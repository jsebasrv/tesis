\documentclass[8pt]{extreport} 
\usepackage{hyperref}
\usepackage{CJKutf8}
\begin{document}
\begin{CJK}{UTF8}{min}
\\	それはとってもいい話だ。	それはとってもいいはなしだ 
\\	私は絵を見るのが好きです。	わたしはえをみるのがすきです 
\\	そこに大きな円を描いて。	そこにおおきなえんをかいて 
\\	京都にはお寺が多い。	きょうとにはおてらがおおい 
\\	家に遊びに来てください。	うちにあそびにきてください 
\\	これをください。	これをください 
\\	彼の車は新しい。	かれのくるまはあたらしい 
\\	私が行きましょう。	わたしがいきましょう 
\\	3月は仕事が忙しい。	さんがつはしごとがいそがしい 
\\	テストを始めてください。	てすとをはじめてください 
\\	彼は医者になりました。	かれはいしゃになりました 
\\	友達と一緒に宿題をした。	ともだちといっしょにしゅくだいをした 
\\	今朝は早く家を出ました。	けさははやくいえをでました 
\\	このパソコンを使ってください。	このぱそこんをつかってください 
\\	私は友達のところに泊まった。	わたしはともだちのところにとまった 
\\	私もそう思います。	わたしもそうおもいます 
\\	私は車を持っています。	わたしはくるまをもっています 
\\	今は時間がありません。	いまはじかんがありません 
\\	今年はイタリアに旅行したい。	ことしはいたりあにりょこうしたい 
\\	誰もその話を知らない。	だれもそのはなしをしらない 
\\	今、朝ご飯を作っています。	いまあさごはんをつくっています 
\\	彼の日本語のレベルは私と同じ位だ。	かれのにほんごのれべるはわたしとおなじくらいだ 
\\	彼は今、勉強しています。	かれはいまべんきょうしています 
\\	工事は3月まで続きます。	こうじはさんがつまでつづきます 
\\	私の国について少しお話しましょう。	わたしのくにについてすこしおはなししましょう 
\\	宿題は未だ終わっていません。	しゅくだいはまだおわっていません 
\\	一緒に宿題をやろう。	いっしょにしゅくだいをやろう 
\\	上司が「一杯、飲もう。	
\\	と言った。 
\\	じょうしがいっぱいのもうといった
\\	今年は雨が少ないです。	ことしはあめがすくないです 
\\	日曜日は図書館に行きます。	にちようびはとしょかんにいきます 
\\	帰国することに決めました。	きこくすることにきめました 
\\	これがこの町で一番高いビルです。	これがこのまちでいちばんたかいびるです 
\\	よく考えてください。	よくかんがえてください 
\\	膝に痛みを感じます。	ひざにいたみをかんじます 
\\	電車で子供が騒いでいた。	でんしゃでこどもがさわいでいた 
\\	大学に行ってもっと勉強したいです。	だいがくにいってもっとべんきょうしたいです 
\\	彼はもう帰りました。	かれはもうかえりました 
\\	駅の近くで食事をした。	えきのちかくでしょくじをした 
\\	彼はかばんに手帳を入れた。	かれはかばんにてちょうをいれた 
\\	彼は真面目な学生です。	かれはまじめながくせいです 
\\	明日の午後、お客様が来る。	あすのごごおきゃくさまがくる 
\\	彼女の髪はとても長い。	かのじょのかみはとてもながい 
\\	本を1冊買いました。	ほんをいっさつかいました 
\\	彼女はよく旅行に行きます。	かのじょはよくりょこうにいきます 
\\	彼女は
\\	です。	かのじょはおーえるです 
\\	彼女はラジオを聞いています。	かのじょはらじおをきいています 
\\	あなたはどう思いますか。	あなたはどうおもいますか 
\\	皆一生懸命生きている。	みんないっしょうけんめいいきている 
\\	この言葉の意味が分かりません。	このことばのいみがわかりません 
\\	彼はタオルで顔を拭きました。	かれはたおるでかおをふきました 
\\	あの本をどこに置きましたか。	あのほんをどこにおきましたか 
\\	それを一つください。	それをひとつください 
\\	この本、あなたにあげます。	このほんあなたにあげます 
\\	こう小さい字は読めない。	こうちいさいじはよめない 
\\	学校は8時半に始まります。	がっこうははちじはんにはじまります 
\\	友達が誕生日プレゼントをくれた。	ともだちがたんじょうびぷれぜんとをくれた 
\\	私は毎朝6時に起きます。	わたしはまいあさろくじにおきます 
\\	今年の春は暖かいね。	ことしのはるはあたたかいね 
\\	午前9時のニュースです。	ごぜんくじのにゅーすです 
\\	郵便局で切手を買いました。	ゆうびんきょくできってをかいました 
\\	別の本も見せてください。	べつのほんもみせてください 
\\	会議は4時に終わります。	かいぎはよじにおわります 
\\	彼はまだ若いです。	かれはまだわかいです 
\\	彼はかなり英語が上手です。	かれはかなりえいごがじょうずです 
\\	彼はかばんをいすの上に置きました。	かれはかばんをいすのうえにおきました 
\\	彼は会社の近くに住んでいる。	かれはかいしゃのちかくにすんでいる 
\\	今、会社に戻ります。	いまかいしゃにもどります 
\\	姉は銀行で働いています。	あねはぎんこうではたらいています 
\\	仕事の後、映画を見た。	しごとのあとえいがをみた 
\\	私は日本語の先生になりたいです。	わたしはにほんごのせんせいになりたいです 
\\	彼はステージに立った。	かれはすてーじにたった 
\\	彼は意外に気が小さい。	かれはいがいにきがちいさい 
\\	ウェイターを呼びましょう。	うぇいたーをよびましょう 
\\	もっと近くに来てください。	もっとちかくにきてください 
\\	家に帰ろう。	うちにかえろう 
\\	彼の気持ちが分からない。	かれのきもちがわからない 
\\	鳥が飛んでいます。	とりがとんでいます 
\\	弟が車を買った。	おとうとがくるまをかった 
\\	旅行の日程を変えました。	りょこうのにっていをかえました 
\\	彼の家はとても広い。	かれのいえはとてもひろい 
\\	質問の意味は分かりましたか。	しつもんのいみはわかりましたか 
\\	グラスの数が足りません。	ぐらすのかずがたりません 
\\	駅はここから近いです。	えきはここからちかいです 
\\	そこに座ってください。	そこにすわってください 
\\	彼は毎晩3キロ走っています。	かれはまいばんさんきろはしっています 
\\	彼は家を売った。	かれはいえをうった 
\\	気持ちのいい朝です。	きもちのいいあさです 
\\	彼は数学を教えています。	かれはすうがくをおしえています 
\\	駅まで歩きましょう。	えきまであるきましょう 
\\	直ぐ行きます。	すぐいきます 
\\	また会いましょう。	またあいましょう 
\\	私の部屋は2階にあります。	わたしのへやはにかいにあります 
\\	町で大事件が起こりました。	まちでだいじけんがおこりました 
\\	彼女は秋に結婚します。	かのじょはあきにけっこんします 
\\	この本は難しいですね。	このほんはむずかしいですね 
\\	彼の家に荷物を送りました。	かれのいえににもつをおくりました 
\\	犬が病気で死にました。	いぬがびょうきでしにました 
\\	駅からはタクシーに乗ってください。	えきからはたくしーにのってください 
\\	この服はとても安かった。	このふくはとてもやすかった 
\\	彼に手紙を書きました。	かれにてがみをかきました 
\\	母と電話で話しました。	ははとでんわではなしました 
\\	今朝から頭が痛い。	けさからあたまがいたい 
\\	私は体が丈夫だ。	わたしはからだがじょうぶだ 
\\	ここに本があります。	ここにほんがあります 
\\	あなたが来るのを待っています。	あなたがくるのをまっています 
\\	彼は背が低い。	かれはせがひくい 
\\	彼女からプレゼントをもらいました。	かのじょからぷれぜんとをもらいました 
\\	昨日タイカレーを食べました。	きのうたいかれーをたべました 
\\	まだ学校へ行くには早い時間です。	まだがっこうへいくにははやいじかんです 
\\	兄は水泳が得意です。	あにはすいえいがとくいです 
\\	あなたの名前を教えてください。	あなたのなまえをおしえてください 
\\	私の夫はサラリーマンです。	わたしのおっとはさらりーまんです 
\\	一から始めましょう。	いちからはじめましょう 
\\	彼女は来月結婚します。	かのじょはらいげつけっこんします 
\\	親の愛は有り難い。	おやのあいはありがたい 
\\	私は古い車が好きです。	わたしはふるいくるまがすきです 
\\	この絵は美しいです。	このえはうつくしいです 
\\	彼は足が長い。	かれはあしがながい 
\\	次はいつ会いましょうか。	つぎはいつあいましょうか 
\\	明日、6時に起こしてください。	あすろくじにおこしてください 
\\	その写真を見せてください。	そのしゃしんをみせてください 
\\	私の娘はアメリカにいます。	わたしのむすめはあめりかにいます 
\\	この本はとてもおもしろい。	このほんはとてもおもしろい 
\\	少し疲れました。	すこしつかれました 
\\	質問のある方はどうぞ。	しつもんのあるかたはどうぞ 
\\	今日は一人の時間を楽しみたい。	きょうはひとりのじかんをたのしみたい 
\\	すてきな色のセーターですね。	すてきないろのせーたーですね 
\\	みんなにお菓子をあげましょう。	みんなにおかしをあげましょう 
\\	テストでいい点を取った。	てすとでいいてんをとった 
\\	私は日本語を勉強しています。	わたしはにほんごをべんきょうしています 
\\	彼女はイタリア語が出来ます。	かのじょはいたりあごができます 
\\	彼は足が短い。	かれはあしがみじかい 
\\	猿も木から落ちる。	さるもきからおちる 
\\	うちの息子は大学1年生です。	うちのむすこはだいがくいちねんせいです 
\\	彼は歯が白い。	かれははがしろい 
\\	誰と会ってみたいですか。	だれとあってみたいですか 
\\	彼女はいつも元気だ。	かのじょはいつもげんきだ 
\\	息子は飛行機のおもちゃが好きです。	むすこはひこうきのおもちゃがすきです 
\\	今日はヤンキースが勝った。	きょうはやんきーすがかった 
\\	祖父が病気になった。	そふがびょうきになった 
\\	カナダの冬はとても寒いです。	かなだのふゆはとてもさむいです 
\\	新しい年が始まりました。	あたらしいとしがはじまりました 
\\	このかばんは重いです。	このかばんはおもいです 
\\	胸に少し痛みがあります。	むねにすこしいたみがあります 
\\	彼女は雑誌を読んでいます。	かのじょはざっしをよんでいます 
\\	私が払いましょう。	わたしがはらいましょう 
\\	この靴はとても軽い。	このくつはとてもかるい 
\\	新しい仕事を見つけました。	あたらしいしごとをみつけました 
\\	彼女はいつ来ますか。	かのじょはいつきますか 
\\	私には友達がたくさんいます。	わたしにはともだちがたくさんいます 
\\	約束を忘れないでください。	やくそくをわすれないでください 
\\	彼女は酒に強い。	かのじょはさけにつよい 
\\	肉と魚とどちらが好きですか。	にくとさかなとどちらがすきですか 
\\	姉は大学生です。	あねはだいがくせいです 
\\	妹は平仮名を全部覚えました。	いもうとはひらがなをぜんぶおぼえました 
\\	私の部屋は狭いです。	わたしのへやはせまいです 
\\	私はワインが好きです。	わたしはわいんがすきです 
\\	私たちは来年、家を建てます。	わたしたちはらいねんいえをたてます 
\\	この中でどれが好きですか。	このなかでどれがすきですか 
\\	赤いバラを買いました。	あかいばらをかいました 
\\	今日は仕事がありません。	きょうはしごとがありません 
\\	今日はスーツを着ています。	きょうはすーつをきています 
\\	赤ん坊が笑っています。	あかんぼうがわらっています 
\\	彼はクラスで一番背が高い。	かれはくらすでいちばんせがたかい 
\\	今日は日本語の授業があります。	きょうはにほんごのじゅぎょうがあります 
\\	その次の週は空いてますか。	そのつぎのしゅうはあいてますか 
\\	漢字は中国から来ました。	かんじはちゅうごくからきました 
\\	毎日、駅まで自転車で行きます。	まいにちえきまでじてんしゃでいきます 
\\	私は電車で通学しています。	わたしはでんしゃでつうがくしています 
\\	彼は郵便局を探していました。	かれはゆうびんきょくをさがしていました 
\\	紙と鉛筆はありますか。	かみとえんぴつはありますか 
\\	私たちは大きな声で歌いました。	わたしたちはおおきなこえでうたいました 
\\	前の車はとても遅い。	まえのくるまはとてもおそい 
\\	きりんの首は長い。	きりんのくびはながい 
\\	彼は走るのが速い。	かれははしるのがはやい 
\\	一緒に帰りましょう。	いっしょにかえりましょう 
\\	私は毎日、日記を付けています。	わたしはまいにちにっきをつけています 
\\	今月はとても忙しい。	こんげつはとてもいそがしい 
\\	子供たちが公園で遊んでいる。	こどもたちがこうえんであそんでいる 
\\	家から学校までは遠いです。	いえからがっこうまではとおいです 
\\	その子は体が少し弱い。	そのこはからだがすこしよわい 
\\	彼女は耳がよく聞こえません。	かのじょはみみがよくきこえません 
\\	私は窓側の席に座った。	わたしはまどがわのせきにすわった 
\\	右のポケットにハンカチが入っています。	みぎのぽけっとにはんかちがはいっています 
\\	私は朝、シャワーを浴びます。	わたしはあさしゃわーをあびます 
\\	肩が凝りました。	かたがこりました 
\\	友達とお酒を飲んでいます。	ともだちとおさけをのんでいます 
\\	もう寝よう。	もうねよう 
\\	昼間は電気を消してください。	ひるまはでんきをけしてください 
\\	おかげさまで元気です。	おかげさまでげんきです 
\\	私たちのチームはその試合で負けた。	わたしたちのちーむはそのしあいでまけた 
\\	それ、全部ください。	それぜんぶください 
\\	私は去年フランスへ行った。	わたしはきょねんふらんすへいった 
\\	このドアは引いてください。	このどあはひいてください 
\\	図書館で料理の本を借りた。	としょかんでりょうりのほんをかりた 
\\	彼は荷物をあみだなに上げた。	かれはにもつをあみだなにあげた 
\\	この町には緑がたくさんあります。	このまちにはみどりがたくさんあります 
\\	彼の腕は太い。	かれのうではふとい 
\\	ドアを開けてください。	どあをあけてください 
\\	教科書を閉じてください。	きょうかしょをとじてください 
\\	時の経つのは早い。	ときのたつのははやい 
\\	日本人はお米が大好きです。	にほんじんはおこめがだいすきです 
\\	この紙を半分に切ってください。	このかみをはんぶんにきってください 
\\	赤い線を2本引いて下さい。	あかいせんをにほんひいてください 
\\	旅行の日程が決まりました。	りょこうのにっていがきまりました 
\\	彼はとても楽しい人です。	かれはとてもたのしいひとです 
\\	昨日、新しい服を買った。	きのうあたらしいふくをかった 
\\	後ろを向いて。	うしろをむいて 
\\	彼に会えて嬉しかった。	かれにあえてうれしかった 
\\	昨日から腰が痛い。	きのうからこしがいたい 
\\	日曜日は海に行きました。	にちようびはうみにいきました 
\\	窓を開けてください。	まどをあけてください 
\\	私は昼のドラマを毎日見ます。	わたしはひるのどらまをまいにちみます 
\\	お母さんによろしくお伝えください。	おかあさんによろしくおつたえください 
\\	彼は留学生です。	かれはりゅうがくせいです 
\\	夏休みももう終わりだ。	なつやすみももうおわりだ 
\\	彼女は背が高い。	かのじょはせがたかい 
\\	私が手伝いましょう。	わたしがてつだいましょう 
\\	鼻がかゆいです。	はながかゆいです 
\\	近所で盗難事件が起きました。	きんじょでとうなんじけんがおきました 
\\	このスープはとても熱い。	このすーぷはとてもあつい 
\\	皿にケーキを乗せました。	さらにけーきをのせました 
\\	その映画はとても悲しかった。	そのえいがはとてもかなしかった 
\\	私は家族を愛しています。	わたしはかぞくをあいしています 
\\	彼女はよくしゃべるね。	かのじょはよくしゃべるね 
\\	私の家は駅の近くです。	わたしのいえはえきのちかくです 
\\	このイチゴは甘い。	このいちごはあまい 
\\	新しいテーブルを買いました。	あたらしいてーぶるをかいました 
\\	日本の食べ物はとても美味しいです。	にっぽんのたべものはとてもおいしいです 
\\	新しい仕事が始まりました。	あたらしいしごとがはじまりました 
\\	私たちはビデオゲームをした。	わたしたちはびでおげーむをした 
\\	その子は指で十数えました。	そのこはゆびでじゅうかぞえました 
\\	今日はいい天気ですね。	きょうはいいてんきですね 
\\	今日はとても暑い。	きょうはとてもあつい 
\\	彼女は足が太い。	かのじょはあしがふとい 
\\	晩ご飯は食べましたか。	ばんごはんはたべましたか 
\\	土曜日の夜はクラブに行きます。	どようびのよるはくらぶにいきます 
\\	彼は神を信じている。	かれはかみをしんじている 
\\	今日は頭が痛いです。	きょうはあたまがいたいです 
\\	お父さんは会社員です。	おとうさんはかいしゃいんです 
\\	彼女は多分家で寝ています。	かのじょはたぶんいえでねています 
\\	時計を見たらちょうど3時だった。	とけいをみたらちょうどさんじだった 
\\	今日はこのホテルに泊まります。	きょうはこのほてるにとまります 
\\	どうして泣いているの。	どうしてないているの 
\\	夫の服をハンガーに掛けた。	おっとのふくをはんがーにかけた 
\\	そこを左に曲がってください。	そこをひだりにまがってください 
\\	おなかが空きました。	おなかがすきました 
\\	ちゃんとドアを閉めてよ。	ちゃんとどあをしめてよ 
\\	私はタバコが嫌いです。	わたしはたばこがきらいです 
\\	金曜日の夜は友達と出かけます。	きんようびのよるはともだちとでかけます 
\\	予約は要りません。	よやくはいりません 
\\	ここには何も無い。	ここにはなにもない 
\\	彼女に黄色いバラを買いました。	かのじょにきいろいばらをかいました 
\\	私は風邪を引きました。	わたしはかぜをひきました 
\\	彼はとても優しい人です。	かれはとてもやさしいひとです 
\\	明日は晴れるといいですね。	あしたははれるといいですね 
\\	私は犬が大好きだ。	わたしはいぬがだいすきだ 
\\	彼女は茶色の靴を履いています。	かのじょはちゃいろのくつをはいています 
\\	彼の部屋はとても汚い。	かれのへやはとてもきたない 
\\	レストランは空いていました。	れすとらんはすいていました 
\\	猫が屋根に上っている。	ねこがやねにのぼっている 
\\	私はパンよりご飯が好きだ。	わたしはぱんよりごはんがすきだ 
\\	私たちは先月11日に結婚しました。	わたしたちはせんげつじゅういちにちにけっこんしました 
\\	髪の毛が伸びたね。	かみのけがのびたね 
\\	電気をつけてください。	でんきをつけてください 
\\	月曜日に会いましょう。	げつようびにあいましょう 
\\	寒いので中に入ってください。	さむいのでなかにはいってください 
\\	お名前を片仮名で書いてください。	おなまえをかたかなでかいてください 
\\	今週は日本語のテストがあります。	こんしゅうはにほんごのてすとがあります 
\\	32ページを開いてください。	さんじゅうにぺーじをひらいてください 
\\	水を一杯ください。	みずをいっぱいください 
\\	あれは何ですか。	あれはなんですか 
\\	その人には二回会った。	そのひとにはにかいあった 
\\	お店はもう閉まっていました。	おみせはもうしまっていました 
\\	彼はシートベルトを締めた。	かれはしーとべるとをしめた 
\\	ここの料理はまずい。	ここのりょうりはまずい 
\\	私は平仮名を全部読めます。	わたしはひらがなをぜんぶよめます 
\\	明日は昼頃から曇るでしょう。	あすはひるごろからくもるでしょう 
\\	絵に触らないでください。	えにさわらないでください 
\\	彼は駄目な男だ。	かれはだめなおとこだ 
\\	この書類を書き直してください。	このしょるいをかきなおしてください 
\\	何か飲み物が欲しいな。	なにかのみものがほしいな 
\\	木曜日は仕事が休みです。	もくようびはしごとがやすみです 
\\	曜日を間違えました。	ようびをまちがえました 
\\	そばにいて下さい。	そばにいてください 
\\	こっちに来て下さい。	こっちにきてください 
\\	父さん、ありがとう。	とうさんありがとう 
\\	火曜日に会議があります。	かようびにかいぎがあります 
\\	喉が渇きました。	のどがかわきました 
\\	彼女は三人の子供の母親だ。	かのじょはさんにんのこどものははおやだ 
\\	水曜日はバイトがあります。	すいようびはばいとがあります 
\\	ソフトクリームを二つください。	そふとくりーむをふたつください 
\\	今晩のパーティーは何時からですか。	こんばんのぱーてぃーはなんじからですか 
\\	千円貸してください。	せんえんかしてください 
\\	息子は六つになりました。	むすこはむっつになりました 
\\	六日前に日本に帰ってきました。	むいかまえににほんにかえってきました 
\\	昨日、あなたのお姉さんに会ったよ。	きのうあなたのおねえさんにあったよ 
\\	クーラーはまだ直りません。	くーらーはまだなおりません 
\\	この問題はかなり易しいです。	このもんだいはかなりやさしいです 
\\	彼はこの辺に住んでいます。	かれはこのへんにすんでいます 
\\	彼女は眠いようです。	かのじょはねむいようです 
\\	あなたのお兄さんは何歳?	あなたのおにいさんはなんさい 
\\	彼は辛いものが好きです。	かれはからいものがすきです 
\\	りんごを八個ください。	りんごをはちこください 
\\	手紙が届くのに三日かかりました。	てがみがとどくのにみっかかかりました 
\\	あそこにバス停があります。	あそこにばすていがあります 
\\	彼は昼過ぎに来ます。	かれはひるすぎにきます 
\\	その店の前で会いましょう。	そのみせのまえであいましょう 
\\	五月五日は祝日です	ごがついつかはしゅくじつです 
\\	プールは人で一杯です。	ぷーるはひとでいっぱいです 
\\	野球は九人で1チームです。	やきゅうはきゅうにんでひとちーむです 
\\	このぶどうは酸っぱいです。	このぶどうはすっぱいです 
\\	答えが違います。	こたえがちがいます 
\\	彼女は指が細いですね。	かのじょはゆびがほそいですね 
\\	コップを三つ買いました。	こっぷをみっつかいました 
\\	八日からイギリスに行きます。	ようかからいぎりすにいきます 
\\	私の弟は高校生です。	わたしのおとうとはこうこうせいです 
\\	妹は歌が上手です。	いもうとはうたがじょうずです 
\\	今日は風が強い。	きょうはかぜがつよい 
\\	バナナが七本あります。	ばなながななほんあります 
\\	来月の二十日は弟の誕生日です。	らいげつのはつかはおとうとのたんじょうびです 
\\	私は二日待った。	わたしはふつかまった 
\\	私は腕時計を四つ持っています。	わたしはうでどけいをよっつもっています 
\\	このコートはとても暖かい。	このこーとはとてもあたたかい 
\\	私の机の上に書類がたくさん有ります。	わたしのつくえのうえにしょるいがたくさんあります 
\\	彼女は良い友達です。	かのじょはいいともだちです 
\\	桃を五つください。	ももをいつつください 
\\	私の妹は小学生です。	わたしのいもうとはしょうがくせいです 
\\	上を向いて。	うえをむいて 
\\	動かないで。	うごかないで 
\\	駅はどこですか。	えきはどこですか 
\\	このケーキはおいしいね。	このけーきはおいしいね 
\\	あの大きい建物は何ですか。	あのおおきいたてものはなんですか 
\\	弟は野球が好きです。	おとうとはやきゅうがすきです 
\\	男の人が私たちに話しかけた。	おとこのひとがわたしたちにはなしかけた 
\\	お酒は大人になってから。	おさけはおとなになってから 
\\	お願いがあります。	おねがいがあります 
\\	店員は若い女の人でした。	てんいんはわかいおんなのひとでした 
\\	これはかなり金がかかった。	これはかなりかねがかかった 
\\	彼は私の上司です。	かれはわたしのじょうしです 
\\	昨日、友達に会った。	きのうともだちにあった 
\\	彼女はとても奇麗だ。	かのじょはとてもきれいだ 
\\	口を大きく開けてください。	くちをおおきくあけてください 
\\	五人で旅行に行きました。	ごにんでりょこうにいきました 
\\	九日に荷物が届きます。	ここのかににもつがとどきます 
\\	娘は明日九つになります。	むすめはあすここのつになります 
\\	私の質問に答えてください。	わたしのしつもんにこたえてください 
\\	お先にどうぞ。	おさきにどうぞ 
\\	この部屋は寒いです。	このへやはさむいです 
\\	四月に大学に入学しました。	しがつにだいがくににゅうがくしました 
\\	財布は机の下にあった。	さいふはつくえのしたにあった 
\\	大丈夫ですか。	だいじょうぶですか 
\\	彼はかばんから教科書を出した。	かれはかばんからきょうかしょをだした 
\\	小さい花が咲いています。	ちいさいはながさいています 
\\	私は父が大好きです。	わたしはちちがだいすきです 
\\	来月の一日は空いてますか。	らいげつのついたちはあいていますか 
\\	今夜は月がとてもきれいです。	こんやはつきがとてもきれいです 
\\	今日は妻の誕生日だ。	きょうはつまのたんじょうびだ 
\\	冷たい飲み物をください。	つめたいのみものをください 
\\	分かった人は手を上げてください。	わかったひとはてをあげてください 
\\	十日後に帰ります。	とおかごにかえります 
\\	彼は時々遅刻します。	かれはときどきちこくします 
\\	私は夏が大好き。	わたしはなつがだいすき 
\\	この子は今年七つになります。	このこはことしななつになります 
\\	夕食には何を食べたいですか。	ゆうしょくにはなにをたべたいですか 
\\	先月の七日に孫が生まれました。	せんげつのなのかにまごがうまれました 
\\	昨日、母と話をしました。	きのうははとはなしをしました 
\\	彼は優しい人です。	かれはやさしいひとです 
\\	そこには私一人しかいなかった。	そこにはわたしひとりしかいなかった 
\\	私の祖母は百才です。	わたしのそぼはひゃくさいです 
\\	彼は私の方を見ました。	かれはわたしのほうをみました 
\\	他に方法がありません。	ほかにほうほうがありません 
\\	僕は学生です。	ぼくはがくせいです 
\\	僕は新しい靴が欲しいです。	ぼくはあたらしいくつがほしいです 
\\	この靴は1万円です。	このくつはいちまんえんです 
\\	ここから富士山がよく見えます。	ここからふじさんがよくみえます 
\\	私はこの店によく来ます。	わたしはこのみせによくきます 
\\	この道を真っ直ぐ行くと駅です。	このみちをまっすぐいくとえきです 
\\	彼女は青い目をしています。	かのじょはあおいめをしています 
\\	あの家には時計が八つあります。	あのいえにはとけいがやっつあります 
\\	話すのを止めてください。	はなすのをやめてください 
\\	新学期は来月の四日からです。	しんがっきはらいげつのよっかからです 
\\	きのうの夜は家にいました。	きのうのよるはいえにいました 
\\	ハワイは四回目です。	はわいはよんかいめです 
\\	来年一緒に旅行しましょう。	らいねんいっしょにりょこうしましょう 
\\	彼には子供が六人います。	かれにはこどもがろくにんいます 
\\	たばこは体に悪い。	たばこはからだにわるい 
\\	お手洗いはどこですか。	おてあらいはどこですか 
\\	ご主人はお元気ですか。	ごしゅじんはおげんきですか 
\\	あなたが本当に好きです。	あなたがほんとうにすきです 
\\	宿題は自分でやりなさい。	しゅくだいはじぶんでやりなさい 
\\	柿の木に実がたくさんなっています。	かきのきにみがたくさんなっています 
\\	これは医者のためのサイトです。	これはいしゃのためのさいとです 
\\	カレーライスは700円です。	かれーらいすはななひゃくえんです 
\\	分からない場合は私に聞いてください。	わからないばあいはわたしにきいてください 
\\	明日、会社を休みます。	あしたかいしゃをやすみます 
\\	メガネが見つかりません。	めがねがみつかりません 
\\	問題が一つあります。	もんだいがひとつあります 
\\	ダムの開発に住民は反対しています。	だむのかいはつにじゅうみんははんたいしています 
\\	その事件の犯人はまだ捕まっていない。	そのじけんのはんにんはまだつかまっていない 
\\	私にはたくさんのお金が必要だ。	わたしにはたくさんのおかねがひつようだ 
\\	彼女はアメリカの企業で働いています。	かのじょはあめりかのきぎょうではたらいています 
\\	その会社は来月、キャンペーンを行う。	そのかいしゃはらいげつきゃんぺーんをおこなう 
\\	現在の気温は30度です。	げんざいのきおんはさんじゅうどです 
\\	私はよく図書館を利用します。	わたしはよくとしょかんをりようします 
\\	今の首相はあまり力がない。	いまのしゅしょうはあまりちからがない 
\\	この車はよく持っているね。	このくるまはよくもっているね 
\\	そのデモについて政府は何もしなかった。	そのでもについてせいふはなにもしなかった 
\\	合格者が発表された。	ごうかくしゃがはっぴょうされた 
\\	電車のドアが開きました。	でんしゃのどあがあきました 
\\	彼は就職試験を受けた。	かれはしゅうしょくしけんをうけた 
\\	これから、より一層努力します。	これからよりいっそうどりょくします 
\\	昨年は地震が多い年でした。	さくねんはじしんがおおいとしでした 
\\	学生たちはインターネットでいろいろな情報を集めた。	がくせいたちはいんたーねっとでいろいろなじょうほうをあつめた 
\\	彼は新製品に興味を示している。	かれはしんせいひんにきょうみをしめしている 
\\	新しいビルの建設が始まった。	あたらしいびるのけんせつがはじまった 
\\	前売券は窓口で販売しています。	まえうりけんはまどぐちではんばいしています 
\\	けがはもう治りましたか。	けがはもうなおりましたか 
\\	このソフトにはいろいろな機能があります。	このそふとにはいろいろなきのうがあります 
\\	彼は何年も地震の研究をしている。	かれはなんねんもじしんのけんきゅうをしている 
\\	ここは一方通行です。	ここはいっぽうつうこうです 
\\	この服は300ドルしました。	このふくはさんびゃくどるしました 
\\	全てのシステムが停止した。	すべてのしすてむがていしした 
\\	子供の頃、よくその公園で遊んだ。	こどものころよくそのこうえんであそんだ 
\\	その質問に対する答えが見つからなかった。	そのしつもんにたいするこたえがみつからなかった 
\\	いい方法を思いつきました。	いいほうほうをおもいつきました 
\\	飛行機が1時間以上遅れた。	ひこうきがいちじかんいじょうおくれた 
\\	それは最近話題の本ですね。	それはさいきんわだいのほんですね 
\\	今後ともよろしくお願いします。	こんごともよろしくおねがいします 
\\	私は世界旅行をしたい。	わたしはせかいりょこうをしたい 
\\	彼はその詩を用いて自分の気持ちを伝えた。	かれはそのしをもちいてじぶんのきもちをつたえた 
\\	今夜は大いに語りましょう。	こんやはおおいにかたりましょう 
\\	新しいコンピューターを買った。	あたらしいこんぴゅーたーをかった 
\\	ここが建物の中心です。	ここがたてもののちゅうしんです 
\\	犬にえさをやった。	いぬにえさをやった 
\\	それはどういう意味ですか。	それはどういういみですか 
\\	彼は一人旅の計画を立てた。	かれはひとりたびのけいかくをたてた 
\\	私は彼から大きな影響を受けました。	わたしはかれからおおきなえいきょうをうけました 
\\	この町は人口が増えた。	このまちはじんこうがふえた 
\\	良い家を選ぶのは難しい。	よいいえをえらぶのはむずかしい 
\\	大学で経済を勉強しました。	だいがくでけいざいをべんきょうしました 
\\	会社で新しいシステムを導入した。	かいしゃであたらしいしすてむをどうにゅうした 
\\	日本での生活は楽しいです。	にほんでのせいかつはたのしいです 
\\	早く授業を進めましょう。	はやくじゅぎょうをすすめましょう 
\\	彼はその事件に関係がない。	かれはそのじけんにかんけいがない 
\\	仕事を続けてください。	しごとをつづけてください 
\\	お金がほとんどありません。	おかねがほとんどありません 
\\	この携帯電話が使えるのは国内だけです。	このけいたいでんわがつかえるのはこくないだけです 
\\	音楽は彼の得意な分野です。	おんがくはかれのとくいなぶんやです 
\\	彼は小さな会社に勤めています。	かれはちいさなかいしゃにつとめています 
\\	その問題に関する記事を読みました。	そのもんだいにかんするきじをよみました 
\\	ここは道の状態がとても悪いです。	ここはみちのじょうたいがとてもわるいです 
\\	フランスの大統領は誰ですか。	ふらんすのだいとうりょうはだれですか 
\\	このアンケートは大学生が対象です。	このあんけーとはだいがくせいがたいしょうです 
\\	事務的な処理に1週間かかります。	じむてきなしょりにいっしゅうかんかかります 
\\	そのパーティーに女性は何人来ますか。	そのぱーてぃーにじょせいはなんにんきますか 
\\	ここが私の家です。	ここがわたしのいえです 
\\	彼はパソコンを2台持っています。	かれはぱそこんをにだいもっています 
\\	トラブルがやっと解決した。	とらぶるがやっとかいけつした 
\\	夢を実現するには努力が必要です。	ゆめをじつげんするにはどりょくがひつようです 
\\	毎年多くの人が海外へ旅行する。	まいとしおおくのひとがかいがいへりょこうする 
\\	まず相手の動きを見ましょう。	まずあいてのうごきをみましょう 
\\	あなたの話は面白いね。	あなたのはなしはおもしろいね 
\\	ガソリンの価格がどんどん上がっている。	がそりんのかかくがどんどんあがっている 
\\	私たちは2階に上がった。	わたしたちはにかいにあがった 
\\	コーヒーをもう一杯ください。	こーひーをもういっぱいください 
\\	弟は切手を集めています。	おとうとはきってをあつめています 
\\	彼は大きな声で話した。	かれはおおきなこえではなした 
\\	東京に来るのは初めてです。	とうきょうにくるのははじめてです 
\\	明らかに彼が悪い。	あきらかにかれがわるい 
\\	計画を一部変更しましょう。	けいかくをいちぶへんこうしましょう 
\\	私たちがその問題を調査しています。	わたしたちがそのもんだいをちょうさしています 
\\	党の代表が質問に答えました。	とうのだいひょうがしつもんにこたえました 
\\	信号が青に変わりました。	しんごうがあおにかわりました 
\\	おばは小さな会社の社長です。	おばはちいさなかいしゃのしゃちょうです 
\\	次は全国のお天気です。	つぎはぜんこくのおてんきです 
\\	この地域は雨が多い。	このちいきはあめがおおい 
\\	これは重要な書類です。	これはじゅうようなしょるいです 
\\	遠くに彼女の姿が見えた。	とおくにかのじょのすがたがみえた 
\\	帰ったら先ず手を洗いましょう。	かえったらまずてをあらいましょう 
\\	私の両親は昭和生まれです。	わたしのりょうしんはしょうわうまれです 
\\	これは大きな社会問題になっている。	これはおおきなしゃかいもんだいになっている 
\\	受付でプログラムを受け取った。	うけつけでぷろぐらむをうけとった 
\\	みんな私たちに期待しています。	みんなわたしたちにきたいしています 
\\	お相撲さんは、みんな力持ちだ。	おすもうさんはみんなちからもちだ 
\\	部長を除く全員が土曜日も働いた。	ぶちょうをのぞくぜんいんがどようびもはたらいた 
\\	一般の方はこちらの席へどうぞ。	いっぱんのかたはこちらのせきへどうぞ 
\\	まあ今回は許してあげよう。	まあこんかいはゆるしてあげよう 
\\	ゴミのリサイクルが普及している。	ごみのりさいくるがふきゅうしている 
\\	彼は歌で自分の気持ちを表現した。	かれはうたでじぶんのきもちをひょうげんした 
\\	作業するにはもっと広いスペースが必要だ。	さぎょうするにはもっとひろいすぺーすがひつようだ 
\\	今日の予定を教えてください。	きょうのよていをおしえてください 
\\	この会社の工場は海の近くにある。	このかいしゃのこうじょうはうみのちかくにある 
\\	彼は食品工場で働いています。	かれはしょくひんこうじょうではたらいています 
\\	よく答えが分かりましたね。	よくこたえがわかりましたね 
\\	ボルドーはワインの生産で有名だ。	ぼるどーはわいんのせいさんでゆうめいだ 
\\	政権が交代した。	せいけんがこうたいした 
\\	彼は非常に高い技術を持っている。	かれはひじょうにたかいぎじゅつをもっている 
\\	電気がついたままですよ。	でんきがついたままですよ 
\\	新しいホームページが完成した。	あたらしいほーむぺーじがかんせいした 
\\	今年は変化の多い年でした。	ことしはへんかのおおいとしでした 
\\	宇宙人は存在すると思いますか。	うちゅうじんはそんざいするとおもいますか 
\\	明日は市民マラソンに参加します。	あすはしみんまらそんにさんかします 
\\	私はテレビをあまり見ません。	わたしはてれびをあまりみません 
\\	ここは父の土地です。	ここはちちのとちです 
\\	体重がかなり減りました。	たいじゅうがかなりへりました 
\\	今日中にこの問題を検討してください。	きょうじゅうにこのもんだいをけんとうしてください 
\\	私は政治に関心がある。	わたしはせいじにかんしんがある 
\\	現在、スペシャルキャンペーンを実施中です。	げんざいすぺしゃるきゃんぺーんをじっしちゅうです 
\\	運動会は9時開始です。	うんどうかいはくじかいしです 
\\	災害に備えて大量の水を買い込んだ。	さいがいにそなえてたいりょうのみずをかいこんだ 
\\	これはイタリアから輸入した服です。	これはいたりあからゆにゅうしたふくです 
\\	これは世界最大の船です。	これはせかいさいだいのふねです 
\\	2社は互いに競争している。	にしゃはたがいにきょうそうしている 
\\	この部分は問題ないです。	このぶぶんはもんだいないです 
\\	突然、電気が消えた。	とつぜんでんきがきえた 
\\	うちは五人家族です。	うちはごにんかぞくです 
\\	この本の内容を説明してください。	このほんのないようをせつめいしてください 
\\	今月と先月の売上を比べた。	こんげつとせんげつのうりあげをくらべた 
\\	姉夫婦に男の子が生まれました。	あねふうふにおとこのこがうまれました 
\\	そのニュースは彼に大きなショックを与えた。	そのにゅーすはかれにおおきなしょっくをあたえた 
\\	彼は選挙に出るつもりだ。	かれはせんきょにでるつもりだ 
\\	その2社は対立しています。	そのにしゃはたいりつしています 
\\	遅れた理由を教えてください。	おくれたりゆうをおしえてください 
\\	システムの構成を変えてみました。	しすてむのこうせいをかえてみました 
\\	彼女が私のスケジュールを管理している。	かのじょがわたしのすけじゅーるをかんりしている 
\\	島の人口は年々増加しています。	しまのじんこうはねんねんぞうかしています 
\\	彼は行政を改革したいと思っている。	かれはぎょうせいをかいかくしたいとおもっている 
\\	会談の内容が発表されました。	かいだんのないようがはっぴょうされました 
\\	初めに会長が挨拶した。	はじめにかいちょうがあいさつした 
\\	このソフトで日本語を勉強することができます。	このそふとでにほんごをべんきょうすることができます 
\\	彼がパーティー会場を提供してくれました。	かれがぱーてぃーかいじょうをていきょうしてくれました 
\\	この温泉はただです。	このおんせんはただです 
\\	今日はこれらの問題について話し合います。	きょうはこれらのもんだいについてはなしあいます 
\\	留学してから私の世界が広がった。	りゅうがくしてからわたしのせかいがひろがった 
\\	この曲線がこの車の特徴です。	このきょくせんがこのくるまのとくちょうです 
\\	この単語の意味を辞書で調べましょう。	このたんごのいみをじしょでしらべましょう 
\\	彼は良いところだけを強調した。	かれはよいところだけをきょうちょうした 
\\	来年から新しい制度が始まります。	らいねんからあたらしいせいどがはじまります 
\\	今後の方針が決まった。	こんごのほうしんがきまった 
\\	このプリンターは従来のものより速い。	このぷりんたーはじゅうらいのものよりはやい 
\\	いずれまたお会いしましょう。	いずれまたおあいしましょう 
\\	彼は事故で怪我をしました。	かれはじこでけがをしました 
\\	あとで電話します。	あとででんわします 
\\	母はまだ外国に行ったことがありません。	はははまだがいこくにいったことがありません 
\\	彼が来た目的が分かりません。	かれがきたもくてきがわかりません 
\\	今日は自由な時間が多い。	きょうはじゆうなじかんがおおい 
\\	私たちは今、結婚資金を貯めています。	わたしたちはいまけっこんしきんをためています 
\\	銀行は3時まで開いています。	ぎんこうはさんじまであいています 
\\	それは全て事実ですか。	それはすべてじじつですか 
\\	会議で重要な決定がありました。	かいぎでじゅうようなけっていがありました 
\\	あの人たちは皆、同じ方向を見ている。	あのひとたちはみんなおなじほうこうをみている 
\\	ここは静かな住宅地だ。	ここはしずかなじゅうたくちだ 
\\	彼がクラスの代表だ。	かれがくらすのだいひょうだ 
\\	激しい雨が降っています。	はげしいあめがふっています 
\\	最近株を始めました。	さいきんかぶをはじめました 
\\	今、値段を交渉しています。	いまねだんをこうしょうしています 
\\	試合の結果を早く知りたい。	しあいのけっかをはやくしりたい 
\\	全体の80パーセントが完成しました。	ぜんたいのはちじゅっぱーせんとがかんせいしました 
\\	後に彼は総理大臣になりました。	のちにかれはそうりだいじんになりました 
\\	彼女は命令に従わなかった。	かのじょはめいれいにしたがわなかった 
\\	お金はまだ十分あります。	おかねはまだじゅうぶんあります 
\\	このビールはあまり美味しくありません。	このびーるはあまりおいしくありません 
\\	我が社の経営はうまくいっています。	わがしゃのけいえいはうまくいっています 
\\	この事故の原因は何ですか。	このじこのげんいんはなんですか 
\\	写真は良い思い出になります。	しゃしんはいいおもいでになります 
\\	彼女は自分の立場を分かっていない。	かのじょはじぶんのたちばをわかっていない 
\\	一緒に対策を考えましょう。	いっしょにたいさくをかんがえましょう 
\\	彼女は同じ間違いを繰り返した。	かのじょはおなじまちがいをくりかえした 
\\	最近、彼の評価が上がった。	さいきんかれのひょうかがあがった 
\\	なお、雨の場合は中止です。	なおあめのばあいはちゅうしです 
\\	その計算は間違っている。	そのけいさんはまちがっている 
\\	私は4つの会社に投資しています。	わたしはよっつのかいしゃにとうししています 
\\	20キロのダイエットは可能だと思いますか。	にじゅっきろのだいえっとはかのうだとおもいますか 
\\	塩、コショウを加えてください。	しおこしょうをくわえてください 
\\	バラには色々な種類があります。	ばらにはいろいろなしゅるいがあります 
\\	国は国語教育を強化しています。	くにはこくごきょういくをきょうかしています 
\\	あなたの意見が聞きたいです。	あなたのいけんがききたいです 
\\	その仕事は危険を伴う。	そのしごとはきけんをともなう 
\\	今日の新聞、どこに置いた?	きょうのしんぶんどこにおいた 
\\	彼は文章がとてもうまい。	かれはぶんしょうがとてもうまい 
\\	彼女の大きな帽子はとても目立つ。	かのじょのおおきなぼうしはとてもめだつ 
\\	兄は船の設計をしています。	あにはふねのせっけいをしています 
\\	この建物の構造は複雑です。	このたてもののこうぞうはふくざつです 
\\	国会が再開した。	こっかいがさいかいした 
\\	仕事がほぼ終わりました。	しごとがほぼおわりました 
\\	試合の相手は誰ですか。	しあいのあいてはだれですか 
\\	交差点で事故が発生した。	こうさてんでじこがはっせいした 
\\	その会社は女性を多く採用している。	そのかいしゃはじょせいをおおくさいようしている 
\\	料理がたくさん残りました。	りょうりがたくさんのこりました 
\\	この段階では、決断するのはまだ早い。	このだんかいではけつだんするのはまだはやい 
\\	昨日の会議について報告があります。	きのうのかいぎについてほうこくがあります 
\\	この商品はよく売れている。	このしょうひんはよくうれている 
\\	東京は日本一大きな都市です。	とうきょうはにほんいちおおきなとしです 
\\	病院はどこですか。	びょういんはどこですか 
\\	この動物園は日本一の規模です。	このどうぶつえんはにっぽんいちのきぼです 
\\	今日の試合はいい経験になりました。	きょうのしあいはいいけいけんになりました 
\\	彼は能力のある社員です。	かれはのうりょくのあるしゃいんです 
\\	5ページの最初を見てください。	ごぺーじのさいしょをみてください 
\\	ここで国際会議が開かれます。	ここでこくさいかいぎがひらかれます 
\\	彼は国会議員です。	かれはこっかいぎいんです 
\\	彼は新聞記者です。	かれはしんぶんきしゃです 
\\	会社の場所を教えてください。	かいしゃのばしょをおしえてください 
\\	彼はもっと運動した方がいい。	かれはもっとうんどうしたほうがいい 
\\	この店はサービスがいい	このみせはさーびすがいい 
\\	駅前に人が集まっています。	えきまえにひとがあつまっています 
\\	新しい政策はあまり良いとは思えません。	あたらしいせいさくはあまりいいとはおもえません 
\\	この道は車が多い。	このみちはくるまがおおい 
\\	この図を拡大コピーしてください。	このずをかくだいこぴーしてください 
\\	前に進んでください。	まえにすすんでください 
\\	気温は35度に達した。	きおんはさんじゅうごどにたっした 
\\	今日の午後、大事な会議があります。	きょうのごごだいじなかいぎがあります 
\\	実験にはこの装置を使います。	じっけんにはこのそうちをつかいます 
\\	上司は私の主張を受け入れた。	じょうしはわたしのしゅちょうをうけいれた 
\\	マラソンで世界記録が出た。	まらそんでせかいきろくがでた 
\\	売り上げは年度によって違います。	うりあげはねんどによってちがいます 
\\	日本の自動車産業は世界的に有名だ。	にほんのじどうしゃさんぎょうはせかいてきにゆうめいだ 
\\	事件の背景に何があったのだろう。	じけんのはいけいになにがあったのだろう 
\\	サラダの材料をそろえました。	さらだのざいりょうをそろえました 
\\	彼は私の要求にこたえた。	かれはわたしのようきゅうにこたえた 
\\	市民の安全は大切だ。	しみんのあんぜんはたいせつだ 
\\	新しい世紀の始まりです。	あたらしいせいきのはじまりです 
\\	計画を大幅に変更した。	けいかくをおおはばにへんこうした 
\\	この工場では魚を加工している。	このこうじょうではさかなをかこうしている 
\\	彼はその計画を実行した。	かれはそのけいかくをじっこうした 
\\	彼の判断は正しい。	かれのはんだんはただしい 
\\	彼はある組織のリーダーだ。	かれはあるそしきのりーだーだ 
\\	彼の会社は車を輸出しています。	かれのかいしゃはくるまをゆしゅつしています 
\\	警察を呼んでください。	けいさつをよんでください 
\\	みんなにこのことを伝えてください。	みんなにこのことをつたえてください 
\\	その映画は今日、公開されます。	そのえいがはきょうこうかいされます 
\\	今、会社に向かっています。	いまかいしゃにむかっています 
\\	私がそれを直しました。	わたしがそれをなおしました 
\\	具体的な例をいくつか見せてください。	ぐたいてきなれいをいくつかみせてください 
\\	彼が元気になるよう願っています。	かれがげんきになるようねがっています 
\\	店に学生のグループが来た。	みせにがくせいのぐるーぷがきた 
\\	これは私個人の意見です。	これはわたしこじんのいけんです 
\\	この話は真実に基づいています。	このはなしはしんじつにもとづいています 
\\	私には全く分かりません。	わたしにはまったくわかりません 
\\	彼に対する見方が変わりました。	かれにたいするみかたがかわりました 
\\	あの子はいつも姿勢が悪い。	あのこはいつもしせいがわるい 
\\	このプロジェクトにはみんなの協力が必要です。	このぷろじぇくとにはみんなのきょうりょくがひつようです 
\\	私はセールスを担当しています。	わたしはせーるすをたんとうしています 
\\	そのアイデアは彼の提案です。	そのあいであはかれのていあんです 
\\	今日が夏休み最後の日だ。	きょうがなつやすみさいごのひだ 
\\	両親を食事に招いた。	りょうしんをしょくじにまねいた 
\\	今ホテルに着きました。	いまほてるにつきました 
\\	彼は再びここに戻ってきた。	かれはふたたびここにもどってきた 
\\	地球の気温は上昇している。	ちきゅうのきおんはじょうしょうしている 
\\	新しい機械が壊れた。	あたらしいきかいがこわれた 
\\	私たちはその会社に注目している。	わたしたちはそのかいしゃにちゅうもくしている 
\\	それは過去の話だ。	それはかこのはなしだ 
\\	この絵はゴッホの作品だ。	このえはごっほのさくひんだ 
\\	彼に直接お願いしなさい。	かれにちょくせつおねがいしなさい 
\\	彼は同僚を批判した。	かれはどうりょうをひはんした 
\\	私はこの国の文化を勉強しています。	わたしはこのくにのぶんかをべんきょうしています 
\\	そのホテルにはレジャー施設がたくさんある。	そのほてるにはれじゃーしせつがたくさんある 
\\	ついに実験が成功した。	ついにじっけんがせいこうした 
\\	彼は実際にはあまり背が高くない。	かれはじっさいにはあまりせがたかくない 
\\	彼女が怒るのも当然だ。	かのじょがおこるのもとうぜんだ 
\\	年間5万人がここを訪れます。	ねんかんごまんにんがここをおとずれます 
\\	この料理はとても簡単です。	このりょうりはとてもかんたんです 
\\	締め切りが迫っています。	しめきりがせまっています 
\\	彼女は荷物をまとめて出て行った。	かのじょはにもつをまとめてでていった 
\\	私に届くメールのうち、迷惑メールが7割を占めている。	わたしにとどくめーるのうちめいわくめーるがななわりをしめている 
\\	彼女はメッセージを残しました。	かのじょはめっせーじをのこしました 
\\	彼はアジアの言語を研究している。	かれはあじあのげんごをけんきゅうしている 
\\	彼は海外で教育を受けました。	かれはかいがいできょういくをうけました 
\\	留学生はアジア人が多く、その他は3割です。	りゅうがくせいはあじあじんがおおくそのたはさんわりです 
\\	私たちは中国の会社と取引しています。	わたしたちはちゅうごくのかいしゃととりひきしています 
\\	彼は自動車会社に就職した。	かれはじどうしゃがいしゃにしゅうしょくした 
\\	私は反対です。	わたしははんたいです 
\\	最近の若者は本を読まない。	さいきんのわかものはほんをよまない 
\\	このケーキはあなたの分です。	このけーきはあなたのぶんです 
\\	夏休みの課題は何ですか。	なつやすみのかだいはなんですか 
\\	通常は夜8時まで営業しています。	つうじょうはよるはちじまでえいぎょうしています 
\\	私たちは彼を家族同様に思っている。	わたしたちはかれをかぞくどうようにおもっている 
\\	この程度の怪我なら大丈夫です。	このていどのけがならだいじょうぶです 
\\	ボールが彼の頭に当たった。	ぼーるがかれのあたまにあたった 
\\	彼女は会社を訴えた。	かのじょはかいしゃをうったえた 
\\	あなたの事情はよく分かりました。	あなたのじじょうはよくわかりました 
\\	会議で全員が発言した。	かいぎでぜんいんがはつげんした 
\\	最近の若者は難しい本を読まない傾向がある。	さいきんのわかものはむずかしいほんをよまないけいこうがある 
\\	京都で国際会議が開催された。	きょうとでこくさいかいぎがかいさいされた 
\\	壁に文字が書いてあった。	かべにもじがかいてあった 
\\	その子は日本語が分からない。	そのこはにほんごがわからない 
\\	上司が意見を述べた。	じょうしがいけんをのべた 
\\	彼は地域の活動に参加した。	かれはちいきのかつどうにさんかした 
\\	テスト期間は10日から15日までだ。	てすときかんはとおかからじゅうごにちまでだ 
\\	申し込むには一定の資格が要る。	もうしこむにはいっていのしかくがいる 
\\	地域の住民が集まって話合いをした。	ちいきのじゅうみんがあつまってはなしあいをした 
\\	昨日、父から手紙が届いた。	きのうちちからてがみがとどいた 
\\	彼はやはり遅刻しました。	かれはやはりちこくしました 
\\	この薬には胃を守る効果がある。	このくすりにはいをまもるこうかがある 
\\	この状況では出発は難しいです。	このじょうきょうではしゅっぱつはむずかしいです 
\\	もう一度、予約を確認した。	もういちどよやくをかくにんした 
\\	私の町は東京の北に位置します。	わたしのまちはとうきょうのきたにいちします 
\\	今日は山がはっきり見える。	きょうはやまがはっきりみえる 
\\	つまり、あなたは何も知らないのですね。	つまりあなたはなにもしらないのですね 
\\	次の駅は通過します。	つぎのえきはつうかします 
\\	あの池はとても深い。	あのいけはとてもふかい 
\\	会長の候補は3人います。	かいちょうのこうほはさんにんいます 
\\	彼はプロのサッカー選手だ。	かれはぷろのさっかーせんしゅだ 
\\	雨の音が聞こえる。	あめのおとがきこえる 
\\	夏には川辺で花火大会があります。	なつにはかわべではなびたいかいがあります 
\\	今、ジャズダンスの基本を習っています。	いまじゃずだんすのきほんをならっています 
\\	その選手は新しいチームと契約した。	そのせんしゅはあたらしいちーむとけいやくした 
\\	新しい星が発見された。	あたらしいほしがはっけんされた 
\\	彼女は当時、まだ3才だった。	かのじょはとうじまださんさいだった 
\\	自宅に電話を下さい。	じたくにでんわをください 
\\	ここは空気がきれいです。	ここはくうきがきれいです 
\\	授業の終わりに復習の時間を設けた。	じゅぎょうのおわりにふくしゅうのじかんをもうけた 
\\	私は歴史に興味があります。	わたしはれきしにきょうみがあります 
\\	各国の代表がニューヨークに集まった。	かっこくのだいひょうがにゅーよーくにあつまった 
\\	特に質問はありません。	とくにしつもんはありません 
\\	科学の授業で実験をした。	かがくのじゅぎょうでじっけんをした 
\\	私は夏に京都を訪れました。	わたしはなつにきょうとをおとずれました 
\\	これは日本一古い建物です。	これはにほんいちふるいたてものです 
\\	私の会社はあのビルの8階です。	わたしのかいしゃはあのびるのはちかいです 
\\	彼らはバスケットの選手です。	かれらはばすけっとのせんしゅです 
\\	少年たちがサッカーをしている。	しょうねんたちがさっかーをしている 
\\	彼はヨーロッパ諸国を旅行した。	かれはよーろっぱしょこくをりょこうした 
\\	もちろん一緒に行きます。	もちろんいっしょにいきます 
\\	後で事務所に来てください。	あとでじむしょにきてください 
\\	彼は失敗の責任を取って、会社を辞めた。	かれはしっぱいのせきにんをとってかいしゃをやめた 
\\	かつて私が学生だった頃のことです。	かつてわたしががくせいだったころのことです 
\\	このサービスは週末に限ります。	このさーびすはしゅうまつにかぎります 
\\	彼は海外での生活が長いです。	かれはかいがいでのせいかつがながいです 
\\	この条件では厳し過ぎます。	このじょうけんではきびしすぎます 
\\	彼は家庭を大切にしている。	かれはかていをたいせつにしている 
\\	彼は化学の教授です。	かれはかがくのきょうじゅです 
\\	その国の人口はどのくらいですか。	そのくにのじんこうはどのくらいですか 
\\	海は広くて大きい。	うみはひろくておおきい 
\\	判断の基準が示された。	はんだんのきじゅんがしめされた 
\\	2003年にアメリカはイラクを攻撃した。	にせんさんねんにあめりかはいらくをこうげきした 
\\	今は便利さとスピードの時代だ。	いまはべんりさとすぴーどのじだいだ 
\\	今はあなたにとって大事な時期です。	いまはあなたにとってだいじなじきです 
\\	この工場ではエアコンを製造しています。	このこうじょうではえあこんをせいぞうしています 
\\	大型のテレビを買った。	おおがたのてれびをかった 
\\	この周辺には大学が多い。	このしゅうへんにはだいがくがおおい 
\\	彼と私はいつも意見が異なる。	かれとわたしはいつもいけんがことなる 
\\	逃げていた犯人が逮捕されました。	にげていたはんにんがたいほされました 
\\	関連のセクションへ連絡した。	かんれんのせくしょんへれんらくした 
\\	父は貿易の仕事をしています。	ちちはぼうえきのしごとをしています 
\\	彼は優秀な社員です。	かれはゆうしゅうなしゃいんです 
\\	ファイルを結合して1つにしました。	ふぁいるをけつごうしてひとつにしました 
\\	みんなで自然を守りましょう。	みんなでしぜんをまもりましょう 
\\	あなたは何かスポーツをしていますか。	あなたはなにかすぽーつをしていますか 
\\	患者は眠っています。	かんじゃはねむっています 
\\	なぜ来なかったの。	なぜこなかったの 
\\	彼はよく映画を見ます。	かれはよくえいがをみます 
\\	ここに並んでください。	ここにならんでください 
\\	このクラスの生徒は30人です。	このくらすのせいとはさんじゅうにんです 
\\	そのままお待ちください。	そのままおまちください 
\\	今度はどこに行きたいですか。	こんどはどこにいきたいですか 
\\	お菓子をみんなで分けました。	おかしをみんなでわけました 
\\	いすを二階に運んでください。	いすをにかいにはこんでください 
\\	このプログラムはとても複雑です。	このぷろぐらむはとてもふくざつです 
\\	月は地球のまわりを回っています。	つきはちきゅうのまわりをまわっています 
\\	妹は高校に通っています。	いもうとはこうこうにかよっています 
\\	工事の音がうるさい。	こうじのおとがうるさい 
\\	私たちは毎日散歩をします。	わたしたちはまいにちさんぽをします 
\\	彼は帰国しました。	かれはきこくしました 
\\	雨が降っています。	あめがふっています 
\\	あれから14年が経ちました。	あれからじゅうよねんがたちました 
\\	きれいな花が咲きました。	きれいなはながさきました 
\\	午後は会議に出席します。	ごごはかいぎにしゅっせきします 
\\	日本に住む外国人が増えています。	にほんにすむがいこくじんがふえています 
\\	この道路は3年前にできました。	このどうろはさんねんまえにできました 
\\	地球は丸い。	ちきゅうはまるい 
\\	あなたは英語が話せますか。	あなたはえいごがはなせますか 
\\	その地域は今、危険だ。	そのちいきはいまきけんだ 
\\	彼女は普通の女の子だ。	かのじょはふつうのおんなのこだ 
\\	この単語の意味を説明してください。	このたんごのいみをせつめいしてください 
\\	お菓子を友達に半分あげた。	おかしをともだちにはんぶんあげた 
\\	車に注意してください。	くるまにちゅういしてください 
\\	安全が第一です。	あんぜんがだいいちです 
\\	友人から手紙をもらいました。	ゆうじんからてがみをもらいました 
\\	もっと努力しよう。	もっとどりょくしよう 
\\	父は胸の手術をした。	ちちはむねのしゅじゅつをした 
\\	明日の会議の準備をした。	あすのかいぎのじゅんびをした 
\\	彼は約束を守る人です。	かれはやくそくをまもるひとです 
\\	彼は血を見て驚いた。	かれはちをみておどろいた 
\\	日本は地震が多いです。	にほんはじしんがおおいです 
\\	私たちはよく話し合いました。	わたしたちはよくはなしあいました 
\\	台風で木が倒れた。	たいふうできがたおれた 
\\	私は彼を許しました。	わたしはかれをゆるしました 
\\	クリスマスに雪が降りました。	くりすますにゆきがふりました 
\\	私は動物が大好きです。	わたしはどうぶつがだいすきです 
\\	この靴は私の足に合っている。	このくつはわたしのあしにあっている 
\\	この国は平和です。	このくにはへいわです 
\\	これは非常に重要です。	これはひじょうにじゅうようです 
\\	今朝彼女は学校に遅れました。	けさかのじょはがっこうにおくれました 
\\	もし雨が降ったら、行きません。	もしあめがふったらいきません 
\\	私は去年、大学を卒業した。	わたしはきょねんだいがくをそつぎょうした 
\\	私は農業を勉強しています。	わたしはのうぎょうをべんきょうしています 
\\	空港までリムジンバスで行った。	くうこうまでりむじんばすでいった 
\\	昨日珍しい果物を食べました。	きのうめずらしいくだものをたべました 
\\	私たちは駅へ急ぎました。	わたしたちはえきへいそぎました 
\\	明日からタバコを止めるつもりです。	あしたからたばこをやめるつもりです 
\\	必ずシートベルトを着けて下さい。	かならずしーとべるとをつけてください 
\\	数字は苦手です。	すうじはにがてです 
\\	やっと仕事が終わりました。	やっとしごとがおわりました 
\\	彼女は旅行が好きです。	かのじょはりょこうがすきです 
\\	私は音楽を聞くのが好きだ。	わたしはおんがくをきくのがすきだ 
\\	小さな川を渡りました。	ちいさなかわをわたりました 
\\	例えば、このソフトで日本語を勉強することができます。	たとえばこのそふとでにほんごをべんきょうすることができます 
\\	家の近くに小学校があります。	いえのちかくにしょうがっこうがあります 
\\	壁に絵が掛かっている。	かべにえがかかっている 
\\	ラジオから美しい音楽が流れています。	らじおからうつくしいおんがくがながれています 
\\	私の夢は先生になることです。	わたしのゆめはせんせいになることです 
\\	鳥が窓から逃げたよ。	とりがまどからにげたよ 
\\	時が過ぎるのは早い。	ときがすぎるのははやい 
\\	東京は日本の首都です。	とうきょうはにっぽんのしゅとです 
\\	彼は高校教師だ。	かれはこうこうきょうしだ 
\\	相談したいことがあります。	そうだんしたいことがあります 
\\	食事の用意ができました。	しょくじのよういができました 
\\	夜のニュースを見ましたか。	よるのにゅーすをみましたか 
\\	新しい法律ができた。	あたらしいほうりつができた 
\\	大変なことが起こりました。	たいへんなことがおこりました 
\\	2003年にイラクで戦争があった。	にせんさんねんにいらくでせんそうがあった 
\\	父は休みの日はずっとテレビを見ている。	ちちはやすみのひはずっとてれびをみている 
\\	親が子供を守るのは自然なことだ。	おやがこどもをまもるのはしぜんなことだ 
\\	日本は島国です。	にっぽんはしまぐにです 
\\	それは良い考えです。	それはいいかんがえです 
\\	私はその番組をビデオに撮った。	わたしはそのばんぐみをびでおにとった 
\\	彼はマンションに住んでいます。	かれはまんしょんにすんでいます 
\\	私は健康な生活を送っています。	わたしはけんこうなせいかつをおくっています 
\\	この通りはにぎやかですね。	このとおりはにぎやかですね 
\\	彼女はとても喜びました。	かのじょはとてもよろこびました 
\\	昨日、母が入院しました。	きのうははがにゅういんしました 
\\	あの町は緑を増やしています。	あのまちはみどりをふやしています 
\\	彼女は三人の子を育てました。	かのじょはさんにんのこをそだてました 
\\	今、部屋の温度は25度だ。	いまへやのおんどはにじゅうごどだ 
\\	この雑誌はよく売れています。	このざっしはよくうれています 
\\	私たちは来月結婚します。	わたしたちはらいげつけっこんします 
\\	今日は上司と食事をする。	きょうはじょうしとしょくじをする 
\\	今日は風が強いです。	きょうはかぜがつよいです 
\\	なるべく早く来て下さい。	なるべくはやくきてください 
\\	彼女は明るい性格です。	かのじょはあかるいせいかくです 
\\	父は安全運転だ。	ちちはあんぜんうんてんだ 
\\	そこでしばらく休んでいます。	そこでしばらくやすんでいます 
\\	富士山の高さは3,776メートルです。	ふじさんのたかさはさんぜんななひゃく-ななじゅうろくめーとるです 
\\	彼にビデオを借りました。	かれにびでおをかりました 
\\	その番組は来週放送されます。	そのばんぐみはらいしゅうほうそうされます 
\\	最近、若い人たちの就職が難しくなっています。	さいきんわかいひとたちのしゅうしょくがむずかしくなっています 
\\	この大きさの封筒が欲しいのですが。	このおおきさのふうとうがほしいのですが 
\\	ケータイをなくして困っています。	けーたいをなくしてこまっています 
\\	毎日、この道を通ります。	まいにちこのみちをとおります 
\\	地震でガスが止まった。	じしんでがすがとまった 
\\	その夫婦はとても仲がいい。	そのふうふはとてもなかがいい 
\\	彼はケーキにろうそくを立てた。	かれはけーきにろうそくをたてた 
\\	髪がだいぶ伸びたね。	かみがだいぶのびたね 
\\	隣の部屋からテレビの音が聞こえる。	となりのへやからてれびのおとがきこえる 
\\	私は生き物を殺すのが嫌いだ。	わたしはいきものをころすのがきらいだ 
\\	今年の夏休みには静かな所へ行きたい。	ことしのなつやすみにはしずかなところへいきたい 
\\	グループの中で意見が分かれました。	ぐるーぷのなかでいけんがわかれました 
\\	もっと詳しく説明してください。	もっとくわしくせつめいしてください 
\\	彼の計算は正確です。	かれのけいさんはせいかくです 
\\	彼は野球がうまい。	かれはやきゅうがうまい 
\\	転んでひざを打ちました。	ころんでひざをうちました 
\\	そこは工業都市だ。	そこはこうぎょうとしだ 
\\	父はサラリーマンです。	ちちはさらりーまんです 
\\	彼は有名な歌手です。	かれはゆうめいなかしゅです 
\\	ゴミを捨ててください。	ごみをすててください 
\\	机を窓の傍に移しました。	つくえをまどのそばにうつしました 
\\	それは正しい答えです。	それはただしいこたえです 
\\	今年の夏はクーラーがよく売れた。	ことしのなつはくーらーがよくうれた 
\\	私はこの番組が好きです。	わたしはこのばんぐみがすきです 
\\	上司に代わって会議に出た。	じょうしにかわってかいぎにでた 
\\	午後8時に大阪に着きます。	ごごはちじにおおさかにつきます 
\\	バスで行こう。	ばすでいこう 
\\	息子は中学に通っています。	むすこはちゅうがくにかよっています 
\\	私はその歌を知らなかった。	わたしはそのうたをしらなかった 
\\	この電子辞書はとても薄い。	このでんしじしょはとてもうすい 
\\	最近、食事を減らしています。	さいきんしょくじをへらしています 
\\	科学は常に進歩している。	かがくはつねにしんぽしている 
\\	約束は守ります。	やくそくはまもります 
\\	あと15分で出発です。	あとじゅうごふんでしゅっぱつです 
\\	部屋の家具を動かしました。	へやのかぐをうごかしました 
\\	両親に彼女を紹介した。	りょうしんにかのじょをしょうかいした 
\\	強風で木が倒れた。	きょうふうできがたおれた 
\\	この席、空いてますか。	このせきあいてますか 
\\	昨日の夜、熱が出ました。	きのうのよるねつがでました 
\\	このベルトの値段は2500円でした。	このべるとのねだんはにせんごひゃくえんでした 
\\	このケーブルの長さは1メートルです。	このけーぶるのながさはいちめーとるです 
\\	この学校は1クラス30人です。	このがっこうはひとくらすさんじゅうにんです 
\\	隣の家には犬がいます。	となりのいえにはいぬがいます 
\\	トラックを運転できますか。	とらっくをうんてんできますか 
\\	彼は大使館に勤めています。	かれはたいしかんにつとめています 
\\	おととい、昔の友人が亡くなった。	おとといむかしのゆうじんがなくなった 
\\	明日、うちでパーティーを開きます。	あしたうちでぱーてぃーをひらきます 
\\	この犬はとても賢い。	このいぬはとてもかしこい 
\\	うまい寿司屋を見つけたよ。	うまいすしやをみつけたよ 
\\	あなたは私にとって特別な人です。	あなたはわたしにとってとくべつなひとです 
\\	私の教室は3階にあります。	わたしのきょうしつはさんがいにあります 
\\	クリスタルガラスの花瓶を買いました。	くりすたるがらすのかびんをかいました 
\\	犬がしっぽを振っている。	いぬがしっぽをふっている 
\\	途中で財布を落としました。	とちゅうでさいふをおとしました 
\\	夕方、雨が降りました。	ゆうがたあめがふりました 
\\	私は日本語コースを取っています。	わたしはにほんごこーすをとっています 
\\	彼はアパートに住んでいます。	かれはあぱーとにすんでいます 
\\	私は月に3冊くらい小説を読みます。	わたしはつきにさんさつくらいしょうせつをよみます 
\\	ジャズのレコードをかけました。	じゃずのれこーどをかけました 
\\	ドアをどんどんとたたいた。	どあをどんどんとたたいた 
\\	私はジムに通っています。	わたしはじむにかよっています 
\\	これは秘密です。	これはひみつです 
\\	母は料理が得意です。	はははりょうりがとくいです 
\\	食べ過ぎておなかが苦しい。	たべすぎておなかがくるしい 
\\	彼はキリストを信じている。	かれはきりすとをしんじている 
\\	道に何か落ちています。	みちになにかおちています 
\\	カメラが壊れた。	かめらがこわれた 
\\	少し音量を下げてください。	すこしおんりょうをさげてください 
\\	1年間、アメリカに留学しました。	いちねんかんあめりかにりゅうがくしました 
\\	5時までに家に帰るのは無理です。	ごじまでにいえにかえるのはむりです 
\\	テーブルの上に花を飾りました。	てーぶるのうえにはなをかざりました 
\\	ロシアは日本の北にあります。	ろしあはにっぽんのきたにあります 
\\	彼はニューヨークの友達を訪ねた。	かれはにゅーよーくのともだちをたずねた 
\\	彼女はその会話をテープに録音した。	かのじょはそのかいわをてーぷにろくおんした 
\\	彼は牧場で馬に乗った。	かれはぼくじょうでうまにのった 
\\	彼の長男は8才です。	かれのちょうなんははっさいです 
\\	近所にカナダ人が住んでいる。	きんじょにかなだじんがすんでいる 
\\	兄は数学の先生です。	あにはすうがくのせんせいです 
\\	妹は九月にアメリカの大学に入学します。	いもうとはくがつにあめりかのだいがくににゅうがくします 
\\	彼女は黒いドレスを着ています。	かのじょはくろいどれすをきています 
\\	私はハンバーガーを頼みました。	わたしははんばーがーをたのみました 
\\	私は料理をテーブルに並べた。	わたしはりょうりをてーぶるにならべた 
\\	彼からメールを受け取りました。	かれからめーるをうけとりました 
\\	太陽が雲に隠れた。	たいようがくもにかくれた 
\\	あの村にはお年寄りがたくさん住んでいます。	あのむらにはおとしよりがたくさんすんでいます 
\\	この料理は味が薄い。	このりょうりはあじがうすい 
\\	大切な用事を思い出しました。	たいせつなようじをおもいだしました 
\\	彼女は細かいことにうるさい。	かのじょはこまかいことにうるさい 
\\	成績が上がりました。	せいせきがあがりました 
\\	母は涙を流していた。	はははなみだをながしていた 
\\	横の長さは1メートルです。	よこのながさはいちめーとるです 
\\	靴下に穴が開いている。	くつしたにあながあいている 
\\	私の一番好きな季節は春です。	わたしのいちばんすきなきせつははるです 
\\	夏はビールがとても美味しい。	なつはびーるがとてもおいしい 
\\	冷蔵庫が故障しました。	れいぞうこがこしょうしました 
\\	会社の裏に公園があります。	かいしゃのうらにこうえんがあります 
\\	教科書の36ページを開いてください。	きょうかしょのさんじゅうろくぺーじをひらいてください 
\\	姉が空港まで迎えに来てくれます。	あねがくうこうまでむかえにきてくれます 
\\	私は毎日たくさん野菜を食べます。	わたしはまいにちたくさんやさいをたべます 
\\	彼に手紙を渡しました。	かれにてがみをわたしました 
\\	母に花を贈った。	ははにはなをおくった 
\\	娘が入学試験に合格しました。	むすめがにゅうがくしけんにごうかくしました 
\\	彼は売上をグラフにした。	かれはうりあげをぐらふにした 
\\	公園に小さな池があります。	こうえんにちいさないけがあります 
\\	部屋の戸が開いています。	へやのとがあいています 
\\	息子は中学生です。	むすこはちゅうがくせいです 
\\	玄関に花を飾りました。	げんかんにはなをかざりました 
\\	写真をたくさん撮りました。	しゃしんをたくさんとりました 
\\	夏休みにお祖母ちゃんの家へ行きました。	なつやすみにおばあちゃんのうちへいきました 
\\	私は毎年、海外旅行に行きます。	わたしはまいとしかいがいりょこうにいきます 
\\	地図を見て来てください。	ちずをみてきてください 
\\	海岸を散歩しましょう。	かいがんをさんぽしましょう 
\\	急に用事を思い出した。	きゅうにようじをおもいだした 
\\	あの女の子を知っていますか。	あのおんなのこをしっていますか 
\\	手紙をポストに入れました。	てがみをぽすとにいれました 
\\	東の空が暗いです。	ひがしのそらがくらいです 
\\	チョコレートとケーキを両方ください。	ちょこれーととけーきをりょうほうください 
\\	友達は向こうにいます。	ともだちはむこうにいます 
\\	忙しいので手伝ってください。	いそがしいのでてつだってください 
\\	来年から給料が上がります。	らいねんからきゅうりょうがあがります 
\\	明日は理科のテストです。	あしたはりかのてすとです 
\\	私たちは朝からあちこち散歩しました。	わたしたちはあさからあちこちさんぽしました 
\\	このはさみはよく切れますね。	このはさみはよくきれますね 
\\	友達と野球をしました。	ともだちとやきゅうをしました 
\\	彼は採用試験に合格した。	かれはさいようしけんにごうかくした 
\\	台風が近づいている。	たいふうがちかづいている 
\\	あの会社の規則は厳しいです。	あのかいしゃのきそくはきびしいです 
\\	遠くに船が見えます。	とおくにふねがみえます 
\\	兄はゴルフを始めました。	あにはごるふをはじめました 
\\	母は看護婦です。	はははかんごふです 
\\	私は毎朝ラジオを聴きます。	わたしはまいあさらじおをききます 
\\	ついに素晴らしい女性にあえた。	ついにすばらしいじょせいにあえた 
\\	タクシーを呼んでください。	たくしーをよんでください 
\\	来月から大学生になります。	らいげつからだいがくせいになります 
\\	日本語の教科書を忘れた。	にほんごのきょうかしょをわすれた 
\\	電気をつけてもらえる?	でんきをつけてもらえる 
\\	急な坂道を上った。	きゅうなさかみちをのぼった 
\\	もっとゆっくり話してください。	もっとゆっくりはなしてください 
\\	彼は3人兄弟です。	かれはさんにんきょうだいです 
\\	夕食に牛肉を買った。	ゆうしょくにぎゅうにくをかった 
\\	彼の昇進は確かだ。	かれのしょうしんはたしかだ 
\\	新しい家にはもう慣れましたか。	あたらしいいえにはもうなれましたか 
\\	その辞書はとても厚い。	そのじしょはとてもあつい 
\\	図書館に本を返した。	としょかんにほんをかえした 
\\	息子の中学校は家から5分です。	むすこのちゅうがっこうはいえからごふんです 
\\	私は森を歩くのが好きです。	わたしはもりをあるくのがすきです 
\\	次の駅で降ります。	つぎのえきでおります 
\\	私はこの曲が大好きです。	わたしはこのきょくがだいすきです 
\\	はさみは机の奥にあった。	はさみはつくえのおくにあった 
\\	彼は歌が得意です。	かれはうたがとくいです 
\\	やっと熱が下がった。	やっとねつがさがった 
\\	レストランでインド料理を食べました。	れすとらんでいんどりょうりをたべました 
\\	私は医者に相談した。	わたしはいしゃにそうだんした 
\\	支払いはカードでお願いします。	しはらいはかーどでおねがいします 
\\	今日は北風が吹いている。	きょうはきたかぜがふいている 
\\	信号が青になった。	しんごうがあおになった 
\\	これは有名な画家の絵です。	これはゆうめいながかのえです 
\\	車を修理に出した。	くるまをしゅうりにだした 
\\	私は地下鉄で通勤しています。	わたしはちかてつでつうきんしています 
\\	ベッドで赤ちゃんが眠っています。	べっどであかちゃんがねむっています 
\\	この本はすごく面白かった。	このほんはすごくおもしろかった 
\\	友人が離婚しました。	ゆうじんがりこんしました 
\\	うちの息子は来年、小学生になります。	うちのむすこはらいねんしょうがくせいになります 
\\	野菜がよく育っている。	やさいがよくそだっている 
\\	いすの数を数えてください。	いすのかずをかぞえてください 
\\	兄はアルバイトをしています。	あにはあるばいとをしています 
\\	母は買い物に出かけています。	はははかいものにでかけています 
\\	私は石につまづいた。	わたしはいしにつまづいた 
\\	この薬を必ず飲んでください。	このくすりをかならずのんでください 
\\	毎年給料が上がる。	まいねんきゅうりょうがあがる 
\\	ねじは左に回すと外れます。	ねじはひだりにまわすとはずれます 
\\	会議で書類のコピーを配った。	かいぎでしょるいのこぴーをくばった 
\\	車が電柱にぶつかった。	くるまがでんちゅうにぶつかった 
\\	彼はタクシーの運転手です。	かれはたくしーのうんてんしゅです 
\\	私たちは教会で結婚式をしました。	わたしたちはきょうかいでけっこんしきをしました 
\\	旅行のためにたくさんフィルムを買った。	りょこうのためにたくさんふぃるむをかった 
\\	未来は誰にも分からない。	みらいはだれにもわからない 
\\	あの人は高校の校長だ。	あのひとはこうこうのこうちょうだ 
\\	私はデパートで靴を買った。	わたしはでぱーとでくつをかった 
\\	彼が外国に行くのを見送りました。	かれががいこくにいくのをみおくりました 
\\	こっちを向いてください。	こっちをむいてください 
\\	風で木が揺れています。	かぜできがゆれています 
\\	昨日、髪を切りました。	きのうかみをきりました 
\\	この箱は重い。	このはこはおもい 
\\	旅行で楽しい思い出ができました。	りょこうでたのしいおもいでができました 
\\	彼はベッドで寝ています。	かれはべっどでねています 
\\	新幹線で京都に行きました。	しんかんせんできょうとにいきました 
\\	うちのネコが子猫を生みました。	うちのねこがこねこをうみました 
\\	彼の奥さんはきれいな方です。	かれのおくさんはきれいなかたです 
\\	私たちは駅の階段をかけ上がった。	わたしたちはえきのかいだんをかけあがった 
\\	高校の時の先生を結婚式に招待した。	こうこうのときのせんせいをけっこんしきにしょうたいした 
\\	寒かったのでコートを着た。	さむかったのでこーとをきた 
\\	山が燃えています。	やまがもえています 
\\	大学の食堂は安い。	だいがくのしょくどうはやすい 
\\	この部屋は天井が高いですね。	このへやはてんじょうがたかいですね 
\\	今から課のミーティングがある。	いまからかのみーてぃんぐがある 
\\	靴が汚れた。	くつがよごれた 
\\	レストランを予約しました。	れすとらんをよやくしました 
\\	一昨年初めて京都へ旅行しました。	おととしはじめてきょうとへりょこうしました 
\\	休日は家でよくテレビを見ます。	きゅうじつはいえでよくてれびをみます 
\\	レストランでピザを注文しました。	れすとらんでぴざをちゅうもんしました 
\\	ここの家賃は12万円です。	ここのやちんはじゅうにまんえんです 
\\	公園を散歩しようか。	こうえんをさんぽしようか 
\\	毎朝、通勤に30分かかります。	まいあさつうきんにさんじゅっぷんかかります 
\\	ノートを開いてください。	のーとをひらいてください 
\\	彼は会社を辞めます。	かれはかいしゃをやめます 
\\	ファイルのダウンロードの仕方が分かりません。	ふぁいるのだうんろーどのしかたがわかりません 
\\	お母さんは台所にいます。	おかあさんはだいどころにいます 
\\	ここに来年、家が建ちます。	ここにらいねんいえがたちます 
\\	運転手がブレーキを踏んだ。	うんてんしゅがぶれーきをふんだ 
\\	会議で書類を配った。	かいぎでしょるいをくばった 
\\	少女は母親の手を握った。	しょうじょはははおやのてをにぎった 
\\	娘は今、バイオリンを練習しています。	むすめはいまばいおりんをれんしゅうしています 
\\	風の強さに驚きました。	かぜのつよさにおどろきました 
\\	明日、久しぶりに友達に会います。	あしたひさしぶりにともだちにあいます 
\\	往復切符をください。	おうふくきっぷをください 
\\	大勢で食事に出かけました。	おおぜいでしょくじにでかけました 
\\	週末、親しい友達を家に呼んだ。	しゅうまつしたしいともだちをいえによんだ 
\\	この番号に電話してください。	このばんごうにでんわしてください 
\\	ワインを少し飲みました。	わいんをすこしのみました 
\\	彼は頭にタオルを巻いていた。	かれはあたまにたおるをまいていた 
\\	ここの海はとても青い。	ここのうみはとてもあおい 
\\	彼の様子がおかしい。	かれのようすがおかしい 
\\	私たちは歩いて橋を渡った。	わたしたちはあるいてはしをわたった 
\\	牛が草を食べています。	うしがくさをたべています 
\\	グラスに氷を入れてください。	ぐらすにこおりをいれてください 
\\	彼女は笑顔で挨拶した。	かのじょはえがおであいさつした 
\\	早く顔を洗いなさい。	はやくかおをあらいなさい 
\\	彼女は道を尋ねた。	かのじょはみちをたずねた 
\\	彼は子供のおしりを叩いた。	かれはこどものおしりをたたいた 
\\	廊下は走らないでください。	ろうかははしらないでください 
\\	郵便局はどこですか。	ゆうびんきょくはどこですか 
\\	皿を落として割った。	さらをおとしてわった 
\\	彼女は来年、大学に進学する。	かのじょはらいねんだいがくにしんがくする 
\\	トイレを掃除しましたか。	といれをそうじしましたか 
\\	桜の花が咲きました。	さくらのはながさきました 
\\	車は駐車場に止めてください。	くるまはちゅうしゃじょうにとめてください 
\\	妹はすぐに泣く。	いもうとはすぐになく 
\\	屋根にカラスが止まっています。	やねにからすがとまっています 
\\	私が中をご案内します。	わたしがなかをごあんないします 
\\	友達とキャンプに行った。	ともだちときゃんぷにいった 
\\	彼は大きな荷物を持ってきた。	かれはおおきなにもつをもってきた 
\\	人に迷惑をかけてはいけません。	ひとにめいわくをかけてはいけません 
\\	先月、友達の結婚式があった。	せんげつともだちのけっこんしきがあった 
\\	昨日、美術館に行きました。	きのうびじゅつかんにいきました 
\\	彼は約束を破った。	かれはやくそくをやぶった 
\\	彼女は愛をこめて手紙を書いた。	かのじょはあいをこめててがみをかいた 
\\	黒のボールペンはありますか。	くろのぼーるぺんはありますか 
\\	私は彼のプロポーズを断った。	わたしはかれのぷろぽーずをことわった 
\\	その店は色々な品物を売っている。	そのみせはいろいろなしなものをうっている 
\\	このカップはプラスチックです。	このかっぷはぷらすちっくです 
\\	私は彼に本を貸しています。	わたしはかれにほんをかしています 
\\	ボートが川に沈んだ。	ぼーとがかわにしずんだ 
\\	波がきらきら光っていました。	なみがきらきらひかっていました 
\\	私は右手で字を書きます。	わたしはみぎてでじをかきます 
\\	信号が赤に変わりました。	しんごうがあかにかわりました 
\\	ニワトリは卵を産みます。	にわとりはたまごをうみます 
\\	今日は疲れました。	きょうはつかれました 
\\	男の子たちがサッカーをしている。	おとこのこたちがさっかーをしている 
\\	彼女はダイエットのために泳いでいる。	かのじょはだいえっとのためにおよいでいる 
\\	白線の内側に下がってください。	はくせんのうちがわにさがってください 
\\	毎朝コーヒーを飲むのが習慣です。	まいあさこーひーをのむのがしゅうかんです 
\\	ボールをこっちに投げてください。	ぼーるをこっちになげてください 
\\	カラーコピーは一枚幾らですか。	からーこぴーはいちまいいくらですか 
\\	お正月にはたいてい、家族が集まる。	おしょうがつにはたいていかぞくがあつまる 
\\	私の趣味は映画とテニスです。	わたしのしゅみはえいがとてにすです 
\\	誰かが私のカバンを盗みました。	だれかがわたしのかばんをぬすみました 
\\	昔、ピアノを習っていました。	むかしぴあのをならっていました 
\\	池に鯉がいます。	いけにこいがいます 
\\	彼は大学院に進みました。	かれはだいがくいんにすすみました 
\\	あの店には何度も行きました。	あのみせにはなんどもいきました 
\\	彼女の家族に不幸があった。	かのじょのかぞくにふこうがあった 
\\	昨夜恐ろしい夢を見た。	ゆうべおそろしいゆめをみた 
\\	母が今日退院します。	ははがきょうたいいんします 
\\	靴に泥が付いています。	くつにどろがついています 
\\	この荷物の重さを計ってください。	このにもつのおもさをはかってください 
\\	この町では商業が盛んだ。	このまちではしょうぎょうがさかんだ 
\\	冬はよくスキーに行きます。	ふゆはよくすきーにいきます 
\\	生まれて初めて布団を敷いた。	うまれてはじめてふとんをしいた 
\\	それを聞いて安心しました。	それをきいてあんしんしました 
\\	母は糸を針に通した。	はははいとをはりにとおした 
\\	荷物がなかなか届きません。	にもつがなかなかとどきません 
\\	彼女は人形をたくさん持っています。	かのじょはにんぎょうをたくさんもっています 
\\	私は夏休みにプールに行った。	わたしはなつやすみにぷーるにいった 
\\	彼は銀メダルを取った。	かれはぎんめだるをとった 
\\	いい匂いがする。	いいにおいがする 
\\	もうすぐこのホームに電車が来ます。	もうすぐこのほーむにでんしゃがきます 
\\	彼は交通事故を起こした。	かれはこうつうじこをおこした 
\\	良い友達がいて私は幸せだ。	いいともだちがいてわたしはしあわせだ 
\\	彼は大きく息を吸った。	かれはおおきくいきをすった 
\\	この川は浅いです。	このかわはあさいです 
\\	今日、工場の見学に行きました。	きょうこうじょうのけんがくにいきました 
\\	私は濃い味が好きだ。	わたしはこいあじがすきだ 
\\	彼女が私のケータイを壊した。	かのじょがわたしのけーたいをこわした 
\\	授業に間に合いました。	じゅぎょうにまにあいました 
\\	これを彼に届けてください。	これをかれにとどけてください 
\\	これは桜の葉です。	これはさくらのはです 
\\	エレベーターで下に降りましょう。	えれべーたーでしたにおりましょう 
\\	メモを取ってください。	めもをとってください 
\\	彼は立派な人です。	かれはりっぱなひとです 
\\	京都では旅館に泊まりました。	きょうとではりょかんにとまりました 
\\	凄い雨になった。	すごいあめになった 
\\	朝ご飯にはいつもパンを食べる。	あさごはんにはいつもぱんをたべる 
\\	大きな音にびっくりしました。	おおきなおとにびっくりしました 
\\	インターネットはとても便利です。	いんたーねっとはとてもべんりです 
\\	コーヒーに砂糖は入れますか。	こーひーにさとうはいれますか 
\\	今日は休みです。	きょうはやすみです 
\\	そのラジオ番組はもう録音しました。	そのらじおばんぐみはもうろくおんしました 
\\	そんな言い方をしてはいけません。	そんないいかたをしてはいけません 
\\	仕事の帰りにビールを飲んだ。	しごとのかえりにびーるをのんだ 
\\	週末は家でゆっくりします。	しゅうまつはうちでゆっくりします 
\\	ズボンが汚れた。	ずぼんがよごれた 
\\	お金が足りなくて買えなかった。	おかねがたりなくてかえなかった 
\\	道で財布を拾った。	みちでさいふをひろった 
\\	彼女は祭りが大好きです。	かのじょはまつりがだいすきです 
\\	昨日、車の博物館に行った。	きのうくるまのはくぶつかんにいった 
\\	私の好きな色は青です。	わたしのすきないろはあおです 
\\	くつ売り場はどこですか。	くつうりばはどこですか 
\\	赤ちゃんが自動車のおもちゃで遊んでいる。	あかちゃんがじどうしゃのおもちゃであそんでいる 
\\	彼は金持ちです。	かれはかねもちです 
\\	明日の試験、頑張ってね。	あしたのしけんがんばってね 
\\	ひき肉を200グラムください。	ひきにくをにひゃくぐらむください 
\\	私は毎朝コーヒーを飲みます。	わたしはまいあさこーひーをのみます 
\\	桜は三月か四月に咲きます。	さくらはさんがつかしがつにさきます 
\\	虫に腕を刺されました。	むしにうでをさされました 
\\	みんなでテントを張りました。	みんなでてんとをはりました 
\\	池でボートに乗りました。	いけでぼーとにのりました 
\\	彼は交通事故に遭った。	かれはこうつうじこにあった 
\\	彼はボールを投げた。	かれはぼーるをなげた 
\\	彼はオートバイに乗っている。	かれはおーとばいにのっている 
\\	この町には外人が少ない。	このまちにはがいじんがすくない 
\\	私は猫が大好きです。	わたしはねこがだいすきです 
\\	親切にしてくださってどうもありがとうございます。	しんせつにしてくださってどうもありがとうございます 
\\	是非、うちに来てください。	ぜひうちにきてください 
\\	毎朝、通学に1時間かかる。	まいあさつうがくにいちじかんかかる 
\\	鍵を彼に預けた。	かぎをかれにあずけた 
\\	おまけを付けました。	おまけをつけました 
\\	私はダンスを習っています。	わたしはだんすをならっています 
\\	日曜日に本棚を組み立てます。	にちようびにほんだなをくみたてます 
\\	壁にペンキを塗っています。	かべにぺんきをぬっています 
\\	彼女は左手で字を書く。	かのじょはひだりてでじをかく 
\\	妹とひどい喧嘩をした。	いもうととひどいけんかをした 
\\	別々に払いましょう。	べつべつにはらいましょう 
\\	私は毎週母に電話をします。	わたしはまいしゅうははにでんわをします 
\\	今日の気温は26度です。	きょうのきおんはにじゅうろくどです 
\\	ゴールキーパーがボールを蹴った。	ごーるきーぱーがぼーるをけった 
\\	外国語を習うのは難しい。	がいこくごをならうのはむずかしい 
\\	彼女は着物がよく似合います。	かのじょはきものがよくにあいます 
\\	私は毎朝、牛乳を飲む。	わたしはまいあさぎゅうにゅうをのむ 
\\	この住所に行ってください。	このじゅうしょにいってください 
\\	先週は海に行った。	せんしゅうはうみにいった 
\\	日本に来て半年になります。	にっぽんにきてはんとしになります 
\\	バスが飛行場に着きました。	ばすがひこうじょうにつきました 
\\	誰か救急車を呼んでください。	だれかきゅうきゅうしゃをよんでください 
\\	そこの階段を下りてください。	そこのかいだんをおりてください 
\\	デザートに果物を食べましょう。	でざーとにくだものをたべましょう 
\\	この文には間違いがあります。	このぶんにはまちがいがあります 
\\	これはあなたのカバンですか。	これはあなたのかばんですか 
\\	彼と近くの映画館に行きました。	かれとちかくのえいがかんにいきました 
\\	彼女はクラブで踊るのが好きです。	かのじょはくらぶでおどるのがすきです 
\\	私は犬が怖いです。	わたしはいぬがこわいです 
\\	彼女は鏡を見て髪を直した。	かのじょはかがみをみてかみをなおした 
\\	庭に草が生えている。	にわにくさがはえている 
\\	お客様には敬語を使いなさい。	おきゃくさまにはけいごをつかいなさい 
\\	風邪のため今日は欠席します。	かぜのためきょうはけっせきします 
\\	彼はカーラジオのスイッチを入れた。	かれはかーらじおのすいっちをいれた 
\\	週末は部屋の掃除をしました。	しゅうまつはへやのそうじをしました 
\\	そのホテルには畳の部屋がある。	そのほてるにはたたみのへやがある 
\\	新幹線の速さはどれ位ですか。	しんかんせんのはやさはどれくらいですか 
\\	私たちは林の中に入っていった。	わたしたちははやしのなかにはいっていった 
\\	誕生日にプレゼントをもらいました。	たんじょうびにぷれぜんとをもらいました 
\\	今日は弁当を持ってきました。	きょうはべんとうをもってきました 
\\	いつでもうちに来てください。	いつでもうちにきてください 
\\	これは寂しい曲ですね。	これはさびしいきょくですね 
\\	あなたは少しも悪くない。	あなたはすこしもわるくない 
\\	祖父は家族みんなに尊敬されています。	そふはかぞくみんなにそんけいされています 
\\	彼らはよくテニスをしています。	かれらはよくてにすをしています 
\\	その家の広さはどれ位ですか。	そのいえのひろさはどれくらいですか 
\\	母に読み書きを習いました。	ははによみかきをならいました 
\\	庭にバラを植えました。	にわにばらをうえました 
\\	彼のわがままには我慢できません。	かれのわがままにはがまんできません 
\\	私は日本国籍です。	わたしはにほんこくせきです 
\\	受付はこちらです。	うけつけはこちらです 
\\	とても恥ずかしかった。	とてもはずかしかった 
\\	ボタンを押してください。	ぼたんをおしてください 
\\	今日は洋服を買いに行きます。	きょうはようふくをかいにいきます 
\\	水と油は混ざらない。	みずとあぶらはまざらない 
\\	彼の話はおかしかった。	かれのはなしはおかしかった 
\\	夕食の料理がたくさん余った。	ゆうしょくのりょうりがたくさんあまった 
\\	新しいコートでテニスをしました。	あたらしいこーとでてにすをしました 
\\	彼女はとても正直だ。	かのじょはとてもしょうじきだ 
\\	私はドラマを見るのが好きです。	わたしはどらまをみるのがすきです 
\\	学生ビザを持っています。	がくせいびざをもっています 
\\	東京は水道の水が不味い。	とうきょうはすいどうのみずがまずい 
\\	あの店員はとても親切です。	あのてんいんはとてもしんせつです 
\\	新しい電池を入れましょう。	あたらしいでんちをいれましょう 
\\	財布をポケットにしまった。	さいふをぽけっとにしまった 
\\	彼らは握手をした。	かれらはあくしゅをした 
\\	私の上司は頭が固い。	わたしのじょうしはあたまがかたい 
\\	そろそろ始めましょうか。	そろそろはじめましょうか 
\\	私はぶどうが好きです。	わたしはぶどうがすきです 
\\	道に迷いました。	みちにまよいました 
\\	ドアを押さえてください。	どあをおさえてください 
\\	皆で東京見物をした。	みんなでとうきょうけんぶつをした 
\\	皆に会議の日にちを知らせた。	みんなにかいぎのひにちをしらせた 
\\	京都には神社がたくさんある。	きょうとにはじんじゃがたくさんある 
\\	彼は注射があまり好きではありません。	かれはちゅうしゃがあまりすきではありません 
\\	彼女はめったに怒りません。	かのじょはめったにおこりません 
\\	彼は恋人を失った。	かれはこいびとをうしなった 
\\	会社に財布を忘れた。	かいしゃにさいふをわすれた 
\\	日本語で作文を書きました。	にほんごでさくぶんをかきました 
\\	今朝、5時に電話が鳴った。	けさごじにでんわがなった 
\\	彼は真面目な人です。	かれはまじめなひとです 
\\	彼女はそのセミナーに申し込んだ。	かのじょはそのせみなーにもうしこんだ 
\\	飲み物は冷蔵庫にあります。	のみものはれいぞうこにあります 
\\	新しい靴がとても気に入りました。	あたらしいくつがとてもきにいりました 
\\	クリスマスにはケーキを食べます。	くりすますにはけーきをたべます 
\\	父の日にネクタイをプレゼントした。	ちちのひにねくたいをぷれぜんとした 
\\	私は板の厚さを測った。	わたしはいたのあつさをはかった 
\\	郵便物は5時まで受け付けています。	ゆうびんぶつはごじまでうけつけています 
\\	彼女の誕生日は7月16日です。	かのじょのたんじょうびはしちがつじゅうろくにちです 
\\	私の兄はまだ独身です。	わたしのあにはまだどくしんです 
\\	私は苦いコーヒーが好きです。	わたしはにがいこーひーがすきです 
\\	彼女はバイオリンを習っています。	かのじょはばいおりんをならっています 
\\	パスポートが見つからない。	ぱすぽーとがみつからない 
\\	私は黒いバッグを持っています。	わたしはくろいばっぐをもっています 
\\	今までどこにいたのですか。	いままでどこにいたのですか 
\\	東京までの切符を買った。	とうきょうまでのきっぷをかった 
\\	ゴムが伸びてしまった。	ごむがのびてしまった 
\\	ジェット機が飛んでいる。	じぇっときがとんでいる 
\\	塩を少々入れてください。	しおをしょうしょういれてください 
\\	随分早く終わったね。	ずいぶんはやくおわったね 
\\	娘が私の手を引っ張った。	むすめがわたしのてをひっぱった 
\\	暑いので帽子を被りましょう。	あついのでぼうしをかぶりましょう 
\\	肉が焼けました。	にくがやけました 
\\	鉛筆を貸して下さい。	えんぴつをかしてください 
\\	親子の会話は大切です。	おやこのかいわはたいせつです 
\\	彼女はギターが得意です。	かのじょはぎたーがとくいです 
\\	私たちは毎日お茶を飲みます。	わたしたちはまいにちおちゃをのみます 
\\	今年の夏は暑さが厳しい。	ことしのなつはあつさがきびしい 
\\	昨日から胃が痛い。	きのうからいがいたい 
\\	そのお年寄りは椅子に座った。	そのおとしよりはいすにすわった 
\\	お茶を入れましょうか。	おちゃをいれましょうか 
\\	この踊りは易しいですよ。	このおどりはやさしいですよ 
\\	彼は今、禁煙しています。	かれはいまきんえんしています 
\\	次男は今、海外にいます。	じなんはいまかいがいにいます 
\\	このセーターはウールだ。	このせーたーはうーるだ 
\\	来月、大阪に引っ越します。	らいげつおおさかにひっこします 
\\	このカメラはよく写りますよ。	このかめらはよくうつりますよ 
\\	喫茶店でコーヒーを飲んだ。	きっさてんでこーひーをのんだ 
\\	この本は全然面白くなかった。	このほんはぜんぜんおもしろくなかった 
\\	やっぱり旅が大好きだ。	やっぱりたびがだいすきだ 
\\	その週刊誌はゴシップばかりだ。	そのしゅうかんしはごしっぷばかりだ 
\\	大きな棚はとても便利です。	おおきなたなはとてもべんりです 
\\	彼は寒くて震えていました。	かれはさむくてふるえていました 
\\	靴を磨いてください。	くつをみがいてください 
\\	朝、カーテンを開けた。	あさかーてんをあけた 
\\	火事です。	119番に電話してください。 
\\	かじですひゃくじゅうきゅうばんにでんわしてください
\\	このシャツはアイロンが必要だ。	このしゃつはあいろんがひつようだ 
\\	ナイフで手を切った。	ないふでてをきった 
\\	皆さんにお話があります。	みなさんにおはなしがあります 
\\	うちの次女は春から中学生です。	うちのじじょははるからちゅうがくせいです 
\\	貯金は十分にあります。	ちょきんはじゅうぶんにあります 
\\	兄はバイクが大好きです。	あにはばいくがだいすきです 
\\	荷物を預かってください。	にもつをあずかってください 
\\	僕はいつかアフリカに行きたい。	ぼくはいつかあふりかにいきたい 
\\	私はオレンジが好きです。	わたしはおれんじがすきです 
\\	もっとよく噛みなさい。	もっとよくかみなさい 
\\	このシャツは少しきついです。	このしゃつはすこしきついです 
\\	彼は家族のために仕事の時間を削った。	かれはかぞくのためにしごとのじかんをけずった 
\\	私たちは高い山を越えました。	わたしたちはたかいやまをこえました 
\\	フランスの友達から小包が届いた。	ふらんすのともだちからこづつみがとどいた 
\\	新年明けましておめでとうございます。	しんねんあけましておめでとうございます 
\\	彼は新聞配達をしている。	かれはしんぶんはいたつをしている 
\\	昨日子供たちと動物園に行きました。	きのうこどもたちとどうぶつえんにいきました 
\\	その手紙を封筒に入れた。	そのてがみをふうとうにいれた 
\\	その道は車が多くて危ない。	そのみちはくるまがおおくてあぶない 
\\	彼女の赤ちゃんは可愛いです。	かのじょのあかちゃんはかわいいです 
\\	さっきの話を続けましょう。	さっきのはなしをつづけましょう 
\\	宿題、手伝ってくれない?	しゅくだいてつだってくれない 
\\	テキストを読んでください。	てきすとをよんでください 
\\	ゴミはそのビニール袋に入れてください。	ごみはそのびにーるぶくろにいれてください 
\\	りんごを一つください。	りんごをひとつください 
\\	私は彼のアルバムを見た。	わたしはかれのあるばむをみた 
\\	私のお祖父さんは毎日散歩します。	わたしのおじいさんはまいにちさんぽします 
\\	母は健康のために水泳をしている。	はははけんこうのためにすいえいをしている 
\\	電車が込んでいる。	でんしゃがこんでいる 
\\	私は柔道を習っています。	わたしはじゅうどうをならっています 
\\	彼女はあまりスカートははかない。	かのじょはあまりすかーとははかない 
\\	シャツのボタンが取れた。	しゃつのぼたんがとれた 
\\	ペンを貸してください。	ぺんをかしてください 
\\	彼女の家には物がたくさんあります。	かのじょのいえにはものがたくさんあります 
\\	布団がとても柔らかい。	ふとんがとてもやわらかい 
\\	さっき郵便が届きました。	さっきゆうびんがとどきました 
\\	コップが落ちて割れた。	こっぷがおちてわれた 
\\	駅員に聞きましょう。	えきいんにききましょう 
\\	妹はとてもおしゃべりです。	いもうとはとてもおしゃべりです 
\\	ここに切手を貼ってください。	ここにきってをはってください 
\\	彼は緑色のジャケットを着ています。	かれはみどりいろのじゃけっとをきています 
\\	私は毎朝ジョギングをします。	わたしはまいあさじょぎんぐをします 
\\	ひざを曲げてください。	ひざをまげてください 
\\	娘は幼稚園に通っています。	むすめはようちえんにかよっています 
\\	明日はきっと雨が降ります。	あしたはきっとあめがふります 
\\	私は理科が得意です。	わたしはりかがとくいです 
\\	私は下着を手で洗う。	わたしはしたぎをてであらう 
\\	夕方は涼しくなりますよ。	ゆうがたはすずしくなりますよ 
\\	私の姉はすぐ太ります。	わたしのあねはすぐふとります 
\\	冷房を入れてください。	れいぼうをいれてください 
\\	駅で友だちと別れました。	えきでともだちとわかれました 
\\	後ろの席が空いています。	うしろのせきがあいています 
\\	出かける時は鍵を掛けてください。	でかけるときはかぎをかけてください 
\\	電車に傘を忘れた。	でんしゃにかさをわすれた 
\\	主人はもう出掛けました。	しゅじんはもうでかけました 
\\	姉は毎晩日記を書いています。	あねはまいばんにっきをかいています 
\\	道の真ん中に人が立っている。	みちのまんなかにひとがたっている 
\\	彼女に指輪をプレゼントしました。	かのじょにゆびわをぷれぜんとしました 
\\	娘さんは幾つになりましたか。	むすめさんはいくつになりましたか 
\\	リンゴが腐っている。	りんごがくさっている 
\\	彼は昔は貧乏だった。	かれはむかしはびんぼうだった 
\\	電話番号を間違えました。	でんわばんごうをまちがえました 
\\	メニューをください。	めにゅーをください 
\\	弟はアイスクリームが大好きです。	おとうとはあいすくりーむがだいすきです 
\\	あの子はおしゃれだね。	あのこはおしゃれだね 
\\	少し休憩しましょう。	すこしきゅうけいしましょう 
\\	舌を出してください。	したをだしてください 
\\	このジュースは甘すぎる。	このじゅーすはあますぎる 
\\	素晴らしい景色ですね。	すばらしいけしきですね 
\\	今日の交通事故はゼロです。	きょうのこうつうじこはぜろです 
\\	今日の朝刊に面白い記事があった。	きょうのちょうかんにおもしろいきじがあった 
\\	彼女はいつも丁寧に仕事をします。	かのじょはいつもていねいにしごとをします 
\\	私は料理が苦手です。	わたしはりょうりがにがてです 
\\	夏は日が長い。	なつはひがながい 
\\	去年、引っ越ししました。	きょねんひっこししました 
\\	歩道を歩きましょう。	ほどうをあるきましょう 
\\	強風で木の枝が折れた。	きょうふうできのえだがおれた 
\\	駅に友人の見送りに行きました。	えきにゆうじんのみおくりにいきました 
\\	そのニュースは夕刊で見ました。	そのにゅーすはゆうかんでみました 
\\	この間彼女に会った。	このあいだかのじょにあった 
\\	クーラーで体が冷えた。	くーらーでからだがひえた 
\\	娘がピンクのドレスを着ている。	むすめがぴんくのどれすをきている 
\\	これはきれいなグラスですね。	これはきれいなぐらすですね 
\\	猫の毛がセーターに付いた。	ねこのけがせーたーについた 
\\	答えを黒板に書いてください。	こたえをこくばんにかいてください 
\\	彼女はダンスが大好きです。	かのじょはだんすがだいすきです 
\\	出口はあそこです。	でぐちはあそこです 
\\	私たちは昨年、富士山に登りました。	わたしたちはさくねんふじさんにのぼりました 
\\	外は雪で真っ白だった。	そとはゆきでまっしろだった 
\\	おじは銀行に勤めています。	おじはぎんこうにつとめています 
\\	棚からその箱を下ろしてください。	たなからそのはこをおろしてください 
\\	その本は貸し出し中です。	そのほんはかしだしちゅうです 
\\	彼はサッカーの選手です。	かれはさっかーのせんしゅです 
\\	冬は暖房が必要です。	ふゆはだんぼうがひつようです 
\\	寒いので手袋をしました。	さむいのでてぶくろをしました 
\\	父は今、留守です。	ちちはいまるすです 
\\	嘘をついてはいけません。	うそをついてはいけません 
\\	私は遠慮します。	わたしはえんりょします 
\\	祖父は足の骨を折りました。	そふはあしのほねをおりました 
\\	母がコーンスープを作っている。	ははがこーんすーぷをつくっている 
\\	私は毎朝バナナを食べます。	わたしはまいあさばななをたべます 
\\	この通りは車の音がうるさい。	このとおりはくるまのおとがうるさい 
\\	その可哀相な子供たちは食べるものがない。	そのかわいそうなこどもたちはたべるものがない 
\\	あそこにパトカーがいる。	あそこにぱとかーがいる 
\\	今日の昼ご飯はハンバーガーでした。	きょうのひるごはんははんばーがーでした 
\\	この毛布は暖かい。	このもうふはあたたかい 
\\	3階までエスカレーターで行きましょう。	さんかいまでえすかれーたーでいきましょう 
\\	お嬢さんはおいくつですか。	おじょうさんはおいくつですか 
\\	私はタオルで顔をふいた。	わたしはたおるでかおをふいた 
\\	新しい手帳を買いました。	あたらしいてちょうをかいました 
\\	次の駅で地下鉄に乗り換えます。	つぎのえきでちかてつにのりかえます 
\\	駅の売店で雑誌を買った。	えきのばいてんでざっしをかった 
\\	彼は毎日パチンコをしています。	かれはまいにちぱちんこをしています 
\\	彼のお母さんは美人です。	かれのおかあさんはびじんです 
\\	駅前に本屋があります。	えきまえにほんやがあります 
\\	みかんを1つください。	みかんをひとつください 
\\	彼は直ぐに謝りました。	かれはすぐにあやまりました 
\\	誕生日にケーキを食べました。	たんじょうびにけーきをたべました 
\\	私は毎朝コップ一杯の水を飲む。	わたしはまいあさこっぷいっぱいのみずをのむ 
\\	日本には四季がある。	にほんにはしきがある 
\\	この辺は田んぼがたくさんあります。	このへんはたんぼがたくさんあります 
\\	明日の天気予報は雨です。	あしたのてんきよほうはあめです 
\\	このジャケットはナイロンでできています。	このじゃけっとはないろんでできています 
\\	変な音が聞こえます。	へんなおとがきこえます 
\\	続きは来週やりましょう。	つづきはらいしゅうやりましょう 
\\	彼は毎日一生懸命働いている。	かれはまいにちいっしょうけんめいはたらいている 
\\	夏は洗濯物がすぐ乾く。	なつはせんたくものがすぐかわく 
\\	空いたお皿をお下げします。	あいたおさらをおさげします 
\\	あなたは間違っている。	あなたはまちがっている 
\\	事務室でコピーを取って来ます。	じむしつでこぴーをとってきます 
\\	消防車が4台も来た。	しょうぼうしゃがよんだいもきた 
\\	あのスーツはそんなに高くない。	あのすーつはそんなにたかくない 
\\	この遊園地のチケットは3000円です。	このゆうえんちのちけっとはさんぜんえんです 
\\	妹はチョコレートが大好きです。	いもうとはちょこれーとがだいすきです 
\\	近所に泥棒が入った。	きんじょにどろぼうがはいった 
\\	靴を脱いでください。	くつをぬいでください 
\\	昨日、年賀状を出しました。	きのうねんがじょうをだしました 
\\	自転車は便利な乗り物です。	じてんしゃはべんりなのりものです 
\\	母が布団を干している。	ははがふとんをほしている 
\\	私は人々の役に立ちたいと思っています。	わたしはひとびとのやくにたちたいとおもっています 
\\	シャツが破れている。	しゃつがやぶれている 
\\	父は用事で出掛けています。	ちちはようじででかけています 
\\	彼は友達の答えを写した。	かれはともだちのこたえをうつした 
\\	私の腕時計は遅れている。	わたしのうでどけいはおくれている 
\\	すてきな贈り物をどうもありがとう。	すてきなおくりものをどうもありがとう 
\\	テレビのチャンネルを変えてください。	てれびのちゃんねるをかえてください 
\\	彼の話はつまらないですね。	かれのはなしはつまらないですね 
\\	マッチを持っていますか。	まっちをもっていますか 
\\	従兄弟と私は同じ年です。	いとことわたしはおなじとしです 
\\	頭を冷やしなさい。	あたまをひやしなさい 
\\	早く部屋を片付けなさい。	はやくへやをかたづけなさい 
\\	お二人の未来に乾杯しましょう。	おふたりのみらいにかんぱいしましょう 
\\	うちではまだストーブを使っています。	うちではまだすとーぶをつかっています 
\\	この電車は東京まで停車致しません。	このでんしゃはとうきょうまでていしゃいたしません 
\\	どこかでネコが鳴いている。	どこかでねこがないている 
\\	この近くにガソリンスタンドはありますか。	このちかくにがそりんすたんどはありますか 
\\	東京まで片道切符を買った。	とうきょうまでかたみちきっぷをかった 
\\	今朝テレビでそのニュースを知りました。	けさてれびでそのにゅーすをしりました 
\\	あそこの交番で道を聞きましょう。	あそこのこうばんでみちをききましょう 
\\	大分ピアノが上手くなりました。	だいぶぴあのがうまくなりました 
\\	梅雨は6月頃です。	つゆはろくがつごろです 
\\	デートで遊園地に行きました。	でーとでゆうえんちにいきました 
\\	ふすまを閉めてください。	ふすまをしめてください 
\\	この道を真っ直ぐ行ってください。	このみちをまっすぐいってください 
\\	紅茶にレモンを入れて飲んだ。	こうちゃにれもんをいれてのんだ 
\\	暑いので上着を脱ぎました。	あついのでうわぎをぬぎました 
\\	彼は偉い学者です。	かれはえらいがくしゃです 
\\	彼はその漢字の書き方が分からない。	かれはそのかんじのかきかたがわからない 
\\	母は毎朝ご飯を炊く。	はははまいあさごはんをたく 
\\	チーズを一切れ食べました。	ちーずをひときれたべました 
\\	今日は群馬までドライブしました。	きょうはぐんままでどらいぶしました 
\\	その踏切は長い。	そのふみきりはながい 
\\	テニスのラケットを買いました。	てにすのらけっとをかいました 
\\	受付は9時からです。	うけつけはくじからです 
\\	彼女が嘘をついたので、彼は怒った。	かのじょがうそをついたのでかれはおこった 
\\	彼女は化粧が上手い。	かのじょはけしょうがうまい 
\\	彼女は寿司を初めて食べました。	かのじょはすしをはじめてたべました 
\\	彼女はスチュワーデスになった。	かのじょはすちゅわーですになった 
\\	今日は都合が悪くて行けません。	きょうはつごうがわるくていけません 
\\	テープレコーダーで自分の声を録音しました。	てーぷれこーだーでじぶんのこえをろくおんしました 
\\	父はひげをはやしています。	ちちはひげをはやしています 
\\	これはとても大きな本棚ですね。	これはとてもおおきなほんだなですね 
\\	外は真っ暗です。	そとはまっくらです 
\\	おふろが沸きました。	おふろがわきました 
\\	その本は直ぐ売り切れた。	そのほんはすぐうりきれた 
\\	布団を押し入れにしまいました。	ふとんをおしいれにしまいました 
\\	革のベルトを買いました。	かわのべるとをかいました 
\\	赤ちゃんがぐっすり寝ている。	あかちゃんがぐっすりねている 
\\	温かい紅茶が飲みたい。	あたたかいこうちゃがのみたい 
\\	邪魔です、どいてください。	じゃまですどいてください 
\\	ソースはどれですか。	そーすはどれですか 
\\	母は古いタイプライターを持っています。	はははふるいたいぷらいたーをもっています 
\\	母は味噌汁に水を足した。	はははみそしるにみずをたした 
\\	私はトマトが大好きです。	わたしはとまとがだいすきです 
\\	彼女はブーツを履いています。	かのじょはぶーつをはいています 
\\	パンにバターをぬって食べました。	ぱんにばたーをぬってたべました 
\\	バスが発車します。	ばすがはっしゃします 
\\	妹はバレーボールが得意です。	いもうとはばれーぼーるがとくいです 
\\	晩ご飯のおかずは何?	ばんごはんのおかずはなに 
\\	カレンダーに予定を書いた。	かれんだーによていをかいた 
\\	次のオリンピックは再来年です。	つぎのおりんぴっくはさらいねんです 
\\	冗談は止めてください。	じょうだんはやめてください 
\\	女性が近付いてきました。	じょせいがちかづいてきました 
\\	電話帳を見せてください。	でんわちょうをみせてください 
\\	ハンドバッグを忘れました。	はんどばっぐをわすれました 
\\	玄関のベルが鳴った。	げんかんのべるがなった 
\\	母は洗濯物を干しています。	はははせんたくものをほしています 
\\	プリンターのインクを買った。	ぷりんたーのいんくをかった 
\\	蚊に足を刺された。	かにあしをさされた 
\\	改札口で会いましょう。	かいさつぐちであいましょう 
\\	父は友だちの死を悲しんでいます。	ちちはともだちのしをかなしんでいます 
\\	暑いのでクーラーをつけました。	あついのでくーらーをつけました 
\\	この靴下は3足で1000円です。	このくつしたはさんそくでせんえんです 
\\	ではそろそろ失礼します。	ではそろそろしつれいします 
\\	西暦1964年に東京でオリンピックがあった。	せいれきせんきゅうひゃくろくじゅうよねんにとうきょうでおりんぴっくがあった 
\\	新しい洗濯機を買いました。	あたらしいせんたっきをかいました 
\\	私は毎日豆腐を食べます。	わたしはまいにちとうふをたべます 
\\	私の子供は毎日昼寝をします。	わたしのこどもはまいにちひるねをします 
\\	私の家は駅から遠くて不便です。	わたしのいえはえきからとおくてふべんです 
\\	赤ちゃんにミルクをあげた。	あかちゃんにみるくをあげた 
\\	私の家は川の向こう側にあります。	わたしのいえはかわのむこうがわにあります 
\\	世界人口は65億人だ。	せかいじんこうはろくじゅうごおくにんだ 
\\	彼女にお礼の手紙を書きました。	かのじょにおれいのてがみをかきました 
\\	娘に可愛いドレスを着せた。	むすめにかわいいどれすをきせた 
\\	サラダをたくさん食べました。	さらだをたくさんたべました 
\\	八百屋でみかんを買いました。	やおやでみかんをかいました 
\\	昨日おじさんに会いました。	きのうおじさんにあいました 
\\	社長に代えて部長を出席させます。	しゃちょうにかえてぶちょうをしゅっせきさせます 
\\	ちょうど急行電車が来た。	ちょうどきゅうこうでんしゃがきた 
\\	次の交差点を左に曲がってください。	つぎのこうさてんをひだりにまがってください 
\\	弟が高等学校を卒業しました。	おとうとがこうとうがっこうをそつぎょうしました 
\\	今朝は6時に目が覚めた。	けさはろくじにめがさめた 
\\	昼休みに公園に行った。	ひるやすみにこうえんにいった 
\\	ベルトがきつくなりました。	べるとがきつくなりました 
\\	インクで手が真っ黒になった	いんくでてがまっくろになった 
\\	皆様、こんにちは。	みなさまこんにちは 
\\	夕食にラーメンを食べました。	ゆうしょくにらーめんをたべました 
\\	店にライターを忘れました。	みせにらいたーをわすれました 
\\	彼は白いワイシャツを着ている。	かれはしろいわいしゃつをきている 
\\	お祖母ちゃんがお風呂場で転んだ。	おばあちゃんがおふろばでころんだ 
\\	彼への気持ちが冷めた。	かれへのきもちがさめた 
\\	朝、急いでシャワーを浴びました。	あさいそいでしゃわーをあびました 
\\	暑いから扇風機をつけよう。	あついからせんぷうきをつけよう 
\\	遅刻しないでください。	ちこくしないでください 
\\	昨日大きな魚を釣りました。	きのうおおきなさかなをつりました 
\\	私はお昼にうどんを食べました。	わたしはおひるにうどんをたべました 
\\	大きなトラックが私たちを追い越した。	おおきなとらっくがわたしたちをおいこした 
\\	お名前に仮名を振ってください。	おなまえにかなをふってください 
\\	家内は九州出身です。	かないはきゅうしゅうしゅっしんです 
\\	今日は両親の結婚記念日です。	きょうはりょうしんのけっこんきねんびです 
\\	もうすぐ下りの電車が出発します。	もうすぐくだりのでんしゃがはっしゃします 
\\	掲示板のお知らせを見ましたか。	けいじばんのおしらせをみましたか 
\\	寒い朝は道路が凍ります。	さむいあさはどうろがこおります 
\\	今日はごみの日だ。	きょうはごみのひだ 
\\	この町には消防署が1つしかありません。	このまちにはしょうぼうしょがひとつしかありません 
\\	スーパーマーケットで人参を買った。	すーぱーまーけっとでにんじんをかった 
\\	ハンカチで手をふきました。	はんかちでてをふきました 
\\	私はビデオテープを30本持っています。	わたしはびでおてーぷをさんじゅっぽんもっています 
\\	朝、近所をぶらぶらした。	あさきんじょをぶらぶらした 
\\	私はアナウンサーになりたい。	わたしはあなうんさーになりたい 
\\	彼女は辞書をよく使います。	かのじょはじしょをよくつかいます 
\\	私は毎日、自炊している。	わたしはまいにちじすいしている 
\\	車を止めて。	くるまをとめて 
\\	カレーライスはスプーンで食べます。	かれーらいすはすぷーんでたべます 
\\	彼は奥さんより年下です。	かれはおくさんよりとししたです 
\\	早く風邪を治してください。	はやくかぜをなおしてください 
\\	箸を上手に使えるよ。	はしをじょうずにつかえるよ 
\\	友達と花見をしました。	ともだちとはなみをしました 
\\	財布は引き出しの中にあります。	さいふはひきだしのなかにあります 
\\	彼女は白いブラウスを着ている。	かのじょはしろいぶらうすをきている 
\\	その文の訳を読みました。	そのぶんのやくをよみました 
\\	プレゼントにリボンを付けた。	ぷれぜんとにりぼんをつけた 
\\	お加減はいかがですか。	おかげんはいかがですか 
\\	今日は一日中家に居ました。	きょうはいちにちじゅううちにいました 
\\	会社までの交通費は一ヶ月8,000円です。	かいしゃまでのこうつうひはいっかげつはっせんえんです 
\\	彼はステレオで音楽を聞いた。	かれはすてれおでおんがくをきいた 
\\	彼は雪の上を滑った。	かれはゆきのうえをすべった 
\\	彼女は早口だ。	かのじょははやくちだ 
\\	フォークをもらえますか。	ふぉーくをもらえますか 
\\	彼は紙を裏返しました。	かれはかみをうらがえしました 
\\	おばは大阪に住んでいます。	おばはおおさかにすんでいます 
\\	次の角で左に曲がってください。	つぎのかどでひだりにまがってください 
\\	星がきらきら光っている。	ほしがきらきらひかっている 
\\	紙を三角に切りました。	かみをさんかくにきりました 
\\	図書室で勉強した。	としょしつでべんきょうした 
\\	入場券は3000円です。	にゅうじょうけんはさんぜんえんです 
\\	妻にネックレスをプレゼントしました。	つまにねっくれすをぷれぜんとしました 
\\	このパジャマを着て。	このぱじゃまをきて 
\\	ボールペンで名前を書いてください。	ぼーるぺんでなまえをかいてください 
\\	父に万年筆をプレゼントしました。	ちちにまんねんひつをぷれぜんとしました 
\\	彼の声は柔らかだ。	かれのこえはやわらかだ 
\\	子供が目を覚ました。	こどもがめをさました 
\\	夏はすいかが美味しい。	なつはすいかがおいしい 
\\	夕方そちらに着きます。	ゆうがたそちらにつきます 
\\	お昼にそばを食べた。	おひるにそばをたべた 
\\	朝食は大抵7時頃に食べます。	ちょうしょくはたいていしちじごろにたべます 
\\	東京まで特急で3時間かかります。	とうきょうまでとっきゅうでさんじかんかかります 
\\	どっちの色が好きですか。	どっちのいろがすきですか 
\\	糊で2枚の紙を貼り合わせた。	のりでにまいのかみをはりあわせた 
\\	姉はバスケットボールの選手です。	あねはばすけっとぼーるのせんしゅです 
\\	明日の天気は晴れです。	あすのてんきははれです 
\\	子供がジュースを欲しがっています。	こどもがじゅーすをほしがっています 
\\	私は毎日味噌汁を飲みます。	わたしはまいにちみそしるをのみます 
\\	この漢字の読み方を教えてください。	このかんじのよみかたをおしえてください 
\\	私は待つのが嫌だ。	わたしはまつのがいやだ 
\\	入り口は向こうです。	いりぐちはむこうです 
\\	彼は色々なことを知っている。	かれはいろいろなことをしっている 
\\	彼は車から荷物を降ろした。	かれはくるまからにもつをおろした 
\\	壁に大きな時計が掛かっています。	かべにおおきなとけいがかかっています 
\\	今日のお昼はサンドイッチです。	きょうのおひるはさんどいっちです 
\\	紙を四角に切ってください。	かみをしかくにきってください 
\\	こっちの四角いテーブルを買おうよ。	こっちのしかくいてーぶるをかおうよ 
\\	突き当たったら右に曲がってください。	つきあたったらみぎにまがってください 
\\	定期券は1万2千円でした。	ていきけんはいちまんにせんえんでした 
\\	彼女はいつもにこにこしています。	かのじょはいつもにこにこしています 
\\	私は馬鹿だった。	わたしはばかだった 
\\	あなたの名字は何ですか。	あなたのみょうじはなんですか 
\\	小さな船が川を下っています。	ちいさなふねがかわをくだっています 
\\	このウールのセーターは暖かい。	このうーるのせーたーはあたたかい 
\\	私の兄はコーラが好きだ。	わたしのあにはこーらがすきだ 
\\	トイレットペーパーを交換しました。	といれっとぺーぱーをこうかんしました 
\\	失礼ですが、どなたですか。	しつれいですがどなたですか 
\\	次の駅で乗り換えです。	つぎのえきでのりかえです 
\\	昨日は友人とハイキングに行きました。	きのうはゆうじんとはいきんぐにいきました 
\\	灰皿をください。	はいざらをください 
\\	次のバス停で降ります。	つぎのばすていでおります 
\\	ハムサンドイッチをください。	はむさんどをください 
\\	私のボーイフレンドはハンサムな方でした。	わたしのぼーいふれんどははんさむなかたでした 
\\	床がぬれている。	ゆかがぬれている 
\\	例文を3つ作ってください。	れいぶんをみっつつくってください 
\\	ローマ字で名前を書いてください。	ろーまじでなまえをかいてください 
\\	彼女はシャツにアイロンをかけた。	かのじょはしゃつにあいろんをかけた 
\\	この苺はとても甘い。	このいちごはとてもあまい 
\\	子供はカレーライスが好きです。	こどもはかれーらいすがすきです 
\\	キャッシュカードでお金を下ろした。	きゃっしゅかーどでおかねをおろした 
\\	今日は一日曇りでした。	きょうはいちにちくもりでした 
\\	魚がたくさん泳いでいます。	さかながたくさんおよいでいます 
\\	もうちょっと塩を入れて。	もうちょっとしおをいれて 
\\	バスの停留所で10分待ちました。	ばすのていりゅうじょでじゅっぷんまちました 
\\	昨日床屋で髪を切った。	きのうとこやでかみをきった 
\\	私は歯医者が嫌いです。	わたしははいしゃがきらいです 
\\	歯磨きはしましたか。	はみがきはしましたか 
\\	祖父は早起きです。	そふははやおきです 
\\	冬休みにお祖父ちゃんの家に行きます。	ふゆやすみにおじいちゃんのうちにいきます 
\\	水を持って行きましょう。	みずをもっていきましょう 
\\	夕飯は寿司でした。	ゆうはんはすしでした 
\\	彼はただの酔っ払いです。	かれはただのよっぱらいです 
\\	一つ例をあげてください。	ひとつれいをあげてください 
\\	彼女はすてきなイヤリングをしている。	かのじょはすてきないやりんぐをしている 
\\	あそこに横断歩道があります。	あそこにおうだんほどうがあります 
\\	皆、社長にお辞儀をした。	みんなしゃちょうにおじぎをした 
\\	先生のお宅はどちらですか。	せんせいのおたくはどちらですか 
\\	昔は彼女のことをガールフレンドと言っていました。	"むかしはかのじょのことをがーるふれんどといっていました 
\\	昔はカセットテープを使っていました。	むかしは カセットテープをつかっていました 。 
\\	背中がかゆいです。	せなかがかゆいです 
\\	ぬれた服を乾かした。	ぬれたふくをかわかした 
\\	学校は今週は休校です。	がっこうはこんしゅうはきゅうこうです 
\\	地震で家がぐらぐらと揺れた。	じしんでいえがぐらぐらとゆれた 
\\	航空便で書類が届きました。	こうくうびんでしょるいがとどきました 
\\	このイチゴでジャムを作りましょう。	このいちごでじゃむをつくりましょう 
\\	スリッパをはいてください。	すりっぱをはいてください 
\\	私はテストが大嫌い。	わたしはてすとがだいきらい 
\\	布団を畳んでください。	ふとんをたたんでください 
\\	友達とトランプをして遊びました。	ともだちととらんぷをしてあそびました 
\\	勝った、万歳!	かったばんざい 
\\	その建物の番地は分かりますか。	そのたてもののばんちはわかりますか 
\\	暑いので日陰で休みましょう。	あついのでひかげでやすみましょう 
\\	父は美容院で髪を切ります。	ちちはびよういんでかみをきります 
\\	新しい文房具を買いました。	あたらしいぶんぼうぐをかいました 
\\	免許証を見せてください。	めんきょしょうをみせてください 
\\	この文を訳してください。	このぶんをやくしてください 
\\	レポートはこれでよろしいですか。	れぽーとはこれでよろしいですか 
\\	私はラッシュアワーの電車が嫌いです。	わたしはらっしゅあわーのでんしゃがきらいです 
\\	1万円を両替してください。	いちまんえんをりょうがえしてください 
\\	では、また明日。	ではまたあした 
\\	その送り仮名は間違っています。	そのおくりがなはまちがっています 
\\	仕事が大分片付いた。	しごとがだいぶかたづいた 
\\	水を1カップ入れてください。	みずをいちかっぷいれてください 
\\	消しゴムを貸して下さい。	けしごむをかしてください 
\\	主人は今、留守です。	しゅじんはいまるすです 
\\	湖でスケートをした。	みずうみですけーとをした 
\\	昨日はたくさん泳ぎました。	きのうはたくさんおよぎました 
\\	父は家の中にいます。	ちちはいえのなかにいます 
\\	鍋にスープが入っているよ。	なべにすーぷがはいっているよ 
\\	怠けていないで、手伝って。	なまけていないでてつだって 
\\	兄は二階にいます。	あにはにかいにいます 
\\	今朝は寝坊しました。	けさはねぼうしました 
\\	この鋏はよく切れる。	このはさみはよくきれる 
\\	母は畑仕事が好きだ。	ははははたけしごとがすきだ 
\\	今日はピクニックに行きましょう。	きょうはぴくにっくにいきましょう 
\\	空が真っ青です。	そらがまっさおです 
\\	彼は服を汚した。	かれはふくをよごした 
\\	昨夜は酔っ払いました。	ゆうべはよっぱらいました 
\\	学校に忘れ物をしました。	がっこうにわすれものをしました 
\\	あちらに行ってみよう。	あちらにいってみよう 
\\	弟は掛け算を習っている。	おとうとはかけざんをならっている 
\\	納豆は臭い。	なっとうはくさい 
\\	明日は自習の時間がある。	あすはじしゅうのじかんがある 
\\	この先は突き当たりです。	このさきはつきあたりです 
\\	彼女に会うとどきどきします。	かのじょにあうとどきどきします 
\\	今日、鍵を無くしました。	きょうかぎをなくしました 
\\	私は納豆をよく食べます。	わたしはなっとうをよくたべます 
\\	妹は来年、二十歳になります。	いもうとはらいねんはたちになります 
\\	日は東から上ります。	ひはひがしからのぼります 
\\	彼の成績はまあまあです。	かれのせいせきはまあまあです 
\\	彼女は首にマフラーをまいていた。	かのじょはくびにまふらーをまいていた 
\\	私の一番好きな果物は桃です。	わたしのいちばんすきなくだものはももです 
\\	このアクセサリーは素敵ね。	このあくせさりーはすてきね 
\\	父があくびをした。	ちちがあくびをした 
\\	缶に飴が入っています。	かんにあめがはいっています 
\\	チケットはもう売り切れだって。	ちけっとはもううりきれだって 
\\	あのお巡りさんに道を聞きましょう。	あのおまわりさんにみちをききましょう 
\\	おしゃべりして喉がからからになりました。	おしゃべりしてのどがからからになりました 
\\	信号は黄色でした。	しんごうはきいろでした 
\\	彼女をデートに誘った。	かのじょをでーとにさそった 
\\	これを速達で送りたいのですが。	これをそくたつでおくりたいのですが 
\\	私の歯は丈夫です。	わたしのははじょうぶです 
\\	娘は学校で引き算を習っている。	むすめはがっこうでひきざんをならっている 
\\	昼前に会議があった。	ひるまえにかいぎがあった 
\\	私は疲れてふらふらです。	わたしはつかれてふらふらです 
\\	7時に目覚まし時計が鳴りました。	しちじにめざましどけいがなりました 
\\	明日の予習をしましょう。	あしたのよしゅうをしましょう 
\\	ソファにワインをこぼした。	そふぁにわいんをこぼした 
\\	一緒にランチに行きませんか。	いっしょにらんちにいきませんか 
\\	レシートを財布に入れました。	れしーとをさいふにいれました 
\\	そのいすを持って来てください。	そのいすをもってきてください 
\\	体に気を付けてください。	からだにきをつけてください 
\\	私は刺身は食べません。	わたしはさしみはたべません 
\\	洗面所で顔を洗った。	せんめんじょでかおをあらった 
\\	娘は学校で足し算を習った。	むすめはがっこうでたしざんをならった 
\\	段々仕事が楽しくなってきました。	だんだんしごとがたのしくなってきました 
\\	チョークで黒板に字を書きました。	ちょーくでこくばんにじをかきました 
\\	プレゼントをきれいな紙で包みました。	ぷれぜんとをきれいなかみでつつみました 
\\	ティッシュを取ってください。	てぃっしゅをとってください 
\\	今朝、事故で電車が止まりました。	けさじこででんしゃがとまりました 
\\	もうお金が無くなりました。	もうおかねがなくなりました 
\\	太陽は西に沈みます。	たいようはにしにしずみます 
\\	入る時はドアをノックしてください。	はいるときはどあをのっくしてください 
\\	新しい歯ブラシが必要だ。	あたらしいはぶらしがひつようだ 
\\	明日は暇ですか。	あしたはひまですか 
\\	豚はすごく鼻が良い。	ぶたはすごくはながいい 
\\	今日は妻と二人で食事をします。	きょうはつまとふたりでしょくじをします 
\\	頑張ったので褒められました。	がんばったのでほめられました 
\\	雑誌を本箱に入れました。	ざっしをほんばこにいれました 
\\	雨が止みました。	あめがやみました 
\\	よろしくお願いします。	よろしくおねがいします 
\\	お湯を沸かしてください。	おゆをわかしてください 
\\	明後日は休日です。	あさってはきゅうじつです 
\\	この靴は幾らですか。	このくつはいくらですか 
\\	友達から絵葉書が届きました。	ともだちからえはがきがとどきました 
\\	今、車を暖めています。	いまくるまをあたためています 
\\	一昨日彼から電話がありました。	おとといかれからでんわがありました 
\\	車のタイヤを換えた。	くるまのたいやをかえた 
\\	これを書留で送りたいのですが。	これをかきとめでおくりたいのですが 
\\	映画館はがらがらでした。	えいがかんはがらがらでした 
\\	缶詰のフルーツはとても甘い。	かんづめのふるーつはとてもあまい 
\\	太陽がぎらぎらしている。	たいようがぎらぎらしている 
\\	私は九月に行く予定です。	わたしはくがつにいくよていです 
\\	彼女は腕を怪我した。	かのじょはうでをけがした 
\\	今日は厳しい寒さになるでしょう。	きょうはきびしいさむさになるでしょう 
\\	5月3日は祝日だ。	ごがつみっかはしゅくじつだ 
\\	昨日の夜はすき焼きを食べた。	きのうのよるはすきやきをたべた 
\\	そっちが私の部屋です。	そっちがわたしのへやです 
\\	家族に葉書を書いています。	かぞくにはがきをかいています 
\\	床をほうきで掃きました。	ゆかをほうきではきました 
\\	昼間は仕事で忙しいです。	ひるまはしごとでいそがしいです 
\\	昨日の夕方、友達とピンポンをした。	きのうのゆうがたともだちとぴんぽんをした 
\\	このプールの深さは2メートルです。	このぷーるのふかさはにめーとるです 
\\	その店員は不親切だった。	そのてんいんはふしんせつだった 
\\	私はおなかがぺこぺこです。	わたしはおなかがぺこぺこです 
\\	彼女はドイツ語がぺらぺらです。	かのじょはどいつごがぺらぺらです 
\\	この本は横書きです。	このほんはよこがきです 
\\	雨なのでレインコートを着ました。	あめなのでれいんこーとをきました 
\\	暑いのでアイスを食べました。	あついのであいすをたべました 
\\	いつごろ夏休みを取りますか。	いつごろなつやすみをとりますか 
\\	このウィスキーは強いよ。	このうぃすきーはつよいよ 
\\	彼にエアメールを送りました。	かれにえあめーるをおくりました 
\\	もっと大きく字を書いてください。	もっとおおきくじをかいてください 
\\	海の水は塩辛い。	うみのみずはしおからい 
\\	そのグループのメンバーは全部で七人だ。	そのぐるーぷのめんばーはぜんぶでしちにんだ 
\\	私のシャープペンシルがありません。	わたしのしゃーぷぺんしるがありません 
\\	セロテープはありますか。	せろてーぷはありますか 
\\	白線の外側を歩かないでください。	はくせんのそとがわをあるかないでください 
\\	私はたばこ屋でライターを買った。	わたしはたばこやでらいたーをかった 
\\	庭に白い猫がいます。	にわにしろいねこがいます 
\\	去年はスニーカーが流行りました。	きょねんはすにーかーがはやりました 
\\	画面の左側を見てください。	がめんのひだりがわをみてください 
\\	袋はいりません。	ふくろはいりません 
\\	昨日、友達とお風呂屋さんに行った。	きのうともだちとおふろやさんにいった 
\\	彼は足の骨を折りました。	かれはあしのほねをおりました 
\\	彼女はいつも私の右側を歩きます。	かのじょはいつもわたしのみぎがわをあるきます 
\\	彼は眼鏡をかけています。	かれはめがねをかけています 
\\	彼は手にやけどをしました。	かれはてにやけどをしました 
\\	靴の汚れを落としました。	くつのよごれをおとしました 
\\	割り算は小学校で習います。	わりざんはしょうがっこうでならいます 
\\	アイスコーヒーをください。	あいすこーひーをください 
\\	雲の間から月が出た。	くものあいだからつきがでた 
\\	顎が痛い。	あごがいたい 
\\	七時に朝御飯を食べました	しちじにあさごはんをたべました 
\\	まだ部屋が暖まらない。	まだへやがあたたまらない 
\\	郵便局はあっちです。	ゆうびんきょくはあっちです 
\\	あんなにいい人はいません。	あんなにいいひとはいません 
\\	行きは新幹線で行った。	いきはしんかんせんでいった 
\\	その子はいたずらが大好きだ。	そのこはいたずらがだいすきだ 
\\	ウェートレスを呼んだ。	うぇーとれすをよんだ 
\\	彼の就職のお祝いをしよう。	かれのしゅうしょくのおいわいをしよう 
\\	お母さんと一緒にお菓子を作りました。	おかあさんといっしょにおかしをつくりました 
\\	母はお釣りを確かめた。	はははおつりをたしかめた 
\\	私の彼女はとても大人しいです。	わたしのかのじょはとてもおとなしいです 
\\	彼女はお祖母さんと住んでいる。	かのじょはおばあさんとすんでいる 
\\	おばさん一家が遊びに来ました。	おばさんいっかがあそびにきました 
\\	父はお土産にお菓子を買ってきた。	ちちはおみやげにおかしをかってきた 
\\	表に人が来ています。	おもてにひとがきています 
\\	話しのポイントを書き取った。	はなしのぽいんとをかきとった 
\\	友達をうちに泊めてあげました。	ともだちをうちにとめてあげました 
\\	その歌手は歌が下手だ。	そのかしゅはうたがへただ 
\\	この風邪薬を飲みなさい。	このかぜぐすりをのみなさい 
\\	次の方、どうぞ。	つぎのかたどうぞ 
\\	その椅子は変わった形をしている。	そのいすはかわったかたちをしている 
\\	その黒い鞄は僕のです。	そのくろいかばんはぼくのです 
\\	帽子を被って外出した。	ぼうしをかぶってがいしゅつした 
\\	雷が鳴っています。	かみなりがなっています 
\\	彼は人前だと上がってしまう。	かれはひとまえだとあがってしまう 
\\	彼は空手を習っています。	かれはからてをならっています 
\\	コーヒーはその缶に入っています。	こーひーはそのかんにはいっています 
\\	元旦に彼から年賀状が来た。	がんたんにかれからねんがじょうがきた 
\\	彼はきれいな服に着替えた。	かれはきれいなふくにきがえた 
\\	私の上司はとても厳しい。	わたしのじょうしはとてもきびしい 
\\	その店は若い客が多いです。	そのみせはわかいきゃくがおおいです 
\\	彼女は金のネックレスをしています。	かのじょはきんのねっくれすをしています 
\\	彼はくしで髪をとかした。	かれはくしでかみをとかした 
\\	今日は雲が多い。	きょうはくもがおおい 
\\	スーツをクリーニングに出しました。	すーつをくりーにんぐにだしました 
\\	ここは景色がきれいですね。	ここはけしきがきれいですね 
\\	喧嘩はやめて。	けんかはやめて 
\\	彼はその問題の答えが分からない。	かれはそのもんだいのこたえがわからない 
\\	テーブルの上にごちそうが並んでいる。	てーぶるのうえにごちそうがならんでいる 
\\	今日の夕食は私がごちそうします。	きょうのゆうしょくはわたしがごちそうします 
\\	この頃、よく肩が凝る。	このごろよくかたがこる 
\\	彼女の目から涙がこぼれた。	かのじょのめからなみだがこぼれた 
\\	会社のパソコンが壊れた。	かいしゃのぱそこんがこわれた 
\\	こんなに面白い本は初めて読んだ。	こんなにおもしろいほんははじめてよんだ 
\\	雨がざあざあ降っている。	あめがざあざあふっている 
\\	再来月まで予約で一杯です。	さらいげつまでよやくでいっぱいです 
\\	再来週は忙しいです。	さらいしゅうはいそがしいです 
\\	電車の中で騒がないでください。	でんしゃのなかでさわがないでください 
\\	その試合は残念な結果になった。	そのしあいはざんねんなけっかになった 
\\	父親が子供を叱っている。	ちちおやがこどもをしかっている 
\\	彼女は大切な書類を机に仕舞った。	かのじょはたいせつなしょるいをつくえにしまった 
\\	ここに住所と氏名を書いてください。	ここにじゅうしょとしめいをかいてください 
\\	ポテトサラダを作るから、ジャガイモを買ってきて。	ぽてとさらだをつくるからじゃがいもをかってきて 
\\	車道の工事が始まりました。	しゃどうのこうじがはじまりました 
\\	祖母は身体が丈夫だ。	そぼはからだがじょうぶだ 
\\	もう少し醤油を足してください。	もうすこししょうゆをたしてください 
\\	白は雪の色です。	しろはゆきのいろです 
\\	明日のプレゼンテーションが心配だ。	あしたのぷれぜんてーしょんがしんぱいだ 
\\	石鹸で手を洗ってください。	せっけんでてをあらってください 
\\	一週間、洗濯をしていない。	いっしゅうかんせんたくをしていない 
\\	掃除機が壊れた。	そうじきがこわれた 
\\	このシャツは袖が短い。	このしゃつはそでがみじかい 
\\	外は暑いよ。	そとはあついよ 
\\	祖父は元気です。	そふはげんきです 
\\	このソファーは気持ちがいい。	このそふぁーはきもちがいい 
\\	祖母は京都で生まれました。	そぼはきょうとでうまれました 
\\	空の写真を撮るのが好きです。	そらのしゃしんをとるのがすきです 
\\	父は毎日髭を剃ります。	ちちはまいにちひげをそります 
\\	そこにちょうど良い台がある。	そこにちょうどよいだいがある 
\\	お体をお大事に。	おからだをおだいじに 
\\	これは母が大切にしていた指輪です。	これはははがたいせつにしていたゆびわです 
\\	そのスキーヤーはポールを倒した。	そのすきーやーはぽーるをたおした 
\\	彼女は病気の犬を助けた。	かのじょはびょうきのいぬをたすけた 
\\	彼はたばこを吸いません。	かれはたばこをすいません 
\\	誰かに聞いてみてください。	だれかにきいてみてください 
\\	血が出ていますよ。	ちがでていますよ 
\\	お茶わんにご飯をよそいました。	おちゃわんにごはんをよそいました 
\\	昼食にお寿司を食べました。	ちゅうしょくにおすしをたべました 
\\	うちの長女は10歳です。	うちのちょうじょはじゅっさいです 
\\	朝食に納豆を食べました。	ちょうしょくになっとうをたべました 
\\	値段はちょうど3万円です。	ねだんはちょうどさんまんえんです 
\\	ちょっとだけ待っていてください。	ちょっとだけまっていてください 
\\	部屋の電気がついています。	へやのでんきがついています 
\\	新しい机を買ってもらいました。	あたらしいつくえをかってもらいました 
\\	私は銀行に勤めています。	わたしはぎんこうにつとめています 
\\	爪が伸びています。	つめがのびています 
\\	あそこに古いお寺があります。	あそこにふるいおてらがあります 
\\	私はえびのてんぷらが好きです。	わたしはえびのてんぷらがすきです 
\\	彼は私より年上です。	かれはわたしよりとしうえです 
\\	あそこにきれいな鳥がいます。	あそこにきれいなとりがいます 
\\	母の肩をとんとんたたいた。	ははのかたをとんとんたたいた 
\\	今日は寒いので長袖を着ました。	きょうはさむいのでながそでをきました 
\\	夕飯になすの天ぷらを食べました。	ゆうはんになすのてんぷらをたべました 
\\	質問は何ですか。	しつもんはなんですか 
\\	兄さん、おめでとう。	にいさんおめでとう 
\\	浅草は賑やかな街です。	あさくさはにぎやかなまちです 
\\	昨日の夜は肉をたくさん食べました。	きのうのよるはにくをたくさんたべました 
\\	その鶏は毎朝鳴きます。	そのにわとりはまいあさなきます 
\\	このバンドはとても人気があるよ。	このばんどはとてもにんきがあるよ 
\\	人間の心は複雑です。	にんげんのこころはふくざつです 
\\	私はにんじんが嫌いだ。	わたしはにんじんがきらいだ 
\\	風呂がぬるかった。	ふろがぬるかった 
\\	雨で濡れてしまった。	あめでぬれてしまった 
\\	姉さん、ごめんね。	ねえさんごめんね 
\\	今日はとても眠いです。	きょうはとてもねむいです 
\\	昨日は8時間眠りました。	きのうははちじかんねむりました 
\\	これは上り電車です。	これはのぼりでんしゃです 
\\	この村の80パーセントは老人です。	このむらのはちじゅっぱーせんとはろうじんです 
\\	火事ですべて灰になった。	かじですべてはいになった 
\\	今日の空は灰色だ。	きょうのそらははいいろだ 
\\	大きな拍手が上がった。	おおきなはくしゅがあがった 
\\	あの橋は日本で一番長い。	あのはしはにほんでいちばんながい 
\\	初めは上手くできませんでした。	はじめはうまくできませんでした 
\\	この家の柱は太い。	このいえのはしらはふとい 
\\	旗が風に揺れている。	はたがかぜにゆれている 
\\	机にシールを貼った。	つくえにしーるをはった 
\\	今日は、半袖のシャツを着よう。	きょうははんそでのしゃつをきよう 
\\	半月前に日本に来ました。	はんつきまえににほんにきました 
\\	今日は半日だけ仕事だ。	きょうははんにちだけしごとだ 
\\	火を消して。	ひをけして 
\\	彼はギターを弾きます。	かれはぎたーをひきます 
\\	スキーでひざを怪我しました。	すきーでひざをけがしました 
\\	彼はひじに怪我をした。	かれはひじにけがをした 
\\	あの村の人々はとても親切です。	あのむらのひとびとはとてもしんせつです 
\\	靴のひもが切れた。	くつのひもがきれた 
\\	この表を見てください。	このひょうをみてください 
\\	今日は昼過ぎまで寝ていた。	きょうはひるすぎまでねていた 
\\	瓶ビールを注文した。	びんびーるをちゅうもんした 
\\	タオルで体を拭きました。	たおるでからだをふきました 
\\	昨日の復習をしましたか。	きのうのふくしゅうをしましたか 
\\	豚肉は美味しいです。	ぶたにくはおいしいです 
\\	アメリカの友人から船便が届いた。	あめりかのゆうじんからふなびんがとどいた 
\\	私たちは船に乗った。	わたしたちはふねにのった 
\\	明日は雨が降るでしょう。	あすはあめがふるでしょう 
\\	父は今お風呂に入っています。	ちちはいまおふろにはいっています 
\\	彼女は平成3年生まれです。	かのじょはへいせいさんねんうまれです 
\\	私は歌が下手だ。	わたしはうたがへただ 
\\	手紙の返事を出しました。	てがみのへんじをだしました 
\\	便所はそこです。	べんじょはそこです 
\\	今夜は星がよく見えます。	こんやはほしがよくみえます 
\\	その話は本当ですか。	そのはなしはほんとうですか 
\\	私は毎月貯金をしています。	わたしはまいつきちょきんをしています 
\\	日本では、二月に豆をまく行事がある。	にほんではにがつにまめをまくぎょうじがある 
\\	私は低い枕が好きです。	わたしはひくいまくらがすきです 
\\	昨日、孫が生まれました。	きのうまごがうまれました 
\\	明日、また来ます。	あしたまたきます 
\\	ここは小さな町です。	ここはちいさなまちです 
\\	紙に大きな丸を書きました。	かみにおおきなまるをかきました 
\\	港に船が着きました。	みなとにふねがつきました 
\\	私の家は町の南にあります。	わたしのいえはまちのみなみにあります 
\\	彼女は紫のドレスを着ていた。	かのじょはむらさきのどれすをきていた 
\\	私の姪は3才です。	わたしのめいはさんさいです 
\\	彼女は木綿のシャツを着ています。	かのじょはもめんのしゃつをきています 
\\	やかんでお湯を沸かしました。	やかんでおゆをわかしました 
\\	今、魚を焼いています。	いまさかなをやいています 
\\	私は少し痩せました。	わたしはすこしやせました 
\\	山の空気はきれいだ。	やまのくうきはきれいだ 
\\	お風呂のお湯が一杯です。	おふろのおゆがいっぱいです 
\\	夕食は7時です。	ゆうしょくはしちじです 
\\	昨夜、流れ星を見ました。	ゆうべながれぼしをみました 
\\	東京行きの列車に乗った。	とうきょうゆきのれっしゃにのった 
\\	彼は指が太い。	かれはゆびがふとい 
\\	このズボンは緩いです。	このずぼんはゆるいです 
\\	彼は良い人です。	かれはよいひとです 
\\	日本へようこそ。	にっぽんへようこそ 
\\	エプロンをしないと服が汚れます。	えぷろんをしないとふくがよごれます 
\\	あそこの四つ角を左に曲がってください。	あそこのよつかどをひだりにまがってください 
\\	夜中に電話がありました。	よなかにでんわがありました 
\\	私の両親は大阪に住んでいます。	わたしのりょうしんはおおさかにすんでいます 
\\	今ちょうど零時です。	いまちょうどれいじです 
\\	ポストはそこの曲がり角にあります。	ぽすとはそこのまがりかどにあります 
\\	ウェーターが水を運んできました。	うぇーたーがみずをはこんできました 
\\	熱があるのでお医者さんに行った。	ねつがあるのでおいしゃさんにいった 
\\	それは細かな問題です。	それはこまかなもんだいです 
\\	絶対に駄目です。	ぜったいにだめです 
\\	一度にそんなにたくさんはできない。	いちどにそんなにたくさんはできない 
\\	私も連れて行ってください。	わたしもつれていってください 
\\	息子が友達を連れて来ました。	むすこがともだちをつれてきました 
\\	私は母に似ています。	わたしはははににています 
\\	このピアノは音があまり良くない。	このぴあのはおとがあまりよくない 
\\	始めにスープが出ます。	はじめにすーぷがでます 
\\	今日は一人で映画を見ます。	きょうはひとりでえいがをみます 
\\	私の父は不動産屋です。	わたしのちちはふどうさんやです 
\\	郵便屋さんはもう来ましたか。	ゆうびんやさんはもうきましたか 
\\	このアルバイトは「学生可」です。	このあるばいとはがくせいかですね 
\\	今日は仕事が早く上がったんだ。	きょう は しごと が はやく あがった 
\\	私は
\\	業界で働いています。	わたし は あいてぃー ぎょうかい で はたらいて います 
\\	車は整備に出しています。	くるま は せいび に だして います 
\\	円高の影響で海外製品が安く買えますよ。	えんだか の えいきょう で かいがい せいひん が やすく かえます よ 
\\	テレビも取材に来たほど有名なレストランです。	てれび も しゅざい に きた ほど ゆうめい な れすとらん です 
\\	証券を売って資金にしようと思うの。	しょうけん を うって しきん に しよう と おもう の 
\\	当社は10年前に設立されたのよ。	とうしゃ は じゅうねんまえ に せつりつ された の よ 
\\	彼は大手のメーカーに勤めています。	かれ は おおて の めーかー に つとめて います 
\\	両社が合併に合意しました。	りょうしゃ が がっぺい に ごうい しました 
\\	彼は一家のお頭だったの。	かれ は いっか の おかしら だった の 
\\	あの会社の幹部は皆とても優秀だね。	あの かいしゃ の かんぶ は みんな とても ゆうしゅう だ ね 
\\	彼は新首相に就任しましたね。	かれ は しんしゅしょう に しゅうにん しました ね 
\\	お店に防犯カメラが設置されたわね。	その みせ に ぼうはん かめら が せっち された わ ね 
\\	議会で区の来年の予定が話し合われたんだ。	ぎかい で く の らいねん の よてい が はなしあわれた ん だ 
\\	庭の木、大きく成長したわね。	にわ の き が おおきく せいちょう した わ ね 
\\	彼は意識を失いました。	かれ は いしき を うしないました 
\\	新たな計画が進んでいます。	あらた な けいかく が すすんで います 
\\	大手チェーン店が進出している。	おおて ちぇーんてん が しんしゅつ して いる 
\\	この古い寺院は今でも完全な形を保っているんだ。	その ふるい じいん は いま で も かんぜん な かたち を たもって いる ん だ 
\\	委員会はその問題について協議しましたわ。	いいんかい は その もんだい に ついて きょうぎ しました わ 
\\	需要が多過ぎて生産が追い付きません。	じゅよう が おおすぎ て せいさん が おいつきません 
\\	事態は深刻です。	じたい は しんこく です 
\\	入学式の後、教科書を購入してください。	にゅうがくしき の あと きょうかしょ を こうにゅう して ください 
\\	ご指摘いただきありがとうございます。	ごしてき いただき ありがとう ございます 
\\	彼は政治に関心が強いね。	かれ は せいじ に かんしん が つよい ね 
\\	外相は来週訪米の予定です。	がいしょう は らいしゅう ほうべい の よてい です 
\\	私たちはレンタカーで移動しました。	わたしたち は れんたかー で いどう しました 
\\	コンピューターをネットワークに接続しました。	こんぴゅーたー を ねっとわーく に せつぞく しました 
\\	当社の製品は3年間の保障つきです。	とうしゃ の せいひん は さんねんかん の ほしょうつき です 
\\	日本は石油のほとんどを輸入しています。	にほん は せきゆ の ほとんど を ゆにゅう して います 
\\	彼は金融関係の会社で働いています。	かれ は きんゆう かんけい の かいしゃ で はたらいて います 
\\	彼女は昨日から私のことを避けているようなんだ。	かのじょ は きのう から わたし の こと を さけて いる よう なん だ 
\\	自分自身を疑ってはいけません。	じぶん じしん を うたがって は いけません 
\\	ここは職員専用の出入り口です。	ここ は しょくいん せんよう の でいりぐち です 
\\	彼の支援がなかったらどうなっていたか。	かれ の しえん が なかったら どう なって いた か 
\\	福祉予算は削減されないことに決まったのよ。	ふくしよさん は さくげん されない こと に きまった の よ 
\\	あの会社との契約が成立しました。	あの かいしゃ と の けいやく が せいりつ しました 
\\	彼女はインターネットを活用しているの。	かのじょ は いんたーねっと を かつよう して いる の 
\\	各地で大雨が降っています。	かくち で おおあめ が ふって います 
\\	このプリンターで写真を出力できます。	この ぷりんたー で しゃしん を しゅつりょく できます 
\\	彼は地震で親を失いました。	かれ は じしん で おや を うしないました 
\\	明日は市内を観光する予定です。	あした は しない を かんこう する よてい です 
\\	写真を額に入れて飾ったんだ。	しゃしん を がく に いれて かざった ん だ 
\\	全員で技術の向上に努めています。	ぜんいん で ぎじゅつ の こうじょう に つとめて います 
\\	私は彼の要望に応じました。	わたし は かれ の ようぼう に おうじました 
\\	最近、銀行の金利が少し上がった。	さいきん ぎんこう の きんり が すこし あがった 
\\	問題を解決できないのが現状です。	もんだい を かいけつ できない の が げんじょう です 
\\	その問題は審議中です。	その もんだい は しんぎちゅう です 
\\	現実は予想より厳しかった。	げんじつ は よそう より きびしかった 
\\	彼の行動は理解できない。	かれ の こうどう は りかい できない 
\\	高速バスで東京に行ったんだ。	こうそくばす で とうきょう に いった ん だ 
\\	それは果物の一種です。	それ は くだもの の いっしゅ です 
\\	彼は暴力事件を起こしたんだよ。	かれ は ぼうりょく じけん を おこした ん だ よ 
\\	都心と地方では収入に大きな差があるね。	としん と ちほう で は しゅうにゅう に おおき な さ が ある 
\\	大きくなったら政治家になりたいです。	おおきく なったら せいじか に なりたい です 
\\	その会社は欧米に進出しているよね。	その かいしゃ は おうべい に しんしゅつ して いる 
\\	会計士に資金運用について相談したよ。	かいけいし に しきん うんよう に ついて そうだん した よ 
\\	彼は営業を担当しています。	かれ は えいぎょう を たんとう して います 
\\	景気が回復してきたね。	けいき が かいふく して きた ね 
\\	社長の辞任が決まりました。	しゃちょう の じにん が きまりました 
\\	学会は予定通りに進行しています。	がっかい は よてい どおり に しんこう して います 
\\	今年の応募作品は水準が高かったね。	ことし の おうぼ さくひん は すいじゅん が たかかった ね 
\\	首相は国家のリーダーだ。	しゅしょう は こっか の りーだー だ 
\\	もっと案を出し合いましょう。	もっと あん を だしあいましょう 
\\	医師に入院を勧められたの。	いし に にゅういん を すすめられた の 
\\	家族で話し合う機会を持ちました。	かぞく で はなしあう きかい を もちました 
\\	彼は大蔵省に勤務しているんだよ。	かれ は おおくらしょう に きんむ して いる ん だ よ 
\\	体がすっかり回復した。	からだ が すっかり かいふく した 
\\	彼女は何とか健康を維持しているわね。	かのじょ は なんとか けんこう を いじ して いる わ ね 
\\	この研究所には最新の設備が揃っています。	この けんきゅうじょ に は さいしん の せつび が そろって います 
\\	彼は結局何を言いたかったのだろう。	かれ は けっきょく なに を いいたかった の だろう 
\\	当時のことはしっかり記憶しているよ。	わたし は とうじ の こと を しっかり きおく して いる よ 
\\	現地の天気は雨です。	げんち の てんき は あめ です 
\\	2社が共同で新製品を開発したんだ。	にしゃ が きょうどう で しんせいひん を かいはつ した ん だ 
\\	その事故では2人死亡したの。	その じこ で は ふたり しぼう した の 
\\	私は彼の意見を受け入れました。	わたし は かれ の いけん を うけいれました 
\\	両者の間に摩擦が生じています。	りょうしゃ の あいだ に まさつ が しょうじて います 
\\	今度のパソコンは性能がすごく良い。	こんど の ぱそこん は せいのう が すごく よい 
\\	私たちのチームに彼が加わった。	わたしたち の ちーむ に かれ が くわわった 
\\	あなたにまで迷惑が及んでごめんなさい。	あなた に まで めいわく が およんで ごめんなさい 
\\	彼は株式の売買で多額の利益を得たのさ。	かれ は かぶしき の ばいばい で たがく の りえき を えた の さ 
\\	今は骨が形成される大切な時期です。	いま は ほね が けいせい される たいせつ な じき です 
\\	会場は人で一杯になりました。	かいじょう は ひと で いっぱい に なりました 
\\	議長が開会の挨拶をしました。	ぎちょう が かいかい の あいさつ を しました 
\\	彼は精神の安定が必要よ。	かれ は せいしん の あんてい が ひつよう よ 
\\	チーム全員に名札が配られました。	ちーむ ぜんいん に なふだ が くばられました 
\\	父は議論好きです。	ちち は ぎろんずき です 
\\	インターネットは情報の概念を変えたよね。	いんたーねっと は じょうほう の がいねん を かえた よ ね 
\\	位が上がれば責任も増えるものだ。	くらい が あがれ ば せきにん も ふえる もの だ 
\\	図書館で資料を借りて来ました。	としょかん で しりょう を かりて きました 
\\	ここは昔、軍事基地でした。	ここ は むかし ぐんじ きち でした 
\\	以前彼はこの町に住んでいたんだ。	いぜん かれ は この まち に すんで いた ん だ 
\\	小型のスーツケースを買いました。	こがた の すーつけーす を かいました 
\\	警察の調べで女性の身元がわかりました。	けいさつ の しらべ で じょせい の みもと が わかりました 
\\	日本は消費大国といわれています。	にほん は しょうひ たいこく と いわれて います 
\\	その事業は国が運営しています。	その じぎょう は くに が うんえい して います 
\\	僕に隠していることは無いですか。	ぼく に かくして いる こと は ない です か 
\\	あの企業では学歴が重視されるよ。	あの きぎょう で は がくれき が じゅうし される よ 
\\	一度の失敗であきらめてはいけないよ。	いちど の しっぱい で あきらめては いけない よ 
\\	私は世界情勢を知るために毎日ニュースを見るわ。	わたし は せかい じょうせい を しる ため に まいにち にゅーす を みる わ 
\\	彼は人物を描くのがうまいな。	かれ は じんぶつ を えがく の が うまい な 
\\	近く交通法が改正されます。	ちかく こうつう ほう が かいせい されます 
\\	彼は会社の重要な人物です。	かれ は かいしゃ の じゅうよう な じんぶつ です 
\\	明日、委員会が開かれます。	あす いいんかい が ひらかれます 
\\	村では最近、不思議な現象が起きています。	むら で は さいきん ふしぎ な げんしょう が おきて います 
\\	私たちは共通の趣味を持っています。	わたしたち は きょうつう の しゅみ を もって います 
\\	そちらに着くのは6時前後です。	そちら に つく の は ろくじ ぜんご です 
\\	新しい代表チームが結成されました。	あたらしい だいひょう ちーむ が けっせい されました 
\\	3人分の代金は計6000円です。	さんにんぶん の だいきん は けい ろくせんえん です 
\\	彼は大きな荷物を抱えているわ。	かれ は おおき な にもつ を かかえて いる わ 
\\	上司の指示に従った。	じょうし の しじ に したがった 
\\	戦後の日本は混乱していました。	せんご の にほん は こんらん して いました 
\\	彼は目的のためには手段を選ばなかったわね。	かれ は もくてき の ため に は しゅだん を えらばなかった わ ね 
\\	代金には規定の手数料が含まれます。	だいきん に は きてい の てすうりょう が ふくまれます 
\\	相手会社から強い圧力が掛かった。	あいてがいしゃ から つよい あつりょく が かかった 
\\	会社は従業員の数を増やす予定だ。	かいしゃ は じゅうぎょういん の かず を ふやす よてい だ 
\\	彼は新しい小説の構想を練っているの。	かれ は あたらしい しょうせつ の こうそう を ねって いる の 
\\	日本の古い料理法についての記述を読んだんだ。	にほん の ふるい りょうりほう に ついて の きじゅつ を よんだ ん だ 
\\	母は苦労して私たちを育てたの。	はは は くろう して わたしたち を そだてた の 
\\	日時はご都合に合わせます。	にちじ は ごつごう に あわせます 
\\	勉強に集中しなさい。	べんきょう に しゅうちゅう しなさい 
\\	これは極めて重要な問題です。	これ は きわめて じゅうよう な もんだい です 
\\	彼は営業で実績を上げたんだ。	かれ は えいぎょう で じっせき を あげた ん だ 
\\	この犬は穏やかな性質だよ。	この いぬ は おだやか な せいしつ だ よ 
\\	父は以前、国鉄に勤めていました。	ちち は いぜん こくてつ に つとめて いました 
\\	ここに生年月日を記入してください。	ここ に せいねんがっぴ を きにゅう して ください 。 
\\	将来はパイロットになりたいです。	しょうらい は ぱいろっと に なりたい です 
\\	環境問題に関する記事を読んだの。	かんきょう もんだい に かんする きじ を よんだ の 
\\	交差点付近は駐車禁止です。	こうさてん ふきん は ちゅうしゃ きんし です 
\\	彼女は国際的に有名な歌手です。	かのじょ は こくさいてき に ゆうめい な かしゅ です 
\\	ここのパネルで機械全体を制御できます。	ここ の ぱねる で きかい ぜんたい を せいぎょ できます 
\\	彼らは家を建築中です。	かれら は いえ を けんちくちゅう です 
\\	この会社の売り上げは昨年の2倍ね。	その かいしゃ の うりあげ は さくねん の に ばい ね 
\\	図書館には各種の雑誌が揃っています。	としょかん に は かくしゅ の ざっし が そろって います 
\\	彼は自分の人生を振り返ったわ。	かれ は じぶん の じんせい を ふりかえった わ 
\\	町の青年たちはボランティア活動をしています。	まち の せいねんたち は ぼらんてぃあ かつどう を して います 
\\	政府は外交に力を入れているの。	せいふ は がいこう に ちから を いれて いる の 
\\	これまでで最高の結果が出たよ。	これまで で さいこう の けっか が でた 
\\	数学が平均点以下だった。	すうがく が へいきんてん いか だった 
\\	アルミニウムは金属の一種です。	あるみにうむ は きんぞく の いっしゅ です 
\\	日本語能力試験を受験したんだ。	にほんご のうりょく しけん を じゅけん した ん だ 
\\	プログラムの過程に問題があるの。	ぷろぐらむ の かてい に もんだい が ある の 
\\	あれは実は私の勘違いでした。	あれ は じつは わたし の かんちがい でした 
\\	牛肉の輸入が厳しく規制されているわね。	ぎゅうにく の ゆにゅう が きびしく きせい されて いる わ ね 
\\	ご注文の品は明日出荷致します。	ごちゅうもん の しな は あす しゅっか いたします 
\\	彼は本社で働くことを希望しています。	かれ は ほんしゃ で はたらく こと を きぼう して います 
\\	正式な招待状を受け取りました。	せいしき な しょうたいじょう を うけとりました 
\\	税に関する新しい法律が定められたぞ。	ぜい に かんする あたらしい ほうりつ が さだめられた ぞ 
\\	コンピュータの電子回路が故障した。	こんぴゅーた の でんし かいろ が こしょう した 
\\	彼の会社は電力を供給しています。	かれ の かいしゃ は でんりょく を きょうきゅう して います 
\\	東京で五カ国の首脳会談が開かれています。	とうきょう で ごかこく の しゅのうかいだん が ひらかれて います 
\\	この国は人口が減少しているわね。	この くに は じんこう が げんしょう して いる わ ね 
\\	指定された席にお座りください。	してい された せき に お すわり ください 
\\	彼は人事を担当しているんだ。	かれ は じんじ を たんとう して いる ん だ 
\\	僕と姉の性格は正反対です。	ぼく と あね の せいかく は せいはんたい です 
\\	現代の科学の進歩には驚くよ。	げんだい の かがく の しんぽ に は おどろく よ 
\\	原則としてキャンセルできません。	げんそく と して きゃんせる できません 
\\	彼は食生活を改善しました。	かれ は しょくせいかつ を かいぜん しました 
\\	皆の意見が一致しました。	みんな の いけん が いっち しました 
\\	棒をテープで固定しなさい。	ぼう を てーぷ で こてい しなさい 
\\	その会社の経営の実態を調査中だ。	その かいしゃ の けいえい の じったい を ちょうさちゅう だ 
\\	去る者は追わず。	さる もの は おわず 
\\	7時5分に駅で集合しましょう。	しち じ ご ふん に えき で しゅうごう しましょう 
\\	彼は校則に違反しました。	かれ は こうそく に いはん しました 
\\	彼は軍事に関わる仕事を続けてきたんだ。	かれ は ぐんじ に かかわる しごと を つづけて きた ん だ 
\\	必死の思いで彼に頼んだよ。	ひっし の おもい で かれ に たのんだ よ 
\\	彼女はクラブの会員です。	かのじょ は くらぶ の かいいん です 
\\	なんとか親を説得してみるよ。	なんとか おや を せっとく して みる よ 
\\	やっと宿題を終えた。	やっと しゅくだい を おえた 
\\	核戦争は絶対に防ぐべきよ。	かくせんそう は ぜったい に ふせぐ べき よ 
\\	このことを決して忘れないでください。	この こと を けっして わすれない で ください 
\\	厳しい練習が良い成果に結び付きました。	きびしい れんしゅう が よい せいか に むすびつきました 
\\	学校で避難訓練がありました。	がっこう で ひなん くんれん が ありました 
\\	日本人の平均収入はどの位ですか。	にほんじん の へいきん しゅうにゅう は どの くらい です か 
\\	私は彼と同じ世代です。	わたし は かれ と おなじ せだい です 
\\	ようやく結論に至ったようね	わたしたち は ようやく けつろん に いたった よう ね 
\\	警官は怪しい男の後を追ったよ。	けいかん は あやしい おとこ の あと を おった よ 
\\	あの国の経済は急速に発展しているのね。	あの くに の けいざい は きゅうそく に はってん して いる の ね 
\\	彼らは長い距離を歩き続けたんだ。	わたしたち は ながい きょり を あるきつづけた ん だ 
\\	書類は形式を守って作ってください。	しょるい は けいしき を まもって つくって ください 
\\	健康が一番大切だ。	けんこう が いちばん たいせつ だ 
\\	彼は生徒の指導が上手ね。	かれ は せいと の しどう が じょうず ね 
\\	彼は盗みの疑いをかけられたの。	かれ は ぬすみ の うたがい を かけられた の 
\\	優れた人材の確保は重要です。	すぐれた じんざい の かくほ は じゅうよう です 
\\	いろいろ工夫して仕事をやりとげたさ。	いろいろ くふう して しごと を やりとげた さ 
\\	試合の相手とユニフォームを交換したよ。	しあい の あいて と ゆにふぉーむ を こうかん した よ 
\\	医療関係の仕事をしています。	いりょう かんけい の しごと を して います 
\\	その会社はリサイクルを推進してるね。	その かいしゃ は りさいくる を すいしん して いる ね 
\\	この頃は、ローマが勢力を伸ばしていました。	この ころ は ろーま が せいりょく を のばして いました 
\\	会議で航空の安全について話し合ったの。	かいぎ で こうくう の あんぜん に ついて はなしあった 
\\	2国間で条約が結ばれました。	にこくかん で じょうやく が むすばれました 
\\	年金問題は行政の最大の課題のひとつよ。	ねんきん もんだい は ぎょうせい の さいだい の かだい の ひとつ よ 
\\	限りある資源を大切にしよう。	かぎり ある しげん を たいせつ に しよう 
\\	この法律は消費者を守るためのものです。	この ほうりつ は しょうひしゃ を まもる ため の もの です 
\\	石油はいろいろな製品の原料になります。	せきゆ は いろいろ な せいひん の げんりょう に なります 
\\	国の財政はとても苦しい状態だな。	くに の ざいせい は とても くるしい じょうたい だ な 
\\	死を恐れるのは自然なことです。	し を おそれる の は しぜん な こと です 
\\	このレシピはいろいろ応用できます。	この れしぴ は いろいろ おうよう できます 
\\	ビザについて外務省に問い合わせた。	びざ に ついて がいむしょう に といあわせた 
\\	沖にボートが浮かんでいます。	おき に ぼーと が うかんで います 
\\	今ドイツ語の基礎を学んでいます。	いま どいつご の きそ を まなんで います 
\\	地球環境の危機が叫ばれているのよ。	ちきゅう かんきょう の きき が さけばれて いる の よ 
\\	彼は非常に優れた選手です。	かれ は ひじょうに すぐれた せんしゅ です 
\\	彼の意向を聞いてみましょう。	かれ の いこう を きいて みましょう 
\\	雨が一層激しくなったね。	あめ が いっそう はげしく なった ね 
\\	叔父には心臓病の恐れがあります。	おじ に は しんぞうびょう の おそれ が あります 
\\	彼はその会社と雇用契約を結んだの。	かれ は その かいしゃ と こよう けいやく を むすんだ の 
\\	あの子は感じの良い子です。	あの こ は かんじ の いい こ です 
\\	彼は地方で演説したのよ。	かれ は ちほう で えんぜつ した の よ 
\\	彼に全てのことを伝えた。	かれ に すべて の こと を つたえた 
\\	この仕事には少なくとも2週間必要でしょう。	この しごと に は すくなくとも にしゅうかん ひつよう でしょう 
\\	東京行きの最終電車は何時ですか。	とうきょうゆき の さいしゅう でんしゃ は なんじ です か 
\\	彼らは教会で式を挙げました。	かれら は きょうかい で しき を あげました 
\\	ドラマの後半は来週放送されるんだ。	どらま の こうはん は らいしゅう ほうそう される ん だ 
\\	その都市は商業が盛んよ。	その とし は しょうぎょう が さかん よ 
\\	来月、隣の州に引っ越します。	らいげつ となり の しゅう に ひっこします 
\\	北欧では社会保障が確立されています。	ほくおう で は しゃかい ほしょう が かくりつ されて います 
\\	去年は航空機の事故が多かった。	きょねん は こうくうき の じこ が おおかった 
\\	焦点を絞って話しましょう。	しょうてん を しぼって はなしましょう 
\\	鼻水は風邪の初期症状のひとつです。	はなみず は かぜ の しょき しょうじょう の ひとつ です 
\\	彼女は出席を拒否した。	かのじょ は しゅっせき を きょひ した 
\\	まずは障害を取り除いてから計画を進めよう。	まず は しょうがい を とりのぞいて から けいかく を すすめよう 
\\	産業はインドの主要産業よね。	あいてぃー さんぎょう は いんど の しゅよう さんぎょう よ ね 
\\	これは深刻な問題です。	これ は しんこく な もんだい で す 
\\	作業の効率を上げるにはどうしたらいいですか。	さぎょう の こうりつ を あげる に は どう したら いい です か 
\\	彼の活躍で優勝したよ。	かれ の かつやく で ゆうしょう した よ 
\\	これからは自己の判断で行動してください。	これ から は じこ の はんだん で こうどう して ください 
\\	その選手の獲得賞金は2億円だって。	その せんしゅ の かくとく しょうきん は におくえん だって 
\\	その監督はアカデミー賞を受賞したよな。	その かんとく は あかでみーしょう を じゅしょう した よ な 
\\	その問題に対する策を皆で考えたの。	その もんだい に たいする さく を みんな で かんがえた の 
\\	彼は怒りを抑えていたの。	かれ は いかり を おさえて いた の 
\\	親切は必ずしも喜ばれるわけではない。	しんせつ は かならずしも よろこばれる わけ で は ない 
\\	あの二人はそっくりで区別できない。	あの ふたり は そっくり で くべつ できない 
\\	将来は海の近くで暮らしたいな。	しょうらい は うみ の ちかく で くらしたい な 
\\	夕方以降にお電話を下さい。	ゆうがた いこう に お でんわ を ください 
\\	大臣の資産が公表されたね。	だいじん の しさん が こうひょう された ね 
\\	彼は緊張していたな。	かれ は きんちょう して いた な 
\\	その広告を新聞で見ました。	その こうこく を しんぶん で みました 
\\	彼女は食事を制限しています。	かのじょ は しょくじ を せいげん して います 
\\	自宅から職場まで1時間かかります。	じたく から しょくば まで いちじかん かかります 
\\	最近ヨガをやる人が急増しています。	さいきん よが を やる ひと が きゅうぞう して います 
\\	彼の今月の業績は素晴らしいです。	かれ の こんげつ の ぎょうせき は すばらしい です 
\\	このケースには
\\	が50枚入ります。	この ケース に は 
\\	が 
\\	まい はいります 。 
\\	彼には清潔なイメージがあるね。	かれ に は せいけつ な イメージ が ある ね 。 
\\	自分の失敗を他人のせいにするな。	じぶん の しっぱい を たにん の せい に する な 。 
\\	彼女の様子がいつもと違う。	かのじょ の ようす が いつも と ちがう 。 
\\	大学に通うのが上京のきっかけでした。	だいがく に かよう の が じょうきょう の きっかけ でした 。 
\\	買い物をすっかり忘れていた。	かいもの を すっかり わすれていた 。 
\\	このシリーズは50話もあるそうよ。	この シリーズ は 
\\	わ も ある そう よ 。 
\\	その国ではエイズが大きな社会問題ね。	その くに で は エイズ が おおきな しゃかい もんだい ね 。 
\\	車のエンジンが調子悪い。	くるま の エンジン が ちょうし わるい 。 
\\	もう一度慎重に見直しましょう。	もういちど しんちょう に みなおしましょう 
\\	この荷物は丁寧に扱って下さい。	この にもつ は ていねい に あつかって ください 
\\	一般に老人は早起きです。	いっぱんに ろうじん は はやおき です 
\\	今日は野外で学習した。	きょう は やがい で がくしゅう した 
\\	国連の本部はニューヨークにあります。	こくれん の ほんぶ は にゅーよーく に あります 
\\	カウンターで料金を支払った。	かうんたー で りょうきん を しはらった 
\\	彼の行為はみんなの誤解を招いたわ。	かれ の こうい は みんな の ごかい を まねいた わ 
\\	困難にあってもあきらめてはいけないよ。	こんなん に あって も あきらめて は いけない よ 
\\	容疑者のアリバイが証明されました。	ようぎしゃ の ありばい が しょうめい されました 
\\	温かい歓迎を受けました。	あたたかい かんげい を うけました 
\\	私は部下を信頼しています。	わたし は ぶか を しんらい して います 
\\	京都で3日間観光しました。	きょうと で みっかかん かんこう しました 
\\	レシートで買い物の金額を確かめたよ。	れしーと で かいもの の きんがく を たしかめた よ 
\\	このドアは自動よ。	この どあ は じどう よ 
\\	これから高齢化が急激に進みます。	これから こうれいか が きゅうげき に すすみます 
\\	今日は組合の集まりがあるわ。	きょう は くみあい の あつまり が ある わ 
\\	時計が12時を指してる。	とけい が じゅうにじ を さして る 
\\	冷めないうちにどうぞお上がり下さい。	さめない うち に どうぞ おあがり ください 
\\	これは一種独特の味がするね。	これ は いっしゅ どくとく の あじ が する ね 
\\	大声で話すと周囲の人に迷惑ですよ。	おおごえ で はなす と しゅうい の ひと に めいわく です よ 
\\	撮影は3ヶ月かけて行われました。	さつえい は さんかげつ かけて おこなわれました 
\\	彼は支店に転勤したよ。	かれ は してん に てんきん した よ 
\\	彼は委員に選ばれました。	かれ は いいん に えらばれました 
\\	今日の日本の若者は欧米化している。	こんにち の にほん の わかもの は おうべいか して いる 
\\	進み具合は全て順調です。	すすみ ぐあい は すべて じゅんちょう です 
\\	彼の会社は移転したよ。	かれ の かいしゃ は いてん した よ 
\\	先生が星の観測に連れて行ってくれました。	せんせい が ほし の かんそく に つれて いって くれました 
\\	彼女は心理学に興味があるよ。	かのじょ は しんりがく に きょうみ が ある よ 
\\	明日の試合は何時からですか。	あした の しあい は なんじ から です か 
\\	手の傷が悪化した。	て の きず が あっか した 
\\	あの人の顔には自信があふれているね。	あの ひと の かお に は じしん が あふれて いる ね 
\\	選挙では3つの政党が争っています。	せんきょ で は みっつ の せいとう が あらそって います 
\\	彼は作家であり学者です。	かれ は さっか で あり がくしゃ です 
\\	これは政府の承認を受けた資格です。	これ は せいふ の しょうにん を うけた しかく です 
\\	夏休みを高原で過ごしました。	なつやすみ を こうげん で すごしました 
\\	彼は課長に昇進しました。	かれ は かちょう に しょうしん しました 
\\	地震の後、町は大混乱だったよ。	じしん の あと まち は だいこんらん だった よ 
\\	これからパスポートの申請に行きます。	これから ぱすぽーと の しんせい に いきます 
\\	もう我慢の限界です。	もう がまん の げんかい です 
\\	教授はカナダ出身です。	きょうじゅ は かなだ しゅっしん です 
\\	この仕事は来月の下旬には終わります。	この しごと は らいげつ の げじゅん に は おわります 
\\	新しい市長が選ばれました。	あたらしい しちょう が えらばれました 
\\	あの巨大な建物は博物館です。	あの きょだい な たてもの は はくぶつかん です 
\\	高度な技術を持つ人材を集めました。	こうど な ぎじゅつ を もつ じんざい を あつめました 
\\	姉は何に対しても積極的です。	あね は なに に たいして も せっきょくてき です 
\\	地図にはいろいろな記号が使われているのね。	ちず に は いろいろ な きごう が つかわれて いる の ね 
\\	今日の話し合いでは結論が出なかった。	きょう の はなしあい で は けつろん が でなかった 
\\	主婦の主な仕事は家事よ。	しゅふ の おも な しごと は かじ よ 
\\	私はもう決心を固めたの。	わたし は もう けっしん を かためた の 
\\	緊急事態が発生した。	きんきゅう じたい が はっせい した 
\\	宇宙の謎は限りなく大きいの。	うちゅう の なぞ は かぎり なく おおきい の 
\\	この機械の仕組みは複雑だ。	この きかい の しくみ は ふくざつ だ 
\\	天皇は日本国の象徴です。	てんのう は にほんこく の しょうちょう です 
\\	それ以来彼女に会ってない。	それ いらい かのじょ に あってない 
\\	独身者パーティーを企画した。	どくしんしゃ ぱーてぃー を きかく した 
\\	もっと他の町との交流を深めましょう。	もっと ほか の まち と の こうりゅう を ふかめましょう 
\\	あなたの事情を考慮して予定をたてました。	あなた の じじょう を こうりょ して よてい を たてました 
\\	裁判の様子はテレビで中継されたわよ。	さいばん の ようす は てれび で ちゅうけい された わ よ 
\\	彼から良い印象を受けたわ。	かのじょ から よい いんしょう を うけた わ 
\\	山で登山客が熊に襲われたよ。	やま で とざん きゃく が くま に おそわれた よ 
\\	感情とは複雑なものです。	かんじょう と は ふくざつ な もの です 
\\	私は彼の意見に賛成だ。	わたし は かれ の いけん に さんせい だ 
\\	冷えて指の感覚がない。	ひえて ゆび の かんかく が ない 
\\	道で警官に呼び止められた。	みち で けいかん に よびとめられた 
\\	これからは経費を節約しましょう。	これ から は けいひ を せつやく しましょう 
\\	彼女は有名な作家です。	かのじょ は ゆうめい な さっか です 
\\	人類は2本の足で歩きます。	じんるい は にほん の あし で あるきます 
\\	これは税金の無駄遣いだね。	これ は ぜいきん の むだづかい だ ね 
\\	この映画は香港で製作されました。	この えいが は ほんこん で せいさく されました 
\\	彼らは次の試合のために作戦を立てたのよ。	かれら は つぎ の しあい の ため に さくせん を たてた の よ 
\\	この本は15年前に出版された。	この ほん は じゅうごねんまえ に しゅっぱん された 
\\	この大学も国際化してきたな。	この だいがく も こくさいか して きた な 
\\	兄は商社に勤めています。	あに は しょうしゃ に つとめて います 
\\	私たちには知る権利があります。	わたしたち に は しる けんり が あります 
\\	ここに去年の所得をご記入ください。	ここ に きょねん の しょとく を ご きにゅう ください 
\\	彼はボールに回転を掛けたんだ。	かれ は ぼーる に かいてん を かけた ん だ 
\\	久しぶりに家族全員で食卓を囲みました。	ひさしぶり に かぞく ぜんいん で しょくたく を かこみました 
\\	ものごとは根気よく継続することが大切です。	もの ごと は こんき よく けいぞく する こと が たいせつ です 
\\	憲法を改正することは難しいわね。	けんぽう を かいせい する こと は むずかしい わ ね 
\\	彼女は今、精神が不安定だ。	かのじょ は いま せいしん が ふあんてい だ 
\\	国会の会期が延長されたわね。	こっかい の かいき が えんちょう された わ ね 
\\	これは外部には秘密です。	これ は がいぶ に は ひみつ です 
\\	これを縮小して印刷してください。	これ を しゅくしょう して いんさつ して ください 
\\	科学技術は目覚しく進歩しているの。	かがく ぎじゅつ は めざましく しんぽ して いる の 
\\	小屋に石炭の山があります。	こや に せきたん の やま が あります 
\\	少年よ大志を抱け。	しょうねん よ たいし を いだけ 
\\	あの一家は仲がいい。	あの いっか は なか が いい 
\\	疲れたでしょう、一旦休みましょう。	つかれた でしょう いったん やすみましょう 
\\	彼は仕事に意欲を燃やしています。	かれ は しごと に いよく を もやして います 
\\	彼の言葉が本当かは疑問です。	かれ の ことば が ほんとう か は ぎもん です 
\\	とても価値のある話を聞いたよ。	とても かち の ある はなし を きいた よ 
\\	試合はすぐに再開されたの。	しあい は すぐ に さいかい された の 
\\	彼女の詩は世界中で有名になったんだ。	かのじょ の し は せかいじゅう で ゆうめい に なった ん だ 
\\	その事故で30人の死者が出たの。	その じこ で さんじゅうにん の ししゃ が でた の 
\\	私は彼の説が正しいと思う。	わたし は かれ の せつ が ただしい と おもう 
\\	この表は年齢区分ごとの人口を表しています。	この ひょう は ねんれい くぶん ごと の じんこう を あらわして います 
\\	食品は日曜日にまとめて買います。	しょくひん は にちようび に まとめて かいます 
\\	彼女は久しぶりに故郷に帰りました。	かのじょ は ひさしぶり に こきょう に かえりました 
\\	彼はこの国で大きな権力を持っているわ。	かれ は この くに で おおき な けんりょく を もって いる わ 
\\	私は郊外に家を買いました。	わたし は こうがい に いえ を かいました 
\\	その男は彼女のバッグを奪ったぞ。	その おとこ は かのじょ の ばっぐ を うばった ぞ 
\\	日本では絶えず米の品種を改良しているの。	にほん で は たえず こめ の ひんしゅ を かいりょう して いる の 
\\	正しい答は後者です。	ただしい こたえ は こうしゃ です 
\\	今後の方針が公式に発表されました。	こんご の ほうしん が こうしき に はっぴょう されました 
\\	この本の作者は誰ですか。	この ほん の さくしゃ は だれ です か 
\\	駅は逆方向です。	えき は ぎゃくほうこう です 
\\	彼女は毎日の暮らしを楽しんでいますね。	かのじょ は まいにち の くらし を たのしんで います ね 
\\	わが社は今年度、黒字となりました。	わがしゃ は こんねんど くろじ と なりました 
\\	祖母は心臓が悪いんだ。	そぼ は しんぞう が わるい ん だ 
\\	ここにタイヤの跡があるわ。	ここ に たいや の あと が ある わ 
\\	アンケートに回答しました。	あんけーと に かいとう しました 
\\	休日は植物の世話をして過ごします。	きゅうじつ は しょくぶつ の せわ を して すごします 
\\	試合の代表メンバーが確定したの。	しあい の だいひょう めんばー が かくてい した の 
\\	荷物はここに重ねてください。	にもつ は ここ に かさねて ください 
\\	ここは芸術を愛する国です。	ここ は げいじゅつ を あいする くに です 
\\	その会社は店舗を拡張していますね。	その かいしゃ は てんぽ を かくちょう して います ね 
\\	あなたの見解を聞かせてください。	あなた の けんかい を きかせて ください 
\\	そのクイズ番組は脳を刺激するね。	その くいずばんぐみ は のう を しげき する ね 
\\	彼は会長に推薦されたよ。	かれ は かいちょう に すいせん された よ 
\\	今日は最高にいい気分だよ。	きょう は さいこう に いい きぶん だ よ 
\\	公園で死体が見つかったの。	こうえん で したい が みつかった の 
\\	そこの角で車同士が衝突したのよ。	そこ の かど で くるま どうし が しょうとつ した の よ 
\\	彼の成長は著しいです。	かれ の せいちょう は いちじるしい です 
\\	感動する映画でした。	かんどう する えいが でした 
\\	新しい型のカメラを買いました。	あたらしい かた の かめら を かいました 
\\	卒業の記念にみんなで旅行したよ。	そつぎょう の きねん に みんな で りょこう した よ 
\\	手術後の経過は順調です。	しゅじゅつ ご の けいか は じゅんちょう です 
\\	余りは次回使いましょう。	あまり は じかい つかいましょう 
\\	彼に協力を依頼しました。	かれ に きょうりょく を いらい しました 
\\	古い服を整理しました。	ふるい ふく を せいり しました 
\\	彼らの海外公演が発表されたよ。	かれら の かいがいこうえん が はっぴょう された よ 
\\	彼らは一斉に拍手したね。	かれら は いっせいに はくしゅ した ね 
\\	彼は国民の支持を得たのよ。	かれ は こくみん の しじ を えた の よ 
\\	その事件は次第に忘れられていったのさ。	その じけん は しだいに わすれられて いった の さ 
\\	この商品は主に女性に人気がありまして。	この しょうひん は おもに じょせい に にんき が ありまし て 
\\	彼は貴重な体験をしたわね。	かれ は きちょう な たいけん を した わ ね 
\\	例を幾つか挙げてみましょう。	れい を いくつ か あげて みましょう 
\\	その川の水は汚染されています。	その かわ の みず は おせん されて います 
\\	見学者に機械の動く原理を説明しました。	けんがくしゃ に きかい の うごく げんり を せつめい しました 
\\	ここは交通の便がよいですね。	ここ は こうつう の べん が よい です ね 
\\	の値を求めなさい。	えっくす の あたい を もとめなさい 
\\	彼は悪い習慣を改めようとしているわね。	かれ は わるい しゅうかん を あらためよう と して いる わ ね 
\\	合計金額を計算してください。	ごうけい きんがく を けいさん して ください 
\\	宗教を持たない人もたくさんいるわ。	しゅうきょう を もたない ひと も たくさん いる わ 
\\	高校卒業後、系列の大学に進みました。	こうこう そつぎょう ご けいれつ の だいがく に すすみました 
\\	テロ集団が警察に捕まりました。	てろ しゅうだん が けいさつ に つかまりました 
\\	彼は1年2組の生徒です。	かれ は いちねん にくみ の せいと です 
\\	地震のあとは津波に警戒してください。	じしん の あと は つなみ に けいかい して ください 
\\	検討の結果、案を修正したよ。	けんとう の けっか あん を しゅうせい した よ 
\\	私の一日の勤務時間は8時間です。	わたし の いちにち の きんむじかん は はちじかん です 
\\	このプロジェクトで私は彼と組んでいるんだ。	この ぷろじぇくと で わたし は かれ と くんで いる ん だ 
\\	解答に誤りが3つ有りますよ。	かいとう に あやまり が みっつ あります よ 
\\	この辺一帯はリンゴ畑です。	この へん いったい は りんごばたけ です 
\\	彼女は最近元気がないですね。	かのじょ は さいきん げんき が ない です ね 
\\	今月はエアコンの売り上げが好調です。	こんげつ は えあこん の うりあげ が こうちょう です 
\\	この製品を使用する前に、説明書をお読みください。	この せいひん を しよう する まえ に せつめいしょ を およみ ください 
\\	私は前者の方が優れていると思います。	わたし は ぜんしゃ の ほう が すぐれて いる と おもいます 
\\	一体何事ですか。	いったい なにごと です か 
\\	首相は内閣の改造を行いました。	しゅしょう は ないかく の かいぞう を おこないました 
\\	そのことは既にみんな知っています。	その こと は すでに みんな しって います 
\\	大いに学び、大いに遊びなさい。	おおいに まなび おおいに あそびなさい 
\\	今日は学校で行事がありました。	きょう は がっこう で ぎょうじ が ありました 
\\	彼は実験を試みたのよ。	かれ は じっけん を こころみた 
\\	このワインは国産です。	この わいん は こくさん です 
\\	私の話を落ち着いて聞いてください。	わたし の はなし を おちついて きいて ください 
\\	彼の発言に対してたくさんの抗議があったよ。	かれ の はつげん に たいして たくさん の こうぎ が あった 
\\	この薬は神経に作用します。	この くすり は しんけい に さよう します 
\\	私は一切その問題とは関係がありません。	わたし は いっさい その もんだい と は かんけい が ありません 
\\	貧富の格差が大きくなっているな。	ひんぷ の かくさ が おおきく なって いる な 
\\	日本語を教える資格を取りたいです。	にほんご を おしえる しかく を とりたい です 
\\	これは強力な接着剤ね。	これ は きょうりょく な せっちゃくざい ね 
\\	証拠を見つけるまで逮捕はできません。	しょうこ を みつける まで たいほ は できません 
\\	先生に許可をもらって早退しました。	せんせい に きょか を もらって そうたい しました 
\\	コンサートは夜7時5分に終了しました。	こんさーと は よる しち じ ご ふん に しゅうりょう しました 
\\	代金は現金でお願いします。	だいきん は げんきん で おねがい します 
\\	地球の資源を守りましょう。	ちきゅう の しげん を まもりましょう 
\\	彼女が将来、大統領になることは確実です。	かのじょ が しょうらい だいとうりょう に なる こと は かくじつ です 
\\	冬は空気が乾燥しますね。	ふゆ は くうき が かんそう します ね 
\\	全ての人に人権がある。	すべて の ひと に じんけん が ある 
\\	金属は高温で溶かします。	こうおん で きんぞく を とかします 
\\	彼は真剣に話を聞いていました。	かれ は しんけん に はなし を きいて いました 
\\	彼の決意は堅いな。	かれ の けつい は かたい な 
\\	その芸術祭は市が主催しています。	その げいじゅつさい は し が しゅさい して います 
\\	あなたの意図はよく分かりました。	あなた の いと は よく わかりました 
\\	留守番電話のメッセージを再生したの。	るすばんでんわ の めっせーじ を さいせい した の 
\\	そんなの常識だよ。	そんな の じょうしき だ よ 
\\	彼は神経が細かいですね。	かれ は しんけい が こまかい です ね 
\\	それは歴史上の大きな革命です。	それ は れきしじょう の おおき な かくめい です 
\\	この線を軸にして図形を回転してください。	この せん を じく に して ずけい を かいてん して ください 
\\	先生のお陰で大学に合格できました。	せんせい の おかげ で だいがく に ごうかく できました 
\\	お支払いはカードもお使いいただけます。	おしはらい は かーど も お つかい いただけます 
\\	彼は受験の失敗を恐れています。	かれ は じゅけん の しっぱい を おそれて います 
\\	もうプリンは固まったかな。	もう ぷりん は かたまった か な 
\\	ごま油の代わりにオリーブ油を使いましょう。	ごまあぶら の かわり に おりーぶゆ を つかいましょう 
\\	腹一杯食った。	はらいっぱい くった 
\\	詳細はお気軽にお問い合わせ下さい。	しょうさい は おきがる に おといあわせ ください 
\\	彼女は京都の生まれです。	かのじょ は きょうと の うまれ です 
\\	彼は少年の頃から学問が好きでした。	かれ は しょうねん の ころ から がくもん が すき でした 
\\	雨の日は子供を室内で遊ばせます。	あめ の ひ は こども を しつない で あそばせます 
\\	これは今までで最低の記録だ。	これ は いま まで で さいてい の きろく だ 
\\	その権力者による支配は50年以上続いたんです。	その けんりょく しゃ による しはい は 
\\	ねん いじょう つづいた ん です 。 
\\	大きく息を吸ってください。	おおきく いき を すって ください 。 
\\	これ以上税が上がると生活できないね。	これ いじょう ぜい が あがる と せいかつ できない ね 。 
\\	最近はテレビを見る回数が減りました。	さいきん は テレビ を みる かいすう が へりました 。 
\\	雪の結晶にはいろいろな形があります。	ゆき の けっしょう に は いろいろ な かたち が あります 。 
\\	欠点のない人間はいません。	けってん の ない にんげん はいません 。 
\\	私たちは万歳をして喜びを表しました。	わたし たち は ばんざい を して よろこび を あらわしました 。 
\\	彼は医学を学んでいます。	かれ は いがく を まなん でいます 。 
\\	そのチームには勢いがありますね。	その チーム に は いきおい が あります ね 。 
\\	その国には物資の援助が必要です。	その く に には ぶっし の えんじょ が ひつよう です 。 
\\	大雨で崖が崩れたね。	おおあめ で がけ が くずれた ね 。 
\\	飛行機の座席はゆったりしていたよ。	ひこうき の ざせき は ゆったり していた よ 。 
\\	もっと照明を明るくしてください。	もっと しょうめい を あかるく して ください 。 
\\	化粧品は化学工場で作られます。	けしょう ひん は かがく こうじょう で つくられます 。 
\\	彼女はショックで口もきけなかったわ。	かのじょ は ショック で くち も きけなかった わ 。 
\\	屋上にアンテナが立っていますね。	おくじょう に アンテナ が たっています ね 。 
\\	彼はあえて危険を冒したの。	かれ は あえて きけん を おかした の 。 
\\	本にカバーを掛けました。	ほん に カバー を かけました 。 
\\	彼女は有能なジャーナリストだ。	かのじょ は ゆうのう な ジャーナリスト だ 。 
\\	彼女はモデルのようにスタイルがいいね。	かのじょ は モデル の よう に スタイル が いい ね 。 
\\	彼はインタビューに、はきはきと答えてたよ。	かれ は インタビュー に 、 はきはき と こたえてた よ 。 
\\	彼女たちはそれぞれ個性が強いですね。	かのじょ たち は それぞれ こせい が つよい です ね 。 
\\	地震は自然災害のひとつです。	じしん は しぜん さいがい の ひとつ です 。 
\\	彼にも一応知らせておいたほうがいいだろう。	かれ に も いちおう しらせておいた ほう が いい だろう 。 
\\	テレビの画面が明るすぎる。	テレビ の がめん が あかるすぎる 。 
\\	暮れは用事が多くて忙しいです。	くれ は ようじ が おおくて いそがしい です 。 
\\	私たちは公害を減らすよう努力しています。	わたし たち は こうがい を へらす よう どりょく しています 。 
\\	この本を参考にして下さい。	この ほん を さんこう に してください 。 
\\	このデザインは数々の賞を受けています。	この デザイン は かずかず の しょう を うけています 。 
\\	労働は国民の義務です。	ろうどう は こくみん の ぎむ です 。 
\\	突然の悲しみが一家を襲いました。	とつぜん の かなしみ が いっか を おそいました 。 
\\	生命は海から始まったと言われている。	せいめい は うみ から はじまった と いわれている 。 
\\	彼ったら怒り爆発だったよ。	かれったら いかり ばくはつ だった よ 。 
\\	今年の夏は異常な暑さですね。	ことし の なつ は いじょう な あつ さ です ね 。 
\\	命より大切なものは無いよ。	いのち より たいせつ な もの は ない よ 。 
\\	大勢が応援に駆けつけてくれたよ。	たいせい が おうえん に かけつけてくれた よ 。 
\\	お気に入りのカップが欠けてしまいました。	おきにいり の カップ が かけてしまいました 。 
\\	高速道路は混んでたよ。	こうそく どうろ は こん でた よ 。 
\\	友達のお父さんが突然失業したの。	ともだち の おとうさん が とつぜん しつぎょう した の 。 
\\	今日は8チームが出場しました。	きょう は 
\\	チーム が しゅつじょう しました 。 
\\	仮にそれが事実だとしてももう遅いよ。	かりに それ が じじつ だ として も もう おそい よ 。 
\\	近代の技術の発展はものすごいです。	きんだい の ぎじゅつ の はってん は ものすごい です 。 
\\	専門家を講師に招いた。	せんもん か を こうし に まねいた 。 
\\	視点を変えて見てみましょう。	してん を かえて みてみましょう 。 
\\	医者に症状を説明したよ。	いしゃ に しょうじょう を せつめい した よ 。 
\\	一列に並んで前進して下さい。	いち れつ に ならん で ぜん しん してください 。 
\\	すみません、以後気を付けます。	すみません 、 いご き を つけます 。 
\\	店の前に長い行列ができていますよ。	みせ の まえ に ながい ぎょうれつ が できています よ 。 
\\	小学校で子供たちの劇を見ました。	しょうがっこう で こども たち の げき を みました 。 
\\	この詩を解釈してみましょう。	この し を かい しゃく してみましょう 。 
\\	一体何の騒ぎですか。	いったい なに の さわぎ です か 。 
\\	彼は学校の劇で主役に選ばれたよ。	かれ は がっこう の げき で しゅやく に えらばれた よ 。 
\\	赤ちゃんの笑顔が可愛いい。	あかちゃん の えがお が かわい いい 。 
\\	窓に男性の影が映っています。	まど に だんせい の かげ が うつっています 。 
\\	彼女はラジオ局で働いています。	かのじょ は ラジオ きょく で はたらいています 。 
\\	それは極端な意見だね。	それ は きょくたん な いけん だ ね 。 
\\	車に駐車違反の警告を貼られました。	くるま に ちゅうしゃ いはん の けいこく を はられました 。 
\\	午後8時に劇場の前で会いましょう。	ごご 
\\	じ に げきじょう の まえ で あいましょう 。 
\\	原子力の安全な利用について考えましょう。	げんしりょく の あんぜん な りよう について かんがえましょう 。 
\\	その殺人事件は白昼に起こったんだ。	その さつじん じけん は はくちゅう に おこった ん だ 。 
\\	図を描いて説明しましょう。	ず を えがいて せつめい しましょう 。 
\\	彼はよく刑事の役を演じるね。	かれ は よく けいじ の やく を えんじる ね 。 
\\	新聞の解説欄を読みました。	しんぶん の かいせつ らん を よみました 。 
\\	高速道路で事故が重なった。	こうそく どうろ で じこ が かさなった 。 
\\	私たちは高級ホテルに泊まったの。	わたし たち は こうきゅう ホテル に とまった の 。 
\\	彼は差別をなくす運動をしています。	かれ は さべつ を なくす うんどう を しています 。 
\\	大学卒業は一昨年です。	だいがく そつぎょう は いっさくねん です 。 
\\	彼は高校の教員です。	かれ は こうこう の きょういん です 。 
\\	彼らは裁判所の前で知らせを待っています。	かれら は さいばんしょ の まえ で しらせ を まっています 。 
\\	学校では音楽部に所属していました。	がっこう で は おんがく ぶ に しょぞく していました 。 
\\	彼は額に汗をかいていたの。	かれ は がく に あせ を かいていた の 。 
\\	彼は数々の名作を生み出した。	かれ は かずかず の めいさく を うみだした 。 
\\	雑音がひどくて声が聞こえません。	ざつおん が ひどくて こえ が きこえません 。 
\\	辺りを見回したの。	あたり を みまわした の 。 
\\	彼はお金をだいぶ稼いだらしいよ。	かれ は おかね を だいぶ かせいだ らしい よ 。 
\\	私は今日、目の検査を受けます。	わたし は きょう 、 め の けんさ を うけます 。 
\\	この地方ではみかんの栽培が盛んです。	この ちほう で は みかん の さいばい が さかん です 。 
\\	ボールが速過ぎて受け止められなかったの。	ボール が はやすぎて うけとめられなかった の 。 
\\	俺の頼みを聞いてくれ。	おれ の たのみ を きいてくれ 。 
\\	彼はとても有名な科学者だ。	かれ は とても ゆうめい な かがく しゃ だ 。 
\\	火災の原因は放火だそうです。	かさい の げんいん は ほうか だ そう です 。 
\\	何を根拠にそんな事を言うのですか。	なに を こんきょ に そんな こと を いう の です か 。 
\\	来年、衆議院が解散されるだろう。	らいねん 、 しゅうぎいん が かい さん される だろう 。 
\\	海水から塩を作ります。	かいすい から しお を つくります 。 
\\	今日は体の具合が悪いです。	きょう は からだ の ぐあい が わるい です 。 
\\	勝負はまだ始まったばかりよ。	しょうぶ は まだ はじまった ばかり よ 。 
\\	意外にも彼は独身です。	いがい に も かれ は どくしん です 。 
\\	ここは危険区域よ。	ここ は きけん くいき よ 。 
\\	幸い、電車に嵐の影響はなかった。	さいわい 、 でんしゃ に あらし の えいきょう は なかった 。 
\\	彼は機械の操作を誤った。	かれ は きかい の そうさ を あやまった 。 
\\	ペットを飼ったことはありますか。	ペット を かった こと は あります か 。 
\\	彼は学会で論文を発表したよ。	かれ は がっかい で ろんぶん を はっぴょう した よ 。 
\\	事務の経験が3年あります。	じむ の けいけん が 
\\	ねん あります 。 
\\	これは人工の湖です。	これ は じんこう の みずうみ です 。 
\\	台風が接近していますね。	たいふう が せっきん しています ね 。 
\\	経済的な効果は計り知れません。	けいざい てき な こうか は はかり しれません 。 
\\	年賀状を印刷したの。	ねんがじょう を いんさつ した の 。 
\\	電車が10分遅れで到着したんだ。	でんしゃ が 
\\	ふん おくれ で とうちゃく した ん だ 。 
\\	彼女の瞳は喜びで輝いていますね。	かのじょ の ひとみ は よろこび で かがやいています ね 。 
\\	彼女は背の高い男性を好みますね。	かのじょ は せ の たかい だんせい を このみます ね 。 
\\	あいうえお順に並べて下さい。	あ いう え お じゅん に ならべてください 。 
\\	その選手は今シーズンも好調だね。	その せんしゅ は こん シーズン も こうちょう だ ね 。 
\\	重大な発表があります。	じゅうだい な はっぴょう が あります 。 
\\	この荷物はかなりの重量ですね。	この にもつ は かなり の じゅうりょう です ね 。 
\\	自分らしい生き方をしなさい。	じぶん らしい いきかた を しなさい 。 
\\	王の墓を見学したよ。	おう の はか を けんがく した よ 。 
\\	彼はあの通りに店を構えているの。	かれ は あの とおり に みせ を かまえている の 。 
\\	足の傷が痛みます。	あし の きず が いたみます 。 
\\	市は昨年から歩きタバコを禁じているの。	し は さくねん から あるき タバコ を きんじている の 。 
\\	心臓は全身に血液を送り出している。	しんぞう は ぜんしん に けつえき を おくりだしている 。 
\\	父の手術が無事に済みました。	ちち の しゅじゅつ が ぶじ に すみました 。 
\\	そんなことは絶対できないよ。	そんな こと は ぜったい できない よ 。 
\\	彼がいいアイデアを出したね。	かれ が いい アイデア を だした ね 。 
\\	一緒に食事しようか。	いっしょ に しょくじ しよう か 。 
\\	外貨を両替しました。	がいか を りょうがえ しました 。 
\\	この花はいい香りがしますね。	この はな は いい かおり が します ね 。 
\\	これは世界で最小のコンピューターです。	これ は せかい で さいしょう の コンピューター です 。 
\\	夏は水分を十分に補いましょう。	なつ は すいぶん を じゅうぶん に おぎないましょう 。 
\\	この角度からは画面が見えにくいですね。	この かくど から は がめん が みえにくい です ね 。 
\\	昨日火山が爆発したの。	きのう かざん が ばくはつ した の 。 
\\	私の父は公務員です。	わたし の ちち は こうむいん です 。 
\\	この事業には大きな意義があります。	この じぎょう に は おおきな いぎ が あります 。 
\\	人質が解放されてよかった。	ひとじち が かいほう されてよかった 。 
\\	彼は新しい仕事に人生を賭けている。	かれ は あたらしい しごと に じんせい を かけている 。 
\\	そのマラソン大会にはたくさんの国々から選手が集まったよ。	その マラソン たいかい に は たくさん の くにぐに から せんしゅ が あつまった よ 。 
\\	ここは子育てに最適な環境です。	ここ は こそだて に さいてき な かんきょう です 。 
\\	二人の実力は互角です。	に にん の じつりょく は ごかく です 。 
\\	彼は80年の生涯を閉じました。	かれ は 
\\	ねん の しょうがい を とじました 。 
\\	医師は異常なしと診断したんだ。	いし は いじょう なし と しんだん した ん だ 。 
\\	彼はいかにもスポーツマンらしいですね。	かれ は いかにも スポーツマン らしい です ね 。 
\\	彼は一見サラリーマン風ですね。	かれ は いっけん サラリーマン ふう です ね 。 
\\	彼は柔軟に対応をした。	かれ は じゅうなん に たいおう を した 。 
\\	活発な意見が交されたの。	かっぱつ な いけん が かわされた の 。 
\\	このソフトで画像を合成できます。	この ソフト で がぞう を ごうせい できます 。 
\\	彼は株で財産の半分を失ったの。	かれ は かぶ で ざいさん の はんぶん を うしなった の 。 
\\	ここは児童の通学路です。	ここ は じどう の つうがく ろ です 。 
\\	運動した翌日は全身の筋肉が痛い。	うんどう した よくじつ は ぜんしん の きんにく が いたい 。 
\\	彼はくじ引きで一等賞を当てたよ。	かれ は くじびき で いっとう しょう を あてた よ 。 
\\	後ろからいきなり肩をたたかれた。	うしろ から いきなり かた を たたかれた 。 
\\	水面に月が映っているね。	すいめん に つき が うつっている ね 。 
\\	改めてあなたのご意見を聞かせて下さい。	あらためて あなた の ご いけん を きかせてください 。 
\\	彼はやっと問題点に気付きました。	かれ は やっと もんだい てん に きづきました 。 
\\	私立の大学は学費が高い。	しりつ の だいがく は がくひ が たかい 。 
\\	最近、事件の真相が明らかになったよ。	さいきん 、 じけん の しんそう が あきらか に なった よ 。 
\\	背中がかゆい。	せなか が かゆい 。 
\\	彼は一流の選手です。	かれ は いちりゅう の せんしゅ です 。 
\\	皆が人質の無事を祈っているわよ。	みな が ひとじち の ぶじ を いのっている わ よ 。 
\\	勝手なことばかり言わないでくれ。	かって な こと ばかり いわない でくれ 。 
\\	彼女は体調を崩しています。	かのじょ は たいちょう を くずしています 。 
\\	メールを経由して広がるウィルスもあります。	メール を けいゆ して ひろがる ウィルス も あります 。 
\\	では、私たちは先に出発します。	では 、 わたし たち は さき に しゅっぱつ します 。 
\\	政治は国民主体であるべきです。	せいじ は こくみん しゅたい で ある べき です 。 
\\	この小説の主題は反戦です。	この しょうせつ の しゅだい は はんせん です 。 
\\	この通りには商店が多いね。	この とおり に は しょうてん が おおい ね 。 
\\	いつまでもあなたを忘れません。	いつ まで も あなた を わすれません 。 
\\	彼は拳銃で撃たれたわ。	かれ は けんじゅう で うたれた わ 。 
\\	この絵はスペインの画家が描きました。	この え は スペイン の がか が えがきました 。 
\\	父はラッシュアワーを嫌っています。	ちち は ラッシュアワー を きらっています 。 
\\	彼女はアレルギーに苦しんでいます。	かのじょ は アレルギー に くるしん でいます 。 
\\	犯人は国外に逃げたようです。	はんにん は こくがい に にげた よう です 。 
\\	彼は若いのにしっかりしてるね。	かれ は わかい の に しっかり してる ね 。 
\\	母の手料理をゆっくり味わいました。	はは の てりょうり を ゆっくり あじわいました 。 
\\	その古い建物は官庁です。	その ふるい たてもの は かんちょう です 。 
\\	期限までに申し込みました。	きげん まで に もうしこみました 。 
\\	水が沸騰して気体になったんだ。	みず が ふっとう して きたい に なった ん だ 。 
\\	この句はどんな意味でしょうか。	この く は どんな いみ でしょう か 。 
\\	この店では新鮮な野菜が買えますよ。	この みせ で は しんせん な やさい が かえます よ 。 
\\	前回の続きから始めます。	ぜんかい の つづき から はじめます 。 
\\	今日は沿岸の波が荒いでしょう。	きょう は えんがん の なみ が あらい でしょう 。 
\\	覚悟はできています。	かくご は できています 。 
\\	彼女は幸福な日々を過ごしているわ。	かのじょ は こうふく な ひび を すごしている わ 。 
\\	部長は京都に出張中です。	ぶちょう は きょうと に しゅっちょう ちゅう です 。 
\\	家族に感謝しています。	かぞく に かん しゃ しています 。 
\\	私は毎年夏に田舎に帰ります。	わたし は まいとし なつ に いなか に かえります 。 
\\	彼の講義はとても分かりやすいですね。	かれ の こうぎ は とても わかりやすい です ね 。 
\\	ゆっくり呼吸してください。	ゆっくり こきゅう して ください 。 
\\	その選手は自主トレーニングを始めたの。	その せんしゅ は じしゅ トレーニング を はじめた の 。 
\\	地面に何か絵が描いてあるぞ。	じめん に なに か え が えがいてある ぞ 。 
\\	7月上旬に夏祭りがありますよ。	7つき じょうじゅん に なつ まつり が あります よ 。 
\\	お二人はどこで知り合ったのですか。	お に にん は どこ で しりあった の です か 。 
\\	彼女は秘書と事務の担当を兼ねています。	かのじょ は ひしょ と じむ の たんとう を かねています 。 
\\	夜の11時に帰宅しました。	よる の 
\\	じ に きたく しました 。 
\\	物事には限度があります。	ものごと に は げんど が あります 。 
\\	彼女は高価な宝石を持っているわ。	かのじょ は こうか な ほうせき を もっている わ 。 
\\	彼女は通りがかりの人に救われたよ。	かのじょ は とおりがかり の ひと に すくわれた よ 。 
\\	祖父は戦場に行ったことがあるそうです。	そふ は せんじょう に いった こと が ある そう です 。 
\\	彼は相変わらず忙しいですね。	かれ は あいかわらず いそがしい です ね 。 
\\	街で偶然友人に会いました。	まち で ぐうぜん ゆうじん に あいました 。 
\\	子供が歩き始めたとき、子供の成長を実感した。	こども が あるきはじめた とき 、 こども の せいちょう を じっかん した 。 
\\	その家の正面には大きなバルコニーがあるの。	その いえ の しょうめん に は おおきな バルコニー が ある の 。 
\\	彼は人民のための政治を行った。	かれ は じんみん の ため の せいじ を おこなった 。 
\\	私は体力では誰にも劣りません。	わたし は たいりょく で は だれ に も おとりません 。 
\\	あの岩まで泳ごう。	あの いわ まで およごう 。 
\\	古代の歴史について勉強しました。	こだい の れきし について べんきょう しました 。 
\\	コストがこんなに高くては採算が取れません。	コスト が こんなに たかくて は さいさん が とれません 。 
\\	車内に忘れ物があったよ。	しゃない に わすれもの が あった よ 。 
\\	彼女は演劇を学んでいます。	かのじょ は えんげき を まなん でいます 。 
\\	嬉しくて思わず涙が出ました。	うれしくて おもわず なみだ が でました 。 
\\	新聞の活字が読みやすくなったね。	しんぶん の かつじ が よみやすく なった ね 。 
\\	古い校舎の修理が必要です。	ふるい こうしゃ の しゅうり が ひつよう です 。 
\\	彼は個人競技のスポーツが好きです。	かれ は こじん きょうぎ の スポーツ が すき です 。 
\\	私は私鉄の職員です。	わたし は してつ の しょくいん です 。 
\\	その住宅地は便利な場所にあるね。	その じゅうたく ち は べんり な ばしょ に ある ね 。 
\\	患者の呼吸は正常です。	かんじゃ の こきゅう は せいじょう です 。 
\\	お互いを信頼することが大事です。	おたがい を しんらい する こと が だいじ です 。 
\\	この道は、ゆるい曲線を描いているよ。	この みち は 、 ゆるい きょくせん を えがいている よ 。 
\\	この国には様々な人種がいます。	この く に には さまざま な じんしゅ が います 。 
\\	窓の外に水田が広がっていたよ。	まど の そと に すいでん が ひろがっていた よ 。 
\\	彼女は曖昧な返事をしたね。	かのじょ は あいまい な へんじ を した ね 。 
\\	運命には逆らえないよ。	うんめい に は さからえない よ 。 
\\	店に苦情の電話をかけました。	みせ に くじょう の でんわ を かけました 。 
\\	言論の自由は国民の権利です。	げんろん の じゆう は こくみん の けんり です 。 
\\	背広をクリーニングに出しました。	せびろ を クリーニング に だしました 。 
\\	雲の間から太陽が現れました。	くも の ま から たいよう が あらわれました 。 
\\	夏祭りで花火を打ち上げます。	なつ まつり で はなび を うちあげます 。 
\\	博士はその道の権威です。	はかせ は その みち の けんい です 。 
\\	今日は3科目のテストを受けたよ。	きょう は 
\\	かもく の テスト を うけた よ 。 
\\	彼は求人広告で仕事を見つけたんだ。	かれ は きゅうじん こうこく で しごと を みつけた ん だ 。 
\\	あなたの経歴をメールで送ってください。	あなた の けいれき を メール で おくって ください 。 
\\	彼女は女の子を出産したんだ。	かのじょ は おんなのこ を しゅっさん した ん だ 。 
\\	彼らは垂直のがけを登り始めたの。	かれら は すいちょく の がけ を のぼりはじめた の 。 
\\	彼はその時スターだったよ。	かれ は その とき スター だった よ 。 
\\	彼は野球部の一員です。	かれ は やきゅう ぶ の いちいん です 。 
\\	外は一面の雪景色だったね。	そと は いち めん の ゆきげしき だった ね 。 
\\	玉ねぎを細かく刻んでください。	たまねぎ を こまかく きざん で ください 。 
\\	電気系統を図面で確かめました。	でんき けいとう を ずめん で たしかめました 。 
\\	他人に知れるとまずいことになるわね。	たにん に しれる と まずい こと に なる わ ね 。 
\\	顧客心理を理解することは重要です。	こきゃく しんり を りかい する こと は じゅうよう です 。 
\\	父は情熱を持って仕事に打ち込んでいます。	ちち は じょうねつ を もって しごと に うちこん でいます 。 
\\	山は鮮やかな緑でした。	やま は あざやか な みどり でした 。 
\\	明日もいい天気になりそうだね。	あした も いい てんき に なり そう だ ね 。 
\\	将来はプロのカメラマンになりたいです。	しょうらい は プロ の カメラマン に なりたい です 。 
\\	彼は書道3級です。	かれ は しょどう 
\\	きゅう です 。 
\\	洪水でたくさんの家が流されたの。	こうずい で たくさん の いえ が ながされた の 。 
\\	何とか最悪の事態を避けることができました。	なんとか さいあく の じたい を さける こと が できました 。 
\\	彼は更に質問を続けたの。	かれ は さらに しつもん を つづけた の 。 
\\	デジカメ市場は急速に拡大している。	デジカメしじょう は きゅうそく に かくだい している 。 
\\	今日、彼女は地味な服装をしていますね。	きょう 、 かのじょ は じみ な ふくそう を しています ね 。 
\\	卒業後の進路を迷っています。	そつぎょう ご の しんろ を まよっています 。 
\\	彼はアルコールに強い体質です。	かれ は アルコール に つよい たいしつ です 。 
\\	運動会で一位になったよ。	うんどうかい で いち い に なった よ 。 
\\	衛星が打ち上げられましたね。	えいせい が うちあげられました ね 。 
\\	彼女のピアノの演奏は素晴らしいね。	かのじょ の ピアノ の えんそう は すばらしい ね 。 
\\	蝶々が空中を舞っています。	ちょうちょう が くうちゅう を まっています 。 
\\	彼は二度と刑務所から出ることができないの。	かれ は にどと けいむしょ から でる こと が できない の 。 
\\	今度こそタバコを止める決心をしました。	こんど こそ タバコ を とめる けっしん を しました 。 
\\	個々の問題を解決しましょう。	ここ の もんだい を かいけつ しましょう 。 
\\	ハイキングは自然に親しむ良い機会ですよ。	ハイキング は しぜん に したしむ よい きかい です よ 。 
\\	ご親切は決して忘れません。	ご しんせつ は けっして わすれません 。 
\\	豆腐は栄養のある食べ物です。	とうふ は えいよう の ある たべもの です 。 
\\	私たちは大声で歌を歌ったの。	わたし たち は おおごえ で うた を うたった の 。 
\\	今度の休暇にフィリピンに行きます。	こんど の きゅうか に フィリピン に いきます 。 
\\	私は彼の成功を確信しています。	わたし は かれ の せいこう を かく しん しています 。 
\\	この薬はあまりよく効かないよ。	この くすり は あまり よく きかない よ 。 
\\	今日の1時間目は国語です。	きょう の 
\\	じかん め は こくご です 。 
\\	氷は固体です。	こおり は こたい です 。 
\\	姉と私は服の好みが似ています。	あね と わたし は ふく の このみ が にています 。 
\\	この棚は水平になっていませんね。	この たな は すいへい に なっていません ね 。 
\\	私は西洋の建築に興味があります。	わたし は せいよう の けんちく に きょうみ が あります 。 
\\	家族で彼の合格を祝ったよ。	かぞく で かれ の ごうかく を いわった よ 。 
\\	刑事が現場を調べているわ。	けいじ が げんば を しらべている わ 。 
\\	彼女はよく自分の成績を自慢するよね。	かのじょ は よく じぶん の せいせき を じまん する よ ね 。 
\\	彼女は出版社で働いています。	かのじょ は しゅっぱん しゃ で はたらいています 。 
\\	消極的な人は成功しないよ。	しょうきょく てき な ひと は せいこう しない よ 。 
\\	この国はイギリスの植民地でした。	この くに は イギリス の しょくみん ち でした 。 
\\	彼女はずっと女優になるのが夢でした。	かのじょ は ずっと じょゆう に なる の が ゆめ でした 。 
\\	身長はどれくらいありますか。	しんちょう は どれ くらい あります か 。 
\\	電気や水を節約しましょう。	でんき や みず を せつやく しましょう 。 
\\	背中に痛みがあります。	せなか に いたみ が あります 。 
\\	運動会は9時に開会します。	うんどうかい は 
\\	じ に かいかい します 。 
\\	お勘定をして下さい。	お かんじょう を してください 。 
\\	彼らは漁業を営んでいるんだ。	かれら は ぎょぎょう を いとなん でいる ん だ 。 
\\	これは日本固有の鳥です。	これ は にっぽん こゆう の とり です 。 
\\	彼はカバンから札束を取り出したんだ。	かれ は カバン から さつたば を とりだした ん だ 。 
\\	家計の収支が合わないの。	かけい の しゅうし が あわない の 。 
\\	車の振動で棚の荷物が落ちた。	くるま の しんどう で たな の にもつ が おちた 。 
\\	自分の信念に従いたいと思います。	じぶん の しんねん に したがいたい と おもいます 。 
\\	せめてこれだけは約束してください。	せめて これ だけ は やくそく して ください 。 
\\	彼女は驚きを隠せませんでしたよ。	かのじょ は おどろき を かくせません でした よ 。 
\\	彼は私より一学年上です。	かれ は わたし より いち がくねん じょう です 。 
\\	彼女は家計を任されているの。	かのじょ は かけい を まかされている の 。 
\\	ガス管が爆発しました。	ガス かん が ばくはつ しました 。 
\\	テレビで明日の気象情報を確認したよ。	テレビ で あした の きしょう じょうほう を かくにん した よ 。 
\\	今日の議題は面白そうです。	きょう の ぎだい は おもしろ そう です 。 
\\	彼らは東京近郊に住んでいる。	かれら は とうきょう きんこう に すん でいる 。 
\\	彼女は結構めがねが似合うね。	かのじょ は けっこう めがね が にあう ね 。 
\\	賞金をもらったら公平に分けましょう。	しょうきん を もらったら こうへい に わけましょう 。 
\\	今日はデパートが混雑していたよ。	きょう は デパート が こんざつ していた よ 。 
\\	信仰は心の支えです。	しんこう は こころ の ささえ です 。 
\\	ここは隣の市との境界です。	ここ は となり の し と の きょうかい です 。 
\\	小麦は色々な食べ物に使われている。	こむぎ は いろいろ な たべもの に つかわれている 。 
\\	火が四方に広がったんだよ。	ひ が しほう に ひろがった ん だ よ 。 
\\	上司に相談してみます。	じょうし に そうだん してみます 。 
\\	食卓に花を飾りましょう。	しょくたく に はな を かざりましょう 。 
\\	会議の初日に市長がスピーチをしたよ。	かいぎ の しょ に ち にしちょう が スピーチ を した よ 。 
\\	アリは土の中に巣を作ります。	アリ は ど の なか に す を つくります 。 
\\	それは彼の推測にすぎない。	それ は かれ の すいそく に すぎない 。 
\\	父は居間でテレビを見ている。	ちち は いま で テレビ を みている 。 
\\	心の内をお話し下さい。	こころ の うち を おはなしください 。 
\\	彼女の訴えは認められたわ。	かのじょ の うったえ は みとめられた わ 。 
\\	大勢の方々にご出席いただきました。	たいせい の かたがた に ご しゅっせき いただきました 。 
\\	仮定の話だけでは結論は出ません。	かてい の はなし だけ で は けつろん は でません 。 
\\	彼女の優勝に感激し興奮しました。	かのじょ の ゆうしょう に かんげき し こうふん しました 。 
\\	では早速書類をお送りします。	では さっそく しょるい を おおくり します 。 
\\	彼の弱点はスタミナが足りないところです。	かれ の じゃくてん は スタミナ が たりない ところ です 。 
\\	彼の言うことは信用できない。	かれ の いう こと は しんよう できない 。 
\\	少年は網でその蝶を捕まえた。	しょうねん は あみ で その ちょう を つかまえた 。 
\\	母は毎日てきぱきと家事をしているよ。	は は はまいにち てきぱき と かじ を している よ 。 
\\	彼は大学で工学を勉強しました。	かれ は だいがく で こうがく を べんきょう しました 。 
\\	彼は論文の仕上げに入ったの。	かれ は ろんぶん の しあげ に はいった の 。 
\\	駅前に新しい書店ができました。	えきまえ に あたらしい しょてん が できました 。 
\\	もうすぐ夜が明けるね。	もうすぐ よる が あける ね 。 
\\	私は木造の家屋が好きです。	わたし は もくぞう の かおく が すき です 。 
\\	彼の態度は誠意を欠いています。	かれ の たいど は せいい を かいています 。 
\\	入学式で学長の挨拶がありました。	にゅうがく しき で がくちょう の あいさつ が ありました 。 
\\	花の成長を観察して日記を書きなさい。	はな の せいちょう を かんさつ して にっき を かきなさい 。 
\\	年をとると血管が硬くなります。	とし を とる と けっかん が かたく なります 。 
\\	彼はやっと報告書を仕上げたよ。	かれ は やっと ほうこく しょ を しあげた よ 。 
\\	市役所で書類をもらって来たの。	しやくしょ で しょるい を もらってきた の 。 
\\	上流には滝があります。	じょうりゅう に は たき が あります 。 
\\	ストレスがいろいろな病気の元になっているの。	ストレス が いろいろ な びょうき の もと に なっている の 。 
\\	その昔航海士達は未踏の地を求め海洋に乗り出しました。	その むかし こうかい し たち は みとう の ち を もとめ かいよう に のりだしました 。 
\\	コードが短くてコンセントに届きません。	コード が みじかくて コンセント に とどきません 。 
\\	警察がその男を捜しているの。	けいさつ が その おとこ を さがしている の 。 
\\	先日の件はどうなりましたか。	せんじつ の けん は どう なりました か 。 
\\	明日は恐らく晴れるでしょう。	あした は おそらく はれる でしょう 。 
\\	その職は大卒の学歴が必要だ。	その しょく は だいそつ の がくれき が ひつよう だ 。 
\\	そのことは承知しております。	その こと は しょうち しております 。 
\\	書類を10枚コピーしました。	しょるい を 
\\	まい コピー しました 。 
\\	雪の重みで枝が折れそうだ。	ゆき の おもみ で えだ が おれ そう だ 。 
\\	デパートは10時に開店しますよ。	デパート は 
\\	じ に かいてん します よ 。 
\\	この病院は24時間看護です。	この びょういん は 
\\	じかん かんご です 。 
\\	儀式には作法があります。	ぎしき に は さほう が あります 。 
\\	霧の深い夜のことでした。	きり の ふかい よる の こと でした 。 
\\	日本の国旗は描くのが簡単です。	にっぽん の こっき は えがく の が かんたん です 。 
\\	私は古典を読むのが好きです。	わたし は こてん を よむ の が すき です 。 
\\	学生たちは自主的に勉強会を開いたわよ。	がくせい たち は じしゅ てき に べんきょう かい を ひらいた わ よ 。 
\\	そちらの品は半額になっています。	そちら の しな は はんがく に なっています 。 
\\	泡立てたクリームに少量のブランデーを加えます。	あわだてた クリーム に しょうりょう の ブランデー を くわえます 。 
\\	あなたの成績が下がったのは当たり前です。	あなた の せいせき が さがった の は あたりまえ です 。 
\\	台風で山も海も荒れています。	たいふう で やま も うみ も あれています 。 
\\	会計を済ませて店を出たんだ。	かいけい を すませて みせ を でた ん だ 。 
\\	彼の我慢強さには感心しました。	かれ の がまんづよ さ に は かん しん しました 。 
\\	店の看板を塗り替えました。	みせ の かんばん を ぬりかえました 。 
\\	その瞬間、猫が跳び出しました。	その しゅんかん 、 ねこ が とびだしました 。 
\\	これを順番通りに並べ替えて下さい。	これ を じゅんばん どおり に ならべ かえてください 。 
\\	彼女は奨学金で大学に行きました。	かのじょ は しょうがく きん で だいがく に いきました 。 
\\	この曲を聞くと青春の頃を思い出します。	この きょく を きく と せいしゅん の ころ を おもいだします 。 
\\	部屋の明かりを点けましょう。	へや の あかり を つけましょう 。 
\\	その町は織物業で有名です。	その まち は おりもの ぎょう で ゆうめい です 。 
\\	彼は私と再会の約束を交わした。	かれ は わたし と さいかい の やくそく を かわした 。 
\\	このあたりは高層ビルが増えましたね。	この あたり は こうそう ビル が ふえました ね 。 
\\	彼は大学の後輩です。	かれ は だいがく の こうはい です 。 
\\	あの国とは国交がない。	あの くに と は こっこう が ない 。 
\\	彼は頻りに時計を気にしていた。	かれ は しきりに とけい を き に していた 。 
\\	新幹線の最高時速は300キロです。	しんかんせん の さいこう じそく は 
\\	キロ です 。 
\\	水は水素と酸素でできています。	みず は すいそ と さんそ で できています 。 
\\	体育館に行って整列しなさい。	たいいくかん に いって せいれつ しなさい 。 
\\	皆の顔と名前を一度には覚えられません。	みな の かお と なまえ を いち ど に は おぼえられません 。 
\\	ここでは喫煙できません。	ここ で は きつえん できません 。 
\\	子供の健全な心を育てましょう。	こども の けんぜん な こころ を そだてましょう 。 
\\	彼はモデルと交際していたんだ。	かれ は モデル と こうさい していた ん だ 。 
\\	彼女は合理的な考え方をする人です。	かのじょ は ごうり てき な かんがえかた を する ひと です 。 
\\	誕生日に小鳥を買ってもらいました。	たんじょう び に ことり を かってもらいました 。 
\\	彼女は結婚生活に失望していたの。	かのじょ は けっこん せいかつ に しつぼう していた の 。 
\\	街で知り合いを見かけたよ。	まち で しりあい を みかけた よ 。 
\\	娘が今年成人します。	むすめ が ことし せいじん します 。 
\\	結婚のお祝いに食器を頂きました。	けっこん の おいわい に しょっき を いただきました 。 
\\	お風呂に花を浮かべて入ったの。	お ふろ に はな を うかべて はいった の 。 
\\	何事も最初が肝心だ。	なにごと も さいしょ が かんじん だ 。 
\\	苦心して絵を描き上げました。	くしん して え を えがき あげました 。 
\\	彼は形式的な質問をしただけだった。	かれ は けいしき てき な しつもん を した だけ だった 。 
\\	入場券をお持ちですか。	にゅうじょう けん を お もち です か 。 
\\	上司から厳重に注意されました。	じょうし から げんじゅう に ちゅうい されました 。 
\\	このフィルムを現像してください。	この フィルム を げんぞう して ください 。 
\\	この道をまっすぐ進むと国道に出ます。	この みち を まっすぐ すすむ と こくどう に でます 。 
\\	夕方の市街は車が渋滞するよ。	ゆうがた の しがい は くるま が じゅうたい する よ 。 
\\	火事は無事消火されました。	かじ は ぶじ しょうか されました 。 
\\	少々のことは我慢します。	しょうしょう の こと は がまん します 。 
\\	衣類の整理をしたよ。	いるい の せいり を した よ 。 
\\	この先にキリスト教の教会があります。	この さき に キリストキョウ の きょうかい が あります 。 
\\	駅前に公立の図書館があるよ。	えきまえ に こうりつ の としょかん が ある よ 。 
\\	海外で語学の勉強をします。	かいがい で ごがく の べんきょう を します 。 
\\	毎朝7時5分に出勤します。	まいあさ 
\\	じ 
\\	ふん に しゅっきん します 。 
\\	彼は去年、家を新築しました。	かれ は きょねん 、 いえ を しんちく しました 。 
\\	私の住まいは東京にあります。	わたし の すまい は とうきょう に あります 。 
\\	私たちはみな彼の死を惜しんだの。	わたし たち は みな かれ の し を おしんだ の 。 
\\	彼の得意な学科は数学です。	かれ の とくい な がっか は すうがく です 。 
\\	新学期が始まったね。	しん がっき が はじまった ね 。 
\\	彼は最近筋肉を鍛えているよ。	かれ は さいきん きんにく を きたえている よ 。 
\\	父の表情が険しくなったの。	ちち の ひょうじょう が けわしく なった の 。 
\\	新しいコーチの指導は厳しかった。	あたらしい コーチ の しどう は きびしかった 。 
\\	その国は次第に資本主義になったわね。	その くに は しだいに しほん しゅぎ に なった わ ね 。 
\\	空き地で工事が始まった。	あきち で こうじ が はじまった 。 
\\	彼は大きな過ちを犯している。	かれ は おおきな あやまち を おかしている 。 
\\	屋上から富士山が見えました。	おくじょう から ふじさん が みえました 。 
\\	結婚すると新しい戸籍が作られます。	けっこん する と あたらしい こせき が つくられます 。 
\\	彼は握手をしようと手を差し出したの。	かれ は あくしゅ を しよう と て を さしだした の 。 
\\	彼はステージに立ったよ。	かれ は ステージ に たった よ 。 
\\	彼のアドバイスはいつも有り難いわね。	かれ の アドバイス は いつも ありがたい わ ね 。 
\\	これはイスラム教の寺院です。	これ は イスラムキョウ の じいん です 。 
\\	論文にその本を引用したの。	ろんぶん に その ほん を いんよう した の 。 
\\	うちの犬は気が荒い。	うち の いぬ は き が あらい 。 
\\	ウサギとカメの話を知っていますか。	ウサギ と カメ の はなし を しっています か 。 
\\	私は両親から多くを教わりました。	わたし は りょうしん から おおく を おそわりました 。 
\\	そのモダンな外観の建物が大使館です。	その モダン な がいかん の たてもの が たいしかん です 。 
\\	電車は5分間隔で来ますよ。	でんしゃ は 
\\	ふん かんかく で きます よ 。 
\\	船がやっと岸に着いたよ。	ふね が やっと きし に ついた よ 。 
\\	野球部はグラウンドで練習しています。	やきゅう ぶ は グラウンド で れん しゅう しています 。 
\\	親類の家に4年間下宿しました。	しんるい の いえ に 
\\	ねんかん げしゅく しました 。 
\\	彼は後方の座席に着いたの。	かれ は こうほう の ざせき に ついた の 。 
\\	暑さは峠を越したね。	あつ さ は とうげ を こした ね 。 
\\	「この会議は退屈だ」と同僚が私に囁いたの。	
\\	この かいぎ は たいくつ だ 
\\	と どうりょう が わたし に ささやいた の 。 
\\	古新聞をひもで縛ったよ。	こ しんぶん を ひも で しばった よ 。 
\\	彼女は会計主任です。	かのじょ は かいけい しゅにん です 。 
\\	雲一つない青空ですね。	くも ひとつ ない あおぞら です ね 。 
\\	その頃彼はまだ赤ん坊だったよね。	その ころ かれ は まだ あかんぼう だった よ ね 。 
\\	彼は仲間を裏切ったの。	かれ は なかま を うらぎった の 。 
\\	彼女は英文の手紙を書きました。	かのじょ は えいぶん の てがみ を かきました 。 
\\	下流に小さな滝があるよ。	かりゅう に ちいさな たき が ある よ 。 
\\	彼はレントゲン技師です。	かれ は レントゲン ぎし です 。 
\\	彼女はとても教養のある人ですね。	かのじょ は とても きょうよう の ある ひと です ね 。 
\\	金銭のトラブルには関わりたくありません。	きんせん の トラブル に は かかわりたく ありません 。 
\\	彼女は毛皮のコートを着ていたの。	かのじょ は けがわ の コート を きていた の 。 
\\	そのカメラには最先端の技術が使われています。	その カメラ に は さいせんたん の ぎじゅつ が つかわれています 。 
\\	彼女の言っていることがさっぱり分からないの。	かのじょ の いっている こと が さっぱり わからない の 。 
\\	彼は織田信長の子孫だよ。	かれ は おだ のぶなが の しそん だ よ 。 
\\	丸いボタンを押すとシャッターが閉まります。	まるい ボタン を おす と シャッター が しまります 。 
\\	私は彼の人格を尊敬しています。	わたし は かれ の じんかく を そんけい しています 。 
\\	ほこりを吸い込んじゃった。	ほこり を すいこんじゃった 。 
\\	勧められて欲しくもないものを買っちゃったよ。	すすめられてほしく も ない もの を かっちゃった よ 。 
\\	このスピーカーは音がいいですね。	この スピーカー は おと が いい です ね 。 
\\	子供はたくさんの愛情が必要です。	こども は たくさん の あいじょう が ひつよう です 。 
\\	雪の上にうさぎの足跡があった。	ゆき の うえ に うさぎ の あしあと が あった 。 
\\	台風で大雨になったの。	たいふう で おおあめ に なった の 。 
\\	この川の河口は太平洋に注いでいます。	この かわ の かこう は たいへいよう に そそい でいます 。 
\\	彼は元来、真面目な人です。	かれ は がんらい 、 まじめ な ひと です 。 
\\	いつでも気軽に遊びに来て下さい。	いつ でも きがる に あそび に きてください 。 
\\	フロアはパーティションで区切られています。	フロア は パーティション で くぎられています 。 
\\	色の組み合わせで印象が変わりますよ。	いろ の くみあわせ で いんしょう が かわります よ 。 
\\	芝生がよく手入れされていますね。	しばふ が よく ていれ されています ね 。 
\\	申し込みの締め切りはいつですか。	もうしこみ の しめきり は い つ です か 。 
\\	彼は出世するタイプだな。	かれ は しゅっせ する タイプ だ な 。 
\\	以前、セールスマンをしていたことがあります。	いぜん 、 セールスマン を していた こと が あります 。 
\\	車掌が発車の合図をしたよ。	しゃしょう が はっしゃ の あいず を した よ 。 
\\	父の教えは「自分に厳しく」です。	ちち の おしえ は 
\\	じぶん に きびしく 
\\	です 。 
\\	彼女たちはおしゃべりに夢中です。	かのじょ たち は おしゃべり に むちゅう です 。 
\\	会社帰りに仲間とカラオケに行きました。	かいしゃ がえり に なかま と カラオケ に いきました 。 
\\	これはスポーツ施設用の器具です。	これ は スポーツ しせつ よう の きぐ です 。 
\\	彼女はこっそりダイエットを始めたの。	かのじょ は こっそり ダイエット を はじめた の 。 
\\	私は毎日ジョギングをしています。	わたし は まいにち ジョギング を しています 。 
\\	服を買うのにあれこれ迷いました。	ふく を かう のに あれこれ まよいました 。 
\\	彼女は自分の姿を鏡に映したの。	かのじょ は じぶん の すがた を かがみ に うつした の 。 
\\	家から駅までおよそ1キロあります。	いえ から えき まで およそ 
\\	キロ あります 。 
\\	高原には気持ちのよい風が吹いていたよ。	こうげん に は きもち の よい かぜ が ふいていた よ 。 
\\	おじいちゃんにお小遣いをもらったよ。	おじいちゃん に お こづかい を もらった よ 。 
\\	どのサークルに入るか迷っちゃった。	どの サークル に はいる か まよっちゃった 。 
\\	この春に彼は社会人になったよ。	この はる に かれ は しゃかい じん に なった よ 。 
\\	彼の趣味は切手収集です。	かれ の しゅみ は きって しゅうしゅう です 。 
\\	毛皮を税関で没収されたんだ。	けがわ を ぜいかん で ぼっしゅう された ん だ 。 
\\	レストランは清潔が第一です。	レストラン は せいけつ が だい いち です 。 
\\	その海峡に橋が掛けられました。	その かいきょう に はし が かけられました 。 
\\	栄養が偏らないように食事に気を付けています。	えいよう が かたよらない よう に しょくじ に き を つけています 。 
\\	その本は貸し出ししていません。	その ほん は かしだし していません 。 
\\	汽車で街まで行った。	きしゃ で まち まで いった 。 
\\	キリストの母親の名前はマリアだ。	キリスト の ははおや の なまえ は マリア だ 。 
\\	寒くて唇が青くなってしまった。	さむくて くちびる が あおく なってしまった 。 
\\	月給は毎月25日に支給されます。	げっきゅう は まいつき 
\\	に ち にしきゅう されます 。 
\\	この曲は誰が作曲したのですか。	この きょく は だれ が さっきょく した の です か 。 
\\	昨日は遅くまで残業しました。	きのう は おそく まで ざんぎょう しました 。 
\\	そうあっさり言わないで。	そう あっさり いわない で 。 
\\	午後に打ち合わせをしましょう。	ごご に うちあわせ を しましょう 。 
\\	年と共に体力が衰えています。	とし とともに たいりょく が おとろえています 。 
\\	彼から思いがけないことを聞いた。	かれ から おもいがけない こと を きいた 。 
\\	この鐘の音はとてもきれいですね。	この かね の おと は とても きれい です ね 。 
\\	私は人前で話すことが苦痛です。	わたし は ひとまえ で はなす こと が くつう です 。 
\\	日が暮れる前に帰りましょう。	ひ が くれる まえ に かえりましょう 。 
\\	昔は結核でたくさんの人が亡くなった。	むかし は けっかく で たくさん の ひと が なくなった 。 
\\	ここはかつてゴールドラッシュで栄えた町だよ。	ここ は かつて ゴールド ラッシュ で さかえた まち だ よ 。 
\\	連絡があり次第出発します。	れんらく が あり しだい しゅっぱつ します 。 
\\	昼ごはんは簡単に済ませましょう。	ひる ごはん は かんたん に すませましょう 。 
\\	彼は朝から晩までせっせと働いたの。	かれ は あさ から ばん まで せっせと はたらいた の 。 
\\	今日は慌ただしい一日でした。	きょう は あわただしい いち にち でした 。 
\\	外はもう薄暗くなりましたよ。	そと は もう うすぐらく なりました よ 。 
\\	カタログを見て注文したんだ。	カタログ を みて ちゅうもん した ん だ 。 
\\	彼は自分の状況を客観的に見てみたのね。	かれ は じぶん の じょうきょう を きゃっかん てき に みてみた の ね 。 
\\	彼は気力にあふれていますね。	かれ は きりょく に あふれています ね 。 
\\	この郡は来年、市になりますよ。	この ぐん は らいねん 、 し に なります よ 。 
\\	あの子は下町育ちだ。	あの こ は したまち そだち だ 。 
\\	昨日の夜、ジャズを聴きに行きました。	きのう の よる 、 ジャズ を きき に いきました 。 
\\	私は心身共に疲れていました。	わたし は しんしん とも に つかれていました 。 
\\	氷は水に浮きます。	こおり は みず に うきます 。 
\\	運動場でサッカーをしよう。	うんどう じょう で サッカー を しよう 。 
\\	家具の配置を変えました。	かぐ の はいち を かえました 。 
\\	日本映画で刀を見た。	にっぽん えいが で かたな を みた 。 
\\	彼は工学部の教授です。	かれ は こうがくぶ の きょうじゅ です 。 
\\	ここにサインしてください。	ここ に サイン して ください 。 
\\	このトラックの車輪は頑丈そうですね。	この トラック の しゃりん は がんじょう そう です ね 。 
\\	食器を全部新しくしました。	しょっき を ぜんぶ あたら しく しました 。 
\\	娘は市立の学校に通っています。	むすめ は しりつ の がっこう に かよっています 。 
\\	その絵は一見に値しますよ。	その え は いっけん に あたいします よ 。 
\\	彼は偉大な事業をなしとげました。	かれ は いだい な じぎょう を なしとげました 。 
\\	彼は試合の途中で棄権したぞ。	かれ は しあい の とちゅう で きけん した ぞ 。 
\\	友人が骨折して外科に入院しました。	ゆうじん が こっせつ して げか に にゅういん しました 。 
\\	僕たちは頑張って決勝まで進んだよ。	ぼく たち は がんばって けっしょう まで すすんだ よ 。 
\\	彼は公衆電話を探したの。	かれ は こうしゅう でんわ を さがした の 。 
\\	彼は質問の答えをごまかしたね。	かれ は しつもん の こたえ を ごまかした ね 。 
\\	要らないファイルは削除して下さい。	いらない ファイル は さくじょ してください 。 
\\	彼女はインテリで、しかも美人ですね。	かのじょ は インテリ で 、 しかも びじん です ね 。 
\\	彼は家から追い出された。	かれ は いえ から おいだされた 。 
\\	丘の上にホテルが建ちましたね。	おか の うえ に ホテル が たちました ね 。 
\\	彼は騙されたと思い込んでいるようです。	かれ は だまされた と おもいこん でいる よう です 。 
\\	浜辺できれいな貝を拾いました。	はまべ で きれい な かい を ひろいました 。 
\\	彼は空想にふけっているの。	かれ は くうそう に ふけっている の 。 
\\	あそこの3区画は売り出し中です。	あそこ の 
\\	くかく は うりだし ちゅう です 。 
\\	この坂を上るのはすごくきついね。	この さか を のぼる の は すごく きつい ね 。 
\\	応募受付は締め切りました。	おうぼ うけつけ は しめきりました 。 
\\	彼女はその映画に出演しているよ。	かのじょ は その えいが に しゅつえん している よ 。 
\\	彼は積極的に攻めたが勝てなかったな。	かれ は せっきょく てき に せめた が かてなかった な 。 
\\	彼はいちいち私に指図する。	かれ は いちいち わたし に さしず する 。 
\\	僕が知っているのは一部分に過ぎない。	ぼく が しっている の は いち ぶぶん に すぎない 。 
\\	その日、国民は平和への祈りを捧げるの。	その ひ 、 こくみん は へいわ へ の いのり を ささげる の 。 
\\	彼女は長年の恨みを晴らした。	かのじょ は ながねん の うらみ を はらした 。 
\\	彼は国の英雄ですね。	かれ は くに の えいゆう です ね 。 
\\	その子は絵本が大好きです。	その こ は えほん が だいすき です 。 
\\	彼女の思いやりが嬉しかった。	かのじょ の おもいやり が うれしかった 。 
\\	顔色がよくありませんね。	かおいろ が よく ありません ね 。 
\\	今日大学に願書を送りました。	きょう だいがく に がんしょ を おくりました 。 
\\	幸運にもチケットを手に入れました。	こううん に も チケット を て に いれました 。 
\\	先生は試験の採点が終わったようね。	せんせい は しけん の さいてん が おわった よう ね 。 
\\	明日はお座敷での宴会になります。	あした は おざしき で の えんかい に なります 。 
\\	彼は司会が上手ですね。	かれ は しかい が じょうず です ね 。 
\\	この新聞の社説は面白いね。	この しんぶん の しゃせつ は おもしろい ね 。 
\\	今日、病院で診察してもらったの。	きょう 、 びょういん で しんさつ してもらった の 。 
\\	従業員は明日からストライキです。	じゅうぎょう いん は あした から ストライキ です 。 
\\	妹はマフラーを編みました。	いもうと は マフラー を あみました 。 
\\	これはあんまり好きじゃないな。	これ は あんまり すき じゃない な 。 
\\	彼の会社は衣服を扱っているよ。	かれ の かいしゃ は いふく を あつかっている よ 。 
\\	外来語は一般にカタカナで書かれます。	がいらい ご は いっぱん に カタカナ で かかれます 。 
\\	彼はアルバイトをして学費を稼いだ。	かれ は アルバイト を して がくひ を かせいだ 。 
\\	こちらにお名前をご記入ください。	こちら に お なまえ を ご きにゅう ください 。 
\\	彼女は強烈な個性の持ち主ですよ。	かのじょ は きょうれつ な こせい の もちぬし です よ 。 
\\	レモンの爽やかな香りがした。	レモン の さわやか な かおり が した 。 
\\	実例を使って説明してください。	じつれい を つかって せつめい して ください 。 
\\	猫が障子を破った。	ねこ が しょうじ を やぶった 。 
\\	彼女は育児で忙しいよ。	かのじょ は いくじ で いそがしい よ 。 
\\	親友に悩みを打ち明けたの。	しんゆう に なやみ を うちあけた の 。 
\\	彼の才能は羨ましい程だね。	かれ の さいのう は うらやましい ほど だ ね 。 
\\	彼は運のいい男です。	かれ は うん の いい おとこ です 。 
\\	弟は落ち着きが足りません。	おとうと は おちつき が たりません 。 
\\	会費が少し高いね。	かいひ が すこし たかい ね 。 
\\	天気予報は当たる確率が高くなりましたね。	てんき よほう は あたる かくりつ が たかく なりました ね 。 
\\	私たちは校歌を合唱したの。	わたし たち は こうか を がっ しょう した の 。 
\\	来週、帰省します。	らいしゅう 、 きせい します 。 
\\	明日は臨時に休業します。	あした は りんじ に きゅうぎょう します 。 
\\	「大きい」は形容詞です。	
\\	おおきい 
\\	は けいようし です 。 
\\	彼は彼女に好意をもっています。	かれ は かのじょ に こうい を もっています 。 
\\	この山は紅葉がとても美しい。	この やま は こうよう が とても うつくしい 。 
\\	消防士は勇敢でなければならない。	しょうぼう し は ゆうかん でなけれ ばならない 。 
\\	今日は食欲がありません。	きょう は しょくよく が ありません 。 
\\	身なりのいい紳士が話しかけてきたの。	みなり の いい しんし が はなしかけてきた の 。 
\\	真実は一つです。	しんじつ は ひとつ です 。 
\\	娘の存在が私の救いでした。	むすめ の そんざい が わたし の すくい でした 。 
\\	朝早く砂浜を散歩しました。	あさ はやく すなはま を さんぽ しました 。 
\\	彼は地位を悪用しています。	かれ は ちい を あくよう しています 。 
\\	彼の趣味は囲碁です。	かれ の しゅみ は いご です 。 
\\	あまりうるさく言わないで下さい。	あまり うるさく いわない でください 。 
\\	あなたを驚かす話があります。	あなた を おどろかす はなし が あります 。 
\\	親指を怪我しました。	おやゆび を けが しました 。 
\\	このカーブは気をつけて。	この カーブ は き を つけて 。 
\\	ガイドつきの旅行を申し込んだの。	ガイド つき の りょこう を もうしこんだ の 。 
\\	田んぼの中でカエルが合唱している。	たんぼ の なか で カエル が がっ しょう している 。 
\\	ごみは各自で持ち帰ってください。	ごみ は かくじ で もちかえって ください 。 
\\	彼は冷静で賢い男だね。	かれ は れいせい で かしこい おとこ だ ね 。 
\\	子供に昔話を聞かしてあげたの。	こども に むかしばなし を きかしてあげた の 。 
\\	彼女は気が利いている。	かのじょ は き が きいている 。 
\\	代金を期日までにお支払いください。	だいきん を きじつ まで に お しはらい ください 。 
\\	その国は共産主義の国だよ。	その くに は きょうさん しゅぎ の くに だ よ 。 
\\	強い日差しが選手たちを苦しめたの。	つよい ひざし が せんしゅ たち を くるしめた の 。 
\\	彼は県立の高校に通っているよ。	かれ は けんりつ の こうこう に かよっている よ 。 
\\	彼女はスキーで足を骨折したの。	かのじょ は スキー で あし を こっせつ した の 。 
\\	猫は砂の上で転がったんだ。	ねこ は すな の うえ で ころがった ん だ 。 
\\	姉は大学に在学しています。	あね は だいがく に ざいがく しています 。 
\\	日本の夏は湿度が高いです。	にっぽん の なつ は しつど が たかい です 。 
\\	何かの説明をする時は実物を使うとわかりやすいの。	なに か の せつめい を する とき は じつぶつ を つかう と わかりやすい の 。 
\\	彼は上級のコースに上がったよ。	かれ は じょうきゅう の コース に あがった よ 。 
\\	地図内の小さい建物は省略してあります。	ちず ない の ちいさい たてもの は しょうりゃく してあります 。 
\\	寝室の壁紙を張り替えました。	しんしつ の かべがみ を はりかえました 。 
\\	水害でたくさんの人が家を失ったの。	すいがい で たくさん の ひと が いえ を うしなった の 。 
\\	その子はおもちゃに飽きたようだね。	その こ は おもちゃ に あきた よう だ ね 。 
\\	虫歯がずきずき痛みます。	むしば が ずきずき いたみます 。 
\\	今ごろそんなこと言わないで。	いま ごろ そんな こと いわない で 。 
\\	衛生には十分気を付けましょう。	えいせい に は じゅうぶん き を つけましょう 。 
\\	彼はお嬢さんと一緒でした。	かれ は おじょうさん と いっしょ でした 。 
\\	それは間接的な原因の一つです。	それ は かんせつ てき な げんいん の ひとつ です 。 
\\	先生が手紙を下さいました。	せんせい が てがみ を くださいました 。 
\\	彼女は恋をしてきれいになったね。	かのじょ は こい を して きれい に なった ね 。 
\\	彼には誰もが親しみを感じます。	かれ に は だれ も が したしみ を かんじます 。 
\\	ポスターの位置が少しずれているよ。	ポスター の いち が すこし ずれている よ 。 
\\	赤道に沿って旅をしました。	せきどう に そって たび を しました 。 
\\	彼は案外いい人かも知れない。	かれ は あんがい いい じん かも しれない 。 
\\	あなたのおっしゃる通りです。	あなた の おっしゃる とおり です 。 
\\	彼女はクラシック音楽が好きです。	かのじょ は クラシック おん が く がすき です 。 
\\	その道は今、下水の工事をしているよ。	その みち は いま 、 げすい の こうじ を している よ 。 
\\	作文を書くのに原稿用紙を使ったの。	さくぶん を かく の に げんこう ようし を つかった の 。 
\\	彼は国費で留学しています。	かれ は こくひ で りゅうがく しています 。 
\\	テレビは彼のいちばんの娯楽です。	テレビ は かれ の いちばん の ごらく です 。 
\\	船から材木が降ろされていますね。	ふね から ざいもく が おろされています ね 。 
\\	こぼれた牛乳をさっと拭き取ったの。	こぼれた ぎゅうにゅう を さっと ふきとった の 。 
\\	ここは市民の運動場です。	ここ は しみん の うんどう じょう です 。 
\\	次回の会議は2週間後に行います。	じかい の かいぎ は 
\\	しゅうかん ご に おこないます 。 
\\	咳が止まらないので病院に行ってきたの。	せき が とまらない ので びょういん に いってきた の 。 
\\	廊下から大きな足音が聞こえた。	ろうか から おおきな あしおと が きこえた 。 
\\	年号が改まりました。	ねんごう が あらたまりました 。 
\\	赤に黄色を合わすと何色になりますか。	あか に きいろ を あわす と なん しょく に なります か 。 
\\	お客様にお菓子を頂いた。	おきゃくさま に おかし を いただいた 。 
\\	尾の長い鳥が飛んでいますね。	お の ながい とり が とん でいます ね 。 
\\	彼の話は大げさだ。	かれ の はなし は おおげさ だ 。 
\\	夏季は電力が不足がちね。	かき は でんりょく が ふそく がち ね 。 
\\	彼は経済学部の学生です。	かれ は けいざいがくぶ の がくせい です 。 
\\	子供たちは広場に駆けて行ったわよ。	こども たち は ひろば に かけていった わ よ 。 
\\	なんて頑固な子だ。	なんて がんこ な こ だ 。 
\\	このシャツは絹でできています。	この シャツ は きぬ で できています 。 
\\	チューリップは茎が長いです。	チューリップ は くき が ながい です 。 
\\	彼らは軽快なステップで踊り出したの。	かれら は けいかい な ステップ で おどりだした の 。 
\\	地震の後、街は地獄のようだったよ。	じしん の のち 、 まち は じごく の よう だった よ 。 
\\	この柿は渋いね。	この かき は しぶい ね 。 
\\	短い時間にずいぶん上達しましたね。	みじかい じかん に ずいぶん じょうたつ しました ね 。 
\\	足の傷を消毒したわ。	あし の きず を しょうどく した わ 。 
\\	このセットで勝敗が決まります。	この セット で しょうはい が きまります 。 
\\	祖父の家に親戚が集まったんだ。	そふ の いえ に しんせき が あつまった ん だ 。 
\\	今日は頭痛がします。	きょう は ずつう が します 。 
\\	彼の息子が家出したそうよ。	かれ の むすこ が いえで した そう よ 。 
\\	ここの大家は近くに住んでいますよ。	ここ の おおや は ちかく に すん でいます よ 。 
\\	歌合戦は赤組の勝ちでしたね。	うたがっせん は あか ぐみ の かち でした ね 。 
\\	予約をキャンセルしたよ。	よやく を キャンセル した よ 。 
\\	香水のいい香りがした。	こうすい の いい かおり が した 。 
\\	メイドがシーツを交換してくれたね。	メイド が シーツ を こうかん してくれた ね 。 
\\	車を車庫に入れておきました。	くるま を しゃこ に いれておきました 。 
\\	予算案が衆議院を通過したな。	よさん あん が しゅうぎいん を つうか した な 。 
\\	受話器を取ってもらえますか。	じゅわき を とってもらえます か 。 
\\	今年は去年よりも順位が上がりました。	ことし は きょねん より も じゅんい が あがりました 。 
\\	服に墨がついちゃった。	ふく に すみ が ついちゃった 。 
\\	彼女の親切に甘えました。	かのじょ の しんせつ に あまえました 。 
\\	彼女は生け花の先生です。	かのじょ は いけばな の せんせい です 。 
\\	彼女は会場整理の係だったの。	かのじょ は かいじょう せいり の かかり だった の 。 
\\	彼女はインフルエンザにかかっていますね。	かのじょ は インフルエンザ に かかっています ね 。 
\\	学力を付けてその大学に進みたい。	がくりょく を つけて その だいがく に すすみたい 。 
\\	博物館で昔の貨幣を見ました。	はくぶつかん で むかし の かへい を みました 。 
\\	神様にお願いしました。	かみさま に おねがい しました 。 
\\	スタジアムは観衆で満員だったよ。	スタジアム は かんしゅう で まんいん だった よ 。 
\\	北風が冷たいです。	きたかぜ が つめたい です 。 
\\	彼女は今、休養中です。	かのじょ は いま 、 きゅうよう ちゅう です 。 
\\	見物人が大勢集まっているね。	けんぶつにん が たいせい あつまっている ね 。 
\\	彼は腕のいいコックです。	かれ は うで の いい コック です 。 
\\	その先生はいつも授業の前に雑談をするの。	その せんせい は いつも じゅぎょう の まえ に ざつだん を する の 。 
\\	何だか寒気がします。	なんだか さむけ が します 。 
\\	うちでは夫も炊事をします。	うち で は おっと も すいじ を します 。 
\\	1日7時間は睡眠を取るようにしています。	1にち 
\\	じかん は すいみん を とる よう に しています 。 
\\	明け方に雨が降り始めましたね。	あけがた に あめ が ふりはじめました ね 。 
\\	彼は悪酔いして暴れたんだ。	かれ は わるよい して あばれた ん だ 。 
\\	学校での苛めはなかなか減らない。	がっこう で の いじめ は なかなか へらない 。 
\\	お芋の料理はお好きですか。	お いも の りょうり は お すき です か 。 
\\	被災地では飲料水が不足しているの。	ひさい ち で は いんりょう すい が ふそく している の 。 
\\	えびの天ぷらは私の好物です。	えび の てんぷら は わたし の こうぶつ です 。 
\\	理科の時間に石の硬さを比べたよ。	りか の じかん に いし の かた さ を くらべた よ 。 
\\	彼は球技が大好きです。	かれ は きゅうぎ が だいすき です 。 
\\	そのグリーンのスカーフ、素敵ですね。	その グリーン の スカーフ 、 すてき です ね 。 
\\	彼女は快い眠りについています。	かのじょ は こころよい ねむり に ついています 。 
\\	このホテルでは最上のサービスが受けられます。	この ホテル で は さいじょう の サービス が うけられます 。 
\\	彼はいつもジーパンをはいているね。	かれ は いつも ジーパン を はいている ね 。 
\\	支度ができたら出かけましょう。	したく が できたら でかけましょう 。 
\\	彼はジャーナリズムを専攻している。	かれ は ジャーナリズム を せんこう している 。 
\\	男の子が3人もいるので食費がかさみます。	おとこのこ が 
\\	にん も いる ので しょくひ が かさみます 。 
\\	妹夫婦は新婚です。	いもうと ふうふ は しんこん です 。 
\\	彼のスピーチは素晴らしかった。	かれ の スピーチ は すばらしかった 。 
\\	出席者を性別で分けてください。	しゅっせき しゃ を せいべつ で わけて ください 。 
\\	ああうるさい人は苦手です。	ああ うるさい ひと は にがて です 。 
\\	アメリカ英語とイギリス英語ではアクセントが違うことがあるね。	アメリカ えいご と イギリス えいご で は アクセント が ちがう こと が ある ね 。 
\\	同じ過ちを繰り返さないことだ。	おなじ あやまち を くりかえさない こと だ 。 
\\	ここに受け取りの印鑑をお願いします。	ここ に うけとり の いんかん を おねがい します 。 
\\	それは最新の学説ね。	それ は さいしん の がくせつ ね 。 
\\	この道は片側通行です。	この みち は かたがわ つうこう です 。 
\\	彼は大きな荷物を担いで来たの。	かれ は おおきな にもつ を かつい できた の 。 
\\	彼は犬の柄の
\\	シャツを着ているよ。	かれ は いぬ の え の 
\\	シャツ を きている よ 。 
\\	彼女は一晩中彼を看病したの。	かのじょ は いち ばん ちゅう かれ を かんびょう した の 。 
\\	将来は気楽な生活がしたいです。	しょうらい は きらく な せいかつ が したい です 。 
\\	父は長い勤労生活を送ったんです。	ちち は ながい きんろう せいかつ を おくった ん です 。 
\\	区役所に書類を届けてください。	くやくしょ に しょるい を とどけて ください 。 
\\	彼女はテレビのコマーシャルに出ているわ。	かのじょ は テレビ の コマーシャル に でている わ 。 
\\	靴の底に釘が刺さってしまったんだ。	くつ の そこ に くぎ が ささってしまった ん だ 。 
\\	祖父は市外の病院に通っているの。	そふ は しがい の びょういん に かよっている の 。 
\\	しつこい迷惑メールに困っている。	しつこい めいわく メール に こまっている 。 
\\	今朝は庭の草に霜が降りていたの。	けさ は にわ の くさ に しも が おりていた の 。 
\\	お年寄りが道端にしゃがんでいますね。	お としより が みちばた に しゃがん でいます ね 。 
\\	アジアの主食は米です。	アジア の しゅしょく は こめ です 。 
\\	書道をすると姿勢も良くなります。	しょどう を する と しせい も よく なります 。 
\\	その男の行動は怪しかったわ。	その おとこ の こうどう は あやしかった わ 。 
\\	そんなに慌ててどこに行くの。	そんなに あわて てどこ に いく の 。 
\\	一同顔を見合わせました。	いちどう がお を みあわせました 。 
\\	校舎から歌声が聞こえて来たよ。	こうしゃ から うたごえ が きこえてきた よ 。 
\\	落ち葉の季節になりましたね。	おちば の きせつ に なりました ね 。 
\\	この帯は長過ぎます。	この おび は ながすぎます 。 
\\	思い切って彼に相談します。	おもいきって かれ に そうだん します 。 
\\	女の勘を甘く見てはいけません。	おんな の かん を あまく みて はいけません 。 
\\	大会のスローガンを掲示したよ。	たいかい の スローガン を けいじ した よ 。 
\\	彼女は毛糸のセーターを編みました。	かのじょ は けいと の セーター を あみました 。 
\\	このコンテストに優勝すると車がもらえるんだ。	この コンテスト に ゆうしょう する と くるま が もらえる ん だ 。 
\\	二人は婚約しています。	に にん は こんやく しています 。 
\\	彼は最強チームの一員です。	かれ は さいきょう チーム の いちいん です 。 
\\	合鍵の作製には2日ほどかかります。	あいかぎ の さくせい に は 
\\	にち ほど かかります 。 
\\	最近、産婦人科の不足が問題になっているね。	さいきん 、 さんふじんか の ふそく が もんだい に なっている ね 。 
\\	知らない言葉を辞典で調べました。	しらない ことば を じてん で しらべました 。 
\\	車掌さんが車内を回って来たよ。	しゃしょう さん が しゃない を まわってきた よ 。 
\\	お邪魔して済みません。	お じゃま して すみません 。 
\\	これは日本の人名ですか。	これ は にっぽん の じんめい です か 。 
\\	ここの水族館にはイルカがいます。	ここ の すいぞくかん に は イルカ が います 。 
\\	筋道を立てて考えなさい。	すじみち を たてて かんがえなさい 。 
\\	素敵なプレゼントをありがとう。	すてき な プレゼント を ありがとう 。 
\\	宿題を済ましてから遊びなさい。	しゅくだい を すまして から あそびなさい 。 
\\	生年月日を教えて。	せいねんがっぴ を おしえて 。 
\\	彼は易しい言葉に言い換えたんだ。	かれ は やさしい ことば に いいかえた ん だ 。 
\\	いい加減なことを言ってはいけません。	いいかげん な こと を いって はいけません 。 
\\	疑わしい場所は全部調べよう。	うたがわしい ばしょ は ぜんぶ しらべよう 。 
\\	この器は上等ね。	この うつわ は じょうとう ね 。 
\\	その国の王様はとても賢い。	その くに の おうさま は とても かしこい 。 
\\	弟は新しい自転車に大喜びです。	おとうと は あたらしい じてんしゃ に だい よろこび です 。 
\\	彼女は郊外の学園に通っているね。	かのじょ は こうがい の がくえん に かよっている ね 。 
\\	彼は過労のために倒れたの。	かれ は かろう の ため に たおれた の 。 
\\	錠を頑丈なものに替えました。	じょう を がんじょう な もの に かえました 。 
\\	決まりを守ることは大切です。	きまり を まもる こと は たいせつ です 。 
\\	キャベツの千切りを添えたよ。	キャベツ の せんぎり を そえた よ 。 
\\	草花を大切にしましょう。	くさばな を たいせつ に しましょう 。 
\\	男女交互に並んでください。	だんじょ こうご に ならん で ください 。 
\\	全員、講堂に集まってください。	ぜんいん 、 こうどう に あつまって ください 。 
\\	うどんは小麦粉から作られます。	うどん は こむぎこ から つくられます 。 
\\	日本の制服は紺が多いね。	にっぽん の せいふく は こん が おおい ね 。 
\\	彼は歯科医師です。	かれ は しか いし です 。 
\\	日本とフランスの時差は8時間です。	にっぽん と フランス の じさ は 
\\	じかん です 。 
\\	牛乳は品切れだったよ。	ぎゅうにゅう は しなぎれ だった よ 。 
\\	先月、そのコースを修了しました。	せんげつ 、 その コース を しゅうりょう しました 。 
\\	パーティーのビンゴの賞品は何がいいだろう。	パーティー の ビンゴ の しょうひん は なに が いい だろう 。 
\\	間違いに印を付けておきました。	まちがい に しるし を つけておきました 。 
\\	これからも両国の親善を深めましょう。	これから も りょうこく の しんぜん を ふかめましょう 。 
\\	彼女は若者のアイドルです。	かのじょ は わかも の のアイドル です 。 
\\	彼の予想は大当たりでした。	かれ の よそう は おおあたり でした 。 
\\	惜しい、もう少しで優勝だった。	おしい 、 もうすこし で ゆうしょう だった 。 
\\	この子は覚えが早いね。	この こ は おぼえ が はやい ね 。 
\\	彼は時々妹をからかいます。	かれ は ときどき いもうと を からかいます 。 
\\	着替えを旅行カバンに詰めたよ。	きがえ を りょこう カバン に つめた よ 。 
\\	全部で6時間の行程です。	ぜんぶ で 
\\	じかん の こうてい です 。 
\\	小雨なので傘はいりません。	こさめ な ので かさ は いりません 。 
\\	彼の名前をご存じですか。	かれ の なまえ を ごぞんじ です か 。 
\\	彼女の心の叫びに誰も気付かなかったよ。	かのじょ の こころ の さけび に だれ も きづかなかった よ 。 
\\	その市の人口は減り続けているの。	その し の じんこう は へりつづけている の 。 
\\	彼は私費で留学したんだ。	かれ は しひ で りゅうがく した ん だ 。 
\\	この週刊誌を毎週買っているの。	この しゅうかん し を まいしゅう かっている の 。 
\\	電車の終点で降りました。	でんしゃ の しゅうてん で おりました 。 
\\	今朝の地震は震度3でしたよ。	けさ の じしん は しんど 
\\	でした よ 。 
\\	ここから道の幅が狭まっています。	ここ から みち の はば が せばまっています 。 
\\	私たちの会社は今月は赤字よ。	わたし たち の かいしゃ は こんげつ は あかじ よ 。 
\\	何をそんなに焦っているのですか。	なに を そんなに あせっている の です か 。 
\\	宛先不明で手紙が戻ってきたの。	あてさき ふめい で てがみ が もどってきた の 。 
\\	案の定、彼は遅刻したな。	あんのじょう 、 かれ は ちこく した な 。 
\\	第一志望の大学に受かりました。	だい いち しぼう の だいがく に うかりました 。 
\\	受け取りに判子をお願いします。	うけとり に はんこ を おねがい します 。 
\\	彼は車を後ろ向きに駐車したの。	かれ は くるま を うしろむき に ちゅう しゃ した の 。 
\\	大急ぎでその仕事を仕上げたよ。	おおいそぎ で その しごと を しあげた よ 。 
\\	弟はようやくベッドから起き上がったの。	おとうと は ようやく ベッド から おきあがった の 。 
\\	学校は生徒たちを午前中に帰したね。	がっこう は せいと たち を ごぜん ちゅう に きした ね 。 
\\	彼には貸しがあるの。	かれ に は かし が ある の 。 
\\	金もうけが彼の趣味だ。	かねもうけ が かれ の しゅみ だ 。 
\\	手首の関節をひねっちゃった。	てくび の かんせつ を ひねっちゃった 。 
\\	後悔しても、しょうがない。	こうかい して も 、 しょうがない 。 
\\	人間は高等な生物と言われている。	にんげん は こうとう な せいぶつ と いわれている 。 
\\	ロックは殊に若者に人気だ。	ロック は ことに わかもの に にんき だ 。 
\\	そこの酒屋さんでビールを買って来て。	そこ の さかや さん で ビール を かってきて 。 
\\	彼は参議院議員です。	かれ は さんぎいん ぎいん です 。 
\\	姉は大学で自然科学を学んでいます。	あね は だいがく で しぜん かがく を まなん でいます 。 
\\	彼は使用人を首にしたよ。	かれ は しようにん を くび に した よ 。 
\\	母に真珠のネックレスをもらいました。	はは に しんじゅ の ネックレス を もらいました 。 
\\	彼は5人兄弟の末っ子です。	かれ は 
\\	にん きょうだい の すえっこ です 。 
\\	よく眠ったら気分がすっきりした。	よく ねむったら きぶん が すっきり した 。 
\\	10問中9問正解しました。	10もん ちゅう 
\\	もん せいかい しました 。 
\\	もっと範囲を狭めて探しましょう	もっと はんい を せばめて さがしましょう 
\\	彼女は夕食に天ぷらを揚げました。	かのじょ は ゆうしょく に てんぷら を あげました 。 
\\	あなたは言い訳が多すぎます。	あなた は いいわけ が おおすぎます 。 
\\	息子はもう異性を意識している。	むすこ は もう いせい を いしき している 。 
\\	彼はタバコの煙を嫌がるの。	かれ は タバコ の けむり を いやがる の 。 
\\	彼を恨んではいけません。	かれ を うらん で はいけません 。 
\\	彼は永久に帰らぬ人となったのよ。	かれ は えいきゅう に かえらぬ ひ と となった の よ 。 
\\	母は今、うちに居りません。	は は はいま 、 うち に おりません 。 
\\	夏休みには海水浴に行きます。	なつやすみ に は かいすいよく に いきます 。 
\\	「手が空く」は慣用句です。	
\\	て が あく 
\\	は かんよう く です 。 
\\	この薬の効き目は素晴らしいの。	この くすり の ききめ は すばらしい の 。 
\\	彼女は大変機嫌がいいね。	かのじょ は たいへん きげん が いい ね 。 
\\	彼の背広に糸屑がついているわ。	かれ の せびろ に いと くず が ついている わ 。 
\\	暦の上では今日から冬ですね。	こよみ の うえ で は きょう から ふゆ です ね 。 
\\	雨が降ってきたので傘を差しました。	あめ が ふってきた ので かさ を さしました 。 
\\	今年も残暑が厳しかった。	ことし も ざんしょ が きびしかった 。 
\\	雷がごろごろ鳴っている。	かみなり が ごろごろ なっている 。 
\\	この三角形の面積を出しなさい。	この さんかっけい の めんせき を だしなさい 。 
\\	ジョギング用のシューズは安くないな。	ジョギング よう の シューズ は やすくない な 。 
\\	このドレスはとても上品なデザインですね。	この ドレス は とても じょうひん な デザイン です ね 。 
\\	これは初級の教科書です。	これ は しょきゅう の きょうかしょ です 。 
\\	今日から新学期が始まります。	きょう から しん がっき が はじまります 。 
\\	食べ物の好き嫌いは特にありません。	たべも の のすききらい は とくに ありません 。 
\\	湖の水は透き通っていたよ。	みずうみ の みず は すきとおっていた よ 。 
\\	正方形の紙を用意しましょう。	せいほうけい の かみ を ようい しましょう 。 
\\	西洋人は正座が苦手よ。	せいよう じん は せいざ が にがて よ 。 
\\	彼は足首を痛めています。	かれ は あしくび を いためています 。 
\\	スープを少し薄めましょうか。	スープ を すこし うすめましょう か 。 
\\	この傘は柄が丈夫だな。	この かさ は え が じょうぶ だ な 。 
\\	シャツを裏返しに着ているよ。	シャツ を うらがえし に きている よ 。 
\\	何かお手伝いしましょうか。	なに か おてつだい しましょう か 。 
\\	初めてのキスは海岸でだったの。	はじめて の キス は かいがん で だった の 。 
\\	旧ソビエトは今はロシアと呼ばれている。	きゅう ソビエト は いま は ロシア と よばれている 。 
\\	石が粉々に砕けましたね。	いし が こなごな に くだけました ね 。 
\\	試合に負けてとても悔しい。	しあい に まけて とても くやしい 。 
\\	猫は私の足元をくるくると回ったの。	ねこ は わたし の あしもと を くるくる と まわった の 。 
\\	彼女はクモを怖がります。	かのじょ は クモ を こわがります 。 
\\	彼女に本を返すように催促したの。	かのじょ に ほん を かえす よう に さいそく した の 。 
\\	また仕事をサボっていますね。	また しごと を サボッています ね 。 
\\	台風が去って風が静まったね。	たいふう が さって かぜ が しずまった ね 。 
\\	新しい事務員が入りました。	あたらしい じむ いん が はいりました 。 
\\	上等なワインを飲んだの。	じょうとう な ワイン を のんだ の 。 
\\	彼女のいちばんの楽しみはショッピングです。	かのじょ の いちばん の たのしみ は ショッピング です 。 
\\	彼女の家は神道です。	かのじょ の いえ は しんとう です 。 
\\	今はほとんどのトイレが水洗ですよ。	いま は ほとんど の トイレ が すいせん です よ 。 
\\	彼はいつもスマートに行動するね。	かれ は いつも スマート に こうどう する ね 。 
\\	玄関に靴がずらりと並んでいたな。	げんかん に くつ が ずらりと ならん でいた な 。 
\\	今夜は天の川が見えますね。	こんや は あまの が わ がみえます ね 。 
\\	彼は金に卑しいね。	かれ は きん に いやしい ね 。 
\\	お客様がいらっしゃいました。	おきゃくさま が いらっしゃいました 。 
\\	ウナギの蒲焼きは美味しいね。	ウナギ の かばやき は おいしい ね 。 
\\	仏像に手を合わせて拝みました。	ぶつぞう に て を あわせて おがみました 。 
\\	歩道に新しくガードが付きましたね。	ほどう に あたらしく ガード が つきました ね 。 
\\	たまには外食しましょう。	たま に は がい しょく しましょう 。 
\\	大雨で崖が崩れたんだ。	おおあめ で がけ が くずれた ん だ 。 
\\	かぼちゃのスープは美味しいです。	かぼちゃ の スープ は おいしい です 。 
\\	僕の財布は空っぽだよ。	ぼく の さいふ は からっぽ だ よ 。 
\\	彼女はいつも聞き手にまわるね。	かのじょ は いつも ききて に まわる ね 。 
\\	彼のマジックはまるで奇跡です。	かれ の マジック は まるで きせき です 。 
\\	この本を君にあげます。	この ほん を きみ に あげます 。 
\\	彼は軽い近視です。	かれ は かるい きんし です 。 
\\	クイズ番組を見るのが好きです。	クイズ ばんぐみ を みる の が すき です 。 
\\	私は洗い物をしたあと、手にクリームをぬります。	わたし は あらいもの を した あと 、 て に クリーム を ぬります 。 
\\	犬が自分のしっぽを追いかけてぐるぐる回っている。	いぬ が じぶん の しっぽ を おいかけて ぐるぐる まわっている 。 
\\	漢字の読み方には、音と訓の2通りがあるの。	かんじ の よみかた に は 、 お と とくん の 
\\	とおり が ある の 。 
\\	この雑誌は月刊ですか。	この ざっし は げっかん です か 。 
\\	この暑さは高気圧のせいです。	この あつ さ は こうきあつ の せい です 。 
\\	こちらを差し上げます。	こちら を さしあげます 。 
\\	それはさぞがっかりしたことでしょう。	それ は さぞ がっかり した こと でしょう 。 
\\	彼はサングラスが似合うね。	かれ は サングラス が にあう ね 。 
\\	その部屋は始終、人が出入りしているね。	その へや は しじゅう 、 ひと が でいり している ね 。 
\\	消火器を交換しました。	しょうか き を こうかん しました 。 
\\	子供を小児科に連れて行くところです。	こども を しょう に か につれていく ところ です 。 
\\	大学で人文科学を専攻しました。	だいがく で じんぶん かがく を せんこう しました 。 
\\	おしゃべりは後回しにしましょう。	おしゃべり は あとまわし に しましょう 。 
\\	彼は威張ってなんかいません。	かれ は いばって なんかいません 。 
\\	大麦はビールの原料になります。	おおむぎ は ビール の げんりょう に なります 。 
\\	脅かさないでよ。	おびやかさない で よ 。 
\\	お前の言うことは信じられない。	おまえ の いう こと は しんじられない 。 
\\	この漢字の音読みは何ですか。	この かんじ の おんよみ は なに です か 。 
\\	インターネットで試験の解答を確認したんだ。	インターネット で しけん の かいとう を かくにん した ん だ 。 
\\	花瓶にお花を生けました。	かびん に お はな を いけました 。 
\\	彼女は娘に可愛らしい服を作りましたね。	かのじょ は むすめ に かわいらしい ふく を つくりました ね 。 
\\	彼女はきっちり3時に来たわ。	かのじょ は きっちり 
\\	じ に きた わ 。 
\\	1年休学することにしました。	1ねん きゅうがく する こと に しました 。 
\\	きゅうりに味噌をつけて食べたの。	きゅうり に みそ を つけて たべた の 。 
\\	インドでは英語は共通語だよ。	インド で は えいご は きょうつう ご だ よ 。 
\\	この庭は草木が枯れているね。	この にわ は くさき が かれている ね 。 
\\	外は木枯らしが吹いているよ。	そと は こがらし が ふいている よ 。 
\\	私はベンチに腰掛けたの。	わたし は ベンチ に こしかけた の 。 
\\	旅先で思いがけない災難にあいました。	たびさき で おもいがけない さいなん に あいました 。 
\\	絵を逆さにしてみてごらん。	え を さかさ に してみて ごらん 。 
\\	列車の窓から雄大な山脈が見えたんだよ。	れっしゃ の まど から ゆうだい な さんみゃく が みえた ん だ よ 。 
\\	始発の電車に乗りました。	しはつ の でんしゃ に のりました 。 
\\	レバーを右に回すと締まります。	レバー を みぎ に まわす と しまります 。 
\\	うちの会社は週休2日です。	うち の かいしゃ は しゅうきゅう 
\\	にち です 。 
\\	言いたいことを言ったら胸がすっとした。	いいたい こと を いったら むね が すっと した 。 
\\	その小学生は難しい本をすらすら読んだの。	その しょうがくせい は むずかしい ほん を すらすら よんだ の 。 
\\	警察はそのグループを怪しんでいます。	けいさつ は その グループ を あやしん でいます 。 
\\	庭に穴を掘ってそれを埋めました。	にわ に あな を ほって それ を うめました 。 
\\	両親を敬うことは大切です。	りょうしん を うやまう こと は たいせつ です 。 
\\	その男はロビーをしばらくうろうろしていたよ。	その おとこ は ロビー を しばらく うろうろ していた よ 。 
\\	彼はお世辞を言うのが上手い。	かれ は お せじ を いう の が うまい 。 
\\	家族でお寺にお参りに行きました。	かぞく で おてら に おまいり に いきました 。 
\\	八百屋さんでたくさんおまけしてもらったよ。	やおや さん で たくさん おまけ してもらった よ 。 
\\	この鞄は重たいです。	この かばん は おもたい です 。 
\\	改札で3時に会おう。	かいさつ で 
\\	じ に あおう 。 
\\	ここで2つの海流が出合っている。	ここ で 
\\	つ の かいりゅう が であっている 。 
\\	背中をお母さんに掻いてもらったの。	せなか を おかあさん に かいてもらった の 。 
\\	眼科で視力検査をしました。	がんか で しりょく けんさ を しました 。 
\\	待ち合わせは2時だと勘違いしていました。	まちあわせ は 
\\	じ だ と かんちがい していました 。 
\\	昨夜はテレビで喜劇を見たよ。	さくや は テレビ で きげき を みた よ 。 
\\	この建物は建築の規準に達していないよ。	この たてもの は けんちく の きじゅん に たっしていない よ 。 
\\	この子は本当に行儀の良い子です。	この こ は ほんとうに ぎょうぎ の よい こ です 。 
\\	この包丁は切れが良いな。	この ほうちょう は きれ が よい な 。 
\\	クッキーと紅茶をいただきました。	クッキー と こうちゃ を いただきました 。 
\\	プラグをコンセントに挿し込んだよ。	プラグ を コンセント に さし こんだ よ 。 
\\	この参考書はとても役に立つよ。	この さんこう しょ は とても やくにたつ よ 。 
\\	終電にやっと間に合った。	しゅうでん に やっと まにあった 。 
\\	この先は徐行して下さい。	この さき は じょこう してください 。 
\\	朝は雀の声で目が覚めます。	あさ は すずめ の こえ で め が さめます 。 
\\	あなたのやり方はずるい。	あなた の やりかた は ずるい 。 
\\	彼はせっかちで困ります。	かれ は せっかち で こまります 。 
\\	秋風が気持ちいいね。	あきかぜ が きもちいい ね 。 
\\	彼の助けは本当に有り難いな。	かれ の たすけ は ほんとうに ありがたい な 。 
\\	彼は胃腸が強くありません。	かれ は いちょう が つよく ありません 。 
\\	彼はいつも受け身の姿勢で、自分からは何もしないんだ。	かれ は いつも うけみ の しせい で 、 じぶん から は なに も しない ん だ 。 
\\	父はうちわで扇いでいるの。	ちち は うちわ で あおい でいる の 。 
\\	大事なメールをうっかり消してしまった。	だいじ な メール を うっかり けしてしまった 。 
\\	バスの回数券を買いました。	バス の かいすう けん を かいました 。 
\\	火力を弱めないと肉がこげるよ。	かりょく を よわめない と にく が こげる よ 。 
\\	知り合いが急病で倒れた。	しりあい が きゅうびょう で たおれた 。 
\\	くだらないおしゃべりはやめなさい。	くだらない おしゃべり は やめなさい 。 
\\	赤い口紅を買いました。	あかい くちべに を かいました 。 
\\	軽率な発言はしないよう気をつけなさい。	けいそつ な はつげん は しない よう き を つけなさい 。 
\\	そんなに謙遜しなくてもいい。	そんなに けんそん しなくて もいい 。 
\\	秋は行楽によい季節ですね。	あき は こうらく に よい きせつ です ね 。 
\\	この仕事は時給1000円です。	この しごと は じきゅう 
\\	えん です 。 
\\	十字路で車とバイクが衝突したぞ。	じゅうじろ で くるま と バイク が しょうとつ した ぞ 。 
\\	私用で出かけなければなりません。	しよう で でかけなけれ ばなりません 。 
\\	英語を初歩から勉強しています。	えいご を しょほ から べんきょう しています 。 
\\	お酢を入れすぎて酸っぱい。	お す を いれすぎて すっぱい 。 
\\	この島は発電を水力に頼っているんだ。	この しま は はつでん を すいりょく に たよっている ん だ 。 
\\	何てずうずうしい人なんだろう。	なにて ずうずうしい ひと な ん だろう 。 
\\	子供がすやすや眠っているね。	こども が すやすや ねむっている ね 。 
\\	この食堂はセルフサービスです。	この しょくどう は セルフサービス です 。 
\\	この子はアルファベットを全部言えます。	この こ は アルファベット を ぜんぶ いえます 。 
\\	彼女は負けずに言い返したの。	かのじょ は まけず に いいかえした の 。 
\\	父のいびきは大きいんだ。	ちち の いびき は おおきい ん だ 。 
\\	彼はいやらしい男ね。	かれ は いやらしい おとこ ね 。 
\\	イントネーションの違いに気を付けて。	イントネーション の ちがい に き を つけて 。 
\\	来週の予定を打ち合わせましょう。	らいしゅう の よてい を うちあわせましょう 。 
\\	試合は雨天中止です。	しあい は うてん ちゅうし です 。 
\\	社長の奥様はきれいな方です。	しゃちょう の おくさま は きれい な ほう です 。 
\\	駅まで駆け足で行ったよ。	えき まで かけあし で おこなった よ 。 
\\	この椅子はがたがたしているね。	この いす は がたがた している ね 。 
\\	この鉢植えは観賞用です。	この はちうえ は かんしょう よう です 。 
\\	カンニングをした人は0点です。	カンニング を した ひと は 
\\	てん です 。 
\\	3は奇数です。	3は きすう です 。 
\\	彼は禁酒しています。	かれ は きん しゅ しています 。 
\\	偶数は2で割り切れます。	ぐうすう は 
\\	で わりきれます 。 
\\	事故現場には入れません。	じこ げんば に は いれません 。 
\\	新しい国立劇場が完成しました。	あたらしい こくりつ げきじょう が かんせい しました 。 
\\	シチューが焦げたよー。	シチュー が こげた よー 。 
\\	この座布団は座り心地がいいね。	この ざぶとん は すわり ごこち が いい ね 。 
\\	彼は最近、失恋したらしいの。	かれ は さいきん 、 しつれん した らしい の 。 
\\	雨の日が続いて家の中が湿っぽいな。	あめ の ひ が つづいて いえ の なか が しめっぽい な 。 
\\	主観だけで物事を見てはいけない。	しゅかん だけ で ものごと を みて はいけない 。 
\\	それは主観的な意見だ。	それ は しゅかん てき な いけん だ 。 
\\	小数は切り捨てて計算して下さい。	しょうすう は きりすてて けいさん してください 。 
\\	水平線に太陽が沈んでいった。	すいへい せん に たいよう が しずん でいった 。 
\\	彼はあぐらをかいてしゃべっていますね。	かれ は あ ぐらをかいてしゃべっていますね 。 
\\	紫の朝顔が咲きました。	むらさき の あさ が お がさきました 。 
\\	彼は時々意地悪な質問をするぜ。	かれ は ときどき いじわる な しつもん を する ぜ 。 
\\	コンビニは至る所にあります。	コンビニ は いたる ところ に あります 。 
\\	あなたも今に分かるでしょう。	あなた も いま に わかる でしょう 。 
\\	その子はいやいや学校へ行ったの。	その こ は いやいや がっこう へ いった の 。 
\\	最近ウエストが太くなったな。	さいきん ウエスト が ふとく なった な 。 
\\	彼は運送会社に勤めているの。	かれ は うんそう かいしゃ に つとめている の 。 
\\	襟の大きいコートを買いました。	えり の おおきい コート を かいました 。 
\\	咳が少し治まりました。	せき が すこし おさまりました 。 
\\	僕たちはオルガンに合わせて賛美歌を歌ったんだ。	ぼく たち は オルガン に あわせて さんび か を うたった ん だ 。 
\\	日本は温帯にあります。	にっぽん は おんたい に あります 。 
\\	この地方は温暖で暮らしやすいな。	この ちほう は おんだん で くらしやすい な 。 
\\	書き言葉と話し言葉はだいぶ違うことがあります。	かきことば と はなしことば は だいぶ ちがう こと が あります 。 
\\	陰で少し休憩しましょう。	かげ で すこし きゅうけい しましょう 。 
\\	祖父は古い仮名遣いで書くことがあるの。	そふ は ふるい かなづかい で かく こと が ある の 。 
\\	部長は僕の仕事に辛い評価を出したんだ。	ぶちょう は ぼく の しごと に つらい ひょうか を だした ん だ 。 
\\	彼はからからと笑いました。	かれ は からから と わらいました 。 
\\	お店のガラス戸が割られたんだ。	お みせ の ガラス と が わられた ん だ 。 
\\	体育館で器械を使って運動したの。	たいいくかん で きかい を つかって うんどう した の 。 
\\	彼の体型じゃ既製のサイズに合わないよ。	かれ の たいけい じゃ きせい の サイズ に あわない よ 。 
\\	彼女は驚いて気絶してしまったの。	かのじょ は おどろいて きぜつ してしまった の 。 
\\	彼女はキッチンで夕食を作っています。	かのじょ は キッチン で ゆうしょく を つくっています 。 
\\	お客さんを客間にお通ししたわよ。	おきゃく さん を きゃくま に おとお し した わ よ 。 
\\	彼は勤勉な人です。	かれ は きんべん な ひと です 。 
\\	この服は上下セットで買いました。	この ふく は じょうげ セット で かいました 。 
\\	犬を鎖でつなぎました。	いぬ を くさり で つなぎました 。 
\\	紙をくしゃくしゃに丸めました。	かみ を くしゃくしゃ に まるめました 。 
\\	きのう、弟と口げんかしました。	きのう 、 お と うと とくちげんか しました 。 
\\	景色が良かったので途中下車したよ。	けしき が よかった ので とちゅう げしゃ した よ 。 
\\	彼は金持ちだけど、けちね。	かれ は かねもち だ けど 、 けち ね 。 
\\	ここには現住所を書いてください。	ここ に は げんじゅうしょ を かいて ください 。 
\\	恋する気持ちを歌にしました。	こいする きもち を うた に しました 。 
\\	まず、肉をパン粉の上で転がします。	まず 、 にく を パンコ の うえ で ころがします 。 
\\	デパートの歳末大売出しが始まったよ。	デパート の さいまつ おおうりだし が はじまった よ 。 
\\	釘に引っかけて袖が裂けてしまったな。	くぎ に ひっかけて そで が さけてしまった な 。 
\\	砂ぼこりで机がざらざらしているね。	すなぼこり で つくえ が ざらざら している ね 。 
\\	彼女はソファーに体を沈めたんだ。	かのじょ は ソファー に からだ を しずめた ん だ 。 
\\	今、論文の下書きをしているところです。	いま 、 ろんぶん の したがき を している ところ です 。 
\\	うちは3人姉妹です。	うち は 
\\	にん しまい です 。 
\\	この本の第6章が特に好きです。	この ほん の だい 
\\	しょう が とくに すき です 。 
\\	ちょっと小便しに行って来る。	ちょっと しょうべん し に いってくる 。 
\\	都会の空はスモッグで灰色ね。	とかい の そら は スモッグ で はいいろ ね 。 
\\	すりに財布をとられた。	すり に さいふ を とられた 。 
\\	奴は裏門から出てきたぜ。	やつ は うらもん から でてきた ぜ 。 
\\	誰かがガラガラとうがいをしているね。	だれか が ガラガラ とう がい を している ね 。 
\\	彼女はかんかんに怒っています。	かのじょ は かんかん に おこっています 。 
\\	彼女は近眼です。	かのじょ は きんがん です 。 
\\	彼女は薬指に指輪をはめています。	かのじょ は くすりゆび に ゆびわ を はめています 。 
\\	彼女の下品なふるまいには我慢できない。	かのじょ の げひん な ふるまい に は がまん できない 。 
\\	これまでの私とは違うんです。	これ まで の わたし と は ちがう ん です 。 
\\	夕食の最中に電話がかかってきたの。	ゆうしょく の さいちゅう に でんわ が かかってきた の 。 
\\	会社から「至急」との連絡があったんだ。	"かいしゃ から 
\\	しきゅう 
\\	と の れんらく が あった ん だ 。 
\\	まず第一に、しっかり下調べをしなさい。	まず だい いち に 、 しっかり したしらべ を しなさい 。 
\\	梅雨時はじめじめする。	つゆ じ は じめじめ する 。 
\\	先生に暑中見舞いを出しました。	せんせい に しょちゅう みまい を だしました 。 
\\	人をじろじろ見るものではありません。	ひと を じろじろ みる もの で は ありません 。 
\\	あなたの姓名を教えてください。	あなた の せいめい を おしえて ください 。 
\\	アリの群れが砂糖にたかっている。	アリ の むれ が さとう に たかっている 。 
\\	学割だとだいぶ安いな。	がくわり だ と だいぶ やすい な 。 
\\	彼の話に私はあいづちを打ったの。	かれ の はなし に わたし は あい づち を うった の 。 
\\	暗いので足下に気を付けてください。	くらい ので あしもと に き を つけて ください 。 
\\	普段から厚着しないようにしています。	ふだん から あつぎ しない よう に しています 。 
\\	手紙に宛名を書き込んだよ。	てがみ に あてな を かきこんだ よ 。 
\\	雨降りで月が見えなかったね。	あめふり で つき が みえなかった ね 。 
\\	彼はようやく姿を現しましたね。	かれ は ようやく すがた を あらわしました ね 。 
\\	それは彼女の期待の現れです。	それ は かのじょ の きたい の あらわれ です 。 
\\	私からご連絡致します。	わたし から ご れんらくいたします 。 
\\	次に、野菜を炒めてください。	つぎに 、 やさい を いためて ください 。 
\\	物価は一段と上昇しそうだね。	ぶっか は いちだんと じょう し ょう しそう だ ね 。 
\\	私たちは庭園を一周しました。	わたし たち は ていえん を いっ しゅう しました 。 
\\	荷物が多いので大きな入れ物が必要です。	にもつ が おおい ので おおきな いれもの が ひつよう です 。 
\\	彼はうわさを打ち消したわよ。	かれ は うわさ を うちけした わ よ 。 
\\	英和辞書をよく使います。	えいわ じしょ を よく つかいます 。 
\\	私たちは出発を1日遅らしたの。	わたし たち は しゅっぱつ を 
\\	にち おくらした の 。 
\\	海上で衝突事故が発生した。	かいじょう で しょうとつ じこ が はっせい した 。 
\\	僕たちは毎朝漢字の書き取りをします。	ぼく たち は まいあさ かんじ の かきとり を します 。 
\\	母は鍋のシチューを掻き回しているよ。	は は はなべ の シチュー を かきまわしている よ 。 
\\	池がかちかちに凍っています。	いけ が かちかち に こおっています 。 
\\	皆が代わる代わる彼らを祝福したよ。	みな が かわる かわる かれら を しゅくふく した よ 。 
\\	今日のフランス語の講座は休講です。	きょう の フランスゴ の こうざ は きゅうこう です 。 
\\	今日はたくさん歩いてくたくたです。	きょう は たくさん あるいて くたくた です 。 
\\	小さなバケツで水を汲んだの。	ちいさな バケツ で みず を くんだ の 。 
\\	先生に月謝を渡しましたか。	せんせい に げっしゃ を わたしました か 。 
\\	論文の項目を分かりやすく整理したわ。	ろんぶん の こうもく を わかりやすく せいり した わ 。 
\\	彼らは交代でボートを漕いだんだ。	かれら は こうたい で ボート を こいだ ん だ 。 
\\	こしょうを入れ過ぎてスープが辛い。	こしょう を いれすぎて スープ が つらい 。 
\\	あの山のすぐ近くが国境です。	あの やま の すぐ ちかく が こっきょう です 。 
\\	それじゃあ上と下が逆さまだよ。	それ じゃあ うえとした が さかさま だ よ 。 
\\	今日は久しぶりに日が射してるね。	きょう は ひさしぶり に ひ が さしてる ね 。 
\\	会長は田中氏に決定。	かいちょう は たなか し に けってい 。 
\\	量より質の方が大事です。	りょう より しつ の ほう が だいじ です 。 
\\	新宿駅まで準急で行ったのよ。	しんじゅく えき まで じゅんきゅう で おこなった の よ 。 
\\	彼女の肌はすべすべしているね。	かのじょ の はだ は すべすべ している ね 。 
\\	結婚して姓が変わりました。	けっこん して せい が かわりました 。 
\\	せっかく来たんだからゆっくりして行きなさい。	せっかく きた ん だ から ゆっくり していきなさい 。 
\\	部屋の空きはありますか。	へや の あき は あります か 。 
\\	この商品は小売店でも買えます。	この しょうひん は こうり てん で も かえます 。 
\\	このブラウスは後ろ開きです。	この ブラウス は うしろ びらき です 。 
\\	彼は留学の夢を諦めていないよ。	かれ は りゅうがく の ゆめ を あきらめていない よ 。 
\\	彼の頑固さにはあきれました。	かれ の がんこ さ に は あきれました 。 
\\	彼は悪女に騙されたんだ。	かれ は あくじょ に だまされた ん だ 。 
\\	海外に住むのは私の憧れです。	かいがい に すむ の は わたし の あこがれ です 。 
\\	彼はパイロットの職に憧れているんだ。	かれ は パイロット の しょく に あこがれている ん だ 。 
\\	最近は暖かです。	さいきん は あたたか です 。 
\\	彼のあだなは「りき」です。	"かれ の あだな は 
\\	りき 
\\	です 。 
\\	食事の後片付けを手伝ったの。	しょくじ の あとかたづけ を てつだった の 。 
\\	バスタブからお湯があふれました。	バス タブ から おゆ が あふれました 。 
\\	雨具の用意を忘れないように。	あまぐ の ようい を わすれない よう に 。 
\\	嵐で庭の木が折れたよ。	あらし で にわ の き が おれた よ 。 
\\	彼は強盗に家の中を荒されたの。	かれ は ごうとう に いえ の なか を あらされた の 。 
\\	その映画はどんな粗筋ですか。	その えいが は どんな あらすじ です か 。 
\\	その2国は資源をめぐって争っています。	その 
\\	こく は しげん をめぐって あらそっています 。 
\\	この本では自然の大切さがよく著されているよ。	この ほん で は しぜん の たいせつ さ が よく あらわされている よ 。 
\\	有りのままを話して下さい。	あり の まま を はなしてください 。 
\\	この気持ちをうまく言い表わせません。	この きもち を うまく いい あらわせません 。 
\\	突然何を言い出すのかと思った。	とつぜん なに を いいだす の か と おもった 。 
\\	私はイカの刺し身が大好きです。	わたし は イカ の さしみ が だいすき です 。 
\\	最近彼のところに行き来してないの。	さいきん かれ の ところ に いきき してない の 。 
\\	ここから先は行き止まりです。	ここ から さき は ゆきどまり です 。 
\\	動物を苛めてはいけません。	どうぶつ を いじめて はいけません 。 
\\	お店は一時休業になったんだ。	お みせ は いちじ きゅうぎょう に なった ん だ 。 
\\	彼はいつのまにいなくなったの。	かれ は いつのまに いなく なった の 。 
\\	春は一遍に花が咲く。	はる は いっぺん に はな が さく 。 
\\	日本は稲の品種が多いよ。	にっぽん は いね の ひんしゅ が おおい よ 。 
\\	彼はソファーで居眠りをしているよ。	かれ は ソファー で いねむり を している よ 。 
\\	市場にはインフレーションの影響が出ているね。	しじょう に は インフレーション の えいきょう が でている ね 。 
\\	今度のウィークエンドは映画を見ます。	こんど の ウィークエンド は えいが を みます 。 
\\	彼は上下おそろいの服を着ていますね。	かれ は じょうげ おそろい の ふく を きています ね 。 
\\	冬はまめにうがいをします。	ふゆ は まめ に う がい を します 。 
\\	もうすぐ夏休みなのでうきうきしています。	もうすぐ なつやすみ な ので うきうき しています 。 
\\	パレードと観衆が道を埋めていたよ。	パレード と かんしゅう が みち を うめていた よ 。 
\\	嘘つきは泥棒の始まりよ。	うそつき は どろぼう の はじまり よ 。 
\\	彼はうつむいて何かを考えているね。	かれ は うつむいて なに か を かんがえている ね 。 
\\	彼女はうなずいて同意を示したの。	かのじょ は うなずいて どうい を しめした の 。 
\\	シャツを裏表に着ていますよ。	シャツ を うらおもて に きています よ 。 
\\	その噂は本当ですか。	その うわさ は ほんとう です か 。 
\\	毎日金魚に餌をやります。	まいにち きんぎょ に えさ を やります 。 
\\	木の枝を折ってはいけません。	き の えだ を おって はいけません 。 
\\	彼は大金を得ましたよ。	かれ は たいきん を えました よ 。 
\\	雲が空を覆っているね。	くも が そら を おおっている ね 。 
\\	その年、この地域では大水がありました。	その とし 、 この ちいき で は おおみず が ありました 。 
\\	日本では、大晦日にそばを食べます。	にっぽん で は 、 おおみそか に そば を たべます 。 
\\	私たちは舟を降りて、陸に上がったんだ。	わたし たち は ふね を おりて 、 りく に あがった ん だ 。 
\\	小川がさらさら流れています。	おがわ が さらさら ながれています 。 
\\	弟は小さい頃は臆病だったんだ。	おとうと は ちいさい ころ は おくびょう だった ん だ 。 
\\	彼女には幼い息子がいます。	かのじょ に は おさない むすこ が います 。 
\\	話はこれでお仕舞いです。	はなし は これ で お しまい です 。 
\\	うちの猫は雄です。	うち の ねこ は ゆう です 。 
\\	デパートでお歳暮を送りました。	デパート で お せいぼ を おくりました 。 
\\	彼は人をおだてるのが上手です。	かれ は ひと を おだてる の が じょうず です 。 
\\	財布の落とし物を拾いました。	さいふ の おとしもの を ひろいました 。 
\\	あなたもだいぶ大人になったね。	あなた も だいぶ おとな に なった ね 。 
\\	彼は溺れている子を助けたんだ。	かれ は おぼれている こ を たすけた ん だ 。 
\\	お目に掛かれて嬉しいです。	お め に かかれて うれしい です 。 
\\	誤解があったことをお詫びします。	ごかい が あった こと を おわび します 。 
\\	ここの温泉はよく効くそうです。	ここ の おんせん は よく きく そう です 。 
\\	会は午後9時に終わりました。	かい は ごご 
\\	じ に おわりました 。 
\\	その国は市場の開放を求められているの。	その くに は しじょう の かいほう を もとめられている の 。 
\\	もう一度原点に返って考えましょう。	もういちど げんてん に かえって かんがえましょう 。 
\\	靴のかかとで彼の足を踏んでしまったの。	くつ の かかと で かれ の あし を ふん でしまった の 。 
\\	医師は人の命に関わる大切な職業だよ。	いし は ひと の いのち に かかわる たいせつ な しょくぎょう だ よ 。 
\\	玄関でがさがさと音がしましたよ。	げんかん で がさがさ と おと が しました よ 。 
\\	弟は美味しそうにりんごをかじっていたよ。	おとうと は おいし そう に りんご を かじっていた よ 。 
\\	階下から微かな音が聞こえた。	かいか から かすか な おと が きこえた 。 
\\	片方の目がかゆいです。	かたほう の め が かゆい です 。 
\\	道に土の塊ができてたよ。	みち に ど の かたまり が できてた よ 。 
\\	お日様が西に傾きましたね。	おひさま が にし に かたむきました ね 。 
\\	彼女は首を少し傾けて笑うの。	かのじょ は くび を すこし かたむけて わらう の 。 
\\	格好のいい青年に会ったよ。	かっこう の いい せいねん に あった よ 。 
\\	彼は水をがぶがぶ飲んだんだ。	かれ は みず を がぶがぶ のんだ ん だ 。 
\\	本にカバーを被せたの。	ほん に カバー を かぶせた の 。 
\\	お上は何を考えてるんだろうね。	おかみ は なに を かんがえてる ん だろう ね 。 
\\	犬が手に噛み付きました。	いぬ が て に かみつきました 。 
\\	このトランクは貨物で送ろう。	この トランク は かもつ で おくろう 。 
\\	この瓶はもう空ね。	この びん は もう そら ね 。 
\\	木の上でからすが鳴いています。	き の うえ で からす が ないています 。 
\\	りんごの皮をむきましたよ。	りんご の かわ を むきました よ 。 
\\	彼女は猫を可愛がっています。	かのじょ は ねこ を かわいがっています 。 
\\	その間に彼は居なくなっていました。	そのかん に かれ は いなく なっていました 。 
\\	祖父はがんで亡くなりました。	そふ は がん で なくなりました 。 
\\	観客は興奮していたよ。	かんきゃく は こうふん していた よ 。 
\\	台風で交通機関がストップしている。	たいふう で こうつう きかん が ストップ している 。 
\\	私は彼女の言葉に傷付きました。	わたし は かのじょ の ことば に きず つきました 。 
\\	あなたを傷付けるつもりはありませんでした。	あなた を きずつける つもり は ありません でした 。 
\\	多くの人が災害の犠牲になったの。	おおく の ひと が さいがい の ぎせい に なった の 。 
\\	この箱にはぎっしり物が詰まっています。	この はこ に は ぎっしり もの が つまっています 。 
\\	電車がぎゅうぎゅうに込んでいますね。	でんしゃ が ぎゅうぎゅう に こん でいます ね 。 
\\	5行以内で答えを書いてください。	5こう いない で こたえ を かいて ください 。 
\\	ここのぎょうざは美味しいよ。	ここ の ぎょ う ざは おいしい よ 。 
\\	余ったきれで人形を作りましょう。	あまった きれ で にんぎょう を つくりましょう 。 
\\	ホテルの金庫に貴重品を入れたよ。	ホテル の きんこ に きちょう ひん を いれた よ 。 
\\	狭い空間に物がたくさん置いてあるね。	せまい くうかん に もの が たくさん おいてある ね 。 
\\	釘を踏んで怪我をしました。	くぎ を ふん で けが を しました 。 
\\	弟に足の裏をくすぐられた。	おとうと に あし の うら を くすぐられた 。 
\\	彼は壁に耳をくっ付けて隣の話を聞いていたの。	かれ は かべ に みみ を くっ つけて となり の はなし を きいていた の 。 
\\	怪我人は軍の病院に運ばれたよ。	けが じん は ぐん の びょういん に はこばれた よ 。 
\\	数百人の群集が集まっているの。	すう ひゃく にん の ぐんしゅう が あつまっている の 。 
\\	彼はマイクで群衆に話しかけたんだ。	かれ は マイク で ぐんしゅう に はなしかけた ん だ 。 
\\	彼女の成績は上の下です。	かのじょ の せいせき は うえ の もと です 。 
\\	相撲の稽古はとても厳しいよ。	すもう の けいこ は とても きびしい よ 。 
\\	煙突から煙がまっすぐ上っていたの。	えんとつ から けむり が まっすぐ のぼっていた の 。 
\\	彼女はいつまでもげらげら笑っていたな。	かのじょ は いつ まで も げらげら わらっていた な 。 
\\	県の代表は2名です。	けん の だいひょう は 
\\	めい です 。 
\\	電話で航空券を予約しました。	でんわ で こうくう けん を よやく しました 。 
\\	家まで交替で荷物を持ちました。	いえ まで こうたい で にもつ を もちました 。 
\\	強盗がカメラに写っていました。	ごうとう が カメラ に うつっていました 。 
\\	ゴールデンウィークに海外旅行をします。	ゴールデンウィーク に かいがい りょこう を します 。 
\\	国民の安全が最も大切です。	こくみん の あんぜん が もっとも たいせつ です 。 
\\	アジアの国名をいくつ知っていますか。	アジア の こくめい を いくつ しっています か 。 
\\	彼は素直な心を持っている。	かれ は すなお な こころ を もっている 。 
\\	乞食が公園のベンチで寝ている。	こじき が こうえん の ベンチ で ねている 。 
\\	名前は五十音順に並んでいます。	なまえ は ごじゅうおん じゅん に ならん でいます 。 
\\	冷えた手を擦って温めた。	ひえた て を こすって あたた めた。 
\\	寒いのでこたつを出しました。	さむい ので こたつ を だしました 。 
\\	彼女への伝言を言付けたの。	かのじょ へ の でんごん を いいつけた の 。 
\\	日本のことわざをいくつくらい知っていますか。	にっぽん の ことわざ を いくつ くらい しっています か 。 
\\	その白い粉は赤ちゃんのミルクです。	その しろい こな は あかちゃん の ミルク です 。 
\\	長いこと御無沙汰いたしました。	ながい こと ごぶさたいたしました 。 
\\	遅れてご免。	おくれて ごめん 。 
\\	これからはもっと気を付けます。	これから は もっと き を つけます 。 
\\	この際はっきり言っておきます。	この さい はっきり いっておきます 。 
\\	うちの会社は祭日は休みです。	うち の かいしゃ は さいじつ は やすみ です 。 
\\	2つの市の境に川が流れているの。	2つ の し の さかい に かわ が ながれている の 。 
\\	その件でしたら、先程上司から許可を得ております。	その けん でしたら 、 さきほど じょうし から きょか を えております 。 
\\	昨日は雨でしたね。	きのう は あめ でした ね 。 
\\	昨晩の雪がまだ庭に残っている。	さくばん の ゆき が まだ にわ に のこっている 。 
\\	米はアジアでは大切な作物だ。	べい は アジア で は たいせつ な さくもつ だ 。 
\\	昨夜の台風で電車が遅れているね。	さくや の たいふう で でんしゃ が おくれている ね 。 
\\	彼女は助けを求めて大声で叫んだの。	かのじょ は たすけ を もとめて おおごえ で さけんだ の 。 
\\	父親には一家を支える責任がある。	ちちおや に は いっか を ささえる せきにん が ある 。 
\\	自転車に油をさしたの。	じてんしゃ に あぶら を さした の 。 
\\	包丁が錆びてしまった。	ほうちょう が さびてしまった 。 
\\	その都市には様々な人種が集まっているわ。	その とし に は さまざま な じんしゅ が あつまっている わ 。 
\\	左右を見てから横断歩道を渡りなさい。	さゆう を みて から おうだん ほどう を わたりなさい 。 
\\	会議は明々後日に延期された。	かいぎ は あきら 々 ごじつ に えんき された 。 
\\	私は通勤に
\\	を使うんだ。	わたし は つうきん に 
\\	を つかう ん だ 。 
\\	明日の授業は時間割り通りです。	あした の じゅぎょう は じかん わり どおり です 。 
\\	マンションの敷金を払いました。	マンション の しききん を はらいました 。 
\\	彼は事業に失敗したの。	かれ は じぎょう に しっぱい した の 。 
\\	来月大阪に支社を開設します。	らいげつ おおさか に ししゃ を かいせつ します 。 
\\	この部屋は暗くて湿気が多いね。	この へや は くらくて しっけ が おおい ね 。 
\\	昨日のコンサートは実に素晴らしかったよ。	きのう の コンサート は じつに すばらしかった よ 。 
\\	この漢字を字引で引いてみて。	この かんじ を じびき で ひいてみて 。 
\\	去年のデータを見せてください。	きょねんのデータをみせてください。 
\\	彼はいつも素早い対応をするね。	かれはいつもすばやいたいおうをするね。 
\\	私はただ彼女と話したかっただけです。	わたしはただかのじょとはなしたかっただけです。 
\\	あの歌手は3部門で賞を取ったの。	あのかしゅは3ぶもんでしょうをとったの。 
\\	野党が与党を厳しく非難していましたね。	やとうがよとうをきびしくひなんしていましたね。 
\\	自民党から新しい党が分離したね。	じみんとうからあたらしいとうがぶんりしたね。 
\\	話の展開についていけない。	はなしのてんかいについていけない。 
\\	三つの町が統合されて新しい市が誕生したのよ。	みっつのまちがとうごうされてあたらしいしがたんじょうしたのよ。 
\\	彼女の仕事はデータの入力です。	かのじょのしごとはデータのにゅうりょくです。 
\\	当初の計画ではもっと早く終わるはずでした。	とうしょのけいかくではもっとはやくおわるはずでした。 
\\	その事件については当局が調査しています。	そのじけんについてはとうきょくがちょうさしています。 
\\	これからは経営の合理化を図りたいと思います。	これからはけいえいのごうりかをはかりたいとおもいます。 
\\	秋は虫の音が心地良いです。	あきはむしのねがここちよいです。 
\\	ひらがなをカタカナに変換しました。	ひらがなをカタカナにへんかんしました。 
\\	東京の地価は上がり続けているんだ。	とうきょうのちかはあがりつづけているんだ。 
\\	船が長い航海に乗り出したの。	ふねがながいこうかいにのりだしたの。 
\\	政治の体制を変える必要があります。	せいじのたいせいをかえるひつようがあります。 
\\	私の予想が当たった。	わたしのよそうがあたった。 
\\	車の費用、予算オーバーだ。	くるまのひよう、よさんオーバーだ。 
\\	彼女は2才の子の母親です。	かのじょは2さいのこのははおやです。 
\\	新しい車が発売された。	あたらしいくるまがはつばいされた。 
\\	大雨で川の水かさが増しているな。	おおあめでかわのみずかさがましているな。 
\\	その土地は民間企業に売却されたんだ。	そのとちはみんかんきぎょうにばいきゃくされたんだ。 
\\	彼女はスペイン語に取り組んでいます。	かのじょはスペインごにとりくんでいます。 
\\	この携帯電話は操作が簡単です。	このけいたいでんわはそうさがかんたんです。 
\\	自転車の乗り過ぎで股が痛いです。	じてんしゃののりすぎでこがいたいです。 
\\	昔のことを思い出したの。	むかしのことをおもいだしたの。 
\\	彼の意見はもっともです。	かれのいけんはもっともです。 
\\	今日は本社で会議があります。	きょうはほんしゃでかいぎがあります。 
\\	あの人は日本銀行の総裁です。	あのひとはにっぽんぎんこうのそうさいです。 
\\	彼女の登場で会場は盛り上がったね。	かのじょのとうじょうでかいじょうはもりあがったね。 
\\	交通費は個人負担です。	こうつうひはこじんふたんです。 
\\	明日、病院に検査に行くので少し不安です。	あした、びょういんにけんさにいくのですこしふあんです。 
\\	彼の意見は無視されたよ。	かれのいけんはむしされたよ。 
\\	社長自らがその会社と交渉したよ。	しゃちょうみずからがそのかいしゃとこうしょうしたよ。 
\\	彼は靴のメーカーで働いています。	かれはくつのメーカーではたらいています。 
\\	この5種類から選択できます。	この5しゅるいからせんたくできます。 
\\	仕事の見通しがたたない。	しごとのみとおしがたたない。 
\\	先月の利益は200万円でした。	せんげつのりえきは200まんえんでした。 
\\	理解してくれてありがとう。	りかいしてくれてありがとう。 
\\	被災者を救うための特別な措置が取られました。	ひさいしゃをすくうためのとくべつなそちがとられました。 
\\	その計画に対する反対の声が高まっているわ。	そのけいかくにたいするはんたいのこえがたかまっているわ。 
\\	最近、教育レベルが低下しています。	さいきん、きょういくレベルがていかしています。 
\\	この場合、保険が適用されますか。	このばあい、ほけんがてきようされますか。 
\\	女の子は話し方まで母親に似るわね。	おんなのこははなしかたまでははおやににるわね。 
\\	今の経済の動きを分析しています。	いまのけいざいのうごきをぶんせきしています。 
\\	彼は明日メキシコへ発ちます。	かれはあしたメキシコへたちます。 
\\	私は旅行から多くの知識を得た。	わたしはりょこうからおおくのちしきをえた。 
\\	知っている範囲で教えてください。	しっているはんいでおしえてください。 
\\	額に汗をかいたよ。	がくにあせをかいたよ。 
\\	夜中もテレビで台風の報道をしていた。	よなかもテレビでたいふうのほうどうをしていた。 
\\	労働者がストをしたそうだよ。	ろうどうしゃがストをしたそうだよ。 
\\	あの会社は目覚しく発展しているね。	あのかいしゃはめざましくはってんしているね。 
\\	テレビで日米野球をやっていますよ。	テレビでにちべいやきゅうをやっていますよ。 
\\	子供が老人の手を引いて歩いていたの。	こどもがろうじんのてをひいてあるいていたの。 
\\	自分のレベルに合った授業を選んでください。	じぶんのレベルにあったじゅぎょうをえらんでください。 
\\	教授はフランス文学の専門家。	きょうじゅはフランスぶんがくのせんもんか。 
\\	川の上流は流れが速いよ。	かわのじょうりゅうはながれがはやいよ。 
\\	新幹線の速度はどれくらいですか。	しんかんせんのそくどはどれくらいですか。 
\\	私は父を目標にしています。	わたしはちちをもくひょうにしています。 
\\	彼女の様子を見てきます。	かのじょのようすをみてきます。 
\\	彼は建築家というよりむしろ芸術家ね。	かれはけんちくかというよりむしろげいじゅつかね。 
\\	その仕事に多数の応募があったよ。	そのしごとにたすうのおうぼがあったよ。 
\\	警察は犯人を特定したらしいよ。	けいさつははんにんをとくていしたらしいよ。 
\\	その事件の直後、捜査本部が設置された。	そのじけんのちょくご、そうさほんぶがせっちされた。 
\\	彼は高校時代からの友人です。	かれはこうこうじだいからのゆうじんです。 
\\	状況はますます悪くなったね。	じょうきょうはますますわるくなったね。 
\\	その雑誌は年に4回発行されているんだ。	そのざっしはとしに4かいはっこうされているんだ。 
\\	旅行中は身の安全が第一です。	りょこうちゅうはみのあんぜんがだいいちです。 
\\	このビルは大地震に耐えられるでしょうか。	このビルはだいじしんにたえられるでしょうか。 
\\	半導体は様々な製品に使われているよ。	はんどうたいはさまざまなせいひんにつかわれているよ。 
\\	新しい内閣が誕生したね。	あたらしいないかくがたんじょうしたね。 
\\	その会社はいろいろな国に投資を分散しているね。	そのかいしゃはいろいろなくににとうしをぶんさんしているね。 
\\	私は哲学を学んでいます。	わたしはてつがくをまなんでいます。 
\\	この辺りの家賃の相場はいくらですか。	このあたりのやちんのそうばはいくらですか。 
\\	彼は素敵な男性です。	かれはすてきなだんせいです。 
\\	あいつが犯人です。	あいつがはんにんです。 
\\	ニュースでは標準語が使われるの。	ニュースではひょうじゅんごがつかわれるの。 
\\	彼はしっかりと責任を果たしました。	かれはしっかりとせきにんをはたしました。 
\\	彼は名の通った会社に就職しました。	かれはなのとおったかいしゃにしゅうしょくしました。 
\\	平均で一日に8時間ぐらい働いています。	へいきんでいちにちに8じかんぐらいはたらいています。 
\\	途中で30分ほど休みましょう。	とちゅうで30ふんほどやすみましょう。 
\\	彼はその問題を容易に解決したわ。	かれはそのもんだいをよういにかいけつしたわ。 
\\	先日会社から独立しました。	せんじつかいしゃからどくりつしました。 
\\	雨はだんだん強まります。	あめはだんだんつよまります。 
\\	課題は7月5日までに提出してください。	かだいは7つき5にちまでにていしゅつしてください。 
\\	彼はイギリスに派遣されました。	かれはイギリスにはけんされました。 
\\	彼女はこの小説の主人公のモデルです。	かのじょはこのしょうせつのしゅじんこうのモデルです。 
\\	その件は終わったと認識しています。	そのけんはおわったとにんしきしています。 
\\	サッカーは団体競技です。	サッカーはだんたいきょうぎです。 
\\	彼の父親は先生です。	かれのちちおやはせんせいです。 
\\	ここが話題のレストランです。	ここがわだいのレストランです。 
\\	イタリア旅行の費用は30万円です。	イタリアりょこうのひようは30まんえんです。 
\\	彼はこの国に平和をもたらしたのよ。	かれはこのくににへいわをもたらしたのよ。 
\\	私が司会を務めます。	わたしがしかいをつとめます。 
\\	彼は先生に反発していたよ。	かれはせんせいにはんぱつしていたよ。 
\\	彼はとても表情が豊かですね。	かれはとてもひょうじょうがゆたかですね。 
\\	私は彼のやり方には不満です。	わたしはかれのやりかたにはふまんです。 
\\	複数のレポートをひとつにまとめています。	ふくすうのレポートをひとつにまとめています。 
\\	その国では英語は通じますか。	そのくにではえいごはつうじますか。 
\\	まだ料金は払っていないけど。	まだりょうきんははらっていないけど。 
\\	車の部品を取り替えたんだ。	くるまのぶひんをとりかえたんだ。 
\\	針に糸を通してください。	はりにいとをとおしてください。 
\\	メートルは長さの単位です。	メートルはながさのたんいです。 
\\	その国の政府は各国に支援を要請したのよ。	そのくにのせいふはかっこくにしえんをようせいしたのよ。 
\\	前年に比べて利益が上がりました。	ぜんねんにくらべてりえきがあがりました。 
\\	あの男の態度にみんなあきれてたよ。	あのおとこのたいどにみんなあきれてたよ。 
\\	法律は私の専門です。	ほうりつはわたしのせんもんです。 
\\	応募者の年齢は問いません。	おうぼしゃのねんれいはといません。 
\\	入国手続きが終わりました。	にゅうこくてつづきがおわりました。 
\\	人はそれぞれ考え方が違います。	ひとはそれぞれかんがえかたがちがいます。 
\\	彼の家は台風の被害にあったの。	かれのいえはたいふうのひがいにあったの。 
\\	来週、妻の実家を訪問します。	らいしゅう、つまのじっかをほうもんします。 
\\	彼女の人生は苦労の連続でした。	かのじょのじんせいはくろうのれんぞくでした。 
\\	1ポンドは454グラムに相当します。	1ポンドは454グラムにそうとうします。 
\\	彼はトップの成績で合格しました。	かれはトップのせいせきでごうかくしました。 
\\	私の免許は来年まで有効です。	わたしのめんきょはらいねんまでゆうこうです。 
\\	彼は数学の博士だそうです。	かれはすうがくのはかせだそうです。 
\\	この商品はトラックで輸送します。	このしょうひんはトラックでゆそうします。 
\\	最近、日本は健康ブームです。	さいきん、にっぽんはけんこうブームです。 
\\	月の表面にはクレーターがたくさんあるね。	つきのひょうめんにはクレーターがたくさんあるね。 
\\	怪我を防ぐためによくストレッチをしてください。	けがをふせぐためによくストレッチをしてください。 
\\	この製品は有害な化学物質を含んでいるわよ。	このせいひんはゆうがいなかがくぶっしつをふくんでいるわよ。 
\\	ようやくゴールが見えてきました。	ようやくゴールがみえてきました。 
\\	昨日株価が暴落しました。	きのうかぶかがぼうらくしました。 
\\	彼はチームの一員です。	かれはチームのいちいんです。 
\\	今日、珍しい体験をしました。	きょう、めずらしいたいけんをしました。 
\\	通信技術の発達は目覚ましいな。	つうしんぎじゅつのはったつはめざましいな。 
\\	新型車に研究の成果が反映されています。	しんがたしゃにけんきゅうのせいかがはんえいされています。 
\\	チームの中では相互の信頼が大切です。	チームのなかではそうごのしんらいがたいせつです。 
\\	その公園は町のほぼ中央に位置するんだ。	そのこうえんはまちのほぼちゅうおうにいちするんだ。 
\\	今、スケジュールの調整をしています。	いま、スケジュールのちょうせいをしています。 
\\	電車の中では新聞を広げないで。	でんしゃのなかではしんぶんをひろげないで。 
\\	同じタイプで色の違うものはありますか。	おなじタイプでいろのちがうものはありますか。 
\\	判決が下された。	はんけつがくだされた。 
\\	火を強めてください。	ひをつよめてください。 
\\	彼の職場は都内にあります。	かれのしょくばはとないにあります。 
\\	このエアコンはあまり電力を使いません。	このエアコンはあまりでんりょくをつかいません。 
\\	彼は絵を本格的に勉強しているんだ。	かれはえをほんかくてきにべんきょうしているんだ。 
\\	試験の日程が発表されました。	しけんのにっていがはっぴょうされました。 
\\	彼はビジネスで大成功を収めたな。	かれはビジネスでだいせいこうをおさめたな。 
\\	彼、友達の意見を否定した。	かれ、ともだちのいけんをひていした。 
\\	母はお釣りを確かめたの。	はははおつりをたしかめたの。 
\\	私は朝早く投票を済ませました。	わたしはあさはやくとうひょうをすませました。 
\\	自然の破壊が進んでいるのよ。	しぜんのはかいがすすんでいるのよ。 
\\	最近、社内ネットワークの調子が悪い。	さいきん、しゃないネットワークのちょうしがわるい。 
\\	彼は大学で音楽の理論を勉強したんだよ。	かれはだいがくでおんがくのりろんをべんきょうしたんだよ。 
\\	今回は長期の滞在です。	こんかいはちょうきのたいざいです。 
\\	プロジェクトメンバーと話し合いをしたの。	プロジェクトメンバーとはなしあいをしたの。 
\\	彼の声はよく響きます。	かれのこえはよくひびきます。 
\\	これらの本を分類してください。	これらのほんをぶんるいしてください。 
\\	彼女はプレッシャーから解放されて本来の自分に戻ったな。	かのじょはプレッシャーからかいほうされてほんらいのじぶんにもどったな。 
\\	この服、私にはやや小さいみたい。	このふく、わたしにはややちいさいみたい。 
\\	文字の大きさは統一してください。	もじのおおきさはとういつしてください。 
\\	車に荷物を積んでください。	くるまににもつをつんでください。 
\\	実家を4年間離れていました。	じっかを4ねんかんはなれていました。 
\\	彼は突然走り出したのよ。	かれはとつぜんはしりだしたのよ。 
\\	今、経済の動向は読みにくいわ。	いま、けいざいのどうこうはよみにくいわ。 
\\	部長に仕事の相談をしたんだ。	ぶちょうにしごとのそうだんをしたんだ。 
\\	あの大学には派閥がたくさんあります。	あのだいがくにははばつがたくさんあります。 
\\	仕事が終わったら連絡します。	しごとがおわったられんらくします。 
\\	姉はよく旅をします。	あねはよくたびをします。 
\\	先月、娘が誕生しました。	せんげつ、むすめがたんじょうしました。 
\\	二人の走者は同時にゴールしたよ。	ににんのそうしゃはどうじにゴールしたよ。 
\\	あの村は伝統を保守している。	あのむらはでんとうをほしゅしている。 
\\	彼は常に姿勢がいい。	かれはつねにしせいがいい。 
\\	日本語の単語をいくつ知っていますか。	にほんごのたんごをいくつしっていますか。 
\\	選手は最高の状態で大会に臨んだわ。	せんしゅはさいこうのじょうたいでたいかいにのぞんだわ。 
\\	来月の総会は東京で行われます。	らいげつのそうかいはとうきょうでおこなわれます。 
\\	これは女性専用の車両です。	これはじょせいせんようのしゃりょうです。 
\\	この電車は地上を走ります。	このでんしゃはちじょうをはしります。 
\\	通信教育はとても便利です。	つうしんきょういくはとてもべんりです。 
\\	彼女はデモに参加したよ。	かのじょはデモにさんかしたよ。 
\\	彼は新聞記者を経て作家になりました。	かれはしんぶんきしゃをへてさっかになりました。 
\\	その物語は京都が舞台だ。	そのものがたりはきょうとがぶたいだ。 
\\	みんなで役割を分担しましょう。	みんなでやくわりをぶんたんしましょう。 
\\	その制度は廃止されました。	そのせいどははいしされました。 
\\	オーダーを追加しました。	オーダーをついかしました。 
\\	今年は比較的景気が好い。	ことしはひかくてきけいきがよい。 
\\	私は民族の歴史に興味があります。	わたしはみんぞくのれきしにきょうみがあります。 
\\	ワープロで報告書を打ったよ。	ワープロでほうこくしょをうったよ。 
\\	その爆撃で兵士が3人負傷しました。	そのばくげきでへいしが3にんふしょうしました。 
\\	彼女は料理の腕が素人の領域を超えているわ。	かのじょはりょうりのうでがしろうとのりょういきをこえているわ。 
\\	彼女は良い反応を示したわ。	かのじょはよいはんのうをしめしたわ。 
\\	グループのメンバーは10人です。	グループのメンバーは10にんです。 
\\	自分を高めるのは大切なことです。	じぶんをたかめるのはたいせつなことです。 
\\	中世の農民は貧しかった。	ちゅうせいののうみんはまずしかった。 
\\	この問題はハードウェアの故障が原因です。	このもんだいはハードウェアのこしょうがげんいんです。 
\\	以前の収入は今の倍はあった。	いぜんのしゅうにゅうはいまのばいはあった。 
\\	これから身長と体重を測定します。	これからしんちょうとたいじゅうをそくていします。 
\\	彼はよくマスコミにも登場するね。	かれはよくマスコミにもとうじょうするね。 
\\	彼は上司に怒りの目を向けたんだよ。	かれはじょうしにいかりのめをむけたんだよ。 
\\	日本は鉄道がとても発達しています。	にっぽんはてつどうがとてもはったつしています。 
\\	このメーカーの製品は高品質だわね。	このメーカーのせいひんはこうひんしつだわね。 
\\	私は日本の伝統を大切にします。	わたしはにっぽんのでんとうをたいせつにします。 
\\	彼女は髪にリボンを結んだね。	かのじょはかみにリボンをむすんだね。 
\\	これは機械の内部の問題です。	これはきかいのないぶのもんだいです。 
\\	新聞の読者の投書欄は好きです。	しんぶんのどくしゃのとうしょらんはすきです。 
\\	去年の売り上げと比較しましょう。	きょねんのうりあげとひかくしましょう。 
\\	列車がホームに入ってきた。	れっしゃがホームにはいってきた。 
\\	ダイナマイトが爆発したんだ。	ダイナマイトがばくはつしたんだ。 
\\	私たちはがんの要因を研究しています。	わたしたちはがんのよういんをけんきゅうしています。 
\\	その殺人事件の捜査は2年間続きました。	そのさつじんじけんのそうさは2ねんかんつづきました。 
\\	親子の対話は大切だよ。	おやこのたいわはたいせつだよ。 
\\	彼は今月の売上目標を達成した。	かれはこんげつのうりあげもくひょうをたっせいした。 
\\	今年はひどい水不足よ。	ことしはひどいみずぶそくよ。 
\\	その商品の宣伝をよく見かけます。	そのしょうひんのせんでんをよくみかけます。 
\\	生産コスト増大のため、値上げします。	せいさんコストぞうだいのため、ねあげします。 
\\	電子辞書はとても便利です。	でんしじしょはとてもべんりです。 
\\	彼女の年齢は27です。	かのじょのねんれいは27です。 
\\	私は料理人を目指しています。	わたしはりょうりじんをめざしています。 
\\	彼は最後まで戦ったよ。	かれはさいごまでたたかったよ。 
\\	彼は農家に生まれました。	かれはのうかにうまれました。 
\\	それは本人に聞いてください。	それはほんにんにきいてください。 
\\	最近甘いものを控えています。	さいきんあまいものをひかえています。 
\\	データを30分おきに保存してください。	データを30ふんおきにほぞんしてください。 
\\	やがて雪も止むでしょう。	やがてゆきもやむでしょう。 
\\	彼は旅行の直前に熱が出たの。	かれはりょこうのちょくぜんにねつがでたの。 
\\	彼は心が豊かな人です。	かれはこころがゆたかなひとです。 
\\	とても雰囲気のいいお店ですね。	とてもふんいきのいいおみせですね。 
\\	川に沿って歩いたんだ。	かわにそってあるいたんだ。 
\\	彼は有能な知事ね。	かれはゆうのうなちじね。 
\\	スミス夫人がいらっしゃいました。	スミスふじんがいらっしゃいました。 
\\	筆者の趣旨を考えてください。	ひっしゃのしゅしをかんがえてください。 
\\	この国の物価はここ10年でだいぶ上がったね。	このくにのぶっかはここ10ねんでだいぶあがったね。 
\\	走者が次々にゴールしました。	そうしゃがつぎつぎにゴールしました。 
\\	両者の意見を聞きましょう。	りょうしゃのいけんをききましょう。 
\\	彼女は若さを保とうと必死だ。	かのじょはわかさをたもとうとひっしだ。 
\\	犯人は警察に抵抗したの。	はんにんはけいさつにていこうしたの。 
\\	彼らは武器を取り、立ち上がった。	かれらはぶきをとり、たちあがった。 
\\	今、歯を治療しています。	いま、はをちりょうしています。 
\\	彼は道路に飛び出したの。	かれはどうろにとびだしたの。 
\\	その手術の成功率は90
\\	だそうです。	そのしゅじゅつのせいこうりつは90
\\	だそうです。 
\\	人に頼らないで、自分でやってごらん。	ひとにたよらないで、じぶんでやってごらん。 
\\	車から燃料がもれていますよ。	くるまからねんりょうがもれていますよ。 
\\	朝は家族全員が揃って食事します。	あさはかぞくぜんいんがそろってしょくじします。 
\\	遥か向こうに目的地が見えてきた。	はるかむこうにもくてきちがみえてきた。 
\\	彼はわずかな貯金で暮らしている。	かれはわずかなちょきんでくらしている。 
\\	彼は選挙に当選しました。	かれはせんきょにとうせんしました。 
\\	ここがいちばん面白い場面です。	ここがいちばんおもしろいばめんです。 
\\	仕事のやり方を教えてください。	しごとのやりかたをおしえてください。 
\\	心配しないで、単に眠いだけです。	しんぱいしないで、たんにねむいだけです。 
\\	予定が変更になりました。	よていがへんこうになりました。 
\\	車を塀に寄せたよ。	くるまをへいによせたよ。 
\\	彼女の行動パターンは興味深いな。	かのじょのこうどうパターンはきょうみぶかいな。 
\\	その問題についてクラスで討論しました。	そのもんだいについてクラスでとうろんしました。 
\\	ストレッチで筋肉を伸ばしましょう。	ストレッチできんにくをのばしましょう。 
\\	彼はついに弁護士の資格を取ったよ。	かれはついにべんごしのしかくをとったよ。 
\\	それはとても不思議な話だね。	それはとてもふしぎなはなしだね。 
\\	彼女には明確な目標があるね。	かのじょにはめいかくなもくひょうがあるね。 
\\	音楽は私の日常の一部です。	おんがくはわたしのにちじょうのいちぶです。 
\\	仏の教えを勉強しました。	ふつのおしえをべんきょうしました。 
\\	私たちのチームが優勝しました。	わたしたちのチームがゆうしょうしました。 
\\	枠の中に答えを書いてください。	わくのなかにこたえをかいてください。 
\\	漢字は中国から伝わった。	かんじはちゅうごくからつたわった。 
\\	チャンピオンがタイトルを防衛した。	チャンピオンがタイトルをぼうえいした。 
\\	資格があると就職に有利です。	しかくがあるとしゅうしょくにゆうりです。 
\\	二つの薬品を1対3の割合で混ぜたんだ。	ふたつのやくひんを1たい3のわりあいでまぜたんだ。 
\\	学校で勉強したことが役立った。	がっこうでべんきょうしたことがやくだった。 
\\	彼と私はいとこ同士です。	かれとわたしはいとこどうしです。 
\\	広場に子供が沢山集まっていたよ。	ひろばにこどもがたくさんあつまっていたよ。 
\\	彼はまるで子供のようにはしゃいだよ。	かれはまるでこどものようにはしゃいだよ。 
\\	その選手は怪我を乗り越えて復活したわね。	そのせんしゅはけがをのりこえてふっかつしたわね。 
\\	彼女は異文化に対する理解を深めたね。	かのじょはいぶんかにたいするりかいをふかめたね。 
\\	来月中旬にフランスへ行きます。	らいげつちゅうじゅんにフランスへいきます。 
\\	警察は犯罪の防止に努めています。	けいさつははんざいのぼうしにつとめています。 
\\	夜9時に東京に到着しました。	よる9じにとうきょうにとうちゃくしました。 
\\	業界は女性の比率が低い。	
\\	ぎょうかいはじょせいのひりつがひくい。 
\\	その二つの国に摩擦が生じたんだ。	そのふたつのくににまさつがしょうじたんだ。 
\\	彼は山道をどんどん進んで行ったの。	かれはさんどうをどんどんすすんでいったの。 
\\	事件現場付近で怪しい人物を見た。	じけんげんばふきんであやしいじんぶつをみた。 
\\	この土地の面積はどれ位ですか。	このとちのめんせきはどれいですか。 
\\	祖母がうちの猫を抱いているわよ。	そぼがうちのねこをいだいているわよ。 
\\	彼がそれほど悩んでいたとは知らなかった。	かれがそれほどなやんでいたとはしらなかった。 
\\	みんなで環境を保護しましょう。	みんなでかんきょうをほごしましょう。 
\\	彼は特殊な能力を持っています。	かれはとくしゅなのうりょくをもっています。 
\\	何か質問はありますか。	なにかしつもんはありますか。 
\\	彼は動作が機敏です。	かれはどうさがきびんです。 
\\	その学校の教育は特色がありますね。	そのがっこうのきょういくはとくしょくがありますね。 
\\	機内に荷物を持ち込んだの。	きないににもつをもちこんだの。 
\\	数学は論理に基づく学問です。	すうがくはろんりにもとづくがくもんです。 
\\	今日は流通の仕組みを勉強しましょう。	きょうはりゅうつうのしくみをべんきょうしましょう。 
\\	彼は会社で微妙な立場にあります。	かれはかいしゃでびみょうなたちばにあります。 
\\	無理はしないでください。	むりはしないでください。 
\\	彼女は水玉模様のスカートをはいているね。	かのじょはみずたまもようのスカートをはいているね。 
\\	この道具の用途を説明します。	このどうぐのようとをせつめいします。 
\\	この都市は工業地帯です。	このとしはこうぎょうちたいです。 
\\	今日は授業を1時間に短縮します。	きょうはじゅぎょうを1じかんにたんしゅくします。 
\\	雨で運動会が中止になったの。	あめでうんどうかいがちゅうしになったの。 
\\	危ないので子供からライターを取り上げました。	あぶないのでこどもからライターをとりあげました。 
\\	私と彼は同じ年代です。	わたしとかれはおなじねんだいです。 
\\	私は今月一杯で退職します。	わたしはこんげついちはいでたいしょくします。 
\\	不況の影響で仕事が少ないね。	ふきょうのえいきょうでしごとがすくないね。 
\\	日本では北海道に滞在しました。	にっぽんではほっかいどうにたいざいしました。 
\\	彼は都心に住んでいます。	かれはとしんにすんでいます。 
\\	父は弁護士です。	ちちはべんごしです。 
\\	彼は私との結婚を望んでいます。	かれはわたしとのけっこんをのぞんでいます。 
\\	彼は結果に満足したようです。	かれはけっかにまんぞくしたようです。 
\\	もはや彼の助けは必要じゃないの。	もはやかれのたすけはひつようじゃないの。 
\\	作業を始める態勢は整っています。	さぎょうをはじめるたいせいはととのっています。 
\\	親が君を一度連れて来いと言うんだ。	おやがきみをいちどつれてこいというんだ。 
\\	皆で暖かく見守りましょう。	みなであたたかくみまもりましょう。 
\\	世の中にはいろいろな人がいます。	よのなかにはいろいろなひとがいます。 
\\	相手チームのリードで前半が終わったよ。	あいてチームのリードでぜんはんがおわったよ。 
\\	彼女の明るさに魅力を感じました。	かのじょのあかるさにみりょくをかんじました。 
\\	直ちに会社に戻ってください。	ただちにかいしゃにもどってください。 
\\	この服のデザインは素敵ですね。	このふくのデザインはすてきですね。 
\\	心理学の授業に登録しましたか。	しんりがくのじゅぎょうにとうろくしましたか。 
\\	彼の説明で納得できました。	かれのせつめいでなっとくできました。 
\\	彼女は4年間熱心に勉強したわ。	かのじょは4ねんかんねっしんにべんきょうしたわ。 
\\	彼は大のサッカーファンです。	かれはだいのサッカーファンです。 
\\	お申し込みなさるのでしたら、こちらにお並びください。	おもうしこみなさるのでしたら、こちらにおならびください。 
\\	野菜が値上がりしていますね。	やさいがねあがりしていますね。 
\\	東西に大きな道路が通っています。	とうざいにおおきなどうろがかよっています。 
\\	レポートを1日で仕上げるのは不可能です。	レポートを1にちでしあげるのはふかのうです。 
\\	島で火山が噴火したよ。	しまでかざんがふんかしたよ。 
\\	彼は成功するに違いないわ。	かれはせいこうするにちがいないわ。 
\\	彼は新しいことに挑戦している。	かれはあたらしいことにちょうせんしている。 
\\	彼は幅広い知識を持っています。	かれははばひろいちしきをもっています。 
\\	全員の意見を総合してみましょう。	ぜんいんのいけんをそうごうしてみましょう。 
\\	彼は物理学の博士です。	かれはぶつりがくのはかせです。 
\\	謎の物体が空を飛んでいます。	なぞのぶったいがそらをとんでいます。 
\\	彼女は会社で高い地位に就いているよ。	かのじょはかいしゃでたかいちいについているよ。 
\\	私の会社で社員を募集しています。	わたしのかいしゃでしゃいんをぼしゅうしています。 
\\	土を掘って木を植えました。	どをほってきをうえました。 
\\	彼は十分反省しています。	かれはじゅうぶんはんせいしています。 
\\	今日は英語の文法を勉強します。	きょうはえいごのぶんぽうをべんきょうします。 
\\	バス代が値上げされました。	バスだいがねあげされました。 
\\	彼は昔、ドイツの大使でした。	かれはむかし、ドイツのたいしでした。 
\\	彼女は都の職員です。	かのじょはとのしょくいんです。 
\\	彼らは1日7時間労働している。	かれらは1にち7じかんろうどうしている。 
\\	コップの底が濡れていますよ。	コップのそこがぬれていますよ。 
\\	彼女は涙を流したんだ。	かのじょはなみだをながしたんだ。 
\\	夜は10時までに寝るのが望ましいの。	よるは10じまでにねるのがのぞましいの。 
\\	国民は総理大臣を非難しているね。	こくみんはそうりだいじんをひなんしているね。 
\\	みんなの意見がまとまりました。	みんなのいけんがまとまりました。 
\\	彼はグループから抜けました。	かれはグループからぬけました。 
\\	姉は福祉の勉強をしています。	あねはふくしのべんきょうをしています。 
\\	彼は貧しい家庭で育ったんだ。	かれはまずしいかていでそだったんだ。 
\\	あの都市はオリンピックの有力な候補地です。	あのとしはオリンピックのゆうりょくなこうほちです。 
\\	最後まで共にがんばりましょう。	さいごまでともにがんばりましょう。 
\\	私は農村で育ちました。	わたしはのうそんでそだちました。 
\\	試験で不正が見つかったよ。	しけんでふせいがみつかったよ。 
\\	この製品の保証期間は5年間となっております。	このせいひんのほしょうきかんは5ねんかんとなっております。 
\\	このゲームのルールは簡単です。	このゲームのルールはかんたんです。 
\\	彼は長年の友人です。	かれはながねんのゆうじんです。 
\\	彼のことを見直しました。	かれのことをみなおしました。 
\\	彼は父親の店を継いだの。	かれはちちおやのみせをついだの。 
\\	「さくら」という言葉を使って俳句を書いたよ。	"「さくら」ということばをつかってはいくをかいたよ。 
\\	事故現場の上空をヘリコプターが飛んでいます。	じこげんばのじょうくうをヘリコプターがとんでいます。 
\\	人々は喜びに沸いた。	ひとびとはよろこびにわいた。 
\\	非常事態です。	ひじょうじたいです。 
\\	あの国は強力な兵器を持っている。	あのくにはきょうりょくなへいきをもっている。 
\\	彼はトップに次ぐ好成績でした。	かれはトップにつぐこうせいせきでした。 
\\	私は山の風景が好きです。	わたしはやまのふうけいがすきです。 
\\	日本人は先輩、後輩の関係をとても大切に考えます。	にっぽんじんはせんぱい、こうはいのかんけいをとてもたいせつにかんがえます。 
\\	彼女は文学に興味を持っているのよ。	かのじょはぶんがくにきょうみをもっているのよ。 
\\	良い成績が取れるように努めます。	よいせいせきがとれるようにつとめます。 
\\	ともかく一度考え直しましょう。	ともかくいちどかんがえなおしましょう。 
\\	翌日、彼に会いに行きました。	よくじつ、かれにあいにいきました。 
\\	かばんの中身を見せてください。	かばんのなかみをみせてください。 
\\	疲れて肩が張っています。	つかれてかたがはっています。 
\\	評判の良いレストランに行きました。	ひょうばんのよいレストランにいきました。 
\\	会議中、ついあくびをしてしまいました。	かいぎちゅう、ついあくびをしてしまいました。 
\\	新しい仕事を引き受けたよ。	あたらしいしごとをひきうけたよ。 
\\	危ないので避難してください。	あぶないのでひなんしてください。 
\\	お振込みは3番の窓口です。	おふりこみは3ばんのまどぐちです。 
\\	私の趣味は読書です。	わたしのしゅみはどくしょです。 
\\	敗戦の原因は何だろう。	はいせんのげんいんはなにだろう。 
\\	3月に大学を卒業の見込みです。	3つきにだいがくをそつぎょうのみこみです。 
\\	彼女は臨時の店員です。	かのじょはりんじのてんいんです。 
\\	交渉が成立するという前提で話を進めます。	こうしょうがせいりつするというぜんていではなしをすすめます。 
\\	このズボンは上着と対になっています。	このズボンはうわぎとたいになっています。 
\\	この人とその人は、同一人物ですか。	このひととそのひとは、どういちじんぶつですか。 
\\	トンネルを抜けると海が見えたよ。	トンネルをぬけるとうみがみえたよ。 
\\	部下が素晴らしい働きをしたな。	ぶかがすばらしいはたらきをしたな。 
\\	彼は受験のことで悩んでいます。	かれはじゅけんのことでなやんでいます。 
\\	彼は大きな悩みを抱えていました。	かれはおおきななやみをかかえていました。 
\\	彼は応募の条件を満たしていない。	かれはおうぼのじょうけんをみたしていない。 
\\	先頭の人はプラカードを持ってください。	せんとうのひとはプラカードをもってください。 
\\	この荷物を倉庫に運びましょう。	このにもつをそうこにはこびましょう。 
\\	彼は都会での生活を楽しんでいるの。	かれはとかいでのせいかつをたのしんでいるの。 
\\	彼はスピーチの前に服装を整えた。	かれはスピーチのまえにふくそうをととのえた。 
\\	彼は30代の半ばです。	かれは30だいのなかばです。 
\\	栄養バランスの良い食事をしよう。	えいようバランスのよいしょくじをしよう。 
\\	出発まで時間の余裕があります。	しゅっぱつまでじかんのよゆうがあります。 
\\	この円は直径6センチです。	このえんはちょっけい6センチです。 
\\	明日は高校の入試です。	あしたはこうこうのにゅうしです。 
\\	この場でお礼を言わせてください。	このばでおれいをいわせてください。 
\\	弟は美術を専攻しています。	おとうとはびじゅつをせんこうしています。 
\\	ユーラシアは世界で最も大きい大陸です。	ユーラシアはせかいでもっともおおきいたいりくです。 
\\	どうしてもその訳を知りたい。	どうしてもそのわけをしりたい。 
\\	私は早朝のジョギングを日課にしています。	わたしはそうちょうのジョギングをにっかにしています。 
\\	長い戦いが終わった。	ながいたたかいがおわった。 
\\	彼は買収されたらしいわ。	かれはばいしゅうされたらしいわ。 
\\	どうかお許しください。	どうかおゆるしください。 
\\	あなたは保険に加入していますか。	あなたはほけんにかにゅうしていますか。 
\\	みんなで製品の名称を考えました。	みんなでせいひんのめいしょうをかんがえました。 
\\	彼はいつも黙って仕事をします。	かれはいつもだまってしごとをします。 
\\	お客様の要望を聞かせてください。	おきゃくさまのようぼうをきかせてください。 
\\	彼はスケートのプロです。	かれはスケートのプロです。 
\\	話を戻しましょう。	はなしをもどしましょう。 
\\	彼が全面的に協力してくれるそうです。	かれがぜんめんてきにきょうりょくしてくれるそうです。 
\\	その国は経済大国よ。	そのくにはけいざいたいこくよ。 
\\	反論がある人はどうぞ。	はんろんがあるひとはどうぞ。 
\\	昨日の夜、社長ご夫妻と食事をしました。	きのうのよる、しゃちょうごふさいとしょくじをしました。 
\\	被災地に物資を送ったよ。	ひさいちにぶっしをおくったよ。 
\\	今日は歴史的な日です。	きょうはれきしてきなひです。 
\\	これは昨年の売り上げの統計です。	これはさくねんのうりあげのとうけいです。 
\\	彼は昨年入社したの。	かれはさくねんにゅうしゃしたの。 
\\	これは素晴らしいチャンスだ。	これはすばらしいチャンスだ。 
\\	彼は適切な言葉で説明してくれました。	かれはてきせつなことばでせつめいしてくれました。 
\\	南北に山が広がっています。	なんぼくにやまがひろがっています。 
\\	会社が二つの派閥に分裂しているんだ。	かいしゃがふたつのはばつにぶんれつしているんだ。 
\\	旅行は父の老後の楽しみです。	りょこうはちちのろうごのたのしみです。 
\\	先生は私たちを導いてくれます。	せんせいはわたしたちをみちびいてくれます。 
\\	仕事の残りは家でします。	しごとののこりはいえでします。 
\\	彼は偉大な発明王です。	かれはいだいなはつめいおうです。 
\\	このビルは完成までに2年を要した。	このビルはかんせいまでに2ねんをようした。 
\\	日中はずっと海で泳いでいました。	にちちゅうはずっとうみでおよいでいました。 
\\	日中貿易は急激に伸びているわね。	にちちゅうぼうえきはきゅうげきにのびているわね。 
\\	今日はどうも体の調子が悪い。	きょうはどうもからだのちょうしがわるい。 
\\	先生の論文が雑誌に載りましたよ。	せんせいのろんぶんがざっしにのりましたよ。 
\\	最近犯罪が減っています。	さいきんはんざいがへっています。 
\\	霧が晴れて視界が開けたね。	きりがはれてしかいがあけたね。 
\\	彼はとても優秀な生徒です。	かれはとてもゆうしゅうなせいとです。 
\\	これが建物の側面の写真です。	これがたてもののそくめんのしゃしんです。 
\\	会社が倒産しました。	かいしゃがとうさんしました。 
\\	最近、仕事がかなりハードです。	さいきん、しごとがかなりハードです。 
\\	彼は来年定年を迎える。	かれはらいねんていねんをむかえる。 
\\	私は留学生として日本に入国しました。	わたしはりゅうがくせいとしてにっぽんににゅうこくしました。 
\\	外の温度はマイナス3度です。	そとのおんどはマイナス3どです。 
\\	とにかく現場へ行ってみましょう。	とにかくげんばへいってみましょう。 
\\	婦人服売り場は5階でございます。	ふじんふくうりばは5かいでございます。 
\\	容器のふたはきちんと閉めましょう。	ようきのふたはきちんとしめましょう。 
\\	若者は体力がありますね。	わかものはたいりょくがありますね。 
\\	面白い出来事がありました。	おもしろいできごとがありました。 
\\	この薬はまだ認可されていません。	このくすりはまだにんかされていません。 
\\	祖父の墓は近くにあります。	そふのはかはちかくにあります。 
\\	論文を仕上げないと卒業できませんよ。	ろんぶんをしあげないとそつぎょうできませんよ。 
\\	彼は新しいアイデアをすぐ取り入れます。	かれはあたらしいアイデアをすぐとりいれます。 
\\	私があなたの安全を保障します。	わたしがあなたのあんぜんをほしょうします。 
\\	お飲み物は無料でございます。	おのみものはむりょうでございます。 
\\	展覧会は連日賑わいました。	てんらんかいはれんじつにぎわいました。 
\\	能は元々大衆の文化でした。	のうはもともとたいしゅうのぶんかでした。 
\\	この問題を解けますか。	このもんだいをとけますか。 
\\	このパイプは詰まっていますよ。	このパイプはつまっていますよ。 
\\	夜間は裏口から入ってください。	やかんはうらぐちからはいってください。 
\\	そんなことは想像できないよ。	そんなことはそうぞうできないよ。 
\\	空がたちまち曇ってきたね。	そらがたちまちくもってきたね。 
\\	湖に鳥の群れがいたよ。	みずうみにとりのむれがいたよ。 
\\	当店ではお酒を取り扱っておりません。	とうてんではおさけをとりあつかっておりません。 
\\	この島は観光で成り立っています。	このしまはかんこうでなりたっています。 
\\	年末のセールはいつも込んでいるね。	ねんまつのセールはいつもこんでいるね。 
\\	あの国は経済的に繁栄しているわね。	あのくにはけいざいてきにはんえいしているわね。 
\\	彼はメガネを外しました。	かれはメガネをはずしました。 
\\	その事故で多くの人が負傷したの。	そのじこでおおくのひとがふしょうしたの。 
\\	全力で走れ。	ぜんりょくではしれ。 
\\	この2社が市場を独占しています。	この2しゃがしじょうをどくせんしています。 
\\	取り敢えずお知らせしておきます。	とりあえずおしらせしておきます。 
\\	私にとって動物は身近な存在です。	わたしにとってどうぶつはみぢかなそんざいです。 
\\	例外は認めません。	れいがいはみとめません。 
\\	道具は全部揃っていますか。	どうぐはぜんぶそろっていますか。 
\\	彼は植物全般に詳しいね。	かれはしょくぶつぜんぱんにくわしいね。 
\\	勤務の体系を見直すことにしました。	きんむのたいけいをみなおすことにしました。 
\\	祖母は大正生まれです。	そぼはたいしょううまれです。 
\\	彼はどんな環境にも適応できます。	かれはどんなかんきょうにもてきおうできます。 
\\	私は雑誌の編集の仕事をしています。	わたしはざっしのへんしゅうのしごとをしています。 
\\	ここで代金をお支払いください。	ここでだいきんをおしはらいください。 
\\	彼は単純な人です。	かれはたんじゅんなひとです。 
\\	沈んだ船を引き上げたんだ。	しずんだふねをひきあげたんだ。 
\\	私の秘書はとても優秀です。	わたしのひしょはとてもゆうしゅうです。 
\\	学生たちは授業についていくのに必死です。	がくせいたちはじゅぎょうについていくのにひっしです。 
\\	そのパーティーはカジュアルな服装で大丈夫です。	そのパーティーはカジュアルなふくそうでだいじょうぶです。 
\\	ついに山頂に到達しました。	ついにさんちょうにとうたつしました。 
\\	パーティーの準備が整いました。	パーティーのじゅんびがととのいました。 
\\	その町に爆弾が落とされたの。	そのまちにばくだんがおとされたの。 
\\	彼の言っていることは矛盾しています。	かれのいっていることはむじゅんしています。 
\\	緑を取り戻す必要があります。	みどりをとりもどすひつようがあります。 
\\	ワインのコルクを抜きました。	ワインのコルクをぬきました。 
\\	私は株の知識がないに等しいです。	わたしはかぶのちしきがないにひとしいです。 
\\	人手が足りなくて忙しい。	ひとでがたりなくていそがしい。 
\\	この経験はあなたにとってプラスになるでしょう。	このけいけんはあなたにとってプラスになるでしょう。 
\\	彼は彼女をリードしながら踊ったね。	かれはかのじょをリードしながらおどったね。 
\\	このソフトには多少問題がある。	このソフトにはたしょうもんだいがある。 
\\	体の調子がとても良いです。	からだのちょうしがとてもよいです。 
\\	猫が窓から外を眺めているよ。	ねこがまどからそとをながめているよ。 
\\	棒の先端を持って下さい。	ぼうのせんたんをもってください。 
\\	たとえ、嵐になっても絶対に行く。	たとえ、あらしになってもぜったいにいく。 
\\	彼女は典型的なイタリア人です。	かのじょはてんけいてきなイタリアじんです。 
\\	なるべく早く仕事を終わらせてください。	なるべくはやくしごとをおわらせてください。 
\\	機械を分解してみたの。	きかいをぶんかいしてみたの。 
\\	役所で住民票をもらってきた。	やくしょでじゅうみんひょうをもらってきた。 
\\	彼は非暴力を説きました。	かれはひぼうりょくをときました。 
\\	最近、体調が少し不安定です。	さいきん、たいちょうがすこしふあんていです。 
\\	叔父は役人として30年働きました。	おじはやくにんとして30ねんはたらきました。 
\\	降りてください、定員オーバーです。	おりてください、ていいんオーバーです。 
\\	腹が減って動けない。	はらがへってうごけない。 
\\	友達と百貨店で買い物をしました。	ともだちとひゃっかてんでかいものをしました。 
\\	ここに穴を掘りましょう。	ここにあなをほりましょう。 
\\	彼は急に立ち上がったの。	かれはきゅうにたちあがったの。 
\\	飛行機の墜落事故があったんだ。	ひこうきのついらくじこがあったんだ。 
\\	このドアの高さを測ってください。	このドアのたかさをはかってください。 
\\	虫歯は予防できます。	むしばはよぼうできます。 
\\	姉が嫁に行ったの。	あねがよめにいったの。 
\\	入場券は当日でも買えますよ。	にゅうじょうけんはとうじつでもかえますよ。 
\\	私に何か用ですか。	わたしになにかようですか。 
\\	この用紙に名前を書いてください。	このようしになまえをかいてください。 
\\	講演のテーマは何ですか。	こうえんのテーマはなにですか。 
\\	彼女はまだ仕事の経験が乏しい。	かのじょはまだしごとのけいけんがとぼしい。 
\\	先生が私の能力を引き出してくれました。	せんせいがわたしののうりょくをひきだしてくれました。 
\\	これは本物のダイヤモンドです。	これはほんもののダイヤモンドです。 
\\	今日、スポーツジムの申し込みをしました。	きょう、スポーツジムのもうしこみをしました。 
\\	彼は強敵を見事に破ったね。	かれはきょうてきをみごとにやぶったね。 
\\	私はまだ両親と同居しています。	わたしはまだりょうしんとどうきょしています。 
\\	彼はこの仕事に適しています。	かれはこのしごとにてきしています。 
\\	あの駅で学生がたくさん乗り込んだね。	あのえきでがくせいがたくさんのりこんだね。 
\\	部屋を暗くしてムードを出してみたよ。	へやをくらくしてムードをだしてみたよ。 
\\	運転手が無線で話していたの。	うんてんしゅがむせんではなしていたの。 
\\	飛行機が無事着陸したわね。	ひこうきがぶじちゃくりくしたわね。 
\\	定期演奏会は年に4回あります。	ていきえんそうかいはとしに4かいあります。 
\\	その質問の適当な答えが見つかりません。	そのしつもんのてきとうなこたえがみつかりません。 
\\	兵隊が銃をかまえていたよ。	へいたいがじゅうをかまえていたよ。 
\\	彼は会社の寮に住んでいます。	かれはかいしゃのりょうにすんでいます。 
\\	冷静に話し合いましょう。	れいせいにはなしあいましょう。 
\\	その車は中古で買ったんだ。	そのくるまはちゅうこでかったんだ。 
\\	彼女は中世の音楽が好きです。	かのじょはちゅうせいのおんがくがすきです。 
\\	停電のため仕事を中断しました。	ていでんのためしごとをちゅうだんしました。 
\\	そこをどいてください。	そこをどいてください。 
\\	彼女は流行に敏感だね。	かのじょはりゅうこうにびんかんだね。 
\\	私の家は川に面しています。	わたしのいえはかわにめんしています。 
\\	そのふたつの町の中間には川が流れているの。	そのふたつのまちのちゅうかんにはかわがながれているの。 
\\	彼は国外に追放されました。	かれはこくがいについほうされました。 
\\	この会社は急成長を遂げました。	このかいしゃはきゅうせいちょうをとげました。 
\\	彼はイタリアのマーケットを開拓しました。	かれはイタリアのマーケットをかいたくしました。 
\\	彼は高い理想を持っているの。	かれはたかいりそうをもっているの。 
\\	他方の視点からも見てみましょう。	たほうのしてんからもみてみましょう。 
\\	私の弟は団地に住んでいます。	わたしのおとうとはだんちにすんでいます。 
\\	果たして彼は現れるだろうか。	はたしてかれはあらわれるだろうか。 
\\	私たち夫婦は家事を分担しています。	わたしたちふうふはかじをぶんたんしています。 
\\	会社で面倒なことが起こったの。	かいしゃでめんどうなことがおこったの。 
\\	かばんに荷物を詰めました。	かばんににもつをつめました。 
\\	彼女はファッションの専門家だ。	かのじょはファッションのせんもんかだ。 
\\	桜のつぼみが膨らんだのね。	さくらのつぼみがふくらんだのね。 
\\	彼は不自由な暮らしに慣れたようだね。	かれはふじゆうなくらしになれたようだね。 
\\	彼の翻訳は分かりやすいですよ。	かれのほんやくはわかりやすいですよ。 
\\	当日は天気に恵まれました。	とうじつはてんきにめぐまれました。 
\\	新しい社員を雇いました。	あたらしいしゃいんをやといました。 
\\	夏休みには家族で登山をします。	なつやすみにはかぞくでとざんをします。 
\\	努力と結果が比例していないの。	どりょくとけっかがひれいしていないの。 
\\	その程度の努力では不十分です。	そのていどのどりょくではふじゅうぶんです。 
\\	赤信号でブレーキを踏んだの。	あかしんごうでブレーキをふんだの。 
\\	ミスは誰にでもあります。	ミスはだれにでもあります。 
\\	ヨットでクルージングを楽しみましたわ。	ヨットでクルージングをたのしみましたわ。 
\\	何とかお願いします。	なんとかおねがいします。 
\\	二人の愛はますます深まっているね。	ににんのあいはますますふかまっているね。 
\\	彼は見事なジャンプを見せたね。	かれはみごとなジャンプをみせたね。 
\\	彼は論理的な人です。	かれはろんりてきなひとです。 
\\	彼は俳優です。	かれははいゆうです。 
\\	ここでは太陽エネルギーを使って発電しています。	ここではたいようエネルギーをつかってはつでんしています。 
\\	その病気は原因不明と言われているんだよ。	そのびょうきはげんいんふめいといわれているんだよ。 
\\	京都から中継で放送しています。	きょうとからちゅうけいでほうそうしています。 
\\	不要になったパソコンを処分したんだ。	ふようになったパソコンをしょぶんしたんだ。 
\\	学生時代を懐かしく振り返ったんだ。	がくせいじだいをなつかしくふりかえったんだ。 
\\	多くの文明は川の近くで始まった。	おおくのぶんめいはかわのちかくではじまった。 
\\	花瓶の向きを変えたの。	かびんのむきをかえたの。 
\\	祖父は明治の生まれです。	そふはめいじのうまれです。 
\\	その国の社会秩序は乱れているわ。	そのくにのしゃかいちつじょはみだれているわ。 
\\	この道は通行できません。	このみちはつうこうできません。 
\\	彼は優秀な部下を持っているわね。	かれはゆうしゅうなぶかをもっているわね。 
\\	雪道で派手に転んでしまったの。	ゆきどうではでにころんでしまったの。 
\\	冬は皮膚が乾燥しますね。	ふゆはひふがかんそうしますね。 
\\	私は毎朝体重を測っています。	わたしはまいあさたいじゅうをはかっています。 
\\	明日から2週間、短期のアルバイトをします。	あしたから2しゅうかん、たんきのアルバイトをします。 
\\	もうすぐ目標の地点に到達します。	もうすぐもくひょうのちてんにとうたつします。 
\\	これはとても手間のかかる料理です。	これはとてもてまのかかるりょうりです。 
\\	彼女は絵がなかなか上手です。	かのじょはえがなかなかじょうずです。 
\\	よく見たら、時計の針が12時を指してるよ。	よくみたら、とけいのはりが12じをさしてるよ。 
\\	彼女には日ごろからお世話になっています。	かのじょにはひごろからおせわになっています。 
\\	みんな彼の温かい人柄が大好きだよ。	みんなかれのあたたかいひとがらがだいすきだよ。 
\\	これが参加者の名簿です。	これがさんかしゃのめいぼです。 
\\	私のチームは1回戦で敗れたよ。	わたしのチームは1かいせんでやぶれたよ。 
\\	道の両側にたくさんお店がありますよ。	みちのりょうがわにたくさんおみせがありますよ。 
\\	俺はロックを聞くのが好きだ。	おれはロックをきくのがすきだ。 
\\	ガソリンが値下がりしました。	ガソリンがねさがりしました。 
\\	私たちの旅は始めはよかったんだ。	わたしたちのたびははじめはよかったんだ。 
\\	私が一番好きな宝石はダイヤモンドなの。	わたしがいちばんすきなほうせきはダイヤモンドなの。 
\\	沖縄方面にお出かけの方は台風にご注意ください。	おきなわほうめんにおでかけのほうはたいふうにごちゅういください。 
\\	わざわざ来てくれてありがとう。	わざわざきてくれてありがとう。 
\\	ここはいやな臭いがする。	ここはいやなにおいがする。 
\\	急に出張を命じられました。	きゅうにしゅっちょうをめいじられました。 
\\	彼は県庁で働いています。	かれはけんちょうではたらいています。 
\\	この本の著者はイギリス人です。	このほんのちょしゃはイギリスじんです。 
\\	彼女は肌がとても白いですね。	かのじょははだがとてもしろいですね。 
\\	その噂はすぐに広まったよ。	そのうわさはすぐにひろまったよ。 
\\	お客さんが続々とやって来ましたよ。	おきゃくさんがぞくぞくとやってきましたよ。 
\\	今年から新しい仕事に就きます。	ことしからあたらしいしごとにつきます。 
\\	台風で停電したよ。	たいふうでていでんしたよ。 
\\	彼はパイロットです。	かれはパイロットです。 
\\	その仕事の報酬として20万円もらいました。	そのしごとのほうしゅうとして20まんえんもらいました。 
\\	彼女はずいぶん大胆なことを言うね。	かのじょはずいぶんだいたんなことをいうね。 
\\	彼はすごく知的な人です。	かれはすごくちてきなひとです。 
\\	彼の家は人の出入りが多いね。	かれのいえはひとのでいりがおおいね。 
\\	明日はどうやら雨らしいよ。	あしたはどうやらあめらしいよ。 
\\	ふと昔の事を思い出したんだ。	ふとむかしのことをおもいだしたんだ。 
\\	彼はベテランの運転手です。	かれはベテランのうんてんしゅです。 
\\	今年はボーナスがたくさん出ました。	ことしはボーナスがたくさんでました。 
\\	月が満ちてきましたね。	つきがみちてきましたね。 
\\	この自転車の持ち主は誰ですか。	このじてんしゃのもちぬしはだれですか。 
\\	子猫が線路に迷い込んだぞ。	こねこがせんろにまよいこんだぞ。 
\\	あの親子はそっくりだね。	あのおやこはそっくりだね。 
\\	彼の人生は変化に富んでいるな。	かれのじんせいはへんかにとんでいるな。 
\\	テニスコートの半面を使って試合をしたの。	テニスコートのはんめんをつかってしあいをしたの。 
\\	この地区は放射能に汚染されたんだ。	このちくはほうしゃのうにおせんされたんだ。 
\\	暴力はいけません。	ぼうりょくはいけません。 
\\	我々の決意は固いです。	われわれのけついはかたいです。 
\\	この車の動力は電気です。	このくるまのどうりょくはでんきです。 
\\	お坊さんが鐘を鳴らしていますね。	おぼうさんがかねをならしていますね。 
\\	南極でペンギンを見ました。	なんきょくでペンギンをみました。 
\\	彼は物理を専攻している。	かれはぶつりをせんこうしている。 
\\	今日は絶えず電話が鳴りました。	きょうはたえずでんわがなりました。 
\\	彼の服のセンスは独特よね。	かれのふくのセンスはどくとくよね。 
\\	彼は皮肉ばかり言います。	かれはひにくばかりいいます。 
\\	ポケットに財布を突っ込んじゃった。	ポケットにさいふをつっこんじゃった。 
\\	世界の平和が私たちの願いです。	せかいのへいわがわたしたちのねがいです。 
\\	日本人がノーベル賞を取ったよ。	にっぽんじんがノーベルしょうをとったよ。 
\\	あの先生は生徒をみな平等に扱います。	あのせんせいはせいとをみなびょうどうにあつかいます。 
\\	眼鏡のレンズを替えました。	めがねのレンズをかえました。 
\\	部屋の中を徹底的に探しました。	へやのなかをてっていてきにさがしました。 
\\	叔父は牧場を持っているんだ。	おじはぼくじょうをもっているんだ。 
\\	昨日友人を見舞ったの。	きのうゆうじんをみまったの。 
\\	静かな音楽で眠気を催したよ。	しずかなおんがくでねむけをもよおしたよ。 
\\	この絵は赤と黒の対比が美しいですね。	このえはあかとくろのたいひがうつくしいですね。 
\\	最近疲れがたまっています。	さいきんつかれがたまっています。 
\\	昨日、絵の展覧会に行ってきました。	きのう、えのてんらんかいにいってきました。 
\\	この単語を発音してください。	このたんごをはつおんしてください。 
\\	その映画はよい批評を得ているんだ。	そのえいがはよいひひょうをえているんだ。 
\\	パチンコで5000円損しました。	パチンコで5000えんそんしました。 
\\	応募作品の大小は問いません。	おうぼさくひんのだいしょうはといません。 
\\	彼女はついに捕まりました。	かのじょはついにつかまりました。 
\\	嘆いていても何も変わりません。	なげいていてもなにもかわりません。 
\\	やっと運転免許を手に入れたよ。	やっとうんてんめんきょをてにいれたよ。 
\\	応援団が試合を盛り上げたんだ。	おうえんだんがしあいをもりあげたんだ。 
\\	彼はタイムを1秒縮めたの。	かれはタイムを1びょうちぢめたの。 
\\	このソフトには色々なツールが付属しています。	このソフトにはいろいろなツールがふぞくしています。 
\\	妹は漫画が好きです。	いもうとはまんががすきです。 
\\	彼は釣りの名人です。	かれはつりのめいじんです。 
\\	前方に山が見えますね。	ぜんぽうにやまがみえますね。 
\\	父が息子を励ましたの。	ちちがむすこをはげましたの。 
\\	うちの子はその本に夢中です。	うちのこはそのほんにむちゅうです。 
\\	新しい仕事の面接に行ってきました。	あたらしいしごとのめんせつにいってきました。 
\\	月が庭を照らしている。	つきがにわをてらしている。 
\\	寺のとなりに墓地があります。	てらのとなりにぼちがあります。 
\\	明日、会議を開きます。	あした、かいぎをひらきます。 
\\	町は台風で大きな損害を受けたんだ。	まちはたいふうでおおきなそんがいをうけたんだ。 
\\	これは、中年の男性によく見られる症状です。	これは、ちゅうねんのだんせいによくみられるしょうじょうです。 
\\	私は通訳です。	わたしはつうやくです。 
\\	「食べる」は動詞です。	「たべる」はどうしです。 
\\	暑さでアイスクリームが溶けてしまった。	あつさでアイスクリームがとけてしまった。 
\\	班のメンバーは5人です。	はんのメンバーは5にんです。 
\\	彼は今、不利な立場にいます。	かれはいま、ふりなたちばにいます。 
\\	最近のプロ野球は面白くなってきたね。	さいきんのプロやきゅうはおもしろくなってきたね。 
\\	私は別に気になりません。	わたしはべつにきになりません。 
\\	彼は仕事の本質をよく理解しているわね。	かれはしごとのほんしつをよくりかいしているわね。 
\\	この商品は免税です。	このしょうひんはめんぜいです。 
\\	資格を有する人のみ応募できます。	しかくをゆうするひとのみおうぼできます。 
\\	私には仕事が第一です。	わたしにはしごとがだいいちです。 
\\	このメガネをかけると平面が立体に見えます。	このメガネをかけるとへいめんがりったいにみえます。 
\\	彼はとても早いペースで走っているね。	かれはとてもはやいペースではしっているね。 
\\	子供が紙飛行機を飛ばしています。	こどもがかみひこうきをとばしています。 
\\	いや、僕は本気なんだ。	いや、ぼくはほんきなんだ。 
\\	それは時間の無駄です。	それはじかんのむだです。 
\\	彼女は自然と調和した暮らし方をしているの。	かのじょはしぜんとちょうわしたくらしかたをしているの。 
\\	彼は大阪に転勤しました。	かれはおおさかにてんきんしました。 
\\	春の太陽が雪を溶かしました。	はるのたいようがゆきをとかしました。 
\\	保険は半年払いにしています。	ほけんははんとしはらいにしています。 
\\	この仕事は君に任せる。	このしごとはきみにまかせる。 
\\	彼はマラソンに出場したよ。	かれはマラソンにしゅつじょうしたよ。 
\\	彼はいつも物事を深く考えるの。	かれはいつもものごとをふかくかんがえるの。 
\\	要するに時機を待つべきだ。	ようするにじきをまつべきだ。 
\\	あと少しで山の頂上です。	あとすこしでやまのちょうじょうです。 
\\	哲学は興味深い学問です。	てつがくはきょうみぶかいがくもんです。 
\\	私たちは同級生です。	わたしたちはどうきゅうせいです。 
\\	電話料金が値下げされた。	でんわりょうきんがねさげされた。 
\\	税金の額は年収によって変わります。	ぜいきんのがくはねんしゅうによってかわります。 
\\	息子に新しい歯が生えました。	むすこにあたらしいはがはえました。 
\\	車のライトが反射していますね。	くるまのライトがはんしゃしていますね。 
\\	店の外にまだ営業中の札がでているよ。	"みせのそとにまだえいぎょうちゅうのさつがでているよ。 
\\	この棒は何に使うのですか。	このぼうはなににつかうのですか。 
\\	今日の遠足はよほど楽しかったらしい。	きょうのえんそくはよほどたのしかったらしい。 
\\	別れはいつでも悲しいものです。	わかれはいつでもかなしいものです。 
\\	ここで部屋の温度が調節できます。	ここでへやのおんどがちょうせつできます。 
\\	直線を描いてください。	ちょくせんをえがいてください。 
\\	駅の手前に郵便局があります。	えきのてまえにゆうびんきょくがあります。 
\\	彼女は本国に帰りました。	かのじょはほんごくにかえりました。 
\\	古い手紙を燃やしたんだ。	ふるいてがみをもやしたんだ。 
\\	贈り物に手書きのカードを添えました。	おくりものにてがきのカードをそえました。 
\\	この笛は竹でできている。	このふえはたけでできている。 
\\	その事故で多量のガス漏れがあったね。	そのじこでたりょうのガスもれがあったね。 
\\	君の長所は明るいところだね。	きみのちょうしょはあかるいところだね。 
\\	彼女はユーモアのセンスがあります。	かのじょはユーモアのセンスがあります。 
\\	彼女はとても陽気です。	かのじょはとてもようきです。 
\\	この村には田畑がたくさんあります。	このむらにはたはたがたくさんあります。 
\\	玄関に車を着けます。	げんかんにくるまをつけます。 
\\	来週、内臓を検査します。	らいしゅう、ないぞうをけんさします。 
\\	この生地は滑らかな手触りが特長です。	このきじはなめらかなてざわりがとくちょうです。 
\\	彼女は子供のスカートを縫ったんだ。	かのじょはこどものスカートをぬったんだ。 
\\	車のギアをバックに入れたよ。	くるまのギアをバックにいれたよ。 
\\	彼はまだ失恋を引きずっています。	かれはまだしつれんをひきずっています。 
\\	彼の態度はどこか不自然だったな。	かれのたいどはどこかふしぜんだったな。 
\\	うちの両親はお見合い結婚でした。	うちのりょうしんはおみあいけっこんでした。 
\\	彼女は政治に無関心です。	かのじょはせいじにむかんしんです。 
\\	二人は一生を共にすることを誓ったの。	ににんはいっしょうをともにすることをちかったの。 
\\	転んで手首を痛めました。	ころんでてくびをいためました。 
\\	日差しが強く照りつけますね。	ひざしがつよくてりつけますね。 
\\	ここは私にとって懐かしい場所です。	ここはわたしにとってなつかしいばしょです。 
\\	びんのふたが外れません。	びんのふたがはずれません。 
\\	娘が突然、発熱したんだ。	むすめがとつぜん、はつねつしたんだ。 
\\	合格を真っ先に両親に伝えたの。	ごうかくをまっさきにりょうしんにつたえたの。 
\\	彼は風紀を乱している。	かれはふうきをみだしている。 
\\	身の回りの世話は私がします。	みのまわりのせわはわたしがします。 
\\	彼女は無口な人ですね。	かのじょはむくちなひとですね。 
\\	どうぞ召し上がってください。	どうぞめしあがってください。 
\\	物音がしたので見に行った。	ものおとがしたのでみにいった。 
\\	秘密を漏らしたのは彼です。	ひみつをもらしたのはかれです。 
\\	仕事には時間と労力が必要です。	しごとにはじかんとろうりょくがひつようです。 
\\	部屋の中から笑い声が聞こえた。	へやのなかからわらいごえがきこえた。 
\\	いくつもの高層ビルがそびえていたんだ。	いくつものこうそうビルがそびえていたんだ。 
\\	今朝の体温は36度でした。	けさのたいおんは36どでした。 
\\	薪を集めて火を焚いたよ。	たきぎをあつめてひをたいたよ。 
\\	彼女は多忙な人です。	かのじょはたぼうなひとです。 
\\	彼はタレントなのでよくテレビに出ています。	かれはタレントなのでよくテレビにでています。 
\\	彼は地理に詳しいの。	かれはちりにくわしいの。 
\\	午後5時には電灯がつきます。	ごご5じにはでんとうがつきます。 
\\	当人は意外に平気のようね。	とうにんはいがいにへいきのようね。 
\\	今日は私が掃除の当番です。	きょうはわたしがそうじのとうばんです。 
\\	子猫にトラと名付けました。	"こねこにトラとなづけました。 
\\	犬の散歩は私の日課です。	いぬのさんぽはわたしのにっかです。 
\\	傷口はもう塞がりました。	きずぐちはもうふさがりました。 
\\	台風がその半島を通過したの。	たいふうがそのはんとうをつうかしたの。 
\\	子供が風船を膨らませているね。	こどもがふうせんをふくらませているね。 
\\	保健室で少し休んだわ。	ほけんしつですこしやすんだわ。 
\\	このクーポンはもう無効です。	このクーポンはもうむこうです。 
\\	友達と遊園地に行きました。	ともだちとゆうえんちにいきました。 
\\	このトイレは有料です。	このトイレはゆうりょうです。 
\\	予報では明日は雨ですね。	よほうではあしたはあめですね。 
\\	彼はたまに料理をします。	かれはたまにりょうりをします。 
\\	手本を見ながら習字をしました。	てほんをみながらしゅうじをしました。 
\\	この問いに答えられますか。	このといにこたえられますか。 
\\	次の冬季オリンピックは2年後です。	つぎのとうきオリンピックは2ねんごです。 
\\	もうここには二度と来ません。	もうここにはにどときません。 
\\	私は毎日寝る前に入浴します。	わたしはまいにちねるまえににゅうよくします。 
\\	昨日、花火を見に行ったよ。	きのう、はなびをみにいったよ。 
\\	このペンは鳥の羽根で作られています。	このペンはとりのはねでつくられています。 
\\	彼女は人前に出ると緊張します。	かのじょはひとまえにでるときんちょうします。 
\\	今、あの国は不景気らしいよ。	いま、あのくにはふけいきらしいよ。 
\\	不良品を返品したよ。	ふりょうひんをへんぴんしたよ。 
\\	床にペンキをこぼしてしまいました。	ゆかにペンキをこぼしてしまいました。 
\\	公園のベンチで本を読んでいます。	こうえんのベンチでほんをよんでいます。 
\\	交通費はめいめい支払ってください。	こうつうひはめいめいしはらってください。 
\\	彼は病気で弱っているんだ。	かれはびょうきでよわっているんだ。 
\\	日本は列島です。	にっぽんはれっとうです。 
\\	彼女は和服がよく似合う。	かのじょはわふくがよくにあう。 
\\	彼の葬式は明日です。	かれのそうしきはあしたです。 
\\	あなたに頼みがあるんですけど。	あなたにたのみがあるんですけど。 
\\	そのホテルで食中毒が発生したんだ。	そのホテルでしょくちゅうどくがはっせいしたんだ。 
\\	低気圧が近づいています。	ていきあつがちかづいています。 
\\	どうにか締め切りに間に合った。	どうにかしめきりにまにあった。 
\\	彼女は内心どきどきしていましたよ。	かのじょはないしんどきどきしていましたよ。 
\\	彼は中国の内陸を旅行しました。	かれはちゅうごくのないりくをりょこうしました。 
\\	年配の人に席を譲りました。	ねんぱいのひとにせきをゆずりました。 
\\	100メートル走のタイムを計ったんだ。	100メートルはしのタイムをはかったんだ。 
\\	このデータはまだ不完全ね。	このデータはまだふかんぜんね。 
\\	包丁で指を切った。	ほうちょうでゆびをきった。 
\\	間違いを見落としたの。	まちがいをみおとしたの。 
\\	彼女はとても有能な部下です。	かのじょはとてもゆうのうなぶかです。 
\\	彼女は恋愛にあこがれる年ごろです。	かのじょはれんあいにあこがれるとしごろです。 
\\	彼はとても体格がいいわ。	かれはとてもたいかくがいいわ。 
\\	校長の退屈な話が続いたんだよ。	こうちょうのたいくつなはなしがつづいたんだよ。 
\\	その建物の屋根は平らだね。	そのたてもののやねはたいらだね。 
\\	朝食はちゃんと食べましたか。	ちょうしょくはちゃんとたべましたか。 
\\	この本の定価は525円です。	このほんのていかは525えんです。 
\\	こんなことも知らないとは情けない。	こんなこともしらないとはなさけない。 
\\	私の田舎はのどかな所です。	わたしのいなかはのどかなところです。 
\\	誰が噂を広めたんだろう。	だれがうわさをひろめたんだろう。 
\\	その文明は何千年も前に滅びてしまったの。	そのぶんめいはなんせんねんもまえにほろびてしまったの。 
\\	この件の担当者は未定です。	このけんのたんとうしゃはみていです。 
\\	このメロディーは聞いたことがある。	このメロディーはきいたことがある。 
\\	あまり欲を出しちゃだめだよ。	あまりよくをだしちゃだめだよ。 
\\	今日は水を2リットル以上飲んだわ。	きょうはみずを2リットルいじょうのんだわ。 
\\	彼はわざと負けたように見えたな。	かれはわざとまけたようにみえたな。 
\\	その子供は大人と対等に話していたよ。	そのこどもはおとなとたいとうにはなしていたよ。 
\\	5月は田植えの季節です。	5つきはたうえのきせつです。 
\\	蝶蝶を捕まえたよ。	ちょうちょうをつかまえたよ。 
\\	私の勤め先にご連絡ください。	わたしのつとめさきにごれんらくください。 
\\	彼ならとっくに帰りましたよ。	かれならとっくにかえりましたよ。 
\\	この村の人たちは長生きです。	このむらのひとたちはながいきです。 
\\	彼は仕事が長続きしません。	かれはしごとがながつづきしません。 
\\	彼はもう彼女を憎んではいないよ。	かれはもうかのじょをにくんではいないよ。 
\\	この部屋は日光がよく当たるね。	このへやはにっこうがよくあたるね。 
\\	花嫁が父親と一緒に入ってきましたね。	はなよめがちちおやといっしょにはいってきましたね。 
\\	無責任な行動は許されません。	むせきにんなこうどうはゆるされません。 
\\	彼は欲望が強い人です。	かれはよくぼうがつよいひとです。 
\\	彼は我が国を代表する作家です。	かれはわがくにをだいひょうするさっかです。 
\\	この本の題名は「坊ちゃん」です。	"このほんのだいめいは「ぼっちゃん」です。 
\\	彼の新しい著書が出版されたね。	かれのあたらしいちょしょがしゅっぱんされたね。 
\\	友人が慰めてくれました。	ゆうじんがなぐさめてくれました。 
\\	私たちは野原で花をつんだの。	わたしたちはのはらではなをつんだの。 
\\	浜辺できれいな貝がらを拾いました。	はまべできれいなかいがらをひろいました。 
\\	彼は晩年を故郷で過ごしたんだ。	かれはばんねんをこきょうですごしたんだ。 
\\	雨が強かったので引き返したよ。	あめがつよかったのでひきかえしたよ。 
\\	海で日焼けしたんだ。	うみでひやけしたんだ。 
\\	「ゆっくり歩く」の「ゆっくり」は副詞です。	"「ゆっくりあるく」の「ゆっくり」はふくしです。 
\\	不必要なファイルは削除してください。	ふひつようなファイルはさくじょしてください。 
\\	私はベストを尽くしました。	わたしはベストをつくしました。 
\\	人と交わって、多くのことを学びました。	ひととまじわって、おおくのことをまなびました。 
\\	待合室はとても込んでいたよ。	まちあいしつはとてもこんでいたよ。 
\\	水着に着替えました。	みずぎにきがえました。 
\\	ぼくは無論きみに賛成だ。	ぼくはむろんきみにさんせいだ。 
\\	余計なことを言ってはだめよ。	よけいなことをいってはだめよ。 
\\	彼女は白いワンピースを着ていたよ。	かのじょはしろいワンピースをきていたよ。 
\\	国民の6割が総理大臣を支持しています。	こくみんの6わりがそうりだいじんをしじしています。 
\\	ダイヤモンドはとても高価ですね。	ダイヤモンドはとてもこうかですね。 
\\	あの子の知能はとても高いそうです。	あのこのちのうはとてもたかいそうです。 
\\	それは彼女が作り上げた話です。	それはかのじょがつくりあげたはなしです。 
\\	あの夫婦は仲良く暮らしているよ。	あのふうふはなかよくくらしているよ。 
\\	彼は並みの人間ではありません。	かれはなみのにんげんではありません。 
\\	その車のナンバーを覚えていますか。	そのくるまのナンバーをおぼえていますか。 
\\	珍しい日本酒が手に入りました。	めずらしいにほんしゅがてにはいりました。 
\\	派出所にだれもいないな。	はしゅつじょにだれもいないな。 
\\	早まらないでよく考えましょう。	はやまらないでよくかんがえましょう。 
\\	今日も遊園地は大変な人出だったよ。	きょうもゆうえんちはたいへんなひとでだったよ。 
\\	彼の話を聞いて不愉快になった。	かれのはなしをきいてふゆかいになった。 
\\	彼女は振り向いて俺に微笑んだんだ。	かのじょはふりむいておれにほほえんだんだ。 
\\	この店は8時に閉店します。	このみせは8じにへいてんします。 
\\	彼は世の中の偏見と戦ったんだ。	かれはよのなかのへんけんとたたかったんだ。 
\\	休む時は前もって連絡ください。	やすむときはまえもってれんらくください。 
\\	今、豆を煮ています。	いま、まめをにています。 
\\	先生の回りに集まってください。	せんせいのまわりにあつまってください。 
\\	息子は学校で読み書きを勉強しています。	むすこはがっこうでよみかきをべんきょうしています。 
\\	この数年、体重は増減していません。	このすうねん、たいじゅうはぞうげんしていません。 
\\	雑誌にその女優の対談が載っていたよ。	ざっしにそのじょゆうのたいだんがのっていたよ。 
\\	この曲は単調でつまらないな。	このきょくはたんちょうでつまらないな。 
\\	とうとう引っ越しの日がきました。	とうとうひっこしのひがきました。 
\\	仕事中に帰るなんてとんでもない。	しごとちゅうにかえるなんてとんでもない。 
\\	彼らはにこやかに挨拶を交わしたけどね。	かれらはにこやかにあいさつをかわしたけどね。 
\\	彼の家柄は武士でした。	かれのいえがらはぶしでした。 
\\	北極にペンギンはいません。	ほっきょくにペンギンはいません。 
\\	コピーの枚数を数えてください。	コピーのまいすうをかぞえてください。 
\\	見本はこちらにございます。	みほんはこちらにございます。 
\\	持ち物には名前を書いてください。	もちものにはなまえをかいてください。 
\\	矢印にそって歩いてください。	やじるしにそってあるいてください。 
\\	預金には利息がつきます。	よきんにはりそくがつきます。 
\\	荷物を駅のロッカーに入れた。	にもつをえきのロッカーにいれた。 
\\	彼から度々メールが来ます。	かれからたびたびメールがきます。 
\\	人を騙してはいけません。	ひとをだましてはいけません。 
\\	私は中途採用で入社しました。	わたしはちゅうとさいようでにゅうしゃしました。 
\\	私の部屋は長方形です。	わたしのへやはちょうほうけいです。 
\\	空港へ彼女を出迎えに行きました。	くうこうへかのじょをでむかえにいきました。 
\\	死んだら天国に行きたいです。	しんだらてんごくにいきたいです。 
\\	今夜中に東京に着くのは到底無理でしょう。	こんやちゅうにとうきょうにつくのはとうていむりでしょう。 
\\	ホテルの予約を取り消したよ。	ホテルのよやくをとりけしたよ。 
\\	駅前の並木が台風で倒れたらしい。	えきまえのなみきがたいふうでたおれたらしい。 
\\	パーティーに行くのは何となく気が進まないな。	パーティーにいくのはなんとなくきがすすまないな。 
\\	公園で犬を放したの。	こうえんでいぬをはなしたの。 
\\	彼はプライドが高い人です。	かれはプライドがたかいひとです。 
\\	今月の付録はアニメの
\\	です。	こんげつのふろくはアニメの
\\	です。 
\\	ママに聞いてみよう。	ママにきいてみよう。 
\\	彼女は政治について無知でした。	かのじょはせいじについてむちでした。 
\\	この町の名物はぶどうです。	このまちのめいぶつはぶどうです。 
\\	私の家は赤い屋根が目印です。	わたしのいえはあかいやねがめじるしです。 
\\	彼らはとても愉快な人たちです。	かれらはとてもゆかいなひとたちです。 
\\	彼は気が緩んでいます。	かれはきがゆるんでいます。 
\\	翌朝は快晴だったね。	よくあさはかいせいだったね。 
\\	勉強と仕事の両立は難しいよ。	べんきょうとしごとのりょうりつはむずかしいよ。 
\\	椅子の荷物をよけて座ったけど。	いすのにもつをよけてすわったけど。 
\\	今は田に水を入れる季節です。	いまはたにみずをいれるきせつです。 
\\	ここは立ち入り禁止です。	ここはたちいりきんしです。 
\\	彼は頼もしい人です。	かれはたのもしいひとです。 
\\	彼に近寄らないで。	かれにちかよらないで。 
\\	ここで直角に曲がってください。	ここでちょっかくにまがってください。 
\\	赤ちゃんの泣き声が聞こえますね。	あかちゃんのなきごえがきこえますね。 
\\	生徒たちは校内では名札をつけます。	せいとたちはこうないではなふだをつけます。 
\\	彼は弁護士バッジを付けていたよ。	かれはべんごしバッジをつけていたよ。 
\\	彼はおまんじゅうを一口で食べたの。	かれはおまんじゅうをひとくちでたべたの。 
\\	彼女はいつも髪をピンで留めているね。	かのじょはいつもかみをピンでとめているね。 
\\	平行に線を引いてください。	へいこうにせんをひいてください。 
\\	娘を保育園に迎えに行きます。	むすめをほいくえんにむかえにいきます。 
\\	学校で防火訓練がありました。	がっこうでぼうかくんれんがありました。 
\\	まあこれでいいだろう。	まあこれでいいだろう。 
\\	駅まで迎えに来て下さい。	えきまでむかえにきてください。 
\\	愛がなければ人生はむなしいよ。	あいがなければじんせいはむなしいよ。 
\\	彼には申し訳ないことをした。	かれにはもうしわけないことをした。 
\\	彼は全校生徒の模範です。	かれはぜんこうせいとのもはんです。 
\\	友人達に協力を呼び掛けたの。	ゆうじんたちにきょうりょくをよびかけたの。 
\\	彼は私の弱みを握っているんだ。	かれはわたしのよわみをにぎっているんだ。 
\\	銭湯は昔より少なくなったわね。	せんとうはむかしよりすくなくなったわね。 
\\	セーターが縮んでしまったな。	セーターがちぢんでしまったな。 
\\	川の中流あたりにその村はあるよ。	かわのちゅうりゅうあたりにそのむらはあるよ。 
\\	ついでだから彼も呼ぼうよ。	ついでだからかれもよぼうよ。 
\\	その国ではコレラの伝染が蔓延しています。	そのくにではコレラのでんせんがまんえんしています。 
\\	この皿を戸棚にしまってください。	このさらをとだなにしまってください。 
\\	長らくお待たせしました。	ながらくおまたせしました。 
\\	彼女は何で来ないんだろう。	かのじょはなんでこないんだろう。 
\\	妹は時々憎らしいことを言うんだ。	いもうとはときどきにくらしいことをいうんだ。 
\\	彼はまだまだ二流の芸人だね。	かれはまだまだにりゅうのげいにんだね。 
\\	この木の根はとても太いな。	このきのねはとてもふといな。 
\\	彼は値切るのが上手です。	かれはねぎるのがじょうずです。 
\\	年賀葉書が売り出されたね。	ねんがはがきがうりだされたね。 
\\	ゴミを残らず拾ったよ。	ゴミをのこらずひろったよ。 
\\	履物は靴箱に入れてください。	はきものはくつばこにいれてください。 
\\	飛行機の運賃が払い戻されたの。	ひこうきのうんちんがはらいもどされたの。 
\\	ビーチで友達とサーフィンしたよ。	ビーチでともだちとサーフィンしたよ。 
\\	ヒ゛ルの非常口を確認したの。	ヒ゛ルのひじょうぐちをかくにんしたの。 
\\	ビタミンを毎日とりましょう。	ビタミンをまいにちとりましょう。 
\\	魚の骨がのどに引っ掛かった。	さかなのほねがのどにひっかかった。 
\\	父兄の方々はこちらへどうぞ。	ふけいのかたがたはこちらへどうぞ。 
\\	フライパンで目玉焼きを作ったよ。	フライパンでめだまやきをつくったよ。 
\\	まあまあ、落ち着いてください。	まあまあ、おちついてください。 
\\	町角にカフェがありました。	まちかくにカフェがありました。 
\\	このベランダは見晴らしがいい。	このベランダはみはらしがいい。 
\\	彼は身振りを交えて説明してくれたの。	かれはみぶりをまじえてせつめいしてくれたの。 
\\	向かいの席が空いていますよ。	むかいのせきがあいていますよ。 
\\	彼女は元々フランスに行くつもりだったの。	かのじょはもともとフランスにいくつもりだったの。 
\\	焼き鳥はビールに合うな。	やきとりはビールにあうな。 
\\	帰宅中、夕立にあったの。	きたくちゅう、ゆうだちにあったの。 
\\	みんなが彼の指差す方を見たんだ。	みんながかれのゆびさすほうをみたんだ。 
\\	夜道は用心して歩きましょう。	よみちはようじんしてあるきましょう。 
\\	人間は理性を持つ動物です。	にんげんはりせいをもつどうぶつです。 
\\	この映画は割に面白いわ。	このえいがはわりにおもしろいわ。 
\\	体温計が壊れてしまった。	たいおんけいがこわれてしまった。 
\\	このかばんには大金が入っています。	このかばんにはたいきんがはいっています。 
\\	タコには足が8本ある。	タコにはあしが8ほんある。 
\\	お風呂に水を溜めておいてください。	おふろにみずをためておいてください。 
\\	彼はその地で残りの生涯を過ごしたんだ。	かれはそのちでのこりのしょうがいをすごしたんだ。 
\\	地下道を通って行きましょう。	ちかどうをとおっていきましょう。 
\\	弟の部屋はいつも散らかっているんだ。	おとうとのへやはいつもちらかっているんだ。 
\\	デザートにケーキを食べました。	デザートにケーキをたべました。 
\\	ネットで得する情報を見つけたよ。	ネットでとくするじょうほうをみつけたよ。 
\\	とっさに彼の名前が出てこなかった。	とっさにかれのなまえがでてこなかった。 
\\	電話があったら取り次いでください。	でんわがあったらとりついでください。 
\\	この話は課長には内緒ですよ。	このはなしはかちょうにはないしょですよ。 
\\	けんかした友達と仲直りしました。	けんかしたともだちとなかなおりしました。 
\\	もうすぐじゃがいもが煮えます。	もうすぐじゃがいもがにえます。 
\\	彼が憎いですか。	かれがにくいですか。 
\\	失敗した時の逃げ道を考えたの。	しっぱいしたときのにげみちをかんがえたの。 
\\	うなぎはぬるぬるしていますね。	うなぎはぬるぬるしていますね。 
\\	部下の方が年始の挨拶に来ましたよ。	ぶかのほうがねんしのあいさつにきましたよ。 
\\	この電車は本当にのろいですね。	このでんしゃはほんとうにのろいですね。 
\\	このくじは外れです。	このくじははずれです。 
\\	友達が蜂に刺されたんだ。	ともだちがはちにさされたんだ。 
\\	知らない人が話し掛けてきた。	しらないひとがはなしかけてきた。 
\\	彼女はいつも独り言を言うんだ。	かのじょはいつもひとりごとをいうんだ。 
\\	カエルがぴょんぴょんはねているね。	カエルがぴょんぴょんはねているね。 
\\	貧乏人が大金持ちになることもあるさ。	びんぼうにんがおおがねもちになることもあるさ。 
\\	犬の毛をブラシでとかしてやったの。	いぬのけをブラシでとかしてやったの。 
\\	彼女は文系です。	かのじょはぶんけいです。 
\\	私の友人はお坊さんをしているの。	わたしのゆうじんはおぼうさんをしているの。 
\\	うちの子が迷子になりました。	うちのこがまいごになりました。 
\\	サラダにマヨネーズをかけたの。	サラダにマヨネーズをかけたの。 
\\	未成年はお酒を飲めません。	みせいねんはおさけをのめません。 
\\	この農薬は人には無害です。	こののうやくはひとにはむがいです。 
\\	今日は寄り道してから帰ります。	きょうはよりみちしてからかえります。 
\\	最近、食事の量を減らしています。	さいきん、しょくじのりょうをへらしています。 
\\	例年8月は雨が少ないね。	れいねん8つきはあめがすくないね。 
\\	3日に1度の割りで彼からメールが来るの。	3にちに1どのわりでかれからメールがくるの。 
\\	大概、風邪は寝ていれば治ります。	たいがい、かぜはねていればなおります。 
\\	彼は大層喜んでいました。	かれはたいそうよろこんでいました。 
\\	昨日大量のゴミが出たの。	きのうたいりょうのゴミがでたの。 
\\	お金がないなら私が立て替えておきます。	おかねがないならわたしがたてかえておきます。 
\\	例えを上げると話が分かり易くなる。	たとえをあげるとはなしがわかりやすくなる。 
\\	私は玉ねぎが嫌いです。	わたしはたまねぎがきらいです。 
\\	だらしない格好をしないでください。	だらしないかっこうをしないでください。 
\\	この服の縮み具合はひどい。	このふくのちぢみぐあいはひどい。 
\\	お盆は実家に帰りました。	おぼんはじっかにかえりました。 
\\	9月には月見を楽しみます。	9つきにはつきみをたのしみます。 
\\	その子が言っていることは作り話です。	そのこがいっていることはつくりばなしです。 
\\	昼の定食は3種類あります。	ひるのていしょくは3しゅるいあります。 
\\	このとうもろこしは甘くておいしい。	このとうもろこしはあまくておいしい。 
\\	彼は両手を握り締めたの。	かれはりょうてをにぎりしめたの。 
\\	彼は不動産の売買をしています。	かれはふどうさんのばいばいをしています。 
\\	白線の内側に下がってお待ちください。	はくせんのうちがわにさがっておまちください。 
\\	この写真を引き伸ばしてください。	このしゃしんをひきのばしてください。 
\\	彼女は人差し指を怪我したの。	かのじょはひとさしゆびをけがしたの。 
\\	友達はそのカップルを冷やかしたんだ。	ともだちはそのカップルをひやかしたんだ。 
\\	木の葉がひらひらと落ちたの。	このはがひらひらとおちたの。 
\\	友達に手紙を書こうと便箋を買ったんだ。	ともだちにてがみをかこうとびんせんをかったんだ。 
\\	今日はぽかぽか暖かい日です。	きょうはぽかぽかあたたかいひです。 
\\	水と油は混ざりません。	みずとあぶらはまざりません。 
\\	大人も子供も交ざって遊んだの。	おとなもこどももまざってあそんだの。 
\\	彼女は町外れの工場で働いているよ。	かのじょはまちはずれのこうじょうではたらいているよ。 
\\	今日は月が真ん丸です。	きょうはつきがまんまるです。 
\\	家から学校までの道順を教えてください。	いえからがっこうまでのみちじゅんをおしえてください。 
\\	そのお寺は長い年月をかけて建てられた。	そのおてらはながいとしつきをかけてたてられた。 
\\	あの老人はとても物知りだね。	あのろうじんはとてもものしりだね。 
\\	幼児は入場無料です。	ようじはにゅうじょうむりょうです。 
\\	昨日の晩御飯は洋食でした。	きのうのばんごはんはようしょくでした。 
\\	彼の家には洋間があります。	かれのいえにはようまがあります。 
\\	食事は余分に用意してあります。	しょくじはよぶんによういしてあります。 
\\	人は見かけによらないな。	ひとはみかけによらないな。 
\\	その店に何十人も行列していたよ。	そのみせになんじゅうにんもぎょうれつしていたよ。 
\\	洗面用具を忘れた。	せんめんようぐをわすれた。 
\\	具合が悪かったので仕事を早退しました。	ぐあいがわるかったのでしごとをそうたいしました。 
\\	病院で大便の検査をした。	びょういんでだいべんのけんさをした。 
\\	作文は縦書きでお願いします。	さくぶんはたてがきでおねがいします。 
\\	この服は大き過ぎてだぶだぶです。	このふくはおおきすぎてだぶだぶです。 
\\	彼の顔に玉のような汗が流れていたの。	かれのかおにたまのようなあせがながれていたの。 
\\	は単数で
\\	は複数です。	
\\	はたんすうで
\\	はふくすうです。 
\\	何か付け足すことはありますか。	なにかつけたすことはありますか。 
\\	床がつるつるすべります。	ゆかがつるつるすべります。 
\\	風邪をひかないように手洗いとうがいをしましょう。	かぜをひかないようにてあらいいとうがいをしましょう。 
\\	この店は水曜が定休日です。	このみせはすいようがていきゅうびです。 
\\	私は作品の出来上がりに満足しています。	わたしはさくひんのできあがりにまんぞくしています。 
\\	お手数ですがよろしくお願いします。	おてかずですがよろしくおねがいします。 
\\	私は毎朝トーストを2枚食べます。	わたしはまいあさトーストを2まいたべます。 
\\	突然の質問にどきっとしたよ。	とつぜんのしつもんにどきっとしたよ。 
\\	あなたたちは仲がいいですね。	あなたたちはなかがいいですね。 
\\	うっかり寝過ごしてしまったんだ。	うっかりねすごしてしまったんだ。 
\\	東京の人口は年々増えています。	とうきょうのじんこうはねんねんふえています。 
\\	寝坊して新幹線に乗り遅れたよ。	ねぼうしてしんかんせんにのりおくれたよ。 
\\	車は雪道をのろのろと走ったの。	くるまはゆきどうをのろのろとはしったの。 
\\	寒いので暖かい肌着を着ました。	さむいのであたたかいはだぎをきました。 
\\	彼は違法駐車で罰せられたんだ。	かれはいほうちゅうしゃでばっせられたんだ。 
\\	早寝は健康のためによいことです。	はやねはけんこうのためによいことです。 
\\	新車はぴかぴかですね。	しんしゃはぴかぴかですね。 
\\	その会社の筆記試験は難しかったよ。	そのかいしゃのひっきしけんはむずかしかったよ。 
\\	ここで一休みしましょう。	ここでひとやすみしましょう。 
\\	今日の日の入りは午後6時でした。	きょうのひのいりはごご6じでした。 
\\	人の記憶は不確かよ。	ひとのきおくはふたしかよ。 
\\	学校から父母に連絡があったんだ。	がっこうからちちははにれんらくがあったんだ。 
\\	私は風呂場で歯を磨きます。	わたしはふろじょうではをみがきます。 
\\	この土地の面積は約100平方メートルです。	このとちのめんせきはやく100ヘイホウメートルです。 
\\	彼は上司にぺこぺこしている。	かれはじょうしにぺこぺこしている。 
\\	この本は薄くてぺらぺらですね。	このほんはうすくてぺらぺらですね。 
\\	この果物はビタミンが豊富です。	このくだものはビタミンがほうふです。 
\\	ほうれん草はビタミンが豊富です。	ほうれんそうはビタミンがほうふです。 
\\	この部屋はほこりだらけですね。	このへやはほこりだらけですね。 
\\	お湯はポットに入っています。	おゆはポットにはいっています。 
\\	彼は正に英雄ね。	かれはまさにえいゆうね。 
\\	この人に見覚えがありますか。	このひとにみおぼえがありますか。 
\\	彼はソウルミュージックをよく聞くの。	かれはソウルミュージックをよくきくの。 
\\	目が疲れたので目薬をさしたよ。	めがつかれたのでめぐすりをさしたよ。 
\\	目覚ましにコーヒーを飲んだの。	めざましにコーヒーをのんだの。 
\\	彼は目下の者にも優しいんだ。	かれはもっかのものにもやさしいんだ。 
\\	読みたい章を目次で探しました。	よみたいしょうをもくじでさがしました。 
\\	赤ちゃんをベッドに寝かせた。	あかちゃんをベッドにねかせた。 
\\	息子は目下勉強中です。	むすこはもっかべんきょうちゅうです。 
\\	お祭りで焼きそばを食べたよ。	おまつりでやきそばをたべたよ。 
\\	湯飲みにはお茶を入れてお茶碗にはご飯をよそいます。	ゆのみにはおちゃをいれておちゃわんにはごはんをよそいます。 
\\	このテーブルは洋室に合わない。	このテーブルはようしつにあわない。 
\\	運転中はよそ見をしてはいけません。	うんてんちゅうはよそみをしてはいけません。 
\\	ニューヨークには芸術家が寄り集まっているの。	ニューヨークにはげいじゅつかがよりあつまっているの。 
\\	国際連合を略して国連といいます。	こくさいれんごうをりゃくしてこくれんといいます。 
\\	彼女はロマンチックな人です。	かのじょはロマンチックなひとです。 
\\	遠足が楽しみでわくわくしています。	えんそくがたのしみでわくわくしています。 
\\	刺身にわさびは欠かせませんね。	さしみにわさびはかかせませんね。 
\\	あの家のトイレは和式です。	あのいえのトイレはわしきです。 
\\	洗面器でハンカチを洗ったの。	せんめんきでハンカチをあらったの。 
\\	郵便局はどこかご存じですか。	ゆうびんきょくはどこかごぞんじですか。 
\\	彼女は短期大学で日本文学を勉強しました。	かのじょはたんきだいがくでにっぽんぶんがくをべんきょうしました。 
\\	彼はとても男性的な人です。	かれはとてもだんせいてきなひとです。 
\\	チキンソテーは私の大好物です。	チキンソテーはわたしのだいこうぶつです。 
\\	いつも月初めに彼と会います。	いつもつきはじめにかれとあいます。 
\\	収入と支出が釣り合っていない。	しゅうにゅうとししゅつがつりあっていない。 
\\	美しさを定義してください。	うつくしさをていぎしてください。 
\\	日本では夫婦同姓が一般的よ。	にっぽんではふうふどうせいがいっぱんてきよ。 
\\	その事故で貴い命が失われました。	そのじこでとうといいのちがうしなわれました。 
\\	この本はページが所々破れているね。	このほんはページがところどころやぶれているね。 
\\	書類はこのファイルにとじてください。	しょるいはこのファイルにとじてください。 
\\	靴が泥だらけになったよ。	くつがどろだらけになったよ。 
\\	彼の名前を度忘れしたぞ。	かれのなまえをどわすれしたぞ。 
\\	この街には家具の問屋がたくさんあります。	このまちにはかぐのとんやがたくさんあります。 
\\	母が電話で長話をしているんだ。	ははがでんわでながばなしをしているんだ。 
\\	何てきれいな人なんだ。	なにてきれいなひとなんだ。 
\\	なぜか彼はにやにやしています。	なぜかかれはにやにやしています。 
\\	居眠りして降りる駅を乗り越した。	いねむりしておりるえきをのりこした。 
\\	彼女は肌色のシャツを着ています。	かのじょははだいろのシャツをきています。 
\\	彼は悪いことをしたので罰を受けたんだ。	かれはわるいことをしたのでばっをうけたんだ。 
\\	公園の鳩にえさをやったの。	こうえんのはとにえさをやったの。 
\\	お祝いにばらの花束を贈りました。	おいわいにばらのはなたばをおくりました。 
\\	白いパンティーを買いました。	しろいパンティーをかいました。 
\\	社長に一言お願いしたの。	しゃちょうにひとことおねがいしたの。 
\\	その場に相応しい服装で来てください。	そのばにふさわしいふくそうできてください。 
\\	彼は府庁で働いています。	かれはふちょうではたらいています。 
\\	彼女は府立大学に通っています。	かのじょはふりつだいがくにかよっています。 
\\	いつも古里を懐かしく思い出すの。	いつもふるさとをなつかしくおもいだすの。 
\\	ふわふわの布団に寝たよ。	ふわふわのふとんにねたよ。 
\\	毎度ありがとうございます。	まいどありがとうございます。 
\\	彼の部屋には真四角な窓があったんだ。	かれのへやにはましかくなまどがあったんだ。 
\\	今日は回り道して帰ろう。	きょうはまわりみちしてかえろう。 
\\	今日中に返答しなければまずい。	きょうちゅうにへんとうしなければまずい。 
\\	母はいつも私の味方です。	はははいつもわたしのみかたです。 
\\	今朝は飯を食べたかい。	けさはめしをたべたかい。 
\\	彼の意外な面を見た。	かれのいがいなめんをみた。 
\\	隣に木造の家が建ったね。	となりにもくぞうのいえがたったね。 
\\	正月には餅を食べますよ。	しょうがつにはもちをたべますよ。 
\\	彼女は物覚えがいい。	かのじょはものおぼえがいい。 
\\	肩をもんでください。	かたをもんでください。 
\\	天井から雨が漏るの。	てんじょうからあめがもるの。 
\\	8時に学校の門が開きます。	8じにがっこうのもんがひらきます。 
\\	魚。	さかな。 
\\	魚だ。	さかなだ。 
\\	魚じゃない。	さかなじゃない。 
\\	学生じゃない。	がくせいじゃない。 
\\	静かじゃない。	しずかじゃない。 
\\	魚だった。	さかなだった。 
\\	学生じゃなかった。	がくせいじゃなかった。 
\\	静かじゃなかった。	しぞかじゃなかった。 
\\	アリスは学生?/うん、学生。	アリスはがくせい。/うん、がくせい 
\\	ジムは明日?/明日じゃない。	ジムはあした?/あしたじゃない。 
\\	今日は試験だ。/ジムは?/ジムは明日。	あしたはしけんだ。/ジムは?/ジムはあした。 
\\	アリスは学生?/うん、トムも学生。	アリスはがくせい?/うん、トムもがくせい。 
\\	アリスは学生?/うん、でもトムは学生じゃない。	アリスはがくせい?/うん、でもトムはがくせいじゃない。 
\\	アリスは学生?/ううん、トムも学生じゃない。	アリスはがくせい?/ううん、トムもがくせいじゃない。 
\\	誰が学生?/ジムが学生。	だれががくせい?/ジムはがくせい。 
\\	誰が学生?	だれががくせい? 
\\	学生は誰?	がくせいはだれ? 
\\	静かな人。	しずかなひと。 
\\	友達は親切。	ともだちはしんせつ。 
\\	友達は親切な人。	ともだちはしんせつひと。 
\\	ボブは魚が好きだ。	ボブはさかながすきだ。 
\\	ボブは魚が好きじゃない。	ボブはさかながすきじゃない。 
\\	ボブは魚が好きだった。	ボブはさかながすきだった。 
\\	ボブは魚が好きじゃなかった。	ボブはさかながすきじゃなかった。 
\\	魚が好きなタイプ。	さかながすきなタイプ。 
\\	魚が好きじゃないタイプ。	さかながすきじゃないタイプ。 
\\	魚が好きだったタイプ。	さかながすきだったタイプ。 
\\	魚が好きじゃなかったタイプ。	さかながすきじゃなかったタイプ。 
\\	魚が好きじゃないタイプは、肉が好きだ。	さかながすきじゃないタイプは、にくがすきだ。 
\\	高いビル。	たかいビル。 
\\	高くないビル。	たかくないビル。 
\\	高かったビル。	たかかったビル。 
\\	高くなかったビル。	たかくなかったビル。 
\\	静かな高いビル。	すずかなたかいビル。 
\\	高くない静かなビル。	たかくないしずかなビル。 
\\	値段が高いレストランはあまり好きじゃない。	ねだんがたかいレストランはあまりすきじゃない。 
\\	値段があんまりよくない。	ねだんがあんまりよくない。 
\\	彼はかっこよかった!	かれはかっこよかった。 
\\	食べる。	たべる。 
\\	アリスは食べる。	アリスはたべる。 
\\	ジムが遊ぶ。	ジムはあそぶ。 
\\	ボブもする。	ボブもする。 
\\	お金がある。	おかねがある。 
\\	私は買う。	わたしはかう。 
\\	猫はいる。	ねこはいる。 
\\	アリスは食べない。	アリスはたべない。 
\\	ジムが遊ばない。	ジムはあそばない。 
\\	ボブもしない。	ボブもしない。 
\\	お金がない。	おかねがない。 
\\	私は買わない。	わたしはかわない。 
\\	猫はいない。	ねこはいない。 
\\	ご飯は、食べた。	ごはんは、たべた。 
\\	映画は、全部見た。	えいがは、ぜんぶみた。 
\\	今日は、走った。	きょうは、はしった。 
\\	友達が来た。	ともだちがきた。 
\\	私も遊んだ。	わたしもあそんだ。 
\\	勉強は、した。	べんきょうは、した。 
\\	アリスは食べなかった。	アリスはたべなかった。 
\\	ジムがしなかった。	ジムはしなかった。 
\\	ボブも行かなかった。	ボブもいかなかった。 
\\	お金がなかった。	おかねがなかった。 
\\	私は買わなかった。	わたしはかわなかった。 
\\	猫はいなかった。	ねこはいなかった。 
\\	魚を食べる。	さかなをたべる。 
\\	ジュースを飲んだ。	ジュースをのんだ。 
\\	街をぶらぶら歩く。	まちをぶらぶらあるく。 
\\	高速道路を走る。	こうそくどうろをはしる。 
\\	毎日、日本語を勉強する。	まいにち、にほんごをべんきょうする。 
\\	メールアドレスを登録した。	メールアドレスをとうろくした。 
\\	ボブは日本に行った。	ボブはにほんにいった。 
\\	家に帰らない。	{いえ
\\	うち}にかえらない。 
\\	部屋にくる。	へやにくる。 
\\	アリスは、アメリカからきた。	アリスは、アメリカからきた。 
\\	宿題を今日から明日までする。	しゅくだいをきょうからあしたまでする。 
\\	猫は部屋にいる。	ねこはへやにいる。 
\\	椅子が台所にあった。	いすがだいどころにあった。 
\\	いい友達に会った。	いいともだちにあった。 
\\	ジムは医者になる。	ジムはいしゃになる。 
\\	先週に図書館に行った。	せんしゅうにとしょかんにいった。 
\\	友達は、来年、日本に行く。	ともだちは、らいねん、にほんにいく。 
\\	友達は、来年に日本に行く。	ともだちは、らいねんににほんにいく。 
\\	ボブは日本へ行った。	ボブはにほんへいった。 
\\	家へ帰らない。	{いえ
\\	うち}へかえらない。 
\\	部屋へくる。	へやへくる。 
\\	勝ちへ向かう。	かちへくかう。 
\\	映画館で見た。	えいがかんでみた。 
\\	バスで帰る。	バスでかえる。 
\\	レストランで昼ご飯を食べた。	レストランでひるごはんをたべた。 
\\	何できた?/バスできた。	なにできた?/バスできた。 
\\	何できた?/暇だから。	なんできた?/ひまだから。 
\\	学校に行った?/行かなかった。/図書館には? /図書館にも行かなかった。	「がっこうにいった?/いかなかった。/としょかんは?/としょかんにもいかなかった。 
\\	どこで食べる? /イタリアレストランではどう?	どこでだべる?/イタリアレストランではどう? 
\\	日本語を習う。	にほんごをならう。 
\\	日本語は、習う。	にほんごは、ならう。 
\\	私が電気を付けた。	わたしがでんきをつけた。 
\\	電気が付いた。	でんきがついた。 
\\	電気を消す。	でんきをけす。 
\\	電気が消える。	でんきがきえる。 
\\	誰が窓を開けた?	だれがまどをあけた? 
\\	窓がどうして開いた?	まどがどうしてあいた? 
\\	部屋を出た。	へやをでた。 
\\	学生じゃない人は、学校に行かない。	がくせいじゃないひとは、がっこうにいかない。 
\\	子供だったアリスが立派な大人になった。	こどもだったアリスがりっぱなおとなになった。 
\\	友達じゃなかったアリスは、いい友達になった。	ともだちじゃなかったアリスは、いいともだちになった。 
\\	先週に医者だったボブは、仕事を辞めた。	せんしゅうにいしゃだったボブは、しごとをやめた。 
\\	先週に映画を見た人は誰?	せんしゅうにえいがをみたひとはだれ? 
\\	ボブは、いつも勉強する人だ。	ボブは、いつもべんきょうするひとだ。 
\\	赤いズボンを買う友達はボブだ。	あかいボンをかうともだちはボブだ。 
\\	晩ご飯を食べなかった人は、映画で見た銀行に行った。	ばんごはんをたべなかったひとは、えいがでみたぎんこうにいった。 
\\	スプーンとフォークで魚を食べた。	スプーンとフォークでさかなをたべた。 
\\	本と雑誌と葉書を買った。	ほんとざっしとはがきをかった。 
\\	友達と話した。	ともだちとはなした。 
\\	先生と会った。	せんせいとあった。 
\\	飲み物やカップやナプキンは、いらない?	のみものやカップやナプキンは、いらない? 
\\	靴やシャツを買う。	くつやシャツをかう。 
\\	飲み物とかカップとかナプキンは、いらない?	のみものとかカップとかナプキンは、いらない? 
\\	靴とかシャツを買う。	くつとかシャツをかう。 
\\	ボブの本。	ボブのほん。 
\\	ボブは、アメリカの大学の学生だ。	ボブは、アメリカのだいがくのがくせいだ。 
\\	そのシャツは誰の?/ボブのだ。	そのシャツはだれの?/ボブのだ。 
\\	白いのは、かわいい。	しろいのは、かわいい。 
\\	授業に行くのを忘れた。	じゅぎょうにいくのをわすれた。 
\\	白い物は、かわいい。	しろいものは、かわいい。 
\\	授業に行くことを忘れた。	じゅぎょうにいくことをわすれた。 
\\	毎日勉強するのは大変。	まいにちべんきょうのはたいへん。 
\\	毎日同じ物を食べるのは、面白くない。	まいにちおなじものをたべるのは、おもしろくない。 
\\	静かなのが、アリスの部屋だ。	しずかなのが、アリスのへやだ。 
\\	今は忙しいの。	いまはいそがしいの。 
\\	今は忙しいのだ。	いまはいそがしいのだ。 
\\	今は忙しいの?	いまはいそがしいの? 
\\	ジムのだ。	ジムのだ。 
\\	ジムなのだ。	ジムなのだ。 
\\	どこに行くの?/授業に行くんだ。	どこにいくの?/じゅぎょうにいくんだ。 
\\	今、授業があるんじゃない?/今は、ないんだ。	いま、じゅぎょうがあるんじゃない?/いまは、ないんだ。 
\\	今、授業がないんじゃない?/ううん、ある。	いま、じゅぎょうがないんじゃない?/ううん、ある。 
\\	その人が買うんじゃなかったの?/ううん、先生が買うんだ。	そのひとがかうんじゃなかったの?/ううん、せんせいがかうんだ。 
\\	朝ご飯を食べるんじゃなかった。/どうして?	あさごはんをたべるんじゃなかった。/どうして? 
\\	ボブは朝ご飯を早く食べた。	ボブはあさごはんをはやくたべた。 
\\	アリスは自分の部屋をきれいにした。	アリスはじぶんのへやをきれいにした。 
\\	映画をたくさん見た。	えいがをたくさんみた。 
\\	最近、全然食べない。	さいきん、ぜんぜんたべない。 
\\	ボブの声は、結構大きい。	ボブのこえは、けっこうおおきい。 
\\	この町は、最近大きく変わった。	このまちほ、さいきんおおきくかわった。 
\\	図書館の中では、静かにする。	としょかんのなかでは、しずかにする。 
\\	いい天気だね。/そうね。	いいてんきだね。/そうね。 
\\	おもしろい映画だったね。/え?全然おもしろくなかった。	おもしろいえいがだったね。/え?ぜんぜんおもしろくなかった。 
\\	時間がないよ。/大丈夫だよ。	じかんがないよ。/だいじょうぶだよ。 
\\	今日はいい天気だね。/うん。でも、明日雨が降るよ。	きょうはいいてんきだね。/うん。でも、あしたあめがふるよ。 
\\	ボブは、魚が好きなんだよね。/そうだね。	ボブは、さかながすきなんだよね。/そうだね。 
\\	明日、映画を見に行く。	あした、えいがをみにいく。 
\\	昨日、友達が遊びにきた。	きのう、ともだちがあそびにきた。 
\\	明日、大学に行きます。	あした、だいがくにいきます。 
\\	先週、ボブに会いましたよ。	せんしゅう、ボブにあいましたよ。 
\\	晩ご飯を食べませんでしたね。	ばんごはんをたべませんでしたね。 
\\	面白くない映画は見ません。	おもしろくないえいがほみません。 
\\	子犬はとても好きです。	こいぬはとてもすきです。 
\\	昨日、時間がなかったんです。	きのう、じかんがなかったんです。 
\\	その部屋はあまり静かじゃないです。	そのへやはあまりしずかじゃないです。 
\\	先週に見た映画は、とても面白かったです。	せんしゅうにみたえいがは、とてもおもしろかったです。 
\\	その部屋はあまり静かじゃないですよ。	そのへやはあまりしずかじゃないですよ。 
\\	その部屋はあまり静かじゃありませんよ。	そのへやはあまりしずかじゃありませんよ。 
\\	私の名前はキムです。	わたしのなまえはキムです。 
\\	僕の名前はキムです。	ぼくのなまえはキムです。 
\\	僕の名前はボブだ。	ぼくのなまえはボブだ。 
\\	俺の名前はボブだ。	おれのなまえはボブだ。 
\\	あたしの名前はアリス。	あたしのなまえはアリス。 
\\	お母さんはどこですか。母は買い物に行きました。	おかあさんはどこですか。はははかいものにいきますた。 
\\	イタリア料理を食べに行きませんか。/すみません。ちょっと、お腹がいっぱいです。	イタリアりょうりをたべにいきませんか。/すみません。ちょっと、おなかがいっぱいです。 
\\	こんなのを本当に食べるか?	こんなのをほんとうにたべるか? 
\\	そんなのは、あるかよ!	そんなのは、あるかよ! 
\\	こんなのを本当に食べる?	こんなのをほんとうにたべる? 
\\	そんなのは、あるの?	そんなのは、あるの? 
\\	昨日何を食べたか忘れた。	きのうなにをたべたかわすれた。 
\\	彼は何を言ったかわからない。	かれはなにをいったかわからない。 
\\	先生が学校に行ったか教えない?	せんせいががっこうにいったかおしえない? 
\\	先生が学校に行ったかどうか知らない。	せんせいががっこうにいったかどうかしらない。 
\\	先生が学校に行ったか行かなかったか知らない。	せんせいががっこうにいったかいかなかったかしらない。 
\\	誰かがおいしいクッキーを全部食べた。	だれかがおいしいクッキーをぜんぶたべた。 
\\	誰が盗んだのか、誰か知りませんか。	だれがぬすんだのか、だれかしりませんか。 
\\	犯人をどこかで見ましたか。	はんにんをどこかでみましたか。 
\\	この中からどれかを選ぶの。	このなかからどれかをえらぶの。 
\\	この質問の答えは、誰も知らない。	このしつもんのこたえは、だれもしらない。 
\\	友達はいつも遅れる。	ともだちはいつもおくれる。 
\\	ここにあるレストランはどれもおいしくない。	ここにあるレストランはどれもおいしくない。 
\\	今週末は、どこにも行かなかった。	こんしゅうまつは、どこにもいかなかった。 
\\	この質問の答えは、誰でも分かる。	このしつもんのこたえは、だれでもわかる。 
\\	昼ご飯は、どこでもいいです。	ひるごはんは、どこでもいいです。 
\\	あの人は、本当に何でも食べる。	あのひとは、ほんとうになんでもたべる。 
\\	私の部屋は、きれいで、静かで、とても好き。	わたしのへやは、きれいで、しずかで、とてもすき。 
\\	彼女は、学生じゃなくて、先生だ。	彼女は、がくせいじゃなくて、せんせいだ。 
\\	田中さんは、お金持ちで、かっこよくて、魅力的ですね。	たなかさんは、おかねもちで、かっこよくて、みりょくてきですね。 
\\	食堂に行って、昼ご飯を食べて、昼寝をする。	しょくどうにいって、ひるごはんをたべて、ひるねをする。 
\\	食堂に行って、昼ご飯を食べて、昼寝をした。	しょくどうにいって、ひるごはんをたべて、ひるねをした。 
\\	時間がありまして、映画を見ました。	じかんがありまして、えいがをみました。 
\\	時間がなかったからパーティーに行きませんでした。	じかんがなかったからパーティーにいきませんでした。 
\\	友達からプレゼントが来た。	ともだちからプレゼントがきた。 
\\	友達だからプレゼントが来た。	ともだちだからプレゼントがきた。 
\\	どうしてパーティーに行きませんでしたか。/時間がなかったからです。	どうしてパーティーにいきませんでしたか。/じかんがなかったからです。 
\\	パーティーに行かなかったの?/うん、時間がなかったから。	パーティーにいかなかったの?/うん、じかんがなかったから。 
\\	時間がなかった。/だからパーティーに行かなかったの?	じかんがなかった。/だからパーティーにいかなかったの? 
\\	ちょっと忙しいので、そろそろ失礼します。	ちょっといそがしいので、そろそろしつれいします。 
\\	私は学生なので、お金がないんです。	わたしはがくせいなので、おかながないんです。 
\\	ここは静かなので、とても穏やかです。	ここはすずかなので、とてもおだやかです。 
\\	なので、友達に会う時間がない。	なので、ともだちにあうじかんがない。 
\\	時間がなかったんでパーティーに行かなかった。	じかんがなかったんでパーティーにいかなかった。 
\\	ここは静かなんで、とても穏やかです。	ここはしずかなんで、とてもおだやかです。 
\\	なんで、友達に会う時間がない。	なんで、ともだちにあうじかんがない。 
\\	毎日運動したのに、全然痩せなかった。	まいにちうんどうしたのに、ぜんぜんやせなかった。 
\\	学生なのに、彼女は勉強しない。	がくせいなのに、かのじょはべんきょうしない。 
\\	デパートに行きましたが、何も欲しくなかったです。	デパートにいきましたが、なにもほしくなかったです。 
\\	友達に聞いたけど、知らなかった。	ともだちにきいたけど、しらなかった。 
\\	今日は暇だけど、明日は忙しい。	きょうはひまだけど、あしたはいそがしい。 
\\	だけど、彼がまだ好きなの。	だけど、かれがまだすきなの。 
\\	デパートに行きましたが、いい物がたくさんありました。	デパートにいきましたが、いいものがたくさんありました。 
\\	マトリックスを見たけど、面白かった。	"マトリックスをみたけど、おもしろかった。 
\\	どうして友達じゃないんですか?/先生だし、年上だし・・・。	どうしてともだちじゃないんですか?/せんせいだし、ねんうえだし・・・。 
\\	どうして彼が好きなの?/優しいし、かっこいいし、面白いから。	どうしてかれがすきなの?/やさしいし、かっこいいし、おもしろいから。 
\\	映画を見たり、本を読んだり、昼寝したりする。	えいがをみたり、ほんをやんだり、ひるねしたりする。 
\\	この大学の授業は簡単だったり、難しかったりする。	このだいがくのじゅぎょうはかんたんだったり、むずかしかったりする。 
\\	映画を見たり、本を読んだりした。	えいがをみたり、ほんをよんだりした。 
\\	映画を見たり、本を読んだりしない。	えいがをみたり、ほんをよんだりしない。 
\\	映画を見たり、本を読んだりしなかった。	えいがをみたり、ほんをよんだりしなかった。 
\\	友達は何をしているの?/昼ご飯を食べている。	ともだちはなにをしているの?/ひるごはんをたべている。 
\\	何を読んでいる?/教科書を読んでいます。	なにをよんでいる?/きょうかしょをよんでいます。 
\\	話を聞いていますか。/ううん、聞いていない。	はなしをきいていますか。/ううん、きいていない。 
\\	友達は何をしてるの?/昼ご飯を食べてる。	ともだちはなにをしてるの?/ひるごはんをたべている。 
\\	何を読んでる?/教科書を読んでいます。	なにをよんでる?/きょうかしょをよんでいます。 
\\	話を聞いていますか。/ううん、聞いてない。	はなしをきいていますか。/ううん、きいてない。 
\\	今日、知りました。	きょう、しりました。 
\\	この歌を知っていますか?	このうたをしっていますか。 
\\	道は分かりますか。	みちはわかりますか。 
\\	はい、はい、分かった、分かった。	はい、はい、わかった、わかった。 
\\	鈴木さんはどこですか。/もう、家に帰っている。	すずきさんはどこですか。/もう{うち
\\	えい}にかえっている。 
\\	先に行っているよ。/美恵ちゃんは、もう来ているよ。	さきにいっているよ。/みえちゃんは、もうきているよ。 
\\	準備はどうですか。準備は、もうしてあるよ。	じゅんびはどうですか。/じゅんびは、もうしてあるよ。 
\\	旅行の計画は終った?/うん、切符を買ったし、ホテルの予約もしてある。	りょこうのけいかくはおわった?/うん、きっぷをかったし、ホテルのよやくもしてある。 
\\	晩ご飯を作っておく。	ばんごはんをつくっておく。 
\\	電池を買っておきます。	でんちをかっておきます。 
\\	晩ご飯を作っとく。	ばんごはんをつくっとく。 
\\	電池を買っときます。	だんちをかっときます。 
\\	鉛筆を持っている?	えんぴつをもっている。 
\\	鉛筆を学校へ持っていく?	えんぴつをがっこうへもっていく? 
\\	鉛筆を家に持ってくる?	えんぴつを{うち
\\	いえ}にもってくる。 
\\	お父さんは、早く帰ってきました。	おとうさんは、はやくかえってきました。 
\\	駅の方へ走っていった。	えきのほうへはしっていった。 
\\	冬に入って、コートを着ている人が増えていきます。	ふゆにはいって、コートをきているひとがふえていきます。 
\\	一生懸命、頑張っていく!	いっしょうけんめい、がんばっていく! 
\\	色々な人と付き合ってきたけど、いい人はまだ見つからない。	いろいろなひととつきあってきたけど、いいひとはまだみつからない。 
\\	日本語をずっと前から勉強してきて、結局はやめた。	にほんをずっとまえからべんきょうしてきて、けっきょくはやめた。 
\\	漢字は書けますか?	かんじはかけますか。 
\\	残念だが、今週末は行けない。	ざんねんだが、こんしゅうまつはいけない。 
\\	もう信じられない。	もうしんじられない。 
\\	富士山が登れた。	ふじさんがのぼれた。 
\\	重い荷物が持てます。	おもいにもつがもてます。 
\\	今日は晴れて、富士山が見える。	きょうははれて、ふじさんがみえる。 
\\	友達のおかげで、映画はただで見られた。	ともだちのあかげで、えいがはただでみられた。 
\\	友達のおかげで、映画をただで見ることができた。	ともだちのおかげで、えいがをただでみることができた。 
\\	久しぶりに彼の声が聞けた。	ひさしぶりにかれのこえがきけた。 
\\	周りがうるさくて、彼が言っていることがあんまり聞こえなかった。	まわりがうるさくて、かれがいっていることがあんまりきこえなかった。 
\\	そんなことはありうる。	そんなことはありうる。 
\\	そんなことはありえる	そんなことはありえる 
\\	そんなことはありえない。	そんなことはありえない。 
\\	彼が寝坊したこともありうるね。	かれがねぼうしたこともありうるね。 
\\	それは、ありえない話だよ。	それは、ありえないはなしだよ。 
\\	彼の日本語が上手になった。	かれのにほんががじょうずになった。 
\\	私は医者になった。	わたしはいしゃになった。 
\\	私は有名な人になる。	わたしはゆうめいなひとになる。 
\\	私は、ハンバーガーとサラダにします。	わたしは、ハンバーガーとサラダにします。 
\\	他にいいものがたくさんあるけど、やっぱりこれにする。	ほかにいいものがたくさんあるけど、やっぱりこれにする。 
\\	去年から背が高くなったね。	きょねんからせがたかくなったね。 
\\	運動しているから、強くなる。	うんどうしているから、つよくなる。 
\\	勉強をたくさんしたから、頭がよくなった。	べんきょうをたくさんしたから、あたまがよくなった。 
\\	海外に行くことになった。	かいがいにいくことになった。 
\\	毎日、肉を食べるようになった。	まいにち、にくをたべるようになった。 
\\	海外に行くことにした。	かいがいにいくことにした。 
\\	毎日、肉を食べるようにする。	まいにち、にくをたべるようにする。 
\\	日本に来て、寿司が食べられるようになった。	にほんにきて、すしがたべられるようになった。 
\\	一年間練習したから、ピアノが弾けるようになった。	いちねんかんれんしゅうしたから、ピアノがひけるようになった。 
\\	地下に入って、富士山が見えなくなった。	ちかにはいって、ふじさんがみえなくなった。 
\\	ボールを落すと落ちる。	ボールをおとすとおちる。 
\\	電気を消すと暗くなる。	でんきをけすとくらくなる。 
\\	学校に行かないと友達と会えないよ。	がっこうにいかないとともだちとあえないよ。 
\\	たくさん食べると太るよ。	たくさんたべるとふとるよ。 
\\	先生だと、きっと年上なんじゃないですか?	せんせいだと、きっととしうえなんじゃないですか? 
\\	みんなが行くなら私も行く。	みんながいくならわたしもいく。 
\\	アリスさんが言うなら問題ないよ。	アリスさんがいうならもんだいないよ。 
\\	図書館はどこですか。/図書館なら、あそこです。	としょかんはどこですか。/としょかんなら、あそこです。 
\\	友達に会えれば、買い物に行きます。	ともだちにあえれば、かいものにいきます。 
\\	お金があればいいね。	おかねがあればいいね。 
\\	楽しければ、私も行く。	たのしければ、わたしもいく。 
\\	楽しくなければ、私も行かない。	たのしくなければ、わたしもいかない。 
\\	食べなければ病気になるよ。	たべなければびょうきになるよ。 
\\	暇だったら、遊びに行くよ。	ひまだったら、あそびにいくよ。 
\\	学生だったら、学生割引で買えます。	がくせいだったら、がくせいわりびきでかえます。 
\\	友達に会えたら、買い物に行きます。	ともだちにあえたら、かいものにいきます。 
\\	お金があったらいいね。	おかねがったらいいね。 
\\	家に帰ったら、誰もいなかった。	いえにかえったら、だれもいなかった。 
\\	アメリカに行ったら、たくさん太りました。	アメリカにいったら、たくさんふとりました。 
\\	もしよかったら、映画を観に行きますか?	もしよかったら、えいがをみにいきますか? 
\\	もし時間がないなら、明日でもいいよ。	もしじかんがないなら、あしたでもいいよ。 
\\	ここに入ってはいけません。	ここにはいってはいけません。 
\\	それを食べてはだめ!	それをたべてはだめ! 
\\	夜、遅くまで電話してはならない。	よる、おそくまででんわしてはならない。 
\\	早く寝てはなりませんでした。	はやくねてはなりませんでした。 
\\	毎日学校に行かなくてはなりません。	まいにちがっこうにいかなくてはなりません。 
\\	宿題をしなくてはいけなかった。	しゅくだいをしなくてはいけなかった。 
\\	毎日学校に行かないとだめです。	まいにちがっこうにいかないとだめです。 
\\	宿題をしないといけない。	しゅくだいをしないといけない。 
\\	毎日学校に行かなければいけません。	まいにちがっこうにいかなければいけません。 
\\	宿題をしなければだめだった。	しゅくだいをしなければだめだった。 
\\	勉強しなくちゃ。	べんきょうしなくちゃ。 
\\	ご飯を食べなきゃ。	ごはんをたべなきゃ。 
\\	学校に行かないと。	がっこうにいかないと。 
\\	ここに入っちゃだめだよ。	ここにはいっちゃだめだよ。 
\\	死んじゃだめだよ!	しんじゃだめだよ。 
\\	全部食べてもいいよ。	ぜんぶたべてもいいよ。 
\\	全部食べなくてもいいよ。	ぜんぶたべなくてもいいよ。 
\\	全部飲んでも大丈夫だよ。	ぜんぶのんでもだいじょうぶだよ。 
\\	全部飲んでも構わないよ。	ぜんぶのんでもかまわないよ。 
\\	もう帰っていい?	もうかえっていい? 
\\	これ、ちょっと見ていい?	これ、ちょっとみていい? 
\\	何をしたいですか。	なにをしたいですか。 
\\	温泉に行きたい。	おんせんにいきたい。 
\\	ケーキ、食べたくないの?	ケーキ、たべたくないの? 
\\	食べたくなかったけど食べたくなった。	たべたくなかったけどたべtくなった。 
\\	ずっと一緒にいたい。	ずっといっしょにいたい。 
\\	犬と遊びたいですか。	いぬとあそびたいですか。 
\\	大きい縫いぐるみが欲しい!	おおきいぬいぐるみがほしい! 
\\	全部食べてほしいんだけど・・・。	ぜんぶたべてほしいんだけど・・・。 
\\	部屋をきれいにしてほしいのよ。	へやをきれいにしてほしいのよ。 
\\	今日は何をしようか?/テーマパークに行こう!	あしたはなにをしようか?/テーマパークにいこう! 
\\	明日は何を食べようか?/カレーを食べよう!	あしたはなにをたべようか?/カレーをたべよう! 
\\	今日は何をしましょうか?/テーマパークに行きましょう!	あしたはなにをしましょうか?/テーマパークにいきましょう! 
\\	明日は何を食べましょうか?/カレーを食べましょう!	あしたはなにをたべましょうか?/カレーをたべましょう! 
\\	銀行に行ったらどうですか。	ぎんこうにいったらどうですか。 
\\	たまにご両親と話せばどう?	たまにごりょうしんとはなせばどう? 
\\	アリスが、「寒い」と言った。	"アリスが、「さむい」といった。 
\\	「今日は授業がない」と先生から聞いたんだけど。	"「きょうはじゅぎょうがない」とせんせいからきいたんだけど。 
\\	「寒い」とアリスが田中に言った。	"「さむい」とアリスがたなかにいった。 
\\	先生から今日は授業がないと聞いたんだけど。	せんせいからきょうはじゅぎょうがないときいたんだけど。 
\\	これは、日本語で何と言いますか。	これは、にほんごでなんといいますか。 
\\	私は、アリスと言います。	わたしは、アリスといいます。 
\\	カレーを食べようと思ったけど、食べる時間がなかった。	カレーをたべようとおもったけど、たべるじかんがなかった。 
\\	今、どこに行こうかと考えている。	いま、どこにいこうかとかんがえている。 
\\	彼は、これは何だと言いましたか。	かれは、これはなんだといいましたか。 
\\	彼は高校生だと聞いたけど、信じられない。	かれはこうこうせいだときいたけど、しんじられない。 
\\	全部食べさせた。	ぜんぶたべさせた。 
\\	全部食べさせてくれた。	せんぶたべさせてくれた。 
\\	先生が学生に宿題をたくさんさせた。	先生ががくせいにしゅくだいをたくさんさせた。 
\\	先生が質問をたくさん聞かせてくれた。	せんせいがしつもんをたくさんきかせてくれた。 
\\	今日は仕事を休ませてください。	きょうはしごとをやすませてください。 
\\	その部長は、よく長時間働かせる。	そのぶちょうは、よくちょうじかんはたらかせる。 
\\	トイレに行かせてくれますか。	トイレにいかせてくれますか。 
\\	トイレに行ってもいいですか。	といれにいってもいいですか。 
\\	同じことを何回も言わすな!	おなじことをなんかいもいわすな! 
\\	お腹空いているんだから、なんか食べさしてくれよ。	おなかあいているんだから、なんかたべさしてくれよ。 
\\	ポリッジが誰かに食べられた!	ポリッジがだれかにたべられた。 
\\	みんなに変だと言われます。	みんなにへんだといわれます。 
\\	光の速さを超えるのは、不可能だと思われる。	ひかりのはやさをこえるのは、ふかのうだとおもわれる。 
\\	この教科書は多くの人に読まれている。	このきょうかしょはおおくのひとによまれている。 
\\	外国人に質問を聞かれたが、答えられなかった。	がいこくじんにしつもんをきかれたが、こたえられなかった。 
\\	このパッケージには、あらゆるものが含まれている。	このパッケージには、あらゆるものがふくまれている。 
\\	レシートはどうされますか?	レシートはどうされますか? 
\\	明日の会議に行かれるんですか?	あしたのかいぎにいかれるんですか? 
\\	朝ご飯は食べたくなかったのに、食べさせられた。	あさごはんはたべたくなかったのに、たべさせられた。 
\\	日本では、お酒を飲ませられることが多い。	にほんでは、おさけをのませられることがおおい。 
\\	あいつに二時間も待たせられた。	あいつににじかんもまたせられた。 
\\	親に毎日宿題をさせられる。	おやにまいにちしょくだいをさせられる。 
\\	学生が廊下に立たされた。	がくせいがろうかにたたされた。 
\\	日本では、お酒を飲まされることが多い。	にほんでは、おさけをのまされることがおおい。 
\\	あいつに二時間も待たされた。	あいつににじかんもまたされた。 
\\	アリスさん、もう召し上がりましたか。	アリスさん、もうめしあがりましたか。 
\\	仕事で何をなさっているんですか。	しごとでなにをなさっているんですか。 
\\	推薦状を書いてくださるんですか。	すいせんじょうをかいてくださるんですか。 
\\	どちらからいらっしゃいましたか。	どちらからいらっしゃいましたか。 
\\	今日は、どちらへいらっしゃいますか。	きょうは、どちらへいらっしゃいますか。 
\\	私はキムと申します。	わたしはキムともうします。 
\\	私が書いたレポートを見ていただけますか。	わたしがかいたレポートをみていただけますか。 
\\	失礼致します。	しつれいします。 
\\	こちらは、私の部屋です。	こちらは、わたしのへやです。 
\\	こちらは、私の部屋でございます。	こちらは、わたしのへやでございます。 
\\	お手洗いはこのビルの二階にあります。	おてあらいはこのビルのにかいにあります。 
\\	お手洗いはこのビルの二階にございます。	おてあらいはこのビルのにかいにございます。 
\\	先生はお見えになりますか。	せんせいほかえになりますか。 
\\	もうお帰りですか。	もうおかえりですか。 
\\	店内でお召し上がりですか。	てんないでおめしあがりですか。 
\\	少々お待ちください。	しょうしょうおたちください。 
\\	こちらにご覧下さい。	こちらにごらんください。 
\\	閉まるドアにご注意下さい。	しまるドアにごちゅういください。 
\\	先生、お聞きしたいことがありますが。	せんせい、おききしたいことがありますが。 
\\	すみません、お待たせしました。	すみません、おまたせしました。 
\\	千円からお預かりいたします。	せんえんからおあずかりいたします。 
\\	いらっしゃいませ。	いらっしゃいませ。 
\\	いらっしゃい!	いらっしゃい! 
\\	ありがとうございました。またお越しくださいませ。	ありがとうございました。またおこしくださいませ。 
\\	どうぞ、ごゆっくりなさいませ。	どうぞ、ごゆっくりなさいませ。 
\\	宿題をやった?/しまった!	しゅくだいをやった?/しまった! 
\\	そのケーキを全部食べてしまった。	そのケーキをぜんぶたべてしまった。 
\\	毎日ケーキを食べて、2キロ太ってしまいました。	まいにちケーキをたべて、2キロふとってしまいました。 
\\	ちゃんと食べないと、痩せてしまいますよ。	ちゃんとたべないと、やせてしまいますよ。 
\\	結局、嫌なことをさせてしまった。	けっきょく、いやなことをさせてしまった。 
\\	ごめん、待たせてしまって!	ごめん、またせてしまって! 
\\	金魚がもう死んでしまった。	きんぎょもうしんでしまった。 
\\	金魚がもう死んじゃった。	きんぎょがもうしんじゃった。 
\\	もう帰っちゃっていい?	もうかえちゃっていい? 
\\	みんな、どっか行っちゃったよ。	みんな、どっかいっちゃったよ。 
\\	そろそろ遅くなっちゃうよ。	そろそろおそくなっちゃうよ。 
\\	また遅刻しちまったよ。	またちこくしちまったよ。 
\\	ごめん、ついお前を呼んじまった。	ごめん、ついおまえをよんじまった。 
\\	宿題をやってしまいなさい。	しゅくだいをやってしまいなさい。 
\\	徹夜して、宿題することはある。	てつやして、しゅくだいすることはある。 
\\	一人で行くことはありません。	ひとりでいくことはありません。 
\\	パリに行ったことはありますか。	バリにいったことはありますか。 
\\	お寿司を食べたことがある。	おすしをたべたことがある。 
\\	日本の映画を観たことないの?	にほんのえいがをみたことないの? 
\\	ヨーロッパに行ったことがあったらいいな。	ヨーロッパにいったことがあったらいいな。 
\\	そういうのを見たことがなかった。	そういうのをみたことがなかった。 
\\	一度行ったこともないんです。	いちどいったこともないんです。 
\\	早くきて。映画は、今ちょうどいいところだよ。	はやくきて。えいがは、いまちょうどいいところだよ。 
\\	彼は、優しいところもあるよ。	かれは、やさしいところもあるよ。 
\\	今は授業が終ったところです。	いまはじゅぎょうがおわったところです。 
\\	これから行くところでした。	これからいくところでした。 
\\	どうしてこなかったの?/授業があったの。	どうしてこなっかったの?/じゅぎょうがあったの。 
\\	どうしてこなかったの?/授業があったもの。	どうしてこなっかったの?/じゅぎょうがあったもの。 
\\	どうしてこなかったの?/授業があったもん。	どうしてこなっかったの?/じゅぎょうがあったもん。 
\\	スミスさんは食堂に行ったかもしれません。	スミスさんはしょくどうにいったかもしれません。 
\\	雨で試合は中止になるかもしれないね。	あめでしあいはちゅうしになるかもしれないね。 
\\	この映画は一回見たことあるかも!	このえいがはいっかいみたことあるかも! 
\\	あそこが代々木公園かもしれない。	あそこがよよぎこうえんかもしれない。 
\\	もう逃げられないかもしれんぞ。	もうにげられないかもしれんぞ。 
\\	明日も雨でしょう。	あしたもあめでしょう。 
\\	あなたは、学生さんでしょうか。	あなたは、がくせいさんでしょうか。 
\\	これからどこへ行くんでしょうか?	これからどこへいくんでしょうか? 
\\	休ませていただけますでしょうか。	やすませていただけますでしょうか。 
\\	あっ!遅刻しちゃう!/だから、時間がないって言ったでしょう!	あっ!ちこくしちゃう!/だから、じかんがないっていったでしょう! 
\\	これから食べに行くんでしょ。/だったら?	これからたべにいくんでしょ。/だったら? 
\\	掃除、手伝ってくれるでしょう。/え?そうなの?	そうじ、てつだってくれるでしょう。/え?そうなの? 
\\	アリスはどこだ?/もう寝ているだろう。	アリスはどこだ?もうねているだろう。 
\\	もう家に帰るんだろう。/そうよ。	もう{うち
\\	いえ}にかえるんだろう。/そうよ。 
\\	りんごだけ。	りんごだけ。 
\\	これとそれだけ。	これとそれだけ。 
\\	それだけは、食べないでください。	それだけは、たべないでください。 
\\	この歌だけを歌わなかった。	このうただけをうたわなかった。 
\\	その人だけが好きだったんだ。	そのひとだけがすきだったんだ。 
\\	この販売機だけでは、500円玉が使えない。	このはんばいきだけは、500えんだまがつかえない。 
\\	小林さんからだけには、返事が来なかった。	こばやしさんからだけには、へんじがこなかった。 
\\	準備が終わったから、これからは食べるだけだ。	じゅんびがおわったから、これからはたべるだけだ。 
\\	ここに名前を書くだけでいいですか?	ここになまえをかくだけでいいですか? 
\\	この乗車券は発売当日のみ有効です。	このじょうしゃけんははつばいとうじつのみゆうこうです。 
\\	アンケート対象は大学生のみです。	アンケートたいしょうはだいがくのみです。 
\\	これしかない。	これしかない。 
\\	これだけ見る。	これだけみる。 
\\	これだけ見ない。	これだけみない。 
\\	これしか見ない。	これしかみない。 
\\	今日は忙しくて、朝ご飯しか食べられなかった。	きょうはいそがしくて、あさごはんしかたべられなかった。 
\\	全部買うの?/ううん、これだけ。	ぜんぶかうの?/ううん、これだけ。 
\\	全部買うの?/ううん、これしか買わない。	ぜんぶかうの?/ううん、これしかかわない。 
\\	アリスからしか何ももらってない。	アリスからしかなにももらってない。 
\\	これから頑張るしかない!	これからがんばるしかない! 
\\	こうなったら、逃げるしかない。	こうなったら、にげるしかない。 
\\	もう腐っているから、捨てるしかないよ。	もうくさっているから、すてるしかないよ。 
\\	これは買うっきゃない!	これはかうっきゃない! 
\\	こうなったら、もうやるっきゃない!	こうなったら、もうやるっきゃない! 
\\	何だよ!おばさんばっかりじゃないか?	なんだよ!おばさんばっかりじゃないか? 
\\	いやだ。おばさんばっかり。	いやだ。おばさんばっかり。 
\\	崇君は漫画ばっかり読んでてさ。かっこ悪い。	たかしくんはまんがばっかりよんでてさ。かっこわるい。 
\\	彼は麻雀ばかりです。	かれはマージャンばかりです。 
\\	直美ちゃんと遊ぶばっかりでしょう!	なおみちゃんとあそぶばっかりでしょう! 
\\	最近は仕事ばっかだよ。	さいきんはしごとばっかだよ。 
\\	佐藤さんは料理が上手で、また食べ過ぎました。	さとうさんはりょうりがじょうずで、またたべすぎました。 
\\	お酒を飲みすぎないように気をつけてね。	おさけをのみすぎないようにきをつけてね。 
\\	大きすぎるからトランクに入らないぞ。	おおきすぎるからトランクに入らないぞ。 
\\	静かすぎる。罠かもしれないよ。	しずかすぎる。わなかもしれないよ。 
\\	時間が足りなさすぎて、何もできなかった。	じかんがたりなさすぎて、なにもできなかった。 
\\	彼には、彼女がもったいなさすぎるよ。	かれには、かのじょがもったいなさすぎるよ。 
\\	昨晩のこと、全然覚えてないな。/それは飲みすぎだよ。	さくばんのこと、ぜんぜんおぼえてないな。/せれはのみすぎだよ。 
\\	昨日、電話三回もしたよ!	きのう、でんわさんかいもしたよ! 
\\	試験のために三時間も勉強した。	しけんのためにさんじかんもべんきょうした。 
\\	今年、十キロも太っちゃった!	ことし、じゅっキロもふとちゃった! 
\\	今日の天気はそれほど寒くない。	きょうのてんきはそれほどさむくない。 
\\	寝る時間がないほど忙しい。	ねるじかんがないほどいそがしい。 
\\	韓国料理は食べれば食べるほど、おいしくなる。	かんこくりょうりはたべればたべるほど、おいしくなる。 
\\	歩いたら歩くほど、迷ってしまった。	あるいたらあるくほど、まよってしまった。 
\\	勉強をすればするほど、頭がよくなるよ。	べんきょうをすればするほど、あたまがよくなるよ。 
\\	は、ハードディスクの容量が大きければ大きいほどもっとたくさんの曲が保存できます。	
\\	は、ハードディスクのようりょうがおおきければおおきいほどもっとたくさんのきょくがほぞんできます。 
\\	航空券は安ければ安いほどいいとは限らない。	こうくうけんはやすければやすいほどいいとはかぎらない。 
\\	文章は、短ければ短いほど、簡単なら簡単なほどよいです。	ぶんしょうは、みじかければみじかいほど、かんたんならかんたんなほどよいです。 
\\	このビルの高さは何ですか?	このビルのたかさはなんですか? 
\\	犬の聴覚の敏感さを人間と比べると、はるかに上だ。	いぬのちょうかくのびんかんさをにんげんとくらべると、はるかにうえだ。 
\\	ここには、誰もいないようだ。	ここには、だれもいないようだ。 
\\	映画を観たようです。	えいがをみたようです。 
\\	学生のようだ。	がくせいのようだ。 
\\	ここは静かなようだ。	ここはしずかなようだ。 
\\	あの人を見たような気がした。	あのひとをみたようなきがした。 
\\	彼は学生のような雰囲気ですね。	かれはがくせいのようなふんいきですね。 
\\	ちょっと怒ったように聞こえた。	ちょっとおこったようにきこえた。 
\\	何も起こらなかったように言った。	なにもおこらなかったようにいった。 
\\	もう売り切れみたい。	もううりきれみたい。 
\\	制服を着ている姿をみると、学生みたいです。	せいふくをきているすがたをみると、がくせいみたいです。 
\\	このピザはお好み焼きみたいじゃない?	このピザはおこのみやきみたいじゃない? 
\\	喫茶店に行くみたいだった。	きっさてんにいくみたいだった。 
\\	秘密を教えてくれるみたいじゃなかった?	ひみつをおしえてくれるみたいじゃなかった? 
\\	もう売り切れのようだ。	もううりきれのようだ。 
\\	このピザはお好み焼きのように見える。	このピザはおこのみやきのようにみえる。 
\\	バランスが崩れて、一瞬倒れそうだった。	バランスがくずれて、いっしゅんたおれそうだった。 
\\	この辺りにありそうだけどな。	このあたりにありそうだけどな。 
\\	この漬物はおいしそう!	このつけものはおいしそう! 
\\	これも結構よさそうだけど、やっぱり高いよね。	これもけっこうよさそうだけど、やっぱりたかいよね。 
\\	お前なら、金髪の女が好きそうだな。	おまえなら、きんぱつのおんながすきそうだな。 
\\	もう10時になったから、来なさそうだね。	もう10じになったから、こなさそうだね。 
\\	これはただの試合じゃなさそうだ。	これはただのしあいじゃなさそうだ。 
\\	その人は学生でしょう。	そのひとはがくせいでしょう。 
\\	その人は学生だろう。	そのひとはがくせいだろう。 
\\	この犬はかわいそう。	このいぬはかわいそう。 
\\	この犬はかわいい。	このいぬ犬はかわいい。 
\\	明日、雨が降るそうだ。	あした、あめがふるそうだ。 
\\	毎日会いに行ったそうです。	まいにちあいにいったそうです。 
\\	彼は、高校生だそうです。	かれは、こうこうせいだそうです。 
\\	今日、田中さんはこないの?/だそうです。	きょう、たなかさんはこないの?/だそうです。 
\\	今日、田中さんはこないの?/こないらしい。	きょう、たなかさんはこないの?/こないらしい。 
\\	あの人は何なの?/美由紀さんの友達らしいですよ。	あのひとはなんなの?/みゆきさんのともだちらしいですよ。 
\\	あの子は子供らしくない。	あのこはこどもらしくない。 
\\	大人らしくするつもりだったのに、大騒ぎしてしまった。	おとならしくするつもりだったのに、おおさわぎしてしまった。 
\\	あの人はちょっと韓国人っぽいよね。	あのひとはちょっとかんこくじんっぽいよね。 
\\	みんなで、もう全部食べてしまったっぽいよ。	みんなで、もうぜんぶたべてしまったっぽいよ。 
\\	恭子は全然女っぽくないね。	きょうこはぜんぜんおんなっぽくないね。 
\\	ご飯の方がおいしい。	ごはんのほうがおいしい。 
\\	鈴木さんの方が若い。	すずきさんのほうがわかい。 
\\	学生じゃない方がいいよ。	がくせいじゃないほうがいいよ。 
\\	赤ちゃんは、静かな方が好き。	あかちゃんは、しずかなほうがすき。 
\\	ゆっくり食べた方が健康にいいよ。	ゆっくりたべたほうがけんこうにいいよ。 
\\	こちらから行った方が早かった。	こちらからいったほうがはやかった。 
\\	マトリックス・レボリューションを観ない方がいいよ。	"マトリックス・レボリューションをみないほうがいいよ。 
\\	そんなに飲まなかった方がよかった。	そんなにのまなかったほうがよかった。 
\\	花より団子。	はなよりだんご。 
\\	ご飯の方が、パンよりおいしい。	ごはんのほうが、パンよりおいしい。 
\\	キムさんより鈴木さんの方が若い。	キムさんよりすずきさんのほうがわかい。 
\\	毎日仕事に行くのが嫌だ。/仕事がないよりましだよ。	まいにちしごとにいくのがいやだ。/しごとがないよりましだよ。 
\\	ゆっくり食べた方が早く食べるよりいい。	ゆっくりたべたほうがはやくたべるよりいい。 
\\	商品の品質を何より大切にしています。	しょうひんのひんしつをなによりたいせつにしています。 
\\	この仕事は誰よりも早くできます。	このしごとはだれよりもはやくできます。 
\\	新宿の行き方は分かりますか。	しんじゅくのいきかたはわかりますか。 
\\	そういう食べ方は体によくないよ。	そういうたべかたはからだによくないよ。 
\\	漢字の書き方を教えてくれますか?	かんじのかきかたをおしえてくれますか? 
\\	吾輩は猫である。	わがはいはねこである。 
\\	混合物とは、2種類以上の純物質が混じりあっている物質である。	こんごうぶつとは、2しゅるいいじょうのじゅんぶっしつがまじりあっているぶっしつである。 
\\	国土交通省は2年後に利用率を70%まで引き上げる考えで、買い替え時に利用する気になるかどうかがカギになりそうだ。	こくどこうつうしょうは2ねんごにりようりつを70%までひきあげるかんがえで、かいかえときにりようするきになるかどうかがカギになりそうだ。 
\\	これは不公平ではないでしょうか。	これはふこうへいではないでしょうか。 
\\	言語は簡単にマスターできることではない。	げんごはかんたんにマスターできることではない。 
\\	花火は、火薬と金属の粉末を混ぜたものに火を付け、燃焼時の火花を楽しむためのもの。	はなびは、かやくときんぞくのふんまつをまぜたものにひをづけ、ねんしょうときのひばなをたのしむためのもの。 
\\	企業内の顧客データを利用し、彼の行方を調べることが出来た。	きぎょうないのこきゃくデータをりようし、かのなめがたをしらべることができた。 
\\	封筒には写真が数枚入っており、手紙が添えられていた。	ふうとうにはしゃしんがすうまいいっっており、てがみがそえられていた。 
\\	この旅館は、様々な新しい設備が備えており、とても快適だった。	このりょかんは、さまざまなあたらしいせつびがそなえており、とてもかいてきだった。 
\\	彼は漫画マニアだから、これらをもう全部読んだはずだよ。	かれはまんがマニアだから、これらをもうぜんぶよんだはずだよ。 
\\	この料理はおいしいはずだったが、焦げちゃって、まずくなった。	このりょうりはおいしいはずだったが、こげちゃって、まずくなった。 
\\	色々予定してあるから、今年は楽しいクリスマスのはず。	いろいろよていしてあるから、こんねんはたのしいクリスマスのはず。 
\\	そう簡単に直せるはずがないよ。	そうかんたんになおせるはずがないよ。 
\\	打合せは毎週2時から始まるはずじゃないですか?	うちあわせはまいしゅう2ときからはじまるはずじゃないですか? 
\\	何かを買う前に本当に必要かどうかをよく考えるべきだ。	なにかをかうまえにほんとうにひつようかどうかをよくかんがえるべきだ。 
\\	例え国のためであっても、国民を騙すべきではないと思う。	たとえくにのためであっても、こくみんをだますべきではないとおもう。 
\\	預金者が大手銀行を相手取って訴訟を起こすケースも出ており、金融庁は被害者の救済を優先させて、金融機関に犯罪防止対策の強化を促すべきだと判断。	よきんものがおおてぎんこうをあいてとってそしょうをおこすケースもでており、きんゆうちょうはひがいしゃのきゅうさいをゆうせんさせて、きんゆうきかんにはんざいぼうしたいさくのきょうかをうながすべきだとはんだん。 
\\	早く帰るべき。	はやくかえるべき。 
\\	早く帰るべく、準備をし始めた。	はやくかえるべく、じゅんびをしはじめた。 
\\	試験に合格すべく、皆一生懸命に勉強している。	しけんにごうかくすべく、みないっしょうけんめいにべんきょうしている。 
\\	今後もお客様との対話の窓口として、より充実していくべく努力してまいります。	こんごもおきゃくさまとのたいわのまどぐちとして、よりじゅうじつしていくべくどりょくしてまいります。 
\\	ゴミ捨てるべからず。	ゴミすてるべからず。 
\\	安全措置を忘れるべからず。	あんぜんそちをわすれるべからず。 
\\	宿題が多すぎて、トイレに行く時間さえなかった。	しゅくだいがおおすぎて、トイレにいくじかんさえなかった。 
\\	お金さえあれば、何でも出来るよ。	おきんさえあれば、なんでもできるよ。 
\\	お弁当を買うお金さえなかった。	おべんとうをかうおきんさえなかった。 
\\	私でさえ出来れば、あんたには楽ちんでしょう。	わたしでさえできれば、あんたにはらくちんでしょう。 
\\	ビタミンを食べさえすれば、健康が保証されますよ。	ビタミンをたべさえすれば、けんこうがほしょうされますよ。 
\\	自分の過ちを認めさえしなければ、問題は解決しないよ。	じぶんのあやまちをみとめさえしなければ、もんだいはかいけつしないよ。 
\\	この天才の私ですらわからなかった。	このてんさいのわたしですらわからなかった。 
\\	私は緊張しすぎて、ちらっと見ることすら出来ませんでした。	わたしはきんちょうしすぎて、ちらっとみることすらできませんでした。 
\\	「人」の漢字すら知らない生徒は、いないでしょ!	「にん」のかんじすらしらないせいとは、いないでしょ! 
\\	「人」 
\\	漢字はおろか、ひらがなさえ読めないよ!	かんじはおろか、ひらがなさえよめないよ! 
\\	結婚はおろか、2ヶ月付き合って、結局別れてしまった。	けっこんはおろか、2がつつきあって、けっきょくわかれてしまった。 
\\	大学はおろか、高校すら卒業しなかった。	だいがくはおろか、こうこうすらそつぎょうしなかった。 
\\	早くきてよ!何を恥ずかしがっているの?	はやくきてよ!なにをはずかしがっているの? 
\\	彼女は朝早く起こされるのを嫌がるタイプです。	かのじょはあさはやくおこされるのをいやがるタイプです。 
\\	うちの子供はプールに入るのを理由もなく怖がる。	うちのこどもはプールにいるのをりゆうもなくこわがる。 
\\	家に帰ったら、すぐパソコンを使いたがる。	いえにかえったら、すぐパソコンをつかいたがる。 
\\	みんなイタリアに行きたがってるんだけど、私の予算で行けるかどうかはとても怪しい。	みんなイタリアにいきたがってるんだけど、わたしのよさんでいけるかどうかはとてもあやしい。 
\\	妻はルイヴィトンのバッグを欲しがっているんだけど、そんなもん、買えるわけないでしょう!	つまはルイヴィトンのバッグをほしがっているんだけど、そんなもん、かえるわけないでしょう! 
\\	私は寒がり屋だから、ミネソタで暮らすのは辛かった。	わたしはさむがりやだから、ミネソタでくらすのはつらかった。 
\\	ボールは爆発せんばかりに、膨らんでいた。	ボールはばくはつせんばかりに、ふくらんでいた。 
\\	「あんたと関係ない」と言わんばかりに彼女は彼を無視していた。	"「あんたとかんけいない」といわんばかりにかのじょはかれをむししていた。 
\\	昨日のケンカで何も言わなかったばかりに、平気な顔をしている。	きのうのケンカでなにもいわなかったばかりに、へいきなかおをしている。 
\\	紅葉が始まり、すっかり秋めいた空気になってきた。	こうようがはじまり、すっかりあきめいたくうきになってきた。 
\\	そんな謎めいた顔をされても、うまく説明できないよ。	そんななぞめいたかおをされても、うまくせつめいできないよ。 
\\	いつも皮肉めいた言い方をしたら、みんなを嫌がらせるよ。	いつもひにくめいたいいかたをしたら、みんなをいやがらせるよ。 
\\	このテレビがこれ以上壊れたら、新しいのを買わざるを得ないな。	このテレビがこれいじょうこわれたら、あたらしいのをかわざるをえないな。 
\\	ずっと我慢してきたが、この状態だと歯医者さんに行かざるを得ない。	ずっとがまんしてきたが、このじょうたいだとはいしゃさんにいかざるをえない。 
\\	上司の話を聞くと、どうしても海外に出張をせざるを得ないようです。	じょうしのはなしをきくと、どうしてもかいがいにしゅっちょうをせざるをえないようです。 
\\	やむを得ない事由により手続が遅れた場合、必ずご連絡下さい。	やむをえないじゆうによりてつづきがおくれたばあい、かならずごれんらくください。 
\\	この仕事は厳しいかもしれませんが、最近の不景気では新しい仕事が見つからないのでやむを得ない状態です。	このしごとはきびしいかもしれませんが、さいきんのふけいきではあたらしいしごとがみつからないのでやむをえないじょうたいです。 
\\	この場ではちょっと決めかねますので、また別途会議を設けましょう。	このばではちょっときめかねますので、またべっとかいぎをもうけましょう。 
\\	このままでは、個人情報が漏洩しかねないので、速やかに対応をお願い致します。	このままでは、こじんじょうほうがろうえいしかねないので、すみやかにたいおうをおねがいいたします。 
\\	確定申告は忘れがちな手続のひとつだ。	かくていしんこくはわすれがちなてつづきのひとつだ。 
\\	留守がちなご家庭には、犬よりも、猫の方がおすすめです。	るすがちなごかていには、いぬよりも、ねこのほうがおすすめです。 
\\	父親は病気がちで、みんなが心配している。	ちちおやはびょうきがちで、みんながしんぱいしている。 
\\	テレビを見ながら、寝ちゃダメよ!	テレビをみながら、ねちゃダメよ! 
\\	二日酔いで痛む頭を押さえつつ、トイレに入った。	ふつかよいでいたむあたまをおさえつつ、トイレにいっった。 
\\	体によくないと思いつつ、最近は全然運動してない。	からだによくないとおもいつつ、さいきんはぜんぜんうんどうしてない。 
\\	電気製品の発展につれて、ハードディスクの容量はますます大きくなりつつある。	でんきせいひんのはってんにつれて、ハードディスクのようりょうはますますおおきくなりつつある。 
\\	今の日本では、終身雇用や年功序列という雇用慣行が崩れつつある。	いまのにっぽんでは、しゅうしんこようやねんこうじょれつというこようかんこうがくずれつつある。 
\\	多くの大学生は、締切日ぎりぎりまで、宿題をやらないきらいがある。	おおくのだいがくせいは、しめきりびぎりぎりまで、しゅくだいをやらないきらいがある。 
\\	コーディングが好きな開発者は、ちゃんとしたドキュメント作成と十分なテストを怠るきらいがある。	コーディングがすきなかいはつしゃは、ちゃんとしたドキュメントさくせいとじゅうぶんなテストをおこたるきらいがある。 
\\	相手は剣の達人だ。そう簡単には勝てまい。	あいてはつるぎのたつじんだ。そうかんたんにはかてまい。 
\\	そんな無茶な手段は認めますまい!	そんなむちゃなしゅだんはみとめますまい! 
\\	その時までは決して彼に会うまいと心に決めていた。	そのときまではけっしてかれにあうまいとこころにきめていた。 
\\	あの人は、二度と嘘をつくまいと誓ったのです。	あのにんは、にどとうそをつくまいとちかったのです。 
\\	明日に行くのをやめよう。	あしたにいくのをやめよう。 
\\	肉を食べないようにしている。	にくをたべないようにしている。 
\\	あいつが大学に入ろうが入るまいが、俺とは関係ないよ。	あいつがだいがくにいろうがいるまいが、おれとはかんけいないよ。 
\\	時間があろうがあるまいが、間に合わせるしかない。	じかんがあろうがあるまいが、まにあわせるしかない。 
\\	最近のウィルスは強力で、プログラムを実行しようがしまいが、ページを見るだけで感染するらしい。	さいきんのウィルスはきょうりょくで、プログラムをじっこうしようがしまいが、ページをみるだけでかんせんするらしい。 
\\	今後50年、人間が直面するであろう問題に正面から向き合って、自ら解決をはかりつつ、そのノウハウが次の産業となるシナリオを考えたい。	こんご50ねん、にんげんがちょくめんするであろうもんだいにしょうめんからむきあって、みずからかいけつをはかりつつ、そのノウハウがつぎのさんぎょうとなるシナリオをかんがえたい。 
\\	もちろん、生徒数減少の現在、学科の新設は困難であろうが、職業科の統廃合や科内コースの改編などで時代に合わせた変革が求められているはずである。	もちろん、せいとすうげんしょうのげんざい、がっかのしんせつはこんなんであろうが、しょくぎょうかのとうはいごうやかないコースのかいへんなどでじだいにあわせたへんかくがもとめられているはずである。 
\\	どんな商品でもネットで販売するだけで売上が伸びるというものではなかろう。	どんなしょうひんでもネットではんばいするだけでうりあがのびるというものではなかろう。 
\\	運動を始めるのが早かろうが遅かろうが、健康にいいというのは変わりません。	うんどうをはじめるのがはやかろうがおそかろうが、けんこうにいいというのはかわりません。 
\\	休日であろうが、なかろうが、この仕事では関係ないみたい。	きゅうじつであろうが、なかろうが、このしごとではかんけいないみたい。 
\\	このドキュメントは間違えだらけで、全然役に立たない。	このドキュメントはまちがえだらけで、ぜんぜんやくにたたない。 
\\	携帯を2年間使ってたら、傷だらけになった。	けいたいを2ねんかんつかってたら、きずだらけになった。 
\\	この埃だれけのテレビをちゃんと拭いてくれない?	このほこりだれけのテレビをちゃんとぬぐいてくれない? 
\\	彼は油まみれになりながら、車の修理に頑張りました。	かれはあぶらまみれになりながら、くるまのしゅうりにがんばりました。 
\\	たった1キロを走っただけで、汗まみれになるのは情けない。	たった1キロをはしっただけで、あせまみれになるのはなさけない。 
\\	白ずくめ団体は去年ニューズになっていた。	しろずくめだんたいはきょねんニューズになっていた。 
\\	女の子と共通の話題ができて、自分の体も健康になる。いいことずくめですよ。	おんなのこときょうつうのわだいができて、じぶんのからだもけんこうになる。いいことずくめですよ。 
\\	彼女は、教授の姿を見るが早いか、教室から逃げ出した。	かのじょは、きょうじゅのすがたをみるがはやいか、きょうしつからにげだした。 
\\	「食べてみよう」と言うが早いか、口の中に放り込んだ。	"「たべてみよう」というがはやいか、くちのなかにほうりこんだ。 
\\	「食べてみよう」と言ったが早いか、口の中に放り込んだ。	"「たべてみよう」といっったがはやいか、くちのなかにほうりこんだ。 
\\	私の顔を見るや、何か言おうとした。	わたしのかおをみるや、なにかいおうとした。 
\\	搭乗のアナウンスが聞こえるや否や、みんながゲートの方へ走り出した。	とうじょうのアナウンスがきこえるやひや、みんながゲートのほうへはしりだした。 
\\	子供が掃除するそばから散らかすから、もうあきらめたくなった。	こどもがそうじするそばからちらかすから、もうあきらめたくなった。 
\\	教科書を読んだそばから忘れてしまうので勉強ができない。	きょうかしょをよんだそばからわすれてしまうのでべんきょうができない。 
\\	昼間だから絶対込んでいると思いきや、一人もいなかった。	ひるまだからぜったいこんでいるとおもいきや、ひとりもいなかった。 
\\	このレストランは安いと思いきや、会計は5千円以上だった!	このレストランはやすいとおもいきや、かいけいは5せんえんいじょうだった! 
\\	散歩がてら、タバコを買いに行きました。	さんぽがてら、タバコをかいにいきました。 
\\	博物館を見がてらに、お土産を買うつもりです。	はくぶつかんをけんがてらに、おみやげをかうつもりです。 
\\	事情を2時間かけて説明をしたあげく、納得してもらえなかった。	じじょうを2じかんかけてせつめいをしたあげく、なっとくしてもらえなかった。 
\\	先生と相談のあげく、退学することにした。	せんせいとそうだんのあげく、たいがくすることにした。 
\\	「ああいうペットが欲しいのかい?」。	
\\	あの店ではひどい目に遭ったので、二度と行かない	
\\	議論はいらない。言われた通りにすればいいんだ	
\\	「彼のことを後でああだこうだ言ってもどうにもらないよ。もう済んだことなんだから」。	
\\	もうコマーシャルにはうんざりだよ	
\\	「映画を見て泣くなんて、お前らしくない」。	
\\	「ビジネスがうまく行かなかったのは残念だ。うまく行く運勢じゃなかったんだな」。	
\\	「あいつとは何となく合わないんだ」。「彼女に会ったが、お互いにピンと来なかった」。「僕たち、相性が合わないだけなんだよ」。	
\\	「親の愛は無条件である」。	
\\	「あいつ態度悪いよ」。	
\\	「彼女、一週間で新しいボーイフレンドをふっちゃったんだ。僕の予想した通りだよ」。	
\\	僕の車は大丈夫だったが、相手の車は事故でめちゃくちゃになっていた	
\\	きみのダンスの相手はだれなのかな?	
\\	はっきり話さないと相手はわからないよ	
\\	彼は君の相手じゃないよ	
\\	「アポを取ろうとしたが、彼は時間をくれなかった」。「彼は女性をあまり相手にしないタイプなんだよ」。	
\\	今はあのチームは僕たちの相手じゃない	
\\	彼女の立場がどのようであるかはきみにはわからない	
\\	この時間に電話して悪かったかな?	
\\	ちょっと僕の立場になってみてよ	
\\	わが社はサービスの質で勝負しています	
\\	無人島で一人になったとしたら、どんな本があったらいいいと思う?	
\\	たいていは車に相乗りして仕事にでかける	
\\	「数社がこの仕事を受注しようとして競合している」。	
\\	銀行との問題は解決したのかい?	
\\	彼の考えは行動と合わない	
\\	「だめだと分かっているのに、なぜあえて給料を上げてくれと要求するのか」。	
\\	「今日の会議どうするのか、君の計画の概要を教えてくれよ」。	
\\	「ここでは言葉にきをつけろよ。青タンになりたくないんだったらな」。	
\\	「おい、若造。調子はどうだい?」。	
\\	「酷い風邪を引いて、寝込んでいた」。	
\\	「どうして英語を知らないとやっていけないんだよ	
\\	「僕たちは離れられないんだ」。	
\\	「少し飲んだでけで酔っぱらっちゃうんです」。	
\\	「この仕事はちょろい」。「この仕事は簡単じゃないよ」。	
\\	「信号が青になるまで待ちなよ」。	
\\	彼の会社は大量の赤字を出している	
\\	このことを振り返って笑える日が来るよ	
\\	「今夜のパーティーのためにちゃんとおめかししなくちゃね」。	
\\	「彼女は伴奏なしでも歌えるよ」。	
\\	「有罪だと、彼の顔中に書いてあった」。「怖いんでしょ。ハッキリ分かるわよ」。	
\\	彼女は明るい性格だ	
\\	外は真っ暗だった	
\\	「スピーチの前になるとひどく緊張しちゃうんだよ僕は」。	
\\	「マッチがなければ花火を上げられないよ」	
\\	あがってるのかい?	
\\	「誰かが家にこっそり入り、何かを盗んだ」。	
\\	その容疑者は住居不法侵入で訴えられている	
\\	もっと辛抱強さが必要だね	
\\	赤ちゃんの顔は見ていて飽きないよね	
\\	そんなに無気力でなければもっとできるのにな	
\\	すぐに諦めるなよ	
\\	「諦め時をわきまえていることは、根気強いことと同じくらい重要である」。	
\\	諦めたらどうだい。勝ち目はないんだから	
\\	「寸を与うれば尺を望む(あいつはすぐにつけあがる)」。	
\\	去年、タバコをやめたんだ	
\\	「彼の馬鹿げた振る舞いには呆れるよ」。	
\\	「もういいよ。うんざりだよ」。	
\\	彼の愚痴にはうんざりだ	
\\	「官僚、議員と実業家が密接な関係にあるために悪循環になっている」。	
\\	あいつは汚い言葉をよく口にする	
\\	「あいつは血も涙もない奴だった」。「何て冷たいの」。	
\\	「会議でお前がしてたことってあくびだけだったな」。	
\\	あくびは伝染するよね	
\\	「あくまでも彼は無実を主張した」。	
\\	彼は僕の手をしっかりと握った	
\\	「僕は夜中まで起きていた」。	
\\	どうして毎日のように同じ物が食べられるのかな?	
\\	「木曜日、君のためにあけとくから」。	
\\	どんなに悪く見えても、いい面もあるのものだ	
\\	彼の趣味はあら探しをすることのようだね	
\\	「彼はあら探しの天才だ」。	
\\	「必ずサイドブレーキを下げる(解除する)ように」。「サイドブレーキ引いた(かけた)?」。	
\\	「彼に恋したわけではないのよ。ただ、ハンサムだなと思っているだけ」。	
\\	「いつも叔父さんのチャーリーを見習ってきた」。	
\\	「彼女が僕の憧れの人なんだ」。	
\\	多くの若者が人気歌手に憧れている	
\\	「私の伯母は、無料ハワイ旅行に当たった」。「支払いは相手持ちで、私は香港に行った」。	
\\	「昨年、事故に遭って、肩を脱臼してしまった」。	
\\	僕を召使いみたいにこき使えると思うなよ	
\\	「テニスって難しいよね。打ったボールが色んなところに飛んで行っちゃうんだから」。	
\\	こんなにあたふたして、むかついた気分で1日が始まるとはね	
\\	朝っぱらからスパイスのきいた物を食べるのかい?	
\\	「明日という日もあるさ」。	
\\	その桃、気をつけてよ。傷んじゃうから	
\\	「体臭は誰にとっても嫌なものだ」	
\\	初めてキスしたとき、天にも昇る気持ちだった	
\\	「週末にフットボールをしていて足をつってしまった」。	
\\	「お金払い過ぎたんだよ」。「この前の休暇でお金使い過ぎたね」。	
\\	「ミニスカートでハイヒールを履くと女の子達の足がスラッとして見えるよね」。「彼女は、えらく足が長いよね」。	
\\	足腰が立つうちは、この仕事を続けるつもりでいる	
\\	あなたにとって私が重荷にならないといいのだけど	
\\	そのいい知らせを聞いたあと、雲の上を歩いているようだった	
\\	「新しい靴がきついので、足の小指が窮屈な感じがする」。	
\\	今のところはあの問題は後回しにしよう	
\\	「飲み過ぎて部屋が回り始めた」。	
\\	彼が弱みを見せるまで待って、それから攻める	
\\	ちょっと立ち上がって、足を伸ばそうよ	
\\	彼は、チームにとっては足手まといだ	
\\	痛い。きみ、僕のつま先を踏んだよ	
\\	何味にしますか	
\\	「このシチューの味、強烈だね」。	
\\	彼と話せば話すほど、彼の性格がわかるようになる	
\\	あれはこれまでの中で最も無味乾燥な議論だった	
\\	来る日も来る日も同じものばかり食べているので嫌になってきた	
\\	彼は無愛想な話し方をした	
\\	「離婚訴訟の場合、大抵は、母親が子供の養育をすることになる」	
\\	よろしければ、そのバッグお預かりしましょう	
\\	「休暇中、犬を預けていってもいいかな」	
\\	「汗ビッショリだね。ここまでずっと走ってきたのかい」。	
\\	焦ることはない	
\\	いらいらすることはないよ。すべてうまくいくから	
\\	「去年は必死になって働いたよ」。	
\\	「慌てず、落ち着いて、冷静に頑張りなさい」。	
\\	「あわてることはない。あくまで冷静に」	
\\	「今は、我が社にとって大変な(ぞっとするような)時期である」	
\\	このチケットを手に入れるために一晩中並んで待ったんだ	
\\	この問題をいいかげんに扱っちゃだめ	
\\	彼女の遊び心あふれる教え方で、勉強するのが楽しくな	
\\	「パーッとやろう」	
\\	スティーブはパーティ好きだよ	
\\	彼は今遊んでいる[職についていない]	
\\	彼女の注意を引こうとしたが、彼女は見向きもしなかった	
\\	「彼は、3人分の仕事をする」。	
\\	彼の声には人間的温かみがなかった	
\\	あまり危ないことには手を出さないほうがいいよ	
\\	パニクることはない。時間はたっぷりあるんだから	
\\	「あんなにおろおろしちゃって恥ずかしいったらありゃしない。馬鹿みたいだったわよ」。	
\\	聞きもせずに私の言い分を拒否するなよ	
\\	雄弁さで彼にかなう者はいない	
\\	この仕事は大きな悩みの種だ	
\\	父は頑固なまでに保守的だ	
\\	「頭の切れるやつを何人か知っているが、彼女は抜きんでている」。	
\\	「(上司の立場を利用して)僕に命令を押しつけようとしてるのですか」。	
\\	人の話を聞かずにわめくのはやめてくれ	
\\	「何故遅れたかその理由を説明した時、彼女は思い切り怒った」。	
\\	僕の教えようとしていることは、片方の耳から入って、もう片方の耳から出ていってしまうんだね	
\\	僕は興味がないんだよ。それがわからないのか?	
\\	きみはのみ込みが早いね	
\\	「心の片隅で僕は彼に対していつも疑いの気持ちを抱いていた」。	
\\	「あの歌が頭の中から抜けなくて」	
\\	「彼はとても知的な男だよ」。	
\\	「これ前と同じパンを当たらしいパッケージに入れただけでしょ」	
\\	「無難な話題からそれないようにした」。「あたりさわりのない話に終始した」。「ジョーと話す時には、給料のことに触れちゃだめだよ。とても気にしてるんだから」。	
\\	きみが昇進するかどうかはくじ運がいいかだよ	
\\	「多分、君は冗談だと思うだろうけど、若い頃、僕はパンクロッカーだったんだ」	
\\	彼女にお礼の手紙を出すべきなのは言うまでもない	
\\	「当たり前だよ、いつ車を使ってもいいさ。聞かなくてもいいよ」。	
\\	僕の学校では厳罰が当たり前のことになっている	
\\	いい成績をとることは当たり前と見なされるべきである	
\\	この役は彼女のはまり役だった	
\\	「私の両親、宝くじでハワイ旅行が当たったの」。	
\\	宝くじ[くじ引き]で何か当たったことある?	
\\	自分にそんなに厳しくすることないよ	
\\	「よかれと思ってやったのにさ、そうは取られなかったんだ」。	
\\	「その演劇にはちょっとがっかりしたね」。	
\\	一方的に勝ってしまうとなんだかあっけない感じがする	
\\	きみって小食だよね	
\\	「仲介役がしっかりしていると取り引きがスムーズに行く」。	
\\	冷める前にコーヒーを飲んでね	
\\	「うー、ここめちゃ寒だね」。	
\\	「じゃんけんしようよ」。「グーがチョキに、チョキがパーに、パーはグーに勝つ」。	
\\	二度とこんなことがないように	
\\	「あっと言う間だね。もう5時か」。	
\\	そのコート暖かそうだね	
\\	「この店では、あらゆる種類の化粧品を売っている」。	
\\	天気予報では、今日の最高気温は何度になるって言ってた?	
\\	「彼女の所に行きすぎたな。もう歓迎されなくなちゃったものな」。「あまりに厚かましいと、歓迎されなくなるよ」。「申し出ありがとうございます。しかし、やっかいをかけたくありませんから」。	
\\	「(大きな)お願いがあるんだけど」。	
\\	「暑さなんて全然気にならない」。	
\\	「彼は話しながら感情がこみ上げてきた」。	
\\	夏の蒸し暑い夜が嫌なんだ	
\\	人間的に深みのある指導者が必要だ	
\\	正確な費用がわからないなら、だいたいでいいから言ってくれないかな?	
\\	「友達が信じられないなら、誰を信じるんだい」。「お前は人を信じすぎだよ」。	
\\	「当たり!(君の推測通りだよ)」。	
\\	私に観光案内をさせてください	
\\	3時に出るから、残った仕事はやっておいてくれるかな	
\\	お昼まであとどれくらい?	
\\	このコーヒーの後味が気に入ってるんだ	
\\	老後をどう過ごされるつもりですか	
\\	「卒業するのに少なくても後1年はかかる」。「後1,2時間で終わります」。	
\\	あとになってああだこうだ言うのは簡単さ。	
\\	彼女はどういうつもりなんだろう、こんなプレゼントを僕にくれて	
\\	「授業中に寝ただろう。ほっぺたに時計のあとがついてる	
\\	犬の糞を踏みそうになった	
\\	「後でね」。	
\\	「時間かかるの?」。	
\\	「後でくよくよしないかな、もうやってしまったことだけど」。	
\\	「僕は何でも後回しにしちゃうんだ(ぐずということにかけては世界的に通用する)」	
\\	「今、好きなこと叫んでもいいけど、何も変わらないよ。大事なとき(ここぞという時)に意見を言うべきだったね」。	
\\	「今日は随分仕事をしたね。後は、掃除をして帰るだけだ」。	
\\	「自分でタイヤを3つ交換した。残りは1つだ」。	
\\	「彼は昨日、過激なことを言っていたが、今日は、逆の態度を取りだした」。	
\\	「このポテトチップス、一つ食べたら止まらなくなるよね」。	
\\	乾杯をするときに、即興で何か言ってください	
\\	「あっ、またへましちゃったよ」。	
\\	「それは私の責任ではありません。この株を買うべきだといったのはあなたでしょう」。	
\\	きみは僕のことを知らないけど、僕はきみのことを知っている	
\\	「彼女をからかわない方がいいよ。酔っぱらってるから」。「彼女をからかわない方がいいよ。3本ビールを飲み干したところだから」。	
\\	あのファストフードのレストラン、ものすごい勢いで増えてるね	
\\	ズボンに穴を開けちゃったよ	
\\	これくらいの成績なら、好きな大学に入れる可能性があるね	
\\	「先週のことお礼を言うつもりだったの」。	
\\	「彼はそんなに厳しくないよ」。	
\\	「ジェーンは太ってるし、態度も悪いし、煙草もぷかぷか吸うし。でも、ケンは彼女のことを可愛くてチャーミングだと思ってるんだな。『あばたもえくぼ(恋は盲目)』っていうけど、まさにそれだな」。	
\\	「少し羽を伸ばしにビーチに行った」。	
\\	楽してお金は稼げないよ。ホントだって	
\\	「危ないところだったが、仕事を終えることができた」。	
\\	この辺りは夜、安全なのかな?	
\\	「それは欲張りだね」。「欲張るなよ」。	
\\	オイルを見てくれますか。減ってると思うので	
\\	「この魚脂っこいね」。	
\\	くつろいで友だちとおしゃべりするのが好きなんだ	
\\	「何やってもテニスでは彼に勝てない。まだまだだな」。	
\\	好き勝手にするのはやめて、ちょっと僕の言うことを聞けよ	
\\	楽観し過ぎてはいけない	
\\	「僕に這いつくばれと言うのか」。	
\\	ずうずうしいかもしれませんが、そうさせていただこうと思います	
\\	娘がとても可愛くて、彼らは彼女のことを甘やかした	
\\	「彼女の息子は、お母さん子なので、いつも一緒にいたがる」。	
\\	「本気でやってるんでしょ」。	
\\	ここは駐車するスペースがかなりある	
\\	「彼、甘党だよ。お酒はほとんど飲まないみたい」。【注】アメリカでは、「甘党」は「お酒を飲まない」という考え方はありません。甘い物好きでもお酒をたくさん飲む人は山ほどいますから。	
\\	「彼は人とは逆のことをする奴だと僕は思っている」。	
\\	「(雨が降ってきたので)友達の傘の中に入った」。	
\\	父はローストの周りを切り取り、それを犬に食べさせた	
\\	「若いころはいつも兄のお下がりを着ていた」。	
\\	「彼女は、電車の中で、濡れて滴がポタポタ垂れている傘を網棚の上に置いた」。「機内持ち込みの荷物は座席の上にある収納庫に収まる大きさでなくてはいけません」	
\\	「彼と一緒に出掛けると、いつもいい天気だよね」。「彼は陽気な性格です」	
\\	「彼は政界にコネを持つ実力者なので、どの会社も彼を雇いたがる」。	
\\	子どもを育てることは罰を与えたり褒美を与えたりすること以上にもっと複雑である	
\\	ここは毎年雨がたくさん降るんですか	
\\	「馬鹿みたいににやにやするのは止めろよ」。	
\\	今日、怪しいやつがキャンパスを歩き回ってるのを見かけた	
\\	私は皆の前で何もかもぶちまけた	
\\	「今夜出かける予定があると前もって言ってくれてたらな」。	
\\	「改めて家族の歴史に触れるために、系図学を勉強する人もいる」	
\\	予定の立て方を変える必要がある	
\\	ほかにどのような手段があるのか	
\\	このサラダどう?あり合わせで作ったんだけど	
\\	「昇進なんて問題外だよ」。	
\\	それはあり得るね	
\\	この天気がよくなればとてもありがたいのだが	
\\	同僚の文句を言うのはやめて、定職についていることに感謝しなさい	
\\	よかった、今朝僕が遅刻したのだれも気づかなかった	
\\	彼のしたことはありがた迷惑だった	
\\	「彼は典型的な重役タイプだった」。	
\\	試合には負けたが、ある意味では勝ったんだよ。なぜなら、接戦だったし、相手はわれわれよりずっと経験があるチームなんだから	
\\	動く歩道があるので、ここは移動しやすいよね	
\\	「これにアルコールどれくらいはいってるの(度数は)?」。	
\\	「誰もが、ある程度は、嘘つきである」。	
\\	「ある程度は、その事故に対して責任を取るつもりではいます」。	
\\	風邪を引かなくする薬ってあるの?	
\\	ほとんどの学生が夏休みにアルバイトを探す	
\\	この部屋を整頓しよう	
\\	泥棒は店にあるものすべてを持っていった	
\\	あのお金、彼はすぐに使ってしまうだろう	
\\	何が何でも、この仕事を明日終わらせたい	
\\	「彼はいつも季節に合わて服を着ている」。	
\\	「今回はそれを考慮に入れる必要はない」。	
\\	「引っ越して来た当時、彼女はこの町が気に入らなかったが、今では慣れてしまった」。	
\\	僕はここには合わないな	
\\	「アメリカのピザって本場のイタリアのピザと違うよね。別物だよね」。	
\\	彼は破産したという事実を素直に認めることができなかった	
\\	「のんびりやれば」。「時間をかけてゆっくりやればいい」。	
\\	「ジムには驚くよ。彼、見かけよりずっと頭がいいから」。	
\\	「安心しろよ。レポートの締め切りは来週なんだから」。	
\\	「この製品は安心だよ」。	
\\	「彼には手を掛ける必要はない(自分でやれる)」。	
\\	「この車は安心して乗れるよ」。	
\\	「先週は風邪を引いてずっと寝ていた」。	
\\	「スピードを落とせよ。皆を殺すつもりなのか」。「なんだよこの運転は。インディー500なのか」。	
\\	安全第一。これが我々のモットーです	
\\	何て暗くて憂うつな話なんだ	
\\	パキスタンにおける核兵器の開発は、その地域を不安定にする可能性がある	
\\	「そのカードには、招待状とホテルまでの道順が同封されていた」。	
\\	「あんなに努力をしたのに無駄だったなんて信じられない」。	
\\	「予定通りに、その業者は、時間がもう少し欲しい、と言って	
\\	「もういいよ。十分だよ(いい加減にしてくれ)」。	
\\	「それで十分だよ」。	
\\	彼は威圧感のある大男でした	
\\	「今夜、電話をしてもいいですか」。	
\\	「彼女は舞台では映えるよね」。	
\\	私のメークが面白いって?それっていい意味で言ってるの、それとも、悪い意味で?	
\\	「彼女は押しが強いよね。いい意味で」。	
\\	「その言葉がそれを説明するにはぴったりだね」。	
\\	別の言い方をすると、あなたは、新しい車を買うことになるだろうと思いますよ	
\\	彼は上司の機嫌を取るのがとてもうまい	
\\	「賛成していないと彼の顔中に書いてあった」。	
\\	「彼にはいいかげんな返事をしておいた」。	
\\	「この仕事をでたらめな無責任なやり方でやって欲しくないんだ」。	
\\	この会社、詐欺してるようなもんだ	
\\	そのレストランの食べ物はよかったが、サービスはたいしたことなかった	
\\	「あいつにはうんざりだよ」。	
\\	わざわざ新しい芸術の展示会に行くことはない。足を運ぶ価値はないから	
\\	「恋人同士にとって、少しの間、お互い離れてみるのも時にはいい」。	
\\	「俺はやってない。はめられたんだ」。	
\\	大人を子ども扱いしたり、子どもを大人のように扱うべきではない	
\\	彼に少し言いたいことがある	
\\	「一生懸命やれば運も向いてくるものさ」。	
\\	彼はとっても気だてのいい子ね	
\\	「確かな情報を得たんだ。調べてみよう」。	
\\	彼女は買い物中毒だ	
\\	「彼女、儲かっているから、週に3	
\\	駐車禁止と書いてあるのはわかってるけど、大丈夫。日曜日だから	
\\	今のは言い過ぎだよ。あんなこと言う必要なかったのに	
\\	この人と結婚したこと、正しい選択だったと思う?	
\\	ここで言ってもいいですか?	
\\	僕は友だちには思ったことを好きなように言うよ	
\\	僕は小食/大食いです	
\\	この話題を持ち出したのはお前じゃないの	
\\	「結局、それについて初めに話してくれる人は誰も来なかったということか。さて、どうしよう」。	
\\	「これよりいいアイディア思いつかないかな」。	
\\	彼女は休暇をとるお金がないと言いつつ、内緒で銀行口座を持っていた	
\\	「本当に言いにくいですが、昨日は済みませんでした。私が悪かったんです」。	
\\	「そうだよ、自分を責めればいいんだ」。	
\\	「子供の頃、彼は我が儘のし放題だった」。	
\\	儘
\\	この車は燃費がいいから気に入ってるんだ	
\\	年齢をわきまえなさい	
\\	「あなたのことはとても好きだし、それに、すごく魅力もあるわ。でも、いいお友達でいましょ」。	
\\	「頭のいいお子さんがいていいわね」。	
\\	彼は上司の言いなりだ	
\\	「時間があれば、ちょっとある問題について話したいんだけど」。	
\\	「花子、あなたっていい人よね」。	
\\	試験に受からなかったんだ。だから何だっていうんだ。来月また受けられるんだから	
\\	「彼女はその計画を意味のない願い事リストとしてしりぞけた」。	
\\	「乗鞍で温泉に入るのは最高だね」。 「私の祖母は、毎月温泉に入りに行く」。	
\\	好きなようにすれば	
\\	「誰だって(人に)いいように使われたくはない」。	
\\	あなたはよくても、ほかの人はどうなの?	
\\	それが良いとか悪いとか言っているんじゃないよ。そのクラブには入りたくないと言っているだけなんだ	
\\	眼鏡が嫌なのはわかるけど、それでよく見えるんだからいいじゃない	
\\	「彼女、可愛い顔してるけど、言葉使いはひどいよ」。	
\\	昔野球で怪我をしたので、肘をちゃんと伸ばすことができないんだ	
\\	「あなたの言っていることは聞こえてるわよ、でも、無理ね」。	
\\	「仕事を辞めて、もっといいのを探せだって?言うのは簡単だけどね。お前なんか親の蓄えで食ってるんじゃないか」。	
\\	「彼はコンサルタントと言うよりセールスマンだね」。	
\\	「ジョージ・ブッシュは裕福な家庭に生まれた」。	
\\	"「""けち""って言ってもいいよ。でも、僕は、友達にはお金を貸さないようにしてるんだ」。
\\	それは言えてるね	
\\	食器用の洗剤で車を洗うのはどうかな?	
\\	そのような発言が適切かどうか疑問に思います	
\\	この機会を活かすべきだよ	
\\	自分で大学を出た	
\\	「どれだけ一生懸命にやっても、収支が合わない」。	
\\	彼は、きみが彼のためにいかに一生懸命働いているかがわからないんだよ	
\\	「彼は根っからの営業だね」。	
\\	「彼はまるで世の中のことは何でも知っているような話し方をする」。	
\\	「自分のしたことに後悔はない」。	
\\	彼を見たとしても、教師だったとは思わないだろう	
\\	「彼、走るの速いよ。大学に入るまで、競技に出たことがないっていうのは意外だよね」。	
\\	簡単な柄を縫い込むのがどれだけ大変かを知ったら驚くわよ	
\\	のどが痛いんだ	
\\	「彼は押しが強いから、一緒に仕事をしていてとても疲れる」。	
\\	今年は半導体の貿易は活況であった	
\\	「来た道とは違う道で家に帰った」。	
\\	通勤電車は行きも帰りも込んでいた	
\\	やっと目覚めたようだね 
\\	「幸運を祈ってサイコロに息を吹きかけた」。	
\\	「そのプロジェクトについて知っているのは僕だけだったので、担当にされた」。	
\\	「あんなに早く話してたら、彼の話は長く続かないだろうな」。	
\\	「彼の両親はとても厳しく、息が詰まるような雰囲気の中で彼を育てた」。	
\\	「彼は前半思い切り飛ばしたので、息切れしてしまった」。	
\\	「あまりに丁寧過ぎると、息苦しくなることがある」。	
\\	「うちの4才の子供は保育園には行きたがらないだろうなと思っていたけど、あの子、行く気になっているのよ。
\\	「この店、僕が来なくなったらつぶれんじゃないの」。	
\\	「雰囲気を変えてもいい頃だ」。	
\\	エアコンなしで、どうやって暮らしていけるわけ?	
\\	「ジムは新しい友だちと直ぐに仲良くなった(気が合った)」。	
\\	間違って、下りの電車に乗ってしまった	
\\	「今日は、あなた最新のファッションを身につけてるわね」。	
\\	彼はいきなり指輪を取りだし、彼女に結婚してくれと言った	
\\	「君には来月、当然のことだが休暇をとってもらうよ(それだけの仕事をした)」。	
\\	「深く息を吸って、管の中に思い切り息を吐き出して下さい」。	
\\	行くあてもなく、ぶらついてるんだ	
\\	彼は口ほど勇気はない	
\\	「彼は1960年生まれだから、もうすぐ50ってことだね」。	
\\	「何度聞いてもいっしょだよ。答えはノーなんだから」。	
\\	「彼がいくら君のことを愛してると言っても構わないよ。全然そんな風じゃないもの」。	
\\	「あなたがいくらお金を持っていても関係ないわ。あなたとはデートしたくないの」。	
\\	「費用がどれだけかかってもかまわない。娘を最高の学校へ行かせたいんだ」。	
\\	どれくらい安ければ車を買おうかなと思うんですか?	
\\	「どれだけ君のことを愛してると言ってるのか知らないけど、あいつ、土曜の夜は君と一緒にいないで、バーでスポーツ番組観てるんだからな」。	
\\	いくらお金が欲しいからって、盗むことの言い訳にはならない	
\\	どうしてそんなことが言えるのかね?	
\\	あいつなんて全然だめだよ	
\\	お酒、ダメなんですよ	
\\	「先週のスキャンダルに対する君の考えは?」。	
\\	この喫茶店、居心地いいよね	
\\	私、デイブが好きなの。一緒にいて気が楽だもの	
\\	「彼が必要だと思った時、彼が現れた」。	
\\	いざとなったら、彼らが実際にどのように行動するかはだれにもわからない	
\\	僕はこの居酒屋の常連なんだ	
\\	「やっても彼には勝てない」。	
\\	「彼女は自分自身で考えることができる」。	
\\	きみの番だよ	
\\	「僕は、危険を犯さない方だね」。	
\\	難しい状況の中で仕事をするのが好きだ	
\\	「土砂降りの中、彼は配達をし続けた。プロの意地として」。	
\\	「付き合い悪いじゃない。こっちへ来て座れよ」。	
\\	「かつては弟をいつもいじめていた」。	
\\	弟をいじめるのはやめなさい	
\\	その子をいじめるのはやめなさい	
\\	もうこれ以上食べられないよ。お腹いっぱいだよ	
\\	「そんなに意地悪しなくてもいいでしょ」。	
\\	そんなに意地悪するなよ	
\\	政府は先住民を不法占拠者として扱った	
\\	「2番目の棚にある商品の中からどれか選んでください」。	
\\	彼は十代の頃からずっと煙草を吸っている	
\\	「夕食ごちそうさまでした。ご馳走になってばかりでは申し訳ありませんので、次回は私がどこかにお連れしましょう」。	
\\	「そういらつくなよ。慌てることはない」。	
\\	人に頼りすぎてはだめだ	
\\	無責任なことをしたため、彼は罰が当たり辛い思いをした	
\\	そのことを考えるだけで頭が痛くなる	
\\	「いたずらっ子でしたか?」	
\\	「いたずら電話か」。	
\\	「部屋に入っただけなのに、僕がぶつとでも思ったのか、彼は嫌な顔をした」。	
\\	「まだ食べちゃだめ。お祈りしてないじゃない」。	
\\	「この仕事が板につくまでに一年かかたったよ」。	
\\	にっちもさっちも行かないよ	
\\	身動きがとれなくなり、損出を減らすために早めに手を引かざるを得なかった	
\\	ジョギングを始めるなら、やりすぎないようにね	
\\	現在に至っても彼らは戦っている	
\\	私たちをもてなしてくれた人たちは、精いっぱいやってぜいたくなパーティーを開いてくれた	
\\	なんでいつもつまらないことにこだわるんだい?	
\\	「この計画に彼を強く推薦します」。	
\\	「諦める前に、もう一度最後の努力をしてみればよかったのに」。	
\\	この件への関与をきみはきっぱりと否定はできないだろう	
\\	「あのレストランは一部の選ばれた客しか入れない」。	
\\	「こんな風に出逢う可能性はどのくらいあるのだろう(まず、ない 
\\	「この川の水が飲める時もあったんだが」。	
\\	「去年はひどかったけどね、今年は少しは楽になってるよ」。	
\\	子どもたちはジャングルジムで遊んでいる	
\\	一度だけでいいから、彼がちゃんと自分の仕事をするのが見たい	
\\	「彼は椅子に座って2冊の本を読み終えた」。	
\\	「一日一善」。	
\\	何をしてもいいから、できるだけ早く駅に連れていってくれ。一秒でも早く	
\\	まだ両親と暮らしているなら、自立しているとは言えない	
\\	即座にインターネットにアクセスできることがビジネスにおいては極めて重要である	
\\	このラザーニャこれまで食べたなかで一番美味しいよ	
\\	「そのバルコニーから市全体が見渡せた」。	
\\	「彼女は生まれながらのリーダーだ」。	
\\	電車の中でいちゃついている10代のカップルがいた	
\\	「お得意さんがまたいらしたわよ(皮肉って)」。	
\\	「彼は夏中化学の勉強に没頭したので、ようやく分かり始めてきている」。	
\\	「営業許可証を取得するまでにどれくらいの書類を書かなくてはいけないのかね」。	
\\	「そのアパートの住人は、完全に自分たちだけの社会を作ることができると思っているようだ」。	
\\	「彼は一言しか聞かなかったが、その言葉が多くのことを語っていた」。	
\\	よろしければ少し発言したいのですが	
\\	彼は月賦で車を買った	
\\	「群衆がポップアイドルが裏のドアから出てくるのを見たとき、一団となって彼女の方に押し寄せた」。	
\\	これ分割払いにしたいんですが	
\\	「大量生産が始まり、この機械の価格がずっと下がった」。	
\\	「その仕事を一気にやってしまった」。	
\\	「彼の行動には皆が注目している」。	
\\	何をたくらんでいるんだ?	
\\	「表面的な話をしただけさ。大した話はしなかったよ」。	
\\	私は父の跡を継がないことにした	
\\	「今日は一週間分の仕事をしたよな気がする」。	
\\	あんな娘と知り合いになれたらな	
\\	「その引き金を引いたら、一生後悔することになるよ」。	
\\	「交通事故を起こし、彼はそれを一生後悔した」。	
\\	「初めの頃の失敗が選挙運動中ずっとついて回った」。	
\\	私の祖父は、一生涯、絵に打ち込んでいた	
\\	彼女は家族の面倒を見るために一生懸命に頑張った	
\\	新品の赤い自転車。それが今一番欲しいんだ	
\\	僕はその日のことを死んでも忘れない	
\\	ストライプがいいですか、それとも、無地がいいですか	
\\	「彼のそばにいるのは大変だよ(つきあいにくい)」。	
\\	僕はそのクラスのほかのみんなと一緒にされた。ここの学生じゃないんだけど	
\\	ハッカーは電力供給網を寸断しようとすることもある	
\\	「電話よりも、顔を合わせて話しがしたいね」。	
\\	彼女を怒鳴っても無駄だよ	
\\	「あの人、あなたにタバコを消すように言ったくせに、自分が吸ってるわよ」。	
\\	フェリーが人を乗せて川を行ったり来たりする	
\\	「言わないでおいたほうがいいこともある」。	
\\	彼女は言葉どおり時間ピッタリに僕のオフィスに現れた	
\\	それじゃあ。今夜ね	
\\	床が濡れていることを注意してくれたらよかったのに	
\\	「ニューヨークタイムズに、昨夜、ダウンタウンで大火事があったと出てる」。	
\\	それを言うためにわざわざここまで来たのかい?	
\\	行ってきます	
\\	買い物に出かける時には、買い物の順番をしっかり決めておかないとね	
\\	「今週は僕の頭の中は数学の方程式のことでいっぱいだ」。	
\\	母は私の今度の旅行のことをとても心配している	
\\	「飲み代を減らすために、ちびりちびり飲んでいる」。	
\\	僕をだまそうとしたってだめだよ	
\\	身体を少し温めるのにブランデーを少し飲みたいね	
\\	「彼らはヒット一本で勝った」。	
\\	原則として、いつもはスーツとネクタイで会社に行く	
\\	「君の一票が大切なんだよ」。	
\\	「この株の価格は上がる一方である」。	
\\	「彼は一つのことしか考えられない」。	
\\	「君の一本後の電車に乗ってたんだね」。	
\\	急がなくていいから、このメモをタイプしておいてね	
\\	「続けてやってれば、すぐに終わるよ」。	
\\	いつまでも独身のつもりで暮らしてはいけないよ	
\\	この仕事を遅くとも午後4時までに終わらせなければ	
\\	「いつまでにチェックアウトすればいいんですか」。	
\\	健康面での心配ばかりしていると精神も病んでくる	
\\	デーブは、いつものようにまた遅刻だ	
\\	「彼は常連だ」。	
\\	この辺りは僕の縄張りなんだ。飲み屋は全部知ってるよ	
\\	君にはいつもそばにいてほしい	
\\	またきみに会うのが待ち遠しい	
\\	私の鼻は母譲りである	
\\	ロバとラバは、馬と他の動物を掛け合わせた動物である	
\\	「あ、そろそろあの時間だよね」。	
\\	「彼女の目はまん丸だ」。	
\\	「なんて可愛い子なの。いないいないばあ~。見ーつけた。あら、泣き出しちゃったわ」。	
\\	「彼は無骨な田舎者みたいだね」。	
\\	「彼、都会的センスがあるわよね」。	
\\	「彼女は夏の服装をしている」。	
\\	バスター・キートンは命知らずのスタントで有名だった	
\\	40マイルのところを65マイルで走ってましたよ	
\\	「彼の態度にはすごく威張ったところがある」。	
\\	「明らかに誰かが薬瓶に異物を混入したんだ」。	
\\	「ここは駐車禁止区域です」。	
\\	「このスラックスのデザイン、期待してたのとちょっと違うよね。どうにかしてくれないかな」。	
\\	「学生時代を思い出すと、今でも何て馬鹿の事をしてたんだと呆れるよ」。	
\\	我々は借金で首が回らない状態にある	
\\	今からでも謝ればいいんだよ	
\\	「会議用のハンドアウトに何か問題があるとしても、今となっては遅すぎる」。	
\\	今頃になって彼は謝ることにした	
\\	「もう仕事はできないなんて言うなよな。できるのは分かってるんだから」。	
\\	今言っていいかどうかわからないが、顔に何かついてるよ	
\\	今更考え直してもだめだ	
\\	残り時間1分になった今でも、コーチは試合をひっくり返して勝てる見込みがあると思っている	
\\	今では、もう気にならないよ	
\\	最初のオリンピックの金メダルを取った今、どういう気持ちですか?	
\\	「最近じゃ、ダウンタウンもかなり良くなってるよ」。	
\\	帰ったほうがいいわ。主人が今すぐにでも帰ってくるから	
\\	今がチャンスだよ。彼女に電話をして誘えよ	
\\	「ビリーは豊かな想像力の持ち主だ」。	
\\	「私の父の世代が、コンピュータ無しで育った最後の世代だ」。	
\\	「今まではこの海峡を渡るのにフェリーに乗らなくてはならなかったのですが、もうその必要はありません」。	
\\	今までに一度も外国に行ったことがない	
\\	「ラインダンスは、日本ではだいたい何に当たるのかな	
\\	こういう意味ありげな質問をされたくない	
\\	「ここで温泉に入らないなんて、意味がないじゃない(時間を無駄にしている)」。	
\\	「10ドルを節約するために、20ドル分の時間とエネルギーを使うなんて無意味だ」。	
\\	「奥が深いと言うより、ありきたりのことのように聞こえるけどな」。	
\\	「同じ色のスーツを2着買うことに何の意味があるの?」。	
\\	ジェイソンは日本人ではないが、彼の振る舞いはまるっきり日本人みたいだ	
\\	「この会社はエドを(しつこく言って)辞めさせようとしている」。	
\\	それで頭に来ることはないよ	
\\	「またかよ。もう会議にはうんざりしてるんだから」。	
\\	「好むと好まざるとにかかわらず、友達は助けなくてはいけないよ」。	
\\	彼のせいで大変な思いをした	
\\	「一日中ふくれっ面して、どうしたんだい」。	
\\	「ねー、何隠してるんだよ。教えてよ。ずるいよ」。	
\\	「あのレストランでひどい目にあったよ」。	
\\	「この試合に関しては、いい感じがするから、勝てるよ」。	
\\	なぜかはわからないんだが、この件に関しては嫌な予感がするんだ	
\\	お前の仕事に対するだらしのない態度にはもううんざりだよ	
\\	また遅刻したので、上司にこっぴどく叱られた	
\\	嫌みのつもりなの?	
\\	「あーあ、あいつらのおでましだよ」。	
\\	議論をして以来、彼らは一言も口を聞いていない	
\\	「私がちょっと言っただけでも彼の怒りは爆発する」。	
\\	「このバーに入り浸っているので、店の人は僕の名前を知っている」。	
\\	この前、仕事で休んだのはいつだった?	
\\	「君に電話だけど、出るの、それとも留守にするの?」	
\\	「この電池切れちゃってるから取り替えなきゃね」。	
\\	「彼女毎週違う男といるよね。ころころ変えてるんだよね」。	
\\	「昨日はバンバン釣れちゃってさ」。	
\\	「早く飲めば飲むほど、それだけたくさん飲めるよ」。	
\\	彼らのせいでいろいろな目にあったんだから、もっと怒ってもいいはずだ	
\\	この赤いシャツ、買ったときからすると色あせちゃったね	
\\	ページに水がかかって、インクがにじんだ	
\\	「正面特別観客席は、旗やまん幕で飾られていた」。	
\\	「今の僕の気持ちを色に例えるなら、明るい幸せ色のピンクってところかな」。	
\\	「彼は世の中を楽観的に見ている」。	
\\	「彼女は色仕掛けで男に好きなことをなんでもさせられる」。	
\\	「紫色の髪をした子たちがそばにいるとどうも落ち着かなくてね」。	
\\	たばこをやめようと自分に言い聞かせてるんだけど、やめられない	
\\	「人に言ってもいないことを言ったなんて決して言わないよ」。	
\\	だれもお前に聞いてないよ	
\\	「失敗したことに対して、説明も謝罪もする必要はない。言わない方がいいよ」。	
\\	言われた通りにしろ	
\\	「いわゆる安くて完全な保険に多くの客が多額のお金を支払ったが、結果的には、価値がないということが分かった」。	
\\	年上を敬うように言われて育った	
\\	「どう考えたって、その意味は分からない」。	
\\	言わないでおくほうがいいこともある	
\\	もう1本もタバコを吸わないことに決めた	
\\	私の母は、エルビス・プレスリーのサインを持っている	
\\	「最近の銀行員どうなってるんだ。態度に問題があるんだよ」。	
\\	「彼女、見えないところで何してるか分からないよ」。	
\\	「いじめには、時に、肉体的なものではなく、精神的なものもある」。	
\\	君は人混みの中にいても本当に目立つね	
\\	「結婚式での彼の面白い話はずっと印象に残っていた」。	
\\	「彼はそのグループの名ばかりの長で、責任者は彼の妻だ」。	
\\	このプロジェクトは初めから失敗する運命にあったんだ	
\\	「このポスターのデザイン、かなり視覚的に訴えるものがあるよね」。	
\\	犯罪現場に最初に来た調査官が証拠を勝手にいじくり回した	
\\	曲がる前に、ウインカーを出すのを忘れないように	
\\	会話が恋愛の話になったら彼女は赤くなった	
\\	「そこは何も書かないでください」。	
\\	「近道より回り道を行く方がいい」。	
\\	「この仕事をきっちりやるには気を張りつめてなきゃだめだよ」。	
\\	「昨日、この手紙をうっかり出すのを忘れてしまった」。	
\\	「どうしたんだよ。友達が一人もいないような顔をして」。	
\\	これがこの事件を解き明かすヒントになるかもしれない	
\\	彼女はボーナスで買い物をしまくった	
\\	両候補者は、お互いの行動について辛辣なことを言い合った	
\\	マックスはいつも仕事場ではとても元気だ。どうしてああなんだろう?	
\\	「なんか場違いって感じだな。ジャケットくらい着てくればよかったよ」。	
\\	彼らの申し出を受けない方がいい	
\\	「フロント(受付)に誰もいなかったら、ベルを鳴らしてください」。	
\\	「好かれるためならどんなことでもする人がいる」。	
\\	このバスは一晩中30分間隔で運行しています	
\\	この世の中、お金で動いてるんだよ	
\\	チームは今、絶好調だ	
\\	彼はぎくしゃくした歩き方をしていた	
\\	「このコンビニのチェーン店がどんどん増えている」。	
\\	「面白半分で、自分で車をチューンアップをすることにした」。	
\\	「彼にはどうも好きになれないところがあるんだ」。	
\\	迷惑をかけていないといいのですが	
\\	彼はいすを後ろに傾け、後ろの脚でバランスを取った	
\\	昨日の夜、この本を初めから終わりまで読んだ	
\\	靴下を裏返しだよ	
\\	彼には話したくない恥ずべき過去がある	
\\	このシチューは濃いから、フォークで食べられる	
\\	「なんか怪しいな	
\\	「暖かくして行きなさい、風邪引くわよ。	
\\	「俺の上司は、禿げ隠しの達人」。	
\\	ジョージの言うことは割り引いて聞くようにしないとだめだよ	
\\	1度嘘をつくと、次々と嘘をつくことになる	
\\	「彼女は嘘泣きをしていた」。	
\\	一度嘘をつくと、次から次へと嘘をつくようになる	
\\	「でかいこと言うのはやめろよ」。	
\\	「減税は、一時的に人気を得るために政治家が使う決まり文句に過ぎない」。	
\\	彼女は私の方を疑い深い目でチラッと見た	
\\	「このシーン怪しいな。インチキみたいだね」。	
\\	「指が何本見える」。	
\\	「彼は、決してぶーたれない」。	
\\	最近仕事がうまく行かない	
\\	「会議、会議、会議の連続で、会議が耳から出てきそうだよ」。	
\\	彼女はボランティアの仕事に没頭した	
\\	この買い物は税控除されない	
\\	3時になってスタートした	
\\	リラックスしてくつろいでよ	
\\	練習を始めて、知らない内に一度に3つのピンを操るようになっていた	
\\	「ブラインドタッチを覚えなかったから、打つのへたなんだ」。	
\\	「家の内と外では性格が変わる人がいる」。	
\\	きみの部署の実情を教えてくれよ	
\\	宇宙空間で生命を維持するのは難しくお金がかかる	
\\	「この計画が日の目を見るようになるといいよね」。	
\\	「うっかり彼に言ってしまった」。	
\\	「立場が逆転したけど、どう思う」。	
\\	「梅雨が始まってから、ずーっとうっとうしい天気が続いている」。	
\\	「毎日長い時間電車に揺られてたら、気分が滅入るよね」。	
\\	大雪のため新幹線の運転を見合わせています	
\\	「いろいろと指示を出すので、混乱しちゃいます」。	
\\	「インフルエンザが学校中に蔓延した」。	
\\	「デイブは重役向きじゃない」。	
\\	「ラジオを聞きながら美容体操をするのが好きなの」。	
\\	「彼は時間と努力を無駄にしただけだ」。	
\\	君はどれくらい料理がうまいの?	
\\	「気合いを入れて仕事に取りかかろう」。	
\\	歴史についてはよく知らない	
\\	居眠り運転は飲酒運転と同じくらい危険である	
\\	株価が急騰している	
\\	彼はとても自己中心的な人だ	
\\	「彼女の言うことは割り引いて聞きなよ」。	
\\	「君の申し出にはとてもそそられるが、受けないよ」。	
\\	「彼らは顔を合わせたとたん、直ぐに仲良くなった」。	
\\	今週は何をしてもうまく行くみたいだ	
\\	ゴルフのほうはどう?うまくなってる?	
\\	「彼女はうぶじゃないよ」。	
\\	この窓から見える砂漠の色は素晴らしいね	
\\	「傷口の手当をしなかったが、それでも化膿せずに治った」。	
\\	「今夜のデート別の日にしてもらえるかな。どうしても仕事から抜けられないんだ」。	
\\	彼のコメントはいっそう事態を混乱させるだけだった	
\\	「彼の話は、スティーブン・キングより込み入っている」。	
\\	「彼にはどこか信用できないところがある」。	
\\	人はお互いに好きだからこそ、憎しみ合ってしまうことがある	
\\	シャツ、後ろ前に着てるんじゃないのか?	
\\	「13才の時に、僕は声変わりをした」。	
\\	このコピー機は両面コピーができるんだ	
\\	おれがお前にたてをつかないのは分かってるだろう	
\\	「彼はコネでこの仕事についたんだ」。	
\\	「リップバンウインクルと浦島太郎の伝説はよく似ている」。	
\\	「時に、人は、言いたいことと逆のことを言ってしまうことがある」。	
\\	「あの歌手はかつてはとても素晴らしかったが、いまでは見る影もない」。	
\\	おれがお前に一体何をしたって言うんだ	
\\	「安全策を取って、好投しているピッチャーを代えたが、裏目に出てしまった」。	
\\	趣味と実益を兼ねて絵を描いている	
\\	ここは和やかにチャットする場所です。つまり、あなたのけんか腰の意見はここにはふさわしくないということです	
\\	「彼は、地下鉄に乗らずに引き返し、彼らの裏をかいた」。	
\\	「彼の主張を裏付ける証拠は何もない」。	
\\	友だちに僕らの秘密を言ったって?もう、ばればれだな	
\\	「日曜日のバイキングがこのレストランの売りだ」。	
\\	「このシャツはよく売れている」。	
\\	「ジョーと彼の兄さんはうり二つだ」。	
\\	「静かにしてくれよ。じっくり考えられないじゃないか」。	
\\	「もう君の不満は十分に聞いたよ」。	
\\	きみが来ることができてうれしいよ	
\\	昨日の夜、君と一緒にいるのを誰にも見られなくてよかったよ	
\\	「健康が世界中の宝物よりも価値がある」。	
\\	「あなたと一緒に仕事ができてとてもうれしいです」。	
\\	嬉しくて泣いているんだ。悲しいからじゃない	
\\	すみません。その本は売り切れです	
\\	彼は青年時代を音楽と酒で無駄に過ごしてしまった	
\\	駅の周りにけんかっ早い奴らがうろついていた	
\\	彼女の夫は彼女が浮気をしていることに気づいた	
\\	「私、変な噂なんてたてたりしないわよ」。	
\\	そうかい、きみが噂の新人か	
\\	「彼の言うことを真剣に受け取ったらだめだよ」。	
\\	彼らは安い外国製品を締め出すために輸入品に税金を課した	
\\	「彼は聞いているようだったが、うわの空だった」。	
\\	頑張って	
\\	僕はついてるぞ	
\\	「土日はその電車運行していないんですよ」。	
\\	悪天候のためマラソンが中止になった	
\\	「もうそれ以上聞きたくない」。	
\\	「自分はカレー好きだと思ってるかも知れないけど、一週間毎日カレーを食べたら嫌になるよ」。	
\\	「彼の演技は涙が出るほど退屈だった」。	
\\	ラッキーな星の下に君は生まれたんだな	
\\	「私たちの学校は毎年運動会をやります」。	
\\	「毎週、どのくらい運動してますか」。	
\\	ジャックはスポーツ万能だ	
\\	「息子は、スポーツに夢中になるタイプじゃないから、あの子が競技してるの見るのは苦痛だわ」。	
\\	毎週、どれくらい運動しますか?	
\\	私の両親は、自分たちが一緒に生活する運命だったといつも言う	
\\	信頼してるからな。へまをするなよ	
\\	今日できることを明日に延ばすな	
\\	素敵なお皿ばかりで、選ぶのが大変	
\\	それ面白そうな映画だね。だれが主演してるの?	
\\	「この会社は、近年、かなりの影響力を持つようになった」。	
\\	「新しい環境法が我々のビジネスのやり方に影響を与えるだろう」	
\\	日曜日はどれくらい遅くまで開いてるんですか?	
\\	あいつは営業向きではない	
\\	「その木は、時代の浮き沈みをそこでじっと見つめてきたのだ」。	
\\	僕の英語の能力はそんなに悪くないよ	
\\	彼はだれに会ってもニコニコしている	
\\	新しいプラズマテレビを買った	
\\	「これは多くのエキストラが登場する叙事詩的な長編映画だ」。 「アルフレッド・ヒッチコックは自分の映画で顔見せ程度に出演していた」。	
\\	「あいつはごますりだよ」。	
\\	「私のことが見えたことを知らせるために彼はうなずいた」。	
\\	ワックスを塗ってすねのむだ毛の処理をしてるの?	
\\	「あなったていやらしいのね」。	
\\	新しいドレスを着た彼女は絵のように可愛い	
\\	「この番組は、テレビドラマのエヌジーを集めたものだ」。	
\\	「怒ると精神的なエネルギーを無駄に使うことになる。」。	
\\	「いい子だからお父さんに新聞取ってきてくれる」。	
\\	「誰がそんな風に話していって言ったんだ」。	
\\	彼女は服装にうるさい	
\\	彼らは縁がなかったんだ	
\\	「食べ過ぎたり、飲み過ぎたりすると、胃が重くなる」。	
\\	「パーッとやろうぜ」。	
\\	「今日の会議は荒れたね」。	
\\	「ジェーンとビルはできてるのかい?」。	
\\	「遠距離恋愛は大変だよ」。	
\\	「僕はあまり迷信深くないんだけど。君はどう」。	
\\	「学生売春は、日本では、よく援助交際と呼ばれる」。	
\\	火を囲んで、みんなで歌を歌った	
\\	最近、円はドルに対してどうなの?	
\\	「今日は暑いなんてもんじゃない」。	
\\	彼はこの会社の縁の下の力持ちだ	
\\	「彼は幸せな家庭に恵まれていた」。	
\\	「依然として円高が進んでいる」。	
\\	「少し楽にすればいいよ。ここにいるのはみんな仲間なんだから」。	
\\	「ビールはいいです。アルコールは少しの間控えていますので」。	
\\	貿易赤字の件は長い間二の次にされてきた	
\\	「かつて数学の先生に熱を上げていた」。	
\\	「スポーツはこれくらいにしまして、次は、天気予報です」。	
\\	「やっと明かりが見えてきたよ(見通しがつく)。今、止める訳にはいかない」。	
\\	「楽ばかりしてると人間だめになるよ。時にはへまをして苦しい思いをした方がいい。それでその窮地からいかに抜け出すかを学ぶことになればね」。	
\\	「こい先に行かせてくれないかな。遅すぎるよ」。	
\\	彼女がガミガミうるさく言うので、それで彼はお酒を飲むようになったんだ	
\\	どこの野球チームを応援してるの?	
\\	「どのチームを応援してるの?」。	
\\	「緊急事態に対処するために、何らかの手を早く打たなければならなかった」。	
\\	「往復切符ですか、片道切符ですか?」。	
\\	去年は仕事に専念して、それが報われたよ	
\\	野球、フットボール、バスケットボールにはシーズンがある	
\\	「両チームとも生彩を欠いていた」。	
\\	彼らは激しい夫婦生活を送ったが、離婚することはなかった	
\\	「彼の音楽の能力に我々は深く感動した」。	
\\	「会社でお偉方と話すことはあまりない」。	
\\	「もっと頻繁に髭を剃るべきだと思わない?」 
\\	「いつ髭を剃るかは自分の勝手だろ」。 
\\	「隣のデービスさん庭の手入れをしてないわね」。 
\\	「人の事に口を出すなよ」。	
\\	大きな顔をするのはやめろよ	
\\	「態度がでかいぞ」。	
\\	我が物顔に振る舞うのは止めろよ	
\\	「彼は、細かいことは気にしない」。	
\\	交渉者たちは原則では同意に達していたが、まだ取り引きにはサインをしていない	
\\	政治スキャンダルが起こるたびに、人々は当然のように駄目な奴は追い出せという態度をとる	
\\	彼女はソフトウエアビジネスでひと山当てた	
\\	「かなり得点を入れたが、逆転できなかった」。	
\\	接待されるのを期待する政治家もいる	
\\	「今日は遠回りをして帰ろう」。	
\\	「彼の頭の中はいつもアイディアであふれそうになっている」。	
\\	「両親が君がタバコをずっと吸っていることを知ったら、しかられるぞ」。	
\\	「彼女は自分の子供には甘いが、他の子には厳しい」。	
\\	コンサルティングで彼は一儲けした	
\\	威張るのはやめろよ。ここではお前がボスじゃないんだから	
\\	「大盛りにすることはないよ。後でおかわりすればいいんだから」。	
\\	「アダックス社の社員が、企業秘密を公にすることにした」。	
\\	友人はその投資は確実だと言っていたが、痛い目にあうんじゃないかな	
\\	その倒産のニュースはウォール街に衝撃を与えた	
\\	この店にはお買い得品がある	
\\	彼は復讐に取りつかれている	
\\	すぐにあいつにこの借りは返してやるからな	
\\	ただいま	
\\	スティーブのすばらしい仕事のおかげで、今年の目標を達成することができた	
\\	「このスキャンダルのおかげで、我が社の株は急落した」。	
\\	「ふざけるんじゃないよ」。	
\\	そんなの比較できないよ	
\\	おかしいな。財布はここのポケットに入ってると思ったんだけどな	
\\	「一方的な勝利だった」。	
\\	「泳いでいる間、財布を預かってもらえるかな」。	
\\	「彼女のことをブスと言ったら、それだけでセクハラだと責められた」。	
\\	政治家が何と言おうが、すべてはお金の問題だ	
\\	このノートに家庭の支出をすべて記録している	
\\	「お代わりしたければどうぞ」。	
\\	ひざまずいて頼むつもりはない	
\\	「出るときにはドアを開けっ放しにしないでくれよ」。	
\\	どうだい、元気だったかい?	
\\	「これが僕のお気に入りのラジオ局なんだ」。	
\\	「馬鹿な質問だと思うけど、これで何するの?」	
\\	「お客様の満足度は去年より今年の方がずっといい」。	
\\	「彼は気楽に話せる奴じゃない」。	
\\	6を4で割るとどうなる?	
\\	「じきお乳をあげるの止めないとね」。	
\\	「君の言うことは奥が深いね。哲学の授業とってるんだろう」。	
\\	「彼女は18だけど、精神年齢はその半分だね」。	
\\	「彼の話は要を得ない」。	
\\	びくびくしてるんだろ	
\\	「あんなこと言うなんて彼女何て嫌な奴なの」。	
\\	仕事に遅れそうだったので、いつもよりも車を飛ばした	
\\	遅ればせながら、誕生日おめでとう	
\\	「ずっと飲み続けてたので彼女は肝臓を悪くした」。	
\\	「ボーナス全部を買い物と旅行に使ってしまった」。	
\\	「あなたの言っていることは分かるのですが、賛成しかねますね」。	
\\	「二人分を払おうかと彼に聞いたが、彼がそれを受けるとは思わなかった」。	
\\	授業をさぼって怒られた	
\\	勝っている方は試合の後半で危ういくらいにのうのうとしていた	
\\	「ダウンタウンに一緒に行こうよ。一杯おごるからさ」。	
\\	「何でもクレジットカードを使うんだけど、万一のことを考えて、少しは現金も持ち歩いてるよ」。	
\\	「感情を抑えたままにしているのは健康的じゃないよ」。	
\\	一番下の子はいつもお下がりを着ている	
\\	「状況はどんどん悪くなってる」。	
\\	「君の気持ちは分かるよ。去年、同じ様なことがあったから」。	
\\	「手の内を明かすな」。	
\\	かなり長い間ひどい風邪を引いてたけど、もう大丈夫だと思うよ	
\\	この風邪は治りそうもない	
\\	「末永くお幸せに」。	
\\	惜しいが、うまくいかなかったね	
\\	「教えてよ。君の会社ではコンサルティングしてるのかな」。	
\\	「彼は議論の仕方は力強いが、信頼性に欠ける」。	
\\	平気でそんなに税金が払えるなんて驚きだね	
\\	彼女は安物買いの銭失いだ	
\\	彼の彼女は、彼におしゃれな服を着させるのが好きだ	
\\	「俺、とっても頭に来たんだ」。	
\\	「時間をとらせてすみません」。	
\\	「もう少しで彼と契約できるところだったのに。あーあ、取り引きがおじゃんになっちゃったよ」。	
\\	「年相応の振る舞いをしろよ」。	
\\	「この音楽聞くと眠くなるよ」。	
\\	「大抵は友人に年に一度か2度連絡を兼ねてクッキーの箱詰めみたいなものを送るようようにしてるの」。	
\\	「ここまで来ても、皆の熱狂はおさまらない」。	
\\	「お世辞を言ってもだめだよ」。	
\\	「そんなお節介はやめてよ」。	
\\	彼は私にとてもよくしてくれました	
\\	ありがとうございました。またお越し下さいませ	
\\	「この1年色々とお世話になりました。どうお礼を言っていいか」。	
\\	今日はスケジュールがかなり遅れている	
\\	「かなり遅くなっていたから家に帰りたかったんだけど、上司がお酒を頼んじゃったんだよ」。	
\\	「恐れ入りますが、この機は禁煙になっているのですが」。	
\\	失礼ながら、違った意見を申し述べさせていただきます	
\\	「その上、今、ハンドバッグを2つ注文していただくと、お揃いの財布がただでついてくるんですよ(テレビショッピング)」。	
\\	暗い通りを歩いていて一度襲われたことがある	
\\	気取ってるんじゃないよ	
\\	私のコレクションの中で、これが最も大切にしているものです	
\\	「彼の立場で考えてみたらいい」。	
\\	今日は、目が回るほど忙しいんだろ?	
\\	「彼、インターネットにはまってるよね」。	
\\	「コンピュータおたくだということは認めるよ」。	
\\	「彼女の心配しているのは、自分のことだけだ」。	
\\	「じゃあね(気楽にやってね)」。	
\\	「僕はおだてられると弱いんだ」。	
\\	「じゃあ、あそこで10時にね」。	
\\	「それ冗談のつもりなの。あんたの話って、ほんと、オチがないよね。面白くも何ともないよ」。	
\\	今日は落ち込んでるんだなんでそんなに落ち込んでるの?	
\\	「社長が長期休暇を取っているというニュースを聞いて、何となく不安な感じがした」。	
\\	「彼女は、お酒を飲み、煙草を吸い、夜更かしをするという不健康な生活を送るようになった」。	
\\	「落ち着けよ。うろたえる理由なんか何もないよ」。	
\\	使わないとだめだよ。外国語ってそういうものさ	
\\	このシャツについたコーヒーのシミの取り方知ってる?	
\\	「濃いお茶の渋みが好きなんだ」。	
\\	「いつかランチしましょう」。	
\\	毎晩、5ページ読めばいいだけ。そんなのちょろいよ	
\\	「今日の間に合わせの報告は、ほとんどが現状の説明だけじゃないか」。	
\\	「からかうのは止めてくれよ」。	
\\	このシャツは洗濯しても色があせないよ	
\\	「何をしても、太る一方だ」。	
\\	「私、根っからの郷ひろみファンなの」。	
\\	「面倒だな。こんなに寒い日には買い物に行きたくないよ」。	
\\	彼女にはおっとりとして上品なところがある	
\\	今朝は調子どう?	
\\	「僕はお前の使い走りじゃないよ」。	
\\	「あっ、もう5時か。皆さん、どうもお疲れさまでした。お先に失礼します」。	
\\	「彼女ここでは長いからね、誰も怖くないんじゃない」。	
\\	このビールに合うおつまみを買おう	
\\	「助けて上げられないよ。今週はお金が足りないから」。	
\\	レジの人がお釣りを少なくくれた。	
\\	座れお手ついてこいいい子だね	
\\	「もうネタ切れだよ(アイディアが尽きた)」。	
\\	試験の途中でわからなくなり、全部終わらなかった	
\\	いい車だね。エンジンの手入れがいいのがわかるよ	
\\	「ええ、どうってことありませんよ」。	
\\	お皿洗わなくてもいいよ。僕がやるから	
\\	子どものことで手がいっぱいですから、仕事はしていません	
\\	「これまでは彼にはやさしかったけど、もうやめた。いい人になるのはやめだ」。	
\\	「彼女に近づくなよ。機嫌が悪いから」。	
\\	「彼女は小さい割に、きついことを言う」。	
\\	「腕時計の上に寝ちゃったから、おでこに時計の跡がついちゃったよ」。	
\\	「これはものすごくいい買い物ですよ」。	
\\	「今がお買い得です」。	
\\	「お金を払っただけのことはあるね」。	
\\	「彼には女っぽいとことがある」。	
\\	「昨年来、彼からから何の連絡もない」。	
\\	「この仕事には危険が一杯ある」。	
\\	奴らにはめられた	
\\	和平へのプロセスはゼロサムゲームとして扱うことはできない	
\\	このズボン洗ったんだけど、染みが落ちないのよ	
\\	昨日彼女に初めて会って、僕はうっとりしてしまった	
\\	「失せろってんだ。訳の分からないこと言いやがって」。	
\\	あいつらのレベルに合わせることはない	
\\	•「彼女おとなしいけど、意思は強いよ」。 •「彼女見た目はおとなしそうだけど、話すのを聞いてごらん(すごいぞ)」。	
\\	今日はとってもおとなしいね	
\\	彼は年齢より若く見える	
\\	「騒音公害が我が街では大きな問題になっている」。	
\\	「バッターとしては彼の半分以下だね」。	
\\	祖母の記憶が薄れ始めている	
\\	会社での彼の影響力は弱くなっている	
\\	その本の辛らつさにはあぜんとした	
\\	最近、女の子の間では、へそ出しが流行っている	
\\	「お腹がものすごい音を立ててるよ。今朝、食べなかったからな」。	
\\	もう一口食べたら、お腹が破裂しちゃうよ	
\\	「彼が冗談を言った時、みんな笑いが止まらなかった」。	
\\	先週食べたものがあたって下痢をした	
\\	「どっちがどっちか区別がつかない」。	
\\	この何年かで歯を治療するのにかかったお金で車が買えたね	
\\	いつもの所でいつもの時間に会おう	
\\	このサンドイッチはみんなほとんど同じだね	
\\	左の眉毛と同じように右の眉毛も動かせますか?	
\\	感傷的な話はよせよ。きみよりひどい問題を抱えている人はいくらでもいるんだから	
\\	「トムとジュディは合うよね」。	
\\	「その子はチームの足を引っ張っている」。	
\\	「レストランの予約お願いしていいかな」。	
\\	これで僕に何かを頼んでさせようっていうわけ?	
\\	きみが彼女のことをあがめるのはいいが、胸にしまっておいてくれよ	
\\	何か食べに来ませんか	
\\	「一生懸命に働かないと首になるよ」。	
\\	お引き取り願えませんか	
\\	「あいつは手強いぞ」。	
\\	「今日はこれで終わりにしよう」。	
\\	「僕がホームステイをしたのは、典型的なアメリカの小さな町だたんだ。野球にお母さんに、アップルパイって感じかな」。	
\\	「裏での話ということなら話してもいいよ」。	
\\	寒い夜は熱いお風呂が一番だね	
\\	悪い癖を捨て去るのは難しい	
\\	この会社は危なっかしい	
\\	彼女は悲しみをお酒で紛らした	
\\	「メキシコの死者の日と日本のお盆との類似点及び相違点を見るのは興味深いことである」。	
\\	「いいかい。5つ通常の値段で買ってよ。そしたら、一つおまけするからさ」。	
\\	「彼女はほんの10才だけど、精神年齢はずっと高い」。	
\\	私の両親は、お見合いで一緒になったの	
\\	「週末に赤ちゃんを近所の神社に連れていくの」。	
\\	おトイレに行きたいの。おしっこかな、うんちかな?	
\\	「彼らはその会社の乗っ取りを目論んでいるが、できるとは思えない」。	
\\	「ドギーバッグもらえますか。全部食べられないので」。	
\\	「知らない方がいいよ」。	
\\	彼は本当におめでたいやつだ	
\\	「一時間その方程式に取り組み、ようやく答えが分かり始めてきた」。	
\\	「ことばの連想ゲームをしよう」。	
\\	学生時代を懐かしむ気持ちが日に日に強くなっていく	
\\	「よーし、とことんやってみよう」。	
\\	「この車はパワーがあるから、高速道路で思い切り飛ばせる」。	
\\	昨日ショッピングセンターで思い切り買い物したんでしょ。いくら使ったの?	
\\	「彼が怖いのは分かるが、勇気を奮い起こして自分の考えを彼に言わなくては」。	
\\	子どもの教育のことをあれこれと考えすぎる傾向にある母親もいる	
\\	「僕が君のことを怖がっているとどこでそう思ったんだろう」。	
\\	彼に思い知らせてやる	
\\	「ナンシーは苦い経験をして電気を使って遊んではいけないことを覚えた」。	
\\	きみの病気は精神的なものだと思うよ	
\\	彼はこうだと決めたら、それを変えない	
\\	「僕の先生はとても厳しいから、彼の声を聞いただけで、座ってても背筋がぴんとしちゃうんだ」。	
\\	「その事を思い出すと、いつも笑ってしまう」。	
\\	「あっ、そうだ。思い出した。あの映画見たことあるよ」。	
\\	昨日の夜は、思い切りカラオケで歌った	
\\	「おい、いい考えがあるぞ」。	
\\	すごいアイデアが浮かんだ	
\\	「仕事に間に合わない、そう思ったんだけど。今までになく飛ばしたら、間に合った」。	
\\	家に帰る途中で買い物をするつもりだったが、忘れてしまった	
\\	「彼女は変な考えにとりつかれている」。	
\\	きみの一番の思い出は何かな?	
\\	最近は何も思い通りには行かないようだ	
\\	「明け方まで働いたんだ。次の日、会社で皆から注目を浴びるんじゃないかと思いながらね」。	
\\	食事は申し分なかった	
\\	彼女が僕に興味を持っていたなんて思いもしなかった	
\\	彼は思いやりがある	
\\	他人に対して少しは思いやりを持ったらどうなんだい	
\\	何も言わなかったけど、僕はいつもきみのことを思ってたんだよ	
\\	彼は思う存分歌った	
\\	「あやつられているのがわからないのか?	
\\	「あやつられているのが分からないのか?」	
\\	彼らは完全に彼のことを思うように操ることができた	
\\	恩を仇で返してはいけない。後悔することになるよ	
\\	彼は彼女の望むことを何でもする	
\\	そのことは考えないようにした	
\\	「物事は思ったようにはならないものさ」。	
\\	彼には昔の面影はない	
\\	この新しいゲーム、面白そうだね	
\\	パーティー楽しかった?	
\\	それって冗談だったの。わからないよ。何がそんなに面白いのかな?	
\\	彼は見た目以上の人物だ	
\\	「悪ふざけも、気を付けないと、大変なことになる」。	
\\	「こちらでお召し上がりですか、それとも、お持ち帰りですか」。	
\\	「昨日の夜見た映画、期待していた通りだった」。	
\\	「思ったようにはならなかった」。	
\\	お前の顔に有罪だと書いてあるよ	
\\	セーターを裏表に着てるよ	
\\	「表向きは僕が一番偉いことになってるんだが、妻が実権を握っているんだ」。	
\\	グレイさんが言うことはここではとても重みがある	
\\	そこに1人で出かけていったのには、彼なりの思惑があったのだ	
\\	「思わず口を滑らしてしまった」。	
\\	あなたが遊びでないといいけど	
\\	彼は真実を言ってるかって?う~ん、さあどうかな	
\\	彼は自分のことを天才だと思っていた	
\\	「人は僕のことをお宅だと思っているみたい。だって、コンピューターのことをいろいろ知っているから」。	
\\	きみは間違った価値観をもっていると思う	
\\	「これで君はこのプロジェクトから離れてもいい」。	
\\	「私の両親は、誰が料理をして、誰が庭の手入れをするか、暗黙のうちに了解している」。	
\\	「何か手伝えることがあったら知らせね」。	
\\	「彼は両親の言うことには全て従う」。	
\\	「親不知が出て来ちゃって」。	
\\	「チンピラ達が、かつあげをするおやじを探してうろついていた」。	
\\	「中年が持っている危機感について冗談飛ばしても、若い子は関心を示さないよ」。	
\\	「親の愛に制限などない」。	
\\	僕に任せてよ	
\\	「痛いっドアに親指を挟んだ」。	
\\	「親指を中に入して拳を握ってください」。	
\\	お湯や冷たい水が出るのは、現代生活のぜいたくの1つである	
\\	「どうしてもお願いしたいことがあるんだ」。	
\\	「何度も転んでスロープを降りてきた」。	
\\	「彼は色々としてくれたので、お礼に行った方がいいね」。	
\\	「休暇も残り数日となった」。	
\\	仕事がたまっている	
\\	「恩着せがましくするのは止めてくれ」。	
\\	音響効果がひどかったので、演奏が台無しだった	
\\	これってきつくて報われない仕事だよね	
\\	「子供達に注意したが、馬の耳に念仏だった」。	
\\	僕は方向音痴なんだ	
\\	「僕の兄は方向音痴だ」。	
\\	「あのセーターすたれちゃったよね(流行遅れ)」。	
\\	女の気持ちがわかる人はいないのかしら	
\\	「急ぎなものですから、バイク便で送っていただけますか」。	
\\	「あの食料品店のオーナーには、どこか少年のようなところがある」。	
\\	「彼女が去っていった時、裏切られたって感じがとてもした」。	
\\	「彼女が君にしてくれたことに対して、何か恩を感じないのかい?」。	
\\	おんぶして上げるよ。脚が痛むなら	
\\	蚊よけの薬[スプレー]持ってきた?	
\\	「鳥かごに手を入れたら、手を噛まれてしまった」。	
\\	「大体、5年か7年の周期で新しい車を買っている」。	
\\	自分を低く評価することはない	
\\	「正直物が馬鹿を見ることはないよ」。	
\\	「今日は素晴らしい演技をしたね」。	
\\	20年間皆勤で働いた	
\\	「僕の兄は東京で会計事務所をやっている」。	
\\	「僕はこれまでに老人問題について考えたことがない」。	
\\	今年はサミットは開かれなかった	
\\	「あいつは仕事中毒だ」。	
\\	「今日の午後は、床屋さんにいた」。	
\\	風邪はなんとか治ったのかな?	
\\	この人は発信者番号を非通知にしてるんだな	
\\	ある程度の期間だったら、定期券がいい	
\\	根本的な改革をしなければ、収益を上げることはできない	
\\	「血なまぐさい話を事細かにしなくていいよ」。	
\\	駐車禁止と書いてある	
\\	明かりはついてるが、家にはだれもいない	
\\	回転台をそんなに速く回すなよ。僕の飲み物がもう少しで倒れるところだっただろ	
\\	この魚、解凍するのに1時間はかかる	
\\	バッティングセンターでは、3球のうち1回くらいしか当たらない	
\\	風邪もう治ったの?	
\\	「彼が鍵穴から覗いているところを捕まえた」。	
\\	降伏か全滅のどちらかしかないということが敵に伝えられた	
\\	私、ブランド品の買い物中毒なの	
\\	弟が電話をしてきて、会話がはずんだ	
\\	「いい買い物どころではない。安い中古車は長い目で見ると、かえってお金がかかることになる」。	
\\	きみは僕に怒る代わりに、感謝すべきなんだよ	
\\	学校帰りに[喫茶店に寄って]コーヒーを飲んだ	
\\	「会議があるから、何時に帰れるか分からないの?じゃあ、帰る前に必ず「帰るコール」を入れてね。夜食の準備しておくから」。	
\\	「彼はここらのバーやパブ、レストランのオーナーは皆知ってるみたいよ」。	
\\	「彼には悩みが沢山あったが、平然とした顔をしていた」。 
\\	「彼の家族は面目を保つために、その問題を秘密にしていた」。	
\\	この学校ではみな 
\\	の友だちみたいだね	
\\	「あっちへ行ってくれ」。	
\\	「彼女が怒ってたのは分かってる。顔中にそう書いてあったもの」。	
\\	ほら立ち上がって、埃を払って、もう一度やってみよう	
\\	昨年、彼らは意見が合わずに口論をし、その時以来、口を聞いていない	
\\	アルフレッド・ヒッチコックは、いつも自分の映画にちょっと顔を出していた	
\\	この産業は、もっと価格競争をする必要がある	
\\	携帯
\\	3プレーヤー無しではやっていけないね	
\\	彼は毎朝熱心に経済面を読んだ	
\\	「ハイヒールは足によくないよ」。	
\\	ここに着くのに、どれくらい時間がかかるのかな?	
\\	彼女はローストビーフに塩を振り掛けた	
\\	タクシーは夜20パーセント割り増しになる	
\\	「彼は車に乗って行った。ビールを6杯も飲んだにもかかわらず」。	
\\	そのペンの書き味はどう?	
\\	消費者の声はほとんどかき消されてしまう	
\\	「デザートにとってもおいしいラズベリーのアイスを食べた」。	
\\	配達をお知らせする書留郵便物受領通知が必要ですか	
\\	道で大金を見つけるのは必ずしもいいこととは限らない	
\\	間違えることは必ずしも悪いことではない。それから学ぶことができるのだから	
\\	雪が降り続くんだったら、家にいるよ	
\\	使い終わったら満タンにしてくれるなら僕の車使ってもいいよ	
\\	「腕を蚊に刺されてひどいんだよ」。	
\\	彼女とは2度と口を聞かないと心に決めたんだ	
\\	何でも聞いてよ。隠すことなんて何もないから	
\\	彼の秘密兵器は何だと思う?	
\\	「彼には財産があるが、自分で稼いだお金は一銭もない」。	
\\	「この写真、5歳は若く見えるね」。	
\\	「よく見かける拡張子は.
\\	である」。	
\\	「現代医学のおかげで、この病気の回復率は約99パーセントである」。	
\\	「いい歳して若いこと結婚したんだって。やるじゃない」。	
\\	「彼女、荒削りだけど磨けば光るよ」。	
\\	「ブライアン・ドーソンはたった一人しかいないのよ」。	
\\	「それは、消費税が導入される前に、買いだめしていたからですよ」。	
\\	「駆け込み乗車ばかりしていたら、いつか怪我するよ」。	
\\	「ありがとう。きみのお陰で助かったよ」。	
\\	町を出てすぐの所にある丘に隠れ家があるんだ	
\\	「この保険の加入者は、満期前に掛け金の支払を止める」。	
\\	「ペンが書けなくなったので、書けるようになるかどうかを調べるために、少し殴り書きをしてみた」。	
\\	裏付けられないなら、突拍子もない言いがかりをつけるのはよしてくれ	
\\	彼女は四六時中、携帯電話で話している	
\\	彼女は道で僕の所に駆け寄って来てハーイと声を掛けた	
\\	「彼が勝つことに1000円賭けるよ」。	
\\	「寝てるときに動き回ったので、布団から出てしまった」。	
\\	「教育にお金をかけたからと言って、より効果が上がる訳ではない」。	
\\	「彼はことば遊びが好きだ」。	
\\	「そんなところに隠れてないで、パーッとやろうよ」。	
\\	我々の将来は素晴らしいものになりそうにない	
\\	「テレビ広告に十分にお金をかければ、この選挙での勝利は確実なものとなるだろう」。	
\\	「これはプロセスチーズで、自然のチーズではない」。	
\\	張り合う方法を見つけられないなら、僕たちは終わりだね	
\\	言い過ぎだよ	
\\	「私の誕生日、今年は、イースターの日曜日なの」。	
\\	だれだって、年下の人間の部下にはなりたくない	
\\	スキーに行くときには重ね着したほうがいいよ	
\\	「このバッグは重いというよりも嵩張るのよね」。	
\\	膝を擦りむいて、ひどいかさぶたになった	
\\	「もう一軒行きましょう」。	
\\	かつてはテレホンカードを集めてたんだ	
\\	「銀行の引き締め製作により会社がどんどん倒産している」。	
\\	自分の誤りを認めろよ	
\\	「それ見せて」。	
\\	「その壁は、何カ所もひび割れしていた」。	
\\	"ダイエットをする積もりだったが、""誰も気にやしない。ちょっとチョコを食べてもどうってことない""と思ってしまった
\\	この車、古くなってるよね。新しいのを買えるかしらね	
\\	家の掃除は、私にとっては趣味のようなものね	
\\	寒さで指の感覚がなくなった	
\\	ほとんどの人が自分が持つ内に秘めた力のことを知らない	
\\	人はまさかの時に、超人的な力を出すことがある	
\\	「ちょっとセーリングをかじっただけでもはまるよ」。	
\\	あれはまるで価値のない本だよ	
\\	このペン、インクが薄いから、きみが何を書いたのかわからないよ	
\\	「7が僕のラッキーナンバーなんだ」。	
\\	午後はみぞれになるって天気予報で言ってたよ	
\\	「ビル、あるいは、飛行機の中で一箇所電気がショートし、それがきっかけとなりとんでもないことが次々と起こり、大惨事になることがある	
\\	今日は風邪気味なんだ	
\\	風に舞う花粉が、春と秋は僕にとっては大敵なんだ	
\\	私に電話をかけてくるなんてどういうことなの?	
\\	「彼女は、髪を風になびかせ、肩に太陽を浴びるのが好きだ」。	
\\	家族は何人ですか	
\\	「その株は上がるよ。確実だよ」。	
\\	怒らしたのならゴメンな。嫌な話題に触れちゃったかな	
\\	彼らはいい関係だよね	
\\	「日本人には、単語や音節を子音で終わらせるのは難しい」。	
\\	「この筋肉のところにこぶ(ぐりぐり)があるんだ」。	
\\	兄が死に彼は復讐をすると誓った	
\\	彼の堅苦しくかしこまった態度が気に入らない	
\\	そんな堅苦しくしてないで、リラックスして楽しめよ	
\\	腰痛がひどくてね	
\\	この部屋きちんとしないと	
\\	「彼は器用貧乏だ」。	
\\	「強そうに見えるけど、あいつらかっこつけて、いきがってるだけだよ」。	
\\	「彼の初期の作品はみなユニークだったが、最近のはどれも同じようだ
\\	彼女が肩をたたいたとき、首の筋肉が緊張するがわかった	
\\	無事に家に帰ったのね。それで安心したわ	
\\	「そんな堅いこと言うなよ。だれだって少しはずるするんだから」。	
\\	ボールの底で豆がかたまっていた	
\\	「コンピュータが動かなくなっちゃった」。	
\\	「考えがかたまりつつあるね」。	
\\	「これで私のこと思い出してね」。	
\\	「彼は通りをジグザグに歩いて行った」。	
\\	「銀行の資金貸し出しの基準は、ハイリスク、ハイリターンのベンチャー企業の方に偏っている」。	
\\	「この料理はどの食べ物に属するのかな」。	
\\	何週間もまともな食事をとっていない	
\\	「当時、彼は素晴らしいセールスマンだった。この会社では今でも彼の話が出る」。	
\\	その手紙を読みながら、彼女は肩を落とした	
\\	彼らはできるだけお互いにくっついて行進した	
\\	「我々は全然価値観が違う」。	
\\	「彼等は活発に議論をした」。	
\\	僕だったら、勝っているうちに止めるけど	
\\	「あいつの言葉って、本当に、いらいらするよね」。	
\\	「昨年は商売は大変だったが、今は盛り返して来てるから」。	
\\	僕、彼女のルックスが好きなんだ	
\\	「動きを覚えることから始めてください」。	
\\	首の筋を違えちゃったんだな。痛くてしようがないよ	
\\	「彼は格好をつけたがる」。	
\\	彼は頭が良さそうに見せようと格好をつけている	
\\	このかっこ付きの減税は、実は大企業が恩恵を被ることになる	
\\	「あいつは女性が回りにいると偉そうにする」。	
\\	何てかっこ悪いんだ	
\\	このフィルターは活性炭を使用している	
\\	「この仕事、コツが掴めるまで、定石通りにやった方がいいよ。初めてなんだろう」。	
\\	ここへ来て、勝手気ままなことをしてもらっては困る	
\\	彼は、僕が怒ってると勝手に思い込んだ	
\\	「彼はさっさとテーブルのところに行って、勝手にパンチを飲み始めた」。	
\\	「どうぞ、冷蔵庫の中の物、好きに取ってください」。	
\\	彼は天才だったが、同時に、問題の多い、絶えず心の中で葛藤している人物であった	
\\	彼にとって人生は、日々自分自身との葛藤だった	
\\	アリたちは砂糖を見つけ、突然動きが活発になった	
\\	「救急車が彼を一番近い病院に運んだ」。	
\\	「今も覚えているけど、昔は、レコードが黒のビニールで作られてたんだよね」。	
\\	「1970年代、彼は出版業界では活躍していた」。	
\\	「彼に励ましの言葉を言った」。	
\\	「一つの出来事で家族なんて崩壊してしまうものだ」。	
\\	「彼には勝てないよ。 対決しない方がいい」。	
\\	「彼には音楽について学ぶべきことが多くあるね。確かにそうなんだが、彼の歌声は素晴らしいよ」。	
\\	彼は浅はかで、神経質な性格である	
\\	「パーティーに何を着ていくかが決まらない」。	
\\	「これはどういう魔よけのお守りなの?」。	
\\	「悲しいけど、本当なんだね。このキャンディーを作っている会社がつぶれるっていうのは」。	
\\	「言いにくいのだが、君はこのクラスをパスすることはできないと思う」。	
\\	旅行には必ず地図を持っていく	
\\	彼は新車にかなりのお金を使った	
\\	「腕を蜂に刺されて、痛くてしょうがないんだ」。	
\\	きみはどの部活に入るつもりなの?	
\\	「先週は、どれくらいのお客さんが加入したのかな」。	
\\	彼はお金に困っている	
\\	「彼は彼女のパパ(気前よくお金を貢ぐおじさん)だ」。	
\\	「僕の誕生日はクリスマスなので、両親はいつも2つのお祝い一つにしている」。	
\\	「友達にお金を貸してはいけない」。	
\\	「我々は彼に大いに期待している」。	
\\	「この建設計画はあまり可能性があるとは思えない」。	
\\	「ジョーは新入社員の面倒をよくみる」。	
\\	「この時期になるとアレルギーが酷くなるんだ」。	
\\	「以前は自由の身だったけど、今では、世話をしなくてはいけない鼻をたらした子供が何人かいるの」。	
\\	「イチゴを食べると発疹が出るんだ」。	
\\	「この部屋うるさいよね。壁に音が反響するからだな」。	
\\	マラソンランナーは20マイル辺りで経験する突然の疲労のことを「壁にぶちあたる」[体力の限界に達する]という言葉で表現する	
\\	「彼、私のこと目に入らないみたい」。	
\\	「彼女は、優しく、純真な振りをしてるだけさ」。	
\\	「30歳になったとたん、彼女の美貌は衰えた」。	
\\	「いいや、その話はしないよ」。	
\\	きみたち2人は話がすれ違っているよ	
\\	「父の髪は横のあたりが白髪になってきている」。	
\\	「その食料品店は午後の5時から6時まではいつも込んでいる」。	
\\	「どうしてもじゃもじゃに髪伸ばしたの?」	
\\	「いつでもお客様にはちゃんと接しないといけない。どんなに嫌な客でも」。	
\\	「昨日は本当にへまなことを言ちゃったね」。	
\\	「ページの端が金色に縁取れていた」。	
\\	「あいつは正直なのか馬鹿なのか微妙なところだよね」。	
\\	あご髭を生やすことにしたんだ	
\\	「彼には語るべき経験や話が山ほどある」。	
\\	カメラを見ないで。自然な姿を写してるんですから	
\\	「君は公認会計士の資格を取った方がいいかも知れないね」。	
\\	「あのレストランのサービスは申し分ない」。	
\\	彼は自分の小さな世界に引きこもった	
\\	「カラーカートリッジはあまり長くもたないよね」。	
\\	「これカラーでコピーしますか、それともモノクロで」。	
\\	歩道の割れ目につまずいた	
\\	からかうのはよしてくれよ	
\\	「どんなにきつく聞こえようが、彼は思ったことを言う」。	
\\	自分の娘からして私のことが信じられないと思っているんだ	
\\	「一日が48時間あったらな」。	
\\	毎日のように残業が続いては、とても体がもたないよ	
\\	「このペースをずっと保つことはできないよ」。	
\\	彼女は体も精神も健康である	
\\	熱っぽくて体中がかゆい	
\\	「トンカチで指を叩いてしまってウー、死ぬほど痛い」	
\\	「彼、君のことえらく嫌っているよ。君はどうなの?」。	
\\	「年を取るに連れ、筋肉のバランスが崩れて行くものだ」。	
\\	「運動は健康には欠かせない」。	
\\	「ストレスかなあ、胃がきりきり痛むよ」。	
\\	「彼、上の方は空っぽだよ。言っていること分かるよね」。	
\\	息子は、英語を勉強しているという口実で、1日中アメリカ映画を見ている	
\\	食べるより飲む方が好きだ	
\\	この贈り物の包装は中身よりいいよね	
\\	「初めからすべてを真っさらにしてやり直せたらいいのだが」。	
\\	彼のジョークは受けなかった	
\\	「あいつは嫌な奴だから、僕だったら、口論はしないな」。	
\\	「我々の努力は無駄に終わった」。	
\\	「ここで止めようよ。喧嘩したくないから」。	
\\	「彼女には人を引きつける所がある」。	
\\	どうしてそう思うのかな?	
\\	もっと力のあるコンピューターが必要だ	
\\	「チェスで弟を負かした時、彼は絶対に借りは返すと言った」。	
\\	本気になって絵を描いているわけではない。道楽ですよ	
\\	「彼は全く死を恐れていない」。	
\\	彼はかれたがらがらの声で話した	
\\	「あの子、なんて愛らしい目をしてるんだ」。	
\\	「今月はウィークデイは営業し、ウィークエンドは営業しません」。	
\\	朝は、子どもたちに学校の準備をさせて、時間通りに登校させるのがやっとだわ	
\\	「彼女の可愛い顔に騙されちゃだめだよ。彼女はどんな議論をしても負けないから」。	
\\	「真っ黒に日焼けした」。	
\\	「ささくれを引っ張ったら、人差し指から血がにじんできた」。	
\\	そのときになったら、やればいい/取り越し苦労はするな	
\\	「僕は授賞式に行けなかったので、同僚が代わりに行った」。	
\\	「金曜日に仕事代わってもらえますか」。	
\\	「ちょっとお待ちください。兄に代わりますので」。	
\\	いつも冬が始まる前に、何日かちょっと暖かい日がある	
\\	彼は映画おたくだ	
\\	ジョーいるかな。近くにいるなら電話に出して	
\\	「僕のせいだよ。みんな僕はカンがよくないって言うから」。	
\\	「ぎりぎりで最終電車に間に合った」。	
\\	「後ろにピッタリついて運転するなよ」。	
\\	この辺りはあまりバスが来ない	
\\	「彼女にとって、音楽を演奏することはちゃんとした仕事であって、趣味ではない」。	
\\	「彼の声、印象よくないんだよね」。	
\\	「こんな口論をしてもどうにもならない」。	
\\	「思った通りに、ジェイクはまた遅刻した」。	
\\	そんな片意地を張るなよ	
\\	考え方次第で、コップに半分入っている、あるいは、半分無くなったとも言える	
\\	「彼、すごいよね」。	
\\	彼女はすぐに感動して涙を流す	
\\	一個人が環境のために多くのことができるのだろうか	
\\	パソコンでいろいろと違った操作環境を試してみるのは面白い	
\\	"「最近の決まり文句は""環境ホルモン""だよね」
\\	「感情的になって、言うべきでないことを言ってしまった」。	
\\	友達以上の関係になりたいんだ	
\\	私たちの関係はうまく行っていない	
\\	「ドラクソージャパンはドラクソーグループの日本支社である」。	
\\	「悪いけど、その問題は僕には関係がないんだ」。	
\\	「それは僕には関係ないよ」。	
\\	いなくなった犬が無事に戻ってきたという知らせはとてもうれしいニュースだった	
\\	「今年は大幅に石油の価格が下がることになっている」。	
\\	直射日光に当たることが、肌には一番よくない	
\\	公式の席なら、ネクタイをした方がいいよ	
\\	今日のユーロとドルの換算レート知ってるかい?	
\\	「指圧なら、彼が最高だよ」。	
\\	料理に関しては彼は天才だよ	
\\	あなたが私にしてくれたことに対して、感謝の気持ちでいっぱいです	
\\	「バランスのとれた食事をした方がいいよ」。	
\\	「夜中のスナックが、一日の中で一番量が多い食事になる」。	
\\	「演劇の感想かい。がっかりだね」。	
\\	「会議の調整をするのって大変だよね。みんなそれぞれスケジュールや優先すべき仕事があるから」	
\\	「あの人感じが悪いよね」。	
\\	「失職したという実感がまだない」。	
\\	彼のそばに居ると何か落ちるかないんだよね	
\\	「彼を見るとぞっとする」。	
\\	時々感情が高ぶってわれを忘れてしまうことがある	
\\	「なんとか涙がこぼれないように彼女は瞬きをしていた」。	
\\	彼女は感情の起伏が激しい	
\\	その時はわからなかったが、とても重要な局面だったんだ	
\\	関節炎の方はどうなの?	
\\	コンピューターウイルスはとても早く感染する	
\\	「やがて彼の意地悪な態度が我々にも影響を与えてた」。	
\\	この本、あまり印象に残らなかったな	
\\	彼が飛行機で発つ前夜に、われわれはジムを盛大に送り出した	
\\	このシャツ、乾燥機で乾かしたの。それとも外に干したの?	
\\	「私、勘違いしてました」。	
\\	思い違いをするなよ。その名著全部を読めるわけがないだろ	
\\	雨が降るな。わかるんだよ	
\\	感極まって、何も言えなかった	
\\	彼はとても辛抱強い	
\\	近ごろ、体調がよさそうだね	
\\	その試合、とってもはらはらしたね。まだ、どきどきしてるよ	
\\	「昨日の夜、松坂は、近鉄にめった打ちにあった」。	
\\	「それ以外は何でもできるんだけどな」。	
\\	完璧を求めるなら、何もできないよ	
\\	「もう一回勧誘の電話が掛かってきたら、怒りが爆発するぞ」。	
\\	新聞の販売員って本当に押しが強いね	
\\	「自己管理は、君が身につけなくてはならない最も大切な技術(能力)の一つだ」。	
\\	「フランス人の女性はアメリカ人の女性と振る舞い方が違うよね」。	
\\	「彼女の言うことが正しければ、来週まで仕事を始める必要はない」。	
\\	私の父は純和風のものが好きだ	
\\	「彼女は3時半まで外出しています」。	
\\	その規則は僕には当てはまらない	
\\	「彼女は意志が強いよ、頑固と言ってもいいくらいだよ	
\\	「これは単なるお使い(仕事)じゃなんいだよ、いいかい」。	
\\	「橋の支えが崩れ、橋全体が湾に落ちてしまった」。	
\\	「君は経験から学ぶということがないのかな」。	
\\	「職業カウンセラーとして働ける仕事を探してるんだ」。	
\\	「あいつらまだ学生みたいだな」。	
\\	「その知らせを聞いた時、体からすべての力が抜けてしまった」。	
\\	「テレビのせいで社会のレベルが下がっている」。	
\\	「アイビーリーグを出たからって、この会社では何の意味もない」。	
\\	子どもは半額で乗れる	
\\	犬が走り抜けて行くと、草のがさがさする音が聞こえた	
\\	「この映画で主演していた男性俳優、見事に演じてたよね」。	
\\	あー、ガソリンがほとんどないよ	
\\	「ジムでボクシングをするのは、ストレスを発散するにはとてもいい方法だ」。	
\\	あれは根も葉もないうわさに過ぎない	
\\	「ガタガタ言ってるだけじゃ何も片づかないよ」。	
\\	「チャラーン、タラーン(「タダー」と発音)	
\\	「君を雇う前は大いに期待していたんだ」。	
\\	学級崩壊は子どもから教育を受ける権利/機会を奪う	
\\	君の数字と僕のとが一致してるかな?	
\\	「写真を撮る時に、彼は勇ましい格好をした」。	
\\	「古いマンションを売って、6割の利益が出たんだ」。	
\\	「一晩中ビールをグイグイのんでた」。	
\\	気をつけて。彼女はすぐに怒るから	
\\	もう我慢ができない	
\\	意思が強くなければ体重を減らせない	
\\	もうこれ以上我慢できない	
\\	我慢するにも限度がある	
\\	「短気は損気」。	
\\	「遅く帰宅したので、父にこっぴどくしかられた」。	
\\	健康に無頓着ではいけない	
\\	「数人の人が車内で話しているだけ。この時間帯は本当に込んでないわね」。	
\\	「その経験で私の自信はガタガタになってしまった」。	
\\	「あの計画にはまだ何かが欠けている」。	
\\	「そんなにクーラーをガンガンにかけなくてもいいだろ」。	
\\	「頭がガンガンする」。	
\\	「ハイヒールにミニスカート、それにシルバーの口紅、私には全然きれいに思えないけど」	
\\	「彼は、人の言うことを一切聞かない頑固じじいだ」。	
\\	「彼は本質的にそういうとんでもないジレンマに陥りやすいんだ。見ていて痛々しいよ」。	
\\	「できるだけのことはやったが、うまく行かなかった」。	
\\	時々、頑張って博士号を取った甲斐があるのか、と思うことがある	
\\	「頑張って」。	
\\	僕たちのバンドがしている努力は報われるのだろうか	
\\	もっと一生懸命にやるべきだったね	
\\	アパートをピカピカにきれいにするのが好きなんだ。	
\\	「そのイスは彼の体重でギシギシいっていた」。	
\\	「彼は取りつかれたようにピアノを弾いた」。	
\\	「コーチに檄を飛ばされてみな気合いが入った」。	
\\	「何があったと思う。今朝、財布なくしちゃったんだ」。	
\\	知りたいから、聞いてみただけなんだ	
\\	「まずい投資をして、彼は徐々に自分の財産を減らしてしまった」。	
\\	そんな大したことのない仕事に躍起になる必要はないよ	
\\	僕は物覚えが悪いんだ	
\\	僕は忘れっぽいんだ	
\\	ボクシングは忍耐力とスピードの勝負だ	
\\	機会があったら今度電話して	
\\	「彼女は機械類の操作がどうしてもできなかった」。	
\\	「子供達に会う機会はほとんどない」。	
\\	あなたの新しい彼氏のこと聞かせてよ	
\\	「うちの旦那にそういうこと言わせてみたいわ」。	
\\	「どんな健康保険に入ってるの」。	
\\	「彼が転勤になるの知らなかったのか。まだ、公になってなかったんだ。今の話し、聞かなかったことにしてくれよ」。	
\\	「どっちがいいか分からないな。それってリンゴとオレンジを比べてるようなものだよ(比較ができないよ」。	
\\	ステーキが彼の気管に入ってしまった	
\\	「気管に吐いちゃった」。	
\\	「さっき目が合ったとき、かなり脈があるなと思った」。	
\\	実はそうでもないのに関心があるようなふりをするなよ	
\\	スーツに着替えないと	
\\	「外国旅行をすると気が大きくなってしまう。それで何も考えずに簡単にお金を使い過ぎてしまう」。	
\\	頭がおかしくなったのか?卒業数週間前になって学校を辞めようなんて	
\\	きみは飲み込みが早いね	
\\	「彼がそばにいると、落ち着かないんだよ」。	
\\	「そんなにカリカリするなよ」。	
\\	ジェームズをパーティーに呼ばないでくれよ。話したくないから	
\\	「このサービスに対して手数料をいただいてもいいでしょうか」。	
\\	「後どれくらい傘をなくすつもりなの?」	
\\	「彼女を見てると、簡単にシャツが縫えそうな気になるけど、実はそうではない」。	
\\	「どうして君の声に悲しさを感じるんだろう」。	
\\	道路をしっかり見て、気をそらしちゃだめだよ	
\\	何が起きたか気づく前に彼に王手をかけられた	
\\	成功したいと思うなら、もっと積極的にならないとね	
\\	彼は気が長いから、何があっても怒らないみたいだね	
\\	「引退後、彼はやる気をなくしてしまったようだ。毎日ただぶらぶらしている」。	
\\	わがチームは今シーズン勢いがなくなってしまったようだ	
\\	「誰かの感情を害しているかも知れないという感じがする」。	
\\	「そんなに緊張するなよ」。	
\\	頭がおかしいんじゃないの?	
\\	「あわてていたから、それを考える余裕がなかったんだ」。	
\\	「彼女、短気だから、怒らせるようなことは言っちゃだめだよ」。	
\\	「その話を聞いて、憂鬱になった」。	
\\	彼は臆病だ	
\\	会議で頑張って言ったことに、まったく耳を傾けてもらえなかった	
\\	この問題が大変なことにならないようにしてるんだ	
\\	「人の話をちゃんと聞けよ」。	
\\	「その話はとても怖かったので鳥肌が立った」。	
\\	彼女に何を言っても無駄だよ	
\\	冷たい水をがぶ飲みするのは胃によくないとどこかで読んだ	
\\	「明日結婚するって?そう言ったのかい?」。	
\\	この薬はアレルギーに効きますか	
\\	彼女はきみの言うことには一切耳を貸さないよ	
\\	危険と便利さは車の運転にはいつでもついてまわる	
\\	「期限までにローンが払えないのなら、期限の延期を申請しなければなりません」。	
\\	「かすかに聞こえる声で彼女は話した」。	
\\	日本では、外国で暮らした経験がある子どもたちのことを帰国子女と呼ぶことがある	
\\	皆さんこのブランドがお好きなんです。使ってみていいということがわかっていますから	
\\	「このスーツかっこはいいけど、着心地がなんか変なんだよな」。	
\\	「彼は一緒にいて気が楽だよね」。	
\\	「あの学校のやつらは、物知り顔をした、いいかっこしいの集まりだ」。	
\\	「このテキスト文字化けしてる」。	
\\	「その学校の合格基 準はとても低い」。	
\\	この件について何時間議論してもむだだよ	
\\	「先生は6頁の偶数の問題だけをやるように言った」。	
\\	長い目で見れば、貿易規制は経済にとってよくない	
\\	もう夏祭りの季節だなんて信じられないよ	
\\	旬の魚についてウエイターに聞いてみて	
\\	彼女の服は季節に合っていない	
\\	「彼女は毅然としている」。	
\\	「彼はバッグの中にヘロイン1キロを持っていて捕まった」。	
\\	「きちんと生活するなんて僕にはできないね」。	
\\	「不規則な時間で仕事をしていたので、睡眠のパターンがおかしくなってしまった」。	
\\	「子供達には大いに期待しています」。	
\\	そんなに期待し過ぎちゃだめだよ	
\\	がっかりするのは嫌だから、あまり期待したくないんだ	
\\	僕はいつも父の期待に応えようとしてきた	
\\	「太股とお腹を鍛える(しめる)ために運動をしてるんだ」。	
\\	「ハーパー氏の発表について直感的にどう思う?」。	
\\	「彼女は明るい(暗い)性格です」。	
\\	「もう通帳は使っていない。インターネットでやってるから」。	
\\	部屋をきれいにしておきたいんだ	
\\	「とんでもない訴訟に巻き込まれて、それがきっかけで弁護士になろうと決めたんです」。	
\\	どういうきっかけで2人は出会ったのかな?	
\\	病気がきっかけで彼は絵を描き始めた	
\\	「彼の話っていつ聞いてもいいよね。うまれつき話がうまいんだろうね」。	
\\	ハッキリ言わせてもらうわ。あなたとはデートしたくないの	
\\	2度と電話してこないでね。キッパリ別れましょう	
\\	「肉体的、精神的、それに、仕事の面でもストレスがたまっている」。	
\\	彼の話し方にはとげがある	
\\	きついことばになるということを承知の上で、君の仕事に対して本当に思っていることを言わせてもらうよ	
\\	「いやな体臭がするね。今日、シャワー浴びた?」。	
\\	「私はきついことを言うと回りの人に言われる」。	
\\	「他の人をそんなに厳しく批判すべきではないよ」。	
\\	「和やかに話していたのだが、気が付いたら喧嘩になっていた」。	
\\	毎日勉強すれば知らぬ間にできるようになるよ	
\\	「何も分からないうちに(何が起きたか気づく前に)、一番の顧客とほとんどの利益を失っていた」。	
\\	彼女はとても思いやりのある人だ	
\\	この仕事で精神的に消耗している	
\\	僕はストレスがたまることがあまりない	
\\	彼と一緒に働くとストレスがたまる	
\\	ここにはよく来るんですか	
\\	来てくれてありがとう	
\\	「彼らは私のローンの申し出をキッパリと断った」。	
\\	「既得権益を持ったグループが革新政権の悩みの種である」。	
\\	気取るのは嫌だね。フランクに行こうよ	
\\	ゆっくりどうぞ。急いでいませんから	
\\	「その洞窟には入るなよ。嫌な感じがするんだ」。	
\\	何事をするにも、曖昧さや混乱に耐えられないとね	
\\	私はあのレストランの食べ物が気に入らなかった	
\\	この車は、僕の好みではない	
\\	「彼女の言ったことにまだひっかかることがあるんだ」。	
\\	費用のことは心配ないよ。僕が払っておいたから	
\\	彼はただのばかなんだから、彼のことで頭に来ることないよ	
\\	あの娘がだれかずっと気になっていた	
\\	[絶対に]英語を勉強するんだ	
\\	「夜、蛇口から水がポタポタたれいていると気になってしょうがいない」。	
\\	「彼の行動って、あれって思うことがあるよね」。	
\\	「ここのローストダックは僕のお気に入りの料理なんだ。一日中頭から離れなかったんだ」。	
\\	子どもが遅くまで帰ってこないと、心配にならないの?	
\\	古いアルバムを見るのが好きなんだ	
\\	彼らは結婚するほど気が合ってるのかな	
\\	「君からお金をだまし取ろうなんて企んでいないよ。君の思い過ごしだよ(想像の産物)」。	
\\	幻覚なのか、それとも、あれってジョンとマーシーが手をつないで歩いているのか?	
\\	「これから仕事が大変になるぞ」。	
\\	「あなたにこんなひどいことが起こるなんてひどい」。 「この度はご愁傷様です」。	
\\	「これぐらいしかできないんです、本当に」。	
\\	君のコメントが激しい議論に火をつけたんだ	
\\	彼の文章はいつでも痛烈だ	
\\	「暮らし向きが厳しいよね」。	
\\	ほかのみんなにそうするように、自分自身にも厳しきすべきだよ	
\\	10キロ走るのは思っているよりはるかに大変だった	
\\	「父が、物事の善し悪しを私に教えてくれた」。	
\\	転地療養をしてもいいよね	
\\	「夜が更けても、彼らはハンティングや釣りの話に夢中になっていた」。	
\\	「彼、あんなに感情の起伏が激しくないといいんだけど」。	
\\	「彼女の議論は、利己的な詭弁ばかりで、とてもついていけない」。	
\\	「借りた物は大切に扱わなくてはね。敢えて言う必要はないと思うけど」。	
\\	電話自体はただなんだけど、高いのは毎月の使用料なんだ	
\\	あいつのこと真面目に受け取ることはないよ	
\\	彼女のそばにいるといつも気まずくなる	
\\	何時に会えるかはっきりしてないんですよ	
\\	出発は決まったのかな	
\\	「デーブに投票するに決まってるでしょ。他に誰がいるの」。	
\\	ここにはたくさんの不文律がある	
\\	結果ははじめからわかり切ってるのに。どうしてここに座ってなくちゃいけないんだ	
\\	この計画はまだ決まっていない	
\\	「この服装のいいところは、いろいろな組み合わせができることです」。	
\\	彼女の声は少し後悔しているようだった	
\\	「君のような人なら、その状況に対処できるよ」。	
\\	ビーチに一緒に行かない?	
\\	どっちの方向に行けばいいのか決めかねている	
\\	細かく説明している時間がありません	
\\	僕がこの仕事には向いていないと彼は心の中で決め込んでいた	
\\	彼は返金してもらうというはっきりとした目的でその店に行った	
\\	「彼、タキシードを着てるから目立つよね」。	
\\	「彼は注意力を持続できる時間が短い」。	
\\	主人とは相性が合ってると思うわ	
\\	「大事なのは気持ちの方よね」。	
\\	「あいつは感情の起伏が激しい」。	
\\	「これでやとこの問題を片づける(気にしなくて済む)ことができます」。	
\\	飲み会に誘ってくれてありがとう、参加させてもらうよ	
\\	あくせくして働く必要のない環境では、生産性が上がる	
\\	彼の話し方ってぞっとするわよね	
\\	あいつむかつくんだよ	
\\	それを見て、吐き気がした	
\\	もう少し配慮するようにしなさいよ	
\\	帯は、着物のための飾り帯のことで、とても長く、カマーバンド[タキシード用]よりも幅が広く、多くの場合、きれいな刺繍がほどこされています	
\\	「スクリーンにモンスターが現れたとき、彼女は悲鳴(金切り声)を上げた」。	
\\	私たちはいつでもお客様にきちんと接しております	
\\	「彼の意見はとても主観的、かつ、感情的なので、真剣に受け止めることができなかった」。	
\\	今購入すれば、20パーセントキャッシュバック/2割引きします	
\\	[電話で]ジーンのことを待たせたままで、すっかり彼女のことを忘れてしまった	
\\	「社会で確固たる地位を築くには、時間とお金が必要だ」。	
\\	キャンセル料は必要ありません	
\\	「手相を見てもらうと、何かいい気持ちになるから、何かあるのよね」。	
\\	「彼女あんなにスリムだから、何でも着られるのよね」。	
\\	おれたちは、狭苦しいダウンタウンのアパートに住んでるんだ	
\\	教える際に、時には、学生に干渉せず、余裕を持たせるのもいい	
\\	その裁判は休憩のために一時的に休廷した	
\\	今日は3コマ目はないんだ	
\\	「直ぐに殺せたところを、ためらったな」。
\\	「お主もそうだろう。お互い、死ぬ運命か」。	
\\	「君の仕事って、暇なときってあるのかい」。	
\\	「ホール監視役の仕事は、授業中にホールに生徒がいないかどうかを先生に教えることである」。	
\\	明日は祭日だ	
\\	首謀者はだれだ?	
\\	「突然のことだったんで、計画を練ってる時間がなかったんだ」。	
\\	この素晴らしい考えがちょうど頭に浮かんだんだ	
\\	「突然、喉が乾き始めた」。	
\\	沈んだ気持ちでその知らせを伝えたんだ	
\\	「このような犯罪は口にするのも嫌だ」。	
\\	「本ばかり読んでないで、パーッとやろうよ(パーティしようよ)」。	
\\	「共有スペースをきれいにするように協力お願いします」。	
\\	その爆発では半径10ヤード以内は真っ平らになってしまった	
\\	狂牛病に感染した肉だって、十分に調理すれば食べても安全なんだよ	
\\	「今日こそ、ここをきれいにするぞ」。	
\\	おい、座をしらけさせるなよ	
\\	「私は幸せな子供時代を送ったが、間違いなくその町では貧しい部類に属していた」。	
\\	ここはバイクの練習コースです	
\\	「こんなによくしていただけるなんて」。	
\\	「言いにくいことなんですが、今日はお手伝いできないんです」。	
\\	「彼の発音て何かはっきりしないよね」。	
\\	「プラグを抜けば」。	
\\	「コンピュータが固まったので、強制終了をしなくてはいけなかった」。	
\\	「気を抜いてなんかいられない。今週はヘレンと対戦するんだから」。	
\\	「若い頃、兄弟同士で対抗意識を持っていた。そして、今でもそうだ」。	
\\	「いつも一人で仕事をしてるから。僕は、チームプレーヤーじゃないんだ」。	
\\	「僕たち共通点はあまりないんだけど、仲がいいんだ」。	
\\	「彼と僕には、共通していることがたくさんある」。	
\\	「6月12日。(今日で)僕は二十歳だ」。	
\\	届け出なければ、あなたもこの犯罪の共犯になりますよ	
\\	リラックスして。心配することは何もないんだから	
\\	3日間寝ずに勉強したんだ。きつかったよ	
\\	「シェークスピアに興味ある?」。	
\\	息子はピアノに興味を持つことはなかった	
\\	教養のある人にはそのような議論は意味がない	
\\	協力はできませんね	
\\	「昨日の巨人・阪神戦は、駆け引きがすごかったね」。	
\\	「我が社には、地方の地域社会にサービスを提供する支店がたくさんある」。	
\\	「持てる力をすべて出して頑張らねば」。	
\\	「あいつには虚言癖がある。本当のことを言うのを聞いたことがない」。	
\\	「話の途中で質問がある場合には、手を挙げてください」。	
\\	最も一般的な摂食障害は過食症と拒食症である	
\\	拒食
\\	「彼は、遠回しな物言いしかしなかったので、とても嫌だった。率直に何も言ってくれなかったんだ」。	
\\	セールスマンになるには、断られることに対処できるようにならないとね	
\\	彼等が彼の嘘を暴露した時、彼は驚いてハッとしたような顔をしてじっと見ているだけだった	
\\	「僕は不器用だから。細かい仕事はできないよ」。	
\\	「彼女と付き合うの嫌じゃないよ」。	
\\	リスニングの練習が嫌だった	
\\	その映画で一番嫌だったのは、音楽だよ	
\\	「食べられない物は何もない」。	
\\	実際に彼に会って嫌いになった	
\\	「この子と僕は合わない」。	
\\	仕事をきり上げる準備ができたら教えてくれよ	
\\	いつもは何時に仕事をやめて帰るのかな?	
\\	「私の兄は、中国語から英語、英語から中国語を考えることもなく切り替えることができる」。	
\\	「この仕事きりがないよ」。	
\\	「彼女のこと好きかって。俺に言わせないでくれよ」。	
\\	「胃がものすごく痛い」。	
\\	人のことに構うなよ	
\\	彼氏がいたんだけど、先月振っちゃったの	
\\	「今月は生活費がギリギリなんだ」。	
\\	新聞の記事を取っておいたことあるかい?	
\\	「都合のいい時に君と話したいんだけど」。	
\\	仕事と家族の生活をはっきりと区別すべきです	
\\	僕が賭け事をするのと仕事の質とは何ら関係がない	
\\	「君は手の内を見せちゃったんだね」。	
\\	家事を切り盛りするのは簡単な事ではない	
\\	締め切りまであまり時間がない	
\\	「このノミ、切れ味が長持ちしない」。	
\\	目的を達成するには、何らかの犠牲を払わねばならぬ	
\\	「ここでは彼が仕切ってるんだ」。	
\\	ストレス解消のために、週末にテニスとジョギングをしている	
\\	「彼はとても短気なので、直ぐに切れる」。	
\\	彼は気が長いから、決して怒らない	
\\	「こなんことしてくださらなくても。高かったでしょう。すみませんね」。	
\\	お構いなく。大丈夫ですから	
\\	僕に気をつかわなくていいよ。僕がここにいないつもりでいいから	
\\	母はいつもほかの人に気を遣っている	
\\	「デービス夫人があまりに気を使いすぎていたので、くつろげなかった」。	
\\	「ことばに気を付けた方がいいよ。デービスさんは直ぐに怒るから」。	
\\	「しっかりしろよ」。	
\\	「気を張りつめていないと、思ってもみないで困ったことになるよ」。	
\\	「手を抜いたらだめだよ。遅れる余裕などないんだから」。	
\\	気を引き締めて。あと5分でステージに立つのよ	
\\	あまり気を使わなくて済む話題のほうに話を向けようとした	
\\	悪気は無かったんだよ 
\\	気分悪くしてないから	
\\	「ここにある本はとても安い」。	
\\	「その規則に従うことはないよ。絶対的なものじゃないんだから」。	
\\	「千ドルじゃロサンジェルスに行って帰って来られないよ」。	
\\	緊急でなければ救急車を呼ぶな	
\\	政治の話はここではタブーだ	
\\	禁酒をしてるけど、煙草は吸い始めた	
\\	独身者にはそれほど金銭的な負担はない	
\\	堅くなる必要はないよ	
\\	「サイレンを聞いただけでも逃げ出したくなるよ」。	
\\	ふくらはぎの筋肉がつった	
\\	最近起きた贈賄スキャンダルについては、会社で話すことはタブーになっている	
\\	「腕力ではなく、知力に頼るべきだ」。	
\\	彼はギャングの振りをするのが好きなんだ。振りをしているだけなんだけどね	
\\	我々は技術力を向上させる必要がある	
\\	「彼らの話すのを聞いても、彼らが飛んでもない偽善者であることは分からない」。	
\\	偽の100ドル札見たことある?	
\\	会議の間、議事録はとられなかった	
\\	「医者は椎間板ヘルニアだと言った」。	
\\	彼の態度には大いに疑問がある	
\\	「(言っていることが)逆だよ」。	
\\	「彼はとても怒りっぽい」。	
\\	この状況がどうして生まれたのか逆算して考えてみよう	
\\	盗聴と逆探知は別だ	
\\	われわれは逆転勝ちした	
\\	「昨日、巨人は逆転勝ちをし、10連勝」。	
\\	きみの車は国産、それとも輸入車?	
\\	「この寸劇見てよ。とっても面白いから」。	
\\	フラッシュを使えば逆光は大丈夫だよ	
\\	「お互いの溝が深かったので、別れることにした」。	
\\	「議論であいつをギャフンと言わせてやった」。	
\\	今夜の調子はどうだい?	
\\	「口を一杯にして話すのは止めなさい」。	
\\	「予約してないんですけで、待つのかな」。	
\\	ここはとても安くておいしいので、ドアを壊ししてしまいそうになるほど込み合っている	
\\	「よーし。仕事が終わったぞ。ギリギリだったな」。	
\\	「日本では、バレンタインデーのチョコレートは女性が男性に上げる物になっている」。	
\\	「若い世代は義理と人情を重んじない」。	
\\	ライバルの会社が我々のマーケットに食い込んできている	
\\	自分自身の空間が必要なんだ	
\\	「スティーブン・キングの最新作はぶっちぎりのベストセラーだ」。	
\\	「映画が終わるまでテレビから目が離せなかった」。	
\\	「ゴミ焼却炉からの悪臭がひどかった」。	
\\	「あの演劇は見なくていいよ。演技がひどい(くさい)から」。	
\\	「なんでそんなにふさぎ込んでるんだよ。次があるじゃないか」。	
\\	これ腐ってるよ	
\\	「大学で会って以来、彼ってずっと私につきまとってるのよ」。	
\\	この仕事であなたの経理の能力を思い切り生かすことができますよ	
\\	「どうしてこんなにすごい仕事に就けたのか分からないよ。くじ運が良かったのかな」。	
\\	我々はえり好みなどできない状況にあるようだね	
\\	「デイブがアンジェラを捨てたって?そんな。いいカップルだったのに」。	
\\	今週は、天気がよくなるんですって	
\\	「彼は天才だが、とても変わっているから付き合いにくい」。	
\\	「髪の毛がちゃんしてればいいんだけど」。	
\\	「どうして何が起きてるか知らない振りができるわけ」。	
\\	「彼はプロのようにゴルフについて話す。全然ゴルフができないくせに」。	
\\	このビデオゲームはかなりはまるよ	
\\	「ビデオゲームは本当に中毒になりやすい」。	
\\	このテレビ番組、癖になるよね	
\\	「彼の人の扱い方は好きになれない」。	
\\	くそっ	
\\	ひどく疲れているみたいだね。休憩したら?	
\\	「大変な(骨の折れる)仕事をした後に、疲労だけが残った」。	
\\	ビールとチリドッグをください	
\\	「旅行したら、いつも友達にちょっとしたお土産(小物)を買って帰る」。	
\\	もうそんな話はよしてくれよ	
\\	「悪いテキーラって飲みやすいけど、飲んだ後がひどいよね」。	
\\	「表向きの方針はもういいよ。本当のところを言ってくれよ」。	
\\	君は話しは上手だけど、僕は納得はしてないよ	
\\	秘密は守るよ	
\\	「彼は口が堅い(秘密を漏らさない)」。	
\\	「食事をする時には、ちゃんと座りなさいってこれまでに何度言ったかしら」。	
\\	「彼はその情報をうっかり漏らしてしまった」。	
\\	僕には弁舌の才があると人は言う	
\\	「彼は何に対しても利口ぶった返事の仕方をする」。	
\\	「そうなんだよ。しばらく英語話してないと、突然話そうとしてもなかなかできないんだよ」。	
\\	「彼の話し方で、私、一目惚れしてしまったの」。	
\\	「その出来事はアッという間に知れ渡った」。	
\\	「僕の話まだ終わってないんだけど」。	
\\	彼女は口ごもっていたが、やっと何があったのかを僕たちに言ってくれた	
\\	「それは無意味なことばばかりでで、誠実な謝罪ではなかった」。	
\\	「お腹がすいているわけじゃないんだけど、何か口の中に入れておきたいんだよね」。	
\\	「彼は60年代の(流行)歌手だ」。	
\\	エディーは口だけで、ゴルフは全然だめだ	
\\	ジーンと話している時は、言葉を挟むことができない	
\\	口の中をきれいにするために水を一杯飲んだ	
\\	ホテルの食事はどうでしたか	
\\	「父は、いつも感情を外に出さないようにしている」。	
\\	子どもは大人にちゃんとした口の利き方をすべきだ	
\\	「彼は色々言うけど、その分仕事もちゃんとするよ」。	
\\	「あいつまただよ、でまかせを言ってるよ」。	
\\	「お世辞を言ってもだめよ」。	
\\	「あいつとはテンポが合わないんだよ」。	
\\	「彼は話すのがうまい」。	
\\	子どもの頃、どうしようもないほど、ハッキリと物が言えなかった	
\\	教師にそんな口のきき方をするものではない	
\\	他人の話に干渉するな	
\\	「彼は、裁判の間、初めから言っていたアリバイに固執した」。	
\\	「ほら、話をゴチャゴチャにしてる。それは僕が言ったことと違うよ」。	
\\	「この靴履いてると、足が痛いのなんのって」。	
\\	「上まで全部靴ひもを結んだらどうだい」	
\\	ゆったり座ってリラックスしなよ。仕事終わったから	
\\	「(道路を)渡る前に、必ず両側を見なさい。まだ、言い方が十分じゃないかしら。両側を見るのよ」。	
\\	「そんなにくどい説明をする必要はないよ」。	
\\	あなた私を誘ってるの?	
\\	「彼らは自暴自棄になっている」。	
\\	新政府は崩壊の危機に瀕していた	
\\	「彼は、一年間失業中である」。	
\\	「彼女が誰が勝ったのかを言うのを皆待ち切れなっかった」。	
\\	「これだけ準備をしたんだから、負けることはない」。	
\\	「君は、欲しいものと必要なものとを分けないとだめだよ」。	
\\	「目の下がたるんできたね」。	
\\	今日は浮き浮きしてるんだ	
\\	あれはきっと雨雲だよ	
\\	空が晴れ上がったね	
\\	お前の軽薄な考えにはうんざりだよ	
\\	「彼は負けてしまって悔しい気持ちで一杯だった」。	
\\	「彼女は金メダルを受賞して、うれし涙を流した」。	
\\	「彼は、小さなことで悩むようなことはしない」。	
\\	この家は6人家族が暮らすには十分な大きさである	
\\	彼はとても粘り強い	
\\	「外国のお金を使う時は、ゼロをしっかり数えないとね」。	
\\	今日はいいニュースがあるんだ	
\\	「後ろからブーブー鳴らすなよ」。	
\\	システムクラッシュがあっても大丈夫なように、バックアップを取っておかないとね	
\\	「それってリンゴとオレンジを比べるようなものだね」。	
\\	「彼が最高さ。誰も比べものにならないよ」。	
\\	「この会社の新商品はヒットしなかったが、それでも、去年の成功があったのでそれでやっていける」。	
\\	最近、お金の面でちょっと大変なんだ	
\\	「気を付けて。後ろから車が来てるよ」。	
\\	「僕ね、去る者は追わないんだよ」。	
\\	そのクレームはうまく解決できたのかね?	
\\	彼にくれぐれもよろしくお伝えください	
\\	「ここに来るのは、大きな山を登るくらい大変だった」。	
\\	どんなことがあっても、そこに時間通りに行くことに決めていた	
\\	「彼は経験豊富だから、物事がよく分かってるよね」。	
\\	このスープ、塊が残ってるね	
\\	今日はあまり具合がよくないんだ	
\\	「世の中狭いよね。ここで君に会うとはね」。	
\\	一日中さぼっていた	
\\	「急ぐことはないよ。ゆっくりやりなさい」。	
\\	もう少し詳細に言ってくれないかな	
\\	正確には、どこに新居を建てるつもりでいるのですか?	
\\	「同じ事をグダグダ言って、彼、壊れたレコードみたいだ」。	
\\	「何を愚痴ってるの」。	
\\	4台も車を持っている人なんて、きみぐらいしか知らないな	
\\	きみとベスが付き合ってるとは知らなかった	
\\	迂回路[道路標識]	
\\	「彼は高校生の時、とんでもない悪だった」。	
\\	「その学校は、創始者に敬意を表して彫像を建てた」。	
\\	予定を立ててお金を使うようにしなさい。そうすればこのようなことにはならないんだから	
\\	「このような状況を切り抜けるには経験が必要になる」。	
\\	きみは熱心なクリスチャンなの?	
\\	「政治家として、彼は、演技のうまさを目一杯活用している」。	
\\	「新人には、昇進する前に、あまり重要でないクライアントで経験を積ませるようにしている」。	
\\	彼は早口で言葉がはっきりしないよね	
\\	蛍光灯は白熱灯よりずっと電気の消費量が少ない	
\\	「そんなに丁寧に(礼儀正しく)しなくてもいいですよ」。	
\\	その新聞にかなり手直しされた私の原稿が載った	
\\	「その件には関わりたくないんだ」。	
\\	「記述式と選択式、どっちがいい?」。	
\\	アイロン掛けが必要ないということなので、この新しいシャツを買ったんだ	
\\	「継続は力なり」。	
\\	「会議の時間がなくなってしまったので、多くの仕事を来週に回した」。	
\\	「その中で、携帯使えるの?」。	
\\	家に電話しようか、それとも携帯にする?	
\\	あなたの年間経費の平均はどれくらいですか	
\\	「人が何を言ってもいいの。私はジープに乗るんだから」。	
\\	読まずに契約書に署名をしてはいけない	
\\	負けそうになったが何とか勝つことができた	
\\	「彼女は電車の中で枝毛を調べながら座っていた」。	
\\	「彼女は化粧を全然しない」。	
\\	「よかれあしかれ、彼女は僕の妻なんだ」。	
\\	「歯医者がドリルで歯に穴をあけるのって嫌だよね」。	
\\	「この案を拒否するなら、代案を出すべきだ」。	
\\	「この車のデザインは、10年間の献身的なリサーチの結果である」。	
\\	どうだったの?	
\\	「すべてうまく行くさ」。	
\\	「彼を助けようと努力したことが却って状況を悪くする結果になってしまった」。	
\\	「終わりよければすべてよし」。	
\\	なんでそれを1週間前に言わなかったの?	
\\	きみの歌声、全然悪くないよ	
\\	彼この分野では結構経験があるんだね	
\\	「このマフィン悪くないじゃない。いけるじゃない」。	
\\	私が子どもの頃は、女性は若くして結婚したのよ	
\\	「彼のやったことは、仕事場から備品を盗みそれを内緒で売ることだった」。	
\\	「ずっと年下の女性と結婚したので、人は彼のことを見下していたが、それは幸せな結婚だった」。	
\\	この計画の初期段階は、非常に実験的であった	
\\	「あいつは典型的な
\\	型だよ」。 「君の血液型は?」。	
\\	猫はいつも体をきれいにしているが、お風呂に入るのは嫌がる	
\\	あいまいな物言いはきみらしくないね	
\\	あいつはしゃべりすぎだよ	
\\	「道路の縁石につまずいた」。	
\\	賛成の方、挙手をお願いします	
\\	息子は新しい門限に対してけんか腰になった	
\\	いつも彼の過去のことについて聞いたの。だけど言ってくれなかったわ	
\\	「彼の(前)腕は毛深いよ」。	
\\	「一晩中どこにいたの?煙草の煙が充満したバーの奥の方にいたんじゃないの」。	
\\	彼の話を聞いてみんな煙に巻かれてしまった	
\\	その燃えているビルは煙に包まれていた	
\\	「片づけるとしよう」。	
\\	その件に関しては今はコメントできません	
\\	子どものころ、私と兄弟は犬猿の仲だった	
\\	「この問題に関しては、お互いの意見がかけ離れているので、解決できるとは思えないね」。	
\\	どうしてそんなけんか腰で話さなくちゃならないんだ	
\\	「彼女は、歳を取って意地悪(喧嘩腰)になった」。	
\\	ひどい離婚の仕方をすると、子どもの人生がめちゃくちゃになる	
\\	「彼女が一晩中酒を飲み、彼がわめき散らす。それって健全な(夫婦)関係じゃないよね」。	
\\	「週に一回、朝鮮人参入りの強壮剤を飲んでいる」。	
\\	以前よりずっと体重のことを気にしてるんだ	
\\	買い物シールを15枚集めると、懸賞に応募できるよ。もしかしたら当たるかもね	
\\	昨年の倒産件数は過去最高だった	
\\	建設的なものなら、批判されても気にはならない	
\\	「謙遜している振りをする人もいる」。	
\\	「彼は先週働きだしたばかりだが、もう既に(仕事に)慣れている」。	
\\	あいつ、余裕だね	
\\	僕の努力がすべて無駄にならないといいのだが	
\\	「しばらく旅行をすることにしたんだ。それで視野が広がることになるかも知れないから」。	
\\	「同時に3つのアルバイトをしていたことがある」。	
\\	コンサルタントは、彼がいろいろしている仕事のうちの1つに過ぎない	
\\	午前2時ごろに寝てしまった	
\\	このワインで本当に酔ってしまった	
\\	「悩みが多くて夜も眠れない」。	
\\	「明日は、バリウム飲んでレントゲン撮影を受けるんだ」。	
\\	「見たぞ。鼻くそをほじってただろ。何もしていないような顔をしたってだめだよ」。	
\\	「風変わりな物が好きなら、面白いお菓子があるよ」。	
\\	「(もう限界だ)うるさくて、これ以上耐えられない」。	
\\	「お前の愚痴や不満はもう聞きたくないよ」。	
\\	「限界だ。この騒音にはもうこれ以上耐えられない」。	
\\	「限界を感じたら引退するよ」。	
\\	「健康が何よりだ」。	
\\	寒さに強い?	
\\	「会社の規則を破ったということで僕の給料を減らした」。	
\\	「隠しカメラが彼が万引きする瞬間を捕らえた(現行犯)」。	
\\	「胸が痛くなったので心臓マヒかた思ったが、医者が言うには、気をつけろという警告だ、とのことだった
\\	食べるために働かなくてはならない。そういうことなんだよ	
\\	「ブロードウェーのミュージカルがとても人気があるのは、現実から離れ素晴らしい時を過ごせるからである」。	
\\	「理想的なことばかり言わないで、たまには現実的になってみらた」。	
\\	「部屋の中の物に手を触れてはいけない。現状のままにしておいてくれ」。	
\\	僕の弟は自分の小さな幻想の世界に住んでいるように見える	
\\	「私の姉は、限定販売の玩具や人形を集めている」。	
\\	彼は信仰を新たにしたクリスチャンだ	
\\	「それはできない相談だ。限度ってものがあるだろう」。	
\\	健康でいたいと思うなら、お酒を飲んだりタバコをすったりする量には限度というものがある	
\\	僕の信用格付けはけっこういいんだよ	
\\	それはそこにいなかったらわからないよ	
\\	「シドニーのキース・ジョンーズからのレポートでした」。	
\\	「どのような商品でもお取り寄せできますが」。	
\\	「資金りがつかないため、会社は電気製品で(給料を)払うことになっている」。	
\\	彼女は自分が興味のある男たちを虜にする術を知っている	
\\	「あの服装ちょっと決まってるね」	
\\	日本ではおいしいワインにチーズとアボガドが手に入るから、なくて困るような物は何もない。	
\\	真の愛を妨げることはできない	
\\	私をもてあそぶのはやめて	
\\	旗が半旗の位置に掲げられている	
\\	彼には威厳がある	
\\	「ああいう感じのヘアースタイルにしようかしら」。	
\\	「メディアがわれわれのことを好意的に取り上げてくれるといいのだが」。	
\\	雨が降ったら、降ったらでいいよ。気にすることはない	
\\	ここに来ることができとても光栄です	
\\	この漢方薬きくと思う?	
\\	お近くの劇場で上映されます	
\\	葬儀は愛する人の死のあとで、心の整理をつける助けになる	
\\	「彼女に教わると、レッスンがよく分かるよ」。	
\\	「彼は会社で出世した」。	
\\	「彼女を雇いましょう。とてもテキパキした人ですから」。	
\\	「今日のクライアントはなかなか好感触だったよ」	
\\	首相の支持率はこれまでで一番低い	
\\	「このカウンターは表面に抗菌処理が施されている」。	
\\	「シートベルトは命を救う役目を果たす」。	
\\	「きっと気に入っていただけると思いますよ」。	
\\	「政治家は利害の対立を避けなければならない」。	
\\	「彼は、獲物をねらうライオンのように、じっと大きなチャンスが来るのを待っていた」。	
\\	突然、攻守が入れ替わった	
\\	あいつの口臭はひどいよ	
\\	「君って、何か怖いものある(恐怖症)」。	
\\	彼は風邪を口実にして仕事を休み、家でビデオを見ていた	
\\	「勉強進んでる?」。	
\\	「君にはどうしても成功したいという強い気持ちはないのかね」。	
\\	インターネット上のポルノは公衆道徳に対する脅威である	
\\	そろそろ攻勢に出る時期だ	
\\	「休んだ方がいいみたいだね。休憩しろよ、僕が代わるから」。	
\\	このレストランは場所はよくないけど、食べ物はいいよね	
\\	交通量が少ない時には、車で会社まで15分で行ける	
\\	映画館では信じられないという気持ちは持たないようにしないとね	
\\	「昨日の夜のパーティーでのお前の行動はとんでもなかったよ」。	
\\	「今日はあまりやる気がしない」。	
\\	さあ、やって見ろよ	
\\	「彼は、唇がひどく腫れていた」。	
\\	人生において、こうなったらいいなと思っていた状態に今いる	
\\	「会社の序列からすると彼女より私のが上なのだが、彼女はそのことを忘れる傾向がある」。	
\\	候補者が選挙前にディベートを行った。	
\\	「見かけからは分からないだろうけど、彼は大富豪なんだ」。	
\\	「政治家の公約など、もうだれも真剣には受け取らない」。	
\\	「6時間交代で働かせた方が、従業員の能率がずっと上がる」。	
\\	本を読むより、自分でゲームの攻略法を見つけるのが好きなんだ	
\\	「コオロギが窓の外で一晩中鳴いていた」。	
\\	「あなたって食べ物にとってもうるさいわよね」。	
\\	私の保険契約は 控除免責金額を超える医療費すべてをカバーしてくれる	
\\	「セールスマンが近づいてきて、売り込みを始めた」。	
\\	とても寒かったので、テーブルの上に置いておいたジュースが凍っていた	
\\	暑い日は、木陰で冷たい物を飲むのが一番だ	
\\	息子の学費のために彼はかなりお金を貯めた	
\\	私の上司は人使いが荒い	
\\	「彼の偉そうな態度が嫌なんだ」。	
\\	反抗的な十代の少女たちを見て腹を立てる人が多い	
\\	「このワインにはとてもこくがある」。	
\\	私は地球の市民になりたい	
\\	「脅しは僕には通じないよ」。	
\\	「ごめんなさい。チキンを丸焦げにしちゃったの」。	
\\	ここはいつもこんなに快適ではないんでしょう?	
\\	秘密がもれてるよ	
\\	「背中のこのあたりが痛いんです」。	
\\	最近何やってるの?	
\\	最近疲れてるようだね	
\\	「最近、調子はどう?」。	
\\	以前、どこかで会いましたか?	
\\	その好意に対して、彼女はここらから感謝の気持ちを表した	
\\	「彼は、一日一冊本を読むと心に決めた」。	
\\	ビーチに行くときには
\\	ローションをつけるようにしなさい	
\\	無理言って悪いんだけど、少しお金貸してくれないかな	
\\	「そこにはいるけど、心は別のところにあるのね」。	
\\	「若い頃に、医者になろうと決めたんだ」。	
\\	君がここにいるので、心強いよ	
\\	あのバンド、なんかまとまりがなかったね	
\\	そういうつもりじゃないんだろう?	
\\	おい、少しは同情してくれよ	
\\	「彼女は人の心の動きにとても敏感だ」。	
\\	失敗するんじゃないかという恐怖感がいつでも心のどこかにあった	
\\	「人に対しては、いつでも公平にね」	
\\	「心構えがすべてだね」。	
\\	討論が
\\	産業の話になったとき、僕は自信を持って話せた	
\\	「直ぐに治るよ」。	
\\	「行いを改めたことを示すために彼は頭を丸めた」。	
\\	行いを改めないなら、別れると彼の妻は言った	
\\	"上司に跳べと言われれば、彼は""どのくらい高く""と答える=[言われたことは何でもする]
\\	「恐怖で縮み上がってしまった」。	
\\	彼女は社交的だから一緒にいて楽しい	
\\	「彼、今日はあまり気合いが入ってないみたいだね」。	
\\	怖じ気づくなよ	
\\	彼に投票するのは断固反対です	
\\	いつか痛い目にあうぞ	
\\	この自動販売機、故障している	
\\	台所から音が聞こえたので、彼女は怖くて凍りついた	
\\	その発疹をかいちゃだめだよ。もっとひどくなるから	
\\	アリの最後のパンチで彼はやられた	
\\	「ここで僕を見かけたって言う必要ないよ。話がこじれるだけだから」。	
\\	チームプレーヤーじゃないとしたら、きみはこの業界ではやっていけないね	
\\	「ビルとジャニスは今年大きな旅行を計画している。2ヶ月かけてヨーロッパを気ままに動き回る積もりでいる」。	
\\	「トラックが私にこするように接触し、そのまま行ってしまった」。	
\\	「あの格好、彼女らしいよね(個性が表れている)」。	
\\	サリーは独特な性格をしているよね	
\\	「彼女が電話が来るとは思っていないから(こそ)、電話をすべきなんだよ」。	
\\	彼らはこそこそ話していた	
\\	「自転車のキーキーいうブレーキの音ってとても聞いていられないよね」。	
\\	広告の内容が正しいことは、極めて重要な原則である	
\\	「ステーキ好きに料理していいよ。僕は何でもいいから(好みがうるさくない)」。	
\\	「彼女はそんなに好き嫌いはないよ」。	
\\	どうして無意味な細かいことにこだわるのかな?	
\\	彼は自分のルックスにこだわっている	
\\	とんでもありません。こちらこそ	
\\	「僕の立場になってよーく考えてみてくれよ。それから、批判をすればいい」。	
\\	「コツが分かってきた。こうだよね。どうだい。分かったぞ」。	
\\	こつこつ勉強すれば、いつかは成功するよ	
\\	きみの1週間[1カ月]のお小遣いはいくら?	
\\	定期預金を始めることにした	
\\	「せこい手を使ってもどうにもならないよ」。	
\\	「お正月にどんなことをしようと誓いをたてたのかな」。	
\\	「今年は一緒に仕事ができてとても楽しかったです。来年も宜しく」。	
\\	「彼女は事なかれ主義だ」。	
\\	もし彼女に言うとしたら、どうやって言うんだい	
\\	その寄付を受け取った慈善団体は感謝の言葉を送った	
\\	「何言っても、彼は真剣に受け止めてくれない」。	
\\	「誤解があったようですね。もうすこしちゃんと説明すべきでしたね」。	
\\	人と話した後、もう少し自分の考えをはっきりと言えたらなといつも思うんです	
\\	「娘の結婚式を見ていて、胸が一杯になった」。	
\\	「ニュアンスをすべて伝えられるような英語の語彙がない」。	
\\	「彼の申し出を断わったんだ」。	
\\	子どもみたいなことをするのは止めなさい	
\\	「彼の子供じみた意見が嫌なのよね」。	
\\	「すごいじゃないか。君のコサックダンスは十分に磨きがかかってるよ」。	
\\	「この業界では彼はかなりコネがある」。	
\\	この辺りに住んでいるんですか	
\\	「今日はスケジュールが一杯なんだ。」	
\\	「あなたのコートの破けてるところ縫って上げるわね。そのついでに取れたボタンもつけてとくわ」。	
\\	こんな時、どうしていいかわからない	
\\	「現在の状況からすると、今年は、新しい営業マンを増やす余裕はないね」。	
\\	最近は何やってるんだい?	
\\	「多分、英語の方はなんとかなると思いますので、翻訳する人を雇う必要はないでしょう」。	
\\	「今日は、見逃してやるが、この当たりで顔を合わせないようにしろよ」。	
\\	触らないで。そのままにしておいて	
\\	このままで行くと、バンクーバーに真夜中までには着くだろう	
\\	このままだと時間どおりには終わらないだろう	
\\	「この分だと、お前、宿題終わらないぞ」。	
\\	議論が堂々巡りをしているだけだ	
\\	これで終わり、それとももっとやるのかな?	
\\	どこかで譲らないと、このままでは大きな戦争になる	
\\	「そのままにしておいてね」。	
\\	あなた、いい服の趣味してるわよね	
\\	「いつまで親のすねをかじってるんだよ」。	
\\	「ごまをするのは止めろよ」。	
\\	お客様がメジャンプしろモと言ったら、メどれくらい高く飛べばいいでしょうモと言うんだ	
\\	「気を付けないと、コップをひっくり返すよ」。	
\\	ゴミは2種類に分けるだけでいい	
\\	お皿を洗うなんてちょろいもんさ。食器洗い機があるから	
\\	彼女はまめに庭の手入れをする	
\\	講演者が突然キャンセルしたので、イベントの主催者たちは困ってしまった	
\\	「大変なことをお願いしているのは分かっているのだが、明日までにこれを仕上げて貰えないだろうか」。	
\\	「森高が結婚するっていう噂だけど。聞いた?」。	
\\	「彼は一日中机に座りっぱなしなのが嫌になって、仕事を辞めた」。	
\\	「今日も何もすることがない。昼寝でもするか」。	
\\	「ここは会社だよ。ティーパーティーやってるんじゃないんだから」。	
\\	新鮮なピクルスはとてもコリコリしている	
\\	「インターネットって癖になるからね。彼、完全にはまってるよ」。	
\\	「今朝起きたら腰が痛くて」。	
\\	また酔ってるのね。懲りないんだから	
\\	「お前も懲りないね」。	
\\	悪いけど、パンクロックには興味がないんだ	
\\	せめて家族を助けることくらいはしないとね	
\\	「これからは、リサイクルしていない紙はここでは使いませんから」。	
\\	「このレストランの(料理の)量、とても少ないよね」。	
\\	「今日は特にすることあるのかな」。	
\\	僕が大学で経済学を専攻してたって信じられる?	
\\	「人目をはばからずに売り込もうとすると、客が怖がって逃げてしまうよ」。	
\\	私の主人は貧しく育ったので、その分子どもを甘やかしている	
\\	彼には私が愛してしまう特別の何かがあるのよ	
\\	人生には浮き沈みがある	
\\	「最近は、ずっと続いている不運から何かを得ようという気持ちになっている」。	
\\	「彼とは2人りきりで話したくないよ。怖いんだもの」。	
\\	「彼は僕を見て嫌な顔をした」。	
\\	1。「怖くて手で目を覆っているんだけど、指の間から見てしまうという衝動に駆られた」。2。「怖くて手で目を覆っているんだけど、指の間から見てしまうという衝動に駆られた」。	
\\	「そんな風にしないの。彼が怖がるでしょ」。	
\\	営業部は3つの課に分けられる	
\\	「彼女が声を張り上げて叫ぶのが聞こえた」。	
\\	1。「その男は、怖そうな顔をしていた」。2。「彼の顔は、トラックに轢かれたたような顔だった」。3。「こいつひどい顔、犯罪者のような顔つきだよ」。4。「昨日の夜、酔った勢いでいかがわしいバーに入ったんだよ。そうしたら案の定、強面の兄ちゃんが出てきてさ。俺達二人、ぼられちゃったってわけ」。	
\\	子どもの時は、[壊れかかった]あばら屋に住んでいたんだ	
\\	「この種の仕事には、忍耐力が必要なんだよ」。	
\\	そのような言葉には根拠はない	
\\	僕の将来どうなると思う?	
\\	この部屋にはコンセントがあまりない	
\\	「ハードコンタクトってもう誰もつかってないんでしょ?」。	
\\	何をたくらんでるんだ	
\\	今日のメニュー献立は何かな	
\\	「この次は頑張って」。	
\\	「こんな遅い時間に電話を掛けない方がいいよ」。	
\\	あれはとっても屈辱的だったよな	
\\	ちょうど昼寝の時間だな	
\\	次に何が起きるかわからないよ	
\\	「彼がそんなに無礼なことを言うとは思わなかった」。	
\\	風邪引きそうだなと思ったら、放っておいちゃだめだよ	
\\	「首?オレの人生ってこんなものかな」。	
\\	「お前の馬鹿げた論理にはついていけないよ」。	
\\	かつては小型車に乗ってたんだけどね	
\\	子どものころよく冷凍テレビ食[色々な具が詰め込まれていて加熱して食べる食品]を食べた。	
\\	「この業界は、10年で完全にデジタル化されるだろう」。	
\\	「保険関係の仕事を何年かしてたんだけど、今は新しい仕事を探してるんだ」。	
\\	「彼女には絶対に謝るつもりはない」。	
\\	「不安になることはありません」	
\\	「力ではなく、うまく人を動かす方がずっといい」	
\\	彼は豪快に食べて飲んだ	
\\	「私は、両親と兄弟二人、祖母と暮らしています。家族みんなで6人になります」。	
\\	父が私にはヒーローなんです	
\\	昨夜のパーティーは、夜のお相手を捜すためのものだった	
\\	なんて安っぽい指輪なんだ	
\\	彼女は駅前で友だちにばったり出会った	
\\	「面倒なことになるよ」。	
\\	「外に出て、新鮮な空気を吸うか。そうすれば、春先のもやもやした気分も吹き飛ぶだろう」。	
\\	今日はどうしてそんなに機嫌が悪いの?	
\\	どうして彼は今日いらいらしてるんだ?	
\\	彼女は派手な色のガウンを着ていた	
\\	「お疲れさまでした、どうも」。	
\\	「このソフト、バグだらけで使い物にならない」。	
\\	そこに時間どおりに着けると思ったのが間違いだった	
\\	ご支援ご援助いただきありがとうございます	
\\	君は夜型、それとも朝型?	
\\	「『五体不満足』とは、健康な体を持つ人のことを表すために使われる『五体満足』をもじった言い方です。その本のタイトルは、『不完全な体』に近い意味を表すように変えられているのです。乙武さんには、手も足もないのですが、暖かい心、知性、そして、才能に満ちあふれています」。	
\\	お前の部屋めちゃくちゃに散らかってるじゃないか	
\\	「いや、どうも、貴重な情報をありがとうございます」。	
\\	ふー、お腹が一杯だ	
\\	「この部屋ちらかってるね。少し整理しなさいよ」。	
\\	「お弁当のおかずがゴチャゴチャになってた」。	
\\	「邪魔しないでくれよ」。	
\\	ぐずぐずしていると、あとでもっと大変なことになるぞ	
\\	去年、株から早めに手を引くべきだったんだ	
\\	じゃあな。ちゃんと連絡しろよ	
\\	一週間休もうと思う	
\\	あいつらにどのくらいだまし取られたんだい?	
\\	そんなごますりはやめろよ	
\\	あいつはごますりだ	
\\	「このあたりでは火曜日と木曜日に燃えるゴミを集めに来る」。	
\\	われわれが使うすべての包装用の包みや箱はゴミの一部となる	
\\	きみが分別したら僕がゴミを出すよ	
\\	この計画に係わるのはごめんだ	
\\	その利益集団は、議会を意のままに操っている	
\\	では、この絵の筆使いを注意して見てください	
\\	黒板を見てください	
\\	1。彼はごり押しした2。彼は自分の地位を利用してその案を承認させた3。会社はそのプランを我々に無理に押しつけた。4。この企画、無理があると思っうんだけど。彼らのゴリ押しで通ってしまった。途中で頓挫するのが関の山だと思うけどな。	
\\	「午後は仕事場でぶらぶらしていた」。	
\\	「午前中はぶらぶらして過ごした」。	
\\	「今日の彼女のパーフォーマンスは最悪だった」。	
\\	「最悪の場合の準備をしていた」。	
\\	私のお気に入りのサイトが一時的にクローズされていたが、再開された	
\\	お願いだから最悪だったときのことを思い出させないでくれよ	
\\	今年こそ最高の年にしよう	
\\	「恋したことってある。その時の気持ちって最高だよね」。	
\\	「このプロジェクトを最後の最後まで見届けたい」。	
\\	この計画は十分できると思うよ	
\\	この駅の最終は何時ですか	
\\	そろそろ転職しようかな	
\\	最後はお金の問題になる	
\\	「第一印象がとても大切である」。	
\\	あの時、彼が薦めてくれたレストランに初めて行ったが、あれが最後でもある	
\\	「我々のビジネスが失敗した理由を一つだけあげることはできない」。	
\\	「ハリーが来ると、いつもパーティーが白けてしまう」。	
\\	母親に、母の日のカードを送れよ。それくらいはしなくては	
\\	最低限ごめんくらい言えばいいのに	
\\	この映画により、ベルボトムが復活した	
\\	「彼は自分の部下をどう使えばいいかを知っている」。	
\\	我々のチームは長いスランプのあと、何とか再浮上しようとしている	
\\	「この番組は、スタジオの聴衆の前でライブ録音された」。	
\\	この会社は借金で身動きが取れない状態だ	
\\	「この話、採用するには長すぎるんだよ」。	
\\	「僕、鉄棒で逆上がりができるんだよ」。	
\\	「彼女に何とか好きになってもらおうとしたが、返って嫌われてしまった」。	
\\	それは僕の性に合わない	
\\	「あの議論はどうもひっかかる(気にくわない)」。	
\\	あの店って酒屋なの、ドラッグストアなの?	
\\	「僕の高校ではサッカーがとても盛んだった」。	
\\	皆が怒りを声に出して言うようになって、会話のレベルが下がった	
\\	私の降りる駅は、1つ先です	
\\	この問題は今は後回しにしよう	
\\	「少し走っただけでもう疲れてる。10キロ走った後にどうなってるか考えたくもない」。	
\\	話がどこに行くのかが見えない	
\\	「アインシュタインは時代を先取りしていた」。	
\\	「お便り心待ちにしております」。	
\\	先に帰るから。それじゃ	
\\	この先どうなるかは誰もわからない	
\\	明日は雨だってさっき聞いたよ	
\\	この取り引きは会社の将来を脅かすかもしれない	
\\	「日刊紙はいつも他紙を出し抜こうとしている」。	
\\	今回の人員削減には解雇も含まれるのだろうか?	
\\	この作文には中身がない	
\\	「彼はお酒を飲むと陽気になる」。	
\\	彼は付き合い程度にお酒を飲む	
\\	「風邪を言い訳にして、会議から(こっそりと)抜け出した」。	
\\	「かつて商売を始めたことがあるんだけど、どうでもいいようなことで随分苦しんだよ」。	
\\	悩みを聞いてくれる人が欲しい	
\\	「彼の贈り物は、上辺だけのものだった」。	
\\	「でしゃばる積もりじゃないんだけど、もう仕事見つけたの」。	
\\	「10パーセントの所得税が自動的に給料から差し引かれる」。	
\\	「今夜は寿司にしようか。どう?」。	
\\	彼は突き刺すような[辛辣な]ことばを言った	
\\	「あなた帰国子女でしょ。英語で分かるもの」。	
\\	「数学できるよね。彼女、5分で僕が聞いた問題問ちゃったもの」。	
\\	「それで、また鍵忘れたわけ。お前らしいよ」。	
\\	いつも通り、ジーンは20分遅刻だ	
\\	さすがに運動選手だけあって、彼は10分足らずで駅まで走っていった	
\\	「君の携帯使わせてもらっていいですか」。	
\\	「彼は会社で窓際に押しやられてしまった」。	
\\	「その事故現場には大勢の人たちが集まっていた」。	
\\	長いフライトのあとでさぞかしお疲れのことでしょう	
\\	「ジュリーとビルが別れたんだった。合わなかったんだろうな」。	
\\	「つねってみて。夢を見ているに違いないよ」。	
\\	「このショーさっきからずっとテレビでやってるね」。	
\\	ちょうど彼女との電話を切ったばかりだったんだ	
\\	熱すると消毒できる	
\\	さっさと食べちゃって。9時までにはここを出ないといけないから	
\\	ねえ、言ってよ。人の心なんか読めないんだから	
\\	待っても意味がない。早速取りかかろう	
\\	会社に着いたらすぐに電話して	
\\	「通販放送後1時間の間に電話がどっと掛かってくる」。	
\\	「このサラダドレッシング、あっさりしてておいしいよね」。	
\\	何かあっさりしたものが食べたいな	
\\	「僕の言ったことが少しは分かったのかね」。	
\\	「一週間仕事を少しもしなかった」。	
\\	彼女をデートに誘ったが、言うだけ無駄だった	
\\	「彼は、真っ正直な人だ」。	
\\	「自動車工場が閉鎖されてから、この場所はゴーストタウンになってしまった」。	
\\	「何人かのモデルと写真家が公園で撮影をしていた」。	
\\	「このペースで仕事してたら死んじゃうよ」。	
\\	自分の失敗に面と向き合わないとだめだ	
\\	「あの先生、いつも原稿の締め切り守らないからね。締め切り日を1週間サバを読んで伝えることにしよう。そうしないとこっちが大変な思いをすることになるから」。	
\\	そのフェスティバルまるで活気がなかったね。人が来てないんだもの	
\\	きみがいなくなってから、この街も寂しくなったよ	
\\	「この街は今は賑わっているが、90年代はひどく寂れていた」。	
\\	それって冗談なの?	
\\	「君の感情を傷つけるつもりはなかったんだ」	
\\	「子供の時、一度も授業をさぼらなかった(ずる休みしなかった)」。	
\\	今日、学校をサボったのかい?	
\\	をつけて人のことを呼ぶのは好きではない	
\\	あなたが居なくなってしまったので、今は寂しい	
\\	「出来上がった原稿をすべてまとめて送る方がいい」。	
\\	「お前、冗談きつよ」。	
\\	「彼は寒冷な気候はまるでだめだった」。	
\\	「その
\\	シャツじゃ凍えるほど寒いだろう」。	
\\	「二人の兄弟とも彼女にふられた(愛を勝ち得ることができなかった)」。	
\\	「違うよ、逆だよ。彼が最初にぶって、その後に僕がぶったんだよ」。	
\\	このビジネスで成功するかどうかはコミュニケーション能力で決まる	
\\	このローションは肌をすべすべにする	
\\	「彼は後からの付け足しでその仕事をした」。	
\\	「自動的に訂正された」。	
\\	だまされた気がする	
\\	騙したこともあれば、騙されたこともある	
\\	「市長の部屋から盗まれた大金が見つかり大騒ぎになった」。	
\\	大変ご迷惑をおかけしました	
\\	子どもたちはいきいきとさわやかな顔をしていた	
\\	「彼を怒らせるのは、眠っている巨人を起こすようなものだ」。	
\\	「多くの人が、依然として、産休は口実に過ぎないと思っている」。	
\\	何が言いたいんだ。俳句の朗読会じゃないんだからさ	
\\	「これって便利だよね」。	
\\	「この参考資料をメールに添付するのを忘れないようにね」。	
\\	このペーパーを書くのに、どのような参考資料を使ったのですか	
\\	彼の研究は、ルパート博士のものをかなり利用している	
\\	ご参考までに、この車はディーゼル燃料で走ります	
\\	「スペシャルゲストに大きな拍手をお願いします」。	
\\	「彼は立て続けに罵りことばを言った」。	
\\	どうしたんだよ。ひどい格好をして	
\\	「その法案はまだ大多数の賛同を得られない」。	
\\	「僕の従兄弟は、家族の元へ帰るよりも、お酒を飲みに行きたいと思っている」。	
\\	「バーベキューに20人以上は来ると思ったが、ここにいる6人だけだった」。	
\\	「私の叔父は、とても多才な人だ。やりてのビジネスマンで、料理もうまく、また、素晴らしい運動能力の持ち主でもある」。	
\\	破産宣告をする前に、賛成と反対の意見を慎重に検討した	
\\	この果物は遺伝子組み替えでできたものだ	
\\	もう犬の散歩はしたの?	
\\	「その話は一面記事にはならない」。	
\\	先週は3連休だった	
\\	在庫数は商売の非効率を測る物差しとなる	
\\	現在、100名の正社員がいます	
\\	「彼は家で仕事をしている」。	
\\	「どのような基準で新しい人を雇うのですか」。	
\\	「去年、彼は個人的に深刻な問題をいくつか抱えたが、見事に立ち直ったよ」。	
\\	「人生を振り返って無駄だったと思ったことは一度もない」。	
\\	僕には正直に言ってくれてるんだよね?	
\\	最近、週刊誌はスキャンダルの記事だらけだ	
\\	だいたいの見積もりを出してくれるかな	
\\	「最初の章をざっと読んで、なかなかいい本だということが分かったよ」。	
\\	「ホワイトノイズは時に心を落ちかせる」。	
\\	午後は喫茶店でおしゃべりをして過ごした	
\\	「僕をいじめるからそういうことになるんだ」。	
\\	「この仕事はいつも同じだ。一歩前進しては、二歩後退する」。	
\\	それはザルで水をすくうようなものだ	
\\	「斬新なデザインで、なめらかな輪郭がいいよね」。	
\\	「今、現金の持ち合わせがあまりないんだ」「おけらなんだよ(一文無し:スラング)」	
\\	「残念だな、今週末、暇じゃないんだ」。	
\\	「彼等が君の才能を分からなかったのは残念だ」。	
\\	サッカーの試合を見に行ったことがあるかい?	
\\	「結婚をして彼は少し太った(ウエストが太くなった)」。	
\\	この
\\	ドライブはディスクに書き込みができるのかな?	
\\	「時には、コマーシャルの方が番組より面白いことがある」。	
\\	「それは控えめに言ってるんだよ」。	
\\	なんだよそれは/もういい加減にしてくれ)	
\\	このスープ、もう少し塩を入れた方がいいね	
\\	彼女はいつもおとなしく遠慮深い	
\\	「仕返ししようとしないで、仲良くしろよ」。	
\\	彼は英語教員の資格を持っている	
\\	彼女の寝室に盗聴器が仕掛けられていた	
\\	の後に目的語として動詞が 来る場合には、
\\	形にします。
\\	不定詞は使えません」。	
\\	彼は何かのことで怒っていると思わざるをえなかった	
\\	政治が腐敗してるのはわかってるよ。だからって、どうするって言うの?僕は考えないようにしてるんだ	
\\	会社に行くのにほかにいい方法がないから、こんなひどい電車に乗って通勤してるんだ	
\\	今日の試験に失敗するわけにはいかない。最後のチャンスなんだから	
\\	「どこか痛いのかい(そういう顔をしている)」。	
\\	遅く家に帰ったので父親にしかられた	
\\	「おまえだって、そんなに長くどうでもいいようなサラリーマンをやるつもりはないだろ」。	
\\	「彼は出来の悪いマネージャーだ」。	
\\	彼女は子ども時代不幸だったので、心に多くの荷物を背負っている	
\\	「つんつんしている奴には関係がない」。	
\\	「ちょうどペーパーを書き終えた時にコンピュータがクラッシュしてしまい、初めから書き直さなければならなかった」。	
\\	そこには入れませんよ。そこはロープで仕切られていますから	
\\	我々の資産は400万ドルを超えている	
\\	その会社には膨大な資金がある	
\\	「彼女の身のこなしはいつもとても優雅だ」。	
\\	「どうしたの。びしょぬれじゃやない」。	
\\	この問題のために勉強をしたのに、試験に出なかった	
\\	「頭の切れる人と話すといつも精神的に刺激を受けるよね」。	
\\	「よい玩具やゲームは子供の好奇心を刺激するように設計されている」。	
\\	歯医者が歯石がいっぱいたまっていると言った	
\\	権力闘争はよく悪感情を引き起こすことになる	
\\	「誰がそんなこと言うかね」。	
\\	「大学一年の時、ひどいしごきにあった」。	
\\	「非番の警察官が泥棒にタックルし、地面におさえつけた」。	
\\	「くだらない話をするために給料払ってるんじゃないよ」。	
\\	彼女は仕事のために生きている	
\\	「どうでもいい電話ばかりかかってくる」。	
\\	「彼は、仕事中毒そのもだ」	
\\	彼は完全な会社人間だ	
\\	彼女は仕事できる人だよ	
\\	彼は不動産業界でひと財産築いた	
\\	「これ味見して、どうか聞かせて」。	
\\	運動する時は、やり過ぎないように	
\\	落ち込むなよ	
\\	僕は野外が好きなんだ	
\\	「マーク・マクガイアは超人だ(非凡な人)」。	
\\	煙草を止めるための12段階の方法を試したことがあるかい?	
\\	この段落の文章、ちゃんと流れてるかな?	
\\	「ほとんどの技術はやってみることで身に付くものだ」。	
\\	自然に料理を覚えたの。実際、勉強することはなかったわ	
\\	「私たちが滞在したホテルはすばらしい環境の中にあった」。	
\\	「大金は持ち歩きたくないんだ。すぐに使っちゃうから」。	
\\	「雨が降りそうだったので、傘を取りに家に戻った」。	
\\	彼は昇進を知らされ、その上に、まもなく父親になることも知った	
\\	「彼女は食べ物にうるさい」。	
\\	きみ、話し方がはっきりしてないよ	
\\	「自分の気持ちについては話したくない」。	
\\	風邪ひきたくてひいたと思ってるの?	
\\	そんなこと言ってほしくもないね。考えただけでも嫌だよ	
\\	ボーイフレンドと行った旅行がばれないように作り話をしたのかい?	
\\	わが党は、新しい法案を発表するための下準備をしている	
\\	彼は扱いにくいよ	
\\	「その夜、彼は素晴らしい手紙を書いた」	
\\	「分かりにくかったらごめん。僕、ときどき訳の分からないことを言うから」。	
\\	「こつを覚えるまでに時間が掛かったが、今ではやれる自信がある」。	
\\	「中古の車より、お金をもっと払って新車を買いたい」。	
\\	今週ロサンゼルスに着いたばかりです	
\\	彼は子ども達がその店で壊したすべての商品の弁償をした	
\\	「もう一度電話してみたら。彼もう家に帰ってるかも知れないから」。	
\\	「靴を脱がずに、彼は、そのまま家に入って行った」。	
\\	実物を見ないで車を買うな	
\\	強盗たちは銀行を襲う前に何日か下見をした	
\\	「早い者勝ち」。	
\\	「新しい車買ったようだね。どんな感じだい」。	
\\	「感心したよ」。	
\\	「これが終わり次第、知らせるよ」。	
\\	「七五三のお祝いは、三歳と七歳が女の子、五歳が男の子だよね。それじゃ、家の娘、今年は七五三だな」。	
\\	彼は今日の地位に築くまでに大変な苦労をした	
\\	「彼は経済的にとても困っていたので、結婚指輪を質に入れた」。	
\\	不注意が火事の大きな原因の一つである	
\\	「ハンバーガーとは名ばかりで、パンとレタスしだけで、肉が入ってないんだから」。	
\\	「あいついいやつだよ」	
\\	ここでまいる訳にはいかない。皆が僕のことを頼りにしているんだから	
\\	「彼の言ったこと、全然分からなかった」。	
\\	今日は空気が湿っぽいね	
\\	本当にわかってるのかよ	
\\	「あの人、僕が知ってる人だっけ」。	
\\	しくじった	
\\	昨日スフレを作ろうとしたが、失敗した	
\\	「彼は部長にべったりだ」。	
\\	リストラが進んでいるお陰で、多くの人が失業している	
\\	「また別の政治家がへまなことを言ってしまった」。	
\\	このワイン、甘ったるくて嫌になる味だね	
\\	「待ってよ。知ってるんだから。喉まで出かかってるんだ」。	
\\	「すみません。気の利かないことを言ってしまいましたね」。	
\\	いきなりすみません。ここで働いているんですか	
\\	「失礼します。営業の高橋です。今、よろしいでしょうか」。	
\\	「そんなこと僕が知らないと思ったの」。	
\\	「クリスマスカードもバースデーカードも送るのを忘れていた人には連絡しにくいよね」。	
\\	僕はきみを幸せにしたい	
\\	お皿がテーブルの上に置いてあった	
\\	「あなたに満足していただくためにわれわれはここにいるんです」。	
\\	仕事が時間通りに終わるようにします	
\\	ここまで来たら後戻りはできない	
\\	受け取った指示に従ってその仕事をしたまでです	
\\	だれか助けに来てくれたらな	
\\	「必要な時に、お前どこにいたんだよ」。	
\\	今日、ランチ買ってきてくれる?	
\\	ビタミン剤のことはいいから、君に必要なのは運動することだよ	
\\	いつも言わなくてもいいことを言ってします	
\\	学校帰りにブラブラしないでね	
\\	お前のすることと言ったら、ウォークマンを聞くことだけだな	
\\	「自分の家を売ってまでフェラーリ欲しくないよ」。	
\\	いつかハンググライダーに乗ってみたい	
\\	この映画、気に入ると思うよ。見てみて	
\\	きみの消しゴム貸してもらっていいかな?	
\\	会議のために秘書にコーヒーを入れてもらった	
\\	やったね。おめでとう	
\\	チームの勝利は彼によるところが大だったことを知り、彼はしてやったりという気持ちだった	
\\	「何をするのが好きですか」。	
\\	「物が二重に見えるので医者に診てもらわなければ」。	
\\	彼の最初の小説は文学界をあっと言わせた	
\\	「彼らが難しい質問をしたので、彼は困ったしまった」。	
\\	熱を下げるために薬を飲まないといけない	
\\	正面のドアの蝶番に油をさしたので、キーキー言わなくなった	
\\	よい弁護士を見つけないわけにはいかないでしょう	
\\	この会話はしなかったというふりをしよう	
\\	食べながら話すのはあまり好きではない	
\\	ここは品揃えはよくないけど、値段では[他の店に]負けないよね	
\\	お腹が空いて死にそうだ。食べようよ	
\\	「目の前で誰かが死ぬのを見たことがありますか」。	
\\	「落ち着けよ。たかが歯痛だろう。死ぬとか生きるとかいうんじゃないんだから」。	
\\	市販されている痛み止めはたくさんある	
\\	今、すごくいいミュージカルやってるよ	
\\	「これ演義じゃなんだから」。	
\\	「元気かい。随分、しばらくだね」。	
\\	「ずっと正座してたから、脚の感覚がなくなってしまった」。	
\\	「かっこいいスーツだね」	
\\	「娘が好きなマンガを見たいというので、いやいや野球からその番組にチャンネルを変えた」。	
\\	彼のお腹の周りに脂肪がつき過ぎている	
\\	「肝臓に余分な脂肪がつくのはよくありませんね」。	
\\	ジューサーって使ったことある?	
\\	「この仕事への応募者を絞り込む必要があるね」。	
\\	この本、元の場所に戻しておいて下さいね	
\\	「やっちゃったよ。ネクタイをスープの中につけちゃったよ」。	
\\	このシミどうしても落ちないんだ	
\\	「一生懸命働くことの大切さについての父のアドバイスが、家を出て独り立ちしてから、身にしみて分かるようになった」。	
\\	ジョーはかなり金離れがいい	
\\	「自分のやり方は変えられないよ」	
\\	「ここの歯、熱い物とか冷たい物食べるととてもしみるんだ」。	
\\	「この会社の請求書の締めは、月末です」。	
\\	「締め切りがあると仕事がはかどる?」。	
\\	このシャツ着られないね。まだ湿ってるもの	
\\	このカードはいつ請求されるの?	
\\	「あいつは下ネタをたくさん知ってるよ」	
\\	「多くのヨーロッパの国が二桁の失業率を抱え悩んでいる」。	
\\	この肉は私には脂っこすぎる	
\\	クリスマスごとに玩具市場に新たな現象が生まれる	
\\	「どこへ行ってたんだい。最近、姿を見せなかったようだけど」。	
\\	「あいつらには、社会人としても自覚がまるでない」。	
\\	「彼は痛くて体を折り曲げた」。	
\\	「彼女は貧しい人たちのことを気に掛けていると言うが、その言葉は上辺だけのように聞こえる」。	
\\	お世辞を言ってもダメだよ	
\\	「最近は、写真写りがよくないと政治家にはなれないよ」。	
\\	「彼はどのような場合でも、自分が正しく、他のみんなが悪いということを証明しようとする」。	
\\	「その手紙は会社のレターヘッドが印刷されている紙にタイプされていた」。	
\\	「サリーはお前の冗談を少しも面白がっていなかったと思うよ」。	
\\	「人生に対して、広い視野を持つ必要があるね」	
\\	旅をして、彼女はより幅の広い人間になった	
\\	もし新しいビジネスを始める計画をしているのなら、客を引きつける特別なものが必要だね	
\\	今週末は3連休だ	
\\	「たいていは週に一日しか休みがない」。	
\\	この合計は累積ですか。それとも個々のものですか?	
\\	「状況を我々の手で掌握しなければならない」。	
\\	「二ヶ月就職活動をしている」。	
\\	仕事の空きができましたら、すぐに連絡します	
\\	「仕事がほとんどないので、考えただけでひるんでしまう」。	
\\	「この学校では、実力主義制度により次の学年にだれが進むかを決定する」。	
\\	「もしあったら、修正液貸してもらえる?」。	
\\	彼女が僕に興味をなくしたとしても、僕は彼女から離れられないんだ	
\\	「子供がまわりにいると、集中できないよね」。	
\\	これは集中講座ですか?	
\\	「五歳の男の子に、長い間集中させられないよ」。	
\\	「彼は上司の飼い犬だ(上司にためなら何でもする)」。	
\\	「親に、自分の子供に対して客観的な意見をもてと言うのは無理な話だ」。	
\\	「このコピー機で書類を拡大できますか」。	
\\	「コーヒーをついでくれないかな」	
\\	「雨の為、スポンサーはそのイベントを中止した」。	
\\	そのストーリーを書く前に、取材しなくちゃね	
\\	講義の内容をかいつまんで説明してくれませんか?	
\\	事務機器は買うより借りた方が安上がりな場合がある	
\\	「合格おめでとう」。	
\\	ここへは出張で来られたのですか、それとも旅行ですか	
\\	今年、そちらの近所にうちのレストランのチェーン店をオープンします	
\\	試験の範囲はどこまでですか	
\\	「これっていい傾向なのかな」「これっていいことなのかな」	
\\	「彼女は良い主婦になる」。	
\\	「あの映画は僕の趣味ではなかった(気に入らなかった)」。	
\\	「そのカタログでは、製品が品目別に並べられていた。	
\\	彼はどの陸上種目にも優れていた	
\\	辛い経験を通じて若者の人格が形成される	
\\	「飲んでる時は、彼、おしゃべりになるよ」。	
\\	私の見方はちょっと右[左]よりかな	
\\	今トマトが旬ですね	
\\	「腕力は子供を教える時には無用である」。	
\\	辞めないでくれ。我がチームにとってきみは一番の財産なんだから	
\\	「夕食の後に少し歩くのは消化にいい」。	
\\	一般的な医者は、必要な時には患者に専門医を紹介する	
\\	こっちに来て。会ってほしい人がいるんだ	
\\	「会社で出世するには、色々な人を踏み台にすることになる」。	
\\	ざっと本に目を通したので、中身を理解する時間はなかった	
\\	「彼には障害があって、杖無しで歩くのは大変だった」。	
\\	わが地域の教育政策は生涯学習を奨励している	
\\	体に障害があるということがどのようなものなのか考えたことがありますか	
\\	「人生にはどうしようもないことがいろいろあるもんさ」。	
\\	「確かに税金は高すぎるけで、どうしょうもないだろ」。	
\\	選ぶといってもいいのがかったので、悪くない方にした	
\\	いいゴルファーはいかにスイングのすべての力をボールに伝えるかを知っている	
\\	それは恐るべき光であった	
\\	「また別の女の子にプロポーズするとは、あいつよくやるよな(懲りないよな)」。	
\\	「細かいことはいいんじゃない」。	
\\	言ってくれよ、正直に	
\\	「悪い風邪の症状がみんな出てるよ」	
\\	日本とアメリカはいつでも今にも貿易戦争をしそうな状態にあるようだ	
\\	この契約がものにできたらいいと思う	
\\	「今日は衝動買いしないように注意しなさいよ」。	
\\	「彼はこの仕事の為に生まれてきたんだ」。	
\\	昨年は気がかなり悪かった	
\\	僕たちは最後の最後まで戦い抜いた	
\\	この写真の人たちは皆、晴れ着を着ている	
\\	「このチーズ賞味(販売)期限が過ぎてるんだよ。だからこんなに安いんだ」。	
\\	牛乳の賞味期限はどれくらいだって書いてある?	
\\	試験の点が、必ずしも本当に知っていること/能力を表しているとは限らない	
\\	「入るには、写真付きの
\\	が必要です。クレジットカードではだめです」。	
\\	「まさかの時のために少し蓄えておいた方がいいよ」。	
\\	クラスみんなで一斉にその答えを言った	
\\	「この牛乳、酸っぱい臭いがするから、飲まない方がいいよ」。	
\\	「僕はそんなに大食いではない」。	
\\	「食事の時にウーロン茶を飲むのが一番だ」。	
\\	「食事の時にウーロン茶を飲むのが一番だ」。 
\\	「私の料理が気に入らないなら、食べないで寝ればいいのよ」。	
\\	「家に帰ってきたら、必ず手を洗いなさいよ」。	
\\	彼女はルックスで雇われたんだ	
\\	「生活費のほぼ半分を食費に使っている」。	
\\	「我がチームにはジェームズがいるので、大いに期待できる」。	
\\	夕食は何?お腹ペコペコだよ	
\\	「先月、生け花を学び始めたばかりですので、まだ、初心者です」。	
\\	「ビールってお腹膨れるよね」。	
\\	彼が試験に落ちたのは驚くことではなかった	
\\	「今朝、ジムかなりふさぎ込んでたからな。どうしてるか見に行こう」。	
\\	「ここんとこ残業続きだからな。寝不足で、目がしょぼしょぼするんだ。今度の週末は、ゆっくり眠れるといいんだけどね」。	
\\	この古いコンピュータはデータの処理が遅い	
\\	「最後の友達をなくしたような顔をして、どうしたんだよ」。	
\\	このボルトは、クライアントが指定した仕様の物ではない	
\\	今週末、釣りに行こうかな	
\\	「カラスが完全に街を占拠してしまった。どうしたらいいんだ」。	
\\	このクレームって、話がうますぎて、本当とは思えないよ	
\\	何しようとしてるんだ?	
\\	その会社はやると言ったことをやろうとしていない	
\\	ここで何をしようとしてるんだ	
\\	「銀行は、ますます、
\\	の使用料金を取るようになっている」。	
\\	社会の高齢化は深刻な経済問題を引き起こす	
\\	「白けた顔してるじゃない」。	
\\	「それは真っ赤な嘘だった」。	
\\	好きなだけ飲んでらっしゃい。でも、朝になって頭が痛いって泣きついてきても知らないからね	
\\	「練習を続ければ、知らぬ間にギターが弾けるようになってるよ」。	
\\	「酔ってるからこそ言えるんだよ」	
\\	君にこの件を調べて欲しいのだが	
\\	無視するだけでは問題は片づかない	
\\	彼女はかまととだ	
\\	「以前は素晴らしかったこのアイディアも、段々とどうでもいいものになった」	
\\	彼の奥さんはかなり寛大だ	
\\	彼らは私一人に責任を負わせた	
\\	「支離滅裂なことをぶつぶつ言う奴に面と向かわない方がいいよ」。	
\\	「ホワイトハウスにはビジョンを持った人物が必要だ」。	
\\	「そのグループ(実力者)の中に留まるには、かなり一生懸命にやらないとだめだ」。	
\\	「週末に少しはテレビを見ることもあるけど、そんなには見ないよ」。	
\\	両親は私の仕事の選択を非難しているように見えた	
\\	「何だよこれ(この代物は)」。	
\\	「こんなに散らかして。ジョーがやったんだな」。	
\\	このコート、しみだらけだ	
\\	「君のミスに対する非難を僕は受ける積もりはない」。	
\\	「まだ叔父の喪中なんだ」。	
\\	「この鉛筆の芯、柔らかすぎる」。	
\\	レタスのしんは最も甘いところである	
\\	彼女は真意を隠した	
\\	「応援を呼ぼうよ」。	
\\	大変なことに挑戦すると、自分の力がわかる	
\\	「引っ越しもして、仕事も代えて、彼は新たなスタートを切った」。	
\\	「彼女は、アドレナリンギンギンのスカイダイビングのようなスポーツが好きだ」。	
\\	「2桁の
\\	で6桁の収入を得ている政治家が嘘の約束をするのには嫌気がさすよ」。	
\\	君がそれをした時に、何を考えていたのか想像がつかないよ	
\\	きみはなんでいつもそんなに神経過敏なの?	
\\	「あいつは細かいからね」。	
\\	「大事な話があるって聞いたけど、何かな」。	
\\	深呼吸をして落ち着いて	
\\	「彼ははまだホヤホヤのカップルである」。	
\\	自分の信用調査書を見てみたい	
\\	彼女の弁護士は、精神異常ということで弁護しようとしている	
\\	きみが約束を守ると信じてるよ	
\\	歴史を通じて思いもよらぬことが何度となく起きてきた	
\\	新たな証拠が出てきたため、警察は昔の未解決事件の捜査を再開した	
\\	本当のことがわかるようになると思うかい?	
\\	爆竹が破裂したのを聞いたとき、心臓が止まりそうになった	
\\	「おー、そこにいたのか。心臓麻痺に起こすかと思ったよ」。	
\\	大統領の辞任はまさに時間の問題である	
\\	眠らなくちゃ。死ぬほど疲れてるんだ	
\\	市長が起工式でリボンをカットした	
\\	「結婚相手を決めるのは、そう簡単にできることではない」。	
\\	そのドレス似合うよ	
\\	僕は平均的な体型かな。少し、痩せてるかな[筋張っている]	
\\	「彼女は華奢で小柄である」。	
\\	「パリで新しいスーツを新調したんだ」。	
\\	「我が社にはもっと能力主義を取り入れる必要がある」。	
\\	「彼は神童と見なされていたが、人生の早いうちに、才能が伸びなくなってしまった」。	
\\	「若い頃は、心配事など一つもなかった」。	
\\	きみの将来が心配なんだ	
\\	「母は、生まれつきの心配性だ」。	
\\	「我慢、我慢」	
\\	文明が何の問題もなく、平和に進歩していくことはめったにない	
\\	文明が技術と一緒に進歩していると思うかい?	
\\	「ジムは息子としんみりと話をした」。	
\\	わが社は消費者の信用を取り戻すために頑張っている	
\\	「この数年で、この会社は信用を落とした」。	
\\	「彼は人が自分のことをどう思っているのかを気にしすぎる」。	
\\	「彼は教師なのだが、振る舞いが教師らしくない」。	
\\	「君はここでの立場を理解していないようだね」。	
\\	「買い物の時間をどれくらいあるのかな」。	
\\	時が経てばすべての傷は癒える	
\\	「時の経過と共に、その単純な物語が、偉大な伝説となった」。	
\\	「4時から5時までの時間帯、会議室を予約しておいた」。	
\\	野球はアメリカの娯楽である	
\\	「スケジュール的には、今日はかなりきついんだ」	
\\	「君が僕を買収できると思っているのなら、それは時間の無駄だよ」。	
\\	文句を言うの止めてくれないか。文句を聞いてる時間なんかないんだよ	
\\	「少し時間をいただけますか」。	
\\	観光シーズンみたいだね	
\\	来週僕のためにスケジュールを空けてくれるかな?	
\\	「結婚について話すのはまだ早いよね」。	
\\	彼は自立することを覚えなくてはいけない	
\\	彼は在職中、スキャンダルで苦しめられた	
\\	「この会話は僕にはレベルが高すぎる」。	
\\	彼は信じられないくらい大根役者だ	
\\	「ニューヨークに住んでいる人たちは、自己主張が強いことで知られている」。	
\\	いい指導者になるには多少自己中心的なところが必要になる時もある	
\\	問題は引き起こしたくないんだ	
\\	「自分で招いたことだから、そうなったんだよ」	
\\	「ハワイから戻ってきたばかりで、時差ボケがひどくて、だから、少し寝ようと思って」。	
\\	そのメーカーは出荷したいくつかの玩具を回収させられた	
\\	「自分で考えることを教わる必要のない子供もいる」。	
\\	独学で偉大な学者になった人たちもいた	
\\	「楽勝だよ」。 
\\	「僕に任せてくれよ」。	
\\	自分を過大評価してはいけない	
\\	「企画が失敗し、彼の自信が大きく揺らいだ(大打撃だった)」。	
\\	自分にどれくらい自信を持ってるのかな?	
\\	「とても奇妙な経験をしたが、それを書いたとしても誰も信じないだろう」。	
\\	自炊してるの?	
\\	これは不規則動詞だよ	
\\	時代遅れって言えばいいさ。でも、お見合い結婚だっていいと思ってるんだ	
\\	この喫茶店に来ると学生時代のことを思い出す	
\\	歳月人を待たず	
\\	「ビルはカンカンに怒っていた」	
\\	「「彼が通りでお金をせがんでいるのを見て自分の目を疑った」。	
\\	自分が目にした現実を理解するまでに数分かかった	
\\	「紙の上ではその計画はよく思えたのだが、実行したときに、多くの問題が出てきた」。	
\\	「君の写真とっても可愛かったけど、実物の方がずっといい」。	
\\	「彼女は、私のラザーニャは興味深いと言ったが、実際は、「ひどい」という意味だった」。	
\\	昨日見たのは僕だったというのは確かなのかい?	
\\	飲酒運転を取り締まる、新しい法律が施行されている	
\\	休みはアパートでぼんやりと過ごすことにしている	
\\	実は、私はこの分野の専門家ではないんです	
\\	このニュースキャスターは画面で見るより実物の方がいいよね	
\\	「彼はやり手のビジネスマンだよ」。	
\\	ラグビーをやろうとしたが、向いてなかったんだ	
\\	「誰でもテレビでマラソンを見て、専門家のように話すことは出来るが、実際に走るのはとんでもなく大変なことだ」。	
\\	「君の力を見せてもらおうか」。	
\\	それじゃ借金をして借金を返すようなものだ	
\\	テロリストたちは小さいころから洗脳されている	
\\	彼女は自費出版をしてくれる会社にお金を払い、自分の犬の写真集を出した	
\\	「自分でガーリックは嫌いだって言ったんだろ」。	
\\	そんなに大声で話さないでくれ。こっちが恥ずかしいよ	
\\	「はっきり自分の考えを言えよ	
\\	「終わりにして、家に帰ろう」。	
\\	「僕が言うのは何ですが、これは素晴らしい絵ですね」。	
\\	これ全部自分で料理したのかい?きみには驚いてばかりいるよ	
\\	「日々の積み重ねが物を言う」。	
\\	そんなに自分に厳しくすることないよ。ちゃんとやってるんだから	
\\	「自分自身のことを愛していなければ、人を愛することはできない」。	
\\	「彼は人生においてどれほど恵まれているか分かっていないと思う」。	
\\	「彼女と僕は住む世界が違うんだよ」。	
\\	「君の人生だ。好きなことをすればいい」。	
\\	この仕事では自分らしさが出るとは思えない	
\\	「自分自身を責める理由などどこにもない」。	
\\	「うぬぼれるのは止めろよ」。	
\\	父は、派手なネクタイが好きだ	
\\	継続は力なり	
\\	「彼は飾り気のない服を好む」	
\\	どうやったらこの形式的手続きさっさと片づけられるんだ	
\\	この部屋はとてもじめじめしているので、壁にカビが生える	
\\	じゃあ、泳ぎに行かない?	
\\	「この計画には一つも欠点がない。	
\\	「ボブに近づくなよ。あいつはこの部署の危険人物だから」。	
\\	たまには少し息抜きをしたほうがいいと思わない?	
\\	「結婚したくないというのじゃなくて、その時期が来たらしたいと思っているだけなんだ」。	
\\	「新しいアパートができて、入り江が見えなくなった」。	
\\	夜行列車でシカゴを発って、明け方にここに着いた	
\\	これが正式な住所変更届けの用紙です	
\\	彼は充実した人生を送った	
\\	「全てのレストランが車椅子でも自由に入れるようにすべきだ」。	
\\	「遅れてごめん。渋滞に巻き込まれちゃって」。	
\\	「信頼できる客を獲得すること、すぐには利益には繋がらないが、それが我々にとって最も大切なこと(関心事)である」。	
\\	「この携帯の充電時間はどれくらいですか」。	
\\	充電池を使ってるから、捨てる必要はないんだ	
\\	「こんなことにはなるとは思わなかったんだ。それに対処するための方法が何かないかな」。	
\\	毎朝柔軟体操をしてるよ	
\\	柔軟な考え方ができずに多くの指導者たちが失脚した	
\\	「カラオケでは、いつも山口百恵のコスモスを歌うの」	
\\	「十八才禁」	
\\	「彼、細かいことばかりいって、全体像が見えてないんだよ」	
\\	「もういいよ。屁理屈を言うの止めてくれ」。	
\\	「その代表者は会社が抱える法律上の問題を軽く見ていた」。	
\\	「講義の最中にうたた寝をしてしまった」。	
\\	受験勉強は僕にとってはとてもいやな経験だった	
\\	テロの後、飛行機を使った旅行の需要が急激に落ちた	
\\	彼は、何人かの役人に賄賂を使って今の地位についた	
\\	「うまくプレゼンをやりたいのなら、自分の考えを順序立てて整理できるようにしなくてはね」。	
\\	このサードパーティーのハードは安いけど、サービスが良くないことがあるよね	
\\	「仕事うまく行っていることと思います」	
\\	「アガシとサンプラスが勝ったのは、順当でしょう」	
\\	「彼はいつも負けたことがなかったが、今回は、負けてしまった」。	
\\	これをアルファベットの逆の順番にファイルしてくれるかな?	
\\	ほかの文化に適応するには、柔軟性がないとね	
\\	「1等、2等それぞれに賞金1000ドルと500ドルが渡されます」。	
\\	「そのファイルをアルファベット順に並べて」。	
\\	引っ越しの準備は全部できたかい?	
\\	「僕も持ってればよかったんだけど」。	
\\	「ジョギングする前のウォームアップにどんなことしてるの」。	
\\	今年は2位だったが、来年は優勝するつもりだ	
\\	「彼は思い切ったことをするのが好きで、毎日、その日が最後のような積もりで生きている」。	
\\	「話の途中でも自由に質問してください」。	
\\	どうぞ好きなものを注文してください	
\\	「その映画の上映時間は3時間だったが、なかなか面白かった」。	
\\	映画は2時間で終わってほしいよね	
\\	「そうさ、認めるよ。僕は君のことが好きだよ」。	
\\	「私の彼に対する気持ちは友情を越えている」。	
\\	ドイツ語の方はどう?	
\\	「あいつの行動は明らかに異常だ」。	
\\	彼は、提案に多くの条件を付けた	
\\	「船が、上下左右それに前後にも揺れたので、船酔いしてしまった」	
\\	「タクシーの運転手が乗車拒否をするのは違法だ」	
\\	「刑務所に入ると、けちな泥棒も常習犯になる」。	
\\	その飛行機は上昇していった	
\\	現状について真剣に話さなくては	
\\	制服を着ることは一体感/帰属意識を生み出すのに役立つ	
\\	やっとつながってうれしいよ。これでいつでもメールのやり取りができる	
\\	冗談だろ	
\\	「これ高級なシチューだね」。	
\\	「彼女は専門家に爪の手入れをしてもらっている」。	
\\	バックミラーとサイドミラーの調節をした?	
\\	「値段が安い割には、食べ物まずくはないだろう」。	
\\	「私の好きなコメディアンはときどき女装をする」。	
\\	親元を離れて、自活してみたい	
\\	「太陽がギラギラ照りつけている」。	
\\	私は両親の元を離れて、18歳で独立独歩の道を歩み始めた	
\\	時には人を頼ることも必要である	
\\	飛行機が遅れてばかりいるからイライラするよ	
\\	銃を所持することは基本的な人間の権利ではない	
\\	高速道路の事故でけが人は出たが、死んだ人はいなかった	
\\	転勤を申し出たが、通らなかった	
\\	彼の人生は人を助けるためのものだった	
\\	ジェリーに会う度に、彼は人生の岐路に立っていると言う	
\\	「静かで平凡な人生でも構わない」。	
\\	この経験から何か学んだかい?	
\\	「イチゴを食べると蕁麻疹が出るんだ」	
\\	かつては二階はもっと水圧が高かったのに	
\\	「何回かチェスをしたことがあるけど、はまることはなかったね」。	
\\	「つま先で水がどれだけ冷たいか調べてみた」。	
\\	「君の目には人を引きつけるような魅力がある」。	
\\	「この手のセールスマンに耳を貸してはいけない。勧誘しようとしてるだけなんだから」。	
\\	「テニス奨学金で大学に入った」。	
\\	その場所はお客でいっぱいだった	
\\	喉がカラカラだよ	
\\	その会社はもっと水平統合される必要がある	
\\	「会議中眠くて仕方がなかった(睡魔と戦っていた)」。	
\\	「この法律は、密室で(タバコの煙が満ちた奥の部屋)書かれ、決まってしまった。一般大衆は何も言えなかった(発言権がなかった)」。	
\\	「これは素晴らしい推理小説だ」。	
\\	「一度に払う余裕がなかったので、新しいテレビを分割で払うことにした」。	
\\	ほんの数時間でバルセロナですよ	
\\	「近所にとんでもない犬がいたので、彼女は犬を怖がるようになった(大人になっても)」。	
\\	結婚は、単に末永く幸せに暮らす以上のことを意味する	
\\	昼寝をしてすかっとしたよ	
\\	「食べ物には特にうるさくないよ」。	
\\	「これは君なら絶対に見たいと思う映画だよね」。	
\\	お腹いっぱいにして泳いじゃだめだよ	
\\	「特にチョコレートが好きって訳じゃないから」。	
\\	焼き魚が一番嫌いなんだ	
\\	恋人はいますか?	
\\	ミッシェルはケンのこと好きなんだって	
\\	「かつてはオフィスで煙草の煙りを我慢しなくてはならなかったけど、もうそういうことはない」。	
\\	背水の陣をしくのは無意味だ。	
\\	「彼の言うことに耳を傾けることはない。ただの脅しにすぎないんだから」。	
\\	「ありがとう。助かったよ」。	
\\	彼女とは話したいとは思わないけど、取り柄と言えば、音楽の才能があることね	
\\	「それ一つ多いね」。	
\\	この決定は教育委員会の政策に大きな影響を与えるだろう	
\\	少なくとも謝れよ	
\\	このスパゲティーソース少し辛いね	
\\	これはスケールの大きい建設プロジェクトである	
\\	「彼は女たらしそのものだ」。	
\\	この仕事を少しずつやりたいですか、それとも一気に片づけますか?	
\\	優れた骨董品鑑定家は、細かいことに気づく。それがどんなに些細なことでも	
\\	「君は、自分の事以外は考えないんだね」。	
\\	この文で動詞を忘れたのは致命的な間違いだった	
\\	息子には健康で元気な青年になってほしいです	
\\	最近の肥り方ったらないね	
\\	あれってすごくない?	
\\	「彼女は一流だよ。雇ってよかった」。	
\\	週末はどうやって過ごしたんだい?	
\\	「夏はどうやって過ごすの」。	
\\	彼の主張に論理を求めちゃだめだよ	
\\	「聞かずにそれをしていいと誰が言ったんだ」。	
\\	「もっとよく考えて議論をすべきだったね」。	
\\	「今、(洗濯機は)すすぎになってるわよ」。	
\\	「君が僕だったら、どんな車を買う?」。	
\\	「この交渉を先に進めましょう」。	
\\	彼らのその報告書は全然進んでいない	
\\	「私の勤めている会社では、社員を喜ばせるために、時々給料を上げる」。	
\\	このドレスの裾がそろっていない	
\\	彼は筋骨たくましいね	
\\	この問題に対するきみのスタンスは?	
\\	「オーケー、そのままの状態でいて」。	
\\	うちの子どもたちも大学生の年齢だからね。じき巣立って行くんだろうな	
\\	「これで、やっとすっきりした」	
\\	人のことを怒鳴りつけるととてもすっきりした気分になることがある	
\\	「行きつ戻りつした後で、すべてが丸く収まるようになるのさ」。	
\\	「待っても待っても、彼、現れなかったの」	
\\	その組織は彼のことを消耗品だと見なしていた	
\\	「今の仕事にも捨て難い点もあるんだけど、辞めて、自分の会社を始めることに決めたから」。	
\\	この古いナイフはまだ十分に使えるから、捨てないでね	
\\	「このピザ、思ったよりおいしいじゃない」。	
\\	「いいこともあれば、悪いこともあるからね」。	
\\	地下鉄の職員は毎年ほぼ同じ時期にストをする	
\\	「ストレスが原因で、一時的に耳が聞こえなくなったんです」。	
\\	「人の話を聞くときには、広い心を持つことが大切だ」。	
\\	率直に彼に言ってやれよ	
\\	このパン、ボール紙みたいに味がないね	
\\	「少し待ってて。ちょっとスピード写真撮ってくるから」。	
\\	どうしたんだよ。びしょ濡れじゃないか	
\\	「スノーボードやるの」。	
\\	皆、ある時点で翼を広げ羽ばたいて行かなくてはならない	
\\	「お前、黙ってようと思ったんだろう」。	
\\	「隅々まで探したが、コンタクトレンズは見つからなかった」。	
\\	「ここにどれくらい住んでるのかな」	
\\	あいにく、その問題を消し去る方法はない	
\\	彼は背が高くヒョロッとしている	
\\	「彼は、すりきり一杯ではなく、スプーン山盛り一杯分を入れた」。	
\\	「お前のズボン破れてるよ」。	
\\	「靴でかかとがすりむけて、血が出ちゃったんだ」。	
\\	きみがそう言うからには、正しいに違いない	
\\	サッカーをやりたいとは思わない	
\\	「彼女に知っていることはすべて話すことにした」。	
\\	「みんな中華かメキシコ料理が食べたかったが、結局、開いている店は、ホッとドック屋しかなかった」。	
\\	「今夜は勉強することになってるんだけど、多分ビデオを観ちゃうだろうな」。	
\\	「結局、この会議のまとめ役になった」。	
\\	きみはお金を決して取り戻せないよ	
\\	「彼はよく辛辣なことを言う」。	
\\	子どもが嘘をついたり、盗みをしたときには、罰せられるべきです	
\\	こんなことが二度と起きないようにします	
\\	毎日、デンタルフロスで歯をきれいにするようにしたほうがいい	
\\	「娘は、補助輪なしで自転車に乗れるようになっている」。	
\\	彼はチームで自分が一番の選手ではないことがわかった	
\\	最近、釣りしてるのかい?	
\\	彼と僕は波長が合わないんだ	
\\	この2つの部署は意見が食い違っている	
\\	もっとバランスの取れた生活を送りたい	
\\	「これお皿で分けた方がいいですか」。	
\\	「彼等はその施設の前で座り込みの抗議を行った」。	
\\	「彼女が目標にされていて、もう少しで誘拐されそうだったって本当?」	
\\	電話が鳴ったとき、ちょうど彼女に電話しようとしていたところだった	
\\	「彼はもう少しで昇進するところまできている」。	
\\	そんなに厚かましくするな	
\\	「最終電車に乗り遅れるんじゃないかとずっと心配しながら、彼が話し続けるのを聞いていた」。	
\\	君の推測通りだよ	
\\	「だらしない時もあるけどね」。	
\\	「それって冷たかったよね」。	
\\	ずるいよ	
\\	何であの2人、あんなふうにくっついてるんだ?	
\\	「今日、彼は授業をさぼった」。	
\\	またカーテンがずれてる	
\\	約束の時間を30分ずらしていただけますか	
\\	手伝えたらいいんだけど、自分の仕事で精一杯なんだ。	
\\	新しい髪型のせいかもしれないけど、今日は、痩せて見えるよね。	
\\	冷たいビール持ってきたって?ちょうど欲しいと思ってたんだよ。	
\\	緑と紫を着ちゃだめだよ。合わないから	
\\	私はアメリカで数年暮らしたが、そこで身を立てられなかった	
\\	生活するために、残業とチップが必要なの	
\\	今年は税率が上がってしまった	
\\	請求書在中	
\\	自分のことを正直な人間だと思いますか?	
\\	「この高速道路はスピード制限がないんだ。好きなだけスピードが出せるよ」。	
\\	地面の上で正座するのは大変だ	
\\	昇進を見送られてしまったので、彼はやる気をなくしてしまった	
\\	大学に7年も行ったけど、ちゃんと(正式には)卒業しなかった。	
\\	彼は少し精神的に不安定なようだね	
\\	僕、ひどい方向音痴なんだ	
\\	大衆の信頼を得るには長い時間がかかる	
\\	森の静けさだけが私のストレスを取り除いてくれる	
\\	あんなやつと別れてほっとしたわ	
\\	人は基本的には善であると信じている	
\\	故郷に帰った時、丁重な扱いを受けた	
\\	大統領は任期中に成長をして大統領としての仕事ができるようになるものだ	
\\	子どもを苦痛や苦難から守りすぎてはいけない。それで人格が形成されるのだから	
\\	「その話し合いがうまく行かなかったのにはいろいろな理由がある。実際、誰かのせいという訳ではない」。	
\\	「コンピュータウイルスのせいで、一週間分の仕事がパーだ」。	
\\	「テレビをつけたら、静電気がすごかった」。	
\\	会社は彼を首にした	
\\	「お客さんが来る前に、部屋を片づけなさい」。	
\\	「この問題をはっきりさせておきたいんだ」。	
\\	""には『自分の部屋、机、ロッカー、家などを、きれいにしておく、それぞれの場所に片づける』という意味があります。
\\	彼には仕事は向かないようだね	
\\	たいていは初春に冬服のセールが行われる	
\\	「彼女は、今年の初めに大変な負債を背負った」。	
\\	「彼は日本政府から人間国宝に認定された」。	
\\	「ジュディは彼女自身の小さな世界に済んでいるように見える」。	
\\	「水泳の自由形で、彼女は世界タイ記録を出した」。	
\\	彼の左フックは世界最強だ	
\\	「彼女は彼にしつこく言って新しい靴を買ってもらった」。	
\\	週末は家具の配置を変えるのに時間を費やした	
\\	走って行って席を取っておいて。僕はボップコーンとお菓子を買ってくるから	
\\	この事故はだれのせいだったんだ	
\\	「それは僕が悪いわけではないが、それでもその事故に対して責任を感じている」。	
\\	このプロジェクトはまるでまとまりがない。はっきりとした責任者がいないからだ	
\\	彼は何事においても決して責任を持とうとはしない	
\\	今週末は一緒に出かけられないな。果たさなくてはいけない責任があるんだ	
\\	「これは自分でやりなさい」	
\\	自分のしたことに対する責任は取れよな	
\\	「午後はずっと外出しています」。	
\\	座れるように横にずれてもらった	
\\	「みんなに同情してもらおうなんて無理だよ」。	
\\	「あの家族は、世間体ばかりを	
\\	僕たちは座って何時間もしゃべった	
\\	そんなふうに背中を丸めるのはやめなさい	
\\	「規則や規制が無ければ、弱肉強食の世界になってしまう」。	
\\	「(せっかく)パリで一ヶ月過ごすんだから、少しフランス語を勉強したらどう」。	
\\	「君がせっかく頑張ってくれたのに、時間通りに終わらなかったんだ」。	
\\	忙しいのにすみません	
\\	なんだよ。財布忘れちゃったよ。これじゃあ、ここに来たの時間の無駄だよ	
\\	「折角来ていただいたのですから、お茶とお菓子でもどうぞ」。	
\\	病気やけがをすると、いかに命が大切かがわかることがある	
\\	「なんでこのせっかくのチャンスを無駄にしなければならなかったんだ」。	
\\	「仕事中に、毎日、直接お客様と接している」。	
\\	可動部分の接触がたいていは悪くなるんだ	
\\	どちらも四つに組む接戦だった 
\\	接戦だったが、何とか勝った	
\\	顧客を持てなすことは、多くの国で当たり前のように行われている	
\\	彼女は彼を説得させてそれを止めさせた	
\\	「仮に会議の時間を午後3時にしておいた」。	
\\	「ハンバーガーにオニオンリングを付けてください」。	
\\	「背水の陣だな。いい加減にやってたらだめだ」。	
\\	「あれはまさにお涙頂戴の映画だよ」。	
\\	「あの映画の主役には共感できないね」	
\\	彼は彼女にまた会うことを切に願っていた	
\\	「君は、安物買いの銭失いだね」。	
\\	ちゃんと予定を立てておかないと、大変なことになるよ	
\\	連戦連勝したのでチームの士気は高まった	
\\	「この町の通りは狭すぎる」。	
\\	時間を無駄にはできない。切羽詰まってるんだ	
\\	我々は競争相手よりも先んじているが、彼らも追いついてきている	
\\	コオロギの鳴き方で気温がわかるんだよ	
\\	「少なくとも簡単な代数くらいは勉強しておいた方がいい」。	
\\	その殺人者の謝罪は犠牲者の家族にとってもなんの慰めにもならなかった	
\\	「同じ事を言おうと思ってたんだよ」。	
\\	ありがとう。君がいなかったらできなかったよ	
\\	「いつもはうつ伏せで寝るの、それとも、仰向け?」。	
\\	小さいころ、輪郭で飛行機を見分けることを覚えた	
\\	1人のリーダーによる専制政治の時代は終わった	
\\	「いつでも物事を長い視野で見るようにするのがいい」	
\\	その映画の良いところは、すべて予告編の中にあった	
\\	「このような仕事の面接の機会がもらえるなんてめったにないことだ」。	
\\	「トレバー先生、言いたいことがあるんですが」。	
\\	「虫を踏みつぶすのはやめろよ。それはよくない(行いだ)よ」。	
\\	「洗濯で忙しくて電話のベルが聞こえなかった」。	
\\	「これが仕事に遅れた時の言い訳ベスト10のリストだよ」。	
\\	「受け付けた順に注文に応じます」。	
\\	「有利に立場になろうと、お互いに機先を制しようとしていた」。	
\\	「どの候補者も有権者の心をとらえられなかった」。	
\\	きみのせいで僕はクラッシック音楽に対して偏見を持つようになっちゃったんだよ	
\\	「警察官に扮するのは犯罪である」。	
\\	次に出る電車はどれですか?	
\\	これ帽子の専門店で買ったのよ	
\\	提携とは、解雇につながる合併を意味する経営者側のことばである	
\\	「彼、先約が入ってたからここにはいないよ」。	
\\	「今は大変だけど、後で良かったって思うよ」。	
\\	カードを買えば、君のラップトップ、インターネットに繋げるよ」。	
\\	「軍事力は重要だが、控えめに使うべきである」。	
\\	慈善のための寄付はたいていの場合は非課税である	
\\	たまには自分の時間を持つようにしろよ	
\\	「文句を言うなよ。君のような立場になれるならどんなことでもするって人は大勢いるんだから」。	
\\	会社が倒産するまで、彼はぜいたくな暮らしをしていた	
\\	この前見たときよりお腹のあたりに少し肉がついたね	
\\	「ナンシーは先月ジムと別れた」。	
\\	「彼女、調子が出てきたね」。	
\\	「そう言うだろうという気がしてたよ」。	
\\	「火曜日のパーティーには絶対来てね」。	
\\	僕の歌うの聞きたくないでしょ。とんでもない音痴だから	
\\	彼は内気だが、ものすごい影響力を持っている	
\\	「彼女の収入は6桁である」。	
\\	「思っていたよりも多くの観客が来ていた」。	
\\	「調査官は、その危険なビルは取り壊されることになるだろう、と言った」。	
\\	「ラグビーの試合の後、全身打ち身だらけだった」。	
\\	「彼らは善戦したので、負けたが、堂々としていた」。	
\\	「喘息のような呼吸器の問題は、成長するとともに治ることがよくある」。	
\\	「新しいテクノロジーにより、爆弾をピンポイントで落とすことが可能になっっている」。	
\\	その本の全体的な印象はよかった	
\\	「そういう意味ではなかったんだ。もう一度言わせてよ」。	
\\	「分かったよ。そういうことにしておくよ」。	
\\	「すみません。あなたの気持ちを傷つけるつもりはなかったんです」。	
\\	「10年前にちょっとした嘘をついたんだけど、女房は今でも覚えてるんだ(忘れさせてくれない)」。	
\\	「好きだとか嫌いだとかじゃなく、そういうのもなんだよ」	
\\	「コンピュータには問題はつきものさ」。	
\\	日本のマンガを読むまでは、コミックは子ども向けだと思っていた	
\\	「自分で会社を持てば税金を節約できる。そういうわけで、最近は、多くの人が会社を作るようになっている」。	
\\	「何で君がそう言うだろうって分かっちゃうんだろ」。	
\\	「足と膝が痛くて、腰もね(言うまでもないけど)。少し寝た方がいいみたい」。	
\\	実際に何があったのか言ってくれ。包み隠さず言ってくれよ	
\\	「手頃なアパートを探しているんだ」。	
\\	人生では多くの困難に遭遇する	
\\	「このプロジェクトがうまく行くか失敗するかはすべて君にかかっているんだ」。	
\\	「彼は、ある専門の分野で欠けているものを総合力で補っている」。	
\\	犯罪者はいつも現場に戻るものだ	
\\	「今日、部屋に掃除機かけた」。	
\\	家に帰ったらすぐに掃除を始めなくては	
\\	この要望にはタイミングよく返事をすることがとても大切なんだ	
\\	「想像するのではなく実物をスケッチした方がいいよ」。	
\\	「今夜は仕事から早めに抜け出した」。	
\\	「今、抜け出すなよ。パーティーが始まるところなんだから」。	
\\	「いろいろ問題があるときに、彼は親身になって聞いてくれるんだ」。	
\\	「いいワインリストがなかったら、ここでは食事しないよ」。	
\\	この骨董品を鑑定する番組は、
\\	の番組と似てるよね	
\\	僕はそういう言い方はしないけど、きみの言いたいことはわかるよ	
\\	僕はそこまでは言わないよ	
\\	そうなんだよ	
\\	「ホテルの部屋で一泊するのに200ドルは高いと思うかも知れないが、それが相場だよ」。	
\\	とにかく冷たいビールを一口飲みたい	
\\	総論では同意しているが、細かい点は詰めなくてはいけない	
\\	お互いに心が離れてしまったことがわかり、私たちは離婚した	
\\	あの若者はいつか有名になるよ。僕にはわかるんだ	
\\	「頭の回転の速い人(即断・即決ができる人をさがしてるんだ」。	
\\	昨日、スピード違反で捕まっちゃったよ	
\\	現実に基づいた議論をすればもっと説得力があるのだが	
\\	「彼は授業ではまあ一生懸命にやってはいるが、もっと頑張れるよね」。	
\\	「ここでしか本物のキャビアは手に入らないよ」。	
\\	この軽食堂では、コーヒーはお代わり自由である	
\\	「それってどういうことなのかな?」。	
\\	彼女は、命を捧げてもいいほど彼のことを愛している	
\\	「どうしてそんなことが言えるの?」2.「それは挑戦的な言葉だね」。	
\\	「今週中に原稿を仕上げていただけないでしょうか。無理なお願いとは分かっているのですが。そこを何とかお願いできませんか」。	
\\	「お願いです、この動物を逃がしてください」。	
\\	「みんなこのソース好きみたいだけど、私はだめだわ」。	
\\	彼にプロのバスケット選手になる素質があるかな?	
\\	いつも政治家は悪いことしても罰せられないんだな	
\\	「彼女の歩き方ってセクシーだよね」。	
\\	話し方でしつけがわかる	
\\	こっちに何時に電話した?	
\\	私は父にそっくりだ	
\\	「こんな雑用で残業したくないよ」。	
\\	「政治のことになると、君は、論理とか理性とかおかまいなしになるようだね」。	
\\	「会社の方針に沿って新聞発表を書いた」。	
\\	少しほおっておいてくれないか。文句を聞く気分じゃないんだ	
\\	「彼は物事を問題なくこなす」。	
\\	「ちゃんと卒業できるんだろうね」。	
\\	「その仕事をなんなくやってのけるのは難しい」	
\\	玄関で靴を脱ぐのを忘れた	
\\	「うちの主人の笑顔ですてきですって。外面がいいだけよ。うちじゃ何もしなくて、ただ、ゴロゴロしてるだけなんだから」。	
\\	「そのうちパーティーでもやりましょう」。	
\\	そのうち風邪がよくなった	
\\	パンクのせいで到着が遅れた	
\\	週末や祝日以外は、仕事場では制服を着ることになっている	
\\	「誕生日プレゼントの代わりに、彼は彼女に花束を買った」。	
\\	「電車の運賃を誤魔化してたことがあるんだけど、3度目に捕まって運賃の3倍の額を支払ったんだ。結局、得したことは何もなかったよ」。	
\\	「君からということなら、みんなそのアドバイスを聞くと思うよ」。	
\\	「それが聞きたかったのです」。	
\\	「その点では、彼のプランは他のよりもいい」。	
\\	「明日、そっちに行きますので、その時に会えばいいでしょう」。	
\\	「ほら時間よ。起きなさい」。	
\\	「この会社にはっきりとした答えを求めても、決まってその場しのぎの答えしか返ってこない」。	
\\	「このスーツの直しは2,3時間でできますので、今日の午後には仕上がります」。	
\\	「子供のころは、家族がその日暮らしをしていたので、みじめだった」。	
\\	スーザンはパーティーに来るけど、デイブはちょっと怪しいね	
\\	きみは2000円払ったけど、500円分しか食べなかったね	
\\	「数字はあまり得意ではない 
\\	「料理の方はどうなのかな?」。	
\\	「中古品は現状のままで売られているので、買う前によく見なさいよ」。	
\\	遊び終わったら、おもちゃをそのままにしておいたらだめよ	
\\	すべての商品は現状渡しです	
\\	「ちっちゃなサラダなんていいから、何かお腹にたまるものを出してよ、ステーキとか」。	
\\	「彼は人の考えをただ機械的にり返しているだけだ」。	
\\	「血色がよく(頬がピンク)、瞳は輝いているし、おばあちゃん、本当に若いよね(若さそのもの)」。	
\\	「馬鹿にするのもいい加減にしろよ。ボートの事、何も知らないくせに」。	
\\	この問題を軽く扱いたくない	
\\	「風が木の枝の間を吹き抜けて行った」。	
\\	「野球のファンというわけではありませんから」。	
\\	「子供を作る前に、親になるとはどういうことか知っておかなくてはいけないね」。	
\\	「彼はシーズンチケットを買うくらい野球に熱心だった」。	
\\	「看護婦なんだから、心肺機能蘇生くらいはできるでしょう」。	
\\	牛とか馬とか豚とか、そういうのを見たんだ	
\\	それぞれがお弁当を持って来た	
\\	「彼は4年連続でこのレースに勝ってきた。それだけ彼が速いってことさ」。	
\\	ここは初めてだけど、それでもガイドは必要ないと思う	
\\	ひどい別れ方をしたけど、僕たちはそれでも友だちなんだ	
\\	「最近、胃の調子が悪くて」。	
\\	「熱があるのに会社に行くですって。そんなの無理よ」。	
\\	きみはパーティーにケーキを持っていくの?それなら僕はワインにするよ	
\\	「彼女は彼女なりにとてもきれいだ」。	
\\	「彼女は彼女なりにとっても綺麗だよね」。	
\\	そうだとしても、だれかが朝までにごみを出さなくてはね	
\\	それは今は置いておきましょう	
\\	それはそうと、昨日はどこにいたんだい?	
\\	「それは別の話だ」 「この話とその話しは別だよ」。 「何度も言わせるなよな。それはそれ、これはこれ。話しをごっちゃにしてほしくないな」。	
\\	「それを言ったら、自分で自分の首	
\\	僕はついてなかったんだ	
\\	どう考えても、彼らは自分たちがしたことは正当化できない	
\\	「そろそのあの時間ですね」。	
\\	多く買えば買うほど、得になる	
\\	すぐにそれをやらないとだめだ	
\\	「彼女は生まれつきのリーダーだ」。	
\\	「彼女はきれいな声をしてるが、ステージ向きではないね」。	
\\	なぜかわからないが、だまされた気がする	
\\	「偽物にだまされないように」。	
\\	「この仕事急いでやるといいことないよ」。	
\\	賛成できなくても、他人の意見は尊重しなくてはいけない	
\\	100万ドル?そんな金持ってないよ	
\\	「強く言われると、そんな気になってくるよね」	
\\	そのお金を返すつもりはないよ	
\\	「選択の余地などないじゃないですか。この取り引きは無しですね」。	
\\	どうしてそんなことが言えるのかな	
\\	ほかに言うことはないの?	
\\	「ほめてくれてありがとう」。	
\\	そんな所で何してたんだ	
\\	「そんなに気に入るとは思わなかったよ」	
\\	「同じ事を何度も言うなよ(壊れたレコードみたいだ)」。	
\\	「お前が勝てるはずないじゃん」。	
\\	「人生とはもがき苦しむこと。だから、それとどう付き合っていくかを身につけるべきなんだよ」。	
\\	なんで僕だけいつもつまらない仕事ばかりやらされるんだ	
\\	パトリックがそんなことするはずない	
\\	「今年、工場が生産を増やさなければ、需要に追いつかない」。	
\\	明日結婚すると思うと、背筋がゾッとするよ	
\\	「原稿がとぎれなく送られて来ている」	
\\	「彼は私に背筋か寒くなるようことを言った」。	
\\	「これじゃだめだな(うまく行かない)」。	
\\	「全力でやってみろ」。	
\\	「あいつらとはレベルが違うよな。ついていけないよ」。	
\\	「企業が変化についていけないようなら、財力、あるいは、力があってもあまり意味がない」。	
\\	「電話では普通の声で話すよ」。	
\\	このプリンターは僕のコンピューターに対応していますか?	
\\	生兵法は大けがのもと	
\\	平穏無事に旅行から帰って来てよかった	
\\	「お風呂のお湯の温度が体温と同じなら、それは熱くもなく冷たくもない温度に感じる」。	
\\	仮想現実ゲームはますますリアルになっている	
\\	「その冗談は通じなかった」。	
\\	「彼は、素晴らしい男です。僕が保証します」。	
\\	交渉に臨んだ人たちは解決策を考え出すことができなかった	
\\	「大したことないから。気にしなくていいよ」。	
\\	「見た目ほどじゃないよ」。	
\\	「僕の体は競泳向きじゃないよ」。	
\\	その状況にどう対処していいかわからなかった	
\\	「そんな風にガツガツ食べるなよ」。	
\\	聞いた?今日、ハリーを首にしたんだってさ	
\\	彼女はものすごく内気な子どもだった	
\\	「今夜はどことどこがやってるの」。	
\\	「彼はこれまでに会った中で、最も無気力な奴だね」。	
\\	今日は体調があまりよくないんだ	
\\	「彼女を見たら、ファッション・モデルだと思っただろうな」。	
\\	こいつが1番の社会の敵だ	
\\	俺にため口きくなよ	
\\	「新人のくせに、あいつ態度悪いよな」	
\\	国際便に乗るといつも体内時計がずれてしまう	
\\	ガス、水道、電気の支払いがたまっている	
\\	「近頃は、セールスマンも大変だ」。	
\\	大学入試の受験勉強は大変だよね	
\\	これ以上は耐えられない	
\\	「大変だぞ。ボスが調べに来るんだって」。	
\\	「先月、歯の治療費を大枚2,000ドルも支払った」。	
\\	あなたのことは色々と伺っています。直接お話ができてうれしいです	
\\	これは我々の組織構造を表したものです	
\\	「とってもいい人を見つけたね、結婚おめでとう」。	
\\	最近、本当に体調がよくないんだ	
\\	仕事場でいい人間関係を保つのは大切である	
\\	「彼女はまさにしなやかで、優雅だった」。	
\\	卒中で彼は寝たきりになった	
\\	そのレストランの食事高いけど美味しかった?	
\\	「だだのブルースじゃないよ」。	
\\	きみは安物買いの銭失いだね	
\\	「僕が払います。いつもおごってもらってばかりですから」。	
\\	車をローンで買うと高くつくよ	
\\	この辺りの起伏の多い丘はとても美しい	
\\	「彼女は誰にでもいい顔する八方美人です」。	
\\	高望みすると、この仕事に合った人を見つけることはできない	
\\	焼き飯がいいですか、それとも白いご飯?	
\\	「彼は煙草を取り出し、火をつけた」。	
\\	彼はいくつかの事業に手を出している	
\\	「徐々に、貧しい生活をするようになった」。	
\\	「彼は思ったことをズバズバ言うよ」。	
\\	その問題について人に言って何かいいことがあるのかい?	
\\	「劇場で君が見たのは僕じゃないよ。僕に似た人だったんだよ」。	
\\	「彼、もうすぐくるよ。それは確かだよ」。	
\\	わかりました。調べておきます	
\\	その法案はそのまま国会を通過した	
\\	「上司が彼に、ままだは先はながいな、と注意するように言った」。	
\\	「彼には、ある程度は教養がある」。	
\\	「ラーメンが4でそばが2だね。じゃ、ラーメンにしよう」。	
\\	「私の父は自力でたたき上げた人だった」。	
\\	「この町には、荒々しい開拓地の町のような雰囲気がある」。	
\\	「野球の試合が雨で中止になった」	
\\	ただほど高いものはないんだよな	
\\	ただいま/戻ったよ	
\\	仕返しをしてやる	
\\	「私はただの大工です。お偉いさんのように扱わなくてもいいですよ」。	
\\	あんなに激しい波の所には行かないよ。プールでも泳げないのに	
\\	「その話し、気を付けた方がいいぞ。ただほど高いものはないって言うからな。くわばら、くわばら」。	
\\	「彼女は週末に絵を描いているその辺にいる絵描きとは違うよ。あの人の作品は市街にある最高の画廊のいくつかに飾られてるんだから」。	
\\	「コンピュータまたフリーズしちゃったのかい」。	
\\	最近、経済は難局を迎えている	
\\	「立ち入り禁止」	
\\	「競争相手を打ち負かすと心に決めている」。	
\\	ガールフレンドと別れるなら、ちゃんと別れなよ。変な期待をもたせちゃだめだよ	
\\	「成功にあぐらをかいてる時ではないよ」。	
\\	が彼の電話に盗聴器を仕掛けた」。	
\\	「台所の洗い場で立ちながら食べることもあるよ」。	
\\	「私ってとにかく何でもため込んじゃうの」。	
\\	きみの邪魔をするつもりはない	
\\	彼女が僕のことをどう思っているのか知りたいんだ	
\\	「エレベーターのドアの前で立ち話しをするなよな」。	
\\	「映画館は、余分に切符を売るべきではない。客に対して公平ではない」。	
\\	「ここは本屋なんだから、図書館じゃないんだ」。	
\\	「コンビニに立ち寄って、お昼買って来るよ」。	
\\	で契約を送ってくださいませんか	
\\	ここに来てから丸1年だなんて信じられない	
\\	「この古いビルはここに100年もの間ずっとある」。	
\\	高校生のときにタバコを吸い始めて、去年、吸うのをやめた	
\\	昼食代貸してもらえるかな?	
\\	ジェニングス氏はこの会社を上意下達方式で経営している	
\\	「息子が2歳になった時、親に逆らうようになった」。	
\\	彼女は長年女性の権利を擁護してきている	
\\	「今日の会は無礼講だよ」。	
\\	かなり大変なスケジュールを立てたね	
\\	「彼と私は似たところがまるでない」。	
\\	「最も厄介な話題は棚上げしたので、交渉を続けることができた」。	
\\	「この仕事を捜したわけじゃないんだ。たなぼただったんだよ」。	
\\	「今、とっても忙しいから、他の人に聞いてよ」。	
\\	まったく知らない人と握手したくない	
\\	「彼はいつもおせっかいで、冷たいから、話したくないんだ」。	
\\	「あの二人、とうとう離婚しちゃったんだ。性格の不一致だって。いやー、他人事とは思えないね」。	
\\	「私のしてることに構わないで。あなたには関係のないことなんだから」。	
\\	彼にきつく当たるなよ。悪意があったわけではないんだから	
\\	「トップにのし上がるには、多くの人を踏みつけて行かなくてはならない」。	
\\	今回そちらにおじゃましてとても楽しかったわ	
\\	「金曜日のパーティーが待ち遠しいよ」。	
\\	スタジアムに歩いていきながら、皆、ゲームを楽しみにしていた	
\\	会社から市内を案内するように言われています	
\\	「妻に頼まれて、午後はずっと使い走りをしていた」。	
\\	この仕事、僕の代わりに終わらせて欲しいんだけど	
\\	「この仕事、君に頼んだよ」。	
\\	「彼女の髪は煙草の煙のような臭いがする」	
\\	このハンバーガー生焼けだよ	
\\	「家の娘、可愛くて食べちゃいたい」。	
\\	「誰かが半分食べかけのサンドイッチをゴミ箱に捨てた」。	
\\	「制限時間は2時間だけど、ディナーが食べ放題、飲み放題だよ」。	
\\	ビールでお腹をいっぱいにするなよ。これからディナーが出るんだから	
\\	「自分が昇進すると聞いて、ぶったまげた」。	
\\	「長い間いなかったので、話したいことが1年分もたまっている」。	
\\	「子供達はどうってことないと思って煙草を吸い始める」。	
\\	「こんな不正行為はもういやだ」。	
\\	夫が夕食を作ることはめったにない	
\\	「社長はほとんど会社には顔を出さない」。	
\\	彼は働く必要はない。お金持ちと結婚したから	
\\	「あー、このビールたまらないね」。	
\\	そんなこと犬にもしないよ	
\\	学校帰りにうろついていないで、真っすぐ家に帰ってらっしゃい	
\\	「休みで出かけている間に、仕事がたくさんたまった」。	
\\	「仕事をし過ぎるとストレスがたまるよね」。	
\\	言い負かすことができると思ったが、妻が泣き出したので、こちらが降参した	
\\	そのケーキはとんでもなくおいしかった	
\\	「いつまでゴロゴロしてるのよ」。	
\\	「あいつ態度でかいんだよ」。	
\\	「若い奴らは、立場をわきまえていないよね」	
\\	「口だけではなく、やってみろよ」。	
\\	「何だよ、これ。抜き打ちテストかよ」。	
\\	「この仕事をするのが君の家族のためになるのかい」。	
\\	リーダーたる者、決定的な瞬間にはためらってはいけない	
\\	何年にもわたって、膨大な顧客データベースを構築してきた	
\\	「より深く理解するために、新しい状況をさまざまな角度からいつも見るようにしている」。	
\\	「仕事のやり方は分かってると言うけど、実際、彼はまるで頼りにならない」。	
\\	彼は人に依存し過ぎる	
\\	「お客様からの電話を、間違った人につないでしまうのは無礼である」。	
\\	「彼女は、彼に目一杯お世辞を言った」。	
\\	"「""もし""は彼の好きなことばだ」。
\\	「外へ出たら、そこはジャングルだよ。自分のことは自分でするように(注意)しなくちゃな」。	
\\	彼らはお釣りを50セント少なくくれた	
\\	このコートで大丈夫ね。今夜は外それほど寒くないから	
\\	「彼は何故あんなダブダブのズボンをはいているのかね」。	
\\	「どうでもいいことをあれこれと話した」。	
\\	「今は、そんな話をする時でもないし、場所でもない」。	
\\	この測定の正確さは1ミリ単位である	
\\	授業落としたことある?	
\\	「彼がかっとするの見たことがない」。	
\\	この計画は去年から棚上げされている	
\\	彼は平穏無事な人生を送った	
\\	「深呼吸をして」。	
\\	ここの責任者が誰なのか知りたいのですが	
\\	「これはみんなでやるんだから、一人が頑張るのとは違うんだよ」。	
\\	「数学の先生は誰なの」。	
\\	「チーズバーガー・スペシャルとオレンジジュースをお願いします」。	
\\	一品料理を頼んだ方が高くつくよ	
\\	その問題は1970年代後半に端を発している	
\\	勘定はいくら?	
\\	「第一印象は信じない」。	
\\	何にもまして、家族に誠実であるべきだ	
\\	「大学院で建築学の学位を取ることにしたんだ」。	
\\	「大学院に行ったが、論文を書き上げなかった」。	
\\	「ベッカーは、2セットダウンをひっくり返して、試合に勝った」。	
\\	「彼女が女優なら、豚が空飛んじゃうよ」。	
\\	「スキューバダイビングをしたときの感覚は言葉では言い表せないよ。是非、やってみるといいよ」	
\\	この鮭の素晴らしい味はたまらないね	
\\	毎月最初の金曜日が給料日だ	
\\	「我々はこれまでに誰も目にしたことがないような好気のまっただ中にいる」。	
\\	このレポート早く終わらせなくちゃ	
\\	今回は本当にだめだったね	
\\	それきみの車、それともレンタカー?	
\\	もしものことを考えて、傘を持っていった方がいいよ	
\\	彼女は君のことをとても大切に思っているんだよ	
\\	彼らは長女をとても大切にしている	
\\	「会議で話をすることになってたんだけど、抜けて来ちゃったよ」。	
\\	僕は大丈夫。心配しないで	
\\	きみ一人で大丈夫かい?	
\\	君は大卒なのかな?	
\\	今、だいたい100ドル持ってるよ	
\\	「ガビーデパートが大セールをやっていると言ってたけど、全然大したことなかった」。	
\\	「計画はきっと成功すると思っていたが、失敗に終わった」。	
\\	「新しい製品に関するメールは私に送ってください。もしよければ友人に転送しますので」。	
\\	彼女が電話に出るのにだいぶ時間がかかった	
\\	「いよいよ佐々木の登板です。さて、このピンチを救えるか」。	
\\	ベトナム戦争は、アメリカとソビエトの代理戦争であったとよくいわれる	
\\	システムは立ち上がったのかな。送らなくちゃならない急ぎのメールがあるんだけど	
\\	自己満足でもいいから最善を尽くすべきだ	
\\	こっちへ来て、抱きしめて	
\\	わが社は、以前の競争相手との提携を考えている	
\\	ちょっと擦っただけじゃない。心配することはない	
\\	「この服、可愛いけど、ベルトだけで予算をオーバーしている」。	
\\	その調査はこの国だけに限られたことではない	
\\	「ジーンズにスニーカーは僕のスタイルじゃないんだな」。	
\\	そのださい眼鏡はずせよ	
\\	「私の好きな作家は、何年も新しい本を出していない」。	
\\	「もちろん君が辞めるって言うんで驚いているよ。人を驚かすような事を言っておいて、落ち着いて対処しろってのが無理な話なんだよ」。	
\\	「そんなに弱腰になるなよ」	
\\	「彼は冗談ばかり言っている」。	
\\	服をたくさん持ってくるで「出す」を表現しています。	
\\	「泣いても、欲しい物は手に入らないよ」。	
\\	メ抱っこ紐モはとても便利だ	
\\	「靴の臭い取りが必要だ」。	
\\	でもでも言うなよ	
\\	「丁重に謝りに彼女の家に行った」。	
\\	彼は証言のなかでいくつか大事な事実を言わなかった	
\\	「やる気がなくなった」	
\\	そのビデオ、もうダビングした?	
\\	「前に会った時より、少し太った?」。	
\\	弟をだまして彼の秘密を言わせた	
\\	何人かの年配の人たちがその信用詐欺に引っかかった	
\\	「私の祖母は貧しい育ちだったので、どんな物でもとことん使ってから捨てる」。	
\\	最初うまく行かなくても、何度もやってみることだ	
\\	「最後の追加点で、巨人は阪神にとどめをさした」。	
\\	「一ヶ月分の仕事が無駄になった」	
\\	「小さなミスですべての努力が無駄になるのを見たくはない」。	
\\	「彼はサラリーマンとしてはまったく使い物にならない」。	
\\	「キムタクにインタビューを申し込んだみようか。いいよね。やってみる価値はあるよね」。	
\\	「そこら中、がらくただらけだな。ボスが来る前に片づけておけよ」。	
\\	「今日は格好が少しだらしないね」。	
\\	「髪の毛と服装が乱れていることから、彼が起きたばかりだということが分かった」。	
\\	「庭仕事をして午後時間を潰した」。	
\\	「だらだら話をしていないで、仕事をしなさい」。	
\\	お前が午後ダラダラしている間に、俺たちは皆働いてたんだ	
\\	「体がなまりのように重い」。	
\\	「どこで聞いたのか知らないけど、それ違うよ」。	
\\	「誰か助けて。誰でもいいから」。	
\\	「僕の知っている限りでは、それは以前にはなかったことだね」。	
\\	ハイビジョンテレビの映像は普通のテレビより断然鮮明である	
\\	「我が党の圧倒的な勝利だった」。	
\\	「いいかい、仕事は段取りよくやるのが大切なんだ。あらかじめ自分なりにスケジュールを決めて。ひとつずつ片づけていくようにしなくちゃね」。	
\\	「私には黙秘権があるじゃないですか?」。	
\\	「大統領の警備をするのは大変な仕事だ」。	
\\	「この新しいホームページ、チェックしておいて」。	
\\	彼の心は長年のプロパガンダによりむしばまれていた	
\\	「私たちは50年間、病気の時も、元気な時も、楽しい時も、つらい時も、ずっと一緒でした」。	
\\	太陽がまぶしくてよく見えない	
\\	「ろうそくの火がちらちらして消えた」。	
\\	そばにいて欲しくないんだよ	
\\	彼女は、チームのために頑張るという気持ちが強い	
\\	「この仕事に身を入れないとね」。	
\\	お腹すかしておいてね。今夜はごちそうだから	
\\	最近、力をつけるためにコンピューターの授業をずっと取っている	
\\	子どもにいたずらする奴は、一生、刑務所に入れておくべきだ	
\\	「それ僕には同じに見えるよ」。	
\\	晴れても雨が降っても、そこの行きます	
\\	多くの血と汗と涙の結果がこのプロジェトなんだ	
\\	「同時通訳するには、言葉の知識以上のものが必要となる」。	
\\	ちくりちくり質問されるのは嫌だね	
\\	旅行をすると物の考え方が広くなる	
\\	「牛の乳搾りは君が思っているより難しいよ」。	
\\	ゆっくりだが、着実に進歩している	
\\	「そのシャツは乾燥機に入れちゃだめだよ。縮むから」。	
\\	内戦は、いつでもひどい戦いになる。それは、血の繋がった物同士が戦うことになるからである	
\\	「20世紀を血に染まった世紀だ、という歴史家もいる」。	
\\	その関連で、わが社の新しい製品について触れたいと思います	
\\	彼は絵に描いた餅のような計画を色々と思いつく	
\\	市場は底をついた	
\\	「夜に叫び声が聞こえ、ぞっとした」。	
\\	「歳のせいか忘れっぽくなっている」。	
\\	「インターネット、インターネットって、耳にするのはインターネットのことばかり」。	
\\	「長い間歩いたので足にまめができた」。	
\\	「知名度を上げないと、選挙では勝てないよ」。	
\\	彼女は飲み込みが早い	
\\	歯は年齢とともに黄色くなる傾向がある	
\\	「電話がどんどんかかってきて、対処できないよ」。	
\\	「話をそっちの方に持って行くのはやめようよ」	
\\	「電車の中で携帯使うなよ」。	
\\	「邪魔するのは止めてくれ」	
\\	「その少年はとんでもないいたずらをした」。	
\\	「初めの状態に戻る必要があるね」。	
\\	君にはこれがどれだけ素晴らしいチャンスであるかを理解すべきだよ	
\\	「男に二言はないよ」。	
\\	今度はちゃんと靴のひもを結びなさいよ	
\\	「今度ガソリンスタンドへ行った時に、ちゃんとオイルを点検してもらえよ」。	
\\	チャンネルをちょこちょこ変えるのは止めて、見るものを決めなさいよ	
\\	「何がすごいのか僕には分からない。何で皆この新しい玩具に夢中になってるんだろう」。	
\\	スピード違反の切符を切ることもできるんだが、今回は注意だけにしておこう	
\\	記者会見でそのレポーターはその会社の会計慣行に注意を向けた	
\\	「彼の立場は大変だよ。そこらじゅうからやんや言われるんだから」。	
\\	私の叔父は逆立ちして歩ける	
\\	中学に行っている子どもが2人いる	
\\	「若いことに、必要なワクチンの接種はすべて受けたかい」。	
\\	お気に入りの映画を散々に言っている批評を読んだ	
\\	彼は自分が宇宙の中心にあると思っている	
\\	だれが最初に飛行機から飛び降りるかを決めるためにくじを引いた	
\\	雨が降り出したので試合を中止した	
\\	「中途半場な仕事しかできないんだよな俺って」。	
\\	「時には、中途半端なことが何もしないことより悪いこともある」。	
\\	「このチキン火が通ってないよ」。	
\\	中途半端な対策はもういい。問題を解決するために、思い切って抜本的なことをしよう	
\\	「中途半端な心構えでは、商売は成功しないよ」。	
\\	「中年になるとどうしても太るよね」。	
\\	近ごろは、このファション・デザイナーの話題で持ち切りだ	
\\	「注文を変えたいんですけど、いいですか」。	
\\	僕にとって10日間は長期休暇だね	
\\	「長期的に見ると、現在抱えている問題は、大したことではない」。	
\\	「この調子だと時間通りに空港につかない。急いでください」。	
\\	「調子が出てきたぞ」。	
\\	「おっ、乗ってるね/調子いいね」。	
\\	「今日は調子がよくない」。	
\\	調子に乗るなよ	
\\	「欠点のない(完璧な)奴などいないよ」。	
\\	今日は気分はどうですか?	
\\	その選挙には何か不正があったに違いない	
\\	「俺のこと怒らせようとしてるのなら、あいつは見事に俺を怒らせているよ」。	
\\	その列はドアから始まり、くねくね曲がって通りのほうに出ていた	
\\	朝のラッシュは9時半ごろがピークだ	
\\	ちょうどいいタイミングだよ。ちょうど会議が終わったときに電話をかけてきたから	
\\	自分の子どもには挑発的な歌詞の音楽にふれさせたくない	
\\	そこに座ってないで、役に立つことをしろよ	
\\	彼女はそのペットの犬をとても大切にしている	
\\	「あの場所には、嫌な感じのやつが一杯いる」。	
\\	「彼等は微笑んで握手をしたが、裏では軽蔑していることを私は見抜いていた」。	
\\	「聴力低下は、ロックファンにとっては深刻な問題である」。	
\\	僕のスピーチどういう感じだったかな?	
\\	銀行にどれだけ貯金してるの?	
\\	この店は生産工場の直営なの?	
\\	「去年、円高で我が社は直接の被害を受けた」。	
\\	「演劇が開演する直前で彼等は台本を変更した」。	
\\	触るなよ	
\\	「3時に新宿で
\\	社との打ち合わせがあります。かなり時間がかかると思いますので。今日は直帰します」。	
\\	「ちょっと言っていいですか」	
\\	「1、2杯飲んでいかないか」。	
\\	大学でフランス語を勉強したから少しは分かる	
\\	「彼の言ったことに対する疑いがどうしてもなくならない(心のどこかにある)」。	
\\	こんな数学の問題なんて朝飯前さ	
\\	この部屋きれいにしなさいよ。竜巻が通ったあとみたいじゃない	
\\	かつては街角でチラシを配っていた	
\\	「彼の言うことは、全く意味をなさない」。	
\\	部屋は静まりかえっていた。だれも最初に話したがらなかった	
\\	「つい先月、姉が引っ越しをした」。	
\\	彼ら、つい最近まで結婚していたのに	
\\	「油っぽい物を食べる積もりはないんだけど、時々、ついでかくておいしいそうなステーキを頼んでしまうことがあるんだ」。	
\\	「もう食べない方がいいのは分かってるんだけど、どうしても(つい)食べちゃうんだよ」。	
\\	脂っこいものは食べないようにしてるんだけど、つい食べてしまう	
\\	「彼女は素敵なドレスを見て、我慢ができなかった」。	
\\	「若い子たちのファッションの流行にはとてもついていけないわ」。	
\\	「ちょっと待って。分からなくなっちゃったよ」。	
\\	「なんてついてるんだ」。	
\\	「あの、ついでなんだけど、金曜日の夜忙しくなかったら、飲みに来ないか」。	
\\	スーパーに行くついでに、クリーニング屋さんに行ってくれる?	
\\	「危うく正面衝突するところだった」。	
\\	最新のエアバッグ・システムは、側面の衝突からも守ってくれる	
\\	「市内を走っていると、高速を走っている時のほぼ倍のガソリンをくう」。	
\\	「ロシアの戦闘機による大韓航空機747の撃墜は冷戦による一番の悲劇であった」。	
\\	「ロシアの戦闘機による大韓航空機747の撃墜は冷戦による一番の悲劇であった」。 
\\	「僕は映画狂だから、少なくとも週に2回は映画館に行く」。	
\\	定期券を使ってるの?	
\\	「通勤が耐えられなくて、仕事を辞めたんだ」。	
\\	毎朝どうやって通学してるの?	
\\	インターネットを使った販売は、大きな新しい市場を意味している	
\\	「誰かが電話に盗聴器を仕掛けていたかも知れない」。	
\\	そのようなレトリックは役員会議では通じないよ	
\\	「その銀行が見えない所でそんなにたくさん手数料をとっていたなんて誰も教えてくれなかった」。	
\\	「そんなはったり無意味だよ」	
\\	工事が完成するまで、臨時出口をお使いください	
\\	「この携帯の人間工学を取り入れたデザインはかなりのものだよ」。	
\\	きみの新しいコンピューターの使い勝手はどう?	
\\	「父親が亡くなって一年で、彼は遺産を使い果たした」。	
\\	「君の新しい車にはどのようにして人間工学が取り入れられてるの?」	
\\	「これは使い捨てカメラ用の防水ケースです」。	
\\	「彼は2時間話しっぱなしだった」。	
\\	君の嘘には愛想が尽きたよ	
\\	「彼は、クラッチとブレーキの区別がつかない」。	
\\	いつも商品をどのように送られているのですか?	
\\	「あいつは口だけで、行動が伴わない」。	
\\	「あいつは、どのような状況でもすりぬけられる」。	
\\	「うちの子、何かにつかまっていれば歩けるのよ」。	
\\	おい、何を言っているのかわからないよ	
\\	「疲れてるね」。	
\\	彼女フラストレーションがたまっているようだね	
\\	「熱いお風呂に入れば疲れが取れるんだ」。	
\\	この頭痛がなかなか治らないんだ	
\\	マッサージをしてもらっている内に、疲れが取れていくのを感じた	
\\	ずっと歩いているので疲れるよ	
\\	このチャンスを無駄にしないようにしよう	
\\	「日本の仕事のつき合いは、なかなか大変だ」。	
\\	私とハリーは付き合いが長い	
\\	「いきなり帰るなよ。付き合い悪いぞ。パーティ盛り上がってるんだから」。	
\\	ビルとアンが手をつないでいるのを見かけたけど、あの2人できてるの?	
\\	買い物に行くの?僕も付き合おうかな	
\\	「その飛行機事故の原因を突き止めるのに1ヶ月かかった」。	
\\	最近のスポーツカーにはがっかりだね	
\\	1。「野球のボールが思い切り指にあたって突き指しちゃったんだ」。2。「転んで指をついたら、反対側に曲がっちゃって」3。「突き指をしちゃってさ。それも指三本だよ。右手の指だから、箸を持つのが大変」。	
\\	彼は次から次へと文句を言ったが、だれも聞かなかった	
\\	「彼は次から次へと冒険をする」。	
\\	「このフルコースにはコーヒーは付いてますか」。	
\\	「痛っ、テーブルにぶつけてすねをすりむいた」。	
\\	ビールおごってよ	
\\	「彼はこの仕事には向かないとつくづく思った。あまりに退屈で、り返しが多いから」。	
\\	この仕事がつくづく嫌になった	
\\	「その庭はきれいにデザインされていた」。	
\\	「育て方がひどいと、ねじ曲がった性格になってしまう」。	
\\	「毎晩のように宿題を先送りしていたツケがまわってきた」。	
\\	「(彼の)過去のつけが回ってきた」	
\\	彼は平気で人を使う	
\\	「出るときにドアを開けっ放しにしないでね」。	
\\	今朝、ドアに鍵を掛けずに家を出た	
\\	彼が受けた教育では付け焼き刃の知識しか身に付かなかった	
\\	「来週、ランチを一緒に食べないか。いつが都合がいいかな」。	
\\	話したいんだけど。いつがいい?	
\\	都合がつかずに、パーティーに出られなかった	
\\	いつが都合がいいですか?	
\\	来週会いたいんだけど。いつが都合がいい?	
\\	「そのスキャンダルへの関与に関して、彼は、支離滅裂なことを言った」。	
\\	どうしてるかと思って手紙/メールを書きました	
\\	「この歌は、世代から世代へと受け継がれて来た」。	
\\	そのプラスチックの袋は、分解して土にかえるのかしら?	
\\	そんなにむきになるなよ。昨日の夜どこにいたのか興味があるだけなんだから	
\\	問題の核心に行くまでに、我々は何時間も話した	
\\	他人の事に首を突っ込むなよ	
\\	彼はホームレスのような暮らしをしていた。株で大儲けしているのに	
\\	「降れば土砂降り(悪いことは続けて起きるもの)」。	
\\	話題を変えたくはないんだけど、この本、もう読んだのかな?	
\\	このコースはどれくらい続くのですか?	
\\	少し自分の時間を取らないと、気がおかしくなるよ	
\\	ダイアルしたらすぐに電話がつながった	
\\	「これって市外通話なの、それともフリーダイヤル?」。	
\\	「角が立たないような言い方を考えるようにしろよ」	
\\	「粒々の入ったピーナッツバターが僕の好物なんだ」。	
\\	「この仕事をしていても何の能力も身に付かないよ」	
\\	「このプログラムを売りに出す前に、彼らはバグ取りをすべきだったんだ」。	
\\	「オフィスがどれくらいの広さがあるかでその人のステータスが分かるよね」。	
\\	「昨日の夜は、友達と冗談を言い合って大笑いした」。	
\\	パレードが通るのを見るためにつま先で立った	
\\	今日は、本当に退屈だったな	
\\	近ごろ、ジャックはアマンダに興味をもっている	
\\	「ちょっとした感謝のしるしです。大した物じゃありませんから、本当に」。	
\\	「今週の僕のスケジュールはぎっしり詰まっている」。	
\\	手短に言うと、破産しちゃったんだ	
\\	引退するために十分なお金をなんとか少しずつ貯めた	
\\	これは積立式の保険ですか、それとも、掛け捨て?	
\\	罪滅ぼしに彼は高価なプレゼントを買った	
\\	「爪伸ばしてるの。マニキュア塗ろうと思って」。	
\\	電車の中で、化粧をして爪の手入れをする女の子がいる	
\\	彼は私のことを怒ってにらんでいた	
\\	「何年もの間に、彼らはお互いを愛いするようになった」。	
\\	「彼女は自分以外の人のことなど考えない」。	
\\	「彼は、無理矢理依願退職させられた」。	
\\	「随分痩せたので、コート(のウエスト)を詰めなくてはならなかった」。	
\\	スペースを空けるために詰めた	
\\	もう少し詰めていただけますか?	
\\	「その件に関する事実について彼女は十分に把握している」。	
\\	「先週は1メートル雪が積もった」。	
\\	「彼女が帰って来るのではないかと思って、私たちは一晩中起きていた」。	
\\	「この写真、艶消しにしますか、光沢のある方にしますか」。	
\\	一年のこの時期には、しとしと雨が降り続くんだ	
\\	「梅雨寒って嫌だよね」。	
\\	「そんなに露骨に感情を表す必要はないだろう」。	
\\	「討論をする時に、手心を加えるようなことはしないよ」。	
\\	強くなれよ。こんな問題気にするな	
\\	この笛をできるだけ大きく吹いて	
\\	「私が状況を説明すればするほど、彼の抱いていた疑念がますます強くなった」。	
\\	「僕の強みは、周りがどんなにひどい状態でも集中できることです」。	
\\	彼の強みの1つに忍耐力がある	
\\	昨日の夜、歯が痛くて大変だったんだ	
\\	彼は辛い経験をしてきた	
\\	いやー、たまんないねは、ビーチでリラックスしているときや、入浴中などに使います。	
\\	弾丸が彼の脚を貫通した	
\\	そんなに過敏になるなよ。彼はそんなつもりで言ったんじゃないんだから	
\\	「ウォーミングアップせずに思い切り走ったら足がつった」。	
\\	「あいつら一緒になって、新しい会社を作るためのとんでもない計画を作り上げたんだ」。	
\\	歳を取って忘れっぽくなっている	
\\	旅行に子どもも連れていくつもりなの?	
\\	よかったらパーティーに友だち連れてきて	
\\	「今回の出張では、通訳が必要になりますかね」。	
\\	「お気遣いから、そう言って下さったんですね」。	
\\	「彼は、誘拐と車の窃盗で有罪になった」。	
\\	彼、何かあってイライラしてるんだな	
\\	「彼はつんとした態度をとった」。	
\\	「週末はずーっと(テレビで)スポーツを観てるよ」。	
\\	「この部屋は(最大で)240人収容できる」	
\\	「それは僕の言ったこととは(正確には)少し違います」。	
\\	「彼は生まれた時から、君主になるための教育を徹底して受けてきた」。	
\\	「利率の高い貯金口座が必要だよね」。	
\\	教授は何を配ってるんだ?抜き打ちテストみたいだな	
\\	カジュアルな服装で会社に行くのはちょっとね	
\\	時間通りに電車が来たね	
\\	「彼は、夜、家に帰りたがらないんだ。尻に敷かれてるからさ」。	
\\	「墓地と幽霊は概念的に密接に結びついている」。	
\\	「仕事を2つ抱えていて、それに赤ちゃんの世話をしなくちゃならないから、彼女手いっぱいなのよ」。	
\\	「電気システムに問題があるために、先週、何度が停電した」。	
\\	「エンジニアとして彼ぐらいの能力があれば、この仕事は十分やっていけるよ」。	
\\	""は妊娠していることを表す時によく女性が使う決まった言い方です」。
\\	エアーバッグはこのモデルの車には標準となっています	
\\	絶えず警戒していなくてはいけない	
\\	「このシャツ、(汚くて)洗濯してもだめよ」。	
\\	「彼は人を批判する時は、決して手加減しない」。	
\\	最近は時間を持て余している	
\\	「あいつキャベツの塊のような札束を持ち歩いているんだぜ」	
\\	今週はこれ以上仕事はできないよ。もう手一杯だから	
\\	「質問があったら気軽に電話してください」。	
\\	「適性検査は、頭の良さを測定するものではない」。	
\\	「これその箱に適当に入れていいよ」。	
\\	何の準備もなしに彼は適当に話をした	
\\	「料理する時は、レシピは使わないの。あり合わせのものを適当に混ぜてつくっちゃうから」。	
\\	「それで苦しむことないよ」	
\\	「喫煙、飲酒、それに、ギャンブルはそれほど悪いことではない。適度にやることができるのなら」。	
\\	「彼はガードマンということになっているが、暴力団まがいの仕事もずいぶんやっている」。	
\\	「彼女は腕まくりをして直ぐに仕事に取りかかった」。	
\\	「駐車禁止は日曜日には適用されない」。	
\\	この種の問題にはどの法律が適用されるのか	
\\	「あの2人は本当に面白いよ。大したもんだよ」。	
\\	「批判が嫌なのは分かってる、しかし、反応を聞くのはとても大切だよ」。	
\\	「この靴、(自分の予算では)とても買えないね」。	
\\	彼等は助けを求めて必死になっている	
\\	申し込み金を支払う必要がありますか	
\\	あの
\\	を使うと手数料を払わなくてはならない	
\\	何となく君が今日そのスーツを着てくる気がしてたんだ	
\\	「会社でセクハラがないか徹底的に調査をしています」。	
\\	報告書をすべて読む必要はない。ざっと目を通せばいいんだ	
\\	「ささっと夕食食べてしまいましょう」。	
\\	まだあなたの手続きは済ませていません	
\\	子どもを養子にするには多くの手続きが必要となる	
\\	「僕は鉄棒は駄目なんだ」。	
\\	「僕は鉄棒の得意(名人)だ」。	
\\	きみの年間の手取り額はいくらなのかな?	
\\	「あなたが舞台に上がる前、私が緊張しちゃったわ」。	
\\	「この仕事量じゃ手に負えないよ」。	
\\	「今、仕事で身動きとれない」。	
\\	「彼女には、手に職があるからね」	
\\	この街のことはよく知ってるんだ	
\\	「手のひらに収まるような小さなコンピュータもある」。	
\\	「彼は世話をするのが大変だ」	
\\	「昨日は、彼、チームプレーに徹していたように見えたけど、今日は全然違ってみえる(がらりと変わった)」。	
\\	「姉の結婚式の準備をするのに2ヶ月かかった」。	
\\	「パーティーに手ぶらで行った」。	
\\	「自慢するわけじゃないけど、服の趣味はいいと自分では思うよ」。	
\\	何があったのか手短に話すよ	
\\	「いいよ、彼とデートしてくれば。でも、振られても知らないよ」。	
\\	「1万ドルぐらいは貯金してあるんだ、まさかの時のためにね」。	
\\	「電車の中で席を譲るときって、周りのことが気になるよね(照れくさいよね)」。	
\\	「センチになるのはよしてくれ」。	
\\	彼は神経質に笑ったが、何も言わなかった	
\\	お世辞を言っても何にもならないよ	
\\	現在の世界状況をきみはどう見るかね	
\\	バスの外に手や腕を出さないようにお願いします	
\\	「(使い方の)分からない機械をいじくらない方がいいよ」。	
\\	「その仕事で手を抜くなよ」	
\\	「行かせてよ」。 
\\	「ずっと手を握っていられないんだから。自分でやらなくては」。	
\\	挨拶のつもりで手を振ったが、彼女には見えなかった	
\\	「わざわざ手伝ってくれてありがとう」。	
\\	「意外な展開が多くて、話の流れが分からなくなっちゃったよ」。	
\\	「これはどこでも行けるフリーパスのようなものですよ」。	
\\	「晴れてるのか、雨が降ってるのか、どっちなんだろう」	
\\	「君は機嫌が悪くなるということがないのかい」。	
\\	天気予報では何て言ってた?	
\\	転勤を申し出たんだけど、まだなんだ)	
\\	「彼は転勤族の一人である」。	
\\	「この若いボクサーのことを知っている人は多くはないが、彼は無視できない存在だ」。	
\\	「仕事を換えてもいい頃だね」。	
\\	「彼はついに理想的な仕事についた」。	
\\	「記者の質問がより具体的になるに連れ、部屋の緊張が高まって行くのが感じ取れた」。	
\\	「彼には心をなごませるような魅力がある」。	
\\	「彼は床下にお金を隠していた」。	
\\	天井が高いので、この家オープンで風通しがいい感じがするよね	
\\	「歯止めがきかなくなったインフレが依然としてロシア経済を苦しめている」。	
\\	「あの二人は完璧なカップルだよ」。	
\\	「彼には、この会社がどのような組織になっているのかまるで分かっていない」。	
\\	彼ってちょっと間抜けなところがあるよね	
\\	「今度入った新人の女の子いるでしょ。あの背が高くて、ショートカットの子。あの子天然ボケだから話してると笑えるよ。」	
\\	「優勝がこのゲームに掛かっている」	
\\	「書類は電子メールに添付してお送りします」。	
\\	「母は、油で揚げるのが好きではない。台所が油で汚れるから」。	
\\	あの男は天罰を受けるであろう	
\\	「今夜は料理したくないから、ピザでも取ろうよ」	
\\	「ここでお召し上がりですか、それともお持ち帰りですか」。	
\\	インターネットを通じて妻と出会ったんだ	
\\	インターネットの出会い系サイトはどんどん増え始めている	
\\	昨日たまたま彼に出くわした	
\\	彼はぐでんぐでんになるまで飲んだ	
\\	「このデータうまくまとめればもっと処理しやすいんだが」。	
\\	「待って。言わないでよ。知ってるんだから、知ってるんだって」。	
\\	「前は水がドバーッと出たのに、今はポタポタたれるだけ」。 
\\	「水のでがよくなかったので、水道屋に電話した」。	
\\	「水がポタポタ垂れてるよ。蛇口から水が漏れてるんだ」。	
\\	息子は得意なものが何もないようにみえる	
\\	「彼は、誰とでも友達になれる」	
\\	もう彼氏はいるの?	
\\	「融通がきけばきくほど、会社は君のことを重宝するよね」。	
\\	出来たてのマフィンが一番だね	
\\	彼らは赤ちゃんができたのでラスベガスで急いで結婚をした	
\\	「テーブルにぶつけて、足にあざができた」。	
\\	彼、インターネットで彼女見つけたんだってね	
\\	「彼がここの仕事につけるように、できるだけのことはすると約束をした」。	
\\	「数ヶ月フランス語を勉強したが、会話が出来るようにはならなかった」。	
\\	「何かちゃんとできるのもないの?」	
\\	「あの二人は正反対だよな」。	
\\	お仕事、お疲れさま/頑張ってくれてありがとう	
\\	せかすなよ。考える時間が必要なんだから	
\\	「あいつはいつも会話に割り込んでくる」。	
\\	「大会9日目、ベスト8のプレーヤーが顔を合わせた(対決した)」。	
\\	「今年のクリスマス用の新しい玩具が棚にずらりと並べられている」。	
\\	「このチームのスターピッチャーは、トリプルA出のルーキーだ」。	
\\	「運に任せることにしよう」。	
\\	「何をぺちゃくちゃ話してるんだよ」。	
\\	この映画の終わり方には不満が残るね	
\\	そのプロジェクトは最初から成功だった	
\\	「品質のことはいいから。より多くの製品をはやく作る必要があるんだから」。	
\\	警告が突然画面に出た	
\\	「そんなにしょっちゅうは夢見ないんだけど、夢を見たときには全部覚えてるよ」	
\\	彼の持ち味が徐々に出てきた	
\\	「答は分かっていたが、とても緊張していたのでそれが出てこなかった」。	
\\	「そんなに家に閉じこもってるなよ。出てきてパーッとやろうよ」。	
\\	「一人で暮らしている頃、毎日のように中華(弁当)を持ち帰って食べてたよ」。	
\\	「思いつきで言ってるんだけど、でも、彼女が何でああいう反応をしたのか分かるような気がするよ」。	
\\	あーあ、 交差点の真ん中で止まりそうだ	
\\	「彼女のことを話すと、彼はうっとりした目になる」。	
\\	「旧式潜水艦は軽油とバッテリーで動いていた」。	
\\	この部屋にコンセントはありますか	
\\	「彼はいつも朝8時に新聞を買いに店に来るんだ。時間通りに来るんだから」。	
\\	「送電線のそばに住んでいると、癌になるという証拠は何もない」。	
\\	「彼の陽気さが部屋中に広がった」。	
\\	「この電池はまだ大丈夫かな	
\\	テープレコーダーのスピードが遅くなっている。新しい電池が必要だな。	
\\	「このラジオはとても感度がいい」。	
\\	「君の携帯って地下鉄の中でも受信できるの?」。	
\\	電話番してるだけでいいからさ	
\\	その電話、手術して耳に取り付けたのか?	
\\	「誰かに聞かれても、僕たちはこの話はしなかったんだから。いいね」。	
\\	この取り引きは、もちろん、帳簿にはつけないよ	
\\	「彼と僕は波長が合うんだ」。	
\\	「メキシコ料理だったら、タコスを食べてみたらいい」。	
\\	「そう言えば、近頃、寒くなってきたよね」。	
\\	「彼は骨董品にとても詳しいから、専門家と言ってもいいくらいだ」。	
\\	「ビールを飲むつもりなら、トイレのそばにいた方がいい」。	
\\	彼の言葉は日によって違う	
\\	彼女は美しくほっそりした体に恵まれていた	
\\	私は高額な
\\	費は、消費者の信頼を得るための投資だと考えている	
\\	この件に関しましては、投票するのを控えることにします。その結果に利害関係があるからです	
\\	「投票により知事を職から退かせた」」。	
\\	我が候補は当選確実だ	
\\	「当然のことですが、彼女には礼儀正しくすべきです」。	
\\	我が社の車にはパワステが標準装備されています	
\\	「彼はエンジニアリングからバロック音楽まで、さまざまな話題についてすらすらと話すことができる」。	
\\	「僕の打順はいつかな」	
\\	「高投票率は新候補には有利に働くが、現職には不利になると思われている」。	
\\	「さしあたりは同じクリーニングサービスを使うことに決めた」。	
\\	「気を付けろよ。君は微妙なところにいて、いろいろ問題になってることがあるからな」。	
\\	この書式に記入してください。登録しますので	
\\	謝ることはないよ。ずっと昔のことなんだから	
\\	「視線は遠くの方を見つめ、彼女の話は遠い昔のことになっていった」。	
\\	車が軽くなれば燃費もよくなる	
\\	「トロントを通ってシアトルに行きます」。	
\\	最近、彼女はよそよそしくなった	
\\	「援助交際は売春の婉曲な言い方にすぎない」	
\\	あと2、3週間で親になるというのはどんな感じなのかな	
\\	「馬鹿な間違えをしたので彼を怒鳴りつけようと思ったが、黙っていた」。	
\\	彼はやる気十分だから、きっと大丈夫だよ	
\\	僕は郊外に住んでいる	
\\	妻は、口紅つけたり、アイラインを引いたりで忙しいんだ	
\\	「彼のことを考えると胸がキューンとするの」。	
\\	時計を逆戻りさせることはできない	
\\	「シザーサラダが私の得意料理です」。	
\\	「背負い投げを練習しなくては」。	
\\	報復するれば満足感を得られるかもしれないが、得策ではない	
\\	「彼って個性があるよね」。	
\\	「彼はいつも無礼だが、昨日の夜言ったことは、特に、ひどかった」。	
\\	「今夜は特に何か食べたいなと思ってた物はあるのかな」	
\\	「今のところは特に必要な物はありません」。	
\\	店内のすべての品が通常の半額になっています	
\\	水曜日の夜の特別番組をお見逃しなく	
\\	私のケースを例外にしてくれるといいのだが	
\\	芸名はありますか	
\\	「一般的に、自動車保険には、損害賠償特約をつけるのがいい」。	
\\	相手がどれだけ強いかが分かって、自信がなくなってしまった	
\\	これはとても厄介な問題だね	
\\	彼は棘のあるコメントをした	
\\	「彼女は、ペットの犬を死ぬほど可愛がっている」。	
\\	その件については徹底的に調査しています	
\\	今、話してもいいですか	
\\	「話からそれるわけじゃないんだけど、君の故郷はどこなのかな」。	
\\	どうやって新年を迎えるの?	
\\	「アッという間の出来事だったんだ」。	
\\	「彼も歳だな」。	
\\	「ミニスカートに厚底靴かい。彼女、年齢に合った服装をすればいいのに」。	
\\	「いつもはどうやって新年を迎えてるの」。	
\\	宝くじで100万ドル当たったとしたら、そのお金どうする?	
\\	明日僕が休んだとして、一人でやれるかい?	
\\	「どんなに電話をしても、彼女は出ないよ」。	
\\	「息子さんの教育のことで悩んでいたので、彼女は年齢よりも10歳は老けてしまった」。	
\\	高層ビルが建ってから、テレビの受信映像がひどくなった	
\\	ここで数カ月間働いて、情勢が把握できた	
\\	「ことばの途中で彼が遮った」。	
\\	通勤の途中で車が故障してしまった	
\\	歩くのに疲れて、途中で引き返した	
\\	「この便は、マニラに16時間立ち寄ることになる」。	
\\	試合の途中でルールは変えられないよ	
\\	「時間になったので、試験を途中で止めなければならなかった」。	
\\	「せりふは全部思えたと思ったのだが、それでもうまく行かなかった」。	
\\	「彼女が誰とデートしても構わないよ。僕らはとっくに終わってるんだから」。	
\\	「君とは一緒に仕事がしにくいね、と上司に言われた」	
\\	夕食後は私がテーブルを片づけて、父がお皿を洗います	
\\	「今夜は、スペインで買ったあのワインを開けましょう」。	
\\	「しまっておいた写真が見つからない」。	
\\	「お世辞を言ってもどうにもならないよ」。	
\\	「その嫌な音を立てるのをすぐに止めなさい」。	
\\	昨日は日陰でも[華氏]100度を超えた	
\\	「初めは皆、僕の計画は非現実的だと言ったが、今ではそれを自分の功績にしようとしている」。	
\\	「突然の知らせで申し訳ない」。	
\\	「予定が詰まっているので」。	
\\	「この車には、スキーと自転車用のラックがついています」。	
\\	「嵐が来ると、テレビの受信がひどくなる」。	
\\	大いに前進はしたが、目標まではまだまだだ	
\\	「できれば一緒に座りたいのですが」。	
\\	一生懸命やることを怖がってはいけないよ	
\\	自分の家族から盗むなんて、もう、物が言えないわ	
\\	医者だからといって、その人が健康であるとは限らない	
\\	「スケジュールが厳しいからな。仕方ない。初校を飛ばすか」。	
\\	何だよ、勝手に入って来て	
\\	「俺達を殺す気かよ。そんな運転して」。	
\\	その事故の後、ガラスが通りのあちこちに飛び散っていた	
\\	どれくらい跳べるかやってごらん	
\\	このおもちゃはとても人気があるので、すぐに売れてしまう	
\\	「このあたりのバーは、薄汚れたいいかがわしいところしかないよ」	
\\	「おいおい、お前がそんなうぶなはずないだろ」。	
\\	「そのことは前に伝えてあるでしょ。聞いてないなんて信じられない。とぼけるのもいい加減にして!」	
\\	ごめん、まだどっちか決めていないんだ。決心がつかないんだよ	
\\	何を言うべきかわからない	
\\	「いつまで親のすねをかじるつもりなんだよお前は」。	
\\	「彼は政界ではよく知られていた」。	
\\	そばに新しい大学が開校したことに伴って、若い人たちの数がどっと増えた	
\\	「コンピュータソフトの売り上げが増えると共に、ビールスを防ぐ為の責任も当然増すことになる」。	
\\	傘を持っているということは、雨が降ると思ってたんだね	
\\	「そのニュースの君の見解はどうかね」。	
\\	「今回の連載は、取りあえずこれで行こう。で、著者を誰にするか。山田先生に頼むとするか」。	
\\	「決めるまで、ビールを飲むことにします」	
\\	「この間違いのパターンは日本人の英語話者に典型的に見られるものです」。	
\\	気をつけて。この人形、とっても壊れやすいから	
\\	「最後に、『以上、取り急ぎ用件のみにて失礼いたします』と打って、このアイコンをクリックするだけ。電子メールって本当に便利だよね」。	
\\	「日本人は、外国のアイディアを吸収し、改善することが得意である」。	
\\	「彼は頼りになるよ。それは言えるね」。	
\\	これは一回払いに致しますか?	
\\	「あんな事、彼に言わなければよかったのに。もう取り消せないわ」。	
\\	この会社は環境に対して積極的なポリシーを持ってるね	
\\	「君は、この仕事に取り組めると思うかね」。	
\\	「彼女は自分の魅力で男達を虜にする」。	
\\	靴が濡れてしまったので、乾かすために日干しした	
\\	きみの噂でもちきりだよ	
\\	「家を出るときには、ドアを開けっ放しにしないでよ」。	
\\	そろそろ犬をトリムしてもらいに連れて行かないとね	
\\	「あいつ自身はいいやつなんだけど、付き合ってるやつらがね」	
\\	「頭を冷やせよ」	
\\	「我々のチームは今シーズン、タイトルを取り返した」。	
\\	「彼女は見る影もないほど落ちぶれている」。	
\\	「インターネットで取り寄せる(注文する)のが好きなんだ」。	
\\	その番組の途中でビデオ[の残り]がなくなってしまった	
\\	「韓国の人は、ユッケと呼ばれる料理を食べます。それは、基本的には、生の牛肉で、その上に生卵をのせることがよくあります」。	
\\	私の正義感が問われている	
\\	そんなに急いでどこに行くんだい?	
\\	そんなばかばかしいことを言うなよ	
\\	その計画には1つだけ困ったことがある	
\\	「親というものは、子供の成功を自分のものとし、失敗の責任は取りたがらない」。	
\\	この仕事は退屈だから辞めたいよ	
\\	「どんな精神状態ならそんな残酷なことができるんだろう?」。	
\\	日本ってどういう所?	
\\	「味見してよ。どう?」。	
\\	どうしたんだよ	
\\	「締め切りを延ばしてもらうのは、どうかと思うよ」。	
\\	「彼女に会う度に、胸がときめく」。	
\\	「近頃、道義的責任が流行りことばになりつつある」。	
\\	このスポーツをするにはどのような道具が必要ですか	
\\	僕たち気が合うんだ。同じスポーツファンだから	
\\	「{それを聞いて感動すればいいわけ」。	
\\	「政治家があなたのためにやると言っていることを決して信じてはいけない」。	
\\	頼むから、このことは女房には言わないでくれ	
\\	サンディーの結婚プレゼントに何を買ったらいいかわからない	
\\	何ができるって言うんだい?	
\\	あきらめてやめるのはよせよ	
\\	こんなときにどうして落ち着いていられるんだ?	
\\	「言ったとたんに彼女はハンマーを渡してくれた」。	
\\	そんな質問をするなんて、君も図々しいね	
\\	それで、僕は何をすることになってるのかな?	
\\	月曜日の午後は何してるの?	
\\	[どうせ]こうなるような気がしてたんだ	
\\	あの2人は同棲しているらしいよ	
\\	「中途半端にやっても意味がない」。	
\\	それでヨーロッパでの休暇はどうだったの	
\\	「自分の間違いが分かったとき、彼は平静を失った」。	
\\	彼の見かけはどうでもいい。問題は、その仕事ができるかどうかだ	
\\	だれも気にしないよ。そのことでやきもきすることはないよ	
\\	「彼はどうどうと机に脚をのせた」。	
\\	振る舞いでその人のことがよくわかる	
\\	「同じことを何度も議論して、元に戻ってしまった」。	
\\	上司がこのことを知ったらどうなる?	
\\	「仕事を止めるの?それでどうするの?」	
\\	「この人どうにかならないの?あの性格、死んでも直らないだろうな。自分のことしか頭にないんだから」。	
\\	新しい税制が5年間にわたって段階的に導入される	
\\	どう見たって、彼は君をだまそうとしてるんだよ	
\\	僕の叔父は、かつて呑み屋をやっていた	
\\	どうやら今日は彼来そうもないな	
\\	どうりで彼らが急いで結婚したわけだ。彼女、同じ年に赤ちゃん生んでるんだもの	
\\	元気を出せよ。個人的なこととして受け取ることはないよ	
\\	レシピどおりにやることが大切なの	
\\	「そのセーター、どれを着ても合わないよ」	
\\	「彼らは腹の皮がよじれるほど笑った」。	
\\	彼はとても興奮していたので、胸がドキドキしていた	
\\	「君は働き者だね、馬力があるね」。	
\\	眼鏡がないと全然見えないんだ	
\\	彼女はまさに生き字引だ	
\\	あーあ、読む本はたくさんあるけど、時間がない	
\\	他人をそんなにひどく言う必要はない	
\\	このコーヒー、独特な味と香りがするね	
\\	これはユニークな車だよね	
\\	「彼の意見は取るに足らないものだ」。	
\\	大学を卒業してから、自立したね	
\\	「彼に跪く必要などないよ。ただ、ごめん、と言えばいいだけさ」	
\\	「ずっと見かけなかったけど、どうしてたの」。	
\\	「仕事の帰りにクリーニング屋に立ち寄った」。	
\\	「どう考えても、それは侮辱に聞こえる」。	
\\	「彼は臆病者ではない」。	
\\	きみがどこへ行こうが、何をしようが、僕はきみについて行く	
\\	どこまで洪水による水が上昇するかはわからない	
\\	「そのクーポンはどのくらい有効なの」。	
\\	今年は多くの女の子達が彼にチョコレートあげたから、私のはどこかに紛れ込んでしまったんだと思う	
\\	昨夜は土砂降りだった	
\\	「財布に、一度も使ってないテレフォンカードがあるよ」。	
\\	彼はどすを効かせた声で話そうとしたが、それを聞いて僕は笑ってしまった	
\\	「彼は鈍感なやつだ」。	
\\	「彼は、土壇場でキャンセルした」	
\\	ドタキャンしてごめん	
\\	最後の一頑張りしてみよう。やってみる価値はある	
\\	「どちらでもいいんですが、本当に、でも、もし選べるなら、オートバイよりも大きなスクーターの方がいいですね」。	
\\	「いつでもカジノへ行けばジムに会えるよ」。	
\\	「ベイクドポテトとフレンチフライどっちか好きかって?とっちとも言えないね」。	
\\	読解力が日々ついてきている	
\\	「早く心を決めろよ」。	
\\	「両方に言い分がある」	
\\	「その調査は些細なことで難航した」。	
\\	「大声で言うことはないよ。ちゃんと聞こえてるから。だから、落ち着いて、何が問題なのか言ってみて」。	
\\	そんな口の利き方をして、何を考えてるんだ	
\\	お力にはなれません	
\\	スピーチはすっかり頭に入っていると思ったが、どういうわか間違えてしまった	
\\	彼ってギャング映画に出てくるような人みたいね	
\\	「そのショーを全て水中で撮影した」。	
\\	だれが勝ったのかは知ってるけど、2位はだれだったの	
\\	ドルで値段がいくらになるか計算した	
\\	子どもたちは万引きがどれだけ重大な犯罪であるかをわかっていないことがよくある	
\\	「どっちがいいかなんて決められないよ」。	
\\	学校が始まるまでにどれくらい時間があるのかな?	
\\	「僕の会社の経営陣には排他的な雰囲気がある」。	
\\	「その交渉は好意的な感じで始まったが、すぐに混乱状態になっていった」。	
\\	洪水の後、ビルのフロアがどろどろしたもので覆われてしまった	
\\	そいつのことらな知ってるけど、ちょっと名前は思い出さないな	
\\	彼らの結婚は破綻している	
\\	あの映画には意外な展開がたくさんあったので、何が起きているのかわからなくなった	
\\	「電車にどんどん人が流れ込んで来た」。	
\\	塵も積もれば山となる	
\\	どんなことをするのでも、いつでも出会う人には礼儀正しく敬意をもって接すべきである	
\\	「彼らは談合をして、それを密かに片づけてしまった」。	
\\	あいつにあんな風にこき使われることはないよ	
\\	車のドアに腕をぶつけたら、ひどいあざになってしまった	
\\	「今言ったこと、誰にも言わないって約束して」。	
\\	「上げられない物を欲しがらないでね」。	
\\	この本、内容が少し乏しいよね	
\\	文句を言うのは嫌がっている訳ではない	
\\	膝のすり傷が二日たってもまだじぐじぐしている	
\\	「先週、膝をひねっちゃったんだけど、全然よくならないんだ」。	
\\	「二人はどうなの」。	
\\	「片づけようか」	
\\	「僕たちはもう終わりだよ。ごめん、そういうことなんだよ」。	
\\	「彼らはバスのボディーの両側に広告を描いて広告板にしようとしている」。	
\\	一度別れてしまったら、元に戻るのはかなり大変である	
\\	このピザ、全然まずくないじゃない	
\\	これ、なかなかきちんとした絵だよね	
\\	「あいつ、やるじゃない(仕事できるじゃない)」	
\\	「誰もが最も恐れているのは、国の経済の空洞化である」。	
\\	「彼は交際の範囲が広い」。	
\\	「火が通ってない豚肉は食べちゃだめだよ」。	
\\	子どものころ、僕は一人でいた	
\\	「もう少し具体的に言ってもらえますか」	
\\	「これからもよい付き合いができるようにしましょう」。	
\\	「今日は土砂降りだ」	
\\	この家では、いくら長居しても構わないですよ	
\\	「彼には先見の明がない(「近眼」の意味もある)」	
\\	そういう短期的な見方をしちゃだめだよ	
\\	1分が1時間のように思えた	
\\	「ゴール前で彼は最後のダッシュをしなかった」。	
\\	「このデータを君のプログラムに入れて、結果を見てくれ」。	
\\	「ジョギングにしばらくはまってたけど、長続きしなかったね」。	
\\	「この家には公園に面した大きなサンルームがある」。	
\\	鶏肉は冷凍しないと長持ちしないよ	
\\	「彼は携帯で大声で話しながら、バイクに乗っていた」。	
\\	「かつてはここに川が流れてたんだけど、もう流れてないね」。	
\\	川の方からかすかに音楽が聞こえてきた	
\\	「ロックの曲が流れていた」	
\\	うちの犬はめったに吠えない	
\\	自分の悲しい話をして人を煩わしてはいけないよ	
\\	「彼女は酔っぱらうと可愛いよね	
\\	「彼女の目には喜びの涙が光っていた」。	
\\	余分なペンは持ってないよね	
\\	「セクハラされてるんだから。ちゃんと言うべきよ。黙ってちゃだめ」	
\\	「鳴いたっていいじゃないか」。	
\\	縦列駐車をするには、縁石の方に車を向けなくてはいけない	
\\	「去年、車をめちゃくちゃにしちゃったので、新しいのを買わなくてはならなかった」。	
\\	「一等はハワイ旅行で、残念賞はハワイのコナコーヒーです」。	
\\	「彼女は、自分の名前を大きな汚い字で殴り書きした」。	
\\	あるもので間に合わせなくてはいけない	
\\	「彼が別れたいと言った時、私は家に帰って思い切り泣いた」。	
\\	会社での待遇の悪さを嘆くより、辞めて別の仕事に就いたら?	
\\	「いい仕事なのに、辞めたいわけ?」。	
\\	「彼はかつてほど誠実ではない」。	
\\	彼は情け深い人だが、タフとは言えない	
\\	「素晴らしいね~。君のお陰で、もっと大変になちゃったね(皮肉)」。	
\\	その知らせは彼に伝えるな。彼を惨めにするだけだから	
\\	「彼は暴力に訴える」。	
\\	「その受賞式で、私は、会社から特別賞に選ばれた」。	
\\	こつこつやって、その仕事を片づけた	
\\	「彼女はいい仕事のセンスをしている」。	
\\	「このキャッチャーミットかなり固いから、なじむまでに数週間かかるだろうな」。	
\\	「責めるんだったら自分だね」	
\\	何事も成せば成る	
\\	「ノックノックジョークをやろうよ」。	
\\	彼は徐々に煙草の煙の臭いが嫌いになってきた	
\\	「会議は水曜日の午後に開かれることになっている」。	
\\	頭痛がしそうだ	
\\	彼は、年と共に辛辣になっていった	
\\	「会社は解雇を避けるためにできるだけのことはしたと言っているが、説得力がない(あやしいもんだ)」。	
\\	「彼女が私より試験の点がいいですって、そんなはずないでしょ」。	
\\	「十年かかったが、彼女を説得して味方に引き入れた」。	
\\	ボーナスがあまりに少なかったので、上司の所へ行って説明を求めた	
\\	「彼の議論は説得力のあるものではなかった」。	
\\	「この歌聞くと、昔のこと思い出すよね」。	
\\	「高校の同窓会、昔のこといろいろ思い出して懐かしかったな」。	
\\	昔懐かしの曲は1日中聴いていられるよ	
\\	「夏風邪は冬の風邪よりひどいよ」。	
\\	「彼は、妻の言うことを聞き、家族の面倒を見る家庭的な男になった」。	
\\	「夏の暑さでグッタリだよ」。	
\\	「信号無視をして横断したからといって警察は逮捕しやしないよ」。	
\\	彼が何を考えているか想像がつかない	
\\	「何かあったら君に最初に知らせるよ」。	
\\	「彼は正直な政治家だと言っているが、何かいんちきくさいところがあるよね」。	
\\	「君の勉強が大変なのは分かってる。でも、頑張らなくちゃね」。	
\\	「あいつは本当のところは分かっていない」。	
\\	絶対に、この試験に合格するんだ	
\\	何もなかったようなふりをするなよ	
\\	「彼の言ったことは聞こえたのだが、初めは分からなかった(印象に残らなかった)」。	
\\	答えてくれないとはどういうことなの?	
\\	「忍耐力も技術と同じように大切である」。	
\\	あの政府の役人みたいな人たちなんでここにいるの?	
\\	「このリストはアルファベット順になっています」。	
\\	彼女がどの国の出身かはよくわからない	
\\	「どんなことがあっても約束を守るべきだ」	
\\	「何不自由なく彼女は育てられた」。	
\\	このことはだれにも言うなよ	
\\	「体操は平均点で点数をつける」。	
\\	「医者というのは名ばかりで、彼は医学を知らない」	
\\	「あの娘、生意気なんだよ」。	
\\	「昨日の夜、雨の中を帰って来たから、今日の朝になっても、靴がまだ湿っていた」。	
\\	「彼の話、どうも怪しいよね。誰かをかばってるんだな」。	
\\	前大統領は汚職はなかったと言っているが、どうも怪しい	
\\	「使わないと、ことばってすぐに忘れちゃうんだよね」。	
\\	「なんでそうなことできるのよ」。	
\\	「ナショナルチームが世界選手権で勝ったとき、僕はそこにいたんだ」。	
\\	「彼は支払いをするときに、お札の束を見せびらかしたがる」。	
\\	言葉を身につける一番の方法はその言葉の中にどっぷりつかることである	
\\	消費者の忌憚のない声を聞きたい	
\\	このショーは、スタジオの聴衆の前でライブ録音されたものです	
\\	「両親に話すときにはいつも南部訛になる」。	
\\	「彼の英語はひどくなまっていた」	
\\	フランス語はちょっと下手になったけど、彼女の言ったことはわかったよ	
\\	私の英語は下手になっている	
\\	彼の給料は上級管理職並みだ	
\\	「文句が殺到している」。	
\\	「うねりが浅瀬では大きな波になる」。	
\\	凍りついた滝を登るのは簡単なことではない	
\\	水が無くなってしまった	
\\	その女性の話を聞いて、彼は感動して涙を流した	
\\	「その演劇はあまりにひどくて、皆、(涙が出るほど)うんざりしていた」。	
\\	涙が止まらなかった	
\\	彼はそれ以上涙をこらえることができなかった	
\\	泣いていたので、彼女は本当のことを言っていると思った。しかし、それはそら涙だった	
\\	「彼女の目に涙がきらりと光った」。	
\\	「酒を飲んで泣かないようにしなくちゃな。歳のせいか感傷的になっちゃってね」。	
\\	もう涙をこらえることはできなかった	
\\	元気を出せよ	
\\	「お客様に、そんな見下すような態度を取るべきではない」。	
\\	「彼の恩着せがましい声の調子が嫌なんだ」。	
\\	免税店って安いと思うでしょう。でも、そうとは限らないんだな	
\\	小さなことで悩むなよ	
\\	「おい、そのことで眠れないなんてことがないようにしないとな」	
\\	彼が言ったことで私の心に疑念が生じた	
\\	「彼にふさわしいクリスマスプレゼントを選ぶのは本当に頭が痛い」。	
\\	この窮地からなんとか抜け出す方法を見つけようともがいている	
\\	色々悩んだが、結局、大丈夫だった	
\\	そうですね、それが本当ならいいんですけど。でも、怪しいですね	
\\	新しい教訓になるといいね	
\\	子どものころバレーを3年間習った	
\\	そのメッセージを彼女を奈落の底に突き落とした	
\\	まさに彼って感じなんじゃない?	
\\	並んでどれくらいになりますか	
\\	クビにはならないだろうけど、小言は言われるだろうな	
\\	君は役になりきっていないよ	
\\	「僕は男だが、声が高いのでときどき電話で女性で通ってしまうことがある」。	
\\	長い一日になりそうだ	
\\	「もしできるのなら、月曜日に予約を入れたいのですが」。	
\\	「ここんところの天気どうなってんだろう。変だよね」。	
\\	「彼は新入社員だから、慣れるまでに2,3週間はかかるだろう」。	
\\	何年にも渡って関節炎の痛みに耐えて来たので、彼女は苦しむのには慣れてしまった	
\\	「結婚生活はどうだい?」。	
\\	彼は音感がいい	
\\	彼は人前で話すと本領を発揮する	
\\	新しい仕事には慣れてきたかな?	
\\	独り善がりはよくないよ	
\\	「自分のスペースが欲しい」。	
\\	「今回の時差ボケは本当にきついわ」。	
\\	「平和部隊での経験で、彼女は何か変わったように見えた」。	
\\	「その辺の通りにいるどうでもいいような浮浪者にお金をやったって言うのかい」。	
\\	彼は私に何度となく何を考えているのか聞いた	
\\	「総理大臣は軟禁されていた」。	
\\	「リンゴなら何個でも食べられるよ」。	
\\	彼は、われわれのプランが気に入らないとハッキリ言った	
\\	朝はどれくらい早く起きるの?	
\\	好きなだけクッキーをお取りなさい	
\\	「なんて臆病で弱虫なんだ」。	
\\	「彼の声にはどこか妙に聞き慣れたところがあった」。	
\\	何か理由を見つけてはいつも食べてるんだね	
\\	自分のしたことについてへたな言いわけをした	
\\	「酔っぱらいばかりで、騒がしくて煙りだらけのパブには行きたくないし、かといって、家に一人でいたくはないしな」。	
\\	「名前は何でしたっけ」。	
\\	「ほら、宝くじ当たったぞ。なんてね」。	
\\	「10チャンネルでは月曜日の夜に面白いインタービュー番組を放送している」。	
\\	ほかの言い方がわかりませんので、あなたのチャック開いてますよ	
\\	それは控えめな言い方だね	
\\	「何の障害もなく全ての計画を実行した」。	
\\	どうしてこんな暮らしができるの	
\\	「必要な物はなんでも揃ってる。海に太陽、それに、素晴らしいビーチ」。	
\\	「彼らが何と言おうと構わない。僕はこれは素晴らしい本だと思うよ」。	
\\	「そっちにあるそれ(何とかいうやつ)取ってくれない」。	
\\	「このタイヤから空気が漏れるのを止める方法が何かあるはずだ」。	
\\	あなたを助けるためにできることは何でもします。約束しますよ	
\\	「なんとか今月の収支を合わせる方法があるはずだ」。	
\\	そこに立っているだけじゃなくて、何かしろよ	
\\	「君のプランどうにかなるさ」	
\\	「それは他の誰かの考えなのか、それともふと思い浮かんだのか」。	
\\	「今は言葉にならないよ」。	
\\	私にとって、彼はどうでもいいの	
\\	ボブはすごいよ。朝に10キロ走って、その後で一緒にサーフィンに行っても平気なんだから	
\\	彼はいとも簡単に100キロを持ち上げた	
\\	何度見てもこの映画は飽きない	
\\	何時間も交渉したが、得る物は何もなかった	
\\	「君の家族の構成は?」	
\\	「彼女に何人デートに誘ったか知らないけど、彼女は決して誘いを受けない」。	
\\	「30年ローンで家を買ったんだ」。	
\\	「こんなぎりぎりになって予約はできませんよ。あなたが誰であってもです」。	
\\	「メアリーは男に言い寄られるのが嫌いだ」。	
\\	ディスコに行ってナンパしようぜ	
\\	「うちの娘は、金曜日の夜になると何時間も電話で話している」。	
\\	「プロ野球選手なら、結果を出しさえすれば、のらりくらりとやっていける」。	
\\	「高電圧機器の周りでは気を付けるように」。	
\\	彼は3輪軽自動車のようなものを運転していた	
\\	テロリストには一般人と軍人の違いはない	
\\	そのスーツを着ると痩せて見えるね	
\\	「その帽子、似合ってるよ」。	
\\	もっと決断力があればいいんだけど	
\\	「車用の空気清浄機を買った」。	
\\	「このバー、何かいかがわしいよね」。 
\\	「彼、やれるって言ったけど、怪しいもんだね」	
\\	「お前にようよ。魚か何かさばいてたのかい」。	
\\	「彼女が肩をポンとたたいた時、びっくりして、振り返った」。	
\\	「ついてないよな。傘を忘れてくると、雨が降るんだから」。	
\\	このようなことが起きるのは日本だけだよ	
\\	「多くのレコード会社は、有名なミュージシャンの全曲をカバーした
\\	の箱入りセットを出している」。	
\\	「人の顔を描くの得意?」。	
\\	「僕、これそんなにはうまくないんだ」。	
\\	辛い食べ物は苦手です	
\\	「僕は政治的には、極端な保守派だ」。	
\\	私、保育園で働いてるの。ときどきすごく賑やかになる時があるのよ	
\\	この強壮ドリンクを飲めば元気が出るよ	
\\	「脇腹についた脂肪を取ろうとしてるんだ」	
\\	「この昆虫は小さすぎて肉眼では見えません」。	
\\	「右脚が肉離れになった」	
\\	彼女には人を引きつける何かがある	
\\	それは捕らえどころのない問題なので、解決しようとしてもなかなか解決できない	
\\	「その話題を避けようとしているの?」。	
\\	「その水はにごっていて、臭いがひどかったので、一滴も飲まなかった」。	
\\	「彼は約束を守るよ」。	
\\	「二次会に行こうよ(場所を変える)」。	
\\	多重人格障害は思っているよりよくあることだ	
\\	「このメガネをかけると、すべてが2重に見える」。	
\\	「一日24時間、年中無休です」。	
\\	「いつでもやってます」。	
\\	「僕の会社は2000年問題の準備に追われているよ」。	
\\	「彼は日本のデービッド・レターマンだ」。	
\\	試合の日は決まっていますが、場所はまだです	
\\	「いつもの会話をしている積もりでコンピュータに話しかけてください」。	
\\	彼はごく当たり前のことのように紅茶にオレンジジュースを入れた	
\\	うちのコンピュータシステムは、いつも決まったようにクラッシュする	
\\	「今は、どれを選択してもダメだ」。	
\\	「昨日、朝から晩までテレビを見てただろう」。	
\\	「頭が働かない。いいアイディアが思いつかない」。	
\\	「ジャックはボブと同じタイプだよね」。	
\\	「目はお父さん似」	
\\	状況が幾度となく変化した	
\\	この会社には二度手間になることがたくさんある	
\\	「朝は、ベッドの中でゴロゴロしていたい」。	
\\	「今日が彼の12才の誕生日だ」。	
\\	「個人的な問題は後回しにして、仕事をしなくては」。	
\\	「あのクラブに入会するにはお金がかかる」。	
\\	彼の作品は独創性に欠け、新しさがない	
\\	彼は飲み込みがかなり遅い	
\\	僕はこのソフトの日本語版を使っている	
\\	「字幕より吹き替えの方がいい」。	
\\	「テキサスの人は本当に肉やポテトが大好物だ」。	
\\	「とても謙虚だね。なんて日本的なんだろう」。	
\\	「彼の言葉は信じられるよ」。	
\\	身軽に旅行したい	
\\	彼は若く見えるかもしれないが、もうすぐ50なんだ	
\\	冬の寒い夜には煮物はいいよね	
\\	この詩のなんとも言えないニュアンスがいいよね	
\\	卒業式に何を着て行ったの?	
\\	その品物、在庫切れなんですが	
\\	「一夜づけで定期試験に臨むのはよくない」。	
\\	関係者以外立ち入り禁止	
\\	立入禁止	
\\	漫画を読むのはやめて、仕事をしなさい	
\\	この街ではとっても寛いだ気持ちになれるんだ	
\\	週末は庭いじりをして過ごします	
\\	すぐに戻るから	
\\	「アメリカ大統領は三期務めることはできない」。	
\\	彼はかつてはよく知られた作家だったんだ	
\\	「グレースは誰ともうまくやれない」。	
\\	「同僚と私は、年に一度人間ドックに入る」。	
\\	「このコラムニストはいつも三面記事を書いている」。	
\\	彼は超人的と言っていいほどの力を持っていた	
\\	彼は人が自分のことをどう思っているか鋭く意識している	
\\	「彼女、妊娠してるよ。ほら、見れば分かるだろう」。	
\\	レジュメは皆さんに渡りましたか	
\\	あれはとてもいい映画だったが、興行収入は大したことなかった	
\\	「救助ヘリが沈みかけた船から船員を引き上げた」。	
\\	「ズボンを脱ぎ、イスの背にかけておいた」。	
\\	あいつを出し抜いてやる	
\\	「彼は禿げている訳ではない。ただ少し薄くなっているだけだ」。	
\\	口を閉じていようとしたが、ビールを2、3杯飲んだらペラペラ話さずにはいられなくなった	
\\	「このあたりに近道があるかな?」。	
\\	おい、逃げるなよな。もうすこし飲んでいけよ	
\\	「彼はこの部署にかなり長くいるから、彼のことゴッドファーザーと呼んでいる」。	
\\	「この部屋、盗聴されているよ」	
\\	手を拭くのに濡れたタオルがいりますか?	
\\	我々の提案に対するクライアントの反応はぱっとしなかった	
\\	「処刑をされた囚人の中には間違って有罪となった者がいるというのはほとんど確かなことである」。	
\\	「彼は6時きっかりに飛び起きた」。	
\\	いつも何時に寝るの?	
\\	「これは特製の焼き鳥のたれタレで、3年寝かせています」。	
\\	「神様、お願い。彼女にハイと言わせてください」。	
\\	私の一つの願いは、孫を見るまで生きていたいということです	
\\	彼女からねぎらいの言葉を一言も聞いたことがない	
\\	旅先で値切るのが楽しんだよ	
\\	「明るい性格が彼女の長所の一つだね」。	
\\	「なまぬるいコーヒーが好きなんだ。熱いのはだめ」。	
\\	「彼はぐうたらではない」。	
\\	「誰がこの仕事引き受けるのかな?」	
\\	彼はけがをして1カ月間ベッドに寝ていた	
\\	彼女はあなたと話すときには爪を隠しているのよ	
\\	「僕の兄は、夢遊病者なんだ」。	
\\	「お別れに一杯やろう」。	
\\	みんな伏せて。あいつは銃を持ってるぞ	
\\	「机でぐっすり寝ているところを上司に見つかった」。	
\\	その古くさい冗談は面白くないよ	
\\	「あのコメディアンのネタもつきかかっている」。	
\\	「来週、ハワイに行くの?羨ましいな」。	
\\	子どもみたいに、しつこくせがむのはやめなさい	
\\	今日は朝から機嫌が悪いね	
\\	「彼女とは馬が合わないみたい。こせこせした性格だから」。	
\\	何年か一緒に暮らし、彼らの愛はだんだんと冷めていった	
\\	「梅雨の時に降る霧雨と夏の夕立は違うよ」。	
\\	「君は、熟睡するほ?」。	
\\	昨日の夜はよく眠れた?	
\\	「レセプションでのスピーチで、君、とっても興奮してたね」。	
\\	「彼は時間をすべて医学の勉強にあてている。食事をする時も、飲むときも、眠る時も医学の勉強のことばかりを考えている」。	
\\	けんか腰だね	
\\	よう、ジャック。元気だったか	
\\	「頑張って交渉するしかないね」。	
\\	レストランに長居しすぎると、帰るように言われるよ	
\\	「マンションの持ち主に、価格を少し下げてもらえるか聞いてみた」。	
\\	多くの人は自分が睡眠不足になっていることがわかっていない	
\\	父は僕が週末に何をしたのか、しつこく聞いた	
\\	「半分寝ぼけていたので、何を言っているのか自分でも分からなかった」。	
\\	「1000番目の投稿者になりたかったのに、誰かに先を越されてしまった」。	
\\	スリは眠っている酔っぱらいを標的にすることがよくある	
\\	で「年がら年中」の意味になります。	
\\	40年も掛かったが、とうとう彼は願いをかなえた	
\\	また足首を捻挫したの?	
\\	「6桁の収入ももはやかつてのような裕福さを意味しない」。	
\\	「おやすみなさい。ちゃんと寝るのよ。虫に刺されないようにね
\\	で韻を踏んでいる)」。	
\\	念のため、余分なお金を持っていくことにするよ	
\\	「この仕事がうまくできるんだったら、その人を雇うよ。年齢は関係ないよ」。	
\\	手の内を直ぐに明かしてはだめだ。	
\\	能書きはいいから、早く仕事に取りかかろう	
\\	"僕のこと""ぐず""って呼んでいいよ。
\\	彼って、(私たちと)調子がずれてるよね。	
\\	夏の夕に吹く涼しい風は実に心地がいい。	
\\	きみの立場がよくないって?僕たちがどう感じてると思う?	
\\	締め切りまでどれくらい時間があるの?	
\\	牛乳残ってるかな?	
\\	口先では自由貿易について同意しているが、彼等は、それを何よりも恐れているのだ。	
\\	「それ食べちゃうの?」。	
\\	仕事は早く終わったが、残りの時間を次の日の準備のために費やした	
\\	試合の残り時間はあと2分しかない	
\\	3人兄弟の一番下だったので、いつもお下がりを着ていた	
\\	いつか見てろよ。	
\\	彼はずうたいがでかい。	
\\	クラスのみんなはお互いにエネルギーを引き出しているように見えた	
\\	彼はのっぺりとした、表情のない顔をしている。	
\\	トレバーさんは年を取らないね	
\\	子どもたちは新しくできた海をテーマにした遊園地に行きたがっている	
\\	この時計が他のなによりも欲しかったんだ。	
\\	「今日は、声がかすれてるね」。	
\\	彼はとても長くひげを伸ばした	
\\	我々はマーケットのシェアをなかなか拡大できない	
\\	彼は、自由に、かつ、オープンに話しをして、その記者会見を難なく終えた。	
\\	「この麺、ゆですぎだよ」。	
\\	今年は彼の契約を更新しないことにした	
\\	「靴下(のゴム)がみんな伸びちゃってる」。	
\\	そのことは彼女に何も言わない方がいいよ	
\\	ナスとカボチャ、どっちのほうが好き?	
\\	私の家は、丘の下の方にある青い家です。	
\\	フィジーで地酒を少し飲んだら、頭がクラクラした	
\\	飛行機は数分で千フィートから1万フィートまで上昇した。	
\\	そのような要求を僕に押しつけようったって駄目だよ	
\\	ほら迎え酒だ。	
\\	この蛇口にはフィルターを付けた方がいいね	
\\	その情報を消化する時間がなかったんだ。	
\\	きみって、飲み込みが遅いよね	
\\	いつも一緒に飲んでるんだ	
\\	彼のこと誘ったんだけど、断られちゃったの	
\\	「あいつにはまったく意気地がない」。	
\\	9ドル99セントで食べ放題だって	
\\	彼は、ビールを一気に飲み干した。	
\\	その薬がなかなか飲めなかった	
\\	彼は甘んじてこれを受ける積もりはない	
\\	これを毎日一錠飲むだけで、やせられるんですよ。	
\\	彼は大酒飲みだ	
\\	「猫は家の中で飼ってるの、それとも、外に出してるの」。	
\\	「みんな乗ってたよね」。	
\\	3回のホームランのあと、試合の流れが変わった	
\\	僕の父は時代に乗り遅れている	
\\	今はやめられないよ。ここまで来たんだから	
\\	もう少し熱意を見せなさい	
\\	色々と考え事をしていたので、降りる駅を通り越しても気が付かなかった。	
\\	映画の後半はずっと寝ていた。	
\\	駅で乗り継ぎの電車に乗り遅れないように。	
\\	昨日の夜は、皆、カラオケでのりのりだった	
\\	「車酔いに、飛行機酔い、船酔い、とにかく何でも酔っちゃうのよ。だから、乗り物に長い間乗りたくないの」。	
\\	その新聞はそのスキャンダルに関する記事を一面に載せた。	
\\	彼はのろまだ彼はきびきびしてる	
\\	「そう言うのは簡単さ。お前は関わってないからな」。	
\\	彼は上司のデスクの前に座り、ガムをかみ、マンガをのんきに読んでいた	
\\	「毎日仕事で(首を切り落とされた鶏のように)せわしなく動き回っているよ」。	
\\	午後は特に何もせずに過ごした。	
\\	君はウインドウズ派、それとも、マック派?	
\\	「10年間放送されていたお気に入りのコメディーがとうとう放送打ち切りになった」。	
\\	「敗者復活戦で勝てば、競技に復活できます」。	
\\	「彼は食べ物をほとんど残したままプレートを返した」。	
\\	「我々にはさばききれないほどの仕事の量だ」。	
\\	今日は電話あったかな	
\\	だれか入ってますか?	
\\	「目はお母さん〔似だけど、顎はお父さんだね」。	
\\	この車は何人乗り?	
\\	「テープが終わったので、番組の最後はとれなっかった」。	
\\	彼女は優しくて寛大な人だ	
\\	この靴はまだ店に出ていない	
\\	マニュアルを見ないで、この修理をしようとするな	
\\	生え際が薄くなるのはどうしようもないよ	
\\	「バリバリの阪神ベテラン選手をトレードに出すのは野村ぐらいだろうな」。	
\\	きみのところは先祖のお墓はあるのかい?	
\\	「ペニシリンは計り知れないほど人類に貢献してきた」。	
\\	「われわれのようにきちんと準備をしてきたチームにとって、彼らは相手ではない」。	
\\	「今日、彼はとてもいらついている。物事がスムーズに進まないからだ」。	
\\	この仕事では、やる気がなければどうにもならない	
\\	「吐いちゃいそうだよ」	
\\	「今日は吐き気がするんだ」。	
\\	「たんのかたまりを吐き出すかのように彼はその名前を言った」。	
\\	「彼女は怒って歯ぎしりをした」。	
\\	「彼女は、歯切れのいい話し方をする」。	
\\	「それでは、皆さん。彼に盛大な拍手をお願いします」。	
\\	我々は彼女を仕事から外し、罰として減給処分にした	
\\	彼は感動的なスピーチをした	
\\	鼻血が出てるよ	
\\	「彼は問題を回避するのがうまい」。	
\\	道から外れちゃだめだよ。危ないから	
\\	「昨日デーパートで、友達と別れ別れになってしまった」。	
\\	ここ数年、派遣社員として働いている	
\\	「彼ら派遣社員を送ってくるらしいよ」。	
\\	「彼女は甘やかされた子供だった」。	
\\	情報と技術の進歩でそれが可能になった	
\\	「『えんがわ』とは、魚のひれの付け根の軟骨と肉の部分を指す」。	
\\	車のドアに親指を挟んでしまった	
\\	「車のドアに親指を挟んじゃったんだ」。	
\\	たまたまドアに指を挟んでしまった	
\\	「彼女がまな板の上で包丁を使ってニンジンを切っている音がトントントントンしていた」。	
\\	「川の土手のそばでキャンプした」。	
\\	「こんなものくずだよ」	
\\	「そんなに興奮するなよ。脅しているわけじゃないんだから」。	
\\	「野球のコーチがチーム全員に午後ずっとグランドを何周も走らせた」。	
\\	水曜日に10キロ走が予定されている	
\\	「馬鹿な質問などない」。	
\\	「彼は若さ一杯だが、あまり知恵がない」。	
\\	警官に違反切符を切られたあとで、車がオーバーヒートしていることに気づいた。踏んだり蹴ったりだよ	
\\	自分の得意手で負けて、倍恥をかかされた	
\\	テレビコマーシャルのちょい役から始めて、ジョーンズさんはメジャー映画の主役を務めるようになった	
\\	「じゃあ、無理なんだね。だったら初めからそう言ってくれよ」。	
\\	「このドレッシング、初めから私が作ったのよ」。	
\\	新しい方ですか?	
\\	どれくらい前に、絵を描き始めたの?	
\\	「どこにも連れていけないよ。いつも困らせるようなことばかりするんだから」。	
\\	「横断歩道ではないところを横切らない方がいいよ。危険だから」。	
\\	「今日は学校のはずじゃないの」。	
\\	「とっても恥ずかしくて、穴に入って死んでしまいたい気持ちだった」。	
\\	困ったね	
\\	あー恥ずかしい	
\\	恥を知りなさい	
\\	「彼女に言おうと思ったんだけど、どうしても言えなかった(勇気が出なかった)」。	
\\	「この超高層ビルは経済の繁栄のシンボルだったが、今では邪魔物になってしまった」。	
\\	「そのドアは鍵なしでは絶対に開かないよ」。	
\\	彼が言ったことが本当であるはずがない	
\\	もうきみはこのプロジェクトをやらなくていいからね	
\\	君には、明日から別のプロジェクトをやってもらうからね	
\\	どれだけの距離を走るかではなく、走ることでどれだけ気持ちがよくなるかなんだよ	
\\	「彼ここに3時までに来るはずだったのに。どこへ行ったんだ」。	
\\	そこに9時に行くはずだったんじゃなかったの?	
\\	もらえるはずだった残業手当がもらえなかった	
\\	「母と叔母が話し始めたら、もうとにかく、気を付けないとね(止まらなくなるので)」。	
\\	僕が政治の話をし始めたら、会話が一気に盛り上がった	
\\	お父さんの誕生日プレゼントはネクタイにしよう。ネクタイなら大丈夫さ	
\\	おい、おれは雨が降るようにお願いしなかったぜ	
\\	「彼女は袖の埃を払った」。 
\\	「彼は襟の折り返しについている糸くずをブラシでとった」。	
\\	「彼は煙草を吸いすぎだね。まるで歩く煙突だよ」	
\\	「何もないところから始めた割には、よくやってるよ」。	
\\	うー、爽やかな風だね	
\\	「この町では皆裸足で歩き回っている」。	
\\	「網が張り裂けるほど魚が一杯はいっていた」。	
\\	君と僕には共通点があるよね。2人とも旅行好きでしょ	
\\	「天気が変わりやすいので、雨が降るのか、晴れるのか分からない」。	
\\	「どっちなの。はい、それとも、いいえ」。	
\\	「インターネットは、とてつもなく巨大で、しかも分散した情報源である」。	
\\	「自分で旅行をすると、随分と予期しないお金がかかるのもだ」。	
\\	ポリエステル製の服は、乾燥した天気の時にはとても静電気が発生しやすい	
\\	「この問題を別の角度から見る必要がある」。	
\\	「きっと彼は開き直ると思ったんだ」。	
\\	少人数の集まりだったが、大いに盛り上がった	
\\	その会社は将来のことを考えて解散した	
\\	「君の議論は成り立たない」。	
\\	「彼女は嫌われるなんて耐えられないんだ」。	
\\	僕の名前を発音記号で書くと、こんな感じになるよ	
\\	「危ないと感じたら、このボタンを押して知らせてください」。	
\\	私の両親はけんかはしないが、冷え切った関係にある	
\\	この新しいルールはいつから実施されるのか	
\\	我が社の新しいコンピューターは7月に店の棚に並ぶ	
\\	我々は、来春、新しい製品を世に送り出す	
\\	一年の最初の三日間が正月休みと呼ばれている	
\\	この雑誌のレイアウト、凝りすぎだよ	
\\	「彼女は目立つ服を着る」。	
\\	彼女は鼻で激しく息を吸っていた	
\\	先週、風を引いて咳がひどかったんだ	
\\	「彼は素敵だよね」。	
\\	彼女は小柄で内気に見えるかもしれないけど、怒ると早口で罵るからね	
\\	「子供達から鼻風邪がうつった」。	
\\	「うちの子供っていつも鼻垂らしてるみたい」。	
\\	鼻をほじるなよ	
\\	背中に毛が生えている男性を女性は嫌う	
\\	彼は女性のような鼻にかかった声で話した	
\\	僕の話はいいからさ、きみはどうしてたんだい?	
\\	こっちに来て、僕の話に少し付き合ってよ	
\\	「言い訳はもういい。本当の説明をしに出直して来い」。	
\\	会話の途中ですみません、緊急の質問があるのですが	
\\	「同じネタを何年も使うコメディアンもいる」。	
\\	君は聞き上手ではないね	
\\	「現金で払えるんですか。それなら話は別です。値引きしてもいいですよ」。	
\\	「彼の言うことは割り引いて聞きなよ。いつもふかすから」。	
\\	話が進むように、いくつか考えを出した	
\\	「話題を変えた方がいい。彼女に火をつけちゃだめだよ(その話をし出したら止まらない)。お願いだから」。	
\\	僕はブッシュに投票したが、だからと言って僕が保守的であるとは限らない	
\\	彼はうぬぼれている	
\\	なんて耳障りな声なんだ	
\\	「新婦の家族は皆、鼻が高かった」。	
\\	お花見って、大抵は桜の花びらが落ちてくる下で夜にやるんでしょ?	
\\	涙目になってる。アレルギーなんだな	
\\	「僕は、景色を見るというよりも食べ物を楽しむ為に旅行をするというタイプだね」。	
\\	兄は私の5才年上だ	
\\	「ゴミを出してよ。とんでもなく臭いから」。	
\\	「そう詮索するなよ」。	
\\	たった一言で僕は彼女に身のほどを思い知らされた	
\\	指を鼻の中に入れないの	
\\	「部長はマーケットが大幅に拡大したことをハリーの功績にした」。	
\\	「彼はすきっ歯だ」。	
\\	「今日は、私の髪、チリチリになっちゃってるの(縮れ毛)」。	
\\	先週、友だちとばか騒ぎをした	
\\	マーチングバンドが
\\	型でフィールドを横切った	
\\	もう少し自分のスペースが必要だ	
\\	政府は汚職まみれだ	
\\	「ずいぶん羽振りが良くなったね(出世したね)」。	
\\	「彼女は最近釣りに夢中になっている」。	
\\	岩の割れ目から雑草が伸びていた	
\\	彼の提案を飲まざるを得なかった	
\\	お酒を飲んでどんちゃん騒ぎをすることが大学生の間では大きな問題になっている	
\\	彼らは3つのパートではもって歌った	
\\	「早めに夕食にしよう」。	
\\	「参議院も衆議院も同じ様なもんだよ」。	
\\	「急ごうよ。見逃したくないんだ」	
\\	返す前に、ビデオ巻き戻した?	
\\	早合点をするなよ。話を全部聞くまで待てよ	
\\	いいかな、さっさと片づけよう。丸1日こんなことしてられないからね	
\\	「彼の話し方は早いよ」。	
\\	「話し終わるまで、口を挟まないでくれよ」。	
\\	"「彼女が部屋に入って来た時に、皆で""誕生日おめでとう""と大声で言うために待ちかまえていたが、フレッドが早まって彼女がドアを開ける前に言ってしまった」。
\\	早まったことをするな。上に言う前に、自分たちでこの問題を解決してみよう	
\\	そんなに早く電話をするなんて、早まったことをしたよね	
\\	「あの手の帽子はかなり前にすたれた」。	
\\	冷静さを失うな	
\\	「何か食べようよ。お腹がすきすぎて全然だめだ」。	
\\	「彼女に悪気はなかったんだ」。	
\\	「ナオミには気を付けた方がいいよ。今日は機嫌が悪いから」。	
\\	「何事も度を越さないようにすべきだ」。	
\\	最近の株式市場は上がったり下がったりしているみたいだね	
\\	「鍼灸師は、体中にあるいくつものツボを見つけてきた」。	
\\	'鍼灸
\\	「あいつらいつもつまらないことで衝突してる」。	
\\	「熱中するのはいいが、悪のりするのはよくない」。	
\\	「ジェイの活力と熱意はどこから来るんだろう」。	
\\	「彼らの赤ちゃんは、めちゃくちゃ可愛い」	
\\	このタイヤのグリップが気に入ってるんだ	
\\	ここからの地平線の眺めは素晴らしい	
\\	「夏(っぽい)服を着るのが好きだ」。	
\\	睡眠を十分とらないと、目が充血しちゃうんだ	
\\	「そのビルは瓦礫と化してしまった」。	
\\	「彼女は飛び降り自殺をする前に、何も書き残さなかった」。	
\\	「彼以外には誰にも言わなかったが、彼がそれを信じたかどうかはハッキリしない」	
\\	「この問題で自己反省をすることになった」。	
\\	最近は、反省しきりです	
\\	政府は高速道路建築プロジェクトのために土地を所有者から買い上げた	
\\	「昨日のインタービューでジムは私に対して丁寧な態度をとらなかった。まるっきりその逆だった」	
\\	最近、体重が増えたり減ったりしている	
\\	「彼を犯罪者と呼んではいけない。まだ、有罪になった訳ではないんだから」。	
\\	「よくバーベキューをやるのかな」。	
\\	「保険会社は、営業の人員を減らそうとしている」。	
\\	「レストランを開こうとするこの計画は、君のいい加減な事業計画の一つみたいだよね」。	
\\	「あいつは大酒のみだ」。	
\\	どれくらい来たのかな?	
\\	10代のころに比べると1.5倍の体重がある	
\\	「どうしたんだよ。はっきり言えよ」。	
\\	「その法律はある程度は役立つが、悪い面もある(両刃の刃)」。	
\\	この仕事は面白い面もあるけど、時々大変なこともある	
\\	「彼はいい人ではなかったが、子供達は立派に育った。悪いところから学んだのである」。	
\\	来週のスケジュールについて今話しましょうかね	
\\	その件に関しては今決めるのは早すぎる	
\\	もちろん、すべては相対的さ。それは僕にだってわかってる。僕はもっと具体的な答えが欲しいんだよ	
\\	傷口が細菌に感染しているようですね	
\\	「このスクーターはとてもよく売れている」。	
\\	何かばかなこと言ったかい?	
\\	なんてばかだったんだ	
\\	俺をばかにするなよ	
\\	「冗談が通じないからと言って、彼を馬鹿すべきではない」。	
\\	「インターネット、インターネットって、壊れたレコードみたいだな」。	
\\	「この本の詩には崇高なものから馬鹿げたものまでいろいろある」。	
\\	あいつは何てばかなんだ	
\\	自転車にそんなにお金をかけるなんてばかげてるよ	
\\	こんなばかげた仕事やってられないよ	
\\	「あなたの話すことって、仕事のことばかりね」。	
\\	この仕事をずっとやってるわけにはいかないだろう	
\\	結婚して10年もたつと、わくわくする気持ちはなくなる	
\\	「表向きは完璧なプランのように見えるけど、そのプラン、欠陥だらけだよ」。	
\\	「彼、ゴルフコースに出るとまるで別人だよね」。	
\\	今は話しかけないでくれ。今日は気分がよくないんだ	
\\	もと大きな冷蔵庫が欲しいんだが、スペースを取り過ぎるかな	
\\	「永福にあるあの薄汚い小さなパブ、とってもいかがわしいところだよね」。	
\\	あー忙しい、忙しい…	
\\	「人にいいことをすると、自分に返ってくるものだよ」。	
\\	「お金の問題を持ち出すなんて、場所と時を考えろよ」。	
\\	「あいつの話すことと言ったら、競馬のことばかりだ」。	
\\	これはバックアップ用のフォルダだよ	
\\	「つまんないから、出ようよ」。	
\\	「心配いらないよ。すべて順調だから」。	
\\	「彼は既に4度目の結婚をしている」。	
\\	「この本、最高だね」。	
\\	「彼は穏やかに客に売り込んで、それがうまく行った」。	
\\	彼は意地の悪いじいさんだ	
\\	「お前、のけものにされているみたいだね」。	
\\	「あの運転手酔っぱらってるんだよ。(車が)ふらふらしてるもの」。	
\\	「経理の田中さんと付き合ってること、みんなにばれちゃってるのか」。	
\\	彼女はばれないように浮気をしていた	
\\	「ばれようがばればいが、カンニングはよくない」。	
\\	秘密が漏れちゃったから。今ではみんな知っている	
\\	嘘がばれ、彼はさらに嘘をついた	
\\	負けると思われているチームが大番狂わせで勝った	
\\	「私の両親は、クイズ番組を見るのが好きです」。	
\\	「この書類、ページ順に揃えてもらえるかな」。	
\\	「来るんだったら自分の飲むビールを持って来て。それじゃ」。	
\\	ジムはルネッサンス的な教養人だからね。彼には隠れた才能が色々とある	
\\	「ジョーは(ここから見て)5列後ろの左へ3つ入った席に座っている人だよ」。	
\\	彼女は明るい性格をしている	
\\	もういい加減にしてくれよ	
\\	「どういう見方をしても、できがいいとは言えないね」。	
\\	もし歯が冷たいものや、熱いものに敏感に反応するようなら、歯医者に診てもらった方がいい	
\\	領収書を書いていただけますか?	
\\	もうすぐお正月休みだ	
\\	あなた何か隠しているような気がするわ	
\\	明かりはこんな感じでいいかな?	
\\	「偽りのコマーシャルにだまされちゃだめだよ」。	
\\	「よくないことばかりくどくど話すなよ」。	
\\	「2週間の休みがあったので、車に乗って旅に出た」。	
\\	明日の日の出は何時かな?	
\\	「会議を終える前に、日が暮れてしまった」。	
\\	ワールドカップがきっかけとなりサッカーの人気が出るといいね	
\\	「ステーキの焼き具合はどのようにいたしましょうか」。	
\\	「お前の 不平不満をここに座って聞かなくちゃらない人の気持ちも考えてみろよ」。	
\\	本日のスープは何ですか?	
\\	「病人を治療し、飢えた人たちに食料を与えたいという気持ちを彼は密かに心に抱いていた」。	
\\	一緒にいて仕事をしてもいいし、出て行ってもいいよ。それはきみ次第だ	
\\	預け入れですか、それとも、引き出しですか?	
\\	「ヘイ氏が会社の記念パーティのスピーチで上司を嘘つき呼ばわりしたために、大変なことになってしまった」。	
\\	「誰もが彼女の話に聞き惚れた」。	
\\	「彼は自分の殻に閉じこもって他の人にめったに会うことはなかった」。	
\\	「両親が離婚したため、彼は(心に)重荷を背負うことになった」。	
\\	彼女、眼鏡を取った方がずっといいよね	
\\	このドレッシングは野菜の味を引き立たせる	
\\	「彼はものすごい数の冗談のネタを持っている」。	
\\	「辞めるんだけど、僕の後任の人に引き継ぐために後2週間はここにいるよ」。	
\\	ニュースのあとにドキュメンタリー番組が続いた	
\\	「給料を上げろと皆で詰め寄った時の、社長の引きつった笑いは見物だったよ」	
\\	「緊張すると、頬がひきつるんだ」。	
\\	「君の一番のミスは、その場から逃げたことだ」。	
\\	「先週、ひどい風邪を引いた」。	
\\	例を出してくれないかな。	
\\	「あんなに一生懸命やられると、丁寧にお辞儀をしてその場から去りたくなるよね」。	
\\	「この商売の秘訣が分かるまでに何年もかかったよ」。	
\\	パーティーに行く前に、夕方になって伸びてきたひげを剃ることにしよう	
\\	「ビーチに行ったのは1年ぶりだよ」	
\\	久しぶり。どうしてた?	
\\	「この曲、懐かしいよね」。	
\\	「お兄ちゃんに後ろから膝を蹴られて、(膝が)カクッとなった」。	
\\	「その音を聞いて彼は怯えてしまい、膝がガクガク震えていた」。	
\\	「彼はとても怖くて、膝がガクガクしていた」。	
\\	ひざを交えて話そうよ	
\\	皆がエレベーターの中に身体を押し込もうとした	
\\	「いつもはどんな写真を撮ってるの。」	
\\	「なるほどね。それなら賛成するよ」。	
\\	「人が僕のうわさ話をしているのは嫌だね」。	
\\	「毎年、生え際が少しずつ後退しているみたいだ」。	
\\	「左利き用のハサミは売ってますか」。	
\\	「少しは飲みます」	
\\	「彼等の家は、私たちの家の斜め向かいにある」。	
\\	「このプロジェクトの舵取りは誰がすることになるのかな」。	
\\	「うちも商売が大変なんだ」	
\\	「彼に言おうとしたんだけど、言葉が喉に詰まってしまって」。	
\\	「それには何かしっくりこないところがある」。	
\\	「その犬は掘り始める前に、前足で地面をひっかいた」。	
\\	いくつかはひっかけ問題みたいだね	
\\	君たち二人はずっと話しているけど止めてくれないかな	
\\	いつ出ていくの?	
\\	自分の殻に閉じこもっちゃだめだよ	
\\	「コンピュータ問題を解決するために懸命に努力致します」。	
\\	華やかな暮らしがしたいとは思わなかった	
\\	バッグのひったくり犯はバイクを使うことがある	
\\	彼女は彼の顔をひっぱたいた	
\\	を取ると、色々な仕事の申し出がある	
\\	「出掛ける時は、忘れずに	
\\	「オンライン登録用紙の必要項目記入欄はすべて記入しなければなりません」。	
\\	「この種の機械を使う時には、皮の手袋は必需品だね」。	
\\	「あなたのクレジットカード番号を送る必要はありません」。	
\\	昨日の夜、あのホテルのロビーにいたことを否定するのかい?	
\\	「競争では、彼はいつも一歩先を行ってる」。	
\\	「この映画でのギネス・パルトロウは、いつもと違うよ」。	
\\	このレストランの食べ物は普通より一段上だ	
\\	「彼女は人に好かれるこつを心得ている」	
\\	「忙しくて一息つく暇もなかったんだ」。	
\\	ここからそこまで駅はいくつありますか	
\\	「大人げないことをするなよ」。	
\\	「彼みたいに人がよくちゃ、営業じゃやっていけないよ」	
\\	「その劇場は混雑していた」。	
\\	「お酒を飲むとジキルとハイドのようになる人もいる」。	
\\	「彼はアイスクリームを一口食べた」。	
\\	妻を亡くし、チャーリーは昨年は大変な思いをした	
\\	「歳をとるにつれ、ほかの人のことが疎ましくなってきた」。	
\\	昨日の夜は本当にへまなことを言ったな	
\\	きみがそう言うのは簡単さ	
\\	犯人は銀行員を人質に取って銀行にたてこもった	
\\	「他の女性に二度と目移りするようなことはないから。約束するよ」。	
\\	この仕事は簡単ではない	
\\	「人それぞれにお酒の飲み方がある」。	
\\	「忙しいのは分かるけど、きりのいいところで話がしたいんだけど」。	
\\	「コーヒーを一つもらえますか」。	
\\	「勤務時間をずらすと、会社はもっと効率的に動くようになる」。	
\\	「長い間親しく付き合って来れたのは、お互いにあまり気を使わずにいられたからだよね」。	
\\	昨夜のフェスティバルはかなりの人出だった	
\\	「一通り新聞の見出しにはざっと見ておきなさい」。	
\\	僕はあまり人付き合いがよくない	
\\	「多くのランナーがマラソンに参加したが、完走したのはほんの一握りだった」。	
\\	人がどう思っているかを考え過ぎない方がいいよ	
\\	「ジャックはいつも人のせいにしようとする。決して自分のせいにはしない」。	
\\	「相手の立場になって考えなければ、人を判断するのは簡単なことだ」。	
\\	「客観的に自分自身のことを見るのは難しい」	
\\	「見かけだけでは判断できないよ	
\\	「一晩おいて考えることにしよう」。	
\\	一晩中踊り明かした	
\\	「君の和訳は及第点を取れるが、正直に言わせてもらうと、もう少し努力をした方がいい」。	
\\	息子は、初めて人に会うと緊張するようだ	
\\	「私の第一印象ってよくないみたいよ」。	
\\	「彼の目は、彼女を見て飛び出した」	
\\	「コンピュータが目新らしい物だと思われていた時代があった」。	
\\	人の中で目立ちたくない	
\\	おい、その時計の扱いには気を付けてくれよ。安物じゃないんだから	
\\	「この都市での、毎日一人当たりの水の消費量の平均はどれくらいかな」。	
\\	「新聞が容疑者の言ったことを文脈をはしょって引用したために、彼が有罪を認めたように伝えられた」。	
\\	彼らは結婚する前に1年間同棲していた	
\\	「パーティーで彼女に引き止められて、放してくれなかった」。	
\\	失礼ながら、あなたは間違っていると思います	
\\	「人の性格を外見(服)からは判断できない」。	
\\	彼の考えはお見通しだ	
\\	そんなに自分を守ろうとするなよ	
\\	そんなふうに僕をだますのはやめてくれ	
\\	弟をいじめるのはよせ	
\\	僕をやりこめることはないだろう	
\\	きみは日焼けしてひりひりする方、それとも、小麦色に焼ける方なのかな	
\\	くるぶしを捻った	
\\	「外交官達は、土壇場で合意点を何とか見つけ出した」	
\\	1。「彼には欠点がない」。2。「彼は完璧だ」。	
\\	「彼、その内、精神的に参っちゃうよ」。	
\\	「公害は我々の生活に影響を及ぼす」。	
\\	大勢の前でステージの上に立つことを考えると身震いがするよ	
\\	この家、少しずつ自分の手で建てたんだ	
\\	「今、暇?ちょっとやって欲しい仕事があるんだけど」。	
\\	起きているときは、この論文を書かなくてはならない	
\\	そのチームは疲れた足取りで少しずつ山を登っていった	
\\	「パイプをふかしながら、チェスをしてのんびりと午後を過ごした」。	
\\	「急いで。暇をつぶす時間などないよ」。	
\\	"それとなく""ひも""と言われて彼は憤慨した
\\	ポルトガルで100ペセタショップに一度行ったことがある	
\\	「あそこの夜景は(目がくらむほど)素晴らしいよ」。	
\\	「君は短距離が得意なの、それとも長距離?」。	
\\	「あの怖い映画のことを思い出すと、冷や汗が出るんだ」	
\\	若い頃、弟をよくからかったものだ	
\\	紫外線用のローション持ってる?	
\\	へー、ビールのグラスも冷やしてあるんだ	
\\	「そんな目で見ないでくれよ。俺が何をしたって言うんだ」。	
\\	授業の評価用紙に記入をしてください	
\\	「彼女は二、三言しか言わなかったが、声の調子が多くのことを語っていた」。	
\\	「『ローリングストーン』誌の表紙に写真が載ることが、ロックンローラーの夢である	
\\	「君のコンピュータのディスプレイは液晶、それとも、ブラウン管?」。	
\\	「人生は賭さ。次に何が起こるか分からないからな」。	
\\	「その映画は期待通りではなかった」。	
\\	「ここの食べ物って彼らが言うほどおいしいのかね」。	
\\	「彼は時々ボーっとしていることがある」。	
\\	「ジェリーは表面的な(中身のない)奴だよ。自分のルックスのことしか気にしていないんだから」。	
\\	「彼は浅薄な奴だ」。	
\\	「彼らが僕を雇いたいと思っている可能性はほとんどないだろうけど、面接の2ヶ月後にその会社に電話をかけてみた」。	
\\	「ヒョンなことから、ハワイでバッタリ出会った」	
\\	「そうしたければ『だから何だったいうんだ』という態度を取ってもいいよ。免許証は戻らないけど」。	
\\	急にわかり始めてきた	
\\	その煙が目にしみる	
\\	「鼻の皮がむけてるね。ビーチかどこかに行ったのかな」。	
\\	「朝食はちゃんと食べて行きなさい」。	
\\	公園で老人がアルミ缶を集めていた	
\\	「今日の講義では、関連したさまざまなテーマについて触れます」。	
\\	彼女は砂の上にタオルを広げた	
\\	「君は風邪を引いたんだ。絶対にそうだよ」。	
\\	「とても粋なことをしたね」。	
\\	英語には8つの主要な品詞がある	
\\	「遺伝子工学によりある野菜は見かけは良くなったが、味が無くなったしまった」。	
\\	「これはよくある表現です」。	
\\	あまり頻繁には手紙を書かない	
\\	お金のかかっていない映画でも最高のできものもある	
\\	そんなに臆病になるなよ	
\\	明日何が起こるかを考えるとぞっとする	
\\	「おい、ハッキリ言えよ」	
\\	「彼は洒落た服を着てるよね」。	
\\	それはお前には関係のないことだ	
\\	「脚捻挫しちゃったから、今日はびっこをひいてるんだ」。	
\\	「皆、長所と短所を持っている」。	
\\	彼女は詳細までよく注意を払い、誠実な仕事をする	
\\	製品の価値に比べれば、価格はただみたいなものだ	
\\	「組合が今年は昇級があると言っていたが、ほとんど無いに等しい」。	
\\	きみは今、微妙な立場にいるみたいだね	
\\	建設的批判と不必要な侮辱の間には微妙な違いがある	
\\	「今年はアレルギーの方はどう?」。	
\\	われわれは皆、平等であるが、それは皆が同じであることを意味してはいない	
\\	乾燥した日には多くの静電気が発生する	
\\	「腕に急激な痛みが走った」。	
\\	「僕のアパートに泥棒が入った」。	
\\	彼は体重のことをすごく気にしているから、体重のことでからかわないほうがいいよ	
\\	「じっと座っていなさい」	
\\	「じっとしてろよ。こっちがいらいらするだろ」。	
\\	「僕の電話とファックスの番号は一緒だよ」。	
\\	「何か気になることがあるんでしょ」。	
\\	「試験の準備をしてなくて不安にならないの?」。	
\\	「身近で見たときにジェットコースターに乗るのが不安になってきた(考え直す)」。	
\\	「結婚をした後、自分たちの家(巣)を持ちたいという気持ちになって、いい家を探し始めた」。	
\\	「女性が結婚後も自分の名字を名乗ることは認められるべきでしょう」。	
\\	「たまには羽を伸ばすのもいい」	
\\	「彼女を見ろよ。今日は粋な格好をしてるぜ」。	
\\	この辺りの観光シーズンはいつですか?	
\\	私の言ったことをフォローするため、ブレンダが話に入ってきた	
\\	彼は何を言おうとしているのか?	
\\	「あの音は神経に障る」。	
\\	「なぜ人がそう言うことをするのか僕には想像ができない」。	
\\	「彼は部屋に入り、深々とお辞儀をした」。	
\\	この保険は不可抗力による被害を保障している	
\\	「葉巻を吸っている人は煙を吸い込みはしないが、ニコチンは体の中に吸収している」。	
\\	「あいつを痛い目にあわせてやる」。	
\\	「思ってることは言ってくれなくちゃ。人の心の中までは分からないんだから」。	
\\	「彼は自分の行動についていい加減な言い訳を並べたてた」。	
\\	「俺なんか負け犬さ」	
\\	「会議で彼は必死になって笑いをこらえようとしていた」。	
\\	「この新しい玩具は今年のヒット商品だね。どこへ行ってもあるもの」。	
\\	ここでは携帯を持っていないのは多分僕くらいだろう	
\\	その会議は嫌な感じで始まった	
\\	「作家として彼女は、さまざまな精神的葛藤を表現している」。	
\\	パンが[焼けて]膨れ上がり始めた時の臭いが好きだ	
\\	「元気だせよ。すねることなんてないだろ」。	
\\	ポテトチップ一袋必ず買ってきてね	
\\	「停滞した経済成長からようやく抜けだしつつある」。	
\\	彼女はいい年の取り方をした	
\\	彼、とてもいい感じに歳を取ってるわよね	
\\	希望の光を探すようにしなさい。そうすれば必ず見つかるはずです	
\\	「午後ずっと電話受付センターに電話が殺到していた」。	
\\	ふさぎ込むことなんてなんにもないじゃないか	
\\	あなたがいないことに慣れるのは、思ったより大変です	
\\	不在者投票をするの、それとも、投票所に行くのかい?	
\\	「大学時代は、真面目な学生ではなく、馬鹿なのことばかりしていた」。	
\\	「彼はパーティで女性とふざけ合う(いちゃいちゃする)のが好きだ」。	
\\	それって山田しか使わない言葉遣いだよね	
\\	明日のこの時間にはバミューダ行きの飛行機に乗ってるなんて信じられる?	
\\	「総理はスキャンダルまみれの政府について何の弁明もしなかった」。	
\\	「これってチューインガムでくっつけてるんでしょ」。	
\\	「このレシピの説明には抜け落ちているところがあるのでついていけないよ」。	
\\	「付箋がなくなっちゃったので、君に読んで欲しいページの隅(端)を折っておいたから」。	
\\	「彼の声は聞こえたが、不明瞭だったので、言っていることが一言もわからなかった」。	
\\	「この場所は趣があるよね」。	
\\	「金欠だ」	
\\	「彼はとても真面目な会社人間だったが、ある日、仕事を辞め大道芸人になってしまった。まさにどうなるかわからないということだね」。	
\\	彼女は二言目には、お金を使わないようにしなくては、と言う	
\\	「あいつ二人の女の子とバランスをとって付き合ってるんだぜ」	
\\	「一人暮らしですか」。	
\\	通路側、窓側、それとも中央の席がよろしいですか	
\\	「彼は二人分食べる」。	
\\	保証書がすべてを負担するわけではない	
\\	「病気の為に、彼のエネルギーが徐々に失せて行った」。	
\\	「彼は典型的なごますりだ」。	
\\	普段は朝食に何を食べてるの?	
\\	「なんでそんなにおめかししてるの」。	
\\	「遠近両用のメガネをかける歳になったんだな」。	
\\	「皆が楽しんでいるのを見て、パーティーに対する心配事が全て消えた」。	
\\	「腹筋何回できる?」	
\\	雨が降ってきた	
\\	「彼女に僕のことを愛しているかと聞くと、曖昧な返事をするんだよ」。	
\\	「普通でしたら、締め切りの後の申し込みは認めないのですが」。	
\\	このドレスは私には平凡すぎるわね。私、人の中で目立ちたいの	
\\	陸に上がって、普通の生活に戻りたいよ	
\\	「2週間続けて残業だよ」。	
\\	彼女の態度ってとっても嫌だよね	
\\	「長い間手紙を書かずに申し訳ない」。	
\\	「ふと彼女の顔を思い出した」。	
\\	きみがどんな気分だったか聞くことなんて思ってもみなかった	
\\	この話のアイデアはふと頭に浮かんだのかい?	
\\	「軽い食事を頼んだのに、気がついたら、彼女はフルコースの食事を持って来たんだ」。	
\\	「もしアイビーリーグを出ていないのなら、彼は君のことを無知なその他大勢の一人として見るよ」。	
\\	ここで会うなんて偶然ですね	
\\	「お前の言い分がよく分からない」	
\\	あなたの申込書は不完全でしたので、受理されませんでした	
\\	「多くの場合、明文化されていない規則が最も大切だったりする」。	
\\	「君のお母さんが病気だという知らせは先週聞いたよ」。	
\\	「彼の質問は右翼かどうかを試すものだった」。	
\\	人にいいようにあしらわれちゃだめだよ	
\\	「正しい生き方を踏み外さないように」。	
\\	「深い水たまりに足を入れ、靴が濡れてしまった」。	
\\	このパッドを何回か踏んでください。気分がよくなりますよ	
\\	この2つの考えをどう結びつけるのか僕にはわからない	
\\	「あの特大の飲み物を飲んだから、お腹がタポンタポンになってる」。	
\\	マイクは一緒にいると気楽でいいよね	
\\	息子は税金のことがあるので、私を扶養家族にしている	
\\	「インフルエンザで一週間寝込んだので、少し足元がフラフラした」。	
\\	少し頭がふらふらするので、座るよ	
\\	本屋で目的もなく本を拾い読みするのが好きなんだ	
\\	「彼女が僕の行動の邪魔をするから、ふったんだよ(別れたんだよ)」。	
\\	「大声で言えば言うほど、説得力がなくなる」。	
\\	「顔をつぶさずに、この場から手を引く方法はない」。	
\\	国民の祝日が日曜に当たる場合は、月曜日に仕事が休みになる	
\\	「今年を振り返ってみると、学ぶべき教訓がいくつかあるよね」。	
\\	勇気を奮い起こしてやっとのことで両親の家を出た	
\\	「彼は彼女の言いなりだ」。	
\\	彼、私が存在していることさえ知らないのよ	
\\	あなたに不良と付き合ってほしくないの	
\\	「何だって言うんだよ」。	
\\	「ここでは臨時雇いで働ければいい。一生勤める積もりはないから」。	
\\	「彼はオフィスにジーンズを履いていく。社長らしくないよね」。	
\\	君の会社はフレックス制なのかい?	
\\	「時々、毎日の生活があまりに自然から離れてしまっていると感じることがある」。	
\\	「このクラブでは、他では会えない人に会うチャンスがある」。	
\\	「最近は公園にあまりホームレスがいないね」。	
\\	「そのセレモニーには厳かな雰囲気があった」。	
\\	この監督は人が楽しめる雰囲気を作り出すことができる	
\\	予定より遅れてるんだ	
\\	結論も出口もないややこしい討論になり動きが取れなくなった	
\\	「仕事に対してそんな傲慢な態度を取るのはよくない」。	
\\	「まず、雨に降られて、次に、デートに遅れて、さらに、財布をなくすなんて、今日は、ついてない(何をやってもダメだ)」。	
\\	「歳なんだろうな。堪え性が全然ない」。	
\\	「かなりいい給料を出すって言ったけど、僕はもっともらうつもりだ」。	
\\	「彼等は自分たちの立場を譲らずに、給料のアップをあくまでも要求した」。	
\\	「この牛肉の霜降りすごいね」。	
\\	教授がそのようなことを言うなんてふさわしくない	
\\	「人にそんなにそっけなくしない方がいいよ」。	
\\	「チャックは僕の後輩だが、僕よりその仕事のことはよく知っているよ」。	
\\	「その部屋ぞっとするような雰囲気がしたよね。どうしてか分からないけど」。	
\\	彼は社会性に欠ける	
\\	「彼はいつも何日かひげを剃っていないような顔をしている」。	
\\	この小説は19世紀のフランスを舞台にしている	
\\	これ以上会議が続くと、僕の予定が滅茶苦茶になる	
\\	「昔とは違うよね」。	
\\	「ハッキリ言っちゃって悪いんだけど、そのヘアースタイル似合ってないよね」。	
\\	「無神経な質問ですみませんが、おいくつなんですか」。	
\\	彼は即興でコメントをするのが特にうまい	
\\	彼のくしゃみの発作は5分間絶え間なく続いた	
\\	腕に発疹ができてるね	
\\	「僕、イチゴを食べると発疹が出ちゃうんだ」。	
\\	私には若い頃のような体力がない	
\\	無難な話題について話を続けた	
\\	きみってすごいね	
\\	今夜は防風ドアを閉めた方がいいと思う	
\\	「一分間で何語打てる?」。	
\\	「カーテンを引いて、プロジェクターのスイッチを入れてくれ」。	
\\	「電車には、普通、しっかりつかまっていられるように横棒につり革がぶら下がっている」。	
\\	「今日は、一日ぶらぶらしようよ」。	
\\	ブラブラしてないで、手伝ってよ	
\\	「数時間街をぶらぶらするか」。	
\\	ブランドではなくて、質が大切なのよ	
\\	「リラックスして、くつろいでくれよ。かしこまることなんかないんだから」。	
\\	「レギュラーを10ドル分お願いします」。	
\\	これが1週間分の食費です	
\\	この飛行機は全部禁煙席ですか	
\\	「試合の終わりのほうになって彼らのチームのまとまりがなくなってしまった」。	
\\	「君は理系型かな、それとも、文系型?」。	
\\	筆無精で済みません	
\\	「多くの人が彼女の文体を真似ようとしている」。	
\\	「私が料理して、主人が洗い物をする。台所ではそういうふうに仕事を分担してるのよ」。	
\\	「急ぐように言えば言うほど、彼は遅くなる」。	
\\	「文法的には、この論文は滅茶苦茶だ(文法のミスが沢山ある)」。	
\\	「彼女は上司のことを平気で馬鹿だと言う」。	
\\	きちんと考えずによく平気でそんなことができるよな	
\\	「私、虫嫌いなの。写真を見るのがやっとよ」。	
\\	「ここには野生動物が多く生息しているので、いつ来ても20種類の鳥を見られるよ」。	
\\	「高い所が怖いんだ」。	
\\	「プレッシャーに強い人もいれば、そうでない人もいる」。	
\\	学校中に規則に従わなくてはいけないという重苦しい感じがあった	
\\	「熱っぽいんだ。熱があるかどうか計った方がいいな」。	
\\	「今夜のパーティーは正装する必要はありません」。	
\\	「自分でも自分のことを目立たないと思います」。	
\\	平和は願っているだけでは実現しない	
\\	状況が厳しいのはわかってる。でも、落ち込むなよ	
\\	こんなことでくじけるな	
\\	「傷ついたのは、他でもない、彼女のプライドだった」。	
\\	「この秋はどのような格好が流行ってるの?」。	
\\	「小さなことでくよくよしないようにしなくては」。	
\\	「どうしてもギターが弾けなかった」。	
\\	勘弁してよ。本気じゃないよな	
\\	「彼、全然平気なんだな」。	
\\	「身を入れてやらないと、使い物にならなくなるよ」。	
\\	彼のユーモアのセンスは[へどが出そうになるほど]ひどいよ	
\\	「あいつは臆病者で、まるで頼りにならないよ」。	
\\	「怠けている訳じゃないよ。調節しながら自分のペースでやってるんだよ」。	
\\	これは言い逃れできないよ	
\\	「お前の言ったこと一言も分からないよ」。	
\\	「お皿にあるものをきれいに食べるまでは、立っちゃだめよ」。	
\\	「僕は人のことをああだこうだとレッテルを貼るのは好きじゃない」。	
\\	「何を言うべきか分からない」。	
\\	ハリスはマッケンジーに接戦で敗れた	
\\	「落ち着いて。慌ててもどうしようもないよ」。	
\\	「若い内に扁桃腺を取っておいたほうがいいよ」。	
\\	「足の親指に腫れ物ができちゃって、それが今日は痛むんだ」。	
\\	静かだったパーティーが大勢が集まりうるさくなった	
\\	彼の謝罪は実は見え透いた言いわけだった	
\\	お金の問題と教育の問題は別にしよう	
\\	「彼の態度には特にガッカリしたね」。	
\\	「この件は別の機会にしょう」。	
\\	「お腹一杯だけど、アイスクリームは入るよ」。	
\\	「これは一緒でいいですか、それとも別々ですか?」。	
\\	「これ有料道路なの?」。	
\\	彼女、9歳にしては話し上手だ	
\\	「こどもを持つといろいろな経験ができて勉強になるわよね」。	
\\	「その試験に合格するにはもっと勉強をしなくては」。	
\\	あの会社は政府高官を思うように使っている	
\\	今日のお弁当は何?	
\\	「インドへ行ってお腹を壊した」。	
\\	「ホストファミリーの方たちとはいまでも連絡を取り合っています」	
\\	彼女は方向音痴なんだ	
\\	「そういうこと聞くなよ。無駄だよ。	
\\	「彼は音痴だ」	
\\	礼拝式は午前8時と11時にメインチャペルで行われます	
\\	「私の会社は、製造業の関連会社です」。	
\\	「私たちは大体において自由放任主義で子供を育てています。もちろん、いたずらをした時は別です」。	
\\	私、それほど話がうまくありませんから	
\\	「彼、誰にでも合わせることができるからね、だから、みんな彼のことが好きなんだ」	
\\	「労働市場はこれ以上外国人労働者を受け入れることはできない」。	
\\	「若い頃に留学したことがありますか」。	
\\	彼は私を捨ててほかの人に走った	
\\	「私にはあなたしかいないの」。	
\\	彼女はきみの妻だろう。どこかの年老いた女性じゃないんだから	
\\	「自己満足でくすくす笑った」。	
\\	「額のほくろを取るための手術を受けた」。	
\\	「この公園は鳥の保護区域になった」	
\\	「棚の埃を払ったら、床に掃除機をかけてね」。	
\\	その部屋のあらゆるところに埃が積もっていた	
\\	「僕のアパート埃がすぐにたまっちゃうんだ」。	
\\	この車は、このクラスでは最高の安全性を誇っている	
\\	「この店販(売代理店に)、僕が欲しくてたまらない車があるんだ」。	
\\	販
\\	「もっと強力なエンジンを積んでいるトラックが君は欲しいんだろう」。	
\\	見てるだけですから	
\\	「彼はその狭い階段を駆け上がって行った」。	
\\	少し太ったみたいだね	
\\	このトラブルの原因は何なんだ?	
\\	「自分の病気が大したことがないと聞いて、本当にほっとしたよ」。	
\\	試験に合格したと聞いて、肩の荷が下りた	
\\	彼女はとても美しかったので、通り過ぎるのを見て、皆ただ驚いて見とれていた	
\\	立ち上がった時、つま先がしびれた	
\\	彼の話はとんでもなく退屈だった	
\\	「すべてがおさまるまでじっとしていよう」。	
\\	「コーヒーを飲めば飲むほど、ますます眠れなくなるよ」。	
\\	「このレストラン、値段は高いけど、それ程おいしくない」。	
\\	「その本は、花柄の模様で飾られていた」。	
\\	それは言うまでもない	
\\	ほんと、大したことじゃないから	
\\	「お酒も程々にしろよ。体を壊すぞ」	
\\	これは冗談じゃないんだよ、いいかい	
\\	「この仕事って、無駄なことをやっているみたいだよね」。	
\\	これはつらく、割の合わない仕事だ	
\\	「昔、シェールに夢中になっていた時があったけど、今は、彼女、痩せこけちゃってね」。	
\\	スタッフを補強することに決めた	
\\	「へばちゃったよ」。	
\\	故郷を捨て、ニューヨークに行った	
\\	「彼女、ダイエットなんてする必要なにのにね。十分、痩せてるんだから」。	
\\	「昨日、君のホームページをじっくり見たけど、面白かったよ」。	
\\	「自画自賛するつもりはないけど、我ながらよくやったと思う」。	
\\	サリーのだんなさんに会ったことある?結婚相手としては申し分のない人よ	
\\	お前の過去はどうなんだい。隠してることはないのか	
\\	「よい母というのは、自分の子供の何もかもを愛し、受け入れるものである」。	
\\	「彼女の好きな音楽ってださいよね(完全に流行遅れ)」。	
\\	「彼女の目が涙できらりと光った」	
\\	僕の言いたいことはそうではありません	
\\	いつもより一本早い電車に乗ろう	
\\	「マラソンをやると言って何年にもなるが、今年、彼はとうとう本格的に練習を始めた」。	
\\	「返還〔移譲の際に、そこにいたのかい」。	
\\	この手の仕事は、気を入れてやらないとな	
\\	「彼は、状況を十分に把握していなかった」。	
\\	レポートの本文はいいんだが、参考文献や脚注には手を加える必要があるね	
\\	「腹を割って話そう」。	
\\	「単刀直入に言います。来週までにこのアパートから出ていってほしいんです」。	
\\	彼はゴメンと言ってるけど、本当かな	
\\	「お世辞はいいから、本当はどうなんだよ」。	
\\	「本当は彼女はそれがいいと思ってると思ってるんじゃないの」。	
\\	本当は2種類の
\\	が必要なんですが、あなたは例外ということにします	
\\	「正直な気持ちを言うよ」。	
\\	「彼の言葉が本音なのか、立前なのかは分からないよ」。	
\\	「私の本当の気持ちです」	
\\	「僕は本好きではないから」。	
\\	「香港は天心の本場だ」。	
\\	ここのメキシコ料理は本物だよ	
\\	「これは練習ではない。明日、大変な交渉があるのでそれに備える必要があるんだ」。	
\\	「リハーサルはこれくらいにして、本番を始めましょう」。	
\\	僕はプレッシャーに弱いんだよ	
\\	「本番のときに不安になることってあるの?」。	
\\	彼女の企業活動は国会議員として職務に反する	
\\	「それじゃごちゃごちゃじゃないか(訳が分からない)」。	
\\	「熱のせいで今日は朦朧とした状態で歩いているような感じだよ」。	
\\	「おい、しっかりしろよ」	
\\	「聞いてるの。お前、なんかゾンビみたいだよ」。	
\\	テレビの前でゴロゴロするのはやめなさい	
\\	「お返事が大変遅れまして」。	
\\	週末は、自分の仕事場でのんびりし過ごします	
\\	鐘を鳴らして、新年を迎えましょう	
\\	会うときはいつでもひげが伸びてるね	
\\	「今週は物価が上がっている」。	
\\	「彼女、気まぐれだけど、そこが魅力的なんだ」。	
\\	「なんだよ。いい警官と悪い警官のパターンかよ」。	
\\	電子メールでの応募は受け付けていませんので	
\\	才能のある人をスカウトしてるんです。興味がありましたら電話ください	
\\	彼には近寄らない方がいいわよ。色々な人に手を出してるんだから	
\\	へりの所に予備のボタンが縫いつけられているシャツもある	
\\	「君をこの議論に引き込むつもりはない」。	
\\	私の家族はこのお寺をずっと後援してきた	
\\	「慌てることないよ。時間はたっぷりあるんだから」。	
\\	「まあ、ぼちぼちってとこかな」。	
\\	「ブレントは以前のように元気には見えない」。	
\\	ポーカーをやってだまされた	
\\	「彼女は絶えず彼にガミガミ言っている」。	
\\	「新宿で酔っぱらって、タクシーに乗ったら、遠回りされて。タクシー代ぼられちゃったよ」。	
\\	「君のことを金持ちにするために、進んで助けを申し出る人など誰もいないよ」。	
\\	「彼に品がないことは話し方から分かる」。	
\\	ボロボロになるまで、お気に入りのジーンズを履いていた	
\\	なんでテーブルの上にパンのくずが散らかってるのかな?	
\\	「会社の汚い部分を公にはしたくはないものだ」。	
\\	「ボブは、僕の知っている中で一番の働き者だね。仕事の鬼だよ」。	
\\	「どうってことないよ。人が笑ったって。このネクタイ、娘がくれてんだから。しめて行こう」。	
\\	ここは魔の交差点だ[事故を起こしやすい]	
\\	いいんじゃないの	
\\	で900点取れたらな」。	
\\	今日はぼんやりしてどうしたんだい?	
\\	「あっちへ行けよ」。 
\\	「遺失物取扱所はどこですか」。	
\\	「シャトルバスは、毎時間15分おきに出ますよ」。	
\\	雪が風に揺れて舞い落ちてきた	
\\	「何でもお見通しなんだね」。	
\\	「景気が悪いからね」。	
\\	「彼には彼なりのペース/リズムがある」。	
\\	「マイレージサービスをわざわざ使うよりも、割安チケットで飛んだ方がいい」。	
\\	立ち見席はいくらなのかな?	
\\	その会議の結果は密かに初めから決められていた	
\\	「髪が目に入って邪魔なんだ」。	
\\	「前もって言っておいてくれたら、今日、空けておくことができたんだが」。	
\\	「私の席は電車の後ろの方だった」。	
\\	「その映画は(派手な)前宣伝通りに面白かった」。	
\\	前々から君は法にひっかかる/捕まると思ってたよ	
\\	「彼女は、この仕事に非常に熱心だ」。	
\\	私の上司はほとんどのプロジェクトを私の好きなようにやらせてくれる	
\\	レポートに助けが必要、それとも自分でやれる?	
\\	「このような事が起きてるのに、なぜ皆、何にもしないで見ていられるのだろう」。	
\\	「ついとんでもない約束をしてまった」。	
\\	「彼は汚いやり方でほとんどのお金を稼いでいる」。	
\\	あんな偽善的なこと聞いたことないよ	
\\	「エレベータに上司と乗り合わせて、落ち着かなかった」。	
\\	「通りを歩いていたら、抗議行動をしている人たちに囲まれてしまった」。	
\\	巻き添えによる被害はだれも望んではいないことだ	
\\	少しずつ暴力と腐敗が我々の社会に入り込んでいる	
\\	すそをつめてもらえますか	
\\	なんてついてるんだ	
\\	「潔く負けるのはたやすいことではない」。	
\\	「もっとそういう格好をすべきだよ」	
\\	「まさかこんなことになるとは思っていなかった」。	
\\	馬鹿なことを言うなよ	
\\	「そんな話でまさか誤魔化せると思っているわけ?」。	
\\	「マラソンか。君はまさにスポーツマンだね」。	
\\	彼の態度はまさにごう慢そのものだ	
\\	「彼は母親に対してコンプレックスを持っている」。	
\\	「負けるにしても、(最後まで)あきらめずに負けた方がましだ」。	
\\	「人間関係は私生活においてと同様に仕事においても重要である」。	
\\	たいした車じゃないけど、ないよりましだ	
\\	缶詰だって無いよりましだよ	
\\	これって冗談だよね 
\\	冗談だろ	
\\	「自分自身であることを忘れずに頑張れば、いいことがあるよ」。	
\\	「彼女はとても熱心に話したので、皆、耳を傾け始めた」。	
\\	彼は悩みをお酒で紛らわした	
\\	昇進してから、ジムはますます自信をつけたようだ	
\\	「彼女とのことが奥さんにばれちゃったって。それはまずいことになったな。とにかく、謝るしかないんじゃないの」。	
\\	これはまずいことになるぞ。郵便受けに税務署からの封筒が入ってる	
\\	私がまずいことを言ったので、彼らはとても怒った	
\\	「僕が話し終わるか終わらないうちに、彼は言い訳をしていた」。	
\\	まだ、その手紙を書いているのかい?	
\\	ぐずぐずするな。時計を見てみろ	
\\	投資してどれくらい利益が上がるかはまだこれからだ	
\\	「長い道のりをやってきたが、先はまだ長い」。	
\\	家が買えるのはまだまだ先だ	
\\	この先、何が起こるかわからない	
\\	「パーティの前に、みんなが待ち合わせる時間と場所を決める必要があるね」。	
\\	「たいていは自転車で学校に、いや、会社に行ってるよ」。	
\\	「君は文法の間違いしていると思うよ」。	
\\	何番にお掛けですか	
\\	彼は、間違いなく、我が国の最高の指導者だった	
\\	「もう少しのところでブレーキとアクセルを間違えるところだった」。	
\\	この時計、合ってないよね	
\\	「劇場の誰もが彼女が歌うのは今か今かと待っていた」。	
\\	待ち時間は2時間と言われ、別のレストランにした	
\\	子どもたちはプレゼントを開けるのを待ちきれない	
\\	「私が送った電子メールへの彼の返事が待ちきれない」。	
\\	「会議が終わると、いつも僕が真っ先にドアを出る」。	
\\	新しいプロジェクトに乗り出す前に、もっとリサーチをする必要がある	
\\	「スピーチを暗記していたと思ったが、群衆の前に立ったら、思い出せなかった」。	
\\	おい、本気じゃないだろうな	
\\	「展示会の芸術作品は言いようのないほどひどかった」。	
\\	彼らのお金のかかる計画に待ったをかけた	
\\	飛行機は墜落して粉々になった	
\\	まさにその通りだよ	
\\	「あのコメントは不適切(無礼)だったね」。	
\\	「彼のところは大家族だが、しっかりとした絆で結ばれている」。	
\\	あれはよくまとまった話だった	
\\	「ナイフに缶切りに栓抜き、みんないっしょになってるんだ」。	
\\	「今回の大統領同士のディベート、誰が司会するのかな」。	
\\	「彼はまだどっちつかずの状態でいる」。	
\\	「大抵は、お皿を重ねて、一気に洗っちゃうのよ」。	
\\	ポリエステル製の服には、よく静電気でまとわりつくという問題がある	
\\	切符売り場があるから、切符が手に入るか行って見てくるよ	
\\	アペックス社の交渉担当者は誰なのかな?	
\\	「話の落ちまでが長いんだよ」。	
\\	もうどうすることもできない	
\\	「いつも締め切りを守っている」	
\\	「彼はすべて規則に従ってやる	
\\	「彼はヱルヴィス・プレスリーの物まねがとてもうまい」。	
\\	「彼は、瞬く間に、魚を開き、はらわたを取り出し、それをバケツの中に入れた」。	
\\	「工場の近くに住んでいる人たちは、煙りの臭いに慣れっこになっている」。	
\\	「このレモネード酸っぱ過ぎるよ。水を入れて少し薄めたら」。	
\\	「ほとんどなにも見えないよ。まぶしくて」	
\\	「家に帰らなくちゃ。(眠くて)目開けてられないよ」。	
\\	目がショボショボして安全に運転ができなかった	
\\	この辺りでは
\\	が多く目撃されている	
\\	彼はスタンドを立てたままで自転車に乗って行った	
\\	きみが思ったようにはいかないよ	
\\	「彼は全力を傾けて仕事をし、誰よりも多くの仕事を片づける」。	
\\	「正直なのはいいが、時には君のこと無神経だと思う人もいるだろう」。	
\\	昨日の夜、上司と結婚しているというとんでもない夢を見た	
\\	「サッチーがコロンビアへ行ったって?もしそうなら、僕はアルバート・アインシュタインてことだな」。	
\\	「青のドレスにするか、緑のスラックスにするか決めかねているの」。	
\\	「実地訓練を受けたお陰で、勉強しなくてもその仕事についてすべて学べた」。	
\\	彼は彫りの深い、いい男だ	
\\	「心配いらないよ。状況はすべて掌握してるから」。	
\\	「そんな話し方をしたら、火に油を注ぐようなものだ」。	
\\	「歳を取ったからって、こだわりを捨てるなんて言い訳にならないよ。僕は違うよ」。	
\\	タートルネックのセーターを着るとのどの周りが暖かい	
\\	マルチ商法とはネズミ講の別の名前だ	
\\	「彼は一言も言わなかったが、彼の姿勢を見れば怖がっているのがはっきりと分かった(彼の姿勢がまるで怖いと叫び声が上げているようだった)」。	
\\	子どものように無邪気なのはいいが、きみはただ大人げないだけだ	
\\	「分かりにくい細かいところはいいから、全体が分かるように説明してくれよ」。	
\\	今夜は何かあっさりした物が食べたいな	
\\	「女子高校生が、痴漢の手を取り締め上げた」。	
\\	お前、ほかの会社の回し者か?	
\\	「彼の言ったことはポイントをついていた」。	
\\	「ラッシュ時って、身動きとれないよ」。	
\\	休暇を十分に楽しんでますか	
\\	「彼は、苦痛を忘れ、バハマでのんびりしている」。	
\\	コンサートのチケットは完売した	
\\	大統領の支持率はかつてないほど低い	
\\	「町を出る前に、満タンにしておくの忘れずにね」。	
\\	型にはまっているので、彼の話し方はあまりうまいとは言えない	
\\	ディック・クラークは万年青年として知られている[いつまでたっても十代のまま]	
\\	「彼は、ダークホースに賭けてて勝った、と言っていた」。	
\\	一度コンビニでチョコの棒菓子を取ったことがある	
\\	「床はがらくただらけで、カーペットがまったく見えなくなっている」。	
\\	彼女はにこっと微笑んだ	
\\	この瞬間を、僕はずっと待っていたんだ	
\\	「みんな田舎者みたいだね」	
\\	「この島は、国際日付変更線をまたがっている」。	
\\	「彼はそこにうっとりとして座っていた」。	
\\	「息をつく暇もない」。	
\\	「好きな本を読んでいて我を忘れた」。	
\\	私の身内には医者がたくさんいる	
\\	「そんな議論をふっかけてくるなよ」。	
\\	「彼らに問いつめられて、私の嘘はばれていた」	
\\	「話をした時、自分が馬鹿みたいだった」。	
\\	「最も重要な目標が見えてきている」。	
\\	「やり返してやる。あいつら後悔するぞ」。	
\\	この投資で、年間で平均してどれくらいの利益があるのかな?	
\\	「未確認の報告によると、地震による死者の数は12人ということです」。	
\\	「鑑定士が私のアンティークの時計を安物の偽造品だと言った」。	
\\	「彼はそんな年には見えないね(年より若く見える)」。	
\\	「友達だと思ってたけど。とにかく、君はどっちの味方なんだ?」。	
\\	きみは知的には洗練されているけど、社交術が身に付いていない	
\\	「この運動をするとお腹が引き締まりますよ」。	
\\	成功したいと思うなら、一生懸命やらないとね	
\\	彼女が檄を飛ばしたので皆とてもやる気になった	
\\	「もっと決断力があればいいんだけど」。	
\\	一カ月でその本全部読み切るつもりなの?	
\\	「善悪の区別が出来ない人がいる」。	
\\	「彼女と話せるのならどんなことでもするよ」。	
\\	「化学記号では下付数字がよく使われる。	
\\	「先月、注文のキャンセルが急に増えた」。	
\\	「彼はこの周りにいる人とは言うことが違う」。	
\\	「どうしてこんな制服を着なくてはならないのかわからない」。	
\\	「彼がとんでもない格好をして会議に出たので、驚いている人がいた」。	
\\	「一ヶ月もの間、男性だらけの船に乗っていたので、彼は彼女に会えてとても嬉しかった」。	
\\	お疲れ様 
\\	お見事	
\\	社長の一番いいところはとても近づきやすいということだ	
\\	彼らはつかの間だったが愛し合った	
\\	「しっかり道路を見なさい」	
\\	この無意味な議論はもうたくさんだ	
\\	現代国家では国境をまったく封鎖することはとてもできない	
\\	「おい、見かけたことのない奴がいるな」	
\\	「先週言われたこと、気にするなよ。過ぎてしまったことだから」。	
\\	きみの努力が水の泡になるのを見たくない	
\\	「平らな屋根には水がたまる傾向がある」。	
\\	「彼女はすごい美人だよね」。	
\\	「いいですよ。お代わりは。まだこれ飲んでますから」。	
\\	彼はタオルで額の汗を拭いた	
\\	「この州では、21歳からお酒が飲めると法律で定められている」。	
\\	「そこではすべてを特価で売り出している」。	
\\	「彼の顔から、ダンサーの優雅さと美しさに彼がうっとりしていることが分かった」。	
\\	「こども達はみんな一緒に遊んでいるが、あの子だけは別で、完全に無視されている」。	
\\	「もう、仕事大変でさぁ。だから今日は会社さぼっちゃおうかな、みたいな。でも、休むと明日が大変だからな」。	
\\	「変えたければチャンネル変えていいよ」。	
\\	彼らを満足させるには何が必要かね?	
\\	正直に言えば、それでいいんだよ	
\\	「このスクーター見た目とは全然違うんだから(見た目にごまかされてはいけない)。すごいパワーなんだから」。	
\\	この業界ではイメージがすべてなんだ	
\\	彼女さわやかだよね	
\\	「忙しいのは分かってるけど、そんな浮浪者みたにな格好をするこはないだろう」。	
\\	「中学教師が生徒にみだらな行為をして逮捕されたって出てたね」	
\\	みだりに誰かが話しているときに割って入ってはいけない	
\\	彼は彼女と別れ、精神的に参ってしまった	
\\	「交差点に出るまでこの道を行って下さい」。	
\\	「銀行はキング通りにあるよ。郵便局の反対側」。	
\\	この前こんなにひどい嵐になったのは2年前のことだった	
\\	「私のアパートは、エレベータの右側の2つ目です」。	
\\	「最後の車両は2つ後ろだ」。	
\\	「君は恥を知るべきだ」	
\\	言いわけをしようとするな。みっともないぞ	
\\	「この仕事につくかどうかは彼が決めればいいのさ(任せる)」。	
\\	もう仕事見つけた?	
\\	テレビを見ていたら、トーストが焦げてしまった	
\\	我々の仕事を彼が軽蔑しているのがわかる	
\\	「僕がそこにいないかのように、彼女は見て見ぬ振りをした」。	
\\	彼の作品は亡くなってから認められるようになった	
\\	キャンパスでは喫煙は認められていない	
\\	「彼は将来有望な若者だ」。	
\\	「結婚とは急いでするものではありませんから」	
\\	「彼の言ったことで、彼に対して別の見方をするようになった」。	
\\	「ジムは何でも規則通りにやるのが好きだ」。	
\\	あなたは大統領の人員整理計画に賛成だと思っていいんですか?。	
\\	このスキルは研修で身につけたのですか?	
\\	「このような仕事はよく見て覚えるようにするんだな」。	
\\	「今年は家族を喜ばせるようなことを一つもしていない」。	
\\	彼女の優しい言葉で私はとても感動した	
\\	「酒造りを身につけるには一生かかる」。	
\\	「この曲を眠りながらでも弾けるようになるまで練習したんだ」。	
\\	通帳を見ると自分がいかに浪費してきたかを痛感する	
\\	「人生色々だな」。	
\\	来週はアクションありコメディーありお見逃しなく	
\\	眺めがいいと、マンションの価格が約3割増しになる	
\\	「ハリウッドは魅力的だと思っている人もいるが、それは見かけだけだ」。	
\\	この棚の本を片っ端から読もうとしてるんだ	
\\	私立探偵がホテルを見張っていた	
\\	「真実は時に受け入れ難いものである」。	
\\	完全に耳が聞こえなくてもそれを直せることもある	
\\	「考えが小さいんじゃない」	
\\	「人の話をよく聞いているので、歳の割には分別がある」。	
\\	「爆発音の後、耳鳴りがした」。	
\\	「そのいつものパターン止めてくれないかな」。	
\\	「衝突した時の音、忘れられないよ」。	
\\	目にゴミが入った	
\\	彼女は僕の言うことにはまるで耳を貸さない	
\\	「耳に手を当てなければ、彼の声が聞こえなかった」。	
\\	ニューヨークに行って、名所を見物した	
\\	この映画は本当にアクション好きな人必見である	
\\	「クッキーが無くなったという話になると、息子は妙に静かだった」。	
\\	「この種の花はもはや野生では見られない」。	
\\	彼女はプラチナアルバムを何枚くらい出したの?	
\\	どうして何度もこの映画を見たがるんだろう	
\\	今から20年後に自分が何をしているかね?	
\\	彼の表情からして、何か心配事があるとわかった	
\\	彼、働き者みたいだね	
\\	映画見たいかい?	
\\	自分で見たら信じるよ	
\\	「彼は、酒を飲んでも彼女のことが忘れられなかった」。	
\\	落ち着いて結婚することがずっと私の人生の主目的だった	
\\	民間の活力が気回復には欠かせない	
\\	きみの行った所では、地元の人たちはどんな服装をしてるのかな?	
\\	「この仕事はまさに君にうってつけだ」	
\\	ジョゼフはこの仕事に向いているとは思えない	
\\	「見送りに空港に来てくれる?」。	
\\	「もう(飲み)始めてるのかい」。	
\\	今、新しい世紀になりました	
\\	「昔は、近所の人を信頼できたものだが」。	
\\	この伝統は昔からずっと守られてきている	
\\	流行遅れの服を着るのが好きなんだ	
\\	「私の世代の人にとっては、結婚前に一緒に住むなんてとんでもないことだった」。	
\\	古き良き時代のこと覚えてる?	
\\	「以前は5セントでコーラが買えたんだよね」。	
\\	昔はこんなじゃなかったよね	
\\	40年代を懐かしいとは思わない。ひどい時代だったから	
\\	「彼は、昔のことについて話すのが好きでね」。	
\\	どこに向かってるのかな	
\\	「舌を鳴らすのは止めてくれ」。	
\\	「あいつの態度の悪さにはむかむかするんだ」。	
\\	何も問題がないような振りをするな	
\\	「やつなんてカスみたいなもんだから無視すればいいんだよ。そのうち消えてなくなるよ」。	
\\	「これただのテレビ番組なんだから。彼等の言うことをそんなに個人的に受け取る必要はないよ」。	
\\	「僕の夢は戦闘機のパイロットになることだったが、ただ向いていなかったんだ」。	
\\	きみの窓はどっちを向いているのかな	
\\	もう少し礼儀正しくした方がいいよ	
\\	ぽっちゃりした男の子ね	
\\	「僕は熟睡するから、砲弾が破裂しても身動きさえしないだろうね」。	
\\	がんばれば報われる	
\\	「それは根拠のある指摘だが、今話している事には関係がないよ」。	
\\	「今、電話に出られませんので、発信音の後のにメッセージを入れ下さい。宜しくお願いします」。	
\\	「だらしのない格好をしていない彼をこれまでに見たことがない」	
\\	軍人だけでなく市民も地雷の犠牲者になる	
\\	彼は同じような議論を蒸し返しただけだった	
\\	二股かけたが、両方ともだめだった	
\\	このリンゴ、虫が食ってて、パサパサしてる	
\\	「このコート、防虫剤の臭いがする」。	
\\	「部長に簡単な質問をしただけなのに、怒鳴られちゃったよ。今日は虫の居所が悪いんだな(スラング:「お尻に虫がいる」の意)。	
\\	「地震が起こりそうな予感がしてたんだ」。	
\\	彼はいい所などない見下げた奴だ	
\\	この歯、詰めてもらわなくちゃ	
\\	私はどの党にも属していない	
\\	それは愚かな議論だ。まるで意味が分からない	
\\	僕はあまり神経質ではない	
\\	「昨日言ったこと撤回するんだね」。	
\\	「その船の乗り組み員は、無人島で身動きが取れなくなった」。	
\\	何人兄弟ですか?	
\\	「背中かいてくれるかな」。	
\\	「犬を買ったんだから、自分で餌をやって、散歩に連れて行かなくちゃだめだよ。お前の責任なんだからさ」。	
\\	午前中は無駄な使い走りばかりしている	
\\	彼はお金を湯水のように使う	
\\	このコースは高いけど、取るだけの価値はある	
\\	我々の努力は結局無駄だった	
\\	無駄な努力をしているように思う	
\\	「何の予告もなしに、彼は僕に向かって大声で叫び始めた」。	
\\	「今年は何回無断欠勤したんだ?」。	
\\	「これ全部を僕一人でやるってわけかい。そんなのむりだよ」。	
\\	近ごろ、仕事ですべての時間とエネルギーが取られてるよ	
\\	「スピーチをする前にはいつもあがってしまうんだ(胸がドキドキする)」。	
\\	「今日、ピクニックに行けると思ってたけどな。いい天気を期待していたのにだめだった」。	
\\	あれだけ一所懸命やってそれが無駄になったのか?	
\\	きみは僕の心配なんてどうでもいいんだね	
\\	よい作家を手本とするのと、アイデアを盗むのは別のことである	
\\	「母親に無謀にもあのような口のきき方をするなんて信じられない」。	
\\	彼女はどうしたら僕が[性的に]興奮するかをよく知っている	
\\	無理をして、その出来事を思い出さなければならなかった	
\\	ジョギングをする時には、無理をしてはいけません	
\\	「欲しくもない景品をもらったから怒ってるのか。どうして怒るのか僕にはわからないな」。	
\\	無理強いしちゃだめだよ。それは人に対するいいやり方ではないよ	
\\	「力が必ずしも正義とは限らない」。	
\\	やり過ぎはいけないよ。疲れるだけだから	
\\	彼の要求は道理に合わない	
\\	運動するときにはやり過ぎてはいけない	
\\	「あまり無理しない方がいいよ」。	
\\	どんなに想像をたくましくしても、うちの子がそんなに行儀の悪いことをするなんて思えない	
\\	「僕の上司は何でも僕に頼めると思っている」。	
\\	「無礼な方が不自然な丁寧さよりも気に障らない時がある」。	
\\	「明日の会議のために何か奇抜なアイディアを考えなくては」。	
\\	「ちゃんと映る中古のテレビを見つける目的で不要品交換会に行った」。	
\\	彼女いつも化粧がかなり濃いよね	
\\	「ご迷惑をおかけしたことをお詫びします」。	
\\	他人に迷惑をかけるべきではない	
\\	「彼のお陰で、初めて問題が見えてきたような気がするよ」	
\\	「彼女が視線をそらしたのは、そのひどい光景を見たくなかったからだ」。	
\\	君はすぐに有名ブランドに目がくらんでしまう	
\\	お前の同情を引くための話は聞きたくないよ	
\\	彼は芸術に対する素晴らしい目を持っている	
\\	「何怒ってるんだよ。そんなに目を剥いちゃって」。	
\\	「あんなことされたら、驚いて口が開いたままになっちゃうよ」。	
\\	チョコレートには目がないんだ	
\\	「メガネが曇ってるよ。拭いたら」。	
\\	「今日は目が疲れてるようだね」。	
\\	「あー忙しい」。	
\\	かつては視力は正常だったが、今は眼鏡が必要なんだ	
\\	「そんなに興奮することじゃないだろ」。	
\\	バンパーに付けたステッカーの角がまくれ上がっていた	
\\	尻込みしちゃだめだよ	
\\	彼には先のことが見通せない	
\\	「彼は一等になることを目指していた」。	
\\	「彼女はブースカグッズには目がきくので、店の棚にあろうものなら、とたんに見つけてしまう」。	
\\	「このビルは目障りだよ。傑作なんかじゃない」。	
\\	彼女の目尻のしわ、可愛いと思うよ	
\\	「ジムあまり見かけないよね。最後に見かけたのは2週間前だな」。	
\\	彼ほど頭の切れるやつはめったにいない	
\\	彼は変な奴だよね	
\\	「その変な(珍しい)シャツどこで買ったの?」。	
\\	「彼は夕食に手を付けなかった」。	
\\	「彼は目立ちたがり屋だ」。	
\\	「一月だけど、冬の天気みたいじゃないね」。	
\\	もう少し顔を見せに来なさいよ	
\\	「彼女は、めったに両親に電話を掛けない」。	
\\	それはとんでもない値段だね	
\\	「そんな嫌な目つきで見るなよ」。	
\\	ここで女性のことをじろじろ見るのはよせよ。教会だぞ	
\\	その公園はここからすぐの所にあるよ	
\\	「来週後半に結論を出すつもりでいる」。	
\\	僕にはわかるんだ。君は、いつか有名な映画俳優になるよ	
\\	「花粉症で鼻は詰まるし、涙目にはなるし」。	
\\	「目に何か入ったの?	
\\	どうしてそんなに目が充血してるの?	
\\	「部屋の空気のせいで目がしみた」。	
\\	涙を浮かべて別れの言葉を言うのは好きではない	
\\	彼女のどぎつい言葉がまだ記憶に焼き付いていた	
\\	生きている限り、この瞬間は忘れないでしょう	
\\	「あいつ俺にとっては一番嫌なやつなんだよ」。	
\\	彼のコメントは大衆の憎しみを買った	
\\	彼が私の骨董品の人形を買いたいって?私の目の黒いうちは駄目だね	
\\	「歳をとって、目の下がたるんじゃうのって嫌よね」。	
\\	「彼女とてっても可愛くて、目が飛び出しちゃったよ」。	
\\	「彼は彼女を溺愛している」	
\\	とても恥ずかしくて、どこを見ていいか分からなかった	
\\	彼の冷たい眼差しが多くのことを語っていた	
\\	「芝生の広さを知る必要がありますので、歩いて測ってみます」。	
\\	彼の経歴は目もくらむほどの業績で一杯だった	
\\	「彼女、男っぽく見えるよね」	
\\	「講義中にテープレコーダーを使うより、ノートを取ったほうがいいときもある」。	
\\	「おい、起きて目を覚ませよ」。	
\\	彼は一生懸命働いて、よく遊ぶ	
\\	メル友探してるの?	
\\	「彼らは相思相愛だ」。	
\\	今週の円の為替レートから目を離すなよ	
\\	遠くにある家をはっきりと見るために、目を細めなくてはならなかった	
\\	「僕の目は、最初から最後までスクリーンに釘付けになった」。	
\\	ほとんどの大人が水疱瘡に対して免疫がある	
\\	「この病棟の面会時間はいつですか」。	
\\	「彼女はいい男に弱い」。	
\\	彼のコメントにちょっとあっけにとられた	
\\	彼は外交特権を持っているので、起訴できない	
\\	あなたに話してるんだから、私の顔を見てよ	
\\	「いろいろ面倒をおかけしました」。	
\\	ここに来るの大変だったんだ	
\\	こんな雑用、面倒だな	
\\	「お前の泣き言はもうたくさんだよ」。	
\\	「今日、芝生も刈らなくちゃならないのか。面倒だな」。	
\\	「自分のことは自分でやる(誰にも個人的なことを心配して欲しくない)」。	
\\	子ども達の面倒を見る時が彼女は一番幸せである	
\\	ゴミ出さなくちゃいけないの?面倒くさいな	
\\	「お金はもらうよ。プライドが高いわけではないから」。	
\\	「後一段だったのに、一番上のステップでつまづいて、お皿を落としてしまった」	
\\	私の叔父は競馬で大穴を当てた	
\\	ただ今あなたのクレジットカードの申し込みのチェックをしています	
\\	今年はどのくらい就職希望者と面接をしたんだい?	
\\	書類にはもう記入したのかな?	
\\	きみには素晴らしい家族があり、いい仕事についている。申し分ないよね	
\\	「言い訳は止めて、謝れよ」。	
\\	「近い内に、家具を買わなくてわ」。	
\\	「盲腸はとっちゃったの?」。	
\\	失業したらどうしよう。そんなことを考えるのはやめよう	
\\	「彼女は違う電話番号をたくさん持っている」。	
\\	「プラスチックのボトルと袋は燃えるけど、燃えないゴミの容器に入れてね」。	
\\	プラスチックボトルは燃えるが、燃えるゴミではない	
\\	「もう、いい加減にしてよ」。	
\\	「本人がここに来た理由は何だと思う?」。	
\\	「世の中にはもぐりの弁護士がたくさんいるから、気を付けないとね」。	
\\	「あなたひょっとしてジェーン・ドーシーさん?ビルー・ドーシーさんの奥さんかしら?」。	
\\	「もしかしてあなたジョートーマスさんではありませんよね」。	
\\	「私の予感が正しければ、今日は傘が必要になるよ」。	
\\	「人の言うことを文字通りに受け取らない方がいい」。	
\\	このファイルにはたくさんの文字化けがある」。「クライアントからメールもらったんだけど。文字化けして読めないんだ。もう一度送ってもららわなきゃな」。	
\\	「この業界では5年ほどでみな燃え尽きてしまう(消耗してしまう)ようだ」。	
\\	「くずくずしている時間はないよ」。	
\\	「この計画がどのような結果をもたらすと君は思うのかね」。	
\\	「脂っこいものを食べると胃がもたれるんだ」。	
\\	「勝手なこと言ってれば」。	
\\	自分で栽培しないと野菜の本当の持ち味はわからないよ	
\\	現金どれくらい持ってる?	
\\	バットをそんなに短く持つな	
\\	「今週のタブロイド誌は映画俳優のゴシップでもちきりだ」。	
\\	彼は僕を議論に引き込もうとした	
\\	新年はどのようなこと[しきたり]をして祝うのかな?	
\\	このような取り引きにはお互いに譲り合う必要がある	
\\	「彼女はもち肌だ」。	
\\	あの格好いい人、司祭なの。なんてもったいない	
\\	テレビを割り引きなしで買ったの?なんともったいない	
\\	遠回しに言うのはやめよう	
\\	「誰かが僕の財布を持っていった」。	
\\	「この時計、45ドルだなんて、格安だったよね」。	
\\	「彼の恐怖話、怖いなんてもんじゃなかった」。	
\\	「まさかの時のためにいくらか貯えておいた方がいい」。	
\\	君ががっかりすのはもっともだよ	
\\	「彼等はいいカップルだけど、結婚が長続きするかどうか」。	
\\	水につけるとセロリとニンジンは長持ちするよ	
\\	現金自動支払い機の所へ行かなくちゃ。お金がもたないから	
\\	友だちなしではやっていけないよ	
\\	「昨日の夜、巨人は2対1で阪神にやっとのことで勝った」。	
\\	きみは大げさに言い過ぎだよ	
\\	「彼女は男に人気がある」	
\\	その賭けには絶対に負けないと言ったが、誤りだったことを認めなければならなかった	
\\	「太陽電池式の温水器は高いが、2,3年で元は取れるよ」。	
\\	位置について。用意。ドン!	
\\	「その委員会で長い議論を重ねたが、結局は出発点に戻ってしまった」。	
\\	「新たな気持ちでスタートしようと思います」	
\\	「この大学では、新入生に献身的努力と知性を求めている」。	
\\	「人前で話すとなると、手に汗かいちゃうんだよ」。	
\\	役所で書類に記入するのっていらいらするよね	
\\	うまく自分を表現できない時は、とてももどかしい	
\\	「フラストレーションがたまって、クラブを辞めることにした」。	
\\	「モーニングサービスには卵が2つにソーセージとホットケーキがつきます」。	
\\	声が震えていたので、彼が緊張していることがわかった	
\\	「単なる暗記だけでは数学は学べない。基本となる概念を理解する必要がある」。	
\\	フレッドは文学について驚くほどよく知っている	
\\	「18才の頃は、すごい肩をしてたんだけどな」。	
\\	道楽で写真をやってるだけですよ	
\\	その演劇は面白かったけど、はっきりとは言えないが、何かが欠けていた	
\\	何だかわからないけど、このスープ何かが足りないよね	
\\	それって仕事とは呼べないでしょ	
\\	「難しいジャンプだ。完璧だよ。あいつ見事にやってのけたよ」。	
\\	私は自然界に対してあまり感受性が強くない	
\\	「まあ、そういうふうに言うなら、僕も賛成してもいいけどね」。	
\\	「彼は自分の周りに武装したボディーガードを置いている。それで、自分が重要人物であるような気になっているのだ」。	
\\	「私の叔父は白内障だが、まだ、目はよく見える」。	
\\	「近頃、物を忘れることが多くなってね。精神的に年を取ったんだな」。	
\\	「君は、この会社にいて、もっと経験を積むべきだね」。	
\\	「あいつらはその出来事を秘密にしておいた」。	
\\	「この保険には、医者が勧めた場合には、鍼と指圧も含まれる」。	
\\	鍼
\\	「彼はここでもめ事を起こそうとしているんだ」。	
\\	スポーツはストレスを解消するにはいい	
\\	「完全に模様替えする時期だよね」。	
\\	居眠り運転は飲酒運転と同様に危険である	
\\	「彼は嘘泣きをしてる」。	
\\	「昨夜のパーティーは信じられないほど盛り上がった」。	
\\	「昨夜のコンサートは全然だめだった」。	
\\	その映画、派手な宣伝通りだったかい?	
\\	群衆の叫び声は熱狂的になった	
\\	失敗したら、起きあがって、埃を払い落とし、もう一度やればいい	
\\	今日はやらなくてはいけない仕事が山ほどある	
\\	「総理大臣の決断のニュースが報道関係者に漏れた」。	
\\	「彼の考えは圧倒的な同意を得た」。	
\\	どうして何か言わなかったの?	
\\	愚痴をこぼすのはよせよ	
\\	「警察はその街に11時以降外出禁止令を出した」。	
\\	テストでどの問題が一番難しかった?	
\\	「私の彼氏はとっても嫉妬深いの。他の男の人を見ただけで、もう大変なんだから」。	
\\	「先延ばしすればするほど、後になってその分だけ自己嫌悪することになるよ」。	
\\	「あなたの言った事には問題があるわね。セクハラともとれるわ」	
\\	「彼らと議論をしようとしたが、耳を貸してもらえなかった」。	
\\	「一日中会議なんて耐えられないよ」	
\\	「このボクサーは、5ラウンドの終わりに、(八百長で)ノックアウトされた振りをしたんだ」。	
\\	彼は歳を取るに連れ惚けて行った	
\\	そんなにやきもち焼くなよ	
\\	「彼の物腰、話し方すべてから独身であることが分かった」。	
\\	このことだれにも言わないって約束して	
\\	3時に人に会う約束をしてたんだ	
\\	このコルク栓抜き、多分、パーティーで役に立つよ	
\\	彼女は実物よりスクリーンのほうがずっとよく見える	
\\	「あいつはとんだ厄介者だ」。	
\\	「交通安全のお守りを持ち歩いてるんだ」。	
\\	株の暴落で彼は破滅した	
\\	彼は、やけになって犯罪に走ってしまった	
\\	「ストレスを発散するために食べない方がいいよ」。	
\\	「ここの官僚的な手続きは、とんでもなく遅い」。	
\\	激しく雨が降っている	
\\	災害を利用して利益を得るやつらはとんでもなくずるいやつらだ	
\\	「いつでも皆に優しくできるわけではない」。	
\\	「彼が優しい言葉をかけたからと言って、何かをする訳ではない(言葉だけのこと)」。	
\\	子どもたちはいじめから自分を守らなくてはいけない。それで気骨が養われるのだから	
\\	いっぱい間食するのかな?	
\\	「いいよ。何でも言って」。	
\\	「いい加減に仕事をしたので、自慢などできなかった」。	
\\	まだ休みの気分だね	
\\	「彼はプライドが高いから聞いてこないだろう」。	
\\	彼女はほっそりとしているね	
\\	あれ、やたらとでかいチーズバーガーだね	
\\	彼は見境なく話し、言うべきでないことまで言ってしまった	
\\	ショッピングシーズンの前にこの製品をなんとか出そうと一生懸命になってやっている	
\\	「多くはないけど、自分の給料で十分にやっていけます」。	
\\	これは先週放映されたショーの再放送だ	
\\	「今日は洗車してもらわなくちゃ」。	
\\	「ここまでだよ。こいつらのためにできることはすべてやったよ」。	
\\	「このチャンネル、変な番組ばかりやってる」。	
\\	「彼は努力に努力を重ね、やっと、商売で大儲けをした」。	
\\	やっとの思いでこれを売ることができたんだ	
\\	「それ提出しなくちゃいけないんだよ。当然のことだよ、ボブ」。	
\\	やっぱりビールでしょ	
\\	「彼がいないって?来ないのは前もって分かってたよな」。	
\\	「財布落としたって?別に驚くことじゃないよ」。	
\\	やっぱり彼女は勉強じゃなくてテレビを見ていたんだ	
\\	やっぱり、ここの食事はいいよね	
\\	きみならやってくれるとわかってたよ	
\\	「親は子供に八つ当たりをすべきではない」。	
\\	われわれのプロジェクトが成功するかどうかは、今はハリー次第だ	
\\	彼は精神的に健康ではないように見える	
\\	「直射日光に当たると、新聞は一日か二日で黄色くなる」。	
\\	「今謝っても遅すぎるよ。それって馬が全部逃げ出した後で納屋の戸をしめるようなもんだよ(手遅れ)」。	
\\	やっぱり、バリーはまた遅刻だ	
\\	「偽医者にはご用心」。	
\\	「盲滅法に物事を急いてやるべきではない」。	
\\	「他社と取り引きをしないと言う条件ならば協力致しましょう」。	
\\	乗馬したことある?	
\\	広告キャンペーンによる一般大衆の支援を得ようとするその計画は失敗に終わった	
\\	「彼のつま先が靴下の穴から覗いていた」。	
\\	「やり残した仕事を片づけなくちゃならないんだ」。	
\\	「(相手からやんや言われた時に)かわすことも必要だよ」。	
\\	大ざっぱに言うと、だいたい摂氏を倍して29を足すと華氏になる	
\\	「そうすべきじゃないよ。危険すぎる」。	
\\	「絶えず机をきれいにしておかないと、整理がつかなくなって、書類がたまっちゃうよ」。	
\\	筋の通らない議論ははっきりわかるんだよ	
\\	有給休暇はどれくらいあるのかな?	
\\	いない間に、郵便と新聞がたまってしまった	
\\	苦戦したが、ここからは楽勝だよ	
\\	「少しずつ取らないと、お皿のスペースがなくなるよ」。	
\\	そのパーティーに行けるならなんでもすするよ。	
\\	今夜はこれくらいにしておこう	
\\	今あきらめちゃだめだよ。タオルを投げる/降参するには早すぎるよ	
\\	「社長は責任を取って辞めることにしたが、退職金を貰い、運転手付きのリムジンは手放さなかった」。	
\\	「そのドキュメンタリーの制作者は、その番組のために嘘の場面を撮ることを拒否した」。	
\\	「これは大変な仕事だよ。でも、打ち込んでやれる仕事が好きなんだ」。	
\\	「彼はその仕事に全力投球した」。	
\\	「何とかお金を捻出しなくては」。	
\\	母はお金がほとんどなくてもなんとかやりくりする達人だった	
\\	やり過ぎなければ、運動するのもいいよ	
\\	「子供の教育に興味を持つ必要はあるけど、度が過ぎるのはよくないよ」。	
\\	思う存分食べていいよ	
\\	自分の人生で何がしたいか決まったかい?	
\\	「子供にやりたい放題にさせて、彼女、一体何考えてるのかしらね」。	
\\	「うっぷんをはらさなければ、爆発しそうだ」。	
\\	「彼女は、他の誰もがやりたがらない大変な仕事に積極的に取り組む」。	
\\	生きる気力をなくすことは、死ぬことよりも悲しいことである	
\\	「試験に受かるかどうかは君の努力次第だね」。	
\\	「派遣社員のやる気を上げる必要がある」。	
\\	「彼は結婚相手としては最高だよ」。	
\\	「出来る限りのことはしたが、十分ではなかった」。	
\\	今夜は映画を見に行こうと思っていた	
\\	「面倒を起こさないでね」。	
\\	「そんなに落ち込むなよ。大丈夫だよ」。	
\\	「遠回しの話はよしてくれ。本題に入ろうじゃないか」。	
\\	昇給を求めるにはガッツがいるよね	
\\	痩せることで自信が持てるようになった	
\\	「今年の有給休暇は使い果たしたので、これからは休みを取ると無給休暇になる」。	
\\	「勇気があれば彼女に本当のことを言うのだが」。	
\\	「この無駄な道路工事はなんだ。税金を(無駄に)使いやがって」。	
\\	26周年記念セールをお見逃しなく!	
\\	「彼はチャンピオンの最有力候補だ」。	
\\	優柔不断が君の最大の問題だ	
\\	「最初にしてほしい大事なこと。車を出す前に、シートベルト締めてね」。	
\\	忙しくて子どもと遊ぶ時間がないって言うなら、何を優先すべきか、ちゃんと考える必要があるわね	
\\	「今日しなくてはいけない仕事のうち、どれを優先させるかな」。	
\\	「彼は念力でいろいろなことができると言っている」。	
\\	「彼の仕事に対する態度は柔軟でない」。	
\\	「彼は何でも(型教科書)どおりにやる」。	
\\	「あんな風に怒りを爆発させるのはよくないな。立場が不利になるだけだ」。	
\\	「この数年で身につけたことは君のためになるよ」。	
\\	有料道路は通らないようにしよう	
\\	「ホンダかハーレーのどちらかを選ぶのは、リンゴとオレンジを比べるようなものだ」。	
\\	「この仕事は嫌だけど、給料と手当が魅力的だったんだよ」。	
\\	彼女は面白半分に、プードルにチュチュを履かせた※チュチュはバレー用の短いスカート	
\\	「お風呂のお湯が熱すぎる」。	
\\	お風呂の後はタオル地のバスローブを着るのがいいですね	
\\	この家にはセントラルヒーティングがついてるだけど、ほとんど使わない	
\\	「シカゴ行きのバスに飛び乗った」。	
\\	雪かき用のシャベルがあったら貸してもらえるかな	
\\	私の叔父は雪色を描くのが好きです	
\\	「先週は斜面に積もったばかりのパウダースノーは全然なかったね」。	
\\	このウイスキーはちびちび飲むもんだね	
\\	あなたの血液型は?	
\\	「バスの中で、妊娠している女性に席を譲った」。	
\\	同意させようと思っても時間の無駄だよ	
\\	値段の交渉をしようとしたが、彼らは価格を変えなかった	
\\	「お金が物を言うのさ」。	
\\	氷枕を額にのせると気分がよくなるよ	
\\	気を抜く余裕はない	
\\	「誰がそんなにゆっくりしていいって言ったんだ」。	
\\	「ゆっくり楽しんでください」。	
\\	週末ゆっくりして頭を冷やそうと思う	
\\	「ゆっくりやれよ。急がなくていいんだから」。	
\\	思う存分休んだらいい	
\\	「このスパゲッティひどいね。ゆで過ぎだよ」。	
\\	「彼には自分で新しい会社へ資金が提供できるほどの財政的ゆとりがある」。	
\\	今の仕事はあまり余裕がないね	
\\	「彼の短い指では、ピアノを弾くのは大変だろう」。	
\\	「彼は二度と盗みをしないと誓った」。	
\\	立つとひざがポキって鳴るんだ	
\\	「君の顎、噛むとポキポキ音がするね」。	
\\	「地に足がついていない夢を見た」。	
\\	「この大学に入学が許可されるのが私の長年の夢のようなものだ」。	
\\	「オペラ歌手になるという私の夢は、まさに夢そのもの」。	
\\	「ロンドンで飛行機から降りた時、以前に来たことがあるような感じがした」。	
\\	「地震が収まった後も、少しの間、天井からぶら下がっている蛍光灯が揺れていた」。	
\\	シャツがきつすぎる 
\\	この靴、きつすぎるわ	
\\	「校外見学へ参加を希望する学生は、まず、両親から許可をもらわなければならない」。	
\\	「ここ取っておいてね。トイレに行って来るから」。	
\\	地震の後、ビルがゆっくり揺れるのを感じた	
\\	「党派間での議論ばかりで、連合政権は完全に無能になっていた」。	
\\	「毎晩カラオケに行くの止めようかな」	
\\	彼女は泥酔した	
\\	「このソフトウエアは大企業用のもので、個人用ではない」。	
\\	ここの蚊はものすごいね/容赦ないね	
\\	絶望的状況では強雨な対策が必要となる。	
\\	「彼はビジネスに対しては断固とした姿勢を持っている」。	
\\	ピーと鳴ったら、名前とメッセージを言ってください。折り返し連絡いたします	
\\	「肉体的に子供を虐待するのは決して許されるべきではない」	
\\	急ぐことはないでしょう。会ったばかりなんですから	
\\	この映画の筋は幼稚だ	
\\	「腰痛がするときは、機嫌が悪くなるんだ」。	
\\	きみはどのようなスローガンを考えてるのかな?	
\\	その手紙はだれから来たのかわからないように書かれていた	
\\	「金曜日は映画の日にしてるから、毎週、映画を見てるよ」。	
\\	何曜日かわかるけど、何日だっけ?	
\\	「水曜日を木曜日だと思って一日過ごした」。	
\\	「そのハンドバッグいいわね。あなたの服によく合ってるわ」。	
\\	「やつは、要領もいいし、仕事も細かいよね」	
\\	「この仕事のやり方のこつが掴めてきましたか」。	
\\	「その議論はここには当てはまらない」。	
\\	「彼以外の人が手伝いに来てくれるといいんだけどな」。	
\\	よかったね	
\\	もう謝っただろう。これ以上何を言えっていうんだ	
\\	僕でよかったら、手伝うよ	
\\	「君のためにしたんだよ。怒ることないだろ」。	
\\	第1作よりも続編がいいことはめったにない	
\\	「彼は物覚えが早い」。	
\\	「そのレポートよく書けてるよ」。	
\\	よく言うよ	
\\	驚くことはないよ。こういうことってよくあるから	
\\	「一体、何様だと思ってるんだ」。	
\\	「すべて君のお陰だよ」。	
\\	よく見ると、彼のスーツはきれいだが、古くすり切れているのがわかった	
\\	そんな探し方をしていたら絶対に見つからないよ	
\\	ばかな質問をすれば、ばかな答えが返ってくる	
\\	「よくあんな男とデートできるわね」。	
\\	「彼の本はとてもよく書けている」。	
\\	「その風邪が良くなるまでに2週間かかった」。	
\\	皆、もともとは欲張りなんだと思うよ	
\\	彼女のこと知ってると思ったが、もう一度見たら知らない人だった	
\\	偏った見方をするなよ	
\\	「よくもまあそんなことをするなんて考えるよな」。	
\\	「よくやったよ」。	
\\	「よくやるよな。自分でこの船作ったのかい」。	
\\	この決定について、きみによく考えてほしいんだ	
\\	「お互いの要求を言っていたらきりがありません。この当たりで妥協しませんか」	
\\	「周り(世の中)がちゃんとしてれば、こんな議論しなくて済むのに」。	
\\	「君は少しかしこまり過ぎだよ」。	
\\	「あんなこと全く言う必要なかったのに」。	
\\	人を批判するときに、彼はやりすぎる傾向がある	
\\	余計なことを言っちゃったかな?	
\\	パーティーに行くのに素敵なプレゼントを買ったんだから、それ以上のことをする必要はないわよ	
\\	ここ数年でお尻がずいぶん大きくなった	
\\	彼女はよく話すので、口を挟むことさえできなかった	
\\	「済みません。通してくれますか」。	
\\	このワックスは、車がほこりっぽくなるのを防ぐ働きがあります	
\\	悪役ばかりやるのにはうんざりだ	
\\	「彼は、いいサッカーの試合ほど楽しいものはないと言うが、僕には全然分からない」。	
\\	「今年の旅行費はいくらくらいなの?」。	
\\	浮気は諸刃の刃である。人にそうしたように、それはいとも簡単に自分に対しても起きうるからである	
\\	「彼女は人に心を許さない」。	
\\	予選はどうだったの?	
\\	この仕事にこんなに時間がかかるとは思ってもみなかった	
\\	「土曜日のレースでは、多分、彼女が勝つよ」。	
\\	「彼の勝利は予想通りだった」。	
\\	この料理はおいしそうな臭いがするね	
\\	「飛び出した目を引っ込めて、よだれを垂らすのは止めろよ。彼女、結婚してるんだから」。	
\\	この仕事ストレスたまるよな	
\\	身体のあちこちが少し痛む	
\\	「その仕事を終えるのにどれくらい時間が掛かるかは分からないな。状況によるよね」。	
\\	1。「さあ、前に来て」。2。「さあ、こっち」。3。「よってらっしゃい、みてらっしゃい。ほらほら、そこのお嬢さん。ちょっと足を止めて、見てってよ」。	
\\	「途中で出て来ちゃうなんて、その映画、よっぽど退屈だったんだね」。	
\\	彼女に電話して、しかりつけるつもりだ	
\\	「明日の予定は何かな」。	
\\	こんな緊急事態に対する予算は組めなかった	
\\	詳細はまだ決まっていないが、ビジネスプランはかたまり始めている	
\\	近ごろは、予定通りに進めるのがとても大変だ	
\\	火曜日に予定入ってる?	
\\	「もっと自然に微妙なとことまで英語で言えるようになりたい」。	
\\	世の中は大変なんだから、お前もしっかりしないとな	
\\	彼女は世の中のことを何も知らない汚れのない少女だ	
\\	スキャンダルのあと、何人かの高官が辞任した	
\\	僕たちのことは言いふらすなよ。秘密にしておきたいんだから	
\\	「この電球は普通のソケットにつけられるけど、白熱灯なんだ」。	
\\	「ポップな曲を携帯の着メロにした」。	
\\	「ごめん、今夜はパーティーに行けないんだ。別の機会に呼んでくれるかな」。	
\\	「プリッツェルを一本だけ食べるなんてとても無理だよ(何本も食べたくなる)」。	
\\	「今日は集中できないな」。	
\\	「泥棒呼ばわりするとは、何と無礼な」。	
\\	「君の文には、余分な動詞がある」。	
\\	君の猫は、もう予防接種は済ませたのかな	
\\	「疲れていることを周りの人に言うことで、彼は、酒の誘いをうまくかわそうとしている」	
\\	彼は人の心を読みとるのがうまい	
\\	パニクルなよ。世の中が終わるわけじゃあるまいし	
\\	「すみません。ここは予約できないんですよ」。	
\\	「こんなの余裕でできるよ」。	
\\	「彼急いでいるようには見えないけど(余裕だね)」。	
\\	今週末にバーベキューするんだけど来る余裕がある?	
\\	もっとゆとりがあるといいのだが	
\\	「与えた被害を隠すために、彼らは時間を稼ごうとしている」。	
\\	「万が一のことを考えて、お金を少しとっておこう」。	
\\	「彼の歴史の知識は、ほとんどの人よりもはるかに上だ」。	
\\	「彼は財布をなくしさらにカッカした」。	
\\	「そこへは電車ではなく飛行機で行ったほうがずっといい」。	
\\	主人と仲直りしたの	
\\	そんなことで喜ぶべきじゃないよ	
\\	「はじめまして。一緒に素晴らしい仕事ができるといいですね」。	
\\	今後ともよろしくお願いします	
\\	よろしくお願いします[あなたと仕事をするのを楽しみにしています]	
\\	「今回助けてくれたら、一生恩に着るよ」。	
\\	「俺、ショートヘアーの女の子に弱いんだ」	
\\	ジャックを怖がることはないよ。あいつは口だけだから	
\\	「意気地なしに見られたくない」。	
\\	「すぐにカッとなるなよな」。	
\\	「あいつは賢い生き方をしてる」。	
\\	おい、泣き言を言うのはよせよ	
\\	いつになったらこの雨やむんだ	
\\	「誰にでもとこかに弱点はあるよ」。	
\\	私の祖父はもうすぐ80だが、とても健康だ	
\\	「五年前の今日、結婚したんです」。	
\\	学生は楽だよね	
\\	一度犯罪者の烙印を押されたら、決して仕事には就けない	
\\	ただの物などない	
\\	「昨夜の試合は、ジャイアンツの楽勝だった」。	
\\	彼は投票で辞めさせられた	
\\	「さっさと片づけて、早く仕事終わりにしよう」。	
\\	「そのインテリアデザインは教会っぽい雰囲気だった」。	
\\	あの2人、結婚間近なんだって	
\\	この会話は堂々巡りをしているだけだ	
\\	「そんなこと言い続けてたら、同意できませんよ」。	
\\	「今週の相場は上がり気味だね」。	
\\	明日、仕事の打ち合わせをするためにランチでもどうですか	
\\	彼に夢中になってなんかいないなんて、まさか。あなたのことよくわかってるんだから	
\\	「あの判事にまったく利害関係が無かったとは思えない」。	
\\	「彼には人を許すだけの包容力がない」。	
\\	「それはすばらしい理論(意見)だが、ここで我々が扱っているのは現実の世界のことなんだよ」。	
\\	「彼は口だけで、行動が伴わない」。	
\\	「利子がとても低いので、銀行に預けても、コーヒーの缶にお金を貯めておくのとかわりがない	
\\	危険を犯さなければ、得るものは1つもない	
\\	「今年リストラされなければいいけど」。	
\\	きみにはリズム感があるのかな?	
\\	君が歌っている間、僕がリズムをとるから	
\\	「できることなら(理想を言わせてもらうと)、洗車をしてからガソリンを満タンにして車を返してほしいね」。	
\\	ジョーンズ氏は市長に立候補するとそれとなく言った	
\\	「彼、丸くなったよね」。	
\\	「彼女は訳の分からないことは言わないよ」。	
\\	「一ヶ月ダイエットしてたんだけど、元に戻っちゃったわ」。	
\\	彼女は髪に赤いリボンをつけた	
\\	「高校時代にドイツでホームステイした」。	
\\	「ある大学では外国の学生に英語の補習授業をしている」。	
\\	「いい環境だね。いかしたおねーちゃんに挟まれて」	
\\	「この機械では片面しかコピーできない」。	
\\	彼は人を利用するタイプの男だ	
\\	調査をすることは私の権限の範囲外である	
\\	電気代は使用量に基づいている	
\\	「物事はどうなるか分からないからね。だから理論と実践という別々のことばがあるんだ」。	
\\	即興で何かをするのは得意じゃない	
\\	「我が社では、毎月、方針に関する稟議書を回し、各部長がそれに同意したことを示すために署名することになっている」。	
\\	「一時しのぎの対策を講じるのではなく、きっぱりとその問題を解決すべきだ」。	
\\	「彼の
\\	小説は、遙か彼方の世界や遠く離れた銀河系に連れて行ってくれる」。	
\\	遙
\\	彼は年齢以上の威厳をもって話をした	
\\	デービスさんはただいまおりません。伝言がありましたらどうぞ	
\\	家を出るときに、留守電にするのを忘れた	
\\	「こちらはジム・ロビンソンです。ただ今、電話に出られません。発信音の後にメッセージを残してください」。	
\\	「彼は午後ずっとボーっとして歩き回っている」。	
\\	「始める前に、きちっと礼をしなさい」。	
\\	「彼女は霊感が強いから、君の運勢が分かるんだよ」。	
\\	彼の態度は、いつも丁寧で礼儀正しい	
\\	「彼は冷静な態度で話した」。	
\\	最近、判断力が鈍ったね	
\\	いつもの夏なら、どのくらい雨が降るんですか?	
\\	「ドレスアップして、決まってるね」。	
\\	今回のイベントには黒のネクタイ着用でお願いいたします	
\\	彼の過去のこと知ってたら、あなたは彼とは付き合わないでしょうね	
\\	「その仕事の質は、評価されなかった」。	
\\	「中に入りたいのなら、列に並んで」。	
\\	「これおいしいね。これ作ったの、それとも、インスタント食品?」。	
\\	「今週末は3連休だ」。	
\\	そのスキャンダルの波及効果で、実業界に不安が広がった	
\\	「そのセミナーは4日連続で行われる」。	
\\	「巨人がまた負けて、これで3連敗」。	
\\	連絡するよ	
\\	「緊急時の為の電話連絡網を作りました」。	
\\	「遠近両用メガネに慣れるのが大変だった」。	
\\	「何年も仕事がなかったが、管理人の仕事を見つけてた」。	
\\	彼は何年も失業している	
\\	"「大学の入試に失敗した学生は、時には、1年、あるいは、2年、中間の状態で次の試験に合格するために勉強をする。そのような学生は""浪人
\\	仕える主のいない侍 
\\	と呼ばれる」。
\\	「親切から彼にはそう言ったんだが」。	
\\	あなたのこれまでの功績に対する感謝のしるしの金時計です。どうぞお受け取りください	
\\	「このビデオ、テレビから録画したの?」。	
\\	「プレイと録画ボタンを一緒に押さないと録画(録音)できないよ」。	
\\	彼女は悔いを残すことなく死んだ	
\\	このカタログには大した物は載ってない	
\\	温泉で週末を過ごした	
\\	「彼は酔っていて、話すことさえできなかった」	
\\	「それは聞かなかったことにするよ	
\\	彼はどっちについてるんだ	
\\	「その事件を表沙汰にしないように試みたが、裏目に出て、かえって大きな騒動を引き起こすことになった」。	
\\	「やっと君の言うことがわかってきたよ」。	
\\	失敗をしても恥ずかしいことはない。それから学ぶんだったら	
\\	痩せたらずっと若く見えるよね。	
\\	自分の責任を逃れようとするな。お前も悪いんだから	
\\	彼の言ったことはとても明確に思えたが、よく考えてみると、まったくはっきりしていなかった。	
\\	分かっていながら、どうして家族を傷つけるようなことができるのかね。	
\\	その件に関して、彼がわかってくれてうれしい	
\\	これは簡単で使いやすい商品です。	
\\	この学校の多くの学生と話してきたが、皆、素晴らしいね。	
\\	行かなくちゃ。じゃあね	
\\	彼らはずっと前に別れたんだよ	
\\	彼女はいつまでも10代のままだ	
\\	ここっていいよね。いつでも歓迎されてるって気になるものね。	
\\	その問題は、表面化する前に、長い間くすぶっていたんだ	
\\	人につけ込まれはしないよ	
\\	この問題に関してドリスに助けを求めてはいけないくらい、君にも分かっているだろう。	
\\	株を発行している会社の数が増えている。	
\\	お湯が沸いてるよ	
\\	彼女の気持ちを傷つけようなんて思ってもいないよ。	
\\	どういうわけで疑ってるんですか	
\\	訳のわからないことを言うのはやめてくれ	
\\	波が怖いからではなくて、水か冷たいからビーチでは泳がないんだ。	
\\	彼に借りなんかないよ。君の命を救ったわけではないんだから。	
\\	学校は休めないよ。そういうわけにはいかないよ。	
\\	チーズバーガーが食べたいだけなんだ。なぜかわからないけど。	
\\	彼は芝居ががかった話し方をしたので、彼が率直に意見を言っているとは誰も思わなかった。	
\\	戸口のところでローラースケートをするなんて、事故が起きるのを待っているようなものだ。	
\\	こんなことまでしなくてもよかったのに。ありがとう。	
\\	わたしのためにわざわざすみませんね	
\\	映画の一番静かな場面でくしゃみをする必要があったわけ?。	
\\	彼女の名前何だっけ。忘れちゃったよ。	
\\	彼のことはいつまでも忘れられない	
\\	この仕事はとても面倒なことになりそうだね	
\\	これとても香りがきつい香水だから、耳の裏に軽くつけるだけにしてね。	
\\	私と残りの人生を一緒に過ごしたいと思っているの。	
\\	私にしてみれば、この仕事はとんでもない詐欺だ。	
\\	あなたは、私にとってすべてなの	
\\	頑張ってね。私のためにホームラン打ってね	
\\	釣り楽しんできてね。私に一匹釣ってきて	
\\	時には、私のことも考えてよ。	
\\	警察官に運転免許書を渡した	
\\	私は1988年にヨーロッパから渡ってきた	
\\	このキャンデーは、派手に宣伝してるわりにはたいしたことないね。	
\\	スポーツは会話のホットな話題である	
\\	これは環状線で、内側が時計回り、外側が反時計回りになっています	
\\	それは東洋と西洋のいいところ混ぜ合わせている。	
\\	彼は荒稼ぎをしている。	
\\	彼が冗談を言って、みな大笑いした。	
\\	あいつらのやってることって馬鹿馬鹿しいよね。	
\\	彼は素敵な笑顔をしている	
\\	君といるといつもニコニコしていられるよ。	
\\	彼にいつ仕事が終わるのか聞いてみたんだが、冗談を言って、話題を換えられてしまった。	
\\	今の状況を現実的に見なくては。後悔するような感情的な決断をしてはいけない	
\\	彼女がどういう女の子かを頭に入れてから、彼女と付き合うんだね。	
\\	夢を見るのもいいが、現実的になって仕事もしてくれよな。	
\\	すみませんが、おトイレに行く間、僕の場所をとっておいてもらえますか	
\\	どのような取引をするつもりでいるのかね	
\\	科学者のわりには、彼あまり頭が良さそうに見えないね。	
\\	年齢のことを考えると、彼、走るの速いよね。	
\\	お前って、ひどいな	
\\	僕が一生懸命やったことが、すべて無駄になった	
\\	悪いんだけど、この書類を印刷してくれるかな。	
\\	悪いんだけど、金曜日はだめだな。	
\\	あなたについてはいいことばかりを聞いています。	
\\	僕だったらその仕事はしないな。いい考えじゃないよ	
\\	すみません。電話で無愛想な話し方をしてしまったようです。失礼致しました。	
\\	怒らないでくれよ、ね。	
\\	彼は泥棒なんだ。被害者として描いちゃだめだ	
\\	最悪の場合のシナリオは、僕がこの試験に落ちて、もう一度受けるってことだな。	
\\	僕が悪者みたいな言い方をするなよ	
\\	昨日の夜は、悪酔いした。	
\\	分かったよ、お前は何でも知ってるんだろ。それで今度はどうすればいいんだい。	
\\	彼はかつて小銭のために街で歌ってたんだ。今じゃ、ラスベガスでは人気歌手の一人なんだから。それが人生ってもんさ。	
\\	彼は本当に腰が重いよ。	
\\	作家としては、彼は他の人たちより1枚上だ	
\\	お酒弱いんですよ。	
\\	あいつお酒強いよ。	
\\	酔っぱらうと絡むからあいつとは飲みたくないんだ。	
\\	お酒は飲まないんですけど、雰囲気が好きなんでよく仲間と飲み屋には行きますよ。	
\\	あいつは笑い上戸だから。	
\\	彼女は泣き上戸だから。	
\\	今日の宴会は無礼講ということで。みんな、パーッとやりましょう。	
\\	遅れて来たんだから、駆けつけ三杯だな。	
\\	もうみんな若くないんだから、イッキはやめようよ。	
\\	とりあえずビールにしよう。	
\\	ビールは最初の一杯だね。後は惰性。	
\\	何か適当なものを見繕って出してよ。	
\\	もう一軒飲みに行こうよ。まだ、終電までは時間があるんだから。	
\\	あいつまた酔った振りして女子社員の体さわってる。そのうちセクハラで訴えられるぞ。	
\\	飲み過ぎたかな。気分が悪くなってきた。	
\\	今夜も飲み会に来られないの。君、つき合い悪いね。	
\\	あまり飲んでもないし、ろくな物も食べてないのに、割り勘かよ。	
\\	この店はこれくらいにして、別の所へ行きましょう。	
\\	いい気分で酒飲んでたら、終電の時間が過ぎてしまった。今夜はビジネスホテル泊まりだな。	
\\	今夜もまた午前様か。かみさんが怒っている顔が目に浮かぶよ。	
\\	酒飲んだんだから、車運転するのは止めろよ。事故ったら取り返しがつかないぞ。	
\\	今夜は酒を飲んで、憂さ晴らしでもするか。	
\\	温泉に入って、一杯やりたい気分だね。	
\\	男だけで飲んでも少しも楽しくないよ。	
\\	昨日飲み過ぎて、どうやって家に帰ったのか全然覚えていないんだ。	
\\	飲むと人間が変わる奴っているよな。	
\\	最近、お酒を飲むと涙もろくなちゃって。	
\\	僕は日本酒党なんですよ。	
\\	今夜は僕がおごるから、付き合えよ。	
\\	あいつは酒飲むとすぐに寝ちゃうんだ。	
\\	空きっ腹にお酒を飲むのは胃に良くないよ。	
\\	夏はやっぱりビールだよね。	
\\	うー寒。熱燗で一杯やっていくか。	
\\	昨日は3軒もはしごしちゃって、結局、家に帰れずに友達のところに泊めてもらったんだ。	
\\	何事も「程良く」がよい。飲み過ぎは体によくない。	
\\	日本酒はどんな料理にも合うね。	
\\	魚料理に赤ワインってことはないだろ。やっぱり白だよ。	
\\	食前酒には何を飲もうか?	
\\	この酒、なかなか手に入らない幻の名酒だよ。	
\\	ノンアルコールのビールありますか?	
\\	ちゃんぽんで飲むと悪酔いするよ。	
\\	酒は百薬の長って言うだろ。もう一杯いこう。	
\\	酒の飲み過ぎは万病のもとだよ。	
\\	お酒は楽しく飲むものです。	
\\	宗教で禁じられているのでお酒は飲めません。	
\\	昨晩飲みすぎて、ひどい二日酔いだ。	
\\	このお酒強いね。胃に火がついたみたいだ。	
\\	一杯やって、パーっと憂さばらししよう。	
\\	デートに居酒屋?ムードなさすぎだよ。	
\\	顔が真っ赤だよ。もうやめといたほうがいい。	
\\	酒は飲んでも飲まれるな。	
\\	一般的に、日本人は白人よりお酒に弱い人が多い。	
\\	アメリカでは、未成年の飲酒は厳しく禁じられている。	
\\	春先は新入生歓迎会などがあるので、急性アルコール中毒の患者が多い。	
\\	大酒飲みには高血圧の人が多い。	
\\	キリンの首はなんであんなに長いのだろう。	
\\	ああ僕もあんな風にずっと木にぶら下がっていたいよ。	
\\	ライオンが動物園から逃げたらしいよ。	
\\	マンドリルって、派手な顔だよね。	
\\	えりまきトカゲがブームの時代があったな。	
\\	トキが絶滅しそうなのはなにが原因だったのだろう。	
\\	コアラはかわいいけど怪力らしいよ。	
\\	ツチノコって本当に存在するのかなあ。	
\\	また人の残り物を食っているよ。ハイエナみたいな奴だな。	
\\	責任感のない人は動物を飼うべきじゃない。	
\\	どうしてもその雑種犬が欲しいんだな。ドーベルマンじゃなくていいんだな。	
\\	アザラシって南極にもいるのかなあ	
\\	アイアイは南の島の猿だよ。	
\\	どうしてカゴから文鳥をだしたの。逃げちゃったじゃない。	
\\	あの象さみしそうだよ。アフリカに帰りたいのかなあ。	
\\	羆は狂暴だから近づいちゃだめだよ。	
\\	もし動物になれるならなにになりたい。	
\\	鯨のおなかに入ってみたい。	
\\	この前公園で捨てシャムネコを見たよ。	
\\	君を動物にたとえると孔雀だな。	
\\	動物保護を広く訴えていきましょう。	
\\	動物虐待は本当に許せない。	
\\	人間だって動物だよ。	
\\	動物の言葉が理解できればいいのになあ。	
\\	きみの奥さんはたいへんな愛犬家なんだって?	
\\	ペットショップで子犬の首輪を買ってきたよ。	
\\	あの家の犬は、品評会で優勝したんだって。	
\\	犬にお座りやお手を仕込むのには根気がいる。	
\\	うちの犬は吠えたり人を噛んだりしません。	
\\	猫にエサと水をやっておいてね。	
\\	今度は毛がふさふさした猫を飼いたいなあ。	
\\	猫が家の柱で爪を研ぐので困っています。	
\\	実は猫アレルギーなんだよ。くしゃみが止まらないや。	
\\	このインコは人の声を真似するのがうまいなあ。	
\\	鳥かごが糞で汚れているから洗わなきゃ。	
\\	ハムスターの寿命ってどのくらいなんですか?	
\\	動物病院でフェレットも診てくれるかな。	
\\	牛のひづめがどんな形だか知ってる?	
\\	ライオンのたてがみは実に迫力がある。	
\\	イルカはとても高度な知能をもっています。	
\\	あなたとは口もききたくないの。	
\\	あいつとはもう絶交だ。	
\\	先に手を出したのはあっちなんだ。	
\\	文句があるなら言ってみろよ。	
\\	ほら、かかってこいよ。	
\\	その言い方がむかつくんだ。	
\\	弱いやつほどよく叫ぶんだ。	
\\	殴り合いのけんかになった。	
\\	お互いにののしり合った。	
\\	人がまじめにやっているのに、あいつがからかうんだ	
\\	今、何て言ったの。もう一度言ってみなさいよ。	
\\	そうかっとなるなって、冗談だよ。	
\\	話せばわかる、いいから落ちつけって。	
\\	何がけんかの原因なの?	
\\	けんかはやめなさい。	
\\	けんか両成敗だ。	
\\	けんかするならおやつ抜きよ。	
\\	けんかするなら外でやりなさい。	
\\	無視するなんて、陰湿ないじめをするんじゃない。	
\\	けんかするくらい元気な子のほうがいいさ。	
\\	あいつのやり方がどうしても気にくわないんだ。	
\\	けんかするほど仲がいいんだよ。	
\\	陰でこそこそ悪口を言うなんて最低だ。	
\\	あまりにばかばかしくて怒る気にもなれない。	
\\	けんかする人の気が知れないわ。	
\\	今になって当時のことを持ち出すなんてひきょうだ。	
\\	子どものけんかに親が出てくるな。	
\\	まとめて相手してやるから、何人でもかかってこい。	
\\	うそをつくな。	
\\	それはこっちのセリフだ。	
\\	余計なことをするな。	
\\	もうがまんできない。	
\\	きみには関係ないことだ。	
\\	ここから出て行け!	
\\	きみの顔なんか見たくもない。	
\\	強いやつはけんかなんてしないもんだ。	
\\	ばかにするな。	
\\	ばかなことを言うなよ。	
\\	けんかするなら素手でやれ。	
\\	簡単に人を刺すから最近のけんかは怖いんだ。	
\\	ただいま。お風呂沸いてるかな。	
\\	お帰りなさい。食事にする、それとも、お風呂にする。	
\\	熱くもなく、ぬるくもなく。ちょうどいい湯加減だよ。	
\\	これじゃ熱すぎて、入れないよ。	
\\	あの人、またお風呂で鼻歌歌っている。	
\\	湯上がりのビール。最高だね。	
\\	今夜は面倒くさいから、お風呂に入るはやめよう。	
\\	汗をかいたから、家に帰ったらすぐにシャワーを浴びよう。	
\\	たまには銭湯に行くのもいいものだ。	
\\	子供の頃、よく銭湯に行って友達と遊んだものだ。	
\\	おーい、バスタオルがないんだけど、持ってきてくれるかな。	
\\	家のお風呂は狭すぎて、足が伸ばせない	
\\	中学生になったとたん、娘が一緒にお風呂に入らなくなった。	
\\	私も孫に背中を流してもらう歳になったか。	
\\	湯船に入る前に、体を流しなさいよ。	
\\	この10年で銭湯の数が激減した。	
\\	今、銭湯っていくらするのかな。最近行ってないからな。	
\\	あれ、もうお風呂出たの?カラスの行水だね。	
\\	夫婦で銭湯に行っていた頃が懐かしいよ。	
\\	今週末は、温泉にでも行って、のんびりするかい?	
\\	疲れたからお風呂につかって疲れをとろう。	
\\	風呂に入ったらさっぱりした。	
\\	お風呂から出るとき、栓を抜いておいてくれる?	
\\	風呂にふたをしてください。	
\\	最近のアパートはユニットバスのところが多いね。	
\\	ユニットバスはどうも慣れないな。	
\\	風呂にお湯を張るのが面倒だからシャワーにしよう。	
\\	シャワーの勢いが弱い。	
\\	シャワーからお湯が出なくて困っています。	
\\	石けんからボディーシャンプーに変えたんだ。	
\\	リンスインシャンプーではどうも洗った気にならないな。	
\\	大変だ、お湯が浴槽からあふれちゃった!	
\\	浴室の中には湯気が充満しています。	
\\	風呂には短くても1時間は入っているよ。	
\\	私は朝シャン派です。	
\\	タオルを浴槽の中に入れないでください。	
\\	露天風呂のある宿に泊まることにしよう。	
\\	やはりこの季節には柚湯がいいね。	
\\	今日は入浴剤を入れてみよう。	
\\	実は私、犬と一緒にお風呂に入ってるんです。	
\\	冬は、肌が乾燥するから、鼻のまわりがガビガビになってしまう。	
\\	最近の湿気のせいで、髪の毛がぺちゃんこだ。	
\\	髪を伸ばしてるんだけど、毛先が細くなってきちゃった、どうしよう。	
\\	コマーシャルに出てくるつやつやヘアーなんて、ウソだと思う。	
\\	ヘアスタイルを変えたけど、誰にも気付いてもらえなかった。	
\\	このシャンプー使うと、髪のコシが出てくるから、おすすめだよ。	
\\	最近、髪の生え際に白髪があって、気になるんだよ。	
\\	最近、お腹のまわりに肉がついてきちゃった。	
\\	洗顔したあとは、肌がモチっとして、気持ちよい。	
\\	頬ににきびの跡が残ってしまって、困っちゃう。	
\\	色白を保つ秘訣って、なに?	
\\	朝起きると、いつも目が腫れぼったい。	
\\	顔色が悪いので、今日は濃いめに化粧をした。	
\\	二日酔いで、顔がはれている。	
\\	唇がかさかさしてても、舐めてはいけないよ。	
\\	今日は、フローラルの香りでさわやかに一日を過ごそう。	
\\	日焼けサロンで、色黒になった人は「いかにも」という感じがする。	
\\	まつげが長いのって、女らいくていいよね。	
\\	恋をすると、瞳の輝きが違うっていうけど、本当だね。	
\\	厚化粧は、男受けしないよ。	
\\	明日のデートに備え、ナチュラルメイクのテクニックを雑誌を読んで研究した。	
\\	私は太るとすぐに顔の頬にでてしまう。	
\\	あなたは、二重が映えるメークをしたほうがきれいですよ。	
\\	お風呂でたくさん汗をかくと、身体が引き締まった感じがする。	
\\	疲れている時に、足のマッサージをすると気持ちいい。	
\\	立ち仕事していると足が太くなるっていうけど、本当?	
\\	私は乾燥肌なので、冬は加湿器が欠かせない。	
\\	左の耳にピアスの穴を二つ開けた。	
\\	おでこがてかっているよ。	
\\	お昼御飯の後、女性用お手洗いは、化粧直しの人たちで込み合います。	
\\	水仕事をすると、手が荒れるわよね。	
\\	新しく買った靴が合わなくて、巻爪になってしまった。	
\\	私は顔の毛穴が目立つので、パックでのお手入れが欠かせない。	
\\	その日の気分で入浴剤を使いわけるのが、密かな楽しみなの。	
\\	疲れたときは、ビタミン
\\	をたくさんとった方が良い。	
\\	指が太くなってしまって、結婚指輪が外れない。	
\\	肩凝り放置は、万病のもと。	
\\	真っ赤なマニキュアは、若い人には似合わない。	
\\	私は出っ歯に悩んでいる。	
\\	磁気のネックレスは、本当に肩凝りに効くのだろうか?	
\\	貧血気味なので、鉄分をたくさん摂取したほうがいいですよ。	
\\	ただ食べなければやせる、というわけではない。	
\\	1度ネイルケアにこだわり始めると、お金がかかって仕方ない。	
\\	深爪して、指に力が入らなくて困った。	
\\	カラーコンタクトをつけたら、私も神秘的な女性に変身できるかな。	
\\	さわやかな笑顔の決め手は、白い歯というのは、言えていると思う。	
\\	ここ1年、髪を脱色しすぎてかなり痛んでしまった。	
\\	かかとがかさかさしているのは、お手入れが足りない証拠だ。	
\\	中国の人がスリムな秘密は、ウーロン茶にあるらしい。	
\\	春から夏にかけては、日焼け止めは欠かせない。	
\\	海で焼きすぎちゃって、背中が真っ赤になって痛い。	
\\	洗顔の後は、すぐに化粧水で入念にパッディングしましょう。	
\\	アイシャドウを少し薬指に取り、まぶたに置いて指でなじませます。	
\\	あのファンデーションは肌に合わないから、もう買わない。	
\\	あーあ。泣いたらマスカラが落ちて目の周りが真っ黒だ。	
\\	デオドラント・スプレーって、本当に汗を抑えてるのかな?	
\\	冬になると肌が乾燥してしまって、クリームが欠かせない。	
\\	無理なダイエットは体に悪いよ。	
\\	運動をして、ちゃんと食事もとってやせるのが一番いい。	
\\	太っちゃうってわかっていても、ついつい夜中にお菓子を食べちゃう。	
\\	「ながら食べ」が一番太りやすいんだって。	
\\	マニキュアを塗ったら、しばらくは何もできない。	
\\	あまり日に焼きすぎると、あとでシミになるわよ。	
\\	最近はコンビニにも化粧品がたくさん置いてある。	
\\	素足に靴をはいたら、靴ずれになってしまった。	
\\	髪が伸びてきたから、また染め直さないと。	
\\	髪をとかすときは、毛先から始めましょう。	
\\	髪を洗ったら、ちゃんと乾かしてから寝たほうがいい。	
\\	このスカート、ウエストのあたりが少しきついわ。太ったのかしら。	
\\	この折り畳み式の自転車、持ち運びができて便利だね。	
\\	しばらく自転車に乗らなかったら、タイヤの空気が減ってる。	
\\	背が大きくなったから、サドルを少し高くしてあげよう。	
\\	運動不足を解消するために、自転車に乗るようにしてるんだ。	
\\	久しぶりに自転車に乗ったら、脚がつっちゃって。	
\\	自転車を買いに行ったら、あまりに種類が多すぎて、結局、買わずに戻ってきた。	
\\	うちの息子は、現在、自転車に乗れるように猛特訓中です。	
\\	うちの子もようやく自転車に乗れるようになったよ。	
\\	タイヤがパンクしたので、自転車屋さんで修理してもらった。	
\\	交通渋滞の中を自転車でスイスイ通り抜けていくと気持ちがいい。	
\\	(車を運転していて)自転車が前を走っている時には気をつけなくちゃ。急に曲がったりするから。	
\\	この自転車、ペダルをこぐと、キーキー音がするんだ。	
\\	この自転車、大分錆び付いてきちゃったな。そろそろ新しいのを買うか。	
\\	最近は、ママチャリでも結構いい値段がする。	
\\	たまには自転車に乗るのもいいものだ。ストレスの解消にもなるし。	
\\	(子供に)自転車で遊びに行くんだったら、自動車に気を付けてよ。	
\\	チェーンが外れちゃったよ。直すの大変なんだよな。手も汚れるし。	
\\	前のタイヤ少し空気が減ってるから、空気を入れておいてね。	
\\	高校生になった長男が、マウンテンバイクを買ってくれってうるさいんだ。	
\\	最近、駅前の放置自転車の数が少し減ったように思える。	
\\	このデパートの地下に駐輪場があったと思うよ。	
\\	2人乗りは危ないからやめなよ。	
\\	このサドルの位置は私には高すぎるわ。	
\\	暗い時間に自転車に乗るときには電灯をつけましょう。	
\\	チェーンが外れちゃった!	
\\	この自転車、カゴがついてないの?	
\\	片手に荷物を持って運転したら危ないよ。	
\\	うちの子どもは補助輪つきの自転車から卒業したよ。	
\\	小さいころは自転車でどこへでもでかけたもんだ。	
\\	最近あまり三輪車を見かけない気がするな。	
\\	スカートをはいて自転車に乗るのは大変よね。	
\\	うちの学校、自転車通学だめなんだ。	
\\	なんで自転車のことを「チャリ」って呼ぶんだろう?	
\\	坂の多いところでは自転車は不便だね。	
\\	学生のころは自転車で新聞配達をしていたよ。	
\\	今年の夏は自転車旅行に出かけるつもりだ。	
\\	旅行先でレンタサイクルを借りよう。	
\\	サーカスで自転車の曲乗りをやっていたよ。	
\\	僕は一輪車にも乗れるよ。	
\\	自転車が将棋倒しになっている。	
\\	今日は漫画喫茶にでも行こう。	
\\	今じゃネットで本が買えるらしい。	
\\	漫画ばっか読んでないでたまには本を読みなさい。	
\\	子どものために絵本を買っていかなきゃ。	
\\	最近は忙しくて、本を読む暇もない。	
\\	ファッション雑誌は時代の最先端を取り込んでいかなきゃならない。	
\\	あの女優がヌード写真集を出すらしいよ。	
\\	この本、面白いから絶対見てね。	
\\	1巻と3巻は持ってるんだけど、2巻だけないんだよね。	
\\	家の息子はマンガばかり読んで、全然、勉強しないの。	
\\	本屋で立ち読みすれば本を買わずにすむなあ。	
\\	あいつ聖書を隅から隅まで読んで理解したらしいぜ。	
\\	この前書いた作文が、コンクールで金賞を取って雑誌に載るらしい。	
\\	最近のミステリー本はマンネリだよね。	
\\	暗い所で本を読んだら目が悪くなるよ。	
\\	学生時代はよく本を読んだけど、勤め始めてからは週刊誌以外はほとんど読まないね。	
\\	どこに行っても売り切れだよ。人気あるんだな、あの本。	
\\	図書券で辞書を買ったよ。	
\\	この本、文庫本になるまで買うのやめよ。	
\\	古本は安くていいよな。	
\\	本は読み返さないと身にならない。	
\\	古本屋に行くと隠れた名著に出合うことができる。	
\\	いい本が売れるとは限らないんだよね。	
\\	部屋の中が本でいっぱいだ。	
\\	洋書を読んで英語力をつけよう。	
\\	若者の活字離れが進んでいる。	
\\	本は偉人との対話だ。	
\\	ペーパーバックを週に1冊は読んでいます。	
\\	乱読はいけないよ、精読しないとね。	
\\	本屋にはハウツー本が溢れている。	
\\	洋雑誌を購読したけど、積読になって困ってるんだ。	
\\	読
\\	最近の英語の辞書にはいろいろな情報が載っている。	
\\	インターネットの普及で本はなくなるのだろうか。	
\\	本は読まない方がいいという人もいる。	
\\	速読によって集中力が増すといわれる。	
\\	読
\\	最近はどの本も似たり寄ったりだ。	
\\	本の虫になるものいいけど、たまには外に出て遊ぼうよ。	
\\	グーテンベルクが活版印刷を発明してから数世紀がたちます。	
\\	最近予定がつまってわからなくなってきたよ、システム手帳を買わないとね。	
\\	喫茶店で本を読むと落ち着きます。	
\\	どうも運転しにくいと思ったら、パンクしてたんだ。	
\\	曲がる時にはウインカーを出せよな。	
\\	この先、5キロ渋滞か。打ち合わせの時間にとても間に合いそうにない。	
\\	この車、10年近く乗ってるけど、まだまだ十分走るよ。	
\\	定期点検のために車出しちゃったから、今、車ないんだよ。	
\\	車、車検に出してるから、今、代車に乗ってるんだ。	
\\	いつも右ハンドルに乗ってるから、左ハンドルに乗るとなんか変な感じだな。	
\\	ずっとオートマ運転してるから、もうマニュアルには乗れないよな。	
\\	雨の日の運転は気を付けないと。	
\\	お前運転荒いな。そんなに急がなくても時間は十分にあるんだから。	
\\	ガソリンがそろそろなくなるな。どこかで給油して行こう。	
\\	フロントガラスに油膜がついて、ギラギラして前が見にくい。	
\\	そこは一通だから入れないよ。	
\\	一度通勤すると、なかなか止められないよね。	
\\	久しぶりにクーラーを付けたら、車中に嫌な臭いが充満してしまった。	
\\	そろそろ洗車に出して、ワックスでもかけてもらうか。	
\\	ハイオク満タンにしてください。それに、オイルも点検してもらえますか。	
\\	タイヤの空気圧見てもらえますか。	
\\	この車、100キロを超えると車体がガタガタし始めるんだ。	
\\	ボーッとしたら、高速の降り口を通過してしまった。	
\\	あの車、ブレーキランプが切れてるよ。気がついてるのかな。	
\\	雪道では、急ブレーキ、急ハンドルは禁物です。	
\\	チェーン付けたりはずしたりするの面倒だから、スタッドレスにしようかな。	
\\	お酒を少しでも飲んだら運転しちゃだめだよ。	
\\	私、ペーパードライバーなの。	
\\	運転免許書は身分証明書代わりね。	
\\	しまった。鍵を車の中に置いたままドアをロックしてしまった。	
\\	細い道を通り抜けようとしたときに、電信柱にドアをこすってしまった。	
\\	信号待ちしていたら、後ろから追突された。	
\\	後ろにぴったりつけて運転するなよな。	
\\	せまい道路でトラックにあおられちゃって、怖かったよ。	
\\	免許の更新に行くのを忘れてて、失効になっちゃったんだ。	
\\	あーあ、前の車、坂道発進でエンストしちゃったよ。	
\\	ちょうどもうじき車検だから、新車に乗り換えようと思ってるんだ。	
\\	日本で左ハンドルの車って、運転しづらくないのかな?	
\\	軽自動車は、小回りがきいていいよね。	
\\	車を駐車場に置いておいたら、当て逃げされた。	
\\	この車、ワイパーのゴムが傷んでるんじゃない?前が見えづらいよ。	
\\	山を下るときはギアをセカンドに入れないと、ブレーキが効かなくなるよ。	
\\	そんなに飛ばすことないよ。時間はたっぷりあるんだから。	
\\	この車、中古だけどあんまり距離も走ってないし、まだまだ乗れそうだね。	
\\	この前駐禁を取られて、ついに免停になっちゃった。	
\\	縦列駐車がどうしても苦手なの。左うしろなんて見えないもの。	
\\	あれ?半ドアになってるのかな。うしろのドア見てみてくれる?	
\\	ハンドブレーキをかけたまま走っててさ。車を止めてようやく気がついたんだ。	
\\	ガソリンスタンドで給油口を開けるつもりが、トランクを開けてしまった。	
\\	この車、ハンドルがやけに重いね。回すのに一苦労だよ。	
\\	この駐車場はバックで入れないと、出るとき面倒だよ。	
\\	信号が変わったのに気がつかないでいたら、うしろの車に思い切りクラクションを鳴らされた。	
\\	この車、無駄に大きいばかりで燃費悪いんじゃない?	
\\	タクシーを1台、4時にお願いしたいんですけど。	
\\	新宿までいくらくらいで行きますか?	
\\	だいたいでいいから、料金を教えてください。	
\\	つきあたりを左に曲がって、郵便局の前で停めてください。	
\\	この辺りです。少し、ゆっくり走ってください。	
\\	車を1日借りたいのですが。	
\\	事故を起こしてしまった場合、保険は適用されますか?	
\\	そっちが飛び出してきたから、ぶつかったんじゃないか。	
\\	こっちの信号は青だったぞ。	
\\	車は大破したけど、運転者は奇跡的に無事だった。	
\\	チャイルドシートはしっかり取り付けないと、事故のとき効果がない。	
\\	エアーバッグが間違って出てきたら、どうやって片づけるんだろう?	
\\	ライトを付けっぱなしで駐車してたから、バッテリーが上がっちゃった。	
\\	寒すぎてエンジンがかからないよ。	
\\	車が燃えています。消防署に電話して!	
\\	父が倒れました。救急車をお願いします。	
\\	国際自動車免許ってどうやってとるの?	
\\	今月、免許の書き換えなんだよね。	
\\	自動車を処分するにもお金がかかるよね。	
\\	この辺りで、車上荒しが多発している。	
\\	思ったことをズバズバ言うタイプ。	
\\	感情の起伏が激しい。	
\\	私は明るい(暗い)性格です。	
\\	ちょっとしたことで直ぐに落ち込んでしまいます。	
\\	私は好奇心の強い18歳の女の子です。	
\\	私おっちょこちょいなんです。	
\\	私せっかちなんです。	
\\	落ち込んだ時に彼女の
\\	を聴くと頑張ろうって気になるんです。	
\\	私、心配性なんです。	
\\	明るい性格なので、つまらないことは笑い飛ばしてしまいます。	
\\	私にはガッツとやる気があります。	
\\	私、何があってもめげません。	
\\	私引っ込み思案なんです。	
\\	私、行動派なんです。	
\\	暗記が苦手なので英語はあまり得意ではありません。	
\\	私の性格は、自分でもよく分かりません。	
\\	私には少しわがまま(自分勝手)なところがあります。	
\\	私おしゃべりで声がでかいです。	
\\	これからはプラス思考で頑張って行こうと思います。	
\\	私のんびり屋さんです。	
\\	私の長所は、何事にも積極的に取り組むことです。短所は、短気なところです。	
\\	最近忘れっぽくて困っています。	
\\	熱しやすくて飽きっぽい性格です。	
\\	真面目で少し頭が固いところがあります。	
\\	物事をじっくり考える慎重派です。	
\\	私は神経質で小さな物音にもすぐに反応してしまいます。	
\\	私はみんなから小心者と言われています。	
\\	優柔不断な私はいつも何を食べるか迷ってしまいます。	
\\	怒りっぽい上司だと仕事がやりにくいな。	
\\	そんなひねくれた性格だと誰も協力してくれないよ。	
\\	彼は裏では何を考えているかわかりません。	
\\	私は人前で喋るととすぐにあがってしまいます。	
\\	彼は冗談が通じない人です。	
\\	私は人を笑わすことが好きです。	
\\	彼はお金にだらしのない人です。	
\\	彼女は誰にでもいい顔する八方美人です。	
\\	彼は昔から努力家で有名です。	
\\	私は陰で人の悪口を言う人は大嫌いです。	
\\	彼女は繊細だからちょっとしたことですぐに傷ついてしまいます。	
\\	失敗しても立ち直りが早いです。	
\\	私は笑いじょうごです。	
\\	私は甘いものには目がないです。	
\\	昔から負けず嫌いです。	
\\	すぐにカーッと熱くなってしまいます。	
\\	彼はいじきたないのでみんなから嫌われています。	
\\	協調性がないと会社ではやって行けません。	
\\	彼はいつまでも根に持つタイプです。	
\\	彼は時間にルーズでいつも待ちあわせの時間に来ません。	
\\	あの人は竹を割ったような性格だ。	
\\	あの子はイタズラばっかりしてワンパクだ。	
\\	あいつは見栄っ張りだ。	
\\	私はお酒を飲むと性格が変わります。	
\\	太っ腹の彼は私にいつもご馳走してくれる。	
\\	彼は何事に対してもおおざっぱだ。	
\\	彼は自己中心的でいつも周りの人に迷惑をかけている。	
\\	彼は口だけで行動しようとはしない。	
\\	私は社交的で人と話しをするのが好きです。	
\\	彼女は誰とでもすぐに仲良くなれます。	
\\	人見知りが激しいのでなかなか友達ができません。	
\\	何事も三日坊主で長続きしません。	
\\	私は物事を割り切って考えます。	
\\	あの人はいうことがコロコロ変わるので信用できません。	
\\	怒られるとすぐにスネてしまいます。	
\\	彼は物事をはっきりさせないと気が済まない性格です。	
\\	私は思っていることをなかなか口に出して言えません。	
\\	物事の途中でもすぐにあきらめてしまいます。	
\\	あの人は社内でも態度がでかくて有名です。	
\\	今年もクリスマスがやってきた。	
\\	クリスマスの由来がなんだか知ってる?	
\\	南半球では真夏のクリスマスなんだよね。	
\\	街にクリスマスの飾り付けが目立ちだした。	
\\	デパートはクリスマス商戦の真っ最中だ。	
\\	ホワイトクリスマスなんて期待しないほうがいいよ。	
\\	クリスマスのイルミネーションを見に、大勢のカップルがやってくる。	
\\	日本ではクリスマスと正月ではどちらが盛り上がるかな?	
\\	イヴの予定は何かありますか?	
\\	クリスマスは家族と過ごす予定です。	
\\	せっかくのクリスマスなのに仕事が入っています。	
\\	恋人と別れてしまったので、さびしいクリスマスになりそうだ。	
\\	私はクリスチャンではないので、クリスマスは祝いません。	
\\	クリスマスソングはちょっと食傷気味だ。	
\\	毎年クリスマス礼拝には参加する。	
\\	ろうそくに火をともして「きよしこの夜」を歌う。	
\\	サンタクロースはトナカイがひくそりに乗っています。	
\\	サンタは大きな袋からたくさんのおもちゃを出します。	
\\	サンタは煙突のない家にどうやって入るんだろう?	
\\	そろそろクリスマスツリーを飾り付けようよ。	
\\	サンタや靴の形をした飾りをもみの木に飾ります。	
\\	シャンパンとクリスマスケーキを買うのを忘れないでね。	
\\	サンタの格好をした人がたくさんいた。	
\\	枕元にプレゼントを置くところを子どもに見つかりそうだ。	
\\	少年法は早く改善されるべきだ。	
\\	犯人は犯行現場に戻るってよくいうよね。	
\\	まだあそこのコンビニの強盗犯は見つかってないらしいよ。	
\\	ねぇねぇ、あそこに住んでいるスミスさんが容疑者として連行されたらしいよ。	
\\	保険金殺人が最近多いよね。結婚する相手をちゃんと知っておかなきゃならないね。	
\\	放火はとても重い罪なんだぞ。	
\\	あんないい人が横領なんて信じられない。	
\\	完全犯罪なんて、できっこないよ。	
\\	シャーロックホームズが本当に存在したら、事件が早く解決するのになあ。	
\\	食い逃げだ。だれか捕まえてくれ。	
\\	この電車、痴漢が多いから乗りたくないわ。	
\\	この前盗まれた車が北海道で見つかったらしい。取りに行くの面倒だよ。	
\\	やっぱりお金が絡んだ犯罪が多いよ。	
\\	人の命は地球よりも重い。	
\\	未成年の飲酒、喫煙はいけないよ。	
\\	早く自首したほうがいい。いつまでも逃げられないぜ。	
\\	危ない橋を渡ってまで金が欲しいか。	
\\	被疑者は身長175センチくらいで、年は25歳くらい。長髪で
\\	シャツにジーパンの男だ。早く探せ。	
\\	八百長で何人ものスポーツ選手が捕まったらしいよ。	
\\	彼は電車の中で痴漢の現行犯で逮捕された。	
\\	彼女が包丁を持った犯人を殴り倒したのは、正当防衛だ。	
\\	判決には、容疑者の精神鑑定結果が大きく影響する。	
\\	前代未聞の贈収賄事件に関与したため、首相は失脚した。	
\\	完全犯罪を実行するには、周到な計画が必要だ。	
\\	25年前の強盗殺人事件の犯人が、時効の直前に逮捕された。	
\\	裁判官が犯罪を犯すなんて、世も末だ。	
\\	彼は犯罪とわかっていながら、会社の金に手をつけた。	
\\	詐欺にひっかかるなんて、我ながら情けない。	
\\	捕まらなければ何をしてもいいというわけではない。	
\\	江戸時代には、犯罪者だけでなくその家族まで罰を受けたそうだ。	
\\	凶悪犯罪が急増している。そういう時代なのかな。	
\\	昨今では、動機が不明な重大犯罪が多い。	
\\	万引きは犯罪だということを、学校でしっかり教育する必要がある。	
\\	スピード違反で1カ月も免停をくらった。	
\\	今回の紙幣偽造事件は、組織的犯行の疑いが濃厚だ。	
\\	繁華街で、外国人の犯罪組織が台頭している。	
\\	警察にとって、愉快犯ほど腹立たしいものはない。	
\\	ある国の政治犯が、日本に亡命を希望している。	
\\	ナチスによるホロコーストは、人道に対する犯罪だ。	
\\	戦争犯罪を裁くのは、国際機関に任せるべきだ。	
\\	最近メールの調子が悪い	
\\	メモリが足りないのかなぁ	
\\	部長が、すごいハイスペック・マシン買ったそうよ	
\\	私は断然ウィンドウズ派	
\\	それ、ウィルスの仕業じゃないかな	
\\	我が社は、2000年問題への対応は万全です	
\\	デスクトップがグチャグチャになってきた	
\\	この回線の細さじゃ、動画は見られないよ	
\\	あの会社もついにホームページを立ち上げたらしい	
\\	サーバーがダウンしてて、メールチェックできない	
\\	液晶画面は、見にくいなぁ	
\\	不正アクセスは、法律で取り締まるべき	
\\	いまだにインターネットの導入を検討している会社がある	
\\	人のパスワードを乱用する「なりすまし」行為が問題になっている	
\\	一日中コンピュータに向かってるから、結構電磁波浴びてるよ。	
\\	どうしよう、コンピュータが立ち上がらない	
\\	このコンピュータ1年前に買ったばかりなのに、もうすっかり時代遅れだわ。	
\\	圧縮したファイルを添付で送ってください	
\\	違法コピーはいけません	
\\	あ、上書きしちゃった	
\\	ハードが断片化を起こしてるみたい	
\\	ウィンドウズは、拡張子を付けないとファイルを読んでくれない。	
\\	そのフロッピー、念の為書きこみ禁止にしておいて	
\\	デスクトップの壁紙をペットの写真にした	
\\	給紙トレーに紙が入ってないよ	
\\	そのフォルダ、共有にしておいて	
\\	初期設定のまま使ってるの?	
\\	専用線で見ると、すごく早いんだけど	
\\	部分一致検索はサーバーに付加がかかるから、やめたほうがいいね。	
\\	外付けのハードディスクに保存しておいて	
\\	メモリ増設しようかな	
\\	表計算ソフトは、覚えておくと何かと便利よ	
\\	イントラネットを使った社内教育がはやってるらしい	
\\	エンドユーザーの視野にたった商品開発が必須	
\\	こんな余計な機能なんでつけたんだろう	
\\	便利な隠しコマンド教えてあげようか	
\\	この画像、解像度が低すぎてすごく汚い	
\\	そろそろプログラミングを覚えようかな	
\\	暗号技術も、かなり信頼のおけるものに進化してるらしい	
\\	インターネットで願書を取り寄せるといいよ	
\\	インターネット中毒って、本当にあるみたいね	
\\	学校は、インターネット接続料金が割り引きされるようになったらしい。	
\\	は、家電製品みたいな感覚で買われてるんだろうね	
\\	個人情報の流出があちこちで起こって問題になった	
\\	大量の文書を入力しなきゃならない	
\\	ホームページをつくるのもいいけど、著作権の侵害になるようなことはしてはいけません。	
\\	郵政省の通信白書を見ればわかるかもしれない	
\\	電子商取引が急速に普及し始めた	
\\	インターネット上で、女の子になりすます男性を「ネットおかま」っていうらしいよ。	
\\	有害情報って、線引きが難しいよね	
\\	よろしければ、私のホームページからリンクを張らせてください	
\\	バックアップとっておいたほうがいいよ	
\\	この時間はアクセスが集中して全然つながらない	
\\	ハッカーにホームページを書きかえられた	
\\	デジカメから画像をとりこんだら、ハードがいっぱいになっちゃった。	
\\	ただのメル友(メール友達)よ	
\\	添付書類が壊れているようです。	
\\	このコピー機、いつも紙詰まりばっかりだ。	
\\	4の紙がないな。いつもこの棚に置いてあるんだけど。	
\\	明日のプレゼン資料、カラーで10部コピーしといて。	
\\	このコピー機、150パーセント以上拡大できないから不便。	
\\	100枚もコピー取るなら、簡易印刷機のほうが安上がりだよ。	
\\	この本分厚いからうまくコピーが取れなかったんだ。	
\\	トナー切れで、文字が薄くなってきてる。	
\\	このフロア、コピー機1台しかないからいつも順番待ちだよ。	
\\	紙詰まりを直したから、手がトナーで真っ黒になっちゃった。	
\\	昔のコピー機ならちょちょいと故障を直せたけど、今のは構造が複雑だからメンテ業者に任せないとだめだよ。	
\\	これ、両面印刷できないから、いちいち手差しで紙を入れなきゃならないんだ。	
\\	写真は写真モードでコピーしないと真っ黒になっちゃうよ。	
\\	なんでこんなにコピーが汚いの。コピー面のガラスに何かついてるんじゃない。	
\\	修正液が乾ききらないうちにコピーしたから、コピー機のガラスが汚れちゃってるよ。	
\\	元原稿が鉛筆で薄く書いてあるから、コピー濃度の設定をできるだけ濃くした方がいいな。	
\\	コピー取るのは女子社員の仕事なんですか。	
\\	カラー写真ってどうしたらうまくコピーできるの。	
\\	カラーコピー機って、維持費高そう。	
\\	コピー機の寿命ってどのぐらいかな。	
\\	これコピーの濃度が濃すぎるよ。もう少し薄くして取って。	
\\	いまどきカラーコピー機もないの。コンビニに行くしかないのか。	
\\	えっ、会社のコピーでも1枚5円もかかるの?	
\\	このコピー機、高機能なのはいいけど、毎週修理してない?	
\\	コピーが終わったら「オールクリア」ボタンを押しておかないと、次の人が困るよ。	
\\	分厚い本ってコピー取りにくいんだよね。何かうまい方法ないかな。	
\\	コピー機が壊れてるから、代わりにスキャナーとプリンターを使って資料作って。	
\\	両面印刷、もうちょっと簡単にできたらいいのにな。	
\\	3の書類を
\\	4サイズにするのって、縮小率50%じゃないの?	
\\	この書類、20部コピーして、第1会議室へ置いといて。原本はおれの机の上ね。	
\\	うちの会社、書類は
\\	4に統一って決まったから、これから
\\	4サイズの資料は全部
\\	4に縮小コピーするように。	
\\	やっぱり大きいサイズの紙のほうが紙詰まりする率が高いみたいだね。	
\\	コピー機の内部って熱いから、開けるときは気をつけて。	
\\	なんで紙が詰まってもいないのに紙詰まりのサインが出るんだろう。	
\\	ここに紙が詰まっちゃうと、修理の人を呼ぶしかないよ。	
\\	急いでいるときに限って紙詰まりだよ。	
\\	これ白い紙じゃなくて、黄色い紙にコピーしといて。	
\\	このコピー機より、総務の前に置いてあるやつのほうがきれいにコピーできるよ。	
\\	きみっていつもコピーを取ってるね。	
\\	こんなにコピー機の性能が上がってきたのに、赤とか青はコピーすると薄くなっちゃうんだよな。	
\\	すごい、この機械の通算コピー枚数、23万枚だって。	
\\	今度のデートは、遊園地に行ってジェットコースターに乗ろうよ!	
\\	7月の上旬は、海に行くのは早いかな?	
\\	9月になると海にはクラゲがいっぱいだよ。	
\\	ボディボードとサーフィンどっちが得意?	
\\	夜までいて花火を見ていこうよ。	
\\	この車の燃費はどれくらい?	
\\	右手に海のさざ波が見えてきたよ。	
\\	山の新緑がきれいにはえている。	
\\	湖で2人きりでボートに乗ってくつろぎたいな。	
\\	この肉と、いちばん上等な白ワインも一緒にお願いします。	
\\	肉に合うワインはどれですか?	
\\	分けて出してもらえますか?	
\\	みんなで分けて食べたいので、取り皿をもらえますか?	
\\	ソースは別に添えてください。	
\\	できるだけ辛くしてください。	
\\	観覧車が一番好きです。	
\\	お化け屋敷に行くの。夜寝られなくなっちゃうよ。	
\\	宙返りするジェットコースターに乗ったら、気持ち悪くなっちゃった。	
\\	初めて彼女の手を握ったのは、お化け屋敷の中でした。	
\\	疲れたから、コーヒーカップに乗ってゆっくりしたいです。	
\\	山でおいしい空気をすうと気分も爽快です。	
\\	山の麓から山頂までは4時間かかるって。止めようよ。	
\\	富士山の雪化粧がすばらしくきれいだね。	
\\	登山道に空き缶やゴミを捨てるのはやめてほしい。	
\\	頂上に着くと皆で「ヤッホー」と叫んだ。	
\\	お酒飲んだんだから、車の運転しちゃだめよ。	
\\	僕の目には君しか映ってないよ。本当さ。	
\\	今日はどこへ行こうか、遊園地?それとも映画でも見る?	
\\	最初にデパートのバーゲンにつきあってほしいんだけど。	
\\	今の二人、見た?あのペアルックはちょっとセンスないよね。	
\\	このペーズリー柄のネクタイ、あなたに似合うんじゃない?	
\\	このワンピースは、君の雰囲気にぴったりだね。	
\\	昨日のけんかを気にしてるの?意外にナイーブなのね。	
\\	ただの男友達っていう立場から彼氏に昇格させてくれるかい?	
\\	私はもうあなたに夢中、メロメロなのよ。	
\\	あのビルの屋上から見る夜景はきれいなのよ。見に行かない?	
\\	こうして二人でずっと海を見ていたいね。	
\\	それ、わたしを口説いてるの?	
\\	もう私、帰らなきゃ。父親がうるさいから。	
\\	今日はいい天気ね。車でどこかへ連れていってくれない?	
\\	俺、ペーパードライバーだから運転には自信ないんだ。電車でどこかに行こうよ。	
\\	ジェットコースターに乗っただけで、はしゃぐなんてかわいいね。	
\\	あの映画の主人公、かわいそうで泣いちゃったわ。	
\\	何が言いたいんだい?君のそのはっきりしないところが好きじゃないんだ。	
\\	もうこれ以上、話にならないわ。私、帰る。	
\\	門限何時だっけ?	
\\	俺の車は、左ハンドルだよ。乗りたい?	
\\	プレイボールの前に、トイレに行った方がいいよ。	
\\	僕は女性にお金を出させるほど、野暮じゃないよ。おごらせて。	
\\	次のデートは水族館がいい?それとも美術館で絵でも見る?	
\\	僕たち気が合うね。	
\\	雨が降ってきたね。相合い傘しようよ!	
\\	いいよ、傘ささなくて。一緒に入ろうよ。	
\\	助手席に乗ってナビしてくれる。	
\\	カーナビの使い方わかる?ごめん、モニターをもう少し斜めにして。	
\\	もう、あなたとは付き合えないわ。	
\\	ごめん、実は二股をかけていたの。許してね。	
\\	僕たち今までに、何回デートした?覚えてる?	
\\	今度のデートで私の両親に会ってくれる?	
\\	火事だ!逃げろ!	
\\	最近この辺りでは放火が頻発している。	
\\	昨日、角の家で家事があっただろ。寝たばこが原因なんだって。	
\\	この辺りは木造の家が多いから、火事が起きたらすごい被害になるだろうな。	
\\	あっ、揺れた。地震かな。	
\\	近いうち東京に大きな地震が来るかも。	
\\	地震による津波の影響はない模様です。	
\\	地震で道路のあちこちに地割れができている。	
\\	最近は地震保険に入る人が増えている。	
\\	地震でビルが横倒しになってるよ。	
\\	地震のときは、すぐにガスを消すように。	
\\	豪雨のため、近所の川が氾濫した。	
\\	大雨で土砂崩れが起きた。	
\\	大型台風が日本に接近しています。	
\\	大雪のため首都圏の交通が麻痺してしまった。	
\\	雪のため東名高速道路で玉突き事故が発生した。	
\\	火山が噴火し、私の家が灰まみれになった。	
\\	火山が噴火する恐れがあるので、近所の住民に避難命令が出された。	
\\	昨日の停電は、雷が原因らしい。	
\\	竜巻はアメリカではよくあるけど、日本ではまれだよね。	
\\	地震予知委員会が地震発生の可能性を指摘した。	
\\	この地震による津波の心配はありません。	
\\	震源は三陸海岸沖15キロ、マグニチュードは7.2でした。	
\\	地震に備えて食料品を備蓄しておこう。	
\\	昨日の地震、大きかったね。	
\\	震度4はあったね。	
\\	関東地区に暴風波浪警報が発令されました。	
\\	埼玉県に出されていた大雨注意報が解除されました。	
\\	南太平洋上で台風が発生しました。	
\\	台風12号が九州に上陸しました。	
\\	台風12号は日本海に抜け、熱帯低気圧に変わりました。	
\\	台風のときは波が高いので海岸には近づかないように。	
\\	強風で大阪行きの飛行機がすべて欠航した。	
\\	津波で大型タンカーが転覆した。	
\\	霧が発生しているので見通しが悪くなっています。	
\\	川の水がはんらんして周囲に水害をもたらした。	
\\	雨が降り続いているので、川がはんらんするのではないかと心配だ。	
\\	洪水で家が押し流された。	
\\	大雨のあとは地盤がゆるんでいるので、土砂崩れが起こりやすい。	
\\	土石流が発生する危険がある。	
\\	一夜漬けでテストに臨んだって、いい結果なんか期待できないよ。	
\\	数学なんてできなくたって、生活していくのに困らないでしょ。	
\\	勉強っていうのは考えることを訓練するのであって、生活の役に立てるためだけにするもんじゃない。	
\\	勉強はさせられるものじゃなくて、自らの知的好奇心を満足させるためにするんだよ。	
\\	「学級崩壊」っていうことがよく言われるけど、小学生の中にも授業中に静かに座っていられない子どもがいるらしいよ。	
\\	中学生や高校生になると、授業を邪魔するやつやさぼるやつも多いだろうしね。	
\\	大学でも講義中の私語が大きな問題になっているって言うし。	
\\	いまみたいな時代に先生をやってるのは、大変だろうなあ。	
\\	生意気なガキにちょっとでも手を出せば、親やマスコミがすぐ大騒ぎするもんね。	
\\	地域のおとながみんなで「悪いことは悪い」って子どもたちに教えていくことが大切なんだろうなあ。	
\\	最近の若い奴らは常識に欠けるよな。もっと、道徳教育を重視しなきゃ日本も終わりだよ。	
\\	不登校が大きな社会問題になっているようだけど、単なる怠けと見分けがつきにくいだろうなあ。	
\\	いじめで自殺しちゃう子どもが出るなんて、現代社会のひずみを象徴しているようだ。	
\\	いじめは、いじめる方もいじめられる方もそれぞれ問題がある、と言ったら語弊があるだろうか。	
\\	体罰は絶対悪のように言われている。それゆえに、教師が思い切った厳しい指導ができなくなっているような気がする。	
\\	子どもをむやみに殴ることはいけないが、時と場合によっては「愛のムチ」も必要だと思う。	
\\	親も教師も子どもを甘やかしすぎるから、つけあがるんだよ。	
\\	暴論かもしれないが、戦前のような厳しい教育を再現したほうがいい社会になるかもしれない。	
\\	人権、人権って権利ばかりが主張され、その裏側の義務を果たすことの重要性がほとんど教育されていないのは片手落ちだ。	
\\	家庭内暴力は親のしつけの問題なんだろうけど、そういう歪んだ環境で育ったやつが凶悪犯罪を犯してるから、無視できない問題だ。	
\\	自分の子どもの教育を放棄するような大人もいるし、生徒に注意できない教師もいるし、世も末だよな。	
\\	家庭教育、学校教育が大切なことはもちろんだが、社会全体でそれをサポートしていく態勢も築いていくべきではないだろうか。	
\\	最近の子どもを注意すると「だってノなんだもん」というように、何でも責任を人に転嫁する傾向にある。	
\\	自分でやったことには責任を持つ、すなわち「自己責任」の原則を徹底的に教育すべきだ。	
\\	家庭内に父権を復活させることも、教育をよくしていく方策のひとつではないだろうか。	
\\	国旗掲揚、国歌斉唱の問題は、毎年卒業式シーズンになると新聞紙上をにぎわす。	
\\	「日の丸・君が代」問題で教育委員会と現場教職員との板ばさみになった校長が自殺しちゃった事件があったよね。	
\\	現代の教育荒廃は、学校、家庭の双方に問題があることは明らかだ。	
\\	教員やってたからわかるけどさあ、自分の子どもの教育を学校任せになんかしてられないよ。	
\\	「金八先生」みたいな先生なんて、そうざらにいるもんじゃない。	
\\	ドラマのように、すぐにメいい子モになっちゃう子どもなんかいないだろう。	
\\	最近では名簿も男女混合になっているらしいが、男女平等意識ってそういうところからも芽生えるのだろうか。	
\\	小学校における英語教育について散々議論された結果、「国際理解教育」の一環としての導入が決まった。	
\\	数学や英語って教科は、年を追うにつれて好き嫌いがはっきりする教科だ。	
\\	私は昔から数学が苦手で、いまだに数字を見るだけで拒絶反応が出てしまう。	
\\	小学校のとき、作文を書かされるのが嫌いでね。原稿用紙に何枚も書いてるやつがいて、すごく劣等感をもったものだ。	
\\	歴史の授業はロマンがあって面白かったんだけど、年号がどうしても覚えられないんだよねえ。	
\\	学校なんて、人間としての社会性を身につけるためだけに機能しているようなものだ。	
\\	中学・高校のときの思い出って言えば、サッカー部でサッカーに明け暮れ、同級生とバカをやったことくらいかな。	
\\	昔から宿題やレポートは直前まで手をつけなかったけど、いまだに追い込まれなければ行動を起こさない性格は変わっていない。	
\\	社会人になったらテスト勉強から解放されると思ってたけど、人生は一生勉強であることがわかってきた。	
\\	今だから思うのかもしれないが、学生時代にもっとちゃんと勉強していればよかったなあと感じることがよくある。	
\\	大学もサービス機関としての意識を向上させないと、つぶれるところが出てくるだろう。	
\\	大学は図書館の開放、公開講座の開設など、地域住民を中心としたサービスを提供し、「象牙の塔」からの脱却を図っている。	
\\	インターネット上で英語を勉強するなんてことが考えられているみたいだけど、私はどうせやるなら紙と鉛筆でやるほうが効果的だと思う。	
\\	アメリカの遠隔講義プログラムを受講し、学位を取得する例が見られるようになった。	
\\	科学技術の進展が目覚しい現代において、教員だけで「教育」を担っていくには限界があるのではないか。	
\\	アメリカの「ネット・デイ」にならい、日本でもボランティアによる学校のネットワーク環境作りが進められている。	
\\	たばこを吸っている中学生を見ても、注意する大人を見かけなくなった。	
\\	教員の役割も、知識を教え込むことから、子どもたちの興味・関心を引き出し、それを追究する方向に導いてあげることにシフトすべきだろう。	
\\	電子メールが流行っているようだけど、どうやって使うの?	
\\	最近は携帯で電子メールのやり取りできるんだね。	
\\	キーボードが苦手だから、電子メールはちょっとね。	
\\	無料のウェブメールって、どうも使いにくいよね。	
\\	メールにこんな大きな書類を添付してくるなよ。	
\\	彼からのメールまた文字化けしてる。何が原因だろう。	
\\	先方に何度メールを送っても、戻ってくるのはどうしてだろう。アドレス間違えていないのに。	
\\	返事を書くのが遅くなり申し訳ありません。	
\\	添付した書類が開けません。
\\	で作成して再度お送りください。	
\\	いちいちメールを書かなくても、携帯で連絡取ればいいじゃない。	
\\	外国にいる人と連絡を取るのは電子メールが一番いいよね。	
\\	電子メールばかり書いてると、漢字忘れちゃうんだよね。	
\\	メールアドレスが新しくなりましたので、お知らせします。これからは
\\	にメールお願いします。	
\\	メールに画像や音声ファイルを添付するのもいいけど、ダウンロードするのが大変だからね。	
\\	これほどまでに電子メールが広く一般に浸透するとは誰も思わなかったんじゃないかな。	
\\	自宅の電話、ファックス、携帯の番号にメールアドレスなんて全部覚えてられないよ。	
\\	4コマ目の授業終了後、皆で飲みに行きませんか。参加できる人は知らせてください。折り返し、集合場所と時間を連絡しますので。	
\\	バースデーカードをメールでもらうっていうのもなかなかいいものだ。	
\\	お忙しい中、リプライありがとうございます。	
\\	この件に関するあなたの意見を是非お聞かせ下さい。お返事お待ちしています。	
\\	最近いたずらメールが届いて困ってるんだ。	
\\	留学するための申込書や資料をメールで請求できるのでとても助かるよ。	
\\	メールでクレジットカードの番号を送っても大丈夫なのかな。	
\\	誰が読むことになるか分からないので、メールで重要な情報は送れないね。	
\\	このメールは事前に知らされているパスワードを入力しないと読めないようになってるんだ。	
\\	本日午後の会議が明日に延期になりましたことをお知らせします。時間については決まり次第連絡致します。	
\\	電子メールで文通をしている人の数が増えているらしい。	
\\	先日、メールサーバーがダウンし、メールの送受信ができなくなって大変な思いをした。	
\\	しばらくメールをチェックしなかったら、こんなに未読のメールがたまっている。	
\\	メールを使い始めた頃、誰からもメールが来ないので、自分宛に送ったことがよくあった。	
\\	英語でメールが書けたら、メール友達の輪がもっと広がるだろうな。	
\\	訳の分からないメールが来てる。しかも、ファイルが添付されてる。ウイルスに感染したくないから、開かずに捨てることにしよう。	
\\	メールのいいところは、相手の時間や都合を考えなくていいところだ。	
\\	会社内での私用メールが、大企業などで問題となっている。	
\\	メールはパソコンよりも携帯でやり取りすることが多い。	
\\	最近ではメールアドレスを高値で販売する会社も多い。	
\\	使用料が無料のプロバイダがあるけど、どうやって採算をとっているんだろう。	
\\	あまりに広告メールが来るので、メーラーのフィルタで振り分けることにした。	
\\	電子メールで個人情報なんかを送るときは、暗号化したほうがいいんじゃない?	
\\	メール友だちって最近多いみたいだけど、メールだけでそんなに続くものなのかな?	
\\	形式のメールは送らないでね。読まないで捨てちゃうから。	
\\	ネット上にメールアドレスを公開するなら、無料のフリーメールなんかを使うといいよ。	
\\	旅行中でもメールを見たいなら、ウェブメールに登録するといい。	
\\	メールは1行の長さを短めにして、改行を多く入れたほうが読みやすい。	
\\	添付ファイルが付いている場合、ウイルスに感染していないかどうか気をつけなければいけない。	
\\	あの子は筆まめだから、すぐに返事を返してくれるよ。	
\\	最近は年賀状や暑中見舞いなんかも、メールで送る人が増えてるよね。	
\\	一瞬で送れるのはいいけど、たまに間違って送信しちゃうことがあるんだ。	
\\	容量の大きいいたずらメールが来ていて、ほかのメールもチェックできない状態になってしまった。	
\\	メールが浸透してからというもの、手紙を書くことももらうことも少なくなった。	
\\	件名は正確に書いたほうがいいよ。空欄になってると削除しちゃう人もいるからね。	
\\	メールマガジンをたくさん取りすぎていて、大切なメールが埋もれちゃうんだ。	
\\	僕は暗記が得意ではないから、なかなか英単語が覚えられない。	
\\	語彙を増やす何かいい方法はないものか。	
\\	英語を一からやり直そうと思うんだけど、文法から始めた方がいいかな。	
\\	いざという時になかなか英語が出てこない。まだまだ練習が足りないな。	
\\	中学で3年、高校で3年。大学へ行ったらそこで何年か英語を勉強することになる。でも、英語が話せないのはなぜ。	
\\	へたな英語でもいいから、もっと堂々と話すようにすべきだよね。	
\\	間違いを気にせずに、話すようにしなさい、と言われてもね。やっぱり、気になっちゃうんだよね。	
\\	英語で簡単な自己紹介くらいはできるようにしておかないと。	
\\	この前の
\\	、あまりにひどい点数だったので、悲しくなっちゃったよ。	
\\	勉強の甲斐あって、
\\	のスコアが150点も上がった。	
\\	継続は力なり。英語だってそう。日々の努力が大切なんだ。	
\\	英語を話す時には、一つ一つの音よりも、リズムやイントネーションが大事だよね。	
\\	英語が少し話せるくらいで、偉そうにするなよ。	
\\	アメリカに1年くらい留学したら、英語が流ちょうに話せるようになるのかね。	
\\	英語を聞く方はいいんだけど、話す方はちょっと。	
\\	単語を並べることはできるけど、文で話すことができないんだ。	
\\	ネイティブと話していると、自分の語彙力の無さを痛感させられる。	
\\	英語を話せるようになりたいのなら、日頃からとにかく英語を口にするように心がけないとだめだよ。	
\\	ただ漠然と英語の
\\	を聞き流しているだけではあまり意味がない。	
\\	高いお金を払って英会話学校へ何ヶ月か通ったけれど、成果が上がらなかった。	
\\	英語とコンピュータ、21世紀には欠かせないツールになる。	
\\	インターネットのおかげで、英語に接する機会が増えた。	
\\	今、電子メールを使って英語でカナダの交換留学生と文通してるんだ。	
\\	先月、試しに
\\	を始めて受験したんだけど、思ったよりできたよ。	
\\	600点目指して、現在、猛勉強中です。	
\\	なんでこんな簡単なことさえ英語で言えないの。	
\\	相手に
\\	と言われると、どきっとするよね。	
\\	相手の英語がまるで分からなかったので、ニコニコして分かっている振りをしているしかなかった。	
\\	英語で電話がかかってきて、あわてて切ってしまったことが何度かある。	
\\	新しい単語を一つ覚えるごとに、一つ忘れるみたい。	
\\	よくわからないので、別の言葉で言い換えてくれますか?	
\\	その単語のスペルを教えてくれますか?	
\\	私の英語は間違っていませんか?	
\\	あなたの発音はとてもはっきりしていてわかりやすいです。	
\\	やはり生の英語は聞き取りにくい。	
\\	日本人はなぜ英語が得意じゃないんだろう?	
\\	恥ずかしがらずに会話に参加しなければ英語は上達しないよ。	
\\	英語圏特有のジェスチャーを教えてください。	
\\	イギリス英語の発音は聞き取りにくいなあ。	
\\	ラジオの
\\	の英語のスピードにはついていけないよ。	
\\	語彙を増やす効果的な方法ってないものか。	
\\	毎日英字新聞を読んでリーディング力を磨いている。	
\\	映画を字幕なしで楽しめればいいんだけどなあ。	
\\	旅行したとき、英語が通じなくて困りました。	
\\	恥ずかしながら、英語は片言しかできません。	
\\	旅行のときに必要な会話くらいはできます。	
\\	英語でディベートをしなければなりません。	
\\	ビジネスシーンで要求されるレベルの英語力を身につけたい。	
\\	彼はネイティブと同レベルの英語力があります。	
\\	外国人との会話にはボディー・ランゲージが非常に大切なときもある。	
\\	そろそろ春もののブラウスを買おうかな。	
\\	秋はさまざまな装いが楽しめる季節だね。	
\\	今年の冬に流行るのはこのジャケットだ。	
\\	去年流行った服が、今年はもう着れない。	
\\	なんでもかんでもブランドものを買えばいいってものじゃない。	
\\	彼女の着こなしはなかなか見事だね。	
\\	君はどんな服でもよく似合う。	
\\	彼は服装にはまったく無頓着だね。	
\\	この街には古着屋が多い。	
\\	このパンツに合ったデザインのシャツを買おう。	
\\	もう少し大きいサイズの帽子はありますか?	
\\	オーダーメイドで背広を作りたいのですが。	
\\	同じデザインで色違いのセーターはありますか?	
\\	ハイネックのセーターは首がチクチクして着れないんだ。	
\\	最近はアジアっぽいテイストを取り入れるのがオシャレらしいね。	
\\	これからはストリート系のファッションに注目だ。	
\\	もう少し地味な柄がいいな。	
\\	ピンクは派手すぎるよ。もっとシックなファッションにしたら?	
\\	彼女はピッタリしたヒョウ柄のシャツを着ていた。	
\\	全身黒でまとめると、ちょっとうさん臭いかな	
\\	今日の服装はパリコレのファッションを手本にしてるの。	
\\	その服のおかげでスリムに見えるよ。	
\\	ふだんの彼は革ジャンに革のパンツ、ブーツという格好だ。	
\\	彼女が身につけている小物にセンスが感じられるね。	
\\	縁のあるメガネのほうが似合ってると思うな。	
\\	このピアス、誕生日プレゼントにもらったんだ。かわいいでしょ?	
\\	通販で買ったネックレス、どうしてつけないの?	
\\	その靴はちょっとダサいんじゃない?	
\\	厚底靴を履いていて足をくじいたわ。	
\\	私の彼って、服のセンスがないのよね。	
\\	このイヤリング、ブラウスとコーディネートしてるの。	
\\	せっかくだから、アクセサリーも買っておこう。	
\\	今日の彼、スーツをパリッと着こなしているね。	
\\	服装がきまっていると、気分も軽やかになる。	
\\	値段が高いだけあって、ブランド物はつくりがしっかりしている。	
\\	流行っているからというだけの理由で、人と同じ服は買いたくない。	
\\	何十万円もする古着のジーンズなんて、コレクター以外は買わないよ。	
\\	センスを磨きたかったら、ファッション誌でも買って研究したら?	
\\	マニュアル通りの服装をしても、センスのなさは隠しきれない。	
\\	あのタレント、かわいいけどファッション感覚はゼロだね。	
\\	私の指導教授、まさにダンディな中年って感じなの。	
\\	あのブランドって最近聞かないけど、まだ売っているの?	
\\	このブティック、内装もすごく凝ってておしゃれ。	
\\	どんな服装で出社してもいいけど、清潔感は大切だよ。	
\\	彼女、スーパーモデル目指してるって?日本人じゃ無理だよ。	
\\	不況で紳士服の売上が激減しているらしい。	
\\	やっぱり世界のファッションの発信地はヨーロッパだね。	
\\	色彩感覚がないと、デザイナーにはなれない。	
\\	彼、アパレル産業に就職を希望してるんだって?	
\\	ファッションショーに出品される服を街中で着たら、さぞや浮くだろう。	
\\	残された自然を守って、地球の未来を明るいものにしたい。	
\\	自然と人間の共生を考えた方がいいかもしれない。	
\\	自然との触れ合いが大切だ。	
\\	森の自然を守ろう。	
\\	ごみはきちんとごみ箱に棄てましょう。	
\\	環境汚染が深刻になっています。	
\\	地球温暖化の原因となる二酸化炭素を減らすために、紙を再資源化する運動があります。	
\\	景観を損なわないようにするために、建物の建設が禁止されている区域があります。	
\\	環境ホルモンが問題になっています。	
\\	再利用できるように、ごみをきちんとと振り分けよう。	
\\	紙を節約するために、裏紙を使わないとね。	
\\	地球に優しい環境作りが大切だ。	
\\	酸性雨で森林に被害が出ているようだ。	
\\	光化学スモッグが出ているから、外には出ない方がいいって。	
\\	富士山がごみの山になっています。ごみのポイ捨てはやめよう。	
\\	オゾンホールがどんどん拡大している。	
\\	ごみを減らそう。	
\\	自然の中で思い切り遊んだ少年時代が懐かしい。	
\\	自然ともっと仲良くなろう!	
\\	森の空気を吸っていると、そう快な気分になるね。	
\\	温泉に漬かって、心身ともにリフレッシュ!	
\\	露天風呂は健康管理にいいみたい。	
\\	森を見ていると心がいやされます。	
\\	地球上の生き物が1年間で何万種類も絶滅しています。	
\\	遺伝子組み換え食品が環境に与える影響について議論されている。	
\\	オゾン層がフロンによって破壊されている。	
\\	室温でしばらく放置した生ものを食べたら、食中毒になりました。	
\\	ゴキブリっていろいろなウイルスをまき散らすんだって。	
\\	俳句は季語を使って、日本の四季を見事に表している。	
\\	風力発電は自然の風を利用したものです。	
\\	ゴミの分別方法がわからない人も多いようです。	
\\	明日は野鳥の観察に出かけます。	
\\	革命でペーパレス化が進むのだろうか。	
\\	最近の子どもは自然と戯れる機会が少ないようです。	
\\	環境問題への関心が急速に高まっている。	
\\	環境問題は地球規模で考えなければならない。	
\\	大気中に放出された汚染物質が環境に悪影響を与えている。	
\\	一定規模以上の大規模開発には環境アセスメントが義務づけられている。	
\\	限りある資源を大切に。	
\\	青く美しい地球を守ろう。	
\\	字幕なしで英語のセリフが分かるようになりたい。	
\\	何だこの映画。期待はずれもいいとこだ。	
\\	最近はアクション映画ばかりで、もううんざりだよ。	
\\	を使っていないっていうけど、最後のシーンはどうやって撮影したんだろう。	
\\	無名の俳優ばかりだけど、思った以上に楽しめたよ。	
\\	一つ当たると、二匹目のドジョウを狙ってすぐに同じような映画を作るんだから。	
\\	あの監督は、この映画で、一体何を言いたかったんだろう。	
\\	黒沢の映画は、やはり、初期の作品の方が断然いい。	
\\	何かお勧めのビデオあるかな。	
\\	結末は誰にも言わないでくださいだって?そんなに大した終わり方じゃないじゃない。期待して損した。	
\\	いい俳優が出ているからといって、それがいい映画だとは限らない。	
\\	これがあの監督の20本目の作品なんだって。	
\\	アカデミー賞作品賞最有力の前評判が高いと聞いて見たんだけど、大した映画ではなかった。	
\\	しばらく見ない内に、あの女優、随分と老けちゃったね。	
\\	ストーリー自体はなかなかよかったけど、3時間半はちょっときつかったな。	
\\	学生時代に見た映画を見ると、その当時のことをいろいろと思い出すよね。	
\\	ベストセラーを映画化しても、原作にはとうてい及ばないことが多いよね。	
\\	あのラストシーンは、切なくて、涙が止まらなかったよ。	
\\	映画を見ている最中に、お菓子を食べるの止めてくれないかね。	
\\	次の回はお立ち見だってさ。もう少し早く来ればよかった。	
\\	映画の台本て覚えるの大変なんだろうな。僕には絶対無理だ。	
\\	お金をかければいい映画ができるってわけじゃないだろう。	
\\	配役も良くないし、演出も今一。
\\	級映画だね。	
\\	昔は2本立ての映画をよく見たけど、今はとても無理だね。お尻が痛くなちゃって。	
\\	すいている時間帯は指定席を解放すればいいのに。	
\\	映画が始まる前の、長い宣伝や予告編どうにかならないかね。	
\\	映画を見るには、平日の午前中が空いていていいよね。	
\\	映画館でいちゃいちゃするなよ。映画を見に来てるんだろ。	
\\	昔は映画館の中を歩いてお菓子とかアイスクリームを売りに来たけどな。	
\\	映画館に入るのに一時間も列に並んだんだ。	
\\	この映画の配給会社はどこ?	
\\	日本の映画館の入場料金は高すぎるよ。	
\\	国際映画祭で入賞したからって、興業的に成功するとは限らない。	
\\	この映画館、自主製作の作品ばかり上映しているんだ。	
\\	昔は、名画座がたくさんあったのにね。	
\\	邦画の黄金時代はもう遠い昔のことだ。	
\\	日本でもシネマコンプレックスが定着してきたね。	
\\	日本のアニメ映画は海外にも根強いファンが多い。	
\\	試写会の招待状もらったんだけど、いっしょに行く?	
\\	この映画、国内での観客動員数の記録を更新中らしい。	
\\	ホラー映画の見どころの1つは、特殊効果だよね。	
\\	リメイク版のほうが、オリジナルよりずっと完成度が高いね。	
\\	ハッピーエンドじゃない映画はどうも苦手だ。	
\\	ドタバタコメディなんかもうたくさんだよ。	
\\	字幕がセリフと一致してないじゃないか。	
\\	往年のアクションスターが、演技派としてカムバックした。	
\\	国際的にも評価されている若手俳優が主演している。	
\\	脇役が完全に主役を食ってるね。	
\\	ただのセクシー女優かと思ったら、けっこう演技もうまいじゃない。	
\\	映画の上映中に携帯でメールを読んでるなんて信じられない。	
\\	ひさしぶりにあのラーメン屋に行こう。	
\\	あそこのレストランで食中毒になった人が3人以上いるのに、なんで営業停止にならないのかな。	
\\	あのレストランのチェーン店がこんどこの町にもできるらしいよ。	
\\	君は本当にあのお好み焼き屋が好きだね。	
\\	ここのデパート・レストランは超まずいよ。	
\\	ここのランチセットは高いけどおいしいよ。	
\\	わざわざあんなところまでラーメン食べに行くことないのに。	
\\	よく食事をしながらお酒を飲めるね。	
\\	またこの店なの。たまにはもっといいものが食べたいな。	
\\	今夜ディナーに行くっていたのになんでお金を持ってないの。	
\\	ワリカンだからな。絶対におごらないぞ。	
\\	あそこのウニ丼を食べたことないなんて人生の半分は損しているよ。	
\\	ここのフランス料理屋に入ってみたいんだけど、いくらとられるか怖くって入れないよ。	
\\	今日一緒にご飯食べませんか。	
\\	あのケチが寿司をおごってくれるっていうからついてったら、回転寿司だったよ。	
\\	やばい今月は食事に金をかけすぎた。来月は貧食になりそう。	
\\	あんたいいとこのおぼっちゃんだから、こんな庶民的定食屋なんて入ったことないでしょう。	
\\	好き嫌いが多すぎるんだよ。一緒に食事にもいけないよ。	
\\	もうしわけありませんが予約は1カ月先まで埋まっております。	
\\	かなり痩せちゃったよ。最近、ちゃんと食べてないから。	
\\	雨も降ってることだし、今日の夕食は近場ですませよう。	
\\	コンビニ弁当ばっかりじゃ、さすがに飽きるな。	
\\	先週から食欲がまったくないんだ。	
\\	今日はご飯ものが食べたいな。	
\\	麺類じゃおなかいっぱいにならないよ。	
\\	朝は忙しいので、朝食はいつも抜いています。	
\\	1日3食しっかり食べないと体を壊すよ。	
\\	食べ過ぎで気持ち悪くなったみたいだ。	
\\	食事中にげっぷなんかしないでよね。	
\\	ちゃんとおわんを持って食べなさい。	
\\	よくかんで食べなさい。	
\\	後片付けを手伝ってちょうだい。	
\\	本当によく食べるわね。	
\\	ライス大盛りにしてください。	
\\	この店はライスと野菜がお替わり自由なんだ。	
\\	この前おごってもらったから、今日はおごるよ。	
\\	店の外まで人が並んでいるよ。	
\\	客がまったく入っていないとさすがに入りづらいな。	
\\	今日、あの店って定休日だっけ?	
\\	カロリーの高そうな食事だな。	
\\	食後のデザートは何かな?	
\\	この程度の仕事はぼくにとって朝飯前さ。	
\\	間食ばかりしてると太るよ。	
\\	このお菓子、変わった食感だね。	
\\	彼はたいへんな美食家だ。	
\\	パーティで残飯がたくさん出た。	
\\	社長主催の晩餐会に出席するから、正装しなくちゃ。	
\\	観劇の前に、レストランで腹ごしらえしておこう。	
\\	立食パーティーの料理って、あまりおいしくないんだよね。	
\\	ベジタリアン向けのメニューはありますか?	
\\	夜食にお茶漬けが食べたい。	
\\	ランチは食べ放題で1500円です。	
\\	夕ごはんはお総菜を買って簡単に済まそう。	
\\	パーティに行くから、この機会に食いだめしておこう。	
\\	こんなに食べ散らかして。片づける人の身にもなってよ。	
\\	この子、もう離乳食を食べているの?	
\\	病院の流動食にはもううんざりだ。	
\\	しまった。ケーキを盗み食いしたのが店長にバレた!	
\\	病気を治したければ、栄養のバランスがとれた食事を続けなさい。	
\\	好き嫌いばかり言ってないで、ぜんぶ食べなさい。	
\\	3日でクリアしました。	
\\	プレイしてみる価値ありです。	
\\	やってみたら、おもいっきりハマりました。	
\\	前作をプレイしていなくても、十分に楽しめました。	
\\	これぞ正統派RPG。	
\\	中古ソフトを購入しました。	
\\	操作性がいまいちです。	
\\	さっそく予約しました。	
\\	送料別でソフトを譲ります。	
\\	やったー、ハイスコアを獲得	
\\	期待してたのに大ハズレでした。	
\\	徹夜で並んで発売日に買いました。	
\\	個人的にはPC版の方が好きです。	
\\	裏技を見つけました。	
\\	戦闘場面の音楽はカッコ良かった。	
\\	新作ゲームの発売が待ちどおしいです。	
\\	音楽もシナリオもとてもよかったです。	
\\	欲しかったゲームが発売延期になった。	
\\	テンポがいいので、サクサク進めます。	
\\	続きが知りたいな、続編は出ないのかな。	
\\	何かオススメがあったら教えてください。	
\\	昔懐かしのインベーダゲームに夢中です。	
\\	グラフィックスや
\\	のセンスが最高でした。	
\\	体験版をプレイしてみたけど、迫力満点でした。	
\\	大作もいいけど、隠れた名作も味があっていいよ。	
\\	ゲームのやりすぎは目が疲れるので、ほどほどに。	
\\	発売から2年近く経っても、今だに人気があります。	
\\	1人で遊ぶよりもネットワークプレイの方が盛り上がるよ。	
\\	アーケード版からの完全移植なので、買って損はないです。	
\\	スタートボタンを同時に押すと、メニュー画面に戻ります。	
\\	画面がぐるぐる回って気持ち悪くなったよ。	
\\	グラフィックもストーリーもよくて、まるで映画を見ているような感じだ。	
\\	このゲームはバグが多すぎて、すぐにフリーズする。	
\\	最近はユーザーの要望が高すぎて、ゲームソフト会社はそれに応えるゲームを作れない。	
\\	たくさんゲーム機が出ているけど、何を基準に買えばいいのだろう。	
\\	昔は人気ソフトがすぐに売り切れになってしまったけど、最近はコンビニでも買えるようになった。	
\\	最近のソフトは似たものばかりで面白くないよ。	
\\	中古ソフト屋にゲームを売りに行ったけど、500円にしかならなかった。	
\\	ソフトが1本5000円なんて当たり前。中には1万円するのだってあるよ。	
\\	1カ月前、6000円だったゲームがこの間1980円で売ってた。	
\\	大人気のゲームが映画化されるんだって。どんな内容なんだろう?	
\\	ネットワークゲームで見知らぬ人と対戦して、そんなに楽しいの?	
\\	のゲームはさすがに迫力がある。	
\\	携帯電話のゲームなら、いつでもどこでも楽しめる。	
\\	シミュレーションゲームって、時間がかかるし頭を使うから嫌い。	
\\	せっかくの休みに、家でゲームしてばかりじゃもったいないよ。	
\\	ゲームで負けたほうが食事を作るってのはどう?	
\\	ゲームは1日30分まで。終わったら勉強しなさい。	
\\	ブラックジャック、ポーカーそれともブリッジにする?	
\\	彼はチェスはめっぽう強いが将棋はそれほどでもない。	
\\	前髪を3センチほど切ってください。	
\\	後ろ髪を10センチほど切ってください。	
\\	後ろは刈り上げてください。	
\\	スポーツ刈りにしてください。	
\\	今度、ストレートパーマをかけようかと思ってます。	
\\	前髪を眉毛にかかる程度にしてください。	
\\	髪を伸ばしているので後ろ髪は、肩までで止めてください	
\\	最後に床屋に来たのは、三ヶ月前です。	
\\	切った毛が背中の中に入ってくすぐったいです。	
\\	眉毛を細くしてください。	
\\	もみあげを伸ばしているので、そろえるだけで結構です。	
\\	一度でいいから、モヒカン刈りにしてみたい。	
\\	髪を緑や赤に染めるのが流行っている。	
\\	天然パーマなので、セットするのが大変です。	
\\	何か、整髪剤はつけますか?	
\\	パンチパーマをかけるのにはどれくらい時間がかかりますか?	
\\	カットの後のマッサージがとても気持ちいい。	
\\	床屋さんから、たまにはくしで髪をとかすように言われた。	
\\	最近、男性のポニーテールが増えてきた。	
\\	絶対にスキンヘッドにはしたくない。	
\\	カツラをつけているのがばれた。	
\\	ドライヤーで前髪をカールしてください。	
\\	床屋さんが、うっかりして僕の眉毛を剃ってしまった。	
\\	三つ編みをして、美容院にいったらいやな顔をされた	
\\	ストレートパーマは人によってはかかりませんよ。	
\\	ワンレンにしてください。	
\\	カットだけならいくらですか?	
\\	パーマとカットの両方はいくらですか?	
\\	パーマをかけたいのですが、どれくらいの時間がかかりますか?	
\\	何カ月おきに床屋に行く?	
\\	10代の頃から髪型は変わっていません。	
\\	思い切って、イメチェンしてみれば!	
\\	床屋によっては、耳の掃除もしてくれる。	
\\	床屋って、切った髪の毛どうしてるのかね?	
\\	髪が多いので、少しすいてください。	
\\	力士のちょんまげって本当の髪?	
\\	どうしてひげ剃りの前に顔を温めるの?	
\\	床屋さんになるには資格がいるの?	
\\	美容師さんによっては、カットにつき1万円以上もらっている人がいる。	
\\	私の髪は剛毛です。	
\\	最近白髪が増えてきたので、黒く染めました。	
\\	髪の毛って1ヶ月にどれくらい延びるの?	
\\	お気に入りの美容師を指名することができますよ。	
\\	髪の毛を切ってもらっている時に寝たことある?	
\\	全体的にムースをつけると髪が落ち着きます。	
\\	もしよろしければ、カットモデルになっていただけませんか?	
\\	ヘアカタログの中から、お好みの髪型をお選びください。	
\\	ふわっとした感じのゆるいパーマをかけてください。	
\\	彼女はベリーショートがとてもよく似合う。	
\\	雨の日などで湿気が多いと髪が広がって大変です。	
\\	あなたのカットやパーマの履歴はカルテとして保存されています。	
\\	ジェルをつけて一つにまとめるときりっとした感じになりますよ。	
\\	傷んだ髪に優しいシャンプーはありませんか?	
\\	数年前、ワンレングスの長い髪が流行りましたね。	
\\	この前、段カットにした段が伸びて、まとまらなくなりました。	
\\	ヘアマニキュアをかけると髪が傷まないそうですね。	
\\	サイドをすっきりさせたいので編み込みをお願いします。	
\\	枝毛ができてしまったので、先の傷んだところだけ切ってそろえてください。	
\\	時々ブラッシングすると髪がつやつやしてきます。	
\\	二本の三つ編みにして、前髪は薄くたらしてください。	
\\	前髪は揃えないでください	
\\	短く刈ってください。	
\\	相手に自分の希望する髪型を伝えるのは難しい。	
\\	最近、自動洗髪機が増えたよね。	
\\	シャンプーされる時って、なんだか緊張する。	
\\	ドレッドヘアーにしたら頭が洗えないな。	
\\	ショートヘアーもお似合いですね。	
\\	ちゃんとトリートメントをしないと髪はすぐ痛みますよ。	
\\	自分で髪を染めればコストがかからない。美容院ならカラーがたくさんあってきれいに染まる。	
\\	美容院に行くと、いつも4時間くらいかかっちゃうんだよね。	
\\	小学生が美容院に行っているよ。時代は変わったな。	
\\	気分転換に美容院でも行こうかな。	
\\	あそこの美容院は技術はよくないし、態度も悪い。もう二度と行かない。	
\\	そのヘアースタイルだと10歳は若く見える。	
\\	ここは特に美容師のセンスがいいと評判のお店です。	
\\	彼は、カリスマ美容師としてテレビでもよく取りあげられる。	
\\	家の近くに美容院がないから本当に困るよ。	
\\	あそこの美容院は完全予約制だから、電話をしてから行ったほうがいいよ。	
\\	昔、床屋だったあそこの店が美容院になってたよ。	
\\	美容師って、自分の髪は自分で切っているのかな。	
\\	最近忙しくてどこにも行ってない、貧乏暇なしとはよくいったもんだ。	
\\	古本屋で本を売ってもお金にならないので、いつも損した気分になる。	
\\	偉人の中には貧しい家庭から育った者も多い。	
\\	コンパに行くお金がないから貸してくれない?	
\\	貧しくても愛があればいい。	
\\	このまま不景気が続けば、私たち貧乏人はどうなるの?	
\\	ああ、はら減った。	
\\	貧乏脱出のためにお金をためよう。	
\\	値引きシールが張られる時間帯に買い物に行きます。	
\\	競馬何てやるんじゃなかった。スッテンテンになちゃったよ。	
\\	ヒッチハイクで旅行費を節約しなくっちゃ。	
\\	こう見えても、昔は貧乏学生をやっていたんだ。	
\\	毎日カレーばかりでいやになっちゃう。	
\\	いいバイトを紹介してもらえませんか?	
\\	今年のボーナスはあまり期待ができない。	
\\	家賃がたまって、大家に会わす顔がないよ。	
\\	スーパーのチラシに目を通し、安いものをチェックしないとね。	
\\	大金は持ち歩かないのが節約のコツです。	
\\	切り詰めてもなぜか赤字、どうしてだろう?	
\\	今月もやりくり頑張るぞ。	
\\	衝動買いはダメだとわかっていても、ついつい買ってしまう。	
\\	金欠なので、現状の生活を維持するだけで精一杯です。	
\\	無駄使いをしてしまったので、今月はピンチだ。	
\\	食費を浮かす方法を教えてくれませんか。	
\\	借金の返済のために一生懸命働いています。	
\\	使わない電化製品はコンセントを抜けば、節電できます。	
\\	節約するには家計簿をつけるのがいいよ。	
\\	笑う角には福来る、貧乏でも明るくいこう。	
\\	電気が止まるとさすがにキツイ。	
\\	日本は物価が非常に高いので、どうにかしてほしい。	
\\	いくら貧乏でも心まで貧しくなったらおしまいだ。	
\\	お金がないと心に余裕がなくなる。	
\\	割りばしは洗えば何度でも使える。	
\\	500円であと3日を乗りきらなくちゃならないんだ。	
\\	今はお金がないから、新しい服を買う余裕はない。	
\\	あいつにだけは金を借りたくなかったけど仕方ない。	
\\	数日間、何も食べてないから頬がこけてしまった。	
\\	3度の食事すら満足に取れない人たちが世の中にはたくさんいます。	
\\	定職がなく、その日暮らしの生活が続いている。	
\\	財布の中身がからっぽだ。今日1日どうやって過ごそう。	
\\	こんなに貧しくては、デートもできやしない。	
\\	金の切れ目が縁の切れ目。	
\\	貧すれば鈍する。	
\\	落ちたものを食べるなんて貧乏臭い真似はやめろ。	
\\	水を飲んで、お腹をふくらませよう。	
\\	今はとにかく我慢して、お金を貯めるしか手はない。	
\\	今の願いは、お腹いっぱい焼肉を食べることです。	
\\	借金が返済できず、家が差し押さえられてしまった。	
\\	不況のせいか、貧乏を苦に自殺する人が増えているようだ。	
\\	もはや、宝くじで一攫千金を狙うしか道はない。	
\\	彼の趣味は道楽を越えている。	
\\	無趣味の人が多すぎる。	
\\	同じ趣味を持っている人たちで同好会を作った。	
\\	趣味がこうじてそれが職業になった。	
\\	趣味をも持つことボケの防止になるそうだ。	
\\	趣味を職業にしない方がいいよ。	
\\	同じ趣味を持った人といる方が気が楽だ。	
\\	同じ趣味がきっかけで、私たちは結婚しました。	
\\	世の中いろんな趣味があるものだね。	
\\	趣味を通じでいろんな人と出会った。	
\\	同じ趣味の人とどうやって知り合えるのかな?	
\\	趣味なんてなくても生きていけるよ。	
\\	バードウォッチングが趣味なんてきっと彼はつまらない人だよ。	
\\	趣味ばっかりでは生きていけないでしょ。	
\\	趣味の欄にはなんて書こう?	
\\	お金のかからない趣味でいいね。	
\\	私は多趣味でお金がいくらあってもたらない。	
\\	私は共通の趣味の友達がいない。	
\\	無趣味が趣味です。	
\\	彼の腕前は趣味を越えている。	
\\	お見合の席で趣味を聞かれてもなんて答えればいいのかしら。	
\\	この趣味、変だよ。	
\\	趣味といえば読書に映画鑑賞だよね。	
\\	趣味といえば、何か収集することじゃないかしら。	
\\	旅行が趣味なんていいわね。	
\\	趣味がまた増えた。	
\\	老後に趣味がないなんてさびしいよ。	
\\	趣味をもっても、仕事が忙しくて何もできない。	
\\	趣味を生き甲斐にしてます。	
\\	ホント多趣味よね。	
\\	今時「ご趣味は?」って聞かれてもね?。	
\\	趣味という共通点があると話がしやすい。	
\\	彼の趣味と私の趣味は全然違うので余暇の過ごし方がまったく違う。	
\\	ロッククライミングが趣味なんて、危険だね。	
\\	彼の場合、何が趣味で何が趣味でないのかわからない。	
\\	趣味で人がわかるよね。	
\\	趣味が同じだと、理解しやすいのかな?	
\\	趣味ばっかりで、仕事しない!	
\\	趣味を通じて、年間何百人といった人と出会える。	
\\	趣味が身を助けたってカンジだね。	
\\	趣味と言って謙遜してるけど、彼のは立派な芸術作品に仕上がっているよ。	
\\	もうすぐ趣味じゃなくなるけど、趣味でやっていたようがよかったかも・・・。	
\\	趣味で彼は父親の財産を食いつぶした。	
\\	趣味といえども、彼の場合はそれが収入源になっている。	
\\	趣味にしてはうますぎる。	
\\	趣味があるってことはいいことだよ。	
\\	サッカーが趣味だとか言っているけど、怪しいものだよ。	
\\	かっこいい趣味をお持ちですね。	
\\	趣味と実益を兼ねてますから。	
\\	今さら趣味なんていわれても何をすればいいのかわからない・・・。	
\\	部屋が汚すぎるぞ。ちゃんと掃除しているのか。	
\\	この部屋ずいぶんスースーするね。どっかから風が入ってるんじゃないの。	
\\	確かに家賃は安いけど、ものすごくぼろいな。	
\\	一人で住むには十分な広さだ。	
\\	駅から近くて便利だけど、狭いんだよなぁ。	
\\	駅から徒歩20分ってかなり遠いんじゃないか。	
\\	仕事先から遠いけど、広いし、静かだ。	
\\	家のすぐ近くに公園があるのがいいね。	
\\	昼間ですら日が全く当たらないのが難点なんだ。	
\\	日当たりがいいから洗濯物が早く乾く。	
\\	蛇口から水が出ないぞ。どこかで水漏れしてるんじゃないか。	
\\	バス、トイレなしで家賃10万なんて信じられない。	
\\	バス、トイレが一緒だとちょっと不便だな。	
\\	台所が広いのがいいね。料理がしやすいよ。	
\\	もっと大きな冷蔵庫が置けたらいいのに。	
\\	やっぱり洋室よりも和室のほうが落ち着くよ。	
\\	このタンスは大きいからこの部屋じゃなくて、奥の部屋に置くことにしよう。	
\\	狭いマンションなんかじゃなくて、やっぱり一戸建ての家に住みたいね。	
\\	将来は庭つきの家に住みたいな。	
\\	庭がもっと広ければ犬を買ってもいいんだけど。	
\\	木々に囲まれた家っていいよね。	
\\	家に車を置けるスペースがあると便利だね。	
\\	家のローンを30年で組んだけど、これから先が大変だな。	
\\	二世帯住居になるんだったら2階建てじゃないと。	
\\	子供も大きくなってきたし、少なくても3部屋くらい欲しいな。	
\\	二日酔いで頭がガンガンしている。	
\\	風邪かな、頭がズキズキ痛むよ。	
\\	かき氷を一気に食べたら、頭にツーンときた。	
\\	私は頭痛持ちです。	
\\	頭が割れるように痛い。	
\\	パソコンにずっと向かってると、目がかすんでくる。	
\\	結膜炎になると目やにが出る。	
\\	最近、夜になると目が見えにくくなるから、暗くなってから車を運転するのが怖い。	
\\	彼は近視で乱視のうえに、最近では老眼にもなってきたらしいよ。	
\\	色盲って、女性には遺伝しにくいんだよ。	
\\	耳元で大きな音がしたから、耳がキーンとしちゃったよ。	
\\	ものを噛むときに自分の舌も噛んじゃって、切れるかと思った。	
\\	親不知は早めに処置しないと、ものが挟まって虫歯になる。	
\\	しっかり歯を磨かないと、歯の裏側に歯垢がつく。	
\\	キャラメルを食べたら、奥歯の詰め物がとれちゃった。	
\\	歯医者で「ギーギー」と歯を削られるときって、何だかとても恐怖心を覚える。	
\\	喉がはれていて、ものを飲み込むとき痛い。	
\\	私の子どもはぜんそくで、具合が悪くなると夜中にヒューヒューと苦しそうにしている。	
\\	彼ったら、あんなに深い咳をして……。 もしかして結核?	
\\	ストレスかなあ、胃がきりきり痛むよ。	
\\	胃の調子が悪いので検査してもらったら、胃潰瘍と診断された。	
\\	それって、まさか胃ガンだということを医者が隠しているわけじゃないよね。	
\\	胃カメラって飲んだことないけど、あんなものを入れられるのはいやだな。	
\\	健康診断でバリウムを飲まされたけど、あれを飲んで「ゲップをするな」っていうのは酷だよなあ。	
\\	採血された跡があざになっちゃった。	
\\	肩こっちゃった、もんでくれる?	
\\	すごいこってるよ。ガチガチだよ。	
\\	すぐ肩がこるっていうのは、メガネの度があってないんじゃないの?	
\\	お腹がしくしく痛む。	
\\	食べ物にあたったかな、下痢がひどいんだよ。	
\\	風邪で鼻をかみすぎて鼻の下が赤くなっちゃった。	
\\	花粉症で鼻水がとまらないから、集中力も出ないよ。	
\\	最近、腰痛がひどいんだけど、椎間板ヘルニアにでもなってるのかなあ。	
\\	重たいものを持ち上げたらギックリ腰になっちゃった。	
\\	蚊に刺されてところがかゆくてかゆくて。	
\\	蚊に刺されたところをかきむしっちゃって、痛がゆいよ。	
\\	私は乾燥肌なので、冬場は寝ている間に背中や腕につめの跡が残るくらいかきむしってしまう。	
\\	キャンプに行ったら、植物でかぶれちゃった。	
\\	一日遅れて筋肉痛が出たんだ。歳かなあ。	
\\	膝に水がたまってしまったらしい。	
\\	正座を長時間すると、足がしびれて立ち上がれなくなる。	
\\	バレーボールをやっていて、足首をひねって捻挫した。	
\\	ボールを取り損ねて、突き指しちゃった。	
\\	骨折でギブスをはめられたら、不自由このうえない。	
\\	ギブスの間に定規を入れて、中をかくのって気持ちいいんだよ。	
\\	朝礼のとき貧血で倒れちゃう子どもが多いらしい。	
\\	夏場は脱水症に気を付けなければならない。	
\\	過呼吸って全身にけいれんを起こすんだよ。	
\\	過呼吸になったら、ビニールなどを口にあてて呼吸ができなくなるようにします。	
\\	おぼれた人を助けあげたら、まず気道を確保して人工呼吸を施します。	
\\	テーピングは、本来怪我しているところを保護するためのものではなく、怪我を予防するために使うのです。	
\\	机の角などに肘をぶつけると、ビリっと電気が走るよね。	
\\	うちの子どもはすり傷がたえないよ。	
\\	レーガン元大統領のなったアルツハイマー病って、日常生活にも支障をきたすくらい深刻な脳の病気らしい。	
\\	最近、脳死判定から臓器移植を実施した事例がいくつか報告されている。	
\\	血液疾患(白血病など)の増加に伴う骨髄提供者の必要性が問われている。	
\\	今年のインフルエンザの流行は
\\	香港型だ。	
\\	昨年、一昨年と病原性大腸菌
\\	-157の猛威はすごかった。	
\\	肥満になると、心臓にかかる負担が大きくなるらしい。	
\\	階段を一気に駆け上がったら、心臓がバクバクしている。	
\\	低血圧だと朝起きられないっていうのは、何の根拠もないことだ。	
\\	朝から何も食べてないから、血糖値が下がってるだろうなあ。	
\\	エイズとは、
\\	に感染して起こる病気で、感染すると身体の抵抗力が低下し、さまざまな感染症や悪性腫瘍にかかってしまう。	
\\	その感染経路は、性的接触、血液感染、母子感染の3つで、ペットや虫を介してあるいは唾液や汗、涙では感染しない。	
\\	私の友人は骨肉腫で亡くなったんだけど、骨肉腫とは骨の癌だ。	
\\	彼のお父さんは脳梗塞で倒れて、半身不随になった。	
\\	知人のおばあさんが老人性痴呆症になって、たばこの吸いがらを食べちゃったりしてたいへんだったとのことだ。	
\\	最近の中高年はリストラなどで精神的に病んでいる人が多い。	
\\	寝ているときに金縛りにあったり、足がつったりするのって何が原因なのだろう。	
\\	深爪しちゃって、指先が痛い。	
\\	体育館で転んで膝をすったら、とてもヒリヒリする。	
\\	熱いシチューを急いですすったら、口の中をやけどしてしまった。	
\\	コンタクトレンズだから、風の強い日は目にゴミが入ってつらいのです。	
\\	聞いた話だけど、コンタクトレンズをつけたまま寝ちゃうと、眼球の裏側にレンズが回っちゃうことがあるんだって。	
\\	傷口がジクジクしている。	
\\	かさぶたを無理矢理はがすと、また血が出てきちゃうんだよね。	
\\	海辺で寝てしまったら、全身やけど状態になっちゃった。	
\\	日に焼きすぎると、着ているものがすれるだけで激痛が走る。	
\\	日焼け後に皮がむけてきて、それをきれいにはがすと快感だ。	
\\	彼は先生に殴られて、鼓膜が破れてしまったということだ。	
\\	歯肉炎になったらしく、口臭が気になる。	
\\	五木寛之氏は、その著書の中で「人間は生まれたときから死のキャリアである」と述べている。	
\\	問題ばっかり噴出して、まったく頭が痛い。	
\\	飲むと暴れるっていう、彼の病気がはじまった。	
\\	私はすねに傷を持っています。	
\\	こういう湿気が多くて気圧が低い日には腰や関節が痛むのよね。	
\\	睡眠不足のせいで立ちくらみがする。	
\\	細かい字ばかり見ていたので、目の焦点が合わなくなってきた。	
\\	ハウスダストが原因でアトピー性皮膚炎を起こした。	
\\	去年辺りから急に花粉症になっちゃって。	
\\	うちは母も祖母もリューマチだったから、私のリューマチは遺伝かな。	
\\	腰痛が背中まで広がって、鉄板が入っているみたいにガチガチになった。	
\\	目の疲れが頭痛や肩凝りの原因になる。	
\\	疲れがたまってくると親知らずが痛みだす。	
\\	最初はただの風邪だと思っていたんだけど、結局風疹だった。	
\\	風邪は万病の元だよ。	
\\	風邪がようやく治ったと思ったらまたぶり返しちゃって。	
\\	手にまめができちゃって。おおきいのが。	
\\	喉が少し腫れちゃってるんだ。	
\\	人差し指にささくれがあったんで、それをひっぱったら、皮がむけて、血がにじんできた。	
\\	やっぱり、見合い結婚より恋愛結婚だよな。	
\\	披露宴の日取りが決まんないよ。	
\\	最近はできちゃった婚が多いよな。	
\\	結婚したあと、親と同居するらしいよ。	
\\	結婚したって幸せになれるとは限らないよ。	
\\	婚約指輪の相場って給料3カ月分だっけ。	
\\	彼女って、あの若さでバツイチなんだって。	
\\	ウエディングドレスを選ぶのに、何でそんな時間がかかるの。	
\\	外国人と結婚すると、いろいろ問題がありそう。	
\\	夫婦円満のために心掛けていることって何?	
\\	結婚をしても、しばらくは共稼ぎです。	
\\	結婚式でブーケを3回も取ったのにまだ結婚できないよ。	
\\	まさか幼なじみの京子と結婚するとは思わなかったよ。	
\\	そんなに落ち込むな。結婚がすべてじゃないぞ。	
\\	華やかな結婚式をすると、もし離婚したら変な目でみられそう。	
\\	最近は結婚しない女性が増えてるよね。	
\\	結婚して半年経つが、あの二人うまくやっているのかなあ。	
\\	あの2人、結婚したかと思ったら、あっと言う間に離婚してしまった。	
\\	彼女は今だに白馬に乗った王子様を待っているんだよ。	
\\	ハネムーンで世界一周なんて、夢の夢ね。	
\\	相手の父親が結婚に反対しているんだ。どうやって説得したらいいかな。	
\\	早く結婚したいと思ったので、相手を紹介してくれる会社に登録してみた。	
\\	高学歴の女性ほど、遅く結婚する傾向がある。	
\\	あまり理想ばかりを追っているから、なかなか結婚できないんだよ。	
\\	家事を公平に分担してくれる人でないと、結婚はしたくない。	
\\	いまどき結婚相手の条件に高学歴、高収入、高身長を挙げるなんて、ぜいたくじゃない?	
\\	彼女、バツイチだって知ってた?	
\\	いなかでは、まだまだ結納の習慣が残っている。	
\\	不況だから、結婚式にかける費用を抑えるカップルが多い。	
\\	彼女、来週結婚式なのに元気がない。マリッジブルーってやつかな。	
\\	身内だけで結婚式を挙げるカップルが増えているそうだ。	
\\	花婿が結婚式で大泣きしてた。彼女と結婚できてよほどうれしかったんだね。	
\\	結婚はしたけど、便宜上、職場では旧姓で通している。	
\\	結婚生活の暴露本を出版する芸能人の気持ちはわからない。	
\\	最初からピッタリの人なんていないよ。一緒に暮らしながらなじんでいくんだ。	
\\	結婚は人生の墓場だっていうけど、結婚して初めてその意味がわかった。	
\\	彼らは今、別居している。	
\\	結婚したら自由に使えるお金が減ってしまった。こんなはずじゃなかったのに。	
\\	彼は結婚に一度失敗しているので、再婚には慎重になっている。	
\\	熟年離婚が増えている。人生をやり直したい人が多いんだな。	
\\	今月は収入が少なかったから財布のひもをしめよう。	
\\	どこの家庭も奥さんが財布のひもを握っている。	
\\	所詮、ギャンブルで儲けたお金なんてあぶく銭だよ。	
\\	銀行に預けても金利が安いからな~	
\\	あの人はけちだからおごってくれないだろうな~	
\\	あの人は太っ腹でうらやましいよ。	
\\	お金のことは心配しなくてもいいからなんでも好きなものを買っていいよ。	
\\	お金はコツコツためないとたまらないよ。	
\\	今月も無駄遣いしちゃったな~	
\\	あ~あ相変わらずふところが寒いな~	
\\	楽して儲かる方法なんてないよ。	
\\	へそくり場所は見つからないようにしないと。	
\\	なけなしのお金を払ってチケットを購入した。	
\\	今年のボーナスは雀の涙ほどだった。	
\\	一攫千金を狙って宝くじを購入した。	
\\	ローンの返済が毎月きついな。	
\\	世の中にはお金で買えないものもたくさんある。	
\\	今月も1万円の赤字だ。	
\\	このまま赤字つづくと会社は倒産だ。	
\\	コインランドリーに行くから小銭が必要だ。	
\\	これが1万円なんてぼったくりだ。	
\\	いま持ち合わせがないから立て替えといて。	
\\	このテレビ番組はずいぶんお金がかかってるな~	
\\	今月は金欠だから外食を控えよう。	
\\	インターネットに夢中になるとあっという間に電話代が高くなっちゃうよ。	
\\	まずはじめに手付金として10万円が必要です。	
\\	お財布の中にあといくらあったかな~	
\\	ボーナスも出たし、どか~んと買っちゃおうかな~	
\\	買うか買わないかは財布の中身と相談だ。	
\\	やっと給料日だ。今月は長かったな~。	
\\	お金を粗末にしたら罰があたるよ。	
\\	来月はバーゲンだし、いまのうちからお金ためとこ。	
\\	ポケットの小銭がじゃらじゃらうるさいな~	
\\	こんなもの買うなんて金銭感覚を疑うよ。	
\\	お金にうるさいと嫌われるよ。
\\	今月は余分な出費がおおかったな~	
\\	今月は金回りがいいな。	
\\	ゲッ、銀行の残高がマイナスになってる!!!	
\\	どうしてお金ってすぐなくなっちゃうんだろう。	
\\	あれも、これも欲しいしお金がいくらあっても足りないよ。	
\\	ご飯食べに行く前に銀行でお金をおろしてくるよ。	
\\	今年の冬で遂に貯金が100万円の大台にのるよ。	
\\	これだけのシステムを構築するにはかなりお金がかかるだろうな~。	
\\	返品したいんですけど送料は負担してもらえますか?	
\\	ここのアルバイトは交通費込みですか?	
\\	残業手当がつかないなんて信じられないよ。	
\\	カードの支払いが来月一気にきちゃうよ。	
\\	お金は計画的に借りないとあとで大変なことになるよ。	
\\	今月はおこづかいを前借りしないとつらいな。	
\\	ビジネスにはお金の話はつきものだよ。	
\\	お金がかかる彼女とはつきあいたくないな~。	
\\	手数料もこれだけ高いとばかにならないな~。	
\\	素人が株に投資したって儲からないよ。	
\\	まだ給料日まで1週間もあるのか~。今月はほんとに金欠だ。	
\\	せっかくの夏休みなのにお金がないからどこにも行けないよ。	
\\	海外旅行に行くために毎月給料の一部を積み立てています。	
\\	彼の音楽の趣味はかたよっているよね。	
\\	家でリラックスするのに何聞いてる?	
\\	頭の中でこの曲がぐるぐるまわってる。	
\\	耳鳴りするぐらい大きいボリュームで聞いていたら、近所の人に怒られた。	
\\	音楽って世界共通語だね。	
\\	音楽の話になると彼女はいきいきするね。	
\\	彼女は音痴だよね。	
\\	ギターの練習しすぎでまめができた。	
\\	彼の音楽の才能は母親ゆずり。	
\\	初めてこの曲を聴いたときは鳥肌がたった。	
\\	このメロディーを聴いていると心がやすらぐ。	
\\	アイドルなんて見かけだけで、ろくすっぽ、歌えやしない。	
\\	この曲の元ネタって知ってる?	
\\	大きな音で聞いていると難聴にななるよ。	
\\	この曲を聴くと若い頃をおもいだす。	
\\	あの
\\	屋に行けば、ほしい物すべてが手にはいる。	
\\	音楽の情報はインターネットで収集してます。	
\\	はやりの音楽はどれもみんな同じに聞こえる。	
\\	この歌詞は心にしみる。	
\\	音楽は自己表現のひとつである。	
\\	彼女の音楽は10年先を行っている。	
\\	やはり音楽は感じるものだろう。	
\\	いい音楽を聴くと踊りたくなる。	
\\	たくさんのバンドがデビューするが、その中で残れるのはほんの一握りの人たちだ。	
\\	最近は
\\	もインターネットで買えるようになって便利になった。	
\\	ショービジネスの世界はほんと下克上だよね。	
\\	だれもが口ずさむメロディーってすごいよね。	
\\	2枚買うと一枚ただなんだって。	
\\	これ、いいってレビューにかいてあったよ。	
\\	音楽好きでバンドをやり始めたけど、やっぱり仕事になると大変だね。	
\\	なんでこんな曲が売れるんだ。今のはやりってわからないな。	
\\	今ではロックは廃れ、
\\	が人気を誇っている。	
\\	あっ、この曲ってタイアップだよね。何の
\\	の曲だっけ。	
\\	曲はいい感じだし、ヴォーカルも演奏もうまい。だけどこのバンド、売れないんだよね。	
\\	あのバンド、不仲がうわさされていたが、やはり解散したね。	
\\	あのバンドのヴォーカルもついに結婚。人気落ちそー。	
\\	海外ではたいして売れてないのに、日本ではすっごく売れてるバンドって多いよね。	
\\	あのバンド、10年ぶりにオリジナルアルバムを出すんだって。	
\\	最近はわけのわからないグループ名が多いよな。	
\\	お笑いの人が歌手デビューする時代になったな。	
\\	最近、若手の演歌歌手が増えてきたね。	
\\	ディナーショーは近くで歌手を見れるのはいいけど値段が高いよね。	
\\	音楽をダウンロードできるようになると、
\\	は買わなくなるね。	
\\	中古
\\	ショップを見てると、新譜を買うのがばかばかしくなるよね。	
\\	インターネットでライブが見られる時代になったな。	
\\	あの人気バンドのコンサートチケットがやっと手に入ったよ。	
\\	この曲を聞くと学生時代を思い出す。	
\\	あの曲をやってくれないと、コンサートに来た気がしない。	
\\	日本はビッグビジネスになるからって、たくさんの海外アーティストが来日してコンサートをするよね。	
\\	コンサートが中止だなんて信じられない。	
\\	最近はストリート・ミュージシャンが増えたね。	
\\	カー・ラジオから大好きな曲が流れてきた。	
\\	大物ミュージシャンのライブで観客が総立ちになっていた。	
\\	彼女、若いころはあるバンドのおっかけやってたんだって。	
\\	ボーカリストがのどを痛めたため、公演は延期になりました。	
\\	古いレコードなので、音があまりよくない。	
\\	あのギター、チューニングが合っていないんじゃない?	
\\	グランド・ピアノが欲しいけど、部屋が狭いしなあ…。	
\\	室内でトランペットの練習をしたいんだけど、近所迷惑かな?	
\\	音楽やってると、モテるんだって?	
\\	ミュージカルに出演するのが夢なんだ。	
\\	オーケストラはやはり生で聞くのが一番だね。	
\\	最近は民族音楽にはまっているんだ。	
\\	レゲエはあの独特のリズムがたまらないよ。	
\\	ジャズはアメリカ生まれの音楽です。	
\\	ビートルズなら、僕はジョンよりポール派だな。	
\\	夏にぴったりの曲を演奏してよ。	
\\	僕の好きな曲を録音したテープを持ってきたよ。	
\\	やっぱりな、あのバンドは売れると思ってたんだ。	
\\	この曲は昔の曲のパクリみたいで、どうも好きじゃない。	
\\	年の瀬が迫ってきました。	
\\	・ 
\\	20世紀も残りあとわずかですね。	
\\	忘年会の幹事がこんなに大変だとは思わなかった。	
\\	お世話になった課長にお歳暮を贈るのを忘れてしまった。	
\\	うちの会社の年末年始の休みは大晦日と三が日だけだ。	
\\	どこの家でも年末の大掃除をやっています。	
\\	大声で客の呼び込みをやっている店が多い。	
\\	来年の干支はなんだっけ?	
\\	・ 
\\	今年が辰年だったから、来年は巳年だね。	
\\	年賀状には年初のあいさつや干支の絵を書くのが一般的です。	
\\	最近は年賀状をパソコンで作る人が増えた。	
\\	お年玉くじつき年賀状の売り上げが落ちている。	
\\	大晦日の晩はテレビで紅白歌合戦を見るのがわが家の習慣です。	
\\	今年は、1999年の年末ほど盛り上がってないみたいだ。	
\\	大晦日はオールナイトで遊ぶぞ。	
\\	初日の出を見に夜中に出かける人が多い。	
\\	年越しイベントが目白押しです。	
\\	こたつに入って年越しそばを食べて、のんびりと年を越します。	
\\	除夜の鐘には煩悩を取り去るという意味があるんだって。	
\\	一年の計は元旦にあり。	
\\	今年がよい一年でありますように。	
\\	・ 
\\	昨年は大変お世話になりました。今年もよろしくお願いします。	
\\	新年早々あいさつ回りに出かけなきゃいけない。	
\\	初詣は近所の神社で済ますつもりだよ。	
\\	三が日の明治神宮への人出はすごい数になる。	
\\	お賽銭を投げ入れて、柏手を打って願い事をします。	
\\	隣の家の門松はうちのよりずっと立派だ。	
\\	初夢で悪い夢を見ると、その一年が思いやられるよ。	
\\	正月にもちをつくのはどうしてなの?	
\\	おせち料理はやはり重箱に入れて、見た目でも楽しみたいものだ。	
\\	もちをのどに詰まらせないように注意してね。	
\\	鏡もちにのってるミカンを食べたのはだれだ?	
\\	彼はお屠蘇を飲んだくらいで酔っぱらってるよ。	
\\	・ 
\\	今年は寝正月だな。	
\\	負けたほうの顔に墨を塗ります。	
\\	書き初めがうまく書けたから気分がいい。	
\\	自分がお年玉をあげる側になるなんて信じられません。	
\\	テレビの正月特番はワンパターンでつまらない。	
\\	七草粥に入れる春の七草を全部言える?	
\\	やっぱりお正月が一年で最大のイベントだね。	
\\	大した議題もないのにまた会議かよ。	
\\	あいつまた課長に怒鳴られてる。	
\\	そんなにストレスをためちゃだめだよ。何かストレスを発散することしろよ。	
\\	コピーぐらい自分で取ってほしいな。	
\\	連日のように残業じゃ体がもたないよ。	
\\	お茶お茶って、私はお茶を入れる為にこの会社に入ったんじゃないんだから。	
\\	あの人達、またうわさ話してる。他にすること無いのかしら。	
\\	リストラか。明日は我が身だな。	
\\	もうすぐ五時だな。一杯飲んでいくかい?	
\\	アルバイト感覚で仕事するなよ。お前は、正社員なんだから。	
\\	今年のボーナスはあまり期待できそうにない。	
\\	また紙が詰まったままになってる。紙詰まりくらい自分でなおせよ。	
\\	このコピー機、拡大・縮小はどうやってやるのかな。	
\\	ホチキスの玉あるかな。	
\\	君、この書類、間違ってるよ。やり直しだ。	
\\	今夜はクライアントの接待か。気が重いな。	
\\	転職するにも、こう不景気じゃね。	
\\	今年も新入社員は採らないらしい。	
\\	休日返上で出勤だなんて、ぺいぺいはつらいよね。	
\\	会社での唯一の楽しみがランチの時間だなんて、なんか寂しいよね。	
\\	愛妻弁当なんて、羨ましいな。俺なんか毎日コンビニ弁当だよ。	
\\	我が社と
\\	社が合併するって噂、本当なかよ。	
\\	コンピュータと英語ができないと、出世できないなんて。俺の将来真っ暗だよ。	
\\	有給が残ってるから、明日は休みを取ろうかな。	
\\	毎月のお小遣い増やしてくれないかな。これじゃ昼休みにコーヒーも飲めやしない。	
\\	あの電車に乗らないと遅刻だ。	
\\	電車に間に合ったと思ったら、カバンがドアに挟まってしまった。	
\\	熱があるみたいだから、会社に病気で休むって電話してくれるかな。	
\\	ここ数年、給料がぜんぜん上がってないような気がする。	
\\	金曜の夜に仕事を代わってくれる奴を見つけることができないから、飲み会には行けないかもしれない。	
\\	議事進行のへたな会議が多すぎる。	
\\	銀行にお金をおろしに行ったら、社長にばったり会った。	
\\	20代で転職する人が多いんだってね。	
\\	会社を興してみたいよなあ。	
\\	考えるのも仕事のうち。仕事は量じゃない。	
\\	お金よりも、やりたいことを優先する人が増えています。	
\\	このあいだ田中がヘッドハントされんだって。	
\\	最近じゃ、どの業界も1人勝ちだよね。	
\\	ジェネラリストの時代から、スペシャリストの時代に移ってきている。	
\\	うちの会社はフレックスタイム制です。	
\\	英語ができないなんて話にならん。	
\\	面接で変なこと言ったのに採用されたよ。	
\\	外資系は完全に実力主義だってね。	
\\	最近は切れる連中が多いので、うかつに叱れやしないよ。	
\\	天職にありつければ幸せだね。	
\\	課長が、肩たたきされてるんだってさ。	
\\	夢のある企画を立てたいね。	
\\	関連企業の求人情報がにぎやかだ。	
\\	2001年度新卒採用情報をお知らせします。	
\\	この企画は世界一だよ。絶対に成功するね。	
\\	シュールレアリスムって、どういうこと?	
\\	キュービズムの絵画を理解するのは、どうも難しい。	
\\	3億円の絵が、実は贋作だったって本当?	
\\	僕は後期印象派の絵が好きだね。	
\\	点描画を描くなんて、気が遠くなるような作業だ。	
\\	風景画を見ていると心がなごむ。	
\\	もっと写実的な絵のほうが、私は好きです。	
\\	ヨーロッパの美術館には、無数の宗教画が展示されている。	
\\	浮世絵は西洋美術に大きな影響を与えたんだ。	
\\	絵画の修復作業は、実に緻密で根気のいる作業だ。	
\\	ポップアートの先駆者というとだれかな?	
\\	キャンバスの上で、さまざまな色が混ざり合っている。	
\\	静物画は簡単そうに見えて、実は奥が深い。	
\\	この絵、何を描いているのかまったくわからない。	
\\	しっかりしたデッサン力をつけないと、いい絵は描けない。	
\\	この絵とタイトルはどういう関係があるんだろうか。	
\\	彼の絵は、うまく遠近感が表現されているね。	
\\	絵よりもこの立派な額のほうが目立っているな。	
\\	僕には絵を描くセンスがまるでない。	
\\	たまには美術館に行って、ゆっくりと絵を鑑賞するのもいいもんだ。	
\\	いいモチーフがあって、はじめていい絵を描くことができた。	
\\	この絵の空間構成は非常に緻密だ。	
\\	この絵からは荘厳な雰囲気が感じられる。	
\\	この彫刻、頭の部分はもとからなかったの?	
\\	絵の端っこにひびが入ってる。	
\\	この絵は何重にも塗り重ねているんだ。	
\\	複製なら有名な画家の絵でも安く買えるんだけどね。	
\\	有名画家の代表作のラフスケッチが見つかった。	
\\	有名な画家だと、1億円以上で取引される作品もある。	
\\	この作品はオークションに出せば100万は下らない。	
\\	よほどの目利きじゃないとこの絵の良さはわからない。	
\\	似顔絵ってうまく描きすぎても下手に描きすぎてもだめなんだ。	
\\	私、有名な画家のモデルにならないかって口説かれているの。	
\\	描かれている人のその時の気持ちが伝わってくるような作品だ。	
\\	今の子どもたちはちぎり絵を作ったことがないんだって。	
\\	この作品には、意図的にモザイクをかけているのかな。	
\\	全体に色がぼけている感じね。輪郭をはっきり描いてみたら。	
\\	この絵、色が淡い感じがしていいわね。	
\\	今やコンピュータで絵を描く時代になったんだ。	
\\	来月、銀座で開かれる個展に行ってみない?	
\\	今日のパーティーはもりあがったね。	
\\	昨日のパーティーはもりあがりにかけた。	
\\	昨日のジョンのパーティーは一晩中続いた。	
\\	パーティーに着ていく服を選ぶのはおっくうだ。	
\\	立食で足が棒になった。	
\\	今日のパーティーは誰が主催なの?	
\\	パーティーには飲み物だけ持ってきて。	
\\	適当に入って勝手にやって。	
\\	遅くまで音楽かけているけど、ご近所さんは平気?	
\\	今日のパーティーは無礼講よ!	
\\	あの人、誰?誰が呼んだのかしら。	
\\	会社のパーティーで、部長が酔っぱらって、社長にからんだんだよ。	
\\	明日、オースティン・パワーズ仮装パーティーがあるけど来る?	
\\	ポールのパーティーいく前に、ちょっと一杯飲んでいく?	
\\	ヤッピーのパーティー行ったけど、全然楽しくなかったよ。	
\\	初めまして。コリーンとはどういった知り合いなの?	
\\	ワインをカーペットにこぼして、カーペットに赤いシミができた。	
\\	ビール買ってくるけど、誰かいる人いる?	
\\	音楽、ガンガンかけようよ!	
\\	これ、私が焼いたケーキ。よかったらみんなで食べて。	
\\	カールが主催するパーティーにはみんなが行きたがるだよ。	
\\	このディナー・パーティーは割り勘?	
\\	私はももこです。ももことよんでください。	
\\	シャンパングラスわって、ごめんなさい。	
\\	あの人誰?誰が呼んだの?	
\\	食べ物はこっち。好きなだけ食べてね。	
\\	一体何が起こったの?人ばっかかりで動けないじゃない!	
\\	社長のスピーチはいつも長くて、パーティーがしらけるよ。	
\\	名刺持ってますか?後で連絡します。	
\\	彼女は一人でパーティーに来てるけど、彼とはどうなったの?	
\\	久しぶりのパーティーでバカ騒ぎしすぎちゃった。	
\\	やっぱり、ガールスパーティーが気楽でたのしいね。	
\\	レベッカのパーティーの招待状がきてない。	
\\	気取ったパーティーはつかれるだけ。	
\\	彼はパーティ好きだ。	
\\	明日のパーティに来て。あなたに会わせたい人がいるのよ。	
\\	今日は楽しかった。また誘ってね。	
\\	あと誰がくる予定なの?	
\\	何か持っていくものある?	
\\	時間通りにはいけないけど、あとから行きます。	
\\	おまねきありがとう。	
\\	すてきなお家ね。	
\\	二次会はどこでする?	
\\	会費はいくらしますか?	
\\	お会計は割り勘です。	
\\	ディナーパーティーの席順を決めるのはいつも気を使う。	
\\	酔っぱらってくるなんて、パーティーが台無しよ。	
\\	彼女のパーティーに行くといつも新しい友達がふえるわ。	
\\	楽しかったわ。また、いつでもよんでね。	
\\	おまねきありがとう。これ、たいしたものじゃないけどおみやげです。	
\\	誰でも好きな人連れてきていいわよ。みんな大歓迎よ	
\\	パーティーっていっても、身内だけのこじんまりしたパーティーだけど。	
\\	パーティーの後かたづけ、大変そうね。	
\\	大使館のパーティーにもぐり込んだけど、誰も知らなくてつまらなかった。	
\\	そんなパーティーに着ていく服なんてないわよ。	
\\	ワルツなんて、ステップ知らないから踊れないわ。	
\\	ボウルなんて気取りやばっかりで肩がこった。	
\\	昨日のパーティー、行ってみたらみんな歌ってて楽しかった。	
\\	みんなフレンドリーで、とっても気軽な雰囲気でいいパーティーね。	
\\	屋上まで人があふれているよ。	
\\	今度、家でもパーティーするから来てよね。	
\\	パーティーするのに理由なんていらないわ!	
\\	彼女のパーティーのフライヤー(お知らせ)、いつもこってていいわね。	
\\	そこの会場にはいったことないので、後で地図くれますか?	
\\	あなた、来るのおそかったわね。パーティーのハイライト見逃したわよ。	
\\	彼らのパーティー行くと、おべっかばっかり聞かされてうんざりよ。	
\\	エスコートなしでいってもいいの?	
\\	ジョンとパーティー行きたいんだけど、どうやって誘うか悩んでいるの。	
\\	明日までにパートナーみつけないと、明日のパーティーいけないわ!	
\\	乾杯の音頭は誰がとるの?	
\\	二日連チャンでパーティーがあったから疲れたわ。	
\\	彼とは久しぶりあったから、おしゃべりに花が咲いたわ。
\\	カレンのパーティーに行くことは、一つのステイタスね。	
\\	シャンパンおつぎしましょうか?	
\\	あいたお皿お下げしましょうか?	
\\	パーティー日和ね!	
\\	芝生の上でごろごろするのって気持ちいいわよね。	
\\	そんなに恥ずかしがっていたら、誰とも知り合いになれないわよ!	
\\	彼女って完璧なホステスね。	
\\	今日、パーティーで知り合った人で誰が一番印象に残ってる?	
\\	バーベキューパーティーって言っていたけど、雨が降ってきたわよ?	
\\	パーティー、パーティーってパーティー三昧よ!	
\\	ワインのお代わりってまだありますか?	
\\	そこのチーズ取っていただけますか?	
\\	必要でしたら、このナプキンお使い下さい。	
\\	ケンっていつも違う女の子とパーティー来てるわよね。	
\\	盛り上げ役がいないと、楽しいパーティにならないよ。	
\\	昨日はお通夜みたいなパーティーだった。	
\\	大きなパーティを成功させるには、段取りが重要だ。	
\\	残業なので、今夜のパーティーには出られない。	
\\	仮装パーティーって楽しそうじゃない?	
\\	残念だけど、一番かわいい子はコンパに来れないらしいよ。	
\\	パーティの主賓が遅刻なんて困るよ。	
\\	ここは2時間飲み放題だから、みんなたくさん飲んでね。	
\\	年末は忘年会。年明けは新年会。この時期は飲み会漬けだ。	
\\	飲み会っていうといつも幹事をやらされちゃうんだよね。	
\\	この店とても広いし、今度みんなでパーティーでもしよう。	
\\	船上パーティーに行くんだって?なんだかロマンティック。	
\\	1回でいいから芸能人がたくさんいるパーティーに行きたいな。	
\\	冬はやっぱり鍋パーティーに限るね。	
\\	せっかく来たんだから、壁の花になるのはよせよ。	
\\	クリスマスイブに男だけでパーティーって、なんか虚しくないか。	
\\	年をとると夜通しで騒ぐのはつらいよ。	
\\	パーティでのあいさつは短ければ短いほどいい。	
\\	招待状なくしたみたい。でも入れてくれるかなあ。	
\\	パーティーに行きすぎて、今月はもうお金がない。	
\\	まっすぐ立って。	
\\	斜めに立って。	
\\	足を少し開いて。	
\\	両手を腰にあてて。	
\\	腕を組んで。	
\\	足を組んで。	
\\	この前に立って下さい。	
\\	背筋を伸ばして。	
\\	胸を張って。	
\\	顎を少しひいて。	
\\	肩の力を抜いて。	
\\	明るい表情をして。	
\\	きりっとした表情をして。	
\\	ニッコリとほほえんで。	
\\	優しい表情をして。	
\\	目線を下さい。	
\\	眩しさを我慢して。	
\\	瞬きが少し多いですよ。	
\\	メガネをふいて下さい。	
\\	メガネを上げて下さい(ずり落ちてきたとき)。	
\\	センターの人の左側に立って。	
\\	もっと前に詰めて(後ろの人に)。	
\\	もっとセンターに詰めて。	
\\	真ん中に向かって少し斜めに立って(片方の肩を出して立つ)。	
\\	皆さん、こっちを向いて下さい。	
\\	いい顔して。	
\\	二人並んで立って下さい。	
\\	少し下がって下さい。	
\\	もっと前に来て下さい。	
\\	フイルム交換なので、少し待って。	
\\	あなたはもう総理を辞めなさい。	
\\	この世には完璧な政治なんてないよ。	
\\	自民党、民主党、共産党どれもどれだな。	
\\	無駄な公共事業が多過ぎます。	
\\	森内閣の支持率は下がりっ放しだよ。	
\\	国会中継って面白い。	
\\	口だけの政治家はもうたくさんだ。	
\\	将来の夢は政治家になることです。	
\\	どんなに頭が良くても、カリスマ性がないと政治家にはなれないよ。	
\\	今はプロレスラーが政治家になる時代だよ。	
\\	女首相は果たして生まれるのだろうか。	
\\	消費税が5パーセントか。なかった時代がなつかしいよ。	
\\	首相公選制は果たして導入されるか。	
\\	どの国でも政治の問題は深刻だ。	
\\	官房長官はつらい仕事だよ。最近は本当にそう思うよ。	
\\	日本の政府はまるで危機管理ができていない。	
\\	与党も野党も同じようなものだ。日本の将来はどうなるのだろう。	
\\	あまりに政治に無関心な若者が多過ぎる。	
\\	選挙権があるんだから、せめて投票ぐらいしろよな。	
\\	雨が降ると投票率が下がるなんて、悲しいね。	
\\	国民の政治不信がピークに達しているね。	
\\	政権交代はもう間もないと思います。	
\\	公約を平気で破る政治家なんて信用できない。	
\\	政治家は官僚の言いなりにはならないでほしいね。	
\\	政界に改革をもたらす政治家が出現した。	
\\	サミットに先立ち、日米首脳同士の非公式会談が行われた。	
\\	サミットでは地球温暖化の問題が話し合われた。	
\\	外務大臣が政府専用機で渡英しました。	
\\	アメリカ大統領が来日し、迎賓館で晩餐会が開かれた。	
\\	衆議院の議長が議事を進行しています。	
\\	こちら側に座っているのが与党の議員たちです。	
\\	連立与党の党首たちが集まって何やら話し込んでいた。	
\\	二世議員がちょっと多すぎると思わない?	
\\	政治に無関心な若者があまりにも多すぎる。	
\\	日本の選挙制度はちょっとややこしいよね。	
\\	憲法には司法、立法、行政の三権分立が明記されている。	
\\	新しい法案が賛成多数で参議院を通過した。	
\\	野党議員は新しい法案の議決をボイコットした。	
\\	住民が県庁舎にダム工事中止の陳情に訪れた。	
\\	ある地方自治体に対して情報開示の命令が下された。	
\\	バーゲンに行ったら、つい余計なものまで買ってしまった。	
\\	店員が来ると見たい物もゆっくり見られない。	
\\	厚底靴を履いていたら、デパートのエスカレータで転びそうになった。	
\\	最近カリスマ店員がはやっている。	
\\	一日中見て回ったが、これはと思うぐらいに気に入った洋服を見つけることはできなかった。	
\\	高そうな店ってやっぱりみんな敬遠するよね。	
\\	バーゲンの時に一つしか残ってなかった物を取るとうれしいよね。でも、みんなの目が怖いけど。	
\\	誰も店に入っていない時に、その店に入るのって勇気がいるよね。	
\\	これはいいなって思う前にどうしても値段を見てしまうんだよね。	
\\	試着した後に、それを買わないと店員さんに断りににくいよね。	
\\	試着
\\	このセーター素肌に着るとチクチクしそうで嫌だな。	
\\	お金がないので取り置きしてもらえますか。	
\\	試着してもいいですか。	
\\	試着
\\	見てるだけです。	
\\	色々他の店を見てから決めます。	
\\	このシャツの長袖ありますか。	
\\	買い物に行ってもあまり「これだ」と思う物に出会えない。	
\\	とりあえず、ひととおり見てから決めた方がいいよ。	
\\	こう品物が多いと、目移りしちゃって困るよ。	
\\	ズボンを試着したがムチムチで恥ずかしかったので、脱いで返してしまった。	
\\	試着
\\	買うつもりもなくただお店に入っただけなのに、直ぐに店員がくっついて来た。	
\\	バーゲンに行くとつい衝動買いしてまう。	
\\	お金ないのに、見ると欲しくなっちゃうんだよね。	
\\	欲しい物がたくさんありすぎて、どれ買おうか迷っちゃうよ。	
\\	バーゲンは安くていいけど、棚の上がグチャグチャで見る気がしないよね。	
\\	可愛いと思う物は全部高いね。	
\\	マネキンが着ている服をそっくり買ってしまうことってない?	
\\	コートは暖かいけど、デパートに入ると邪魔になるから、どこかで預かってくれないかな。	
\\	私は貧乏性だから、流行のものよりずっーと使える物を買ってしまうの。	
\\	自分で言うのもなんだけど、私って買い物上手なのよ。	
\\	とてもお似合いですよ。今着ているセーターにもぴったり。	
\\	もうひとつ大きいサイズはありますか?	
\\	そちらの色は、今在庫を切らしてるんですよ。	
\\	マネキンが着ているぶんには素敵でも、それが似合うとは限らないよ。	
\\	アウトレットのお店で買うと、ブランド物も手ごろな値段で手に入る。	
\\	もう春物の服が出てるんだね。早いなあ。	
\\	お金を持っていないときに限って、欲しい物見つけたりするんだよなあ。	
\\	試着してみるとイマイチなことって多いよね。	
\\	試着
\\	バーゲンに朝早くから並ぶ人の気がしれないよ。	
\\	試着室に入るのも並んでるよ。今日はあきらめようかな。	
\\	試着
\\	コートって値段もけっこうするからね。ついつい慎重になっちゃうよ。	
\\	休日はどこのお店も混んでいて、ゆっくり買い物できないよね。	
\\	「安いから」と買ったものって、結局は着なかったりするよね。	
\\	高いものがいいものとは限らないよね。	
\\	迷った物は、またほかの日に見に来るんだ。衝動買いしちゃうからね。	
\\	雨の日の買い物って、傘を開いたり閉じたりが面倒だよね。	
\\	試着の際には、ファンデーションがつかないようにティッシュをご利用ください。	
\\	試着
\\	少し予算はオーバーしたけれど、イメージにぴったりだったので買ってしまった。	
\\	今日は気晴らしに友だちとショッピングに行こうかしら。	
\\	デパートで買い物をしながら歩き回るとどうしてこんなに疲れるのかしら。	
\\	昨日は徹夜だったから、眠くてしょうがない。	
\\	ここ1週間、仕事が忙しくて連日5時間しか眠っていない。	
\\	このところ毎日、朝4時に自然と目が覚めちゃうんだ。	
\\	最低7時間は寝ないと、身体の調子が悪くなる。	
\\	二度寝って、なぜか気持ちいいんだよね。	
\\	うつぶせで寝ていて、よく息ができるもんだ。	
\\	あおむけで寝てる人がほとんどだよね。	
\\	デリケートだから、枕が変わると眠れない。	
\\	なんだか目がさえちゃって眠れないんだ。	
\\	寝違えて首が痛いよ。	
\\	寝相が悪くて、よく夜中にベッドから落ちるんだ。	
\\	口を開いたまま寝てたから、のどが痛い。	
\\	おまえのいびきがひどくて、なかなか寝られなかった。	
\\	寝言ですごいこと言ってたぞ。	
\\	いい夢を見てたんだけど、起きたら忘れちゃったよ。	
\\	うなされてたみたいだけど平気か?	
\\	夏はクーラーなしじゃ、暑くて寝られないよ。	
\\	夜中に布団を蹴っ飛ばしたから、朝は寒かったよ。	
\\	やっぱり、ベッドよりも布団がいいね。	
\\	朝は寝ぼけているので仕事にならない。	
\\	きみの歯ぎしりはすごい音だよ。	
\\	睡眠学習をやっているんだけど、まったく効果がないよ。	
\\	不眠症はつらいよ。寝たくても寝れないもん。	
\\	今日は寝すぎてなんだか体がだるい。	
\\	最近眠れないから病院で睡眠薬をもらってきた。	
\\	最近いやな夢ばっかり見るから寝たくないな。	
\\	疲労回復には寝るのが一番だ。	
\\	赤ん坊の仕事は寝ることなんだよ。	
\\	赤ん坊の夜泣きがひどくてぜんぜん眠れないよ。	
\\	彼は1週間ほとんど寝ていなかったので、2日間気絶したように眠り続けた。	
\\	寝ることは大切だけど寝すぎはよくないよ。	
\\	とてもじゃないけど、いびきがひどくてきみの横には寝れないよ。	
\\	会議中に居眠りしている。すごいやつだ。	
\\	ぼくは眠りが浅くて、雨の音でも目が覚めちゃうんだ。	
\\	私、眠りが本当に深くて、地震が起きても気がつかないんだ。	
\\	私、枕を高くしないと眠れない。	
\\	昨日の夜寝言を言ってたわよ。	
\\	激しい運動をすれば朝までぐっすり眠れるよ。	
\\	徹夜で勉強しても効率が悪いよ。寝なさい。	
\\	明日は休みだ。ぐっすり眠るぞ。	
\\	よだれを垂らして寝てるなんて、よっぽど疲れてるんだな。	
\\	寝息をたてて気持ちよさそうに眠っているわ。	
\\	睡眠が足りないから仕事中にあくびなんかするんだ。	
\\	睡眠不足のせいか、目がしょぼしょぼする。	
\\	お腹が減ると眠たくなるんだ。	
\\	寝返りを打つのは健康な証拠だ。	
\\	僕は目を開けたまま眠ってるらしいんだ。	
\\	夜中にコーヒーなんか飲むと眠れないよ。	
\\	10分でも昼寝をすれば、頭がすっきりするよ。	
\\	社会人になったら早寝早起きの習慣をつけたほうがいいよ。	
\\	僕は寝だめができる体質なんだ。	
\\	室内が暖かくて、ついうとうとしてしまった。	
\\	なんでそんなに目が充血してるんだい?徹夜したんだろう。	
\\	何歳までおねしょしてた?	
\\	足が冷えて眠れないんだ。	
\\	背中を丸めて寝るなんて猫みたいね。	
\\	あのソファは気持ちよくて、すぐ眠ってしまう。	
\\	最近、不眠症ぎみで悩んでいるんだ。	
\\	寝る暇も惜しんで勉強しなきゃ、宿題が間に合わない。	
\\	布団に入るときが一番幸せだよ。	
\\	これまでに何度禁煙しようとしたことか。	
\\	健康によくないと分かっていてもタバコ止められないんだな。	
\\	禁煙、禁煙ってうるさく言うけど、タバコ吸うのは本人の勝手でしょ。	
\\	タバコを吸う人が肺ガンになる確率は吸わない人と比べ物にならない程高い。	
\\	タバコを吸っている人のそばにいると服に臭いがつくから嫌なんだ。	
\\	レストランの禁煙席に座って、平気な顔をしてタバコを吸うなんて、どういう神経してるんだろう。	
\\	吸い殻を道路に捨てるのは止めてほしいね。	
\\	学生服を着たままタバコを吸っている高校生をよく見かける。	
\\	こんな煙だらけの部屋でよく仕事していられるな。	
\\	この煙どうにかならないの。窓を開けて、空気を入れ換えれば。	
\\	女性の喫煙者の数が増えているらしい。	
\\	タバコを吸ったら、臭いですぐに分かる。	
\\	タバコの自動販売機が多すぎるんだよ。	
\\	食後の一服がたまらなくおいしいんだよね。	
\\	仕事中にイライラするとつい吸っちゃうんだよね。	
\\	タバコを吸う時には、吸ってもいいですか、と相手に聞くのが礼儀でしょう。	
\\	子供が産まれてから、家でタバコが吸えないので、吸うときには外にでなくてはいけない。	
\\	未成年のくせに、格好付けてタバコを吸うなよ。全然、様になってないよ。	
\\	酒にたばこ。どちらか一つでも止められたらいいんだけど。	
\\	寝煙草をして、畳を焦がしてしまった。	
\\	タバコ代も馬鹿にならないから、禁煙しようかな。	
\\	タバコ吸う人の息って臭いのよね。本人は気づいてないんでしょうけど。	
\\	歩きながらタバコを吸うのは止めて欲しいね。タバコの火が腕についてやけどしたことがあるんだ。	
\\	一日に二箱も。それは吸い過ぎだよ。	
\\	意志が弱いから、なかなかタバコが止められないんです。	
\\	禁煙ガムを試してみたけど、僕には効き目はなかった。	
\\	タバコを吸ったことがないので、タバコを吸う人の気持ちが全く理解できない。	
\\	子供の時はタバコなんて絶対に吸わないって思ってたけど、周りの影響で吸うようになってしまった。	
\\	お爺ちゃんの思い出と言ったら、タバコの臭いかな。	
\\	主人がヘビースモーカーだから、壁にヤニがついてきれいにするのが大変なの。	
\\	よくもまあそうひっきりなしにタバコを吸うね。	
\\	あなたにタバコを少しずつ減らして、止めて欲しいんだけど。	
\\	昨日、ジャイアンツぼろ負け!!	
\\	松井のサヨナラホームラン見た?	
\\	現在の首位は、阪神タイガースで、2位の広島とは2ゲーム差です。	
\\	バッターのイチローは、目の覚めるような2塁打を打った。	
\\	今日の勝ちで、西武ライオンズのマジックは5に減りました。	
\\	今日の成績は最悪。4タコに終わってしまった。	
\\	今、何回の裏ですか?	
\\	どっちが勝ってるの?	
\\	野球はどこのチームのファンですか?	
\\	ハラハラするな。一打逆転のチャンスだよ!	
\\	今タッチしてないよ。審判ちゃんと見ろよ。	
\\	上原のストレートは150
\\	を越える。	
\\	先頭打者がヒットで出塁した。	
\\	オープン戦が終わり、いよいよ開幕だ。	
\\	デットボールを当てられ、激怒した。	
\\	スリーバントを成功させ、意気揚々とベンチに戻った。	
\\	ファアボールでいいから、なんとかして出塁してくれ。	
\\	一塁手は、走ってきた打者にタッチしアウトを取った。	
\\	彼は、上手いよ! シングルプレーヤーだよ!	
\\	スライスばかりで
\\	連続だよ	
\\	「あ、ダフった。また、ボギーだよ。」	
\\	パー確実だね。	
\\	ピン抜く?	
\\	テンプラだ!	
\\	調子でてきたね。	
\\	左のラフに入っちゃったんじゃないかな。	
\\	グリーンまでどれくらい?	
\\	グリーンをねらうのはちょっと危険だね。刻んだ方がいいよ。	
\\	そのクラブじゃ大き過ぎるよ。	
\\	あの池を越えるにはどれくらいキャリーが必要かな?	
\\	入れーーーーーー!	
\\	ホールインワンを一度でいいから出したいよ。	
\\	テニスにもあるけど、4大大会って何?	
\\	このサーブを受けてみろ!	
\\	うわー、すごいドライブがかかってるね。	
\\	しまった、ガットが切れちゃった。	
\\	あいつのサーブってすごい強いんだよね。	
\\	手首をこねちゃだめだよ	
\\	今のが入ってるって?うそだろ、完全にはずれてるじゃん。	
\\	俺はバックで打つ方が得意だな。	
\\	彼女はフットワークがいいね。	
\\	板をハの字にして、止まるときはもっと大きなハの字にすればいいんだよ。	
\\	ストックを曲がりたい方向に突くとターンしやすいよ。	
\\	ここはコブ斜だからさけた方がいいね。	
\\	ボーダーってほんとに邪魔だよね。	
\\	膝を曲げて、前を見て。	
\\	私のレベルはボーゲンです。	
\\	透明度はどれくらいですか?	
\\	耳抜きをきちんとしてね。	
\\	エアーの残り具合は?	
\\	体験ダイビングコースはいつやっていますか?	
\\	クロールできる?	
\\	平泳ぎ得意?	
\\	飛び込みしたことある?	
\\	背泳ぎすると、水飲んじゃう。	
\\	私、実はすごいカナヅチ。	
\\	水球は、見るとのやるのとは大違いのスポーツです。	
\\	私は犬かきしかできない。	
\\	ばた足をして進もう!	
\\	ラクビーって何人でするの?	
\\	ワントライって何点?	
\\	今日は、彼、絶対にハットトリックを決めるぞ!!	
\\	ワールドカップは何年ごとに行われるの?	
\\	野球と違いどんな天候コンディションでもサッカーは行われます。	
\\	サッカーの試合は何分間ですか?	
\\	オリンピックの由来は?	
\\	ワァーーーオ、金銀銅総なめ。	
\\	今度のオリンピックはどこで開催されるのですか?	
\\	日本では今まで何回オリンピックが開かれましたか?	
\\	100メートル、10秒切れる?	
\\	走り高跳びと走り幅跳びどっちを見たい??	
\\	砲丸投げ、やり投げ、円盤投げ、どう違うの?	
\\	マラソンで2時間を切ることは、不可能なの?	
\\	あの選手は2回もフライングをした。	
\\	私は、ゴールインした後、倒れた。	
\\	リレーにおいてバトンタッチは非常に大切である。	
\\	競技によって、試合開始や終わりの合図は違います。	
\\	しまった、またガーターだ。	
\\	いい波が来てるね、よし、行くぞ。	
\\	今日はクラゲがいるから気をつけて。	
\\	今日のディフェンスはマンツウマンですか、ゾーンですか?	
\\	やっぱ、バスケの醍醐味はダンクだよ。	
\\	いくつのファールで退場ですか?	
\\	パッシングではなくロブで勝負だ。	
\\	ペンとシェークとどちらを使っていますか?	
\\	どうして卓球の選手のユニフォームは色つきなの?	
\\	カヌーをこぐのは疲れる。	
\\	体操競技の中では、鉄棒の大車輪が好きです。	
\\	吊り輪は、非常に力がいる競技です。	
\\	鞍馬は、力よりもバランス感覚が必要です。	
\\	段違い平行棒は、女子の競技にしか用いられません。	
\\	一般の人は平均台の上にただ立っているだけで難しいです。	
\\	混んだゲレンデで人を捜すこと難しい。	
\\	転んだ時に、ストックが折れてしまった。	
\\	私のシュプールの後を追ってきてください。	
\\	アイスバーンだから気をつけてね。	
\\	競馬場には多くの人がいる	
\\	そのレースは写真判定になった。	
\\	今回は穴馬を買った。	
\\	本命の馬がこけた	
\\	私は運動神経がまったくない。	
\\	僕はスポーツ得意じゃないから。	
\\	僕は運動音痴なんだ。	
\\	僕は運動部系じゃないからさ。	
\\	運動はだめだな。	
\\	スポーツ見るのはいいけど、やる方はね。	
\\	彼はスポーツも勉強も得意だ。	
\\	彼は文武両道だ。	
\\	彼、スポーツしてるからって勉強ができないということはないよ。	
\\	大きなボールを使うスポーツはいいけど、テニスとか、卓球、野球はね。	
\\	インクはまだあるのに、ボールペンが書けなくなってしまった。	
\\	シャーペンの芯を2、3本もらえるかな。	
\\	僕のは水性なので、油性のサインペンを持っていたら貸してもらえるかな。	
\\	そこにある赤と黒のマジックを取ってくれないか。	
\\	鉛筆削りも電動になって、手で回すやつはあまり見かけなくなった。	
\\	あっ、数字を間違えちゃった。修正液はどこだっけ。	
\\	ここにある書類を番号順に並べて、ホチキスで留めておいてくれ。	
\\	この封筒じゃ小さすぎるな。もう少し大きいのはないかな。	
\\	このセロテープあまりよくくっつかないね。	
\\	この万年筆、インクが漏れてる。	
\\	この電卓小さすぎて使いにくいよ。もう少し大きいのない?	
\\	このクライアントのファイルはどこにあるのかな?	
\\	消しゴムがすり減って、使いにくくなってきた。	
\\	の鉛筆で該当する項目の丸を黒く塗りつぶしてください。	
\\	コンパスに三角定規、分度器。君の筆箱には何でも入ってるんだね。	
\\	このボールペン、12色なんだね。どうりで太いわけだ。	
\\	画びょうを床にばらまいてしまった。	
\\	クリップがはずれて、書類がバラバラになってしまった。	
\\	このカッター切れが悪いね。	
\\	付箋がついているページをコピーするんですね。	
\\	そのポスター、輪ゴムで広がらないようにしてくれるかな。	
\\	大事なことは忘れないようにメモに書いて、虫ピンで壁にとめておくようにしているんだ。	
\\	消しカスが出ない消しゴムってないのかな。	
\\	この消しゴム、まったく消せないよ。	
\\	会社にも左利き用のはさみがあればいいのに。	
\\	紙を挟みすぎてクリップが広がっちゃったよ。	
\\	ファイルの留め金が壊れて書類が抜け落ちちゃったよ。	
\\	スティックタイプののりって手が汚れなくていいよね。	
\\	大事な部分は蛍光ペンでチェックしておこう。	
\\	蛍光ペンでマークしたら文字がにじんで読めなくなった。	
\\	いちいち定規で線を引くなんて神経質すぎないか。	
\\	1メートルの長さの定規なんて持ってないよ。	
\\	このダンボールをガムテープで補強しておいて。	
\\	両面テープを使うと、手にくっついちゃうんだよね。	
\\	クリアファイルなら中身がわかって便利だよ。	
\\	普段からメモ帳を持ち歩いて、思ったことを書きとめているんだ。	
\\	4サイズの封筒を3枚用意しておいて。	
\\	替えのコピー用紙ってどこにあるの。	
\\	会社でコンパスを持ってるなんて風水にでも使うのかい。	
\\	領収書を書くならカーボン紙でコピーを取っておきなさい。	
\\	テープ式の修正液も持ってない?	
\\	筆ペンなんて会社に置いてないよね。	
\\	電子メールのおかげで、海外の大学への問い合わせが簡単にできるようになった。	
\\	が今年10月からコンピュータ化されるから、コンピュータの使い方に慣れておかないと。	
\\	のライティングのスコアが低いのが悩みの種なんだ。	
\\	前回受けた
\\	のスコアが目標点に少し足りなかったので、もう一度受験しないとだめだな。	
\\	現在、
\\	600点突破を目指して猛勉強中です。	
\\	できれば奨学金がもらえればいいんだけど。	
\\	事前に留学先の住宅事情を調べておくに越したことはない。	
\\	こちらに来て、自分の語彙力のなさを痛感しています。	
\\	ようやくアメリカの生活にも慣れ、充実した毎日を送っています。	
\\	この大学は、日本の大学とは比べ物にならないほど施設が充実している。	
\\	いろいろな国からの留学生と知り合いになるのがとても楽しい。	
\\	日本食が無性に食べたくなることがたまにある。	
\\	最初の学期だから、無理して授業を多く取らないほうがいいな。	
\\	あの先生の講義はきついよ。リーディングの課題がとにかく多いから。	
\\	授業の予習に追われて、アルバイトどころじゃない。	
\\	授業を録音するのもいいけど、後で聞く時間がないので、結局は意味がない。	
\\	分からないことがあって、どう英語で質問しようかなと考えているうちに、次の話題に移ってしまった。	
\\	初めて英語で質問した時には、胸がドキドキしたね。慣れちゃえばどうってことないんだけど。	
\\	(授業中に)今の説明をもう一度していただけますか。一部聞き取れなかったものですから。	
\\	(クラスメートに)先週風邪で休んだので、ノートを見せてもらえるかな。	
\\	(クラスメートに)今日の講義で分からないことがいくつかあったんだけど、教えてもらえるかな。	
\\	課題のレポート仕上げるのに2日も徹夜した。	
\\	(書店で)ブラウン先生のテキストは売り切れなんですか。注文したらどれくらいかかりますか。	
\\	こっちの図書館は夜中までやってるので、夜遅くまで多くのが学生が熱心に勉強している。	
\\	図書館が大きすぎて、本を探していて何度か迷子になってしまったことがある。	
\\	(研究室のドアをノックして)失礼します。質問があるのですが、今よろしいでしょうか。	
\\	講義の内容についていくつかお聞きしたいことがあるのですが、先生の都合のい時間を教えていただけますか。研究室にうかがいますので。	
\\	週末も勉強していたら、とても体がもたない。少し息抜きも必要だよ。	
\\	こちらへ来てしばらくは、カルチャーショックのためか、体調がすぐれなかった。	
\\	勉強についていくのが大変で、ホームシックになるどころではなかった。	
\\	留学という新たな目標があるので、英語の勉強にも身が入る。	
\\	短期間でできるだけ効果が上がる語学プログラムはありますか?	
\\	英会話学校に通うなら、できるだけ日本人が少ないところがいいね。	
\\	初めの3カ月は食事が合わなくて苦労した。	
\\	目的をはっきりもたないと、だらだらと過ごしてしまうことになりかねない。	
\\	大学生が1年くらい留学しても、帰国子女のような英語が身につくわけじゃない。	
\\	英語を学ぶのではなく英語で何を学ぶかが重要だ。	
\\	留学を成功させるには、前もっていろいろ準備する必要がある。	
\\	インターネットを使って単位が取れる海外の大学が増えてきた。	
\\	遊びの誘惑がなくて、勉強に集中できそうだから田舎の学校を選んだんだ。	
\\	ホームステイすると、現地の人の生活習慣がよくわかる。	
\\	いくつかの日用品を英語でなんと言えばいいのかわからなくて困った。	
\\	せっかく外国に来ているんだから、休日には勉強を忘れて旅行しよう。	
\\	日本人ばかりでかたまっていたら、英語は上達しないよ。	
\\	いろんな国からの学生と親しくなれるのも留学のよい点だ。	
\\	留学生だからといって特別扱いしてくれるわけではない。	
\\	テストの点がよくても、授業で積極的に発言しないと評価は低くなる。	
\\	帰国の日が近づいて、たまらなく寂しくなってきた。	
\\	せっかく留学したのだから、帰国したら国際的な仕事に就きたい。	
\\	ファーストフードばかり食べているから、日本食が恋しくなってきた。	
\\	この語学学校では、学生ビザを取るために必要な書類を用意してくれますか?	
\\	夏休みにほかの国に行くので、授業に出席していることを確認できる書類を学校でもらう必要がある。	
\\	体験入学は何日間できるのですか?	
\\	私の学生ビザは今月いっぱいで切れるので、急いで書類を集めなければならない。	
\\	入学テストはいつ行われるのですか?	
\\	この学校に入学するためにはどうすればよいのですか?	
\\	学割で定期を買うにはどうしたらよいのですか?	
\\	学生証はいつもらえるのですか?	
\\	私の授業の教室が変わったのですが、どこでやっているか知りませんか?	
\\	教科書はどこで手に入れればよいのですか?	
\\	授業が難しすぎてついていけないので、クラスのレベルを1つ落としてもらいたいのですが。	
\\	留学生サロンは毎日とても混雑している。	
\\	海外での生活は、何もかもが新鮮で勉強になる。	
\\	学生であるという証明ができれば、税金を払う必要はない。	
\\	国営の学校なら、税金を払っていれば学費が安くなるよ。	
\\	ここの語学学校の授業内容はお勧めできない。	
\\	来週までにこの課題をやっておかないと、単位が取れない。	
\\	あの教授は、留学生だからと言って成績を甘くしてはくれない。	
\\	ホームステイしていたら、きっと私の英語力はもっと伸びたのに。	
\\	大学の寮に入ったら、現地の学生と相部屋になった。	
\\	この味噌汁は、ちょっと味濃いねえ。	
\\	一人暮しをしてると、お袋の味が恋しくなる。	
\\	このレストランは、味付けが抜群にいい。	
\\	なんか、こってりしたもの食べたいなぁ。	
\\	取引先の社長と会食だなんて、おいしいもの出されても味わうどころじゃないよ。	
\\	このパフェ、大きくて嬉しいけどちょっとしつこい。	
\\	にんにく風味の料理は、食欲を増すね。	
\\	うちの主人は味にうるさいもので、食事を作るほうも苦労しますよ。	
\\	うちの食事は、どちらかというとうす味のほうです。	
\\	関東の料理は、味が濃い。	
\\	このおしんこ、鷹の爪が入ってるからピリッとしておいしい。	
\\	くー、わさびが効いてるー。	
\\	サバの塩焼きは、ちょっと生臭いね。	
\\	魚の煮物は、しょうがを入れると味が締まるよ。	
\\	何かさっぱりしたものが食べたい。	
\\	この煮物、椎茸からいい出汁が出てるわ。	
\\	私は、甘党です。	
\\	そんな油っこいものばっかり食べたら、気持ち悪くなるよ。	
\\	あなたのお母さん、お料理上手ね。あれはほとんど料亭の味よ。	
\\	このケーキ、アーモンドの香りが香ばしくって美味しい。	
\\	赤ワインだったら、辛口でこくがあるものが好きです。	
\\	最近の発泡酒は、のどごしがいいし味もしっかりしてるから、ビールとたいして変わらないよ。	
\\	秋に出るみかんは、まだすっぱいね。	
\\	その薬は苦いですか?	
\\	日本のほうれん草は、灰汁が強くてサラダには向いてない。	
\\	いやだ、牛乳がもう酸っぱくなってる。	
\\	レトルト物って、防腐剤の味がするから好きじゃない。	
\\	このお肉は硬くて、噛みごたえがある。	
\\	このパン、ボソボソする。	
\\	甘いもの食べた後は、口直しにしょっぱい物が欲しくなる。	
\\	旬のものは、食材の味を生かして調理したほうがいい。	
\\	味がわからない人に、何食べさせたってむだよ。	
\\	魚を冷凍すると、あきらかに味が落ちるね。	
\\	羊の肉は、ちょっと癖があるよね。	
\\	風邪ひいてて、何を食べても味がしない。	
\\	酸っぱいものはちょっと苦手なんだ。	
\\	辛いもの、大丈夫?	
\\	甘ったるくて、これ以上食べられないよ。	
\\	これ、すごくしょっぱい!	
\\	もうちょっと甘くしたほうがいいみたい。	
\\	このスープ、味がしないよ。	
\\	ここのラーメンはこってりしていておいしいね。	
\\	このジュース、なんだか薬みたいな味がしない?	
\\	ちょっと味見してみてよ。	
\\	甘酸っぱいイチゴはいかがですか?	
\\	このカレー一度食べたら忘れられない味だね。	
\\	最近の若い人はいろんなものにマヨネーズをかけるんだって。	
\\	このお肉、味付けしてないんじゃない?	
\\	このカクテルはジュースみたいな味だね。	
\\	すごくフルーティーな味がする日本酒だ。	
\\	年を取ると薄味のものが食べたくなるな。	
\\	あまりグルメになりすぎるのも考えものですね。	
\\	僕は味オンチだってよく言われるよ。	
\\	やっぱり見た目よりも味が重要でしょ。	
\\	うちの夫は味にうるさくて困るわ。	
\\	最近無言電話が多くて困る。	
\\	あ、キャッチ入ったから、また後でかけなおす。	
\\	あの人あんまりうっとうしいから、留守電にして居留守つかっちゃった。	
\\	昔は長電話して、よく怒られたものだ。	
\\	お父さん、電話口で頭下げたって、相手には見えないよ。	
\\	だれか、電話出て。	
\\	この掃除機、テレフォンショッピングで買ったのよ。	
\\	ただいま留守にしております。ピーという発信音の後に、お名前とメッセージをお願いします。	
\\	新商品の広告出したら、問い合わせの電話が鳴りっぱなしだよ。	
\\	電話でのお問い合わせは受け付けておりません。	
\\	あの子いい子なんだけど、電話魔なのよね。	
\\	昨日留守電聞かなかったから、電話くれたの知らなかった。	
\\	いじめの電話相談室では、年々相談件数が増えてるらしい。	
\\	あの人は電話嫌いで有名だから、
\\	メールで連絡とったほうがいいよ。	
\\	田中はいま、他の電話に出ておりますが。	
\\	田中はちょっと席をはずしておりますが、すぐに戻ると思います。	
\\	鈴木と申しますが、電話があったことだけお伝えください。	
\\	鈴木と申しますが、折り返しお電話いただけるよう、お伝え願えますか。	
\\	田中さん、鈴木さんという人から電話がありましたよ。	
\\	田中さん、鈴木さんという人から電話が入ってます。	
\\	最近、電話セールスが多くって仕事にならない。	
\\	イタズラ電話がしょっちゅうかかってくるんだけど、電話番号変えたほうがいいかな。	
\\	その時間は外にいると思うから、家の電話じゃなくて携帯に連絡入れて。	
\\	失礼ですが、どちらでうちの電話番号をお知りになりました?	
\\	中村と申しますが、一郎さんはご在宅でしょうか。	
\\	電話機が変わってから、内線と外線の音の区別がつかなくて、何度も失敗しちゃった。	
\\	太郎はまだ帰っておりませんが、帰ったらお電話させましょうか?	
\\	いま出先ですので、こちらからまたお電話いたします。	
\\	たまには電話ぐらいしてきなさいよ!	
\\	はじめてお電話させていただきます
\\	出版の吉田と申しますが、広報ご担当の方をお願いできますでしょうか。	
\\	お礼のお手紙書かなきゃいけないんだけど、面倒臭いから電話で済ましちゃおう。	
\\	今手が離せないから、後でかけなおすって言っておいて。	
\\	サポートセンターっていっつもこうだよ。さっきから、何分保留音聞かされてるんだろう。	
\\	家でインターネットもいいけど、電話代がばかにならないよ。	
\\	息子?今年の正月に電話してきたっきりよ。	
\\	今どこ?車の音がするけど、外から?	
\\	ここの公衆電話、テレフォンカードしか使えないよ。	
\\	電話口で怒鳴らないでよ!	
\\	あの人会ってると楽しいんだけど、電話だと間が持たないのよね。	
\\	この同好会も人が増えてきたから、連絡網作ったほうがいいんじゃない?	
\\	ホームページを立ち上げようと思うんだけど、もう一本電話線ひいたほうがいいかしら?	
\\	来月開演される劇のチケットなんですが、電話でも予約できますか?	
\\	講堂に入る前に、携帯の電源切っておいたほうがいいよ。	
\\	電話で喧嘩して、切った後で後悔するのよね。	
\\	住所がわかってるなら、電話帳調べれば電話番号わかるよ。	
\\	夕方に
\\	社から電話が入ると思うんで、用件だけ聞いておいてください。	
\\	お電話いただいたそうで。どういったご用件だったでしょう。	
\\	あの会社、電話の受け答えが悪すぎる。	
\\	電話だと、もっと英語が聞き取れない。	
\\	覚えにくい電話番号だね。	
\\	電話が遠いんですが、もう少し大きな声でお話いただけますか?	
\\	今日は電話が少なかったから、仕事がはかどった。	
\\	電話が混線してるみたいね。	
\\	これ以上電話で話し合っても埒があかないから、どこかで会わない?	
\\	みんな外出しちゃったから、今日は一日電話番。	
\\	夜中に電話が鳴ると、ドキッとするよ。	
\\	この電話だって、盗聴されてるかもしれないよ。	
\\	今朝は、実家からの電話で起こされたよ。	
\\	コンサートのチケット取りたいんだけど、なかなか電話がつながらない。	
\\	鈴木さんのお家、さっきからずっと話し中なの。	
\\	忙しい、忙しいって、電話もできないほど忙しいの?	
\\	あの子もひまそうだから、電話して呼び出そうか。	
\\	よし子ちゃん、何かあったのかなあ。電話の声がおかしかった。	
\\	1週間前からずっと一郎君に電話してるんだけど、つかまらないの。	
\\	そんなに困ってったなら、電話してくればよかったのに。	
\\	一件電話待ってなきゃいけないから、先に行ってて。	
\\	せっかく電話したのに、文句ばっかり言わないで。	
\\	この電話を営業部に回していただけますか?	
\\	この1カ月、彼から電話が無いんだけど、もうだめなのかしら。	
\\	電話しようしようと思いながら、すっかりご無沙汰しちゃってすみません。	
\\	あ、受話器がはずれてた。	
\\	さっき、電話に出た人、あなたのお父さん?	
\\	そろそろ
\\	にしようかな。	
\\	夜分遅くにすみません。ご主人様はお帰りでしょうか?	
\\	先月お金が払えなくって、電話止められちゃった。	
\\	いま、公衆電話からかけてるんだけど、テレフォンカードがなくなりそうだからすぐに切るね。	
\\	今日は誰とも話したくないから、電話線抜いておこう。	
\\	最近、国際電話が安くなったよね。	
\\	みんな凝った着メロ使ってるよね。	
\\	お母さんだって心配してるんだから、一本電話入れておいたほうがいいよ。	
\\	先ほどは、電話が切れてしまいまして、すみませんでした。	
\\	田中は現在出張中で、今月一杯は戻らないのですが。	
\\	先週から、クレームの電話がひっきりなしにかかってくる。	
\\	(キャッチホンで電話を取って)いま電話中だから、あとでかけなおすね。	
\\	商品に関するお問い合わせは、フリーダイヤルをご利用ください。	
\\	今月の電話代見て、目が飛び出ちゃった。	
\\	この公衆電話は、国際電話はかけられないみたいよ。	
\\	明日からアメリカに出張なんだけど、コレクトコールで電話するからね。	
\\	ホテルの部屋に戻ったら、寝る前にモーニングコールをセットしておいたほうがいいよ。	
\\	私に電話?今手が離せないから、ちょっと保留にしておいてくれる?	
\\	失礼ですが、電話番号をお間違いじゃないでしょうか。	
\\	その電話の呼び出し音うるさい。もう少し音量下げられない?	
\\	ダイヤル式の電話って、ほとんど見なくなっちゃったね。	
\\	受話器から声がもれてたから、さっきの会話全部聞こえちゃったよ。	
\\	あんまりもたもたしてると、催促の電話がかかってくるよ。	
\\	一家に一台テレビ電話なんて、そんな時代本当にくるのかしら。	
\\	最近の女子高生は、携帯でメールしてるらしい。	
\\	電話でご説明するのもなんですから、一度伺ってよろしいでしょうか。	
\\	最近うちのお兄ちゃん、伝言ダイヤルにはまってるみたい。	
\\	電話番号が変更になったので、ご連絡しておきます。	
\\	フジテレビの局アナはかわいい子が多い。	
\\	このコメンテーター、テレビなれしてるね。	
\\	あの女優さんは、テレビ映りがいい。	
\\	連続ドラマの最終回見逃した!	
\\	テレビ見ながら勉強できるわけないでしょ!	
\\	あの人は、ワイドショーネタに強い。	
\\	このアナウンサー、最近さかんにマスコミでたたかれてるね。	
\\	新聞は、テレビ欄しか見ない。	
\\	昨日は、一日中テレビを見てた。	
\\	金曜の夜8時からやってる
\\	のドラマ、なんていうタイトルだっけ?	
\\	この女優さん、子役のころはかわいかったのよ。	
\\	あの俳優さん、すっかり落ちぶれちゃったね。	
\\	深夜番組でしか見なかったお笑いタレントが、ついにゴールデンタイム進出だって。	
\\	あの女優さん、
\\	が多いらしいよ。	
\\	司会者が変わって、すっかり人気番組になったね。	
\\	この映画、音声多重放送だって。	
\\	「サリーちゃん」って、まだ再放送してるの?	
\\	オリンピックのときは、一日中テレビにかじりついてたよ。	
\\	視聴者プレゼンとで、温泉旅行が当たった!	
\\	兄弟が多いと、チャンネル争いで大変だよ。	
\\	ヒロスエ人気は、いつまで続くのかね。	
\\	あの番組、ビデオになったよ。	
\\	勝手にチャンネル変えないでよ。	
\\	次は、ニューヨークからの中継です。	
\\	この子、国民的アイドルって言われてる子でしょ?	
\\	衛星放送で、いい映画がたくさんやってるよ。	
\\	あのアナウンサー、
\\	から民放に移るらしい。	
\\	テレビは、もはや一家に一代どころか、一人一代の時代になっちゃったね。	
\\	そんなことあるわけないじゃん。テレビの見すぎ!	
\\	テレビ業界って、汚い世界だって言うね。	
\\	地方局にも、かわいいアナウンサーはたくさんいる。	
\\	あっ、ニュース速報だ。	
\\	あの新人、絶対ブレイクするよ。	
\\	秋の番組改変が楽しみだ。	
\\	視聴率とれないから、メインキャスターが下ろされたんだって。	
\\	サスペンスドラマって、二流俳優ばっかりだよね。	
\\	昔エキストラのバイトしてたんだ。	
\\	あの女優さんたいしてきれいじゃなかったよ。テレビ映りがいいだけみたい。	
\\	ただいま、生放送でお送りしております。	
\\	もっと画面から離れて見なさい!	
\\	ビデオの録画予約しておけばよかった。	
\\	くだらない番組が多すぎる。	
\\	あの俳優さん、最近バラエティ番組に出始めた。	
\\	テレビつけっぱなしで寝ちゃった。	
\\	どこも視聴率争いで必死だ。	
\\	サッチー騒動でスポンサーが降りたって。	
\\	このクイズ番組は、司会者がよくない。	
\\	コマーシャルばっかり!	
\\	芸能レポーターに追いかけられて大変だ。	
\\	最近の深夜番組は、面白い番組が増えてきた。	
\\	君は、本当にテレビっ子だね。	
\\	あの番組のメンバーは、アイドル崩ればっかりだ。	
\\	最近、
\\	の映りが悪いなぁ。	
\\	お正月の特番は、どこも似たり寄ったりだ。	
\\	寝過ごしちゃった。	
\\	(冬場)足下が暖かく、背中からお日様が当たってると眠くなるよね。	
\\	あの揺れがたまらなく眠気を誘うよ。	
\\	寝入るとスリに財布をとられちゃうよ。	
\\	きれいな女の人が口開けて寝てるのをみると、がっかりするよ。	
\\	大きないびきをかいて寝てるオヤジがいた。	
\\	ラッシュ時って、身動きとれないよ。	
\\	満員電車のなかでは、人に寄りかかっていられる。	
\\	人に押されて、とんでもない格好でひと駅我慢しなきゃならなかった。	
\\	中のほうに入ったら、降りれなくなっちゃった。	
\\	おじさんに囲まれて息もできなかった。	
\\	肌と肌が触れ合って気持ち悪い。	
\\	汗でシャツがはりついてる人には近づきたくない。	
\\	「これでもか」って後ろから無理矢理乗り込もうとする人がいて、体が宙に浮いたよ。	
\\	満員電車の中で「シャカシャカ」いわせてる人って多い。	
\\	通勤時間は本や新聞を広げることもできない。	
\\	中吊りをボーっとみて過ごした。	
\\	おじさんのポマードの臭いで気持ち悪くなった。	
\\	人にもたれかかってくるヤツがいて、腹が立った。	
\\	背が高いと、中吊りが頭にぶつかるからすごくイライラするんだよ。	
\\	満員電車の中で背中がかゆくなって、思わずからだをよじっちゃった。	
\\	雨が降ってる日は、電車の中がムワっとする。	
\\	濡れた傘を人のズボンでふきやがってノ。	
\\	傘がだれかに引っかかって、降りるときに無理に引っ張ったら柄だけになってた。	
\\	新しい靴をおろしたときにかぎって、電車の中で人にふまれるよね。	
\\	ハイヒールの女性に足を踏まれて穴があくかと思った。	
\\	スーツに口紅つけやがってノ。	
\\	「痛勤」電車とはよく言ったものだ。	
\\	満員電車で女の人と向き合っちゃうと、けっこうバツが悪いもんだ。	
\\	電車の中で携帯電話で話してる光景って妙だよね。	
\\	やけにすいてる車両があると思ったら、酔っぱらいが横たわってたよ。	
\\	ブツブツ独り言を言ってるおじさんがいて不気味だった。	
\\	「この人痴漢です」って言うのは勇気がいるだろうなあ。	
\\	隣の人が頭を肩にのせてくるときって、どうしたらいいか困っちゃうよね。	
\\	お年寄りが立ってるのに、平気な顔してシルバーシートに座り続ける若者のモラルはいったいどうなってるんだ。	
\\	最近の高校生って足を広げて、ふんぞり返って座ってるヤツが多いから迷惑だ。	
\\	電車の中で化粧をしている女性を最近よく見かけるけど、どういう神経してるんだろう。	
\\	電車の中でいちゃいちゃしてる若いカップルがいたんでびっくりした。	
\\	こういう若者を見ていると「プライベート」と「パブリック」の感覚がずれてきてるのかなあ、と疑問に思っちゃう。	
\\	リュックを背負ってる若者が多いけど、満員電車の中では邪魔なんだよなあ。	
\\	乗り継ぎがいいと意外と早く着ける。	
\\	帰宅時に電車を待ってる時間って、無性に腹立たしい。	
\\	終電で帰ると午前様だ。	
\\	終電を降りたらダッシュしないと、タクシーを待つ長い列の後ろに並ばなければならない。	
\\	終電はお互いの路線で乗り継ぎを待つから、かえって早く家に着ける。	
\\	終電で乗り過ごしちゃったらたいへんだ。	
\\	上りの終電って早くなくなっちゃうよ。	
\\	ここを何時に出れば家まで帰れるの?	
\\	吊り棚においた荷物は忘れがちだ。	
\\	雨上がりは傘の忘れ物が多くなる。	
\\	傘を手すりにかけ忘れてる光景ってよく見かけるよね。	
\\	定期を落としたときは、どうすればいいの?	
\\	忘れ物や落とし物は、駅で数日預かった後その路線の拾得物収集所に集められるらしい。	
\\	満員電車で気づかないうちに時計のベルトが切れちゃった。	
\\	その時計は駅係員に聞いたけど、結局出てこなかった。	
\\	拾得物で多いのはやはり傘。持ち主の現れない拾得物はどう処分するんだろう。	
\\	吊り棚にのせた新聞や雑誌って、片づけるのがたいへんだろうなあ。	
\\	吊り棚にのってる新聞や雑誌を拾ってるおじさんがよくいるけど、何とも切ないなあ。	
\\	東京の地下鉄ってものすごく複雑で、路線図をよく見ないと目的地にたどり着けない。	
\\	最近はインターネットを使えば、出発駅から目的駅までどの路線を使えばいいか、時間がどのくらいかかるか、料金がいくらか、などがすぐにわかる。	
\\	地下鉄ってけっこう他の路線との相互乗り入れがあるから、知らないうちに地上を走ってたりするよね。	
\\	最近、地下鉄の壁面を広告として使う、なんてことがはじまったらしい。	
\\	早稲田に行くなら、吉祥寺まわりでいった方が乗り換えも少ないし、ゆったり座っていける。	
\\	子どもが電車の中で泣き出すと、おかあさんは慌ててなだめたり、すかしたりする。	
\\	小さな子どもが一生懸命に吊革につかまろうと背伸びしている光景ってほほえましい。	
\\	子どもつて靴を脱いで窓の方を向いて、流れいく景色を眺めるのが好きだ。	
\\	子どもにとっては、電車のいちばん前に立って線路が迫ってくる感じっておもしろいだろうなあ。	
\\	車両事故が起こると、他の路線に振替輸送が行われる。	
\\	人身事故の後始末って、気持ち悪いんでしょ。	
\\	「お客様が線路に立ち入ってるため安全が確認できるまで運転を見合わせています」。	
\\	「発車間際のかけ込み乗車は、ドアに挟まれる危険があるのでおやめください」。	
\\	昨日の台風で線路が冠水したらしく、電車は止まっていた。	
\\	大雪が降ると、首都圏の鉄道網はマヒ状態になる。	
\\	日本の駅って階段が多くて、車椅子の人たちには不便だろうなあ。	
\\	障害者の人たちがもっと気楽に外出できるように、鉄道会社はエレベーターを設置するなどの整備を進めるべきだ。	
\\	外国人にとっては、車内放送や案内表示などが不親切でわかりにくいだろう。	
\\	近郊連絡線の切符を買うのってなかなか難しい。	
\\	路線図を見て目的地までの運賃を探すのって、目が悪いとたいへんだ。	
\\	160円の切符を買うときに、210円入れて10円玉が5つ出てくると頭に来るよね。	
\\	切符を買う長い列に並んでいると、順番が来てはじめて運賃を確認したり、財布からお金を出したりする人がいて「並んでいるときに準備しておけよ」って言いたくなっちゃう。	
\\	時刻表をめくって、時間や運賃のスケジュールを立てるのは慣れが必要だ。	
\\	列車のトラブルで新幹線に閉じこめられと、電気もつかないような状態になるらしい。	
\\	この前、新幹線のトンネル内の壁面が剥がれ落ちたらしいけど、大事故にならなくてよかったねえ。	
\\	カラスのいたずらで、線路に置き石がされている。	
\\	いくら混んでいるからって、連結部分に立っていたら危ないよ。	
\\	つり革やてすりに直接触るのって気持ち悪いよね。	
\\	いやあ、子供が戸袋に手をはさまれちゃってさあ、電車がしばらく止まっちゃったんだ。	
\\	電車が動かないので、今日は遅刻します。	
\\	なんだかうるさいねえ。隣りの車両に移ろうか。	
\\	行きは何番線ですか?	
\\	すみません、具合が悪いんです。席、ゆずってもらえませんか?	
\\	お客さん、終点ですよ。	
\\	あ、線路に靴を片一方落としちゃった。	
\\	私の芸能ネタは、毎朝晩中吊り広告で仕入れています。	
\\	疲れてて立ったまま寝ちゃって、膝がカクってしちゃったよ。	
\\	「乗り越し、乗り継ぎのお客様は車内係員にお声をおかけください」。	
\\	キセルしたら通常運賃の3倍の料金を請求されるらしい。	
\\	自動改札って、定期や切符を裏返しに入れても大丈夫なんだよ。	
\\	自動改札にしたことによって、人件費は浮いたのかなあ。	
\\	学生時代に終電の終わった線路の上を歩いて帰ったことがある。	
\\	最近は単線なんて田舎にしかなくて、都心では複々線化しようとしているんだよ。	
\\	横浜駅は急行も特急も止まるので、何に乗っても大丈夫です。	
\\	ぶらりと一人旅でもしてくるかな。	
\\	旅の恥は掻き捨て。	
\\	会社の慰安旅行で箱根に行きます。	
\\	修学旅行では、よく枕投げをした。	
\\	時差ボケで辛い。	
\\	車酔いするほうだっけ?	
\\	お隣さんは北海道まで家族旅行だって。	
\\	夢は、世界一周旅行	
\\	ニューヨークまでの直行便をお願いします。	
\\	窓際の席にしてください。	
\\	飛行機は早割でとれば安いよ。	
\\	格安航空券を利用しよう。	
\\	キャンセル料3割もとられるんだって。	
\\	結局はツアーに入っちゃうのが安いよね。	
\\	平日料金と休日料金で、1万円も違うんだ。	
\\	あのツアー、キャンセル待ちの状態だって。	
\\	万が一の時のために、保険には入っておいたほうがいいよ。	
\\	政府観光局から資料をもらおう。	
\\	期間内乗り放題チケットを利用しよう。	
\\	指名通話で国際電話をかけるといいよ。	
\\	旅は情報収集が重要。	
\\	飛行機の到着予定時刻は?	
\\	予防接種を受けていかなきゃいけない。	
\\	今、日付変更線越えたって。	
\\	空港へのアクセスがいいところに泊まろうよ。	
\\	入国手続きって緊張しちゃう。	
\\	ツインの部屋にエキストラベッドを入れてください。	
\\	角部屋でお願いします。	
\\	海に面した部屋でお願いします。	
\\	二組隣同士の部屋でとってください。	
\\	ホテルの部屋にドライヤーはついていますか?	
\\	あの民宿なら、素泊まり5千円だよ。	
\\	空港からホテルまでの送迎バスはありますか?	
\\	何日前までに予約すれば格安料金の対象になりますか?	
\\	このバッグ、機内持ちこみ
\\	なはずなんだけど。	
\\	出発20分前になったら、搭乗案内のアナウンスがかかるって。	
\\	荷物検査で引っかからなければいいけど。	
\\	飛行機の時間が遅れてるらしい。	
\\	台風で飛行機が飛ばないかもしれない。	
\\	予定より少し送れて成田に到着した。	
\\	パスポートなくした。再発行しなくちゃ。	
\\	お昼は駅弁にしよう。	
\\	機内食って、美味しくないね。	
\\	背もたれもっと倒せるよ。	
\\	離陸の時って耳がツンとする。	
\\	飛行機でお酒飲むと、気圧のせいか酔いやすい。	
\\	スリッパでロビーに降りたらだめだよ。	
\\	旅館は、上げ膳据え膳がうれしいね。	
\\	チェックインまで荷物を預かってもらえますか?	
\\	ホテルまでタクシーを呼んでください。	
\\	チェックアウトのときにまとめて清算してください。	
\\	この近辺で、美味しい郷土料理を食べさせてくれるお店はありますか?	
\\	この郵便物を出しておいてもらえますか?	
\\	今釣ってきた魚を、生き造りにしてもらえますか?	
\\	この旅館は露天風呂がすばらしいのよ。	
\\	宿が見つからなかったら、野宿しかないね。	
\\	こういう無人販売所は、都会では見かけないよね。	
\\	もう一日延泊できますか?	
\\	この温泉は穴場だね。	
\\	朝食はバイキング形式だって。	
\\	タクシー一日借り切ろうか。	
\\	せっかだから土地のものを食べようよ。	
\\	今日はビーチでのんびりしよう。	
\\	明日の予定は、天候によるね。	
\\	明日は、寺巡りをしよう。	
\\	うわぁ、観光客だらけ。	
\\	海をバックに記念写真撮ろうよ。	
\\	土地の人に案内してもらうのが一番だよ。	
\\	都会の喧騒を忘れるね。	
\\	日焼け止め塗ったほうがいいよ。	
\\	旅先で仕事の話しないでよ。	
\\	食事がまずいと気がめいるよね。	
\\	水着の上に羽織るれるもの持っていったほうがいいね。	
\\	今年はビキニが流行ってるみたいね。	
\\	ここの地ビールは美味しい。	
\\	旅に出たら地酒を飲まなきゃ。	
\\	9月の海は、人がまばらだね。	
\\	日焼けして痛い。	
\\	海に潜ったら耳抜きしてください。	
\\	観光案内所にいってパンフレットをもらってこよう。	
\\	旅の間の共通のお財布をつくろう。	
\\	こっちの人はのんびりしてるね。(←旅先でよく言う「こっちの人」という言い方)	
\\	遊覧船が出てるらしいよ。	
\\	川くだりでびしょびしょになった。	
\\	水道水は飲まないほうがいいよ。	
\\	白タクには乗らないように。	
\\	異国情緒あふれる街並みが続く。	
\\	ガイドブックを過信しちゃいけない。	
\\	ちょっと足をのばして、次の町まで行ってみようよ。	
\\	タクシーの初乗り料金が450円だって。	
\\	このお土産かさばるなぁ。	
\\	いきあたりばったりの旅もいいね。	
\\	いかにもリゾートってかんじの格好ね。	
\\	ぶらぶら寄り道していこうよ。	
\\	この通りのモーテルはどこも満室だった。	
\\	レンタカーの乗り捨て料金はいくらだろう?	
\\	明日は、朝市に行こう。	
\\	次のドライブインで休憩していこう。	
\\	この山道は、観光客用に足場が整えられている。	
\\	景色のいいところで車をとめて、休憩しよう。	
\\	ビッグバンによって宇宙は誕生したといわれている。	
\\	あの星までの距離はいったい何光年あるんだろう?	
\\	惑星って太陽の周りを公転している星のことだよね。	
\\	火星には生命体がいるかもしれない。	
\\	木星には衛星はいくつあるの?	
\\	地球からでも土星の輪は見えるんだって。	
\\	あの星はガスの膜で覆われている。	
\\	太陽系は銀河系のどのあたりにあるんだろう。	
\\	ヘール・ボップ彗星が地球に接近した。	
\\	ブラックホールの謎はまだ解明されていない。	
\\	ガガーリンはソ連の宇宙飛行士だったよね。	
\\	僕の将来の夢は宇宙開発事業団で働くことです。	
\\	これが世界最大の天体望遠鏡です。	
\\	太陽の寿命が気になる。	
\\	宇宙には、使われなくなった人工衛星が無数にある。	
\\	火星の探査機によって水の存在が確認されるかもしれないね。	
\\	西暦2800年ころに惑星直列が起こるらしいよ。	
\\	オーロラの発生は太陽の活動が大きな要因となっています。	
\\	宇宙遊泳には当然宇宙服が欠かせない。	
\\	これが新しくできた星雲の写真です。	
\\	人類で初めて月面歩行をした人はだれだっけ?	
\\	宇宙ステーションの建設には莫大な費用がかかった。	
\\	無重力状態を体験してみたいな。	
\\	大変だ、うちの庭に隕石が降ってきた!	
\\	隕石
\\	宇宙旅行ができる日もそう遠くないらしい。	
\\	最近の天気予報はよくあたります。	
\\	あんまり天気予報をあてにしたらだめだよ。	
\\	毎朝必ず天気予報をみてから外出します。	
\\	さっきまでの快晴がうそのようだ。	
\\	雨が降ったり止んだりではっきりしない天気です。	
\\	雲ひとつない素晴らしい天気です。	
\\	ここ数日空は厚い雲に覆われている。	
\\	昨日とはうって変わって素晴らしい天気です。	
\\	行くか行かないかは天気次第です。	
\\	ここ数日天気がめまぐるしく変わっている。	
\\	ここ何年か異常気象がつづいている。	
\\	このところいい天気がつづいている。	
\\	いい天気だから窓をあけてもいいですか?	
\\	さわやかな風が窓からはいってくる。	
\\	雲行きがあやしくなってきました。	
\\	山の天気は変わりやすいから気をつけてください。	
\\	陽射しが強いので日射病に気をつけてください。	
\\	陽射しが強いので帽子をかぶろう。	
\\	雲の切れ間から太陽が見えます。	
\\	急に雨が降ってきたから喫茶店で時間をつぶしてました。	
\\	どっか雨宿りする場所がないかな~	
\\	久しぶりに天気がよかったので遠出をしました。	
\\	朝と夜の気温差が大きいので体調に気をつけないと。	
\\	午後には夕立が降りそうだ。	
\\	通り雨だからすぐ止むよ。	
\\	ここ数日大気の状態が不安定です。	
\\	今夜から天気は荒れ模様です。	
\\	天候に恵まれ行楽地は大勢の人で賑わった。	
\\	太陽の照り返しが強い。	
\\	天気がいいので洗濯物を乾していこう。	
\\	試合が行われるか行われないか微妙な天気だ。	
\\	彼女はみんなから「あめおんな」と言われている。	
\\	気温が下がると関節が痛むんだ。	
\\	日を追う毎に暑くなっていくみたい。	
\\	今日は抜けるような青空だ。	
\\	午後から天気は下り坂になります。	
\\	天気は西から除々に回復してきます。	
\\	午後から晴れ間がのぞくでしょう。	
\\	雨あしが強くなる前に家に帰ろう。	
\\	梅雨入りは来週以降になりそうだ。	
\\	明日から梅雨前線が活発になります。	
\\	梅雨時期は洗濯物が乾きにくい。	
\\	毎日雨で気分が憂鬱です。	
\\	空気がべとべとして気持ち悪い。	
\\	明日の夜から雨だから置き傘をしていこう。	
\\	各地で雷を伴って激しい雨が降りました。	
\\	今日の風は心なしか湿って感じられる。	
\\	梅雨らしく空はどんよりしている。	
\\	さっきりより雨あしが強くなっている。	
\\	外はどしゃ降りです。	
\\	夕焼けがきれいだし明日も晴れるだろう。	
\\	今年の夏の降水量は平年並みだそうだ。	
\\	毎日熱帯夜で寝不足です。	
\\	今年の台風の発生率は去年以上です。	
\\	台風は各地におおきな被害をもたらしました。	
\\	台風の今後の進路に充分注意してください。	
\\	台風は明日の正午に温帯低気圧にかわる模様です。	
\\	台風の接近に伴って海は大荒れです。	
\\	関東地方は2時間後に暴風域にはいります。	
\\	発達した低気圧が通過するので強風に気をつけてください。	
\\	明日は台風一過でよく晴れるでしょう。	
\\	大雨のため野外コンサートは中止になった。	
\\	海水浴は悪天候のため来週に延期になった。	
\\	激しい雨で視界がよく見えません。	
\\	昨日からの激しい雨で河川が増水している。	
\\	今年も集中豪雨の被害が相次いだ。	
\\	大雨のため客足が思っていたほど伸びなかった。	
\\	今年の夏は冷夏になりそうだ。	
\\	気象庁の発表によると今年は猛暑らしい。	
\\	さすがにこれだけ暑いと夏バテしてきた。	
\\	こんなに晴天がつづくと水不足が心配だ。	
\\	各地で雷雲が発生した。	
\\	落雷の影響でほとんどの家庭が停電した。	
\\	日に日に秋らしい天気になってきた。	
\\	天候に恵まれて運動会は無事終わった。	
\\	強風で看板が倒れている。	
\\	玉突き事故は道路の凍結が原因です。	
\\	空気が澄んでいるので星がよく見える。	
\\	空気が乾燥しているので火の元には気をつけてください。	
\\	視界が不良のため飛行機の到着時間が遅れた。	
\\	大雪のため交通機関がマヒした。	
\\	朝起きてみるとあたり一面銀世界だった。	
\\	明日は雪だからいつもより早く家を出よう。	
\\	例年よりはやく霜がおりた。	
\\	いつの間にか雨がみぞれに変わっていた。	
\\	底冷えする寒さが何日も続いている。	
\\	寒さが身にしみる季節になってきた。	
\\	雪が積もっているので雪合戦をしよう。	
\\	喫茶店にはいって寒さをしのいだ。	
\\	あまりの寒さに鳥肌がたった。	
\\	朝からしとしとと雨が降っている。	
\\	春一番が各地で吹き荒れました。	
\\	早く春になるといいんだけどな~	
\\	この雨でせっかくの桜も散っちゃうな。	
\\	曇ってきたけど、雨は降らないと思うよ。	
\\	花が咲き、鳥が鳴いている。春だね。	
\\	港が霧に煙っている。	
\\	雨が降るときには、右肩で感じるんだ。	
\\	日に日に夏(冬)らしくなるね。	
\\	日に日に春(秋)らしくなるね。	
\\	ああいう子供がそのまま大きくなるから、若者の犯罪が増えるんだ。 
\\	「ああいう」は
\\	「あのような」で表現します。
\\	「干渉」。
\\	「上がる傾向にある」。	
\\	あの事故以来、もう二度とああいう思いをしたくなくて、周りにも迷惑をかけたくないので自転車に乗ることをやめました。 
\\	この場合の「思い」は 
\\	(経験)で表現すればいいでしょう。	
\\	きみの、ああ言えばこう言う的態度には実際頭に来てるんだよ。 
\\	(言い返し)と 
\\	を組み合わせて「ああ言えばこう言う」を表現しています。	
\\	済んでしまったことをああだこうだ言ってもどうにもならないだろう。 
\\	は「好きなだけ論評する」、
\\	は「いつまでも話す」の意味になります。	
\\	勧誘って、何でああもしつこいのだろうか。何度来られても買わないものは買わないっていうの。まったく。 
\\	「ああもしつこい」のニュアンスを 
\\	(…にうんざりする)を使って表現しています。
\\	は「神経にさわる」の意味。	
\\	彼は相変わらず泣き虫だ。 
\\	「相変わらず」は
\\	で簡単に表現できます。
\\	は「泣き虫、弱虫」のことです。	
\\	星占いによると彼女と僕はあまり相性がよくないようだ。 
\\	「相性」は
\\	「化学反応」で表現できます。
\\	「一致する」。	
\\	あいつとはどうも相性が合わないんだ。 
\\	「相性が合う」とは「うまくやっている」、つまり、
\\	で表現できますので、これを否定にすればいいのです。また、
\\	「波長が合う」という表現を用いてもいいでしょう。	
\\	(どんなことがあっても)貴方を愛することに変わりはないから。 
\\	「愛する」は
\\	、「どんなことがあっても」は
\\	を使って表現します。	
\\	店の店員は、なんか働かされているって感じで、愛想が悪い人がやけに多かった。 
\\	は「無愛想な」という意味の形容詞。
\\	「上辺だけの」。	
\\	毎晩飲み歩いてばかりいたら、奥さんに愛想尽かされるぞ。 
\\	「愛想をつかす」は
\\	「立ち去る」でいいでしょう。
\\	一緒にいる」。	
\\	この電話機壊れてるみたい。相手はこっちの声が聞こえるみたいだけど、こっちは相手の声が全然聞こえない。 
\\	「(電話の)相手」は 
\\	でいいでしょう。	
\\	お互いに相手を思いやる気持ちがだんだん薄れてきている。 
\\	で「お互いに相手を思いやる気持ち」の意味になります。
\\	は「薄れる」の意。	
\\	こういう時は相手の出方を見てその上でどうするか判断しなきゃ。 
\\	「相手」を 
\\	(相手、敵)、
\\	で表現しています。
\\	は「…に基づいて」の意	
\\	前年度優勝校と一回戦で当たってしまい惨敗した。相手が悪かった。 
\\	「釣り合いの取れていない試合」で「相手が悪かった」を表現しています。	
\\	約束を守らなと、みんなから相手にされなくなるぞ。 
\\	「相手にしない」は
\\	「真剣に受け止めない」で表現できます。「約束を守る」は
\\	、「破る」は
\\	を使います。	
\\	その会社に新製品を売り込もうと電話を掛けたが、相手にされなかった。 
\\	で「対応したがらなかった」の意味になります。
\\	で「(その人に)時間をあげない」の意	
\\	時には相手の立場になって考えてみることも必要だよ。 
\\	が「相手の立場になる」に当たる言い回しです。
\\	は「見方、考え方」の意。	
\\	メールって時間を気にせず送れるからとても便利だと思う。電話だとどうしても時間とか相手の都合を気にするし。 
\\	と 
\\	を組み合わせて「相手の都合を気にする」を表現しています。
\\	も使えます。	
\\	相手の身になって物事を考えることができる人になりなさいよ。 
\\	「相手の身になって考える」とは
\\	(他の人の視点から物事を見る)ということですね。
\\	は「思いやりのある」の意。	
\\	この計画はアイディア勝負だ。 
\\	「勝負」を
\\	「鍵」、
\\	で表現しているところがポイントです。	
\\	君の愛読書って何? 
\\	「愛読書」とは、
\\	「一番好きな本」ということですね。	
\\	今回の選挙で、敵陣営は多党相乗り候補だったので本当に勝つのに苦労した。 
\\	で「複数の党が支援する」で「多党相乗り」を表現しています。
\\	は「承認、推薦」の意。	
\\	あいみつ取るべきだね。 
\\	とは「入札にする」の意味。
\\	は「請負業者」、
\\	は「競い合う」。	
\\	ワイシャツ、アイロンかけるの面倒くさいな。 
\\	は動詞として使えば「アイロンをかける」の意味になります。「ワイシャツ」は
\\	といいます。	
\\	きみと僕のスケジュールが合うということなどこれまでなかった。 
\\	には「噛み合う、調和する」という意味があります。
\\	は「うまく合う」の意。	
\\	あえてあの店には行かないようにしているんだ。 
\\	で「特に 
\\	するようにしている」の意味。	
\\	その計画はまだ青写真の段階だ。 
\\	「青写真」はそのまま英語でも
\\	と言います。
\\	「計画の段階にある」。	
\\	派手にすっ転んだので膝に青タンが できてしまいました。 
\\	には名詞「打ち身」、動詞「傷める」の意味があります。この場合の「青」は英語では
\\	と言います。	
\\	お前のような青二才が偉そうなことを言うなよね。ろくな仕事もできないくせに。 
\\	「青二才の」は
\\	「未熟な」、
\\	で表現できます。
\\	「とるにたらない奴、くだらない奴」。	
\\	(レントゲン撮影で)右からぐるっと回って、もっと、もっと、はい、そのまま仰向けになってください。 
\\	「仰向け」は
\\	と言います。ちなみに、「うつぶせ」は
\\	です。	
\\	どうして、英語ができないと生き残れないなんてあおるんだ? 
\\	「あおる」は
\\	と
\\	を組み合わせて
\\	しないといけないと言う」で表現できます。また、
\\	「絶えず、... するようにプレッシャーをかけられる」とも言えますね。	
\\	私たちは、赤い糸で結ばれていたのよ。なんてね 
\\	「赤い糸で結ばれている」は
\\	する運命にある」で表現できます。
\\	「運命」。
\\	は「そんなことないか」の意。	
\\	お酒を飲むとすぐに赤くなっちゃうんです。 
\\	「赤くなる」は
\\	、あるいは、
\\	「パッと赤くなる」でいいでしょう。	
\\	そんな事は、彼にとっては赤子の手を捻るようなものだ。 
\\	「赤子の手を捻るようなもの」をここでは
\\	「赤ちゃんからキャンディーを取るようなもの(ちょろい)」で表現しています。
\\	は「楽なこと」の意。	
\\	赤信号みんなで渡れば怖くない。 
\\	は「赤信号で渡る」、
\\	は「怖い」の意。
\\	「数の多い方が有利(安全)」	
\\	うーむ、私の会社赤字だからなあ。倒産しなけりゃいいんだけど…。 
\\	で「赤字の状態にある」という意味になります。
\\	は「破綻する」、
\\	は「倒産する」の意。	
\\	私が目標を達成した暁には何を買ってくれる? 
\\	「暁には」とは 
\\	(…の日が来たら)という意味ですね。	
\\	あの子は髪を切ってから、あかぬけたね。 
\\	は「洗練された、上品な」という意味です。
\\	「粋な、流行の」。	
\\	アカペラで歌う。 
\\	「アカペラで歌う」は、そのまま
\\	と言えます。
\\	とは「伴奏やコーラスがなくても」の意。	
\\	それって、あからさまないやがらせだね。 
\\	「あからさま」は
\\	「明らかな」を使って表現できます。
\\	「嫌がらせ」、
\\	「追い回す」。	
\\	彼女って明るい人でしょ? 
\\	ここでは「明るい」を
\\	「朗らかで元気な」で表現しています。	
\\	こちらは日本と比べて、夏の日はとても長く、朝6時前から夜の9時30分ごろまで外が明るいです。 
\\	「明るい」は名詞 
\\	(日中)、形容詞 
\\	(明るい)で表現できます。	
\\	僕ってあがり性だから、人の前で話すの苦手なんだよ。 
\\	「あがり性」を
\\	「人の前で緊張するタイプ」と説明しています。	
\\	さっきから花火らしき音がしているけど、どこで上がっているんだろう・・めずらしい平日に。 
\\	「(花火を)上げる」は
\\	、「上がる」は
\\	で表現できます。	
\\	人前に出るとあがってしまって、うまく話せない。 
\\	で「あがる」の意味になります。
\\	を使ってもいいでしょう。
\\	は「ものが言えなくなる」の意味。	
\\	最近、近所で空巣が多発している。共働きの人は気をつけないとね。 
\\	「空き巣」は
\\	と言います。
\\	「注意する」	
\\	空き巣に入られたことありますか?入られないために何かしていることはありますか? 
\\	「空き巣」には 
\\	という言い方がありますが、ここでは 
\\	(泥棒に入る)を使っています。	
\\	彼は飽きっぽい性格だから、一つの仕事が長続きしないんだ 
\\	で「落ち着きがなく飽きやすい」の意味になります。	
\\	これは今までいろいろ試した中で、一番私に合っている英語の勉強方法かもしれないと思っています。これなら、飽きずに長続きしそうです。 
\\	で「飽きずに」の意味になります。
\\	は動詞で、「…に飽きる」の意。	
\\	あきらめが早いね~。もっと粘れば手に入れられるかもしれないのに。 
\\	で「いとも簡単にあきらめる」の意味になります。
\\	は「粘り強い」の意。	
\\	おまえは相変わらず諦めが早いな。もう少し頑張ってみたらどうだい。 
\\	が「諦めが早い」に当たる言い回しです。
\\	を使ってもいいでしょう。	
\\	ふられたくせに、また、彼女に電話するなんて、お前も諦めが悪いな。 
\\	で「諦め時を知らない」の意味になります。	
\\	時には諦めも肝心だよ。 
\\	の文字通りの意味は「譲る(諦める)のが一番いいことだってある」です。	
\\	今回はあきらめな。しょうがないよ。 
\\	「あきらめる」は
\\	を使えばいいでしょう。
\\	「無駄にする」。	
\\	何か新しいことを始めるには何かをあきらめなければならないこともある。 
\\	「あきらめる」は 
\\	で表現すればいいでしょう。	
\\	あいつの馬鹿さ加減には、もう、あきれちゃうよ。 
\\	は「ぎょっとさせる」、
\\	は「嫌になる」の意味。
\\	「馬鹿な行動」。	
\\	もう呆れてものも言えないよ。 
\\	で「愕然とする」の意味になります。
\\	「うんざりする」。	
\\	今ダイエット中だからなんとか言いつつお菓子食べてるし。あきれるよ。 
\\	「…は十分だ」を使って「あきれる」を表現しています。
\\	は「むしゃむしゃ食べる」、
\\	は「うんざりする」の意。	
\\	ストレスがたまっては食べて、食べ過ぎてはストレスがたまって…こんな悪循環の繰り返しー。 
\\	「悪循環」は
\\	と言います。「ストレスがたまる」の表現の仕方にも注意してください。	
\\	責められた彼は、開き直って悪態をついた。 
\\	は「悪態をつく」、
\\	で「早口でののしる」の意味。
\\	は「挑戦的な態度を取る」の意味で、
\\	でも同じような意味です。	
\\	自分たちの悪事がばれないようにするなんて、なんて悪どい奴等! 
\\	とは「無情、冷酷な」の意味になります。
\\	「悪行」	
\\	お前、会議中あくびばかりしてたじゃないか。 
\\	「あくびをする」は
\\	ですね。「ばかりしている」を
\\	を使って表現しています。	
\\	彼の話があまりに長くてつまらないので、あくびをこらえるのに一苦労した。 
\\	で「…するのを差し控える」の意味になります。
\\	は「あくびする」の意。	
\\	誰のせいでもない。責任は、あくまでも君に取ってもらうよ。 
\\	ここでは「あくまでも」を
\\	「固く決心している」を使って表現しています。	
\\	若いころは、右手の握力が60キロ以上あったんだが。 
\\	「握力」は 
\\	といいます。	
\\	明け方まで仕事をしていたから、眠くてしかたがない。 
\\	は「夜明け」、
\\	は「空が白み始めるころ」の意味。
\\	「疲れ切った人」。	
\\	練習に明け暮れる毎日を送ってます 
\\	「…ばかりする」と
\\	(来る日も来る日も)を使って「明け暮れる毎日を送る」を表現しています。	
\\	前もって言ってくれれば、その日、空けておいたのに。 
\\	「その日をあける」は
\\	あるいは
\\	で表現できます。	
\\	そんなに気を落とさないで。明けぬ夜はないんだから。 
\\	(夜明け前は真っ暗だ)を使って「明けぬ夜はない」を表現しています。
\\	でもいいでしょう	
\\	人の揚げ足を取るのは止めろよ。 
\\	共に「…のあら探しをする」という意味です	
\\	彼は揚げ足とりの天才だ。 
\\	「揚げ足をとる」とは
\\	「あら探しをする」ことですね	
\\	この間、気づいたらサイドブレーキ上げっぱなしで運転してた。自分でも、よく車動いてたと思うけど。 
\\	で「サイドブレーキを上げた状態で」の意味。「サイドブレーキ」は英語では
\\	と言います。	
\\	彼を好きというより、ただあこがれているだけなの。 
\\	「あこがれ」をここでは
\\	「とっても魅力的だなと思っているだけ」で言い換えています。
\\	「べたぼれ、のぼせ」。	
\\	ディズニーランドのパレードを見て、ああいう人たちに憧れて劇団に入団したのだけど、現実にはハードルが高かった。 
\\	「憧れて」は
\\	のようになりたい」、
\\	を尊敬する」で表現できます。
\\	「劇団」。	
\\	言葉で表現できないくらい、とにかく感動した。だって、憧れの人が目の前にいるんだよ。 
\\	で
\\	に憧れる」に意味になりますので、これを使えばいいでしょう。
\\	に首っ丈、熱を上げる」。	
\\	昔から、メガネかけている人を見ると憧れる。私は目がいいからメガネをかける必要がないから。 
\\	「憧れる」を 
\\	(羨ましい)で表現しています。
\\	と 
\\	の組み合わせにも注目。	
\\	今回の出張は、先方の負担で顎足つきだ。 
\\	「顎足つき」とは
\\	「食事(顎)と交通費(足)を相手が持つ」ということですね	
\\	笑いすぎて、顎がはずれそうになった。 
\\	は「脱臼する」という意味。
\\	「カックッとはずれる」。	
\\	自分のこと何様だと思ってるんだあいつは。人をあごで使いやがっ 
\\	(人をこき使う)で「あごで使う」を表現しています。
\\	は「指図のままに」の意。	
\\	テニスは難しい。ボールがすぐあさっての方向に飛んでいってしまう。 
\\	「あさっての方向へ飛んでいく」とは「どこへ飛んでいくか分からない」ということですから、
\\	で表現すればいいでしょう	
\\	この寒いのに冷房入れるなんて、嫌がらせとしか思えないよ、まったく。朝っぱら気分悪い 
\\	「朝っぱら」は
\\	で表現できます。また、
\\	のように具体的に時間を言ってもいいでし	
\\	朝っぱらからテレビでサッカーの観戦をしました。ゴールが決まったときは空しく1人で興奮して叫びました。 
\\	(朝早くから)で「朝っぱらから」を表現しています。
\\	(空しい)の使い方にも注目。	
\\	朝の来ない夜はない。 
\\	ここでは「朝の来ない夜はない」を
\\	「夜明け前が最も暗い」、
\\	「いつも明日はやって来る」で言い換えています。	
\\	最近、朝起きるとあざができているのです。どうも足を重ねて横になっているみたいで、上の足が重くてそうなっているみたい。 
\\	この場合の「あざ」は 
\\	や 
\\	ではなく、
\\	で表現すればいいでしょう。
\\	(軽い痣になる)を使ってもいいですね。	
\\	お前の足臭いな。ちゃんと洗ってるのか 
\\	は
\\	よりも強烈な臭いがする場合に使います。	
\\	この計画は足がちっとも地に着いていないじゃないか。 
\\	は「現実性に欠けるもの」、
\\	は「絵空事」という意味です。	
\\	電車に乗ろうとして階段を駆け下りたら、足がつっちゃって、階段の途中で座り込んでしまった。 
\\	は「痙攣を起こす」、
\\	は「つる」の意味	
\\	札幌に出張に行って、足が出てしまった。 
\\	「足が出る」は
\\	「予算をオーバーする」を使えばいいでしょう。
\\	は「超える」の意。	
\\	最近の若いやつらは足が長いよね。俺の倍はある。 
\\	「足が長い」は
\\	と言います。「倍はある」の言い方も覚えておくといいでしょう。	
\\	きみもかなり足腰が弱ってきたようだ 
\\	が「足腰が弱くなる」に当たる言い回しです。	
\\	みんなの足手まといになりたくないから、今回の自転車旅行はやめておくよ 
\\	が「足手まとい」に当たる表現です。前置詞 
\\	と一緒に覚えるようにしましょう。動詞 
\\	の使い方にも注目。	
\\	叱られることがわかっていたので、職員室への足取りが重かった。 
\\	は「足を引きずる」、
\\	は「重い足取りで」、
\\	は「叱りつける」の意。	
\\	歩きすぎて、足の裏と足の指が痛い。靴のサイズがあってないのかな。 
\\	「足の指」は
\\	と言います。「サイズが合う(合わない)」の言い方にも注意。	
\\	当プロジェクトは、審議事項が多すぎて前に進むことができず、足踏み状態が続いている。 
\\	(保留となっている、待機状態の)で「足踏み」を表現しています。	
\\	"飲み過ぎて、なんか足元がフラフラする。 
\\	「足元がフラフラする」は
\\	という言い方をします。
\\	の代わりに
\\	を使ってもいいでしょう。その他にも、
\\	のような言い回しもあります。
\\	は「一杯余分に(飲んだ)」の意。
\\	医療費の高さってまるで患者の足元を見てるようだ。 
\\	の文字どおりの意味は「一番弱い時につけ込む」です。	
\\	大阪に行ったついでに足を延ばして京都見物をした。 
\\	「…に足を延ばす」を 
\\	で表現しています。
\\	は「観光」の意。	
\\	私、あなたたちのプロジェクトの足を引っ張ってるんじゃないかって心配なの 
\\	で「…の邪魔をする」の意味になります。
\\	を使っても同じようなことが言えます。	
\\	足を踏んだら、謝れよな。 
\\	「(足を)踏む」は 
\\	で表現します。	
\\	あの俳優の演技って味があるよね? 
\\	とは「やり方」の意味。
\\	(味わい)を使ってもいいでしょう	
\\	このスープ少し味がうすいよね。塩を入れようか。 
\\	は「さっぱりしている、味がうすい」の意味。
\\	でも「味があまりない」の意味になります。	
\\	この商品は革製品なので、使っている間にだんだん味がでてきます。 
\\	ここでは「味が出る」を 
\\	(より特質、特性が出る)を使って表現しています。	
\\	旅行から帰ったあとに日常生活が味気なく思えることって、よくあるよね。 
\\	は「味気ない、活気のない、単調な」の意味を表します。
\\	は「意味のない」の意。	
\\	なんかこのサラダ葉っぱだけ残って、だんだん味に飽きてきた。ほかのをオーダーすればよかった。 
\\	「味に飽きる」を 
\\	で表現しています。	
\\	冬になった今このあたりは、実に味も素っ気もない風景だ。 
\\	(つまらない、無感動な)と 
\\	(面白くない)を組み合わせて「味もそっ気もない」を表現しています。
\\	は「特色のない」の意	
\\	男の子であれ、女の子であれ、結婚するまでの預かり物だと私は思うことにしている。 
\\	「預かり物」を文脈から判断して
\\	「世話をする、面倒を見る」を使って表現してみました。	
\\	すみません、チェックアウトした後で荷物を預かってもらうことはできますか? 
\\	この場合の「預かる」は 
\\	で表現すればいいでしょう。動詞 
\\	を使っても同じようなことがいえます。	
\\	先日預けた洋服もうできていますか?(クリーニング屋で) 
\\	「預ける」は
\\	「持ち込む」、
\\	「置いていく」で表現できますね。	
\\	僕は汗っかきだから、カレーを食べたら汗だくになっちゃうんだ。 
\\	で「汗を一杯かく」の意味。「汗だく」の表現の仕方にも注目。	
\\	って言うから、一瞬駄目なのかと思ったら、
\\	ってことなんですよね?あ~焦った、よく間違えるんですよね。 
\\	(慌てる、狼狽する)で「焦る」を表現しています。この場合の 
\\	とは「言葉がつかえる
\\	の意味。	
\\	この間、高速を走っていたらいきなり光ったのを感じでスピード違反のカメラが作動したかと焦った。 
\\	この場合の「焦る」は 
\\	(パニックになる)で表現すればいいでしょう。
\\	も使えますね。	
\\	毎日汗水流して働いても、給料これだけだもんな。 
\\	で「血みどろになるまで働く」の意味になります。
\\	だけでも「汗水流して精一杯働く」という意味を表します。	
\\	焦らずに頑張ってね。 
\\	「焦らず」をここでは、
\\	「平静な」、
\\	「冷静に」、
\\	「落ち着いた」で表現しています。	
\\	司会者から、一言お祝いのスピーチって言われて焦ちゃったよ 
\\	この場合の「焦る」は
\\	「緊張する」でいいでしょう。	
\\	気持ちだけ焦ってしまって、全然行動が伴わない。 
\\	「焦る」は
\\	で表現できます。前置詞の
\\	は
\\	にしては、割には」の意味になります。	
\\	なんで皆あそこまで並んで、あのレストランで食べたいのだろう。ま、一度行ってみる価値はありそうだね。 
\\	「あそこまで並ぶ」のニュアンスを 
\\	で表現しています。
\\	と 
\\	の使い方がポイントです。	
\\	お前、完璧に遊ばれてるよ。実は見ちゃったんだ、お前の彼女がほかの男といるのを。 
\\	は「…をおもちゃにする、もてあそぶ」の意味。	
\\	彼の言動・行動には遊び心があって楽しい。 
\\	「遊び心」は 
\\	(遊び好きな)で表現すればいいでしょう。	
\\	ゴールデンウィークに、遊びすぎちゃってもうお金が底をついたよ。 
\\	は「出歩き過ぎる」の意味。
\\	と
\\	「破産」の使い方にも注目。	
\\	あの黒いスーツを着た老人、昔はかなりの遊び人だったらしい 
\\	「遊び人」は 
\\	でいいでしょう。
\\	は「遊び人」の意。	
\\	夏の日が長い分、冬の日は短いので今、思いっきり遊びまくってます。でないと、後悔しますから。 
\\	「遊ぶ」は 
\\	でいいでしょう。また、
\\	と表現することもできます。	
\\	最近あの子私と遊んでくれないんだよね。どうしたんだろう? 
\\	この場合の「遊ぶ」は 
\\	(出かける)で表現すればいいでしょう	
\\	朝勉強した方が効率がいいらしい。朝の1時間は夜の3時間にあたいするそうだ。 
\\	この場合の「あたいする」は
\\	、あるいは、
\\	「価値がある」を使えばいいでしょう。	
\\	彼はとても温かみのあるきれいな瞳をしている。 
\\	「温かみ」は 
\\	で表現すればいいでしょう。	
\\	所詮、無理な話なんだから、当って砕(くだ)けろだ。 
\\	(試しにやってみる)で「当たって砕けろ」を表現しています。
\\	で「無理なことを言う」の意。	
\\	砕
\\	慣れないことばかりで最初はあたふたしていたけど、やっと接客のコツをつかんできたって感じです。 
\\	で「絶えずうろたえている」の意味。
\\	は「…のこつをつかむ」、
\\	は「ひどく慌てる」の意味	
\\	あたふたしちゃって、みっともないったらありゃしない。 
\\	「あたふたする」は
\\	「パニック状態になる」、
\\	「あわてふためく」で表現できます。	
\\	住民の申し出は頭からはねつけられた。 
\\	「頭からはねつける」を 
\\	(にべもなく断る)で表現しています。
\\	は「即座に」の意。	
\\	あいつは婿養子だから奥さんに頭が上がらないんだ。 
\\	(人の言いなりになる)で「頭が上がらない」を表現しています。	
\\	これからのことを考えると、頭が痛いよ。 
\\	~で「~するのは頭が痛い」の意味になります。
\\	は「痛い」の意。	
\\	あいつは頭が固いから、付き合いにくい 
\\	で「自分のやり方で凝り固まっている」の意味になります。
\\	は「頑固な」の意。	
\\	彼の努力には、頭が下がるよ。 
\\	で「兜を脱ぐ」という意味になります	
\\	あの人、何でも人の頭越しにやっちゃうのよね。 
\\	「人の頭越しに」は
\\	と言います。日本語と同じですね。	
\\	何でそんな頭ごなしに決めつけるの?あなたはいつもそう。人の話を聞こうともしないで。 
\\	で「盲目的(よく考えず)に決めつける」の意味になります。
\\	でも同じようなニュアンスになります	
\\	あの時は頭に血が上って、何がなんだか分からなかったんだ。 
\\	「頭に血が上る」とは「カッとなる」ことですから
\\	で表現すればいいでしょう。	
\\	英会話のレッスンを受けたけど今日は疲れていて先生の話す英語がまるで頭に入ってこなかった。 
\\	と 
\\	を組み合わせて「何も意味をなさなかった」と言っています。
\\	は「印象に残る、記録に残る」の意。	
\\	英語の本を読んでも全然単語とか頭に入らない。他のこといろいろ考えすぎているのかな… 
\\	ここでは 
\\	(十分に理解される)で「頭に入る」を表現しています。
\\	は「(心に)残る、焼き付く」の意。	
\\	彼は頭の回転が速いから、どんな仕事でもてきぱきとこなす。 
\\	は「頭の回転が速い、素早く判断する」の意味。
\\	を使ってもいいでしょう。	
\\	その学校は、中・高一貫校だということを頭の片隅にでも入れておいて下さい。 
\\	で
\\	を心に留めておく」の意味になります。
\\	「結合させる」。	
\\	昨日、アメリカから帰ってきたので、まだ頭の中が英語環境だ。 
\\	ここでは「頭の中が英語環境」を
\\	「英語で考える」、
\\	「頭の中で英語がはね回っている」と言い換えています。	
\\	今更だけど、今の仕事に向いてないと思う。どうも頭を使うっていうのが~。年と共になかなかいいアイデアが思い付かなくなってきたし。 
\\	「頭を使う」はそのまま
\\	と言えます。
\\	に向いている」。	
\\	私って、新しもの好きで、店で、新製品(食べ物)をみるとすぐ買ってしまう。でも、そのうち半分は失敗するね。 
\\	「新しもの好き」はそのまま
\\	で表現できます。	
\\	とりあえず、あたりさわりのない話をしておいたよ。 
\\	には「操縦する、舵取りする」の意味があります。これを使って「微妙な話
\\	はしないように
\\	した」と表現しているのです。
\\	は
\\	に触れる」の意。	
\\	機械って当たりはずれがあるよね。 
\\	とは「不確実なこと、予想不可能なこと」という意味です。 
\\	は「運次第
\\	の意	
\\	なんでそんな当たり前のことをいうのかと思うでしょう?でもこれって凄く大事なことだよ。 
\\	「当たり前のこと」は
\\	という言い方をしますが、ここでは文の流れを考えて
\\	「つまらないこと(当たり前のようなこと)で大騒ぎをする」を使って表現してみました。	
\\	ご飯をよそったら炊飯器の蓋を閉めるのって当たり前のことでしょ。そんなこと教えられなくたってわかるじゃん。本当にあきれるわ。 
\\	ここでは「当たり前」を 
\\	と 
\\	で表現しています。	
\\	こたつで寝ちゃったら風引くのはあたりまえ。自己管理がなっていない。 
\\	「当たり前」は
\\	「もちろん」を使って表現できます。「こたつ」は
\\	とするか、
\\	と言ってもいいでしょう。	
\\	生地が薄くても当たり前だよ、安いんだから。それを承知で買うんだから仮にすぐ駄目になってもあとで文句言ったりするなよ。 
\\	(当然)を使って「当たり前」を表現しています。
\\	(予期する)の使い方にも注目。	
\\	日本は不況だと聞いていたけど、こちらも負けずに不況で2つ仕事をもっているのは当たり前のようです。 
\\	「当たり前」は 
\\	、あるいは、
\\	で表現できます。	
\\	昔、007シリーズのジェームズ・ボンドはショーン・コネリーの当たり役だった。 
\\	ここでは「当たり役」を 
\\	で表現しています。	
\\	ちょっと私にあたらないでよ!。私には関係ないことだもん。 
\\	は
\\	をいじめる」、
\\	は
\\	に八つ当たりする」の意味になります。	
\\	昨日のバザーで、ラッフルを買ったら、宿泊券が当たった。これで運を使い果たしたかな。 
\\	「当たる」は 
\\	で表現します。
\\	は「(慈善事業などで販売する)富くじ」の意。(注)2つ目の日本語は英訳していません。	
\\	時間があるといろいろ考えすぎて主人にあたってしまうので、悪い癖だとは思いつつ文句を言ってしまいます。 
\\	ここでは「あたる」を 
\\	で表現しています。
\\	は「…に八つ当たりする」の意。	
\\	よかれと思ってしたことがあだとなった。 
\\	ポイントは
\\	「期待に反した結果になる」の使い方です。	
\\	その事件は、あっけない幕切れだった。 
\\	で「ガッカリしたのまま状態だった」の意味。
\\	「不満足な」。	
\\	さっきまでぴんぴんしていた人が、心臓発作で倒れて亡くなった。人間なんて本当にあっけないものだ。 
\\	(突然終わる)を使って「あっけないもの」を表現しています。
\\	は「心臓発作」の意。	
\\	ああ、疲れた。重いものよりあっさりとしたものが食べたいな。冷蔵庫に何かあったかな? 
\\	この場合の「あっさり」は
\\	を使えばいいでしょう。
\\	は「ヘトヘトに疲れている」の意。	
\\	今回のホームステイは斡旋会社をとおさずに、自分で探していっったから、安くすんだ。 
\\	「斡旋会社」は
\\	「仲介をする会社」でいいでしょう。	
\\	冷めると美味しくないから、あったかいうちに遠慮しないで召し上がれ。 
\\	この場合の「うちに」は 
\\	(間に)を使って表現すればいいでしょう。	
\\	あっちっち!このお風呂、熱くて入れないよ。 
\\	と
\\	はそれぞれ「アオー」「イーク」と発音します。	
\\	ジャンケンポイ。 あっちむいてほい。 
\\	「ジャンケンポイ」は
\\	と言うようです。この後の
\\	!と
\\	は日本語の意味になるように英語にしてみました。ちなみに、アメリカでは「あっちむいてほい」という遊びはありません。	
\\	またもや、あってはならない飛行機事故が発生した。 
\\	「あってはならない」を 
\\	(弁解できない、言い訳の立たない)で表現しています。
\\	にも注意。	
\\	楽しい時って、あっと言う間に時間過ぎちゃうんだよね。 
\\	「あっと言う間に過ぎる」をここでは
\\	「飛んでいく」で表現しています。	
\\	これとあそこにあるのはどちらが厚い生地ですか? 
\\	厚さを表す場合には
\\	を使います。	
\\	その商品は、うちの店では扱いがありません。 
\\	には「店に売っている」という意味があります。
\\	「扱う」	
\\	暑かったり寒かったりして、毎日着る服を決めるのが難しい。 
\\	とは「温度の高低」のことです。
\\	は「変化」の意	
\\	すき焼きをごちそうになって、一晩泊めてもらったうえに、朝めしまで食べてきたなんて、ちょっと厚かましすぎるんじゃないか。 
\\	「厚かましい」は
\\	「(人の好意などに)乗じる、やっかいをかける」を使えばいいでしょう。
\\	人
\\	飲食物で「人に飲食物を出す」。これを
\\	を用いて受け身にして、
\\	で「すき焼きをご馳走になる」の意味を表しています。	
\\	誠に厚かましいお願いではありますが、お受けいただけると幸いです。 
\\	が「厚かましい」に当たる形容詞です。
\\	「大胆な、図太い」。	
\\	僕は暑がりだから、冬でも薄着で平気なんだ。 
\\	「暑がり」を
\\	「暑さにとても敏感である」で表現しています	
\\	まあそう熱くなるなよ。彼も本心からそう言った訳ではないんだから 
\\	この場合の「熱くなる」は
\\	で「取り乱す、怒る」ということですね。
\\	にカンカンになる」。	
\\	そろそろいいかなと思ってウールのワンピースを着ていったら、暑苦しいと言われた 
\\	で「不愉快なほど熱い」の意味になります。
\\	は
\\	の意味で使われています。	
\\	彼は精神的に成長し、人間としての厚みがでてきた。 
\\	ここでは「厚み」を 
\\	(深み)を用いて表現しています。	
\\	当てずっぽうで答えを書いたらあってたなんてラッキー! 
\\	は「当てずっぽうで(答える)」の意。
\\	でも同じような意味になります。	
\\	人の噂って当てにならないからな。 
\\	「当てにならない」とは
\\	「信じられない」ということですね。	
\\	彼が何て言ったか、当ててみて。 
\\	この場合の当てるは
\\	「推測する」を使えばいいでしょう。	
\\	こちらに来られたら私がアテンドします。 
\\	「アテンドする」とは 
\\	(面倒をみる、世話をする)ということですね。	
\\	後は私がやっておくから、もう寝ていいよ。 
\\	この場合の「後」は
\\	「残り」という意味ですね。	
\\	あと残り1キロ走れば、ゴールです。 
\\	「あと残り」を 
\\	で表現しています。この場合の「ゴール」は 
\\	で表現します。	
\\	つい感情的になって部下を怒鳴りつけてしまい、後味が悪かった。 
\\	が「後味が悪かった」に当たる言い回しです。	
\\	後々のために、今、少しでも貯金をしておいたほうがいい。 
\\	「後々のために」は 
\\	(将来のために)で表現すればいいでしょう。
\\	は「蓄える」の意。	
\\	えー、まだ11時。お昼まで後1時間もあるよ。 
\\	と
\\	の使い方がポイントですね。	
\\	試験が終わって、なんかあとからあとから、自分の間違った個所に気づく。あ~もういやだー。 
\\	(過去のことを振り返ってみる)を用いて「あとから」を表現しています。
\\	は「あとから考えると」の意。	
\\	プレゼントを貰うのはいいけれど、後が怖いな。 
\\	の文字どおりの意味は「それに伴う義務(義理)を除いては」です。	
\\	(通勤電車の中で隣の人を見て)この人、きっと寝坊したに違いないな。顔に枕のあとがばっちり残ってる。 
\\	「枕のあとが残る」をここでは
\\	「枕のあとがついているのが見える」で表現しています。
\\	を使ってもいいですね。「枕のあと」は
\\	と言いますので、一緒に覚えておきましょう。	
\\	犬の糞の後始末くらいしろよな。 
\\	「…の後始末をする」を 
\\	で表現しています。
\\	は「(犬や鳥などの)糞」の意。	
\\	後でね。 
\\	「後で」は
\\	を使えばいいでしょう。	
\\	後どのくらいでお風呂のお湯いっぱいになる? 
\\	と
\\	の使い方がポイントですね。
\\	「浴槽」。	
\\	後になって、やめておけばよかったなんて、思ったりしない?(結婚に関して) 
\\	ここでは「後になって」を
\\	と
\\	「そうなったとたんに」で表現しています。	
\\	そうやって面倒な事は後伸ばしにしているから、全然終わらないのよね。 
\\	が「後(先)のばし」に当たる言い回しです。
\\	「困難なこと」。	
\\	今さら騒いでもどうにもならないよ。後の祭りだよ。 
\\	で「どうにもならない」、
\\	「大切な時にそうすべきだったんだ」で「後の祭り」を表現しています。
\\	「今となっては遅すぎるよ」とも言えます。
\\	「発言する」。	
\\	後は、上司が認めてくれるかどうかだ。 
\\	で「残っているのは 
\\	だ」の意味になります。	
\\	あと一人、いればなあ。 
\\	「あと一人」は
\\	などで表現できます。	
\\	もうここまで来たらやるしかないでしょう。もう後戻りはできない。 
\\	「後戻りする、逆に戻る」は
\\	で表現できます。
\\	「ここまで(こんなに遠くまで)」。	
\\	このポテトチップおいしくて後を引くね。 
\\	は「おいしくてもっと欲しくなる」の意味。2番目の例文では「一つ食べたら止められない」と表現しています。	
\\	あの漫才師コンビの突っ込み役はアドリブがきかない。 
\\	「アドリブ」はそのまま
\\	が使えます。ここでは「突っ込み役」を 
\\	([ぼけ役を]引き立てるまじめ役)で表現しています(少しニュアンスが日本語とずれますが…)。	
\\	こんなへまをして、穴があったら入りたいよ。 
\\	は「くぼみ」の意味。
\\	「やりそこなう」、
\\	「どじ」。	
\\	何を言っているんですか。あなたの方が、適任でしょう。 
\\	「あなたの方が」はそのまま
\\	を使えばいいでしょう。
\\	「資格のある、適任の」、
\\	できる、やれる」。	
\\	きみは、ずっと以前からブラインド越しにきみの姿を探している僕を知らない。 
\\	「きみは…を知らない」を
\\	で表現しています。
\\	は「手がかり」の意。	
\\	いきなりビールを3本飲んでしまうなんて、佐藤さんも侮れないなあ。 
\\	は「軽量級、たいしたことのない」、
\\	「ごくごく飲み干す」の意味。	
\\	このラーメン屋は、安くて美味しいけど、あまり混んでなくて穴場だ。 
\\	とは「人目につかない場所」のことです。
\\	は「奥まった」の意。	
\\	そのワイシャツ買ったばかりなのに、もうタバコでそでに穴を開けちゃったの?もう~! 
\\	で「焦がして穴を開ける」の意味になります。
\\	は「穴を開ける」の意。
\\	という言い方はしません)	
\\	であのぐらいの点数があれば、もっと話せるはずだけどなー。 
\\	を副詞として用いて、
\\	を強調しています。
\\	の使い方にも注目。	
\\	まだ、あのときのお礼を言ってなかったわね。本当にありがとう。 
\\	「あの時のお礼」を
\\	「先日はありがとう」と言い換えて表現しています。	
\\	あいつもアバウトな性格だよな 
\\	は「几帳面な、細かいことに気を使う」の意味。
\\	「完全主義者」	
\\	「あばたもえくぼ」って言うからね。 
\\	「あばた」、「えくぼ」はそれぞれ英語で
\\	と言いますが、「あばたもえくぼ」を英語で表現する場合には、この2つの表現は使いません。「あばたもえくぼ」の意味に当たる一番ポピュラーな言い方が
\\	です。また、
\\	のような言い方もあります。	
\\	一夏のアバンチュールを求めて南の島へ行きました。 
\\	「アバンチュール」とはフランス語で冒険の意味ですが、日本語には、多少、ロマンチックなニュアンスが含まれていますので、
\\	と表現してみました。	
\\	所詮は泡銭だ。すぐに使ってなくなってしまうだろう。 
\\	とは「簡単に手に入れたお金(泡銭)」のことです。
\\	は「お金がすぐになくなる」、
\\	は「消えてなくなる」の意。	
\\	危ないところだった!買ったもの忘れて帰るところだった 
\\	には「きわどい、もう少しで 
\\	するところ」という意味があります。	
\\	思った通り、あの辺りはなんだか危なげな雰囲気の所だった。夜に出歩くのは気をつけたほうがいいね、きっと。 
\\	で「…は危ない感じがする」の意味。	
\\	結局は虻蜂とらずになってしまった 
\\	「欲張る」、
\\	「結局何も手に入らない」を組み合わせて「虻蜂取らず」を表現しています。
\\	は「強欲な」の意。	
\\	油がはねて危ないからこっち来ちゃだめ!あっちで待ってて。 
\\	「はねる」を 
\\	([油などが]ジュージューいう)で表現しています。	
\\	お好み焼き屋に行ったので、なんか服が油臭い! 
\\	「油臭い」は
\\	で表現できます。
\\	「いろいろと具の入っているクレープ(お好み焼き)」。	
\\	そんなところで油を売ってないで、デスクに戻って仕事をしなさい 
\\	は「ぶらぶら時間を過ごす」、
\\	は「無駄話をする」の意味。	
\\	彼にはどうしても勝てない。まだまだ甘いな。 
\\	「まだまだ甘い」は
\\	「学ぶことが沢山ある」、
\\	「まだまだ先は長い」などを使って表現すればいいでしょう。
\\	は「しのぐ」、
\\	人+
\\	で
\\	で(人)に勝つ」の意。	
\\	あの子、自分に甘すぎるよね。 
\\	この場合の 
\\	は「甘い、寛大で」の意味になります。
\\	は「気ままな、わがままな」の意。	
\\	アメリカに住んで、6カ月も住めば英語がペラペラになるだろうと思ってたけど、それは甘い考えだったようだ。 
\\	で「甘い考え方をする」の意味になります。
\\	は「希望的観測」の意。	
\\	何甘えたこと言ってんの? 
\\	で「甘えた、こびへつらうような調子」の意味になります。
\\	「おべんちゃらを使った」。	
\\	ひとに甘えるのもいい加減にしろ。 
\\	「甘える」とは人に頼る
\\	ということですね。	
\\	落ち込んでいるときって、誰かに思いっ切り甘えたくなる。 
\\	ここでは「甘える」を 
\\	「…に頼る」を使って表現しています。
\\	は「落ち込む」の意。	
\\	うちの猫って甘えん坊で、どこでも私のあと着いて来るのよね。 
\\	「甘えん坊」は
\\	「赤ちゃんみたい」で表現できます。
\\	溺愛する」の意。	
\\	舞台で踊ってみている人見て、憧れて劇団に入ったけど世の中そんな甘くなかった。ああいう世界を甘く見過ぎていた。 
\\	ここでは「甘く見る」を
\\	「あまりに単純な見方をする」で表現しています。
\\	「劇団」、
\\	「素人(芸術家)。	
\\	あなたの引き出しはまだ余っているじゃない。半分使わせてよ。 
\\	「余っている」を 
\\	を使って表現しています	
\\	俺、甘党だけど、お酒も飲むよ。 
\\	「甘党」には
\\	という言い方がありますが、ここでは
\\	より
\\	を好む」を使って表現しています。	
\\	彼ってあまのじゃくだから思っていることをそのまま言いたがらないの。 
\\	とは「つむじ曲がりの人」のことです。
\\	「率直に」。	
\\	ザーっときたから喫茶店で雨宿りしてたら、あっというまに雨が上がった。 
\\	ここでは「雨宿り」を
\\	「(避けるようにして)... に入る」と
\\	「土砂降りの雨から逃れるために」、
\\	「雨宿りをする場所」を使って表現しています。	
\\	シャツ生地のあまり切れを再利用したい。 
\\	「あまり切れ」は
\\	で表現できます。ここでは具体的に
\\	を付け足しています。
\\	(再利用する)、
\\	(切り落とす)。	
\\	スキーシーズンも終わりに近づいて、ウエアとかのバーゲンやっているけど、余り物しかなくて丁度いいサイズのモノがない。 
\\	「残り物」を使えば「余り物」が表現できます。「丁度いいサイズ」を英語でどう言うかもチェック!	
\\	電車の網棚に鞄を置いたまま、降りてしまった。 
\\	「網棚」は
\\	と言います。ちなみに、飛行機内の頭上の収納スペースは
\\	と呼びます。	
\\	あの人と出かけるときは必ず雨が降るので、雨男と呼ばれている。 
\\	日本語の「雨男」に当たる英語がありませんので、ここでは呼び名として
\\	と
\\	を使っています。
\\	には「(呪術で)雨を降らせる人」という別の意味もあります	
\\	うちらの中の誰かが雨男/雨女にちがいない。 
\\	英語には「雨男/雨女」に当たる表現はありませんので、ここでは
\\	を使ってその内容を説明しています。	
\\	大衆を従えるにはアメとムチが必要だ。 
\\	が「アメとムチを使う」に当たる言い回しです。	
\\	梅雨入り宣言されると、だいたいその日や次の日は、すごい晴れたりしている気がする。結局、梅雨といっても雨の量は少ないかも。 
\\	「雨の量」はそのまま 
\\	で表現できます。「梅雨入り」の表現の仕方にも注目。	
\\	さっきからあの人一人でにやにやして。怪しい。 
\\	「怪しい」をここでは
\\	「裏に何があるのか」、
\\	「何がそうさせているのか」で表現しています。	
\\	最近忙しいせいか、時々独り言を言っている自分に気づくのだけど、端から見たらきっと怪しいやつに思われているだろうな。 
\\	で「ちょっと変なやつ、薄気味悪いやつ」の意味。	
\\	もうこうなったら、洗いざらい白日の下にさらけだしなさい。 
\\	で「すべて白状する」の意味になります。
\\	は「一切白状する」の意。	
\\	あらかじめ申し上げておきますと、弊社の提供する求人情報は転職希望者を対象としておりますのでご了承下さいませ。 
\\	は「前もって知らせる」の意味。
\\	「対象とする」。	
\\	改めて、日本の文化や歴史に触れてみるのもいいものだと思った。 
\\	「改めて」のニュアンスを
\\	「戻る」、
\\	「新たにする」で表現しています。	
\\	悪い習慣を改めることによって、人生が大きく変わることがあります。 
\\	は「作り直し、補正する」、
\\	は「入れ替える、逆転する」の意味です。
\\	の「すっかり、劇的に」の意。	
\\	あらゆる手段で試したけど無理だった 
\\	で「あらゆる術策を弄する」の意味になる。	
\\	昼食は、ありあわせのもので済ました 
\\	で「手元にあったいくつかの素材で」の意味を表します。	
\\	彼女が、佐藤君と結婚するなんてありえないよ。 
\\	「ありえない」は
\\	で表現できます。
\\	は強調するための表現です。	
\\	この日に申し込みにいっても、希望者が多い場合は、すぐにクラスを受けられない事がありえたりしますか? 
\\	この場合の「あり得る」は 
\\	で表現すればいいでしょう。
\\	は「応募者」の意。	
\\	ああ参ったなぁ、1つ頼むの忘れてた。もう1回行ってきてくれるとありがたいんだけどなぁ。 
\\	「ありがたい」は 
\\	、あるいは、
\\	(感謝する)で表現できます。	
\\	せっかくの申し出なんだから、ありがたく受け取っておきなさい 
\\	「ありがたく受け取る」を 
\\	と 
\\	を組み合わせて表現しています。
\\	は「喜んで」の意。	
\\	寝坊したときに、改めてフレックス制度のありがたさを感じる。 
\\	「ありがたさを感じる」を 
\\	と 
\\	を組み合わせて表現しています	
\\	そういう事されるとかえってありがた迷惑だってわからないのかな? 
\\	(まったく好意にはならない)で「ありがた迷惑」を表現しています	
\\	今度の上司? 個性的どころか、ありがちなタイプね。 
\\	「ありがち」をここでは
\\	「他のみんなとほとんど変わりない」で表現しています。	
\\	宿が取れなくてスキー旅行は中止になったけど、あまりスキーしたい気分でもなかったので、ある意味では良かったのかもしれない。 
\\	が「ある意味で」に当たる表現です。
\\	は「そんなに …ない」の意。	
\\	大阪では、エスカレーターは左側が歩く人用で東京と反対なんだね。 
\\	で「エスカレーターで歩く人は左側に寄る」の意味になります。	
\\	ワインは飲みやすくてついつい飲みすぎるけど、アルコール度はかなり高いよね。 
\\	で「かなりアルコールの度が高い」の意味になります。
\\	「飲みやすさ」、
\\	「飲む」。	
\\	夢を叶えるためにはある程度の犠牲・努力・苦労は覚悟している。 
\\	で「ある程度の...」の意味になります。
\\	しようと心に決める」。	
\\	ハワイみたいにあまり日本人が多いのも考えものだけど、ある程度日本人が近くにいた方が多少心強い。気持ちの問題かもしれないけど 
\\	ここでは「ある程度」を
\\	「少なくとも何人かは」で表現しています。	
\\	先生のストってよくあることなの?明日って学校あるの? 
\\	この場合の「あるの?」は 
\\	で表現できます。
\\	は「ストをする」の意。	
\\	(この店では)アルバイトの募集ってしてますか? 
\\	「アルバイト」を 
\\	で表現しています。
\\	は「応募」の意。	
\\	写真を現像したらばらばらに置いておかずに、アルバムにきちんと整理したらどう? 
\\	で「きちんと整理する」の意味になります。	
\\	今日は残業で遅くなりそうだから、冷蔵庫にあるものを適当に食べておいて 
\\	で「冷蔵庫にあるものは何でも」の意味。	
\\	彼はホント後先のことを考えないで、あればあるだけ使うので目が離せません。 
\\	「あればあるだけ」のニュアンスを 
\\	(手に入れたとたんにすべてのお金を使ってしまう)で表現しています。「後先のことを考えずに」の表現の仕方にも注目。	
\\	あなたの事は何があっても一生忘れないわ。 
\\	で「何が起きようとも」の意味を表します。	
\\	彼女は私の好みに合わせて、料理を作ってくれる。 
\\	「(私の好みに)合わせて」は、簡単に
\\	、あるいは、
\\	でいいでしょう。	
\\	証拠は揃った。アリバイがないことを合わせて考えると、彼が犯人に違いない 
\\	「合わせて考える」は
\\	を活用すれば表現できますね。
\\	「証拠」、
\\	「アリバイ」。	
\\	メアリーももう長く日本に住んでいるのだから、そろそろ日本の習慣とかに合わせたほうがいいんじゃないの? 
\\	は
\\	に合わせる、調節する」、
\\	は
\\	に慣れる」の意味になります。	
\\	ちょっとこれ試着は無理ということなんですが、(体に)合わせてみるだけならいいですか? 
\\	で「自分の身体にあてて合うかどうか見る」の意味になる。	
\\	試着
\\	日本のあるインドカレー屋のカレーの味って、やっぱり日本人の口に合わせて作られているよね。 
\\	この場合の「合わせる」は
\\	「適合させる」でいいでしょう。
\\	「味覚」。	
\\	あいつには迷惑ばかりかけて、合わせる顔がない。 
\\	「合わせる顔がない」は
\\	(向き合えない)で表現すればいいでしょう。	
\\	そんなにあわてることはないよ。時間はたっぷりあるんだから。 
\\	「パニクる必要なない」、
\\	は「落ち着いている」の意。	
\\	案外、この本は読みやすいね。 
\\	は「驚くほど」、
\\	は「皆が思っている以上に易しい」の意味。	
\\	それを聞いて安心しました。 
\\	「安心する」とは
\\	「ホッとする」、
\\	「気分がずっとよくなる」ということですね。	
\\	既製品を買うよりも、自分で作ったほうが何が入ってるのか分かって安心するよね 
\\	この場合の「安心する」は
\\	でいいでしょう	
\\	あいつに仕事を任せておけば安心だ。 
\\	「安心だ」は
\\	「信頼できる」でいいでしょう。
\\	は「一つも心配することはない」の意。	
\\	こう何か安心できる材料が欲しいし、もっと自分自身に自信もつけたい。 
\\	で「安心させてくれるような何かが起こってほしい」の意味になります。
\\	「元気づける」、
\\	「取り戻す」。	
\\	お医者さんが、熱が下がるまでは安静にするようにって言ってたでしょ。 
\\	「安静にする」は
\\	「ベッドで(ゆっくり)横になる」ということですね。
\\	「指示する、処方する」、
\\	「おさまる、静まる」。	
\\	そんなに慌てて運転して、一体、どこへ行こうってんだよ。時間はあるんだから、安全運転してくれよ。 
\\	「安全運転」はそのまま
\\	で表現できます。	
\\	まだ安全性が確認されてないのに出荷してしまったらしい。大丈夫だと思うけど食べてしまったので心配。だれか責任とってくれるの? 
\\	で「安全であると確認する」の意味になります。
\\	は「確認する」の意。	
\\	その悲惨な記事を読み、暗たんたる思いを禁じえませんでした。 
\\	「暗たんたる思い」は 
\\	で表現すればいいでしょう。
\\	は「痛ましい」、
\\	は「ぞっとする」、
\\	は「絶望的に」の意。	
\\	気候が安定しないようですね。 
\\	動詞
\\	は「安定する」、形容詞
\\	は「不安定な」という意味	
\\	こちらに9時に来るように案内をいただいたのですが。 
\\	「案内をもらう」とは
\\	「招待を受ける」、
\\	するように言われる」ということですね。	
\\	あんなに悲しくて辛いものなら恋愛なんてもうしたくない。 
\\	この場合の「あんなに」は
\\	、あるいは、
\\	で表現すればいいでしょう。	
\\	案の定、あの業者今日になって泣きついてきたよ。 
\\	「案の定」は
\\	「思った通り」、
\\	するだろうと思った」で言い換えられます。また、「泣きついてきた」をここでは
\\	「もっと時間が欲しいと頼む」で表現しています。	
\\	ボーナスが出ないなんて、そりゃあんまりだよ。 
\\	までは名詞節。
\\	で「ひどすぎる」の意味になります。	
\\	それはあんまりだ! 
\\	!は「それで十分だ、もういい」の意味を表します。	
\\	今まで何もなかったところにこんな大きな家具を置くと威圧感があるね。でもすぐ慣れるか。 
\\	は「圧倒的な」の意味。
\\	(圧倒する)を使っても「威圧感」を表現することはできますね。	
\\	電話でならいいですよ。 
\\	この場合の「いい」は
\\	、あるいは、
\\	「気にしない」で表現すればいいでしょう。	
\\	この役者いい味だしているよね。なくてはならない存在だ。 
\\	で「素晴らしい魅力(風采、貫禄)がある」の意味になります。	
\\	あ、もちろんこれはいい意味で言ってるんだよ。 
\\	と 
\\	の組み合わせがポイントです。	
\\	ほんと、バカが付くぐらい人がいいのね。あ、勘違いしないでね。いい意味で言ったんだから。 
\\	で「いい意味で言う」の意味になります。
\\	「極端に」。	
\\	彼のあだ名は言い得て妙だよね。 
\\	「言い得て妙」とは
\\	「ぴったり」、
\\	「まさに図星」で言い換えられます。	
\\	そのイディオムを1つの動詞で言い換えるのは難しいと思うし、そんなに重要な事でもないと思いますが。 
\\	は「代わりに用いる」、
\\	は「言い換える」の意味です。	
\\	外面はすっごくいいわけ。だから、ユーザ受けはいいんだよね。だれにでもいい顔をする嫌なやつなんだよ。 
\\	「いい顔する」をここでは 
\\	(…をとてもよく扱う)を使って表現しています。	
\\	彼は堅物だから、どうも息子が転職するのにいい顔をしないんだよね。 
\\	「いい顔しない」をここでは
\\	「気に入らない」、
\\	「悩ます」で表現しています。
\\	「厳格な」。	
\\	いいかげんな話はやめろ。 
\\	「いいかげん」はいろいろな場面で使う表現です。例えば、
\\	「いいかげんにしろ」、
\\	「いいかげん嫌になった」など。	
\\	最近は何をするにもいいかげんになっちゃてるこんな自分はいやだけど、でもやめられない。どうにかなるでしょ。 
\\	で「ぞんざいなやり方で」の意味になります。
\\	「本気でなく、本腰を入れずに」。	
\\	あの英語の先生って、ちょっといいかげんだと思わない?だって、間違った英語を話していても全然直してくれないだもん。 
\\	「いいかげん」を
\\	(標準に達していない)で表現しています。
\\	は「見かけ倒しの人」の意。	
\\	いくつかツアー会社があるけど、1つはすごく対応がよくてあとは対応がいいかげんだと聞きました。 
\\	ここでは「いいかげん」を 
\\	で表現しています。
\\	の代わりに 
\\	(無責任な)を使ってもいいでしょう。	
\\	何その顔! もういい加減にしてよ!何か文句でもあるの? 
\\	が「いい加減にしてくれ」に当たる言い回しです。	
\\	もし機会があったら、行ってみるのもいいかもしれないよ。 
\\	この場合の「いいかもしれない」は 
\\	(…する価値があるかもしれない)で表現すればいいでしょう。	
\\	少し離れて相手を見るのもいい考えだと思う。 
\\	「いい考え」はそのまま
\\	を使えばいいでしょう。
\\	「新しい見方で」。	
\\	俺が上司に告口をしたって? いいがかりだよ、それは。 
\\	で「言いがかり」の意味になります。	
\\	だれだってそんなことされたらいい気しないよ。 
\\	「いい気持ちになる」は 
\\	でいいでしょう。
\\	は「嫌な思いをする」の意。	
\\	いろいろ悩みはあるけど、前向きに考えないとって自分に言い聞かせています。 
\\	で「…だと自分に言い続けている」という意味になります。
\\	の使い方にも注目。	
\\	一生懸命にやれば、いい結果が出るのもだよ。 
\\	で「いい結果を生む」の意味になります。	
\\	ママ、お洗濯干してくるから、ここでいい子で待っててね。 
\\	「いい子でいる」は、そのまま
\\	と言えばいいでしょう。「洗濯物を干す」の表現の仕方にも注目。	
\\	良いことを聞いた!今度試してみようっと! 
\\	には「情報、ヒント、こつ」という意味があります。	
\\	自分は働かずダンナの給料でブランド品買いあさってさぞやいいご身分ね。 
\\	(暮らし向きがいい)を使って「いいご身分」を表現しています。
\\	は「何もせずに」の意。	
\\	この間、歯医者行ったら、念のためとか言われて、歯形とレントゲンをとられた。いい商売してるよ。 
\\	「いい商売」は
\\	でいいでしょう。
\\	「(鋳)型」。	
\\	その位のこと、してくれたっていいじゃないねえ。 
\\	で「…してもどうってことない(害を与えない)」の意味になります。	
\\	自分が思ったことはバシバシ言うタイプです。で、たまに言い過ぎたかなって思うこともあるのですが。 
\\	は「行き過ぎる」、
\\	は「率直過ぎる」の意味。	
\\	日本も不景気だというので、今はアメリカで勉強しているのはいい選択だったのかもしれません。 
\\	で「賢い選択」の意味になります。
\\	も使えますね。	
\\	言いたい人には言わせておけ! 
\\	が「言いたい人」に当たる言い回しです。	
\\	酒に酔って言いたい放題のことを言ってたくせに、次の日には自分が何を言ったのか覚えてないなんて、あいつは最低だよ。 
\\	(わめき散らす)で「言いたい放題」のニュアンスを出しています。	
\\	いい食べっぷりだ。あいつ、さっきあんなに食べたのによく食べるな。しかも余裕で食べてるよ。 
\\	感嘆文
\\	!を使って「いい食べっぷり」を表現しています。
\\	は「衰える」の意。	
\\	このコース受けようって言い出したのは彼なのに、なんで言い出しっぺの彼が申し込んでないわけ? 
\\	で「…と言ったのは彼だ」の意味になります。	
\\	結局、言い出しっぺの人が来てないのだけど、どういうことだ? 
\\	は「最初に持ち出す」の意味ですね。	
\\	何かいい知恵がないものかね。 
\\	「いい知恵」は
\\	でいいでしょう。	
\\	彼って忙しい忙しいって言いつつ、超暇そうなんだけど。 
\\	「言いつつ」を
\\	「いつも…だと言っている」でそのニュアンスを表現しています。	
\\	とにかく、言いづらい事を打ち明けてくれてうれしいよ。 
\\	「言いづらい事」は
\\	と言えばいいでしょう。
\\	「(話を)切り出す」。	
\\	自分で片付けたくせに、私がどこかにやったに違いないなんて、勘違いもいいとこよね。 
\\	「勘違いもいいとこだ」をここでは
\\	「180度ずれている」で表現しています。
\\	「完全に間違っている」。	
\\	あいついいとこのお坊ちゃんだから、お金で苦労したことがないんだよ。 
\\	と
\\	「裕福で」、
\\	「周りから守られ」、
\\	「甘やかされて」を組み合わせて「いいとこのお坊ちゃん」のニュアンスを出しています。	
\\	彼のいいところと言えば、まじめでよく働くことかな。 
\\	「いいところ」を 
\\	で表現しています。	
\\	(電車の中で)ちょっと寄り掛からないでよ。酒くさいし。まったくいい年してちゃんとしてほしいよ。(独り言) 
\\	(あなたの年令なら)で「いい年して」のニュアンスを表現しています。	
\\	あの人、彼女から、いい友達でいましょうって言われたんですって。 
\\	「いい友達でいましょう」は
\\	でいいでしょう。ここでは
\\	追っ払う、鼻であしらう」も一緒に覚えておきましょう。	
\\	彼はいつも仕事で海外に行けていいなあ。 
\\	この場合の「いいな」は
\\	(仮定法)で表現すればいいでしょう。	
\\	僕は家内の言いなりになっている。 
\\	で「…を厳しく拘束する」の意味。
\\	は「(人の)言いなりになる」の意。	
\\	ちょっと言いにくい話なんだけど。 
\\	「言わなくてすめばいいんだけど... 
\\	と仮定法を使っているところがポイントですね。	
\\	花子さんって意外にいい人だったんだねえ。 
\\	「いい人」は
\\	と言えばいいでしょう。
\\	「驚いたことに」。	
\\	あっ、食べようと思ったメロンが腐ってる!いいもん、イチゴ食べるから。 
\\	この場合の「いいもん」は 
\\	で表現すればいいでしょう。	
\\	あいつ、俺のことを無用の長物だと言いやがった。 
\\	は「しりぞける、捨て去る」、
\\	「あざける」で「言いやがる」のニュアンスを出しています。	
\\	と「温泉、鉱泉水」、
\\	は「抜群、抜きんでている」の意。	
\\	気にしていても仕方ないから、良いように(前向きに)考えるようにした。 
\\	の文字通りの意味は「それを良いこととして考える」です。
\\	は「…について前向きに考える」の意。	
\\	彼は結局いいように使われただけだった。 
\\	で「好きなように 
\\	を使う」という意味になります。
\\	「操り人形」。	
\\	あなたはいいわよね。そうやって何でも言いたいこと思う存分言えるから。 
\\	この場合の「いいわね」は 
\\	を使えばいいでしょう。	
\\	良い悪いは別にして宗教は人々に多大な影響を及ぼしている。 
\\	で「それをきみが良いと思おうと、悪いと思おうと」の意味になります。
\\	でも同じようなニュアンスが出せます。	
\\	忙しいってことは商売が繁盛しているってことで、まあある意味いいんじゃない? 
\\	ここでは「いいんじゃない」を 
\\	(感謝すべきなんじゃない)で表現しています。	
\\	彼女顔に似合わず、言うことがきついよね。 
\\	は「きつい話し方をする」、
\\	「毒舌だ」の意味。	
\\	先日老体にむちをうって、炎天下の元テニスをしたので、今日は体が言うこと聞きません。 
\\	で「体が言うことを聞かない」を表現しています。
\\	でも同じように意味になります。	
\\	このロボット何でも私の言うことを聞くんだ。疲れて、何にもしたくない時は、助かるね。 なんて、ありえない話だよね。 
\\	で「私の言うことは何でも聞く(する)」の意味。	
\\	それを運命って言うんだって彼が言ったんだけど、言うねー、って感じじゃない? 
\\	ここでは「言うねー」を、
\\	「どうしてあんなことが言えるんだろう」で表現しています。	
\\	仕事をしているというよりかは、ただ会社にいるだけって感じ。 
\\	は
\\	の代わりに」、
\\	は
\\	というより」の意味。
\\	「備え付けの備品、居座っている人」、
\\	って感じ」。	
\\	もし生まれ変われたら、金持ちの家の子になりたい。それで何不自由なく暮らしたい。 
\\	で「お金持ちの子」の意味になります。
\\	「生まれ変わる」。
\\	は
\\	の意味	
\\	その行事があると知ってて行ったわけではなくて、たまたま行った日にその行事が行われていた。ま、ラッキーと言えば、ラッキーだったかな? 
\\	と言えば」のニュアンスを
\\	と言えないことはない」で表現しています。
\\	「ついていたんだろうな」。	
\\	確かに僕は時間にルーズだ。でもそれはきみにも言えることだろ。 
\\	で「同じことがきみにも言える」の意味。
\\	は「時間を守る」の意。	
\\	これまでのしがらみを一切無視して改革を断行するのはいかがなものか。 
\\	ここでは「いかがなものか」を 
\\	と 
\\	を組み合わせて表現しています。
\\	を使ってもいいでしょう。	
\\	この時期に議員がそのような発言をするのはいかがなものかという批判が相次いだ。 
\\	「いかがなものか」を 
\\	(疑わしい、問題のある)で表現しています。
\\	は「国会」の意。	
\\	卒業後、自分の英語力を活かせる職業に就きたい。 
\\	で「自分の英語力を最大限に活かす」という意味。
\\	は「…を最大限に利用する」の意。	
\\	彼がアメリカで就職する前に学校に行く事を望んでいるので、なんとか行かせてあげないと、と思っている。 
\\	「行かせたい」を
\\	「行かせる(送り出す)べきだ」で表現しています。	
\\	いかにストレスを発散するかって事が大切だね。この世の中生きていくには。 
\\	「如何に」は
\\	を使えばいいでしょう。ここでは
\\	と
\\	にも注目。	
\\	あーあ、きみが日本語理解できたらなぁ、いかに僕の日本語がすばらしいかわかってもらえるのに。 
\\	この場合の「いかに」は 
\\	で表現すればいいでしょう。
\\	は「…しさえすれば」の意。	
\\	彼はいかにも「営業」って感じだね。 
\\	「いかにも営業」、
\\	「徹底的に」の意。	
\\	こんな格好をしているといかにもスポーツしてそうに見えるけど、全然やってないんだよ。 
\\	の
\\	の後に
\\	「すっかり、ぴったり」を使って「いかにも運動をしている人のように見える」という意味を表しています。	
\\	皆さん方の準備の成果を、本日遺憾無く発揮して下さい。 
\\	「遺憾なく」とは「十分に」ということですから、
\\	「遠慮せずに全力を出す」と言えばその意味になりますね。	
\\	彼がコンピューターに詳しいとは意外だった。 
\\	「意外」とは 
\\	(考えてもみない)ということですね。	
\\	この間たまたまあの店の厨房が汚いのを見てしまったけど、意外とどこの店も見えない個所はそうなのかもしれないね。 
\\	「意外とそうかも」をここでは
\\	「(どこでも)共通して... のような気がする」で言い換えています。
\\	「汚れた台所用品」。	
\\	2、3年ぶりにピアスしたけど、まだちゃんと穴が開いていたよ。もう絶対にふさがったと思っていたけど、意外にふさがらないもんだ。 
\\	「意外に」は
\\	(…して驚く)、
\\	(驚いたことに)で表現すればいいでしょう。	
\\	なんか喉がいがらっぽいな。 
\\	は「ひっかかる、チクチクする」、
\\	は「荒れた」の意味。	
\\	仕事が見つかって生活が安定するまで帰ってこない!くらいの勢いで行かないと途中で弱音はきそう。 
\\	ここでは「勢い」を
\\	「活気、元気」、
\\	「やる気」を使って表現しています。
\\	「奮い起こす」。	
\\	今や国境を越えて資本や人材が行き交う時代だ。 
\\	で「どんどん国境を越える」の意味になります。
\\	は「浸透できる、穴の多い」の意。	
\\	行きは電車、帰りはバスで行ってきました。 
\\	「行き帰り」は
\\	ですね。	
\\	たぶん長野まで行き帰り車で運転したから、疲れているんじゃない? 運転ってけっこう、神経使うし。 
\\	「行き帰り」を 
\\	で表現しています。
\\	は「…にこたえる」、
\\	は「消耗させる」の意。	
\\	やっぱり、クーラーっていいよね。一気に、生き返る~。 
\\	が「生き返る」に当たる言い回しです。
\\	に活気をよみがえらせる」。	
\\	私の顔に誰かの息がかかって、気持ち悪いわ。 
\\	は「息をする」の意味です。
\\	「嫌な」、
\\	「不快な」。	
\\	行きがかり上、僕が担当者にされてしまった。 
\\	は「状況からして」の意味になります。	
\\	このスピーチ、早口言葉みたいで息が続かないよ。 
\\	は「息が切れる」、
\\	は「息が詰まる」の意味。	
\\	一日中オフィスで机に向かっていると息が詰まりそうになる。 
\\	で「息苦しくなる」の意味。
\\	「息が詰まるような」。	
\\	初めから意気込みすぎると、途中で息切れするぞ。 
\\	で「途中で息が詰まる」の意。
\\	「自分のペースでやる」。	
\\	あんな裕福な生活ができても、自由がきかないし、きっと息苦しくて逃げ出すだろう。 
\\	が「息苦しい」に当たる形容詞です。
\\	「裕福」。	
\\	せっかく、英語で注文しようと意気込んで入ったのに、日本語で「いらっしゃいませ」って言われた。あれ?って感じ。 
\\	「意気込む」をここでは
\\	「やる気になる」、
\\	「(心の)準備を整える」を使って表現しています。	
\\	近所に行きつけの店があるから、今日はそこでランチを食べよう。 
\\	「行きつけの店」は
\\	「店の常連」で言い換えて表現すればいいでしょう。	
\\	仕事で行き詰まったら、それ以上考えても無駄なので気分転換するようにしている。 
\\	で「行き詰まる、身動きがとれなくなる」の意味を表します。
\\	「違った視点から物事を見る」。	
\\	あなたなしでは生きていけない。 
\\	で「…なしでは生きられない」の意味。	
\\	最初2人はぎこちなかったけど、話して行くうちに2人は意気投合し た。 
\\	「意気投合する」は
\\	で表現できます。
\\	「そりが合う」。〔注)投稿文を一部書き換えました。	
\\	この航空会社、行きと帰りの食事がえらく違うね。 
\\	「行きと帰り」が英語では 
\\	となっている点に注意。
\\	は「外国行きの」の意。	
\\	今日の格好はとっても粋ですね。 
\\	には「粋な、スマートな」の意味があります。
\\	「粋な服装」。	
\\	彼女は結婚式の途中で、酔っぱらっていきなり歌い出した。 
\\	「いきなり」は
\\	「突然に」を使えばいいでしょう。	
\\	たまには旅行でもして息抜きしないと、毎日仕事仕事じゃやってられないよ。 
\\	「息抜き」は
\\	「(仕事を)中断する、休憩する」で表現すればいいでしょう。	
\\	息を吸って、止めて下さい。 
\\	「息を吸う」は
\\	、あるいは、
\\	で表現します。
\\	「抑える」。	
\\	行くあてがどこかにあるわけじゃないけど、ここにいる訳にもいかないんだ。 
\\	と 
\\	を組み合わせて「行くあて」を表現しています。
\\	は「目的地」の意。	
\\	たまに、意気地なしの自分に気がつく。ただ、強がっているだけなのかもね。 
\\	は「意志の弱い、優柔不断の」という意味。
\\	は「いくじのない」の意。	
\\	君の息子さん、今度いくつになるの? 
\\	と
\\	と
\\	の組み合わせがポイントです。	
\\	最近、いくら寝ても眠い。なんか、変な病気にでもかかったのかな。 
\\	「いくら...しても」は
\\	を使って表現できます。	
\\	いくら私を愛してるって言っても、彼女のことも好きなことには変わりはないでしょ。 
\\	この場合の「いくら」は
\\	で表現できます。	
\\	私がいくら言っても、聞こうとしないのよ。 
\\	で「何度 
\\	しても」の意味になります。	
\\	いくら強くアクセル踏んでも全然スピードでないと思ったら、サイドブレーキあがってた。 
\\	この場合の「いくら」は
\\	を使って表現できます。後半の
\\	の使い方にも注目してください。	
\\	海外から物品を輸入したとき、その価値がいくら以上だと課税されるの? 
\\	ここでは「いくら以上」を 
\\	(最小限の額)、
\\	を使って表現しています。	
\\	いくら好きとは言え、3日連続カレーはきついよな。 
\\	「きつい」をここでは
\\	「十分過ぎる」を使って表現しています。	
\\	試験の問題なんぞ、難しくしようと思えばいくらでも難しくできる。 
\\	の使い方がポイントです。	
\\	いくらなんでもそういう態度は相手に対して失礼だろう。 
\\	(きみが何を言おうと)で「いくらなんでも」を表現しています。
\\	は「そう(失礼な態度だと)言われても仕方がない」の意味。	
\\	彼女って本当にイケてないわ。 
\\	「いけてない」は 
\\	で表現すればいいでしょう。	
\\	結構いける口なんでしょう。 
\\	で「お酒が強い、飲んでも乱れない」という意味になります。	
\\	彼女とだいたい同じ意見です。 
\\	で
\\	におおむね賛成です」、
\\	で「だいたい同じ意見です」の意味になります。	
\\	働き始めて最初の3年は毎日のように辞めたい辞めたいって思っていたけど、さすがに10年も経つと居心地がよくなるみたい。 
\\	「居心地がいい」を 
\\	で表現しています。	
\\	安くて、うまくて、居心地がいい店が繁盛する。 
\\	「居心地がいい」を 
\\	(いい感じがする)、
\\	(その中でくつろげる)で表現しています。
\\	は「繁盛している」の意。	
\\	いざという時役に立つ。 
\\	は「ピンチの時」、
\\	「その時が来たら」の意味。
\\	「役に立つ、使いやすい」。	
\\	退職まであと、一カ月だ。早く辞めたい辞めたいと思ってたけど、いざその日が近づいてくるとなんかちょっと寂しい気もする。 
\\	ここでは「いざ」のニュアンスを 
\\	を使って表現しています。	
\\	以前テレビで見て、ホット赤ワインを一度飲んでみたいと思っていたのですが、居酒屋ではなくもっと本格的なお店で飲んでみたいです。 
\\	「居酒屋」は
\\	でいいでしょう。
\\	は「普通の」、
\\	は「本物の」の意。	
\\	あなたから離れようと思っているのにいざそうしようとしてもうまくいかない。 
\\	で「たとえやってみてもできない(仮定法)」の意味を表します。
\\	「別れる、離れる」。	
\\	いざという時にあわててしまうから、僕はだめなんだ。 
\\	で「いよいよという時」の意味。
\\	「最後の一瞬になる」。	
\\	私、いざ本番!となるといっつも緊張するんだよね。しかも、上がり症だしさ。 
\\	で「その場になると」の意味。
\\	は「息が詰まる、言葉が詰まる」、
\\	は「上がる、固くなる」の意。	
\\	私って石橋をたたいて渡るタイプなのよね。 
\\	「石橋をたたいて渡る」を
\\	「危険に近寄らない」で表現してみました。「タイプ」はそのまま
\\	が使えます。	
\\	私は、石橋を叩いて渡るタイプです。ちょっと慎重すぎるとは思うのですが、計画なしでは思い切ったことはできない。 
\\	「石橋を叩いて渡る」を 
\\	と 
\\	(注意深い)を組み合わせて表現しています。
\\	は「慎重な」、
\\	は「全力を尽くす」の意。	
\\	花火の途中でかなり強い雨が降ってきたけど、(花火は)上がりつづけていたよ。まさに、花火職人の意地だね。 
\\	ここでは「意地」を
\\	「職人としての責任、プライド」で表現しています。	
\\	いつまでも意地張ってないで、こっちに来なよ。 
\\	ここでは
\\	「意地っ張り、頭の固い奴」、あるいは、
\\	「気にしながらもピリピリしている」を使えばいいでしょう。	
\\	いじめは思春期だけじゃなく、 更年期でもあるよね。 
\\	「いじめ」は
\\	を使って表現します。「更年期」の表現の仕方にも注目。	
\\	彼は小学生のころ、いじめられっ子だった。 
\\	「いじめる」は 
\\	で表現できます。
\\	は「からかう」の意。	
\\	ニキビをいじるなって言われても思わずいじめたくなっちゃうの 
\\	ここでは「いじめる」を
\\	「…手を出す」、
\\	「…をつつく」を使って表現しています。	
\\	こうなった以上、じたばたしても仕方がない。 
\\	で「こういう状況なので」の意味になります。
\\	は「こうなったので」の意。	
\\	彼って意地悪だと思わない? 私がホラー物の映画が嫌いなのを知ってるくせにいつも見るんだよ。もう、いやんなっちゃう 
\\	は「ひどい、嫌な(奴)」、
\\	は「意地悪な、卑劣な」の意味。	
\\	神様って何て意地悪なんだろう!! 
\\	(残酷な)、
\\	([ひどく]苦しめる)で「意地悪」を表現しています。	
\\	ぎりぎりまでここのお店で粘って、居れるだけ居座って、それからコーヒー屋に行こう。 
\\	と 
\\	を組み合わせて「居られるだけ居座って」を表現しています。	
\\	その週のいずれかであれば、いつでもかまいません。 
\\	この場合に「いずれか」は
\\	で表現できますね。	
\\	以前から好きだったんだけど、最近ワインにはまっているんだよね。ポリフェノール3倍とか言う文句にひかれ、ついつい飲んでしまう。 
\\	と過去形
\\	を組み合わせて「以前からずっと好きだった」という意味を表しています。	
\\	彼はいつもどこかの家に居候している。 
\\	「家賃なしで」の使い方がポイント。	
\\	そういらいらするなよ。急がば回れって言うだろ。 
\\	ここでは
\\	(諺)を使わずに、
\\	「物事をさっさと片づける」と
\\	「結果的には、時間がもっと掛かることになる」で表現しています。	
\\	彼は依存心が強すぎるからだめなんだよ。 
\\	で「依存心が強すぎる」という意味になります。	
\\	あなたは世の中を甘く見すぎている。ちゃんとしないと、あとで自分が痛い目に遭うよ。 
\\	ここでは「痛い目に遭う」を 
\\	(ひどい目に遭う、火傷する)で表現しています。
\\	は「罰が当たる」の意。	
\\	あそこの映画館の椅子のクッションはいいと言われているけど、やっぱり2時間も座りっ放しだとおしりが痛くなった。 
\\	「痛くなる」は 
\\	でいいでしょう。動名詞の 
\\	を主語にして、動詞に 
\\	を使っている点に注意。	
\\	あの子は本当にいたずら好きだ。 
\\	の文字通りの意味は「いたずらでいっぱい」です。
\\	「いたずら好きな」。	
\\	こんな夜遅くにいたずら電話をかけてくるなよな。 
\\	「いたずら電話」には
\\	という決まった言い方があります。	
\\	(歯医者で)そんな痛そうな顔しないで。 痛くないようにするからさ。 
\\	「しりごみする、ひるむ」を使って「痛そうな顔(をするな)」を表現しています。	
\\	いただきま--す。 
\\	1番目の例文は、食事の前のお祈りです。2番目は、子供がふざけてお祈りをするときのことばです。「いただきます」をそのまま英語にすると
\\	になりますが、あまり言いませんね。個人的経験からすると、父親が、
\\	などと言って、食事を始める、ということがあるようです。	
\\	最近主婦業がやっと板についてきたって感じなのよねえ。 
\\	「板につく」とは
\\	が快適にできるようになる」ということですね。
\\	を管理する、操る」。	
\\	局長は、次官と大臣の板ばさみになって苦しんでいる。 
\\	(真ん中に挟まる)が「板挟み」に当たる言い回しです。
\\	の使い方にも注目。	
\\	その対決では、両陣営とも手ひどいダメージを負って、痛み分けに終わった。 
\\	と
\\	「引き分けに終わる」を使って「痛み分け」を表現しています。	
\\	年甲斐もなく張り切ってゴルフ頑張りすぎて、手首を痛めてしまったらしい。 
\\	で「手首を痛める」の意味になります。「年甲斐もなく」の表現の仕方にも注目。	
\\	そういう理由で現在に至ります。 
\\	が「現在に至る」に当たる表現です。	
\\	きのうの官官接待は至れり尽くせりだったよ。 
\\	「至れり尽くせり」は 
\\	(申し分がない)ということですね。(注)ここでは「官官接待」を「接待」として英訳しています。	
\\	いちいちうるさいなあ、そんなことどうでもいいじゃない。 
\\	(つまらないことをあら探しする[形容詞])、
\\	(こうるさい)を使って「いちいちうるさいな」を表現しています。	
\\	この新しい清涼飲料水は、この夏の一押しです。 
\\	「一押し」を
\\	「飲むには最高の清涼飲料水」と表現しています。	
\\	今日お気に入りの芸能人が出てるミュージカルを見に行った。彼は、今年の一押しだね。 
\\	「一押し」とは
\\	「是非見るべき」ということですね。
\\	「呼び物」。	
\\	良いか悪いか、そう一概には言えない。 
\\	「一概に」を 
\\	(断定的には)で表現しています。
\\	(一言で)の使い方にも注目。	
\\	あの店、一見さんでも入れるのかな。 
\\	「一見さん」とは
\\	「馴染みのない客」ということですね。	
\\	あの人と巡り会えたのは、一期一会なのでしょうか。 
\\	「一期一会」をここでは
\\	「一生に一度(の出会い)」で表現してみました。	
\\	一時期は、コーラにはまって、毎日のように飲んでたよ。 
\\	「一時期」は
\\	、あるいは、
\\	で表現できます。
\\	にはまっている」。	
\\	この間ミュージカルに見に行ったんだけど、前から3列目でさ、もう憧れの人が目の前で、もう最高だったね。一段と彼を好きになった。 
\\	「一段と」は
\\	を使えばいいのですが、ここでは
\\	「まいる、いかれる、魅了される」にも注目。	
\\	雲梯(うんてい)で一段飛ばしができるんだ。 
\\	「一段飛ばし」を 
\\	(バーを1つ飛ばす)で表現しています。
\\	は「雲梯」の意。	
\\	そのショーは一度だけでも見たほうがいいよ。 
\\	「一度だけでも」は 
\\	で表現すればいいでしょう。	
\\	一度に一つしか、だめよ。 
\\	で「一度にひとつずつやる」の意味。	
\\	一日一善。良いことをした後は、気持ちがいいね。 
\\	「一日一つは良いことをする」は
\\	と言います。	
\\	一生懸命勉強して一日も早く英語がしゃべれるようになりたい。 
\\	(人間の力で可能な限り早く)で「一日も早く」を表現しています。	
\\	どんな仕事でも一人前になるには長い年月がかかるものです。 
\\	「一人前になる」を 
\\	(自己を確立する)で表現しています。	
\\	アメリカではいち早く映画が公開されるので、映画ファンにはたまらないね。 
\\	(すぐに公開される)で「いち早く」を表現しています。「たまらない」の表現の仕方にも注目。	
\\	一番おいしいのはどれ? 
\\	「一番おいしい」は 
\\	と 
\\	を組み合わせれば表現できます。	
\\	この露天風呂は海が一望できるのでいいね。 
\\	で「海を広く見渡せる展望」の意味になります。	
\\	皆が彼には一目置いている。 
\\	で「敬意の念をもって扱う」の意味。
\\	「優位性、優れていること」。	
\\	彼らが皆の前でいちゃいちゃするので、目のやり場に困った。 
\\	「いちゃつく」は 
\\	で表現できます。	
\\	例のお客さん、またいちゃもんつけてきたわよ。 
\\	「いちゃもん」は
\\	「不平・不満を言う」を使えばいいでしょう。
\\	は「また文句を言ってる」の意味を表します。	
\\	来週から中間テストだ。一夜漬けじゃ、太刀打ちできない問題ばかりだぞ。今日から始めろよ。 
\\	「一夜漬けをする」は
\\	という言い方があります。
\\	は「詰め込む」の意。	
\\	ビザの一連の手続きをお願いしている方がこの学校を薦めてくれました。 
\\	ここでは「一連」を
\\	「過程」、
\\	「段階」を使って表現しています。	
\\	私の授業は、一話完結ですので、先週休んだ方も心配いりません。 
\\	で「一つでまとまっている」の意味。	
\\	一を聞いて十を知る。 
\\	第一文の意味は「一言で十分だった。それですべてが分かった」。	
\\	これから1回の発言を2分以内としますので、ご協力よろしくお願いします。 
\\	「1回の発言」を 
\\	で表現しています。
\\	(感謝する)、
\\	(協力)の使い方にも注目。	
\\	そのダイヤのネックレスは大変高価だったが、思い切って一回払いで払うことにした。 
\\	で「いっぺんで…を払う」の意味になります。
\\	は「一回払い」の意。	
\\	我が国の雇用形態は、4月の大卒一括採用である。 
\\	「雇う」と
\\	「募集する」、
\\	「ひとまとめにして」を組み合わせて「一括採用」を表現しています。	
\\	お代金のお支払いは一括でよろしいですか。 
\\	と
\\	で「一度に払う」の意味になります。
\\	は「分割払いで」の意。	
\\	ここの工場は、部品から製品までを一貫生産している。 
\\	日本語の意味を表すように忠実に英語に訳してみました。
\\	「統合した」、
\\	「規格化する」の意。	
\\	こういう仕事はいっきにやた方がいいんだよ。 
\\	「一気に」は
\\	「一辺に」、
\\	「一度で」で表現できます。	
\\	あの人の一挙手一投足が注目の的です。 
\\	「一挙手一投足」とは「あらゆる行動」のことですから、
\\	で表現すればいいでしょう。
\\	は「注意を引きつける」の意。	
\\	政府はその難局を乗り切るために一計を案じた。 
\\	には「作り上げる」という意味があります。
\\	は「苦労して考え出す」の意。	
\\	彼女は一見おとなしそうだけど、やる時はやるって感じ。中に何か秘めた思いを持っているよ。 
\\	で「一目見て」の意味。
\\	「おとなしい」。	
\\	我々の拳法の奥義は一子相伝である。 
\\	が「一子相伝」に当たる言い回しです。	
\\	今日は随分しゃべるのね。いつも無口なのに。1週間分はしゃべったよ。 
\\	「~分」は
\\	「値する」で表現できます。
\\	「長々と話すこと」、
\\	「おしゃべり」。	
\\	何があってもずっと一緒にいようね。 
\\	「一緒にいる」は 
\\	、あるいは、
\\	で表現すればいいでしょう。
\\	は「何が起きても」の意。	
\\	事故は一瞬、保証は一生。やはり、運転するんだったら、歩いた方がいい。 
\\	「一生」は
\\	を使えばいいでしょう。
\\	「負債、責務」。	
\\	彼女はこの曲のヒットによって一生食べて行ける。 
\\	「一生」は
\\	でいいでしょう。	
\\	この失敗、きっと一生、言われつづけるなあ。 
\\	とは「後から何とかして償う、乗り越える」の意味になります。
\\	「ついて離れない、ついて回る」。
\\	「大失敗」。	
\\	今の仕事は体力的にきつくなってきたので、何か一生やっていける仕事を見つけたい。 
\\	「一生涯」は
\\	で表現します。	
\\	どうしてそんなに人のために自分の時間をさいて一生懸命になれるの?一文の得にもならないじゃん。 
\\	「一生懸命」は 
\\	で表現すればいいでしょう。「一文の得にもならない」の言い方にも注目。	
\\	一生のお願いだから、何も言わずに5万円貸してくれない? 
\\	「一生のお願い」のニュアンスを
\\	(生きている間ではこれが最後のお願い)で表現しています。	
\\	はりきってフランス語のクラスを取り始めたのに、ついていけなくて途中でやめてしまって、一生の不覚です。 
\\	「一生の不覚」を 
\\	(ずっと後悔することになる失敗)で表現しています。	
\\	今、日本はワールドカップ一色です。 
\\	ここでは「一色」を 
\\	(見る物すべて)で表現しています。	
\\	仕事より家族と一緒にいる時間を大切にしたいと思うし、そうすべきだと思う。 
\\	「家族と一緒にいる」は
\\	でいいでしょう。	
\\	(あなたと)一緒にしないでくれる? 
\\	で「私を含めないでくれる」の意味になります。	
\\	週末いきなり停電になって、最初ブレーカーが落ちたのかと思ったけど、その一帯が停電だったみたい。 
\\	が「一帯(にわたって)」に当たる表現です。
\\	は「停電」の意。	
\\	彼とはいずれ一対一の話し合いをしなければならない。 
\\	「一対一で(の)」は副詞、形容詞として使えます。	
\\	もし、アメリカで職に就けなくても、せめて英語だけは習得して日本に帰れば、行った意味あるし後悔もしないと思う。 
\\	「行った意味がある」を 
\\	と
\\	を組み合わせて表現しています。
\\	は「価値がある」の意。(注)「取得」を「習得」に変更しました。	
\\	自分から良いって言ったくせに、後になってやぱっりだめって言うなよ。 
\\	で
\\	だと言ったのは君だろう」の意味になります。	
\\	今後1年間、アメリカと日本を行ったり来たりするつもりです。 
\\	「行ったり来たり」は 
\\	と語順が逆になります。	
\\	言っていいことと悪いことがあるだろう。彼女に謝れよ。 
\\	と
\\	の使い方がポイントですね。	
\\	あなたが言っていたように、あの噂は本当だった。 
\\	で「まさにあなたが言っていたように」の意味になります。	
\\	お母さん、行ってきます。今日は帰りが遅くなると思うよ。 
\\	で「出かける」の意味にまります。簡単に
\\	と言ってもいいですね。	
\\	え!スープ濃かった?なんだ、言ってくれたら薄めたのに。 
\\	で「言ってくれていたなら(仮定法過去完了)」の意味になります。	
\\	酔って英語を話すと効果があるってテレビで言ってた。 
\\	「言ってた」は
\\	(2番目の英訳)でもいいのですが、
\\	も使えます。	
\\	あんなところに行ってまで映画を見たいとは思わない。 
\\	で「わざわざそこまで行く」の意味になります。	
\\	気をつけてね!いってらっしゃい! 
\\	「いってらっしゃい」は 
\\	でいいでしょう。
\\	の代わりに 
\\	も使えます。状況によっては 
\\	とも言えます。	
\\	あいつは、狭い部屋に住んでるくせに、一点豪華主義で、AV機器だけには金を惜しまない。 
\\	「1つの物にすべてのお金を使う」で「一点豪華主義」を表現しています。
\\	は「一方に偏った」、
\\	は「優先権」の意。	
\\	今は、来月の海外旅行のことでもう頭がいっぱいよ。だって、久々なんだもん。 
\\	この場合の「(頭が)いっぱい」は
\\	「考えられることは...だけ」を用いて表現すればいいでしょう。	
\\	今後のことを考えると不安でいっぱいで爆発しそうになる。英語は上達するだろうか、子どもが生まれたらちゃんと育てていけるかとか。 
\\	で「将来のことで不安でいっぱいになる」の意味。
\\	は「破裂、爆発する」の意。	
\\	最近残業してないし、旅行に行くために食費をけずりながらいっぱいいっぱいの生活をしている。 
\\	で「ぎりぎりでやっていける物だけで生活する」の意味になります。	
\\	しまった!彼女に一杯食わされた! 
\\	([人を]だます」で「一杯食わされる」を表現しています。	
\\	今夜あたり会社帰りに一杯やりましょうか。 
\\	「一杯やる」はそのまま 
\\	でいいでしょいう。
\\	を付け足しても構いません。	
\\	その歌手は、一曲しかヒットを飛ばさず、結局一発屋で終わってしまった。 
\\	が「一発屋」に当たる言い回しです。	
\\	わが国の決済方法として一般的な慣習は、手形決済である。 
\\	「一般的」は 
\\	で表現すればいいでしょう。
\\	は「手形で支払う」の意。	
\\	私が選挙で一票入れたところで、何も変わらないと思ってしまう。 
\\	は名詞で「投票(一票)」、動詞で「「投票する」の意味があります。	
\\	本屋で英語関係の本を見つけると、次々に買ってしまって、読まずにいるので、たまる一方です。 
\\	の
\\	で「(たまる)だけ、一方」というニュアンスを出しています。	
\\	彼女は聞かれた事に頷くだけで、会話が一方通行なんだよなぁ。 
\\	は形容詞で「一方通行の」という意味。
\\	をつけて
\\	で副詞としても使えます。	
\\	最近は、一本早い電車で来ているんだ。 
\\	「一本早い電車」は
\\	と言います。	
\\	この間言ってた写真だけど、送ってくれるの、いつでもいいからね。 
\\	で「いつでもいい」の意味になります。
\\	(急がない)を使ってもいいでしょう。	
\\	どうしよういつのまにか汚れてる。この汚れちょっと目立つね。 
\\	「いつの間にか」を
\\	「見ていない間に」で表現しています。	
\\	いつまで経っても散らかしっ放しで、いつになったら片付けるの? 
\\	「いつまでも」を
\\	(無期限に)、
\\	(永遠に)で表現しています。
\\	は「放り出す」の意。	
\\	何時までに仕上げればいいですか? 
\\	「何時まで」は 
\\	と 
\\	を組み合わせて表現します。	
\\	予約をあなたにお願いするにはいつまでに言えば良いですか? 
\\	「いつまで」はそのまま
\\	で表現できます。
\\	も面白い言い回しですね。	
\\	私ってちょっと考え過ぎなのかな。自分の話す英語が合っているかどうかいつも気になる。 
\\	「いつも気になる」は 
\\	で表現できます。動詞は 
\\	を使ってもいいでしょう。	
\\	昨日飲んで遅く帰ったのに、よく今朝いつも通り起きれたと思うよ。 
\\	「いつも通り」は
\\	で表現すればいいでしょう。	
\\	いつもの定食お願いします。 
\\	「いつもの」がそのまま使えます。
\\	「いつも食べているもの」と名詞としても使えます。	
\\	仕事が終わったら、いつもの店で一杯やって行かないか。 
\\	には「よく出入りする場所、たまり場」という意味があります。	
\\	あのドラマの中の役のように、彼女は年を取らないで小さいままでいて欲しかったなあ。だってイメージっていうものがあるし。 
\\	「いて欲しい」は
\\	と
\\	を組み合わせて表現します。	
\\	彼らの
\\	の発売が待ちどおしくて、居ても立ってもいられない。だって、あの幻のバンドのだよ。 
\\	は「じっと座っていられない」、
\\	は「我慢できない」、
\\	は「伝説の」の意。	
\\	私の視力の悪さは遺伝なの。 
\\	は「遺伝の」、
\\	は「受ける、遺伝する」の意味。	
\\	遺伝子組み換え食品の隠された危険性について、もっと知る必要がある。 
\\	「遺伝子組み替え食品」は
\\	といいます。	
\\	今夜はそろそろおいとまします。みなさんよい週末を。 
\\	「おいとまする」には色々な言い方がありますが、ここでは
\\	と
\\	を使ってみました。	
\\	五木ひろしは糸目の男。 
\\	には「細長い切れ目」という意味がありますので、これを使って「糸目」を表現してみました。	
\\	なんて可愛い子なの。いない、いない、ばー。 
\\	(「ピーカブー」と発音)が「いない、いない、ばー」に当たる表現。この後に
\\	「見ーつけた、見えますよ」を続けて言うことがよくあります。	
\\	今時めずらしく都会ずれしてなく田舎っぽい人だね。新鮮でいいかも。 
\\	で「都会ずれしていなくて田舎っぽい人」を表現しています。
\\	「生き生きとした、すれていない」。	
\\	彼は行動も言動も田舎もんそのものだよ。 
\\	「首の後ろが日に焼けている奴」が「田舎もん」の意味で使われることがよくあります。
\\	「田舎者」。	
\\	鯔背(いなせ)だね。 
\\	「いなせ」は
\\	「かっこいい、粋」を使えばいいでしょう。
\\	とは「粋な着こなしをする人」の意味です。
\\	「彼はいつもいなせな格好をしている」という言い方もあります。	
\\	映画の撮影と言えども、本当の炎の中に入っての撮影はさぞかし大変で命がけだったと思う。 
\\	ここでは「命がけ」を 
\\	(生きるか死ぬか)で表現しています。
\\	は「命をかける」の意。	
\\	お前、スピード違反で捕まったんだって? 何キロオーバーだったの? 
\\	で「スピード違反で捕まる」の意味。	
\\	一年先輩だからって、そんなに威張らなくていいじゃない。 
\\	には「ふんぞり返る、威張り散らす」という意味があります。
\\	「一年先輩」も一緒に覚えておくといいでしょう。	
\\	最近の一連の異物混入騒動には うんざりしている。 
\\	が「異物混入」に当たる表現です。	
\\	ここの通りは違法駐車の車だらけだな。警察を何をしてるんだよ。 
\\	とは「違法駐車をしている車」のことです。	
\\	ん~、イマイチなんだよねえ。この脚のあたりの線とかさぁ。なんとかなんないの? 
\\	とは「素晴らしいと言えるまでにはなっていない」、
\\	は「ちょっと的がはずれている」の意。	
\\	今思うと、平気で昆虫が触れたなんて昔はたくましかったな。今は見るだけでもだめ。 
\\	「今思うと」は
\\	を振り返ると」で表現できます。	
\\	最近の暑さにやられたみたいで、だるくて何もする気がしません。今からこんな状態では8月が思いやられる。 
\\	で「今からこんな状態」を表現しています。
\\	は「無気力な、眠気がさす」の意。	
\\	7時に仕事が終わって来たので、少し遅れましたが今からでも授業受けられますか? 
\\	この場合の「今からでも」は 
\\	と 
\\	を組み合わせて表現すればいいでしょう。	
\\	今頃言うなんて遅いよ。事が起こた時に言わないと。 
\\	で「そんなことで今頃僕のところに来るなよ」の意味になります。	
\\	先月の米国出張の疲れが、今ごろになって出てきた。 
\\	「今頃になって出てきた」のニュアンスを 
\\	で表現しています。	
\\	ここまで計画したのに、今更やめるって言わないでよ。 
\\	は「やめる
\\	は「諦める」の意味ですね。	
\\	今更だけど聞いていい?日本にどれぐらい住んでいるのですか?特に深い意味はないけど、ふと気になったもので。 
\\	「今更だけど」のニュアンスを
\\	(これまでどうして…を聞かなかったのかわからないけど)で表現しています。	
\\	カナダは空き巣が多いと聞きましたが本当ですか?入られないほうが不思議なくらいって。今更だけど引っ越し前にして不安になってきた。 
\\	「今更」のニュアンスを 
\\	で表現しています。	
\\	今でも思い出すとあれは最高に嬉しい出来事だった。 
\\	「今でも」は
\\	で表現します。
\\	は「際立つ」の意。	
\\	今となってはもうそういう問題じゃない。 
\\	ここでは「今となっては」を 
\\	(この時点では)で表現しています。
\\	は「ここまでくる」の意。	
\\	以前から興味があったのでそのクラスを受けてみたけど、今となってはほかのにすれば良かったと後悔してる。 
\\	「今となっては」のニュアンスを 
\\	で表しています。
\\	のあとには 
\\	が略されています。	
\\	髪のカット料、
\\	000だったんだよ。今時珍しいですよね。無免許だったりして。 
\\	「今時珍しい」は
\\	で簡単に表現できます。
\\	「出くわす、見つける」。	
\\	僕が家に帰ったとき、うちの弟が今にも泣き出しそうな顔で「腹減った」って言ったんだ。 
\\	ここでは「今にも」を 
\\	を用いて表現しています。
\\	は「わっと泣き出す」の意。	
\\	今日って天気どう?悪くなるの?今のうちに、洗濯しておこうっと。 
\\	ここでは「今のうちに」を 
\\	(できる間に)で表現しています。
\\	は「…を片づける」の意。	
\\	今の仕事には、創造力が必要とされている。 
\\	ここでは「今の仕事」を
\\	「今している仕事」、
\\	「現在の仕事」と言い換えています。	
\\	こういう考え方って昔の人の考えだと思う。今の世代の人はこうは思わないと思うけど。 
\\	「今の世代」は
\\	、あるいは、
\\	で表現できます。
\\	「考え方」。	
\\	最近うちの会社新しいカメラ発売したんだけど、今までとはちょっと違ったデザインなんだ。 
\\	「今までとは違ったデザイン」は完了形を使って
\\	のよう表現すればいいでしょう。
\\	は「市場に出す」の意。ちなみなに、アメリカでは
\\	の代わりに
\\	を使うことがよくあります。	
\\	あなたの部の人で、今までに外国の支店に転勤した人はいますか? 
\\	「今までに」は 
\\	で表現すればいいでしょう。現在完了だけでも「今までに」の意味は表せます。	
\\	お正月のおせちというと、意味合い的にはアメリカでいうターキーみたいな感じなのかな? 
\\	は「同じような含みを持つ」、
\\	は「いくぶん同じニュアンスになる」の意味を表します。「おせち」の表現の仕方にも注目。	
\\	彼女に意味ありげに見つめられて、どきどきした。 
\\	には「意味ありげな、こびるような」という意味があります。
\\	は「意味ありげな」の意。	
\\	北海道に行ったら、やっぱり海の幸を食べないと行った意味がないでしょう。 
\\	ここでは「行った意味がない」を
\\	で「北海道へ行ったことにならない」と言い換えて表現しています。	
\\	週末の勉強だけじゃ、全然進歩しないよ。やっぱり、毎日やらないと意味がない。 
\\	「意味がない」は
\\	、あるいは、
\\	「時間の無駄」を使えばいいでしょう。	
\\	あのドラマ最後にいつも意味深な言葉を言って終わるよね。 
\\	「意味深」は
\\	「意味の深い、奥の深い」を使えば表現できますね。	
\\	せっかく月謝袋を忘れないようにって持ってきたのに、お金が入ってないんじゃ意味ないじゃん。 
\\	で「まったく意味がない」の意味になります。
\\	「役に立たない」を使ってもいいでしょう。	
\\	来年アメリカに移民するって言うと、皆、私が日本人ではなくなるのではと心配するんです。特に、祖母や祖父が。 
\\	で「…に移民する」という意味。接頭辞 
\\	-、接尾辞-
\\	の使い方に注目。	
\\	彼の行動や態度は嫌がらせとしか思えない。むかつく! 
\\	は「嫌がらせをする」、
\\	は「嫌がらせ」の意味です。	
\\	彼女すごく嫌がってたじゃない。もうやめようよ。 
\\	ここでは「嫌がる」を 
\\	(困っている、腹を立てている)で表現しています。
\\	と 
\\	の組み合わせにも注目。	
\\	丸田さんたちの騒がしさにはほんとうに嫌気がさした。 
\\	に嫌気がさす」は
\\	また、
\\	できます。(「うんざりする」参照)。
\\	は「(うるさい)人たち、一団」の意。	
\\	ここにいるといやでも感覚は研ぎ澄まされる。 
\\	「いやでも 
\\	する」は
\\	せざるをえない」を使えばいいでしょう。	
\\	他人に迷惑をかけたり、嫌な思いをさせたりするようなことは絶対したくない。 
\\	で「(人に)嫌な思いをさせる」という意味になります。	
\\	ねねね、そんないやな顔しないでよ。 
\\	「いやな顔をする」は
\\	の他に、
\\	「(話を聞いて)うんざりしたような顔をする」でも表現できます。	
\\	ねねね、何か隠しているでしょ!教えてよ。教えてくれないなんて、いやな感じ! 
\\	の本来の意味は「ぞっとするような」ですが、ここでは「嫌な感じを与える」といいう意味で使われています。
\\	「秘密主義の、隠し立てする」。	
\\	今日は嫌なことがあったから、ビールのんでさっさと寝ることにしょうーっと。 
\\	「嫌なこと」とは
\\	「ひどい経験」ということですね。
\\	「つらい、大変な」。	
\\	この計画うまく行かないんじゃないかと、初めから嫌な予感がしてたんだ。 
\\	「嫌な予感がする」は
\\	で表現できます。	
\\	昨日占いでさそり座は今日は運勢が悪いって、特に風邪に注意っていうからなんか嫌な予感がしてたら、本当に風邪ひいちゃった。 
\\	が「…について嫌な予感がする」に当たる言い回しです。
\\	は「予感」の意。	
\\	早く帰りたいときに日限って、電車が事故とかで遅れてたりするんだよね。嫌になるぜ。 
\\	で「…にうんざりする」の意味。「嫌になるぜ」のニュアンスは
\\	(こんなの信じられないよ)でも出せますね。	
\\	バイト先のお局から嫌味を言われる。 
\\	で「しつこく困らせる」、
\\	(悩ます)で「嫌味を言う」を表現しています。「バイト先のお局」を 
\\	(上司)に変えて英訳しています。	
\\	嫌みのつもりで、彼の会話力を誉めたのに真に受けたようだ。 
\\	は「皮肉な」、
\\	は「嫌みな」の意味。
\\	は「額面通りに受け取る」の意。	
\\	(あいつら)いよいよおいでなすった。 
\\	この場合、
\\	がよく使われます。また、
\\	!と言ってもいいでしょう。
\\	となっているのは「ヒーーーーア」と発音を伸ばすことを表すためです。	
\\	この間、居酒屋でホット赤ワインを飲んで以来、はまっています。 
\\	「以来」は
\\	で表現します。
\\	は「…にはまる」、
\\	は「日本風の」の意。	
\\	最近、彼のちょっとした言葉にイライラする。 
\\	には「いらだたせる、怒らせる」という意味があります。「ちょっとしたこと」を
\\	で表現している点にも注意。	
\\	親が仕事で忙しいのをいいことに子供が友達の家に入り浸ってしまっている。 
\\	ここでは「入り浸る」を
\\	「多くの時間を費やす」で表現しています。	
\\	連続して取れる休暇は最長でも1週間。あと、風邪で病院に行った場合には、レシートがあれば医療休暇として別で処理されます。 
\\	「医療休暇」は 
\\	でいいでしょう。
\\	は「…するのは認められていない」の意。	
\\	もし電話がかかってきたら、居留守を使っておいて。 
\\	「居留守」は「実際には居るが、居ないことにする」と言い換て、
\\	をポイントにして、前後に必要な表現を付け加えればいいのです。また、
\\	のようにも表現できます。	
\\	まず、英語を話すときの考え方をマスターして、後は単語を入れ替えるだけで話せるようになりたい。 
\\	「入れ換える」は
\\	は「取り替える」で表現すればいいでしょう。
\\	「繋げる、順番に並べる」。	
\\	最近、社員の入れ替わりが激しくないですか? 
\\	この場合の
\\	は「転職率」という意味になります。	
\\	いやぁ、昨日は入れ食いでねえ。 
\\	「入れ食い」は
\\	「魚が餌を食う」を使えば、簡単に表現できます。
\\	「狂ったように」て強調してもいいでしょう。	
\\	うちの会社はいればいる(長い時間いる)だけ残業代がもらえるから 
\\	ポイントとなる日本語の部分だけを英語に訳してみました。「いればいるだけ」は
\\	で表現できますね。
\\	は「稼ぐ、利益を上げる」の意味。	
\\	この2週間、色々あってさ、あまりよく眠れなかったんだ。 
\\	で「色々経験する」の意味になります。	
\\	このタオル色落ちするよ。白いものと分けて洗わなきゃ。 
\\	は「色がにじむ」という意味です。
\\	でも「色落ち」を表現できます。	
\\	合成洗剤でこの帽子を洗うと色がにじむことがあるので、注意した方がよい。 
\\	「にじむ」は
\\	で表現します。
\\	は「合成洗剤」の意。	
\\	伝統と栄光に彩られたその学校の歴史に、消しようのない汚点が加わった。 
\\	で満ちた」、
\\	で包まれた」を使って「彩る」を表現しています。
\\	「汚れていない」、
\\	で汚される」、
\\	「消すことができない」、
\\	をたくさん持っている」、
\\	その町を色にたとえていうと、ワインレッドかな、 こう、なんか気品があって、落ち着くのよね。 
\\	で「もしその町が色だとすると」の意味になります。
\\	「落ち着きのある」、
\\	「威厳、気品」。	
\\	人を色眼鏡で見てはいけない。 
\\	で「先入観を持って 
\\	に近づく」の意味。
\\	「早まった判断をする」。	
\\	彼女は色(目)仕掛けで男を誘惑する。 
\\	「色仕掛け」をここでは
\\	「自分の容姿、セックスアピールを使う」で表現しています。文脈をはっきりさせるために、「誘惑する」を「男から欲しい物を手に入れる」に変えて英語にしました。	
\\	これまで我が社にそうしたシステムがなかったけど、別に違和感はないよ。 
\\	「違和感はない」は
\\	「居心地が悪いと感じることはない」ということですね。	
\\	来年からといわず、今すぐにでも留学すべきかなって最近思ってきた。 
\\	「~とはいわず」を
\\	(…をいうのをやめた方がいい)を使って表現しています。	
\\	もう何度も言わせないでよ。さっきから、何回起こしていると思ってるの?(朝、なかなか起きない人を起こす) 
\\	「言わせる」は使役動詞の
\\	を使って
\\	と言えばいいでしょう。	
\\	俺から言わせれば、それはド素人のセリフだ。 
\\	(僕の意見を求めるなら)で「俺から言わせれば」を表現しています。
\\	は「ずぶの素人」の意。	
\\	愛するがゆえに言葉にできないこともあるし、言わぬが花ということもあります。 
\\	で「言わないままにしておく方がいいこともある」の意味になります。	
\\	言われたことしかできないなんて、新人でもできるっつうの。プロジェクトから外されるのも当然の結果だね。 
\\	「言われたことしかできない」は 
\\	でいいでしょう。
\\	は「…と違いがない」の意。【注】2番目の和文は英訳していません。	
\\	電車で行った方が景色が見れていいと言われて行ったけど、寝てしまって意味なかった。 
\\	この場合に「言われて」は
\\	、あるいは、
\\	でいいでしょう。
\\	「無駄な」。	
\\	「いつ誰が来ても恥ずかしくないように常に部屋をきれいにしておきなさい」と言われて育った。 
\\	で「…するように言われながら育てられる」の意味になります。
\\	は「決まりが悪い思いをする」の意。	
\\	そう言われてみれば、そんな気 がする。 
\\	には「(ことばで)言う」の意味がありますので、
\\	で「そういうふ風に言うんだったら」と言っていることになります。
\\	「(ちょっと)言う」。	
\\	言われなくてもわかってるよ。私って、ばか正直なんです。言わなくていいことまで言ってしまったりする。 
\\	で「そんなこと言わなくてもいいよ」の意味になります。	
\\	もうこれ以上洋服を買わないように、今日意を決してクレジットカードを解約したんだ。 
\\	「意を決する」とは 
\\	(決心する)ということですね。	
\\	こことここに印鑑を押してください。 
\\	ここでは
\\	「印鑑」の前に
\\	(署名、サイン)を付けています。	
\\	最近の銀行員の態度 はなんと慇懃無礼なことか。何様だと思ってるんだ。 
\\	「慇懃無礼」は
\\	「上辺は丁寧なようで実は無礼」でいいでしょう。	
\\	なんか彼がやる事って陰険だよね。かなり性格が悪いに違いない。 
\\	は「ずるい」、
\\	は「策略を使う」の意味。
\\	「ペテンにかける」。
\\	「陰険な」も使えます。	
\\	中学・高校では陰湿ないじめが増える傾向にあるという。 
\\	「陰湿」は
\\	で表現できます。
\\	「微妙な点」。	
\\	今年、印象に残った出来事はありますか? 
\\	(目立つ、際だつ)と
\\	を組み合わせて「印象に残る」を表現しています。	
\\	初の海外で一緒に旅行にいったうちの一人がスリに遭ってもう大変だったんだ。だから、あまりいい印象が残ってないのだと思う。 
\\	で
\\	の印象を残す」という意味になります。	
\\	その会社では、会長が院政を敷いていて、社長はお飾りにすぎない。 
\\	「院政」をここでは「引退した社長が会社を裏で操っている
\\	と説明しています。
\\	「名目上の長」、
\\	「操り人形」	
\\	決勝戦は、例年どおり、AチームとBチームとの因縁の対決となった。 
\\	「因縁の」を
\\	(…する運命の)で表現しています。
\\	は「…することになる」の意味。	
\\	あのコマーシャルってインパクト強いね。特に、音楽が耳に残るよ。 
\\	で「強い印象を与える」の意味になります。
\\	でも同じような意味が表せます。「耳に残る」は
\\	を使わずに
\\	と言います。	
\\	証拠を隠滅しようと思ったのに、しくじってしまった。 
\\	が「証拠を隠滅する」に当たる言い回しです。
\\	は「しくじる」の意。	
\\	前の車、ウィンカーを出すのが遅過ぎるよ。 
\\	「ウインカーを出す」を
\\	で表現しています。ちなみに「ウインカー」はアメリカでは
\\	、イギリスでは
\\	といいます。	
\\	知り合って以来長いが、彼の浮いた話を聞いたことがない。 
\\	は名詞で「触れること、言及すること」という意味。
\\	を使ってもいいでしょう。	
\\	(領収書の宛名はどうされますか)上様でお願いします。 
\\	で「名前を明記しなくてもいい」と言っているのです。
\\	「顧客」。	
\\	迂回のルートは確かになだらかだけど、道幅は狭くしかも道の両サイドって何も柵がないんだよ。 
\\	「迂回」には
\\	という言い方がありますが、ここでは
\\	「回り道」を使っています。	
\\	もう面接まで日にちがない。うかうかしてられない。 
\\	は「怠ける、だらだらする」の意味。
\\	「機敏な、頭が切れる」。	
\\	うかつにも昨日投稿するのを忘れてしまった。 
\\	「うかつ」は
\\	「軽率」で表現できます。
\\	「愚かにも」。	
\\	何でそんなに浮かない顔をしてるんだよ。何か嫌なことでもあったなのかな。 
\\	は「ふさぎ込んでいる」という意味になります。
\\	「浮かない顔」。	
\\	色々な情報を統合して、企業の実態を浮かび上がらせる。 
\\	ここでは「浮かび上がらせる」を
\\	(…を明るみに出す)を使って表現しています。
\\	は「表面化する」の意。	
\\	あいつ、浮かれてるよ。私が気があるって勘違いしてんじゃないの? 
\\	(有頂天になる)、
\\	(得意になる)で「浮かれる」を表現しています。	
\\	彼がこのごろ愛想がいいのは何かたくらんでいるからだと、うがった見方をする人もいる。 
\\	「うがった見方をする」とは
\\	(洞察力のある見方をする)という意味ですね。
\\	(適切な)。
\\	(愛想がいい)、
\\	(たくらむ)。	
\\	彼女やっと結婚が決まったみたいだよ。もう毎日うきうきで、それに付き合わされるこっちは大変よ。 
\\	は「陽気な」、
\\	は「うきうきした」の意味。
\\	は「興奮状態」、
\\	は「分かち合う」の意。	
\\	若者ばかりのそのコンサート会場でおじさんのわたしはすっかりまわりから浮いていた。 
\\	「浮く」は
\\	「合わない」、
\\	「場違い」で表現すればいいでしょう。
\\	「(のろまな)老人」。	
\\	我が大学は、交流と謳っているにも拘わらず、学生を送り出すばかりで、留学生を受け入れる受け皿がないのは困ったもんだ。 
\\	ここでは「受け皿がない」を
\\	(…を受け入れる準備ができていない」で表現しています。
\\	は「大いに宣伝する」の意。	
\\	受付は、大体毎年夏頃にはじまってた気がするよ。違ったけ? 
\\	「申し込みを受け付ける」は
\\	で表現できます。
\\	「申し込む」。	
\\	彼は、うけねらいで、わざと間違った答えをした。 
\\	「うけねらい」は
\\	「みなを笑わせる」でもいいのですが、ここでは
\\	「うける」、
\\	「(くすくす)笑いをとる」を使って表現してみました。	
\\	受けたいコースが夜の7時から9時なのですが、授業が終わる時間帯でもまだバスが動いているか心配なのですが。 
\\	この場合の「動いている」は 
\\	で表現します。
\\	は「…を心配する」の意。	
\\	自分の職業を決めるとき、まずもって世の中の動きを学ぶことが大切であるとわかった。 
\\	ここでは「動き」を 
\\	(あり方)で表現しています。
\\	と 
\\	を組み合わせてもいいでしょう。	
\\	(サッカーの試合で)あいつ今日は動きが悪いな。交代させるか。 
\\	で「調子がよくない」の意味。
\\	はここでは「鈍い」の意。	
\\	このマウス、動きが悪いな。 
\\	「動き」については自動詞 
\\	を使って表現すればいいでしょう。	
\\	一時は雨後の筍のようにカラオケ屋ができたけど、近頃は数がかなり減ったよね。 
\\	で「キノコのようにどんどん出てくる」の意味。
\\	「繁茂した状態ではなくなる(薄くなる)」、
\\	「雑草のようにどんどん出てくる」。	
\\	憂さ晴らしにみんなで飲みに行くか。 
\\	とは「浮かれて騒いで楽しむこと」の意。	
\\	彼はどうもうさんくさい。 
\\	「うさんくさい」とは
\\	「怪しい(ところがある)」ということですね。	
\\	あいつ、何度も電話してきてうざい! 
\\	とは「厄介者」のことです。
\\	は「迷惑な」の意。	
\\	座席を後ろに倒してもいいですか? 
\\	「後ろに倒す」は
\\	を使って表現すればいいでしょう。	
\\	授業は残り5回しかないので、テキストの後ろの方はきっと授業でやる時間ないですよね。 
\\	「後ろの方」は
\\	で表現すればいいでしょう。
\\	は「取り扱う、カバーする」の意。	
\\	おい、セーターを後ろ前に着てるぞ。 
\\	「後ろ前」は
\\	を使って表現します。	
\\	何か後ろめたいことがあると相手の言動にハラハラしちゃうよね。 
\\	(恥ずべきことをする)を使って「後ろめたいこと」のニュアンスを出しています。	
\\	このみそ汁、薄すぎるよ。 
\\	「薄い」は
\\	、または、
\\	(水っぽい)を使えばいいでしょう。	
\\	うちの上司は、私が辞めることをうすうす感づいているかもしれない。 
\\	「かすかに」、
\\	「予知する」を組み合わせれば「うすうす感づく」が表現できます。
\\	「あいまいな」。	
\\	薄着をしてクーラーのきいた部屋にずっといたら、風邪を引いてしまった。 
\\	で「薄着をしている」の意味になります。ここでは
\\	(分詞構文)の
\\	が省略されています。	
\\	頭の後ろの方が薄くなってきたみたい。 
\\	は動詞「薄くなる」、形容詞「薄い」として使えます。「頭の後ろ」は
\\	や
\\	などを使って表現すればいいでしょう。また、
\\	「28歳の時に、禿げ始めたんです」のように
\\	を動詞としても使えます。	
\\	"初対面でいきなり
\\	""なんて言われたらちょっと嘘っぽくない?なんか、そういう人ってだれにでも言ってそう。 
\\	で「ちょっと不誠実に聞こえる」の意味。
\\	を使ってもいいでしょう。
\\	間違った情報を伝えたからって、僕のことを嘘つき呼ばわりするのはやめてくれ。 
\\	が「嘘つき呼ばわりする」に当たる表現です。
\\	は「…ということになる」の意。	
\\	嘘泣きしてもだめだよ。 
\\	「嘘泣き」は
\\	、あるいは、
\\	で表現します。	
\\	よくもまあ、あそこまで嘘八百を並べられるものだ。 
\\	は「大量の…」、
\\	は「…の連続」の意味。	
\\	「この程度の借金は1カ月で返せる」と、男はうそぶいた。 
\\	「うそぶく」は
\\	「自慢する」、あるいは、
\\	「大きいことを言う、豪語する」を使って表現します。	
\\	心豊かで、自立的、なおかつ国際性ある教育などと、謳い文句だけは立派だねぇー。 
\\	「謳い文句」は
\\	を使えばいいでしょう。
\\	「立派に聞こえる」、
\\	「大いに宣伝する」。	
\\	そういう疑い深い目で見るよな。浮気なんかしてないんだから。 
\\	で「疑い深い目で見る」の意味になります。
\\	で「(…に内緒で)浮気をする」の意。	
\\	この番組やらせっぽいな。一度こういうのがわかると、どの番組もそうじゃないかって疑うよね。 
\\	「疑う」は
\\	ですね。ここでは「やらせ」の表現の仕方にも注目。	
\\	初めは自分の目を疑ったね。あれ?って。 
\\	で「ハッとして見直す」という意味があります。	
\\	彼はうたれ強い。 
\\	は「ひどい目にあう」、
\\	「こき使われても平気」を使えば「打たれ強い」に近い意味が表せます。	
\\	あの男はうだつの上がらない男だ。 
\\	「うだつが上がらない」を 
\\	(成功しない)を使って表現しています。	
\\	会議するほどじゃないけど、ちょっと打ち合わせが必要だよね。 
\\	「打ち合わせ」は
\\	でいいでしょう。
\\	「時間を取る」の意	
\\	高校時代はクラブ活動に打ち込んだ。 
\\	で「…に専念する」という意味になります。
\\	も使えます。	
\\	不動産を買うとき、消費税って内税、外税それとも非課税? 
\\	「内税、外税、非課税」をそれぞれ 
\\	を使って表現しています。	
\\	昔は父親とよくゴルフの打ちっ放しに行ったけど、コースを回ったのは、旅行でサイパンに行ったときにやった1回だけです。 
\\	「ゴルフの打ちっ放し練習場」は
\\	と言います。	
\\	彼は打ち解けるまでにちょっと時間がかかるタイプみたい。 
\\	(くつろぐ)、
\\	(好意を寄せるようになる)で「打ち解ける」を表現しています。	
\\	読み始めの頃は、何がなんだかわからなかったけど、読んでいくうちに徐々に話の内容が理解できてきた。 
\\	ここでは「~のうちに」を
\\	「…する過程で」を使って表現しています。
\\	は「読むに連れて」の意。	
\\	このキーボード、僕には小さすぎて打ちにくい。 
\\	「打ちにくい」は
\\	と
\\	、あるいは、
\\	と
\\	を組み合わせて表現すればいいでしょう。	
\\	うちの息子は内弁慶ですから。外ではおとなしくしているらしいんですけど。 
\\	の文字通りの意味は「外ではネズミ、家ではライオン」です。
\\	「おとなしい」、
\\	「思い切って外へ出る」。
\\	「社交的な」。	
\\	長い間秘密にされていたクレムリンの内幕が、いま暴露される。 
\\	「内幕」は 
\\	を使って表現すればいいでしょう。
\\	は「暴露する」の意。	
\\	ある科学者が発見したらしいんだけど、宇宙ってカフェラテみたいな色をしてるんだって。 
\\	この場合の「宇宙」は 
\\	でいいでしょう。	
\\	あの企画、内輪受けで終わらなければいいけど。 
\\	「内輪受けで終わらない」をここでは
\\	「自分たちの範囲を出る、越える」を使って表現しています。
\\	「派閥」。	
\\	うっかりやってしまった。 
\\	「うっかり」は、
\\	や
\\	あるいは
\\	など表現できます。	
\\	社との駆け引きでは、前回は土壇場でうっちゃられ、今回は向こうのいいように寄り切られた。 
\\	で「最後で形勢逆転する」という意味。
\\	「議論」、
\\	「なすがまま」、
\\	「土壇場の策略で」。	
\\	梅雨でうっとうしい天気が続いている。 
\\	「憂鬱な、薄暗い」、
\\	「(雨などが)止む」。 
\\	は名詞で「休止」の意味になる。	
\\	この天気、うっとうしいね。 
\\	「うっとうしい」には
\\	「憂鬱な、気が滅入る」がピッタリです。	
\\	人身事故のため、中央線は現在不通になっています。 
\\	で「一時不通になっている」の意味になります。
\\	は「死傷者」の意。	
\\	(レントゲン撮影で)左腰を上げて、ぐるっと回って、うつぶせになってください。 
\\	「うつぶせ」は
\\	、「仰向け」は
\\	と言います。	
\\	私、風邪気味なのでうつらないように気をつけてね。 
\\	「うつる」を
\\	で表現しているところがポイントですね。	
\\	皆、彼女は社長の器じゃないと言ってるけど、器なんてあとからついてくるもんだ。 
\\	で
\\	になるのに必要なもの」という意味になります。これで「器」が表現できますね。
\\	に向いていない」。	
\\	いきなり腕立て伏せ100回やれって言われたって、そんなのできないよ。 
\\	「腕立て伏せ(一回)」は
\\	と言います。「腹筋」は
\\	シェフが腕によりをかけて作った料理というわりにはたいしたことない味だね。 
\\	「腕によりをかける」は「持てる技術
\\	を使って努力 
\\	した」と言い換えられますね。	
\\	彼女、料理の腕を上げたね。 
\\	「腕を上げる」は
\\	(進歩する、上達する)で表現すればいいでしょう。	
\\	(心電図検査で)裾をちょっと上げて、腕もまくってください。 
\\	「(裾を)上げる」は
\\	、「(腕を)まくる」は
\\	を使います。	
\\	私は日本の歴史とか地理については、ほんと疎いので今更だけど勉強したほうがいいのかなと感じています。 
\\	「うとい」とは 
\\	(…を知らない)ということですね。	
\\	夕方からの英会話クラスを受けようと思いつつ、ついついうとうと寝てしまい気づいたら8時でした。 
\\	ここでは「うとうとする」を 
\\	で表現しています。	
\\	最近、彼の評価はうなぎ登りだね。 
\\	ここでは「うなぎ登り」を
\\	(急速に上がる)を使って表現しています。	
\\	それは自信ではありません。自惚れです。 
\\	とは「自己陶酔」のことです。
\\	は「うぬぼれの強い」の意。	
\\	彼の言うことを鵜呑みにしちゃだめだよ。いい加減なことばかり言ってるんだから。 
\\	で「人が言ったことを額面通りに取る」の意味です。
\\	「不十分な、いい加減な」。	
\\	コネとかあれば簡単に仕事見つけられるけど、そんなうまい話ないしな。 
\\	「うまい話」をここでは
\\	「そんもの僕にはない」で言い換えています。	
\\	あいつとは馬が合う。 
\\	「馬が合う」とは、「仲がいい」ということ。そこで直ぐに思いつくのが
\\	です。同じ様な意味で、
\\	もよく使います。
\\	とは「直ぐに仲良くなる」の意味。これも一緒に覚えておくといいでしょう。	
\\	今、あいつには勢いがあるから、何をやってもうまく行くんだよ。 
\\	「うまく行く」は
\\	で表現できます。
\\	で「絶好調である」の意。	
\\	何度練習してもうまくならないと、嫌になって最後には諦めてやめてしまうんです。ま、きっとセンスがないってことなんだね。 
\\	「うまくなる」は 
\\	で表現できます。
\\	は「(持って生まれた)才能がない」の意味。	
\\	奴がどこまで信用できるか、海千山千だからな。 
\\	「海千山千」を
\\	「多くのことくぐり抜けてきた、経験してきた」、
\\	「色々なことを経験し、見てきた」で表現しています。	
\\	海の見える部屋を予約したんじゃなかったの? 
\\	が「海の見える部屋」に当たる表現です。
\\	の使い方がポイントですね。	
\\	その傷、ほっておいたら膿んじゃうから、消毒した方がいいよ。 
\\	「膿む」は
\\	と言います。
\\	「殺菌(消毒)する」。	
\\	今度、埋め合わせするからさ。 
\\	「埋め合わせする」は
\\	「償う」で表現すればいいでしょう。	
\\	その問題をうやむやにするな。 
\\	は「ごちゃごちゃの、ぼんやりした」の意味。
\\	(曖昧な)を使ってもいいでしょう。	
\\	あの問題も、紆余曲折を経て解決した。 
\\	がまさに「紆余曲折」のことですね。
\\	は「込み入った、複雑な話」の意味になります。	
\\	あの人、どうも裏がありそうで信用できないよね。 
\\	「裏がある」とは
\\	「密かな動機がある」ということですね。
\\	「隠された」。	
\\	嫌悪とは、結局愛の裏返しなんだ。 
\\	(表と裏)を使えば「裏返し」が表現できますね。	
\\	靴下脱ぐとき裏返しに脱がないでくれる? 畳むときに面倒くさいから。 
\\	で「…を裏返しにする」の意味になります。	
\\	あーあー、彼、興奮し過ぎて声が裏返ってるよ。 
\\	で「裏声になる」の意味。
\\	「声がうわずって... になる」。	
\\	この紙捨てるの勿体無いから、裏紙にでも使おうっと。 
\\	「裏紙に使う」を文字通りに
\\	と英訳しています。
\\	を使ってもいいでしょう。	
\\	あいつに裏切られるとは思ってもいなかった。 
\\	「裏切る」は
\\	ですね。
\\	でも同じ意味になります。	
\\	彼って裏口入学したらしいね。 
\\	「裏口入学する」は英語でも同じように
\\	と言います。	
\\	アメリカ北東部にも「浦島太郎」に似たようなタイムスリップの民話があるらしいよ。 
\\	「浦島太郎」はそのままローマ字書きします。ここでのポイントは
\\	に似ている」と
\\	に相当する、類似する」の使い方ですね。	
\\	彼女は思いとは裏腹な態度をとった 
\\	「裏腹」は
\\	と反対の」、
\\	「反対、逆」を使えば表現できます。	
\\	先日深夜番組で、うらぶれたトシちゃんを見た。 
\\	は「みすぼらしい姿の」、
\\	「人生において1段も2段も落ちぶれてしまった」の意。	
\\	俺は、彼から恨みを買うようなことは何一つしていないよ。 
\\	で
\\	から反感を買う」という意味になります。
\\	は「…の怒りを招く」の意。	
\\	大事をとって、好調の投手をあえて交代したのが裏目にでたね。 
\\	「裏目」は
\\	「期待したとは反対の結果になる」か
\\	「被害の方が大きい」を使えばいいでしょう。	
\\	趣味と仕事が同じって人をみると羨ましい。好きなことをして稼げるって最高だよね。 
\\	「羨ましい」は
\\	(形容詞)を用いて表現します。
\\	は「趣味でお金を稼ぐ」の意。	
\\	裏を返せば、それは私はあなたのことには関わりたくないですよということだ。 
\\	(はっきり言えば)、
\\	(言い換えれば)で「裏を返せば」を表現しています。	
\\	追手のウラをかいて、地下道を南下したアッシュの運命や如何 に。まて次号。 
\\	日本語をできるだけ忠実に英語に訳してみました。どのように英語で表現されているかチェックしてみるといいでしょう。
\\	「裏をかく、出し抜く」、
\\	「思いがけなく」の意。	
\\	博士のアリバイ、ウラをとったか?ライアン君 
\\	「ウラをとる」とは
\\	「証拠を得る」ということですね。	
\\	裏をばらすと、実は先週私が言ったジョークは、このテキストに載っていたのです。そして、前日から練習していたんです。 
\\	で「(人に)秘密を打ち明ける」という意味になります。
\\	は 
\\	と言い換えられます。	
\\	そこの売りは、手ぶらで行けるバーベキュー場です。 
\\	「売り」は、
\\	「一番いいのは」、
\\	「最も魅力的なのは」と言い換えて表現すればいいでしょう。	
\\	欲しいと思った時に買わないと、後でと思うと売り切れてて悔しい思いをよくする。 
\\	「売り切れ」は
\\	、あるいは、
\\	を使って表現します。
\\	「後悔する」。	
\\	あの親子うり二つじゃない。 
\\	「うり二つ」には
\\	という決まった言い方があるのですが、この表現は、親子には使わないようです。したがって、ここでは、
\\	「似ている」を使って表現しています。また、
\\	「彼は父親そっくりだ」とい	
\\	うるさい。ちょっと静かにしてくれる? 
\\	「うるさい」は
\\	を使って表現すればいいでしょう。
\\	「低く抑える」の意。	
\\	"今月の予算が予算がってうるさいな。まったく。節約すればいいんでしょ! はいはい。 
\\	""で「今月の予算がああだこうだ(とうるさく言う)」の意味を表しています。
\\	時々お客さんで日本人の店員がいてくれてよかったと言われるとうれしい。 
\\	ここでは「うれしい」を 
\\	を使って表現しています。
\\	は「ほっとする」の意。	
\\	うれしいことに、彼はいやな顔をせず、私の頼みごとを受け入れてくれた。 
\\	「うれしいことに」は
\\	「幸運なことに」を使って表現すればいいでしょう。この文脈では
\\	とも言えます。	
\\	健康でいられるほど、うれしいことはない。 
\\	「満足の行く」と
\\	を組み合わせて「うれしいことはない」を表現しています。	
\\	私は、こうやってあなたと文通できることを、うれしく思っています。 
\\	「うれしく思う」は
\\	ですね。
\\	も使えます。
\\	「…と文通する」。	
\\	1年後、合格した暁には嬉し泣きしたいものです。 
\\	が「嬉し泣き」に当たる表現です。	
\\	見てこれ、どう考えてもこんなの売れないと思わない?この店よくこんなの置いてあるね。 
\\	「売れる、売れない」は自動詞 
\\	を使って表現できます。	
\\	どうせ私なんて売れ残りよ。 
\\	が「売れ残り」に当たる表現です。
\\	を文字通りに訳すと「結婚市場でうまく行かない(負ける)」になります。	
\\	試着室が空くまでその辺でうろうろしていますから、空いたら呼んでくれませんか? 
\\	は「歩き回る、ぶらぶらする」の意味。
\\	は「うろうろする」の意。	
\\	試着
\\	もし彼が浮気したら、すぐに別れます。 
\\	が「浮気をする」に当たる動詞です。
\\	は「浮気、情事」の意。	
\\	私が帰ってから変な噂が広まってないといいな。 
\\	「噂」は
\\	ですね。
\\	「陰口をたたく」。	
\\	これが噂の行列のできる店か。 
\\	「噂の」は 
\\	(皆が話している)で表現すればいいでしょう。
\\	を使うこともできます。	
\\	うわついた気持ちで仕事をしているとミスするよ。 
\\	「うわついた気持ち」を「真剣でない」と言い換えて、
\\	で表現しています。
\\	「台無しにする」、
\\	「軽々しい」。	
\\	今の給料に、特別手当を上乗せするので、うちの会社に来ないか。 
\\	「上乗せ」は 
\\	(付け足す)で表現すればいいでしょう。
\\	の使い方にも注目。	
\\	彼は、人の話しをいつもうわの空で聞いている。 
\\	「うわの空」は
\\	「心は他のどこかにある」で表現できます。
\\	「うなずく」。	
\\	きみの能力と運をもってすれば、必ず来週の試験には合格するよ。 
\\	この場合の「運」は 
\\	でいいでしょう。	
\\	私って何て運が悪いんだろう! 
\\	「運が悪い」は 
\\	で表現します。	
\\	(あれえ,電車こないなああ。)その電車は土日運休なんですよ。 
\\	「運休」とは
\\	「運転していない」、
\\	「走っていない」ということですね。	
\\	この日のこの時間のバスにあまり乗る(予約)人がいない場合は、運行を中止することはありますか? 
\\	「運行中止」は 
\\	で表現すればいいでしょう。
\\	は「時間帯」の意。	
\\	サッチー論争、もう、うんざりだね。 
\\	ここでは「サッチー」の代わりに
\\	を使っています。「うんざり」は
\\	「(聞きすぎて)まひした」の意。
\\	の意味は「聞きたいと思っている以上に聞いてしまた」ですね。
\\	「いらだたせる」。	
\\	いくらカレーが好きだからって、さすがに1週間も食べ続けるとうんざりする。 
\\	自分のことを言う場合でも、上の英文のように主語を
\\	にすることがよくあります。
\\	で「どんなに... でも」の意。ポイントとなる「うんざりする」は
\\	で表現しています。	
\\	きっと彼は、うんざりするよ。 私だって、そうよ。 
\\	「うんざりする」は
\\	で表現できます。
\\	「興味を失う」。	
\\	星占いによると、今年の僕の運勢はいいらしい。 
\\	ここでは「今年の運勢はいい」を簡単に
\\	で表現しています。	
\\	運動会は、今度の日曜日に延期になったよ。 
\\	が「運動会」に当たる表現です。	
\\	すごい運動ができそうに見えますね。 
\\	で「運動をたくさんする」の意味。
\\	「スポーツマンのように見える」。	
\\	私は運動神経がないので何をやってもどんくさいです。 
\\	(ひどい反射神経をしている)で「運動神経がない」を表現しています。
\\	は「動きの鈍い」の意。	
\\	運動神経が鈍い子をもつと、運動会がつらいです。 
\\	で「大した運動選手じゃない」の意味。また、この場合の
\\	は「強み、長所」の意味で使われています。	
\\	運動不足解消のため週末にジョギングをすることにした。 
\\	で「運動不足に関して何かするために」の意味になります。
\\	は「ジョギングをする」の意。	
\\	あなたと私は別れる運命だったのよ。 
\\	は「運命」、
\\	は「…する予定になっている」の意味。	
\\	運を天に任せて、やるしかないか。 
\\	で「…を運命に任せる」の意味になります。	
\\	明日は永遠に続く。 
\\	ここでは 
\\	を使って「永遠に続く」を表現しています。	
\\	英会話教室がたくさんありすぎて、選ぶのに困るよ。 
\\	「英会話教室」は 
\\	を使えばいいでしょう。	
\\	この俳優って、何とかって映画に出てた人だよね? 
\\	ここでは「何とかって映画に出る」を 
\\	で表現しています。	
\\	私って影響されやすいタイプなんだよね。映画見たらその日の夜はその夢を見るし。 
\\	この場合の「影響される」は
\\	「影響する」、
\\	「暗示にかかりやすい」を使って表現すればいいでしょう。	
\\	英会話レッスンでは、何人で受けるかによって、一人あたりの発言量に影響してくると思うので少ない(人数の)方がだんぜんいい。 
\\	「影響する」をそのまま使えます。
\\	「ずっと、はるかに」。	
\\	もうすぐ夏時間にあわせて営業時間を変更すると聞いたけど、いつから営業時間を延長するのですか? 
\\	「営業時間」は 
\\	ですね。
\\	は「延長する」の意。	
\\	普段は無愛想な彼女も、商談のときは、営業用の笑みを浮かべていた。 
\\	(商売の目的で笑う)で「営業用の笑みを浮かべる」の意味が表せます。
\\	は「冷淡な」、
\\	は「無愛想な」の意。	
\\	あの樹は数百年もの間、 黙って、世の栄枯盛衰を眺めてきたのだ。 
\\	「栄枯盛衰」は
\\	、あるいは、
\\	で表現します。	
\\	ネイティブ・スピーカーの友達の中にいると、自分の英語力不足を痛感させられる。 
\\	ここでは「英語力不足」を
\\	(自分の英語がいかにひどいか)で表現しています。	
\\	いつも笑顔でいてね。 
\\	「笑顔」は 
\\	(動詞、名詞)で表現できます。	
\\	パソコン画面は液晶が主流になってきた。 
\\	「液晶画面」は 
\\	と言います。
\\	は 
\\	の頭文字語です。	
\\	学生の頃、某劇団に入団していたのだけど、セリフなしのエキストラばっかりだった。 
\\	「エキストラ」はそのまま
\\	が使えます。
\\	「セリフがない」。
\\	「セリフがない端役」。	
\\	田中先生ってさ、高橋さんのことえこひいきしてるよね。 
\\	「えこひいき」は
\\	(動詞)を使えば表現できます。	
\\	さっき、君に会釈した人って、君の知り合いなの? 
\\	「会釈する」は
\\	(動詞)、
\\	(名詞)を使います。	
\\	足のレーザー脱毛をエステでしている。 
\\	「エステ」は 
\\	、あるいは、
\\	でいいでしょう。	
\\	もー、あなたって本当にエッチなんだから。 
\\	和英辞典で「エッチ」を引くと
\\	という語が出ていますが、ここでは
\\	「そのことばかり(セックス)考えている」と
\\	「セックス以外のことは考えられないの」で表現してみました。	
\\	白人がビーチで読書するのは絵になるが、日本人がやると似合わない。 
\\	で「美しい絵になる」、
\\	は「似合わない」の意味。
\\	と 
\\	の組み合わせにも注目。	
\\	映画のエヌジー特集って意外に面白いんだよね。 
\\	とは「カットしたシーン」のことです。
\\	「どじ、失敗」を使ってもいいでしょう。
\\	「とても楽しい、大笑い」。	
\\	怒ることで、マイナスにエネルギーを使うよりも、許すことにエネルギーを使ったほうが、後々の自分の精神状態に良い。 
\\	で「エネルギーを無駄に使う」の意味になります。「使う」の意味では
\\	あいるいは
\\	も使えます。	
\\	お母さんのお手伝いして、偉いね。 
\\	この場合の「偉いね」は
\\	「いい子だね」でいいでしょう。	
\\	こんなこと偉そうに言える立場じゃないけど。 
\\	の文字通りの意味は「高圧的(高飛車に)言えるような立場にない」になります。
\\	「(偉そうに)言えない」。	
\\	えり好みしなければ、仕事はなんでもあるよ。 
\\	には「好みのやかましい」という意味があります。
\\	は「えり好みする」の意。	
\\	日本人は人との出逢いを大切にします。私達には縁があったのよ。 
\\	ここでは「縁がある」を
\\	(…することになっている、運命づけられている)を使って表現しています。	
\\	今月は宴会続きで胃がもたれっぱなしだよ。ひどいときは3日連続とかね。 
\\	「宴会続き」は
\\	と複数形にすることで表現できます。後は動詞の使い方ですね。
\\	「3日連続で」。	
\\	今夜は宴会部長の山田がいないから、今ひとつ盛り上がらないね。 
\\	「宴会部長」に当たる英語は... 
\\	「パーティが好きでたまらない人」でしょうか。「盛り上がらない」は
\\	を使えばいいでしょう。	
\\	今日の会議、すべて円滑に運んでよかった。 
\\	で「円滑に運ぶ」の意味になります。
\\	「(見事に)やる」、
\\	「障害」。	
\\	彼女とは縁がないと思って、あきらめるんだな。 
\\	この場合の
\\	は「運命づけられている」という意味になります。	
\\	遠距離恋愛って、結局、長続きしないんだよね。 
\\	で「遠く離れていながらもお互いの関係を保つこと」の意味になります。
\\	「続く」。	
\\	靴のひもが切れるなんて縁起が悪いな。 
\\	「縁起が悪い」は
\\	で表現できますね。「切る、切れる」は
\\	を使います。	
\\	中高生による薬物乱用、援助交際等の犯罪が増加してきている。 
\\	「援助交際」には
\\	という言い方もありますが、ここでは
\\	「高校生売春」を使っています。	
\\	人々は手をつないで円陣を組んでその建物を取り囲んだ。 
\\	が「円陣を組む」に当たる表現です。	
\\	最近円高が絶好調だ。換金するのをもう少し待っていればよかった。 
\\	「円高」は 
\\	と 
\\	を使って表現すればいいでしょう。
\\	は「両替する」の意。	
\\	この炎天下にマラソンしている人みたよ。 
\\	は「焼けつくような」、
\\	は「猛烈に熱い」、
\\	は「激しい」という意味。この場合の「マラソン」は
\\	のことですね。
\\	とは42.195キロ走ることですから。	
\\	関連は言うまでもなく、英語も堪能な彼は、いつもマネージャーの縁の下の力持ちだ。 
\\	で「影で大変な仕事をする」の意味になります。
\\	「目立たない)
\\	「…することになっている)(注)英文には「いつでも裏方で大変な(つまらない)仕事をさせられている」というニュアンスが含まれています。	
\\	夫婦円満が何よりです。 
\\	「円満」とは
\\	「幸せな結婚をしている」ということですね。	
\\	円安が進んでいるね。 
\\	「円安」とはドルに対してですから、
\\	「ドルに対して低い」で表現します。「円高」は
\\	と言います。	
\\	なんか言いたそうな顔ですね。遠慮せず言って。 
\\	は「抑える」、
\\	は「ためらうことはなない」の意味。	
\\	遠慮しときます。(わざと嫌いな食べ物を食べる?とか聞かれて返事する) 
\\	は「どちらかというと 
\\	したくない」の意味。
\\	「離れる」。簡単に、
\\	と言ってもいいでしょう。	
\\	まぁ、その件は次回までお預けということにしておきましょう。 
\\	は「…を保留にする」、
\\	は「棚上げする」の意味です。	
\\	またあいつら、キスしてるぜ。見ろよ。お熱いね。いつも。見てられないよ。 
\\	で「お互いに強く(性的に)ひかれている」の意味。
\\	の真っ最中」。	
\\	お後がよろしいようで。 
\\	とは「次の人にバトンタッチします」の意味になります。
\\	には、例として、
\\	と使っています。状況に応じて、名前を替えればいいのです。	
\\	最後の追い込みだ。出来るだけのことはやろうと思う。 
\\	で「最後の追い込み(ホームストレッチだ)」の意味になります。	
\\	人間楽ばかりしてるとだめになる。たまには、自分で自分の首をしめるくらい窮地に追い込むのもありだと思う。 
\\	には動詞として「窮地に追い込む」という意味があります。
\\	「窮地」。	
\\	私の乗っているバスを、次々と車が追い抜かしていっている。 
\\	「追い抜かす」は
\\	で表現します。
\\	「…に追い抜かされる」。「次々と」の表現の仕方にも注目。	
\\	私って、無理やり勉強しなきゃいけないような環境においやらないと勉強しないタイプだから。 
\\	「おいやる」を
\\	「環境によって無理矢理(勉強するように)させられる」を使って表現しています。
\\	の前には
\\	が、
\\	の後には
\\	が略されています(1番目の英訳)。
\\	で「死に物狂いになって勉強しなくてはいけない状況)の意。	
\\	皆が応援してくれているからそれに応えるべく頑張らなくては。せめて英語だけでもマスターして帰らないと。 
\\	「…を応援する」は 
\\	で表現します。
\\	と 
\\	を組み合わせてもいいですね。	
\\	頑張ってください、応援してます! 
\\	で
\\	を応援する」の意味になります。
\\	「味方する」の使い方にも注目。	
\\	詳しい原因はわからないので、応急処置をしただけだからまた壊れる可能性はあるよ。(パソコンの場合) 
\\	で「応急処置をする」の意味になります。
\\	「生じる、現れる」。	
\\	運動といえば、毎日家と会社の往復ぐらい。やっぱり運動してないとだんだん体もなまってくるね。 
\\	この文脈では
\\	「通勤」で「往復」の意味になりますね。
\\	「往復」という言い方もあります。
\\	「なまる」。	
\\	決まったフレーズばかり覚えても、それらを応用できないと上達しないだろうな。 
\\	「応用する」は
\\	、あるいは、
\\	を使って表現します。	
\\	アイスホッケーって、オールシーズンのスポーツですか? 一度、会場で生で見てみたいのですが。 
\\	とは「一年を通してやるスポーツ」のことです。	
\\	大味な試合だった。 
\\	「大味な」をここでは
\\	「興奮を欠いた」、
\\	「活気のない」を使って表現しています。	
\\	台風が近づいているので、週末は大荒れの天気になるらしい。 
\\	で「大荒れの天気」の意味。
\\	を使って「大荒れ」を表現してもいいでしょう。	
\\	今回の経験は、今後の私の英語の勉強の仕方に大いに役立つだろう 
\\	「大いに役立つ」は
\\	「非常に価値のある」で表現できます。
\\	「とても」。	
\\	彼女はうちの部署に来た久々の大型新人です(期待の新人)。 
\\	「大型新人」をここでは
\\	「大いに期待している人物」と表現しています。
\\	「楽しみにして待つ」。	
\\	新しい靴買おうなんて思ったことあるかって?大きなお世話だよ。 
\\	とは「(他人のことではなく)自分自身のことを心配しろ」という意味。
\\	「そんなのお前に関係ないだろ」とも言えます。	
\\	あいつ新人のくせに、ずいぶん大きな顔してるじゃないか。 
\\	「大きな顔」は
\\	(態度がでかい)で表現すればいいでしょう。	
\\	彼が大きな顔をしていた。 
\\	で「その場所の所有者のように振る舞う」の意味になります。
\\	「放漫な」。	
\\	新入社員のくせに、大きな口をたたくな。 
\\	が「大きな口をたたく」に当たる表現です。
\\	は「雇われ人」、
\\	は「豪語する」の意。	
\\	あの人って大ざっぱな性格だからね。 
\\	「大ざっぱ」とは「細かいこと
\\	を気にしない」ということ。
\\	は「とやかく言う」の意。	
\\	彼と僕の考えは大筋では一致している。 
\\	は「全体的に見ると」という意味。	
\\	問題の多い政治家が増えてきたので、政界の大掃除が必要なのではないか。 
\\	で「大掃除をする」の意味になります。	
\\	彼は株の取り引きで失敗して大損した。 
\\	が「大損する」に当たる言い回しです。
\\	(すってんてんになる)の代わりに 
\\	(大損する)も使えます。	
\\	ヒンギスが一回戦敗退の大番狂わせ。 
\\	「大番狂わせ」は
\\	と言います。
\\	は動詞としても使えます。
\\	は「ふるい落とす」、
\\	は「敗北」の意。	
\\	彼は競馬で大穴を当て、友だちに大盤振る舞いをした。 
\\	(大きな宴会をする)で「大盤振る舞い」のニュアンスを表現しています。
\\	(おごる)の使い方にも注目。	
\\	あそこまで、こっちの道で行くと相当大回りになるよ。 
\\	「大回り」は
\\	でいいでしょう。
\\	の使い方にも注目。	
\\	(コーヒーのお替わりいる?) うん、ぼくのは少なめ、弟のには多めについでね。 
\\	「多め」は
\\	を使えばいいでしょう。
\\	「一杯になるまで注ぐ」。	
\\	夜遅く帰ったら、父親に大目玉を食らった。 
\\	で「しかりつける」の意味になります。
\\	「厳しく言われる」。	
\\	彼は、まだ新人なんだから大目に見てやれよ。 
\\	は
\\	に優しくする」、
\\	は
\\	に厳しくしない」という意味。	
\\	株で大もうけする時代は終わった。 
\\	ともに「大儲けする」という意味を表します。
\\	と語順が倒置されている点にも注意。	
\\	ちょっとテレビに出たくらいで、あいつ大物気取りでいるよ。 
\\	が「大物」に当たる表現です。
\\	は「重要人物」の意	
\\	いつもの定食とライスの大盛りね。 
\\	「大盛り」は
\\	、あるいは、
\\	「盛る」を使って表現します。「いつもの」はそのまま
\\	が使えます。	
\\	今回の事は公にすると今後の売り上げに影響が出るから、今回だけは許してやる。 
\\	「公にする」は
\\	で表現できます。
\\	と
\\	「大目に見る」の使い方にも注目。	
\\	知人に勧められて株に手を出し、彼は大火傷をした。 
\\	(火傷をする/痛い目にあう)、
\\	(スッテンテンになる)を使えばいいでしょう。
\\	は「…をちょっとやってみる」の意。	
\\	不正な会計操作で大揺れに揺れているアメリカで、画期的な解決案が提示された。 
\\	「大揺れ」を動詞 
\\	(揺さぶる)、
\\	(揺るがせる)を使って表現しています。	
\\	このビデオテープお買い得だったのよ。だって、3本でたったの500円だよ。私、質にはこだわらないからさ使えればいいし。 
\\	が「お買い得」に当たる表現です。
\\	は「…に口やかましい」の意。	
\\	ヴァレンタインのお返しのプレゼントだよ。 
\\	ここでは「お返し」を 
\\	(…に対する私なりのお礼の言い方)で表現しています。
\\	は「…と交換に」の意。	
\\	今日マージャンで負けたお返しは近いうちに必ずするから、楽しみに待っておれ。 
\\	(取り戻す)で「お返し」を表現しています。
\\	で「麻雀」の意。	
\\	あら、おかえりなさい! 
\\	「お帰りなさい」は
\\	で表現します。	
\\	私のおかげではないよ。私が出勤したときはすでに売上は
\\	700だったし。彼女がいたからこそ、その数字が出たんだよ。 
\\	(手柄にする)で「おかげ」を表現しています。
\\	の使い方にも注目。	
\\	私にとって両親は友達感覚でなんでも話せる存在だった。そのおかげで、隠し事をすることなく素直に育った。 
\\	「おかげで」は
\\	を使って表現できます。
\\	「素直な」。	
\\	何がおかしいの?だって本当のことだもん。しょうがないじゃん。 
\\	「おかしい」は
\\	「何笑ってるの」、あるいは、
\\	で表現すればいいでしょう。	
\\	それって比べる方がおかしいんじゃない? 
\\	「おかしい」は
\\	を使って表現すればいいでしょう。	
\\	おかしいな。確か先週提出したと思っていたけど。ま、いいや。家に帰って確認してきます。 
\\	この場面での「おかしいな」は
\\	でいいでしょう。
\\	は「とても」の意。	
\\	どちらが勝ってもおかしくない試合だった。 
\\	第一訳の意味は「どちらが勝手も驚かなかっただろう」の意味になります。これで「おかしくない」のニュアンスが出せます。	
\\	家が決まるまでここに、タンスとか洋服とかその他もろもろの荷物をおかせてもらってていいですか? 
\\	で「置かせてもらう」を表現しています。
\\	「保管する」。	
\\	「お前みてえなおかちめんこは初めて見た」と言ったら、「セクハラだ」と言われた。 
\\	ここでは「おかちめんこ」を
\\	「不細工な」、
\\	「最も不器量な」で表現しています。このような言葉を使えば、当然、相手は怒り心頭してあなたを
\\	「避難する」ことになります。	
\\	再来年に留学するって決めたのは、結局はお金の問題だけなんだ。 
\\	で「…の理由はお金の問題だけである」の意味。
\\	は「金銭上の」の意。(注)和文を一部カットしました。	
\\	はじめから、英語の先生になりたいと思ってあのコースを受講したわけではなかったので、そんなにお金をかけるつもりはなかった。 
\\	で「…に多くのお金をかける」の意味になります。
\\	は「(結果として)... するようになる」の意。	
\\	ライスとキャベツをお代わりお願いします。 
\\	「お代わり」は
\\	あるいは
\\	と言います。	
\\	彼を拝み倒して、この仕事を引き受けてもらったんだ。 
\\	で「…に嘆願する」の意味になります。
\\	は「懇願する、頼む」の意。	
\\	もしかして、食卓の上におきっぱなし? 冷蔵庫に入れておかないと、くさっちゃうよ。 
\\	には「そのままで放置してある」という意味があります。
\\	を使ってもいいでしょう。
\\	で「(食べ物が)腐る」の意味。	
\\	最近面白いことが起きないんだよねー。つまらないな。 
\\	「起きる」は 
\\	で表現すればいいでしょう。	
\\	あのドラマのエンディングで使われている曲、今ちょっとお気に入りなのよね。 
\\	「ちょっとお気に入り」を
\\	と
\\	を組み合わせて表現しています。	
\\	お決まりの質問で悪いのだけど、これ以外には思いつかない。 
\\	「お決まりの質問」とは、「よくする質問」ということですから、
\\	で表現できますね。	
\\	私共の開発したシステムはお客様に必ずや満足頂けるかと自信を持っております。 
\\	「お客様」は
\\	、あるいは、
\\	を使えばいいでしょう。	
\\	そんなお気楽なことを言っている場合じゃないよ。 
\\	で
\\	を気楽に受け取る」という意味になります。	
\\	分母に3を、分子に1をおいて下さい。 
\\	この場合の「おく」は
\\	、あるいは、
\\	でいいでしょう。	
\\	お食い初めしなくちゃね。 
\\	英語では「お食い初め」という言い方はありませんので、ここでは
\\	「離乳させる」を使って表現しています。
\\	「固形食」。	
\\	長年プログラマーやっているけど、今だにすぐ解決できないエラーとか発生したり、凡ミスをしてしまう。この仕事は、奥が深いよ。 
\\	「奥が深い」とは
\\	「それには多くのことがある」ということですね。	
\\	凡
\\	この子は、とても奥手ですからね。 
\\	「奥手」をここでは
\\	「年相応の行動をしない」、
\\	「未熟な」で表現しています。	
\\	奥歯に物のはさまったような言い方をしないで、何があったのかはっきり言ってくれ。 
\\	「遠回しに言う」、
\\	「上辺を飾ってハッキリ言わない」を使えばいいでしょう。
\\	「ハッキリ言う」、
\\	「しゃべる」。	
\\	以前、英語の電話が全然聞き取れなくて、恥ずかしいと同時に悔しい思いをしたことがあって、それ以来電話で英語を話すことに臆病になっています。 
\\	この場合の「臆病になる」は
\\	(…を怖がる)で表現すればいいでしょう。ここでは
\\	(臆病な)は使えません。	
\\	お礼状をくれるなんて、彼女はなんて奥床しいんだろう。 
\\	「奥床しい」を
\\	「丁寧な、優雅な」で表現してみました。
\\	「思いやりがある」。この場合の
\\	は「行為、行動」で「見せかけだけの、うわべだけの」という意味は含まれていません。	
\\	この間、パーティーに遅れて行くって言ったら地図をもらったんだけど、結局迷ってしまいました。 
\\	「パーティーに遅れていく」は
\\	でいいでしょう。
\\	は「迷う」の意。	
\\	遅れ馳せながら、本年もご愛顧の程よろしくお願い致します。 
\\	(少し遅くなりましたが)で「遅ればせながら」のニュアンスを出しています。	
\\	脳震盪を起こして倒れて、その後のことは何も覚えていない。 
\\	ここでは「脳震盪を起こす」を
\\	と
\\	を使って表現しています。
\\	「倒れる」。	
\\	一年分のおこずかいを先に渡しておくわ。無駄使いしないようにね。計画的に使ってね。 
\\	「お小遣い」は
\\	と言います。「一年分」を
\\	で表現している点にも注目。	
\\	おことばですが、それは違うと 思います。 
\\	は「失礼ですが」、
\\	は「間違って取らないでください」の意味になります。	
\\	お言葉に甘えて、夕食をご馳走になります。 
\\	この場合の
\\	には「(申し出に)応じる」という意味があります。	
\\	しまった!頼まれていた番組の録画セットしてくるの忘れた!怒られる!!! 
\\	ここでは「怒られる」を
\\	を使って表現しています。	
\\	長年の成功が社内におごりを産んでしまった。 
\\	「おごり」は 
\\	(慢心、自惚れ)で表現すればいいでしょう。
\\	は「…で慢心する」の意。	
\\	えっ?おごってくれるの?でも次は私に払わせてね。 
\\	「おごる」は
\\	を名詞、あるいは、動詞として使って表現できます。	
\\	もうだいたい決まったんだけど、押さえにあと何人か当たってみるよ。 
\\	とは「安全の為に」の意味。これを使えば「押さえに」の意味が表せます。	
\\	私は嫌な事はすぐ嫌と言ってしまう性格です。もう少し、自分の気持ちを抑えた方がいいのかな? 
\\	「気持ちを抑える」は
\\	で表現できます。	
\\	ゴルフを始めたのですが、最初は道具を買うのはもったいないので親のお下がりを使っています。 
\\	「お下がり」は 
\\	と言います。
\\	は「…を始める」の意。	
\\	これじゃお先真っ暗だよ。 
\\	で簡単に表現してもいいですし、また、
\\	「トンネルの出口に明かりが見えない(見通しがきかない)」を使ってもいいでしょう。	
\\	ご主人に先立たれた今のお気持ちお察しいたします。 
\\	「察する」は
\\	「想像する」を使えばいいでしょう。
\\	を奪われる」。	
\\	そんな話し方すると、お里が知れるよ。 
\\	「お里が知れる」とは「素性がばれる」ということですから、
\\	「中身が見える」、
\\	「正体を暴露する」を使って表現します。	
\\	急におなかに激痛が走って救急車を呼んだんだけど、到着したころにはもう治まっちゃってさ。いやあ参った、参った。 
\\	(減少する)、
\\	(静まる、やわらぐ)を使って「治まる」を表現しています。
\\	は「救急車」、
\\	は「当惑」の意。	
\\	どんなことをしてもこの怒りは収まらない。 
\\	この場合の「収まる」は 
\\	(消え失せる)、
\\	(…を取り除く)で表現すればいいでしょう。	
\\	結婚おめでとう。お幸せにね! 
\\	「お幸せに」は
\\	と
\\	を組み合わせて表現すればいいでしょう。	
\\	あっ、惜しい!もうちょっとで入るところだったのにね。(ゴルフで) 
\\	ここでは「惜しい」を 
\\	(もう少し)で表現しています。	
\\	教えて欲しいことがあるんだけど。 
\\	この場合の「教えて欲しい」は
\\	「聞きたい」ということですね。
\\	するつもり」。	
\\	自分の考えを人に押し付けるのは止めてよ。 
\\	は「押しつける」、
\\	は「強いる」の意味になります。前置詞は
\\	を使います。	
\\	あなたは自分の好きなことには惜しみなくお金を使うけど、それ以外の出費についてはとやかく言うのね。 
\\	ここでは「惜しみなく」を 
\\	(何の問題もなく…する)で表現しています。
\\	は「大喜びで」の意。	
\\	授業料は高いかもしれませんが、それなりの効果が得られれば授業料を惜しむこともないでしょう。 
\\	([授業料を]払うのをとても嫌がる)、
\\	(…を切りつめる、節約する)を使って「惜しむ」を表現しています。	
\\	彼女、おしゃれが上手ね。 
\\	は「おしゃれな人」のことです。
\\	と 
\\	を組み合わせても「おしゃれ」を表現できますね。	
\\	(健康診断で)カップのこの目盛りまでお小水をとってください。 
\\	「おしっこをする」は
\\	という動詞を使います。
\\	は「お小水のサンプル」の意。	
\\	お邪魔します。/お邪魔しました。 
\\	「お邪魔する」は
\\	「面倒、やっかいをかける」で表現できます。	
\\	お前のせいで計画がおじゃんになっちゃっただろ。 
\\	は「はかなく消える」の意味。	
\\	彼はまだ20代のくせに、言うことがどうもオジン臭い。 
\\	で「中年を過ぎた男のように」という意味になります。	
\\	寝る時に聞くおすすめの
\\	は何かある? 
\\	「お薦め」は動詞
\\	「推薦する」を使って表現すればいいでしょう。	
\\	田中さんにはお歳暮何贈ろうかしら。 
\\	「お歳暮」は
\\	、「お中元」は
\\	と言います。	
\\	ついにここ迄、押せ押せで来てしまった。 
\\	ここでは「押せ押せで来る」を
\\	「夢中になる(熱狂の波に乗る)」と表現しています。
\\	「頂点、波の峰」の意。	
\\	あんなお世辞にうかれちゃって、ばかみたい。 
\\	「お世辞」は
\\	と言います。「うかれる」を
\\	を使って表現している点にも注目。	
\\	彼女ってほんとおせっかいね! 
\\	で「他人事に鼻を突っ込む」という意味。
\\	に干渉する」。	
\\	いつもお世話になっております。これからもよろしくお願いいたします。 
\\	(感謝する)と 
\\	を組み合わせて「お世話になる、よろしくお願いする」のニュアンスを表現しています。	
\\	最後皆の前で挨拶なんてすると思ってなかったので、何も考えてきていません。とりあえず、今までお世話になりました。 
\\	が「お世話になりました」に当たる言い回しです。	
\\	今回の件では彼にお世話になったから、挨拶に行った方がいいだろう。 
\\	ここでは「お世話になる」を
\\	「私たちのために多くのことをしてくれた」で表現しています。	
\\	遅いなー。まだ、帰ってこないな。先に寝ちゃおうかな・・・ 
\\	「遅い」は
\\	ですね。
\\	で「先に寝る」の意。	
\\	この間、あんな遅い時間に電話してごめんね。 
\\	「とても遅い時間に電話する」は
\\	ですね。	
\\	恐れ入りますが、お客様の予約を承った者の名前を覚えていらっしゃいますか? 
\\	で「恐れ入ります」の意味が表せます。
\\	「許す」。	
\\	社長の書いた文章に手を入れるなんて、恐れ多くてできないよ。 
\\	「恐れ多い」とは
\\	(あまりに無礼である)ということですね。	
\\	お揃いのセーターを着て、あなたたち熱々ね。 
\\	「お揃い」は
\\	で表現できます。
\\	「本当に愛し合っている」、
\\	「本当にお互いに惚れ合っている」。	
\\	あそこの公園で、女性がジョギング中に襲われたって話知ってる?それ以来公園ではかなり用心してるのだけど。 
\\	「襲われる」は 
\\	で表現します。名詞 
\\	(襲いかかること)を使ってもいいでしょう。	
\\	彼女は少し英語が話せるからってお高くとまっている。 
\\	(思い上がった)で「お高くとまっている」を表現しています。
\\	は「高慢でよそよそしい」の意。	
\\	私は、そのアイドルがブレイクする前の、無名時代のお宝写真を持っている。 
\\	(珍しい貴重な)で「お宝」を表現しています。
\\	は「(値がつけられないほど)価値のある」の意。	
\\	困っているときはお互い様だろ。何があったか言ってみろよ。 
\\	ここでは「お互い様」を
\\	「友達同士は助けあうもの」と言い換えています。	
\\	忙しいのはお互い様だ。 
\\	「お互い様」は 
\\	で表現すればいいでしょう。2つ目の英文は「忙しいのはきみだけじゃない」の意味になります。	
\\	インターネットおたくが増えてるらしいよ。 
\\	とは「変な奴」、
\\	は「頭がいいがださい奴」の意。	
\\	いかにもアニメおたくっていう人っているよね 。 
\\	「おたく」にはいろいろな言い方があります。
\\	「熱狂的ファン」、
\\	「はまっている、とりつかれている」、
\\	「頭はいいが社会性に欠けるやつ」などなど。	
\\	お為ごかしの親切はやめろ! 
\\	「おためごかし」とは「表面的には相手のことを考えているようで、実は、自分の利益のことを考えている」ということですね。これは
\\	「私利的な」で表現すればいいでしょう。	
\\	まだ、カゼなおってないんだろ。お大事に。 
\\	「お大事に」に当たる英語は、皆さんよくご存じの
\\	です。	
\\	あんまり、あいつをおだてるなよ。すぐ、ずにのるからさ。 
\\	「おだてる」は
\\	「お世辞を言う」、
\\	「おべっかを使う」で表現できます。	
\\	5時に駅前の喫茶店で落ち合うことにしましょう。 
\\	「落ち合う」とは
\\	「集まる」、
\\	「会う」ということですね。	
\\	おちがない 
\\	が「おち」に当たる表現です。	
\\	ヘマしちゃって落ち込んでるの。 
\\	「落ち込む」は
\\	「落胆する」でいいでしょう。
\\	は強調のために使われています。
\\	は「失敗する」の意。	
\\	確かに電子メールって便利だけど、しばらく返事が来ないと妙に落ち着かない。ちゃんと届いたのかなとか気に障ること書いたかなとか。 
\\	この場合の
\\	は「心を乱す、平静を失わせる」という意味になります。	
\\	そんな毎日ぷらぷらしてないで、そろそろ落ち着きなさい! 
\\	「落ち着く」は
\\	で表現します。
\\	「落ち着かない」。	
\\	早く 面接終わらせて 落ち着きたいよ。待っている間が一番いやだね。 
\\	この場合は
\\	「くつろぐ」を使えばいいでしょう。	
\\	試験が終わってからというもの、すっかり気が抜けて勉強しなくなりました。最近自分の英語力が落ちてきていると感じます。 
\\	「(能力が)落ちる」は 
\\	、あるいは、
\\	を使って表現できます。
\\	は「やる気」の意。	
\\	食器洗い機って確かに便利だとは思うけど、汚れはあまり落ちないような気がする。 
\\	自動詞の 
\\	には「汚れを落とす」という意味があります。
\\	は「きれいになる」の意。	
\\	今日のお茶、ずいぶん渋いね。誰が入れたのかな。 
\\	「渋い」は
\\	「苦い」を使います。	
\\	雨降ってきたし、歩きつかれたからどこかでお茶でもしない? 
\\	「お茶をする」とは
\\	「コーヒー(紅茶)を飲む」ということですね。	
\\	こんなの日本で働いていたときと比べたら、お茶の子さいさいだよ。まだまだ、頑張れるよ。 
\\	この場合の 
\\	とは「楽な仕事」のことです。
\\	は「…とは比べ物にならない」の意味。	
\\	あいつら今日はとりあえず現状報告だけしてお茶をにごそうと思ってるんだ。 
\\	は「本腰が入っていない」、
\\	は「体裁を整える」の意味。	
\\	彼は人をおちょくる。 
\\	は「悩ます」、
\\	は「引っかき回す」の意味になります。
\\	をからかう」を使ってもいいですね。	
\\	長いフライトのせいで、化粧がいつの間にか落ちちゃった。 
\\	この場合の「落ちる」は 
\\	で表現すればいいでしょう。
\\	は「なくなる、減少する」の意。	
\\	私の英語の勉強方法の何が悪いのかわからないけど、なかなか上達しないし、落ちる一方です。 
\\	「落ちる一方」は
\\	、あるいは、
\\	でいいでしょう。	
\\	田中さん、昔郷ひろみの追っかけやってたってホント? 
\\	「追っかけ」は
\\	「熱狂的ファン、親衛隊」で表現します。	
\\	冬になると外に出るのが億劫になって、だんだん家にこもるようになった。 
\\	「億劫」は
\\	「苦痛」、
\\	「困難なこと」、
\\	「大変」を使って表現すればいいでしょう。
\\	「(家に)閉じこもりきりの人」。	
\\	見かけはおっとりしてるけど、いざとなったら彼は頼りになる。 
\\	「おっとり」を 
\\	(落ち着いた)と
\\	(物静かな)を使って表現しています。	
\\	おっはー、みんな。 
\\	「おっはー」は「おはよう」を短くした言い方ですので、ここでは
\\	'を使ってみました。第2訳では、
\\	と
\\	を強調して伸ばしています。	
\\	昨日、うちの娘におつかいを頼んだんだ。すぐ近くのスーパーなんだけどさ、ちゃんと頼んだ卵買ってこれたよ 
\\	をお使いにやる」は
\\	、「お使いに行く」は
\\	と言います。	
\\	どうもお疲れさまでした。 
\\	「お疲れさま」だからと言って、
\\	を使って表現できません。このことばが使われる状況を考えて英語にする必要があります。
\\	が無難な言い回しでしょう。この後に、
\\	などを付け加えて別れることになります。	
\\	彼女は、会社ではお局だから、もう怖い物なしだね。 
\\	「お局」をここでは
\\	「君が思っている以上に影響力(権力)がある」、
\\	「上司を黙らせる(つけあがらせない)ことができる唯一の人物」と意訳してみました。	
\\	この漬け物はビールのおつまみにちょうどいい。 
\\	「おつまみ」は 
\\	でいいでしょう。
\\	は「…によく合う」の意。	
\\	(コンビニのレジで)おつりが足りません 
\\	で「お釣りを少なく渡す」の意味になります。
\\	「足りない」。	
\\	お釣りが100円多いですよ。 
\\	この場合の「多い」は
\\	(お釣りが多すぎる)で表現します。	
\\	うちの犬はお手もお座りもできないんだよ。 
\\	「お手」は 
\\	と言います。
\\	は「(動物の)脚」の意。	
\\	君にはお手上げ。 
\\	「お手上げ」とは「諦める」ということですから、
\\	を使って表現すればいいでしょう。コンピュータを教えていて、もうお手上げ、という状況なら、
\\	と言えます。	
\\	この問題にはお手上げだよ。ヒントくれない? 
\\	「お手上げ」とは 
\\	(身動きが取れない)ということですね。	
\\	お肌のお手入れは毎日欠かさずしています。 
\\	で「毎日する肌の手入れ」の意味になります。	
\\	お手数をおかけして申し訳ありませんが、パーティー会場までの地図をファックスしていただけませんか。 
\\	が「お手数をおかけして申し訳ありませんが」に当たる言い回しです。	
\\	色々とお手数をお掛けしまして、ありがとうございました。 
\\	で「…のためにわざわざ面倒なことをする」の意味になります。	
\\	お手すきでしたら、事務室までお願いします。 
\\	「お手すき」とは
\\	(手がふさがっていない、忙しくない)ということですね。	
\\	今日のゴルフはお手柔らかにお願いしますよ。 
\\	は「厳しくしないでくれ」、
\\	「やさしくする」の意味ですね。	
\\	彼はお天気屋だから付き合いにくい。 
\\	は「お天気屋の」、
\\	は「気まぐれな」という意味です。	
\\	彼女は、とてもおてんばな女の子です。 
\\	「おてんば」は
\\	と言います。
\\	は「女の子の割には、ずうずうしい(厚かましい)」の意味。	
\\	負けたらおでこにしっぺね。 
\\	「おでこ」は
\\	、「しっぺ」は
\\	「ピシャリとたたく」でいいでしょう。	
\\	これは絶対お得ですよ。 
\\	は「絶対に買うべき品」、
\\	はの「一生にあるかないかのお買い得品」の意。	
\\	外貨って、日本で換金していくのと、現地で換金するのでは、どっちがお得? 
\\	この場合の「お得」は
\\	「よりよい換金率を得る」ということですね。	
\\	ここのランチセットってかなりお得感があるよね。 
\\	には「よい買い物、取引き、大した物」の意味がありますので、これを使えば「お得感」を表現できます。
\\	は「払ったお金に値する」の意味。	
\\	あいつなかなか男らしいところがあるんだぜ。 
\\	「男らしい」の名詞形
\\	を使っています。
\\	は「強い男の一面」の意。	
\\	仕事を依頼した相手から音沙汰がないので心配になってきた。 
\\	「音沙汰がない」とは「連絡がない」ということ。これは
\\	を使って表現すればいいでしょう。	
\\	この間のテストは散々だったよ。あんなとことに落とし穴があるなんて。 
\\	ここでは「落とし穴」を
\\	「ひっかけの問題」と言い換えています。
\\	「大失敗する」。	
\\	私を陥れようとして、彼はわざと私が失敗したように見せかけているような気がします。 
\\	で「…に罠をしかける」という意味。
\\	は「…を失敗させる」の意。	
\\	その法案の扱いをめぐって、与党と野党の間で、落としどころを探る動きが活発になっている。 
\\	「落としどころ」とは
\\	(合意点)ということですね。「与野党」の表現にも注意。	
\\	このドアの汚れを落としてください。 
\\	この場合の「落とす」は
\\	(取り去る)、
\\	(きれいにする)いいでしょう。	
\\	彼を落とす(自分にほれさせる)にはどうしたらいいかな。 
\\	「ほれさせる」の意味の「落とす」も 
\\	で表現できます。	
\\	ふざけたこと言いやがって。おとととい来やがれってんだ。 
\\	もちろん
\\	とは言えませんから、ここでは 
\\	で「ちゃんと話ができるようになってから(出直して来い)」と言い換えています。	
\\	大人げないからこんな事で怒るのやーめた。 
\\	で「あまりに子どもじみた」の意味。これで「大人げない」のニュアンスを表現しています。	
\\	あの娘、見た目はおとなしそうだけど、かなり気が強いよ。 
\\	「おとなしい」は
\\	「静かな」、
\\	「従順な」で表現します。	
\\	朝から悪いことばっかり起きると思ったら、やっぱり今日は、占いによると運勢が良くないみたいだ。おとなしくしていようっと。 
\\	(鳴りをひそめる、身を低くする)、
\\	(冷静さを失わないでいる)を使って「おとなしくしている」のニュアンスを表現しています。	
\\	きみは同年代の子よりも大人っぽく見えるね。 
\\	「大人っぽく見える」は 
\\	(ずっと年を取って見える)で表現すればいいでしょう。	
\\	電車の中ではたまにウォークマンで音楽聞いているけど、音漏れが気になるのよね。 
\\	で「音が漏れているかも知れない」の意味。
\\	「音漏れ」。	
\\	だいたい写真より実物が劣ることが多いのだけど、そこの街並みは写真よりすごかった。 
\\	で
\\	ほどよくない」の意味になります。
\\	「逆」。	
\\	彼は40代になり体力の衰えを感じ始めていた。 
\\	は自動詞で「衰える、なくなる」の意味を表します。
\\	は「なくなる、消え去る」の意。	
\\	ここ最近、体力の衰えをひしひしと感じます。年は取りたくないねぇ。 
\\	は「衰える、減退する」、
\\	は「弱くなる」、
\\	は「はっきりと」の意味。	
\\	当時の日本海軍は、驚くべき強さを誇っていた。 
\\	この場合の 
\\	は「びっくり仰天させるような」という意味になります。
\\	は「驚くばかりの」の意。	
\\	ちょっと、お腹が出てるよ。お腹が痛くなるから、ちゃんとシャツを入れなさい。 
\\	「出ている」は自動詞
\\	で表現します。【注】英語では「お腹を出して寝るとお腹が痛くなる」というよりも、「風邪を引く
\\	と考える方が自然のようです。	
\\	朝から何も食べてないから、お腹がグーグー鳴ってるよ。 
\\	「(お腹が)グーグー鳴る」は
\\	、あるいは、
\\	で表現します。
\\	「抜く」。	
\\	ピザを食べ過ぎて、お腹が破裂しそうだ。 
\\	で「はち切れそうになる」の意味。
\\	は「裂く」。
\\	(爆発する)も使えます。	
\\	お腹がよじれるほど笑ったよ。 
\\	は「(笑いで)体を折り曲げる」の意味。
\\	「脇腹」、
\\	「痛い」。	
\\	昨日の夜食べた牡蠣がよくなかったのかな。お腹の調子がよくないの。 
\\	キーワードは「お腹をこわす」になっています。これは
\\	で表現できます。「お腹の調子がよくない」は
\\	でいいでしょう。	
\\	あいつもお前も同じ穴のむじなだろ。 
\\	は「2人とも同類」、
\\	に関わっている」の意味になります。	
\\	同じお金を使うなら、買い物をして物が残るほうが有意義じゃない?飲み食いやゴルフとかってその場限りじゃない。 
\\	この文脈では、
\\	(同じお金を使うなら)で表現すればいいでしょう。
\\	は「外食する」の意。	
\\	昨日と同じぐらいの時間に帰るから、悪いけど洗濯物取り込んでおいてくれる? 
\\	で「昨日と同じような時間に」の意味になります。	
\\	この番組と同じような始まり方をする番組を以前に見たことがある気がするな。 
\\	ここでは「同じような」を
\\	(このような)を使って表現しています。
\\	は「多かれ少なかれ」の意。	
\\	きみが自分の両親を大事に思うのと同じように、僕だって僕の両親が大事なんだ。 
\\	「同じように」は 
\\	を用いて表現すればいいでしょう。
\\	のあとには
\\	が略されています。	
\\	僕はあの手のお涙ちょうだいの映画は好きじゃない。 
\\	が「お涙ちょうだいの映画(番組)」に当たる表現です。
\\	は「操作する」の意。	
\\	あの二人、お似合いのカップルだね。 
\\	「お似合いのカップル」に当たる英語表現は、
\\	である。	
\\	あの子って、チームのおにもつじゃん!? 
\\	「お荷物」をここでは
\\	「まったくの負担」、
\\	「重荷を背負わせる」で表現しています。	
\\	じゃあ飛行機チケットの手配とかお願いしていいかな。時間とかは全部任せるから。 
\\	この場合の「お願いする」は
\\	で表現できますね。	
\\	それってなんだか私がお願いしたみたいだね? 
\\	「お願いする」は 
\\	、「みたいだね」は 
\\	(…のように聞こえる)で表現すればいいでしょう。	
\\	彼は結婚が決まってからというものの、毎日のようにおのろけ話をしてきます。もう、うんざり。 
\\	「おのろけ話」を 
\\	(毎日彼が話すのは彼女のことばかり)で表現しています。	
\\	お茶の準備ができましたのでこちらへお入り下さい。 
\\	難しく考える必要はありません。「お入り下さい」は
\\	ですね。	
\\	私の古いステレオはもうお払い箱だ。 
\\	は「(不要ながらくたを)捨てる」、
\\	は「放る(捨てる)」の意味。	
\\	これ以上お話しすることはありません。お引き取りください。 
\\	ここでは「引き取る」を 
\\	(あきらめる)で表現しています。	
\\	なんでこうお金にもならない仕事を引き受けちゃうほどお人好しなんだろ。 
\\	この場合の
\\	とは「だまされやすい、おめでたい人」の意味になります。
\\	「はした金」。	
\\	今夜はこのへんでお開きにしましょう。 
\\	で「今夜はこの辺で終わりにしよう」の意味になります。
\\	「終える」。	
\\	一人暮らしをしていると、お袋の味が恋しくなるよ。 
\\	「お袋の味」は
\\	で表現できます。「恋しい」は
\\	を使えばいいでしょう。	
\\	この話につきましてはオフレコということでお願いします。 
\\	「オフレコ」は
\\	と言います。冠詞の
\\	を忘れないように!	
\\	今誰かお風呂に入っているの? 
\\	「お風呂に入っている」は 
\\	を進行形を用いて表現します。	
\\	多くの外国人の名前を覚えるのは、難しい。 
\\	この場合の「覚える」は
\\	でいいでしょう。	
\\	昨日フルマラソンを完走したが、今日は全身が筋肉痛で,歩くことすらおぼつかない。 
\\	で「足元が不安定である」の意味。
\\	は「痛い」の意。	
\\	プールで泳いでいたら両足がつって、おぼれそうになった。 
\\	「おぼれる」は 
\\	で表現します。
\\	は「(脚が)つる」の意。	
\\	毎年8月のだいたい3週目あたりはお盆と言われている期間になります。 
\\	「お盆」は
\\	のように説明することになります。	
\\	この
\\	シャツ3枚買うから、500円おまけしてくれないかな。 
\\	で
\\	を値びく」という意味になります。	
\\	この子は、とてもませています。 
\\	「ませている」とは
\\	「歳の割には成熟している」ということですね。
\\	「(話し方が)大人のように聞こえる」。	
\\	母がお見合いをしろってうるさいのよ。 
\\	「お見合い」には
\\	という表現がありますが、ここでは
\\	(仲介者を通して結婚する)を使っています。
\\	は「仲介者、まとめ役」の意。	
\\	週末、神社にお宮参りに行きます。 
\\	「お宮参り」をここでは
\\	「祝福を受ける」で表現しています。
\\	「洗礼をする」。	
\\	おしっこいっぱい出たね、オムツ替えようねー。 
\\	「オムツ」は
\\	、「替える」は
\\	ですね。
\\	は「とても、すごく(副詞)」、
\\	は「おしっこする」の意。	
\\	その店で、お目当てのバッグをやっと見つけた。 
\\	「お目当て」は
\\	「探していた」ということですね。	
\\	こちらでお召し上がりでしょうか? 
\\	ファーストフードで「お召し上がり」と言う場合には、
\\	でいいでしょう。	
\\	彼はなにも知らず、おめでたい奴だ。 
\\	は「知らぬが仏」、
\\	「生き証人」の意。また、
\\	で「何も気にとめずに生きていく」と言っています。	
\\	これは「おめでたい」とか「縁起のいい」って意味があるの。 
\\	この場合の「おめでたい」は 
\\	で表現すればいいでしょう。	
\\	なぜ彼女が怒っているのか、思い当たることがあるだろう。 
\\	人 
\\	で「(人に).... が思い浮かぶ」の意味になります。	
\\	偏見かもしれないけど、アメリカって聞くとなんか危険って言葉が思い浮ぶんだよね。 
\\	で「心に浮かぶ」の意味になります。
\\	「喚起する」。	
\\	日に日に彼への想いが募っていく。 
\\	ここでは「日に日に想いが募る」を 
\\	と 
\\	で表現しています。	
\\	さあ、思いきってやってみよう。 
\\	とは「持てる力すべてを出し切ってやってみよう」の意。	
\\	思い切って短く切ったね。でも、すっごく似合ってるよ。 
\\	「思い切って」のニュアンスを
\\	「確かに」を使って表現しています。
\\	「冗談を言う」。	
\\	久しぶりの休暇だ。温泉にでも行って、思い切り羽を伸ばしてくるか。 
\\	「思い切り」を 
\\	(すっかり)、
\\	(徹底的に)で表現しています。「久しぶり」の表現の仕方にも注目。	
\\	確かに10年も働いた会社を今更辞めるのは勿体無いことだと思うけど、ここで思い切らないと前に進めないしさ。 
\\	で
\\	する勇気を奮い起こす」という意味になります。	
\\	あの人はとても思いこみが激しいんだよね… 
\\	(…掴む、手に入れる)と、
\\	(離さない)を組み合わせて「思いこみが激しい」を表現しています。
\\	(あれこれと…を考える)。	
\\	(実際は土曜日なのに)今日は日曜日だと思い込んでいた。 
\\	で
\\	と思い込む」の意味になります。
\\	「確信する」。	
\\	今回、いろいろ物件を探してみてだいたいどこも同じ値段なんだなって思い知らされたよ。 
\\	「思い知らされる」は 
\\	(教える)、
\\	(学ぶ)を使って表現すればいいでしょう。	
\\	母国語が同じ人と話すと深いところまで話せるし、説明しなくてもいいから、本当に楽だと言うことを思い知った。 
\\	と
\\	「じかに、直接に」を組み合わせて「思い知る」を表現しています。	
\\	それはきみの思い過ごしだよ。 
\\	「思い過ごし」は 
\\	(名詞)と 
\\	(動詞)を使って表現します。	
\\	何かやるにしても始めにいろいろ考えて慎重に調べてから行動するタイプなので、思い立ってもすぐには行動できないですよね。 
\\	(何かやるにしても)と「思い立っても」が似たような意味になっていますので、ここでは後半で「思い立っても」を表現していません。	
\\	歯医者さんの治療する音を思い出しただけで、歯が痛くなる。 
\\	を考えただけで」、
\\	「その音だけで」の主語の立て方に注目。	
\\	こんなところで思い出し笑いするなよな。 
\\	で「思い出したことを笑う」の意味になります。
\\	「くすくす笑う」。	
\\	そうそう急に思い出したけど、昨日
\\	子に道でばったり会ったよ。 
\\	「思い出す」は
\\	をそのまま過去形で使えばいいでしょう。	
\\	あのレジの男の人のスマイルって思いっきりひきつってて、作り笑顔以外の何物でもないよね。 
\\	の後に
\\	を付け足して 「思い切り」を表現しています。	
\\	うーん。いいアイデアが思いつかない。 
\\	「思いつく」は
\\	や
\\	で表現できます。	
\\	自分でもびっくりなのですが、ほんと急に思いついたアイデアなんです。我ながらすごい。 
\\	「思いつく」を 
\\	で表現しています。	
\\	(海外でタクシーに乗って、)折角だからドライバーと何か話そうと思いつつ、結局何も話さないまま着いてしまった。 
\\	この場合の「思いつつ」は
\\	と考える、思う」、あるいは、
\\	を使えばいいでしょう。	
\\	彼女の新しい連絡先を控えておこうと思いつつ、ついつい忘れていました。 
\\	(…するつもり)を過去完了進行形にして「思いつつ」を表現しています。	
\\	今回のことであまり思い詰めないほうがいいよ。だれにでも失敗はあるんだから。 
\\	には
\\	をくよくよ心配する」という意味があります。
\\	「へまをする、しくじる」。	
\\	中学時代に、受験勉強で必死に勉強したことは、今ではいい思い出になっています。 
\\	「いい思い出」を 
\\	で表現しています。
\\	は「猛烈な努力」の意。	
\\	英語が話せないから、思い通りに伝えられないのが悔しい。 
\\	が「思い通りに」に当たる言い回しです。
\\	は「いらだたしい」の意。	
\\	あと1ヶ月我慢すれば冬休みだと思いながらがんばります。 
\\	この場合の
\\	は「知った上で、頭に入れる」という意味だと思えばいいでしょう。	
\\	思い残すことない?充分楽しんだ? 
\\	が「思い残すことがない」に当たる言い回しです。	
\\	私はあなたがそんなふうに思うなんて思いもしなかった。 
\\	(まったくそういう考えはない)で「思いもしない」を表現しています。	
\\	彼は思いやりがない。 
\\	は「思いやり、同情」の意味。
\\	「…に欠ける」の意。	
\\	あの人は本当にあなたのことを思ってくれてるよ。こんなにいい友だちはいないよ。 
\\	この場合の「思う」は 
\\	で表現すればいいでしょう。
\\	は副詞の働きをしています。	
\\	なかなか会えなくてもいつもあなたのこと思っているの!大好き! 
\\	で「いつも思っている」を表現しています。もちろん、
\\	も使えます。	
\\	めったに来れないから、思う存分楽しんだ。 
\\	で「思う存分楽しむ」という意味になります。
\\	は「羽目をはずす」の意。	
\\	それでは奴らの思うつぼだ。 
\\	「思うつぼ」とは
\\	「相手の仕掛けた罠にまんまとはまる」ということですね。	
\\	今やめれば、奴らの思うつぼだ。 
\\	で「(相手の)思うつぼにはまる」の意味になります。
\\	は「まさに彼らが求めていることをする」の意。	
\\	人前では感謝しておきながら、陰で悪口を言いふらすなんて、恩を仇で返すようなものじゃないか。 
\\	の文字通りの意味は「恩を悪口で返す」。
\\	でも「恩を仇で返す」が表現できます。	
\\	しばらく英語を話していないと、いざ話そうとしても口が思うように動かないんですよ。特に、朝とか。 
\\	で「思ったように」の意味になります。
\\	は 
\\	のことです。	
\\	先のことを考えるといろいろ不安になりますが、今が楽しければいいかなぁと思うようにしています。 
\\	「思うようにしている」を 
\\	(…することに決める)、
\\	(…だと自分自身に言い聞かせる)を使って表現しています。	
\\	結局、人生、自分の思うようになってる。 
\\	ここでは「思うように」を
\\	を使って表現しています。	
\\	彼女はちょくちょくメールをしに、留学センターに現れるらしいのだけど、芸能人の面影は全然なさそうだね。 
\\	「面影」を 
\\	(名残、痕跡)と 
\\	(外見)を組み合わせて表現しています。
\\	は「表れ、示す物」の意。	
\\	(遊んでる最中に)面白ーい!! 
\\	「面白い」は
\\	の他に、
\\	が使えます。	
\\	あー面白かった 
\\	「面白い」は 
\\	で表現します。	
\\	なんか何にも面白くないのに皆笑うんだよね。私には理解できない。(特に、こっちが仕事で忙しい時に笑い声が聞こえるとすっごいムカツク) 
\\	ここでは
\\	(笑えることがない)を使って「面白くない」を表現しています。(注)後半の文は英訳していません。	
\\	将来は小学校のバスケットボールのコーチになって、子どもたちにバスケットの面白さを伝えたいの。 
\\	ここでは「面白さ」を 
\\	(いかに魅力的か)を用いて表現しています。
\\	の使い方にも注目。	
\\	面白半分に友達をからかうのはやめなさい。 
\\	「面白半分に」は
\\	、あるいは、
\\	で表現できます。
\\	「困らせる」。	
\\	お客様、お持ち帰りでしょうか? 
\\	「お持ち帰り」は
\\	、あるいは、
\\	を使って表現します。	
\\	やっぱり、あなたは私の思ったとおりの人だった。 
\\	「思った通り」は
\\	と
\\	「想像する」を使って表現します。	
\\	物事ってのは思ったように行かないものだな。 
\\	で「計画したように」の意味。
\\	でも同じよな意味になります。通例、冠詞
\\	はつけないで使います。	
\\	こう見えても私にだって悩みはあるし、ストレスだってたまるんだよ!ただ面に表れないだけだよ! 
\\	「面には表れない」は
\\	で表現できます。
\\	は「ストレスがたまって疲れている」の意。	
\\	この生地を裁つ時「表と裏が逆」と指摘されたのに「上下逆」と勘違いしてしまった。 
\\	「表と裏が逆」を
\\	で、「上下逆」を
\\	で表現しています。	
\\	表向きには何もかわらない。内部で組織がかわるだけ。(会社の組織替え等) 
\\	で「表面は」の意味。
\\	「内部の」。	
\\	あなたがアメリカに行きたいって気持ちと、私が行きたいって気持ちでは重みが全然違う。 
\\	は「深さ」、
\\	は「重大さ」の意味です。	
\\	国会に喚問された参考人には、さまざまな思惑があったのではないかと思う。 
\\	ここでは「思惑がある」を 
\\	(秘めた動機)で表現しています。
\\	で「思惑がある」の意。	
\\	わっ!くさ!思わず息とめちゃったよ 
\\	ここでは「思わず」を
\\	「何が起きてるか分かる前に」と言い換えています。
\\	「悪臭」。	
\\	彼は私と付き合うつもりもないのに、思わせぶりなことばかり言う。 
\\	「思わせぶり」を
\\	「…だと思わせようとする」、
\\	「…だと思わせるふりをする」で表現しています。	
\\	きみの言うことが正しいと思わないことはない。 
\\	と 
\\	(それ自体が悪い)を組み合わせて「正しいと思わないことはない」のニュアンスを出しています。
\\	は「まったく」の意。	
\\	きっと自分はいい子だと思われたいんだよ。 
\\	で「みんなに…だと思ってほしい」の意味になります。	
\\	英語で手紙を書くときは、考える時間もあるし参考書も参照できる。なので、(手紙を受け取った相手からは)かなり英語が話せるかのように思われるのよね。 
\\	ここでは「思われる」を
\\	のように見える」、
\\	という印象を与える」で表現しています。	
\\	私は自由を重んじる人間です。ですから束縛されることは私にとって拷問に等しい仕打ちなのです。 
\\	「重んじる」とは
\\	(…にとても高い価値を置く)ということですね。
\\	は「…に等しい」、
\\	は「危うくする、危険にさらす」の意味。	
\\	君は御役御免だよ。(その仕事から開放する) 
\\	「御役御免」は
\\	「解放する」と
\\	「義務」を組み合わせて表現します。	
\\	そのバラエティ番組では、司会者がゲストの名前をわざと間違えるのが「お約束」となっていた。 
\\	「お約束」とは
\\	「暗黙の了解」ということですね。
\\	「決まっていること」。	
\\	お役に立てて光栄です。 
\\	で「役に立つ」という意味になります。
\\	「光栄に思う」。	
\\	親孝行な息子さんがいるから、老後の心配しなくていいわね。 
\\	「親孝行な息子」を
\\	「誠実な息子」で表現してみました。
\\	「衰える」。	
\\	親不知を抜いてもらわなきゃ。 
\\	「親不知」は
\\	と言います。「抜いてもらう」は
\\	を使って表現します。	
\\	「おやじ狩り」とは、若者が夜・帰宅途中の年配者を狙い窃盗することである。 
\\	「おやじ狩り」を
\\	と直訳し、その後で具体的に説明しています。投稿された日本語をそのまま英語に訳してみました。	
\\	そんなおやじギャグ言ってたら、若い女の子に相手にされないよ。 
\\	ここでは「おやじ」を
\\	「40/50代かそこらの」と表現しています。
\\	は「見向きもしない」の意。	
\\	あいつは子どもが生まれてから、すごい親ばかになったよね。 
\\	ここでは
\\	「溺愛する」を使って「親ばか」のニュアンスを出しています。	
\\	引っ越しの手続きがこんなに面倒だとは思わなかった。おかしくなりそうだ。今までは全部親任せだったから気づかなかったけど。 
\\	で「…すべてを親に任せる」の意味。
\\	と 
\\	を組み合わせても同じようなことが言えます。	
\\	親指はパパ。人差し指はママ。中指は兄さん。薬指は姉さん。小指は赤ちゃん。 
\\	「親指」は
\\	ですね。日本語をそのまま英語にしてみました。	
\\	(採血で)親指を中にしてぎゅっと握ってください。はい、ちょっと、チクッとしますよ。 
\\	で「拳をギュッと握る」の意。この場合、敢えて「親指を中に入れる」と言う必要はありません。もし、言いたいのなら、
\\	を付け加えます。「チクッとしますよ」の2通りの言い方にも注意してください。	
\\	シャワーの水温設定を高温にしているんですけど、お湯が出ないんですが。 
\\	「出る」は 
\\	で表現します。使役の 
\\	の使い方にも注目。	
\\	折り入ってご相談があるんですが。 
\\	とは「大きな頼み事」の意。「折り入って」は
\\	でそのニュアンスが出せるでしょう。	
\\	私の場合滑るというより降りてくるって感じだね(スノボーやスキー)かなりゲレンデの邪魔者だよ。 
\\	で「降りてくる」の意味ですが、その後に、
\\	「何度もころぶ」を付け足して、より具体的に説明しています。
\\	「混乱させる」、
\\	「ピューッと降りてくる」。	
\\	あいつには、いろいろ世話になったからお礼参りに行かなくちゃ。 
\\	「彼のところにちょっと寄る」の意味ですが、
\\	の部分から、これが「訪問する」だけではないことが伝わります。
\\	「こらしめる」の意。	
\\	8月も終わりに近づくと、クラゲが増えて海水浴ができなくなる。 
\\	で
\\	の終わりが近づく」の意味になります。
\\	「近づく」。
\\	「クラゲ」。	
\\	ここのところ毎日宿題に追われる毎日で、睡眠不足です。 
\\	「宿題に追われる」を 
\\	(やらなくてはいけない宿題がたくさんある)で表現しています。
\\	は「雪だるま式に増える」の意。	
\\	彼って恩着せがましい偽善者なんじゃない? 
\\	「恩着せがましい」は
\\	「相手を見下すような」で表現できます。
\\	「偽善者」。	
\\	そのホールは音響効果抜群で、臨場感に溢れていた。 
\\	「音響効果」は
\\	、あるいは、
\\	で表現します。
\\	を使った「臨場感」の表現の仕方にも注目。	
\\	なんて恩知らずなやつなんだ。あいつとはもう口も聞きたくない。 
\\	が「恩知らず(の人)」のこと。
\\	(恩知らずな)。	
\\	音痴だから、恥ずかしくて人前で歌を歌うなんてとんでもない。 
\\	「音痴」は
\\	と言います。
\\	「恥ずかしくなる」。	
\\	私は音痴だから歌うのが恥ずかしいの。 
\\	が「音痴」に当たる表現です。	
\\	たまに車の中とかで歌を口ずさんでいて、自分の音痴さに驚くことがある。 
\\	「音痴な」は
\\	、「さ」は
\\	で表現できます。
\\	「鼻歌を歌う」。	
\\	家にいるときでも、もう少しおしゃれをしなければ女がすたる。 
\\	は「女性の魅力がなくなる」、
\\	「下り坂になる」の意味です。	
\\	女心がまったく理解できない人とは、早く別れたほうがいいわよ。あとで後悔することになるから。 
\\	「女心」はそのまま 
\\	で表現すればいいでしょう。
\\	と 
\\	を組み合わせてもいいですね。	
\\	それ、うちの(部署の)女の子に取りに行かせるから。 
\\	「うちの女の子」に当たる表現は
\\	でしょうか。
\\	と言えないこともありませんが、この場合、
\\	を使うのが自然でしょう。	
\\	(彼女が)彼の話を始めると、彼女って女の子らしい声と表情になるの。 
\\	「女の子らしい」は
\\	で表現できますね。	
\\	ふられるなんて悔しい!女をあげてあいつを見返してやる!! 
\\	ここでは「女を上げる」を
\\	「いい(本当の)女になる」と言い換えて表現しています。
\\	「捨てる」。	
\\	今回のことは恩に着ます。 
\\	は「(恩などを)負う」、
\\	は
\\	に恩がある」の意味です。	
\\	彼は完全に彼女におんぶにだっこだ。 
\\	「おんぶにだっこ」とは
\\	「…に完全に頼る」ということですね。	
\\	どうして私ばっかりこんなに蚊に刺されるの?きっと私の血がおいしいのかな? 
\\	「蚊に刺される」を 
\\	で表現しています。	
\\	あんなに面倒を見てきたのに、裏切るなんて。飼い犬に手を噛まれるとはこのことだよ。 
\\	ここでは
\\	「とても裏切られた気持ちになる」で「飼い犬に噛まれる」のニュアンスを表現しています。	
\\	電子レンジの調子がおかしい。修理するより買い換えるほうがいいかな。 
\\	「買い換える」は
\\	を使えばいいでしょう。
\\	「調子が悪い」。	
\\	私にそんな難しい仕事ができると思ってるなんて、社長は私のことを買いかぶってるよ。 
\\	「買いかぶる」は 
\\	(過大に評価する)で表現すればいいでしょう。
\\	は「…を高く評価し過ぎる」の意。	
\\	この間投稿した記事がやっと雑誌に載ったんだよ。頑張って出し続けた甲斐があったよ。 
\\	ここでは「甲斐がある」を
\\	「よい結果に終わる、うまくいく」で表現しています。	
\\	"日本人が決勝に進むのは、初の快挙って競技が今回は目立った。 
\\	で「初の快挙」を表現しています。この
\\	とは
\\	""という意味です。
\\	皆勤賞に近いぐらい出席したのに、なんで評価が悪いんだろう。 
\\	で「ほとんど授業を休んでいない」の意味。	
\\	議長、副議長を1人ずつ、書記、会計を2人ずつ選ぶ。 
\\	「会計」は
\\	と言います。	
\\	来年から日本では介護保険制度が始まるね。 
\\	「介護保険」は
\\	、あるいは、
\\	でいいでしょう。	
\\	年に1度のサミット、今年はどこで開催されるのだろう。 
\\	「開催する」は 
\\	を使って表現します。
\\	は「例年の、年1回の」の意。	
\\	彼はコチコチの会社人間だから。 
\\	とは「頑固な(保守的な)」、
\\	は「根っからの会社人間」の意味になります。他に、
\\	あるいは
\\	「完全な(生粋の)会社人間だ」という言い方もあります。	
\\	あの日立さんとNECさんが手を組むなんて、時代が変わった。 
\\	会社名に「さん」をつけるような習わしはありませんので、ここではそのまま
\\	を使っています。
\\	「提携する」の意。	
\\	夏ばて解消に、ウナギのカバ焼きでも食べるか。 
\\	この場合の 
\\	は「払いのける」の意味になります。
\\	は「けだるさ、不快感」の意。	
\\	非通知設定のままになってる。解除しなきゃ。 
\\	ここでは「解除する」を 
\\	で表現しています。
\\	は「発信者番号通知サービス」の意。	
\\	毎回切符買っているの?もったいなーい。回数券買えば、10枚分の料金で11枚買えるんだよ。 
\\	「回数券」は
\\	と言います。また、カード式になっているものは
\\	と呼びます。	
\\	そんな事をしても汚染の問題は改善されないと思うけど。 
\\	で「よくなる」の意味になります。
\\	は「解決する」の意の動詞。	
\\	何があったのか、かいつまんで話してくれよ。 
\\	「かいつまむ」とは
\\	「要約する」ということですね。
\\	「概要」。	
\\	10ドルって書いてあったけど、レジに行ったら5ドルで買えちゃった。 
\\	で「(ラベルに)…と書いてあった」の意味。
\\	は「(印、価格などを)つける」の意。	
\\	この不景気で、商売も開店休業状態だ。 
\\	と 
\\	(客は一人も見えない)を組み合わせて「開店休業」を表現しています。	
\\	昨日回転寿司へ食事に行った。 
\\	「回転する」を使って、
\\	と言えばいいでしょう。
\\	は「流れて行く」、
\\	は「コンベヤー」のこと。	
\\	パンを解凍してあるから後で焼いて食べてね。じゃ、私は出かけるね。 
\\	で「解凍する」、
\\	で
\\	を解凍しておいた」の意味。	
\\	お寿司食べたいなと思っても10回のうち8回は我慢しているんだから、私が食べたいって言ったときは駄目って言わないで。 
\\	で「10回のうち8回は」の意味になります。	
\\	彼は目の手術を受けて視力が回復した。 
\\	が「回復する、改善する」に当たる動詞です。
\\	は「(外科)手術」の意。	
\\	言葉ではうまく表現できないけど、日本とアメリカの文化や生活スタイルの違いを垣間見たきがした。 
\\	「垣間見る」とは
\\	「ちらっと見る」、
\\	「のぞき見る」ということですね。	
\\	ミサイル攻撃により、敵陣は壊滅状態となった。 
\\	(完全に破壊される)で「壊滅状態になる」を表現しています。
\\	は「徹底的に破壊する」の意。	
\\	こんないらないものまで買って。お前は買い物中毒なんじゃないの。 
\\	で「買い物中毒である」の意味。	
\\	以前受けた英語のクラスは、今のクラスよりレベルが低かったけど、皆もっと英語で会話してたけどなぁ。今のクラスのメンバーって皆シャイなのかな。 
\\	「会話」は
\\	ですね。ここでは
\\	(活気がある)と
\\	の使い方にも注目。	
\\	安いと思って買ったのに、腐っていたのでかえって損をした。 
\\	「かえって」を
\\	「結局は」、
\\	「(自分のしたことが)自分に返ってくる」を使って表現しています。	
\\	まだ時期が早かったみたいで、全て雪景色でした。でも、かえってまたいい時期に行こうという気になりましたけどね。 
\\	「かえって…したくなる」を 
\\	(…しようという気持ちになる)で表現しています。
\\	の使い方にも注目。	
\\	お!久しぶり。仕事の帰り? 
\\	が「仕事帰り」に当たる表現です。
\\	は「久しぶり」の意。	
\\	帰る前に必ず「帰るコール」を入れてね。 
\\	「帰るコール」は、そのまま
\\	と言えばいいでしょう。	
\\	彼はこのあたり/店では顔がきくんだ。 
\\	「顔がきく」とは
\\	「常連」、
\\	「皆が知っている」ということですね。	
\\	それじゃ、彼の顔が立たない。 
\\	「顔を立てる」に当たる英語表現は
\\	「面目を保つ」です。	
\\	彼はこの業界では顔が広い。 
\\	「顔が広い」とは 
\\	(たくさんの人を知っている)ということですね。	
\\	歩いていて人の吸っているタバコの煙が顔にかかるのがとっても嫌なの。 
\\	この場合の「顔にかかる」は
\\	でいいでしょう。	
\\	自分でもわかっているんだけど、嫌なことがあると顔に出ちゃうんだよね。 
\\	「顔に出る」とは
\\	「顔を見れば分かる」ということですね。	
\\	こんなこともわからんのか。顔を洗って出直してこい。 
\\	とは「最初からやり直せ」の意味になります。
\\	(顔に水をかけて、ゼロからやり直せ)	
\\	ちょっと顔を貸してもらえないかな。知らせておきたいことがあるんだ。 
\\	「顔を貸してもらう」とは
\\	(…とちょっと話す)ということですね。	
\\	時間があったら、今夜の飲み会に顔を出してよ。 
\\	で「…に顔を出す」の意味。
\\	と
\\	の間に
\\	が略されています。
\\	は「顔を見せる」の意。	
\\	外食産業の価格戦争はもうしばらく続きそうだ。 
\\	「価格戦争」はそのまま 
\\	と英訳すればいいでしょう。	
\\	彼女は家事をサボって、かかさずこのテレビを見ています。 
\\	は「…を見逃すとは夢にも思わない」という意味で、「かかさず」のニュアンスを表しています。	
\\	日本の芸能ニュースは欠かさず、インターネットでチェックしているので日本人より詳しいかもしれません。 
\\	(こつこつと)、
\\	(定期的に)で「欠かさず」を表現しています。	
\\	あんなにかかとの高い靴はいてたら、すぐに捻挫しちゃうよな。 
\\	「あんなにかかとの高い靴」は
\\	で表現できますね。
\\	の使い方がポイントです。	
\\	英語が上達するまでには、もう少し時間がかかりそうです。 
\\	で「もう少し時間がかかる」の意味になります。	
\\	そのアイスクリーム上に掛かってるの何?チョコレート? 
\\	は「振り掛ける」(動詞)、「粒状のもの」(名詞)として使うことができます。	
\\	インターネットにつなげなくても、毎月基本料金はかかってくるから、使ったほうがいいよね。 
\\	ここでは「基本料金はかかってくる」を 
\\	(基本料金を払わなくてならない)で表現しています。
\\	は「…した方がいい」の意。	
\\	滑れないにもかかわらずリフトで上に上がってしまって大変な目にあった。 
\\	「かかわるず」は
\\	で表現できます。
\\	「とっても困る」。	
\\	このボールペン、高かった割には書き味が悪いんですけど。 
\\	「書き味が悪い」は
\\	で表現します。	
\\	一方的に電話を切られたので、僕の声は,受話器の信号音にかき消された。 
\\	は「(音を)消す」、
\\	は「次第に消えていく」の意味。	
\\	そろそろかき氷の季節だね。 
\\	「かき氷」は
\\	と言います。お店の看板には
\\	と書いてあることが多いようです。	
\\	これを書留でお願いします。 
\\	が「書留」にあたる表現です。	
\\	まったく休んだときに限って、(仕事の)トラブル起きるんだもん。まったく嫌になるよ。 
\\	「ときに限って」を
\\	で表現しています。
\\	は「問題が持ち上がる」の意。	
\\	のスコアが上がったからと言って、スピーキング力もそれに伴ってあがるとは限らない。 
\\	「限らない」は
\\	「必ずしも…ではない」で表現します。	
\\	英語の運用能力がつかない限り、会社での出世は望めない。 
\\	この場合の「限り」は
\\	で表現できます。
\\	は「出世する、成功する」の意。	
\\	今持っているカードは自分でハサミを入れない限り永遠に使えるのかな。落としたら危ないね。 
\\	この場合の「…限り」は 
\\	で表現できます。
\\	の使い方にも注目です。	
\\	蚊に刺されたところをかいちゃだめだよ。はれちゃうから。 
\\	「掻く」は
\\	を使います。ここでは
\\	「はれる」も一緒に覚えておきましょう。	
\\	せめて英語が話せるようになるまでは、帰ってこない覚悟で行きます。 
\\	(…すると決心している)で「覚悟している」を表現しています。	
\\	なんかあやしいなー。隠し事でもしてるの!? 
\\	「…に隠し事をする」は 
\\	で表現します。
\\	は「あやしい行動をとる」の意。	
\\	交渉が暗礁に乗り上げても、当方には隠し球があるから、心配しなくても良い。 
\\	で「いざという時のための巧みな策がある」の意味になります。
\\	は「座礁する」、
\\	は「失敗する」の意。	
\\	私は、ギャンブルには向いてないと思う。そういう運がないというか、働いて稼いだ方が確実だし。 
\\	で「より確実なこと」の意味。
\\	に向いている」。	
\\	あの写真よりも、確実に5才は年をとっているのだから。 
\\	「年をとる」をここでは
\\	を動詞として用いて表現しています。	
\\	ファイルが開けない。拡張子つけてメール送って。 
\\	「拡張子」のことは
\\	と言います。	
\\	この試験ではどれぐらいの確率で合格するのか全然情報がない。運らしいよ。 
\\	「確率」は
\\	「見込み」、
\\	「割合、率」を使えばいいでしょう。	
\\	よっくんのおとうさんね、幼稚園の若い先生とかけおちしたらしいわよ。 
\\	と駆け落ちする」は
\\	、あるいは、
\\	を使って表現します。	
\\	彼女は、僕にとって掛け替えのない人だ。 
\\	とは「取り替えることができない」の意味。
\\	「代わる」。	
\\	あなたにはわからないかもしれないけど、あなたは私にとってかけがえのない存在なのよ。 
\\	「かけがいのない存在」を
\\	「誰もあなたの代わりにはなれない」と言い換えています。
\\	「取り替えることのできない」。	
\\	「今月の売り上げすごいじゃないか」。「でも、消費税がアップする前の駆け込み需要が相当ありますからね」。 
\\	「駆け込み需要」をここでは
\\	「急いで買う」で表現しています。	
\\	駆け込み乗車は危険だよ。 
\\	は「電車に乗るために走る」の意味。
\\	「争って電車のドアに殺到する」。	
\\	この
\\	は、英語を勉強する人にとって、 駆け込み寺みたいだよね。 
\\	この場合の 
\\	は「苦境などから救ってくれるもの(場所)」という意味です。	
\\	その教会は、逃げ込んでくる女性を救う、まさに現代の駆け込み寺だ。 
\\	「駆け込み寺」は
\\	でいいでしょう。
\\	は「避難所」の意。	
\\	この種の保険は掛け捨てが多い。 
\\	で「満期の際の返金がない」、
\\	「払い戻しが含まれていない」で「掛け捨て」を表現しています。	
\\	このボールペン、まだインクがあるのに書けないじゃない。 
\\	ここでは自動詞
\\	の使い方に注目。	
\\	現実からあまりにもかけ離れたお世辞は、逆に失礼だ。 
\\	で「現実からかけ離れている」の意味になります。
\\	は「逆の結果を生む」、
\\	は「突飛な」の意味。	
\\	彼は、食費を削ってまで携帯電話をかけまくっている。 
\\	ここでは「かけまくる」を
\\	(使い過ぎる)を使って表現しています。
\\	は「電話をたくさんかける」の意味。	
\\	この間、銃声のような音がしたので思わず窓に駆け寄って見たけど、どうも爆竹だったようだ。 
\\	「窓に駆け寄る」は 
\\	でいいでしょう。「思わず」の表現の仕方にも注目です。	
\\	彼が明日時間通りに来るかどうかジュース一本賭けない? 
\\	「賭ける」は
\\	を使います。
\\	とは
\\	果汁」のことです。念のため。	
\\	お腹出して。ちゃんと布団もかけて寝なさい。風邪ひくよ。 
\\	「(布団を)かける」は
\\	という言い方をしまうす。
\\	「お腹を出して」にも注目。	
\\	親たちは、子供たちを良い学校に行かせようと塾に行かせたり、教育にかなりお金をかけています。 
\\	「(お金を)かける」は
\\	でいいでしょう。
\\	「塾」。	
\\	あのコマーシャルって彼の名前と会社名をかけているよね。結構、そういうCMって多い。 
\\	この場合の「かける」は
\\	「彼の名前を使って遊ぶ(語呂合わせなど)」ということですね。	
\\	陰でこそこそ、意地悪をするなよな。 
\\	「陰でこそこそ」は
\\	を使えばいいでしょう。この場合の
\\	は「下劣な、卑しい」の意味になります。	
\\	栄華を極めたさすがの
\\	産業にも、陰りが見え始めた。 
\\	(嵐に直面しそうである)で「陰りが見え始めた」を表現しています。
\\	は「栄華の頂点」の意。	
\\	新しいシステムで既存顧客の囲い込みを行う。 
\\	で「縛り付ける」、
\\	で「囲い込む」の意味になります。	
\\	生のままじゃなくてちょっと加工されたお寿司は食べれるようになった。ネギトロみたいなのとか。 
\\	が「加工した」という意味になります。
\\	「刻んだ」。	
\\	彼とのことは、もう過去の事なの。 
\\	「過去の事」はそのまま
\\	が使えます。
\\	(歴史)でも同じようなことが言えます。	
\\	世の中お金が一番と言っても過言ではないと思う。でも、考えると空しいから、世の中お金でないと自分に言い聞かせている。 
\\	が「過言」に当たる言い回しです。
\\	は「…と言っても言い過ぎではないと思う」の意。	
\\	ちょうど学生の試験の週と重なったようで、渋滞に巻き込まれる事もなく、スキー場も空いてました。 
\\	「重なる」を
\\	「ぶつかる」を使えば表現できますね。
\\	「込み合っていない」。	
\\	あいつは親父の威光を笠に着て、威張りちらしている嫌なやつだ。 
\\	(地位を悪用する)、
\\	(権威を利用する)で「笠に着る」のニュアンスを表現しています。
\\	は「思い上がった」、
\\	は「ばか者」、
\\	は「役人、経営幹部」の意。	
\\	そんなに重ね着して、きみは寒がりなんだね。 
\\	が「重ね着する」に当たる表現です。
\\	は「寒さに敏感である」の意。	
\\	箱のまま持ってくると嵩張るから、袋に入れ替えて持ってきたんだ。 
\\	で「大きすぎて、扱いにくい」の意味。
\\	「嵩張る」。	
\\	そのかさぶた取ったらだめだよ。傷跡が残っちゃうから。 
\\	「かさぶた」は 
\\	と言います。
\\	は「傷跡」の意。	
\\	ちょいとここらで河岸をかえましょうや。 
\\	「河岸を変える」とは
\\	「ここを出る(別の場所へ行く)」、
\\	「どこか他へ行きましょう」ということですね。	
\\	海外で売られている
\\	には何故歌詞カードがついてないのだろうか。こちらの人はあまり気にしないのかな 
\\	は「印刷された歌詞」の意味。日本の歌詞カードを具体的に説明すると 
\\	となります。	
\\	最近、銀行の貸ししぶりで倒産する会社が増えている。 
\\	で「(貸す)お金に関して引き締めている」の意味。また、
\\	「銀行がお金を貸すのを嫌がっていること」でも表現できる。	
\\	お互い過失を認めたのですが、相手はけがをして私は平気だったので、向こうは働けない間の賃金の半分を請求してきた。 
\\	が「過失を認める」に当たる表現です。
\\	の代わりに 
\\	(責任)を使ってもいいでしょう。	
\\	ちょっと、それ貸して! 
\\	「貸して」は
\\	を使って表現します。	
\\	3箇所に輪ゴムを掛けてください。 
\\	で「3箇所」の意味になります。
\\	「輪ゴム」。	
\\	彼女はストレスから過食症になり、1カ月で10キロも太ってしまった。 
\\	「過食症」には
\\	という言い方がありますが、ここでは
\\	(何かに取り憑かれたように暴食する)を使って表現しています。	
\\	なんか彼のヘアスタイルって中途半端だな~。自分では満足しているのかしら。 
\\	が「かしら」に当たる表現です。
\\	「中途半端な)	
\\	少しは家事の手伝をしてくれてもいいんじゃないの。 
\\	とは「家庭内の雑用」のことです。	
\\	うー、寒い。手がかじかんじゃってるよ。 
\\	が「かじかむ」に当たる表現です。	
\\	彼は火事場のくそ力でその困難と思われた仕事も乗り切った。 
\\	「火事場のくそ力」を 
\\	(緊急時に出る信じられない力)で表現しています。
\\	は「奮い起こす」、
\\	は「超人的力」の意。	
\\	そんなことをやり遂げるのはとても無理だと思っていたが、いざとなると火事場のばか力が出て、やり遂げられた。 
\\	(隠れた力)で「火事場のばか力」を表現しています。
\\	は「いよいよとなった時、必要となった時」の意味。	
\\	の経験があると言ってもほんのかじった程度です。ビジネスに活かせるどうかはわかりません。 
\\	ここでは「かじる」を
\\	「テキストを使ってちょっとやっただけ」で表現しています。
\\	「ちょっとテキストは開く」。	
\\	ちょっと、お菓子のかすボロボロこぼしてるよ。 
\\	とは「クッキーのくず」のことです。
\\	は「まき散らす」の意。	
\\	プリンターのインクがかすれてきた。新しいのと交換しなくては。 
\\	この場合の 
\\	は「薄い」、
\\	は「はっきりしない、かすかに」の意味。	
\\	ひまわり2本で500円かあ。2本じゃ寂しいし、4本じゃ数が悪いしなあ。 
\\	この場合の「悪い」は
\\	や
\\	「不運な、運がよくない」で表現すればいいでしょう。	
\\	今日は、風が冷たいね~。風さえなければ、天気がいい時は冬でもそんなに寒くないのに。 
\\	で「今日は冷たい風が吹いている」の意味になります。
\\	は「冷たい」の意。	
\\	彼のちょっとした失言のせいで、計画全体がだめになったなんて、風が吹くと桶屋がもうかるというようなものだ。 
\\	ここでは「風が吹くと桶屋がもうかる」を
\\	(それがきっかけとなり関連したことが続けて起こり)と
\\	(結果として…になる)を組み合わせて表現してみました。
\\	は「結果」の意。	
\\	飲みに行くのはいいけど、風邪気味なんだから早く帰ってきなよ。 
\\	「風邪気味」の「気味」は 
\\	で表現できます。	
\\	夜になってライトがつくと虫が集まってきて、そこに扇風機がおいてあってその風に吹かれて虫が飛んできてすごい嫌だ。 
\\	「吹かれる」は 
\\	を受け身にして表現します。
\\	で「風に吹かれて自分のほうに飛んでくる」の意味。	
\\	あなたから連絡くれるなんて、どういう風の吹きまわし? 
\\	「どういう風の吹きまわし」は 
\\	(何があったのか)で表現すればいいでしょう。	
\\	バイクが好きなのは、運転してると風を感じるからかな。 
\\	この場合の
\\	は名詞で「感触」という意味になります。
\\	「(そよ)風を感じる」。	
\\	私の家族は父、母、弟です。 
\\	と 
\\	で家族構成を説明しています。	
\\	昨日の競馬のメインレースは、とってもかたくおさまったよ。ああいうのを、銀行レースって言うんだよね。 
\\	この場合の「かたい」は
\\	「驚かない(こと)」ということですね。
\\	の使い方にも注目。	
\\	堅い話は抜きにして、今夜は楽しく飲みましょう。 
\\	この場合の
\\	は「微妙な、注意する必要がある」の意味。
\\	は「難しい」の意。	
\\	私の片思いなの。彼は、私のことなんて全然知らないと思うわ。 
\\	と
\\	は共に「一方的な」という意味です。	
\\	日本人の発音はカタカナ英語になりがちだ。 
\\	ここでは「日本人が英語を話すと、カタカナの影響を強く受ける
\\	と表現しています。	
\\	一日中パソコンに向かっていると肩がこってしょうがないなあ。 
\\	「肩こり」は
\\	といいます。これと動詞の
\\	を組み合わせれば「肩がこる」の意味になります。一緒に
\\	も覚えておくといいでしょう	
\\	あの蚊に刺されたのね。よし、私がこの殺虫剤で敵を取ってやるわ。 
\\	で「…に仕返しをする」の意味。
\\	は「仕返し、報復」の意。	
\\	彼女は、カメラの前では緊張して硬くなってしまった。 
\\	「緊張する」は
\\	、「堅くなる」は
\\	で表現します。	
\\	まあいいじゃないの。1年に1回のお祭りの日なんだからそんな堅苦しいこと言わず、ぱっといこうよ。 
\\	で「堅苦しく形式張った」の意味になります。
\\	は「堅苦しい」の意。	
\\	肩こりがひどくなってきた。 
\\	「肩こり」は
\\	と言います。
\\	は「かたくなる」の意。	
\\	浴衣って着た後の片づけが面倒なんだよね。タンスに入れる前に干して乾かしたり。 
\\	で「やらなくてはいけない後かたづけ」の意味になります。	
\\	片手間にやっているっと結局両方中途半端で終わってしまう。 
\\	は「内職する、アルバイトする」という意味です。
\\	「片手間に」。	
\\	肩で風を切って歩いていた頃が懐かしいよ。 
\\	は「かっこつけて(気取って)歩く」という意味です。
\\	を使った「懐かしい」という言い回しにも注目。	
\\	お母さんて、私が幼稚園の頃、よく野菜なんかを型で抜いてお弁当を動物の顔みたいにしてたよね。 
\\	ここでは「型で抜く」を
\\	の後に
\\	「クッキーの型抜きのようなもの」を付け加えて表現しています。	
\\	そんなに緊張しないで。肩の力を抜いて、リラックスしなさい。 
\\	が「肩の力を抜く」に当たる言い回しです。	
\\	末の娘を嫁に出し、これで肩の荷が下りました。これからは主人と2人で人生を楽しもうと思います。 
\\	が「肩の荷が下りる」に当たる言い回しです。	
\\	私の父は曲がったことの嫌いな、堅物でした。 
\\	は「堅苦しい、厳格な」という意味の形容詞です。
\\	「曲がった」、
\\	「不正行為」。	
\\	あ!失敗だ。全然ボールがバラバラにならなかった。ほとんど固まってる。(ビリヤードの最初の一突きで) 
\\	で「かたまっている」の意味。
\\	は「しくじる」の意。	
\\	あ、またコンピュータが固まっちゃったよ。 
\\	「固まる」は
\\	でも表現できますが、
\\	を使うことが多いようですね。	
\\	お!かき混ぜが足りなかった。底の方がまだかたまっていた。 
\\	粉末スープをかき回す、という場面を想定して英訳してみました。
\\	で「かたまっている」の意味になります。
\\	「かたまり(名詞)」。	
\\	これ父の形見の時計なんだ。 
\\	が「形見」に当たる語です。この文脈での
\\	の使い方も覚えておくといいでしょう。	
\\	あ!ちょっとカレンダー傾いている。もうちょっと右側を上に上げて。そうそう。 
\\	「傾く」は
\\	を使って表現します。
\\	「真っ直ぐにする」。	
\\	大手銀行の貸出先は、重厚長大企業に偏っていた。 
\\	は
\\	の方に曲がる」、
\\	は
\\	の方へ傾く」の意味。「重厚長大」を
\\	と
\\	を組み合わせて表現しています。	
\\	ハンバーガーばかり食べていると、栄養が偏るよ。 
\\	「偏る」とは
\\	「バランスがとれていない」ということですね。	
\\	なんかちゃんと栄養管理しないと栄養が偏って病気になりそうだ。もう若くないし。 
\\	「偏る」は 
\\	を使って表現すればいいでしょう。
\\	は「…になる」の意。	
\\	ビルはこの会社での語り草になっている。 
\\	「語り草」とは
\\	「伝説的人物」ということですね。	
\\	今回の契約が取れなかったからって、肩を落とすなよ。 
\\	で「意気消沈する」という意味。
\\	と 
\\	の組み合わせにも注意。	
\\	僕たちは肩を寄せながら歩いた。 
\\	「肩を寄せる」は 
\\	(お互いにくっついて)で表現すればいいでしょう。	
\\	価値観の違いが、大きな問題です。 
\\	「価値観」は
\\	(複数形)を使って表現します。	
\\	あの勝ち気な女には参ったよ。 
\\	「勝ち気」は
\\	で表現できます。	
\\	(麻雀で)おい、勝ち逃げするつもりかよ。 
\\	(立ち去る)と
\\	(勝ち越して)を組み合わせれば、「勝ち逃げする」が表現できます。	
\\	彼の言うことって、いちいちカチンとくるんだよね。 
\\	は「嫌な、不快な」の意味。
\\	「無味乾燥なことば」。	
\\	あそこの英会話学校、月謝制を導入してからなかなか活気付いてきたよ。 
\\	に生命を吹き込む」、
\\	「回復する」、
\\	「景気がよくなる」で「活気付く」を表現しています。	
\\	あの人、かっこよくない? 
\\	「かっこいい」は 
\\	で表現します。	
\\	先ずは、格好から入らないとね。 
\\	「格好から入る」は
\\	「そのための服装を揃える」で表現すればいいでしょう。	
\\	そんな所にそんな格好で寝ていると、首が痛くなるよ。ちゃんとベッドで寝なさい! 
\\	で「そんな変な姿勢(格好)で」の意味になります。
\\	は「首の筋を違える」の意味。	
\\	昨日こたつで変な格好で寝たらしく、体中が痛い。 
\\	「格好」は
\\	でいいでしょう。
\\	「痛い」。	
\\	国防副長官の発言は、大統領の対イラク強硬姿勢を裏付ける格好となった。 
\\	ここでは「格好となる」を 
\\	(…のように見える)で表現しています。
\\	は「強硬姿勢」の意。	
\\	単に人減らしをするだけでは、かっこ付きの「合理化」に過ぎず、本当の意味の合理化にはならない。 
\\	が「かっこ付きの、いわば」に当たる言い回しです。
\\	は「見え透いた」の意。	
\\	あいつは、女の子がいるとかっこつけたがるんだから。 
\\	とは「見せびらかす人、かっこつける人」という意味です。これは動詞
\\	としても使えます。	
\\	カナダで楽しい第二の生活を送ってくる!ってたんかきって日本出てきたのに1年で帰ったらかっこ悪いよ。 
\\	ここでは「かっこ悪い」を 
\\	で表現しています。
\\	は「公言する」の意。	
\\	道路や橋さえ作れば、街を活性化できるという考え方が昔あったんだろう。 
\\	は「新しい活力を与える」、
\\	は「活気づける」の意味。	
\\	君もこの手の仕事は初めてなんだろう?勝手がわかるまではマニュアル通りにやっておいたほうがいいよ。 
\\	「勝手が分かる」を、ここでは
\\	「もっと経験を積む」、
\\	「コツを掴む」で表現しています。
\\	「規則に従って」の意。	
\\	こう言っては勝手に聞こえるかもしれませんが、今まであまり自然保護とか考えたことはありません。 
\\	は「気にかけない、無頓着な」、
\\	は「手前勝手な」、
\\	は「自然保護」の意味。	
\\	勝手な思い込みで言葉を並べ、無意識にひどく人を傷つけた。 
\\	には「勝手な、恣意的な」という意味があります。
\\	は「出し抜けに言う」、
\\	は「苦痛を与える」の意。	
\\	ちょっと一人で勝手に先にやらないでよ。先生がこれは一緒にやって下さいって言ってたじゃない。(問題を解く際に一人が勝手に問題を解いてしまう場合) 
\\	この場合の「勝手に」は
\\	「自分で」で表現できます。
\\	することになっている」。	
\\	何とでも言えば。 後で 後悔しても知らないからね。勝手すれば。 
\\	「勝手にする」は
\\	「自分の思うようにする」と言い換えて表現すればいいでしょう。	
\\	行くか行くまいか、今心の中で葛藤してるんだ。 
\\	で「…について精神的に葛藤している」の意味になります。
\\	は「戦う」の意。	
\\	私の中の天使と悪魔が葛藤してるんだよね。食べてもいいよ。食べるなってさ。ちょっと天使がリードしてるけど。 
\\	ここでは「葛藤している」を 
\\	(私の中で戦っている)を使って表現しています。	
\\	昨日テレビで見たんだけど、30分ぐらい下を向いていると、腸の動きが活発になってトイレに行きたくなるらしいよ。 
\\	で「より活発になる」という意味になります。
\\	は「向上する」の意。	
\\	父がこちらの病院にかつぎこまれたと聞いて飛んできたのですが。部屋番号を教えて下さい。(受付で) 
\\	「担ぎ込む(込まれる)」は
\\	「急いで連れて行く」を使えば表現できます。	
\\	日本もかつては平和な国だと言われていたが、だんだん危険な国になりつつあると思う。 
\\	「かつて」とは
\\	「過去において」ということですね。
\\	「平和を愛する」。	
\\	来年も活躍してね。あなたの事、応援してます。 
\\	で「頑張ってね」の意味になります。
\\	を応援する」。
\\	「活躍する」。	
\\	彼は、自分の部下に喝を入れた 
\\	「気合いを入れる、檄をとばす」の意味で、
\\	がよく使われます。
\\	は「やる気を起こさせる」の意。	
\\	彼女はかなりひねくれている。 家庭環境のせいだと思う。 
\\	の後に
\\	「環境」を付け足してもいいですが、少し専門的な響きを持つようになります(例:カウンセラーのことば)。「ひねくれている」は
\\	で表現できます。	
\\	あなたが帰ってくるまで起きてようと思ったのだけど、眠たさには勝てなかった。 
\\	この場合の「勝てない」は
\\	「眠気を撃退する」ことができなかったとうことですね。	
\\	まだ長袖のスーツいくには、暑そうだけど、かといって、上着脱ぐと普段着っぽくなっちゃうし。 
\\	「かといって」は
\\	「そうは言っても」で表現できます。
\\	「カジュアルな、普段着の」。	
\\	あいつ、最近、角が取れて丸くなったね。 
\\	で「でこぼこが取れる」の意味になります。
\\	は「成熟した、分別のある」、
\\	は「柔らかくなる」の意。	
\\	明日、パーティーに参加しようかなぁ? 
\\	かな」は
\\	で表現できます。	
\\	この狸の置物は、家内安全、商売繁盛のお守りです。 
\\	が「家内安全」、
\\	が「商売繁盛」に当たる言い回しです。
\\	「狸」、
\\	「繁盛した」。	
\\	まさか、先生の授業が今日が最後だなんて、そんな悲しいこと言わないでくださいよ。 
\\	「悲しいこと」は
\\	で表現すればいいでしょう。	
\\	悲しいことに世の中、金がないと何にもできないのよね。 
\\	「悲しいことに」には
\\	という言い方があります。
\\	「不幸なことに」。	
\\	彼女は、会えば必ず自慢話ばっかりだから、皆から嫌われるんだよ。 
\\	ここでは 
\\	と 
\\	~を組み合わせて「必ず(~なしでは 
\\	できない)」を表現しています。
\\	と 
\\	を使ってもいいでしょう。	
\\	元々本読むのって苦手だから、一冊を読み終えるのにはかなりの根気が必要なんだよ。 
\\	「かなりの」を
\\	を使って表現しています。
\\	は「忍耐」の意。	
\\	そこらじゅう蚊に刺されてかゆくてしょうがない。 
\\	「蚊に刺される」は
\\	と
\\	の過去分詞形
\\	を組み合わせて表現します。	
\\	いま、どこのインターネットプロバイダーがいいのか考えているんだ。あと、いま加入すべきか、はたまた引っ越してから入ったほうがいいのかとかさ。 
\\	「加入する」は 
\\	で表現できます。この場合は 
\\	を使ってもいいでしょう。	
\\	うち衛星放送に加入しているから、24時間いつでもテレビ見れるんです。 
\\	「加入する」は
\\	「契約する」でいいでしょう。	
\\	予定してた飲み会がつぶれてお金が浮いた。 
\\	で「お金が使えるようになる
\\	という意味になります。	
\\	「金蔓」は
\\	「パトロン」、あるいは、
\\	「主な収入源」と言い換えて表現できます。	
\\	今度、合格のお祝いも兼ねて飲みに行きましょう 
\\	「兼ねて」は
\\	を使えば表現できます。	
\\	あいつ昔はいい奴だったんだけど、今じゃ金の亡者だよ。 
\\	で「お金にとりつかれている」という意味になります。	
\\	インターネットを使うとビジネスの可能性が広がるよね。 
\\	「可能性」をここでは
\\	「範囲」、
\\	「潜在能力、発展性」を用いて表現しています。	
\\	インターネットを有効に使用すれば、いろんな可能性が広がる。 
\\	「可能性」は
\\	「潜在能力」、あるいは、
\\	を使えばいいでしょう。	
\\	あの人をかばうことなんかない。あなたを裏切ったんだから。 
\\	「かばう」とは
\\	「保護する」ということですね。
\\	「裏切る」。	
\\	明日は花粉が多いそうだ。憂鬱だ。 
\\	「花粉」は
\\	(複数形の
\\	はつけない)と言います。	
\\	花粉症の人って、鼻がむずむずしてつらいんでしょ? 
\\	「花粉症」は
\\	と言います。
\\	「かゆい」、
\\	「かゆい、ちくちくする」、
\\	「鼻水が出る」。	
\\	日焼け止めのクリームを塗ったら、顔がかぶれて赤くなってしまった。 
\\	「かぶれ」は
\\	と言います。
\\	で「顔に赤い発疹が出る」。	
\\	話し声が壁に反響して、うるさいなこの部屋。 
\\	「反響する」をここでは
\\	から跳ね返ってくる」と言い換えています。もちろん
\\	「反響」も使えます。	
\\	アメリカに来てさっそく大きな壁にぶちあたったって感じがした。だって、インターネットはすぐに使えるものだと思ってたし。 
\\	この場合の「壁にぶちあたる」は 
\\	(大きなハードルに出くわす)、
\\	(大きな障害にぶつかる)で表現すればいいでしょう。	
\\	仕事が忙しいからって、最近私のことちっともかまってくれないんです。 
\\	「かまってくれない」とは
\\	「自分に対して時間を割いてくれない」ということですね。
\\	「忙しいと訴える」の意。	
\\	彼女のカマトトぶりは、うんざりする。 
\\	とは「気だてがよく、純真な振り」の意味ですね。	
\\	結婚してからというもの、あまり格好に構わなくなった。 
\\	「構わない」とは
\\	(…はそんなに心配しない)ということですね。
\\	は「それほど注意を払わない」の意。	
\\	鎌を掛けるのはめろよ。 
\\	「鎌を掛ける」をここでは
\\	「だまして... させる」、
\\	「ことばを巧みに使って相手をはめる」を使って表現しています。	
\\	ちょっと待って。私たちってなんだかかみ合ってないようだから、ちょっと整理してみよう。 
\\	自動詞
\\	には「気持ちが通じる」という意味があります。
\\	は「(歯車などが)かみ合う」の意。	
\\	お互いの髪が銀色に(白く)なっても、アイラブユーと言っていたい。 
\\	「髪が銀色に(白く)なるっても」は
\\	「年老いて、白髪になっても」で表現できます。	
\\	このコピー機しょっちゅう紙が詰まるんだから。新しいのに買い換えればいいのに。 
\\	「詰まる」は
\\	を使って表現できます。	
\\	髪が伸びたらから切りに行くか。 
\\	「髪が伸びる」は
\\	で表現します。	
\\	どんなに嫌な奴だって、お客様は神様なんだから。 
\\	「お客様は神様」をここでは
\\	「お客様はいつでも正しい」と言い換えています。
\\	「嫌な」、
\\	「付き合いにくい」。	
\\	神の国発言で森首相は窮地に立たされている。 
\\	「神の国発言」は
\\	でいいでしょう。	
\\	痛い!紙の端で手を切っちゃった。 
\\	「紙の端で手を切る」をここでは
\\	で表現しています。ちなみに、「端」は
\\	と言います。	
\\	その話を聞いて、やっぱり天才と紙一重だと思いました。 
\\	「紙一重」をここでは
\\	「微妙な(境界)線」を使って表現しています。	
\\	前髪は伸ばしているので、深く切らないでください。 
\\	「伸ばす」は 
\\	で表現します。
\\	は「前髪」の意。	
\\	あの人、面白いよ。するめみたいに噛めば噛むほど味が出るんだから。 
\\	「噛めば噛むほど」を
\\	と
\\	を使って「目ればみるほど味が出る(もっと見たくなる)」と言い換えています。
\\	「底を尽きない、無尽蔵の」。	
\\	出演者は素人のはずなのに、カメラ目線でインタビューに答えていた。 
\\	で「カメラを真っすぐ見て」の意味。
\\	も同じ意味。
\\	は「…だと言われている」の意。	
\\	こんな就職難だから、専門学校にでも行って資格を取った方がいいかもしれない。 
\\	ここでは「~いいかもしれない」を
\\	を使って表現しています。「就職難」の表現の仕方も覚えておくといいでしょう。	
\\	痒いところに手が届くとまではいかないが、あの旅館のもてなしはなかなかのものだよ。 
\\	で「改善するところは何もない」を使って「痒いところに手が届く」のニュアンスを表現しています。
\\	「完璧」。	
\\	そんなに自分の殻に閉じこもってばかりいるなよ。 
\\	が「殻」に当たる表現です。
\\	は「引っ込む」の意。	
\\	カラー印刷はインクがもったいないので、できるだけ白黒印刷してください。 
\\	「カラー印刷」は
\\	と言います。	
\\	悪いけど、これを三枚カラーコピーしてください。 
\\	「カラーコピー」はそのまま
\\	が使えます。	
\\	階段を降りているとき、空足を踏んで、危うく転びそうになった。 
\\	ここでは「空足を踏む」を
\\	(踏み外す)と
\\	(突き出す)を組み合わせて表現しています。	
\\	あの少年は,目の前を歩いている女性をからかっていた。 
\\	「からかう」は 
\\	で表現できます。	
\\	彼って何でもズバズバ言って、かなり辛口だね。 
\\	とは「はっきりしないことを言う」の意味。「辛口」は
\\	「どぎつい」、
\\	「毒舌」で表現できます。	
\\	彼って名前からして怪しいよね。 
\\	で「…から判断して」の意味。
\\	(彼の名前だけでも…)。	
\\	こう忙しくちゃ、体がいくつあっても足りないよ。 
\\	の使い方に注目。
\\	となっているのは仮定の話をしているからです。	
\\	やる気は十分あるのだけど、なんせこの年になると体がついてこないだよね、最近。 
\\	は「(最後まで)やり通す」という意味です。
\\	は「やる気」の意。	
\\	毎日のように残業してたら、体がもたないよ。 
\\	は「持続する、耐える」、
\\	は「すり減る」の意味になります。	
\\	最近彼、顔色が悪いよね。どこか体が悪いのかな? 
\\	「体が悪い」は
\\	(病気である)で表現すればいいでしょう。	
\\	喘息がひどくてずっと咳き込んでいたら、体中筋肉痛ですごくつらいです。 
\\	「体中」は 
\\	で表現します。
\\	は「喘息」、
\\	は「発作」の意。
\\	昨日久しぶりにテニスをしたから、体中がいたくて。 
\\	「体中が痛む」は
\\	を動詞として使って、
\\	、また、名詞にして、
\\	のように言います。「久しぶりに」の言い方も一緒に覚えておくといいでしょう。	
\\	そんなに仕事ばかりしていると体に毒だよ。 
\\	「体に毒だ」は
\\	でもいいのですが、ここは
\\	「害する、毒を盛る」を使って表現してみました。	
\\	昔のような身体の切れを取り戻したい。 
\\	「体が動ける状態」と
\\	「敏捷性」で「体の切れ」を表現しています。
\\	「勢い」、
\\	「機敏」。	
\\	最近ちょっと体を動かしただけで、筋肉痛になるのよね。 
\\	「体を動かす(こと)」とは
\\	「運動」するということですね。
\\	「痛い」。	
\\	そんなに仕事をすると、体壊しちゃうよ。 
\\	は「働き過ぎで病気になる」、
\\	「へとへとになるまで働く」の意味です。「過労死(する)」は
\\	と言います。	
\\	あの人にあえなくなって心がからぽになった様な気がする。 
\\	「からっぽ」はそのまま
\\	が使えます。
\\	「空、うつろ」。	
\\	私が英語を勉強しているからといって、皆私に英語に関しては聞いてくるけど、別に私もまだ英語は完璧ではないのだけどなぁ。 
\\	「…だからといって」は 
\\	を使って表現できます。
\\	は「決して…でない」の意。	
\\	彼は辛党で、キムチが大好物なんだ。 
\\	「辛党」とは
\\	「酒好き」ということですね。「辛い食べ物が好き」ならば、
\\	で表現すればいいでしょう。	
\\	これ空箱だけど持って帰って何かに使う? 
\\	「空き箱」は
\\	を使って表現します。(注)投稿文の「後、これは裏紙にでもどう?」は省略しました。	
\\	床にワックスを掛ける前に、乾拭きをしてね。 
\\	で
\\	を拭いて乾かす」の意味。	
\\	そのスーパーはお客の数を増やそうとバーゲンセールをしたが空振りに終わった。 
\\	「空振り」とは
\\	(無駄に終わる)、
\\	(失敗)ということですね。	
\\	昨日、新宿で酔っぱらいに絡まれた。 
\\	「口論(喧嘩)に巻き込まれる」の意味。	
\\	彼女って、一生懸命なんだけど、どこか空回りしてるのよね。 
\\	「空回り」は
\\	「努力を無駄にする」で表現すればいいでしょう。	
\\	もう、そんなにからまないでよ。おねがいっ! 
\\	「からむ」は
\\	「刺激する」、あるいは、
\\	「喧嘩を売る」で表現できます。	
\\	カリスマ美容師って言われてるから似合うよにお任せしますって言ったら、こんな頭にされたんだよ。センスのかけらもないよね。 
\\	ここでは「カリスマ美容師」を
\\	「才能と個性のあるへーアースタイリスト」で表現しています。
\\	という言い方もできます。
\\	に好きなようにさせる」。	
\\	彼の決意を駆り立てたものは一体何なのだろう。 
\\	「駆り立てる」のニュアンスは 
\\	(可能にする)で表せます。	
\\	このガスコンロ、最近火力が弱くなってきた。 
\\	は自動詞で「熱くなる」の意味。
\\	は「熱くする力がなくなった」の意味。	
\\	後でこの借りはちゃんと返すから、とりあえず、君がこの壷を壊した事にしておいてくれない? 
\\	は「埋め合わせをする」、
\\	は「それだけの価値があるものにする」という意味。。	
\\	軽い気持ちで株に手を出してはいけない。 
\\	この場合の 
\\	は「何気なく、のんきに」、
\\	は「本気の」の意味。	
\\	最近の若い子は死を軽くみているのでは? 
\\	で「軽く受け取る」の意味になります。
\\	の代わりに
\\	を使ってもいいでしょう。	
\\	もう午前中は忙しくて、お客の対応でてんやわんやで声がかれそうだよ。 
\\	「(声が)かれる」は 
\\	を使って表現すればいいでしょう。
\\	は「大慌ての」、
\\	は「大騒ぎ」の意。	
\\	彼女の可憐な目を見てると、たまらないね。 
\\	「可憐な」は
\\	「可愛らしい」や
\\	を使えばいいでしょう。
\\	「耐える」の意。	
\\	今月はカレンダー通りに出勤します。 
\\	「カレンダー通り」は
\\	として、その後に具体的に説明すればいいでしょう。また、
\\	「通常のカレンダー通り」と表現することも可能です。	
\\	最近、だらだらしていて怠けて掃除もしてないんです。かろうじて、洗濯はしているのですけど。 
\\	「かろうじて」を 
\\	(…だけはしている)で表現しています。	
\\	あの子、かわいい顔してけっこう言うね。 
\\	この場合の
\\	は
\\	の割には」の意味を表しています。
\\	「言いたいことを言う」、
\\	「意志が強い」の意。	
\\	私って、日に当たると直ぐに真っ赤になって、皮がむけちゃうのよ。 
\\	自動詞
\\	「日に焼ける(火傷)」、
\\	「皮がむける」の使い方がポイントですね。	
\\	人差し指のささくれをひっぱたら、皮がむけて、血がにじんできた。 
\\	ここでは動詞の使い方に注目してください。
\\	「ぐいっと引っ張る」、
\\	「むける」、
\\	「引き出す」。
\\	「ささくれ」。	
\\	捕らぬ狸の皮算用 
\\	が「捕らぬ狸の皮算用」に当たる言い回しです。	
\\	ねねね、旅費半分もつから、その代わりにあなたの実家に泊めてもらえるように、話つけてくれない? 
\\	ここでは「その代わりに」を
\\	を使って表現しています。流れからして
\\	でいいでしょう。
\\	の埋め合わせをする」も使えますね。	
\\	彼に代わりにやっておいてもらおうと思っている。 
\\	「代わりに」は前置詞
\\	で表現できます。	
\\	その事に関してはわからないので、主人にかわります。(電話で) 
\\	この場合の「代わる」とは
\\	と話をさせる」ということですね。	
\\	季節の変わり目に風邪を引く人が多い。 
\\	「季節の変わり目」は
\\	でいいでしょう。	
\\	あんな男が好きだなんて、まったく花子は変わり者だなあ。 
\\	は「変人」、
\\	は「奇妙な人」の意味。	
\\	今、彼横にいるよ。彼に代わろうか?(電話をしていて) 
\\	で「(その人を)電話に出す」の意味。	
\\	悪かったわね。どうせアタシゃカンが悪くってよ。 
\\	は「霊感がない」、
\\	「ピンと来る」の意味。	
\\	間一髪のところで最終電車に間に合った。 
\\	共に「かろうじて」の意味。	
\\	車を運転するときは、前の車との間隔を数メートルあけて運転しないと危ないよ。 
\\	「間隔をあける」をここでは
\\	「数メートルあける」、
\\	「数メートル下がる」で表現しています。	
\\	ケーブルカーは何分間隔で運行されていますか。 
\\	この場合の「間隔」は 
\\	を使って表現すればいいでしょう。	
\\	彼らはゲーム感覚で犯罪を起こして楽しんでいる。 
\\	「感覚」は
\\	でいいでしょう。
\\	は動詞としても使えます。	
\\	人の名前を覚えるのが得意というかなんか感覚で覚えているんだよね。 
\\	この場合の「感覚」という表現がとても曖昧なので、ここでは
\\	「その人の印象と名前を直感的に(感覚的に)結びつける」と具体的に表現してみました。	
\\	国会は、盗聴法案を巡ってまさに侃々諤々の体であった。 
\\	「侃々諤々」とは「激しい議論を交わす」ことですから、
\\	「わいわい大騒ぎの議論」、
\\	「熱論」を使って表現すればいいでしょう。	
\\	人生は、その人間が考えたとおりになっていくと思いませんか? 
\\	で「その人がこうなるだろうなと思ったように」の意味になります。	
\\	君は考え違いをしてるよ。人の話をよく聞いて、もう1度、よーく考えてごらん。 
\\	で「間違った考え方をしている」の意味。
\\	と
\\	を組み合わせても同じようなことが表現できます。	
\\	考えようによっては、彼の弱点も長所とみることができる。 
\\	(それの見方次第で)で「考えようによっては」を表現しています。	
\\	引っ越しのことを考えると、頭が痛くなる。だって、やることがいっぱいあるし。本当にちゃんと引っ越しできるのかな。 
\\	(引っ越しすると考えること)を主語にして表現している点に注目。
\\	は「やってのける」の意。	
\\	あいつ今日はカンが冴えてるな。 
\\	には「霊感を与える」の意味がありますので、これを受け身にすれば「カンが冴えている」に近い意味が表現できます。	
\\	感極まって泣いてしまった。 
\\	で「感動して泣く」の意味になります。	
\\	皆それぞれ育ってきた環境が違うからね。そういう人が一緒に住むというのは難しい事なのかも。 
\\	で「違った環境で育つ」の意味。	
\\	そちらの環境で日本語が見れるという事なので、お言葉に甘えて日本語でメールします。 
\\	「環境」は
\\	でいいのですが、ここではより具体的に言うために
\\	を付け足しています。	
\\	最近、アルミ缶の飲料を飲まないようにしている。環境ホルモンの影響を受けると言われているので。 
\\	「環境ホルモン」は、
\\	と言います。
\\	「含む」。
\\	「摂取する」。	
\\	自分の受験番号を見つけたとき、彼は感極まって泣き出してしまった。 
\\	「感情的になる、感激する」の意味。
\\	「新入生」。	
\\	だから、この先も友達という関係でいましょう。 
\\	「友達という関係」をここでは
\\	(友達でいる)で表現しています。	
\\	何度もお電話いただいているみたいですが、うちの娘とはどういうご関係ですか? 
\\	「どういうご関係」は 
\\	と 
\\	を使って表現すればいいでしょう。	
\\	会社の関連会社(子会社?)で勤めています。 
\\	は「関連(系列)会社、支部」という意味ですね。	
\\	親友になるのに、期間は関係ないと思う。 
\\	で「関係がない」の意味を表します。	
\\	今日のレッスンとは全然関係ないのですが、質問があるのですが。 
\\	「関係ない」は
\\	で表現します。第一文の
\\	の後に質問をすればいいのです。	
\\	彼が独断でやってくれたことに腹を立てるのではなく、むしろ歓迎すべきだ。 
\\	ここでは「歓迎すべき」を 
\\	で表現しています。	
\\	あそこで今買い物すると20%還元だって。いろいろ考えているよね。 
\\	この場合の「還元」は「値段を下げる、割り引く」ということですから、
\\	「下げる」、
\\	「カットする」を使えばいいでしょう。
\\	「引きつける」。	
\\	ああいう人は観光客の敵だ。1回甘やかすと癖になるから絶対せがまれてもお金はあげるもんじゃない。 
\\	(…に注意する)で「敵」を表現しています。この文脈では 
\\	も使えます。	
\\	冠婚葬祭に通用する礼服がないんだよ。 
\\	「礼服」とは
\\	(公式の場に着ていく服)ということですね。
\\	とも言えます。	
\\	ドルで値段を言われると、どうしても、一度日本円に換算して値段の価値を考えてしまいます。 
\\	が「換算する」に当たる動詞です。
\\	(計算する)を使っても同じようなことが言えます。	
\\	食べ物に関して言えば、このステーキセットで間に合うと思うけど、飲み物は別に注文したほうがいいと思うよ。 
\\	に関して言うと」は
\\	や
\\	を使って表現します。	
\\	地図に関しては、彼のほうが得意なので彼に道順を説明して下さい。私は、自他ともに認める方向音痴なもので。 
\\	ここでは「関しては」を前置詞 
\\	で表現しています。
\\	は「地図のことになると」、
\\	は「ひどい方向音痴である」の意。	
\\	お世話になった人たちに対する感謝の気持ちを忘れないようにしなさい。 
\\	は「…に感謝する」、
\\	は「感謝の念」の意味。	
\\	最近、間食もせずに 規則正しい生活を送っています。 
\\	「間食をしない」をここでは
\\	「お菓子(軽食)に手を出さない」で表現しています。
\\	「十分な食事」。	
\\	間食するのを止めないと、痩せないよ。 
\\	には動詞として「軽く食べる」という意味があります。
\\	の方が「間食」の意味がはっきり出ますね。	
\\	彼の行動力にはホント感心するね。どこからあのパワーはくるんだろうか。 
\\	「感心する」は
\\	で表現できます。	
\\	幹事の仕事って、案外大変なんだね。だって、皆予定が全然合わなくて、合わせるのに大変! 
\\	「幹事」は
\\	という言い方がありますが、ここでは
\\	「幹事の仕事」と表現しています。	
\\	何あれ!あの店員、男と女で客の対応が全然違う。感じ悪い。 
\\	で「あれって嫌だね」という感じになります。	
\\	なんか明日から仕事って感じがしないなあ。もうちょっと計画的に休みを過ごせばよかった。 
\\	で
\\	本当のようには思えない」の意味になります。	
\\	お姉さんと顔は似てないけど、感じとか仕草とかがそっくりだよね。 
\\	「感じが似ている」のニュアンスを
\\	と
\\	「印象」を組み合わせて表現しています。	
\\	あの人ってなんだか感じがわるいよね。しゃべりかけにくいよ。 
\\	「身の毛がよだつような」、
\\	「きみが悪い」と
\\	「雰囲気」の組み合わせがポイントです。	
\\	感情のコントロールができるのが大人ではないでしょうか。 
\\	「感情」は
\\	と複数形にします。
\\	は「…の一部分」の意。なお、
\\	の前の冠詞
\\	はよく省略されます。	
\\	悲しい映画は、つい感情移入し、涙顔を見られるのがいやで見たくない。 
\\	「感情移入する」とは
\\	「同一視する」ということですね。
\\	「引き込まれる」。	
\\	彼は感情の起伏が激しいから、付き合いにくい。 
\\	(むら気の、気分屋の)で「感情の起伏が激しい」を表現しています。	
\\	お前は肝心な時にいつもいないんだから。役に立たないやつだ。 
\\	とは「とても重要な時、場面」のこと。
\\	は「いざという時」の意。	
\\	風邪ひくと関節とか痛くなるよね。 
\\	「間接」は
\\	と言います。
\\	は「痛む(自動詞)」、
\\	は「痛い(形容詞)」の意。	
\\	パソコンがウイルスに感染しちゃって、パソコン再インストールする羽目になって散々だったよ。 
\\	で「ウイルスに感染する」の意味になります。
\\	は
\\	を再インストールする」の意。	
\\	日本人学生のさぼりぐせが外国人留学生に感染し困っている。 
\\	には「(癖などが)... にうつる」という意味がありますので、この文脈ではこれを使えばいいでしょう。
\\	「傾向、性癖」。	
\\	その本を読んだきみの感想は? 
\\	で「どう思ったか教えてくれ」の意味になります。
\\	は「印象、感じ」の意。	
\\	あなたの歓送会なんだから主役が行かないと意味がない。 
\\	「歓送会」は
\\	と言います。
\\	のみの場合は「歓送、送別」の意味になります。	
\\	今日は天気悪いから、これじゃあ洗濯物は乾かないな。仕方ない、乾燥機にかけるか。縮んじゃうからほんとは嫌なんだけどな。 
\\	「乾燥機」は
\\	ですね。
\\	は「洗濯物」、
\\	は「縮ませる」の意味です。	
\\	勘違いして逆方向の電車に乗ってしまった。 
\\	「勘違い」は
\\	「誤って」や
\\	「間違える」で表現できます。	
\\	1年アメリカに住めば英語がうまくなると勘違いする人が多い。 
\\	で「思い違いをして…と考える」の意味になります。
\\	も使えます。	
\\	あまりそのことについては突っ込まないでくれよ!勘で答えただけだからさ。 
\\	は「直感で」の意。	
\\	感動のあまり開いた口がふさがらなくなり、最後の方ではちょっと泣きそうになった。 
\\	ここでは「感動のあまり」を
\\	「感動して、感極まって」を使って表現しています。
\\	は「口を大きく開ける」、
\\	は「口を開けて(形容詞)」の意。	
\\	堪忍袋の緒が切れた。あいつ、今度は絶対に許さないぞ。 
\\	「忍耐」と
\\	「すり切れる」を組み合わせて「堪忍袋の緒が切れる」を表現しています。	
\\	調子はそれほど芳しくない 
\\	で「身体の調子がよくない」の意味になります。
\\	は「調子がいい」の意。	
\\	最近の日本経済には勢いがないので、カンフル剤を打つ必要がある。 
\\	で「カンフル剤を打つ必要がある(比喩的表現)」という意味になります。	
\\	怪物ルーキー松坂が近鉄を1安打完封。 
\\	「完封」とは
\\	「点数なし」ということですね。「怪物」をここでは
\\	「大評判(の)」で表現してい点にも注意してください。	
\\	それだけは勘弁してよ。 
\\	は「何でもするから、それだけはさせないでくれ」の意。	
\\	うちの旦那は写真に関しては完璧主義者。妥協ができない人なのです。 
\\	「完全主義者」は 
\\	と言います。
\\	は「…でよしとする」の意。	
\\	最近、勧誘の電話が多くて困っているんだ。 
\\	「勧誘電話」は
\\	と言います。現在、この手の電話はアメリカでも問題になっています。
\\	「我慢する」、
\\	「困らせる」の意。	
\\	今朝はなんかボーっとしていて電話の対応が無愛想でごめんね。しかも、勧誘かと思って出たから。 
\\	とは「電話で販売する人」のこです。
\\	は「セールス(勧誘)の電話」の意。	
\\	プログラム開発をする時は、自分の下についている人に作業指示したり、彼らの作業管理をしたりします。 
\\	「管理する」は
\\	「管理・監督をする」で表現すればいいでしょう。	
\\	30才になって、お前も貫禄がついてきたな。 
\\	は「大人の振る舞い・態度」という意味です。
\\	「振る舞う」。	
\\	彼が言うには、事前に申し込みが必要だということだけど。 
\\	「~が言うには」は
\\	「彼によれば」を使えば表現できますね。	
\\	妹はイギリス留学から帰国後外国かぶれしている。 
\\	(外国のすべてのことに夢中になっている)を使って「外国かぶれ」を表現しています。	
\\	伊藤は今、外出しています。 
\\	「外出している」は
\\	で表現します。	
\\	3番目のインタビュー項目に該当する人は手を上げてください。 
\\	「該当する」は 
\\	を使って表現すればいいでしょう。
\\	は前の 
\\	を修飾しています。	
\\	あいつ見た目はおとなしそうだけど、かなり我が強いよ。 
\\	「自分の事ばかり考えている」と
\\	を組み合わせて「我が強い」を表現してみました。
\\	「臆病な、内気な」、
\\	「自意識過剰な」。	
\\	は文字通り「お使いに来た子供じゃない」の意味になります。
\\	で「もっとちゃんとしろよ」と言っているのです。	
\\	吊革につかまったまま熟睡してしまって、たまにガクっとなって恥ずかしい時がある。 
\\	「膝がガクッとなる」は
\\	と言います。	
\\	お前は、本当に、学習しないね。 
\\	「学習しない」はそのまま
\\	と言えばいいでしょう。	
\\	どんな講座があるかわからないなら、学習相談に聞いてみてはいかが。 
\\	とは「学生がさまざまな事柄について相談をしに行くオフィス」のことです。	
\\	あいつら仕事にまるで身が入ってないようだけど、まだまだ学生気分が抜けてないんだろうな。 
\\	「学生気分」をここでは
\\	「まだ学生のように見える」と表現しています。
\\	「様子、態度」。	
\\	その知らせを聞いて、彼はガクッときた。 
\\	「ガクッとくる」とは
\\	「ショックを受ける」ということですね。	
\\	学生の学力が低下したということで、数学の補習をしている大学があると新聞に出ていた。 
\\	「学力低下」は
\\	「学力」と
\\	「低下する(動詞)、低下(名詞)」を組み合わせて表現できます。
\\	「補修の、治療の」。	
\\	君は学歴が通用すると思っているが、今はそんな時代じゃないんだ。 
\\	で「有名大学に行ったという強みで出世する」の意味。ここでは
\\	の後に
\\	「アイビースクール(ハーバード、プリンストンなど)」を使っていますが、状況に応じて、具体的な大学名を入れればいいでしょう。	
\\	学割を取るため学生部に行った。 
\\	「学割」は 
\\	でいいでしょう。
\\	は「…として資格を得る」の意。	
\\	かかとの裏ががさがさなので、軽石で毎日こすってお手入れをしなければいけません。 
\\	「がさがさ」は 
\\	、あるいは、
\\	で表現すればいいでしょう。
\\	は「軽石」の意。	
\\	来年公開されるシュワルッツネッガーの映画っておもしろそうだね。なんか、あの人が主演する映画って見ちゃうんだよね。 
\\	「主演する」は
\\	を動詞として使えばいいでしょう。
\\	「ヒット作になりそうな」。	
\\	高速道路を走っていて、ガス欠で車が動かなくなってしまった。 
\\	「ガス欠になる」は 
\\	と言います。
\\	は「(車を脇に寄せて)止める」の意。	
\\	規律の厳しい団体生活には、時々ガス抜きが必要だ。 
\\	には「ストレスを発散する、鬱憤を晴らす」という意味があります。	
\\	あそこのスポーツ新聞の記事ってがせネタばっかなんだよなー。よくだまされるよ。 
\\	は「事実無根の」という意味。
\\	は「スポーツ紙」、
\\	は「新聞(軽蔑的に)」の意。	
\\	いつまでもガタガタ言ってないで、少しは実行してみたらどう? 
\\	「ガタガタ言うな」は簡単に
\\	と言えばいいでしょう。
\\	「仕事をする」の意。	
\\	"ガチョーン 
\\	「ガチョーン」は
\\	""ですから... ここではどのうよな場面で使うのかを説明しましたが、これでいいでしょうか?谷啓さんに聞いてみる必要がありますか?
\\	「じっと見つめる」、
\\	「急いで逃げる」の意。
\\	あまり期待していると、私に会った時にがっくりするかも。 
\\	「がっくりする」は
\\	「失望する」、あるいは、
\\	「失望(名詞)」を使えばいいでしょう。	
\\	今、教育の現場では学級崩壊から一層悪化した学校崩壊が進行しつつある。 
\\	「学級崩壊」を
\\	、「学校崩壊」を
\\	で表現しています。
\\	(しつけ、規律)を使った補足説明にも注目。	
\\	この数字は、私の常識に合致するが、理屈に反する。 
\\	で「…と適合する」の意味になります。
\\	は「常識的に」の意。	
\\	ドーセットは第1セットで早くもガッツポーズをしていた。 
\\	「ガッツポーズ」は
\\	(和製英語)とは言いません。
\\	「勝ち誇ったポーズをする」のように説明する必要があります。
\\	「勝ち誇って拳をあげる」と表現してもいいでしょう。	
\\	競馬で大穴を当てて、がっぽり金が入ったから、今日の飲み代は僕が出すよ。 
\\	には「お金をどっさりもっている」という意味があります。
\\	「たんまりお金が入る」の意味。
\\	「大穴」。	
\\	こう暑いと、コーラがぶ飲みしたくなるよ。 
\\	が「がぶ飲み」に当たる動詞です。
\\	「グイグイ飲む」。	
\\	日本人は我慢強いほうだと思う。 
\\	「我慢」は
\\	(忍耐力)を使って表現すればいいでしょう。
\\	は「苦難に耐えることができる」、
\\	は「文句を言わない」の意。	
\\	あートイレ行きたい。でも家に着くまで我慢しよう。 
\\	この場合の「我慢する」は 
\\	で表現すればいいでしょう。	
\\	これ前から欲しかったんだ。でも高いからずっと我慢してたの。 
\\	には「(感情を)抑える」という意味があります。
\\	は「誘惑に抵抗する」の意。	
\\	歯医者で、痛かったら手を挙げてくださいと言われたので、そうしたら、我慢してくださいと言われた。 
\\	で「痛みをこらえる」の意味になります。
\\	は「我慢する」の意。	
\\	あなたの言動にはずーっと我慢してたの。 
\\	「我慢する」は 
\\	で表現できます。	
\\	最近の若い奴らは、我慢するってことを知らないからな。 
\\	「我慢する」を
\\	「忍耐力がない」で表現しています。	
\\	そうガミガミ言うなよ。さっき謝ったじゃないか。 
\\	が「ガミガミ言う」に当たる動詞です。	
\\	最初の3年はがむしゃらに働いたけど、最近はやる気が全然起きない。 
\\	(狂ったように働く)で「がむしゃら」を表現しています。
\\	は「ものすごく」の意。	
\\	今日、がらがらだね。この時間帯って空いているのかね。 
\\	「がらがら」とは
\\	「ぜんぜん込んでない」ということですね。	
\\	私の自信は、ガラガラと音をたてて崩れた。 
\\	「(シューシューと)音をたてて消える」、
\\	「あえぐように最後の息をする」、
\\	「バタッと倒れる」。この3つの動詞を使って「ガラガラと音をたてて崩れた」を表現してみました。
\\	「しぼむ」。	
\\	あの件に関しては、 画竜点睛を欠いてしまった。 
\\	「画竜点睛を欠く」とは「最後の大切な部分が欠けている」の意味ですから、
\\	「仕上げが必要である」を使って表現します。	
\\	ちょっと寒いじゃない!誰よ、こんなに冷房ガンガンに入れる人。 
\\	は
\\	上げる、始動させる」、また、
\\	は「過熱状態」の意味になります。	
\\	私のストレス解消法は、ウォークマンでガンガンにロックを聞くことです。 
\\	は「猛烈な、激しい」、
\\	は「たたきつけるような」の意味。 前置詞
\\	の使い方にも注意。	
\\	茶髪に顔黒か?俺の趣味じゃないね。 
\\	「顔黒」は
\\	「真っ黒に日焼けしている」でいいでしょう。顔を出さなくても意味は通じます。	
\\	彼は自分の意見を全く変えないがんこじじいだ。 
\\	「頑固じじい」は
\\	「頑固な」、
\\	「頭が固い」と
\\	「とんまな老人」を組み合わせて表現します。
\\	「妥協する」、
\\	「動く」の意。	
\\	義理と人情でがんじがらめになって動きがとれない彼。見ていて可哀想。 
\\	でがんじがらめ」は
\\	で身動きがとれない」を使って表現できます。「義理と人情」の表現の仕方にも注意。	
\\	若い内から目標を持って頑張っている子をみると尊敬してしまう。 
\\	で「熱心に粘り強く 
\\	を追い求める(続ける)」の意味になります。	
\\	これからは、頑張り甲斐のある社会を構築していかなければならない。 
\\	「…し甲斐がある」は 
\\	で表現できます。「頑張る」は 
\\	でいいでしょう。
\\	は「…に貢献する」の意。	
\\	仕事辛そうだけど、頑張ってね。 
\\	「頑張る」は、
\\	を続けてやる」、
\\	「やり通す」で表現できます。
\\	「大変なこと、うんざりするようなこと」。	
\\	このバンド頑張っているね。3週連続で2位だって。 
\\	「頑張る」とは
\\	(とてもよくやっている)ということですね。
\\	の使い方にも注意。	
\\	お互いに頑張ろうね! 
\\	の文字どおりの意味は「持っているものをすべてぶつける」です。	
\\	このドア、開けたり閉めたりするとキーキー音がするんだけど、直してくれない。 
\\	「キーキー(音がする)」は
\\	を使って表現します。2番目の訳文では
\\	が名詞として使われています	
\\	このイス、座るとキーキー言うんだ。 
\\	「キーキー」は自動詞
\\	あるいは
\\	で表現します。	
\\	君の(弾く)ピアノは気合いが入ってるんだろうね。 
\\	には「素晴らしい、見事な(啓示を受けたように)」の意味があります。	
\\	皆、気合がはいってきたな。私も負けてられないぞ。 
\\	「気合が入る、入れる」を
\\	「持っているもの全てをぶつける」で表現しています。
\\	「全身全霊を傾けて」。	
\\	ねえ、聞いてくれるぅ、バスに乗るまで1時間も待たされちゃってさあ。 
\\	「聞いてくれる」は
\\	の他に、
\\	「何だと思う」も使えますね。	
\\	気にしないで、ちょっと聞いてみただけ。 
\\	と 
\\	の組み合わせで「聞いてみただけ」を表現しています。
\\	は「気分を悪くする」の意。	
\\	給料はのほとんどは食費に消えるよ。 
\\	は
\\	に行ってしまう」、
\\	は「占める、奪う」の意味。	
\\	何事においても気負いが失敗を招くから、常に平常心を保ちなさい。 
\\	「気負い」を
\\	(熱心過ぎる)、
\\	(興奮し過ぎる)で表現しています。	
\\	ここのところ勉強していないので、日々、1単語ずつ記憶から消えていっている気がする。 
\\	で「記憶から消える」という意味になります。
\\	を使ってもいいですね。	
\\	彼は信じられないくらい記憶力がいいんだ。昔あったことを事細かに覚えてるんだから。 
\\	「記憶力」は 
\\	で表現できます。
\\	は「とっても細かいこと」の意。	
\\	英語ってさ、数学の応用問題とかとは違ってあまり考えることがないじゃない?すなわち記憶力の勝負だよね。 
\\	ここでは「勝負」を 
\\	(…に頼る)を用いて表現しています。
\\	と 
\\	(記憶する)の組み合わせにも注目。	
\\	浴衣は夏しか着る機会がない。 
\\	は「好機、機会」の意味。「浴衣」は 
\\	でいいでしょう。	
\\	僕は機械音痴だから、ファックスとかコンピュータ使うの全然だめなんだ。 
\\	で「機械をひどく苦手としている」という意味になります。	
\\	インターネットが広まってメールで済ませてしまう事が多くなって、手紙を書く機会が減ったと思う。 
\\	「機会が減る」とは
\\	する機会がどんどん少なくなくなる」ということですね。	
\\	明日久々にあいつに会うけど、今あいつがはまっているゴルフの話をおもいっきり聞かされそうだ。 
\\	で「…についてうんざりするほど聞かされる」の意味になります。
\\	は「長い話」の意。	
\\	うちの旦那にもその話聞かせたいよ。 
\\	の使い方がポイント。	
\\	歯の治療も保険がきかないとこんなにするんだね。 
\\	には「(保険で)保証する」という意味があります。
\\	はその名詞形です。	
\\	今の話、聞かなかったことにしてくれ! 
\\	第1文は、「私がそれを言うのを君は聞かなかったんだ、いいかい」、第2文は、「この話はしなかったんだ」の意味になります。
\\	「それは聞かなかったという振りをしてくれ」でもいいですね。	
\\	どっちがいいかって、聞かれたら、やっぱりこっちかなぁ。 
\\	「聞かれたら」をここでは
\\	言わなくてはいけないのなら」と言い換えています。	
\\	水を飲んだら気管に入っちゃって、咳が止まらない。 
\\	で「(間違った)気管に入る」の意味になります。	
\\	ウッ、コーヒーが気管に入っちゃったよ。ゴホ、ゴホ、ゴホ。 
\\	とは「(食道ではなく)いつもとは違う管の中に入る」つまり「気管に入る」ということ。	
\\	ちょっと目が合ったからって、自分に気があるなんて思うなよ。 
\\	「気がある」は、
\\	「好き」で簡単に表現できますね。	
\\	あなたのことを想ってる人に対して、好きでもないのに気があるふりをするのは残酷だと思う。 
\\	で「…に気があるふりをする」の意味になります。
\\	の使い方にも注目。	
\\	そんなに汗をかいて。早く着替えないと、風邪引くよ。 
\\	「着替える」は 
\\	、あるいは、
\\	だけで表現できます。
\\	は「汗をかく」の意。	
\\	海外に行くと、気が大きくなって 思わず(お金を)使いすぎちゃうんだよね。 
\\	「気が大きくなる」をここでは
\\	「開放的な」を使って表現しています。
\\	「(お金を)使い過ぎる」。	
\\	あんなに毎日報道陣にあとをつけられたり囲まれたりしたら、気がおかしくなっちゃうよ。私には無理だね。 
\\	「気がおかしくなる」は 
\\	で表現します。
\\	は「追い回す」、
\\	は「執拗に、情け容赦なく」の意。	
\\	あなたはホントに気が利くねぇ。 
\\	ここでは「気が利く」を 
\\	(気に掛ける)、
\\	(察しのいい)で表現しています。	
\\	彼の話の内容が気が気でなかった。 
\\	は「不安な気持ちでいる」、
\\	は「はらはらさせる」の意味になります。	
\\	彼は、いつ悪事がばれるかと、気が気でなかった。 
\\	「気が気でない」は
\\	「不安、落ち着かない」で表現できます。「悪事」は
\\	とも言えますね。
\\	「いらいらしている」の意。	
\\	あんなちょっとの量じゃ食べた気しないよね。歯に詰まって終わりだったよ。 
\\	で「全然食べた気がしない」の意味になります。「歯に詰まって…」の部分は意訳をしています。
\\	は「一口分」の意。	
\\	宿泊費としていくら支払えばいいですか?無料というのはこっちの気がすまないというか、それは絶対よくないです。 
\\	で「気分がすきっとしない、何かひっかっかる」の意味。	
\\	何度傘を無くしたら気が済むの? 
\\	「気が済む」とは、「満足する」という意味ですから、ここでは
\\	を使って表現しています。	
\\	自分でもできるような気がしてきた。 
\\	「気がする」は
\\	する気分になる」で表現できます。
\\	でもいいですね。	
\\	髪って意外と重いんだよね。 髪切ってから、肩凝りが少なくなった気がする。 
\\	この場合の「気がする」は
\\	の感じがする」を使えばいいでしょう。	
\\	もう!さっきから私の席の前で世間話されてすごく気が散るな。ちゃんと仕事しろ! ってホント言いたくなる。 
\\	「気が散る」とは 
\\	(集中できない)ということですね。
\\	は「(気を)散らす」の意。	
\\	インターネットやっていると時間が経つのが早い。気がついたら既に1時間経っている事が多い。 
\\	ここでは「気づいたら」を文脈に合わせて
\\	(時間のことに気づく前に)と言い換えています。	
\\	彼の妻は気が強くて、彼は尻に敷かれている。 
\\	は「積極的な、有無を言わせない」の意味。
\\	は「…を尻に敷く、こき使う」の意。	
\\	35年のローンだなんて、気の長い話だなぁ。私には考えられない。 
\\	「気の長い話」を
\\	で表現しています。35-
\\	は形容詞なので、
\\	に
\\	を付ける必要はありません。	
\\	あの人、定年したらすっかり気が抜けちゃったみたいで、毎日なにもせずブラブラしてるのよ。 
\\	「気が抜ける」は
\\	に対する熱意、情熱がなくなる」で表現できます。
\\	とは「彼の帆には風が吹かなくなった」つまり「やる気がなくなった」という意味です。	
\\	なんか学校も終わってしまって、気が抜けて英語の勉強を怠っていたので、英語力がだんだん落ちてきている気がします。 
\\	「気が抜ける」を 
\\	(やる気がしない)で表現しています。
\\	は「…をおろそかにする」、
\\	は「悪化する」の意。	
\\	1レッスン単位に支払うことができれば、気兼ねなく払える金額なんだけどね。 
\\	「気兼ねする」とは「少しためらう」ということですから、
\\	「よく考える、二の足を踏む」を使って表現すればいいでしょう。	
\\	平日は常に気がはっていて、週末になると気が抜けるせいか体がだるかったり頭痛がする。 
\\	は「張りつめている、緊張している」の意味です。
\\	「気力がなくなる」。	
\\	来週末に引っ越しというのに、全然部屋が片付いていないよ。もう、考えただけでも気が変になりそうだ。 
\\	で「頭がおかしくなる、気が変になる」の意味。
\\	でも同じような意味になります。(※和文を一部修正しています)	
\\	忙しすぎて、そこまで気が回らなかった。 
\\	で「その考えが浮かばなかった」、
\\	は「そんなこと考えもしなかった」の意味。	
\\	彼は気が短いから、接客業には向いてないと思う。 
\\	「気が短い」は
\\	と言います。
\\	に向いている」。	
\\	こんな話を聞くと、気がめいってしまう。 
\\	「気が滅入る」は
\\	で表現できます。
\\	「落ち込ませる」は動詞としても使えます。	
\\	気が弱いので、会社では後輩にさえ言いたいことが言えず情けない。 
\\	は「臆病な、小心な」、
\\	は「気の弱い、意気地のない」、
\\	は「部下」の意味。	
\\	話の内容に聞き入ってしまって、質問する余裕がなかった。 
\\	(…に聞き入る)に
\\	(…に魅惑される)を付け足しています。
\\	は「うっとりする」の意。	
\\	日本は世界の国々に比べ、危機管理がまるでできていない。 
\\	「危機管理」は
\\	といいます。
\\	は「無能な」の意。	
\\	彼は聞き上手だよね。 
\\	「聞き上手」はそのまま
\\	と言えばいい。	
\\	夏のデパートって、冷房が効き過ぎてるから、外に出たとき、温度差で鳥肌(立つよね。 
\\	で「エアコン強くかけすぎている」という意味になります。
\\	「鳥肌が立つ」。	
\\	あいつの言うことなんて聞き流していればいいんだよ。 
\\	ここでは「聞き流す」を 
\\	(右の耳から入って左の耳から出る)で表現しています。	
\\	メイプルシロップは、虫歯になるのを防いでくれると聞きました。ホントかどうかは知りませんが。 
\\	で「…だと(言われるのを)聞いたことがある」の意味になります。	
\\	を
\\	""と聞き間違えてしまった 
\\	「聞き間違える」は
\\	「誤って」と
\\	を組み合わせて表現します。
\\	薬局でかぜ薬買うより、医者に行ってもらったほうが安いし、効き目あるよ。しかも、診断もしてくれるし。 
\\	「効き目がある」は
\\	「効果がある」を使えばいいでしょう。
\\	は「風邪薬」のこと。	
\\	あの子は聞き分けのいい子だ。 
\\	(従順な)で「聞き分けのいい」を表現しています。
\\	をそのまま使ってもいいでしょう。	
\\	スポーツはいつも危険と隣り合わせだ。面白半分でやっていると危険な目にあう。 
\\	(…と連動する、密接な関係にある)と 
\\	を使って「危険と隣り合わせ」を表現しています。
\\	でも同じ意味になります。	
\\	レッスンが期限までに終わりそうに無かったので、期限延長の手続きをしたら、2ヶ月伸びたんだ。 
\\	で「期限延長を申し込む」の意味になります。	
\\	隣の座席の人のヘッドフォンから邦楽が聞こえてくる。 
\\	~で「…が~から聞こえてくる」の意味になります。	
\\	あの子、帰国子女のくせに英語全然だめだよね。 
\\	ここでは「帰国子女」を 
\\	(外国に住んでいたことがある)で表現しています。	
\\	お酒は気心の知れた仲間と飲むのが一番だ。 
\\	(隅から隅までよく知っている親友)で「気心の知れた仲間」を表現しています。	
\\	このスーツ、形はかっこいいけど着心地がいまいちなんだよ。 
\\	「着心地」は
\\	を使えばいいでしょう。
\\	「よくない、ぶざまな」。	
\\	あの人はとっても気さくな方です。 
\\	は「率直な、隠し事のない」、
\\	「気の合う、楽しい」の意。	
\\	東京もんはキザであかんわ。 
\\	は「気取っている、かっこつけている」、
\\	は「気取っている奴」の意味。	
\\	機種依存文字をお使いではありませんか? 
\\	ここでは「機種依存文字」を専門用語を使わずに、意訳して
\\	「特殊なフォント」として表現しています。	
\\	合格の基準は、私が実際海外で住んだ時に困らない程度の英語力を身につけているかどうかです。 
\\	「基準」は
\\	を使えばいいでしょう。
\\	「何とかやっていく」。	
\\	彼の主張は単なる机上の空論にすぎない。 
\\	が「机上の空論」に当たる表現です。
\\	は「思索する」の意。	
\\	衛兵交代が奇数の日だけなんて知らなくてせっかく行ったのに見れなかった。 
\\	「奇数」は
\\	と言います。「奇数の日に」は
\\	ですね。	
\\	中国では人口を増やさないために子どもの数が規制されていると以前聞いたことがあります。 
\\	「規制する」は 
\\	、あるいは、
\\	を使えばいいでしょう。	
\\	桜が終わるとつつじの季節がやってくる。 
\\	「つつじの季節」は
\\	でいいでしょう。(注)和文の最後に「がやって来る」を付け足しました。	
\\	今だけの季節限定のケーキってありますか?これだけは、食べるべきだという一押しのケーキでもいいですけど。 
\\	「季節限定」とは 
\\	(この季節だけに入手できる)ということですね。「一押し」の表現の仕方にも注目。	
\\	天気予報によると今日の午後は季節はずれの雪が降るらしい。 
\\	「季節はずれの」は 
\\	で表現できます。
\\	も使えます。	
\\	あの場面でも、彼は毅然としていました。 
\\	「毅然」は
\\	「冷静でしっかりしている」で表現できます。
\\	「冷静である」の意。	
\\	彼は麻薬所持で起訴された。 
\\	「起訴する」に当たる英語は
\\	ですね。
\\	「所有」。	
\\	規則正しい生活、食事が健康の秘訣です。 
\\	に「規則正しい、きちんとした」という意味があります。
\\	も同様です。	
\\	規則正しい生活をしないと、いずれは体を壊すことになるよ。 
\\	で「きちんした生活」という意味。	
\\	あんなに皆から期待されると、それがプレッシャーになって、彼女、オリンピックで金メダルが取れないかもしれない。 
\\	で
\\	に大いに期待する」の意味になります。
\\	「期待」。	
\\	どうしていつも守らない約束をするの?期待させないで、悲しくなるだけだから。 
\\	で「(その人に)期待させる」の意味になります。	
\\	彼らの期待にこたえたいけど、それがまたかなりのプレッシャーになっているのだよね。 
\\	が「期待にこたえる」に当たる言い回しです。
\\	は「がっかりさせる」の意。	
\\	彼は医学部に入学できずに、両親の期待を裏切ることになってしまった。 
\\	が「…の期待を裏切る」に当たる表現です。
\\	は「…に反して」の意。	
\\	ちょっと空手でもやって体鍛えようかな…、護身にも役立ちそうだし。一石二鳥だ。 
\\	ここでは「鍛える」を
\\	の代わりに
\\	「シェープアップする、体調を整える」で表現してみました。「一石二鳥」の表現の仕方にも注目。	
\\	皆さんの忌憚のないご意見をお聞かせください。 
\\	は「率直に、忌憚無く」の意味。
\\	「控える、言わずにおく」。	
\\	女の子は、気だてのいいのが一番だね。 
\\	が「気だてのいい、人のいい」に当たる形容詞です。	
\\	念のために、2000年になる前に記帳して、残高確認しておこう。 
\\	「記帳する」は
\\	「印刷(記入)してもらう」を使えばいいでしょう。	
\\	仕事だから教えている感じで、テキスト以外のことや質問には一切答えてくれないし、きちんとチェックしてくれない。 
\\	で「ちゃんとした(誠実な)仕事をする」の意味になります。
\\	は「徹底的な」の意。	
\\	英語を習い始めたきっかけは、 映画を字幕無しで見たいと思ったからです。 
\\	「きっかけ」をここでは
\\	「そういう訳で」、あるいは、
\\	「元々は..することにした」で表現しています。	
\\	あなたに言われたことがきっかけで、自分を変えてみたいと思うようになった。 
\\	で「(人に)…するきっかけを与える」という意味になります。
\\	(…のおかげで)の使い方にも注目。	
\\	これをきっかけに今後のことを真剣に考えてみよう。 
\\	「きっかけ」は 
\\	(理由)、
\\	(機会)で表現すればいいでしょう。	
\\	あいつは生粋のアメリカ人だね。 
\\	で「心底、根っから」の意味になります。
\\	「生粋の」。	
\\	そんな仕事、きっぱり断ったほうがいいよ。 
\\	で「きっぱりと断る」の意味になります。	
\\	オーストラリアに行くのをきっかけに今の仕事はキッパリやめて、何か新しいものに挑戦したいと思っています。 
\\	で「…をキッパリやめる(手を切る)」の意味になります。	
\\	仕事がきつい。体がきつい。精神的にきつい。ズボンがきつい。 
\\	「緊張している」を使えば、4つの英文とも日本語の意味に近くなります。	
\\	こういう言い方したらきついかな?でも、実際彼の行動には少しむかついているのよね。 
\\	には「どぎつい」という意味があります。
\\	は「…にうんざりしている」の意。	
\\	お前は相変わらず言うことがきついね。 
\\	は「きつい、鋭い」、
\\	は「ぶっきらぼうな」の意味。	
\\	うわ! この臭いきつくない?? いい臭いなんだけど、長時間はつらいな。 
\\	は「圧倒するような、強烈な」という意味になります。
\\	「鼻を刺激する、きつい」。	
\\	私は、きつい性格だとよく言われる。 
\\	は「どぎつい、とげとげしい」、
\\	は「不快な、癪に障る」の意味。	
\\	それは、冗談にしてもきつい一言だな…、もうショックで立ち直れないよ。 
\\	は「厳しい、どぎつい」の意味。
\\	「きつい、鋭い」。	
\\	特にプロポーズの言葉はなかったな。気づいたら一緒に住んでいたって感じ。 
\\	この場合の
\\	は「いつのまに... 近づいてくる」というニュアンスになります。
\\	「知らぬ間に」。	
\\	家に帰ってテレビを見ていたのですが、疲れていたようで居眠りしてしまって、気づいたら8時を過ぎていた。 
\\	「気づいたら」のニュアンスを 
\\	で表しています。	
\\	勉強してたのはいいのだけど、気づいたらそのままイスの上で寝ていた。 
\\	この場合の「気づいたら」は
\\	「知らぬ間に」と言い換えてもいいでしょう。	
\\	あなたのお気遣いに感謝いたします。 
\\	で「関心を持つ、心配する」の意味になります。	
\\	新しい上司が難しい人で、われわれ部下一同、気疲れしてしまっています。 
\\	で「精神的に疲れ切る」の意味になります。	
\\	急遽、友人が一緒にホテルに泊まることになり、別に嫌ではなかったのですが、かなり気疲れしました。 
\\	ここでは「気疲れする」を 
\\	(精神的に疲れる)で表現しています。
\\	は「緊張、負担」の意。	
\\	あそこで働くのは自分にとっていいことだとは思いますが、とても気疲れします。でも、これを乗り越えないといけないんでしょうね。 
\\	で「気苦労が多い」という意味になります。
\\	は「精神的に疲れる」の意。	
\\	ついさっきまで彼女が来ていたんだ。 
\\	「来ていた」とは 
\\	(ここにいた)ということですね。	
\\	本日来ていただいた理由は、皆さんの意志を再確認するためです。 
\\	「来ていただく」は 
\\	と 
\\	を組み合わせて表現すればいいでしょう。
\\	も使えます。
\\	は「再確認する」の意。	
\\	借金を申し込んだら、木で鼻をくくるような返事が返ってきた。 
\\	「木で鼻をくくる」は
\\	「そっけない」を使って表現できます。	
\\	自民党は既得権益のしがらみがあるため、改革は遅々として進まない。 
\\	が「既得権益」に当たる表現です。	
\\	気取ってなんかいないで、本当にどう言っていいかわからないことを聞こうよ。 
\\	の文字通りの意味は「ありのままの自分を出すことを恐れるな」です。その後に、さらに
\\	「気取るのを止める」を続けています。
\\	は「ためらう」の意。	
\\	気長に楽しんでやってますよ。 
\\	で「快適な時を過ごす」の意味になります。	
\\	あの会社のやっていることはきな臭くて危険である。 
\\	「きな臭い」をここでは
\\	「うさんくさい、うたがわしい」、
\\	「嫌な感じがする」で表現しています。	
\\	彼は、全てにおいて、自分の思い通りに事が運ばないと気に入らない人だ。 
\\	ここでは「気に入らない」を 
\\	(我慢できない)で表現しています。
\\	は「対処する、扱う」、
\\	は「計画通りに」の意。	
\\	彼は彼女に敵対心をもっているようで、私が彼女と仲良くしているのが気に入らないようです。 
\\	「気に入らない」は 
\\	で表現すればいいでしょう。	
\\	バースデーカードは気に入ってくれた? 
\\	「気に入る」は
\\	を使えばいいでしょう。
\\	でも同じような意味が表せます。	
\\	彼女に何か気に障ること言ったかな。 
\\	「気に障る、神経に触る」は
\\	で表現できます。
\\	は「触る、不快感を与える」の意。また、
\\	「悩ます」も使えます。	
\\	はるばるアメリカから来てくれたんだよ。時間のことを気にしないで思う存分楽しませてあげたい。 
\\	で「時間のことは気にすることないよ」の意味になります。
\\	でも同じことが言えますね。
\\	は「とても楽しい時を過ごす」の意。	
\\	いちいちそんな事を気にしちゃだめだよ。 
\\	「悩ませる」を使って「気にする」を表現しています。	
\\	そのビデオってパッケージにバラの写真があるやつ? 前からビデオ屋に行くたびに気になっていたけどまだ借りたことない。 
\\	「気になる」は 
\\	(気づく)を使えばいいでしょう。
\\	は「目に留まる、注意を引く」の意味。	
\\	なかなか英語力が伸びないって事は、自分では勉強している気になっているだけで、実はしてないって事だね。要するに、努力が足りないって事だ。 
\\	この場合の「気になっている」は、
\\	(自分では勉強していると思っていた)で表現できますね。
\\	(納得させる)の使い方にも注目。	
\\	時計の音が気になって眠れないよ。 
\\	この場合、
\\	の中に「気になる」が含まれています。	
\\	あいつの行動が、非常に気になる。 
\\	「気になる」は
\\	「関心がある」を使って表現します。
\\	「大きな関心事」の意。	
\\	ずっと前から、あそこで売っているパンが気になっていたんだけど。 
\\	「気になっていた」をここでは
\\	したいとずっと思っていた」と言い換えて表現しています。	
\\	最近、彼のことが気になって仕方ないのは、恋してるせいかしら? 
\\	「気になる」は 
\\	で表現すればいいでしょう。
\\	も使えます。	
\\	記念に写真を撮ってあげるよ 
\\	ここでは「記念」を 
\\	(…によりこの機会を思い出す)を使って表現しています。	
\\	落ち込んだ時には、気の合う仲間と飲むのが一番だね。 
\\	ここでは「気の合う」を 
\\	(同じような気持ちを持った)で表現しています。簡単に 
\\	といってもいいでしょう。	
\\	私の気のせいだったかもしれない。ごめん、何でもない。でも、おかしいな。 
\\	「気のせい」をここでは
\\	「想像する」、
\\	を使って表現しています。	
\\	気のせいか、最近白髪が増えたような気がする。 
\\	「気のせい」を
\\	(名詞)、
\\	(動詞)を用いて表現しています。	
\\	これからさき、気の遠くなるような業務が我々を待ち受けている。 
\\	「びっくりするような」、
\\	「茫然となるような」で「気の遠くなるような」のニュアンスを表現しています。	
\\	彼、交通事故で入院したんですって。お気の毒に。 
\\	「気の毒」は
\\	を使って表現できます。
\\	には「恥」という意味がありますが、ここでは「気の毒な(残念な)事」の意味になります。	
\\	そうは言っても、気は心ですから。 
\\	「気は心」をここでは
\\	「感謝の気持ちを表す」、 
\\	「友情のしるし」を使って表しています。	
\\	大学のキャンパス誘致がこの町の起爆剤になることを願っています。 
\\	ここでは「起爆剤」を
\\	「触媒」、
\\	「いい機会」を使って表現しています。	
\\	彼の記事は、その昔舌鋒鋭かったが、今では牙を抜かれてしまった。 
\\	が「牙を抜く」に当たる動詞です。
\\	は「爪を抜く」、
\\	は「痛烈な」、
\\	は「厳しい」、
\\	は「辛辣な」の意。	
\\	給料前できびしいから今日は飲みに行くのやめとくよ 
\\	で「お金の面で厳しいので」と言っていることになります。
\\	「乏しい」。	
\\	うちのユーザは納期にあまり厳しくないから、ついついだらだらと作業してしまうんだよね。 
\\	「厳しい」は
\\	「厳しい、厳格な」を使えばいいでしょう。
\\	は「締め切り」のこと。	
\\	3カ月もカナダに滞在すれば、英語がぺらぺらになるだろうと思ってたけど、現実は厳しいね。 
\\	「現実は厳しい」を 
\\	で表現しています。	
\\	それは器物破損で訴えるべきよ。そうしないと、やつらは図に乗るばかりだよ。ちゃ んと、善悪をわからせないと。 
\\	「器物破損」は
\\	と言います。
\\	「故意に破損する」。「善悪を分からせる」の英語の言い回しも一緒に覚えておきましょう。	
\\	気分転換に映画でも見に行こうかしら。 
\\	「気分転換」は
\\	でいいでしょう。	
\\	もうすぐ息子のセンター試験があるので、今年はのんびりとお正月気分に浸っていられなかった。 
\\	は「のんびりする、くつろぐ」の意味。
\\	「捧げる、向ける」。	
\\	あの人って本当に気分屋ね。私、ついていけない。 
\\	が「気分屋な」に当たる形容詞です。
\\	について行く」。	
\\	あいつ、またお得意の詭弁で、責任逃れをしようとしてる。 
\\	は「屁理屈(詭弁)」のこと。
\\	「詭弁」。	
\\	お金を借りたら返す。それって基本だろ。 
\\	で「基本の一つ」の意味。
\\	「基本的な」。	
\\	日本の携帯電話は基本料金が高い上に通話料も別に取られる。 
\\	「基本料金」は 
\\	、あるいは、
\\	といいます。「通話料」は
\\	です。	
\\	あいつ、生真面目にもほどがあるよ。 
\\	で
\\	「真面目な」を強調して「生真面目」を表現しています。	
\\	あなたと気まずくなりたくなかったから今まで言えなかったの。 
\\	「気まずくなりたくない」を 
\\	(…とうまくやる、仲良くする)を使って表現しています。
\\	の使い方にも注目。	
\\	詳しい旅行日程が決まりましたのでメールしました。そちらにはお昼ごろ着く予定ですが、空港から直接お宅に行っていいですか? 
\\	を付け足して、後ろから「決まりました」の意味を表現しています。
\\	の代わりに 
\\	を使ってもいいでしょう。
\\	は「旅行日程」の意味。	
\\	彼女の帰国はまだ決まっていない。 
\\	「決まっていない」は 
\\	で表現すればいいでしょう。	
\\	そんなの痛いにきまってるよ。 
\\	「きまっている」のニュアンスは
\\	「痛いと分かっている」で出せますね。	
\\	特に何曜日の何時に働くって決まってないんですよ。暇なときに働かせてもらっているという感じで。 
\\	「決まっていない」を 
\\	(週の特定の日には働いていない)で表現しています。	
\\	もう二度と浮気しないなんて言ったって、また絶対やるに決まってるんだから信じちゃだめだよ。 
\\	「決まってる」を 
\\	(確実なこと)を使って表現しています。
\\	は「確実に」の意。	
\\	彼のアメリカ留学が決まった。 
\\	(…することになっている)で「決まった」を表現しています。	
\\	この洋服って着まわしきくかな? 
\\	は両方とも「合う」の意味ですね。	
\\	最近、風邪気味なんだよねぇ。 
\\	「気味、ちょっぴり」は
\\	で表現できます。	
\\	なぜ、彼女がそんなことを言うの? 君が言うのなら、わかるけど。 
\\	で「君だったら想像がつくけど」の意味になります。	
\\	今度みんなで旅行に行こうと思うんだけど、きみもどう? 
\\	「きみもどう」は 
\\	ですね。具体的に 
\\	の代わりに 
\\	と聞いてもいいでしょう。	
\\	満足している物件には出合えてないけど、今のところ70%ぐらい満足している物件は1つあって決めかねているんだ。 
\\	「決めかねる」は 
\\	(決められない)ということですね。
\\	は「決定をする」の意。	
\\	あの先生の指導はきめ細やかなことで評判だ。 
\\	「きめ細やかな」を 
\\	(詳細な)を使って表現しています。
\\	は「微妙な差異をつけた」の意。	
\\	私が散らかしたって決めつけないでよ。 
\\	で「…だと思い込むのを止める」という意味になります。	
\\	今日こそは絶対(彼に)告白すると決めて行ったのに、取りとめのない話で終わっちゃった。 
\\	で「決心をして…へ行く」の意味。
\\	は「…することになる」ということ。	
\\	今日は、髪型、洋服共にビシッと決めてきたのに、この雨で台無しだよ。 
\\	「ビシッ」を髪型の場合には
\\	「整える」、服装の場合には
\\	「特別な服装をする」を使って表現しています。	
\\	気勢をそがれて、気もそぞろになってしまった。 
\\	「気もそぞろ」とは
\\	「集中できない」ということです。
\\	「やる気」、
\\	「萎える」、
\\	「気が散る」の意。	
\\	もう気持ちが離れちゃったんならしょうがないか。こうなったら早く離婚したほうがいいよ。 
\\	(疎遠になる)を使って「気持ちが離れる」を表現しています。
\\	は「離婚」の意。	
\\	あなたのお気持ちだけで十分です。 
\\	が「気持ちだけで十分だ」に当たる表現です。
\\	「慰める、元気づける、安心できる」。	
\\	今は、気持ちに余裕がないな。 
\\	とは「喜怒哀楽の感情を持つ余裕がない」の意味。	
\\	これでやっと気持ちの整理がつきました。 
\\	何か悲しいことが起きた後で、「気持ちの整理ができた」という時に使うのが
\\	です。	
\\	お気持ちはうれしいのですが、今回は遠慮させていただきます。(招待などに対して) 
\\	(とても嬉しく思う)、
\\	(招待してくれたことをとても感謝する)で「お気持ちはうれしいのですが」のニュアンスを表現しています。	
\\	相手が納得してくれない場合は、納得してくれるまで説明して、お互いが気持ちよく仕事できるように心がけています。 
\\	で「一緒に気持ちよく仕事をする」の意味になります。	
\\	シャワー浴びないの?体ベタベタして気持ち悪くない? 
\\	は「不快な」、
\\	は「不潔な、汚い」の意味。	
\\	胃がムカムカして気持ち悪い。 
\\	は「気分が悪い」、
\\	は「ムカムカする」、
\\	は「むかつく」の意。	
\\	見て、カエルがつぶれてる。ゲッ気持ち悪い。 
\\	「気持ち悪い、むかつく」は形容詞 
\\	で表現できます。
\\	は「むかつく、嫌な」、
\\	は「押しつぶす」の意。	
\\	プライドの高い彼が、そこまで譲ったんだから、その気持ちをくんであげてもいいんじゃない。 
\\	「気持ちをくむ」は 
\\	([その人の]気持ちを考慮する)で表現すればいいでしょう。	
\\	次回は着物の帯の締め方を練習するそうだ。 
\\	「着物の帯」は
\\	でいいでしょう。
\\	の文字通りの意味は「蝶結びをする」。	
\\	キムタクが出てきたとたん、会場の女の子がきゃーきゃー言っちゃって、すごかったよ。 
\\	「かん高い悲鳴を上げる」は
\\	などで表現できます。	
\\	彼は客の扱いがとてもうまい。 
\\	で「客を扱う」の意味になります。
\\	の代わりに 
\\	も使えます。	
\\	この企画について、第三者としてのきみの客観的な意見が聞きたいんだが。 
\\	「客観的な」は
\\	ですね。「第三者」の言い回しにも注目。	
\\	只今キャッシュバック・キャンペーン実施中。 
\\	「払い戻し」が「キャッシュバック」に当たる表現です。	
\\	キャッチホンが入ったから、ちょっと待っててくれるかな。 
\\	「キャッチホン」は和製英語です。「キャッチホンが入る」は
\\	で表現します。	
\\	海外に住むのは、キャリアを積んでからでも遅くないかなと思っている。 
\\	は「確立する」、
\\	は「築き上げる」の意味。	
\\	いつからキャンセル料は発生しますか? 
\\	「キャンセル料」は
\\	と言います。
\\	は「効力を発生する」、
\\	は「受ける、負う」の意。	
\\	占いって気休めに過ぎないと思う。だって、同じ星座、血液型の人は山ほどいるのに皆同じ運命なわけないじゃん。 
\\	「気休め」をここでは
\\	「気休めの言葉」を使って表現しています。
\\	「精神療法」。	
\\	あなたって着やせするからいいわよね。 
\\	「着やせする」を
\\	で「服を着た時にやせて見える」で表現しています。
\\	「うらやむ」。	
\\	海外で暮らすようになってから、日本で生活することがいかに窮屈であったかがわかった。 
\\	は「息苦しい、窮屈な」の意味。
\\	(圧迫される)。	
\\	今までは左側には誰もいなかったのに、席替えをして両端に人がいるのでなんか窮屈だ。 
\\	ここでは「窮屈」を
\\	([ひじを自由に動かす]余裕がない)で表現しています。
\\	は「狭苦しい」の意。	
\\	のテストって、間に休憩があればいいのになあ。長すぎるよ。 
\\	この場合の「休憩」は
\\	ですね。
\\	は「途中で」の意。	
\\	1限目の授業休講なんだって。もっと早く知らせてくれよ。 
\\	「休講にする」は 
\\	、あるいは 
\\	で表現できます。	
\\	「わざと急所を外したな。」///「お互い様だろ。しかし、お互い助からねえ。」 
\\	「急所」は
\\	「致命的な場所」で表現できます。	
\\	今年が最後のチャンスかもしれないし、今しかできないからって相談したら、会社が1ヵ月休職する事を許してくれました。 
\\	で「一ヶ月の休暇を取る」という意味になります。
\\	に一ヶ月の休暇を与える」。	
\\	クラス毎に給食当番がいて、その人たちが配膳から片づけまでやります。 
\\	「給食当番」を
\\	で説明しています。	
\\	あなたは休日何をしていますか? 
\\	ここでは「休日」を 
\\	で表現しています。	
\\	与党議員への疑惑追及の急先鋒だった野党議員が、みずから墓穴を掘って議員辞職した。 
\\	ここでは「急先鋒」を 
\\	(先頭に立つ)、
\\	(…の先頭になって)で表現しています。	
\\	急な用事ができちゃって、今日の飲み会行けないんだけど。 
\\	「急用」は動詞に
\\	を使うといいでしょう。「飲み会」は
\\	という言い方をします。一緒に覚えておきましょう。	
\\	あ!そうだ。急に思い出したけど、そういえば昨日の野球の試合見た?なかなか見ごたえがある試合だったよ。 
\\	「急に思い出す」は 
\\	を使えばいいでしょう。
\\	は「思い浮かぶ」の意。	
\\	今、急にラーメンが食べたくなった。 
\\	で「突然… が欲しくなる」の意味。	
\\	せつなくって胸がキュンってしちゃった。 
\\	(キュッと締まる)、
\\	(絞る、悩ます)を用いて「キュン」のニュアンスを出しています。
\\	は「(心の)苦しみ」、
\\	は「あこがれ」の意。	
\\	凶悪犯罪が多発している。物騒な世の中になったものだ。 
\\	で「ぞっとするような(ひどい)犯罪」という意味になります。
\\	「残虐な」。	
\\	うちのクラスのがり勉の田中さんのお母さんって、やっぱり教育ママなんだってね。 
\\	は「教育のことで頭が一杯の(とりつかれている)」という意味になります。	
\\	このアパートは家賃の他に、水道代、電気代、ガス代、そして共益費が必要になります。 
\\	、または、
\\	が「共益費」に当たる表現です。	
\\	この間テレビで、女性がどこまで化粧なしで出かけられるかその境界線を調べてみようってやってた。 
\\	「境界線」は 
\\	と言います。
\\	は省略できます。	
\\	焼き肉が食べたいけど、狂牛病が心配だからやめておこう。 
\\	「狂牛病」は 
\\	と言います。
\\	は 
\\	(牛海綿状脳症)の頭文字語です。	
\\	今日こそ、きっちりカタつけるぞ。 
\\	「今日こそ」は
\\	「今日がその日だ」と言えばいいでしょう。
\\	「解決する」。	
\\	とても素敵なホテルでやっとイタリアを満喫できると思っていたら、エレベーターで「何階ですか?」って日本語で聞かれ興ざめした。 
\\	が「興ざめ(楽しさを台無しにする)」に当たる言い回しです。
\\	は「典型的な、神髄の」の意。	
\\	デジタル・デバイドとは、情報強者と情報弱者の間に大きな所得格差が生じることである。 
\\	ここでは「コンピュータ強者」を
\\	「コンピューターに精通した、豊富な知識を持った人達」で表現してみました。	
\\	最近の教習所って
\\	とか使ってるんだね。私の時代にはそんな高級車使ってなかった。時代は変わったね。 
\\	「教習所」は
\\	と言います。
\\	は「贅沢な、豪華な(形容詞)」の意。
\\	はその名詞形。	
\\	お花までいただいてしまって、恐縮です。 
\\	「恐縮する」を
\\	「本当に恩義を感じる」、
\\	「それほどのことをあなたにはしていないので恥ずかしく思う」で表現しています。	
\\	あの人のように気を使われると、かえってこちらまで恐縮してしまう。 
\\	ここでは「恐縮する」を
\\	「かえってこちらが意識(気を使って)しまう」と言い換えて表現しています。	
\\	英語の発音に強弱をつけるのは難しい。特に、私発音だめだから。 
\\	「強弱」はそのまま
\\	を使って表現します。	
\\	あっ!パソコンがストールした。強制終了しよう。 
\\	「強制終了」は
\\	、あるいは、
\\	「電源を切る」と言います。	
\\	強制的に一度パソコンを立ち上げてみよう。 
\\	は名詞で「強制的立ち上げ」の意味。動詞
\\	を用いても同じようなことが言えます。	
\\	競争相手がいないと、なんか張り合いがない。 
\\	「競争相手」は
\\	「ライバル」を使えばいいでしょう。	
\\	子供の頃、よく兄弟げんかをしたものだ。 
\\	これはあまり難しく考えずに
\\	を使えばいいでしょう。	
\\	日本の企業では、いかに協調性があるかが重要視されている気がします。 
\\	は形容詞で「協調性のある」、
\\	は名詞で「協調性」の意味。	
\\	海外旅行が唯一の共通趣味。後は、ほとんど違う。見るテレビとか。 
\\	は「分かち合う」、
\\	「共通して持っている」の意味になります。
\\	「などなど」。	
\\	それらの事件の犯人にはいくつかの共通点があったらしい。 
\\	で「共通した特徴」の意味。
\\	で
\\	と共通した何かがある」。	
\\	今日で、二十歳になりました。 
\\	「今日で、二十歳になる」は
\\	で初めて、その後、現在形を使って
\\	でいいでしょう。ちなみに、
\\	という言い方はしません。	
\\	彼が被害者ということはあり得ず、むしろ共同正犯の可能性だってあり得る。 
\\	「共同正犯(共犯)」を 
\\	(主犯の1人として同等の責任を有している)で表現しています。
\\	は「連帯の」の意。	
\\	怖くて怖くて仕方がないの。何も怖いことなんてないって頭ではわかっていても、この恐怖感はぬぐいきれない。 
\\	「この恐怖感」は 
\\	で表現できます。
\\	の代わりに 
\\	を使ってもいいでしょう。	
\\	うちの子最近凶暴になってきたのよね、すぐ手を上げるのよ。 
\\	で「ますます凶暴になる」の意味。
\\	は「身体に虐待を加える」の意。	
\\	私は英語があまり得意じゃありませんが、外国の文化にとても興味があります。 
\\	「…に興味がある」は
\\	でいいでしょう。	
\\	たとえ親が無理矢理に子どもに英語を慣わしても、子どもが自分から興味を持たない限り続けるのは難しいと思うな。 
\\	で「個人的に…に興味を持つ」の意味になります。
\\	は「続ける」の意。	
\\	その言葉はスラングで、教養のある人は使わないって聞きました。 
\\	「教養のある」は 
\\	で表現します。	
\\	アンケート調査にご協力頂きますよう何卒よろしくお願いいたします。 
\\	で「協力に感謝する」の意味。
\\	は「アンケート」、
\\	「助力」の意。	
\\	昨日の巨人・阪神戦は、虚虚実実の駆け引きが見物だった。 
\\	で「駆け引き(戦術)と知力の戦い」で「虚々実々の駆け引き」を表現しています。
\\	は「複雑な戦術面のやりとり」、
\\	とは「見物(みもの)」の意味ですね。	
\\	局地的な大雨だったらしいけど、行かなくて正解だったよ。 
\\	「局地的」は
\\	「その地域の」を使って表現できます。	
\\	極力、レンズは交換しないようにしてください。 
\\	「極力」を
\\	「できるのなら」と言い換えています。	
\\	彼女の虚言癖はどうにかならないものかなあ。もううんざりだよ。 
\\	は「病的なまでに嘘をつくこと」の意味。	
\\	この案に賛成の方は挙手してください。 
\\	が「挙手をお願いします」に当たる言い回しです。もちろん、
\\	とも言います。	
\\	無理なダイエットをした結果、拒食症になり、彼女は今入院している。 
\\	で「拒食症になる」の意味。
\\	の名詞形は
\\	(拒食症)。
\\	は「猛烈な」の意。	
\\	拒食
\\	面接では、 虚心坦懐に話すことが必要だ。 
\\	「虚心坦懐話す」とは「率直に話す」ことですから、
\\	「隠し隔てのない」、
\\	「誠実な」、
\\	「率直な」などを使って表現できます。	
\\	時々、英語に対して拒絶反応するんだよね。きっと、そういうときってスランプに陥っているのだと思うけど。 
\\	ここでは「拒絶」を 
\\	(はねつける、拒絶する)で表現しています。
\\	は「嫌悪」、
\\	は「マンネリ化する」の意。	
\\	日本語がわからないから、彼ったらきょとんとして私たちの顔を見てたわ。 
\\	は「当惑した」、
\\	は「あぜんとした」の意味。	
\\	手先が器用で細かい作業に向いているということで彼女が選ばれた。 
\\	が「器用」に当たる語です。
\\	でも「手先が器用な」の意味になります。	
\\	彼のこと嫌いじゃない。 
\\	は
\\	のこは全然嫌じゃない」の意味。	
\\	イギリス英語は嫌いなわけではないけど、私にとっては聞きづらく難しく感じるだけです。 
\\	で「…が嫌いではない」の意味になります。
\\	と
\\	の使い方にも注目。	
\\	あなたの嫌いな食べ物は何ですか? 
\\	「嫌う」は
\\	で表現できます。
\\	は「…するのが耐えられない」の意味。	
\\	嫌いな食べ物って、小さい頃に毎日のように食べさせられたから嫌いになったって人が多いよね。 
\\	「(私の)嫌いな食べ物」は
\\	で表現すればいいでしょう。	
\\	彼女は映画の中でしか見たことないけど、実際に会ったら全然イメージが違って嫌いになるかもね。 
\\	には「興味をなくす」という意味があります。
\\	でも「嫌いになる」は表現できますね。	
\\	二人とも今は舞い上がっているけど、会うは別れの始め、ってこと知っているのかね。 
\\	「会うは別れの始め」を
\\	で文字通りに説明しています。
\\	「うっとりしている」。	
\\	彼、今日、精彩がないね。青菜に塩だね。 
\\	「青菜に塩」とは「しおれている様子」のことですから、
\\	英訳全体でこの意味を表していることになります。	
\\	親父も立派な経営者だったけど、息子の活躍もすごいね。まさに青は藍より出でて藍より青しだね。 
\\	「青は藍より出でて藍より青し(先生より生徒の方が優れている)」を
\\	「勝る」、
\\	を使って文字通りに表現しています	
\\	彼をやり込めるくらい、赤子の手を捻るより簡単だよ。 
\\	とは「赤ちゃんから飴を取る」ということです。それくらいにたやすいことだと言っているのです。日本語にかなり近い言い回しですね。
\\	「バケツの中にいる魚を釣る」。	
\\	うまくいっていたあのふたりも、とうとう秋風が立ち始めたね。 
\\	「秋風が立つ(相手が嫌になる)」を
\\	「よそよそしさ(冷淡さ)を感じる」で表現しています。	
\\	もう辺りが暗くなってきた。秋の日は釣瓶落としだね。 
\\	は「(岩のように)素早く沈む」、
\\	は「やって来る」の意味。	
\\	都合の悪いこと、そうそう隠しとおせるもんじゃないよ。悪事千里を行く、っていうだろう。 
\\	は「悪い知らせはすぐに伝わる」という意味です。	
\\	悪銭身につかず、というじゃない。結局は地道に稼ぐしかないと思うよ。 
\\	とは「不正な手段で得た金(利益)ははかないものである(すぐになくなる)」の意味。
\\	「あぶく銭」。
\\	でも「悪銭身につかず」の意味になります。	
\\	人の揚げ足を取るばかりじゃなく、もっと建設的な意見を言ってくれないかな。 
\\	「あら探しをする、重箱の隅をつつく」。	
\\	"朝寝坊の宵っ張りか身上の俺としては、飲み会は朝まて付き合うよ 
\\	「朝寝坊の宵っ張り」を文字通りに英訳すると
\\	""になります。
\\	コンピュータのトラブル処理なんて、彼にとっては朝飯前だよ。 
\\	「朝飯前」は 
\\	(とても簡単なこと)で表現できます。	
\\	休日なのに出勤なんて足が重くなるよ。 
\\	で「ぐずぐず歩く、足を引きずって歩く」の意味。	
\\	失敗したからってそんなにクヨクヨするな。明日は明日の風が吹くさ。 
\\	は 
\\	とも言えます。	
\\	あんな汚い商売からは足を洗ったよ。 
\\	「足を洗う」は英語では 
\\	(手を洗う)と言います。
\\	は「ペテン、信用詐欺」の意。	
\\	彼は一度キセルが成功したことに味をしめて、常習犯になった。 
\\	で「…の味を覚える」の意味。ここでは「キセル」を 
\\	(電車代をごまかすこと)で表現しています。	
\\	当たって砕けよが俺の持ち味だからね。 
\\	は「思い切ってやってみなければ、何も得られない」の意。
\\	「とことんやってみる」。	
\\	人のお節介などしないで、自分の頭の上の蝿を追ったらどうなんだね。 
\\	で「他人事に鼻をを突っ込む」の意。	
\\	君の指摘、中らずといえども遠からずだね。 
\\	「的が少し外れている」と
\\	「近い」を組み合わせて表現しています。	
\\	いまさら嘆いても、それこそ、後の祭だよ。 
\\	「後の祭り」は
\\	「手遅れ」でもいいのですが、ここでは
\\	「今となっては(後悔しても)何のためにもならない」で表現しています。二番目の英文の
\\	「彼女は戻って来ないよ(たとえて言っている)」の意味。	
\\	後先考えず、とりあえずやってみるか。後は野となれ山となれ、だよ。 
\\	で「後のことは誰も気にしない」の意味。	
\\	いやあ面目ない。穴があったら入りたい心境だよ。 
\\	で「穴に入る」の意。
\\	「消える」、
\\	「目に見えない」。	
\\	顧客にあの手この手で商品を売り込んでみたが、まったく興味を示さなかった。 
\\	(あらゆる策を試す)で「あの手この手」を表現しています。	
\\	彼はあんなわがままな人とよく結婚したね。まあ、あばたもえくぼってやつだね。 
\\	で「人それぞれだから」の意味。 
\\	「恋は盲目」でも「あばたもえくぼ」の意味になります。	
\\	彼の会社、最近結構危ない橋を渡っているって、もっぱらの噂だよ。 
\\	で「とても危ない」の意。
\\	「生きるか死ぬか(倒産するか)の瀬戸際にある」、
\\	「綱渡りをする(危ない状況にある)」。	
\\	そんなに欲をかいていると、虻蜂取らずになっちゃうよ。 
\\	「虻蜂取らず」とは「欲をかくと何も手に入らない」ということですから、このニュアンスは
\\	「結局何も残らない」で表現できます。
\\	「欲張る」。	
\\	プラプラ油を売っていないで、とっとと仕事にもどったらどうなのさ。 
\\	で「ぶらぶらしながら無駄話をする」の意。
\\	「時間を無駄にする」。	
\\	いろいろトラブルもあったけど、雨降って地固まるってことになってほしいね。 
\\	「地固まる」とは
\\	をより強くする」ということですね。
\\	「大変な思いをする」。	
\\	二人の沈黙はなんなんだ。これが嵐の前の静けさってやつか。 
\\	が「嵐の前の静けさ」にあたる言い回しです。	
\\	あんな大失敗をしてしまって上司に合わせる顔がない。 
\\	この場合の
\\	は動詞で「顔を向ける、直面する」の意味。	
\\	いつまでも思い悩まないで、とりあえず動いてみたら。案ずるより産むが易しだよ。 
\\	「案ずるより産むが易し」には
\\	という言い方があります。
\\	「何もしないことから来る不安の方が、実際に直面する危険の方よりもずっとひどい」でも同じようなことが言えますね。	
\\	彼女は試合に勝って安堵の表情を浮かべた。 
\\	は「喜びでホッとした表情」の意味。	
\\	彼、能書きばかり立派だけど、言うは易く行うは難し、ってこと分かってんのかなあ。 
\\	という言い方もありますが、ここでは
\\	「実際にやることがどれだけ体験か」で表現しています。
\\	「自己宣伝、激賞」、
\\	「楽天的な説明」。	
\\	わが社は倒産寸前だったが、このヒット商品のおかげで息を吹き返した。 
\\	「息を吹き返す」を 
\\	(…に新しい息吹を吹き込む)で表現しています。	
\\	石の上にも三年、というじゃない。もう少し辛抱しなさいよ。 
\\	「石の上にも三年」とは「辛抱すればうまくいく」という意味ですから、このことわざのポイントは
\\	「辛抱する」と
\\	「根気強い」で言い表せます。
\\	「時間がかかる」、
\\	「成し遂げる」。	
\\	彼は何事にも慎重で、まさに石橋をたたいて渡るタイプだ。 
\\	で「細心の注意を払う」という意味になります。	
\\	あの先生タバコで肺がんになったんだって。医者の不養生、そのものだね。 
\\	は「私の言う通りにしなさい。私のやっていることをしてはいけない」の意味。
\\	「医者の指示に従わない」	
\\	あの美人姉妹、いずれ菖蒲か杜若、二人とも優劣つけがたいね。 
\\	「菖蒲か杜若」を
\\	「ユリ」と
\\	「バラ」を使って表現してみました。
\\	と
\\	で表現できないこともありません。でも、やはり、美しさを言っているのですから、花を使った方がいいでしょうね。	
\\	面倒なようでも、基本に立ち返って検討し直したらどうだろう。急がば 廻れ、というじゃない。 
\\	で「あわてると無駄になる」の意味。
\\	という言い方もあります。	
\\	痛くもない腹を探られるのはやりきれないね。 
\\	で「嫌疑をかけられるなんて信じられない」の意味になります。この後に
\\	「何もしていないのに」を付け足しています。	
\\	年の功というのか、彼の意見はさすがに一日の長があるね。 
\\	で「他の人に比べより経験に基づいている」の意味。	
\\	一難去ってまた一難、またまた難題が社長からトップダウンで降りてきたよ。 
\\	「一難去ってまた一難」は
\\	とも言えますが、ここでは
\\	するやいなや」を使って表現しています。	
\\	"第二子は男の子だって。これでめでたく一姫二太郎ってわけだね。 
\\	日本語をそのまま英語にすると
\\	""となります。
\\	一を聞いて十を知る、とまではいかないまでももうちょっと物分りがよくなってほしいよ。 
\\	で「少しの手がかりで、全体の状況が理解できる」の意味。
\\	「見通しのきく人」。	
\\	彼あんなに元気だったのに。突然亡くなるなんて。一寸先は闇だね。 
\\	とは「次に何が起こるか本当に分からない」の意味です。
\\	「いつ死ぬことになるか」。	
\\	いつもお世話になっていますから、一旦緩急あれば、すぐさま馳せ参じます。 
\\	で「窮境にある」の意味。	
\\	彼はうちの社員の中でも一頭地を抜く存在として一目置かれている。 
\\	「一頭地を抜く存在」とは
\\	「高くそびえ立つような人物(他より抜きんでている)」ということですね。	
\\	この中古カメラ、すぐに動かなくなったよ。店員に一杯くわされた。 
\\	(かつがれる、口車に乗せられる)で「一杯くわされる」を表現しています。	
\\	残念ながら、サッカーの試合、一敗地に塗れてしまったよ。 
\\	「一敗地に塗れる」とは「2度と立ち上がれないほどにやられる」という意味ですから、
\\	「こてんぱんにやられる」、
\\	「惨敗」を使えばその意味が表せるでしょう。	
\\	事故で車は全損って嘆いているけどさ、怪我がなくてなにより、命あっての物種だよ 
\\	で「生きていることに感謝するべき」の意。
\\	「全損になる、完全に壊れる」。	
\\	世間知らずにもほとがある。井の中の蛙大海を知らず、ということかね。 
\\	で「世間知らず」の意味になります。
\\	「何も分かっていない」、
\\	「井戸」	
\\	あの2人、いつもけんかばかりしてるけど本当に仲悪いのかな?嫌よ嫌よも好きのうちってこともあるしさ。 
\\	(本当に好きな物を嫌う)を使って「嫌よ嫌よも好きのうち」にニュアンスを出しています。	
\\	もっと経費を抑えろよ。入るを量りて出るを為す、ということを知らんのかね。 
\\	「入るを量りて出るを為す」とは
\\	「お金を使う前に、どれだけ収入があるのかを見る」ということですね。
\\	「収入」。	
\\	色気より食い気からはそろそろ卒業してもいいお年頃だと思うけどね。 
\\	「色気より食い気」には「形より実質」という意味がありますが、ここでは文字通りの意味で使われていますので、そのまま
\\	「容姿、見てくれ」と
\\	を使って英訳しています	
\\	いくら金持ちといっても、アラブの石油王に比べたら…。上には上があるよ。 
\\	の文字通りの意味は「いつでももっと持っているやつがいる」です。	
\\	連中、頭数だけは揃えたようだけど、所詮は烏合の衆。こちらは僕一人で論破してみせるよ。 
\\	とは「ただうるさいだけの群衆」ということです。
\\	「群」、
\\	「理詰めで説き伏せる」。	
\\	あいつにいくらお説教をしてもむだだよ。牛に経文なんだから。 
\\	「牛に経文」とは
\\	「聞かない」、
\\	「耳を傾けない」ということですね。	
\\	後ろ指を指されるようなことなど、決してしていないよ。 
\\	「後ろ指を指される」とは
\\	「陰口を言う(言われる)」ということですね。	
\\	患者を安心させるには多少のウソも必要かもしれない。ウソも方便ということもあるしな。 
\\	は「罪のない嘘が必要な時もある」の意味。
\\	「無理に解釈する」。	
\\	あんなウダツがあがらない男といつまで付き合ってるの。 
\\	「ウダツがあがらない奴」を
\\	「失敗者、ださい奴」、
\\	「将来性のない奴」で言い換えています。	
\\	腕によりをかけて作った自慢の料理よ。さあ召し上がれ。 
\\	で「自分の持つ技術をすべて使う」の意味。	
\\	彼はいいね。打てば響くようにすぐ動いてくれるよ 
\\	「打てば響く」とは
\\	「物わかりが早い」、
\\	「反応がいい、敏感な」ということですね。	
\\	奴はうどの大木なんかじゃないよ。本当に頭の切れる奴だ。 
\\	とは「でくの坊、のろま」のことです。	
\\	営業成績はまさに鰻上り、というところかね。 
\\	は「急騰する」、
\\	は「どんどん(螺旋状に)上がる」の意味。	
\\	彼に忠告したところで、馬の耳に念だよ。 
\\	とは「煉瓦の壁」という意味。文字通りに英訳すると第2文のようになります。これでも意味は通じるはずです。	
\\	奇襲攻撃で相手の裏をかくっていうのはどうだ。 
\\	で「まさに我々がやらないだろうと相手が思っていることをする」の意味になります。
\\	は「奇襲攻撃」の意。	
\\	彼のせりふは割り引いて聞いてよ。単に売り言葉に買い言葉というだけなんだから。 
\\	で「彼は、君の言ったことに対して言い返してただけなんだから」の意。
\\	「仕返し」。
\\	「報復的行為の」。	
\\	お受験ママもいいかげんにしたら。瓜の蔓に茄子はならないよ。 
\\	とは「そのような人物になる素養(下地)が無いのに無理してもどうしようもない」の意味。	
\\	噂をすれば陰。あいつが来たぞ。 
\\	が「噂をすれば陰」に当たる諺です。	
\\	レースに勝つためにできるだけの準備はした。あとは運を天に任せるだけさ。 
\\	で「残りは運命に任せる」の意味になります。	
\\	君の夢一杯のビジネスプラン、絵に描いた餅にならんことを切に望むよ。 
\\	「絵に描いた餅」には
\\	という言い方がありますが、
\\	「希望的観測」でも同じようなことが表現できます。	
\\	文句は言わず縁の下の力持ちだと思って頑張ろうよ。 
\\	(賛美されないヒーロー)で「縁の下の力持ち」を表現しています。	
\\	大関との対決を制した横綱がいよいよ優勝に王手をかけました。 
\\	で「…は確実である」の意味。	
\\	もう大丈夫。大船に乗った気で心配しないで。 
\\	「大船に乗る」とは
\\	「僕に100パーセント頼る」ということですね。
\\	「頼りれる人と一緒にいる」。	
\\	当事者の君は彼を結構評価しているけど、傍目八目とはいえ、私の厳しい見方があたっているかもね。 
\\	「傍目八目」とは
\\	「(当事者よりも)物事が客観的に見られる」ということですね。
\\	「慈悲/情け深い」。	
\\	同じことを何回繰り返すの。屋上屋を架すようなくどい説明は勘弁してよ。 
\\	「屋上屋を架す(無用のことをする)」をここでは
\\	「時間を無駄にする」と
\\	「長たらしい」を使って表現しています。	
\\	いつまでダラダラ小田原評定してるの。とっとと会議の結論を出しなさいよ。 
\\	「小田原評定」とは「長引いてなかかな決まらない相談(会議)」のことですから、ここでは
\\	「だらだらと果てしなく会議を続ける」を使ってその意味を表しています。	
\\	そんな言いわけでお茶を濁そうったってそうはいかないよ。 
\\	(ごまかす)を使って「お茶を濁す」を表現しています。
\\	の代わりに
\\	も使えます。
\\	は「言いわけの」の意。	
\\	男は度胸、女は愛嬌だ。男なら当たって砕けろよ! 
\\	「度胸」は 
\\	(勇気)、「愛嬌」は 
\\	(魅力、可愛らしさ)で表現すればいいでしょう。
\\	は「一か八かやってみる」の意。	
\\	鬼が出るか蛇が出るか。この提案書の結果が見物だね。 
\\	で「この提案書がどのようなトラブルを引き起こすかは分からない」の意味になります。	
\\	君がここにいてくれるなら、なにも怖いものなし。鬼に金棒だよ。 
\\	とは「要塞」のことです。
\\	「かわす」。	
\\	課長の奥さん、今旅行中なんだって。それで、鬼のいぬ間の洗濯とばかり、毎日飲み歩いているんだ。 
\\	の文字通りの意味は「猫がいない時に、ネズミは遊ぶ」。	
\\	あの元気な男が病気で倒れるなんて、鬼の霍乱ともっぱらの噂だよ。 
\\	で「悪魔が日射病になるようなものだ」の意。	
\\	霍乱
\\	そんな些細なことで、鬼の首を取ったようにはしゃがなくってもいいじゃない。 
\\	で「征服した英雄のように」の意。	
\\	相手チームから1点を取っただけで、うちのチームは鬼の首を取ったような喜びようだ。 
\\	は「意気揚々とした、勝ち誇った」の意味です	
\\	尾羽打ち枯らした彼の姿を見るのは忍びないね。 
\\	で「落ちぶれ果てた」の意。
\\	「耐え難い」。	
\\	この歌お前の十八番なのに、先に歌われちゃったね。 
\\	は「得意、得手」の意味。
\\	「機先を制する」。	
\\	思い立ったが吉日、さっそくプロジェクトの企画案を書き始めよう。 
\\	「徹底的に追求する」と
\\	「思いつく」の使い方がポイント。
\\	「ぐずぐずする」。	
\\	自殺なんてするもんじゃない。親に先立つは不孝だよ。親の気持ちも考えろ。 
\\	で「子どもに先立たれる」の意味になる。ここでは「不幸」を 
\\	(ひどいこと)で表現しています。	
\\	あの子はどれだけご両親が心配しているのかわかってるのかしら。親の心子知らずね。 
\\	「親の心子知らずね」を 
\\	(子どもってそういうもんだよ)で表現しています。	
\\	彼のプレゼンのうまさは折り紙付きだ。 
\\	とは「すでに立証済みの」ということです。	
\\	ちょっと待ってよ、そんなに単純に、終わりよければ総て良し、ってわけにはいかないよ。 
\\	これには
\\	という決まった言い方があります。2番目の英文は、ハイフンをつけて形容詞的に使っています。	
\\	うちは3人姉妹だから、にぎやかだよ。女三人寄れば姦しいよね。 
\\	で「女が三人集まれば大騒ぎ」という意味になります。	
\\	あんなに目をかけていた部下がライバル会社に転職するなんて、飼い犬に手を噛まれた気分だよ 
\\	「飼い犬に手を噛まれる」は英語でも
\\	と言います。	
\\	この飲み屋は顔が利くから、安心して飲んでいいよ。 
\\	「顔が利く」とは
\\	「有名である、よく知られている」ということですね。	
\\	ここにいる人のほとんどが彼の友達らしいよ。本当に顔が広いね。 
\\	「顔が広い」とは
\\	「友達の輪が広い」ということですね。	
\\	学生時代のドジ話をみんなの前でされて、顔から火が出るほど恥ずかしかった。 
\\	で「恥ずかしさで顔が真っ赤になる」の意味になります。
\\	「へま、大失敗」。	
\\	お前、よくもおれの顔に泥を塗るようなまねをしてくれたな! 
\\	「顔に泥を塗る」とは
\\	「(相手の名を)汚す」ことですね。	
\\	あいつ、俺の顔に泥を塗りやがって。 
\\	「顔に泥を塗る」とは
\\	(面目を失う)ということですね。	
\\	このクラブの店長と知り合いだから顔パスで入れるよ。 
\\	で「僕たちのことを見たら、すぐに入れてくれる」という意味。	
\\	今日の打ち上げだけど、仕事が終わったら8時くらいに顔を出すよ。 
\\	「顔を出す」とは
\\	(現れる)、
\\	(顔を見せる)ということですね。	
\\	学問に王道なし。毎日、地道に勉強することだね 
\\	ここでは王道を
\\	「近道」で表現しています。	
\\	さあ、駆けつけ三杯、威勢よくぱっといきましょう。 
\\	には「一気に飲み干す(動詞)、一気のみ(名詞)」という意味があります。駆けつけ三杯は日本特有の(悪しき?)習慣でしょう。	
\\	型にはまったつまらない企画ばかり出すなよ。 
\\	の文字どおりの意味は「古い鋳型」です。	
\\	暑いからってだらけすぎだ。熱いシャワーでも浴びて活を入れるか。 
\\	(再び元気づかせる)を使って「活を入れる」を表現しています。
\\	は「無気力な」の意。	
\\	どんな汚い手を使っても、勝てば官軍だ。 
\\	ここでは 
\\	(勝者がいつも正しい)で「勝てば官軍」を表現しています	
\\	鉦や太鼓で探しあてた割には、あいつ期待はずれだったよな。 
\\	「鉦や太鼓で探す」とは「大騒ぎをして探す」ことですから、
\\	「大変な思いをして探す」を使って表現すればいいでしょう。
\\	「努力」。	
\\	金が物言う、世知辛いこのご時世に、彼の生き方は一服の清涼剤だね。 
\\	「金が物を言う」は英語でも同じように
\\	と言います。
\\	「厳しい、残酷な」、
\\	「けちな」。	
\\	金に糸目を付けないから、いい人材を紹介してよ。 
\\	で「お金は問わない」の意味。
\\	「惜しむ」。	
\\	つまらんことにいらいらしないで、金持ち喧嘩せずの精神をもったらどうだい。 
\\	文字通りに訳せば
\\	となりますが、ここでは
\\	「~する余裕がある」を使っています。
\\	で「(余裕があるので)いちいち議論をしない」の意味。	
\\	しーっ!壁に耳あり障子に目ありよ。会社の中で社長の悪口なんて言わないほうがいいわ。 
\\	が「壁に耳あり」に当たる言い回しです。障子を使った表現は英語にはありません。
\\	で「中傷する」の意。	
\\	できることはすべてやったよ。後は、果報は寝て待て、の心境だね。 
\\	で「それがどうなるか見守る」の意味になります。	
\\	雷を落とす 授業に遅れて教室に入ったら、教授の雷が落ちた。 
\\	(…をひどく叱る)を用いて「雷が落ちる」を表現しています。	
\\	僕も結構風呂の時間が短いけれど、君は本当に烏の行水だね。 
\\	で「入ったらすぐに出てくる」の意。
\\	「(水を)パシャッとかけること」。	
\\	この子ったらうちでは走り回っているのに、おじさんの所では借りてきた猫のようにおとなしかったわ。 
\\	で「(子羊のように)おとなしい、従順な」の意味になります。
\\	(とても静か)も使えます。	
\\	なかなか面白い映画だが、クライマックスがいまひとつで画竜点睛を欠いている。 
\\	で「最後の場面での詰めを欠いていた」の意味になります。	
\\	何の役にもたたないけど、枯れ木も山の賑わいでよければパーティに参加するよ 
\\	比較級、
\\	比較級)で「多ければ多いほど、より楽しい」の意味になります。第2文では、ハイフンを付けて形容詞として用いています。	
\\	可愛い子には旅をさせた方がいいのよ。今の内に少しは苦労させた方が本人のためよ。 
\\	「放り出す」、
\\	「苦しいこと」、
\\	「その人にとっていい、利益になる」を使って表現しています。
\\	「思い切って外に出す」、
\\	「厳しい」。	
\\	ボスの経費チェックは厳しいね。まさに、眼光紙背に徹す、だよ。 
\\	「眼光紙背に徹す」とは
\\	「行間を読む(深い意味を読みとる)」ということですね。
\\	「洞察力が鋭い」。	
\\	あの店は閑古鳥の鳴くわびしだ 
\\	彼のビジネスプランは立派だけど、勘定あって銭足らず、という気もするけどね。 
\\	「紙の上」と
\\	「実際」を使って「勘定あって銭足らず」の意味を表現しています。	
\\	彼とは学生時代から肝胆相照らす仲だよ。 
\\	「肝胆相照らす仲」とは
\\	「親友同士」ということですね。	
\\	チャンピオンは自分の敵は自分だって言ってた。眼中人なしって感じ。 
\\	この場合の
\\	は「重要である、問題である」の意味になります。	
\\	もう我慢できん。堪忍袋の緒が切れた。 
\\	「堪忍袋 
\\	は
\\	「怒る、切れる」を使えばいいでしょう。	
\\	間髪を入れず、迅速かつ的確な応対、畏れ入ります。 
\\	で「稲妻のように速い」で「間髪入れず、迅速」を表現しています。
\\	「ことばで言い表せない」。	
\\	あの男鳴り物入りで入社したけど、実力のほど看板に偽りなしだったね。 
\\	で「(期待に)こたえる、沿う」。
\\	「派手な歓迎(宣伝)」。	
\\	休日に出社しなきゃいけないなんて気が重いな。 
\\	「気が重い」を 
\\	(気が滅入る)で表現しています。	
\\	もうやめろよ。これ以上傷口に塩を塗るようなことは彼に可哀想だ。 
\\	で「一層ひどい思いをさせる(傷口に塩をすり込む」の意味。	
\\	お前はほんとに着たきり雀だな。そろそろ新しい服を買ったらどうだい? 
\\	「着たきり雀」とは 
\\	(いつも同じ服を着ている)ということですね。	
\\	木で鼻をくくったその返事の仕方、相手に失礼だと思わんのかね。 
\\	「木で鼻をくくたような」とは
\\	「ぶっきらぼう、そっけない」ということですね。
\\	「無愛想な」。	
\\	よくそんな人の気に障るようなことが言えるな。 
\\	「気に障る」を 
\\	(嫌な思いをさせる、悩ませる)で表現しています。	
\\	あいつ追いつめられたら何するか分からないぞ。窮鼠猫を噛むってこともあるからな。気を付けるに越したことはない。 
\\	は「どうしようもなければ、運のつきたネズミは猫にも噛みつく」の意味。
\\	「牡鹿」、
\\	「敵」。	
\\	いけすかないボスだけど、仕事の悩みを相談したら案外、窮鳥懐に入れば猟師も殺さず、てなことになるかもね。 
\\	「窮鳥懐に入れば 
\\	を文字通りに
\\	英訳しています。	
\\	本命と対抗馬が競り合ったお陰で、ダークホースの彼が漁夫の利を得たってわけだ 
\\	共に「(相手より)先んじる、前に出る」の意味。	
\\	清水の舞台から飛び降りるつもりで勝負にでるよ。 
\\	「全力を傾ける」、
\\	「駆り立てる」、
\\	「一か八かの戦い」を使って「清水の舞台... 
\\	を表現しています。
\\	「深い方」、
\\	「勝者がすべてを取る」。	
\\	彼は細かいことには気がつくけど、木を見て森を見ずだな。大局的見地がない。 
\\	が「木を見て森を見ず」に当たる言い回しです。「森を見て木を見ず」は 
\\	といいます。	
\\	この業界は食うか食われるかの世界だ。気を抜くな。 
\\	で「食うか食われるか」の意味になります。
\\	は「警戒する」の意。	
\\	この企画にぴったりのライタ-、草の根分けてでも探しだしてほしい。 
\\	で「あらゆる所を探す」の意味になります。更に、
\\	「あらゆる手段を使う(石を一つ残らずひっくり返して探す)」を付け加えています。	
\\	あいつは口が軽いから、大事なことは話さないほうがいい。 
\\	(口を閉じたままにすることができない)で「口が軽い」を表現しています。	
\\	うっかり口が滑るということはあるけど、いくらなんでもあれは言い過ぎだね。 
\\	は「うっかり口を滑らす」、
\\	は「失言」の意味。	
\\	まあよく喋るね。口から先に生まれたんじゃないの。 
\\	で「生まれながらのおしゃべり」の意味になります。
\\	という言い方はしませんが、使われた文脈により意味は通じるはずです。
\\	「おしゃべりする」、
\\	「おしゃべりな人」。	
\\	口は災いの元。つまらんことはしゃべらない方がいい 
\\	で「つまらないことを言って、困ったことになる」の意味。
\\	「ペラペラしゃべる」、
\\	「ぺちゃくちゃしゃべる」。	
\\	勉強はいやだな。でも「苦は楽の種」と思うしかないか。 
\\	は「骨折りなければ利益なし」の意味です。	
\\	このプロジェクトが失敗したら首が飛ぶな。 
\\	「首が飛ぶ」を 
\\	と 
\\	(転がる)を組み合わせて表現しています。	
\\	あちこちに借金して、もう首が回らない。 
\\	の 
\\	の代わりに 
\\	も使えます。	
\\	今までお喋り三昧していた連中も、ボスが来た途端に蜘蛛の子を散らすように席に戻ったよ。 
\\	で「ハエのように散り散りになる」で「蜘蛛の子を散らす」のニュアンスを表現しています。	
\\	お前と俺は、今回のプロジェクトでは車の両輪ということだろう。仲良くしないことにはうまくいかないよ。 
\\	の文字通りの意味は「密接に一緒になって働く」です。これで「車の両輪」のニュアンスが出せるでしょう。	
\\	カラオケでの彼の歌、玄人はだしだね。 
\\	で「プロをしのぐ(恥ずかしい思いをする)」の意味になります。	
\\	チキンは食べられない、というけど、単なる食わず嫌いじゃないの。 
\\	で「食べて(試して)もいないのにないぜ分かるのか」の意味になります。	
\\	彼は本当になにもできない男だったが、編物だけは得意で今じゃ編物教室の先生だよ。芸は身を助けるってことだね。 
\\	(得意なことがなんであれ、それを使って何とかやっていける)で「芸は身を助ける」のニュアンスを出しています。	
\\	入院したおかげで、いろいろとゆっくり考えることができてさ。けがの功名だったよ。 
\\	で「それはけががもたらしたいいことの一つである」の意味になる。	
\\	彼、ボスの逆鱗に触れ、地方支店に追いやられたよ。 
\\	この場合の
\\	は動詞で「くどくど言う」の意味になりますこの場合の
\\	は動詞で「くどくど言う」の意味になりますこの場合の
\\	は動詞で「くどくど言う」の意味になります
\\	で「人の怒りを招く」の意味。
\\	「いなか、奥地」。	
\\	せっかく忠告してやったのに「下衆の逆恨み」かよ 
\\	(これが貴重なアドバイスをしたことに対するお礼か)で「下衆の逆恨み」のニュアンスを表現しています。	
\\	この案件をどう処理するか、君に下駄を預けるよ。 
\\	「下駄を預ける」とは
\\	に任せる」ということですね。	
\\	そんな煙に巻くことばかり言ってないでちゃんと説明して。 
\\	「煙に巻く」を 
\\	(謎めいた口振りで話す)で表現しています。	
\\	あいつらは今じゃ犬猿の仲らしいよ。女性関係でもめたからだろうな。 
\\	英語の場合には猿の代わりに
\\	を使います。	
\\	よし、明日の競馬は乾坤一擲の大勝負でいくぞ。 
\\	「乾坤一擲の大勝負をする」とは 
\\	(すべてを賭ける)ということですね。	
\\	彼女をデートに誘ったら、けんもほろろに断られた。 
\\	「けんもほろろに」を 
\\	(きっぱりと、にべもなく)を使って表現しています。	
\\	卒業から5年たって、結婚して子どもができた同級生もいる。光陰矢のごとしだな。 
\\	で「時が経つのはとても早い」の意味。
\\	を付ける必要はありません。	
\\	好機逸すべからず。この機に一気に仕事を仕上げてしまおう。 
\\	「好機逸すべからず」を
\\	「素晴らしい機会」、
\\	「馬鹿な」、
\\	「逃す」を使って表現しています。
\\	「一挙に」、
\\	「思い切って」。	
\\	仕事でもプライベートでも、いつも彼の後塵を拝する格好になっている。 
\\	で「…の方が勝っている、しのいでいる」の意味。	
\\	さすが、弘法は筆を選ばずだね。借りたバットでホームランを打つなんて。 
\\	の文字どおりの意味は「本当に才能がある人は道具に文句を言わない」です。	
\\	10年間がむしゃらに働いて、なんとか事業に成功した。これで故郷に錦を飾れるよ。 
\\	少し大げさですが 
\\	(故郷への凱旋)を用いて「故郷へ錦を飾る」を表現しています。
\\	は「こつこつ働いて」の意。	
\\	この劇場のこけら落としは、有名な演出家の作品だったらしい。 
\\	「劇場のこけら落とし」は 
\\	でいいでしょう。	
\\	ここはひとつ心を鬼にして、君に忠告しておくよ。 
\\	で「心を鬼にする、冷たい態度をとる」の意味になります。	
\\	部長はこのプロジェクトを実現させるため、さんざん心を砕いた。 
\\	「心を砕く」とは 
\\	(考え抜く、知恵を絞る)ということですね。	
\\	君達は自分の提案が最高にいいと思っているのだろうけど、僕から見れば、五十歩百歩としか思えないけどね。 
\\	で「どっちもどっち」の意味になります。
\\	「はるかに」。	
\\	いつまでも御託を並べていないで、やるべきことをやれよ。 
\\	この場合の
\\	は動詞で「くどくど言う」の意味になります	
\\	子供の使いじゃあるまいし、もうちょいとまともな交渉してくれんかね。 
\\	とは「使い走り」のことです。
\\	「交渉」。	
\\	最近、廻りでも離婚が多いようだけど、子は鎹って事は、今は昔となったのかね。 
\\	「絆」と
\\	「結び付ける」の使い方に注目。	
\\	鎹
\\	海外旅行に行くなら保険に入りなさい。転ばぬ先の杖っていうでしょ。 
\\	が「転ばぬ先の杖」に当たる言い回しです。	
\\	彼は結構しぶといね。転んでもただでは起きぬ、いい根性しているよ。 
\\	で「どのような状況(問題)も自分にとって有利なものにしてしまう(利用する)」と言っていることになります。
\\	から利益を得る」。
\\	「粘り強い」。	
\\	僕、ホラー映画が大好きなんだ。怖いもの見たさってやつさ。 
\\	で「怖い物を見たいという気持ちを抑えられない」の意味になります。	
\\	さいは投げられた。あとはうまくいくことを祈ろう。 
\\	が「さいは投げられた」に当たる言い回しです。
\\	は単数形。複数形は
\\	先んずれば人を制すのだから、迷っては駄目です 
\\	は「曖昧なことを言う、曖昧な態度を取る」、
\\	は「先導する」の意味。	
\\	酒は百薬の長と言うじゃないか。少しぐらい飲んでも病気は悪化しないさ。 
\\	を英語と言い換えると 
\\	となります。	
\\	身長180センチって、さばを読みすぎじゃないの。すぐばれるぞ。 
\\	(水増しする)を使って「さばを読む」のニュアンスを表現しています。	
\\	あの人がそんな間違いをするなんて信じられないよ。サルも木から落ちるだね。 
\\	ここでは「サルも木から落ちる」を
\\	(だれにでも失敗はある)と言い換えています。	
\\	去る者は日々に疎しだな。お互いに離ればなれになると、自然と疎遠になるよね。 
\\	の文字通りの意味は「会わなくなると、心から消えていく」です。
\\	「連絡が途絶える」。	
\\	あんなもめごとに関わりたくないね。触らぬ神にたたりなしだよ。 
\\	が「触らぬ神にたたりなし」に当たる言い回しです。	
\\	2浪したけど、三度目の正直でようやく大学に合格した。 
\\	「三度目の正直」は 
\\	(三度目の幸運)で表現すればいいでしょう。	
\\	三人寄れば文殊の知恵って言うだろ。みんなで考えれば、いい案が浮かぶかもしれないよ。 
\\	これには
\\	「一人の頭で考えるより2人の頭で考えた方がいい」という決まった言い方があります。	
\\	あの時の君の助け舟。まさに地獄で仏といったところだよ。 
\\	「地獄で仏」をここでは、
\\	「困った時の本当の友達」と言い換えて表現しています。	
\\	最近思いもよらない事件が多いね。事実は小説より奇なり、とはよく言ったもんだよ。 
\\	「事実は小説より奇なり」には
\\	という決まった言い方があります。	
\\	人生沈む瀬あれば浮かぶ瀬あるわけだから、あまりくよくよしないように。 
\\	「浮かぶ沈む」は
\\	で表現すればいいでしょう。
\\	「山あり谷あり」。	
\\	いくら仲がいいからって、そんな言葉づかいをしちゃいけないよ。親しき仲にも礼儀ありって言うでしょう。 
\\	が「親しき仲にも礼儀あり」に当たる言い回しです。	
\\	舌の根も乾かぬ内に、またいいかげんな話をするなよな。 
\\	したかしない内に、~をする」を使えばいいでしょう。	
\\	十把一からげにしないで、もう少し個別案件をきちんと検討してくださいよ。 
\\	には「ひとつにまとめる、一緒に扱う」という意味があります。	
\\	死に物狂いでレポートを2時間で終わらせた。 
\\	は「半狂乱で」の意味。
\\	と
\\	を使っても同じような意味が表せます。	
\\	四の五の言わずに、とにかくひとくち食べてみなさい。 
\\	(ぶつぶつ言うのをやめる)で「四の五の言わずに」を表現しています。	
\\	皆から飲み代を集めたが、少し足りなくて自腹を切った。 
\\	で「差額(足りない分)を自分のポケットから払う」という意味になります。	
\\	日本人やイギリス人にはやはり独特の島国根性が見受けられる。 
\\	「島国根性」には 
\\	という言い方がありますが、ここでは 
\\	を使っています。	
\\	釈迦に説法とは存じますが、先生(医者)も禁煙されたほうがよろしいのではないでしょうか。 
\\	で「もうすでに知っていることを多分言っている(と思う)」の意味。	
\\	彼の話し方は、どうもしゃくにさわる。 
\\	(神経に障る)で「しゃくにさわる」を表現しています。	
\\	いくら私の実家にあいさつに行くからって、そんなに鯱張ってられるとこっちだって落ち着かないわよ。 
\\	「鯱張る」とは
\\	(かしこまる、堅くなる)ということですね。	
\\	彼の父親は、自分が経営する会社の業績が悪化して借金で首が回らないらしい。 
\\	が「借金で首が回らない」に当たる言い回しです。
\\	の代わり
\\	も使えます。	
\\	孤軍奮闘したけれど、衆寡敵せず、彼のビジネスプランは受け入れられなかったよ 
\\	で「圧倒的な優劣の差に対して」という意味になります。「孤軍奮闘」の表現の仕方にも注目。	
\\	本来、柔道の神髄は、柔よく剛を制すというものだ。 
\\	で「柔」、
\\	(暴力・力ずく)で「剛」を表現しています。
\\	は「効果的な」の意。	
\\	朱に交われば赤くなる。友だちはよく選ぶべきだよ。 
\\	の文字通りの意味は「悪い品行はうつることがある」になります。	
\\	小食は長生きのしるしを実践してるんだけど、食費も抑えられるので一石二鳥だね。 
\\	「小食」を
\\	(少ない食事)、「長生きのしるし」を
\\	とそのまま英訳しています。「一石二鳥」の英訳にも注目。	
\\	宝くじで一等が当たったって?おいおい、冗談も休み休みいえよ。 
\\	「冗談も休み休み言えよ」とは、つまり、
\\	「冗談を言うのは止めてくれ」ということですね。	
\\	車の運転は慣れてきたころが一番危ないんだよ。初心忘れるべからずっていうだろ? 
\\	で「初心者であった時のことを忘れる」の意味。	
\\	知らぬが仏って言うだろ。何があったか知らない方がいいよ。 
\\	が「知らぬが仏」に当たる英語の諺です。	
\\	英語が堪能な人を探していたので、2年間アメリカに留学していた彼女に白羽の矢が立ったのです。 
\\	で「…に限定して選ぶ」の意味になります。	
\\	いつも怒鳴り散らしてばかりの部長だけど、家では尻に敷かれてるらしいよ。 
\\	が「尻に敷かれる」に当たる表現です。
\\	は「(子羊のように)おとなしい、従順な」の意。	
\\	どう考えてももう打つ手なし。進退これ谷まったよ 
\\	で「追いつめられる」の意味になります。	
\\	あの人の彼、ハンサムで優しくておまけにお金持ちで、みんなの垂涎の的よ。 
\\	「垂涎の的」には
\\	という言い方がありますが、ここでは動詞
\\	(羨ましがる)を使っています。	
\\	娘のピアノはグングン上達していった。好きこそ物の上手なれとはよく言ったものだ。 
\\	の文字通りの意味は「楽しんでやればうまくできる」です。	
\\	フラれたからって、彼女を悪く言うのはよせ。酸っぱいブドウってやつさ。 
\\	「酸っぱいブドウ」はそのまま 
\\	(負け惜しみ)が使えます。	
\\	通信教育なんて成果が上がるのか疑問だったけど、まんざら捨てたものじゃない。 
\\	で「少しも悪いことはない」の意味。
\\	は「通信講座」の意。	
\\	捨てる神あれば拾う神あり。やっと採用試験に合格したよ。 
\\	「神、神霊」、
\\	「見捨てる」、
\\	「拾う」を組み合わせて表現しています。	
\\	あいつはほめられるとすぐ図に乗るからな。 
\\	で「ほめられると有頂天になる」の意味。
\\	は「調子に乗る、図に乗る」の意。	
\\	間取りも狭いし、周りもうるさいけど、住めば都なんだよ、この部屋。 
\\	この場合の
\\	は「自分が暮らしてる(生活している)快適な場所」の意味。	
\\	急いては事を仕損じるよ。もう少々慎重に。 
\\	で「慌ててやると仕事を台無しにしてしまうことになる」の意味。
\\	「慎重な」。	
\\	彼女、恋人からいきなり別れ話を切り出されたんだって。あんなに仲がよかったのに、まさに青天の霹靂ね。 
\\	は「晴天の霹靂」に相当する慣用句です。	
\\	金を借りるためとはいえ、あいつにだけは頭を下げたくなかったな。しかし、背に腹は代えられない。 
\\	で「頭を下げずに、お金を貸してもらうことは無理だ」と言っています。	
\\	あいつは線が細いけど、力は人の2倍はあるよ。 
\\	ここでは「線が細い」を 
\\	(繊細な)で表現しています。「2倍」の表現の仕方にも注意。	
\\	そうと決まれば、善は急げだ。さっそく細かい打ち合わせに入ろう。 
\\	(鉄は熱いうちに打て)で「善は急げ」を表現しています。	
\\	先が見えないように思うかもしれないけど、千里の道も一歩から。まずは始めないとね。 
\\	が「千里の道も一歩から」に当たる言い回しです。	
\\	そっぽを向いてないで、きちんと話を聞きなさい。 
\\	の代わりに 
\\	も使えます。	
\\	備え有れば患え無し、いつも万全の下準備を、とは思うけど、 現実はね。 
\\	「準備しておけば、大丈夫」で「備え有れば患え無し」の意味になります。。
\\	「控えを作っておく」。	
\\	今度の商売は、損して得取れ、という戦略でやろうよ。 
\\	「損して得取れ」は英語でも
\\	と言います。
\\	「損失を蒔いて、利益を刈る」。	
\\	風邪はほとんど治ったが、大事を取ってもう1日会社を休もう。 
\\	が「大事を取る、念のために」に当たる言い回しです。	
\\	ちょっと大きいけど、この封筒でいいや。大は小を兼ねるって言うし。 
\\	で
\\	で失敗した方がまし」の意。	
\\	コンピュータの使い方も知らないのに、あんなに高いマシンを買って。宝の持ち腐れだよ。 
\\	「宝の持ち腐れ」とは
\\	「なんたる無駄」ということですね。	
\\	彼は竹を割ったような性格だ。 
\\	で「爽やかなほどに率直な」の意味になります。	
\\	今回の僕の失敗を他山の石にして、君の一層の活躍を祈ってるよ。 
\\	「僕の失敗を他山の石にする」とは
\\	「僕の失敗から学ぶ」ということですね。	
\\	多勢に無勢。私の企画は残念ながら取り下げますよ。 
\\	は「~より数でまさる」という意味の動詞です。これを受け身にすれば「多勢に無勢」が表現できます。	
\\	会議でのきみのプレゼンは非常にわかりやすかったが、最後の一言だけは蛇足だったな。 
\\	「蛇足」とは
\\	(余分な)ということですね。	
\\	叩けば埃が出る身体、あまり厳しく責めないでよ。 
\\	「表面をひっかく」だけで、必要な証拠がみんな出てくる。これで「叩けば埃が出る」のニュアンスを出しているのです。	
\\	彼、今は目だたないけど、ただの鼠ではないと思うよ。 
\\	「ただの鼠ではない」とは
\\	「普通の 
\\	ではない」ということですね。ここでは
\\	など、具体的な名詞を入れておきました。
\\	「国会議員」、
\\	「ありふれた」。	
\\	泥棒を見つけたんで大声で叫んだら、脱兎のごとく逃げ出したよ。 
\\	(できるだけ早く逃げた)で「脱兎のごとく逃げ出した」を表現しています。
\\	は「大声で叫ぶ」の意。	
\\	立つ鳥跡を濁さず。次に入居する人のことを考えて、引っ越しの後片づけは入念に。 
\\	で「…(散らかった)ままにしておくのははしたないことである」の意味になります。	
\\	立て板に水の如きプレゼンテーション、おそれ入りました。 
\\	「立て板に水」とは
\\	「とても滑らか」ということですね。	
\\	まさかあんなひどい出来の絵が売れるとは、蓼食う虫も好きずきだね。 
\\	とは「人それぞれ」の意味。
\\	という言い方もあります。	
\\	宝くじが当たったなんて棚からぼたもちだね。 
\\	「意外な授かり物」を使えば「棚ぼた」が表現できます。	
\\	あいつ、自分に都合の悪い話しになると狸寝入りをする。 
\\	「狸寝入り」とは
\\	「寝ている振りをする」ということですね。	
\\	あいつまた待ち合わせの時間に遅れてくるかもしれないから、集合は7時だって駄目を押しておいてよ。 
\\	は「二重に確認するために」、
\\	は「念のために」の意味。	
\\	便りの無いのは良い便り。息子さんから連絡がないのは、元気でやってるからよ。 
\\	「便りの無いのは良い便り」に当たる英語の言い回しは
\\	です。	
\\	他力本願じゃ、世の中渡っていけないよ。 
\\	で「あまりに人に頼りすぎる」という意味。	
\\	つまらないことで腹を立てるなよ。短気は損気っていうだろ。 
\\	という言い方もありますが、ここでは
\\	「短気」を主語にして表現しています。
\\	「無一文の」、
\\	「貧乏への近道」。	
\\	ボクシングの試合を見ると血が騒ぐのは男だけよ。 
\\	(興奮する)で「血が騒ぐ」を表現しています。	
\\	彼とは幼稚園から大学まで一緒で(ちくばのとも)と呼べる仲だよ。 
\\	「竹馬の友」は
\\	と言います。
\\	の代わりに
\\	(終生の)を使ってもいいでしょう。	
\\	昨晩お父さんはかなり酔って、千鳥足で帰ってきました。 
\\	には「よろよろ歩く、千鳥足で歩く」という意味があります。	
\\	あの姉妹は昔からひどく仲が悪かったけれど、血は水より濃いもんだね。 
\\	が「血は水より濃い」に当たる言い回しです。	
\\	お昼になると、あのソバ屋の前にはいつも長蛇の列ができている。 
\\	とは「くねくねと続く長い列」のことです。	
\\	毎日、5個ずつ単語を覚えてるんだ。ちりも積もれば山となるってね。 
\\	の文字通りの意味は「一回に少しずつ、それが積み重なって多くなる」です。	
\\	血を分けた兄弟なのに、なぜ憎しみ合わなくてはいけないんだ。 
\\	が「血を分けた兄弟」に当たる言い回しです。
\\	は「敵意」の意。	
\\	ニューヨークの夜景に比べたら、ここの景色なんて月とスッポンだよ。 
\\	「夜と昼ほど違う」で「月とスッポン」を表しています。	
\\	彼は深く思い悩んでいたが、今では憑き物が落ちたように晴れやかな様子だ。 
\\	の文字どおりの意味は「悪霊が取り除かれる」です。	
\\	経理をやっていると、つぶしがきいていいわね。 
\\	(ほかの分野でも仕事を得ることができる)で「つぶしがきく」を表現しています。	
\\	試験の予想がツボにはまり、あまり勉強しなくてもいい点が取れた。 
\\	(ラッキーな予想をする)で「ツボにはまる」を表現しています。	
\\	確かに彼は大罪を犯したけれど、罪を憎んで人を憎まずと言うじゃないか。 
\\	が「罪を憎んで、人を憎まず」に当たる言い回し。
\\	は「とがめる」、
\\	は「犯罪者」の意。	
\\	恋愛なんてものは、出会い頭に車に衝突するようなもんだよ。 
\\	で「運命的な衝突のようなもの」という意味になります。	
\\	彼、陽子とつきあい始めたの?手が早いやつだ。 
\\	ここでは「手が早い」を 
\\	で表現しています。	
\\	あいつは理屈の合わんことには梃子でも動かないよ。 
\\	で「少しも動かない」の意味。	
\\	手塩にかけて育てた部下が、転職してしまった。 
\\	で「優しく愛情を持って訓練する」の意味になります。	
\\	彼女、過去に年上の男性を何人も手玉に取っているらしい。 
\\	(思うままにする)で「手玉に取る」を表現しています。	
\\	この子には3歳になったらテニスをやらせよう。鉄は熱いうちに打てだ。 
\\	が「鉄は熱いうちに打て」に当たる言い回しです。	
\\	手八丁口八丁の彼だけど、誠がないのが玉に疵だね。 
\\	「仕事が早い」、
\\	「ものをとてもハッキリ言う」を組み合わせれて「手八丁口八丁」を表現しています。
\\	「不誠実」、
\\	「欠点」。	
\\	出る杭が打たれる会社じゃ、大きな成長は望めないね。 
\\	の文字通りの意味は「革新を押さえつける」。
\\	「皆一緒でなくてはいけないという脅迫観念に取り憑かれている」。	
\\	きみの出る幕じゃない。ぼくに任せておけ。 
\\	で「ここで果たす役割はない」という意味になります。	
\\	友だちが困っているのに、手をこまねいて見ているだけなんてひどすぎる。 
\\	ここでは「手をこまねく」を 
\\	「助けるために指一本上げることなしに」で表現しています。	
\\	上司がいないからといって手を抜かないように。 
\\	が「手を抜く」に当たる言い回しです。	
\\	灯台下暗しだな。本当に大切な人って身近にいるもんだ。 
\\	が「灯台下暗し」に当たる言い回しです。	
\\	ダイエットするのもいいが、度が過ぎると身体をこわすよ。 
\\	で「やり過ぎる」の意味になります。
\\	は「破滅させる」の意。	
\\	つまらん会議はとっとと終わってほしいよ。時は金なりということを彼等は知らいのかね。 
\\	「時は金なり」は
\\	ですね。
\\	「意味のない」。	
\\	毒にも薬にもならないような、いい加減な提案をしないでほしいなあ。 
\\	は「無害の」、
\\	は「無意味なほど平凡な」の意味になります。	
\\	彼、床の間の置き物じゃあるまいし、たまにはお役にたってもらいたいもんだよ。 
\\	「盆栽」を使って「床の間の置き物」のニュアンスを表現しています。	
\\	昔はこんなこと簡単だったのに。年には勝てないか。 
\\	の文字通りの意味は「年齢の私に対する扱いがよくない」です。	
\\	そんな取って付けたようなお世辞を言うなよ。 
\\	「取って付けたような」とは
\\	「無理な、不自然な」ということですね。	
\\	そんなに人をうらやましがるなよ。隣の芝生は青いって言うだろ。 
\\	が「隣の芝生は青い」に当たる言い回しです。	
\\	彼、今のところは飛ぶ鳥も落とす勢いだけれど、あの威勢いつまでもつのか、疑問だね。 
\\	「飛ぶ鳥も落とす勢い」のニュアンスを
\\	「(誰も)止められない」で表現しています。
\\	「頂点」。	
\\	株に手をだすなんて「虎の尾を踏む」ようなことはするなよ。 
\\	で「危険なことをする(虎の尾を掴む)」の意味になります。	
\\	みんな自分が一番だって言うけれど、実際はドングリの背比べみたいもんなんだよな。 
\\	「ドングリの背比べ」とは
\\	(どの人も同じようなものだ)ということですね。	
\\	お金を貸してやりたいのはやまやまだが、ない袖は振れないよ。 
\\	ここでは「ない袖は振れぬ」を文字通りに、
\\	(ない物はあげられない)と訳しています。	
\\	泣いても笑っても、後一ヶ月で大学卒業だ。 
\\	「泣いても笑っても」は英語でも
\\	と言います。	
\\	長い目で見ようよ。あの子は大丈夫さ。 
\\	と
\\	(長期的に)を組み合わせて「長い目で見る」を表現しています。	
\\	上司に反抗したって何にもならないぞ。人生、長いものには巻かれたほうが楽に決まってる 
\\	の文字通りの意味は「負かすことができないなら、仲間になれ」です。	
\\	昨シーズン、本当に鳴かず飛ばずだったが、今年はいい感じだぞ。 
\\	ここでは「鳴かず飛ばず」を
\\	(目立たないままでいる)を使って表現しています。	
\\	道で転んだうえに、溝に財布を落としちゃった。泣きっ面に蜂だよ。 
\\	(踏んだり蹴ったり)を使って「泣きっ面に蜂」を表現しています。	
\\	情けをかけても彼のためにならないわよ。 
\\	には動詞で「かわいそうに思う」という意味があります。
\\	を主語にしてもいいでしょう。	
\\	彼女にメール送っても返事がまったく来ないんだよ。なしのつぶてでさぁ。 
\\	「なしのつぶて」とは「相手からまるで音沙汰がないこと」ですから、
\\	を使って表現できます。	
\\	人間何事もその気になればなんとかなるさ。為せば成る、だよ。 
\\	「為せば成る」を 
\\	(努力はいつも報われる)で表現しています。	
\\	そんなに落ち込むなよ。人生、七転び八起きさ。 
\\	は「人生には浮き沈みがある」の意味。	
\\	彼女、遅刻してきたのに何食わぬ顔をしている。 
\\	で「何事もなかったかのように振る舞う」の意味。	
\\	身長190センチで名前が大なんて、名は体を現すだな。 
\\	と 
\\	(ぴったりの)を組み合わせて「名は体を現す」を表現しています。 
\\	とも言えますね。	
\\	英語は習うより慣れろだよ。 
\\	「習うより慣れろ」には 
\\	という言い方がありますが、ここでは 
\\	(…に慣れる)と 
\\	を使って表現しています。	
\\	父の名を汚すような真似はできない。 
\\	は「汚す、傷つける」という意味です。	
\\	モーツアルトは地位を手に入れられなかったが、後世に名を残した偉大な人物だ。 
\\	で「並はずれた才能の持ち主(第一人者)として認められる」の意味。
\\	は「名声を得る」の意。	
\\	彼を信じてお金を貸したのに返してくれないの。煮え湯を飲まされた気分だわ。 
\\	「煮え湯を飲まされた気分」とは 
\\	(裏切られた気持ちになる)ということですね。	
\\	夜遅くに彼女を家まで送ったら、彼女の父親が玄関の前で仁王立ちで待ってたんだ。 
\\	「威圧するような姿勢をとる」で「仁王立ち」を表現しています。	
\\	このテストで満点を取るのは二階から目薬をさすようなものだ。 
\\	で「見つかりそうもない物を探す(無駄骨を折る)」の意味になります。	
\\	あの取引がライバル社に横取りされたんだって? そりゃあ、逃がした魚は大きいな。 
\\	が「逃がした魚…」にあたる言い回しです。	
\\	彼女は苦虫をかみつぶしたような顔をして、テストの答案を見ていたよ。 
\\	には「不機嫌な」という意味があります。	
\\	人生は、行うべきか、行わないべきか、という二者選一の連続だよね。 
\\	と
\\	を組み合わせれば「二者選一」を表現できます。	
\\	またケガしたの? 二度あることは三度あるって言うから気をつけてね。 
\\	「必ず... だ」。
\\	で「二度、三度」を表現している。	
\\	二兎追うものは一兎も得ずって言うでしょう。どっちつかずはやめて、彼一人に絞りなさい。 
\\	「二兎追う者は一兎も得ず」はそのまま 
\\	と直訳できます。
\\	は「迷う、ためらう」の意。	
\\	彼に何をアドバイスしても糠に釘。もう止めた止めた。 
\\	「糠に釘」とは
\\	しても意味がない」ということですね。	
\\	君、本当に企画通ったこと確認しただろうね。糠喜びだけはごめんだよ。 
\\	を直訳すると「早まった幸せ」になります。これで「糠喜び」のニュアンスが出せますね。	
\\	いま思えば、当時はぬるま湯につかったような日々を送っていた。 
\\	(気楽に過ごす)で「ぬるま湯につかる」のニュアンスを表現しています。	
\\	彼は前科があったために泥棒の濡れ衣を着せられた。 
\\	で「間違って訴えられる、責められる」の意味になります。	
\\	先方から好条件を提示してきたので、今回の転職は願ったり叶ったりだ。 
\\	で「まさにずっと願っていたことだ」の意味になります。	
\\	こんなに上等なステーキも、味音痴の彼には猫に小判だ。 
\\	で「豚に真珠を投げるようなもの(猫に小判)」の意味になります。	
\\	店が繁盛してるのはいいけれど、忙しくて猫の手も借りたいほどだ。 
\\	ここでは「猫の手も借りたい」を
\\	「目が回るほど」を使って表現しています。	
\\	やっと郊外に家を買ったけれど、猫の額ほどの庭がついているだけだ。 
\\	ここでは「狭さ」を
\\	(切手)を使って表現しています。	
\\	いまさらアイドルに熱を上げるような年じゃないでしょ。 
\\	で「…にのぼせ上がる」の意味。	
\\	今週は根も葉もない噂に振りまわされどおしだったね。 
\\	は「根拠のない、事実無根の」という意味。	
\\	もう年貢の納め時だ。もう音楽なんてやめて店を継ぐんだ。それがお前のためだ。 
\\	(真剣になる時期である)を使って「年貢の納め時」を表現しています。	
\\	能ある鷹は爪を隠すって言葉は、まさに僕のためにある言葉だな。 
\\	で「力のある人は見せびらかす必要はない」の意味。
\\	で「おあつらえ向きの」の意。	
\\	先週あれだけ先生に叱られたのにもう宿題忘れて、喉元過ぎれば熱さ忘れるってやつね。 
\\	の文字通りの意味は「危険が去ったとたんに、反省した態度を忘れる」です。	
\\	あそこの奥さんは大柄でたくましいわね。だんなさんは小柄で頼りなさそう。まさに蚤の夫婦だわ。 
\\	ここでは「蚤の夫婦」を文字通りに
\\	と英訳しています。	
\\	飲む打つ買うの毎日はもう卒業。今からはまじめに働くぞ!! 
\\	が「飲む打つ買う」に当たる言い回しです。	
\\	乗り掛かった船だ、君に協力するよ。 
\\	は「引き返せない」、
\\	は「遠くまで来すぎている」の意味になります。	
\\	彼になにを注意しても、暖簾に腕押し。もうやめたやめた。 
\\	は「無駄なことをする」の意味。代わりに
\\	とも言えます。	
\\	おとなしくしていると思っていたが、ついに化けの皮を現したな。 
\\	(本性)と
\\	(現れる)を使って「化けの皮を現す」を表現しています。	
\\	もうこれ以上恥の上塗りになるような真似はしないでほしい。 
\\	が「恥の上塗り」に当たる言い回し。	
\\	何事も初めが肝心だから。しっかりと準備をして仕事にとりかかってくれたまえ。 
\\	は「どのような努力をするにも初めが一番大切である」という意味。	
\\	私、明夫とつきあいたいんだけど、橋渡しをしてくれない? 
\\	(お膳立てをする、仕組む)で「橋渡しをする」を表現しています。 
\\	「仲立ちをする」を使ってもいいでしょう。	
\\	僕の専門は法律ですので、国際政治については畑違いですから僕にはよく分かりません。 
\\	ここでは「畑違い」を
\\	と
\\	する資格がない」を組み合わせて表現しています。	
\\	ショップにギターが売っているわけないだろう。それは「畑に蛤」だ。 
\\	「畑に蛤」とは「見当違いをすること」ですから、
\\	(混乱する)を使ってその意味を表現しています。	
\\	サッカー日本代表、これでもう20連勝だよ。まさに破竹の勢いってやつだね。 
\\	(絶対に止められない)で「破竹の勢い」のニュアンスを出しています。	
\\	舞台上とは違って、楽屋裏では蜂の巣をつついたような騒ぎだ。 
\\	が「蜂の巣をつついたような騒ぎ」に当たる言い回しです。
\\	は「せわしい」、
\\	は「蜂の巣」の意。	
\\	あの手この手と策はつくしたけれど、いよいよ八方塞がりになってしまった。 
\\	とは「行き詰まってしまう」の意味。
\\	「どこにも行けない地点(身動きがとれない)」。	
\\	突然の昇進の発表で、彼、鳩が豆鉄砲を食らったような顔をしていたよ。 
\\	で「ポカンとする」の意味。
\\	「ヘッドライトに照らされた鹿のような様子だった」。	
\\	君が司法試験に合格して、お父様もさぞかし鼻が高いことでしょう。 
\\	で「誇らしげに微笑む」という意味になります。	
\\	彼にちょっと鼻薬を嗅がせておけば、この話、もっと楽に通っただろうにね。 
\\	「鼻薬を嗅がせる」とは
\\	「賄賂を使う」ということですね。	
\\	あいつの話は大げさだから、話半分で聞いたほうがいい。 
\\	で「…を割り引いて聞く、加減して聞く」の意味になります。	
\\	購入した商品の欠陥について販売店に文句をつけたが、レシートがなかったので鼻であしらわれた。 
\\	は「冷たくあしらう」の意味。	
\\	彼は面白いやつだけど、だれにでも歯に衣着せぬ言い方をするから周りは敬遠するんだよ。 
\\	は「遠慮せずにはっきり言う」の意味。
\\	(彼は遠慮なくずけずけ言うから)を使ってもいいでしょう。	
\\	羽目を外すのもいいけど、明日の仕事に響かないようにね。 
\\	「遠慮なくやる」が「羽目を外す」に当たる言い回しです。
\\	や
\\	という言い方もあります。	
\\	いろいろ理由を並べたけど、早い話、あなたとは別れたい。 
\\	で「結局のところ…なんだ」の意味。	
\\	朝練に行く途中に千円拾ったんだ。早起きは三文の得だね。 
\\	が「早起きは三文」に当たることわざです。	
\\	あの政治家、本当に腹が黒いな。今すぐ辞職すべきだよ。 
\\	で「意地の悪い」という意味になります。「腹黒い」は 
\\	でも表現できます。	
\\	今日は入試だというのに頭は痛いし遅刻しそうだし、万事休すだ。 
\\	で「おしまいだ」の意味になります。	
\\	階段を一気に駆け上がったら膝が笑ってしまった。 
\\	は「ぐらぐらする」の意味。もちろん、ここでは 
\\	は使えません。	
\\	彼、ストックオプションで大金を手に入れ、最近は左団扇で暮らす 毎日というもっぱらの噂だよ。 
\\	で「日々を楽に快適に過ごす」の意味。	
\\	そんな失敗なんて気にするな。人の噂も七十五日。みんなすぐに忘れるさ。 
\\	で「人の記憶力など大したことがない(すぐに忘れる)」の意味。
\\	「へま」、
\\	「噂ははかないものである」。	
\\	かわいい姪っ子のために私がひと肌脱ぎましょう。 
\\	(…のために何かする)で「一肌脱ぐ」を表現しています。
\\	(助ける)を使ってもいいでしょう。	
\\	あいつ、ついに社長になったぞ。人は人、おれはおれ。あいつは頑張ったからな。おれは自分のやり方で頑張るよ。 
\\	で「よかったじゃないか。彼は頑張ってそうなったんだから。でも、僕は僕だからね」の意味になります。	
\\	今更そんなことを言った日にゃ、火に油を注ぐことになるだけだよ。 
\\	「火に油を注ぐ」は、英語でも文字通りに
\\	と言います。
\\	「油(燃料)を注ぐ」。	
\\	いろいろと出費が多くて、今月は火の車よ。 
\\	「火の車」とは、
\\	(お金が不足している)ということですね。	
\\	説明はこのくらいにして、まずは実物を見てほしい。百聞は一見に如かずというからね。 
\\	「百聞は一見に如かず」は 
\\	ですね。	
\\	この事件は政権内腐敗の氷山の一角に過ぎない。 
\\	が「氷山の一角」に当たる言い回しです。	
\\	君の雑な提案ではこの事業、うまくいくはずがないのは、火を見るよりも明らかだよ。 
\\	が「火を見るより明らか」に当たる言い回しです。
\\	「ちょっと雑な」。	
\\	へえ、パソコンってのは機能も価格もピンからキリまであるんだね。知らなかったな。 
\\	「ビンからキリまで」とは
\\	(とても異なる/幅がある)ということですね。	
\\	30歳までに結婚するという彼の計画は、もはや風前の灯火だ。 
\\	この場合の
\\	は「危なっかしい、不確かな」という意味です。	
\\	やってしまったことは仕方ない。覆水盆に返らずだ。 
\\	で「それについては何もできない」、
\\	は「過ぎてしまったこと」の意味。	
\\	もう逃げ道はないぞ。あきらめろ。すでに袋の鼠だ。 
\\	「逃げ道はない」が「袋の鼠」に当たる言い回しです。
\\	という言い方もあります。	
\\	あの大事故で、骨折だけで済んだのは、不幸中の幸いだ。 
\\	(もっとひどいことになった可能性があった)で「不幸中の幸い」のニュアンスを表現しています。	
\\	明日、本当に金を返すよ。武士に二言はないさ。返すよ。 
\\	で「必ず約束は守る」の意味になります。	
\\	無理なお願いだったが、彼は二つ返事で引き受けてくれた。 
\\	は「躊躇なく、快く」の意味。
\\	を使ってもいいでしょう。	
\\	急に雨は降ってくるし、行きたかったコンサートは中止になるし、踏んだり蹴ったりだよ、今日は。 
\\	が「踏んだり蹴ったり(ひどい目に合わせた上に、さらに、侮辱する)」に当たる表現です	
\\	そんなばかなこと言うなよ。へそが茶を沸かすよ。 
\\	ここでは「へそが茶を沸かす」を
\\	「笑わせるな」で表現しています。	
\\	きみがあんまりからかうから、彼女はへそを曲げちゃったよ。 
\\	ここでは「へそを曲げる」を 
\\	(怒る)で表現しています。	
\\	何でもいいからとにかく企画を出そう。下手な鉄砲も数撃てば当たるさ。 
\\	が「下手な鉄砲も数撃てば当たる」に当たる言い回しです。	
\\	居間の電球を取り替えてくれって? 任せて、そんなの屁の河童だよ。 
\\	「屁の河童」とは 
\\	(朝飯前)ということですね。
\\	ともいえます。	
\\	減らず口をたたくのはやめなさい。 
\\	(議論のための議論をする)で「減らず口をたたく」を表現しています。	
\\	この見積もり金額でOKだって。坊主丸儲けとはこのことだね。 
\\	で「ただで何かを得る」の意味。
\\	「認める」。	
\\	さすが高級料亭だけあって、料理はほっぺたが落ちるほどおいしかったよ。 
\\	英語には味を表すのに「頬」を使った表現はありません。「ほっぺたが落ちるほどおいしい」とは
\\	(びっくするほど美味しかった)ということですね。	
\\	ここに遺跡が埋まっているというからあちこち掘り返したけど、何も出てきやしない。まったく骨折り損のくたびれもうけだよ。 
\\	で「色々苦労しても何も得る物がない」の意味になります。
\\	は「発掘する」の意。	
\\	うそばっかりついていると、いつかボロが出るぞ。 
\\	とは「お里が知れる」ことです。	
\\	うちの娘二人が相次いで出産したもので、わが家は盆と正月がいっぺんに来たような騒ぎでしたよ。 
\\	英語圏には盆と正月はありませんから、ここではその代わりに
\\	と
\\	を使って表現してみました。
\\	は「(興奮して)混乱している状態」の意。	
\\	結果はどうであれ、やってみようよ。まかぬ種は生えないんだから。 
\\	(…を考えているだけではどうにもならない)で「まかぬ種は生えない」を表現しています。	
\\	あのスポーツ解説者は、間口ばかりで奥行きがないな。 
\\	「何も言うことがない」が「奥行きがない」、
\\	(大声で話す/うるさいだけ)が「間口ばかりで」に当たります。	
\\	決着が付いてから文句を言ったって、負け犬の遠吠えにしか聞こえないよ。 
\\	で「無駄なことを言う」という意味になります。	
\\	へえ、今日はおしゃれじゃない。馬子にも衣装てのはこのことだね。 
\\	「磨き上げる」、
\\	「こぎれいにする」。これを受け身で用いている。	
\\	そんな眉に唾をつけるような話に簡単に乗らないでよ。 
\\	「眉に唾をつける」とは
\\	「うさんくさい、いかがわしい」ということですね。	
\\	真綿で首を締めるような言い方をしないで、お願いだから、もっとはっきり言ってくれない。 
\\	「真綿で首を絞める(遠回しに責める)」を「微妙な手がかりを集めてついていく
\\	ようなことをさせるのは止めてくれ」と言い換えて表現しています。	
\\	挑戦者がリングに上がりました。さあいよいよチャンピオンが満を持しての登場です! 
\\	「丁度いい瞬間まで待つ」で「満を持して」の意味を表しています。	
\\	迷子を探しに行って自分も迷うなんて、ミイラ取りがミイラになるようなものだ。 
\\	の文字どおりの意味は「解決する代わりに、その問題の一部になってしまう」です。	
\\	あいつは体は大きいけど、実はみかけ倒しで小心者なんだ。 
\\	「見かけ倒し」を 
\\	(それは誤解を招く)で表現しています。
\\	は「臆病な、小心な」の意。	
\\	いいかげんな仕事をしてるから、左遷されちゃうんだよ。身から出た錆だね。 
\\	とは「自分で蒔いた種は自分で刈れ」の意味。
\\	「いい加減な」。「左遷される」の表現の仕方にも注目。	
\\	この会議では火の粉が降ってこないように、見ざる言わざる聞かざる、で臨むしかないね。 
\\	が「見ざる言わざる聞かざる」に当たる言い回しです。
\\	「面倒」。	
\\	金が必要ならそう言ってくれればよかったのに、黙っているなんて水くさいじゃないか。 
\\	「…を見ず知らずの人のように扱う」で「水くさい」のニュアンスを出しています。	
\\	あの二人は水と油だね。まったく反りが合わない。 
\\	「水と油」は英語では
\\	と油が先に来ます。	
\\	水も漏らさぬ周到な準備をしたとはいうけど、開始早々トラブル続き。だいじょうぶかいな。 
\\	は「完璧な(空気をもらさない)」という意味です。
\\	「問題に囲まれている」。	
\\	息子はもう50歳になるというのに、短気なところは子どものころと変わらない。三つ子の魂百までとはよく言ったものだ。 
\\	で「子どものころに身につけた習慣は一生ついてまわる」の意味になります。
\\	(性格は変わらない)とも表現できます。	
\\	きみは本当に身の程知らずだね。だれに口をきいていると思っているんだ。私はきみの会社の社長だぞ! 
\\	(自分の立場を知らない)で「身の程知らず」を表現しています。	
\\	きみの忠告はいつも耳が痛いよ。 
\\	この場合の 
\\	は動詞で「恥ずかしく思わせる」の意味。	
\\	生演奏ばかり聞いていて耳が肥えたのかな。
\\	のクラシック演奏を聞いても、いいと思えない 
\\	で「…に対する耳が肥える」を表現しています。	
\\	もうそのニュースを知っているなんて耳が早いね。 
\\	「耳が早い」は 
\\	で表現できます。	
\\	そこまで言われちゃ、私としては身も蓋もないよ。少しは俺の立場もわかってよ。 
\\	「身も蓋もない」とは「露骨過ぎる」という意味ですから、
\\	「残忍なまでに率直に」でそのニュアンスが表現できるでしょう。	
\\	彼女の態度からはまだ脈がありそうだ。 
\\	「脈がある」とは「少しは望みがある
\\	ということですね。	
\\	見よう見真似でこんな出来のいいプログラムを書いたんだって。すごいな。 
\\	は「真似て...を身につける」の意味です。	
\\	あの人、今では見る影もないけど、10年前はすごい売れっ子歌手だったのよ。 
\\	が「見る影もない」に当たる言い回しです。	
\\	彼は元コックの腕をふるって食事を作ってくれた。さすがは昔とった杵柄で、とてもおいしかった。 
\\	「昔とった杵柄」を
\\	、つまり、「一度身についたものは決して忘れない」
\\	を使って表現しています。	
\\	あの選手、圧倒的に強いけどなぜか金メダルは取れない。無冠の帝王だな。 
\\	が「無冠の帝王」に当たる表現です。	
\\	彼女はお高くとまってる感じがしてどうも虫が好かない。 
\\	(…を好きにさせない)で「虫が好かない」を表現しています。
\\	は「高慢」の意。	
\\	あいつの顔を見るだけで虫酸が走るんだ。 
\\	「虫酸が走る」をここでは
\\	(ぞっとする)を使って表現しています。	
\\	ボス、結構荒れているけど、どこか虫の居所が悪いのかね。 
\\	で「機嫌が悪い」という意味になります。	
\\	彼をだますなんて胸が痛まないの。 
\\	は形容詞で「胸が張り裂けるような」の意味になります。	
\\	大好きな歌手が来日すると聞いて胸が躍った。 
\\	と 
\\	(喜びで踊る)を組み合わせて「胸が躍る」を表現しています。	
\\	どうも君は未だ胸に一物もっているような気がする。文句があればはっきり言ってよ。 
\\	が「胸に一物ある」に当たる表現です。	
\\	まったく新しい観点からのお話に、目から鱗が落ちる思いでした。 
\\	(明かりが見える、納得する)を使って「目から鱗が落ちる」を表現しています。また、
\\	と、日本語と同じ言い方もできます。
\\	は「鱗」の意。	
\\	車に乗る必要はないよ。アルクは駅から目と鼻の先だから。歩けばじゅうぶん だよ。 
\\	「目と鼻の先」とは
\\	「すぐ目の前」ということですね。	
\\	あんなに言われて黙ってちゃだめだって。目には目を歯には歯をだ。 
\\	「目には目を歯には歯を」はそのまま
\\	を使って表現します。	
\\	ピッチャーが目にも留まらぬ速さでボールを投げた。 
\\	は「ものすごい速さで」の意味。	
\\	あの人がいると、仕事がやりづらくてしょうがない。目の上のたんこぶだよ。 
\\	(絶えず邪魔になる物)で「目の上のたんこぶ」を表現しています。	
\\	祖父は初孫の誕生を心待ちにしていたので、目の中に入れても痛くないとばかりにかわいがっている。 
\\	「目に入れても痛くないほど可愛がる」を
\\	(大事にする)、
\\	(熱愛する)を使って表現しています。「目に入れても痛くないほど可愛い子」は 
\\	といいます。	
\\	彼女の部屋に行ったんだけど、全然掃除していないみたいで目も当てられない有様だった。 
\\	で「見ていられないほどひどい散らかりよう」の意味になります。	
\\	最近の子どもはテレビゲームばかりで、日本の伝統的な遊びには目もくれないよ。 
\\	で「…にまったく関心がない、注意を向けない」の意味になります。	
\\	試合に負けたのは僕のエラーのせいだ。本当に面目ない。 
\\	「恥ずべきことをする、体面汚す」の意味。	
\\	そういう仕事なら彼にお願いしたほうがいいよ。餅は餅屋ということさ。 
\\	「餅は餅屋」のニュアンスを
\\	(彼はその仕事のちゃんとしたやり方を知っている)を使って表現しています。	
\\	紆余曲折はあったけど、あのふたり、結局は元の鞘へ収まったよう だね。 
\\	で「結果的にまた一緒になる」の意味。
\\	を使った「紆余曲折」の表現の仕方にも注目してください。	
\\	手を変え品を変え、いろいろ努力はしたけれど、結局は、元の木阿弥ということになったようだね。 
\\	で「初めに戻ってしまった」の意味になります。
\\	「想像できる限りの」。	
\\	もし失敗したら元も子もないだろ! 
\\	「元も子もない」を
\\	「すべてを失いそうだ」で表現しています。	
\\	不景気を物ともせず、高級ブランドが売り上げを伸ばしている。 
\\	「物ともしない」を 
\\	(まるで…が存在さえしないかのように)を用いて表現しています。「高級ブランド」は 
\\	でもいいでしょう。	
\\	交通事故は災難だったけど、物は考えよう。人身事故でなかっただけよかったと思わなきゃ。 
\\	は「どのようにそれを見るか」、 
\\	は「どのようにそれを考えるか」の意味。	
\\	今度新規ビジネスを始るんだ。それで物は相談なんだけど、君、一枚噛んでくれないか。 
\\	「物は相談」をここでは
\\	「そのことを君と話したいんだが」と言い換えています。	
\\	物は試しにやってはみたけれど、このプログラムつくり、僕が太刀打ちできるような簡単なものじゃなかったね。 
\\	で「どんなことでも試してみる価値があると思って」という意味になります。
\\	にはとうていかなわない」。	
\\	欲張りもほどほどにした方がいいよ。物は八分目がちょうどいいんだよ。 
\\	「八分目」を
\\	を使って表現しています。
\\	は「欲張りな」の意。	
\\	その作戦は両刃の剣だな。こちらも多くの損害を被るだろう。 
\\	「両刃の剣」は 
\\	と言います。	
\\	彼の今回の判断、理解に苦しむね。焼きが回ったということかね。 
\\	で「かつてほど頭が切れない」という意味になります。	
\\	やきもちを焼くなんてみっともないぞ。 
\\	が「やきもちを焼く」に当たる表現です。
\\	は「…にふさわしい」の意。	
\\	ローン地獄で、ボーナスもらっても焼け石に水だよ。 
\\	「ほど遠い」、
\\	「ほんの少し、大海の一滴」。	
\\	コスト意識が高いのは結構だけど、安物買いの銭失い、ということにはならないか? 
\\	が「安物買いの銭失い」に当たる英語の諺です。	
\\	その仕事はおれにはつらいな。やせ馬に重荷だと思う。 
\\	「やせ馬に重荷」を
\\	(その重荷は扱えない)と意訳しています。	
\\	あいつにはまいったよ。どんなに厳しく怒っても、柳に風と受け流すだけだよ。 
\\	で「冷静に対処する」の意味になります。	
\\	競馬の大穴、ビギナーズラックで止めたほうがいいんじゃない。柳の下の泥鰌は2匹いるとは思えないよ 
\\	「柳の下に泥鰌は2匹いない」を
\\	「あんなについていることって一度だけしかないよ」と言い換えています。	
\\	藪から棒に何を言い出すの。今の話題と関係ないでしょう。 
\\	「藪から棒に」とは
\\	「出し抜けに」ということですね。	
\\	病は気からよ。しっかりしなさい 
\\	「気」を
\\	「精神、心」で表現している点に注意。	
\\	あいつは1回ギャンブルで大儲けしてから、病みつきになってしまった。 
\\	で「…に取りつかれる、病みつきになる」の意味。	
\\	闇夜に鉄砲じゃあるまいし、そんな場当たり的なやり方でうまくいくとは思えないよ。 
\\	「当てずっぽう」が「闇夜に鉄砲」に当たる言い回しです。	
\\	彼は引退試合でホームランを打って、有終の美を飾りました。 
\\	で「最高潮で自分のキャリアを終える」の意味になります。	
\\	彼は仕事はできるけど、融通が利かないからやりにくいよ。 
\\	とは「柔軟性がない」ことです。	
\\	ちょっと揺さぶりをかければ、彼は自分から話し出すと思うよ。 
\\	が「揺さぶりをかける」に当たる言い回しです。	
\\	うちの子どもは遠足の日が来るのを指折り数えてるよ。 
\\	が「…までの日を数える」に当たる言い回しです。	
\\	要領を得ない説明をされても困るのはこっちなんだ。 
\\	(意味をなさない、訳がわからない)で「要領を得ない」を表現しています。	
\\	欲に目がくらんでとんでもないことをしてしまった。 
\\	の前には 
\\	が略されています。
\\	は「惑わす」の意。	
\\	ふたりでまじめな話をしているんだから、君、無責任な横槍を入れな いでほしいんだけど。 
\\	「横やりを入れる」とは
\\	「口を差し挟む」ということですね。
\\	「無責任に」。	
\\	あのチームは去年もこの大会で優勝していて、今年も優勝候補との呼び声が高い。 
\\	で「優勝候補として考えられている」の意味になります。	
\\	ちょっと走っただけで息切れする。寄る年波には勝てないよ。 
\\	で「年を取ることは止められない」の意味になります。
\\	でも同じようなことが言えます。	
\\	相手の弱みにつけ込むような振る舞いだけはやめようや。 
\\	「かもにする」、
\\	「弱点」を組み合わせて「弱みに付け込む」を表現しています。	
\\	駐禁で切符を切られた直後にUターン禁止でつかまった。運が悪いというか、弱り目に祟り目というか。 
\\	「物事がますます悪くなる」で「弱り目に祟り目」のニュアンスを表現しています。	
\\	そんな先のことばかり言って。来年のことを言えば鬼が笑うわよ。 
\\	(将来がどうなるかはだれも知らない)で「来年のことを言えば鬼が笑う」を表現しています。	
\\	いつまでもいい思いができると思うかい。楽あれば苦あり、ということになるかもね。 
\\	が「楽あれば苦あり(苦労しなければ楽な思いはできない)」に当たる言い回しです。
\\	「至福」。	
\\	きみと話したってらちがあかないから、ほかの人と話すよ。 
\\	「らちがあかない」を 
\\	(どうにもならない)で表現しています。	
\\	彼はなんでそんなことを言うのか理解に苦しむよ。 
\\	で「僕にはわからない」の意味になります。	
\\	理屈をこねるなんてみっともないぞ。 
\\	「理屈をこねる」は 
\\	、あるいは、
\\	で表現できます。
\\	は「不名誉、恥辱」の意。	
\\	会場は人であふれ、立錐の余地なしという大盛況だったね。 
\\	は「すし詰めの」、
\\	は「余地、場所」の意味。	
\\	何でもかんでも利に走ろうとしてはダメだ。 
\\	で「…から利益を得ようとする」という意味になります。	
\\	女の子二人とドライブに行ったんだって。両手に花だねぇ。 
\\	ここでは「両手に花」を文字通りに
\\	と英訳しています。	
\\	彼、別の女性とも付き合っていたの。両天秤にかけられてたなんてショック。 
\\	が「両天秤にかける」に当たる言い回しです。	
\\	言いたくないけど今日は言わせてもらうぞ。良薬は口に苦しだと思って聞きなよ。 
\\	(耐えなくてはいけない嫌なこと)を使って「良薬口に苦し」を表現しています。
\\	には「嫌なこと」「丸薬」という意味があります。	
\\	あの2人を見ていると、類は友を呼ぶんだなってわかる。 
\\	が「類は友を呼ぶ」に当たることわざです。	
\\	法律の歴史を見ていると例外のない規則はないって感じ。 
\\	が「例外のない規則はない」に当たる言い回しです。	
\\	日本の武道というものは、礼に始まり礼に終わる。 
\\	「礼に始まり礼に終わる」はそのまま 
\\	で表現すればいいでしょう。	
\\	礼も過ぎれば無礼になるんだから、毎日見舞いに行くのはやめなよ。 
\\	(礼儀)の後には 
\\	が略されています。	
\\	また戦争が始まるなんて、歴史は繰り返すとしか言いようがない。 
\\	が「歴史は繰り返す」に当たる言い回しです。	
\\	1回ミスしただけで、無能のレッテルを貼らないでほしい。 
\\	が「レッテルを貼る」に当たる動詞です。ちなみに、「レッテル」はオランダ語です。	
\\	親が子どもの人生のレールを全部敷くなんて、間違っている。 
\\	(綿密に計画を立てる)で「レールを敷く」を表現しています。	
\\	新しいことに挑戦してみたんだけど、労多くして功少なしだったよ。 
\\	と
\\	(…に終わる)を組み合わせて「労多くして功少なし」を表現しています。	
\\	皆が期待している以上、老骨に鞭打ってでも頑張らなくては。 
\\	の文字通りの意味は「年齢にもかかわらず全力を尽くす」です。	
\\	最近、酒を飲むとすぐにろれつが回らなくなる 
\\	が「ろれつが回らない」に当たる表現です。	
\\	3年かかりましたね。「ローマは一日にしてならず」っていうだろう。この作品は大作だよ。 
\\	「ローマは一日にしてならず」は
\\	ですね。	
\\	論より証拠、実際にやって見せればいいじゃない。 
\\	「プリンの味は食べてみれば分かる」が「論より証拠」に当たる英語の諺です。	
\\	旅行でいろんな所に泊まってきたが、わが家に勝る所なしだね。 
\\	が「わが家に勝る所なし」に当たる言い回しです。	
\\	若い時は二度ない、どんとやれ、男なーら、人のやれないことをやれ。 
\\	「若い時は二度ない」はそのまま
\\	と言えます。	
\\	彼も悪気はなくやったことですから、ここは若気の至りということでお許しください。 
\\	「若気の至り」を
\\	「若いがゆえの向こう見ずさのせいにする」で表現しています。	
\\	親ならだれだってわが子荷にならずって思ってるものさ。 
\\	は「不必要な重荷」の意味。	
\\	他人の敷地内を、我が物顔で歩くのはよせ。 
\\	で「その場所を自分が所有しているように振る舞う」の意味になります。
\\	は「所有地」の意。	
\\	この1年はわき目も振らずに勉強してきた。 
\\	で「ひたむきに…に専念する」の意味になります。	
\\	彼は会社をくびになったんだけど、わざわい転じて福となすで、見事独立したから偉いよ。 
\\	「不運を幸運に変える」。
\\	「とんでもない目にあわせる」。	
\\	話すときは気をつけたほうがいいよ。災いは口からって言うし。 
\\	は「口が原因で大変な思いをすることがある」の意味。	
\\	旅行に行きたいと思っていたら、特別ボーナスが出たんだ。渡りに舟だったよ。 
\\	で「…に飛びつく」の意味になります。	
\\	契約を取るためにもまずはなんとかして社長と渡りをつけよう。 
\\	「渡りをつける」を 
\\	(…に近づく)で表現しています。	
\\	彼女は笑い上戸だから、いっしょにお酒を飲むと楽しいよ。 
\\	「笑い上戸」は 
\\	と言います。「泣き上戸」は 
\\	です。	
\\	いつも威張ってる上司が失敗して、笑いをかみ殺すのに苦労したよ。 
\\	で「大声で笑うのをこらえる」の意味になります。
\\	は「…に威張り散らす」の意。	
\\	笑う門には福来たるっていうからいつも笑っているけど、なかなかいいことないな。 
\\	で「笑いは幸運をもたらす」という意味になります。	
\\	彼に相談するなんて、彼女、わらにもすがる思いなんだろうね。 
\\	が「わらにもすがる」に当たる表現です。	
\\	この商売は苦労するばかりで割に合わない。 
\\	には自動詞で「引き合う、もうかる」という意味があります。
\\	は「苦労、骨折り」の意。	
\\	はっきり自己主張しないと、割を食うことになるよ。 
\\	が「割を食う」に当たる言い回しです。
\\	は「…を擁護する」の意。	
\\	おまえだっていつか彼女ができるさ。われ鍋に閉じ蓋だよ。 
\\	「われ鍋に閉じ蓋」には 
\\	という言い方がありますが、ここでは 
\\	で表現しています。	
\\	我に返って、初めて自分のしたことを後悔した。 
\\	「我に返る」は 
\\	で表現できます。また、
\\	「落ち着きを取り戻す」を使ってもいいでしょう。	
\\	そのゲームの面白さといったら、思わず我を忘れるほどだった。 
\\	で「…に夢中になる」の意味。	
\\	私の母、私に輪をかけておしゃべりなの。 
\\	この場合の「輪をかけて…」は
\\	と比較級を使って表現できます。	
\\	私彼女に嫌われているのかしら。なんか気に障る事でも言ったかな。 
\\	には
\\	に嫌われる」という意味があります。	
\\	きりのいい所で休憩にしようか。 
\\	(都合がいい)を使って「きりのいい」を表現しています。	
\\	この辺で仕事を切り上げて、夕食を食べに行きましょう。 
\\	で「今日は終わりにしよう」の意味になります。
\\	は「(仕事を)やめる」の意。	
\\	ちょっと日本語ばかり使ってたから、とっさの時に英語が出てこない。早く頭を切り替えないと。 
\\	の文字通りの意味は「ギアを入れ換える」ですね。
\\	を使っても同じような意味になります。	
\\	きりのいいところで声かけてね。 
\\	で「(仕事を)止められることろで」の意。
\\	は、文字通り「声を掛けてね」の意味になります。	
\\	今の職場には不満だらけ。言い出したらきりがないよ。 
\\	で「(言い出したら)際限なく言える、止まらない」の意味になります。	
\\	胃がキリキリいたくて。 
\\	「胃がキリキリ痛む」は
\\	「刺すように胃が痛む」と言えばいいでしょう。	
\\	いわゆる族議員の既得権益にどこまで斬り込んでいくかが、今後の課題だ。 
\\	(…を急に取り去る)で「斬り込む」のニュアンスを表現しています。
\\	は「骨抜きにする、略奪する」の意。	
\\	毎年20日有給休暇がもらえますが、最高で保持できるのは40日分です。使い切れなかったのは、切り捨てられます。 
\\	で「捨てられる」、
\\	で「(権利などを)失う、失う」の意味。	
\\	毎日旅行費貯めるために切り詰めた生活をしているんだよ。 
\\	で「最低限必要なものだけで暮らす」の意味になります。	
\\	ここに置いてあった新聞の切り抜きどこへ行っちゃったんだろう。 
\\	「切り抜き」は
\\	ですね。
\\	(まさしく、ちょうど)は副詞。	
\\	今中途半端なので、きりのいい所までやってから昼食とるので気にしないで先行ってていいよ。 
\\	「きりのいい所」をここでは
\\	「(仕事を)中断するのにいい所」で表現しています。	
\\	プライベートと仕事は切り離して考えているので、そういう個人的な感情を仕事には持ち込みません。 
\\	で
\\	と
\\	をはっきり分ける」の意味になります。	
\\	その問題とこの問題は切り離して考える必要がある。 
\\	「切り離す」は
\\	を使えばいいでしょう。
\\	は「分離する、切り離す」の意。	
\\	切り札は最後まで見せるな。見せるなら、さらに奥の手を持て。 
\\	「切り札」は
\\	「(トランプの)エース」でいいでしょう。
\\	で「エースをいざという時のためにとっておく」の意味になります。	
\\	彼の母親は、その食堂を一人で切り盛りしていた。 
\\	には「経営する、運営する」という意味があります。
\\	は「厳しく監督する」の意。	
\\	誕生日までもう1カ月きったんだよね。 
\\	「1カ月きる」を 
\\	(あと1カ月ない)で表現しています。	
\\	この包丁、切れ味がいいね。 
\\	自動詞
\\	に
\\	をつけて「よく切れる」という意味になります。
\\	「切れ味」。	
\\	きれい事ばかり言っていたら、世の中渡っていけないよ。 
\\	「きれい事」はそのまま
\\	と言えばいいでしょう。
\\	は「愛想がいい」の意。	
\\	うちのかみさんとってもきれい好きでね。 
\\	が「きれい好きな」に当たる形容詞です。	
\\	ずっと上司にはむかついていたみたいだけど、彼女、今日という今日はとうとうキレて出て行ったね。 
\\	「キレる」は
\\	「ポキッと折れる」を使って表現できます。	
\\	近頃、すぐにキレる若者が多い。 
\\	とは「感情をあらわに出す」、
\\	は「全く抑制せずに」の意味になります。	
\\	先日父親と初めてゴルフに行きました。うちの父は怒りっぽいので、友だちも一緒だったので突然きれないか気が気でありませんでした。 
\\	とは「怒り出す」ことです。
\\	は「怒りを爆発させる」。
\\	は「怒りっぽい」の意。	
\\	こんな高価な物いただいてしまって。かえって気をつかわせてしまいましたね。すみません。 
\\	を思いやる」が「気を使う」に当たる表現になります。
\\	とは、思ってもみない贈り物をされた時に感謝の気持ちを表すことばです。	
\\	彼って、健康にはかなり気をつかっているように見えたけど、すぐに病院にいかなったせいで入院するはめになったそうだ。 
\\	「かなり気をつかう」は
\\	で表現できます。
\\	は「入院させる」の意。	
\\	今はホームステイだから、やっぱりテレビとかシャワーとか使うのに時間がかち合わないようにとか気をつかう。 
\\	で「…しないように注意する(気をつかう)」の意味になります。
\\	を使ってもいいですね。	
\\	お母さんのところいくと気を遣うからやだ。 
\\	で「物事がちゃんとできているかどうかわからない」の意味。
\\	でも「気を遣う」のニュアンスは表現できますね。	
\\	やはり部下が、既婚者だと仕事を与える時に気を使ったりつるものですか?あまり、残業になりそうな仕事は渡さないとか。 
\\	「よく考える」、
\\	「気を付ける」で「気を使う」を表現しています。	
\\	このワイン甘くて飲みやすいけど、気をつけないと飲みすぎるよ。しかも、アルコール度は、ビールの倍だから。 
\\	「気を付ける」は
\\	を使います。「飲みやすい」「倍」の表現の仕方も一緒に覚えておくといいでしょう。	
\\	クサクサすることばっかりだけど、気を取り直して頑張っていきましょう! 
\\	「しっかりする、落ち着く」。これで「気を取り直す」を表現できます。
\\	「落ち込む、ふさぎ込む」。	
\\	リスニングは集中力が大事だと思う。ちょっと気を抜くともう話しについていけなくなる。 
\\	には「(気を)緩める」という意味がありますので、これを使えば「気を抜く」のニュアンスが出せるでしょう。
\\	「流れ」。	
\\	今度の試験はかなり難しいらしいので、気を抜けない。 
\\	この場合の
\\	は「緩める」という意味ですね。
\\	「大変・困難なこと」。	
\\	ああいう有名な人って大変だよね。だって、いつ写真撮られるかわからないし、いつも気を引き締めておかないといけないじゃない。 
\\	(警戒している、油断しないでいる)を使って「気を引き締める」を表現しています。
\\	は「構える」の意。	
\\	そんなに落ち込まないで。そうだ、カラオケにでも行こうか?気を紛らわすのにちょうどいいよ。 
\\	で「…から気をそらす」の意味になります。
\\	で「~に…させる」の意。	
\\	もし、私が言ったことで気を悪くさせちゃったのならごめんなさいね。 
\\	には「気分を害する」という意味があります。	
\\	だれも見向きもしない100円均一本のコーナーで、稀覯本を発見することぐらい嬉しいことはないね。 
\\	特に「均一」と表現する必要はありません。
\\	で十分です。
\\	「無視された」の意。「嬉しいことはない」を
\\	で表現している点にも注意してください。	
\\	社員は皆社長の言を金科玉条としている。 
\\	「金科玉条」は
\\	「黄金律」あるいは
\\	「絶対律」と表現できます。	
\\	この金額でA駅まで行けますか? 
\\	「この金額で」は
\\	、あるいは、
\\	で表現できます。	
\\	先日主人が急にお腹が痛いというので医者に連れていったら、緊急でいろいろ検査をされて1万5千円も請求された。 
\\	で「緊急で検査をする」の意味になります。
\\	は「緊急に」の意。	
\\	ちょっとそれ禁句だってー。もう! 
\\	「禁句」は 
\\	(タブー)で表現できます。
\\	を使ってもいいでしょう。	
\\	お前、禁酒してたんじゃないのか。 
\\	は「アルコールを禁止する」という意味です。	
\\	いつまでも親に金銭的な負担をかけちゃだめだよ。しっかりしなよ!! 
\\	「金銭的な負担をかけない」を 
\\	(…のすねをかじらない)を使って表現しています。
\\	は「親のすねをかじる」の意。	
\\	緊張して来た…。 
\\	「緊張する」は
\\	で表現します。	
\\	何も悪いことしてないのだけど、警官が近寄ってくると、妙に緊張したりしない? 
\\	「緊張する」は
\\	で表現します。これと
\\	「変に」、
\\	「ちょっと」を組み合わせて、「妙に緊張する」を表現します。	
\\	フィットネスクラブで運動しすぎて、すごい筋肉痛です。 
\\	で「体中が痛い」の意味。
\\	は「ひどい筋肉痛になる」の意。	
\\	雪道での急ブレーキは禁物だよ。 
\\	とは名詞で「やってはいけないこと、禁じられていること」という意味になります。	
\\	運動不足のせいか、筋力が落ちたような気がする。 
\\	「筋力」は
\\	でいいでしょう。	
\\	偽善的なのはいやだが、偽悪的なのも嫌いだな。 
\\	ここでは「偽悪的」を 
\\	(威嚇するような振りをする)、
\\	(悪意に満ちた振りをする)で表現しています。
\\	は「偽善」、
\\	は「偽善者」の意。	
\\	あの会社のコンピュータチップに関する技術力はすごい。 
\\	「技術力」は 
\\	でいいでしょう。
\\	は「信じられない、驚くべき」の意。	
\\	きれい事ばかり並べやがって、この、偽善者めが~! 
\\	「偽善者」、
\\	もっともらしく言う、うわべを飾る」の意。
\\	「偽善の」。	
\\	なんかあそこの店でカードで買い物すると、レシート切るとき店の裏に1度カードを持っていくから、ちょっと心配なんだよね。最近、カード偽造とか多いし。 
\\	が「カード偽造」に当たる表現です。
\\	は「落ち着かない」、
\\	は「偽造する」の意味。	
\\	今年のサミットの議題は何だろう。 
\\	で「議題になる(予定されている)」という意味になります。	
\\	重い物を持ち上げようとしたら、ぎっくり腰になってしまった。 
\\	には「痛める、筋を違える」という意味があります。
\\	「捻挫する、筋を違える」。	
\\	私の人生このままでいいのかな?と疑問を感じ始めた。 
\\	で「疑問を持ち始める」の意味。
\\	の使い方にも注目。	
\\	違うよ、その逆だよ! 
\\	「逆、逆さま」には、
\\	と
\\	2通りの言い方があります。	
\\	彼女のほうが悪いのに逆ギレされちゃった。 
\\	で「切れる」を表現した上で、
\\	を使って「逆」を説明しています。	
\\	現状から逆算すると、去年の今頃が転機だったな。 
\\	この場合の「逆算する」は 
\\	(振り返る)表現すればいいでしょう。
\\	は「逆に数える」の意。	
\\	犯人の電話を逆探知できませんでした。 
\\	が「逆探知する」に当たる動詞です。
\\	は「犯人」の意。	
\\	大統領選は、逆転につぐ逆転でおもしろい。 
\\	で「形勢を逆転させる」という意味になります。
\\	は「紆余曲折」の意。	
\\	昨日の試合で逆転勝ちして、巨人、これで10連勝! 
\\	「逆転勝ちする」をここでは
\\	を使って表現しています。「10連勝」の言い方にも注目。	
\\	その商品は、初め日本で売り出されたときは全然売れなかったが、アメリカでヒット商品になり、日本に逆輸入された。 
\\	が「逆輸入」に当たる表現です。
\\	は「大失敗に終わる」の意。	
\\	え~。あの映画ってただのギャグの固まりみたいなやつじゃん。やっぱり映画と言えば感動がないと。 
\\	「ギャグ」は
\\	をそのまま使えばいいでしょう。
\\	とは体を使ったコント
\\	のようなものです。
\\	「…に過ぎない」、
\\	「感動させる」。	
\\	逆光だから撮れないよ(写真を撮ろうとした時)。 
\\	「逆光」は
\\	「光がまぶしすぎる」で表現すればいいでしょう。	
\\	実際会ってみて、あなたが想像していた人物と私とであまりギャップがなかったのなら良いのですが。 
\\	「ギャップ」はそのまま
\\	が使えます。	
\\	相手をギャフンと言わせてやった。 
\\	「ギャフンと言わせる」は、 
\\	「言葉が出ない程にやっつける」、あるいは、
\\	「打ち負かす」、
\\	「圧倒する、ペチャンコにする」、
\\	「意気消沈させる」などを使って表現すればいいでしょう。	
\\	自分にはギャンブル運はないとわかっているから、ギャンブルは遊びでやるのがいいと思う。 
\\	とは「幸運の女神」のことです。
\\	は「楽しみのために」の意。	
\\	ご飯食べながら携帯で話すなんて、お行儀悪いわよ。 
\\	「行儀が悪い」には、色々な言い方があります。
\\	など。	
\\	この店、急に人気が出ちゃって、いつも行列ができている。 
\\	「行列」は
\\	を動詞あるいは名詞として使って表現できます。
\\	は
\\	の意味で使われています。	
\\	行列のできる店だと聞いて来てみたけれど、このスパゲティー全然おいしくないじゃない。 
\\	で「入るために並ぶ」の意味になります。	
\\	何か食べるつもりで早く出たのですが、着いてみたら時間ギリギリで食べる時間なんてなかったんです。 
\\	「ギリギリ」ということは
\\	「余った時間がない」ということですね。
\\	「ちょうどぴったりに」。	
\\	あー義理チョコあげるのも大変。 義理チョコもらってうれしいかな? 
\\	「義理チョコ」を文字通り
\\	と英語に訳してみました。	
\\	義理人情なんて今の若いのには分からないんだろうな。 
\\	ここでは「義理」を
\\	「義務感」、「人情」を
\\	「人間愛」で表現しています。	
\\	インクが爪の中まで食い込んじゃった。こうなるとなかなか落ちないんだよなぁ。 
\\	ここでは「食い込む」を 
\\	(じわじわ染みこむ)で表現しています。
\\	を使ってもいいでしょう。	
\\	私はあなたと同じ空間にいられるだけで幸せです。 
\\	ここでは「同じ空間にいる」を
\\	「あなたと同じ空気を吸う」と言い換えています。	
\\	彼女の新曲は空前の大ヒットとなった。 
\\	は共に「記録破りの」という意味になります。	
\\	ウインブルドンが始まったので、毎日テレビに釘付けになっている。 
\\	「釘付け」は
\\	と言います。
\\	は「接着剤」のことです。また、
\\	「根が生える」も使えます。	
\\	この辺いっつも臭くない?それとも、私の服が臭うのかな。 
\\	「臭い」は
\\	(動詞)、
\\	(形容詞)で表現します。	
\\	そんなくさい芝居をしてもお見通しよ。 
\\	「臭い芝居」をここでは「見え透いた
\\	芝居はするな」と言い換えて表現しています。
\\	は「何を企んでいるか分かっている」の意。
\\	「やめてくれ」。	
\\	そんなにくさらないで。次はきっとうまくいくさ。 
\\	「くさる」とは「がっかりして、落ち込む」ことですから、
\\	を使えばいいでしょう。
\\	「嘆く、不満を言う」。	
\\	これ腐ってるよ。臭いで分かるよ。 
\\	は「腐ったような臭いがする」の意味。
\\	は「腐っている」、
\\	は「悪くなる」の意。	
\\	彼とは学生時代からの腐れ縁でね 。 
\\	「腐れ縁」をここでは
\\	「彼との縁を切れない」、
\\	「彼とくっついている」で表現しています。	
\\	このエッセイは私のあらゆる知識を駆使して書き上げたものです。 
\\	が「駆使する」に当たる言い回しです。	
\\	くじ運がある人とない人っていると思うよ。 
\\	「くじ運がある」は簡単に
\\	でいいですね。	
\\	党として、彼に議員辞職を勧告したことは苦渋の選択だった。 
\\	とは「苦しい決定」のことです。
\\	は「苦痛に満ちた」の意。	
\\	くすん、あんなに楽しい夢を見ていたのに途中でいきなり起こされるなんて。 
\\	は「嫌なこと、がっかりすること」の意味。	
\\	明日お天気崩れるみたい。せっかくの日曜日なのに。雨が降ると、気分までブルーになるんだよね。 
\\	「崩れる」とは
\\	(悪くなる)ということですね。
\\	は「気持ちが落ち込む」の意。	
\\	彼女って癖がなくって付き合いやすい人だよね。 
\\	は「癖、奇癖」の意味。
\\	「一緒にいて楽」。	
\\	癖毛なので、空気が乾燥している日は、広がちゃってもう大変よ。 
\\	「癖毛」は
\\	でいいでしょう。
\\	は「縮れ毛」のことです。
\\	「ふわりとふくらむ」。	
\\	嬉しいくせにー。もっと、喜びなよ。 
\\	「くせに」は
\\	「分かってるんだから」で表現すればいいでしょう。
\\	「大喜びしている」、
\\	「隠す、抑える」。	
\\	ディズニーランドって千葉にあるくせになんで東京ディズニーランドっていうのだろう。 
\\	「くせに」は
\\	だとしても」、あるいは、
\\	にもかかわらず」で表現すればいいでしょう。	
\\	覚えたフレーズを実際に使って通じるとすっごくうれしいね。癖になりそう。 
\\	は「病みつきになる」、
\\	は「習慣性の、中毒性の」という意味。	
\\	僕の兄にいわせると、お風呂で読書することは、癖になるらしい。 
\\	とは「習慣(癖)を形成する」という意味です。	
\\	運動しているときに肩痛めちゃってさ、こういうのって一度やると癖になるんだよね。 
\\	(絶えずつきまとう)を用いて「癖になる」を表現しています。
\\	は「また起きる傾向がある」の意味。	
\\	あいつは、とんだ曲者だ! 
\\	「曲者」は
\\	「人を陥れるやつ」、
\\	「策士」で表現できます	
\\	くそっ。あなただけには負けたくなかった。(試合やゲームで) 
\\	が「くそっ」に当たる言葉です。	
\\	自分ではもうちょっと頑張れると思っていても、実際には体はすでにくたくたで、その状況で無理して頑張っていることがあるんだよね。 
\\	で「肉体的に消耗した、くたくたになっている」の意味になります。
\\	は「こつこつ働いていく」、
\\	は「疲労困憊して倒れる」の意味。	
\\	骨折り損の草臥れ儲け。 
\\	「骨を折る(一生懸命働く)」を
\\	で表現しています。一生懸命に頑張ったけれど、それで何が見せられるかといったら、
\\	「疲労」だけだった、と言っているのです。
\\	「疲労」。	
\\	その牛肉、500グラムください。 
\\	「下さい」は
\\	と
\\	を使って表現します。	
\\	くだらないお土産をもらった。 
\\	とは「安っぽいお土産」のこと。
\\	「つまらないもの、ちゃちなもの」の意。	
\\	そんなくだらない話をいつまでしているんだ。 
\\	で「くだらない戯言」の意味になります。	
\\	このワイン口当たりが良くて飲みやすいわ。 
\\	「口当たりがいい」は
\\	「滑らか」、あるいは、
\\	「芳醇な」で表現できます。	
\\	あの日に話したことは内緒にしておきたいので、私たちの中で口裏を合わせておかなきゃね。 
\\	と
\\	を組み合わせることで「口裏を合わせる」のニュアンスが出せます。	
\\	あの人は口がうまいから、ついその気になっちゃうんだよね。 
\\	「口がうまい」は
\\	で表現できますね。	
\\	僕は口が堅いから、そのことは誰にも言わないよ。 
\\	の文字通りの意味は「秘密を守るのがうまい」です。	
\\	あいつは口が軽いからこのことは言わない方がいい。 
\\	「口が軽い」は
\\	「口にしまりがない」、 
\\	「胸の内にしまっておけない」で表現できます。	
\\	遅刻するなと口が酸っぱくなるほど言ったのに、また、遅刻したのか。 
\\	「口が酸っぱくなるほど言う」とは
\\	「何度も何度も言う」ということですね。	
\\	つい口が滑って、昨日のこと彼女に言っちゃったんだよ。 
\\	は「(口が滑って)うっかり言ってしまう」、
\\	「口を滑らせる」の意味。	
\\	あの子、小学生にしては口が達者だね。 
\\	で「たいした話し手」の意味。
\\	は「…の割には」の意。	
\\	ああ言えば、こう言う。まったく、口が減らない奴だ。 
\\	「口出しができない」で「口が減らない」を表現しています。
\\	「返事を用意している」の意味。	
\\	やっぱりいつも英語を話していないと急に話そうとしても口がまわらないね。 
\\	とは「錆び付く、下手になる」の意味です。
\\	「長い間ほったらかしておく」。	
\\	私に一目ぼれしたと言うのが、彼の口癖せでした。 
\\	「口癖」とは
\\	「いつも言う」ということですね。	
\\	この手の商品は口コミで広まるらしい。 
\\	「口コミ」には
\\	というピッタリの表現があります。
\\	「宣伝、広く知れ渡ること」の意。	
\\	口答えするな!後でゆっくり話は聞いてあげるから。 
\\	ここでは
\\	と
\\	「口答え」を組み合わせて表現しています。	
\\	なに口ごもってるんだよ。言いたいことをはっきり言えよ。 
\\	が「口ごもる」に当たる言い回しです。	
\\	何!その口先だけの心のこもってない返事は! 
\\	「口先だけの心のこもっていない」とは
\\	「全く偽善的な」ということですね。	
\\	お腹いっぱいなんだけど、口寂しくて、なんか食べちゃうんだよね。 
\\	「口寂しい」を
\\	「まだ何か食べていたい」と言い換えて表現しています。
\\	「間食する」。	
\\	彼は、あの曲ってなんだっけ、と言って曲を口ずさんだ。 
\\	が「口ずさむ」に当たる動詞です。
\\	「小節」。	
\\	あいつは口だけで、実行が伴わない。 
\\	で「口だけで、何もしない」の意味になります。
\\	は「口先だけのことを言う」の意。	
\\	いちいち口出ししないでよ。これは私のことなんだから私が自分で決める。 
\\	が「いちいち口出しする」に当たる言い回しです。	
\\	アイスクリームには口直しにクラッカーが添えられています。 
\\	ここでは「口直し」を
\\	「舌をきれいにする」を使って表現しています。	
\\	この料理、お口に合いますでしょうか。 
\\	で「(その人の)好みに合う」の意味。
\\	は「気に入る」の意。	
\\	彼、口には出さないけど、心の中では心配しているかもしれませんね。 
\\	「口には出さない」は
\\	「表には出さない」、
\\	「内にしまっておく」で表現すればいいでしょう。	
\\	あいつに口の利き方を教えてやれよ。 
\\	のあとに 
\\	(上司に対する)を付け足してもいいでしょう。
\\	は「きちんと」の意。	
\\	あの人はほんとうに口八丁手八丁だから。 
\\	「口八丁手八丁」は
\\	「口もうまいし」、
\\	「仕事もできる」と説明すればいいでしょう。	
\\	口幅ったい言い方をするようですが、ホームページ作成に関しては誰にも負けないと思います。 
\\	「口幅ったい」とは
\\	「大きいことを言う、自慢する」ということですね。	
\\	あんな調子のいいこと言っておいて、ホント口ばっかりなんだから! 
\\	「口ばっかり」は
\\	「話だけ(名詞)」で表現できます。	
\\	「口ばく」は
\\	と言います。
\\	「やりおおす、うまくかわす」。	
\\	僕、口べただから、思ったことがうまく言えないんだ。 
\\	で「話し上手ではない」。
\\	「伝える]、
\\	を通じさせる」の意。	
\\	僕は口べただから、自分の気持ちを相手にうまく伝えることができないんだ。 
\\	の意味は「ハッキリ言えない」です。
\\	~で「…を~にわかってもらう)	
\\	親に向かって生意気な口をきくんじゃありません。 
\\	で「反抗的な(ふてぶてしい)口のきき方をする」の意味になります。
\\	は「無礼に、礼儀をわきまえずに」の意味。	
\\	お前が口を挟むと、話がややこしくなるだけだよ。 
\\	は「口を差し挟む」、
\\	「余計なお節介をする」の意味です。	
\\	このコースター、グラスの底にくっついちゃうから使いにくいよね。 
\\	「くっつく」は
\\	を使います。皆さんよく知っている「ステッカー
\\	の動詞形です。
\\	には「~に固執する」の意味もあります。	
\\	彼は僕の意見をくつがえした。 
\\	で「ひっくり返しす」の意味になります。	
\\	靴擦れしちゃって、歩けない。 
\\	「靴擦れ」は
\\	を主語にして、動詞
\\	「擦りむく」を使ってで表現します。その他に、
\\	「死ぬほど痛い」、
\\	「締め付ける」、
\\	「まめになる」などを用いても同じ様な症状を説明できます。	
\\	あっ、靴のひもが切れた。何か嫌なことが起きなければいいけど。 
\\	この場合は
\\	「靴ひもを切る」でいいでしょう。	
\\	カフェでゆっくりコーヒー飲んでくつろいで過ごす休日が好きです。港の近くのオープンカフェなんかだったらもういうことないね。 
\\	「くつろぐ」は 
\\	(リラックスする)で表現できます。
\\	は「居心地のいい」の意。	
\\	ねねね、今の説明ってちょっとくどかったかな。失敗!! 
\\	「くどい」は
\\	「長たらしい」を使って表現できます。
\\	は「少しやり過ぎる」の意。	
\\	くどくなってきたから、もう食べられない。 
\\	には「濃厚な」という意味があります。
\\	「しつこい」と
\\	「脂っこい」を組み合わせてもいいでしょう。	
\\	彼女をくどいても無駄だよ。来月結婚するんだから。 
\\	で「…に言い寄る」の意味。
\\	で「(性的に)誘惑する」の意。	
\\	苦肉の策で企画したんですが。本がよくが売れまして。 
\\	とは「もがき苦しんで(窮余の策として)」の意。また、
\\	「いいものがないも思い浮かばない」でも表現できますね。	
\\	我が社は倒産寸前だが、首の皮一枚でつながっている。 
\\	ここでは
\\	「風前の灯火である」を使って「首の皮一枚」のニュアンスを表現しています。
\\	は「…に瀕している、今にも 
\\	しそうな」、
\\	は「かろうじて」の意。	
\\	いつ首を切られるか分からないなんて、窓際族はつらいね。 
\\	で「首を切る」、
\\	で「首を切られる」の意味。「窓際族」の表現の仕方にも注目。
\\	「解雇通知」。	
\\	あの人の帰りを首を長くして待っている。 
\\	は「熱心に」、
\\	は「大いに楽しみにされている」の意味。	
\\	あの企画は、九分九厘うまくいかないだろうな。 
\\	は「百の内一つのチャンスもない」、
\\	は「見込みはゼロに等しい」の意味。	
\\	現実と空想の区別ができなくなってきている若者が増えている気がする。 
\\	ここでは「区別する」を
\\	で表現しています。	
\\	目の下に隈ができてるぞ。仕事のし過ぎじゃないのか。 
\\	で「目の下の隈」の意味になります。	
\\	平社員の私にとって、社長は雲の上の人だ。 
\\	(別の世界に住んでいる)で「雲の上の人」を表現しています。	
\\	なんか雲行きが怪しいけど、今日って雨降るのかな。朝、急いでて天気予報見てくるの忘れた。 
\\	(雲の様子が嫌だね)で「雲行きが怪しい」を表現しています。	
\\	曇ってきたね。雨が降るのかな。 
\\	(曇る)は動詞としても使えます。もちろん 
\\	とも言えます。	
\\	そんな雲をつかむような話をしたって、どうしようもないだろう。 
\\	ここでは「雲をつかむよな」を
\\	(あいまいな)、
\\	(ぼやけた)を使って表現しています。	
\\	答えられなくてくやしかった 。 
\\	「くやしい」は
\\	「ひどい思いをする」と言い換えればいいでしょう。また、
\\	「悔しい思いをする」という言い方もあります。	
\\	彼女たちは決勝で負けて、悔し涙を流した。 
\\	で「悔し涙を流す」の意味になります。
\\	「苦々しく」。	
\\	つまらないことで、あまりくよくよしない方がいいよ。 
\\	には「くよくよ考える」と意味があります。	
\\	添削してもらう必要がないくらい完璧な英語を書けるようになりたい。 
\\	この場合の「…くらい」は 
\\	で表現できます。
\\	は「完全な、欠点のない」の意。	
\\	何が何でも食らいついていくという気迫が彼に求められている。 
\\	(頑張る、しがみつく)で「食らいつく」を表現しています。
\\	は「しつこさ」、
\\	は「ねばり強く」の意。	
\\	外国通貨のときは、数字の位取りに気を付けないと。 
\\	「位取り」は
\\	と言います。	
\\	最近暗いニュースばっかりだね。しかも、メディアって1つの事件をしつこいぐらいに長期に渡って報道するしね。 
\\	「暗いニュース」を 
\\	(気が滅入るニュース)で表現しています。
\\	は「…に執着する、しがみつく」、
\\	は「いやになるほど」の意。	
\\	三叉路でどっちへ行っていいか分からなくて止まったら、後ろからクラクションを鳴らされた。 
\\	「クラクションを鳴らす」は
\\	と言います。	
\\	コンピュータがクラッシュして、すべてのデータを失ってしまった。 
\\	が自動詞・他動詞としてそのまま使えます。	
\\	上級者と比べて私の英語力には何がたりないのですか? 
\\	「比べる」は
\\	と
\\	を組み合わせて表現します。
\\	「欠けている、不足している」。	
\\	あの子と、私を比べるときがあると思うけど、それはどんな時?どんな事で? 
\\	「比べる」は
\\	を使って表現します。	
\\	有給休暇は毎年使い切れなかった場合は繰り越しができますが、最大30日まででオーバーした分は切り捨てられます。 
\\	で
\\	に繰り越される、持ち越される」の意味になります。	
\\	今の仕事はきっと夏だけなので、冬に向けて何か考えておかないと生活が苦しくなるかもしれません。 
\\	で「収支を合わせるのに困る」、
\\	で「家計上大変なことになる」の意味。	
\\	前を歩いていた猫が車に引かれそうになったので、助けようと思って行ったら自分がひかれそうになった。笑っちゃうでしょ? 
\\	で「車に引かれそうになる」の意味になります。	
\\	紀香?涼子?ど~んといらっしゃい!僕は来るもの拒まずだからね。 
\\	(一般的な女性の名前)は
\\	に代えても構いません。
\\	は「断る、拒絶する」の意。	
\\	この製品にまたクレームをつけられました。 
\\	この場合は 
\\	は使えません。「クレームをつける」とは 
\\	(文句を言う)ことですね。
\\	は「苦情を訴える」の意。	
\\	坂を下る時は,転ばないようにくれぐれも注意してください。 
\\	「くれぐれも注意する」のニュアンスを 
\\	(…しないようにできるだけのことをする)で表現しています。	
\\	日本語の勉強頑張って下さいね。私も英語を学んでいるので苦労はよくわかります。 
\\	「苦労が分かる」とは
\\	「それがいかに大変かが分かる」ということですね。	
\\	シャワーの位置が固定されているから、バスタブを洗っても流すときにかなり苦労するのだけど。 
\\	で「…するのに苦労する」の意味になります。
\\	は「固定する」、
\\	は「困難なこと」の意。	
\\	彼は苦労人だから、言うことが違うね。 
\\	で「いろいろな事を見聞きしてきた」という意味になります。	
\\	この豚汁、だしが利いてるし、具も多くておいしいね。 
\\	とは「塊がたくさん入っている」という意味です。これで「具が多い」を表現しています。
\\	は「…でいっぱいの」の意。	
\\	風邪引いて病院行くと、当然具合が悪い人が病院には来ているわけで、そういうなかで待っていると余計悪くなりそうな気がする。 
\\	「具合が悪い」は 
\\	(気分が悪い)ということですね。
\\	で表現してもいいでしょう。	
\\	先週、友人と旅行の事を話していたら、偶然にも同じ飛行機で行く事を知った。 
\\	!は「何という偶然」という意味。	
\\	グズグズしてないで、さっさと支度しなさい。 
\\	で「グズグズ/ノロノロする」の意味。
\\	は「準備」。	
\\	何をぐずぐずしてるんだよ。早くしないと電車の時間に間に合わないよ。 
\\	で「ぐずぐずする(だらだら足をひきずる)」、
\\	「ぐずぐずする」の意味を表します。	
\\	もう少し具体的に打合せしましょう。 
\\	が副詞で「具体的に」の意味になります。	
\\	あそこに書いてる日没からプレイ料金が安くなるというのは、具体的に何時からのことを指しているのですか? 
\\	(何時から何時まで)で時間を具体的に聞いていることになります。
\\	は「厳密には」の意。	
\\	何ぐだぐだ言っているの?そんな事言う暇があったら、家の事手伝いなさい。 
\\	で「(同じ様なことを)繰り返して言う」の意味になります。
\\	「不平を言う」、
\\	「くだらないことを言う」。	
\\	ごめん、つい愚痴を言っちゃって。 
\\	「愚痴を言う」は
\\	で表現できます。	
\\	ブリスベンに来て、ゴールドコーストに行かないで日本に帰る人なんて、あなたぐらいだよ。 
\\	ここでは「あなたぐらい」を 
\\	で表現しています。	
\\	あの二人って、ぐるだったんだ!くっそーやられた。悔しい。 
\\	「ぐる」は
\\	で表現できます。
\\	は「裏切る」の意。	
\\	今駅前工事してて通行止めになってるからあっちからグルッと回ってこなきゃいけないよ。 
\\	は「遠回りする、迂回する」の意味。
\\	でも同じことがいえますね。	
\\	私はねー、こう見えても中学生の頃、めちゃめちゃグレてたんですよ。 
\\	「ぐれていた」は 
\\	「不良」、あるいは、
\\	「手に負えない奴」、
\\	「乱暴物」を使って表現すればいいでしょう。	
\\	では、彼に敬意を表しまして、乾杯しましょう。 
\\	が「…に敬意を表して」に当たる言い回しです。
\\	「尊敬の印」。	
\\	ちょっと今日は給料日前なので、自分の食べた分だけ支払うって事でいいかな?私って情けなーぃ。もっと計画的にお金は使わないとだめだね。 
\\	この場合の
\\	は「予定を立ててお金を使う」の意味になります。
\\	は「一文無しになる」の意。	
\\	こういうときには経験が物を言うんだよ。 
\\	で「経験を必要とする」の意。	
\\	彼が敬虔なクリスチャンだとはね。 
\\	が「敬虔な」に当たる形容詞です。	
\\	次、転職するとしたら、自分の今までの経験を活かせる職につきたい。 
\\	「経験を生かす」は
\\	「経験を活用する」で表現できますね。	
\\	彼はもっとこの会社で経験を積む必要があるね。 
\\	は「蓄積する」という意味です。
\\	も使えますね。	
\\	(最近の若者には)そういった傾向があります。 
\\	は「…する傾向にある」の意味。
\\	を使ってもいいですね。	
\\	蛍光灯が切れちゃてるから、新しいのと取り替えなくては。 
\\	この場合の「切れる」は
\\	で表現します。
\\	は「蛍光灯」のこと。	
\\	彼について覚えていることと言えば、確か私と初めて会った時、彼は敬語で話していたくらいかな。 
\\	「敬語」には
\\	という言い方もありますが、ここでは
\\	「丁寧な言葉」を使えばいいでしょう。	
\\	あなたの記事が掲載されている雑誌は何でしたっけ? 
\\	この場合の「掲載する」は
\\	で表現できます。	
\\	こんなことで警察沙汰にはしたくないの。 
\\	で「このことを警察に届ける(知らせる)」の意味。	
\\	穴埋め問題か、正誤問題か、長文和訳か、試験の形式を教えて下さい。 
\\	「形式」は
\\	で表現できますが、それを使わなくても(第二英訳)同じような意味が表せます。	
\\	最近は形状記憶のシャツが多いから、アイロン掛けが楽になった。 
\\	「形状記憶」は
\\	といいます。形容詞として用いる場合には
\\	とハイフンを付けます。	
\\	何事も継続は力なりでしょ。 
\\	は「継続性、ねばり強さ」の意味。
\\	「うまく行く」。	
\\	この案件に関しては、継続審議と致します。 
\\	「継続審議」とは
\\	「次の会議に 
\\	を持ち越す」ということですね。	
\\	そこ、携帯入るのかな? 
\\	まず、「携帯電話」は
\\	と言います。「携帯入る」とは、「受信可能な範囲にある」ことなので、これは
\\	あるいは
\\	で表現できます。また、
\\	を使っても同じようなことが言えます。「電波が届く」は	
\\	電車に乗るときぐらい、携帯電話のスイッチを切れよな。 
\\	「携帯電話」は 
\\	、「公衆電話」は 
\\	といいます。	
\\	経費削減でコーヒー用の紙コップは1人1日1個しか使えないから、違うものを飲むときは洗って使い回さなければいけません。 
\\	で「経費削減の努力」の意味になります。「洗って使い回す」の表現の仕方にも注目。	
\\	軽蔑されてもしょうがないけど、ブランド物持ってるとなんか安心できるの。 
\\	は「周りの人は何か言うだろけど、私は気にしないの」、
\\	は「言いたかったら好きなことを言えばいいのよ」の意味になります。「ブランド物」をここでは
\\	で表現しています。	
\\	レッスンは1年間の契約だから、来週でここでのこのレッスンは最後になります。今、継続するかほかのコースを受けるか自分で勉強するか迷っています。 
\\	で「一年コースの契約になっている」の意味。
\\	は「登録」の意。	
\\	第1志望の大学には落ちたけれど、滑り止めで入った大学が自分に向いていたのはけがの功名だ。 
\\	が「けがの功名」に当たる言い回しです。	
\\	(美容院で 
\\	毛先をそろえるぐらいでお願いします。 
\\	「毛先」は
\\	と言います。
\\	が「切りそろえる」、
\\	は「短くしないで」、
\\	は「でこぼこをなくす」の意。	
\\	ちょっと化粧直してくる。 
\\	「直す」を使って
\\	で「化粧を直す」の意味になります。
\\	「鼻に化粧をつける」。	
\\	あいつにはけじめというものがない 
\\	「けじめ」をここでは
\\	「善悪の区別をつける」で表現しています。	
\\	ちょっとした虫歯なのに、虫歯でない個所を50%近く削ってニセの歯をかぶせるなんて納得いかない。まだ、十分使える歯なのに。 
\\	この場合の「削る」は
\\	「すり減らす」ということですね。
\\	で
\\	を削られる」の意味を表しています。	
\\	人の案にけちを付けるなら、代案を出して見ろよ。 
\\	「けちをつける」は
\\	「あら探しをする」、あるいは
\\	「冷や水をかける」を使えばいいでしょう。
\\	は「見つける、示す」の意。	
\\	いろいろ考えた結果、
\\	語学学校に行く事を決めました。でも、実際に行ってみないといいか悪いかわからないけどね。 
\\	この場合の「結果」は接続詞の
\\	、あるいは、
\\	で表現できます。	
\\	毎日面接の練習したのに、当日面接官は私には何にも質問しないで終わってしまい、結果オーライで良かったけどちょっと残念。 
\\	(結果として…になる)と 
\\	を組み合わせて「結果オーライ」を表現しています。
\\	は「拍子抜け」、
\\	は「期待はずれ」の意味。	
\\	どのくらいで健康診断の結果は出ますか? 
\\	この場合の「出る」は
\\	「戻って来る」ということですね。	
\\	結果として、キャンセルすることになったけど、お互いの気持ちを知るいい機会だった。 
\\	「結果として...になる」は
\\	を使って表現できます。	
\\	いろいろあったけど、結果良ければすべて良しだよ。 
\\	は「結果がよかったんだから。それが重要なんだよ」の意味です。	
\\	そんなの結果論だよ。 
\\	とは「後になってとやかく言う」の意味です。
\\	は「後で人のことを批判する(動詞)」の意。(日曜日に見たフットボールの試合について、月曜日になってああだこうだ言う人がいることから、このような意味で使われるようになったのです)	
\\	あの作品にはけっこう自信があったのになあ。どうして採用されなかったのかな 
\\	「けっこう」は
\\	(かなり)で表現すればいいでしょう。
\\	は「…にとても自信がある」の意。	
\\	まだちゃんと計算してないけど、今月はすでに結構使ってるよ。そろそろ切りつめようよ。 
\\	で「かなりお金を使う」の意味になります。
\\	は「予算内で生活する」の意。	
\\	インターネットってやっていると時間を忘れるので、気をつけないといけない。やってみないとわからないと思うけど結構(意外に)はまるんだよ。 
\\	は「中毒性の」の意味。これを使って
\\	で「思ったよりはまる」と言っているのです。	
\\	私たちこれからどうなるの? 結婚するの? 
\\	「結婚する」は 
\\	ですね。「どうなる」を 
\\	で表現している点にも注目。	
\\	結婚詐欺にあった。 
\\	ここでは
\\	「だまして金を取る」と
\\	「…と結婚の約束をする」を組み合わせて「結婚詐欺」を表現しています。	
\\	最近では、結婚適齢期など話題にもならない。 
\\	で「何歳で結婚するか」の意味。	
\\	結婚当初に買ったカメラ、ここしばらく使ってなくて飾りになっているけど、ただで人にあげるにはちょっと考えちゃうな。 
\\	で「結婚初期に」の意味になります。
\\	は「埃をかぶる」の意。	
\\	血液型で人の性格を分析するなんてナンセンスだよ。 
\\	「血液型」は
\\	と言います。
\\	「分類する」。	
\\	私のうさぎは日の当たる所で毛繕いをするのが大好きだ。 
\\	「毛繕い」は
\\	を使って表現します。	
\\	年のせいか決断力が鈍くなってきた。 
\\	が「決断力のある」の意味を表す形容詞です。	
\\	そんな小さなことにこだわるなんて、意外とケツの穴の小さい奴だな。 
\\	は「融通がきかない、しみったれの」という意味。
\\	は「堅苦しい、鈍重な」の意。	
\\	階段を降りる途中で何かにつまずいた。 
\\	が「つまずく・けつまずく」に当たる表現。主語を人ではなく、もの、例えば、
\\	「道路上にあった何かにつまずいた」のような言い方もします。	
\\	ではこのへんで、決を採ることにしましょう。 
\\	で「その件について投票する」の意味になります。	
\\	あいつが、今度そんなことしたら俺だって、けつまくっちゃうよ。 
\\	「けつをまくる」には 
\\	(けんか腰になる)という言い方もありますが、ここでは 
\\	(絶対に許さない)を使って表現しています。
\\	は「行為」の意。	
\\	アメリカに来てみたけど、やっぱりカナダがいいと思った。 
\\	「けど」は
\\	で表現します。	
\\	彼はものすごく毛深いよ。 
\\	が「毛深い」に当たる形容詞です。	
\\	結婚して一緒に住んで初めて煙草がこんなに煙いという事に気がついたよ。 
\\	「煙い」は
\\	で表現できます。
\\	「初めてする」の使い方にも注意。	
\\	彼は人のことを煙に巻くのがうまい。 
\\	は「迷わすような」、
\\	は「(人を)巻く」の意味。	
\\	その工場火災で、多くの従業員が煙に巻かれて亡くなった。 
\\	で「煙に圧倒される、ぐったりさせられる」の意味。
\\	は「煙に巻かれて死ぬ」の意。	
\\	ダラダラやっててもしょうがないから、もうこのへんでけりをつけようじゃないか。 
\\	は「だらだらやる」、
\\	「時間を無駄に過ごす」の意。	
\\	その件についてはこちらで処理させていただきましたので、ご安心下さい。 
\\	「その件」は 
\\	で表現すればいいでしょう。	
\\	あの2人は犬猿の仲だから、宴会で近くに座らせないほうがいい。 
\\	で「対立している」の意味を表します。
\\	のように言ってもその意味は通じます。	
\\	この問題は見解の相違では片づけられない。 
\\	で「この問題に関して、我々の意見がどれだけ違っているか」の意味になります。	
\\	会議の場で、彼はけんか腰で話し始めた。 
\\	は「けんか腰の、好戦的な」という意味。	
\\	お前、相変わらず喧嘩っ早いな。 
\\	で「喧嘩をしたくてむずむずしている」の意味。	
\\	あの2人、あんなに仲が良かったのに、けんか別れしちゃったんだって。 
\\	が「けんか別れをする」に当たる言い回しです。
\\	は「怒って別れる」の意。	
\\	毎週水泳しているなんて健康的ですね。私も何か運動始めようかな。 
\\	「健康的」はそのまま
\\	を使えばいいでしょう。	
\\	今日は、眠すぎて頭が働かない。健康ドリンク飲んで頑張ろう。 
\\	ここでは「健康ドリンク」を
\\	「カフェイン入りのドリンク」と説明しています。
\\	「元気が出る強壮剤」を使ってもいいでしょう。	
\\	最近健康に気を使ってて、日曜日に健康に関するテレビをやってるのだけど、それでいいと聞くと次の日は早速買いに行きます。 
\\	で「健康により注意を払う」の意味になります。
\\	は「健康を意識する、気を使う」の意。	
\\	懸賞に応募したいからって、そんな無理して買わなくてもいいのに。 
\\	「懸賞」は
\\	でいいでしょう。
\\	で「…する意味がない」の意。	
\\	海外で日本人女性が大使館に強盗等で助けを求めた件数は過去最高に上ったらしい。やっぱり海外って日本より危険なんだね。 
\\	で「件数が過去最高」の意味。
\\	でも同じようなことが言えます。	
\\	今日の会議は、いろいろと建設的な意見が出てよかったよ。 
\\	が「建設的な意見」に当たる表現です。
\\	は「見地」の意。	
\\	彼は褒めても素直に受け入れず、謙遜してばかりいる 
\\	で「謙遜する」の意味になります。	
\\	ここ初めてだから、どこに何があるのか全然見当がつかない 
\\	とは「方向が分からない、混乱する」の意味です。
\\	は「どこに何があるのか」に当たる表現です。	
\\	その問題に関しましては、検討の余地があると思います。 
\\	この場合の 
\\	は「余地」の意味。
\\	は他動詞なので前置詞 
\\	は必要ありません。	
\\	彼は、健闘むなしく入賞できなかった。 
\\	(努力したのにもかかわらず)で「健闘むなしく」を表現しています。	
\\	みなさんの健闘を心から祈っています。 
\\	「健闘を祈る」は
\\	と
\\	を使えばいいでしょう。	
\\	若いうちに見聞を広げるために、海外旅行をした方がいい。 
\\	ここでは「見聞」を
\\	「知識」と
\\	「経験」で、「広げる」を
\\	で表現しています。	
\\	部長が第一課の課長を兼務しています。 
\\	動詞
\\	には「二役を務める」という意味があります。	
\\	彼は、職場で課長と部長を兼務している。 
\\	で「…として2役を兼務する」の意味になります。
\\	は「…として務める」の意。	
\\	睡魔の撃退法を教えてよ。 
\\	には「撃退する、退治する」という意味があります。	
\\	ビールちょっと飲んだだけであんなに酔うなんて、彼女相当下戸だね。 
\\	「相当な下戸」を 
\\	で表現しています。	
\\	下剋上の世の中だぎゃあ、枕を高くして寝られんがなも。 
\\	「下克上」を
\\	「下級の武士」が
\\	「絶えず階級の上の武将を打倒する」で説明しています。
\\	「階級」、
\\	「帰する、移る」、
\\	「混乱の」の意。	
\\	(レントゲン撮影で)これを一気に飲み込んでください。ゲップが出そうになっても、がまんしてください。 
\\	「ゲップ(をする)」は
\\	と言います。文全体の流れも見ておくといいでしょう。面白い言い回しが使われていますので。	
\\	歩きながらげっぷするな!! しかも、すまし顔で。 
\\	「げっぷをする」は
\\	と言います。
\\	「何もなかったような顔をする」、
\\	「偉そうな顔をするな(とぼけるなよ)」。	
\\	あいつは下手物食いだから、どんな料理でも食べるよ。 
\\	で「今までに食べたことのない不気味な食べ物を試してみる」の意味になります。	
\\	もう限界です。これ以上飲めません。 
\\	で「限界に達する」の意味。これは、例えば、
\\	「これば私の限界です」のようにも言い換えられます。この場面では
\\	と言ってもいいでしょう。	
\\	今回の彼の態度で(私は)かなり(我慢の)限界きている。 
\\	ここでは「限界」を
\\	で表現しています。
\\	の言い回しにも注目。	
\\	英語を一人で勉強するのって限界があるから、やっぱりネイティブと話す機会をもって、生の英語に触れるのが一番だと思う。 
\\	で
\\	には限界がある」の意味になります。
\\	に触れる」。	
\\	自分自身の限界を感じましたので、引退します。 
\\	「限界に到達する」を
\\	で表現している点に注意してください。	
\\	さすが新人!元気が有り余っているな。その元気分けてほしいよ。 
\\	この場合の「元気」は、
\\	「やる気」を使って表現すればいいでしょう。	
\\	元気だね、そんな格好で寒くないの?なんだか、見ているこっちが鳥肌立っちゃうよ。 
\\	この場合の「元気」は
\\	(頑丈な、強健な)、
\\	(頑丈な)で表現すればいいでしょう。
\\	は「鳥肌が立つ」の意。	
\\	彼ってここんとこ欠勤しているみたいだけど、それって減給になるんじゃないの? 
\\	で「減給する」の意味。
\\	「減給」。	
\\	彼女はひったくりの現行犯で逮捕された。 
\\	で「現場を押さえられる」の意味になります。
\\	「ひったくる」。	
\\	英語は毎日少しでもやらないと上達しないっていうのはわかっているのだけど、現実にはなかなかね。 
\\	「現実には」は
\\	を使えばいいでしょう。
\\	「現実は 
\\	だ」。	
\\	アメリカへ渡って3年は仕事が見つからなくて、貯金は全部食い尽くしたっていう話を聞くと心配になる。でも、それが現実かもね。 
\\	で「それが現状だ」の意味になります。
\\	をそのまま使ってもいいでしょう。
\\	は「混乱させる、悩ませる」の意。	
\\	お前のやっていることは現実からの逃避に過ぎない。 
\\	が「現実から逃避する」に当たる言い回しです。また、
\\	には名詞として「現実逃避」の意味があります。	
\\	理想ばかり追わないで、もっと現実的にならなくちゃ。 
\\	「現実的」をここでは
\\	「実践的な」を使って表現しています。	
\\	様々な教材、英会話学校など何らかの英語の勉強をしていますが、時間とお金が英語の実力となって返ってこないのが現状です。 
\\	は「厳しい現実」という意味です。
\\	「(現時点で)一つ学んだことは」。	
\\	"この商品のキャッチフレーズは ""あなたを幻想の世界へ導きます
\\	でどうかな? 
\\	ここでは「幻想の世界」を 
\\	で表現しています。
\\	今度出る
\\	の
\\	って、30万 枚の限定版らしいよ。やっぱり、
\\	と同じ発売日だから、いろいろ考えているのかね。でも、
\\	には勝ってこないよ。 
\\	「限定版」は
\\	ですね。
\\	「争って」、
\\	「より多く売り上げる」。	
\\	そのとき生きるか死ぬかの体験をしたのが、私の原点となっている。 
\\	ここでは「原点」を
\\	で表現しています。	
\\	我慢にも限度があるよ。 
\\	で
\\	に限度がある」の意味になります。
\\	「無限の」。	
\\	どのくらいのお酒の量で酔うのか、自分では自分の限度がいまだにわかりません。強いほうだとは思いますが。 
\\	「自分の限度」は 
\\	で表現すればいいでしょう。
\\	は「お酒が強い」の意。	
\\	きみのクレジットカードの限度額っていくら? 
\\	「限度額」は 
\\	で表現すればいいでしょう。	
\\	この怒りは、実際に現場にいないとわからないよ。口では説明できない。 
\\	この場合の「現場」は 
\\	(そこ、その場所)で表現すればいいでしょう。
\\	は「不可能にする、受け付けない」の意。	
\\	では、マネージャーからのご説明はこの辺で終わらせていただいて、現場の人間から現状を報告します。 
\\	で「現場の人間」を表現しています。
\\	「職員として 
\\	に勤める(動詞)」。	
\\	申し訳ございませんが、販売できるのは現品限りで、あとは取り寄せになります。 
\\	「現品限り」は、
\\	「あるのは棚の上にある商品だけです」と言い換えて表現します。「お取り寄せ」は、
\\	「注文する」、
\\	「頼む」を使えばいいでしょう。	
\\	景気がよくないので、今月の給料は現物支給だそうだ。 
\\	とは「商品で支払う」の意味。	
\\	彼女には、小悪魔的な妖しい魅力がある。 
\\	「小悪魔的な怪しい魅力」を
\\	を使って表現しています。	
\\	彼女、小粋な服装してるよね。 
\\	「小粋」を、ここでは、
\\	「少し押さえ気味の魅力」、
\\	「控えめの粋さ」として表現しています。	
\\	外国旅行をすると、日本食が恋しくなるよね。 
\\	「恋しい」はそのまま
\\	を使えばいいでしょう。
\\	の使い方に注目。	
\\	人の恋路をじゃまするやつは馬に蹴られて死んじまえ。 
\\	「恋路」は 
\\	でいいでしょう。「馬に蹴られて…」の表現の仕方にも注目。	
\\	恋の駆け引きは無駄だよ。 
\\	(駆け引きをする)と 
\\	(巧みに操縦する)を組み合わせて表現しています。
\\	は「巧みに扱う」の意。	
\\	5月5日の1週間前くらいから鯉のぼりを飾る家をよく見かけるんだよ。 
\\	ここでは「鯉のぼり」を 
\\	で表現しています。
\\	は「吹き流し」の意。	
\\	あいつの高圧的な態度が気に入らない。 
\\	「高圧的な」は
\\	(高飛車な)、
\\	(傲慢な)で表現できます。	
\\	こういう格好している人って最近多いよね。(テレビを見ていて) 
\\	「こういう、ああいう」は
\\	で表現できます。	
\\	メディアは、その事件を好意的に書いた。 
\\	で「…を好意的に扱う」の意味。	
\\	私の座右の銘は「行雲流水」です。 
\\	は「流れにしたがって行く」、
\\	は「物事にあわせて行動する」の意味です。「座右の銘」は
\\	(モットー、標語)で表現すればいいでしょう。	
\\	僕が最優秀賞に選ばれるなんて、とても光栄に思います。 
\\	「光栄に思う」は
\\	を名詞、あるいは、動詞として用いて表現できます。	
\\	このナイトクリームはどんな効果があるんですか? 
\\	(~は何々をする)で「効果がある」を表現しています。「効果」は 
\\	を使ってもいいでしょう。	
\\	日本ではだいたい数カ月遅れて映画が上映されますが、今回のこの映画は数日後には公開されるんです。 
\\	「公開される」は 
\\	で表現します。
\\	(封切り)は名詞としても使えます。	
\\	済んでしまったことを、今さら後悔しても始まらないよ。 
\\	で「すでに起きてしまったことをくよくよ考える」の意味。
\\	は「後悔」の意。	
\\	もう何か言ってやってよ。この子全然家の手伝いしないんだから。他人から言ってくれた方が効果あるかもしれないし。 
\\	「効果」は
\\	ですから、これと
\\	を組み合わせればいいでしょう。
\\	「少しもしない」。	
\\	鈴木部長、降格して課長になるらしい。 
\\	「降格する」は
\\	「昇格する」の反意語ですね。
\\	は「落とされる(降格される)」の意味になります。	
\\	どっちのクラスで勉強した方が効果的だと思いますか? 
\\	「効果的」はそのまま
\\	を使えばいいでしょう。	
\\	今日のクライアントはなかなか好感触だったよ。 
\\	「好感触」はそのまま英語にして
\\	でもいいですし、また、
\\	「感触、感じ
\\	を使っても表現できます。	
\\	彼女は好感度ナンバーワン女優だから、CMの引き合いも多いんだ。 
\\	ここでは「好感度ナンバーワン」を 
\\	を使って表現しています。
\\	でもいいでしょう。	
\\	最近いろいろな抗菌グッズが出まわっている。 
\\	「抗菌の」は
\\	と言います。	
\\	どんなに売上に貢献しても給料は変わらないんだもんな。これじゃやる気なくなるよな。 
\\	「貢献する」は
\\	を使えばいいでしょう。
\\	「上げる、増加する」。	
\\	どんな料理が出てくるか、乞うご期待! 
\\	が「乞うご期待」に近い言い回しですね。	
\\	それは公私混同もはなはだしい。 
\\	ここでは「公私混同」を
\\	「会社の仕事と個人の事柄をはなはだしく混同している」と言い換えています。
\\	「目に余るほどに」。	
\\	チャンスの到来を虎視眈々と狙う。 
\\	英語でも
\\	を使って表現できます。
\\	は「チャンスを待つ」、
\\	は「飛びかかる」の意	
\\	これまで奴にはディベートでやられっ放しだったが、攻守ところを入れ替えて反撃した。 
\\	「攻守ところを入れ替える」とは 
\\	(相手を守勢に立たせる)ということですね。	
\\	お前、口臭がするぞ。歯ちゃんと磨いてるのか。 
\\	「口臭がする」は
\\	と言います。	
\\	彼は高所恐怖症なので、高い所には上がれないんだ。 
\\	で「高い所に対する恐怖感を持っている」の意味。
\\	「高所恐怖症」。	
\\	友だちが結婚するから出席しないといけないって言えば、会社を休むいい口実になる。さすがに上司もだめとは言えないでしょう。 
\\	「口実」は
\\	(言いわけ)で表現できます。
\\	は「仕事を一日休む」の意。	
\\	リスニングを向上させる秘訣ってなんだろう 
\\	で
\\	がよりうまくなる」の意味になります。
\\	「向上させる」。	
\\	君には向上心というものがないのかね。 
\\	「向上心」は
\\	「野心、待望、熱望」で表現できます。	
\\	公序良俗に反するホームページは、事前通告無く削除されます。 
\\	が「公序良俗」に当たる表現です。
\\	は「品位」の意。	
\\	受け身の人生を送っていた彼も、物心ついてから攻勢に転じた。 
\\	で「攻勢に出る」の意味になります。
\\	で「ハッキリとわかる」の意。	
\\	先生は交代で休暇を取る。 
\\	には「交代する」の意味があります。
\\	「お互いに代わりをつとめる」。	
\\	ここは交通の便が悪いからね。最寄りの駅からバスで1時間もかかるんだから。 
\\	「交通の便が悪い」は
\\	と
\\	で表現できます。
\\	の使い方にも注意。	
\\	もし東京で車を制限するような制度を導入したら、交通量が減ると思いますか? 
\\	が「交通量」に当たる言い回しです。
\\	「…を設置する」。	
\\	メグ・ライアン主演の映画は荒唐無稽だったが、大変面白かった。 
\\	「荒唐無稽」を 
\\	(とても信じられない)、
\\	(少し不自然な)を使って表現しています。	
\\	私たちの行動はおかしいのでしょうか。 
\\	「行動」は
\\	(動詞)、
\\	を使って表現すればいいでしょう。
\\	「奇妙な」。	
\\	もっと行動力があればなあ・・。 
\\	ここでは「行動力」を
\\	「物事を起こす力」、
\\	「エネルギー」で表現しています。	
\\	恋愛に関しては特に臆病になってしまい、自分から行動を起こすことなんてできない。 
\\	の文字どおりの意味は「物事を進める」です。
\\	は「イニシアチブを取る、自ら乗り出す」の意。	
\\	口内炎ができて食事する時、痛くて痛くて。 
\\	「潰瘍」と
\\	を組み合わせて「口内炎」を表現します。	
\\	昔は今の仕事はやりがいのある仕事だと思ったのに、今ではもう早く転職したいって感じ。なんでこうなったんだろう。 
\\	で「どうしてこのように感じるようになったのかな」の意味になります。	
\\	彼は、会社では後輩だけど、年は上で、彼が入社した時、私が教育担当でした。 
\\	は「部下」の意味になります。
\\	「後輩の立場にいる」。	
\\	今年の夏は、どこに行こうかと迷っている。一応、候補はあるんだけどね。 
\\	「候補」はそのまま
\\	使えます。また、
\\	でも同じような意味になります。	
\\	こう見えても意外に人前で話すの苦手で、手なんか震えるし、口ごもちゃって、自分でも何言っているかわからない時がある。 
\\	「こう見えても」をここでは
\\	「自信があるように見えるかも知れない」と言い換えています。また、
\\	は「僕のことを見ても分からないかも知れないけど」の意。	
\\	選挙が終われば、どの立候補者も公約のことなんて忘れちゃうんだろうな。 
\\	が「公約」に当たる表現です。
\\	したとたん」。	
\\	ただ、タイピングするのではなくて、文章の内容を理解してからやるほうが、効率がいいよ。 
\\	は「編集する、校正する」の意味。「内容を編集する」とは、つまり、「内容をしっかり理解する」ということですね。
\\	「一語一語」。	
\\	ベストセラーになっているのは、ゲームの攻略本やタレントの写真集ばかり。 
\\	とは「攻略するためのマニュアル」のことです。	
\\	セミの声って暑さを倍増させるね。 
\\	この場合の「声」は
\\	ではなく
\\	を使います。	
\\	今度、だんなの上司を食事に招待するんだけど、相当舌が肥えてるらしいから、メニューを考えるのが大変だわ。 
\\	「舌が肥えている人」のことを
\\	と言います。
\\	は「えり好みする、うるさい、気むずかしい」の意味。	
\\	2万円までは非課税ですが2万円を超える部分に関しては課税されます。 
\\	が「超える部分」に当たる言い回しです。
\\	は「超過」の意。	
\\	街で知らない人に声をかけられた。 
\\	「声をかける」は
\\	、あるいは、
\\	「会話を始める」で表現すればいいでしょう。	
\\	今朝、池に氷が張っていた。 
\\	(氷が張る)は
\\	(凍る)、あるいは、
\\	(形をなす)を使って表現します。	
\\	時には木陰でボーッとするのもいいもんだ。 
\\	は「ぼんやりする」、
\\	は「うっとりする、幸福感に浸る」という意味です。	
\\	この調子で小金をため込めば、そのうち銀座で豪遊できるぞ。 
\\	で「小金を貯める」の意味になります。
\\	で「どんちゃん騒ぎをする」、
\\	で「貯め込む」の意。	
\\	彼は唯一の男性社員らしいので、いろいろこき使われないか心配です。 
\\	「こき使う」は 
\\	(過度に働かせる)、
\\	(酷使する)で表現すればいいでしょう。	
\\	彼は毎日こき使われて、一度ぎゃふんと言わせてやろうと、その機会を今か今かと待っていた。 
\\	には「(偉そうにして)指揮する、仕切る」という意味があります。
\\	で「こき使われる」のニュアンスが出せるでしょう。	
\\	あれが今世の中で話題になっているコギャル?テレビで見たのと一緒だ。 
\\	ここでは「コギャル」を 
\\	といったあとで、
\\	で説明しています。
\\	は「活発に動き回る」、
\\	は「派手な」の意。	
\\	このソース、こくがないのよね。何が足りないのかしら。 
\\	には「濃度、密度」という意味があります。
\\	「こくがない」。	
\\	彼は、日本とアメリカの国籍を持っている。 
\\	「国籍」は
\\	ですね。
\\	は「二重の」の意。	
\\	そんなこけおどしは僕には通用しないよ。 
\\	この場合の
\\	は「はったりをかます」の意味になります。
\\	「簡単にははったりは通じない」。	
\\	トーストが焦げてるじゃない。これ食べろっていうのかい? 
\\	で「焦げてカリカリになる」の意味。	
\\	ここ最近この辺りの天気はこんな感じなのですか。日本と比べると過ごしやすいですね。 
\\	「ここ最近」は
\\	を使えばいいでしょう。
\\	は「住みやすい」の意。	
\\	ここだけの話だけど、田中が経理の高橋さんと結婚するらしいよ。 
\\	で「…をこの部屋から出さない」の意味になります。
\\	は「経理課」の意。	
\\	君の作文はここのところが大変いい。 
\\	「ここのところは」は
\\	、あるいは、
\\	でいいでしょう。	
\\	ここのところずっとラザニアとかパスタとかばっかり食べていたから、こう何かあっさりしたものが食べたいな。 
\\	「ここのところ」は 
\\	で表現できますね。
\\	は「あっさりした」の意。	
\\	なんだかここのところ、夜10時を過ぎるとすごく眠たくなるのです。 
\\	「ここのところ」は 
\\	(最近)で表現すればいいでしょう。	
\\	明日は晴れてくれないと困るな。だって、ここのところ週末雨ばっかりで洗濯物が全然乾かないんだよ。 
\\	「ここのところ」とは
\\	「最近は」ということですね。	
\\	あなたに心当たりがないのです。 
\\	「心当たりがない」とは
\\	(見覚えがない)ということですね。	
\\	君を心から愛してます。 
\\	「心から」はそのまま
\\	と言えます。	
\\	毎日30分でもいいから勉強するように心がけている。でも、現実にはできてない。 
\\	「心がける」をここでは
\\	「自分に... すると言い続けている」、
\\	「心に決める」で表現しています。
\\	の使い方にも注目してください。	
\\	健康維持のため、砂糖はとりすぎないよう心がけています。 
\\	「心掛ける」は
\\	(注意する)で表現すればいいでしょう。	
\\	私のことはいいから行ってきて。せっかくのパーティーに私のせいで行けないんて心苦しくてしょうがない。 
\\	で「…して気持ちが落ち着かない」の意味になります。	
\\	どこ見てんのー。心ここにあらずって感じね。 
\\	「心ここにあらず」をここでは
\\	「注意を払っていない」と言い換えています。	
\\	若いんだから、志しは高く持たなくてはね。 
\\	は「大望、野望」の意味です。
\\	はその形容詞形。	
\\	それは、私にとってたいへん心強い言葉です。 
\\	「心強い」は 
\\	(励みになる)を使って表現すればいいでしょう。	
\\	最近、あるテレビがきっかけになって栄養補助食品に凝ってるのですが、心なしか体の調子がいい気がします。 
\\	「心なしか」のニュアンスを 
\\	(どういうわけか)、
\\	を使って表現しています。	
\\	心にもないおせじやよいしょはむなしいです。 
\\	(空虚な)を使って「心にもない」のニュアンスを表現しています。	
\\	もっと心にゆとりを持たないとダメだねぇ。 
\\	(…を快く受け入れる)で「心にゆとり」を表現しています。
\\	は「…に理解を示す」の意。	
\\	あいつには人の心の機微など分からないよ。 
\\	は「微妙な点」という意味です。
\\	「洞察力」。	
\\	あの頃のことをいつも心のどこかで考えている。 
\\	(心のなかに特別な場所がある)で「心のどこかで」を表現しています。	
\\	心のバリアフリーを大切に! 
\\	「心のバリアフリー」とは
\\	「偏見のない、公平な(心を持つ)」ということですね。	
\\	すべては心のもち方次第だ。 
\\	「心の持ち方」を「物の見方」と言い換えて表現しています。
\\	次第」、
\\	「結局... ということになる」。	
\\	私の英語力でプレゼンをするにはかなり心もとないです。 
\\	「心もとない」とは
\\	(自信がない)ということですね。
\\	は「ぐらつく、心もとない」の意。	
\\	心を癒すピアノ音楽 
\\	は「鎮静させる」、
\\	は「癒す、治す」の意味です。(注)次回は、文(主語+動詞)の形で投稿してください。	
\\	これからは心を入れ替えて働きますので、許してください。 
\\	で「改心する」という意味になります。
\\	「きちんとする」。	
\\	心を入れ替えて、今日から英語で日記をつけ真剣に勉強を始めました。 
\\	(心機一転する)を使って「心を入れ替える」を表現しています。	
\\	彼は相変わらず腰が重くて、なかなかこちらが頼んだ仕事をしてくれない。 
\\	が「腰が重い」に当たる言い回しです。
\\	の代わりに 
\\	(仕事に取りかかる)を使ってもいいでしょう。	
\\	後ろから急に驚かすから、腰が抜けそうになったよ。 
\\	「動けなくなる」で「腰を抜かす」を表現しています。
\\	「その場で動けなくなる」。	
\\	彼は相変わらず腰が低いね。 
\\	「腰が低い」とは 
\\	(謙虚で礼儀正しい)ということですね。
\\	は「謙虚な」の意。	
\\	あいつ昨日の勢いにくらべると、今日はずいぶん腰が引けてるな。 
\\	には「力、元気」の意味があります。
\\	「奮い起こす」、
\\	「積極性」。	
\\	外務省の改革は、腰砕けに終わった。 
\\	は「失敗、挫折」、
\\	は「失敗に終わる」の意。	
\\	何でこのメーカーに固執する必要があるわけ? あなたがこだわらなければ選択肢が広がるんだよ。 
\\	ここでは「固執する」を 
\\	(頑固なほど義理堅い)で表現しています。
\\	は「固く決心をしている」の意。	
\\	こしゃくなやろうだ、名を名乗れ。 
\\	(生意気な若造)で「こしゃくな」のニュアンスを表現しています。	
\\	私のマックが買ったばかりなのに故障して修理に出してたの。 
\\	「故障する」は
\\	を使えばいいでしょう。	
\\	突然、暗闇から人が出てきたので、彼は腰を抜かすほど驚いた。 
\\	ここでは「腰を抜かす」を 
\\	(立ちすくむ)、
\\	(麻痺させる)を使って表現しています。	
\\	彼は風邪をこじらせ肺炎になり、今、入院している。 
\\	この場合の
\\	は「ひどくなって…になる」の意味。
\\	(肺炎)に冠詞は必要ありません。	
\\	カゼをこじらせて、1週間も寝込んでしまった。 
\\	「こじらせる」は
\\	「とても悪くなった」で表現できます。
\\	は
\\	をうち負かす」の意味。	
\\	共同生活って大変だと思うよ。特に同じ部屋に住むとなると四六時中気を遣わなければならないし、ちょっとしたことで仲がこじれて気まずくなったりしそうだし。 
\\	「物事を複雑にする」で「こじれる」のニュアンスを表現しています。
\\	「共同生活」、
\\	に配慮する」、
\\	「ゴチャゴチャになる」。	
\\	チュニジアは個人技に長けたチームらしい。 
\\	「個人技」は 
\\	、あるいは、
\\	で表現すればいいでしょう。
\\	は「…に熟達している」の意。	
\\	最初は、個人旅行をしようと思っていろいろ考えていたんだけど、考えている内にめんどくさくなってきて、結局 ツアーを申し込んだ。 
\\	は「一人で旅行する」、
\\	は「自分で旅行する」という意味。	
\\	ちょっとよそ見をしてたら、電信柱に車こすっちゃってさ。 
\\	「こする」は
\\	を使って表現します。この機会に、
\\	「よそ見する」、
\\	「電信柱」も一緒に覚えておきましょう。	
\\	最近の若者は、皆同じ格好をしているから、個性がないよね。 
\\	「個性がない」をここでは
\\	「区別がつかない」で表現しています。	
\\	いいんじゃない。そういう個性的な人がいても。だれも否定できないと思うけど。 
\\	ここでは「個性的」を 
\\	を使って表現しています。
\\	は「ユニークな、唯一の」の意。(注)最後の和文は英訳していません。	
\\	愛があるからこそ、私たちは生きていけるのだと思う。 
\\	「こそ」は
\\	なので」ということですね。	
\\	こそこそ隠れて何やってるの? 
\\	「こそこそ隠れる」は
\\	(潜む)で表現すればいいでしょう。
\\	は「こそこそする」の意。	
\\	ひどい砂利道だなあ。二日酔いにはこたえるよ。 
\\	で「すごくガタガタ揺れて感じる」で「こたえる」のニュアンスをだしています。
\\	「不快感を与える」。	
\\	誇大広告にだまされて、使い物にならない商品を買ってしまった。 
\\	で「おとり販売詐欺にだまされる」の意味。
\\	は「人を惑わす広告」の意。	
\\	結婚してからは、クリスマスとかあまりこだわらなくなった。なんか寂しいね。 
\\	ここでは「こだわらない」を、
\\	をあまり重要視しない」、
\\	にあまり注意を払わない」で言い換えています。	
\\	彼は料理にはこだわりますね。 
\\	で
\\	に関して好みがうるさい」の意味になります。
\\	「うるさい、気にする」。	
\\	今回は、前置詞とかにあまりこだわらなくていいんじゃない?そんなこと言ってたら全然先に進まないし、取りあえず文章を作っちゃおうよ。 
\\	(…をくよくよ心配する)で「こだわる」を表現しています。	
\\	彼ももういい大人なんだから、そんなことにこだわらないでほしい。もっと、広い心を持たないと。 
\\	ここでは「こだわる」を 
\\	を使って表現しています。	
\\	こちらこそ申し訳ありません。 
\\	「こちらこそ」は 
\\	で表現すればいいでしょう。	
\\	4月実施なのに、全然詳細は決まってない。早く詳細を決めてもらわないとこちらは全然動けない。こちらの身にもなってほしいよ。 
\\	「こちらの身になる」は
\\	「我々の靴を履いてみてほしい」という言い方をよくします。
\\	を我々の視点から見る」。	
\\	だんだんコツがつかめてきたよ。こうだろ?どう?いい?よっしゃ! 
\\	「コツがつかむ」は、
\\	で表現できます。また、
\\	「のみこむ」を使ってもいいですね。	
\\	毎日コツコツ英語を勉強すれば、知らない間に上達していると思うよ。自分では気づかないかもしれないけど。 
\\	が「毎日こつこつ勉強する」に当たる言い回しです。
\\	は「根気強く、継続して」の意。	
\\	聞いた話だけど、彼らは小遣い稼ぎに、飲酒運転とスピード違反を捕まえているらしい。 
\\	で「小遣いを稼ぐ方法として」の意味。
\\	でも同じような意味になります。	
\\	この代行手数料というのは固定金額?それとも金額によって違うのですか? 
\\	「固定」をここでは 
\\	を使って表現しています。
\\	は「…によって」の意。	
\\	小手先だけで作ってちゃいいものはできないよ。 
\\	とは「器用さ以上のものが必要になる」の意味。	
\\	毎年言っていることなんだけど、今年こそはスノボーやってみょうと思う。 
\\	「今年こそ」は
\\	「今年が... する年だ」で表現できます。	
\\	旧年中はお世話になりました。今年も宜しくお願い致します。 
\\	に恩を受けている」と
\\	を楽しみにしている」の使い方がポイントです。	
\\	役所のことなかれ主義には呆れてしまう。 
\\	ここでは「ことなかれ主義」を
\\	「危険や変革につながるようなことは何もしない」で表現しています。
\\	にうんざり」、
\\	「役人」、
\\	「避ける」。	
\\	ことになったらで、実際に行くことになったら、いつ行くの?(実行するかわからない計画があって「実際にすることになったら」と言いたいとき) 
\\	で「…することになる」の意味。強調するために
\\	を使っています。	
\\	みんなからのおめでとうの言葉ってうれしいよね? 
\\	「おめでとうの言葉」は
\\	でいいでしょう。	
\\	あの人は、言葉が通じないから何話しても無駄だわ。 
\\	「言葉が通じない」をここでは
\\	「こちらの言うことを一言も聞いてくれない」と言い換えて表現しています。	
\\	言葉足らずで、誤解されてしまったようです。申し訳ありませんでした。 
\\	「言葉足らず」とは「はっきりと説明しなかった」ということですから、
\\	を使えばいいでしょう。	
\\	最近の若者は、言葉遣いがなっとらん。 
\\	で「ちゃんとした話し方を知らない」の意味になります。
\\	は「きちんと自分を表現する」の意。	
\\	彼女はそれまで冷静に話していたが、その当時の話になった途端、感極まって言葉につまった。 
\\	「言葉につまる」は
\\	「胸が一杯になって言葉が出ない」を使って表現します。「感極まる」の表現の仕方にも注目。	
\\	私が心配しているのは、言葉の壁だけです。 
\\	「言葉の壁」は
\\	と言います。	
\\	ハッキりと断るべきです。 
\\	は「断わる」、
\\	は「明確な言葉で」の意味。	
\\	いつまでそんな子供じみたことを言っているんだ。 
\\	共に「子どものような」という意味。	
\\	彼の子供っぽい笑い方が素敵よね。 
\\	「子供っぽい」は
\\	「子供のような」と言います。似たような単語に
\\	がありますが、これは「子供じみている、大人げない」という意味になりますのでここでは使えません。	
\\	頑張ったねえ。こなれた動きのコサックダンスだ。 
\\	「こなれた」とは
\\	「熟達した」、
\\	「十分に洗練された」で表現すればいいでしょう。	
\\	あいつ部長のコネでここに就職できたんだぜ。 
\\	「コネ」はそのまま
\\	が使えます。また、
\\	「コネ、てずる」という言い方もあります。	
\\	そんなに前の方に乗ると後で乗り換えるのが大変だから、この辺りでいいよ。(電車に乗るとき)。 
\\	「この辺り」は
\\	でいいでしょう。
\\	「困難なこと)	
\\	この後の予定は何ですか。 
\\	「この後」は
\\	「今度の、次に来る」で表現できます。	
\\	散らかっている部屋だけ片づけようと思ったんだけど、この際だから全部屋片づけようかと思って。 
\\	「この際だから」を
\\	「どうせやるんだから」で言い換えて表現しています。
\\	「散らかった」。	
\\	いつも早く来ているけど、さすがの彼もこの時間には駅にいないね。 
\\	「この時間」を 
\\	(こんなに早い時間に)で表現しています。	
\\	今のこの状況からして、どう考えてもこれから忙しくなるとは思えないんだけど。 
\\	の文字通りの意味は「このような状況が与えらているので」。	
\\	なんかこのところ風邪気味で頭がぼーっとしていて、勉強に身が入らない。 
\\	「このところ」は 
\\	(ここ2、3日)でいいでしょう。「風邪気味」の表現の仕方にも注目。	
\\	この場合、英語では、何と言ったほうがいいのですか? 
\\	「この場合」は
\\	でいいでしょう。	
\\	今日のところはこのへんにしといてやろう。 
\\	「このへんにしてやる」は
\\	「十分 
\\	した」でいいでしょう。	
\\	人生一度しかないのに、このままでいいのかって時々思うんだよね。 
\\	が「このまま」の意味を表しています。「人生一度」の表現の仕方にも注目。	
\\	アメリカに来て4カ月、全然ヒアリングの上達が見られません。1年過ぎてもこのままだったらどうしようと不安です。 
\\	「このままだったら」を 
\\	で表現しています。	
\\	なんか嫌な予感がする。結構相手に押されている。このままこの試合負けそう。(テレビを見ながら) 
\\	ここでは「このまま」を 
\\	で表現しています。	
\\	このまま行くと何時の電車に間に合いそうかな? 
\\	で「この調子(速度)で行く」という意味になります。	
\\	このまま続けていても結論が出そうにないので、ちょとやり方を変えませんか?取りあえず、1人ずつ案を出して行きましょう。 
\\	で「このままで先に進む」の意味になります。
\\	でも同じような意味です。	
\\	このままで済むと思ったら大間違いだからな、覚えておけよ。 
\\	で「とても簡単に終わる」の意味になります。
\\	は「思い込む」の意。	
\\	たまにこのままではいけないって自分で感じるんだよね。もっと、将来の事真剣に考えないとって。 
\\	「このまま」を
\\	(同じように暮らす)、
\\	(このように)で表現しています。	
\\	この悪くなった植物、抜かないでこのままにしとく(このまま放置しておく)と他の植物に影響しないかな。 
\\	で「今ある状態のままにしておく」の意味になります。
\\	「悪影響を及ぼす」。	
\\	私たち、テニス以外の好みはバラバラです。あ、でも食べ物の好みは少し似ているかも。一緒に住んでいると食べ物の好みは似てくると思う。 
\\	ここでは「好み」を対象に合わせて、
\\	(興味)と
\\	(好み)を使って表現しています。	
\\	日本人の母親は、子離れ出来ていないとよく外国人に言われる。 
\\	に対する愛情から脱皮できない」を使って表現しているところがポイントですね。
\\	を放す」。	
\\	上司とかに媚びたりってどうしてもできない性分なのよね。 
\\	で「媚びる、ペコペコする」の意味。	
\\	あいつなんか、部長の子分みたいなもんだよ。 
\\	とは「おべっか使い」のことです。
\\	は「ごますり」の意。	
\\	キーボードにコーヒーをこぼしたら、いくつかのキーが使えなくなってしまった。 
\\	「こぼす」は
\\	でいいでしょう。
\\	は「はねかける」の意。「使えない」を
\\	で表現している点にも注目。	
\\	うちの地区のごみ捨ての仕分けが細かくなって、とても面倒くさい。 
\\	には「細かい」という意味があります。
\\	(複雑な)でも「細かい」というニュアンスは出せますね。	
\\	歳を取ると、細かいことはあまり気にしなくなるもんだよ。 
\\	「細かいこと」は
\\	でいいでしょう。
\\	は「悩ませる」、
\\	は「気に病む」の意。	
\\	会合には途中で途切れることなく、こまめに顔を出すことが当選の秘訣だ。 
\\	「こまめに」のニュアンスを 
\\	で表現しています。
\\	は「まめな」の意。	
\\	そんなわがままばかり言って困らせないで。 
\\	「困らせる」を 
\\	(悩ませる)で表現しています。	
\\	無理難題を突きつけられ、ほとほと困っている。 
\\	で
\\	(が殺到して、あふれて)身動きがとれない(困る)」の意味。
\\	「ひどい、とんでもない」。	
\\	ちょいと小耳にはさんだんだけどさあ、森高が結婚するってほんと? 
\\	「小耳にはさむ」は
\\	「偶然耳にする」でいいでしょう。	
\\	一日中部屋に籠もって、一体何をしているの。 
\\	ここでは「籠もる」を
\\	「閉じこめる」、
\\	「隠れる」で表現しています。
\\	も使えますね。	
\\	木漏れ日の中、昼寝もいいか! 
\\	で「日差しが木々の葉っぱの間を通り抜けている状態で」の意になる。
\\	「漏れる」、
\\	「体を伸ばす」。	
\\	こらっ!しっかり働け! 
\\	「こらっ」に当たる英語は、そうですね、
\\	や
\\	でしょうか。
\\	は「だらけた姿勢でいる」の意。	
\\	ミル貝はコリコリしている。 
\\	「コリコリ」は
\\	(パリパリ、サクサク)で表現すればいいでしょう。
\\	は「コリコリした噛みごたえ」の意。	
\\	あいつ今インターネットにはまってるらしいよ。凝り性だからな。 
\\	「凝り性」とは、
\\	「何事も中途半端にはやらない」、あるいは、
\\	「徹底的にやる」で表現すればいいでしょう。ここでは、
\\	「はまっている」もいっしょに覚えておきましょう。	
\\	肩が凝りすぎたせいで、集中力が無くなってすっごくいらいらする。 
\\	「肩凝り」は
\\	で表現します。
\\	「いらいらする」。	
\\	あの一件で彼は懲りてまじめに仕事しているかと思っていたけど、全然懲りてなかった。 
\\	「懲りてない」とは
\\	(何一つ学んでない)ということですね。	
\\	あんなひどいめにあったのに、まだ懲りてないの?! 
\\	「懲りない」とは
\\	「一つも学ばない」ということですね。	
\\	最近たこ焼きに凝っててさあ、毎日必ず食べてるよ。 
\\	で「…にはまっている」の意味になります。
\\	で「たこ焼き(たこ団子)」を表しています。	
\\	これ以上あなたにしてあげられることはない。 
\\	「これ以上…ない」は 
\\	で表現できます。	
\\	これからそちらに伺いますとお伝え下さい。 
\\	「これから」は
\\	「今すぐ」ということですね。	
\\	これだけの量の食料で足りるかい。 
\\	「これだけの量」は
\\	で表現できます。	
\\	海外でクリスマスと言えばターキーだけど、日本では特にこれという食べ物はないな。 
\\	「これという」は
\\	でそのニュアンスが出せるでしょう。
\\	「指定された」。	
\\	(すごい下手な絵を書いて見せながら)これでも学生時代は美術部だったんだけど。 
\\	「これでも…だった」のニュアンスを
\\	を使って表現しています。この
\\	とは、もちろん自分のことです。	
\\	その広告には、商品の長所がこれでもかというほど並べてあった。 
\\	で「公然と(人目をはばからずに)ひけらかす」という意味になります。	
\\	最初蒔いた種が全滅で悔しくて、今度はこれでもかというぐらいたくさん蒔いちゃった。 
\\	と 
\\	(過大な償いをする、補償し過ぎる)を組み合わせて「これでもかというぐらい」を表現しています。	
\\	だれに投票するか決めた?どの人もこれといって特徴がないのよね。 
\\	で「何か特別なあるもの」という意味。この場合の 
\\	は「あるもの」の意味で、形容詞 
\\	が後ろではなく前に来ています。	
\\	私はコロコロ気が変わる人間だから。 
\\	とは「気まぐれな人」のことです。
\\	と 
\\	の組み合わせにも注目してください。	
\\	転んでもただで起きちゃいけないぞって最近思うようになった。 
\\	の意味は「どんな経験でも、たとえどれほど不快な思いをしたとしても、無駄にはならない」になります。その後に、さらに、
\\	「もし、それから学ぼうという強い気持ちがあるのなら」を続けて「転んでも 
\\	の気持ちを表しているのです。	
\\	試験の結果なんて怖くて聞けないよ。 
\\	「怖い」は
\\	「おびえる」、
\\	を怖がる」を使えばいいでしょう。	
\\	ちょと遅く帰ってきたからって、そんなにこわい顔するなよね。 
\\	ここでは「こわい顔」を
\\	「そんな風に(そんな顔で)見ないでくれ」と意訳しています。	
\\	怖いものみたっさっていうのかな。いつもばっちり化粧をしている田中さんの素顔を見てみたいと思わない?どんな顔してるんだろうね。 
\\	ここでは「怖い物見たさ」を
\\	「好奇心から見てしまう」と言い換えています。厳密に訳せば、例えば、
\\	「怖い物を見たいという衝動」のようになるでしょうか。また、
\\	「見ちゃいけない、だけど、我慢できない」のように表現してもいいでしょう。ちなみに、
\\	とはスラングで「顔」の意味。	
\\	一度事故を起こすと一生だめになるかもって思うと恐くて運転できない。 
\\	ここでは
\\	「怖がらせる」、
\\	「怖くなる」と
\\	を組み合わせて「怖くて」を表現しています。	
\\	離乳食を作る時、大人のを作る途中で少しだけ小分けにしておいて、味付けを別々にすると楽よ。 
\\	「小分けする」は 
\\	を使って表現すればいいでしょう。	
\\	長年、政界の浄化が声高に叫ばれているが、一向によくならんね。どうしたものか? 
\\	が
\\	を声高に叫ぶ」に当たる言い回しです。
\\	を求めて)叫ぶ」。	
\\	バーで強面の兄ちゃんが僕の隣に座った。 
\\	「顔つき」に関しては、...-
\\	が使えます。例えば、「強面」なら
\\	あるいは
\\	、「いい男」は、
\\	、「醜い、ぶす」は、
\\	のように
\\	の前の形容詞を使い分ければいいのです。	
\\	このプラモデル、すぐ壊れそう。 
\\	は「薄っぺらな、もろい」という意味です。	
\\	根気強く頑張れば、きっとうまく行くよ。 
\\	は「忍耐強い」、
\\	は「一貫した」の意味。
\\	「いい結果に終わる」。	
\\	何を根拠にそんなことを言うんだ。 
\\	「根拠」は
\\	(基礎、根拠)で表現します。
\\	は「支持する」の意味。	
\\	このまま不景気が続くと、わが社が今後どうなるかが心配だ。 
\\	で「この先…に何が待ちかまえているのか」という意味になります。	
\\	寒いからコタツのコンセント入れたら? 
\\	で「コンセントを差し込む、繋ぐ」の意味になります。
\\	(スイッチを入れる)を使ってもいいでしょう。ちなみに、コンセントは和製英語で、英語では
\\	と言います。	
\\	眼鏡をやめて、コンタクトレンズにしたんだ。 
\\	「コンタクトレンズ」は
\\	、あるいは、
\\	と言います。複数形になっているところに注意してください。
\\	「取り除く」、
\\	「除く」。	
\\	今日はなんであんなに愛想がいいんだろう。何か魂胆があるに違いない。 
\\	の文字通りの意味は「隠された動機」です。
\\	は「たくらみ」の意。	
\\	毎日夕食の献立を考えるのが大変なの。 
\\	で「…の献立を考える」という意味になります。	
\\	今度こそ失敗しないようにしないと。 
\\	「今度こそ」は
\\	でいいでしょう。
\\	「しくじる」。	
\\	こんな朝早くから元気だな。 
\\	「こんな」は
\\	を使って表現すればいいでしょう。	
\\	私の命は私のもの、私以外のだれの物でもない。それをほかのだれかに脅かされるなんて、こんな屈辱的なことはない。 
\\	と
\\	(屈辱)を組み合わせて「こんな屈辱的な 
\\	を表現しています。
\\	は「屈辱的な」の意。	
\\	こんなことなら昼寝でもしてればよかった。 
\\	で「このようになるのなら」の意味になります。
\\	の代わりに 
\\	も使えます。	
\\	こんなこともあろうかと思ってちゃんと用意してきたよ。 
\\	で「このようなことが起きるんじゃないかという気がする」の意味。	
\\	ホップってこんなに背の高い植物だったんだ。(テレビを見ていて) 
\\	「こんなに」は
\\	で表現できますね。目の前にある場合には、
\\	、離れている場合には
\\	と
\\	と
\\	を使い分ける必要があります。	
\\	彼女太りすぎで命に関わるから胃を小さくする手術を受けなきゃいけないんだって。こんなになるまでほっとくなんて信じられない。 
\\	で「そんなに悪くなるまで放っておく」の意味。
\\	は「命に関わる」の意。	
\\	ま、こんなもんかな。 
\\	は「それでいい(それなりのことはしている)」、また、
\\	「仕事を仕上げる」の意味。	
\\	そんなこんにゃく問答にはつきあっていられないよ。 
\\	「こんにゃく問答」とは「訳の分からない話」のことですから
\\	「ねじ曲がった論理」や
\\	「馬鹿げた議論」で表現すればいいでしょう。	
\\	見て!この雑誌、今月号からこんなにコンパクトになったんだよ。これだったら持ち歩きに便利だね。 
\\	「コンパクト」は
\\	がそのまま使えます。
\\	は「持ち運びできる」の意。	
\\	夕食にコンビニ弁当を出したら、主人に叱られちゃったわ。 
\\	「コンビニ弁当」を
\\	(出来合いのランチ)と
\\	を組み合わせて表現しています。	
\\	今後いろんなものがコンピュータ化されていくと思う。それらは、便利であるという反面、いろいろ問題もでてくるだろう。 
\\	で「コンピュータ化される」の意味になります。
\\	「現れる、生じる」。	
\\	コンピューター関係の仕事をやっているっと言うと、皆コンピューターに詳しいと思うみたいだけど、私みたいにそんなに詳しくない人もいる。 
\\	で「コンピューターの分野で働いている」の意味になります。	
\\	金輪際うちの娘には電話を掛けないでくれ。 
\\	「金輪際 
\\	するな」は
\\	で表現すればいいでしょう。	
\\	ご安心ください。 
\\	は動詞で「不安がらせる」の意味。
\\	「落ち着く」。	
\\	君のやり方は相変わらず強引だな。 
\\	は「力を使う、力で押し切る」、
\\	「強引なやり方に頼る」の意。	
\\	彼は豪快に酒を飲み、そして豪快に笑った。 
\\	(うまそうに飲む)と 
\\	(うれしそうに笑う)で「豪快」のニュアンスを出しています。
\\	は「心から楽しんで」の意。	
\\	この前、友達と合計4人でアラスカに行って来ました。 
\\	「全員で4人」で「合計」の意味を表しています。	
\\	彼は、何事にも恐れず、ひるまず、そしていつもどっしり構えている真の意味での豪傑だ。 
\\	「豪傑」は 
\\	でいいでしょう。また、
\\	(真に大物の、英雄の)も使えます。
\\	は「ひるまない」、
\\	は「威厳をもって振る舞う」の意。	
\\	明日の合コンは気合入れよう! 
\\	(独身者)と 
\\	(懇親会)を組み合わせて「合コン」を表現しています。	
\\	彼の、ゴージャスな服装はまさに成金だ。 
\\	は「ぜいたくな」という意味。ここでは 
\\	を使ってもいいでしょう。	
\\	友だちが来月日本からカナダに来るのですが、彼女たちはトロントに最初滞在する予定でそこで合流しようということになってます。 
\\	「合流する」は 
\\	、あるいは、
\\	で表現すればいいでしょう。	
\\	ちゃんと話しの内容を理解しておかないと後で誤解を招く恐れがあるよ。 
\\	「誤解を招く」は
\\	「招く」と
\\	「誤解」を組み合わせればいいのです。	
\\	入社10年目にして5月病になった。 
\\	英語には「五月病」に当たる表現がありませんので、
\\	「スランプになる、落ち込む」でそのニュアンスを出してみました。
\\	「スランプ」。
\\	「春先に憂鬱な気分になること」が「五月病」に近い表現であると言えないこともありませんが。	
\\	今まではわざわざ外まで吸いに出て行かないといけなかったが、空港内に喫煙ルームが新設されていたので、彼は終始ご機嫌だった。 
\\	で「とっても喜んでいる」の意味になります。
\\	で「とても楽しい」という意味。	
\\	今日は部長、ご機嫌斜めだな。 
\\	が「ご機嫌斜め」に当たる言い回しです。	
\\	彼女は極彩色の服に身を包み、さっそうと舞台に登場した。 
\\	で「鮮明な色の衣装」の意味になります。	
\\	今日はわざわざ来てもらって御苦労さまでした。 
\\	「ご苦労様」をここでは
\\	「わざわざしてくれたこと」で表現しています。	
\\	それって、
\\	のせいでコンピューターが誤作動して起こったらしいよ。でも、何はともあれ大きな問題が起きなくて良かったよ。 
\\	「誤作動」は
\\	や
\\	を使って表現します。	
\\	その仕事はすぐに片づけられると思っていたのが誤算だった。 
\\	(…だと決めてかかるのは間違いである)で「誤算」を表現しています。	
\\	慣れないことだらけで何かとご迷惑をおかけするとは思いますが、早く慣れるように頑張りますのでご支援ご指導の程お願します。 
\\	で「ご支援ご指導」を表現しています。
\\	の代わりに 
\\	を使ってもいいでしょう。	
\\	また午前様か。どうやって言い訳しようかな。 
\\	で「夜中を過ぎて帰宅する」の意味になります。	
\\	乙武さんは「五体不満足」という本を書いた。 
\\	「五体不満足」をここでは、まず、
\\	とローマ字書きして、その後で
\\	と説明してみました。下の説明も参考にするといいでしょう。	
\\	彼は社内のごたごたを上手く収めた、なかなかのやり手だって評判だよ。 
\\	この場合の「ごたごた」は 
\\	で表現すればいいでしょう。
\\	は「雑然とした」、
\\	は「めちゃくちゃな」の意。	
\\	「私のカレシ~、世界一やさしい男(人)なの」。///
\\	「ごちそうさま」。 
\\	は「はい、はい」、
\\	「よかったね」、
\\	「素敵ね」の意味。すべて皮肉って言っていることになります。	
\\	ごちそうさまでした。とても美味しかったです。 
\\	「ごちそうさまでした」は
\\	で表現すればいいでしょう。	
\\	旅行は好きでいろんな所に行っているのだけど、なんかどこで何を見たのかゴチャゴチャになっている。 
\\	で「ごちゃ混ぜになる」の意味になります。
\\	を整理できない」。	
\\	私、アリみたいな小さい虫がごちゃごちゃ固まっているの見ると、ぞっとしちゃう。 
\\	で「ごちゃ混ぜになった集まり(かたまり)」の意味。
\\	「震える、ぞっとする」、
\\	「ゴチャゴチャの」、	
\\	横で彼がごちゃごちゃ言うから、全然相手の電話の声が聞こえなかったので聞き間違えたのです。 
\\	「ゴチャゴチャ」のニュアンスを
\\	「でたらめな」を使って表現しています。	
\\	政府の今回の事件への対応は、どうも後手後手にまわっているとしか言えない。 
\\	ここでは「後手後手にまわる」のニュアンスを 
\\	(ぐずぐず引き延ばすこと)、
\\	(対応ののろさ)で表現しています。	
\\	資金が調達できないということなら、この建設計画はご破算ですな。 
\\	で「…を白紙にする」に当たる表現です。
\\	は「手に入れる」の意。	
\\	ご無沙汰しております。ご家族の皆さんはお元気ですか。 
\\	で「長い間連絡をとっていない」の意味になります。
\\	の使い方にも注目。	
\\	私は自分へのご褒美として指輪を買った 
\\	で「自分自身にご褒美を与える」の意味になります。	
\\	あの店でお釣りを誤魔化された。 
\\	は「お釣りを少なく渡す」という意味です。	
\\	彼は上司に対していつもごまをすってるので、上司からは気に入られてるが同期の同僚からは嫌われている。 
\\	で「…にお世辞を言う」の意味になります。
\\	は「上司」、
\\	は「同僚」の意。	
\\	あいつは上司にごまをすってばかりいるから嫌なんだ。 
\\	は「…におべっかを使う」、
\\	は「ご機嫌を伺う」の意。	
\\	ごみの日っていつですか?燃える物と燃えない物を分けないとだめですか? 
\\	はつけなくてもいい)で「ゴミの収集日」の意味になります。簡単に
\\	と言ってもいいでしょう。	
\\	日本のゴミ問題は深刻です。1人1人の自覚が大切だと思います。 
\\	で「深刻なゴミ問題を抱えている」の意味。
\\	(自覚)、
\\	(過剰なゴミ)の意。	
\\	あっ、いけない。ゴミを出すの忘れてた。 
\\	が「ゴミを出す」に当たる表現です。	
\\	これ以上のトラブルはごめんだ。 
\\	この場合の「ごめん」は
\\	「…はもうたくさんだ、いらない」で表現すればいいでしょう。	
\\	あの教授は、政府の方針は何でも弁護する御用学者だ。 
\\	(…の支配下にある、…の意のままで)と
\\	(後援者、パトロン)を使って「御用」を表現しています。	
\\	こちらの部屋にある額の絵もご覧ください 
\\	「ご覧ください」は
\\	でいいでしょう。	
\\	よろしければこの絵もご覧になって行ってください。 
\\	と
\\	を組み合わせて「ご覧になって行ってください」を表現しています。
\\	は
\\	に注意を引く」の意。	
\\	彼らはごり押しでその計画を委員会で通した。 
\\	「ごり押しする」とは「力で無理矢理にする」こと。この意味に近いのが
\\	「全くの力でやる」。また、
\\	を無理矢理に通す」、
\\	「自分の地位を利用して...する」、
\\	を無理矢理押しつける」などを	
\\	1週間ぐらい休みがあれば旅行に行くけど、土日程度の休みだったら、家でごろごろしている。 
\\	「ごろごろ」は
\\	「寝転がる、だらだらする」でいいでしょう。	
\\	ごろごろしてないで、たまには家事を手伝ってよ。 
\\	「ごろごろする」は
\\	は「ぶらぶらする、のらくろする」で表現すればいいでしょう。	
\\	滅茶、最悪だ。もうだめだ。 
\\	「最悪」は
\\	!で表現できますね。	
\\	よかったー。最悪の事態は間逃れた。 
\\	「最悪」は
\\	ですから、これを使って、
\\	と言えばいいでしょう。
\\	「避ける」。	
\\	先生の都合でお休みになっていたクラスは再開のめどはたちましたか? 
\\	ここでは「再開」を 
\\	(動名詞)、
\\	(動詞)で表現しています。	
\\	犯人の記憶に基づいて事件が詳しく再現された。 
\\	で「詳しく再現される」の意味になります。	
\\	最高にいい1年でした。 
\\	で「最高に」を表現しています。
\\	は「いままでで」の意。	
\\	エアーズロックの頂上に着いた時、これほど最高の気分はないと思いました。 
\\	で「最高にうきうきした気持ちになる(うきうきした気分の頂点に到達する)」の意味。	
\\	来月アメリカに行くことになったので、このクラス最後まで受講することができなくなりました。残念です。 
\\	で「このクラスを最後まで見届ける」という意味になります。	
\\	採算性を度外視して、理想ばかり追求していたんじゃ組織はもたないよ。 
\\	が「採算性」に当たる表現です。
\\	は「実現可能性」の意。	
\\	こんなにバスの最終が早いと思わなかった。もう少しで、乗り遅れるところだったよ。あぶない、あぶない。 
\\	この場合の「最終」は 
\\	で表現すればいいでしょう。
\\	は「乗り損ねる」の意。「あぶない、あぶない」の表現の仕方にも注目。	
\\	こんな時代にリストラされたら、再就職はとても無理だな。 
\\	「再就職」を 
\\	で表現しています。
\\	は「ほとんど」の意。	
\\	最終的には自分で決断する事です。 
\\	「最終的に」は
\\	で表現すればいいでしょう。	
\\	何事も最初が肝心だよな。 
\\	は「うまいスタートを切る」の意味。「肝心」は
\\	「非常に重要な」、あるいは、
\\	でいいでしょう。	
\\	30歳にして、サイン会に並んでちょっと恥ずかしかった。でも、きっと人生で最初で最後だと思う。 
\\	「最初で最後」を 
\\	で表現しています。	
\\	私が仕事を辞めた最大の原因は人間関係です。 
\\	で「一つ最も大きな原因」の意味になります。
\\	「要因」。
\\	「同僚」。	
\\	昨日のデート雨にふられて最低だったよ。 
\\	ここでは「最低だった」をここでは
\\	「台無しにする」を使って表現しています。
\\	は「最悪」の意味。	
\\	電車に乗るときには携帯のスイッチ切れよな。最低限のマナーってものがあるだろ。 
\\	ここでは「最低限のマナー」を
\\	(最低限の礼儀)で表しています。	
\\	まずは、最低限必要なものから買おう。そうでないと、また物だけ増えて結局使わずじまいで捨てることになりそうだから。 
\\	で「必要最低限の…」という意味になります。
\\	は「散らかす」の意。	
\\	なかば死語になった「日米貿易摩擦」という言葉が、突然再燃する事態に至った。 
\\	「再燃する」を 
\\	(復活させる)を使って表現しています。
\\	は「目立つようになる」の意。	
\\	長島監督と野村監督の采配ぶりには際立った違いがある。 
\\	「采配」を
\\	で表現できます。
\\	は意味が掴みにくいかも知れませんね。この言い回しで「これ以上違うということはない--全く違っている」という意味になるのです。
\\	は「正反対の」の意味。	
\\	お!またこの歌売れ出した。ランキング再浮上だ!根強い人気だね。 
\\	で「上に戻る」の意味。
\\	は「しっかりしている」、
\\	は「根深い」の意。	
\\	マイケルJフォックスがでているドラマの再放送が始まったんだ。毎回楽しみで楽しみで。 
\\	「再放送する」は動詞、名詞の両方に使えます。	
\\	制度変更のため、多くの企業が債務超過を余儀なくされる。 
\\	「債務超過」は
\\	と言います。
\\	で「…を引き受ける」の意。	
\\	えっ、僕の投稿は結局採用されなかったんですか?! 
\\	「作用する」は、簡単に
\\	「使う」でいいでしょう。
\\	は動詞で「認める、オーケーにする」の意。	
\\	ぼくは逆上がりができます。 
\\	「逆上がりをする」は
\\	という言い方が辞書には載っていますが、ここでは
\\	を使って文字通りに説明してみました。	
\\	親切のつもりだったのに、逆恨みされた。 
\\	「憤って反応した」、
\\	「反対の結果になる」の意。	
\\	人の神経を逆撫でするようなことを言うのは止めろよ。 
\\	が「逆撫でする」に当たる言い回しです。	
\\	魚の小骨が喉にひっかかってなかなか取れない。 
\\	「小骨」は
\\	でいいでしょう。
\\	「ひっかかる」。	
\\	以前、あなたが言っていたお酒は酒屋に言って手配してもらっていますが、まだ届いてないんです。 
\\	「酒屋」は
\\	でいいでしょう。
\\	は「配達する」の意味。
\\	~で「~から…を取り寄せる、~に…を注文する」の意。	
\\	うちの中学校は、水泳と縦笛えが盛んだった。おかげで、今平泳ぎならいくらでも泳げるよ。 
\\	ここでは「盛ん」を
\\	「とても人気がある」、
\\	に夢中になる、関心を持つ」を使って表現しています。	
\\	正直言って、私の英語力って以前よりも下がったと思いますか? 
\\	この場合の「下がる」とは 
\\	(悪くなる、ひどくなる)ということですね。
\\	も使えます。	
\\	私はあなたより2駅先で降ります。 
\\	この場合の「先」は
\\	(後)で表現します。	
\\	そんな中途半端な改革では、問題を先送するだけで、根本的な解決にはならない。 
\\	この場合の「先送り」は
\\	(後回しにする、延期する)で表現できますね。
\\	は「中途半端なやり方」、
\\	は「遅らせる」の意。	
\\	(プロジェクトが)初めからこんなにつまずいてたら、ほんと、先が思いやられるね。 
\\	で「先に何が待ちかまえているかを考えるのが嫌だ」の意味になります。
\\	「行き詰まる」、
\\	「つまづく」。	
\\	この会社どうなるんだろう。まるで先が見えない。 
\\	で「道の前方が見えない」の意味になります。
\\	は「はっきりしない」の意。	
\\	その高校では、文部省の新指導要領の内容をはるかに先取りした英語教育が行われていた。 
\\	「(はるかに)先取り」は
\\	「(ずっと)進んでいる、先進的な」を使えばいいでしょう。
\\	「文部省」。	
\\	春を先取りして薄着したんだけど、まだ早過ぎたらしい。おかげで、風邪引いちゃったよ。 
\\	ここでは「先取り」のニュアンスを
\\	を楽しみにする、期待する」を使って表現しています。	
\\	彼女は仕事があるから私より先に帰ります。 
\\	この場合の「先に」は 
\\	でいいでしょう。	
\\	先のことなんてわからないんだし、今自分がどう思うかを大事にしたいと思う。 
\\	「先のこと」は 
\\	(将来、未来)で表現すればいいでしょう。	
\\	先ほど、鈴木様からお電話がありまして、あとで会社のほうに電話をいただきたいと言っていました。 
\\	「先ほど」は 
\\	、あるいは、
\\	で表現すればいいでしょう。
\\	は「感謝する」の意。	
\\	来年もこのまま不景気が続くと、うちの会社も先行きが不安だな。 
\\	「先行き」は
\\	、「不安」は
\\	で表現できます。
\\	は「不景気」の意。	
\\	う!先越された。早く出てきてよ!(トイレで) 
\\	で「私より先に飛び込む」の意味。	
\\	良品質の製品を安く他国から輸入できる時代なので、大手電気メーカーは人員削減に踏み切った。 
\\	で「人員削減を実行する」の意味になります。
\\	も使えます。	
\\	政府の改革案は、結局、官僚の作文に過ぎなかった。 
\\	ここでは「作文」を 
\\	(手仕事)を使って表現しています。
\\	は 「結果として…になる」、
\\	は「やっつけ仕事」の意味。	
\\	私は酒癖が悪い上に記憶を無くしてしまうの。 
\\	とは「ひどい酔っぱらい」のことです。	
\\	彼は酒が強くて、少々飲んでも顔に出ません。 
\\	が「お酒が強い」に当たる言い回しです。	
\\	ちょっと、風邪引いているからってそんなに避けなくてもいいじゃん。 
\\	この場合の「避ける」は
\\	でいいでしょう。
\\	に近づくのを怖がる」。	
\\	山をのぼるに連れて、お菓子の袋がふくらむんだよ。今思えば、小さい時は、そんな些細なことでも興味を持って、楽しんでたなあ。 
\\	「些細なこと」は
\\	や
\\	「ほんの小さなこと」で表現できます。	
\\	私はあなたをずっと支えていきたい。 
\\	で「あなたのためにそこにいる」と言っていることになります。
\\	は「…に頼る」の意。	
\\	写真を送ってもらったお礼にささやかなプレゼントをした。 
\\	「ささやかな」は
\\	「質素な、簡単な」、
\\	「大したことのない」で表現できます	
\\	差し出がましいようですが、私から意見を述べさせて頂きます。 
\\	の文字通りの意味は「出しゃばりすぎに聞こえるのは承知の上で」になります。	
\\	デポジットから最終月請求分の金額を差し引いて返金して欲しい。 
\\	「差し引く」は
\\	でいいでしょう。
\\	「残り」。	
\\	この鮑お土産でもらったはいいけど、やっぱりこのまま刺し身で食べた方がいいかな。でもこの時期、生物はちょっと気になるな。 
\\	「刺身」は
\\	と言いますが、ここでは
\\	を
\\	「刺身にする(生のまま切って)」と副詞的に使っています。	
\\	切なさが胸を刺す。 
\\	ここでは「刺す」を 
\\	(突き通す)を使って表現しています。	
\\	さすが外人さんだね。英語うまいね。 
\\	「さすが」をここでは
\\	と分かる」、
\\	のはずがない」を用いて文脈に合うように言い換えてみました。	
\\	大きいプレートに大盛りの料理が出てきて、さすがアメリカ料理って感じでした。 
\\	ここでは「さすが」を
\\	であるということが分かる」で表現しています。
\\	「これがまさにアメリカ料理だね」。	
\\	さすがだね。君ならやってくれると思っていたよ。 
\\	「さすが」をここでは
\\	「君らしいね」と言い換えています。	
\\	さすがに週3回英会話のレッスンを会社帰りに受けるのは大変かな。 
\\	「さすがに 
\\	かな」のニュアンスを
\\	を使って表現しています。	
\\	さすがのあなたも、このジェットコースターには参るでしょ。 
\\	(あなたのような人でさえ)で「さすがの」のニュアンスを出しています。
\\	は「冒険好きな、大胆な」、
\\	は「ためらる」、
\\	は「…するのを嫌がる」の意味。	
\\	当面、その会社で好きな勉強を続けさせてもらいなさい。 
\\	この場合の「~させてもらう」は
\\	(使役用法)を使えばいいでしょう。
\\	「機会」。	
\\	あいつ地方の支店に左遷されるらしいよ。 
\\	はそれぞれ「降格させる」、「追いやる」の意味。	
\\	この虫、臭いに誘われてやってきたな。ちょっとそこのお菓子片づけといて。 
\\	この場合の
\\	は「引きつける(誘惑する)」という意味です。	
\\	このあいだあれだけ自分の英会話力を自慢したんだから、今日は外国人とさぞかしよく意思疎通できるよね。 
\\	「さぞかし」を 
\\	(きっと)を使って表現しています。また、2番目の英訳ではそのニュアンスを 
\\	で出しています。	
\\	君と僕は結婚する定めだったんだよ。 
\\	で
\\	する運命にある」の意味になります。	
\\	周りの皆さんの英語があまりにも自然なので、ここはアメリカじゃないかと錯覚した。 
\\	この場合の「錯覚」は
\\	で簡単に表現できますね。	
\\	って誰かが役でやってたなーとさっきから考えているんだけど、誰だったか思い出せない。うー気になる!!! 
\\	「さっきから」とは
\\	「ちょっと前から」ということですね。	
\\	さっき食べたばっかりなんだよね。 
\\	「さっき…したばっかり」は、
\\	と
\\	を組み合わせて表現すればいいでしょう。	
\\	ただの気休めにすぎないかもしれないけど、殺菌効果があると聞いて毎日緑茶でうがいしています。 
\\	「殺菌効果がある」は
\\	で表現します。
\\	は「気休め」、
\\	は「うがいする」の意。	
\\	今日録画したビデオさっさと見てしまわない?10時から見たいテレビがあるんだ。 
\\	「さっさ」は 
\\	で表現できます。	
\\	とにかく、相手の気持ちを察することが必要だ。 
\\	この場合の 
\\	は動詞で「感じる」の意味。
\\	は「直感で知る」の意。	
\\	早速来月にでもそこの旅行会社に電話して聞いてみるね。 
\\	「早速」を 
\\	で表現しています。
\\	は「旅行業者」の意。	
\\	家に帰ったらさっそく試してみよう。安かったから味のほうは今ひとつかもな。 
\\	「さっそく」は 
\\	(…したらすぐ)で表現すればいいでしょう。
\\	は「…した瞬間に」の意。	
\\	通販の受注電話は番組放送直後から1時間くらいに殺到する。 
\\	「殺到する」は
\\	「洪水のように押し寄せる」、
\\	に圧倒される」で表現できます。	
\\	髪を短く切って、さっぱりしたよ。 
\\	「さっぱりした」は
\\	「さわやかな気分になる」で表現できます。	
\\	唐揚げにレモンかけるとさっぱりしておいしいよ。 
\\	ここでは「さっぱり」を 
\\	で表現しています。	
\\	今の話さっぱりわからなかった。 
\\	で
\\	が一言も理解できない」という意味になります。	
\\	私、運動関係は学生時代からさっぱり。毎年、運動会が憂鬱だったもん。 
\\	は「運動はだめだった」の意味。
\\	「運動会」。	
\\	英検1級の面接では、しゃべれるかなと思っていたけど、さっぱりだめだった。 
\\	(一言も言えなかった)で「さっぱりだめだった」を表現しています。	
\\	彼女はさっぱりとした性格だからさ。 
\\	「さっぱり」とは
\\	「正直」で
\\	「率直な、まっすぐな」ということですね。	
\\	ずいぶん殺風景なところに住んでるね。駅前にコンビニが一軒しかないじゃない。 
\\	は「寂しい、住む人のない」、
\\	は「荒れた、もの悲しい」の意味。	
\\	この間、うちの会社の前でドラマの撮影が行われたんだ。 
\\	「撮影する」は
\\	を動詞として使えばいいでしょう。	
\\	ここ1ヶ月ほど、殺人的な忙しさが続いた。 
\\	には「ひどく大変なこと(不可算名詞)」という意味があります。	
\\	今日の負けはしょうがないと、彼はさばさばした表情でインタビューに応じた。 
\\	「さばさば」のニュアンスを 
\\	(率直な、開けっ広げな)で表現しています。
\\	は「率直に」の意。	
\\	彼には原稿の締め切り日を1週間サバを読んで伝える。 
\\	「サバを読む」をここでは
\\	「一週間遅れることを考慮しておく」と言い換えて表現しています。
\\	「本当の締め切りの一週間前」の意。	
\\	本当に2人しか来なかったら,寂しいパーティーになるよね。 
\\	は「陰気な、元気のない」、
\\	は「ださい、調子はずれの」という意味。	
\\	留学を決意して友だちとかにもう会えないかもって会ったりしていると、だんだんその日が近づくにつれて寂しくなってきた。 
\\	で「その日が近づくにつれ、どんどん悲しくなる」の意味。
\\	は「悲しませる」の意。	
\\	ハワイの様な常夏でリゾート地だと思っていましたが、人通りも少なく、店もあまりなく意外と寂れた街だった。 
\\	この場合の「寂れた街」は
\\	「ひどく(悲しげな)小さな街」で表現すればいいでしょう。
\\	「常に」。	
\\	さぶ~!そんなくさいこと言わないでよ。 
\\	!(おい、頼むよ)で「さぶ~」のニュアンスを出しています。
\\	とは「頼むからそんな面白いことを言って僕を笑い転げさせないでくれよ
\\	の意味を表し、つまらないことを言った相手を皮肉る時に使う言葉です。	
\\	彼はその言葉の意味が差別用語だと知ってて使ったのだろうか? 
\\	で「差別的なことば」の意味。	
\\	最近、サボり気味で来てなかったんです。 
\\	「さぼり気味」は
\\	「ちょっと怠惰である」を使えばいいでしょう。
\\	「授業の出席に関してはあまりよくない」。	
\\	今日、4時限目の授業サボっちゃった。 
\\	「サボる」は 
\\	を使って表現します。	
\\	こんどのプロジェクトは、佐藤君のおかげで成功したようなものだ。佐藤様様だな。 
\\	「…に敬意を表する、兜を脱ぐ」を使って「様様」のニュアンスを表しています。	
\\	帰国してしまうなんて寂しくなるわ。 
\\	「…がいなくて寂しい」は
\\	で表現できますが、ここでは
\\	を使っています。	
\\	五月雨式に入稿していただけると助かります。 
\\	「五月雨式に」を
\\	「出来上がり次第」、
\\	「進行にしたがって少しずつ」と表現しています。	
\\	君はなんでいつもそんなさむいギャグが言えんの? 
\\	「さむいギャグ」とは
\\	「つまらない、取るに足らない冗談」ということですね。
\\	は「そんな(さむい)冗談には取り締まる法律が必要だ」と言っていることになります。	
\\	なんだよそんなに服を着込んじゃって。寒がりだな。 
\\	で「寒さに耐えられない」の意味になります。
\\	をいらいらさせる、悩ませる」。	
\\	冬に短パン半袖の人を見ると、見ているこっちが寒くなるよ 
\\	「寒くなる」は使役動詞の
\\	を使って、
\\	と言えばいいでしょう。	
\\	兄弟で恋の鞘当て。 
\\	「恋の鞘当て」と
\\	「恋敵になる」ということ。	
\\	鏡に写る物体は左右が逆になるのに、どうして上下が逆さまにならないのだろう? 
\\	動詞の
\\	を使えば「逆」を表すことができます。「物体」は
\\	、「上下さかさま」は
\\	ですね。	
\\	言葉が通じる通じないはコミュニケーションを大きく左右すると思います。自分の言いたいことが言えないときほど、悲しいことはない。 
\\	「左右する」とは 
\\	(決定する)ということですね。	
\\	血液の凝固を防ぎ、血液をさらさらにする薬を現在服用しています。 
\\	ここでは「さらさらにする」を 
\\	(薄める)、
\\	(さらさらと流れる)で表現しています。
\\	は「凝固する」の意。	
\\	梶田さんのさりげない心遣いがうれしい。 
\\	とは「でしゃばらない(控えめな)思いやり」のこと。
\\	は「控えめな種類の」の意。	
\\	あ、さりげに漢字直ってる。 
\\	「さりげなく」は
\\	「そっと」で表現できます。	
\\	人からされて嫌なことは、あなたもしてはいけません。 
\\	ここでは「される」を 
\\	を使って表現しています。	
\\	マッサージされる方は気持ちよくていいかもしれないけど、する方はすごく疲れるんだよね。 
\\	で「マッサージをされる方」を表現しています。
\\	で「マッサージをするのは疲れる」の意味。
\\	は「マッサージ師(男性)」、
\\	は「マッサージ師(女性)」の意。	
\\	財布を盗まれてビックリして騒いでいたら、近くにいた私服ガードマンらしき人たちが駆け寄ってきてくれました。v 
\\	で「大騒ぎする」の意味になります。「私服」の表現の仕方にも注意。	
\\	先ほどあなたに文字化けした
\\	メールばかりを何通も送ってしまい、お騒がせしました。 
\\	この場合の「お騒がせしました」とは「迷惑をおかけしました」ということですから、
\\	を使って表現すればいいでしょう。
\\	は「文字化けした」の意。	
\\	彼はさわやかな青年ですね。 
\\	「さわやか」は 
\\	、あるいは、
\\	(清々しく)を使って表現すればいいでしょう。	
\\	彼は今機嫌(きげん)が悪いのでそっとしておこう。さわらぬ神に祟りなしだ。 
\\	が「さわらぬ神に祟りなし」に当たる表現です。
\\	「つま先で(静かに)歩く」、
\\	「怒らせる」。	
\\	彼女、産休を取ることになったので、この仕事、君が引き継いでください。 
\\	「産休」は
\\	と言います。	
\\	俳句は短い詩ですが、一行に書くこともあり、三行分かち書きにすることもあります。しかし、行末で韻を踏んでいるわけではありません。 
\\	が「三行分かち書き」に当たる表現です。
\\	とも言えます。	
\\	(ありがとう!) 参考にしてみるよ! 
\\	は
\\	に目を通す」の意。	
\\	天気予報は100%は信じられないけど、何か計画を立てる時とかに参考にはするね。 
\\	「参考にする」は
\\	を使えばいいでしょう。
\\	は信じられない」の表現の仕方にも注目。	
\\	おかしいなあ、辞書にはこう書いてあったのに。辞書も参考程度にしかならないってことか。 
\\	ここでは「参考程度」を
\\	(ヒント、示唆)で表現しています。	
\\	昨日の夜、インターネット上の情報を参考にしながら、このレポートを書いたんだ。 
\\	ここでは「参考にする」を
\\	「…を利用する」を使って表現しています。
\\	は「参考資料」の意。	
\\	このホームページは、とても参考になる。 
\\	「参考になる」とは 
\\	(有益な情報(内容)がたくさんある)ということですね。
\\	は「有益な、情報を提供する」の意。	
\\	三三七拍子で締めましょう! 
\\	英語圏には「三三七拍子」の習慣はありませんので、ここではそのまま
\\	と直訳しています。
\\	「いつもの」。	
\\	彼は、酒を飲みながら散々文句を言った後、今度は、泣き出した。 
\\	ここでは「散々」を
\\	「あらゆる種類の」を使って表現しています。
\\	と動詞を繰り返すことでもそのニュアンスは出せますね。	
\\	今朝は朝から散々だったよ。寝坊はするし、その上電車は遅れるし。 
\\	「散々」のニュアンスを
\\	と
\\	で出しています。
\\	は「発展して…になる」の意。	
\\	会議で散々な目にあった。 
\\	「散々な目にあう」は
\\	でいいでしょう。
\\	とは「大変な状況」の意。	
\\	賛成多数でこの件に関しましては認められました。 
\\	が「大多数」、
\\	が「…に賛成である」の意味。
\\	「認める」。	
\\	彼は三度の飯より映画を観るのが好きで、1週間に10本は映画を観てるらしいよ。 
\\	で「食事をするより映画を観ていたい」の意味。
\\	~「~よりむしろ... がしたい」。	
\\	前は2人で行ったけど、今度は山田も入れて3人で行こうよ。 
\\	は「三人組」の意味。
\\	を加えて、3人にしよう」。	
\\	彼は、走って、守って、攻撃できるという三拍子がそろっている。 
\\	ここでは「三拍子」を
\\	「多才な」を使ってそのニュアンスを表現しています。自動詞
\\	の使い方にも注目。	
\\	あるホームページ上でその映画の批評が載ってて、最高って言う人もいれば、いまいちでお金と時間の無駄っていう人の賛否両論だった。 
\\	ここでは「賛否両論」を 
\\	(コインの裏表)で表現しています。	
\\	所詮、政治とは妥協の産物だ。 
\\	「産物」は 
\\	で表現すればいいでしょう。	
\\	昨日は家に帰ってから夕飯食べて、ダウンタウンを散歩しました。だって、夜8時でもまだ明るいんだもの。いい運動になったよ。 
\\	で「ダウンタウンを散歩する」の意味。
\\	は「いい運動」の意。冠詞 
\\	は必要ありません。	
\\	その話は、まるで3面記事のようだ。 
\\	「三面記事」は
\\	または
\\	と言います。	
\\	やっぱり3連休だと、自分の時間が作れていろいろやりたい事ができるなぁ。でも、また明日から仕事か、憂うつだ。 
\\	「3連休」は
\\	ですね。
\\	は「憂うつな」の意味。	
\\	その本は在庫がございませんので、お取り寄せになります。 
\\	で「その本は在庫がない」の意味。
\\	は「在庫切れ」の意。	
\\	今回の学費改訂は、在校生に影響を及ぼさない。 
\\	で「現在、登録(入学)している学生」の意味になります。	
\\	在宅で仕事をするようになってから自由な時間が増えた。 
\\	「在宅(で働く)」は
\\	でいいでしょう。	
\\	の点数がそのまま会話力とイコールになるとは思えないけど、判断する材料がそれしかないから仕方がない。 
\\	ここでは「材料」を
\\	「(判断するための)基準」を使って言い換えています。
\\	「協力し合う」、
\\	「直接...になる」。	
\\	何度も挫折を味わって、人として大きく成長する。 
\\	は「頓挫、挫折」の意味。
\\	「自分の足で立ち上がる」。	
\\	僕、人生で1度も挫折感味わったことないんですよ。 
\\	「挫折感を味わう」を
\\	「失敗をして絶望的な気持ちになる」で表現しています。	
\\	ざっくばらんに語り合おう。 
\\	(遠慮のない)と 
\\	(率直な)を組み合わせて「ざっくばらん」を表現しています。
\\	は「隠し事をしない」の意。	
\\	私は毎月
\\	をとっています。 
\\	で「…を購読する」の意味になります。	
\\	ざっとかき混ぜればいいんじゃない? 
\\	は「雑に」の意味。	
\\	これをざっと読んでチェックしてくれる? 
\\	は「ざっと読む」、
\\	「素早く読む」の意味です。	
\\	周りの雑音に惑わされずに、自分が正しいと思ったことをやりなさい。 
\\	この場合の「雑音」はそのまま
\\	で表現できます。
\\	「異議、苦情」。	
\\	雑談してないでさっさと終わらせて下さい。 
\\	で「あれこれおしゃべりする」の意味。
\\	は「ぺちゃくちゃしゃべる」の意。	
\\	ざまあみろ!いつも意地悪な事ばかりしてるから罰が当たったんだ。 
\\	で「当然の報い」の意味になります。
\\	「いじめる、あら探しをする」。	
\\	社と交渉するのはざるで水をすくうようなものだ。 
\\	とは「ざるで水をすくうように簡単なことだ」、つまり、「大変だ」と言っていることになります。
\\	「水銀のしずく」。	
\\	そのスパゲッティーあと5分でゆであがるから、そしたらザルにあげてくれる? 
\\	ここでは「ザルにあげる」を 
\\	(水切り器で水気を切る)で表現しています。	
\\	君の新しい車、斬新なデザインで洗練されたフォルムしてるよね。 
\\	「斬新な」なに当たる形容詞には、
\\	「革新的な」、
\\	「新奇な」、
\\	「ユニークな」などがあります。
\\	「外形」、
\\	「スマートな」。	
\\	引き落とし口座が残高不足のままだったので、水道料金が未納になっていた。 
\\	「残高不足」は、
\\	と
\\	、あるいは、
\\	「不十分な財源」を使って表現します。
\\	は「実行を怠る」の意。	
\\	残念だったね。次は頑張ってね。 
\\	「残念」、「頑張って」は、
\\	で表現すればいいでしょう。	
\\	あのパーティに招待してくれるって!ああ残念無念。もうその日は別の約束が入っているんだ。 
\\	で「心から... は言いたくない」の意味。
\\	「残念」。
\\	「招待される」。	
\\	今までの試合の中で一番印象に残っている試合は何ですか? 
\\	「試合」は 
\\	ですね。
\\	(忘れられない)と 
\\	(際だつ)の使い方にも注目。	
\\	幸せ太り 
\\	「幸せ太り」に近い意味になるように、
\\	「食事を与える」を使って表現してみました。
\\	「太った」の意。次回は、句だけではなく、ポイントとなる表現が使われている文を投稿するようにお願いします。	
\\	ついにあの教材のカセットテープが
\\	化されるみたいだよ。 
\\	で「…の
\\	を作る」という意味になります。
\\	でも同じような意味が表せます。	
\\	これ、CMで見てずっと気になっていたやつだ。 
\\	は
\\	のことですね。前置詞は
\\	を使います。
\\	「気になっている」。〔注)英語では
\\	とは言いません!	
\\	これと言って趣味はないんですが、強いて言えば、音楽鑑賞ですかね。 
\\	には「無理に解釈する、誇張する」の意味があります。	
\\	なんであたしがそんな仕打ち受けなきゃいけないの? それって、お門違いもいいとこよ。 
\\	(そんな扱いを受ける)で「仕打ちを受ける」を表現しています。
\\	は「ずうずうしく…する」の意。	
\\	この料理、塩味がきついんじゃないの。 
\\	で「塩味がきつすぎる」の意味になります。	
\\	普段はすぐふて腐れるのに、今日はやけにしおらしいな。 
\\	(子猫のようにおとなしい)、
\\	(素直な)で「しおらしい」を表現しています。
\\	は「ふくれっ面をする」の意。	
\\	彼女は友達に裏切られ、いつか仕返ししてやろうと思っている。 
\\	「仕返しする」は
\\	「復讐する」で表現します。
\\	「裏切る(背中を刺す)」。	
\\	最初に裏切ったのはきみだ。きみに僕を責める資格なんてないね。 
\\	ここでは「資格」を 
\\	(権利)を使って表現しています。
\\	は「裏切る」の意。	
\\	あんな所にああいう仕掛けがあったんだ。なるほどね。 
\\	とは「装置、道具」のことです。
\\	は「装備する、セットする」の意。	
\\	の後ろには国名しかこない(言えない)のですか? 
\\	「~しか来ない」は
\\	と
\\	を組み合わせて表現します。
\\	「文法的に言って」。	
\\	彼のことを誤解してしまったようね。まぁ、そう思われても仕方がないけど。 
\\	「仕方がない」は
\\	を使って表現でします。
\\	で「(その人の)せいではない」ということが表現できる。	
\\	あー!映画始まってるみたい。どうする?仕方がないから、また今度にする?それとも、違うの見る? 
\\	この場合の「仕方がない」は 
\\	(遅すぎる)ということですね。
\\	は「どうすることもできない」の意。	
\\	今の仕事は、ただ生きるためにやっているようなもので、全然興味もわかないし仕方なく働いているって感じです。 
\\	(ほかに選択の余地がない)で「仕方がない」を表現しています。	
\\	風邪気味だけど、今からではもうキャンセル料を取られてしまうし、それに今ごろキャンセルしたらみんなにも迷惑がかかるから行くしかないか。 
\\	「行くしかないか」を
\\	(ほかに選択の余地はないかな)で表現しています。	
\\	どうしたの?お父さん、そんなしかめっ面して。 
\\	「しかめっ面」は
\\	といいます。これは動詞としても使えます。	
\\	あまりにも悪ふざけがひどかったから、上司に彼をしかってもらった。 
\\	で「(人を)うんとしかる」の意味になります。
\\	は「しかりつける」の意。	
\\	どうせ俺はしがないサラリーマンさ。そんな高級レストランじゃとても食事できないよ。 
\\	ここでは「しがない」を
\\	「取るに足らない」で表現しています。
\\	「つまらない、地味な」、
\\	「働く人」。	
\\	(すごい英語力ですがお仕事は何なんですか。)なあに、しがないサラリーマンですよ。 
\\	「しがないサラリーマン」をここでは
\\	「ごく当たり前のサラリーマン」と表現しています。この
\\	は
\\	「普通の」という意味を表しています。ちなみに、「サラリーマン」は
\\	とも
\\	とも言います。	
\\	業界の古いしがらみに縛られて、思うような事業が展開できない。 
\\	には「しがらみ、義理」という意味があります。
\\	は「縛りつける」の意。	
\\	あんな敷居が高い人、相手にしません。 
\\	は「あんなに自分の地位、立場を意識している」、
\\	「あんなにお高くとまっている」の意味になります。	
\\	部長の都合が悪くなってしまったので、飲み会の日程は仕切直しだね。 
\\	「仕切直し」とは
\\	「飲み会をまた設定し直す」という意味ですね。	
\\	昨夜家の近所にパトカーがいっぱい集まってて警官がロープである一角を仕切ってて、とても不気味でした。 
\\	「…をロープで仕切る」は 
\\	で表現します。
\\	は「集まる」の意。	
\\	何かビジネスを始めるには、だいたいいくらくらいの資金が必要なんですか? 
\\	この場合の 
\\	は「資本金」の意味になります。
\\	は「資金を提供する」の意。	
\\	今後の旅の資金のために節約しなくてはと言われると、宿代払えとは言えませんでした。 
\\	この場合の「資金」は 
\\	、あるいは、
\\	(お金[資金]を貯める)で表現すればいいでしょう。	
\\	彼は芸人だからいつも馬鹿な事やっているけど、たまに見せる恥ずかしそうにする仕草がかわいいのよね。 
\\	には「(身のこなしや動作などの)癖」という意味がありますので、これで「仕草」が表現できるでしょう。
\\	「へまをする」。	
\\	あらやだ、このお煎餅もう湿気(しけ)ってる。 
\\	「湿気る」は
\\	「湿っぽい」あるいは
\\	「湿気のある」で表現します。「煎餅」は
\\	。これに
\\	を組み合わせれば英文のできあがり。では、「お煎餅そこに置いたら、すぐに湿気っちゃうよ」は何と言う?
\\	あの山掛けの問題、試験に出た? 
\\	「試験にでる」は
\\	でいいでしょう。
\\	とも言えます。	
\\	彼が毎日日記を付けているのに刺激されて、私も英語で日記を付けることにしました。 
\\	には「鼓舞する、その気にさせる」の意味がありますからこれを使えばいいでしょう。	
\\	確かに、プライベートレッスンもいいけど、グループレッスンでライバルがいたほうが、お互い刺激しあって向上も早くなるんじゃないのかな。 
\\	ここでは
\\	「刺激する」をそのまま使えばいいでしょう。
\\	「お互いに」。	
\\	私の歯は歯垢がたまりやすいのですが、どうにかたまらないようにする方法はないかしら。 
\\	「歯垢」は 
\\	と言います。
\\	は「ためる、蓄積する」の意。	
\\	次期社長の座をめぐり、どろどろした権力闘争のあげく、社内にはしこりが残った。 
\\	ここでは「しこり」を
\\	(悪い後味を残す)で表現しています。
\\	は「わだかまり、不和」、
\\	は「跡、通った道」の意。	
\\	それって死語じゃないの。 
\\	「死語」をここでは
\\	「もう言わなくなっている(使わない)」と言い換えています。	
\\	新入りをしごくのもほどほどにしろよ。 
\\	で
\\	を酷使する」の意味になります。	
\\	あなたの仕事は何時からですか、今日会いに行きます。 
\\	「仕事」は
\\	ですね。この場合には冠詞をつける必要はありません。	
\\	おしゃべりばっかしてないで、仕事しろ! 
\\	「仕事しろ」は
\\	「仕事に取りかかれ」でいいでしょう。
\\	「雑談する」、
\\	「無駄話をする」の意。	
\\	一日中働いてばかりいると、仕事中毒になっちゃうよ。ときには息抜きも必要だよ。 
\\	「仕事中毒」は 
\\	ですね。
\\	は「…の中毒になる」、
\\	は「一休みする」の意。	
\\	どうでもいい電話ばっかりかかってきて、仕事にならない! 
\\	「仕事にならない」は
\\	でいいでしょう。ここでは
\\	の使い方にも注目してください。	
\\	彼は典型的な仕事人間だね。 
\\	「仕事人間」は
\\	「仕事中毒」で表現すればいいでしょう。
\\	「典型的な」、
\\	「典型」の意。	
\\	彼は相変わらず仕事の虫だね。今夜も残業かい。 
\\	「仕事の虫」は 
\\	(仕事中毒の人)で表現すればいいでしょう。	
\\	すっごいムカツクやつがオフィスにいるんだよ。もう、超ムカツク。でも、仕事はできるから、上司は彼のこと買ってるんだよね。 
\\	「仕事はできる」は 
\\	(仕事のやり方を知っている)ということですね。	
\\	成功の要素は、個人の資質と能力とそして運だ。 
\\	が「個人の資質」に当たる表現です。	
\\	デパートの試食コーナーへ行くと、いろいろな物がただで食べられるよ。 
\\	「試食コーナー」は
\\	、または、
\\	でいいでしょう。	
\\	きみの場合、いくら勉強してもし過ぎるってことはないから。 
\\	で「勉強し過ぎることはない」の意味になります。	
\\	年を取るにつれてどうも涙もろくなったようです。先のことを考えるとどうしても沈んでしまうので何かやっていたいのです。 
\\	この場合の「沈む」は 
\\	(落ち込んでいる)で表現すればいいでしょう。
\\	は「涙もろい」の意。	
\\	昔は何かと不便であったが、今よりもずっと自然が豊かであった。 
\\	で「もっと多くの自然が残されていた」の意味になります。	
\\	アラスカでは、自然現象の「オーロラ」が見られるんですよね。 
\\	「自然現象」は
\\	(単数:
\\	と言います。	
\\	値段が上がれば自然と皆、煙草を買うのやめるんじゃないかな。 
\\	「自然と」は
\\	で表現できますね。	
\\	私が考えたこの文章、もうちょっと短くて自然な文章になりますか? 
\\	には「自発的な」の他に「(文体が)自然な」という意味があります。	
\\	なんだか知らないけど自然に覚えちゃったんだよね 
\\	は「自然に身につける」の意味。
\\	は、例えば、「ことばや複雑な技術
\\	などです。	
\\	休みの日は、体が必要としているだけの睡眠をとりたいので、目覚まし時計はセットしません。自然に目が覚めた時に起きるようにしています。 
\\	「自然に」は
\\	でいいでしょう。
\\	は「頼りにしている、任せている」の意。	
\\	田舎はたくさんの自然に囲まれているのはいいんだけど、刺激的な事があまりないからちょっと退屈。 
\\	「囲まれる」は
\\	で、「たくさんの自然」は
\\	で表現すればいいでしょう。
\\	「刺激的な」。	
\\	もっていると使ってしまいそうだから、先に払っておくよ。 
\\	「しそう」は
\\	で表現できます。
\\	「前もって」。	
\\	締め切りに間に合わなそうだったから、徹夜したんだ。 
\\	で「少しも 
\\	しそうではなかった」の意味。	
\\	これだけの大金を払った上で偽物かもしれないというリスクを背負うのはどうかと思うよ。 
\\	「~した上で」のニュアンスは 
\\	、あるいは、分詞構文
\\	で出すことができます。
\\	は「躊躇する、二の足を踏む」の意。	
\\	彼はおいしい物を食べ歩いているから、舌が肥えている。 
\\	で「洗練された味覚を持っている」の意味。
\\	「味覚」。	
\\	あいつ酔って、舌が回らなくなってるよ。 
\\	は「不明瞭に発音する」、
\\	は「不明瞭な」の意。	
\\	彼女はショックを受けて、何も話したがろうとしませんでした。落ち着くまでに時間がかかりそうだ。 
\\	「~したがらない」は
\\	で表現すればいいでしょう。
\\	「通常の状態に戻る」、
\\	「落ち着く」。	
\\	事故を起こしたくて起こしたんじゃないんだから、そんなに責めないで。 
\\	と 
\\	を組み合わせれば「…したくて~したんじゃない」が表現できます。	
\\	あいつの話なんかしたくもない。 
\\	で「…したくもない」の意味になります。	
\\	この前の旅行、親には鈴木さんと行ったことになってるんだ。本当は彼氏と行ったんだけどね。 
\\	「~したことになっている」のニュアンスを 
\\	を使って表現しています。
\\	(私の親に関する限り)を使ってもいいでしょう。	
\\	明日のパーティーに持って行く料理の下準備を今朝してたらバスに乗り遅れた。 
\\	ここでは「下準備をする」を 
\\	で表現しています。
\\	でも「下準備」の意味になります。	
\\	彼女は一見清純そうに見えるが、かなりしたたかな女だ。 
\\	には「したたか者」という意味があります。
\\	は「手強い」、
\\	は「世間知らず」の意味。	
\\	裕美は、彼に送る最後の手紙を認(したた)めていた。 
\\	「したためる」は簡単に
\\	で表現すればいい。	
\\	認
\\	彼の説明、舌足らずでよく分からなかった。 
\\	とは「不完全な、表面的な」の意味になります。	
\\	彼、下積み時代が長かったから、やっと花が咲いたって感じだね。 
\\	で「下っ端で多くの時間を過ごす」の意味になります。
\\	「子分、弟子、下っ端」、
\\	「花が咲く」。	
\\	その古いパソコンは下取りしてもらえば? 
\\	「下取りする」は
\\	と言います。	
\\	ここに来たばかりだというのに、暗いニュースが多くて残念です。 
\\	「…したばかり」は 
\\	を使って表現します。
\\	は「どんどん出てくる、フル回転で」の意。	
\\	今からレッスンの予約を変更すると、今回予約した分のレッスン料はキャンセル料として、次回払わないとだめですか? 
\\	が「予約した分のレッスン料」に当たる言い回しです。	
\\	彼にもわかるように、私達も英語で話した方がいいですね。 
\\	「した方がいい」は
\\	でも表現できますが、ここでは
\\	を使ってみました。	
\\	靴を履いたままで上がって下さい。 
\\	この場合の「~まま」は前置詞
\\	と
\\	を組み合わせて表現すればいいでしょう。	
\\	今回は来年の春の留学のための下見に2週間ほど行ってくるつもりです。 
\\	で「…の感触(雰囲気)をつかむ」の意味になります。
\\	は「…に慣れる」の意。	
\\	かなりの方向音痴なので、一度下見に行かないと、案内できないと思います。 
\\	「下見に行く」を 
\\	と 
\\	(…に慣れる)を組み合わせて表現しています。	
\\	アメリカの企業では、日本とは違ってアピールした者がちって感じがします。能力がある人はそれだけ自分の力を試すには絶好の場所かもしれない。 
\\	で「利益(仕事)は 
\\	した物に行く」を使って
\\	した者勝ち」を表現しています。	
\\	肩を手術したようですが、大丈夫? 
\\	のようですが」は
\\	だそうですが(聞いています)」でいいでしょう。	
\\	あいつのタイピングの速さを見たら、お前も舌を巻くよ。 
\\	「舌を巻く」とは
\\	「感動する、感銘を受ける」ということですね。	
\\	品物が着き次第御連絡差し上げます。(店で在庫切れで予約した際の定員の言葉) 
\\	「次第」は
\\	でもいいのですが、ここでは
\\	と
\\	を使って表現しています。	
\\	今年は娘の七五三だ。 
\\	「七五三」はそのまま
\\	と言います。	
\\	私は胆石がもとで七転八倒の苦しみを経験した。 
\\	で「激しい痛みでのたうち回る」という意味になります。
\\	は「胆石」、
\\	は「耐え難い」の意。	
\\	このプラダのバッグ、質屋で買ったの。半値以下だったわ。 
\\	「質屋」は
\\	と言います。
\\	「質屋(人)」。	
\\	現在あの火事の原因を、失火と放火の両面から捜査している。 
\\	この場合の「失火」は、英文の最後に 
\\	があるので 
\\	でいいでしょう。
\\	は「放火」の意。	
\\	料理は凝ったもんは作らないし、掃除も汚れが目立ってから仕方なくやる感じ。私って主婦失格ね。 
\\	は「ひどい主婦」、
\\	は「主婦とは名ばかり、悪い主婦の見本」という意味です。	
\\	彼は若いのになかなかしっかりしてるよ。 
\\	には「~の割には」の意味があります。
\\	は「(人柄が)真面目な、しっかりした」、
\\	は「成熟した、分別のある」の意。	
\\	しっかりしてよ。あなたが頼りなんだから。 
\\	この場合の
\\	は「しゃんとする、しっかりする」という意味です。
\\	でも同じような意味が表せます。	
\\	わかったような、わかってないような、なんかしっくりこないんですけど。 
\\	「しっくりこない」をここでは
\\	「何かが欠けている、何かが足りない」で表現してみました。	
\\	このせんべい、湿気てるよ。駄目だよ、ちゃんと封しなきゃ。 
\\	は「湿気のある」、
\\	は「ふやけた」、
\\	は「せんべい」の意味。	
\\	偉そうに、知ったようなこと言うな。 
\\	(自分でも何を話しているかわからない)で「知ったようなことを言うな」を表現しています。
\\	(専門家)の使い方にも注意。	
\\	彼の奥さんってうちらが知っている人? 
\\	「知ってる人」は
\\	で簡単に表現できます。	
\\	今となってはこのコースをとって失敗したかなと実は思っています。だって、こんなに大変だとは思わなかったのだもの。 
\\	で「しくじる、失敗する」の意味。
\\	を使ってもいいでしょう。	
\\	この間、初めて巻きずしを作ってみたのだけど、水の量が多かったみたいで、ご飯が柔らかすぎて失敗した。 
\\	で「…を台無しにする」という意味になります。
\\	は「柔らかな」の意。	
\\	あいつは部長に尻尾を振ってばかりいる。 
\\	で「とても熱心に…を喜ばせようとする」の意味になります。
\\	「忠実な」。	
\\	俺の親父、リストラされて、今、失業中なんだ。 
\\	「失業中」は 
\\	で表現できます。
\\	は「リストラされる」の意。	
\\	公の場であんなに無責任なことを言っておいて、失言ではすまされない。 
\\	で「つい口を滑らせること、失言」の意味。
\\	「無責任な」。	
\\	しつこいよ! 
\\	この場合の
\\	は「厄介者、困り者」という意味です。
\\	は「うるさい」の意。	
\\	私は失語語で言葉が喉まで出ているのですが、実際には言葉になりません。 
\\	で「一種の失語症に違いない」の意味になります。第二訳では「失語症」を使わずに、「言葉が出てこないとがしょっちゅうある」という風に英訳しました。	
\\	海外では、待ち合わせ時間より早く行くと失礼だと聞きましたが、本当ですか? 
\\	「失礼」は
\\	「無礼な」でいいでしょう。	
\\	失礼かと思いますが、ご返送いたします。 
\\	は「無礼」、
\\	は「ぶっきらぼうな」の意味です。	
\\	(部屋に入るときや退出するとき)失礼します。 
\\	「失礼します」の基本は
\\	ですね。部屋に入る時には、ノックをしてから、
\\	などと言います。出るときには、
\\	などを使えばいいでしょう。	
\\	そんな事知ってるよ。失礼な。 
\\	この場合の
\\	は「ばーか」の意味になります。もちろん、友達同士での会話という前提でこの表現を使っています。	
\\	こう長い間連絡取ってないと、なにかきっかけでもないと連絡しづらい。 
\\	「しずらい」は
\\	で表現できます。
\\	「突然」、
\\	「知らん振りをする」。	
\\	一度日本のおいしいお寿司をきみに食べさせてあげたい。 
\\	「~させてあげたい」は 
\\	で表現すればいいでしょう。	
\\	彼の部屋にはたくさんの食べカスが捨ててあった。 
\\	この場合の「してある」は 
\\	で表現すればいいでしょう。
\\	(散らかっている)の前には 
\\	が省略されています。	
\\	皆様に満足して頂けるよう、当スタッフ全員一丸となって頑張ります。 
\\	「していただけるよう」は
\\	「~の為に」、あるいは、
\\	「保証する」を使って表現すればいいでしょう。	
\\	私が帰ってくるまでに食器洗っておいてくれるかな。 
\\	この場合の「しておく」は
\\	目的語+過去分詞」で表現すればいいでしょう。	
\\	地震が起きてからでは手遅れだ。 だから日頃から対策を講じる必要がある。 
\\	(…するのは遅すぎる)と 
\\	を組み合わせて「…してからでは手遅れだ」を表現しています。
\\	と 
\\	を使ってもいいでしょう。	
\\	彼も好きで口が臭いわけではなく、きっと胃が悪いんだよ。だから、それを指摘するのはどうかと思うけど。 
\\	「指摘する」は
\\	で表現します。「どうかなと思う」の表現の仕方にも注目。	
\\	うちの上司、時間に厳しすぎるんだよね。もっと大目に見てくれてもいいのになぁ。 
\\	「(うちの上司)~してくれてもいいのになあ」のニュアンスは 
\\	で表現できます。
\\	は「時間に厳しい」の意。	
\\	どうして、もっと早く言ってくれなかったのですか?そうすれば対処できたのに。 
\\	「もっと早く... してくれなかったの」は
\\	と
\\	を使えば表現できます。	
\\	ねえ、帰ってくるついでにタバコを買ってきてくれる気はない? 
\\	で「…を買ってきてくれるかな」の意味になります。	
\\	語学の勉強って週1、2回では意味ないと思う。やはり毎日少しずつやってこそ上達するものだと思う。 
\\	で「…してこそ」を表現しています。
\\	「上達する、進歩する」。	
\\	最近、安心して食べ物が食べれなくなった。食べる前にちょっと臭いかいだり変なものが入ってないかついチェックしてしまう。 
\\	「ついしてしまう」のニュアンスを
\\	「…のようなことをする」で表現しています。	
\\	明日も仕事なんだから遅くまで起きてないで早く寝なさいよ。 
\\	「~しないで」は
\\	(…してほしくない)で表現すればいいでしょう。	
\\	ゲームばかりしていないで勉強しなさい。 
\\	で「…ばかりする」の意味になります。
\\	は「絶えず」の意。	
\\	人からお金借りてまで、買い物する事ないと思うけど。 
\\	してまで」は
\\	を使って表現できます。
\\	「借金する」。	
\\	なんでもないよ。ちょっと言ってみたかっただけ。 
\\	「ちょっと言ってみたかっただけ」を 
\\	で表現しています。	
\\	あなたの日本語の勉強にいいと思って、
\\	でやっている日本語講座を録画しておいたので、ぜひ見てみて! 
\\	は「(試しに)見てみる」、
\\	は「試してみる」の意味。	
\\	金曜日は午後から仕事で時間がなかったので、先生にレポートを添削してもらう時間がなかった。 
\\	この場合の「してもらう」は 
\\	(使役)を使えばいいでしょう。
\\	~の場合は「…に何とか頼んで~してもらう」というニュアンスになります。	
\\	昨日ね、彼に誕生日のプレテントを買ってもらったの。 
\\	「買ってもらった」は 
\\	で表現すればいいでしょう。	
\\	よくそんなことで、してやったりみたいな顔できるな。 
\\	(とてもうまくいった)で「してやったり」を表現しています。
\\	は「とても勝ち誇ったように振る舞う」の意。	
\\	私は彼のほえ面をかいている様を見て、してやったりとほくそ笑んだ。 
\\	「してやったり」を 
\\	(僕がしたことだ)、
\\	(僕がやった)で表現しています。
\\	は「目に涙をためた」、
\\	は「ほくそ笑む」の意。	
\\	何してるときが一番楽しい? 
\\	この場合の「~しているとき」は
\\	で表現できます。
\\	を使う場合には、2番目の英訳のように言えばいいでしょう。	
\\	うっ! 日本酒が効いてきた。視点があわなくてまっすぐ歩けない。 
\\	で「ぼやけて見える」の意味です。	
\\	何か世間を驚かすようなことをしでかしたい気分だ。 
\\	ここでは「しでかす」を 
\\	を使って表現しています。
\\	は「驚かす」の意。	
\\	彼は質問をされて、しどろもどろになった。 
\\	「しどろもどろ」とは
\\	「何を話しているか分からない」状態になることです。
\\	は「困らせる」の意。	
\\	あの文は最初が過去形で始まっているので、
\\	の後ろは過去完了形にしないといけないですか? 
\\	「~しないといけない」は 
\\	、あるいは 
\\	で表現すればいいでしょう。
\\	は「過去完了」の意。	
\\	雨の日に、車が跳ねる水しぶきを彼女が浴びないよう彼女と場所を交替し道路の内側を歩いてあげた。 
\\	と 
\\	と 
\\	の使い方に注目。	
\\	しょうがないじゃん、払わないわけにはいかなかったんだよ。 
\\	「しないわけにいかない」を 
\\	「…する以外に選択の余地はない」で表現しています。	
\\	先生の授業は難しすぎて単位取得を諦めましたので、最初から授業を登録しなかったことにしていただけませんか。 
\\	「授業を登録しなかった」を 
\\	で表現しています。
\\	は「抹消する」の意。	
\\	テレビ見ながら、勉強できるわけないでしょ。 
\\	「…しながら」は
\\	で表現できます。
\\	(同時に)の使い方にも注目。	
\\	この店、なかなか品揃えがいいね。食料品なら何でもあるね。 
\\	(品揃え、選ばれた品々)の使い方がポイントです。	
\\	死にそうにおかしかった。 
\\	で「笑って死にそうになる」の意味。	
\\	あまり遠くに住むと親の死に目に会えないかもしれないな。 
\\	「死に目」は
\\	ということですね。	
\\	その仕事は、死ぬか生きるかのものすごいプロジェクトだったんだ。 
\\	「死ぬか生きるか」はそのまま
\\	で表現できます。
\\	「やり遂げる」。	
\\	花粉症にかかり、市販の薬が効かないので耳鼻科で治療してもらった。 
\\	が「市販の薬(処方箋のいらない薬)」に当たる表現です。
\\	は「耳鼻咽喉科の医者」の意。	
\\	最近この人テレビで見ないと思っていたら、お芝居やっていたのね。 
\\	この場合の「芝居」は
\\	(舞台劇)でいいでしょう。	
\\	遅刻したのを、病気のせいにするなんてお前もお芝居がうまいね。 
\\	「芝居がうまい」は、そのまま
\\	を使えます。また、
\\	「うまい演義だった」とも言えます。
\\	は「~の振りをする」の意。	
\\	しばらく車を運転していないと、アクセルとブレーキがごっちゃになることない? 
\\	「しばらく」をここでは
\\	「長い間の休みの後で」、
\\	「久しぶりに」で表現しています。	
\\	ずっと正座してたからか、足がしびれちゃって。 
\\	「足がしびれる」は
\\	「寝てしまった」を使って表現します。
\\	が「正座する」に当たる言い方です。	
\\	そのスーツ渋いね。 
\\	「趣がある、趣味のよい」が「渋い」に近い形容詞でしょう。また、
\\	「おしゃれ、品のいい」を使っても表現できます。ちなみに、「かっこいい」は
\\	と言います。	
\\	無人駅であるのを良い事にかなりの人がキセルしていたらしく、たまたま来ていた駅員に、皆、運賃を渋々払っていた。 
\\	は「いやいや」の意味。
\\	「料金をごまかす」。	
\\	お腹の周りの余分な脂肪を取るいい方法はないものか。 
\\	で「余分な脂肪」の意味になります。
\\	は「ぜい肉」、
\\	は「胴」の意。	
\\	(超音波検査で)これは典型的な脂肪肝ですね。脂肪は超音波を吸収するので、脂肪があると黒く写るんです。 
\\	「脂肪肝」は
\\	「脂肪がたくさんついた肝臓」と言います。	
\\	すみません、レモンの絞り汁いただけますか? 
\\	「レモンの絞り汁」は 
\\	でいいでしょう。	
\\	シドニー、ケアンズ、メルボルン。一個所に絞るのは難しいな~。だって、それぞれいい雰囲気持っているし。 
\\	には「絞り込む」という意味があります。この場合、
\\	「選ぶ」を使ってもいいでしょう。	
\\	洗濯物たたんどいたから、しまってくれる? 
\\	「しまう」は 
\\	「片づける」で表現すればいいでしょう。
\\	は「洗濯物」の意。	
\\	う、しまった!今まで平気だったズボンがきつい。 
\\	この場合の「しまった」は、
\\	!でいいでしょう。
\\	「ぴったり合う」。	
\\	あの、これシミが付いてるんですけど、ほかに在庫ありますか? 
\\	「シミ」は 
\\	といいます。
\\	は「在庫がある」の意。	
\\	昨日の夜は、親友と酒を飲みながら、しみじみと人生について語り合ったよ。 
\\	は「心からの、腹を割った」の意味。
\\	「有意義な」。	
\\	そんなしみったれたことを言うなよ。 
\\	とは「けちん坊、しみったれ」のことです。
\\	も
\\	も同じような意味です。	
\\	体に染み付いた癖は、なかなか直せない。 
\\	「体に染み付く」は
\\	「(習慣など)深くしみ込ませる」を使えばうまく表現できます。
\\	でもいいですね。
\\	「不可能に近い」。	
\\	虫歯ができちゃって、冷たい物を食べるとしみるんだ。 
\\	「しみる」は
\\	「敏感な」を使って表現します。
\\	両方を出している点にも注意。	
\\	僕のアルバイトの給料は、毎月20日締めです。 
\\	第一訳の意味は「僕のアルバイトの給料は、毎月20日まで働いた期間に基づいて計算されます」となります。	
\\	締め切りに間に合わせるために、3日徹夜した。 
\\	「締め切り」は
\\	と言います。	
\\	祖母は湿っぽいのが嫌いな人だったので、告別式では明るく送ってやりたいと思います。 
\\	は「湿っぽさ」、
\\	は「陰気」、
\\	は「快活な、生き生きとした」の意味になります。	
\\	このカードの締め日っていつ? 
\\	が「締め日」に当たる言い回しです。	
\\	下ネタ言いにくい雰囲気のサークルってあるよね。 
\\	「下ネタ」は「汚い冗談」ということで
\\	でいいでしょう。また、
\\	も使えます。
\\	は「受け入れられる」の意味。	
\\	今回の2000年問題の対応で、年を下2桁で扱っている個所を4桁で扱うように修正しました。 
\\	「下二桁」は
\\	と言います。
\\	「対応する」、
\\	「扱う」。	
\\	霜降り肉って日本人しか好まないのかな? 
\\	「霜降り肉」は 
\\	と言います。
\\	は「脂肪」の意。	
\\	モー娘とかハリポタとか、一種の社会現象だよね。 
\\	「社会現象」は 
\\	ですね。
\\	は「それ自体で」の意。	
\\	他人の論文を丸写ししようものなら、社会的に抹殺されてしまうよ。 
\\	で
\\	を社会的に追放する」の意味になります。
\\	「遠ざける」。	
\\	社会に出たら、自分の意見は言うようにした方がいいよ。でないと、自分が損するよ。 
\\	「社会に出る」は
\\	、あるいは、
\\	と言えばいいでしょう。
\\	「思ったことをはっきり言う」。	
\\	急にお腹が痛くなって、駅のホームでしゃがみ込んでしまった。 
\\	が「しゃがみ込む」にあたる表現です。
\\	「(苦痛で)体を折り曲げる」、
\\	「しゃがむ」。	
\\	彼の言い分は聞いたが、どうも釈然としないところがある。 
\\	「不満が残る」、
\\	「説得力がない」で「釈然としない」を言い換えて表現しています。	
\\	社交辞令でないならうれしいけどね。 
\\	は「お世辞がうまい、外交上の」という意味。	
\\	写真写り悪いな~。写真と実物で随分違う。 
\\	主語を人と物を使って、2通りに英訳しました。
\\	が「写真写りがいい」に当たる言い回しです。自動詞
\\	と
\\	を組み合わせても同じことが言えます。	
\\	彼の斜に構えた態度はどうも可愛いげがないね。 
\\	「斜に構えた態度」を
\\	「皮肉っぽい態度、物言い」、
\\	(を疑いの目で)横目に見る」を使って表現しています。	
\\	以前は、勤続30年で社名と本人の名入りの金時計がもらえたそうですよ。 
\\	が彫られている」を使えば、
\\	入りの」が表現できます。	
\\	こんな長い時間並んで、目の前で売り切れたら、シャレにならないなあ。 
\\	ここでは「シャレにならない」を
\\	「面白くも何ともない」でそのニュアンスを表現しています。	
\\	コンタクトレンズにしたら、視野が広くなったみたいで快適だよ。 
\\	「コンタクトレンズ」は
\\	と複数形にします。「視野、視界」は
\\	あるいは
\\	、「広がる、広げる」は
\\	両方とも使えます。「快適」は
\\	や
\\	でいいでしょう。	
\\	視野を広げるためにも、アメリカに留学するのもいいんじゃないかな。 
\\	で「物事に対する見方を広げる」の意味になります。	
\\	この店の集客力アップのため、駅でパンフレットを配った。 
\\	には「客を引きつける物」という意味があります。	
\\	今時、週休2日制じゃない会社なんて珍しいよね。 
\\	が「週休2日制」に当たる表現です。
\\	ともいえます。	
\\	週休3日だったらいいのにな。 
\\	「週休三日」とは
\\	「週末が三日ある」と言い換えることができます。	
\\	月別の集計表を出してください。 
\\	この場合の「集計」は
\\	で表現すればいいでしょう。
\\	は「月ごとの(形容詞)」の意。	
\\	脱いだら脱ぎっぱなし、使ったら使いっぱなしだから、収拾がつかなくなる。 
\\	で「まったく手がつけられない、収拾がつかない」の意味になります。
\\	「めちゃくちゃ(な状態)」。	
\\	4年生になってから就職活動をするなんて、遅すぎるよ。 
\\	「就職活動」は
\\	でいいでしょう。
\\	「4年生の」。	
\\	この不景気ではなかなかいい就職口はないから、しばらくこの会社で働くしかない。 
\\	「就職口」は
\\	といいます。
\\	は「少ない、まれ」の意。	
\\	就職難だし、これからの事を考えると、息が詰まりそうだよ。 
\\	「就職難」とは
\\	「仕事を見つけるのがとても大変だ」ということですね。
\\	「息の飲む、息が止まる」、
\\	「待ちかまえている」、
\\	「胸がつかえる」、
\\	「容赦のない」。	
\\	クラスを習熟度別にして英語の授業を行う大学が増えてきているようだ。 
\\	で「学生がどれくらい進んでいるかにしたがって」の意味になります。
\\	「習熟度」。	
\\	あっ、書き間違えちゃった。修正液はどこだっけな。 
\\	「修正液」は
\\	(白の修正液)、あるいは、
\\	と言います。	
\\	何でそんなにあの男に執着するの? 
\\	で「…に惚れる」、
\\	は「…に執着する」の意。	
\\	私の英語のヒアリングは集中して聞かないとまだ理解できないレベルです。ちょっと気を抜くとあれ?あれ?って感じで 
\\	には「注意して」という意味があります。「気を抜く」の表現の仕方にも注目。	
\\	なんだか久々に集中して勉強して、頭を使ったという感じがする。 
\\	で「集中して勉強する」の意味になります。
\\	は「集中的な」の意。	
\\	歳のせいか、集中力がなくなった。 
\\	「集中力」は動詞
\\	、あるいは、名詞
\\	を使って表現できます	
\\	あいつは執念深いから気をつけた方がいいよ。 
\\	「執念深い」は
\\	「恨みを抱く」で表現できます。
\\	を使っても同じ様なことが言えますね。	
\\	会議でのきみの主張はあまりに主観的過ぎたから、みんなが耳を貸さなかったんだよ。 
\\	「主観的な(に)」は
\\	で表します。
\\	に耳を貸す」。	
\\	3から
\\	4に縮小コピーすれば、1ページに入りきると思うけど。 
\\	「縮小」は
\\	、「拡大」は
\\	と言います。
\\	の使い方にも注意。	
\\	付加疑問
\\	を使う時は、主語に合わせないといけないのですね。 
\\	に合わせる」は
\\	「一致する、調和する」で表現できます。
\\	に付加疑問の例文を挙げておきました。	
\\	皆、あれは故障だっていっているけど、主催者側はそういうものなんだって言いはっているらしいよ。 
\\	「主催者側」は
\\	でいいでしょう。
\\	を使っても文脈からわかるでしょう。	
\\	雑誌の特集のため、貴社の社長の1週間を取材させて頂きたいのですが。 
\\	で「…について一週間取材をする」という意味になります。
\\	は「特集」の意。	
\\	編集者が趣旨を変えない程度に、投稿文を修正することもあります。 
\\	「趣旨」は 
\\	(趣旨、意味)で表現します。
\\	は「原稿」の意。	
\\	今月半ばから出向で日本橋勤務なの。でも1カ月だけどね。 
\\	「出向して働く」は
\\	と言います。
\\	は動詞としても使えます。	
\\	ご出産おめでとうございます。(カード、手紙での表現) 
\\	「出産」は
\\	を使って表現します。すこしかしこまった言い方ですが、
\\	でもいいでしょう。	
\\	母の介護のため田舎に戻るんだけど、そこの支店にも顔出すから交通費が出張扱いにならないかな。 
\\	で「出張として処理する」の意味になります。
\\	は「交通費」の意。	
\\	私たちの秋のイベントの飲食コーナーに、出店してもらえませんか。 
\\	の文字通りの意味は「屋台を経営する」です。
\\	は「売店」の意。	
\\	本日学習したところまで、(試験の)出題範囲とします。 
\\	「出題範囲とする」は 
\\	で表現すればいいでしょう。	
\\	"最近、主夫と呼ばれる男の人が増えているようだ。外に働きに出ず、炊事、洗濯を奥さんの代わりに行う。本当にそれでいいのかどうか。 
\\	投稿された日本語がかなり長いので、とりあえずその意味になるように英語にしてみました。「主夫」を
\\	""と説明しています。
\\	彼女は10代で結婚しましたが、ちゃんと主婦をやっています。 
\\	「ちゃんと主婦をやっている」は
\\	で表現できます。(注)日本語を一部書き換えました。	
\\	何だそのネクタイ。趣味が悪いね。 
\\	で
\\	の趣味が悪い」という意味になります。	
\\	陸上のこの種目では1位取ったことあるよ。 
\\	この場合の種目は
\\	「種目、行事」でいいでしょう。	
\\	彼は陸上部に入っている。種目は幅跳びと3段飛び、それに円盤投げだ。 
\\	ここでは「種目」を 
\\	で表現しています。「入っている」を前置詞 
\\	で表している点に注意。	
\\	あの政治家は修羅場をくぐってきたから打たれ強い。 
\\	で「修羅場をくぐる」を表現しています。
\\	は「戦いを通じて鍛えられた」の意。	
\\	彼女には酒乱の気がある。 
\\	「酒乱」を
\\	と
\\	を組み合わせて表現しています。	
\\	最近の携帯電話は折り畳み式が主流になっている。 
\\	ここでは「主流」を 
\\	(ほとんど)を使って表現しています。
\\	は「主な、支配的な」の意。	
\\	このお店はいつも旬のメニューを出してくれて、しかも飾り付けがきれいなの。 
\\	「旬」は 
\\	で表現すればいいでしょう。
\\	は形容詞。	
\\	ホームランを打つには、腕力よりも瞬発力が必要だ。 
\\	「瞬発力」とは
\\	「瞬間的な力」ということですね。
\\	「一瞬で体を動かす」。
\\	「腕力」。	
\\	資本主義の根本は、私有財産制を認めるということである。 
\\	が「私有財産性」に当たる表現です。	
\\	今夜はごちそうだって。おなか空かせるために、公園でも歩いて消化させようよ。 
\\	「消化する」は
\\	ですね。ここでは
\\	「食欲を増進させる」も一緒に覚えておきましょう。	
\\	ビザを斡旋してくれた人がいい不動産屋も紹介してくれるそうだ。 
\\	「紹介する」は
\\	ですね。
\\	~で「…を~に差し向ける、紹介する」の意。	
\\	2年前にこちらに来たとき1週間学校に行ったのですが、そのとき学校が紹介してくれたホストファミリーが彼らだったのです。 
\\	「紹介する」は 
\\	ですね。この文脈では 
\\	(用意する)も使えます。	
\\	高橋課長、昇格して部長になったんだって。 
\\	で
\\	に昇格される」の意味になります。
\\	でも同じ様なことが言えます。	
\\	3月は、公共事業費の予算を消化するためだけの工事がやたらと目につく。 
\\	「消化する」を
\\	(使い切る)で表現しています。
\\	は 
\\	の意味で使われています。
\\	は「正当化、理由」、
\\	は「割り当て」の意味。	
\\	あの人は目と耳に障害があるんだよ。 
\\	「障害」は
\\	を使って表現します。
\\	するのに問題がある」。	
\\	英語ってホント生涯学習だと思います。いくら勉強しても終わりが見えてこない。 
\\	ここでは「生涯学習」を 
\\	で表現しています。
\\	は「使い尽くす」の意。	
\\	障害者に対する配慮に欠ける人がかなりいる。 
\\	で「身体障害者の人たち」という意味になります。
\\	の使い方にも注意。	
\\	今日は、もうだるくてしょうがない。 
\\	ここでは
\\	過ぎる」で「しょうがない」のニュアンスを出しています。	
\\	(試験前に全然勉強してない人 に対して)そんなんでどうするの?これじゃ試験受からなくてもしょうがないな。 
\\	「どうやっても 
\\	は無理」で「しょうがない」を表現しています。
\\	「態度、心構え」。	
\\	特に応援したい候補者がいないので、消去法で、一番欠点の少ない人に投票した。 
\\	は「除く、除去する」の意味。
\\	「消去法」。	
\\	ゴルフで正しい位置を打つとボールの衝撃は全然受けないが、変な位置を打つとかなりの衝撃が手に来て、寒いときは余計痛い。 
\\	が「衝撃を感じる」に当たる表現です。
\\	は「とんでもなく」の意。	
\\	あまりにも衝撃的過ぎて何も言葉はありません。現実にあったことだと受け止めたくない。 
\\	で「言葉にするにはあまりにすさまじい」の意味。
\\	の代わりに 
\\	を使ってもいいでしょう。	
\\	彼は性懲りもなく、また別の女性にプロポーズした。 
\\	「性懲りもなく」をここでは
\\	「決して学ばない」で表現しています。	
\\	いつパーティーの時間等の詳細は掲示板に張り出されますか? 
\\	「詳細」は
\\	ですね。動詞
\\	の使い方にも注目。	
\\	正直言って、想像していたのと全然違ってつまらなかった。 
\\	が「正直に言って」に当たる表現です。
\\	を使ってもいいでしょう。	
\\	去年から花粉症の症状が急にでてきました。 
\\	「症状」は
\\	ですね。
\\	「花粉症」も一緒に覚えておきましょう。	
\\	12月に入り、クリスマス商戦が本格化してきている。 
\\	ここでは「クリスマス商戦」を
\\	で表現しています。	
\\	彼は今月大きな商談を3つもまとめた。 
\\	「商談をまとめる」は 
\\	で表現すればいいでしょう。
\\	の代わりに 
\\	も使えます。	
\\	今日は衝動買いしないように気を付けなくちゃ。 
\\	「衝動で」は
\\	と言います。これと
\\	を組み合わせればいいでしょう。	
\\	営業は僕の性に合わない。 
\\	で
\\	に向かない」の意味。
\\	「人材、人物」。	
\\	最近は雨ばかり降っているから、商売あがったりだ。 
\\	で「商売あがったりだ」を表現しています。
\\	は「沈み込んでいる」の意。	
\\	このマラソンは、最後の5キロが勝負だな。 
\\	(決定的な、重大な)、
\\	(決める、決定する)の使い方がポイントですね。	
\\	今日は人生が決まる日だから勝負服を着ていこう。 
\\	「勝負服を着る」を 
\\	([服装を]完璧に決めていく)で表現しています。
\\	は「注意深く、念入りに」の意。	
\\	このソーセージ、とっくに賞味期限が切れてるじゃない! 
\\	「賞味期限」は
\\	と言います。
\\	「美味しく食べられる期限」も使えます。	
\\	賞味期限が昨日までだったので、無理して大福を3つたいらげた。まだ、3つ残っている。 
\\	が「賞味期限」に当たる表現です。	
\\	結局、なんだかんだいっても自分の英語の力を証明するのは、
\\	のスコアしかないんだよね。 
\\	この場合の「証明」は
\\	(証拠)、
\\	(証明)で表現すればいいでしょう。	
\\	証明写真っていっつも目をつぶっていたり失敗が多いんだよね。 
\\	「証明写真」は
\\	ですね。
\\	「まばたきする」、
\\	「なんやかや」、
\\	「失敗する」。	
\\	今回このプロジェクトに参加できたことは、私にとって将来の糧になります。 
\\	「将来の糧」を
\\	「将来にとって重要な経験」で表現しています。	
\\	それでは乾杯をいたしますので、皆さんご唱和ください。 
\\	で「みんなで言いましょう」の意味になります。
\\	は「乾杯する」の意。	
\\	食あたりでお腹が痛くて、2日間会社を休んだ。 
\\	「食あたり」は
\\	「食中毒」でいいでしょう。	
\\	歳をとるとともに、食が細くなるものだな。 
\\	ここでは「食が細くなる」を
\\	「食べる量が少なくなる(少なく食べる人)」で表現しています。	
\\	食事以外の時にもワインを飲むことが多くなってきている。 
\\	「食事以外の時」を文字通り
\\	と英訳しています。	
\\	そんなこと言うんだったら、今日は食事抜きよ。 
\\	「食事抜き」は簡単に
\\	で表現できます。	
\\	牛乳を飲んで食中毒になった人の数が1万人を越えた。 
\\	「食中毒」は
\\	と言います。
\\	(の病気)にかかる」。	
\\	彼女は職場の花だよね。 
\\	「花」を 
\\	(飾り)で表現しています。
\\	は「実際に仕事をするというより飾り的存在である」の意。	
\\	給料日までまだまだだ。今月残りどうやって暮らそう。やっぱり、うちは食費がかかりすぎだな。 
\\	で「食料品にお金を使い過ぎる」の意味。名詞で表現したければ
\\	が使えます。	
\\	将来を嘱望されていたのに、彼はなぜ会社を辞めたんだろう。 
\\	で
\\	に大いに期待してる」の意味になります。	
\\	少し運動したらいいよ。食欲が出て、何でもおいしく食べられるから。 
\\	「食欲が出る」は 
\\	と 
\\	(向上する)を組み合わせて表現します。
\\	は「刺激する」の意。	
\\	海外生活初心者にとっては、カナダはいい国だと思います。 
\\	「初心者」を
\\	「初めての経験」、
\\	「初めて」で表現しています。	
\\	ワニの肉の食感はどうも好きになれない。普段食べなれないものだから、どうも食べる時に動物が思い浮かんじゃって。 
\\	ここでは「食感」を
\\	を食べた時の何か」で表現してみました。
\\	「心に浮かぶ」。	
\\	今日、すごくショックな事がありました。すぐには立ち直れないかもしれません。 
\\	で「ショックな事」の意味。形容詞 
\\	の位置に注意。	
\\	何をそんなしょぼくれた顔してるんだよ。元気出せよ。 
\\	は「落ち込んでいる、意気消沈している」の意味。
\\	「浮かない顔」。	
\\	寝不足で、目がしょぼしょぼする。 
\\	とは「目がかすむ、ぼやける」ことです。
\\	とも言えますね。	
\\	一度、パソコンを強制終了させたいのですが、何か処理途中のものありますか? それらは保存されないで消えてしまうけどいいですか? 
\\	この場合の「処理」は
\\	(動詞)を使って表現できます。「強制終了」に当たる言い回しにも注目。	
\\	彼、あなたがいない事がわかると、なんかしょんぼりして帰っていったよ。 
\\	は「しょげた様子で」の意味。
\\	「ふさぎ込んだ」。	
\\	仕様の変更が多くて、ちょっと作っては修正、ちょっと作っては修正で全然進まない。すごく効率が悪い。 
\\	「仕様」は
\\	、あるいは短くして
\\	といいます。日本語に合わせて、
\\	と
\\	を繰り返しています。	
\\	ちょっとワインでも飲んじゃおうかな。いや、明日も仕事だし、ここはぐっと我慢だ。 
\\	で「…しちゃおうかな」のニュアンスが出せます。
\\	は「意志の力」の意。	
\\	もう、まったく。しょうがないな(人から何か依頼され、しぶしぶする時)。 
\\	は「選択の余地がない」、
\\	は「逃げられないからな」の意味になります。	
\\	この「使用前・使用後」ってほんとかなあ。なんか嘘っぽいんだよね。 
\\	「使用前・使用後」は
\\	と
\\	を使って表現します。
\\	は「あやしい、うさんくさい」の意。	
\\	私が彼をまたごうとしたとき、すごく感じ悪い口調で「おい!」って言ったんだよ。 
\\	「~しようとする」は 
\\	で表現します。
\\	は「今にも…しようとする」の意味。	
\\	彼は警察から車を修理するように言われたが直そうとしなかった。 
\\	で「(警察の言うことに)従おうともしなかった」の意味になります。
\\	は「努力する」の意。	
\\	リスニングの問題で、質問文の内容を理解しようとしている間に次の質問に移っているので聞き逃してしまうことがよくある。 
\\	で「まだ…しようとしている」の意味になります。
\\	は「…を聞き逃す」の意。	
\\	インターネットが1時間1000円というのは、30分使っても1000円かかるんですか? 
\\	で「インターネット接続料金」という意味になります。
\\	「(料金を)請求する」。	
\\	最近、白髪が増えてきた。 
\\	(動詞)には「白髪になる」という意味があります。	
\\	飲み会の席で議論なんか始めるから、みんな白けちゃったよ。 
\\	「どっちらけ」に当たる表現と言えば、
\\	です。火に濡れた毛布は掛けると消えてしまうでしょう!「飲み会」は、
\\	でもいいのですが、
\\	という言い方も一緒に覚えておきましょう。	
\\	よくもそんな白々しい言い訳ができるもんだね。 
\\	で「まずい言い訳」の意味。「よくも」の表現の仕方にも注目
\\	もう起きてよ。何度も同じ事を言わせないで。寝ててもいいけどさ、遅刻しても知らないからね。 
\\	「知らないからね」のニュアンスを
\\	(私のせいにしないで)を使って表現しています。	
\\	知らない内に癖ってでちゃうもんだね。気をつけなきゃ。 
\\	は「ほとんど思ってもみない時に」の意味になります。カッコの中には、具体的な癖、「拳を鳴らす
\\	や「足でトントン音を立てる
\\	を入れておきました。	
\\	こんなこと素面(しらふ)じゃ言えないよ。 
\\	言えないだろうな」は仮定法を使って
\\	で表現します。「素面」は
\\	と言います。この場合の
\\	は副詞
\\	として使われています。	
\\	自分で調べてわかるような質問はしないで、後で自分で調べて欲しい。 
\\	この場合の「調べる」は
\\	を使えばいいでしょう。	
\\	日本って、だれかが困ってても知らんぷりする人が多いよね。 
\\	で「気づかないふりをする」の意味になります。
\\	は「無視する」の意。	
\\	彼女は尻が軽い。 
\\	には「だらしのない、尻軽な」という意味があります。	
\\	彼らの企画、尻すばみにならないといいけど。 
\\	「尻すぼみ」にぴったりなのが
\\	です。これは「ポシャる」(「ポシャる」の項参照)の意味にも使えます。また、
\\	「段々小さくなる」も同じような意味を表します。	
\\	あいつは相変わらず女房の尻に敷かれている。 
\\	が「尻に敷かれている」に当たる表現です。
\\	は「…を短いロープにつないでおく」の意。	
\\	これはおまえがやったことだろう。何でおれがおまえの尻ぬぐいをしなくちゃならないんだ。 
\\	ここでは「尻ぬぐい」を 
\\	(…を引き受ける)、
\\	(片づける、整理する)で表現しています。	
\\	あいつの説明は支離滅裂で何を言っているか分からない。 
\\	「不明瞭な」、
\\	「ばらばらの、支離滅裂な」、
\\	「首尾一貫していない」、
\\	「ごちゃごちゃの」を組み合わせて「支離滅裂」を表現しています。	
\\	年を取るに連れ、視力って落ちるもんだね。小さい字が読めなくなってきた。 
\\	「視力」は
\\	です。
\\	「歳を取る」、
\\	「細かい文字」。	
\\	その画家は、知る人ぞ知るという存在だ。 
\\	で「仲間内では知られている(人気がある)」の意味。
\\	「事情に通じている人たち」。	
\\	自分がアジア圏に住んでいるかもしれないけど、あまりアジアには興味がないし、行きたいと思わない。 
\\	「知れないけど」のニュアンスを
\\	で出しています。もちろん、助動詞
\\	も使えます。	
\\	日本ではゲイに関しては、まだ素直に受け入れられてないので、白い目で見られます。 
\\	(軽蔑の目で見られる)で「白い目で見られる」を表現しています。
\\	は「軽蔑、軽視」の意。	
\\	こんな海のものとも山のものとも分からない代物に、社運を賭けられるか! 
\\	第1文では、「代物」を文脈から判断して
\\	「取り引き」で表現しています。第2文では、
\\	「物」を使っています。
\\	「賭ける」。	
\\	こんないたずらをするなんて、あいつの仕業だな。 
\\	が「仕業」に当たる言い方です。
\\	は(彼)しかいない」の意味。	
\\	このワイシャツ、しわしわになってるから、アイロンかけてくれるかな。 
\\	で「しわしわになっている(状態)」を表します。	
\\	経営者の失策は、結局われわれ従業員にしわ寄せがきて、最後に尻拭いをしてやることになる。 
\\	ここでは「しわ寄せ」を
\\	「責任を押しつける」、
\\	に向ける、追いやる」で表現しています。
\\	「損害を被る」。	
\\	彼女の死を悼む声がその掲示板に寄せられた。 
\\	は「哀悼」、
\\	は「慰め」の意味になります。
\\	「掲載する」。	
\\	シャーペンの芯がなくなっちゃったんだけど、何本かもらえるかな。 
\\	「芯」は
\\	(元々は「鉛」の意)と言います。「シャープペンシル」は和製英語です。英語では
\\	と言います。	
\\	白菜の葉っぱとしんの部分どっちが好き? 
\\	「しん」は 
\\	といいます。
\\	は「白菜」の意。	
\\	彼の真意がはかりかねる。 
\\	と 
\\	(目的、動機)で「真意」を表現しています。
\\	は「推測する、見抜く」の意。	
\\	混乱を収拾するために、遂にあいつが乗り出してきたか。真打登場というわけだ。 
\\	ここでは「真打ち」を
\\	「何をなすべきかよく分かっている(人物)」で表現してみました。
\\	「大混乱」。	
\\	改革の本当の真価は、これから現れてくる。 
\\	「真価」は 
\\	で表現できます。
\\	は「明らかな」、
\\	は「はっきり現れる」の意。	
\\	転職も引っ越しもして、心機一転、新たなスタートを切った。 
\\	「心機一転」は、
\\	「(過去のことを忘れて)新たな気持ちで始める」を使えばうまく表現できます。	
\\	寺巡りに盆栽----若い身空でずいぶん辛気くさい趣味だこと。 
\\	「辛気くさい」を
\\	「つまらない」、
\\	「退屈な」で表現しています。	
\\	彼はシングルだから、とてもかなわないよ。 
\\	この場合の「シングル(ゴルフのハンデ)」は
\\	でいいでしょう。「かなわない」の言い回しも一緒に覚えておくといいでしょう。	
\\	ここ禁煙なのに、平気でタバコすっているよ! どういう神経しているんだか。 
\\	(何を考えてるんだよ)で「どういう神経をしているんだか」を表現しています。	
\\	私神経質だからちょっとでもシーツにシワがあるといやなの。 
\\	「神経質」は 
\\	で表現すればいいでしょう。
\\	は「神経過敏な」、
\\	は「しわ」の意。	
\\	あの人としゃべると神経使うよね~。 
\\	「神経を使う」は
\\	「色々と考える(気を使う)」で表現すればいいでしょう。	
\\	人を散々こき使っておきながら礼の言葉一つないんだよ。神経疑っちゃうよね。 
\\	「神経を疑う」とは 
\\	(とても無神経だ、思いやりがない)ということですね。	
\\	真剣な顔つきで相談があるっていってたけど、なんだろう? 
\\	「真剣な顔つきで」は
\\	で表現できます。
\\	「真面目な」。	
\\	緊張を和らげるために、深呼吸をするといいよ。 
\\	「深呼吸」は 
\\	といいます。
\\	は「深呼吸をすること」の意味。	
\\	あいつは新婚だから、飲みに誘っても来ないよ。 
\\	「新婚」は
\\	「最近結婚したばかり」ということですね。	
\\	クレジットカードやローンの申し込みって審査があるんでしょ?収入がないのに通るかな? 
\\	この場合の「審査」は 
\\	(…を調べる)、
\\	(…を調査する)で表現すればいいでしょう。	
\\	その犯人は、心神喪失状態の犯行と見せかける用意周到さとしたたかさを持っていた。 
\\	の文字通りの意味は「犯罪的に精神異常である」です。
\\	は「心神喪失状態にある」、
\\	は「極めて慎重であること」の意味。	
\\	信じていたのに、どうして言ったとおりに仕事をしないの? 
\\	ここでは 
\\	(…を頼りにする)を用いて「信じる」を表現しています。	
\\	日本人にとって、土足で家に上がることは便器で顔を洗うのと同じくらい信じられないことだ。 
\\	ここでは「信じられないこと」を 
\\	(とても考えられない)で表現しています。
\\	は「信じられない」の意。	
\\	こちらに来て、最初の3カ月は何もかも新鮮かついろいろ不安もあったけど、ようやく慣れてきました。 
\\	「新鮮」は 
\\	で表現すればいいでしょう。「慣れる」の表現の仕方にも注目。	
\\	その事件の真相は一部の人以外には知らされなかった。 
\\	「真相」は
\\	「真実」で表現すればいいでしょう。ここでの
\\	(選ばれた)は形容詞。	
\\	暗闇から突然人が出てきたので、びっくりして心臓が止まりそうになった。 
\\	「(心臓が)止まりそうになる」は 
\\	を用いて表現します。
\\	は「心臓発作(麻痺)」の意。	
\\	このシーンはもうドキドキもので心臓に悪いよ。 
\\	ここでは「心臓に悪い」を
\\	「心臓麻痺を起こす」と言い換えて表現しています。「ドキドキ」は
\\	や
\\	を使ってもいいでしょう。	
\\	ここのところ森首相の進退問題が、新聞やテレビで頻繁に取り上げられている。 
\\	で「辞任を余儀なくさせられる」の意味になります。
\\	は「報道」、
\\	は「窮地に追いやる」の意味。	
\\	彼は疲れたのか、死んだようにぐっすり寝込んでいたので、全然朝の皆の声には気づかなかった 
\\	が「死んだように眠る」に当たる表現です。
\\	は「ぐっすり眠る」の意。	
\\	今日弟の家の新築祝いをします。 
\\	が「新築祝い(の宴)」に当たる表現です。	
\\	家を購入する時って、慎重にならなかった?だって、そこに一生住むんだよ。賃貸の方が気軽だと思うけど。 
\\	は「慎重な、注意深い」、
\\	は「用心深い」の意味です。	
\\	このスーツ新調したんだ。似合うかな? 
\\	で「…を仕立ててもらう」の意味になります。	
\\	もう少し身長があったらな。 
\\	「身長」は
\\	ですね。ちなみに、「体重」は
\\	です。	
\\	私は、身長がとても小さいですがいやだと感じたことは、ありません。 
\\	「身長が小さい」は
\\	を使って表現します。	
\\	最近メガネの度が合わないと思っていたら案の定視力が下がってたよ。だから、メガネを新調したんだ。 
\\	「メガネを新調する」は
\\	と言えばいいでしょう。	
\\	競争原理を、大学教員間にも浸透させなきゃいかん。 
\\	は「しみ通る、広がる」、
\\	は「…が吹き込む、染み込む」の意味。	
\\	子供の頃は神童と言われた彼が、今は、ただの人か。 
\\	が「神童」に当たる表現です。
\\	「普通の」。	
\\	うちの課長って、急に風邪ひいて休むって連絡したら、体の事より先ずは仕事の事を心配するのよね。信じられないよね。 
\\	この場合の「心配する」は、
\\	で表現すればいいでしょう。
\\	「電話をかけて病気で欠席すると伝える」。	
\\	アメリカ留学の前にいろいろ現地では起きてて、周りの人はかなり心配している。 
\\	ここでは「心配」を 
\\	(案じて)、
\\	を用いて表現しています。	
\\	私は、もう大人なのにうちの両親は心配性で嫌になる。 
\\	で
\\	のことで思い悩む」の意味になります。	
\\	人間辛抱だ。 
\\	は「辛抱(我慢)する」、
\\	は「忍耐、辛抱」の意味。	
\\	医療の世界って常に進歩しているから、今不可能な事も先々には可能になるだろう。 
\\	「進歩」は
\\	、あるいは、
\\	でいいでしょう。	
\\	情報や技術が進歩することは、もう止まらないだろう。 
\\	には「進歩、前進。上達」という意味があります。	
\\	彼女は、その時のことをしんみりと語った。 
\\	「しんみりと」は
\\	「心(の底)から」を使えばいいでしょう。	
\\	あそこの会社って、問い合わせする度に言うことが違うから、今イチ信用できないんだよね。 
\\	ここでは「信用できない」を 
\\	で表現しています。
\\	は「…する度に」の意。	
\\	約束を破ってばかりいると、信用を落とすよ。 
\\	で「信用を失う」という意味。
\\	「信頼」。	
\\	あの人って自意識過剰なとこあるよね。 
\\	は「ひどく自分を意識している」、
\\	は「自分のことに没頭している」の意味。	
\\	就職したっていうのに、あいつは社会人だという自覚がまるでない。 
\\	「自覚がない」をここでは
\\	に気づいていない」を使って言い換えています。
\\	「一人前の」。	
\\	彼もリーダーとしての自覚が出てきたみたいで後輩をおいて帰るって事はなくなった。 
\\	ここでは「自覚が出てきた」を
\\	するようになる」、
\\	という感覚が出てくる」を使って表現しています。
\\	「後輩」。	
\\	テスト時間はどれくらいですか。 
\\	この場合の「時間」は
\\	を使えばいいでしょう。	
\\	今はいろいろ悩みがあると思うけど、時間が解決してくれるんじゃないかな。 
\\	(解決する、うまく行く)と
\\	(やがて)を組み合わせて「時間が解決する」を表現しています。	
\\	地震に会ったときは、二度と旅行はしないと思ったけど、時間が経つと忘れるもんだ。 
\\	は「時と共に、やがて」の意味。
\\	「時の経過と共に」。	
\\	日曜日の2時頃行くから、その時間帯は空けといてね。 
\\	「時間帯」の「帯」は
\\	でいいでしょう。
\\	「取っておく」。	
\\	時間つぶしに喫茶店でコーヒーでも飲んでいくか。 
\\	「(時間を)つぶす」は
\\	、あるいは、
\\	を使います。	
\\	どっちから行っても時間的には一緒だと思うよ。 
\\	で「時間に関しては」という意味になります。	
\\	こんなやり方は、時間の無駄だと思います。もっと、効率よくやれるよう考えましょう。 
\\	「時間を無駄にする」は
\\	で表現できます。
\\	「時間的に効率のいい」、
\\	「効果的な、有効な」。	
\\	仕事もプライベートも忙しいので、あなたと遊ぶ時間はありません。 
\\	「…する時間はない」は、
\\	で表現します。	
\\	忙しいのに時間とらせてしまって、ごめんね。 
\\	「時間をとらせる」は
\\	、あるいは、
\\	「無駄にする」を使えばいいでしょう。	
\\	毎年今の時期はこのような天気です。 
\\	が「毎年今の時期は」に当たる言い回しです。	
\\	今回は時期が合わないので、コースの申し込みは次回まで待つことにしました。 
\\	ここでは「時期が合わなかった」を 
\\	を使って表現しています。(注)「コース」を「英会話のコース」にして英訳しています。	
\\	この新しい部品を使うのは時期尚早だよ。 
\\	「時期尚早」とは
\\	「(時期が)早すぎる」ということですね。	
\\	自給自足の生活が理想だよ。 
\\	が「自給自足の生活」に当たる表現です。	
\\	最近いろんな事件がおきるけど、一番の原因は便利になってしまった世の中のせいかもしれない。インターネットとかさ。 
\\	「事件」は 
\\	といいます。
\\	(犯罪)を使ってもいいでしょう。	
\\	そんな次元の低い話は、やめましょう。 
\\	を使って会話のレベルを少し上げようと言っています。
\\	「低俗なこと」。	
\\	あそこまで自己顕示欲が強いのは、ほんと、鼻につくよ。 
\\	で「目立とうという強い意識を持っている」の意味になります。
\\	は「自己顕示欲の強い」、
\\	は「かんに障る」の意。	
\\	自己主張をするのもいいが、自分の仕事くらいはちゃんとやれよな。 
\\	で「自己主張をする」の意味になります。	
\\	なぜ彼はあんなにも自己中心的なんだろう。 
\\	「自己中心的」は 
\\	で表現します。	
\\	お酒飲んで運転して事故を起こしたらどうするの? 
\\	で「事故を引き起こす」、
\\	は「衝突」の意味。	
\\	あいつ、自業自得だよ/ざまあみろ。 
\\	「自業自得だ・当然の報いだ」には
\\	人+
\\	という決まった言い方があります。主語を省略して、
\\	のようにも言います。また、
\\	を使って、
\\	「自分でそうしたんだろ(招いた)」とも言えます。	
\\	昨日、ハワイから帰ってきたばかりだから、時差ボケで眠くて仕方がない。 
\\	で「ひどい時差ボケにかかっている」の意味。	
\\	週末に食べたお菓子が昨日自主回収されたってニュースを聞いたのだけど、食べてしまったのはどうすればいいのって感じ? 
\\	「自主回収する」は 
\\	で表現します。
\\	は「…することになっている」の意。	
\\	お前には自主性ってものがないのか。いつでも人の意見ばかり聞いて。 
\\	「精神の独立」が「自主性」に当たる表現です。	
\\	授業の復習や自主勉強を少しでもやったほうがいい。 
\\	で「自主的に(自分で決めた)勉強をする」の意味になります。
\\	は「自分で」の意。	
\\	これは、自信があるよ! 
\\	で「自信がある」の意味。
\\	「自分でできるってわかってる」。	
\\	あいつ自信過剰なんだよ。、自分は仕事ができると思ってるんだから。 
\\	「自信過剰」は
\\	で表現します。
\\	は「大物」のこと。	
\\	最近は何をやってもうまく行かないから、自信喪失だよ。 
\\	「自信喪失」は
\\	「自信を失う」で表現できます。
\\	とは「最低」の意味。	
\\	もっと自分に自信を持てよ。そんな弱気じゃ、次の試合に勝てないぞ。 
\\	で「自分に自信を持つ」の意味になります。
\\	は「弱腰の」の意。	
\\	事実は小説より奇なりって言うだろ。 
\\	が「事実は小説より奇なり」に当たる言い回しです。	
\\	近所にコンビニがあって、もったいないとは思いつつ、自炊しないでコンビニ弁当で済ますことが多々あるけど、これじゃあお金がもたない。 
\\	が「自炊」に当たる表現です。
\\	は「無一文になる」の意。	
\\	参考書によるとこの場合、動詞の時制を一致させないといけないらしいよ。 
\\	で「動詞の時制を同じにする」の意味になります。	
\\	君の会社は時代遅れだね。 
\\	(流行遅れの)の代わりに
\\	を使ってもいいでしょう。
\\	は「時代遅れの」という意味。	
\\	イチローは日本でプレイしていた時代からファーストネームを背番号につけてプレイしていました。 
\\	で「日本で野球をしていた時代から」の意味になります。	
\\	あの老舗が店じまいだって。これも時代の流れでいたしかたのないことか。 
\\	「時代の流れ」はそのまま
\\	と言えばいいでしょう。	
\\	思うように事が進まないのでじだんだを踏んでいた。 
\\	「地団太を踏む」とは「いらいらして足を踏みならす」ことですから、
\\	も可)で表現できます。	
\\	試験日が近づくにつれて、だんだん実感がわいてきてやる気が出てきた。 
\\	で「その現実をじわじわと感じる」の意味になります。	
\\	英語に限らず言葉は奥が深いよ。そう簡単にマスターできるものではないと最近実感しました。 
\\	「実感するようになる」を 
\\	で表現しています。
\\	でも同じような意味になります。	
\\	彼は言ったことは必ず実行する奴だ。 
\\	は「言ったことはやる」の意味ですね。
\\	「最後までやり通す」。	
\\	この教会って写真でこんなに綺麗なんだから、実際はもっとすごいんだろうな。 
\\	この場合の「実際」は
\\	を使えばいいでしょう。	
\\	えーっと、それで実際に私に会ってみて、どう思った? 
\\	「実際に」を使えばいいでしょう。
\\	「本人に、実物に」。	
\\	2、3年って残りの人生の中では、短いような気がするけど、実際にはきっと長いんだろうな。 
\\	「実際には」は
\\	でいいでしょう。
\\	(比べる)の使い方に注意。	
\\	全国交通安全運動を実施中です。 
\\	「交通安全」は
\\	といいます。
\\	は「実施中である」の意。	
\\	足を捻挫して、週末は家でじっとしてたんだ。 
\\	で「週末を家で何もしないで過ごす」という意味になります。	
\\	実はさ、僕大学を辞めたんだ。 
\\	この場合の「実はさ」は 
\\	で表現すればいいでしょう。	
\\	彼女、写真より実物の方がずっとかわいいじゃない。 
\\	「実物」は
\\	で表現します。	
\\	そんなにマラソン好きなら、大会とか出てみたら?自分の実力も試せるし。 
\\	「実力」は
\\	「実際の能力」を使えばいいでしょう。
\\	「尺度、評価基準」。	
\\	このコースは
\\	と比べてもっと質が高く実用的で、修了すればほかの授業についていける実力がつきます。 
\\	「実力がつく」を 
\\	(得る)と
\\	「…するのに必要となるもの(力)」を組み合わせて表現しています。	
\\	このコースは侮れない。こういう所で実力の違いがわかる。 
\\	とは「大人と子供を区別する」、つまり、「力があるかどうかを区別する」という意味になります。
\\	「下手なプレーヤー」。	
\\	あまりに期待が大きすぎて、彼女は実力を発揮できずに初戦で負けてしまった。 
\\	で「自分の持つ能力を十分に利用する」という意味になります。	
\\	それじゃあ自転車操業になっちゃうよ。 
\\	(その日暮らしをする)で「自転車操業」を表現しています。	
\\	中東での自爆テロの報道を見ると、いつも複雑な気持ちになる。 
\\	が「自爆テロ」に当たる表現です。「複雑な気持ち」の表現の仕方にも注目です。	
\\	今までに作りためた俳句を自費出版したいと思います。まず何をしたらよいでしょうか。 
\\	が「自費出版する」に当たる動詞です。	
\\	約束を破ったらどうなるかわかっているよね。自分から今度約束破ったらこづかい半分でいいって言ったんだからね。 
\\	「自分から」のニュアンスは
\\	で出せますね。
\\	「お小遣い」。	
\\	自分で言って、言った自分が恥ずかしくなってきた。 
\\	(決まりが悪い、恥ずかしい)と 
\\	を組み合わせて「自分が恥ずかしい」を表現しています。	
\\	彼っていっつも自分で言わずに、人に言わせようとする。 
\\	「自分で言わずに」を
\\	「自分で言う代わりに」で表現しています。
\\	動詞」の使役用法にも注目。	
\\	海外のスーパーでは店員が袋に入れてくれるけど、日本では自分でやらなければならない。 
\\	「自分で」は
\\	を使えばいいでしょう。動詞
\\	にも注目。	
\\	自分で言うのも何だけど、これでも昔は成績はクラスでも一位2位を争うぐらいだった。 
\\	「いい気になる、得意になる」、
\\	「自画自賛する」の使い方がポイントですね。	
\\	自分でもびっくりするくらい料理が下手なんだよね。ちょっと真剣に習ったほうがいいかも。 
\\	で「自分でもびっくりする」の意味になります。
\\	は「真剣に」の意。	
\\	こんなに忙しいのに、自分でもよくやってると思うよ。 
\\	を使って「自分の仕事に自分でも感心している」と表現しています。
\\	「自分の基準からしても」。	
\\	できない自分に腹が立つ。ああ悔しい。 
\\	で「自分自身が嫌になる」、
\\	は「みじめ、痛ましい」の意。	
\\	自分に正直に生きようよ。 
\\	「正直に生きる」は、
\\	で表現できます。	
\\	できるだけ自分の置かれた状況を良い方向にみて生きたい。 
\\	「環境」、
\\	「状況」と
\\	の使い方がポイントですね。	
\\	彼って煙草吸う時自分の世界にはいっているよね。端から見るとそれがすっごく笑えるんだけど。 
\\	で「自分の世界に入り込む」の意味になります。	
\\	自分らしく生きてみろよ。 
\\	で「自分が決めた通りに生きる」という意味になります。
\\	「確立する」、
\\	「従う」。	
\\	いつでも自分らしさが大切だ。 
\\	「自分らしさ」は 
\\	(自分自身になる)、
\\	(自分自身に忠実である)で表現すればいいでしょう。	
\\	そんなに自分を責めることはないよ。 
\\	は「自分自身に厳しくする」、
\\	「責める」の意。	
\\	おいまたかよ。あいつは酒飲むと自慢話を始めるから嫌なんだよ。 
\\	は「自慢をする」の意味。
\\	「自慢する(大きなことを言う)」。	
\\	そんな地味なネクタイは、その背広には合わないよ。 
\\	は「控えめな」、
\\	は「目立たない」の意味。	
\\	地道な努力をすることを忘れてはいけない。努力すればきっといい結果が出るよ。 
\\	は「こつこつやる、働く」の意味です。
\\	でも「地道な努力」の意味になります。	
\\	彼の服の趣味は地味だ。/ 彼の性格は地味だ。 
\\	「服が地味だ」は
\\	「保守的な(に)」、「性格が地味だ」は
\\	「控えめな」を使えばいいでしょう。	
\\	引っ越しするにあたって、いろいろ事務手続きがあってとても面倒だ。しかも、対応の悪い担当者とかもいたりするし。 
\\	とは「お役所(官僚)的な手続き」のことです。
\\	は「煩わしい手続き」の意。	
\\	毎日のように雨が降って、じめじめした日が続くと、気分が滅入るよね。 
\\	「じめじめ」を 
\\	(湿気のある)と 
\\	(湿っぽい)を使って表現しています。	
\\	じゃあ最初からそう言えばいいでしょ。 
\\	「じゃあ」は 
\\	でいいでしょう。	
\\	海外旅行とかに行って実際に英語を使う事によって、自分の英語の弱点がわかる事が多い。そこで間違えた英語はなかなか忘れないしね。 
\\	「弱点」は
\\	でいいでしょう。	
\\	じゃじゃ馬ならしは男の醍醐味。 
\\	とは「手に負えない人」の意味になります。
\\	「言うことをきかない、制御できない」。	
\\	もう! また変な画像をメールで送ってきて! ウィルスとか入ってるんじゃないの? 
\\	「…じゃないの」を 
\\	(…ではないかしら)を使って表現しています。
\\	は「変な、奇怪な」の意。	
\\	歯を一生ケアするってことじゃなくて、ただ単に痛みが止まればいいっていう考えなわけね 
\\	ここでは「じゃなくて」を
\\	を使って表現しています。
\\	に関心を持つ」。	
\\	そこにいると邪魔なんだよね。どいてよ。 
\\	「邪魔」は
\\	を使って表現します。	
\\	この犬、人が好きで、誰かが通りかかるとすぐにじゃれるんだ。 
\\	は「元気がいい」
\\	は「遊び好きな」の意味	
\\	目が充血してるけど、昨日も残業だったのかい。 
\\	が「(目が)充血している」に当たる形容詞。	
\\	前回登録した住所を変更しにきたのですが、どうすればいいですか? 
\\	「住所を変更する」は 
\\	、あるいは、
\\	で表現します。	
\\	当プログラムを一層充実させることが小生の課題だ。 
\\	「充実させる」とは
\\	「…をよりよくする」ということですね。
\\	は「肉付けする」の意。	
\\	ここは、障害者のための設備が充実している。 
\\	で「必要な物を整えてある」の意味になります。	
\\	この先10キロも渋滞してるんじゃ、高速道路を降りた方がいいな。 
\\	「渋滞」は
\\	ですね。
\\	「もつれさせる」。	
\\	今回の海外旅行で何を重点的に(英語の)勉強するすればよいかを教えてくれた気がする。 
\\	で
\\	に最も重点を置く」の意味になります。
\\	に集中する」。	
\\	次の就職が決まるまでは、人生の充電期間だと思って頑張りなさい。 
\\	で
\\	のために(バッテリーを)再充電する」という意味になります。
\\	十分に利用する」。	
\\	その髭剃り充電切れだよ。充電してあげるから貸して。 
\\	で「パワー切れ」の意味になります。
\\	は「(電池が)なくなる」の意。	
\\	どのような状況でも臨機応変に対応できる柔軟性が彼には欠けている。 
\\	で「柔軟な」の意味。
\\	がその名詞形。
\\	「欠ける」、
\\	「対策を講じる」。	
\\	運動する前には必ず柔軟体操をしたほうがいいよ。 
\\	「柔軟体操をする」は 
\\	で表現できます。
\\	と言ってもいいでしょう。	
\\	もっと柔軟な考え方をしなくちゃだめだよ。 
\\	「柔軟な」は 
\\	で表現できます。	
\\	カラオケでの私の十八番は、山口百恵の『コスモス』です。 
\\	「18番」は、数字を使っては表現できませんので、
\\	のように言い換える必要があります。	
\\	十八才未満の方はこのビデオを借りることはできません。 
\\	「十八才未満」は
\\	18と言います。	
\\	そんな重箱の隅をつっつくようなことばかり言うなよ。 
\\	で「小さいことにこだわる」の意味。
\\	の代わりに、
\\	「いちいち細かいことを言うな」という言い方もあります。	
\\	重箱の隅をつつくような質問ばかりするなよ。 
\\	は「うるさい」、
\\	は「意味のない」、
\\	は「細かいことにこだわること」の意味。	
\\	毎年、ダイアリーを買うときに、どんな点を重要視していますか? 
\\	「重要視する」をここでは、
\\	「あなたが求める最も重要なこと」で表現しています。	
\\	会議中に思わず熟睡してしまった。 
\\	には「ぐっすり」という意味があります。	
\\	志望校に無事合格できて、これでやっと受験地獄から抜け出せる。 
\\	ここでは「受験地獄」を文字どおりに 
\\	で表現しています。
\\	は「とんでもなく競争の激しい」の意。	
\\	今後需要がありそうな分野を勉強して、安定した仕事につき生活を安定させたい。 
\\	「需要がある」は 
\\	で表現できます。
\\	は「安定した」の意。	
\\	お酒は、人間関係を滑らかにする潤滑油です。 
\\	「潤滑油」は
\\	(可算名詞)と言います。	
\\	たまに自分の仕事の事を自己紹介の時に話す機会があるのだけど、なかなかうまく順序だてて説明できないのよね。 
\\	この場合の「順序立てて」は
\\	「まとめる、体系づける」を使って表現すればいいでしょう。
\\	「順番に並べる」。	
\\	車に
\\	プレイヤーをつけようと思ってるんだ。ちょっと高いけどやっぱり純正品がいいな。 
\\	(自分の車のブランドと同じものにする)で「純正」を表現しています。	
\\	順調に行っても12時間かかるのに、1時間遅れたから13時間もかかったんだよ。 
\\	で「すべてが順調に行く」という意味になります。	
\\	アガシ、サンプラスら、シード選手はそろって順当勝ちをし、ベスト16に進んだ。 
\\	「順当勝ち」は、
\\	「予想したとおり」、あるいは、
\\	「驚くべきことではない」で表現できますね。「シード選手」は
\\	と言います。	
\\	彼は順当に勝ち進んだが、決勝で負けてしまった。 
\\	ここでは「順当に」を
\\	「皆が期待した通り」で表現しています。	
\\	小さい番号から順に当てて落としていけばいいの?(ビリヤードで) 
\\	で「小さい番号の順に」の意味になります。
\\	は「番号順に」の意。	
\\	彼には順応性があると思う。1週間もしたらすぐ新しい環境に染まれる。この間、オーストラリアに行ったときにそう感じた。 
\\	が「順応(適応)性がある」に当たる形容詞です。
\\	は「本領を発揮する」、
\\	は「順応(適応)性」の意。	
\\	それじゃ、ここのダイアログを山口、野村、小杉の順番で読んでみて。 
\\	で「その順番で」の意味になります。前置詞は
\\	を使う点に注意。
\\	人 
\\	動詞」で「(人に)~させる、してもらう(使役用法)」の意。	
\\	注文によっては、(食事を届ける)順番が前後してしまうことがあります。ご了承ください。 
\\	「前後する」をここでは
\\	「逆にする、入れ替える」を使って表現しています。
\\	「首尾一貫した、矛盾のない」。	
\\	やっとここ最近引っ越しの準備を少しずつ始めました。でもいざ当日はどうなるか、全然準備が終わってないのではと嫌な予感がします。 
\\	「引っ越しの準備をする」は 
\\	でいいでしょう。
\\	(準備)も使えます。	
\\	(雨が急に降ってきてにもかかわらず、ちゃっかり傘を持っている人に対して)準備いいね。 
\\	「準備万端である」の意味で
\\	をよく使います。
\\	は「準備してきたんだね」の意味になります。	
\\	泳ぐ前には、ちゃんと準備運動しなさいよ。 
\\	は「(体を)しなやかにする」という意味です。
\\	も使えます。	
\\	惜しかったね、準優勝で。 
\\	で「2位に終わる」の意味。
\\	とも言えます。	
\\	あいつって自由気ままに暮らしてるよな。ちょっとは落ち着いた方がいいんじゃないかな。 
\\	には「自由奔放な」という意味があります。
\\	は名詞で「気まま」の意。	
\\	ご自由に中をお回りください。 
\\	で「自由に... する」の意味になります。	
\\	私のお金で買うんだから何を買おうと私の自由でしょ。 
\\	ここでは「自由に」を 
\\	を使って表現しています。
\\	は「…な限りは」の意。	
\\	その映画の上映時間は、90分です。しかも、3D映画では初めての大作らしい。90分っていうのは。 
\\	で「上映時間は 
\\	である」の意味。
\\	「大ヒット作」。
\\	を使っても簡単に表現できます。	
\\	この映画の上映時間はどのくらいですか? 
\\	が「上映時間」に当たる表現です。
\\	を使ってもいいでしょう。	
\\	怪我や病気のために捕獲した野生動物にはあえて名前は付けないんです。付けてしまうと情が移って別れるとき辛くなるので。 
\\	で
\\	に愛情を抱くようになる」の意味。
\\	の後の
\\	不定詞は省略されることがあります。	
\\	情がわいて、別れられない。 
\\	ここでは「情がわく」を
\\	に強い愛着を持つ」、
\\	「強い関係を築く」で表現しています。	
\\	彼女の英語力が上級者並みになっているなんて、なんだかちょっと悔しいな。 
\\	で「上級者(並み)になる」という意味。ここでは「悔しい」を
\\	を使って表現しています。	
\\	彼の言動は常軌を逸している。 
\\	は「突飛な、普通ではない」の意味の形容詞。	
\\	このコースに申し込むのに、何か条件がありますか? 
\\	とは「必要条件」のことです。	
\\	(視力検査で)上下左右のうちどこが開いているか言ってください。 
\\	この場合の
\\	は「(開いてる部分は)どっちを向いていますか」の意味で使われています。	
\\	タクシーに乗ろうとしたら、嫌な顔をされ、その上、乗車拒否された。 
\\	「運転したままだった(通り過ぎた)」、
\\	「私を乗せなかった」で「乗車拒否」を表現しています。
\\	「呼び止める」、
\\	「嫌な」。	
\\	高校時代、僕は遅刻の常習犯だった。 
\\	「遅刻の常習犯」とは
\\	「ほとんど時間通りについたことがない」ということですね。	
\\	クーラーなどの冷房機器で室内の温度を下げると、外の空気がその分上昇するらしい。 
\\	「上昇する」は 
\\	で表現すればいいでしょう。動詞 
\\	の使い方にも注意。	
\\	周りの友だちはわざわざこういう時期に留学しなくてもって口をそろえていて、この情勢のなか本当に行くのかって話しているらしい。 
\\	この場合の「情勢」は 
\\	(現在の状況)で表現すればいいでしょう。
\\	は「皆が一致して」の意。	
\\	管理職との間に、親近感や一体感を醸成することが重要である。 
\\	ここでは「醸成する」を 
\\	(育てる、育成する)、
\\	(作る)を使って表現しています。	
\\	ようやくインターネットができる状態になったから、またここにメール下さい。 
\\	で、「ようやくインターネットに接続できる状態になっている」の意味。	
\\	冗談じゃない、病院なんか行くものか。 
\\	「冗談じゃない」は 
\\	や 
\\	!(勘弁してくれよ)で表現します。	
\\	この煮物、上品な味ね。 
\\	「高級な味」、
\\	「グルメ向きの」の意味。	
\\	マニキュア落としたいんだけど、除光液持ってる? 
\\	「除光液」は
\\	と言います。	
\\	車を運転するときは、助手席にだれかに座ってもらって反対側を見ててもらわないと運転できません。 
\\	ここでは「助手席」を 
\\	で表現しています。	
\\	ボーリングは、意外と得意で平均スコアは130ぐらい。女性にしてはかなりいい方だと思いません? 
\\	「~にしては、割には」は前置詞の
\\	を使って表現します。	
\\	犯人は皆変装していて、その内の一人は女装していた。でも、傑作な事に、かつらがちょっとずれていた。 
\\	「一人女装している」の意味になります。
\\	「犯罪者」、
\\	「かつら」、
\\	「曲がる」。	
\\	今週は忙しくて医者に行く暇はないし、休暇も取れないから、どうにかして自力で治さなきゃ。 
\\	「自力で」は
\\	で表現できます。
\\	は「…する余裕がある」の意。	
\\	太陽がじりじり照りつける。 
\\	は「生きたまま焼く」、
\\	は「容赦なく照りつける」の意味です。	
\\	最近の女性は自立している人が多い。 
\\	「自立している」は
\\	(自活する)、
\\	(独立した)を使って表現すればいいでしょう。	
\\	人に頼ってばかりではいけない。自立心を持たなければ。 
\\	は「独立独歩の」という意味です。
\\	は「独立心のある」の意。	
\\	じれったいな。何が不満なのかはっきり言ってくれよ。 
\\	「いらいらさせる」で「じれったい」を表現しています。
\\	は「馬鹿げている、お話にならない」の意味。	
\\	彼は人権派の弁護士として知られていた。 
\\	で「人権派の弁護士」の意味になります。
\\	は「…を擁護する」の意。	
\\	なんだか最近、毎日のように人身事故が起きているような気がする。 
\\	とは「死傷者が出た事故」のことです。	
\\	今日人事異動の発表日だね。なんか緊張しちゃうよ。いきなり呼ばれて異動だったらどうしよ! 
\\	が「人事異動」に当たる表現です。
\\	は「入れ替える」の意。	
\\	彼の野球人生は、今期で幕を閉じた。 
\\	「野球人生」は 
\\	でいいでしょう。
\\	の使い方にも注目。	
\\	就職すべきか、大学院に行くべきか。彼は今、人生の岐路に立たされている。 
\\	で「岐路に立っている」の意味。
\\	は複数形ですが、単数として扱います。	
\\	人生は一度しかないから、チャンスがあればいろいろやってみたいと思う。 
\\	「人生は一度」を
\\	で表現しています。もちろん、
\\	も使えます。	
\\	やっぱり「人生日々勉強」だね。 
\\	の組み合わせに注目。	
\\	食べ物に好き嫌いはないんですが、サバだけはどうしてもだめなんです。蕁麻疹がでちゃうもんですから。 
\\	「好き嫌いがない(食べ物にとやかく言わない)」は
\\	で表現できます。「サバ」は
\\	と言います。ポイントとなる「蕁麻疹
\\	が出る」は、
\\	で表せます。また、具体的に
\\	とも言えます。	
\\	このシャワー、水圧が低くてイライラする。 
\\	「水圧」は 
\\	といいます。「低い」は 
\\	で表現します。	
\\	この前、健康のために水泳でも始めようかと思ったんだけど、結局まだ何もしていない。 
\\	「~を始める」には
\\	の他に、
\\	という言い方もあります。
\\	に取りかかる」。	
\\	今日は水温が低いから、プールには入れないな。 
\\	「水温が低い」とは
\\	「水が冷たすぎる」ということですね。ちなみに、「水温」は
\\	と言います。	
\\	あなたの瞳に、吸い込まれそう! 
\\	「吸い込まれる」は
\\	「引き込まれる」で表現すればいいでしょう。	
\\	すばらしい景色で思わず吸い込まれそうになった。 
\\	で「引き込まれるような感じがする」。
\\	を引き込む」。	
\\	大学は、テニスの推薦入学で入りました。 
\\	「推薦で」は
\\	を使って表現できます。前置詞
\\	がポイントです。	
\\	ランチの時間なのに、この店ずいぶん空いてるね。 
\\	で「がら空き」という意味になまります。	
\\	炎天下で運動する場合には、水分補給を忘れずに。 
\\	の意味は「水を加えて元に戻す」です。
\\	は「焼けつくような」の意。	
\\	きみの改善提案は全社に水平展開された。 
\\	が「水平展開する」に当たる言い回しです。	
\\	お酒を飲むとトイレが近くなり睡魔に襲われる。 
\\	「睡魔に襲われる」は
\\	「眠くなる」で表現すればいいでしょう。	
\\	水面下では、もう業界再編が始まっている。 
\\	で「舞台裏の(形容詞)」という意味になります。	
\\	探偵物の本って、読みながら自分も一緒に犯人を推理できて面白い。 
\\	で「犯人が誰かを推理する」の意味になります。
\\	「見つける」。	
\\	先日撮った写真は、データが大きくなってしまうので数回に分けてメールで送ります。 
\\	「分ける」は
\\	、あるいは、
\\	を使えばいいでしょう。数回は
\\	で表現します。
\\	「送信」。	
\\	ここからそこまでなら、数十分で行けるよ。 
\\	「数分」なら
\\	が使えますが、「数十分」は10 
\\	や20 
\\	のように表現します。	
\\	まだ幼いのにそんな世の中をなめたようなことを言っているとは、末恐ろしい。 
\\	で「彼が成長した時に、どのようになるのかを考えるのが怖い」という意味になります。
\\	「世慣れた」、
\\	「恐ろしい」。	
\\	末永くお幸せに。 
\\	と
\\	を繰り返すことで「ずっと永久に」という意味を表現しています。	
\\	日頃から傲慢な上司に、言いたいことをぶつけてやったのですかっとした。 
\\	で「すかっとした」を表現しています。
\\	は「大胆に立ち向かう」の意。	
\\	食べ物の好き嫌いはあまりないんだけど、ピーマンだけは食べられないわ。 
\\	「好き嫌いはない」は
\\	「何でも食べる」と言い換えて表現すればいいでしょう。「ピーマン」は
\\	あるいは
\\	と言います。	
\\	この番組って、君が好きそうな番組だね。 
\\	は「君が好きなそうな種類の番組」、
\\	は「人の好みに合っている」の意味。	
\\	空きっ腹にビールを飲んだからすぐに酔いが回ってしまった。 
\\	で「空きっ腹に」の意味になります。前置詞が 
\\	になる点に注意。	
\\	スキーって好きでもないけど、嫌いでもない。 
\\	文字通りに
\\	と言えます。
\\	は「どちらでもいい」の意味。	
\\	好きな食べ物はパスタと寿司です。 
\\	「好きな食べ物」は
\\	、または、
\\	を使って表現すればいいでしょう。	
\\	最近好きな人ができたんだ。 
\\	「好きな人」は 
\\	と 
\\	を組み合わせて表現すればいいでしょう。
\\	は「…に首ったけになる」の意。	
\\	僕は、だんだん ひろ子のことが好きになってきた。 
\\	「だんだん好きになる」は 
\\	(…を好きになり始める)を使って表現すればいいでしょう。	
\\	どうも家事って好きになれない。主婦としてこんなんじゃだめだってわかっているんだけど。 
\\	するようになる」は
\\	で表現できます。
\\	には
\\	を身につける」という意味があります。
\\	「罰当たりの行い」。	
\\	過ぎたことは水に流そうよ。 
\\	が「過ぎたことは水に流す」に当たる言い回し。	
\\	試験に受かったとは言え気を抜けない。やっと、スタート地点にたどりついたに過ぎないから。これからいろいろ試練がまっている。 
\\	で
\\	に到達したすぎない」の意味になります。
\\	「試練」、
\\	「苦難」。	
\\	彼が悲観していないということが唯一の救いだ。 
\\	「唯一の救い」に当たる英語は
\\	です。
\\	「欠如」、
\\	「沈まない」。	
\\	今度以前事故があった機種と同じ飛行機を利用するのですが心配だ。救いの種は、専門家はあの機種はいい飛行機だと言っていたこと。 
\\	(それで気分がよくなった、安心できた)で「救いの種」を表現しています。
\\	は「安心させる」の意。	
\\	お弁当が一つ少ないよ。10人なのに、9人分しかない。 
\\	「少ない(足りない)」は
\\	で表現できます。また、
\\	という言い方も一緒に覚えておきましょう。	
\\	両親の不仲は、私の結婚観に少なからず影響を与えたと思う。 
\\	で「…するのに大きな役割を果たす」の意味になります。
\\	は「少なからず」の意。	
\\	少なくとも、英語だけはマスターするつもりで来たので、習得するまでは帰らないつもりです。 
\\	「少なくとも」は 
\\	で表現します。
\\	は「習得する」の意。(注)「取得」を「習得」に変更しました。	
\\	巻きずしには、固めのご飯が合うのだなと学んだよ。つまり水を少なめにして炊かないと駄目なんだよね。 
\\	で「少なめの水」の意味になります。
\\	は「不足した」の意。ちなみに、
\\	とは「少し歯ごたえのある」ということです。	
\\	洋画って邦画と比べると、やっぱスケールが違うよね。 
\\	で「違ったスケールで」の意味になります。	
\\	彼ってスケベおやじそのもの!!! 
\\	で「好色そのもの」という意味。
\\	「好色な」。	
\\	週1日一気に勉強するより、少しずつでもいいから毎日勉強した方が効果的だと思います。 
\\	で「一度に少し」の意味になります。
\\	(少しずつ)。	
\\	完璧を目指しているので少しでもミスがあったら言ってください。 
\\	で「ほんのわずかなミスをする」の意味になる。
\\	は「…を目指す」の意。	
\\	あなたは、私のことを少しも考えてくれてない。 
\\	「少しも考えない」は
\\	でいいでしょう。	
\\	少し問題があるけど普通に使えるよ。 
\\	(少し問題がある)を
\\	で表現しています。
\\	(使える)。	
\\	健やかに成長なさいますようお祈りしています。 
\\	「健やかに」は 
\\	で表現するといいでしょう。	
\\	痩せてなんかないよ。見えてない部分はもうすごいんだから。年とともに、下の方に肉が下がっていってる気がする。 
\\	この場合の「すごい」は
\\	(ひどい)で表現すればいいでしょう。
\\	は「垂れ下がる」、
\\	は「ずんぐりした」、
\\	は「垂れた」の意味。	
\\	宙返りできるなんて、すごーい! 
\\	ここでは「すごい」を 
\\	で表現しています。	
\\	見かけはぱっとしないが、彼は凄腕の刑事らしい。 
\\	ここでは「凄腕」を
\\	「誰も負かすことがでいない」で表現しています。
\\	「優秀な」。	
\\	私の理想の休暇の過ごし方は、映画を見ながらお菓子をつまみゆっくりすることです。それで、ワインなんかあると最高だね。 
\\	が
\\	の過ごし方」に当たる言い回しです。	
\\	夢とか目的をもって若い頃を過ごすべきだった。 
\\	「過ごす」は
\\	で表現できます。
\\	過去分詞」で「~すべきだったのに(しなかった)」の意。	
\\	私は筋の通ってないことは嫌いだ。 
\\	は「筋の通らない」、
\\	は「つじつまの合わない」の意味。
\\	(動詞)は「論理的に考える」の意。	
\\	あんたにそんな事言われる筋合いはない。 
\\	「筋合いはない」とは
\\	「権利、理由はない」ということですね。	
\\	筋を通して話せば、彼だってわかってくれるさ。 
\\	「筋を通して話す」ということは
\\	「論理的に話す」ということですね。
\\	「よく考えた説明」。	
\\	その雑巾、使った後でちゃんとすすいでから絞ってそこに掛けておいてね。 
\\	「すすぐ」は
\\	を使って表現します。
\\	「絞る」、
\\	「掛ける」。	
\\	コンピュータを買おうと思うんだけど、君が薦める機種は何かな。 
\\	「薦める」は
\\	「推薦する」を使います。	
\\	原則として、このシラバスにしたがって1年間授業を進めていきたいと思います。 
\\	「進める」とは
\\	「行う」ということですね。
\\	「前進する」でもいいですね。	
\\	各金融機関で、不良債権の処理が遅々として進んでいないことぐらい小学生でも知っている。 
\\	(何とかなる)を否定形で用いて「進んでいない」を表現しています。
\\	は「不良債権」の意。	
\\	雀の涙ほどのボーナスでも、出ないよりましか。 
\\	は「小銭、はした金」、
\\	は「ほとんど無いようなもの」の意味。	
\\	このズボン長さが中途半端でみっともないんです。裾上げしてください。 
\\	「(裾を)上げる」は 
\\	で表現します。
\\	は「おかしい」の意。	
\\	スタイルがいいですね。その体型を維持する秘訣は? 
\\	「スタイルがいい」は 
\\	、あるいは、
\\	と言います。	
\\	社の我が社に対するスタンスは最近、より好意的に変化した。 
\\	をそのまま使えばいいでしょう。
\\	は「よくなる、改善する」の意。	
\\	よし、そろそろ、スタンバっといて。 
\\	「スタンバイする」とは「準備する」ことですから、
\\	を使えばいいでしょう。	
\\	小泉研究室から、多くの優秀な研究者が巣立っていった。 
\\	「巣立つ」は 
\\	、あるいは、
\\	(…を生み出す)で表現すればいいでしょう。	
\\	あの容疑者の証言はどうもすっきりしないよなあ。 
\\	「スッキリしない」とは「疑問が残る」ということで
\\	「答えている以上に疑問が生じる」を使って表現しています。
\\	「出来過ぎている」の意。	
\\	思いっきり言いたいこと言ってやった。あースッキリした。 
\\	「すっきりした」は 
\\	(気持ちよかった)で表現すればいいでしょう。	
\\	すったもんだの末、なんとか円く収まる。 
\\	は「色々な事が起きてそれが収まる」の意味。
\\	「まとめる、継ぎ合わせる」。	
\\	デートをすっぽかすなんて、あんまりだよ。 
\\	が日本語の「すっぽかす」に当たる表現です。
\\	は分詞構文で後から「すっぽかすなんて」と付け加えて言っています。	
\\	彼は、その組織の捨て石となって働いた。 
\\	(…のために自分を犠牲にする)で「捨て石」を表現しています。
\\	は「…のためだけに自分を捧げる」の意。	
\\	ホテルで
\\	500のランチっていうのも捨て難いけど、丑の日だからやっぱりうな重でしょう。 
\\	「魅力(引きつける物)が無いわけではない」で「捨て難い」のニュアンスを表現しています。
\\	「馬鹿にならない物」。「丑の日」と「うな重」の表現の仕方にも注目。	
\\	この間貧血で倒れたときはどうなることかと思ったよ。でも何人か声をかけてくれて、世の中まだまだ捨てたものじゃないと思ったよ。 
\\	ここでは「捨てたものじゃない」を 
\\	(まだ …する十分な理由がある)を使って表現しています。	
\\	このピザそう捨てたものでもないよ。 
\\	「捨てた物じゃない」=「悪くない」。
\\	がこの意味になります。	
\\	世の中まんざら捨てたもんじゃあないな。 
\\	この場合の
\\	は「(世の中の)さまざまなこと」という意味ですね。それが「思ったほど悪くない」と言っているのです。
\\	は「薄汚い、くだらない」の意。	
\\	まだバスのストライキは続いていますか?長いですね。バスが走ってなくても中心部に行ける手立てがありますか? 
\\	「ストライキが続く」は 
\\	と 
\\	で表現できます。
\\	も使えます。	
\\	ストレス性難聴にかかって、しばらく耳が聞こえなくなってしまいました。 
\\	「ストレス性の」は「ストレスのために、原因で」と言い換えればいいでしょう。	
\\	もう少し素直に人の話を聞けないのかな? 
\\	「素直に聞く」とは、
\\	「もっと真剣に」、
\\	「もっと注意して(聞く)」ということですね。	
\\	素直に、ありがとうって言っとけばいいのよ。ホント、素直じゃないね~。 
\\	ここでは「素直」のニュアンスを 
\\	と 
\\	で表現しています。
\\	は「(暗に)意味する」の意。	
\\	果てしなく続く単調な毎日の繰り返しを想像すると、砂を噛むような思いがした。 
\\	「砂を噛むような思い」とは 
\\	(退屈で活気のない)という意味ですね。
\\	は「面白みのない」の意。	
\\	スピード写真って鮮度があまりよくないけど、奇麗に撮るコツがあるんだよ。知ってる? 
\\	「スピード写真」は
\\	と言います。
\\	「ボックス、箱」。	
\\	週末は最悪だったよ。いきなり雨降ってきて、干してた洗濯物ずぶ濡れで洗濯しなおしだし。 
\\	この場合の 
\\	は「ずぶ濡れにする」という意味になります。
\\	は「洗濯物」の意。	
\\	前回より自分では滑れるようになったと思う。(スノーボード) 
\\	この場合の「滑る」は「スノーボードで滑る」ということですから、
\\	を動詞として使えばいいでしょう。	
\\	鳥は飛ぶ前に、羽をじゅうぶんすぼめる。 
\\	「十分すぼめる」を
\\	「しっかりと中に引っ張り入れる」で表現しています。
\\	は「備える」、
\\	は「緊張させる」の意。	
\\	おたくのミスなんだからどうにかしてちゃんと責任とってくださいよ。すみませんでしたでは済まされませんよ。私の人生がかかっていたんだから。 
\\	「済まされない」とは、
\\	「逃げられない」、
\\	「どこかにやってしまえない」ということですね。	
\\	そんな隅っこに座ってないで、こっちに来いよ。 
\\	「隅っこに」は
\\	を使えばいいでしょう。	
\\	住み始めの頃は、賃貸だからと思って、毎週掃除していたけど。日が経つと、やらなくなるもんだよ。 
\\	「住み始め」とは
\\	「引っ越してきた当初は」ということですね。
\\	「長続きしない」。	
\\	謝って済む問題じゃないだろう。 
\\	の文字通りの意味は「簡単に謝って逃げる」です。
\\	は「片づける、解決する」の意。	
\\	彼、スラッとしていて、格好いいわね。 
\\	が「スラッとした」に当たる形容詞です。	
\\	(お鍋に)砂糖をすりきり一杯入れます。 
\\	が「すりきり一杯」に当たる表現です。
\\	「平らにする」。	
\\	このスーツ何年も着てるから、袖口がすり切れちゃってる。 
\\	には「すり切れる、ボロボロになる」という意味があります。
\\	「ボロボロになる」。	
\\	昨日、駅の階段で転んで、右手の肘をすりむいてしまった。 
\\	「すりむく」は
\\	を使って表現します。	
\\	同僚に英作文頼まれたんだけど、やっぱり回答するからには、ちゃんと正しい答えを教えたい。 
\\	で「…する限りは」の意味になります。
\\	は「…したほうがいい」の意。	
\\	彼はちょっと仕事でミスをして、電話する気分ではないって言うから、代わりに私が電話しました。 
\\	「…する気分ではない」は 
\\	で表現します。
\\	でも同じような意味になります。	
\\	夏に一回取りに帰って来る事にして、取り合えず今回持って行ける物だけ持っていこう。 
\\	で
\\	することにする」の意味。
\\	ということにして」。	
\\	予約取りたかった日は取れなくて、今日しか空いてなかったので、今日4レッスン連続で取ることになったんです。 
\\	「(結果として)~することになった」は
\\	を使って表現できます。
\\	「連続で」。	
\\	見たくなくても、見る事になりそうだ。だって、その時はほとんどの番組がオリンピックだろうからね。 
\\	で「(結果として)... することになる」の意味。	
\\	来月、英語の勉強のために渡米することになりました。 
\\	が
\\	ということになる」の意味です。
\\	だと決まった」。	
\\	彼はこの町を出たまま、二度と戻ってくることはなかった。 
\\	「…することはない」は 
\\	で表現すればいいでしょう。	
\\	会議での彼の鋭い指摘に、どう答えていいか言葉に詰まってしまった。 
\\	は「鋭い、単刀直入な」、
\\	は「痛烈な」という意味。
\\	「ことばが出ない」。	
\\	これからも度々言い争いをするようなことがあれば、離婚を考えます。 
\\	が「…するようなこと」に当たる言い回しです。	
\\	来年留学で向こうに行ってからの話題に困らないように、最近洋楽を聞くようにしています。努力してるでしょ? 
\\	で「必ず…するようにしている」の意味。
\\	も同じような意味を表します。	
\\	だれかと話しているとき、知らない単語とか聞いたらメモるようにしているんだ。そしたらあとで調べることできるでしょ。 
\\	で「…することにしている」の意味になります。
\\	を使ってもいいでしょう。	
\\	英語の勉強を始めてから、いろんな事に対してよく考えるようになったと思う。 
\\	で
\\	することが(より)できるようになる」の意味になります。	
\\	アメリカに引っ越してきてから、うちの主人は家事を手伝ってくれるようになった。日本では全然しなかったのに。 
\\	ここでは「するようになった」を 
\\	で表現しています。
\\	(何もしない)	
\\	クッキングクラスの影響を受けたのか、最近料理するようになりました。 
\\	「最近料理するようになる」を現在完了進行形と 
\\	を組み合わせて表現しています。	
\\	最近お互いの心がすれ違っている気がする。 
\\	ここでは「心がすれ違う」を
\\	(気持ちがうまく通じ合わない)を使って表現しています。	
\\	彼は週末休みで私が平日休みなのですれ違いの生活を送っていると思っている人もいます。 
\\	ここでは「すれ違い」を 
\\	(…と合致しない)、
\\	(断絶、食い違い)を使って表現しています。	
\\	私は水、木が仕事が休みで、彼は週末が休みなのですれ違いの生活です。 
\\	とは「亀裂、不和、食い違い」という意味です。
\\	は「食い違い」の意。	
\\	これをお皿に分ければいいのですか? 
\\	この場合の「すればいい」は
\\	すべき」、あるいは、
\\	して欲しい」で表現できますね。	
\\	子供が車内の真ん中で座り込んでまわりの邪魔なのに、なんで親は注意しないのだろう。 
\\	「座り込む」のニュアンスは
\\	で表現できます。
\\	「ドスンと腰を下ろす」。	
\\	引田天功が拉致寸前のところだったってホントの話? 
\\	「寸前」は
\\	「もうすこしで...のところだった」を使えばいいでしょう。
\\	「拉致」、
\\	「誘拐」、
\\	「もう少しで... するこころ」の意。	
\\	いざ行くぞっという寸前で、やだとか言い出しやがって。 
\\	で「…する間際になって」の意味になります。
\\	は「ちょうど…しようとする」の意。	
\\	突然、角から車が飛び出して来て、すんでのところで車にひかれるところだった。 
\\	で「きわどいところで 
\\	するところだった」の意味になります。	
\\	最初からおごってもらうつもりだったみたい。なんて図々しい。 
\\	この場合の 
\\	とは「ずうずうしさ、厚かましさ」のことです。
\\	は「恥知らずの、ずうずうしい」の意。	
\\	せっかくアメリカまで来たのだから、どこか観光に行こうよ。ホテルにずっといるなんてもったいないよ。 
\\	「ずっと」を
\\	で表現しています。
\\	「閉じこもっている」。	
\\	お前、あの娘に気があるな。図星だろう。 
\\	「図星」は
\\	!(推測が当たった)、
\\	(僕の言うとおりだろう)で表現できますね。	
\\	彼はずぼらだから、細かい仕事は頼まない方がいい。 
\\	は「もさっとした人、だらだらした人」の意味。
\\	「いいかげんな」。	
\\	ずるいよ!そんな自分勝手な!自分だけ見たいテレビみてさ。 
\\	「ずるい」は
\\	「フェアーでない」で表現できます。	
\\	それはやさしさではなくて、男のずるさだ。 
\\	ここでは「ずるさ」を
\\	(巧みに操作する、ごまかす)や
\\	(不正直な)を使って表現しています。	
\\	2人はけんかばかりしているくせに、ずるずるとつきあっている。 
\\	(長い間一緒にいる)で「ずるずると付き合う」を表現しています。	
\\	今日会社ずる休みしちゃったよ。 
\\	「ずる休み」には
\\	という言い方があります。	
\\	ああ、また布団がこんなにずれて下がっている。何で毎回こうもずれるんだ?よっぽど寝相が悪いんだな。 
\\	とは「ずるずると滑り落ちる」ことです。
\\	は「だらしのない」、
\\	は「移る、位置が変わる」の意。	
\\	毎週録画してるこの番組、いつも10分位ずれてるんだけど、タイマーのセット間違えているのかな? 
\\	この場合の「ずれる」は 
\\	を使って表現すればいいでしょう。	
\\	帰りの電車の中ではもう疲れすぎてて立っているのが精一杯だった。 
\\	で「自分のできることは 
\\	だけだった」の意味になります。	
\\	お煎餅を食べた後のせいか、ヨーグルトが苦く感じる。ちょっと、飲んでみて。 
\\	で「多分 
\\	したからだろう」と言っていることになります。
\\	「一口」。	
\\	前の会社、とうとうリストラ始めたんだって。景気のいいうちに辞めて、正解だったかも。 
\\	の文字通りの意味は「やるべき正しいこと」になります。
\\	「リストラ」。	
\\	離婚の原因は、性格の不一致と価値観の違いです。 
\\	「性格の不一致」は 
\\	と 
\\	(かち合う、釣り合わない)で表現できます。
\\	は「一致する」の意。	
\\	アメリカで生活するのと滞在するのではわけが違う。 
\\	この場合の「生活(すること)」は 
\\	を使って表現すればいいでしょう。
\\	は「一時的に」の意。	
\\	彼女、どうやら生活残業しているらしいよ。残業代がもらえる今のうちに残業して稼いでおこうって考えだね。 
\\	で「自分の生活スタイルを維持する」の意味になります。
\\	は「収支を合わせる」の意。	
\\	給料から40%も税金で引かれるなんて、ちゃんと生活していけるかすごい心配だよ。 
\\	は「何とかやっていく」、
\\	は「(収入の範囲内で)何とかやりくりする」の意味。	
\\	送料はあとで請求します。 
\\	は動詞「請求書を送る」、名詞「請求書」としても使えます。	
\\	彼は正義感が強く、決して人をだますようなことはしない。 
\\	で「正義感が強い」の意味になります。	
\\	学生ビザだと、外国で生活していく上で、いくつか制限があります。 
\\	「制限」は
\\	「制約」、
\\	「限界、制限」を使えばいいでしょう。	
\\	ずっと正座していたから、足がしびれて立ち上がれなかった。 
\\	と
\\	「折り畳む」の使い方がポイントです。
\\	は「(足が)しびれる」の意。	
\\	上司に今月末で退職する旨を正式に言ったとたん、急にやる気がなくなって仕事に対する意欲がさっぱりなくなった。 
\\	「正式」は 
\\	を使って表現します。
\\	は「消える、なくなる」の意。	
\\	正式に旅行の日程が決まりました。 
\\	は「公式の」、
\\	は「正式に決められる」の意味。	
\\	この頃、精神的にまいってる。 仕事が夢にまででてくるもの。 
\\	「精神的に」は
\\	、あるいは、
\\	(感情的に)で表現すればいいでしょう。
\\	は「不安、心配」の意。	
\\	僕は政治音痴だから、政局にはまるで興味がない。 
\\	(政治的に)と
\\	(向いていない、適性がない)を組み合わせて「政治音痴」を表現しています。	
\\	政治家の不祥事が続き、国民の多くが政治不信に陥っている。 
\\	(…に導く)、
\\	(不信)、
\\	(政治)を組み合わせて「政治不信に陥る」を表現しています。
\\	は「違法行為、職権乱用」の意。	
\\	朝もやの湖畔は、静寂が支配していた。 
\\	は「静穏」、
\\	は「…に包まれる」の意。	
\\	不潔で、収入も低く、女の気持ちなんてさっぱりわからないあんな嫌なやつと別れてせいせいした。 
\\	で「せいせいする、すっきりする」の意味。
\\	の代わりに 
\\	を使ってもいいでしょう。	
\\	日本人は性善説でモノを考えるから、凶悪事件が再発してしまう。 
\\	で「本来的(根本的に)にいい」の意味になります。
\\	は「生来の、先天的に」、
\\	は「凶悪な」の意。	
\\	今日、彼らの披露宴が盛大に行われた。 
\\	「豪華な披露宴」と
\\	で、「皇室にふさわしい」を使って「盛大に行われた」のニュアンスを表現しています。
\\	(全力をあげる)	
\\	彼女自信、プロとして成長する事を望んでいます。それは自分との戦いですね。 
\\	で「プロとしての能力をさらに伸ばす」の意味になります。	
\\	何事も最初は大変ですもんね。これを乗り越えてこそ、人として大きく成長するってものだ。なんてね。 
\\	「大きく成長する」を 
\\	(よりよい人間になる)で表現しています。
\\	を使ってもいいですね。(注)「なんてね」は英訳していません。	
\\	おまえのせいで、風邪をひいた。 
\\	「~せい」は
\\	「落ち度、誤り」を使えばいいでしょう。	
\\	(靴を試しにはいて)きっと靴下のせいでちょっときつく感じるのだと思います 
\\	で「きっと... のせいだ」のニュアンスが出せます。	
\\	冬になると静電気が起きていやだ。 
\\	「静電気」は
\\	といます。
\\	は省略できます。	
\\	「せーの」で持ち上げるよ。いい、いくよ? せーのーで! 
\\	は「アーンド」と伸ばして言います。
\\	は動詞で「よいしょと持ち上げる」という意味ですが、ここではかけ声として使っています。	
\\	ちょっと頭の中を整理させて。 
\\	で
\\	を整理する、きちんとする」の意味。
\\	「整理する、まとめる」。	
\\	このまま話していても埒があかないから、先ず話すべき事を整理しないか? 
\\	は「整理する、分類する」、
\\	は「組織化する、まとめる」という意味。	
\\	身の回りのものをきちんと整理整頓しなけりゃだめだよ。 
\\	「整理整頓」は
\\	で簡単に表現できます。
\\	「個人の持ち物」の意。	
\\	なんだかあの人、生理的に受け付けない。 
\\	の文字通りの意味は「…が本能的(反射的)に嫌い」です。	
\\	あそこの店、近々、セールをする予定ないかな。今度聞いてみよう。 
\\	「セールする」は 
\\	で表現できます。	
\\	やりたいことがある時は思い切ってやるべき。多少のリスクを背負ってもね。 
\\	で「ある程度の危険を負う(引き受ける)」の意味になります。	
\\	あの一帯は世界遺産に認定されています。 
\\	「世界遺産に指定された場所」は
\\	といいます。	
\\	英語が話せると世界が広がった気がするね。 
\\	「(世界が)広がる」は
\\	、あるいは、
\\	を使えばいいでしょう。	
\\	世界記録保持者が、オリンピックで勝てるとは限らない。 
\\	「世界記録保持者」は
\\	と言います。
\\	すると保証されている」。	
\\	ボクシング世界ヘビー級チャンピオン、レノックス・ルイス!「世界最強の右」でマイク・タイソンを倒せるか? 
\\	「世界最強の右」はそのまま 
\\	と言います。「最強」は 
\\	でもいいでしょう。	
\\	子供にせがまれて、昨日、ディズニーランドに行った。 
\\	~で
\\	にうるさくせがんで~させる」という意味になります。	
\\	席替えしたので、心機一転!新しい環境で仕事できる。 
\\	この場合の「席替え」は
\\	と
\\	を組み合わせればいいでしょう。
\\	は「改良する、改修する」の意。	
\\	お花見の席取りも楽じゃない。 
\\	は「杭で印を付ける」の意味。
\\	でも「場所取り」を表現できます。	
\\	責任逃れするなんてズルイよ。 
\\	で「責任逃れをする」の意味。
\\	は「いい加減な、浅薄な」	
\\	彼の仕事に対する責任感は半端でない。感心するよ。 
\\	「責任感」は
\\	ですね。
\\	「きまぐれな、いい加減な」。
\\	「賞賛する」。	
\\	彼が今回のプロジェクトの責任者に選ばれた。 
\\	動詞
\\	には「率いる」という意味があります。
\\	は「…の担当の、責任者である」の意。	
\\	自分の失敗を他人に責任転嫁するなよな。 
\\	で「…のせいにする」という意味になります。	
\\	この場におよんで責任逃れをするな。 
\\	で「責任を回避する」の意味になります。
\\	は「かわす、避ける」の意。	
\\	人に責任を押しつけるなよ。 
\\	「責任を押しつける」は
\\	を使えばいいでしょう。
\\	は「はぐらかす、自分でやるべきことを人に回す」の意。	
\\	これは完璧な医療ミスであって、医者なり病院側が責任を取るべきだ。 
\\	で「…の責任を取る」の意味。
\\	は「紛れもない」の意。	
\\	田中は今席を外しております。 
\\	「席を外している」は
\\	、あるいは、
\\	で表現できます。	
\\	あの高校生、前におばあさんが立ってるんだから、席を譲ればいいのに。 
\\	「譲る」は 
\\	を使って表現できます。
\\	は「かなり年配の」の意。	
\\	世間様に申し開きができない。 
\\	「世間様」は
\\	でいいでしょう。ここでは「申し開きができない」を
\\	「理解してもらうのは無理」で表現しています。	
\\	いつまでも無職でいると、世間体が悪い。 
\\	で
\\	を高く評価しない」の意味。
\\	「ぐうたら」。	
\\	近所のおばちゃんたちは、毎日世間話に花を咲かせています。 
\\	が「世間話」に当たる表現です。
\\	は「(退屈などを)紛らす」の意。	
\\	ポケットから手を出して、背筋を伸ばして歩きなさい。 
\\	は「背筋を伸ばして立つ」ということです。
\\	の使い方にも注目。	
\\	せち辛い世の中になったものだ。 
\\	ここでは「せち辛い」を
\\	「(ジャングルのように)生存競争が大変な場所」、
\\	「冷酷で、厳しい世の中」で表現しています。	
\\	せっかく休み取ったのに、今日行けないの。 
\\	「せっかく... する」は
\\	で表現できます。	
\\	せっかく、あなたのために朝早く起きて作ったお弁当なのに、食べられなかったってどういうことなの! 
\\	「わざわざ 
\\	する」が「せっかく」にあたる言い回しです。
\\	を手を付けないでおく」。	
\\	せっかく忙しい中、朝に髪の毛セットしたのに、この天気でもうめちゃくちゃだわ。もう、最悪! 
\\	で「とても忙しいにもかかわらず」の意味になります。
\\	は「わざわざ…する」の意。	
\\	せっかく充電したのに、携帯電話持ってくるの忘れた。これじゃ、意味ないよ。 
\\	で「わざわざ…する」という意味になります。
\\	にも同じような意味があります。
\\	は「無駄」の意。	
\\	アメリカで
\\	に一ヶ月通って折角耳が慣れたのに、日本に帰ってたら元に戻ってしまった。 
\\	「丁度した時に」と
\\	のコツを掴む」を組み合わせて「折角慣れる」のニュアンスを表現しています。	
\\	せっかくの人生、自分のやりたいことをやりたいけど、他人に迷惑かけてまではやりたくない。 
\\	「せっかくの人生」を 
\\	(一度しか生きられない)で表現しています。
\\	は「…を侵害する」の意。	
\\	海外旅行して、せっかく英語を話す良いチャンスだったのに、あまり話しかけられなかった。 
\\	「せっかくのチャンス」は
\\	「素晴らしいチャンス」と言えばいいでしょう。
\\	「始める」。	
\\	私は、接客業より事務職の方が向いていると思う。自分のペースでできるから。 
\\	「接客業」は
\\	でいいでしょう。
\\	「自分のペースで」。
\\	「管理上の」。	
\\	このウォークマンたまに音が聞こえないんだけど、どっか接触が悪いのかな? 
\\	「接触」は 
\\	ですね。
\\	は「すり減る」の意。	
\\	今日の試合は接戦だった。 
\\	には「互角の」という意味があります。
\\	は「接戦(互角の試合)」の意。	
\\	外国のクライアントを接待したんだけど、英語が通じなくてもう大変だったよ。 
\\	(楽しませる、もてなす)で「接待する」が表現できます。	
\\	彼女の発言には説得力がある 
\\	「説得力がある」は動詞の
\\	の形容詞形である
\\	を使います。	
\\	6時17分って、また中途半端な時間にアラームセットしているんだね。 
\\	「セットする」はそのまま
\\	が使えます。
\\	「変な、妙な」。	
\\	セットを飲み物はコーラで下さい。それと単品でイチゴシェイクを一つ下さい。 
\\	と
\\	の使い方がポイントです。
\\	「添え物として」。	
\\	いよいよ切羽詰まってきたな。のんびりしてられない。 
\\	「せっぱ詰まる」は
\\	「窮地に陥る」、
\\	「追いつめられる」で表現すればいいでしょう。	
\\	あの映画のラストシーンには、思わず胸が切なくなりました。 
\\	「胸がせつなくなった」は「ちょっと悲しくなった」と考えて、
\\	で表現すればいいでしょう。
\\	は「包む」の意。	
\\	あの映画の主人公の刹那的な生き方にあこがれる人は多い。 
\\	「刹那的」をどう英語で表現するか。
\\	とは「太く短く生きる人生」に当たる表現。また、
\\	は「その瞬間を生きる」の意味を表す。
\\	「賞賛する」、
\\	「夢見る」。	
\\	彼が無事でいることを、切に願う。 
\\	(心から)、
\\	(熱心に)と 
\\	を組み合わせて「切に願う」を表現しています。	
\\	今月の給料は少なかったから、食費を削ったりしてちょっと節約しないといけないな 
\\	で
\\	を少し切りつめる」の意味。	
\\	2人だと材料買って家で作るより、外で買って食べたほうが安いことが多いです。時間の節約にもなりますし。 
\\	で「時間を節約する」という意味。
\\	は「時間を節約すること(手段)」の意。	
\\	栄光と挫折は、まさに背中合わせだ。 
\\	の文字どおりの意味は「コインの両面」です。
\\	は「不名誉な」、
\\	は「反対の面」の意。	
\\	なんでここだけ、こんなに狭いの?定員より多く座ってない?(電車の座席で) 
\\	で「狭苦しい」の意味になります。
\\	「詰め込む、押し込む」。	
\\	今、大学は大きな変革を迫られている。 
\\	「迫られている」とは
\\	(…するように圧力がかけられている、せかされている)ということですね。	
\\	願書申し込みの期日が3日後に迫っていた。 
\\	で「迫ってくる」の意味です。
\\	は「締め切り」の意。	
\\	蝉がミーン、ミーン鳴いている。 
\\	「蝉」は 
\\	ですね。鳴き声の表現の仕方にも注意。	
\\	せめて日常会話ができるくらいになりたい。 
\\	「せめて」は
\\	「少なくとも、最低でも」と言い換えられます。	
\\	最後の最後で自首したことが、せめてもの救いだよ。 
\\	で「我々にとっての唯一の慰め」の意味になります。
\\	は「自首する」の意。	
\\	それはこっちのセリフよ。 
\\	が「こっちのセリフ」にぴったりの言い方です。
\\	「一行(ことば)」。	
\\	この度はいろいろとお世話になりありがとうございました。 
\\	ここでは「お世話になる」を
\\	を使って表現しています。	
\\	背中が日焼けして、イスにもたれかけできないし、背を下に向けて寝れない。 
\\	「背中」は
\\	と言いますから、
\\	で「背を下にして寝る」という意味になります。	
\\	はるかかなたに、船影が見えた。 
\\	この場合の「(船)影」は
\\	、あるいは、
\\	で表現すればいいでしょう。	
\\	内閣人事は、総理大臣の専決事項だ。 
\\	で「…は個人の自由裁量による」という意味になります。
\\	で「専決」の意味。	
\\	少子化と人口の老齢化を見越して業務内容を大幅に変更したあの会社は先見の明があったわけだ。 
\\	「先見の明がある」、
\\	「適応させる、調節する」、
\\	「社会の老齢化」、
\\	「人口統計学上の変化」の意。また、「先見の明がある」を... 
\\	で表現している点にも注意。	
\\	先行オールナイトを見にいったのに、途中で寝てしまって、気づいたらエンディングが流れていた。 
\\	で「オールナイトで行われる映画の先行上映」の意味になります。	
\\	西暦2000年、2001年という文字通り千載一遇の歴史的な節目を迎えた今、人々の関心は来る21世紀の方に向いてばかりいる。 
\\	「千載一遇」をここでは
\\	「めったにない経験(百万回の内一回)」で表現してみました。
\\	「歴史的節目」、
\\	「熱心に」、
\\	「しっかりと」。	
\\	先生!質問があります。 
\\	英文の後に簡単な説明をつけておきました。「先生」と呼ぶ時は、相手の名前を言うのが普通です。小学校では
\\	と呼ぶこともあります。	
\\	先祖供養をして因縁を解く。 
\\	「(先祖を)供養する」をここでは
\\	「祈る」、
\\	「崇拝する」、
\\	「敬う」で表現しています。
\\	「先祖」、
\\	「悪い業(因縁)を洗い清める」。	
\\	こう雨ばかりじゃ、洗濯物が乾かないわ。 
\\	「洗濯物」は
\\	ですね。
\\	という言い方もあります。	
\\	先着10名にもれなく粗品プレゼント。ついついこういうのに引かれるよな。 
\\	で「先着10名様」の意味になります。
\\	「引きつけること(名詞)、引きつける(動詞」)。	
\\	この商品は先着順に販売しておりますので、お買い求めの方は列にお並びください。 
\\	で「先着順で」の意味になります。
\\	「列を作る」。	
\\	ちくしょー、あの子狙ってたのに、あいつに先手を打たれちまった! 
\\	には「出し抜く」という意味がありますので、これを使えば「先手を打たれる」を表現できます。
\\	は「先を越される」の意味。	
\\	あの広告宣伝効果抜群だよね。消費者の心をとらえている。 
\\	「宣伝効果」は
\\	「広告」と
\\	「効果的」を組み合わせて表現すればいいでしょう。
\\	「消費者」。	
\\	彼は共産主義に洗脳された。 
\\	「洗脳する」は 
\\	と言います。
\\	は「共産主義者」の意。	
\\	あいつ先輩面してる。 
\\	「先輩(の社員)として」と
\\	で「その役を演じる」で「先輩面してる」を表現しています。	
\\	この電車は次発です。 先発の電車は、
\\	駅には止まりません。 
\\	「次発」は 
\\	、「先発」は 
\\	で表現できます。(注)
\\	のような表現は避けてください。	
\\	あのデパートには介護用品専門店が入ってるよ。 
\\	「専門店」は
\\	(…を専門にしている店)、あるいは、
\\	でいいでしょう。	
\\	このコメンテーターの解説は、経済の専門用語ばかりでよくわからない。 
\\	で「経済の専門用語」の意味になります。	
\\	今週の金曜日は先約が入っていて、飲み会には行けないな。 
\\	「先約」は
\\	「すでに計画が入っている」と言えばいいでしょう。	
\\	リ-ダ-は先憂後楽の心情に徹すること。 
\\	は「信念を待つ」、
\\	は「苦難に耐え」、
\\	は「利益、恩恵」の意味。	
\\	インターネットの専用線を入れたから、電話代を気にせずにネットサーフィンが楽しめるよ。 
\\	「専用線」のことを
\\	と言います。	
\\	今年の新人は、戦力になる奴が一人もいない。 
\\	「プラスになる(役に立つ)もの」で「戦力」を表現しています。	
\\	家賃て税金かかるの? 
\\	この場合の 
\\	は動詞で「課税する」の意味です。	
\\	贅沢な時間を過ごす。 
\\	で「贅沢に」の意味。	
\\	それはぜいたくな悩みだよ。 
\\	「贅沢な」は表現しなくても、
\\	「そういう問題を持ってみたいよ」と言えば「贅沢な悩みだよ」の意味になります。	
\\	日本を懐かしむことはあまりありませんが、日本でいかに自分がぜいたくに暮らしていたのだなと思うことがあります。 
\\	はともに「ぜいたくな」という意味の形容詞です。	
\\	あなたお腹の周りに贅肉がついてきたんじゃないの。 
\\	(太い)と
\\	(お腹の周り)の使い方がポイントですね。	
\\	お前なんかとは絶交だよ。顔も見たくないよ。 
\\	で
\\	とは終わりだ(絶交だ)」の意味になります。	
\\	絶好調! 
\\	とは「これ以上にないほどいい」の意味。
\\	「最高の状態」。	
\\	絶対それ聞かれると思った。 
\\	「絶対」は
\\	「分かってた」、
\\	「確信してた」で表現できます。	
\\	このリゾット、この前来た時に食べてすごく美味しくて、今度来た時は絶対買って帰ろうと思っていたんですよ。 
\\	ここでは「絶対…する」を
\\	「…すると心に決める」、
\\	で表現しています。	
\\	小さい時からピアノを引いていたから、彼女には絶対音感がある。 
\\	が「絶対音感」に当たることばです。	
\\	彼女は、当時独裁者として絶大な権力を振るっていた。 
\\	で「絶大な権力を振るう」の意味になります。
\\	は「巨大な」の意。	
\\	7万円か・・・ 0が一個多いなあ
\\	一個少なければなあ)・・ 
\\	「ゼロが一つ多すぎる」は
\\	と言います。「少なければ」は
\\	を使って表現します。	
\\	めずらしく今日は全員参加のようだね。 
\\	で「皆、現れた、やって来た」の意味。	
\\	昨日の地震で5つの家屋が全壊した。 
\\	で「完全に破壊する」という意味。	
\\	全身筋肉痛だ。無茶するんじゃなかた。 
\\	「全身」は
\\	でいいでしょう。
\\	「頭のてっぺんからつま先まで痛む」。	
\\	善戦はしたのですが、結局は負けてしまいました。 
\\	「善戦する」とは
\\	「いい戦いをする」ということですね。
\\	でも同じような意味になります。	
\\	うちの子供、喘息がひどくてね。発作が起こると大変なの。 
\\	「喘息」は
\\	と言います。
\\	「ひどい」、
\\	「発作」。	
\\	この組織変更は、個別最適でなく全体最適をめざしたものです。 
\\	は「最高にする」の意味。これと
\\	「全体」、
\\	「個々の」を組み合わせて「全体最適、個別最適」を表現しています。	
\\	色々な色が混じっていますが、全体的なイメージはオレンジなんです。 
\\	「全体的な」は 
\\	で表現すればいいでしょう。	
\\	そういう意味で言ったんじゃなくて。 
\\	「そういう意味」は
\\	と
\\	を組み合わせれば表現できます。	
\\	わかった。わかった。そう言うことにしといてあげるよ。(誰かが、自分の事を自慢したり、自分のおかげで、物事が成功したんだと言った人に対して) 
\\	は「何でもお前の言う通り」の意味。	
\\	ごめん!そういうつもりで言ったんじゃないんだけど。 
\\	「そういうつもり」は
\\	「意味する、意図する」を使って表現します。	
\\	一度悪事が明るみになると、世間はそういう目でしか見なくなる。 
\\	で「(そういうイメージが)人々の目に焼き付いてしまう」の意味になります。
\\	の動詞としての使い方に注目。	
\\	どうしてって言われたって、そういうものなのよ! 
\\	「そういうもの」は、
\\	「物事はそういうものである」で表現します。	
\\	コンピュータっていきなり壊れたりするからね。ま、でもそういうものかもしれないね。 
\\	を受け入れる、一緒に生きていく」で「そういうもの」のニュアンスを表現しています。	
\\	バスって後ろから乗るものだとばかり思っていたのですが、そうとも限らないのですね。 
\\	「そういうものだと思っていた」を 
\\	と 
\\	を組み合わせて表現しています。
\\	は「乗車する」の意。	
\\	先のプロジェクトは、資金不足のためお流れとなりました。そういうわけで、新しいプランを準備しなければなりません。 
\\	「そういうわけで」は
\\	また、
\\	を使えばいいでしょう。	
\\	そう言うんじゃないかと思って、ちゃんとカメラを持ってきたよ。 
\\	と
\\	の使い方がポイントですね。	
\\	そういえば、最近あいつの姿を見かけないな。どうしたんだろうね? 
\\	「そういえば」は
\\	だから」と
\\	「話に出す」を組み合わせて表現すればいいでしょう。	
\\	そう言われてみれば、そうだね。 
\\	この場合の 
\\	は「言葉にする」という意味になります。	
\\	値段相応のホテルだった。 
\\	で「ほどよい(適切な)料金のホテル」の意味になります。	
\\	会話のなかでまったく知らない単語に遭遇すると、そこで思考が停止してしまう。 
\\	で「まったくなじみのない語に出くわす」の意味になります。
\\	は「遭遇する」の意。	
\\	このプロジェクトの成否は君の双肩にかかっている。 
\\	が「双肩にかかっている」に当たる表現です。	
\\	その英語試験では、読んで、書いて、会話する総合力が要求される。 
\\	「総合力」は
\\	「全部の、総合的な」と
\\	「能力」を組み合わせればいいでしょう。	
\\	この事件で、捜索個所は10カ所にも及んだ。 
\\	この場合の「捜索個所」は
\\	でいいでしょう。
\\	は「捜索、調査」の意。	
\\	最近の掃除機って、音も静かだし、臭いもあまりしないんだってね。 
\\	「掃除機」は
\\	ですね。
\\	「悪臭で満たす」。	
\\	帰ってきてそうそううるさいな。こっちは、疲れてるっちゅうの。 
\\	この場合の「そうそう」は
\\	(…したとたん)で表現できます。
\\	(悩ます)。	
\\	早々にご返事いただきましてありがとうございます 
\\	は「即刻の、迅速な」という意味です。	
\\	想像で話をするのって難しい。 
\\	で「想像だけに基づいて」の意味になります。	
\\	父が急に倒れて、容態があまりよくないらしいので早引きしたいのですが。 
\\	「早引き」は
\\	「早く帰る」と言うことですね。
\\	「倒れる」。	
\\	今日は気分が良くなかったので、会社を早退した。 
\\	で「早く帰る許可を得る」の意味になります。	
\\	学校の先生は、担当教科を教えるだけでなく、いろんな悩み事の相談にものってくれる 。 
\\	「相談にのる」をここでは
\\	「耳を貸す、話を聞く」で表現しています。	
\\	誰かの結婚式でもあれば集まったりするけれど、そうでもない限り集まることなんてないね。 
\\	以下に注目してください。「他の理由では、そうはならないだろう(皆が集まらない)」と言っているのです。	
\\	この短期集中コースは、2単位の授業に相当します。 
\\	「相当する」は 
\\	(…と同等である)で表現すればいいでしょう。	
\\	そうとも言うね。 
\\	が「そうとも言うね」に当たる言い回しです。	
\\	昔は、ゴルフは金持ちのスポーツだと言われていましたが、今は手軽にプレイできるコースも増えそうとも言えないと思います。 
\\	で「必ずしもそうではない」の意味。
\\	でも同じような意味になります。	
\\	へー、そうなんだ。私はてっきりあなただと思ってた。 
\\	ここでは「そうなんだ」をこの文脈に合わせて 
\\	を使って表現しています。	
\\	この辺の相場はわからないけど、友人は一日3食付で一ヶ月6万円払っているって言ってた。 
\\	は「時価、現行の価格」という意味ですね。
\\	「標準価格」。	
\\	そうまでしてタバコ吸いたいかね? 
\\	と 
\\	の使い方がポイントです。	
\\	抜本的な改革案も、各部署が総論賛成各論反対だったので、一向に実現しなかった。 
\\	の文字通りの意味は「原則としては…に賛成、細かい点については反対」となります。	
\\	私が遠くの町に引っ越してから、彼女とは疎遠になってしまった。 
\\	ここでは「疎遠になる」を 
\\	を用いて表現しています。
\\	の使い方にも注目です。	
\\	彼は、じっくりと腰を据えて研究に打ち込んだ結果、その分野に多大なる足跡を残すことができた。 
\\	は「名を残す」という意味。
\\	でも同じような意味になります。
\\	は「…に専念する」の意。	
\\	わが社は、もう新人から時間をかけて育てる時間的・金銭的余裕も無いので、即戦力を採用しよう。 
\\	「即戦力」とは、
\\	「入ってきて、いつでもいい仕事ができる」ということですね。
\\	「新人(新しく雇った人)」。
\\	「競技にすぐ参加する」。	
\\	高級車っていくらスピードを出しても速度感を感じない。 
\\	で「スピードを感じない」の意味になります。
\\	は「高級車」の意。	
\\	彼は即物的だから、好きになれないんだ。 
\\	「即物的」とは
\\	(現実的)ということですね。	
\\	彼は、プロ野球選手としての現役時代、そこそこ活躍した。 
\\	「そこそこ」は
\\	で表現できます。	
\\	そこでしか学べない何かを学びたい。まだ、今はそれが何だかわからないけど。 
\\	「そこでしか」は
\\	と
\\	を使って表現します。	
\\	あいつは飲んだら底なしだ。 
\\	は「大酒を飲む」、
\\	は「相手を酔いつぶすまで飲む」の意味。	
\\	そこのところをうまく伝えてくれ 
\\	「そこのところ」とは
\\	「その個所、部分」ということですね。	
\\	彼女がそこまで彼を好きだって言うなら、友だちとして応援すべき? 
\\	「そこまで」のニュアンスを 
\\	で表現しています。
\\	は「応援する、協力する」の意。	
\\	「あの人ってほんとに仕事できないの。一日も早く辞めてほしいわ」「そこまで言うか」。 
\\	は「そこまでは言わないよ(そんなに遠くまでは行かない)」の意。
\\	!は「きつい事言うね」に当たる表現。	
\\	そこを何とかお願いできませんか。 
\\	「そこを何とか... 
\\	をここでは「助けていただける何らかの方法はありませんか」と言い換えています。	
\\	無理なのはわかっているのですが、そこをなんとかしていただけないでしょうか。 
\\	で「何か方法はないでしょうか」の意味になります。	
\\	このお料理、素材の味が生きていてとってもおいしいわ。 
\\	「素材」は
\\	と言います。
\\	「上質の」。	
\\	彼には素質があるが、練習をしないからゴルフが上達しないんだ。 
\\	(…になるために必要な物)で「素質」を表現しています。	
\\	きみの行為は、軽率のそしりを免れない。 
\\	と 
\\	(うまく逃げる)を組み合わせて「そしりを免れない」のニュアンスを表現しています。
\\	は「結果を直視する」の意。	
\\	うなじにそそられた女性は過去にひとり。 
\\	「そそられる」をここでは
\\	「本当にエロチックな(性欲を刺激する)」で表現しています。
\\	「うなじ」、
\\	「しびれさせる」の意。	
\\	何この部屋?汚い!!本当にだらしがないね。育ちがわかるわ。 
\\	で「どのようにあなたが育てられてきたかがわかる」の意味になります。
\\	は「しつけがわかる」の意。	
\\	先ほどそちらに電話したら、赤は在庫がないと言われましたけど? 
\\	「そちらに電話する」は 
\\	、あるいは、
\\	でいいでしょう。
\\	は「…がない」の意。	
\\	あなた、ジュリア・ロバーツにそっくりって言われない? 
\\	とは「そっくりな人」のことです。
\\	も使えます。	
\\	今日は、仕事が終わったら、速攻で帰るぞ。 
\\	は
\\	した瞬間に」、
\\	「きっかりに」の意味。	
\\	夏休みに入った途端、うちの息子達は勉強そっちのけで遊んでいる。 
\\	という考えが消えてなくなる」で「そっちのけで」のニュアンスを出しています。	
\\	お客のニーズにそって、システムを作るのって面白いけど大変だ。 
\\	は
\\	にしたがって」の意味になります。
\\	「とても面白い」。	
\\	彼、落ち込んでいるみたいだから、しばらくそっとしておいた方がいい。 
\\	には「そっとしておく、構わないでおいておく」という意味があります。	
\\	あいつはそつがないよな。 
\\	ここでは「そつがない」を
\\	「欠点がない、完全な」で表現しています。	
\\	卒業見込み証明書を、 発行していただきたいのですが? 
\\	が「卒業見込み証明書」に当たる表現です。	
\\	つなくこなすってのはむずかしいなあ。 
\\	「そつなくこなす」は
\\	「落ち度のない仕事をする」ど表現します。
\\	の代わりに
\\	「均一の、継ぎ目のない」を使ってもいいでしょう。	
\\	お袖通すからこっちのおててかして。はい、反対側。ほらできた。 
\\	「袖に手を通す」は 
\\	で表現すればいいでしょう。
\\	の代わりに 
\\	も使えます。	
\\	うちの主人って、外面がいいだけ。うちではひどいものよ。 
\\	「ふりをする」と 
\\	「出掛ける時」を組み合わせれて「外面」を表現しています。	
\\	そのうち一杯やりましょう。また、連絡しますよ。 
\\	「そのうち」は
\\	、あるいは、
\\	でいいでしょう。	
\\	最近あまりついてないけどそのうちいいことあるよ!元気出してね! 
\\	が「そのうち」に当たる言い回しです。	
\\	今日は、浜辺で寝て過ごした。そのおかげで、とてもリラックスした時間を過ごせた。 
\\	「…のおかげで」は 
\\	を使って表現できます。関係代名詞 
\\	の使い方にも注意。	
\\	原則、全員に支払っていただきますが、既に寄付した方についてはその限りではありません。 
\\	この場合の「その限りではありません」は 
\\	(以外は)を使って表現すればいいでしょう。
\\	は「寄付」の意。	
\\	今日の夕食は、一緒に食べられませんが、その代わりに、明日の昼食を一緒に食べましょう。 
\\	「その代わりに」は
\\	で簡単に表現できます。	
\\	いろいろ技術が進歩して、世の中便利になって、その結果人々は楽な方を選ぶようになってしまった。 
\\	「その結果」は
\\	、あるいは、
\\	を使って表現すればいいでしょう。	
\\	君のその言葉を、ずーっと、待っていたんだ。 
\\	「その言葉」は
\\	(複数形)、
\\	(単数形)を文脈に応じて使い分けるようにします。	
\\	あなたの、その言葉を、待っていました。 
\\	最初の英文は「あなたがそれを言うのを期待していた」、2番目は「それが私が聞きたいと思っていた言葉だ」という意味。	
\\	残業したら、した分だけお金はもらえるので、その点では良いと思う。 
\\	で「その点では」の意味になります。	
\\	来週伺えると思うので、その時にパンフレットをいただきます。 
\\	「その時」は
\\	、あるいは、
\\	を使えばいいでしょう。	
\\	その時になったら、必ず知らせてね! 
\\	「その時になったら」はそのまま
\\	、あるいは、
\\	で表現できますね。	
\\	上司に相談しても、その場は「はいはい。わかった」ってその場しのぎの返事で、結局何もしてくれない。そういう人なんだよ。 
\\	ここでは「その場しのぎ」を
\\	を片づけるために、関心を持っているような振りをする」で言い換えて表現しています。	
\\	いつも利用しているクリーニング屋は、朝の10時までに出せば、その日の夕方にはできている。 
\\	で「その日に仕上げるサービスをしている」という意味になります。	
\\	なんかその日暮らしって感じで、羨ましい。 
\\	「その日暮らし」はそのまま
\\	で表現できます。	
\\	その日は、午前中からゴルフに行く予定なんだ。だから、夕方にそちらに行けるかどうかは微妙です。 
\\	「その日は」は 
\\	と前置詞 
\\	を使います。
\\	の使い方に注意。
\\	は「あやふやな」の意。	
\\	電話で問い合わせしたら超過請求したのでその分戻してくれるって言ったのに、どうも口だけだったみたい。 
\\	「その分」を 
\\	(その額)で表現しています。
\\	は「払い戻す」、
\\	は「過剰に請求する」の意。	
\\	お金があると使ってしまうタイプだから、そのへんの管理はしかっりしておかないと。 
\\	「そのへん」を
\\	「範囲、領域」、
\\	「傾向」を使って表現しています。	
\\	これこのまま食べる?それとも温める? 
\\	「そのまま」は
\\	、あるいは、
\\	で表現すればいいでしょう。	
\\	昨日の夜こたつに入ってテレビを見てたら、そのまま朝までこたつで寝てしまった。 
\\	「そのまま」は 
\\	で表現すればいいでしょう。
\\	を使ってもいいですね。	
\\	そのままにしておいて下さい。自分で何とかしますので。 
\\	で「そのままにしておく」の意味。	
\\	若い頃って、食べたらそのまま身になっていたから。 
\\	「そのまま身になっていた」とは
\\	「食べたものが全て私が太るのを助けた」、
\\	「食べたもの全てが私の肋骨についた(血となり肉となる)」ということですね。	
\\	お前、それ、そのままやないか。 
\\	最初の英文は「それ説明のつもり?」、次は「オームだってそんな説明できるよ」の意味になります。これで「そのままやないか」の意味にかなり近くなるのでは?	
\\	実際やってみることによって、そのものの面白さ、難しさが分かる。 
\\	「そのものの(面白さ、難しさ)」を
\\	で表現しています。	
\\	その男は彼女の前なので見栄を張って、自分がワインにとても詳しい素振をした。 
\\	「素振りをする」は
\\	か
\\	を使えばいいでしょう。
\\	「けばけばしさ」、
\\	「派手な身振り」、
\\	「愛好者、マニア」。	
\\	物を粗末にしてはいけないよね? 
\\	で「物を軽く扱う」の意味になります。
\\	「何も考えずに)	
\\	木の葉が風にそよいでいる。 
\\	で「風でかさかさ音がする」の意味になります。	
\\	野球のことに関してはあまり詳しくない。それ以前に興味がない。 
\\	「それ以前に」のニュアンスをここでは
\\	「少しも 
\\	にない(強調)」で表現しています。	
\\	私、隣にどんな人が住んでいるのか知らない。それが、マンションというものなのかな。 
\\	「それは~というもの」は
\\	で表現できます。
\\	と言えば「それが人生というもの」の意味になります。	
\\	日本語を話したら罰金らしいけど、それぐらいの方が真剣に英語を勉強したい私にはもってこいの環境だ。 
\\	で「それくらいの(環境)」の意味になります。	
\\	あゆはもうお姉ちゃんなんだから、それぐらいのことは我慢できるでしょ? 
\\	この場面では「それぐらいの事」を簡単に
\\	で表現すればいいでしょう。	
\\	この人、さすがにジェットコースターを作ったというだけあって、それ系の乗り物は全然平気なんだろうな。 
\\	「それ系」は
\\	(その種の物)で表現できます。	
\\	それぞれの理由がある。 
\\	で「すべてに理由がある」の意味です。
\\	を主語にして言いかえてもいいでしょう。	
\\	朝御飯代わりに、甘い物食べることもあるよ。それだけ好きってこと。 
\\	「それだけ好き」は
\\	でいいでしょう。	
\\	今はピンとこないけど、あとで自分で考えてみます。それでわからなかったらまた聞きます。 
\\	「わからなかったら」は
\\	で表現すればいいでしょう。
\\	は「ピンと来る」の意。	
\\	彼女は遊ばれてるのも利用されているのもわかっているよ。それでも彼と別れたくないんだって。 
\\	「それでも」を 
\\	で表現しています。
\\	は「別れる」の意味。	
\\	今おなかが痛くてそれどころじゃないんだよ。 
\\	「それどころじゃない」を「他の誰かに聞いてくれ」、「僕のところに来てもだめだ」と言い換えているところがポイントです。
\\	とは「助けられるような立場(状況)にない」の意。	
\\	ごめん! 今日夕飯準備してないんだ。疲れて、それどころじゃなかったの。 
\\	「それどころじゃない」は
\\	を使って表現できます。	
\\	それならそれで私にも考えがある。 
\\	で「そういうことなら」の意味になります。	
\\	昔は、それなりに特別な日には、パーティーとかやったけど、最近は何もしないな。 
\\	で「ちょっとした(どうでもいいような)口実をつけて」で「それなり」を表現してみました。	
\\	あんな車でも、それなりに役に立っているんだ。 
\\	は「その車なりに」、また、
\\	は「そんなにすごいってわけじゃないんだけど(それなりに)」の意味。	
\\	それにしても、今年は花粉が多いね。去年まで平気だったのに、おかげでひどいもんよ。 
\\	この文の前にどのような話をしたのかわかりませんので、取りあえずここでは「それにしても」を
\\	を使って表現しています。
\\	は「花粉」(不可算名詞)の意。	
\\	それはさておき、本題に移ります。 
\\	「さておき」のニュアンスを 
\\	(棚に載せる)で表現しています。
\\	は「重要ではない」の意。	
\\	それはそうとさぁ、彼女の噂聞いた? 
\\	「それはそうと」は 
\\	(ところで)、あるいは、
\\	で表現すればいいでしょう。	
\\	それはそれ、これはこれ 
\\	まず、注意するのは「これ」と「それ」の語順。日本語では「これ、それ」になりますが、英語では
\\	の順で使うことが多いので、「それはそれ、これはこれ」は
\\	と言うのが自然です。もちろん、話しの流れの中で
\\	と言うこともあります。ちなみに、
\\	のような言い方はしません。(注)「あれこれ」- 
\\	「あちこち」- 
\\	「遅かれ早かれ」のように日本語と英語の語順が入れ替わることも覚えておくといいでしょう。	
\\	あの立場にいる人がそれを言っちゃおしまいだよ。 
\\	この場合の「おしまい」は
\\	「自殺行為」と言い換えて表現します。「その立場にいる人」は
\\	と言います。	
\\	それを見た彼女は、ただぼうぜんとその場に立ちすくんでいた。 
\\	「それを見た…」は 
\\	で表現すればいいでしょう。 
\\	は「麻痺する」、
\\	は「展開された」の意。	
\\	この部署は、そろいもそろって使えない奴ばかりだな。 
\\	で「ひとりひとりが」の意味。
\\	(一人もいない)の使い方にも注意。	
\\	そろそろ失礼します。帰りの電車がなくなりますので。 
\\	で「そろそろ失礼します」のニュアンスを出しています。	
\\	同じ値段なら、大きい方買わないと損じゃない? 
\\	で「悪い買い物(取り引き)をする」の意味になります。
\\	は「損する」の意。	
\\	遠慮しないで。こんなおいしいケーキ食べないと損よ。 
\\	この場合の「損」は 
\\	(損失、損害)でいいでしょう。
\\	は「遠慮する」の意。	
\\	彼は、リーダー的存在です。 
\\	「存在」の意味では
\\	や
\\	は使いません。そのまま
\\	動詞を使って表現すればいいでしょう。	
\\	あの人今日来てたんだ。存在感ないから全然気がつかなかった。 
\\	で「一目を引かない存在」という意味になります。	
\\	昨日カナダの現地口座を開設してお金を送金したんだけど、半年前と比べると10円も円安になっていて、なんか損した気分でした。 
\\	には「(取り引きなどで)損をする」という意味があります。
\\	は「割を食う」の意。	
\\	お客さん、これはそんじょそこらにある品物とは違うんですよ。 
\\	で「めったにない品物」の意。
\\	「出くわす」。	
\\	なんだ!全然空いているじゃん。急いで損した。 
\\	で「何の得にもならないことのためにえらく急いでしまった」と言っているのです。	
\\	あなたにとってはくだらないものでも、私にとっては大切なものなの。人の価値観は尊重すべきだよ。 
\\	は動詞(尊重する)、名詞(尊重、敬意)の用法があります。
\\	は「価値観」の意。	
\\	どこにそんなお金があるんだよ。 
\\	「そんな」は 
\\	(その種の…)で表現すればいいでしょう。	
\\	君にそう強く言われるとそんな気がしなくもないけれど。 
\\	「そんな気がしなくもない」を
\\	「賛成しないとは言えない」と言い換えています。
\\	は「賛成しない訳にはいかない」の意。	
\\	何?延滞金払えって?そんな気さらさら無いね、おととい来やがれ! 
\\	で「…する意図はない」の意味になります。
\\	(豚が空を飛ぶ)で「あり得ないこと」を表しています。	
\\	そんな事を言われましても私どもでは何もできません。 
\\	この場合の「そんな事」は
\\	「そのような種類の話」ということですね。	
\\	さっきそんな事言ってた?ごめん、聞いてなかった。 
\\	「そんな事」は
\\	でいいでしょう。	
\\	そんなこと言ってたら何にもできないじゃん。 
\\	で「それに対してそんなふうなら」の意味。文字どおりに 
\\	を使ってもいいでしょう。	
\\	(お世辞を言われて) いいえ、そんなことないですよ。 
\\	例えば、
\\	とピクニックで作ったサラダをほめられたとしましょう。その時に、
\\	と答えると「そんなことないですよ」と言っていることになります。	
\\	そんな所で寝ていたら風邪引くよ。風邪引いてもしらないからね。 
\\	「そんな所」はそのまま 
\\	と表現すればいいでしょう。
\\	は「責める」の意。	
\\	そんなに急いでどうしたの? 
\\	この場合の「そんなに」は
\\	を使って表現すればいいでしょう。	
\\	じゃぁ、そんなに言うなら、みんなに聞いてごらんよ。みんなだって、きっとそう言 うはずだよ。 
\\	「そんなに言うなら」をここでは
\\	「そんなに繰り返し言うなら」と言い換えています。	
\\	そんなのって、あり? 
\\	で「そんなはずはない」の意味になります。この後に
\\	(付加疑問)を付けて「あり?」を表現しています。	
\\	宝くじなんてそうそう当たるものじゃないよ。人生そんなものさ。 
\\	で「そういうもの」の意味になる。
\\	「宝くじ」。	
\\	係長というのは、下からは突き上げられ、上からは命令され本当に損な役回りだ。 
\\	とは「(うらやむに足りない)やりたくない仕事」のことです。
\\	は「上からこき使われる」の意。	
\\	バースが阪神に復帰するって、そんなわけないやん。 
\\	が「そんなわけない」に当たる言い回しです。	
\\	夏休み期間中は、臨時電車が増発されるそうです。 
\\	この場合の「増発」は
\\	「増やす」でいいでしょう。
\\	「追加の」。	
\\	なんか背中がゾクゾクする。熱いお風呂に入ってすぐに寝よう。 
\\	「(背中が)ゾクゾクする」は
\\	といいます。	
\\	続々と投稿が寄せられています! 
\\	を使って、
\\	のように言っても「続々」を表現できますが、ここでは
\\	「途切れることなく受け取る」や
\\	「どんどん
\\	メールが届いている」の	
\\	変な人に声かけられてぞっとしちゃったよ。 
\\	「ぞっとする」には
\\	という決まった言い方があります。
\\	「変な人」。	
\\	そんなんじゃ、全然だめじゃん。 
\\	で「どうにもならない」の意味になります。	
\\	体当たりでやってみよう 
\\	とは「持てる物すべてを出してやってみろ」、
\\	は「体ごとぶつかれ
\\	の意味になります。	
\\	ふ~。彼らの体育会系ノリにはついていけないよ。 
\\	「体育会系の乗り」という発想が英語にありませんので、ここでは
\\	「あいつらにはついていけない」、
\\	「あいつらの仲間には入れない」を使って表現してみました。	
\\	今度の正月は、2000年問題に対応しないといけないんだ。だから休日返上ってことになりそうだよ。 
\\	「対応する」は
\\	「対処する」を使えばいいでしょう。
\\	「コンピュータ2000年問題」。	
\\	彼女電話だと裏声で人が変わったように対応が良いね 
\\	ここでは「対応」を
\\	「態度、物腰」を使って表現しています。	
\\	どうせ新しいパソコンを買うなら、
\\	対応してるのがいいな。 
\\	ここでは「対応」を 
\\	が使える)で表現しています。前置詞 
\\	を使ってもいいでしょう。	
\\	君は、英語の大家になる素質を持っていると思っていたが、どうも生半可な知識ばかりだね。 
\\	ここでは「大家」を
\\	「すぐれた学者」、
\\	「一流の権威」を使って表現しています。	
\\	今回のイベントが大過なく終わってほっとした。 
\\	で「1つの障害もなく」の意味。
\\	は「事故、災難」の意。	
\\	人間の体感温度がそれぞれ違うのは、住んでいる場所によって大きく影響されると思う。遺伝もあるかもしれないけど。 
\\	で「人によって熱さ寒さを感じる温度のレベルが違う」の意味になります。	
\\	ジェットコースターに乗って、時速100
\\	のスピードを体感した。 
\\	ここでは「体感する」を 
\\	(…の興奮を感じる)、
\\	(経験する)で表現しています。	
\\	あんたら、たいがいにせえよ。 
\\	ここでは「たいがいにせいよ」を「冗談もほどほどにしろ」にとって、
\\	で表現しています。	
\\	飛田課長が、太鼓判を押した企画書ならクライアントも一発OKだろう。 
\\	は「認可する」の意味。また、
\\	で「その企画書を支持している(裏には彼がいる)」と言っているのです。	
\\	今回の問題に関しては、早急に対策を練る必要がある。 
\\	で「…の解決策を考え出す」の意味になります。
\\	は「対応策」の意。	
\\	佐藤さんわざわざ御礼いいにきたよ。返って気を遣わせちゃったかな。たいした物じゃなかったのに。 
\\	「たいした」は
\\	で表現することができます。
\\	「わざわざ 
\\	にする」。	
\\	本当、大したことないよ 
\\	で「何でもないことだよ、本当に」の意。2番目の文は「皆がほめるほどのものではない」の意味になる。	
\\	お酒が体質に合わないんだ。 
\\	には
\\	が性に合う」という意味があります。
\\	「体質」。	
\\	今後また、このような問題がこのコンピューターで起きた場合、どのように対処すればいいですか? 
\\	ここでは「対処する」を 
\\	を使って表現しています。	
\\	彼は大食漢だから、人の2倍以上は食べる。 
\\	「大食漢」は
\\	と言います。「二倍以上」の表現の仕方にも注目。	
\\	3月31日付けで退職届を出すつもりだ。 
\\	「退職届」は
\\	といいます。
\\	は「退職する」の意。	
\\	僕は子どものころ、対人恐怖症で人前に出るとよくどもった。 
\\	とは「恐怖症」のことです。
\\	でも「対人恐怖症」を表現できます。
\\	は「どもる」の意。	
\\	今日の巨人の対戦相手ってどこ?どっちが勝ってるの?巨人? 
\\	で「どこと対戦しているのか」の意味になります。この場合、もちろん、
\\	は使えません。	
\\	夏ばて気味で、あまり体調が良くないんだ。 
\\	「体調があまり良くない」は
\\	で表現できます。
\\	「体の具合が良くない」。
\\	「けだるさ」、
\\	「無気力」。	
\\	体調が悪いんだからしっかり寝て、栄養のあるもの食べなくちゃだめだよ。 
\\	「体調が悪い」とは 
\\	(気分がよくない)ということですね。	
\\	彼、見た目はがっしりしているけど、よく体調を崩すよ。 
\\	ここでは「体調を崩す」を
\\	「そんなに健康ではない」で表現しています。
\\	「がっしりした」、
\\	「じょうぶな」。	
\\	紫外線はお肌の大敵です。ちゃんとケアしておかないと後で大変なことになります。 
\\	「大敵」は文字どおりに 
\\	と訳せばいいでしょう。
\\	という言い方もあります。	
\\	2、3年後には、先生と対等にジョークも交えながら英語で会話できていると思うよ。いや、そうなっていたい。 
\\	には名詞として「対等の人」の意味があります。
\\	は「…と対等に」の意。	
\\	あいつ、新人のくせに態度でかいと思わないか? 
\\	で「でかい態度をとる」の意味。また、
\\	の
\\	は
\\	のことを受けていますので、「新人らしく振る舞っていない」と言っていることになります。	
\\	人間には体内リズムというものがあって、そのリズムに合わせて活動することが成功の秘訣なんだ。 
\\	「体内リズム」は 
\\	で表現できます。
\\	は「合わせる」の意。	
\\	昨日はいきなり水道が止まってしまって、ホント散々だった。滞納してたかな? 
\\	「滞納」とは
\\	(支払期限を過ぎている)ということですね。	
\\	どこか旅行にでも行きたいけど、行っても日帰りかな。今からじゃ全部手配するの大変だし。 
\\	「大変」は
\\	、あるいは、
\\	で表現できますね。
\\	「(結果として)... になる」。	
\\	子どもだらけのパーティーって賑やかそうだけど、いろいろと大変じゃなかった? 
\\	「大変」は
\\	、あるいは、
\\	(とても疲れさせる)を使えばいいでしょう。	
\\	田中さんも大変だねぇ。子どもが熱が出たってしょっちゅう奥さんから携帯に電話かかってきて。子どもをもつと私もああなるんだろうな。 
\\	で「大変な思いをする」の意味になります。前置詞
\\	は状況を表しています。	
\\	水道の調子が悪いみたいだけど大丈夫かな。留守中に水道管が破裂でもしたら大変だな。 
\\	で
\\	したら大変なこと(大災害)になるだろう」と言っています。	
\\	大枚をはたいてパソコンを買ったが、使い道がなく、部屋の隅で埃をかぶっている。 
\\	この場合の
\\	は「大金」の意味になります。
\\	「埃」。	
\\	やっぱり授業は、インターネット方式より顔を突き合わせる対面式の方がいい。 
\\	「対面式」は
\\	(顔をつき合わせて)で表現できます。	
\\	アラブとイスラエルの対立の構図は長年に渡って続いている。 
\\	ここでは「構図」を 
\\	を用いて表現しています。
\\	は「敵対」、
\\	は「対立」の意。	
\\	私はさっぱり釣れなかったのに、隣の家族は下で誰かが魚をつけているんじゃないかって思うぐらい大漁だった。 
\\	「大漁」は
\\	「大漁である」で表現すればいいでしょう。	
\\	こんな運動もできないの?本当に体力弱いね。 
\\	この場合の「体力」は 
\\	で表現すればいいでしょう。	
\\	僕は人間関係に耐えられなくなって会社を辞めたんだ。 
\\	「耐えられない」は 
\\	で表現できます。
\\	は「駆け引き」の意。	
\\	たおやかな花が咲いている。 
\\	「たおやか」とは「しとやかな、優雅な」の意味ですから、
\\	「繊細な」、
\\	「優雅な」、
\\	「しなやかな」などを使えばいいでしょう。	
\\	具合が悪くて倒れちゃった。 
\\	は「倒れる、卒倒する」、
\\	は「気を失う」の意味。	
\\	ブランド物って高いだけあって物がしっかりしているよね。 
\\	(あなたが多分予測したとおり)と 
\\	を組み合わせて「高いだけあって」のニュアンスを表現しています。	
\\	たかが野球、されど野球。 
\\	と
\\	「ただ... に他ならない」と
\\	「それでもなお~だ」の組み合わせに注目。	
\\	たかが電気代ぐらいでケチケチするな。 
\\	「たかが」を 
\\	(小さな)、
\\	(ささいな)で表現しています。	
\\	この間、上司にビアガーデンでおごってもらったんだけど、大変なプロジェクトを任されて高くついちゃったよ。 
\\	「かなりの代償を払うことになる」で「高くつく」を表現しています。
\\	は「ただのランチはない(ただほど高いものはない)」の意味。	
\\	家で食事をするより外食したほうが高くつく。 
\\	「高くつく」は 
\\	(ずっと高い)、
\\	(もっとお金がかかる)を使って表現すればいいでしょう。	
\\	この間、初めてこのコースでゴルフをしたのですが、高台にあって眺めがよくまだ紅葉が楽しめました。 
\\	この場合の 
\\	は名詞で「高台、丘」のことです。
\\	は「紅葉」の意。	
\\	彼女は僕にとっては高嶺の花なんです。 
\\	は「達成できない夢」、
\\	は「手の届かない宝」という意味になります。	
\\	なかなか仕事が見つからないのは、高望みしすぎてるからかもしれない。 
\\	が「高望みする」に当たる言い回しです。
\\	は「野心を持ちすぎる」の意。	
\\	炊き込みご飯だとおかずがあまりいらないから助かる。 
\\	ここでは「炊き込みご飯」を 
\\	で表現しています。	
\\	昔よく、眠れない時や、気を静めたりしたい時に、よくお香をたいたな。 
\\	で「線香を燃やす」の意味になります。
\\	「静める」、
\\	「和らげる」。	
\\	あいつら今度は何を企んでいるんだ。 
\\	で
\\	をやっている、企んでいる」の意味になります。
\\	は「考え出す、作り出す」の意。	
\\	父が退職してから、我が家はたけのこ生活をしなければならなかった。 
\\	「たけのこ生活」とは「身の回りの持ち物を売って生活する」ということですから、
\\	と表現すればいいでしょう。
\\	とは「財産」のことです。	
\\	彼は竹を割ったような性格なので、時として誤解されることがある。 
\\	「隠し立てしない」、
\\	「率直な」を使って「竹を割ったような性格」を表現しています。	
\\	これじゃあ何の足しにもならない。 
\\	「足しにならない」を 
\\	で表現しています。	
\\	最後にコンサートに行ったのは、確か3年ぐらい前かな 
\\	ここでは「確か」を
\\	「~に違いない」を使って表現しています。	
\\	確かに、一度に多くのレッスンを買えば買うほど、一レッスンあたりの金額は安くなるけどね。 
\\	「確かに」は
\\	でいいでしょう。	
\\	自信がないので確かめてからご連絡します。 
\\	「確かめる」は
\\	「確認する」、あるいは、
\\	を使えばいいでしょう。	
\\	できあがった案は、結局、対立する2つの案を足して二で割ったようなものになった。 
\\	ここでは「足して二で割る」を 
\\	「2つの…の組み合わせ」で表現しています。
\\	(薄めた)、
\\	(薄めた、弱めた)を付け足している点にも注目。	
\\	彼は部下にまだまだだなとたしなめるように言った。 
\\	「たしなめる」は
\\	「しかる」、
\\	「とがめる」で表現する。	
\\	自分では気づかないけど多少なりとも上達しているのではと思う。そういうのって、自分では気がつかないものじゃないかしら? 
\\	「多少なりとも」は
\\	「幾分、多かれ少なかれ」を使って表現できます。	
\\	どこへ食べに行くか多数決取る? 
\\	この場合の「多数決」は
\\	「投票する」でいいでしょう。	
\\	前の社長は、天下りだったが、今度の社長は、現場からのたたき上げだ。 
\\	が「たたき上げ」に当たる言い回しです。
\\	「自力でたたき上げた人」。	
\\	この辺りの田園は、落ち着いた佇まいを見せている。 
\\	「佇まい」は
\\	「雰囲気」でも表現できますが、ここでは
\\	「感じ」を名詞で使ってみました。
\\	「落ち着いた、平穏な」。	
\\	昨日のウインブルドンは雨にたたられて全試合順延となった。 
\\	「雨にたたられる」は
\\	でいいでしょう。
\\	は「悩ませる(名詞で『疫病』)」の意。簡単に
\\	でも表現できますね。	
\\	サービスでワインがもらえるかわりに、アンケートに答えなければならないみたい。ただではもらえないってことね。 
\\	で「ただではもらえない」の意味になります。
\\	は「無料の」の意。	
\\	ただいま。今夜の夕食は何かな。 
\\	が「ただいま」に当たる表現です。	
\\	思い出しただけでも腹だたしい。ショックを通り越して、もう怒り爆発寸前。今度会ったらただじゃおかない。 
\\	で「ただではすませない」という意味になります。
\\	は「かんかんになって怒る」の意。(注)2つ目の和文は英訳していません。	
\\	今の私に出来ることは、ただ、ただ、あなたを信じるということだけです。 
\\	この場合の「ただ、ただ」を
\\	「私に残されていること」、
\\	「私にできる全てのこと」で言い換えています。	
\\	ただでさえ英語で話されてわからないのに、そんなに遠くからじゃ余計にわからないよ。 
\\	(…は十分に大変だ)と
\\	(今のままで、現状で)を組み合わせて「ただでさえ」のニュアンスを出しています。	
\\	結局は、ただより高い物はないってことさ。 
\\	「この世に、ただの物はない」でも同じような意味になりますが、ここでは
\\	と
\\	を使って表現しています。	
\\	あいつはただ者じゃない。あの話しぶりからすると相当コンピュータに詳しいぞ。 
\\	で「君が知っている普通のコンピュータユーザーではない」の意味になります。
\\	「ふざける」。	
\\	パソコン、もう立ち上がっていますか? 
\\	「立ち上げる」は
\\	と言います。
\\	で
\\	が立ち上がっている状態になっている」の意味。	
\\	このままでは、わが国の経済が立ち行かなくなるので、構造改革に一刻の停滞も許されない。 
\\	は「前進する」、
\\	は「安定させる」の意味。	
\\	線路に人が立ち入ったせいで、運転を一時見合わせているようだ。しかも、今のところ、再開の目途はたってないらしい。最悪! 
\\	「線路に立ち入る」は
\\	と言えばいいでしょう。
\\	の使い方にも注目。	
\\	ホームページ作りでは、あいつにはとても太刀打ちできない。 
\\	で
\\	にはかなわない」の意味になります。
\\	「競う」。	
\\	時として1%の希望は、絶望よりたちが悪い。 
\\	この場合の「たちが悪い」は 
\\	で表現すればいいでしょう。
\\	は「絶望」の意。	
\\	不安、恐怖感をたち切って、彼女は見事に栄冠を勝ち取った。 
\\	この場合の「断ち切る」とは
\\	「克服する」、
\\	「打ち勝つ」ということですね。
\\	「上手に、手際よく」、
\\	として現れる」。	
\\	あの人また立ち聞きしてる。 
\\	が「立ち聞き」に当たる表現です。	
\\	腹が減ってるんだったら、立ち食いそばでも食ってくか。 
\\	「立ち食いそば(屋)」は
\\	と言います。
\\	にちょっと立ち寄る」の意。	
\\	ものをすぐ捨てられないたちで、いつか使うだろうと思って置いていたらもう部屋中そんなモノでいっぱいになっている。 
\\	この場合の「たち」は
\\	「そういう種類の人」で表現できますね。
\\	「散らかっている」。	
\\	今回は賛成すると思われていた社長が、プロジェクト推進に大きく立ちはだかった。 
\\	ここでは「立ちはだかる」を 
\\	(…の邪魔をする)で表現しています。	
\\	今、自分がどういう立場にいるかわかってるのか? 
\\	「立場」は 
\\	、あるいは、
\\	で表現すればいいでしょう。	
\\	自販機の前で立ち話しするなよ。飲み物が買えないじゃないか。 
\\	「立ち話」は
\\	で表現できます。
\\	「邪魔をする」。	
\\	映画館は立ち見が出るほど人で一杯だった。 
\\	「立ち見する」は
\\	でいいでしょう。
\\	は「(ただ今)お立ち見です」の意味。	
\\	雑誌のコーナーは立ち読みが多くてもうけにはならない。 
\\	「立ち読みする」は
\\	あるいは
\\	で表現できます。
\\	は「利益を上げる」の意。	
\\	帰りに本屋にちょっと立ち寄っていい? 
\\	「立ち寄る」には、
\\	などの言い方があります。	
\\	先程、宅急便で資料をお送りしましたので、明日の朝にはお手元に届くかと思います。 
\\	ここでは「宅急便」を
\\	を使って表現しています。
\\	でもいいでしょう。ちなみに、アメリカでは
\\	がよく知られています。	
\\	ニューヨークに来てまだ1カ月も経っていない。 
\\	この場合の「経っていない」は完了形 
\\	で表現します。	
\\	あそこに建っている建物は、大きい。 
\\	この場合の「建つ」は
\\	でいいでしょう。	
\\	彼は健康のことを考えて、酒を断つことにした。 
\\	「断つ」とは 
\\	(あきらめる)ということですね。
\\	は「断つ」、
\\	は「心配、関心」の意。	
\\	今月の彼女の電話代、とりあえず立て替えると言っておいてくれるかな? 今度会ったときにもらうから。 
\\	(彼女の電話代を払う)で「立て替える」が表現できますね。この文脈では 
\\	も使えます。(注)和文を一部手直ししました。	
\\	日本は義理人情を重んじ、本音と建て前を使い分ける縦社会だ。 
\\	が「縦社会」に当たる表現です。「義理人情」、「本音と建て前」の英訳にも注目。	
\\	新入社員のくせに、会議の席で社長に盾突くとはね。 
\\	は「(公然と)反抗する」の意味。
\\	に立ち向かう」。
\\	「雇われ人」。	
\\	最近、日本では立続けに地震が起きているので心配です。 
\\	「立て続け」は
\\	(…の連続)で表現できます。	
\\	たて前だけで人付き合いしていたら、気疲れしちゃうよ。 
\\	ここでは「立前」を
\\	「表面的に」、
\\	「堅苦しく付き合う」を使って表現しています。	
\\	うちのボスは大変だと知りつつも、この売上の目標を毎回立てているので社員は大変です。 
\\	「売り上げ目標を立てる」は 
\\	と 
\\	を組み合わせて表現すればいいでしょう。	
\\	芸能人に喩えると誰に似ていると自分で思いますか? 
\\	で「一人選ばなくてはならないとしたら」の意味です。
\\	「喩える」。	
\\	自分のことは棚に上げて、人の悪口ばかり言うなよな。 
\\	とは「自分の問題を無視する」こと。
\\	「けなす」。
\\	「厳しい目、精査」。	
\\	会社に来るとちゅうで 
\\	「意外な授かりもの、たなぼた」、
\\	「(思いがけなく起こった)幸い」の意。	
\\	邪魔だな。もうちょっと他人の迷惑も考えてよ。 
\\	で「他の人のことも少し/もっと考える」の意味になる。
\\	「うるさい」、
\\	「やっかいな奴」。	
\\	いくら義理の親でも血がつながっていないんだから他人だよ。 
\\	で「血縁(肉親)ではない」の意味。
\\	は「姻戚」の意。	
\\	そんな他人行儀な話し方はしないでくれよ。 
\\	は「堅苦しく振る舞う」の意味になります。	
\\	いやー、他人事とは思えないね。 
\\	第1文は、「彼らだけに起こることだと思うべきではない」、第2文は「自分に起こるとは思わないだろう(でも起こるかも)」の意味。	
\\	私に任せるって。何それー。他人事みたいにこれは、2人の問題でしょう。 
\\	「他人事」は
\\	「あなたにはまったく影響を及ぼさないかのように」、
\\	「あなたにはまるで関係がない」を使って表現すればいいでしょう。	
\\	彼女は自分に甘く他人に厳しいがモットーです。 
\\	この場合の「甘い、厳しい」は 
\\	で表現できます。前置詞 
\\	の使い方にも注意。	
\\	他人のふんどしで相撲をとる 
\\	ここでは「他人のふんどしで相撲をとる」ことを
\\	「人を使って利益を得る」と言い換えています。	
\\	今の私の英語力ではあなたを十分に楽しませてあげられないかもしれないけど。 
\\	で「(人を)楽しませる」の意味になります。
\\	を使ってもいいでしょう。	
\\	プレステ2が来年の3月に発売されるそうだけど、今から楽しみだね。 
\\	「楽しみ」は
\\	、また、
\\	「待ちきれない」を使えばいいでしょう。	
\\	ねぇねぇ、たまにさ、仕事から帰るとき今日の夕飯は何かな?なんてことを考えて楽しみにしながら家に帰ることってない? 
\\	「楽しみにしながら」は 
\\	を使えばいいでしょう。	
\\	しまった!頼まれていた番組の録画セットしてくるの忘れた!怒られる! 
\\	(…することになている)で「頼まれる」を表現しています。もちろん、
\\	も使えます。	
\\	娘に頼まれて買いに来たので、詳しいことはよくわからないのですが。 
\\	「娘に頼まれる」は
\\	「娘の為に」と言い換えられますね。
\\	を使ってもいいでしょう。	
\\	今日急用で早帰りするから、この仕事頼んでもいいかな? 
\\	使役
\\	~(…に~させる)の使い方がポイントですね。	
\\	後のことは頼んだ。よろしく。 
\\	「頼む」を
\\	を頼りにする」、
\\	に任せる、がやるべき」で表現しています。	
\\	隣で煙草をすってると、服がたばこ臭くなるから嫌なんだ。 
\\	で
\\	の臭いがする」の意味。
\\	「服が臭うようになる」を使っても表現できます。	
\\	だれだよ、テーブルの上に食べかけのピザを置きっぱなしにしたのは。 
\\	が「食べかけ」に当たる形容詞です。	
\\	家の娘、食べちゃいたいくらいかわいいんだから。 
\\	「かわいい」は
\\	を使い、「食べちゃいたい」は
\\	「がつがつ食べる」、
\\	「平らげる」で表現します。	
\\	臭いと思ったら、こんな所に魚の食べ残しが落ちている。 
\\	「食べ残し、残飯」は
\\	と言います。形容詞的に使う場合には
\\	を取って
\\	とします。	
\\	お昼に中華の食べ放題に行かない? 
\\	が「食べ放題」に当たる表現です。	
\\	喉が渇いて、お茶をがぶ飲みしたら、お腹がたぽたぽになった。 
\\	には「(液体が)バシャバシャ動く」という意味があります。	
\\	彼女が彼と結婚するなんて、たまげたね。 
\\	「たまげる」とは
\\	「驚く」ということですね。
\\	「信じられない」。	
\\	たった2日じゃ一週間のたまった疲れはとれないよ。 
\\	「たまった」は
\\	「蓄積する」を使えば表現できます。
\\	に値する」。	
\\	今頃たまった疲れがでてきたらしく、すっごくだるくて眠い。 
\\	で「疲労をためる」の意味。
\\	「疲労がたまる」。	
\\	面接官の気分次第で合格不合格が決まるなんて、たまったもんじゃないよ。 
\\	「たまったもんじゃない」をここでは
\\	「とても我慢できるようなことではない」と言い換えています。	
\\	たまに、私の英語を誉めてくれる人がいますが私はまだまだだと思っています。 
\\	この場合の「たまに」は 
\\	でいいでしょう。
\\	も使えます。	
\\	いっつも人にばっかり頼んでいないで、たまには自分でやったら? 
\\	は「時たま」の意味。
\\	の代わりに」。	
\\	24歳で医師と結婚し玉の輿に乗ったつもりの彼女だったが後年、夫の浮気と暴力に悩まされ離婚した。 
\\	と
\\	「おとぎ話」を組み合わせて「玉の輿」を表現しています。	
\\	風呂上がりの冷たいビール、この一杯がたまらなんだよな。 
\\	とは「もってこいだ、たまらい(的に当たっている)」の意味。	
\\	いくら家が近所だからって、質問がある度に呼び出されたんじゃたまんないよ。 
\\	「たまらない」は 
\\	(耐えられない)を使って表現すればいいでしょう。	
\\	その空き家は、不良少年たちのたまり場になってしまった。 
\\	とは「たまり場、行きつけの場所」のことです。
\\	は「うろつく」、
\\	は「不良」の意。	
\\	外国のお金は慣れるのに時間がかかり、旅行の際は初め(コインを使い分けるのが)面倒で紙幣ばかり使うのでコインがたまってしまう。 
\\	「結果として、たくさんの小銭が残る」で「たまる」という状態を説明しています。
\\	「面倒」。	
\\	それはいいアイデアだと思うけど、それができないから困っている。ストレスがたまる一方だ。 
\\	で「ストレスがたまるのを感じる」の意味。
\\	「ストレスにやられる」。	
\\	ここでへこたれてたまるか。 
\\	(降参する)、
\\	(諦める)と否定形を組み合わせて「へこたれてたまるか」を表現しています。	
\\	たまんねーー。 最高の気分だよ。 
\\	「とても抵抗できない、我慢できない」で「たまらないほど」の意味を表している。	
\\	そんな所でたむろして何やってるんだ。 
\\	「たむろする」は
\\	「集まる」で表現できますが、ここでは
\\	「ぶらつく」を使ってみました。	
\\	あいつ新人のくせに、俺にため口きくんだぜ。 
\\	で「俺と同じ立場(身分)のように話す(行動する)」の意味を表します。	
\\	最近は、年下のくせにため口を聞く奴が多すぎる。 
\\	「ため口を聞く」とは、年齢が上でも同年齢の相手のようにして話すこと。これは
\\	で表現できます。
\\	は「対等に」、
\\	「尊敬の念を持って話さない」の意。	
\\	いつものパターンだ。終わった試しないもんな、口ばっかりで。 
\\	「試しがない」とは
\\	「一度も 
\\	するのを見たことがない」ということですね。	
\\	それって、私を試したってこと? 
\\	「試す」は
\\	(動詞、名詞)を使って表現できます。	
\\	二人で一緒にレッスンを受けると、お互い頼りそうなので 別々で受けた方が自分たちのためになる。 
\\	で
\\	から利益を受ける」の意味。
\\	で「その人のためになる」。	
\\	彼女一人にそれをさせるのは、ためらいを感じるな。 
\\	で「…するのをためらう」の意味。	
\\	毎日どんなに忙しくても欠かさず英作文書いてはいるけど、ある程度たまってから添削してもらおうかと思って。 
\\	「ためる」は 
\\	(蓄積する)を使って表現すればいいでしょう。	
\\	多面観察とは上司、部下、同僚など複数の人が多面的に人物を評価することを言う。 
\\	「多面観察」をここではそのまま
\\	と英訳しています。	
\\	うちのリーダーってなんか頼りないんだよね。リーダーには向いてないと思うけど。 
\\	「意気地のない、決断力のない」、
\\	「信頼できない、頼れない」を使えばいいでしょう。
\\	は「彼には、人を引っ張るのに必要な素質がない」の意。	
\\	人にばかり頼っては成長しないよ。 
\\	「頼る」は
\\	を使えばいいでしょう。	
\\	お客様からいただいた電話をたらい回しにするのは失礼です。 
\\	が英語の「たらい回しにする」に当たる言い方である。	
\\	今年の夏は例年になく暑く、ジッとしていてもタラタラ汗が流れ出てくる。 
\\	で「汗がタラタラ流れる」の意味になります。
\\	を使うと流れる汗の多さを強調できます。	
\\	あいつの話は、いつもたらればの話しかしない。 
\\	は「もしこうだったら、もしあーだったら」、
\\	は「もし... なら、と言い過ぎる」の意。	
\\	他力本願はいけないよ。自分で困難を乗り越えるようにしなくては。 
\\	「他力本願」を
\\	「誰かに救ってもらうことを期待する」と言い換えています。
\\	を救い出す」の意。	
\\	現金が500円足りなかったからカードで払った。 
\\	「足りない」は 
\\	を使って表現できますので覚えておきましょう。	
\\	日本語を教えるのに何か参考書を買おうと思ったけど、なんとかインターネットで用が足りそうだ。 
\\	自動詞 
\\	には「用が足りる、間に合う」という意味があります。	
\\	最近、少したるんでるんじゃないの。 
\\	は「だらだらする」、
\\	は「(仕事などが)遅れる」という意味。	
\\	ゆっくり時の流れるままに彼女とたわいない会話を楽しんだ。 
\\	「たわいのないこと」とは「なんでもないこと」ですから、
\\	で表現すればいいでしょう。
\\	とは「無駄話をする」の意。	
\\	ちょっと汚い話なんだけど、たんがからむのよね。 
\\	で「たん(のかたまり)が詰まる」の意味。	
\\	この機械の操作には、ミリ単位の正確さが要求される。 
\\	で「正確さの許容範囲が1ミリ以内である」の意味。
\\	(単位)の使い方は、次の
\\	を参照のこと。	
\\	あの先生は、安易に単位を出さないよ。 
\\	「単位を出す」とは 
\\	(合格させる)ということですね。	
\\	そんなにカッカするなよ。短気は損気って言うだろ。 
\\	の使い方に注目。「(短気を起こすと)高くつくよ」と言っているのです。	
\\	使わない道具が箪笥の中で肥になっている。 
\\	が「箪笥の肥やしになる」に当たる言い回しです。	
\\	北野映画の特徴は、淡々としていることだ。 
\\	ここでは「淡々としている」を 
\\	(波乱のない)、
\\	(特色のない)で表現しています。	
\\	丹田を意識して呼吸してください。 
\\	「丹田」とは「おへそのしたあたり」のことを指しますので、
\\	「腹部」で大体の場所を表現します。
\\	「保つ」、
\\	「集中する」の意。	
\\	この件に関しましては、担当の高橋が説明いたします。 
\\	「…担当の」は 
\\	で表現します。
\\	は「…に責任がある」の意。	
\\	小さい頃から皆と同じなのが苦手で、単独行動をとっていた。 
\\	とは「一匹狼」のことです。
\\	「独立した、自立した」。
\\	「あまりチームプレーヤーではない」。	
\\	私のクラス担任は英語の先生です。 
\\	「担任」は
\\	と言います。	
\\	チーズバーガーとフィッシュバーガー、どちらも単品で。 
\\	で「それだけで 
\\	の意味になります。「フィッシュバーガー」は
\\	ということもあります。	
\\	それは、箱で売られてるのもあるし、単品でも売られています。 
\\	ここでは「単品」を
\\	(1つずつ)を使って表現しています。	
\\	米国企業の不正会計に端を発し、いまや米国型の資本主義にまで不信感が広がった。 
\\	が「…に端を発する」に当たる言い回しです。	
\\	この前のタクシー代と夕食代あわせて3500円、早く返してね。 
\\	ここでは「代」を 
\\	と 
\\	で表現しています。
\\	で「(人に)…を借りている」の意。	
\\	第一印象で、あなたの英語を教えるにあたっての熱意を感じました。 
\\	「第一印象」は
\\	ですね。
\\	「情熱、熱意」。	
\\	中高生は第一に勉強が大切です 
\\	(最も重要である)で「第一に」を表現しています。	
\\	就職できなかったら大学院に行くって?お前、考えが甘いんだよ。 
\\	「大学院」は
\\	ですね。
\\	と言うこともあります。	
\\	私は大学院を出ています。 
\\	「大学院を出た」とは
\\	「大学院に行った」ということですね。	
\\	ベッカーは2セットダウンから大逆転劇を演じた。 
\\	「大逆転をする」は
\\	と言います。
\\	は「驚異的な逆転勝利」の意味になります。また、
\\	は「不利な立場、不足」の意。	
\\	あの女優まったくダイコンだね。 
\\	がまさに「大根役者」のことです。	
\\	ダイビングの醍醐味は、やってみなければわからないよ。 
\\	「醍醐味」は
\\	「快感、スリル」や
\\	「素晴らしさ」で表現します。	
\\	今までありがとう。先生のクラスでアメリカドラマの醍醐味を知った気がします。スクリプトなしで理解できるように今後頑張ります。 
\\	ここでは「醍醐味」を 
\\	(魅力)で表現しています。
\\	は「感謝する」の意。	
\\	授業は11月の第三週の水曜日から始まります。 
\\	「第三週」はそのまま
\\	を使います。	
\\	これまでにない大惨事になった。 
\\	「大惨事」は
\\	ですね。
\\	「前例のない」、
\\	「聞いたことがない、前代未聞の」、
\\	「大きさ」。	
\\	大至急、報告書をまとめて、彼に届けて欲しい。 
\\	「大至急」とは
\\	(できるだけ早く)ということですね。	
\\	一度大失敗したことがあるので、お客さんが日本人ってわからない限り声をかけないようにしてます。 
\\	(台無しにする)を使って「大失敗」を表現しています。
\\	は「だめにする」の意。	
\\	車を修理している間に代車をいただきたいのですが。 
\\	が「代車」に当たる言葉です。
\\	でも「代車」は表現できますね。	
\\	具合が悪いので大事を取って、今日は寝ています。 
\\	(用心のために)が「大事を取って」に当たる言い回しです。	
\\	一生、君を大事にしたい。 
\\	「大事にする」は
\\	、あるいは、
\\	で表現できます。	
\\	僕が車を持っていたときは自分で洗車して、ワックスを掛けてとても大事にしていたよ。 
\\	は「とても大事にする」の意味。
\\	を使っても同じようなことがいえます。	
\\	今仕事抜けてきて大丈夫だったの? 
\\	「大丈夫」は
\\	を使って表現します。
\\	「こっそり抜け出す」。	
\\	まだ大丈夫、諦めちゃダメ。 
\\	(ちゃんとやっている)で「大丈夫」を表現しています。	
\\	トルシエ監督で大丈夫かな? 
\\	「大丈夫」は
\\	を使って表現すればいいでしょう。	
\\	大卒だからって、仕事ができるとはかぎらない。高卒でも優秀なやつはいくらでもいる。 
\\	を動詞「卒業する」と名詞「卒業生」で使っているところがポイントです。	
\\	あれ?英検に合格したのっていつだったかな。忘れちゃったな、だいたいで書いておくか。 
\\	で「だいたい合っている日付を書く」の意味になります。
\\	は「おおよその」の意味。	
\\	あの英会話学校って、そんな大々的に宣伝していないけど、口コミで生徒が集まってくるんだって。 
\\	「大々的に宣伝する」を
\\	で表現しています。
\\	「口コミ」。	
\\	せっかくの花火が急な雨で台無しだったよ。毎年楽しみにしてたのに。 
\\	「台無しにする」は
\\	「めちゃめちゃにする」、
\\	「だめにする」で表現できます。	
\\	他の人のアドレスがわからなかったので、代表で送らせてもらいました。皆に連絡お願いします。 
\\	この場合の「代表」は
\\	「気付け」を使えばいいでしょう。	
\\	そっちで公開されてても日本に来るのは半年後ぐらいだから、雑誌で見たい映画を見つけても、見るのはだいぶ先なんだよ! 
\\	「見るのはだいぶ先」を 
\\	(かなりの間見ることはできない)で表現しています。	
\\	さあ、ここで、いよいよ、大魔人佐々木の登場です。 
\\	は「名人、達人、すごい人」の意味です。第2文では、佐々木選手を
\\	「手品師」と呼んでいます。	
\\	そこの2つの商店の争いは、バックにいる親会社の代理戦争だ。 
\\	が「代理戦争」に当たる表現です。	
\\	メールサーバーがダウンして、今朝からメールが使えないんだ。 
\\	「ダウンする」は 
\\	で表現できます。
\\	は「アクセスする」の意。	
\\	株価が下がっているときだからこそ、その会社を安く買い叩けるのだ。 
\\	で「まさに…という理由で~する」の意味になります。	
\\	一度でいいからあの人に抱かれてみたい。 
\\	「抱かれる」を 
\\	と 
\\	を使って表現しています。	
\\	2、3年前に流行ったあれも、今となっては抱き合わせでないと売れなくなってしまった。 
\\	とは「抱き合わせ販売」のことです。
\\	は「売りさばく」、
\\	は「…と抱き合わせ販売にする」の意味。	
\\	それはただ単に、体が疲れてなくて眠れないだけじゃないの? 
\\	「だけじゃないの」のニュアンスを
\\	と
\\	と 
\\	を組み合わせて表現しています。	
\\	立っているだけで汗がでてくる。 
\\	この場合の「だけで」は
\\	を使って表現できます。
\\	は自動詞で「汗をかく」の意。	
\\	前首相の支持率が支持率だっただけに、現首相の支持率は際立って見える。 
\\	(…に注目する)を使って「支持率が支持率だっただけに」のニュアンスを出しています。	
\\	彼の服装いつもださい。 
\\	は「流行遅れの、流行している」の意味。	
\\	彼は私の靴をダサいといって、ばかにします。私に言わせれば彼の靴のほうが、今どき履いている人いない。 
\\	ここでは「ださい」を 
\\	(流行遅れの)で表現しています。	
\\	この車ってどこの会社がだしているやつ? 
\\	この場合の「だしている」は
\\	を動詞、あるいは、名詞として使って表現できますね。名詞
\\	には
\\	製」という意味があります。	
\\	出し抜けに会社辞めるなんて、一体、部長との間で何があったんだよ。 
\\	「青天の霹靂」という表現がありますが、この
\\	は「突然に、出し抜けに」の意味で使うことができます。
\\	は「急に... する」の意。	
\\	母親についた嘘の言い訳をしていた時にするどいつっこみがはいり、たじたじになってしまった。 
\\	ここでは「たじたじになる」を
\\	「だじろぐ、ひるむ」、
\\	「声が震え始めた」で表現しています。
\\	「痛烈な」。	
\\	だじゃれの連発でみんな 寒くなった。 
\\	「だじゃれ」は
\\	でいいでしょう。
\\	には「嫌な気分にさせる、やる気をなくさせる」という意味があります。	
\\	たくさん出したら安くしてくれますか?(クリーニング屋にて) 
\\	(服をたくさん持ってくる)で「出す」を表現しています。	
\\	その子はプレーステーションの新しいソフトが欲しいとだだをこねた。 
\\	「だだをこねる」は
\\	「だだっ子のように振る舞う」で表現します。
\\	「がき、子供」。	
\\	どうしたの、そんなに泣いて。だっこして欲しいの? 
\\	名詞
\\	は「抱きしめて(かわいがる)こと」の意味。ここでは
\\	も使えますね。	
\\	これは脱臭、除湿効果があるので、靴の中に入れておくといいんですよ。 
\\	「脱・除」を
\\	「取り除く」で表現しています。
\\	「臭い」、
\\	「湿気」。	
\\	だって~そう言ったじゃない! 
\\	「だって」は
\\	、あるいは、
\\	を使えばいいでしょう。	
\\	彼は脱サラし、ラーメン屋を始めたらしい。 
\\	「脱サラ」を
\\	(会社の熾烈な争いから抜ける)を使って表現しています。「ラーメン屋」は
\\	でいいでしょう。	
\\	発音とリスニング力に関しては、彼には脱帽します。 
\\	で「かぶとを脱ぐ」の意味。また、
\\	とも言えます。	
\\	映画を見ていてよく思うのだけど、文章で見ると簡単な英語の文も、リスニングでは脱落している音とかあって難しい。 
\\	は「落とされる、略される」、
\\	は「省略される」の意味。	
\\	その話をきいて一気に脱力した。 
\\	「脱力する」は
\\	「力が抜ける」で表現できます。また、
\\	「麻痺する」を使ってもいいでしょう。	
\\	昨日借りてきたCDを明日返さないといけないから、ダビングしようっと。 
\\	この場合の「ダビング」は
\\	を使います。
\\	は「吹き替える」の意味で使われることが多いようです。	
\\	お腹のあたりが少しだぶついてきたな。 
\\	は「蓄積する」、
\\	は「贅肉」の意味になります。
\\	「太鼓腹」。	
\\	だまされたと思ってやってごらん。必ずうまくいくから。 
\\	「だまされたと思って」とは 
\\	(僕のことを信じてくれ)ということですね。	
\\	彼は、優しすぎて人を信用するタイプだから、いずれだれかに騙されるんじゃないかって心配。 
\\	は「騙す、(お金を)騙し取る」、
\\	は「…に騙される」の意。	
\\	私は、古くてよく故障するパソコンをだましだまし使っている。 
\\	は「無理に引き出す、絞り出す」という意味です。
\\	「搾り取る」、
\\	「十二分に」。	
\\	今回がダメでも、また、次があるさ。 
\\	で「物事がうまくいかない」の意味になります。
\\	はともに「しくじる」の意味。	
\\	ジャイアンツにダメ押しの1点が入った。 
\\	「ダメを押す」は「それで安全になった」と考えて、
\\	「保険」を使って表現できます。
\\	「勝利を決定づける」の意。	
\\	確かに年を取るにつれて脂っこい物がだめになるね。 
\\	「だめになる」とは
\\	するのが大変になる」ということですね。	
\\	このままほっておくと、日本そして日本人は駄目になる。 
\\	で「どんどん駄目になる」という意味になります。
\\	「破滅する、駄目になる」。	
\\	掃除はやらない、仕事はできない。私って駄目人間の象徴だわ。 
\\	ここでは「象徴」を
\\	「典型的な」を使って表現しています。	
\\	だめもとで、キムタクに取材を申し込んでみたら? 
\\	「だめもと」を
\\	「失う物が何かあるのか」、
\\	「試しにやってみても困ることなどない」で表現しています。	
\\	不安な事だらけだよ。でも、それを一つ一つ克服していくっていうのが人生だと思う。 
\\	「どっさり」を使って「だらけ」のニュアンスを表現しています。	
\\	女にだらしのない奴は好きじゃないんだ。 
\\	「女にだらしがない」をここでは
\\	「いろいろな女性とほっつき歩いて」、
\\	「一人の女性に優しくしない」と言い換えて表現しています。	
\\	そんなだらしない格好のままでいないで、ちゃんと服きなさい。誰か来たらどうするの? 
\\	で「浮浪者のように見える」の意味になります。
\\	「だらしのない、みすぼらしい」。	
\\	時間があるときに限ってダラダラ過ごしてしまいがちだよね。 
\\	で「(時間を)潰す」の意味。
\\	「のんびりする」。	
\\	だらだらしてる場合じゃないぞ。行かなきゃ。 
\\	「だらだらする」とは
\\	「時間を無駄にする」ということですね。
\\	「無駄に費やす」。	
\\	せっかく海外に行くチャンスだから、何か目的を持って行きたいけどそれが見つからない。何か見つけないとだらだら過ごしそうだ。 
\\	で「向こうでの時間をだらだら過ごす」の意味になります。
\\	(無駄にする)を使ってもいいでしょう。	
\\	もう今日は、だるくて何もする気がしない。 
\\	とは「エネルギーがあまりない=だるい」の意味。
\\	「気だるい、元気がない」とも言えます。	
\\	そのこと、誰から聞いたの? 
\\	この場合の「誰から」は
\\	でも
\\	でも表現できますね。	
\\	彼女は新しいコンピュータを手に入れて、うれしくてしょうがなく誰でもいいから自慢したかった。 
\\	「誰でもいい」を文末に
\\	「誰でも構わなかった」をつけて表現しています。	
\\	今までに誰もやったことがない。 
\\	「誰も」は
\\	ですね。	
\\	アメリカに引っ越してきてから、どこへ行ってもコーヒー屋があるので断然コーヒーを飲む量が増えた。 
\\	で 
\\	を強調して「断然」の意味を表してます。
\\	は「急激に増える」の意。	
\\	彼女ダントツで勝ったよ。 
\\	は「楽勝の」という意味。
\\	「楽々と」。	
\\	仕事はきちっと段取りよくやらなきゃね。 
\\	とは「規則にしたがって」の意味です。これは
\\	「一つずつ片づけてる」と言い換えてもいいでしょう。	
\\	あいつはいつも自分が不利になるとだんまりを決め込む。 
\\	は「口を閉ざす」の意味。
\\	「一言も言わない」を使っててもいいでしょう。	
\\	アメリカは観光するにはいい所だと思うけど、住むとなるとちょっと考えちゃうね。特に、治安がね。 
\\	ここでは「治安」を
\\	「犯罪」、
\\	「犯罪発生率」を使って表現しています。	
\\	あの噂が、ただしいのかどうなのか、ホームページでチェックしないとなぁ~。 
\\	この場合の「チェック」は
\\	「見る」でいいでしょう。
\\	「ホームページ」。	
\\	うちの部下は初めは素直だったのに、だれかがよけいな知恵をつけたから、うまく手を抜くようになった。 
\\	ここでは「知恵をつける」を 
\\	を使って表現しています。	
\\	健やかなる時も病める時も汝の妻とすることを誓います。 
\\	誓う時には
\\	、あるいは、
\\	という表現をよく用います。	
\\	うわー。こんなにテレビの画面がいっぱいあると、目がチカチカする。気持ち悪くなってきた。 
\\	「目がチカチカする」を 
\\	(悩ませる、困らせる)と 
\\	を組み合わせて表現しています。
\\	は「気分が悪い」、
\\	は「目がくらむ」の意。	
\\	このテレビ、チカチカしてどうも映りがよくないな。 
\\	が「チカチカする」に当たる表現です。
\\	「受像」。	
\\	うるさい!近寄るな! 
\\	は「…から離れる」、
\\	は「あっちへ行け」の意味。	
\\	皆で力を合わせて頑張ろう。 
\\	は「協力して働く」の意味。
\\	でも「力を合わせる」が表現できますね。	
\\	我が社では電子取引に力を入れています。 
\\	と
\\	「資産、力」、
\\	「努力」を組み合わせることで「力を入れる」を表現しています。	
\\	明日の決戦のため、力を貯めておこう。 
\\	「力を貯める」はそのまま
\\	と言えます。	
\\	彼が就職する前に学校に行って力をつけたい気持ちはわかるけど、悲しいことにお金がないとそういうことってできないんだよね。 
\\	ここでは「力をつける」を 
\\	(技術力、能力を上げる)で表現しています。	
\\	電車の中で痴漢を捕まえたことがあるの。 
\\	は動詞で「痴漢する」、
\\	は名詞で「痴漢」の意味。	
\\	それとそれの違いはなに? 
\\	「違い」は
\\	「名詞」、
\\	「動詞」で表現できます。	
\\	明日雨が降っても、彼は来るにちがいない。 
\\	は「疑いなく」、
\\	は「確実に」の意味です。	
\\	彼は血がにじむような努力をして、司法試験に合格した。 
\\	とは「血の汗を流す、懸命にがんばる」ということです。	
\\	同時通訳と逐次通訳はどちらが難しいんでしょうか。 
\\	「同時通訳」は
\\	、「逐次通訳」は
\\	といいます。	
\\	上司に退職することを言うのは全然いいんだけど、言ったあと、辞めるまでがちくちく嫌味言われそうで嫌なんだよね。 
\\	で「いろいろと悩ませる」の意味になります。
\\	は「嫌な」の意。	
\\	アメリカと日本の教育って違うと思うので、こちらで教育を受けてこちらの知識を広げたいのです。いい仕事に就くために。 
\\	が「知識を広げる」に当たる言い回しです。
\\	でも同じようなことが言えます。	
\\	週末,乳搾り体験に行かないなにそれ乳搾りの疑似体験が出来るんだ。本物の牛じゃなくて,模型の牛を使っておもしろそう。 
\\	(動詞)に「乳を搾る」という意味があります。「乳搾り体験」は
\\	でいいでしょう。	
\\	与党の構造改革が遅々として進んでいない。 
\\	で「どうしようもないほどゆっくりとしたペースで進む」の意味になります。
\\	で「与党」の意。	
\\	このセーター、クリーニングに出したら縮んじゃったみたい。袖こんなに短くなかったよね。 
\\	「縮む」は
\\	を使って表現します。	
\\	二つのグループ間で、血で血を洗う復讐戦が始まった。 
\\	は「凶暴な、物騒な」、
\\	は「容赦しない」の意味。	
\\	最近、血生臭いニュースが多いよね。 
\\	「血なまぐさい」は
\\	あるいは
\\	を使って表現します。	
\\	日本語でも「ボーイフレンド」は使います。ちなみに、その場合「彼氏」という意味になります。 
\\	「ちなみに」は 
\\	(ついでに言えば)、
\\	(ところで)を使えばいいでしょう。ここでは「彼氏」を 
\\	で表現しています。	
\\	彼女は地に足が着いている人だから、この先きっと成功するだろう。 
\\	ここでは「地に足が着く」を
\\	(分別のある)と
\\	(現実的な)を使って表現しています。	
\\	今回の事件で、米国企業への信頼は完全に地に落ちた。 
\\	は「どん底、最低の段階」、
\\	は「(新たな)どん底」の意味。	
\\	その事故の話を聞いて、皆、血の気が引いた。 
\\	「血の気が引く」とは
\\	「真っ青になる」ということですね。
\\	「青白くなる」。	
\\	あいつのところ、お母さんが痴呆症になちゃって大変なんだってさ。 
\\	「痴呆症になる」は
\\	と言います。
\\	「歳を取る」。	
\\	ちまたではインターネットなるものがはやっています。 
\\	「ちまたでは... だ」は
\\	と
\\	を使って「みなが話している」で表現すればいいでしょう。	
\\	指をドアに挟んだら、血豆ができた。 
\\	「血豆」は
\\	と言います。ここでは動詞の
\\	「挟む」、
\\	「(まめなどを)作る}の使い方にも注意してください。	
\\	知名度の点から言うと彼の方が日本で人気があると思うな。 
\\	で「知名度の点からすると」という意味になります。	
\\	その着眼点はするどい! 
\\	で「どこを見るべきか知っている」の意味になります。	
\\	歯の着色が気になります。最近、コーヒーを飲んでるからかな。 
\\	「歯の着色」は 
\\	でいいでしょう。
\\	は「…が気になる」の意。	
\\	(携帯の表示が)着信ありになっている。 
\\	「着信があった」は
\\	で表現できます。
\\	の使い方もポイントですね。	
\\	彼の話は回りくどくて、着地点が見えない。 
\\	「着地点が見えない」は
\\	「話をどこへ持っていこうとしているのか分からない」と言えばいいでしょう。
\\	で「堂々巡りをしている」、
\\	は「だらだら話す」の意。	
\\	新しい着メロを携帯に入れたんだ。 
\\	「着メロ」を英語にするのは少し大変です。ここでは「電話が掛かってきた時に流れるメロディ」と説明しています。	
\\	一生懸命、受験勉強しているのに、妹が茶々を入れたがって困る。 
\\	「茶々を入れる」とは「邪魔する」ということ。
\\	あるいは
\\	を使えばいい。	
\\	あの娘、ちゃっかりしているよね。 
\\	「ちゃっかりしてる」は
\\	「賢い、抜け目がない」で表現すればいいでしょう。	
\\	あのおとなしそうな江藤さんが茶目っ気をたっぷり出していたずらをした。 
\\	ここでは「茶目っ気たっぷり」を
\\	「ふざけた、おどけた」、
\\	「とっても楽しい」を使って表現しています。
\\	「うまくやる」、
\\	「いたずら」。	
\\	この話、チャラにしようよ。 
\\	は「この話は忘れてくれ」、
\\	「新しくやり直す」の意味です。	
\\	その企業にとって、合併は最後のチャンスをものにする絶好の機会だ。 
\\	ここでは「ものにする」を
\\	「…を大いに利用する」で表現しています。	
\\	お前、言ったことはちゃんとやれやよな。 
\\	と
\\	で「ちゃんとやる」のニュアンスを出しています。	
\\	電子メールって便利だけど、ちゃんと着いたかどうか不安じゃない?かといって、着いたかどうかメールするのもなんだし。 
\\	ここでは「ちゃんと」のニュアンスを
\\	を使って表現しています。
\\	は「確認する」の意。	
\\	せっかく資格取得しても、ちゃんと毎年更新しておかないと無効になるよ。 
\\	「ちゃんと」のニュアンスは
\\	「忠実に、しっかりと」で出せます。
\\	を使ってもいいでしょう。	
\\	チャンネル争いは止めなさい。 
\\	この場合の「争い」は
\\	(言い争う)を使えばいいでしょう。	
\\	みんながあいつのことちやほやするから、調子に乗っちゃうんだ。 
\\	で「もてはす」という意味になります。「調子に乗る」の表現の仕方にも注目。	
\\	あいつ、この間も仕事中にインターネットで遊んでいて注意されたばかりなのに、全然懲りてないようだ。今でもやってるよ。 
\\	で「…のことで注意される」の意味になります。
\\	は「1つも学ばない(=懲りない)」の意味。	
\\	宇宙人襲来という根も葉もない噂に踊らされぬよう注意を呼びかけた。 
\\	「注意を呼びかける」は 
\\	でいいでしょう。
\\	は「(人に)~するように訴える」の意。(注)日本語の主語がはっきりしませんので、ここではとりあえず 
\\	を使っています。	
\\	下からはつきあげられ、上からは文句を言われる。中間管理職はつらいよ。 
\\	「中間管理職」は
\\	と言います。
\\	「上(上層部)」。	
\\	宙返りしたり体をすごい曲げたり、逆立ちしたりバック転したり、とても人間とは思えないパフォーマンスだった。 
\\	が「宙返りする」に当たる表現です。
\\	は「逆立ち」、
\\	は「後ろ宙返り」、
\\	は「ねじ曲げる」の意。	
\\	うちの子どもには、男女共学の中高一貫校に行かせるつもりだ。 
\\	12で「一貫教育」を表現しています。
\\	と言ってもいいでしょう。	
\\	僕は子どものころから注射が苦手でね。 
\\	「注射」は
\\	と言います。
\\	「注射される」。	
\\	彼は、あるホームページでかなり中傷されていた。ま、それだけ有名人ってことで、ある意味いいのかね。 
\\	この場合の 
\\	は「散々に言う」の意味です。
\\	は「名誉を傷つけるための工作」の意。	
\\	物事何でもあなたを中心に動いていると思ったら大間違いだからな。 
\\	は「…のまわりを世界が回る」の意味。動詞 
\\	の使い方にも注目。
\\	は「悲しいほど」の意。	
\\	あそこの物件って抽選なんだって。立地条件もいいし倍率高そうだよね。 
\\	が「抽選で、くじ引きで」に当たる表現です。
\\	は「確率」の意。	
\\	雨で30分間野球の試合が中断した。 
\\	この場合は、
\\	「中断する」を受け身で使います。	
\\	何をやっても中途半端にしかできない。 
\\	「中途半端」は
\\	で表現できます。	
\\	なんか彼のひげは中途半端にはやしているから私は好きじゃないのよ。生やすなら生やす、剃るならそるってどっちかにしてほしい。 
\\	「中途半端」は
\\	を使えば表現できます。
\\	や
\\	の使い方にも注意。	
\\	私の髪の長さって中途半端だと思わない?ちょうど毛先がはねるんだよね。 
\\	「中途半端(な長さ)」を
\\	「短くもなく、長くもない」で表現してみました。
\\	「曖昧な領域」。	
\\	親の英語が中途半端な状態で子どもに教えても、子どもはやっぱり中途半端な英語を覚えるんだろうな。 
\\	は「生半可な」、
\\	は「中途半端な能力の」という意味です。	
\\	一度にいろいろ始めたら、きっと全部が中途半端になりそうだから、一つずつ終わらせていこうかと思っている。 
\\	で「仕事が中途半端に終わる」の意味になります。	
\\	最近、お腹の周りに肉がついてきたみたい。中年太りかな。 
\\	「中年太り」を、ここでは、
\\	「お腹の周りに余分な肉がつく」、
\\	「中年が忍び寄る」を使って説明しています。
\\	とは「中年になるとお腹の周りに肉がついてくるものだ(お腹の周りに余分なつくことも中年になることに含まれる)」の意。	
\\	自分の買いたい商品を事前に試すことができるシステムがあるというのはとてもいいアイデアだと思うし、今後注目されると思う。 
\\	で「どんどん注目を集める」の意味。
\\	は「どんどん話題になる」の意。	
\\	(注文した料理がこないときに)注文は通っていますか? 
\\	この場合の「通る」は
\\	を使えばいいでしょう。	
\\	一カ月くらい長期休暇を取って、のんびりしたいよな。 
\\	「長期休暇」は 
\\	、あるいは、
\\	でいいでしょう。	
\\	君は目先のことばかり考えていて、長期展望がない。 
\\	が「長期展望」に当たる言い回しです。	
\\	この調子で行けば、8時に間に合うかも。 
\\	「調子」は
\\	「進度」、
\\	「ペース」で表現できます。	
\\	寝不足のせいか、今日はどうも調子が出ない。 
\\	で「物事がうまく行く、調子よく進む」の意味。
\\	「元気が出ない」。	
\\	人に頼んでばかりで、調子よ過ぎやしないか? 
\\	「調子よすぎる」をここでは
\\	「物事を自分の都合のいいようにする」を使って表現しています。
\\	「快い、うまみのある」。	
\\	(ボーリング場で)調子が悪いの? (いいえ、そんなことはないけど) 
\\	で「調子が悪い」の意味。
\\	(自動詞)「ボウリングする」。	
\\	こういう雰囲気に弱いんだよ。ついつい調子に乗って一気飲みしてしまうんだよね。 
\\	で「やりすぎる、自制心を失う」の意味になります。
\\	は「一気飲みする」の意。	
\\	だれにでも長所短所はあるものだよ。欠点のない人などいないよ。 
\\	ここでは「長所短所」を
\\	で表現しています。
\\	「欠点、不完全」。	
\\	この間4カ月ぶりにゴルフしたら、最初はぐたぐただった。後半になるについて調子を戻したけどね。 
\\	で「調子が戻る」の意味になります。(注)「最初はぐたぐただった」は英訳していません。	
\\	国民にはまったくわからない水面下で総理候補者の調整が進んでいたなんて、たまったもんじゃない。 
\\	この場合の「調整する」は
\\	でいいでしょう。
\\	は「取り決める」の意。
\\	は「水面下で、舞台裏で」、
\\	は「密室の」、
\\	は「うまく処理する」の意味。	
\\	ファックスしてって何度電話しても、ファックスよこさないのよ。これって私に対する挑戦かな 
\\	ここでは「挑戦かな」を
\\	「刺激する、いらいらさせる」を使って表現しています。
\\	とも言えます。	
\\	ちょうあのラーメン屋、いつ行っても長蛇の列ができてるけど、本当においしいのかな。だのれつ 
\\	「長蛇の列」は 
\\	を動詞、名詞として用いて表現できます。	
\\	夕方に疲れが頂点に達して、時折プッツンしてしまう。 
\\	「頂点に達する」は
\\	と
\\	を組み合わせて表現できます。また、
\\	(最高潮に達する)でも同じような意味です。ここでは「プッツンする」を
\\	(あきらめる)で表現しています。	
\\	コンサート中止になっちゃったの?テスト勉強しようと思ってたからちょうどいいや。 
\\	「ちょうどいい」は 
\\	で表現すればいいでしょう。	
\\	相手の挑発行為に乗ってはいけない。 
\\	「挑発する」は 
\\	を使って表現します。
\\	は「敵、相手」の意。	
\\	彼は唯一会社でコンピュータを知っているのでかなり重宝がられているようです。 
\\	(なくてはならない)、
\\	(絶えず必要とされる)で「重宝がられる」を表現しています。	
\\	このホームページは大変便利で、いつも重宝しています。 
\\	「重宝する」とは
\\	「役に立つ」ということですね。
\\	は「貴重だと思う」の意。	
\\	あの店の店員態度って、超むかつくよね。 
\\	「超むかつく」は「信じられないくらい嫌な」と言い換えて、
\\	と表現します。また、
\\	「とんでもない、言語道断」を使ってもいいでしょう。	
\\	起業家を税制面で優遇することは、今や世界的な潮流だ。 
\\	「潮流」とは
\\	「趨勢、動向」、
\\	「傾向」ということですね。(注)日本語をに「起業家」を付け足しました。	
\\	最近、歳のせいか、聴力が落ちてきたような気がする。 
\\	で「かつてのようによく聞こえない」の意味になります。	
\\	直感でそう感じました。 
\\	とは「直感、第六感」のことです。
\\	は複数形で「内蔵」という意味がありますが、ここでは「本能、直感」の意味で使われています。	
\\	アルバイトで稼いだお金は無駄遣いしないで銀行に貯金しなさい。 
\\	「銀行に貯金する」は 
\\	で表現します。
\\	は「浪費する」の意。	
\\	ヴィトンを買うなら直営店で買ったほうがいいよ。偽物が出回っているから。 
\\	「直営店」を 
\\	で表現しています。
\\	は「正式に認可された」、
\\	は「流通する」の意。	
\\	天気予報によると、週末に大型台風が関東を直撃するらしい。 
\\	で「直撃を受ける」の意味になります。
\\	に直接向かう」。	
\\	ちょっと直前で申し訳ないけど、時間を変更したいのですが。 
\\	は「土壇場で」の意味になります。	
\\	僕の彼女にちょっかい出さないでくれ。 
\\	で「…に手を出すな」の意味になります。	
\\	今日は(出先から)直帰します。 
\\	「直帰」はそのまま
\\	「そこから真っ直ぐ家に帰る」でいいでしょう。また、「会社に戻らない」
\\	と言い換えることもできます。	
\\	最後にちょっと聞いていいですか? 
\\	「ちょっと聞く」は
\\	「さらにもう一つ(聞いていいですか)」で表現すればいいでしょう。	
\\	会社の帰りに、ちょっと一杯やってかないか。 
\\	は「急いでする」、
\\	は「飲む」の意味になります。	
\\	ちょっと英語のできる秘書が、同時通訳をすることになったのでさあ大変だ。 
\\	「ちょっとできる」を
\\	「英語をちょっと知っている」で表現しています。
\\	は「同時通訳」のこと。	
\\	インターネットショッピングって便利だと思うけど、ネット上でクレジットカードの番号を入力するのがちょっと気になるのよね。 
\\	は
\\	について不安を抱く」、
\\	は
\\	にちょっと疑いを持つ(用心する)」の意味。	
\\	そんなのちょろいよ。おれに任せれば、30分で終わるよ。 
\\	「ちょろい」とは
\\	(楽なこと、楽な仕事)という意味ですね。	
\\	カラスがゴミ袋を破ってゴミを散らかすので掃除が大変だ。 
\\	は共に「まき散らす」という意味。	
\\	このチラシ見てよ。パソコンが4万5000円ですって。 
\\	「チラシ」は
\\	と言います。
\\	とは「(新聞の)折り込みビラ」のことです。	
\\	彼の言っていることはいつもちんぷんかんぷんだ! 
\\	「理にかなったことを話す」、
\\	「理解する」の意。
\\	と言っても同じような意味になります。	
\\	みんなに会うの3年ぶりだから、ちょっと緊張するな。沈黙が続いたりしたらどうしよう。 
\\	「沈黙」は
\\	を使えばいいでしょう。
\\	は「黙りこくる」の意。	
\\	ついこの間、新しいアトラクションができたらしいよ。もうちょっと早ければ先月行ったから乗れたのにな・・。 
\\	「ついこの間」は
\\	でいいでしょう。「アトラクション」は
\\	でいいのですが、この場合は「新しい乗り物」ということなので、
\\	を使って表現しています。	
\\	ついこの間まで、きみはそう言っていたじゃないか。 
\\	で「つい最近まで」の意味。	
\\	コタツ出したのはいいんだけど、ついついコタツで寝てしまうんだよね。 
\\	ここでは「ついつい」のニュアンスを
\\	「ついうっかり」を使わずに
\\	「少しすると、じきに」で出しています。	
\\	ごめん!電話しようと思ってたのに、ついつい寝てしまった! 
\\	と
\\	の組み合わせ、
\\	「どうしても 
\\	してしまう」の使い方がポイント。	
\\	仕事で疲れると、つい外食してしまうので、やめなきゃと思いつつついつい。 
\\	「ついつい」は 
\\	(つい…をしてしまう)で表現できます。
\\	は「外食する」の意。	
\\	新製品を見るとついつい買いたくなる。 
\\	動詞」で「~せざるをえない」の意味。	
\\	最近の若い子たちのファッションにはついていけないわ。 
\\	「ついていく」は
\\	でいいでしょう。	
\\	歴史や政治関係の知識は全然なくて、知らなくてもどうにかなると思っていたけど、たまにそういう話題になるとすっごく困るんだよね。話についていけなくて。 
\\	「ついて行く」は
\\	で表現できます。
\\	「置いて行かれる」。(注)投稿文がかなり長くなっています。次回は、もう少し短くお願いします。	
\\	ついてない時は、何をやってもうまくいかないものだ。 
\\	「ついてない」は
\\	や
\\	「不運続き」を使って表現すればいいでしょう。	
\\	佐藤さんが店屋物をたのむのなら、私もついでにたのもうかな。 
\\	「ついでにたのむ」は
\\	でいいでしょう。ここでは
\\	「店屋物をたのむ」も覚えておくといいでしょう。	
\\	掃除機かけるならついでに雑巾がけもしておいて。 
\\	この場合、「ついでに」のニュアンスは 
\\	で表現できるでしょう。	
\\	信号で止まっていたら、後ろからトラックに追突された。 
\\	は「激しくぶつかかるの」の意味。
\\	「後ろから」。	
\\	この辺りは追突事故が多いので、早急に道路整備を行う必要がある。 
\\	「追突(事故)」は
\\	といいます。
\\	は「急を要する」の意。	
\\	だいたい1日の3分の2は通勤時間も含めて仕事の時間に費やしている。 
\\	で「(時間を)取る、奪う」という意味。	
\\	離陸直後コンコルドが墜落したが、日本人乗客はいない模様。 
\\	「墜落する」は
\\	、あるいは、
\\	で表現します。
\\	の模様」の使い方にも注目。	
\\	牛タンを頼むなんて、なかなかの通だね。 
\\	ここでは「通」を
\\	「自分のしていることが分かっている」で言い換えて表現しています。
\\	「くろうと」。	
\\	あそこは通勤には便利だけど、そこで働くとなると1人で全部管理するのですよね。そうなると自信ないな。 
\\	で「通勤に便利な所」という意味になります。
\\	は「…を担当する」の意。	
\\	1時間半っていう通勤時間は平均だと思うよ。 
\\	は名詞で「通勤、通学(時間)」の意味。	
\\	毎朝の通勤電車ってうんざりだけど、それに慣れてきている自分が怖い。あ~、こんな生活は嫌だ、嫌だ。 
\\	「通勤電車」は 
\\	といいます。
\\	は「電車で通勤すること」の意。	
\\	このコートね、通信販売で買ったのよ。 
\\	「通信販売」は
\\	と言います。現在は、インターネットを使った通信販売もありますね。	
\\	通信傍受法成立の前から既に、家の電話「盗聴」されてるカモ。 
\\	「通信傍受法」は
\\	と言います。
\\	は「盗聴する」、
\\	「制定する」。	
\\	で高得点とっても、通じる英語が話せないと意味がないと思いますけど。 
\\	「通じる」とは
\\	(言いたいことが伝えられる)ということですね。
\\	は「ずっと遅れる」の意。	
\\	試験の詳細は試験の約一ヶ月前に通知されるので、それから計画をたてようと思います。 
\\	「通知する」は
\\	を使います。
\\	「通知(名詞)」。	
\\	あの強力打線に二軍上がりのルーキーじゃとても通用しないよ。 
\\	「通用しない」をここでは
\\	「1イニングもたない」、あるいは、
\\	「もちこたえられない」で表現しています。	
\\	ご面会の方は、時間外通用門をお通り下さい。(病院にて) 
\\	で「時間外の入り口」の意味になります。	
\\	このマウス、使い勝手があまりよくないね。新製品なのに。 
\\	で「全く使いにくい」の意味。	
\\	立地条件と交通の便はなかなかいいのだけど、ちょっと使い勝手の悪そうな部屋なんだよね。 
\\	「使い勝手が悪い」を 
\\	(住みにくい)で表現しています。
\\	は「使い手にやさしい」の意。	
\\	外国のコインは日本に帰ったら日本円に両替できないので、現地で使い切れないといけない。 
\\	「使い切る」は
\\	と言います。	
\\	新しいコンピュータの使い心地はどう? 
\\	「使い心地」は
\\	のパターンを使います。	
\\	使い捨て製品は買わないようにしている。 
\\	「使い捨ての」は
\\	、または、
\\	と言います。	
\\	使ったら使いっぱなしだから、いろんなものがちらかっている。 
\\	ここでは「使いっぱなし」を
\\	「一度使いだしたら、そのまま使っている」で表現しています。
\\	「散らかす」。	
\\	早く5時にならないかな。朝のラッシュでもう今日1日の気力は使い果たしたって感じ。 
\\	が「使い果たす」に当たる動詞です。
\\	で「近づいてくる」、
\\	で「朝の通勤」の意。	
\\	これらの単語の使い分け方を教えて下さい。 
\\	「使い分け」をここでは
\\	「区別する」で表現しています。	
\\	日本から荷物を送るときは郵便局を使ったのだけど、この間
\\	会社の送料を見たら、それより安かったよ。 
\\	「使う」は 
\\	ですね。
\\	は「通常郵便」の意。(注)
\\	会社」のような表現は避けるようにしてください。	
\\	あいつ口ばっかりで、全然、使えないじゃない。 
\\	は「大口をたたく、でかいことを言う」の意味。
\\	は、まさに「口(話)だけ」のこと。「使えない」は、そのまま
\\	か
\\	「何の役にも立たない」でいいでしょう。	
\\	あいつ、使えない奴だよな。 
\\	「使えない」は、文字通り
\\	、あるいは、
\\	「何の役にも立たない」で表現します。	
\\	ハイハイしていた子供が、最近、つかまり立ちができるようになった。 
\\	「つかまり立ち」は
\\	「何かにつかまりながら、ぎこちなく歩く」で表現できます。	
\\	あの映画は悪くはなかったけど、いまひとつつかみ所がなかった。あの映画で何を言おうとしてたのかが理解できなかった。 
\\	「つかみ所がない」を 
\\	(要領を得ない、無意味な)で表現しています。(注)和文の一部をカットしました。	
\\	今頃旅の疲れがでてきた。もう年だな。 
\\	が「疲れが出る」に当たる表現です。
\\	「疲れが遅れて出る」。	
\\	最近今までの疲れがどっと出てきたのか、家に帰ると即寝てしまいます。 
\\	(疲労)と 
\\	を組み合わせて「疲れが出る」を表現しています。
\\	は「…に大きな被害を与える」の意。	
\\	年のせいか、この頃なんだか疲れがとれにくいんだよねぇ。 
\\	と
\\	を組み合わせれば「疲れが取れない」のニュアンスが出せます。	
\\	年を取ると、一度無理をするとなかなか疲れが取れません。 
\\	で「疲れを取る(取り除く)」の意味になります。	
\\	ここんとこ全然疲れが取れないんだよね。もうこの暑さで何もやる気もしないけど。 
\\	と 
\\	(疲労)を使って「疲れを取る」を表現しています。
\\	の使い方にも注目。	
\\	最近ちょっと出かけて、うろうろしただけで疲れる。年だな~ 
\\	で「疲れさせる」の意味。
\\	も使えます。
\\	は「ぶらつく」の意。	
\\	頑張って勉強した甲斐があった。やはりせっかく自分の英語を試せる機会があるのだから、使わない手はないと思う。 
\\	で「…しないのは意味がない」と言っていることになります。
\\	は「無駄にする」の意。	
\\	明日の飲み会行きたくないけど、皆行くから行かないとまずいかな・・・付き合いも大変だ。 
\\	で「(人間)関係を維持する」の意味になります。
\\	と交際する」。	
\\	彼女は、一番付き合いの長い友達です。 
\\	「付き合いの長い(友達)」は
\\	や
\\	を使えばその意味が表現できます。	
\\	パーティーがこんなに盛り上がっているのにもう帰るなんて、付き合いが悪いな。 
\\	で「それはないんじゃないの(そういう行動をとるなよ)」の意味になります。
\\	「非社交的な」。	
\\	彼と付き合い始めてどれくらいになるの。 
\\	「付き合う」は
\\	を使って表現できます。	
\\	ゴルフなんておもしろくないよ。私はただ彼氏に付き合ってやってるだけ。 
\\	(彼がゴルフをやるから)で「付き合う」を表現しています。	
\\	プログラム上でエラーの原因を突き止めるのは、まさに推理小説で犯人を突き止めるような感じ。 
\\	「突き止める」は
\\	「見つける」でいいでしょう。
\\	「突き止める」。
\\	「犯人」。	
\\	月並みな質問で、もう答えるのや聞くのはうんざりかとは思いますがほかに思いつかないので。どちらのご出身ですか? 
\\	は「月並みの、平凡な」という意味の形容詞。
\\	は「とても退屈な思いをする」、
\\	は「ありふれた」の意。	
\\	先週、久しぶりにバレーボールをやったら突き指をしてしまった。 
\\	「突き指する」は
\\	と言います。この
\\	は「捻挫する」にも使えます。例えば、
\\	は「足首を捻挫した」の意味。	
\\	よくそんなに次から次へとベラベラしゃべれるね。しかも英語で。 
\\	が「次から次」に当たる言い回しです。
\\	は「ぺちゃくちゃしゃべる」、
\\	は「早口にしゃべる」の意。	
\\	買いたい本が次から次へと出てきて、いくらお金があっても足りないな~。 
\\	が「次から次へと」に当たる表現です。
\\	する余裕がある」。	
\\	このCDって何の本に付いていたやつ? 
\\	この場合の「付く」は
\\	と一緒に付いてくる」、
\\	に伴う、属する」を使って表現すればいいでしょう。	
\\	痛い!! 脚の小指を机の角にぶつけた!こういうのが、一番痛いんだよね。 
\\	「角」は
\\	ですから、
\\	で「机の角」になります。
\\	「死ぬほど痛い」、
\\	「引っかける」。	
\\	友だちの結婚式で何かやって驚かしたいのですが、何かいいアイデアありませんか?例えば、ウエディングケーキを作ってあげるとか。 
\\	で「…を作ってあげるとか」の意味になります。	
\\	つくづくこの仕事に向いていないと思った。 
\\	ここでは「つくづく思った」を
\\	「痛烈に感じた」、
\\	「痛いほどハッキリと分かった」を使って表現しています。
\\	に向く」。	
\\	アメリカに来て、日本のことをいろいろ聞かれることが多く、もっと日本のことについて勉強しておけばとつくづく思います。 
\\	(…に痛切に感じられる)を用いて「つくづく」を表現しています。
\\	は「率直に…という印象を持つ」の意。	
\\	地下にリビングルームがあるなんて、日本ではあまりない作りです。 
\\	は「設計」、
\\	は「作られている」の意味。	
\\	日本の環境そのものが、コミュニケーションをとれない家庭状況を作り出しているのかもしれない。 
\\	この場合の「作る」は
\\	「創り出す」、
\\	「生み出す」を使って表現すればいいでしょう。	
\\	日頃の勉強を怠っていたツケがまわってきた 。 
\\	この
\\	は「雪だるま式に増える」の意味を表します。
\\	「つきまとう」。	
\\	悪いことばかりしているとそのうちつけがまわってくるよ。 
\\	「つけがまわる」とは、「結果的に、支払うことになる」ということですから、
\\	で表現できます。
\\	は
\\	追いつく」の意。	
\\	彼はあなたの彼に対する気持ちにつけこんでいるのよ。 
\\	は「…の弱みにつけ込む」、
\\	は「操る」の意味。	
\\	昨日、クーラーをつけたまま寝てしまって、どうも調子がよくない。 
\\	「つけたまま」は
\\	、あるいは、
\\	で表現できます。	
\\	昨日はラジオをつけっ放しで寝たでしょう。 
\\	で「ラジオをつけたままで」の意味。前置詞
\\	の使い方がポイント。
\\	でも同じような意味になります。	
\\	一夜漬けで勉強した英語は付け焼き刃で、
\\	の試験はぼろぼろでした。 
\\	「付け焼き刃」のニュアンスを 
\\	(多くは身に付かない)で表現しています。
\\	は「効き目がある、うまく行く」の意。	
\\	彼って、めんどうな事は一切やらずに人任せで、自分のやりたいことしかやらない。都合がいいよね。 
\\	で「自分にとって物事が楽に進むようにする」の意味になります。
\\	の後についていく」。	
\\	都合がつき次第、そちらに伺います。 
\\	で「都合がつきましたらすぐに」の意味。
\\	は「訪問する」の意。	
\\	来週末は仕事が夜当番なので、再来週なら都合がつくかもしれません。また電話します。 
\\	「都合がつく」は 
\\	を使って表現すればいいでしょう。	
\\	私は特にその日は予定がないので、あなたの都合に合わせられます。でも、朝は弱いから午後がいいな。 
\\	~で「~に合わせて…を予定する」の意味になります。
\\	で「朝は弱い」の意味。	
\\	土曜日なんだけど、どうする?どこで何時に待ち合わせしますか?そっちの都合のいい時間でいいよ。 
\\	ここでは「都合のいい」を簡単に 
\\	を使って表現しています。	
\\	お前の言っていることはまるでつじつまがあわない。 
\\	は「矛盾する」の意味です。「まるで」を
\\	「完全に」で表現しています。	
\\	お友達にもよろしくお伝えてください。(手紙文) 
\\	「…によろしく」は
\\	、あるいは、
\\	で表現します。	
\\	私が今迄学んできたことを伝えたいし、アメリカでの仕事の進め方も習得したい。 
\\	には「渡す」という意味がありますので、これを使えばいいでしょう。
\\	「蓄積する」。	
\\	化学物質は土にかえらないから困るんだ。 
\\	「土にかえる」とは
\\	(生物分解する)ということですね。
\\	は「分解する」の意。	
\\	あなたがそうやっていつも突っ掛かってくるからけんかになるんでしょ。 
\\	で「好戦的な姿勢を取る」の意味になります。	
\\	その件に関しては、もう少し突っ込んだ話をした方がいい。 
\\	で「…をもう少し深く掘り下げる」の意味になります。	
\\	私もたまにずる休みとかして、人のこと言える立場ではないので、それ以上は突っ込まなかった。 
\\	「突っ込む」とは
\\	(追及する)ということですね。
\\	は「ずる休みする、さぼる」、
\\	は「鋭い」の意。	
\\	彼女って、誰にも言わないでねって言いつつ、皆に言い回っているんだもん。ホント呆れるよ。 
\\	は「その間中」の意味です。
\\	は「…する一方で」の意。	
\\	どうして悪いことって続いて起きるんだろう。 
\\	「続いて起きる」は
\\	でいいでしょう。	
\\	さっきの話の続きだけど、何で彼のこと嫌いになっちゃったの? 
\\	この場合の「続き」は
\\	「…に戻る」を使って表現すればいいでしょう。	
\\	この特別コースは、1回2時間の授業で3カ月続きます。 
\\	「続く」は 
\\	でいいでしょう。
\\	(会う)の使い方にも注目です。	
\\	ここのところ、毎晩パーティー続きでろくに自分の時間がなく忙しい毎日でした。 
\\	と現在完了進行形を用いて、「続く」を表現しています。	
\\	うちはインターネットと電話線と同じ回線を使っているので、夜は電話がつながらないことが多いと思います。 
\\	「つながらない」は 
\\	で表現できます。
\\	は「インターネット接続」の意。	
\\	この電話でフリーダイアルってつながりますか? 
\\	この場合の「つながる」は
\\	「(フリーダイヤルに)かけられる」ということですね。	
\\	そんな角が立つような言い方をしなくてもねえ。 
\\	「どぎつい言い方をする」、
\\	「もっと優しい言い方を見つける」の意	
\\	私はつぶあんよりこしあんのほうが好き。 
\\	とは「ピューレ(こしたもの)」、
\\	は「果肉」のことです。これで「こし」と「つぶ」を表現してみました。
\\	「かたまりの入った」。	
\\	つぶしがきかない仕事だからねえ。 
\\	「つぶしがきない」は「他の職種に使える技術が身に付かない」ということと考えて、
\\	と英語で表現します。
\\	を主語にすると動詞は、
\\	を使うことになりまうす。	
\\	自分で出したエラーとは言え、つぶすのも一苦労だ。(プログラミングの際) 
\\	この場合の「つぶす」は、
\\	「直す」、
\\	「取り除く」ということですね。	
\\	オフィス何坪あるかがステータス? 
\\	英語では「坪」とは言いませんので、ここでは
\\	「大きさ、広さ」、
\\	「四方の広さ」で表現します。	
\\	くくくく。面白すぎる!ツボに入ったよ! 
\\	とは「ユーモアのセンス」の意味。
\\	「笑い転げる」。	
\\	この靴、つま先のあたりが少しきつい。 
\\	「つま先」は 
\\	ですね。「きつい」は 
\\	で表現します。	
\\	退屈なんじゃない、毎日がつまらないの。 
\\	この場合の「つまらない」は 
\\	(単調な)、
\\	(退屈な)で表現すればいいでしょう。	
\\	あなたの気持ちはうれしいけれど、私、つまらない女よ。(プロポーズされた時などに) 
\\	ここでは「つまらない女」を
\\	と
\\	を用いて表現しています。
\\	は「…をうれしく思う」の意。	
\\	つまらないものですが気に入っていただくとうれしいです。 
\\	「つまらないもの」に当たる英語は
\\	でしょうか。英語では、実際につまらない物でも、そうは言いませんので、注意が必要です。	
\\	もう予定がつまっていて、これじゃゆっくりできやしない 
\\	で「あふれるほど一杯である」という意味。
\\	「詰まっている」。	
\\	つまるところ、その原因は何なんだ。 
\\	は「結局のところ」、
\\	は「最終的には」の意味。	
\\	アル中って日々の積み重ねでなる病気でしょ。だから治るにも同じ年月が必要なんだって。 
\\	で「徐々に蓄積する」の意味になります。
\\	は「(いろいろの物が混ざって)混合物になる」の意。	
\\	積立預金の利息は何パーセントですか? 
\\	が「積立預金」に当たる表現です。
\\	は「利子」の意。	
\\	そんなに優しくしたって罪滅ぼしにはならないよ。 
\\	の文字通りの意味は「本当の謝罪の代わりにはならない」です。
\\	(償う)を使ってもいいでしょう。	
\\	爪が伸びたからそろそろ切るか 
\\	「伸びる」は
\\	で表現します。
\\	「切りそろえる」の意。	
\\	爪が伸びてるから切りなさいよ。 
\\	「爪」は
\\	、「伸びる」は
\\	で表現できます。	
\\	ちょっと約束に遅れたくらいでそんなに冷たい目で見るなよ。 
\\	で「人を冷たい目で見る」の意味になります。
\\	は「冷たい視線」の意。	
\\	彼、最近、私に冷たくなったような気がする。 
\\	で「冷たくなる」の意味になります。	
\\	彼には人を思いやる気持ちなど爪の垢ほどもない。 
\\	「爪の垢ほどもない」を
\\	「少しもかまわない」で表現しています。	
\\	彼もとうとう詰め腹を切らされたってわけだね。 
\\	「詰め腹を切らされる」とは
\\	「無理矢理辞めさせられる」ことです。	
\\	ズボンの丈を2センチほど詰めてもらえますか。 
\\	この場合の「(丈を)詰める」は
\\	を使います。
\\	「内側の縫い目」。	
\\	席を少し詰めてもらえますか。 
\\	この場合の「詰める」は
\\	と
\\	を組み合わせて表現すればいいでしょう。	
\\	単語と単語の間をもう少し詰めなさい。 
\\	「間を詰める」は動詞の 
\\	と 
\\	を組み合わせれば表現できます。	
\\	自分では、分かったつもりでいたけど、改めて考えると分かんなくなってきた。 
\\	この場合の「つもり」は
\\	「確信を持っていてた」、あるいは、
\\	「そう思っていた」と言い換えてもいいでしょう。	
\\	うお!すごい雪。これは積もるね。積もるのはいいんだけど、その後凍ると歩きにくいのよね。 
\\	には「積もる、積み重なる」という意味があります。
\\	「凍る(自動詞)」。	
\\	会社時代の親しい友人が死んだので、通夜に参列しなければならない。 
\\	が「通夜」にあたる語です。
\\	「親しい、近しい」。	
\\	うーんやっぱり、艶消しの金のネックレスがいいわ。 
\\	で「艶消し仕上げ」の意味になります。	
\\	梅雨の季節は、気分も滅入る。 
\\	「梅雨」は 
\\	ですね。
\\	は「…を落ち込ませる」の意。	
\\	梅雨寒って、イヤー。 
\\	ここでは「梅雨寒」を「梅雨とともにやって来る寒さ」と表現しています。
\\	「寒さ、冷え」。	
\\	強がってあんなこと言ったけど、本当はすごくさびしい。 
\\	で
\\	に対して平気な顔をした」という意味になります。
\\	「動じない」。	
\\	彼女はいつも強気な発言をする。 
\\	「強気な発言をする」とは
\\	「厳しいことばを使う(強硬な言い方をする)」ということですね。	
\\	ごめんね。弱音ばかり言って。ホント正直言って不安なことだらけでさ。もっと強くならないとね。 
\\	この場合の「強くなる」は 
\\	でいいでしょう。
\\	は「めそめそ愚痴をこぼして泣き言を言う」の意味。	
\\	想いが強ければ強いほど、裏切られたときのショックって大きいよね。 
\\	、あるいは、
\\	を
\\	比較級、
\\	比較級」の構文で使えば、「強ければ強いほど」が表現できます。	
\\	今回のオーストラリアの滞在で、実際に現地の生活に触れてみて、やはり子供のためにもそこに住む決意が強まりました。 
\\	「決意が強まる」は
\\	「以前にもまして確信する」ということですね。	
\\	どんな事があってもめげないのが、彼の強みだよね。 
\\	「強み」は
\\	で表現できます。ここでは
\\	と
\\	をめいらせる」の使い方にも注目。	
\\	三カ国語も話せるなんて、かなりの強みだよ。それを前面に出して履歴書書いたほうがいいよ。 
\\	「強み」は 
\\	(有利な点)で表現すればいいでしょう。
\\	と 
\\	の使い方にも注目。	
\\	右目だけ極端に視力が悪くて、ぼやけてよく見えないんだ。これって結構辛いんだよ。 
\\	「辛い」は 
\\	で表現できますね。
\\	は「ぼんやりした」の意。	
\\	人生、辛いことばかりじゃないよ。 
\\	(苦難と苦痛)、
\\	(試練と苦難)で「辛いこと」を表現しています。	
\\	日本での生活で辛かった事は何でしたか? 
\\	ここでは「辛かった」を
\\	と最上級を使って表現しています。	
\\	沖縄の音楽は、最近の音楽と違い、流行りとは無縁で沖縄の意志を貫いているようで日本の誇りだと思う。 
\\	ここでは「貫く」を
\\	「忠実な」を用いて表現しています。
\\	は「…を忠実に守る」の意。	
\\	あんなに面の皮の厚い奴は見たことがない。 
\\	は「図々しい、厚かましい」の意味。	
\\	手がつっちゃったよ。 
\\	「つる」は
\\	と言いますが、これは足にも手にも使えます。
\\	「痙攣する」。	
\\	お前らつるんでるんじゃないの。どうもあやしいな。 
\\	「つるむ」のニュアンスを
\\	と
\\	で表現しています。
\\	「でっち上げる、ごまかす」、
\\	「たくらむ」。	
\\	年を取るにつれて、だんだん洋菓子より和菓子の方が好きになってきた。 
\\	「~につれて」は
\\	で表現できます。第二英訳の
\\	の使い方にも注意してください。	
\\	動物を連れて飛行機に乗ることは許されているのですか?そうであれば、一緒に連れていってあげれば? 
\\	「一緒に連れていく」は 
\\	を使って表現できます。
\\	も使えます。	
\\	私なりにどこに連れて行ってあげたらいいか考えておくね! 
\\	「連れて行く」は 
\\	で表現します。
\\	は「…について少し考える」の意。	
\\	英語の話せる者を連れてきます。 
\\	この場合は
\\	「行って連れてくる」で表現すればいいでしょう。
\\	を捕まえる」。	
\\	そんな~。つれないこと言わないで。 
\\	「つれないことを言うな」とは「優しくしてくれ」ということなので、
\\	を使えばうまく表現できます。
\\	「痛い」は、文脈によっては「それはきついね、きついこと言うね」の意味にもなります。	
\\	犯人は、40時間もその少年を連れ回したそうだ。 
\\	で
\\	を連れてあてもなくさまよう」の意味になります。
\\	「誘拐する」。	
\\	なんで今日はそんなにつんつんしてるんだよ。 
\\	「つんつんする」を
\\	「機嫌が悪い」を使って表現しています。	
\\	このミント味の飴、鼻につんと来るね。 
\\	「鼻につんと来る」は
\\	「鼻に真っ直ぐ行く」で表現できます。	
\\	ほとんどの競技は、昼間行われるんだよね。となると、真っ昼間からオリンピックづけ? 
\\	で「オリンピック漬けになる」の意味。(例)
\\	「昨日は相撲漬けだった」。
\\	の
\\	とは、喉の所に手を当てて、一杯だった状態(喉の所まで来ている)を示す、ということです。	
\\	電車もエレベーターみたいに定員を作れば良いのに。例えば、従量制とかでオーバーすると警告音がなるとか。 
\\	「定員」は
\\	「決められた乗客の収容人数」で表現できます。
\\	「最適の」。	
\\	従量
\\	ていうか、テニスをしたいだけなんだよね。 
\\	「ていうか」をここでは「あの、僕が言いたいのはね」と言い換えています。	
\\	彼は幼少時から帝王学を仕込まれてきたんだ。 
\\	ここでは「帝王学を仕込まれる」を「会社の社長になるように仕込まれる」の意味に解釈して
\\	で表現しています。
\\	「幼少から」。	
\\	最近は、定額も通常貯金も利率は変わらないね。 
\\	で「定額貯金口座」の意味。
\\	「通常貯金」。	
\\	定期テストが、もうすぐ始まるんだ。だから勉強しなくちゃいけないの。 
\\	「定期テスト」を 
\\	、あるいは、
\\	で表現しています。	
\\	日本人としては、土足で家に上がるというのはどうも抵抗感があります。 
\\	「抵抗感」のニュアンスを
\\	(ちょっと変な感じがする)で出しています。	
\\	雪で多少電車が遅れているようですが、試験は定刻通り始めます。 
\\	で「予定通り進む」の意味になります。
\\	(…により、通りに)	
\\	あいつ亭主関白だって言ってるけど、実は女房の尻に敷かれてるんだ。 
\\	は「威張り散らす」の意味。
\\	短いロープで繋いでいる(尻に敷いている)」、
\\	「(亭主を)尻に敷く」。	
\\	なんで日本では、お化けというと夏っていうイメージが定着しているんだろう? 
\\	「定着している」をここでは
\\	につきもの」を使って表現しています。
\\	に密接に結びついている(連想させる)」。	
\\	仕事が手いっぱいなので、これ以上手伝えません。 
\\	とは「山ほど仕事がある」、
\\	「たくさんの仕事を抱えている」の意味になります。	
\\	あっ、停電だ。真っ暗で何も見えないよ。 
\\	「停電」は
\\	と言います。
\\	「電気の」。	
\\	彼女程度の英語力で、受付業務ってできるんだ。 
\\	の
\\	で「程度」の意味が表せます。
\\	は「彼女と同じくらい英語を知っていれば」の意。	
\\	いくら定番フレーズを習得しても、どうして(英語が)全然上達しないのだろう。 
\\	「定番フレーズ」は
\\	でいいでしょう。
\\	「向上する」。	
\\	すみません、このバックは定番ですか?他に何色を作ってるんですか? 
\\	「定番」を
\\	(標準的な)、
\\	(標準の)を使って表現しています。	
\\	警備の手薄な建物を犯人に狙われた。 
\\	で「警備の厳しくない」という意味になります。
\\	でも同じようなことが言えます。	
\\	(うちの)旦那の煙草のせいで、壁が真黄色。あれだけ汚くなったらもう手後れだよ。きっと、掃除しても色落ちないよね。 
\\	「手遅れ」は
\\	で簡単に表現できます。ここでは動詞
\\	「黄ばむ」の使い方にも注目。	
\\	手加減して叩いたんだから、痛いわけないじゃん。 
\\	「手加減する」は
\\	「まともに」や
\\	「きつく」を使うことで表現できます。	
\\	手があいてたら、こっちに来て手伝ってくれないか。 
\\	「手があいている」は 
\\	、あるいは、
\\	で表現できます。	
\\	そんなに高いんじゃ、とても手が出ないな。もう少し安くならない? 
\\	「手が届かない」は
\\	「手の届く範囲を超えている」で表現します。
\\	は「(値段を)まける」の意。	
\\	今、手がふさがっているから、だれか電話出てくれない? 
\\	「手がふさがっている」は 
\\	で表現します。
\\	は「(忙しくて)手が離せない」の意。	
\\	野菜ジュースは飲み安いし、一番手軽に野菜を補給できる手段だと思う。 
\\	「手軽」をここでは
\\	「最も簡単な方法の一つ」と言い換えています。	
\\	どうやら私は速記者に向いているらしい。昔、適性検査で向いている職業の欄によく書かれていた。 
\\	「適性検査」は
\\	と言います。
\\	に向いている」。	
\\	この単語のスペル適当だから、あんまり当てにしないでね。 
\\	で
\\	あまりはっきりしない、自信がない」の意味。
\\	を頼りにする」。	
\\	適当に口座番号を入力した場合、ちゃんとそれが実際存在しているものかってすぐにわかるのかな? 
\\	この場合の「適当」は 
\\	(でたらめの、任意の)を使って表現すればいいでしょう。
\\	は「連続」の意。	
\\	なんだかよくわからないけど、適当にいろいろな所をいじっていたら直った。 
\\	で「適当にいじくる」という意味になります。	
\\	それ適当にやればいいから。 
\\	「適当にやる」は、
\\	「好きなようにやる」、あるいは、
\\	「注意をしてやる必要はない」で表現します。	
\\	忙しすぎず、暇過ぎず、適度に仕事があるのが一番いいのよね。 
\\	で「ちょうどいい量」という意味になります。	
\\	主に、システムの開発をしていますが、システムエンジニア的な仕事もします。システムの提案、保守等。 
\\	が
\\	的な仕事」にぴったりの言い方ですね。	
\\	彼女は、てきぱきと仕事を片づけた。 
\\	まず、「片づける」は
\\	でいいでしょう。ポイントとなる「てきぱき」は
\\	「時間を無駄にせずに」、
\\	「有効に」で表現できますね。
\\	「精力的に取りかかる」の意。	
\\	この間、車で事故ったんだけど、自分の車の修理には保険が適用しないんだよ。 
\\	この場合の「適用」は
\\	「含む」を使えばいいでしょう。	
\\	新しい移民法が先月末から適用されましたが、それによるとかなり規制がゆるくなった気がします。 
\\	ここでは「適用される」を 
\\	(実施される、施行される)、
\\	(実施される)で表現しています。	
\\	外貨交換の手続きが早いかどうかは、そこの銀行の大きさや時間帯、担当者の手際の良さによると思う。 
\\	「手際の良さ」は
\\	「能率の良さ」、
\\	「手腕、技量」で表現すればいいでしょう。
\\	とは
\\	のことです。	
\\	あいつと話していても何の反応もしないから、手ごたえがない。 
\\	「手応えがない」とは
\\	「反応がない」ということですね。	
\\	このジャケット、手頃な値段よね。 
\\	「手頃な」は
\\	を使えばいいでしょう。ここでは
\\	が動詞として使われています。
\\	「手頃な価格」。	
\\	これといった解決策がなくて、手探り状態です。 
\\	で「暗闇の中で手探りする(暗中模索)」の意味になります。	
\\	こんなことで手数料を2000円も取るの? 
\\	とは「(金融取引の)手数料」のことです。	
\\	このカードを作った際に申請書に書いたサインを変更したいのですが手数料はかかりますか? 
\\	「…に手数料はかかりますか」は 
\\	で表現できます。
\\	は「手数料を払う」の意。	
\\	僕はてっきりお前がコピー機を壊したと思っていたよ。 
\\	「てっきり」のニュアンスは
\\	(確信する)で表せます。	
\\	ドイツでは、ゴミの分別等リサイクルに関しては徹底しているよね。 
\\	で
\\	に徹底している」の意味。
\\	「徹底した、十分な」。	
\\	私ってかなりの負けず嫌いで、何か人が知ってて自分が知らなかったりしたことがあると、家に帰ったら徹底的に調べます。 
\\	で「…を徹底的に勉強する」の意味。
\\	の動詞としての使い方にも注目。	
\\	夕飯は手っ取り早くピザですませよう。 
\\	は「さっと作る」、
\\	は「がつがつ食べる」の意。	
\\	もしこの英語のクラス続けるのであれば、今月中に更新手続きをしないといけないんです。 
\\	「更新手続きをする」は
\\	を使えばいいでしょう。
\\	は「申し込み、申請」の意。	
\\	手続きを取りますのでこちらへ来てください。 
\\	ここでは「手続きをとる」を
\\	「処理する」を使って表現しています。
\\	は「用紙」、
\\	は「申込書」のこと。	
\\	ぼくは鉄棒が得意です。 
\\	「鉄棒」は
\\	と言います。
\\	は「「鉄の棒」のことです。	
\\	私は子供のころ鉄棒が苦手だった。 
\\	学校の校庭にあるような鉄棒は
\\	と言います。体操競技に使う鉄棒は
\\	ですね。	
\\	なんだかんだ引かれて手取り15万円くらいだよ。やっていけないよ。 
\\	「手取り」は 
\\	と言います。
\\	は「差し引き」の意。	
\\	彼がソロを歌うとき、音程が外れるんじゃないかって、手に汗握っちゃうよね。 
\\	「手に汗握る」に当たる表現は
\\	と言います。	
\\	彼の怠けぐせは、もう手に負えない。 
\\	「手に負えない」は
\\	「どうしていいか分からない」で表現することができます。	
\\	ここ数ヶ月きちんと掃除してなかったもんだから、一度にやろうとしてらもう大変。 汚すぎて手に負えない。 
\\	で「自分が何とかできる範囲を越えている」と言っている。	
\\	手に職系は転職がラクよね~ 
\\	「手に職がある」とは
\\	「売れる技術を持っている」で表現できます。	
\\	あなたの手紙からどんなに楽しかったのかが手にとるようにわかります。 
\\	で「きわめてハッキリと」の意味。	
\\	雑誌を手に持ったままでは、本屋とかコンビニには入りづらいね。やっぱり、入る前に鞄の中にしまっちゃうよ。 
\\	前置詞
\\	と
\\	の組み合わせポイント。	
\\	彼って手の掛かる人だね。 
\\	は「注意して...を扱う」、
\\	「多くの世話を必要とする」の意味。	
\\	手の平を返したように、急に違う態度をする人っているよね。 
\\	とは「がらりと変える(とんぼ返りをする)」の意味。	
\\	今回の社員旅行の手配の方、よろしく頼むね。 
\\	「手配」は
\\	「準備、手配」、あるいは、
\\	「手はずを整える」を使えばいいでしょう。	
\\	彼のお見舞いに行くのに、手ぶらでは行けないだろう。 
\\	で「手ぶらで行く」の意味になります。	
\\	手前味噌になりますが、当店の豚カツは最高の素材を使っておりますので。一度、是非、召し上がってみてください。 
\\	「自慢をするわけではありませんが」で「手前味噌になりますが」のニュアンスを表現しています。「豚カツ」の表現の仕方にも注目。	
\\	あまり時間がないので、手短に用件を言ってくれ。 
\\	「手短」は 
\\	(手短な、簡潔な)で表現すればいいでしょう。	
\\	遅れても知らないからね! 
\\	この場合の「ても」は
\\	で表現します。	
\\	手持ちがなくてもクレジットカードがあれば買えちゃうからついつい買ってしまって、後がこわいよ。 
\\	「手持ちの金がない」は
\\	で表現できます。
\\	があるのでより日本語に近い言い方ですね。	
\\	電車の中で席を譲るのって、照れくさいよね。 
\\	ここでは「照れくさい」を
\\	「きまりが悪い」で表現しています。	
\\	そんなふうに言われるとなんか照れるな。 
\\	とは「恥ずかしくなる、はにかみ屋の」の意味。	
\\	日本人同士で英語話すとなんか変な感じだし照れる。 
\\	ここでは「照れる」は
\\	「きまりが悪い思いをする」で表現しています。
\\	は「変な」の意。	
\\	まいったなー、照れるなー。そんなに褒めないでよ。 
\\	「照れる」のニュアンスを
\\	(何を言っていいかわからない)と
\\	(顔が赤くなる、恥ずかしくなる)、
\\	(照れくさい思いをさせる)を使って表現しています。	
\\	旅行!?今は怖くて行けないよ。きみはニューヨークでのテロについてどう思いますか? 
\\	「どう思う」は 
\\	ですね。「テロ」は 
\\	でいいでしょう。
\\	は「…に関するきみの見方は?」の意。	
\\	手を出して運転してると危ないよね。 
\\	で「窓から手を下に下げる」の意味になります。	
\\	何事も中途で辞めるのが嫌いで、一度手を付けたら最後までやらないと気が済まない。 
\\	で
\\	に着手する」の意味になります。
\\	に手を付ける」。	
\\	少しは手を抜くことも覚えた方がいいよ。 
\\	この場合の「手を抜く」は
\\	「力をセーブする」で表現できます。
\\	「保存する」の意。	
\\	彼が彼女の手をわざと放したシーンあたりから、ちょっと彼は怪しいと思っていた。 
\\	「放す」は
\\	を使って表現します。	
\\	通りの向こうで手を振っている人がいるけど、きみの知り合いかい? 
\\	「手を振る」は 
\\	ですね。
\\	は「通りの向こう」の意。	
\\	こんなことで手を煩わせてしまってすみませんね。 
\\	「手を煩わす」とは
\\	「やっかいをかける」ということですね。	
\\	早く来週にならないかな。だって、来週の予告によるとすごい展開になるらしい。待ち遠しい! 
\\	ここでは「すごい展開になる」を
\\	「何かとってもすごいことが起こる」で表現しています。	
\\	それさえあれば、天下御免だ。 
\\	「天下御免」をここでは「自由になんでもできる」と解釈して
\\	と表現しています。
\\	「名声」、
\\	「富」。	
\\	太陽が出ているのに雨が降っている。天気雨って言うんだね。こういうの。 
\\	ここでは一部日本語を使って「天気雨」を説明しています。	
\\	彼は機嫌が良かったり悪かったりと、お天気屋だ。 
\\	「天気が変わるように気まぐれに気分が揺れる」で「お天気屋」を説明しています。
\\	「気分屋の」。	
\\	天気予報は晴れだって言ってたのに、何で雪が降るのよ、うそつき! 
\\	「天気予報」は
\\	ですね。
\\	で 「天気予報では…と言っていた」の意味になります。	
\\	40過ぎて、外国の支店に転勤になるとは夢にも思わなかった。 
\\	動詞
\\	はには「転勤させる」という意味があります。名詞としても使えます。	
\\	我が家は転勤族なので、2、3年毎の転勤で、引越し貧乏です。 
\\	3年毎に転勤する」、
\\	「絶えず引っ越しをしている」で「転勤族」のニュアンスを出しています。
\\	「一文無しになる」、
\\	「ぎりぎりでやっていく」。	
\\	シックス・センスに出ているあの男の子は天才子役以外の何者でもない。 
\\	は「天才」、
\\	は「〔天賦の)才能を持った」の意味。	
\\	この不景気なご時世に、転職しようとしても仕事なんてないよ。 
\\	で「転職」の意味になります。	
\\	僕は、教師が天職だと思っています。 
\\	は「天職、神のお召し」の意味。	
\\	彼女とはテンションが同じだから、一緒にいて疲れない。 
\\	ここでは「テンション」を
\\	「エネルギーのレベル」を使って表現してみました。
\\	「緊張」はそのまま使えませんので。2番目の英訳では、お互いにテンションが高ければ
\\	、低ければ
\\	を使うことになります。
\\	「疲れさせる」。	
\\	あの娘はいつも天真爛漫でいいよね。 
\\	で「子供のような天真爛漫さがある」の意味になります。
\\	「たくらんでいない」。	
\\	天井裏に上がって電気配線を調べてもらえますか? 
\\	「天井裏」は
\\	「屋根の下にあるはって入れるスペース」、あるいは、
\\	「屋根裏」で表現すればいいでしょう。	
\\	この部屋、天井が低いね。 
\\	「低い」は 
\\	ですね。
\\	を主語にした場合には 
\\	を使う点に注意。	
\\	あの会社の株は天井知らずだ。 
\\	は「天井を突き抜ける」の意味。
\\	「上限がない」。	
\\	あの二人、お互いに天敵同士だと思ってるみたいだよ。 
\\	「天敵」は
\\	とそのまま英語にすれば通じます。
\\	「確信している」、
\\	「見なす、扱う」。	
\\	お前は、社会の仕組みというものをてんで分かってないね。 
\\	で
\\	がまるで分かっていない」の意味。	
\\	彼女のぼけっぷりは天然だよね。 
\\	「天然」はそのまま 
\\	で表現できます。
\\	は「あほ」、
\\	は「軽薄さ、突飛さ」の意。	
\\	彼女、天然ぼけだから。 
\\	「天然ボケ」をここでは
\\	「本当にボーっとしている」、
\\	「ボーっとしている人」で表現しています。	
\\	国立競技場、鹿島対磐田の天王山決戦は、延長の末鹿島が勝利し首位をキープ。勝ち点差を4に増やした。 
\\	日本語を忠実に英語に訳してみました。「天王山」は、
\\	「雌雄を決する(すべてか無か)」、
\\	「今後を左右する」を使って表現しています。
\\	「保つ」、
\\	「押し上げる」、
\\	「やってのける」、
\\	「進ませる」の意。	
\\	あなたからのメールに添付ファイルは付いていましたか?私のメールソフトからは、開けなかったので残念です。 
\\	「添付ファイル」は
\\	と言います。
\\	は省略できます。	
\\	今夜接待することになっているクライアントは、エビの天麩羅が大好きなんだよ。 
\\	「天麩羅」はそのまま
\\	と言えばいいでしょう。具体的に説明すると
\\	となります。
\\	「仕事の関係者」。	
\\	天網恢々疎にして漏らさず。悪いことはできねえ! 
\\	ここでは「天網恢々疎にして漏らさず」を 
\\	で表現しています。	
\\	恢
\\	今日はごはん作るのめんどくさいから、店屋物でも取ろうか。 
\\	で「何か配達してもらう」、
\\	「夕食を頼む」。これで「店屋物を取る」の意味を表します。	
\\	今夜の夕食は店屋物にしよう。 
\\	で「何かを出前してもらう」の意味になります。
\\	「注文する」。	
\\	あの2人は、出会い系サイトで知り合って付き合い始めたんだって。 
\\	で「相手を捜すためのサイト」の意味になります。	
\\	あそこは、英語を話す場というより、出会いを目的で来る人が多いと聞いたので、あまり行く気がしないのですが。 
\\	「出会い目的」を 
\\	(デートの相手を捜しに行く)で表現しています。
\\	は「場所」の意。	
\\	アナタと出会えて良かった。 
\\	「出会う」は
\\	、あるいは、
\\	(お互いの道が交わった)で表現できます。	
\\	昨日の夜は泥酔してしまって、どうやって家に帰ったか覚えてないんだ。 
\\	とは「酔ってボーッとする」という意味。
\\	で「ぐでんぐでんに酔って」の意。	
\\	我が社でも情報をデータベース化する必要がある。 
\\	で
\\	のデータベースを構築する」の意味。	
\\	あっ!それなんだっけ? ここまで出かかっているのに。 
\\	「舌先に乗っている」が「ここまで出かかっている」に当たる言い回しです。	
\\	(水道の)水の出がよくないな。何か詰まってるのかな。 
\\	で「流れがよくない」の意味になります。
\\	「蛇口」。	
\\	水の出が悪いな。どこか詰まってるのかな。 
\\	「水の出が悪い」ということは
\\	「水がいっぱい出てこない」ということですね。
\\	「水圧が低い」。	
\\	出来が悪い生徒でごめんなさい…。 
\\	「できが悪い」は 
\\	で表現すればいいでしょう。
\\	は「失敗者」の意。	
\\	この前のパーティーでたくさんの友達ができたんだ。 
\\	「友達ができる(作る)」は
\\	と言います。	
\\	私、ほかに好きな人ができたの。だからあなたとはこれ以上つきあえない。別れてください。 
\\	で「ほかに興味を持っている人がいるの」という意味になります。
\\	の使い方にも注目。	
\\	奥さんに逃げられたのは彼にも問題がある。他人がとやかく言う問題ではないとは思うけど、どんなできた奥さんだってあれじゃ我慢できないでしょう。 
\\	「できた」は
\\	で表現すればいいでしょう。
\\	「こんなことが言える資格はないが」の意。	
\\	う~ん、出来たてよりは劣るけど、でもおいしいよ。(ピザを食べていて) 
\\	は「焼きたての」という意味。
\\	は 
\\	を強調するために使われています。	
\\	そのアイドルタレントは、「できちゃった婚」をした。 
\\	で「予期しない妊娠で結婚する」の意味になります。	
\\	写真で顎のあたりに影ができるとちょっと痩せて見えるのよね。 
\\	この場合の「できる」は
\\	「現れる、見える」ということですね。	
\\	ねぇ、彼女できた? 
\\	この場合の「できた」は 
\\	で表現すればいいでしょう。	
\\	この英文チェックしてくれませんか?ちょっと量が多いのですが、できるだけでいいので。 
\\	「できるだけ」は
\\	で表現します。	
\\	英語だけでコミュニケーションとれるようになりたい。 
\\	で
\\	することができるようになる」の意味です。	
\\	本当に木偶坊なんだから、あんたは。 
\\	「木偶の坊」とは
\\	「役立たず」を使って表現します。
\\	とは「何一つとしてちゃんとできるものがない」の意。	
\\	"あの二人はでこぼこコンビだね。 
\\	60年代から70年代にかけてアメリカで
\\	""というテレビ番組が人気を博しました。日本でも「おかしな二人」というタイトルで放映されました。まさにあの二人が「でこぼこコンビ」すね。「正反対の二人」と言いたければ、
\\	を使って表現できます。また
\\	お仕事、お疲れさまでした。 
\\	「お疲れさまでした」は状況に応じて表現しなければなりません。ここでは「仕事」ということで、
\\	を使っています。	
\\	彼女はでしゃばりだから、皆から嫌われるんだ。 
\\	は「我が強すぎる」、
\\	は「でしゃばり」の意味。	
\\	お前がでしゃばることはない。あいつにやらせればいいんだ。 
\\	は「口をはさむ」、
\\	「関わる」という意味です。	
\\	大会9日目、ベスト8が出揃った。 
\\	「出揃った」は
\\	「生き残る」、
\\	「残る」を使って表現します。「ベスト8」には
\\	という言い方もあります。	
\\	こんなに好条件が出揃ったんだから、これは買うしかないでしょ! お客さん! 
\\	で「どうして他の物に決めるのか(それはないでしょう)」と言っているのです。
\\	「(素晴らしい商品なので)売れる」。	
\\	ねねね、あそこの宝くじ売り場から一等がでたんだって。ほら、お店の前に書いてある。 
\\	この場面での「出た」は
\\	「売った」、あるいは、
\\	を使って表現すればいいでしょう。	
\\	どうなるか分からないけど、でたとこ勝負で行こう。 
\\	「出たとこ勝負」には
\\	という決まった言い方があります。
\\	は「流れにまかせて、何とかする」の意。	
\\	そんなでたらめなことを言うなよ。 
\\	は「支離滅裂な」、
\\	は「とりとめもなくしゃべる」の意味。	
\\	この曲の出だしが好きなの。 
\\	で「…の始まり方」という意味になります。
\\	「(曲の)最初の小節」。	
\\	新生活の出だしからトラブル続きでホントまいった。でも、ある程度は予期してたけどさ。 
\\	ここでは「出だし」を 
\\	(初めから)で表現しています。
\\	は「(問題などが)持ち上がる」、
\\	は「予想する」の意。	
\\	6月の国際会議に間に合わせるために、論文をやっつけ仕事ででっち上げなければいけないんだ! 
\\	で「質の悪い物をさっと作る」の意味になります。
\\	「手早く作る」。	
\\	おたくのウェブサイトでオンラインサインアップをしたけど、エラーメッセージが出てサインアップできなかった。 
\\	「エラーメッセージが出た」を 
\\	を使って 
\\	で表現しています。	
\\	昨日の夢に、ゴジラとウルトラマンが出てきた。 
\\	で「登場する」の意味。
\\	「ゴズィラ」、
\\	「アルトラマン」の発音に注意。	
\\	日本チームは、最初固くなっていたが、そのあと持ち味がだんだんと出てきた。 
\\	「出てくる」を 
\\	(現れる)で表現しています。「持ち味」の表現の仕方に注目。	
\\	(英語で質問されたとき) あなたの言っている事は分かっているんだけど、私の英語力では、返事の文が出てこない。 
\\	「出てこない」は
\\	でいいでしょう。
\\	「答え方が分からない」。	
\\	彼は出不精だから飲み会に誘っても出てこないよ。 
\\	で「出掛けるのが嫌い」の意味になります。
\\	とハイフンを付けて形容詞にしている点にも注目。	
\\	今夜は料理作るの面倒だから、出前でも取るか。 
\\	「出前を取る」は
\\	「何かを届けてもらう」、
\\	「出前を頼む」を使って表現します。	
\\	できもしないことをできるなんて、そんな出任せを言うのは止めろよ。 
\\	で「いい加減なことを言う」、
\\	「うっかり... を言ってしまう」の意味になります。
\\	「無責任な」。	
\\	ここは通行量が多いからもっと前まで出ておかないと永久に曲がれないよ。 
\\	この場合の「出る」は 
\\	と 
\\	を組み合わせて表現すればいいでしょう。	
\\	彼は、でれでれした顔で新しい彼女のことを話してくれた。 
\\	「でれっとした(本来の意味は『どろどろの』)」。
\\	にも同じよな意味を表す。	
\\	これ何で動くの? 電気の力で動くんのかな? 
\\	この場合の「動く」は
\\	を使って表現します。
\\	「電気で」。	
\\	全然部屋暖かくならないね。暖房の電源いれてないんじゃないの? 
\\	で「電源が入っている」の意味になります。
\\	で「プラグを差し込む」の意。	
\\	いつも電車の時間に合わせて帰るから、同じ時間に帰っている。 
\\	「電車の時間に合わせる」は
\\	「同じ電車に乗る」と言い換えて表現すればいいでしょう。	
\\	毎日、パソコン使っているから、電磁派を浴びまくっているよ。 
\\	「電磁波」は
\\	と言います。
\\	は動詞で「おおわれる」の意味。	
\\	怒って不機嫌な時は、周りの人にまで伝染するから、それならば、逆にニコニコを伝染させた方が何倍も良い。 
\\	は「伝染する」、
\\	は「うつりやすい」の意味。	
\\	電池が切れたのかな。リモコンのスイッチを押してもテレビがつかない。 
\\	「(電池が)切れる」は
\\	を使って表現します。
\\	は「リモコン」のこと。	
\\	腕時計が止まってる。電池を交換しなくちゃ。 
\\	「電池」は
\\	と言います。交換は
\\	を使います。主語を
\\	にして、
\\	とも言えますね。	
\\	地下にいたから、携帯電話の電波が届かなかったんだ。 
\\	携帯電話の電波は
\\	と言います。ここでは
\\	と
\\	の使い方を覚えるようにしましょう。	
\\	電波が届かなくて声がよく聞こえないんだけど。 
\\	「電波が届かない」とは、
\\	「受信があまりよくない」ということですね。	
\\	今日の電話番はだーれ? 
\\	ここでは「電話番」を
\\	「電話に出る」を使って表現しています。
\\	「順番」、
\\	「見張る」の意。	
\\	彼は電話魔だね。暇があればいつでもだれかに電話をかけている。 
\\	(電話中毒になっている)で「電話魔」を表現しています。
\\	は「電話魔」の意。	
\\	知らないことにしておいてくれ。 
\\	この場合は、
\\	「忘れてくれ」、
\\	「振りをしてくれ」を使えばいいでしょう。	
\\	実際はしないんだけど、帳簿上、2万円入金して1万円引き出したことにしてくれる? 
\\	~で「…を~ということにする」のニュアンスを表現しています。	
\\	海外旅行が好きという点では、私と同じだね。 
\\	「という点では」は
\\	を使えばいいでしょう。	
\\	メキシコと言えば、やっぱタコスでしょ。 
\\	と言えば」は
\\	に触れる」を使えばいいでしょう。また、
\\	でも同じ様な意味になります。	
\\	明日お好み焼きを食べに行きます。食べ物といえば わたしは一人暮(ぐ)らしにもかかわらず料理をあまりしません。 
\\	と言えば」は
\\	を使って表現します。	
\\	その兄弟は、親子と言ってもおかしくないくらい年が離れていた。 
\\	ここでは「言ってもおかしくない」を
\\	だと思うってしまうほど」と言い換えて表現しています。	
\\	ビールを飲むとトイレが近くなっていけねえ。 
\\	「トイレが近い」は
\\	「トイレに行きたくなる」で表現すればいいでしょう。
\\	「小便をする」。	
\\	悪くないんだけどねぇ…、ちょっと統一性に欠けてない? 
\\	「統一性」は 
\\	、あるいは、
\\	でいいでしょう。
\\	は「欠ける」の意。	
\\	彼は体格に恵まれていたので、バスケット選手として頭角を現すのにさほど時間はかからなかった。 
\\	(すばらしい姿を現す、かっこよく見える)の意味。
\\	は「目立つ」、
\\	は「…をするのにいい体格をしている」、
\\	は「体格」の意。
\\	の使い方にも注意。	
\\	インフラにこれだけの額を投資したのだから、もう後戻りはできない。 
\\	「投資する」は 
\\	、あるいは、
\\	を動詞として用います。	
\\	一方の当事者は賛成するが,もう一方の当事者は、そうは言わない。 
\\	には「当事者、関係者」という意味があります。それに
\\	(関係している)、
\\	(関与している)を付け足しています。	
\\	どうせなら自分が投票した人に当選して欲しいね。 
\\	で「選ばれる」の意味になります。
\\	「候補者」。	
\\	今回の参議院選挙で、彼は当選確実だと言われている。 
\\	で「必ず勝つ」の意味。
\\	は「参議院」の意。	
\\	辛いことを経験すると、今まで当たり前だと思っていたことに感謝できるようになる。 
\\	で
\\	を当然のこととみなす」の意味になります。	
\\	この機能は当然付いているものと思っていました。 
\\	(標準の)で「当然付いている」を表現しています。
\\	は「当たり前のこと」の意。	
\\	携帯電話の人体への影響が取りざたされている最中、かれはその良さをとうとうと述べた。 
\\	は「しゃべりまくる」、
\\	は「長々と話す、熱弁をふるう」の意味。	
\\	うわ!お茶当番がまわってきた。面倒くさいな。 
\\	「当番」は
\\	を使えばいいでしょう。	
\\	今日は雨が降っているから、選挙の投票率は下がるだろうな。 
\\	で「投票者数」の意味になります。
\\	「選挙、投票」。	
\\	当分の間は、今の仕事を辞められそうにありません。辞めても、他の就職先がありませんから。 
\\	で「当分の間、さしあたり」の意味になります。	
\\	今回の選挙では、彼は当落線上にあるので、後援者も気が気でない。 
\\	が「当落線上にいる」に当たる言い回しです。
\\	「後援者」。	
\\	1学期では、8科目を登録する予定です。 
\\	「…を登録する」は
\\	といいます。	
\\	そんな遠い昔の話なんて覚えてないよ。 
\\	「遠い昔」は 
\\	で表現すればいいでしょう。	
\\	昔の話をしながら彼女は遠い目をした。 
\\	で「遠くを見る/見つめる」の意味になります。	
\\	やはり、家から遠くになるにつれて、ちゃんと化粧して出かける人が多くなるって結果が出ていた。 
\\	「遠くになるにつれて 
\\	ますます~する可能性が高くなる」を 
\\	で表現しています。	
\\	バックミラーを通して、後ろの車の運転手が運転しながら携帯で電話しているのが見えた。 
\\	「バックミラーを通して見る」は
\\	でいいしょう。「バックミラー」は
\\	ではなく、
\\	と言います。	
\\	ごめん!すぐお祝いの返事を書くつもりだったのだけど、何せ英語から遠のいてたので、英語でメール書くのに時間がかかったのよ。 
\\	この場合の 
\\	は「放っておく」の意味になります。	
\\	そんな遠回しな言い方しないで、ハッキリいってくれよ。俺のどこがいけないんだ。 
\\	が「遠回しに言う」に当たる表現です。
\\	「謎めいたことを言う」。	
\\	来年留学するので、そろそろ少しずつ準備しないといけないけど、10カ月も先だと思うと、まだまだいいやと思ってしまう。 
\\	「まだまだいいやと思う」を、
\\	(まだ時間がたくさんあると思い始めてしまう)で表現しています。	
\\	口内炎ができて痛いんだけど。触わったらだめだって思えば思うほど触わっちゃうんだよね。 
\\	この場合の「~と思うほど」は
\\	すればするほど、... である」を使えばうまく表現できます。
\\	「口内炎」。	
\\	彼、自分の企画が会議を通ったので、やる気満々だね。 
\\	(承認する、認める)を使って「通る」を表現しています。
\\	は「…であふれそうになる」の意。	
\\	都会に近くなればなるほど、ゴルフのプレイフィーは高い。でも、田舎のほうにいけば、安いところはいっぱいあるよ。 
\\	「都会」は 
\\	でいいでしょう。	
\\	忙しいとか何とか言って断っちゃえば? 
\\	「とか何とか」は 
\\	で表現できます。
\\	は「何やかや」の意。	
\\	彼のことを思うとときめくんです。 
\\	で「どきどきする」の意味になります。	
\\	今、時を止められたらいいのに。 
\\	で「時を止める」を表現しています。
\\	を使ってもいいでしょう。	
\\	得意料理っていうのはないな。料理苦手だし、いつも簡単な料理しか作らないから。 
\\	「得意料理」は
\\	でいいでしょう。
\\	「さっと作る」、
\\	「大した 
\\	ではない」。	
\\	背負い投げが彼の得意技だもんね。 
\\	「得意技」は
\\	でいいでしょう。
\\	「…が得意である」。	
\\	今回の亡命事件で、政府は事態を長引かせるのは得策でないと判断したようだ。 
\\	で「…するのは得策(最善の策)ではない」の意味になります。	
\\	特徴がある人だと名前は覚えやすいと思うけど。名前・ヘアスタイルが変わっているとかさ。 
\\	「特徴」は
\\	でいいでしょう。「ある」の部分は文の流れの中で表現しています。	
\\	特に時間は言ってないのですが 、今日お伺いすることは電話で話しているので高橋さんはご存じだと思います。 
\\	「明確(具体的に)言う」を使えば「特に」が表現できます。	
\\	教会に特に興味があるわけではないのだけど、この教会だけは行って実際見てみたい気がする。 
\\	「特に」は
\\	を使えばいいでしょう。ここでは
\\	の動詞、名詞としての使い方にも注目して下さい。	
\\	(何か質問はありますか?)今のところ、特になにもありません。 
\\	共に「今のところは(質問は)ありません」の意味。	
\\	今日は野菜の特売日だから、スーパーに行って来るわ。 
\\	が「特売日」に当たる表現です。	
\\	昨日は特別番組だったので、いつもの番組はやってなかったみたいだな。 
\\	「特別番組」は
\\	でいいでしょう。
\\	「代わる)	
\\	今回だけ特別に例外として認めていただけませんか? 
\\	で「特別な例外」の意味。
\\	も可。	
\\	感謝されるのが目的で寄付するわけではないので、匿名にするつもり。 
\\	は「匿名で、名前を明かさないで」の意味。
\\	は「獲得する」の意。	
\\	この保険は、特約を付けられるのが特徴だ。 
\\	が「特約」に当たる表現です。	
\\	道路に積もった雪がなかなか溶けない。 
\\	「溶ける」は
\\	、「積もる」は
\\	ですね。	
\\	親指に刺さったとげがなかなか取れないんだ。 
\\	「とげ」は
\\	ですね。
\\	は「取り除く」の意。	
\\	あいつの言うことには刺がある。 
\\	は「棘のある、辛辣な」、
\\	は「痛烈な」という意味の形容詞です。
\\	(毒舌である)	
\\	今夜はとことん付き合いますよ。 
\\	で「とことん最後まで」の意味になります。	
\\	システムを作る上で自分が納得が行かなかったら、とことん話しあいます。 
\\	で「…を徹底的に話し合う」の意味になります。
\\	は「考えられる」の意。	
\\	忙しいところ申し訳ないですがよろしくおねがいします。 
\\	が「お忙しいところ」に当たる言い回しです。	
\\	ところでさぁ、これって(英語で)何て言えばいいのかなぁ? 
\\	前後の文脈がありませんので、ここでは「ところで」をそのまま
\\	を使って表現しています。	
\\	年明けはいつから仕事ですか? 
\\	「年明け」は
\\	、あるいは、
\\	で表現できます。また、
\\	と言ってもいいでしょう。	
\\	来年は年男だね。もう36か、早いもんだ。 
\\	「年男」は
\\	「十二支の私の年」と説明すればいいでしょう。	
\\	一気に階段をかけあがったら、息切れしちゃったよ。俺も年かな。 
\\	「歳かな」は
\\	「歳を取っているに違いない」で表現します。
\\	「息切れ」。	
\\	こういう飲み会の雰囲気に弱いんだよね。年甲斐もなく思わず、一気飲みとかしてしまうんだよ。 
\\	「年甲斐もなく... 
\\	のニュアンスを
\\	する代わりに年相応の振る舞いをすべき」で出しています。
\\	「一気に飲む」。	
\\	ちょっと贅沢して海外で年越しをするのもなかなかいいものだ。 
\\	で「海外で新年を過ごす(迎える)」の意味になります。	
\\	最初に本屋に行くでしょ、それでもしそこからデパートに行くとしたら、24号線をまっすぐ行けばいいのかな? 
\\	この場合の 
\\	は「…だとすれば」という意味。
\\	を使っても同じような意味になります。	
\\	仮に明日僕が行かないとして、きみが代わりに行ったとしても、状況は何一つ変わらないんじゃないかな。 
\\	「~として(仮定)」は 
\\	で表現すればいいでしょう。	
\\	たとえ車にカバーをかけていたとしても、ひょうのダメージはうけていたと思う。かなりすごかったみたいだよ。 
\\	ここでは「としても」を
\\	と
\\	を強調するための
\\	を使って表しています。
\\	「ひょう」。	
\\	最近体力が落ちちゃってね。歳には勝てないね。 
\\	「歳には勝てない」をここでは
\\	「若くはなれない」と言い換えて表現しています。	
\\	ロケットが大気圏突入の際、ほんの少しの時間だけれども通信が途絶えるそうだ。 
\\	「途絶える」は 
\\	でいいでしょう。
\\	は「再び入る」の意。	
\\	旅行する際に、その土地についての勘を養うにはバスを利用するのがもってこいである。 
\\	「土地勘を養う」とは、
\\	(場所の感覚をつかむ)ということですね。	
\\	途中で投げ出さないで、最後までやってよ。 
\\	この場合の「途中」は
\\	、あるいは、
\\	で表現できます。	
\\	ここに来る途中、えらい目に合った。 
\\	「ここへ来る途中で」は
\\	で表現できます。	
\\	あの教授の授業に途中から出席しても、単位取るのは難しいよ。 
\\	は「学期の途中で」の意味。
\\	(途中で)を使ってもいいでしょう。	
\\	途中下車して、どこかで昼飯でも食べていこうか。 
\\	で「途中下車」の意味になります。	
\\	面倒で途中投げ出しそうになったけど、なんとか頑張ったよ。 
\\	が「途中で」に当たる言い回しです。この文脈では 
\\	も使えます。	
\\	私も同じ方向の電車に乗るので途中まで一緒に行きましょう。 
\\	で「途中まで」の意味になります。	
\\	この新人のアナウンサー、さっきからとっちてばかりいる。 
\\	「とちる」とは
\\	「つかえる」、
\\	「間違える、失敗する」ということですね。	
\\	あいつ遅いな。約束の時間とっくに過ぎているのに、何やっているんだ? 
\\	「とっくに」を
\\	「ずっと前に」、
\\	「かなり前に」で表現しています。	
\\	「君はとっつきにくいね」と社長に言われてしまいました。 
\\	に近づきにくい」、
\\	は「声を掛けにくい」の意。	
\\	ちょっと洗うから机の上グラスとってくれる? 
\\	この場合の
\\	は「片づける」の意味になります。	
\\	3年前にポルトガルで買ったとっておきのワインを今日はごちそうしよう 
\\	「とっておき」は、簡単に、
\\	を使ってもいいですし、この場合には、
\\	「3年間寝かせておいた」とも言えますね。	
\\	とっておきの写真をなくしてしまった。 
\\	「とっておき」を「しまっておいた」と言い換えて、
\\	を使って表現しています。	
\\	とってつけたようなお世辞をどうも有難う 
\\	は「うわべだけに聞こえる」の意味。
\\	「お世辞、おべっか」、
\\	「おべっかを言う」。	
\\	そんな仕事とっとと片づけちゃえよ。 
\\	「とっとと」は
\\	「急いで」、
\\	「素早く」ということですね。	
\\	前回は、
\\	400点突破用の問題やったから、次はもう一段階上の500点用の問題をやってみよっと。 
\\	ここでは「突破する」を
\\	(突き破る)を使って表現しています。	
\\	突拍子もないこと言うなよ。僕が経理の田中さんと付き合ってるわけがないだろ。 
\\	ここでは「突拍子もないことを言う」を
\\	「夢みたいなことを言う」と言い換えています。
\\	「空想する」。	
\\	すみません。突然の事で、まだ状況が把握できてないのですが。 
\\	「突然の事」はそのまま
\\	「突然起きた」と言えばいいでしょう。	
\\	忙しすぎて、とてもじゃないが、そこまで首がまわりません。 
\\	で「とてもじゃないが」のニュアンスを出しています。	
\\	ここの遊園地は障害者のための施設が整っている。従業員だけじゃなくて、普通のお客もとても障害者に対して協力的だ。 
\\	「整っている」を
\\	「用意されている」、
\\	「備えられている」を使って表現しています。	
\\	携帯のアンテナ伸ばした方が電波の届きがいいんじゃない? 
\\	ここでは「電波の届きがいい」を
\\	「よりよく受信ができる」と言い換えています。	
\\	目標の730点には届かなかったけど、前回より100点も上がったんだからすごいよ。でも、今回そんなに上がったから次回受けるのちょっと怖くない? 
\\	この場合の「届く」は
\\	を使えばいいでしょう。
\\	は「上回る」の意。	
\\	(劇場や飛行機の席の予約で)つれと隣り合わせの席がいいのですが。 
\\	「座らせる」と
\\	を使えばいいでしょう。
\\	でも同じ意味になります。	
\\	夕方5時になるとさっさと店じまいをするなんて殿様商売をしていたら、すぐ店がつぶれるぞ。 
\\	「殿様商売」のニュアンスを
\\	「ずぶの素人のように」で表現しています。	
\\	あのまじめそうな人が殺人事件を起こすとはね。 
\\	「とはね」のニュアンスを 
\\	で表現しています。
\\	の使い方にも注意。	
\\	プログラマーだからといって、コンピュータのこと全部知っているとは限らない。特に 私を例に取るとね。みんな勘違いしているよ。 
\\	ここでは「…とは限らない」を 
\\	(…を意味しない)、
\\	を用いて表現しています。	
\\	飛ばして次の話題へ進んだ。 
\\	「飛ばす」に当たる動詞は
\\	です。
\\	「前置きを飛ばす(省く)」。
\\	「リハーサルを飛ばす」などのように使います。また、
\\	でも同じような意味が表せます。
\\	「次の章に飛んだ」。	
\\	だれも知り合いもいない、言葉も通じない場所に飛び込むのは、かなり勇気のいることだと思う。 
\\	は「投げ込む、突っ込む」、
\\	は「放り出す」の意。	
\\	急に車の前に飛び出してくるなよ。危ないじゃないか。 
\\	は「突進する」という意味ですね。
\\	を使ってもいいでしょう。	
\\	シャワーカーテンしてるのに外に水が飛び散って床がびっしょりになっちゃった。 
\\	この場合の「外に飛び散る」は 
\\	で表現すればいいでしょう。	
\\	彼は幅跳びで8メートルを跳ぼうと頑張っている。 
\\	ここでは「跳ぶ」を 
\\	「(記録を)破る」と 
\\	で表現しています。	
\\	テレビの宣伝効果のお陰で、その新製品は飛ぶように売れた。 
\\	が「飛ぶように売れる」に当たる言い回しです。	
\\	ウチの近所にゃほんとにトホホな飲み屋しかないもんね。 
\\	「トホホ」とは「ぱっとしない、汚い、情けない」の意味と考えて、ここでは
\\	「薄汚い」、
\\	「みじめな」を使って表現しています。
\\	はそれぞれ「安酒場」「いかがわしい飲み屋」の意味です。	
\\	とぼけるんじゃないよ! 
\\	「とぼける」は「知らない振りをするな」で表現できます。	
\\	聞いてないなんて、とぼけるのもいい加減にして! 
\\	「どぼける」は、
\\	「知らない振りをする」と言えばいいでしょう。
\\	「馬鹿にする」。	
\\	突然転勤を言い渡されて、彼は戸惑いを隠せなかった。 
\\	の文字どおりの意味は「そのことで驚いたという事実を隠せなかった」です。
\\	は「ためらい」の意。	
\\	お客さんにうちのお店のほうがほかのお店より安いわと言われ、なんて返していいかわからずちょっと戸惑った。 
\\	この場合の「戸惑う」は 
\\	(言葉に詰まる)でいいでしょう。	
\\	両親が共稼ぎとか、塾があって、一人で夕飯を食べる子供が3人に1人いるなんて、かわいそう。 
\\	「共稼ぎ」は簡単に
\\	で表現できます。
\\	「3人に1人」。	
\\	電子メールを使うようになって、友達の輪が広がった。 
\\	で「より多くの人と知り合いになる」の意味。
\\	「友達の輪」。	
\\	来月日本の有名歌手がライブをダウンタウンで開くのですが、それに伴ってファンも日本ついてくると思うので観光客が増えそう。 
\\	この場合の「伴って」は特に訳出しなくてもそのニュアンスは英文の中に含まれます。	
\\	時代と共に言葉も進化していると思わない? 
\\	「共に」は
\\	と一緒に」ということですね。	
\\	ずいぶん大きなスーツケースだこと。でも売られてるということは、機内に預け入れできるサイズなんだよ。 
\\	「…ということは」を 
\\	(…の事実からすると)で表現しています。
\\	は「機内持ち込み荷物」の意味。	
\\	彼の意見を如何とらえるかは人によって違うと思う。 
\\	この場合の「とらえる」は
\\	で表現できます。
\\	と
\\	の使い方にも注目。	
\\	とりあえずこれでいこう。 
\\	「とりあえず」のニュアンスを
\\	で出しています。
\\	「さしあたり」。	
\\	とりあえずビール持ってきてくれるかな。 
\\	「とりあえず」をここでは
\\	で始める」、
\\	「最初、口火」で表現してみました。この他にも
\\	の文を使ってもいいでしょう。	
\\	これらのフレーズってよく日本人が間違うパターンで取り上げられている。 
\\	この場合の「取り上げる」は
\\	「指摘する」を使えばいいでしょう。	
\\	"中身は壊れ物なので""取扱注意""の札を貼っていただけますか? 
\\	小包などに
\\	(壊れ物、取り扱い注意)というシールが貼られているのをよく見かけます。
\\	以上、取り急ぎ用件のみにて失礼いたします 
\\	とりあえず、日本語の意味に近い英文を2つ挙げておきました。また、直訳すれば、
\\	あるいは 
\\	のようにも言えます。文の締めくくりとしては、
\\	日本での経験を活かし、日本でのやり方のいい部分をうまく取り入れる事によって、新しい視点でシステムの構築ができると思います。 
\\	「採用する、取り入れる」、
\\	「適合させる」の使い方がポイントです。	
\\	誠実さだけが彼の取り柄だね。 
\\	には
\\	に有利、役に立つ」という意味がありますので、これを使って「取り柄」を表現しています。
\\	「誠実、信頼度」、
\\	「忠誠」。	
\\	すみません、黄色のバックを取り置きしてもらっているんですけど。 
\\	「取り置き」は 
\\	(取っておく)で表現すればいいでしょう。
\\	は「留め置き方式」の意。	
\\	しまった。重複表現を投稿するなんて、取り返しのつかないことをしてしまった。 
\\	この場合の「取り返しがつかない」とは、
\\	「その投稿を取り下げることができない」ということ。
\\	とは「投稿を取り消す」の意。	
\\	将来は、環境への取り組みに積極的な企業に勤めたい。 
\\	(…に対して積極的な姿勢を持つ)の意味。
\\	で「取り組み」を表現しています。	
\\	自分の仕事を説明できないなんて、ちゃんと真剣に取り組んでなかったことがバレバレだ。 
\\	に真剣に取り組む」は
\\	で表現できます。	
\\	彼の演奏には人を虜にする魅力がある。 
\\	「魅力」、
\\	「引きつける」を組み合わせれば「虜にする魅力」が表現できますね。
\\	「カリスマ」、
\\	「うっとりさせる」の意。	
\\	雨が降ってきたから、洗濯物、取り込んでくれない? 
\\	で「洗濯物を取り込む」の意味になります。	
\\	最近、政界、官界、財界の間のどす黒い癒着が取り沙汰されている。 
\\	「取り沙汰される」とは「いろいろ噂が流れる」ということですから、ここでは 
\\	(いろいろな噂の主題)を用いて表現しています。
\\	は「後ろめたい」の意。	
\\	この間オーストラリアに行った時に、デジタルカメラでいろいろ撮ったけど、撮りっぱなしでまだパソコンに撮り込んでない。ま、いつもの事だけどさ。 
\\	「撮りっぱなし」をここでは
\\	「まだカメラの中に入ったままになっている」で表現しています。
\\	でも同じことが言えますね。	
\\	動物が好きなので、将来はトリマーになりたいです。 
\\	ここでは「トリマー」を 
\\	を使って表現しています。	
\\	彼本人は好人物なんだが、その取り巻きがどうもいけないね。 
\\	「取り巻き」は、
\\	「仲間」、あるいは、
\\	「彼が付き合っている人たち」で表現できます。	
\\	何があろうとも、取り乱しちゃ、いかん。 
\\	は「自制心」。この場合の
\\	は名詞で「冷静さ」のことです。	
\\	仕事のボリュームが思ったより多くて、作業が予定より2,3日遅れている。早く遅れを取り戻さなければ。 
\\	「取り戻す」は
\\	「埋め合わせる」で表現すればいいでしょう。	
\\	私は彼にスポーツ選手として誇りをもっていた頃の自分を取り戻して欲しいと願う。 
\\	「取り戻す」をここでは
\\	「に戻る」、
\\	「取り戻す」で表現しています。
\\	「運動選手」。	
\\	この本をとりよせていただきたいのですけど。 
\\	「取り寄せる」は
\\	「注文する」でいいでしょう。	
\\	昨日会社から帰ってきてからビデオ録画したから、途中からしか撮れてないと思う。 
\\	この場合の「撮れる」は簡単に
\\	を使って表現すればいいでしょう。	
\\	僕の好物はとろろかけユッケなんだ。 
\\	「ユッケ」を
\\	「韓国式のタルタルステーキ」で表し、その後に、
\\	「とろろを上にかけた」と説明を加えています。次回は、文の中にポイントとなる表現を入れて投稿してください。	
\\	そういうときって、人間性が問われるよね。 
\\	ここでは「問われる」を
\\	(試す)を使って表現しています。
\\	は「…を試す」の意。	
\\	今日はクラスがあると思って、仕事が終わって飛んで帰ってきたのにキャンセルなんてショックが大きい。 
\\	が「飛んで家に帰る」に当たる言い回しです。	
\\	ストライキって毎年やっているのですか?バスで通勤通学している人にとってはとんでもないですね。 
\\	は「けしからぬ、ひどい」の意味。
\\	はその名詞形。	
\\	そのプロジェクトはとんとん拍子に進み、思った以上の成果を挙げた。 
\\	で「何の支障(つまずき)もなくうまくいく」の意味になります。	
\\	結果的には、とんびに油揚をさらわれた。 
\\	ここでは「とんび」を「部長
\\	にして、「僕たちがやって来たことを自分の手柄にしてしまった
\\	と言い換えて英語で表現しています。	
\\	もう辞めたい度はピークだけど、あと一カ月だと思って頑張って仕事しますか。人生のうちでたった一カ月だもんね。どうにかなるでしょ。 
\\	ここでは「辞めたい度」を 
\\	と、
\\	(欲求、願望)を使って表現しています。
\\	は「ピークにある」の意。	
\\	ここ駐車禁止って思いっきり書かれているのに、よく平気で駐車してるな。どういう神経してんだ? 
\\	「何考えてるんだろうね」で「どういう神経」のニュアンスを出しています。	
\\	アメリカは旅行で行っただけなので、地元の人が住んでいる場所はどういう所かはわからないなぁ。 
\\	「どういう所」は 
\\	でいいでしょう。	
\\	本当はどう思っているの? 
\\	「どう思う」は
\\	です。
\\	を使うなら、
\\	になります。	
\\	君にそんな事を言うなんて、どうかしてたんだ。 
\\	ここでは「どうかしてた」を
\\	(頭がおかしい、正気を失っている)を使って表現しています。	
\\	あまりに几帳面なのもどうかと思うね。 
\\	「ちょっと疑いを持っている」で「どうかと思う」のニュアンスを出しています。
\\	「細かいことにうるさい人」。	
\\	階段を駆け上がったら、動悸が激しくなって、めまいがした。 
\\	は「激しく打つ」の意味。
\\	「頭がクラクラする」、
\\	「めまい」。	
\\	彼に道義的責任はないのだろうか 。 
\\	「道義的責任」を
\\	「道徳的に義務を負っている」、
\\	「道徳的な義務がある」と表現しています。	
\\	もう全然うまくならないから、しまいには道具のせいにしたりなんかして。でも、現にいいクラブだとよく飛ぶらしい。 
\\	ゴルフの道具は 
\\	と言います。
\\	は「…を~のせいにする」の意。	
\\	日本人同士で英会話をするのはなんだか恥ずかしい。 
\\	ここでは「日本人同士」を 
\\	で表現しています。	
\\	昨日、田中さんと食事をした?それがどうした。 
\\	文脈がはっきりしませんので、こちらで文を作ってみました。「それがどうした」は
\\	でいいでしょう。	
\\	うちの不動産屋に何度いっても水道直してくれないけど、もし何かあったらどうしてくれるの?って感じ。 
\\	「どうしてくれる」は
\\	でいいでしょう。また、ここでは「不動産屋」の代わりに
\\	「管理会社」を使っています。	
\\	私どうしても日本に行って彼に会いたいの。飛行機代貸してくれる? 
\\	で「何があっても」の意味。
\\	と大文字にしてあるのは、その部分を強調するためです。	
\\	花束って、もらったその時はうれしいけど、その後にどうしようか迷うんだよね。 
\\	で「…をどうすべきか」の意味になります。
\\	は「束」の意。	
\\	自分でもどうしようもできないことってあるでしょ? 
\\	と 
\\	の使い方がポイントです。	
\\	このままでは駄目だと思うのですが、考えれば考えるほど嫌になってしまって、どうしようもなくなるんです。 
\\	「どうしようもない」は
\\	(望みがない、見込みがない)で表現できます。
\\	の使い方にも注意。	
\\	有名な人ってよっぽど精神的に強くないとやっていけないよね。ちょっとやそっとのことじゃ動じない精神をもてないとね。 
\\	で「動じない、うろたえないでいる」の意味になります。
\\	も「動じない」の意。	
\\	年末の海外旅行って発売と同時に売り切れたんだって。 
\\	「同時に」は
\\	などで表現できます。
\\	「一杯になる」。	
\\	彼は心臓で話すタイプだから、英語が話せなくても全然動じない。そういう点は、見習うべきだと思う。 
\\	は「平然としている、動じない」、
\\	は「決してあわてない」の意味。
\\	は「心臓が強い」、
\\	は「思わず言ってしまう」の意。	
\\	ねぇどうする?車で行く?電車で行く?電車だと、11時にはここを出ないとだめだよ。 
\\	この場合の「そうする」は
\\	(どうしたい?)、
\\	(どういうことになる?)で表現すればいいでしょう。	
\\	そういえば、今週の土曜日どうする?英会話に行ってから佐藤さん家に行く?それとも英会話は日曜日にする? 
\\	「どうする」は
\\	または 
\\	で表現すればいいでしょう。	
\\	(彼は)どうせ遅れるに決まってるじゃん! 
\\	を使うと「どのようにしても、いずれにしても」の意味になります。	
\\	あなたと同棲したいです。 
\\	「同棲する」は 
\\	でいいでしょう。
\\	(引っ越す、移り住む)	
\\	どうせやるなら、思い切りよくね。 
\\	「どうせ(やる)なら」は
\\	で表現すればいいでしょう。
\\	「思い切りやる、目一杯やる」。	
\\	スパゲティーどうだった?味薄くなかった? 
\\	「どうだった」は
\\	を使って表現します。	
\\	その知らせを聞いて彼は動転してしまい、何も言えなかった。 
\\	には
\\	ひどく驚かす」という意味があります。
\\	「唖然とさせる」。	
\\	なんかすべてがどうでもよくなっちゃった。私の人生って何なんだろう。 
\\	(…を気にしない)で「どうでもいい」を表現しています。	
\\	彼らは、どうでもいいことにいちいち騒ぎすぎ。もっとさぁ心の広い人間になろうよ。 
\\	で「まったく重要でないこと」、
\\	で「こせこせする」の意味。	
\\	仕事中にどうどうとゲームするなんて、彼女もいい度胸しているよね。 
\\	ここでは「どうどうと」を
\\	「それをする完全な権利がある」を使って表現しています。	
\\	人の前ではもっと堂々としなくてはダメだよ。 
\\	は「威厳のある(形容詞)」、
\\	は「もっと威厳を持って」の意味。	
\\	議論が堂々巡りする。 
\\	「堂々巡り」とは
\\	「ぐるぐる回る」を使えばいいでしょう。	
\\	例のハワイに行く件どうなった? 
\\	「どうなった」は 
\\	で表現すればいいでしょう。	
\\	彼が今度の担当者になるんだって?どうなることやら。 
\\	で「これ/この企画がどういう結果になるのか」の意味になります。
\\	「責任を持つ」。	
\\	この人どうにかならないの?死んでも直らないよ。 
\\	「どうにかならないの」は
\\	「何かできることはないのか」と言い換えることができます。
\\	「最後の最後まで」。	
\\	新制度の導入には国民の合意が必要であろう。 
\\	には「紹介する」のほかに「導入する」という意味もあります。
\\	は「必要条件」の意。	
\\	どっからどうみても、だれが見てもこの建物怪しいって。 
\\	で「どのような見方をしても」の意味になります。
\\	は「いかがわしい」の意。	
\\	どんな博打でも胴元が儲かる仕組みになっている。 
\\	ここでは 
\\	(賭博場)を使って「胴元」を表現しています。	
\\	待ち合わせ場所で1時間、待ってみたが、どうやら彼女は来ないらしい。 
\\	ここでは「どうやら」のニュアンスを
\\	(かなりハッキリしている)を使って出しています。
\\	は「現れる」の意。	
\\	(体温計を見て)わあっ、熱がある。どうりで頭がフラフラするわけだ。 
\\	「どうりで」は
\\	(不思議ではない)を使って表現できます。
\\	は「頭がフラフラする」の意味。	
\\	同僚が先に出世したからといって、ひがむのはよせよ。 
\\	「同僚」は 
\\	でいいでしょう。
\\	という言い方もあります。
\\	は「昇進する」の意。	
\\	料理の本どおりにちゃんと作ったはずなのに、失敗しちゃった。何がいけなかったのだろう。 
\\	で「料理の本に書いてあったとおりに」の意味になります。
\\	は「文字どおりに」の意。	
\\	ほんま、どぎつい色やなあ、そのセーター。 
\\	「どぎつい(色)」は
\\	で表現できます。また、
\\	「けばけばしい、派手な」を使ってもいいでしょう。	
\\	心臓が、どきどきして破裂しそう! 
\\	「どきどき」は
\\	「心臓がどきどきする」、
\\	「激しく打つ」を使えばいいでしょう。	
\\	1人で海外旅行に行くのはドキドキする。 
\\	で「ドキドキする」の意味になります。
\\	は「ドキドキする」の意。	
\\	あなたは、努力家ですね。 
\\	「努力家」は
\\	「一生懸命働く
\\	ということですね
\\	私いつもはコンタクトレンズなんだけど、実はすごいド近眼なの。 
\\	(近眼の)に 
\\	(超)、
\\	(すごく)を付け足して「ド」を表現しています。	
\\	彼は読書家だから、とても物知りだ。 
\\	で「とても多読である(博識である)」の意味。	
\\	あいつの読書量は半端じゃない。 
\\	ここでは「読書量が多い」を 
\\	で表現しています。
\\	は「気まぐれな、その時々の」の意。	
\\	彼は相変わらずの毒舌だ。 
\\	で「どぎついことばを使う」の意味。
\\	は「毒舌である」の意。	
\\	日本語にも地方によって独特のなまりがあってその地方の人じゃないとわからない時もある。 
\\	とは「他とは違う」という意味です。
\\	は「…固有の、独特の」の意。	
\\	都会と違ってこういう田舎っぽい所も、独特な雰囲気を醸し出してて、こういう所もそれなりにいいと思う。 
\\	「独特な」は 
\\	で表現すればいいでしょう。
\\	は「素朴な、田舎風の」、
\\	は「都会の」の意。	
\\	彼の意見は毒にも薬にもならない。 
\\	「あいまい」、
\\	「中立の、はっきりしない」、
\\	「安全な」、
\\	「面白みのない」、
\\	「役に立たない」などを組み合わせて「毒にも薬にもならない」を表現しています。	
\\	彼、独立して、レストランを始めたんだ。 
\\	(自分で)を使って「独立して」を表現しています。
\\	で「独立する、自立する」の意味。	
\\	謝る時は土下座だろ。 
\\	「ひれ伏す」を使えば「土下座」が表現できます。
\\	「悔恨」の意。	
\\	お!久しぶり。どこか旅行へでも行っていたの? 
\\	と
\\	か何か」の使い方がポイントです。	
\\	食事は済んだ?何処かに寄って何か食べていこうか? 
\\	「何処かによる」はそのまま
\\	を使えばいいでしょう。	
\\	どこから見ても、誰が見ても、間違いなくこれは幽霊屋敷だよ。 
\\	と
\\	の組み合わせに注目。「この家を見て、幽霊が出ないと言う人は誰もいない」=「幽霊屋敷だ」という意味。	
\\	私の、一体どこが好きになったのですか? 
\\	「どこ」を
\\	を使って表現している点に注意してください。	
\\	いろんな英語学校があるけど、私の経験から言わせてもらうと、どこへ行っても結局はその人のやる気次第だと思う。 
\\	「どこへ行っても」は 
\\	で表現します。
\\	は「やる気」の意。	
\\	前回のメールでどこまで話したか忘れたけど、働き始めたって言ったっけ? 
\\	ここでは「どこまで」を 
\\	を使って表現しています。	
\\	どこまでの切符かったの? 
\\	と
\\	の使い方がポイントですね。	
\\	どさくさにまぎれて帰ってきてしまった。 
\\	「どさくさにまぎれて」を
\\	「混乱の中を」を使って表現しています。
\\	「どうにか抜け出す)	
\\	こんな土砂降りの中、急いでどこに行くんだい? 
\\	「土砂降り」は 
\\	、あるいは、
\\	と言います。	
\\	あれ、テレフォンカードの度数がもう10を切ってる。 
\\	「度数」は
\\	でいいでしょう。動詞には
\\	を使います。	
\\	彼女は痴漢の手をつかみ、どすを効かせた声で睨みつけて、「10年早いんだよ」ってそいつに言った。 
\\	ここでは「どすをきかせた」を 
\\	(太くて低い)、
\\	(威嚇的な)を組み合わせて表現しています。「10年早い」の表現の仕方にも注目。(注)和文を一部カットしました。	
\\	あいつのやり方はまるで他人の家に土足であがりこむようだ。 
\\	で「何でもないように他の人の気持ちを踏みつける」の意味になります。
\\	「心づかい、配慮」。	
\\	最終的には、ドタキャンが3人いた。 
\\	「ドタキャン」に当たる名詞表現は
\\	です。動詞
\\	を使って、
\\	で「最後になってキャンセルする」の意味になります。
\\	「最後になって断りを入れる」も一緒に覚えておくといいでしょう。	
\\	もう!あいつはしょっちゅうドタキャンする! 
\\	「ドタキャン」は 
\\	で表現します。
\\	は「約束を破る」の意。	
\\	彼は土壇場で力を発揮するタイプだ。 
\\	「土壇場」は
\\	で表現できます。
\\	は「奮い起こす、呼び出す」、
\\	は「最後まで頑張る(形容詞)」の意。	
\\	どちらかと言えば、ロングヘアの子がいいな。でも、まあ似合っていればどっちでもいいけど。 
\\	「どちらかと言えば」をここでは
\\	「選択する余地があるなら」で表現しています。	
\\	1年を通して、ホテル住まいと家とどっちが多いのですが? 
\\	と
\\	の組み合わせに注目。	
\\	ノリカとレイコどっちが好きかって?どちらでもないな。 
\\	「どちらでもない」は
\\	でいいでしょう。
\\	「好きな人」。	
\\	僕は英語の読解力がないから、もう少し英語の本を読むようにしないといけないな。 
\\	「読解力」は 
\\	でいいでしょう。
\\	は「ほとんどゼロ」の意。	
\\	(イタリアンとフレンチどっちがいい?)どっちでもいよ。 
\\	「どちらでも」は
\\	を使えばいいのですが、ここでは
\\	という言い回しも覚えるようにしましょう。	
\\	どっちもどちだわ。 
\\	「どっちもどっち」は、
\\	を使って「どっちも正しい」あるいは「どっちも間違っている」と言い換えればいいでしょう。	
\\	調べ始めたらどつぼにはまってしまった。 
\\	「どつぼにはまる」とは
\\	「動きがとれなくなる」の意味ですね。
\\	「難航する、動きがとれなくなる」。	
\\	また隣りのおやじが怒鳴りちらしてるよ。毎日何をそんなに怒ることがあるんだろう。 
\\	「怒鳴り散らす」、
\\	「わめく」、
\\	「大声で叫ぶ」、
\\	「叫び声を上げる」などを組み合わせて表現すればいいでしょう。
\\	「腹を立てる」。	
\\	あんなことしたあとで、どの面下げて会いに来たんだ? 
\\	(よくもまあ…できたものだ)で「どの面下げて」のニュアンスを表しています。	
\\	主人は外出しておりますが、どのようなご用件でしょうか? 
\\	が相手に用件を聞くときの言い回しです。	
\\	緊張するとどもっちゃうんだ。 
\\	「どもる」は
\\	と言います。	
\\	あのドラマに出てくるようなケーキ屋が近所にあればなあ。毎日通っちゃうね。 
\\	で「あのドラマに出てくるような…」の意味になります。	
\\	これからテレビ局でドラマの収録をしてから、地方へロケに出かけます。 
\\	「収録する」は
\\	を動詞として使えばいいでしょう。
\\	「撮影する場所」。	
\\	よく頑張ったきみには、努力賞をあげましょう。 
\\	「努力賞」を
\\	(特に努力をしたことに対する賞)で表現しています。	
\\	ドル決済のカードが欲しい。 
\\	で「ドルで請求される」という意味になります。
\\	は「ドル建ての」の意。	
\\	潜水艦の中で死んじゃった人たち、どれだけ苦しくて怖かったことか…。考えたことある? 
\\	「どれだけ」は 
\\	を使って表現します。
\\	は「恐ろしい」、
\\	は「苦しむ、惨めな」の意。	
\\	どれにするか、私が選べるのですか? 
\\	は「選択をする」、
\\	は「どちらかを選ぶ」の意味になります。	
\\	東京なんかより大阪のほうがどれほどいいかわからない。 
\\	「どれほどいいか」は 
\\	で表現します。
\\	は「わからない」の意。	
\\	どろどろした人間関係がいやで、クラブを退会したんですよ。 
\\	は「ごちゃごちゃした」の意味。
\\	に巻き込まれる」、
\\	「複雑さ」。	
\\	物事、理論通りには運ばないよ。現実は、もっとどろどろしているんだ。 
\\	ここでは
\\	「乱雑な、ごちゃごちゃしている」、
\\	「混乱している、ごちゃまぜの」を使って「どろどろしている」を表現しています。	
\\	だれの心の奥底にも、どろどろとしたものが潜んでいる。 
\\	ここでは「どろどろした」を
\\	(醜い)、
\\	(汚れて真っ黒な)を使って表現しています。
\\	は「眠っている(潜んでいる)」」の意。	
\\	あの歌手の名前なんて言ったっけ。度忘れして思い出せないよ。 
\\	は「うっかり忘れる、度忘れする」の意味。	
\\	景気は今がどん底。 
\\	で「落ち込んでいる」、
\\	は「どん底にある」の意。	
\\	このドラマの最終回でどんでん返しがあるかもよ? 
\\	が「どんでん返し、驚くような結末」に当たる言い回しです。
\\	という言い方もあります。	
\\	あなたも彼女を見習って自分の意見をどんどん言うべき。 
\\	で「もっと強く、強力に」の意味になります。
\\	「自分の意見を持つ」、
\\	「ハッキリ言う」。	
\\	どんなに小さなほこりだって、時間がたてば積もって山になるんだよ。 
\\	この場合の「どんなに」は 
\\	で表現すればいいでしょう。
\\	は「ちり、小片」、
\\	は「蓄積する」の意。	
\\	コンピューターグラフィックスが、まずはどんなものか試しに体験レッスンを受けてみようかな。 
\\	で「コンピューターグラフィクスがどんなものか学ぶ(知る)」の意味になります。
\\	は「実地の」の意。	
\\	その事件はなあなあで処理されて、表に出なかった。 
\\	「なあなあ」をここでは
\\	「お互いに困ることがないような方法で」と説明的に表現してみました。
\\	「避けるようにしてつま先で歩く」。	
\\	友だちをないがしろにするやつは、嫌いだ。 
\\	は「…を軽く考える、ないがしろにする」の意味。
\\	は「…を真面目に受け止める」の意。	
\\	ひじに鞄をかけただけで、内出血してる。それだけ重たかったってこと?私ってなんてか弱いのかしら。 
\\	で「内部で出血する」の意味になります。
\\	でも同じことがいえます。	
\\	みんなで何内緒話してるのよ。私も仲間に入れて。 
\\	「内緒話」は
\\	を使って表現すればいいでしょう。	
\\	彼女は無いものねだりばかりしている。 
\\	で「不可能な物を求める」の意味になります。	
\\	内容の濃いとても充実したクラスでした。これぞ探し求めていたクラスだと思いました。 
\\	で「素晴らしい内容が詰まっている」の意味になります。	
\\	私はお金がない訳じゃない。ただこんなことには使いたくない。 
\\	と 
\\	を組み合わせて「ない訳ではない」を表現しています。
\\	は「無一文で」の意。	
\\	かさぶたを取ったらだめだよ。治りが遅くなるから。 
\\	で「治るのに時間がかかる」の意味になります。
\\	は「かさぶた」のこと。	
\\	この程度の捻挫なら、1週間程で治るでしょう。 
\\	「治る」は自動詞
\\	を使って表現します。	
\\	彼らは、めちゃくちゃ仲がいいね。 
\\	で「驚くほど親密な仲だ」の意味になります。
\\	「信じられないほど」。	
\\	宴会の途中で、そろそろ宴会を打ち切りたいときに使う日本の伝統的習慣。 
\\	もちろん英語にな「中締め」に当たる言い回しはありません。ここでは「中締め」が使われる状況を考えて、英語で再現してみました。	
\\	君とのことは無かったことにしてくれないか。 
\\	で「二人の間には何もなかったように振る舞う」の意味。
\\	「カップル」。	
\\	僕は、芸能ニュースを電車の中吊り広告から仕入れてるんだ。 
\\	ここでは「中吊り広告」を
\\	「電車の天井からぶら下がっている大衆向け雑誌の広告」と説明しています。	
\\	どうやったら仲直りできるかなあ…? 
\\	「…と仲直りする」は 
\\	で表現します。
\\	は「よりを戻す」の意。	
\\	なかなかいい出来じゃない? 
\\	「なかなか」は
\\	でいいでしょう。
\\	(悪くない)も使えますね。	
\\	風邪がなかなか治らない。もう咳が出過ぎてお腹が痛い。 
\\	で「なかなか…できるようには見えない」の意味を表しています。
\\	の使い方に注目。	
\\	あいつなかなかやるじゃないか。 
\\	「悪くない」が「なかなかやる」に近い言い方です。また、
\\	「思ったよりやる」、
\\	「仕事ができる」でも同じような意味が表現できます。	
\\	コマースがこのまま進展すれば、中間業者などが不要な中抜き状態が起こり、商品価格がますます下落するだろう。 
\\	は「空洞にする、中身をなくす」の意味です。
\\	「現在の調子で」。	
\\	ねねね、ちょっと私も話の仲間に入れてよ! 
\\	で「中に(仲間に)入れる」の意味になります。	
\\	この肉分厚いから、中まで火が通るかな。 
\\	で「中まで調理される(火が通る)」の意味になります。	
\\	あの子だけ赤い洋服着てない。仲間はずれだ。 
\\	この場合の 
\\	は「のけ者」の意味。
\\	は「のけ者、はみだし者」の意。	
\\	もう少し中身のある話をしろよ。 
\\	「中身」は
\\	「内容」で表現できます。	
\\	これからも仲良くしてやってね。 
\\	「仲良くする」は
\\	を使えばいいでしょう。	
\\	梅雨の長雨って、ないね、今年。 
\\	は「やまずに降り続く雨」の意。「梅雨」は
\\	と言います。
\\	の前には
\\	が省略されています。	
\\	随分長居をしてしまいました。このへんで失礼します。 
\\	で「思ったより長居する」の意味になります。
\\	は「長居して嫌がられる」の意。	
\\	この娘、トロいけど長い目で見てやってよ。きっとモノになるからさあ。 
\\	「長い目で見る」を、ここでは、
\\	「時間を与える」、
\\	「長い期間で見る」で表現しています。
\\	「モノになる」も知っておきたい言い回しですね。	
\\	今この出費は痛いけど長い目で見れば断然お得だよ。 
\\	で「長い目で見れば、長期的に見れば」の意味。
\\	も使えます。	
\\	なぜ休みの日の1日は早く感じて、働いているときの1日って長く感じるのかな。 
\\	で「とても長く思える」の意味。
\\	は「だらだら長引く」の意。	
\\	彼はゴール前で軽く流した。 
\\	は「ゆるめる」という意味ですね。「ゴールライン」は
\\	と言います。	
\\	システムをテストする時は本番データを使って、最初から最後まで流してテストします。 
\\	この場合の流すは
\\	を使えばいいでしょう。	
\\	今年の目標は、何か長続きできる趣味を持つことです。 
\\	「長続きする」とは
\\	を続ける、固執する」ということですね。	
\\	この部屋なかなか眺めがいいじゃない。 
\\	で「眺めがいい」という意味になります。
\\	でも同じ様なことが言えます。	
\\	きみがくれたあの花、意外に長持ちしていてビックリだよ。 
\\	と 
\\	を使って「長持ちする」を表現しています。	
\\	ねね、ここに座りながらショー見ない? 
\\	しながら」は
\\	で表現すればいいでしょう。この
\\	は省略可能です。	
\\	人が生まれ、人が死に、たとえ風景がかわっても川は流れ続ける。 
\\	難しく考えることはありません。「流れ続ける」は
\\	ですね。	
\\	先月あなたが連れて行ってくれたバーで流れていた曲あったじゃない? あのあと、レンタル屋に借りにいって、ハマッってしまったんだよ。 
\\	「流れている」は
\\	(自動詞)を使って表現します。他動詞として使う場合は、
\\	となります。「はまる」は 
\\	で表現できます。	
\\	あまり音楽は聞かないな。特にお気に入りのはないし。最新音楽がテレビで流れているのを聞いたりはするけど。 
\\	ここでは「テレビで流れている」を
\\	「テレビでやっている音楽(は何でも)」、
\\	で「テレビで演奏しているもの」を使って表現しています。	
\\	昨日の夜は、隣の犬の鳴き声がうるさくてよく眠れなかったんだ。 
\\	(声を上げて鳴く)、
\\	(うるさく鳴く、騒音を立てる)で「鳴き声」を表現しています。	
\\	(お前の)泣き言はもうたくさんだ。 
\\	は「泣き言を言う、愚痴を言う」、
\\	は「同情を引くような話(弁解)」の意味。	
\\	あいつは泣き上戸だから、一緒に飲みたくないんだよ。 
\\	「泣き上戸(の)」は
\\	(形容詞)、あるいは、
\\	と言います。	
\\	励ましのメッセージ有り難う。まさか来るなんて思ってなかったから、嬉しくて泣きそうだったよ。 
\\	が「泣きそう」に当たる言い回しです。
\\	「励ましの」。	
\\	その可能性はなきにしもあらずだな。 
\\	「なきにしもあらず」とは
\\	(不可能ではない)ということですね。	
\\	それってセクハラじゃない。泣き寝入りなんてしちゃだめよ。 
\\	「泣き寝入り」は
\\	「秘密にしておく、自分の心の内にしまっておく」を使って表現します。「セクハラ」は
\\	ですね。	
\\	お前は相変わらず泣き虫だな。 
\\	が「泣き虫」に当たる表現です。	
\\	この傾斜板と床とがなす角度は何度ですか 
\\	「角度は何度ですか」は
\\	といいます。
\\	(作る、形成する)の使い方にも注目。	
\\	彼は愛車を泣く泣く廃車にした 
\\	「泣く泣く」をここでは2通りに訳しています。
\\	する他なかった」、	
\\	慰めなんかいらないわ。今回の件は私の責任なんだから。 
\\	「慰める」は
\\	で表現します。	
\\	彼からは殴り書きの手紙が一通届いただけで、その後音沙汰が無い。 
\\	「走り書きする、ぞんざいに書く」を使って、
\\	で「殴り書き」の意味になります。	
\\	なけなしのお金をはたいて、彼女に指輪をプレゼントした。 
\\	の文字どおりの意味は「やっとのことでかき集めた少しのお金で」になります。	
\\	その戦争映画で、主人公の女の子は親孝行だった。泣ける映画だった。 
\\	「泣ける(お涙頂戴の)映画」には
\\	という決まった言い方があります。
\\	に尽くす、献身的な」。	
\\	最近向かいの本屋さんの店員さんと顔見知りになったのですが、土曜日の売上はさっぱりだと嘆いていました。 
\\	「嘆く」は 
\\	(不満をこぼす)で表現できます。
\\	は「嘆く」、
\\	は「ちょっとした知人」の意。	
\\	みんな文句ばっかりいうから、もう投げ出したくなってきた。 
\\	には「投げる、ほうる」という意味があります。
\\	も使えます。	
\\	最近彼の態度には投げやりなところが見られるね。 
\\	は「実にぞんざいな態度をとる」の意味。
\\	「気まぐれな」。	
\\	うちのオーナーは、ほんといい人なので、情けで私を雇ってくれたのだと思います。 
\\	ここでは「情けで」を 
\\	(親切さから)、
\\	(同情から)で表現しています。	
\\	試験前に風邪を引くなんて、我ながら情けない。 
\\	この場合の
\\	とは「こんなはずじゃなかった、こんなの必要じゃなかった」と逆の意味を表しています。 
\\	自分がこんなにも無能で何にもできないと思うと、自分で自分が情けないし、悔しい。 
\\	ここでは「情けない」を 
\\	(惨め)、
\\	(落胆する)を使って表現しています。
\\	は「無能な、役に立たない」の意。	
\\	情け容赦ないよねあいつは。 
\\	で「同情・哀れみなど一切ない」の意味になります。
\\	「冷淡な」。	
\\	名指しで非難されたからには、受けて立つしかないでしょう。 
\\	で「批判をするために 
\\	を選ぶ(名指す)」の意味になります。
\\	「名指しで 
\\	を非難する」。
\\	「対立」。	
\\	その規則は、はじめのうちこそ守られていたが、なし崩し的に誰も守らなくなった。 
\\	「なし崩し的に」は
\\	(少しずつ)で表現すればいいでしょう。	
\\	海外で日本で馴染みのファーストフードを見つけると、やっぱり安心感があるせいか、ついつい入っちゃうね。 
\\	「馴染み」は
\\	「よく知っていること」で表現できますね。「安心感」を
\\	で表している点にも注目。	
\\	明日から新しいクラスだけど、馴染めるか心配。 
\\	「なじむ」は
\\	「慣れる」で表現できます。
\\	のコツを掴む」。	
\\	あいつはなんでも人に責任をなすりつけようとするから、皆から信頼されないんだ。 
\\	と
\\	「非難」を組み合わせれば表現できます。
\\	「(責任などを)転嫁する」。	
\\	あいつとうとう国家試験パスしたんだよ。成せば成るよな。 
\\	「成せば成る」のニュアンスを 
\\	で表現しています。	
\\	今の子供達はなぞなぞ遊びなんてしないんだろうな。 
\\	「なぞなぞ」は
\\	と言います。
\\	「なぞなぞ遊びをする」。	
\\	彼さ、最近たばこのことグチグチ言わなくなっちゃったんだけど、もう言っても無駄だと思って諦めちゃったのかな? 
\\	(もう…しない)で「言わなくなっちゃった」のニュアンスを出しています。
\\	は「…だと思う、…と判断する」の意。	
\\	今日電話くれることになっているのにまだ来ないな。 
\\	で
\\	することになっている」の意味。	
\\	咳をし過ぎて、お腹が痛くなってきた。 
\\	「…になってきた」とは
\\	(…し始める)ということですね。	
\\	年を取るに連れて、ずうずうしくなってきた気がする。 
\\	で「より図々しくなる」の意味。
\\	は「厚かましさ」の意。	
\\	スポーツの世界って、負けたら監督が辞任するとか選手が引退するけどそれってなんか納得行かないな。 
\\	で「そのことがとても満足のいくこととは思えない」の意味。	
\\	彼女が私より試験の成績がよかったのは納得いかない! 
\\	「納得いかない」は
\\	「何故か分からない」と表現すればいいでしょう。
\\	「確信している」。	
\\	そんなに私の事を疑うのでしたら、納得のいくまで調べてもらって結構ですよ。 
\\	「納得のいくまで」とは
\\	「確信するまで」ということですね。	
\\	なぜ、われわれの企画が採用されなかったのか、納得がいく説明をしてほしい。 
\\	は「満足のいく、得心のいく」という意味です。	
\\	もっと納得できるようないいわけをしてほしいよ。 
\\	は「説得力のある、納得のいく」という意味です。	
\\	この音楽なつかしいね。 
\\	「なつかしい」は
\\	で表現できます。	
\\	懐かしい風景だ。全然変わってないね。 
\\	「懐かしい」は
\\	を使えばいいでしょう。	
\\	これって、なんだか昔懐かしの味だ。 
\\	「昔懐かしの味」を
\\	と
\\	を使って表現しています。
\\	は「懐かしい」の意。	
\\	クーラーつけっぱなしで寝たから、夏風邪引いちゃったよ。 
\\	「夏風邪」は
\\	と言います。
\\	の使い方にも注目!	
\\	犬達は、人間に懐いていなかったので、少し驚きました。 (犬は猫より人懐っこいイメージがある) 
\\	で「懐いていない」の意味になります。
\\	「愛着を持っていない」。	
\\	夏バテで食欲はないし、やる気はない、もうバテバテさ。 
\\	が「夏バテする」に当たる言い回しです。
\\	は「けだるさ」の意味。
\\	「憂鬱、沈滞」。	
\\	薬局は、ホテルを出て斜め向いにあります。 
\\	「斜めに」と
\\	を組み合わせれば「斜め向かい」が表現できます。
\\	「斜め向かいの」。	
\\	何を思ったのか老人はぼくに身体を寄せてきました。 
\\	ここでは「何」を 
\\	と 
\\	で表現しています。
\\	は「にじり寄る」の意。	
\\	また何かあったら連絡するね。 
\\	が「何かあったら」に当たる言い回しです。
\\	「知らせる」。	
\\	往復でたった3万円なんて、何かあるに違いない。 
\\	は「うさんくさい、怪しい」の意味。
\\	「いかがわしい、いんちきな」。	
\\	今はなにかと大変だけど、これを乗り越えたら、きっとうまくいくでしょう。 
\\	「なにかと」は
\\	「色々なこと」を使って表現すればいいでしょう。	
\\	皆焦ってしまっていて、何がなんだかわからない状態でした。 
\\	で「何が起こっているのか訳が分からない」という意味。
\\	「あわてる、混乱する」。	
\\	何が何でも合格するんだという気持ちがなければ、志望校に合格しないよ。 
\\	で「何があっても」の意味になります。
\\	は「どんなことが起きようとも」の意。	
\\	あいつは何食わぬ顔をしてきついことを言う。 
\\	で「まるで関心がない顔をして」の意味になります。	
\\	なにげに話を聞いていたのね。 
\\	「なにげに」をここでは
\\	「何も考えずに」、
\\	「消極的に」を使って表現しています。	
\\	僕のような静かな男をつかまえて、うるさいとは何事ですか。 
\\	「何事か」のニュアンスを 
\\	(どうして…と言えるのか)で表しています。
\\	は「ばかげた、不合理な」の意。	
\\	何事も続けることが大切だ! 
\\	で「たとえそれが何であろうとも」の意味になります。
\\	「頑張る、やり通す」。	
\\	せっかく遠路はるばる来たのに朝から晩まで食っちゃ寝していったい何しに来たんだ? 
\\	が「何しに来たんだ」に当たる言い回しです。
\\	の前の 
\\	が略されています。	
\\	これは何順に並べればいいのですか? 
\\	「何順、どのような順番」は、簡単に
\\	で表現できます。	
\\	あなたは何人ですか? 
\\	「何人」は 
\\	(国籍)、
\\	(民族、人種)を使って表現すればいいでしょう。	
\\	何はともあれ、約束だけは守って下さい。 
\\	は「何が起きても」の意味。
\\	とは、つまり、「最低限すべきこと」。	
\\	よくも大学にまで行かせてくれてここまで何不自由なく育ててくれたと思うと感謝。 
\\	で「不自由することなしに」の意味になります。
\\	「感謝」。	
\\	何もそんな言い方しなくたっていいのに…すごく傷ついた。 
\\	この場合の「何も」は 
\\	(…する理由はない)を用いて表現すればいいでしょう。	
\\	あの競技は何を基準に採点しているのだろう。 
\\	は
\\	「(判断の)基準」の複数形。	
\\	彼は名ばかりで実力がない。 
\\	「名ばかり」をここでは
\\	「有名な名前を利用(悪用)している」で表現しています。	
\\	今度入ってきた山田、生意気なんだぜ。 
\\	「生意気」は
\\	「横柄な」、
\\	「厚かましい」で表現します。	
\\	シャツが生乾きで気持ち悪い。 
\\	「生乾き」は
\\	「湿っている」で表現すればいいでしょう。
\\	「乾く」、
\\	「べとべとする」、
\\	「湿っている」。	
\\	イカを料理したから、手が生臭い。 
\\	「生(魚)臭い、」は
\\	で表現します。「イカ」は
\\	と言います。	
\\	政治家と政商との間で、利権がらみの生臭い話があったらしい。 
\\	「生臭い」は
\\	、あるいは、
\\	で表現すればいいでしょう。
\\	は「影響力のある商人」の意。	
\\	語学力ってちょっと怠けるとすぐに衰えるからね。 
\\	「怠ける」を
\\	「勉強するのを延ばす(延期する)」で表現しています。
\\	「無視する
\\	寧々さん。それじゃ生殺しだよお。 
\\	は「どっちつかずの状態にする」の意味。
\\	「中間の状態」。	
\\	優勝決定戦を生でみたい。 
\\	「生で」は
\\	「ライブで」の他に、
\\	「自分で」を使っても表現できます。	
\\	現金でこの場で支払うって言うのはなんか生々しいね。 
\\	この場合の「生々しい」は「洗練されていない、がさつな」という意味の
\\	を使えばいいでしょう。	
\\	英語力を向上させるには、英語の学校に行くよりアルバイトでもして働きながら生の英語にふれているほうがいいと思いますか? 
\\	で「実際に使われている英語に触れる」の意味になります。
\\	は「向上させる、上達させる」の意。	
\\	消費者の生の声が聞きたいね。 
\\	(消費者から直に聞く)で「生の声」を表現しています。	
\\	これって再放送?それともビデオ?生放送? 
\\	「生放送」は 
\\	(副詞)で表現します。	
\\	彼女フランス人だから、彼女の英語はフレンチなまりなのかな。 
\\	「なまり〔訛)」は
\\	(名詞、あるいは、動詞)を使って表現します。	
\\	オーストラリア出身にしては、全然英語がなまってないね。 
\\	「なまる」は
\\	「なまりなしで話す」で表現できます。	
\\	週末、ゴルフの打ちっぱなしにいったのですが、体がなまっていたので、ちょうどいい運動になった。 
\\	(体が鈍って、体調を崩して)を使って「なまる」を表現しています。	
\\	しばらくピアノを弾かなかったら、指がなまって動かなくなってしまった。 
\\	この場合の 
\\	は「(能力などが)鈍くなる、下手になる」という意味。
\\	は「機敏な」の意。	
\\	凶悪犯罪の増加を見ていると、日本もだんだんアメリカ並みになってきた。 
\\	「並み」は
\\	を使って表現します。
\\	は「…と並んで」の意。	
\\	わが社にも情報化、国際化の波が押し寄せている。 
\\	ここでは「波が押し寄せる」を
\\	と
\\	を組み合わせて表現しています。	
\\	今日は、晴れてるけど、風があるから(海の)波が高いね。 
\\	で「波を高くする(かき立てる)」の意味。	
\\	簡単に言うけど、ビジネスを始めるっていうのは並大抵のことではできないよ。 
\\	とは「大変なこと、無理な注文」ということです。
\\	は「簡単なことだと言う」
\\	の意味。	
\\	泣きすぎて、涙がかれるよ。 
\\	ここでは「涸れる」を 
\\	を使って表現しています。(注)和文を過去形にして英訳しています。	
\\	今日はびっくりしすぎて涙が出たよ。 
\\	「涙が出る」を 
\\	を使って表現しています。
\\	の使い方にも注目。	
\\	涙が出るほど、あなたのことを、愛している。 
\\	「涙が出るほど」は
\\	あるいは
\\	「泣きたいほど」で表現できます。	
\\	そのドラマのエンディングを見た後、涙が止まらなかった。 
\\	「涙が止まらない」とは
\\	(泣くのをやめることができない)ということですね。	
\\	彼女は涙声で、自分の身の上話をした。 
\\	で「涙声で」の意味になります。
\\	は「涙を流しながら」の意。	
\\	彼女は涙ながらに、会社で何があったかを話し始めた。 
\\	で「涙を流しながら」の意味になる。
\\	は「涙ながらの説明(報告)」の意。	
\\	どうしたんだい。涙目になってるじゃないか。 
\\	で「こみ上げてくる」という意味。
\\	「潤んでいる」。	
\\	最近、涙もろくなっていけない。年のせいかな。 
\\	で「感傷的な涙が溢れるのを感じる」の意味になります。
\\	「涙もろくなる」。	
\\	遺族たちは涙をこらえながら、その事故原因の説明を聞いていた。 
\\	と
\\	は共に「抑える、食い止める」という意味です。	
\\	さあ、涙を拭いて、元気を出しなさい。 
\\	が「涙を拭く」に当たる言い回しです。	
\\	最近 ちょっと疲れただけでも会社休んじゃえーって思っちゃうんだよね。たるんでるよ。会社をなめてるね。 
\\	「なめる」とは
\\	を重要視しない」ということですね。
\\	軽く見る」。
\\	「たるむ」。	
\\	何かうちの会社って取引先になめられてるよ。 
\\	ここでは「なめる」を
\\	を馬鹿にする」で表現しています。
\\	は「だます」の意。	
\\	コンサートが始まってしまえばダフ屋が売っているチケットも安くなるもんだと思っていたのですが、そうとも限らないのですね。 
\\	で「かつては…だと思っていた」の意味。
\\	は「ダフ屋」の意。	
\\	来週親が来るので、今からどこへ連れて行こうかと、頭を悩ましています。 
\\	「悩ます」は
\\	を使えばいいでしょう。
\\	は「…について悩む」の意。	
\\	近頃、悩み事があってよく眠れないんだ。 
\\	「色々な悩み」は
\\	を複数形にして用います。
\\	「悩むことがたくさんある」と言い換えてもいいでしょう。
\\	は「全然眠れない」の意。	
\\	コレだけがたった一つの悩みの種なのよね。 
\\	「悩みの種」を 
\\	と 
\\	(心配させる)、
\\	(悩ませる)を組み合わせて表現しています。	
\\	ホームステイに持っていくお土産ですっごく悩んでいるの。 
\\	「悩む」をここでは
\\	するのに苦労する」、
\\	で悩む、心が乱れる」で表現しています。	
\\	ヒアリング力をどうやって向上させるかで今悩んでいます。いろいろ試してはいますが、伸び悩んで困っています。 
\\	「悩む」は 
\\	(途方に暮れる、困る)で表現すればいいでしょう。	
\\	悩んだ末わからなくて適当に答えを選んだのにあっていた。ラッキー! 
\\	で「…で色々と苦しんだ後」の意味になります。
\\	の使い方にも注目。	
\\	何か用事があって来るんならいいけど、僕たちに会うためだけに来るんだったら悪いから今度にしよう。 
\\	「(ぜひ)どうぞ」という意味の
\\	と、
\\	で「ならいいけど」を表現しています。	
\\	最近新しい習い事を始めました。 
\\	「習い事」は
\\	を使って表現すればいいでしょう。	
\\	何か習い事してるの?(子どもに) 
\\	で「何かの習い事をする」の意味を表しています。
\\	を使ってもいいでしょう。	
\\	米国を奈落の底に突き落とした
\\	氏を国際手配した。 
\\	で「奈落の底に突き落とす」の意味になります。
\\	は「犯人捜査」の意。(注)
\\	氏」のような表現は避けるようにして下さい。	
\\	それは、彼ならではの発言だ。 
\\	「ならではの発言」を
\\	(…が言うだろうと期待していること)で表現しています。	
\\	どのレジも人が並んでるわね。 
\\	「並ぶ」は
\\	を動詞・名詞として用いて表現できます。	
\\	やばい、この分だとクビになりかねない。 
\\	で「…という結果になる」の意味。	
\\	自分が主人公になりきって、謎を解いたり推理したりできる話の本が好きだな。 
\\	「主人公になりきる」を
\\	(主人公の中に入り込む)で表現しています。
\\	(引き込む)、
\\	(演繹法)。	
\\	インターネットって、相手の本当の正体がわからないから、ちょっと怖いよね。男性が女性になりすましてたりさ。 
\\	「なりすまし」とは
\\	として通る(通用する)、なりすます」ということですね。	
\\	今月は、祝日もないし、特に予定もないし、長いつまらない月になりそうだなー。 
\\	が「…になりそうな」に当たる言い回しです。
\\	は「…になる」の意。	
\\	なるべく返事をください。 
\\	で「なるべく、できるだけ」のニュアンスを出してみました。	
\\	えっ彼女まだ知らないの? なるほどね。サプライズパーティーだからな。 
\\	この場合の「なるほど」は
\\	「分かった」で表現すればいいでしょう。	
\\	車の運転は、理屈ではなく、慣れだよ。 
\\	この場合の
\\	は「(腕や技術に)磨きをかける」という意味です。もちろん、
\\	も使えます。	
\\	最近の若者は慣れていて、
\\	カメラを突然向けられても物怖じせずどうどうと受け答えしている。 
\\	「慣れている」は
\\	で表現できます。
\\	は「びっくりする」、
\\	は「ぺちゃくちゃしゃべる」の意。	
\\	彼は、日々日本に慣れてきている。 
\\	に慣れる」は
\\	ですね。	
\\	3日間ぐらい英語ばかり聞いていると、さすがに耳が慣れてくるね。 
\\	「…に慣れる」は
\\	で表現します。
\\	は「途切れなく、連続して」の意味。	
\\	あなたは女の子に慣れてるね。いつも遊んでるの? 
\\	ここでは「慣れている」を 
\\	(かなり…の経験がある)を使って表現しています。	
\\	慣れてないから上手くできないよ。 
\\	「慣れる」は
\\	を使えばいいでしょう。
\\	は「…を熟知している」の意。	
\\	何でも慣れてきた頃が一番危ないので、気を抜かずに頑張れと親からメールをもらいました。 
\\	「慣れる」は 
\\	で表現すればいいでしょう。
\\	の使い方にも注意。
\\	は「危険な」の意。	
\\	猫はそれぞれ縄張りを持っている。 
\\	は「領地、領域」、
\\	は「芝地、縄張り」の意。	
\\	時差ボケって、ただ眠いって感じではなくて、眠い上になんか気持ちも悪くなるんだよね。 
\\	「なんか」のニュアンスを
\\	で表現してみました。	
\\	なんか自分で言っておきながら恥ずかしくなてきた。 
\\	が「なんか」に当たる表現です。	
\\	ただの
\\	メール友達なんかに、こんなことおしえられないね! 
\\	この場合の
\\	は「大したことのない」の意味になります。	
\\	おまえまたふられたのか?今度で何回目だよ。 
\\	で「今回で何度目になるの?」という意味。	
\\	昨晩は、数時間にわたって軟禁され、体も心もへとへとだ。 
\\	「軟禁される」には
\\	という言い方がありますが、ここでは
\\	「閉じこめられる」を使っています。
\\	「心身共にへとへと」。	
\\	今、チーズにはまっているんだけど、チーズだったら一日何個でも食べられるよ 。 
\\	は不可算名詞ですから、「何個」はこの場合
\\	と表現すればいいでしょう。	
\\	彼はこちらが出した条件に難色を示した。 
\\	この場合の
\\	は「ためらい、危惧」という意味になります。
\\	を使っても「難色」のニュアンスは出せます。	
\\	さぁて、何時から勉強始める?さっさと始めないと眠たくなってやる気なくなるよ。 
\\	「何時から」は
\\	を使って表現します。
\\	は「やる気になる」の意。	
\\	1月3000円を支払えば、何時間インターネットをしてもそれ以上の電話料金を払う必要はありません。 
\\	「何時間」は 
\\	、あるいは、
\\	で表現すればいいでしょう。	
\\	あいつ軟弱だからな。 
\\	で「気骨がない」の意味になります。
\\	「意気地のない弱虫」。	
\\	日本人同士で、英語を話してるなんて、なんだかおかしいね。 
\\	「なんだかおかしい」は
\\	、あるいは、
\\	で表現できますね。	
\\	何だかんだ言っても、彼のこと好きなんでしょ。 
\\	「何だかんだ」は
\\	(結局)を使っても表現できますが、ここでは、
\\	(何言ってもいいけどさ)で日本語に近く英訳しています。	
\\	きみはいつも何だかんだ理由をつけては断るんだね。 
\\	で「何だかんだ」を表現しています。
\\	は「言いわけ」、
\\	は「口実」の意。	
\\	一度行って見たいけど、日帰りで行くにはちょっと遠いな。かといって、1日中家にいるのもなんだし。 
\\	「なんだし」のニュアンスを
\\	は分からないし」で出しています。
\\	「日帰り旅行」。	
\\	ななんだっけ?どこかで聞いた事があるんだけど。 
\\	の使い方がポイントです。
\\	「何だったけ(もう一度言ってみて)」。	
\\	なんだか知らないうちに、日本語教師になったわたし。なんちゃって。 
\\	「なんちゃって」は
\\	!「ほんの冗談さ」か
\\	!「(前の内容を受けて)そなことないよ」を使って表現します。	
\\	今日のサッカーの中継って、何チャンネルでやるの? 
\\	と
\\	、及び、
\\	「放送する」の使い方がポイントですね。	
\\	彼は、何ていうか…ちょっと変わってるの。 
\\	で「どのように言えばいいか」の意味。
\\	を使ってもいいでしょう。	
\\	彼は英語ができるなんてもんじゃない、文句の付けようがないくらい完璧だ。 
\\	(…と言うのは、控えめな言い方だ)で「…なんてもんじゃない」のニュアンスを出しています。	
\\	ちょっと、家から遠い(英会話学校が)のが難点だけど、入っちゃえばどうにかなるか。 
\\	「難点」は
\\	を使えばいいでしょう。
\\	「申し込む」は自動詞・他動詞両方に使えます。	
\\	なんでこう休み時間って過ぎるのが早いんだろう。仕事しているときは時間が経つのが長く感じるのに。 
\\	で「なんでこう」のニュアンスは出せますね。
\\	は「辛く思えるほどに」の意。	
\\	車なんて走ればなんでも良いって感じ。 
\\	で「どんな車でも」の意味。
\\	する限りは」。	
\\	何と言っても、あの緑の多さと、人々が時間に縛られていない環境がいいね。 
\\	ここでは「何と言っても」を
\\	「確実に...はすごいと言えるよね」で言い換えています。	
\\	それ、何とかって言うんだったよね。うーん、思い出せないなあ。 
\\	「何とか」は
\\	、あるいは、かなりくだけた言い方になりますが、
\\	「何とかいうやつ(「ワチャマコーリッ」と発音)」を使えばいいでしょう。	
\\	前の人が注文しているのを盗み聞きしてなんとか英語で注文する事ができた。 
\\	「なんとか 
\\	する」は
\\	を使って表現します。	
\\	何とか助けてあげたいけど、私には力(権限)がなくてどうもしてあげられない。申し訳なさでいっぱいです。 
\\	「何とか助ける」は 
\\	、あるいは、
\\	で表現すればいいでしょう。
\\	は「権限」、
\\	は「影響力」の意。	
\\	なんとかなるさ。あまり気にするなよ。 
\\	の文字通りの意味は「それから何かが生じる」。
\\	「なんとか切り抜ける」。	
\\	私はヒアリングのスキルに問題があるのでなんとかしなければと思っています。何かいいアイデアありません? 
\\	「何とかする」は 
\\	を使えばいいでしょう。
\\	は「…に問題がある」の意。	
\\	あまり気にするなよ。なんとかなるさ! 
\\	「なんとかなる」は
\\	「うまく行く」を使えば表現できます。主語は、その状況に応じて、
\\	など使い分けるようにすればいいでしょう。また、
\\	とも言えます。
\\	は
\\	から生じる、起きる」の意。	
\\	なんとなく思い立って、手紙を書いたんだけど、あの人元気かな。 
\\	「どういうわけか」と
\\	「思い浮かぶ」を組み合わせて「なんとなく思い立って」を表現しています。
\\	「耐える、持ちこたえる」。	
\\	何ともいえない気持ちになる。 
\\	で
\\	を言葉にすることができない」という意味になります。	
\\	もう諦めなよ。彼はあなたのことなんか何とも思ってないよ。 
\\	の文字通りの意味は(きみが存在していることさえ知らない)です。	
\\	最近、仕事が遅れていても全然焦らないし、なんとも思わなくなってしまった。仕事に対しての責任感がなくなったのかなあ。 
\\	(真剣に受け取らない)で「何とも思わない」を表現しています。
\\	でも同じようなことがいえます。	
\\	若い頃は、徹夜しても、何ともなかったけどな。 
\\	「何ともない」は
\\	、あるいは、
\\	で表現できます。	
\\	何度見ても、彼の髪型ってワイルドだ。でも、手入れが大変そう。 
\\	「何度見ても」は 
\\	と 
\\	を組み合わせて表現すればいいでしょう。	
\\	働きすぎて、体を壊したら何にもならないよ。 
\\	「何にもならない」は 
\\	で表現できます。	
\\	あなたは、何人兄弟の何番目ですか? 
\\	第一例文で注意するのは、
\\	動詞が
\\	になっている点。現在も家族と一緒にすんでいるなら
\\	でもいいが、別々に暮らしているのなら
\\	を使うことにります。また、
\\	の使い方もチェック!	
\\	もう何人も日本人がその家庭にはホームステイしているから、きっと日本人の好みの味を知っているんだと思うよ。 
\\	「何人もの」は
\\	「かなり多くの」、
\\	「多くの」を使えばいいでしょう。
\\	「泊める、接待する」。	
\\	何年払いのローンで家かったの? 
\\	この場合の「何年」は
\\	、あるいは、
\\	「期間」を使えばいいでしょう。	
\\	何年経っても英語力が上達しない。 最近、さぼってたからな。 
\\	ここでは「何年経っても」を
\\	「何年勉強しても」で表現しています。また、「さぼる」を
\\	「ほうっておく」で表現している点にも注目。	
\\	街で男になんぱされた。 
\\	は「引っかける」、
\\	は「言い寄る」の意味を表します。	
\\	ここ最近、怪しげな人に出会います。最初は、道を聞いてきたのかと思いきや、ただのナンパでした。 
\\	には「ナンパする、引っかける」という意味があります。
\\	は「怪しげな」の意。	
\\	布団乾燥機のタイマー何分間にセットしたの? 
\\	と
\\	を組み合わせればいいでしょう。	
\\	プロ野球は、結果を出してなんぼの世界だろ。 
\\	「なんぼ」をここでは
\\	「お金を稼ぐ」で表現しています。	
\\	アメリカって電圧何ボルトだっけ?プラグの形は日本と一緒? 
\\	「何ボルト」は
\\	、あるいは、
\\	で表現します。
\\	「一致する」、
\\	「アダプター」。	
\\	私たちはなんらかの形で報復はすべきだと思う。 
\\	は「何らかの方法で」、
\\	は「ある種の…」の意味。
\\	は「報復する」、
\\	は「懲罰、当然の報い」の意。	
\\	犯人だけが標的になればいいのだが、市民を巻き添えにしたら向こうがやったこととなんら変わりない。報復でなくテロになりかねない。 
\\	で「…と変わりがない」の意味になります。
\\	は「報復」の意。	
\\	昨日買ったワンピースなんだけど、どう?似合っている? 
\\	「似合う」は 
\\	を使って表現します。
\\	との組み合わせにも注目。	
\\	彼女って体に似合わず大食いだね。 
\\	で
\\	に合う」の意味。
\\	の割には」。	
\\	相変わらずおまえは煮え切らないやつだな。彼女との結婚どうするつもりなんだ。 
\\	には「態度をはっきりさせないでいる」という意味があります。
\\	は「優柔不断な、煮え切らない」の意。	
\\	この植物って何か効果があるの?臭い消しとか? 
\\	「臭い消し」をここでは
\\	「防臭剤」を使って表現しています。
\\	「除去する」。	
\\	ちょっとこの辺怪しいんじゃない?臭うよねえ。 
\\	とは「怪しい、うさんくさい」の意味。また、
\\	の臭いがする」の意。	
\\	ごめん。カバンの中にポテトが入っているから、ちょっとにおうかもしれない。 
\\	「におう」は
\\	で表現します。	
\\	ソフトクリームは2回転半させてつくるのがこつだと教わりました。 
\\	「2回転半させる」は
\\	と言います。
\\	とも言えます。	
\\	傘を持ってくるのを忘れた日に限って、雨が降ってきたりするんだよね。まったく、いやになるよ 
\\	~で「~する時には、確実に... する」という意味を表します。
\\	「困惑」。	
\\	カナダって学費が高校まで無料って聞いたけど本当ですか?当然、公立に限ることだと思いますが。 
\\	で「…に限る」の意味になります。
\\	でも同じようなニュアンスが表現できます。	
\\	やっぱり、アメリカで売ってるタイタニックのビデオも2巻セットなの?日本と同じなんだ。 
\\	「2巻セット」は
\\	、あるいは、
\\	と言います。	
\\	彼、似顔絵を描くのがとてもうまいんだ。プロになれるよ。 
\\	「似顔絵」は
\\	と言います。「描く」は
\\	を使えばいいでしょう。	
\\	これは、私が苦手とする物のひとつです。 
\\	「苦手」は
\\	で表現します。
\\	のようにも言えます。	
\\	彼の事は嫌いなわけではないけど、苦手だな。絶対友だちにはなれないタイプ。 
\\	(…がいると居心地が悪い)で「苦手」を表現しています。
\\	の使い方にも注目です。	
\\	どんな分野でも、勝ち組と負け組がはっきりする二極分化の時代に突入した。 
\\	は
\\	の間には明確な区別がある」、
\\	は「はっきりと大きく別れている」の意味になります。	
\\	外は祭りで賑やかだね。夏って感じがする。これは、家でゆっくりテレビなんか見ている場合じゃないぞ。見に行ってこよっと。 
\\	は「活気がある、賑やかな」の意味の形容詞です。
\\	(ごろごろする)。	
\\	このドリンクを飲めば疲れにくいです。 
\\	「疲れにくい」とは
\\	(簡単には疲れない)ということですね。	
\\	最近、お腹の当たりに少し肉がついてきたみたいだ。 
\\	は「ビール腹」のことです。では、
\\	は?「お腹の周りについた余分な脂肪」のことですね。
\\	は「発達させる」という意味ですから、これで「お腹の周り
\\	に余分な肉(脂肪)がつく」となるのです。	
\\	あそこ杉の木だらけで、肉眼でも花粉が飛んでいるのが見えたよ。 
\\	「肉眼」は
\\	という言い方をしますが、ここでは
\\	「実際に(目で)見える」で表現しています。	
\\	太股が肉離れになった。 
\\	「肉離れ」は
\\	と言います。「肉離れになる」は、動詞
\\	を用いて、例えば、
\\	のように表現します。また、
\\	や
\\	などのような言い方もできます。	
\\	彼ってあんなひどいことしても、なんか憎めないんだよね。 
\\	が「…を憎めない」に当たる表現です。	
\\	犯人は窓から逃げたようだ。逃げ足の速い奴だ。 
\\	「逃げ足の速い」は
\\	を使って表現すればいいでしょう。
\\	は「つるりと逃げる」の意。	
\\	自分に都合が悪い話になったからって、(会話から)逃げてない? 
\\	は「…を回避する」、
\\	は「…から手を引く」の意味。	
\\	彼の英語の発音って、なんかにごってない? 
\\	が「にごる」に当たる表現です。
\\	「不明瞭な、ぼそぼそ話すような」。	
\\	男に二言はありません。言ったことは実行します。 
\\	で「前言を翻す」という意味になります。
\\	「最後までやる」。	
\\	結婚式の二次会はクラブで行ないます。 
\\	「…の2次会」は
\\	、あるいは、... 
\\	で表現できます。	
\\	あいつは二重人格だからな。切れると何をするか分からないよ。 
\\	「二重人格」は
\\	と言います。
\\	は「癇癪を起こす」の意。	
\\	目をこすったら、物が二重に見える。 
\\	「(物が)二重に見える」は
\\	と言います。
\\	「こする」。	
\\	24時間いろんな番組を放送しているから、家に帰るのが遅くなっても見れるし、再放送を何度もやっているので私にはピッタリです。 
\\	投稿された日本語をできるだけ忠実に英語に訳してみました。ポイントとなる「24時間」は24 
\\	でいいでしょう。	
\\	「24時間営業」とは
\\	「24時間開いている」ということですね。
\\	「24時間」。	
\\	この支店のコンピュータは2000年問題に対応していない。 
\\	とは「対応している」の意味。「2000年問題」は
\\	という言い方をするので一緒に覚えておきましょう。	
\\	日本にもこれに似たものがあります。もうちょっと濃い味だけど。 
\\	「似たもの」とは
\\	「同じ(ような)もの」ということですね。
\\	「味が濃い」。	
\\	では、次回の集まりの日時と場所を決めることにしましょう。 
\\	「…の日時」は
\\	でいいでしょう。
\\	の中に
\\	も含まれています。	
\\	2、3ヶ月ホ-ムステイをして日常会話を学びたい。 
\\	「日常会話」は
\\	でいいでしょう。
\\	とも言えます。	
\\	今日図書館に行く途中で映画の撮影をしているのを見たんだけど、思わず興奮してしまった。地元の人には日常茶飯事らしいけど。 
\\	で「毎日あること」の意味になります。
\\	(ごく自然なこと)でも同じようなことが言えます。	
\\	毎朝牛乳を飲むのを日課にしているので、今更やめられないなぁ。 
\\	「日課」は
\\	ですね。また、
\\	(たいていは)を使ってもいいでしょう。	
\\	いろいろ問題があって、にっちもさっちもいかない状態です。 
\\	「本当に困った状態に陥る」の意。また、「身動きが取れない状況」をアメリカの小説のタイトル 
\\	を使って、
\\	と言うこともあります。	
\\	いくつかのチャンネルでは、それを生放送でやるだろうね。日中は仕事で見れないから、夜の再放送を見る。 
\\	「日中」は
\\	を使えばいいでしょう。
\\	を生で放送する」。	
\\	もう煮詰まっちゃって、アイデアが出てこないよ。 
\\	は「へとへとになる、疲れる」、
\\	は「知恵を絞る」の意味。	
\\	自分の友達を見てみると、結構自分に似ている(性格的に)人が多いかもしれない。 
\\	「性格的に似ている」は
\\	で表現できますね。	
\\	誰かに似てるって言われない? 
\\	ここでは「誰か」を具体的な名前にしています。「似ている」は
\\	ですね。	
\\	本社の方針が二転三転するから、現場の人間は、振り回されっぱなしだ。 
\\	「二転三転する」を
\\	(何度も…を変える)で表現しています。
\\	は「回転木馬」の意。	
\\	手間を省こうと思ってしたことが、かえって二度手間になってしまった。 
\\	で「結果としては手間が2倍になる」の意味。
\\	は「余分な」の意。	
\\	すっかり寒くなって、朝起きるのが辛い。ついつい二度寝してしまう。 
\\	で「どうしてもまた寝てしまう」という意味になります。
\\	「寝返りを打つ」。	
\\	今日、12歳になった。 
\\	になった」を
\\	を使って表現しています。	
\\	海外旅行の一番の目的は、自分の英語力を試す事であって、観光は二の次です。な~んてね。 
\\	が「主の、第一の」、
\\	が「第二の、2の次」の意味になります。
\\	!は前に言ったことを否定する場合に使います。
\\	の後には、
\\	が省略されています。これで「そんなことはない」と言	
\\	このバスを利用するには、チケットが必要ですか? 
\\	この場合の「には」とは
\\	するためには」ということですね。
\\	「必要とする」。	
\\	その歌がヒットしたとたんに、二匹目のドジョウを狙って、似たような歌が次から次へと出てきた。 
\\	「また幸運をつかもうと」、
\\	(成功を再現する)で「二匹目のドジョウ」を表現しています。	
\\	おまえも鈍いね。彼女がおまえのこと好きだってことがわからないのか? 
\\	「鈍い」は 
\\	、あるいは、
\\	で表現すればいいでしょう。	
\\	各授業内容に関して詳しく書いてあるパンフレットのようなものは置いていますか?日本版があればそちらをいただきたいのですが。 
\\	「日本語版」は 
\\	と言います。
\\	は「小冊子」の意。	
\\	これは、日本語吹き替えです。 
\\	が「吹き替える」に当たる動詞です。	
\\	やはり、ここのタイ料理の味付けは、日本人好みにしてあるんだね。 
\\	「日本人好みに」を
\\	「日本人が気に入るように」で表現しています。
\\	に合わせて」。	
\\	オリンピックで日本選手がメダルを取ると、もらい泣きするのって日本的かな? 
\\	で「とても日本的なこと」の意味になります。	
\\	あいつは二枚舌だから気を付けた方がいい。 
\\	「二枚舌」は
\\	といいます。
\\	「2つの顔を持つ、裏表のある」でも同じ様な意味が表せます。	
\\	荷物になるから向こうで買うわ。 
\\	「(無理して)持ち運ばなくてはならないもう一つの物になるだけ」で「荷物になる」を表現しています。
\\	は「…に追加される」の意。	
\\	30にもなってそんなこともわからないの? 
\\	とイントネーションを上げて「30にもなって」を表現しています。
\\	を使ってもいいでしょう。	
\\	今日の夕飯、タケノコの煮物でいい? 
\\	「煮物」は 
\\	を使って表現します。
\\	は「タケノコ」の意。	
\\	英語で話すとき、自分の言いたいニュアンスで相手に伝わってるかが気になるんだよね。 
\\	で「ニュアンスを伝える」という意味になります。
\\	は「適切な」の意。	
\\	あの家族、みんなスーツだね。多分入園式に出るんだよ。 
\\	「入園式」は
\\	でいいでしょう。
\\	は「盛装する」の意。	
\\	いつ頃入荷しますか? 
\\	(売られている)、
\\	(在庫にある)で「入庫」を表現しています。	
\\	昔、入試結果発表の前にはよく発表の夢を見た。でも、不思議と現実には受かっているのに、夢の中ではいつも不合格。 
\\	「入試(結果)発表」はそのまま
\\	と言えます。
\\	「掲示する」。	
\\	朝から散々だったよ。電車が信号機故障で遅れてホームへは入場規制され、やっとホームに行けても電車混んでて乗れないし。 
\\	で「駅のホームへの入場を制限する」の意味になります。
\\	は「規制する」の意。	
\\	電車の事故で遅れが出ていて、ホームは入場制限されているみたいだな。これで、会議に間に合わなかったらどうしてくれる! 
\\	で「込み合った駅のプラットホームへ行くのが制限されている」の意味になります。	
\\	海外の友だちに
\\	メール送るときって、いつも辞書とにらめっこよ。 
\\	で「自分自身を…から引き離す」の意味になります。	
\\	あそこの場所は、自分の庭みたいなもんだ。 
\\	で「…を知り尽くしている」の意味。	
\\	私の趣味はガーデニングです。 
\\	「ガーデニング」は 
\\	ですね。
\\	の使い方にも注目。	
\\	ワールドカップのおかげで、普段サッカーを見ない母まで、にわかサッカーファンになっちゃったよ。 
\\	ここでは「にわか」を 
\\	で表現しています。
\\	は「すぐに、たちまち」の意。	
\\	アメリカと比べて日本の総理大臣の任期って短いよね。 
\\	「任期」を
\\	「在職する」で表現しています。
\\	「在職期間、任期」。	
\\	ベッカムは帰国後もすごい人気だね。たいしたもんだ。 
\\	「ものすごい人気」は 
\\	で表現すればいいでしょう。	
\\	職場では人間関係ってすっごく大事だと思う。たとえ忙しくても理解しあえる仲間がいれば頑張れる。 
\\	「人間関係」を
\\	「人との付き合い方」を使って言い換えています。
\\	とうまくやっていく」。	
\\	君も四十を過ぎたんだから、年に一度は人間ドックに入って検査してもらった方がいいよ。 
\\	「人間ドック」をそのまま英語に直訳しても意味は通じませので、
\\	「精密検査」や
\\	「身体検査」などを使って表現します。	
\\	電子メールって便利だけど、ちょっと人間味に欠けると思わない? 
\\	「人間味」は
\\	でそのニュアンスが出せます。
\\	「欠ける」、
\\	「論じる」、
\\	「伝える」。	
\\	よく目をつぶった瞬間から眠ることできるね。人間業とは思えない。 
\\	「人間業とは思えない」とは 
\\	(人間業を超えた)ということですね。
\\	は「ほとんど」の意。	
\\	日本人は、喫煙による害が人体に及ぼす危険性に対してあまりにも認識が薄すぎる。 
\\	で「…への認識がはっきりしていない」の意味になります。	
\\	妊娠6ヶ月にもなると、だいぶお腹が目立つようになるね。 
\\	「妊娠している」。その名詞形が
\\	です。
\\	「目立つ」の使い方にも注目。	
\\	レジュメを会場に、人数分だけ持っていってください。 
\\	ここでは「人数分」を 
\\	を使って表現しています。
\\	の使い方にも注目。	
\\	スパイダーマンがハリーポッターの興行収益を抜いて1位になったらしい。相当の人気だね。 
\\	「抜いて」は 
\\	(追い越す)、
\\	(追い抜く)で表現すればいいでしょう。	
\\	私たちがきゃーきゃー騒いでいる間に、後ろにいた人が友達の財布をバッグから抜き取ったのです。 
\\	には「引き抜く」という意味があります。
\\	「盗む」。	
\\	リビングにズボン脱ぎ捨ててあったよ。 
\\	「脱ぎ捨てる」は
\\	「置きっぱなしにする」で表現できますね。	
\\	彼女はこの職場みんなのアイドルだから、抜け駆けするなよ。 
\\	(巧みに身をかわす)、
\\	(デートに誘う)を組み合わせて「抜け駆けする」を表現しています。
\\	は「勝る」の意。	
\\	近頃抜け毛がひどくてね。 
\\	には「毛が抜け落ちる」という意味があります。	
\\	英語を話すとき、初めに日本語で考えてしまう癖がどうもまだ抜けないみたいで。よくないとはわかっているのですが。 
\\	で「…する癖を直す」の意味になります。
\\	を使っても「まだ抜けない」のニュアンスは出せますね。	
\\	この当たりに新宿へ出る抜け道があったはずだが。 
\\	への抜け道」は
\\	で表現できます。	
\\	昨日は飲み会、途中で抜けちゃて悪かったな。 
\\	で「…の途中で抜ける」の意味になります。	
\\	彼、この部署長いから、ここの主みたいなもんだな。 
\\	は「動かす、運営する」、
\\	は
\\	の責任を担う」の意味。	
\\	盗み聞きするなんて、感じ悪いと思わない? 
\\	で「盗聴する」という意味がありあます。
\\	「気味が悪い、嫌な」。	
\\	すみません、ちょっとこのタオル、濡らしてもらえますか? 
\\	この場合の「濡らす」は動詞 
\\	を使って表現すればいいでしょう。	
\\	このお茶、ぬるいね。入れなおしてくれないか。 
\\	「ぬるい」は 
\\	で表現します。
\\	は「なまぬるい」の意。	
\\	そのお金を盗んだのは僕ではない。それは濡れ衣だ。 
\\	で「間違って非難される」の意味になります。	
\\	うちの子寝起きが悪くて、毎朝起こすのが大変なんだよ 
\\	「寝起きが悪い」とは
\\	「起きるのが大変」ということですね。
\\	「寝坊」。	
\\	なるべく9時までに、子どもを寝かしつけるようにしているけど、なかなか寝てくれない。 
\\	「寝かしつける」は 
\\	で表現します。
\\	も使えます。	
\\	3年じっくり寝かせた秘伝の焼き鳥のタレ。 
\\	「熟成させる」が「寝かせる」にぴったりの言い方です。	
\\	神様お願い。私の願いを叶えて。 
\\	「認める」の使い方がポイント。
\\	「私の祈りを聞いてください」。	
\\	願わくば、その時あなたといっしょにいられますように。 
\\	「願わくば」は
\\	を名詞、動詞として用いて表現できます。	
\\	顔を合わせるたびに文句ばっかり言ってないで、たまにはねぎらいの言葉の1つでもかけてくれたらどうなんだい。 
\\	で「ねぎらいの言葉を1つかける」の意味になります。
\\	の代わりに 
\\	も使えます。	
\\	きみは値切るのがとってもうまいね。 
\\	動詞 
\\	には「値切る」という意味があります。
\\	は「値切る人」の意。	
\\	彼女もねくらな性格で損してるよね。 
\\	「ねくら」は
\\	「暗い、陰鬱な」で表現できます。
\\	「ふさぎ込んでいる」。	
\\	私は、とても、猫舌なんです。 
\\	「猫舌」は
\\	「熱い物を食べたり飲んだりできない」、あるいは、
\\	「熱い物にとても敏感である」を使って表現できます。	
\\	猫背で歩くのは止めなさい。 
\\	は「前かがみになる、猫背になる」の意味。	
\\	で、だれがネコの首に鈴をつけるんだい? 
\\	「猫に鈴をつける」とは「誰がそれをするのか」ということですから、
\\	「誰がそれをやるガッツがあるのか」で表現することができます。
\\	は「やる気」の意。	
\\	楽しい旅行のはずが、雪で予想以上に寒く、帰ってきてから2日間調子が悪くて寝込んでしまいました。 
\\	が「(病気で)寝込む」に当たる言い回しです。
\\	も使えます。	
\\	初対面の人と話すときって、どうしても猫かぶって話しちゃうんだよね。 
\\	ここでは 
\\	(子猫のように害を与えない振りをする)で「猫をかぶる」を表現しています。	
\\	昨日の夜、寝言いってたけど、何の夢見てたの。 
\\	「寝言をいう」はそのまま
\\	で表現できます。	
\\	寝酒にこのワインを買っていこう 
\\	が「寝酒」に当たる表現です。	
\\	彼はなかなか屈服しなかったので、力でねじ伏せた。 
\\	で「…に無理矢理頭を下げさせる」の意味になります。
\\	は「…を同調させる」の意。	
\\	うちのカミさん、寝相が悪いのなんのって。 
\\	とは「寝ながら絶えず動き回っている人」という意味。
\\	「寝返りを打つ」。	
\\	彼も火山ストーリーのネタが尽きちゃったんじゃない?だから、今回は監督をやらなかったとか。 
\\	(ちょっと古くなる)で「ネタが尽きる」を表現しています。
\\	は「…が尽きる、なくなる」の意。	
\\	もうネタが尽きちゃって、何も思いつかない。 
\\	「ネタ(素材)が無くなる」の意味。	
\\	人のことをあまり妬んではいけないよ。 
\\	「妬む」は
\\	を使えばいいでしょう。
\\	でも
\\	父親にねだって、やっとパソコンを買ってもらったの。 
\\	は「しつこく追い回す」、
\\	は「うるさくせがむ」の意味。	
\\	昨日、寝違えて首がいたいのよ。 
\\	で「首の筋を違える」の意味。
\\	は「悪い姿勢で寝る」の意。	
\\	あいつはねちねちした性格だから付き合いにくいよね。 
\\	は「しつこい」という意味ですね。
\\	「恨みを持つ」。	
\\	彼の付き合った女って星の数ほどいて、自分で言うには熱しやすく冷めやすいタイプなんだって。 
\\	(好きになって、すぐに嫌いになる)が「熱しやすく冷めやすい」に当たる言い回しです。	
\\	こう熱帯夜が続くと、クーラー無しでは眠れないよ。 
\\	「熱帯夜が続く」を
\\	で表現しています。	
\\	私は寝つきが悪くて、布団に入ってから1時間は眠れない。 
\\	「寝つきが悪い」を
\\	「眠りにつくのが大変」で表現していまうす。	
\\	最近いろいろ心配事ばかりしているせいか、夜寝つきが悪いです。もっと、気楽にならないとと思ってはいるのですが。 
\\	(なかなか眠れない)で「寝つきが悪い」を表現しています。	
\\	この頃、誰も彼もが、熱に浮かされたように、
\\	革命」
\\	革命」と騒いでいる。 
\\	「夢中になって、我を忘れて」と
\\	「熱狂して」を組み合わせて「熱に浮かされたように」を表現しています。
\\	で「多くの人たちが
\\	という意味になります。
\\	「息を切 「息を切らして」。	
\\	寝ても覚めても考えるのは彼女のことばかり。 
\\	が「寝ても覚めても」に当たる表現です。
\\	「昼も夜も、寝ているときでさえ」。	
\\	彼は、私があなたにコーヒーを作ってあげたのを根にもっているようだ。なんて心の狭いやつだ。 
\\	が「…に対して根に持つ、恨む」に当たる言い回しです。
\\	は「心の狭い」の意。	
\\	ねねね、ちょっと聞いて。今から私が言う英語、正しいと思う? 
\\	「ねねね」と相手の注意を引くために声を掛けているわけですから、
\\	を使えばいいでしょう。	
\\	粘り強く交渉するしかないよ。 
\\	には「へこたれずに頑張る」という意味があります。
\\	「粘り強い」。
\\	しかない」を
\\	「唯一の選択」と表現している点にも注意。	
\\	粘ったけど、あそこに居座るのは2時間が限界だった。 
\\	は「最後まで頑張る」という意味です。
\\	(できるだけ長く)を使ってもいいでしょう。	
\\	いくらか値引きしてください。他にもまだ買いますから。 
\\	「値引き」は
\\	を使えばいいでしょう。	
\\	最近映画にはまっていて、ビデオを借りては毎晩見て、寝不足気味です。 
\\	ここでは「寝不足」を 
\\	(十分睡眠が取れない)で表現しています。
\\	は「…に夢中になる」、
\\	は「…に夢中になっている」の意。	
\\	義母ったら私のお給料の使い道を根掘り葉掘り聞くのでやんなっちゃう。 
\\	で「ありとあらゆる詮索をするような質問をする」の意味になります。
\\	は「詮索する」の意。	
\\	うちの弟は、よく寝ぼけて変なことを言い出す。 
\\	は「意識がもうろうする、うわごと言うような状態になる」の意味を表します。	
\\	1000番目の投稿をねらっていたのに、先を越された。 
\\	「ねらう」は
\\	あるいは、
\\	で表現できます。	
\\	日本はアメリカを支援しているので、敵対国から狙われる可能性はありますね。 
\\	「狙われる」は 
\\	を動詞、名詞として用いて表現しています。
\\	は「敵の」の意。	
\\	あいつ、年がら年中仕事ばかりして、何が楽しいのかね。 
\\	で「年がら年中」の意味になります。	
\\	念願のアメリカに来れて嬉しい! 
\\	ここでは「念願」を
\\	を使って表現しています。	
\\	階段から落ちて、捻挫しちゃった。 
\\	「捻挫する」は 
\\	と言います。
\\	は「足首」、
\\	は「捻る」の意。	
\\	こんなに一生懸命に働いても、年収500万円そこそこだぜ。 
\\	で「年に~円稼ぐ」の意味になります。	
\\	よーし、いい子ね。ねんねしなさい 
\\	が「ねんねしなさい」に当たる表現です。	
\\	念のため、下記の件を確認して下さい。 
\\	で「確認のため」の意味になります。
\\	は「何かあるといけないので」の意。	
\\	年齢に関係なく犯罪者の顔は公表すべきだと思う。未成年者だけ公表しないのはおかしい。 
\\	で「何才であろうが」の意味になります。
\\	「年齢に関わらず」。	
\\	「能ある鷹は爪を隠す」っていい日本語の表現だよね。 
\\	ここでは「能ある鷹は爪を隠す」をそのまま英語で表現してみました。これでも諺の持つ意味は伝わります。
\\	「(猛禽類の)爪」。	
\\	能書きはいいから、早くやれってんだよ。 
\\	「能書き」は「話しや説明はもういい」と言い換えて、
\\	で表現すればいいでしょう。	
\\	納期がまだまだ先だと思うと、いけないと思いつつも、ついついだらだら仕事してしまう。 
\\	「納期」は
\\	「配達の日」で表現すればいいでしょう。
\\	「楽しい、快い」。	
\\	彼は「のうてんき」でいいよなあ。 
\\	「脳天気」とは、「周りののことを気にせずに自分のことをやる」の意味にとって、ここでは、
\\	表現しています。	
\\	今年の納涼企画は、屋上ビヤガーデン! 
\\	「納涼企画」をここでは
\\	「夕方のそよ風を楽しむ」と言い換えてみました。	
\\	被害者の気持ちになって考える。 
\\	で
\\	の立場に自分自身を置く」の意味。
\\	は「見方、考え方」の意。	
\\	残すところあと10日で今年も終わりか。 
\\	と 
\\	を組み合わせて「残すところあと…日」を表現しています。	
\\	こっちのボディーソープから先に使ってね。まだ少し残っているから。 
\\	で「まだ少し残っている」の意味になります。
\\	は「底」の意。	
\\	あんなの単なるリップサービスだっつうのに、のこのこ出かけていく奴があるか! 
\\	「彼の招待に応じる」と
\\	「ずうずうしく、厚かましく」で「のこのこ」のニュアンスを出しています。	
\\	これ残り全部食べちゃっていいの? 
\\	で「残っているものすべて」の意味になります。	
\\	あのクラスを受講していたのは、ほとんどが韓国人で、残りはいろいろでした。メキシコ人、中国人など。 
\\	この場合の「残り」は 
\\	で表現すればいいでしょう。	
\\	残り時間後5分だ。ディフェンスをかためろ。 
\\	「残り時間」は
\\	を使って表現します。
\\	では
\\	の前に
\\	が略されています。
\\	は「支える、強化する」の意。	
\\	前の日の残り物を使って子どもたちのお弁当を作ることが多いわね。 
\\	「残り物」は
\\	と言います。	
\\	そんなもん、のし付けてお返ししてやる! 
\\	「のしをつけて返す」をここでは「おまけをつけて返す」と解釈して、
\\	「利子をつけて」を使って表現しています。	
\\	森の中から熊がのっそりと現れた。 
\\	「のっそり」は
\\	で表現できますね。
\\	「ずうたいのでかい」、
\\	「いばる」の意。	
\\	うちのクラスはのりが悪くて、文化祭の劇もぱっとしそうにないよ。 
\\	(クラスとしての意気込みがあまりない)で「のりが悪い」を表現しています。	
\\	よく日本の外交政策は、のっぺらぼうのようで顔がないといわれる。 
\\	「のっぺらぼう」をここでは
\\	と
\\	「味気ない、無表情の」を使って表現しています。	
\\	彼は、俺の昔(学生時代)にそっくりだ。 
\\	~で「まるで~のようだ」の意味ですね。	
\\	これは以前からのどから手がでるほど欲しかった漱石の「心」の初版本だ。 
\\	で「たまらないほど...したい」の意味になります。
\\	は「ずっと...したいと思っている」の意。	
\\	この時計喉から手が出るほど欲しかったんだ。 
\\	英語では「喉」の代わりに
\\	「味わう」、
\\	「右手を代わりに差し出す」を使って表現します。	
\\	昨日の夜、カラオケを歌いすぎて喉がかれてしまった。 
\\	が「喉がかれている」に当たる形容詞です。	
\\	今、髪を伸ばしているんだ。 
\\	「伸ばす」は 
\\	で表現します。	
\\	中学時代は名選手だったが、その後伸び悩んだ。 
\\	で「伸び悩む」を表現しています。
\\	は「向上する、上達する」という意味ですね。
\\	は「一流選手、有名人」の意。	
\\	あなた方の子供がのびのびと生活している環境を見て、私も子供にそういう生活をさせたいと思った。 
\\	ここでは「のびのびと」を
\\	「自由に、気楽に/活発に」を使って表現しています。	
\\	このラーメンのびてるよ~。 
\\	は「ふやけた、水浸しの」の意味になります。	
\\	彼はコンピュータについて詳しいので3カ月の契約でしたけど、延びる可能性もあるって本人が言っていた。 
\\	「延びる」は 
\\	を使えばいいでしょう。第一英訳では 
\\	の代わりに 
\\	も使えます。	
\\	このTシャツ着ているうちに首周りが伸びないか心配だな。 
\\	この場合の「伸びる」は
\\	と言います。	
\\	誕生日を覚えていてくれる事の方が大事だと思うの。そう思わない? 
\\	で
\\	する事の方が大事の意味になります)	
\\	寒いときと暑いときでは、ちょっと擦りむいただけでも、寒いときのほうが痛いと感じませんか? 
\\	この場合の
\\	のほうが」は 
\\	を使って表現すればいいでしょう。
\\	は「擦りむく」の意。	
\\	じゃあ、明日秋葉原駅の上野方面のホーム前の方で待っているね。 
\\	「~の方」をここでは
\\	を使って表現しています。	
\\	長湯し過ぎて、のぼせちゃったよ 
\\	「のぼせる」は
\\	「めまいがする」、
\\	「頭がフラフラする」で表現すればいいでしょう。	
\\	やっぱり、フランスに行ったらエッフェル塔にのぼってみないとね。 
\\	「のぼる」は
\\	で表現します。	
\\	交渉では、こちら側の要求を100%のませなければならない。 
\\	に無理矢理~を受け入れさせる)を使って「のませる」を表現しています。
\\	は「飲み込む、やむを得ず承伏する」の意。	
\\	酒は飲んでも飲まれるな。 
\\	は「適度に飲む」、
\\	「酔いすぎない限りは」の意味。	
\\	なんかこの瓶の飲み口が汚れている。なんか気持ち悪いから、替えてもらおうかな。 
\\	「飲み口」は
\\	で表現します。
\\	は「気持ちが悪い、嫌な」の意。	
\\	私って飲み込みが悪いのかもしれない。全然スノボー上達しないし。 
\\	「飲み込みが悪い」とは
\\	「学ぶのが遅い」ということですね。	
\\	彼はとてものみ込みが早い。 
\\	で「理解が早い」という意味になります。	
\\	あの人は飲み友達なんだ。 
\\	ここでは
\\	の代わりに親しみを込めて
\\	「相棒」を使っています。	
\\	あいつはつきあいが悪いから、飲みに誘っても来ないよ。 
\\	が
\\	を飲みに誘う」に当たる言い回しです。
\\	は「社交的である」の意。	
\\	俺なんかノミの心臓だから、言いたいことの10分の1も言えないんだ。 
\\	「気が弱い、小心な」が「ノミの心臓」にあたる表現です。	
\\	ここは、飲み放題ですか? 
\\	「飲み放題」は
\\	と言います。ハイフンを忘れずに!	
\\	もうじき会議始まるからこのコーヒー飲みほさねば。 
\\	は「急いで飲む(食べる)」の意味。ここでは
\\	を使ってもいいですね。	
\\	このチューハイは口当たりがいいので飲みやすい。 
\\	(舌にいい感じ)を使って「口当たりがいい」を表現しています。「チューハイ」ということなので「口当たりのよさ」を 
\\	(軽くて泡の多い)でも表現してみました。	
\\	こっちに不利になるような条件を飲むわけにはいかない。 
\\	この場合の「飲む」とは
\\	(受け入れる)ということですね。	
\\	これを飲むだけでやせられるなんてそんな都合のいい話あるか~。そんなのあるわけない! 
\\	「だけで」は
\\	で表現できますから、
\\	で「飲むだけで」の意味になります。	
\\	アメリカに来てから、すごくコーヒーを飲む量が増えました。だって、どこへ行ってもコーヒーショップが目につくし。 
\\	して以来、ずっと[以前より]たくさんのコーヒーを飲み続けている)で「飲む量が増えた」の意味を表しています。
\\	は「摂取量」、
\\	は「急上昇する」の意。	
\\	夜になると野良猫の鳴く声がうるさくて寝られやしない。 
\\	が「野良猫」に当たる表現です。	
\\	ノリの悪いコンサートだったね。 
\\	「ノリがいい・悪い」は、ロックコンサートなら
\\	「ロックする」、ジャズなら
\\	「ジャムる(即興演奏をする)」を使って表現すればいいでしょう。	
\\	9回裏、ボールに執念が乗り移ってヒットになり、試合をひっくり返した。 
\\	(ボールが行ってほしいところに飛んでいった)で「執念が乗り移る」のニュアンスを出しています。	
\\	情報時代に乗り遅れないように、私の父はコンピュータースクールに通い始めた。 
\\	を追い越す、通り過ぎる)で「乗り遅れる」を表現しています。
\\	は「チャンスを逃す」の意。	
\\	計画通りに仕事が進まないが、乗りかかった船だから最後までやり遂げるつもりだ。 
\\	「乗りかかった船」をここでは
\\	(ここまで来たんだから)で表現しています。	
\\	実は、その映画を見ることは全然乗り気でなかったのだけど、始まったらもうくぎ付けになって見ている自分に気づいた。 
\\	することにとても熱狂的である)を用いて「乗り気」を表現しています。
\\	でもいいでしょう。
\\	(引きつける、くぎ付けにする)の使い方にも注目。	
\\	考え事をしてたら、乗り越してしまった。 
\\	「乗り過ごす」とは、つまり、
\\	「降りる駅を逃す」ということですね。
\\	「考え事をする、もの思いに耽る」の意。	
\\	知らぬ間に寝込んでしまい、降りる駅を乗り過ごしてしまった。 
\\	が「降りる駅を乗り過ごす」に当たる言い回しです。	
\\	乗り継ぎが悪いと、30分、1時間は待たなければならない。 
\\	「乗り継ぎ」は
\\	と言います。複数形になっている点に注意。	
\\	彼はもうのりのりで演奏していて、観客も最高潮に盛り上がっていた。 
\\	ここでは「のりのり」を
\\	「とても興奮した調子になる」を使って表現しています。	
\\	私、乗り物酔いするから、長い時間バスに乗るのは無理よ。 
\\	「車酔いする」、
\\	「乗り物酔いする」を使えばいいでしょう。	
\\	あれ?この単語さっきのページに載ってた? 
\\	で「そのページに掲載されている」の意味。	
\\	あいつ、牛みたいなのろまなやつだから、たまにゃあ、ムチ打たないと、気合いが入らないんだ。 
\\	「のろま」を
\\	「水牛のような動き」で表現しています。
\\	とは「活発さ、元気さ」の意味ですが、皮肉を込めてそう言っているのです。
\\	とは「カバ」のことです。
\\	「むち打ってやる気を出させる」、
\\	「ゆっくりと(大股で)動く」の意。	
\\	お前は当事者じゃないから、そんなのんきなこと言ってられるんだよ。 
\\	は「きにしない、思いつきの、むとんちゃくな」の意味ですね。
\\	「関わっている」。	
\\	あの店員ってたとえ客が並んでいてもぺちゃくちゃ隣りの人としゃべってたりして、まったくのんきだよね。こっちは急いでいるというのに、まったく。 
\\	ここでは「のんき」のニュアンスを、
\\	(何も気にせずに)で出しています。
\\	は「好きなように」の意。	
\\	つる, 
\\	のんびりとした生活で、時間に縛られない生き方ができればな。 
\\	は「もっとのんびり暮らす」の意味になります。「時間に縛られる」を
\\	と表現している点にも注意。	
\\	のんべんだらりと夏が過ごせるのも今年が最後だな。来年は社会人だもんな。 
\\	で「だらだら過ごす」という意味になります。「社会人」の表現の仕方にも注目。	
\\	あなたは、携帯電話派か手帳派か。僕は、断固、手帳派だ。 
\\	ここでは「~派」を
\\	を使って表現しています。	
\\	この本を取り扱っている本屋はあまりないようで、廃刊に近いらしいよ。と言うことは、かなり貴重な本だね。 
\\	で「廃刊間近」の意味になる。	
\\	(クイズ番組などで)途中で脱落した回答者にも敗者復活戦の機会が用意してあります。 
\\	「敗者復活戦」は
\\	と言います。
\\	「競技者、参加者」、
\\	「競技、戦い」。v	
\\	海外のスーパーのレジでは、中身をすべてかごから出して精算してもらうけど、日本ではかごにはいったまま出します。 
\\	「入ったまま」を
\\	で一杯の」を使って表現しています。	
\\	有り難いことに、次から次へと難しい仕事が入ってくる。 
\\	「入ってくる」は
\\	ですね。
\\	!は「何てついてるんだ」の意味ですが、ここでは当然、皮肉っぽく言っていることになります。	
\\	佐藤さん、1番に
\\	カンパニーからお電話が入ってます。 
\\	「…が入ってます」は
\\	、あるいは、
\\	を使って表現できます。	
\\	(トイレでノックに答えて)入ってます。 
\\	の代わりに 
\\	も使えます。
\\	は「ふさがっている」の意。	
\\	彼の顔って、少し木村拓也が入ってない? 
\\	で「ちょっと... のようなところが見える」の意味になります。	
\\	小さいボトルに入りきらなかったから、余ったのを別の大きいのに入れておいたよ。(初めから大きい方に入れればよかったんだけどさ) 
\\	で「…に収まる」、
\\	は「残り」の意味。(注)後半の文は英訳していません。	
\\	何分テープで録画すれば入りきるかな。 
\\	「入りきる」を
\\	「何分のテープが必要か」で言い換えて表現しています。	
\\	彼女には他人に対する配慮がまるでない。 
\\	この場合の
\\	は「心遣い」の意味になります。
\\	で「気に掛ける」の意。	
\\	あそこの入り口にある自動販売機に、新しいドリンクが入ったんだね。もう飲んでみた? 
\\	この場合の「入る」は
\\	で表現できます。また、
\\	(現れる、登場する)も使えます。	
\\	この仕事に関しては田中さんがハウツーを持っているから聞いてみるといいよ。 
\\	「ハウツー」は
\\	(ノウハウ、実際的な知識)で表現できます。
\\	(歩くハウツー本)でもそのニュアンスは出せます。	
\\	しばらく会わないうちに、お前、生え際が随分薄くなったな。 
\\	「生え際」は
\\	と言います。
\\	は「薄くなる」の意。	
\\	"阪神の生え抜きベテラン選手をトレードに出すなんて、野村監督でなきゃできない相談だよね。 
\\	「トレードに出す」はそのまま
\\	が使えます。「生え抜き」は「阪神野球で経験を積んできた(育ってきた)ベテラン選手」、「阪神野球の血が流れているベテラン選手」という言い方をしています。
\\	とは、英語で説明すると
\\	年に1度、お盆には必ず、先祖の墓参りに行きます。 
\\	「お墓参り」を 
\\	(お墓)、
\\	(お供え物をする)を組み合わせて表現しています。	
\\	あ! 駅から何分かかるか計ろうと思ったのに忘れてた。 
\\	「時間を計る」は
\\	を動詞として使えばいいでしょう。	
\\	われわれの完敗です。全然、歯が立ちませんでした。 
\\	とは「力が(はるかに)及ばない
\\	ということです。
\\	にはとうていかなわない」。	
\\	なかなか思い通りに行かなくて歯がゆい思いをする。 
\\	で「いらいらする、もどかしい思いをする」の意味。
\\	「いらいらする」。	
\\	あのインストラクターって声も小さいし感じも暗いし、何か覇気がないんだよね。 
\\	この場合の「覇気」は 
\\	、あるいは、
\\	(活力)で表現すればいいでしょう。
\\	は「陰気な」の意。	
\\	あの缶ジュースをのんで、すぐに吐き気がした。 
\\	「吐き気がする」は
\\	と言います。	
\\	時々、吐き気がするほど肩がこるんです。 
\\	で
\\	に吐き気をもよおさせる」の意味。
\\	「吐き気」。	
\\	彼ははき捨てるようにその名前を口にした。 
\\	「ことばを吐く」、
\\	「嫌悪感を持って」の意。
\\	「汚い」。	
\\	あなたの歯ぎしりで、昨日の夜はよく眠れなかったわ。 
\\	が「歯ぎしりする」に当たる言い回しです。	
\\	身内の不手際を言及されると、彼の返答は途端に歯切れが悪くなった。 
\\	とは「不明瞭な、はっきりものが言えない」という意味。
\\	「いい加減な」。	
\\	皆様もう一度盛大な拍手をお願いします。 
\\	この後に、人の名前が来ます。
\\	「拍手」。	
\\	その選手は、薬物使用が発覚して金メダルをはく奪された。 
\\	が「…をはく奪される」に当たる言い回しです。
\\	は「発見、発覚」の意。	
\\	昨日観た映画は迫力があり、釘づけになりました。 
\\	「迫力がある」は 
\\	、あるいは、
\\	(感動的、印象的な)で表現すればいいでしょう。	
\\	歯を磨くと歯茎からかなり血が出るんだけど、歯医者に見てもらおうかな。 
\\	が「歯茎から血が出る」に当たる言い回しです。	
\\	話をはぐらかせるな。ちゃんと聞いてよ。 
\\	が「はぐらかす」に当たる動詞です。
\\	「かわす」。	
\\	昔はよく親とはぐれて迷子センターにお世話になったもんだ。 
\\	は「…から離れる」、
\\	は「…から歩いてどこかに行ってしまう」の意味です。	
\\	昨日デパートで友達とはぐれた。 
\\	「はぐれる」とは
\\	「お互いを見失う」ことですね。	
\\	私はある会社から派遣されて働いています。このほうがお金になる場合もあるの。 
\\	「派遣」は 
\\	といいます。
\\	は「派遣する」の意。	
\\	今度、
\\	カンパニーから派遣社員が来るそうです。 
\\	ここでは「派遣社員を送る」を
\\	「一緒に働く人を送ってくる」と表現しています。
\\	「臨時雇い」を使ってもいいでしょう。	
\\	悲惨な事故で夫を亡くした彼女を、長女が必死に励ました。 
\\	「励ます」は 
\\	ですね。
\\	は「心の支えになる」の意。	
\\	あの子が箱入り娘だって?冗談だろ。高校時代かなり悪かったらしいよ。 
\\	は「純真な、汚れのない」、
\\	は「世間から離された」の意味になります。
\\	「(世間から)保護されている」。
\\	に~だという印象を与える」。	
\\	本日、皆様のお陰をもちまして、ようやく開催の運びとなりました。 
\\	で「…するという段階に到着する」の意味になります。
\\	は「…を可能にする」の意味。	
\\	(すしネタの)えんがわって、あのこりこりした歯ごたえがたまらないんだよね。 
\\	で「パリパリした噛みごたえがたまらない」と英語で表現してみました。
\\	「軟骨」。	
\\	電車のドアに鞄が挟まってしまった。 
\\	「挟まる」は 
\\	で表現できます。自動詞 
\\	の使い方にも注目。	
\\	電車のドアにコート挟まれて、身動きがとれなくなった。 
\\	「挟まる」は
\\	「つかまる」で表現します。また、
\\	で閉まる」を使ってもいいでしょう。	
\\	電車に駆け込んだら、ドアに挟まれてしまった。 
\\	「挟まれる」は 
\\	で表現できます。
\\	は「動きがとれなくなる」の意。	
\\	お母さんさんなら、今、直子の髪をチョキチョキ切ってるところだよ。 
\\	「(ハサミで)チョキチョキ切る」は
\\	と言います。(注)日本語を一部書き換えました。	
\\	左端に置いといて。 
\\	「端」は
\\	、あるいは、
\\	で表現します。	
\\	こんなものはしにもぼうにもかからないじゃないか。 
\\	「はしにもぼうにもかからない」とは「使い物にならない」ということ。したがって、
\\	「まったく使えない」、
\\	「がらくた」を使って表現します。	
\\	彼は雷が鳴るとはしゃいで見にいくんです。 
\\	「はしゃぐ」とは
\\	「興奮する」ということですね。
\\	「元気にさせる」。	
\\	学生時代は冬になると、体育の時間はよく走らされたな。 
\\	「走らされる」を
\\	「走らなければならない」と言い換えています。	
\\	モーリス・グリーンは100メートルを9.79秒で走り、世界記録を樹立した。 
\\	この場合の「走る」は 
\\	、あるいは、
\\	で表現すればいいでしょう。	
\\	聞くは、一時の恥。知らぬは、一生の恥。 
\\	「恥」を動詞
\\	「恥ずかしい思いをさせる」を使って表現しています。	
\\	そのミーティングは、若さで弾けんばかりの熱気とエネルギーに満ち溢れていた。 
\\	「爆発しそうな」で「弾けんばかりの」を表現しています。
\\	「満ちあふれている状態」。	
\\	焦ってそんなことをしたら、恥の上塗りをするだけだ。 
\\	で「2倍恥ずかしい思いをする」の意味になります。
\\	は「急いで行動する」の意。	
\\	そんなことをすれば恥の上塗りになるだけだよ。 
\\	(倍恥をかく)で「恥の上塗り」を表現しています。
\\	は「陥れる」の意。	
\\	小泉首相はアメリカをはじめG8各国首脳と会談し、温室効果ガス削減を柱とする地球温暖化防止条約への同調を求めた。 
\\	この場合の「はじめ」は 
\\	、あるいは、
\\	で表現すればいいでしょう。
\\	は「地球温暖化」、
\\	は「温室効果ガス」の意。	
\\	なんとかなりそうなんだったら、初めから行けないかもって脅かすなよ。 
\\	この場合の「初めから」は
\\	「だったら」で表現すればいいでしょう。
\\	「脅かす」、
\\	「困惑させる」。	
\\	この缶の底は始めっからこんなんだったっけ? 
\\	「初めから」は
\\	でいいでしょう。	
\\	初めての方でも安心してご利用になれます。 
\\	「初めて」とは
\\	(以前に試したことがない)ということですね。	
\\	最近ゴルフ始めたんだ。でも、なかなか上達しないんだよね。向いてないのかな…。 
\\	には「(趣味などを)始める」という意味があります。
\\	は「…に(生まれつき)向いている」の意。	
\\	頼むから俺に恥をかかせるようなことだけはしないでくれよな。 
\\	の意味は「恥ずかしい思いをさせる、困らせる」です。	
\\	田中さんなら、斜向かいの家ですよ。 
\\	は「対角線の、斜向かいの」という意味の副詞。
\\	ともいいます。	
\\	あれ?彼って長野に住んでいるはずなのに、なんで東京駅にいるんだ? 
\\	で
\\	することになっている」の意味。	
\\	食事中、おならをしてしまい、恥ずかしい思いをした。 
\\	「おならをする」は
\\	「ガスを出す」で表現します。
\\	「屁をする」は少し下品な言い方です。	
\\	日本人なのに日本の文化を知らないのは恥ずかしい事だ。 
\\	この場合の「恥ずかしい」は
\\	(ばつが悪い、当惑させるような)を使えばいいでしょう。	
\\	せっかく意を決して初日にプレゼンテーションしたのに、途中で先生にとめられて恥ずかしい思いをした。 
\\	この場合の「恥ずかしい」は 
\\	(ばつが悪い)、
\\	(恥ずかしい思いをする)で表現すればいいでしょう。	
\\	あのときは、恥ずかしくてあまり話ができなくって後悔しています。 
\\	「恥ずかしい」を 
\\	を使って表現しています。	
\\	メールでは恥ずかしくなるようなことを書けるけど、いざ目の前にしたら、絶対にそんなこと言えないだろうな。 
\\	「恥ずかしくなるようなこと」は
\\	「恥ずかしい思いをさせる」を使って表現すればいいでしょう。	
\\	トークは彼の本業なのでよかったけど、ダンスとなると恥ずかしそうに下向きながら踊っていました。 
\\	は「決まりが悪い、恥ずかしい」の意味。
\\	「本職、専門」。	
\\	どうやってこの箱開けたの?開くはずないのに。 
\\	は「開けられるようには見えなかった」の意味になります。でも、開いてしまった訳ですね。	
\\	彼が強制するはずがない。 
\\	ここでは 
\\	(見ることはない)、
\\	(想像できない)で「はずがない」を表現しています。
\\	は「力ずくの、無理強いする」の意。	
\\	彼はプロジェクトから外され、たいそうご立腹で周りの人に怒りを当り散らしている。外されたのは、自分のせいでしょ!つうの。 
\\	「プロジェクトから外す」は 
\\	で表現できます。受け身にした場合は 
\\	となります。	
\\	彼は経験が足りないから、この企画から外した方がいい。 
\\	「外す」は
\\	「取り除く」を使えばいいでしょう。	
\\	単に長い距離を走ればいいというのではないはずだ。 
\\	ここでは「ないはずだ」のニュアンスを 
\\	(必ずしも…ではない)で表現しています。	
\\	せっかく早く帰れたはずだったのに、人身事故で電車が遅れ、接続列車にも乗れずさんざんだったよ。 
\\	と
\\	で「はずだった」が表現できます。
\\	でも同じような意味になります。
\\	「台無しにする」、
\\	「おじゃんにする」。	
\\	焼飯を作るはずだったのに炊飯器がなくてお鍋でご飯たいたもんだから、どうも水の量を間違えたみたいでねっとりしてしまった。 
\\	ここでは「…するはずだった」を 
\\	で表現しています。
\\	は「ねっとりした、水浸しの」の意。	
\\	昇給が3カ月先送りになるってことは、本来もらえるはずだった3カ月分がカットされたってことじゃないか。 
\\	「…するはずだった」は 
\\	で表現できます。	
\\	話がはずんで帰る時刻をとっくに過ぎてしまった。 
\\	「会話がはずんだ」は
\\	で表現できますね。	
\\	人と話をするのはいろいろ学ぶことも多いからすごく好きなのだけど、会話を弾ませるのってなかなか難しいです。 
\\	で「会話を弾ませる」の意味になります。	
\\	中華料理って、どこの国でもどこで食べても外れがなくて美味しい。たとえ、見た目古びた店でも。 
\\	(…なら間違えることはない)で「外れがない」を表現しています。
\\	でも同じようなことが言えます。
\\	は「荒れ果てた」、
\\	は「見栄えのさえない」の意。	
\\	今日の午後は雪が降るって言ってたけど、天気予報またはずれたね。 
\\	ここでは「はずれる」を
\\	(間違える)、
\\	(的はずれ)を使って表現しています。	
\\	袋の中をよくはたいてから、可燃ごみとして捨てましょう。 
\\	は「払う、はたく」、
\\	は「(ブラシ、手で)払う」の意味になります。
\\	「可燃ゴミ」。	
\\	隣でたばこなんか吸われちゃ、はた迷惑だわ。 
\\	「迷惑になる」は
\\	で表現します。また、
\\	「嫌なんだよ」でも同じ様なことが言えます。	
\\	彼は、裸一貫でその商売を初めて、今では全国に100を越すチェーン店を持っている。 
\\	で「何もないところから」という意味になります。
\\	「ゼロから」。	
\\	最近、肌寒くなってきたね。 
\\	は「薄ら寒い」の意味です。	
\\	裸足で(試しに)履いてもいいですか? 
\\	この場合の
\\	は「裸足で」の意味で副詞として使われています。	
\\	信じられないかもしれないけど、これでも若い頃ははち切れんばかりに太ってたんです。 
\\	「はち切れる」は
\\	「爆発する」、
\\	「ふくれる」を使って表現しているところがポイントです。
\\	は強調するための副詞。	
\\	彼とはどうも波長が合わない。 
\\	が「波長が合わない」に当たる言い回しです。	
\\	雨が降るんだか、晴れるんだか、はっきりしない天気だね。 
\\	は「言うのが難しい」、
\\	は「決めかねている」の意。第2文の主語が
\\	になっているところが英語らしいですね。	
\\	もうどっちかはっきりしてよ。いらいらするな~。 
\\	「どっちかはっきりする」は
\\	「心を決める」ということですね。	
\\	わが社も情報の受け手としての機能ばかりでなく、これからは情報の発信基地として有益な情報を世界に向けて発信していかなければならない。 
\\	には「情報源」という意味がありますので、これを使えばいいでしょう。
\\	「受け手」、
\\	「広める」、
\\	「配布する」。	
\\	広告を作ると費用が発生する。その費用を管理するのがこのシステムです。 
\\	この場合に「発生」は
\\	「生じる、発生する」を使えばいいでしょう。	
\\	冬は空気の乾燥により、静電気が発生して物に触るたび、バチバチ言います。 
\\	「発生する」は
\\	を使うのですが、文の流れに合わせて、
\\	(蓄積する、たまる)を使っています。	
\\	発想の転換が必要だね。 
\\	「発想の転換」は
\\	で表現できます。	
\\	彼ははったりで、ピンチを切り抜けた。 
\\	とは「ふかす、はったりをかます」のことです。	
\\	日常会話は、話題が色々な方向に発展するので本当に難しい。 
\\	「発展する」はそのまま 
\\	を使えばいいでしょう。
\\	は「変化する」の意。	
\\	私的な集まりが発展的に解消して、会社組織になった。 
\\	ここでは「発展的解消」を
\\	(解消する、解散する)と
\\	(再組織する)を組み合わせて表現しています。	
\\	必死で考えた企画は却下、一生懸命作った本は売れない。山下さんは八方塞がりの状態に陥っている。 
\\	は「どこにも出口がない状態で身動きがとれなくなっている」、
\\	「どこにも行くところがない」の意味になります。	
\\	彼女って八方美人なんだよね。 
\\	「八方美人」とは
\\	「皆を喜ばせようとする」ことですね。	
\\	私の英語の発音が悪いからなかなか伝わらない。 
\\	「発音」は
\\	ですね。「悪い」は
\\	を使います。	
\\	彼は発言しようとして何度も手を挙げたが、議長に無視されてしまった。 
\\	「発言しようとした」は
\\	でいいでしょう。
\\	「合図する」。	
\\	今朝、初霜が降りた。 
\\	「初霜」は
\\	ですね。「降りた」を
\\	で表現している点にも注意。	
\\	政府は、3種類の農産物に対しセーフガードを発動した。 
\\	「発動する」は 
\\	で表現します。
\\	は「…に関して」の意。	
\\	あの雑誌いつも毎週木曜日発売なのに、めずらしく今週は金曜日に発売してる。 
\\	「発売する」は 
\\	で表現できます。	
\\	私たちの開発チームが手がけた商品が、先週、発売開始となった。 
\\	で「発売される」の意味になります。
\\	と 
\\	を使っても「発売開始」は表現できます。	
\\	日本はお正月に初詣でに行く習慣があります。 
\\	ここでは「初詣」を文字通りに
\\	と英訳しています。
\\	は「参拝する」の意。	
\\	今日は随分派手なセーターを着てるね。 
\\	には「派手な、けばけばしい」という意味があります。	
\\	彼はいつも派手な服を着ている。 
\\	は「派手な、けばけばしい」、
\\	は「とんでもない、突飛な」の意味。	
\\	電車で長身の人の近くに立つと、頭にその人の鼻息がかかって気持ち悪い。 
\\	「鼻息がかかる」を
\\	(その人の鼻息が聞こえる)と言い換えています。	
\\	鼻を噛みたかったけど電車の中ではちょっと恥ずかしくてできないし、下を向くと鼻水がたれるし鼻風邪が一番やっかいだ。 
\\	「鼻風邪」は
\\	、または、
\\	と言います。
\\	で「鼻をかむ」の意。	
\\	彼女は華のある女優です 
\\	「魅力的な雰囲気」で「華」を表現しています。	
\\	3年ぶりに中学の時の友だちに会ったんだけど、思い出話に花が咲いて5時間も飲み屋で話していた。 
\\	は「とうとうと話す」の意味。
\\	は「会話がとても弾む」の意。	
\\	花粉症のせいで鼻はつまるし、鼻水がぐじゅぐじゅ出てくるしで、もう仕事にならないよ。 
\\	「鼻が詰まる」は
\\	と言います。上の例文では、
\\	を形容詞的に用いています。
\\	「花粉症」、
\\	「鼻水が出ている」。	
\\	さっきまで鼻が通っていたのにまたつまりだした。だから花粉症っていや。 
\\	「詰まる」は
\\	で表現できます。
\\	「花粉にアレルギーがある」。	
\\	鼻くそがつまって、気持ち悪いから、トイレでほじくってこよう。 
\\	「鼻くそ」は
\\	と言います。
\\	は「ほじる」の意。	
\\	鼻毛が出てるぞ。 
\\	(突き出る)、
\\	を組み合わせて(鼻毛が出てる)を表現しています。	
\\	あなたちょっと鼻声ね。風邪のひき始めじゃない? 
\\	で「鼻が詰まっている」の意味。
\\	は「風邪をひく」の意。	
\\	ところで話変わるけど、あのワイドショーの司会者は、実は昔、伝説の名子役と呼ばれていたんだって。 
\\	が「ところで話変わるけど」に当たる言い回しです。	
\\	彼女の話に毎日付き合うのもだんだん疲れてきたな。だって、同じ話を何回もするんだもん。こういうのって言ってあげるべき? 
\\	(彼女の話し相手になる)で「話に付き合う」を表現しています。	
\\	あー、もう話しにならない。もう言い訳なんか聞きたくないね。 
\\	は「どうにもならない」の意味です。また、
\\	で「この会話は無駄だ」と言っているのです。	
\\	おいおい、途中で話の腰を折るなよ。 
\\	「話の腰を折る」は 
\\	で表現できます。
\\	と 
\\	の組み合わせにも注目。	
\\	(タクシーでドライバーと)最初は話が弾んでいい雰囲気だったけど、そのうち話のネタがつきてしまって、シーンとしてしまった。 
\\	「話のネタ」とは
\\	「話す事柄」ということですね。
\\	「活気のある」、
\\	「いい調子で」。	
\\	彼はなかなか話の分かる人だね。 
\\	は形容詞で「物わかりがいい、理解のある」という意味。	
\\	その腕時計は、他にはない何か特殊な機能でもついているの?それなら話は別。買ってもいいよ。 
\\	ここでは
\\	「それなら話は別だ」、
\\	「それで見方が変わる」を使っています。	
\\	あいつは大げさだから話半分で聞いた方がいいよ。 
\\	「額面通りに」の使い方がポイント。
\\	「誇張する」の意。	
\\	何でそんなに怒るの?ちょっとその話を出しただけじゃん。 
\\	で「持ち出す」の意味になります。
\\	は「ちょっと言う」の意。	
\\	話せば長くなるんだけど、とにかくそれを決めたには理由はいっぱいある。 
\\	で
\\	について話すと長くなる」の意味になります。
\\	とだけ言っておく」。	
\\	の点数がいいからといって、英語が話せるとは限らない。 
\\	「話せるとは限らない」は
\\	と
\\	を組み合わせれば表現できますね。
\\	が主語になっている点に注意。	
\\	彼は英語が話せることを鼻にかけている。 
\\	ここでは「鼻にかける」を 
\\	(…をとても自慢している)で表現しています。
\\	の代わりに 
\\	も使えます。
\\	は「…で大騒ぎをする」の意。	
\\	あの人の喋り方はどうも鼻につくよね。 
\\	「鼻につく」は 
\\	(嫌な)で表現すればいいでしょう。	
\\	新郎の親戚は全員、鼻ぺちゃだった。 
\\	「鼻ぺちゃをしている」は
\\	と言います。	
\\	いよいよお花見の季節だね。 
\\	「お花見」は
\\	といいます。	
\\	鼻水が止まらない。かぜひいたかなぁ~。 
\\	で「鼻水が出る」の意味になります。	
\\	どちらかといえば、花より団子ってタイプです。 
\\	ここでは「花より団子」を
\\	を使って「きれいな物より、実用的な物の方が欲しい」と言い換えて表現しています。
\\	「見て楽しい」。	
\\	あなたと妹さんは年はどのくらい離れているのですか?かなり差があります? 
\\	この場合の「離れる」は 
\\	で表現すればいいでしょう。	
\\	つい先ほどまで、このあたりに化学薬品特有の鼻をつく刺激臭が漂っていました。 
\\	には「鼻をつんとつく」という意味があります。
\\	「鼻を刺激する」。	
\\	人の問題に鼻を突っ込むのはよしてくれよ。 
\\	が「鼻を突っ込む」に当たる言い回しです。	
\\	あいつの鼻をへし折ってやりたいよ。 
\\	で「鼻っ柱を折る」の意味になります。
\\	は「(過大評価されている人を)実力相応に評価する」の意。	
\\	人前で鼻をほじるのはやめなさい。 
\\	「鼻をほじる」は 
\\	と言います。
\\	は「みんなの前で」の意。	
\\	今回は彼に花を持たせることにしよう。 
\\	この場合の
\\	とは「名誉、功績」の意味になります。	
\\	ボードが箱に歯抜けで入れられてるから、詰めていれれば、箱がもっと有効につかえるよ。 
\\	「歯抜け」という言い方自体あまり頻度が高くないように思うのですが、ここでは「隙間
\\	を使って、その状態を説明してみました。
\\	「ぴったりと」。	
\\	朝、髪とかす時間もなかったの? 後ろの髪がはねてるよ。 
\\	この場合の「(髪の毛が)はねる」は
\\	でいいでしょう。	
\\	今週末は温泉にでも行って、羽を伸ばすか。 
\\	は「くつろぐ、のんびりする」という意味です。
\\	は「リラックスする」の意。	
\\	"アイススケートはスキーと同じような感覚と言われたけど、難しくてまっすぐ立てません。ハの字になってしまう。 
\\	英語にはカタカナのハはありませんので、ここでは
\\	(くさび形)を使って表現しています。
\\	この人がもうちょっと詰めてくれれば、座れるのにな。自分が幅を取ってるってことを気づいてないのかな。 
\\	「幅を取る」とは 
\\	(スペースを取る)ということですね。
\\	は「詰める」の意。
\\	の代わりに 
\\	も使えます。	
\\	今の世の中には悪がはびこっている。 
\\	には「はびこる、流行する」という意味があります。
\\	は「…で一杯である」の意。	
\\	近頃、彼、羽振りがいいね。 
\\	「羽振りがいい」は
\\	を使えばいいでしょう。	
\\	彼女は今釣りにはまっている。 
\\	で
\\	に熱中している」の意。
\\	にも同じ様な意味があります。	
\\	クローゼットから少し洋服のはしがはみ出でてると、赤ちゃんってそれを引っ張り出したくなるの。 
\\	(…から突き出る)で「はみ出す」を表現しています。	
\\	バス停でいつもと同じ時間にバスを待っていたのに、素通りされて歩いてくる羽目になった。 
\\	(…をせざるを得ない)を使って「羽目になった」を表現しています。
\\	は「…する以外に選択の余地がない」の意。	
\\	いつも金曜日の夜は羽目をはずして、飲み屋をはしごしてしまう。 
\\	が「羽目をはずす」に当たる表現です。
\\	は「はしご酒をして度を越す」、
\\	は「はしご酒」の意味。	
\\	あの2人組みのコーラスグループは、はもりが最高! 
\\	「はもり」とは
\\	のことですね。	
\\	外で食べていっちゃおうか。夕食にはまだ早いけど。 
\\	で
\\	には少し早いけど」の意味になります。	
\\	参議院も早い話が養老院だよね。 
\\	は「簡単に言うと」、
\\	を述べる一番の(手っ取り早い)方法は」の意。「参議院」は
\\	と言います。	
\\	早い者勝ちだから、早くいこうよ! 
\\	「早い者勝ち」には
\\	という決まった言い方があります。	
\\	この歌興味ないから、早送りして飛ばしちゃおうよ。 
\\	が「早送りする」に当たる表現です。	
\\	早合点しないでくれ。彼女とは食事をしただけで、付き合っているわけではない。 
\\	は「すぐに結論を出す、即断する」という意味です。「付き合っている」は
\\	もしくは、
\\	で表現します。	
\\	(レジ等で)すみませんが急いでいるので早くしてもらえませんか。 
\\	「早くする」は
\\	を使って表現すればいいでしょう。	
\\	お前、早口だから、何言ってるんだか全然分からないよ。 
\\	「早口」とは
\\	「早く話す」ということですね。	
\\	早とちりをするな。人の話をちゃんと聞けよ。 
\\	は「早合点する」の意味。
\\	!「そんなに慌てるなよ」でも「早とちり」のニュアンスが表現できます。	
\\	早とちりするなよ。俺と彼女がつき合ってるわけがないだろ。 
\\	「はやとちりする」を
\\	「先に進む」を使って表現しています。
\\	も使えますね。	
\\	就職協定廃止後、就職活動開始の時期がどんどん早まっている。 
\\	で「始まるのがどんどん早くなる」の意味になります。
\\	は「ゆっくり進む」の意。	
\\	こういう時にこそ、早め早めと手を打つ必要がある。 
\\	で「できるだけ早く物事を片づける」の意味になります。
\\	は「迅速」の意。	
\\	最近はどんなものがはやってるの? 
\\	で「はやっている」の意味。
\\	「流行の(形容詞)」。	
\\	彼は腹が据わっているから、何が起きても動じない。 
\\	「腹が据わる」は 
\\	でもいいのですが、ここでは 
\\	(冷静である)を使って表現しています。
\\	は「落ち着き払っている」の意。	
\\	腹が減っては戦はできぬっていうだろ。まず腹ごしらえをしておかないとね。 
\\	の文字通りの意味は「お腹が一杯だからこそ軍隊は行進する」。	
\\	あんなに腹黒い奴、これまでに会ったことがない。 
\\	で「性質が悪い」、
\\	「悪意」。	
\\	一生懸命やってきたのに、あんなに文句ばかり言われては、腹の虫が治まらない。 
\\	には「慰めることができない」という意味があります。
\\	「慰める」。	
\\	健康のためには腹八分目が一番だよ。 
\\	「腹八分目」とは、
\\	「ほどほどに食べる」ということですね。	
\\	今国会は、波乱含みの様相を呈している。 
\\	は「騒然とした(動乱の)時」、
\\	は「どたばた騒ぎ(混乱)」の意味。	
\\	肩こりがひどいので鍼でもしてもらおうかな。 
\\	で「鍼治療を受ける」という意味。
\\	「鍼灸師」。	
\\	鍼
\\	同い年だからってだけで、すぐ張り合おうとするなんて子供っぽいですよ 
\\	ここでは
\\	「子供っぽい」ではなく
\\	「大人になってない」を使っています。「張り合う」は
\\	「競争する」でいいでしょう。
\\	と対抗する」の意。	
\\	全身筋肉痛だよ。 彼女にいい所を見せようとしてちょっと昨日のサッカーでハリキリすぎたかな。 
\\	で「度を超す」という意味になります。	
\\	今日も一日、張り切って行きましょう。 
\\	とは「元気一杯の」の意味になります。
\\	「やる気」。	
\\	長い間お会いできず胸が張り裂ける思いです。 
\\	で「胸が張り裂ける思い(こと)」という意味になります。	
\\	高級車でスピード出すと車体に重みがかかって地面を張り付くように走るけど、安い車はその反対でまるで宙に浮いているように感じる。 
\\	で「道路の中に引き込まれるような」の意味になります。
\\	は「浮き上がる」、
\\	は「…に張り付く」の意味。	
\\	はるかかなたに見える地平線までがおれの土地だ。 
\\	「はるかかなたに見える」を
\\	(見渡す限り)で表現しています。	
\\	春っぽい髪型(服装)ですね。 
\\	「春っぽい」を
\\	「とても春らしい雰囲気を持っている」で表現しています。	
\\	はれぼったい目をしてるね。昨日、飲んで遅かったのかい? 
\\	には「はれた」という意味があります。
\\	(分詞)の使い方にも注目。	
\\	あの地震でいくつかの建物は半壊したけど、負傷者が出なかったのは何よりだと思う。 
\\	で「部分的に壊れる」の意味になります。	
\\	その前例のない事件では、犯人が事前に犯行声明文を新聞社に送りつけていたそうだ。 
\\	の文字通りの意味は「告白の手紙」です。
\\	「前代未聞の」、
\\	「公表する」。	
\\	彼女の話を聞いた時、僕は半信半疑だった。 
\\	「半信半疑」とは
\\	「完全に信じた、納得した訳ではない」ということですね。
\\	を使って日本語に近い言い方もできます。	
\\	その点は日頃から反省してます。 
\\	上の最初の文の意味は「どうしてそなったのか、また、二度とそうならないようにずっと考えてきました」となります。
\\	「じっくり考える」、
\\	「自己反省」の意。	
\\	余計なことを言うべきではなかったと、今更ながら反省しています。 
\\	「反省する」は 
\\	(後悔する)で表現すればいいでしょう。	
\\	ゴミ工場のための追い出し反対! 
\\	で「絶対反対」、
\\	で「彼らに…させるな」の意味になります。
\\	は「ゴミ消却所」の意。	
\\	タクシーで値切った事はないけど、反対にぼられそうになったことはある。 
\\	この場合の「反対に」は
\\	「その一方で」を表現すればいいでしょう。
\\	「値切る」、
\\	「だます」、
\\	「法外な金額を取られる」も一緒に覚えておきましょう。	
\\	私、最近異様に食べちゃうんだけど、これってダイエットの反動だと思う。 
\\	「反動」は 
\\	で表現できます。
\\	は「むちゃくちゃに」の意。	
\\	これをやった犯人は、田中君だね? 
\\	で「それをやった人」の意味。映画
\\	の中でも「救世主」の意味で
\\	が使われていましたね。	
\\	今日の給食はハンバーグと野菜サラダでした。 
\\	とは「挽肉を丸く平らにした料理」のことです。冠詞のない
\\	(パンなし)でもいいでしょう。	
\\	私は18歳から22歳まで着物の販売員をしていました。 
\\	「販売員」は
\\	と言います。	
\\	海外生活を決めたのは、半端な気持ちでじゃないよ。こう見えても、いろいろ先を考えているんだから。苦労するのは百も承知してる。 
\\	は「生焼けの、不完全な」の意味。
\\	「悪ふざけで」。	
\\	彼の飲みっぷりといったらはんぱじゃない。 
\\	とは「君の知っている普通の酒飲みじゃない(以上)」の意味になります。また、
\\	には「酒を飲む」の意味がありますが、
\\	で「ガンガン飲む」の意味を表します。	
\\	もう半分ぐらい歩いたかな。すでに疲れて無事に家にたどり着けるかどうか。 
\\	この場合の「半分」は 
\\	で表現できます。
\\	は「家に着く」の意。	
\\	今まで見たなかで一番泣けた映画かも。話の半分いかないうちに泣けてきたもん。終わりまでに5カ所は泣けるシーンがあった。 
\\	で「半分終わらないうちに」の意味になります。
\\	は「涙する」の意。	
\\	緊張して、言いたいことの半分も言えなかった。 
\\	と
\\	「出てくる」の使い方がポイントですね。	
\\	その法律は、便利な半面もあるが、 (拡大解釈すると)危険な半面もある。 
\\	「面」は
\\	を使って表現します。
\\	とは「何事にも両面がある(いい面と悪い面がある)」の意。
\\	「拡大解釈する」。	
\\	卒業できるのはうれしい反
\\	面、今日で皆とお別れだと思うと淋しいです。 
\\	この場合の「反面」は、
\\	(同時に)、
\\	(その一方で)を使って表現すればいいでしょう。	
\\	僕は、酒とギャンブルに溺れているが、決してまねをしちゃいけないよ。ある意味で反面教師と思ってほしい。 
\\	で「やってはいけないことの例として役に立つ」の意味になる。	
\\	今はそんなこと言っている場合ではない。直ちに解決方法を考えないと大変なことになる。 
\\	が「…する場合ではない」に当たる表現です。	
\\	その時の場合と状況によってどうするか決めます。 
\\	文字どおりに訳すと 
\\	になります。
\\	は「その時は、その時だ(その時のことはその時になってから考える)」の意味。	
\\	人による場所によるって、そんなこと言ってたらきりがないよ。それじゃ、全部場合によるって答えになってしまうよ。 
\\	「場合による」は
\\	(…次第である)で表現できます。	
\\	傷口を消毒しないとばい菌が入っちゃうよ。 
\\	は「感染する」、
\\	は「消毒薬」の意。	
\\	最近バカ売れしている本があるらしくて、昨日テレビで取り上げられていた。 
\\	で「飛ぶように売れている本」というニュアンスになります。
\\	「特集する」。	
\\	ね?ばかげてるでしょ? 
\\	「ばかげてる」は 
\\	で表現すればいいでしょう。	
\\	生活のために仕事をしている自分が、ばかに思えてきた。なんで、好きでもない仕事を我慢してまで働かないといけないのだろう。 
\\	「ばかに思える」はそのまま 
\\	と言えばいいでしょう。
\\	は「…を耐える」の意味。	
\\	何なのその態度は?人をばかにしているとしか思えない。 
\\	は「あざける、ばかにする」、
\\	は「人を嘲弄するかのように振る舞う」の意味。	
\\	もっとうちの会社駅から近くならないかな。 通勤時間もばかにならないよ。 
\\	「軽視できないこと」で「ばかにならない」というニュアンスが出せます。
\\	「どんどん増える」。	
\\	馬鹿の一つ覚えのように、インターネット、インターネットって言うなよ。 
\\	ここでは「馬鹿の一つ覚え」を
\\	「そのことしか言わない」、
\\	について繰り返す」と言い換えています。	
\\	ばかばかしい 
\\	「ばかばかしい」は
\\	で表現します。	
\\	今考えると自分でもばかみたいだと思うけど、弟とはよくテレビのチャンネル争いでけんかしてたの。 
\\	「ばかみたい」を 
\\	と 
\\	(ばかばかしい)を組み合わせて表現しています。	
\\	お金は使うものだし、ただ貯めるだけではばからしい。確かに貯めるのは必要だと思うけど、やっぱり使わなきゃ。 
\\	ここでは「ばからしい」を
\\	(意味がない)、
\\	(愚かな)で表現しています。	
\\	たまに、なんでこんな興味がない仕事を続けているんだろうと思って、生活のためだけに働いていることがばからしくなる。 
\\	ここでは「ばからしい」を 
\\	(ばかばかしい)、
\\	(ばかげた)で表現しています。	
\\	頭の中は試験のことでいっぱいで、ふと気づくとその事ばかり考えている。 
\\	「ばかり考えている」を
\\	「他のことは考えられない」で表現しています。	
\\	予想が当たったからといって、喜んでばかりいられない。 
\\	(喜ぶという理由にはならない)で「ばかりいられない」を表現しています。
\\	でもそのニュアンスは出せます。	
\\	結婚したばかりの頃は優しかったのに。愛情が冷めてきたね。 
\\	と 
\\	(新婚夫婦)を使って「結婚したばかりの頃」を表現しています。	
\\	私って、誰もださないエラーとかバグとかだしちゃうんだよね。 
\\	「バグ」はそのまま
\\	が使えます。
\\	損害をこうむる、苦しむ」。	
\\	彼女舞台に上がると化けるよね(変身するよね)。 
\\	「化ける」とは
\\	「変身する」ということですね。
\\	「役になりきる」の意。	
\\	風で髪がバサバサになっちゃった。 
\\	を使って髪が乱れている状態を表現しています。
\\	で「乱す、乱雑にする」の意。	
\\	オーブンレンジとガスコンロが一緒になっていれば、そんなに場所もとらないし、便利なのにな。 
\\	で「多くのスペースを残しておく」を使って「場所を取らない」の意味を表しています。
\\	は「節約する」の意。	
\\	永福町のあの居酒屋って、場末だよね~。 
\\	は「みすばらしい」、
\\	は「安酒場」、
\\	は「いかがわしい場所」の意味になります。	
\\	今日は一日中ばたばたしてて、本を読むゆとりなどありませんでした。 
\\	「一日中ばたばたする」のニュアンスを 
\\	(一日をとてもあわただしく過ごす)で表現しています。	
\\	ざまあみろ。悪い事ばかりしてるから、罰があたたんだよ。 
\\	この場合の
\\	は
\\	の罰を受ける」の意味になります。
\\	「付きまとう、悩ます」。	
\\	こんな高いレストランだとは思わなかった。何か場違いな所へ来ちゃったって感じ。 
\\	「場違い」は
\\	「間違えた場所」で表現できます。	
\\	最近、警察の不祥事のニュースばっかりだ。 
\\	ここでは「ばっかり」を
\\	ばかりを耳にする」を使って表現しています。	
\\	大切なデータはいつでもバックアップを取るようにしている。 
\\	「バックアップ」はそのまま
\\	が動詞として使えます。	
\\	今日の講義ってつまんないよね。ばっくれよっか。 
\\	には「さぼる」という意味があります。	
\\	バッチ・グーです。 
\\	「バッチ・グー」は
\\	でもいいのですが、ここでは
\\	「いい、すてき、かっこいい」を使ったみました。	
\\	あなた私がばついちだって知っててプロポーズしてるの? 
\\	「ばついち」とは、
\\	「最初の結婚ではない」、
\\	「一度離婚したことがある」ということですね。	
\\	彼女、足が長くて、スタイル抜群よね。 
\\	この場合の「抜群」は
\\	でいいでしょう。
\\	「容姿、スタイル」。	
\\	バナナのたたき売りじゃあるまいし、あれも、これもただでやるといっちゃいかん! 
\\	英語圏にはバナナのたたき売りはありませんので、ここでは
\\	「(火事の焼け残り)処分特売」、
\\	「(デパート地下の)特売場」を使ってそのニュアンスを出してみました。	
\\	こんな婆くさい洋服恥ずかしくて着れないよ。 
\\	は「時代遅れの、薄汚い」の意味。これと 
\\	を使って「婆くさい」のニュアンスを出しています。
\\	は「老婆」の意。	
\\	結局、僕がばばをつかまされたわけだ。 
\\	で「外れくじを引く」という意味になります。
\\	「貧乏くじを引く」。	
\\	彼女たちの歌ってバラバラで全然合ってないね。 
\\	とは「ごちゃごちゃになっている」という意味です。	
\\	ばれたか!実は、経理の田中さんと付き合ってるんだ。 
\\	ここでは「ばれたか」を
\\	「分かっちゃったか」を使って表現しています。
\\	は「万事休す」の意味。	
\\	浮気をするならばれないようにやってほしい。そして、一生ばらさないでほしい。 
\\	で「明るみに出る」の意味になります。
\\	は「浮気をする」の意。	
\\	仕事中に眠いときは、ばれないように寝るのが一番。結構、スリルがあるよ。 
\\	で「人に見つからないように」の意味。
\\	「気づく」。	
\\	大丈夫だよ、ばれやしないって。 
\\	「ばれはしない」とは 
\\	(誰にもわからない)ということですね。
\\	は「知れてしまう」の意。	
\\	私って嘘ついてもすぐ顔に表れるらしいので、すぐにばれるんだよね。 
\\	「顔に表れるのでばれる」を 
\\	(顔中に書いてある)で表現しています。	
\\	今回のワールドカップでは番狂わせが多かったね。 
\\	「番狂わせ」は 
\\	で表現できます。
\\	も使えます。	
\\	「日本語でくらそう」って、あの日本人が外国人に日本語を教える番組? 
\\	「番組」は
\\	と言います。
\\	の使い方に注意。	
\\	その書類、番号順にならべてくれる。 
\\	「並べる」は
\\	「揃える」を使って表現します。「番号順」は、状況に応じて、
\\	「ページ毎に」などを使うようにします。	
\\	今回の出版記念パーティーには、万障お繰り合わせの上、ご出席ください。 
\\	これが英語の「万障お繰り合わせの上 
\\	に該当する言い回しです。	
\\	インターネットといえども万能ではない。インターネットではできないことはいくらでもある。 
\\	は「全能の」という意味。
\\	~で「…に~するこを可能にする」の意。	
\\	あのスクリーンから2番目の席です。(飛行機内で) 
\\	「~から2番目の席」は
\\	と言います。
\\	とも言えます。	
\\	この部屋、日当たりはいいけど、少し狭いよね。 
\\	「日当たりがいい」は
\\	で表現できます。
\\	も一緒に覚えておくといいでしょう。
\\	(狭苦しい)も知っておこう。	
\\	そんなことやったら、結局ひいきの引き倒しになっちゃうぞ。 
\\	「ひいきの引き倒し」とは「かえって迷惑をかける」ということですから、
\\	(益より害を与えている)で表現すればいいでしょう。	
\\	どうひいき目にみても、あまりいいできではないね。 
\\	「好意的に見ようとする」を使えば「ひいき目」に近い意味が表現できます。	
\\	冬になると手や足がすぐ冷えるの。 私って冷え性なんだ。 
\\	「冷え性」とは
\\	「寒さに敏感である」ということですね。
\\	「冷える(動詞)」。	
\\	すみません、控えはいただけないのですか? 
\\	何の控えかはっきりしませんので、ここでは 
\\	(写し、控え)、
\\	(領収書)を使っています。	
\\	千葉県では知事選に突入したばかりだが、7月にはもう次の選挙戦が控えている。 
\\	「控えている」とは
\\	(もうすぐやってくる)ということですね。
\\	は「間近になっている」の意。	
\\	もう若くはないんだから、少しはお酒を控えたら。 
\\	は「遠ざける、避ける」という意味です。
\\	は「減らす、控える」の意。	
\\	彼女は、そこそこかわいいという程度だが、昨日会ったときは、光の加減できれいに見えた。 
\\	「光の加減」のニュアンスを 
\\	で表現しています。	
\\	ビタミンC入りとか何か健康にいい事が書いてあったりすると、その言葉に引かれて買っちゃうんだよ。 
\\	で
\\	に引きつけられる」の意味を表します。
\\	とは「だまされやすい人」のことです。	
\\	あなたには物事を悲観的に見る傾向があるわね。 
\\	「悲観的な(に)」は
\\	で表現します。
\\	「悲観論者」。	
\\	日帰りで京都に行ってみない? 
\\	「日帰り旅行」は
\\	と言います。	
\\	12月になり、日が落ちるのが早くなった。 
\\	が「日が落ちるのが早い」に当たる言い回しです。	
\\	そんなにちんたら仕事をしていたら、日が暮れちゃうぞ。 
\\	で「日が暮れる前に」の意味。	
\\	ワールドカップが近いから、またサッカー人気に火がつくかな。日本人ってイベント好きだからさ。 
\\	「火がつく」とは 
\\	(さらに人気が出る)ということですね。
\\	(引き金となる、引き起こす)の使い方にも注目!	
\\	このハンバーグ、中まで火が通っていないじゃないか。 
\\	は「生焼けの」、
\\	は「ちゃんと料理されていない」の意味。	
\\	ひがみごとって聞くほうも辛いんだよねえ。わかっちゃいるんだけどね。 
\\	「ひがみごと」を
\\	「ゆがんだ不平」で表現しています。
\\	「不当に扱われている」、
\\	「利用されている」の意。	
\\	この店の日替わりメニューを教えてください。 
\\	「日替わりメニュー」は 
\\	や 
\\	でいいでしょう。	
\\	何を食べようかな?みんな、日替わりランチでいいかな? 
\\	「日替わりランチ」は 
\\	でいいでしょう。
\\	とも言えますね。	
\\	悲願の金メダル獲得。ヤワラちゃんの表彰式での笑顔がとても印象的だった。 
\\	とは「長い間抱いていた願い」のことです。
\\	「強い印象を与える」。	
\\	きみがそんなことを言うんだったら、我々はこのプロジェクトから引き上げる。 
\\	(歩いて去る)、
\\	(離れる)で「引き上げる」を表現しています。	
\\	(食堂で)トレイに載せたお皿の分の料金が(ミール)カードから自動的に引き落とされます。 
\\	で「…から引き落とされる」の意味になります。
\\	は「差し引く」の意。	
\\	観光客が現地の人たちの写真を勝手に撮ったことが引き金になって、暴動が起きた。 
\\	には「引き金になる、引き起こす」という意味があります。	
\\	ドラマを見ていると、時々自分もその話の中に引き込まれたりするんだよね。なん か、主人公になった気になったり。 
\\	で「引き込まれる」、
\\	で「吸い込まれる」の意味。
\\	になりきる」。	
\\	最近ひきこもりの人が増えているようだ。 
\\	「ひきこもり」は
\\	「家で孤独に暮らす」で表現できます。
\\	「引っ込む」。	
\\	前を見ろ!いつまでも過去を引きずっていてはいけないよ! 
\\	「引きずる」は
\\	「持ち歩く」で表現すればいいでしょう。
\\	「前向きの」。	
\\	あの完璧9頭身のロレナと2人で歩くなんて、私ただの引き立て役になっちゃいそうで、怖いよ。 
\\	の文字通りの意味は「比較することで彼女をよりよく見せる」です。ここでは「9頭身」を
\\	(均整の取れた)で表現しています。	
\\	塩味が豆の甘味を引き立てます。 
\\	には「引き出す」という意味があります。	
\\	芸人は「引き出し」が多いほうがいいよね。 
\\	「引出」を
\\	「商品目録」、
\\	「在庫」を使って表現しています。	
\\	引継ぎの期間が必要なので、辞める場合は最低3ヵ月前には会社に申し出ないといけません。 
\\	とは「円滑に移行すること」の意。第2文では、
\\	で更に詳しく「引き継ぎ」を説明しています。
\\	「知らせる」。	
\\	引き続き、わたしは研究の推進に努めます。 
\\	「引き続き…する」は 
\\	で表現すればいいでしょう。	
\\	給料を上げろ、と大勢の社員に迫られて、さすがの社長もひきつり笑いをしていた。 
\\	「神経質な」、
\\	「ピクピク引きつる」の意。この形容詞を使えば「引きつったような笑い」が表現できます。動詞表現としては、
\\	のパターンを活用します。
\\	とは「詳しい説明を求める」の意味。	
\\	緊張のあまり、自分でも顔がひきつってるのが分かった。 
\\	「ひきつる、ピクピクする」は
\\	で表現できます。ちなみに、「頬がひきつる」は、
\\	と言います。	
\\	その会社員はひき逃げ犯として、今朝、逮捕された。 
\\	「ひき逃げ」は
\\	と言います。
\\	の容疑で」。	
\\	風邪は引きはじめが肝心だから、気をつけて。 
\\	ここでは「引き始め」を
\\	「風邪を引きそうだなと感じ始めたら」と言い換えて表現しています。	
\\	内容をわかり易くするために、1つ2つ卑近な例をあげよう。 
\\	「卑近な」は 
\\	(身近な)を使えばいいでしょう。
\\	は「…の言い方で」の意。	
\\	あんまり熱心にやられると、ひいちゃうよね。 
\\	この場合の「ひく」は
\\	「後ずさりする」で表現できます。	
\\	英語の勉強は楽しみながらやらないと続かないよ。これが、長続きする秘訣だと思います。 
\\	「秘訣」は
\\	を使えばいいでしょう。
\\	「ねばり強い、持続する」、
\\	をやり通す」。	
\\	あごのほうにひげのそり残しがあるよ。 
\\	で「ひげを少しそり残す」の意味になります。	
\\	久しぶりだね。2年ぶりくらいかな。 
\\	「久しぶり」を
\\	「もう随分経つよね」で表現しています。	
\\	久しぶりに買い物にでも行きたいな。 
\\	で「長い間買い物に出かけていない」の意味になります。	
\\	久々に聞いたこの曲。なつかしい。 
\\	「久々に聞く」をここでは
\\	と動名詞で表現しています。ここで
\\	ではなく
\\	を使ったのは、その曲が自然に耳に入って来たからです。	
\\	吊革につかまりながら寝てしまい、膝がカクッとなってしまった。 
\\	「カクッとなる」は
\\	「力が尽きる」で表現すればいいでしょう。	
\\	少し走っただけで膝がガクガクしてるよ。 
\\	は「ぐらぐらする、よろよろする」の意味です。	
\\	階段を一気に駆け上がっただけで、膝が笑ってるよ。 
\\	は「ぐらぐらする」、
\\	は「ぐらつく」の意味。
\\	は使えません。	
\\	機会があったら、きみとひざを交えて話したいと思っていたんだ。 
\\	と 
\\	(有意義な話)を組み合わせて「ひざを交えて話す」を表現しています。	
\\	今野球の3チームが、1ゲーム差でひしめきあっている。 
\\	は「肩を並べて、互角で」、
\\	は「死闘を演じる」の意。	
\\	彼女、被写体としては最高なんだけど、写真嫌いだからな。 
\\	「とても写真写りがいい」で「被写体としては最高」を表現しています。
\\	「被写体」。
\\	「写真嫌い」。	
\\	非常識なことばかり言っていると、皆から相手にされなくなるぞ。 
\\	で「理にかなったことを言う」の意味になります。これに
\\	を付けて「非常識なことを言う」を表現しています。	
\\	そんなところで何をひそひそ話ししてるんだよ。 
\\	は「囁く」の意味ですね。ここでは
\\	「うわさ話をする」を使ってもいいでしょう。	
\\	しばらく会わない間に、ずいぶん額が広くなったね。 
\\	「生え際」と
\\	「後退する」を使って「額が広くなる」を表現しています。
\\	を使った言い方もあります。	
\\	このギター、左利きの人用なんだ。 
\\	「左利き」は
\\	、「右利き」は
\\	と言います。	
\\	僕は左党だから甘いものは嫌いだ。 
\\	「左党」は「酒飲み」のことですね。これは簡単に
\\	と言えばいいでしょう。ちなみに、「酒飲みは甘い物を好まない」という傾向は、アメリカであまりないようです。お酒も飲むし、甘い物も食べるアメリカ人はたくさんいます。	
\\	山田さんの左斜め向かいの席の人って誰? 
\\	「左斜め向かい」は
\\	「斜めに」と
\\	を組み合わせて表現します。
\\	「斜め向かいの」。	
\\	左ハンドルの車は、乗ってみると、意外に運転しやすい。 
\\	「左ハンドルの車」は
\\	、あるいは、
\\	と言います。
\\	は使いませんので要注意。	
\\	この一帯、バブルのころは羽振りよかったが、今じゃ軒並み左前。 
\\	「左前」とは「景気が悪くなる」ことですから、
\\	「下り坂になる」や
\\	「お金に困っている」で表現すればいいでしょう。ちなみに、「バブル経済」は
\\	と言います。	
\\	喉に魚の小骨がひっかかってなかな取れないんだ。 
\\	がひっかかる」は
\\	を使えばいいでしょう。また、
\\	「つかえる」、
\\	「ちょっと動く」も一緒に覚えてくだい。	
\\	その事について、ちょっとひっかかっている事(気になる事)があるんだよね。 
\\	とは「おさまりが悪い」の意。
\\	「飲み込む、鵜呑みにする」。	
\\	猫と遊んでたら、顔をひっかかれちゃったんだ。見てよ、この傷。 
\\	「ひっかく」は
\\	ですね。この場合の「傷」は
\\	でいいでしょう。	
\\	入学試験のその問題は、ひっかけ問題だった。 
\\	「ひっかけ問題」は
\\	といいます。	
\\	新製品に対するクレームの電話がひっきりなしにかかってきた。 
\\	で「…が次々来る」の意味になります。
\\	は「…が殺到する」の意。	
\\	ここ数日、引越しの準備で忙しい日が続いています。 
\\	「引っ越す」は
\\	で表現します。
\\	は動詞、名詞両方に使えます。	
\\	彼は引っ込み思案だから、人前で話すのが苦手なんだ。 
\\	「恥ずかしがり屋の」と
\\	「内気な」を使って「引っ込み思案」を表現しています。	
\\	あの頃は、司法試験に合格するために必死で勉強していました。 
\\	は
\\	するために必死にもがいていた」という意味。	
\\	その家族はひっそりと小さなアパートで暮らしていた。 
\\	「ひっそりと」は
\\	「静かに」で表現すればいいでしょう。	
\\	最近、この辺りでひったくり事件が多発していますので、自転車のかごにはひったくり防止ネットをつけましょう。 
\\	とは「財布、ハンドバッグなどのひったくり」のことです。「多発」と「防止ネット」の表現の仕方にも注目。	
\\	母親に嘘をついたことがばれて、ひっぱたかれた。 
\\	とは「(お尻などを)平手で叩くこと」です。
\\	は「思い切りぶつ、打ちのめす」の意。	
\\	金メダルを取った高橋尚子選手はどこでも引っ張りだこだ。 
\\	「引っ張りだこ」とは、
\\	(とても人気がある)ということですね。
\\	(引っ張りだこ)ともいえます。	
\\	これは、外出するには必需品でしょ。 
\\	「必需品」は
\\	「なくてはならない物」で表現できますね。	
\\	この書類に必要事項を記入して、明日の朝までに提出してください。 
\\	「必要事項」は
\\	と
\\	を組み合わせればいいでしょう。ここでは
\\	「記入する」と
\\	「提出する」も一緒に覚えるようにしましょう。	
\\	急に待ち合わせの場所とか時間が変更になったり、急用ができたりした時は、携帯電話の必要性をすっごく感じる。 
\\	で「どれだけ携帯電話が必要か」の意味。
\\	「緊急時」。	
\\	パスワードを変更する必要はありません。 
\\	「必要ない」は
\\	で表現できます。ここでは「名前」の代わりに
\\	を使いました。	
\\	私はそれに関しては否定しないよ。確かに私は衝動買いするけど、それが悪い?それによって、何かあなたに迷惑かけてます? 
\\	「否定する」は 
\\	でいいですね。
\\	は「…を衝動買いする」の意。	
\\	今夜は友達と飲む約束がありますので、一足お先に失礼します。 
\\	で
\\	よりも少し前に去る」という意味になります。	
\\	今回のギネス・パルトロウの演技、いつもとひと味違うね。 
\\	日本語を少し変えて英語にしてみました。「ひと味違う」を
\\	で表現しています。
\\	「少し、少々」。	
\\	今夜のビーフシチューは一味違うわよ。2日がかりで作ったんだから。 
\\	ここでは「一味違う」を
\\	「ちょっと特別な味がする」で表現しています。	
\\	彼は、腰が低くくて人当たりがいいよね。 
\\	「気持ちのよい」、
\\	「控えめな」、
\\	「でしゃばらない」の意。	
\\	なんかふっと一息つくと、いろんな事を考えてしまう。 
\\	とは「一休み、一息」のことです。	
\\	英会話の先生が来月引っ越すのですが、今までよりひと駅手前になり、しかも駅から今より近くなるので私には好都合です。 
\\	と 
\\	を組み合わせて「ひと駅手前」を表現しています。
\\	とも言えますね。	
\\	あいつアメリカに行って、一皮むけて帰って来たよな。 
\\	「一皮むける」とは
\\	「より成熟する」、
\\	「成長する」ということですね。	
\\	人がいいから営業にはむかないかもね。 
\\	「人がいい」は
\\	で簡単に表現できます。
\\	は「~になるには、あまりに人がよ過ぎる」の意味。	
\\	名の知られた温泉は、どこも人が多くて、ゆっくり温泉に浸かっていられない。 
\\	「人が多い」は
\\	「混雑している」で表現すればいいでしょう。
\\	「浸かる」。	
\\	あいつ酔うと人が変わるから一緒に飲みたくないんだよ。 
\\	で「別人のようだ」の意味になります。	
\\	この料理、結構いけるよ。一口食べてみる? 
\\	で「一口、一噛み」の意味になります。	
\\	そんな事いわれても、こっちだって英語でメール書くのってひと苦労なんだから。たまには、自分でやってよ。 
\\	ここでは
\\	(骨折り)を使って「ひと苦労」を表現しています。
\\	で「困っている」の意。	
\\	秋は人恋しい季節です。 
\\	で「仲間が欲しいと思う」という意味になります。	
\\	お前はいつも一言多いんだよ。 
\\	で「一言多く言う」の意味になります。
\\	で「少し度が過ぎる」の意。	
\\	ひとごとだと思って言いたいこと言うなよな。 
\\	で「(その人の)立場にない」の意味になります。	
\\	野党は、予算審議を人質にとって、証人喚問を要求した。 
\\	(人質にとる)をこの文脈でそのまま使うことができます。
\\	は「乗っ取る」の意。	
\\	あいつ彼女一筋だったのに、振られちゃったんだ。 
\\	を熱愛する」で「一筋」表現しています。
\\	「捨てる」。	
\\	彼は偏屈者だから、同意を取り付けるのは一筋縄ではいかない。 
\\	ここでは「一筋縄ではいかない」を 
\\	(簡単な仕事ではない)で表現しています。	
\\	お酒の好みは人それぞれですからね。 
\\	は「好きな飲み物」の意。
\\	「人それぞれ」。	
\\	今回のプロジェクトが終わって、やっと一段落したって感じだね。 
\\	「一段落する」とは
\\	「休める場所に来た」という意味ですね。	
\\	このラーメンは一つ50円。 
\\	この場合の「一つ、一杯」は
\\	「一杯につき」を使えばいいでしょう。	
\\	しまった。テストで解答欄をひとつずらして書いちゃった。 
\\	「一つずらす」は
\\	で表現できます。
\\	を受け身に使ってもいいでしょう。	
\\	人付き合いを気にしなくてもいいという点ではいいけど、隣に誰が住んでいるのかわからないっていうのもちょっとこわいね。 
\\	「近所との関係」で「人付き合い」を表現しています。
\\	「怖い」。	
\\	今日はすごい人出だね。何かイベントでもあるのかな。 
\\	ここでは「人出」を 
\\	で表現しています。
\\	は「催す」の意。	
\\	一応一通りやれる事は全部やった。 
\\	訳文全体で「一応一通り... やった」というニュアンスを表現しています。	
\\	彼は人に合わせるのがうまいから、皆に好かれるんだよ。 
\\	(…とうまくやる)で「人に合わせる」を表現しています。	
\\	実際有名になれるのは、ほんの一握り。でも、良い経験になった。 
\\	で「一握りの... 
\\	の意味になります。
\\	「実用的な」。	
\\	人のことなど構っていられない。自分のことで精一杯なんだから。 
\\	で「人のことを気にする」の意味になります。
\\	「…で思い悩む」。
\\	の使い方にも注目。	
\\	お前はすぐ人のせいにするからな。 
\\	~で
\\	を~のせいにする、責任にする」の意味。	
\\	人の立場になって考えることも時には必要だよ。 
\\	「人の立場になる」は
\\	で表現できます。
\\	の代わりに
\\	も使えます。	
\\	人の振り見て我が振り直せって言うだろ。 
\\	を使って説明的に表現しています。	
\\	彼って、優秀なシステムエンジニアなんだって。人は見かけによらないわね。 
\\	で「見かけで 
\\	は判断できない」の意味。	
\\	一晩ゆっくりと考えることにするか。 
\\	「一晩寝て考える」は
\\	を使います。	
\\	この年になると、若い子たちに一晩中付き合って飲むのは大変だ。年を感じるよ。 
\\	「一晩中」は 
\\	でいいでしょう。
\\	は「年を感じる」の意。	
\\	あなたの和訳はまあまあ及第点と言えるが、欲をいえばもうひとひねりあってもいい。 
\\	で「ひとひねり」が表現できます。	
\\	僕は人前で話すのが苦手なんだ。 
\\	「人前で話す」は
\\	と言います。	
\\	私人見知りするタイプだから、初対面の人とはなかなか仲良くなれない。 
\\	ここでは「人見知り」を
\\	「初めて会う人たちには引っ込み思案になってしまう」と言い換えて表現しています。
\\	「仲良くやる」。	
\\	その家から現れたのは、瞳涼しい好青年であった。 
\\	「瞳涼しい」は、ちょっと英語にしにくいですね。ここでは
\\	「うるんだ、純真無垢な」、
\\	「夢を見ているような」を使って表現しています。ちなみに、
\\	は「露」、
\\	は「霧」の意、	
\\	時代も変わったよね。一昔前までは、海外旅行なんてそう行けるもんじゃなかったし、一生に一度行ければいいなーなんて思っていたのに。 
\\	「一昔前」のニュアンスを
\\	「かなり前」、
\\	「かつては... だった」で表現しています。	
\\	君は人目を気にしすぎるよ。 
\\	とは「自意識過剰」ということです。	
\\	このリンゴ、一山いくらなのかしら。 
\\	「一山」は
\\	で表現できますが、アメリカでは量り売りをするのが普通です
\\	一人あたりで考えると、シングルルームよりツインルームの方が安いよね。 
\\	で「一人当たりの基準で見てみると」の意味になります。	
\\	発言の一部だけがマスコミに取り上げられて独り歩きした。 
\\	を直訳すると「それ自身で命を持つようになる」という意味になります。	
\\	一人暮らしってなにかと大変なんだよね。 
\\	「一人暮らし」は
\\	ですね。
\\	は強調のための副詞として使われています。	
\\	独り占めするなよ!皆で分けようぜ。 
\\	で「自分だけのものにする」という意味。	
\\	きみも人を立てることを覚えないといけないね。 
\\	(…に敬意を払う)と 
\\	(値する)を組み合わせて「人を立てる」を表現しています。	
\\	お前には人を見る目がない。 
\\	ここでは
\\	「判断する人(名詞)、判断する(動詞)」と
\\	を組み合わせて「人を見る目」を表現しています。	
\\	彼は人を見る目がある。 
\\	とは「人の性格を見抜くのがうまい人」のことです。
\\	は「飲み込みの早い人」の意味。(注)和文を一部修正しました。	
\\	彼には人を寄せ付けないところがある。 
\\	には「よそよそしい、冷ややかな」という意味があります。	
\\	その仕打ちはあまりにもひどすぎる。 
\\	(値する)、
\\	(あのような扱いを受ける)を組み合わせて「ひどい(仕打ちを受ける)」のニュアンスを表現しています。	
\\	私、頭にきちゃって彼にひどいこと言っちゃったの。 
\\	(残酷な)を副詞 
\\	(まったく)で強調しています。
\\	は「卑劣な、たちの悪い」の意。	
\\	別に非難するつもりでいったわけではありませんが、そうとらえたのならすみません。 
\\	ここでは「非難する」を 
\\	(批判的な)を使って表現しています。	
\\	私すぐ日に焼けて、赤くなっちゃうの。 
\\	で「すぐに日に焼ける」の意味。
\\	は「真っ赤な」の意。	
\\	"風呂上りに体を拭いていたら、""ぐぎ!""って捻ったみたいで、それ以来痛い。 
\\	「捻る」は 
\\	で表現します。「グキッ」の表現の仕方にも注目。
\\	みんないい企画書を作るべく、アイデアをひねり出そうとしている。 
\\	「ひねり出す」は
\\	「見つける」、
\\	「取り組む」で表現すればいいでしょう。	
\\	優しくて、よく気が付いて、気が利いて、それでいて頭も切れる。彼は本当に非の打ちどころがない。 
\\	は「これ以上、何を望むことがあるのか(ない)」、
\\	「欠点が一つもない」の意。	
\\	このコップひびが入ってるから使わない方がいいよ。 
\\	「ひび」は動詞としても用いることができます。	
\\	もし納期が遅れでもしたら、相手からの信用度はがた落ちで、今後の取引に響きますよ。 
\\	「響く」とは
\\	に大きな影響を持つ」ということですね。
\\	「商売の関係」。	
\\	動かすと腰に響くからじっとしてるね。 
\\	「腰で感じる」で「腰に響く」を表現しています。	
\\	日々の努力なしに、英語は上達しない。 
\\	で「絶え間ない日々の努力」の意味になります。	
\\	最近、仕事も落着いて暇になった。 急に暇になると何をしていいかわからない。 
\\	「暇になる」は
\\	でいいでしょう。	
\\	旅行が好きで、暇さえあれば旅行に行きます。多いときで年に5回とか。 
\\	で「暇なときはいつでも…する」の意味になります。	
\\	日増しに春めいて来るね。 
\\	が「日増しに」に当たる言い回しです。	
\\	暇つぶしに映画でも見に行くか。 
\\	「暇をつぶす」は
\\	、あるいは、
\\	で表現できます。	
\\	あまりにも何も無さすぎて暇をもてあましてしまった。 
\\	で「時間を持てあます」の意味。	
\\	旦那は今学生で、私が働いているのですが、旦那はヒモにはなりたくないって仕事をしたくなったみたいです。 
\\	ここでは「ひも」を 
\\	(…に頼る)を使って表現しています。
\\	は「金銭面で」の意。	
\\	結構、百円均一ショップで掘り出し物があるよ 
\\	「百円均一ショップ」は 
\\	でいいでしょう。
\\	は「掘り出し物」の意。	
\\	あそこは百万ドルの夜景で有名だ。 
\\	「百万ドルの」は
\\	とハイフンを付け、複数形の
\\	を取って形容詞にします。	
\\	100メートル走で2人もこけたよ。 
\\	「100メートル走」は100-
\\	ですね。	
\\	会議で川添さんの追及を受けて冷や汗ものだったよ。 
\\	「冷や汗をかく」は
\\	という言い方をします。
\\	は
\\	に詳しい説明を求める、迫る」、
\\	「圧力をかける」の意味。	
\\	「彼のことが好きだ」と友だちに告白したら、彼女はにやにやしながら、私を冷やかし始めた。 
\\	「冷やかす」は 
\\	または 
\\	(…をからかう)を使って表現すればいいでしょう。	
\\	外出するなら日焼け止めクリームを塗っていったほうがいいよ。 
\\	「日焼け止めクリーム」は 
\\	といいます。	
\\	冷蔵庫にお茶冷やしてあるから、いつでも飲んでいいよ 
\\	「冷やす」と
\\	の使い方がポイントです。	
\\	ちょっと会議に遅れてくらいで、そんな冷ややかな目で見ないでくれよ。 
\\	は「冷たく、きつい視線」のこと。
\\	「現れる」。	
\\	先生や、クラスによっては、評価レポートをもらえなかった時もありました。 
\\	「評価レポート」は
\\	でいいでしょう。	
\\	彼女は歌声はとても表現力豊かだ。 
\\	「表現力豊か」は
\\	「表現に富む」を使えばいいでしょう。	
\\	あの女性誌、どうして今月はタッキーが表紙なんだろう。 
\\	「表紙」は
\\	で表現します。
\\	は「呼び物にする、特集する」の意。	
\\	"通帳に記入するには、タッチパネルの""通帳記入""の表示をタッチしてください。 
\\	「表示」とは
\\	「ディスプレイ上のボタン」ということですね。(注)投稿文を一部書き換えてあります。
\\	彼女は感情が表情に出やすいのですぐわかる。 
\\	で「…の表情から」の意味になります。	
\\	瓢箪から駒って言うのかな、知人の紹介で就職決まっちゃったんだよ。 
\\	で「ほとんど期待していない時に物事が起きる」という意味を表しています。
\\	「突然に」。	
\\	このレストラン、評判通りのおいしさだね。 
\\	で「(褒めちぎっていた)評判にふさわしい、恥じない」の意味になります。	
\\	このケーキ評判のわりには、あまりおいしくないね。 
\\	は
\\	に恥じない、こたえる」、
\\	は「評判である」の意味。「おいしくない」は表現しなくてもいいでしょう。	
\\	あのひと、いつもひょうひょうとしてるね。 
\\	「ひょうひょうとしている」ということは「物事に動じない
\\	ということですね。
\\	「のんきな、気楽な」。	
\\	表面が良い人や要領が良い人は、出世するのよ。 
\\	「表面が良い」をここでは
\\	「いい印象を外に出す(与える)」で表現しています。	
\\	心の内はどうだかわからないけど、表面上は応援してくれているんだ。 
\\	「表面上」は
\\	を使えば表現できます。	
\\	ひょっとすると、彼は来ないかも知れない。 
\\	には「見込み、可能性」という意味がありますので、これを使えば「ひょっとすると」のニュアンスが出せるでしょう。	
\\	ヒョンなことから、同じ飛行機便に乗り合わせた。 
\\	「ひょんなことから」は
\\	「不思議な巡り合わせで」を使って表現できます。文脈によっては、
\\	「全くの偶然で」を用いてもいいでしょう。もちろん、
\\	も使えます。	
\\	「ああ、飲んださ!」なんて、そんな風に開き直ったってねえ。免停は免停だから。はい、これ切符ね。 
\\	ここでは「開き直る」を
\\	「つんつんする」で表現しています。
\\	「取り消す」。また、
\\	は
\\	の意味で使われています。	
\\	今、とてもよいアイデアが閃いた! 
\\	には「(考えが)急に浮かぶ」という意味があります。	
\\	紫外線が強いせいか、外にいると肌がひりひりします。 
\\	「ひりひりする」は 
\\	で表現できます。
\\	は「紫外線」の意。	
\\	日に焼きすぎて、背中がヒリヒリして眠れないよ。 
\\	に「ヒリヒリする」という意味がありますが、ここでは
\\	「かゆい」も付け加えています。	
\\	今日は忙しくて昼を抜いたから、会議中にお腹がなり始めた。 
\\	この場合の「抜く」は
\\	を使います。
\\	には冠詞
\\	をつけません。	
\\	道端に、だれかが椅子を捨てていたので、まだ使えそうだったから拾ってきた。 
\\	この場合の「拾う」は 
\\	(素早く手に入れる)で表現できます。	
\\	若いうちは、ある一つのことに打ち込んで勉強するもよし、何でも広く浅く勉強するもよしだ。 
\\	が「広く浅く勉強する」に当たる表現です。
\\	「探求する」、
\\	「深入りしない」。	
\\	そちらのスカーフもぜひ広げて見て下さい。見た目の印象が全然違いますよ。 
\\	この場合の「広げる」は 
\\	を使って表現すればいいでしょう。(注)2番目の和文は英訳していません。	
\\	その計画が失敗するのは火を見るより明らかだよ。 
\\	で「ここで君に話しているのと同じくらい確かだ」の意味になります。
\\	「明白な」。	
\\	お前の言うことには品がないよ。 
\\	はそれぞれ「低級な」、「品の悪い」という意味です。	
\\	この文の 
\\	は何の品詞? 
\\	「品詞」は 
\\	と言います。それを使わずに表現することもできます。2つ目の英訳を参考にしてください。	
\\	昔のトマトジュースはまさにトマトって味がしたけど、品種改良されて飲みやすいものになった。 
\\	で「今は品種改良された特別なトマトがあるので」という意味になります。	
\\	どのくらいの頻度でシャトルバスは運行されていますか? 
\\	「頻度」は
\\	「どれくらい頻繁に」で表現できます。
\\	「頻繁な」。	
\\	これからも頻繁にメール交換しようね!!! 
\\	ここでは「頻繁に」を
\\	(親密な)、
\\	を使って表現しています。	
\\	私は、B級ホラー映画が大好きだ。 
\\	とは「低予算」ということです。	
\\	怖い、怖いと思うとただの木だって幽霊に見えちゃうもんだよ。そんなにビクビクするなよ。 
\\	(おびえる)と 
\\	(ビクッとする)を使って「ビクビクする」を表現しています。	
\\	新しい職場で仕事を覚えるのが大変!毎日ビクビクしているよ。 
\\	「ビクビク」を 
\\	(おびえた)を使って表現しています。	
\\	そんな注意の仕方じゃだめだよ。 もっと、びしっと言わないと。 
\\	「もっとハッキリと」、
\\	「きっぱりと(副詞)」を使えば「びしっと」に近い意味が表現できます。
\\	「慎重にやる」、
\\	「気取って言う、ハッキリ言わない」。	
\\	今日はビシッと決めちゃって、デートでもあるのかい。 
\\	が「(服装を)ビシッと決める」に当たる言い回しです。	
\\	私はあそこで働くことに対しては、全然問題ないけど、ビジネス的に私のスキルで問題ないのかなと心配しています。 
\\	で「ビジネスの観点からすると」の意味になります。	
\\	脚どうしたの?なんかびっこひいているように見えるけど。 
\\	「びっこをひいて歩く」は
\\	と言います。
\\	「片足をひきずる」は動詞としても使えます。	
\\	日本では謙虚が美徳と言われている 
\\	「美徳」は
\\	ですね。
\\	「謙虚」。	
\\	彼の微に入り細を穿つようなプレゼン資料の準備は誰もまねできないだろうな。 
\\	「微に入り細を穿つ」とは、
\\	(詳細に注意を払う)ということですね。	
\\	僕の稼ぎなんか微々たるものだ。 
\\	には「わずかな金額」という意味があります。
\\	は「取り立てて言うほどのものではない」の意。	
\\	私が預けている銀行は、確か年2回利子がつくよ。びびたるものだけどね。 
\\	「びびたるもの」は
\\	「ほとんど無いに等しい」で表現すればいいでしょう。
\\	「話にならない」。	
\\	私と彼には微妙な距離があって、なかなかうまくいかなかった。 
\\	「微妙」は 
\\	(微妙な、とらえがたい)を使えばいいでしょう。
\\	は「はっきりしない」の意。	
\\	彼のプレーが反則になるかどうかは微妙なところだ。 
\\	(…かどうかを判断するのは難しい)で「微妙なところだ」を表現しています。
\\	も「微妙な」の意。	
\\	彼女、肺炎で入院したと聞きましたが、その後の病状はどうですか。 
\\	「病状」について聞くには、
\\	を使えばいいでしょう。
\\	「入院している」。	
\\	人間はみな平等だ。生まれて死んでいくのはすべての人に与えられてる。大して差なんてないのよ。 
\\	「平等」は
\\	で表現します。
\\	は「重要ではない」、
\\	は「2次的な、付随する」の意。	
\\	静電気でびりっときた! 
\\	「びりっとくる」を 
\\	を使って表現しています。
\\	は「静電気」の意。	
\\	痛っ!肘をぶつけちゃったよ。ビリビリする。 
\\	とは「(ぶつけると電気が走ったようにしびれる)肘の骨」のこと。
\\	は「ビリビリしびれる」の意。	
\\	最近ビル荒らしが増えているみたいだけど、案外セキュリティが十分でないビルの方が安心だったりするんだよね。 
\\	「ビル荒らし」は
\\	「夜盗」、
\\	「住居侵入」と
\\	を組み合わせて表現します。(注)和文の後半部分は論理的ではありません(説明不足?)が、そのまま英訳しています。	
\\	彼女は流行にとても敏感だ。 
\\	で「…に注意を払っている」の意味になります。	
\\	その貧乏ゆすりやめてくれないかな。 
\\	は「そわそわする」の意味で、落ち着きのない動きを表します。	
\\	いやね。あの人の貧乏揺すりで気が散るわ。 
\\	英語では「貧乏ゆすり」とは言いませんので、ここは
\\	「そわそわする(落ち着かない)」を使って表現しています。	
\\	あとで、ファックス、送ってくれる。 
\\	は名詞・動詞両方に使えます。	
\\	言葉ではうまく表せないけど、私を何か不安にさせるものがある。 
\\	は「悩ませる」、
\\	は「気になる、心配になる」の意味。	
\\	初心者レベルでこんなに難しいと先が不安だな・・・ 
\\	ここでは「不安」を
\\	を考えたくない」で表現しています。	
\\	海外生活を夢見る私にとって、海外生活関連の本を読むと、現状の厳しさを思い知らされて、すっごく不安になることがある。 
\\	「不安になる」は
\\	と
\\	「不安な」、
\\	「心配する」を組み合わせて表現すればいいでしょう。	
\\	あの二人、あんなに若いのに結婚しちゃって、まるで夫婦ごっこしている子供を見てるみたいね。 
\\	「夫婦ごっこ」は
\\	「ままごとをする」でいいでしょう。	
\\	夫婦別姓って、結局認められたんだっけ? 
\\	「夫婦別姓」をここでは
\\	で表現しています。
\\	「配偶者」。	
\\	たまには夫婦水入らずで温泉に行ってのんびりすのもいいもんだ。 
\\	「夫婦水入らず」は
\\	「二人だけだ」、または、
\\	で簡単に表現できます。	
\\	花火に風鈴、蚊取り線香、風流だねえ。 
\\	ここでは「風流」を
\\	「より繊細な」を使って表現してみました。	
\\	これからはいい季節になるので観光客が増えそうですね。 
\\	「増えそう」を 
\\	(来るようになる)で表現しています。
\\	でも同じようなことが言えます。	
\\	課長は彼にちょっと言い過ぎたようだったので、私がそっとフォローしておいた。 
\\	がそのまま使えます。
\\	は「手際よく」、
\\	は「適切にする」の意。	
\\	これに深い意味は全然ないよ。 
\\	「深い意味」は 
\\	で表現すればいいでしょう。
\\	は「隠された」の意。	
\\	首相は、同僚の発言に不快感を示した。 
\\	「不快感」は
\\	、あるいは、
\\	「不満足」で表現できます。
\\	が気にくわない、不機嫌で」。	
\\	家族がコミュニケーションを取らなすぎだと思う。だから不可解な事件が最近起きるんだよ。 
\\	が「説明できない、不可解な」の意味になります。	
\\	椅子に深く腰掛けていないと、誰かがふざけて椅子を引いたらすぐひっくり返るぞ。 
\\	が「椅子に深く腰掛ける」に当たる言い回しです。	
\\	それは不可抗力だ。 
\\	で「コントロールできない」の意味。	
\\	私には煙草の良さがわからない。以前、ふかした事はあるけど、ちゃんと吸った事はない。 
\\	「ふかす」は
\\	と言います。
\\	「吸い込む」。	
\\	深爪しちゃって、人差し指が痛くてね。 
\\	で「爪を痛いところまで切る」の意味になります。
\\	「切り整える」。	
\\	君はいつも深読みしすぎるきらいがあるね。世の中もっと単純だと思うんだけどなあ。 
\\	「深読みする」は
\\	や
\\	などを使って表現できます。	
\\	勝っていた試合が雨で中止になり、選手は、不完全燃焼のままだった。 
\\	「不満足の状態ままである」、
\\	「もっとやれた」で「不完全燃焼」のニュアンスを表現しています。	
\\	あんな奴に負けるなんて。ああ、不甲斐ない。 
\\	「役立たず、いいところが何もない」、
\\	は「哀れ」の意味。	
\\	たまに電車で他人の会話を聞いて吹き出しそうになるんだよ。笑い上戸だから、堪えるのがた大変でさ。 
\\	「吹き出す」は
\\	「急に笑い出す」を使えばいいでしょう。ここでは「笑い上戸」の表現の仕方も覚えておくといいでしょう。	
\\	コンピュータゲームが普及したせいで、子供が外で友達と遊ぶということをしなくなってきていると思う。 
\\	「普及する」とは
\\	「広まっている」、
\\	「流行っている」ということですね。	
\\	携帯電話の普及率は、急速に伸びている。 
\\	(率、割合)と 
\\	(所有権)を組み合わせて「普及率」を表現しています。
\\	は「高騰する」、
\\	は「普及」の意。	
\\	先日、責任のなすりあいをやってからというもの、両者の間には不協和音が流れている。 
\\	が「(はっきりとした)不協和音」に当たる言い回しです。
\\	は「責任を指摘すること」、
\\	は「不協和音」の意。	
\\	イチローが大リーグへ行くんだって。ファンとしては複雑な心境だね。 
\\	で「相反する気持ちを持つ」の意味。
\\	「感情が複雑に入り交じった状態」。	
\\	初めてケーキ作りに挑戦してみたけど、どうも材料の分量を間違えたようで、スポンジが膨らまなかった。 
\\	には「(パンなどが)膨れ上がる」という意味があります。
\\	は「材料」のこと。	
\\	ふくれっ面してないで、機嫌を直して笑ってごらんよ。 
\\	には「ふくれっ面をする、すねる」という意味があります。	
\\	このヌードルってカップと袋入りの両方があるけど、袋入りの方が辛い気がする。 
\\	で「カップ入りと袋入りがある」の意味。	
\\	こんな不景気な時代に、転職は無理だろうな。しばらく、今の仕事で我慢するか。 
\\	が前の
\\	を後ろから説明しています。これで「こんなに悪い経済状態で」の意味になります。	
\\	彼ずいぶん老けて見えるね。まだ20代なんだろう。 
\\	で「年の割には老けて見える」、
\\	は「年齢よりも年を取っている」の意味。	
\\	彼、20代なのに随分老けて見えるね。 
\\	で「老けて見える」の意味になります。前置詞
\\	には「…の割には」の意味があります。	
\\	考えようによってはそれは不幸中の幸いだよ。けがしなくて済んだんだからさ。 
\\	(もっとひどいことになっていた可能性もあった)を使って「不幸中の幸い」を表現しています。
\\	は「軽く済む」の意。	
\\	耳に(ピアスの)穴あいているけど、最近面倒くさくてつけてないのよね。そのうち、穴がふさがってしまうかも。 
\\	この場合の「ふさがる」は
\\	を使って表現できます。動詞としての
\\	の使い方も一緒に覚えておきましょう。	
\\	今日は朝からふさぎ込んでるけど、どうしたんだい? 
\\	「ふさぎ込む」は 
\\	(落ち込む、意気消沈した)で表現すればいいでしょう。
\\	は「ふさぎ込む」の意。	
\\	別れたときに何が辛かったって、彼の存在があまりに大きかったってこと。 
\\	で「私の人生に大きな穴がポカッと開いてしまいました。その穴を、以前は彼が埋めていてくれたのです」という意味になります。	
\\	選挙当日は出かける予定があるので、不在者投票するつもりだ。 
\\	で「不在者投票する」の意味になります。
\\	は「不在者投票」の意。	
\\	会社にあんな格好でくるなんて、絶対ふざけているよね。 
\\	「ふざける」は
\\	「冗談を言う、ふざける」を使えばいいでしょう。
\\	でも同じような意味が表せます。	
\\	あぶないから ふざけるのよしなさい。 
\\	「ふざける」は
\\	でいいでしょう。	
\\	山田先生の講義には独特のリズムがあって、「山田節」と呼ばれている。 
\\	この場合の「節」は
\\	で表現すればいいでしょう。
\\	は「節、韻文」の意。	
\\	昨日までアメリカにいたなんて何だか不思議な感じ。 
\\	この場合の「不思議」は 
\\	(信じられない)、
\\	で表現すればいいでしょう。	
\\	これほど警察の不祥事が続くと、警察は信頼できないと思われても仕方がない。 
\\	「不祥事」は
\\	「スキャンダル」でいいでしょう。
\\	「面目を潰すような不名誉なこと」。	
\\	家は安普請なので、雨漏りが大変。これから梅雨。どうしよう。 
\\	「クラッカーの箱のように作られている」、
\\	「急いで作る(貼り付ける)」で「安普請」を表現しています。ちなみに、
\\	には「(陸軍の)救急車」という意味もあります。
\\	「ざるのように	
\\	すみません。私の説明では不十分だったようですね。補足してくれてありがとう。 
\\	ここでは「不十分」を
\\	「うまく説明していない」で表現しています。
\\	「不十分な」。	
\\	コピーするページに付箋をつけておいてください。 
\\	「付箋」は
\\	と言います。動詞
\\	の元々の意味は「旗を立てる」です。	
\\	先程いただいたファックスですが、ところどころ不鮮明な箇所がありますので、再送していただけませんか。 
\\	「不鮮明な箇所」は
\\	「ぼやけた所」で表現すれべいいでしょう。	
\\	東京の下町にもまだ風情がある。 
\\	「風情」をここでは
\\	「地域色」、
\\	「雰囲気」で表現しています。	
\\	数年後には、コンピュータエンジニアが不足すると言われている。 
\\	「十分でない」 
\\	、あるいは、
\\	「不足」を使えば「不足する」が表現できます。	
\\	マスコミによると現政権の支持率は3割以下だったが、選挙が終わっていざ蓋を開けてみると自公保の議員が過半数を優に超えて当選した。 
\\	「何が起こっているか実際にわかった時」で「蓋を開くけてみると」を表現しています。	
\\	うちの親は、二言目にはもっと勉強しろと言うので嫌になる。 
\\	(すべてに…と言って反応する)を使って「二言目」を表現しています。	
\\	あいつ二股をかけてたってわけか。 
\\	は「うまくやりくりする」、
\\	は「あいまいな態度を取る」の意味。	
\\	2人暮らしだから、キャベツを買ってもいっつも使い切れなくて腐らしてしまう 
\\	で「2人りぐらし」が表現できます。
\\	「腐る」。	
\\	飛行機の後部にある2人席のみのシートって空いてますか? 
\\	「2人席のみ」を
\\	で表現しています。	
\\	うち何するにも2人分かかるからな。ま、しょうがないか。 
\\	で「二人分払う」、
\\	で「二倍払う」の意味になります。例えば、双子の場合には、
\\	の後にに
\\	を付け足せばいいでしょう。	
\\	それは当然買い主の負担でしょ。 
\\	ここでは「負担」を 
\\	(責任)で表現しています。
\\	にも「負担」の意味があります。	
\\	今の仕事って忙しい時と暇な時の差がはげしすぎるんだよね。だから、体に負担がかかるんだよ。 
\\	この場合の
\\	は「徐々に消耗させるもの
\\	という意味で使われています。
\\	「不規則な」、
\\	「たくさん」、
\\	「欠乏」、
\\	あいつは札付きの不良だから、付き合わない方がいい。 
\\	ここでは「札付きの不良」を
\\	「悪名高い非行少年」を使って表現しています。	
\\	せっかくだから普段食べられないものを食べようよ。 
\\	「普段」は 
\\	で表現すればいいでしょう。
\\	は「…なのだから」の意。	
\\	わたしは、普段着とよそ行きの差が激しいと自分でも思っている。 
\\	「普段着」は
\\	といいます。
\\	「めかし込んだ」、
\\	「衣装」。	
\\	いつもと感じが違うなと思ったら、縁なしのメガネにしたんだ。 
\\	「縁なしメガネ」は
\\	と言います。
\\	「気づく、感づく」。	
\\	(クラスメートとの)長年のわだかまりが、久しぶりの同窓会でふっきれた。 
\\	ここでは「ふっきれる」を
\\	「消える、蒸発する」、
\\	「消える、溶けてなくなる」を使って表現しています。
\\	「恨み」。	
\\	かつては楽に腹筋100回はできたけどな。 
\\	「腹筋(一回)」は
\\	といいます。	
\\	何か上から降ってきた。 
\\	「降ってくる」は 
\\	(落ちてくる)で表現すればいいでしょう。	
\\	彼女にぼくのこと好きって聞いたら普通って言うんだよ。 
\\	この場合の、
\\	と
\\	は「まあね」の意味になります。	
\\	普通の生活ができればそれで十分だ。 
\\	この場合の「普通」は
\\	「普通の、平凡な」でいいでしょう。	
\\	いいアイデアだと思ったけど、ちょっと普通すぎるね。もっと、斬新なアイデアを出さないといけないな。 
\\	は「月並みの、ありふれた」という意味です。
\\	(普通の)と 
\\	を組み合わせてもいいでしょう。
\\	は「革新的な」の意。	
\\	いくら右目が極端に悪くても、本を読んだりパソコンのディスプレイを見るとき以外、普通の生活をするぶんには何の支障もない。 
\\	で「普通の生活にとって」の意味になります。
\\	は「…の邪魔になる」の意。	
\\	もう2日連続で電車遅れて、今週は最悪だったよ。v 
\\	「連続」は
\\	を使って表現します。	
\\	あいつ新人のくせに、ずいぶんふてぶてしいね。 
\\	は「生意気な」、
\\	は「でかい態度をとる」の意味。	
\\	生来の筆無精のため、返事が遅くなり申し訳ない。 
\\	「筆無精」は
\\	で表現できます。	
\\	子供の頃、夜中にトイレに行こうとして、ふと窓を見たら窓を透けて何かが見えた気がしてとても怖かった。 
\\	「ふと」を
\\	「突然」、
\\	「偶然に」で表現しています。
\\	「怖がらせる」、
\\	「怯えさせる」。	
\\	ふと今思ったんだけど、日本語どのくらい話せるのですか? 
\\	で「…が浮かぶ」の意味になります。
\\	と 
\\	の使い方に注目。	
\\	今話していてふと思ったのだけどさ、彼らはあんなにサービスが良くてどうやって商品の売上の採算をとっているのだろう。 
\\	で「ふと頭に浮かぶ」の意味になります。
\\	は「利益」の意。	
\\	音楽を聞いたりCMを見た後で、ふと気づいたら頭の中で、その曲とかが流れていることってない? 
\\	「ふと気づくと」を
\\	「気づく前に」、
\\	「次に気づくのは」を使って表現しています。
\\	「コマーシャルソング」。
\\	「くっついている、はりついている」。	
\\	そのビルには、不特定多数の人が出入りしていた。 
\\	「不特定多数の人が出入りする」とは
\\	「一般の人が誰でも入れる」ということですね。	
\\	彼とはふとしたことで知り合ったの。 
\\	(まさに偶然)で「ふとしたことで」を表現しています。	
\\	それはどうも腑に落ちない話ですね。 
\\	「腑に落ちない」は
\\	「どうも理解できない」で表現できます。
\\	「しっくりこない」の意味。	
\\	この書類には不備な点がかなりあるね。書き直してくれたまえ。 
\\	とは「欠点、不備」のこと。
\\	は動詞としても使うことができます。	
\\	どのグループにも必ずなにがしかの不文律はあるはず。 
\\	「不文律」とは
\\	「文章化されていない規則」のことです。	
\\	小渕前首相の訃報が突然テレビから流れ、びっくりしてしまった。 
\\	で「訃報を聞く」の意味になります。	
\\	その大胆な法案は、各議員に踏絵を迫った。 
\\	ここでは「踏み絵」を
\\	「リトマス試験紙」を使って表現しています。
\\	「立場を明らかにする」。	
\\	彼女のためによかれと思って色々なことをしてあげたのに、彼女は私の好意をことごとく踏みにじるようなことをした。 
\\	ここでは「踏みにじる」を 
\\	(…に唾を吐く)で表現しています。	
\\	階段を踏み外して、右足を捻挫してしまった。 
\\	が「踏み外す」に当たる言い回しです。
\\	「捻挫する」。	
\\	ガム、踏んじゃった。 
\\	「…を踏む」は
\\	ですね。	
\\	立ち仕事で足が疲れた時は、ゴルフボールを踏むと疲れがとれると聞きました。 
\\	「踏む」は 
\\	でいいでしょう。
\\	と 
\\	の組み合わせにも注目。	
\\	携帯電話の普及で幽霊の目撃が少なくなったっていう記事を読んだのだけど、それらの因果関係についてはいまだ不明らしい。 
\\	「不明らしい」は 
\\	(はっきりしないようだ)で表現すればいいでしょう。
\\	は「頻度」の意。	
\\	アイスのコーンがふやけちゃったよ。 
\\	は「ふやける」、
\\	は「柔らかな」の意味。	
\\	彼女は自分のことは棚にあげて人の悪口ばかり言うので、一緒にいると不愉快になる。 
\\	「一緒にいると不愉快になる」をここでは 
\\	(彼女のそばいにいたくない)で表現しています。
\\	(棚上げにする)の使い方にも注目。	
\\	扶養家族のことを考えると、そう簡単には今の会社を辞められない。 
\\	「扶養家族のことを考える」を
\\	(家族がいかに私のことを頼っているかを考える)で表現しています。
\\	は「食べさせなくてはならない家族」の意。	
\\	飲み過ぎて足元がふらふらするよ。 
\\	は「不安定な、ぐらつく」の意味です。
\\	「真っ直ぐ歩けない」でも同じことが表現できます。	
\\	朝貧血起こして倒れてしまったんだけど、いまだに頭がふらふらしている。まだ、座っているからどうにかもっているけど。 
\\	(めまいがする)で「頭がふらふらする」を表現しています。
\\	は「貧血」の意。	
\\	会社の帰りにふらりと立ち寄った本屋で、面白い本を見つけたんだ。 
\\	「ふらりと立ち寄る」を
\\	(たまたま…する)と
\\	(立ち寄る)を使って表現しています。	
\\	彼女に振られたくらいで、そんなに落ち込むなよ。 
\\	「振られる」は
\\	といいます。
\\	「ふる、捨てる」。	
\\	間違い探しって、後の人ほど絶対不利だよ!だって、先に簡単なの全部言われちゃうもん。 
\\	ここでは「不利」を
\\	を使って表現しています。また、
\\	比較級、
\\	比較級」の構文にも注意。
\\	「変なところ」。	
\\	自分が誤解して怒っているのに気づいて、振り上げたこぶしをどう収めて良いか困ってしまった。 
\\	で「怒りまくる」という意味です。
\\	「撤回する」、
\\	「挑戦する、反抗する」。	
\\	来週は振替休日があるから3連休になる。 
\\	「振り替え休日」は 
\\	でいいでしょう。
\\	は「…を補う、埋め合わせる」の意。	
\\	(携帯電話で)ねぇ、わたし、今どこにいると思う?振り返ってみて! 
\\	「振り返る」は
\\	ですね。	
\\	彼は最後の気力を振り絞って目標を達成した。 
\\	は「寄せ集める、奮い起こす」、
\\	は「奮い起こす、呼び出す」の意味。	
\\	彼はいつも彼女を自分の都合によって振り回す。 
\\	は「こき使う」の意味。
\\	「適応させる」、
\\	「気まぐれ」。	
\\	振り向いてくれない人をいつまでも想い続けるなんてバカみたいだって自分でも思う。 
\\	ここでは「振り向く」を 
\\	([人のことを]考える)、
\\	を使って表現しています。	
\\	彼は今では真面目な好青年だが昔は不良で有名でした。 
\\	が「不良」に当たる表現です。
\\	は「立派な、しゃんとした」、
\\	は「悪名高い」の意。
\\	とは 
\\	のこと。	
\\	私に話をふるな! 
\\	ここでは「話をふる」を
\\	「私のことを見るな」と言い換えています。	
\\	腰掛けのつもりだったのに、気がついたら私もこの会社で古株になってしまった。 
\\	「古株」に当たる表現には、
\\	「古参」や
\\	「古くからいる一人」、
\\	「古顔」などがあります。
\\	考える」、
\\	「一時しのぎ」。	
\\	彼にはもっと課長らしく振るまってほしいよ。あれじゃ、部下は誰もついていかないよ。 
\\	「もっと 
\\	らしく振る舞う」は
\\	で表現できます。	
\\	フレックス制度のお陰で、朝寝坊したけど遅刻しないですんだ。 
\\	「フレックス」は
\\	と
\\	を付けて表現します。
\\	は「間に合う」の意。	
\\	今回の旅行では、大自然に触れることができた感じがした。 
\\	「大自然に触れる」は
\\	「自然に戻る」、
\\	「自然に近づく」で表現すればいいでしょう。	
\\	実際に海外で生の英語に触れて、こんなにもリスニング力がないというのを改めて思い知らされて落ち込んだ。 
\\	で
\\	に接触する」の意味になります。
\\	「発達していない」。	
\\	最近、駅で寝ている浮浪者達が増えている。 
\\	「浮浪者」に当たる英語は
\\	や
\\	です。	
\\	髪型かえたら、なんか雰囲気かわったね。一瞬、君だとわからなかったよ。 
\\	「雰囲気」は
\\	を使えば表現できます。
\\	「違って見えるようになる」。	
\\	床がみしみしいうのは、古いって事もあるけど、古さの雰囲気を出すためにしているというのもあります。 
\\	「雰囲気を出す」は 
\\	と 
\\	を組み合わせて表現すればいいでしょう。
\\	は「キーキーいう」、
\\	は「人が住んでいる(いた)ような」の意。	
\\	大統領は分刻みの過密スケジュールで動いている。 
\\	ここでは「分刻み」を 
\\	(計画を練る)と 
\\	を組み合わせて表現しています。	
\\	国会では、自衛隊法案を通すか通さないかで紛糾した。 
\\	で「…のことで大騒ぎする」の意味。
\\	は「身動きの取れない状態になる」の意。	
\\	そんなところでふんぞり返ってないで、さっさと仕事に取りかかれよ。昼休みはとっくに終わってるぞ。 
\\	は「ふんぞり返る、そり身になる」の意味。
\\	「仕事に取りかかる」。	
\\	雨には降られるわ、デートに遅刻するわ、財布を落とすわ、今日は踏んだり蹴ったりだよ。 
\\	「踏んだり蹴ったり」をここでは
\\	「次から次に起こる」で表現しています。現在形を使って、それが今起きているかのように言っている点にも注意して下さい。	
\\	歳のせいか、踏ん張りがきかなくなった。 
\\	「踏ん張り」は
\\	「ねばり強さ」、
\\	「根気強い」を使って表現します。	
\\	ここが踏ん張りどころだ。みんなで力を合わせて頑張ろう。 
\\	「踏ん張りどころ」を
\\	「最善を尽くす場面」で表現しています。	
\\	親指を怪我していたので足に力が入らず、ちょっとした揺れにたいして踏ん張ることができずころんでしまった。 
\\	ここでは「踏ん張る」を
\\	「立ったままでいる」と言い換えています。	
\\	今日は給料日だし奮発して高級な霜降りの松坂牛を買っちゃおう。 
\\	のは「贅沢をする」、
\\	で「思い切り... する」の意味。ここでは「霜降り」の表現の仕方にも注目!	
\\	学生の分際でこんな高価なものを買うなんて分不相応だ。 
\\	と
\\	(その人の身分に合っている)を組み合わせて「分不相応」を表現しています。
\\	は「ふさわしい」の意。	
\\	あいさつしてもニコリともしない。あいつ、無愛想な奴だな。 
\\	には「ぶっきらぼうな、ぞんざいな」という意味があります。
\\	「荒々しい、無愛想な」。	
\\	そんなに部下のことが信じられませんか? 
\\	「部下」は簡単に
\\	でいいでしょう。多少説明的になりますが
\\	「あなたの下で働いている人達」とも言えます。	
\\	急に笑い出さないでよ。不気味だから。 
\\	「不気味」は
\\	「気味が悪い」でいいでしょう。	
\\	人間というのは本質的にみんないい人なんだけど、なかにはそれを表現するのに不器用な人がいるんだと思う。 
\\	で「…するのが下手な」という意味。
\\	は「…するのがあまりうまくない」の意。	
\\	その無精ひげ似合わないからそった方がいいよ。 
\\	「無精ひげ」は
\\	と言います。
\\	「何日もひげを伸ばす」。	
\\	今やっているドラマでケーキ屋さんを舞台にしているのがあって、毎回出てくるケーキが美味しそうで見るたびに食べたくなる。 
\\	「舞台にしている」は 
\\	(ケーキ屋さんで起こる)で表現すればいいでしょう。
\\	は「…にセットされている(舞台にしている)」、
\\	は「とても美味しい」の意。	
\\	虫歯を見つけられたおかげで、思ったより治療に時間がかかり、そのあとのすべての計画がぶち壊しになった。 
\\	この場合の「ぶち壊す」は 
\\	で表現すればいいでしょう。
\\	は「乱す」の意味。	
\\	また新宿で発砲事件があったんだって。ぶっそうな世の中になったもんだ。 
\\	「ぶっそうな世の中」は
\\	を使って表現します。
\\	とすると「どんどんぶっそうになっている」という感じになります。	
\\	ぶっちゃけた話、今までこちらでホームステイされた方は、宿泊費としていくらぐらい払われていましたか? 
\\	で「こんなことを聞いて気を悪くしないでほしいんだけど」を使って「ぶっちゃけた話」のニュアンスを出しています。	
\\	ぶっちゃけた話し、これっていくらぐらいしたの? 
\\	は「ぶしつけで悪いんだけど」の意味。	
\\	もうぶっつけ本番で発表するしかないな。 
\\	(…を勘でやる、臨機応変にやる)で「ぶっつけ本番」のニュアンスを表現しています。	
\\	この1カ月間、ぶっ続けで出勤した。毎日深夜残業、徹夜もした。 もう死にそうです。 
\\	「ぶっ続け」を 
\\	(休むことなく続けて)で表現しています。
\\	は「一日も欠かさずに」の意。	
\\	海外製の化粧品がどうも肌に合わないのか、最近ぶつぶつが出てきました。 
\\	には「(吹き出物が)出る」という意味があります。
\\	は「にきび、吹き出物」の意。	
\\	顔にぶつぶつができてる。最近、寝不足だから。 
\\	とは「ニキビ、吹き出物」のことです。
\\	「発疹」。	
\\	この製品は普通に使用しても壊れないから、割れているということは何らかの物理的な力が加わったということだ。 
\\	で「だれかがかなりの力を…に加えたに違いなかった」の意味になります。
\\	は「割れる」の意。	
\\	初対面では女性の年齢は聞かないのが無難でしょう。 
\\	(最も安全なこと)で「無難」を表現しています。
\\	は「…するのを避ける」の意。	
\\	彼は若いころ遊び人で、数々の武勇伝が伝えられている。 
\\	「武勇伝」は
\\	(英雄的行為)ともいえますが、遊び人ということなので、
\\	の代わりに
\\	を使っています。	
\\	光が眩しいのでブラインドを閉めてもいいですか? 
\\	「ブラインド」は
\\	、あるいは、
\\	と言います。	
\\	ブラインドタッチを身につけるには、当たり前のことですが、できるだけキーボードを見ないようにして練習することが必要です。 
\\	が「ブラインドタッチで打つ」に当たる動詞です。
\\	「当たり前のこと」。	
\\	このブラインドを上げ下げするのってどうやるの。 
\\	この場合の
\\	は「動かす、動く」という意味になります。	
\\	つり革につかまってぶら下がっちゃだめよ。人の迷惑になるでしょ。 
\\	からぶるさがる」は
\\	で表現します。
\\	「つり革」、
\\	「迷惑(不快)な事」。	
\\	たまにはぶらっと町並みを散策して、ゆっくり過ごす旅もいいものだね。 
\\	で「あてもなくぶらつく」という意味。	
\\	昼間家にいることはあまりないな。1人でいるのはつまらないので、時間があれば出かけてぶらぶらしている。 
\\	「ぶらぶらする」は 
\\	で表現すればいいでしょう。
\\	は「当てもなく」の意。	
\\	彼は、リストラで会社を辞めてからしばらくぶらぶらしていた。 
\\	には「さまよう、漫然と過ごす」の意味があります。
\\	「数を減らす」の意。	
\\	ブランド服には興味がないので、今着ている服もどこの商品なのか知りません。 
\\	で「ブランド服」を表現しています。
\\	の使い方にも注目。	
\\	皆さん、今夜は無礼講で行きましょう。それでは、乾杯。 
\\	「堅苦しい振る舞いはやめる」が「無礼講」に当たる言い回しです。	
\\	ガソリンを2000円分だけ入れて下さい。 
\\	分」は
\\	を使って表現します。	
\\	すみません、バッグの分はこのカードで払って、ポーチの分はこっちのカードで支払いたいのですが。 
\\	ここでは「分」を前置詞 
\\	で表現しています。ほかの前置詞の使い方にも注目。	
\\	わが大学では、全面禁煙ではなく分煙を実施していますので、よろしくご協力のほどお願い致します。 
\\	「分煙する」を 
\\	(煙草が吸える場所と吸えない場所に別れている)で表現しています。
\\	は「応じること」の意。	
\\	これは脂肪を分解してくれるので、ダイエットに効果的です。 
\\	「分解する」は
\\	で表現します。
\\	「効果的な」。	
\\	高校3年生になると大学進学のために文系か理系かを選び、それによってクラス分けされます。 
\\	ここでは「文系・理系」を
\\	と
\\	で表現しています。
\\	「それに応じて」。	
\\	文章が下手だから、彼にこの原稿を頼まない方がいい。 
\\	で「書くのがとても下手」の意味になります。
\\	は「与える、割り当てる」のこと。	
\\	君の文体は、口語体と文語体が混ざってちぐはぐな印象を与えている。 
\\	「文体」は
\\	でいいでしょう。
\\	「まとまっていない」、
\\	「ごた混ぜ」。	
\\	宿題って、自分でやらないと意味ないと思うよ。ましてや、友達同士で分担してやるなんてもってのほか。 
\\	「分担する」は
\\	「分ける」、あるいは、
\\	「分割する」を使って表現します。
\\	「ばらばら」、
\\	の目的を台無しにする」、
\\	「とんでもない、馬鹿げている」。	
\\	涙を流せば流した分だけ、幸せになれるというもんだ。 
\\	この場合の「分だけ」は
\\	するればするほど 
\\	だ」を使えば表現できますね。	
\\	文法的に理解できない箇所が何カ所かあります。 
\\	で「文法的に」の意味になります。
\\	「文法的見地からすると」。	
\\	昔しばしば辛いカレー食べていた時は全然平気だったけど、食べなくなったら甘口カレーさえ辛く感じる。 
\\	で「平然として」の意味になります。
\\	「ためらわずに」。	
\\	この間、メールで15日は平気だって書いたけど、予定が入っちゃったんだ。ほかの日でもいい? 
\\	この場合の「平気」は
\\	、あるいは、
\\	で表現すればいいでしょう。	
\\	昔は、平気で昆虫触れたのに、今は、図鑑を見るだけでもだめ。 
\\	ここでは「平気で」を
\\	を使って「全然気にならない」と言い換えて表現しています。
\\	する気になる」。	
\\	ぼくの兄は、いつも平行して三冊の本を読んでいる。昼休みに読む本、カバンに入れてなんかの待ち時間に読む本、そしてお風呂で読む本! 
\\	「並行する」は
\\	「同時に」で表現すればいいでしょう。	
\\	僕、閉所恐怖症なので、狭いところは駄目なんですよ。 
\\	で「閉所恐怖症の」。
\\	はその名詞形。	
\\	閉所
\\	よくもまあこんなときに平常心でいられるな。 
\\	で「平静を保つ」の意味。
\\	「平静」。	
\\	企業を取り巻く閉塞的な経済環境が続いているために、依然日本経済はよくない。 
\\	は「息苦しい」、
\\	は「窒息するような」の意味。	
\\	夕方少し熱があったが、熱いお風呂に入って、早めに寝たら、平熱に戻った。 
\\	で「平熱に戻る」の意味になります。	
\\	今夜のパーティーには平服で来てくださいね。 
\\	「平服」は
\\	でいいでしょう。	
\\	こんな平凡な日々がいつまで続くのだろうか。 
\\	は「平凡な」、
\\	は「平穏無事な、波乱のない」という意味。	
\\	日本人は平和ぼけだとよく言われる。 
\\	で「平和主義の気持ちのいいぼーっとした状態になる」という意味を表します。
\\	は「弱る、活気がなくなる」の意。	
\\	一度大学受験に失敗したくらいでへこたれるなよ。 
\\	で「がっかりする」の意味。
\\	は「くじける」の意。	
\\	ちょっとへこみましたが、そうそう物事は始めからうまくいくものとは思ってないんで、こんなことではへこたれません。 
\\	ここでは「へこたれる」を 
\\	で表現しています。
\\	は「挫折、失敗」の意。	
\\	あらら、こんなに車をへこましてくれて、どうしてくれるんですか? 
\\	「へこます」は
\\	を名詞、あるいは、動詞として使えば表現できます。	
\\	またへそ出しルックが復活してきたみたいだね。 
\\	で「おへそを出す」の意味。
\\	「へそ出しルック」。	
\\	そんなつまらないことで、へそを曲げるなよ。 
\\	で「ご機嫌斜めになる」の意味。	
\\	へたくそ!ちゃんとやれよ。 
\\	「へたくそ」は
\\	「ひどい」で表現できます。	
\\	見てこれ、「本物100
\\	だって。こんな物に本物もへったくれもないじゃんねぇ、ブランド物じゃあるまいし。 
\\	「へったくれ」のニュアンスを 
\\	で表現しています。
\\	は「本物の」の意。	
\\	先生に怒られたってへっちゃらだい。 
\\	で「平気、少しもかまわない」の意味。	
\\	そんなへっぴり腰じゃ、力がはいらないぞ! 
\\	「へっぴり腰」を
\\	「腰をしっかり入れないと」で言い換えています。
\\	「力を入れる」。	
\\	あいつの顔を見るとへどが出そうになる。 
\\	は「吐く、もどす」の意味。
\\	は「(胸が)むかつく」ということ。	
\\	彼ってほんとにへなへなしてるわよね。 
\\	「へなへなしている」は
\\	「小心な」、
\\	「意気地なしの」などを使って表現します。	
\\	先は長いんだから、そんなに飛ばすと途中でへばっちゃうぞ。 
\\	は「疲れて倒れる」、
\\	は「疲労から」の意味。
\\	「落伍する」。	
\\	減らず口をたたかずに、素直に自分の負けを認めろ。 
\\	で「うまく言い抜ける」の意味になります。
\\	は「…のための」の意。	
\\	屁理屈ばかり並べて、お前の言ってること全然筋が通ってないよ。 
\\	「こじつけの考え」を
\\	「並べる」で「屁理屈を並べる」に近い意味になります。
\\	「屁理屈を言う」。	
\\	このジュースいくら飲んでも減ってない気がする。 
\\	で「減らない」という意味になります。	
\\	茶髪の人って、ちょっと遊んでいるように見られがちだけど、それは偏見だよ。 
\\	「偏見」は
\\	ですね。
\\	「偏見のある、偏った」。
\\	「ぶらぶらする」。	
\\	今、返事に困っている? へんなことを質問しちゃったかなあ。 
\\	で「返事が思いつかない?」という意味ですね。	
\\	君の酒の強さに脱帽して、僕がもっている酒豪のタイトルを返上するよ。 
\\	で「…のタイトルを返上する」の意味になります。
\\	は「酒が強い」、
\\	は「…に脱帽する」の意。	
\\	現地で地震は初めてだったらしく、現地の人は大慌てでしたが、その点、日本人は慣れているせいか平然としていました。慣れって怖い。 
\\	で「少しも驚かすことはなかった」の意味。
\\	「恐怖に怯える」、
\\	「冷静でいる」。	
\\	最近の風邪は、扁桃腺が腫れて、高熱が出るらしい。 
\\	で「腫れた扁桃腺」の意味になります。複数形の
\\	がついている点に注意してください。	
\\	あいつは扁平足だから、走るのが遅いんだよ。 
\\	「扁平足」は
\\	と言います。ちなみに、「(扁平足を)している」は
\\	を使って表現します。	
\\	今、大学は研究機関から教育機関へと完全に変貌した。 
\\	で「~から…に一変する」の意味になります。
\\	は「少しずつ変化する」の意。	
\\	遂に新型スカイライナーがベールを脱いだ。 
\\	は「ベールを取る」の意味。【注】「遂に」は英訳していません。	
\\	韓国への旅行って最近流行っているけど、食べるだけとか買い物だけで行きたいとは思わないな。何か見所があれば別だけど。 
\\	が「別(のこと)」に当たる言い回しです。
\\	は「呼び物、人の注意を引きつける物」の意。	
\\	別に結婚していたことを隠していたわけではないけど、誰も聞かなかったからあえて自分から言わなかっただけ。 
\\	「別に 
\\	していたわけではない」を
\\	する為に特別な努力はしていない」で表現しています。
\\	「発表する、公表する」。	
\\	その件に関しては、別の機会に話し合うことにしよう。 
\\	「別の機会」は
\\	を使えばいいでしょう。	
\\	デザートは別腹だからね。 
\\	ここでは「別腹」を
\\	「デザートを食べるスペースをあけられる」と言い換えています。	
\\	会計は別々にしてください。 
\\	で「別々に計算(合計)する」の意味。	
\\	(高速道路で)ここから先は別料金って書いてあるよ。 
\\	「高速道路の料金」は
\\	といいます。これと
\\	を組み合わせれば「別料金」が表現できます。	
\\	あの人は、良くも悪くも弁が立つので、討論では太刀打ちできない。 
\\	は「雄弁な、話し上手な」、
\\	は「口が達者だ」の意。	
\\	いいこと聞いたわ。それって勉強になったわ。 
\\	「勉強になる」は
\\	、あるいは、
\\	を使って表現できます。	
\\	私の英語力なんてまだまだ勉強不足です。このままでは、海外生活なんてできたものじゃない。 
\\	「勉強不足」は
\\	と
\\	を使うことで表現できます。	
\\	クリーンイメージで知られている現職の知事が、業者に便宜供与を図ったとして逮捕された。 
\\	「便宜供与を図る」とは 
\\	(…に対して物事を容易にする)ということですね。
\\	は「…を特別扱いする」の意。	
\\	私はお弁当屋で働いてます。 
\\	「弁当屋」を 
\\	(お弁当を売っているファストフードの店)で表現しています。	
\\	不規則な生活をしているせいか、ちょっと便秘気味でね。 
\\	「便秘」は
\\	と言います。
\\	は
\\	気味」の意味。
\\	「便秘になっている」。	
\\	この夏休みに、ロサンゼルスに3週間ホームステイをして英語を勉強するつもりです。 
\\	「ホームステイ」はそのまま
\\	でいいのですが、どのようにして前置詞
\\	と組み合わせるかがポイントです。	
\\	最近引越した先輩いるのだけど、引越してから悪いことが立て続けに起こるんだって。方角が悪いのかな~って。 
\\	「方角」は 
\\	(位置)、
\\	(方向)で表現すればいいでしょう。	
\\	あなたはどうしていつも話をそういう方向に持って行きたがるかな。 
\\	「方向に持っていく」は
\\	を動詞として使えばいいでしょう。	
\\	彼女、方向音痴だから、すぐに道に迷っちゃうんだ。 
\\	「方向音痴」は
\\	と言います。	
\\	今日は祖父の命日で、法事にたくさんの親戚が集まった。 
\\	が「法事」に当たる表現です。
\\	は「記念する」の意。	
\\	うちの会社は、チケット会社の法人会員なので、会社を通してチケットが買えて、わざわざ取りに行く必要もない。 
\\	「法人会員」は
\\	と言います。
\\	と
\\	と
\\	の間には
\\	が省略されています。	
\\	放任主義だと言って、しつけを放棄している母親がかなりいる。 
\\	は「勝手にさせておく」、
\\	は「自由にさせる」という意味です。
\\	「回避する」。	
\\	彼は話題が豊富だから、話していて楽しい。 
\\	で「話すことがたくさんある」の意味。	
\\	彼の魅力って、包容力があるところよね。 
\\	で「他の人を受け入れられる包容力(大きな収容能力)がある」の意味。
\\	「忍耐力、寛大さ」、
\\	「欠点」。	
\\	我が国のパソコン市場もそろそろ飽和状態に近づいてきたので、他国の市場を模索しなければならない。 
\\	が「飽和状態」に当たる表現です。
\\	「探し出す」、
\\	「模索する、探す」。	
\\	1年間アメリカでホームステイする予定です。 
\\	は動詞としては使えませんので、文脈に合わせて前置詞
\\	などと組み合わせます。	
\\	ほかに好きな人がいるんですか。 
\\	この場合の「他に」は 
\\	で表現すればいいでしょう。	
\\	他の女のところへ行かないって、約束して。私が好きなのは、あなただけよ。 
\\	「他の女」はそのまま
\\	と言えばいいでしょう。	
\\	ありがとう、他のだれから言われるより嬉しい。 
\\	の使い方がポイントです。	
\\	彼、昇格したんだんって?もう、ほくほく顔だったよ。 
\\	「ほくほく顔」は、
\\	と
\\	「顔を見れば分かる」を組み合わせて表現できます。
\\	は「顔中に書いてある」の意。	
\\	彼女の口元のほくろ、セクシーだね。 
\\	「ほくろ」のことは
\\	と言います。
\\	を使うこともあります。	
\\	堂々と通りの真ん中を歩けるから、やっぱり歩行者天国はいいわよね。 
\\	「歩行者天国」には決まった言い方がありませんので、ここでは
\\	「道路が閉鎖されて車が通れなくなっている」と説明しています。	
\\	このマウス動きが悪いな。中にほこりがたまってるのかな。 
\\	「埃がたまる」は
\\	と言います。	
\\	埃が立つからドタバタしないで。 
\\	で「埃を舞い上がらせる」、
\\	は「まき散らす」の意。	
\\	このマウスどうも動きが悪いと思ったら、中に埃がたまってたんだ。 
\\	が「埃がたまる」に当たる言い回しです。
\\	「たまる、集まる(自動詞)」。	
\\	この会議場は、科学技術の粋を結集して作られており、規模においても東洋一を誇る。 
\\	は「…を誇る」、
\\	は「粋、神髄」の意味。	
\\	いままでに、あの車を欲しいと思ったことはあるの? 
\\	「欲しいと思う」は
\\	と
\\	を組み合わせて表現すればいいでしょう。	
\\	彼らは、英語ができる人間よりも、コンピュータスキルを備えている人材を欲しがっている。 
\\	この場合の「欲しがっている」は
\\	「探し求めている」を使って表現できます。	
\\	ウインドーショッピング?遠慮しとく。見れば欲しくなるから。 
\\	この場合の「欲しくなる」は 
\\	と 
\\	を組み合わせて表現すればいいでしょう。
\\	は「ウインドウショッピングに出かける」の意。	
\\	もう少し細い釘はありませんか。 
\\	「(釘の)細い」は
\\	を使って表現します。	
\\	ウエストをあと2センチ細くしたい。 
\\	~で
\\	から…を取る」の意味になります。
\\	は「やせさせる」の意。	
\\	大臣と役人のごたごたが、この事件のそもそもの発端である。 
\\	「発端」は 
\\	を使って表現すればいいでしょう。
\\	は「源を発する」の意。	
\\	一緒にいてほっとできて、私を大事にしてくれる人が、タイプかな。 
\\	「ほっとする」は
\\	「くつろぐ」を使えばいいでしょう。	
\\	全員無事と聞いて、ほっと一安心といったところだ。 
\\	で「ほっとため息をつく」の意味になります。	
\\	この料理は美味しくて頬っぺたが落ちそうだ。 
\\	(あごがほとんど落ちそうだ)で「頬っぺたが落ちそうだ」のニュアンスを出せるでしょう。	
\\	皮下脂肪が厚いと体温が体内にこもり、からだや顔が熱く、ほてりを感じる。 
\\	ここでは「ほてり」を 
\\	で表現しています。
\\	を使ってもいいでしょう。
\\	は「皮下脂肪」の意。	
\\	あなたの大人げない言動にはホトホト呆れます。 
\\	「ほとほと」は 
\\	や 
\\	で表現すればいいでしょう。
\\	は「子どもっぽさ、幼稚」の意。	
\\	ほとぼりが冷めるまで待ちましょう。 
\\	「死の灰が消えて無くなるのを待つ」を使って「ほとぼりが冷める」を表現しています。
\\	「無事におさまる」。	
\\	マンションって上の階に行けば行くほど、家賃って高くなるんでしょ。 
\\	比較級、
\\	比較級」の使い方がポイントですね。
\\	「分譲マンション」。	
\\	自分が思っている程他人は見てないって。 
\\	この場合の「程」は
\\	で表現できます。
\\	「注意して」。	
\\	今度発行される新しいお札は、偽札防止に最新技術が施されて いるらしい。 
\\	この場合の「施す」は
\\	印刷されている」と言い換えて表現すればいいでしょう。
\\	「飾る」。	
\\	わざわざ言うほどのことじゃないでしょ。 
\\	で「わざわざ出てきて言う」の意味。
\\	は「…する価値がある」の意。	
\\	料理って程の物じゃないんだけど、今食べるもの作ってるんだ。 
\\	「程の物」のニュアンスを 
\\	(大した…ではない)で表現しています。	
\\	お酒もほどほどにしないと、体に悪いぞ。 
\\	で「適度に、ほどほどに」の意味。
\\	と副詞にしても使えます。ちなみに、「何事もほどほどに」は
\\	と言います。	
\\	冗談も程々にしてよ。 
\\	ここでは「程々」を
\\	(そんなに冗談を言うのは止めろ)で表現しています。	
\\	わざわざ電車に乗ってここのランチを食べに来たのに。今日は休みかよ。骨折り損のくたびれ儲けだったな。 
\\	で「まったくの時間の無駄」と言っていることになります。
\\	「すべての苦労が無駄になる」。	
\\	私の仕事は大変骨が折れるけど、やりがいがあります。 
\\	「骨が折れる」は 
\\	(大変な)で表現すればいいでしょう。	
\\	そんな無理なダイエットをしてたら、骨と皮になっちゃうよ。 
\\	「骨と皮」は、英語では
\\	「皮と骨」という語順になります。	
\\	政府は骨太の構造改革を発表した。 
\\	「骨太」を 
\\	(大胆な)、
\\	(補強された)で表現しています。	
\\	たまには温泉にでも入って、骨休めをしないとな。 
\\	がまさに「骨休め」の意味になります。
\\	は「羽を伸ばす、息抜きをする」の意。	
\\	彼は骨を埋めるつもりでアメリカに渡った。 
\\	ここでは「骨を埋めるつもり」を
\\	(2度と帰らないつもり)で表現しています。	
\\	見てみろよ、あいつ頬がこけちゃって。あの部署、仕事きついからな。 
\\	「頬がこける」を、ここでは、
\\	「骨と皮になってる」で表現しています。もう一つ
\\	「(骨身を削って)馬車馬のように働く」という言い回しも一緒に覚えておくといいでしょう。
\\	は
\\	をへとへとになるま	
\\	私の会社の命令なので、会社のホームページを明日中に見ないと、叱られます。 
\\	「ホームページを見る」は
\\	ですね。「ホームページ」は
\\	と言うことが多いようです。
\\	「困ったことになる」。	
\\	ああ、それは誉めすぎ。 それじゃ、一時はやった誉め殺しだよ。 
\\	この場合の「誉め殺し」は
\\	「べた誉め」を使って表現すればいいでしょう。
\\	「政界」、
\\	「お世辞を言う」。	
\\	やるだけのことはやったと自分を誉めたい。 
\\	で「自画自賛する」という意味になります。	
\\	この壺は掘り出し物だね。どこで手に入れたんだい。 
\\	には名詞で「掘り出し物」という意味があります。	
\\	今ごろそんな昔の事を掘り出さないでよ。もう終わった事じゃない。まだ、根にもっているの?しつこいなぁ。 
\\	で「昔のことを掘り出す」の意味になります。
\\	で「根に持つ」の意。	
\\	まあ、彼には嫌なところもいっぱいあるけど、惚れた弱みだから仕方がないな。 
\\	で「好きになったのは私だから」で「惚れた弱み」のニュアンスが出せるでしょう。
\\	「何もかも」。	
\\	彼女、あんなださい奴にどうして惚れたんだろう 
\\	で「好きになる、惚れる」という意味になります。
\\	「ださい奴」。	
\\	あの映画のラストシーンにはホロリとさせられるよね。 
\\	は「涙がこみ上げてくる」、
\\	は「まばたきをする」の意。	
\\	それは私にとっては本意ではありません。 
\\	ここでは「本意」を
\\	(意味する)、
\\	(意図する)で表現しています。	
\\	昨日はいつもより2本遅い電車で帰りました。 
\\	で「2本遅い」を表しています。
\\	とも言えます。	
\\	もちろん学生時代に英語は勉強したけど、本格的に始めたのはここ2,3年くらい前です。 
\\	「本格的に... する」は
\\	「真剣に取り組む」で表現すればいいでしょう。	
\\	香港返還後、カナダに移民した香港人が多くいるらしい。 
\\	「返還」は
\\	を使って表現します。
\\	「移民する」、
\\	「復帰」。	
\\	本腰を入れて仕事をしないと、期日までに終わらないぞ。 
\\	で「本気になって取り組む」の意味になります。	
\\	お前にはこの問題の本質が分かっていないんだよ。 
\\	「本質」は
\\	、あるいは、
\\	「本当の問題」でいいでしょう。	
\\	オープンカーのルーフと本体って一緒にセットで売られているんじゃないの?別々? 
\\	この場合の「本体」は 
\\	でいいでしょう。
\\	は「別々」の意。	
\\	前置きはこれくらいにして、本題に入りましょう。 
\\	とは「本題、主要な問題」ということです。
\\	「中心の」。
\\	「前置き」、
\\	を省く」。	
\\	早速本題に入るけど、 飲み会の件なんだ。今週の金曜日あいてるかな。 
\\	は「聞きたかったことを直ぐに聞く」の意味ですね。	
\\	本当にごめんね。 
\\	この場合の「本当」は
\\	でいいでしょう。	
\\	あの時、彼女に本当の気持ちを言えばよかった。今となっては手遅れだが。 
\\	「本当の気持ち」は
\\	でいいでしょう。
\\	「まさに…すべき瞬間」。	
\\	本当は彼女の方が、いいって思ってるんじゃない? 
\\	は「これって~だというあなた流の言い方なんじゃないの」の意味になります。この中に「本当は」が含まれているのです。
\\	ももちろん使えます。	
\\	あーもう散々だった。雪で電車は止まるし、寒いし。ホントだったら、もう家に帰ってゆっくりできている時間だ。 
\\	ここでは「ホントだったら」のニュアンスを 
\\	(普段は、普通は)を使って表現しています。	
\\	是非、本音を聞かせて下さい。 
\\	「本音」とは
\\	「本当に感じていること」で表現すればいいでしょう。	
\\	日本人は本音と立前を上手に使って人との調和を崩さないようにしている。 
\\	「本音と立前」を英語にするには一工夫する必要があります。ここでは「本音」を
\\	「裏表がない、正直な気持ち」、「立前」を
\\	「表向きの儀礼」で表現しています。
\\	「形だけの行為、ことば」。	
\\	ほんの気持ちですから。ぞうぞお納めください。 
\\	「ほんの気持ち」は
\\	「これあなたに」と簡単に表現すればいいでしょう。	
\\	彼は子供の頃から本の虫だったからね。いまでも毎月相当な数の本を読んでるんじゃないかな。 
\\	が「本の虫」に当たる名詞です。	
\\	やっぱり本場北海道のシャケって味が違うね。 
\\	とは「それが取れた場所
\\	という意味です。	
\\	日本のインドカレーは日本人に合わせて作られてるから食べれるけど、本場のインドカレーだったら辛すぎで食べれないと思う。 
\\	「本場の」は 
\\	(本物の)で表現すればいいでしょう。
\\	を使ってもいいですね。	
\\	今でもう寒いなんて、まだまだ甘いね。冬はこれからが本番だよ。 
\\	「冬本番」は
\\	「定まる、落ち着く」を使って表現すればいいでしょう。	
\\	(面接の)本番に強いなんて羨ましい限りだ。私なんかあがっちゃって何いっているかわからない時がある。 
\\	「本番」をここでは
\\	「実際の」を使って表現しています。	
\\	彼は本番に強いから、そんなに心配しなくていいよ。 
\\	(うまくやる)と
\\	(その場で、現場で)を組み合わせて「本番に強い」を表現しています。
\\	と 
\\	の使い方にも注目。	
\\	私、子供の頃から本番に弱いのよね。 
\\	で「大事な時に、結果を出すのが得意ではない」という意味になります。
\\	と
\\	の組み合わせた表現の仕方にも注目。	
\\	学生の本分にもとることのないよう注意しなさい。 
\\	ここでは「本分」を 
\\	(責任)で表現しています。
\\	は「そそのかして…させる」の意。	
\\	それじゃ本末転倒じゃないか。 
\\	とは「逆さまにする」の意味。	
\\	熱があるせいで、なんか頭がぼーっとしている。 
\\	に「ぼーっとした」という意味があります。
\\	「ぼーっとした状態」。	
\\	たまには何も考えずにボーっとしているのもいいよね。 
\\	は「静かに、ボーっとしている時間を持つ」、
\\	は「何も考えずに」の意味。	
\\	すみません。ぼーっとして聞いていませんでした。 
\\	「ボーっとする」は
\\	、あるいは、
\\	「お昼を食べに行っている」で表現します。	
\\	雨の日は、いつもよりさらに輪をかけて、何もやる気がしない。家で、ボーっとしてたいなぁ。 
\\	は「ダラダラする、ゴロゴロする」、
\\	は「ブラブラする」の意味。	
\\	返事が遅れて申し訳ありませんでした。このところ、仕事に忙殺されておりまして。 
\\	は「仕事で身動きがとれない」、
\\	「仕事の山に埋もれている」の意味になります。	
\\	寝不足で頭がボーッとする。 
\\	が「ボーッとする」に当たる表現です。
\\	で「頭の働きが鈍い」の意	
\\	ここ2週間疲れて何もする気がしません。でも早く予定を立てないと何もせずボーっと来月過ごすのも嫌だし。 
\\	「ボーっと」を 
\\	(無気力で)、
\\	(ぼんやりする)を使って表現しています。	
\\	そろそろ忘年会のシーズンだね。 
\\	「忘年会」は
\\	と言います。	
\\	朝しっかり剃っても、時間が経つに連れてひげが生えてくる。夜には、ぼうぼうだ。 
\\	(伸びる)と 
\\	(濃い)を組み合わせて「ぼうぼう」を表現しています。
\\	は「ぼさぼさ(もじゃもじゃ)に見える」の意。	
\\	株価が暴落したために、多くの会社が倒産した。 
\\	は「急落する」、
\\	は「暴落する」の意味。	
\\	最近、物忘れがひどい。ボケてきたかな。 
\\	「ぼける」は
\\	で表現できます。
\\	「アルツハイマー病」。	
\\	あのコンビのボケとツッコミは絶妙だ。 
\\	まず、2つの表現の説明を見て下さい。
\\	「すべてを言葉通りに受け取るタイプ」。
\\	「相手の言葉を茶化すタイプ」。この説明から分かるように、
\\	が「ボケ」、
\\	が「ツッコミ」ということになります。	
\\	新聞に載っていた募集の広告を見て履歴書を持ってきたのですが、まだ募集していますか? 
\\	ここでは 
\\	(雇用してもらうための申し込みを受けつける)を用いて「募集している」を表現しています。	
\\	結構きつい仕事だからあまりなりたいと思う人がいないのかもね。だって、よく募集しているポスター見かけるし。 
\\	で「募集するためのポスター」の意味になります。	
\\	私、今、ボーイフレンド募集中です。 
\\	は「…を求めている」の意味です。	
\\	ワイシャツのボタンが取れちゃったから、つけておいてね。 
\\	は「取れる、はずれる」の意味。
\\	で「もぎ取る」の意。	
\\	その議論は、最初からボタンを掛け違っていて、かみ合わなかった。 
\\	で「へたに始める」の意味。
\\	「うまく計画されていない」。	
\\	増上寺は、徳川家の菩提寺である。 
\\	「菩提寺」は
\\	と言います。
\\	は「一族」の意。	
\\	まあ、焦らずぼちぼち行きましょか。 
\\	は「ゆっくりやる」、
\\	は「のんびり行く」の意。	
\\	ぼちぼちですよ。仕事の方も適当に頑張っているし。何よりも今は、休みが欲しい。 
\\	「少しずつ」で「ぼちぼち」という感じを表現しています。
\\	があったらいいなあ」。	
\\	元気そうな奴ほど、ある日突然ぼきっと折れてしまう。 
\\	「ぼきっと折れる」は
\\	「(肉体的、精神的に)参ってしまう」、
\\	「重病になる」ということですね。	
\\	なにげなく入った飲み屋で、ひどいボッタクリに遭った。 
\\	が「ボッタクリに遭う、ぼられる」に当たる言い回しです。
\\	は「戯れに」の意。	
\\	彼は彼女のことをいつもぼやいている。 
\\	「ぼやく」は
\\	「不満を言う」、
\\	「嘆く」を使えばいいでしょう。	
\\	ぼられちゃったよ。 
\\	「ぼられる」に当たる英語は
\\	といいます。第2文の意味は「彼に2倍も払わせられた」の意味になります。	
\\	(お仕事は何ですか)ボランティアで通訳をしています。 
\\	「ボランティア」はそのまま
\\	が使えます。
\\	を動詞として用いて、
\\	のようにも言えます。	
\\	しゃべらなきゃいいのに、ついしゃべってしまってぼろが出た。 
\\	ここでは「ぼろが出る」を
\\	「何を話しているか分かっていないことば皆にばれてしまった」と言い換えて表現しています。	
\\	(スピーチの後で友人に)最初の方はよかったんだけどさ、途中からぼろが出ちゃって、あまり練習してないことがばれちゃったよ。 
\\	(途中で心もとなくなる、ぐらつく)を使って「ぼろが出る」を表現しています。
\\	は「ばらばらになる」の意。	
\\	そのパン、ボロボロ崩れるから気をつけな。 
\\	は「ボロボロに崩れやすい」の意味です。
\\	はその動詞形です。	
\\	あいつでかいことばかり言ってるけど、そのうちぼろを出すよ。 
\\	「ぼろを出す」を
\\	「彼の欠点がさらされる、丸出しになる」、
\\	「彼の弱点が出る」で表現しています。	
\\	彼女はアルコールが入ると「キス魔」になります。こわいね。 
\\	「キス魔」は
\\	「みんなにキスをし始める」と言い換えればいいでしょう。	
\\	まだ何か話すことなかったかな。ま、いいや。ごはんにしよう。 
\\	「ま、いいや」は
\\	か
\\	「気にしないで」を使って表現すればいいでしょう。	
\\	魔の月曜日だ。月曜日の事を考えると、胃が痛くなる。 
\\	「魔の月曜日」は 
\\	でいいでしょう。
\\	の動名詞の使い方にも注目です。	
\\	世の中にはいろんな人がいるし、それぞれ考え方も違うからまあいいんじゃないの。 
\\	ここでは「まあいい」を 
\\	と 
\\	を使って表現しています。	
\\	で900点をマークするのが目標です。 
\\	そのまま
\\	を使うことができませんので、
\\	「点を取る」でマークの意味を表現しています。
\\	目指す」。	
\\	彼はその知らせを聞いて舞い上がった。 
\\	「舞い上がる」は 
\\	で表現できます。
\\	でも同じようなことがいえます。	
\\	迷子にならないように、ちゃんとお母さんの手を握ってるのよ。 
\\	「迷子になる」は
\\	と言います。
\\	にしっかりつかまる」。	
\\	シャトルバスは毎時00分、15分、30分、45分と、ちょうど15分おきにやってきます。 
\\	この場合の「毎時」は
\\	で表現できます。	
\\	舞い散る桜吹雪を見ながらどんちゃん騒ぎをするのは、今や上野公園の風物詩だ。 
\\	で「ヒラヒラ舞い落ちる」の意味になります。
\\	は「花びら」の意。「風物詩」の表現の仕方にも注目。	
\\	まいったな。あなたには何でもお見通しなんですね。 
\\	「参った」は
\\	で表現できます。	
\\	まいどっ。もうかりまっか? 
\\	「まいどっ」は
\\	!「また会ったね」でいいでしょう。
\\	「稼いでる?」、
\\	「商売の方どう?」が「もうかりまっか」に近い言い回しですね。	
\\	彼っていつもマイペースよね。 
\\	を使う場合には前置詞が
\\	になる点に注意!	
\\	このホテルの宿泊代もマイレージに積算できるんですか? 
\\	「マイレージ」は、文脈によって
\\	を使い分ければいいでしょう。	
\\	相撲を見に行きたいんだけど,前売り券を買わなくても,当日券でも見れるかな。 
\\	で「その場でチケットを買う」の意味。
\\	は「当日券を買う」の意。	
\\	アメリカでいろいろ起きているけど、留学はずっと前から決めていたことだから今更やめるつもりもない。確かに不安はあるけど。 
\\	で「ずっと前から決められている」の意味になります。
\\	は「ずっと前から…することを計画してきた」、
\\	は「懸念」の意。	
\\	前髪が邪魔だ。前髪がうっとおしい。 
\\	が「前髪」に当たる表現ですが、第一文のように
\\	とも言えますね。	
\\	政府は、その事業を前倒しで行うという考えを示した。 
\\	は「繰り上げる」という意味です。
\\	「時期を早める」。	
\\	前の方の席まだ空いてますか?で、できれば、通路側からの席3つお願いしたいのですが。 
\\	「前の方」は
\\	を使えば表現できます。
\\	「通路」。	
\\	あの映画は前評判通りの面白さだった。 
\\	が文字通り「前評判」という意味になります。	
\\	前々から気になっていたんだけど、海外では手紙とかで夫婦の名前を書く時、女性の名前を先に書くの? 
\\	で「前からずっと思っていた」の意味になります。現在完了を使うことで「前々から」が表現できます。	
\\	彼の何事にも前向きに取り組むという姿勢がいいよね。 
\\	で「前向きな(積極的な)姿勢」という意味になります。	
\\	お店を一人で任されることはそれだけ信用されているということ。その反面売上は自分にかかっていると思うとかなりのプレッシャー。 
\\	で「~に 
\\	の責任を任せる」の意味になります。
\\	と 
\\	の使い方にも注目。	
\\	猫の世話は、全部親に任せっきりにしたから親も大変になって保健所へ連れていったのだと思う。 
\\	~で「…を~に完全に任せる」の意味になります。
\\	は「動物収容所」の意。	
\\	このプログラムの講師など外注しなくとも、学内の教員で充分賄えるんだという事を示すべきだ。 
\\	「賄える」とは
\\	(…を処理することができる)ということですね。	
\\	こんな理不尽な事がまかり通っていいのですか? 
\\	「まかり通る」とは、「そのままにしておく(行かせる)」ということですから、
\\	を使えばいいでしょう。
\\	「不公平」、
\\	「道理をわきまえていない」の意。	
\\	魔が差したとはいえ、してはならないことをしてしまった。 
\\	「魔が差す」は
\\	「誘惑に負ける」と言い換えることができます。
\\	は「誘惑に屈する」の意味。	
\\	俺は曲がったことが嫌いなんだ。 
\\	ここでは「曲がったこと」を
\\	「不正直な」、
\\	「不公平な」で表現しています。	
\\	僕の父は曲がったことをするのが大嫌いだ。 
\\	「曲がったこと」を 
\\	(不正直な)、
\\	(不正直)を使って表現しています。	
\\	その家を訪ねたら、夫婦喧嘩の最中で、間が悪かった。 
\\	で「悪い時に」の意味。
\\	「ばつが悪い思いをする」。
\\	「いさかい」。	
\\	交通渋滞に巻き込まれて、30分で着くところが、1時間半もかかってしまった。 
\\	とは「身動きが取れなくなる」ということです。	
\\	報復といっても、市民を巻き添えにしたらテロと同じことになる。何かほかにいい方法はないものか。 
\\	で「…に巻き込まれる」の意味になります。
\\	は「巻き添えにする」の意。	
\\	このデータには、異常値が紛れ込んでいる。 
\\	で「…にいつのまにか入り込む」の意味になります。
\\	は「異例の」、
\\	は「不適切な」の意。	
\\	ズボンのすそをまくり上げていてもすぐ落ちてくる。 
\\	「まくり上げる」は
\\	でいいでしょう。	
\\	ホームランを打てたのはまぐれですよ。 
\\	が「まぐれ」に当たる表現です。	
\\	君は昔っから負けず嫌いだからな。 
\\	「負けるのが嫌い」を現在完了形にして「昔からずっと」の意味を表しています。
\\	「我慢する」。	
\\	馬子にも衣装よねえ。 
\\	「馬子にも衣装」には
\\	「着ている服装でそれなりに見える」という決まった言い方があります。	
\\	びっくりした!まさかこんなとこで会うとは思わなかったよ。 
\\	ここでは「まさか」を
\\	でそのニュアンスを表現しています。	
\\	まさかぁ、そんなことあるわけないじゃない。 
\\	「まさか」は動詞 
\\	(からかう)、
\\	を使って表現します。	
\\	いつスノボー行ったの?まさか、私が仕事してた正月休みなんて言わないよね? 
\\	「まさか」は
\\	「(まさか)私に 
\\	だとは言わないでしょうね」で表現できます。
\\	は「たたみこむように言う」。	
\\	サンマが夕飯にでるなんて、まさに秋って感じだね。 
\\	「…だとわかる」、
\\	で「まさに…」を表現しています。	
\\	以前は旅行に行くとデジタルカメラで撮ってたけど、全然整理してない。撮ったら撮りっぱなしで、まさに怠け者でしょ? 
\\	「まさに怠け者」をここでは 
\\	で表現しています。
\\	は「整理(分類)する」の意。	
\\	あの人、ちょっとマザコンじゃない。 
\\	は「お母さん子、マザコン男」、
\\	は
\\	にっべったり」の意味。	
\\	やらないで後悔するぐらいなら、やってみて後悔したほうがマシだ。 
\\	「まし」は
\\	「よりいい」で表現できます。
\\	「前進」。	
\\	仕事の能力は彼より劣るかもしれませんが、人間的には彼よりマシだと思います。 
\\	とは「より人格が伴っている、誠実さがある」の意味。	
\\	月曜日の仕事のことを考えると、気が重い。こんなことなら、学校に通ったほうがましだ。 
\\	「まし」は 
\\	(むしろ…したい)で表現すればいいでしょう。	
\\	2つ仕事を持たないと暮らしていけないなんてちょっと悲しいです。まだ、日本のほうがマシな気がします。 
\\	ここでは「まし」を 
\\	で表現しています。	
\\	由美ちゃん飲み会に来ないの?まじかよ。 
\\	「まじ」は
\\	「まじめ、本気」で表現します。	
\\	真面目にやってれば、きっといいことあるよ。 
\\	とは「自分に正直になる(頑張る)」の意味ですね。
\\	「うまく収まる」、
\\	「成功する」。	
\\	いつになく真面目な顔をして、どうしたんだい。 
\\	共に「真面目な表情」という意味。	
\\	歯医者で麻酔を打ったから、口がよく回らない。 
\\	とは麻酔薬の名称です。
\\	は「麻酔」の意。	
\\	おせじでも誉めてくれるなんてうれしいよ。ますますやる気がでてきたよ。 
\\	ここでは「ますますやる気が出る」を
\\	(本当に励みになる)で表現しています。	
\\	それはまずいことになったね。 
\\	「まずい」は
\\	「事態、状況がよくない」、
\\	「大変だ」で表現できますね。	
\\	まずいなー。ここ一カ月英語を口にしてない。そろそろ家でも練習しないと今まで覚えてきたことを忘れてしまいそうだ。 
\\	「まずいな」のニュアンスを 
\\	で表現しています。	
\\	私が言ったことまずかったかな? 
\\	「まずい」を 
\\	で表現しています。
\\	は「怒らせる」の意。	
\\	私達は、またたく間に、恋に落ちました。 
\\	は「知らぬ間に」、
\\	は「考える時間などなく」の意味になります。	
\\	すみません、まだですか?ずいぶん長いこと待っているですけど。 
\\	が「まだですか(出来上がってないのですか)」に当たる表現です。
\\	は「順番」の意。	
\\	まだ10時なの? 
\\	この場合の「まだ」は 
\\	を使えばいいでしょう。	
\\	旅行の詳しい日程が決まるのはまだこれからです。 
\\	(まだ…していない)で「まだこれから」を表現しています。	
\\	英会話の授業にはやっと慣れたけど、ちゃんと話せるようになるのはまだまだ先のようだわ。 
\\	「まだまだ」は
\\	するには長い時間がかかる」で表現すればいいでしょう。	
\\	アメリカドラマの内容が完璧に理解できるようになるには文化背景等を知る必要があるわけだから、まだまだ時間がかかりそうだな。 
\\	で「…できるまでにはまだまだだ」の意味になります。
\\	で「そこに到達するにはかなり時間がかかる」の意。	
\\	実際引越すのは冬になってからだから、まだまだ先のこと。また、近くなったら引越しの手伝いで連絡するよ。 
\\	「まだまだ先」は 
\\	で表現すればいいでしょう。
\\	は「連絡をする」の意。	
\\	図書館の前に3時に待ち合わせをしよう。 
\\	「待ち合わせる」は
\\	「会う」を使えばいいでしょう。	
\\	え?一週間に10冊も本を読むの? 一ヶ月の間違いじゃないの? 
\\	で「一ヶ月ということじゃないの」の意味になります	
\\	は 
\\	の間違いだと思います。 
\\	この場合の「間違い」は「綴りの間違い」ですから、
\\	、あるいは、動詞
\\	を使えばいいでしょう。	
\\	どうやら、間違い電話のようですね、お掛け直しください。 
\\	「間違い電話です」は 
\\	と言います。	
\\	このコースに登録した人は、間違いなく必ず授業に出てください。 
\\	(必ず…するように注意する)で「間違いなく」を表現しています。
\\	は「…に登録する」の意。	
\\	アクセルとブレーキをよく間違えそうになる。だから自分でも恐くてあまり運転は気が進まない。 
\\	で「…を間違いそうになる」の意味になります。
\\	「…する気分」。	
\\	メールのアドレス間違えてますよ。 
\\	で「…を間違える」の意味にまります。	
\\	やっと会えた。 この時(日)を、どれだけ待ち焦がれたか! 
\\	「待ち焦がれる」は
\\	と現在完了を使えばそのニュアンスが出せます。	
\\	待ち時間はどれくらいですか。 
\\	には名詞で「待つ時間」という意味があります。	
\\	早く読みたくてその本が出版されるのが待ち遠しいんだ。 
\\	「待ち遠しい」とは 
\\	(待ちきれない)ということですね。	
\\	後3日で待ちにまったゴールデンウィークだね。思いっきり寝られるぞ。 
\\	で「長い間待っていた(心待ちにしていた)」の意味になります。	
\\	彼のいいとこは、助けが必要な人がいると、いつも真っ先に駆けつけてくれるところね。 
\\	で「真っ先にやって来る人」の意味になります。	
\\	その企業は、不況の波をもろに受けて倒産へまっしぐらだ。 
\\	「まっしぐら」は
\\	(まっさかさまに)でそのニュアンスが出せるでしょう。
\\	(一斉攻撃)。	
\\	緊張のあまり頭の中が真っ白になり、自分が何をしているのか分からなくなってしまった。 
\\	が「頭の中が真っ白になった」に当たる言い回しです。
\\	「圧倒する」。	
\\	まったく!あなたって困った人ね。 
\\	(本気でそう言っている」、
\\	!「信じられないよ」で「まったく」のニュアンスを表現しています。	
\\	それは、私達日本人には、全く考えられないことです。 
\\	「考えられない」は
\\	(形容詞)を使えば表現できます。「全く」は
\\	「完全に」、
\\	「絶対に」を使えばいいでしょう。	
\\	その法案成立に待ったをかけたのは、共和党であった。 
\\	で「…にブレーキをかける」の意味。
\\	は「やれるならやってみろと挑む」の意。	
\\	速い球だったので、バットが真っ二つに折れた。 
\\	で「あざやかに2つに折れる」の意味になります。
\\	は「…を真っ二つに割る」の意。	
\\	彼の意見は非常に的を射ていると思います。 
\\	が「的を射る」に当たる表現です。	
\\	あれ? もしかして、私さっき全然的外れな答えをした?ごめん!なんかあまり考えずに適当に答えちゃった。 
\\	「的外れ」は
\\	で表現できます。
\\	「見当違い」。	
\\	うちのクラスって、みんな仲が良くって、まとまってるよね。 
\\	「しっかりと結束した」を使って「まとまっている」を表現しています。
\\	「一緒にくっついている」。	
\\	この文章はまとまりがないね。何が言いたいのかまったくわからないよ。 
\\	で「焦点がぼけている、まとまりに欠ける」の意味。
\\	は「とりとめのない」の意。	
\\	2人分まとめて送金したので、手数料は一人分でいいかと思ったのですが。 
\\	この場合の「まとめて」は
\\	「一回の処理で」で表現すればいいでしょう。	
\\	この委員会のまとめ役は彼が適任だと思います。 
\\	この場合の「まとめる」は
\\	、あるいは、
\\	でいいでしょう。	
\\	ある物事に対して、自分の意見をまとまるというのは、難しいことだと思います。 
\\	この場合の「まとめる」は
\\	「形成する、形作る」で表現できます。	
\\	今日ゴミの日だから、まとめといて!捨ててきてくれてもいいけど。 
\\	「(ゴミを)まとめる」は
\\	「袋に入れる」、
\\	「束にする、包む」で表現できます。
\\	「ゴミの日」には冠詞
\\	は必要ありません。	
\\	スカートがまとわりついて歩きにくい。 
\\	で「…にくっつく」という意味。
\\	は「(静電気による)まといつき」の意味。	
\\	お会計は患者用窓口でお願いします。 
\\	で「患者用窓口」の意味になります。	
\\	社の電話窓口は細かい対応をしてくれるので使いやすい。 
\\	ここでは「電話窓口」を 
\\	で表現しています。
\\	は「良心的な」、
\\	は「几帳面な」の意。	
\\	そんなまどろっこしい言い方しないで、はっきり言えよ。 
\\	で「遠回しに言うのをやめる」の意味になります。
\\	「遠回りの」。	
\\	今の私はまな板の上の鯉です。 
\\	「まな板の上の鯉」の状態を 
\\	(私にできることは何もない)で表現しています。
\\	と 
\\	の使い方に注目。	
\\	今日、会社間に合った?間に合わないんじゃないかと心配してたんだ。 
\\	で「時間通りに」の意味。
\\	「間に合う」。	
\\	ま、物事はマニュアル通りにはいかないよなぁ。 
\\	この場合、
\\	は使えません。物事を運ぶための具体的なマニュアルはありませんから。
\\	のように「計画通りに」を使うのが自然です。コンピュータのようにマニュアルがある場合には、例えば、
\\	「コンピュータは、大抵、マニュアル通りには動かない」のように言います。	
\\	日本人って人のまねをすることが好きで、それでもってそれが格好いいと思っている。 
\\	「まねをする」は
\\	を使えばいいでしょう。	
\\	写真を撮りますから、まばたきするのを我慢してください。 
\\	は動詞で「まばたきする」の意味。	
\\	さっき、臭かったけど、だんだん鼻が麻痺してきて、感じなくなっ た。 
\\	「麻痺する、感覚がなくなる」、あるいは、
\\	に慣れる」を使えばいいでしょう。	
\\	新芽が出てきたら、間引きしてあげるとよく育つよ。 
\\	「間引き」は
\\	「木を刈り込む」で表現できます。
\\	「形成する」、
\\	「新芽が出る」、
\\	「周りを薄くする(刈り込む)」の意。	
\\	(光が)まぶしくて前が見えない。 
\\	「まぶしい」は
\\	「まぶしい光」を使えばうまく表現できます。	
\\	試験前、3日連続徹夜したせいで、まぶたが重くって、もうくっつきそうだよ。 
\\	「(まぶたが)重い」は
\\	を使って表現できます。
\\	「くっつく」も一緒に覚えておくといいでしょう。	
\\	眠たくてまぶたがくっつきそうだ。 
\\	「瞼」は
\\	、「閉じる」は
\\	ですね。
\\	は「ぴったりくっつく」の意。	
\\	最後のコンサートの映像で、幻の映像と言われているくらい貴重なものなんだよ。あなたには、わからないとは思うけど。 
\\	「幻」を 
\\	(幻影)を使って表現しています。
\\	は「幻影の」の意。	
\\	それって、フロッピー入れたままパソコン立上げたからエラーがでたんじゃないの? 
\\	「まま(状態で)」は前置詞
\\	を使って表現できます。	
\\	バブルのときは経理で飲み代持ってくれたけど、今では500円以下の昼飯代もままならない。 
\\	「ままならない」のニュアンスを 
\\	で出しています。
\\	は「バブル経済」の意。	
\\	彼女ってまめだね。デスクの上をきちんと整理整頓して。 
\\	で「ぐうたらではない」の意味になります。
\\	「きちんと」、
\\	「全力を傾ける」、
\\	「整然として」。	
\\	テニスをやりすぎて、右の手のひらに豆ができてしまった。 
\\	「(硬くなった)豆(たこ)」のことを
\\	と言います。「水ぶくれ」は
\\	といいます。	
\\	これは昔の人が、子ども部屋に飾って、悪夢から子どもたちを守ろうとして使われていたものです。 
\\	この場合の「守る」は 
\\	、あるいは、
\\	(寄せ付けない)で表現すればいいでしょう。	
\\	え?サッチーがコロンビア大をでてるって?ウッソー、まゆつばもんよねえ。 
\\	「まゆつばもの」とは
\\	「いんちきの、見せかけの」ということです。
\\	は「ほら、大風呂敷」の意。	
\\	いつも個人で行こうかツアーで行こうか迷うのだけど、結局ツアーの方が安いからそっちにしちゃうんだよね。 
\\	でああだこううだと悩む(行ったり来たりする)」で「迷う」を表現しています。
\\	「いろいろと考える」。	
\\	丸暗記じゃ、すぐ忘れます。まずは内容をよく理解すべきだね。 
\\	「丸暗記」をここでは
\\	の2通りで表現しています。
\\	「乱暴な」の意。	
\\	あそこに座っている丸顔の女性の名前を教えてくれないか。 
\\	とは「丸顔の女性」のことです。
\\	は「丸みを帯びた」の意。	
\\	私さえ我慢すれば全て丸くおさまるのよね。 
\\	「全て丸くおさまる」は
\\	「全てが結果としてうまく行く」、
\\	「全てのことがおさまる、安定する」ということですね。	
\\	彼はその場をなんとか丸く収めようとしたが、うまく行かなかった。 
\\	が「丸く収める」に当たる言い回しです。
\\	「和解」。	
\\	年取ったからといって、まだまだ丸くはなりたくないね! 
\\	「軟弱になる」、
\\	「棘がなくなる
\\	の意味。	
\\	丸首の黒のセーターはありますか。 
\\	「丸首」は
\\	と言います。	
\\	僕をマルチ商法に巻き込むのはやめてくれ! 
\\	とは「ネズミ講方式」のことです。	
\\	まるで、あなたがすぐそばにいるようです。 
\\	「まるで」は
\\	「あたかも...」、
\\	「丁度...のようだ」を使えばいいでしょう。	
\\	彼女はもう40近いのに、行動、仕草や考え方がまるで子ども。そういうの見てると、時々イライラする。 
\\	ここでは「まるで子ども」を 
\\	で表現しています。
\\	は「(言動などの)癖、特徴」の意。	
\\	特に丸秘事項ってわけではないのだけど、専門的な話だから一般の人が聞いても理解しにくいじゃないかと思う。 
\\	、あるいは、
\\	が「最高機密」の意味を表しますので、これを使えばいいでしょう。	
\\	このシチュー、味がとってもまろやかね。 
\\	には「まろやかな、熟した、芳醇な」という意味があります。	
\\	痴漢にまわしげりをくらわした女子高生を電車の中で見た。 
\\	ここは「まわしげり」を
\\	で表現しています。
\\	「回る」、
\\	「痴漢する」、
\\	「撫で回す」。	
\\	あんたこの番組をいやに勧めるけど、ひょっとしてテレビ局の回し者か? 
\\	ここでは「回し者」を
\\	(スパイ、職員)を使って表現しています。
\\	でも同じようなことがいえますね。	
\\	回りくどい言い方はやめて、用件を早く言ってくれよ。 
\\	「回りくどい言い方は止めろ」には
\\	という決まった言い方があります。
\\	は「たわごと、くそ」、
\\	とは「要点を言え」の意。	
\\	毎日、満員電車に揺られて通勤か。これじゃ体がもたないよ。 
\\	「満員電車」は
\\	「ぎっしり詰まった」を使って
\\	と言います。	
\\	いいなー佐藤さん。今ごろハワイの海を満喫してるんだろうな…、こっちは仕事してるっていうのに。 
\\	が「満喫する(思う存分楽しむ)」に当たる言い回しです。	
\\	彼は女性に囲まれて、まんざらでもない様子だ。 
\\	ここでは「まんざらでもない」を
\\	「苦痛を感じない」で表現しています。
\\	「悪くない」。	
\\	すみません。今、満席なんです。20分ほどお待ちいただけますか。 
\\	が「満席、満室」に当たる表現です。	
\\	あなたのあのアパートの満足度はどのくらい?100%満足の部屋ってそう簡単には見つからないよね。 
\\	「満足度」は 
\\	と 
\\	(満足)を組み合わせて表現すればいいでしょう。
\\	は「…に満足する」の意。	
\\	ガソリンを満タンにして、ついでにオイルも見てくれる。 
\\	「満タンにする」は
\\	ですね。	
\\	あの先生の授業はマンネリ化してて、新鮮味がない。 
\\	ここでは「マンネリ」を 
\\	(一本調子のサンバのようだ)で表現しています。
\\	は「お決まりのやり方で」の意。	
\\	あの俳優は、もういい年なのに、まるで万年青年だ。 
\\	ここでは「万年青年」を
\\	(年を取っているようには見えない)を使って表現しています。
\\	「若者」の意。	
\\	最終レースで万馬券が出たんだってよ。 
\\	英語では「万馬券」という言い方はしませんので(当然ですが)、ここでは
\\	「大穴」を使って説明しています。	
\\	コンビニで万引きをして捕まったって?どうしてそんなことしたの。親として恥ずかしいよ。 
\\	「万引き」は
\\	と言います。
\\	は「恥ずかしい」の意。	
\\	塩をふる際は、mの字を書くように様にふるとまんべんなく行き渡ります。 
\\	の文字通りの意味は「カバーされない所はない」になります。
\\	「パラパラと振りかける」。	
\\	試合の後、彼は満面の笑みで、喜びの気持ちを語った。 
\\	で「喜びで顔を輝かせる」の意味になります。	
\\	いよいよ満を持して、ジャックの登場です。 
\\	とは
\\	(今度は彼の番です)の意味。
\\	(彼がずっと待っていた瞬間)で「満を持して」を表現しています。	
\\	何だか、ミーハーな人たちばっかりね。 
\\	は「低俗な」という意味。
\\	「洗練されているところ」。	
\\	ほとんどの飛行機は、年をまたがる日、つまり12月31日から1月1日に変わる時間帯は、運転を見合わせるって聞きました。 
\\	この場合の「見合わせる」は
\\	「延期する」でいいでしょう。
\\	「またがる」。	
\\	思わず見入ってしまって、すっかり写真を撮るのを忘れていました。 
\\	この場合の「見入る」は、「その景色に圧倒される
\\	、魅せられる
\\	と言い換えて表現しています。	
\\	電車が込んでいて身動きが取れず、降りることができなかった。 
\\	は「(ちょっと)動く、身動きする」の意味。
\\	「すし詰めの」。	
\\	あまり物事にはまりすぎると、自分を見失ってしまうこともある。 
\\	「自分を見失う」は
\\	と言います。	
\\	高速道路建設については、党内の身内からも批判の声が湧き上がった。 
\\	とは「(政府、体制の)支持者」のことです。
\\	は「支持者、味方」の意。	
\\	騙されないぞ!そんな見え透いたお世辞には。 
\\	は「まやかしの、偽りの」の意味。
\\	「恩着せがましくする」。	
\\	追求されて、つい、見え透いた嘘をついてしまった。 
\\	とは「お見通し」の意味。また、
\\	は「透明な」の意味ですから、
\\	で「僕の嘘は見え見えだった」と言っているのです。	
\\	英語を話している時の自分が、馬鹿に見えてくる。 
\\	で「馬鹿に見える」の意味。
\\	「物笑いになる」。	
\\	(観光バスに乗っているとき)前に、美術館が見えてきた。 
\\	は「前方に見える」、
\\	は「見えてきた」の意味。	
\\	俺を馬鹿にしてきたやつらをいつかきっと見返してやるぞ。 
\\	「見返す」は
\\	で表現できます。
\\	「最後に笑う」。	
\\	英語のためなら惜しみなくお金を使ってしまう自分に気づいた。投資しただけの見返りがあればいいのだが。 
\\	ここでは「見返り」を 
\\	と 
\\	を使って表現しています。
\\	の使い方にも注目。	
\\	これは未確認情報だけど、うちの会社と
\\	社が合併するらしいんだ。 
\\	で「まだ確認していない」の意味になります。
\\	「合併する」。	
\\	高額で購入したアンティーク時計が、鑑定してもらったら、実は見かけ倒しのものだとわかった。 
\\	には「安物のコピー品」の意味があります。
\\	「偽物」。	
\\	彼って見かけによらずタフだよね。 
\\	「見かけ」は
\\	で表現できます。
\\	で「見た目よりずっとタフだ」の意味。	
\\	こわがらずに本当のあなたを見せて欲しい。私はいつだってあなたの味方よ。 
\\	で
\\	の味方」の意味になります。
\\	「あなたの後ろにいる」。	
\\	会話力ばかり磨いていないで、もっと人間を磨け。 
\\	ここでは「磨く」を 
\\	を用いて表現しています。
\\	は「才気あふれる」、
\\	は「教養のある」、
\\	は「洗練された」の意味。	
\\	寒い方が魚の身がしまっているから、冬の方が寿司はおいしいんだよ。 
\\	には「堅く引き締まる(形容詞)」という意味があります。
\\	で動詞としても使えます。
\\	「肉」。	
\\	週に4日アルバイトをしているので、疲れてしまって、勉強にあまり身が入らない。 
\\	と
\\	「気乗りのしない」の組み合わせがポイントです。
\\	は「使う、費やす」の意。	
\\	校長訓話を聞いて、皆身が引き締まる思いがした。 
\\	(元気づける、身を引き締めるような)、
\\	(精神を高揚させる)で「身が引き締まる」を表現しています。	
\\	もうこうなったら、計画を十分練っている余裕なんか全く無い。見切り発車でいこう! 
\\	で「さっさと取りかかって、仕事を進めながらなんとかやっていく」の意味になります。
\\	「仕事にとりかかる」、
\\	「先に進む」。	
\\	この間ビデオレンタルが1本100円だったので思わず20本も借りてしまったけど、結局全部見きれなかった。 
\\	で「全部を見る時間がなかった」の意味になります。
\\	は「結局…することになる」の意。	
\\	親たちが世の中の善悪を教えれば子供たちもその辺の見極めはできるようになると思う。 
\\	「見極める」とは、言い換えれば、
\\	「違いが分かる」ということですね。
\\	「決める、限定する」。	
\\	社長の右腕と言われるくらいだから、彼はかなりのキレ者だよ。 
\\	「右腕」は
\\	と言います。	
\\	数字の右肩に2と書くと、二乗という意味になります。 
\\	「右肩」は文字通りに
\\	と訳します。「二乗」の表現の仕方のも注目。
\\	「累乗」、
\\	「書く」、
\\	「右上に書く文字、記号」。	
\\	あの会社のオンラインショッピングの売り上げは、右肩上がりで増えている。 
\\	「右肩上がり」を
\\	「着実に増えている」で表現してみました。
\\	「景気が上向く」。	
\\	右目と左目で見え方が違うんですよね・・・(眼科で) 
\\	の使い方がポイントですね。	
\\	海外生活に関しては、右も左もわからない。けど、苦労したほうが、自分の能力を知ることができるし、人として大きくなれると思う。 
\\	「右も左もわからない」とは
\\	について何の手がかりもない」ということですね。
\\	「手がかりがない」。	
\\	彼って、神経質で怒りっぽくて、いつも眉間にしわを寄せてるよね。 
\\	「眉間にしわを寄せる」は
\\	「眉毛と眉毛の間に悩みによる皺を寄せている」を使って表現しています。
\\	「とても心配して」。	
\\	そのテニスの試合では、見応えのあるラリーが続いた。 
\\	「見応えがある」とは
\\	「見る価値がある」ということですね。	
\\	彼はその仕事を一人で見事にやってのけた。 
\\	は「素晴らしく」、
\\	は「模範となる」の意味。	
\\	どうやら裁判制度が、われわれにとって身近なものになるようだ。 
\\	(より近づきやすい)を使って「身近」を表現しています。
\\	は「…に関わる」の意。	
\\	今日の午後半日という本当に短いお付き合いでしたが、大変楽しかったです。 
\\	で「午後の短い間だけ一緒にいる」という意味になります。
\\	は「短い間」の意。	
\\	我々は将来を見据えつつ、今そこにある危機を乗り越えていかねばならない。 
\\	「見据える」は、
\\	をしっかり見る」で表現できます。
\\	「差し迫った」。	
\\	水掛け論ばかりしていて、とうとう結論が出なかったのは遺憾なことだ。 
\\	(意味のない議論)、
\\	(ぐるぐる回る)を使って「水掛け論」を表現しています。	
\\	政府は、ウイルスの進入を水際で阻止する方針だ。 
\\	「水際」は 
\\	(国境)で表現すればいいでしょう。
\\	は「…を締め出す」の意。	
\\	水臭いなあ、親子じゃないか。 
\\	「水臭い」をここでは
\\	「話せばいいだろ」、
\\	「遠慮するなよ」と言い換えています。また、この場面では、「親子じゃないか」を
\\	「僕は、お前の父親だろ」、
\\	「僕は、お前の父親じゃないのか?」で表現しています。ちなみに
\\	とは言いません(発音しにくいでしょう?)。	
\\	今日の会議でのことはお互い水に流して、今夜は楽しく飲みましょう。 
\\	で「過ぎてしまったことを水に流す」の意味になります。
\\	「起こる」。	
\\	これまでの努力が水の泡になってしまった。 
\\	は「無駄になる」、
\\	は「無に帰する」の意味。	
\\	あのグランドは水はけが悪いから、こんなに雨が降ったら、明日の試合は延期だな。 
\\	には「(土地が)排水される」という意味があります。
\\	「延期する」。	
\\	水も滴るいい女。 
\\	は「最高の」、
\\	は「気絶するような」の意。	
\\	ウイスキーの水割りをダブルでお願いします。 
\\	「ダブル」はそのまま
\\	を使って表現できます。
\\	は「ダブルにしてくれ」の意味ですね。	
\\	そうめんをおつゆにつけるとき、水をよく切らないとつゆがすぐ薄くなっちゃうよ。 
\\	には「水切りをする」という意味があります。
\\	は「バーミセリ(スパゲティーより細長いパスタ」の意。	
\\	未成年の飲酒は法律で禁止されている。 
\\	「未成年の」は
\\	、「未成年」は
\\	で表現できます。
\\	は「法律に反する」、
\\	は「法律で...を禁じている」の意味ですね。	
\\	店じまいセールということで、すべての商品を通常の半額以下で販売しております。 
\\	が「店じまいセール」に当たる言い回しです。
\\	「店じまいする」。「半額以下」の表現の仕方にも注目。	
\\	想像していた通りの国でした。町の中心部からは山が見え、海が見え、もうその風景には魅せられたました。 
\\	で
\\	を魅せる」の意。
\\	でうっとりする」。	
\\	みんなで遊んでいるのに、あの子だけみそっかすにされている。 
\\	「みそっかす」には
\\	「役に立たない奴」という言い方がありますが、ここは「みっそかすにされている」と動詞表現になっていますので、
\\	「追放する、相手にしない」で表現しています。	
\\	今日は会社さぼっちゃおうかな、みたいな。 
\\	「みたいな」は、文脈によって訳し分ける必要があります。上のような文脈の「みたいな」は「ならいいのに」の意味を表していますので、
\\	や
\\	あるいは、
\\	!のように言えばいいでしょう。	
\\	見たい番組に変えていいよ。 
\\	「見たい番組」をここでは
\\	「みたいものは何でも」で表現しています。	
\\	私はただ、あなたと一緒に話をして、テレビを見て、いちゃついて、満たされたいだけなの。 
\\	「満たされる」は 
\\	で表現すればいいでしょう。ここでは「いちゃつく」を 
\\	を使って表しています。	
\\	それって見たままじゃん。(例えば、丸いオブジェクトに名前を付けてと言われ、「丸ちゃん」とか答えたとき) 
\\	ここでは「見たまま」を
\\	「明らか、すぐに分かる」を使って表現しています。
\\	(描写)と
\\	を組み合わせても「見たまま」のニュアンスが出せます。	
\\	これ見た目よりずっとおいしいよ。 
\\	「見た目」は自動詞
\\	を使って表現できます。	
\\	ガラスが汚れていると飾っているものが映えないでしょ?けっこう通りがかりにちらっと見ていく人もいるから見た目は肝心だよ。 
\\	で「見た目(外見)はとても大切である」の意味になります。
\\	は「視覚効果」の意。	
\\	もんじゃ焼きって、味は悪くないんだけど、見た目がちょっとね~。 
\\	で「見た目が大してよくない」の意味。	
\\	忙しいのはわかるけど、もう少しみだしなみにも気を使いなさい。 
\\	で「身だしなみ(外見)に注意する」の意味。
\\	「身づくろいをする」。	
\\	中学校教師が生徒にみだらな行為をして逮捕された、という新聞記事を目にした。 
\\	「みだらな行為」をここでは
\\	「猥褻行為」、
\\	「性的暴力」で表現しています。	
\\	この話をみだりに他の部署の人間にしないように。 
\\	「みだりに」を 
\\	(私の許可なしに)、
\\	(私が言っていいといわなければ)を使って表現しています。	
\\	人身事故のため、電車のダイヤが乱れている。 
\\	ここでは「乱れる」を
\\	「混乱」を用いて表現しています。
\\	「死傷者」。「ダイヤ」は
\\	で表現します。	
\\	道なりにそのまま進むと左手に交番が見えます。 
\\	「道なりに進む」は
\\	、あるいは、
\\	でいいでしょう。	
\\	そのベーグル屋は、角の本屋の道を挟んで隣にある。 
\\	と
\\	の使い方がポイントですね。	
\\	今日は3日ぶりに晴れたね。 
\\	と 
\\	を組み合わせて「3日ぶり」を表現しています。	
\\	渋谷は新宿から3つ先の駅です。 
\\	「3つ先」は方向によって
\\	と
\\	を使って表現します。
\\	とも言えますね。	
\\	遅刻しないようにいつもより3つ前の電車に乗らなければならない。 
\\	「3つ前」は
\\	と
\\	を組み合わせて表現します。	
\\	人の見ている前で、そんなみっともないことをするな。 
\\	で「恥ずべき振る舞いをする」の意味。	
\\	そんな穴のあいたシャツ着ないでよ、みっともないから。 
\\	は「みすぼらしい」、
\\	は「薄汚い」の意味。	
\\	おつまみを何か適当にみつくろって出してくれるかな。 
\\	ここでは「みつくろう」を
\\	「あなたがいいと思う物」、
\\	「よさそうに思える物」と動詞を組み合わせて表現しているとことがポイントですね。	
\\	面白そうなクラスをガイドブックで見つけて、それには5月から開始って書いてたのに 
\\	では9月開始になってた。 
\\	「見つける」は 
\\	ですね。「書いていた」を 
\\	で表現している点にも注意。	
\\	アクセサリーはあまりつけるタイプではなかったのですが、ここで働き始め毎日見ているうちに欲しくなりました。 
\\	「見ているうちに」は 
\\	と 
\\	を使って表現すればいいでしょう。	
\\	この論文を読むと、彼の日本への偏見が見て取れる。 
\\	(はっきり表れる)で「見て取れる」を表現しています。
\\	は「明瞭な」の意。	
\\	巻き込まれたくなかったので、思わず見て見ぬふりしてしまった。 
\\	は「気が付いていない振りをする」の意味。	
\\	あなたに認められて嬉しいです。 
\\	は「認められること」の意味です。	
\\	私は彼がタバコを吸うのを認めてないので、きっとどこかで隠れて吸っているのだと思います。 
\\	「認めない」を 
\\	で表現しています。
\\	は「こっそりタバコを吸う」の意。	
\\	もし、彼が僕の言った事を今日まで継続しているなら、ほんとうにみどころのある奴だ。本格的に鍛えよう。 
\\	で「大いに見込みがある」の意味。
\\	「大いに可能性がある」。	
\\	結婚?いや~、こればっかりは縁ですから。 
\\	ここでは「縁」を、
\\	「ちょうどいい時に、いい所にいる」と説明しています。さらに、「年齢とは関係ない」ことも付け足しています。	
\\	あの発言で彼のことを見直しました。 
\\	は
\\	に対する意見を見直す(考え直す)」の意味になります。	
\\	この料理本を見ながら作ったんだ。見た目はイマイチだけど、味は保証するよ。 
\\	「見ながら」は
\\	と分詞を使えばいいでしょう。
\\	にしたがってする」。	
\\	添削されてない文章は正しいとみなしていいんですよね? 
\\	は
\\	と見なされる、考えられる」、
\\	は
\\	だと思う、決めてかかる」の意味。	
\\	日本ではコンピューター系の仕事をやってた人たちも、こちらではほとんどが寿司見習いとか、キッチンヘルパーです。 
\\	で「…として見習いをする」という意味になります。
\\	は「コンピューター関連の」の意。	
\\	ガーデニング好きな彼女を見習って私も庭に花でも植えようかな。 
\\	で「彼女の例にならって」の意味になります。	
\\	運動不足が体に悪いと言われると、身に覚えがあるだけに、ぎくっとする。 
\\	「身に覚えがある」をここでは
\\	「状況を変えるために一切何もしていない」で表現しています。	
\\	最近、海外での生活がいかに大変なんだなって身に染みてきました。レストラン勤務では何年食べていけるのやら。 
\\	で「痛切に感じる」という意味になります。
\\	でも同じような意味が表現できます。	
\\	よく発音を注意されるので、正しい発音を身につけたい 
\\	「身につけたい」は
\\	、あるいは、
\\	を使えばいいでしょう。	
\\	覚えるという勉強法ではすぐあきるし忘れるから、なんとか自然に身につける方法はないかな。 
\\	「身につける」は
\\	「習得する」を使えばいいでしょう。
\\	「身につける」。	
\\	同世代の人が住宅ローンに苦しめられていると聞いて身につまされた。 
\\	が「身につまされる」に当たる言い回しです。
\\	は「同情する」の意。	
\\	彼女の身の上話は涙無しには聞けないよ。 
\\	「身の上話」は
\\	でいいでしょう。「涙無しには聞けない」を
\\	「乾いた目で」を使って表現している点にも注目。	
\\	今日先に帰るね。今夜毎週見ているテレビ番組があって、ビデオセットしてないから見逃せないのよ。 
\\	この場合の「見逃す」は
\\	ですね。
\\	は
\\	の頭文字語です。	
\\	この展望台からの見晴らしは素晴らしい。 
\\	「見晴らし」は 
\\	(眺め)で表現すればいいでしょう。
\\	は「(景色などを)見渡す」の意。	
\\	最近の若者は、見栄えのする華やかで簡単なことにはすぐ飛び付くが、地道な努力が要求される日の当たらない仕事には見向きもしない。 
\\	ここでは「見栄え」を
\\	「表面的に魅力がある」で表現しています。
\\	「注意をそらす」、
\\	「華々しさ」、
\\	「容姿上の魅力」。
\\	を嫌がる」。	
\\	最近は、生きた英語を学ぼうと、片っ端からレンタル屋でビデオ借りて、見まくってます。 
\\	が「片っ端から見まくる」に当たる表現です。	
\\	頑張りな、ここで見守っててあげるから。 
\\	「見守る」は 
\\	(じっと見ている)でいいでしょう。	
\\	それは私にとって、耳が痛い話です。 
\\	「耳が痛い」をここでは
\\	を聞くのが恥ずかしい」と言い換えています。
\\	「きまりが悪い」。	
\\	おじいちゃんは耳が遠いから大きな声で話さないとだめだよ。 
\\	で「耳が遠い」の意味になります。
\\	は「難聴」の意。	
\\	そんなみみっちいことを言うなよ。 
\\	「みみっちい」は
\\	「けちな」、
\\	「けちくさい」で表現できます。	
\\	彼女は若いのに、耳年増だ。 
\\	「耳年増」は
\\	「(話を)聞いて人生のことについて多くを学ぶ」と言い換えられますね。	
\\	あまりに激しい騒音のせいか耳鳴りがやまない。 
\\	「耳鳴りがする」は
\\	を使って表現します。	
\\	その話しは耳にタコができるほど聞いたよ。 
\\	は「疲れる、すり減る」、
\\	は「タコができている、皮膚が硬くなる」の意。	
\\	まだ、あなたの声が、耳に残っている。 
\\	「耳に残る」は
\\	「まだ声が聞こえる」と言い換えることができます。	
\\	プールで耳に水が入って、なかなか取れないんだ。 
\\	で「…に水が入る」の意味になります。
\\	は「水が流れて出てくる」の意。	
\\	私が言うことには耳を貸さないので、言ってやってくださいよ。タバコやめるやめるって、口だけなんだもん。もう耳にたこだよ。 
\\	「耳を貸さない」は
\\	を使えばいいでしょう。
\\	は「…に慣れる」の意。	
\\	耳を澄まして聞いてごらんよ。虫の鳴く声が聞こえるから。 
\\	「耳を澄まして聞く」は
\\	でいいでしょう。
\\	は「聞こえる」の意味。	
\\	タバコ止めたんだって?いつまで続くか見物だわ。 
\\	(私の見たいのは)で「見物」を表現しています。	
\\	その映画って、ゲームで大ヒットしたやつを映画化したやつでしょ?スタントなしのアクションが見ものらしいよ。 
\\	(見せ物的な、壮観な)で「見もの」を表現しています。	
\\	両親と離れて暮らすようになってから、なんか妙にやさしくなった気がするのよね。 
\\	「妙に」のニュアンスを
\\	「どういうわけか」で表現しています。	
\\	この木は、日本北部に多く見られます。 
\\	「見られる」は
\\	と組み合わせて
\\	を受け身に使えばいいでしょう。	
\\	この曲って、ボーカルが女性のバンドというか、2人組のやつでしょ。確か去年出した
\\	は、ミリオンいったとか。 
\\	の文字通りの意味は「昨年、彼らはミリオンセラーになった
\\	を出した」です。	
\\	タイに特別に魅力を感じて行きたいと思ったことはないな。何か有名なものあったかな? 
\\	「魅力」を
\\	を使って表現しています。	
\\	人の良いところを見ようとしないで、悪いとこばっかり見てる。 
\\	「良いところ(悪いところ)を見る」を 
\\	(見る)と 
\\	(見える)を使って表現しています。	
\\	ここに展示されているものを見る限り、彼女の将来は有望です。 
\\	「見る限り」をここでは
\\	(…から判断して)を使って表現しています。	
\\	見るからに甘そうだね。 
\\	で「見るだけで…だとわかる」の意味。
\\	は「明らか、明白な」の意。	
\\	毎日少しでも勉強しないと意味がないのはわかっているのですが、仕事から帰ってくると疲れて英語のテレビは見る気がしません。 
\\	「…する気がしない」は 
\\	で表現します。
\\	でも同じようなニュアンスが出せます。	
\\	見ればわかるだろっていう感じじゃない? 
\\	「見ればわかる」は
\\	と
\\	を組み合わせて表現すればいいでしょう。	
\\	あなたぁ~恋しいぃ~未練でしょうかぁ~、という歌がありました。 
\\	「未練」を
\\	「忘れられない」、
\\	「ほろ苦い思いで」と表現してみました。どうでしょう?	
\\	おまえもそろそろ30なんだから、身を固めたらどうだい。 
\\	には「落ち着く、身を固める」という意味があります。
\\	を使ってもいいでしょう。	
\\	日本の景気対策を議論するとき、民間活力の導入をおいてほかに考えられない。 
\\	「民間活力」を 
\\	で表現しています。
\\	は「刺激」、
\\	は「新しい活力を与える」の意。	
\\	チマチョゴリは民族衣装です。 
\\	「民族衣装」は 
\\	でいいでしょう。	
\\	私、今の仕事向いてないと自分で思うのよね。 
\\	がちょうど「向いている」に当たる言い回しです。	
\\	私には何が向いていると思いますか?ずっと考えているのですが、自分ではわからないんですよ。 
\\	で「生まれついて…するのに向いている」の意味。
\\	は「素質、適性」の意。(注)後半の日本語を一部書き換えました。	
\\	着いたら電話するから迎えに来て。 
\\	「迎えに来る」は
\\	、あるいは、
\\	と言います。この
\\	はよく省略されます。	
\\	二日酔いがひどいから、迎え酒でも飲むか。 
\\	「迎え酒」には
\\	という決まった表現がありますので、この機会に覚えておくといいいでしょう。	
\\	オーストラリアでは時差の関係で21世紀を日本より早く迎える。しかも、夏だし、もう申し分ないね。 
\\	ここでは「迎える」を
\\	「入る」を使って言い換えています。	
\\	昔と比べて、子供の遊び方って変わったんじゃないかな。昔は、お金があまりかからない遊びしかやらなかった気がする。 
\\	で「今は昔とは違う」の意味を表しています。
\\	と比べて」。	
\\	中には昔ながらの伝統を守り続けてる人もいると思うよ。この辺では伝統を感じることはできないけど、もっと田舎に行けば感じる。 
\\	で「昔からの方法に従って暮らす」の意味になります。
\\	でも同じような意味になります。	
\\	海外用の日本のお土産ってなんでああいう古い昔の絵柄が入ったものが多いのだろう。 
\\	この場合の「昔の」は 
\\	でいいでしょう。
\\	は「お土産」の意。	
\\	家の親って昔の人だから、何かにつけていろいろうるさくて。 
\\	「昔の人」は
\\	「古風な」で表現できます。
\\	についてうるさく(くどくど)言う」。	
\\	しばらくぶりに帰ってきたけど、この辺りは全然変わってないね。昔のままだね。 
\\	が「昔のまま」に当たる言い回しです。
\\	は省略できます。	
\\	こう見えても昔は料理が得意だったんだ。 
\\	「昔」は
\\	「かつては」を使えばいいでしょう。
\\	に凝る、はまっている」。	
\\	最近、自分でもいけないとわかっているのだけど、ちょっとしたことでむかつくんだよね。昔はこんなんじゃなかったのに。 
\\	がポイントです。主語 
\\	と 
\\	の使い方にも注目。
\\	は「取るにたらない」の意。	
\\	昔はよかった 
\\	「昔」は
\\	で表現します。	
\\	久しぶりに会ったんだから、酒でも飲みながら、昔話にでも花を咲かすとするか。 
\\	「昔話に花を咲かす」は
\\	「古き良き時代の話をする」で表現できます。	
\\	今そっちに向かっている途中だからあと10分くらいで着くよ。 
\\	で「そこに行く途中」という意味になります。	
\\	彼はむかつく行動を私に対してよくする。 
\\	「むかつく」を
\\	「耐えられない」、
\\	「嫌悪を表す」で表現しています。	
\\	あのわざとらしい態度がむかつくんだよね。なんか文句があるんだったら、直接言ってって感じ。 
\\	ここでは「むかつく」を
\\	「嫌な思いをさせる」、
\\	「我慢できない」を使って表現しています。	
\\	近ごろの若者は政治に無関心である。 
\\	で「…に無関心である」、
\\	は「無関心」の意。	
\\	そうむきになるなよ。あんなやつ相手にすることないよ。 
\\	は「怒る、取り乱す」、
\\	「無視する」、
\\	「怒らせる」、
\\	「注意を払うに値しない」の意。	
\\	彼は冗談のつもりで言ったんだから、そんなにむきになることはないわよ。 
\\	ここでは「むきになる」を
\\	「真剣に受け止める」で表現しています。
\\	「気にする」。	
\\	なんでよくクラスリーダーに指摘されるんだろう。私、全然リーダー向きでないのに。 
\\	とは「リーダー向きの人」のことです。
\\	に向いている」。	
\\	ちょっとあっち向いてくれる? 
\\	この場合の「向く」は 
\\	を使えばいいでしょう。	
\\	受付にいたのは若い女性ではなく、むくつけき男性だったのでがっかりした。 
\\	「むくつけき」とは
\\	、つまり「粗野で野蛮な(むさくるしい)」ということですね。	
\\	歩きすぎて足がむくんでいる。靴がきつい。 
\\	は「はれ上がる」、
\\	は「とんでもなく痛い」の意味。	
\\	寝ていると思ったら、ムクムクと起き上がってきた。 
\\	共に「動き出す」の意味。	
\\	一生懸命努力をすれば、必ず報われる。 
\\	は「成果を上げる、利益をもたらす」の意味になります。
\\	は「報いる」の意。	
\\	このクーポン券もう期限が切れているからもう無効だよ。 
\\	「無効」は
\\	でいいdしょう。
\\	「無効である」。	
\\	最近、無言電話がよくかかってくるので気味が悪い。 
\\	で「何も言わずに電話を切る」という意味になります。
\\	「悩ませる」、
\\	「電話に出る」。	
\\	アイツはいつ会ってもむさ苦しい。 
\\	は「だらしない」、
\\	は「みすぼらしい」の意。	
\\	無差別テロというのは、とても卑怯で卑劣で不公平なやり方だと思う。 
\\	「無差別テロ」は 
\\	と言います。
\\	は「区別する」の意。	
\\	同じ話を蒸し返すなよ。それはもう終わったことだろう。 
\\	ここでは「蒸す返す」を
\\	「繰り返す」を使って表現しています。
\\	は「済んだことについて話す」の意。	
\\	両方欲しいなんて虫がよすぎるんじゃない? 
\\	「欲張るのはよそう」、
\\	(欲張りすぎてない?)で「虫がいい」のニュアンスを表現しています。	
\\	このセーター虫食いしてない? 
\\	は形容詞で「虫に食われた」の意味。
\\	は衣類を虫食いにしてしまう虫、つまり「イガ(衣蛾)」の意。	
\\	防虫剤入れといたのに、まだ1回しか着てないセーター虫に食われちゃった。 
\\	「衣服に害を与える虫」のことを
\\	といいます。これに
\\	を続ければ「虫が食う」の意味が表現できます。
\\	でも同じ様なことが言えます。
\\	「防虫剤」。	
\\	部長は今日、虫の居所が悪そうだから、話しかけない方がいい。 
\\	「~な気分」は
\\	で表現できます。「虫の居所が悪い」は「気分が悪い」ということですね。
\\	は「そばに寄らないようにする」の意味です。	
\\	虫の知らせがあった。 
\\	「虫の知らせ」を、ここでは、
\\	「(それが起きるような)そのような気がした」で表現しています。
\\	の予感がする」も使えますね。	
\\	みんなが真剣に話しているときにニヤニヤしやがって、虫の好かない奴だ。 
\\	「虫の好かない」は 
\\	(好きではない)で表現すればいいでしょいう。
\\	は「実に嫌な」、
\\	は「ニヤニヤする」の意。	
\\	冷たい物を食べると歯がしみるんだ。虫歯かな。 
\\	「虫歯」は 
\\	といいます。
\\	は「…に敏感な」の意味。	
\\	彼女は、参院選に無所属で立候補した。 
\\	が「無所属で立候補する」に当たる言い回しです。	
\\	この景色を見てると無心になれる。 
\\	で「乱雑なことを忘れられる」の意味になります。	
\\	あいつの無神経さにはあきれるよ。 
\\	は「無感覚な、平気な」、
\\	は「無神経な、鈍感な」の意味です。	
\\	彼が言う事って矛盾してない? 
\\	で「矛盾したことを言う」の意味。	
\\	あの駅無人駅なのに、今日はめずらしく駅員が数名いた。 
\\	は「人が乗っていない、無人の」という意味。	
\\	3歳の私の娘が、泣かない日はない。 
\\	「3歳の娘」は 
\\	で表現します。
\\	の後に複数形の 
\\	を付けないように注意。	
\\	朝方、鼻がむずむずすることがよくある。 
\\	「むずむず」は
\\	「かゆい」で表現すればいいでしょう。	
\\	勝手に早退するなんて、あまりに無責任だよ。 
\\	で「君の責任感はどこにあるのか=無責任だよ」と言っていることになります。
\\	「帰る」、
\\	「不足」。	
\\	無駄足にならないように、アポを取ってから、彼に会いに行くんだな。 
\\	で「時間を無駄にすることになる」の意味。
\\	(努力)を付け足してもいいでしょう。	
\\	また新しいパソコンを買ったの?家計が大変なんだから、お金の無駄遣いをしないでよ。 
\\	「お金の無駄遣い」は 
\\	ですね。
\\	は「家計」の意。	
\\	今まで自分の勉強法にいつも不安を感じていたけど、今回のテストの結果で無駄ではなかったことがわかっただけでも嬉しい。 
\\	「無駄ではない」とは 
\\	(やるだけの価値がある)ということですね。	
\\	無駄な抵抗はやめて、あきらめたら。このゲームは僕の勝ちだよ。 
\\	(抵抗)と 
\\	(無駄な、役に立たない)を使って「無駄な抵抗」を表現しています。
\\	は「勝ったも同然だ」の意味。	
\\	最近時間を無駄に過ごしているなぁってすっごく感じるのだけど。毎日毎日働いて終わってさ。 
\\	で「時間をつぶす」の意味。
\\	でもいいでしょう。
\\	は「明けても暮れても」の意。	
\\	無断で僕のパソコンを使うなんて、あいつ許せないよ。 
\\	は「聞きもしないで」という意味。
\\	「許可」	
\\	入社早々、無断欠勤するとは。今の若いやつらは何を考えてるんだ。 
\\	で「何も言わずに仕事を休む」の意味になります。
\\	「無断欠勤する(軍隊用語:
\\	この仕事を2日で仕上げろなんて、そんな無茶なこと言わないでくれよ。 
\\	で「期待し過ぎだよ=無理な話だ」の意味になります。
\\	「無茶な、ひどすぎる」。	
\\	彼女は彼に夢中で、ほかの物が見えなくなっている。彼がくれるプレゼントだったら、どんな物でもうれしく思う。 
\\	で「…に夢中になる」の意味。
\\	は「…にのぼせ上がる、熱を上げる」の意。	
\\	何か大変なことが起きるのではないかと胸騒ぎがする。 
\\	「胸騒ぎ」は
\\	「不安になる」を使えばいいでしょう。	
\\	そのドラマを見終わった後、なんとなく、虚しくなってしまった。 
\\	は「すり減った、疲れ切った」、
\\	は「むなしい、空虚な」の意味。	
\\	どんなに誠意を尽くしても、相手に伝わらなければむなしいだけだよね。 
\\	(無益さを感じる)で「むなしい」を表現しています。	
\\	私の好意を無にしないで。 
\\	は「無駄になる」、
\\	は「ささいなものにする」の意味。	
\\	相手チームはすべての点で格上だから、胸を借りるつもりで臨むつもりだ。 
\\	「胸を借りる」を 
\\	(負けないよう頑張って、…から学ぶ」を使って表現しています。	
\\	スノボーでまだ止まる事もできないのに、無謀にもリフトに乗って上へ上がってしまいました。 
\\	で「ずうずうしくも 
\\	する」の意味。
\\	「向こうみずにも」。	
\\	エッチな本を読んでいてムラムラときた。 
\\	で「性的に刺激を受けた」の意味になります。
\\	は「性的に興奮した」の意。	
\\	気に入ったのが見つからないなら、今無理して買うことないじゃん。 
\\	で「無理して…する」の意味になります。	
\\	自分では疲れていないと思っていても体は正直なので無理はできません。 
\\	ここでは「無理」を 
\\	(やりすぎる)で表現しています。
\\	(嘘をつく)の使い方にも注目です。	
\\	大きさが違うのに、同じ価格にしろだなんて、所詮無理があるよ。 
\\	で「…する意味がない。無駄だ」の意味。
\\	「本質的に不合理である」。
\\	「結局」。	
\\	そもそも最初からこの予算じゃ無理があったんだ。 
\\	ここでは「無理がある」を 
\\	(不合理な)、
\\	(現実を無視した)を使って表現しています。	
\\	もう無理がきかない年だなと感じた。体力が全然ない。 
\\	で「かつてのように体力を使うことができない」で「無理がきかない」を表現しています。	
\\	先日はごめんなさい。無理したら行けたのですが、次の日仕事があったし安静にしたほうがいいかなと思ったので。 
\\	で「全力を尽くす」、
\\	で「無理をする」を表現しています。	
\\	着ている服に合わせようと思って、無理してヒールの高い靴を履くんじゃなかった。 
\\	「やりすぎる」で「無理して」を表現しています。
\\	は「わざわざ…する」の意。	
\\	身体の調子はどう?無理しないでね。 
\\	「無理しないで」を 
\\	(気楽にやる)で表現しています。
\\	は「やり過ぎる」の意。	
\\	そんなに無理することはないよ。マイペースでやればいい。 
\\	「無理をする」は
\\	「やり過ぎる」、
\\	「無理矢理やる」で表現できます。	
\\	とても手が届かない金額だと簡単にあきらめがつくけど、ちょっと無理をすれば買える金額だとどうしても諦めきれない。 
\\	には「無理をする」という意味があります。
\\	は「無理をしすぎる」の意。	
\\	上司から無理難題をふっかけられて弱っている。 
\\	で「不可能なこと」の意味になります。これを使えば「無理難題」が表現できますね。	
\\	クリスマスだのバレンタインだからって何かするっていうのは無理矢理っぽくてすきでない。 
\\	「無理矢理っぽい」は
\\	「わざとらしい、人工的」を使えばうまく表現できます。
\\	「こじつけの」。	
\\	韓国に旅行に行ったのですが、韓国は私にとって、目新しいものばかりでした。 
\\	「目新しい」は
\\	、あるいは、
\\	「新しい経験」を使えばいいでしょう。	
\\	保険金目当てに我が娘を殺そうとするなんて、とても信じられない。 
\\	で
\\	を企てる」の意味になります。目的を表す不定詞
\\	だけでも「目当てに」を表せます。	
\\	今日のメイクすごくかわいいね。コツを教えてくれない? 
\\	「メイク」は
\\	と言わなければ通じません。「コツ」の表現の仕方にも注意。	
\\	毎日のように子供がお邪魔して、迷惑でありませんか? 
\\	は「面倒」、
\\	は「負担」の意味。	
\\	忙しいのにホント無理言ってすみません。迷惑はかけませんから。 
\\	で「あとで迷惑をかけないと約束する」の意味になります。
\\	は「要求」の意。	
\\	あの先生の説明を聞いて、目からウロコが落ちたよ。 
\\	「目からうろこが落ちる」は英語でも全く同じような言い方をします。
\\	とは、まさしく「うろこ」のことです。	
\\	彼女と目が合ったとたん、つい、目をそらしてしまったよ。 
\\	「(目が)合う」は
\\	を使って表現します。
\\	「視線をそらす」。	
\\	彼女はお金に目がくらんで、愛してもいないのに彼と結婚をした。 
\\	動詞
\\	には「目をくらます、惑わす」という意味があります。
\\	は「欲」の意。	
\\	彼の話を聞いて、目頭が熱くなった。 
\\	の文字通りの意味は「感動させて涙を流させる」です。	
\\	これが彼の作品だとわかるとは、さすがお目が高い。 
\\	「目が高い」は 
\\	で表現できます。	
\\	何をそんなに怒ってるんだよ。目がつり上がってるぞ。 
\\	英語で「怒って目がつり上がる」という表現の仕方はしないようです。ここでは
\\	を使って「目を見れば怒っているのがわかる」を使って言い換えています。	
\\	いきなりあんな事をされたら目が点になっちゃうよ。 
\\	「目が点になる」をここでは
\\	「唖然とする」で表現しています。
\\	「言葉を失う」	
\\	僕、甘いものには目がないんですよ。 
\\	で「…が大好きである」の意味になります。	
\\	メガネが曇って、よく見えないんだ。 
\\	が「曇る」に当たる表現です。	
\\	寝不足のせいかな。目が腫れてるよ。 
\\	で「目が腫れている」という意味になります。	
\\	ここのところ目が回るほど仕事が忙しくて、ビデオを見ている暇もない。 
\\	「目が回るほど」は
\\	「頭が回る、めまいがする」で表現します。
\\	「めまいがずる、ふらふらする」。	
\\	目が悪いから、小さな文字はよく見えないんだ。 
\\	「(目が)悪い」は 
\\	で表現します。
\\	は「細かい」、
\\	は「視力」の意。	
\\	なにもそんなに目くじらたてなくてもいいんじゃない? 
\\	「目くじらをたてる」は
\\	「そんなに腹を立てる価値はない」で表現できます。
\\	「興奮する」。	
\\	風のせいで車にかけてあったシートがめくれていた。 
\\	は自動詞(まくり上げる)としても使えます。	
\\	最近ネイティブの人との付き合いに言葉の壁を感じ悩むことも多いが、今が我慢のときだと思ってめげずに立ち向かっていこうと思います。 
\\	「今が我慢のときだと思ってめげずに立ち向かって行こうと思います」のニュアンスを 
\\	(今がなんとか努力をして切り抜けていく大変な段階であると思う)で表現しています。あえて、「めげずに」には英訳していません。	
\\	目先のことばかりを考えていてはダメだよ。将来のことも考えるようにしなくちゃ。 
\\	で「目の前のこと」という意味になります。	
\\	上のレベルを目指すよりも、今のレベルを保つ事の方が大事だと思います。 
\\	「上のレベルを目指す」は
\\	「次のレベルに上がる」、
\\	「向上を目指す」ということですね。	
\\	ブースカ大好きの水田さんは、偶然入った店でめざとくブースカグッズを見つけて早速買うことにした。 
\\	「目標に向かって進む」、
\\	「視線が素早く... に向けられる」が「めざとく見つける」に近い言い方です。	
\\	この人向こう行ってくれないかな。貧乏揺すりがすっごい目障りだ。 
\\	は「悩ませる、いらだたせる」の意味。「貧乏揺すり」をここでは
\\	「そわそわ動く」を使って表現してみました。	
\\	30を過ぎると、目尻のしわが気になるわよね。 
\\	が「目尻の小じわ」に当たる表現です。
\\	は「心配事」の意。	
\\	おっと珍しい!君がこんな朝早く来るなんて!雪でも降るかも! 
\\	で「めったに見られない光景」の意味。
\\	は、その時期に併せて、
\\	を、例えば、
\\	や
\\	などに置き換えればいいでしょう。	
\\	こんな私を愛してくれるなんて、あなたは珍しい人ね。 
\\	で「めったにいない人」の意味になります。	
\\	韓国に行ったんだけど、商店街に珍しい物がいっぱい売られていたよ。 
\\	は「まれな、珍しい」、
\\	は「めったにない」、
\\	は「変な」の意味。	
\\	ゴキブリが平気なんて、あなたは珍しい人ね。 
\\	この場合の「珍しい」は
\\	「変な」でいいでしょう。
\\	「(とても)びっくりさせる」。	
\\	いつも残さず食べるのに、めずらしく残していると思ったら、かなり美味しくなかったのね。 
\\	この場合の「めずらしい」は
\\	「普通でない、まれな」で表現すればいいでしょう。	
\\	あの時のパーティーでジーンズ履いてたのあなただけだったもん。一際目立っていたわよ。だから、覚えているのよ。 
\\	「目立つ」は
\\	を使って表現します。	
\\	最近、めっきり秋らしくなってきた。 
\\	「めっきり」を
\\	「間違いなく」で言い換えて表現しています。	
\\	これはめっけもんだな。滅多に出る品物じゃないよ。 
\\	は名詞で「掘り出し物、見つけ物」という意味があります。	
\\	そんなことはめったに起こりませんね。 
\\	「めったに」は
\\	で表現します。	
\\	彼は数学にめっぽう強かった。 
\\	は「信じられないほど」の意味。
\\	は「…を圧倒する」の意。	
\\	あいつは目つきが悪いから、皆から嫌われるんだよ。 
\\	「不快な」と
\\	を組み合わせて「目つきが悪い」を表現しています。	
\\	僕、よく目つきが悪いって言われるんです。 
\\	で「目つきがよくない」の意味。
\\	の文字通りの意味は「目との目の合わせ方がうまくない」です。	
\\	彼は学校の目と鼻の先に住んでいるのに、遅刻ばかりしている。 
\\	(石を投げたら届く距離)が「目と鼻の先」に当たる言い回しです。
\\	は「声の届く距離」の意。	
\\	これからどうすべきか目途が立たないんだよ。 
\\	「目途が立たない」とは
\\	「言うのが大変だ」、
\\	「分からない」ということですね。	
\\	この指輪を渡した時に彼女が喜ぶ顔が目に浮かぶようだ。 
\\	ここでは「目に浮かぶ」を
\\	(心に描く)を使って表現しています。	
\\	私の場合に鼻はそうでもないけど、眼にくるんだよね。(花粉症の症状) 
\\	と
\\	の言い回しに注目。
\\	「鼻孔」。	
\\	目にゴミが入って、なかなか取れないんだ。 
\\	が目に入る」は
\\	と言います。「ゴミ」は
\\	でいいでしょう。
\\	「飛んで来て入る」の意。	
\\	目にゴミが入って、なかなか取れないんだ。 
\\	この場合の「ゴミ」は
\\	を使えばいいでしょう。
\\	や
\\	が目に入るわけがありませんから。	
\\	タマネギが目にしみる。涙が出て来ちゃった。 
\\	ここでは
\\	「ひりひりさせる」を用いて「しみる」を表現しています。	
\\	彼女は目に涙を浮かべて、本当のことを話し始めた。 
\\	で「こみ上げてくる」の意味になります。
\\	は「きらめく」の意。	
\\	あのときの光景が今でも目に焼き付いてます。 
\\	「目に焼き付く」とは「目を閉じてもまだ見える
\\	ということですね。
\\	は「華やかに彩る」の意。	
\\	この瞬間を目に焼き付けておきたい。 
\\	ここでは 
\\	(…を記憶に焼き付ける)、
\\	を使って「目に焼き付ける」を表現しています。	
\\	あいつにとって山田は目の上のたんこぶみたいなものさ。 
\\	とは
\\	の姿を見るのは耐えられない」、また、
\\	は「悩みの種(脇腹のとげ)」の意味。	
\\	なんで彼は僕のことを目の敵にするんだろう。何か彼の恨みを買うようなことをしたかな? 
\\	(変わらぬ敵)で「目の敵」を表現しています。
\\	「侮辱」、
\\	は「(人の)怒りを招く」の意。	
\\	私の目の黒いうちは、この土地をだれにも売るつもりはない。 
\\	「目の黒いうち」を 
\\	(びんびんしている間は)で表現しています。	
\\	目の下に隈ができてるじゃないか。残業ばかりしてるからだよ。 
\\	「目の下の隈」は
\\	と言います。「残業」は
\\	ですね。	
\\	彼女が欲しいって言ってたコート、目の玉が飛び出るほど高かったんだ。 
\\	は日本語と同じ「目が飛び出るほど」という意味ですね。
\\	「(思わず)あっと言うほど」。	
\\	彼、中学の娘さん目に入れても痛くないんだよ。 
\\	「(可愛くて仕方がなくて)目に入れても痛くない」には
\\	という決まり文句があります。	
\\	生演奏、生歌で、ステージの上ではあちらこちらでパフォーマンスをやっていて、目のやり場にこまった。 
\\	で「目をどこに向ければいいか困る」の意味。	
\\	目は口ほどにものを言うんだよ。 
\\	「目は口ほどに…」は 
\\	を使って表現します。
\\	は「雄弁な」の意。	
\\	測るものがないから目分量でやろう。 
\\	「何か測るもの」。
\\	「目で測る」の使い方も一緒に覚えておくといいでしょう。	
\\	最近少しめまいがするんだ。医者に診てもらった方がいいかな。 
\\	「めまい」は
\\	と言います。「診てもらう」の表現の仕方にも注意。	
\\	そんな女々しいこと言うなよ。 
\\	は「弱虫、泣き虫」、
\\	は「女々しい、軟弱な」の意味。	
\\	さっきいいアイデアが浮かんだのに。何だったっけ。メモっておけばよかったな。 
\\	で
\\	を素早く書き留める」の意味。
\\	を使う場合には、次のような言い方をします。
\\	など。	
\\	右目に目やにがついてるよ。 
\\	(名詞)には「眠り」とは別に「目やに」という意味もあります。	
\\	もっと生活にメリハリをつけよう。 
\\	(…のペースを変える)で「メリハリをつける」を表現しています。
\\	は「よりうまくバランスをとる」の意。	
\\	自分のメル友が一体どんな人なのか、いつも気になるんだよね。 
\\	「メル友」は 
\\	、あるいは、
\\	でいいでしょう。	
\\	あの娘にめろめろなんだ。 
\\	には「べたぼれ」の意味がありますので、これで「めろめろ」が表現できます。	
\\	あの子、何でも口に入れちゃうから、目が離せないのよ。 
\\	で「見えないところに置いておく」で「目を離す」を表現しています。	
\\	彼は孫の顔を見て目を細めた。 
\\	「目を細める」は
\\	といいますが、この場合は「うれしい」の意味になりますので、
\\	(うれしくてほほ笑む)を使って表現しています。	
\\	彼のステージはどれも目を見張るものばかりだった。 
\\	「目を見張る」は
\\	「驚くほどの」や
\\	「素晴らしい」で表現できますが、ここでは
\\	「見るべきもの、その価値がある」を使ってみました。	
\\	あの島では今、感染病が流行っているって新聞で読みました。特に、免疫力が下がっている人が危ないらしい。 
\\	ここでは「免疫力が下がっている人」を 
\\	で表現しています。
\\	は「免疫反応が十分でない」の意。	
\\	彼は、交通事故で入院し、今は面会謝絶で会えない。 
\\	「面会謝絶」は
\\	という言い方をします。	
\\	あいつかなりの面食いだからな。だからなかなか彼女ができないんだよ。 
\\	で「綺麗な人に弱い」の意味。
\\	「決まった恋人」。	
\\	彼女から突然別れ話をされ、面食らってしまった。 
\\	には「あわてさせる、面食らわせる」という意味があります。
\\	は「まごつかせる」の意。	
\\	申し訳ないですが、この程度のキズは免責とさせていただきます。 
\\	ここでは「免責」を
\\	(カバーしない、含まない)で表現しています。	
\\	何か文句があるなら堂々と、面と向かって言うべきだ。 
\\	が「面と向かって」に当たる言い回しです。
\\	は「堂々と」、
\\	は「不平、不満」の意。(注)「戦う」を「言う」に変更しました。	
\\	これ、一度修理に出してから調子が悪いんですよ。また持って行くのも面倒だし。 
\\	には「厄介なこと」という意味があります。	
\\	職場ではよくあることですが、人間関係の問題はいろいろと面倒ですよね。 
\\	で「いろいろ面倒なこと」の意味になります。
\\	は「よくあることだ」の意味。	
\\	ビデオ返しにいくのめんどうだなー。でも、遅くなったらもっと行く気がなくなるから、さっさと行っておくか。 
\\	には「悩ませること、厄介な問題」という意味があります。	
\\	もう面倒がらずにやってよ。お風呂掃除は、あなたの担当って結婚当初決めたことじゃない。 
\\	は「たらたら不平を言う」、
\\	は「愚痴をこぼす」という意味。	
\\	明日のパーティーに行くの、急に面倒くさくなってきちゃった。 
\\	「面倒くさい」は
\\	「わざわざ... する」を使うとうまく表現できます。	
\\	彼は、後輩の面倒見がとてもいいから。 
\\	「面倒を見る」は
\\	で表現できます。「後輩」の言い方にも注目。	
\\	彼は後輩の面倒見がいいからね。 
\\	「面倒見」は
\\	を使って表現できます。	
\\	ちょっと、何で私の話聞いてくれないの? わかった、面倒くさいんでしょ。 
\\	は名詞で「面倒くさいこと」の意味。
\\	でも同じようなことが言えますね。	
\\	面目ない。俺がこんなドジを踏むなんて。 
\\	「面目ない」とは
\\	「とっても恥ずかしい」ということですね。
\\	「へま(どじ)をする」。	
\\	せっかく階段を走って降りたのに、もう一歩のところで電車に乗れなかった。 
\\	「もう一歩(少し)のところで... できない」は
\\	で表現できます。
\\	「ダッシュ、突進」。
\\	「目の前だ閉まる」。	
\\	一度に大金を賭けることができなくて、少しずつ賭けるから当たっても全然もうからない。 
\\	ここでは「もうかる」を
\\	を使って表現しています。
\\	、または 
\\	は「…を賭ける」の意。	
\\	来年4月から始まるコースの申し込みの受付が開始されて、今どれにしようか悩んでいます。 
\\	で「入学(登録)の申し込み」の意味になります。	
\\	申込者が多かったから(英会話の)クラスが2つに分かれただけで、レベルはどちらも同じです。 
\\	「申し込み」は
\\	、「申込者」は
\\	ですね。	
\\	彼の名前は日本ではよくある名前で、よく申し込み用紙の書き方などの例で使われています。 
\\	「申込用紙」は 
\\	と言います。
\\	は「記入する」の意。	
\\	今日は、天気もいいし、気分もいいし、申し分ないね。 
\\	「申し分ない」とは
\\	(これ以上欲しい物があるか?何もない)ということですね。	
\\	今回の事は本当に申し訳ないと思っております。 
\\	ここでは「申し訳ない」を
\\	に対してちゃんとした言い訳があるといいんですが(実は、ない)」を使って表現してみました。	
\\	もうすぐ終わるよ。ちょっとそこで待ってて。 
\\	「もうすぐ」は
\\	を使えばいいでしょう。	
\\	彼、盲腸で一週間入院してたんだって。 
\\	「盲腸」は
\\	と言います。冠詞の
\\	はつけません。	
\\	お寺で座禅を組み、妄念を払いなさい。 
\\	「妄念を払う」を 
\\	(払いのける)と 
\\	(心を惑わす考え)を使って表現しています。	
\\	もう一つメールアドレスがあるから、こっちに送って下さい。 
\\	「もう一つ」は
\\	ですね。	
\\	今日は燃えないゴミの日だっけ? 
\\	(冠詞なし)が「ゴミの日」に当たる表現です。	
\\	燃えるゴミと燃えないゴミを一緒に出しちゃだめよ。 
\\	「燃えるゴミ」は 
\\	、「燃えないゴミ」は 
\\	といいます。
\\	は「捨てる」の意。	
\\	もお、どうしてそんな事を言うの? 
\\	「もう」は
\\	(「オー」と発音)、
\\	!で表現すればいいでしょう。	
\\	特に目的も無く行ったものだから、何もする事もなくボーッとしてました。 
\\	で「特別な目的なしで」の意味になります。
\\	「(悪化して)... になる」。	
\\	この大学にいて、あの名物教授を知らないなんて、もぐりだ。 
\\	「もぐり」とは
\\	「本当の学生ではない」ということですね。	
\\	それって、もしかして、あれのこと? 
\\	この場合の「もしかして」は
\\	や
\\	(仮定法)を使って表現すればいいでしょう。	
\\	もしかして、わかってなかったのって私だけ? ごめん!! 
\\	「もしかして」をここでは
\\	「よっとして... なのかな」で表現しています。	
\\	もしかすると、行けないかも。 
\\	「もしかすると」は
\\	を使えばいいでしょう。
\\	は
\\	ということになるかも知れない」の意。	
\\	「ボーンコレクター」って文字通り、骨を集めている人の話なの? 
\\	が「文字通りの」の意味になります。	
\\	「文字化け」を表現する場合には、
\\	を使うといいでしょう。これは名詞「ゆがめられたもの」、動詞「ごっちゃにする」、あるいは、
\\	「不明瞭な、はっきりしない」として使えます。また、
\\	「妙な文字」でも「文字化け」の意味になります。	
\\	週の初めから残業じゃ、一週間もたいよ。 
\\	と
\\	「長続きする」の使い方に注目。	
\\	時間がない、もたもたするな。 
\\	「もたもたするな」は
\\	「時間を無駄にするな」で表現できます。
\\	は「時間を無駄に過ごす」の意。	
\\	面接では自分がこの会社で何ができるか、その結果会社にどのような利益をもたらす事ができるという事をアピールする事が大切だと思う。 
\\	「もたらす」を
\\	「引き起こす、創り出す」を使って表現しています。	
\\	昨日食べた肉のせいかな。なんか胃がもたれるんだよな。 
\\	「もたれる」とは
\\	「重く感じる」ということですね。	
\\	それって、絵に描いた餅じゃない? 
\\	「絵に描いた餅」には
\\	という決まり文句があります。
\\	「ほしい物の一覧(実際には手に入らない)」。	
\\	生徒それぞれの持ち味を引き出すことが教師の役割だ。 
\\	ここでは「持ち味」を 
\\	(持って生まれた能力)、
\\	(特性)で表現しています。	
\\	この間、大掃除してる時にパスポートに挟まっていたドル札を見つけて、換金しようと思って持ち歩いていたんだけど、なかなか銀行に行く機会がなくてさ。 
\\	「…を持ち歩く」は
\\	で表現できます。「大掃除」や「なかなか行く機会がない」の表現の仕方にも注目です。	
\\	ゴルフの練習のしすぎで、親指の皮がむけた。持ち方がおかしいのかな。 
\\	「持ち方」は 
\\	でいいでしょう。
\\	は「間違った持ち方をする」、
\\	は「まめがむける」の意味。	
\\	きっと、月曜日に会社行ったら君の噂でもちきりかもしれないね。 
\\	は
\\	で一杯」、
\\	は「がやがや言う」の意味です。	
\\	これを場内に持ち込んでもいいですか? 
\\	「持ち込む」は
\\	で表現します。	
\\	毎年、年末になるともちつきをします。 
\\	「もちつき」は
\\	と説明すればいいでしょう。
\\	は「ドンドンたたく」の意。	
\\	外国人の友達とお互いに英語と日本語の勉強をしているんだ。つまり、僕と、彼らは持ちつ持たれつなんだ。 
\\	とは「お互いに有益である」の意味。
\\	は「お互いに得をする状況」の意。	
\\	彼女は、美人だけど、ちょっと鮫肌っぽいのが気になる。 
\\	「鮫肌」をここでは
\\	「乾いた肌」、
\\	「うろこのような肌」で表現しています。	
\\	こんな高い
\\	シャツを運動着にするなんて、もったいなくてできない。 
\\	「もったいなくてできない」は 
\\	(無駄にする)を使えば表現できます。	
\\	まだ、半分もクラスが終わってないので、今辞めたら授業料がもったいないのだけど、行くのも苦痛です。 
\\	「もったいない」とは 
\\	(無駄)ということですね。
\\	は「面倒なこと、骨折り」の意。	
\\	何もそんなにもったいぶって話さなくてもいいじゃないか。 
\\	(回りくどい話し方をする)で「もったいぶって」のニュアンスを表現しています。
\\	は「要点を言う」の意。	
\\	そんな所に置きっぱなしにしていたら、誰かに持って行かれちゃうよ。 
\\	で「それを持って歩き去る」の意味になります。	
\\	気に入ったぜ、出血大サービスだ。もってけ泥棒! 
\\	この場合の
\\	は「ただみたいな物」の意味で使われています。
\\	「事実上(ただで)上げてるようなものだ」の意。	
\\	遊園地のお化け屋敷でもだめなのに、本物のお化け話なんてもってのほか。 
\\	なんて考えないで」と言えば、「もってのほか」のニュアンスが表現できます。	
\\	やりたい事をやりつつお金を貯める。これが私のモットー。やりたい事を我慢してまで貯金したくない。 
\\	「モットー」はそのまま
\\	が使えます。
\\	の後には
\\	が省略されています。	
\\	彼女が怒るのはもっともだよ。デートをすっぽかしたお前が悪いよ。 
\\	「もっとも」は
\\	(理由)、
\\	(権利)を使えば表現できます。	
\\	充電するの忘れたからあまりもたないかも。 
\\	には「続く、持続する」という意味があります。	
\\	すみません、このお魚冷蔵庫で何日くらい保ちますか? 
\\	この場合の「保つ」は
\\	(自動詞)、あるいは、
\\	で表現できます。	
\\	あと半分しか残ってないや。これで一週間もつかな? 
\\	自動詞 
\\	には「もつ、持続する」という意味があります。	
\\	やはり持つべきものは友だちだね。 
\\	の文字通りの意味は「きみのような友だちがいるっていうことはいいよね」です。
\\	は「手に入れる」の意。	
\\	もつれたゲーム展開。 
\\	「もつれた」は
\\	「きわどい、接戦の」を使って表現します。	
\\	私には彼の作品が何でああまでもてはやされて世界的に有名になったのかわからない。 
\\	で「…をもてはやす」の意味になります。
\\	は「惜しみなく与える」、
\\	は「こびへつらうような」の意。	
\\	彼は女にもてるからね。 
\\	「もてる」をここでは「磁石のように引きつける
\\	と言い換えて表現しています。	
\\	彼女は美人だ。…もとい、整形美人だ。 
\\	で「…を撤回する」の意味。
\\	は「褒める」、
\\	は「褒められる、功績が認められる」の意味」。	
\\	飲み放題
\\	1500か~。3杯飲めば元が取れるでしょう。 
\\	で「支払ったお金と同じ価値の物を手に入れる」という意味です。	
\\	座席は元あった位置に戻したほうがいいですか? 
\\	で「前にあった位置」の意味になります。
\\	は「元の位置」の意。	
\\	あのふたり、なんだかんだ言って結局もとの鞘におさまっちゃうんだよね。 
\\	「もとの鞘に収まる」は
\\	「結局は一緒にいる」でいいでしょう。
\\	「どぎつい」。	
\\	それじゃあ元の木阿弥じゃないか。 
\\	「元の木阿弥」とは「元に戻る」ことですから、
\\	や
\\	「最初からやり直す」を使って表現できます。	
\\	今回の面接では、そんな高いレベルの英語力は求められてない。 
\\	この場合の「求められる」は
\\	「要求する」、
\\	「必要とする」を使えばいいでしょう。	
\\	結婚式で歌を歌ったんだけど、もう恥ずかしくて。もともと人前で歌うの得意じゃないし。 
\\	「もともと得意じゃない」は完了形を使って
\\	と言えばそのニュアンスが出せるでしょう。	
\\	うまく話せなくて、もどかしく思われませんか? 
\\	「もどかしい」とは 
\\	(いらいらする)ということですね。	
\\	早く先を読みたくてたまらないのにわからない単語が多くて辞書を引くのがもどかしい。 
\\	ここでは「もどかしい」を 
\\	で表現しています。
\\	は「いらだたしいほどに」の意。	
\\	いいたいことがうまく言えずもどかしさが募り、やがて英語での会話に消極的になってしまいました。 
\\	「もどかしさ」は
\\	、あるいは、
\\	で表現できますね。	
\\	あのホテルのモーニングサービス、安くておいしいよ。 
\\	「モーニングサービス」は和製英語です。
\\	と表現すればその意味が伝わります。	
\\	そういうことを聞くこと自体、あなたのごう慢さを物語っている。 
\\	この場合の「物語る」は 
\\	(示す)で表現すればいいでしょう。英文2では自動詞 
\\	(現れる)を使っています。
\\	は「ごう慢な」の意。	
\\	日本人って、暗記するばかりの教育で、物事について考えるという教育を受けていない気がする。 
\\	この場合の「物事」は
\\	「様々な問題」、
\\	でいいでしょう。
\\	「暗記」。	
\\	今年で80になるうちの祖母は、とても物知りです。 
\\	で「とてもたくさんのことを知っている」の意味になります。
\\	は「生き字引」の意。	
\\	マリナーズの佐々木はものすごい球を投げるから絶対活躍すると思っていた。 
\\	には「卑劣な、いやしい」の他に、「すばらしい、すごい」という意味があります。
\\	「すごい腕〔肩)をしている」。	
\\	なかなか単位をくれないって評判のあの先生の授業をわざわざ取るなんて物好きだな。 
\\	「物好きだ」のニュアンスを 
\\	(気まぐれな)、
\\	(好奇心から)を使って表現しています。	
\\	なんか物足りないな。スパゲティーも食べようかな。 
\\	「物足りない」とは
\\	「まだお腹が一杯ではない」ということですね。ここでの
\\	は「あき、余地」の意。	
\\	監督
\\	の作品にしては、なんか物足りなかったな。彼の作品だということで、かなり期待してたのだけど。 
\\	(何となく不足している)で「物足りない」を表現しています。
\\	でも同じような意味になります。「期待する」の表現の仕方にも注目。	
\\	いつも決まった仕事だけしていればいいというものではない。 
\\	「いいというものではない」は 
\\	で表現すればいいでしょう。	
\\	この間初めてスノーボードやって全然コツがつかめなかったんだけど、反対に今度は絶対モノにしてやるという気が強まった。 
\\	「スノーボードに必要な技術を得る」で「モノにする」を表現しています。
\\	のコツを掴む」。	
\\	日本人の精神の根幹を成すものは、「もののあわれ」という感情である。 
\\	ここでは「もののあわれ」を 
\\	(自然[物事]に対する深い感受性)で表現しています。
\\	は「内在する悲哀」の意。	
\\	君はただの怠け者なのに、自分は自然体で生きていると言うのか。物は言い様だな。 
\\	で「それも一つの言い方だな」の意味になります。	
\\	海外からの要人が来られるので空港の警戒が物々しい。 
\\	「警戒が物々しい」は
\\	(警備)と
\\	「きつい」を組み合わせて表現できます。
\\	「位の高い人」。	
\\	ものもらいができたので夏休みはプールに人れなかった。 
\\	「ものもらいができる」は
\\	、あるいは、
\\	と言います。
\\	は「できる、形成される」の意味になります。	
\\	最近、物忘れが激しくて。少しぼけて来たのかな? 
\\	「物忘れ」は
\\	を使えばいいでしょう。	
\\	お前もまだまだだな。もう少し社会で揉まれた方がいい。 
\\	「揉まれる」を
\\	「社会に出て経験をして多くの教訓を得る」で表現しています。
\\	「世間の」。	
\\	警官という立場を利用して事件を揉み消すなんて、何てことだ! 
\\	で「隠す」という意味。	
\\	あー、肩が凝った。揉んでくれないかな。 
\\	「揉む」とは
\\	「マッサージする」ということですね。
\\	「擦る」、
\\	「固くなる」。	
\\	今、温泉に行くかスキーに行くかもめているんだよ。どうしようかな。 
\\	は「分裂している」、
\\	は「迷う、ためらう」という意味ですね。	
\\	もやもやした気持ちのままでは仕事が手につかない。 
\\	「もやもやした」を 
\\	(うっせきした)で表現しています。
\\	も同じような意味になります。	
\\	春も近づいてきたので、彼女は部屋の模様替えをすることにした。けど、今一家具の配置が気にいらなかった。 
\\	「模様替え」は
\\	「作り直す」、
\\	「作り替える」で表現すればいいでしょう。	
\\	この薬を飲むと眠気を催しますので、車の運転はしないでくださいね。 
\\	「眠気を催す」を
\\	「眠くさせる」で表現しています。	
\\	あんまり可哀想な話なんで、思わずもらい泣きしちゃった。 
\\	で「涙が出る」の意味になります。
\\	「シクシク泣く」。	
\\	盛り上がり方が全然ちがうというか、海外の年越しの方が華やかなのよね。 
\\	で「クライマックスに向けて盛り上がっていく」という意味になります。
\\	「陽気な、爽快な」。	
\\	昨日のライブもりあがったね。 
\\	「盛り上がる」は
\\	「興奮する」、
\\	は「熱狂させる」の意味。	
\\	音楽がロック調でカッコよくて、一段とその映画を盛り上げていた。それに思っていたよりかは面白かった。 
\\	この場合の 
\\	は「高める、増進する」、
\\	は「あおり立てる」の意味。	
\\	カーレースを盛り上げるために、戦闘機を飛ばしているのを見ました。さすが、カナダ。 
\\	ここでは「盛り上げる」を 
\\	(興奮を高める)で表現しています。
\\	も使えます。
\\	は「…の上をすれすれに飛ぶ」の意。	
\\	いっとき不利であっても、後半に何が何でも盛り返すという気概がそのチームには欠けている。 
\\	この場合の「盛り返す」は 
\\	でいいでしょう。
\\	は「気骨」、
\\	は「(精神的)たくましさ」の意。	
\\	引っ越しってやらなければいけないことが盛りだくさんで、だんだん面倒くさくなってきた。 
\\	で「細かい仕事が多岐に渡ってある」の意味になります。
\\	を使ってもいいでしょう。	
\\	この万年筆、インクが漏れてるじゃない。 
\\	「漏れる」は
\\	を使って表現します。この場合は進行形にします。	
\\	彼の提案に、皆、諸手を挙げて賛成した。 
\\	「諸手を挙げて」は
\\	と言います。
\\	は「大歓迎」の意味。	
\\	悪いのは向こうなんだから、ちょっと文句言いに行かなくては。このままでは納得いかない。 
\\	(何か言う)で「文句を言う」を表現しています。
\\	(文句を言う)を使ってもいいでしょう。	
\\	おかしなことに、彼は働けることがうれしいって言ってます。働いている時は、文句ばっかり言ってたけど。 
\\	「文句を言う」は 
\\	(ぶつぶつ泣き言をいう)で表現すればいいでしょう。	
\\	門限があるから、もう帰らなくちゃ。 
\\	「門限」は
\\	と言います。	
\\	これは頭がいいとか悪いとかの問題じゃなく、知っているか知らないかの問題だよ。 
\\	「問題」は 
\\	、あるいは、
\\	でいいでしょう。
\\	は「知性、知的能力」の意。	
\\	君の髪の色や写真の写り方を問題にするような会社は、君の方からお断りすればいいよ。 
\\	が
\\	を問題にする」に当たる表現です。	
\\	今回も玉虫色の決着になり、 問題は先送りされた。 
\\	「先送り」は
\\	「延期する、遅らせる」で表現できます。ここでは「玉虫色の決着をする」という表現の仕方にも注目してください。
\\	「あいまいな」、
\\	「故意に」。	
\\	それって問題発言ですよ。セクハラすれすれじゃないですか。 
\\	「問題発言」は、
\\	「(相手の言ったことに対して)全くいい感じがしない」と表現してもいいでしょう。また、直接的に、
\\	とも言えます。
\\	すれすれ」、
\\	「接している、触れている」の意。	
\\	反論してみたが、問答無用で押し切られてしまった。 
\\	「問答無用」をここでは
\\	「少しも聞いてもらえない、相手にされない」で表現しています。
\\	「反対する」。	
\\	朝から会議でやぁーんなっちゃうよ。 
\\	「嫌になる、うんざりする」は
\\	や
\\	で簡単に表現できます。	
\\	相撲に八百長があるって本当かな。 
\\	には「八百長を組む」という意味があります。	
\\	おれも50を過ぎて、少し焼きが回ったようだ。 
\\	には「腕が鈍る、落ちる」という意味があります。
\\	で「頭の回転が速い」の意。	
\\	彼がヤキモチ焼きなので、電話もできません。 
\\	「やきもち焼き」とは
\\	(嫉妬深い)ということですね。	
\\	彼は、やくざっぽいところがあって、なるべく会いたくない。 
\\	で「ギャングっぽい、やくざっぽい」という意味になります。
\\	「暴力団(の一員)」。	
\\	この前の約束どおり、そのことについて話すよ! 
\\	で「前に約束したとおり」の意味になります。	
\\	そう言えば友だちとお昼を一緒に食べる約束をしてたんだ。悪いけどそろそろ行かなきゃ。じゃあまた。 
\\	ここでは「約束」を 
\\	を使って表現しています。	
\\	その番組は、毎週健康に関することを取り上げていて、とても役に立つのです。 
\\	「役立つ」は 
\\	でいいでしょう。
\\	は「健康に関連した」の意。	
\\	彼は、役のうえでいやらしい男を演じているだけで実際はいいやつだ。 
\\	「役のうえで…」を 
\\	(スクリーン上で)を使って表現しています。
\\	は「低俗な、ポルノ調の」の意。	
\\	彼って、何やってもまわりの人に迷惑かけるんだよね。疫病神だよ、まったく。 
\\	には「厄介者、疫病神」の意味があります。
\\	「罰する」。	
\\	最近、悪いことが続いて起こり、厄除けしてもらおうかと思っている。 
\\	ここでは「厄除けをしてもらう」を
\\	「悪魔を追い払ってもうらう」で表現してみました。	
\\	彼はリストラされ、やけになって毎晩酔いつぶれていた。 
\\	この場合の「やけ」は
\\	は「自暴自棄の、やけくその」を使えばいいでしょう。	
\\	会社でリストラされ、彼はやけくそになって毎晩大酒を飲んでいた。 
\\	「やけくそ」は
\\	「自暴自棄の」を使えばいいでしょう。
\\	は「自暴自棄、やけ」のこと。	
\\	あ~、なんかムシャクシャする。やけ食いしたい気分。焼き肉の食べ放題にでも行かない? 
\\	「ストレスを発散させるために大食いする」で「やけ食い」のニュアンスを表現しています。
\\	「度を過ごして食べること」。	
\\	今日は時間が経つのがやけに遅く感じる。まだ、3時か・・・、定時まで後、2時間もある。 
\\	「やけに」をここでは
\\	「苦しいほどに」で表現しています。
\\	「ひどく」。	
\\	今年って台風がやけに多いと思わない? 
\\	(大変な)と 
\\	を組み合わせて「やけに多い」を表現しています。
\\	は「猛烈に」の意。	
\\	独立行政法人化によって存続する特殊法人が焼け太りする懸念がある。 
\\	ここでは「焼け太り」を 
\\	(不当に栄える)を用いて表現しています。	
\\	好きじゃないのなら、優しくしないで。 
\\	この場合の「優しい」は
\\	でいいでしょう。	
\\	何か困ったことがあったら言ってねって、とか優しい言葉をかけられると、反対に頼みずらくなる。 
\\	「優しい言葉」はそのまま
\\	でいいでしょう。
\\	しずらくなる」の
\\	の使い方にも注意。	
\\	生きていく上で自分と戦わなければならない時がある。そして自分にうち勝ってこそ強い精神力が養われていくのだ。 
\\	ここでは「養う」を
\\	(発達させる)、
\\	(築き上げる)を用いて表現しています。	
\\	夜食を食べないようにすれば、少しはやせるんじゃないの。 
\\	「夜食」は 
\\	でいいでしょう。
\\	は「やせる」の意。	
\\	なんでもかんでも安請け合いするから、身動きがとれなくなるんだよ。 
\\	「直ぐにたくさんのことを約束する」で「安請け合い」のニュアンスが表現できるでしょう。	
\\	安かろう悪かろうでは消費者は買ってくれない。 
\\	で「安くて汚い商品」の意味になります。
\\	「値段と品質を交換する(安いから品質が悪くてもいい)」。	
\\	正月休みぼけで、頭は働かないし、まだ何もやる気がしない。 
\\	ここでは「正月休みぼけ」を
\\	「正月休み中、だらだらしていたので」を使って表現しています。
\\	は「鈍る、強さを失う」、
\\	は「頭の鈍い」、
\\	は「やる気のない」の意。	
\\	そんなにやせ我慢しなくても、はやくこっちに来れば良かったのに。 
\\	「やせ我慢する」を
\\	(プライドから)、
\\	(無理して)と2通りに訳してみました。	
\\	どうしたんだよ、そんなに痩せこけちゃて。 
\\	で「やせ細って」の意味になります。	
\\	この道はやたらと一時停止が多いと思わないか? 
\\	この場合の 
\\	は副詞で「はるかに」の意味を表しています。
\\	は「…で散らかっている」の意。	
\\	海外旅行に行くと、やたらにブランド品ばかり買いあさっている女性をよくみかける。 
\\	ここでは「やたらに」を
\\	「見境なく、無差別に」を使って表現しています。
\\	は「買いまくる」の意。	
\\	どの企業も優秀な人材を雇用しようと躍起になっている。 
\\	で「…するのに苦しんでいる」の意味。
\\	は「もがく」の意。	
\\	旦那の給料だけでやっていくのはきついな。 
\\	は「何とかやっていく」、
\\	は「(収入の範囲で)やっていく」の意味。	
\\	この番組1年前に日本でやってたよ。 
\\	「放映する」を受け身の進行形にして「やってた」を表現しています。	
\\	美容院でやってもらっても(切ったり パーマしてもらっても)、だいたい自分が思っているのと違う事が多い 
\\	「やってもらう」は
\\	(使役)を使えばいいでしょう。
\\	「(結果が)... になる」。	
\\	こんな連中相手に、これ以上やってられない。 
\\	「連中」は
\\	、「やってられない」は「これ以上できない」と言い換えて表現すればいいでしょう。	
\\	今日って、いつもの番組やってる? 
\\	「やってる」は
\\	、あるいは、
\\	を使って表現できます。	
\\	やっと会えたわね! よく、あなたの顔をみせて! 
\\	「やっと」は
\\	を使って表現します。	
\\	わかる?やっとの思いでここまで痩せたの。私の努力を無にするようなことしないで。 
\\	「やっとの思い」は 
\\	(ようやく)で表現すればいいでしょう。	
\\	やっぱ、これって出さなきゃいけないの。 
\\	「やっぱ」をここでは
\\	「思った通り」、
\\	「このこと心配してたんだ」で表現しています。	
\\	夏はビール!!やっぱコレでしょ! 
\\	(おあつらえ向きなもの)で「やっぱこれでしょ」を表現しています。
\\	の使い方にも注目。	
\\	やっぱりね、そうだと思った。 
\\	は「確かにね」、
\\	は「はい」の意味。	
\\	やっぱり彼か。こんなの考えるの彼ぐらいだもん。 
\\	「やっぱり」は
\\	「分かってた」、
\\	「はっきりしてた」で表現できます。	
\\	お前(恋人)とは、よく喧嘩をするけど、それでもやっぱりお前のことが好きだ。 
\\	この場合の「やっぱり」は
\\	で表現すればいいでしょう。	
\\	釜飯を注文しようかな~。やっぱり、雑炊にしよう! 
\\	「やっぱり」のニュアンスを 
\\	(結局)、
\\	(代わりに)で表現しています。「釜飯」と「雑炊」の表現の仕方にも注目です。	
\\	やっぱり彼はやってくれたよ。彼ならやってくれると思ってたけどね。 
\\	(思った通り)と 
\\	を組み合わせて「やっぱり…してくれた」を表現しています。	
\\	機嫌が悪いからって私にやつ当たりしないでよ。 
\\	が
\\	に八つ当たりをする」に当たる言い回しです。	
\\	まだ決定したわけではないのですが、その日はスキーに行こうかっていう話が出ていて、宿次第って感じです。 
\\	で「泊まる所を見つけることにかかっている」の意味になります。	
\\	健康な肉体に健康な精神が宿る。 
\\	「宿る」は 
\\	(…に存する)で表現できます。
\\	は「健全な」の意。	
\\	タバコのヤニで、前歯が黄色くなってるよ。 
\\	「タバコのヤニ」は
\\	といいます。動詞
\\	の使い方にも注目。	
\\	とにかく矢は放たれたのです。やるしかありません。 
\\	「矢」ではなく「馬」を使って、
\\	「馬は納屋から出ている」と言い換えています。似たような表現に
\\	「賽は投げられた」があります。	
\\	やはり寒い地方で育った人は、寒さには強いのかな 
\\	この場合の「やはり」は、
\\	でそのニュアンスが出せます。	
\\	ヤブの歯医者にかかると歯の詰め物がとれていけねえ。 
\\	「ヤブの歯医者」は
\\	「無能な、役立たずの歯医者」と言えばいいでしょう。
\\	「詰め物」。	
\\	やぶからぼうに試してもだめだよ。 
\\	「やぶからぼうに試す」とは
\\	「考えずにやる」、
\\	「後先のことを考えずに急いでやる」ということですね。	
\\	この条件を呑んでいただけるのであれば、協力する事はやぶさかではないですよ。 
\\	「やぶさかでない」とは
\\	「喜んで... する」ことですね。	
\\	流鏑馬という伝統行事があるのであなたに見せてあげたい。実は私もテレビでしか見たことないんだよね。 
\\	ここでは「流鏑馬という伝統行事」を 
\\	で表現しています。第2英訳では流鏑馬を具体的に説明しています。	
\\	会社に経費削減を提案したら、自分のプロジェクトの経費を減らされてやぶ蛇になった。 
\\	(…にとって期待に反した結果になる)で「やぶ蛇」を表現しています。
\\	で「やぶ蛇になる」の意。	
\\	おい、ズボンが破れて、パンツが見えてるぞ。 
\\	「破れてる」は
\\	あるいは
\\	で表現します。「見えてる」は
\\	の他に、
\\	を使っても言えます。	
\\	ちょっとやぼ用で今日の飲み会行けないわ。 
\\	「やぼ用がある」は
\\	「ちょっと用事がある」、
\\	「やらなくちゃいけない雑用がある」などで表現できます。	
\\	人生、山あり谷ありだよ。 
\\	「山あり谷あり」には色々な言い方があります。
\\	「山頂と谷底」、
\\	「浮き沈み」など。また、
\\	「いいこ事もあれば、悪いこ事もある」のように言い換えてもいいでしょう。	
\\	きみのヤマ勘は当たったためしがないな。 
\\	で「物事を正しく推測する」という意味。
\\	は「当てずっぽう」の意。	
\\	彼は、山っ気があるからね。 
\\	「山っ気がある」とは
\\	「危険を犯す」の意味。	
\\	一日風邪で休んだだけなのに、机の上に書類が山積みになっている。 
\\	で山積みになる」は
\\	で表現すればいいでしょう。また、
\\	を使うこともできます。	
\\	大和魂をもつ僕としては、筋の通らない話には納得ができない。 
\\	「大和魂」を 
\\	で表現しています。
\\	は「生来の、固有の」、
\\	は「染みこませる」の意。	
\\	せっかくのまとまった休みで自分の時間が作れるから、普段やれないことをやってしまおう。 
\\	ここでは「まとまった休み」を 
\\	(合わせた)と 
\\	を組み合わせて表現しています。	
\\	やらなくちゃいけない仕事が山のようにある。 
\\	で「山のような仕事」の意味になります。	
\\	来週が、今取っているクラスの山場だな。それを越せば、暇になると思うけど。 
\\	ここでは「山場」を 
\\	(最も大変な部分)で表現しています。
\\	は「峠を越す」、
\\	は「重要な、重大な」の意。	
\\	うちの子どもは、バイキング形式のレストランで、何でも山盛りにして取ってくるので恥ずかしい。 
\\	が「山盛りにする」にあたる動詞です。
\\	「バイキング」。	
\\	行きたいのはやまやまなんだけど、先約があるので。 
\\	で「本当は行きたいんだけど」と言っています。
\\	「先約がある」。	
\\	この企画も山を越したから、あとはきみたちに任せるよ。 
\\	で「峠を越える」の意味になります。	
\\	英語の記事を最初頑張って読んでみたけど、難しかったのでやめて日本語記事を読み始めた。 
\\	この場合の「やめる」は
\\	(あきらめる)ということですね。
\\	(途中で)を付け足してもいいでしょう。	
\\	多分、仕事を辞めるかもしれません。なぜなら、自分の能力の限界を感じたからです。 
\\	「辞める」は
\\	を使えばいいでしょう。「限界を感じる」の表現の仕方にも注目。	
\\	あの番組はやらせにきまってる。 
\\	とは「ねつ造、偽造する」ことです。	
\\	コンピュータの世界は、次から次へと新しい物がでて、ついていくのが大変だよ!でも、またそこが、やり がいがあるけどね。 
\\	ここでは「やりがい」を
\\	「価値がある」、
\\	「満足を与えてくれる」を使って表現しています。
\\	の使い方にも注目!	
\\	なんか今の仕事ってやりがいがない。 
\\	「やりがいがない」は
\\	「かいがない、価値がない」、
\\	「てごたえがない、やりがいがない」を使えばいいでしょう。	
\\	やりくり算段する。 
\\	「やりくり」ですぐに思い浮かぶのが
\\	ですね。これと
\\	「帳尻を合わせる」を組み合わせればいいのです。
\\	は「何とかやっていく」の意。	
\\	この給料で毎月やりくりするのは大変だ。 
\\	で「かろうじて生計を立てる」の意味になります。
\\	は「収支を合わせる」の意。	
\\	スノーボードやり過ぎて、足がパンパンだ。年だなぁ。 
\\	「やり過ぎ」は
\\	を使って表現すればいいでしょう。
\\	は「堅い、硬直した」の意。	
\\	教育熱心なのもいいけど、あんまりやりすぎるとかえって子供のためにならないかも。 
\\	「やりすぎる」は
\\	を使えば簡単に表現できます。また、2番目の文では、
\\	「とりつかれる」を使っています。	
\\	やりたい放題することを「自由」ととり違えている人がいる。 
\\	(思うままに滅茶苦茶やる)で「やりたい放題」を表現しています。
\\	は「浮かれ騒ぎ」の意。	
\\	アメリカに行ってから、やりたいものを見つけるのでも遅くはないかな。本当に見つけられるか不安だけど。 
\\	で「自分がやりたい何か」の意味になります。
\\	(動名詞)を主語にしている点に注意。	
\\	子供には自分のやりたいようにさせたい。もちろん 間違った事をした時は、注意するけどね。 
\\	「やりたいようにさせる」は
\\	と
\\	「何でも」を組み合わせて表現できます。	
\\	この遣り場なき怒りをどこにぶつければいいのか? 
\\	のはけ口を見つける」、
\\	「怒りを置く場所」の意。	
\\	やる気は十分にあるんだけど、なかなか実行できないのが現実なのよ。 
\\	「やる気」は
\\	「モーティベーション」、あるいは、その動詞形
\\	を使って表現すればいいでしょう。	
\\	やる気がないやつは、このチームには必要ない。 
\\	で「やる気がない」、
\\	は「やる気に欠ける」の意味。	
\\	どこどこの英会話学校がいいとか悪いとかよく聞くけど、結局はその人のやる気次第だと思う。どこも、似たり寄ったりよ。 
\\	で「結局、結果は努力次第」の意味。
\\	は
\\	と書き換えてもいいでしょう。	
\\	今日は、朝からやる気十分だね。何かいいことあったのかな。 
\\	「やる気がある」は
\\	で表現します。
\\	「やる気満々の」。	
\\	一回り下の結婚相手見つけるなんて、彼女もやるじゃない。 
\\	「やるじゃない」は
\\	「かなりうまくやる」、
\\	「悪くない」で表現できます。	
\\	今日やれるところまでやってしまおう! 
\\	は「進歩、前進」の意味。
\\	「できるところまで進む」の意味。	
\\	いつも宿題は金曜日の夜にやろうと思っているのですが、結局土曜日の朝に急いでやることが多いです。 
\\	で「やろうと思う」のニュアンスを表現しています。
\\	(…するつもり)を使ってもいいでしょう。	
\\	ほんと、やんちゃな子供ね。 
\\	は「いたずらな」、
\\	は「腕白、いたずらっ子」の意味。	
\\	夏になると、幽霊物の番組が多くなるので憂鬱だ。 
\\	で
\\	を暗い気分にする」の意味になります。
\\	「憂鬱」。	
\\	海外旅行するたびに、現地で美容院に行ってみようと思うのだけど、いざとなると勇気が無くてなかなかね。 
\\	「勇気」は
\\	ですね。
\\	は「おじけづく」の意。	
\\	誰もいない店に入るのってちょっと勇気いると思わない? 
\\	で「…するには勇気がいる」の意味になります。	
\\	その新聞への投書を読んで勇気づけられた。 
\\	は「励みになる、元気づける」、
\\	は「…で励まされる」の意味です。	
\\	来週の金曜日に有休を取ります。 
\\	「有給休暇」は
\\	と言います。	
\\	彼は勇気を出して憧れの彼女をデートに誘った。 
\\	で
\\	する勇気を奮い起こす」の意味になります。
\\	にのぼせる、いかれる」。	
\\	うちらが払っている税金がそんな事に使われているの?もっと、有効に使って欲しいよ。 
\\	この場合の「有効」は
\\	「効果的」、
\\	「有益な」でいいでしょう。	
\\	近鉄バファローズの優勝記念セールには、初日だけでも、ものすごい人々が押し寄せた。 
\\	「優勝記念セール」は 
\\	でいいでしょう。
\\	は「群衆、集団」の意。	
\\	あの選手は、優勝候補の筆頭です。 
\\	で「…を勝ち取ると支持されている(有力だと思われている)」の意味。	
\\	君は相変わらず優柔不断だな。 
\\	は「煮え切らない、決断力のない」という意味。	
\\	朝御飯を食べないと言うよりは、時間がない。食べる時間があったら、寝る方を優先させるね。 
\\	「優先する」は
\\	を使えばいいでしょう。
\\	というよりはむしろ
\\	だ」の構文にも注意。	
\\	仕事より家庭を優先している環境が気に入りました。 
\\	「優先する」は
\\	「優先」、あるいは、
\\	~(…を~より前に置く)を使って表現すればいいでしょう。	
\\	会員になれば、優先的に予約が取れるらしいよ。 
\\	で「優先権を得る/持つ」の意味になります。	
\\	私は幽体離脱したことがあるんです! 
\\	で「幽体離脱の経験」という意味になります。	
\\	あいつは本当に融通が利かない。 
\\	「堅苦しい」、
\\	「柔軟でない」、
\\	「扱いにくい」の意。	
\\	彼女の答えは、優等生的で、面白みに欠けていた。 
\\	とは「教科書から取ってきた」の意味。
\\	「(機械的に)繰り返す」。	
\\	やっぱり、スピードスケートの競技って、小柄なほど風の抵抗が少なくなるから有利なのかな。 
\\	「有利」は
\\	でいいでしょう。
\\	「小柄」も一緒に覚えておくといいでしょう。	
\\	彼らは、コンピュータエンジニアを欲しがっている。ってことは、それ関係の仕事をしている私は、その点 (面接で)他の人より有利かもしれない。 
\\	「有利である」を
\\	に一歩先んじている」を使って表現しています。
\\	も使えます。	
\\	このパンフレットは有料ですか、それとも無料ですか? 
\\	「有料」は 
\\	(料金)を使って表現すればいいでしょう。	
\\	どちらの小説もとてもよく書けているので、優劣がつけ難い。 
\\	は「どちらか一つを選ぶのは難しい」の意味。
\\	を使ってもいいですね。	
\\	家で勉強しようと思っても、テレビとか誘惑するものが多くて続かないんだよね。 
\\	は「心をそそる」、
\\	は「注意をそらすもの」の意味になります。	
\\	一連の放火は愉快犯の仕業だった。 
\\	で「面白半分に」の意味になります。
\\	は「放火」の意。	
\\	ちょうどいい湯加減だね。これでお酒があれば言うこと無し。 
\\	で「いい湯加減」の意味になります。
\\	「ぴったりの温度」。	
\\	この浴衣、もう一つ上のサイズと換えてもらえますか。 
\\	「浴衣」は
\\	と言えばいいでしょう。「浴衣」を説明するなら、次のように言います。
\\	家を選ぶ時は、子供のことも考えて、学校、公園等の施設が近くにあるかどうかも重視します。床暖房なんかがあればなおよし。 
\\	「床暖房」は
\\	と言います。「なおよし」の表現の仕方にも注目。	
\\	ここから出ているバスは、東京駅行きのみですか? 
\\	行き」は
\\	、あるいは、
\\	で表現します。	
\\	道路に雪が積もっているので、雪かきをしないと。 
\\	は「シャベルですくってどかす」の後に
\\	を続ければ「雪かきする」ことになります。	
\\	山々は一面の雪景色できれいですよ。 
\\	ここでは「雪景色」を
\\	で表現しています。	
\\	やっぱり北海道は雪質が全然違いますか? 
\\	「雪質」は
\\	と
\\	を使うことで表現できます。	
\\	たまには雪見酒もいいもんだ。 
\\	で「雪景色を見ながらお酒を飲む」という意味になります。	
\\	輸血ミスしただけで死んじゃうんだね。も、病院も信用できないな。恐い恐い。 
\\	「輸血」は 
\\	と言います。
\\	は「…を信頼する」の意。	
\\	座席を譲り合う老夫婦。 
\\	「座席を譲り合う」をここでは
\\	「相手に座るように申し出る」で表現しています。	
\\	その金額ではこの品は譲れませんね。話になりませんよ。 
\\	この場合の
\\	は「売りに出す」という意味になります。「話にならない」の表現の仕方にも注目。	
\\	彼は、テレビを買うにあたってサイズは小さいので妥協したけど、フラットスクリーンだけは譲れないらしい。 
\\	この場合の 
\\	は「意見を変える」という意味です。
\\	(あきらめない)で「譲れない」を表現してもいいでしょう。	
\\	お金をかけなくても生活を豊かにすることはできると思うよ。 
\\	には「豊かにする」という意味があります。	
\\	冬は足が冷えて寝つけないので 、子どものころから湯たんぽを入れて寝ています。 
\\	が「湯たんぽ」に当たる表現です。	
\\	なんてやつだ。油断も隙もありゃしない。 
\\	の文字通りの意味は「彼がそばにいる時には、注意し過ぎてもし過ぎることはない」です。	
\\	ゆっくりしていって下さい。 
\\	で「腰を下ろしてくつろぐ」の意味になります。	
\\	今回は時間がなくてあまりじっくり見れなかったので、次回はゆっくり見てまわりたい。 
\\	で「(ゆっくり)時間を掛けて(取って)」の意味。
\\	「ゆっくりとしたペースで」。	
\\	今日一日ゆっくりしていたら、大丈夫だと思うので。ほんと心配かけてすみません。 
\\	で「1日ゆっくり過ごす」の意味。	
\\	今シーズンは、お疲れさまでした。オフは、ゆっくりと休んでね。 
\\	「ゆっくりと」は
\\	「時間をとって」で表現すればいいでしょう。	
\\	今月は残業続きなので、ゆっくり休める時間が欲しい。 
\\	で「十分な休養を取る」の意味になります。	
\\	このスパゲッティゆですぎで、全然おいしくない。 
\\	「ゆで過ぎ」は
\\	と言います。
\\	でも同じことが言えますね。	
\\	新車に買い換えるゆとりなどまったくない。 
\\	で
\\	する立場にはない」の意味になります。	
\\	ゆっくり本読んだり、テレビを見たりできる、ゆとりのある生活がしたいです。 
\\	ここでは「ゆとり」を 
\\	で表現しています。
\\	は「寝転がって丸くなる」の意。	
\\	俺、指が太いから、小さなキーボードじゃだめなんだよ。 
\\	は「(体や顔が)丸ぽちゃの」の意味ですが、指にも使えます。	
\\	指切りげんまん、嘘ついたら針千本飲ます。 
\\	「胸に誓う(十字を切る)」、
\\	「死んでもいい」の組み合わせれば「指切りげんまん... 
\\	に近い意味になるでしょう。
\\	「突き刺す」。	
\\	気になるから指を(ポキポキ)鳴らすの止めてよ。 
\\	が「(ポキポキ)鳴らす」に当たる動詞です。
\\	でも同じような意味になります。	
\\	指をポキポキ鳴らすの止めてくれないかな。気になちゃって集中できないよ。 
\\	が「指をポキポキ鳴らす」に当たる言い回しです。
\\	「集中」。	
\\	おやすみ、いい夢見ろよ 
\\	「いい夢を見る」には
\\	という決まった言い方があります。
\\	「ぐっずり(副詞)」。	
\\	将来、アメリカの大学院に留学しようと思っています。その夢の実現に向けて、今は、働きながら一生懸命に勉強しています。 
\\	で「夢を現実のものとする」という意味になります。	
\\	アメリカに留学することを夢のままで終わらせたくない。 
\\	で「夢のままで終わらせる」の意味になります。この
\\	は省略しても構いません。	
\\	すっごく不思議なんだけどよく同じ夢を見るんだ。でも、実際には行った事がない場所なの。 
\\	は「繰り返される」の意。これを用いて
\\	で「同じような夢を見る」を表現しています。	
\\	煙が煙突からユラユラ立ち上っている 
\\	は「曲がりくねって行く」、
\\	は「ゆっくりと流れるように上がって行く」の意味。	
\\	このズボン、腰の周りが少しゆるい。 
\\	は「ゆるい、だぶだぶの」という意味。	
\\	親も私が流暢に英語を話しているのを見れば安心して海外生活を許してくれるだろう。 
\\	この場合の「許す」は
\\	「相手の思ったようにさせる」を使えばいいでしょう。	
\\	みんな並んでいるのに、列の途中で割り込みをするなんて許せない。 
\\	「列に割り込む」は
\\	と言います。
\\	「許し難い」。	
\\	離陸してから5時間ぐらいたったときにしばらく揺れが続いた。 
\\	この場合の「揺れる」は 
\\	で表現できます。	
\\	なんでまた、リストラと汚職に揺れるそんな会社に就職するんだい。 
\\	は「ぐらつく、よろめく」という意味です。
\\	「苦悩」、
\\	「不安定な」。	
\\	夜遊びもほどほどにしないと。何?かみさんが怖くて帰れない?そりゃ気の毒だ。 
\\	ここでは「夜遊び」を
\\	「夜出掛ける」、
\\	「仕事の後で酒を飲み回る(はしご酒)」で言い換えています。	
\\	酔いつぶれるまで飲むなんて、会社で嫌なことがあったんだろうな。 
\\	で「酔いつぶれる」の意味になります。
\\	は「意識を失う」の意。	
\\	これらのサンダルは女性用(男性用)ですか? 
\\	「~用」は
\\	「~のため」を使って表現できます。	
\\	虫がすごいからスプレー用意して来たの。 
\\	「用意してきた」は
\\	(持ってくる)、あるいは、
\\	(…を準備する)で表現できます。	
\\	リスクの高い証券は、その代償としてより高い収益を要求する。 
\\	「要求する」は
\\	で表現します。
\\	(代償)。	
\\	蜂が容赦なく襲ってきた。 
\\	「容赦なく」は
\\	「情け容赦なく」でいいでしょう。	
\\	さっき電話もらったようだけど、何か用事だったの? 
\\	ここでは「用事」を
\\	(…な何かがあるのかな)で表現しています。	
\\	幼児虐待の問題が新聞やテレビでさかんに取り上げられている。 
\\	「幼児虐待」は
\\	と言います。	
\\	彼にはむかつくところがあるけど、まだ彼のことをよく知らないから、少し様子を見ようと思う。 
\\	で「これからどうなるか見るために、今はこのままにしておく」という意味。
\\	は「判断を保留する」という意。	
\\	君たちの発想はあまりに幼稚だ。 
\\	は「幼い」、
\\	は「子どもじみた」という意味。	
\\	最近、腰痛がひどくて同じ姿勢で長い間座っていられないんだ。 
\\	「腰痛」には
\\	という言い方がありますが、ここでは
\\	を使っています。	
\\	だめだなぁ、もっと商品が売れるような宣伝文句考えてくれなくちゃ。 
\\	ここでは「ような」を 
\\	(…の種類の)を使って表現しています。
\\	でもいいでしょう。	
\\	彼ってズル賢いのよ。親に見えないように私に意地悪するの。 
\\	で「両親に気づかれないようにして」の意味になります。
\\	は「ずるい」、
\\	は「卑怯な」の意。	
\\	うちの学校の給食って曜日によってメインメニューが決まってたんだ。月曜日は麺類、火曜日は、パンとか。 
\\	「曜日」は
\\	ですね。
\\	「(食事などを)出す」。	
\\	待ち合わせの時間と曜日を間違えて伝えてしまったのかと心配してました。 
\\	「時間と曜日」は 
\\	で表現します。
\\	は「…することになっている」の意。	
\\	今日何曜日なの?忙しくて、まるで曜日の感覚がなくなっちゃったよ。 
\\	「感覚がない」をここでは
\\	を忘れる」を使って言い換えています。	
\\	このスカートには上はどんな洋服を合わせるといいですか? 
\\	「(洋服に)合う」は
\\	で表現できます。
\\	を付け加えると「ピッタリ」という意味になります。	
\\	あいつは要領はいいけど、仕事は雑だよね。 
\\	「要領がいい」を「賢く仕事をする」と言い換えて
\\	で表現しています。
\\	「いいかげんな」、
\\	「見掛け倒しの」の意。	
\\	私、幹事をやるのが始めてだから 要領が悪いのかもしれません。 
\\	「要領が悪い」をここでは
\\	「こつを掴んでいない」と言い換えています。
\\	「幹事、まとめ役」。	
\\	お前の話はどうも要領を得ない。 
\\	で「要点が外れている」の意味。
\\	「的はずれ」。	
\\	この問題わからなかったの。 あなたに聞いてよかった。 
\\	この場合の「よかった」は
\\	「嬉しい」ということですね。	
\\	へぇ、チケット取れたんだ。よかったね。 
\\	この場合の「よかった」は 
\\	(すばらしい)を使えばいいでしょう。	
\\	仕方ないじゃん。ほかにどう言えばよかった? 
\\	で「ほかに何が言えたのか」の意味になります。	
\\	私でよかったら、なんでも聞いてくださいね。 
\\	(もしきみが、私が助けになると思うなら)で「私でよかったら」のニュアンスを表現しています。	
\\	きみのためによかれと思ってやってあげたのに。そんなに非難されるとは心外だな。 
\\	「よかれと思ってする」は
\\	でいいでしょう。	
\\	あの映画は、次回作を予感させるような終わり方だった。きっと、パート2が作られるね。 
\\	ここでは「予感させる」を 
\\	(…のお膳立てをする)で表現しています。	
\\	彼は彼女にとって、良き理解者であった。 
\\	「良き理解者」は
\\	で表現できます。	
\\	この映画よくできているなと思ったら、やっぱりハリウッド映画か。 
\\	「よくできた」は
\\	でいいでしょう。	
\\	彼女って昨日あんなに遅くまで残業してたのに、よく次の朝もきちんと来れるよね。感心するよ。 
\\	ここでは「よく」を
\\	(なんでもなかったように)で表現しています。	
\\	私はあの人とは合わないから一緒に仕事をしたくないとかって職場ではよくある人間関係ですよね。 
\\	「よくある」は 
\\	で表現すればいいでしょう。
\\	は「職場」の意。	
\\	よく言うよ!それ自分で壊したくせに! 
\\	で「どうしてそういうことが言えるわけ?」の意味。	
\\	「すべてあなたさまのお陰です」。 「よく言うわ」。 
\\	は「そう思ってもいないくせに」の意味になります。	
\\	よく考えたら、英語に触れたのって今週は今日が初めてだ。週に1時間じゃ、上達しないのも無理ないよね。 
\\	(振り返ってみると)で「よく考えたら」のニュアンスを表現しています。「無理ないよね」の表現の仕方にも注目。	
\\	よく探しもしないで「ないない」言うんじゃありません。 
\\	で「急がずにちゃんと探す」の意味。
\\	は「どこにもない」の意。	
\\	よく調べ上げた調査に基づいて質問されたので、たびたび答えに窮した。 
\\	は「十分調査された」という意味の形容詞。
\\	は「十分な調査に基づいた」の意。	
\\	よくそんなところで、働けるわねぇ! 
\\	「よくそんな」は
\\	と
\\	を組み合わせることで表現できます。	
\\	その映画は動物たちが主役なんだけど、かなりよくできた作品でした。 
\\	は「よくできた」の意味。	
\\	怪我、早く良くなるといいね。 
\\	「良くなる」は
\\	を使えばいいでしょう。	
\\	欲張ってあれやこれやと買いすぎで、一人で家に持って帰るのが大変だった。 
\\	は形容詞で「欲張りの」という意味です。
\\	は「家に着く」の意。	
\\	あの映画の中のキャラクターのロボットって、よく見ると気持ち悪いと思いませんでした?主役でかつ心温まる話なんだけど。 
\\	が「よく見る」に当たる言い回しです。
\\	は「嫌な」、
\\	は「心温まる」の意味。	
\\	"それはまさに""親の欲目""ってやつよ。 
\\	で「親が自然に持つ偏った見方」の意味になります。
\\	は「偏見を持った」の意。
\\	よくもまあ、こんなにたくさんの本を今までに買ったな~。自分でも感心するな。 
\\	「信じられない」で「よくもまあ」のニュアンスを出しています。	
\\	メダルは取れなかったが、彼はよくやったよ。 
\\	が「よくやる」に当たる言い回しです。
\\	と
\\	の使い方にも注目。	
\\	また大垣鈍行で帰省するの?ようやるわ。 
\\	「ようやるは」は
\\	「信じられない」や 
\\	「君の忍耐力はすごいよ」で表現できます。「鈍行」は
\\	と言います。	
\\	よくよく考えたら、私一人でこんなことしなきゃいけないなんて、バカバカしくなってきちゃった。 
\\	で「実際に注意深く考える」の意味になります。
\\	は「真剣な」の意。	
\\	まあ、欲を言ったらきりがないから、この辺で手を打つとするか。 
\\	は「終わりがない、際限ない」、
\\	「取り引きをする」、
\\	「妥協する、手を打つ」の意。これを組み合わせて表現すればいいでしょう。	
\\	欲を言えば、生徒達に何を学ぶかを決めさせることができたらいいと思う。 
\\	は「一つ変えられるのなら」、
\\	は「自分の思うようにできるなら」の意味。
\\	「発言権」。	
\\	寝過ぎると余計眠くなるって言うよね。たぶん今の私の状態はまさにそれ! 
\\	この場合の「余計」は比較級を使って表せばいいでしょう。
\\	で「なおさら眠くなる」の意味。
\\	「思いがけなく」。	
\\	余計なことを言っちゃったかも。今頃後悔しても遅いけど。 
\\	で「(余計なことを)言い過ぎる」の意味になります。
\\	について思い悩む」。	
\\	ちゃんと計画を立てて今後実行してくださいって励ましのメールをもらったけど、余計プレッシャーに感じるなぁ。 
\\	で「その代わりにものすごいプレッシャーを感じる」の意味になります。
\\	は「激励の」の意。	
\\	余計なたわ言は私には必要ない。 
\\	「余計な」は 
\\	(不必要な)で表現すればいいでしょう。	
\\	このコンタクトレンズ私に合わなくて、つけているとよけいに目が疲れる。 
\\	の文字通りの意味は「いつもよりもっと早く疲れる」です。
\\	で「余計に」を表現しています。	
\\	この古い建物は、横に広くない? 
\\	は「幅が広い」の意味。
\\	も使えます。	
\\	彼が横やりを入れれなければ、会議はもっと早く終わったんだよ。 
\\	「横槍」とは
\\	「途中で口を挟むこと」ですね。	
\\	彼の横を通るのが、とても怖かった。 
\\	「横を通る」は
\\	でいいでしょう。
\\	「怖い思いをする」。	
\\	おにぎりはこうやってラップを使うと、手も汚れないし手にはばい菌がいるからそれも防げる。で、そのまま包めば持っていける。 
\\	で「手が汚れないようにする」の意味。
\\	は「きれいなままである」の意。	
\\	あの女優は、デビュー当時は優等生的な役が多かったが、最近では大胆に汚れ役にも挑戦している。 
\\	で「社会から外れた(のけ者にされた)人物の役柄」の意味になります。
\\	は「描く」、
\\	は「社会のくず」の意味。	
\\	特に買いたいものがあるわけではないのだけど、行ってぶらっと見るのがいいんです。君にはこの良さがわからないだろうな。 
\\	は「良さが分かる」という意味です。
\\	「ぶらつく」、
\\	「あてもなく」。	
\\	値段によって3種類のコースを御用意しております。ご予算はいくらぐらいですか? 
\\	「どれくらいの金額を使われるのですか」で「ご予算はいくらくらいですか」と聞いている。	
\\	便利になるのはいいけど、機械に頼りすぎるのも善し悪しだね。 
\\	ここでは「善し悪し」を
\\	(諸刃の刃)を用いて表現しています。
\\	は「どのコインにも表と裏がある」の意。	
\\	人を寄せ付けないところがある。 
\\	で「知り合いになるのが大変」の意味。
\\	「追い払う、寄せ付けない」。	
\\	日本チームが予選通過するといいですね。 
\\	「予選」は 
\\	でいいでしょう。	
\\	あの映画は予想外の大ヒットだった。 
\\	は「予想外の」という意味の形容詞。
\\	(驚かす)を使っても同じようなことが言えますね。	
\\	なんか最初から目つきが悪いと思っていたらやっぱり彼は犯人だったか。予想的中だ! 
\\	は「予感が正しかった」の意味。
\\	「犯人」。	
\\	彼は大方の予想に反して、金メダルを獲得した。 
\\	と動詞
\\	と名詞
\\	の組み合わせに注目。	
\\	隣の人が食べているステーキを見ていたらよだれが出てきた。 
\\	この場合の
\\	は動詞で「よだれが出る」の意味。	
\\	会議中に寝てしまい、メモ帳の上によだれを垂らしてしまった。 
\\	「よだれを垂らす」は
\\	を使って表現します。	
\\	この間からエッチな話ばっかりしてるけど、もしかして欲求不満なの? 
\\	「欲求不満」は 
\\	で表現できます。副詞 
\\	を付け足してもいいでしょう。	
\\	よっこらしょっと。最近からだが重くなったなあ。 
\\	「よっこらしょ」は
\\	!や
\\	!でその感じが出せるでしょう。
\\	「動きがのろい」の意。	
\\	あれ、袋1枚しかもらえなかったの?おかしいな、スーパーでは買った量によって、袋もらえるはずだけど。 
\\	「よって」は
\\	次第」で表現します。2番目の訳文では、
\\	「必要なだけの袋」を使っていますので、「買った量によって」がその中に含まれることになります。	
\\	よってらっしゃい、みてらっしゃい 
\\	これは通行人や客を呼び止めるときによく使われる言い回し。他にもいくつか別の言葉を挙げておきましょう。
\\	「このすんばらしいスパーモップ見て、見て、見て」、
\\	「キッチンナイフが必要なら、ここよ」。【注】
\\	とは
\\	の意味を表します。	
\\	よっぽど印象に残る事じゃないと覚えてないんだよね 。 
\\	「よっぽど」は
\\	「かなり」を使えばいいでしょう。	
\\	大変だったんだよ、昨日。よっぽど電話しようかと思ったんだけどあまりにも遅いからやめたんだ。 
\\	で「本当は電話をするつもりだった」の意味になります。
\\	は「ものすごい」の意。	
\\	予定は未定。 
\\	「未定」は
\\	「決定していない、宙ぶらりんの」でいいでしょう。	
\\	予定外の出費しちゃった。 
\\	が「予定外」に当たる形容詞です。	
\\	来月は今月より仕事は暇だと思うから来月飲みにいきましょう。また予定がはっきりしたらメールするよ。 
\\	で「予定が固まる(具体化する)」の意味になります。
\\	は「決定する」の意。	
\\	明日の会議は予定通りに行われます。 
\\	で「予定された時間と場所で開かれる」の意味。	
\\	この間メールで10日大丈夫って書いたけど、昨日10日に予定入れちゃったからほかの日でもいい? 
\\	動詞 
\\	には「予定に組み入れる」という意味があります。
\\	は「起こる、生じる」の意。	
\\	英語を話す以前に、世の中の事について真剣に考えた事がないので自分の意見がない。 
\\	「世の中」は
\\	でいいでしょう。
\\	「本当の世の中の」。	
\\	世の中そんなに甘くなかった。 
\\	と 
\\	の組み合わせに注目。
\\	は「そんなに簡単ではない」の意。	
\\	彼女は世の中の汚い物を見たことがないんだ。いわゆる箱入り娘だよ。 
\\	とは「人生(社会)の恥部」のことです。
\\	は「保護する」、
\\	は「汚い物」、
\\	は「卑劣なこと」の意。	
\\	外務大臣の発言の余波が広がった。 
\\	「余波」は 
\\	と言います。
\\	は「結果」、
\\	は「広範囲に及ぶ」の意味。	
\\	あの人勝手に私のこと恋人呼ばわりしてるの。全然関係ないのに。 
\\	この場合の「呼ばわり」は 
\\	(…だと呼ぶ)で表現すればいいでしょう。	
\\	台所の蛍光灯が切れてる。予備の蛍光灯あったかな? 
\\	「予備」は
\\	を使って表現します。
\\	「蛍光灯」。	
\\	電話かけたけど、ずっと呼び出し音のままだったから、後でかけようと思って一度切りました 
\\	「呼び出し音のまま」を
\\	「リンリン鳴っている」で表現しています。
\\	「受話器を置く」。	
\\	今度の火曜に運動会があるんだけど、次の日は雨が降った時のための予備日になっているの。 
\\	で「…を取っておく」の意味。
\\	でも「予備日」を表現できますね。	
\\	1個食べたら呼び水になっちゃって結局1袋食べちゃった。 
\\	この場合の「呼び水」は
\\	に繋がる、導く」を使えばうまく表現できますね	
\\	夜更かししかたら、マジで眠い。顔洗ってシャキッとしよう。 
\\	「夜更かし」は
\\	でいいでしょう。ちなみに「徹夜する」は
\\	と言います。
\\	は「元気を出す」の意。	
\\	痴漢呼ばわりするとは、無礼じゃないか。何もしてないんだから。 
\\	呼ばわり」は
\\	を使えばいいでしょう。「痴漢」は
\\	と言います。	
\\	の 
\\	は一つ余分だと思います。 
\\	「余分」は
\\	、あるいは、
\\	で表現します。
\\	明日は予防接種だよ。泣かないでがんばろうね。 
\\	「予防接種」は
\\	と言います。
\\	は「注射」の意。	
\\	飲みに誘おうと思ってたのに、最近疲れぎみだなんて、あいつ予防線をはったな。 
\\	「予防線をはる」とは「自分自身を守る」ことですから
\\	が使えます。	
\\	読みがあまかったな。この仕事1週間で終わると思ったのに。もう2、3日かかりそうだ。 
\\	「読み」は
\\	(判断、評価する)を使えば表現できます。
\\	(読み違える)、
\\	(判断を誤る)。	
\\	こんなに官僚の汚職が続くと、世も末という気になってくる。 
\\	で「世の中が終わりになる」という意味。
\\	も使えます。	
\\	予約制なので、自分が都合のいい日に予約して受講できます。 
\\	「予約制」は
\\	で表現します。「予約する」は、
\\	を使えばいいでしょう。	
\\	(試験日前夜に映画だなんて)余裕だね。羨ましいよ。 
\\	で
\\	する余裕がある」の意味になります。
\\	「羨む」。	
\\	余裕で間に合ったね。急ぐ必要なかった。 
\\	で「十分時間を残してそこに着く」の意味になります。	
\\	レジの作業に手いっぱいで、お客さんと話す余裕がない。 
\\	ここでは「余裕がない」を 
\\	(…する時間がない)で表現しています。	
\\	今月は少し余裕があるから何かおいしいものでも食べにいこうか? 
\\	(余分な)を使って「余裕がある」を表現しています。	
\\	週末にゆっくりする余裕などない。仕事が山ほどあるんだ。 
\\	で「…する時間がない」の意味。	
\\	失敗した!もっと、余裕をもって計画たてるべきだった。 
\\	ここでは「余裕を持って計画を立てる」を
\\	「適切な計画を立てる」と言い換えて表現しています。	
\\	今回、オーストラリアに行ってきて、すごしやすい気候で、よりいっそう住みたいという気持ちが高まりました。 
\\	「よりいいそう....」をここでは
\\	と
\\	で表現してみました。
\\	の使い方がポイントですね。	
\\	あなたと会って、より一層あなたのことが、 好きになりました。 
\\	この場合、「一層」は
\\	、あるいは、
\\	を使って表現すればいいでしょう。
\\	(動名詞)を主語にする言い方はなかなか出てきませんね。	
\\	このラーメン、1個で買うよりも3個買ったほうが、1つ当たり10円安いよ。 
\\	「割引」と
\\	文を使って「もし…すれば、10円安くなる」を表現しています。	
\\	結婚してすぐ別居してたあの2人だけど、最近よりを戻したみたいで嬉しい。 
\\	「よりを戻す」とは
\\	(再び一緒になる)ということですね。
\\	は「和解する、仲直りする」の意。	
\\	人の不幸を喜ぶなんて最低!! 
\\	で「…で喜ぶ」の意味。
\\	は「引き出す」の意。	
\\	はじめまして、よろしくお願いします。 
\\	「よろしくお願いします」をここでは、
\\	を使って表現しましたが、この英語は決して自然とは言えません。
\\	が「はじめして、よろしくお願いします」の意味になると考えた方がいいでしょう。	
\\	(会議後に別れる時)では、そこのところよろしくお願いします。 
\\	「よろしくおねがいします」のニュアンスを
\\	と、
\\	(よく考える)を使って出しています。	
\\	彼の事、今後ともよろしくお願いします。 
\\	(任せる)の使い方がポイントです。	
\\	申し訳けないけどよろしく頼む(たとえば誰かに頼み事をしたとき) 
\\	ここでは「よろしく頼む」を
\\	「君の助けが必要だ」と言い換えています。	
\\	なんで限定品という言葉に弱いのだろうか 
\\	は「我慢する」、
\\	は
\\	に弱い人」の意味。「限定品」という言い方も一緒に覚えておきましょう。	
\\	弱い犬ほどよくほえるって言うだろ。 
\\	の文字通りの意味は「吠える犬はめったに噛むことはない」です。	
\\	そんな弱気になってどうするの?やってみないとわからないよ。 
\\	は「臆病な」、
\\	は「弱腰の」の意味。	
\\	そんな弱腰でどうするんだよ。一か八かやってみろよ。 
\\	は「意気地のない」、
\\	は「弱腰である」の意味。「一か八か」の表現の仕方にも注目。	
\\	私、要領が悪くていつも損ばかりしてる気がする。あーあ、世渡り上手になりたいな。 
\\	「世渡り上手」をここでは
\\	「もっと要領がいい(効率的な)人になる」、
\\	「もっと賢く生きる」と言い換えています。	
\\	こんなことで弱音を吐いてどうするんだ。 
\\	で「…について泣き言を言う」の意味。
\\	は「取るに足らない」の意。	
\\	お!雪がだんだん弱まってきた。今夜にはやんでもらわないと明日大変だな。 
\\	には「(雨や雪が)やむ」という意味があります。これを進行形にして「弱まる」を表現しています。
\\	は「和らぐ、緩む」の意。	
\\	人の弱みにつけ込みやがって、何て嫌な奴なんだ。 
\\	が「弱点」、
\\	が
\\	を責め立てる」という意味。	
\\	会社に40歳近い女性の先輩がいるのですが、いつも同じ話をするのでいらいらします。 
\\	40で「40歳に近い」の意味になります。
\\	(近づく)は動詞としても使えます。	
\\	来年の今ごろは、君と一緒に暮らしてるだろう。 
\\	で「来年の今頃(この時期)」の意味になります。
\\	「今から一年後」。	
\\	仕事がないときは無性に仕事がしたいと思いましたが、いざ仕事を始めるとプレッシャーを感じ、やはり学生が楽だなと感じてます。 
\\	「楽」は 
\\	で表現すればいいでしょう。「プレッシャーを感じる」の表現の仕方にも注目です。	
\\	そんな烙印は押されないほうがいいに決まっている。 
\\	「烙印を押す」は
\\	を動詞として用いて表現できます。	
\\	楽して成果を得ようという考えがそもそも間違いだ。 
\\	「楽して」は 
\\	で表現できます。
\\	の使い方にも注目。	
\\	そんな仕事、楽勝だよ。1時間もあれば終わるよ。 
\\	で「楽な仕事」の意味になります。
\\	「朝飯前のこと」。	
\\	まさかあの候補が落選するとはね。 
\\	で「落選する」の意味になります。	
\\	この仕事早いとこ終わりにして楽になりたい。 
\\	「楽になりたい」は、
\\	、あるいは、
\\	で表現すればいいでしょう。	
\\	照れてそっぽむくなんて、あいつらしいよ。 
\\	「~らしい」と
\\	を組み合わせことで「とっても~らしい」の意味になります。
\\	「顔を赤らめる、恥ずかしがる」。	
\\	千葉にドイツ村っていうのができたらしいよ。今度行って見ようよ。きっと、本場のドイツ料理とかが食べれるんじゃないの? 
\\	ここでは「らしい」を
\\	(…だそうだ)で表現しています。
\\	は「本物の」の意。	
\\	こんな所に居ては、いつまでたってもらちが明かない。 
\\	「らちが明かない」を 
\\	(何も成し遂げられない)、
\\	(どうにもならない)で表現しています。	
\\	そんなこと言っても、らちがあかないよ。 
\\	ここでは「らちがあかない」を
\\	「少しも話が進まない」と言い換えて表現しています。
\\	「同意」。	
\\	あなたはいつも楽観的でいいわね。 
\\	「楽観的な」は
\\	ですね。	
\\	今日、一緒にランチしない? 
\\	が「ランチする」に当たる言い回しです。	
\\	彼女は私のよき理解者です。 
\\	「よき理解者」のニュアンスを
\\	(僕より自分のことをよく知っている)でだしている。	
\\	あの二人は利害関係で結びついてるんだ。 
\\	で「利害関係がある」の意味になります。	
\\	家で自分ですべて英語で話そうとしても、結局自分の力量でしか話してないんだよね。これじゃ、上達しないと思うのだけど。 
\\	「力量」は
\\	「能力、才能」で表現できます。	
\\	あいつは理屈っぽいから、話していて疲れるんだよ。 
\\	ここでは「理屈っぽい」を
\\	「理屈・理論のことばかりを繰り返す」、
\\	「詭弁、こじつけ」で表現しています。	
\\	あいつは何かと理屈をこねてばかりいて、実行しないやつだ。 
\\	とは「合理的に説明する、正当化する」という意味になります。	
\\	利子がほとんどつかないので、銀行にお金を預けても意味がない。 
\\	には「興味」だけではなく「利子」の意味もあります。
\\	は「ほとんど」の意。	
\\	我々がリスクを背負わねばなるまい。 
\\	「リスクを背負う」は 
\\	と 
\\	を組み合わせて表現すればいいでしょう。	
\\	あの会社もとうとうリストラで年内までに1000人ほど解雇するらしい。 
\\	「リストラ」は
\\	「数を減らす」で表現できます。	
\\	あ、昨日のクラブでのノリがまだ残ってて、知らずにリズム取ってる。 
\\	ここでは「リズム」を 
\\	を使って表現しています。	
\\	彼にはリズム感がないからね。打楽器はダメじゃないかな。 
\\	(リズム感がない)は
\\	で表現します。
\\	(打楽器奏者)。	
\\	僕の理想は、自分の興味がある物、好きな事をやって稼ぐ。そうなったら、最高だね。 
\\	「理想的には」を使えばいいでしょう。文末に
\\	「完璧なのに」があるので、
\\	を省略しても構わないでしょう。	
\\	今回、うちの地区ってだれが立候補しているのだろう? 
\\	には「立候補する」という意味があります。
\\	は「立候補者」の意味。	
\\	彼は最近、人間的に立派になったね。 
\\	「立派な」をここでは簡単に
\\	を使って表現しています。	
\\	彼女はいつも理にかなったことをいう。 
\\	「理にかなう」は
\\	で表現します。これは、
\\	のように言い換えることもできます。	
\\	無理してダイエットしない方がいいよ。リバウンドでかえって太っちゃうから。 
\\	「リバウンド」はそのまま
\\	が使えます。また、
\\	でも同じ様なことが言えますね。	
\\	これ贈り物なので箱に入れるか包装してリボンをかけてください。 
\\	で「…にリボンをかける」という意味になります。
\\	でもいいでしょう。	
\\	アメリカに留学したいって簡単に言うけど、どれくらいお金がかかるか分かってるのかい。 
\\	「アメリカに留学する」は
\\	でいいでしょう。	
\\	あの科目は必修だから、落とすと留年することになるぞ。 
\\	「一年を繰り返す」が「留年」に当たる言い回しです。
\\	「落とす、落第する」。	
\\	両手に花でいい環境じゃん! 
\\	「それぞれの腕に可愛い子」で「両手に花」を表現しています。また、
\\	の間に挟まれている」を使ってもいいですね。
\\	とは「絶世の美女」のこと。	
\\	両面コピーの仕方を教えてくれないかな。 
\\	この場合の「両面」は形容詞
\\	か
\\	を用いて表現すればいいでしょう。	
\\	彼女を愛してたんじゃないの? 利用価値がなくなったからもう会わないなんてどういうこと? 
\\	「利用価値がなくなる」とは 
\\	(使えない、役に立たない)ということですね。
\\	は「…を愛している」の意。	
\\	図書館の利用範囲を学外者にまで拡大してはいけない。 
\\	(…だけが)と
\\	(…を使用する資格がある)を組み合わせて「利用範囲を拡大してはいけない」を表現しています。	
\\	水道代についてですが、何日から何日までの利用分が何日に請求されるのですか? 
\\	(水の使用)で「利用分」を表現しています。
\\	は「請求書を送る」の意。	
\\	理論と実践は違うから、実際にやってみな いと分からないよ。 
\\	「理論」は
\\	、「実践」は
\\	ですね。	
\\	そのへんは臨機応変に対応して下さい。 
\\	で「臨機応変にやる」の意味になります。	
\\	稟議書とは通らないものと相場が決まっている会社もある。 
\\	ここでは「稟議書」のことを
\\	「回覧用書類」をよびその後の
\\	で説明しています。	
\\	臨時の出費が多かったので、今月は家計のやりくりが大変だ。 
\\	で「一回限りの出費」の意味。
\\	をめちゃくちゃにする」、
\\	を台無しにする」。	
\\	彼女は事件の状況を臨場感たっぷりに語った。 
\\	とは「本当にそこにいるような気にさせる」という意味。
\\	「説明、描写」。	
\\	彼女の凛とした姿には惚れ惚れする。 
\\	「凛とした」を 
\\	(威厳のある、品位のある)で表現しています。
\\	は「魅惑的な」、
\\	は「素晴らしい」の意。	
\\	私は留守をまかされているもので、この家の者は今出かけているのですが。 
\\	「留守をまかされる」は
\\	と
\\	(家主がいない間)を組み合わせて表現すればいいでしょう。
\\	は「(人の家の)留守番をする」の意。	
\\	昨日何度か電話かけたけど、留守電になっててつながらなかったんだけど。 
\\	で「留守電になってた」を表現しています。	
\\	留守番電話にメッセージ残すの苦手なんだ。 
\\	「留守番電話」は
\\	と言いますが、ここではそれを出さずに、
\\	で「留守番電話にメッセージを残す」という意味を表しています。	
\\	あの子、早々と就職が決まったから、ルンルン気分よね。 
\\	は「有頂天になって」、
\\	は「幸せ一杯な」の意味になります。
\\	「すぐに」。	
\\	柔道は礼に始まり礼に終わる。 
\\	は「礼儀正しいお辞儀」の意味になります。	
\\	霊感がないようで、未だ幽霊は見たことありません。 
\\	で「向こう側(霊界)とは通じていない(接点がない)」の意味になります。	
\\	君の息子は、礼儀正しくて良い子だね。 
\\	は「行儀のよい」という意味です。
\\	は「礼儀正しい」の意。	
\\	人って一度パニックになると何がなんだかわからなくなるじゃん。後で、冷静に考えればちゃんと理解できるんだけど。 
\\	「冷静に考える」は
\\	でいいでしょう。
\\	を見失う、分からなくなる」。	
\\	審判に求められることは、冷静な判断力と決断力だ。 
\\	この場合の
\\	は「慎重な、じっくり考えた」という意味になります。
\\	は「冷静な」の意。	
\\	今年の梅雨は例年になく雨が少なかった。 
\\	は「例外的に」の意味。
\\	も使えますね。	
\\	夏用の礼服を新調したんだ。 
\\	「礼服」は
\\	ですね。
\\	「衣類」は数えられませんので、
\\	のような言い方をします。	
\\	週末に招待されたパーティーは礼服着用なのかい? 
\\	とは「礼服着用のパーティー」のこと。	
\\	私、彼氏いない歴2年です。 
\\	ここでは「歴」を現在完了を用いて表現しています。
\\	の使い方にも注目。	
\\	仕事には見えないかもしれませんが、れっきとした仕事ですよ。 
\\	には「ちゃんとした、立派な」という意味があります。	
\\	後から来た方は、この列の後ろに並んでください。 
\\	は「整列する」、
\\	は「列に並ぶ(入る)」の意味。	
\\	今日は夕飯作るのが面倒だから、レトルトカレーですまそう。 
\\	ここでは「レトルトカレー」を
\\	「パックに入ったカレー」で表現しています。	
\\	連休はどこ行っても混んでるから、家にいるのが一番だよ。 
\\	「連休」を
\\	「(五月の)ゴールデンウイーク」を使って表現しています。もし、「三連休」と言いたければ、
\\	と言えばいいでしょう。	
\\	1人が買うと、連鎖反応のようにほかの人もそれを見て買いたくなる傾向があります。 
\\	ここでは「連鎖反応」を 
\\	で表現しています。	
\\	2日連続で海に行ったから焼けたよ。 
\\	「連続で」は
\\	で表現できます。	
\\	巨人また負けたのかよ。これで3連敗か。 
\\	「連敗」は
\\	「伸ばす」と
\\	「連続」を使えばうまく表現できます。	
\\	こっちから連絡するまで何も言ってこないなんて、もう2度とあの旅行会社には頼まないぞ。 
\\	「連絡する」とは 
\\	(電話する)ということですね。
\\	も「…に連絡する」の意味。	
\\	明日の運動会が実施されるかどうかは、電話連絡網を使ってお知らせいたします。 
\\	ここでは「連絡網を使って知らせる」を
\\	と言い換えて表現しています。
\\	の表現の仕方にも注目。	
\\	最近、小さな字が見えにくくてね。老眼鏡をかける年になったのかな。 
\\	「老眼鏡」は
\\	と言います。
\\	は「遠近両用メガネ」のことです。	
\\	長い浪人時代を経て、彼はようやく職にありついた。 
\\	この場合の「浪人」とは
\\	「仕事がない」ということですね。	
\\	どこの入試にも合格できなかったので、浪人することにした。 
\\	英語には「浪人」に当たる言葉がありませんので、
\\	(もう一度受験するためにさらに1年自分で勉強する)と具体的に説明しています。	
\\	彼は、第一志望校に合格しなかったら浪人する覚悟で入試に臨んだ。 そして見事に合格した。 
\\	ここでは「浪人する」を
\\	「二度目受験のために(高校と大学の)中間の状態で
\\	1年間を勉強に費やす」で表現してみました。
\\	の可能性に対して心の準備をする」、
\\	「見事に」、
\\	という恐ろしい思いに立ち向かいながら」。	
\\	だまっていようと思っていたのですが、老婆心ながらアドバイスさせていただきます。 
\\	「老婆心ながら」は
\\	「親切から」で表現できますが、ここでは
\\	「どちらでもいいのですが」を使っています。	
\\	日本には定年制度というものがあります。父は今月25年間勤めた会社を定年退職するので労をねぎらう意味で家族で旅行に行きます。 
\\	「労をねぎらう」を
\\	(感謝の気持ち表す)と言い換えて表現しています。
\\	(定年制度)。	
\\	これ、録音したからあげるね。 
\\	「録音」は可算名詞ですから冠詞
\\	がつきます(2番目の英訳)。	
\\	えっ! 録画されてなかったの?ちゃんとセットしたはずなのに。今日は、それを見ることだけが唯一の楽しみだったのに 
\\	「録画する」は
\\	を使えばいいでしょう。ここでは「唯一の楽しみ」の表現の仕方にも注目。	
\\	ああいうひどいことばかりしているとロクな死に方をしないね。 
\\	で「ちゃんとした死に方をする」の意味になります。
\\	は「心安らかに」の意味。	
\\	バーゲンだから急いできたけど、ろくな物が残ってない。 
\\	で「何もいい物は残っていなかった」の意味になります。	
\\	露天風呂とは名ばかりで、屋根のない小さな湯船があるだけじゃないか。 
\\	「露天風呂」は
\\	と言います。
\\	は「本物の」の意。	
\\	彼は酔っぱらって、ろれつが回らなくなっていた。 
\\	は「不明瞭に言葉を話す(早口で)」、
\\	とは「訳の分からないことを言う程酔っている」の意味です。	
\\	そんな話は論外だよ。 
\\	は「問題外の」、
\\	は「考慮にも値しない」の意味になります。	
\\	この言葉の定義は、論者によって違う。 
\\	ここでは「論者」を 
\\	で表現しています。	
\\	その事件に対する世界の論調を新聞でチェックしてみてくれ。 
\\	「論調」をここでは
\\	「意見がどうなっているか(どのように意見が形成されているか)」で表現しています。	
\\	彼の話は論理性に欠けているので、説得力がない。 
\\	で「論理性に欠ける」の意味。この場合の
\\	(不足する)は形容詞。	
\\	若いうちに多く失敗しておいた方がいい。 
\\	「若いうち」は
\\	で表現できます。	
\\	髭剃ったら、10歳は若返って見えるよ。 
\\	「人から年齢を10歳取る」の
\\	の使い方に注目。	
\\	改革により生じる痛みを、国民全員が分かちあうという認識が重要だ。 
\\	「分かちあう」は 
\\	で表現できます。
\\	は「負担、重荷」、
\\	は「耐える」の意。	
\\	そのパンフレットには、わかったようなわからないような抽象的なスローガンが並べられていた。 
\\	で「何となく分かっているようだが、はっきりはしていない」という意味になる。	
\\	研究者たちは、それが人体に影響を与えるとわかっていて何故作り続けるのだろう。 
\\	「わかっているのに」は
\\	(分詞構文)「知っていて」で表現できますね。	
\\	あなただったら、わかってくれると思って相談したのに。どうも見当違いだったようだ。 
\\	「わかってくれる」は 
\\	(理解する)、
\\	(共感を得る)で表現すればいいでしょう。	
\\	彼女の授業はすごくわかりやすい、わからない時はわかるまで教えてくれる。 
\\	「分かりやすい」は
\\	でいいでしょう。
\\	と
\\	の使い方にも注目!	
\\	いろいろ情報を集めているうちに、一番この店で買うのが安い事がわかった。 
\\	この場合の「わかる」は
\\	でいいでしょう。
\\	「徹底的な」、
\\	「集める」。	
\\	もうすぐお別れですね。お店は、楽しかったですか。 
\\	で「そろそろさよならを言う時間です」という意味になります。(注)ここでは「店」は英訳していません。	
\\	もう彼氏とは別れちゃったの。 
\\	で
\\	と別れる」という意味になります。	
\\	彼はいつ見ても若々しいね。うらやましいよ。 
\\	「若々しい」は 
\\	で表現すればいいでしょう。
\\	は「うらやむ」の意。	
\\	ここは和気あいあいとした雰囲気でいいですね。 
\\	「和気あいあい」は
\\	「仲のいい」、
\\	「好意的な」、あるいは、
\\	「暖かい」、
\\	「気が合う」を組み合わせて表現すればいいでしょう。	
\\	党内の身内からも反対の声が沸き上がった。 
\\	で「表面化する」の意味になります。	
\\	その政治家は、脇が甘くて、暴力団に弱みを握られている。 
\\	「脇が甘い」をここでは「(人が)簡単に弱みにつけこむ
\\	を使って表現しています。
\\	は「搾取者」のことです。	
\\	もう少し自分の立場をわきまえないと。 
\\	「わきまえる」とは
\\	「もっと慎重に考える」ということですね。	
\\	今年は(受験の)あの大学の帰国子女の枠は少ないらしいよ。 
\\	この場合の「枠」とは
\\	「数」のことですね。
\\	「制限、範囲」。	
\\	お風呂沸いてるけど、食事の前に入る? 
\\	は「熱くなっている」、
\\	は「浴槽に熱い湯を満たす」の意味。	
\\	私が盗むわけないじゃん。何か証拠でもあるの? 
\\	「絶対に 
\\	しない人」の意味。
\\	「絶対に 
\\	にない」。	
\\	たった一杯、とか言ったってそれで済むわけがない。 
\\	「それで済むわけがない」を 
\\	(そんなことあるはずがない)で表現しています。	
\\	彼の説明を聞いていたら、訳がわからなくなってきた。 
\\	で
\\	が理解できない」の意味になります。	
\\	僕が夜寝るのが遅い理由は、子供の勉強に付き合っているわけではない。11時からインターネットすることにしているからだよ。 
\\	、この
\\	の使い方に注目!	
\\	そんなに急いでいるわけではないので暇な時に添削して下さい。 
\\	で「そんなに急いでいるわけではない」のニュアンスが出せます。
\\	「修正する、手直しする」。	
\\	今食べるわけにはいかない。ここは、ぐっと我慢しよう。 
\\	で
\\	するのはダメ」の意味。
\\	「意志」。	
\\	わけもなくビールが飲みたくなることがあるんだ。 
\\	この場合の
\\	は「何としても」という意味になります。	
\\	わざとらしい。そんなことしても、ばればれだよ。 
\\	「意識して」、
\\	「故意に」、
\\	「わざとらしい」の使い方がポイント。	
\\	信頼して相談したのに、結局その事が後になって災いを招いて会社の問題にまで発展してしまった。 
\\	で「問題を招く(引き起こす)」の意味になります。
\\	「やっかいなことになる」。	
\\	今日は、遠いところわざわざ来てもらったんだから、おごらせてよ。 
\\	を直訳すると「ここまではるばる特別にやって来る」となります。	
\\	他にもっと楽な生き方もあるだろうに、なんでわざわざ大変な道を選ぶの? 
\\	で「わざわざ...する」の意味になります。
\\	は「あえて(わざわざ) 
\\	する」の意。	
\\	わざわざこっち向いて鼻をかむなんて最低! 
\\	は「故意に、わざと」の意味。
\\	「わざわざ」。	
\\	パソコンを使い出してから、漢字がわすれがちになってきた。自分で書く機会が少なくなったからね。 
\\	しがち」は
\\	する傾向がある」を使って表現します。	
\\	私のことを忘れないでください。 
\\	「忘れないで」は 
\\	を使って表現すればいいでしょう。	
\\	合コンに誘ってくれるのはいいんだけど、1人でいるほうが好きなんでどうも煩わしいんだ。 
\\	ここでは「煩わしい」を 
\\	と 
\\	(厄介な問題)を使って表現しています。
\\	は「親睦会」の意。	
\\	これ、私がつけている香水。 いつも、私のことを、思い出してね。 
\\	「(香水を)つける」は
\\	と言います。	
\\	(プロポーズされて) 本当に、私でいいの? 
\\	には「十分である、間に合う」という意味があります。簡単に
\\	と言ってもいいですね。	
\\	私にしてみれば、その試合はただお互いの力の強さを見せびらかしているだけにすぎない。 
\\	で「私の立場からすると(見ると)」の意味になります。簡単に
\\	と言ってもいいでしょう。
\\	「誇示する」。	
\\	私の存在は、あなたにとってたいした物ではない。 
\\	「存在」を敢えて使う必要はないでしょう。この文脈では
\\	で十分です。
\\	と
\\	の使い方にも注意してください。	
\\	私のぶんもがんばって下さい。 
\\	で「私のために出来ることをして」の意味になります。	
\\	今度の温泉旅行に一緒に行けなくて残念だけど、私の分まで楽しんできてね。 
\\	「私の分まで」は 
\\	、あるいは、
\\	(私の代わりに)を使って表現すればいいでしょう。	
\\	もし私の身に何か起こったら、ここを調べてみて。 
\\	「私の身」は
\\	でいいでしょう。	
\\	お手数ですが、本日、そちらのホテルに宿泊している山田さんに、このファックスを渡してください。 
\\	「渡す」は 
\\	で表現します。
\\	の使い方にも注意。	
\\	3番線ホームへ行くには、連絡橋を渡ってください。 
\\	この場合の「渡る」は 
\\	でいいでしょう。
\\	は「通路」、
\\	は「歩道橋」の意。	
\\	これが話題の新商品です。 
\\	「話題」は
\\	「皆が話している」と言えばいいでしょう。また、
\\	「絶対に手に入れたい品」とも言えますね。	
\\	今、韓国ではどこへ行ってもワールドカップの話題でもちきりだ。 
\\	「話題でもちきり」を 
\\	で表現しています。
\\	は「(至る所まで)それ一色の」の意。	
\\	山手線は輪っかになっていて、右回りが上り、左回りが下り線なんだよ。 
\\	「輪っかになっている」を
\\	(輪を形成する)で表現しています。	
\\	その家は典型的な和洋折衷の作りだった。 
\\	「和洋」は
\\	、「折衷」は
\\	「混ぜたスタイル」、あるいは、
\\	「混ぜたもの」を使って表現します。	
\\	もうかってもうかって、笑いが止まりませんね。 
\\	は「苦労せずにもうける」の意味。	
\\	笑いすぎてお腹が痛くなっちゃったよ。 
\\	で「お腹が痛くなるまで笑う」の意。	
\\	もうあまりのバカバカしさに、なんだか笑っちゃうよね。 
\\	は「笑わざるをえない」の意味ですね。
\\	「馬鹿馬鹿しい」。
\\	の前に不定詞の
\\	が省略されています。	
\\	もう泣かないで、さぁ笑って。 
\\	「笑って」は 
\\	でいいでしょう。	
\\	私が今笑っていられるのは、あなたのおかげです。 
\\	この場合の「笑っていられる」とは
\\	「気楽にいられる」ということですね。	
\\	お前の責任なんだから、笑ってごまかすなよ。 
\\	は「笑って済ます、片づける」の意味です。	
\\	仕事も見つからないから今はいい勉強できる機会だと割り切って勉強に専念することにした。 
\\	について現実的になると決める)で「割り切る」を表現しています。
\\	は「実際的に、実用的に」の意。	
\\	彼女はあーゆー人だと割り切らないと、つきあっていられない。 
\\	「割り切る」は
\\	だと心に留めておく」で表現すればいいでしょう。	
\\	仕事は仕事って割り切ってやっているから、今の仕事続いているんだと思う。 
\\	「割り切る」をここでは
\\	「そういう姿勢でいる」と言い換えています。
\\	に感傷的な思いこみを持たない」。	
\\	列に割り込むのはよしてください。 
\\	「列に割り込む」は
\\	と言います。	
\\	小さいボトルって大きいのに比べると割高だよね。 
\\	の文字通りの意味は「悪い取引」。
\\	は「値段の割に得にならない買い物」の意。	
\\	それ、3980円のわりにはいいじゃない。 
\\	のわりには」は前置詞の
\\	を使って表現します。	
\\	値段の割りにはいいホテルだった。 
\\	「~の割には」と言う時には、前置詞の
\\	を使います。
\\	「(料金を)請求する」。	
\\	昔はそんな悪だったのね。知らなかった。今明かされる真実って感じだね。 
\\	「悪」は 
\\	、あるいは、
\\	で表現すればいいでしょう。
\\	は「意外な新事実」の意。	
\\	この期に及んで、悪あがきはよせよ。 
\\	「悪あがき」とは
\\	(意味のない努力)のことですね。
\\	は「無益な」の意。	
\\	お願いしようかと思ったんだけど、忙しそうだし悪いかなと思って。 
\\	「悪いと思う」は
\\	がそのまま使えます。	
\\	皆の日程調整悪いけどよろしく頼む 
\\	と
\\	で「悪いけどよろしく頼む」のニュアンスが表現できます。	
\\	その事についてインターネットで調べたんだけど、殆ど悪いことしか書いてなかった。 
\\	「悪いこと」とは
\\	「否定的な」ということですね。(注)一部投稿文をカットしました。	
\\	まだ出発まで時間あったから、悪いと思って起こさなかったんだけど。 
\\	ここでは「悪いと思って」を
\\	すべきじゃないと思って)と言い換えて表現しています。
\\	は
\\	するのはよくないと思った」の意。	
\\	忙しいのにメールの返事お願いして悪いね。 
\\	「悪いね」は
\\	で表現できます。
\\	共に「要求する」の意。	
\\	彼も、悪気があったわけじゃないと思う。 
\\	で「危害を与える積もりはない」の意味。	
\\	彼女の性格って、良く言えば物静かだけれど、悪く言えばネクラよね。 
\\	と 
\\	の対比がポイントです。
\\	は「(ひねった)解釈」、
\\	は「ダサイ奴」の意。	
\\	うーん、悪く考えても80点は取れたと思うな(低めに見積もっても)。 
\\	は「最悪でも」の意味。
\\	「80点以下を取ったということはないだろう」。	
\\	私一人を悪者にして自分はいい顔しようとしてるんじゃん。 
\\	「悪者」は 
\\	でいいでしょう。
\\	は「間違っている、落ち度がある」の意。	
\\	そんなに飲んだら悪酔いするぞ。 
\\	で「飲んで気分が悪くなる」の意味。
\\	「もどす」。	
\\	あの人は何かにつけて我知り顔で説明する。 
\\	「我知り顔」とは
\\	「何でも知っているように振る舞う」と言うことですね。
\\	「一流の権威」。	
\\	われながら、驚いたね。自分でも、まさか宝くじにあたるなんてね。 
\\	は「僕がどれだけ驚いたか想像してみて」の意味。	
\\	あなたは、どうして、そういつも、ワンテンポずれるわけ? 
\\	「ワンテンポずれる」をここでは
\\	「リズムが必要だ」で言い換えています。
\\	「ちょっと遅れる、ずれる」の意。	
\\	(私は)ワンランク上の暮らしがしたい。 
\\	「段、級」を使って
\\	で「ワンランク上の」の意味になります。
\\	は「一段上の」の意。ここでは「上の暮らし」を「今住んでいるところよりも1つ上の場所で暮らす」と英訳しています。	
\end{CJK}
\end{document}