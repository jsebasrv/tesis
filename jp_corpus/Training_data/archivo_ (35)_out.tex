\documentclass[8pt]{extreport} 
\usepackage{hyperref}
\usepackage{CJKutf8}
\begin{document}
\begin{CJK}{UTF8}{min}
\\	ワンカラ	わんから	一人でするカラオケのこと。	ワンカラとはワン
\\	とカラオケから成る言葉で、一人でカラオケをすること。一人でカラオケに行くことを意味する。もともとは二人以上の複数による娯楽として普及したカラオケだが、2000年代半ば辺りから一人で部屋を借りてカラオケをする人が急増。一人でカラオケをする人が2割を超えるようになり、業界側も一人カラオケ用の専門店を作り、対応している。ワンカラは、そんな中で生まれた言葉である。(一人カラオケの略で、ヒトカラともいう。下記関連語参照)	2012年	合成語(和製英語)	
\\	アゲぽよ	あげぽよ	テンションがあがっている状態のこと。	あげぽよとはテンションがあがっている状態をあらわす『アゲアゲ』に、かわいらしさを強調する接尾語『ぽよ』を付けたもの。『ぽよ』自体は特に意味を持たないため、単純に『アゲアゲ』のかわいらしい言い回しと考えてよい。 なお、アゲぽよは2010年にギャル的流行語大賞で1位をとった他、2011年には女子中高生ケータイ流行語大賞銀賞、ユーキャン新語流行語大賞の候補にもなった(受賞はしていない)。	2011年	若者言葉	
\\	アフター4	あふたあふぉお	終業後の時間のこと。	アフター4とは終業後の時間を意味する『アフター5』の派生語で、サマータイム導入後の仕事明けの時間を意味する。もともとは官庁の退庁時間が5時(17時)であったことから、民間企業もこれにならい、5時終業の企業が多かった。そんな終業後(5時以降)の時間の使い方に対する意識が高まるにつれ、終業後の時間をアフターファイブと呼ぶようになった。 しかし、2011年。原発事故による夏場の節電対策のため、始業・終業をそれぞれ1時間早めたサマータイムを導入する企業が増え、16時終業になってきたことから、アフター4と呼ばれるようになった。 アフター4に主に行うこととして、ジム通い、家族サービス、会食などが挙げられている。また、アフター4のサラリーマンをターゲットにした外食産業、サービス業も多い。 なお、アフター4は2011ユーキャン新語・流行語大賞にノミネートされた(受賞はしていない)。	2011年	
\\	エダる	えだる	
\\	不眠不休で働くこと。 
\\	上司に恵まれず、必要以上の仕事を強いられること。	エダるとは民主党所属の衆議院議員であり、内閣官房長官でもある枝野幸男の『枝』に動詞化する接尾語『る』をつけたもので、寝る時間を惜しんで働く、不眠不休で働くという意味で使われる。これは、2011年3月11日に起きた東北地方太平洋沖地震以降の枝野官房長官の会見に望む姿から
\\	を中心に浸透したもので、。特に震災数日後は
\\	(枝野寝ろ)」というハッシュタグができ、「少し、休んでください」というエールも多数あった。後に、管内閣総理大臣の震災への対応に疑問視や批判の声があがり、枝野官房長官がそれをフォロするイメージがつきはじめると、上司に恵まれず、必要以上の仕事を強いられることという意味でもエダるは使われるようになった。 なお、エダるは2011年ユーキャン新語・流行語大賞で候補語になったが受賞はしていない。	2011年	ネット用語	
\\	推しメン	おしめん	アイドルグループの中で自分が推しているメンバー(好みのメンバー、応援しているメンバー)のこと。	推しメンとは『イチ推しのメンバー』を略したもので、アイドルグループの中のイチ推しのメンバー(特に気に入っているメンバー、応援しているメンバー)を意味する。複数の推しメンがいる場合、一番好きなメンバーを神推し、二番目が激推し、三番目を推しという。また、推しメンを変えることを意味する推し変といった派生語も存在する。推しメンは
\\	48を中心に、主に女性アイドルグループに対して使われる。ジャニーズ(ジャニーズ用語)の場合、自担がこれにあたる。 なお、推しメンは2011年ユーキャン新語・流行語大賞で候補語になったが受賞はしていない。	2011年	若者言葉	
\\	ガラケー	がらけー	日本市場を対象に、独特の進化を遂げた日本製の携帯電話。	ガラケーとはガラパゴス・ケータイ(携帯)の略で、日本市場を対象にして独自の進化を遂げた日本製の携帯電話のことである。 ガラパコスとは外の世界と断絶された環境を指して使われることが多い。これは外部から断絶された場所にあるため、生物が独自の進化を遂げたガラパゴス諸島からきている。 日本の携帯電話が世界市場とは違う独特な進化を遂げたことから、そういった日本製の携帯電話をガラケーと呼ぶようになった。ガラケーの主な特徴としてはオサイフケータイ、ワンセグ、赤外線通信などがある。 日本は通信方式が世界標準と異なっていたため、海外メーカーの携帯電話が入ってこなかったことからガラケーのような独自の進化を遂げる携帯電話が生まれたとされる。	2011年	若者言葉 ・ 略語	
\\	絆婚	きずなこん	絆を大切にした結婚のこと。絆を求めてする結婚のこと。	絆婚とは2011年の東日本大震災をキッカケに出来た言葉である。東日本大震災により、それまで付き合いのあったカップルや友人関係にあった男女の中には関係(絆)を再認識し、籍を入れるものも多数あった。そういった経緯での結婚を東日本大震災復興のキーワードともなった絆にちなんで絆婚と呼ぶ。また、震災の影響から家族という絆を求める人の婚活が積極的となったり、収入などの条件にとらわれず、本当に一緒にいたい人との絆婚を望む人(独身女性に多い)が増えたとされる。 また、既に結婚しているカップルの間でも、震災を受け、改めて絆を感じたという夫婦が多い中、逆に震災をキッカケに離婚する震災離婚も話題になった。	2011年	
\\	震災離婚	しんさいりこん	震災が原因の離婚のこと。	震災離婚とは1995年1月17日にあった阪神淡路大震災や2011年3月11日の東日本大震災など、震災が原因・キッカケとなった離婚のこと(震災離婚という言葉が出来たのは1995年の阪神淡路大震災時)。 地震と離婚がどう結びつくのか?これは震災が起きたときに妻を置いて逃げた(逃げようとした)夫が原因で不仲となり、離婚にいたったというもので、こういった夫婦のことを当時のメディアが震災離婚という言葉で報道した。	2011年	
\\	スマホ	すまほ	スマートフォンのこと。	スマホとはスマートフォンの略で、パソコン機能をベースとして作られた多機能携帯電話のことをいう。スマフォと略されることも。スマホの具体的な定義として、広義には携帯電話にパソコンや
\\	の機能が備わったもののこと。狭義にはアプリケーションを追加し、カスタマイズが出来る携帯電話のことをいうが、後者を指して使われることが多い。 2008年に日本に上陸し、多くのユーザーを有する
\\	を指してスマホと呼んだり、
\\	に対し、
\\	など、それ以外のものを指してスマホと呼ばれることもあるが、どちらも誤用である。 ちなみに、スマートホンのスマートは
\\	(賢い)からきている。なお、スマホは2011年ユーキャン新語・流行語大賞にてトップテンを受賞している。	2011年	略語	
\\	セレブニート	せれぶにーと	不労所得などでお金があり、働く必要がないためにニートな生活を送っている者。	セレブニートとはお金持ちで品のある高級感漂う女性を意味する『セレブ』と、職に就いておらず、学生でもなく、求職活動もしていない人を意味する『ニート』の合成語。ニートが自身の怠慢からそのような状態にあるのに対し、セレブニートは不労所得など働かなくても収入が充分あるから働かない人を意味する。 大抵は上記のような者をセレブニートと呼ぶが、稀にセレブと結婚したことでニートのような生活が出来るようになった人やセレブと結婚したニートという意味で使われることもある。	2011年	和製英語、合成語	
\\	どや顔	どやがお	得意気な顔のこと。	どや顔の『どや』とは関西方面で「どうだ?」という意味で使われる方言。語尾をあげて「どや?」と聞くこの言葉、何かが完成したり、うまくいったときなど、「どうだ?」の後に「すごいだろう?」という意を含んで用いることが多い。つまり、どや顔とは「どや?」と言わんばかりのしたり顔、自慢気な顔のことをいう。 なお、どや顔は2011年流行語大賞のノミネート60語に含まれている。	2011年	
\\	モバ美 ・ モバビュ	もばびゅう	携帯できる美容家電のこと。	モバ美とはモバイル(携帯)とビューティーから成る造語で、持ち歩きが出来る美容家電のこと。ビューティーとの合成語であるため、モバ美と書いても“もばび”ではなく、“モバビュー”と読む。 ミスト美顔器やヘアアイロン、まつ毛カーラーの他、電動歯ブラシ、携帯できる空気清浄器、イオン発生器などもモバ美に含まれる。小型&軽量で使いやすいだけでなく、女性の好まれるようなおしゃれなデザインのものが多く出回っている。	2011年	美容 ・ ファッション関連	
\\	イクメン	いくめん	育児を楽しむ男性。育児を積極的に行う男性のこと。	イクメンとはイケメン(下記関連語参照)が変化したもので、育児を積極的に率先して行う男性、育児を楽しんで行う男性を意味する。産休による出産後、女性が引き続き育児を行うのが一般的であった。これに対し、男性が育児休業基本給付金といった制度を利用し、育児休暇をとって積極的に育児を行う男性が増えた。こういった男性を賛美する言葉として出来たのがイクメンである。ただし、休暇をとって育児をしたい男性は多いものの、収入(給与)が下がる、会社の評価が低くなるといった理由から、まだまだ日本における事実上のイクメンの数は少ないのが現状である。(2009年現在) また、イクメンはユーキャン新語・流行語大賞にて、2010年のトップテンとなっている。受賞者はタレントで4児の父として産休を取ったつるの剛士。	2010年	
\\	痛絵馬	いたえま	キャラクターが描かれているなど、内容が痛い絵馬のこと。	痛絵馬とは「痛い絵馬」の略(ここでいう痛いとは肉体的な痛みのことでなく、「恥ずかしい」「情けない」といった意味になる。下記関連語参照)。具体的にはキャラクターが描かれていたり、キャラクターの口癖や語尾を願い事に絡めたものを指して痛絵馬という。他に、2ちゃん用語やニコニコ動画の書き込みに見られるようなフレーズ、語尾を願い事に絡めて書いたものなどがある。	2010年	
\\	イラドル	いらどる	人をイライラさせるアイドルのこと。	イラドルとは苛立ちをあらわす『イラッ(イライラ?)』と『アイドル』から成る合成語で、言動(主に発言)がイラッとさせるアイドルのことをいう。
\\	ドルという形で、アイドルが細分化される中で出来た言葉である。イラドルの代表といえば、自身の1
\\	のタイトルが「イラドル 山下若菜」である山下若菜。インタビュアーの質問に対し、的外れな応対をすることで、周りをイライラさせるキャラクターが定着。それが注目されている異色のアイドルである。	2010年	
\\	おはあり	おはあり	「おはよう」という挨拶に対するお礼の言葉。	おはありとは「おはよう」と「ありがとう」から成る合成語で、主に
\\	で使われる用語である。
\\	の
\\	(=タイムライン)上でフォロワーから「おはようございます」などの挨拶をもらったときに「おはようの挨拶をありがとう」といったニュアンスで用いる。また、自身が「おはよう」の挨拶後、即
\\	からログアウトし、フォロワーさんからの「おはよう」の返信に後から気付いた際にも「今朝はおはようの挨拶をありがとう」といったニュアンスでおはありという。	2010年	ネットスラング ・ 合成語	
\\	おやあり	おやあり	「おやすみ」という挨拶に対するお礼の言葉。	おやありとは「おやすみなさい」と「ありがとう」から成る合成語で、主に
\\	で使われる用語である。
\\	の
\\	(=タイムライン)上で眠りに就くことを告げ、フォロワーから「おやすみなさい」などの挨拶をもらったときに「おやすみの挨拶をありがとう」といったニュアンスで用いる。また、眠りに就くことを告げた後、即
\\	からログアウトし、フォロワーからの「おやすみ」の挨拶に気付いたのが翌朝だった場合にも、「昨夜はおやすみの挨拶をありがとう」といったニュアンスでおやありという。	2010年	ネットスラング ・ 合成語	
\\	金活	きんかつ	金運アップするための活動のこと。	金活とは金運アップするための活動のこと。金運アップといえば、金運を司る西方位を整頓したり、タイガーアイなどのパワーストーンを持ったり、黄色や金色の財布を持つといった風水に絡んだものが主にあげられる。もちろんこれらも含まれるが、金活という言葉を使い始めたのが「満月の日にお財布をフリ、満月の気を取り込むことで金運アップ」という活動をしているフリフリーゼ(満月お財布フリフリ会:下記関連サイト参照)によるものであるため、満月の日に財布を振ることやフリフリーゼの活動に参加・賛同することを指して使うことが多い。また、フリフリーゼという通り、女性がメインであることから金活女子という言葉も使われる。	2010年	
\\	ぐだぐだ	無気力な状態。同じこと ・ つまらないことを長々と話すさま。	
\\	とは「ぐだぐだ」「ぐだぐだする」といった意味で、メールを含むインターネット上で使われる言葉である。具体的には、つまらないことを長々と話すこと。無気力でヤル気のなさそうな様。気分が緩んで締りのない様をあらわす。このようなローマ字の頭文字
\\	の場合ならグダグダ=
\\	を使った略語は、通常、アルファベット読み(この場合なら「じー ・ でぃー ・ じー ・ でぃー)をするが、
\\	に関しては、そのまま「ぐだぐだ」と読む。	2010年	若者言葉、アルファベット略語	
\\	雑食系	ざっしょくけい	
\\	肉食系でも草食系でもない人または両方を兼ね備えてる人のこと。 
\\	多趣味な人のこと。	①雑食系とは肉食系男子(恋愛に対し、積極的に行動をおこすタイプ)や草食系男子(協調性が強く、家庭的だが恋愛に不向きなタイプ)のどちらにも属さないタイプ、両方を兼ね備えたタイプ、中間的位置にいるタイプの男性を意味する。 ②雑食系とは多趣味な人や好みが散漫な人のこと。またここから特に拘りのない人、さらにマイペースな人といった意味で使われることもある。こちらの意味では2010年以前から使われている。	2010年	
\\	3同男	さんどうおとこ	女性が結婚相手として男性に求める3つの条件のこと。	3同男とは『3高男(三高)』、『3低男(三低)』に続く、女性が結婚相手(パートナー)として男性に求める三つの条件のこと。具体的には、自分の仕事や働き方を理解してくれる『仕事観』、どちらかが職を失っても支えあっていける『金銭感覚』、常識が共有出来るよう『育った環境』。以上三つが同じ(または似ている)男性のことを3同男という。 なお、3同男は日経
\\	の7月号で使われた造語である。	2010年	
\\	自宅難民	じたくなんみん	仕事から家に帰っても、時間や居場所がない父親のこと。	自宅難民とは仕事から帰り、自宅でくつろごうと思っても、落ち着ける時間が持てなかったり、居場所がないような父親を指す言葉である。調査会社アイカフェが40~50代で有職~年収300万円以上、既婚、子ありという条件で関東エリアに住む男性450人に調査。2010年5月に発表された結果によると、「ひとりになれる時間・場所がない」「見たいテレビ番組が見られない」「妻や子どもがうるさい」などの不満を持つ人は51.2
\\	。中には帰宅しても夕飯がないという不満を漏らす人も。そんな自宅難民と呼ばれるお父さんが居場所としているのがファミリーレストランや自分の嗜好にあったカフェで、同じ趣味仲間が出来たりすることもあり、案外自宅よりよいという意見もある。	2010年	
\\	終活	しゅうかつ	人生の終焉をより良く迎えるための前準備のこと。	終活とは(自分の)人生の終わりのための活動の略で、就活や婚活の派生語である。終活は人生の最期をより良いもの(=自分の理想的なもの)とするため、事前に行う準備のことで、例えば葬儀の内容やお墓のことを事前に決めておいたり、財産配分が主目的である遺言とは別に、自分の思いや意思、願いを綴るノートを書いておくといったものがある。遺言状の場合、葬儀後に開封されることが多く、葬儀に対する願いを書いても希望が叶えられずに終わってしまう。そこで事前に家族や業者に相談するといった終活をする人が増えている。また、終活をすることで、死や人生を見つめなおすといった人も多い。 なお、終活はユーキャン新語 ・ 流行語大賞の2010年候補語60語にノミネートされた。	2010年	
\\	女子会	じょしかい	女性だけで集まる食事会 ・ お茶会 ・ 飲み会のこと。	女子会とは女性だけで集まって飲食をしたり、話し(情報交換)をする会のこと。仕事を持つ女性の過半数が月一回以上女子会に参加しているという調査結果もある。女子会の意義や目的で多いのが本音トークや秘密トークが出来る点。他に女性同士だから共感してもらえる点やストレス発散が出来るという理由も多い。男性の集まりだと飲み会というイメージが強いが、女子会では食事やお茶、スイーツなど、お酒のない飲食の方が多い。また、ショッピングや映画鑑賞、旅行など多岐にわたっている。また、こういった動向に着目し、女性限定プランを打ち出す飲食店もある。 2008年辺りから使われるようになった女子会だが、2010年にはユーキャン新語・流行語大賞のトップテンを受賞している。受賞者はモンテローザ社長の大神輝博。	2010年	
\\	スマホ	すまほ	スマートフォンのこと。	スマホとはスマートフォンの略で、パソコン機能をベースとして作られた多機能携帯電話のことをいう。スマフォと略されることも。スマホの具体的な定義として、広義には携帯電話にパソコンや
\\	の機能が備わったもののこと。狭義にはアプリケーションを追加し、カスタマイズが出来る携帯電話のことをいうが、後者を指して使われることが多い。 2008年に日本に上陸し、多くのユーザーを有する
\\	を指してスマホと呼んだり、
\\	に対し、
\\	など、それ以外のものを指してスマホと呼ばれることもあるが、どちらも誤用である。 ちなみに、スマートホンのスマートは
\\	(賢い)からきている。なお、スマホは2011年ユーキャン新語・流行語大賞にてトップテンを受賞している。	2010年	略語	
\\	装飾男子	そうしょくだんし	華美に着飾り、様々な装飾をファッションに取り入れる男子のこと。	装飾男子とはメンズモード誌
\\	(ヴォーグオムジャパン)』による造語で、2008~2009年にかけて流行った草食男子の派生語である。文字通り、アクセサリーやピン、コサージュといった装飾をファッションに貪欲に取り入れ、華美に着飾る男子を意味するが、特に定義がされているわけではない。また、装飾男子の中には化粧をし、レディースの服を着たり、メンズスカートを履いたりといった女性的アプローチが色濃いものもあるが、装飾男子はあくまでそのファッション性に対する言葉であり、精神性 ・ 性的嗜好については関係ない。	2010年	美容 ・ ファッション用語	
\\	ツイドル	ついどる	
\\	上のアイドルのこと。	ツイドルとは
\\	(ツイッター)とアイドルから成る造語で、
\\	上で人気があり、フォロワー数の多い人やキャラクターのこと。ツイドルは
\\	によって人気が集まった人やキャラクターのことをいい、ネットアイドルやグラビアアイドルなど、もともと
\\	以外で活動し、人気のあったアイドルが
\\	を始め、多くの人にフォローされたとしても、ツイドルには該当しない。 ツイドルのハッシュタグ:
\\	2010年	ネットスラング	
\\	~なう。	~なう。	現在の所在地や状態、行動を表現する言葉。=~中(ちゅう)	"~なう。とはインターネット上のコミュニケーション ・ サービスサイト
\\	で使われるようになった言葉で、自分が現在どこにいるか(所在地 ・ 下記使用例1)、どうなっているか(状態 ・ 下記使用例2)、何をしているか(行動 ・ 下記使用例3)などを書き込む際、文末につけて使う言葉。営業しているお店に掛かっている『営業中』という札の『中(ちゅう)』にあたるもので、英語で「今」を意味する
\\	""からきている。 また、~なう。はユーキャン新語 ・ 流行語大賞にて、2010年のトップテンとなっている。受賞者はスーパー高校生の梅崎健理。 の使用例: 今、北海道にいます→北海道なう 
\\	(こちらは)快晴です→快晴なう 
\\	今、ジョギングをしています→ジョギングなう 
\\	2010年	ネットスラング	
\\	ふぁぼる	ふぁぼる	
\\	のつぶやきをお気に入りに登録すること。	ふぁぼるとはインターネット上で
\\	(ツィート ・ 日本では「つぶやき」とも呼ばれる)と呼ばれる短文の投稿 ・ 閲覧によるコミュニケーション ・サービス
\\	(ツイッター)』内で使われる言葉である。
\\	にはお気に入りのツィートを登録するお気に入りボタンというものがある。このお気に入りを意味する英語
\\	に日本語で動詞化する接尾語『る』を付けたものがふぁぼるである。
\\	ると表記されることもある。 また、他者からツィートをお気に入り登録されることをふぁぼられるという。	2010年	ネット用語	
\\	ブラリ	ぶらり	
\\	(ブラック ・ リスト)のこと。	ブラリとは「タオルをぶらりと下げる」といったように垂れ下がっているさま。「ぶらりと立ち寄る」といったように当てのないさまをあらわす。「ぶらりと過ごす」といったように何もしないさま。通常はこれらの意味で使われるブラリだが、ネット上で用いられる場合、これとは別に
\\	(ブラック ・ リスト)の略として使われることがある。こうした意味で使われるのは、主にモバゲーとネットオークションサイト。モバゲーでは絡みたくない人を指し、オークションサイトでは過去にトラブルがあった人を指して用いる。
\\	ともいう。	2010年	略語、
\\	用語	
\\	ママ鉄	ままてつ	鉄道好きなお母親のこと。	ママ鉄とは母親を意味する『ママ』と『鉄道』を合成したもので、鉄道好きなお母さん、鉄道オタクなお母さんを意味する。ただし、ママ鉄は子供に鉄道の絵本やオモチャを買ってあげたり、電車を見に行くなどする中で鉄道マニアになっていった母親を意味し、子供が生まれる前や独身の頃から鉄道マニアであった人は含まない。ちなみに、後者のもともと鉄道マニアだった女性は鉄子(下記関連語参照)に含まれる。	2010年	
\\	山ガール	やまがーる	アウトドアブランド、グッズに身を包み、登山を楽しむ女性のこと。	山ガールとは山(=登山)を楽しむ女性(ガールという通り、大抵は20~30代の女性を指す)のこと。2000年代後半のアウトドアブーム(特に登山ブーム)にのり、アウトドアブランドやアウトドアグッズに身を包み、登山を行う女性のこと。山ガールの主なスタイルは防水加工の上着に、コーディネイトが可能で機能性(保湿性、着替えやすさ)があり、ファション性もある山スカート(通称:山スカ)。パワースポットブームも手伝ってか、富士山を登る山ガールが多い。 また、アウトドアアイテムや小物を部分的に取り入れ、野外フェスティバルに行ったり、散歩といった身近なアウトドアを楽しむ雰囲気山ガールと呼ばれる者もある。 なお、山ガールはユーキャン新語 ・ 流行語大賞の2010年候補語60語にノミネートされた。	2010年	
\\	アシュラー	あしゅらー	阿修羅像が好きな女性のこと。	"アシュラーとは阿修羅(あしゅら)に英語で人化する接尾辞
\\	""を付けたもので、阿修羅に関心のある人、阿修羅像が好きな人を意味する。ただし、アシュラーという言葉が誕生した経緯として、興福寺の阿修羅像が2009年3月~6月に上野の東京国立博物館で公開された際、中高年層だけでなく、若い女性の来館が多かったことから、こうした女性を指して付けたことがあり、単に阿修羅が好きな人というより、女性ファンといったニュアンスで用いられる。 アシュラーは2009年のユーキャン新語 ・ 流行語大賞でノミネートされた言葉でもある。ただし、入賞はしていない。
\\	2009年	
\\	アラ還	あらかん	60歳(60代)前後の人のこと。	アラ還とは「アラウンド還暦
\\	還暦)」の略で、還暦前後の世代を意味する。2006年に30代前後を意味するアラサーという言葉が使われ、2008年には派生語であるアラフォーが流行語大賞を受賞。アラ還はその流れの中で誕生した言葉である。2009年は団塊の最後の世代が還暦を迎える年ということも、アラ還が注目を集める一要因である。また、2008年に61歳で紅白歌合戦に初出場した秋元順子さんがインタビューで「アラ還の星を目指し、声が続く限り唄っていきます」とコメントしたことでも話題になった。 ※還暦とは:数え歳で61歳。還暦と呼ぶ由来などは下記参考サイトを参照	2009年	
\\	イクメン	いくめん	育児を楽しむ男性。育児を積極的に行う男性のこと。	イクメンとはイケメン(下記関連語参照)が変化したもので、育児を積極的に率先して行う男性、育児を楽しんで行う男性を意味する。産休による出産後、女性が引き続き育児を行うのが一般的であった。これに対し、男性が育児休業基本給付金といった制度を利用し、育児休暇をとって積極的に育児を行う男性が増えた。こういった男性を賛美する言葉として出来たのがイクメンである。ただし、休暇をとって育児をしたい男性は多いものの、収入(給与)が下がる、会社の評価が低くなるといった理由から、まだまだ日本における事実上のイクメンの数は少ないのが現状である。(2009年現在) また、イクメンはユーキャン新語 ・ 流行語大賞にて、2010年のトップテンとなっている。受賞者はタレントで4児の父として産休を取ったつるの剛士。	2009年	
\\	ウェブ飲み会	うぇぶのみかい	ウェブ上のサイトなどを介して行う飲み会のこと。	ウェブ飲み会とはウェブ上のホームページやメッセンジャーなどのチャット機能を使い、ネット上で会話をしながらお酒を飲むバーチャルな飲み会のことで、ネット飲み会、オンライン飲み会ともいう。実質的には家で一人お酒を飲むことになるが、外で飲むのと違い、お金がかからないといった理由やその時々でさまざまな話題が出来るといった利点から、若者を中心にウェブ飲み会が浸透している。また、酒類メーカーであるサントリーが運営する『ほろよい.
\\	といったサイトもある(下記関連サイト参照)。このサイトでは自身のアバター(ネット上で自分の分身として画面に登場するキャラクター)を作り、「血液型で語ろう」「地元を語ろう」「趣味で語ろう」など、話題別の部屋(建物)に入るといった形で飲み会に参加する。サントリー『ほろよい』の
\\	でウェブ飲み会のシーンがあるが、そこに登場する水嶋ヒロや堀北真希らがサイト上にゲストとして参加したことでも話題になった。	2009年	ネットスラング	
\\	えい ・ てぃー ・ えむ	アルファベット略語
\\	語)として複数の意味を持つ言葉である。	
\\	は通常
\\	の略で銀行や郵便局の現金自動預け払い機を意味するが、これとは別に若者の間では
\\	語(日本語をローマ字表記し、その頭文字のみによる言葉)として下記のように複数の意味で使われる。 具体的に
\\	語としての
\\	とは「アイツたまにムカツク
\\	、「アホな父ちゃん、もういらへん
\\	、「あなたの便りを待っている
\\	といった意味で使われる。	2009年	
\\	語 ・ アルファベット略語	
\\	おうちごはん		自宅でとる食事のこと。	おうちごはんとは家 ・ 自宅を意味する「おうち」と「ごはん(ご飯)」から成る合成語で、文字通り、自宅で食事をすること。また、自宅でとる食事を意味し、外食の対になるものと考えてよい。不況の影響から、外食を抑え、自宅で食事をとる人が増えたこと。健康やダイエットという点から、自炊をする人が増えるなど、おうちごはんが見直されている。また、それぞれの家庭の献立やレシピを紹介するおうちごはんブログを書く人も増えている。	2009年	合成語	
\\	オシオ男	おしおおとこ	ナルシズムが強く、過剰な自信を持っている男性のこと。	オシオ男とは「押尾学(俳優)のような男」の略で、ナルシズムが強く、過剰な自信を持っている男性を意味する。オシオ男は2009年8月、押尾学が麻薬及び向精神薬取締法違反の疑いで逮捕され、更に矢田亜希子との離婚、女性問題が明るみになるといったことを受け、カップルコンサルタント:西郷理恵氏が作った造語である。西郷氏はオシオ男の特徴として、いかの8項目を挙げている。1.ルールが守れない。2.自分大好きナルシスト。3.夢がでかすぎる。4.常に上から目線。5.すぐに感情的になる。6.彼女に自分のセンスを押しつける。7.女性に対して積極的(肉食男子)。8.いざ問題が起きると逃げる。※各項目の詳細については下記参考サイトを参照。	2009年	
\\	オシャンティー	おしゃんてぃー	オシャレなこと。	オシャンティーとはおオシャレをスタイリッシュにした言葉。意味自体はお洒落と同意である。ゆとり語(註参照)のひとつとされている。 註)ゆとり語とは、略語やラ抜き言葉。「おまぇ」「ぅち」といった小文字を多用したものや言葉を耳で聞いたまま表記するものなど、ゆとり世代(ゆとり教育世代)の若者が使う独特な言い回しや表記のこと。	2009年	若者言葉	
\\	介護離婚	かいごりこん	親の介護を理由に離婚すること。	介護離婚とは双方の親の介護が主な原因となる離婚を意味する。例えば、妻が親の介護のために実家へ帰り、そのまま別居状態となって離婚する場合がこれにあたる。「夫が家庭を顧みなかったから」「夫から愛情を感じないから」といった熟年離婚のような要因が根底にある場合でも、離婚の直接原因や引き金が介護であった場合、介護離婚ともいえる。また、近い将来想定される介護が原因の離婚を意味する介護前離婚という言葉もある。	2009年	
\\	ガチムチ	がちむち	
\\	筋肉の上に脂肪がついている体型のこと。 
\\	とても太っている人のこと。	ガチムチとは筋肉質な体型をあらわす『ガッチリ~ガチガチ』と太った人をあらわし『むっちり~ムチムチ』から成る言葉で、筋肉質が付いてガッチリしているが、大柄でムッチリした体型。筋肉の上に脂肪がついた感じの体型を指す。 また、これとは別に「真剣に」「真面目に」といった意味の『ガチ』と『むっちり』の合成語として、真剣にむっちりした体型~とても太っている人といった意味で使われることがある。	2009年	若者言葉	
\\	北島康介	きたじまこうすけ	強い睡魔のこと。(俗語的解釈)	"北島康介とはアテネオリンピック、北京オリンピックの2大会で、平泳ぎ100
\\	、平泳ぎ200
\\	の連続2冠王に輝いた日本男子競泳選手である。そして、泳ぎ手 ・ 泳ぐ人のことを英語で
\\	(スイマー)""ということから、北島康介選手のことを強い睡魔(スイマーにかけたもの)という意味で使用。例えば「北島康介に負けた」と言った場合、北島選手と戦って負けたという意味でなく、睡魔に負けたという意味になる。
\\	2009年	
\\	ケーマー	けーまー	懸賞マニアのこと。または健康マニアのこと。	"ケーマーとはさまざまな懸賞へ応募することを楽しみとしている懸賞マニアのこと。ゲームマニアのことを
\\	""に
\\	(「~する人」という意味の英接尾辞)""をつけ、ゲーマーと呼ぶが、ケーマーがその派生語なのか、単に懸賞マニアの頭文字(ケとマ)を伸ばして出来た言葉なのかは不明。 懸賞マニア以外に、健康マニアという意味で使われることもある。
\\	2009年	
\\	恋活	こいかつ	恋愛するための活動のこと。	恋活とは恋愛をするための出会いを求めた活動を意味する(「こいかつ」の他、恋愛活動の略ということで「れんかつ」ともいう)。恋活は純粋に彼氏 ・ 彼女を探したり、恋愛に発展しそうな異性を探したり、休日一緒に過ごせる異性を探すといったもので、恋愛の先に結婚があるという意識で探す場合は婚活扱いとなる(婚活については下記関連語参照)。例えば、男性ばかり ・ 女性ばかりの職場で働くような出会いの少ない人が、仲間を通じたり、インターネットを使うといった形で、職場以外に人間関係を築くといった活動を指す。	2009年	
\\	焦げキュン	こげきゅん	好きな異性を思って胸が熱くなるさま。	焦げキュンとは感動や恋心で胸がいっぱいになることを意味する『胸キュン(下記関連語参照)』の派生語で、恋い焦がれ、焦げてしまうくらいに心が熱く燃え上がるさまを意味する。順番としては、焦げキュンの方が胸キュンよりも強く胸が締め付けられた状態として用いられる。	2009年	
\\	渋原系	しぶはらけい	渋谷と原宿のスタイルを取り入れた女性ファッションのこと。	渋原系とは渋谷のギャルと原宿のストリートカジュアルのスタイルを取り入れたレディース ・ ファッションのこと。
\\	シャツにジーンズといったカジュアルなファッションにアイライン、マスカラ、つけまつげで目元を強調したギャルメイク、金髪の盛り髪といったスタイル。渋原系ファッションともいう。	2009年	
\\	しまラー	しまらー	しまむらの服でコーディネートしている人のこと。	しまラーとは衣料品チェーンストアの「ファッションセンターしまむら(株式会社しまむら)」を略したものに、英語で人化する接尾語
\\	をつけたものである(この語が使われる前に、同じく衣料品チェーンの名を人化したユニクラーという言葉が流行しているため、その派生語と考えてよい)。“しまむら”で購入した衣類で全身をコーディネートしていること。またはそういった人を意味する。また、ユニクラー同様、安い服でコーディネートしているという意を含んで使われることが多い。“しまむら”が都心にはほとんどないため、しまラーも地方を中心に使われていたが、2009年に入り、メディアで取り上げられることが増え、広く浸透している。	2009年	美容 ・ ファッション用語、若者言葉	
\\	ジョソコ ・ 女装子	じょそこ	女装をした男性(男子)のこと。	ジョソコとは女装子と書けばわかる通り、女装をした男性・男子を意味する。女装子は狭義では日常的に女装をしている人など、普段から女性として生活している人を指すが、広義では趣味で女装しているだけの男性も含む。ただし、どちらも性的指向についての定義はなく、同性愛者、両性愛者、異性愛者のいずれかは人によって異なる。また、近年はコスプレの一種として軽い感覚でジョソコをする若者も東京を中心に増えている。	2009年	若者言葉	
\\	赤る	せきる	赤外線通信で情報交換すること。	赤るとは携帯電話の機能の一つである赤外線通信の『赤』に、動詞化する接尾語『る』を付けたもので、赤外線通信で情報交換(主にメールアドレスや電話番号の交換)をすることを意味する。この場合の赤るは赤外線通信(せきがいせんつうしん)から来ているので「せきる」と読む。「あかる」と読んだ場合、赤面する、赤点を取るといった意味で用いられる。	2009年	
\\	草食男子	そうしょくだんし	恋愛に積極的でない男性のこと。	草食男子とは恋愛やセックスに積極的ではない男性のことで、コラムニスト;深澤真紀が2006年に命名した造語である。それまで一般的な男性像として、恋や女性を手に入れるため、自ら行動を起こすというイメージがあった。この自ら行動を起こすという姿勢を肉食獣に見立て、その対として出来たのが草食男子である。婚活が話題となった2008年、女性誌で草食化する男性が取り上げられるようになり、話題となった。同年7月には森岡正博 著『草食系男子の恋愛学』という本も出版されている。草食男子は単にそれを嘲う言葉ではなく、女性を一人の人として見てくれているとして、好意的に捉えている女性も多く、単なる奥手とは異なるニュアンスで使われる。 また、2009年にはユーキャン新語・流行語大賞のトップテンに入賞している。ちなみに受賞者は草食男子という言葉を生んだ深澤真紀さんと歌手の小池徹平であった。	2009年	
\\	ダサイ族	ださいぞく	暴走族のこと。	ダサイ族とは沖縄県宜野湾署が暴走族のイメージダウンを狙い、2009年につけた暴走族の俗称である。同管内から寄せられた685点の応募の中から選ばれたもので、他に「よわむし族」「ゴキブリ族」が二次選考まで残り、最終的にダサイ族に決定。一部メディアでも話題になったが、浸透には至っていない。	2009年	
\\	チャリリーマン	ちゃりりーまん	自転車通勤をする会社員のこと。	チャリリーマンとは自転車を意味する「チャリ(チャリンコの略)」と、(役員を除く)会社員、月給取りを意味するサラリーマンの合成語で、自転車通勤をするサラリーマンを意味する。チャリリーマン(またはチャラリーマン)という呼び方は以前からあったが、経済評論家であり、ビジネス書・自己啓発本が人気の勝間和代氏が健康と集中力
\\	のため、自転車活用を推奨したこと、また、2008年の原油価格の高騰などから自転車通勤が見直され、話題となった。	2009年	合成語	
\\	チョメる	ちょめる	男女が仲良くすること。性交すること。	ちょめるとは男女が仲良くすること。いちゃいちゃすること。性交することを意味する。1980年代前半、故:山城新吾が流行らせた『チョメチョメ』という言葉がある。これh伏せ字に音を付けたものだが、後に男女の性交のこともチョメチョメというようになった。ちょめるはこの『チョメチョメ』を略したものに、動詞化する接尾語である『る』を付けたものである。	2009年	
\\	チリモン	ちりもん	チリメンジャコに混じっている魚介類などの生物のこと。	チリモンとはチリメンモンスターの略で、チリメンジャコ(しらす)の中に入っている小さな生物を意味する。チリモンの中には魚の稚魚、貝類、タコ、カニやエビ、タツノオトシゴ、ヒトデやクラゲといったものも入っているという。きしわだ自然資料館など、チリモンを探すことで、子供たちに自然に接してもらおうという新たな試みのイベントも行われている他、チリモンに関する書籍が出されたり、
\\	で紹介されるなど、注目されている。また、株式会社カネ上のように選別前のいろいろ混じったしらすを商品化しているところもある(なお、チリメンモンスターは株式会社カネ上の登録商標となっております)。	2009年	
\\	てぃい ・ けい ・ じぃ	卵かけご飯のこと。	
\\	とは「卵かけご飯
\\	の頭文字をとったアルファベト略語
\\	という言葉が元で流行したことから
\\	語ともいう)。 不景気でおうちごはん(下記関連語参照)が主流となった。中でも、
\\	(卵かけご飯)はその作りやすさと美味さから支持され、中には独特なレシピの
\\	(卵かけご飯)が登場。
\\	などで特集されることも増えるとともに、
\\	と呼ばれるようになった。	2009年	
\\	語 ・ アルファベット略語	
\\	てぃい ・ わい	ありがとうのこと。	
\\	とは
\\	の頭文字から成るもので、
\\	同様、「ありがとう」という意味で使われる。
\\	語の一種でもあるが、どちらかといえば
\\	(下記註参照)のオンラインゲーム時のやりとりに用いられ、
\\	11用語(下記註参照)として扱われることが多い。また、サンクス
\\	から
\\	から
\\	という略も使われる。 註)
\\	とはファイナルファンタジーのこと。
\\	11(ファイナルファンタジー
\\	からオンライン化され、そのやりとりから
\\	11用語が多数誕生した。	2009年	
\\	語 ・ アルファベット略語	
\\	デコクロ	でこくろ	ユニクロの服に独自の飾りを施し、オリジナルな服にして着ること。	デコクロとは装飾や飾付けを意味する
\\	(デコレーション)』と衣料ブランド ・ ショップの『ユニクロ
\\	から成る合成語で、ユニクロ商品にデコレーションを施して着ること。また、飾り付けがなされ、オリジナルなデザインとなった
\\	シャツやスカート、パンツを意味する。ユニクロ製の服は安価で手ごろなことからユーザーが多い分、街で同じユニクロ製の服を着た人に合うことも増える。それを回避するため、デコクロをする人が増えている。具体的には刺繍を施したり、ピンやビーズ、レースを付けるといったもの。デコクロ以外につけクロ(付ける+ユニクロ)、リメクロ(リメイク+ユニクロ)ともいう。	2009年	美容 ・ ファッション用語	
\\	デコ服	でこふく	バッジやブローチを付けるといった装飾を施した服。	デコ服とはデコレーションされた服のこと。手持ちの服に、バッジやブローチを付けたり、刺繍をあしらうといった飾りつけをしたものをいう。デコ服は、自分で装飾することから、他にないオリジナルな一着という点が好まれる他、新しい服を買わなくても、いろんなデザインが楽しめるという不況時代の賢いお洒落として、男女共に好まれている。	2009年	
\\	肉食女子	にくしょくじょし	恋やセックスを求め、積極的に行動する女性のこと。	肉食女子とは草食男子の対語で、恋やセックスを手中に収めるため、自ら積極的に行動する女性を意味する。男性の草食化(恋やセックスに対する積極性の低下)に伴い、そういった男性をどう攻略するかが女性誌で度々取り上げられるようになった(具体的には自ら触れていく、甘えていくなど)。また、女性ファッション誌
\\	が2009年1月号で肉食女子自体の特集を組み、更に話題となった。この中で肉食女子は「セックス大好きタイプ」「恋愛体質タイプ」「草食男子攻略タイプ」の3タイプにわけられている。	2009年	
\\	ネガキャン	ねがきゃん	ネガティブキャンペーンのこと。	ネガキャンとはネガティブキャンペーンの略で、もともとは選挙において、対立候補を貶めるような発言 ・ 広告宣伝をし、相対的に自候補のイメージアップを謀るという選挙戦術の一種である。ここから、選挙に限らず、対立する人や組織に不利な情報を流すことで自らのイメージアップを謀ることをネガキャンという。ただし、ネガキャンは誹謗中傷合戦に見えることも多く、逆にイメージダウンとなってしまうことも多い。	2009年	略語	
\\	ハムト	はむと	公衆トイレのこと。	ハムトとは若者言葉で、公衆トイレを意味する。これは、まず公衆トイレの“公”の字を『八(はちがしら)』と『ム』に分けて「ハム」。そこにトイレの『ト』を付けたものである。ただし、ハムトは若者の間でもそれほど浸透 ・ 認知されていない言い回しである。	2009年	若者言葉	
\\	東国ばる	ひがしこくばる	宮崎に行くこと。	東国ばるとはお笑いタレントであり、2009年現在宮崎県知事を勤めている東国原英夫の苗字:東国原(ひがしこくばる)をひらがなで終わらせ、動詞風にしたものである(「出しゃばる」との合成語という説もある)。東国原氏の特徴を動詞として使うもので、「宮崎に行く」、「忙しくても、いろんな場所に行く(顔を出す)」、「呼ばれもしないのに、どこにでも出てくる」など、複数の意味で使われる。	2009年	
\\	ファストファッション	ふぁすとふぁっしょん	低価格で手軽なファッション(ブランド)のこと。	ファストファッションとは低価格帯な品揃えで手軽なファッションブランドのこと(ファーストフードの衣類版とも)。店頭に並ぶ商品が高頻度で変わっていくのも特徴のひとつで、数週間で主力商品のラインナップががらりと変わってしまうことも珍しくない。日本ブランドではユニクロが知られる(他にはしまむら、
\\	など)。百貨店が不調な中、売り上げを伸ばしていること。また、海外から
\\	、フォーエバー21、
\\	といったブランドが進出してきたことが社会現象になったこともあり、ファストファッションは2009年のユーキャン新語 ・ 流行語大賞でトップテンに入賞している(受賞者はモデルでタレントの益若つばさ)。	2009年	美容 ・ ファッション用語	
\\	ホス系	ぶいほすけい	男性ファッションスタイルの一種。	
\\	ホス系とはヴィジュアル系
\\	系)とホスト系から成る合成語で、両者を取り入れた形のファッションスタイルを言い、そういうスタイルの男性を
\\	ホス系男子ともいう。ホスト系といえばスーツ姿が一般的だが、ジーンズやジャケットを着たり、骸骨や鋲などのロック系のアクセサリーを付けたものをいう。
\\	ホス系はリイド社の男性向けファッション誌
\\	(メンズスパイダー)』による造語で、同雑誌が提案したファッションである。	2009年	
\\	弁当男子	べんとうだんし	自分で弁当を作って持参する男性のこと。	弁当男子とは自分で弁当を作り、職場に持っていく男性を意味する(学校へ自作弁当を持参する男子学生も含む)。こうした弁当男子と呼ばれる男性が出てきた背景として、不況による食費の節減(外食やコンビニ弁当をやめ、弁当を作る)。他には、ダイエットを目的とした者もある。どちらにしても、嫌々やるというより、自ら楽しんでやるという傾向が強い。 また、こうした弁当男子向けの料理本なども出版されるなど注目され、2009年のユーキャン新語・流行語大賞にノミネートされた(ただし、入賞には至っていない)。	2009年	
\\	ミソる	みそる	年齢が30歳または30台になること。	ミソるとは30歳または30台という年齢を意味する『三十路(みそじ)』に動詞化する接尾語『る』を付けたもので、30歳になること。年齢が30台に突入することを意味する。 同様に40歳になること、40台に突入することを、四十路(よそじ)になるということからヨソるという。	2009年	
\\	ムカ男	むかお	非常にかっこいい男性のこと。	ムカ男とは「無駄に格好いい男(ムダに、カッコいい、男)」の略。「無駄にカッコいい」というのは、女性からモテるのに十分なカッコ良さ以上にカッコいい(=必要以上にカッコいい)というニュアンスで用いられる。つまり、ムカ男とは非常にかっこいい男性を意味する。 これとは別に「むかつく男」の略として、見ていて癪にさわる、存在自体腹が立つ男性という意味で用いられることもある。	2009年	
\\	ユニ隠し	ゆにかくし	周囲にユニクロの服とバレないように着こなすこと。	ユニ隠しとはユニクロ製の衣類を身につけていると、周りの人に悟られないように着こなすことをいう。簡単なものでは、下着やくつ下など、目につかない所でユニクロの商品をまとい、上から他のブランド商品を着て隠すといったものから、ユニクロ商品自体に刺繍を施すといったものまで様々。中には大胆にカットしたり、スプレーで色を塗るといった原型を留めないユニ隠しもある。	2009年	若者言葉	
\\	ユニバレ	ゆにばれ	ユニクロの服を着ていることを周りに気付かれること。	ユニバレとは衣料品の生産 ・ 販売をしている「ユニクロ」と、秘密や嘘などが露顕することを意味する「バレる」から成る造語で、身につけている服がユニクロの商品であると周りの人に気付かれることを意味する。ユニクロブランドのカジュアルな服には安価というイメージがあり、ユニバレ=恥ずかしいという意をもって使われることが多い。同じく、安価な服を売っている店というイメージが強い「しまむら」の服に対しても、シマバレという言葉が使われている。 ユニバレは2009年以前から使われている言葉だが、ファッション ・ 女性誌 ・ トレンドをウォッチするブログ
\\	が2009年1月に取り上げ、広く普及した。	2009年	若者言葉 ・ 合成語	
\\	ヨソる	よそる	年齢が30歳または30台になること。	ミソるとは30歳または30台という年齢を意味する『三十路(みそじ)』に動詞化する接尾語『る』を付けたもので、30歳になること。年齢が30台に突入することを意味する。 同様に40歳になること、40台に突入することを、四十路(よそじ)になるということからヨソるという。	2009年	
\\	離活 ・ リカツ	りかつ	将来の離婚にむけての調査や準備のこと。	離活とは離婚活動を略したもので、婚活(下記関連語参照)の派生語(対語?)として出来た言葉である。2008年、離婚件数は減少したものの、将来の離婚に向けての準備といった離活をするアラフォー女性が増えたことから、取り上げられた。具体的には、離婚後の収入のための資格取得、慰謝料や財産分与のための法的知識を身につけるといったもので、特に子供がいる場合は、子供の成長 ・ 養育費 ・ 学校などの節目を考慮して行われる。 2009年春には
\\	で『コンカツ ・ リカツ』という、桜井幸子、清水美沙、国生さゆり、大桑マイミが出演のドラマタイトルとしても使われている。また、同年のユーキャン新語 ・ 流行語大賞の候補語に選ばれている。	2009年	
\\	リキイシ	りきいし	パワーストーンのこと。	リキイシとはパワーストーンのこと。パワーストーンを直訳すると力石となることから、リキイシと呼ぶようになった(「ちからいし」でなく「りきいし」と読んだのは『あしたのジョー』の力石 徹(りきいし とおる)から?)。 ちなみにパワーストーン自体、和製英語である。また、力石と書いてパワーストーンと読ませることも多い。	2009年	
\\	歴女	れきじょ	歴史好きな女性のこと。	歴女とは歴史好きな女性(女子)のことで、それを略したものである。これは歴史好きなアイドルの美甘子(みかこ)を「歴ドル」と呼ぶことから、その派生として出来た呼び方とされる。トニー ・ レオン、金城武が出演の「レッドクリフ 
\\	が2008年末に日本で公開、2009年には妻夫木聡が
\\	大河ドラマ「天地人」の主演になったことも歴女増加の要因と考えられる。また、2010年の大河ドラマ「龍馬伝」では主役の坂本龍馬を福山雅治が演じることが決まっており、ますます歴女が増えると思われる。 2009年のユーキャン新語 ・ 流行語大賞ではトップテンに入賞。歴女の代表ともいわれる歴史好きな女優;杏が受賞した。	2009年	
\\	アゲ嬢	あげじょう	小悪魔ファッションの女性 ・ 女の子のこと。	アゲ嬢とはインフォレスト株式会社が発行している雑誌「小悪魔
\\	(下記関連サイト参照)」を愛読し、「小悪魔
\\	に出てくるようなファッションをしている女性 ・ 女の子を意味する。「小悪魔
\\	はキャバクラ嬢の教科書とも言われている雑誌で、具体的には髪は逆毛や巻き髪でボリューム感を出し、アイライナーやつけまつ毛でアイラインを強調。デコネイルやそれに合わせたキラキラ系のアクセサリーも特徴のひとつである。	2008年	若者言葉、美容 ・ ファッション用語	
\\	厚姫	あつひめ	自分の都合で周りを巻き込む厚かましい女性のこと。	厚姫とはもともと
\\	のバラエティ番組「サラリーマン
\\	のコーナーのタイトルであり、そのコーナーに出てくる堀内敬子が演じるキャラクター名
\\	姫島あつみ”からきている)のことだが、ここから自分に都合良く周りを巻き込んだり、振り回したりする厚かましい女性のことを厚姫と呼ぶ。ちなみにサラリーマン
\\	の厚姫は、
\\	で2008年に放送された大河ドラマ「篤姫」からきている。	2008年	
\\	あなる	あなる	理解した ・ 合点がいったときに発する言葉。	"あなるとは「あ~、なるほど」が略されたもので、相手の話しが理解出来たとき、合点がいったときに発する言葉。英語で「肛門の」を意味する
\\	(アナル)""との語弊がウケて浸透。2008年に発売された『渋谷語事典2008』に掲載されたことで、広く認知される言葉となったが、関東エリアの若者の間では、それ以前から使われている。
\\	2008年	若者言葉	
\\	アラフォー	あらふぉー	40歳前後である女性のこと。	アラフォーとはアラウンドフォーティー
\\	の略で、40歳前後(35~44歳)の女性を意味し、アラフォー世代ともいう。アラフォーはアラサー
\\	の派生語として生まれた言葉で、2007年から使われるようになる。2008年4月には天海祐希、藤木直人が出演する
\\	系列金曜ドラマ
\\	の略として定着している。この言葉が使われるようになった2007年のアラフォーの特徴として、男女雇用機会均等法のもとで仕事をし、キャリアを積んだ世代。結婚と仕事の選択がそれまでの世代より自由に行えるようになり、キャリアウーマン、女性ベンチャー企業家も珍しくなくなる世代といえる。経済的余裕と拘りからアラフォーをターゲットとする女性誌も多い。また、2007年という時代に関係なく、結婚 ・ 出産 ・ 仕事を決める人生の岐路にある世代といえる。 なお、アラフォーは2008年のユーキャン新語・流行語大賞にてエド・はるみの「グ~!」とともに年間大賞に選ばれている。受賞者は天海祐希。	2008年	和製英語	
\\	アリージョ	ありーじょ	小柄な女性のこと。	アリージョとは蟻女とも書き、蟻のように小柄な(小さな)女性を意味する。大抵は身長155
\\	以下の女性をアリージョと呼ぶ。女性ファッション誌でも小柄なモデルがアリージョモデルとして注目されている通り、アリージョは背が低いことを嘲う言葉でなく、むしろ褒め言葉として使われることが多い。益若つばさ、井上奈保といったアリージョモデルがいる。	2008年	
\\	池様	いけさま	かっこいい人のこと。	池様とはイケメンを略し、『池』という漢字を当てたものに、『様』という敬称を付けたもの。つまり、池様とはイケメン様で、容姿が整った人、格好いい人を意味する。例えば「若い池様募集中」と書かれている場合、「若くてカッコいい人を募集しています」という意味になる。池様はヴィジュアル系
\\	系)を好む人を中心にインターネットを介するやりとりの中で普及した言葉である。	2008年	若者言葉	
\\	うさんぽ	うさんぽ	ウサギの散歩のこと。	うさんぽとは「兔(うさぎ)」と「散歩(さんぽ)」から成る合成語で、文字通り、ウサギの散歩を意味する。独身女性を中心に、ペットとしてウサギを飼う人が増えており、そうした人たちのペットとのコミュニケーションの時間がうさんぽである。散歩方法は犬同様、ウサギにリードを付けて行うが、公園や堤防など、目的地まではバッグなどに入れて連れて行く形が多い。インターネットではうさんぽ専用サイトが立ち上がっている他、
\\	でも多くの関連コミュニティがあり、情報交換が成されている。	2008年	合成語	
\\	おもてなし婚	おもてなしこん	ゲストに喜んでいただくことをメインとする結婚式のこと。	「おもてなし」とは客に歓待 ・ 馳走 ・ 世話をすること。おもてなし婚はゲスト(友人や仕事仲間など)を持て成す(もてなす)ことがメインとなる結婚式を意味する。これまで結婚式の結婚式は派手婚や地味婚、オリジナル婚などいろいろあれど、メインはあくまで新郎・新婦であり、お披露目の場であるからこそ、披露宴といった。しかし近年、足を運んでくれたゲストに喜んでもらうことを目的とする結婚式が増えているという。おもてなし婚には大きくわけて二つある。一つは豪華な料理やミュージシャンによる演奏などで楽しんでもらうもの。もう一つはゲームなどゲストにも参加してもらうタイプのもの。一見これまでと変わりないが、例えば披露宴の定番であった高砂(新郎・新婦の席を一段高くする台)をなくし、ゲストと同じ目線にするといったところにゲスト中心であることが伺える。	2008年	
\\	亀る	かめる	遅れる、遅くなること。とろい(間が抜けている)こと。	亀るとは遅い、ゆっくりといったイメージの動物『亀』に動詞化する接尾辞『~る』を付けたもので、遅れる、遅くなる、遅刻するといった意味で、若者の間で普及。もともとはメールの返事が遅くなることを指したが(メールの返事が遅いことを意味する「亀レス」からきた言葉かどうかは不明)、後に待ち合わせなど、メール以外を対象としても使われるようになる。 また、これとは別にとろい、鈍臭いという意味で亀ると言う場合もある。	2008年	若者言葉	
\\	カレセン	かれせん	50代以上の男性を好む30代以下の若い女性のこと。	カレセンのカレは“枯れ”のことで、ここでは50代以上の枯れた男性を意味し、センはデブ専やブス専に見られる性趣向としての
\\	専門」の“専”である。つまり、カレセン(=枯れ専)とは50代以上の枯れた男性を好む30代未満の女性やそういった趣向自体を意味する。2006年に流行したちょいワルおやじとは一線を画し、背中に哀愁を漂わせ、さりげない知識や経験を持つ枯れた男(おじさん)に癒しを求める若い女性が増えているという。アスペクトから『カレセン―枯れたおじさん専科』という本も出ている。 ※若い女性の中には上記と別に「私は彼だけのもの」というニュアンスの「彼専用」を略してカレセンと言っている場合があるので注意が必要。	2008年	
\\	キター!	きたー	何かが起きた、または起きつつあるときに発する言葉。	キタ-!とは何かが起こった(来た)ときや起きつつあるときに発する言葉で、大抵は期待や緊張、興奮とともに待っていたものが起きた際に発しられる。これはもともと参天製薬の目薬「サンテ
\\	の
\\	内で俳優:織田裕二による、「きたー」と叫ぶシーンからきたもの。掲示板サイト2ちゃんねるにて、これが元となって出来たとされるアスキーアート
\\	とも言う
\\	や携帯の文字を使った絵)が登場、ドラマ「電車男」で頻繁に使われた。2008年には織田裕二のものまねでブレイクした山本高広が頻繁に使ったことから、更に広く普及した。 キターの
\\	の一例:キタ
\\	(゜
\\	゜)
\\	ッ!!	2008年	
\\	鬼太郎	きたろう	「最低」を意味する言葉。	鬼太郎とは水木しげる原作のアニメ「ゲゲゲの鬼太郎」からきた言葉で最低を意味する。これは主題歌の「ゲ ・ ゲ ・ ゲゲゲのゲ~
\\	を「下 ・ 下 ・ 下下下の下~
\\	と捉え、下の下の下だから最低という意味になる。 ちなみに「ゲゲゲの鬼太郎」の『ゲゲゲ』というのは、水木しげるが幼少の頃、“しげる”という名前がうまく発音出来ず、“げげる”になっていたことからきたあだ名で、そこから取ったものと水木氏本人がインタビューで語っている。	2008年	
\\	近距離恋愛	きんきょりれんあい	親友など、身近な人との恋愛のこと。	近距離恋愛とは遠距離恋愛の派生的に出来た言葉で、幼なじみやクラスメート、異性という枠を超えた親友関係にあった人など、身近な存在であった人との恋愛を意味する。また、インターネットの出会い系サイトでは近所での出会い~恋愛を指して近距離恋愛という。 2008年公開された
\\	というアメリカ映画の邦題として使われた。また、『近キョリ恋愛』という表記ではあるが、2007年から別冊フレンドににて連載のマンガのタイトル(作:みきもと凜)にもなっている。	2008年	
\\	けい ・ けい ・ わい	
\\	場の空気がひどく読めない人のこと。 
\\	場の空気を過剰に読み過ぎる人のこと。	
\\	とは
\\	(空気が読めない人のこと)の派生語で、ひとつは「かなり、空気が読めない
\\	という意味で、もうひとつは「過剰に空気を読みすぎ
\\	という意味で用いられる。つまり、同じ
\\	ではあるが、使用する場面によって意味は全く逆になる。 この他に「昨日は結構良かった
\\	、「こらこら、やめなさい
\\	、「コツコツやる
\\	、「今日、会社を辞めた
\\	といった意味で用いられることもある。	2008年	アルファベット略語 ・ 
\\	語、若者言葉	
\\	嫌儲	けんちょ	金儲けを悪とする思想。また、そういった思想の人 ・ 団体のこと。	嫌儲とは金儲けを悪とする思想や感情および、そういった考えを持つ人や団体、そういった考え方が元となる活動や態度を意味する(悪いことをして儲けることを嫌うのみでなく、金儲けを無差別に非難する感情や思想)。嫌儲はインターネット内で使われる俗語(インターネットスラング)で、2005年に話題となった嫌韓(けんかん)をもじったものである(2008年4月現在、メディアなど、インターネット以外での普及は見られない)。嫌儲の読みは特に定まっていないが、多くは「けんちょ」と読み、他に「いやちょ」「いやもう」「けんもう」といった読み方もされる。童話や昔話で悪役に富豪が多いこと、汚職など社会的事件の多くにお金が絡んでいることなどが、嫌儲という考え方を生んだ要因とされている。	2008年	
\\	こそアド	こそあど	密かにメールアドレスを交換すること。	こそアドとは『こっそり』と『メールアドレス』から成る言葉で、こっそりと(=密かに)メールアドレスの交換することを意味する。例えば、合コンで友人に抜け駆けをして目当ての異性とメールアドレスの交換をする(教え合う)といったことがこそアドにあたる。	2008年	若者言葉	
\\	婚活	こんかつ	結婚するための活動のこと。	婚活とは結婚活動の略。家族社会学者である中央大学の山田昌弘教授とジャーナリスト白河桃子の共著『「婚活」時代』からの造語で、就職活動を意味する略語「しゅうかつ」の派生語と考えてよい。晩婚化 ・ 非婚化が進み、恋愛関係が直接結婚に結びつかなくなった昨今、もはや結婚は人生のデフォルト(標準、最初から準備されているもの)ではないとし、就職活動同様、自分を磨き、出会いの場へ出向いて積極的に自分をアピールする婚活をしないと良い結婚は出来ないと言われている。 また、2009年春にはフジテレビが『婚カツ』という表記でドラマタイトルに使用。中居正広が演じる草食男子の雨宮が、ひょうんなことからついてしまった嘘のために婚活をするというもの。	2008年	
\\	ザビる	ざびる	髪が薄い人のこと。	ザビるとはカトリック教会の司教 ・ 宣教師であるフランシスコ ・ ザビエルの『ザビ』に動詞化する接尾語『る』を付けたもので、髪が薄い人を意味する。これはフランシスコ ・ ザビエルの髪型(有名な胸のハートに手を添えている頭頂部に髪の毛が無いザビエル画像)から来たものである。機動戦士ガンダムに登場するザビ家のデギン ・ ソド ・ ザビはスキンヘッド。長男のギレン ・ ザビは額が若干広いが、ザビるとは関係ない。	2008年	若者言葉	
\\	しゃみーす	しゃみーす	相手に合わせること。嘘をつくこと。	しゃみーすとは『三味線』と、「あざーす」や「ちーっす」など崩した挨拶で使われる『~っす』を合成したもので、適当に相手に合わせることや嘘をつくことをいう。これは三味線が語り手に調子を合わせて弾く楽器であることや嘘を意味する口三味線などからきている(詳しくは下記関連語参照)。例えば、誰かに調子を合わせただけで、実際には思っていないことを言った場合に「嘘だよ~」といったニュアンスで「しゃみーす」と言う。	2008年	若者言葉	
\\	女子飲み	じょしのみ	女性だけの飲み会。そういった場でのコミュニケーション。	女子飲みとは「女子だけの飲ミニケーション」の略で、女性だけの飲み会。またはそういった場でのコミュニティ、コミュニケーションを意味する(女性だけという意味の「女子のみ」との掛詞?)。男性がいないことでお酌などの気を使わなくてよい。恋愛から仕事の話しまで、男性とは共有出来ない深い話しが出来、逆に女性同士だから共感が出来る。男性と一緒だと店が固定化しやすいのに対し、女性のみだと開拓出来るといった理由から、女子飲みが増えている。 ただし、次第にお酒の有無に関係ない『女子会』という言葉に集約されている。	2008年	
\\	外こもり	そとこもり	海外で引きこもり生活をすること。	外こもりとはアルバイトなど日本で短期間集中的に稼いだ金を持って物価の安い国に行き、そこで長期滞在して生活することをいう。また、そういった人を外こもり族という。言葉自体は2006年に使われ始め、2008年にそうした人が増加していると再度話題になった。なお、外こもり族の長期滞在先として多いのはタイ王国の首都バンコク(ネットカフェやコンビニがあり、街は近代的だが物価は安いため)。一見、優雅な海外生活にも見える外こもりだが、海外と引きこもりの合成語であることからもわかる通り、滞在先で外出することは少なく、中には一日中宿にこもっている者もあるという。	2008年	
\\	タヒる	たひる	死ぬ。絶望的。非常に疲れていること。	タヒるとは漢字の『死』を分解したもの(『歹(かばね偏、または、いちた偏 ・ がつ編という)』の上部の横棒は端折られている)に、動詞化する接尾語『る』を付けたもの。単純に「死ぬ」という意味でも使われるが、多くは「(死ぬほど)絶望的な状況になる:使用例①」「(死ぬほど)疲れた:使用例②」という意味で使われる。 タヒるの使用例: 店長が後ろに居ると知らず、悪口言ってしまった~。タヒる~。① 一昨日からバイトと飲む会で全然寝てなくてタヒった。②	2008年	若者言葉	
\\	ダルビッシュ	だるびっしゅ	疲れて億劫なさま。	ダルビッシュとは北海道日本ハムファイターズに所属するプロ野球のダルビッシュ有のこと。関東の若者の間では、このダルビッシュ有の“ダル”と、疲れて億劫なさまを意味する怠い(だるい)をかけ、ダルビッシュは怠いと同意に使われる(単に接頭部が同じというだけで、ダルビッシュ有選手が怠いに通じるエピソードを持つわけではない)。また、2008年北京オリンピックで初戦キューバ戦で敗戦投手になった翌々日、突如丸刈りにして球場入りしたことが話題となり、一部では丸刈りのことをダルビッシュという。ただし、先述の意味で
\\	で紹介されたことから、基本的には怠いという意味で使われている。	2008年	若者言葉	
\\	チョクる	ちょくる	会ってすぐに告白すること。	チョクるとは「直ぐ(すぐ)」の直を音読みしたものに、動詞化する接尾語『る』を付けたもの。ここから「すぐにする」という意味になるが、大抵は会ってすぐに告白するという意味で使われる。 これとは別に直(チョク)を「直接」の略、または「直に(じかに)」の音読みしたものという解釈で、直接する、直接言うという意味で用いられることもある。	2008年	若者言葉	
\\	チョリース	ちょりーす	「こんにちは」という挨拶、または「了解」の意。	チョリースとは「こんにちは」という挨拶や「了解」「わかった」といった意味で使われる言葉である(同義 ・ 類似語に「チーッス」があるが、その派生語か否かは不明)。チョリースは渋谷を中心に関東エリアの若者の間で普及。タレント木下優樹菜が用いたことで、全国的に浸透した(下記註参照)。 註)木下優樹菜が所属するプラチナムプロダクションに問合わせたところ、チョリースは木下優樹菜による造語でなく、仲間内で使っていた言葉とのこと。 註2)ディスコ(クラブ)や一部エリアでは1990年代から使用(読者情報)	2008年	若者言葉	
\\	デコ弁	でこべん	おかずを型抜きしたり、色の組み合わせで綺麗に飾った弁当のこと。	デコ弁とは「デコレーション弁当」を略したもの。おにぎりや型抜きした根菜、ウィンナー、果物などで顔や動物を作ったり、緑 ・ 赤 ・ 黄といった色とりどりの野菜を使って綺麗に盛りつけ(飾り付け)られた弁当をデコ弁という。アンパンマンやピカチュー、キティちゃんといったキャラクターを弁当で形作ったものもデコ弁の一種だが、これらは特にキャラ弁という(順番的にはデコ弁のほうがキャラ弁より後に普及した言葉である)	2008年	略語	
\\	デコラー	でこらー	携帯電話や衣服などに好んで装飾をする人のこと。	デコラーとは装飾(デコレーション)することを意味する「デコる(下記関連語参照)」に英語で人化する接尾辞
\\	を付けたもの。例えば、携帯電話にビーズで装飾を施したり、服や鞄などに刺繍や飾り付けを好んでする人のことをデコラーという。 また、携帯メールを送る際、好んでデコメール
\\	ドコモのサービスの一つ)する人のこともデコラーと呼ぶ。	2008年	
\\	デパ弁	でぱべん	デパートで売られている駅弁のこと。	デパ弁とはデパートやスーパーの催事場で売られる駅弁のこと。デパ弁といえば、京王百貨店が1966年から行っている『元祖有名駅弁(別名:駅弁の甲子園)』が有名だが、他にも多くの百貨店で全国の駅弁を集めた催しがあり、集客の一つとなっている。駅弁業者側から見ても、デパ弁が年間売り上げの9割強という弁当もあり、80年代以降、下降気味となっていた駅での販売数をカバーする収入の要となっている。デパ弁がここまで人気となった背景として、全国の駅弁が一同に介すため、複数で様々な弁当を購入して楽しめる点、調理実演販売で見て楽しめる点、待たせないことを目的に作り置きされた駅で販売される弁当と違い、出来立てほやほやの弁当が食べられる点があげられる。	2008年	
\\	どつかれ様	どつかれさま	殴られた人に言う言葉。ひどく疲れた人に言う言葉。	どつかれ様とは関西弁で「殴られる」を意味する「どつかれる」と「お疲れ様」という挨拶をかけた言葉で、殴られた人に言う(挨拶的な)言葉である。 また、「お疲れ様」の『お』を、後に続く言葉を強調する『ど(例:ど真ん中、ど素人)』にしたもので、ひどく疲れた人にかける言葉としても使われる。	2008年	
\\	内定長者	ないていちょうじゃ	複数の企業から内定をもらっている大学生のこと。	内定長者とは就職活動を受けた結果、複数の企業から内定をもらった大学生を意味する。就職氷河期と呼ばれるほど、就職難の時代が続き、大学側は内定をもらった学生に対する別会社の就職斡旋をしない傾向にあった。しかし、緩やかながらも回復の方向に向かっている経済状況、団塊世代が大量に退職したことによる労働者不足、さらにインターネットを使った個々の自由な就職活動により重複内定者が増加。こうした重複内定者を内定長者という。ただし、こういった時代にあっても内定がひとつももらえない学生もおり、就職活動に関しても格差が現れたという見方もある。	2008年	
\\	ナマドル	なまどる	訛りのあるアイドルのこと。	ナマドルとは「訛り(なまり)」と「アイドル」から成る合成語(「なまりアイドル」の略語とも)で、訛り(方言)を使って話すアイドルを意味する。1990年代辺りから地方アナウンサーがあえて標準語でなく、地元言葉で訛ったまま話したり、2004年にはロコドル、ジモドルと呼ばれる地方だけで活動するアイドルがその土地の訛りを好んで使った。更に全国区で活動するアイドルの中にも故郷の訛りを直さず、そのまま話す人が現れ、2008年ブームになりつつある。主なナマドルに福島県出身の野紗耶香、山形県出身の佐藤唯がいる。	2008年	合成語	
\\	貧格	ひんかく	品格が欠如していること。	貧格とは品位 ・ 気品などを意味する『品格』をもじったもので、品格が欠如していること(=品格が貧しい)を意味している。貧格は2005年11月に発売された藤原正彦著『国家の品格(新潮新書)』や2006年9月に発売された坂東眞理子著『女性の品格
\\	新書)』といったベストセラーによる「品格ブーム」から生まれた言葉である。また、逆に2008になると土屋賢二著『ツチヤの貧格』、ビートたけし著『貧格ニッポン新記録(小学館101新書)』、横澤彪 ・ 
\\	ニュース編集部著『テレビの貧格』といった書籍が出版されている。	2008年	
\\	ペラい	ぺらい	薄っぺらいの略。	ぺらいとは「薄っぺらい」の略で、ギャルを中心に若者の間で使われる。ただし、若者が使う場合、多くは人の軽薄さ、中身の無さを指して使うことが多い。次にコミュニケーションや言動など、人がすることの軽さ、薄さに使われ、ペラい本(薄っぺらい本)というように、物に対して使うことは少ない。これは、ペラいがチャラいと同義の新たな言葉として浸透したためと思われる。チャラい男性のことをチャラ男というように、ペラ男という言葉も存在する。	2008年	若者言葉 ・ 略語	
\\	ペリる	ぺりる	図々しい態度をとること。	ペリるとは江戸時代、鎖国をしていた日本に艦隊(黒船)を率いて来航し、開国させたことで知られるアメリカ海軍のマシュー・ペリーの『ペリー』に動詞化する接尾語『る』を付けたもので、図々しい態度や物事を強引に推し進めること、何かに固執することをいう。これらはペリー来航から開国までの様子から付けられた意味である。	2008年	若者言葉	
\\	ポニョる	ぽにょる	
\\	太ること、太っていること。 
\\	映画「崖の上のポニョ」を見ること及び、それに付随すること。	
\\	ポニョるとは厚みがあり、柔らかい感じを表現する擬態(音)語「ポニョポニョ」に動詞化する接尾語「る」を付けたもので、太ることや太っていることを意味する。比較的親しみを込めて用いられることが多い言葉である。 
\\	ポニョるとは宮崎駿監督、2008年公開の映画「崖の上のポニョ」のポニョに動詞化する接尾語「る」を付けたもの。「崖の上のポニョ」を観に行くという意味での使用が多く、他に藤岡藤巻と大橋のぞみが歌い、大ヒットした主題歌を口ずさむという意味でも使われる。また、「ポニョのようになる(ポニョのような行動をする)」という意味でも使われるが、具体的には不明(調査中)。	2008年	若者言葉	
\\	メタボ	めたぼ	内臓脂肪型肥満により、病気が引き起こされやすくなった状態。	メタボとは「メタボリックシンドローム
\\	の略で、内蔵脂肪型肥満によって、さまざまな病気(肥満症・高血圧・糖尿病・高脂血症といった生活習慣病)が引き起こされやすくなった状態のことをいう。2006年の医療制度改革により、2008年4月から健康保険組合、国民健康保険などに対し、40歳以上の加入者を対象としたメタボリックシンドロームに着目した特定健診及び、保健指導の実施が義務付けられた。これにより、メタボという言葉も普及。単に太った人(特に腹回りが太い人)、デブといった意味でも使われるようになる。	2008年	略語	
\\	モナる	もなる	不倫をすること。	モナるとはアナウンサー、タレントの山本モナの名前モナに、動詞化する接尾語『る』を付けたもので、山本モナの2度の不倫騒動から不倫することをいう。また、1度目の民主党の細野豪志議員との不倫騒動では路上キスシーンが写真週刊誌のスクープ画像として掲載されたことから、路上キスのことをモナると言ったり、騒動後、一時的に活動休止し、復帰したが、再び二岡智宏(当時読売ジャイアンツ投手)との不倫騒動が取り上げられるとチャンスを自らの手で潰してしまうという意味でも使われた。	2008年	
\\	ゆう ・ けい	糞 ・ 大便。またはそれを排泄すること。	
\\	とは糞・大便の俗称である「うんこ
\\	のアルファベット略語
\\	の流行と共に普及したことから
\\	語、
\\	的日本語ともいう)で、大便及び、大便を排泄すること。大便をしに行くことを遠まわしに表現する際に「ちょっと
\\	行ってくる」というように用いる。ただし、一般的には
\\	の頭文字として、イギリス・英国という意味で用いられる。	2008年	若者言葉	
\\	指恋	ゆびこい	
\\	好きな人と携帯メールでやりとりをすること。 
\\	携帯メールのやりとりの中で芽生えた恋のこと。	指恋の指とは携帯メールを打つ際に使う指ということで、携帯メールや携帯メールを打つという作業を指す言葉として若者言葉で用いられる(親指族など)。つまり、指恋とは携帯メールの恋ということになり、好きな人とメールでやりとりすることや携帯メールのやりとりをする中で育まれた恋愛関係といったことを意味する。	2008年	若者言葉	
\\	ゆるキャラ	ゆるきゃら	国や地方の公共機関が作ったキャラクターのこと。	ゆるキャラとは「ゆるいキャラクター」の略で、主に国や地方が町(村)おこし、イベント、地方名産の
\\	のために作ったキャラクターを指す(特に定義はなく、作りや見た目がしっかりしているなど除外されこともある=公共機関のキャラクターであればゆるキャラというわけではない)。ゆるキャラは漫画家“みうらじゅん”による造語で、2002年には『ゆるキャラショー』を、2004年には『ゆるキャラ図鑑』という書籍も出版している(下記通販検索に関連
\\	、書籍が掲載)。2007年に彦根城築城400年を記念して作られた『ひこにゃん』で火がつき、2008年発表の「せんとくん・まんとくん」問題も手伝い、ゆるキャラが広く認知される。2008年新語・流行語大賞のノミネート60語に入っている。	2008年	
\\	ラグい	らぐい	オンラインゲームにおいてネットワークのパケット送受信の遅延によるゲーム動作の鈍いさまを表す言葉。	ラグいとはタイムラグに見られる「遅れる」「進み方が遅い」といった意味の英語
\\	(ラグ)』に形容詞化する接尾辞『い』をつけたもの。ここでいうラグはネットワークゲームで指示 ・ 命令が実行されるまでの時間及び、ネットワークの中でパケット送受信にかかる時間を指す。つまり、ラグいとはオンラインゲームの反応の悪さやその要因であるネット回線の遅さを意味する言葉である。(ホームページで表示が遅い場合に「重い」と言うが、それと同類の言葉である)	2008年	ネットスラング	
\\	ラスボス	らすぼす	勝てそうにない相手、解決が難しそうな問題のこと。	コンピュータゲームで各ステージの最後に出てくる強い敵キャラクターをボスキャラクターといい(ボスキャラや単にボスともいう)、それらをクリアし、ゲームの最終局面に出てくるボスキャラクターのことを最後のボスという意味でラスボス(ラスト ・ ボスの略)という。当然、ラスボスは簡単には勝てない設定になっている。そんなラスボスの特徴から、ゲーム以外でも、簡単には勝てそうにない相手、簡単に解決出来そうもない問題のことをラスボスというようになる。	2008年	若者言葉	
\\	リクラブ	りくらぶ	就職活動がキッカケで始まる恋愛のこと。	リクラブのリクとはリクルートの略で就職活動を意味する。つまり、リクラブとは就職活動で芽生えた恋、就職活動をキッカケにスタートした恋愛を意味する。普段のカジュアルファッションや崩した着こなしと違い、会社説明会や就職試験といった就活の場での、リクルートスーツに身を包んだ姿が新鮮にうつり、恋愛に発展するケースが増えているという。また、サークル活動など、出会いの場に参加する学生が減っていることも、リクラブ増加の一要因といわれている。	2008年	若者言葉	
\\	冷ター	れいたー	何でも冷凍してから食べる人のこと。	冷ターとは「冷凍」と「イーター(食べる人の意。
\\	に人化する接尾辞-
\\	をつけた和製英語)」から成る合成語で、様々なものを冷凍して食べる人のこと。バナナを凍らせて食べたり、コンビニの冷凍庫に並ぶペットボトル(冷凍飲料)を利用するといった人は多い。しかし、冷ターはこれにとどまらず、プリンや大福、果てはカレーやおでんといった温かい料理も冷凍して食べるという。 ちなみに冷ターの読みは「つめたー」ではなく「れいたー」。	2008年	
\\	あある ・ てぃ	様々な言葉の頭文字であり、それらの意味を持つ言葉。	
\\	とは
\\	語(日本語言葉(単語)をローマ字表記し、その頭文字だけを使う若者言葉。
\\	略語、
\\	式日本語、アルファベット略語ともいう)の一種で、多くの意味を持つ。
\\	類は友を呼ぶ
\\	ラジオ体操
\\	ラーメン食べる
\\	路駐
\\	など。 日本語の頭文字でないため
\\	語ではないがリアルタイム
\\	略、またリプレイタイム
\\	の略でパチスロ用語としても使われる。	2007年	
\\	語 ・ アルファベット略語	
\\	愛方	あいかた	恋人のこと。彼氏 ・ 彼女のこと。	愛方とは漫才など二人組の一方が他方を指す場合に使う「相方(下記関連語参照)」が転じたもので、恋人 ・ 彼氏 ・ 彼女という意味で使われる。基本的に二人組の一方が他方を指す言葉であるため、カップルとして成立している相手方を指す。また、恋愛関係に限らず、親友など深い仲の人に対しても用いる。	2007年	
\\	あい ・ けい	思い通りにならず、気が焦る(苛立つ)さま。	
\\	とは「イラッとくる
\\	の頭文字をとったアルファベット略語で、苛立つさま、思い通りにならず、気が焦るさまを意味する。これは週刊誌
\\	2007年11月27日号」の中のイラッと来るメール文例集という企画の中で、イラッとくるメールを
\\	メールと呼んだことから普及したものである(ただし、それ以前からイラッと来ることを
\\	と言ったり、苛立たせる人を
\\	野郎と呼んでいたエリア、世代、グループもある)。2007年は
\\	語
\\	式日本語)ブームでこうしたアルファベットの頭文字による言葉が頻出したが、
\\	もそのひとつである。	2007年	
\\	語 ・ アルファベット略語	
\\	アサヒる	あさひる	捏造すること。	アサヒるとは朝日新聞の“アサヒ”に動詞化する接尾語“る”をつけたもので、捏造する、執拗に攻撃するといった意味で使われる。これは石原壮一郎氏が安倍元首相の突然の辞任発表を受け、朝日新聞のコラムで「アタシ、もうアベしちゃおうかな」という言葉が流行していると問題提議したことによる。これに対し、ありもしない流行語をあたかもあるかのように報じ、執拗に安倍元首相を攻撃しているとしてインターネット上で反論が多数出る。そして逆にアサヒるという言葉が生まれ、話題となった。	2007年	
\\	アニドル	あにどる	アイドル的存在の動物のこと。	アニドルとは日経流通新聞による造語で、アニマル・アイドルの略。アイドル的存在の動物を意味し、
\\	で人気となっている動物の他、ブログから火がついて話題となった動物、更にある特定地域や動物園、その他特定施設で人気の動物を指す。一般的にはある程度広く認知されている話題の動物を指すが、ここから「我が家のアニドル」といった具合にご近所 ・ 家族レベルでも使われる。	2007年	
\\	いかほどー	いかほどー	「どんだけー」に対して返す言葉。	いかほどーとは2007年にメイクアップアーティスト“
\\	が使った流行語「どんだけー」に対し、返答として普及した言葉である。会話中に「どんだけー」と突っ込まれた際、「いかほどー」と言い返す形で使用する。「どんだけ?」という疑問(質問)系の言葉に対し、「こんだけ(これだけ)」でなく、疑問系で返すいかほどーはそれ自体が特に意味を持つ言葉ではない。	2007年	若者言葉	
\\	痛車	いたしゃ	アニメやゲームのキャラクターがあしらわれた車のこと。	痛車とは痛い車の略で、アニメやゲームのキャラクター及びロゴのシールが貼ってあったり、そういったものが塗装されている車を指す。こうした2次元キャラクターをあしらった車は、所有者(ドライバー)がオタク、アキバ系であると主張していることから痛い車、略して痛車と呼ぶ(正確には単に略しただけでなく、イタリア車の略称「イタ車」とかけた言葉である)。所有者本人が自嘲して使うことも多く、自身を痛車乗りというドライバーもある。	2007年	若者言葉	
\\	色落ち	いろおち	色気がなくなること。	色落ちとは本来、衣服の染め色が洗濯などにより落ちることをいう。ここから、若者の間で、人としての色気がなくなることを色落ちという。男性(の色気)に対して使われることもあるが、大抵は女性(の色気)に対して用いられる。	2007年	
\\	ウソ電	うそでん	写真を加工して作った架空(実在しないデザイン)の電車のこと。	ウソ電とは「ウソ(嘘)の電車」の略で、鉄道写真を画像編集ソフトなどで加工して作られた架空の電車(実在しないデザインや色の電車)を意味する。加工方法で多いのが色を変えたウソ電。例えば新車両の画像に旧車両の配色で加工するといったものがある。他にもドア数の増減(本来ドアの少ない特急形車両のドアを増やしたり、逆に通勤形車両のドア数を減らすといったもの)、背景の入れ替え(山手線の車両が大阪駅に停車していたり、貨物列車が地下鉄道を走るなど)、異なる路線の先端と中間車両の入れ替えなどがある。	2007年	
\\	えいち ・ けい	話題を変えることを告げる繋ぎの言葉。	
\\	とは話題を変えるときに言う「話、変わるけど
\\	の頭文字である。
\\	など、様々な言葉を頭文字で語る形が若者の間で使われるが、
\\	もその一種。これら頭文字言葉には、ギャル文字に見られた暗号的要素や携帯メールで打つ際の手軽さがある。 若者の間では以前から使われている言葉だが、2007年、こうした頭文字略語の存在が話題となり、広く知られるようになる。 ※人名の頭文字など、他の意味で使用していることも多いので注意	2007年	コギャル語 ・ ギャル関連、アルファベット略語	
\\	えいち ・ えむ	前に話していた話題に戻すことを告げるときの言葉。	
\\	とは「話し戻る(戻す)けど
\\	の頭文字から成るアルファベット略語
\\	の流行から普及したため、
\\	語・
\\	式日本語ともいう)である。いったん変わった話題を元に戻す際に言い、話題を変えるときに使う
\\	とセットとして使える。 音楽ジャンルのひとつヘヴィメタル(通称:ヘビメタ)の略としての使用も多い。	2007年	若者言葉、アルファベット略語	
\\	えー ・ けい ・ わい	周囲の雰囲気・状況を気にしないこと。	
\\	とは「あえて空気読まない」の略で、「空気読めない」を意味する
\\	の派生語にあたる。
\\	(空気読めない)が注目され、集団生活や仕事において空気が読めることの重要性が叫ばれる中、あえて空気を読まない=周囲の雰囲気や状況を気にしない、振り回されないといった姿勢が
\\	である。 このようなアルファベット略語(頭文字略語)は以前から若者の間で使われているが、2007年、
\\	と共に多くの略語が広く紹介された。	2007年	アルファベット略語、若者言葉	
\\	えー ・ しー	場の空気を乱す人のこと。	
\\	とは“
\\	の頭文字で、場の空気を壊す人。つまり、場の雰囲気を乱したり、場の空気を読まずに不用意な言動をする人のことをいう。 
\\	のようなアルファベット略語は若者の間では以前から使われているが、2007年、
\\	と共に多くの略語が広く紹介された
\\	語
\\	式日本語)ともいう)。	2007年	アルファベット略語、若者言葉	
\\	えす ・ けい ・ えぬ	「そんなの関係ねえ」の頭文字。	
\\	とは小島よしおのギャグ「そんなの関係ねえ
\\	を頭文字で略したもの。「そんなの関係ねえ」は「オッパッピー」とともに2007年にブレイクしたギャグで、同年の流行語として様々な場面で用いられる。同時に2007年は
\\	など頭文字略語も話題に。こうして若者の間で普及した
\\	だが、日本ハムの田中幸選手が日本シリーズ前のインタビューで
\\	ですから」と発言。新聞も
\\	野球」
\\	精神」と報じたことから、若者以外にも広く知られるようになった。 ※元の「そんなの関係ねぇ」は2007年流行語大賞トップテンに入賞。	2007年	アルファベット略語、若者言葉	
\\	えす ・ けい ・ わい	場の空気を察することが出来ないこと。またはそういった人。	
\\	とはスーパー
\\	。つまり、「スーパー空気読めない
\\	の頭文字で、場の雰囲気を察することが人並み外れて出来ないこと及び、察することが出来ていない人を意味する。他の
\\	語同様、「エス・ケイ・ワイ」とも読むが、
\\	に限っては空を意味する英単語の綴りと同じため、「スカイ」とも読む。他に
\\	を強調したものとして
\\	といった言葉がある。	2007年	アルファベット略語、若者言葉	
\\	えぬ ・ だぶりゅー	ノリが悪いこと。またはそういった人のこと。	
\\	とは北西
\\	ノースウェストの頭文字)という意味で用いられることが多いが、若者の間では「ノリが悪い
\\	という意味で使われる。例えば複数人でカラオケに行き、順番に歌っているのに一人だけ歌わなかったり、飲み会で盛り上がっているのに一人だけ誰とも話さず部屋の隅にいるような人を
\\	という。こういったアルファベット略語は以前から使われているが、その中の
\\	が2007年に流行。
\\	語
\\	式日本語)と呼ばれて話題になった。	2007年	アルファベット略語、若者言葉	
\\	えむ ・ あい ・ だぶりゅー	相手の言動や状況が理解出来ていないときに発する言葉。	
\\	とは「マジで意味わかんない
\\	の頭文字から成るアルファベット略語
\\	の流行とともに注目されたことから
\\	語、
\\	式日本語ともいう)である。相手の言っている言葉の意味が全く理解出来ないときや状況が把握出来ないときなどに
\\	という。	2007年	アルファベット略語、若者言葉	
\\	えむ ・ えい	予測 ・ 想像が出来ないことが起きたときに発する言葉。	
\\	とは「マジありえない
\\	の頭文字からなるアルファベット略語
\\	の流行とともに注目されたことから
\\	語、
\\	式日本語ともいう)である。予測出来ないことや想像を絶することが起きたとき、また信じられないようなヒドイ言動・理解不可能な言動を誰かがしたときに
\\	という。	2007年	若者言葉、アルファベット略語	
\\	えむ ・ けい ・ わい	場の雰囲気や状況を察することが出来ない人。	
\\	とは「マジ、空気読まない」の略で、「空気読めない」を意味する
\\	(空気読めない)の派生語にあたる。
\\	同様、場の雰囲気や状況を察することが出来ない人を指すが、
\\	が軽く、冗談っぽく言う場合にも使われるのに対し、
\\	はそういった人への苛立ちや呆れが伴い、真剣に言う場合の使用が多い。	2007年	アルファベット略語、若者言葉	
\\	おー ・ えむ ・ じー	驚いたときや感動したときに発する言葉。	
\\	とは“
\\	の頭文字で、驚いたときや感動したとき、助けを乞うようなときに発する言葉である。一応、
\\	語
\\	的日本語)として普及した言葉だが、日本語の頭文字ではなく、
\\	と共にイレギュラーな部類に入る。 また、上記以外に“
\\	の頭文字として、「俺にとって女(彼女)は君だけ」といったニュアンスでも使われる。	2007年	アルファベット略語	
\\	オタージョ	おたーじょ	オタクな女性のこと。	オタージョとはオタクと女子(じょし)の合成語で、オタクな女性を意味する(この数年前にアデージョという言葉が流行っており、ここからオタク版として文字ったものとも考えられる)。オタクな女子を意味する言葉は既に腐女子が定着していた。しかし、腐女子は文字から来るイメージが
\\	などで使用するのに不適切とし、マスコミが考えた同義の造語がオタージョである。(ただし、実際に浸透するには至っておらず、腐女子で通っている)	2007年	合成語	
\\	オッパッピー	おっぱっぴー	様々な意味で使われる囃子言葉。	オッパッピーとは沖縄出身のお笑い芸人“小島よしお”のギャグのひとつで、小島よしお曰く、
\\	(オーシャン ・ パシフィック ・ ピース:太平洋に平和を)の略とされているが、特に深い意味はなく、囃子言葉と考えてよい。ちなみに太平洋は
\\	であり、オッパッピーでは「太平洋に平和を」という意味にはならない。一説には卑猥な意味合いであったが、ダメ出しが出たために現在の意味になったとも言われる。小島よしお自身は「そんなの関係ねえ!」というネタの合間にオッパッピー(その後、チントンシャンテントンという言葉も使う)を入れ、ネタとネタの繋ぎ的役割で使っている。 2007年夏にブレイクしたこのギャグは若者を中心に即座に普及。誰かが冗談を言った際に「面白いー!」といったニュアンスで使ったり、オッパッピーな週末、オッパッピーな子供たちといった様々な形で使われている(特に意味を持たない言葉のため、各々が独自のニュアンスで使用しており、会話の流れからどういったニュアンスで使っているか汲み取る必要がある)。 なお、2007年流行語大賞にノミネートされている(入賞は逃している)。	2007年	若者言葉	
\\	オトメン	乙男(おとめん	乙女チック(乙女のような趣向)な若い男性のこと。	オトメンとは菅野文著作のマンガ ・ タイトル で、『乙女』と男性を意味する英語
\\	の合成語だが、このマンガの主人公:飛鳥のような男性が女性にうけるのではないかとして話題になった言葉である。飛鳥は剣道部の主将を努めるイケメンという顔の裏に乙女チックな趣味 ・ 思考 ・ 特技を持つというのが特徴である。ここから、乙女的趣味を持ち、料理 ・ 掃除 ・ 洗濯など家事全般に長けるといった乙女要素を持ちつつも男らしさを兼ね備えた若い男性をオトメンという(最後の男らしさを兼ね備えているというのがポイントであり、オカマやニューハーフとは異なる)。2007年4月現在ではマンガが人気を博し話題になっている段階で、実生活レベルでオトメンがモテているという例はないが、今後注目のキーワードとされる。	2007年	若者言葉	
\\	オラニャン	おらにゃん	普段は男らしいが二人きりの時は甘えん坊な男性のこと。	オラニャンとは普段「オラオラ」と強気で男っぽいが、彼女など異性と二人きりになると「ニャンニャン」と甘える男性のことをいう。2005年に普及したツンデレ(男性と二人きりになると変貌する2面性のある女性のことで、2006年にはその派生語ツンデレラーが流行語大賞にノミネートされている。詳しくは下記関連語参照)の男性版にあたるオラニャンはギャルサーなどを介し、ギャルの間で使われるようになる。普段は虚勢を張っていても、自分の前では子供っぽさを見せたり、甘えん坊に変貌する姿が母性本能を刺激。こうしてオラニャンは女性から好かれるとも言われる。	2007年	若者言葉、コギャル語 ・ ギャル関連	
\\	ガオる	がおる	気分が悪くなる、驚く、恐れるさまを表す言葉。	がおるとはもともと山形県および宮城県方面で使われた方言で「(病気などで)具合が悪くなる ・ 弱る ・ 疲れる」、または「(都合の悪い事態に)青くなる ・ 落ち込む ・ ヤル気がなくなる」といった意味で使われる(北海道の一部でも同様の意味で使う地域がある)。これが若者の間で普及。表記がカタカナ混じりのガオるになり、「ガオりまくり」など使い方は今風だが、意味の変化はない。どういった経緯で若者が使い始めたかは不明。	2007年	若者言葉(元方言)	
\\	格差婚	かくさこん	収入や家柄、名声などに格差のある二人が結婚すること。	格差婚とは収入や家柄、名声などに格差がある二人の結婚のことだが、特に女優:藤原紀香とタレント:陣内智則の結婚を指して使われた(披露宴の席で陣内自身がこの結婚を格差婚と発言するシーンもあった)。そのためか、2007年現在、男性の収入や名声に対し、妻となる人のそれが格段に高いという意を含み、逆玉と同義に使われている。こうした格差婚は昔から本や舞台、映画など様々な形で取り上げられているが、2005年に話題となった『電車男』、更に藤原と陣内が結婚するキッカケとなった共演ドラマ『59番目のプロポーズ』も格差婚(格差恋愛?)を扱い、富や名声より、純粋さやひたむきさを優先することの美徳が注目され始めている背景があった。	2007年	
\\	過疎る	かそる	
\\	ある場所から人がいなくなったり、物が少なくなること。 
\\	男性の髪の毛が薄くなること。	過疎るとはある地域の人口が少なすぎることを意味する『過疎』に動詞化する接尾語『る』を付けたもので、ある場所から人が減ったり、物が少なくなることをいう。例えば、賑わっていた商店街から人が減ったり、インターネットの掲示板の書き込みが減るといったことを過疎るという。また、減るということから、中年男性の髪の毛が減る(=薄くなる)ことも過疎るという。	2007年	若者言葉、ネットスラング	
\\	カミラー	かみらー	よく言葉を噛んでしまう人のこと。	カミラーとは「噛む(噛み)」に英語で言葉を人化する接尾語
\\	をつけたもので、会話の最中に言葉をよく噛む人(言葉が詰まる、つっかえる人)のことである。
\\	を付けただけだと『カマー(カミャー)』となるが、「ラー」で終わっているのはこの手の人化された言葉の多くが「ラー」で終わっているためと考えられる(下記関連語参照)。また、噛む ・ 噛まないに関係なく、滑舌の悪い人をカミラーという。2006年末に発売されたホリウチダイスケの著書『滑舌悪いんです 
\\	はカミラーを主人公とした作品として話題となる。	2007年	若者言葉	
\\	カラアゲ	からあげ	
\\	空元気だけど、テンションを上げるさま。 
\\	カラオケに行って、テンションを上げるさま。	カラアゲとはアゲアゲ(下記関連語参照)の派生語で、二つの意味を持つ略語である。まず、多く使われるのは「空元気(からげんき)でアゲアゲ」。元気がないときや落ち込んでいるとき、空元気でテンションをあげるさまを意味する。落ち込んだ友人に言ってあげたり、落ち込んでいる当人が「カラアゲ~」と言うことで自己暗示的にテンションをあげるのに用いる。 もう一つは「カラオケでアゲアゲ」の略で、文字通り、カラオケに行き、テンションがあがっているさまを意味する。	2007年	若者言葉	
\\	ギザどん	ぎざどん	突っ込みを入れる際に使う言葉。	ギザどんとは「凄く~」「とても~」「超~」といった意味でアイドル中川翔子が使う“しょこたん語”の『ギザ』と、突っ込みを入れる際の言葉として2007年の流行語である『どんだけー』の合成語である。ただし、『どんだけー』は美味いや面白いといった言葉と違い、強調させ得る類いの言葉ではないため、ギザはしょこたん語(しょこたん風)にするために付けたものと考えてよい。 「どんだけー」はメイクアップアーティストの
\\	がよく使う言葉だが、
\\	番組
\\	には香取慎吾がこの
\\	に扮し、
\\	と名乗るコーナーがある。ここに中川翔子がゲスト出演した際、ギザどんが使われ話題となった。	2007年	若者言葉	
\\	キャラ立ち	きゃらだち	個性を際立たせ、他との違いがはっきりしていること。	キャラ立ちのキャラとは「個性」「特性」といった意味のキャラクター(英語では
\\	の略。つまり、キャラ立ちとは個性や特性を際立たせ、他との違いを明確にすることで、キャラクター(個性)が確立していることをいう。アニメや漫画の登場人物、お笑いを中心とした芸能人に対して使うことの多い若者言葉であったが、2007年の安倍元首相辞任に伴う自由民主党総裁選挙の選挙演説の中で麻生太郎氏が「私は非常にキャラが立ち過ぎている」と発言。広く知られる言葉となった。	2007年	
\\	けい ・ ぜっと、けい ・ ずぃー	絡みづらいさまを表す言葉。	
\\	とは「絡みずらい
\\	本来、絡み辛いなので「絡みづらい」が正しいが、頭文字が
\\	のため、あえて「ずらい」と表記しています)」の頭文字からとった略語である。絡みづらい、つまり(人やグループ、話題に)接しにくい、関わりにくいさまのことで、「アイツ、
\\	だからなあ」
\\	な話題」といった使い方をする。これはアルファベットの頭文字を使った略語は以前から使われているが、2007年にその中の
\\	が流行語になったことから、
\\	語
\\	式日本語)と呼ばれ、話題となった。	2007年	若者言葉	
\\	けい ・ わい	
\\	場の雰囲気 ・ 状況を察することが出来ない人のこと。 
\\	現場作業における危険予知のこと。	
\\	その場の雰囲気や状況などを察する(感じる・掴む)ことを「空気を読む」とも表現する。
\\	はこの「空気」と「読む」の頭文字で、主に空気が読めない人を意味する。また、逆にそういった人に「空気を読め」と提言する際にも
\\	と耳元で囁くなどして使われる。女子高生がメールのやりとりで使い、普及した。 若者の間では以前から使われているが、2007年、こういった頭文字略語の存在が話題となり、広く知られる。 
\\	とは「危険」と「予知」の頭文字である。主に工場や現場作業の際、安全衛生・ゼロ災害運動のひとつとして
\\	活動(危険予知活動)という形で使用。	2007年	アルファベット略語 ・ 
\\	語、若者言葉(コギャル語)	
\\	けい ・ わい ・ あーる	場の雰囲気 ・ 状況を察することが出来る人のこと。	
\\	とは「空気読める
\\	の略で、場の雰囲気や状況が察知出来る人や察知出来ることをいう。こうしたことが出来ない人のことを
\\	(空気読めない)といい、2007年には世代を超え、幅広く使われたが、
\\	はその対語にあたる。ただし、
\\	のようには浸透していない。 
\\	など、こうしたアルファベット略語は若者の間で以前から使われているが、2007年の
\\	の普及を機に世代を超え、広く知られるようになった。	2007年	若者言葉	
\\	ケーチュー	けーちゅう	携帯電話に依存し、いつも触って手放せなくなっていること。	ケーチューとは「携帯電話中毒」の略で、椙山女学園大学人間関係学部人間関係学科の加藤主税教授による造語である。携帯電話依存とも言われる状態のことだが、加藤教授はケーチューの定義(チェックリスト)として、以下10項目を挙げている。 
\\	一人の時は、必ずケータイを見ている。 
\\	食事するとき、必ずケータイを見ている。 
\\	朝起きると、必ずケータイを見ている。 
\\	家の中でも、ケータイを持ち歩く。風呂やトイレにも持って行く。 
\\	いろいろな振動に、すぐ反応する。 
\\	圏外の場所には居たく(行きたく)ない。 
\\	メールをして、すぐ返事が来ないとイライラする。ずっと返事を待っている。 
\\	通話料金が月5万円を超える。 
\\	通話料金が払えなくて、使えなくなったことが何回もある。 
\\	メモリーに電話番号、アドレスが200人以上保存してある。	2007年	
\\	光合成	こうごうせい	日光に当たること。子作りをすること。	光合成とは本来、太陽光を用い、二酸化炭素
\\	と水
\\	から炭水化物
\\	を合成し、酸素
\\	を放出することをいう。ここから日光にあたることを光合成という。特に屋内から出ない生活が続いている人による、「最近、光合成していない」といった使い方が多い。 また、男性と女性が新たなものを作るという意味で、子作りを指して光合成という場合がある。子作り=性行為を意味することもあるが、この場合、あくまで子作りのための性交を指し、快楽だけを目的とする性交は含まない。	2007年	若者言葉	
\\	合ラン	ごうらん	合同ランニングの略。	合ランとは合同ランニングの略で、新しいスタイルの合コンのことである。まず、合コン同様、初対面の男女が集う。ここで合コンなら自己紹介をし、飲み会が進んでいくわけだが、合ランはその前にランニングをして汗をかき、銭湯で汗を流し、スッキリしてから飲み会を始める。他にも新たなスタイルの合コンはあるが、中でも合ランはランニングブームに乗って注目されている。 また、これとは別に合同ランチの略としても使われる。こちらの合ランはランチタイムを使ってコンパをしようというものである。	2007年	
\\	ご褒美貧乏	ごほうびびんぼう	自分へのご褒美によって金銭的に貧しくなること。	ご褒美貧乏とは自分へのご褒美(仕事で成果があった場合や自ら課した目標が達成出来たときなどに、自分自身にご褒美すること)によって、金銭的に貧しくなることをいう。ご褒美貧乏にも種類がある。一つは、ご褒美にブランド商品など高価なものを購入することで貧乏になるタイプ。もう一つは、「今日は頑張ったし、自分へのご褒美」と、目標をたてず、漠然と自分にご褒美するもので、不要に外食をしたり、衝動買いをするなど、自分へのご褒美という名目で財布の紐を緩めるタイプがある。	2007年	
\\	コム持ち	こむもち	ウィルコムの携帯端末を持っている人のこと。	コム持ちとは株式会社ウィルコム
\\	の携帯端末
\\	機)を持っている人のこと。ウィルコムでは
\\	への通話料(2009年現在、日本での
\\	はウィルコムのみのため、実質的にはウィルコム同士の通話料)を24時間無料(=基本料だけの音声通話定額制)とした。これが学生など若者の間で普及。出会い系など各種掲示板や
\\	の自己紹介でコム持ちと書き、「ウィルコムを持っている(=ウィルコムを持っていれば無料で電話出来ます)」という意思表示をする若者が増えている。	2007年	
\\	三手	さんて	独身女性がパートナーに望む三つの条件のこと。	三手とは『三高』や『三低』の派生語(それぞれの意味は下記関連語参照)で、女性が結婚において男性に求める三つの条件を意味する。そして、この三つの条件が「手伝う」「手をとりあう」「手をつなぐ」といった「手」という字で始まることから三手という。三高や三低はあくまで男性の特長など、女性は受動的であった。これに対し、三手は女性も能動的に関わる内容となっており、肉食女子といった流行語に見られる女性の意識変化が見られる条件となっている。	2007年	
\\	しー ・ あい ・ けい	場違いな人。場の雰囲気から浮いている人のこと。	
\\	とは「ちょっと痛い子
\\	の頭文字である。ここでいう「痛い」とは身体の痛みでなく、恥ずかしい ・ 場違い ・ みっともないといったニュアンスで使われている(詳しくは下記関連語参照)。つまり
\\	とは場違いな人や場の雰囲気から浮いてしまっている人を意味する。	2007年	若者言葉	
\\	しー ・ けい ・ わい	場の雰囲気や状況が全く察知出来ない人のこと。	
\\	とは「超空気読めない」の頭文字から成るアルファベット略語である。場の雰囲気や状況が読めない人や読めないことを意味する若者言葉
\\	(空気読めない)」が2007年には総理や政治を皮肉る言葉としてニュースにも出てくるようになった。
\\	はその派生語として出来た言葉である。ちなみに当初は「超
\\	という言い回しが使われていた。	2007年	若者言葉	
\\	じー ・ じぇい	労をねぎらう言葉。	
\\	とは「お疲れさま」「よく頑張った」といった意味の英語
\\	""の頭文字で、意味も英語同様である。
\\	のようなアルファベットの頭文字を使った略語は以前から使われていたが、2007年に
\\	が流行語となり、
\\	語ともいう。
\\	もこうした
\\	語のひとつとして紹介された。ただし、他の多くの
\\	語が日本語をローマ字読みした際の頭文字であり、その点は異なる。
\\	2007年	若者言葉	
\\	じぇい ・ けい	女子高校生のこと。	
\\	とは女子高校生
\\	または女子高生
\\	の頭文字である。
\\	など、様々な言葉を頭文字で語ることが若者の間で使われるが、
\\	もその一種である。2007年、こうした頭文字略語の存在が話題となり、広く普及(若者の間ではそれ以前から使われている)。 また、これとは別に大手掲示板サイト“2ちゃんねる”を中心するインターネット内では「常識的に考えて(→常考→
\\	という段階を経て出来た略)」という意味で使われることも多い。 ※他にも人名、社名などの略で使われることも多いので注意。	2007年	若者言葉、アルファベット略語 ・ 
\\	語	
\\	じぇい ・ しー	女子中学生のこと。	
\\	とは女子高生を意味する
\\	の類語で女子中学生
\\	の頭文字である。
\\	など、様々な言葉を頭文字で語ることが若者の間で使われるが、
\\	もその一種である。2007年、こうしたアルファベット略語の存在が話題となり、世代を超えて広く普及。 ※ただし、
\\	は日本中央競馬会のジャパンカップなど、様々な名称の頭文字として使われるため、文脈からの判断が必要。	2007年	若者言葉	
\\	十二月病	じゅうにがつびょう	クリスマス ・ イブに伴う精神的ストレスからくる症状のこと。	十二月病とは恋人のいない人がクリスマス・イブが近づくにつれて抱える精神的ストレス及び、こうしたストレスからくる症状のことである。12月は
\\	番組や
\\	、雑誌がクリスマスを扱い、街へ出ても多くのショップや家がクリスマスの飾りつけをする。恋人のいない男女にとってはこれらが孤独感を煽る形となり、十二月病の沈んだ気持ちにさせることになる。中にはこうした気分を引きづったまま、二月病に突入する者もある。	2007年	
\\	白点	しろてん	試験で0点またはそれに近い点数のこと。	白点の白とは答案用紙に何も書き込めず、白紙状態を意味する。つまり、白点とは試験で無回答の0点、またはそれに近い点数であったことを意味する。ちなみに落第点は赤点、落第点をギリギリ免れた点数を桃点という。赤と白を混ぜたものが桃色と考えると、白点は高得点でなければならないが、赤点 ・ 白点 ・ 桃点は各々別の由来からきた言葉なため、その通りにはなっていない。	2007年	若者言葉	
\\	ステキ女子	すてきじょし	容姿 ・ スタイルが良く、上品な女性のこと。	ステキ女子とは漫画 ・ ドラマ『ホタルノヒカリ』の中で、干物女に対し、理想的な女性を表すものとして使われた言葉である。特に2007年のドラマ化により、広く普及。干物女こと綾瀬はるか演じる雨宮螢が使った言葉であることから、女性から見たパーフェクトな女性像というニュアンスでステキ女子は使われる。具体的には、容姿やスタイルが良いのはもちろんのこと、仕事が出来ても嫌味がない。さりげない気配りが出来る。香水をつけなくてもイイ香りがするなどである。ただし、全てが完璧であるが故、男性からは逆に魅力に欠けるといった印象を与えることもある(ホタルノヒカリでもそういったキャラクター)。	2007年	若者言葉	
\\	すぱい	化粧をしていない状況に問題を感じた際、発する言葉。	
\\	とは本来、「密偵」や「間諜」、「スパイをする」、「見つけ出す」といった意味の英語だが、それとは別に「すっぴんヤバくね?
\\	のアルファベット略語として、すっぴん(化粧していない状態)に問題を感じた際に使われる。2007年、
\\	というアルファベット略語が普及する中、
\\	語
\\	式日本語)の一つとして普及。通常、
\\	語はアルファベット読みのため、エス ・ ピー ・ ワイと読むが、
\\	に関しては(カタカナ)英語読みのスパイで浸透している。	2007年	若者言葉	
\\	セレビッチ	せれびっち	上品かつ、たくましい女性のこと。	セレビッチとは優雅なお金持ちやお上品といった意味の和製英語『セレブ
\\	の略:下記関連語参照)』と、雌犬・あばずれといった意味の英語『ビッチ
\\	の合成語である。ただし、ここでのビッチは逞しい女性、強気な女性といったニュアンスで使われている。つまり、セレビッチとは上品かつ逞しい女性を意味する。セレビッチは漫画家の辛酸なめ子さんによる造語である。後に同名雑誌
\\	表記)が出版され、(好意的か、批判的かは別にして)若者の間で広く普及した。	2007年	合成語(和製英語)	
\\	タバい	たばい	「すごい」「魅力的」といった意味の言葉。=やばい	たばいとは“やばい”と同義に使われる言葉で、本来の「危ない」「怪しい ・ 胡散臭い」といった意味よりも、1990年代以降若者に使われている「すごい」「のめり込みそうなほど魅力的」といった肯定の意味での使用が多い。パソコンのキーボードで隣り同士にある
\\	と
\\	の打ち損じで出来た言葉なのか、「多(た)」と「やばい」の合成(たばいは“やばい”を強めた言葉として使用されるため)なのか、語源については定かでない(調査中)。ジャニーズの
\\	に所属する小山慶一郎と加藤成亮のユニット「コヤシゲ」が使ったことにより、ジャニーズファンを中心に広まっている言葉である。	2007年	若者言葉	
\\	だぶりゅ ・ けい	しらけること。	
\\	とは「しらける」という意味で都心部を中心に女子高生など若者の間で使われるアルファベット(頭文字)略語のひとつである。ただし、
\\	など他の頭文字による略語と違い、しらけるの頭文字
\\	ではなく、
\\	が使われている。これは、しらける(白ける)=白+蹴る=
\\	という流れで出来たため。	2007年	若者言葉	
\\	でぃー ・ けい	大事な場面で言葉を噛むこと。男子高校生のこと。	
\\	とは大事なところで噛む
\\	の頭文字からきた言葉で、感動的なシーンや告白の最中など、大切な場面で言葉(セリフ)を噛んでしまうこと、またそういった人のことをいう。 これとは別に、男子高校生
\\	のことも
\\	というが、先述の意味で使われることのほうが多い(文脈から推察が必要)。 
\\	は以前から使われていたが、2007年の
\\	の流行に伴い、同様のアルファベット略語
\\	語ともいう)のひとつとしてメディアやウェブで紹介された。	2007年	若者言葉、アルファベット略語 ・ 
\\	語	
\\	ディスる	でぃする	軽蔑し、罵り ・ けなすこと。	ディスるとは英語で「不~ ・ 非~」といった否定形にする接頭語“
\\	に動詞化する接尾辞「る」つけたもの(無礼 ・ 軽蔑という意味の
\\	からきたという説もある)で、
\\	で相手を貶す(けなす)行為をいう
\\	の本場アメリカではむしろディスることから生まれた音楽性という見方もある)。ヒップホップシーンではもちろん嫌いな相手を攻撃するためにもディスるが、自身の実力を見せ付けるためにディスったり、ディスり返したりすることもある。こうしたやりとりはエミネム主演の映画
\\	に見ることが出来る。	2007年	若者言葉	
\\	鉄子	てつこ	女性の鉄道ファンのこと。	鉄子とは菊池直恵(著)のマンガ『鉄子の旅 
\\	から来た言葉で、鉄道ファンの女性を意味する。『鉄子の旅』は2000~2006年にかけ、小学館の月間
\\	で連載されていたマンガで、作者の菊池直恵とトラベルライターの横見浩彦による鉄道紀行という設定。マンガの中の鉄子は女性の鉄道ファンを意味する造語で、特にそういった名前のキャラクターが設定されているわけではない。また、一般にはこうした意味以外に『鉄子の旅』の略称としても使われる。2007年6月から『鉄子の旅』が
\\	でアニメ化、鉄道アイドルと呼ばれ、『鉄子の旅』にも出演している豊岡真澄の存在もあり、話題にはなっているが、実際に鉄子と呼べる女性が増えているのかは定かでない。	2007年	
\\	どんだけー	どんだけー	突っ込みを入れる際に使う言葉。	どんだけーとは「どれだけ
\\	なのー」が略されたもので、新宿二丁目界隈のおかま(ゲイ)が使用していた言葉である。そして新宿二丁目で使われる光景がダウンタウン司会の番組『リンカーン』の世界ウルリン滞在記というコーナーで放送され普及(お笑いコンビ
\\	の藤本敏史が新宿二丁目のゲイのブラスバンドに体験入団中のワンシーン)。さらにメイクアップアーティストの
\\	が頻繁に使ったことから、若者を中心にツッコミを入れる際の言葉として広く浸透した。2007年春の月9ドラマ『プロポーズ大作戦』で主人公の山下智久が使ったことでも知られる。例えば、大量に食べている人に「どんだけ食うんだよ」と言ったような「どんだけ
\\	なんだよ」という言葉を略し、どんだけーと感嘆詞的に一言で突っ込みを入れたり、「うそー?」「マジー?」と同様のニュアンスで使ったり、様々な意味合いのツッコミ文句として使われる。どんだけぇ~の綴りで2007年流行語大賞トップテンに入賞。	2007年	若者言葉	
\\	ななけい	
\\	企業が就職で敬遠されるカ行で始まる7つの理由のこと。	1980年代終わり、学生が就職活動で敬遠する職場の理由「危険」「汚い」「きつい」をカ行で始まる三つの理由であることから3
\\	と呼び、流行語にもなった。その派生語で不人気な
\\	企業を意味する新3
\\	の汚いが帰れないに変更)という言葉が2005年以降使われるようになる。7
\\	とはそこへ更に「規則が厳しい」「休暇がとれない」「化粧がのらない」「結婚できない」という4項目が付け加えられたもので、新3
\\	同様不人気な
\\	企業を意味する言葉である。	2007年	
\\	ナンパ箱	なんぱばこ	ナンパ目的の人が多いとされるクラブ、ディスコのこと。	ナンパ箱とはナンパ(男性が女性に声をかけ、口説くこと)をする人が多い箱(イベントを行う施設 ・ 建物の意)のことだが、ここでいう箱とは特にクラブやディスコをさす。つまり、ナンパ箱とはナンパ目的の男性が多いクラブやディスコのことだが、クラブ ・ ディスコ側が特にナンパを奨励したり、ナンパしやすい雰囲気作りをしているわけでなく、自然とそうした人が増えたり、ナンパが成功の確立が高いといった噂からナンパ箱として知られることが多い。予定のわからない女性に闇雲に声をかける路上ナンパに比べ、クラブでのナンパは成功率が高いことから、クラブやディスコをナンパ箱として利用する若者が増えている。	2007年	若者言葉	
\\	二月病	にがつびょう	バレンタインに伴う精神的ストレスからくる症状のこと。	二月病とは五月病の派生種で、2月の恋愛イベント“バレンタインデー”に際し、チョコレートがもらえるあてのない男性や2月14日当日、実際にチョコレートがもらえなかった男性に見られる、孤独感など精神的ストレスからくる症状のこと。大抵は男性がかかりやすいが、中には本命の男性に彼女がいてチョコレートを渡せないなどの理由から女性がかかることもある。また、類似の十二月病が落ち着きかけたところで二月病になる人も多いという。	2007年	
\\	2娘1	にこいち	仲良し二人組のこと。	2娘1とは「二人でひとつ」ということを「ニコイチ(不具合のある2つの製品を使って1つのものを作ること:下記関連語参照)」にかけたもので、二人でひとつと言えるほど仲がよいという意味になる。仲のいい友達二人でツーショットプリクラを撮ったときのコメントなどに使われる。2娘1という当て字が使われてはいるが、必ずしも女性同士とは限らず、男女のカップルで用いることもある。また、人数が増えた場合、3娘1(さんこいち)、4娘1(よんこいち)とも言う。 一姫二太郎の逆バージョンという意味はないので、ご注意ください。	2007年	若者言葉	
\\	2個持ち	にこもち	同じものを2個持つこと。	2個持ちとは同じ物を持つことを意味する。ここでいう同じ物とは全く同一物、同メーカーの物に限らず、例えばポーチとトートバッグといったように用途別にバッグを2つ持つことを2個持ちという。バッグ以外ではメーカー違いの携帯電話を2台持ったり、
\\	とマッキントッシュの2台のパソコンを持つといった2個持ちが多い。	2007年	
\\	ネットカフェ難民	ねっとかふぇなんみん	インターネットカフェを宿代わりにしている人のこと。	ネットカフェ難民とは家賃が払えず住処を失い、インターネットカフェを宿代わりにして暮らす人のことである。日雇い労働(派遣)という雇用形態で働く若者に多く見られる。また、長時間勤務のため、自宅まで帰ると睡眠時間がなくなることから、職場近くのインターネットカフェで寝泊りする者もある。どちらにしても労働環境 ・ 雇用状況の悪さがもたらした産物であり、社会問題のひとつとして注目されている。また、ネットカフェ難民になる所持金もない人の中にはマクドナルドで仮眠するマック難民と呼ばれる人も増えている。 ※ネットカフェ難民は2007年流行語大賞トップテンに入賞。	2007年	
\\	寝モバ	ねもば	布団やベッドに寝転がってモバイルを使うこと。	寝モバとは「寝たままモバイル」の略である。モバイル
\\	とは直訳すると可動 ・ 移動だが、ここではノートパソコンや携帯電話、
\\	(情報携帯端末)など持ち歩き可能なモバイル端末を指す。本来外出時に持ち歩くために軽量小型化されたモバイル端末だが、これを活用し、布団の中でゴロゴロとモバイル端末を使って遊ぶのが寝モバである。携帯電話で閲覧 ・ 利用可能なウェブサービスが増えており、これに伴って寝モバ人口も増加しているとされる。	2007年	
\\	脳筋	のうきん	体力はあるが頭は悪そうな人のこと。(筋肉バカ)	脳筋とは「脳みそまで筋肉(で出来ている)」の略で、体は鍛え上げられたマッチョな体型をしているが、頭は悪そうな人を嘲う言葉である。何か事を成すときに作戦を練ったり、計画をたてるといった頭を使うことはせず、猪突猛進、体力任せで挑むような人。腕力勝負の喧嘩では負け知らずだが、戦国時代の軍師のような策士にかかると簡単に負けてしまうような人を脳筋という。2007年には脳トレが流行り、脳トレで鍛えられた頭の良い人というイメージを持たれがちだが、実際はその対に位置する人を指す言葉ということになる。(ただし「脳筋を鍛える」というように、脳の筋肉という意味で使われている場合もあります)	2007年	
\\	バメン	ばめん	気分次第、状況次第といった意味の若者言葉。	バメンとは場面のことである。この場面は光景や情景(シーン)以外に「人の個々の行為を成り立たせている環境 ・ 状況(参考資料:広辞苑)」といった意味もある。ここから気分次第 ・ 状況次第のことを若者の間ではバメンという。未来のことでどうするか質問された際に「気分次第で」と返答する代わりに「バメンで」と言う(下記使用例参考)。 また、これらとは別にメンバーの倒語(業界用語によく見られるもので、前後を逆にした言葉。うまい→まいうー が有名)として使われる場合もある。	2007年	若者言葉	
\\	ぴー ・ けい	頭文字として複数の意味を持つ言葉。	
\\	とは「パンツ食い込んでる(くいこんでる)」の略。2007年、
\\	を筆頭に若者の間で重用されているアルファベット略語のひとつ
\\	語 ・ 
\\	式日本語ともいう)。スカート、ズボンのシワなどからパンツ(下着)が食い込んでいると予測出来る人、またはズボン(=パンツ)が食い込んでいる人に対して使う。 
\\	とはパソコンソフトやゲームソフトにおけるパワーアップ ・ キットの頭文字。主にシュミレーションゲームに見られる。
\\	ともいう。 
\\	とはプレイヤーキラーの頭文字。複数参加型のオンラインゲームにおいて、本来、共に戦うべき他の参加プレイヤーを攻撃して倒す人のこと。他の参加プレイヤーの武器や所持品を奪う目的で行われることが多い。	2007年	若者言葉、アルファベット略語 ・ 
\\	語	
\\	ぴー ・ てぃー	頭文字として複数の意味を持つ言葉。	
\\	とは「パンを食べる
\\	のアルファベット略語
\\	語ともいう)である。2007年の
\\	語ブームでは、こちらの意味で使われることが多い。 
\\	とは「プリクラを撮る
\\	のアルファベット略語
\\	語)である。ただし、
\\	に比べると、こちらの意味での使用は少ない。 
\\	は上記以外にも「プロジェクト・チーム
\\	「ポイント
\\	など、さまざまな意味で使われるため、文脈から意味を汲み取る必要がある。	2007年	若者言葉	
\\	ヒガモ	ひがも	被害妄想や被害妄想癖がある人のこと。	ヒガモとは被害妄想(ひがいもうそう)の略で、文字通り被害妄想(下記※印参照)を意味する。また、被害妄想癖のある人や被害妄想に陥っている最中の人を指して使う。2007年11月にフジテレビのバラエティー番組『はねるのトびら』の短縮鉄道というコーナーで出題され、広く認知される。 ※被害妄想とは:他人から害を加えられる、苦しめられるなど、被害をこうむると信じる妄想。(広辞苑第五版より)	2007年	若者言葉 ・ 略語	
\\	ビる	びる	
\\	ライブサーチを使って検索すること。	ビるとはマイクロソフト社創業者であるビル ・ ゲイツ
\\	と動詞化する接尾辞『る』を掛けたもので、マイクロソフト社が提供する検索機能“
\\	(ライブサーチ)”を使って検索することを意味する。これは2007年に発売された人気ソフト「初音ミク」。この初音ミクに関連の画像が、
\\	や
\\	の画像検索結果ではほとんど現れないのに対し、
\\	での検索結果では現れたことから普及した。これまで、ビる以外にも
\\	る」「ライブサーチる」といった類語もあったが、「ググる」や「ヤフる」のようには浸透していなかった。	2007年	
\\	昼シャン	ひるしゃん	
\\	昼の休憩時間に近くの美容室で洗髪すること。 
\\	昼の休憩時間にシャンパンを飲むこと。	
\\	昼シャンとは
\\	が昼休みに会社近辺の美容室で洗髪することである。朝シャンが流行した後の1998年に話題となった。仕事が終わってからデートなどの用事がある日に昼シャンを行う
\\	が増え、昼休みの時間帯に割引をしたり、ランチを出す美容室も出て話題になった。 
\\	昼シャンとは昼間にシャンパンを飲むことである。(株)マチュアライフ研究所代表取締役:渋井真帆が発案した言葉で、上司に怒られたときなどブルーな気分になった際、気分転換のための昼シャンを提唱。ちょっとした悪いことをする爽快感、またビールよりもグレードの高いシャンパンを薦めている。2007年現在、六本木ヒルズ近辺のレストランではランチにシャンパンを出す店もある。	2007年	
\\	ピン写	ぴんしゃ	一人で写真に写ること。一人で写った写真。	ピン写とはポルトガル語で点を意味する“
\\	と写真から成る合成語である。ピンは点の他、カルタやサイコロの目の
\\	、さらに独り者、単独といった意味でも使われる(ピン芸人のピンなど)。ピン写の「ピン」は一人という意味からきたもので、一人で写真に写ることや、一人で写っている写真を意味する。カメラや写真の形態は特に関係なく、アナログ、デジタル、携帯写真で撮られた全ての写真を対象とするが、携帯写真に限り、ピン写メともいう。	2007年	若者言葉	
\\	ピンプリ	ぴんぷり	一人でプリクラに写ること。一人で写ったプリクラ。	ピンプリとはポルトガル語で点を意味する“
\\	と、ゲームセンターなどで写真を撮り、シール印刷してくれる機械“プリント倶楽部(通称:プリクラ)”からくる言葉である。ピンは点の他、カルタやサイコロの目の
\\	、さらに独り者、単独といった意味でも使われる(ピン芸人のピンなど)。ピンプリの「ピン」は一人という意味からきたもので、一人でプリクラに写ることや、一人で写っているプリクラ画像を意味する。	2007年	若者言葉	
\\	ブスドル	ぶすどる	ブサイクなアイドル(タレント)のこと。	ブスドルとは「ブス風アイドル(ブサイクなアイドル)」の略で、インターネットを中心に話題となっている森三中 ・ 大島似の藤子まい(16歳)を写真週刊誌
\\	が取り上げる際に形容した言葉である。2007年3月現在、ブスドルは小田切まいを形容する肩書きとなっているが、松田聖子を形容する言葉であったママドルが次第に子を持つアイドル全般に使われるようになったと同様、ブスドルも今後森三中などブスを売りとしているタレント全般を形容する言葉になる可能性もある。	2007年	若者言葉	
\\	フニーター	ふにーたー	最低限の労働しかしないフリーターのこと。	フニーターとはアルバイトや派遣社員など、非正社員雇用で生計を立てている人を意味するフリーターと、職に就かず、就職活動もしていない人を意味するニートの合成語である。具体的な定義は特にないが、ニートとフリーターの中間的存在、またはニートに限りなく近いフリーターという二つの意味で使われている。ただし、仕事をすればニートでなく、仕事をしなければフリーターではなくなるといった相反する言葉の合成語であり、中間的な人というのは考え難いため、後者の最低限必要な労働しかしないフリーターが自然である。	2007年	若者言葉	
\\	フルボッコ	ふるぼっこ	徹底的に殴りつける様を表す言葉。	フルボッコのフルとはフルパワーの略で「力の限り」「全力で」といった意味になる。ボッコはひどく殴りつける様を表すボコボコの語感を強めたボッコボコの略である。つまり、フルボッコで力の限り殴りつける様=徹底的に殴りつける様を表す。実際のケンカの様子を現す際に使われる他、格闘技に関するブログ記事や格闘ゲームなどに見ることが出来る(=「フルボッコにしてやんよ」)。 また、これとは別に、「(古くて)壊れそう」という意味の『ぼっこ(ぼっこいの略)』に『古い』をつけ、古くておんぼろ(ボロボロ)なものを指して使うこともある。	2007年	不良用語~若者言葉	
\\	幕の内	まくのうち	内容が多く、わかりづらいメールや話しのこと。	幕の内とは一通のメールや一度の話しに内容が複数入っていることで、これらを一つの弁当箱に多種のおかずが入った『幕の内弁当』に例えたものである。もともとは幕の内メールといい、一通に複数の内容が書かれていて、何が言いたいのかわからないメールを意味した。これが幕の内と簡略化され、メール以外に、一度に多くの内容が盛り込まれた朝礼や発表など、話しに対しても使用。どちらにしても「わかり難い」といった悪い意味合いで使われる。	2007年	若者言葉	
\\	マック難民	まっくなんみん	インターネットカフェを宿代わりにしている人のこと。	ネットカフェ難民とは家賃が払えず住処を失い、インターネットカフェを宿代わりにして暮らす人のことである。日雇い労働(派遣)という雇用形態で働く若者に多く見られる。また、長時間勤務のため、自宅まで帰ると睡眠時間がなくなることから、職場近くのインターネットカフェで寝泊りする者もある。どちらにしても労働環境 ・ 雇用状況の悪さがもたらした産物であり、社会問題のひとつとして注目されている。また、ネットカフェ難民になる所持金もない人の中にはマクドナルドで仮眠するマック難民と呼ばれる人も増えている。 ※ネットカフェ難民は2007年流行語大賞トップテンに入賞。	2007年	
\\	ミテコ	みてこ	18歳未満のこと。	ミテコとは「身分証明が提出出来ない子」の略で、18歳未満または未成年という意味である。ミテコはスナックやクラブなどの水商売、さらに風俗業界といった、18歳未満の者が客および従業員として入店出来ない業界での業界用語である。主に後者の(女性)従業員の年齢に関する話題で使われる。店側だけでなく、客が「この店、ミテコが混じってない?」といった使い方もする。	2007年	略語	
\\	むちゃぶり	むちゃぶり	漫才、トーク番組において、会話やネタを相手(他人)任せにし、強引に振ること。	むちゃぶりとはもともと漫才で使われる振りの一種。通常、漫才では仕込んだ話しの流れに沿ってネタを振り、ボケやツッコミを行う。これに対し、仕込んだネタとは全く違う流れに話しを広げ、(収拾がつかなくなった状態で)オチを相方に強引に振ることをむちゃぶりという。後にトーク番組でも、トークに行き詰まった際に、前後の脈略なく、他人任せで会話を振ることなど、他者に強引に会話やネタを振ることをいう。2007年には、くりぃむしちゅーの有田哲平がレギュラー出演するバラエティ番組「むちゃぶり!」や関ジャニ
\\	がレギュラー出演する番組「おもてなし音楽バラエティむちゃ
\\	ブリ(こちらはゲストに無茶なお願いをするというコンセプトであり、ここでいうむちゃぶりとは意味が異なる)」が放送開始。一般にも普及。	2007年	
\\	モテプヨ	もてぷよ	程好い肉付きでふっくらした女性のこと。	もてぷよとは異性から人気があるという意味の「モテる」と、体に脂肪が付いたさまを表す「ぷよぷよ」から成る言葉で、痩せすぎず、ふっくらした体型を指す。もてぷよは女性向けファッション誌
\\	による造語である。
\\	は主に40代女性をターゲットにした雑誌だが、そういった層に向け、ふっくらした体型を生かしたモテるファッションの提案としてもてぷよを使い始めている。あくまでモテるぷよぷよであり、ぷよぷよはモテるという意味ではない。また、顔がかわいいぽっちゃり体型をもてぷよ、そうでない場合をデブとする意見もある。お笑いコンビ“インパルス”の堤下敦が
\\	モデルの土岐田麗子から告白された際、もてぷよは女性だけでなく、男性にも及ぶとして話題になった。	2007年	合成語	
\\	もののけ	もののけ	非常に不細工な人のこと。(俗語的解釈)	もののけとは不細工な人や雰囲気が良くなかったり、暗い人を意味する。もののけとは本来、死霊 ・ 生霊 ・ 邪気及びそれらが祟る(たたる)ことを意味する。ここから、それら霊や霊の祟りに匹敵するほど不細工 ・ 暗いというニュアンスで用いられるようになった。主にコンパやオフ会など、初対面の人が多い場所で使用。女性が不細工な男性に、男性が不細工な女性に、双方に用いられる。	2007年	
\\	桃点	ももてん	落第点に近い点数のこと。	桃点とは試験の落第点を意味する赤点の派生語で、桃色は赤色を薄めた(白を混ぜた)色である。つまり、桃点とは落第点をギリギリクリアした赤点寸前の点数を意味する。	2007年	若者言葉	
\\	安カワ	やすかわ	「安くてかわいい」「安っぽいがかわいい」の略。	安カワとは「(価格が)安い」「(質が)安っぽい」の安(やす)と「可愛い」から成る言葉だが、男性が使う場合と女性が使う場合で若干意味合いが異なる。 まず、女性が使う安カワは洋服やアクセサリー、小物といった物を対象とし、「安くてカワイイ」「安いけどカワイイ」といった値段の割りに可愛いさまを表し、「安カワ
\\	「安カワアクセ」といった形でも使う。 男性は主に女性を対象に「安っぽいけど可愛い」といった意味合いで使用。可愛いで締める誉め言葉にも聞こえるが、中には「安い扱いの女」という意を含む場合もある。そもそも本命の女性に「安っぽい」といった意を含む安カワを誉め言葉としては使わない。	2007年	若者言葉	
\\	ヤンデレ	やんでれ	心が病んだヒロイン(2次元キャラクター)のこと。	ヤンデレとは「病み」と「デレデレ」から成る合成語で、アニメ、ゲームなどの2次元キャラクターの中で、精神的に病んだキャラクターが他のキャラクターに対して過激な愛情表現をし、異常な行動をしてしまう様および、そういったキャラクターのこと。ただし、これは病み鍋パーティーというサイトで打ち出された基本定義であり、2007年現在、ヤンデレという言葉を発する人によって流動的に定義が変化している状態にある。また、現時点ではアニメやゲームといった2次元キャラクターの萌え属性の一種であり、実在する女性を対象とする言葉にはなっていない。 また、上記とは別に「ややツンデレ」というニュアンスで使う場合もある。	2007年	若者言葉	
\\	リア充	りあじゅう	現実の生活が充実していること。またはそういった人。	リア充の『リア』はリアルの略で、インターネット上のコミュニティに対する現実の生活を意味し、『充』は充実を略したもの。つまり、リア充とはリアル(=現実の生活)が充実していること。または充実している人を意味する大手掲示板サイト“2ちゃんねる”で生まれた言葉である(この場合の充実とは、男女交際や友人など、主に人間関係の充実を指す)。携帯電話によるインターネット接続の普及から、現実生活が充実しつつ、インターネットのコミュニティにも参加するという者が増えた。こういった人に対し、以前からインターネットコミュニティ(チャットや
\\	、掲示板など)に入り浸り、現実生活が充実していない者が嫉妬・皮肉(自虐)を込めた呼び方としてリア充は使われた。	2007年	ネットスラング	
\\	ルーってる	るーってる	言動や雰囲気がルー大柴に似た人のこと。	ルーってるとはタレントの「ルー大柴」と、何かに似ていることを意味する「~入ってる」から成る合成語で、ルー大柴の独特なキャラクターに似た人を意味する。具体的には態度・仕草が大袈裟な人、くどい人。会話の中でやたら英単語を使う人がこれにあたる。 ルーってるは1990年代前半、ルー大柴の最初のブレイク時に大学生の間で普及。ルー大柴のメディアでの活動が減るのに伴って死語となるが、2007年の書籍の出版やメディアでの再ブレイクから再び使われるようになった。	2007年	若者言葉	
\\	ワンコールワーカー	わんこーるわーかー	一日または短期の単位での派遣労働者のこと。	ワンコールワーカーとは短期間(基本的に1日単位)の仕事を請け負う派遣労働者を意味する。こうした派遣労働者は派遣会社から電話やメールによって仕事の指示を受け、派遣先に出向いて働くことから、ワンコールワーカー
\\	和製英語)と呼ばれるようになる。こうしたワンコールワーカーは労働派遣法が改正され、人材派遣業が介在出来る範囲が広がった2004年から急増。労働条件の悪さや収入の不安定さが取り上げられ、ネットカフェ難民とともに社会問題の一例として使われることが多い。	2007年	和製英語	
\\	ヲタギャ	をたぎゃ	オタクなギャル。またはオタクなバンギャ。	ヲタギャとは『ヲタク(オタク)』と『ギャル』から成る合成語で、ギャル・ファッションに身を包むギャルを意味する。また、ヲタギャはヴィジュアル系バンドが好きな女性を意味する『バンギャ』のもじりとして出来た言葉であり、ヴィジュアル系バンドが好きでオタクな女性という意味でも使われる。	2007年	合成語	
\\	アザス	あざす	「ありがとうございます」を意味する若者言葉。	アザスとは人気お笑いコンビ“アンタッチャブル”の決まり文句『あざーっす』が簡略化されたものである。あざーっすは「ありがとうございます」が崩れたもので、アザスも同様の意味で若者が使用。ブログや携帯メールなど文字による使用が多い。 アザスといえば・・・ アザスといえば、元となる「あざーっす」を普及させたお笑いコンビ“アンタッチャブル”でしょう。 人力舎のタレント養成所『スクール
\\	の同期生、山崎弘也(1976年生まれ)と柴田英嗣(1975年生まれ)によるコンビで1994年に結成。2003年の
\\	-1グランプリの好成績、更に翌年の
\\	-1グランプリの優勝を機にブレイク。テンポの早いボケ ・ ツッコミ漫才が特徴、漫才以外のトークも冴えていることから『伊東家の食卓』など多くのバラエティ番組 ・ クイズ番組に出演してる。	2006年	
\\	アスドル	あすどる	スポーツ界でアイドル的存在の女性のこと。	ジドルとは「アスリート ・ アイドル」の略で、スポーツ選手の中でアイドル的存在の女性を意味する。アスドルと呼ばれるには容姿だけでなく、スポーツ選手としての実績 ・ 実力があることが前提となる。主なアスドルに空手の小林由佳、ビーチバレーの浅尾美和、モーグルの上村愛子がいる(2007年現在)。	2006年	若者言葉、略語	
\\	アラサー	あらさー	30歳前後の女性のこと。	アラサーとは「アラウンドサーティー
\\	の略で、30歳前後(周辺)の女性を意味し、アラサー世代ともいう。アラサーという言葉はアパレル業界(ファッション業界)が生んだ造語で、2006年秋からの婦人服新ブランドのターゲット層として位置づけたものである。2006年現在30歳前後の女性といえば、ルーズソックス、茶髪、プリクラといった流行を次々と作り上げたコギャル世代にあたる。そんなコギャルが大人になったアラサーは独自の発想でさまざまな組み合わせの着こなしをし、かつ消費サイクルも早いことから大手アパレルメーカーを中心に注目されている。	2006年	和製英語、美容 ・ ファッション用語、略語	
\\	ウーバーセクシャル	うーばーせくしゃる	マッチョかつ、知的で美しい男性のこと。	ウーバーセクシャル
\\	とは英語の“
\\	にあたるドイツ語“
\\	と英新語“
\\	から成る造語で、メトロセクシャルの美意識とマッチョの筋肉質で野生的かつ自信に満ち溢れた姿、さらに政治やビジネスにも詳しい知的さを兼ね備えた男性を意味する。ワイルドなステージを行いながら、詞や発言では繊細な表現をし、かつ政治的活動もしている
\\	2のボーカリスト:ボノや、マッチョな俳優として登場し、後に政治の世界で活動するアーノルド・シュワルツネッガーがウーバーセクシュアルの例として挙げられている。2006年現在、日本では馴染みの浅い言葉だが、一部雑誌などではメトロセクシャルの進化形として注目されている。	2006年	カタカナ語	
\\	ウケミン ・ 受け民	うけみん	受け身な人のこと。	ウケミンとは恋愛において受け身な人のことである。ウケミンにはトラウマ型ウケミン、プライド型ウケミンなど、ウケミンになる要因から様々なタイプがあり、自分からは何もしない人から相手の好意を確認するまで動かない人までランク付けもされている。 ※ウケミンはこういった意味でネット内で使われている言葉だが、今後普及する中で恋愛以外に使われるなど、意味の拡大の可能性があります(2006年7月現在)。	2006年	ネットスラング	
\\	うちエコ	うちえこ	環境省が提唱する地球温暖化防止策のひとつ。	うちエコとは家を意味する『うち』と自然環境(保護運動)を意味する『エコロジー
\\	から成る造語で、各家庭における衣食住を通して二酸化炭素の削減を求める環境省の地球温暖化防止策である。これまでクールビズ、ウォームビズと職場向けの提案を行ってきていたが、それを家庭レベルで提唱したものがうちエコである。衣食住それぞれに提案がされており、衣では家着・普段着でのウォームビズの取り入れ、食ではサラダを温野菜サラダにするなど体が温まる料理の提案や調理時の無駄なエネルギー使用を抑えるよう促している。住では使わない電気製品のコンセントを外したり、買い替え時には省エネ製品を選ぶよう謳われている。	2006年	カタカナ語	
\\	裏友	うらとも	顔を合わせたことのない友達。または学校 ・ 職場以外の仲間のこと。	
\\	裏友とは実際に顔を合わせることのない友達のことである。この定義から考えるとペンパル(文通友達)も含まれるが、基本的にはメールやメッセンジャー、チャットなどインターネットを介してやりとりする友達をいう。 
\\	裏友とは学校や職場、近所の仲間と違い、普段の生活では顔を合わせない友達をいう。
\\	と異なるのは、実際に会って遊ぶ仲間であっても、日常的に会うことがなければ裏友と呼べる。例えば、知り合ったキッカケはインターネットだが、オフ会などを通じ、会って遊ぶようになった友人、好きなアイドルのファンクラブで知り合って遊ぶようになった友人がこれにあたる。	2006年	若者言葉	
\\	駅モニ	えきもに	駅構内や周辺の店で朝食を摂ること。	駅モニとは
\\	東日本フードビジネス(株)による造語で「駅ナカ ・ モーニング(駅構内 ・ 周辺でモーニング)」の略である。朝食を摂らずに出勤する人、会社についてからおにぎりやパンを慌しく食べるといった層をターゲットに同社が駅構内及び周辺のコーヒーショップの朝食セットを強化したり、開店時間を早めるといった策を打った。その際に造った言葉である。	2006年	
\\	エコ買い	えこがい	賞味期限の古い商品を選んで買うこと。	エコ買いとはエコロジー(生態学 ・ 地球規模の環境を考えた社会活動 ・ 経済活動)に基づく買い物のことで、家庭にある食材を古い物から順に使うように、お店で購入する際も古い物から購入し、廃棄することがないようにしようというものである。エコ買いはジェイアール東海エージェンシーの畑中大平による造語で、2006年新聞広告クリエーティブコンテストで最優秀賞受賞作品のタイトルである。同広告には「賢い主婦はスーパーで手前に並んでいる古い牛乳を買う。」というキャッチコピーが書かれた牛乳パックの絵と「自宅の冷蔵庫に新しい牛乳と古い牛乳があれば、どちらから飲みますか?古い牛乳からですよね。賞味期限が過ぎて、棄ててしまうのがもったいないですから。しかし、スーパーでは新しい牛乳を選んで買っていませんか?新しい牛乳から売れていくと、そのぶん古い牛乳は売れ残ってしまいます。日本では毎日約2000万人分の食料が、賞味期限切れなどの理由で棄てられています。できるだけ、売り場の手前にある古い牛乳を買いましょう。飽食や贅沢を見直すことで、食料輸送や焼却処分時の環境負荷を減らすことができます。ムダを減らして、
\\	2排出量を減らしましょう」といった文が書かれていた。	2006年	
\\	エコロジジィ	えころじじぃ	環境問題に関して口うるさく説く高齢男性のこと。	エコロジジィとは自然環境(保護運動)を意味する『エコロジー
\\	と男性高齢者を意味する俗語『ジジィ』から成る合成語で、環境問題を説く高齢者を意味する。ただし、エコロジジィには「説教くさい」「若い頃に好き勝手したことは棚に上げ、自分が落ち着く年齢になったら環境問題など言い出す人」といった嫌悪や嘲う意を伴い、そういった高齢者をたたえたり、尊敬の念を抱いては使われるものではない。	2006年	若者言葉、合成語	
\\	エハラー	えはらー	スピリチュアルカウンセラー江原啓之を信奉する人のこと。	"エハラーとはスピリチュアルカウンセラー江原啓之の苗字に「~する人」という意味にする英接尾語
\\	""をつけたもので、江原啓之を信奉する人を意味する。江原啓之氏はスピリチュアルカウンセリング活動(スピリチュアルカウンセラーとは俗にいう霊能者)を行う他、2001年に発売された著書『幸運を引きよせるスピリチュアル・ブック』がベストセラーとなり、同時にラジオ、
\\	出演と活動の幅を広げている。特に2005年から放送されている美輪明宏とのダブルカウンセラーによる
\\	番組『オーラの泉』が多くの人から支持され、エハラーという言葉が使われるようになる。なお、エハラーは2006年、流行語大賞にノミネートされている(受賞はしていない)。
\\	2006年	若者言葉	
\\	1世代	えふわんせだい	20~34歳の女性のこと。	
\\	1世代とはマーケティングで使われる言葉で、20~34歳の女性を意味する
\\	1層ともいう)。ちなみに
\\	1世代の
\\	は英語で「女性の」という意味の形容詞“
\\	の頭文字。1は20~34歳という世代を表すもので、次の35~49歳が2となる。例えば、35~49歳の男性なら
\\	2世代
\\	2層)と呼ぶ。 中でも
\\	1世代が話題となっているのは、この世代独特のセンスや口コミ、ネットによる情報伝達の早さによるもので、この世代に受け入れられるかどうかが商品ヒットの鍵となるトレンドリーダーとして注目されているためである。	2006年	美容 ・ ファッション用語	
\\	炎上	えんじょう	ブログのコメント欄に記事及びその投稿者への意見が殺到すること。	炎上とは本来、火が燃え上がること(特に大きな建物が火事で燃え上がること)をいうが、これとは別にブログ
\\	で投稿した記事や投稿者に対し、閲覧者のコメントが殺到するさまを炎上という。また、そのコメントの多くは反対意見や誹謗中傷が多く、中にはコメント投稿が集中し過ぎ、ブログが機能停止となることもある。こうした現象は著名人のブログにのみ起こるわけではなく、普段はアクセスがないような無名個人のブログであっても、記事での発言が問題となり、他のブログや掲示板で問題提議され、炎上することもある。	2006年	ネットスラング	
\\	おサセ	おさせ	ふしだらな女性のこと。	おサセとは誰にでも簡単に肉体関係を許してしまうふしだらな女性のことで、同じ意味で使われる「サセ子(すぐにサセてくれる子の意)」の派生語にあたる(サセ子に尊敬の意を表す接頭の「お」をつけたものなのか、サセ子と“おませ”の合成語なのかは不明)。 2006年、『ダウンタウンのガキの使いやあらへんで!!』の中で、演出家の斉藤敏豪こと通称ヘイポーが整形外科医兼タレントである西川史子に「とんでもないおサセがやって来た」と発言したことで広く浸透(特にここで生まれたわけではなく、それ以前から書籍タイトルに使われたりもした言葉である)。	2006年	
\\	大人食い	おとなぐい	低価格商品を豊かな経済力で大量購入すること。	大人買いとは幼少期に出来なかった購入(コレクション)に対する夢を、大人になり、経済力がついてから果たす購入の仕方をいう。具体的には玩具付お菓子のケース買いやマンガを全巻一度に購入などがある。類似行為に『まとめ買い』があるが『まとめ買い』は一度に大量購入することで安価に入手したり、購入回数を減らすことで時間を節約するといった目的があり、大人買いとは意味合いが異なる。大人買いの発祥はトレーディングカードを収集するコレクター間という説が濃厚。オマケ付食品の流行とともに大人買いという言葉も一般的に普及する。更にケース売りの商品や大量数のパッケージ商品を『大人買いセット』とネーミングして販売するメーカーもある。子供が大量購入する場合でも代金を支払うのが大人(親など)であれば大人買いと呼ぶ。	2006年	若者言葉	
\\	大人ニート	おとなにーと	25歳から34歳までのニートのこと。	ニートとは15歳から34歳までの仕事をしておらず、教育も職業訓練も受けていない者のことだが、その中でも25歳から34歳までを特に大人ニートという。2006年3月に発表された「厚生労働省の2005年のまとめ」を受けて出来た言葉で、その内容とは、ニート全体の数は64万人台と4年連続で横ばいだったが25~34歳に関しては増加、全体の6割強を占めたというものだ。ちなみに横ばいだった4年間に大人ニートの数は4万人増加、その分15~24歳のニートが減少している。景気回復により、新卒者の就職状況はよくなったが、バブル後の不況の中でニートになった人はそのまま歳を重ね、ニートの高齢化を招いたといわれている。	2006年	
\\	鬼買い	おにがい	狂ったように買うこと。	鬼買いとは「鬼のように買いあさる(買いまくる)」、つまり狂ったように買いまくったり、尋常じゃない量や種類を買うことをいう(鬼~=激しく~、異常なほど~といった意味で使われる)。当初、若者の間で使われた鬼買いだが、次第に世代を超えて普及。鬼買いに値する購入機会が多いためか、トレーダー(株売買をする人)による、ブログ日記での使用も多い。	2006年	
\\	オレオレ症候群	おれおれしょうこうぐん	自己中心的な会話をする人のこと。	オレオレ症候群とは「オレは」「オレなら」と会話の主語に一人称が多く、自分が中心となる会話ばかりしてしまう人のことをさす。また、相手の話しを遮り、自分の側へ会話をもっていこうとするタイプもオレオレ症候群にあたる。オレオレ症候群は症状が重いほど自覚症状がなく、当人に悪気がないため、周りも注意しずらいことから更に症状を悪化させるという。対処法は、まず自覚症状を持つこと。自分がオレオレ症候群だと気付いたら、会話に「なるほど」「それで?」といった相槌や漫才でよく使われるオウム返しを多用し、相手の話を膨らませるクセをつけることで回避できる。	2006年	
\\	がばい	がばい	「凄い」を意味する形容詞(本来の佐賀弁とは若干意味が異なる)	がばいとは「とても」「大変」という意味の佐賀弁で、「がばいうまか(とてもおいしい)」「がばいきつか(大変疲れた)」といった使い方をする。このがばいを使った島田洋七(漫才コンビ
\\	の一人)の小説(自伝小説)「佐賀のがばいばあちゃん」のヒットから「凄い」という意味の形容詞として全国的に普及(ただし、本来の佐賀弁では名詞を修飾する使い方はしない)。自費出版でスタートした小説は2004年に徳間出版から再出版。現在も発行部数を伸ばしており、シリーズトータルでは500万部を超えている。また、2006年には映画化され、こちらもヒット。2007年初めにドラマ化。夏には高校野球で佐賀県立佐賀北高等学校が優勝し、“がばい旋風”という言葉も使われた。なお、2007年の新語・流行語大賞にノミネートもされている(受賞は逃している)。	2006年	
\\	壁本	かべぼん	つまらない本のこと。	壁本とは「読後、壁に叩きつけたくなる本」の略で、内容に納得いかなかったり、つまらないといった理由で壁に投げつけたくなるような本を意味する。特にどういった本という定義はないが、ミステリーや推理小説のトリックによる結末のつまらなさが要因となることが多い。また、一般的な書評や他者の感想に関係なく、自身が壁に投げたくなるほどつまらないと感じれば壁本となる。	2006年	
\\	ガリマッチョ	がりまっちょ	着痩せするが、脱ぐと筋肉質な人。	ガリマッチョとは体が細く、服を着ていると単に痩せているように見えるが、実際は筋肉質で引き締まった体をした人のこと。服を着ているとガリガリ(体が痩せているさまを表す言葉)だが、脱ぐとマッチョという意味合いで使われる。 2006年にフジテレビの人気番組
\\	(スマスマ)』のワンコーナー「おにいさんといっしょ(中居正広がうざみちお兄さんという体操インストラクター佐藤弘道(通称:ひろみちお兄さん)に扮するもの)」の中で行った同コーナー用創作体操「ガリマッチョ体操」の歌詞 ・ 内容が話題となった。	2006年	若者言葉	
\\	キモカワイイ	きもかわいい	見た目的によくはないがかわいいこと。	キモカワイイとは見た目的に気持ち悪い ・ よくない(=キモイ)が、かわいらしさを感じさせるなど、「キモイけどカワイイ」または「キモくてカワイイ」といった意味の感嘆詞である。キモカワイイはお笑いコンビ“アンガールズ”の代名詞的にも使われており、「キモカワ~」と略しても使われる。2005年辺りからこうした一見共通点のない言葉や対になる言葉を繋げる表現が増えているが、一説には物の多様化に伴う表現の拡大とも言われている。 なお、1990年代後半には既に一部の若者が使い始めている。	2006年	若者言葉	
\\	ギャルサー	ぎゃるさー	ギャルサークルの略。	ギャルサーとはギャルサークルの略でギャルが集うサークルのことである。ギャルサーにはサロンなどで焼いた色黒ギャルだけが集うものやエリア別のギャルサー、以前ギャルだったギャルママによるものまで様々なタイプがある。有名なギャルサーになると携帯メールによる試験(学校や会社でいう筆記試験)と面接を経て初めて加入出来るものや後輩に対しても敬語を使う規律の厳しいギャルサーも存在する。 ギャルサーは2005年からメディアでも取り上げられ一般にも知られる存在になったが、2006年4月から日テレで放送された連ドラ『ギャルサー』で一気に浸透。同ドラマにはギャルサーに対峙する米国帰りのカウボーイ役として藤木直人、ギャルサーのメンバー役として矢口真里、戸田恵梨香、鈴木えみ、新垣結衣、岩佐真悠子、佐津川愛美が出演。	2006年	若者言葉 ・ 略語	
\\	キュン死に	きゅんじに	胸が締め付けられて骨抜き状態になること。	キュン死とは「胸がきゅんとする(~とくる、~となる)」またはそういった心情を表した擬態語「キュンキュン」の『きゅん』と『死』の合成語で、心臓が止まってしまいそうなほど胸がきゅんとすること。つまり、好きな人の言動で胸が締め付けられ、心を奪われて骨抜きとなる状態を意味する。またこれが転じ、一瞬で心を奪われ、死にそうなほど深い恋に落ちることをキュン死というようになる。 キュン死には中原アヤ原作の漫画「ラブ
\\	コン」の主人公:小泉リサが使っていたもので、同漫画の映画化(2006年公開)を機にキュン死にも中高生を中心に普及した。	2006年	若者言葉	
\\	キラ男	きらお	外見 ・ 内面ともにかっこいい男性のこと。	キラ男とは輝くさま、生き生きとして輝くように見えるさまを表す「キラキラ」と「男」の合成語でキラキラと輝く男(=容姿・中身ともにかっこいい男)を意味する。キラ男を取り上げた女性ファッション誌
\\	では、外見はほどよく筋肉がつき、服装は人気セレクトショップで揃え、アクセサリーはシンプル。内面は女性に対する気配りがきき、優しく、自分を磨いているといった点をキラ男の特徴として挙げている(キラ男が
\\	による造語なのか否かは不明)。	2006年	若者言葉	
\\	キレカジ	きれかじ	若者ファッションの一種。	キレカジとは「きれいめカジュアル」の略で、服装自体はデニムに
\\	シャツといったカジュアルなものだが、そこにゴージャスなアクセサリーをつけたり、メイク(化粧)をしっかり綺麗にするといった若者のファッションスタイルをいう。また、普段着姿でも、センスの良さからキレイに見える人のこともキレカジという(ただし、こちらの意味ではあまり用いられない)。	2006年	若者言葉	
\\	銀の卵	ぎんのたまご	定年退職したシニア世代の中で、まだ職能を有する人材のこと。	1960年代はじめ、中学を卒業し、就職先を求めて東京や大阪など都会へ集団就職する団塊世代の若者を金の卵と呼んだ。そんな団塊の世代も2007年になると定年を迎え始め、一斉に退職者が出ることから起こる2007年問題が話題となる。そんな中、長年の知識や経験が生かせる、これら団塊の世代を新たな人材として確保しようという動きが出てくる。こうして高齢となった団塊世代の中で人材として求められるような高度な職能のある者を、金の卵とシルバー世代をかけた銀の卵と呼ぶ。	2006年	
\\	クーデレ	くーでれ	一見クールな性格だが、親しくなると甘える人のこと。	クーデレとはクール
\\	とデレデレから成る合成語で、一見クールな性格だが親しくなると甘えたり、可愛らしさといったデレデレ面を見せる女性(キャラ)を意味する。2005年にツンデレ(ツンツンとデレデレの合成~下記関連語より参照)が流行ったが、クーデレはその亜種であり、萌え属性の一種である。	2006年	若者言葉	
\\	グダる	ぐだる	グダグダを動詞化した言葉。	グダるとはグダグダという擬態語を略したものに、動詞化する接尾辞「る」を付けたもので、グダグダに値する様になること。また「グダグダ言う」「グダグダする」と同義に使われる。最後になって往生際の悪いことを言ったり、終了の段階に入っているのにずるずると無意味に続けてしまうこと。無意味に怠けて過ごすこと。また、疲れてダラダラ過ごしたり、グッタリすることをダグるという。	2006年	若者言葉	
\\	クレしん	くれしん	人気マンガ ・ アニメ『クレヨンしんちゃん』のこと。	クレしんとは臼井儀人による漫画作品であり、人気アニメである『クレヨンしんちゃん』を略したものである。クレしんは1990年夏に双葉社の週刊
\\	漫画アクションで連載開始。1992年4月よりテレビ朝日にてアニメ放送が開始。主人公幼稚園児の野原しんのすけを中心に家族、友達、周りの大人たちとの間に起こる騒動を面白おかしく描いたギャグマンガである。なお、2009年9月には作者:臼井儀人が不慮の死を遂げている。	2006年	略語	
\\	ゲセレブ	げせれぶ	品のないセレブのこと。	ゲセレブとは「下品なセレブ」の略で、急に大金を手にし、財的にはセレブと呼べるが品や風格までは備わっていないような人、お金の使い道がわかっていないセレブもどき、なりそこないセレブといった意味で使われる。	2006年	
\\	コンビニスト	こんびにすと	コンビニエンスストアをよく利用する人のこと。	コンビニストとは小規模な日用雑貨食料品店を意味“コンビニエンスストア
\\	の略称『コンビニ』に、英語で「~する人」という意味にする接尾辞
\\	(ピアニスト、ジャーナリストなどに見られる)』を付けたもので、コンビニエンスストアをよく利用する人を意味する。この「利用」とは、大半が客としての利用を意味するが、複数店のコンビニでアルバイトをしているなど、従業員 ・ アルバイトとしての利用をさす場合もある。また、特に頻繁に利用する人をヘビー ・ コンビニストともいう。	2006年	若者言葉	
\\	サーセン	さーせん	謝罪の言葉。	サーセンとは「すいません」という謝罪の言葉が崩れたものである(こんにちは=ちーっす、ありがとうございます=アザースなどの類い)。意味的には「すいません」と変わりないが、軽いニュアンス、ふざけたニュアンスの言葉であり、仕事上の謝罪といった真剣に詫びる必要がある場面での使用は逆効果となる。 なお、サーセンという文字表記を始めたのは掲示板サイト2ちゃんねる内のスレッドである。	2006年	若者言葉	
\\	サバメシ ・ サバ飯	さばめし	サバイバル飯、またはサバイバル飯炊き術の略。	サバメシとはサバイバル ・ メシダキ術という飯炊き方法、又はその方法で炊かれたサバイバルメシの略称である。サバメシはアルミ缶と牛乳パックで飯を炊くもので(具体的な方法は下記参照)、防災科学技術研究所の内山庄一郎氏が考案した方法である。内山氏は2004年に起きた新潟県中越地震で現地入りした際、ガスコンロで飯炊きを行っていたが、ある書籍でアルミ缶による飯炊き方法を知り、実践する。しかし、全くうまく炊けず、以来半年をかけて実践的なアルミ缶による簡易炊飯を考案。2006年にはサバメシというネーミングで
\\	などメディアでも話題となった。	2006年	
\\	サムライブルー	サムライブルー	サッカー
\\	杯日本代表のキャッチ ・ フレーズ	サムライブルーとは
\\	ワールドカップ2006ドイツ大会に向けて付けられた日本代表のキャッチ ・ フレーズである(サムライブルーの「ブルー」は日本代表のユニフォーム ・ カラー)。日本サッカー協会がワールドカップ本大会のため、2005年10月に5つの候補の中から一般投票を行い、2006年1月27日に
\\	と発表。5月17日には
\\	による応援歌『サムライブルー 
\\	もリリースされている。ちなみに他の候補は上位から
\\	「世界を驚かせよう。」「頂点へ、全員で。」
\\	。2位の倍以上の得票でサムライブルーが選ばれた。	2006年	
\\	シカメ	しかめ	メールを無視すること。または無視されたメールのこと。	シカメとは無視を意味する『シカト』と『メール』から成る合成語で、届いたメールに返事をしないこと。また読むことすらしないといったように、届いたメールを相手にしない ・ 無視することをいう。または、そのように無視されたメールのことをいう。シカメール、シカメルといった呼び方もする。	2006年	若者言葉、合成語	
\\	執事カフェ	しつじかふぇ	女性向けのカフェのこと。(日本語俗語辞書サイトポリシー)	執事カフェとは女性(主にオタク女性)向けのカフェ(喫茶店)のことで、アキバ系を中心とした男性オタクをターゲットとするメイドカフェのブームに対し、「男の子が接客する逆バージョンがあったら・・・」という女性(主に腐女子、オタクガール)の声から生まれた形態である。執事カフェのスタッフは当然男性で、メニューはスイーツをメインに構成、燕尾服の男性ホールスタッフが女性客を「お嬢様(男性客の場合はお坊ちゃま ・ 旦那様)」と呼び、混雑時に設けられた制限時間になると「お嬢様、乗馬のお時間でございます」といった言葉がかけられるという。執事カフェは2006年4月現在、池袋で3月にオープンした『執事喫茶
\\	のみだが、インターネットやメディアでは既に話題となっている。	2006年	施設分類	
\\	ジモる	じもる	地元で遊ぶこと。	ジモるとは『地元(じもと)』を略したものに、動詞化する接尾辞『る』をつけたもので、地元(出身地、実家のある場所及びその周辺)で遊ぶことをいう(広義には地元に帰る、地元に行くといった意味でも使用)。主に実家から離れた場所の高校 ・ 大学に通う生徒が使う。	2006年	若者言葉	
\\	写メ交	しゃめこう	写真付き携帯メール交換の略。	写メ交の『写メ』とは写真付き携帯メールの略だが主にその写真(画像)を指し、『交』は交換の略。つまり写メ交とはメールで画像交換することを意味する。写メ交は主にインターネットで知り合った友人と互いの容姿を知るために行われるため、顔写真など当人が写った画像の交換をいう。メル友募集の掲示板では、写メ交をメル友になる条件として挙げている者が増えている。	2006年	若者言葉	
\\	塾弁	じゅくべん	子供が学習塾に持っていく弁当のこと。	塾弁とは学習塾へ持っていって食べる弁当のこと。子供たちの多くが学校から帰宅後、学習塾に通っていますが、その時間は夕方から21時台と長時間に渡る。帰宅後に夕食を取っていては時間が遅いため、弁当を持たせる親が増えている。また、学習塾側も塾弁を食べる時間を配慮してか、7時台に若干長めの休憩を取るところが増えている。また、脳の活性化や休憩時間内に簡単に食べられることを考慮した塾弁
\\	僖
\\	轡塰
\\	も出ている。	2006年	
\\	ジュヴァーナ ・ 十番
\\	女	じゅばーな	麻布十番界隈に勤めるキャリアウーマンのこと。	ジュヴァーナとは麻布十番界隈に勤め、活躍するキャリアウーマンのことである。ジュヴァーナは東京都文京区で発行されているフリー雑誌『都市生活向上マガジン 
\\	で麻布十番特集が組まれた際に作られた造語で、漢字では十番
\\	女と書く。その後、
\\	番組で使われ普及した。	2006年	
\\	趣味父子	しゅみふし	父と子で同じ趣味を持って楽しむこと。	趣味父子とは学校にあがるくらいの“様々なことに興味を持ち始める年齢の子供”を持った父親が自分の趣味を子供に教え、父子が一緒に同じ趣味を楽しむこと、またはそういった父子のことをいう。 我が子が将来、一流企業に勤めるよりも、クリエーターや職人といった個々の技術力や独創性、センスを生かした職に就くことを望む親が増えている。そんな考えからか、勉強以外にも芸術や料理、物を作るといった趣味を子供に教え、一緒に楽しみながら子供を育てていくという趣味父子が増えている。	2006年	
\\	準ヒキ	じゅんひき	外出はするが、家族以外の交流がない(出来ない)人のこと。	準ヒキとは富山国際大学専任講師である樋口康彦が提唱する準ひきこもりの略。準ヒキ及び準ひきこもりとは、通常の引きこもりのように自宅に籠もるわけでなく、学校へ行くなど外出はするが、家族以外との人との交流が無い・持てない人を指す。学校にも登校し、一見問題がないように見えるが、対人経験が少なく、社会人になってから問題が表面化するともいわれる。	2006年	
\\	人工天然	じんこうてんねん	本来天然(ボケ)ではないが、天然のフリをする人のこと。	当人は特に笑わせる気はないが、言動がズレていて面白いこと(またはそういった人)を天然という。これに対し、本質的には特に天然でないが、人(特に異性)から好かれたり、楽しい人と思わせるために天然や不思議ちゃんといったキャラクターを装うことを人工天然という。	2006年	若者言葉	
\\	新成田離婚	しんなりたりこん	熟年夫婦が海外旅行を機にする離婚のこと。	新成田離婚とは夫の退職祝いなどで海外旅行に行き、その旅が引き金となってする熟年夫婦の離婚のことである。退職後、毎日家にいる夫へのストレスが熟年離婚の原因になることは多いが、そうした状況下で絶えず行動を共にする海外旅行が悪い影響を及ぼすこと。また、家でしているように旅先でも横柄に振舞う夫の態度が新成田離婚を決意する原因といわれている。	2006年	
\\	退職型ニート	たいしょくがたにーと	一旦就職するが、ヤル気をなくしてニートになる人のこと。	退職型ニートとは一旦は職に就いたものの、就労意欲をなくして退職し、そのままニートになってしまう人のことである(ニートについては下記関連語参照)。退職型ニートは新卒ニート(学業を終え、そのままニートになってしまう者)に比べ、就職希望をしない人の率が高く、社会問題になっている。また、企業側にとっても初期投資をかけた人材を失うことになり、大きな痛手である。こうした退職型ニートを出さないためには、社内ニートの段階での対処が必要であるが、それ自体が困難といわれており、多くの課題を残す問題になっている。	2006年	
\\	体入	たいにゅう	風俗店などに体験入店して一日働くこと。	体入とは体験入店の略で、キャバクラやその他風俗店で使われる風俗業界の用語である。体入はホステス及びホスト希望者が正式な契約をする前に一日店に出て働き、その店の雰囲気を知るために設けられたもので、この場合も大抵は日払い給与が支払われる。中には勤める意思は全くないが、体入で受け取る給与を目的に、あちこちの店で一日体験入店をして回る体入荒らし(体入女ともいう)も存在する。	2006年	略語	
\\	だきょる	だきょる	妥協を意味する若者言葉。	だきょるとは「妥協する(だきょうする)」を略した言葉である。意味的な変化は特にないが、例えば何かを途中で諦めた際、「妥協した」と言うより「だきょった」と言ったほうが深刻さが薄れ、軽く言えることから若者が使用。(ただし、若者の間でも日常的に使われるほど浸透している言葉ではない:2008年現在) だきょるの使用例: 今回は最後まで頑張るつもりだったけど、結局、だきょっちゃった。	2006年	若者言葉	
\\	タベホ	たべほ	食べ放題のこと。	たべほとは「食べ放題」の略で、食べ放題の店のことをいう。食べ放題の店とは定額料金でおかわり、追加が自由、多くは盛り付けはセルフで行う(バイキングスタイル、ビュッフェスタイル)。また、2時間1500円食べ放題といった制限時間を設けている店もある。食べ放題の店へ行くことは「たべほ行く」。「たべほする」と言った場合、食べ放題の店で食事を摂ることを意味する。たべほー(たべほ~)と伸ばすこともある。	2006年	
\\	ダメンズ	だめんず	恋愛 ・ 結婚対象としてダメな男のこと。	だめんずとは「駄目(ダメ)」と「メンズ」をかけた言葉で、恋人や結婚相手としてダメな男を意味する。だめんずは倉田真由美の漫画タイトル『だめんず ・ うぉ~か~』からきたもので、具体的には金がない男、浮気症、
\\	(暴力)などがある。漫画は2000年から週刊誌
\\	にて連載。2002年に飯島愛、原千晶というキャスティングでドラマ化。更に2006年には藤原紀香主演で再びドラマ化。これが話題となり、広くだめんずという言葉も普及。また、2007年には倉田真由美著『だめんず症候群』という書籍が発売。	2006年	
\\	チャラ字	ちゃらじ	雑に書かれた汚い字のこと。	チャラ字のチャラは軽薄なさまを表す「チャラチャラ」、更にそれを形容詞化した「チャラい」からきたものである。つまり、チャラ字とは軽薄な感じのする文字ということになるが、具体的には雑な字、速く書いた汚い字などがこれにあたる。(古くは「ちゃら」だけで出鱈目という意味で使われたが、チャラ字についてはチャラチャラやチャラいから派生で出来た言葉と考えられる) 字の乱れ(汚い字)の原因として、パソコンや携帯メールの普及により、書く機会が減ったなど挙げられるが、チャラ字をやめ、一画ずつ意識して書くだけでも随分正されるという意見もある。	2006年	若者言葉	
\\	ちょいオタ	ちょいおた	オタクっぽい部分を出して女性にモテようという考え方。	ちょいオタとは「ちょっと」が変形した『ちょい』と『オタク』の合成語だが、ちょっとしたオタクという意味でなく、ちょっとオタクっぽい部分を見せて女性の心を射止めようという考え方をさすものである(言葉の意味的には「ちょいワル(おやじ)」の派生語と考えたほうが正しい)。ちょいオタはメンズノンノが2005年末に提唱した案である。2006年6月には月刊少年エース(角川書店)連載
\\	(日本ひきこもり協会)へようこそ」のキャンペーンポスターにある「ちょいオタ、いいじゃない?」というキャッチコピーで普及。なお、このキャンペーンのイメージモデルは「ちょいワル」で知られるパンツェッタ ・ ジローラモで、ポスターの赤褌姿が話題となった 。	2006年	合成語	
\\	ツンテレ	つんてれ	普段はツンツンし、二人きりになると照れる人のこと。	ツンテレとはツンデレの派生語で、普段はツンツンしているが、二人きりになると照れる。または照れて逃げ出したり、照れ隠しで暴言を吐いて逃げ出す女性を意味する。例えば、男性が二人きりになったとき「結構、かわいいとこがあるね」と言ったとする。ここでデレデレと男性に寄っていくのがツンデレ。真っ赤になって逃げ出したり、「バカ~」などの暴言を吐いて去ってしまうタイプがツンテレにあたる。二人きりになったとき、好きと認めてしまうのがツンデレ。認めたいが認められずにいるのがツンテレともいえる。	2006年	若者言葉	
\\	ツンデレラ	つんでれら	ツンデレ属性をもつキャラクターのこと。	ツンデレラとは『ツンデレ』と『シンデレラ』の合成語で、ツンデレの普段はツンツン、二人きりになるとデレデレといった特徴をもつキャラクターのこと。また、そういったタイプの実在する女性に対してもツンデレラが使われる(ツンデレについては下記関連語参照)。これとは別にツンデレ愛好家という意味でツンデレラが使われる場合もある。ただしこの意味で使用する場合、英語で「~する人」という意味にする接尾語“-
\\	をつけ、ツンデレラーとのばすことが多い。 なお、ツンデレラは2006年新語 ・ 流行語大賞にノミネートされた。	2006年	若者言葉	
\\	デレかけ	でれかけ	ツンデレのデレになる瞬間のこと。	デレかけとは「デレデレになりかけ」「デレ状態に入りかけ」といったものが略された言葉で、ツンデレにおけるデレになる瞬間を意味する。ツンデレには普段はツンツンしている女性(キャラクター)が何らかの要因でデレデレに変わる「ツンツンからデレデレ移行型」とツンツンとデレデレの同居型がある。デレかけは移行型タイプでデレに完全に変貌する前のデレデレになりかけた状態を意味する。なお、デレかけは2005年8月26日に“きゃんでぃそふと”から発売された18禁恋愛アドベンチャーゲーム『つよきす』のスタッフが提唱した言葉である。	2006年	若者言葉	
\\	鉄板	てっぱん	「間違いない」「確実な」という意味の言葉。	鉄板は「硬い」ことから、「堅い」にかけ、ギャンブルや芸人の世界で「間違いない」「確実な」といった意味で使われる。特にお笑いの世界では確実にうけるネタ ・ ギャグ(主に大御所芸人の定番ギャグ)を『鉄板ギャグ』という。また、これを略して鉄板ともいう(例えば、ビートたけしの定番ギャグであれば「たけしの鉄板」という)。 ※鉄板はギャンブラー(主に競輪競馬、ボート)の間で、昭和時代(中後期)には既に使われていた言葉ですが、お笑い芸人を通じて一般(主に若者)にも広く浸透した言葉ということで、芸人から広まった年、2006年の俗語と設定しています。	2006年	若者言葉	
\\	読モ	どくも	読者モデルの略。	読モとは読者モデルの略で、ファッション誌のモデル募集やスカウトによって選ばれたファッション雑誌の素人モデルをいう。普段は学生や
\\	といった一般的な生活をしている読モは親近感が持てるなど、好意的に受け入れられることが多いが、逆に嫉妬 ・ 嫉みを持たれたり、モデルとしては素人故に軽んじられることもある(掲示板サイト2ちゃんねるでは侮蔑した毒モという表記も使われる)。こうした読モの中には、それを機にプロダクション契約し、プロのモデルやタレント、女優として活動する者もある。逆に、既にプロダクション契約しているモデルが宣伝のために読モをすることもあるという。 雑誌名を頭につけ、
\\	読モ、
\\	読モといった呼び方もする。	2006年	若者言葉	
\\	仲仔	なかこ	仲の良い友達のこと。	仲仔とは仲の良い子、親友を意味する。仲仔を崩すと「中イイ子」なり、そこから仲のいい子という意味になったとされる(メール情報より)。仲仔は中学生向けのファッション誌
\\	から発信~普及した言葉で、
\\	では仲がいいことをアピール出来る言葉として紹介されている。こうして中学生を中心に小学生を含む低年齢層で普及している。 仲のいい友達との写真を「仲仔画像」、生涯仲のよい友達の意で「一生仲仔」といった使い方もする。	2006年	若者言葉	
\\	ナルハヤ	なるはや	依頼 ・ 指示したことを早く済ませて欲しいという意思表示の言葉。	ナルハヤとは「なるべく早く」の略で、お願いしたことや依頼、支持したことを一刻も早く完了させて欲しいという願いを表した言葉である。もともとは若者の間で使われたナルハヤ。「なるべく早く」では相手に負担 ・ 威圧を与えてしまう可能性があるため、早めに結果を出しては欲しいがそんなに切羽詰っているわけではないというきなど、軽いノリでお願い出来るナルハヤを使う社会人もあるという。	2006年	若者言葉	
\\	ニコモ	にこも	小中学生向け雑誌『ニコラ』のモデルのこと。	ニコモとは小学生~中学生向けのファッション雑誌『ニコラ』(新潮社より1997年創刊)に出てくる女性ファッションモデルのことで、「ニコラ(の)モデル」の略である(正確にはモを
\\	で囲んだ表記を使う)。ニコモは大抵10代前半で、もともとモデルとして活動していた者の他、読者オーディションによって起用されるようになった者がある。ニコモを卒業し、その後も活躍しているタレントには栗山千明、蒼井優、沢尻エリカ、新垣結衣がいる。また、ニコラに出てくる男性モデルのことをメンモという。	2006年	若者言葉	
\\	ネガティブルー	ねがてぃぶるー	落ち込んで、言動が消極的なこと。	ネガティブルーとはマイナス思考や否定的 ・ 消極的なさまを表す「ネガティブ」と、気分が沈んださまを表す「ブルー」から成る言葉で、気分が沈んでいるために、行動 ・ 言動が消極的になっていることを意味する。	2006年	合成語	
\\	ネット弁慶	ねっとべんけい	インターネット上では強気だが、実際にはおとなしい人のこと。	ネット弁慶とはインターネットの掲示板やチャット、
\\	など、ネット上では威勢がよく強気だが、実際にはおとなしかったり、気弱な性格をした人のこと。外では意気地がないが、家では威張り散らす内弁慶をもじった言葉である。ネット上では仕切ったり、威張り散らしてた人がオフ会などで実際に会ってみたら無口なネット弁慶だったということも少なくない。同様にネットではおとなしくしていたが、実際に会ったら態度が大きな人もいるという。	2006年	ネットスラング	
\\	寝飛ばす	ねとばす	寝坊をして予定を棒にふってしまうこと。	寝飛ばすとは寝坊をして、本来計画していた予定を飛ばしてしまうことをいう。例えば、寝過ごしてしまい、仕事を休んだ。チケットを購入していたコンサートに寝坊して行けなかったといった際に用いる。寝飛ばすはあくまで予定を飛ばした(=履行しなかった)際に用いる言葉であり、遅刻してでも参加した場合は用いない(使用例2参照)。 寝飛ばすの使用例: 楽しみにしてたデートなのに、仕事疲れで寝飛ばしてしまった。 寝飛ばすほどではないけど、ほとんど番組は終わってた。(使用例2)	2006年	若者言葉	
\\	パギャル	ぱぎゃる	中途半端なギャルのこと。	パギャルとは「中途半端 ・ ハンパ」の「パ」と「ギャル」から成る合成語で、中途半端なギャル(なんちゃってギャル)を意味する(会話の流れからパギャルと認識出来る場合『パ』とも略される)。パギャルの定義は使う人やサークルによって異なるが、色白で成りは普通の女子高生なのにギャル文字を使い、ギャル気取りの若者を指すことが多い。そもそも最近のギャルの間ではギャル文字を「古臭い」「読みにくいだけ」と捉える者も多く、携帯メールでギャル文字を使っている若者全般をパギャルと呼び、非難 ・ 軽視する傾向がある。	2006年	コギャル語 ・ ギャル関連、若者言葉	
\\	柱本	はしらぼん	つまらない本のこと。	柱本とは「読後、柱に叩きつけたくなる本」の略で、内容に納得いかなかったり、つまらないといった理由で柱に投げつけたくなるような本を意味する。特にどういった本という定義はないが、ミステリーや推理小説のトリックによる結末のつまらなさが要因となることが多い。また、一般的な書評や他者の感想に関係なく、自身が柱に投げたくなるほどつまらないと感じれば柱本となる。ただし、類似語の『壁本』のほうが認知度は高い。	2006年	若者言葉	
\\	パネェ	ぱねぇ	程度を強調する言葉「半端ではない」が変化したもの。	パネェとは「半端ではない(半端じゃない)」という程度を強調する言葉が次のように変化していくことで出来たものである。まず『半端ではない(半端じゃない)』が簡略化され『半端ない』に、次に『ない』の部分の語感が荒くなった『半端ねぇ(ハンパねぇ)』となり、更に略されたのがパネェである。パネェは女子中学生向け雑誌
\\	の中で頻繁に使われたことから若者を中心に普及。カタカナの他にぱねぇやパねぇといった表記も使われる。 パネェの使用例: パネェくらいカレのことが好き パネェくらい頑張ったー!	2006年	若者言葉	
\\	パネェ画	ぱねぇが	携帯待受画像の一種。	パネェ画とは携帯電話の待受け画像の一種である。パネェとは程度を強調する文句「半端じゃない(半端ではない)」が変形した言葉だが(詳しくは下記関連語参照)、パネェ画はこのパネェを使った文章を画像化したものである(「パネェ好き(半端じゃなく好きの意)」や「パネェ頑張る!(半端じゃないくらい頑張るの意)」など)。他の文字による待受け画像の多くが手書き文字を使用しているのに対し、パネェ画は無地(主に黒地)の背景に太め白抜きのワープロ文字で書かれているのが特徴である。	2006年	若者言葉	
\\	バモス	ばもす	(見た目的 ・ 生理的に)非常に気持ち悪いこと。	バモスとは「非常に ・ とても」という意味の俗語『バリ』又は『バカ(バカデカイなどに使用)』と、「気持ち悪い」を意味する俗語『きもい』が変化した『キモス(註参照)』から成る合成語で、見た目的に非常に気持ち悪いものや生理的に気持ち悪いものを指し、侮蔑の意を込めて使われる。 註)某大手掲示板を中心に若者の間で、形容詞や動詞の語尾を『ス』に変える言葉が増えている(例:「ワロス(笑う ・ 笑えるの意)」「ウレシス(嬉しいの意)」)。「キモス」もそういった若者言葉のひとつである。	2006年	若者言葉	
\\	ハラミー	はらみー	妊婦のこと。	ハラミーとはみごもる=妊娠することを意味する『孕む(はらむ)』に、「~する人」という意味にする英語の接尾辞
\\	をつけたものである。つまり、ハラミーとは孕んだ人=妊婦を意味する。	2006年	若者言葉	
\\	ビニ弁	びにべん	コンビニエンスストアで売っている弁当のこと。	ビニ弁の『ビニ』はコンビニという略称で親しまれている小規模な日用雑貨食料品店“コンビニエンスストア
\\	とは「便利」「好都合」の意)”の略。『弁』は“弁当”の略である。つまり、ビニ弁とはコンビニエンスストアで売っている弁当を意味する。一般的にはコンビニ弁当と呼ばれているが、それを更に略したのがビニ弁である。	2006年	若者言葉	
\\	ビニる	びにる	コンビニエンスストアに行くこと。	ビニるとは小規模な日用雑貨食料品店でコンビニという略称で親しまれている“コンビニエンスストア
\\	ちちなみに
\\	とは「便利」「好都合」の意)”を略した『ビニ』に、動詞化する接尾辞『る』を付けたもので、コンビニへ行くことを意味する。オケる、マクるなど様々な店舗 ・ 施設の略称を動詞化する言葉が生まれているが、ビニるもその一つ。	2006年	若者言葉	
\\	干物女	ひものおんな	様々な事柄で面倒臭がったり、適当に物事を済ませる女性のこと。	干物女とは様々なことに対していい加減だったり、面倒臭がったりする女性のことで、“ひうらさとる”のマンガ『ホタルノヒカリ』からきた言葉である。『ホタルノヒカリ』では主人公の蛍が27歳であることから、干物女は20代後半~30代の女性に対して使われる。2007年7月には綾瀬はるか主演で日本テレビより『ホタルノヒカリ』がドラマ化され、干物女も流行語に。同年の新語 ・ 流行語大賞で候補60語にノミネートされた(ベスト10 ・ 受賞は逃している)。	2006年	
\\	貧乏クジ世代	びんぼうくじせだい	就職難の時代に就職活動をした世代のこと。	貧乏クジ世代とはバブルが崩壊し、企業が新卒の採用を抑制した1992年以降に就職活動をすることになった世代のことで、1970年から1980年代初めに生まれた世代をさす。貧乏クジ世代は精神科医の香山リカによる造語で、香山氏の書籍タイトル にもなっている。この世代は幼少期にバブルを経験し、勉強さえ頑張れば幸せになれると信じてきたのが、就職活動時期にバブルがはじけたために一変。新卒採用に抑制がかかったうえ、団塊ジュニアと呼ばれる人口が多い世代でもあるため、失業率があがった。この就職難の時期を就職氷河期と呼ぶことから就職氷河期世代ともいう。	2006年	若者言葉	
\\	プロい	ぷろい	プロ級の技術や完成度のこと。	プロいとは玄人とか、それで生計をたてている人という意味の『プロ(英語の
\\	からきたもの)』に形容詞にする接尾語『い』を付けたもので、技術や完成度がプロ級、プロさながらであるさまをいう。また、プロ ・ アマといった区別がないような分野 ・ 事柄においても、他者より秀でているさまをプロいということがある。	2006年	
\\	ペア画	ぺあが	携帯電話の待受け画像の一種。	ペア画とは携帯待受画像の一種で、2枚の待受画像(2台の携帯電話)を合わせることでひとつの絵になるものや、2枚の絵が対になっている待受画像のことである。前者の場合、ハートマークが半分づつ描かれ、2台を合わせることでハートが形作られるものや、一枚は男性が口を突き出している画像、もう一枚は女性が口を突き出している画像で、合わせると男女がキスをしている絵になるといったペア画もある。後者のペア画は、例えば一方に男性のイラストを、もう一方に女性のイラストを対になる構図で描くことでペアとなる。こうしたペア画に互いのメッセージを入れて持ち歩き、会ったときに画像を合わせる若いカップルが増えている。	2006年	若者言葉	
\\	ヘビロテ	へびろて	頻繁に着る服のこと。	ヘビロテとはヘビーローテーションの略で、もともとは
\\	やラジオ局が気に入っている、薦めたい、売り込みたいといった理由で頻繁にかける曲やアーティストを意味する。ここから、ファッション業界(主に女性誌)で、短いインターバルで何度も頻繁に着る服のことをヘビロテと言うようになる。	2006年	美容 ・ ファッション用語 ・ 若者言葉 ・ 略語	
\\	まくる	まくる	
\\	前につく動詞の行為が勢いよく行われるさまを表す言葉。 
\\	電話やメールを無視すること。	
\\	まくるとは動詞の後につけることで、その動詞の行為が勢いよく行われるさまや、むやみに成されるさまを表す言葉である。若者言葉で名詞の後に『する』や『る』を付けて動詞化するものがあるが、平成以降、まくるも名詞の後に付けて動詞化する言葉としても使われている(例:ゲームまくる、ネットまくる』)。 
\\	まくるとはかかってきた電話に出なかったり、届いた携帯メールに返事を出さないなど、電話やメールを無視するという意味で使われる若者言葉である。 
\\	から、メールを出しまくる、電話をかけまくるの意で使う場合もあえう。	2006年	後者は若者言葉	
\\	マッパネェ	まっぱねぇ	「半端ではない」が強調された言葉。	マッパネェとは「本気 ・ 真剣 ・ 冗談ではない」という意味の俗語『マジ』と「半端ではない(半端じゃない)」を意味する俗語(略語)『パネェ』を合成し、語呂をよくした言葉である。意味もマジとパネェを合わせた「真剣に半端ではない」となるが、単に「パネェ(=半端じゃない)」が強調されたものと考えてよい。	2006年	若者言葉	
\\	メール残業	めーるざんぎょう	メールを使った持ち帰り残業のこと。	メール残業とは仕事を家に持ち帰ってする持ち帰り残業の一種。これまで持ち帰り残業といえば、資料や書類を家に持ち帰り、自宅で作成して職場に戻すという荷物を運ぶ手間がかかった。しかし、電子メールの発達により、自宅と会社間の資料の持ち帰りや戻すといった作業がメールの送受信で済むようになった。こうした手法でする残業をメール残業と呼ぶ。しかし、メール残業を含む持ち帰り残業は労働時間の特定がしにくく、サービス残業となる問題を抱えていることに変わりない。また、安易な資料の持ち出しが情報漏洩の原因になると指摘する意見もある。	2006年	
\\	メガドル	めがどる	メガネをかけたアイドルのこと。	メガドルとは「メガネアイドル」または「メガネっ娘アイドル」の略で、メガネをかけたアイドルを意味する。特にメガネが特徴となっているアイドル、終日メガネをしているアイドルをさす。それまでアイドルのメガネはタブーとされていたが、オタクが好む萌え属性としてメガネっ娘が浮上。時東ぁみがメガドルの火付け役として話題となる。2007年現在、他に主だったメガドルの存在はなく、むしろ時東ぁみの代名詞的言葉として使われている。	2006年	若者言葉 ・ 略語	
\\	メトセク	めとせく	都市部に住み、ファッションや美容に興味をもつが男らしさも失わない男性のこと。	メトセクとは1994年にイギリスのマーク ・ シンプソンが作った造語『メトロセクシャル』の略である。具体的には都会に住み、ファッションやスキンケア(男性コスメ)に気を使う男性がメトセクである(他にもお洒落なお店を知っているなど、様々な要素がある)。メトロセクシャル自体は日本でも2005年辺りから男性誌などで使われるようになっていたが、2006年にメトセクと略した形で広く普及。	2006年	略語	
\\	文字画	もじが	携帯電話の待受画像の一種。	文字画とは携帯待受画像の一種である。それまでの待受画像は好きなアイドルの写真やキャラクター、風景画など、種類は多かったが、すべて画像がメインであった。これに対し、文字がメインになったものが文字画である。文字画の内容は「好き」「仲良し」といったお気に入りの単語を画面いっぱいに書いたものから数行にわたる文章タイプのものまであり、字体も様々である(当初、明朝体やゴシック体など、
\\	にデフォルトで入っているものだったが、次第に手書き文字が主流になっている)。また、文字の背景も無地一色のものから、文字にあうデザインやキャラクターをほどこしたものまで様々である。文字画の代表的なものに「パネェ画」「恋愛画」「友情画」がある(個々の内容は下記関連語参照)。	2006年	若者言葉	
\\	友情画	ゆうじょうが	携帯電話の待受画像の一種。	友情画とは文字(文章)が配された携帯待受け画像の一種である。内容は「一緒にいてくれてありがとう」「いつもありがとう」「アナタは私の宝物」といった友人へのメッセージ的なものが多い。これらの文章の長さはさまざまで、数行に渡る手紙のようなものもあれば「親友(真友 ・ 心友とも書く)」や「仲良し」など一言だけのものもある。同時期に流行ったパネェ画に比べ、背景に風景やキャラクターなどがカラフルに使われている点、字体が手書き文字である点が友情画の特徴である(類似構成で恋愛をテーマにした恋愛画というものもある)。友情画もパネェ画や恋愛画同様、背景や文章を指定すれば作成してくれるサイトが存在する。	2006年	若者言葉	
\\	よさのってる	よさのってる	髪が乱れるさまを表す言葉。	よさのってるとは作家 ・ 歌人である『与謝野晶子』と、状態を表す『~してる(「~している」の意)』を合成させた言葉で、髪が乱れているさまを表す言葉である。これは与謝野晶子の代表作『みだれ髪 
\\	という歌集からきている。つまり、よさのってる=与謝野晶子+している=みだれ髪+している=“髪が乱れている”となる。	2006年	美容 ・ ファッション用語	
\\	よさまくる	よさまくる	激しく髪が乱れるさまを表す言葉。	よさまくるとは『よさ』に行為が勢いよく成されるさまを表す俗語『まくる』をつけたものである(『~まくる』については関連ページ参照)。『よさ』とは明治時代~昭和時代にかけて活躍した作家:与謝野晶子(よさのあきこ)の頭2文字をとったもので、与謝野晶子の歌集『みだれ髪 
\\	から『よさまくる』→『与謝野晶子+まくる』→『みだれ髪+まくる』。つまり「髪が乱れまくっている」という意味になる(類語『よさのってる』が普及しており、よさまくるは『よさのってる』よりも更に髪が激しく乱れたさまをあらわす派生語でもある)。	2006年	若者言葉	
\\	恋愛画	れんあいが	携帯電話待ち受け画像の一種。	恋愛画とは文字(文章)が配された携帯待受け画像の一種である。内容は「ワタシ幸せなの、アナタがそばにいるから」といった恋人や好きな人へのメッセージ的なものから、「一緒にいることだけが恋じゃない、離れてても心を通わせることが恋」といった恋愛観的なものまで様々で、文章の長さも細かい文字で長文になっているものから「スキ」「恋愛上等」といった単語だけのものまである。同時期に流行った『パネェ画』に比べ、背景にキャラクターやハートマークなどの模様がカラフルに使われている点と字体が手書き文字である点が恋愛画の特徴である。恋愛画もパネェ画同様、背景や文章を指定すれば作成してくれるサイトが存在する。	2006年	若者言葉	
\\	老ティーン	ろうてぃーん	積極的な行動が起こせない10代前半のこと。	老ティーンとは『ローティーン
\\	と『老人』をかけた言葉で、博報堂生活研究所の中村恭子が現代(2006年現在)の10代前半に対して名付けた言葉である。10~60代を対象に「キャリアアップのために転職しても構わないか」という質問に対し、20~30代は過半数が肯定的だったのに対し、10代は22.9
\\	と60代よりも少なく、受験においても、もう少し勉強すれば偏差値の高い学校に入れるとしてもあっさりと諦めてしまう老ティーンが増えているという。老ティーンを生んだ原因としてはバブル崩壊後に生まれ、夢を持つには厳しい時代背景であったことや、頑張っている姿よりも疲れて無気力な姿を子供に見せる親が増えたためと言われている。	2006年	合成語	
\\	ワンセグ	わんせぐ	1セグメント放送の略。	ワンセグとは移動体端末(モバイル)で受信出来る地上デジタルテレビジョン放送のことで、正式には1セグメント放送(ワンセグメント放送)といい、『ワンセグ放送』といった略され方もする。 地上デジタル放送では1チャンネル
\\	を14に分割し、この分割された値(約429
\\	が1セグメント
\\	という単位になる。また14分割されたもののうち1セグメントは隣りのチャンネルとの混信防止に使われ、残る13セグメントを使って映像や音声信号が送信される。ハイビジョンでは12セグメントを使用して同時に1番組を放送、通常画質の場合は4セグメントで済むため同時に3つの番組が放送出来る(スポーツで時間が延長した場合や緊急のニュースが入った場合でも予定の番組と並行して3つまでは同時放送が出来る)。どちらの放送をするにしても13セグメントのうち1セグメントが空くことになるが、この1セグメントがモバイル用に割り当てられている。これがモバイル用の地上デジタル放送をワンセグ(1セグメント放送)と呼ぶ由縁である。 ワンセグの受信については主に携帯電話が考えられており、携帯各社からワンセグ受信可能な携帯電話が発売されている。また、ワンセグは2006年4月1日から放送開始されている。	2006年	略語	
\\	する	あい ・ てぃーする	アイスクリームを食べに行くこと。	
\\	するの
\\	とは
\\	業界などに見られる“
\\	の頭文字でなく、「アイス(クリーム)食べる
\\	の頭文字である。こういった言葉はアルファベット略語、またはこうした略語の中で2007年に流行語になった
\\	から
\\	語
\\	式日本語)ともいう。ただし、他の
\\	語がアルファベットだけであるのに対し、
\\	するの場合、これでひとつの言葉として認知されている。この場合の「する」とは、カラオケする(=カラオケに行く)のように「~に行く」という意味である。ちなみに、この「する」を付けず、
\\	だけで使われることもある。	2005年	
\\	語	
\\	アイパチ	あいぱち	人と目が合うこと。瞬きをすること。	
\\	アイパチとは人と目が合うことを意味する。アイは目を意味する英語“
\\	。パチについてはバッチリやシャッター音のパチリなどが考えられるが詳しいことは不明(情報募集中)。 
\\	アイパチとは瞬き(まばたき)を意味する。アイは目を意味する英語“
\\	。パチは瞬くさまを表す擬態語「パチパチ」や「パチクリ」からきている。	2005年	若者言葉	
\\	アウェイ	あうぇい	場違いであること。	アウェイとは「離れて ・ 不在で ・ あちらへ」といった意味の英語“
\\	だが、俗語的には「場違いなこと」という意味で使われる。これはスポーツにおける『ホーム ・ アンド ・ アウェー』で使われているアウェイの「敵地」という意味が転じたものである。2005年に放送されたフジ
\\	のドラマ『電車男』で主人公:電車男のオタク仲間が「場違いなこと」という意味でアウェイを使ったことから普及した。更にそこから派生して「苦手なこと」という意味でもアウェイは使われる。	2005年	若者言葉	
\\	アニート	あにーと	ニートな兄のこと。	アニートとは職に就かず、求職活動もしていない未婚の社会人という意味の
\\	(ニート)に、兄(あに)をかけたもので、ニートをしている兄のことである。例えばニートな兄を持つ妹 ・ 弟が友人から兄弟がいるか問われた際、「アニートがいるよ」といった形で使用する。ただし、ニート人口が急増しているとはいえ、アニートのように妹 ・ 弟がいるという限定をクリア出来る者、更にこの言葉が使えるシチュエーション自体がまだまだ少なく、限定要素が多いためか、若者の間でも一部でしか普及していない。	2005年	若者言葉	
\\	甘辛系	あまからけい	男っぽい物と女っぽい物を同時に着こなすファッション。	甘辛とは砂糖など甘味の調味料と唐辛子など辛味調味料という対となる味を同時に使うことで引き出される独特な味や調理法のことだが、コギャルの間で甘辛系はファッション用語としても使われ、男性的要素の強い衣類や小物と女性的要素(特にかわいい系をさす場合が多い)の強い衣類や小物を同時に着こなすこと、またはそういったファッションを好んでする人を意味する。ただし、先の料理法と混同されるためか、あまり広くは普及していない。	2005年	ファッション用語、コギャル語 ・ ギャル関連、若者言葉	
\\	アレリーマン	あれりーまん	会話を「アレ」ばかりで済ませる人のこと。	アレリーマンとは「アレとってくれる」「アレってなんだっけ?」「アレだよ、アレ」といったように会話を「アレ」で済ます人のことをいう。アレリーマンは『アレ』と『サラリーマン』から成る合成語だが必ずしもサラリーマンが対象というわけでなく、学生同士の会話でも使われる。 アレリーマンは
\\	が
\\	(プレイステーション ・ ポータブル)『脳力トレーナー ポータブル』のテレビ
\\	用に作った造語で、2005年10月1日から放送されていた。 
\\	の内容は二人のサラリーマンがテーブル越しに  
\\	あの映画みました?  
\\	あー、アレだろ、あの女優が出てる  
\\	そうそう、アレがアレする といった「アレ」を連発する会話をし、二人の間にアレリーマンの文字が表示されるというものであった。	2005年	
\\	インキャラ ・ 陰キャラ	いんきゃら	暗い人や家に閉じこもりがちな人のこと。	
\\	インキャラとは「陰気」と「キャラクター」から成る合成語で、陰気な人や根暗な(暗い)人という意味である。 陰キャラという表記も使われる。ただし、実際の性格が暗いか否かに関係なく、クラスやグループの人(特にいじめっ子 ・ いじめグループ)から嫌われることでインキャラと呼ばれることも多く、そういう意味では『いじめられっ子』『嫌われキャラ』に近い。 
\\	インキャラとは「インドア」と「キャラクター」から成る合成語で、マンガを読んだり、ゲームをしたりといった家で遊ぶ(暮らす)ことを好む人を意味する(オタクや引きこもりとはニュアンスが異なり、単に人付き合いが悪い程度の意味で使われることが多い)。また、学校の休み時間に人と遊んだり、会話をしたりせず、自分の殻に閉じこもる人という意味でも使われる。 インキャラは主に上記2種類の意味で使われ、それぞれそういった人を嘲う言葉だが、一部「インターナショナル」「インテリ」など“イン”で始まる他の言葉と「キャラクター」で合成されている場合があり、前後の文脈から意味を読み取る必要があります。 ※便宜上、
\\	と意味をわけて記載しましたが、
\\	のいじめられっ子という状況から
\\	のタイプになっている人もいますし、
\\	は
\\	の意を含んで使われることも多いので特にわけて考える必要はないです。また、インキャラは普及過程の意味が曖昧な言葉です。学校や地域、人によって
\\	を基準に様々な派生した意味で使われることがあります。	2005年	若者言葉 ・ 合成語	
\\	インパ	いんぱ	
\\	(東京ディズニーランド)へ行く(入園する)こと。	インパとはテーマパーク
\\	東京ディズニーランド)へ行く(入園する)という意味の『インパーク
\\	が略されたものである。インパーク同様、インパ日記、インパ写真といった使い方もされる。また、一人でディズニーランドへ行くことを『一人インパ』という。さらにインパの前にディズニーランドへ行った回数を入れ、『2インパ』といったような単位としても使われる。	2005年	若者言葉	
\\	ウォームビズ 
\\	うぉーむびず	温暖化防止を目的とした冬の厚着スタイルのこと。	ウォームビズとは環境省が2005年夏に行ったクールビズの成功を受けて提唱した省エネファッションの秋冬版である。ウォームビズでは「暖房設定温度を20
\\	以下にしましょう」と謳っている。外気温が8
\\	の場合、暖房設定温度を21
\\	から20
\\	にすることで年間で25.7
\\	の
\\	2が削減でき(一日9時間使用時)、クールビズで提唱した冷房設定温度による効果よりもウォームビズのほうが4倍強の効果が見込まれた。このことから小池大臣を筆頭に環境省は精力的にアピールをしたが、クールビズのときのように
\\	議員はどんなファッションで登場?」といった話題性に欠け、ウォームビズはあまり浸透していない。またクールビズは「ノーネクタイ、ノースーツ」といった男性を主眼とする運動であったが、ウォームビズでは「スカートをパンツにしましょう」と女性にもアピールを行った。	2005年	美容 ・ ファッション用語	
\\	オタパパ	おたぱぱ	オタクなパパの略。	オタパパとはオタクなお父さんのことである。 オタクという言葉が生まれてから20年以上、一般的に使われるようになってからも既に15年近く経過し(2005年12月現在)、当時アニメや鉄道模型、ミニカーなどに凝っていたオタク少年の中には結婚し、子供を持つ者も増えてきた。そうしたお父さんの中で現役オタクでいる人をオタパパという。また、オタパパの子供は同じ種のオタクになる可能性が高いようである。これは単に親に似たというだけでなく、例えばアニメオタクなオタパパは子供にアニメを見せたり、一緒にフィギュアで遊んだりし、鉄道オタクなオタパパなら子供に鉄道のビデオや写真を見せたりする。幼少の頃からそういった環境で育っていることが一番の要因のようである。親子で同じ趣味を持ち、スキンシップがはかれるという点では幸せな親子像ともいえる。 オタパパの特徴としては探求の仕方がより専門的で(知識面、精神面ともに)拘りが強い。また、少年時代に比べ経済的余裕があるため大人買いをするオタパパも多く、昔から集めている物も含め膨大な数のコレクションを保有しているという。	2005年	略語、若者言葉	
\\	おなペコ	おなぺこ	非常にお腹が空いたさまを表す言葉。	おなペコとは「おなかペコペコ」の略で、お腹が非常に空いているさまをあらわす言葉である。基本的には「おなかペコペコ」と同じように用いるが、「おなペコタイム」といった形容詞的な使い方もする。また、空腹時のお腹が鳴る音をおなペコアラームという。	2005年	若者言葉、略語	
\\	おめでた婚	おめでたこん	子供を授かった後にする結婚のこと。	おめでた婚とは子供を先に授かった状態でする結婚のことで、実質的には妊娠したことが引き金でする結婚である場合が多い。1990年代に『できちゃった婚』という同義語が普及。増加傾向にある『できちゃった婚』に対し、結婚産業が「できちゃった」というマイナスイメージを払拭するために考えた言い換えがおめでた婚である。一般にはあまり普及していないが「ママ」と「マリッジ」を合成した『ママリッジ』という言葉も業界では使われている。	2005年	
\\	親トモ	おやとも	携帯メールで繋がった友達のこと。	親トモとは交流手段が携帯メールの友達を意味する。携帯メールを打ち込む際、主に親指を使うことから「親指だけの友達」となり、これを略したのが親トモである。 名刺を何枚も持つことで安心するサラリーマンやメモ帳のアドレス欄がうまることで安心する大人がいるのと同様に、現代の若者は親トモが携帯電話のアドレス帳に大勢いることで安心感を覚え、携帯メールの着信音が全く鳴らないことに不安を感じるという。携帯メールのやりとりをすることで人と繋がっているという安心感を覚えているのは若者に限ったわけでなく、携帯メールや携帯サイトの普及とともにそういった大人も増えている。	2005年	若者言葉	
\\	オリ画	おりが	オリジナル画像の略。	オリ画とはオリジナル画像の略で、インターネットのウェブサイト(主にアダルトサイトや同人系サイト、画像掲示板サイト)で使われる言葉である。例えばサイト内の画像がオリ画と謳っている場合、そのサイト内の画像はサイト管理人らが撮影 ・ 描くなどしたオリジナル画像であることを意味し、画像掲示板で「オリ画以外禁止」と謳われていた場合、投稿者は自身が撮影 ・ 描くなどしたオリジナル画像しか添付投稿してはいけないことを意味する。また、オリジナルの画像であっても他サイトで既に公開したことのあるものはオリ画に含まない場合がある。	2005年	若者言葉、
\\	用語(スラング)、略語	
\\	おるず ・ おるつ	失意を意味するアスキーアート(文字絵)の一種。	
\\	とは失意や落胆、挫折を意味するアスキーアートの一種で、失意体前屈というが、「おるず」「おるつ」「おあず」などの読みで浸透している。アスキーアート
\\	とはパソコンや携帯で使われるアスキーコードの文字や記号を使ったアートのことで、顔文字もアスキーアートの一種である。
\\	の場合、
\\	が垂れた頭、
\\	の縦線がうなだれた手、横に流れる部分が背中、
\\	は地面に膝をつけている足にあたり、全体で落胆し、手と膝をついてうなだれているさまを描いている。失意体前屈は他に 
\\	など数種類あるが、
\\	は打ち込みが楽といった理由からか、ブログやメール、
\\	、掲示板など広く使われている。	2005年	ネットスラング	
\\	かまやつ女	かまやつおんな	かまやつひろしのような風貌の女子のこと。	かまやつ女とは夏なら古臭いおじさんみたいな帽子・冬ならニット帽をかぶって、髪はもっさり、服装はラフでゆったりしたものを着た女子のことである。2003年、その風貌が歌手の“かまやつひろし”に似ていることからカルチャースタディーズ研究所がかまやつ女と命名。原宿、下北沢で見かけることができるというかまやつ女。2005年には書籍のタイトル に、2006年には『負け犬』に次ぐ女性をさして使う俗語の候補として『干物女』とともに話題となる。	2005年	若者言葉	
\\	ガン見せ	がんみせ	ファッションとして肌を露出すること。	ガン見せとは「ガンガン見せる」の略で、女性(ギャル・コギャル)がファッションとして肌を露出することを意味する。また、肌を露出したファッション自体やそういったファッションをした人のこともガン見せという。	2005年	若者言葉	
\\	ギザ	ぎざ	「凄く~」「とても~」といった意味で強調する言葉。	ギザとは中川翔子(通称:ショコタン)が使う“しょこたん語”の一つで「凄く~」や「とても~」のように後に付く言葉を強調するものである。例えば「ギザカワユス(とてもかわいいの意)」「ギザうれしす(とても嬉しいの意)」といった形で使う。中川翔子本人が
\\	やラジオで「ギザはパソコンの大きい単位から出来た言葉」といっている通り、国際単位系の接頭語である
\\	(ギガ)が音的に崩れたものである。ただし、掲示板サイト2ちゃんねるではギガを強調語として以前から使っており、ギザは2ちゃん用語におけるギガの派生語と考えられる。	2005年	若者言葉	
\\	機変	きへん	携帯電話の機種変更のこと。	機変とは機種変更の略で、携帯電話の回線契約・電話番号を継続したまま、端末だけ変えることをいう(=機種変)。2006年から携帯電話の番号ポータビリティが実施され、電話番号を引き継いだまま、他の携帯電話会社へ契約を移転出来るようになったが、番号が変わらなくても他社の端末へ変更する場合は機変(機種変更)とは呼ばず、新規になる。	2005年	若者言葉	
\\	教中	きょうちゅう	教室の真ん中で告白すること。	教中とは「世界の中心で、愛を叫ぶ」という映画の俗称『セカチュー』からきたもので、「教室の中心で、愛を叫ぶ」を略したもの。みんなが居る教室の真ん中で、意中の人に愛の告白をするという意味だが、実際にそういったシチュエーションで告白するのは希で、教中は一種の罰ゲームとして行われるものである。	2005年	若者言葉	
\\	キラースマイル	きらーすまいる	一瞬で悩殺 ・ 魅了してしまうような笑顔のこと。	キラースマイルとは一瞬で人を虜にしたり、骨抜きにしてしまうような笑顔 ・ 微笑みのことで、女性がそういった男性の笑顔に対して使うことが多い。特に微笑みの貴公子とも呼ばれるペ ・ ヨンジュンなど韓流スターの笑顔に対して使われる他、笑顔が人気の元となっているアイドル、俳優といった国内外の有名人に対して使われることが多く、一般を対象にはあまり使われない。	2005年	
\\	クールビズ 
\\	くーるびず	温暖化防止を目的とした夏の軽装スタイルのこと。	"クールビズとは英語の
\\	(クール)""とビジネスを略した造語
\\	(ビズ)""をから成る和製英語で、「夏を涼しく過ごすための新しいビジネススタイル」という意味である。温暖化防止の狙いから夏の冷房設定温度を28
\\	にし、28
\\	でも快適に過ごせる軽装(上着なし ・ ノーネクタイ)を推進したもので、この運動自体をさしてクールビズともいう。平成17年(2005年)2月の京都議定書の発効を受け、環境省が名称を公募。6月1日にスタートしました。 クール ・ ビズ以前に『省エネルック』というものもあったが、当時推奨された『半袖スーツ』の見栄えからかほとんど普及しなかった。
\\	2005年	政策語 ・ 和製英語	
\\	クラシックパンツ	くらしっくぱんつ	ふんどしのこと。	クラシックパンツとは褌(ふんどし)のことである。百貨店の三越では越中褌(えっちゅうふんどし)に高級感を持たせるため、クラシックパンツという名称で1970年代から販売していた。2005年4月13日にフジ
\\	系のバラエティ番組『トリビアの泉』でこのことが取り上げられ話題となる。放送が父の日前ということもあり、同年の売上げは急激に伸び、クラシックパンツという名で褌が再評価された。	2005年	
\\	グラドル	ぐらどる	グラビアアイドルのこと。	グラドルとは「グラビアアイドル」の略で、雑誌のグラビア、ポスターを中心に写真集やイメージビデオなどが活動の主であるアイドル、またそうした活動から知名度をあげていったアイドルをさす。ちなみにグラビアアイドルのグラビアとは本来写真画像の印刷に特化した印刷技法の名称である。ここから雑誌巻頭などグラビア印刷で刷られた写真ページをグラビアページと呼ぶ。ただし、現在では印刷技法がグラビア印刷であるか否かに関係なく、女性アイドル(グラドル)による写真ページのことをグラビアページと呼ぶ。	2005年	若者言葉	
\\	ゲーム婚	げーむこん	オンラインゲームがキッカケでする結婚のこと。	ゲーム婚とはゲームの中の仮想結婚という意味ではなく、オンラインゲームをキッカケに知り合った男女が結婚することで、正確にはオンラインゲーム婚という。パソコンゲームやゲーム機器の中にはインターネットに接続し、コンピュータでなく人と対戦出来るものや多数の人と一緒に遊べるタイプのものがある。チャットをしながらゲームをする中で惹かれあったり、互いのキャラクターがゲーム内で仮想結婚~結婚生活をする中で惹かれあい、現実にもゲーム婚に辿り着く。外見や肩書きに惑わされることなく、時には本性(本音)が垣間見えるゲームで知り合うゲーム婚のほうが本当に性格が合うか知った上で一緒になれていいといった意見もある。	2005年	若者言葉	
\\	ゴギる ・ 呉儀る	ごぎる	直前に約束をキャンセルすること。	ゴギるのゴギとは中華人民共和国の女性政治家:呉儀副総理のこと。2005年5月23日、呉儀副総理は来日し、小泉元首相と会談を予定していたが緊急公務のため急遽帰国。帰国直前の日本経団連の奥田碩会長との会食は予定通り行われ、話題となった。ここからドタキャンすること(=直前に約束をキャンセルすること)をゴギるという。	2005年	若者言葉	
\\	コスプリ	こすぷり	コスプレプリクラの略。	コスプリとは「コスプレ」と「プリクラ(プリント倶楽部)」から成る合成語で、コスプレをした状態で撮ったプリクラをを意味する。単にコスプレ姿を収めたコスプリもあるが、文字入れが可能なプリクラ機を使用し、コスプレに合うセリフを入れた1コマ漫画風のコスプリが多い傾向にある。	2005年	若者言葉	
\\	こっさり	こっさり	こってりとあっさりの中間の味のこと。	こっさりとは「こってり」と「あっさり」から成る合成語で、こってりとあっさりの中間的な味を意味する。こっさりは元々そういった味のラーメンに対して使われた言葉で、ラーメン通の間で普及していたが、2005年辺りから日本酒やワインなどの酒類、和洋中のスープ、カレーなどジャンルを越えて使われるようになる。また中間の味以外に、見た目はこってりしているが味はあっさりといったもの、口に入れた瞬間はこってりとコクがあるが後味はさっぱりしたものなどもこっさりという。	2005年	
\\	サギ写	さぎしゃ	別人のように美しく撮られた本人による撮影画像のこと。	サギ写とは詐欺写真の略で、写真に写る当人(モデル)を目の前にしても誰の写真なのか判断出来ないような、別人のように美しく撮れている画像をいう。画像種類の限定はなく、カメラ、カメラ付き携帯による写真、さらに街角に設置されたセルフの証明写真(スピード写真)やプリクラなど様々。撮り直しが容易なデジカメ、カメラ付き携帯電話の普及により、自分が美しく見える角度や表情を研究 ・ 把握している人(主に女性)も多く、出来の良いサギ写
\\	を撮るため、日夜研究している人もいる(逆に偶然美しく撮れた写真がサギ写扱いされる場合もある)。	2005年	
\\	授かり婚	さずかりこん	子供を授かった後にする結婚のこと。	授かり婚とは子供を先に授かった状態でする結婚のことで、実質的には妊娠したことが引き金でする結婚である場合が多い。1990年代に『できちゃった婚』という同義語が普及。増加傾向にある『できちゃった婚』に対し、結婚産業が「できちゃった」というマイナスイメージを払拭するために考えた言い換えが授かり婚である。一般にはあまり普及していないが「ママ」と「マリッジ」を合成した『ママリッジ』という言葉も業界では使われている。	2005年	
\\	サンダる	さんだる	深夜、コンビニエンスストアへ行くこと。	サンダるとは履物のサンダルの語尾を動詞化する接尾語『~る』にしたもので、深夜にコンビニエンスストアへ行くことである。サンダルの動詞化であれば、サンダルを履いて出かけることを意味し、場所は関係ないように思える。しかし、サンダるはお笑いコンビ:トータルテンボスが2004年末の
\\	-1グランプリで見せたネタから来る言葉であり、そのネタに沿って深夜にコンビニへ行くという意味でのみ使われる。(ただし、若者言葉として今後普及する際に意味を広げる可能性はある)	2005年	若者言葉	
\\	時差家族	じさかぞく	妻(母)の仕事ですれ違いが多い家族のこと。	時差家族とは妻(母)の仕事(パート、アルバイト)ですれ違いが多くなった家族のことで、コンビニ、ファミレスなどサービス業の24時間営業化に伴い、主婦の労働力が深夜~早朝にまで及ぶようになったことに起因する。時差家族は2005年11月に日本経済新聞で使用された言葉である。 時差家族というと悪いイメージが先行しがちだが、子供を保育園に預けず家族で面倒が見れるという利点もあるという。昼間は妻が面倒を見て、夫が帰宅後、バトンタッチしてから妻が仕事にいくという形態である。ただし、妻の睡眠時間なども考えると、夫婦の労働時間帯がよほど恵まれた稀なケースではある。どちらにしても時差家族には夫婦の会話や家族団欒の食事は少なく、課題の残る形態であることに変わりはないようである。	2005年	
\\	社内ニート	しゃないにーと	会社勤めしているが、労働意欲が全くない人のこと。	社内ニートとは就職し、会社に通ってはいるものの、何らかの理由で労働意欲を失ってしまっている人のことで、仮面ニート ・ 隠れニートともいう。社内ニートの中には、希望する職種や企業に勤めることが出来なかったなどの理由でもともと働く意思のない者と、本来は働く意欲があったが企業や労働内容 ・ 労働条件 ・ 賃金などの問題で労働意欲を失ってしまった者の2種類がある。ニート=若者というイメージが強いが、終身雇用制が崩壊し、将来の不透明さから社内ニートになる30~40代も少なくない。社内ニートの中には転職意欲さえも失い、毎日出勤するだけで何もしない者もある。また、社内ニートは退職型ニートの予備軍ともいわれており、社会問題になっている。	2005年	
\\	ジャパニメーション	じゃぱにめーしょん	日本製のアニメーションのこと。	"ジャパニメーションは英語の
\\	(日本)""と
\\	(アニメーション)""から成る合成語で「日本製のアニメーション」を意味する。 ジャパニメーションは
\\	""という綴りで1990年代から使われいたアメリカン ・ スラングが日本に渡ったカタカナ語である。ただし
\\	""は日本人の蔑称である
\\	""と
\\	""という分け方も出来ることから、アメリカでは偏見的な意味合いを含んで使われることが多く、日本では単にアニメーションの略として使われている
\\	(アニメ)""が一般的には日本製のアニメという意味で使われている。
\\	2005年	カタカナ英語	
\\	就活メイク	しゅうかつめいく	就職活動で優位に立つためのメイクのこと。	就活メイクとは就職活動用メイクの略で、就職活動(主に面接)において優位に立つためのメイクのこと。リクルートメイクともいう。就職戦線氷河期といわれる中、女性の就職活動ではメイクによる印象作りも重要視されるようになり、生まれた言葉である。学生にとっては父親世代の面接官に好印象をもってもらわなくてはならない。流行の目元を強調する濃いアイメイクなど、派手なメイクは敬遠される。かといって、薄すぎると印象に残りにくくなる。就活メイクの基本は清楚 ・ 知的 ・ フレッシュといったキーワードに沿ったもので、雑誌や
\\	サイトに特集が組まれたり、合同企業説明会で指導が行われることもある。	2005年	美容 ・ ファッション用語	
\\	熟年離婚	じゅくねんりこん	長い結婚生活の末にする離婚のこと。	熟年離婚とは長い結婚生活の末にする離婚のことである。この場合の「長い」とは人により解釈が異なるものの、一般的には20年以上とされている。「熟年離婚=熟年者の離婚」と思っている人もいるが、例えば55歳で結婚し、65歳で離婚したとしても熟年離婚とは呼ばない。また、籍が入っているだけで別居していた中高年夫婦の離婚も熟年離婚とは呼ばない。これは長く連れ添った夫婦が夫の定年退職などを機に出来た不満やそれまで我慢していた不満による離婚という意味で使われるためである。 1990年代後半から中高年齢層の離婚がメディアで取り上げられるようになり、熟年離婚という言葉も頻繁に使われた。特に2005年10月には渡哲也、松坂慶子が夫婦役のドラマのタイトルとなり、他の同意語や類語を一本化するほどの勢いで流行語的に使われた。	2005年	
\\	スイーツ	すいーつ	ケーキなど甘いお菓子のこと。(日本語俗語辞書サイトポリシー)	"スイーツとはケーキやプリンなど甘いお菓子のことで、英語(イギリス圏)で甘いデザート ・ お菓子を意味する
\\	(米語では
\\	""からきたカタカナ英語である。 日本でのスイーツは当初、有名店のパティシエによる高級洋菓子など子供のお菓子と区別し、大人が味わって食べるお菓子という意味で使われていた。2005年に入ると女性雑誌を中心にスイーツがブームとなる。 このブームによりスイーツという言葉が普及していく中で、国内の菓子メーカーもスイーツ ・ ブームに参入。「お手軽スイーツ」や「和風スイーツ」といった形でスイーツは甘いお菓子全般を意味する言葉となる。
\\	2005年	カタカナ英語	
\\	スローフード	すろーふーど	ファーストフードに対して作られた概念及びその概念に沿った運動のこと。	スローフードとはイタリアの
\\	(ブラ)という町で1986年に起こったスローフード運動という
\\	=非営利運動)のことや、その運動の根源である「その土地や季節の食材、料理を食べ、食文化を守ろう」という概念のことである。 また、特に日本では「食事の時間くらいはゆっくりとろう」という意味でもスローフードという言葉が使われる。	2005年	カタカナ英語	
\\	スローライフ	すろーらいふ	ゆったり過ごすライフスタイルのこと。	スローライフとはイタリアで起こったスローフード運動から派生したもので、効率や利便性を追求する現代人に、あらためて自分自身、また自分の生活を見直そうといった考え方である。スローライフは毎日忙しい生活をしている日本人ならではの言葉である。スローライフはあくまで普段の生活の中で他人や社会に翻弄されないゆったりとした時間を過ごし、個々の衣食住に関して考えなおしてみようというもので、田舎での生活を指すものではない。スローライフの具体例を下記に掲載しておきますのでご参照ください。	2005年	和製英語	
\\	席朝族	せきあさぞく	朝食を会社に着いてから食べる人のこと。	席朝族とはアサヒ飲料株式会社による造語で、朝食を出社し、会社の自席で食べる人のことをいう。同社はワンダブランドからモーニングショット発売に合わせ、全国9都市の若手ビジネスマン1,200を対象に朝の実態調査を実施。実に1/3が席朝族であることがわかったという。ただし、会話の中でそういった人を指して席朝族と呼ぶほどには浸透していない。	2005年	
\\	絶対領域	ぜったいりょういき	ミニスカートとオーバーニーソックスの間の太股の部分。	絶対領域とはアキバ系などに好まれる女性ファッションの部位で、ミニスカートとオーバーニーソックス(オーバーニー参照)の間に見える太股部分のことである。ミニスカートの丈:絶対領域:オーバーニーソックスの膝上部分の比率は4:1:2.5がベストとされており、絶対領域といわれる絶対的な効力を発揮するためには、この比率の
\\	以内でなければならない。 絶対領域はもともとは
\\	アプリケーション、アニメ、ゲームのキャラクターの中に見られ、
\\	サイトや同人誌などで使われていた言葉だが、
\\	などメディアによるアキバ系の解説の際、メイド喫茶の紹介と合わせて絶対領域の説明も行われようになり一般的にも浸透した。 株式会社バンプレストが2005年8月9日に『絶対領域』を商標登録出願。	2005年	若者言葉	
\\	タカラ	たから	大勢でカラオケに行くこと。	タカラとは「多人数でカラオケに行く」を略したものである(この場合の多人数が何人以上といった定義は特に定められていない)。本来、カラオケは複数人(大勢)でする娯楽であり、タカラという言葉をあえて用いる必要なかった。しかし、2005年になると一人でカラオケに行く人が増加。こうして一人でカラオケに行くことをヒトカラと呼ぶようになったことから、その対語としてタカラという言葉が使われるようになった。ただし、先に説明した通り、カラオケに行く=タカラ的な要素を含んでいるためか、ヒトカラのようには普及していない。	2005年	
\\	ダブルハッピー婚	だぶるはっぴーこん	子供を授かった後にする結婚のこと。	ダブルハッピー婚とは子供を授かった状態でする結婚(妊娠を機にする結婚)のことで、結婚と妊娠というダブルのハッピーを意味する。こうした妊娠発覚後の結婚を1990年代には『出来ちゃった婚』と呼んだが、主催者側にとって印象が悪いことからブライダル産業各社が『出来ちゃった婚』に変わる言葉を考案。ダブルハッピー婚はそのひとつである。ダブルハッピー婚以外に『おめでた婚』『授かり婚』という言い回しも使われる。	2005年	
\\	ダルカジ	だるかじ	ヨレヨレでだらしない服。家の近くへちょっと出かける際に着る服。	ダルカジとは「しまりがない」という意味の『怠い(だるい)』または「怠い」を畳語(註参照)にした『ダルダル』と『カジュアル』から成る言葉で、ヨレヨレになっただらしない服。または家で着たり、近所のコンビニエンスストアに行くときなど、ちょっとした外出の際に着る服をいう。 註)畳語:「多々」「国々」「たまたま」など同じ成分を2回重複させた言葉	2005年	若者言葉	
\\	チープリ	ちーぷり	安価だがカワイイこと。	"チープリとは安価という意味の英語
\\	(チープ)""とカワイイという意味の英語
\\	(プリティ)""による合成語で、安価だがカワイイという意味で使用。主に女性向けファッションやコスメ、女性誌で用いられる。
\\	には安価とは別に安っぽいという意味もあるが、同様にチープリも安っぽいけどカワイイといった意味で使われることがある。
\\	2005年	若者言葉、和製英語	
\\	直留守	ちょくるす	携帯電話にかけた際、直接留守番電話サービスに繋がること。	直留守とは携帯電話に対して電話をかけた際、呼び出し音が鳴らず、数秒の無音状態後、いきなり留守番電話サービス(伝言サービス)に繋がることをいう。直留守は「直接留守電」の略だが、他に直留守電という略し方もある。受話側の携帯電話の電源が切れていたり、圏外の際、またはそういった設定となっている場合に起こる。	2005年	
\\	ツンデレ	つんでれ	ツンツンデレデレ(キャラ)の略。	ツンデレとはツンツンデレデレの略で、ギャルゲー(恋愛物のアドベンチャーゲーム)やエロゲーにおけるヒロイン ・ キャラクターの属性の一つである。ツンデレはゲームのストーリー開始時~普段はツンツンしているが、二人きりになったときや、恋人になったときなど、なにかのきっかけで急にしおらしくデレデレしてくるキャラクターのことをいう。 ツンデレの起源は『あやしいワールド』という匿名掲示板で、2002年に書かれた「ツンツンデレデレキャラに弱い、萌え」という投稿にはじまる。当初、ツンデレはツンツンからデレデレに時間軸で変わっていくキャラクターに使われ、現在のようなツンツンとデレデレの2面性をもったキャラクターという意味ではなかった。2002年末には『2ちゃんねる』に「ツンデレ派集まれ」というスレッドが立ち、エロゲーを中心とした一部のコミュニティに浸透していく。2005年に入り、メディアもツンデレを取り上げるようになり、萌え属性としては『メガネっ娘』に次いで一般にも浸透。今後、キャラクターのみでなく、一般的な女性に対して使われる可能性を秘めている。 また、ツンデレは『悪友タイプ』『主人公嫌悪タイプ』『素直になれないタイプ』『孤立タイプ』『高飛車タイプ』『堅物タイプ』などの細分化が成されている。	2005年	若者言葉、略語	
\\	デイする	でいする	デイトレードすること。	デイするとは「デイトレードする」を略したものである。デイトレードとは株や為替、通貨などの金融商品の売買を一日の間に何度も行い、その日の間に売買を完結させる取引方法のことである。日本語では「日計り商い」という。インターネット取引の整備や手数料の自由化などをうけ、日本でもデイトレーダー(デイする人)が増える。特に2003年~2005年にはデイトレードに関する書籍の販売や
\\	での紹介からデイトレーダーが急増。カリスマデイトレーダーと呼ばれる人も話題となった。ただし、2006年のライブドア事件による市場の暴落(いわゆるライブドア・ショック)以降は減少傾向にある。	2005年	略語	
\\	デパ屋	でぱおく	デパートの屋上のこと。	デパ屋とは「デパートの屋上」の略で、デパート屋上にある施設やショップなどのスペースを意味する。昭和時代、デパ屋といえば子供向けの遊戯施設とレストランが定番であった。後にビアガーデンや映画の上映といったものに移行していくが、特に差別化のためのスペースというわけではなかった。2005年、日本橋高島屋、銀座三越が犬用のスペースを設け話題になる。他にも新宿にある京王百貨店のお洒落なカフェ『スカイカフェ』や三越日本橋店の庭園さながらのガーデニング売り場『チェルシーガーデン』など、それぞれの色を出したスペース作りがされる。こうしてデパートの屋上は『デパ地下』同様にデパートの特徴付をし、デパ屋という言葉とともに再注目されるようになった。	2005年	略語	
\\	電車男	でんしゃおとこ	"掲示板サイト""2ちゃんねる""から生まれた恋愛ストーリーのこと。(日本語俗語辞書サイトポリシー)
\\	"電車男とは掲示板サイト""2ちゃんねる""での書き込みを基にした恋愛ストーリーのことで、投稿者のハンドルネームからきている。ここから通勤・通学の電車で恋に落ちた男性を『電車男』、更に出会った場所によって
\\	男(図書館男、バス男など)』といった形で使われている。 電車男は2004年3月14日、""2ちゃんねる""の独身男性板(通称:毒男板)内のスレッドに、アキバ系ヲタクで彼女いない歴22年=当時の年齢という""731(後の電車男)""の「酔っ払いに絡まれていた女性を助けたら、お礼を言われ、住所を聞かれた(実際の書き込みはかなりの長文で詳細が書かれている)」といった書き込みから始まった。その数日後、助けた女性から『エルメスのティーカップ』がお礼として届いたことから、相手の女性は『エルメス』と呼ばれるようになる。このティーカップについていた伝票にエルメスの電話番号が書かれていたが、ダイヤル出来ずに悩む電車男をスレッドの住人が応援 ・ アドバイスする。この辺りから徐々に盛り上がりをみせ、電車男を応援するスレッドが派生、""731こと電車男""~""電車男""とハンドルネームも固定のものになっていく。以降もスレッド住人に支えられながら一歩づつエルメスに近づいていき、同年5月9日に告白。電車男とエルメスは結ばれるというストーリーである。
\\	2005年	若者言葉	
\\	同志	どうし	男性同性愛者のこと。(俗語的解釈)	同志とは男性同性愛者(一般的にはゲイと呼ばれる人)の俗称である。これは中国語で同性愛者を同志と書くことからきている。ただし、日本で広まった理由は中国のコンドームメーカーが同志という商品名の男性同性愛者用コンドームを発売したと報道したことによる。この報道により一時的に同志と呼ばれたが、それほど浸透しておらず、男性同性愛者の中にも知らない人は多い。	2005年	
\\	ドルチェ	どるちぇ	イタリアのデザートやお菓子。(日本語俗語辞書サイトポリシー)	"ドルチェとはイタリア料理におけるデザートやお菓子のことで、イタリア語の
\\	""からきたカタカナ伊語である。日本でも浸透しているドルチェの代表的なものにパンナコッタ、ティラミス、タルトがある。これらが浸透していく中でドルチェという言葉も女性を中心に普及。2005年にはスイーツ ・ ブームに乗り、ドルチェも女性雑誌などで特集が組まれるようになる。 本来
\\	""には「甘い」「やさしい」「柔らかい」という意味があり、そこからイタリアではお菓子のこともドルチェ(甘美)と呼ぶようになった。このため、イタリア人に醤油で味付けされた『せんべい』などをお菓子として出すと違和感をもった表情をされることがある。
\\	2005年	カタカナ語	
\\	妊娠	にんしん	任天堂に対し、狂信的なユーザーのこと。(俗語的解釈)	妊娠とは本来、女性が体内に受精卵またはそれが発育した胎児を包容している状態を意味するが、これとは別に任天堂信者という意味でも使われる。任天堂信者を任信に略し、それをあてたもので、任天堂(任天堂商品)を狂信的に支持するユーザーを意味する。2ちゃんねるのハード ・ 業界板で、そういったユーザーを侮蔑する言葉として使われる。妊娠という表記での使用は妊婦に失礼であるということから、任信の字のまま使ったり、任豚(妊豚)といった言い回しも使われる。	2005年	ネットスラング	
\\	パンコ	ぱんこ	性的にふしだらな女性のこと(=やりまん)。	パンコとは戦後の進駐軍兵士相手の娼婦を意味するパンパン(またはパンパンガール、パン助;下記関連語参照)を女性名っぽくしたもの。パンパンの派生ということで、一部には売春 ・ 援助交際をする女性 ・ 女子という意味でも使われるが、パンコといった場合、大抵は簡単に性行為を許してしまう女性という意味(尻軽女、やりまんと同義)で使われることが多い。	2005年	
\\	非乙女	ひおとめ	恋をしていない女性のこと。	非乙女とは乙女ではない人という意味だが、ここでいう乙女とは「恋する乙女」にみられるような恋をしている女性、または乙女のようにときめいている女性を意味する。つまり、非乙女とは恋をしていない女性、ときめいていない女性を意味する。	2005年	若者言葉	
\\	ヴィクトリアン	びくとりあん	ファッションスタイルや美術様式の一種。	ヴィクトリアンとは、英国のビクトリア王朝時代(1837~1901年)のファッションや家具、美術をモチーフにしたスタイルのことで、「ヴィクトリア調」、「ヴィクトリア風」といった意味である。ファッションの世界でいうヴィククトリアン(ヴィクトリア調)は、夫を亡くしたビクトリア女王が長期間喪に服していたときに身に着けていた服装を、一般が真似たことから生まれたスタイルである。 ヴィクトリアンはヨーロッパコレクションでも何度か登場しているが、日本では2005年秋にトレンド ・ ファッションとして紹介され、スカートの裾やブラウスの襟などにレースやベルベットを施したヴィクトリア調のものが流行している。	2005年	カタカナ英語	
\\	ビジョナリー	びじょなりー	先見の明がある人のこと。	"ビジョナリーとは英語
\\	""からきているカタカナ語で、2005年辺りからビジネスの分野で使われるようになる。
\\	""は直訳すると「先見の明のある」「概念的な」「幻想的な」という意味になるが、ビジネス分野におけるビジョナリーとは、先進的 ・ 独創的なビジョンを現実化し、社会に大きな影響(貢献)をした経営者のことである。つまり、現存する市場や商品、サービス、技術を駆使して経営するのみでなく、新たな市場や商品、サービス、技術といったものを模索~具現化することで成功し、更にそれが社会的にも影響力を与え、尊敬される経営者である。 例えば、馬車が主流の時代に自動車を大量生産し、普及させたヘンリー ・ フォードなどがそれにあたる。日本では孫正義氏など
\\	長者に使われることでビジョナリーという言葉は普及したが、具体的にどういったビジョンで、どこまでの功績を残せばビジョナリーと呼べるといった定義があるわけではなく、ビジョナリーという言葉を使う人の価値観や見解で大きく異なる。 また、「ビジョンを持っている企業」「未来志向の企業」「先見的な企業」をさすビジョナリーカンパニーや、「将来ビジョンのある生活」を意味するビジョナリーライフといった言葉もある。
\\	2005年	カタカナ英語	
\\	ヒトカラ	ひとから	一人でカラオケに行くこと。	ヒトカラとは一人でカラオケに行くことで、「一人カラオケ」がさらに略された言葉である。通常、二人以上の複数による娯楽として普及したカラオケだが、ストレス発散、歌の練習、人前では苦手だが歌自体は歌いたいといった理由からヒトカラをする人が2005年辺りから増えてきた。しかし、元来、複数人で利用されることの多いカラオケボックス。店舗側はこうした複数での来店を前提に料金設定しているところも多く、中にはヒトカラを断る店もある(逆にヒトカラ歓迎が売りの店舗もある)。また、こうしたヒトカラをする人のことをヒトカラーという。 なお、ヒトカラは株式会社
\\	の登録商標です(登録日:2005/4/22)。	2005年	
\\	ヒラリーマン	ひらりーまん	何の役職もないサラリーマンのこと。	ヒラリーマンとは役職についていない一般社員を意味する「平社員」と「サラリーマン」から成る合成語で、役職についていないサラリーマンを指す(平社員と同義)。ヒラリーマンが使われ始めた時期 ・ 出所などは不明だが、広く普及した要因は2005年の衆議院議員総選挙(俗にいう郵政解散に伴う総選挙)で小泉チルドレンとして初当選した杉浦大蔵議員による。当選後のインタビューで自らを「地味なヒラリーマンだった」と称したことから世代を超え、広く知られる言葉となった。また、これを略してヒラリーともいう。	2005年	合成語	
\\	ヒルズ族	ひるずぞく	六本木ヒルズ内のマンションに入居している企業の経営者のこと。	ヒルズ族とは東京 ・ 港区にある六本木ヒルズ内のマンションに入居している企業の経営者をさし、ヒルズ長者とも呼ばれる。主にIT企業の若い経営者やベンチャー企業家など、いわゆる「勝ち組」に対して使われるが、広義では六本木ヒルズのオフィスで働いている人全般をさす場合もある。 2003年に開業した六本木ヒルズであるが、このとき
\\	ベンチャー企業が相次いでオフィスを設けた。それら企業の経営者が2005年に入り、ニュースやワイドショーなどメディアでの露出が増える中でヒルズ族と呼ばれるようになった。 ヒルズ族の代表的な人には三木谷浩史(楽天)、藤田晋(サイバーエージェント)などがあげられる。	2005年	
\\	ブサイケ	ぶさいけ	嫌になるほど厚かましいこと。	ブサイケは『ブサイク』と『イケてる』から成る合成語で、背も低く小太りで外見はブサイクなのに、きれいな女性を連れているようなイケテル男性をさす。なぜブサイクなのにモテるのか?これは主に知識や話題が豊富というところにある。外見がいくらよくても知識がなかったり、話題を持っていない男性は飽きるのに対し、外見はブサイクでも話題が豊富で会話がはずむ男性や博識な男性のほうが魅力的と捉える女性にブサイケはモテる。ブサイケの中でも、特にパソコンやインターネットに詳しい男性の人気があがっている。これは単に物を知っているということ以上に、パソコンやインターネットを使う女性が増えた中で「困ったときに助けてくれた」ということが頼もしく、また優しく映ることから人気が高いようである。	2005年	若者言葉	
\\	フラゲ	ふらげ	発売日より前に商品を入手すること。	フラゲとは競争競技でスタートの合図よりも先にスタートすることを意味する和製英語『フライング(英語では
\\	と「得る」「受け取る」という意味の英語
\\	(ゲット)』を組み合わせた『フライング ・ ゲット』を略したもので、
\\	やゲーム、雑誌などを発売日よりも前に入手することを意味する。フラゲは発売日の前日以前に入荷した商品をすぐ(発売日前)に販売する店での購入や個人的に店員と仲良くなり、入荷後すぐに売って貰うといった形で行われる。また、インターネット上にはフラゲ出来る店の情報を交換する
\\	サイト(掲示板)も存在する。	2005年	和製英語	
\\	フリ卒	ふりそつ	フリーターをしていた人が就職活動し、正社員雇用で勤め始めること。	フリ卒とはフリーター卒業の略で、アルバイトやパートタイマー、派遣社員など正社員以外の雇用形態による収入で生計をたてていた人が、就職活動をし、正社員として勤め始めることをいう。フリ卒はリクルート
\\	の「そろそろフリ卒フロム ・ エー」というキャッチコピーで普及(フリーターもフロム ・ エーの造語)。正社員を集う企業の募集パンフレットやホームページにも使われている。	2005年	
\\	ポリ電	ぽりでん	110番のこと。	ポリ電とは警察を意味する英語
\\	(ポリス)』と、『電話(または電話番号)』から成る合成語で、警察の電話番号。つまり、警察機関への緊急通報用電話番号である110番(ひゃくとおばん)を意味する。	2005年	
\\	マルガリーマン	まるがりーまん	丸刈りのサラリーマンのこと。	マルガリーマンは『丸刈り』と『サラリーマン』から成る合成語で、丸刈りのサラリーマンのことである。 サッカーのデビッド ・ ベッカムや高原直泰、三都主アレサンドロ、野球の清原和博や阿部慎之助など、スポーツの世界で丸刈りにした選手が目立ち、そういった選手のファンが真似をするようになり、丸刈りが俄かにブームとなった。サラリーマンの丸刈りに関しては、過去に大きな失敗をした社員が反省の意味を込めてしていた髪型でもあるため、「清潔感があっていい」という意見と賛否両論話題となり、マルガリーマンと呼ばれるようになった。	2005年	若者言葉	
\\	ミーツ	みーつ	友人が集うこと。	ミーツといえば、大抵は「会う」という意味の英語“
\\	の変化形)”のことだが、ここでは「集合」「会議」を意味する“
\\	(ミーティング)”の略で、友人が集まることを指す若者言葉である。「ミーツ5時(5時に集合の意)」「ミーツ渋谷(渋谷に集合の意)」といった使い方もする。	2005年	若者言葉	
\\	みのる	みのる	よく働くこと。	みのるとはタレント『みのもんた』を略したものに、動詞化する接尾辞『る』を付けたものである。みのもんたといえば、フリーアナウンサー、司会者として数多くのレギュラー番組を持ち、2006年には一週間でもっとも長時間、テレビの生番組に出演する司会者としてギネスにも登録されている。また、こうしたタレント業以外にも父親が経営する会社を継ぎ、代表取締役社長になるといった実業家としての顔もあることで知られる。そんな“みのもんた”のように、多くの仕事をバリバリとこなしていくことをみのるという。	2005年	若者言葉	
\\	ミリオネーゼ	みりおねーぜ	年収1000万円以上の女性のこと。(日本語俗語辞書サイトポリシー)	ミリオネーゼとは年収1000万円以上の女性のことで、株式会社ディスカヴァー ・ トゥエンティワン(以下、
\\	が2003年に女性向けビジネス本を翻訳する際、『ミリオネーゼになりませんか?』というタイトルで使った造語である。
\\	の定義では「自分の才覚による年収が1000万以上で、経済的にも精神的にも自立し、恋も遊びも楽しむおしゃれな女性のこと」となっており、玉の輿や親の七光りなど単にお金持ちというだけの女性や、自力で1000万円以上稼いでいても仕事だけの生活でプライベートがおろそかになっている女性は含まれない。2005年、ヒルズ族が注目される中、女性企業家もメディアで取り上げられるようになり、その際にミリオネーゼという言葉も普及した。 ミリオネーゼはシロガネーゼのもじりという解説を見かけるが、
\\	に問い合わせたところ、大金持ち ・ 百万長者を意味する英語
\\	(ミリオネア)』の女性形を作ろうという考えが発端で、イタリア語風にしようという案から生まれた言葉である。よってミリオネアのもじりと考えたほうが正しい。また、年収1000万円という定義に対し、なぜ『ミリオン(100万)』という言葉を使ったかだが、これはもともと「月収100万円」という考えからきており、後に年収1000万円という定義付けがされたためである。 なお、ミリオネーゼは
\\	の登録商標となっています。
\\	については下記参考サイト参照)	2005年	和製英 
\\	伊語	
\\	メトロセクシャル	めとろせくしゃる	都市部に住み、ファッションや美容に興味をもつが男らしさも失わない男性のこと。	"メトロセクシャルとは1994年にイギリスのマーク ・ シンプソンが作った造語で、
\\	(都会人)""と
\\	(異性愛者=ゲイでない男性)""の合成語である。メトロセクシャルを簡単にいえば「都会に住むお洒落な人」ということになるが、お洒落というよりは一歩突っ込んでファッションやスキンケアに気を使う男性をさす。ただし、ゲイや女装を趣味とすることとは違い、あくまでも男性としてのケアである。 このイギリスで生まれたメトロセクシャルは、後にマイケル ・ フラッカーというアメリカの作家によって紹介され、イギリス人に比べ田舎っぽさの残るアメリカ人に衝撃を与える。このマイケル ・ フラッカーの著書『メトロセクシャル 
\\	は2004年に翻訳され日本でも紹介されている。日本では元々トレンディといった形でメトロセクシャル的要素を備えていたことに加え、2003年9月に伊勢丹がメトロセクシャル層をターゲットにした『メンズ館』をオープンしていたことや資生堂が高級スキンケアブランド
\\	を立ち上げたことから、メトロセクシャルという言葉もスムーズに受け入れられた。
\\	2005年	カタカナ英語	
\\	メル画	めるが	携帯電話の待受け画像の一種。	メル画とは携帯待受画像の一種で、文字通り、携帯メールを開いたような画像である。携帯メール同様、最上部は「受信 
\\	で始まり、受信日時、送信者、件名、ラインが入って本文(絵文字・顔文字なども再現)といった構成を画像化したものである。また、多くは件名・本文に合う画像がバックに配されている。内容は友情論・愛情論・人生論など持論が書かれたものや、実際に受け取ったメールで思い入れがあるものなど様々だが、傾向として送信者に歌手やアーティスト名、件名に曲名、本文に歌詞が書かれたものが多い。こうした歌詞が書かれたものはメル画であると同時に歌詞画にも区分される。とは携帯待受画像の一種で、文字通り、携帯メールを開いたような画像である。携帯メール同様、最上部は「受信 
\\	で始まり、受信日時、送信者、件名、ラインが入って本文(絵文字 ・ 顔文字なども再現)といった構成を画像化したものである。また、多くは件名 ・ 本文に合う画像がバックに配されている。内容は友情論 ・ 愛情論 ・ 人生論など持論が書かれたものや、実際に受け取ったメールで思い入れがあるものなど様々だが、傾向として送信者に歌手やアーティスト名、件名に曲名、本文に歌詞が書かれたものが多い。こうした歌詞が書かれたものはメル画であると同時に歌詞画にも区分される。	2005年	若者言葉	
\\	萌え	もえ	アニメ ・ 漫画 ・ ゲームなどのキャラクターに対する愛情表現。	"萌えとはアニメ ・ 漫画 ・ ゲームに登場する架空の人物(キャラクターのことで、主に美少女系キャラ)に対する愛情や執着を表現した言葉である。萌えは1990年代初頭、""パソコン通信""の草分け
\\	というコミュニケーション媒体で、オタクを中心に使われるようになる。インターネットの普及により、オタク系コミュニティを中心に一般的にも広がり、2004年には『萌え』が流行語大賞でノミネートされるほどになった(流行語に関しては下記の『追加情報参照』)。更に萌え
\\	は
\\	(オタク)とセットで欧米にも広がり、「わび」「さび」に次ぐ日本人独特の精神として認知される。 後に萌えの対象はキャラクターのみでなく、萌えに属するファッション(メガネ、メイド服などのコスプレ)、萌えに値するパーツ(目や耳、足など)へと広がり、それら萌えに属する(値する)ファッションやパーツという意味の萌え属性、萌えツボといった言葉も生まれる。更に一般に広がる中で単純に物や異性に対する欲情を表現する言葉としても使われるようになり、萌え株、萌え業界、萌え本といった使われ方をするようになる。
\\	2005年	若者言葉	
\\	ゆるデコ	ゆるでこ	ゆるめの服をデコラティブに着こなすこと。	ゆるデコとは大きめのゆるい服を装飾的(デコラティブ)に着こなすファッション ・ スタイルのことで、それまでに流行した『ゆるカジ』の流れで生まれた言葉である(ただし、ファッション的にはレイヤードルック(重ね着)の流れで生まれたスタイルである)。『ゆるカジ』は緩めのカジュアルを着れば成立するが、ゆるデコはデコラティブという点がポイントとなり、重ね着などの装飾という要素が必要になる。	2005年	若者言葉	
\\	恋愛ニート	れんあいにーと	異性との交際を諦め、恋愛に興味を無くした者のこと。	恋愛ニートとは異性との交際を諦め、恋愛自体への興味を無くし、恋愛のために自分を磨くということをしない人のことで、職にも学業にも就かず、就こうという意思のない人を意味するニート
\\	の恋愛版である。(ニートについては関連ページ参照) 恋愛経験や恋愛に関する知識がない人の中には、彼女が出来たときにどうしていいかわからず、逆に努力の末にそんな煩わしい悩みを抱えるなら恋愛などしないほうが楽だと考え、出会いのための自分磨きをすることもなく恋愛ニートとなるという。	2005年	若者言葉	
\\	ロハス 
\\	ロハス	環境 ・ 自然 ・ 健康にやさしいライフスタイルのこと。(日本語俗語辞書サイトポリシー)	"ロハスとは
\\	""の頭文字をローマ字読みしたもので、直訳すると「健康と持続可能な社会を求めるライフスタイル」となる。ロハスの明確な定義付けはされていないが、自分自身の日常生活や身近な自然保護だけでなく、地球規模での環境や人間の健康を最優先するライフスタイルという意味で浸透している。 ロハスはアメリカの社会学者ポール ・ レイと心理学者シェリー ・ アンダーソンが15年間に渡る調査結果から1998年に発表した「新しい文化やライフスタイルを創造する生活創造者がアメリカの30
\\	を占め、それら生活創造者が望むライフスタイルがロハスである」という発言から生まれた言葉である。日本ではロハスという言葉よりも有機農法、リサイクル運動、企業における環境問題を視野に入れた生産 ・ 資源利用まで具体的な活動が先に行われており、ロハスという言葉自体は『愛 ・ 地球博(愛知万博)』にあわせた形で使われるようになった。
\\	2005年	カタカナ英語	
\\	あえん	あえん	予測 ・ 想像が出来ないことが起きたときに発する言葉。	あえんとは予測 ・ 想像が出来ないことが起きた際に発する「ありえん(「あり得ない」という語が崩れたもの)」という言葉が簡略化されたものである。ただし、この意味では若者の一部にしか普及していないため、前後の脈略なく「あえん」といった場合「会えない」が崩れたものととられることが多い点、注意が必要である。 あえんの使用例: 
\\	急いでてズボン履かずに出社しちゃった。 
\\	それはあえんだろ。	2004年	若者言葉 ・ 略語	
\\	アキバ系	あきばけい	秋葉原の文化を好み、秋葉原に集う人のこと。	アキバ系とは秋葉原独特の文化を好み、秋葉原に集う人のことを指す。秋葉原が家電の街から
\\	(周辺機器) ・ ゲームの街へと変貌するに際し、オタクの街というイメージが定着する。これに伴い
\\	系やアニメ系のオタクをアキバ系と呼ぶようになる。ただしオタクはアキバ系よりも広義なためオタク=アキバ系ではない。また、平仮名のあきば系や漢字の秋葉系という表記も使われるが、カタカナ表記のアキバ系が最もポピュラーである。更にオタク文化(アキバ系が好むショップなど)が秋葉原同様に中野でも増えており、中野系(ナカノ系)という言葉も使われる。今後全国各地で
\\	系といった形でアキバ系は点在するであろうとの見方もある。	2004年	
\\	アキバ系ファッション	あきばけいふぁっしょん	アキバ系と言われる人に多いとされるファッション。	アキバ系ファッションとはバンダナ、リュックサック、ミリタリールック、チェックのシャツ、スリムジーンズ、アニメショップの紙袋といったアキバ系(及びオタク)をイメージするファッションをさす。ただし実際にこのような格好をしたオタクを秋葉原で探してもほとんど遭遇することはない。これはメディアのオタク報道が増える中で秋葉原に通う人たちの身だしなみに対する意識改革があったためと言われる。現在アキバ系ファッションは電車男の劇場版やドラマで確認することが出来る程度といっても過言ではない。	2004年	美容 ・ ファッション用語	
\\	アデオス ・ 艶男	あでおす	アデージョがテクニックを駆使して手に入れるべき男性のこと。	アデオスとは女性ファッション雑誌
\\	(ニキータ)」の造語で、艶男と書き、男をオスと読ませ、スペイン語で「さよなら」を意味する『アディオス』っぽい読みにしたものである。アデオスは大人のいい女『アデージョ(詳しくはコチラ)』がテクニック(お洒落 ・ 色気)を駆使して手に入れるべき男性を意味する。同誌は女性向け雑誌のため、アデオスに関しては特に具体的な定義付けをしていないが、大物アデオスをゴルフ場で
\\	するための方法など、大人で高収入の男性をイメージさせる内容がある。 2006年11月には『艶男艶女の
\\	バラード』という楽曲も発売された。	2004年	美容 ・ ファッション用語	
\\	アデージョ ・ 艶女	あでーじょ	大人(30代?)でイイ女(モテる女)のこと。	アデージョとは女性ファッション雑誌
\\	(ニキータ)」の造語で、 若い女性(同誌ではコムスメと呼ぶ)にはない経験やテクニックをもつ大人のイイ女(モテる女)を意味する。艶女と書き、アデージョと読む。アデージョのキーワードとして貫禄 ・ お金 ・ 自立 ・ テクニック(お洒落)などがある。
\\	では若さでなく、テクニックでモテる大人の女“艶女”をコンセプトにさまざまな情報が発信され、多くの造語を生んでいる。 また、フジテレビ系列のバラエティ番組『ココリコミラクルタイプ』の中で、お笑いコンビ“ココリコ”による
\\	女
\\	に出てくるファッションを身にまとった女性を演じるコント)』が話題となり、
\\	読者以外にも広くアデージョという言葉が浸透した。	2004年	美容 ・ ファッション用語	
\\	イベサー	いべさー	イベント ・ サークルのこと。	イベサーとは「イベント ・ サークル」の略で、クラブでのパーティーなどイベントを行うことを前提(目的)とするサークルのことである。以前は若者のサークルといえば大学生が行うものを指したが、2000年辺りから渋谷センター街を中心に10代の高校生男女によるイベント団体が増えた。これら10代のイベント ・ サークルをイベサーと呼ぶ。イベサーには主にギャルやギャル男、
\\	系
\\	によるものがあるが、ギャルだけで構成されるイベサーを特に『ギャルサー』と呼ぶ。後にイベントの内容や世代に関係なく、何らかのイベントを行うサークルという意味で、広くイベサーが使われるようになる。	2004年	若者言葉、コギャル語 ・ ギャル関連、略語	
\\	インスコ	いんすこ	ソフトウェアなどをコンピュータにインストールすること。	インスコとはパソコン(コンピュータ)にソフトウェアやオペレーションシステムをインストール
\\	などの記憶媒体にあるファイルを展開したり、圧縮ファイルからファイルを展開して、プログラムを実行可能な状態にすること)することをいう。インストールのことをインスコールと誤表記したことから出来たインターネット ・ スラングで、逆にソフトウェアなどをシステムから削除するアンインストールのことはアンインスコまたはウンインスコという。	2004年	ネットスラング	
\\	ウザもい	うざもい	暴れてしまいそうなほど頭にきている状態で発する言葉。	ウザもいとは「鬱陶しい」「わずらわしい」といった意味の俗語『ウザい』と「気持ち悪い」という意味の俗語『キモい』から成る言葉で、暴れたり、相手に殴りかかってしまいそうなほど頭にきている状態を表したり、そういう状態のときに発する言葉として使われる。キモいに通じるイヤな感じのときやウザいに通じるイライラが重なったとき、又は一方が相当ひどいときに使う。 ウザもいの使用例: はっきり言ってウザもい!視界から消えないと何するかわからんぞ。	2004年	若者言葉	
\\	オサレ	おされ	センスのズレたお洒落や自信のないお洒落のこと。	オサレとはお洒落(オシャレ)のもじりで、当人はセンスの良いお洒落な格好をしているつもりだが、周囲はダサイと思っているような格好及びこのような当人と周囲とのお洒落に対する温度差を指すこと。また、普段お洒落に気を使わない人がお洒落をしたときなど、照れ隠しやセンスに対する自信のなさからオサレを用いたりする。	2004年	ファッション関連	
\\	ガシる	がしる	目頭が熱くなること。	ガシるとは「目頭(めがしら)が熱くなる」の略である。感動して目頭を熱くすることや、目頭が熱くしながら涙を流すことをいう。ガシる自体に目頭という意味は含まれているが、「目頭がガシる」「ガシラ(目頭の略)がガシる」といった言い方もする。	2004年	若者言葉 ・ 略語	
\\	俄然	がぜん	「にわかに」「いきなり」の他、「断然」と同意に使われる言葉。	俄然とは徳川夢声(大正~昭和にかけて活躍。弁士 ・ 漫談師)の造語で、「俄かに(にわかに)」「だしぬけに」「急に」といった意味で使われた。徳川夢声が造ったこの当初の意味で1930年代に流行し、そのまま定着する。 これが2000年代前半になると、「断然」と同意で若者を中心に普及。俄然と断然の聞き間違いによるとされている。2004年10月には
\\	から『俄然パラパラ』というオムニバス
\\	シリーズが発売開始。俄然が断然の意で広く浸透した要因といわれる(聞き違いでなく、こちらが誤用の発端という意見もある)。	2004年	
\\	カラ友	からとも	カラオケに行く友達のこと。	カラ友とはカラオケ友達の略で、一緒にカラオケに行く友達(カラオケ仲間)を意味する。もともと友人 ・ 知人関係にあった人の中で、頻繁にカラオケに行く人をカラ友と呼ぶ場合もあるが、インターネット上の掲示板や
\\	に代表される
\\	サイトのコミュニティなどで募集してカラ友になることが多い。またイレギュラーだが、ネット上でやりとりするヒトカラ(一人でするカラオケ:下記関連語参照)が趣味の仲間を指す場合もある。	2004年	
\\	韓流	かんりゅう ・ はんりゅう	韓国大衆文化の流行のこと。	韓流とは韓国の大衆文化の流行を意味し、2000年頃に中国で生まれた言葉である。日本ではドラマ『冬のソナタ(通称:冬ソナ)』が話題となった2004年から使われるようになる。韓流に伴い韓国語講座テキストの売上が伸びたり、日本から韓国への旅行者が30
\\	以上増加するなど様々な経済効果を生んでいる。しかし、韓流ブームの火付け役でもあるペ ・ ヨンジュン、チェ ・ ジウ、更にイ ・ ビョンホン、チャン ・ ドンゴン、キム ・ スンウといった韓流スターの来日時の騒ぎとは裏腹に韓国ドラマの視聴率は減少。業界では韓流=冬のソナタといった声もある。	2004年	
\\	旧ダチ	きゅうだち	旧友のこと。	旧ダチとは『旧友』と『友達(またはそれを略したダチ)』をかけた言葉(旧ダチが使われる数年前に『元ダチ』という言葉が普及していることを考えると、元ダチの派生とも考えられる)で、意味は旧友と同意である。ただし、旧友は昔の友達、または現在も親交のある昔からの友達といった意味で使われるが、旧ダチは現在も付き合いのある昔からの友達は含まない。逆に現在も友達ではあるが、数日メールのやりとりがないといった程度でも旧ダチに含まれる場合がある。つまり旧ダチは現時点で親交があるか否かに焦点を当てた言葉である。	2004年	若者言葉	
\\	ギリセー	ぎりせー	ギリギリセーフの略。	ギリセーとは「ぎりぎりセーフ」を略した言葉で、ぎりぎりのところで間に合う、ぎりぎりの点数で合格するなど、ぎりぎり(瀬戸際)のところで問題回避(解決)が間に合ったさまを表す。ギリセーは「遅刻しそう」「落第しそう」といった問題を抱えた人がギリギリのところで問題回避した際に発する言葉としても使われる。	2004年	若者言葉	
\\	クチャラー	くちゃらー	食事をするとき、クチャクチャと音をたてて食べる人のこと。	クチャラーとは物を食べる際の咀嚼音をあらわした擬音語『クチャクチャ』に、
\\	する人」という意味にする英接尾辞
\\	(この場合はラー)』を付けたもので、咀嚼音が周囲に聞こえてしまうような食べ方をする人を意味する。当初、そういった食べ方をする人をからかったり、冷やかす呼び名であったが、次第に汚い人、非常識な人といった非難・嫌悪の意を込めて使われるようになる。基本的に、ガムを音をたてて食べるように意識的に音をたてる人、意識して口を閉じれば静かに食べられるのにそれをしない人を指し、高齢や病気といった身体的問題で咀嚼音が漏れる人のことをクチャラーとは呼ばない。	2004年	若者言葉	
\\	けまらしい	けまらしい	他人同士仲良くすることを疎ましく思うこと。	けまらしいとはインターネットを介して広まっていった言葉で、他人同士が仲良く話したり、遊んだりすることを疎ましく思うさまを表す。ちなみに、ここでいう他人とは友人 ・ 知人と赤の他人の両者を含む。例えば、知人同士が仲良くしてることに対する嫉妬も、自分が孤独なときに赤の他人同士が仲良くしているのを見て妬む心もけまらしいに通じる。	2004年	
\\	ケミる	けみる	ハーモニーを織り成すこと。	ケミるとは人気デュオグループ“
\\	(ケミストリー)”を略したものに動詞化する接尾語『る』をつけた言葉で、ケミストリーのように綺麗なハーモニーを織り成すという意味で使われる。ケミるは主にカラオケでハーモニーを奏でる(複旋律を歌う)ことを願う際に使う。更にスポーツやゲームで息の合う動きをするという意味で使われる場合もある。	2004年	若者言葉	
\\	原始メール	げんしめーる	手紙や葉書のこと。	原始メールとは電子メールのもじりで、従来の郵便を使った手紙や葉書のこと。パソコンの電子メール
\\	メール)及び携帯メールは年々普及し、文章のやりとりはこれらデジタル媒体が主流となる。年明けの挨拶も年賀状ではなく、携帯メールというのが若者の間では当たり前となっているように電子メール、携帯メールが基準である現代の若者にとっては従来の手紙(封書 ・ 葉書)は原始メールとなる。	2004年	若者言葉	
\\	現ダチ	げんだち	現時点で付き合いのある友達のこと。	現ダチは意味的には「現在付き合いのある友達」の略であるが、現ダチが使われる数年前に同義語『今ダチ』が普及していることを考えると、今ダチの派生語とも考えられる。現ダチは主に現在進行形で一緒に遊びに行ったりする友人を指す。また、以前は恋愛関係にあった恋人と別れ、友達の関係として続いている場合、それを強調するために現ダチが用いられることがある。現ダチは小 ・ 中学生を中心とする若者言葉だが、今ダチに比べると2006年現在も普及率の低い言葉である。	2004年	若者言葉	
\\	サー人	さーじん	サークルに参加する人のこと。	サー人とはサークルに参加する人のことである。ただし、大学のサークルや社会人サークルに参加する人を対象にしたものでなく、「イベサー」「ギャルサー」「和サー」といった10代の若者によるサークルに参加する人を意味する(主だった活動のない和サーは含まれないことが多い)。 ※サー人が今後世代を超えて普及した場合、大学生サークルや社会人サークルに参加する人という意味での使用も考えられます。	2004年	若者言葉	
\\	三低	さんてい	独身女性がパートナーに望む三つの条件のこと。	1990年代、独身女性が求める結婚相手の条件として『三高』が流行したが、三低とはその2000年代版である(三高については下記関連語参照)。三低は「低姿勢 ・ 低リスク ・ 低依存」の3つの“低”から成っており、低姿勢=腰の低い男性(レディーファースト)、低リスク=リスクの低い安定した職業に就く男性(公務員や資格を必要とする仕事)、低依存=束縛しない男性を意味する。三高が引っ張っていってくれる(頼り甲斐のある)男性像であったのに対し、三低では女性が対等な関係を求める傾向が出ている。これは三低が、働く女性が一般的になったことや経済的にも共働きが必要になってきているという時代を反映したためともいわれている。	2004年	
\\	シモネーター	しもねーたー	猥談ばかりする人のこと。	シモネーターとは「下ネタ(性に関する淫らな話題)」と1984年からシルーズ化されているアーノルド ・ シュワルツェネッガー主演の人気映画「ターミネーター」をかけた言葉である。映画ではやられても、やられても立ち上がるターミネーターの如く猥談を繰り返す人を意味する。ただし、しつこいといった悪い意味ばかりでなく、猥談が多くて楽しい人という意味合いでも使われる。また人でなく、下ネタ ・ 猥談という意味で使われることもある。	2004年	若者言葉	
\\	スクハラ	すくはら	学校で起こる嫌がらせのこと。	スクハラとは『セクハラ』のもじりで、「スクール ・ ハラスメント」の略である。
\\	(学校)+
\\	(嫌がらせ)』。つまり、学校で起こる嫌がらせのことで、先生が生徒にするもの、生徒同士 ・ 教師同士の間で起きたもの全てを含み、行為も特に性的なものと限定されていない。ただし、スクハラという言葉が生まれた背景には、男性教員による女子生徒への性的嫌がらせがニュースなどで話題になったことが関係しており、実質的には教師から生徒へのセクハラ行為(性的嫌がらせ)をさして使われることが多い。	2004年	若者言葉	
\\	スラダン世代	すらだんせだい	アニメ
\\	(スラムダンク)』を見て育った世代のこと。	スラダン世代とはテレビ朝日系列で放送された人気アニメ
\\	(スラムダンク:井上雄彦原作の高校バスケットボールを題材にした少年漫画で、1990~96年に週刊少年ジャンプで連載。アニメは1993~96年に放映)』を見て育った世代のことである。
\\	世代で有名なものに団塊の世代がある。この団塊の世代は単にベビーブーム時に生まれた世代というだけでなく、その世代特有の思想や活動を含んで語られることが多い。スラダン世代の場合、安西先生(湘北バスケ部)の名台詞「あきらめたらそこで試合終了だよ」を信条としている世代という意味を含んでいる。例えば、何か壁にぶつかったときに「諦めませんよ。俺らスラダン世代だし」といった使い方をする。	2004年	若者言葉	
\\	セカチュー	せかちゅう	小説『世界の中心で、愛を叫ぶ』の俗称。	セカチューとは2001年4月発売の片山恭一のベストセラー『世界の中心で、愛を叫ぶ』の略称である。2004年5月に映画化、同年7月には
\\	系列でドラマ化され、『セカチュー現象』といわれるほどのブームとなり、2004年の流行語大賞 ・ トップテンも受賞している。セカチューの内容は恋人が白血病だったというもので、時代時代にブームになる病で成就しない恋物語のひとつではあるが、書籍が村上春樹の『ノルウェーの森』を抜いて小説過去最多部数になったことでもわかるように若者を中心に大きな支持を受けた作品である。 またセカチューは「セカチューを知らない=若者文化を知らない」という形で、世代や世代間交流を計る尺度的な言葉としても使われる。	2004年	若者言葉 ・ 固有名詞の俗称	
\\	チキる	ちきる	おびえること。おびえて足がすくんだり、弱腰になること。	チキるとは臆病者、怖がりという意味の俗語『チキン』に動詞化する接尾辞『る』を付けたもので、怯えることや怯えて足がすくんだり、怖気づいて弱腰になることをいう。「チキってる」「チキってるんじゃない」といった現在進行形での使用が多い。	2004年	若者言葉	
\\	チャケバ	ちゃけば	「本心言うと」「はっきり言うけど」といった意味の若者言葉。	ちゃけばとは「ぶっちゃけ話」を略したコギャル語である。ぶっちゃけ話とは暴露話のことである。しかし、ちゃけばと言った場合、暴露話という名詞でなく、木村拓哉が流行らせた「ぶっちゃけ」に近い使い方をする。例えば「ちゃけば、ムカつくのよね(=本音を言うと、ムカつくの)」というように使用する。	2004年	若者言葉、コギャル語 ・ ギャル関連、略語	
\\	チャッキョ ・ 着拒	ちゃっきょ	着信拒否のこと。	着拒とは携帯電話の機能のひとつである着信拒否及び、着信拒否をすることで、ちゃっきょと読む。着拒は本来迷惑電話、イタズラ電話の対策としてある機能で、登録した番号から電話がかかってきても着信音が鳴らないといったものである。ただし、そういった迷惑電話より、ケンカして口も利きたくなくなった人(別れた恋人など)を対象に着拒することが多い。	2004年	
\\	チョーグンジョー	ちょうぐんじょう	憂鬱なさま。落ち込んでいるさま。	チョーグンジョーとは色の一種である『群青(ぐんじょう)』の前に意味を強める言葉『超』をつけたもの。憂鬱な気分や落ち込んでいるさまを表す言葉に「ブルー」があるが、群青色はブルー(青色)を深くした色である。つまり、チョーグンジョーとは超ブルーよりも更に深く落ち込んでいるさま、ひどく憂鬱な気分であることを表す言葉である。(チョーグンジョーは、これでひとつの言葉として浸透しており、群青(グンジョー)だけでは使われない)	2004年	若者言葉	
\\	つゆだく	つゆだく	汗を大量に溢れ出すこと。またはその状態。	つゆだくとはもともと牛丼チェーン店で、客の好みに合わせ、牛丼の汁を多めに入れるサービス(正式にメニューとして存在するものではない)のことをいう。ここから、たっぷりの汁のように、汗が大量に溢れ出すことや溢れ出したさまをつゆだくという。高市早苗議員が2006年に内閣入りした際、就任会見で大量の汗をかいていたこつから“つゆだく大臣”と呼ばれた。 また、汗以外に怪我などで大量の血が噴き出すさまもつゆだくという。	2004年	
\\	デコ電	でこでん	デコレーション携帯電話の略。	デコ電とはデコレーション携帯電話の略で、ビーズやライトストーンを貼ったり、ペイント更には彫刻といった装飾が施された携帯電話のこと。デコ電は主に自分の携帯電話を自分で装飾することにより、世界に一台しかない携帯電話として2004年から主婦や
\\	、ギャルの間で流行。2005年に入ってからはデコレーションを施すショップも増えている。 デコ電の発祥はネイルアートショップであるとか、叶姉妹であるなど諸説あるが定かではない。むしろ、平成に入ってから流行していたビーズ細工やライトストーン細工の延長線として浸透したようである。	2004年	若者言葉	
\\	デコリ	でこり	装飾や飾りつけのこと。	"デコリとは装飾や飾付けを意味する英語
\\	(デコレーション)""を略したものに「飾り」の『り』をつけたもので、意味的には装飾や飾付けと同義に使われている。デコ電という言葉もある通り、デコリの対象は主に携帯電話であるが、他にもノートやプリクラ手帳、カバン(ポーチ)にペン、上履きまでさまざまな物に使われる。内容も自分でシールやプリクラを貼って簡単にデコったものから業者にお金を出してデコってもらうものまでさまざまである。
\\	2004年	日英合成語 ・ 若者言葉	
\\	デコる	でこる	飾りつけをすること。	"デコるとは装飾や飾付けを意味する英語
\\	(デコレーション)""を略したものに動詞化する接尾語『る』をつけたものである(ただし、『デコリ』という言葉がそれ以前に使われているため、デコリの『リ』を『る』に変え、動詞化したとも考えられる)。意味は装飾 ・ 飾りつけをすることをデコるという。デコ電という言葉もある通り、デコる対象は主に携帯電話だが、他にもノートやプリクラ手帳、カバン(ポーチ)にペン、上履きに至るまでさまざまな物に使われる。内容も自分でシールやプリクラを貼って簡単にデコったものから業者にお金を出してデコってもらうものまでさまざまである。
\\	2004年	若者言葉	
\\	デブス	でぶす	デブでブスのこと。	デブスとは太っているという意味の「デブ」と不美人という意味の「ブス」からなる合成語で、文字通り、デブでブスなことを意味する(デブなためにブスな人というニュアンスでも使われる)。ブスが形容する対象が女性であるように、デブスも女性を対象に使われる。デブスは他者がそういった女性を嘲う際にも使うが、当人が自嘲して使うことも多い。	2004年	若者言葉	
\\	長嶋ジャパン	ながしまじゃぱん	2004年のアテネオリンピック野球日本代表チームの愛称。	長嶋ジャパンとは、2004年アテネオリンピックにおける男子野球日本代表チームの愛称で、監督の長嶋茂雄にちなんでつけられた。 
\\	(国際オリンピック委員会)の方針で2000年のシドニーオリンピックからプロ野球選手のオリンピック出場が解禁となり、日本はプロとアマチュアの混合チームで参加したが4位止まりだった。そこで日本野球連盟はアテネオリンピックでは全選手をプロ野球選手にし、監督には巨人軍終身名誉監督の長嶋茂雄の起用を発表。以後、『長嶋ジャパン』という愛称で呼ばれるようになる。しかし、2004年4月3日に長嶋監督が脳梗塞で倒れ、実質的なチームの指揮はヘッドコーチの畑中清が行った。予選リーグは6勝1敗で、順調に決勝リーグへコマを進めたものの、決勝リーグ1試合目に0対1でオーストラリアに敗れる。2試合目のカナダ戦は11対2で勝利し、銅メダルを獲得している。 プロ野球選手を送っていながら銅メダル止まりであったことに対しては、オリンピック中もプロ野球はシーズン中で、戦力への影響がないよう各球団から2名づつの拠出という制限があったことや、畑中采配が柔軟性に欠けていたなどの意見があげられている。	2004年	固有名詞の俗称	
\\	和サー	なごさー	「和みサークル」の略。	和サーとは「和みサークル(和み系サークル)」の略で、『イベサー』の対として生まれた言葉であり、イベントを行わない(イベントが主ではない)サークルを意味する。イベサーにはイベント参加のための前納金やイベントを成功させるための規律が存在するが、和サーにはそういったものがなく、花火大会や飲み会、海に行ったりと一昔前でいう仲良しサークル的(合コンサークルの中高生版?)色合いが強い(ただし、一部の和サーではイベントを行うものもある)。	2004年	若者言葉	
\\	なでしこジャパン	なでしこじゃぱん	サッカー女子日本代表チームの愛称。(日本語俗語辞書サイトポリシー)	なでしこジャパンとは2004年アテネオリンピックにおけるサッカー女子日本代表チームの愛称で、アテネ ・ オリンピックの出場決定を受け、
\\	(日本サッカー協会)が一般公募を行い、約2,700通の応募の中から選定。2004年7月7日に『なでしこジャパン』と発表された。なでしこジャパンの選定に関してはもともと報道などで『大和撫子』と呼ばれていたからで、日本女性の美称である『大和撫子』から日本を限定する「大和」の部分を取り除き、「世界へ羽ばたけ」という意味を込めて『ジャパン』がつけられたものである。アテネオリンピックにおけるなでしこジャパンの結果は予選リーグを1勝1敗で決勝トーナメントへ進出。準々決勝でアメリカに1対2で敗れ、メダルを逃している。 なお、なでしこジャパンは
\\	の登録商標で、2004年流行語大賞にノミネートされている。 追記: 2011年7月17日、佐々木則夫監督率いるなでしこジャパンが女子W杯ドイツ大会にて初の優勝を果たした。男女通じ、
\\	主催による大会での優勝は初であり、震災で暗雲が立ち込めていた日本に光を与えたことが称えられ、同年8月には国民栄誉賞が授与された。そんな経緯もあり、なでしこジャパンは2011年新語・流行語大賞にて年間大賞に選ばれた。	2004年	固有名詞の俗称	
\\	ニート 
\\	にーと	職に就いておらず、学生でもなく、求職活動もしていない未婚者のこと。	"ニート
\\	とは
\\	""の頭文字で、英国で生まれた言葉である。ニートは1990年代、英国の若者の就労意識低下に対する労働政策に用いられた言葉で「職に就いておらず、学校などの教育機関にも属しておらず、職業訓練も受けていない15~34歳の未婚者」を対象とした。フリーター、退職者、失業者はニートに含まれず、逆に就業意識はあっても求職活動をしていない場合や家事手伝いはニートになる。 
\\	という言葉はイギリスではあまり普及せず、欧米でニートといった場合、「きちんとした」「こぎれいな」といった意味の
\\	""を連想されることが多い。1981年に大手化粧品メーカーの
\\	ソングとして起用された『ニートな午後3時』のニートはこちらの意味になります。また、
\\	(地球に接近する小惑星観測プロジェクト)を略した
\\	という言葉もあり、このプロジェクトで発見された彗星でニート彗星と呼ばれるものもあります。
\\	2004年	カタカナ英語	
\\	パケ死	ぱけし	パケット通信によるメール、ウェブ、ゲームなどのし過ぎで通信費が支払えなくなる状態や疲れによって死んだような状態になること。	パケ死のパケとはパケット及びパケット通信の略で、この場合、携帯電話や
\\	によるメールやウェブサービス、アプリ(ゲーム ・ その他ソフト)などを指す。当初(2001年頃)、パケ死はこれらパケット通信による通信費がかさみ、支払い不能または不能に近い支払い困難な状態を指した。後に各携帯会社が通信費の定額制を打ち出し、費用の問題はクリアされたが、出費を気にせず長時間出来ることから、疲労で死んだようになる者が現れる。するとパケ死も先述の意味からパケット通信のやり過ぎで死んだようになった人を指す言葉となった。	2004年	若者言葉	
\\	バッキー	ばっきー	ひどく束縛する人のこと。	バッキーとは束縛する人、束縛したがる人を意味する「ソクバッキー」の略。ソクバッキー同様、ひどく束縛する人や束縛したがる人を意味する。基本的に男女間に生ずる束縛に対して使われる言葉であり、過干渉な親など、子供を束縛する親に対して用いることは少ない。	2004年	若者言葉	
\\	ハンカレ ・ ハンカノ	はんかれ ・ はんかの	ハンゲーム内の恋人のこと。	ハンカレ、ハンカノの「ハン」とはハンゲーム
\\	という無料ゲーム ・ オンラインゲームサイト名の略。「カレ」「カノ」はそれぞれ彼氏、彼女を意味する。つまり、ハンカレ、ハンカノとはハンゲーム内で付き合っている彼氏や彼女のことである。ハンゲームでは単にゲームをするだけでなく、チャットやブログ
\\	などの機能が充実している。こうした機能からハンゲーム内で出会い、ハンゲーム内で付き合いが始まる。ただし、ハンカレ、ハンカノは基本的にハンゲームの中だけの恋人であり、実際に会ってデートするといったことはしない。現実でお付き合いする恋人とは別にハンカレ、ハンカノがいる者も多いという。(中にはゲーム婚(関連語参照)の発展する場合もある)	2004年	若者言葉	
\\	ひじきまつげ	ひじきまつげ	アイメイクに失敗したまつ毛の一種。	ひじきまつげとはアイメイクの失敗例の一つで、マスカラが上手く塗れず、ダマになった状態。またはマスカラを塗りすぎた状態を意味する。こういった状態のまつ毛がひじきに似ていることからひじきまつげという。会話の流れからアイメイクの話しとわかる場合はひじきと略すこともある。	2004年	
\\	微味	びみ	美味しいとも不味いとも言えない、よくわからない味のこと。	微味とは微妙な味の略であり、美味をもじった言葉である。2000年辺りから、人から出された提案や評価を受け入れることも完全に否定することも出来ないようなものに対して微妙という表現を使うようになるが、この場合の微妙もそれにあたる。不味くはないが美味くもないもの、一般的に食べなれた味付けとかけ離れているため、不味くはないが受け入れられないものを微味という。	2004年	若者言葉	
\\	ブチアゲ	ぶちあげ	非常にテンションが高い状態のこと。	ブチアゲとは「ブチギレ ・ ブチかわいい」など後に続く言葉を強調する『ブチ』とテンションが高い状態を意味する俗語『アゲアゲ』から成る言葉で、アゲアゲよりもテンションが高い状態を表す言葉として用いられる。アゲアゲはクラブ発祥の言葉だが、ブチアゲは当初からファッションや人間関係など様々な場面で使われている。2004年にはメディアがコギャル語のひとつとしてブチアゲを紹介。2005年には“まえけん
\\	トランス ・ プロジェクト”による『恋のブチアゲ
\\	天国』が発売され、話題となる。 また、ブチアゲより更にテンションが高い状態を表す言葉に『ブヒアゲ』があり、アゲアゲを基本とするとブチアゲは比較級に位置する。	2004年	若者言葉	
\\	ブヒアゲ	ぶひあげ	非常にテンションがあがっている状態のこと。	ブヒアゲとはテンションがあがっている状態を表す俗語『アゲアゲ』『ブチアゲ』の中で、最もテンションが高い状態のときに使われる言葉である。ブチアゲ同様、2004年にはメディアでコギャル語のひとつとして紹介されているが、アゲアゲやブチアゲに比べ知名度が低い。アゲアゲやブチアゲが曲タイトルなどに使われたのに対し、ブヒアゲにはそういった経緯がないことも要因として考えられる。また、ブヒアゲの同義語(テンションの高さも同程度)にブヒブヒがあるが、ブタの幼稚語(又はブタの泣き声)としての使用 ・ 認知のほうが高く、こういった意味で使われることは少ない。	2004年	若者言葉	
\\	振り込め詐欺	ふりこめさぎ	電話や葉書で相手を騙して金銭を振り込ませる詐欺のこと。	振り込め詐欺には『オレオレ詐欺』『なりすまし詐欺』『架空請求詐欺』など、手法によって様々な呼ばれ方をするものがあるが、葉書や電話などで相手をだまし、指定の口座へ金銭を振り込ませるタイプの詐欺全般のことをいう。当初はそれぞれ別名称の詐欺事件とされていたが、手法の多様化、複雑化に伴い、2004年12月に警視庁で『振り込め詐欺』という名称に統一されている。	2004年	
\\	プリマジ	ぷりまじ	プリクラで実際よりも綺麗に、かわいく写ること。	プリマジとはプリクラ(プリント倶楽部の略称)とマジックから成る言葉で、実物より綺麗に写っていたり、かわいく写っているプリクラ。またはプリクラで実物より綺麗に写ることを意味する。プリクラの機種によっては発光が強く、色白に写ったり、肌が綺麗に写るものがある。そういったプリクラで実物より綺麗に、かわいく写ったものをプリマジという。	2004年	若者言葉	
\\	ブログ 
\\	ぶろぐ	
\\	(ウェブログ)を簡略化した言葉。	"ブログの歴史や定義に関しては様々な説があります。その中でも1999年頃に
\\	と呼ばれる
\\	サイト(ホームページ)の更新や管理を簡単にするソフト""の登場がブログの起源というのが大方の見解である。2000年以降のアメリカでは、個人が発信するジャーナリスト系のサイトを
\\	(ウェブログ=
\\	(ウェブ)
\\	(記事)=ウェブ上の記事)と呼び、これが略され、ブログ
\\	という呼び方で定着している。また、
\\	の切る部分を変え
\\	=私たちはブログする』というキャッチコピー的な呼ばれ方をしたこともある(ウェブログよりも簡略化されたブログのほうが普及しており、一般的な俗語の定義からは外れています)。日本でも2003年頃から多くのブログ ・ サイトが誕生しているが、アメリカのブログが911テロ事件をキッカケとした意見表明やジャーナリスティックな内容のものが多いのに対し、日本のブログは個人日記風なものが多い。中でも芸能人を中心とした著名人ブログに人気が集中している。
\\	2004年	略語、
\\	用語	
\\	ホワイトキック	ほわいときっく	場の空気がしらけること。	ホワイトキック
\\	を直訳すると「白(しろ)+蹴る(ける)」になる。これを続けて読むと「しらける」に。つまり、ホワイトキックとは場がしらけることを意味する。後に頭文字だけ使い、
\\	という
\\	語でも用いられたが、現在は死語となっている。	2004年	若者言葉	
\\	マイレージ	まいれーじ	マイレージ ・ サービスの略。	"マイレージとは航空会社が提供する多利用搭乗客向けのプログラムのことで、ユナイテッド航空のマイレージ・プラスというサービスからきている。ちなみに英語の
\\	""は「総マイル数」という意味だが、同様にこういったサービスに対しても使われ、
\\	という言い方もする。 マイレージの内容は航空会社ごとで異なるが、基本的には搭乗飛行距離が一定に達するとポイント(マイル又はマイレージと呼ばれる)が加算され、その貯めたポイントを無料航空券や上級座席への切り替えサービスと交換出来るといった内容。マイレージで使用するマイル(ポイント)を貯める基本は搭乗飛行距離だが、それ以外にもショッピングや食事、クレジットカードの利用額によって加算されるものもある。更に、インターネットの懸賞サイトでポイントをマイルに交換出来るサイトもある。
\\	2004年	カタカナ英語	
\\	負け犬	まけいぬ	30代以上、未婚、子なしの女性のこと。	負け犬とは30代になっても結婚をせず、当然ながら子供もいない女性のことで、エッセイスト酒井順子が2003年に発売したベストセラー『負け犬の遠吠え 
\\	からきている。同書では「どんなに美人で仕事ができても“30代以上 ・ 未婚 ・ 子ナシ”は女の負け犬」と提示したうえで、負け犬になる原因や特徴、負け犬にならないため ・ なってしまってからの10箇条などが紹介されている。書籍が週刊誌で紹介され話題になる中で、そういった女性自身が「どうせ私は負け犬よ」「負け犬生活も結構楽しい」といった自嘲的使い方をするようになる。次第に負け犬が普及すると、嫌いな女上司(お局)に対して「今日の負け犬、ご機嫌ななめね」といった嘲る言葉としても用いられるようになる。こうして負け犬は2004年の流行語大賞トップテン受賞するほどの言葉となった。	2004年	
\\	ユニクローゼ	ゆにくろーぜ	ユニクロの商品を好んで着る女性のこと。	ユニクローゼとはカジュアルファッションの代名詞ともなっているファッションブランド『ユニクロ』をイタリア語風にしたものだが、この語が使われ始めた頃、白金に住む高級感のある主婦を『シロガネーゼ』と呼んでおり、ユニクローゼはその派生語と考えてよい。また、類語の『ユニクラー』は性別に関係なく使われるのに対し、ユニクローゼは女性のみを対象とした語である。どちらにしても、この言い回しでの普及率は低い。	2004年	若者言葉	
\\	ヨンフルエンザ	よんふるえんざ	ペ ・ ヨンジュンに熱をあげること。	ヨンフルエンザとはヨン様こと韓国の人気俳優『ペ ・ ヨンジュン』と病名の『インフルエンザ』から成る合成語で、病にかかったかのようにヨン様に熱をあげることをいう。ヨンフルエンザはヨン様ブーム(韓流ブーム)のキッカケ『冬のソナタ』が地上波で放送された2004年から使われるようになる。同年4月のヨン様初来日時には5000人近くもの
\\	(ヨンフルエンザ患者?)が羽田空港に押し寄せるなど話題になった。ヨンフルエンザはそんな光景を見た他者が嘲う意を込めて使う場合と、当人がヨン様
\\	であることをブログなどで語る場合にも使われる。	2004年	合成語 ・ 若者言葉	
\\	レゲー	れげえ	昔のゲーム、または貴重なゲームのこと。	レゲーとはレトロ ・ ゲームの略で「昔のゲーム」「懐かしいゲーム」といった意味で使われる。どういったゲームがレゲーにあたるかは世代によって異なるが、2007年現在、多くはファミコンというネーミングで一斉を風靡した任天堂のファミリーコンピュータを指して使われることが多い。また、これとは別にレア ・ ゲームの略としてレゲーを使う場合がある。ただし、意味合いが異なるだけで、対象となるゲームに差異はない。 ※発音は同じだが、音楽ジャンルのレゲエとは表記が異なるので注意!	2004年	若者言葉	
\\	ロコドル	ろこどる	地方のアイドルのこと。	ロコドルとはローカルアイドルを変則的に略したもので、地方のラジオや
\\	番組、タウン誌といった地方(地元)を拠点として活動するアイドルを意味する。このロコドルという言葉はシャ乱
\\	のメンバーであり、モーニング娘。を生み出した音楽プロデューサー“つんく
\\	が、
\\	「ポップジャム」でローカルアイドルを特集したコーナー内で命名した造語である。ロコドル以外に地方アイドル、ご当地アイドル、地元アイドル(ジモドル)と一般的には呼ばれる。	2004年	
\\	イーモッシュ	いーもっしゅ	男性と思われるほど垢抜けしない女性のこと。	イーモッシュとはイモい(又はその元となるイモ臭い ・ イモっぽい)の派生語で、男性と間違われてしまうほどイモい女性(垢抜けない ・ 田舎臭い ・ かっこ悪い女性)を意味する。イーモッシュは若者の間で使われる言葉だが、関東外エリアなど若者なら知っているというほど普及した言葉ではない。	2003年	若者言葉	
\\	イタ告	いたこく	イタズラで告白すること。	イタ告とはイタ電(イタズラ電話の略)の派生語で、“イタズラ”と“告白”から成る言葉。つまり、イタズラで異性に愛の告白をすることを意味する。単に特定人物をからかってするイタ告の他、罰ゲームとして行われるものなどがある。	2003年	若者言葉	
\\	イナタい	いなたい	田舎臭くて冴えないことや、くだらないこと。	いなたいとはミュージシャンの間で「泥臭い」「ブルージー」さらに「へたうま」といったニュアンスで使われる言葉である。関西のミュージシャンや音楽ファンを中心に良い意味での田舎臭さを表現したいなたいだが、音楽関係以外の若者に普及する中で「パッとしない」「ダサい」など悪い意味での田舎臭いというニュアンスでも使われるようになる(更に「くだらない」と同義にも使われる)。いなたいの語源については「田舎」に形容詞化する接尾語「い」をつけ、音が変化したとする説と「田舎」と「野暮ったい」の合成語という説があるが、もともと「野暮ったい」といった悪い意味の言葉ではなかったことから、前者が有力と思われる。	2003年	若者言葉	
\\	汚ギャルーズ	おぎゃるーず	何日も履いてボロボロになったルーズソックスのこと。	汚ギャルーズとは何日も風呂に入らず、汚い格好をした少女を意味する『汚ギャル(おぎゃる)』と1990年代に流行したコギャルファッションのひとつ『ルーズソックス』をかけた言葉で、汚ギャルが身につけている履き潰され、汚れきったルーズソックスを意味する。不潔という点以上に悪臭が話題となったが汚ギャル。そんな彼女たちが履く汚ギャルーズがバラエティ番組で取り上げられた。ただし、言葉自体は番組側が作ったものなのか、若者の間で自然発生的に出来たものなのかは不明。	2003年	美容 ・ ファッション用語、コギャル語 ・ ギャル関連、若者言葉、合成語	
\\	大人買い	おとながい	低価格商品を豊かな経済力で大量購入すること。	大人買いとは幼少期に出来なかった購入(コレクション)に対する夢を、大人になり、経済力がついてから果たす購入の仕方をいう。具体的には玩具付お菓子のケース買いやマンガを全巻一度に購入などがある。類似行為に『まとめ買い』があるが『まとめ買い』は一度に大量購入することで安価に入手したり、購入回数を減らすことで時間を節約するといった目的があり、大人買いとは意味合いが異なる。大人買いの発祥はトレーディングカードを収集するコレクター間という説が濃厚。オマケ付食品の流行とともに大人買いという言葉も一般的に普及する。更にケース売りの商品や大量数のパッケージ商品を『大人買いセット』とネーミングして販売するメーカーもある。子供が大量購入する場合でも代金を支払うのが大人(親など)であれば大人買いと呼ぶ。	2003年	若者言葉	
\\	輝き離婚	かがやきりこん	「輝くために」または「輝いている間に」する離婚のこと。	輝き離婚とは後の人生を輝いたものにするための離婚のことで映画プロデューサー黒沢久雄(故黒澤明監督の長男)とタレントの林寛子が離婚に際して交わした以下のセリフが元となる。 久雄:離婚したら今より輝けるのか? 寛子:輝くと思う 久雄:じゃ、いいよ。お互い自由になったらいい友達になれるだろう このやりとりからワイドショーが二人の離婚を輝き離婚と呼ぶようになった。後に輝き離婚は定年離婚や濡れ落ち葉離婚の対語として、夫の定年まで待つのでなく女性として賞味期限切れになる前にする離婚という意味でも使われるようになる。	2003年	
\\	カンコる	かんこる	人が少なく寂れている様を表す言葉。	カンコるとは本来人が集まって賑わうべき場所だが人が少なく寂れた感じを表す俗語『閑古鳥が鳴く』を略し、これに動詞化する接尾語『る』をつけたものである。意味自体に変化はないが、『カンコる=閑古鳥が鳴いている』という現時点寂れている感じが強調されている。また、カンコるは大手掲示板サイト“2ちゃんねる”でスレッド、板(掲示板)に人が集まらないことを意味して使われ始めた言葉である。	2003年	若者言葉	
\\	消えネタ	きえねた	忘れ去られたギャグや人(著名人)の話題。	消えネタとは一発屋など、一時的に流行になったが、次第にメディアで取り上げられなくなり、忘れられたギャグや著名人の話題のことである。つまり、消えネタとは「記憶から消えたギャグ(人)のネタ」を略したものである。
\\	や雑誌などのメディアが行う「あの人は今?」と題した名前を聞かなくなった著名人の現状を追跡するコーナーなどは消えネタの代表である。もちろん、個人的な会話や個人ブログで扱われるそういった話題も消えネタである。 上記とは別に、仲間内で話題にのぼらなくなったようなネタのこともいう。	2003年	若者言葉	
\\	ギザ十	ぎざじゅう	周囲に溝の入った十円玉のこと。	ギザ十とは(2006年現在の)50円玉や100円玉のように、周囲にギザギザの溝が入った10円玉のことである。そんなギザ十は自動販売機が受け付けないことから不便な硬貨という見方もあるが、逆に希少価値のある硬貨として収集している人も多い。また、2003年にはギザ十の価値がフジ
\\	『トリビアの泉』で取り上げられ広く浸透する。 ギザ十が発行されたのは1951年(昭和26年)から1958年(昭和33年)までで、発行されていない1956年(昭和31年)を除く7年間におよそ17億7千万枚が発行されている。中でも昭和33年のギザ十は2500万枚と発行数が少なく、希少価値があることから注目を集めている。	2003年	
\\	ギャ男	ぎゃお	ヴィジュアル系バンドが好きな男性のこと。	ギャ男とは「バンギャル男」の略。バンギャルとはヴィジュアル系バンドの女性ファンやヴィジュアル系バンドを好む女性のことでバンギャともいう。つまり、バンギャル男を略したギャ男とはヴィジュアル系バンドの男性ファンやヴィジュアル系を好む男性を意味する。	2003年	若者言葉	
\\	ゲトる	げとる	欲しいものを手に入れること。	ゲトるとは「得る ・ 取る ・ とらえる」という意味の英語“
\\	に接尾語『する』をつけた俗語『ゲットする』が略されたもので、ゲットする同様、欲しいものを手に入れることを意味する。また、同様に好きな異性の心を射止めることやナンパで異性を口説き落とすという意味でも使われる。	2003年	若者言葉	
\\	高ジャー	こうじゃー	高校生が着るジャージのこと。	高ジャーとは高校のジャージの略で、高校生の頃に着ていたジャージ、高校生が着ているジャージを意味する。高ジャーはあくまで学校で着る(着ていた)ジャージのことである。お笑いのテツ
\\	トモや仲間由紀恵の『ごくせん』出演によるジャージ・ブームなど、流行・ファッションとして着るジャージや家着として着るジャージの中で、学校で着るものと異なるジャージは含まない(逆に学校で着ているジャージであれば、学校以外で着たり、卒業後に着ていても高ジャーと呼べる)。	2003年	若者言葉 ・ 略語	
\\	シャカコン	しゃかこん	社会人(男性)とのコンパのこと。	シャカコンとは「社会人コンパ」の略で、文字通り社会人男性とのコンパという意味で若い女性(大学生、短大生など)が使う言葉である。ただし、この場合の社会人とは(恋愛を含む)交際をするに値する収入のある者をさし、無職やフリーター、ニートは含まれない。また、シャカコンには「奢ってもらえる」という意を含んで使われることも多い。	2003年	若者言葉	
\\	趣味カノ	しゅみかの	性行為を目的とした女性のこと。	趣味カノとはホスト用語のひとつで本命の彼女(本カノ)でもお金目当てで店に呼び出すために色恋を匂わせる女性(色カノ)でもない女性のことで、時間が空いたときだけ会う女性のことである。仕事でも本気でもない趣味程度の女性ということから趣味カノという。こうした趣味カノが必ずしも体目当ての女性というわけではないが、都合のいい女ということに変わりなく、一般に普及する中で趣味カノ=性行為を目的とした女性という意味合いで使われることが多くなっている。	2003年	若者言葉	
\\	正直ベース	しょうじきべーす	「実のところ」「本当のところ」といったニュアンスの言葉。	正直ベースとは賃金ベースや情報ベースといった「~を基礎(基準・土台)に」といった意味の「~ベース」に正直をのせたもの。社会人が仕事上で「正直であることを基準に=正直に言いますと・・・」というニュアンスで使ったり、若者が「実のところ・・・」「本当のところ・・・」といった軽いニュアンスで用いたりする。特に社会人の間では、
\\	のバラエティ番組『サラリーマン
\\	内の「サラリーマン語講座」というコーナーで取り上げられたことにより、浸透した。	2003年	
\\	着うた	ちゃくうた	着信音用に始められた携帯電話のサービスのひとつ。	着うたとは携帯電話の着信音やアラーム音をボーカル入りの楽曲に出来るサービスである。
\\	と同じ内容の音源から30秒程度にカット、これを配信しやすいように
\\	といったフォーマットに圧縮したものである。 着うたはソニー ・ ミュージックエンターテイメント
\\	の登録商標で、着うたサービスを最初に行ったのは2002年12月に開始した
\\	である。次いで2003年12月からはボーダフォンが、2004年2月からは
\\	ドコモも同様のサービスを開始。また、
\\	では楽曲の一部でなく、1曲丸ごとダウンロード出来る『着うたフル』というサービスを2004年11月から始めている。これにより、着信音としての着うたから聴くための着うたとなり、携帯電話の用途に広がりを持たせることになる。	2003年	サービス名	
\\	チャラリーマン	ちゃらりーまん	服装や言動が軽薄なサラリーマンのこと。	チャラリーマンとは言動 ・ 振る舞いが軽薄であったり、服装が派手で安っぽいといった意味の俗語『ちゃらちゃら』と『サラリーマン』から成る合成語で、チャラチャラしたサラリーマンを意味する。具体的には仕事がいい加減であったり、就業後のコンパやデートといった遊びばかりに力を注ぐサラリーマン。服装が派手で軽い感じのサラリーマンなど、チャラチャラに相当するサラリーマンに使われる。また一部で、これとは別に自転車を意味する俗語『ちゃりんこ』と『サラリーマン』の合成による、自転車通勤のサラリーマンという意味でも使われている。	2003年	若者言葉	
\\	チャリンダー	ちゃりんだー	自転車愛好家。自転車移動で旅をする人のこと。	チャリンダーとは自転車の俗称「チャリンコ」と、乗り手・騎手といった意味の英語
\\	(ライダー)」による合成語で、自転車愛好家を意味する。そうした人を表す言葉にサイクリストやチャリンカーという言葉があるが、それらに比べ、チャリンダーは自転車に乗って旅をする人というニュアンスが強い。特に定義はなく、サイクリングの延長で町内を旅する程度から、自転車で日本横断といったものまで、様々なチャリンダーが存在する。	2003年	合成語	
\\	中ジャー	ちゅうじゃー	中学生が着るジャージのこと。	中ジャーとは中学校のジャージの略で、中学生の頃に着ていたジャージ、中学生が着ているジャージを意味する。中ジャーはあくまで学校で着る(着ていた)ジャージのことである。お笑いのテツ
\\	トモや仲間由紀恵の『ごくせん』出演によるジャージ ・ ブームなど、流行 ・ ファッションとして着るジャージや家着として着るジャージの中で、学校で着るものと異なるジャージは含まない(逆に学校で着ているジャージであれば、学校以外で着たり、卒業後に着ていても中ジャーと呼べる)。	2003年	若者言葉 ・ 略語	
\\	なつい	なつい	
\\	「懐かしい」の略。 
\\	夏の暑さのこと。	
\\	なついとは「懐かしい」の略で、使い方も「懐かしい」と同様である(簡略化することで感嘆詞としても使われる)。なついは若者を中心にこちらの意味で使われることが多い。 
\\	なついとは夏の暑さのことで、「暑い」同様の使い方をする。「夏は暑い」を冗談で「アツはナツい」とひっくり返して言う人がいるが、なついはこの「アツはナツい」から生まれた言葉である。ただし、若者の間でもこの意味で使われることは少ない。	2003年	若者言葉	
\\	ブサメン	ぶさめん	不細工な顔の男性のこと。	ブサメンは「不細工(ブサイク)」と「面(顔の意)」または
\\	から成る合成語で、顔の造作が不細工な男性を意味する。ただし、語源的には『イケメン』の対として出来た言葉と考えるほうが無難である。また、顔の良し悪しに関係なく、ある程度の年齢になっても女性から相手にされない(モテない)男性という意味でも使われる。	2003年	若者言葉	
\\	ぶっちゃけ	ぶっちゃけ	暴露話のこと。	ぶっちゃけとは『打ち明ける』が崩れた『ぶっちゃける』を略したもので、当初「暴露話(ここだけの話)」を意味し、「ぶっちゃけトーク」「ぶっちゃけニュース」といった形で使われた。2000年に入ると「要するに」「実のところ」「思うところ」といった軽い意味でも使われるようになる。2003年にはキムタクこと
\\	の木村拓哉が
\\	系ドラマ
\\	でキーワード的に頻繁に使用したことから流行語となる。	2003年	若者言葉	
\\	プリン髪	ぷりんがみ	金髪または茶髪の根っこから黒髪が生えてきている状態。	プリン髪とは脱色(ブリーチ)や染めることで茶髪または金髪にした髪の根元から黒髪が生え始めた状態をいう。こういった状態のとき、髪一本一本を見ると根(下)が黒く、先(上)が茶や金ということになるが、頭全体で見た場合、分け目や頭頂部の方から黒くなっていくように見える。これが上部に黒いカラメルのかかったプリンに似ていることから、こう呼ばれるようになった。プリン髪以外に髪の毛プリン、プリン頭ともいう。	2003年	
\\	ペレ判定	ぺれはんてい	模擬試験の合否判定が
\\	や
\\	のこと。	ペレ判定のペレとは「20世紀最高のプレイヤー」「サッカーの王様」といわれるブラジル出身のサッカー選手のことである(ちなみにペレは裸足という意味のニックネーム。家が貧しく、新聞紙を丸めた玉を裸足で蹴って練習したことに由来する。本名はエドソン ・ アランテス ・ ド ・ ナシメント)。このペレが2002年にファイザー株式会社の
\\	啓発キャンペーンの
\\	に出演。ここから、模擬試験の合否判定で
\\	といったランクをつけられることをペレ判定という。	2003年	若者言葉	
\\	ホテイチ	ほていち	ホテルにあるショップ ・ 惣菜コーナーのこと。	ホテイチとは「ホテルの一階」の略で、デパ地下の派生語として使われる語句である。つまり、ホテイチとはホテルの一階にある惣菜のテイクアウトコーナーやショップを意味する。ホテルの一階にはそれぞれ、ホテル特有の高級食材を使った惣菜や焼きたてパンなどが販売されている。こうしたホテイチは、宿泊客はもちろん、宿泊をしていない人も利用出来、本物志向の
\\	や主婦など、ホテイチで買い物するためだけにホテルを訪れる人もある。	2003年	
\\	ヤン毛	やんげ	襟足が他に比べ極端に長い髪形のこと。	ヤン毛とは襟足(頭の後ろ部分)が他に比べ、極端に長い髪形、または襟足の長い毛の部分を意味する。ヤン毛はヤンキーの間でリーゼントで襟足を伸ばすスタイルが流行ったことからきた言葉で、「ヤンキーの毛」の略である(ヤンキーやヤンママが息子にそういった髪型をさせたことからきたという説もある)。ただし、ヤン毛は若者など一部で使われる言葉であり、理髪店 ・ 美容院で「ヤン毛にしてください」と言っても通じないことが多い。	2003年	若者言葉	
\\	リバースする	りばーすする	酒を飲み過ぎて嘔吐すること。(俗語的解釈)	リバースするとは逆(にする) ・ 逆回転(する) ・ 後退(する)といった意味の英語“
\\	からきたものである。
\\	は日本でも同様の意味でカタカナ英語として浸透しているが、この逆回転する、後退するというのは「戻る~戻す」とも言える。そして、この「戻す」には嘔吐という意味もあることから、飲み過ぎて嘔吐することをリバースするという(リバースだけでも用いられる)。	2003年	
\\	ロケット団	ろけっとだん	悪いことをしたときにすぐさま飛んできて注意する先生のこと。	ロケット団とは本来、人気アニメ「ポケットモンスター(通称:ポケモン)」に出てくる悪の軍団
\\	のことだが、そのネーミングから中 ・ 高生の間で「悪いことをしたときにすぐさま飛んできて注意をする教師」を指して使われている。具体的には修学旅行の消灯後、騒いでいる部屋に飛んでくる教師や教室での騒ぎに気付いて飛んでくる先生などである。ポケモンにはロケット団の他に『アクア団』『マグマ団』が存在するが、学校によってはこれらを上記の意味で使うところもある。	2003年	若者言葉	
\\	イケツラ	いけつら	かっこいい男性のこと。	イケツラとは「かっこいい」「魅力的」という意味の『イケてる』と顔 ・ 顔つきを意味する『ツラ』から成る言葉で、かっこいい顔 ・ モテそうな顔を意味する。イケツラは1990年代末頃からギャル ・ コギャルが使い始め言葉で、それまで使われた『イケメン』の「メン=面」を「ツラ」と読んだ派生語なのか、全く別の出所なのかは不明。どちらにしても死語となったイケメンに比べても認知度が低い言い回しではある。	2002年	若者言葉、コギャル語 ・ ギャル関連	
\\	今ダチ	いまだち	現時点で付き合いのある友達のこと。	今ダチとは『元ダチ』の対語として生まれた言葉で、「今の(今は)友達」の略になる。一般的な意味は現在進行形で付き合いのある友達のことで、一緒に遊びに行ったりする友達をさすことが多い(インターネットで使われる今ダチの中にはメル友やブログなどネット上での交流をさしても使われる)。また、基本的な意味は同じだが、以前は恋人で現在は友達としてだけ交流しているということを強調するために今ダチを使う場合もある。	2002年	若者言葉、略語	
\\	エモい	えもい	
\\	エロくて気持ち悪いこと、またはそういった人。 
\\	なんとなく寂しい ・ 悲しい気持ちのこと。	
\\	エモいとは助平(スケベ)な人を意味する『エロ(又はエロい)』と気持ち悪いという意味の俗語『キモい』の合成語で、気持ち悪いほどエロい、気持ち悪いエロといった感じでエロいが強調された言葉である。 
\\	エモいとは感情・感動といった意味の英語“
\\	(エモーション)”を略したものに形容詞化する接尾語『い』をつけたもので、なんとなく寂しい気持ちや悲しい気持ちを表すコギャル語である。エモいは当初こちらの意味で使われ始めたが、現在こちらの意味での使用は少ない。	2002年	コギャル語 ・ ギャル関連、若者言葉	
\\	丘ラッパー	おからっぱー	格好だけで
\\	系の音楽そのものには疎い人のこと。	サーフボードを担いで陸をウロウロし、格好だけで実際はサーフィンが出来ない人のことを陸サーファー(丘サーファー)というが、この『丘』を『ラッパー』に付けたのが丘ラッパーである。つまり、丘ラッパーとは格好だけは
\\	系ファッションをしているが、ラップやヒップホップ、
\\	といった
\\	系の音楽そのものについては疎く、音楽の話題になるとついていけない偽ラッパー、偽
\\	をいう。	2002年	若者言葉	
\\	汚ギャル	おぎゃる	汚い格好(不潔)をしている若い女性のこと。	汚ギャルとは何日も風呂に入らず、下着や靴下なども履き替えずにいる不潔で汚らしい格好の若い女性のことである。『汚物(おぶつ)』と『ギャル
\\	の合成語という説、『コギャル』の「コ」を『汚物』にかけて出来たとされる説があるが詳細は不明。プチ家出などで家に帰らない日が続くことから汚ギャルになる少女が多いとも言われるが、部屋自体が汚い汚ギャルも多く、一概にプチ家出が要因とはいえない。頭皮や足から出る悪臭が横を通り過ぎただけで不快感を与えるほどの汚ギャルが
\\	で取り上げられ話題となった。	2002年	コギャル語 ・ ギャル関連、若者言葉	
\\	教えて君	おしえてくん	わからないことを自分で調べず、すぐ人に聞いて解決しようとする人のこと。	教えて君とはわからないこと、知らないことを自分で調べようとせず、すぐ誰かに聞いて解決しようとする人を意味する。教えて君はもともとインターネットで使われた言葉で、一旦自分で調べることをせず、いきなり掲示板やチャットで聞く人をいう。特に検索サイトなどで検索すれば簡単にわかる内容を質問する人や掲示板への書き込みが質問ばかりの人に使われる。少しずつではあるが、インターネット以外でも使われるようになってきている。	2002年	ネットスラング	
\\	オタい	おたい	オタクなさまをあらわす形容詞。	オタいとは特定分野関して強い興味を抱き、関連する物を収集したり、詳しく知ることに時間を費やす『オタク』に、形容詞終止形の『い』をつけたもので、オタクなさまをあらわすオタクの形容詞形である。「オタい店」「オタい奴」「最近の俺、オタい」といった使い方をする。	2002年	若者言葉	
\\	オレオレトーク	おれおれとーく	自己中心的な会話のこと。	オレオレトークとは自分(オレ)中心の会話ばかりすることをいう。オレオレトークの典型として、出会い系サイトの掲示板やメールのやりとりの初期には自分の自慢話を書き連ね、少し仲良くなると、今度は仕事の愚痴ばかりといったものが多い。こうしたオレオレトークばかりする人を意味するオレオレ症候群という言葉もある(下記関連語参照)。なお、オレ(俺)が男性に対する一人称であるように、オレオレトークも男性がする自己中心的会話を対象としている。	2002年	
\\	過去問	かこもん	過去の試験問題のこと。	過去問とは「過去の(試験)問題」の略で、過去に行われた入試(入学試験)、資格試験、定期試験の問題を意味する。単なる試験の練習や自信付けのために行うだけでなく、出題傾向を知り、集中的に勉強するため過去問を分析する人も多い。	2002年	
\\	カップルつなぎ	かっぷるつなぎ	手を繋ぐ際の繋ぎ方の一種。	カップルつなぎとは人と手を繋ぐ際の繋ぎ方の一種。手の平をあわせ、双方の指を交互に重ねる繋ぎ方をいう。指の一本一本が絡み合う形になり、密着度が高く、恋人同士が好んで行うことからカップルつなぎという。 ※あくまで繋ぎ方の名称であり、カップル以外がしても呼び方に変わりはない。	2002年	
\\	カナリア軍団	かなりあぐんだん	サッカー ・ ブラジル代表の愛称。	カナリア軍団とはサッカーのブラジル代表に対する愛称で、国旗の色である黄色のユニフォームからそう呼ばれるようになった。 ちなみに現在、カナリア軍団のユニフォームカラーはホーム試合のみ黄色でアウエーの試合では青いユニフォームを着用している。	2002年	固有名詞の愛称	
\\	ギャル文字	ぎゃるもじ	携帯メールなどの平仮名 ・ カタカナ(一部漢字)を異なる文字で表した当て字のこと。	ギャル文字とは携帯メールやインターネットの平仮名、カタカナ(一部漢字)を異なるカタカナ ・ ロシア文字 ・ ギリシア文字 ・ 特殊記号を使って表す当て字である。ギャル文字の出所がインターネットの掲示板やチャットからなのか携帯メールからなのかは不明だが、普及した要因はギャルの携帯メールである。ギャル文字は一種の暗号としても使われており、ギャルの間ではギャル文字を用いることで他者に解読出来ないようにすることがステイタスとなっている。このため次々と新しい形のギャル文字が登場している。下記に記したのはその一例である。 また、ギャル文字は『へた文字』とも呼ばれ、『ギャル文字へた文字公式
\\	といった書籍が発売されたり、クイズ番組の問題、カラオケの歌詞に使用されたりと社会現象にもなっている。	2002年	若者言葉	
\\	黒歴史	くろれきし	触れてはいけないこと。なかったことにしたいこと。	黒歴史とはアニメ
\\	ガンダム(ガンダムシリーズの一つで1999年4月から2000年4月にかけ放送。2002年には映画化もれさている)』で使われた言葉で、過去の宇宙戦争の歴史を意味した。これが転じ、忘れたい過去や消し去りたい過去、ある人と話す際に触れてはいけないその人の過去という意味で用いられるようになる。有名人
\\	が隠していた下積み時代のエピソードや過去のスキャンダルを「有名人
\\	の黒歴史」といった使い方もする。	2002年	若者言葉	
\\	ケーアド	けーあど	携帯電話のメールアドレスのこと。	ケーアドとは携帯電話のメールアドレスのことである。ケーアドという略仕方はコギャルを中心に若者の一部が使っているが、携帯アドレスや単にアドレスと略すのが一般的である。同様の法則でパソコンのメールアドレスをパソアド、携帯電話の電話番号をケー番、家庭用電話の電話番号を家番と略す。	2002年	若者言葉	
\\	ケー番	けーばん	携帯電話の電話番号のこと。	ケー番とは携帯番号の略で、携帯電話の電話番号を意味する。ケー番はコギャルを中心に若者の間で使われたが、家庭用電話機の電話番号同様「番号」と略すほうが一般的である。ちなみに同様の法則で携帯電話のメールアドレスをケーアド、家庭用電話の電話番号を家番と略す。	2002年	若者言葉	
\\	コスる	こする	コスチューム ・ プレイをすること。	コスるとはコスチュームを略したものに動詞化する接尾語『る』をつけたもので、コスプレをすることである。この場合の「コスプレをする」とはアニメキャラクターの格好をしたり、制服などのコスチュームを身につけることのみで、コスチュームを作ることやコスプレイベントを催すといった着る行為以外は対象とならない。これとは別に、もともとコスるは演劇用語に見られた言葉で、衣装を身につけて演じることを意味する。	2002年	若者言葉	
\\	コモラー	こもらー	ひきこもる人(ひきこもり)のこと。	コモラーとは“ひきこもる”の『こもる』に人化する英接尾時
\\	をつけたもので、ひきこもる人(=ひきこもり)を意味する。同義語にヒッキーがあるが、コモラーは言葉の響きのよさ、柔らかさからカウンセラーがネーミングしたとされている(カウンセラー名、ネーミングした時期など詳細は不明)。ただし、ヒッキー同様、症状としてのひきこもり以外に「今日はコモラーしてた」というように終日家から出なかった程度で使う人も多い。	2002年	
\\	サッカー	さっかー	フットボールの一種。	"サッカーとは
\\	杯や
\\	リーグでお馴染みの球技の一種である。しかし、サッカーという呼び方は日本、アメリカ、オーストラリアなど一部地域のみで、大半の国では
\\	(フットボール)""と呼ばれる。日本でもチーム名は
\\	というが、これは
\\	""の略である。 サッカー発祥の地イギリスでは、1860年過ぎまで様々なルールのフットボールが存在していた。これに対しロンドンのフットボール協会が1863年、ボールを持つことが許される
\\	(発祥地に由来したラグビー)""と、足しか使ってはいけない
\\	(協会のフットボール)""とにわけ、現代に繋がるルールへと統一した。当時の学生が、この
\\	""の略語
\\	""に「~する人」という意味にする
\\	""をつけた
\\	(アソッカー)""という言葉で呼び、更にこれが略されて
\\	""となったのが語源である。ちなみにこの言葉が入ってきた明治時代には英国読みの『ソッカー』と発音されたが、英国では
\\	""は普及せず、あくまで
\\	""という呼び名が主流であったため、アメリカ読みの『サッカー』となった。 また、日本語で
\\	""のことは『蹴球(しゅうきゅう)』というが、これもサッカーは『サ式蹴球』ではなく、正式名のアソシエーション ・ フットボールから『ア式蹴球』という。 日本のサッカーの歴史: そんなサッカーであるが、日本では1921年に日本蹴球協会が創立。1929年には世界サッカー連盟
\\	に加盟。ワールドカップ予選は第五回のスイス大会から参戦している。1981年からは『キャプテン翼』というサッカーを題材にした漫画が連載され、野球に次ぐ人気スポーツとなる。1993年には日本プロサッカーリーグである
\\	リーグが開幕。プロ野球をも凌ぐといわれるほどの人気となる。更に1994年には第14回ワールドカップ ・ フランス大会で初の本戦出場、2002年には韓国と合同でワールドカップの本戦地となるなど、サッカーは国民的スポーツの仲間入りを果たしている。
\\	2002年	カタカナ英語(略語)	
\\	シカツ	しかつ	大勢で無視すること。	"シカツとは「無視する」という意味の俗語『シカト』に英語の
\\	""をつけて複数形にしたものである。シカトが個人単位で無視する行為に対し、シカツは大勢で無視することを意味する。例えば許せない行為をした人を皆で無視しようというとき、「アイツ、ムカつくからシカツするぞ」といった使い方をする。
\\	2002年	若者言葉	
\\	写メ	しゃめ	カメラ付携帯電話で撮った画像を添付したメールのこと。	写メとは
\\	-フォン(現:ソフトバンク)が2001年に命名したサービス『写メール
\\	の略で、カメラ付き携帯電話で撮影した画像を添付したメールのことである。ただし、写メは特にソフトバンク以外の画像添付メールに対しても使われる。また、広義には携帯電話で撮影した画像や撮影する行為自体を写メという。動詞として使う場合は「写メする」「写メる」という(写真付きメールを送るという意味より、撮影することを指すことが多い)。 ※写メールは当初、
\\	-フォンが行ったキャンペーンの名称でしかなかったが、後にサービス名として定着。
\\	-フォン(現ソフトバンク)の登録商標となっており、
\\	では同サービスを
\\	ショット」、
\\	では「フォトメール」という。	2002年	若者言葉	
\\	ジョモい	じょもい	非常に気持ち悪い人のこと。	ジョモいとは「極上きもい(ごくじょうきもい)」の略で、『きもい(見た目的に気持ち悪いの意)』を強めたコギャル語のひとつである。またジョモいは不精髭を生やした人や毛深い人など、体毛による不快さ(キモさ)を感じさせる人をさすことが多い。これは「ジョ」の語感からきたとする説があるが定かではない。どちらにしてもジョモいの使用は減っており、死語になっている。	2002年	若者言葉	
\\	タムる	たむる	集団で群れ集まること。	集合した軍隊や軍隊が集まった場所(陣営)、更に軍隊以外でも仲間や同じ会社の者、同業者が集まった場所を屯(たむろ)と言う。これに「~する」を付け、動詞化した「屯する(たむろする)」を略したものがタムるである。つまり、タムるとは集団で群れ集まることを意味する。	2002年	若者言葉	
\\	天然	てんねん	当人は笑わせる気がないのに、言動がずれていて面白いこと(人)。	天然とは1980年代から芸人を中心に普及した天然ボケの略で、特に笑わせる気はないが、話しの内容や行動がずれていて面白いこと、またはそういった言動をする人を指す(芸人(漫才)のボケとは本来笑わせるために計算された人工的 ・ 人為的なものである。それに対する言動ということから天然ボケと呼ぶ)。天然と略した形では1990年代末辺りから使われるようになり、同時に仲間内など一般の身近な人を対象にしても使われるようになる。また、天然アイドル、天然野郎、天然娘など多くの言葉が生まれている。	2002年	若者言葉	
\\	ドクハラ	どくはら	医師による患者への嫌がらせのこと。	ドクハラとはドクターハラスメントの略。ドクターは医者を意味し、ハラスメントは嫌がらせを意味する(セクハラが有名)。つまり、ドクハラとは医師による患者への嫌がらせを意味する。ドクターハラスメンと及びドクハラは外科医の故 土屋繁裕による造語で、主に医師の心ない言葉、暴言がこれにあたる。医師の悪意の有無に関係なく、患者が嫌がらせと感じた言動がドクハラにあたる。 これとは別に、独身ハラスメントの略としても使われる。独身者への「結婚しないの?」といった質問など、独身者が不快に思うような言動を意味する。ドクハラはもともとこちらの意味で使われだした言葉だが、近年、先述の意味で使われることが多くなっている。	2002年	略語	
\\	トリビア	とりびあ	くだらないこと。雑学的な事柄や知識。	"トリビアとは英語の
\\	""からきているカタカナ語で「くだらないこと」「些細なこと」「つまらないこと」を意味するが、2002年10月からフジテレビ系列で放送の番組『トリビアの泉』を機に「雑学的な知識」という意味で普及した言葉である。 トリビア
\\	は
\\	(トライアングル=三角形)」
\\	(トリオ=3人組)」に含まれる
\\	を意味する
\\	""と「・・・経由で」といった意味を持つ
\\	""が結合したもので、当初ラテン語では「三叉路」という意味であった。古代ローマの都市は三叉路が多かったため、トリビアは「ありふれた場所」という意味を持ち、転じて「くだらないこと」「些細なこと」という意味になった。更に中世ヨーロッパの大学が主要7科目の中の文法 ・ 修辞 ・ 論理の3科目を
\\	""と呼んだ。これら3科目は残る4科目に比べ、必要度の点で劣るとされたことから、トリビアが「無駄な知識」という意味を持つようになる。 また、トリビアはアバロン ・ ヒル社から発売されたゲームの名前でもある。1981年に発売され、トリビア(雑学的知識)を競いながらコマを進めていくボードゲームである。
\\	2002年	カタカナ英語	
\\	パワハラ	ぱわはら	会社の地位や立場を利用した嫌がらせのこと。	パワハラとはパワーハラスメント
\\	の略で、地位や立場、権力を利用した嫌がらせを意味する。パワーハラスメントは2002年、パワハラ対策の相談を請け負っている株式会社クオレ ・シー ・ キューブの岡田康子氏による造語である。パワハラは主に会社の上下関係を利用した嫌がらせや要求の押し付けを指す。例えば後の給与や出世への影響をちらつかせた上でのサービス残業の強要。上司が犯したミスの部下への転嫁。叱咤や疑念を持った態度も内容 ・ 程度によってはパワハラにあたる。ちなみに上下関係を利用した性行為の強要はセクハラと同時にパワハラにもあたる。	2002年	和製英語	
\\	専	びーせん	ブサイクを好む人のこと。	
\\	専とは「ブサイク専門」の略で、ブサイクな男性を好む女性やブサイクな男性ばかりを彼にする女性を意味する。デブ専同様にそういった好みという場合もあるが、
\\	専についてはレベルの高い男性と付き合うより気楽、自分(女性)のほうが優位に立てるといった理由が裏にあることも多い。こうしたことから
\\	専はブサイク専門だけでなく、
\\	級専門 ・ 
\\	ランク専門の略とする見方もある。	2002年	若者言葉	
\\	フィニる	ふぃにる	物事を終わらせること。	フィニるとは「終える」「済ます」といった意味の英語
\\	(フィニッシュ)』を略したものに、動詞化する接尾語『る』を付けたもので、物事を終わらせるという意味のコギャル語である。また、ここから「使い切る」といった意味でも使われる。例えば、残っているポットのお茶を飲みきる場合、「このお茶、フィニるね(=このお茶、全部飲んじゃうわね)」といった風に使う。	2002年	若者言葉	
\\	不思議ちゃん	ふしぎちゃん	不思議な個性を持った人のこと。	不思議ちゃんとは不思議を女性名っぽく呼んだもので、不思議な個性 ・ 感性を持ち、マイペースで、周囲とズレた言動をする女性を意味する(芸能人ではグラビア
\\	バラエティアイドルの小倉優子が不思議ちゃんにあたる)。大抵はそういった不思議な言動 ・ 個性に対し、面白い ・ 愛嬌があるといったニュアンスで好意的に呼ぶ言葉だが、中には皮肉や嘲う意を込めた否定的な意味合いで使う者もある。不思議少女とも言い、対象が男性の場合は不思議くんという。	2002年	
\\	ブチる	ぶちる	約束を破ること。無視すること。	ブチるとは約束を破ること、人の行為や応対を無視することを意味する「ぶっちする」の略である(または「ぶっちする」を略した同義語「ブッチ」に動詞化する接尾語「る」をつけたものとも考えられる)。ブチるも「ぶっちする」と同義に使われる言葉だが、かかってきた電話や届いたメールを無視するという意味での使用が特に多い。	2002年	若者言葉	
\\	ポイ捨て	ぽいすて	「ポイと捨てる」の略。	ポイ捨てとは「ポイと捨てる」を略した名詞形で、タバコの吸殻や空き缶、ガムなどを無造作に道路や道路脇、公園など公共の場に捨てることを指す。 ポイ捨てという言葉自体は昭和時代から使われていたが、2002年(平成14年)10月、東京都千代田区の『生活環境条例』の中にポイ捨て防止が盛り込まれたことが話題となり、改めて脚光を浴びる。以降、全国各地で『ポイ捨て禁止条例』『ポイ捨て防止法』といったものが敷かれる。 ※ポイ捨てに関する条例は千葉県成田市の1997年(平成9年)ポイ捨て禁止条例制定といったように、千代田区が発祥というわけではありません。	2002年	略語	
\\	やり目	やりめ ・ やりもく	性行為を目的として異性に近づくこと。	やり目とは「やり目的」の略で、やること(=性行為)を目的として異性に近づくなど、性行為を目的とした行動を意味する。またここから派生して、性行為を目的として近づいくる男性や性行為を目的に近づいたり、付き合っている女性を意味する。やり目は「やり目的」の略であり、「やりもく」と読むのが正しく、当初はそう読まれていたが、次第に「やりめ」という読みの方が多く使われるようになる(隠語にするためか?)。	2002年	若者言葉	
\\	ゆるカジ	ゆるかじ	ゆったりとしたファッションのこと。	ゆるカジとは『ゆるい』と「普段着」を意味する英語
\\	(カジュアル)』から成る合成語で、ゆったりとした着こなしのファッション ・ スタイルのことである。ゆるカジの主なものにゆったりしたパーカーにパンツがあるが、ゆるカジはゆったりした(ゆるい)感じのファッション全般を指し、
\\	系
\\	・ ゆるデコはゆるカジの1種になる。	2002年	若者言葉	
\\	レイヤー	れいやー	コスプレをする人、コスプレを趣味とする人。	レイヤーとは『コスプレイヤー』の略で、コスチューム ・ プレイ(通称:コスプレ)をする人やコスプレを趣味とする人を意味する。ただし、レイヤーという言い回しはコスプレを趣味とする人やコスプレ愛好者が使う言葉で、一般にはあまり普及していない。 これとは別に英語“
\\	からきた「層」「階層」という意味でもレイヤーは使われる。パソコンではグラフィックソフトで重ね合わせることの出来る層(透明シート)の単位やそういった機能、またネットワーク階層を意味する言葉として使用。女性の髪型では毛先に段差をつけるスタイルをレイヤーと呼ぶ。	2002年	若者言葉、略語	
\\	アゲアゲ	あげあげ	テンションがあがっている状態のこと。	アゲアゲとはテンションが高い状態を表す若者言葉で、クラブで気持ちが高揚し、踊り出すときや踊りたくなるような曲が流れたときにアゲアゲは使われる(気分の点で差異があるが、使い方という点だけみれば「ノリノリ」に似た言葉である)。アゲアゲは後にラジオの
\\	も使うようになり、クラブと関係のない若者にも広く浸透。同時にダンスや曲調といったものに関係なく、テンションをあげる様々なものやテンションがあがる様々な場面で使われるようになる。 また、下記関連語のような、アゲアゲよりも更にテンションが高い状態を意味する派生語も使われる。	2001年	若者言葉	
\\	アジドル	あじどる	アジアン ・ アイドルの略。	アジドルとはアジアン ・ アイドルの略であるが下記のような3種類の意味で使われる。 
\\	アジドルとはアジア出身の日本で活躍しているアイドルのことで、当初この意味で使われた言葉である。 
\\	アジドルとはアジアで活躍している日本のアイドルのこと。 また広義にはアジアでの活躍を目指す日本のアイドルも含まれる。 
\\	アジドルとはアジアで人気のアジアのアイドルのこと。	2001年	略語	
\\	アシヤレーヌ	あしやれーぬ	兵庫県芦屋市の高級住宅地に住む奥様。	アシヤレーヌとは兵庫県芦屋市の高級住宅地に住む奥様(基本的には専業主婦)をさす。アシヤレーヌは女性誌
\\	による造語で、シロガネーゼの関西版として出来た言葉である。対象エリアが東京都港区白金台と兵庫県芦屋市で異なるものの、基本的な意味合いに差異はない。ただし、アシヤレーヌが身につけるアクセサリーのほうが派手という意見が多い。	2001年	
\\	安全牌	あんぜんぱい	益も害も与えない人の例え。	安全牌とは人に益も害も与えない人のことで、主に結婚のするのに無難な相手といった意味で使われる。安全牌はもともと麻雀用語で、既に他者が捨て牌しているといった理由から捨て牌しても安全な牌のことである。そんな自分にとっては特に必要ない(利益を与えない)が危険でもない牌ということから安全牌は先述の意味で使われるようになった(麻雀用語では安全牌は安牌(あんぱい)と略されることもある)。 安全牌は主に女性が男性の収入や将来性などから使っていた言葉だが、男女のスタンスが変わっていく中で男性も安全牌を使うようになる。 安全牌の使用例: 遊び相手としては物足りないけど、結婚相手としては安全牌だし、キープしておこうかなあ	2001年	賭博用語	
\\	イエデン ・ 家電	いえでん	家にある固定電話のこと。	家電とは「家の電話」の略で、家や自分の部屋に設置された固定電話を意味する(読みはいえでん。「かでん」と読む場合、家庭用電気製品を意味する。誤読防止のため、イエデンとカタカナ表記することも多い)。携帯電話の普及により、それと区別するために家電という言葉が普及。例えば「一日家にいるから、家電のほうに連絡して」というように使う。また、「携帯ならいいけど、家電は教えられない」といったように、家の固定電話の電話番号という意味で用いることもある。	2001年	
\\	おくごめ	おくごめ	時間に遅れた際のお詫びの言葉。	おくごめとは「遅れてごめんなさい」の略で、文字通り、待ち合わせなどで時間に遅れた際、お詫びの言葉として使う若者言葉である。メールでは『遅ご』と略され、そのあとに頭を下げている顔文字や下向きの矢印を入れることで謝っているという意思表示をする。また、メールの際は事前のお詫びであることも多く、「遅れそうでごめんなさい」という意味でも使われる。 どちらにしてもおくごめと略した言い回しは砕けた表現であり、仲間内で使う言葉である。	2001年	若者言葉、略語	
\\	オファー	おふぁー	好きな人に告白すること。(俗語的解釈)	オファーとは「提案する」「申し出る」といった意味の英語“
\\	で、カタカナ語として日本でも使われる言葉である。オファーは大抵、初めて仕事で付き合う相手など、それまであまり付き合いのない相手や継続的な付き合いのない相手に新たに仕事を持ちかけたり、誘うことをいう。ここから、それまで恋愛対象として付き合いのない人に交際を求める告白を若者の間でオファーという。	2001年	若者言葉	
\\	ガチンコ	がちんこ	真剣勝負のこと。	ガチンコとは「真剣勝負」を意味する相撲用語で、真剣勝負をする際の真正面からぶつかる様を表した言葉とされており、「八百長をしない力士」という意味でも使われる(相撲の取り組み全てはガチンコ(真剣勝負)が前提である)。後に力道山など角界からプロレスへの転進者が増えたことでガチンコはプロレス界でも使われるようになる。 しかし、ガチンコという言葉が世代を超えて多くの人に浸透したキッカケという意味では1999年から
\\	系で放送された番組『ガチンコ!』があげられる。
\\	が司会、『ファイトクラブ(プロボクシングの元ミドル級チャンピオン竹原慎二が監督)』『ラーメン道(支那そばや店主:佐野実が監督)』『漫才道(オール巨人が監督)』といったコーナーでプロを目指す素人がその道の一流と呼ばれる人に志願。プロが毎週出す難題をクリアし、多くの応募者の中から1名(及び数名)がプロになれるという番組であった。プロと志願者は正にガチンコでぶつかり合い、視聴者に感動を与えた人気番組で2001年には『ファイト
\\	という特番も組まれた。しかし、週刊誌で「ヤラセ番組」と報じられてからは視聴率が低下、2003年に打ち切りとなっている。	2001年	相撲用語	
\\	ギャル男	ぎゃるお	色が黒く、髪色 ・ 髪型が派手な男性のこと。	ギャル男とは文字通りギャルの男版で色が黒く、髪色 ・ 髪型は派手、渋谷センター街などにたむろしてはナンパに明け暮れる男性をいう。ギャル男はコギャルの普及から4年後の2000年から使われ始め、翌2001年に普及した言葉である。また、ギャル雑誌
\\	に対するギャル男用の雑誌として
\\	といったものもある。しかし、ギャルがお姉系といわれる小奇麗な格好になって姿を潜めたようにギャル男も大人らしい着こなしをするようになり、ギャル男という言葉も次第に風化していく(2006年、東京 ・ 渋谷
\\	が男性売り場を展開。ギャル男の聖地として話題となる)。	2001年	若者言葉	
\\	ことよろ	ことよろ	「今年もよろしく」の略。	ことよろとは「今年もよろしく」の略で、主に携帯メールで使われた言葉だが、次第に会話の中でも使われるようになった。ことよろは1990年代末期から使われるようになった言葉である。 ことよろは『あけおめ』とセットで使われることが多く、「あけおめことよろ」で「あけましておめでとうございます。今年もよろしくお願いいたします」という年始の挨拶になる。当時、年が明ける瞬間に携帯電話で「あけおめことよろ」年賀状代わりにメールすることが流行った。しかし同時刻に多くの人が携帯メールを送るため回線がパンク。携帯電話各社は元旦0時前後の回線強化と携帯メールの規制を行うようになった。	2001年	若者言葉、略語	
\\	さくあり	さくあり	「昨年はありがとうございました」の略。	さくありとは「昨年はありがとうございました」の略で、主に携帯メールで使われた言葉であるが、次第に会話の中でも使われるようになった。さくありは1990年代末期から使われるようになった言葉である。 さくありは『あけおめ』や『ことよろ』とセットで使われることが多いが、『あけおめ』『ことよろ』に比べると普及度は低い。年が明ける瞬間に携帯電話で「あけおめことよろ、さくあり」などのメールを送ることが年賀状に代わりに普及。同時刻に多くの人が携帯メールを送るため回線がパンク。携帯電話各社は元旦0時前後の回線強化と携帯メールの規制を行うようになった。	2001年	若者言葉	
\\	ずいまー	ずいまー	不味い(味が悪い)を意味する感嘆詞。	ずいまーとは「不味い」「ひどい味がする」の意味で使う感嘆詞で、業界用語によくある倒語のひとつである。倒語とは通常使っている言葉をひっくり返して表現するもので、業界でよく使われるものに『ワイハー(ハワイの意)』『デルモ(モデルの意)』などがある。ずいまーも単純に「まずい」がひっくり返っただけである。 反義語である『まいうー』同様、ずいまーもテレビ東京系列の番組『元祖!でぶや』でホンジャカマの石塚英彦、振付師兼ダンサーのパパイヤ鈴木が使ったことから一般にも浸透した言葉である。ただし、番組内でもずいまーはあまり使われておらず、『まいうー』に比べると普及率は低い。	2001年	業界用語(倒語) ・ 若者言葉	
\\	ずく	ずく	調子に乗ること。	ずくとは調子付くの「付く(づく)」からきた言葉で、調子付く⇒チョズク⇒ずくという形で略されたものである。調子付く、調子に乗るという意味で使われる。 これと別に長野方言(信州弁)などエリアによっては「ずく無し」で「やる気がない ・ 億劫で行動にうつろうとしない」といったようにヤル気に近い意味でも使われる。	2001年	若者言葉	
\\	即る	そくる	ナンパした女性と会ったその日に性交に及ぶこと。	即るとは「その場 ・ その時にすぐ行うこと」という意味の“即(そく)”に動詞化する接尾語“る”をつけたものである。ただし、即るはナンパする人やその仲間内で使われる言葉で、ナンパで出会ってデートした相手とその日に性行為することをいう(コンパで即日性交に持ち込んだ場合にも使われることもあるが、多くはナンパを対象に使われている)。また、名詞形として使用する場合は即、ナンパで知り合い、2度目のデートで性交することは準即という。	2001年	若者言葉	
\\	ちょボラ	ちょぼら	ちょっとしたボランティアのこと。	ちょボラとは「ちょっとしたボランティア」の略で、
\\	(公共広告機構)が提唱したキャッチフレーズである。ボランティア団体に入ったり、被災地に行くといった大がかりなものでなく、日常の中で出来るちょっとしたボランティアをすすめるものである。
\\	ではゴミ箱の周りに散らばったゴミを片付ける。車椅子の人のために道をあける。子供が横断歩道を渡ろうとするとき、一緒に渡る。老人がバスから降りようとするとき、手を貸すといったことをちょボラとして紹介。ちょボラをした人に人形(ちょボラ人形?)をプレゼントするという内容であった。	2001年	
\\	珍走団	ちんそうだん	騒音を出しながら走る車 ・ バイクの集団のこと。	珍走団とは騒音を出し、交通法規を無視して走りまわる集団、いわゆる暴走族のことである。珍走団は『暴走族』という暴力的な語感を迫力のないものにすることで青少年の参加 ・ 見物する意思をそごうという考えから生まれた言葉で『珍』と略して使われることもある。こうした言葉は笑福亭鶴光(このときはリスナーの投書による提案)をはじめ、所ジョージ、みうらじゅんも提案しているが不発。その中で珍走団は大手掲示板サイト『2ちゃんねる』で生まれ、ネットを介して広まったこと。更に2001年、ダウンタウンの松本人志による人気番組での紹介や警察、裁判官、一部メディアも取り上げたことで、他の同義語よりは普及した。しかし、一時的な盛り上がりで終わっており、インターネット内の一部で使われる程度になっている。	2001年	不良 ・ ヤクザ用語、若者言葉
\\	トラチラ	とらちら	トランクス(男性下着)がチラリと見えること。	トラチラとはパンチラの派生語で、短パンなどから男性下着“トランクス”がチラッと見えることを意味する。トラチラという概念は以前からあるが、言葉として出てきたのは遅く、やおい、ボーイズラブといった男性同性愛を描いたマンガ人気による男性下着フェチ(愛好家)の増加とインターネットの普及後といわれている。ブリーフに比べ、トランクス人口が多いためか、ブリチラと比べ、希少価値が低い。	2001年	
\\	ノンパラ	のんぱら	結婚をせず、なおかつ親から独立している女性のこと。	ノンパラとは「ノン ・ パラサイト ・ シングル」の略である。パラサイト ・ シングルとは結婚をせず、親と同居し、いつまでも親に甘えた生活を送る女性を意味する(詳しくは下記関連語のノンパラを参照)。これに否定の“
\\	(ノン)”を付け、略したのがノンパラで、パラサイト ・ シングルの親に甘えた生活という部分に否定形がかかった意味になる。つまり、ノンパラとは結婚をしていないが、親元から離れて一人暮らしをし、自分で稼いだ収入で生計を立てている女性を意味する。2001年には山本貴代が『ノンパ
\\	という書籍も出版している(ノンパラが山本貴代による造語なのか否かは不明~調査中)。	2001年	
\\	はっちゃける	はっちゃける	はめをはずしてはしゃぐこと。	はっちゃけるとは日常生活の嫌なことやイライラを忘れ(忘れるために)、思う存分にはしゃぐことである。はっちゃけるは2000年辺りから若者を中心に使われたが、次第にストレスの多いサラリーマンなど社会人にも使われるようになる。同時にストレス発散のためにはしゃぐという行為以外に、単に「はりきる」といったニュアンスでも使われるようになる。また、2003年辺りから「はっちゃけ野郎 ・ はっちゃけ日記」といった形でも使われる。	2001年	若者言葉	
\\	埴輪ルック	はにわるっく	女学生が制服スカートの下にジャージを履くスタイル。	埴輪ルックとは女子中学生や女子高校生が制服スカートの下にジャージパンツや短パンを履くスタイルのことで、この格好が古墳から出土する埴輪に似ていることから、こう呼ばれるようになった(埴輪スタイルともいう)。もともと防寒を目的に寒い地域で行われたとされている(発祥として主に北海道、または長野県という説があるが、その他エリアを含め、詳細は不明)。	2001年	若者言葉	
\\	バンギャ	ばんぎゃ	ヴィジュアル系バンドが好きな女性のこと。	バンギャとはヴィジュアル系バンドのファンやヴィジュアル系を好む人のことである。ただし、バンギャはバンギャルの略である通り、男性ファンはもちろん、女性であってもギャルと呼ぶに相応しくない年齢のファンは含まない。ちなみに高齢の女性ファン(昔からの年季の入ったファン)のことはオバンギャといい、バンギャをやめる(卒業する)ことは『あがる』という。	2001年	コギャル語 ・ ギャル関連、若者言葉	
\\	半ケツ ・ 半尻	はんけつ	一つの椅子に二人で座ること。ファッションでお尻の一部を出すこと。	
\\	半ケツとは一つの椅子に二人で座ることで、二人で並んで一つの椅子に腰掛けると、尻のワレメに沿って縦半分ずつ分け合う形になることから半ケツと呼ぶ。もともとはこちらの意味で使われた言葉だが、こちらの意味ではほとんど使われることのない死語となっている。ニケツともいう。 
\\	半ケツとはローライズを履いたり、わざとパンツやスカートを下げて履くことで尻の上部を出すファッションのことである。半ケツと言っても尻を半分も出すわけではなく、尻の割れ目が見えそうな程度でも半ケツと呼ぶ。また、スカートがめくれあがったときなど、尻の下部分が露出したさまを逆半ケツという。	2001年	若者言葉	
\\	パンパンマン	ぱんぱんまん	顔がむくんでいること。またはそういった人。	パンパンマンとはやなせたかし原作の人気アニメ『それいけ!アンパンマン』の主人公アンパンマンと、物が膨らんだ様を表す擬態語パンパンを合成したもので、顔がむくんだり、太ったためにパンパンになった状態及び、そういった状態の人を意味する。後に、顔以外でもパンパンに張った状態を指して使われるようになる(例:肩がパンパンに張った場合、「肩がパンパンマンになった」といった風に用いる)。	2001年	合成語	
\\	ブス専	ぶすせん	不細工な顔の異性を好むこと。そうした人。	
\\	専」とは仕事やスポーツなど、様々な場面で
\\	を専門に扱う」
\\	専門に担当する」といった意味で使われる。つまり、ブス専とは不細工な人を専門として扱う ・ 担当するという意であり、不細工な顔の異性ばかりを好むこと、またはそういった嗜好の人をさす(ブス専は主に不細工な女性を好む男性に対して使用。不細工な男性を好む女性の場合は
\\	専を使うことが多い)。	2001年	若者言葉	
\\	プリ帳	ぷりちょう	プリクラを貼るための手帳このと。	プリ帳とはプリント倶楽部(通称プリクラ)で撮った写真シールを貼るための手帳のことでプリクラ手帳の略である。1995年に登場したプリクラは1997年に第一次ブームを迎えるが、その後需要が落ち始める。様々な機種の登場で再び2000年から需要が増え始め、2002年には第二次ブームを迎えることになるが、プリ帳がそれらブーム間のプリクラ離れを食い止めたと言われる。小学生の女の子の間で流行ったシール帳の延長とされるプリ帳には、様々な友人とのプリクラや遊びに行った場所で撮った思い出のプリクラで埋められ、日記のような役割もしている。近年ではそんなプリ帳にワッペンを貼ったり、ラメ入りペンで装飾するデコリというものが流行っている。(デコリについては関連ページ参照)	2001年	若者言葉	
\\	ペコキュー	ぺこきゅう	お腹が空いた状態のこと。	ペコキューとはお腹が減った状態を意味する擬態語の『ペコペコ』とお腹が空いて締まった状態を意味する『きゅう』から成る言葉で、お腹が空いたことを意味する。「朝から何も食べてなくてペコキュー」とか「お腹ぺこぺこ」といった感じで「お腹ペコキュー」といった使い方をする。 また、お腹がペコペコになるほどお金がないということから、貧乏および貧乏人という意味でも用いられる。(お金がないため、財布がペコペコでキュッとなっているということから貧乏人という意味をもったという説も・・・)	2001年	若者言葉 ・ 合成語	
\\	ヘタレ ・ 屁垂れ	へたれ	一人前でないことや腰抜けな人のこと。また、それらを嘲う言葉。	へたれはもともと芸人の楽屋言葉で一人前ではない芸人を意味し、そういった芸人を嘲う言葉として使われていた。漢字で屁垂れと書くと理解しやすいと思うが、小便垂れ同様、まだまだ子供という意味に繋がっている。こうした意味で昭和時代から使われていたへたれだが、平成時代以降、芸人が
\\	番組内で使うようになり、若者を中心に一般にも浸透(関西ではそれ以前に普及)。意味も「腰抜け」「根性なし」と広がりを見せ、表記もカタカナのヘタレが使われるようになる。	2001年	若者言葉	
\\	まいうー	まいうー	旨い、美味しいを意味する感嘆詞。	まいうーとは「旨い」「美味しい」を意味する業界用語によくある倒語のひとつである。倒語とは通常使っている言葉をひっくり返して表現するもので、業界でよく使われるものに『ワイハー(ハワイの意)』『デルモ(モデルの意)』などがあり、まいうーも単純に「うまい」がひっくり返っただけである。まいうーが特に一般でも使われるようになったキッカケはテレビ東京系列の『元祖!でぶや』という番組で、この番組内でホンジャカマの石塚英彦と振付師兼ダンサーのパパイヤ鈴木が旨いものを食べた際に「まいう~」と言ったことから広まった。特に石塚英彦にとっては「まいう~」がひとつのキャッチフレーズ的にもなっている。	2001年	業界用語(倒語) ・ 若者言葉	
\\	ミリタリー	みりたりー	「軍の」「軍隊の」「陸軍の」という形容詞。又「軍の物」という意味でも使われる。	"ミリタリーとは英語の
\\	""と同じ「軍の」「軍隊の」「陸軍の」という意味で使われる。ミリタリーは主にファッションや雑貨(時計、小物)の世界で使われる言葉である。特に衣料に関しては、その機能性、デザインがうけて流行となる。 本物のミリタリーは衣料や雑貨を含む全てを国が管理する。国の指揮の下、競売にかけられ市場に出回る物もあるが、日本で『ミリタリー ・ ファッション(ミリタリールック)』『ミリタリー ・ グッズ』として販売されているものの多くはレプリカや単なるコピー商品といわれている。
\\	2001年	カタカナ英語	
\\	メリクリ	めりくり	「メリークリスマス」の略。	メリクリとは「メリー ・ クリスマス」の略で、主に携帯メールで使われた言葉である。パソコンと違い、携帯電話でメールの文字を打つのは時間がかかるため、簡略出来る言葉はどんどん簡略化されている。そういった中で生まれた「メリクリ」のような言葉は若者を中心に会話の中でも使われるようになる。 また、メリクリは
\\	がシングル曲のタイトルとしても使っている。	2001年	略語	
\\	ユニクラー	ゆにくらー	ユニクロの服を着こなす人のこと。	ユニクラーとは衣料品の生産~販売を行う企業『株式会社ユニクロ
\\	に英語で「~する人」という意味にする接尾語
\\	をつけたもので、ユニクロの服を着こなす人やユニクロの服を(好んで)着る人を意味する。それまでもシャネルを身につける人を『シャネラー』、ルイ・ヴィトンを身につける人を『ヴィトラー』と呼んだが、ユニクラーはこの流れから生まれた言葉である。これはユニクロが単なる衣料品販売店というイメージから、カジュアル ・ ファッションの一ブランドというイメージが強くなったためで、低価格 ・ 高品質のユニクロ商品が売上げ ・ 利益ともにピークとなった2001年にユニクラーも普及した。	2001年	若者言葉	
\\	ラブアド	らぶあど	恋人の名前が入ったメールアドレスのこと。	ラブアドとは
\\	とメールアドレスから成る合成語で、携帯やパソコンで使用するメールアドレスのユーザーが指定出来る部分(通常は@より前の部分)に恋人の名前を入れたものである。メールアドレスには自分の名前や誕生日、趣味、好きなものなど、利用者と関連のある文字と数字を組み合わせて作る。そこに恋人の名前を入れるラブアドで恋人がいることをアピールしたり、恋人に独占されている(または独占している)という意思表示が出来る。ただし、ラブアドにしたことで他人にアドレスを教えるのが恥ずかしい。また、別れた後に使いづらいといったマイナス要素もある。	2001年	若者言葉	
\\	あけおめ	あけおめ	「あけましておめでとう」の略。	あけおめとは「あけましておめでとう」の略で、携帯メールで挨拶する際、この略式が使い勝手がよいことから、2000年以降、広く浸透した言葉である。ただし、所ジョージなど、古くから使っている人もあり、ネットにおいても1980年代末には
\\	チャットなどで若者に使われ始めている。(※また、あけおめという言葉自体は古典落語で使われており、江戸時代の下町で使われた言葉という情報も頂いております。 ※この点に関しては裏づけ調査中)。 あけおめは『ことよろ』とセットで使われることが多く、「あけおめことよろ」で「あけましておめでとうございます。今年もよろしくお願いいたします」という年始の挨拶になる。当時、年が明けると同時に「あけおめことよろ」と年賀状代わりにメールすることが流行たが、同時刻に多くの人が携帯メールを送るため回線がパンク。携帯電話各社は元旦0時前後の回線強化と携帯メールの規制を行うようになった。	2000年	若者言葉、略語	
\\	朝スポ	あさすぽ	出勤前にスポーツクラブに行くこと。	朝スポとは「早朝スポーツジム」及び、これに類するもの(動詞がつくなど、文章化されたものなど)の略で、早朝にスポーツジムに通うことを意味する。サラリーマンや
\\	など、昼 ・ 夜に運動する時間の取れない人が出勤前の時間を利用し、朝スポをする。広義には「早朝にスポーツする」の略とし、スポーツジム以外にもボウシング(ボクササイズ)ジムやスイミング、また、そういった施設に行かなくても出勤前に近所をジョギング、ウォーキングすることも含む。	2000年	
\\	イケメン	いけめん	かっこいい男性のこと。	イケメンとは「かっこいい」「魅力的」を意味する『イケてる』と顔を意味する『面』又は男性を意味する英語の
\\	から成る合成語で、「かっこいい男性」を意味する。イケメンはあくまで男性を対象にした言葉で「イケメン君」とも言われる。また、「イケメン男」「イケメン俳優」というように形容詞的にも使われる。 インターネット上では『逝け面』という表記も見られるが、この場合は「不細工な男性」という正反対の意味になる。ただし、逝け面は2ちゃんねるを中心としたインターネット上で使われる言葉であり、通常の会話で「いけめん」と言った場合、前者のイケメンと考えてよい。	2000年	若者言葉	
\\	痛い	いたい	他人の的外れな言動に対する「恥ずかしい」「情けない」「気の毒に」などの気持ち。	痛いは通常、自身の体(又は心)の痛みをうったえたり、人の身体の痛みを問いかける際に使われる。この痛いが平成に入り、人の言動に対して用いられるようになる(この場合、イタイ ・ イタいといったカタカナ表記も使われる)。痛いが用いられる主な言動として、勘違い ・ 場違いなどの見当外れな言動、人が不快 ・ 不満に感じる言動など、「恥ずかしい」「情けない」「みっともない(不様)」に通じるものである。また、「痛い子」「痛い話」「痛いサイト」など、後ろに対象を付けて使われることが多い。	2000年	若者言葉	
\\	インパーク	いんぱーく	テーマパーク
\\	に行くこと。	"インパークとは
\\	""をカタカナ読みしたもので、パーク(テーマパーク)に行くことを意味する。ただし、英語圏で
\\	""は特にこういった意味で使われる言葉ではなく、和製英語と考えてよい。また、インパークは
\\	(東京ディズニーランド)のファンが好んで使うため、
\\	(または
\\	ディズニーシー)に行く(入園する)という意味の言葉と考えてよい(インパークが
\\	関係者 ・ ファンから生まれた言葉か否か、また
\\	以外を対象にしての使用があるかどうかは不明)。「インパーク日記」「インパーク写真」といった形でも使われる。
\\	2000年	和製英語	
\\	ウーハイ	うーはい	焼酎をウーロン茶で割った飲み物「ウーロンハイ」の略。	ウーハイの「ウー」はウーロン茶の略、「ハイ」はハイボール(または酎ハイ)の略。つまり、ウーハイとはウーロン茶のハイボールで、多くは焼酎をウーロン茶で割ったものをさす。日本でハイボールといえばウィスキーを炭酸で割ったものが一般的で、酒類をソーダやトニックウォーターなど炭酸で割ったものというのが通説になっている。このため炭酸が入っていないのにウーハイ(ウーロンハイ)と呼ぶのはおかしいという意見を耳にするが、広義にはアルコールをジュースなど各種ドリンクで割ったものをハイボールといい、ウーハイ(ウーロンハイ)という呼び方も間違いではない。	2000年	略語	
\\	駅ナカ	えきなか	駅構内にある商業スペースのこと。	駅ナカとは駅構内(改札内)にある商業スペースのことで、主にコンビニエンスストアや書店、飲食店などが展開している。従来、駅構内にはキヨスクと立ち食いソバ屋がある程度であったが、1995年、阪急電鉄が十三駅に直営のコンビニエンスストアをオープン。以後、各鉄道会社が駅ナカに力を入れていった。駅から出ずに買い物などを済ますことが出来るという利便性から人気があるものの、周辺の商店に人が流れないという問題も抱えており、実際、訴訟問題に発展しているケースもある。駅中 ・ エキナカという表記も使われる。	2000年	
\\	オキラ	おきら	嫌い(苦手)な物や人のこと。	オキラとは「お嫌い」の略で、嫌いであったり、苦手であったりする物や人を意味する。オキラは「お気に入り」の略“オキニ”の対語として生まれた言葉で、主にジャニーズ ・ ファンがお目当ての人から嫌われている人(ファン)に対して用いるジャニーズ用語である。	2000年	若者言葉	
\\	オッハー	おっはー	「おはよう」の崩れた言い方。	おっはーとは朝の挨拶「おはよう(ございます)」が崩れたもので、もともとは『オーハー』とも言われていた。「オーハーまたはおっはー」は1997年からテレビ東京が子供向けに放送している朝の番組「おはスタ」で司会者が使っていたものである。これを
\\	の香取慎吾が許諾を得て使用。香取扮する慎吾ママがフジ
\\	系列の番組「サタ
\\	スマ」で独特な振り付けとともにおっはーを流行語にし、2000年新語 ・ 流行語大賞では見事大賞を受賞している。	2000年	若者言葉	
\\	おなクラ ・ 同クラ	おなくら	同じクラスの人のこと。	おなクラとは「同じクラス」の略で、学校で同じクラスの人を意味する。あくまでクラスが同じクラスか否かという点を答える言葉であり、友達と呼べる仲か否かは関係ない。同クラという表記も同程度に使われる。また、おなクラという響きから、自慰を意味するオナニーに絡んだ言葉と見せかける引っかけ問題としてクイズ番組に使用されることもある(ただし、「おな中(おなちゅう:下記関連語参照)」と比較すると圧倒的にクイズ番組での使用頻度は低い)。	2000年	若者言葉	
\\	おやっさんギャル	おやっさんぎゃる	おじさんのような行動をする若い女性のこと。	おやっさんギャルとは1990年に流行語大賞をとった『オヤジギャル』の進化形(パワーアップしたもの)で、基本的な意味はオヤジギャル同様、おじさん特有といわれた行動をする若い女性のことである。具体的には、くわえタバコでパチンコをするといったもので、有間しのぶのマンガ『モンキー ・ パトロール』に登場する“やいち”におやっさんギャルを見ることが出来る。	2000年	若者言葉、コギャル語 ・ ギャル関連	
\\	親指族	おやゆびぞく	
\\	パチンコばかりしている人のこと。 
\\	携帯電話の機能を使いこなし、頻繁に利用する人のこと。	親指族とは名古屋で流行していたパチンコが全国的に人気となった1950年代始め、パチンコに熱を上げ、パチンコばかりしている人を嘲う言葉として使われた。当事のパチンコは今のように玉を電動で打つのではなく、一発づつ親指でレバーを弾いて打った。ここから親指族と呼んでいた。	2000年	
\\	がっつり	がっつり	「思いきり」「思う存分」といった意味の言葉。	がっつりとは「たっぷり」「しっかり」「十分」「思いきり」「思う存分」と同意に使われる言葉で、「がっつり食べる」「がっつり稼ぐ」「がっつりサポート」といった使い方をする。もともと北海道で使われていた方言で、それを釧路出身のラジオ
\\	やまだひさし”が全国区の番組で使用し、広まったという説がある(参照:三省堂デイリー新語辞典)。	2000年	
\\	カラーギャング	からーぎゃんぐ	アメリカのストリートギャングを真似た不良グループのこと。	カラーギャングとは暴走族やチーマーにかわって1990年代後半から首都圏を中心に出てきた不良グループ(非行グループ)のことである。ヒップホップを好みアメリカのストリートギャングに憧れていたことからギャングと名乗っていたが、日本のギャングの特徴である赤 ・ 白 ・ 黒 ・ 黄といったカラフルなシャツやバンダナからカラーギャングと呼ばれるようになる。このカラーは所属するグループのシンボルカラーで、その色から敵か味方かを判別する。こういったファッション感覚や
\\	文化を取り入れた辺りがうけ、ダサイと言われるようになっていた暴走族と入れ替わる形で規模を広げていった。 カラーギャングはアメリカのストリート ・ ギャングの凶悪性や凶暴性も受け継いでおり、リンチやケンカの際にナイフやスタンガン、バットといった凶器をためらいなく使うことも特徴である。逮捕事件も起こるようになる中でこういった部分にスポットを当てた報道番組が増え、一般にもカラーギャングという言葉は浸透していった。 また、『池袋ウエストゲートパーク 
\\	というカラーギャングのドラマもある。2000年4月から6月に
\\	系列で放送され、脚本は宮藤官九郎。キャストに長瀬智也、加藤あい、窪塚洋介、山下智久、坂口憲二、妻夫木聡、佐藤隆太、小雪、森下愛子、渡辺謙と豪華な顔ぶれであった。2003年には同じキャストで2時間のスペシャル番組も放送されている。	2000年	不良 ・ ヤクザ用語	
\\	ギャルママ	ぎゃるまま	ギャルのような格好をした母親のこと。	"ギャルママとはガングロに茶髪、厚底サンダルといったギャルのような格好をした母親のことをいう。ギャルママのほとんどは過去にギャル(コギャル)の格好をし、そのまま母親になったというパターンである。また、ギャルママのサークルも存在する。これは独身時代、同じギャル仲間で交流していたグループによって運営されていることが多い。また、ギャル雑誌
\\	""には『ギャルママ紹介コーナー』が設けられている。 類語にヤンママがあるが、こちらは実年齢が若い母親が対象で、年齢に関係なく格好から使い分けられるギャルママとは意味が異なる。
\\	2000年	若者言葉	
\\	食い逃げ	くいにげ	性行為をするだけして、姿を消したり、別れること。(俗語としての解釈)	食い逃げとは本来、飲食店で食事をし、代金を支払わずに逃げ去ることをいうが、男性が性欲を満たすためにセックスすることを「食う」と表現することから、性行為をした後、すぐに姿を消したり、別れ、去ってしまうことも食い逃げというようになる。こうした流れから出来た言葉なため、同義語のやり逃げと違い、食い逃げは女性の同様の行為に対して使うことはない。	2000年	
\\	ケチャラー	けちゃらー	ケチャップ好きな人のこと。	ケチャラーとは調味料のケチャップの略に、人化する接尾語「ラー」をつけたもので、ケチャップが好きな人やいろいろな食材 ・ 料理にケチャップをつけて食べる人を意味する(マヨネーズ好きな人を意味するマヨラーの流行語を受けて使われるようになったことから、マヨラーの派生語と考えてよい)。また、ケチャラーは「マヨラー
\\	ケチャラー」といったように、マヨラーの対として使われる程度であり、あまり使われない言葉である。また、同じ意味の言葉でケチャップに人化する英接尾辞“-
\\	を単純に付けたケチャッパーという言葉も使われる。	2000年	若者言葉	
\\	圏外	けんがい	興味の対象外ということ。	圏外とは本来、携帯電話などの移動通信サービスにおいて、基地局の電波が受信出来ない状態のことをいうが、これとは別に、若者の間で興味の対象外(興味の対象範囲から外れている)という意味でも使われる。この場合、物や事象に対してより、異性に対して恋愛対象外という意味で用いることが多い。	2000年	若者言葉	
\\	コスプレ	こすぷれ	コスチューム ・ プレイの略。	コスプレとはコスチューム ・ プレイの略で、もともと「立派なのは衣装だけで内容が伴っていない演劇」に対する揶揄的な言葉であったが、現代ではアニメ ・ 漫画 ・ ゲームのキャラクターや歌手(主にビジュアル系ロックミュージシャン)が着ている衣装を真似て作った服を着て、そのキャラクターになりきること、更にそういった格好のことをいう。 また、コスプレを行う人をコスプレイヤー又はそれを略したレイヤーと呼ぶ。	2000年	略語 
\\	若者言葉	
\\	コミケ	こみけ	コミックマーケットの略称で、コミケットとも呼ばれる。	コミケとはコミックマーケットの略称。 コミックマーケットとはコミックマーケット準備会主催の大規模な同人誌展示即売会を意味し、正式には『コミック=マーケット』と表記する。他にもコミックマーケットの略称としてコミケットがあるが、最もポピュラーな呼び方がコミケである。 コミケは現在、8月中旬と12月末の年2回、東京国際展示場で催され、それぞれ夏コミ、冬コミと呼ぶ。ジャンルはアニメ ・ 漫画が主体ではあるが特に限定はされていない。その歴史は古く、第1回コミックマーケットは1975年(昭和50年)12月21日に日本消防会館会議室で行われる。また、最近のコミケでは同人誌販売のみでなく、コスプレのイベントも行われている。 コミックマーケット、コミケット、コミケはいずれも有限会社コミケットの登録商標(1998年登録)であるが、コミケの知名度があがるとともに他の同人誌展示即売会もコミケと呼ばれるようになる。サランラップは旭化成株式会社の一商品名であるが、ラップ全体がサランラップと呼ばれるのと同様の現象である。	2000年	略語 
\\	若者言葉	
\\	シャカシャカ	しゃかしゃか	防寒用のナイロン着のこと。	シャカシャカとはスポーツをするときに履く防寒用のナイロン着やウィンドウブレイカーのことで、歩いたり走ったりしたときに「シャカシャカ」という音がすることから若者にそう呼ばれるようになった。同義語に『しゃかパン』があるが、こちらはパンツと限定されており、シャカシャカの一部である。	2000年	若者言葉、美容 ・ ファッション用語	
\\	シロガネーゼ	しろがねーぜ	高級感のある主婦のこと。	シロガネーゼとは「ミラノっ子」という意味のイタリア語
\\	(ミラネーゼ)』をもじった言葉で、高級住宅街である東京都港区白金台周辺に住む専業主婦や高級感を醸し出すために白金台周辺でショッピングを楽しみ女性のことを指す。シロガネーゼは光文社の女性誌
\\	が1998年に使った造語で、2000年辺りからメディアが使うようになり普及。キンチョーが
\\	で北斗晶に「シロガネーゼもダラシネーゼ」と言わせたことで更に広く浸透した。	2000年	
\\	セメラー	せめらー	陰陽師;安倍晴明のファンのこと。	セメラーとは平安時代の陰陽師である安倍晴明(あべのせいめい)に、人化する接尾語「~ラー
\\	をつけたもので、安倍晴明のファンを意味する。1993年~2005年に連載された岡野玲子の漫画『陰陽師(夢枕獏の小説『陰陽師』を漫画化したもの)』で描かれた安倍晴明が人気となり、晴明ブームなるものが起こる中、セメラーという言葉も誕生した。2001年には野村萬斎が安倍晴明を演じた映画『陰陽師』も話題となった。	2000年	
\\	男介の世代	だんかいのせだい	病気の妻を介護する夫のこと。	男介の世代とは『団塊の世代』のもじりで、病気の妻を介護する60~70代の男性を意味する。男介の世代は「亭主の代わりはないけれど、市長の代わりはある」といって辞任した大阪府高槻市の江村利雄元市長、妻の介護を理由に相談役へ退いた東海銀行の西垣覚元会長がメディアで取り上げられる際に使われた言葉である。後に母を介護する男性など、家族を介護する男性という意味で使われるようになる。	2000年	
\\	つーか	つうか	相手に反論したり、話しを切り返す際の言葉。	つーかとは「と言いますか」が崩れた「って言うか」が更に崩れたもので、相手の発言に対し、言い換えたり、反論する際に文頭に付けて使用。 2000年になると反論だけでなく、「才能も実力もあって凄いと思う。つーか、凄いんだよ」、「見ててイライラする。つーか、うざい」といったように、自分が話した内容を強め、畳み込むように言う際にも用いられる。また発言を切り出す時にも文頭に付けて使用する(この場合、つーか自体は特に意味を持たない	2000年	
\\	テクハラ	てくはら	ハイテク技術のスキルの高い人が低い人にする嫌がらせのこと。	テクハラとは『セクハラ(セクシャル ・ ハラスメントの略)』のもじりで、「テクノロジー ・ ハラスメント(テクニカル ・ ハラスメント)」の略である。
\\	(科学技術)+
\\	(嫌がらせ)、つまり、テクハラとはパソコンなどハイテク技術に詳しい人がそうでない人にする嫌がらせのことである。具体的にはコンピュータが疎い人にわかりそうもない専門用語で説明する、簡単に解決する内容をわざと難しい手法で教えるといったものがテクハラに相当する。テクハラは当初、パソコンに疎い中高年に対する嫌み言葉であったが、次第に年齢に関係なく使われるようになり、更に事務機器や
\\	機器など幅広いジャンルで使われるようになる。	2000年	若者言葉	
\\	デパ地下	でぱちか	デパートの地下売り場のこと。	デパ地下とは『デパートの地下売り場』の略だが、意味としては「地下の食品売り場」を指すため、正確には『デパート地下の食品売り場』の略になる。デパ地下は
\\	1(ビーワン)』『地下食(ちかしょく)』とも呼ばれ、以前から人気があったが、特に2000年以降はデパ地下ブームと言われるほど多くの
\\	番組や雑誌が取り上げるようになる。それまでも特産品展など特徴的な品揃えをしていたが、特にこの頃から有名店の出店やスイーツの品揃え強化など女性をターゲットにした展開 ・ 差別化したためで、デパ地下グルメという言葉も生まれる。昼間の主婦はもちろん、夕方には仕事帰りの
\\	で賑い、デパ地下が客寄せの要とも言われる。	2000年	略語	
\\	デフォ	でふぉ	デフォルトのこと。	デフォとは英語で初期設定 ・ 初期値(にする)、怠慢、欠席(する)といった意味の英語
\\	(デフォルト)』の略である。日本ではデフォルト自体、初期設定、初期値の意で使われることが多く、それを略したデフォもそうした意味で使われる。車やパソコンで顧客の希望で付けられるオプションがあるが、例えばこれらオプションがついていない初期出荷状態。また外食では、大盛 ・ 小盛に対する基本量や濃い味 ・ 薄味、辛口 ・ 甘口といった味に対する基本の味がデフォにあたる。また、女性の顔に関して、普段から化粧をしている人は化粧した状態をデフォ、化粧は特別なときにする程度で普段スッピンの人はスッピンの状態をデフォという。	2000年	略語	
\\	デリバる	でりばる	出前をとること。	デリバるとは英語で出前を意味する
\\	(デリバリー ・ サービス)」のデリバリーに動詞化する接尾語「る」をつけたもので、出前をとることを意味する。また、同じくデリバリーに動詞化する接尾語「る」を付けたもので、デリバリーヘルス(=女性を派遣(出張)させるタイプのファッションヘルス)を利用するという意味でも使われる。それぞれ、デリると略して使われることもある。	2000年	
\\	2ちゃん	にちゃん	いくつもの匿名掲示板群で構成されたサイト『2ちゃんねる』のこと。	2ちゃんとは西村博之(通称:ひろゆき)が主宰する匿名掲示板群が置かれたサイト『2ちゃんねる』(1999年5月30日開設)のことで、2005年9月現在、日本で最大のアクサス数を有する掲示板サイトである。匿名性が強いため、他では知ることの出来ない情報を得ることが出来たり、本音を書き込める利点がある反面、それを逆手にとったイタズラや無責任な誹謗中傷などの書き込みも後を絶たず、2ちゃんの存在は絶えず賛否両論をよんでいる。 2ちゃんねるの語源に関しては諸説ある。主催者の西村博之は、関東地方のビデオやゲーム用(様々なビデオやゲームが利用出来る)チャンネルである2chと、様々な話題を扱う掲示板サイトという内容がリンクして出来たとしているが、開設当初から知っている人の間では、それまであった『あめぞう』という掲示板の後継サイトであったため、『あめぞう』が1ch、その後継であるから2ちゃんねるであるという見解が多い。 事件などメディアで取り上げられるまでは、「ツーちゃんねる」とも読まれていたが、メディアが「にちゃんねる」と報じるようになり、ほぼこの読みで統一されている。また、2
\\	とも表記される。2ちゃんを利用する人を『2ちゃんねらー』や『住人』、各掲示板のことを『板』、掲示板内のスレッドのことを『スレ』など、2ちゃん特有の呼び方がある他、2ちゃんから生まれた2ちゃん用語は数々あり、書籍にもなっている。 また、2004年~2005年にかけて書籍、映画、ドラマとして一般的にも話題となった『電車男』で使われている掲示板も2ちゃんねるのことで、『電車男』は2ちゃんねるの『独身男性板』に書き込まれたログを元にしたラブストーリーである。 註)日本語俗語辞書では、一般的に普及した俗語の解説を行っています。特定サイトのみで使われている用語の掲載追加依頼に対しては対応出来ませんので、ご了承ください。「萌え」や「電車男」「ツンデレ」のように、一般にも普及した言葉については随時掲載していきます。	2000年	略語(固有名詞の俗称)、ネットスラング	
\\	ネット番長	ねっとばんちょう	インターネット上では強気だが、実際にはおとなしい人のこと。	ネット番長とはインターネットの掲示板やチャット、
\\	、他者のブログへのコメントなど、ネット上では威勢がよく強気だが、オフ会などで実際に会ってみたら無口でおとなしかったり、気弱な性格をした人のこと。つまり、ネット番長とは「ネットの中の番長」ではなく、「ネットの中だけの番長」を意味する。	2000年	ネットスラング	
\\	パラサイト	ぱらさいと	結婚せずに親と同居している人。	"パラサイトとは適齢期を過ぎても結婚をせずに親元で暮らす人のことで、正確にはパラサイト ・ シングルという。パラサイトは英語の
\\	""からきているカタカナ語だが、
\\	の意味が「寄生虫」「寄生植物」であることからもわかるように、「結婚をせず、親元で甘えて生活(寄生 ・ 居候)する」といった悪いイメージで使われることが多い。実際、十分な収入があるにも関わらず、裕福な暮らしをするためにパラサイト ・ ライフ(結婚をせず、親に甘えた生活)を過ごす人や、育児 ・ 家事をしたくないといった理由でパラサイト ・ ライフを送る人が多い。逆に結婚はしたいけれど出会いがないなどの理由で結果的にパラサイトとなっている人もいる。どちらにしてもパラサイトの増加が少子化問題を深刻化させるのも事実であり、社会問題のひとつとしてクローズアップされている。
\\	2000年	カタカナ英語	
\\	ヒッキー	ひっきー	
\\	ひきこもる人(ひきこもり)の略語。 
\\	宇多田ヒカルの愛称。	ヒッキーとは外出せず、インターネットやゲームばかりしている非社交的な人をさす『引きこもり』を略し、英語で人化する接尾語
\\	をつけたもので、2ちゃんねるを中心としたインターネットで普及した言葉である。元となる『ひきこもり』に関しては該当ページ参照。 また、これとは別にヒッキーは歌手『宇多田ヒカル』の愛称でもある。
\\	とも表記される。	2000年	略語 ・ 若者言葉	
\\	ボラバイト	ぼらばいと	ボランティアとアルバイトの中間的労働スタイルのこと。	ボラバイトとは英語の
\\	(ボランティア)と独語の
\\	(アルバイト)から成る合成語で、賃金よりも人手の足りない仕事を手伝うこと、未経験の仕事を経験すること、そこでの人との触れ合いを目的とする労働(活動)を意味する。例えば、農家の農繁期、宿泊施設のハイシーズン
\\	、夏休みなど)といった地方で人手を必要とする時期の手伝いの他、
\\	やボランティア団体の臨時スタッフといったといったものもある。労働期間は日帰りから住み込みで1ヶ月以上といったものまで様々。最低賃金法に定められた最低賃金を最低基準とするボラバイト料は時給300~600円前後と低賃金だが、そういった場所で働きたいと思っている人が募っており、また農家など雇う側からは感謝され、その満足感で苦にならないという。また、ボラバイトをする人はボラバイターという。	2000年	
\\	ボンレスハム	ぼんれすはむ	網タイツを履いた足の太い女性のこと。	ボンレスハムとは豚のもも肉を使ったハムのことだが、作る際にタコ糸で肉塊をしばり、燻煙(スモーク)する。このタコ糸で縛った状態の肉塊を網タイツを履いた女性の足にみたて、網タイツを履いた太い足の女性をボンレスハムと言うようになる。2000年の網タイツブームの際に普及した言葉である。	2000年	
\\	ママギャル	ままぎゃる	ギャルのような格好をした母親のこと。	ママギャルとはガングロに茶髪、厚底サンダルといったギャルのような格好をした母親のことをいう。ママギャルのほとんどは過去にギャル(コギャル)の格好をし、そのまま母親になったというパターンである。また、ママギャルのサークルも存在する。これは独身時代、同じギャル仲間で交流していたグループによって運営されていることが多い。ギャルママともいう。	2000年	若者言葉	
\\	メッセ	めっせ	インスタント ・ メッセンジャーの略。	メッセとはインターネット上で同じソフトを使って文字などによる会話やファイル転送を行う「インスタントメッセージング」や、それを行うためのソフト「インスタントメッセンジャー」を略したものである。日本でも1990年代後半から
\\	でメッセが浸透。その後、ヤフーチャットと連動し、それ自体で友達を作ることが容易な
\\	メッセンジャーや
\\	(ホットメール)と連動している
\\	メッセンジャーといった大手ポータルサイト提供のメッセが
\\	カメラやヘッドセットを使って
\\	電話のように使えることもあり人気を博している。また、最近では通話に特化した
\\	(固定電話や携帯電話との通話も可能:その場合は有料)も人気がある。	2000年	ネットスラング	
\\	元ダチ	もとだち	旧友のこと。	元ダチとは「元友達」の略、つまり旧友のことである。 「友達」のもじりという解釈も出来るが、元ダチが普及する以前に「元彼(もとかれ ・ もとかの)」という言葉が普及しており、元ダチもこの流れで出来た言葉と思われる。ただし、元彼が現時点では彼氏 ・ 彼女という関係でない場合に限られるのに対し、元ダチは音信不通になっているだけで友達としての関係が特に終わっていない場合も対象になる。	2000年	若者言葉	
\\	ヤマンバ	やまんば	山姥のような髪型のギャル(コギャル)のこと。	ヤマンバとは顔を日焼けサロンで焼いたり、黒系のファンデーションを使って黒くし、髪型は金や銀 ・ 白といった色に脱色、乱れた髪型が特徴のコギャル ・ スタイルのひとつである。民話に登場する山姥(やまうば ・ やまんば)の多くが白髪の乱れ髪であることからヤマンバと呼ばれるようになった。『ヤマンバギャル』という呼び方や『マンバ』と略されることもある(ただし、ヤマンバとマンバはメイクなど厳密には異なる)。	2000年	若者言葉	
\\	やり逃げ	やりにげ	性行為をするだけして、姿を消したり、別れること。	やり逃げとは性行為を済ませた後、すぐに姿を消したり、別れてしまうことである。本来は男性のこうした行為を指したが、近年、インターネットの出会い系サイトなどを介し、やり逃げをする女性も増えており、必ずしも男性を対象とした言葉ではなくなってきている。また、厳密には性行為を目的に近寄り、目的を果たしたら姿を消すことをいうが、体目当てでなく、恋愛及び結婚を考えて近寄った場合であっても、性行為後、結果的に音信不通になっていればやり逃げと呼ばれる。 更にここから性行為以外の場合でも、楽しむだけ楽しみ、その後始末や代償としての約束を放棄して姿を消すことをやり逃げというようになる。	2000年	
\\	ヤン車	やんしゃ	ヤンキーが好む仕様の車のこと。	ヤン車とは「ヤンキー(仕様)の車」の略で、窓にスモークを張ったものや車高を下げたもの、マフラーに穴をあけたり交換して排気音を大きくしたものなど、このような改造が施された車のことである。 ただし、ヤン車仕様の車がひとつのスタイルになっており、こういった改造を施した車のドライバーが必ずしもヤンキーであるとは限らない。	2000年	不良 ・ ヤクザ用語	
\\	夜スポ	よるすぽ	終業後にスポーツクラブに行くこと。	夜スポとは「深夜スポーツジム」及び、これに類するもの(動詞がつくなど、文章化されたものなど)の略で、終業後、スポーツジムに通うことを意味する。サラリーマンや
\\	など、昼に運動する時間の取れない人が終業から帰宅までの時間を利用し、夜スポをする。広義には「深夜にスポーツする」の略とし、スポーツジム以外にもボウシング(ボクササイズ)ジムやスイミング、また、そういった施設に行かなくても近所をジョギング、ウォーキング(散歩)することも含む。	2000年	
\\	オタカラ	おたから	オタク的楽しみ方をするカラオケのこと。	オタカラとは厳密にはオタ ・ カラと書き、カラオケをする際のオタク的な楽しみ方、またはそういった楽しみ方をするカラオケのことをいう。例えばアニメオタクであれば、アニソン(アニメソング)だけをひたすら歌う。ミュージカルオタクであれば、好きなミュージカルの曲を役になりきって歌う。アイドルオタクであれば、一人がアイドルになりきって歌い、他が親衛隊として応援コールをするといったものがオタカラである(これらは一例であり、同じアニメオタクでも異なるオタカラの方法が存在する)。オタカラは同好の者だけで集まって行われるものであり、単にカラオケをしたい ・ 歌いたいという人を交えては行われない。	1999年	若者言葉	
\\	カラハラ	からはら	カラオケで強制的に歌わせる行為。	カラハラとは「カラオケ ・ ハラスメント」の略。ハラスメントとはセクハラ(セクシャル ・ ハラスメント)に見られる通り、嫌がらせという意味である。つまり、カラハラとはカラオケによる嫌がらせという意味になるが、具体的には当人が歌いたいか否かに関係なく、強制的に歌わせる行為をいう。ただし、大半は自分がカラオケ好きであることから、他者もカラオケ好きと思い込んで歌本やマイクを回すといった善意から起こす行為であり、他のハラスメント系の言葉
\\	ハラ)のように嫌がらせを意図した行為とは限らない。	1999年	
\\	逆ギレ ・ 逆切れ	ぎゃくぎれ	注意を受けた人が指摘した人に怒ること。	逆ギレとは違反や間違いなどし、注意されたり、苦情を言われている者が逆に怒り出す現象をいう(逆にキレるの略)。逆ギレは1900年代に入り、『キレる』という言葉の普及と併行して若者の間で使われるようになる。1990年代末には世代を超えて広く普及。普段からキレやすい人が注意を受けて逆に怒り出すものだけでなく、普段はおとなしいがひどく注意を受ける間に堪忍袋の緒が切れて怒り散らすといった場合もある。後者が転じ、普段いじめられている子が逆上して暴れ出すことも広義に含まれる。また、逆ギレと逆上の混同から間違いを正すため、怒るべくして怒っている人を含めて使う人もいる。	1999年	若者言葉	
\\	きょどる	きょどる	挙動不審な行動をとること。	きょどるとは挙動不審の『きょど』に俗語でよくある名詞を動詞化する接尾語『る』をつけたもので、挙動不審な行動や怪しい態度をとることである。また、広義にはそういった行動 ・ 態度の人や動物を指しても使われる。	1999年	若者言葉	
\\	コシる	こしる	怖気づくこと。	コシるとは「腰が引ける」の略で怖気づくこと。ビビるの類似語で、喧嘩で腰が砕けたり、好きな異性の前で口が利けなくなるといったことがコシるにあたる。ここから、口先だけで実際には逃げ腰なこともコシるというようになる。 これとは別に「腰を振る」の略として用いられることもある。	1999年	若者言葉	
\\	サラウーマン	さらうーまん	正社員雇用で勤めている女性のこと。	サラウーマンとはサラリーマンと女性を意味する英語“
\\	(ウーマン)”から成る合成語で、女性サラリーマンを意味する(サラリーマンとは正社員雇用され、サラリー(俸給、月給など)で生計をたてている会社員)。
\\	に「お気軽」「腰掛け」といったイメージがあるが、実際には女性の仕事に対する姿勢 ・ 考え方にそういった点は無く、企業の中核を成す女性も増えている。そういった風潮の中で生まれたサラウーマンだが「会社員」「ワーキングウーマン」「ブジネスウーマン」が使われて浸透せず、死語となっている。	1999年	
\\	ザラリーマン	ざらりーまん	平凡なサラリーマンのこと。	ザラリーマンとは「ざらにいるサラリーマン」を略した若者言葉である。「ざら」とは世の中にあって珍しくないさまを表す俗語であり、ザラリーマンは珍しくないサラリーマン、つまり平凡なサラリーマンを意味する。第三者が揶揄するよりも自嘲する際に使われることが多い。 2005年にはお笑いコンビ千原兄弟のコント「矢ザラリーマン」がグッズ販売されるほどの人気となる。ただしこれはザラリーマンからきたものでなく、矢ガモ、矢バトなど、矢が刺さると濁点で始まるという考えからきたものである。	1999年	若者言葉	
\\	試婚	しこん	同棲のこと。	試婚とは試験的に結婚生活を営むことで、俗にいう同棲のことである。知り合ってからプロポーズまでの時間が短いとき、相手の生活や性格でわからない点が多い。そこで一緒に暮らしてもやっていけるかどうかを確かめるためにする同棲が試婚である。1960~70年代に流行った同棲とは違い、女性が採点者として優位にたっているのが特徴である。 して優位にたっているのが特徴である。 また、広義には何らかの理由(法律婚に反対といったものや借金など)で単に籍を入れていないだけの同棲生活(=事実婚)のことも試婚という。	1999年	
\\	ジモ着	じもぎ	家の近くへ、ちょっと出かける際に着る服。	ジモ着とは「地元(じもと)」と、上着 ・ 下着 ・ 家着に見られる「着」を合わせた言葉で、地元で着る服を意味する。例えば、家の近所にあるコンビニエンスストアへ、ちょっと買い物に行く際や郵便を出しに行く場合、また、犬の散歩などで着る服をいう。ジモ着はパジャマやジャージなどの家着と普段着の間に位置するものだが、具体的にどういった服がジモ着という定義はなく、その人の感性によって異なる。	1999年	若者言葉	
\\	週末婚	しゅうまつこん	週末だけ一緒に過ごす夫婦のこと。	週末婚とは何らかの事情で週末(休日)のみ会ったり、週末のみ生活をともにする夫婦のことである。週末婚は当初、子供をもった人が再婚する際、子供が新しい親に成れるために行われたものを指した。後に単身赴任などの理由で週末しか会えない夫婦のことも週末婚と呼ぶようになる。1999年には内館牧子脚本、永作博美 ・ 松下由樹出演のドラマ『週末婚 
\\	で広く認知される言葉となる。流行語的に週末婚が普及してからは、平日は一人の生活を楽しみ、週末だけ夫婦生活を楽しむといった、特に事情なく週末婚という形態を望む若者も増えている。	1999年	
\\	シュノラー	しゅのらー	収納好きな人のこと。	シュノラーとは『収納』に英語で人化する接尾語
\\	をつけたもので、収納好きな人(主に主婦)を意味する(アムラー、シノラーの流れで出来た言葉)。
\\	で朝から昼にかけて放送される奥様向け番組に収納の達人と言われる人が登場したり、収納本も売上げを伸ばす中で出来た言葉である。部屋のちょっとした隙間を利用し、パズル感覚で収納していくシュノラー。一見、いいことのように思えるが、収納のために揃えた工具や材料購入費のために生活費を削ったり、見よう見真似で片付けたはいいが、どこにしまい込んだかわからなくなるなど、本末転倒な事態を招いているシュノラーも少なくない。	1999年	
\\	セッチャ	せっちゃ	盗まれた自転車のこと。自転車を盗むこと。	セッチャとは窃盗(せっとう)と自転車の俗称であるチャリンコ(下記関連語参照)から成る言葉で、盗まれた自転車、または自転車を盗むという行為を意味する。窃盗チャリ、窃チャリといった略も使われたが、セッチャはこれが更に簡略化されたものである。	1999年	若者言葉	
\\	セレブ	せれぶ	お金持ちで品のある、高級感漂う女性のこと。	セレブとは英語
\\	(セレブリティ)の略だが、日本のセレブはセレブリティの本来の意味である有名人(著名人)、名士とは異なる意味合いで普及しており、新たに出来た和製英語と考えてもよい。海外では例え大富豪でもメディアを嫌い、一般的に名が浸透していない人をセレブとは言わないが、日本では逆に一般に知られていない人でも、優雅なお金持ちであればセレブという。例えば東京の白金台に住むシロガネーゼと呼ばれる優雅な暮らし向きの一般人もセレブにあたる。また、高級という意味合いが強くなったセレブはセレブなお部屋、セレブ感のあるファッションなど、人以外にも使われるようになる。	1999年	和製英語(略語)	
\\	早朝族	そうちょうぞく	早起きして趣味や勉強をする人のこと。	早朝族とは早起きして趣味や習い事、勉強など、出勤までの時間を利用して行う人のことをいう。こうしたことは終業後のアフター5にするのが一般的だったが、その場合、突然の残業などで予定が狂うことも多い。しかし、早朝族にはそういった心配がない他、勉強会や教室へ向かうため、早めに都心入りすることになり、通勤ラッシュも回避出来る。更に早起きで健康にもよく、一石二鳥も三鳥もあるとして注目された。ただし、こうした考え方は前々からあり、以前は『ビフォア9』といわれた。また、昼過ぎまでダラダラと寝て過ごしがちな週末に早起きし、有意義に過ごす人を意味する『週末早朝族』という言葉もある。	1999年	
\\	タオラー	たおらー	白いタオルを頭や首に巻いている若者のこと。	タオラーとはバンダナやスカーフの代わりとして、白いタオルを頭や首に巻いている若者のこと。もともとは運動部員から普及したとされるタオラー。冬は首に巻いて寒さを防ぎ、夏は長い髪をまとめれば、防暑と吸汗が同時に出来るという実用的な面から若者の間で浸透。タオラーと呼ばれるような頃には機能性以上に一つのファッションとして普及していった。大人の間でも浸透しており、休日や職種によっては仕事中にタオルを巻く者もある。	1999年	若者言葉	
\\	着メロ	ちゃくめろ	着信音用に始められた携帯電話・
\\	のサービスのひとつ。	着メロとは着信メロディーの略で、携帯電話 ・ 
\\	の着信音用に単音から64和音までの電子音で作られた楽曲 ・ メロディーのこと。近年の携帯電話機はデフォルト(購入時)の状態でいくつかの着メロが入っている。また、最新のヒット曲や名曲などの着メロを配信する『着メロサイト』からダウンロードも出来る。著作権が発生する楽曲の着メロを配信する場合、作曲者に使用料を支払わなければならない。そのため、着メロも有料配信や広告メールの受信会員などに対するサービスという形で配信される場合が多いが、中には無断で着メロを作成~配信する違法サイトもある。 また、『着メロ』は株式会社
\\	の登録商標である。	1999年	サービス名	
\\	中二病	ちゅうにびょう	思春期にありがちな行動 ・ 思考(症状)のこと。	中二病とはタレント伊集院光がラジオ番組「深夜の馬鹿力」の中で提唱した思春期(中でも男子)特有の行動 ・ 思想のことで、それまで曖昧に語られてきたものを具体的に考察 ・ 提示したものである。症状としては「美味いと思っているわけでもないコーヒーを飲みだす」「洋楽を聞き出す」「タバコを吸うわけでもないのにジッポも持ち始める」などの背伸び(大人の真似事)や「サラリーマンにだけはなりたくない」「大人は汚い」「アメリカは汚い」といった反抗 ・ 批判的言動など多くの項目があげられ、サブカル系 ・ 尾崎系 ・ 自慢系 ・ 妄想系といった分類もされている。そしてこうした症状が出始めるのが中学二年生前後であることから中二病と呼ぶ。なお中二病はネタとして生まれた言葉が番組の枠を超え、賛同者を介して広まった言葉であり、医学的にも心理学的にも立証 ・ 承認された病気 ・ 症状ではない。また、中二病は大人ぶったり、不良ぶったり、又はそれらに憧れる程度に留まるものであり、校則違反程度ならまだしも、喫煙 ・ 飲酒 ・ 破壊 ・ 暴力といった違法行為は中二病の枠を逸脱した単なる不良 ・ 
\\	とされる。 中二病を意識し過ぎるあまり何も行動出来なくなる人を「裏中二病」、中二病の人を批判 ・ 注意 ・ 嘲笑する人のことを高二病、高二病の人が中二病を非難 ・ 嘲笑する姿を見て非難 ・ 嘲笑することを大二病などの派生語も存在する。	1999年	若者言葉	
\\	成田破局	なりたはきょく	婚前海外旅行で婚約解消すること。	成田破局とは婚前にする海外旅行がキッカケで婚約を解消すること、さらに婚約はしていないまでも結婚は考えていたカップルで婚前海外旅行を機に冷めてしまった場合も成田破局に含まれる。成田破局に至る理由は海外旅行という特殊な環境の中で相手の器の小ささや粗など悪い面が見えたことによる。元となる成田離婚は男性が原因で女性が冷めてしまう場合に限られるのに対し、成田破局は女性が原因で起こる破局も含まれる。ただし、成田離婚や同じくその派生語である新成田離婚に比べ、普及度の低い言葉である。	1999年	
\\	ネットアイドル	ねっとあいどる	インターネット上のアイドルのこと。	"ネットアイドルとは""インターネット""と「偶像」を意味する英語
\\	""から成る和製英語で、簡単にいえばインターネット上のアイドルである。ネットアイドルは活動の場がネット上であることから、自身もしくは協力者が運営する専用のサイトを持っていることが前提となる。それ以外の定義は曖昧で、自称ネットアイドルのようにファンの存在の有無も特に問われない。また、最初からネットアイドルとしてサイトを開設するタイプや元々はコスプレイヤーなど他の目的でサイト開設したものが人気を呼びネットアイドル化した者、サイトに写真は掲載していないが日記などの文面からアイドルのように人気を得た者、更に実在する人物ではなく、架空の2次元キャラクターを使ったバーチャルネットアイドルといったものまである。ネットアイドルが最初にブームになったのは1990年代末で
\\	出演をした者もいるが、中には画像による性的アピールでファンや話題性を獲得しようとするネットアイドルもあり問題になる。 ネットアイドルは基本的に女性や女性キャラクターに対して使われ、男性の場合は『ネットボイドル』または『ボイドル』と呼ばれる。また、ネットアイドルはネトアと略される場合がある。
\\	1999年	ネットスラング ・ 和製英語	
\\	ネチケット	ネチケット	ネットワーク ・ エチケットの略。	ネチケットとは、ネットワーク ・ エチケット
\\	の略です。国際的にはインテル社のサリー ・ ハンブリッジによる
\\	にガイダンスが公開されており、インターネットを中心としたネットワークを使う上でのマナーが記されています。当初は知的財産の共有するためのシステムだったインターネットですが、利用者の増加とともにネットワーク内のネチケットが重要視されています。法的な拘束力はありませんが、ネットワークを利用する場合(特に
\\	サイト(ホームページ)を運営される場合や電子メールを扱う場合)、最低限のマナーとして知っておきましょう。	1999年	ネットスラング、略語(和製英語)	
\\	ネバラー	ねばらー	ネバネバした食感の食材を好む人のこと。	ネバラーとは「ネバネバ」という食感を表現する言葉に「~する人」という意味にする接尾辞「ラー(註1)」をつけたもので、オクラや納豆、山芋、とろろ芋、モロヘイヤなどネバネバとした食感の食材を好む人のこと。ネバラーという言葉が誕生する少し前にマヨラーが普及しており、その派生語と考えてよい。ただし、ネバラーはマヨラーのように広く認知されることなく死語となっている。 註1) 本来は英接尾辞“-
\\	をつけ、「ネバー」となるが、1990年代に人化されたマヨラーを含む一連の語句(下記関連語参照)はアムラーの派生語であるため「ラー」がつけられた。	1999年	
\\	はにわ状態	はにわじょうたい	呆然とした状態のこと。	はにわ状態とは顔(表情)が埴輪(はにわ)の顔のようになる状態をいう。埴輪の多くは目と口の部分に同程度の丸い穴が開けられ、鼻が付けられたシンプルな造りになっている。この顔がポッカリ口を開いているように見えることから、口をポッカリと開けてしまうような呆然とする状態をはにわ状態という。	1999年	
\\	バンツー ・ 番通	ばんつう	電話の発信者の番号を通知するサービスのこと。	バンツーとは「電話番号通知サービス(でんわばんごうつうちさーびす)」の略で、固定電話機のナンバー ・ ディスプレイといったサービスや携帯電話の発信者番号通知機能を意味する。こうしたサービスや機能により、電話に出る前にディスプレイに表示された番号(または番号で登録した名前)から誰からの電話か察することが出来る。バンツーを見て、都合の悪い相手のときはブチるということもある。(ブチるとは電話に出ず、無視すること:関連語参照)	1999年	若者言葉 ・ 略語	
\\	プチ家出	ぷちいえで	数日~1週間程度の家出のこと。	プチ家出とは小さいという意味のフランス語“
\\	が転じ、「少し」といった意味で使われるカタカナ仏語『プチ』と『家出』から成る言葉で、数日から1週間程度の家出をいう。親子喧嘩などで対話が途切れ、そのまま自部屋に行くような感覚で友達宅へ家出し、数日後に戻ってくるプチ家出が急増。特に親子喧嘩といった原因もなく、プチ家出をする子供も多い。携帯電話で連絡したり、居場所を告げて出るなど、これまでの家出に比べ、安心出来る側面もある。しかし、犯罪に巻き込まれる子供も多く、社会問題となっている。また、家出を意味する俗語『やさぐれ』から成るプチやさぐれという同義語も使われる。	1999年	
\\	別居婚	べっきょこん	住居を別にしたまま結婚する形態のこと。	別居婚とは住居を別にしたまま結婚するスタイルのことである。働く女性が増え、互いのライフスタイルを崩さないために別居婚を選択するという理由が多く、他には再婚で子供がいる場合、とりあえず籍を入れ、子供が慣れさせるためにとりあえず別居婚から始めるという場合もある。仕事の都合などから単身赴任になるなど、結婚当初は同居し、途中から別居になる場合は別居婚と呼ばない。また、別々に暮らし、一緒に過ごすのが週末だけの場合、特に週末婚という。別居婚により自分のライフスタイルが守られるというメリットがある反面、なかなか互いを理解しあうことが出来ないというデメリットもある。	1999年	
\\	ベランダー	べらんだー	ベランダで花や野菜を育てる人のこと。	ベランダーとはベランダに人化する英語の接尾辞“-
\\	を付けたもので、ベラダンにプランターや鉢を置き、そこで花や野菜を育てる人のことをいう。一軒家の庭で園芸(ガーデニング)をする人のことをガーデナーと呼ぶが、ベランダーはその対語にあたる。アパートやマンション、庭のない一軒家に住む人向けにベランダ ・ ガーデニングを提案する園芸ライフスタイルマガジン
\\	(プランテッド)」の編集長“いとうせいこう”による造語である。	1999年	
\\	メガコン	めがこん	メガネコンシャスの略、叉は18以上のスクリーンを持つ映画館。	
\\	メガコンとはメガネ ・ コンシャスの略である(コンシャス
\\	は意識している、意識的なという意味。日本では他の言葉と組み合わせて使われることが多く、ボディコンで使われて普及した言葉)。視力補助のためのメガネだが、その形状や色から顔のイメージを大きく左右もする。ここから
\\	に合わせて複数のメガネを持つなど、メガネをファッションアイテムのひとつとして意識的にはめ変えることをメガコンという。この意味では1999年に普及。ただし関東ではメガネ ・ コンタクトのチェーン店『メガコン』をさして使われることが多い。 
\\	メガコンとは18以上のスクリーンを持つ映画館(複合施設)のことで、シネコンの中のひとつである。ただし、シネコンと違い、認知度は低い。	1999年	若者言葉	
\\	ヤリコン	やりこん	性行為を目的とするコンパのこと。	ヤリコンの『ヤリ』とは性交を下品に表現した「やる」の変形で、「やり逃げ」「やりまくり」にも見られる。『コン』はコンパの略。つまりヤリコンとは男女合意の上で行われる性行為を目的としたコンパや飲み会のことである。ヤリコンには一次会にホテル(ラブホ)の一室を借り、始めから飲み会と性交を自由に行うもの、一次会は居酒屋などで飲み会をし、二次会 ・ 三次会でホテルや参加者の家で宅飲みというように性交が行える場所に移動するものなどがある。	1999年	若者言葉	
\\	ワン切り	わんぎり	電話をかけ、呼び出し音を一回だけ鳴らし(又は鳴らさず)、電話を切ること。	ワン切りとは携帯電話や
\\	に電話をかけ、呼び出し音を1回だけ鳴らして電話を切り(数回の場合や鳴る前に切る場合もある)、相手の電話機の着信履歴に番号を残す行為である。ワン切りは知り合い同士でかけ直して欲しい時や、電話番号を伝える(交換する)時にも使われるが、高額なサービスへ誘導するための業者からのものが多い(主にアダルト関係やツーショット ・ ダイヤルなど)。こうした業者からのワン切り対策として、知らない番号からの着信履歴にはかけ直さない。仮にかけてしまった場合も、サービス利用の合意を業者と交わしていなければ無効な契約なので支払う必要はない。ただし、最近の業者のやり方は手が込んでいますので、請求書が来た場合は、最寄の国民センターや弁護士に相談することをお薦めします。脅迫じみた取り立てや電話がある場合は、慌てて支払うのでなく、警察へ相談するようにしましょう。 ワン切りの相手を確認したい場合は、公衆電話を利用するのが一番安全です。電話機からかけたい場合は、そのままダイヤルするのではなく、ダイヤル前に
\\	を回してから、番号をダイヤルしましょう。この非通知通話により、相手には自分の電話番号を知られずにすみます。 頻繁なワン切りの着信音で困っている方は、着信があってから数秒後にメロディーが鳴り始めるタイプの着メロのダウンロードや、着信音をステップトーン(徐々に着信音が大きくなる)に設定することで解消出来ます。 また、登録した電話番号に対して、自動で応答するタイプの携帯電話もあります(ワン切りであっても自動応答)。かかってきたワン切り業者からの番号を登録しておけば、業者に通話料金が発生します。これを利用しようという消費者運動もあります。ワン切り業者の電話番号を共有し、大勢でその番号を登録します。業者の多くはコンピューターなどで無作為に何万件ものワン切りをしていますから、業者に高額な通話料金が発生し、ダメージが与えられるわけです。	1999年	
\\	ワンコ	わんこ	
\\	犬のこと。 
\\	携帯電話で呼び出し音を一回だけ鳴らして切ること。	
\\	ワンコとは犬を意味する俗称のひとつ。他にも犬を嘲う俗称「ワン公」があるが、ワンコがこの略かは不明。ワンコはこちらの意での使用が多い。 
\\	ワンコとは「ワンコール」の略で、携帯電話で呼び出し音を一回だけ鳴らすことを意味する。携帯電話では一回コールするだけでも相手に誰からの着信か知ることが出来る。これを利用し、無料で意思伝達をする若者も多い。例えば昼前に
\\	さんからワンコがあった場合、ランチの誘いということを事前に決めておく。こうしておけば電話に出ることなく、ワンコによる着信だけで
\\	さんとランチが出来るといったものである。	1999年	若者言葉	
\\	アイコラ	あいこら	アイドルコラージュの略。	アイコラとはアイドルコラージュの略で、タレントの写真を利用した加工画像のことである。コラージュ
\\	とは新聞や布片などをキャンパスに貼りあわせて作る現代絵画の技法のひとつだが、アイコラも多くはタレントの顔写真を他の写真や画像に貼り込んだものである。ただし、タレント画像の首から下の部分にヌード絵など上書きで描かれたものもアイコラと呼ばれている。 アイコラには大きくわけて2種類ある。ひとつは有名な女性タレントの顔とヌード写真や水着写真の体を合成させたエロ画像系のもので、アイコラというと多くはこちらがイメージされる。もうひとつは有名人の顔を利用した
\\	アニメなどによる面白画像系である。 
\\	による画像加工が容易になったことで様々なアイコラが画像掲示板などで見られるようになった。しかし日本では肖像権やプライバシーに重きを置いているため違法という考え方が主流で、アイコラを載せた雑誌社の社長が名誉毀損で逮捕されるといった事例もある。ちなみにアメリカではアイコラは著作権の引用の範囲内であり、これを規制することは言論の自由を妨げるものとして合法との処置がとられている。	1998年	略語、
\\	用語	
\\	アク代官	あくだいかん	鍋料理の際、灰汁取りを行う人のこと。	アク代官とは時代劇の悪代官をもじった言葉であり、鍋奉行の対になる言葉(派生語?)で、鍋料理を食べる際に灰汁(アク)取りを行う人(=灰汁取り係)のこと。アク代官は
\\	サイト「鍋の館(下記参考サイト参照)」で提唱された言葉で、後に
\\	などのメディアでも取り上げられるようになり、浸透した。同サイトでは、アク取り係は美味しい鍋を作る上で重要な仕事であるにも関わらず、軽視されていると捉え、そこにスポットを当てる目的でアク代官と名付けた。また、元となる悪代官同様、鍋料理の影の支配者という意図を含むとしている。	1998年	
\\	おな中 ・ 同中 ・ オナ中	おなちゅう	同じ中学または同じ中学出身のこと。	おな中とは『同じ中学』を略したコギャル語で「同じ中学に通っている生徒」または「同じ中学の出身(卒業生)」という意味で使われる。 『チェベリバ』『チョベリグ』など様々なコギャル語が存在するが、中でもおな中は「オナニー中」との引っ掛け問題としてクイズ番組などに多用され、一般にも広く知られるようになった言葉である。 ※おな中は'80年代のツッパリブームから使われていますが、コギャル語として普及した1998年にメディアでも話題となり広く普及したため1998年の俗語としています。	1998年	若者言葉、コギャル語、略語	
\\	カーキカラー	かーきからー	
\\	産業に従事する人のこと。	カーキカラーとは襟
\\	の色から職種を表す言葉ブルーカラー、ホワイトカラーの派生語にあたり、
\\	(情報技術)産業に従事する人を意味する。ただし、
\\	産業に携わる人の襟がカーキ色というわけではなく、
\\	産業に従事する人の特徴であったカーキ色のチノパンからきたものである。カーキカラーはブルーカラーとホワイトカラーの中間に位置する新しい時代を支えるビジネス層として注目されたが、ブルーカラーやホワイトカラーのようには定着していない。	1998年	
\\	ガングロ	がんぐろ	顔を真っ黒にしている人のこと。	ガングロとは「ガンガン黒い」の略(顔が黒い→顔黒→ガングロという説もある)で、顔を真っ黒に日焼けさせること。またはそういった人のことである。日焼けサロンで焼いたり、黒いファンデーションを塗って黒くする(元の肌が地黒の人は対象としない)。1990年代後半から普及したガングロだが、2001年に入ると白塗り(美白ブーム)が好まれるようになり、当時のような流行スタイルではなくなった。ちなみに風貌の奇抜さからメディアが取り上げ話題になったが、東京(関東)以外で実際にガングロにした少女は少なかったとされている。また、ガングロより黒さの強いゴングロ、更に強いバチグロという言葉も使われた。	1998年	若者言葉	
\\	ギコる	ぎこる	挙動不審な行動をすること。ネット中毒になること。	
\\	ギコるとは「ぎこちなくなる」の略で、挙動不審な行いや怪しい言動をすることをいう。 
\\	ギコるとはインターネットスラング『ギコギコ(=電子掲示板に張り付く~入り浸ること)』の動詞形で、電子掲示板に長時間居つくこと。更にそれが転じ、ネット中毒(=インターネットに長時間接続し、現実生活に支障をきたすほど、チャットや掲示板、
\\	、オンラインゲームばかりすること)になることをいう。1990年代末に使われたが、この意味での使用は減っている。	1998年	ネットスラング	
\\	個族	こぞく	家族と離れ、または家族を持たず、一人で暮らすこと。	個族とは家族の対語として生まれた言葉で、家族と離れて個人(一人)で暮らす人や家庭を持たずに個人(一人)で過ごす人を意味する。人のプライベートを干渉しないといった意識の変化、コンビニや飲食業、コインランドリーなど利便性の向上が家族を持つことの必要性を下げ、個族が増えた要因とされる。また、同じ個族でも単に一人で暮らすというという者と、地域や親戚など身内のネットワークを完全に断ち切り、孤立する者がある。近年見られる後者タイプの増加は犯罪や高齢者の孤独死に繋がる恐れもあり、問題視されている。	1998年	
\\	小局	こつぼね	お局の配下的存在の人のこと。	職場を仕切る古参の女性社員を、江戸時代に大奥を仕切った女官に例え「お局」という。このお局の配下的存在にあるのが小局である。お局は自身が職場で活動しやすく、意見も通せるようにするため、後輩
\\	をランチや飲みに誘い、取り込んでいく。こうして何人かの小局を従えた後、上司を味方につけたり、中には自分が気に入らない人をイジメたりするという。ただし、小局は一般的にはあまり知られておらず、2007年現在ではほとんど死語となっている。	1998年	
\\	貞子	さだこ	暗い女性、気味の悪い女性、髪の長い女性、幽霊のこと。	貞子とは鈴木光司原作のホラー作品「リング」に登場する超能力者:山村貞子のこと。リングでは
\\	に映し出された井戸から白いワンピースに長い髪で顔を隠した姿で這い上がり、そのまま
\\	モニターから這い出し、白目を剥くシーンが有名だが、この特徴から暗い女性や気味の悪い女性、髪の長い女性の他、幽霊の代名詞としても使われた。ただし、リングが流行から外れるのに合わせ、こういった比喩としての使用もなくなっている。	1998年	若者言葉	
\\	式場離婚	しきじょうりこん	式および披露宴当日にキャンセルすること。	式場離婚とは予約している結婚式や披露宴の当日、新婦が現れずにキャンセルになることをいう。式や披露宴を済ませてから入籍することも多く、式直前にドタキャンをしたからといって、必ずしも離婚という言葉が適切とは言えないが、式場失恋ではなく、式場離婚で定着している。式場離婚に至る原因は親との同居や妻の仕事の継続といった結婚後の問題が未解決または曖昧な結論しか出していない状態で当日を迎えたことによるものが多い。こうした理由のため、女性がドタキャンしたものを式場離婚と呼ぶことになっているが、1996年の人気ドラマ『ロングバケーション』のように男性がドタキャンする場合も存在する。イレギュラーではあるが、これも式場離婚と呼ぶ。	1998年	
\\	自己中	じこちゅう	自己中心なこと。自己中心な人のこと。	自己中とは「自己中心(または自己中心的)」の略で、自分を中心(第一)に物事を考え、周囲(他人)に配慮した言動が出来ないこと。また、そういった人のことをいう。後ろにその対象となる代名詞を付け、自己中男 ・ 自己中女、自己中上司、自己中日記といった使い方もする。	1998年	
\\	しゃかパン	しゃかぱん	防寒用のナイロンパンツのこと。	しゃかパンとはスポーツをするときに履く防寒用のナイロンパンツやウィンドウブレイカーのパンツのことで、歩いたり走ったりしたときに「シャカシャカ」と音がすることから若者の間でしゃかパンと呼ばれるようになった。同義語に『シャカシャカ』があるが、こちらはパンツとは限定されていない。	1998年	若者言葉	
\\	シャッター通り	しゃったーどおり	衰退した通り(商店街)のこと。	シャッター通りとは閉店 ・ 撤退のため、シャッターを下ろしたままの店や事務所が多い通り(=衰退した通り)を意味する(商店街に限定した場合、シャッター商店街ともいう)。シャッター通りが出来る要因としては、炭坑街のように町の機能低下による人工そのものの減少の他、大型店の出店などによる客の流出が要因となることも多い(特に大規模小売店舗法の改正以降は、後者が要因のシャッター通り化した商店街が少なくない)。	1998年	
\\	縄文顔	じょうもんがお	彫りの深い顔立ちのこと。	縄文顔とは彫りの深い顔のことをいう。遺跡から発掘された人骨を元に縄文時代の顔を復元したところ、角ばった輪郭に彫りの深い顔立ちという特徴であることがわかった。また、眉や髭などの毛も濃かったとされている。これら縄文顔の特徴は1980年代に流行したソース顔の特徴に似ており、一時的にではあるがこうした彫りの深い顔を縄文顔と呼んだ。ちなみにソース顔の対であるしょうゆ顔は弥生顔の特徴に似ている。	1998年	若者言葉	
\\	そそうコール	そそうこーる	一気飲みをさせる掛け声のこと。	そそうコールとは飲み会で「そそう」というフレーズを使った一気飲みをさせるための掛け声のこと。同様の掛け声に「一気(イッキ)」があるが、これは状況に関係なく使われたのに対し、そそうコールは「そそう」、つまり何か失敗があったときに使う。例えば、自己紹介を間違えたり、披露したネタが失敗または面白くなかったときに使われる。単純に3拍子で「そ ・ そ ・ う」と繰り返すもの以外に
\\	(エス) ・ 
\\	(オー) ・ 
\\	(エス) ・ 
\\	(オー) ・ そそう」といった掛け声がある。	1998年	
\\	ダンサリアン	だんさりあん	段差のあるところに座り込む若者のこと。	ダンサリアンとは階段や駐車場の輪止め、道路の縁石など、段差のあるところに座り込む若者のことで、『ジベタリアン』と『段差』をかけた言葉である(ジベタリアンについては下記関連語参照)。ダンサリアンはジベタリアン同様、迷惑 ・ 非常識といった意を伴って使われる言葉であり、段差があるところに腰掛けていても場所によってはダンサリアンと呼ばない場合がある(公園内の座るために作られた段差など)。	1998年	
\\	ネカマ	ねかま	インターネット上で女性のふりをする男性のこと。	ネカマとは『ネットおかま』の略で、ネットワーク上のオカマという意味になる。具体的にはチャットや掲示板、
\\	などに女性のふりをして参加する男性だが、通常のオカマのように女性として生きたいといったポリシーや性同一性障害といった理由があるからではなく、単に男性をからかうためだったり、女性のふりをすることで女性のガードを緩め、接っしやすくすることを目的にネカマをしている者が多い。ネットおかま自体は1990年代始めのパソコン通信時代から存在するが、ネカマと略されたのは1990年代後半に入ってからである。	1998年	若者言葉 ・ 合成語、ネットスラング	
\\	バリ3	ばりさん	携帯電話のアンテナ受信状況がよいこと。	バリ3とはバリバリと3本から成る合成語で、バリバリは物事を勢いよく行うさま。3本は携帯電話の受信状況を表すアンテナの本数を意味する。受信状況が最もよいとき、画面上のアンテナは3本全て点灯状態になる。3本はこの状態を意味する。つまり、バリ3とはバリバリに3本立ってる=携帯のアンテナが余裕で3本点灯するほど受信状況がよいという意味になる。 また、小売店が30
\\	(3割引)することをバリ3ということがある。	1998年	
\\	バリ0	ばりぜろ	携帯電話の電波が圏外で通信不能な状態のこと。	バリ0とは『バリバリ』と『0本』から成る言葉で、バリバリは物事を勢いよく行うさま。0本は携帯電話の受信状況を示すアンテナの本数のことである。携帯電話のモニター内に記されたアンテナは、受信状況が良いときは3本全てが点灯し、悪いときは点灯本数が0本(圏外という文字が表示)になる。つまりバリ0とは携帯電話の電波受信状況が非常に悪く、通信出来ないことを意味する。	1998年	若者言葉	
\\	引きこもり	ひきこもり	自分の殻や空間に閉じこもることや、そういった人のこと。	"引きこもりに関しては下記の各機関にて定義が定められていますが、根本は社会と一線を引き、自分の殻(空間)に閉じこもること。またはそれをする人をさします。 【厚生労働省の定義】 さまざまな要因によって社会的な参加の場面がせばまり、就労や就学などの自宅以外での生活の場が長期にわたって失われている状態 【精神科医 斎藤環氏の定義】 
\\	代後半までに問題化し、 
\\	ヶ月以上、自宅にひきこもって社会参加しない状態が持続しており、ほかの精神障害がその第一の原因とは考えにくいもの 【朝日新聞社記者 塩倉裕氏の定義】 対人関係と社会的活動からの撤退が本人の意図を超えて長期間続いている状態であり、家族とのみ対人関係を保持している場合を含む
\\	これらの定義とは別に、""2ちゃんねる""を中心としたネットワークで使われる引きこもりは、単に外出せず、家でインターネットやゲームばかりしている非社交的な人全般をさすことが多い。
\\	1998年	
\\	ピタソ	ぴたそ	緩みのないピッタリした靴下のこと。	ピタソとは「ぴったりしたソックス」の略で、ピタ靴下ともいう。当時、ルーズソックスが完全に定着しており、逆にルーズソックスを履かないことに拘りを持つ若者も出てくるようになる。また同時期、体にぴったりした(多少サイズが小さめのものも多い)
\\	シャツをピタ
\\	(ピチ
\\	ともいう)と呼んでいたことから、こうした足にぴったりフィットする靴下のことをピタソと呼んだ。ただし、ピタソやピタ靴下という言葉は普及度は低い。	1998年	若者言葉	
\\	フケ専 ・ 老け専	ふけせん	
\\	老けた人を好む性嗜好のこと。 
\\	中年女性だけを扱う風俗店のこと。	
\\	フケ専とは「老けた人専門」の略で、老けた人ばかりを性的に好む嗜好、またはそういった人を意味する。フケ専はもともとゲイ(男性同性愛)の間で使われた隠語だったが、1990年代後半から一時的に老けた異性を好む人という意で男性同性愛者以外にも普及。ただし、老けた男性を好む女性が追加されただけで、老けの対象が女性にまでは広がっていない。 
\\	フケ専とは老けた人を専門に扱う店ということで、中年女性ばかりを雇った風俗店を意味する。ただし、こちらの意では熟女専門(熟専?)と呼ぶのが一般的である。	1998年	若者言葉	
\\	凹む	へこむ	落ち込んだり、気が滅入ったりすること。	へこむとは外圧によって物の表面がくぼむことをいうが、ここから(外的要因によって)落ち込んだり、気が滅入るといった心的変化のことをへこむという。また、落ち込むは長期に渡るような致命的問題に対して使い、へこむは一時的な軽い落ち込みに対して使うといった傾向がある。 ※へこむを上記のような意味で使った文献が江戸時代にありますが、1990年代後半の若者が当時の文献を参考にしたとは考え難く、独自で使い始めたという見解から、当辞書では1998年の俗語に分類しています。	1998年	若者言葉	
\\	ボキャ貧	ぼきゃひん	語彙の少ないさま。またはそういった人のこと。	ボキャ貧とは「ボキャブラリーが貧困」の略で、語彙(言葉)が少ないさま及び、そういった人を意味する(当時流行っていたフジ
\\	系の番組「タモリのボキャブラ天国」がボキャ天と略して呼ばれたことに起因したものと思われる)。当時、小渕首相は前任の橋本首相に比べ、弁が立たず、マスコミの交わし方も劣ると言われた。これを受け、小渕首相自らがボキャ貧と自嘲し、広く普及した。	1998年	
\\	マヨラー	まよらー	マヨネーズが好きな人のこと。	マヨラーとはマヨネーズに人化する接尾語「ラー」をつけたもので、マヨネーズが好きな人のことである(1990年代後半、「ラー」はアムラーなど様々な語句に使われる:下記関連語参照)。狭義にはマヨネーズをご飯にかけて食べたり、そのままチューブから舐める(この行為をマヨチュチュともいう)といった一般的でないマヨネーズの使い方をする人のみをマヨラーと呼ぶ場合がある。また、マヨラーの間ではメーカーごとで
\\	派、エコナ派など派閥がわかれている。	1998年	若者言葉	
\\	メル友	めるとも	メール友達の略。	メル友とはメール友達の略で、掲示板やチャットで知り合い、
\\	メールや携帯メールが交流手段の主となっている友達のことである。インターネットや携帯サイトでの出会いの場合、相手がどういった人物であるかわからないこともあり、いきなり会ったり、電話番号を教えるのは不安なため、とりあえずメル友から始め、親交を深めていくことが多い。こうして親交を深めたメル友と結婚をするカップルも少なくない。またメル友を募集したり、メル友を探すことを目的としたウェブサイト ・ 携帯サイトも多数存在するが、こういった場所で知り合った人とのトラブル、事件なども増加し、問題となっている。メル友は
\\	98によるパソコンの普及でインターネット人口が増えた1998年に使われ始め一気に浸透。携帯メールの普及も後押しし、現在でも使われている言葉である。	1998年	略語 ・ 若者言葉、ネットスラング	
\\	もさい、もっさい	もさい、もっさい	格好悪い ・ 垢抜けしないこと。	もさいとはもさっとしたさま(=気がきかないさま、かっこ悪いさま、田舎くさいさま、垢抜けしないさま)をあらわす言葉で、もともと愛知県の尾張地方で使われた方言であった。もさいの他、もっさいとも言う。	1998年	
\\	モバリーマン	もばりーまん	モバイル機器を使っていつも仕事をしている人のこと。	モバリーマンとは携帯端末機器を意味するモバイル
\\	英語では「可動性の~」「移動可能な~」)とサラリーマンの合成語である。モバリーマンは博報堂による造語で、この言葉が出来た1990年代末にはモバイル機器を使ってネットに接続し、移動時間や食事中など、いつでもどこでも仕事をするサラリーマンを意味した。後に、モバイルを使っていつでもどこでも株取引をする人やプライベートな情報収集 ・ サイト閲覧をする人など、仕事か否かに関係なく、モバイル機器を使いこなすサラリーマン全般を指すようになる。	1998年	合成語	
\\	弥生顔	やよいがお	あっさりとした顔立ちのこと。	弥生顔とはあっさりとした顔のことをいう。遺跡から発掘された人骨を元に弥生時代の顔を復元したところ、丸型の輪郭に全体が曲線でなめらか、平坦な作りが特徴であることがわかった。また、眉や髭などの毛も薄く細かったとされている。これら弥生顔の特徴は1980年代に流行したしょうゆ顔の特徴に似ており、一時的にではあるがこうしたあっさりとした顔立ちを弥生顔と呼んだ。ちなみにしょうゆ顔の対であるソース顔は縄文顔の特徴に似ている。	1998年	若者言葉	
\\	ランコン	らんこん	ランチタイムに行う合同コンパのこと。	ランコンとはランチと合コン(または合同コンパ)の合成語で、ランチタイムを利用して行うコンパのこと。ただし、内容はさまざま。時間帯が昼間というだけで、アルコールも出るものから、近隣オフィスに勤める男女が昼食を取りながら知り合うための会、また、目的が男女の出会いではなく、異業種交流といったものもある。ただし、ランコンといった場合、男女が知り合うことを目的とした食事会及び飲み会を指すことが多い。合ランともいう。	1998年	合成語	
\\	路チュウ ・ 露チュウ	ろちゅう	路上でキスをすること。	路チュウとは「路上でチュウ」を略したもの(路上駐車を意味する同音異義語『路駐』のもじりでもある)で、路上でするキスを意味する。道路に限らず、駐車場や公園、駅など人の往来がある場所でのキスを含むが、この場合、路チュウ以外に「露出」からとった露チュウという表記が使われることもある。1990年代、人前で平然とキスをする若者が話題となり、路チュウという言葉も普及。しかし、路チュウが当たり前になるとともに死語となりつつある。	1998年	若者言葉	
\\	割り込み婚	わりこみこん	20歳になったばかりの女性や30代後半以上の女性が適齢期の女性に割り込むようにして30代前半の男性と結婚すること。	割り込み婚とは20歳になったばかりの若い女性や30代後半以上の女性(主にバツイチなど離婚経験者を指す)が20代後半から30代前半の結婚を考えるようになった女性に割り込むように、この年代に相応しい30歳から30代半ばの男性と結婚してしまうことをいう。ただし、
\\	婚の中ではあまり浸透していない言葉で、現在、ほとんど使われない死語となっている。	1998年	
\\	ありござ	ありござ	「ありがとうございます」の略。	ありござとは「ありがとうございます」の略で、親しい仲間内で使われる若者言葉である。ありござは1990年代後期から使われるようになった。 パソコンと違い、携帯電話でメールの文字を打つのは時間がかかるため、簡略出来る言葉は極力簡略化されている。そういった中で生まれたありござのような言葉は若者を中心に会話の中でも使われるようになる。	1997年	略語	
\\	する	えふえふする	
\\	スキャンダル写真を撮ること。 
\\	ファイナルファンタジーをすること。	
\\	するの
\\	とは2大写真週刊誌「フライデー」と「フォーカス」の頭文字である。つまり、
\\	するとはこういった写真週刊誌に掲載されるようなスキャンダル写真を撮ったり、スキャンダル写真を雑誌に投稿することを意味する。この意味では1980年代中頃に使われた。 
\\	するの
\\	とはファイナルファンタジー
\\	の頭文字である。つまり、
\\	するとはファイナルファンタジーをすることを意味する。シリーズ中、特に人気のあったファイナルファンタジー
\\	が発売された1997年、この意味で使われるようになった。(現在、大半がこちらの意で用いられている)	1997年	
\\	ゲードル	げーどる	アイドル的活動をするゲームキャラクターのこと。	ゲードルとは「ゲーム」と「アイドル」からなる合成語で、アイドルのような活動をするゲームキャラクターを意味する。写真集や
\\	、映画に出演するゲードルもあり、その活動範囲はアイドル顔負けといったものもある。こうしたゲードルは
\\	の進化および、2次元キャラクターを愛するオタク需要の増加から今なお新たなゲードルが生まれている(2007年現在)。 これとは別に、ゲーム好き(ゲームオタク)を売りにし、なおかつゲームが得意なアイドルのこともゲードルという。	1997年	若者言葉 ・ 合成語	
\\	ゲットする	げっとする	欲しいものを手に入れる(又は手に入れた)こと。	ゲットするとは「得る ・ 取る ・ とらえる」という意味の英語“
\\	に接尾語『する』をつけたもので、欲しいものを手に入れることである。また、好きな異性の心をつかむことやナンパで異性を口説き落とすという意味でもゲットするは使われる。ゲットするという言い回しは1990年代に入った辺りから使われているが、人気アニメ『ポケットモンスター』の主人公サトシが使ったことにより、同アニメがスタートした1997年、特に広く普及した。	1997年	若者言葉	
\\	シノラー	しのらー	篠原ともえのファンや篠原ともえみたいな人のこと。	シノラーとはタレント“篠原ともえ”の篠原に英語で人化する接尾語
\\	をつけたものだが、少し前にアムラーが流行語となっており、シノラーはその篠原ともえ版である。広義には篠原ともえのファンをさすが、当時の篠原ともえは前髪を横一直線に切り、安っぽいおもちゃの腕輪や指輪をジャラジャラつけ、服も幼い感じで、ロリコンというより子供っぽい雰囲気。更に「~ですぅ」「キャッキャッ」といった幼稚な言葉使いや仕草が特徴だった。ここからシノラーも単にファンというだけでなく、そういった篠原ともえの「シノラー語 ・ メイク ・ ファッション」を取り入れた女性を指す。ただし、篠原ともえの低迷とともにシノラーも死語となる。	1997年	若者言葉	
\\	ジベタリアン	じべたりあん	ところ構わず地面に腰をついて座る若者のこと。	ジベタリアンとはコンビニの駐車場や繁華街、ところ構わず腰をついて座り込む若者のこと。ジベタリアンは菜食主義を意味する『ベジタリアン』と地面を意味する俗語『地べた』をかけたものである。当初、ジベタリアンは道徳性の問題として話題になる程度だったが、次第に電車の中など、いたる所で見られるようになり、社会問題にもなる。ジベタリアンが増えた原因として、洋式トイレの普及+正座をする機会が減ったことの他、食生活の変化や単に体力が落ちたためとも言われている。	1997年	
\\	ストーカー妻	すとーかーつま	夫の浮気を疑ってつけまわす妻のこと。	ストーカー妻とはストーカーをする妻。ストーカーの定義をあてはめると、特定の人物に対し、執拗につきまとう妻ということになる。ただし、ストーカー妻が対象とする特定人物とは夫。また、多くのストーカーが異性に対する恋愛感情 ・ 好意が原動力となっているが、この場合は夫に対する浮気 ・ 不倫という疑惑が要因となっており、ストーカーというより、警察や探偵の尾行に似た行為である。サングラスに帽子やカツラで変装し、職場まで尾行したり、休日もつけまわす主婦もあるという。	1997年	
\\	象足	ぞうあし	象のように太い足のこと。	象足とは象のように足首から太股まで全体に太い足のことで、主に女性を対象に使われる。象足は1980年代頃から若者を中心に使われていたが、ルーズソックスがブームとなった1990年代半ば、実際に足首が太いか否かは別にしてルーズソックスを履いた足が足首まで同じ太さに見えたことから象足と呼ばれ、特に普及した。	1997年	若者言葉	
\\	卒サラ	そつさら	定年退職または独立起業などで会社勤めを辞めること。	卒サラとは卒業とサラリーマンの合成語だが、「脱サラ」のもじりとして出来た言葉とも考えられる。卒サラはもともと定年退職の「会社に行けなくなり、やることがなくなる」といった暗いイメージを払拭し、卒業という言葉を使うことで「新たなスタート」というイメージを与える言葉であった。ただし近年、独立起業したり、株で生計を立てるようになって会社を辞めるなど、脱サラと同意に使われることが増えている。	1997年	
\\	チャイドル	ちゃいどる	チャイルド ・ アイドルの略。	チャイドルとはチャイルド ・ アイドルの略で、アイドル的な活動をする子供やアイドルのように人気のある子供を意味する。チャイドルの年齢は使う人の主観によっても異なるが、多くは労働基準法で規制のかかる中学生までを対象に使われる。チャイドルはこうした規制からドラマや映画、舞台しか活動の場がなかったが、インターネットの普及によるアピール場所の増加で多くのチャイドルが登場している。 また、「我が子=我が家のアイドル」であることから「我が家のチャイドルは今日も元気いっぱい」といった使い方もされる。	1997年	略語	
\\	デブ専	でぶせん	太った人を好む人。太った人のみを雇った風俗店のこと	デブ専とはデブ専門の略で、太った人を性的に好む人や太った人のみを専門に雇う風俗店を指す。更に太った人を専門に扱っているという意味でデブ専サイト、
\\	のデブ専コーナーといったものもある。ブス専を
\\	専と言うように、デブ専も
\\	専と言うが、
\\	専の場合はそれだけでなく、野球のドラゴンズ専門の略として使われていることもあるので注意。 また、デブ専とは別に、ぽっちゃり専門の略でぽちゃ専というものもある。ただし、デブとぽっちゃりの境界線が人によって曖昧なように、デブ専とぽちゃ専の境界も特に定義されていない。サイトや店舗では併記されていることも多い。	1997年	
\\	ドッグイヤー 
\\	どっぐいやー	
\\	業界の技術革新スピードを表す例え。	"ドッグイヤーとは英語の
\\	""、直訳すると「犬の年」だが、
\\	業界は技術革新などの変化が早いことを意味する例えとして使われる。犬と人間の寿命から換算すると、1年間に犬は人間の7年分の出来事が起こる計算になる。このことから情報技術分野
\\	分野)の1年(ドッグイヤー)は他分野の7年に相当すると例えたものである。つまり、
\\	業界は通常よりも7倍のスピードで変化しているということを意味する。
\\	1997年	
\\	用語 ・ カタカナ英語	
\\	トレカ	とれか	トレーディングカードの略。	トレカとはトレーディングカードの略である。トレーディングカードとは文字通りトレード(交換)を意図したカードのことである。トレカの歴史は古く、日本でも1950年代には登場している。当初は野球選手物が多かった。後にエヴァンゲリオンのシリーズ(1996年発売開始)がヒット。トレカブームのキッカケとなる。以降、様々なジャンルのトレカが発売され、2007年現在もムシキングなど人気トレカが存在する。トレカの大きさは一般的にテレホンカードやキャッシュカードサイズで、菓子の付録として付いているものやカードのみを不透明な袋に入れて販売されるものもある。どちらも中身(カードの種類)が見えない構造になっており、原則として交換によって揃えられるようになっている。	1997年	
\\	弾ける	はじける	踊り狂うこと。はめをはずすこと。思い切ったことをする。	はじけるには本来、「裂けて開く」「成熟して割れる」といった意味がある(広辞苑第五版参照)。ここから、まず昭和のディスコ~クラブ ・ ブームの中で、我を忘れ、狂ったように踊りまくることをいうようになる(使用例
\\	。更に平成に入ると踊る以外の行為に対しても使われるようになり、世間体や常識を忘れ、羽目を外すという意味で使われるようになる。(使用例
\\	1997年	若者言葉	
\\	パチモン	ぱちもん	偽物のこと。	パチモンとは「パチモノ」が音的に崩れた言葉で、パチモノ同様に偽物を意味する。主にバッグや財布、時計など高級ブランド品の偽物をさすが、他にもオマケ付き菓子、トレカ、ゲーム機やゲームソフトなど様々な偽物商品をパチモンという。パチモンは正規品よりも安価だが粗悪品が多い。しかし、その安さからパチモンとわかって購入する人、また安価という理由でなく、パチモンをひとつのジャンルやブランドとして収集する人もいる(コレクションとしてはキャラクターグッズのパチモンに多い)。	1997年	
\\	鼻ピ	はなぴ	鼻ピアスの略。	鼻ピとは鼻ピアスの略で、文字通り鼻に付けるピアスを意味する。鼻の出入り口付近の左右の壁および、左右の穴を隔てる中央の鼻柱隔に穴を開けて付ける。鼻ピにはスタッドタイプ(まっすぐなシャフトが特徴。先端の突起で外れなくする)、スクリュータイプ(シャフトの先端をスクリュー状にすることで外れなくするタイプ)、セブダム(鼻の中央の鼻柱隔に付けるタイプ)がある。日本ではファッションのひとつとして付けるだけだが、インドで花嫁の証しとして付けるように、鼻ピをすることに意味のある国 ・ 集落もある。	1997年	若者言葉	
\\	ボコる	ぼこる	袋叩きにすること。	ボコるとはひどく殴りつけるさまや大勢で袋叩きにするさまを表す『ボコボコ』を略し、俗語によくある動詞化する接尾語『る』をつけたもので、「ボコボコにする」の略と考えてよい(ただし、ボコボコの場合は「ボコボコに言う」など暴力行為以外にも使われるがボコるの場合、袋叩きにする ・ ひどく殴りつけるといった暴力行為 ・ 破壊行為のみが対象である)。ボコるという言い回しは1990年代後半から暴走族など不良を中心に普及した。	1997年	若者言葉	
\\	マイブーム	まいぶーむ	個人的に夢中になっていること(個人的な流行)。	"マイブームとは漫画家
\\	みうらじゅんが作った造語で、彼が薦めた「世の中の流行に流されず、自分だけの流行(ブーム)を持とう」という生き方のキャッチフレーズ的なものとして作られた和製英語である。特にマイブームに関する定義付けはされておらず、仮に流行のドラマや格好であったとしても、それが流行だから見たり着たりしているのでなく、 自分の興味心が元で夢中になっているのであればマイブームとなる。 本来、
\\	""とは「物価の急騰」「にわか景気」という意味が転じ、「爆発的に多くの人から支持をされる事 ・ 物」といった意味で使われる。これに個人レベルである
\\	""をつけることは矛盾しているが、この矛盾と個性が重要視された時代背景から若者を中心にうけ、浸透していった。また、マイブームは自由国民社による新語 ・ 流行語大賞で1997年のトップ10入賞にもなっている。
\\	1997年	和製英語 ・ 若者言葉	
\\	アウトオブ眼中	あうとおぶがんちゅう	「眼中にない」「論外」を意味する若者言葉。	アウトオブ眼中は「~から外れて」「~を離れて」といった意味をもつ英語の
\\	と『眼中』を組み合わせた造語で、「眼中にない」「問題外」「論外」といった意味で使われる。様々な場面で使われる言葉ではあるが、興味のない異性を対象に使われることが多い。 アウトオブ眼中という言い方は対象を馬鹿にした意が含まれるので使用に際しては注意が必要である。 アウトオブ眼中は1980年代に大学生を中心とした若者が使い始め、1990年代後半に最も普及した言葉である。	1996年	若者言葉	
\\	アムラー	あむらー	安室奈美恵のファッションを真似た女性のこと。	アムラーとは歌手:安室奈美恵(あむろなみえ)の「アムロ」に英語で人化する接尾語“-
\\	をつけたもので、安室奈美恵のファッションを真似た女性を意味する。1990年代、シャネラーなどブランド名を人化する言葉が流行する中で、当時、安室奈美恵がファッション ・ リーダー的存在だったことから生まれた言葉である。具体的にはロングヘアに細い眉、ミニスカートに厚底ブーツといったスタイルの女性がアムラーと呼ばれた。安室奈美恵ブームが去るとともにアムラーも死語となった。	1996年	若者言葉	
\\	アムる	あむる	安室奈美恵のファッションを真似すること。アムウェイ商品を売ること。	アムるとは歌手:安室奈美恵のアムロの「ろ」を俗語でよくある名詞を動詞化する接尾語「る」に変えたもの(俗語『アムラー』の動詞形と考えてよい)で、ロングヘアに細い眉、ミニスカートに厚底ブーツといった安室奈美恵のファッションを真似することである。アムラー同様、安室奈美恵がファッションリーダー的存在でなくなるのと並行して死語となっている。 これとは別に日用品を扱うアメリカ企業「アムウェイ
\\	の商品を個人販売員となって売ったり、アムウェイのビジネス勧誘をするディストリビューターになることもアムるいう。こちらの意味では他者が揶揄嘲弄する際に使う。	1996年	若者言葉	
\\	アララー	あららー	アムラーになりきれていない女性のこと。	アララーとは歌手:安室奈美恵(あむろなみえ)のファッションを真似ているが、それになりきれていなかったり、似合っていない女性を意味する。アララーは安室奈美恵を真似る女性を意味する『アムラー』と、人が何か失敗した際にかける感嘆詞「あらら」をかけたものである。	1996年	若者言葉	
\\	ウーロン茶	うーろんちゃ	長髪で茶髪の男性を卑しんでいう言葉。	ウーロン茶とは「うっとうしい」を意味する俗語『うざったい』、長髪を意味する俗語『ロン毛』、これらと『茶髪』から成る合成語で、長髪で茶髪の男性を若い女性が嘲ったり、卑しみを込めて使った言葉である。大人からの否定的な声は以前からあったヘアスタイルのウーロン茶。若者からもそういった髪型に対して否定的言葉が生まれたとして1996年に話題となった。	1996年	美容 ・ ファッション用語、コギャル語 ・ ギャル関連	
\\	オール	おーる	
\\	・・・”という英熟語の略として様々な意味で使われる語。	"オールとは
\\	""という英熟語の略として様々な意味で使われるが、最も頻度が高いのは
\\	""の略で「一晩中 ・ 夜通し」という意味になる。若者が「オールで飲み」や「オールでダンス」という場合、大抵はこの意味で使われる。日程的なものでは
\\	""の略で「終日」の他、会話の中で特定の期間が読み取れる場合(夏休みや
\\	、出張期間など)、その期間全てを指してオールという場合もある。 日程以外には
\\	""や
\\	""の略として「(その場にいる人や会話内で特定した人の)全員」という意味でも使われる。
\\	1996年	カタカナ英語+略語	
\\	おやじギャグ	おやじぎゃぐ	古臭いギャグのこと。	おやじギャグとは古臭いギャグや面白くない(うけない)ギャグのことで、中年(40歳前後)以上が好んで使う駄洒落(「電話に出んわ」、「トイレに行っといれ」など)を指すことが多い。なお、おやじギャグという呼称はこういった駄洒落など、古臭いギャグを嘲うものであり、「つまらない」というニュアンスを多く含んだ嫌味言葉である。	1996年	
\\	オンザ眉毛	おんざまゆげ	前髪を眉毛の上で切り揃えた髪形のこと。	"オンザ眉毛とは眉毛の種類ではなく、前髪に関する言葉である。オンザというのは英語の
\\	""。つまり、オンザ眉毛
\\	眉毛)とは前髪を眉毛の上でバッサリと切り揃える髪型のことである。オンザ眉毛という言葉自体は1990年代前半からあるが、1990年代後半、特異なキャラクターやファッションで人気を博した篠原ともえが好んでしたことから広く普及した。オンザ眉やオンザともいったが、現在ではあまり使われない死語となっている。
\\	1996年	美容 ・ ファッション用語	
\\	ガンベル	がんべる	ポケットベルを激しく鳴らすこと。	ガンベルとは「凄く~」「超~」「とてもよく~」といった頻度 ・ 度合いを強調する言葉「ガンガン」と、数字メッセージを受け取る小型受信機「ポケットベル(通称ポケベル)」の合成語で、何度も頻繁にポケットベルを鳴らす(=ポケットベルに頻繁にメッセージを送ってくる)ことを意味する。現在はツールがポケットベルから携帯メールに変わっており、同時にガンベルも死語となっている。	1996年	若者言葉	
\\	ギムエン族	ぎむえんぞく	セックスレス及びセックスレス寸前の夫婦。	ギムエン族とは内館牧子の著書で、1996年には浅野ゆう子、黒木瞳、大浦龍宇一といったキャスティングによりドラマ化された『義務と演技(ギムとエンギ)』というタイトルからきたものである。このストーリーには、仕事で疲れている妻を喜ばせようと義務的に体を求める夫とそれに演技で応える妻という夫婦。逆に妻は抱いてもらおうと求めるが仕事疲れを理由に夫が拒み、セックスレスになっている夫婦といった2カップルが出てくる。ここから夫婦間でセックスレスになっていたり、滅多に性交のないセックスレス寸前の状態をギムエン。また、そういった状態にある夫婦をギムエン族という。	1996年	
\\	逆ナン	ぎゃくなん	女性から声をかけてナンパすること。	逆ナンのナンとは男性が女性に声をかけて誘う「ナンパ」の略である。逆ナンとはナンパの逆、つまり女性から男性に声をかけて誘うことをいう。昭和末期にはこういった積極的女性が増え、話題となった。	1996年	
\\	コクる ・ 告る	こくる	告白すること。	告るとは「告白する」を略したもので、意味も同じである。告るも「告白する」と同様に好きな人に恋心を伝えることや、人に隠していたこと打ち明けることをさすが、告るの場合、主に前者の愛を告げるという意で使われることが多い。「メールで告る」「今夜、アイツに告る」といった使い方をする。	1996年	若者言葉	
\\	コマダム ・ 小マダム	こまだむ	若くしてブランド物に身を包む既婚女性のこと。	コマダムとは奥様といった意味の英語“
\\	(フランス語で既婚女性の名に冠する
\\	から)”に小さい(若い)を付けたもので、若くして豪華 ・ 高価なブランド物に身を包んだ既婚女性を意味する。小マダムとも書く(大抵はカタカナ表記を使用)。夫がエリートでお金のあるコマダムは、結婚しても独身時代のようにブランド品を買ったり、スタイルに気を配ったり、高級レストランで食事をしたりする。また、そういった家庭に育つ子供は幼い頃からファッショナブルに装わされ、マゴマダムと呼ばれる。なお、コマダムの年齢に関しては20代前半とするものや30代までとするものなど、定義は様々である。	1996年	
\\	コムロする	こむろする	徹夜すること。	コムロするとはミュージシャン、プロデューサーといった音楽活動をしている小室哲哉の苗字に「~する」という動詞化する言葉をつけたもので、小室哲哉の哲哉(てつや)にちなんで「徹夜する」という意味で使われる。コムロするは1990年代中頃、コギャルの間で普及したが、現在は世代に関係なく、ほとんど使われない死語となっている。また、コムロするが使われる前は同様の理由から『タケダする』という言葉が使われた。	1996年	若者言葉	
\\	コヤジ	こやじ	
\\	実年齢は若いがオヤジっぽい人のこと。 
\\	お洒落な中年男性のこと。	
\\	コヤジとはコギャルに使われる『コ(子)』と中年男性を意味する『オヤジ』から成る言葉で、年齢は若いがオヤジ(中年)臭さのある者やオヤジ予備軍的な者をさす。1990年代半ば、コギャルが好んで使ったであることから、20代でも人によってはコヤジ扱いされ、30代までが対象範囲となる。 
\\	コヤジとは「コジャレたオヤジ」の略で、お洒落な40~50代男性を意味する。この意味では2005年に消費ターゲットとして一時的に話題となる。ただし、コヤジからくるイメージ ・ 語感のせいか、あまり浸透はしていない。	1996年	
\\	地味婚	じみこん	挙式や披露宴が質素な結婚のこと。	地味婚とは挙式や披露宴に費用をかけない結婚のことである。ただし、同じ地味婚でも挙式や披露宴を全く行わないものから、単に招待数を抑えただけで、式の内容自体は豪華といったものまで様々である。 地味婚を行う理由は、単に挙式や披露宴にかかる費用が出せない場合もあるが、新婚旅行やその後の生活に費用を回した方がよいという合理的な考え方や形式だけの挙式~披露宴をしたくないといった理由が多い。	1996年	
\\	地雷	じらい	
\\	禁句など触れてはいけないこと。 
\\	利用してみて始めて問題があることに気付くもの。	地雷とは地中に設置し、人や車両が接触(踏む)ことで爆発する武器のことだが、この接触すると爆発するという特徴から、以下のような意味で使われる。 
\\	地雷とは触れると爆発するという特徴から、場の空気が悪くなるため、皆が避けている話題(禁句)のことをいう。例えば、人が気にしている身体面 ・ 能力面の短所の話題、誰かが悩んだり、落ち込んだりしている要因となる事象(失恋 ・ 失業 ・ 落第など)に関連した話題が地雷にあたる。場の空気を悪くしたり、誰かを怒らせてしまう地雷に周りは触れないよう気遣うが、後から来て何も知らず、その話題に触れてしまうことがある。これを「地雷を踏む」という。 
\\	地雷は踏んで爆発し、初めてそこが危険と気付くことから、購入~開封してから気付く粗悪品、入店してから容姿の酷い女性しかいない店と気付く風俗店、ぼったくりバーなどは地雷の典型的である。インターネットではページを開いたり、ダウンロードして初めて気付くウィルス付ファイルなどを地雷と呼ぶ。	1996年	
\\	タレ弁	たれべん	タレントのような活動をする弁護士。	タレ弁とはタレント弁護士の略で、単に弁護士としてテレビに出るといった域を超え、タレント的活動する弁護士のことである。タレ弁という言葉自体は1990年代半ばから使われているが、2002年春から日本テレビ系列で放送されている島田紳助司会『行列のできる法律相談所』に出ている弁護士陣に対して使うことが多く、中でも
\\	発売までした丸山和也氏や多くのレギュラー番組を持つ茶髪の風雲児こと橋下徹氏を指してタレ弁ということが多い。ちなみに、本業(弁護士業務)に対するメディア活動の割合やどういった活動内容がタレ弁に値するのかといった定義は特になされていない。	1996年	略語	
\\	チョバチョブ	ちょばちょぶ	最悪なことがあり、とても憂鬱なさまを意味する感嘆詞。	チョバチョブとは「超 
\\	超 
\\	の略で、チョベリバとチョベリブの合成語にあたり、「最悪で悲惨な状況」「最悪な事態にとても憂鬱」といった意味になる。チョバチョブは他のチョベリバなどと同様、1990年代後半にコギャルの間で使われたコギャル語で、現在はあまり使われることのない死語となっている。なお、チョベリブよりも悲惨な状況のときに使われる語で、チョバチョブ
\\	チョベリブという関係になる。ただし、普及度の点では当時からチョベリブより低い。	1996年	若者言葉	
\\	チョベリグ	ちょべりぐ	最高の意の感嘆詞。	チョベリグとは「超 
\\	の略で、「最高(サイコー)」という意味で使われる感嘆詞である。「超」と「ベリー」の両方が付く言葉だが、よほど良いことがあったときのみ使う言葉かというと、そういうわけではない。1996年には対語『チョベリバ』とセットで流行語大賞トップテンに入賞している。なお、最初にチョベリグを言い出した人物がわからないため受賞者はなく、該当者が名乗り出るのを待ったままの状態になっている。また、チョベリグは現在、ほとんど使われない死語になっている。	1996年	若者言葉	
\\	チョベリバ	ちょべりば	最悪 ・ 最低を意味する感嘆詞。	チョベリバとは「超 
\\	の略で、「最悪(サイアクー)」「最低(サイテー)」という意味で使われる感嘆詞である。もともとコギャルの間で使われていたチョベリバだが、1996年に木村拓哉と山口智子主演の人気ドラマ『ロングバケーション』で使われたことから広く普及。対語である『チョベリグ』とともに同年の流行語大賞トップテンに入賞している(なお、最初に言い出した人物がわからないため受賞者はなく、該当者が名乗り出るのを待ったままの状態になっている)。また、コギャルを扱う番組やクイズ番組などでも使われ、コギャル語の代表的言葉のひとつになっている。	1996年	若者言葉	
\\	チョベリブ	ちょべりぶ	とても憂鬱なさまを意味する感嘆詞。	チョベリブとは「超 
\\	の略である。ここでいう“
\\	とは青色でなく、マリッジ ・ ブルーなどに見られる憂鬱な気分を意味するブルーであり、チョベリブは「とても憂鬱」という意味になる。チョベリブは同時期に流行したチョベリバ、チョベリグ同様、コギャルの間で使われたコギャル語で、他の語句同様、現在は死語となっている。また、類語にチョビチョブという言葉があるが、チョベリブのほうが意味合いが弱く、チョビチョブ
\\	チョベリバという関係になる。	1996年	若者言葉	
\\	チラ見	ちらみ	瞬間的に見ること。	チラ見とは「チラッと見る」及び「チラチラ見る」が略されたもので、瞬間的に見ること、わずかな間隔で視線を送ったり外したりする見方をいう。チラ見は大抵、見てはいけないもの、見ると失礼にあたるものだが、興味 ・ 関心から見たいという欲求がある際にされる(代表的なものは女性の胸の谷間やミニスカートなど露出された部分を男性が見る場合)。ただし、チラ見という言葉が普及した1990年代中頃には女性が好みの男性にチラッと見て、視線を送ることをいった。これは好みの男性に寄ってこさせたり、声をかけさせるために行うもの。つまり当事のチラ見は逆ナンパのひとつの方法を意味していた。	1996年	若者言葉	
\\	チン妻	ちんさい	インスタント食品で料理を済ませる妻のこと。	チン妻のチンとは電子レンジで調理完了を知らせる音で、電子レンジで調理することを意味する「チンする」からきている。つまり、チン妻とはチンする妻=冷凍食品や惣菜など、電子レンジで簡単に済ませられる料理ばかりを食卓に出す妻をいう。 これとは別に楽チンのチンからきた「何でも気付いてしてくれ、一緒にいると楽な妻」や、逆に「旦那が何でもしてくれ、楽をしている妻」という意味で使われている場合もある。(基本的には先述の意味で用いられる)	1996年	
\\	中抜け	なかぬけ	学校や会社を途中で抜け出すこと。	中抜けとはもともとコギャル語で学校を途中で抜け出して遊びに行くことを意味した。中抜けという語感から途中から間だけ抜け、最後は戻ってくるイメージを持つ人もあるが、戻るか否かは特に関係ない。2007年現在、若者の間では死語となりつつあるが、2007年4月、兵庫県宝塚市職員(クリーンセンター職員)が勤務中に中抜けした問題が法廷に持ち込まれ、話題となった。 これとは別にオートフォーカス(以下
\\	のカメラで被写体が中心以外にある構図を撮影する際、
\\	が中心にある背景にピントを合わせ、肝心な被写体がピンボケしてしまうことがある。これを中抜けという。	1996年	
\\	ノーブラ	のーぶら	ブラジャーをつけていないこと及び、ノーブランドのこと。	
\\	ノーブラとは「ノー ・ ブラジャー」の略で、ブラジャーをつけていないことを意味する。ただし、ノーブラは卑猥 ・ 下品な意味合いを伴い、もともとブラジャーを着用しない年齢である幼女 ・ 少女は対象としない。この意味では1980年代から使われており、現代でもノーブラの大半がこちらの意味で使われる。 
\\	ノーブラとはノーブランドの略で、(有名)メーカー製品でないことを意味する。ただし、この意味で使われるのは秋葉原の一部(主に
\\	パーツを扱うショップ)で、アキバでも店によっては
\\	の意味と誤解される場合があるので注意。また、ノーブラといっても大手メーカーのパーツを下請けの製造している場合もあり、ノーブラ=粗悪品という意味で使われているわけではない。	1996年	
\\	ハデ婚 ・ 派手婚	はでこん	お金をかけた派手な挙式や披露宴のこと。	派手婚とは「派手な結婚式」の略で、お金をかけた派手な演出や衣装、料理を振舞う挙式 ・ 披露宴を意味する。1990年に入り、有名芸能人が高額で派手な演出の披露宴を行い、その模様が特別番組になるほど話題となった。こうした結婚式をメディアが派手婚と報道。一般の結婚式でもお金をかけた派手なものに対して派手婚と呼ぶようになる。	1996年	
\\	はぶる	はぶる	仲間はずれにすること。	はぶるとは仲間はずれにするという意味の俗語『はぶにする』が略されたもので、『はぶにする』同様、学生の間で使われた言葉である。 これとは別に一部では「はぶりがいいことをする」という略から気前良く奢るという意味ではぶるを使う場合がある。	1996年	若者言葉	
\\	ピッチ	ぴっち	
\\	という携帯電話の一種の俗称。	"ピッチとは、
\\	という種類の携帯電話に対する俗称で、元の
\\	は
\\	""の略である。ピッチという言葉は、
\\	が本サービス開始から2年が経過した1996年に使われ始めた(1997年版現代用語の基礎知識に掲載)。 
\\	の解説】 
\\	とは、第二世代デジタルコードレス電話を意味し、デジタル方式の
\\	は、当初、傍聴されにくい点と音質がいい点からコードレス電話の子機として利用されてきた。また、当初は
\\	""を略した
\\	という名称であったが、書籍などで知られる
\\	研究所があり、混乱を避けるため
\\	をつけた
\\	に変更される。 また、簡易な基地局により公衆回線に接続出来ることから、野外への携帯を目的とした電話機として普及。携帯電話登場後も、料金の安さや、当時の携帯電話は地下街での利用が出来なかったことから、
\\	に軍配はあがった。しかし、携帯電話の値下げや地下街でも利用出来るようになる中で、
\\	の利用エリアの狭さ、途切れやすさが浮き彫りとなり衰退していく。 この対策として、高速な通信速度を生かしたデータ通信へと転換。ノートパソコンなどの普及により、
\\	など、データ通信のみのものが主軸となる。 逆に音声通信に関しては、主力企業であったアステルが2004年にサービスを終了。2005年4月には
\\	ドコモが新規加入の打ち切りをしている。一方、ウィルコムが2005年5月より、
\\	同士の通話を定額制にするといった新たな動きもある。
\\	1996年	若者言葉	
\\	フリマ	ふりま	中古品や不用品を持ち寄って売買する市のこと。	フリマとはフリーマーケットの略で、中古品や不用品を持ち寄って売買や交換をする市である。ちなみにフリーマーケットのフリーは“
\\	(自由)”ではなく“
\\	(名詞では蚤[ノミ]のことだが、この場合「汚れた~」という形容詞として使用)”のこと。以前、日本では蚤の市 ・ ガラクタ市などと呼んでいたが、近年、フリマやフリーマーケットと呼ぶのが一般的となっている。	1996年	略語	
\\	マゴギャル	マゴギャル	コギャルより若い世代のギャルのこと。	マゴギャルとはコギャルのような奇抜な格好や言動をする中学生のことである。コギャルは本来“高校生ギャル”が略されたものだが、この「コ」の部分を「子」と捉え、子よりも更に下の世代ということから孫をつけたのがマゴギャルである。つまり、高校生ギャルよりも下の世代=中学生ギャルのことをマゴギャルと呼ぶ。しかし、コギャル自体が死語になりつつあり、マゴギャルも次第に使われなくなっている。	1996年	若者言葉	
\\	リスハラ	りすはら	リストラ ・ ハラスメントの略。	リスハラとは企業再構築のための解雇を意味するリストラと、嫌がらせを意味する
\\	(ハラスメント)を組み合わせ、さらに略した言葉である。リスハラはリストラという言葉が普及した翌々年の1996年より普及。リストラ対象者に会社側が行う嫌がらせのことをいう。例えば、些細なミスで減俸や降格をしたり、全く仕事を与えないといったものがリスハラである。リスハラを行うことで、企業からの解雇ではなく、雇用者の自己退職を促す。ただし、リスハラはセクハラなど一連の
\\	ハラという言葉の中で普及度の低い言葉である。	1996年	略語	
\\	ルートさん ・ 
\\	るーとさん	
\\	ケチな人のこと。 
\\	人並みな人のこと。	
\\	ルートさん
\\	は1.73・・・であることから「人並み」を意味する言葉として戦後の学生の間で普及した。 
\\	ルートさん
\\	は「人並みにおごれや」という語呂合わせで覚えることから「人並みにおごれや」と言われる人、つまり「ケチな人」を意味する言葉として、平成時代以降、若者の間で普及した。	1996年	若者言葉	
\\	ロンゲる	ろんげる	男性が長髪にすること。	ロンゲるとは男性の長髪を意味するロンゲ(ロン毛)に動詞化する接尾辞『る』をつけたもので、若い男性が長髪にすることを意味する。ただし、この言葉が使われた1996年はキムタクこと
\\	の木村拓哉がドラマ『ロングバケーション』でブレイクした年で、ロンゲるも単に長髪にするというより、キムタクのようなサラサラヘアの長髪にするという意味で使われた。ロン毛は現在も使われるが、動詞化したロンゲるのほうは死語となっている。	1996年	若者言葉	
\\	アカハラ	あかはら	大学で教授が下の立場の教職員や生徒にする嫌がらせのこと。	アカハラとはアカデミック ・ ハラスメントの略である。ここでいうアカデミックとは特に「大学の」「大学教職員の」という意味で用いる。つまり、アカハラとは大学内での教授による嫌がらせをいう(生徒に行うもの、立場の低い教職員に行うものがある。上下関係による嫌がらせということから、パワハラの一種でもある)。生徒に対する具体例として、授業を受けさせない、私的思惑で学位を与えないといったものがあり、下の立場の教職員に対しては、研究の妨害、不当に昇進を拒んだり、退職勧告するといったことがある。また、両者に関係するものとして、性行為の強要、私的用事をさせるといったこともある。アカハラは東京大学教授:上野千鶴子氏による造語で1995年に普及。後に「キャンパス性差別事情 
\\	ストップ ・ ザ ・ アカハラ」という書籍の副題にも使っている。	1995年	
\\	えむけいふぁいぶ	マジで切れる5秒前の略。	
\\	5とは「マジ
\\	で切れる
\\	5秒前」の略で、不快な環境や腹立たしい言動などで、今にもキレそう(感情が制御できなくなりそう)な状態にあることを表すコギャル語である。また、キレそうな感情が特に激しい場合には超
\\	5ともいう。 余談だが1997年に広末涼子が歌手デビューする際の竹内まりや作詞 ・ 作曲によるデビュー曲
\\	で
\\	する5秒前(マジで恋する5秒前)』は、この「マジで切れる5秒前」のもじりである。	1995年	コギャル語 ・ ギャル関連、アルファベット略語	
\\	官官接待	かんかんせったい	公務員同士による接待のこと。	官官接待の官とは官職の略で役人を意味する。つまり、官官接待とは官職に就く人から官職に就く人への接待=公務員同士の接待を意味する。多くは地方自治体の役人が補助金の決定権を持つ中央官僚に対して行われた(現在、こうした接待は禁じられている)。1995年7月30日、名古屋で開かれた全国市民オンブズマン連絡会議にて、全国の地方自治体で約29億円が官官接待に使われていると公表。これが話題となり、官官接待は同年の流行語大賞トップテン入賞した。受賞者は全国市民オンブズマン連絡会議代表:井上善雄氏。	1995年	
\\	公園デビュー	こうえんでびゅー	初めて子供連れで公園に行くこと。	公園デビューとは初めて子供を連れて公園に行くことで、子供が1歳になった辺りで行われることが多い。公園デビューの目的は子供が他の子の仲間入りすることと、母親が他の母親と交流を持つことにある。特に後者は、子育ての情報交換や母親同士で井戸端会をすることによる育児ストレス ・ 家事ストレスの発散として大きな役割をもっている。こうしたことから、育児の重要なポイントのひとつとされるようになった公園デビューだが、子供が他の子と馴染めなかったり、母親同士がしっくりいかないなどの理由で失敗することも多く、公園を転々とする親子もいる。	1995年	
\\	コンビニ離婚	こんびにりこん	食事がコンビニの惣菜ばかりという理由でする離婚のこと。	コンビニ離婚とは食卓にコンビニエンスストアで買った惣菜や弁当ばかりが並ぶことが原因でする離婚のこと。一人暮らしで親から料理を学ばず、コンビニ弁当に慣れ親しんだ
\\	が結婚後もその習慣のまま料理をしないことによるもの。また、コンビニエンスストアの特徴である「手軽」という点から、コンビニ離婚には手軽(=気軽、簡単)な離婚といったニュアンスが含まれる。	1995年	
\\	ダイエッター	だいえったー	ダイエットをする人、様々なダイエットグッズに手を出す人。	ダイエッターとはダイエットに英語で人化する接尾語
\\	をつけたもので、文字通りダイエットをする人のことである。「今日から私もダイエッター」といったように、始めたばかりの人を含め、ダイエットをしているあらゆる人を対象に使われる。ただし、ダイエッターという言葉が使われ始めた1990年代には単にダイエットをする人のことではなく、通販番組で話題のダイエットグッズ(運動器具)やダイエットサプリをいろいろ持っている人、雑誌や
\\	で新たなダイエット法が紹介される度に試すような人を嘲う言葉であった。当時の意味合いで認識している人には嫌味言葉になるので使用に注意が必要。	1995年	若者言葉	
\\	ティッシャー	てぃっしゃー	ティッシュ配りをする人のこと。	ティッシャーとはティッシュペーパーに英語で人化する接尾辞“-
\\	をつけたもので、駅前など街頭でポケットティッシュを配っている人を意味し、1990年代半ばに普及。こうしたティッシュ配り自体はそれ以前から行われているが、この頃から様々な言葉に“-
\\	を付けて人化させた言葉が流行(ダイエッター、アムラーなど)。ティッシャーもその中のひとつにあたる。また、広義には街頭でビラ配りしている人など、街頭で物を配る人のことをティッシャーと呼ぶ。	1995年	若者言葉	
\\	なんちゃって	なんちゃって	言動が本気ではない、嘘 ・ 冗談と表す言葉。	なんちゃってとは出来もしないことをやると言ってみたり、ありえないことを見たと言った後に付けることで、それまでの話しが本気ではない ・ 冗談 ・ 嘘 ・ 誇張していたことを表明する言葉である。1977~1978年、電車の中でこのな~んちゃってを使って笑わせるなんちゃっておじさんが話題に。
\\	でも取り上げられ、なんちゃっては全国的流行語となった。また、1990年代半ばになると偽物 ・ まがい物、見せかけだけといった意味で使われるようになる(例:なんちゃってセレブ、なんちゃってギタリスト、なんちゃってブランドなど)。	1995年	
\\	ノーティー	のーてぃー	大学の講義でノートを取り、友人に貸す人のこと。	ノーティーとはノートに、英語で「~する人」という意味にする接尾語
\\	を付けたものである。つまり、ノーティーは大学の講義に出席してノートをとり、それを友人に貸す人のことである。ノートを借りる側は授業もそこそこにサークルやアルバイト、コンパに遊びと好き勝手をし、試験前になるとノーティーにノートを借りて試験に挑むという形が多く、ノーティーをイジメ用語と考える向きもある。ただし、ノートをとって貸す人がノーティーであり、貸す理由が金品などの収入という場合もある。このため一概にイジメ用語とは決めがたい。	1995年	若者言葉	
\\	パソ婚	ぱそこん	インターネットを介して知り合った人と結婚すること。	パソ婚とは結婚とパソコンをもじった言葉で、パソコン通信を通じて知り合い、結婚することを意味する。後にパソコン通信からインターネットへ移行していくが、パソ婚という言葉はそのまま残り、現在もチャットや掲示板、出会い系サイトなどで知り合って結婚することや、そういったカップルのことをさして使われる。また、より開かれたインターネットになったことでパソコン(インターネット)を通じ、海外の異性と知り合って結婚するパターンも増えている。こうしたパソコンを介して海外の異性と知り合い、結婚することを国際パソ婚という。	1995年	
\\	ビル診	びるしん	ビルの一角で開業している診療所のこと。	ビル診とは「ビル診療所」の略で、ビルの一角で開業している診療所を意味する。規模は一室だけのものから、複数フロアを借りて開業しているものまで様々で、大都市の中心部やオフィス街、郊外の駅前などに多く見られる。なお、ビル診断といった場合、ビル診で診断を受けるという意味でなく、ビル(建物)自体の診断を意味するので注意。	1995年	略語	
\\	ポア	ぽあ	殺人のこと。	ポアとはもともとチベットの仏教用語“
\\	で「(人の意識を)移す」、中でも「死後、人の意識を仏界に移す」ことを意味する。オウム真理教が施設の捜査を受け、代表の麻原彰晃こと松本智津夫が逮捕、さらにその後の公判の報道から、オウム真理教がポアを「殺人」「殺す」という意味で使用していたことが発覚。若者を中心に殺すという意味で普及した。また、実際の殺人ではなくても、それに近いような苦痛を与えることもポアというようになる。例えば、
\\	さんをイジメのターゲットにする、
\\	さんを(活躍していた)業界から追放することなどを
\\	をポアする」といった形で使う。インターネットでも掲示板やチャットでよく使われたが、現在ではほとんど死語となっている。	1995年	
\\	目がハート	めがはーと	好きな人 ・ 物が視界にあり、それを見入るさまを表す言葉。	マンガ、アニメで登場人物やキャラクターの目がハートマークになるシーンがある。目がハートとはそういったシチュエーションに自分があることを意味する。例えば、好きな異性が視界に入ったり、好みのデザインの服や小物、ぬいぐるみなどが目に止まり、それを見入るさまをいう。中でも一目惚れのように、視界に入った瞬間、強い興味をひくといった際の使用が多い。	1995年	若者言葉	
\\	横の出世	よこのしゅっせ	地域活動や趣味仲間などで友達を作ること。	横の出世とは地域活動やボランティア活動、趣味といった(仕事以外の)自分が興味ある分野で友達を増やすことを意味する。企業における出世が昇給や昇格など上へ昇ることから縦の出世とし、これに対して仕事以外の仲間を増やすことは横へ幅を広げることから横の出世という。バブルの崩壊、リストラなどで縦の出世を諦めて横の出世を考える人、また仕事以外の生きがいの重要性を提唱するメディアが増えたことにより、横の出世を考える中高年が増えたとされる。横出世ともいう。	1995年	
\\	老いらく離婚	おいらくりこん	老いてからする離婚のこと。	老いらく離婚とは長い結婚生活の末、老いてからする離婚のことである。老いの基準は使う人の感覚で異なるが、70代以降の離婚に対して使われるのが一般的である。 老いらく離婚という言葉自体は以前からあるが、厚生省の統計で同居生活35年以上の夫婦の離婚が前年比を大幅に上回った1994年、この報道とともに老いらく離婚という言葉も普及した。2005年、同義語の『熟年離婚』がドラマタイトルになってからはほとんど使われない表現である。	1994年	
\\	肩たたき	かたたたき	上司が部下に退職を勧めること。	肩たたきとは上司が部下に退職を勧めることで、主にリストラなど企業(会社)が人員削減の際に行うものである。 肩たたきは元々大蔵省を中心とした役所関係の隠語で、当人の同意がないと辞めさせることが難しい役人ならではの言葉であった。一般に広く普及したのはバブル(景気)がはじけた1990年代からで、特に1994年にはリストラとともに終身雇用制崩壊の代名詞として使われた。	1994年	
\\	サビラン	さびらん	一人寂しく食べるランチのこと。	サビランとは「寂しいランチ」の略で、一人寂しくランチを食べる上司を嘲う言葉として
\\	が使用。ランチは同僚や部署仲間と一緒に楽しくとるという
\\	のスタンスに対し、会議や商談の連続でランチタイムくらいは一人ゆっくりと過ごしたいという中年サラリーマンの考え方のギャップによって生まれた言葉だが、的を得た表現がウケ、
\\	を中心に若者の間で普及。動詞として使う場合は「サビランする」「サビラン食べる」といい、サビラン族という言い回しも使われる。	1994年	若者言葉	
\\	サボテン女	さぼてんおんな	潤いがなく、とっつきにくい女性のこと。	サボテン女とはサボテンの特徴である乾きとトゲを持つ女性、つまり潤いがなく、とっつきにくい女性を意味する。1994年6月に週刊誌
\\	が「サボテン女の生態レポート」という企画取り上げ、広く普及。ずぼらな一人暮らしをする女性をさし、食事はコンビニ弁当、掃除 ・ 洗濯もほとんどせず、風呂もたまにしか入らないさまは水やりなどの手間がかからないサボテンそのもの。また、そんな汚れた部屋に人を寄せ付けないさまはトゲのあるサボテン。こうしてサボテン女と呼ばれるようになった。2007年7月から放送の
\\	系列のドラマ『肩ごしの恋人』。主演:米倉涼子が演じる早坂萌を貴島誠一郎プロデューサーがサボテン女と称したことから、負け犬、干物女に続く語句として再注目されている。	1994年	
\\	しゃばい	しゃばい	危ない(状況がよくない)又はさえない(ぱっとしない)こと。	しゃばいはその元となる語によって下記の二つの意味で使われる。しゃばいは形容詞だが双方の意味でシャバ、シャバッという感嘆詞としても使われる。 
\\	しゃばいとは俗語『やばい』が音的に変化したもので、「危ない」「状況がよくない」を意味する形容詞である。 
\\	しゃばいとは俗語『しょぼい』が音的に変化したもので、「さえない」「ぱっとしない」という意味で使われる。	1994年	
\\	シルハラ	しるはら	介護を受ける高齢者に対する嫌がらせのこと。	シルハラとはシルバー ・ ハラスメントの略である。シルバーとはシルバー世代(還暦を過ぎた60歳以上の人)のことで、ハラスメント
\\	は嫌がらせという意味の英語である。ただし、シルハラは単純に60歳以上の人に対する嫌がらせという意味でなく、老人介護が絡んだ嫌がらせのことで、介護疲れした肉親や介護施設の職員による肉体的 ・ 精神的嫌がらせを指す。また、介護を必要とする老人のプライド(誇り)を傷つけるような言動や介護拒否もシルハラにあたる行為であり、他のハラスメント(虐待 ・ 嫌がらせ)にはない特徴である。	1994年	略語 ・ 和製英語	
\\	たそがれ離婚	たそがれりこん	長い結婚生活の末にする離婚のこと。	たそがれ離婚とは長い結婚生活(同居生活)の末にする離婚のことである。 たそがれ離婚という言葉自体は以前からあるが、1994年に「35年以上同居した夫婦の離婚率が前年を大きく上回った」という厚生省の発表を報道する際、新聞の見出しにも使われ一時的に浸透した。2005年には同義語の『熟年離婚』がドラマタイトルになったことから、現在ではほとんど使われななくなった表現である。	1994年	
\\	デパガ	でぱが	デパートで働く女性のこと。	デパガとは『デパート ・ ガール』の略で、デパートで働く女性を意味する。デパガは「美人」「派手」「化粧が濃い」といったイメージを伴って使われるため、デパートの中でも主に化粧品 ・ ジュエリー ・ ファッション ・ 受付(案内 ・ サービスカウンター)といった部署に勤める女性を指し、特に食品売り場は除外されることが多い。また、デパート内でもエレベーターで案内をする女性はエレベーターガール(又はその略のエレガ)と別格で表される。 デパガは1990年代から普及した言葉で、それまでは『デパート ・ ギャル』を略したデパギャルが使われていた。	1994年	若者言葉	
\\	バカップル	ばかっぷる	度を越した熱々ぶりを見せるカップルのこと。	バカップルとは「馬鹿」と「カップル」の合成語で、『週間女性自身』による造語(同誌の「馬鹿っ母」に続いて出来た言葉)である。こうして1980年代半ばに造られた言葉が、実際に流行したのは1994年のこと。羽賀研二と梅宮アンナをメディアがバカップルと呼んだことによる。具体的には人前で平気でキスをしたり、イチャイチャするカップル、惚気話を延々とするカップルなど、こういった度を越した熱々ぶりを見せるカップルを嘲笑ったり、皮肉ってバカップルと呼ぶ。	1994年	
\\	パチーノ	ぱちーの	パチンコのこと。	パチーノとは作家 ・ 評論家の室伏哲郎による造語でパチンコ(パチンコ屋、パチンコ産業)を意味する。パチーノ
\\	はパチンコに対する「怖い」「汚い」「暗い」といった負のイメージの言葉にある頭文字
\\	をパチンコ
\\	から除いたものである。パチンコ店の大型化、近代化が進み、1992年には初の
\\	機も登場。パチーノが使われた1994年には総売上げが30兆円に達していたが、同時にパチンコへのめり込むことによる事故が多発するという問題も増えていた。	1994年	
\\	ピタ
\\	ぴたてぃー	身体に対し、サイズが小さめのぴったりフィットした
\\	シャツのこと。	ピタ
\\	とは「ぴったり(フィット)した
\\	シャツ」の略で、身体に対してサイズの小さいピチピチの
\\	シャツのこと(ここからピチ
\\	ともいう)。1990年代前半に1970年代のファッションが復活。ピタ
\\	はその中で再流行となったファッションのひとつで、男女共に浸透した。	1994年	若者言葉 ・ 略語	
\\	フェミ男	ふぇみお	体型、ファション、言動など、女性っぽい雰囲気の漂う少年のこと。	フェミ男とは女性を意味する
\\	または
\\	からきたもので、体型や物腰、ファッションなど、女性っぽい雰囲気を漂わせている少年のこと。具体的には体の線が細く、なで肩。体にはりつくようなサイズの小さいTシャツ、すその広がったパンツを着用し(中にはスカートを履く者もあった)、ピアスやアクセサリーを身につける(これらはあくまで一例)。当時、いしだ壱成や武田真治がフェミ男の代表として挙げられている通り、あくまで雰囲気が女性的ということであり、オカマとは異なる。現在はほとんど使われない死語である。	1994年	
\\	マリハラ	まりはら	未婚女性に「結婚まだなの?」と聞くこと。	マリハラは結婚を意味する「マリッジ
\\	と嫌がらせを意味する「ハラスメント
\\	を略したものである。具体的にマリハラとは未婚女性に「まだ、結婚しないの?」と質問することをいう。質問する側の悪意の有無や急かす意思の有無に関わらず、質問された女性が不快に感じればマリハラとなる。ただし、稀にマリファナ ・ ハラスメント(マリファナ(その他違法薬物)を無理に薦めること)の略として使われる場合もある。どちらにしてもマリハラはハラスメントを使った言葉の中では普及度の低い言葉である。	1994年	
\\	ヤン婆	やんばば	
\\	若くて孫のいる女性のこと。 
\\	元ヤンキーのおばさんのこと。	ヤン婆とはヤンママのもじりで、ヤングなお婆さんの略。若いのに孫のいる女性を指すが、ヤンママ同様、特に年齢の定義はない(一般的には40代以下で孫のいる女性)。 また、ヤンママ同様にヤン婆も元ヤンキーだったおばさんやヤンキーのように派手な格好のおばさんといった『ヤンキーなおばさん』の略という別の意味でも使われる。	1994年	若者言葉	
\\	ヤンママ	やんまま	
\\	十代の若いお母さんのこと。 
\\	ヤンキーの格好をしたお母さんのこと。	ヤンママとはヤングとママから成る合成で、若いお母さんを意味する。若いお母さんの年齢は主観的なものであり、特に定義はされていないが、一般的には10代のお母さんのことをいう。ヤンママは1990年代から使われ始めた言葉で、受賞はしていないが1994年の流行語大賞にノミネートされている。 また、ヤンママの『ヤン』をヤンキーの略とし、ヤンキーの格好をしたお母さんという意味もある。本来ヤンママはヤングママが略された言葉だが、若いお母さんの中には元ヤンキーやヤンキーの格好をした人も多かったこと、更にメディアがヤンママを取り上げる際、そういった若いお母さんを面白おかしく取り上げたことからヤンキーママの略という別の解釈が生じた。最終的にヤンママは二つの意味を持つ言葉として定着した。	1994年	若者言葉	
\\	リストラ	りすとら	リストラチャリングの略。	"リストラとは、英語
\\	(リストラチャリング)""の略で、英語本来の意味は再構築である。日本でカタカナ語として使われるリストラは、企業再構築における解雇や、企業 ・ 組織の縮小の部分のみを意味することが大半であるが、本来は企業が事業規模(収入)に合わせて、経費(支出)を見直し、組織を再編成することであり、単に人員削減のみを指すものではない。 1990年のバブル崩壊とともに、リストラを行う企業が急増。メディアで毎日のように取り上げられ、リストラという言葉も年代を超え、広く浸透した。
\\	1994年	略語	
\\	イエローカード	いえろーかーど	警告に値する失敗 ・ 問題をした人に発する言葉。	イエローカードとはもともと一部のスポーツで使われる用語(サッカー用語としての認識が高い)で、反則した選手へ警告として出す黄色いカード及び、警告を一回受けたことを意味する(イエローカードによる罰則は種目により異なる)。
\\	リーグが開幕し、サッカー用語が広く認知されるようになると、イエローカードはサッカーなどのスポーツ以外に仕事や日常生活の中でも使われるようになる。大抵は警告に値する問題行為や失敗をした人対し、審判がサッカーで反則をした選手にするように片手を上げ、「イエローカード」と言う。	1993年	
\\	オケる	おける	カラオケに行くこと。	オケるとはカラオケを略したものに、動詞化する接尾辞「る」をつけたもので、カラオケをしに行くことを意味する。こうした言い回しは1990年代前半に普及したが、年々使われなくなってきている。また、「オケする」といった言い回しもあるが、「オケする」の場合、オーケストラをする=オーケストラで演奏するといった意味で使われることもある(追記:バンドで歌を入れず、楽器だけで練習したり、インスト曲を演奏するという意味でも使われる)。文脈からどういった意味で使われているか判断する必要がある。	1993年	若者言葉	
\\	過労児	かろうじ	塾、習い事に忙しく、息もつけずにいる子供のこと。	過労児とは過労死のもじりで、学習塾や家庭教師、更に習い事や部活と息つく暇もなく、過労死してしまいそうな子供を意味する(児は子供という意味で特に6~12歳までの学齢児童に限定するものではない)。「いい会社に入るためにはいい大学に入学。そのためにはいい高校に・・・」といった考え方が蔓延し、子供たちは学校以外にも学習塾や家庭教師で勉強をするのが当たり前となっていた。過労児はそんな風潮への皮肉の言葉でもある。	1993年	
\\	恋人未満	こいびとみまん	男女の関係で、恋人と呼べるほどではない関係のこと。	恋人未満とは男女の間で友人以上に仲はよいが、恋人といえるほど親密な関係ではない(恋人という関係では考えていない)ことを意味する。二人の関係を恋人と勘違いしている周囲や恋人と思い込んでいる異性に恋人ではない(または恋人と考えていない)ことを伝えるために言うことが多く、「友達以上恋人未満」というフレーズでもよく使われる。また、性行為をし、肉体関係にはあるが恋人とは考えていない関係を「セフレ以上恋人未満」と言う。 2006年にはテレビ朝日で放送されたドラマのタイトルにもなっている。松本莉緒、安居剣一郎が幼馴染で恋人未満な関係を演じた。	1993年	
\\	コギャル	こぎゃる	1990年代の奇抜な格好をした女子高生のこと。	コギャルとは1990年代にみられたミニスカートにルーズソックス、ガングロに茶髪といった奇抜な格好の女子高生のことである。コギャルは本来、高校生ギャルを略したものだが、次第に「子ギャル」という認識(ギャルデビューする前の子)で使われるようになる。コギャルが登場した頃はその奇抜なファッションや言葉使いが流行の先端として取り上げられた。後にマスコミからモラルの低さや援助交際といった問題が取り沙汰されるようになり、コギャルという言葉とともにその存在も影を潜める。	1993年	若者言葉	
\\	知ったか	しったか	知らないのに知っているようなふうをするさま。	知ったかとは「知ったかぶり」の略で、本当は全く知らなかったり、あまり詳しくは知らなかったりするのに知っているようなふうで得意になるさま、知っているようなふうで語るさまをいう。また、知ったかぶりが多い人を指すこともある。この場合、人名っぽく知ったか君と言ったり、知ったか野郎ともいう。	1993年	
\\	シャネラー	しゃねらー	シャネルを好んで身につける女性のこと。	シャネラーとはフランスの女性ファッションデザイナーが興したブランド『シャネル
\\	に人化する英語の接尾辞
\\	をつけたもので、シャネルのブランド商品を好んで身につける人のこと。特に服から小物、アクセサリーまで、全身をシャネルで固めた人や大量にシャネル商品を購入する人をさす。 後に「アムラー」や「シノラー」など、
\\	を付けて人化する言葉がいろいろ出てくるが、シャネラーはその先がけである。	1993年	若者言葉	
\\	チャパツ ・ 茶髪	ちゃぱつ	「茶色い髪」の略で、茶色く染めたり脱色した髪のこと。	茶髪とは文字通り、茶色に染めたり、脱色した髪のことでチャパツと読む。ヤンキーなどの不良が自分を誇示するために茶色く染めていた頃から茶髪という言葉は使われていた。1990年代に入り、ギャルを中心に若者のファッションの一つとして広く普及、茶髪という呼び方も世代を超え、広く浸透した。逆に今なお茶髪=不良ファッションと考えることの方が滑稽に扱われる傾向にある。髪全体でなく、一部が茶色いもの、また茶色というより金色に近い髪に対しても茶髪という場合がある。これは主観によるもので、特に定義付けされていない。	1993年	若者言葉	
\\	謎本	なぞぼん	アニメ、漫画、小説などのフィクション作品の謎を考察する本。	謎本とは漫画やアニメ、ドラマや小説など、フィクション作品の設定やキャラクターにまつわる作品内の謎を考察 ・ 研究した本のこと。1992年に「サザエさん」を研究した『磯野家の謎 -サザエさんに隠された69の驚き(東京サザエさん学会)』が180万部のベストセラーに。さらに翌1993年に発売された『サザエさんの秘密 (世田谷サザエさん研究会)』もヒット。こうしてドラエもんを研究したもの、ちびまる子ちゃんを研究したものなど、次々に発売され、謎本というひとつのジャンルになった(謎本という言葉 ・ ジャンルを生むキッカケとなった書籍として『磯野家の謎』を紹介しています。これ以前にもこの手の本は存在します)。	1993年	
\\	ニケツ ・ 二尻	にけつ	
\\	二輪車による二人乗りのこと。 
\\	ひとつの椅子に二人で座ること。	
\\	ニケツとは自転車 ・ バイクといった二輪車で二人乗りすること。当初、不良を中心に普及した言い回しだが、次第に若者全般が使う言葉となる。なお、ニケツといった場合、多くはこちらの意味で使われる。 
\\	ニケツとは一人掛けの椅子に二人で座ること。主に若いカップルが一人掛けの椅子一脚しかないときにする。半ケツともいう。	1993年	若者言葉	
\\	西海岸	にしかいがん	トイレのこと。(俗語的解釈)	西海岸は英語で
\\	(ウェスト ・ コースト)」という。水洗式便所は英語で
\\	(日本でも
\\	という形で広く普及)」というが、それぞれ頭文字が
\\	であることから、トイレのことを西海岸と呼んだ。トイレに行く際、「ちょっと西海岸に行ってくる」といった使い方をするのだが、現在はあまり使われない死語である。	1993年	若者言葉	
\\	ババシャツ	ばばしゃつ	中高年の女性が好んで着る厚手の肌着のこと。	ババシャツとは中高年女性が好んで着る厚手の下着や長袖の下着のこと。1990年代前半に使われ始め、
\\	や女学生など、若い女性の間で普及し、定着している。色はベージュ、白、グレーなどでシンプルなデザイン。ファッション性より、温かさを保つといった機能性を重視したものだが、近年のババシャツは薄手ですっきりした感じに見せるものやキャミソールタイプなど、ファッション性にも気を使ったものが増えている。2005年にはウォームビズ運動の中でババシャツが注目を浴びた。	1993年	
\\	ベルサする	べるさする	終業のベルと同時に帰宅すること。	ベルサするとは終業のベル(チャイム)と同時に帰宅する人を意味する俗語『ベルサッサ』を簡略化し、『する』を付けて動詞化したもので、終業のベルと同時に帰宅することである。ベルサするは1990年代に入ってから若者の間で使われた。	1993年	若者言葉	
\\	ベル友	べるとも	ポケットベルでメッセージ交換する友達のこと。	ベル友のベルとはポケットベル(ポケベルとも言う)のことである。ポケットベルは電話機でダイヤルした数字を受信し、表示する小型受信機である。本来は連絡して欲しい電話番号を送信したり、番号による簡単なメッセージ(例えば01は「帰社せよ」というメッセージといったもの)を送るためのものだが、次第に数字の語呂合わせによるやりとりが成されるようになり(例:0840=オハヨオ(おはよう)、3470=サヨナラ、14106=アイシテル(愛してる)など)、若者の間にもポケベルが普及。こうしたポケベルによるメッセージ交換をする友達のことをベル友と呼んだ。1993年には緒方拳、裕木奈江主演のドラマ「ポケベルが鳴らなくて」の放送。さらに同タイトルのドラマ主題歌を歌った国武万里がレコード大賞新人賞を受賞するなど、ポケベルが流行の先端となったが、次第に
\\	、携帯電話が普及。これによりポケベル離れが起こり、ベル友も死語となった。	1993年	若者言葉	
\\	マクる	まくる	マクドナルドへ行くこと。	マクるとはハンバーガーショップ大手チェーン『マクドナルド』を略したものに動詞化する接尾辞『る』をつけたもので、マクドナルドする、つまりマクドナルドへ行く、マクドナルドで食事するといった意味で使われる。 ※盛んにする、無視するといった意味の「まくる」に関してはコチラ。	1993年	若者言葉	
\\	・ マルイチ	まるいち	離婚経験がある人や離婚をしたこと。	マルイチとは離婚経歴のある人を意味する俗語『バツイチ』を改変したもので、基本的な意味はバツイチと同じである。ただし、マルイチは×(バツ)が
\\	(マル)に改変されていることからもわかる通り、経験した離婚に対してよいイメージの言葉として使われる。例えば双方の未来のためにする離婚など、離婚が良い経験のひとつとして捉えられる場合にマルイチは使われる。このように主観的なイメージが影響するため、マルイチを使うのは当事者で第三者による「彼(彼女)はマルイチ」といった使い方はあまりされない。	1993年	
\\	メタルカラー	めたるからー	高度な技術が必要な職に就く技術者のこと。	メタルカラーとは襟
\\	の色から職種を表す言葉ブルーカラー、ホワイトカラーの派生語で、作家の山根一眞による造語である。メタルカラーは高度な技術が必要な作業を行う技術者 ・ 職人、また創造的な技術者を意味し、日本の技術を世界水準に引き上げた支えた人として賛辞されている。ただし、メタルカラーはあくまでイメージからきた色であり、そういった職に就く人の襟の色とは関係ない。	1993年	
\\	~モード	もーど	体調や状況 ・ 状態をあらわす言葉。	モードとは英語の“
\\	で「方法 ・ 様式」や「流行」といった意味で使われるが、これとは別に現在の体調や状況 ・ 状態を表す言葉としても使われる。この場合、モード単独では使わず、前にその状態を表す言葉を伴う。例えば、疲れているときなら「お疲れモード」、長期休暇や休憩 ・ 睡眠に入ったとき(入ろうとしているとき)なら「お休みモード」といった形で使用。1990年代から使われた表現で、当時に比べれば頻度は減っているものの、2007年現在も使用されている(ただし、当時使っていた人による継続的使用が大半である)	1993年	若者言葉	
\\	モニタリアン	もにたりあん	モニター募集にあちこち応募する女性(おばさん)のこと。	モニタリアンとは新製品など商品の試用や放送番組、新聞 ・ 雑誌などの視聴をし、内容に対する意見や感想を述べる人を意味するモニターと、羞恥心のないおばさんを意味するオバタリアンの合成語で、商品を無料でもらうことや謝礼を目的にあちこちのモニター募集に応募しまくるおばさんのこと。	1993年	
\\	リーマン	りーまん	サラリーマンの略。	"リーマンとはサラリーマンの略で、1990年代に入ってから使われるようになる 。リーマンはサラリーマンを嘲う意を込めて使われることが多い。ちなみに、サラリーマンとは、
\\	""という英語からきたカタカナ語で、
\\	""(俸給、給料、月給)で生計をたてる
\\	""(人)という意味である。ただし、日本でいうサラリーマンはアルバイトやパートなど、正社員雇用されていない人や会社員であっても役員の場合は含まないことが多い。更にイメージ的なものとして、ブルーカラーでなく、ホワイトカラーを連想する言葉として使われる。また、サラリーマンはリーマン以外にサラリーと略す場合もある。 註)英語では
\\	""は
\\	""(主に肉体労働に対する賃金労働者)の対になり、階級的意識が伴うため
\\	""など具体的職業名で呼び、
\\	""はほとんど使われない。
\\	1993年	略語	
\\	六月病	ろくがつびょう	6月に入って新入社員がかかる病(症状)のこと。	六月病とは新入生(主に大学生)が入学一ヶ月後にかかる症状『五月病』の派生語で、五月病が大学生を主に対象としているのに対し、新入社員、新社会人を対象とする病(症状)である。新入社員(新社会人)の場合、新入生(新大学生)と違い、スタート前に研修期間が設けられることが多い。その後、正式に配属が決まるのだが、その研修期間のズレにより、五月病のような症状が6月に出ることがあり、それを六月病と呼ぶようになった。 ただし、六月病も五月病同様、医学用語ではなく、特に定義付けされた病気というわけではない。	1993年	
\\	ロン毛	ろんげ	男性の長髪のこと。	ロン毛のロンは英語“
\\	(ロング)”の略で、ロン毛は長い髪を意味する。ただしロン毛といった場合、特に男性の長髪を指す。1980年代まで男性の長い髪は単に「長髪」と呼ばれていたが、1990年代に入り、男性の長髪が若者ファッションのひとつとして取り入れられるようになる中でロン毛と呼ばれるようになる。このため、男性がファッションや拘りから意図的に伸ばしている長髪を意味し、不精で単に髪が伸びているだけの長髪に対してロン毛とはいわない(嫌みやからかって言う場合を除く)。	1993年	若者言葉	
\\	アナドル	あなどる	アナウンサー ・ アイドルの略。	アナドルとはアナウンサー ・ アイドルの略で、アイドル的な人気のあるアナウンサーのことである。またアナドルが使われた1990年代始め、ドラマや音楽活動、グラビアといった本職以外の仕事をするアナウンサーが現れたことからタレント(アイドル)のような活動をするアナウンサーといった意味でも使われる。アナドルは当初そういった華やかなアナウンサーを讃美する言葉であったが、次第に「外見だけで中身がない」という意を含んで使われるようになる。 元祖アナドルは1989年にドラマ『同級生』に出演したフジテレビの中井美穂という意見と、女子アナ3人のグループ
\\	に所属した日本テレビの永井美奈子、藪本雅子、米森麻美(故)という意見がある。日テレがアナウンサーをアイドル的に扱う方向性を打ち出していたことやアナドルという言葉が普及した時期を考えると後者の意見が妥当と思われる。	1992年	略語	
\\	アルハラ	あるはら	アルコール ・ ハラスメントの略。	アルハラとはアルコール ・ ハラスメントの略である。ハラスメント
\\	とはセクシュアル ・ ハラスメントに見られるように「嫌がらせ」という意味の言葉である。つまりアルハラはアルコール(酒)絡みの嫌がらせをいう。具体的にはお酒が飲めない人に飲むことを強要したり、一気一気をさせることをいう。上司や顧客、先輩など立場が上の人間が「俺の酒が飲めないというのか?」と言って半強制的に飲ませる行為は典型的アルハラにあたる。	1992年	略語	
\\	アンビリ ・ アンビバ	あんびり ・ あんびば	超常現象。また、そういったことを体験したときに発する言葉。	アンビリとは「信じがたい~、驚くべき~」といった意味の英語“
\\	が略されたものである。松田聖子が
\\	の中で「アンビリーバボー」といった事から1992年に
\\	の間で普及。当時は「信じられなーい」「うっそー」といった驚きを表す言葉として「アンビリー!」と言っていた(アンビバというグループ(エリア?世代?)もある)。一度は死語となりかけたアンビリだが、フジテレビ系列放送、ビートたけし出演の
\\	番組『奇跡体験!アンビリバボー』の人気とともに再度使われるようになる。アンビリ現象、アンビリ事件といった使い方や単に超常現象の別称としての使用が増えている。	1992年	略語	
\\	エイハラ	えいはら	中高年(高齢者)への嫌がらせやイジメのこと。	エイハラとはエイジハラスメントの略で、もともとは企業内での中高年者に対する年齢を理由にした差別 ・ 嫌がらせのことであった。後に家庭や高齢者施設内での高齢者に対する嫌がらせ ・ イジメという意味でも使われる。企業については主に役職についていない中高年社員に対する無視がある。また、永年勤務の
\\	に対するセクハラは内容や理由によってはエイハラにも含まれる。家庭や施設でのエイハラについては心理的虐待 ・ 身体的虐待 ・ 経済的虐待 ・ 性的虐待とそれぞれ具体的に定義されており、無理矢理食事を口に運ぶことやベッドに縛り付ける、(侮辱的に)子供のように扱うなどがあげられている。	1992年	略語	
\\	借りパク	かりぱく	人から借りた物をそのまま自分の物にすること。	借りパクとは「借りてぱくる」「借りた物をぱくる」の略である。「ぱくる」の意味のひとつに「盗む」がある。要するに借りパクとは人から借りた物を盗ってしまうことである。ただし、万引きや泥棒のように初めから盗ることを前提にしていることは少なく、借りたことを忘れ、結果的に私物になった(主にマンガなど安価な物の貸し借りに見られる)、当初は返すつもりであったが返せなくなった(主に金銭の貸し借りに見られる)といったものが多い。貸した側が忘れていることも多い。また、エリアによっては借りパチともいう。	1992年	
\\	着まわし ・ 着回し	きまわし	一つの服を組合せを変えることで違う感じに見せること。	着まわしとは服やパンツ、スカートの組み合わせを変えたり、異なる装飾品を身につけることで幾通りもの違う感じに見せて着こなすことをいう。バブル崩壊後、節約という言葉が重要なキーワードとなる中、ファッションの世界での節約として普及。以降、着こなしにより、少ない服をいかにセンスよく組み合わせ、バリエーションを増やせるかがファッション ・ センスのポイントの一つと言われるまでになり、ファッション誌でも度々取り上げられるキーワードとなる。	1992年	
\\	ゲーセン	げーせん	ゲームセンターの略。	ゲーセンとはゲームセンターの略で、100円前後のコインゲームが出来る遊技場のことである。ゲームセンター自体は1970年代末のインベーダー ・ ブーム以来人気を博しているが、ゲーセンという略し方が普及したのは10年以上を経た1992年以降である。 ちなみにゲーセンの元となる『ゲームセンター』、英語の“
\\	からきていると思っている人も多いが、実は和製英語である。ちなみに英語では“
\\	といい、ゲームセンターの客のことを“
\\	という。最近、ゲームセンターやボーリング場、カラオケなど複合的な娯楽施設の入ったものを『アミューズメント』と呼ぶが、イギリス英語ではゲームセンターを“
\\	という。	1992年	若者言葉 ・ 和製英語	
\\	ゲロ男	げろお	不細工な男性のこと。	ゲロ男とは不細工な男性という意味で1990年代初めに使われた言葉である。この場合のゲロとは吐瀉物や食べた物を吐き戻すことを意味するが、吐瀉物のような男という意味で使われたのか、吐き気がするような男という意味で使われたかは不明。またゲロ男が普及する前に女性を対象とした類語『ゲロブス』が使われており、その派生語とも考えられるが、ゲロブスのゲロは「すごく」といった意味であり、ニュアンスは若干異なる。 ※上記意味での使用は年々減少しており、現在ゲロ男といった場合、単純にゲロをした男という意味で使われることが多い。	1992年	
\\	午後一	ごごいち	午後の最初の授業、午後一番目にすること。	午後一とは朝一(あさいち)の派生語(対語)として出来た言葉で、午後一番の略である。学生の場合、給食時間後や昼休み後の午後最初の授業を午後一と言い、社会人の場合、昼休みを終え、最初にすることを言う。ここで社会人の場合、その時間が問題となるが、具体的に午後一が何時かは定義されていない。ただし、12~13時を昼休みとする官公庁 ・ 企業が多いことから、13時頃(13時過ぎ)を指すのが一般的である(サービス業など昼休みの時間帯が異なる業種や企業、また休日の約束事などで使う場合、この通りではないこともある)。	1992年	
\\	サービス残業	さーびすざんぎょう	手当て ・ 賃金無しで行う残業のこと。	サービス残業とは労働基準法に定められた手当・賃金が雇用主から支払われない中で行う残業(時間外労働)のことである。例えば企業側は残業の禁止を謳う。残業をする労働者はそれを無視して勝手に残業したという構図を作り、残業の申請をしにくくする。また、単に会社や部署の雰囲気で残業=サービス残業といった図式を定着させてしまっていることも少なくない。家に持ち帰ってする仕事はサービス労働というが、社内ですべき残務を家に持ち帰っているものが大半であり、サービス残業と考えてよい。これらサービス残業は過労死の原因にもなり、社会問題のひとつとなっている。	1992年	
\\	事実婚	じじつこん	婚姻届は出していないが、事実上婚姻状態にある関係のこと。	事実婚とは婚姻届を出していないが、それ以外は婚姻状態にある関係をいう。家族や友人など周囲から夫婦と認められている関係であり、結婚式もあげたりと、同棲とは一線を画す。婚姻届を出さず、事実婚を選ぶ理由は様々だが、姓を変えず、別姓のままでいたい、法律上の結びつきでなく、双方の意思で夫婦として結びついていたいといった精神的理由が多いとされる。事実婚の場合、住民票の続柄は同居になるが、夫 ・ 妻のどちらも独身であれば夫(未届)、妻(未届)という表記に出来る(どちらか一方、または双方が他と婚姻関係にあっても、事実婚は法的に婚姻するわけではないので、重婚には値し無い)。また、事実婚(内縁)の夫婦としての法的権利や義務も民法で示されている。	1992年	
\\	社内妻	しゃないつま	社内の女性社員を妻のように扱うこと。	社内妻とは職場内の女性社員に妻のような雑事をさせている状態やそういった雑事をさせられている女性社員をさす。具体的にはタバコや文具などのお使い、脱いだ上着の片付け、お茶汲み(通常のお茶汲みと違い、お茶が欲しくなった際、女性社員に湯呑みを突き出すことでお茶を入れるよう指示する横着なもの)が社内妻の雑事にあたる。1990年代初め、セクハラが話題となる中、女性社員の意思表示の場が増えた。社内妻は以前からあったこうした男性社員(主に中堅男性社員)の怠慢が取り沙汰されるようになる中で生まれた言葉である。社内妻という響きから不倫などを想像しがちだが、そうした性的関係やセクハラ行為とは別次元の言葉である。	1992年	
\\	スチールカラー	すちーるからー	産業ロボットのこと。	労働者の種類を従事るす職によって変わる襟
\\	からブルーカラー、ホワイトカラー、更にその派生語としてゴールドカラー、ピンクカラーなどと呼ぶがスチールカラーもその一種である。ただし、他のカラーが職種やその職種に従事する人を指すのに対し、スチールカラーは職種や人ではなく、その襟の色が示す通り、鋼鉄製のもの、つまり産業ロボットを意味する。バブルが崩壊後、大幅なリストラとともにオートメーション化が叫ばれた。そんな中でスチールカラーという言葉も使われた。	1992年	
\\	ダサダサ	ださださ	非常にかっこ悪い ・ 非常に野暮ったいこと。	ダサダサとはかっこ悪い ・ 田舎臭い ・ 野暮ったいといった意味の俗語『ダサい』及びその名詞形『ダサさ』の『ダサ』を畳語にしたものである(畳語とは「多々」「国々」「たまたま」など同じ要素が2度繰り返される言葉)。ダサダサと畳語にすることで語り手の「かっこ悪い」「野暮ったい」といった思いが強いことを強調している。このため『ダサい』や『ダサさ』と違い、肯定的な表現(「ダサいところがかわいい」「ダサさがよかったりする」など)には使われない。	1992年	若者言葉	
\\	チーマー	ちーまー	盛り場にたむろし、喧嘩を売ったりする不良ぶったグループ。	"チーマーとはゲームセンターにたむろしたり、コンビニの駐車場でしゃがんで話し込み、喧嘩を売ったりする不良ぶったグループやグループに属している人のことである。チーマーは「団 ・ 組」を意味する
\\	""に「~する人」という意味の
\\	""を付けた造語であることからもわかる通り、単なるヤンキーや不良少年少女とは違い、グループで活動することが特徴である。このことから「集団でないと行動出来ない」という蔑した意を含んで使われることが多く、1990年代初頭に普及したチーマーという言葉は2000年以降あまり使われなくなった。 ちなみに中国語にチーマーという言葉があるが、これは「馬に乗る」という意味で上記のチーマーとは無関係である。
\\	1992年	不良用語 ・ 和製英語 ・ 若者言葉	
\\	天動説世代	てんどうせつせだい	転職を重ねる世代のこと。	天動説世代とは地球の周りを太陽や星が公転しているとする天動説同様、自分を中心に世の中が回っていると考える世代のことで、具体的には転職を重ねる世代をさす。核家族の中で叱られることも少なく、何でも思い通りになっていた学生時代を過ごしたこの世代は、社会人になるとちょっとした不満から退職し、転職を重ねた。ただし、こうした傾向の若者が毎年社会に出る中で、特定世代の特徴ではなくなったことやフリーター、ニートといったタイプが登場する中で、天動説世代は使われなくなり、死語となっている。	1992年	
\\	ドタキャン ・ どたキャン	どたきゃん	直前に約束をキャンセルすること。	ドタキャンとは『土壇場』と『キャンセル』から成る合成語で、約束や申し込み、契約などを直前にキャンセルすることである。デートなど待ち合わせ時間直前のキャンセルや結婚の準備が整い、式当日を待つ状態での婚約解消などがドタキャンにあたる。ドタキャンはもともと芸能界で使われていた言葉だが、後に旅行会社など顧客との契約で仕事を進める営業マンの間で使われるようになり、一般にも普及。ドタキャンはこういった行為をする側よりも、された側に使われることが多く、中には単なる『すっぽかし』に対して使われることもある(すっぽかされた側の捉え方による)。	1992年	合成語	
\\	なつマン	なつまん	「懐かしいマンガ」の略。	なつマンとは「懐かしいマンガ」の略で、1980年代後半辺りから行われたリバイバル版のマンガに対して使われた言葉である。ちなみに、リバイバルされた懐かしいアニメのこともなつマンという(「なつアニ」とは言わない)。当時、横山光輝の『鉄人28号』など、1950~60年代に発売されたマンガを対象になつマンと呼んだが、年を追うごとに発売時期と懐かしいと言われるまでの期間が短くなっている。どちらにしてもなつマンという言い回しは使われなくなり、死語となっている。	1992年	若者言葉	
\\	生足	なまあし	女性のファッションのひとつ。	
\\	生足とは、スカートのときにストッキングやタイツをはかず素足を出したファッションのことで、生足という言葉自体は1990年代始めから普及。ファッションとしては1996年に流行となった。ファッションの場合は靴下をはいた状態も生足と呼ばれる。ただし、ハイソックス、ニーソックスは除く。 
\\	上記のファッションとは別に、男性の性的趣向(フェチ)のひとつとしても生足という言葉は使われる。この場合、靴下もはいていない完全な素足の状態を意味することが多い。	1992年	
\\	パセリ君	ぱせりくん	適齢期を過ぎても結婚出来ずにいる男性のこと。	パセリ君とは結婚適齢期を過ぎ、何歳になっても結婚出来ずにいる男性を嘲う言葉である。これはこういった男性を、いつまでも手を付けられず、料理の皿に残るパセリに例えたものである。 ちなみにパセリちゃんという言葉もあるが、これは防衛省 ・ 自衛隊のキャラクターの名前であり、パセリ君に通じる意味では使われていない。	1992年	
\\	・ バツイチ	ばついち	離婚経歴のある人や離婚をしたこと。	バツイチとは離婚経験のある人のことである。戸籍謄本に配偶者を書き込む欄があるが、離婚し除籍となった配偶者を×(バツ)と書いて消していたことから、離婚経験者をバツイチと呼ぶ(戸籍謄本がコンピュータ管理されるようになってからは配偶者を×印で消すのではなく、身分事項欄に「離婚」や「除籍」と書かれる)。バツイチは明石家さんまが大竹しのぶとの離婚会見の際に使ったことで広く浸透し、会見のあった1992年の流行語となった。また、離婚を2回経験している者を×2(バツニ)、3回の者を×3(バツサン)という。バツイチについては2回以上の離婚経験者にも使われるため、1回以上離婚を経験した者と考えてよい。	1992年	
\\	ピンクカラー	ぴんくからー	女性が多く占める職種のこと。	ピンクカラーとは襟
\\	の色から職種を表す言葉ブルーカラー、ホワイトカラーの派生語である。ただし、ピンクカラーのピンクは襟の色でなく、女性をイメージ色としてあてがわれたもので、ピンクカラーは昔から女性が多く占める職種をさす。例えば、看護師(看護婦)、客室乗務員(スチュワーデス)、保育士(保母さん)、更に枠を広げると秘書、図書館司書などもピンクカラーに分類される。ピンクカラー ・ ジョブ、ピンクカラー ・ ワーカーズともいう。	1992年	
\\	不燃ゴミ	ふねんごみ	家庭や夫に対し、不満を抱いている妻のこと。	定年退職し、家で何もせず、ブラブラしている夫のことを粗大ゴミというが(下記関連語参照)、不燃ゴミはその対語で家庭や夫に不満を抱く妻を意味する。心の奥でくすぶる家庭に対する不満や夫に対して燃えなくなったさまを不燃ゴミに例えたもので、妻自身が自嘲気味に使用。ただし、粗大ゴミのようには普及せず、現在ではほとんど使われなくなっている。	1992年	
\\	冬彦さん	ふゆひこさん	マザー ・ コンプレックスな男性のこと。	冬彦さんとは1992年、
\\	系で放送されたドラマ「ずっとあなたが好きだった」で佐野史郎が演じた役名である。この冬彦はマザコン男で性行為は
\\	といったアブノーマルなものを好み、蝶のコレクションをするオタクな面を持ち合わせる特異なキャラクターであった。こうしたキャラが話題となり、同年の民放連続ドラマの中で最高視聴率を樹立。冬彦さん現象といったものが流行となり、冬彦さんはマザコン男の代名詞となった。ただし、番組終了後は次第に使われなくなり、現在では死語となっている。	1992年	
\\	フレカジ	ふれかじ	フレンチ ・ カジュアルの略。	フレカジとは「フレンチ ・ カジュアル」というファッション ・ スタイル名を略したものである。フレカジの特徴は黒、グレー、デニムといったベーシックな色調を用いることで、デザインもシンプルなものが多い。こうしたフレカジはそれ自体の主張が弱く、他の様々な服や小物とコーディネートしやすいことから好まれた。特にバブル崩壊後は合理性重視という需要に移行していたこともあり、それにマッチしたという見方もある。	1992年	
\\	マザコン	まざこん	マザー ・ コンプレックス
\\	の略。	"マザコンとはマザー ・ コンプレックス
\\	を略した和製英語で、この場合の
\\	""は「依存心」「執着心」と訳し、「男性 ・ 男の子が母親に抱く執着心」や「母親に対し、執着心を持つ人」という意味になる。マザー ・ コンプレックス自体は精神分析の創始者“フロイト
\\	が生んだ言葉だが、マザコンといった場合、いくつになっても乳離れ(親離れ)出来ない男性や母親に頭があがらない男性、彼女や妻に対し母親像を求める男性を嘲うときに使われる。また、マザコンは
\\	系列の「冬彦さん」で一斉を風靡したドラマ『ずっとあなたが好きだった』(1992年7月~9月放送、佐野史郎・賀来千香子・野際陽子 出演)により流行語的にも使われた。
\\	1992年	和製英語(略語)	
\\	ワクドキ	わくどき	期待と不安が入り混じったさまを表す言葉。	ワクドキとは期待や喜びで心がはずみ、落ち着かないさまを表す「ドキドキ」と不安や緊張で動悸のするさまを表す「ドキドキ」の合成語で、期待と不安 ・ 興奮と緊張が入り混じり、落ち着かないさまを表す。1990年代始めに女学生から広まった言葉で、2008年現在も使われている。また、三重テレビでは「ワクドキ!元気」やその後継番組「とってもワクドキ!」といった番組名に使用している。	1992年	若者言葉 ・ 合成語	
\\	油ギッシュ ・ 脂ギッシュ	あぶらぎっしゅ	顔などが脂ぎっていること。	油ギッシュとは『脂ぎる』と「精力的」「元気旺盛」を意味するドイツ語
\\	(エネルギッシュ)』から成る合成語で、顔や首筋などが脂ぎっていることである。油ギッシュは1990年代から若者の間で使われるようになり、油ギッシュの他に脂ギッシュという表記も使われる。 また、油ギッシュは外見的なことだけでなく、会話(主に説教)や性格がしつこいという意味でも使われるようになる。この場合、多くは大人(主に中年男性=オヤジ)を対象に使われる。	1991年	合成語 ・ 若者言葉	
\\	オキニ ・ お気に	おきに	「お気に入り」の略。	オキニとは「お気に入り」の略で、お気に入りの物や人、店などを指して主に女の子(ギャル ・ コギャル)が使う言葉である。 また、バンギャの間では特にアーティスト(バンドメンバー)に気に入られている女性ファンのことをいう。	1991年	若者言葉、コギャル語、略語	
\\	オバンチュール	おばんちゅーる	中年女性との恋の冒険(不倫)のこと。	オバンチュールとは中年の女性を意味する「おばん」とフランス語で冒険 ・ 浮気を意味する
\\	(アバンチュール)」から成る合成語で、中年女性との恋の冒険、不倫を意味する。1990年代始めに流行ったが、死語となっている。	1991年	カタカナ仏語+合成語	
\\	キープ君	きーぷくん	女性が予備の結婚希望相手という対象で考えている男性。	キープ君とは若い女性が本命の結婚希望相手である男性とうまくいかなかったときのために、予備として考えている男性を意味する。また、本命の相手は特にいないが、本命にするには物足りない男性を本命が見つかるまでのキープ君にするというパターンもある(最終的に本命が現れなかったときはキープ君と結婚することになる)。キープ君はバブル期の若い女性の間で普及。不景気になり、時代が厳しくなるとともに死語となっていった。	1991年	若者言葉	
\\	きしょい	きしょい	気色悪いの略。	きしょいとは気色悪いの略で、見た目に気色悪い場合に使われることが多い。きしょいは関西を中心に会話の中で生まれた言葉で平成に入ってから普及。きしょいは『きしょー』という感嘆詞として使われることが多く、関西では『きっしょい』ともいう。また、気色悪いけどカワイイという意味を持つ『きしょかわいい』という言葉もある。 類義語に『きもい』があるが、これは気持ち悪いからきている言葉できしょいよりも意味合いは弱い。きもいの語意を強める場合は言葉の前に『超』をつけ『ちょーきもい』となるが、きしょいの場合、関西が中心だったこともあり『超』ではなく『めっちゃ(めっさ)』をつけ『めっちゃきしょい』となる。 きしょいの使用例: あの映画、きしょいシーンが多くて吐き気がした。 あの人、暗いというより、きしょいのよね。	1991年	若者言葉(略語)	
\\	キレる	きれる	感情が高ぶり、理性が利かなくなる状態のこと。	キレるとは感情の高ぶり(怒り)を理性で抑えきれなくなる状態のことで、腹を立てる程度から人に危害を加えてしまうものまで様々である。キレるという衝動は人の言動や態度、自分の思い通りにならない事象など要因があって起こる。マンガで怒りを表現する際、こめかみの血管を浮き立たせる(腫れる?)描写が使われるが、キレるはこのこめかみの血管や脳の血管(回路)が怒りで切れるという意を含む。1990年代までは『ぷっつん』という表現が使われ、キレるも並行して使われてはいたが、不良仲間や不良マンガで使われる言葉であった。一般に広く使われるようになるのは1990年代に入ってからで、後にキレやすくなった若者が社会問題のひとつとして取り上げられるようになる。	1991年	若者言葉	
\\	近所姑	きんじょしゅうと	(余計な)世話をやく近所のおばさんのこと。	近所姑とは嫁いだ先のご近所で、世話をやいてくるおばさんを意味する。嫁いだばかりでどこに何があるかわからないときに教えてくれる人、その土地の風習を教えてくれる人という良い意味で使われる場合もあるが、姑という言い回しはマイナスイメージで使われることが多いように、近所姑も「うるさい」「お節介をやく」「監視されているよう」といった悪い意味合いで使われることがほとんどである。またそういった人の中で、親に近い年齢の女性を近所姑、少し年上の女性を近所小姑と使い分けもする。	1991年	
\\	ジモティー ・ ジモッピー	じもってぃー ・ じもっぴー	地元の人のこと。	ジモティーとは『地元』に「~する人」「~される人」という意味にする英語の接尾語
\\	をつけたもので、「その土地の人」という意味で使われる若者言葉である。 ジモティーはジモッティー(当初はこの呼び方で普及)や語尾の『ティー』を『ピー』(恐らくピープルから取ったものと思われる)に入れ代えたジモッピーとも言われる。	1991年	合成語(和製英語)・若者言葉	
\\	セクハラ	せくはら	性的いやがらせのこと。	セクハラとはセクシュアル ・ ハラスメント
\\	の略で、相手を不愉快にさせる性的な言動を意味する(主に職場の上司と部下など、上下のある人間関係の間で行われるものに対して使われる)。1991年3月、日本弁護士連合会が『男女雇用均等法』の改正を求める意見書を提出。セクハラ防止の明文規定を設け、雇用主に対するセクハラ予防対策の義務を要求した。このことをメディアが取り上げる際、セクハラという言葉も普及。当初、セクハラ対策は女性が快適な職場で働けるようにするためのもので、男性が女性に対して行うセクハラ行為が対象であったが、次第に逆の事例による訴えも出ている。これらは『逆セクハラ』というが、男女雇用均等法では逆セクハラに関する内容は触れられていない。セクハラは身体に不要に触れるなどの行為から性的な冗談、食事に執拗に誘うといったものまで受け手が不快と思えば全て対象となる。しかし、これがエスカレートし、些細なことでもセクハラと言って騒ぐ風潮が逆に問題にもなった。 また、セクハラ行為をした人物の職業や役職などをセクハラの後につけた言葉も数々生まれた(例:セクハラおやじ、セクハラ教師、セクハラ社長など)	1991年	カタカナ英語+略語	
\\	ゾッキー	ぞっきー	暴走族のこと。	"ゾッキーとは「暴走族」を簡略化した俗語『族(ゾク)』に英語で人を表す接尾語
\\	""を付けたもので、「暴走族に所属する人」や単に「暴走族」という意味でも使われる。また、ゾッキーは語感から暴走族に対して嘲う意を込めて使われることが大半である。このため、ゾッキーは第三者が使用する言葉で、当事者が自らや暴走族仲間を指して使うことはない(自嘲する場合や冗談として使用するなど例外は存在する)。
\\	1991年	若者言葉	
\\	チキンレース	ちきんれーす	根競べ ・ 度胸試しの要素が強い駆け引きのこと。	チキンレースとは本来、ジェームスディーン主演の映画『理由なき反抗 
\\	に見られる度胸試しのためのレース(又はゲーム)のことで、壁や崖に全速力で向かい、どちらが壁(崖)近くまでブレーキを踏まずにいられるかを競うものである(チキンゲームともいう)。ルールは上記以外にも多種存在するが、基本的には相手との根競べであることから根競べの状態を指してチキンレースというようになる。特に株取引の際、こういった場面が多いことから株の世界でよく使われる他、ボードゲームやカードゲーム、オンラインゲームなど複数参加のゲームで、リスクを伴うが勝ちに繋がるような駆け引きの状態を指しても使われる。	1991年	
\\	昼食難民	ちゅうしょくなんみん	オフィス街で昼休憩に食事にありつけない状態のこと。	オフィス街では昼休憩時(主に12~13時)、一斉に大勢の人が飲食店に駆け込み、一時的に飲食店の数と客数の需給バランスが崩れる。昼食難民とはこうした状況により、昼休憩の間にランチをとれなかった人や出遅れて店の行列に並ぶ人たちを指す(ランチ難民ともいう)。時間帯をずらしたり、弁当を持参するといった回避策をとる人も増えている他、これを逆手にとった車による移動型の飲食店 ・ 弁当販売で成功している人もいる。	1991年	
\\	内食	ないしょく ・ うちしょく	家で素材から調理して食べる食事のこと。	内食とは外食の対語で家で、食べる食事を意味する。ただし、惣菜や弁当など、外で買ってきたものを家で食べることを外食と内食の中間という意味で中食と呼び、内食は家で素材から調理したものを食べることをいう。ないしょくと読むことが多いが、うちしょくと読まれることもある。「内職」との混同を避けるためか、家のことを「うち」とも呼ぶからか、単に人や地域、団体によって異なる読まれ方がされていただけなのかは不明。(広辞苑第六版ではないしょくと記した上で、補足的にうちしょくと書かれている)	1991年	
\\	中食	なかしょく	惣菜や弁当など、外で購入し、家で食べる食事のこと。	中食とはレストランや定食屋などの外食に対し、家で自炊および家族が作った手料理を食べることを内食とし、その中間にあたる食事を意味する。具体的にはスーパーの惣菜やコンビニ ・ 仕出し屋の弁当、調理済みのパン(サンドウィッチ、惣菜パン)などが中食に当たり、こういった商品の製造販売業を中食産業と呼ぶ。当初、中食はちゅうしょくと呼ばれることも多かったが、昼食と区別するためか、なかしょくという読みで定着しつつある。また、中間食(ちゅうかんしょく)ともいう。	1991年	
\\	ねるとん	ねるとん	「お見合い」や「コンパ」のこと。	"ねるとんとは「お見合い」や「コンパ」を意味する言葉で、1987年10月から1994年12月に放送されたお見合い番組『ねるとん紅鯨団』からきている。そもそも『ねるとん』という言葉は番組司会者である""とんねるず""を引っくり返して作られた造語である。番組は23時台という時間帯にも関わらず、1989年には20
\\	を超える高視聴率を獲得。この頃から若者の間でコンパのことをねるとんと言うようになり始める。この番組が始まるまでのコンパは現在と違い、「暗い人たちの集い」といったイメージが強かったが、『ねるとん紅鯨団』がそのイメージを払拭。女性をコンパに誘うときも「ねるとんしない?」といった軽いノリで誘える、そんな理由から『ねるとん』は若者に重宝された。企業もこれに便乗し、結婚相談所など様々な企業が『ねるとんパーティー』と銘打ち、大会場やクルーズでのお見合いパーティーを催している。1991年には、リクルート出版の旅情報誌""じゃらん""が『ねるとんツアー』という特集をし、大反響を呼んだが、毎号10ページ、トータルで100本近くの関連ツアーを掲載したことからも、いかに多くの企画 ・ イベントがねるとんに便乗して行われていたか、また人気のある企画であったかがうかがえる。 余談だが同番組から『ツーショット』という流行語も誕生している。
\\	1991年	若者言葉	
\\	ばあや君	ばあやくん	召使いのように女性に尽くす男性のこと。	ばあや君の「ばあや」とは年老いた召使いの女のこと。ばあや君とはばあやが仕えるように、女性に尽くす男性を意味する。それまでに流行したキープ君やアッシー君、みつぐ君を総括したものと考えてよい。ただし、ばあや君はバブルがはじけ、不景気になってから出来た言葉であったため、先の類語と比べ、それほど普及しないまま、死語となっている。	1991年	若者言葉	
\\	バッチグー	ばっちぐー	万事良好で完璧なさま及び、そういった際に発する言葉。	バッチグーとは申し分ないさま、見事なさまを表す「ばっちり」と英語の
\\	(グッド)」を崩したものから成る合成語で、万事良好なさま、物事が予定通り順調に進むさまを表す。また、そういった状態のときに発する言葉としても使われる。バッチグーは1990年代初め、当事バラドルとして人気のあった森口博子が使ったことから普及。若者のみならず、広い世代に使われたが、2007年現在、会話上ではほとんど使われない死語となっている。	1991年	合成語	
\\	ハラッサー	はらっさー	性的嫌がらせをする人のこと。	ハラッサーとは「悩ませる ・ 困らせる」という意味の英語“
\\	に「~する人」という意味にする接尾語
\\	をつけたもので、英語では「困らせる人」という意味で使われるが、日本では特にセクハラする人、つまり性的いやがらせをする人という意味で使われる。後に『モラル ・ ハラスメント』など『セクハラ』以外の
\\	ハラ』が普及する中でハラッサーもそれら嫌がらせ(虐待)をする人全般を指すようになる(モラルハラスメントをする人の場合はモラハラッサーという場合もあるが、ハラッサーとだけ言われることが多く、前後の文脈から行為の内容を察する必要がある)。	1991年	カタカナ英語	
\\	ピーチさん	ぴーちさん	コピー機を使いこなせない人のこと。	ピーチさんとは「コピー音痴」を略したものに敬称「さん」をつけたもので、コピー機など
\\	機器を使いこなせない人のことである。ただし、ピーチさんは中年管理職など目上の人を
\\	同士で嘲う際に使う言葉であり、飲み込みの悪い新入社員がいつまでも
\\	機器を使いこなせずにいたとしてもピーチさんとは言わない。現代では死語となっており、あまり使われることはない。	1991年	
\\	複業	ふくぎょう	複数の仕事(本業的な仕事)を持つこと。	「ふくぎょう」といった場合、通常「副業」という書き、本業以外に持つ仕事を意味する。複業も本業以外の仕事のことだが、副業はあくまで本業が空いた時間を利用して行うサブ的な仕事を指すのに対し、複業はどちらが本業か区別出来ないようなものをいう。1991年に中谷彰宏が『複業の達人』を出版しているが、複業という言葉が中谷氏による造語であるかは不明。	1991年	
\\	マ行女	まぎょうおんな	男性が嫌う女性のタイプを表す言葉。	マ行女とは男性が嫌う女性のタイプを表した言葉で、内容の頭文字が「ま ・ み ・ む ・ め ・ も」であることからマ行女と呼ぶ。マ行女の中身は『ま=魔女(のような女)』『み=見えっ張り』『む=無神経』『め=目立ちたがり』『も=文句ばかり』。当時、『アッシー君』『めっしー君』『みつぐ君』という俗語にあるような女性優位な状況に反感を抱く男性がいたこと、同時期に男性を評価する『サ行男』という俗語があったことから生まれた言葉である。ただし、『サ行男』に比べるとマ行女の普及率は低い。	1991年	
\\	まったり	まったり	ダラダラ ・ のんびりするさま。	まったりとは本来、まろやかでコクのある味を表現する言葉として関西エリアで江戸時代から使われていた。1980年代のグルメブーム時に関西の料理評論家や料理人が
\\	番組の中で使ったことから全国的に普及。また、まったりはこうした味がゆっくり口に広がるさまを表現する言葉でもあり、これが転じて1990年代に入ると何もすることがなくてダラダラ(ゆっくり ・ のんびり)するさまを表す言葉として使われるようになる(この場合、退屈 ・ つまらないといった意が含まれることもある)。現代の若者の間でまったりは味覚としてよりも、ダラダラ ・ のんびりといった意味での使用のほうが多い。	1991年	若者言葉	
\\	メッシー君	めっしーくん	女性に食事を奢る男性のこと。	メッシー君とは飯(メシ)に英語で「~する人」という意味にする接尾辞
\\	をつけ、さらに人名っぽくするために「君」をつけたものである。この場合の飯とは食事(特に外食)をさし、メッシー君で女性に食事を奢る男性を意味する。男性側が食事以上の関係を望むか否かに関係なく、女性側にとって暇なとき食事に付き合ってくれ、それを奢ってくれるだけの男性をさしてメッシー君と呼ぶ。バブル期に流行語になったメッシー君はアッシー君、ミツグ君とセットで都合のいい男性の代名詞として使われたが、バブルが崩壊し、時代が厳しくなるとともに、これらの言葉は死語となっていった。	1991年	若者言葉	
\\	ヤギ男	やぎお	育児や家事に協力的な男性のこと 。	ヤギ男とは温厚で人に優しく、育児や家事も積極的に協力する男性を意味する。日経ウーマンの読者調査で仕事をバリバリこなす従来の男性(=トラ男)より、ヤギ男のほうが恋人としても、結婚相手としても好まれていると報じられ、話題となった。特にキャリアウーマンの結婚相手として最適とされているが、逆に出世欲、向上心が足りないとする見方もある。	1991年	
\\	ランチ難民	らんちなんみん	オフィス街でランチタイムに食事にありつけない状態のこと。	オフィス街ではランチタイム(主に12~13時)、一斉に大勢の人が飲食店に駆け込み、一時的に飲食店の数と客数の需給バランスが崩れる。ランチ難民とはこうした状況により、昼休憩の間にランチをとれなかった人や出遅れて店の行列に並ぶ人たちを指す(昼食難民ともいう)。時間帯をずらしたり、弁当を持参するといった回避策をとる人も増えている他、これを逆手にとった車による移動型の飲食店 ・ 弁当販売で成功している人もいる。	1991年	
\\	ターン	あいたーん	自分が生まれ育った故郷以外で就職すること。	
\\	ターンとは元参議院議員の小山峰男による造語で、自分が生まれ育った場所以外で就職することを意味する。Iターンはもともと、長野県が首都圏に住む学生やサラリーマンに長野で仕事することを呼びかけるキャッチフレーズとして出来た言葉であるが、後に生まれ育った場所(主に過疎化の進む市町村)での就職を指して使われるようになった。 
\\	ターンはUターンを文字った言葉で、ある場所からある場所へ(この場合なら首都圏から長野へ)、人の流れを地図に示すと一直線になることから
\\	が使われたが、この場合、『ターン』という表現は誤用になる。ただし、Iターンには、
\\	=私 ・ 
\\	=愛とし、自分にかえる、愛にかえるという意味も含まれている。	1990年	
\\	あげまん	あげまん	付き合った男性の運をあげる女性のこと。	あげまんとは付き合った男性に幸運をもたらしたり、運気をあげる女性のことである。もともと花柳界(色街)で使われていた隠語だが、1990年(平成2年)、伊丹十三監督の映画『あげまん』によって広く普及した。	1990年	
\\	イタメシ ・ イタ飯	いためし	イタリア料理のこと。	イタ飯とは「イタリアの飯」を略したもので、文字通りイタリア料理を意味する。ファッションや家具などイタリア風のデザインが世界的にブームとなっていたが、その流れの中でイタリア料理がブームに。また、ティラミスのブームも後押しする格好となった。料理を飯と表現するイタ飯が広く若者の間で普及した背景には、イタリア料理がフランス料理と違い、庶民的というイメージがあったことによるという見方もある。	1990年	略語	
\\	一姫二若三老人	いちひめにわかさんろうじん	危険なドライバーのタイプ。	1960年代、「一富士二鷹三茄子」という諺をもじり、危ないドライバーのタイプを示した言葉「一姫二虎三ダンプ」という言葉が流行った。一姫二若三老人はその平成版として出来た言葉で、1番危険なのは女性ドライバー、次に若いドライバー、老人ドライバーが危険であることを意味する。	1990年	
\\	オジタリアン	おじたりあん	中年男性特有の態度をした人を侮蔑する言葉。	オジタリアンとは中年男性を意味する『おじさん』と1986年公開のホラー映画『バタリアン』から成る合成語で、やたら態度が大きい ・ 偉そう ・ マナーがないといったおじさん特有の要素を持つ中年男性を意味し、そういった人を侮蔑する言葉である。オジタリアンはオバタリアンの派生的に生まれた言葉であり、オバタリアンに比べると普及度は低い。	1990年	合成語	
\\	オヤジギャル	おやじぎゃる	年は若いがオヤジのような行動をする女性のこと。	オヤジギャルとは週刊誌
\\	に連載されていた中尊寺ゆつ子のマンガ『スイート ・ スポット』からきた言葉で、年は若いがオヤジのような行動をする女性(ギャル)のことである(下記具体例参照)。オヤジギャルは当初、同マンガの主人公:小山田ノン(大手物産の問題
\\	に対して使われた言葉だが、後にそういった女性に対して使われるようになり、1990年には流行語大賞で新語部門 ・ 銅賞を受賞している。	1990年	コギャル語 ・ ギャル関連	
\\	開婚	かいこん	女性が条件を妥協してする結婚のこと。	開婚とは女性が『三高』という条件を妥協して行う結婚のことである。もともと三高派(三高に拘る女性)が条件を妥協する際、自分の年齢とともにズルズルとなし崩しに妥協する女性や「お金や外見が全てではない」と自身に言い訳しながら妥協する女性がいた。こうした姿を江戸幕府が行った開国に例えた言葉である。ただし、開婚はあまり普及していないマニアックな表現である。	1990年	
\\	家事てつ	かじてつ	家事手伝いのこと。	家事てつとは家事手伝いの略である。ただし、ここでいう家事手伝いとは自営業をしている親の手伝い、病に臥せっている母親や仕事で忙しくい母親の代わりに家事をするといったものでなく、裕福な家庭に育ち、仕事をする必要がない女性の花嫁修業としての家の手伝いを意味する(近年、ニートや無職といった人が体裁の悪さから自称している家事手伝いも含む)。	1990年	若者言葉	
\\	カルコール族	かるこーるぞく	就業時間に軽くアルコールを飲む人のこと。	カルコール族のカルコールとは「軽い」と「アルコール」の合成語である。つまり、カルコール族とは就業時間や就業時間内のちょっとした休憩時間に軽くアルコールを飲む人のことをいう。缶コーヒーや缶ジュースで喉を潤すように仕事中にアルコールを飲む人を嘲う言葉で、ビールに限った人の場合、特にカルビ族ともいう。どちらにしても一時的に普及した言葉で、現在はほとんど使われない死語となっている。	1990年	
\\	キャピる	きゃぴる	若い女性のようにはじける(はしゃぐ ・ 浮かれる)こと。	キャピるとは1980年代中頃に流行した明るくピチピチした若い女性を意味する「キャピキャピギャル(キャピキャピガール)」を略したものに動詞化する接尾辞「る」を付けたもので、若い女性のようにふるまったり、ミーハーぶることをいう。	1990年	若者言葉	
\\	ごけい	若者に人気のない職場の条件を表す言葉。	
\\	とは前年に流行した
\\	の派生語である。3
\\	は「危険」「汚い」「きつい」の頭文字であったが、5
\\	はこれに「暗い」「臭い」が加わったものである。5
\\	は新卒者を中心とした若者に人気のない職種(企業)の代名詞として使われたが、元となる3
\\	のようには普及していない。	1990年	若者言葉	
\\	コンサバ	こんさば	ベーシックなファッションスタイルのこと。	"コンサバとは「保守的」「控え目な」を意味する英語
\\	""を略した和製英語で、流行に左右されないベーシックなファッションのことである。 コンサバという言葉が普及した1990年には、主にロングヘアにスーツといった
\\	ファッションをイメージして使われたが、女性向けファッション雑誌でコンサバが使われる中で、ベーシックなスタイルの中にも流行を取り入れたものなど幅広いファッションスタイルに対して使われるようになる。
\\	1990年	和製英語、美容 ・ ファッション用語、略語	
\\	サ行男	さぎょうおとこ	女性にとっての男性の好みを表す言葉。	サ行男とは『三高』に続いて女性にとっての男性の好みを表す言葉で、内容の頭文字が「さ ・ し ・ す ・ せ ・ そ」であることからサ行男と呼ぶ。具体的には『さ=サラサラ髪』『し=しょうゆ顔』『す=すべすべ肌』『せ=清潔感』『そ=痩身(そうしん)』。 サ行男は『三高』に対して外見面をうたったものだが、『三高』同様にじきに死語となる。	1990年	若者言葉	
\\	さげまん	さげまん	付き合った男性の運気を下げる女性のこと。	さげまんとは付き合った男性の運気を下げる女性や、後ぐされのある、つきの悪い女陰のことである。 さげまんはもともと花柳界(色街)で使われていた隠語だが、反義語の『あげまん』が伊丹十三監督の映画『あげまん』で流行になったのを受け、さげまんという言葉も広く使われるようになった。	1990年	
\\	三高	さんこう	女性が結婚相手を選ぶ際に望む三つの条件のこと。	三高とは女性が結婚相手を選ぶときに「高収入 ・ 高学歴 ・ 高身長の三つの『高』が揃った男性がいい」としたことから使われるようになった言葉である。 三高は1990年代、主に20代の女性の間で使われたが、一時的なブームの後は影を潜めていった言葉である。 2001年に発売された橘奈緒著『女子大生、スチュワーデスになる 
\\	で「世間では今どき流行ではない三高のお坊ちゃま」というくだりがあるように、三高という基準そのものが一時的なブームだったようである。	1990年	
\\	3ナイ族	さんないぞく	団塊の世代を皮肉った言葉。	3ナイ族とは「コピー機が使えない」「ワープロが使えない」「ファックスが使えない」という3つの『ない』から成る団塊の世代を皮肉った言葉である。この言葉が普及した1990年、団塊の世代は中間管理職や会社役員という上司の立場にあったこと、また団塊の世代に多い理屈っぽく、自分の理論を押し付けようとする姿勢が若者から煙たがられた。3ナイ族はそんな若者が「偉そうにしている割に
\\	機器すら使いこなせない」という皮肉を込めて使った言葉である。	1990年	若者言葉	
\\	渋カジ	しぶかじ	1980年代後期~1990年代の若者のファッションスタイル。	渋カジとは「渋谷カジュアル」の略で、1980年代後半~90年代の若者のファッションスタイルのことである(『渋いカジュアル』の略という説もあるが、ファッション的に渋いというイメージは薄く、俗説と考えられる)。渋谷の高校生から発信された渋カジはそれまでの
\\	ブランドブームに相反し、ポロシャツ、ローファー、ジーンズといったシンプルなものであった。1990年に入り、そういった着こなしで歩く渋谷の男子高校生をメディアが渋カジという言葉とともに取り上げ全国的に普及。渋カジの特徴であった『紺ブレ(紺のブレザーの略)』という言葉も同時に普及した。	1990年	
\\	スッチー	すっちー	スチュワーデスを略した会話語。	"スッチーとはスチュワーデスの略で、旅客飛行機の客室乗務員のことである。 スッチーは作家(元長野県知事)田中康夫が広めた言葉で1980年代後半から使われるようになる。スッチーは当初、スチュワーデスに対し親しみを込めて使われた言葉だが、アルバイト乗務員の増加による質の低下やスチュワーデス自体のイメージの低下から嘲りの意を込めて使われるようにもなる。 スッチーの語源であるスチュワーデスは英語の
\\	""からきたカタカナ語で、
\\	""の女性形にあたる。ただし、英語圏では性別をわける呼び名を避ける傾向があり、スチュワーデスのことを
\\	""と呼ぶ航空会社が多い。
\\	1990年	略語	
\\	タカビー	たかびー	高飛車な人のこと。	タカビーとは『高飛車』を略したものに英語で人を表す接尾語“-
\\	をつけたもので、「高飛車な人」を意味し、主にそういった人を軽蔑したり、罵る意を込めて使われる。 後にタカビーの「ビ」を半濁音にした『タカピー』という言い回しも普及している。	1990年	若者言葉、略語	
\\	タカピー	たかぴー	高飛車な人のこと。	タカピーとは『高飛車』と
\\	から成る合成語で、「高飛車な人」を意味し、そういった人を軽蔑したり、罵る意を込めて使われることが多い。 もともとは『高飛車』を略し、人化する英接尾語“-
\\	をつけた『タカビー(ピの部分が濁音)』が使われていたが、直にタカピー(半濁音)も普及。現在は同程度に使われている(タカピー(半濁音)がタカビー(濁音)の聞き違いによって生まれた語なのか、
\\	との合成として、同時期に別の出所から普及した言葉なのかは不明)。	1990年	若者言葉	
\\	チバリーヒルズ	ちばりーひるず	千葉にある高級住宅街:ワンハンドレッドヒルズの俗称。	チバリーヒルズとは千葉県千葉市緑区あすみが丘の一角に開発された高級住宅街『ワンハンドレッドヒルズ』の俗称で、アメリカのカリフォルニア州ロサンゼルスにある世界的に有名な高級住宅街ビバリーヒルズと千葉(チバ)をかけたものである。1989年に分譲開始。一戸5億~十数億円もする豪邸はメディアでも多く取り上げられ、チバリーヒルズという呼び名も普及した。	1990年	和製英語	
\\	ドメ男	どめお	海外勤務経験のない男性。家庭内で暴力を振るう男性のこと。	ドメ男とは「国内の」という意味の英語「ドメスティック
\\	の略に「男」を付け、男性名っぽくした言葉である(ダメ男との掛詞でもある)。当初ドメ男は国内勤務だけで海外勤務をしたことが男性という意味で、1990年代の
\\	がそういった男性社員を嘲う言葉として使用。ここから留学をしたことがない男性や海外旅行をしたことがない男性など、海外慣れしていない男性をドメ男という。 後にドメスティックは先述の意味よりも「家庭内の」という意味、特にドメスティック ・ バイオレンス(家庭内暴力)に使われる言葉として定着。バブルがはじけ、死語となっていたドメ男も家庭内で暴力を振るう夫という別の意味合いで使われるようになる。	1990年	
\\	成田離婚	なりたりこん	新婚旅行中や旅行前のトラブルが原因でする離婚のこと。	成田離婚とは新婚旅行や旅行前の空港でのトラブルが原因でする離婚のことである。お見合いで結婚し、新婚旅行で相手の意外な一面を知り離婚に至るというケースは想像出来ると思うが、むしろ以下のような原因が話題となり、成田離婚という言葉は浸透した。結婚前、金銭面 ・ 時間面で自由がきき海外旅行慣れした新婦、仕事に追われ新婚旅行が初の海外という新郎。そんな二人が無事挙式を済ませ新婚旅行に向かったはいいが、海外旅行慣れしていない新郎の出航前の段取りの悪さや海外でのあたふたした様子に愛想をつかし、新婦が帰国後離婚を切り出すといったものである(成田離婚が話題になった当時は愛想をつかして一人先に帰国した新婦や出航前の空港で離婚を決意して旅行を取り止めた新婦が取り上げられたりもした)。また、成田離婚回避を目的とした新郎向けに、タクシーの乗り方からレストランのオーダーまで体験出来る「新婚旅行下見ツアー」というパッケージが企画 ・ 実施され話題になった。 1997年10~12月には草薙剛、瀬戸朝香主演で『成田離婚』(フジテレビ)という連ドラもあったが、言葉としてはこれが最後のピークで、以降『スピード離婚』という言葉にまとめられる形で死語化している。	1990年	
\\	バックレ会議	ばっくれかいぎ	犯罪を犯した少年による証拠隠滅行為のこと。	バックレ会議の『バックレ』とはしらばっくれる(とぼける)ことで、バックレ会議とは犯罪を犯した少年が証拠隠滅のために行うミーティングのこと。ミーティングで打ち合わせた内容に従い、警察での取調べで口裏を合わせてしらばっくれることからバックレ会議と呼ぶ。主に暴走族など集団非行の仲間内で使用。	1990年	若者言葉	
\\	バブル	ばぶる	
\\	泡 ・ 気泡のこと。 
\\	実体経済とはかけ離れた相場や景気のこと。	"バブルには大きくわけて下記の二つの意味があります。 
\\	バブルとは英語の
\\	""のことで、泡や気泡を意味するカタカナ語。また、そこから派生して「実態のない見せかけのもの」といった意味ももつ。 
\\	バブルとは投機によって高騰した資産価格が支える経済で、実態経済とはかけ離れた相場や景気のことを意味する。 主に1980年代後半のバブル景気をさすが、他にもインターネット ・ バブルといったものがある。 バブル景気が命名された1987年は一般にはあまり浸透せず、むしろ、バブルが崩壊した1990年代、まさに膨らみきった泡がはじけるように経済が破綻したこから『バブル崩壊』という言葉でメディアを中心に一般にも浸透していった。
\\	1990年	カタカナ英語	
\\	バラドル	ばらどる	バラエティアイドルの略。	バラドルとはバラエティアイドルの略で、バラエティをメインに活動しているアイドルを指す。昭和時代のアイドルは山口百恵、桜田淳子から次世代の松田聖子、中森明菜、小泉今日子に代表されるように歌番組やドラマをメインにしており、バラエティは親近感を出すためのイメージ戦略としてサブ的な位置にあった(バラエティに出演する場合も多くは間で曲を歌ったりしていた)。そこへ山瀬まみ、森口博子、井森美幸といった歌や芝居よりも面白い語り口を武器とし、バラエティをメインに活動するアイドルが登場。今でこそこういった売り出し方をするアイドルも珍しくなくなったが、当時は独特な位置にあったことからバラドルという言葉でまとめられていた。しかし、こういった切り口のアイドルは珍しくなくなってからはバラドルという言葉も廃っていく。その要因の一説として、浜崎あゆみのような歌専門、深田恭子のようなドラマ専門といった立ち位置をハッキリさせる傾向からアイドルという漠然とした存在がいなくなったこと、番組側もそういった人をメインに扱うようになり、単にグラビア的に人気のあるアイドルはバラエティやクイズ番組しか枠がなくなったからとも言われている。	1990年	略語	
\\	ビンバー ・ ビンベスト	びんばー ・ びんべすと	貧乏を強調した言葉。	ビンバーとは『貧乏』に英語で形容詞につけて比較級にする接尾語
\\	をつけたものである。同様にビンベストは『貧乏』に英語で形容詞を最上級にする接尾語
\\	をつけたもの。つまり、貧乏の活用形がビンボー
\\	ビンバー
\\	ビンベスト(「お金がある
\\	お金がない」という関係)ということである。また、ビンボーは貧乏の初期段階、ビンバーは貧乏の第二段階、ビンベストは完全な貧乏という使い分けもされるが、特にどういう状況がどの段階といった定義はない。	1990年	若者言葉 ・ 和製英語	
\\	ファジー	ふぁじー	「あいまいな」という意味の形容動詞。	ファジーとは英語の“
\\	からきたカタカナ英語で、英語同様「あいまいな」「ぼやけた」といった意味で使われる。1990年代に入り、人間の思考や行動にある曖昧さをを取り入れた家電製品が登場。こうした製品が話題になる中でファジーという言葉も流行した。1991年には悪くなりそうでなかなか落ち込まない経済状況を意味するファジー不況という言葉も使われた。ただし、こちらは現在あまり使われない死語となっている。	1990年	カタカナ英語	
\\	別腹	べつばら	お腹いっぱいでも好きな物は食べられること。	別腹とはお腹いっぱいの状態でも好きな食べ物は食べられることを意味する言葉である。女性が甘い物を対象に使うことが多く、食後に「ケーキは別腹だから大丈夫」といった形で使う(他には飲酒の後のラーメンなど)。もちろん、実際に好きな食べ物が入る別の腹(胃)や臓器があるわけではなく、好きなものであれば別腹があるかのように食べられるという例えである(最近の調査では美味しいものを見たとき(正確には血糖値が下がったとき)に出るオレキシンという分泌液が胃を緩め、先に食べたものを小腸へ送り出すことで胃にスペースを作るといわれている)。	1990年	
\\	マック	まっく	
\\	ファーストフードチェーン:マクドナルドのこと。 
\\	社のパソコン:マッキントッシュのこと。	
\\	マックとはハンバーガーを主力商品とするファーストフードチェーン『マクドナルド』の俗称で、関西圏ではマクドという呼び方で親しまれている。1980年代には既にマックやマクドと呼ばれていたが、全都道府県に出店が完了した1990年辺りから「関西でマクドナルドはマックではなくマクドだ」といった発言がメディアで見られるようにり浸透。ちなみ朝食用メニューの朝マックやマックシェイクといった企業側がマックと謳っているものは関西圏でもマック(朝マック、マックシェイク)で呼ばれる。また、「マクドナルドで食事をする」ことを『マックする』といった使い方もされる。 
\\	マックとはアップル社が1984年から販売しているパソコン ・ シリーズ「マッキントッシュ
\\	の俗称で、カタカナ表記のマック以外に英語表記の
\\	が使われる。こちらの意味では関東 ・ 関西ともにマックで統一されているため、関西圏でマックといった場合はマッキントッシュをさして使われる。	1990年	固有名詞の俗称	
\\	貢ぐ君	みつぐくん	好きな女性にお金や品物をどんどん貢ぐ男性のこと。	貢ぐ君とは好きな女性に金品や品物などをせっせと貢ぎ、女性の気を引こうとする男性のこと(食事をせっせと奢るメッシー君は貢ぐ君の一種ともいえる)。貢ぐ(みつぐ)だけで人名っぽく呼ぶこともある。貢がれている女性が特に深い仲ではなく、単に貢ぐ君としか考えていないという意志を示すなど、対象を嘲ったり、蔑んで使われることが多い。どちらにしてもバブルの産物的な言葉(存在)であり、バブル崩壊以降は死語となっている。	1990年	若者言葉	
\\	メダカ馬鹿	めだかばか	手の施しようのない愚かなことをした人を嘲う言葉。	メダカ馬鹿とは細かくて通常のたも網では救えないメダカのような馬鹿。つまり、メダカ馬鹿は救いようのない馬鹿という意味で使われる。例えば、想像を超えるような失敗 ・ 失態 ・ 失言をした人、救いようのない(手の施しようのない)愚かなことをした人を嘲う際に用いる。1980年代始めにメダカブスという言葉が使われたが、メダカ馬鹿がその派生なのか、それとは別に自然発生した言葉なのかは不明。	1990年	若者言葉	
\\	ヤンエグ	やんえぐ	青年実業家のこと。	ヤンエグとは「ヤングエグゼクティブ」の略である。ヤングエグゼクティブ=
\\	(若い)+
\\	(管理職)。つまり、ヤンエグとは若くして管理職に就いた人や自らの会社を立ち上げた青年実業家を意味する。また、広義には単に「仕事の出来る若者」という意味でも使われる。ヤンエグは平成初期に流行した言葉だが、次第に死語になりつつある。	1990年	
\\	アッシー	あっしー	女性が移動手段としている男性のこと。	アッシーとは足(あし)に英語で「~する人、される人」という意味にする接尾語
\\	をつけたものである。足には動物の下肢以外に「足がない」「事故で足止めされた」といった表現で車やタクシー、電車といった移動手段(交通機関)を指すことがある。つまり、アッシーとは女性が足(移動手段)として考えている男性を意味する。アッシーと女性の間には恋愛関係はおろか、友情関係すらなく、仮に友達としての付き合いがある場合でも、足代わりという意識のほうが強い場合にしかアッシーとは呼ばない。主にバブル期の女性が使っていた言葉でアッシー君ともいう。	1989年	若者言葉	
\\	アリギリス	ありぎりす	働くばかりでも遊ぶばかりでもなく、仕事も遊びもする人のこと。	アリギリスとは「アリとキリギリス」というイソップ寓話に出てくるアリとキリギリスを合成した言葉である。物語でアリは冬の食料に困らないよう、ひたすら働き続け、キリギリスは夏場歌を歌って遊び、冬に入って食料に困り、アリのところへ助けを乞いに行く。アリギリスはそんな働くだけのアリでも、遊び呆けてばかりのキリギリスでもなく、仕事も遊びも一所懸命するスタイル及びそういったスタイルの人を意味する。また、アリギリスの究極のスタイルとして、仕事が遊び ・ 遊びが仕事といった姿勢 ・ 考え方を挙げる人もいる。	1989年	
\\	アンマリ族	あんまりぞく	結婚せず、優雅な独身生活を楽しむ人のこと。	アンマリ族のアンマリとはアンマリッド
\\	の略で、未婚・独身といった意味だが、ここでは非婚といったニュアンスで使われる。つまり、アンマリ族とは結婚をせず、独身生活を謳歌する人のことである。結婚による時間的制約や金銭的制約を受けることのない自由なライフスタイル。こういったライフスタイルの女性を意味する。このため、本人に結婚願望はあるが、相手がいないため独身でいる人をアンマリ族とは言わない。	1989年	
\\	オタク	おたく	特定の分野に関して強い興味を抱き、関連するものを収集したり、詳しく知ることに時間を費やす人のこと。	オタクはアニメ愛好家が互いを二人称の「オタク」と呼びあうことから、コラムニスト中森昭夫が『漫画ブリッコ』の中でおたく族と名付けたことに由来する造語で、正しくはおたくと平仮名表記する。当時はほとんど普及しなかったが平成に入り、アニメ ・ マニアが仲間内で使用。当初は家(お宅)にこもって特定な趣味分野(アニメや漫画などメディアから発信されたサブカルチャーが対象の主)を追求する様をおたくと呼んだ。ただし、近年のオタクはイベントに出かけるなど家にこもっての活動とは限らず、当初のおたくとは意味合いを画する。また、2ちゃんねるを中心としたネットワークではヲタクという表記が使われる。ヲタクはヲタと略して使われることが多く、
\\	分野のオタク』を
\\	ヲタ』、『オタクな学生』を『ヲタ学生』などバラエティーに富んだ使い方がある。更にオタクであることを隠して生活する人を意味する『隠れオタク』という言葉もある。 『オタク』といえば・・・: オタクといえばネガティブなイメージが強かったのですが、そんなオタクの「興味を持ったジャンルには財を惜しまず出費する収集欲」に着目する企業も多く、『オタク産業』という言葉も生まれている。2004年8月に株式会社野村総合研究所がオタク市場規模は2,900億円であると発表。これは当時のデジカメ市場を上回っている。それ以降も更に延び続け、2005年10月6日には同じく株式会社野村総合研究所からオタク市場規模は4,110億円であると発表されている。 今後も更に延びが期待される市場として注目されている。	1989年	若者言葉	
\\	オタッキー	おたっきー	オタクっぽい人のこと。	オタッキーとはゲームやパソコン、マンガを読んで家にばかりいる人、異常なほどにひとつのことを追求したり収集したりする人『オタク』に人を表す英接尾語
\\	をつけたもので、オタクっぽい人のことである。オタッキーは当初、オタクの中でも
\\	ブランドに身を包んだりといったファッションにも気を使う、女性にモテるオタクという意味で、漫画家
\\	みうらじゅんが1989年に作った造語である。普及する過程で単にオタクと同意に使われるようになるが、オタッキーのほうが明るい、かわいいといったイメージを伴うことが多い。1990年にはオタクよりもオタッキーという呼び方のほうが流行した。	1989年	若者言葉	
\\	お局	おつぼね	職場を仕切る古参
\\	のこと。	おつぼねとは職場を仕切る古参の女性社員を意味し、お局と書く。おつぼねは「意地悪」「口うるさい」といった意を含み、そういった
\\	を嘲ったり、侮蔑する際に使われることが多く、尊敬する先輩
\\	に使われることはほとんどない。また「おつぼね=行き後れ(婚期を過ぎても結婚出来ずにいる女性)」というイメージも強く、多くは独身の古参
\\	に使われる。 お局という言葉自体は、宮中や江戸時代の大奥などで「局(仕切りで隔てた個室)を与えられた女官(女中)」として古くから使われている。ただし、当時のお局(御局)は指導的立場にあったものの、現在のように悪い意味はなく、単に階級的なものであった。おつぼねが悪いイメージを伴い、現代の古参
\\	に対して使われるようになったのは、1989年に放送された大原麗子出演の
\\	大河ドラマ『春日局』がキッカケで、同年は流行語的に使われた。また、おつぼねさまと丁寧な言い回しも使われるが、意味自体に差異はなく、かえって皮肉や嫌みが込められた感もある。	1989年	
\\	オバタリアン	おばたりあん	おばさん特有の図々しさ、羞恥心のなさを持つ中年女性。	オバタリアンとは中年女性を意味する『おばさん(おばはん)』と1986年公開のホラー映画『バタリアン』から成る合成語で、羞恥心がない ・ 図々しい ・ 無神経といったおばさん特有の要素を持つ中年女性を意味する。オバタリアンは漫画家“堀田かつひこ”による造語であると同時に堀田氏が1986年から“まんがライフ”で連載した4コマ漫画のタイトルでもある。1989年には土井たか子(当時社会党)が使い、話題となり、同年、流行語大賞 ・ 流行語部門の金賞を受賞している(受賞者は土井たか子と堀田かつひこ)。	1989年	合成語	
\\	帰宅子女	きたくしじょ	離婚して実家に戻ってきた女性のこと。	親の仕事により、長年海外で生活して日本に戻った女子(帰国時期が小学校~大学入学前後の者)を帰国子女という。帰宅子女はこのもじりで、結婚して一旦実家を出たが、後に離婚をし、再び実家に戻ってきた女性を意味する。それまでこういった女性を出戻り娘と呼んだが、これに比べて語感が明るいことから帰宅子女が普及。ただし、現在はあまり使われない死語になっている。	1989年	
\\	ゴールドカラー	ごーるどからー	若手経営者 ・ 青年実業家のこと。	ゴールドカラーとはサラリーマンのタイプを表すホワイトカラーやブルーカラーの派生語で、アメリカの経営コンサルタント及び経営学者であるロバート ・ 
\\	・ ケリーによる造語である。ただし、ロバートが提起したゴールドカラーと日本で普及したそれとは意味が異なっている。本来はコンサルタント的や経営者的な能力 ・ 知識を持ち、会社の規則に縛られず、独自の見解やライフスタイルで活動する新しい種類のサラリーマンをさす。これが日本ではビジネスシーンにおける新人類として紹介され、軽いノリで仕事に接する有能なサラリーマンというイメージで定着。最終的にヤンエグ(青年実業家)や創業者2代目の若手経営者に対する称号的言葉となる。どちらにしても現代では経営研究家など一部の人しか使わない言葉となっている。	1989年	
\\	困ったちゃん	こまったちゃん	言動が他者を困らせる人。処置に困る人のこと。	困ったちゃんとは仕事でミスばかりする人や周りの人の気分を害するような言動を平気でする人など、他者を困らせる人、周りが処置に困るような言動をする人、無神経な人のことで、こういった困った人のことを人名的に呼んだものである。1980年代後半に社会人の間で普及。会社の中の困ったちゃんに対して好んで使われ、後に学生など世代を超えて広く使われるようになった。	1989年	
\\	サクセススーツ	さくせすすーつ	学生が就職活動するためのスーツのこと。	サクセススーツとは就職活動をする学生が会社訪問や面接のために着るスーツのことである。面接官に個性よりも安心感を与えることが目的とされるサクセススーツは、紺またはグレーの無地といった色にオーソドックスなデザインが特徴である。それまではこういったスーツをリクルートスーツと呼んだが、1989年のリクルート事件(汚職事件)によるイメージの悪さから、デパートがサクセススーツに変名。ただし、リクルート事件のほとぼりが冷めてくると、再びリクルートスーツと呼ぶようになり、サクセススーツは死語となっていった。	1989年	
\\	さんけー	「危険」「汚い」「きつい」職場のこと。	
\\	とは「危険(きけん)」「汚い(きたない)」「きつい」の3語の頭文字からきており、こういった嫌われる職場 ・ 職種に対して使われる。また、このような職場や職種を、それぞれ
\\	職場』
\\	職種』ともいう。 
\\	は1980年代末期から普及し、1989年には流行語としてノミネートされている。 また、3
\\	は上記とは別に住宅の間取り(キッチンと部屋で3部屋あるもの)や国際単位の3キロ
\\	など)としても使われる。	1989年	
\\	しばく	しばく	
\\	殴る ・ 蹴るなど暴力をふるうこと。 
\\	ある場所へ(食べに ・ 飲みに ・ 遊びに)行くこと。	
\\	しばくとは鞭や細い棒で強く叩くことを意味するが、関西エリアではこういった道具の使用に関係なく、殴る ・ 蹴るなどの暴力をふるうという意味で、主に喧嘩の前の脅し文句として使われた。
\\	も含め現在は死語。 
\\	しばくとは「~へ行く(~する)」という意味の若者言葉で、関西で使われていたものが上方芸人によって全国的に広まったものである。馴染み深い使い方として「茶しばく=お茶を飲みに行く」、他に「牛しばく=吉野家へ行く」「ネズミしばく=ディズニーランドへ行く」といった形で使われる。	1989年	不良言葉 ・ ヤクザ用語、若者言葉	
\\	チクリ	ちくり	
\\	密告することや密告者のこと。 
\\	針状のものを刺すさまをあらわす言葉。	
\\	チクリとは告げ口や密告するという行為を意味する名詞である。また、ここから告げ口をする人(密告者)という意味でも使われる。チクリは1970年代末から使われた『チクる』から派生したもので、1980年代末から不良少年を中心に若者の間で普及。他愛無いこともいちいち告げ口するような人を『チクリ魔』ともいう。 
\\	チクリとは注射やハチなど針状のものが肌を刺すさまを表す擬態語である。ここから針を刺すような刺激的な物言いも表す(例:ちくりと言う)	1989年	
\\	チャッカマン	ちゃっかまん	抜け目のない人のこと。	チャッカマンとは抜け目ないさまを意味する「ちゃっかり」と人という意味の英語“
\\	。から成る合成語で、抜け目のない、ちゃっかりした人を意味する。ちなみにチャッカマンとは本来、株式会社東海から出ている柄の長いライターの商品名だが、今回のチャッカマンはこれに掛けた掛詞である。ただし、現在この意味では死語となっており、ほとんど使われない。	1989年	
\\	ツーショット	つーしょっと	男女が二人きりになること。	ツーショットとは男女が二人きりになることを意味することが多い。しかし本来は、二人の出演者をひとつの画面に入れ、クローズアップさせることを意味する映画
\\	の撮影用語である。1987~94年に放送されたフジテレビの『ねるとん紅鯨団』という集団見合(出会い)番組で、男女二人が他の人から離れて行動を始めると、その二人を追うようにしてカメラをクローズアップ。このとき、司会のとんねるずが『ツーショットだー』と言ったことが、男女が二人きりになることを意味するようになったきっかけである。 同じ時期に流行した、
\\	2ダイヤルの男女二人で話せる番組回線をツーショット ・ ダイヤル、更にインターネット時代に入り、二人きりでやりとりが出来るチャットをツーショット ・ チャットと呼んで定着した。	1989年	若者言葉	
\\	月の輪おやじ	つきのわおやじ	
\\	シャツの下のランニングシャツの線が見えている男性。	月の輪おやじとはワイシャツの中に着たランニングシャツの線が見えている男性のこと。このランニングシャツの透けて見えるラインを月の輪熊(ツキノワグマ)の胸元にある三日月型の白い胸毛に例えて出来た言葉である。主に
\\	が中年上司を嘲う際に使用。	1989年	
\\	つくしん坊	つくしんぼう	女性に尽くす男性のこと。	つくしんぼうとは春先に生える植物「ツクシ(スギナ:トクサ科)」の別称『土筆んぼう(つくしんぼう)』と『尽くす』をかけた言葉で、女性にプレゼントをしたり、足代わりとして車を出すなど、女性に尽くす男性を意味する。ただし、彼女 ・ 妻といった関係の女性に尽くす男性に使われることは少なく、そういった関係にない女性の関心をひくために尽くす男性が主な対象となる。 なお、つくしんぼうは1980年代末に普及。現在はほとんど使われることのない死語である。	1989年	若者言葉	
\\	天安門する	てんあんもんする	意見が通らなかった際、ストライキなど実力行使に出ること。	天安門するとは1989年6月4日、中華人民共和国の北京市にある天安門広場で行われた六四天安門事件の「天安門」に、動詞化する「~する」を付けたものである。天安門事件は民主化を求める学生や市民のデモ活動(抗議運動)からきたもの。それを動詞化した天安門するは、意見や主張が取り入れてもらえなかったとき、ストライキなどの実力行使をするという意味で使われた。現在はほとんど使われない死語となっている。	1989年	
\\	濡れ落ち葉	ぬれおちば	定年退職後の夫のこと。	濡れ落ち葉とは定年退職後の夫のことで、仕事人間だった夫が家では邪魔な存在であることを表現した言葉である。濡れ落ち葉という比喩が出てくるまでは『粗大ゴミ』などと表現されていたが、それに比べると若干キレイな(遠まわしな)言い方なので好まれているようである。 定年後の夫がなぜ濡れ落ち葉なのか?これは妻が出かけようとすると「わしもついて行く」と言い、どこにでもついてくる様が払っても払ってもなかなか落ちない濡れた落ち葉に似ているからで、ある主婦がしたこの発言を評論家の樋口恵子が聞き紹介。それが広がる中で濡れ落ち葉という言葉で定着した。	1989年	
\\	バサロ恋愛	ばさろれんあい	人にバレないように隠れてする恋愛のこと。	バサロ恋愛とは水泳の泳ぎ方のひとつ「バサロ泳法」からきた言葉である。バサロ泳法とは1970年代にアメリカのジェシー ・ バサロが考案した泳法で、背泳ぎの際、両手を伸ばし、足はバタフライのドルフィンキックをしながら潜行する。この潜伏して進む姿から人に知られないように隠れてする恋愛のことをバサロ恋愛と呼ぶ。1988年のソウルオリンピックで鈴木大地がバサロ泳法で金メダルを獲得し話題となったことからバサロ恋愛という言葉も生まれたが、現在ではほとんど聞かれない死語となっている。	1989年	
\\	光り物	ひかりもの	宝石などアクセサリーのこと。	光り物とはもともと人魂や流星など、発光体をさす言葉として平安時代の書物には使われている言葉である。後に金属や鯖(さば)や鰯(いわし)など青く光る青魚と呼ばれるものも光り物と呼ばれた。そして1980年代後半のバブル期には、女性
\\	がもらって喜ぶアクセサリーを光り物と呼ぶようになる。ただし、バブル期に使われるようになったことからもわかる通り、光り物は豪華な宝石がついたものなど、暗に高価なアクセサリーという意味で使われることが多い。ただし、バブル崩壊後はラメ入りのグッズや発光機能のあるオモチャ、プラモデルなど、アクセサリー以外を対象とした使用率が増えている。	1989年	
\\	ファミレス	ふぁみれす	ファミリーレストランの略。	ファミレスとは外食産業の形態のひとつで、家族向けに想定されたメニューや店舗作りがなされているファミリーレストランの略である。ファミレスの特徴は、まず多様なニーズに応えられる豊富なメニュー。特にファミレスということから、子供向けメニューの充実など、家族連れで来ても満足出来る品揃えをリーズナブルな価格帯で揃えているところが多い。他にも広い通路や駐車場など、ファミリー向けの対応が様々に施されている。また、ファミレスと呼ばれる外食チェーンは多数あるが、それぞれが独自の方針や特徴を打ち出している。 こうした形態の店は1970年代から日本でも見られたが、ファミレスと略して呼ばれるようになったのは1980年代末からである。	1989年	略語	
\\	フリーター	ふりーたー	非正社員雇用形態で生計を成り立てている人。	"フリーターとはアルバイトやパートタイマー、派遣社員など、正社員以外の雇用形態による収入(主に時給や日給)で生計をたてている人をいう。 フリーターは1987年にリクルート社のアルバイト情報誌『フロム ・ エー』編集長の道下裕史が作った造語で、好景気であった1980年代後半、組織に縛られないという形態が脚光を浴び、フリーターという言葉も定着していった。ちなみにフリーターは英語のフリーと、ドイツ語のアルバイト(働くの意)、更に
\\	する人」の意味を持つ英接尾語
\\	""を合成させたフリー ・ アルバイター(自由に働く人)の略である。
\\	1989年	
\\	ホタル族	ほたるぞく	ベランダでタバコを吸う人のこと。	ホタル族とはベランダでタバコを吸う人のことで、夜間、ベランダで吸うタバコの火種がホタルに見えることからこう呼ぶようになった。1980年の「嫌煙権訴訟」に始まり、飛行機 ・ 国鉄(現
\\	車両での禁煙など、嫌煙権が認められつつある中で、ホタル族は嫌煙権確立の象徴的な言葉となる。なお、当時ホタル族をする(させられる)理由としては副流煙が体に悪いといったことよりも、壁がヤニで汚れたり、衣類に臭いがつくからといったものが多かった。また、ホタル族は父親の威厳の失墜という意味でも当時話題となった。	1989年	
\\	マスオさん	ますおさん	妻の実家で結婚生活をする人のこと。	マスオさんとは人気アニメ『サザエさん』でサザエさんの夫であるフグ田マスオのことだが、マスオさんがサザエさんの両親とサザエさんの実家で同居していることから、妻の実家で結婚生活をする人、さらに広義には婿養子のことをマスオさんと呼ぶようになる(アニメでマスオさんはフグ田姓のままであり、婿養子ではない)。特に1989年にはこうした男性が増えたことからマスオさん現象という言葉が流行となった。	1989年	
\\	焼きを入れる	やきをいれる	
\\	人に制裁をくわえること。 
\\	日焼けすること。	
\\	焼きを入れるとは人に活を入れることで、鍛冶で刃物を焼いて鍛えることからきている。これが転じ、制裁をくわえるという意味でも焼きを入れると言うようになる。この意味では主に不良が使用。意に沿わない相手に多勢でくわえる制裁が多い。不良 ・ ヤクザ映画のセリフに見られる他、昭和のツッパリブームでは若者を中心に広くこの意味で普及し、定着している。 
\\	焼きを入れるとは肌に焼き色を入れる。つまり、日焼けすることを意味する。バブル期の
\\	が用いた表現で、当時人より早く焼きを入れるために海外旅行をする者まであった。後にギャル、コギャルへ引き継がれた言い回しである。	1989年	不良 ・ ヤクザ用語	
\\	リゾラバ	りぞらば	女性が海外リゾート地で男性とその場限りの恋愛をすること。	リゾラバとは
\\	、リゾート ・ ラバーズ(和製英語)」を略したもので、女性が海外リゾート地(ハワイ、グアムなど)で外人男性(現地男性)とするその場限りの恋愛を意味する。こうした行為をするには、よほどの魅力を備えるか、金品による合意を取り付ける必要がある。ここからリゾラバ=海外で男性を買うという意味で使われることが多い。1989年には爆風スランプがシングル『リゾ ・ ラバ 
\\	をヒットさせ、広く知れる言葉となる。ただし、リゾラバはバブル時代を象徴する行為~言葉であり、現在は死語となっている。	1989年	
\\	あきすとぜねこ	あきすとぜねこ	恋占いの一種。	あきすとぜねことは主に昭和時代の少女が好んでやった名前を数字化して答えを導く恋占いのことである。その占い結果が「愛してる」「嫌い」「好き」「友達」「絶好」「熱烈」「恋人」のいずれかであったため、それぞれの頭文字をとってあきすとぜねこと呼ばれた。また、占い結果の「恋人」が「結婚」であった地域もあり、この地域ではあきすとぜねけと呼ばれていた。以前からこの相性占い自体は存在したが、1988年、光
\\	がビデオ、書籍などにあきすとぜねこを使ったコンセプトで販促。これにより女学生の間で普及。2001年には
\\	「みんなのうた」で『アキストゼネコ』という曲が発表。内容はあきすとぜねこを交えたラブソングだが、ここでは「き」は「嫌い」ではなく「キス」になっていた。	1988年	
\\	イタ電	いたでん	イタズラ電話の略。	イタ電とは「いたずら電話」の略でいた電とも書く。イタ電には様々な種類があるが、代表的な3つを紹介。 【無言電話】 受話器をとり応答しても何も返答せず、無言のままでいるもの 【ガチャ切り】 受話器をとった瞬間に電話を切るものや応答した瞬間に切るもの 註)携帯普及後に増えた、受話器をとる前(通常は1~数回のコールで)切ってしまう『ワン切り』とは異なります。 【エロ電】 いきなり「ハアハア」という卑猥な声(エロ声)を出すものや下着の色、格好などを尋ねるもの、他にも自分の陰部の状態を説明しだすものなど猥褻な内容のもので、主に女性をターゲットとしたイタ電です。 イタ電は特定の人に対する嫌がらせのみでなく、ランダムにダイヤルを回し、不特定の相手にする場合もあります。イタ電だと思う相手であったら電話を切るか放置しておく(イタ電をする側に通話料金が発生するので効果があります)のがよいでしょう。また同じ内容のイタ電が何度もかかってくる場合は警察に通報するのもひとつの手段です。内容が脅迫じみたものやエロ電の場合、録音機能のある電話機をお使いでしたら内容を録音しておきましょう。 ※イタ電は刑法の威力業務妨害にもあたる犯罪行為です。	1988年	略語(若者言葉)	
\\	億ション	おくしょん	一戸が1億円以上する高級マンションのこと。	億ションとは一戸が一億円以上で売買される分譲マンションのこと。バブル経済期、この手のマンションがよく売られたことから生まれた言葉で、通常のマンションを万ションとし、それより高価な物件として億ションと呼んだ。	1988年	合成語(略語?)	
\\	カイワレ族	かいわれぞく	管理社会の中でしか生きられない中 ・ 高生のこと。	カイワレ族とは医師であり文筆家でもある村崎芙蓉子の造語で『カイワレ族の偏差値日記』という著書のタイトルにも使われている。同書で村崎氏は当時の中学生 ・ 高校生を、土ではなく、ウレタンの苗床で完全管理の下に育てられるカイワレに例えた。また、カイワレ族である息子を「都会っ子のためドングリではないが、モヤシっ子ほどひ弱でもない」とし、「ドングリの背比べ」をもじった「カイワレの背比べ」という表現も使っている。つまり、管理社会でしか生きられない成績が普通の中学生 ・ 高校生を意味している。なお、カイワレ族は1988年流行語大賞で新語部門 ・ 表現賞に入賞している(受賞者は村崎芙蓉子)。	1988年	
\\	ケチャップ顔	ケチャップがお	造りが濃くて親しみのある顔のこと。	ケチャップ顔とはソース顔より濃く、更に親しみのわく顔をさす。また、濃い顔の中で少年っぽさの残るタイプもケチャップ顔に分類される。少年隊の東山紀之がしょうゆ顔の代表、同じく少年隊の錦織一清がソース顔の代表と呼ばれたのを受け、残るメンバーの植草克秀に対してつけられたのがケチャップ顔である。こうした経緯から後付けされたケチャップ顔の定義は感じ方に依る部分が大きく、曖昧であったためか、しょうゆ顔・ソース顔に比べ、早い段階から使われなくなっている。 ちなみに
\\	の国分太一、
\\	の堂本剛がケチャップ顔の例として引き合いに出される。	1988年	若者言葉	
\\	合コン	ごうこん	合同コンパの略。	合コンとは「合同コンパ」の略で、本来は二つ以上のグループが合同でするコンパを意味するが、多くは男性と女性のグループが交流を持つ会のことをさす。合コンの多くは男女それぞれ同数で行われる。仕事やサークル、旧友といった異性の知人(友人)を発端にし、双方が人数を調整して行うものが多い。合コンはあくまで男女が知り合うためのキッカケであり、その後の個人的交際を目的にして行われる(中にはその場の飲み会だけを楽しむ目的のものもある)。その目的を達するには合コン中の話術などテクニックを要し、その方法を伝授する書籍も多い。 以前は合同コンとも略されたが、現在、ほとんど使われていない。	1988年	
\\	5時から男	ごじからおとこ	終業時間が来ると元気になるサラリーマンのこと。	5時から男とはタレント:高田純次が出演した栄養ドリンク『グロンサン』の
\\	から生まれた言葉で、終業時間である5時になると元気になるサラリーマン、つまり仕事よりも遊びになるとイキイキするサラリーマンを意味する(当時、企業の多くが8時~17時を就業時間の定時としていた為)。
\\	の内容は仕事に疲れてダラダラしていたサラリーマンが終業時刻になるとグロンサンを飲んで元気になり、疲れ知らずで夜の街を遊び回るというものだった。
\\	は極端にしても、仕事を終え、遊びになると元気になるサラリーマンは多く、これらを5時から男と呼ぶようになる。5時から男は1988年に流行語大賞 ・ 大衆賞を受賞している(受賞者は高田純次)。	1988年	
\\	コタツムリ	こたつむり	日本版カウチポテト~こたつに入ってばかりいる人のこと。	コタツムリとは暖房具の一種「こたつ」と「かたつむり」から成る合成語で、もともと当時の流行語だったカウチポテトに対する日本版として作られた言葉である。うさぎ小屋と呼ばれる日本の住宅事情でカウチポテトは不向き、そもそもカウチを知っている人がどれだけいるのかと提言したうえで、日本人のビデオ鑑賞はこたつにミカンが似合うとし、コタツムリという言葉が生まれた。現在、この意味では死語となっているが、一旦こたつに入るとなかなか出てこない人や、そういったさまを表す言葉として使われている(人気アニメ:キテレツ大百科1990年1月21日放送分『ポッカポッカ!コタツムリで春がきた』の影響?)。また、2007年には書籍のタイトルにもなっている。	1988年	合成語	
\\	しょうゆ顔	しょうゆかお	あっさりとした顔だちのこと。	"しょうゆ顔とはあっさりとした日本的な顔のことで、切れ長の目、鼻筋が通り、顔全体が細くて小さいといった特徴は涼しげだが平凡でもある。1980年代後半以降のアイドルに多い顔で2005年現在も人気がある。しょうゆ顔という言葉は1987年から女性が男性の顔を評するのに使い始め、1988年には『ソース顔』と共に流行語大賞の""流行語部門 ・ 大衆賞""を受賞している。このときしょうゆ顔の代表として受賞したのは少年隊の東山紀之である。他に木村拓哉、風間トオル、巨人軍の清原和博などがしょうゆ顔にあたる。 当時、しょうゆ顔 ・ ソース顔以外にも『マヨネーズ顔』『ケチャップ顔』『みそ顔』など様々なものがあったが、この二つだけが広く普及した。
\\	1988年	若者言葉	
\\	しょぼい	しょぼい	冴えない ・ ぱっとしないこと。	しょぼいとは「目がしょぼしょぼする」や「しょぼくれる」で使われる『しょぼ』に形容詞形化する接尾語『い』を付けたもので、「冴えない」「ぱっとしない」「使えない(使いものにならない)」といった意味で使われる。またしょぼいは『しょぼ(ショボッ)』といった形で感嘆詞としても使われる。	1988年	若者言葉	
\\	ソース顔	そーすかお	彫りの深い顔のこと。	"ソース顔とは西洋的な彫りが深く、少し暑苦しさも感じられる顔のことである。1970年代まではこういった顔が好まれた。ソース顔という言葉は1987年から女性が男性の顔を評するのに使い始め、1988年には『しょうゆ顔』とともに流行語大賞の""流行語部門 ・ 大衆賞""を受賞している。ちなみに、このときソース顔の代表として受賞したのは少年隊の錦織一清である。他には、郷ひろみ、阿部寛、元巨人軍の定岡正二などがソース顔にあたる。 当時、ソース顔 ・ しょうゆ顔以外にも『マヨネーズ顔』『ケチャップ顔』『みそ顔』など様々なものがあったが、この二つだけが広く普及した。
\\	1988年	若者言葉	
\\	チンする	ちんする	
\\	電子レンジで調理すること。 
\\	鼻をかむこと。	
\\	チンするとは電子レンジで調理したり、料理を温めることを意味する。これは家庭用電子レンジの調理完了を知らせる音が「チーン(現在はピピピなど様々)」であることに由来。この音に動詞化する接尾語『~する』をつけたものがチンするである。幼児語にも聞こえるが、世代に関係なく使われ、コンビニエンスストアでも「チンしますか?(温めますか?の意)」と聞かれることがある。 
\\	ちんするとは幼児語で鼻をかむことである。	1988年	
\\	とか語	とかご	「~とか」を頻繁に会話のこと。	とか語とは「~とか」という言葉が頻出する会話のこと。特に断定する名詞の後など、不要な部分に付ける会話を指す(例:今は卒業して、サラリーマンとかしています)。こういったとか語は言葉を曖昧にし、発言に責任を持たない若者の特徴として問題視された。とか弁ともいう。	1988年	
\\	はな木 ・ 花木	はなもく	木曜日の夜に遊ぶこと。	花木とは花金の前日である木曜の晩に遊ぶ(飲み歩く)ことである。これは週休二日制による土 ・ 日を有意義に過ごすため、金曜の夜はゆっくり休んで、木曜の晩に遊んでおくという考え、または週末気分を一日早く味わうために起きた風潮と言われる。花木は1988年の新語 ・ 流行語大賞新語部門で銀賞受賞。ただし、バブル時代が終わるとともに死語となっている。	1988年	
\\	パンピー	ぱんぴー	一般ピープルの略。	パンピーとは「一般人」「素人」「普通の人」といった意味の俗語『一般ピープル』を略したもの。パンピーという略し方は1970年代のツッパリブーム時、ヤンキーや暴走族など不良の間で普及。不良が不良以外を呼ぶ際に使われた。後に芸能人がメディアで使用し、若者を中心に広く普及したが、現在はほとんど使われることのない死語となっている。オタクがオタク以外を指す場合など、一部では現在も使われている。	1988年	若者言葉 ・ 略語	
\\	級グルメ	びーきゅうごるめ	庶民的な安い価格で美味しい料理を探し、楽しむこと。	グルメとは食通 ・ 美食家という意味のフランス語で、主に高級料理店の料理を探求する(楽しむ)人のことだが、そういった高級料理を楽しむことを
\\	級グルメとし、その対として庶民が気軽に食べられる価格帯の店で美味しい料理を探求することを
\\	級グルメと呼ぶ。例えば、ファーストフードを含む外食チェーンや大衆食堂、近所のうどん屋での美味しい料理の探求がこれにあたる。	1988年	
\\	ビフォア9	びふぉあないん	出勤前の時間を有効に使うこと。	ビフォア9とはアフター5に対して出来た言葉で、出勤時間(9時始業の企業が多いためナイン)の時間を自己啓発のために使おうという考え方である。それまでアフター5を有効に使うべきとする提案がなされてきたが、実際、アフター5を有効に使おうと思っても残業や夕食 ・ 飲み会の誘いなど急な予定で時間を奪われることが多い。ビフォア9にはそうした心配がないことから推奨されている。具体的には出勤前の英会話教室、スポーツジム、人脈作りなどがある。ビフォア9という言葉は1988年には既に使われ始めているが、2007年現在も新しい時間管理術 ・ 自己育成術として紹介されることも多い。	1988年	
\\	プー	ぷー	仕事をせずにぷらぷらと遊んでいる者のこと。	プーとは就労可能な年齢でありながら職に就かず、ぷらぷらと遊んでばかりいる者を意味する俗語『プータロー』が略されたもので、1980年代末から若者の間で普及した言葉である。(プータローについては関連ページ参照) また、プータローが男性をイメージする名前『太郎』から成っていることから、同様の女性に対してプー子といった派生語も使われたが、1990年代以降は男女の影響がなく語呂もいいプーで済ませられることが多い。	1988年	若者言葉 ・ 略語	
\\	プーする	ぷーする	
\\	社会人になっても定職に就かず、プラプラすること。 
\\	おならをすること。	
\\	プーするとは大学、高校を卒業して社会人になっても定職に就かず、ぷらぷらしている人を意味する「プータロー」に「~する」をつけて動詞化し、更に略したものである。プータローが動詞になっただけで、基本的な意味は変わらず、アルバイト(フリーター)をしながらプラプラ過ごすこともプーするに入る。 
\\	プーするとはオナラ(屁)の鳴る音として一般的に使われる「プー」に、動詞化する接尾語「~する」をつけたもので、おならをすることである。プーするは当初、幼児に話しかける言葉として使われたが、次第に世代に関係なく使われるようになる。人がしたおならを茶化す際に使われることが多い。	1988年	若者言葉	
\\	ワンコン	わんこん	ワンレン ・ ボディコンの略。	ワンコンとは女性の髪形のひとつ“ワンレングスカット”を略した俗称:ワンレンと、女性のファッションのひとつ“ボディ ・ コンシャス”を略した俗称:ボディコンから成る合成語である(それぞれの詳しい内容は下記関連語を参照)。1980年代後半、ワンレンにボディコンというスタイルが
\\	及び女子大生の間で普及。こうしたスタイルは『ワンレン ・ ボディコン』と連語で呼ばれていたが、後にワンコンと略された。ただし、ワンコンという言い回しが普及する頃にはこうしたスタイル自体が下火となり始めており、現在では全く使われない死語である。	1988年	若者言葉 ・ 略語	
\\	ワンレンブス	わんれんぶす	ワンレンが似合っていない不美人のこと。	ワンレンブスとはバブル期に流行った女性の髪型『ワンレングス(カット)』をもじったもので、ワンレングスカットが似合っていない不美人のことを言った。後にワンレングスカットが背後から見ると似合っていて美しいが、正面から見ると不美人な女性のこともワンレンブスと呼ぶようになる。ただし、ワンレンボディコンという言葉(ファッション)が使われなくなるとともにワンレンブスも死語となっている。	1988年	若者言葉	
\\	朝シャン	あさしゃん	朝起きてする洗髪のこと。	朝シャンとは朝起きてするシャンプー(洗髪)のことである。1987年、朝食を抜いてもシャンプーはするという女子高生が大半を占めているとわかり話題となった。こうした状況を受け、ハンドシャワーが付いた洗面台が登場。なお、朝シャンという言葉は資生堂の
\\	の「朝のシャンプー」というフレーズを女子高生が略したものである。また、朝シャンは1987年新語 ・ 流行語大賞の新語部門 ・ 表現賞を受賞。受賞者は資生堂のセールス商品事業部であった。	1987年	若者言葉	
\\	追いコン	おいこん	追い出しコンパの略。	追いコンとは「追い出しコンパ」の略で、サークル(クラブ)やゼミ、寮の先輩の卒業を祝って行われるコンパのことである。「追い出す」という響きから、先輩に対する悪意(お礼参り的意識)を伴って行われるコンパというイメージを持たれるかもしれないが、実際には通常の飲み会となんら変わりはない。しいて言えば、先輩の卒業ということから生まれる独特な空気(寂しい ・ 悲しいといったものから無礼講的なものまで)がある程度である。 また、音楽系のサークルでは追いコンを「追い出しコンサート」の略として使っている場合もある。	1987年	若者言葉、略語	
\\	鬼のように	おにのように	「非常に」「すごく」といった意味を強める言葉。	鬼のようにとは動詞や形容詞の前に付け、「すごく」「非常に」「大変に」といった意味で、後に続く言葉を強める修飾語である(例:鬼のように早い=凄く早い ・ 非常に早い;鬼のように眠る=すごく眠る=(死んだように)深く眠る)。後に「鬼~」だけで、「超~」や「激~」のような強調語として使われるようになる。	1987年	
\\	家教	かきょう	家庭教師のこと。	家教(かきょう)とは家庭教師の略で、小学校~高校生を対象に家庭に訪問して教育をする人のこと(「かてきょう」「かてきょ」ともいう)。この語句が使われるようになった1987年(昭和62年)はバブル景気の最中で家庭教師を雇う家庭も年々増加していた。また、株式会社トライグループが運営する家教の派遣会社「家庭教師のトライ」を始めたのもこの頃。	1987年	略語	
\\	テレカ	てれか	テレホンカードの略。	テレカとは公衆電話を利用するためのプリペイドカード『テレホンカード』の略である。テレカが日本で発売され始めたのは1982年末。大量の硬貨を公衆電話のために所持しておく必要がなくなったことから直に普及。後に表の絵柄に観光地の写真になっている土産テレカや、タレントやアニメキャラクターを施し、雑誌のプレゼント用となるテレカなど様々なオリジナルテレカが登場。こうしたオリジナルテレカを収集する人も増え、金券ショップやオークションで高額取引されたり、マニア向け雑誌も発行、1987年にはテレカ収集協会も発足している。ただし、携帯電話の登場で公衆電話自体の利用が減っており、収集目的でない一般利用のためのテレカの販売数は減少している。	1987年	
\\	パンキョー ・ 般教	ぱんきょう	一般教養科目の略。	パンキョーとは一般教養科目または一般教育科目の略で般教と書く。パンキョーは大学の講義の中で各学部 ・ 学科に特化した講義を行う専門科目に対し、全ての学科を対象とした科目のことである。人文科学、社会科学、自然科学といった基礎科学系の他、外国語学系、保健体育系などがこれにあたる。	1987年	若者言葉	
\\	ブッチ ・ ぶっちする	ブッチ ・ ぶっちする	授業など出席しないといけないものを理由もなく意図的に休むこと。	ブッチとは授業や部活、アルバイトなど出ないといけないものを風邪など正当な理由がないにも関わらず意図的に休むことで、『ぶっちぎる』の略である。またここから転じ、「約束をやぶる」「無視する」更に「電話に出ない=電話を無視する」~「一方的に電話を切る」という意味でも使われるようになる。ブッチはそれだけでも動詞として使われるが、動詞として使用する場合の多くは『ぶっちする』という形で使われる。 余談だがブッチは漢字で『仏智』と書き「仏の円満な智慧」という意味の言葉もある。 ブッチ ・ ぶっちするの使用例: その電話、原稿の催促だからブッチしていいよ 約束?そんなのどうでもいいじゃん。ブッチブッチ	1987年	不良 ・ ヤクザ用語	
\\	ベルサッサ	べるさっさ	終業のベルと同時に帰る人のこと。	ベルサッサとは「ベルが鳴るとさっさと帰ってしまう」の略で、授業や就業が終わる合図のベルがなると同時に帰宅する人を意味する。ベルサッサは1980年代後半に若者の間で普及、学生の場合は帰宅以外にも授業が終わるチャイムとともに運動場に遊びに出る生徒や購買へパンを買いに走る生徒に対しても使われた。	1987年	若者言葉	
\\	ママドル	ままどる	母親になっても活動しているタレント(アイドル)のこと。	ママドルとは英語の“
\\	(ママ)”と“アイドル”から成る合成語で、母親になってもアイドル的存在で活動をするタレントのことである。 昭和アイドルの代表格である松田聖子が1986年10月1日、神田正輝との間に長女:沙也加(2002年に
\\	名義でデビュー)を出産。それ以降もアイドル的な人気は衰えず、むしろ同性
\\	の支持者が増えカリスマ的な存在になっていく。活動自体も以前と変わらず精力的に行った。そんな松田聖子の代名詞としてママドルという言葉は生まれた。後にママドルは上記の意味で松田聖子以外にも使われるようにる。	1987年	和製英語	
\\	ミツバチ族	みつばちぞく	北海道をバイクでツーリングする人たちのこと。	ミツバチ族とは夏(7~8月)にバイクで北海道に渡り、ツーリングをする人たちのこと。こうしたライダーが乗っていたバイクの排気音がミツバチの羽音に似ていることから、こう呼ばれるようになった。1986年には大阪 ・ 東京からそれぞれ北海道を結ぶ、バイク旅行のための列車(バイクを運搬する列車やバイクと一緒に移動出来る列車)も運行を開始し、ミツバチ族という言葉も広く普及。ただし、この列車が運行をやめた1998年辺りから死語となりつつある。	1987年	若者言葉	
\\	ヤンキー座り	やんきーすわり	ヤンキーが好んでする座り方のこと。	ヤンキー座りとはヤンキーが好んでする座り方のことで、うさぎ飛びの姿勢から足をやや開き、かかとは地面につける。更に手はひざへ持っていった形である。 和式トイレで大便をする時の姿勢と同じことから『うんこ座り』とも呼ばれる。	1987年	若者言葉	
\\	嘘ぴょん	うそぴょん	嘘を言った後、それを打ち明けるときに使う言葉。	嘘ぴょんとは事実より大袈裟に表現したり、冗談で済ませられる程度の脚色をするなど、軽い嘘をついた後、それが嘘(冗談)であると打ち明ける際に使う言葉である。「嘘だよーん」またはそれが転じた「嘘だぴょーん」を簡略化したものと思われる。なお、嘘ぴょんはあくまで場を和ませる感じで使うものであり、あまり深刻な嘘についての使用は不向きである。 2009年に行われた
\\	ランキング「今でもつい使ってしまう死語ランキング」で堂々の1位となっている(ランキングではうそピョーンという表記を使用)。	1986年	
\\	ウニる	うにる	頭が混乱すること。	ウニるとは頭が混乱した状態や混乱した人を意味する俗語『ウニ』に、俗語でよくある名詞を動詞化する接尾語『る』をつけたもので、頭が混乱するという意味の動詞である。焦りや動揺から混乱した際に「頭がウニってる」といった使い方をする。ただし、ウニるは名詞形の『ウニ』に比べ、当時から普及率の低い言い回しである。	1986年	若者言葉	
\\	サリーちゃん	さりーちゃん	足首が太いこと。	人気アニメ「魔法使いサリーちゃん(初代:1966~68年に放映分)」では主人公のサリーちゃんを筆頭に足の太さが太ももから足首までほとんど変わらない円柱に近いものであった(1989年にスタートした二代目サリーちゃんでは足首に向かって細くなっている)。ここから足首が太いこと、逆に足が全体的に細く、足首から腿まであまり太さが変わらないこと、更にそういった足の女性をサリーちゃんという。また、こういった足のことはサリーちゃん足という。 ただし、近年、こういった意味での使用はほとんどなくなってきている。	1986年	若者言葉	
\\	族	さんでーぞく	言い訳ばかりする若者(ギャル)のこと。	
\\	族の3
\\	とは「だって~」「でも~」「どうせ~」の3語の頭文字
\\	のこと。つまり、3
\\	族とはこれら3語が口癖のように頻繁に使う若者(ギャル)を意味する(三語族ともいう)。そんな3
\\	族は「言い訳ばかり」「いつでも言い訳を準備している」といった意味合いで、若者の態度を非難する言葉でもある。	1986年	若者言葉	
\\	新人類	しんじんるい	1960~65年頃の間に生まれた世代のこと。	新人類とは1960年から1965年頃に生まれた世代のことで、団塊の世代と団塊ジュニアの中間に位置する。ジャーナリストの筑紫哲也が若者との対談の中で使ったことから広まり、1986年には流行語大賞・流行語部門金賞を受賞。 新人類は世代によって様々な意味を含んで使われるが、事実検証的にみた場合、生まれた(物心ついた)ときから家に
\\	のあった最初の世代とされ、それまでの世代がした共有体験が戦争や全共闘運動など社会と深く関わっていたのに対し、新人類の共有体験は受験勉強、マンガ、アイドルを含む産業音楽などパーソナルかつサブカルチャー的なものであった。こうしたことから、それまでの世代と違い、新しい価値観をもつ世代ということで新人類という言葉は使われるようになる。当初新人類は「自分勝手」「無感覚・無感動」などのマイナスイメージを持たれた。しかし、プロ野球で当時活躍した新人類の代表:清原和博、工藤公康、渡辺久信(流行語大賞ではこの3人が受賞)がイメージを一掃。「物怖じしない」「クヨクヨしない」といったプラスイメージも持ち合わせるようになる。しかし、新人類といわれた世代が社会の中心的位置にたつ年齢になったことや、更に新しい価値観をもつ若者が登場してくる中で新人類という言葉も使われなくなる。	1986年	
\\	何気に	なにげに	何となく、さりげなく及び、実は、本当はといった形容動詞。	
\\	何気にとは「何気なく」「何となく」「さりげなく」といった意味で1980年代中頃、若者に使われた言葉である。 
\\	何気にとは「本当は」「実は」といった意味で使用する若者言葉である。	1986年	若者言葉	
\\	ハイレグ	はいれぐ	ハイレグカットの略。	"ハイレグとは英語の
\\	(ハイレグカット)""の略で、女性の水着や下着で股間部分の三角形が鋭角になっているデザインのことである。1980年代半ばにハイレグのワンピース水着が流行し、ハイレグという言葉自体は1986年から普及している。ハイレグはその大胆な
\\	ラインから男性に人気があるのはもちろん、足が長く見えることから女性にも人気がある。現在でもレースクイーンやキャンギャルが着用する水着にハイレグが採用されることが多い。
\\	1986年	カタカナ英語 
\\	略語	
\\	プータロー	ぷうたろう	仕事をせずにぷらぷらしている人のこと。	"プータローとは厳密には学校を卒業しても定職に就かない者(日雇い労働者を含む)をいうが、一般的には無職でぷらぷらしている人(日雇い労働者は含まない)、更に遊んでばかりいるために進学及び進級出来ずにいる浪人生や大学の留年生を意味する。これは表記の変化によるもので、プータローは終戦後の横浜を中心とした日雇い沖人夫や陸上の自由労務者をさして使われた風太郎が語源である。次第に日雇い労働者を含む定職に就いていない者全体を風太郎と呼ぶようになり、表記もプータローで定着。1980年代、プータローは若者を中心に全国的に普及するが、当初の語源を知らない世代にはプータローの『ぷー』が「ぷらぷら」や
\\	を連想させ、日雇い労働者を省く無職者に限定して使われるようになる。(ただし、現代でも人によっては「日雇い労働者=定職に就かずにぷらぷらしている者」と位置づけ、日雇い労働者にプータローが使われる場合がある) また、2000年以降は類語のニートが普及するが、ニートは働く意思がないの者に限定されているのに対し、プータローは意思はあるが就職先が見つからない者も含まれるため、同じ無職でも意味合いは異なる。 余談だが住居にプータローやプータローホームと呼ばれるものがある。これはフィンランド語の
\\	(木)
\\	(家)
\\	(木の家)""である。ちなみに岩手県八幡平スキー場近くにあるプータロー村はこちらのプータローからきており、無職者を意味するプータローが集う村というわけではない。
\\	1986年	
\\	ボディコン	ぼでぃこん	女性特有の体のラインを浮きだたせたファッションのこと。	"ボディコンとはボディラインに密着し、胸や腰、お尻といった女性特有の体の線を浮き立たせたファッションのことで、英語の
\\	""を略した和製英語である。(英語の
\\	""は「体型(肉体)を意識している」といった意味で使われ、ファッションのみを指す言葉ではないのでご注意ください) ボディコンは北アフリカ出身のデザイナー
\\	""によって作られ、1981年にはミラノコレクションで発表、世界的に話題となる。日本では1986年頃、ディスコ ・ ブームとともに話題となったファッションである。その際に、ワンレンというヘアスタイルと一緒に普及したことから『ワンレンボディコン』といった形でセットにして記憶している人も多い。
\\	1986年	和製英語	
\\	耳がダンボ	みみがだんぼ	聞き耳を立てること、またはそういったさま。	耳がダンボのダンボとは小象をモチーフにしたディズニーキャラクターの一種で、羽のように広げて空が飛べるほど大きな耳が特徴である。漫画で聞き耳を立てる姿を表現する際、耳を大きく描く描写が使われる。この2点から聞き耳をたてることや、聞き耳をたてるさまを耳がダンボという。耳ダンボ、ダンボと略されることも多い。また自動詞として使う場合、耳をダンボにするともいう。	1986年	
\\	目が点	めがてん	驚いているさまを表す言葉。	目が点とはマンガの中で驚いた人を描く際、目を点で表す表現方法からきた言葉で、驚いているさまを表す言葉である。驚くさまや呆然とするさまを目が点と言うになったのは1980年代後半からで、当初は「目が点になる」といわれていた。1980年代末になると若者の間で「目が点~」と略し、語尾を伸ばして感嘆詞的にも使われるようになった。	1986年	
\\	ワシントンクラブ	わしんとんくらぶ	トイレのこと。	ワシントンクラブは英語で書くと
\\	になる。水洗式便所は英語で
\\	(日本でも
\\	という形で広く普及)」というが、それぞれ頭文字が
\\	であることから、トイレのことをワシントンクラブと呼んだ。例えば合コンの最中にトイレへ行きたくなった際、「ちょっとワシントンクラブに行ってくる」とトイレへ行くことを遠回しに表現するために使用する。ただし、2007年現在、こういった言い回しはせず、死語となっている。	1986年	若者言葉	
\\	あちゃあ	あちゃあ	「しまった」「やばい」と思ったときに発する言葉。	あちゃあとは物を壊してしまったときや人に隠していたことが知られたとき、自分の言ったギャグがうけず場がしらけたときなど、「しまった」「やばい」「イタイ」と思ったときに発する言葉である。 あちゃあの使用例: あちゃあ、会議は明日と思ってたから資料を持ってこなかったよ。	1985年	若者言葉	
\\	うざい	うざい	うざったいの略で使われる会話言葉。	うざいとは『うざったい』の略で、「鬱陶しい」「わずらわしい」「うるさい」「面倒臭い」「気持ち悪い」「邪魔」といった意味を持つ。うざいは1980年代のツッパリブームから関東圏を中心に使われるようになり、1990年代には不良以外にも使われ、全国的に普及する。うざいが更に簡略化された『うざ』や、うざいの語感が荒くなった『うぜー(うぜえ)』という言い方もある。2006年、学生の相次ぐ自殺が社会問題となるが、うざいと言われたことが原因になったり、うざいの一言が発端で殺傷事件になるほど荒い言葉なので使用には注意が必要である。 うざいの使用例: 毎晩のように電話してくる友達がうざいんだよな 毎日毎日小言ばっかりで、あ~うざい! いい加減、まとわりつくな!うぜえんだよ	1985年	不良 ・ ヤクザ用語、若者言葉	
\\	ウニ	うに	頭が混乱している状態のこと。(俗語的解釈)	寿司ネタなどに使われるウニは柔らかく、力が加わるとグチャグチャになりやすいことから、頭が混乱してわけがわからなくなった状態(脳味噌がウニのような状態=頭がグチャグチャになった状態)をウニと言い、多くは「頭がウニ」という形で使われる。また、ウニはそういった状態の人自体を指しても使われる。	1985年	若者言葉	
\\	デー 
\\	えっくすでー	重大な出来事が起こると(予想)される日。	
\\	デーとはいつなのか断定(又は発表)出来ないが、「重大なこと」「大切なこと」が起こるとされる日のことで、比較的それが近い将来のときに使われる。
\\	デーは元々第二次世界大戦の軍事用語で、秘密裏に進められる工作の決行日を指した。昭和末期になると、昭和天皇が死去する可能性の高い日をさす隠語として報道マンが使用。昭和天皇が崩御された1989年には一般にも広く普及した。現代では出産予定日、店舗やイベントでの
\\	人目来客予定日、地震が起こる日、プロ野球の優勝決定予想日、人気商品の発売日など幅広い用途で
\\	デーは使われる。 
\\	デーは和製英語です。英語では『ノルマンディー上陸作戦』で使われた
\\	という軍事用語が
\\	デーに近い使われ方をする。ただし
\\	は
\\	デーと違い『ノルマンディー上陸作戦』自体や上陸作戦が決行された1944年(昭和19年)6月6日を指す固有名詞としての意味合いが強い。	1985年	軍事用語 ・ 和製英語	
\\	カエルコール	かえるこーる	夫が妻に帰宅すること(又は帰宅時間を告げる)電話のこと。	カエルコールとは「帰るコール」と書けばわかる通り、夫が仕事を終え、妻に「今から帰る」と電話することや帰宅予定時刻を告げる電話をすること。カエルコールは
\\	が提唱したもので、「帰る」と「蛙」をかけた巨大蛙が出てくる
\\	も話題となった。また、仕事や学業で故郷を離れた子供から親向けにかける「ふるさとカエルコール」というものもあった。	1985年	
\\	家庭内離婚	かていないりこん	夫婦関係が破綻しながらも同居はしている状態のこと。	家庭内離婚とは離婚届けを提出し、実際の夫婦関係(婚姻関係)は破綻しているが、表面上は夫婦関係を保っている(装っている)状態のことである。家庭内離婚をした夫婦の多くは同じ家に住んでいても寝食を共にしないことはもちろん会話もほとんどない。 家庭内離婚を選択する理由は、夫婦間が冷め切り離婚はしたいが経済的問題や子供のために別居までは出来ないといったもので、他に世間体(近所や会社、親族に対する体裁)が理由という意見もある。 家庭内離婚は1983年に作家の林郁が
\\	で発言し、1985年に書籍のタイトル として使用し、流行語にもなった。	1985年	
\\	サクッと	さくっと	すばやくテキパキと行動するさま 。	サクっととはすばやく行動することを意味する「サクサク(と)」を略したもので、すばやくテキパキと行動するさまを表す。『サクサク』は賭博用語から広まった言葉だが、サクッとはもともと麻雀用語として使われたものが若者に広まった言葉である。現代の若者の間でサクッとは単にすばやく行動するだけでなく、簡略化したり、簡単に済ませるといった意が含まれることが多い。	1985年	
\\	チャラい	ちゃらい	言動が軽薄で浮ついていること。	チャラいとは言動が軽い様を表す俗語『チャラチャラ』を略し、形容詞化する接尾語『い』をつけたもので、言動が軽く浮ついている様や服装が派手で安っぽい様を表す。チャラいは1980年代に使われ始めた言葉で、徐々に使用度が減っていたが、近年、同様の意味で再び若者に使われるようになっている。	1985年	
\\	デパギャル	でぱぎゃる	デパートで働く女性のこと。	デパガとは『デパート ・ ガール』の略で、デパートで働く女性を意味する。デパガは「美人」「派手」「化粧が濃い」といったイメージを伴って使われるため、デパートの中でも主に化粧品 ・ ジュエリー ・ ファッション ・ 受付(案内 ・ サービスカウンター)といった部署に勤める女性を指し、特に食品売り場は除外されることが多い。また、デパート内でもエレベーターで案内をする女性はエレベーターガール(又はその略のエレガ)と別格で表される。 デパガは1990年代から普及した言葉で、それまでは『デパート ・ ギャル』を略したデパギャルが使われていた。	1985年	若者言葉	
\\	トラキチ	とらきち	熱烈な阪神ファンのこと。	トラキチとは1985年に21年ぶりとなるリーグ優勝、更に初の日本一となった阪神ターガースを支えてきた熱狂的な応援団(及び
\\	のことで、『虎(タイガース)きちがい』の略である。また、トラキチは1985年の流行語大賞で流行語部門 ・ 銀賞を獲得。優勝の盛り上がりムードから阪神ファンになった人を意味する『にわかトラキチ』という言葉も使われた。	1985年	
\\	はな金 ・ 花金	はなきん	金曜日の夜に遊ぶこと。	花金とは「花の金曜日」の略で、金曜日の晩に遊んで(飲み歩いて)楽しむことを意味する。これは週休二日制の導入により、土曜休日の企業が増えたことで、休日前の金曜の晩は遊びまわる風潮が出来たことから生まれた言葉である。ただし、バブル時代が終わるとともにサラリーマンにこうしたゆとりがなくなり、花金も死語となっていく。	1985年	
\\	はぶにする	はぶにする	仲間はずれにすること。	はぶにするとは仲間はずれにすることで、名古屋圏及び関西の一部でははばにするともいう。はぶにするは「村八分(註参照)にする」の略、または「省く(はぶく)」が変化したものと言われるが、同時期にはちにするという言い回しの同義語があることを考えると、前者が妥当と思われる。また「はぶ(はば)」で仲間はずれを意味する名詞としても使われる。 村八分:村民が規約違反を犯した際、全村申し合わせでその家との交際を断つ制裁のこと。転じて仲間はずれを意味する(参考文献:広辞苑)	1985年	
\\	ヘタウマ ・ 下手巧	へたうま	一見(一聴)するとへたそうだが、個性や味のある作品のこと。	へたうまとは主に絵画や音楽(楽器演奏)の世界で使われる言葉で、各々の世界で求められる技術や完成度を満たしていないため、一見(一聴)するとへたそうだが、逆にそれが個性や味になっている作品を指す。ただし、特にへたうまの定義はなく、個性 ・ 味といった感性による曖昧な部分が尺度となるため、同じ作品でもへたうまと感じる人もあれば、単に下手と捉える人もある。 また1990年代には若い世代が使う字体を意味する『へたうま文字』という言葉も使われた。	1985年	
\\	ヤラセ	やらせ	メディアが作った虚偽の報道をすること。	やらせとはテレビや新聞などで事実にはないことを演技させ、その映像や写真を使った虚偽の内容を報道することである。やらせは元々業界用語であったが、1985年テレビ朝日『アフタヌーンショー』の報道にやらせがあったとして話題となり、一般にも浸透した。 また、報道に関係なく、事前に打ち合わせた内容を自然に振る舞わせる様々な行為に対してやらせが使われるようになる。バラエティ番組における見学客のスタッフの合図で起こる拍手、露天の『さくら』もやらせの一種である。	1985年	
\\	わちゃあ	わちゃあ	「しまった」「やばい」と思ったときに発する言葉。	わちゃあとは物を壊してしまったときや人に隠していたことが知られたとき、自分の言ったギャグがうけず場がしらけたときなど、「しまった」「やばい」「イタイ」と思ったときに発する言葉である。	1985年	若者言葉	
\\	一円玉ブス	いちえんだまぶす	不美人な女性をからかう言葉。	一円玉ブスとはブスを更に強調した言葉で、一円玉が両替でそれ以上崩せない額(硬貨)であることから「これ以上崩しようのないブス」という意味で使われる。	1984年	若者言葉	
\\	可取り専攻	かとりせんこう	成績が可ばかりの人のこと。	可取り専攻とは成績が優 ・ 良 ・ 可の中でギリギリ合格ラインにある可ばかりの人を皮肉を込めて呼んだ言葉である。可取り専攻は蚊を駆除する蚊取り線香のもじりで1980年代に普及。また、評価の良し悪しに関係なく、合格さえすればいいという考え方をさして使うこともある。どちらにしても現在は死語となっており、全く使われていない。	1984年	若者言葉	
\\	くれない族	くれないぞく	「~してくれない」という言葉をよく言う人のこと。	くれない族とは「~してくれない」とよく口にする人やそういった思考の強い人を意味する。くれない族は1984年、
\\	テレビで放送された田村正和、大原麗子主演のドラマ『くれない族の反乱』から生まれた言葉である。このドラマは、それまで子供の使っていた
\\	が××してくれない」という言葉が主婦層にまで浸透している実態を描いたものとして評価され、くれない族という言葉は同年の流行語大賞で流行語部門銀賞を獲得。自己啓発 ・ 人材教育などの書籍 ・ セミナーで使われることも多い。	1984年	
\\	だ埼玉	ださいたま	埼玉県を嘲う言葉。	だ埼玉とは東京周辺エリアにあって垢抜けしない埼玉を嘲った言葉である。もともと似た言い回しで『ど埼玉』という言葉があったが、1970年代後半より流行した俗語『ダサい』と『埼玉』をかけてだ埼玉になった。1980年代中頃、タレントのタモリ(森田一義)
\\	番組内でが頻繁に使い普及(最初にだ埼玉と言い始めたのもタモリといわれている)。ただし、次第に使われなくなり、死語となっている。	1984年	若者言葉	
\\	ナンパーマン	なんぱーまん	ナンパをする男性のこと。	"ナンパーマンとは俗語の『ナンパ』と男性を意味する英語の
\\	""から成る合成語(語の響きから『スーパーマン』との合成という説もある)で、ナンパをする男性を意味する。ただし、他の同義語と異なりナンパーマンという表現は現在あまり使われなくなっている。 ナンパーマンはもともとテレビ朝日の深夜番組『ミントタイム』(1984年放送)で俳優の松澤一之が扮した役名(キャラクター名?)で、内容はナンパーマンが街で声をかけて捕まえた女性宅を訪問するというものであった。
\\	1984年	合成語~若者言葉	
\\	顰蹙	ひんしゅく	眉をひそめさせるようなことをした人への非難の言葉(俗語的解釈)	顰蹙(ひんしゅく)とは「顰めっ面(しかめっつら)」に見られる顰(しかみ)と、ちぢこまるという意味の蹙まる(しじまる)から来た言葉で、本来は顔をしかめることや眉をひそめるさまをいう。また、周りに居る人の顔をしかめさせたり、眉をひそめさせるような行為をして軽蔑されることを「顰蹙を買う」というが、後にこれを略して顰蹙と言うようになる(例:あの態度は顰蹙だったんじゃない?)。また、1980年代半ばになると、顰蹙を買うような行為をした人への非難の言葉としても使われるようになる。この場合、多くが「ヒンシュクー」と語尾を延ばす傾向にある。	1984年	若者言葉	
\\	百合族	ゆりぞく	女性同士の同性愛や女性同性愛者のこと。	百合とは女性同士の同性愛やそれをする女性同性愛者をいう。これは男性同性愛者向けに発行されていた雑誌「薔薇族」が女性読者(女性同性愛者)向けに始めた投稿コーナーのタイトル「百合族コーナー」からきている。1980年代に入るとポルノ映画や男性誌が百合族という言葉を使うようになり、そういった嗜好がない人にも広く浸透する。しかし、直にレズという言葉に移行し、現在、百合は女性同性愛の中でも精神面を重視した比較的プラトニックな恋愛関係をさす傾向の言葉となっている。	1984年	
\\	ロンタイ	ろんたい	ロングタイトスカートの略。	ロンタイとはロングタイトスカートの略である。1970~80年代のツッパリブーム、不良少女はくるぶしに掛かるほど丈の長いスカートに丈の短い上着がステータスであった。このロングスカートをロンタイといい、1988年にはロンタイ
\\	というコミックのタイトルにも使われている。1990年代には短いスカートにルーズソックスが主流となり、こうしたロンタイは影を潜めていく。ただし、タイトなロングスカートを指すファッション用語として言葉は残っている。また、スケバンの格好でブレイクした芸人、桜塚やっくんに当時のロンタイを見ることが出来る。	1984年	若者言葉	
\\	うんこ座り	うんこすわり	和式便所で大便するときのしゃがむ姿勢。	うんこ座りとは和式便所で大便するときのしゃがむ姿勢のことで、不良がたむろして座るときにすることから『ヤンキー座り』ともいう。また、読みはうんこすわりの他にうんこずわりともいう。具体的にはうさぎ飛びの姿勢から足をやや開き、かかとはついて手はひざに乗せた姿勢である。尻は浮かせた状態であるが、現代の食生活で育った若者には尻を浮かせるという姿勢がきつく、うんこ座りが出来る若者が減っているという見解もある。ただし「うんこ座り=ダサイ」と考える若者が年々増えていることも事実であり、うんこ座りをする人が減った直接の要因であるかは定かでない。	1983年	不良 ・ ヤクザ用語	
\\	おしん	おしん	耐え続けること。	おしんとは1983年4月から放送が始まり、60
\\	以上の視聴率をとった
\\	の人気連続ドラマのタイトルであり、主人公の名前である。橋田壽賀子原作のこのドラマでおしんは貧乏な生活と次々起こる苦難を耐えに耐え、やがて幸せを掴むことになる。この耐え続けるおしんの姿が人気となり「おしんブーム」に。耐え続けることを「おしん」、また耐えるという動詞として「おしんする」という言葉が使われた。ただし、ドラマ自体が忘れられていくのと並行してそれらの言葉も死語となっている。	1983年	
\\	お茶する	おちゃする	喫茶店へ行くこと。(俗語的解釈)	お茶するとは喫茶店に行くこと及び、喫茶店で何か飲みながら休憩したり、話したりすることを意味する。1970年代にどこかへ行くことを
\\	する」といった表現が流行(映画に行く場合なら「映画する」など)。お茶するもその一つだが、他と違い、ナンパの常套文句など、後々まで使われ、1980年代前半には独立して流行語となっている。ナンパでは「お茶する?(お茶しない?)」と疑問形にして使用。関西地区では「茶しばく」という。	1983年	
\\	クリスマスケーキ	くりすますけーき	25歳を過ぎ、男性の理想 ・ 条件を下げる女性のこと。	クリスマスケーキはクリスマス当日の25日を過ぎると安い値段でさばかれる。また、適齢期を過ぎても結婚出来ずにいる女性のことを「売れ残り」という。ここから、24歳までは男性に対し、結婚のための条件をいろいろつけていたが、25歳を過ぎて条件を下げる(=自分を値引きする)女性をクリスマスケーキという。ただし、結婚適齢期が高くなるとともにこの意味では使われなくなり、現在、死語となっている。	1983年	
\\	ケバイ ・ ケバい	けばい	「けばけばしい」の略。	ケバイとは「目立っている ・ はでやかである」という意味の「けばけばしい」を略したものである。ただし、ケバイと略した場合、化粧や服装が派手な女性を対象に使われる(あくまで装飾が派手な女性であり、顔の作りが派手(濃い)女性に対してケバイとは言わない)。また、ケバイは嘲う意や嫌悪感を伴って使われることが多い(派手な容姿を好む男性が好意的に使用)。	1983年	
\\	ジャパゆきさん	じゃぱゆきさん	日本へ出稼ぎに来ている東南アジアの女性のこと。	ジャパゆきさんとは唐ゆきさん(からゆきさん:明治~戦前、海外に渡り、娼婦として働いた日本人女性のこと。唐といっても中国に限ったものでなく、アジアを中心に世界に渡る)のもじりで、海外から日本へ出稼ぎにきている東南アジアの女性を意味する。この言葉が流行になった1980年代には不法滞在や売春強要など、社会的問題として話題となった。以降も数的には増えているが、逆に珍しい存在でなくなったのか、ジャパゆきさんは死語となりつつある。	1983年	
\\	熟女	じゅくじょ	円熟した年頃の女性のこと。	熟女とは円熟した年頃の女性のことで、1970年代後期に作られた造語「熟年」の派生語である。1983年には中森明菜のヒット曲「少女
\\	に対し、五月みどりが「熟女
\\	という曲を発表し、話題となる。熟女は年齢ではなく、風貌や雰囲気から区分されるため、具体的な年齢は定義されていないが、この言葉が使われ始めた1980年代初めには20代後半から40代辺りの女性を指した(これは目安であり、最近では30~50代という意見も多い)。 また、先に述べた五月みどりは70年代に当時30代で男性誌のグラビアやポルノ映画に出ていたことから、元祖熟女ヌードとして扱われることが多い。ただし、熟女が五月みどりをモチーフとして生まれた言葉であるかは不明。	1983年	
\\	ダサさ	だささ	野暮ったさ、かっこ悪さのこと。	ダサさとは古臭い ・ 野暮ったい ・ 田舎臭い ・ かっこ悪いといった意味の俗語『ダサい』の名詞形で、古臭さ ・ 野暮ったさ ・ 田舎臭さ ・ かっこ悪さといった意味で使われる。また、意味から嘲ったり、批判のみに使われるように思われるが、「ダサさ加減が好き」「そのダサさがかわいい」といった肯定表現に用いられることがある。こうした場合「初心(うぶ)」という意が含まれていることが多い。	1983年	若者言葉	
\\	トラバる	とらばる	
\\	女性が転職すること。 
\\	のトラックバックをすること。	"トラバるとは俗語『とらばーゆする(トラバーユする)』を略したもので、女性が転職することを意味する。元のとらばーゆするはリクルート出版から出ている女性向け転職雑誌『とらばーゆ』に動詞化する接尾語の『する』をつけたものである。ちなみに『トラバーユ』とは仏語の
\\	""で「仕事 ・ 労働」を意味するが、日本では雑誌名としての認知のほうが高いため「トラバる=女性の転職」という意味で定着している。 また、これとは別にトラバるとは
\\	(ブログ)のトラックバックをすることやトラックバックを活用して遊ぶという意味もある。
\\	1983年	合成 
\\	略語	
\\	ネアカ ・ 根明	ねあか	根が明るい性格のこと、またはそういう人のこと。	ネアカとは「根が明るい」または「根っから明るい」の略で、話しやすい、周りや自分自身を楽しませたり、笑わせるのが得意など、第一印象や見た目に関係なく、本質が明るい性格やそういった人を意味する。ただし対語の『ネクラ』に比べると性格よりも前向き ・ ポジティブといった考え方 ・ 生き方が対象になっていることが多い。ネアカは『ネクラ』とともに漫画家のいしかわじゅんによる造語で、タレントのタモリこと森田一義がはやらせたことで流行語となった。	1983年	
\\	ネクラ ・ 根暗	ねくら	根が暗い性格のこと、またはそういう人のこと。	ネクラとは「根が暗い」または「根っから暗い」の略で、無口そう、いつまでも根に持ちそう、不平 ・ 不満 ・ 愚痴ばかり言ってそうなど、第一印象や見た目ではなく、本質が暗い性格やそういった人を意味する。ネクラは対語の『ネアカ』とともに漫画家のいしかわじゅんによる造語で、タレントのタモリこと森田一義がはやらせたことで流行語となった。	1983年	
\\	ホコ天	ほこてん	通常車道となっているところを曜日 ・ 期間を限定して歩行者に開放すること、又は開放された道路のこと。	ホコ天とは歩行者天国の略で、通常は車道して使われている道路を限定曜日や限定期間、歩行者に開放すること、また開放された道路のことを指す。自動車による死亡事故が最も多かった1970年に銀座、新宿、池袋、浅草の4箇所で限られた区間の車道が歩行者に開放され話題となった。当時、自家用車の普及にともない交通事故の多発が深刻な問題となっており、そうした環境の中でこういった道路が『歩行者天国』と呼ばれた。1980年代に入ると、歩行者天国はホコ天と略して呼ばれるようになり、竹の子族やストリートパフォーマンス、バンドブームなど、若者カルチャーの発信源的な役割を担うようになる。こうして当初の目的であった道路の安全と街の活況という意味では成功を果たしたホコ天だが、歩行者が出すゴミや騒音、近隣道路の渋滞が問題となり、廃止するところも出ている。なお2006年現在、前後の文脈なしにホコ天といった場合、秋葉原を指すことが多い。 ※ホコ天が最初に話題になったのが先述の東京の4箇所であったため、最初にホコ天が実施された場所と紹介されていることがありますが、これらは1970年8月2日のスタート。それ以前(1969年8月6日)に北海道旭川市の平和通りが行っており、こちらがホコ天の発祥となります。	1983年	略語 ・ 若者言葉	
\\	胸キュン	むねきゅん	感動や恋心などで胸がいっぱいになること。	胸キュンとは「胸がきゅんとする ・ 胸がきゅんとなる ・ 胸がきゅんとくる」といった言葉を略したもので、恋心や感動 ・ 歓喜 ・ 悲哀といった感情で胸がいっぱいになること、胸がしめつけられることを意味する。ただし、胸キュンと略したものは、恋心が原因となっている場合に使われることが多い。胸キュンは
\\	(イエローマジックオーケストラ)が1983年に発売したシングル『君に胸キュン』に使われている。この曲がカネボウの
\\	に起用され、ヒットソングになったことから、胸キュンも若者を中心に広く普及。ただし、現在は広告のキャッチコピーなどに使われる程度で死語となっている。	1983年	
\\	メルヘンチック	めるへんちっく	童話の中のようなかわいらしいさま。	メルヘンチックとはドイツ語で「お伽噺 ・ 童話」を意味する『メルヘン
\\	に、英語で「~的」「~っぽい」といった意味にする接尾辞『チック
\\	をつけた和製英語(和製独語)である。童話の世界に出てくるようなかわいさ、不思議さ、神秘さを含む雰囲気やそういったさまをメルヘンチックという。 ※メルヘン
\\	作り話、物語という意味の 
\\	が変化した 
\\	に、前に付く言葉を「小さい ・ かわいい」という意味で形容するする接尾辞 
\\	をつけたもので童話やおとぎ話をさす。また、ドイツでは「でたらめ」「ほら話」といった俗的意味でも使われる。	1983年	和製外語	
\\	めんたん切る	めんたんきる	睨みつけること。	めんたん切るとは睨みつけるという意味で、関西を中心に使われた『めんた切る』が音的に変化したものである。この場合の睨みつけるとは、主にケンカを売ったり、言いがかりをつけるために行うものを指す。 ただし、めんたん切る同様、『めんた切る』が音的に変化した言葉『メンチ切る』に比べると、関西以外での認知度は低い言い回しである。	1983年	若者言葉	
\\	メンチ切る	めんちきる	睨みつけること。	メンチ切るとは睨みつけるという意味で、関西を中心に使われる言葉である。この場合の睨みつけるとは、主にケンカを売ったり、言いがかりをつけるために行うものを指し、メンチ切るも1980年代のツッパリブーム以来、不良が好んで使う言葉であった。メンチ切るという行為は不良同士がすれ違いざまにしたり、ヤンキー座り(うんこ座り)して下から睨みつけるといった形が多い。ただし、メンチ切るはそういった行為をされた側が抗議(難癖)をつける際に用るのがほとんどで、 行為をする側が「メンチ切ってやる」といった使い方はしない。 また、メンチ切るの語源解説で「面子切るの変形」というものもあるが、1970年代の関西では『めんた切る』と言い、メンチ切るはその派生語に当たる。この点から「目ん玉切る」が元になっている可能性も考えられる(詳細不明)。 死語になりつつあったメンチ切るだが、2005年にボクシングの亀田3兄弟を扱う記事でメンチ切り、メンチ戦といった言葉が使われ、再普及している。	1983年	不良 ・ ヤクザ用語	
\\	めんどい	めんどい	非常に面倒、わずらわしいの意。	めんどいとは「面倒臭い(めんどうくさい)」の略で、面倒臭い同様、非常に面倒、わずらわしいという感情を表す言葉として使われる。1980年代から若者の間で使われるようになっためんどい、現代の若者が複雑なもの、難解なものに対し、回避傾向にあることを表す言葉とも言われている。ただし、めんどいという言葉自体は江戸時代、既に使われていたという説もある(調査中)。	1983年	
\\	ヤンキー	やんきー	
\\	不良行為を行う少年及び不良っぽい雰囲気の少年。 
\\	アメリカ人の俗称。	ヤンキーとは不良行為を行う10代の少年のことで、パーマをかけ、剃り込みをいれるといった独特のヘアスタイルが特徴である。会話の内容から明らかにヤンキーに関するものとわかる場合はヤンと略されることもある。 また、ヤンキーはアメリカの南北戦争時、南軍が北軍兵士に対して使った蔑称で、後にアメリカ人をさす俗称となった言葉でもある(大リーグのニューヨーク ・ ヤンキースはここからきている)。不良を意味するヤンキーはここから来たという説と関西の不良が好んで使う
\\	やんけ」という言葉から「やんけ言い」と呼ばれ、これがヤンキーになったという説がある(詳細不明~調査中)。	1983年	若者言葉	
\\	横浜	よこはま	便所または小便のこと。	横浜の市外局番は
\\	だが、これを「おしっこ」と読み、小便のことを横浜という。また、トイレに行くことを「オシッコしてくる(オシッコに行ってくる)」というように「横浜に行ってくる」という。例えば接客中や合コンのときなど、公然とトイレに行くことが言いにくいときに「横浜に行く」と隠語風に使う。ただし、近年こうした言い回しは使わず、死語となっている。	1983年	
\\	いいとも	いいとも	同意 ・ 了解を表す言葉。	いいともとは1982年10月から放送しているタモリ司会の長寿番組「笑っていいとも」によって広まった言葉で、「大丈夫です ・ いいですよ ・ わかりました」といった了解 ・ 同意を表す言葉である。同番組ではオープニングでタモリが客席に「笑っていいかな?」と問いかけた際、客席が一斉に「いいともー」と答えたり、テレフォンショッキングというコーナーで翌日出演依頼したゲストに「明日来てくれるかな」と問いかけた際、問われたゲストの決めセリフとして使われる。番組が話題になるとともにいいともという言葉も世代を超え広く普及。	1982年	
\\	ええかっこしい	ええかっこしい	人前で実際よりも格好良く見せようとすること。	ええかっこしいとは『ええ格好(=良い格好)』に、「~する人」という意味にする接尾語として関西方面で使われる『~しい(~しぃ)』を付けたもので、人前で実際よりも良い人を装ったり、格好良く見せようとする人のことをいう。	1982年	
\\	おいしい	おいしい	好都合、好条件、お得の意。	おいしいとは本来、「好ましい」「見事」「味がよい」といった意味の『美し(いし)』に、形容詞化する接尾語『い』と丁寧にする接頭語『お』をつけたもので、「美味い(うまい)」の上品な言い回しとして使われていた。1982年になると、コピーライター:糸井重里が「おいしい生活」という西武百貨店のキャッチコピーを作ったことから、「好都合な~」「好条件な~」「お得な~」といった意味でも使われるようになる。後に対象となるものを付け、「おいしい話し」「おいしい仕事」「おいしい状況」といったように用いることが多い。	1982年	
\\	追っかけマン	おっかけまん	芸能人を追いかける取材記者のこと。	追っかけマンの「追っかけ」とはアイドル、タレント、俳優、歌手やスポーツマンなど、著名人を会場の出入り口やホテル、中には自宅まで追っかけていく行為やそういった人(主にファン)を意味する。これに対し、追っかけマンは仕事で追っかけをする取材記者を意味する。特にアナウンサーの松宮和彦が
\\	系の音楽番組『ザ ・ ベストテン』の追っかけマンとして、生放送当日、地方で活動中していたり、コンサート中でスタジオ入り出来ない歌手を追っかけ、中継していたのが有名である。	1982年	
\\	ノンケ	のんけ	異性愛者のこと。	ノンケとは異性愛者のことで、男性同性愛者の間で普及した言葉である。ノンケは
\\	気」と書けばわかる通り、「その気がない」という意味に繋がります。つまり、男性同性愛者に対しその気がない=異性愛者ということになります。 ※ノンケが使われるようになって20年以上経過していますが、現在も男性同性愛者だけに普及する言葉であるか(女性同性愛者に普及していないか)は不明。	1982年	
\\	パシリ	ぱしり	強い人や偉い人の命令で使い走りする人のこと。	パシリとは「使いっ走り(つかいっぱしり)」の略で、自分より強い人や偉い人の命令で使い走りする人やそういう行為自体を意味する。先輩の命令で買い物に行く後輩がパシリにあたるが、全後輩が同等程度に交代で使い走りする場合、全後輩をパシリと呼ぶことはない。これはパシリに「アゴで使われる人」といった嘲う意が込められるためである。ただし、現代では特に嘲う意を込めず、「ちょっとパシリに行ってくれ」といった形で単に使いを頼む際にも使われる。このため一概にパシリ=いじめであるとは限らない。ちなみにパシリはツッパリブームの不良が使用、直に学生を中心に広く若者に普及した。	1982年	若者言葉	
\\	パッパラパー	ぱっぱらぱー	アホのこと、または「理解できない」「わからない」の意。	パッパラパーとは馬鹿・阿呆という意味で使われる「パー」をふざけて言う軽い語感にした言葉である。語感が軽いだけで基本的な意味は馬鹿 ・ 阿呆と同義であるが、他の類語に比べ、自嘲する際に用いられることが多い。また、パッパラパーの語源がドイツ語の“
\\	とする説があるが、この語が「くだらない話しをするな」「べちゃべちゃと話すな」という意味であることから、俗説または音的なものだけを参考にしたと思われる。また、先述の意味が転じ、「(全然)わからない」「(全く)理解出来ない」といった意味で、「僕、物理はパッパラパーだから…」というようにも使われる。	1982年	
\\	はまちっ子	はまちっこ	男子の前でかわいい子ぶる女の子のこと。	はまちっ子とは男子(男性)の前で無知や非力を装い、おとなしくてかわいい子や良い子を演じる女の子のことである。魚のハマチはブリの成長過程の呼び名である。ここから“ぶりっ子”の子供版をはまちっ子と呼んだ(ぶりっ子に関しては下記関連語参照)。ただし、ハマチと略すとぶりっ子と同じ大人版を意味する。これは、ぶりっ子=ブリの子=ハマチという解釈による。どちらにしてもぶりっ子と違い、ハマチやはまちっ子は定着せず、死語となっている。	1982年	若者言葉	
\\	むかつく	むかつく	癪に障る ・ 腹が立つこと。	むかつくとは「胃がむかつく」に見られるように、胸焼けや吐き気をもよおすことだが、関西などエリアによっては江戸時代から「癪に障る ・ 腹が立つ」という意味でも使われる。1970年代後半のツッパリブーム時には不良が教師や親、警察、敵対する人 ・ グループといった自分たちの自由や思惑の邪魔になる者を対象に、後者の意味で全国的に使うようになる。1980年代に入ると世代 ・ エリアを超え、こちらの意味でも広く浸透。また、1998年に栃木県で起きた教師刺殺事件で犯人の中学生の発言に使われ話題となった。マンガなどではムカつくという表記も使われる。	1982年	
\\	赤テープ	あかてーぷ	不良が意思表示のため、鞄の取っ手に巻いたテープのこと。	赤テープとは1970年代後半から1980年代の不良の間で「ケンカ売ります」という意思表示のため、学生カバンの取っ手に巻かれた赤いテープのことである。これは赤テープの色が攻撃的なこと、また信号で赤は「止まれ」にあたることから「ケンカを売られたくなければ来るな(止まれ)」という意味で使われた。ちなみに逆の「ケンカ買います(=喧嘩上等)」という意思表示には白テープ、信号の「止まれ」に対して「進め(=ケンカ買うから売りたいなら進んで(突っ込んで)来いの意)」で青テープが使われた。ただし、エリアによっては赤テープが「ケンカ買います」で白テープが「ケンカ売ります」であったり、白テープは白旗同様「降参(ケンカしません)」の意思表示であったりと、エリアによって意味や色が異なる(上記以外の色が使用されるエリアもあり)。 1981年にはロックバンド“横浜銀蠅”が『ツッパリ
\\	(登校編) 
\\	の歌詞中に赤テープという言葉を使ったことから、ツッパリや若者以外にも広く認知されたが、現在では赤テープを巻いて歩く不良自体が存在せず死語となっている。	1981年	不良 ・ ヤクザ用語	
\\	えぐい	えぐい	「気色悪い」「きつい」また「すごい」こと。	えぐいとは本来「あくが強く、喉や舌を刺激するような味がする」という意味(この意味では平安時代から使用)だが、これが転じ「気色悪い ・ 気味が悪い」更に「(気色悪いほど)残忍な ・ 残虐な」「きつい」「きびしい」「つらい」など様々な意味で使われるようになる。1981年に女優:中原理恵がコンタック600の
\\	で「えぐいんじゃないの~」と言ったことから流行語になった。また、同じ頃えぐいは本来の意味とは全く異なる「(いい意味での)凄い」「かっこいい」という意味でも使われるようになる
\\	自体は本来の意味で使用していたため関連性はないと思われる)。現在もえぐいは様々な意味で若者を中心に使用されるが本来の意味を知っている人は減少している。	1981年	若者言葉	
\\	かまぼこブス	かまぼこブス	不美人な女性のこと。	かまぼこブスとは意味的にはブス同様、不美人な女性を意味する。ブスの前に付いている蒲鉾(かまぼこ)は板にのせた状態で売られている。ここから板についたブス(=ブスがいたについている)といったニュアンスでかまぼこブスという。意味合い的にはブスよりもきつい表現だが、実際には愛嬌 ・ 親しみをもって使われた。ただし、近年ほとんど使われない死語となっている。	1981年	若者言葉	
\\	精神ブス	せいしんぶす	容姿はいいが性格が悪い人(女性)のこと。	精神ブスとは容姿がよく、周りから「美人」と言われることで図に乗っている、感じの悪い女性や悪くなってしまった女性のことである。精神ブスは1980年代から使われ始め、単に「美人だけど性格が悪い人」といった意味でも使われるようになる。『ブス』が女性対象の言葉であるため、精神ブスも女性を対象に使われたが、次第に男性に対しても使用されるようになる。 ※対義語である『性格ブス』と文字からくるイメージが似ており、混同して使われることも多いので、上記とは異なった意味で使われている場合があります。(性格ブスに関しては関連ページ参照)	1981年	若者言葉	
\\	粗大ゴミ	そだいごみ	定年退職後の夫のこと。	粗大ゴミとは定年退職後の手間がかかり邪魔な夫に対する嫌みを込めた例えで、評論家の樋口恵子が主婦から聞いた粗大ゴミという表現を1981年に新聞で紹介したことから広まった。また、粗大ゴミは1990年に発売された『拝啓「粗大ゴミ予備軍」殿―30代 ・ 40代夫婦の生の声』や2003年発売の東海林のり子著『夫を粗大ゴミにしない銀の法則 
\\	といった書籍のタイトルにも使用されている。	1981年	
\\	タイマン	たいまん	1対1のケンカのこと。	タイマンとは1対1のケンカのことで、1980年代に当時『ツッパリ』と呼ばれる不良少年が好んで使った言葉である。 当時は学校同士やグループ同士のケンカの際、最後は相手の権力者(大将 ・ 番長)とタイマンでケリをつけることが美徳とされた。 また、タイマンは本宮ひろ志の漫画など当時の不良やツッパリ漫画でよく使われた。	1981年	若者言葉、不良 ・ ヤクザ用語	
\\	ダチ公	だちこう	仲間 ・ 友達のこと。	ダチとは「友達」の略で、意味は友達と同様である。ダチは当初、的屋や不良の間で使われるようになる。昭和時代に入った辺りから人を侮辱するときにつける接尾語『公』をつけたダチ公という呼び方も使われる(ただし、この場合の『公』は特に侮辱の意を込めているわけでなく、むしろそれだけ気軽に接する親しい仲を意味している)。このダチ公がツッパリブームに乗って1970年代末から若者に浸透(1974年に公開した松田優作主演映画『あばよダチ公 
\\	の影響という説もある)。1980年代後半にはダチ公という呼び方が古臭いというイメージを伴い死語化。逆にダチが不良以外の若者にも使われるようになる。	1981年	略語	
\\	袋	ふくろ	
\\	袋叩きの略。 
\\	池袋の略 。	
\\	袋とは袋叩き(ふくろだたき)の略で「ふくろ」と読む。1970年代末から不良の間で普及。ツッパリブームに頻出した言葉のひとつである。動詞として使う場合は「袋にする」となる。 
\\	袋とは東京副都心のひとつである池袋(いけぶくろ)の略で「ぶくろ」と読む。袋と略した読み方自体は戦後まもなく不良が好んで使っていたが、平成時代に入るとブクロとカタカナ表記で若者全般に浸透した。	1981年	若者言葉	
\\	マブダチ	まぶだち	親友のこと。	マブダチとは「本物の」「本当の」を意味する形容詞の『まぶ』と「友達」を意味する略語『ダチ』から成る合成語で「親友」を意味する。マブダチは『まぶい』同様、もともとは的屋が使っていた隠語であった。不良文化がマンガや音楽などで大衆化する1970年代後半のツッパリ ・ ブームを機に若者を中心に広く普及。 マブダチをタイトルにした商品には1983年に横浜銀蝿ファミリーの杉本哲太がシングルマブダチを、同じく1983年に車田正美がコミック“真友仁義(真友と書いてマブダチと読む)”を、新しいところでは2004年に氣志團が“結婚闘魂行進曲「マブダチ」”を発売している。	1981年	不良 ・ ヤクザ用語	
\\	メダカブス	めだかぶす	不美人な女性をからかう言葉。	メダカブスとはブスを強調した言葉で、メダカが網や手ですくい上げにくい魚であることから、「救いようがないブス」という意味で使われる。また「救いようがないブス」を「考え方(性格)が悪い(ブスであることで卑屈になったり、逆に開き直ったりすること)」と捉え、『性格ブス』という意味でもメダカブスは使われる。(性格ブスに関しては関連ページ参照)	1981年	若者言葉	
\\	カセットテープ	かせっとてーぷ	便所のこと。	カセットテープとは音楽や議事録など音声を録音 ・ 再生するための媒体のひとつだが、その活用の主である録音は「音入れ」ともいう。ここから『音入れ=おトイレ』というもじりで、トイレのことをカセットテープと呼ぶようになる。例えば接客業など、面と向かってトイレに行くことが告げられないとき、「カセットテープを取ってきます」と言い、他の従業員に接客を交代してもらうといった形で使われた。ただし、現在ではカセットテープ自体が
\\	や
\\	レコーダーといった媒体に代わっており、死語となっている。	1980年	
\\	族	ぞく	暴走族の略。	族とは暴走族の略で、カタカナ表記でゾクと書かれることも多い。不良仲間など一部の若者の間では1970年代末から族と略して使われていたが、1980年より週刊誌などメディアが使うようになり、一般にも浸透。後に『ゾッキー』とも言われるようになるが、『ゾッキー』は第三者のみが使うのに対し、族は当事者(暴走族に所属する人や元暴走族など)も使う言い回しである。	1980年	若者言葉	
\\	団塊の世代	だんかいのせだい	終戦後(第二次世界大戦後)のベビーブームに生まれた世代のこと。	団塊の世代とは終戦後のベビーブームに生まれた世代のことで、堺屋太一が命名、『団塊の世代』というタイトルで1980年に書籍化もしている。団塊の世代をベビーブームという観点から正確に捉えた場合、1947(昭和22年)から1949年(昭和24年)の3年間に生まれた806万人という人口の塊を指すことになるが、堺屋太一の定義を含む一般的な解釈では、学生生活や家庭 ・ 地域生活をともにしたベビーブーム後の2年間を含む1951年(昭和26年)生まれまでを意味する(ちなみに1946年に対し1947年の出生数は59.6
\\	増)。 2006年現在、団塊の世代といわれる人たちの年齢は55~58歳。世代人口が突出して多いうえに、本物志向が強い世代であることから、団塊の世代をターゲットにした商品やサービスにビジネスチャンスがあるということ、また逆にこれら多くの人がまもなく定年を迎える時期であることから、その穴を埋めるためにどうすべきかといった問題の浮上(特に定員制の公務員は深刻で、団塊の世代が退職後、東京の小学校の教員採用倍率は1倍台になるともいわれている)で団塊の世代という言葉も頻繁に使われるようになる。	1980年	
\\	中坊 ・ 厨房	ちゅうぼう	中学生のこと。	中坊とは中学坊主の略で、中学生を意味する。1970年代後半から不良の間で使われていた中坊は学生を中心に一般にも普及。「中坊の分際で」といったセリフに見られるように、中学生を子供扱いする際によく使われる。 また、掲示板サイト2ちゃんねるを中心にインターネットでは中坊(ちゅうぼう)と読みが同じであることから、台所 ・ 調理場を意味する厨房と書く傾向にある(誤変換によるものいわれる)。ちなみに高校生は工房、小学生を消防という。	1980年	
\\	バックレる	ばっくれる	しらばっくれるの略。	
\\	ばっくれるとは「とぼける」「白を切る」「しらんふり」といった意味で、主にヤクザや盗人、不良の間で使われる言葉である。 
\\	ばっくれるは
\\	の意味から転じ、「逃げる」といった意味でも使われるようになる。更にそこから「学校を無断で早退する(サボる)」といった意味で若者に使われた言葉である。	1980年	不良 ・ ヤクザ用語	
\\	フケる	ふける	逃げること。授業を途中で抜けてサボること。	ふけるとは逃げる、行方をくらます、駆け落ちするといった意味で江戸時代から使われた言葉である。逃げる、行方をくらますという意味では主に盗人の間で使われたが、1970年代末から1980年代のツッパリブーム時になると、その派生として授業を途中抜けしてサボるという意味で不良を中心に若者の間で普及。この場合、フケるという表記が好んで用いられた。	1980年	若者言葉	
\\	ぶりっ子	ぶりっこ	男性の前でかわいい子ぶる女性のこと。	ぶりっ子とは男性の前で無知や非力を装うなど、おとなしくてかわいい子を演じる女性のことで、タレントの山田邦子が「かわい子ぶりっ子、ぶるぶるぶりっ子」というギャグで使用したことがキッカケで1980年の流行語となった。また当時、歌手の松田聖子がぶりっ子の代表としてあげられ、元祖ぶりっ子とも言われる。更にぶりっ子は先生や親、先輩や上司など目上の者の前で良い子(目上の者が好むであろう態度)を装う者にも使われるようになる。ただし、この場合も女性を対象に使用される。(『いい子ぶりっ子』になると男性にも使用される場合がある)	1980年	若者言葉	
\\	マジ	まじ	真面目(まじめ)の略。	マジとは真面目の略で、「真面目」「本気」「真剣」「冗談ではない」といった意味で使われる。マジは江戸時代から芸人の楽屋言葉として使われた言葉だが、1980年代に入り、若者を中心に広く普及。「マジで」「マジに」といった副詞として、また「マジ
\\	(下記関連語参照)」といった形容詞的にも使われる。 また、「本気」と書いてマジと読ませるマンガなどもある。	1980年	若者言葉	
\\	マッチョ	まっちょ	筋肉質な体型のこと。(俗語的解釈)	マッチョとは「男らしい ・ 男気のある」という意味のスペイン語“
\\	(英語でも意味 ・ 綴りは同様)”からきたカタカナ語である。ただし、日本でマッチョといった場合、一般的には体型に対して使われ、ボディービルダーのように筋肉質な体つきを指す言葉になっている(近年、蔑称としても使われているため、筋肉美という表現は避けています)。特に女性ビルダーが増えてからは「マッチョな女」といったように筋肉質な体型を指し、男らしいかどうかとは関係なく使われている。また、マッチョな人という意味の『マッチョマン』という言葉も普及したが、現在は死語となっている(2006年3月に東芝
\\	からマッチョマンがデビューしている)。ちなみにマッチョマンは男性に限定して使われた言葉である。	1980年	カタカナ語	
\\	ワザトラマン	わざとらまん	言動がわざとらしい人のこと。	ワザトラマンとはいかにも言動が不自然でわざとしたようであるという意味の『わざとらしい』と『ウルトラマン』から成る言葉で、言動がわざとらしい人を嘲う言葉である。また、同じグループ、同じクラスなど、身近にこういった複数人いる場合、ワザトラマンエース、ワザトラマンセブンといった具合にウルトラ兄弟の名前を付けていくこともある。	1980年	合成語	
\\	イマイチ ・ 今一	いまいち	求めていたものより少し足りないさまを表す言葉。	いまいちとは「今一つ(いまひとつ)」の言い換えで、期待している(願っている)値や結果、状況に対し、少し不足していて物足りないさまを表す言葉である(例:いまいちの出来 ・ 彼がいまいち好きになれない)。いまいちは1970年後半に若者の間で普及。直に世代を超え、広く使われるようになる。また1980年代に入ると、今二(いまに) ・ 今三(いまさん)~今百 ・ 今千といったように「今」の後に続く数字を増やすことで物足りなさの度合いを表した。	1979年	
\\	イモい	いもい	田舎臭い ・ 野暮ったいこと。	イモいとは田舎者やセンスがない人を意味する俗語で野菜の『いも』に形容詞化する接尾語『い』をつけたものである(イモ臭い ・ いもっぽいの略とも考えられる)。つまりイモいとは田舎臭い、野暮ったいといった意味の若者言葉である。また、平成に入ると単に「かっこ悪い」「たいしたことない」といった意味でも使われるようになる(ダサいと同義に使用)。 イモいの使用例: 今い人よりイモい人の方が私はホッと出来て好きだな。 あんなイモい曲なら俺でも作れる。	1979年	若者言葉	
\\	江川る	えがわる	強引に物事を進めること。ゴリ押し。	江川るとは元プロ野球選手(読売巨人軍:ピッチャー)であり、野球解説者である江川卓の苗字に動詞化する接尾語『る』を付けたもので、強引に自分の意見を押し通す。物事を強引に進める。ゴリ押しといった意味で使われる。これは1978年のドラフト問題、1979年のトレード問題~巨人軍入団に至るまでの一連の騒動(江川問題ともいう)やそれ以前の江川氏の言動を受けて出来た言葉である。	1979年	
\\	ガメ子	がめこ	がめつい女性のこと。	ガメ子とは利益に対して抜け目なく、押しが強いという意味の「がめつい」を女性の人名っぽくしたもので、がめつい女性を指して使う。(がめついに関しては下記関連語を参照)	1979年	若者言葉	
\\	きもい	きもい	気持ち悪いの略。	きもいとは気持ち悪いの略で、生理的に気持ち悪いというより、見た目が気持ち悪い場合に使われることが多い。きもいは1970年代後期には既に存在した言葉だが、若者を中心に会話の中での使用頻度が増したのは1990年代後期に入ってからである。きもいはカタカナのキモい、キモイといった表記も使われる。また、きもいの前に語意を強くする『超(ちょー)』をつけ、『超きもい』という形で使われることも多い。類義語に『きしょい』があるが、これは気色悪いからきている言葉できもいよりも強い意味合いがある。 きもいの使用例: あの人、いつも何考えてるかわかんなくてきもいよね。 今度のクラス、きもいヤツが多過ぎる。	1979年	略語	
\\	ギャル	ぎゃる	若くて元気のいい女性のこと。	"ギャルとは米国スラング
\\	""からきたカタカナ語で
\\	(ガール)""とほぼ同意である。
\\	""は1970年代から米国で使われた言葉で、1972年にジーンズメーカーの
\\	(ラングラー)が女性用ジーンズを
\\	""というネーミングで売り出したのが始まりである。 ギャルが日本で広まったのは1979年で、まず若者の間で流行し、世代を超え広く普及するのは1980年代に入ってからである。特に1980年代後半には『ピチピチギャル』など
\\	ギャル』という言葉が数多く作られ一気に浸透した。1990年代に入りギャルは一旦死語扱いされるものの、1997年に『コギャル』など
\\	ギャル』という言葉が流行。これ以降、ギャルはコギャル(を含むガングロ、ヤマンバなどコギャル系)に対して使われるようになり、当初のギャルとはイメージも意味合いもかなり異なる。また、
\\	""といったギャル雑誌もある。 ちなみに英語圏のホワイトカラーの間では、
\\	""を秘書やアシスタント、
\\	""をガールフレンドや娘といった使い分けをする。
\\	1979年	若者言葉(カタカナ英語)	
\\	グリコ	ぐりこ	ジャンケンのグーのこと。お手上げのこと。	
\\	グリコとはジャンケンで勝った者が出した手に応じて進むゲームの際にグーを意味する。また、エリアによってはこのジャンケンゲーム自体をグリコという。ちなみにグーのグリコは3歩、チョキはチヨコレートで6歩、パーはパイナツプルで6歩進めるという遊びである(エリア ・ 世代により、ルールが異なる)。 
\\	グリコとはお手上げのことである。これは江崎グリコのキャラクターであるランナー(商品パッケージや大阪道頓堀の看板に見られる)が両手をあげていることによる。ちなみにこのキャラクターは1923年(大正12年)から描かれているが、グリコがお手上げという意味で使われたの1970年代末になってからである。ただし、現在、こちらの意味での使用はほとんどない。死語。	1979年	幼児語 ・ 若者言葉	
\\	シャコタン	しゃこたん	車体を低くした改造車のこと。	シャコタンとは漢字表記の車高短と書くとわかる通り、車高を低く改造した車のことである。シャコタンは1970年代末から主に暴走族が使用。1980年代中頃には暴走族以外にも浸透する。車高を低くする主な方法はスプリングの改造や取り外しによるもので、直径の小さいタイヤに交換したり、車体上部を改造することで低くしたものはシャコタンと呼ばないことが多い。こうしたシャコタンは外観の向上及び、重心を下げることによる安定性のために行われる。	1979年	若者言葉	
\\	省エネルック	しょうえねるっく	石油節約を目的とした夏の軽装スタイルのこと。	省エネルックとは第2次石油ショック後の1979年(昭和54年)6月に大平正芳内閣が提唱した夏の軽装のこと。 省エネルックで推奨された『半そでスーツ』はほとんど普及することもなく、省エネルック自体も浸透せずに終わったが、1994年(平成6年)、羽田孜元首相が半袖スーツを愛用したことで一時的に話題になった。 2005年(平成17年)に地球温暖化防止を目的とした同内容の運動を行うに際し、省エネルックに変わる言葉を公募。クールビズと命名される。	1979年	政策語	
\\	スケバン	すけばん	女番長のこと。	スケバンとは「女」を意味する俗語『助(すけ)』に「番長」の略をつけたもので、女番長(女子不良グループのリーダー)を意味する(このためスケ番という表記も使われる)。ただし、スケバンは番長と違い、必ずしもリーダー格の女性を指すわけでなく、「スケバングループ」という言葉もあるように単に不良ぶった風貌の女子を対象に使われることが多い。そんなスケバンの特徴はロングスカートにパーマのかかった髪、薄く潰した学生鞄などがある。1985年には斉藤由貴主演のドラマ『スケバン刑事 (後に南野陽子、浅香唯主演の続編も放送)』により、世代を超えて認知されるが、平成時代には死語となる。2005年にお笑い芸人“桜塚やっくん”のネタ『スケバン恐子』のブレイクで再びスケバンという言葉も普及する。	1979年	
\\	タメ口	ためぐち	対等な言葉使いのこと。	タメ口とは「対等」「同じ」を意味する俗語『タメ』に『口(口ぶり)』をつけたもので、対等な言葉使い、つまり友達口調を意味する。『タメ』は元々賭博用語だが、これが不良少年に広まる過程でタメ口という言葉は生まれた。こうして当初は不良少年が好んで使ったが、1970年代末のツッパリ ・ ブームで若者全体に広まる。タメ口は後輩など目下の者に「タメ口でいいよ」といった好意的なものから「なれなれしい言葉使い」という意を含み「タメ口きいて(使って)んじゃねえぞ」と敵意を表すものまで様々なシーンで使われる。	1979年	若者言葉	
\\	タメ年	ためどし	同じ年のこと。	タメ年とは「同じ」「対等」を意味する俗語『タメ』に『年』をつけたもので、「同じ年」を意味する。(タメに関しては関連ページ参照) タメ年はもともと不良少年の間で浸透した言葉だが、1970年代後期、マンガや音楽などで不良文化が普及する(ツッパリ ・ ブーム)中、若者全体に広まった。	1979年	若者言葉	
\\	チクる	ちくる	告げ口や密告をすること。	チクるとは告げ口や密告をすることで、ちっくるともいう。「ちく」が「口(くち)」の倒語でチクるというようになったというもの、また「ちくりと言う」の「ちくり」が変化したものという説がある。当初、チクるはツッパリブームの不良少年など若者の間で普及した言葉であったため、先生や先輩、親などへ告げ口をするという軽い意味合いであった。しかし、当時の若者が社会人になってからも使用。上司や警察、報道機関などへ密告するという意味合いでも使われるようになる。また、若者の間では名詞形の『チクリ』という形でも使われる。	1979年	
\\	チョメチョメ	ちょめちょめ	伏せ字を音声化(単語化)したもの。	チョメチョメとは文章中に出てくる
\\	などを使って表される伏せ字を音声化したものである(例:私の上司は××である⇒私の上司はチョメチョメである)。チョメチョメは1979年から1985年にフジテレビで放送されたクイズ番組『アイ ・ アイゲーム』が生んだ流行語である。同番組はある空白のある文章を穴埋めするもので、一般視聴者とその視聴者が選んだ芸能人の穴埋めの解答が最も近い人が正解という内容であったが、司会の故:山城新伍が出題する際、この空白部分をチョメチョメと読んだことから流行った。後に男女がいちゃいちゃしたり、男女の性交のこともチョメチョメというようになる。	1979年	
\\	チョンボ	ちょんぼ	うっかりして間違えること。ミス。	ちょんぼとはもともと麻雀用語で、役が揃ってない牌の状態で間違ってあがってしまうことをいう。これを中国語では錯和(ツァホウ:間違ったあがり)または冲和(チョンフォウ:虚しいあがり)といい、このいずれかが変化してちょんぼになったとされる。ここから1970年代後期にはうっかりミスや間違いのことをちょんぼというようになる。また一部では万引きという意味でも使われる。	1979年	
\\	ツッパリ	つっぱり	若者が虚勢を張ること。	ツッパリとは若者が虚勢を張ったり、いきがること。その多くは非行 ・ 不良行為で具現化されたため、非行 ・ 不良行為をする少年 ・ 少女をツッパリ兄ちゃん、ツッパリねえちゃんと呼び、そういった少年少女の総称としてもツッパリは使われた。ツッパリの特徴は丈の長いズボンやスカート、刺繍の入った学生服など規定外の制服を着用、男子は剃りやリーゼント、女子はパーマといった髪型が多かった。ツッパリの多くは、些細に思える原因で喧嘩を始めたり、喫煙、飲酒、バイクでの暴走といった違法行為を行っていた。更に暴力団と関係をもち、薬物を乱用する者や恐喝、万引きといった行為をする者もあった。ツッパリは1970年代から1980年代に使われた言葉で、1980年代末から1990年代にはヤンキーという言葉がそれに取って代わった(ツッパリとヤンキーは、厳密にはタイプが異なります。ヤンキーについては関連ページ参照)。	1979年	若者言葉	
\\	どつぼ	どつぼ	深く落ち込んだ状態や最悪な状況のこと。	どつぼとは深く落ち込んだ状態や最悪な状況を意味し、そういった状況になるという意味の「どつぼに嵌まる(はまる)」といった形で使われることが多い。どつぼはもともと関西エリアで肥溜め(肥溜めは野にあることから野壷ともいい、それが音的に崩れたものか?)のことをいうが、関西芸人が最悪な状況を肥溜めにはまった状況に例え、楽屋言葉としてどつぼというようになったとされる(壷に閉じ込められ、落ち込む様をどつぼと言い出したのが最初という説もある)。一般には1970年代末辺りからよく使われるようになる。	1979年	
\\	ナウい	なうい	「今風の」「流行の」という意味の形容詞。	ナウいとは「今風の」「流行の」といった意味の形容詞で、その中でも最先端なものといった意を含んで使われることが多い。英語で「今」という意味の“
\\	に形容詞形にする接尾辞「い」をつけたものではあるが、ナウいが流行る1970年代終わりまでに形容動詞形の『ナウな』が流行っており、ナウいはこの『ナウな』の派生語と考えてよい。どちらにしても現代ではほとんど使われなくなっており、全くナウくない死語である。 【追記:2009年6月】 ナウいという言葉自体が古臭くなったことから、「流行に乗り遅れている」という意味合いで、あえてナウいと言う場合がある。	1979年	若者言葉	
\\	パープリン	ぱーぷりん	バカ ・ アホ ・ 気違いを意味する言葉。	パープリンとは漫画家:小島よしのり氏による造語で、アホ、バカ、気違いと同義に使われる言葉である。パープリンだけでそういった人を表すこともあるが、人を指して使う場合、パープリン ・ ギャル、パープリン野郎といった使い方をすることが多い。パープリンは小林よしのりが、漫画『東大一直線』(週刊少年ジャンプにて1976年~1979年に連載)で使われ、当時は流行語にもなったが、現在ほとんど使われない死語となっている。『東大一直線』連載当時、バカや気違いといった言葉が差別用語として非難されていたことから、回避策のための代用として造られた言葉という説もあるが、定かではない(調査中)。	1979年	若者言葉	
\\	ハマトラ	はまとら	横浜トラディショナルファッションの略。	ハマトラとは1970年代後半から1980年代にかけて流行した横浜ニュートラディショナルという女性ファッションの略で、
\\	など女性雑誌に取り上げられたことから、1979年にはハマトラと略した名称とともに全国的に広まった。ハマトラは横浜の名門『フェリス女学院大学』に通う生徒をイメージしたファッションで、上はポロシャツにベスト ・ カーディガン、下は巻きスカート、ハイソックス、かかとの低い靴といったスタイルが特徴。横浜の老舗ショップで揃えるのが正統派とされた。流行に左右されにくいハマトラには現在も根強いファンが存在する。	1979年	略語	
\\	ビビる	びびる	おじけづくこと。	ビビるとはおじけづくという意味で古く平安時代から使われた言葉である。江戸時代には舞台前に緊張や気後れから萎縮することを意味する楽屋言葉として芸人の間で使われ、特に関西エリアでは一般にも普及。1970年代後半のツッパリブームになると、ケンカ前に相手におじけづくことや悪事をする前におじけづくことを指し、不良少年の間で重用される。1980年代には世代 ・ エリアを超え、広く普及すると同時に、おじけづいている人や萎縮しやすい人を意味する名詞形のびびりも使われるようになる。	1979年	
\\	ファーストフード	ふぁーすとふーど	注文してすぐに供され、持ち帰り可能な食品。	"ファーストフードとは英語の
\\	""からきたカタカナ語で、意味も英語と同じく「注文して待たずに食べられ、持ち帰りも可能な食品」である。
\\	""ではなく
\\	""からきているので、本来は『ファストフード』というのが正しいが、日本ではイギリス英語の『ファーストフード』という発音で浸透している。ただし、イタリアのスローフード運動が日本でも浸透する中で、スローフードに対してファーストフードという発音はおかしいという意見から、意図的に『ファストフード』と呼ばれる場合もあるが、ファーストフードで完全に定着しているため、なかなか浸透の域にまでは至っていない。 ファーストフードの主なものには、ハンバーガー、ホットドッグ、牛丼などの丼物、サンドウィッチなどがある。ただし、日本では「持ち帰り可能」という部分が省略され、単に「早くて簡単に食事を済ませられるお店」という意味でも浸透しているため、駅の立ち食いそばなど更に広義に使われるが、ジャンクフードのイメージを逸脱したものには使われない。 ちなみに英語の
\\	""は食品自体に使用し、ファーストフードを扱うお店といった場合は
\\	""と表記する。
\\	1979年	カタカナ英語	
\\	女デート	めでーと	女同士のデートのこと。	女デートとは女性同士でするデートのことで、おデートをもじって出来た言葉である。1970年代末から80年代初頭に使われた。 女デートは女同士のデートと説明したが、女友達同士でランチをしたり、遊園地などへ遊びに行くことであり、レズビアンがするデートというわけではない。	1979年	若者言葉	
\\	夕暮れ族	ゆうぐれぞく	中年男性と若い女性のカップルのこと。	夕暮れ族は、吉行淳之介の『夕暮れまで』(1978年)という著書に描かれたカップルを取り上げて出来た言葉で、中年男性と若い女性のカップルを意味する。 また、1980年代初頭には売春買春を仲介する愛人バンク『夕暮れ族』というものも登場。発起人である筒見待子のルックスがよかったこともあり、マスコミでかなり話題になったが、1983年に摘発され解散している。 どちらの夕暮れ族からきているかは不明であるが、1980年代には「あの子、おとなしい顔して夕暮れ族なんだって~」といった形で、「金持ちの愛人になる若い女性」という意味でも使われた。また、平成に入ると中年女性と青年のカップルを意味する逆夕暮れ族という言葉も使われた。	1979年	若者言葉	
\\	ダサい	ださい	時代遅れ、野暮ったい、田舎臭いこと。	ダサいとは古臭い(時代遅れ)、野暮ったい、田舎臭い、かっこ悪いといった意味の形容詞で、1970年代から関東の若者を中心に普及した言葉である(一説には1950年代から一部の若者の間で隠語的に使われており、1970年代に広く浸透したともいわれる)。1980年代後半に入ると対語の『ナウい』が死語となるが、ダサいはそのまま使われ続け、現代に至っている。田舎を「たしゃ」と読み、ダサいになったとの説もあるが正確な語源は不明(語源については下記サイト参照)。 ダサいの使用例: いまどき、リーゼントなんてダサいよ。 裕二君って田舎から出てきたばかりでダサいとこあるけど、そこがかわいかったりするのよねえ。	1978年	若者言葉	
\\	窓際族	まどぎわぞく	第一線を退き閑職に就く中高年サラリーマンのこと。	窓際族とは出世ラインに乗れなかった中高年サラリーマンの中で、実質的な仕事を与えられず、遊軍的な立場に置かれた人を揶揄する言葉である。1977年(昭和52年)6月の北海道新聞のコラムで窓際族の元になったとされる『窓際おじさん』という言葉が使われている。当時、高度成長期に雇用した中高年層があふれ、ラインの管理職から外れた者は仕事も与えられず、窓際に追いやられたデスクで新聞を読んだり、外を眺めては時間を潰すという光景がみられた。そんな光景を北海道新聞は『窓際おじさん』という言葉でコラムに載せた。翌年の1月には日本経済新聞が新年連載「ニッポン ・ 生きる条件」で、
\\	の雑談中にあった言葉として窓際族を紹介している。 窓際族の『族』という言葉から複数人の集団をイメージするが、このような状況に追いやられた個人に対して使われることのほうが多い。窓際族は日本企業が終身雇用を謳っていたことから確立したポストであり、人員削減を中心としたリストラが行われるようになってからは、窓際族と呼べる悠長なものではなくなってきている。また、1980年代に入ると窓際族は学生の間で休み時間などに人と接しず、一人で時間を潰す生徒に対しても使われた。 窓際族といえば・・・: 窓際族といえば『負け組』とイメージする人も多いと思うが、必ずしもそうとは言えない。日本人初の世界規格である家庭用ビデオの
\\	というフォーマットは、当時日本ビクター株式会社の窓際技術者と呼ばれる人たちが、本社に極秘で進めたプロジェクトの中で生まれたものである。要するに、個々人の裁量、考え方如何で先は変わってくるわけである。 ※窓際技術者による
\\	誕生ストーリーは『陽はまた昇る』というタイトルで、西田敏行、渡辺謙、緒方直人といったキャストにより映画化もされている。	1978年	
\\	むずい	むずい	「難しい」の略語。	むずいとは「難しい(むずかしい)」を略したもので、1970年代末辺りから若者を中心に使用されるようになり、1980年代半ばに広く浸透。以降、若者の間で使われ続け、定着している。一部には「ムズムズする」「むず痒い」といった意味で用いられる場合もあるが、多くは「難しい」の略として、難しいと同意に扱われている。 むずいの使用例: このゲーム、マジでむずい。	1978年	若者言葉 ・ 略語	
\\	めんた切る	めんたきる	睨みつけること。	めんた切るとは睨みつけるという意味で、関西を中心に使われる言葉である。この場合の睨みつけるとは、主にケンカを売ったり、言いがかりをつけるために行うものを指す。めんた切るの語源については目玉を視線と捉え、「目ん玉切る」を略したものといわれているが、「目玉(視線)を切る」という意味なのか、「目玉(視線)で切る」なのか詳しいことは不明。 めんた切るは後に『めんたん切る』『メンチ切る』という形でも普及する。	1978年	若者言葉	
\\	落ちこぼれ	おちこぼれ	
\\	学校の授業についていけない学生のこと。 
\\	組織 ・ 集団についていけない人のこと。	落ちこぼれとは不登校生や引きこもりなど不良学生を指す言葉として、週刊誌などメディアが使った言葉である。1970年代から使われた言葉で、特にツッパリブームといわれた1970年代後半に広く普及。同時に、不良学生であるか否かに関係なく、授業についていけない生徒という意味合いで使われるようになる。また、学生だけでなく、社会人にも普及し、組織や集団についていけない人、成績の悪い社員を指して使われるようになる。	1977年	
\\	激~	げき
\\	「非常に」「とても」という意味の接頭語。	激とは「とても」「非常に」といった意味の接頭語で「劇的に」「刺激的なほど」といった強い意味で使われる言葉である。篠山紀信が1975年に雑誌の中で使った造語『激写』が流行、以来「激
\\	という言葉を使った新聞 ・ 雑誌 ・ 広告が増え、1970年代後半には一般にも使われるようになる。 「激」を使ったいろいろありますが、その中でもよく耳にする(耳にした)ものに下記の語句があります。 
\\	激うま:とても美味しいこと 
\\	激まず:とてもまずいこと 
\\	激辛:とても辛いこと 
\\	激かわ:とてもかわいいこと 
\\	激プリ:激かわに同じ。プリはプリティの略 
\\	激似:とても似ていること 
\\	激痩せ:突然、すごく痩せること 
\\	激安:とても安いこと 
\\	激レア:とてもレアな(珍しい、貴重、手に入り難い)こと	1977年	若者言葉	
\\	新人五月病	しんじんごがつびょう	新社会人がかかる病のこと。	新人五月病とは入社、転勤などで新年度の4月から新しい環境に置かれた社会人がかかる病気のことで、4月の緊張感が途切れた
\\	(ゴールデン ・ ウィーク)明け辺りから活力がなくなるものである。具体的症状は仕事に対する意欲の低下といった程度から出社拒否といったものまである。同じ症状で学生(新入生)を対象にした『五月病』という言葉があったが、社会人においてもそういった症状が見られるようになり、新人五月病(新五月病ともいう)という言葉が生まれる。しかし、数年後には新人五月病は使われなくなり、学生 ・ 社会人を問わず『五月病』でまとめられていく。	1977年	
\\	独身貴族	どくしんきぞく	独身の気楽なさまを羨む(または揶揄する)言葉。	独身貴族とはお金 ・ 時間が自由に使える独身者を揶揄する言葉である。もともと独身貴族は親元に暮らし、生活費は親任せで、給料を趣味や余暇、ファッションなど自分のために使える若い独身者を指した。しかし、既婚者から見れば、そうした金銭的な贅沢以外にも独身というだけで自由な点が多く、後に一人暮らしをする独身者や離婚して独身に戻った人を含め、独身者全般に対して使われるようになる。ただし、独身貴族はあくまで独身者を羨む(または揶揄する)言葉であり、病など当人の意思では解決出来ない事情で結婚出来ない人、親の介護など当人以外の問題が原因で生活に余裕のない人は独身であっても独身貴族に含まれない。	1977年	
\\	デコチャリ	でこちゃり	装飾をほどこした自転車のこと。	デコチャリとは「デコレーション」と自転車の俗称「チャリンコ」から成る合成語で、派手な装飾をほどこした自転車のことである。デコチャリは1975年~79年にかけて公開された映画『トラック野郎』を見て憧れた人たちがデコトラを真似たものである。デコチャリが登場した1970年代後半はあくまで自身のチャリ(自転車)をデコトラに近づけることに主眼が置かれ、トラックに使用されるパーツを着けたものが多かった(当時のデコチャリはトラック野郎シリーズ『トラック野郎 望郷一番星(1976年公開)』で見ることが出来る)。ただし、重さが原因で走行が出来ないものもあった。そこで1980年代に入ると装飾手法に変化が見られ、走行性を考慮したものが増えてくる。しかし、1990年代に入るとこうしたデコチャリが姿を消し、言葉も死語となる。ただし、現在もデコチャリの根強いファン存在し続けており、1990年代後半辺りから徐々にその数は増えているという。	1976年	合成語	
\\	デコトラ	でことら	デコレーション ・ トラックの略。	デコトラとはデコレーション ・ トラックの略で、豪華な電飾や派手なペイントで装飾されたトラックのことである(アートトラックともいう)。時代によりペイントの内容(龍や虎、観音像、芸能人、中にはディズニーまで様々)やパーツの部品に変化はあるものの、派手で豪華な装飾が多い。デコトラは当然公道でも見られるが、1975年~79年にかけて公開した菅原文太主演映画『トラック野郎』で見ることが出来る(この映画を見てデコトラにしたトラック運転手も多い)。ちなみにデコトラという言葉はプラモデルの大手メーカーである青島文化教材社が、トラック野郎封切の翌年(1976年)、こういったトラックをプラモデル化する際に作った造語である。	1976年	略語	
\\	サラ金	さらきん	消費者金融のこと。	サラ金とは「サラリーマン金融」の略で、サラリーマンなど個人を対象とする高利貸し業者のことである(サラ金=高利貸しというイメージがあり、低金利の消費者金融を対象にサラ金とはあまり言わない)。収入さえあれば無担保で即融資してくれることから、小口で気軽に利用する人が増えた。しかし、法定金利を超える高金利から返済出来なくなる人も多く、『サラ金地獄』という言葉とともに社会問題になった。現在、消費者金融という呼び方が一般的となり、サービスの向上や低金利業者の介入など質も変わってきている。	1975年	
\\	先様	せんさま	先生 ・ 教師のこと。	先様とは「先生」を略したものに敬称の「様」をつけたもので、意味はそのまま先生 ・ 教師となる(読みは「せんさま」。「さきさま」と読むと「先方」の敬称となる)。「様」がつくことで、尊敬の念をあらわす言葉にも見えるが、実際は『先公(センコー)』の流れで出来た教師を侮蔑した呼び方である。 また、1990年代には先ちゃんという呼び方も使われる。これは音楽ユニット“
\\	の二人が指導的立場にあったプロデューサーの奥田民生を先ちゃんと呼んだことから普及した。	1975年	若者言葉	
\\	ニュートラ	にゅーとら	ニュートラディショナルの略。	ニュートラとはニュートラディショナルという1970年代後半から1980年代に流行ったファッションスタイルの略称である。ニュートラは神戸のお嬢様風のファッションスタイルのことで、ワンピースにブレザーといったベーシックな格好にエルメスやグッチなど海外ブランドの小物、バッグを合わせるといったものである。これを女性ファッション誌
\\	がニュートラと紹介。都会の
\\	を中心に流行った。なお、同誌では横浜発信のニュートラディショナルを『ハマトラ』と呼ぶ。 近年、これと別に政府運営の就職支援サイトを指すことが増えている。	1975年	略語	
\\	ナウな ・ 
\\	な	なうな	「今風な」「いかした」という意味の形容動詞。	"ナウなとは「今」を意味する英語の
\\	""に形容動詞にする接尾語“な”をつけたもので、「今風な」「いかした」という意味で1970年代初頭から使われた。ナウなが使われ始めたときは英語のまま
\\	な』と表記されていた。1972年から『ナウな』というカタカナ表記が増えるとともに流行語として広まったが、ナウなの形容詞形である『ナウい』と入れ替わる形で1980年代には使われなくなる。 また当時、ナウなは『ヤング』と組み合わせ『ナウなヤング』というセットで流行り、セットで死語となった。
\\	1972年	若者言葉	
\\	アイ
\\	あい	中絶または出産を意味する隠語。	男女交際をするうえでの性的進度を表す“
\\	という言葉があるが、この中の“
\\	は妊娠を意味する。さらにこの
\\	の文字はお腹が大きくなった妊婦を横から見た形にも見える。ここから
\\	に対し、
\\	の膨らんだ部分を取り去った形のアルファベット“
\\	を中絶という意味で付け加えられる。“
\\	はもともと女学生の間で隠語として普及し、後に男子学生の間にも広まる。また、学生の妊娠が即中絶ではなく、出産という可能性も出てくるようになってからは“
\\	も出産の意を含むようになる。	1971年	
\\	ウハウハ	うはうは	上機嫌なさまをあらわす言葉。	ウハウハとは上機嫌なさまや嬉しいさまをあらわすもので、1970年に放送されたハウス食品「ジャワカレー」の
\\	(故:伊丹十三、宮本信子夫妻が出演)で流行した言葉である。例えば、思わぬ収入が入ったことを「財布がウハウハ」。そういった気分をウハウハ気分といったように使う。また、性行為のことをウハウハという場合もある(例:今日は彼女とウハウハ)。	1970年	
\\	一ちゃん農業	いっちゃんのうぎょう	おじいちゃん、おばあちゃん、おかあちゃんのいずれか一人でする農業のこと。	働き盛りの男性が出稼ぎで不在となり、残された者のみで行う農業を『三ちゃん農業』と呼んだが、1960年代後半に入ると三ちゃんの中の一人だけで農作業を行う農家が現れ、これを『三ちゃん農業』のもじりで一ちゃん農業と呼んだ。 一ちゃん農業はおじいちゃん、おばあちゃんが体力的に限界で働けなくなり、おかあちゃん一人で作業をする農家を指すことが多い。他におかあちゃんはパートへ、更におじいちゃん、おばあちゃんのいずれかが体力的限界となり一人で作業を行ったというパターンもある。	1968年	
\\	五月病	ごがつびょう	主に新入生や新社会人がかかる病(症状)のこと。	五月病とは入学や入社、転勤などで新年度の4月から新しい環境に置かれた人がかかる病気のことで、4月の緊張感が途切れた
\\	(ゴールデン ・ ウィーク)明け辺りから活力がなくなるものである。具体的症状は勉学や仕事に対する意欲の低下といった程度のものから出社拒否 ・ 登校拒否といったものまである。ただし五月病は医学用語ではなく、病とはいうものの、特に定義付けがされている病気ではない。五月病は一人暮らしを始めた大学生に多い症状であったため、もともと大学の新入生を対象に使われたが、後に新社会人に対しても使われるようになる。	1968年	
\\	ノンポリ	のんぽり	政治や学生運動に無関心であったり、関与しないさま。	ノンポリとはノンポリティカル
\\	の略で、政治に関心がないこと、またはそういった人を意味する。全共闘など学生運動が盛んに行われた頃の言葉で、政治に関心があってもこうした過激化する学生運動に関与せず、どの党派にも属さないような学生も指し、ノンポリ学生という言い回しも使われた。 しかし、当事のような政治に深く関心を持つ若者が少なくなった現在、無党派という意味で使われる程度で、あまり使われない死語となっている。	1968年	和製英語、略語	
\\	ハレンチ	はれんち	イカす、カッコイイといった意味の感嘆詞。(俗語的解釈)	ハレンチとは本来、破廉恥と書き、恥を恥と思わないこと(恥知らず)といった意味の言葉だが、カタカナ表記のハレンチの場合、イカす、カッコイイ、ニクい、サイコーといった意味でも使われた。元々ゲイの間で使われた意味で、破廉恥な格好や行為を大胆 ・ 突飛と捉え、カッコイイの意を込めて「ハレンチ」と言っていたことからきている。永井豪がマンガ『ハレンチ学園』を連載(後に
\\	化、映画化もされている)したことから、広く普及、流行語にもなった。ただし、現在はカタカナ表記のハレンチも大半が破廉恥と同意に使われている。	1968年	若者言葉	
\\	アングラ	あんぐら	アンダーグランドの略。	アングラとは「地下の」「秘密の」といった意味の英語“
\\	の略で、秘密組織(地下組織) ・ 反体制運動や前衛演劇 ・ 前衛美術を指した。日本では1960年代後半から流行。『アングラ族』『アングラ映画』『アングラ喫茶』といった言葉も使われた。ただし、インターネットの掲示板などで使われるアングラは違法アダルト画像、無許可のコピーソフトや音楽
\\	3ファイルなど、違法なものを扱ったサイト、グロ画像(死体画像など)、裏情報など、広く公に出来ないものを扱うサイトを指すことが多い(この場合、
\\	という略も使われる)。	1967年	略語	
\\	フーテン ・ 瘋癲	ふうてん	仕事も学業もせず、ぶらぶらしている人のこと。	フーテンとはもともと瘋癲と書き、精神状態が異常なこと及び、そういった人をさした。ここから1967年の夏、新宿東口に集まる長髪にラッパズボン、(妙なデザインの)サングラスといった格好をし、定職にも就かず、ブラブラしている無気力な若者集団をフーテン族と呼ぶようになる。瘋癲がカタカナ表記されたフーテンはこうしてアメリカのヒッピーに近いイメージで使われた。1968年には「フーテンの寅さん」で馴染みの「男はつらいよ」が
\\	ドラマでスタート(翌1969年には映画上映)。ただし、車寅次郎には的屋という定職がある。さらに正月 ・ 盆の稼ぎ時をむかえ、的屋業に戻るため、旅立つシーンで毎回映画は終わる。実はフーテンと呼ばれるような無職でもなければ、ぶらぶらしてばかりの無気力な人でもなかったりする。	1967年	
\\	マッチポンプ	まっちぽんぷ	自分で問題を起こしておき、自ら解決することで賞賛や利益を得ること。	マッチポンプとはマッチで火をつけておきながら、それを自らポンプで消すという意で、自ら起こした問題を自分で解決することをいう。1966年、自民党の代議士であった田中彰治は国会である問題を取り上げ、裏でその問題の該当者に金品を要求。相手がそれに応じると提議した問題を自らもみ消すといった事件があった。マッチポンプはもともとこの事件や田中代議士のあだ名として使われたものである。マッチポンプが単なる自作自演の問題解決でなく、それによって賞賛や利益を得ることが目的という意味合いが強いのはこのためである。例えば新聞記者が自ら事件を起こし、現場に最初に駆けつけた記者として記事に重みを付けるといった行為などがマッチポンプにあたる。	1966年	和製英語(英語+蘭語)	
\\	シェー	しぇー	驚いたときに発する感動詞。	シェーとは赤塚不二夫のマンガ『おそ松くん(1962~69年の週刊少年サンデー連載)』に出てくるイヤミが驚いたときにいう感動詞で、1963年から独特な両手のポーズを使った「シェー」が登場する。1965年にポーズも含めシェーという言葉が大流行、翌1966年には映画『怪獣大戦争』の中でゴジラがキングギドラをやっつけた際にシェーのポーズをしたことが話題となる。 ちなみにこのシェーという言葉が生まれたキッカケだが、原稿を取りにきた編集者に原稿が出来ていないことを告げたとき、「ひえ~」と発したのが編集者の前歯がなかったためシェーと聞こえたからだと後のインタビューで語っている。 【追加情報】 本文ではゴジラがシェーのポーズをとったことが話題になったと書きましたが、他にも浩宮徳仁親王(2005年現在の皇太子殿下)がこのポーズをとって話題になりました。1970年8月5日大阪万博でのエピソードで、当時まだ小学生だった浩宮徳仁親王が三菱未来館を見学されている最中に「シェー」のポーズを披露。関係者は突然の出来事に困惑したそうですが、新聞にも掲載され話題となりました。	1965年	若者言葉	
\\	インド人もびっくり	いんどじんもびっくり	驚いたときに発する言葉。	ンド人もびっくりとは何かに驚いたときに発する言葉として使われるもの。インド人もびっくりの語源は1964年に放送された
\\	の『特製エスビーカレー』の
\\	から。インド人に扮した芦屋雁之助がカレーを食べた後、あまりの美味さに飛び上がり「インド人もビックリ」と言うものであった。この
\\	と同時にセリフも流行し、現在でも驚いたときに出る言葉として用いられている。こういった経緯の言葉であることから、不意打ちなど、突然の出来事で驚く場合より、良くも悪くも予想以上の結果に驚くといった際に用いられることが多い。	1964年	
\\	ウルトラ
\\	うるとら ・ しい	「奇策」「大逆転技」「ものすごい」の意。	ウルトラ
\\	とは1964年に開催された東京オリンピックで生まれた言葉で、本来は体操の日本男子チームが生み出した難易度
\\	以上の技のことをいう(2009年現在、
\\	までの難度が設定されており、体操でウルトラ
\\	という表現は使わない)。これが流行語となり、大逆転技、奇策、さらに物凄いといった意味で用いられる。	1964年	
\\	みゆき族	みゆきぞく	日本で流行したファッション文化の一種。	みゆき族とは1964年に流行したファッションのことで、アイビー・リーグ(アメリカ合衆国東部の名門私立大学8校から成る連盟)の普段着を真似たもの(=アイビールックと呼び、創業者:石津謙介による
\\	ブランド(ヴァンジャケット社)が取り扱っていた)。こうした格好で銀座みゆき通り(御幸通り)をたむろしていたことから、そういった若者をみゆき族と言い、特に1964年4月末に創刊された週刊誌「平凡パンチ」が取り上げたことで話題となった。しかし、同年9月12日には周辺の苦情による一斉取り締まりが行われ、姿を消している。	1964年	
\\	録勉	ろくべん	録音して勉強すること。	録勉とは『録音』と『勉強』から成る合成語で、ラジオ講座や講義などを録音し、それを使ってする勉強を意味する。録勉は東芝がオープンリールのテープレコーダー『カレッジエース』を宣伝する際に使った造語である。1963年に「昔、ガリ勉。いま、録勉」というキャッチフレーズで紹介され、翌1964年に録勉が流行語となった。ちなみにカレッジエースは大卒初任給が18,930円であった当時に19,800円という高価なものであった。	1964年	
\\	一姫二虎三ダンプ	いちひめにとらさんだんぷ	危ないドライバーのタイプのこと。	一姫二虎三ダンプとは初夢で見ると縁起がよいとされる諺「一富士二鷹三茄子」のもじり(語呂的にはこちらだが、「一姫二太郎」からのもじりとも考えられる)で、危険なドライバーのタイプをあらわした言葉である。まず一姫は女性ドライバー。二虎の虎は酔っ払いをあらわし(下記関連語参照)、飲酒運転をするドライバーを意味する。三ダンプは文字通り、トラック運転手のことである。ただし、この言葉が使われ始めたのは1960年代始めであり、三番目にトラック運転手がくる辺り、そのまま現代に通用する言葉ではないかもしれない。	1963年	
\\	おーえる	女性会社員、女性事務員のこと。	
\\	とは
\\	の頭文字で女性会社員、女性事務員を意味する。ちなみに
\\	自体が和製英語である(欧米では性差別となるこうした言葉は好まず、男女ともに
\\	などが用いられる)。
\\	という言葉が使われるまでは
\\	の略で
\\	と呼んでいた。ただし英語では
\\	同様、
\\	のような性差別する言葉はなく、
\\	は娼婦、売春婦を意味する
\\	の略になる。この情報を受けて
\\	が
\\	を放送禁止に。また、東京オリンピック前に当たる1963年、来日する欧米人の誤解回避を目的に雑誌『女性自身』が代替語を募集。こうして
\\	から
\\	に変わり、定着した。ただし、
\\	には腰掛け、お茶汲みといったイメージが強いことから、
\\	と呼ばれることを拒む女性も多く、雇用上の男女平等が叫ばれるとともに使用が減っている。	1963年	和製英語+略語	
\\	カギっ子	かぎっこ	両親が共働きで留守の家に帰るため、いつも鍵を持ち歩く子供のこと。	カギっ子とは両親が共働きで帰宅の時間には留守になるため、外出の際、いつも鍵を持ち歩く子のことをいう。この言葉が出来た1963年、そういった子は学校へ行くときも遊びに行くときも、いつも首から鍵をぶら下げていたことからカギっ子と呼ぶようになる。団地族と呼ばれる都会の中流階級世帯によく見られた光景で、不良を作る原因として問題にもなった。何歳から何歳をカギっ子とするかは特に定義されていないが、教育者 ・ 行政 ・ 一般ともに大抵は小学生を対象としている。	1963年	
\\	三ちゃん農業 ・ 3ちゃん農業	さんちゃんのうぎょう	働き盛りの男性がいない農業のこと。	日本が高度経済成長に入った1960年代、それまで農業を営んでいた働き盛りの男性の中には東京など街へ出稼ぎに出る者、平日はサラリーマンで休日のみ農業を行うという者も多かった。こうして働き手を失った農村(農家)では残されたおじいちゃん、おばあちゃん、おかあちゃんが農業を行うことになる。この三つの『ちゃん』が行う農業ということから三ちゃん農業と呼ばれた。 1963年には国会でも「三ちゃん農業」という言葉が使われ、これを新聞が報道。同年の流行語となっている。	1963年	
\\	バカンス	ばかんす	長期休暇のこと。	バカンスとはフランス語の“
\\	で(フランス人のような)長期休暇の過ごし方、またはそういった過ごし方を意識した長期休暇を意味する(英語では
\\	。フランスでは労働者の権利として3週間以上の連続休暇(1982年以降は5週間以上)が認められているが、1960年代当時の日本では週休1日が一般的。連続休暇は盆 ・ 年末年始 ・ 
\\	(飛び石連休)程度であったため、実質的なバカンスは存在せず、2~3日程度の休暇を指した。ザ ・ ピーナッツのヒット曲『恋のバカンス』により、言葉自体は流行ったが、そういった事情からか、直に死語となっている。	1963年	カタカナ仏語	
\\	ハッスル	はっする	張り切る。頑張るの意。乗って(ノリノリで)いく。	"ハッスルとは「押し進む」「乱暴に押す」「押し売りする」「押し合い」といった意味の英語
\\	""からきたものだが、日本でカタカナ英語として使用する際は張り切る、頑張る、ノリノリでいくといった意味で使われる。日本球団のタイガースがアメリカへキャンプに行った際に持ち帰ったとする説。1962年に公開されたポール ・ ニューマン主演映画の「ハスラー」からきたとする説があるが、全国区で普及させたのはクレイジーキャッツの映画「クレージー作戦 ・ くたばれ!無責任」の中でハッスル ・ コーラの社員として登場した植木等が歌った『ハッスル ・ ホイ』という歌とされている。また、2004年になると、プロレスラーの小川直也がプロレスイベント「ハッスル」のために作った『ハッスルポーズ(腰に腕を添え、「ハッスル、ハッスル」と言いながら、腰と腕を振るもの)』が話題~流行となった。
\\	1963年	カタカナ英語	
\\	当り屋	あたりや	「当り」「当る」にちなんで複数の意味を持つ。	
\\	当り屋とは思った通り、順調に事が運んだ人や店のこと。努力の末というより、ついている人や店という意味合いが強い。明治時代末辺りから使われる。 
\\	当り屋とは野球で短期的に打率をあげている調子のよい選手のこと。昭和初期から使用。 
\\	当り屋とは走行中の車にわざとぶつかり、法外な示談金 ・ 賠償金を求める人のこと(主にこれを職とするような常習的な人をいう)。1962年には流行語になるほど頻繁に話題となった当り屋。最近では車に当たり、交渉しようと運転手が車を降りた隙に乗り逃げするという手口もある。	1962年	
\\	調	しーちょう	調子いい人や調子いいさまをあらわす言葉。	
\\	調とは「調子いい(ちょうしいい)」を「ちょうしー」読み、その前後をひっくり返したもの(しーちょう=
\\	調)で、軽薄で調子のいい人、またはそういったさまを表す言葉である。
\\	調はもともとその明るい調子
\\	調=ハ長調)にひっかけ、1960年代からジャズマンを中心とするミュージシャンの楽屋言葉として使われた。また、植木等が使用したことで、ミュージシャン以外の一般にも普及。更に1979年10月にはサザンオールスターズが5枚目のシングルタイトル
\\	調言葉に御用心』に使用したことで広く浸透。しかし、1990年代には使用する人も減り、死語となっている。	1962年	若者言葉	
\\	トサカにくる	とさかにくる	かっとなること(腹立たしく感じること)。	腹立たしく感じ、カッとなることを『頭に来る』というが、トサカにくるとは、この『頭に来る』を強めた言葉である。トサカ(漢字では鶏冠)とは鶏の頭にある赤い部分のことだが、カッときて頭に血が上った様をトサカに例えて出来た言葉と思われる。トサカにくるは映画『お姐ちゃんシリーズ』の中で中島そのみが使ったことから流行った言葉である。	1961年	
\\	ラリる	らりる	シンナーなどの薬物で酔ったようなフラフラの状態になること。	らりるとは睡眠薬 ・ シンナー ・ 麻薬などの薬物によって酔ったようなフラフラの(意識がとぶ)状態になることで、主に不良の間で使われたが、次第に若者、更に世代を超えて広く浸透。薬物に酔った状況下ではラ行がハッキリ発音出来ないことから、それをからかってラリるという言った。 平成時代に入ると上記の意味が転じ、寝不足などで頭が働かず、変な行動をとることややたら陽気になることもらりるというようになる。	1961年	若者言葉	
\\	六本木族	ろっぽんぎぞく	六本木に集う若者のこと。	六本木族とは戦後復興期を終え、高度成長期に向かう頃、六本木に集まった若者を意味する。当時の六本木は外国公館と進駐軍の街としてアメリカ兵の往来が目立つのが特徴だった。それに惹かれて六本木に集まったロカビリー族が後に六本木族と呼ばれるようになった。六本木はテレビ朝日、
\\	が近いことからテレビ関係者が多く、六本木族もそういったテレビ、芸能関係者が多かった。中でも女優の加賀まりこ、俳優の峰岸徹が有名。	1961年	
\\	オスタルジア	おすたるじあ	男性を恋しがること。	オスタルジアとは男性を意味する『雄(オス)』と、故郷を懐かしんだり、過ぎ去った時代を懐かしむといった意味の『ノスタルジア』から成る言葉で、(女性が)男性を恋しがることを意味する。また、ボーイフレンド募集中という意味でも使われる。	1959年	
\\	カミナリ族	かみなりぞく	暴走族のこと。	カミナリ族とは現代でいう暴走族のことである。これはマフラーを外し、バリバリと騒音を立てて走るさま、また車の間を縫うように走るさま(ジグザグ走行)を雷(稲光)に例えたものである。当時、特に暴走族と走り屋の区別がなかったため、騒音を出して走る車輌(主に二輪車)全てを対象にカミナリ族と呼んだ。後に走り屋の前身とする見方と暴走族の前身とする見方にわかれるが、多くは暴走族の前身と言われている。ただし、1950年代にバイクを所持出来る若者は裕福な家庭の子であり、格好はお洒落で暴力性もなく、暴走族とはイメージの異なる集団であった。どちらにしても、現代では死語になっている。	1959年	若者言葉	
\\	サッチョン族	さっちょんぞく	札幌で単身赴任すること。またはそういった人。	サッチョン族とは『札幌』と独身を意味する『チョンガー』を合成したもので、札幌で単身赴任すること。または札幌で単身赴任する人を意味する。1959年、目覚ましい発展のあった札幌を中心に、北海道へ進出する大企業が多く、転勤で札幌へ単身赴任する者が多数出来たことから流行した言葉である。	1959年	合成語	
\\	曲がり角	まがりかど	転換期 ・ 転換点のこと。	曲がり角とはある状況から異なる状況へ移行する転換期 ・ 転換点を意味する(比較的、上昇から下降、安定から不安な状態など、良い状況から悪い状況への転換点を指すことが多い)。例えば、肌の新陳代謝がピーク及び停滞時期から下降曲線になる時期をお肌の曲がり角(24~25歳といわれる)。リストラ ・ 大きな病気 ・ 離婚といった変化の節目を人生の曲がり角といった形で用いる。 1959年に公開されたマルセル ・ カルネ監督のフランス映画『危険な曲り角』がキッカケで流行した言葉である。	1959年	
\\	メスタルジア	めすたるじあ	女性を恋しがること。	メスタルジアとは女性を意味する『牝(メス)』と、故郷を懐かしんだり、過ぎ去った時代を懐かしむといった意味の『ノスタルジア』から成る言葉で、(男性が)女性を恋しがることを意味する。また、ガールフレンド募集中という意味でも使われる。	1959年	
\\	シビれる ・ 痺れる	しびれる	激しく興奮し、魅了されること。	しびれるとは本来、体の一部または全身が感覚を失い、自由に動けない状態。また、電気などに触れ、体がビリビリ震えることをいう。この後者の電気などに触れて麻痺したさまから転じ、音楽や芝居、容姿などに刺激を受け、非常に興奮したさまや魅了されたさまをしびれるという。三橋三智也の節回しに対して使われたのが初め。現在はほとんど死語となっている。	1958年	若者言葉	
\\	ながら族 ・ 乍ら族	ながらぞく	ラジオや音楽をかけながら勉強や仕事をする習慣の人。	ながら族とはラジオや音楽を聴きながら勉強したり、
\\	を見ながら食事をするといったように、何か他事をしながらでないと集中出来ない人をいう。こうした症状を日本医科大学の木田文夫教授が「ながら神経症」と名付けたことから生まれたながら族は広く一般にも浸透。当事は「集中出来ない若者」と、ながら族は問題視されていたが、近年、ながら族のほうが効率のよい頭の使い方をしているといった見解や脳の活性化に良いという意見もある。	1958年	
\\	シスターボーイ	しすたーぼーい	言葉使いや態度が女性っぽい男性のこと。	シスターボーイとは言葉使いや態度、仕草などが女っぽい男性のことで、アメリカ映画『お茶と同情』でジョン ・ カーが演じた女っぽい男性が“
\\	と呼ばれていたことから生まれた言葉である (アメリカ映画からきていることからカタカナ英語ともとれるが、米国ではそういった意味で普及しなかったため和製英語ともいえる)。シスターボーイはあくまで言動が女っぽくなよなよした男性や女っぽい美男子のことで、ニューハーフやオカマというわけではない。当時、シスターボーイは丸山明宏(現:美輪明宏)を指して使われた。	1957年	
\\	神風タクシー	かみかぜたくしー	スピード違反など、無謀運転をするタクシーのこと。	神風タクシーとは客の回転率をあげるため、スピード違反や信号無視など、無謀な運転をするタクシーのことをいう。命知らずな運転を神風特攻隊(神風特別攻撃隊)になぞらえた呼び名である。1956年、週刊新潮の記事で使われて普及。同記事によれば、東京に来た外人が日本のタクシーを見て神風タクシーと言ったとされている。1960年代に入ると、東京オリンピックに向け、交通面の取り締まりも厳しくなり、神風タクシーのようなタクシーはなくなっていった。	1956年	
\\	太陽族	たいようぞく	無軌道で不道徳な若者のこと。	太陽族とは大宅壮一が「太陽の季節」という作品から作った造語で、1950年代後半の流行語である。1956年、石原慎太郎(東京都知事)が「太陽の季節」(新潮社)で作家デビュー。同年1月には「太陽の季節」で芥川賞を受賞。また、弟の石原裕次郎も同年5月、「太陽の季節」の映画化の際に俳優デビューを果たしている。小説 ・ 映画ともにヒットし、そこに描かれる無軌道な若者を太陽族と呼んだ。太陽族映画と呼ばれるものに「狂った果実」「処刑の部屋」があるが、これらは「太陽の季節」とともに
\\	から上映制限運動を起こされるほど問題になると同時に話題となった。	1956年	
\\	ドライ	どらい	物事を割り切って捉える合理的 ・ 現実的なさま。	ドライとは「乾燥した」及び「禁酒」という意味の英語“
\\	からきたもので、物事を割り切って捉える合理的 ・ 現実的なさまをあらわす。これは“
\\	の「酒の入っていない」という俗的な意味からきたものである。『太陽族』とともに1950年代半ばから流行したドライは感情的にならず、現実的に判断する姿から、無感情 ・ 非情といった否定的な意味で使われることも多かった。	1956年	カタカナ英語	
\\	オランコリー	おらんこりー	恋人がいないことによる悩みや哀愁のこと。	オランコリーとは「居ない」が崩れた言い回し『おらん』と、「憂鬱 ・ 物思い ・ 哀愁」といった意味の英語
\\	(メランコリー)』の合成語で、恋人が居ないことによる悩みや哀愁、またはそういったさまを表す言葉である。オランコリーは1955年の流行語だが、1970年代にも学生の間で好んで使れている。	1955年	和製英語	
\\	コネ	こね	便宜 ・ 利益が見込める人 ・ 組織との親しい関係のこと。	コネとは関係 ・ 縁故といった意味の英語
\\	(コネクション)の略で、就職 ・ 進学 ・ 出世などで影響力があり、便宜をはかってもらえるような人 ・ 組織、または客として、情報源として利益をもたらしてくれるような人 ・ 組織 ・ 業界との親密な関係のことである。 例えば
\\	社の人事決定権を持ち、
\\	社への就職に便宜をはかってもらえるような人と親密な関係にある
\\	さんを
\\	さんは
\\	社にコネがある」という。	1955年	
\\	三種の神器	さんしゅのじんぎ	新しい生活をするうえで持っておきたい(持ちたい)とされる3種類の耐久消費財のこと。	三種の神器とは一歩進んだ生活をするうえで持っていたいとされる3種類の耐久消費財のことで、神武景気の消費ブーム時、家電販売のキャッチコピーとして白黒テレビ ・ 電気洗濯機 ・ 電気冷蔵庫を三種の神器とし、1955年に流行語となった。その後も時代に適した3種類の耐久消費財(主に電化製品)をあてて三種の神器は使われる。高度経済成長期にはカラーテレビ(又はピアノ) ・ クーラー ・ 車、2000年以降ではデジカメ ・ 
\\	レコーダー ・ 薄型テレビなどがある。 三種の神器とはもともと皇位継承のしるしとして天皇に受け継がれてきた3種の宝物(八咫鏡 ・ 天叢雲剣 ・ 八尺瓊勾玉)のことで、それを現代人の宝物という意味で上記の形で使われるようになった。更に三種の神器は単に3種の必需品という意味で使われたり(野球
\\	の三種の神器=メガホン ・ ハッピ ・ 帽子など)、各種メーカーやお店が人気商品3つを売り出すためのキャッチコピーとしても三種の神器は使われる。	1955年	
\\	ソーラー ・ ソーラー族	そーらーぞく	ミーハーよりまともな人のこと。	ソーラーとはミーハーの対語(類語)として生まれた言葉である。音階でソラはミファ(ミハ)より高いことから、「ミーハーよりも上(つまりミーハーよりまとも)」又は「ミーハーと違ってまとも」という意味で使われた。ソーラーは第三者が「彼(彼女)はソーラー」といった使い方よりも、当事者がミーハーを嘲う意を込め、自負的に使われた。また、まったく同じ意味でソーラー族ともいう(1台で暴走するバイクのことも暴走族と呼ぶように、特に集団でいなくてもソーラー族とよぶ)。	1954年	若者言葉	
\\	ヤメ検	やめけん	検事を辞めて弁護士になった人のこと。	ヤメ検とは一般的に検事を辞めて弁護士になった人をさすが、広義には定年退官者を含め、単に検察の任を辞した人をいう。ヤメ検=検察官を辞して弁護士となった人という意味合いで使われる理由は、実際、退職後に弁護士へ転進する人が一定数いること(検察官の多くが弁護士資格を有しており、弁護士には定年がないことから転進者が多い)、またそういった形で弁護士になった人は検察時代の人脈によって刑事事件に強いといった特徴から特にヤメ検と呼ぶ。ただし、弁護士の間では嘲ったり、見下した意味合いで使うことが多い。また、依頼人=お客という意識が通常の弁護士よりも低く、中には取調べ口調で接する人もあり、依頼人からの受けもよくないことが多い。	1954年	
\\	さいざんす	さいざんす	「そうです」「その通りです」のくだけた言い方。	さいざんすとは故トニー谷
\\	が使い、流行した言葉で、「左様でございます」を茶化し、下品な言い回しにしたものである。意味的には「左様でございます」同様、「その通りです」「そうです」「そのようです」となる。1950年頃からトニー谷の司会者としての人気があがり、さいざんすも流行語に。1953年にはビクターレコードから「さいざんす ・ マンボ」という曲を発表。翌54年には「さいざんす二刀流」という映画も公開された。 ※トニー谷:ソロバンを使った芸やカタコト英語のトニングリッシュが特徴のヴォードヴィリアン(司会、歌、映画、舞台など)。赤塚不二夫の「おそ松くん」に登場する「シェー」で馴染み深いイヤミのモデルでもある。1987年7月16日死去。	1953年	
\\	車用族	しゃようぞく	社用車や公用車を私用で使う人のこと。	車用族とは斜陽族のもじりである社用族から派生した言葉で、社用車(企業や会社組織で利用する車)や公用車(官公庁で使用する車)を私用目的で使う人のことをいう。社用族は社費を社用と称して使う人のこと(公用と称して接待を受けたり、公費を使用する人を公用族)だが、車用族はそんな社用族 ・ 公用族の一種にあたる。	1953年	
\\	プーバイ	ぷーばい	チケットを高値で転売してお金を儲けること。	プーバイとは売ることを目的に購入したチケットを、入手しそびれた人に高値(大抵は倍額以上)で売ってお金を儲けることで、未だとダフ屋行為と呼ばれるものである。ただし、ダフ屋はコンサートやスポーツのチケットを対象とするが、プーバイは符売と書く通り、電車の切符(主に急行券や寝台券)がメインで、他に劇場などの切符を売っていた。ちなみにプーバイをする人をプー屋という。	1953年	
\\	ほんわか	ほんわか	穏やかに、のんびりと心が和むさま。	ほんわかとは強い刺激や緊張感のないノンビリしたさま。穏やかな雰囲気に包まれ、心が和むさまをいう。ほんわかは独特な英語(トニングリッシュ)によるギャグや『さいざんすマンボ』といった楽曲が有名なコメディアンであり、マンガ『おそ松くん』に登場するイヤミのモデルでもあるトニー谷の造語で1950年代始め頃に普及。以後、現代も親しまれている言葉である。	1953年	
\\	ミーハー	みーはー	流行などに熱中しやすい人のこと。	ミーハーとは流行に熱中しやすい人や、程度の低いことに夢中になっている人のことで、軽蔑の意を込めて使われることが多い。 ミーハーは『みいちゃんはあちゃん』の略であることから、当初は女性に対して使われることが多い言葉であった。しかし、次第に性別に関係なく、上記のような人全般に対して使われるようになる。(『みいちゃんはあちゃん』に関しては該当ページ参照 
\\	また、ミーハーから生まれた言葉に軽い若者文化を総称した『ミーハー族』、ミーハーっぽい人を意味する『ミーハーチック』といったものがある。	1953年	若者言葉	
\\	アジャパー	あじゃぱー	驚きと困惑を同時に表す感嘆詞。	アジャパーとは驚きと困惑を表す感嘆詞で、喜劇俳優の“ばんじゅん”こと伴淳三郎が映画『吃七捕物帖』(1951年公開)でうけたセリフ「アジャジャーにしてパーでございます」を略したものである。つまり、アジャパーは『アジャジャー』と『パー』からなる合成語だが、パーはクルクルパーに見られる馬鹿の意、アジャジャーは伴淳三郎の出身地:山形県の方言で「あれまあ」と同義に使われる言葉である。アジャパーは映画の公開から若者を中心に普及。当時の流行語にもなった。	1951年	若者言葉	
\\	煙突	えんとつ	タクシーの運転手が料金をごまかす方法。	煙突とはタクシーの運転手が乗車料金をごまかす方法のひとつで、料金メーターを空車の状態で客を乗せて車を走らせ、その分の乗車料金をタクシー会社に申告せず着服することである。昔のタクシーの料金メーターはハンドル横辺りにある空車レバーを倒すことで実車となった。空車時はこのレバーを立てた状態で走り、外にいる客から空車であるという案内にもなっていた。この立った状態の空車レバーを煙突とみたて、先述の行為を煙突と呼んだ。 ただし、現在はノルマの関係から『ぶっこみ』が行われることのほうが多い。(ぶっこみに関しては関連ページ参照)	1951年	業界用語	
\\	社用族	しゃようぞく	社用と称し、社費で役得する人のこと。	社用族とは1948年の流行語「斜陽族」のもじりで、1951年に朝日新聞記者で天声人語を執筆したことで知られる荒垣秀雄氏の造語である。社用にかこつけ、社費で役得する人たちのことで、接待と称して会社の接待費で飲み食いしたり、社用と称して交通費で移動する人たちがこれにあたる。	1951年	
\\	チラリズム	ちらりずむ	男性の性的興奮に関する主張(主義)のひとつ。	チラリズムとはチラリと、英語で
\\	主義という意味にする接尾語
\\	(この場合エロティシズムからとられている)』から成る合成語で、全てが見えてしまうよりもチラッと見える、又は見えそうで見えないほうが興奮するという男性の性的興奮に関する主義の一種である。スカートからチラッと見える(見えそうな)女性の下着、シャツや上着からチラッと見える(見えそうな)女性の胸元などがこれにあたる。また、下着姿や水着姿も全裸に対してはチラリズムとなる。 もともとチラリズムは浅香光代の剣劇で、立ち回りの際に乱れた裾からチラリと見えた太股に対して使われた言葉で、1951年(昭和26年)の流行語である。後に浅香光代本人が「立ち回りの際にハプニングとして見えてしまった太股が受けたため、見せ場のひとつとして意識的に取り入れた」と語っている。	1951年	合成語(和製英語)	
\\	アメション	あめしょん	アメリカへ短期渡航した人をからかう言葉。	アメションのアメとはアメリカ、ションとはしょんべん(=小便)を意味し、「アメリカへ小便をしに行った」という言葉を略したもの。戦後占領下の日本では代議士、芸能人といった著名人が箔をつけるため、続々とアメリカへ渡った。そんな著名人に「アメリカへ行って小便しただけだろう」という冷やかしの意を込めてアメションと言った。アメションという言葉自体は大正時代から使われたが、こうした経緯から戦後流行語となり、イギション(イギリスへ・・・)、フラション(フランスへ・・・)といった派生語もうまれた。	1950年	
\\	イエスマン	いえすまん	上司など目上に対してご機嫌取りをする人の一種。	イエスマンとは上司や先輩など自分にとって影響力のある人や権威のある人に対してご機嫌取りをする人の一種で、目上の人からの頼まれ事や意見に対して「はい」「いいですねえ」といった肯定的な返答しかしない人のことである。イエスマンは主にサラリーマンなど企業人に使われる言葉で、ワンマン経営者やカリスマ性のある社長の周りに多いとされる。こういったイエスマンは扱い易い良い人材に見えるが、明らかに誤った方向に進んでいるときや間違った意見に対しても指摘をせず「はい」と答えるだけなのでよいことばかりではない。実際、社長の取り巻きがイエスマンばかりであったために危機に瀕した会社もある。現代ではイエスマンを自分の意見を持たない者として低く評価する企業や上司も多い。 ちなみに元々は、戦後占領下の政府当局者で国政の細部までいちいち指示を受けないと行動が出来ない人を意味する政治用語であった。	1950年	
\\	イカレポンチ	いかれぽんち	軽薄で頭が悪い間抜けな男性のこと。	いかれぽんちとは「してやられる」という意味の俗語『いかれる』と「坊ちゃん ・ ぼんぼん」を意味する関西弁の『ぼんち』が転化した『ぽんち』から成る合成語で「軽薄で間抜けな男性」を意味する(『ぽんち』に関しては、『ポンチ絵』という滑稽な絵(マンガ)からきたという説もある)。 いかれぽんちは1950年前後の流行語で昭和時代にはよく使われたが年々使われることが少なくなっている。	1950年	合成語	
\\	顔パス	かおぱす	その人の地位や名誉で切符がなくても乗車 ・ 入場が出来ること。	"顔パスのパスとは英語の
\\	""で「通過する」「無料乗車」「無料入場券」といった意味。また、ここでいう顔とは人体の部位としての顔でなく、「あの人は会社の顔」といったような知名度、他人に対する影響力のある人といった意味。つまり、顔パスとはその人の知名度や影響力、地位、権威といったもので、本来入場料、乗車料、面倒な手続きを要する場所に顔を見せるだけで素通り~利用出来ることを意味する。戦後辺りから使われた言葉である。 またここから顔馴染みになったことで上記のように素通り出来ること、更に顔馴染み ・ 常連になることも顔パスというようになる。
\\	1950年	
\\	十一	といち	十日で一割の金利のこと。	といちとは十一と書き、十日で一割という金利(利息)を取ること及び、そういった金利を取る金融業者を意味する。金融業者に対しては厳密に十日で一割という利子ではなくても、十日に二割、三割といった高利貸し全般を対象にといちまたはといち金融と呼ぶ。 なお、最近では十日で一割という利子からではなく、東京都知事登録を初めて受ける際、その業者に振られる「東京都
\\	第
\\	号」といった番号からそう呼ぶことが多い。こうした初登録後3年を過ぎていない東京都
\\	の番号がついた金融業者にはヤミ金融、高利貸しが多いことから都
\\	=といちと言う。	1950年	
\\	とんでもハップン	とんでもはっぷん	「とんでもない」「まさか」といった意味の言葉。	とんでもはっぷんとは日本語の「とんでもない」と英語
\\	の合成語で、「とんでもない」「まさか」といった意味で使われる。もともと戦後の学生間で使われていた言葉だが、後に獅子文六が朝日新聞に連載した長編小説「自由学校」で使用。更に同小説の映画化の際も使用し、流行語となった。1980年代には派生語「飛んでも8分歩いて10分」という言葉も生まれている。	1950年	合成語	
\\	二枚目半	にまいめはん	外見はカッコいいが、滑稽な役を演じる俳優及びそういった人。	二枚目半とは色男 ・ 美男子役を意味する『二枚目』、滑稽な役 ・ 道化役を意味する『三枚目』の両方を兼ね揃えた役者を意味する。また、ここからカッコいいが当人は自覚がなく(または鼻にかけず)、愉快な人を意味する。昭和時代、俳優としては草刈正雄、岡田真澄が二枚目半として評価された。最近では玉木宏が二枚目半に当たるが、二枚目半という言葉自体使われなくなってきており、「三枚目も出来る二枚目俳優」「三枚目が入ってきた(二枚目)俳優」といった表現が使われる(以前と違い、二枚目俳優やアイドルが三枚目を演じることは当たり前になり、区分けする必要性がなくなったためか?)。	1950年	
\\	自転車操業	じてんしゃそうぎょう	売上げのほとんどを仕入れに当て、赤字覚悟で操業すること。	自転車操業とは文芸評論家:臼井吉見による造語で、操業を止めると倒産してしまうため、売上げのほとんどを仕入れ金に当て、赤字覚悟で操業し続けることをいう。戦後、
\\	の経済顧問:ドッジが行った財政政策(ドッジ ・ ライン)で金詰りに陥った中小企業が事業を続けるため、原価割れの仕事も請け負った。赤字になることはわかっていても、生産を止めれば倒産必至なため、操業し続けたこの状態を、ペダルを漕ぐ足を止めると倒れる自転車の運転に例え、自転車操業といった。バブル崩壊後、再び使用が増えた言葉である。	1949年	
\\	ニコヨン	にこよん	日雇い労働者のこと。	1949年6月、東京都の失業対策事業として職業安定所が支払う日雇い労働者への定額日給を240円と定めた。そしてこの百円2枚と十円4枚という日当から日雇い労働者のことをニコヨンと呼んだ。ただし、日当の額の変化とともに意味を成さなくなり、この呼び方は使われなくなる。現在ニコヨンは死語となっており、ニコヨン世代と現代のニートやネットカフェ難民を比較する際に使われる程度となっている。	1949年	
\\	斜陽族	しゃようぞく	華族令があった頃の旧上流階級のこと。	斜陽族とは1947年に月刊文芸雑誌「新潮」で連載されていた太宰治の「斜陽」からきた言葉で1948年の流行語。「斜陽」は戦後、華族令の廃止とともに没落した貴族を描いた物語。そしてこの物語に出てくるような没落貴族のことを斜陽階級と呼び、後に斜陽族という言葉で流行語になった。 また、斜陽族をもじった社用族、車用族という言葉もある。	1948年	
\\	13階段	じゅうさんかいだん	処刑場のこと。	13階段とは戦後の処刑台の階段の段数からきた言葉で、「処刑場」を意味して使われていたが、後に「絞首刑」「死刑」といった意味でも使われるようになる(註:現在の処刑台の階段が13段かは不明)。 13階段は極東国際軍事裁判(いわゆる東京裁判)で東条英機元首相を含む
\\	級戦犯7名に死刑が宣告~執行された1948(昭和23年)に普及。2001年には高野和明が『13階段』というタイトルで小説を発表(第47回江戸川乱歩賞受賞)。2003年には反町隆史、山崎努、笑福亭鶴瓶、田中麗奈出演による同小説の映画化もされている。	1948年	
\\	額縁ショー	がくぶちしょー	初期ストリップショーのひとつ。	戦後、初めてストリップショーが行われたのは1947年1月15日、東京新宿:帝都座の「名画アルバム」であった。ただし、女性(モデル)が動くことは当事の規制にかかり、検挙の恐れもあるため、現在一般的に知られるようなショーと異なり、額縁の中で半裸または上半身裸の女性が名画のポーズをして見せるというものであった。こうした手法のストリップを額縁ショーと呼び、当事の流行語にもなった。	1947年	
\\	ロードショー	ろーどしょー	特定の劇場で行う映画の独占封切興行。	"ロードショーとは英語の
\\	""からきたカタカナ語で、一般公開に先駆け、特定の劇場(主に大都市圏)で行われる封切上映のことをいい、ロードショーを行う映画館のことを『封切館』という。ちなみに日本で初めてロードショーが行われたのは1947年(昭和22年)3月25日。東京・有楽町のスパル座で『アメリカ交響楽』が上映された。 ロードショーはもともと(地方巡業の)演劇から生まれた言葉で、公演前に宣伝目的で行われた路上での予告上演のことであった。このため英語で
\\	""といった場合、地方巡回興行をさすことが多く、日本でいうロードショーを英語でいう場合は
\\	""のほうが適切である。近年の映画作品は全国一斉に封切りされることが多くなったため、ロードショーも先行上映という意味だけでなく、映画興行全般をさす言葉として使われるようになってきている。
\\	1947年	カタカナ英語	
\\	カストリ	かすとり	カストリ酒、またはカストリ雑誌の略。	
\\	カストリとは「カストリ酒 ・ カストリ焼酎」の略である。カストリ酒とはイモや酒粕からとった粗悪な焼酎及び、闇市に氾濫した粗悪な密造酒のことである。また、カストリを飲んで意気盛んに語るインテリをカストリゲンチャという。 
\\	カストリとは「カストリ雑誌」の略で、戦後、出版の自由化に伴って発売された大衆娯楽雑誌のことである。粗悪な紙に内容はグロいものがメインだった当事の娯楽雑誌は3号程度で廃刊になっていたことから、3号も飲めば潰れるといわれたカストリ酒に掛けた言葉といわれる。(物資不足から当事の雑誌には再生紙が使われ、紙のカスを取って作った雑誌=カストリという説もある)	1946年	略語	
\\	パンパン	ぱんぱん	進駐軍兵士相手の娼婦のこと。	パンパンとは戦後、連合国軍による保障占領で日本にきた進駐軍兵士
\\	を相手に街頭に現れた私娼のことで、パンパンガールやパン助ともいう。こういった女性をなぜパンパンと呼ぶのか?これは「性交」を意味する英語
\\	からきたという説や「女性」を意味するインドネシア語
\\	(プロムパン)』のアメリカ訛りという説、更にペンペンという三味線(つまり芸者)を表す擬音語からきたという説などあるが詳細は不明。どちらにしても、米兵
\\	が使ったこれらの言葉が日本人の耳にパンパンと聞こえ、そう呼ばれるようになった。ただし、現代人には物が膨らんで張った様を表す擬態語として用いることの方が多い。	1945年	カタカナ英語
\\	インハラベビー	インハラベビー	妊婦のこと。	インハラベビーとは妊婦のこと。
\\	腹 
\\	と書けばわかる通り、お腹の中に赤ちゃん
\\	がいるということを、英語と日本語を織り交ぜ、英語っぽく発音したものである。昭和初期、女学生の間で使われた隠語で、現在は全く使われない死語となっている。	1931年	
\\	テクシー	てくしー	徒歩のこと。	テクシーとは歩くさまを表す擬音語「てくてく」と「タクシー」をかけた言葉で、てくてく歩いて行くこと、徒歩で行くことを意味する。タクシーが東京 ・ 銀座で走るようになったのは1912年(大正元年)からだが、その数年後にテクシーという言葉が使われるようになる。ただし、テクシーが使われたのは昭和初期くらいまでで、現在は死語となっている。	1931年	
\\	オフィスワイフ	おふぃすわいふ	肉体関係が伴う女性秘書のこと。	オフィスワイフとは直訳すると職場の妻ということになるが、ここでは単に肉体関係のある秘書(女性社員)を意味する。これは1930年にアメリカで上映されたロイド ・ ベーコン監督の映画『オフィスワイフ(原題:
\\	からきたものである。映画は出版社社長と女性秘書との仕事を超えた恋愛関係が絡むストーリーである。	1930年	カタカナ英語	
\\	モマ	もま	モダン ・ マダムの略。	モマとはモダンマダムの略で、大正末期から昭和初期にかけて流行した『モガ』『モボ』を受けて出来た言葉である(ただし、モマは他のモガ ・ モボのように流行語となるほど広くは普及していない)。モマは当時の流行を追った夫人を嘲う言葉である。	1930年	略語 ・ 若者言葉	
\\	ルンペン	るんぺん	浮浪者になること。	ルンペるとは独語からきた「定住地をもたず、公園や道路など公共の場で生活する人」を意味する『ルンペン』の最後の『ン』を俗語によくある動詞化する接尾語『る』に置き換えたもので、「ルンペン(浮浪者)になること」を意味する。更にこの意味から転じて、ルンペるは「失業する」という意味でも使われる。	1930年	
\\	小田急る ・ オダキュール	おだきゅる	小田急線に乗ってどこかへ行くこと。	小田急るとは小田原急行鉄道(現在の小田急電鉄)を略したものに、動詞化する接尾語『る』を付けたもので、小田急線に乗ってどこかへ行くこと。または郊外へ行くことを意味する。1929年、西條八十が作詞した『東京行進曲(同名映画の主題歌)』が流行。この歌詞の中の「いっそ小田急(おだきゅ)で逃げましょうか」というくだりから、小田急る(おだきゅる)という言葉が使われるようになった。読みが外国語っぽいことから、オダキュールという表記も用いられる。	1929年	
\\	モガ	もが	モダン ・ ガールの略。	モガとはモダンガールの略で、大正末期から昭和初期にかけて流行した言葉である。当時、モガと呼ばれた若い女性の特徴は断髪、洋装、引き眉毛、頬紅、ルージュなどのメイキャップといった格好で、どちらかというと不良的な悪い意味で使われることが多かった。 昭和中期を過ぎてからは、モダンという言葉自体が使われなくなったこともあり、モガも廃れていった。代わりに『ナウいギャル』『イケてる』といった言葉が似たような意味合いで使われるようになる。	1926年	美容 ・ ファッション用語、略語、若者言葉	
\\	モボ	もぼ	モダン ・ ボーイの略。	モボとはモダンボーイの略で、大正末期から昭和初期にかけて流行した言葉である。当時、モボと呼ばれた若者の特徴は山高帽氏、ロイド眼鏡、セーラーパンツ、ステッキといった格好で、どちらかというと不良的な悪い意味で使われることが多かった。 昭和中期を過ぎてからは、モダンという言葉自体が使われなくなったこともあり、モボも廃れていった。代わりに『ナウなヤング』『イケてる』といった言葉が似たような意味合いで使われるようになる。	1926年	美容 ・ ファッション用語、略語、若者言葉	
\\	ジリ貧	じりひん	次第に状況が悪くなること。	ジリ貧とは「ジリジリと貧しくなる」の略で、もともとは第一次世界大戦後、不況のため次第に業績が悪化していく企業をさして使われた言葉である。後にジリ貧は個人レベルでも使われるようになり、内容も金銭的なもの以外に学校やスポーツの成績、人間関係 ・ 国家間関係など「次第に状況が悪くなるさま」を表す言葉として様々な場面で使われるようになる。 また、株用語でジリ貧といった場合、相場がジリジリと安くなることを意味し、『ジリ安』ともいう。逆にジリジリと相場があがることを『ジリ高』という。	1923年	
\\	閻魔帳	えんまちょう	教師が生徒の成績などを書き留めておく手帳のこと。	閻魔帳とは本来、閻魔様が亡者の生前の罪悪を書き留めておくとされる帳面のことだが、ここから、学校の先生が生徒の成績 ・ 採点や日頃の行いを書きとめておく帳面のことをいう。また、巡査 ・ 警官が罪科を調べて書き留めておく手帳を指して閻魔帳という場合もある。	1920年	
\\	サボる	さぼる	学校や会社を怠けて休むこと。	サボるとは「仕事などを意図的に行わないこと」を意味するフランス語“
\\	からきている(フランスはストライキの多い国で主にストライキの際に使われる言葉である)。1919年(大正5年)9月、神戸川崎造船所でサボタージュ作戦が行われ、賃上げに成功したことが話題となり、同年“
\\	を略したサボが流行語になる。このサボに動詞化する接尾語『る』をつけたものがサボるである。こうしてサボるは学校や職場を病気などの理由もなく、自分の気分で怠けて休んだり早退するという意味で使われるようになる。サボるはこうした経緯で生まれたカタカナ語であり、サボの部分はカタカナ表記するのが正解だが、もともと日本語として存在する言葉と勘違いしている人が多く、平仮名表記のさぼるも使われる。	1919年	カタカナ仏語+略語	
\\	ガチャガチャ	がちゃがちゃ	
\\	新米巡査のこと。 
\\	カプセルトイの別称。	
\\	ガチャガチャとは新米巡査を嘲う言葉として大正時代に使われた言葉である。1917年(大正6年)は大戦景気で警察志願者が激減。通常なら不採用になるような気弱な人も採用になった。こうして警官になった気弱な人が繁華街を巡回する際、サーベルをガチャガチャと鳴らし、威嚇しながら歩いたことからこう呼ばれるようになった。 
\\	ガチャガチャとはフィギュアなどのオモチャが入った丸い透明カプセルを販売する自動販売機のこと。コインを入れ、商品を取り出すハンドルを回す際に「ガチャガチャ」と音がすることからこう呼ばれるようになった。この意味では『ガチャポン』『ガシャポン』『ガチャンコ』など、エリアによって呼び方が異なる。 ※『ガチャガチャ』『ガチャポン』『ガシャポン』は株式会社バンダイの登録商標、『ガチャンコ』は株式会社タカラトミーの登録商標になっています。 
\\	ガチャガチャとはうるさく物が触れ合う音。物が無秩序に入り乱れるさま。	1917年	
\\	ニコポン	にこぽん	人を懐柔する態度のこと。処世術のひとつ。	ニコポンの「ニコ」とは和やかに笑う表情をあらわしたニコニコの略、「ポン」は肩を叩くさまをあらわす擬態語である。つまりニコポンとはニコニコしながら相手の肩をポンと叩き、親しげにすることで人を懐柔したり、人に物を頼むという処世術の一つである。ニコポンは1913年(大正2年)に当時の首相:桂太郎に対するニコポン宰相 ・ ニコポン主義という評に使われて流行。ニコポン上司など一般にも使われたが、こうした態度で人が動かせる時代ではなくなったせいか、現在ではほとんど使われない死語となっている。	1913年	
\\	ハイカラ	はいから	西洋かぶれした人やその様。	"ハイカラとは舶来品や西洋の格好を好む西洋かぶれした様やそのような人のことで、キザや軽薄といった意を込めて使われた明治時代の流行語である。 ハイカラは英語の
\\	(高い襟)""を略した造語で、もともとは高い襟をつけ、進歩主義 ・ 欧米主義を主張した若い政治家を指した。また、保守主義者の『チョン髷(まげ)党』に対し、開国主義 ・ 欧米主義者の『ハイカラ党』という言葉もあった。これが転じ、西洋かぶれした人全般をハイカラと呼ぶようになったが、西洋の服装や文化を取り入れることが当たり前となるとともにハイカラという言葉も使われなくなった。しかし昭和後期に入り、少女マンガ『はいからさんが通る』で再びハイカラという言葉が脚光を浴びた。
\\	1909年	和製英語	
\\	ハイカる	はいかる	西洋風な格好をしたり、西洋風に振る舞うこと。	ハイカるとは西洋風の格好をしたり、西洋風に振る舞うことを意味する流行語『ハイカラ』に名詞を動詞化する接尾語『る』をつけた造語で、西洋風の格好や物を取り入れたり、西洋風に振舞うことをいう。(ハイカラに関しては該当ページ参照)	1909年	若者言葉	
\\	レッテル	れってる	ある人や物事に対する特定の評価こと。	レッテルとはオランダ語の“
\\	で、もともとは商品に貼る商品名や内容、容量などを書いた小札のことである。この小札は商品の案内という目的と同時に商品のイメージを一方的に作るという考え方も出来る。ここからある人や物事に対する特定の評価や固定的なイメージをレッテルという。例えば、「医者だから金持ち」「過去にを不良品を出したから危険」といったものがレッテルにあたる。レッテルは小札からきているため、こうしたイメージ付け ・ 特定の評価をすることを「レッテルを付ける」「レッテルをくだす」とは言わず、「レッテルを貼る」と言う。また人や物をおおまかに分類し、イメージ付けすることをレッテル貼りという。中高年層の「最近の若者は
\\	だ」といった発言や血液型によるイメージ付けがこれに当たる。	1909年	カタカナ蘭語	
\\	出歯亀	でばかめ	覗きの常習者、変態、助平のこと。	出歯亀とは女湯覗きの常習犯“池田亀太郎”からきている言葉である。1908年3月22日、東京 ・ 大久保で銭湯帰りの電話交換局長夫人が殺害され、その犯人として池田亀太郎が逮捕される。マスコミは池田が出っ歯であったことから出歯亀(デバカメ)と呼び、出歯亀事件という名で大きく報道した(池田が出しゃばりであることから出張(でば)と呼ばれていた物が出歯と報道されたという説もある)。 ここから覗きの常習者を出歯亀と呼ぶようになる。後に出歯亀は変態、変質者、更に単に助平といった形で意味を広げて使われるようになる。	1908年	
\\	成金	なりきん	急に大金持ちになることや、急にお金持ちになった人のこと。	"成金とは元々将棋用語で『歩』が敵陣地に入ると『金』に成る(変わる)ことをいう(""ときん""とも言う)。前方に一駒進むだけの『歩』が相手陣地に入った途端『王 ・ 玉』の次に重要とされる『金』に成る、そんな様から急に金持ちになるという意味でも使われるようになる。このような意味で江戸時代には既に使われている。成金は努力を積み重ねてお金持ちになったというより、宝クジなどで楽にお金を手に入れたというイメージが強い。このことから急にお金持ちになった人を侮蔑する意を含んで使われることが多い。成金は江戸時代から使われている言葉だが、株式市場が暴落~高騰した1906年、鈴木五郎が株で儲けたことが話題となり、成金という言葉も広く普及した。 また、成金になる要因を頭につけた『土地成金』『自動車成金』『株成金』といった使い方や『成金趣味』『にわか成金』といった言葉もある。
\\	1906年	
\\	花電車	はなでんしゃ	ストリップ ・ ショーでの芸のこと。	花電車とは本来、装飾を施した電車のことで、特に路面電車が市街地を走っていた頃は祭り ・ 祝事になると華やかな電灯や花などの装飾を付けて走っていた(現在、各地で祭りになると走らせる花バスに名残がある)。この花電車の特徴として、華やかな飾りとともに客を乗せないといった点がある。これはそうした装飾を見せるためだけに走らせるからだが、こうした特徴から、風俗業において客を乗せない(=客に直接触れない)ストリップでのショー、パフォーマンスといった見せるだけの芸を花電車と呼ぶ。	1904年	
\\	兵隊勘定	へいたいかんじょう	割り勘で支払うこと。	兵隊勘定とは複数人で飲食をした際、代金を各自均等に負担して支払うことをいう(現代では「割り勘」という呼び方が一般的)。兵隊勘定は日露戦争が開戦となる1904年(明治37年)に流行となった言い回しだが、現在では当時を知る人やそこから聞いて人が使う程度の死語となっている。兵隊勘定の由来には同じ兵隊という立場だから均等に負担するという考えからきた説、明日生きているか定かでない兵隊だから貸し借りなしで均等に支払うという考えからきた説がある。	1904年	
\\	たちんぼ ・ 立ちん坊	たちんぼう	立ったままで長時間いること。	たちんぼとは車の性能がよくなかった時代に坂の下に立って車を待ち、車がきたら坂を後押しして金をもらう者や職業安定所近辺でぶらぶらと立ちながら仕事を待つ日雇い労働者を意味した(当時、建築で人が必要な場合、親方が職業安定所で仕事にあぶれウロウロしている中から体格のいい者を使っていたため)。更に立って仕事を待つということから、道路に立って客を待つ娼婦のこともたちんぼと呼ぶようになる。たちんぼはこういった意味が転じ、「立ち仕事」「立っている人」、「長時間立っていること=立ちっぱなし」と意味を広げる。たちんぼは本来立ちん坊と書き、たちんぼうと読むが、テンポのよさからかたちんぼと略された形で定着している。 たちんぼの使用例: 今日のバイトはたちんぼで疲れた~。	1903年	
\\	バンカラ ・ 蛮カラ	ばんから	言動や風貌が保守的で粗雑な人のこと。	バンカラとは『野蛮』と『ハイカラ』から成る合成語で蛮カラとも書く。バンカラは同じ時期に流行した「西洋かぶれ」を意味する俗語『ハイカラ』の対語として生まれたものである。バンカラの格好は昭和時代のマンガに登場する応援団に見られる学生服に高下駄、マントといったもので、日本古来の精神を守るという姿勢は骨太であるがハイカラに対しては保守的ともとれる。 そういった風貌や気質が後に硬派な番長をイメージし、番カラという表記も使われた。	1902年	和製英語+合成語、美容 ・ ファッション用語	
\end{CJK}
\end{document}