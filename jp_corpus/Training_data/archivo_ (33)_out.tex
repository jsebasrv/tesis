\documentclass[8pt]{extreport} 
\usepackage{hyperref}
\usepackage{CJKutf8}
\begin{document}
\begin{CJK}{UTF8}{min}
\\	けれども	けれども	
\\	雨が降っていたけれども、みんなは楽しい時を過ごした。 
\\	お引き受けしたいんですけれど・・・ 
\\	彼は16歳だけれど、両親から独立しています。
\\	彼はあまり勉強しないけれども、成績がいい   
\\	ようにも〜ない	ようにも〜ない	
\\	しようとしてもできない; したいのにできない
\\	頭が痛くて、起きようにも起きられなかった  
\\	新しいパソコンを買おうにも、お金がなくて買えない。  
\\	会社を辞めようにも、次の仕事が見つからない。  
\\	寿司を作ろうにも材料が手に入らない。  
\\	母は僕の日本人のガールフレンドと話そうにも日本語ができなかった。   
\\	に過ぎない	にすぎない	
\\	あれは大混乱期の始まりに過ぎない。  
\\	それは夢に過ぎない。  
\\	ある手紙で彼は、「人生は一瞬の夢に過ぎない」と言っている。   
\\	その新聞の発行部数は競争紙の三分の一に過ぎない。  
\\	それじゃ女は大きく育った子供に過ぎない。   
\\	サンフランシスコの人口は、ニューヨークのわずか10分の1に過ぎない。  
\\	ルチアーノは大声援を背負って戦えるかもしれないが、彼はまだヒヨコに過ぎない。   
\\	雨林は地面わずか2%をしめるに過ぎないが、世界の野生の動植物、及び昆虫の半分以上の種類がそこで生息している。  
\\	永久的な平和など幻想に過ぎない。  
\\	我々の世界は宇宙のほんの小さな一部分に過ぎない。  
\\	君の計画は砂上の楼閣に過ぎないよ。  
\\	恒久的な平和など幻想に過ぎない。   
\\	私の考えでは、恒久的な平和など幻想に過ぎない。   
\\	私は貧しい農民に過ぎない。  
\\	前のやっていることは現実からの逃避に過ぎない。   
\\	大丈夫です。ただ、ちょっと疲れたにすぎません。  
\\	(赤毛)
\\	てしょうがない; てたまらない	てしょうがない; てたまらない	
\\	睡眠不足のせいで眠くてしょうがない。  
\\	しょうがないとおもうよ  
\\	彼の発言は他に解釈のしようがない。   
\\	誕生と死との間を享楽する以外には、この両者に対して回復の手の施しようがない。   
\\	最近お腹が空い てしょうがない。  
\\	勉強が楽しくてしょうがないという人もいるんですか。信じられませんね。  
\\	今日は何もすることがなくて、ひまでしょうがない。   
\\	いい成績が取れる、うれしくてしょうがない。   
\\	ひどい天気だけどしようがない。   
\\	今出かけてもしようがない。   
\\	〜堪えない	〜にたえない	
\\	~をとても強く感じる・話してがそのことに対して強い感情を持っているときに用いる。	
\\	色々お世話になりまして、感謝にたえません  
\\	氏の当会へのご援助はまことに感謝にたえません  
\\	彼の私に対する態度は憤慨に堪えない   
\\	と共に	とともに	
\\	大阪は東京とともに日本経済の中心地である  
\\	病気が回復すると共に食欲も出てきた  
\\	である]経済的であると共に  
\\	興味深いと共に  
\\	スタインベックはとともにヘミングウェイアメリカの大分な作家の一人だ。  
\\	私は職場の同僚達 とともにそのデモに参加した。  
\\	終戦とともに激しいインフレが人々を襲った。   
\\	年をするとともに耳が遠くなってきた。   
\\	その提案は建設的であるとともに革命的であった。  
\\	人口が増えるとともに、それまで見られなかった都市問題が発生した。  
\\	にほかならない	にほかならない	
\\	まさに〜だ・である	
\\	戦争は死にほかならない。  
\\	ここでやめるのは敗北を認めることにほかならない。  
\\	シャルルマーニュの目的はまさにローマ帝国の栄光を取り戻すことにほかならなかった  
\\	世界の歴史は自由意識の進歩にほかならない  
\\	この仕事が成功したのは皆さんのご協力の結果にほかなりません。  
\\	その問題についていろいろな意見が出ているが、それは、みんなが関心を持ってい るからにほかならない.  
\\	彼が他ならぬ大統領である。   
\\	彼をなぐったのは他ならぬ私だったのです。   
\\	失敗は初めての成功にほかならない。  
\\	極まりない/ 極まる	きわまりない/ きわまる	
\\	私は、彼の失礼きわまりない態度に我慢ならなかった。 
\\	その机は、乱雑極まりない状態だ。  
\\	こんな風は強い日に小型のボートで沖に出るなんて、危険極まりない。  
\\	(赤毛)
\\	私は、彼の失礼きわまりない態度に我慢ならなかった。  
\\	に当たって; に当たり	にあたって; にあたり	
\\	サイトのご利用にあたって  
\\	科学論文を書くにあたって  
\\	私の留学に当たり父は自分の経験を話してくれた  
\\	もし困った事があればユーザーズガイドにあたってください。   
\\	我々は決定にあたって偏見に左右されない。   
\\	研究部長はその部門が新製品のテストをするにあたって、徹底的な仕事をさせた。   
\\	私が部屋を掃除する番にあたっていた。   
\\	私たちが到着する日は日曜日にあたっている。   
\\	凄い明かりが野球場にあたっていた。   
\\	日にあたってからだを暖めた。   
\\	彼が外人客の接待にあたっている。   
\\	彼女は火にあたってからだを暖めた。   
\\	母親が、彼の看護にあたってる。   
\\	私は就寝 に当たって少量の洋酒を喫すること習慣にしている。  
\\	小説家は時代小説を書くに当たり、その時代の歴史を詳しく調べた。  
\\	新年を迎えるにあたって,一年の計画を立てた.  
\\	かえって	かえって	
\\	安いと思って買ったのに、腐っていたのでかえって損をした。  
\\	あまりにしっかりうなぎを握ると、かえって逃げられる。  
\\	これだけ悪口雑言を浴びせられれば、後は怖いものなどありゃしない。かえってすっきりするよ。  
\\	これはあまり自分を守ろうとすると、かえって大きな間違いを犯すことになるだけだ、ということです。  
\\	そんなに美辞麗句を並べられると、かえって嘘に聞こえるよね。  
\\	完全を求めるとかえって失敗する。  
\\	休養するどころかかえっていつもより仕事に励んだ。  
\\	事故は減らずにかえって多くなっている。  
\\	この問題が分からなくて、先生に聞いたけど、説明してくれて、かえって頭が混乱してきた。  
\\	欠点があるからかえって彼女が好きだ。  
\\	それはかえって家庭問題をさらに悪化させるだけだろう。  
\\	彼はえさをやりすぎてかえって犬をだめにしている。   
\\	に限って	にかぎって	
\\	うちの子供に限ってそんなひどいことはしませんよ。  
\\	傘を持っていない日に限って、雨が降る。  
\\	参加者は女性に限る  
\\	夏はビールに限る。  
\\	その日に限って財布を家に忘れた。   
\\	その日に限って子供たちは会館に入ることを許される。   
\\	その日に限って彼は欠席だった。   
\\	一緒にいるのを見られたくない人と一緒にいるときに限って、知り合いに会うことがおおい。   
\\	急いでいる時に限ってバスが遅れる。   
\\	彼女に限ってそんなことはない。  
\\	中野先生はたいてい研究室にいらっしゃるのに、今日に限っていらしゃらなかった。   
\\	嫌なことがある時に限って、嫌なことが重なる。  
\\	恋がなくって、連れて行ったことに限る。  
\\	彼女の携帯電話は話したい時に限って、話中だ。  
\\	朗らかは洒落に限る。  
\\	〜んです	〜んです	
\\	「どうしたんですか」「ちょっと気分が悪いんです」  
\\	「どうして遅れたんですか」
\\	バスが来なかったんです。 
\\	行きたいのですが、、、(行けない)  
\\	確かに違法じゃないんです。そんな簡単に儲からないし、絶対に破綻するシステムでありながら、一律に禁止できないから見逃されているんです。  
\\	〜につき	〜につき	
\\	工事につき電車は不通です。  
\\	教授の海外出張につき休講。  
\\	雨天につき、試合は延期いたします。  
\\	この道は工事中につき、通行止めになっています。  
\\	冷房中につき、ドアをお閉めてください。  
\\	こちらは特売品につき、お一人様一点限りとなっています。  
\\	この機械はただ今調整中につき、ご使用になれません。   
\\	本日は祭日につき、休業させていただきます。  
\\	雨天につき試合は延期いたします。   
\\	まで	まで	
\\	人間は食べ物がなくても40日まで生きられるが、水がないと7日間しか生きられない。  
\\	私は東京から九州まで飛行機で行った。  
\\	おまけに雨まで降っていた。  
\\	私は昼までに手紙を書き終えた。  
\\	10時まで起きてた。  
\\	そこまで言えるのかなあ.  
\\	10時までに帰らなくっちゃ。  
\\	夕方まで勉強しました。ゆうがたまでべんきょうしました。  
\\	とする	とする	
\\	日本へ行くとすれば  
\\	温度は一定であるとする。  
\\	あなたが立派な世界市民になろうとするなら、偏見は持たないように努力すべきである。  
\\	私たちは みな、 人生に おいて 成功を 収めようとするが、成功するのは、きわめて 数少ない人たちである。   
\\	進んで学ぼうとする気の無い物には教える甲斐がない。   
\\	歩こうとするとここがズキンとします。   
\\	労働者たちにはもっと働こうとする刺激がなにもない。   
\\	によって	によって	
\\	アメリカ大陸はコロンブスによって発見された。  
\\	さんによって書かれた本  
\\	列車の代わりに飛行機に乗ることによって失った時間を埋め合わすことができる。  
\\	理想は言葉によって表現される。  
\\	リンゴは彼女によってナイフで半分に切られた。  
\\	卵は大きさと重さによって選別された。  
\\	友達を見ることによってどんな人かわかる。  
\\	輸入の増加によって需要は下がった。  
\\	法律によって我々は子供たちを学校に通わせなければならない。  
\\	わが社では、最新のコンピューター・ソフトの使用によって仕事の合理化を目指すことになった。  
\\	この 薬 は 人 に よって は 副作用 が 出ること が あります。   
\\	イエスキリストの名によって、アメン。  
\\	しかし (然し・併し)	しかし	
\\	けれども。だが。前に述べられた事を受けて、それとは対立する事柄を述べ始める時に使う。	
\\	しかし今や事態は大きく変わった。  
\\	アカスギを材料に巨大でグロテスクな、しかし美しい円柱を作った。  
\\	この文書には契約法上の問題はほとんどない、しかし税法上の問題は多々ある。  
\\	しかし、私はあなたが大好きですよ。  
\\	しかし、外国人と仲良くするのはあまり簡単ではありません。  
\\	〜とあれば	〜とあれば	
\\	〜普通とは違う特別な状況が条件になっていることを表す。後ろには、その状況の時に当然起ころ予想される状況や講堂を表す文がくる。
\\	だれも行かないとあれば、わたしが行くしかないだろう。  
\\	遠来の客が来るとあれば、腕をふるってご馳走を用意しましょう  
\\	彼女の両親が来るとあれば、部屋をきれいにしなければならない。  
\\	ことなしに	ことなしに	
\\	〜(し)ないで、
\\	の動作をしないでは、
\\	の可能性はないという内容を表す
\\	良いお返事をいただくことなしに、帰るわけにはまいりません  
\\	担当教師の許可を得ることなしに、履修科目の変更はできない。  
\\	外相は「真の相互理解に基づく関係は、領土問題を解決して平和条約を締結することなしには構築し得ない」  
\\	もので	もので	
\\	初心者なもので  
\\	月謝を払う余裕がないもので、少しお金を貸してくれますか。  
\\	だって、 高いもん〜!  
\\	得る	うる	
\\	考え得る方法は全部試してみたが、どれも失敗に終わった。  
\\	そのようなことをしても得るべきものはなにもない。  
\\	これらのガスは地球規模の温暖化をもたらし得る。  
\\	そういう事故は時折起こり得る事だ。  
\\	勤勉さが経験不足を補うこともあり得る。  
\\	計画を立てる前に、あまり慎重にし過ぎると、計画そのものがひっくり返ることも、時にはあり得るだろう。  
\\	彼女はオリンピックでメダルを獲得しうる。  
\\	として	として	
\\	彼は単に冗談としてそれを言った。  
\\	先生として働いています。  
\\	私は医者としてあなたに言います。   
\\	18歳以上の者はだれでも大人としてみなされる。   
\\	1月の輸出は100億ドルと、同月としては最高記録になった。   
\\	250キロというのは相撲取りとしても並外れた体重だ。   
\\	2つのことを一度にしようとしてはいけません。  
\\	4億人の人が英語を第一言語として話します。   
\\	は一定期間に国によって生産されたすべての商品とサービスの市場価値の総額として計算される。   
\\	あいつはギャンブラーとして名が通っている。   
\\	あいつは海千山千だから、取り引き相手としてはかなり手強いぞ。   
\\	あなたがばか者のように振る舞えば、ばか者として扱わなければならない。   
\\	あなたの解答には私たちとしてはまったく不満だ。   
\\	あなたの言うことは認めるとしても、私はそれでもその計画には反対だ。   
\\	あなたの言うとおりだとしても、私達はまだあなたの計画に賛成できない。   
\\	あなたは初心者としてはテニスがうまい。   
\\	けんは早起きを習慣としている。   
\\	ここが彼女が秘書として働いている所です。  
\\	あなたは芸術家としての無限の可能性を秘めている。  
\\	このような(子供の)態度は、自分たちがある程度独立していることを示そうとしての結果なのである。   
\\	この国ではイデオロギーや宗教が人としての人格形成に役立つ例があまりに少ない。   
\\	そのコンサート旅行が成功したので彼女の歌手としての名声は固まった。   
\\	その会社はすでにロボットメーカーとしての名声を確立した。   
\\	その結果として、大切な躾や社会道徳の教育がおろそかにされてきたことは否めない。多くの親自身が、子供の勉強を優先するあまり、他人への思いやりなど社会の一員としての責任を忘れがちになっているのも事実である。   
\\	ジャーナリストとしての彼の経歴は優れた実績でいっぱいだった。   
\\	フローレンス・ナイチンゲールは専門職としての看護を始めた人として有名である。   
\\	この部屋は物置として使っている。   
\\	李さんは自動車の会社でエンジニアとして働いている。  
\\	彼女は彼の手伝いさん兼料理人として働いた。   
\\	あなたの医者として、食事の量を減らすように忠告します。   
\\	私はその報告を信頼できるものとして認めた。   
\\	その本お代金として800円払った。   
\\	撃ェどこへ行ってしまったのか誰が一人として知らない。   
\\	当選されたとして、まず何がしたいですか?   
\\	提案は時期尚早だとして却下された。   
\\	大きい	おおきい	
\\	形容詞の使い方	
\\	がてら	がてら	
\\	散歩がてら本屋に寄った。  
\\	ドライブがてら駅に私を迎えにきてください。  
\\	息抜きがてら同僚とチャットをする。  
\\	せっかく上京したのだから見物がてら、買い物でもしよう。  
\\	気味	ぎみ	
\\	風邪気味だから行きたくない。  
\\	彼はいくぶん太り気味だ。  
\\	風邪気味 
\\	仕事が忙しくて最近少し疲れ気味だ。  
\\	私は最近 太りぎみなのでジョギングを始めました。  
\\	私の時計はおくれぎみなので、修理に出した。  
\\	今日は風邪気味だから、家にいよう。  
\\	(赤毛)
\\	新入社員は緊張気味の顔をしていた。  
\\	この所忙しくて少し疲れ気味だから、今日は早く帰る事にした。   
\\	今朝は一寸風邪気味だ。  
\\	彼は少し太り気味だ。   
\\	ものか	ものか	
\\	あきらめるものか。最後までがんばりぬくぞ。  
\\	そうはいうけど、わかるものか。  
\\	そんなものか。  
\\	どんなものかよく調べてみないと判らない。  
\\	俺の目の黒いうちはそんなことさせるものか  
\\	あんな店、二度と行くものか。   
\\	元気なもんか。くたくただよ。   
\\	うれしいもんですか。困っているんです。  
\\	に反して	にはんして	
\\	両親の期待に反して女の子が生まれた。  
\\	国民の期待に反して政府は年金制度改正法案の強行採決に踏み切った。  
\\	研究社員になる希望に反して大学を止めらなければなりませんでした。   
\\	神の意向に反して人間は自然を破壊している。  
\\	(僕の期待に反して)実のところ物価は毎週上昇している。   
\\	この結果は、当初の期待に反して、変数の数が処理速度に影響を与えないということを意味しているのである。   
\\	その結果は私の期待に反していた。   
\\	意思に反して何かを無理にさせる事はできない。   
\\	私は意に反してこれをせざるをえなかった。   
\\	私は意に反して彼と結婚した。   
\\	彼は、自分の意志に反して契約書に署名させられた。   
\\	彼は意に反して解雇された。   
\\	彼は私たちの期待に反して落選した。   
\\	彼は彼自身の望みに反してそれをするように説得された。   
\\	彼は陽気そうに見えるが、本当はそれに反して悲しんでいるのだ。   
\\	彼らの上司は彼らの意志に反して彼らを働かせた。   
\\	彼らは彼の意志に反して彼にその契約書に署名させた。   
\\	彼女は、彼らがしていることは会社のセクハラ政策に反している、と言った。   
\\	彼女は自分の意思に反して計画を断念した。   
\\	神の意向に反して,人間は自然をはかいしている。  
\\	期待に反して、実験結果が出てしまった。  
\\	彼女は母親の意向に反してその若者と結婚した。   
\\	ということだ	ということだ	
\\	今回のレッスンテーマ= 恋ということだ。  
\\	あいつが来ないのは何か後ろめたいことがあるということだ。  
\\	なんということだ!  
\\	その報告書によると不況になるということだ。   
\\	なるほど君の計画はもっともらしく聞こえるが、肝心なのは、それで取り引きが増えるのかどうかということだ。   
\\	ボブとルーシーは交際をやめたということだ。   
\\	警察によれば、真夏のうだるような暑さがその一揆に火をつけたということだ。  
\\	結局は自分がかわいいということだな。   
\\	手紙が戻ってきたということは、彼女はもうこの住所にはいないということだ。  
\\	一度〜たら	いちど〜たら	
\\	一度決めたら、やれるところまでやった方がいい。  
\\	私の日本語の先生は一度言ったら、テストの日程は絶対に変えない  
\\	〜をもって (を以て)	〜をもって	
\\	期限を浮キ。〜で。	
\\	本日の営業は午後7時をもって終了いたします。  
\\	これをもちまして本日の披露宴をめでたくお開きとあさせていただきます。   
\\	店内改装のため、今月末をもって一時閉店することになりました。  
\\	からといって	からといって	
\\	子供だからといって許せない。  
\\	忙しいからといって連絡してくれないのは怠慢だ。  
\\	歩いたからといって遅れた言いわけにはなりません  
\\	ウチは電子メールはないからというのはいつまで通用するだろうか。早くホームページを立ち上げることをおすすめします。  
\\	あなたが天才でないからといって、がっかりする必要はありません。   
\\	あなたは人を単に貧しいからといって見下してはならない。   
\\	あまり才能ないからといってがっかりしてはいけない。   
\\	ある人が間違いをしたからといってそれを笑うのは不作法である。   
\\	ある人が間違いをしたからといってそれを笑うのは無作法である。   
\\	ある人間が間違いをしたからといってそれを笑うのは無作法である。   
\\	ある本が読者の興味を引かないからといって、その原因がその本にあるということにはならない。   
\\	いくら親しいからといってそんな事を彼に頼めません。   
\\	おいしいからといって、なんでも食べ過ぎてはいけない。   
\\	おいしいからといって食べすぎてはいけない。   
\\	おばは非常なお金持ちだからといって、それだけ幸せというわけではない。   
\\	そう言ったからといって彼の重要性が低下するわけではない。   
\\	その人が完璧からといって、恋に落ちるわけではない。   
\\	だからといって異議があるわけではない。   
\\	まちがえたからといって彼の事を笑うな。   
\\	感じるからといって、本当に居る訳じゃない  
\\	こととて	こととて	
\\	他でもない〜の(ひと・こと・もの)だから	
\\	何も知らぬこととて、間違いをしてしまいました。  
\\	店が狭いこととて、お客様には窮屈な思いをさせて申し訳ございません  
\\	何も知らぬこととて、失礼を致しました。  
\\	引っ越ししてきたばかりで、辺りの様子も分からないこととて、どうぞ宜しくおねがいいたします。  
\\	今回の転勤は急なこととて、ゆっくりに挨拶にも伺えませんでした。  
\\	何分にも若い二人のこととて、皆様のご指導をよろしくお願い垂オ上げます。  
\\	かたわら	かたわら	
\\	メアリーは読書をしており、1匹の猫がかたわらで眠っていた。   
\\	王妃は王のかたわらに立っていた。   
\\	審判はコートのかたわらにある高い椅子に座る。   
\\	仕事のかたわら、子供達にフランス語を教えるつもりです。  
\\	私はスペイン語を習うかたわら、趣味でフラメンコを習うことにしました.   
\\	私の母は自分で編み物を習うかたわら、人にも教えている  
\\	(酢渇)
\\	妹は会社に勤めるかたわら、夜は専門学校で英語を勉強しています。  
\\	〜の	〜の	
\\	「納得いかないんですよ」「何が?何がですの?」  
\\	男・女:もうお店やっていないの?!まだ4時だよ。  
\\	なんで彼女がいないの?  
\\	悪いけど他に用事があるの。  
\\	「どうしたの。」「少しほうっておいて。あなたには関係ないことなの。」  
\\	「今出かけてはいけない。」「どうして だめなの。」  
\\	私、去年から新しいことを始めたの。  
\\	「私、自動車にしょっちゅう乗ってるの。だから、クライアントが私にコンタクトが取れないって文句を 言うの。」  
\\	女:「時々、彼と離婚しようと思うの。」男:「冗談だろう!」   
\\	「太っている人はみんなダイエットすべきだと思う。」「あなた、私のことを言ってるの。」  
\\	7時半に起きなければいけなかったんじゃないの。  
\\	をどれくらい持っているの。  
\\	あなたのことを友達以上には考えられないの。  
\\	としては	としては	
\\	ミキさんは一年生としては英語が上手だ。  
\\	あの人は学者としては立派だが人間としては尊敬できない。  
\\	このステーキは日本のステーキとしては安いです。  
\\	ジョンソンさんは日本語の一年生としては、日本語が上手だ。  
\\	これは日本のアパートとしては、大きい方です。  
\\	というより	というより	
\\	私にはこの音楽は楽しいというより、むしろ やかましい.  
\\	彼は天才と言うよりむしろ変人だ  
\\	今年の夏は涼しいというより寒いと言った方が良いくらいです。  
\\	この雑誌は面白いというよりつまらないです。  
\\	彼女は可愛いというよりこわいです。  
\\	あの人はきれいというよりじょうひんだ。  
\\	「奥様はとても陽気な方ですね。」 「いやあ、陽気というより、うるさいというべきでしょうね。」  
\\	山田さんは書物が大好きで、技術者というより学者といった方がいい。  
\\	〜といい〜といい	〜といい〜といい	
\\	このスーツ、色といいデザインといい、お客様はよく似合いですよ   
\\	内容といい話し方といい、見事なスピーチでした。   
\\	家柄といい学歴といい、彼は非の打ちどころがない。   
\\	スピードといいパワーといい、今の彼に勝てる選手はいない。   
\\	については	については	
\\	お願いだから、二度とこの件については言及しないで。  
\\	氏については何とも申し訳ありません。  
\\	あなたの趣味について話してください。   
\\	あの件については彼女の指示に従っていいでしょう。   
\\	この問題についてはもう少しみんなでよく考えてみてください。  
\\	かれらの関係についてはさっぱりわからない。   
\\	このようなことについては直接お会いして、図面に書いたり、ロジックアナライザやオシロでタイミングを見たりしながら話し合ったほうが解決が速いです。   
\\	この奇妙な植物についてはほとんど知られていない。  
\\	この計画については黙っていてください。   
\\	この件については、危険はたとえあるとしてもほとんどない。   
\\	この件については、私達は政府に反対だ。   
\\	この件については彼が間違っているということを、私は彼にはっきり悟らせねばならない。   
\\	と言うと	というと	
\\	スイスというと何をイメージしますか?  
\\	なぜクビにならなかったかと言うと彼には超能力があり、彼の予知で数々の難事件が解決をみたからだ  
\\	何かというとあいつはいつも口を出そうとする。  
\\	何故かといわれても、知りません  
\\	は言うまでもなく	はいうまでもなく	
\\	いつものカクテルは言うまでもなく、フランス製のシャンペンも出された。   
\\	ここには、結果の不正確さは言うまでもなく、方法論上の問題が数多く存在している。   
\\	その先生は、英語は言うまでもなく、立派なフランス語を話す。   
\\	ジェーンはギターは言うまでもなく、バイオリンもひける。   
\\	ボストンは私には寒すぎるが、シカゴは言うまでもなくもっと寒い。   
\\	英語は言うまでもなく、彼はドイツ語も話せる。   
\\	核兵器は言うまでもなく、原子力発電も危険である。   
\\	核兵器は言うまでもなく、原子力発電所も危険である。   
\\	私たちにとって彼の勇気は言うまでもなく、彼の知恵にも感嘆した。   
\\	私達は彼の勇気は言うまでもなく、彼の知恵にも感嘆した。   
\\	人を嫌うのは言うまでもなく、人を信頼できないことが、人間の苦しみの根源だ。   
\\	彼の妻は今や彼らの2人の子供は言うまでもなく、彼の祖父も面倒を見なければならなかった。   
\\	彼はスペインは言うまでもなく、ポルトガルへも行ったことがある。   
\\	彼はフランス語は言うまでもなく、英語も話せない。   
\\	彼は英語とフランス語は言うまでもなく、ドイツ語も話せる。   
\\	彼は英語は言うまでもなくフランス語とドイツ語を話せる。   
\\	彼は英語は言うまでもなくフランス語も話せる。   
\\	彼は化学は言うまでもなく数学もほとんど知らない。  
\\	彼は和書は言うまでもなく洋書も読んでいる。   
\\	彼は贅沢品は言うまでもなく、日用品を買うのにも事欠く。   
\\	彼女は英語は言うまでもなく、フランス語とドイツ語も話せる。   
\\	どうしても	どうしても	
\\	どうしても事故は起こるものだ。  
\\	どうしても痩せたい  
\\	かのようだ	かのようだ	
\\	かれは全てを知っているのになにも知らないかのような顔をしている。  
\\	まるで生きているかのようだ。  
\\	彼はまるで父親に会ったことがないかのようだ。   
\\	あなたはなぜ彼のことをまるで老人であるかのように話すのか。   
\\	あの人はまるで私を知っているかのようにおじぎをしているわ。   
\\	その子供はまるで大人であるかのような話し方をする。   
\\	その男の子は、自分がまるで偉大な学者であるかのような話し方をする。   
\\	赤いドレスの彼女がほほえむと美しい花が開いたかのようだった。  
\\	(赤毛)
\\	まだ三月の初めなのにとても暑い。まるで夏になったかのようだ。  
\\	(赤毛)
\\	彼女はまるで徹夜をしたかのような顔をしていました。  
\\	(赤毛)
\\	この集会は下記のように開催されます。  
\\	我が家の屋根裏部屋は蒸し風呂のようだ。  
\\	彼女はやり手のようだ。  
\\	ならいざ知らず	ならいざしらず	
\\	辞書があるならいざ知らず、何もなくてはこの文章はとても読めるものではない。  
\\	専門家ならいざ知らず、素人ではこの化学品を作ることができない。  
\\	他の人ならいざしらず、私はそんなことができない。  
\\	そのとき	際して・時 (さいして 
\\	とき) 
\\	その時点	
\\	その時から貴方以外見えなくて」.  
\\	てたまらない	てたまらない	
\\	非常に	
\\	大学に合格して、うれしくてたまらない  
\\	家族からの手紙を読んでいたら、声が聞きたくてたまらなくなった  
\\	寒くてたまらない  
\\	不便でたまらない  
\\	あのカメラが欲しくてたまらない。  
\\	このうれしい知らせを家族に知らせたくてたまらない。  
\\	そのことを話してよ。聞きたくてたまらないの。  
\\	そのことを話してよ。聞きたくてたまらないの。  
\\	その話をしてよ。聞きたくてたまらないよ。  
\\	オーストラリアに行きたくてたまらない。  
\\	トムはそのリンゴがほしくてたまらない。  
\\	ニューヨークに行きたくてたまらない。  
\\	君は彼女に会いたくてたまらないのですか?  
\\	子供達は風船が欲しくてたまらない。  
\\	私たちは外国に行きたくてたまらない。  
\\	暑くてたまらない。  
\\	窓を開けてくれ。ここは暑くてたまらない。  
\\	痛くてたまらない。  
\\	彼は先ごろの成功を自慢したくてたまらない。  
\\	彼らは昼食が待ち遠しくてたまらない。  
\\	母に会いたくてたまらない。  
\\	冷たいものを飲みたくてたまらない。  
\\	限り	かぎり	
\\	アメリカ人でない限りこの仕事には就けない  
\\	嬉しい限りだ  
\\	彼女が太らない限り分かれない  
\\	私が知るかぎりでは、あの人は無実です  
\\	お金を持ってる限り働かない  
\\	私が 理解する限りでは、こんな風に使う  
\\	僕の日本語はあまり上手じゃないけど、「君が好き」と書ける限り、これで十分だよ。  
\\	あらん限りの大声で。  
\\	あなたの仕事に精を出さないかぎり成功しないでしょう。   
\\	あなたは静かにしているかぎりここにいてもよい。   
\\	あなたは彼と一緒にいるかぎり、幸福にはなれません。   
\\	さきうるかぎりの時間は読書についやす。   
\\	その国民の社会生活に関するかぎりでは、進歩は実に遅々たるものである。   
\\	できるかぎり話をはっきりさせよう。   
\\	われわれは生きているかぎりは働かねばならない。   
\\	ウェンディは必ずしも日本の祭りに興味があるとはかぎりませんでした。   
\\	給料に関するかぎり、それはよい仕事である。   
\\	君が最善を尽くすかぎりは君を援助してあげよう。   
\\	君はここに居るかぎり安全です。   
\\	見わたすかぎり砂のほかには見えなかった。   
\\	見渡すかぎり、すべては雪におおわれていた。   
\\	見渡すかぎり砂のほかには何も見えなかった。   
\\	思いもよらないことが起こらないかぎり、きっとうまくいくはずだ。   
\\	私が生きているかぎりあなたに不自由はさせない。   
\\	私が知るかぎり、1学期の間は何も問題がなかった。   
\\	私が知るかぎり彼女は大変よい人だ。   
\\	私たちは食料と水の蓄えがもつかぎりここにいられる。   
\\	私の知るかぎり、彼は信用のできる人だ。   
\\	私の知るかぎり、彼は無罪です。   
\\	私の知るかぎりでは、誰一人それをした人はいません。   
\\	私の知るかぎりでは。   
\\	私の知るかぎりでは彼は正直です。   
\\	私はできるかぎりの援助を彼にさしのべるつもりです。   
\\	私はできるかぎりの援助を彼に差しのべるつもりです。   
\\	私はできるかぎり熱心に働いた。   
\\	私は日曜日にはいつも家にいるとはかぎりません。   
\\	社会生活に関するかぎり、その進歩は実に遅々たるものであった。   
\\	食べすぎないかぎり、何を食べてもかまわない。   
\\	生きているかぎり、ご恩は決して忘れません。   
\\	生きているかぎり彼女は彼の親切を忘れなかった。   
\\	出来る限り協力します。  
\\	彼がここにいる限り私の仕事が増えるだろう。  
\\	彼はあまり疲れていない限り夜遅くまで働きます。   
\\	彼はテストのために勉強するのを忘れない限り、大丈夫だと思います。  
\\	につき	につき	
\\	この仕事は一時間につき十ドル貰える。  
\\	1ページにつき、3枚ずつコピーして下さい。   
\\	1ヤードにつきキス一回でごぜえやす。   
\\	20以上の注文につきましては、見積りより10%引きの特別価格でご提供します。   
\\	この牛肉は1ポンドにつき4ドルです。   
\\	その品物を1ダースにつき3ドルで買う。   
\\	それらは1ポンドにつき25セント安かった。   
\\	水道代は一ヶ月につき3000円かかります。  
\\	(赤毛)
\\	昼休みにつき、事務所は1時まで休みです。  
\\	(赤毛)
\\	医師1人につき患者が50人いた。   
\\	彼女は1年につき300万円稼ぐ。   
\\	それは1グラムにつき8円だ。   
\\	〜は	〜は	
\\	田中さんは大学生です。
\\	私はマイケルです。  
\\	林檎は緑です。
\\	皆さんは何処ですか? みんなさんはどこですか?   
\\	私はアメリカ人です。 わたしはアメリカじんです。   
\\	私たちは火星人です。わたしたちはかせいじんです。   
\\	という風に	というふうに	
\\	彼はどうしようもないという風に首を振った  
\\	これはどんな風に持ったらいいんですか?  
\\	こんな風に持って下さい  
\\	あんな風に運転していたら、きっと事故に会うよ。  
\\	こんな風にやりなさい。  
\\	それはこんな風に起こった。  
\\	こんな風にそれをしたのですか  
\\	にきまっている	にきまっている	
\\	著作権の侵害に決まってるだろーが  
\\	彼は負けるに決まっている  
\\	そんな子供じみた計画は失敗するにきまっている。   
\\	冗談を言っているにきまっている。   
\\	彼は怠け者にきまっている。彼が働いているところを見たことがない。   
\\	先月結婚した息子の家に行きたいけれど、私が行けば邪魔になるにきまっているので、行かないことにします。  
\\	〜っぱなし (〜っ放し)	〜っぱなし	
\\	新幹線の中では、東京から青森まで3時間立ちっぱなしだった。足が痛くなっちゃった!  
\\	ここに置きっぱなしにしちゃダメでしょ。  
\\	ゴミを散らかしっぱなしにしないでね。  
\\	友達のビデオが借りっぱなしだ。早く返さなくちゃ。  
\\	〜は	〜は	
\\	私は肉は食べない。  
\\	「セリーナをみせて!」「心配は要らない。いきてはいる、いきて。」  
\\	いや、でも野球を見に行くのは好きだけどね。  
\\	私は、足し算は好きだけど、引き算は嫌いだ。  
\\	彼は上海で住んでいたので、戦争は彼とは全く関係のないところで行われていた
\\	魚は食べますが、肉は食べません。さかなはたべますが、にくはたべません。  
\\	〜が	〜が	
\\	バスが来るよ。 
\\	台所にテレビがあります 。
\\	ここだと2万円前後のテレビが買える。
\\	この店ではほかでは売っていない特別仕様のCG機能を持つPCを買うことができる。  
\\	日本語ができます 「にほんごができます。」  
\\	雨が降っています。(あめがふっています)   
\\	お母さんがパンを買いました。 おかあさんがパンをかいました。   
\\	アリスさんがそれをしました。   
\\	お父さんが映画を見ました。 おとうさんがえいがをみました。   
\\	新聞がテエブルの上にあります。しんぶんがテエブルのうえにあります。   
\\	これが私の家です。これがわたくしのうちです。   
\\	だれがあの彼女ですか?  だれがあのかのじょですか?   
\\	雨がふっています。 あめがふっています。  
\\	から見ると	からみると	
\\	しかし、外から見ると、なぜこんなに非効率な作業をしているのだろう、と疑問に思うことが少なくない  
\\	私の立場からみると当然でしょう。  
\\	月から見ると、地球はボールのように見える。   
\\	現在の空模様から見るといつ土砂降りになるかわからない。   
\\	雇用者の数という見地から見ると、すべての工場のうちここは最も大規模なところだ。   
\\	山頂から見ると、その島はとても美しい。  
\\	私たちの家は西欧の基準から見ると小さすぎるとのことです。   
\\	私の目から見ると彼なんかまだ赤ん坊だ。   
\\	小屋の様子から見ると、そこには誰も住んでいないようだ。   
\\	〜を限りに	をかぎりに	
\\	今まで続いていたことが今後はもう続かなくなることを浮キ	
\\	今日を限りに、この学校ともお別れです。  
\\	その映画の上映は15日を限りに打ち切られることとなった。   
\\	人事部の山田さんは、明日を限りに退職します。   
\\	〜をもって (を以て)	〜をもって	
\\	道具や手段を表す。〜で。	
\\	彼は人一倍の努力をもって優勝カップを手にすることができた   
\\	こんな難問は、優秀な彼の頭脳をもってしても解決できないだろう。   
\\	身をもってその仕事の大変さを知った   
\\	彼の迫ヘをもってすれば,この問題の解決は容易であろう   
\\	〜ことから	〜ことから	
\\	文字のことから看板のことまでご相談下さい。  
\\	皆さん、話がおしゃむ君のことから逸れてるんですけど。  
\\	そういうことだから。  
\\	このことから当然彼は潔白にちがいないということになる。   
\\	このことから彼は当然無罪潔白にちがいないということになる。   
\\	ふとしたことからその噂は嘘だと分かった。  
\\	クモの巣を作るには、このきぬの枠を作り、それを木や垣根のような固い物体に固定させることから始める。   
\\	眼鏡の形をしていることから、その橋を眼鏡橋とよぶ。  
\\	店の前にいつも人が並んでいることから、人気のある店だということがわかる。  
\\	町並みが昔のままに保存され、古い寺も多いことから、その町は「小京都」と呼ばれている。  
\\	この地方には良質な水が豊富なことから、人々が酒造りをはじめるようになりました。  
\\	道がぬれていることから、昨夜、雨が降ったことが分かった。  
\\	二人が一緒に仕事がしたことから,交際が始まった。  
\\	かけだ	かけだ	
\\	何か始めたけれど、途中で止めてしまいました。完成はまだです。	
\\	お風呂に入りかけたときに電話が鳴った。  
\\	〜まい	〜まい	
\\	問題が起これば、学校だけで何とかしよう、外にはもらすまい。  
\\	一兵たりとも逃すまいぞ!
\\	あいつはそこまではやるまい。  
\\	ものだ	ものだ	
\\	携帯電話でメールをやり取りするのは楽しいものですね?  
\\	出来るものです  
\\	朝人に会ったら「おはよう」と言うものです。  
\\	意外でしたが、やってみたらこれが出来るものです。  
\\	あんな大事故にあって、よく助かったものだ。   
\\	地震のときは、だれでもあわてるものだ   
\\	人の話はよく聞く ものです。  
\\	子供のころ、いたずらをして、よく父に叱られたものだ。   
\\	弟:あなたの漫画を読ませてくださいな。 兄:とんでもない、あれは見るものではなく逃げるものだ。  
\\	(赤毛)
\\	子供は親の言うことを聞くものだ。  
\\	(赤毛)
\\	教職にある者は単に物質的快楽ばかり求める ものではない。  
\\	(赤毛)
\\	リーグは負け犬と言ってやればいいんだ。それがサッカーというものだ。  
\\	(赤毛)
\\	人間はアイドルを求めてしまうものらしい。  
\\	(赤毛)
\\	お母さん:どうして人参を残したの。子供:だって嫌いだもの。  
\\	(赤毛)
\\	若いころはよく親に反抗したものだ。  
\\	本ものだ。  
\\	生け花を習い始めて10年目になる田口さんは,自分ではまだまだ下手だと言っている.しかし,まだ2年目のわたしから見ればあの人の作品は実に見事な ものだ.  
\\	〜にたえる	〜にたえる	
\\	十分に〜に値する・どうにか〜することができる。	
\\	十分読むにたえる作品だ。  
\\	趣味で始めた焼き物だが、ようやく市販するにたえる作品ができるようになった  
\\	あの子は大人の鑑賞にたえる絵を描く  
\\	ずいぶん値段が安いけど、使用にたえるかどうか、ちゃんと確認して買った方がいい。   
\\	〜まい	〜まい	
\\	この病気で死ぬまいと思うが、念のため医者に相談して下さい。  
\\	ハイテクばかりが起業ではあるまい。  
\\	5人に1人が車を持つようになる日も遠くあるまい。  
\\	ああいう風ではたいした者になれまい。  
\\	いかなる状況のもとでも、王女が王位を継承することはあるまい。  
\\	この牛乳は明日までもつまい。  
\\	そのことですぐに何か手を打たねばなるまい。  
\\	それは君の役には立つまい。  
\\	そんな噂は誰も信じまい。  
\\	火星へ旅行できる日も遠くあるまい。  
\\	今度は罪をまぬがれまい。  
\\	少々のウイスキーを飲んでも害にはなりますまい。  
\\	彼はもう英語は教えまい。   
\\	彼女はそんなことはすまい。   
\\	こんな機会は二度とこまい。  
\\	清のパーティーにはもう行くまいと思った。  
\\	もう橋本には何も頼むまいと心に決めた。  
\\	恵子は清のパーティーにはもう行くまいと思った。  
\\	恵子は清のパーティーにもう行くまい。  
\\	彼は私のハンドバッグを取ろうとしたが、私は取られ(る)まいとして脇の下に強くはさんだ。  
\\	私は負け(る)まいと頑張った。  
\\	日本へ行こうか行くまいか迷った。  
\\	それは大した金にはなるまい。  
\\	って	って	
\\	知らないって  
\\	幾らたのまれたってできない  
\\	「あばたもえくぼ」って言うからね。  
\\	「テレビの音を小さくしてもらえませんか」って言ったんですよ!  
\\	彼が探してた鍵ってこれ?  
\\	[立ち入り禁止]っていうのは, 入ってはいけないっていう意味です.  [立ち入り禁止] 
\\	[使用禁止]は使うなっていう意味です.  [使用禁止] 
\\	果たして	はたして	
\\	果たして本当だろうか?  
\\	果たしてまた会うことができるだろうか。  
\\	彼は果たして成功した。  
\\	あいつは果たして失敗した。  
\\	あなたの言う事が果たして事実なのか調べてみよう。  
\\	だめだろうと思っていたが果たして失敗だった。  
\\	経営陣は果たして本気で我々のボーナスについて考えようとしてるのか、それとも単に口先だけだったのだろうか。  
\\	っこない	っこない	
\\	ノーベル賞なんて、僕は貰えっこない  
\\	そんなこと有り得っこない!  
\\	その点はまちがいっこない。  
\\	私がどんな目にあったか、君にはわかりっこない。  
\\	将来何が起こるかなんて、誰にもわかりっこない。  
\\	彼を彼の弟と間違えっこない。  
\\	頑固な妹はそんなに方法で心が溶けりっこない。  
\\	プロレスラーには勝てっこないよ!  
\\	せいで	せいで	
\\	寝不足のせいで、今日は頭がぼんやりしている  
\\	あいつのせいで、先生に叱られた。  
\\	気のせいか、彼は今日は何となく元気がなく感じられた。  
\\	私は失敗したのは、彼のせいだ。  
\\	電車が遅れたせいで、遅刻した。  
\\	暑いせいか、食欲がない。  
\\	台風のせいで、旅行にいけなかったんです。  
\\	(赤毛)
\\	天気のせいか、気分が悪い。   
\\	学生が上手にならないのは、教え方が下手なせいかもしれない。  
\\	熱がある せいで何を食べてもおいしくない。  
\\	彼女がパーティーに来なかったのは病気のせいだ。   
\\	私が失敗したのは怠惰のせいだ。   
\\	彼らの到着が遅れたのは雨のせいだった。   
\\	彼女の美貌は年のせいで衰えた。   
\\	ところ	ところ	
\\	今着いたところです  
\\	このホテルは公共交通機関の点から見ると便利な所に位置している。  
\\	あなたはそれを終えましたか。それどこか、今始めたところです。   
\\	あなたは昨晩私が彼女と二人っきりでいるところを見つけて驚いたにちがいない。   
\\	あなたは当分の間私のところにいても良い。   
\\	に加えて	にくわえて	
\\	サイズ的魅力に加えて画質的なノープロブレム感もある「ペンタックス 
\\	わたしは 日本語に加えて英語の文法もあまり分かりません。  
\\	このことに加えて、他にもまだ理由がある。  
\\	この増加に加えて、世界の経済構造の変化があった。  
\\	そのことに加えて、彼は試験に失敗した。  
\\	イギリスに加えて、フランスもドイツからの廃棄物の持ちこみを禁止しそうだ。  
\\	君が勉強した学校ではタイプに加えて常識というものを教えてもらわなかったかい。  
\\	豪邸に加えて、彼はベントレーを持っているのです。  
\\	有名な物理学者であることに加えて、彼は偉大な小説家でもある。  
\\	私は大学で医学生物学の勉強に加えて日本語も学ぶつもりです。  
\\	かと思うと	かとおもうと	
\\	これからこの状態が続くのかと思うと落ち込んでしまう。  
\\	今、自分が親にやってもらっていることをやらないといけないかと思うとぞっとします。  
\\	イギリス人かと思ったらアメリカ人だって  
\\	クリスがどんなにむかつくかと思うと、頭が痛くなりそう。  
\\	について	について	
\\	晩餐会の席でソフトウェアの新バージョンの発表について触れたい。  
\\	計画について  
\\	私が行くことについて  
\\	正午ごろについてどうですか?  
\\	コンピューターの使い方についての本がほしい。  
\\	私が子供の頃、日本という国について知っていたことは、東アジアの島国だということぐらいでした。
\\	ぐらい 
\\	私たちは趣味について話した。   
\\	彼女は旅行について随筆を書いた。   
\\	新聞はその地震についての長い記事を載せていた。   
\\	その件について私に報告を欠かさないで下さい。   
\\	判事はその事件について公正な評決を下すよう陪審に説示した。   
\\	子供の教育について誰かが親に責任を負うのか   
\\	その事件について彼方はどう思いますか   
\\	彼はその結果についてまったく無頓着だ。   
\\	どこで待ち合わせるかについて彼女は何か言っていましたか  
\\	前に	まえに	
\\	あなたが去る前には火を確実に消しなさい。  
\\	試験を答える前に、指示を読んでください。  
\\	朝ご飯を食べる前に手を洗って  
\\	たかだか (高高)	たかだか	
\\	この仕事だったらたかだか三日もあればできる。  
\\	パーティーの出席者はたかだか20人だろう。  
\\	たかだか五万円稼いだくらいでいばるな。  
\\	に応じて、に応じた	におうじて	
\\	世の中のお金の量であるマネーサプライは、経済活動の大きさに応じて伸び縮みします。  
\\	無理をしないで体力に応じた運動をして下さい。  
\\	車の速さに応じてガソリンの消費量が変わる  
\\	みなさんのご希望におうじて商品を生産していくつもりです。  
\\	(赤毛)
\\	彼の招待に応じてそのディナーへ出席した。   
\\	君がする仕事の量に応じて給料を払おう。   
\\	彼らはリンゴを大きさと形に応じて等級づけした。   
\\	ごとき, ごとく, ごとし	ごとき, ごとく, ごとし	
\\	実際はそうではないが、まるで〜のような/に (文語体)	
\\	烈火のごとき非難。  
\\	私のごとき初心者にとって、。。。  
\\	ジェット機のごとく空を飛ぶ。  
\\	以上のごとく  
\\	常のごとく  
\\	生けるごとく  
\\	鳥のごとく空を飛ぶこと、これは人間の夢である。  
\\	ライオンのごとく勇敢な人だ。  
\\	その少年は飛ぶがごとき勢いで、駆け去った  
\\	その少女はのごとき乙女であった。  
\\	前述したごとく、会議の日程が変更になりましたので、ご注意ください。  
\\	あの社長は世の中の全てが自分の物であるかのごとくふるまっている。  
\\	人生は航海のごとし。  
\\	東京人は万人走っているかのごとし。  
\\	彼女は女王のごとく芸能界に君臨している。  
\\	夢のごとき新婚生活の後に悲劇が訪れた。  
\\	〜ものなら	〜ものなら	
\\	行けるものならヨーロッパ へ(に) 行って見たい。でも、僕お金ないし、無理かな。  
\\	父は、たいへん厳しいんです。夜遅く帰ろうものなら、家に入れてもらえません  
\\	そうすれば目にみえるものならなんだって。   
\\	その伝言が重要なものなら彼に電話しなさい。   
\\	意志の弱い人は批判の的になる。だから友達でさえも、その欠点を直せるものならとおもってかれをいじめようとするだろう。   
\\	意思の弱い人は批判の的になる。だから友だちでさえも、その欠点を直せるものならとおもってかれをいじめようとするだろう。   
\\	仮にも為す価値があるものならば何でも良く為す価値がある。   
\\	自分のものならガチョウも白鳥に見える。   
\\	自分のものなら何でも最高。   
\\	選んでいただいたものなら何でも結構です。  
\\	彼はちょっと面白いものなら何でもとびつきます。   
\\	彼は甘いものならなんでも好きです。   
\\	楽しかった子供のころに、戻れるものなら戻りたい。  
\\	(誠)
\\	彼女に秘密を教えようものなら、すぐ皆に知られてしまう。  
\\	(誠)
\\	帰れるものなら、今すぐ国へ帰りたい。   
\\	やれるもんなら、やってみろ。  
\\	「変な格好」をしている若者が何か失敗でもしようものなら、その外見ゆえになにかと後ろ指を差される。  
\\	君とは渋谷に遊びに行けるものなら行きたい。  
\\	〜でなくてなんだろう	〜でなくてなんだろう	
\\	このケーキが世界一でなくてなんだろう。  
\\	あの国の大統領は国民の意見は無視して何でも自分一人で決定する。これが独裁者でなくてなんだろう。  
\\	自分の命を犠牲にして多くの人を救ったあの男が英雄でなくてなんだろう。  
\\	でなくてなんだろう 
\\	を禁じ得無い	をきんじえない	
\\	孤児には同情をきんじえない。  
\\	同情を禁じえない  
\\	彼の愚かな振る舞いに怒りを禁じえなかった。  
\\	不安を禁じえない。  
\\	彼は彼女に驚きを禁じえなかった。  
\\	途端(に)	とたん(に)	
\\	疲れていたので、ベッドに入ったとたんに、眠ってしまった。  
\\	座った途端  
\\	犯人は警官の姿を見たとたん逃げ出した。  
\\	店に入ったとたん、いいにおいがした。急にお腹が空いてきた。  
\\	一日中歩き回ってくたびれていたので、布団に入ったとたんに眠ってしまった。  
\\	がる	がる	
\\	小川さんはどうも新しい車を欲しがるんですね。  
\\	バカ!放せよ!彼女は苦しがってるじゃない!  
\\	お乳欲しがるあの子がかわい..  
\\	そんなに強がるんじゃねえ!  
\\	怖がっていないで、ジェトコースターに乗ってみて、おもしろいよ!  
\\	(タン)
\\	彼は、彼女と別れてから、いつも悲しがっています。  
\\	その吠えている犬をこわがるな。  
\\	彼女はいやがる子を学校に連れて行った。  
\\	〜さえ〜ば	〜さえ〜ば	
\\	地図さえあれば、何の迷いもなく目的地に到着します。.   
\\	お金さえあれば、この車を買います。  
\\	資料を調べ さえすれば、すぐ分かることです。  
\\	住所さえ分かれば、地図で探して行きます。  
\\	あなたが幸せでありさえすれば私は高
\\	ない。   
\\	それが欲しかったら頼みさえすればいいですよ。   
\\	彼女がもう少し速く来てさえいればなあ。   
\\	ボタンを押しさえすればよい。   
\\	だけ	だけ	
\\	彼女はよく働くだけでなく、魅力的だ。   
\\	お土産に十分なものを買うだけの十分な時間がありますか。  
\\	ローマ字だけ書けます。  
\\	日本語がこれだけ書ければとてもすばらしいと思います。  
\\	言いたいだけ言って、あいつは行ってしまった。  
\\	この本は読むだけの価値がある。  
\\	「何をさしあげましょう」「いや、結構。ただ見て回っているだけだから。」  
\\	一度だけチャンスが残っている。  
\\	あぁ、この用紙に書き込むだけです。  
\\	努力しただけ、結果が出ますよ。  
\\	これができるのは君だけだ。  
\\	午前中だけそれを借りていいですか。  
\\	新聞だけではなく週刊誌にも出ていた。  
\\	東京だけでもそのような事件は10件あった。  
\\	「ごめん」と一言だけでも言ってほしかった。  
\\	言ってみただけだ。  
\\	一目見ただけでその家が気に入った。  
\\	そう言った(それを見た)だけで彼は怒った。  
\\	ちらっと見せてくれるだけでいい。  
\\	やるだけはやってみなさい。  
\\	日曜日は、寝たいだけ寝ることにしているので、起きるのは昼ごろになる。  
\\	今日はあなたのお誕生日だから、特別よ。ケーキやチョコレートも好きなだけ食べていいわよ。  
\\	電力不足が心配されています。電気をできるだけ節約しましょう。  
\\	たくさんありますから、どうぞほしいだけお取りください。  
\\	できるだけ、がんばりましょう。  
\\	(赤毛)
\\	最近は職場だけでなく家庭においてもパソコンが使われている.  
\\	それはかえって家庭問題をさらに悪化させるだけだろう。  
\\	性だけを大文字で書きなさい。  
\\	私にお礼を言ってくれたのは彼らだけだった。  
\\	彼女をただ愛しているだけではない。  
\\	それについて考えただけで胸がつまってが出た。  
\\	ただ金もうけのためにだけそれをした。  
\\	桜さんだけがそこに行った。  
\\	彼女はかぜをひいただけで欠席した。  
\\	彼女は単に彼にうまく利用されただけだろう。  
\\	テストのことを思うだけで不安になった。  
\\	そこへ行くことを考えるだけでも恐ろしい。  
\\	今度だけは見逃してください。  
\\	彼女は彼の愛だけが欲しかった。  
\\	聞こえるのは波の音だけだ。  
\\	君に言っているのは事実だけです。  
\\	〜ずにはいられない	〜ずにはいられない	
\\	買い物せずにはいられない  
\\	愛さずにはいられない。  
\\	きみがまちがっていると思わずにはいられない。   
\\	しかし今は、一目見て、サメがその人にくらいつくのは避けられないようだということに気づかずにはいられない。   
\\	その本を読むと誰でも泣かずにはいられない。   
\\	それを笑わずにはいられない。  
\\	君の愚かさには笑わずにはいられない。   
\\	私は、彼の才能を称賛せずにはいられない。  
\\	私はあなたを好きにならずにはいられない。   
\\	私はあなたを笑わずにはいられない。   
\\	私はくしゃみをせずにはいられない、きのうかぜをひたんだ。   
\\	私はその少女を気の毒に思わずにはいられない。   
\\	私はその冗談を聞いて笑わずにはいられない。   
\\	私は英国の田舎に愛着を感じずにはいられない。   
\\	私は自国を愛さずにはいられない。   
\\	私は笑わずにはいられない。  
\\	私は赤ちゃんを見るとほほえまずにはいられない。   
\\	ぼんやり	ぼんやり	
\\	ぼんやりと暮らす	
\\	ケンは何時も授業中ぼんやりしています。   
\\	海の向こうに島がぼんやり見えます  
\\	彼女はぼんやりと窓の外を眺めていた。  
\\	〜に沿って	〜にそって	
\\	いきなり線に沿って切らずに、まず大まかに型抜きをする。そして、線に沿って切り抜く。  
\\	彼の計画に沿って物事を進めた  
\\	私達はパレードが通りに沿って進んでいくのを見た。   
\\	川に沿って歩きなさい、そうすれば学校が見えるでしょう。   
\\	通りに沿って桜が植えられている。   
\\	通りに沿って店が並んでいる。   
\\	彼は川に沿って歩いた。   
\\	彼は通りに沿って車を運転していた。   
\\	彼らはアフリカの西海岸に沿って航海した。   
\\	郵便局は通りに沿ってあります。すぐわかるよ。   
\\	社の方針に沿って、既存コンテンツの機種対応に追われている様子。   
\\	この川に沿って一時間ぐらい歩くと、海に出ます。   
\\	子供たちの希望に沿った公園が作られた。その公園は大人の考えたものとはかなり違っていた。   
\\	会社の経営方針に沿って、来年度の計画が立てられた。  
\\	両岸に沿ってずっと並木がある。   
\\	道路は海岸に沿って延びている。   
\\	上で述べた線に沿って。。。   
\\	とはいえ	とはいえ	
\\	とはいえ 
\\	は
\\	は事実だが、しかし実際は
\\	の意味	
\\	相手に言い分もわかったとはいえ、心から納得したわけではない  
\\	(赤毛)
\\	春になったとはいえ朝晩はまだ寒い。  
\\	(赤毛)
\\	幼いとは言え、その子は自分の家庭の苦しい事情を理解している。  
\\	(赤毛)
\\	便利とはいえパソコンは高いのでなかなか買えない。  
\\	(赤毛)
\\	80才の祖母は、この間階段で転んで足を痛め、歩くのが不自由になってしまった。とはいえ、全く歩けないということではないので、家事をするには問題ないとのことだ。  
\\	(赤毛)
\\	時間がなかったとは言え、ジェーンにこれくらいのレポートが書けないはずはない。  
\\	(赤毛)
\\	いかに暫定的処置とは言え、これはずさんすぎる  
\\	(赤毛)
\\	地球温暖化を防止するための国際会議の結果、京都議定書が作成された。とはいえすべての国がその基準を守るまでには、まだ年月がかかりそうだ。  
\\	今度の学会が楽しみだ。とは言え、まだ発表準備は全然できていない。  
\\	〜をものともせずに	〜をものともせずに	
\\	にもかかわらず	周囲の反対をものともせずに、二人は結婚した。	
\\	彼女は足の痛みをものともせずにマラソンを走り抜いた。  
\\	雪の深い道をものともせずそいつはひたすら車を駆った。  
\\	敵の攻撃を物ともせず部隊は突進した。  
\\	周囲の反対をものともせずに、二人は結婚した。  
\\	寒さをものともせずに、彼は薄着で外出した   
\\	氷のような水をものともせずに彼は川へ飛びこんだ   
\\	そのおてんば娘は危険をものともせずにその木に登った   
\\	立木さんは病気をものともせずにレースを走り切って勝った。  
\\	ないで	ないで	
\\	朝ご飯を食べないで学校へ行った。  
\\	払わないで去る  
\\	宿題をしないで、学校へ行った。  
\\	(反戦)
\\	今さら〜たところで	いまさら〜たところで	
\\	今さら説明したところで先生は許してくれないと思う。  
\\	今さら騒いだところで、どうにもならないよ。後の祭りだよ。  
\\	今さら、戻ったところで、彼はもうおまえを愛しないようになった。   
\\	この問題をいまさら議論したところで無意味でしょう  
\\	子供時代に過ごした村を訪ねたいと思っていますが、いまさら行ったところで、古いものは何も残していないでしょうね。   
\\	次第	しだい	
\\	その結果を聞き次第  
\\	事と次第によっては許さないよ。  
\\	ビアガーデンに客がたくさんいるかどうかは天気次第だ。   
\\	ホームページの人気は内容次第。   
\\	雨がやみ次第、家に向かって出発しよう。   
\\	駅に着き次第、電話します。   
\\	音は次第に弱くなり、そうしてとうとう聞こえなくなった。   
\\	母は赤ちゃんの世話をしてもられ次第、私の仕事の未来でしょう。  
\\	手当たり次第に土地を買い占める。   
\\	万事は君のこの取り扱い方次第だ。   
\\	私たちが行くか行かないかはあなた次第です。   
\\	昇進は迫ヘ次第である   
\\	この世のなかはおかねしだいと言う人もいる。 
\\	この世のなかはおかねしだいと言う人もいる。 
\\	ことになっている	ことになっている	
\\	ここでは禁煙ということになっている.  
\\	法律で、子供を働かせてはいけないことになっている。  
\\	8時に駅で会うことになっている。   
\\	の低さの原因を人種に求めるという議論を論駁するどころか、リンのデータはそれを事実上補強することになっている。   
\\	明日スミスさんのお宅に伺うことになっている。  
\\	ここで野球をしてはいけないことになっている。   
\\	この契約によって私は彼らに1万ドル払わなければならないことになっている。   
\\	この劇場では酒は飲めないことになっている。   
\\	この建物からでては行けないことになっている。   
\\	この宿題は月曜日に提出することになっている。   
\\	この貸付金は高い利子をとられることになっている。   
\\	そのパーティーではだれもがネクタイをすることになっている。  
\\	娘は来年から留学することになっている。  
\\	試験が受けられなかった学生はレポートを提出することになっている。  
\\	学校を欠席する場合は、必ず連絡をしなければならないことになっている。  
\\	山田さんとの約束では、3時までに東京駅に行くことになっている。  
\\	(赤毛)
\\	山田先生の授業は、試験を受けるかわりにレポートを出してもよいことになっている。  
\\	(赤毛)
\\	この部屋には、関係者以外入ってはいけないことになっている。  
\\	(赤毛)
\\	午前中は文法を勉強する。午後は会話を中心に勉強することになっている。  
\\	(赤毛)
\\	今度の日曜日には、友達と会うことになっています。  
\\	この寮は,玄関は12時に閉まることになっている。  
\\	この席は、たばこを吸ってもいいことになっている。  
\\	60点以上が合格ということになっている。  
\\	つれて、につれ	つれて、につれ	
\\	ば...ほど	
\\	山は高くなるにつれて、気温が下がる。  
\\	だから山は高くなるにつれて、気温が下がる。  
\\	だから山は高くなるにつれて、気温が下がる。  
\\	やっと	やっと	
\\	彼はやっと試験を合格できた。  
\\	数学の問題はやっと分かったか。  
\\	じかんがかかりましたが、やっとおわりました。  
\\	(赤毛)
\\	〜てこそ	〜てこそ	
\\	「〜てはじめて」と同じ意味だ。	
\\	スポーツでもゲームでも自分でやってこそ、おもしろさが分かる。  
\\	日本の文化は、日本へ行ってこそ分かるものだ。  
\\	ぬく	ぬく	
\\	完全に〜する; 最後まで〜する	
\\	頑張りぬく  
\\	あなたは多くの困難に耐え抜かなけねばならない。  
\\	頑張りぬいたが結局できなかった。  
\\	困難にあっても最後までやりぬく気持ちを捨てないで、がんばりなさい。   
\\	難しい数学の問題を考えぬいて答えが出せたときは、うれしかった。   
\\	勉強にはぜんぜん興味を持たない子供に、両親は困りぬいている。  
\\	私の祖父は2つの戦争を行き抜いた。   
\\	最後までやり抜け。  
\\	〜な	〜な	
\\	バカなことするな!  
\\	あまり 飲むな   
\\	馬鹿なことをするな。   
\\	誰も動くな。  
\\	心配するな。  
\\	気にするな。   
\\	決して約束を破るな。   
\\	下さい	ください	
\\	それをください。  
\\	みせてください。  
\\	あの方が言われる事を、何でもしてあげてください。  
\\	みなさま、ようこそご搭乗くださいました。  
\\	「塩をとってください。」「はい、どうぞ。」  
\\	「芝生に入らないでください」という標識があった。  
\\	彼によろしくお伝えください。  
\\	5時にお電話下さいませんか。  
\\	あのビールをください。  
\\	あの窓を閉めてくださいませんか。  
\\	この花を送ってくださったのは佐藤さんかもしれません。  
\\	お母さんが私にかわいらしいお人形を買ってくださいました。  
\\	ようこそパーティーにお越しくださいました。  
\\	カードをもう一枚ください。   
\\	ビール二本 ください。  
\\	何か飲み物を下さい。   
\\	万一留守中にだれかが訪ねて来たらすぐにもどると伝えください。   
\\	彼女が来るかどうか知らせてください。   
\\	どうぞうお入りください。   
\\	お名前が聞きとれませんでした。 もう一度言ってください。   
\\	これをどうやったらよいか教えてくださいませんか  
\\	に際して	にさいして	
\\	卒業に際して、私たちは感謝の印として彼に置き時計を贈りました。  
\\	読書に際しては、過去の偉大な作家に最も多くの注意が払われなければならない。   
\\	ご予約に際し、クレジットカードの番号および有効期限などの情報が必要です。  
\\	卒業式に際して、市長からお祝いの言葉が卒業生に送られた。   
\\	工場見学に際して、係の人から様々な注意をうけた。  
\\	たりとも	たりとも	
\\	ご飯の一粒たりとも無駄にするな!  
\\	こんなテレビゲームをやる時、一瞬たりとも気を抜くことが出来ない。  
\\	1センチたりとも進まない。  
\\	一分たりとも無駄に出来ないぞ。  
\\	試合の最中は、一瞬たりとも油断はできない。  
\\	何人たりともここに入ることは許されない。  
\\	訳	わけ	
\\	と言うわけで、
\\	僕は決して農業が嫌いな訳じゃない。   
\\	私は訳もなく悲しくなった。  
\\	わけありの二人  
\\	それを纏めるのはわけはない  
\\	それとこれとはわけが違う  
\\	そういうわけには行かない.  
\\	訳の判らぬことを言うな。  
\\	たとえアルバイトしても、勉強を残るわけにはいかない。.  
\\	そんなにたくさん食べられる訳ないよ!   
\\	彼女の部屋に行った事がある訳ないよ!  
\\	毎日三時間も日本語を勉強しているんですか。 よく出来るわけですね。
\\	いかん (如何)	いかん	
\\	「いか(如何)に」の略。どのようであろうか。	
\\	試験の結果いかんでは、卒業できないこともある  
\\	実力テストの結果いかんでレベルを決める  
\\	天気いかんにかかわらず来ます  
\\	占いの結果いかんでは、アメリカに帰ることになります。  
\\	天気いかんでは行きます  
\\	それは如何ともしがたい。  
\\	その問題はあなたの態度如何によって決まることだ。  
\\	スパイダーマンの運命や如何  
\\	その試合には国籍のいかんを問わず誰でも参加出来る。   
\\	幸福は環境いかんによるものでなくて、むしろ自分の人生に対する見方いかんによる。   
\\	仕方がない、それはいかんともしがたい。   
\\	成功するかどうかは、君自身の努力いかんによる。   
\\	成功は努力いかんによる。   
\\	天候のいかんにかかわらず駅へ迎えに参ります。   
\\	肌の色のいかんを問わず、彼は万人の言論の自由を擁護した。   
\\	彼の命は判決いかんにかかっている。   
\\	をもとにして	をもとにして	
\\	以前は経験を元にして設計していましたが、 今や経験と勘だけではレースに勝てるような マシンを作ることはできません。  
\\	この映画は小説をもとにしている。   
\\	自分が集めた事実をもとにして、科学者は、その事実に意味と秩序と勝ちを与える論理の模様すなわち理論を織りあげるのである。   
\\	彼は親譲りの財産をもとにして富を作った。   
\\	いくら (幾等)	いくら	
\\	葉書は幾らですか。 
\\	「それはいくらかかりましたか。」
\\	000円かかりました。」  
\\	いくらお金をもってますか。  
\\	それがいくらするのか前もって彼に聞いたほうがよい。  
\\	このTシャツの値段はいくらですか。  
\\	このTシャツはいくらですか。  
\\	このオートバイがいくらか、私は知りません。  
\\	このカメラはいくらですか。  
\\	このスイミングスクールの月謝はいくらですか。  
\\	このツアーは1人いくらですか。  
\\	このバスの料金はいくらですか。  
\\	このペンはいくらしますか。  
\\	このホテルの宿泊料金はいくらですか。  
\\	イクラはいくらですか  
\\	このシャツと靴はいくらですか。  
\\	あるまじき	あるまじき	
\\	〜立場の人にあってはならない  〜立場の人はしてはいけない	
\\	他人の文章を盗んで書くとは、プロの作家にあるまじき行為だ。   
\\	その言葉は市長にあるまじきものであった。   
\\	人のお金を盗むとは、警察官にあるまじき行為だ。   
\\	にかけては	にかけては	
\\	卓球にかけては弟におよばない  
\\	法律にかけては私はよくわからない  
\\	わが社の製品は知名度は低いものの、性狽フ良さにかけては世界の最高水準だ。  
\\	であろう	であろう	
\\	あなたの助けがなかったら、我々は仕事を間に合わせられなかったであろう。  
\\	おそらくそれは嘘であろう。  
\\	きっと彼女の病気はすぐによくなるであろう。  
\\	それはナンセンスだ。誰もそれを信じないであろう。  
\\	現在の私たちにとって、もっとも重要なのは、地球の環境を守ることであろう。自分と関係がないと考えずに、自分の問題として、この問題を考えることが求められている。  
\\	(赤毛)
\\	のみならず	のみならず	
\\	東アジアのみならずヨーロッパでもインターネットではユニコード化が進んでいると聞き及んでおります。  
\\	時は虚偽のみならず真実も明らかにする。  
\\	日本人学生は極度に内気であるのみならず、時としてほとんど全く話したがらないように見える。   
\\	彼が日本のみならず、世界においても有名な物理学者である。   
\\	彼は英語のみならずフランス語も話すことができる。   
\\	彼女のみならず彼女の息子達も幸せだった。   
\\	僕はこの女の人のみならず、あの女の人も大好きです。  
\\	たとえ	たとえ-ても	
\\	たとえ医者でもガンには勝てません。  
\\	たとえ冗談でも、それはないでしょう.  
\\	たとえどんなことがあったとしても....  
\\	たとえ難しいことが多くても、成功するためにがんばろう。  
\\	たとえきらいでも、野菜を食べないわけにはいきませんよ。  
\\	たとえ手術を受けても、完全には治らないだろう。  
\\	たとえみんなに反対されても、私は正しいと思ったことをする。   
\\	たとえ雨が降っても出かける。  
\\	たとえ彼がいやだと言っても、私たちの言い分を彼に聞いてもらう。   
\\	両者の間には、たとえあったとしてもごくわずかしか、相違はありません。   
\\	たとえ100歳まで生きても、私はピカモ
\\	理解できないでしょう。   
\\	運転する人は、たとえどんなに腕がよくても、運転免許試験に合格しなければならない。  
\\	たとたん	たとたん	
\\	[とたん(に)] 
\\	おもちゃを取り上げたとたん、子供は激しく泣き出した。  
\\	窓を開けたとたんに小鳥が外へ逃げてしまった。  
\\	家に着いたとたん嵐になった。   
\\	帰宅したとたん雨が降り始めた。   
\\	私が立ち去ったとたんにけんかが始まった。   
\\	べきだ	べきだ	
\\	子供たちには真実を話すように教えるべきです。  
\\	私が彼を助けるべき理由はまったくありません。  
\\	「金を選ぶべきだった。」   
\\	「太っている人はみんなダイエットすべきだと思う。」「あなた、私のことを言っているの。」  
\\	10月に総会を開催すべきだという提案があった。  
\\	あなたにはもっと早く全てを打ち明けるべきだった。  
\\	あなたはこの機会を利用すべきだ。  
\\	あなたはそれを初めに言っとくべきだった。  
\\	あなたは、この二行を省くべきだ。  
\\	年上の人には敬意を払うべきだ。   
\\	彼女を軽視すべきではない。  
\\	トムはきのう歯医者に行くべきだったのに。   
\\	彼女の欠点には目をつぶってやるべきだ。  
\\	通商障壁は低くすべきだ。  
\\	刃物は赤ん坊の手の届かない所に置くべきだ。  
\\	ジョージは体重が減っている。医者にみてもらうべきだ。  
\\	どんな場合でも、まじめに職を捜している人は職業安定所に当たってみるべきです。  
\\	借金せずに暮らすぐらいの分別はあってしかるべきだ。  
\\	と同様	とどうよう	
\\	同じである事。	大都市と同様、地方でも交通事故が増えている	
\\	多くのアメリカ人と同様に彼は気さくだ。  
\\	あればこそ	あればこそ	
\\	皆様の協力があればこそ無事故で完工できたことを感謝しております。  
\\	きものを着ると言う行為は、洋服以上に全身の筋肉を事細かに使うことにも驚かされます。何事もほんとうに健康で あればこそですね。  
\\	エア切れで20メートルの深度からエアエンボにもかからず浮上できたのも、日頃の訓練あればこそでした。  
\\	私にはその仕事がやさしいとわかった。前にもそんな仕事をしたことがあればこそだ。  
\\	あなたのためを思えばこそ、厳しくしているのだ。  
\\	頭がよければこその悩みですね。  
\\	優勝できたのは、チーム全員の協力あればこそだ。  
\\	抜き	ぬき	
\\	全部食べないとデザート抜きですよ。  
\\	その話はぬきにして飲みましょう.  
\\	昼飯ぬきで仕事をした.  
\\	私は、塩ぬきのフライドポテトを注文した。  
\\	冗談は抜きにして  
\\	冗談抜きで、ちょっと大切な相談があるんです。   
\\	お世辞抜きに、君の日本語は本当にうまいよ。   
\\	コーヒーは、砂糖ぬきでお願いします。  
\\	(赤毛)
\\	冗談ぬきでまじめに考えて  
\\	(赤毛)
\\	朝食ぬきで学校に行く子供が多いです。   
\\	堅苦しいあいさつ抜きでやりましょう。   
\\	コーヒー抜きで働くのは苦痛であった。   
\\	てからというもの	てからというもの	
\\	てから後はずっと	
\\	娘が帰って来てからというもの、年老いた父親を見違えるほど元気になった   
\\	水泳を習い始めてからというもの、冬でも風邪を引かなかった。   
\\	〜に足る	〜にたる	
\\	信頼するに足る人物だと思っていた、その人にだまされてしまった  
\\	いやしくもなすに足る事なら立派にやるだけの価値がある。   
\\	この農園は私たちの必要を満たすに足るだけの野菜を産出する。   
\\	ビルは印刷業でいい暮らしをするに足るお金をかせいだ。   
\\	大衆によって重んじられている人が必ずしもそれを受けるに足る人とは言えない。   
\\	君たち全員に足るだけのパンがある。   
\\	色々	いろいろ	
\\	ワインには色々なタイプがあり、それによって育成の仕方もさまざまです	
\\	御社のことは同僚より聞き及んでおります。  
\\	やるって言ってませんでしたっけ?   
\\	お褒めの言葉ありがとうございます  
\\	了解です、進めて下さい。   
\\	先日の件ですが、添付した契約の翻訳をして貰いたい。
\\	添付しました契約を翻訳して頂きたい。   
\\	いつまでに届けて頂けるか、お知らせ下さいますか?  
\\	〜まい	〜まい	
\\	母に言うと心配するから、このことは母に言うまい  
\\	もう言うまい。  
\\	行くまいと決めた。  
\\	彼女は慰謝料など払うまいと決心した。   
\\	彼女は寝るまいと決心しました。  
\\	までに	までに	
\\	明日までにそれを済ませてしまうべきです。  
\\	お昼までに戻らないといけない。   
\\	いかにも	いかにも	
\\	あのかばんはいかにも重そうだが、中身はからで、1キロしかない。  
\\	あの人は見かけはいかにもこわそうだが、本当はとても優しい人です。  
\\	いかにもあいつの言いそうなことだ。  
\\	あのスーツはいかにも高そうだ。  
\\	いかにもビルのやりそうなことだ。  
\\	いかにもリゾートって感じの格好ね。  
\\	こんなおふざけはいかにもベーカー氏らしい。  
\\	こんな格好をしているといかにもスポーツをしてそうに見えるけど、全然やってないんだよ。  
\\	そうしたしゃべりかたはいかにも彼らしい。  
\\	そうした話し方はいかにも彼らしい。  
\\	そのような行いはいかにもメグらしい。  
\\	そんな発言はいかにも彼らしい。  
\\	事象として簡単なことを、いかにも難しそうに表現する人は、あまり頭がよさそうではない。  
\\	遅刻するなんていかにも彼らしいと思う。  
\\	遅れてやってくるのは、いかにも彼のやりそうなことだった。  
\\	朝食前に出勤するなんていかにも彼らしい。  
\\	彼はいかにも「営業」って感じだね。  
\\	彼はいかにも私の親友らしく見せかけた。  
\\	彼女の口振りはいかにもそらぞらしい。  
\\	有り得る	ありうる	
\\	自分の携帯のデータが他人に流出してる事も有りうるって事ですからねぇ  
\\	そんなことがあり得るということ自体がおかしいですね  
\\	いかなる善意の考えもそれ自体は悪でありうる。  
\\	それは大いにありうることだ。   
\\	計画を立てるさいに、あまりにも慎重すぎると、計画そのものが、ひっくりかえることも、時にはありうるだろう。   
\\	人間が、動物から病気をもらうことは、まれではあるがありうる。   
\\	人生はあきらめが肝心。もうけた財産も消えさることがありうるのだから。   
\\	人生はあきらめが肝心。儲けた財産も消えさることがありうるのだから。  
\\	先生でも間違いをすることはありうる。   
\\	彼が来ることはありうると思う。   
\\	それはありうることだ。   
\\	しか	しか	
\\	ローマ字しか書けません。  
\\	田中さんだけ来ました。  
\\	田中さんしか来ませんでした。  
\\	百円しかない。  
\\	赤いものしかない。  
\\	そこは車でしか行けない。  
\\	英語しか知りません。  
\\	彼らしか私に礼を言わなかった。  
\\	ほんの数人の友達しかいない。  
\\	火曜のフットボール試合の観客はわずかしかいなかった。   
\\	彼女は金を浪費することしか能ない。   
\\	彼はせいぜい平均点しか取れないだろう。   
\\	彼女はせいぜい5千円しか払わないだろう。   
\\	君に賛成だとしか言えない。   
\\	彼女は金もうけのことしか考えない。   
\\	やむをえない (已むを得無い)	やむをえない	
\\	しかたがない そうするより方法がない	この大雪では、試合を中止にするのもやむをえない 
\\	やむを得ず雨天で試合が中止になる場合はある。  
\\	やむを得ず雨天で試合が中止になる場合はある。  
\\	にあって	にあって	
\\	幼児期にあって、彼はストレスの多い状況を経験した。  
\\	この核時代にあっては、世界中どこの国だって安全ではない。   
\\	その行いが神にあってなされたことが明らかにされるためである。」   
\\	その宝はまだ森の中にあって、そこの木の下に埋めてある。   
\\	それは信じるものはみな、人の子にあって永遠の命を持つためです。   
\\	自然がいつもそこにあって私たちを助けてくれることを当然だと思ってはいけない。   
\\	日本は完全に近代化したという点で、アジア諸国の中にあって独特の存在である。   
\\	彼の家は丘の中腹にあって、見晴らしがすばらしい。   
\\	彼は混乱のただ中にあっても冷静だ。   
\\	彼女のドレスは私の趣味にあっていない。  
\\	豊かさの中にあっても餓死する人がいるのは残念なことである。   
\\	数学は高度情報社会にあって、必要な教養となっている。  
\\	彼女はどんな逆境にあっても、決して希望を失わなかった。   
\\	に伴って	にともなって	
\\	とともに; につれて; に従って	
\\	地震に伴って火災が発生することが多い。  
\\	軽石噴火:粘性の高いマグマの活動に伴って発生す ることが多い。  
\\	少子化に伴い高齢者が高齢者を介護するようになるだろう。  
\\	人口が増えるに伴って、 いろいろな問題が起ってきた。  
\\	経済発展に伴う環境破壊が問題になっている。  
\\	ペットの販売に関するトラブルが、最近のペットブームに伴って増加しています。  
\\	秋が深まるに伴って、山の紅葉が進む。   
\\	地震に伴う火災で、多くの人命が失われた。  
\\	発熱に伴って、吐き気もしてきた。  
\\	(赤毛)
\\	この町でもどんどん人口が増加している。それ にともなってスーパーや銀行もでき、便利になってきた。  
\\	なければなりません	なければなりません	
\\	野菜を食べなければなりません。   
\\	お店で卵を買わなければなりません。なければなりません。  
\\	なければなりません 
\\	〜反面	〜はんめん	
\\	その反面  
\\	普通兄は優しい反面、卑しいところもある。  
\\	その仕事は十分にお金になるが、その反面1日に12時間働かなくてはならない。   
\\	この布は水に強い反面,熱には弱い.  
\\	インターネットは子供にとって役に立つ学習ツールだが、反面、しっかりその使い方を指導しない子供がポルノや沫ヘシーンの画像に簡単に接してしまうことになりかねない。  
\\	しまう (仕舞う・終う・了う)	しまう	
\\	もう〜。また皿を壊してしまったか。  
\\	夢に耽ったり、生きることを忘れてしまうのはよくない。  
\\	あなたは笑われてしまうでしょう。  
\\	より	より	
\\	私は肉より魚の方がいい。  
\\	現実は聞いたよりももっと悲しい。  
\\	これよりほかに方法がないんだよ  
\\	一般に日本人は、自分たちがかつてより暮し向きがよくなったと考えている。  
\\	彼はわたしより速く走ることができる。  
\\	それより、バスを降りてから電話してください。  
\\	私は彼よりサッカーがうまくできる。  
\\	に関して	にかんして	
\\	最近テロに関してニュースがよく新聞に出ています   
\\	あの件に関しては、画竜点睛を欠いてしまった。   
\\	このことに関してまた君に電話するよ。   
\\	この議事録に関して、訂正、追加がありましたらお願いします。   
\\	この件に関して、あまりお役に立てなくてすみません。   
\\	この件に関してはまったく知らない。   
\\	この件に関しては彼らに賛成できない。   
\\	この件に関して何か言うことはありますか。   
\\	この研究の最初の部分では、大衆統制の心理に関してすでに行われている研究について言及する。   
\\	この現象に関してはまだ物理的な説明がなされていない。   
\\	この事に関してはあなたの意見に賛成です。   
\\	この問題に関して、3つの問題がある。   
\\	大気汚染に関する報告書が委員会に提出された。  
\\	先週、東京で超伝導に関する学会が開かれた。  
\\	インターネットの利用に関するアンケートを行なう。  
\\	ないことには	ないことには	
\\	「人間は意味のないことには耐えられない」と言ったのは、フランスの哲学者のサルトルだったでしょうか。   
\\	誰かに聞かないことには難しい問題にいつも答えることができません。  
\\	自分に関わりのないことには口を出すな。  
\\	その食べ物を食べてみないことには好きかどうかわからない。  
\\	住所がわからないことには、連絡しようがありません。   
\\	小説を書いても、おもしろくないことには売れませんよ。   
\\	何といっても、健康が一番。いくら有能でも、健康でないことには活躍できない。  
\\	頑張らないことには今度の試験に合格できないよ。  
\\	もっと付き合わないことには、彼が実際どんな人かは分からない。  
\\	(赤毛)
\\	どんな人か、実際あってみないことには、よくわからない。  
\\	(赤毛)
\\	一度行ってみないことには、どんな所かわからないだろう。  
\\	(赤毛)
\\	今までにない型の伝染病が各地に広がっている。くわしい調査をしないことには、政府も対策がたてられない。  
\\	(赤毛)
\\	文章を書くということは、何かを表現することであり、自分は何を表現したいのか、それをはっきりさせないことには、何も書けない。  
\\	(赤毛)
\\	寮が立派でないことには、寮費は高くない。  
\\	かかわらず	かかわらず	
\\	このデパートは曜日にかかわらずいつも込んでいる  
\\	金額の多少にかかわらずいつでもご相談下さい。  
\\	そのような境遇であったにもかかわらず、彼は自分一人で道を切り開いた。   
\\	その患者はその薬を飲んだにもかかわらず少しも気分が良くならなかった。   
\\	その計画は強い反対にもかかわらず続行された。   
\\	その工事は住民からの強い反対にもかかわらず、続けられた。   
\\	その講義はとても長かったが、それにもかかわらず私はその講義を楽しんだ。   
\\	その車は五千ドルもしたが、それにもかかわらず、まともに走らなかった。   
\\	それにもかかわらず彼女は私たちの言い値を受け入れた。   
\\	それにもかかわらず彼女は魅力的だ。   
\\	ひどい雨にもかかわらず彼らは出発した。   
\\	ひどい渋滞にもかかわらず、私は定刻に間に合った。   
\\	やめろといったにもかかわらず彼は我を通した。   
\\	トラックは検査に合格しなかったが、にもかかわらず運転手たちは引き取っていった。   
\\	マラソンは晴雨にかかわらず開かれます。  
\\	メアリーは病気であるにもかかわらず、働き続けた。   
\\	悪天候にもかかわらず、彼らは車で行く決心をした。   
\\	悪天候にもかかわらずお年寄りたちは御機嫌だった。   
\\	一行は悪天候にもかかわらず出発した。   
\\	雨にもかかわらず、私は外出した。   
\\	彼女は出費に拘らず自分の計画を実行するだろう。  
\\	疲れきっていたにもかかわらず眠れなかった。   
\\	彼女のあらゆる努力にもかかわらずその企ては挫折した。   
\\	彼は年にもかかわらず非常に生活的だ。   
\\	あだ名 (綽名・渾名)、ニックネーム	あだな、にっくねーむ	
\\	他人を親しんで、本名以外につける名。	
\\	身長が高いから、私のあだ名は「ラクダ」です。   
\\	あだ名で呼ばないで下さい。   
\\	も〜ば	も〜ば	
\\	心から始まる恋もあればセックスから始まる愛もある  
\\	嫌いな人もいれば、好きな人もいる。  
\\	ことはない	ことはない	
\\	〜しなくても いい	
\\	ただの風邪ですから、心配することはありません  
\\	この世に思いわずらうことはない。  
\\	時間は十分あるから急ぐことはない。  
\\	ちょっと待たされただけで、なにもそんなに怒ることはないでしょう。  
\\	君があやまることはないよ。  
\\	で・に・前	で・に・ まえ 
\\	駅前でお会いします   
\\	カフェー内で   
\\	カフェで   
\\	駅で  
\\	駅長室で  
\\	金曜日の夜にお会い、、   
\\	時間に間に合います   
\\	バーで   
\\	いつもの時間と場所で 
\\	いつもの時間にいつもの場所で  
\\	七時に   
\\	七時までに到着できます。(行けます。)   
\\	金曜日の午後に待ち合わせしましょうか?   
\\	金曜日の3時に待ち合わせしましょうか?   
\\	一人で 映画を 見に 行く。   
\\	〜ものがある	〜ものがある	
\\	彼女の小説には人を勇気づけるものがある。  
\\	家族に会うと私の心に複雑なものがある。  
\\	彼の学問的な業績には感銘を与えるものがある。   
\\	彼の功績は特筆大書すべきものがある。   
\\	彼の性格には単なる正直以上のものがある。   
\\	彼女には何か磁力のようなものがある。   
\\	物には限界というものがある。   
\\	僕の人生には何か欠けているものがあるような気がするんだ。   
\\	問題の中にはゴリラに不公平なものがある。   
\\	彼の話し方にはどこか強引なものがあった。  
\\	彼は、性格は別として、絵の才能には見るべきものがある。  
\\	(赤毛)
\\	彼女の演奏には、人の心を動かすものがある。  
\\	(赤毛)
\\	難病した人は今まで耐えられて、霊験なものがあら。  
\\	この二つには何か共通したものがある。  
\\	なあ、な〜、な	なあ、な〜、な	
\\	自分のしたいことや欲しいものなど	
\\	明日は釣れるといいなあ!  
\\	冷たいビールを飲みたいなあ。  
\\	腹減ったなあ。   
\\	きれいだなぁ〜。   
\\	冷たいビールをのみたいなぁ〜。  
\\	一方では	いっぽうでは	
\\	またその一方ではその経験から学んだものも大きかった。   
\\	また一方ではその経験から学んだものも大きかった。   
\\	一方では君は間違っている。かといって、私はそのことであなたを攻めることはできない。   
\\	一方では彼は私の報告書を賞賛したが、他方ではそれを批判した。  
\\	一方では彼は親切だが、他方では怠け者だ。   
\\	彼女は、自分や自分の教え子達はとても豊かであるのに、一方では非常に貧しい人々がいるという事実を受け入れることができなかった。   
\\	上は	うえは	
\\	動物を飼う上は責任をもって世話をするべきです。  
\\	彼のことだから、「やる」と言った上は、絶対にやるだろう。   
\\	事態がこうなったうえは、もう彼一人には任せてはおけない。  
\\	(赤毛)
\\	にとって・として・の上で・からいうと・から見ると	にとって・として・の上で・からいうと・から見ると	
\\	英語教師として、一番難しいのは生徒と関係作りだと思います。   
\\	あかるい	あかるい	
\\	太郎君、もっと明るいところで読みなさい。  
\\	頂きます	いただきます	
\\	頂きます。  
\\	お言葉に甘えてもう一杯頂きます。  
\\	コーヒーをやめて紅茶を頂きます。  
\\	先生に辞書をいただきました。  
\\	(赤毛)
\\	社員「すみません、あした用事があるので、仕事を休ませていただきたいんですが。」社長「あしたか。ちょっと急でこまるんだけれど。」  
\\	(赤毛)
\\	火曜日、木曜日、土曜日にしていただけませんか。  
\\	(赤毛)
\\	といっても	といっても	
\\	大変だといっても3ヶ月間だけですから。  
\\	日本語ができると言ったって、平仮名や片仮名しか読めないから、難しい漢字が出ると困るわ。  
\\	日本の春を代表する花は何と言っても桜でしょう。  
\\	「愛と哀しみの果て」は俗受けしたけど何といっても映像のきれいな映画だった。   
\\	たとえ大会があるといっても、彼らにそんなにたくさんのお金を請求してはいけないな。   
\\	メアリーはまた失敗した。なんといっても彼女はまだ若い。   
\\	一口に英会話教材といっても、千差万別だ。   
\\	何といっても私がまちがっておりました。  
\\	彼女が上手な歌手だといってもお酒を飲み過ぎた後でカラオケで歌うだけです。   
\\	庭があるといっても猫の額ほどです。  
\\	熱帯の国といっても、朝晩は涼しくて過ごしやすくなる。  
\\	中国語ができるといっても、基礎を勉強しただけですから、まだまだです。  
\\	わたしは甘いものがあまり好きではない。といっても、クッキーやケーキをまったく食べないということではない。  
\\	(赤毛)
\\	料理の勉強を始めたといっても、また三か月にすぎない。  
\\	(赤毛)
\\	旅行をしたといっても、近い観光地へいった。  
\\	代わりに	かわりに	
\\	昼間遊ぶ代わりに夜勉強するつもりだ。  
\\	お金は健康の代わりにはならない。  
\\	この 部屋 は 狭い かわりに、 家賃 が 安い.   
\\	この辺は静かな代わりに店も遠くて不便だ    
\\	校長先が病気になった私たちの先生の代わりをした。  
\\	まみれ (塗れ)	まにれ	
\\	(名詞+に+まみれる); (名詞+まみれ)	
\\	ソースまみれ、ソースにまみれる	
\\	その子供たちは全身ほこりまみれだった   
\\	彼は全身泥まみれだった。   
\\	そのナイフは血にまみれていた。  
\\	彼の額は汗まみれだった。  
\\	インドにお互いを色でまみれる祭がある。  
\\	あの子は泥と埃塗れだった。  
\\	〜時	〜とき	
\\	私は日本に行った時日本語を勉強した  
\\	いつでも都合の いい時にお金を返してください。  
\\	何というときに出てきたんだ!  
\\	せっかく	せっかく	
\\	せっかく会いに行ったのに、友達はいませんでした。  
\\	せっかくいい大学に入ったのだからよく勉強するつもりです。  
\\	人がせっかく作ってくれた料理を食べないのは失礼だ。   
\\	せっかく彼女に手紙を書いたのに、送るのを忘れちゃった。  
\\	せっかく母親が買ってきたシャツをどうしても着ようしなかった。  
\\	ばかりか	ばかりか	
\\	柔道は若者の健康によいばかりか、人格形成にもおおいに役立つ。  
\\	気温ばかりか湿度も高い。  
\\	この報告書は字が汚いばかりか間違いがたくさんある  
\\	そればかりか、会社の直営店の商品は、たいていよその店より高価だったのです。   
\\	君ばかりか僕まで巻き添えを食らった。   
\\	柔道は若者の健康によいばかりか、人格形成にもおおいに役立つ。   
\\	彼女は美しいばかりか、心もやさしく、しかも聡明である。   
\\	彼女は美人であるばかりか気立てもいい。   
\\	目が覚めているときばかりか眠っているときにもそれが必要だ。   
\\	労働はただ単に必要なものであるばかりか、楽しみでもある。   
\\	彼はサッカー,ばかりでなく水泳もダンスも上手なんですよ.  
\\	アンさんは頭がいい、ばかりでなく親切で心の優しいです。  
\\	財布をなくしたばかりか怪我までしてしまった。  
\\	〜をめぐって	〜をめぐって	
\\	[〜] を中心点にして、どんな議論や対立関係が起こっているかを言う時	
\\	この規則の改正をめぐって、まだ討論が続いている。   
\\	土地の利用をめぐって、二つの対立した意見が見られる。   
\\	マンション建設をめぐる争いがようやく解決に向かった。  
\\	町の再開発をめぐり、住民が争っている。  
\\	法律改正をめぐって、議論が続けられた。  
\\	(赤毛)
\\	この小説は、1 人の女性をめぐって、5人の男性か戦う話です。  
\\	教室の使い方をめぐってまだ議論が続いている。  
\\	に先立って; に先立つ	にさきだって; にさきだつ	
\\	のまえに	
\\	フィルタリング導入に先立って、親子の信頼関係を損ねないために考慮すべきこと。  
\\	お金に先立つのは家族だ。   
\\	映画の一般公開に先立って主演女優のサイン会が開かれた。  
\\	会に先立って彼らは夕食をとった。   
\\	海戦に先立って出港した後に、乗組員が船に慣れる時間がないことが、しばしばあるものだ。   
\\	このミーティングの目的は、1月の商品企画会議に先立ち、新製品のプランを考えることです。   
\\	スミスさんは奥さんに先立たれた。   
\\	会に先立って彼らは夕食をとった。  
\\	彼は息子に先立たれた。   
\\	彼女は1人息子に先立たれた。  
\\	話し合いの糧にするために、ミーティングに先立ち、皆様からいただいた提案をとりまとめてみました。   
\\	なくちゃ	なくちゃ	
\\	シャツを買わなくちゃ。   
\\	検査して貰うため病院に行かなくちゃ。   
\\	明日試合があるから、早く寝なくちゃ   
\\	かのごとく	かのごとく	
\\	あたかも永遠を傷つけることがないかのごとく暇をつぶしているが・・・。  
\\	誰も見ていないかのごとく自由に踊りましょう  
\\	その森の中は、何万年も前に戻ったかのごとく神秘的だった  
\\	(誠)
\\	であろうと	であろうと	
\\	どんな人であろうと、この映画は楽しめます。  
\\	いかに粗末であろうとも、我が家は我が家である。   
\\	この本を書いたのが誰であろうとも、その人はとても利口だ。   
\\	みたい	みたい	
\\	彼は風邪をひいたみたいです。  
\\	まるで夢みたい  
\\	お前みたいな奴は見たことがない  
\\	京都みたいな所に行くのが好きです  
\\	外国に行くみたいなことを言っていました  
\\	明日雨が降るみたいなあ。  
\\	きり	きり	
\\	寝たきり老人  
\\	手紙を貰ったきり、連絡がない。  
\\	彼はアメリカに行ったきりだ。  
\\	息子は友達に家に行ったきり帰ってこない。  
\\	店はたたまれ、それっきりだった。  
\\	彼は38歳なのに、両親に頼りきりです。  
\\	彼は3年前に家を出たきり、全然消息がない。  
\\	彼は自分の子供達の教育の問題を妻に任せきりだった。  
\\	彼は病気でずっと床についたきりだ。  
\\	「物を人から借りたきり返さないのは、どろぼうと同じですよ」と母に言われてしまった。  
\\	大学の登山パーティーは、3日前から北アルプスの立山に登ったきり、下りてこない。救助隊がパーティーの行方を探している。  
\\	あれきり私たちは彼女に会っていない。  
\\	〜を	〜を	
\\	目的	
\\	それをください。
\\	新しいカメラを買った。  
\\	私はチキンをもらいます。  
\\	彼は「イマジン」を歌いました。
\\	仕事を探している。
\\	薬を飲む。
\\	少し日本語を話します。「すこしにほんごをはなします。」  
\\	コーラを飲みました。  
\\	を契機に	をけいきに	
\\	大統領が中国に行ったのを契機に万里の長城を訪問しておりました。  
\\	オンデマンドを契機にダンボールから印刷までサービス.  
\\	新しいCDを契機に、彼は人気者になりました。  
\\	(赤毛)
\\	それを契機に彼女は新しい商売を始めた。   
\\	としたら、とすれば、とすると	としたら、とすれば、とすると	
\\	温泉に行くとしたらどこがいいでしょうね。  
\\	この質問を答えるとしたら何が貰いますか。  
\\	生まれ変われるとしたら、男がいい?女がいい?   
\\	君が僕の立場としたら、この場合どうする?   
\\	このまま不況から抜け出せないとすれば、日本の将来に希望はない。  
\\	病気の療法するとしたら、いい所はどこですか。  
\\	積もりだった	つもりだった	
\\	僕は彼女と結婚するつもりだった。/ 
\\	僕は彼女と結婚するつもりだった。  
\\	(赤毛)
\\	今日は誰も来ないし、私も出かけないつもりだった。  
\\	(赤毛)
\\	正しく見直したつもりだったが、間違ってしまった。  
\\	(赤毛)
\\	自分では大人になったつもりだったが、子どもっぽい行動してしまった。  
\\	(赤毛)
\\	優しく言ったつもりだったが、「それをしなさい!」と聞こえてしまった。  
\\	(赤毛)
\\	自分では元気になったつもりだったが、またどこか痛くなってしまった。  
\\	(赤毛)
\\	医者になる積もりだった。   
\\	これはほんの冗談の積もりだった。   
\\	この夏ハワイへ行く積もりだった。   
\\	の余り	のあまり	
\\	心配のあまり 
\\	心配のあまり  
\\	彼女は一番行きたかった大学に合格し、嬉しさのあまり跳び上がった。  
\\	父親の死を知らされて、悲しみのあまり、寝込んでしまいました。  
\\	ショックのあまり、涙も出ませんでしたよ。  
\\	韓国には空腹のあまり家主の犬を調理しようとした男を逮捕された。  
\\	恥辱のあまり数回自殺未遂している。  
\\	そのニュースを聞いた時、ニコルは悲しみのあまり我を忘れた。   
\\	その映画を見て彼女は感動のあまり涙を流した。   
\\	その光景を見て私は恐怖のあまり身の毛がよだつ思いだった。   
\\	その知らせを聞いたとき、誰もが喜びのあまり叫び声をあげた。   
\\	スフィンクスは激怒のあまりうなった。   
\\	感動のあまり、彼は一言も発することができなかった。   
\\	嬉しさのあまりとび上がった。  
\\	驚きのあまり口もきけなかった。   
\\	驚きのあまり話す力がなくなってしまいました。   
\\	私は空腹のあまり口もきけなかった。   
\\	昇任したさのあまり彼は他人の感情を考えなかった。   
\\	悲しみのあまり彼女は気が狂った。  
\\	ように言う	ようにいう	
\\	医者に控えるように言われているのでお酒が飲めなくて、ウズウズしている。  
\\	医者は彼に酒を控えるように忠告した。  
\\	医者なら誰でも君に禁煙するように言うだろう。  
\\	私は図書館の職員に金曜日締め切りのレポートのためにその本が必要だと言った。そうしたら彼女は借りている人に返却を求めるように言うと言った。   
\\	先生は私たちに毎日教室を掃除するように言う。  
\\	〜こそ	〜こそ	
\\	しばらく休みましたが、ことしこそは英語力をさらに向上させましょう。  
\\	今度こそやってやる。  
\\	これこそわれわれが探し求めていたものだ。  
\\	わたしこそお詫びしなければなりません。  
\\	止めるのを聞かばこそあいつは飛び出して行った。  
\\	ようこそおいで下さいました。  
\\	マイケルソンーモーリーの光速度に関する実験があったからこそアインスタインは特殊相対性理論を確立することができた。  
\\	これこそ求めていたものですよ。  
\\	明日こそ、勉強するぞ!
\\	まさかの時の友こそ真の友。   
\\	君自分のためを思えばこそこう言っているのです。  
\\	これこそ今私に必要な本です。  
\\	これ(此・是・之・惟)、それ(其), あれ(彼)	これ, それ, あれ	
\\	それはりんごケーキです。
\\	あれは何のケーキでしょうか。
\\	これはイチゴケーキです。
\\	にくい	にくい	
\\	出来るごとが難しいです。	
\\	彼は分かりにくいでしょうかなぁ?   
\\	食べにくい食べ物がきらいだ。  
\\	このめずらしい切手は手にはいりにくいんだよ。  
\\	タイガーティの説明はとてもわかりにくいですね。  
\\	英語での本を読みにくいです。  
\\	多くの外人にとって、納豆を食べにくいんだ。   
\\	凍った食べ物が食べにくいだ。  
\\	〜はさておき	〜はさておき	
\\	それはさておき	
\\	それはさておき本題に入ることにします。  
\\	太郎は花子の行方を追って山に登りはじめた。今はそれはさておき、花子はそのとき東京行きの高速バスの上だった  
\\	それはさておき、他に必要なものはないか。   
\\	金の問題はさておき、そういう旅行は疲れるよ。  
\\	(誠)
\\	仕事の話はさておき、今日は思い切り楽しみましょう。  
\\	(誠)
\\	文句を言いがちはさておき、彼女は優しい人です。  
\\	何はさておき月曜日の朝一番にいらっしゃい。  
\\	に相当する	にそうとうする	
\\	この日本語に相当する英語はない  
\\	この絵には莫大な額のお金に相当する価値がある。   
\\	その語に相当するようなフランス語を知っていますか。  
\\	地図上の青い線は川に相当する。   
\\	地図上の幅の広い線は道路に相当する。   
\\	鳥の羽は人間の腕と足に相当する。   
\\	日本の「将棋」は、チェスに相当する。   
\\	日本語の「義理」に相当する英語は見つからない。   
\\	木にとっては枝は人間の手足に相当する。  
\\	その当時1ドルは大体110円に相当していました。  
\\	彼の月給は、私の半年分の給料に相当します。  
\\	彼らが一日に食べる食べ物の量は、彼らの体重に相当している。  
\\	兼ねる	かねる	
\\	その点は賛成しかねる。/ 
\\	言いかねる   
\\	たまりかねる  
\\	こんな高価な贈り物、頂きかねます。  
\\	彼女に本当の事を言いかねて彼はだまった  
\\	そのことについては、即答しかねます。 
\\	こういう事情ですから、残念ながら、せっかくのご招待をお受けいたしかねるのです。   
\\	その点は賛成しかねる。   
\\	一年間で浪費するはした金がいくらになるか、だれにも決めかねる。   
\\	彼女の態度は少々腹にすえかねる。   
\\	その意見に賛成とは言いかねます。  
\\	(赤毛)
\\	労働条件の変更について会社から説明を受けたが、私はどうも納得しかねる。  
\\	(赤毛)
\\	面接試験ではどんな質問が出るのかと聞かれたが、そのようなことを聞かれてもちょっと答えかねる。  
\\	(赤毛)
\\	そんな方法には賛成し兼ねる。  
\\	髪をもっと短く切ろうかどうか決め兼ねている。  
\\	今日は晴天だから、雷雨がありかねる。  
\\	〜ものの	〜ものの	
\\	パソコンを買ったものの使い方がぜんぜんわからない。  
\\	熱は下がったものの、まだ咳が止まらない。  
\\	軽いけがで済んだからいいようなものの、次回からは気をつけて下さい。  
\\	あのホワイト・ライオンの子どもはおとなしいとはいうものの、なんといっても獣ですから、気を許さないで下さい。  
\\	いろいろな説はあるものの、恐竜がなぜ突然絶滅してしまったのかはまだ謎のままである。  
\\	パーティーに行くことにしたものの気がはずまない。  
\\	パーティーに行くことにしたものの気がはずまない。   
\\	流言飛語に惑わされるべからず、とはいうものの、言うは易く行うは難し、と思わない?   
\\	認知は中企業以上では高まってきているものの、小企業ではまだ低いとみている。   
\\	ようではないか	ようではないか	
\\	雨が降るようではないか  
\\	空にのびる枝が我々を導いてくれてるようではないか。  
\\	動いているのに、進んでいないようではないか。  
\\	らしい	らしい	
\\	動かないらしい  
\\	そんな事をするなんて君らしくない。  
\\	私は彼と彼のガールフレンドらしい人を見かけた。  
\\	みんなさんは夕方来るらしい。  
\\	〜だに	〜だに	
\\	想像だにしなかった  
\\	聞くだに恐ろしい  
\\	今だに  
\\	その時計は私の祖父が買ったのだが、いまだに調子が良い。  
\\	だに 
\\	その質問はいまだに頭から離れない。   
\\	その州では未だに古い慣習が根強い、とその文化人類学者は言っている。   
\\	その傷はいまだに直らない。  
\\	その母親は子供の死をいまだに嘆き悲しんでいた。   
\\	なぜ彼が自殺をしたかはいまだに謎である。   
\\	わが国の税制にはいまだに一貫した哲学がない。   
\\	考えるだに恐ろしいことだ。  
\\	彼らはこのアイデアを一顧だにしない。   
\\	そばから, 傍から	そばから	
\\	とすぐに	
\\	子供が掃除するそばから散らかすから、もうあきらめたくなった。 
\\	私は聞いたそばから人の名前を忘れてしまう。   
\\	を込めて	をこめて	
\\	彼女は心をこめて歌ったので、聴衆は深い感動をうけた。   
\\	心を込めて祈りなさい。   
\\	怒りを込めて告発人の方に向き直った。   
\\	彼は満身の力を込めてそれを持ち上げた。   
\\	彼女のテクニックはすばらしいが、もっと感情を込めて演奏する必要がある。   
\\	彼女は、愛情と優しさを込めてお祈りをした。  
\\	愛をこめて、セーターを編みました。  
\\	(赤毛)
\\	怒りを込めて、核実現反対の署名した。  
\\	病気回復の祈りを込めて、皆で鶴を折った。  
\\	彼は彼女に愛情をこめてキスした。  
\\	〜か	〜か	
\\	これはあなたの本ですか。  
\\	家に帰りましょうか。  
\\	ひとつ質問をしてもいいですか。  
\\	今日は鶏肉を買いましたか。  
\\	ほとんど (殆・殆ど)	ほとんど	
\\	(私は)ほとんど読みません。  
\\	(私は)ほとんど食べました。  
\\	日本は石油のほとんどを輸入しています。  
\\	(アヴェタル)
\\	ほとんどのひとはでんきをつかいます  
\\	のに	のに	
\\	せっかく階段を走って降りたのに、もう一歩のところで電車に乗れなかった。  
\\	風が強いのに釣に行きました  
\\	めったに遅れなかったのに、遅刻常習犯の烙印を押された。  
\\	彼はとても年をとっているのに元気だ。   
\\	彼はお腹が痛いのに学校に行った。   
\\	に対して	にたいして	
\\	アメリカでは離婚に対しての考え方が大分変わってきた。  
\\	先生に対して  
\\	女性の地位はまだ低いのに対して欧米では比較的に高い。  
\\	話すのに対して  
\\	これまで日本は外交に対して、閉鎖的な政策を取ってきた。   
\\	日本の大学に入学するのが難しいのに対して、アメリカの大学は卒業するのが難しい。   
\\	市民は増税に対して強く反対してる。   
\\	私は政治に対して強い関心がある。   
\\	手数料は一万円に対して、五百円です。  
\\	その経済学者の説に(対する/対して)反論はなかった。  
\\	〜であれ〜であれ	〜であれ〜であれ	
\\	男性であれ、女性であれ、人間としての権利は同じはずだ。  
\\	有名であれ、無名であれ、人生は短い。   
\\	注射であれ飲み薬であれ、良く効く薬はかえって副作用が心配だ  
\\	父は菓子であれ、果物であれ、甘い者は一切口にしない。  
\\	正社員であれ、パートであれ、仕事に対する責任は変わりません。  
\\	あの学者は哲学であれ、言語学であれ、人類学であれ、同じように精通している。  
\\	ということですね (と云う事ですね)	ということですね	
\\	少しずつそうなったということですね   
\\	に違いない	にちがいない	
\\	確かに と おもう	
\\	あのシチューが当たったに違いない。   
\\	あの煙を見なさい。あそこの建物が燃えているに違いない。   
\\	あの狐がめん鳥を殺したに違いない。   
\\	あんなに短時間で1マイル走ったなんて、君はすぐれた運動選手に違いない。   
\\	あんなふうに話すとは彼はばかに違いない。   
\\	あんな風に振る舞うなんて彼は気が狂ってるに違いない。   
\\	お百姓さんはそれを聞いて喜ぶに違いない。   
\\	きっと無罪になるだけの決定的切り札を隠し持っているに違いない。   
\\	ここに彼の書き置きがある。かれはここに来たに違いない。  
\\	この急流で泳ぐのは危険に違いない。   
\\	この作文の英語はうますぎる。それを彼女が書いていないのに違いない。  
\\	この事の根本には何かあるに違いない。   
\\	この車は乱暴使い方をされてきたに違いない。  
\\	これは本物のダイヤに違いない。   
\\	こんなことを信じるなんて彼はよほどの間抜けに違いない。   
\\	こんなにかわいい人形をあなたに送ってくれるとは、彼女は心が優しいに違いない。   
\\	こんなに長い距離を歩いたなんて彼は健脚家に違いない。   
\\	そこでおそらくこれは久しぶりに戻ってきた古代の神に違いないと考えた。   
\\	そのことはお金に関係があったに違いない。   
\\	かぎがどこかに落ちたに違いない。  
\\	これは、若いときの彼女の写真に違いない。  
\\	(赤毛)
\\	まじめなあの人のことだから、時間どおりにくるに違いない。  
\\	(赤毛)
\\	彼女は30歳ぐらいに違いない。  
\\	この事件の裏には何かがあるに違いない。   
\\	ジェーンはとても顔色が悪い。病気に違いない。   
\\	彼は20分前に家を出たので、もう会社に着いているに違いない。   
\\	彼の計画は失敗に違いない。   
\\	君は冗談を言っているに違いない。   
\\	て以来	ていらい	
\\	日本に来て以来、日本に対する考え方が少しずつ変わってきた。  
\\	第二次世界大戦の敗北以来日本国民は本来の意味での「愛国心」を失ってしまった。  
\\	彼女を知って以来、僕の人生はバラ色だ。  
\\	卒業して以来、クラスメートには会っていない。  
\\	数年前に大きな病気をして以来、健康に注意するようになった。  
\\	彼と一緒行って以来あたしの思いはいつも彼に向けています  
\\	彼は学校を出て以来働いている。   
\\	彼女は当地に来て以来6回引越しをした。   
\\	ともなると	ともなると	
\\	プロともなると体つきが変わってくるというが、目の前の彼女のきゃしゃで凛とした立ち姿からは、とても激しいダンスを一人で踊り通す人のようには見えない。  
\\	大学へ進学するともなると、学問、バイト、サークルなど、人それぞれのライフスタイルができてくる。  
\\	桜の季節ともなると、この公園は花見客でにぎわう  
\\	〜にしたら	〜にしたら	
\\	弟にしたら、私のようなうるさい姉はいやでたまらない。  
\\	わたしにしたらアメリカの冷たいビールよりイギリス生ぬるいビールがおいしい。  
\\	経営者にしたらボーナスは安いほうがいいが、社員の身にしたら1000円でも高いほうがいい。   
\\	この階段は、私にはたいしたことはありませんが、足の弱い人にしたら、とても大変だろうと思いますよ。  
\\	子供にしたら、博物館めぐりはつまらないだろう。  
\\	(赤毛)
\\	のみ	のみ	
\\	家族のみが来た   
\\	めぐの家のみに行ってもいいですよ!  
\\	この規則は、大人のみに適用されます。  
\\	この種の植物は熱帯地方にのみ育ちます。  
\\	この切符は、購入後2日間のみ有効である。  
\\	ただ結果を待つのみだ。  
\\	知識は学習によってのみ習得することができる。  
\\	私が大好きな人は、貴方のみだ。  
\\	要点のみを話してください。   
\\	月は反射によってのみ輝く。   
\\	老兵は死なず。ただ消え去るのみ。   
\\	彼がだれなのか神のみぞ知る。   
\\	勇者のみが尊敬に値する。   
\\	まさか	まさか	
\\	まさか来るとは思わなかった。  
\\	まさか!  
\\	びっくりした!まさかこんなところで会うとは思わなかったよ。  
\\	彼の言ったことはまさか本当であるとは思えない。  
\\	ロニーが日本語能力試験が合格したと聞いて先生つい「まさか」と言い出しました。  
\\	がり	がり	
\\	彼は暑がりで、冬の間も半袖を着ている。  
\\	あいつは恐がりでトカゲを見ただけで怖気をふるう。  
\\	そんな強がりを言ってみても何になる。  
\\	ように〜て	ように〜て	
\\	(私に)電話(を)するように言って下さい。  
\\	明日妹さんを早く来るようにさせて下さい、聞きたいことがあるんですよ。  
\\	一年以内にあなた方が全員、流暢な英語を話しているようにしてあげます。  
\\	5時にここに来るようにしていただけませんか。  
\\	あなたが出かけるときドアに鍵をかけるのを忘れないように気をつけてください。  
\\	私は部長にこの点について榎本さんに確認するように言われました。  
\\	日本語の場合、先ずは私に連絡するように伝えて頂ければ幸いです。  
\\	ドアの所にいる人が誰であれ、待つように言ってください。   
\\	急ぐときほど、安全の確認を怠らないようにして下さい。  
\\	(赤毛)
\\	転ばないように足元に気をつけてください。  
\\	(赤毛)
\\	になる	になる	
\\	ざっと計算しても大変なことになる 
\\	未来には先生になりたいです。  
\\	この薔薇の花はもっと美しくなりました。  
\\	注意しなかったので,病気になりました。  
\\	て形
\\	いる (て居る)	〜ている	
\\	私 は 食べている   
\\	あなた は 見ている   
\\	為	ため	
\\	彼が来る事を確かめる為に電話をした。  
\\	彼は、去年老齢の為、なくなった。  
\\	このキーは何の為の物ですか?  
\\	外国人の為のいい辞書がない  
\\	日本人は会社の為によくはたらきます  
\\	何の為に?  
\\	雪がたくさん降ったため試合が中止された。  
\\	に従って; に従い	にしたがって; にしたがい	
\\	とともに; につれて; に伴って	
\\	予算が不足している。したがってこの計画は実行できない。  
\\	南に行くにしたがって気温が高くなる。  
\\	私の指示にしたがって下さい。  
\\	経験を積むにしたがって給料があがる   
\\	契約に従い、雇用期間を三年とする。   
\\	収入が増えるに従って、支出も増える。  
\\	医学が進歩するにしたがって平均寿命が延びた。  
\\	高くにしたがって、見晴らしがよくなった。  
\\	彼女は私の忠告に従って退学しなかった。   
\\	わが社は労働者の安全に関する政令に従って操業している。   
\\	法に従って   
\\	彼の希望に従って   
\\	見た目はともかく	みためはともかく	
\\	[見た目] 
\\	あの女優は、顔はともかくとして演技がすばらしい。  
\\	母の料理は見た目はともかく味は美味しい。  
\\	精一杯	せいいっぱい	
\\	できるかどうかわからないが、精一杯がんばろう  
\\	あれが彼としては精一杯の微笑だった。   
\\	これが精一杯です。  
\\	私は彼女に追いつこうと精一杯走った。 
\\	〜にして	〜にして	
\\	かれはノーベル賞に値する研究を成し遂げた。これは、偉大な科学者にして初めてできることであろう。   
\\	留学生にして日本に行った。  
\\	90歳 にして彼はいまだに驚くほど元気である。  
\\	これは長年訓練を積んだ彼のような人にして初めてできる技だ。  
\\	ようにする	ようにする	
\\	知らせるようにします  
\\	覚えるようにします  
\\	あなたは英語で自分自身が理解されるようにすることができますか。   
\\	ドアが閉まるようにすることができますか。   
\\	悪い状況を最大限に生かすようにするべきだ。   
\\	一つの事に専心して、それがうまくできるようにするべきだ。   
\\	運動をしてもっと健康になるようにする。   
\\	私は私の言うことを人に理解されるようにすることが出来なかった。   
\\	私は時代に遅れないようにするため本や雑誌を読む。   
\\	私は彼の曲と彼の人生にあまり興味がないようにする。   
\\	私は彼女に2度とそんなことをさせないようにするつもりだ。   
\\	時勢に遅れないようにするために、新聞を読みなさい。   
\\	自分たちの悪事がばれないようにするなんて、なんて悪どい奴等!   
\\	自分の言語を他の人たちの言語と同じようにするために必要な変更を加える。   
\\	交渉	こうしょう	
\\	ある事を実現するために、当事者と話し合うこと。かけあうこと。	
\\	-9の計算機に一括購入の割り引きはありますか。もし5%値引きしていただけるなら、1,000台買いたいと思います。お知らせください。  
\\	見積り交渉の表現  
\\	ご要望のありましたお見積りをお送り致します。   
\\	貴社の経営コンサルティングサービスについてのお見積りを頂戴したいと存じます。   
\\	ご提案を再検討いただければ幸いです。   
\\	少し高すぎて契約書にサインすることができません。   
\\	4月27日までにご注文いただければ4%割引することができます。  
\\	当社の事務所への輸送費込みなら貴社の見積りで発注したく存じます。   
\\	当校の教師のために無料で辞書を2冊付けてくださるのであれば、貴社の見積りは悪いものではありません。   
\\	貴社の価格はいつもより少し高いように思えますが、分割払いが可能であればお受けさせていただきたく存じます。  
\\	貴社の値上げについてはご通知頂いていませんでしたので、値上げ前の価格でお支払いしたく存じます。   
\\	ご送付いただいた商品のおよそ10%が破損しておりましたということを書面でお知らせいたします。したがいまして、その金額を請求書の総額からお引きください。   
\\	前回評判の高い貴社に発注いたしましたとき、今年は値上げはないとお約束されました。 ←価格上昇  
\\	確か、この見積りよりも低い金額で合意したように思うのですが。  
\\	貴社にも同意いただけるものと思います。そうでないと納入をお受けするわけには参りません。   
\\	貴社の納入は2日遅れましたので、5%の値引きか搬送費を無料にしてください。  
\\	貴社が当社地区に納入しなくなったとお聞きし、たいへん残念に思います。そのことに対して当社に金銭的弁償を行うか、別の供給業者をご紹介ください。   
\\	貴業界では季節割引をするのが一般的ですが、貴社ではありますか。  
\\	商品をクリスマス用ペーパーで包装して納入してくださるのであれば、価格があがることには反対しません。   
\\	紙箱ではなく透明プラスチック箱に入れて納品してくだされば、よく売れると思います。貴社が包装を変えてくださるようであれば、即座に3000個の花瓶の発注したいと思います。  
\\	貴社が1箱に5個ずつ包装しなおしてくださるのであれば、今後納入後8日以内にお支払いをさせていただきます。  
\\	今後箱には梱包材を入れてください。   
\\	貴社が輸送期間を3週間から2週間に短縮してくださるのであれば、同意した時期よりも1週間早くお支払いができます。   
\\	貴社が2人のショーウィンドウ装飾の専門家を2日間派遣してくださるのであれば、次回に大きな発注をすることができます。   
\\	新規子供用
\\	シャツの助ェな見本を含めていただけるようでしたら、納入時に現金払いすることが可能です。  
\\	来年度当社に対する納入価格を据え置いてくださるなら、貴社からの購入量を8%引き上げる所存です。   
\\	貴社が21営業日以内に請求に対する支払いをしてくださるという条件であれば納期を早めることは可能です  
\\	貴社が綿糸に絹10
\\	を混紡して頂けるのであれば今後3年間毎年、発注量を15%増やしたく存じます。  
\\	問い合わせへの回答表現  
\\	お問い合わせいただき誠にありがとうございます。添付の価格リストをご覧下さい。  
\\	この型を2000ドル分以上ご購入いただきますと、特別に10%割引させていただきます。   
\\	価格表は、暫定的なもので、予告なしに変更する場合があります。  
\\	見積価格は5月12日までの購入に有効です。  
\\	価格表の価格は
\\	が上がれば変化します。  
\\	価格表の価格は当社の供給業者の価格に左右されます。   
\\	貴殿は大切なお客様ですので、8%の特別割引をさせていただきます。  
\\	貴社が4月23日までに発注いただけるようでしたら、3%の割引をさせていただきます。  
\\	もちろん当社はリスト価格でお品をお届けすることはできますが、当日配送をご希望の場合はその価格に4%を足さざるを得ません。  
\\	お品を取りにきていただけるようでしたら、8%の割引になります。  
\\	見積価格は同色の製品のみです。   
\\	見積価格は同サイズの製品のみです。   
\\	この価格には100キロ以内の配達料しか含まれていません。  
\\	当社が次のカタログを送付するまで価格表の価格は有効です。   
\\	前季の物を販売しているので、価格はカタログのものより安くなっています。   
\\	この価格には当社の専門家による4時間の講習料金も含まれております。   
\\	この価格には電話によるホットラインへの無制限のご利用料金も含まれています。  
\\	この語学学校への入学金には年4回開催されるパーティへの招待も含まれています。  
\\	この航空券代金には、大きなコンテストへの参加料も含まれており、その特賞として世界一周旅行の航空券が二名様に当たります。   
\\	当社の価格は石油価格に左右されますので、ここ2〜3ヶ月で上がるかもしれません。   
\\	残念ながら当社の供給業者が納入価格を4%引き上げましたので、当社も同様の措置をとらざるを得ません。  
\\	来週の月曜に当社の価格が上がるので、今週中に倍の発注を是非お願いいたします。   
\\	この価格表では価格が少し高くなっておりますが、それは仕入価格が高くなったからです。   
\\	この価格表では価格が上がっていますが、最高級コーヒー豆が不足しているからです。   
\\	このオファーは10日間だけ有効です。   
\\	このオファーは2週間だけ有効です。   
\\	このオファーは、当社と2年間以上のおつきあいのあるお客様がご利用いただけます。   
\\	このオファーはほかのオファーと組み合わせることはできません。   
\\	このオファーは予告しないで変更することがあります。  
\\	当社は貴社に最高級の新商品ラインをご提供できますことを誇りに思っております。   
\\	貴殿を当社のお得意様リストに含めさせていただきます。お得意様は最初にこの特別オファーをご利用いただけます。   
\\	貴社には当社としてかねがね格別の敬意を払わさせて頂いておりますので、このオファーもその反映一端とお考えいただければ幸いです。   
\\	貴社とは10年間パートナーの関係にありますので、最初にこのオファーを提供したく存じます。   
\\	このオファーは当社と特別な関係にある、ごく限られたお得意様だけに提供しております。   
\\	当社は貴殿に、特製ワイングラスの特別販売をさせていただきたく存じます。   
\\	当社は貴殿に、この美しいワイングラスの特別販売をさせていただきたく存じます。   
\\	当社は貴殿に、特別包装を施したワイングラスの特別販売をさせていただきたく存じます。   
\\	4月15日までにご注文いただけましたら、ワイングラスの特別販売をご利用いただくことができます。   
\\	ワイングラス3ダース以上のご注文をいただけましたら、おそろいのカラフ(エ)を無料で差し上げます。   
\\	当社の大量購入割引に関しておたずねください。   
\\	12日以内に当社のコンピュータをお買い上げいただけましたら、2ヶ月間の無料保守契約を結ばさせていただきます。   
\\	母の日のプレゼントに、この人形の収集家向けの特別版を差し上げてはいかがでしょうか。   
\\	このタイプの人形は限定販売で、定期購読者にしかお分けしておりません。   
\\	3ダース以上の教科書をお買い上げいただければ、無料で教授マニュアルを差し上げます。   
\\	すでに教授マニュアルをお持ちでしたら、先生のご選択で2種類の言語ゲームを差し上げます。   
\\	当社の英語教科書最新版をお買い上げいただきましたら、辞書を割引価格で提供させていただきます。   
\\	これは今年最後のセールで、在庫があるうちにご注文をお願い申し上げます。  
\\	8月21日までにこのバックパックをお買い上げいただきましたら、全費用込み東京タワーへのツアーへご招待いたします。  
\\	2003年11月5日までにビジネスクラスの航空券をご購入いただきましたら、貴殿の口座に2000マイルを入れさせていただきます。  
\\	2000個以上ご購入の際、5%の値引きをさせていただきます。  
\\	企画およびコンサルティング業務に関する貴社の条件と料金について確認したいと存じます。   
\\	見積価格は6月23日まで有効です。  
\\	見積価格は小口客用です。   
\\	見積価格は1000個以上購入の際に適用されます。   
\\	マーケティングキャンペーンが始まりましたので、見積価格がいつもより低くなります。   
\\	価格は通常よりも2%下回っています。   
\\	いつもの価格よりも5%割引できます。   
\\	貴殿は大切なお客様ですので、5月にはご購入の際4%割引させていただきます。   
\\	在庫一掃のため10%引きにします。   
\\	在庫一掃のため価格を少し下げる予定です。   
\\	価格には配送料金も含まれています。   
\\	この価格には配送料金は含まれていません。   
\\	この価格には当日配送料金が含まれています。   
\\	この有利な価格には1年間完全保証がついています。   
\\	価格表の価格は、連絡なしに変更になることもあります。   
\\	見積価格は綿製の衣服のものです。   
\\	毛製品の場合は、綿製品の見積価格に12%足してください。   
\\	商品が少し傷んでおりますので、通常価格の25%引きで販売させていただきます。   
\\	輸送費の上昇は承諾しますが、この機会に、小切手ではなく現金払いの場合、3%の割引をお願いできないでしょうか。  
\\	2500個のかわいいバレンタインカードとリボンを付けてくださるのであれば、その価格で合意させていただきます。  
\\	授業料は大学が負担する。  
\\	費用のX%を負担する 
\\	御社の価格はいつもより少し高いようですので、分割払いが可能であれば、その価格でお受けできます。  
\\	最低限の保障。   
\\	最低限・最小限。   
\\	保障。   
\\	て初めて	てはじめて	
\\	2、3日たって初めて彼が到着した。   
\\	20才になって初めて、彼女は歌手になろうと決心した。   
\\	あとになって初めて、それをやった理由を彼は説明した。   
\\	ずっと後になって初めて私は子供の教育の大切さを理解するようになりました。   
\\	そして14歳になって初めてピアノに触れたのです。   
\\	その時になって初めて私たちが置かれている危険にきづいた。  
\\	その時になって初めて私は危険な事態に気がついたのである。   
\\	つい最近になって初めて彼女は考えを変えた。   
\\	比べものにならない	くらべものにならない	
\\	速さという点では飛行機は船とは比べものにならない。  
\\	食べる量という点では馬は鳥とは比べものにならない。  
\\	このほうが比べものにならないほど優れている。   
\\	私の業績はあなたの業績とは比べものにならない。   
\\	私の料理などお母さんのとは比べものにならない。   
\\	バッハの作品は他のバロック音楽とは比べものにならない。   
\\	芸術的才狽フ点では、 彼女は私とは比べものにならないほど優れている。  
\\	によって	によって	
\\	天気によっては来ます  
\\	結果によっては採用しないことがあります。  
\\	天気によっては中止することがあります。  
\\	景気にようってはメキシコに行きます。  
\\	〜つ〜つ	〜つ〜つ	
\\	ラシューアワーの車内で乗客は押しつ押されつしている  
\\	彼が通りを行きつ戻りつしているのが見えた  
\\	遠くに見えつ隠れつする富士山  
\\	彼と僕とは持ちつ持たれつだ  
\\	社とは長年の付き合いで、持ちつ持たれつの関係にある。  
\\	世の中はみな、もちつもたれつ、困った時はお互い様ですよ。  
\\	久しぶりに旧友に会い、さしつさされつ、夜遅くまで語り合った。  
\\	彼女はそのセーターをためつすがめつして、買おうかどうしようかと迷っている。  
\\	数人の男たちが組んずほぐれつの大げんかを始めた  
\\	〜もかまわず	〜もかまわず	
\\	子供は服がぬれるのもかまわず川の中に入って遊んでいる。  
\\	人の迷惑もかまわず電車の中で携帯電話で話している人がいる。  
\\	姉は家族の心配もかまわず危険な冬、山へいこうとする。  
\\	かたがた(旁)	かたがた	
\\	仕事かたがた京都見物にゆく  
\\	来春結婚する孫が、結婚の報告かたがた婚約者を連れてやってきた。  
\\	無事に卒業できたので、恩師に報告かたがた、手紙を書いた.  
\\	が欲しい	がほしい	
\\	靴がほしいのです。  
\\	「わたしは、妻と子供が欲しいので、家が必要なんだよ。」  
\\	だれか話し相手がほしい。  
\\	もう一杯ビールがほしいな。  
\\	デザートにアイスクリームがほしい。  
\\	君は何がほしい。  
\\	領収書がほしいのですが。  
\\	欲しい。  
\\	僕の宿題をして欲しいですが。  
\\	なにがほしいか言ってください。  
\\	(赤毛)
\\	どんなかばんがほしいですか。  
\\	(赤毛)
\\	わたしがほしいちずはどこにもありませんでした。  
\\	(赤毛)
\\	わたしはもっと小さいテープレコダーがほしいです。  
\\	(赤毛)
\\	あたらしいようふくがほしいです。  
\\	(赤毛)
\\	〜が	〜が	
\\	突然ですがボードゲーム告知をてっきり忘れてました。
\\	悪いですがちょっと静かにしてください。
\\	冷静になろうとしたのだが、ついかっとなった。  
\\	料理はあまりおいしくなかったが、その他の点では、そのパーティーは成功だった。  
\\	立ち聞きする気はなかったが、あなたの話しが聞こえた。  
\\	彼は友だちというほどではないが知り合いだ。  
\\	友だちとテニスの試合をしたが負けた。  
\\	明日晴れるかどうかわからないが、もし晴れたら、私たちはピクニックへ行く。  
\\	友人を作るのはたやすいが、友情を保つのは難しい。  
\\	友人のKは美人だが、人に思いやりがない。  
\\	木は浮くが鉄は沈む。  
\\	明日の11時に予約をしてありますが、時間を変更できますか。  
\\	名前は知らないが、彼女の顔だけは知っている。  
\\	娘に電話をかけようとしたが、通じなかった。  
\\	毎日一生懸命に英語を勉強したが、あまり身につかなかった。  
\\	本当に申し訳ないのですが、私はあなたの傘をなくしたらしいのです。  
\\	できるだけ早く走ったが、最終電車に間に合わなかった。  
\\	魚は食べますが、肉は食べません。  
\\	あまりの〜に	あまりの〜に	
\\	程度が高すぎて、その結果〜。	
\\	あまりのおいしさに、つい食べ過ぎてしまいました。  
\\	あまりの激痛に、思わず叫び声をあげた。  
\\	10年ぶりで兄に会った。兄のあまりの変化に言葉が出なかった。  
\\	〜で	〜で	
\\	車で来ました。 
\\	ひとりで行きますか。
\\	ペンで書く。  
\\	町角で  
\\	食料品店でそれを買いました。  
\\	彼女はその市で働いている。  
\\	海で泳ぐのが好きです。  
\\	通りで遊ぶ。  
\\	この店は八時で閉店です。  
\\	私は三回目で試験に合格しました。  
\\	彼は八十歳で亡くなりました。  
\\	二、三日で帰るつもりです。  
\\	ナイフで肉を切るのは本当に難しい。  
\\	みずほでお勤めですか。  
\\	彼は鉛筆で手紙を書きました。  
\\	〜ともなく,〜ともななしに	〜ともなく,〜ともななしに	
\\	特に〜しようと思うのではなく	
\\	ゴロゴロしながら見るともなく外を見ると、アンパンマンがいました!   
\\	どこからともなく   
\\	待つこともなく彼女が現れた。   
\\	彼女はどこからともなくひょっこりやってきた。   
\\	見るともなくぼんやり外を見ていたら、不意に大きなカラスが飛んできた。  
\\	見るともなくぼんやり外を見ていたら、不意に大きなカラスが飛んで来た   
\\	散歩をしているうちにくるともなく駅まで来てしまった。   
\\	我が家ではいつからともなしに正月にスキーに行く習慣が始まった。  
\\	目撃者の話を聞くともなく後ろで聞いていたが、突然その話の中に事件解決のための重要な情報が含まれていることに気がついた。   
\\	いつも母がその歌をロシア語で歌うのを聞いていたのでなるともなく自然に覚えてしまった。   
\\	どこからともなくいい匂いが漂ってきた   
\\	誰ともなく地震の被災者のためにボランティア活動をしようと言い始めた。   
\\	儚い	はかない	
\\	はかない恋  
\\	はかない人生  
\\	儚いお花の美しさ  
\\	それははかない夢だった   
\\	(ラブレス)
\\	はかない努力   
\\	(ラブレス)
\\	はかない幸せ   
\\	(ラブレス)
\\	はかない喜び   
\\	(ラブレス)
\\	このはかない人生   
\\	(ラブレス)
\\	宝くじに庶民ははかない望みをかける   
\\	(ラブレス)
\\	でなくては	でなくては	
\\	こんなおいしい料理は母親でなくては作れない。  
\\	そうでなくては話にならない。  
\\	どんなにお金があっても健康でなくては人間は幸福に暮らせない。   
\\	栗は少なくとも15分は茹でなくてはいけない。   
\\	君たち学生は勤勉でなくてはならない。   
\\	食物は生きるうえでなくてはならない。   
\\	約束には忠実でなくてはいけない。   
\\	君でなくてはこの本は書けない。  
\\	〜ことに 「は」	〜ことに 「は」	
\\	嬉しいことに4月から、給料が1割りほど上がる。  
\\	残念な ことに、ポケットの穴から全部のお釣りを落としてしまいました。  
\\	ボスは猿を拷問することで快感を得る 
\\	残念なことに、ストで新幹線が止まり、旅行は中止になってしまった。  
\\	驚いたことには、彼女は7か国語が話せるそうだ。  
\\	驚いたことになくした財布が無事に警察に届けられていた。  
\\	残念なことに、楽しみにしていたキャンプが雨で中止になった。  
\\	がち	がち	
\\	私のボスは、短気で怒りがち。   (怒りっぽい 
\\	担当者の好き嫌いによる誤解も場合によっては起こりがち。  
\\	曇りがち   
\\	遠慮がち   
\\	お金持ちは貧乏人を見下しがちである。  
\\	学生たちは同じ間違いをおかしがちだ。  
\\	運動が健康の鍵であることを私たちは、とかく忘れがちだ。  
\\	果物はすぐに腐りがちである。  
\\	私たちはどんな番組が行われているかに関係なく、テレビを見がちである。  
\\	寝不足だと注意が散漫になりがちです。  
\\	彼女は病気がちで青白い顔をしている。   
\\	冬になってから曇りがちの毎日です。   
\\	彼は欠席が多いので勉強が遅れがちです。  
\\	私は子供の頃、病気がちだった。  
\\	彼は最近休みがちだ。   
\\	そのケーキはおいしそうなあ。。。残念ながら、私は太りがちだから。。。  
\\	先日、病気がちのおばを見舞いに行った。  
\\	彼女の小説は退屈になりがちだ。  
\\	彼は怠惰に走りがちだ。  
\\	にわたって	にわたって	
\\	の間(の); に広がって; の間続いて	
\\	1月4日16時から1月6日20時頃にわたって一部のグリーティングカードが届いていないという障害が発生いたしました。  
\\	東京から静岡にわたって地震がありました。  
\\	老後にわたって社会保険制度の安定が必要です。  
\\	あの子は英語、ドイツ語、中国語、フランス語、日本語にわたっていろいろな言葉を知っています。  
\\	ゲイツ氏は、今後10年間にわたって7億5000万ドルを寄付することにしました。   
\\	彼のファンは三世代にわたっています。   
\\	彼の仕事は、その業界のあらゆる領域にわたっている。   
\\	この曲は現在数週間にわたってチャートの1位を記録しています。  
\\	デモは、10年間にわたって行われた。  
\\	(赤毛)
\\	人類は長い年月にわたって努力を重ね、ついに月へ飛行に成功した。  
\\	(赤毛)
\\	参加国の首相による話し合いは、五回にわたっておこなわれた。  
\\	(赤毛)
\\	チンさんは病気のため、2か月にわたって学校をやすんだ。  
\\	広範囲にわたった海の汚染が、問題になっている。  
\\	以上	いじょう	
\\	私は二時間以上も待った。  
\\	1000銘柄以上の株が取引所に上場されている。  
\\	10ドル以上は使いたくありません。  
\\	15種類以上のパイをご用意しています。  
\\	1820年から1973年の間に合衆国は、4600万人以上の移民を受け入れた。  
\\	18歳以上の者はだれでも大人としてみなされる。  
\\	20人以上の人  
\\	あなたのことを友達以上には考えられないの。  
\\	あなたは必要以上のカロリーを摂っている。  
\\	あの音にはもうこれ以上我慢できない。  
\\	あらゆる事を考慮に入れれば、結果は私の期待以上だ。  
\\	この計画以上にすぐれたものはまず作れない。  
\\	これ以上まけられませんか。  
\\	君が私を愛している以上に私は君を愛している。  
\\	私の祖母は私が欲しがる以上のものをくれた。  
\\	人生には目に見える以上のものがある。  
\\	動物も植物も、一般に考えられている以上に日光を必要とする。  
\\	彼は期待以上のことをなしとげた。  
\\	彼の負債は支払い限度以上に達している。  
\\	彼は名誉を富以上のものと見ている。  
\\	二年以上前この写真を撮ったんです。  
\\	関係	かんけい	
\\	あるものが他のものと何らかのかかわりを持つこと。	
\\	彼女は文学と音楽の関係に関する研究を続けている。  
\\	1917年にアメリカはドイツとの外交関係を断った。  
\\	彼らの関係は決裂して元に戻らなかった。  
\\	2人は性的関係を持つようになった。  
\\	彼女は母親との関係がうまくいっていない。  
\\	続ける	つづける	
\\	ダイエット中のくせにあくまでも甘い物を食べつづけます  
\\	お金のために、愛してもいない男との生活をつづけるならば、すっかり失望して、進退きわまる時が将来やってくることだろう。  
\\	この失敗、きっと一生、言われつづけるなあ。  
\\	私は母国を愛しているし、これからも愛しつづけるだろう。  
\\	無視しろぅ! じゃ、書き続けて!  
\\	小さい字を書きつづけて、手がつかれました。  
\\	(赤毛)
\\	長い時間テレビをみつづけると、目がいたくなる。  
\\	(赤毛)
\\	て形
\\	ある (て在る・て有る)	〜てある	
\\	窓が開けてある。  
\\	窓が閉めてある。  
\\	窓を(は)開けてある。  
\\	窓を(は)閉めてある。  
\\	ように	ように	
\\	だから、英語がうまく話せるようになりたいです。  
\\	どのようにしても、わたしはかまわない。  
\\	21世紀がどのようになるのか想像できますか。  
\\	人間は赤ん坊のときは四つんばいで、それから2本の足で歩くようになり、最後は年をとって杖が必要になる。  
\\	あなたが早くよくなるように祈っています。  
\\	どころではない、どころではなく	どころではない、どころではなく	
\\	まじめどころではない、堅物です!  
\\	お金がなくて、買い物どころではない。  
\\	昼寝どころではない。さっさと仕度しな!  
\\	寒いどころではない- 凍え死ぬよ!  
\\	忙しくて遊ぶどころではない。  
\\	引越をしたばかりなので、旅行どころではなくうちの整理に手一杯です。  
\\	私は怒ってなどいない、それどころではない。   
\\	彼はあんな小さな字が読めるのだから近視どころではない。   
\\	彼はよく間違いをするが、馬鹿どころではない。   
\\	彼は紳士どころではない。   
\\	彼は正直どころではない。   
\\	来週試験だから旅行どころではない!  
\\	この国の経済はどんどん悪くなっていて、よくなるどころではない。   
\\	彼の奥さんはやさしいどころではなく、鬼のような人ですよ。   
\\	卒業も危ないのだから、就職活動どころではない。  
\\	天気が悪くては、海水浴どころではない。  
\\	(赤毛)
\\	当時はお金もなく、 誕生日と言っても祝うどころではなかった。  
\\	(赤毛)
\\	春だというのに、 お花見どころではなく、夜遅くまで働いている。  
\\	(赤毛)
\\	仕事が残っていて、酒を飲んでいるどころではないんです。  
\\	(赤毛)
\\	二三月は年度末で多忙につき 遊ぶどころではなさそう。  
\\	(赤毛)
\\	わたしは仕事でしばしば出張するので、あちこち旅行できていいとみんなに言われるが、いつも忙しくて見物するどころではない.   
\\	忙しくて食事どころではなかったんです。   
\\	〜べく	〜べく	
\\	エンジニアになる べく一生懸命勉強した  
\\	このお支払いの問題が解決できるよう、ご指摘の件を改善すべく努力いたします。  
\\	彼らは男を逮捕すべく追いかけた。  
\\	恩師に会うべく、久しぶりに母校を訪れた  
\\	本学は、経済的な理由で大学教育を受けられない海外の優秀な学生にその機会を与えるべく、 奨学金を供与している。  
\\	とにかく	とにかく	
\\	それはとにかくとして、この問題はどうするの。  
\\	とにかく私に言えるのは  
\\	とにかく頼むからちょっと聞いてよ  
\\	とにかくやってみないことには  
\\	とにかく、きみは上京すべきだ。  
\\	とにかく、君の知ったことじゃない。  
\\	とにかく、君はそこに行った方がいい。  
\\	とにかく、言いづらい事を打ち明けてくれて嬉しいよ。  
\\	とにかく、私はあなたの意見には反対だ。  
\\	につれて	につれて	
\\	とともに; に伴って; に従って; と一緒に	
\\	食べるにつれて頭が良くなった  
\\	気温の変化につれて  
\\	夜がふけるにつれて、嵐はますます激しさを増した。  
\\	車の数の増加につれて、地球温暖化が問題になってきた、と言われている  
\\	演説を進むにつれて僕はますます退屈になった。   
\\	寒くなるにつれて、セーターがほしくなった。  
\\	飛行機は急角度で上昇し、それから海岸を離れるにつれて水平飛行に移った。   
\\	興奮するにつれて、彼はますます早口になった。   
\\	金持ちになるにつれて、彼はますます欲深くなった。   
\\	空気が抜けるにつれて風船は徐々に降りてきた。   
\\	経験を積むにつれて更に知恵が身につく。   
\\	芸術家は年を取るにつれて画風が変わるであろう。   
\\	研究するにつれてますます心理学が面白く感じるようになった。   
\\	高く昇るにつれて大気は薄くなる。   
\\	高く上るにつれて、空気はより冷たくなっている。   
\\	歳をとるにつれて、人はだんだん無邪気で無くなる。   
\\	私が登るにつれて、枝が曲り始めた。   
\\	時代の変化につれて結婚の形も変わってきた。  
\\	時間が経つ につれ、したこともだんだん忘れてしまう。  
\\	たら	たら	
\\	その夜半に目覚めたら呼吸が酷く不規則で苦しくなっていました。  
\\	私ももし、イギリスに行ったらそうなるのか。  
\\	まず説明書を読んだらIKEAの家具のパックが作れる。  
\\	「あなたに何かあげられたらなぁ…」  
\\	100万円獲得したら、どうしますか。  
\\	あなたが済んだらその新聞を僕に読ませてください。  
\\	このにわか雨がやんだら、私たちは出かけるつもりです。  
\\	この仕事がすんだらたっぷり休める。  
\\	もし子供が毒を飲んだら、病院に急いで連れて行きなさい。  
\\	もし彼が休んだら、日本語のテストはないでしょう。  
\\	飲んだら運転してはならない。  
\\	君がもう一度彼に頼んだら、彼も気が変わるかもしれない。  
\\	私が鳥だったら、一日中空を飛び回りたい  
\\	(誠)
\\	食堂へ行ったら、マリーさんに会いました。   
\\	アリスさんが朝起きたら、雨が降っていました。  
\\	さすが (流石)	さすが	
\\	さすが、彼は期待を裏切らないね。  
\\	さすがだね。君ならやってくれると思っていたよ。  
\\	さすがに偉大な学者だけあって、彼はその問いに容易に答えた。  
\\	年の功と言うのか、彼の意見はさすがに一日の長があるね。  
\\	彼はさすがに大学へ行っただけのことはある。  
\\	本当の事を言ったとはさすがに勇気がある。  
\\	に至る	にいたる	
\\	この道は公園に至る。  
\\	義務の路は栄光に至る道であった。   
\\	公民権運動はある夢に至る。   
\\	重要なのはゴールではなく、そこに至る道程である。   
\\	昔から今に至るまで存在する、あらゆる社会の歴史は階級闘争の歴史である。   
\\	宣戦布告に至るまでの時期はこのような状況であったと言える。   
\\	大事に至る前に火事は消し止められた。   
\\	彼らは、カルカッタからニュウーヨーク市に至るまで、世界中に支部を持っている。   
\\	滅びに至る門は大きく、その道は広い。   
\\	関東から東北にいたるまで  
\\	中国から地中海に至るシルクロードには、さまざまな文化交流の跡が見られる。  
\\	にとっての	にとっての	
\\	〜から見た	
\\	視覚障害者の方にとっての「ホームページの読みやすさ」を診断するソフトウェアを無償提供開始.  
\\	この世は我々の想像力にとっての画布そのものである。   
\\	我々自身にとっての法律は、我々の理性が法律である。   
\\	原爆は人類にとっての重大な脅威だ。   
\\	雇い主にとっての倒産は、従業員にとっての解雇と同じである。   
\\	私にとってこの本は、あなたにとっての聖書のような物だ。   
\\	習慣や考え方は人によって異なるので、自分にとっての常識は他人にとっての非常識でないともかぎらない。  
\\	事実は科学にとって、ちょうど詩人にとっての言葉のようなものである。   
\\	精神にとっての読書は肉体にとっての運動に当たる。   
\\	男性にとっての最大のストレスの原因は仕事であるが、女性の場合は人間関係が1位である。   
\\	なり	なり	
\\	会社を出たなり、いつまでも帰らない。  
\\	彼女を見つめたなり、しばらくの間微動だにしなかった。   
\\	かけた	かけた	
\\	作文は今日中に書かなければならないのにまだ書きかけだ。  
\\	読みかけた。  
\\	完全に忘れている。読んだ(いや、読みかけた)ことは、確かなのに。  
\\	冷蔵庫の中にある飲みかけの物を捨てて下さい。  
\\	この食べかけのお弁当はだれの?  
\\	ケーキを食べかけた時に、電話が掛かって来た。  
\\	おぼれかけた人は息を吹き返した。  
\\	その子供は車にひかれかけた。  
\\	半分忘れかけた音楽が彼の心の中で踊りました。  
\\	彼が何かを言いかけたが先手を打ってやった。  
\\	彼が倒れかけた時私は彼の腕をつかんだ。  
\\	彼が眠りかけたとき名前が呼ばれるのを耳にした。  
\\	彼はその仕事に飽きて、やりかけたままで放っておいた。  
\\	辛うじて	かろうじて	
\\	やっとのことで。何とか。ようやくのことで。わずかに。	
\\	走って行ったらかろうじて2時の電車に間に合いました  
\\	かろうじて落第せずにすんだ  
\\	かろうじて付き合っている  
\\	11時までにかろうじてなんとかその手紙を書いた。   
\\	かろうじて間に合いましたね。   
\\	かろうじて事故に遭わずにすんだ。   
\\	かろうじて車に当てられずにすんだ。   
\\	かろうじて列車に間に合った。   
\\	ところを	ところを	
\\	〜時(なの)に・状況(なの)に ・相手の状況を配慮する阜サとして、話の前置きに用いることも多い。	
\\	お忙しいところをおいで下さいまして、ありがとうございます。  
\\	本日は、雨で足元が大変悪いところを、このようにたくさんの方にご参加いただきまして、誠にありがとうございます。  
\\	お休みのところをお邪魔致しまして、垂オ訳ありません。  
\\	〜それまでだ	〜それまでだ	
\\	今までどんなに頑張ったには関係なく、あきらめたらそれまでだ。  
\\	限りだ	かぎりだ	
\\	とても、たいへん、きわめて	
\\	試験にパスできなくて、悔しい限りだ。  
\\	ないことは(も)ない	ないことは(も)ない	
\\	お箸は使えないことはないんですが、ナイフとフオークの方が使いやすいです。  
\\	納豆は食べないこともないんですが、あまり好きじゃないんです。  
\\	どんなに年をとっていても学べないことはない。  
\\	新聞を読めないことはないんですが、時たまですね。   
\\	「日本語は難しくありませんか。」「いえ、難しくないこともないんですが、日本語の難しさは強調されすぎていると思いますよ。」  
\\	「田中さんのふるさとは雪がたくさん降るところだから、スキーはお得意でしょう。」 「ううん、できないことはないけど、もう何年もやってないから…」  
\\	(赤毛)
\\	に先立って	にさきだって	
\\	試合に先立って開会式を行う  
\\	だけのことはある	だけのことはある	
\\	彼はさすがに大学へ行っただけのことはある。   
\\	このメロン、甘くて、とってもおいしい。高かっただけのことはあるね。  
\\	好きなときに電話できない間柄の女性と付き合っていたとき、ただ携帯電話に電話番号を表示させてみるだけのことがよくあった。  
\\	後で後悔しないように出来るだけのことをしなければならない。  
\\	彼女はテストのリスニング部分で満点をとった。さすがにアメリカ人の彼氏といつも話しているだけのことはある。  
\\	その映画は面白かった。
\\	が作っただけのことはある。  
\\	風	ふう	
\\	日本風のマンションでいいです。  
\\	たまにタイ風焼きうどんを食べます。  
\\	露天風呂のある純和風旅館に泊まった。  
\\	どんな風に卵を作れますか?  
\\	フランス風に料理する。  
\\	この家はジョージ王朝風に建てられた。  
\\	掛かる	かかる	
\\	〜	死にかかった= 
\\	ドアが開き かかったが、また閉まってしまった。  
\\	僕は車の事故で死に掛かった。   
\\	森の上に月が出かかっていた。   
\\	月曜日に始めた仕事も金曜日には出来かかっていた。  
\\	旅はおよそ5時間くらい掛かるでしょう。  
\\	-方	-かた	
\\	私は英語の手紙の書き方を知っています。  
\\	この漢字の読み方を教えてください。  
\\	いろいろな人の考え方を見出すのが好きです。  
\\	うちに	うちに	
\\	暗くならないうちに帰ってきなさい。/ 
\\	暗くならないうちに帰ってきなさい。  
\\	雨が降らないうちに帰りましょう。  
\\	電車が動いているうちに降りてはなりません。  
\\	雨が降らないうちに、帰ったほうがいいと思うから、これで、失礼します。   
\\	母が元気なうちに、会いにいきたいと思っています。  
\\	(赤毛)
\\	彼女は雇われて5ヶ月もしないうちにくびになった。   
\\	その村に着かないうちに彼女はすっかりへたばってしまった。   
\\	〜も	〜も	
\\	これも私のです。
\\	あなたもジャズが好きですか?  
\\	あなたも私たちと一緒にいらっしゃいませんか。  
\\	お客様もオレンジジュースになさいますか。  
\\	あの小さいのもわたしのです。  
\\	いくつかこの土地のワインもございます。  
\\	「ぼくもだよ。」  
\\	ピザも食べるのが好きですか。  
\\	〜でたまらない・〜てたまらない	〜でたまらない・〜てたまらない	
\\	は状態を表すもの	
\\	子供のことが心配 でたまらない。  
\\	良い表現が見つかりませんけど、おいしくて、おいしくて たまらないです。 紅茶ありがとうございました。  
\\	彼女は金が欲しくてたまらない。  
\\	私たちは行きたくて堪らない。   
\\	彼は名声欲しくて堪らない。  
\\	私は一杯飲みたくて堪らない。  
\\	のことだから	のことだから	
\\	弘子のことだから、遅れるでしょう.   
\\	あなたのことだから上々でしょう。  
\\	方が	ほうが	
\\	3月1日の方が都合がよいのですが。  
\\	9時以降に電話した方が安いですか。  
\\	物事を生半可に知っているくらいなら全然知らない方がましだ。  
\\	遅くても着実な方が競争に勝つ。  
\\	私は肉より魚の方がいい。  
\\	無理をしない方がいいですよ。   
\\	もう帰ったほうがいいですか。  
\\	〜わけだ	〜わけだ	
\\	〜(の)は当り前だ  〜(の)は当然の結果だ	
\\	また、テロがあったそうだ。空港の手荷物検査が厳重なわけだ。   
\\	「この部屋寒いですね。」「寒いわけです。窓が開いています。」   
\\	波が高いわけだ。台風が近づいているんだった。   
\\	子供が4人もいるから、私は毎日がんばって働いているわけです。  
\\	この道の辺りは渋滞であって、回り道わけだ  
\\	にひきかえ	にひきかえ	
\\	に比べると反対に	
\\	私の古い車にひきかえかれのはとても早いです。  
\\	去年はトマト50個収穫できたのよ。それにひきかえ今年は30個だけ。多分気候のせいね…  
\\	同じ年の人 にひきかえ彼はとても頭がいいです。  
\\	几帳面な父にひきかえ息子はどらしなかった。  
\\	この大学は教育の質がとてもいいのにひきかえ、学内の設備が劣る(おとる)   
\\	先月の食料品の売り上げが著しく伸びたのにひきかえ、衣料品の売り上げが落ち込んだ   
\\	16世紀のフランスでは、貴族の食事は贅沢だったのにひきかえ、市民の食事はとても貧しいものだった。   
\\	前回の作品にひきかえ、今回のはすばらしいできだ。   
\\	だした	だした	
\\	だす	
\\	車が動き出した  
\\	1マイル歩くか歩かないうちに雷が鳴りだした。   
\\	2人の友人は深い友情のきずなをつくりだした。   
\\	いったんはやりだしたら、リーバイスのジーンズは決してすたれなかった。   
\\	お金を全部使ってしまい、私たちは仕事を探しだした。   
\\	さらに悪いことに雨がふりだした。   
\\	しばらくして彼らは歩きだした。   
\\	そのときジョーンズさんが何かいいだした。  
\\	その少女はわっと泣きだした。   
\\	その話を聞いたら父のことを思いだした。   
\\	それは彼の身に応えだした。   
\\	ゆっくりと彼は状況が分かりだした。   
\\	サーカスを見にみんなくりだした。   
\\	ホイッスルの音を聞くやいなや、彼らは全速力で走りだした。   
\\	一人ぼっちにされたので、その小さな女の子は泣きだした。   
\\	雨が降りだしたとき何をしていましたか。   
\\	雨が降り出すと洗濯物をとりこみだした。  
\\	家に帰り着くとすぐ雨が降りだした。   
\\	彼女は私を見て、急に笑いだした。   
\\	彼女は非常に喜んだので踊りだした。   
\\	彼女は夫との口論の際、とても興奮して突然泣きだした。  
\\	彼女はマンガを読んでいて、笑いだした。  
\\	弟とお姉ちゃんが笑い出した。  
\\	彼女が話しだした。  
\\	ものですから	ものですから	
\\	どうして遅刻したんですか?目覚まし時計が壊れていたものですから。  
\\	いつもの列車に乗り遅れたものですから。   
\\	ご迷惑をおかけしますが、急を要することが起こったものですから。   
\\	その件に関しては発言する立場にないものですから。   
\\	それは本当に役にたつものですから、ぜひ買うべきです。   
\\	もう一度おっしゃって下さいませんか。お名前がはっきりと聞き取れなかったものですから。   
\\	メキシコでは、多くの困難を経験しました。スペイン語が全然わからなかったものですから。   
\\	家族の所へ行きたいものですから、2〜3日休みをとってもいいでしょうか。   
\\	子供が6人いるものですからね。   
\\	私の代わりにこれをしていただけませんか。私は大変に忙しいものですから。  
\\	事故で電車が遅れたものだから、遅くなったすみません。  言い訳 
\\	て形からでないと; て形からでなければ	てからでないと; てからでなければ	
\\	上司に相談してからでないとお返事できません。   
\\	舞鶴に行くには京都に行ってからでないと行けません。  
\\	ビザが取れてからでなければ、旅行の予定は立てられない。  
\\	この問題は、よく検討してからでないと、決定は難しい。  
\\	子供が小さい時、寝かせてからでないと、ゆっくりテレビも見れなかった。  
\\	しっかり考えてからでないと、お答えできません。  
\\	(赤毛)
\\	もう少し具体的な説明を聞いてからでないと、その計画には賛成できません。  
\\	(赤毛)
\\	彼に示してからでないと、彼はおまえの感情が分からないよ。  
\\	臨席をしてからでなければ、なにも分かってない。  
\\	ながら	ながら	
\\	私はその薬が危険だと知りながら飲みつつけていた。  
\\	二人はほほえみながらハリーを見つめるばかりだった。  
\\	各社は供給過剰などのリスクもにらみながら、事業展開を進めることになる  
\\	くる、くると言いながら来なかったじゃない。  
\\	この掃除機は、小型ながら性能がいい。  
\\	今週、風邪を引いていながら、仕事へ行った。  
\\	秋子さんは, ピアノを 弾きながら 歌いました。  
\\	運転しながら, 音楽を聞きます  
\\	森さんはお茶を飲みながらテレビを見ています。  
\\	歩きながら話しませんか。   
\\	お茶お飲みながら話しましょう。   
\\	話を聞きながらメモしてください。  
\\	運転しながら電話をしないでください。   
\\	ピアノを弾きながら歌えますか?   
\\	彼は話しながらタバコを吸っていた。   
\\	本を読みながら眠ってしまった。   
\\	私たちは食事をしながらその問題を論じた。   
\\	彼はパイプをくわえながら話す。   
\\	彼は金はありながら幸せではない。   
\\	彼女は病気でありながら学校へ行った。   
\\	遺憾ながら彼女は重傷を負っておられると垂オ上げねばなりません。   
\\	彼女と離れながら、恋が残ってある。  
\\	〜ず	〜ず	
\\	思わず裸足で外に出た。  
\\	女だと間違えず働きやがって!スケベなリモコンだ!  
\\	日本語を覚えずに日本でのクラス(授業)は難しい。  
\\	なぜあなたは食わず嫌いになったのか?  
\\	日本語を覚えずに、日本で暮すのは難しい。  
\\	愛を信じられずに育ったカン・ジェホは、働きながら大学に通う苦学生。   
\\	困っている人を見たら、助けずにはいられない。  
\\	(赤毛)
\\	あまり将来のことを考えずにこの仕事を選んだばかりに、今苦労ばかりしている。  
\\	(赤毛)
\\	この布は見ずに強い反面、熱には弱い。  
\\	(赤毛)
\\	雨が降らず、川の水が干上がった  
\\	こんな	こんな	
\\	こんな面白いことはない  
\\	こんなに夜遅く  
\\	こんな風に持って下さい  
\\	こんなことになろうとは、正に「お釈迦様でも気がつくめい」だね。  
\\	「マイク、飛行機って普通はこんなふうに揺れるものなのかい」  
\\	あなたがこんなに早くまた起きて動き回れるようになって本当に良かった。  
\\	あなたは昔私をこんなふうに扱わなかった。  
\\	あなたをこんなに長い間待たせてすみません。  
\\	いつもこんなに混雑しているのですか。  
\\	いやはや、こんなにはっきりものを言う秘書は初めてだ。  
\\	かつて一度もこんな大群衆を見たことがない。  
\\	きみがこんなに料理が上手だとは知らなかった。  
\\	こんなあしらいをされたら誰だって承知しないでしょう。  
\\	こんないい天気の日に、あなたは家の中に居なければならないなんてかわいそうだ。  
\\	こんなおいしいスープを味わったことがありますか。  
\\	こんなきつい仕事はこりごりだ。  
\\	こんなきれいな夕焼けは見たことが無い。  
\\	こんなくだらないことに多額の金を支払うとはけしからん。  
\\	こんなに心配しないでください。  
\\	を問わず	をとわず	
\\	学歴、経験は問わず、やる気のある社員を募集します。  
\\	私が書いたかを問わず、(全て)「作者:あまつか」です。  
\\	著作権表示があると否とを問わず真実の作者は法律上当然著作権者になります。  
\\	その試合には国籍のいかんを問わず誰でも参加できる。  
\\	試合は晴雨を問わず行われます。  
\\	年齢を問わず人々はこの歌が好きだ。  
\\	肌の色のいかんを問わず、彼は万人の言論の自由を擁護した。  
\\	老いも若きも、また男女を問わず、地元の浴場で一緒になることを楽しみにしていた。  
\\	たり する 
\\	たり	
\\	たり、 
\\	たり します。	
\\	飲んだり、食べたり、歌ったり、ダンスをしたり、大いに楽しみました。  
\\	パーテイーで飲んだり, 食べたり, 踊ったりした。  
\\	私達は, 先生に誉められたり, 叱られたりします。  
\\	夏には, 登山したりする。  
\\	私は, 朝コーヒーを飲んだり, 飲まなかったりする。  
\\	天気によって富士山が見えたり, 見えなかったりする。  
\\	金曜日には木村さんは来たり, 来なかったりします。  
\\	あの店の包装紙は, 赤かったり, 白かったり, 青かったりします。  
\\	このレポートのデータは, 正しかったり(正しく)なかったりする。  
\\	くれる (呉れる)	くれる	
\\	彼女は、「これを 洗って くれないか い」と いった。   
\\	私の先生は 私に 本を読むためにを くれました。    
\\	先生は私に教えてくださいました。   
\\	友達は私に本を買ってくれました。   
\\	私は先生に教えていただきました。 
\\	私は友達に本を買ってもらいました。 
\\	今日、来てくれてありがとうございます。   
\\	今日、来てくれてありがとう!   
\\	手伝ってくれてありがとうございます。   
\\	くらいなら〜〜ほうがいい	くらいなら〜〜ほうがいい	
\\	友達を傷つけるくらいなら、自分が我慢したほうがいい  
\\	あなたと結婚するくらいだったら鳥と結婚するわ。  
\\	ここにいるくらいならすぐに出発したほうがいいですよ。  
\\	あんな男と結婚するくらいなら死んだほうがましだ。  
\\	ここでタクシーを拾おうとするくらいなら家まで歩いたほうがましだ。  
\\	このお金を彼にあげるくらいなら、捨てたほうがいい。  
\\	この会社でそんな仕事をするくらいなら会社をやめる(ほうがいい)。  
\\	この暑い中を外出するくらいなら家にいた方がよい。  
\\	こんな料理を食べるくらいならドッグフードを食べたほうがましだ。  
\\	君を捨てるぐらいなら死んだほうがましだ (ほうがいいだ)。   
\\	降参するぐらいなら死んだほうがましだ (ほうがいいだ)。   
\\	と同時に	とどうじに	
\\	ほぼ同時に	
\\	彼が入ってきた、それと同時にベルが鳴った。	
\\	彼は厳しいと同時に優しい。  
\\	私たちがプラットホームに着くのと同時に電車が出てしまった。  
\\	彼は科学者であると同時に音楽家出もある。  
\\	彼らは笑うと同時に泣いた。  
\\	(赤毛)
\\	駅に着くと同時に彼女は母に電話した。  
\\	(赤毛)
\\	生まれると同時に死は始まる。  
\\	(赤毛)
\\	スポーツの逆説はスポーツが(人を)分けると同時に結びつけることだ。  
\\	(赤毛)
\\	知識は力であるが、それは善を行う力であるのと同時に、それと全く同じ程度に悪を行う力でもある。  
\\	(赤毛)
\\	自分の意見も言うと同時にほかの人の考えもよく聞くことが大切だ。  
\\	(赤毛)
\\	彼女の長所を認めるが、それと同時に欠点も目につく。   
\\	地震と同時に火事が発生した。   
\\	彼らはチャイムが鳴ると同時に席を立った。   
\\	〜なり〜なり	〜なり〜なり	
\\	ジュースなりコーラなり、お好きなものをどうぞ   
\\	君が来るなり,僕が行くなりしなくてはならない   
\\	勉強ばかりしていいないで、たまには外に遊びに行くなりスポーツをするなりして、気分転換をしたらどうですか。   
\\	風邪を引いたのなら、薬を飲むなり何なりすればよかったのに   
\\	うろうろ探しまわらないで、交番で聞くなりすれば良かったのに   
\\	英語なりスペイン語なり、なんか外国語を身につけたい。   
\\	も〜ば〜も	も〜ば〜も	
\\	父はお酒ものめばタバコも吸うので、健康が心配だ。  
\\	小泉さんは英語も上手なら、フランス語も上手です。  
\\	小泉さんは英語もできればフランス語もできる。  
\\	彼は歌も上手なら踊りもうまい、パーティーの人気者だ。  
\\	彼は地位もなければお金もない。  
\\	結婚生活というものは楽しいこともあれば大変なこともある。  
\\	まるで	まるで	
\\	まるで眠れやしない。  
\\	フレッドには忍耐心というものがまるでないから、彼のカンシャク玉を破裂させるのはわけもない。  
\\	フレデイはこの一ヶ月というものずっと夜勤だったため、正常な時間に働いている友人達に会う機会がまるでなかった。  
\\	寒くて手の感じがまるでない。  
\\	君の考えと僕のとまるで違うね。  
\\	彼女はまるで徹夜をしたかのような顔をしていました。   
\\	(赤毛)
\\	彼女はまるで彼に信じている。  
\\	彼には紳士らしいところがまるでない。   
\\	彼女がどんな人なのかまるで見当がつかない。   
\\	た形ところ	たところ	
\\	マニュアルの説明通りにやってみたところうまくいった。  
\\	彼女に電話をかけようと考えていたところ、彼女の方から、電話をかけて来た  
\\	自宅でのインターネット利用についてたずねたところ、「利用している」が28.6%であった。  
\\	彼女に電話をしようとしたところ、電話がかかってきた。  
\\	犯人が店から出てきたところで警官に捕らえられた。  
\\	担当者に電話で問い合わせてみたところ、社員旅行の申し込みはもう終わっていた。  
\\	実際、日本へ行ってみたところ、想像以上に素晴らしかったです。  
\\	(赤毛)
\\	しっ、静かに。少し前に赤ちゃんがやっと寝たところなんだから。  
\\	(赤毛)
\\	電話帳で調べたところそういう名前の会社はなかった。  
\\	(赤毛)
\\	会社の問い合わせをしたところ、地図を送ってくれた。  
\\	(赤毛)
\\	新しいヲープロを使ってみたところ、とてもつかいやすかった。  
\\	角を曲がった所に郵便があります。   
\\	仕事が終わったところで電話します。   
\\	駅に着いたところ列車はもう出てしまっていた。   
\\	くらい〜はない	くらい〜はない	
\\	外国で病気になるくらい不安なことはない。	
\\	外国で病気になるくらい不安なことはない  
\\	鈴鹿に買い物に行くくらいいいことはない。  
\\	赤ちゃんぐらいかわいいものはない。  
\\	漢字を憶えるくらい難しいことはない。  
\\	し	し	
\\	ここでの討論はおもしろいものでしたし、困惑させたり、ときにはぞっとする内容のものもありました。  
\\	ロッククライミングも、海で深く潜ることもしたし、インドネシアの熱帯雨林で眠ったこともある。  
\\	開拓者達は辛い生活をしたし、彼らの子供もそうだった。  
\\	彼は幸福だ。1つには、昇進したし、それに結婚したばかりだから。  
\\	母はその知らせを喜んでいたし、私もそうでした。  
\\	洋子はジョンを完全に無視したし、ジョンのほうも彼女を同じように無視をした。  
\\	1つには私は忙しいし、もう1つには興味もない。  
\\	あの絵も美しいし、この絵もまた美しい。  
\\	ことだから	ことだから	
\\	彼のことだから、すぐにここへ飛んでくるに違いないと思った。  
\\	そうゆうことだから彼が来るのは期待できないよ。   
\\	健康は富にまさる。なぜなら前者は後者より大事なことだから。   
\\	自分でやったことだから仕方がないね。   
\\	彼のことだから、きっと最後まで頑張りぬくと思います。  
\\	子供のことですから、そんな難しいことを言ってもわかりませんよ。  
\\	のんきなルミのことだから、何も心配していないだろう。  
\\	致す	いたす	
\\	今後とも努力を致す所存であります。  
\\	その頃わたくし共は北山の掘立小屋同様寝起きをいたして...   
\\	なしに	なしに	
\\	〜をしないで 
\\	がないままで
\\	あなたのおもてなしにはどんなに感謝しても感謝しきれません。  
\\	この辞書なしには済まされない。   
\\	その取り決めは、予告なしに変ることがある。  
\\	だれも努力なしには何事も成し遂げられない。   
\\	なんの気なしに言っただけで、別に悪気があってのことではありません。   
\\	ぶどう酒なしには一日も過ごせない、と彼は言う。   
\\	男たちは何の許可もなしに事務所のコンピュータを勝手に使ってた。  
\\	一切の政治権力なしに人々の自由を実現することは、はたして可狽ネのだろうか。  
\\	せんがため	せんがため	
\\	ぜひ実現させたい積極的な目的をもってあることをする	
\\	キリスト教義の正当性をより強固にせんがための伏線ととれないこともありません。  
\\	あの頃わたしは大学に進学せんがために毎日必死でがんばった。   
\\	研究を完成させんがため、彼は昼夜寝ずにがんばった。   
\\	一日も早く自分の店を持たんがために、必死で働いている。  
\\	〜など	〜など	
\\	彼は失敗などしないと言う。  
\\	私は野球、テニス、バレーボールなどをすることができます。  
\\	お母さん、アップルジュース、コーラ、お茶等を買って欲しいんですけど。  
\\	鉛筆やペンや消しゴムや本などを買いました。  
\\	彼らは私の年齢、姓名、国籍などを聞きました。  
\\	〜としても	〜としても	
\\	ゲーム用
\\	としても進化を遂げつつある
\\	彼はとてもいい人だとしても、私は彼を本当は信頼していない。   
\\	息子さんがこの世でいちばん注意深い人物だったとしても、あなたはまだ心配されると思います。  
\\	彼女と結婚できるとしても、2、3年後になるだろう。   
\\	その人が親切そうに見えたとしても、本当に親切かどうか分からない。  
\\	親が有名だとしても、本人に実力がなければ社会で認められないでしょう。   
\\	たとえ相手が部下だとしても、世話になったらお礼の気持ちをちゃんと伝えるべきだ。  
\\	来週の陸上競技大会に走るとしても負けると思う。  
\\	よもや	よもや	
\\	よもやそんなことは有り得ない  
\\	よもやそんなことはあるまい。  
\\	私はよもや彼女が有罪だとは思ってもみなかった。  
\\	うちに	うちに	
\\	はじめは予想しなかったような変化が現すことを表す。	一週間の外国出張のうちに、近くの公園の桜が全部散ってしまった。W
\\	今は上手に話せなくても練習を重ねるうちに上手になる。  
\\	友達に誘われて何回か山登りをしてい るうちに、僕も山が好きになった。  
\\	(よ)うが, (よ)うと ようが ようと	(よ)うが, (よ)うと ようが ようと	
\\	ても、ようとも
\\	人がなんと言おう(と/が)、私は自分の子供を信じている。  
\\	頑固と言われよう(と/が)バカと言われよう(と/が)、私は自分の信じることをやります。  
\\	あの教授は学生が分かろう(と/が)分かるまい(と/が)かまわず授業を進めていく。   
\\	報酬が多かろう(と/が)少なかろう(と/が)私はやるべきことをやるだけだ。   
\\	周囲に反対されようが、自分でやると決めた以上は最後までやり抜くつもりだ  
\\	か〜ないかのうちに	か〜ないかのうちに	
\\	電車のドアが開くか開かないかのうちに乗客が飛び出して来る  
\\	ピストルの音が鳴ったか鳴らないかのうちにスタートしました。  
\\	その一行が出発するかしないかのうちに、雨が振り出した。  
\\	私がホールに入るか入らないかのうちに式が始まった。  
\\	私が寝入るか寝入らぬかのうちに、電話がなった。  
\\	彼は本当にプロ並みの
\\	のように、一曲が終わったか終わらないかのうちに、次の一曲を整えてくる。  
\\	電話が鳴るか鳴らないかのうちに、受話器を取りました。  
\\	この店の寿司弁当は、店頭に並ぶか並ばないかのうちに、飛ぶように売れていく。  
\\	〜からこそ	〜からこそ	
\\	常識に反する理由だが、その理由を特に言いたい時に使う表現	
\\	雪だからこそ、うちにいたくない。雪の日にうちにいるのは寂しすぎる。  
\\	頭がいいと思っているからこそ、厳しくしつけるのです。  
\\	実は、知らない人ばかりだったからこそ、言いにくいことも言うことができたのだ。  
\\	手伝ってくれたからこそ、仕事が早く出来ましたよ。  
\\	のに	のに	
\\	日本語を覚えるのに、時間がかかります  
\\	あなたに会うのには何時(いつ)/何処(どこ)がいいですか。  
\\	パスポートは海外旅行に行くのに必要です。   
\\	水着は泳ぐのに使う。  
\\	私がその仕事を仕上げるのに5ヶ月かかった。   
\\	少し	すこし	
\\	かれは少しおそいですね。
\\	もう少し居てもいいですか  
\\	「水を少しくださし。」「いいですよ。」  
\\	11時15分前を少し過ぎています。
\\	あそこは少し高すぎると思います。  
\\	あなたの報告書には少し間違いがあります。
\\	お茶が濃すぎるので少し水を加えてください。
\\	今日は少し寒い。
\\	この靴は少しゆるい。
\\	このスープは塩が少し足りない。  
\\	このスカートは少しきついのです。くのすかあとはすこしきつのです。
\\	なくして	なくして	
\\	愛なくして、結婚生活は続けられない。   
\\	民主主義なくしては達成できない。  
\\	苦労なくしてもうけなし。  
\\	私の許可なくして、君はそんなことをすべきではなかったのだが。  
\\	忍耐なくしては、誰も成功することはできない。  
\\	山川教授なくしては今の私は存在しなかったであろう。   
\\	それっきり	それっきり	
\\	それ以来ずっと、それ以来	
\\	僕はそれっきりスパイクと会っていない  
\\	店はたたれたまま、それっきりだった。   
\\	去年11月プネに移りました。それっきり先生と連絡しなかったです。  
\\	なので	なので	
\\	この情報は内々に知らされたものなので、あなたに何も教えることはできない  
\\	それは真実なのでいまさら四の五の言ってもはじまらない  
\\	あんなに子供が好きなので、スーは先生になるべきだ。   
\\	いすをどかしてください。邪魔なので。   
\\	イヤホンを頼んだはずなのですが。   
\\	お天気なので、釣りに行きたいような気がする。   
\\	お母さんが病気なので、今日はお父さんが料理をします。   
\\	こういうわけで私はあなたに反対なのです。  
\\	今ごろになって ~も	いまごろになって ~も	
\\	今ごろになって勝訴しても、亡くなった人は帰ってこない。  
\\	明日出発なのに、今頃になってビザがないなんて。  
\\	今ごろになってそのようなことを言われても遅いよ!  
\\	今ごろになってやめろと言われても遅い  
\\	今頃になって(から)、切符を買うのは遅すぎる  
\\	今ごろになって(から)、注文しても遅い   
\\	今頃になってそんなこというの?  
\\	最中に	さいちゅうに	
\\	考えている最中に、話しかけられて困った。  
\\	[スクリーンセイバーのオフ] ゲームをやっている最中にスクリーンセイバーが起動されるのを防止します。   
\\	冬の真っ最中の登山は十分な冬山の経験がなかったらやめるべきです。  
\\	食事をしている最中に地震が起きて、大騒ぎになった。  
\\	彼の演説の最中に中座する。   
\\	彼女は食事の後片付けの最中だった。   
\\	わりに	わりに	
\\	彼は年のわりに老けて見える。  
\\	高いわりに美味しくない  
\\	二年間居たわりには日本語が・悪くはない・うまくはない  
\\	日本に長く住んでいるわりに日本のことを知らない  
\\	あなたは年のわりに若く見える。  
\\	年のわりにはトムはよくやった。  
\\	この布は値段の割に品質が良い。  
\\	ダイエットは非常な努力を要するが、その割に成果は少ない。  
\\	彼は経験がない割には良くやっている。  
\\	お金がないと言っているわりに、よく買い物をするね。  
\\	父は、普通ならとっくに引退している年齢だが、年をとっているわりには元気だ。  
\\	なるほど彼は若いが、年の割りには賢い。  
\\	その子は年の割りに小柄だ。   
\\	彼はブラジルにいた割りにはポルトガル語が下手だ。   
\\	あっての	あっての	
\\	意味 
\\	があるからこそ
\\	がある。
\\	がなければ
\\	もない。	
\\	あなたの援助あっての私の成功。  
\\	今日の私があるのも、20数年前に私を産んでくれた両親あっての事です。  
\\	なんの気なしに言っただけで、別に悪気があってのことではありません。   
\\	事故で車は全損って嘆いているけどさ、怪我がなくてなにより、命あっての物種だよ。   
\\	命あっての物種。   
\\	日々の練習あっての勝利だ。  
\\	デパートはお客様あっての商売、お客様は何より大切です。  
\\	みせる	みせる	
\\	私は彼に顔を見せるのが恥かしかった。  
\\	そんな手紙を私に見せるなんて、どうかしてるのとちがう?  
\\	テッド叔父さんはパンダを見せるために、私たちを動物園へ連れて行った。  
\\	今の学生に暇な時間がもっとあれば、政治にもっと関心を見せるかもしれない。  
\\	私が買ったばかりの車を君に見せるよ。  
\\	私はあなたに見せる写真がある。  
\\	自分の本当の感情を見せることは、日本では美徳とは考えられていない。  
\\	手の内を見せるな。  
\\	乗客は全員切符を見せることが要求される。  
\\	切り札は最後まで見せるな。見せるなら、さらに奥の手を待て。  
\\	彼がいつか姿を見せることは確実だ。  
\\	彼女は背を高く見せるためにハイヒールをはいている。  
\\	僕らは、彼が土曜の午後に姿を見せるのを期待している。  
\\	手の内をみせるな。  
\\	天候は回復のきざしを見せている。   
\\	彼女はグラフをコンピュータ画面に出して見せた。   
\\	彼は消火活動の際にすばらしい勇敢を見せた。   
\\	入り口で会員証を見せなくてはなりません。   
\\	あなたの新しい車を見せてくれませんか   
\\	腕のどこを怪我したのか医者に見せなさい。   
\\	彼を私にバンジョーを弾いて見せた。   
\\	気が付いた	きがついた	
\\	納豆は美味しいと気が付いた。  
\\	気がついたら病院のベッドに横たわっていた。  
\\	公園のベンチにかばんを置き忘れたことに今気がついた。  
\\	(赤毛)
\\	彼が好きになったのに気が付いた。  
\\	ような気がする	ようなきがする	
\\	予感・予測・不確かな記憶・予兆を表現する方法としてもっとも幅広く用いられる言い方っす。	
\\	もう他に方法はないような気がする。   
\\	明日雨が降るような気がする。  
\\	その話は本当のような気がする。   
\\	あれはまだ終わってなかったっけ。もう終わったような気がする。  
\\	新しい仕事ではうまくいくような気がする。   
\\	心臓が張り裂けるような気がする。   
\\	自分がこの世で最も幸せな者のような気がする。  
\\	なら	なら	
\\	外からなら撮ってもいいよ と言われたので,外に出て撮影しました。  
\\	乗るなら飲むな。  
\\	自分の信念を隠すくらいなら死んだほうがましだ。   
\\	よんどころない理由で参加されないのなら前払い金はお返しします。  
\\	ご都合により参加されないのなら前払い金はお返ししません。  
\\	その仕事がだめだと思うなら、辞めればいいと思うよ。  
\\	仙台に来ていたなら、電話をしてくれればよかったのに。   
\\	彼が来るならこのパーティーには出席しなかったのに。   
\\	「打ち合わせの時間は7時どうですか。」「7時はだめですが、8時ならいいですよ。」   
\\	(タン)
\\	中国に行くなら、航空券を早くしといた方がいいですよ。  
\\	(タン)
\\	どうせ	どうせ	
\\	どうせ金なんてそのうちなくなるんだ  
\\	どうせ二人は枯すすき、花の咲かない枯ススキ  
\\	どうせ郵便局へ行くのなら、切手を買ってくださいませんか?  
\\	どうせ行かなければならないのなら早いほうがいいです。  
\\	あなたが手伝ってくれないことは分かっている。どうせあなたはそういう人だ。  
\\	て形
\\	みる (て見る)	てみる	
\\	着てみてもいい?  
\\	あなたは自分で行って見るべきです。   
\\	老いた故郷の両親のことを考えて見るべきだ。   
\\	やって見ようか?  
\\	から	から	
\\	面接に行くからはげましてください  
\\	必要から発明が生まれる。  
\\	あの日帰ったのは体調が悪かったからです。  
\\	ああ、ゆっくりやって下さい。急ぎませんから。  
\\	昨夜病気だから、眠れなかった。  
\\	(反戦)
\\	にかけては, にかけても	にかけては, にかけても	
\\	〜についていえば	
\\	について	
\\	社のテレビは、画面の美しさにかけては他社の製品に勝っている。  
\\	数学にかけては彼女はクラスでいつも一番だった  
\\	たび、たびに	たび、たびに	
\\	にサインインするたびに、プログラムをアップグレードするように求められます。   
\\	東京に行くたびに言っている気がする  
\\	かのじょはあうたびにきれいになる。  
\\	課長は出張のたびに、みんなにお土産を買ってくる。  
\\	彼女はやるたびに失敗した。   
\\	会いに行くたびに彼女は留守だった。   
\\	私は旅行のたびに、絵葉書を買います  
\\	とばかりに	とばかりに	
\\	課長は余計な質問はするなとばかりに私ににらみつけた。   
\\	霧が晴れたので私がチャンスとばかりに写真を撮りまくった。   
\\	竹本はこの時とばかりに日頃の不満を吐き出した。   
\\	彼女が舞台に登場すると、「待ってました!」とばかりに大きな拍手が起こった   
\\	彼女は、信じられないとばかりに、口を開けたまま彼を見つめていた。   
\\	にとって	にとって	
\\	から見て(も)	
\\	外人にとって納豆は食べにくい  
\\	この写真は私にとって、何よりも大切なものです。  
\\	ユダヤ人にとって、まんじは悪のしるしだよ。  
\\	この大学で教えられることは私にとって極めて名誉なことです。   
\\	町の人々にとってこの工場の閉鎖は死活問題だ。   
\\	ここは君にとって一番安全な場所だ。  
\\	私にとって、この環境は本当にありがたいです。  
\\	(赤毛)
\\	花づくりはわたしにとって一番の楽しみである。  
\\	(赤毛)
\\	この寿司は私にとって美味しくないです。   
\\	私にとってのるかそるかだった。  
\\	人間にとって重要な事柄   
\\	スポーツマンにとって視力が強いのは不利だ。   
\\	母乳は赤ちゃんにとってとても大事である。   
\\	日本の貿易にとってこわい存在  
\\	アパートの火事は入居していた家族にとってたいへんな災難であった。   
\\	今日の試験は君の将来にとってとても大切なものになるでしょう。   
\\	それは彼女にとって重大であった。   
\\	挙げ句	あげく	
\\	挙げ句の果てに彼は刑務所行きになった。  
\\	4頭の象が酔っ払った挙句の果てに高圧線に襲い掛かって感電死。  
\\	迷ったあげく彼の誘いを断ってしまった。  
\\	散々考えた挙げ句その計画を実行にうつした  
\\	両親に相談したあげくに、彼と離婚することにした。   
\\	彼女はみんなにさんざん迷惑をかけたあげく、あいさつもしないで会社をやめていった。  
\\	コンピュータを物色して歩いたあげく、デイヴィッドより200ドル安い値段で手に入れた。  
\\	私は十分考慮したあげく申し出に応じた。  
\\	色々勉強したあげく、卒業しました。   
\\	一杯食べた挙げ句に吐き出した。  
\\	どの大学に留学しようかと、さんざん悩んだあげく、日本の大学に決めた。  
\\	(赤毛)
\\	さんざん文句を言ったあげくに、出て行ったんです。  
\\	(赤毛)
\\	ざんざん迷った あげく彼のプロポウズを断れた。  
\\	練成を完成したあげく、体が逞しくなりました。  
\\	かねない	かねない	
\\	あの男はどんなばかなことでもやりかねない。  
\\	真面目なように見えるが彼は嘘をつきかねない。  
\\	あの男は裏切りもしかねない。   
\\	この手の雑誌は若者に害を与えかねない。   
\\	しかし贈り物をひけらかすことは、何も持ってこなかった客を当惑させることになりかねない。   
\\	その少年は盗みをしかねない。   
\\	イスラム系共和国が独立国家共同体に参加しない場合、それらの国々は独自の勢力を形成し、一触即発の民族的、宗教的な亀裂を生じかねない、と
\\	連、西側の消息筋は警告しています。   
\\	インフレを抑制しようとして金融政策に偏重すると、金融、したがって景気を必要以上に締め付けることになりかねない。   
\\	スキャンダルは嘘を魔ォかねない。   
\\	一瞬のちゅうちょでもパイロットは命を落としかねない。   
\\	一瞬躊躇すればパイロットは命も落としかねない。   
\\	学校へ行ったことのない人は貨物自動車からものを盗みかねないが、大学教育を受けた人なら鉄道全部を盗みかねない。   
\\	規則にうるさいレフェリーは試合を台無しにしかねない。   
\\	新しい調査では、65歳以上の病院患者の診察記録には誤りが多く、重大な診察ミスにつながりかねない、ということです。   
\\	奴はうそもつきかねない男だ。   
\\	冬に暖かい服装をしていないと、ひどいかぜをひくことになりかねない。   
\\	彼ならそれくらいの事はいいかねない。   
\\	彼はそんなことをやりかねない。   
\\	彼はどんな悪事でもやりかねない。   
\\	彼は望む物を手に入れるためには人をもだましかねない。   
\\	ジョンさんは作りたてチョコケーキを全て食べかねない。  
\\	嘘をつきかねない男だ。   
\\	準備は足りないので、落第しかねない。  
\\	をめぐって	をめぐって	
\\	国会は、盗聴法案を巡ってまさに侃々諤々の体であった。  
\\	その土地をどうするかをめぐって反目が生じた。   
\\	インドのバイヤーと日本のビジネスマンは新製品の値をめぐって激しいやりとりをした。   
\\	委員会はその計画をめぐって意見が割れた。   
\\	委員会はその計画をめぐって意見が分かれた。   
\\	候補者たちは減税法案をめぐって応酬しました。   
\\	国会はその法案をめぐって暗礁に乗り上げてしまいました。   
\\	先生たちはその問題をめぐって賛否が分かれた。   
\\	彼の死をめぐってあるうわさが広まっている。   
\\	彼らの間でつまらないことをめぐってけんかが始まった。   
\\	この小説は、1 人の女性をめぐって、5人の男性か戦う話です。  
\\	その事件をめぐって、さまざまなうわさがながれている。  
\\	〜に難くない	〜にかたくない	
\\	(名詞+にかたくない/ 
\\	る+にかたくない)	
\\	彼の不安は想像にかたくない  
\\	彼の喜びは、察するにかたくない。  
\\	〜をぬきにして・はぬきにして	〜をぬきにして・はぬきにして	
\\	ふつうはふくまれるもの、当然あるものを加えずに	
\\	強い意志を抜きにしては、この仕事を完成させることはできない。  
\\	料理の上手な山田さんをぬきにしては、パーティーは開けません。  
\\	皆さん、仕事の話は抜きにして楽しく飲みましょう。  
\\	皆さん、仕事の話は抜きにして楽しく飲みましょう。  
\\	気がする	きがする	
\\	悪い人のような気がしました。  
\\	高いような気がしている。  
\\	いつも排尿したいような気がする。  
\\	その話は本当のような気がする。  
\\	ちょっと危ない気がする。  
\\	彼にどこかで会ったような気がする。  
\\	例外が多いような気がしました。  
\\	電車の中での勉強はとても無理なような気がします。   
\\	旅行で行きましたけど、そこには住みたくないような気がしました。  
\\	「天気予報、今日は涼しいそうよ。」「反対に、暑い気がする。」  
\\	一週間田舎にいてすっかり元気になったような気がする。  
\\	うしろめたい気がする。   
\\	お天気なので、釣りに行きたいような気がする。   
\\	やはりもっと話さないとだめな気がします  
\\	彼、最近、私に冷たくなったような気がする。   
\\	彼がその秘密を知っているような気がする。   
\\	彼が近いうちにひょっこり訪ねてくるかもしれない気がする。   
\\	彼は医者のような気がする。   
\\	彼は勉強を始めるとどうして眠気がするのだろうか。  
\\	彼をどう思っているか話してみたい気がする。   
\\	彼女は今日来るような気がする。   
\\	分かるような気がする。   
\\	僕の人生には何か欠けているものがあるような気がするんだ。   
\\	旅に出かけたい気がする。   
\\	扁桃腺がはれている気がするのですが。   
\\	日本語は、勉強すればするほど難しくなるような気がする。  
\\	(赤毛)
\\	ほかない	ほかない	
\\	もはや、盧武鉉(ノ・ムヒョン)大統領の再信任を問う国民投票に進むほかない。それが好ましいからではなく、その他の途が閉ざされているからだ。  
\\	風邪をひいた。暖かくして寝るよりほかない。  
\\	勉強するよりほかない  
\\	本当のことを言うよりしかたなかった  
\\	政府は他の歳入源を考え出すほかない。   
\\	彼らは計画全体をあきらめるよりほかないと意見が一致している。   
\\	雪のため、新幹線が駅と駅の間で止まってしまった。乗客はじっと車内で待つよりほかない。   
\\	台風で家が押し流された。人々はどうすることもできず、じっと見ているほかなかった。  
\\	雨が降っているので、残念だが今日の遠足は延期するほかない.  
\\	居られない	いられない	
\\	お客様がそろそろお着きになって大変なので、のんびりお茶など飲んではいられない。  
\\	眩しいというのは目をあけていられないほど光が強い様子だ。  
\\	(赤毛)
\\	出かけたとき、外にいられないくらい風がつきささった。  
\\	(赤毛)
\\	〜へ	〜へ	
\\	私は来年外国へ行きます  
\\	乗客への呼びかけ
\\	(バスの)中へお詰め願います。  
\\	1000人もの人がそのパーティーへやってきた。  
\\	7月のある日、私たちは海へ行った。  
\\	切る	きる	
\\	3巻もある長い小説を1年かけて読みきった。  
\\	このセマイ部屋に、 全員入りきるのか。  
\\	ダイエット中のくせに彼女はケーキを食べきりました。  
\\	さあさあ、座って疲れきった脚を休めなさい。  
\\	その事実は誰にでもわかりきったことだ。  
\\	ハリーはなんとかその川を泳ぎきった。  
\\	彼女は、自分が正しいと言いきった。  
\\	その動きが伝えきれない部分もありますので、是非デモサイトの方をご覧ください。   
\\	1875年に、
\\	はイギリス海峡の幅を泳ぎきった。   
\\	イギリス海峡の幅を泳ぎきるのは簡単なことではありません。  
\\	私は、1000ページもある本を、あまり面白かったので一晩で、読みきった。   
\\	彼は、山の中で道に迷って10日間も水だけで過ごしたので、発見されたときはすっかり体が弱りきっていた。  
\\	父は、毎日疲れきって会社から帰ってくる。  
\\	自分たちの持っている力を試合で出し切れば、きっと勝てますよ。  
\\	こんなにたくさん、食べきれません。少し残してもいいですか。  
\\	彼女は100メートルを走り切りました。  
\\	私は疲れ切っていたが、それにも拘らず眠れなかった。  
\\	彼は読書にひたり切っていた。   
\\	彼女の目はガラスのように澄み切っている。  
\\	人間はいつかは地球に残っている石油をすべて使い切ってしまうだろう。  
\\	ヤンキースは7回2点を入れ、そのまま5対3で逃げ切った。  
\\	〜に比べて	〜にくらべて	
\\	今年は去年に比べ、雨の量が多い。  
\\	ノーパンクタイヤはエアタイヤに比べて寿命が長く、理想的です。  
\\	この作品は君の前のに比べて進歩している。   
\\	この本はあの本に比べて、ためになるばかりでなくおもしろい。   
\\	それは以前に比べて格段に進歩した。   
\\	コンピューターに比べて、ワープロは一つの目的にしか使えない。   
\\	古い機種に比べてこちらのほうがずっと扱いやすい。   
\\	昨年に比べて今年は春の訪れが早かった。   
\\	私の収入は支出に比べて、500ポンド足りない。   
\\	私の足はあなたのに比べて小さい。  
\\	前に比べて、私は今よく忘れている  
\\	外国にて、日本はが高いと言われている。  
\\	諸外国にて、日本はが高いと言われている。  
\\	余り	あまり	
\\	あまり面白くない 
\\	あまり遅い 
\\	この話は本当にしてはあまりにもうますぎる。  
\\	あなたのいらっしゃるのがあまり遅いので、今電話をするところでした。   
\\	あなたはあまりにも素早すぎる。  
\\	あなたはあまり夜更かしをしない方がいい。  
\\	あの人はあまり先が長くない。  
\\	英語は余り難しすぎて私には分かりません。   
\\	驚きの余り声もでなかった。  
\\	彼女にあまり期待をかけすぎるな。  
\\	自分を余りまじめに考えすぎてはいけない。  
\\	「その女性は年をとっていますか」「いいえ、余り。」  
\\	私は水泳は余りうまくない。  
\\	余り遠くなかった。  
\\	余りそれは気に入らない。  
\\	彼は余り背が高くない。   
\\	彼は余り金持ちじゃない。   
\\	そこに着いたのは余り早くなかった。   
\\	余り寒くないですよ。   
\\	その案は余り感心しない。   
\\	雨の心配は余りない。   
\\	「本はどれくらい持っていますか」「余りありません。」   
\\	僕はおばさんを訪問する機会が余りない。  
\\	て形
\\	しまう (て仕舞う・て終う・て了う)	てしまう	
\\	その動作・作用が完了して、もとへ戻らない意を表す。特に、取り返しがつかない意を表す。	
\\	その映画を見て泣いてしまった。  
\\	宿題はもうやってしまいました。  
\\	いつも本を読んでいるうちに眠ってしまう。  
\\	彼は稼ぐ金をすべて本に使ってしまう。  
\\	そういった話をしていると議論になってしまう。  
\\	それをやってしまうまで、ここを出てはなりません。  
\\	「あら、ごめんなさい。どうやら間違え電話をしてしまったようだね。」  
\\	2日後に食べ物がなくなってしまった。  
\\	5分と走らないうちに息切れしてしまった。  
\\	前から欲しかった本がやっと手に入ったものだから、早速徹夜して終わりまで読んでしまった。  
\\	(赤毛)
\\	遅れてしまいました、すみません。   
\\	遅れちゃった、すみません。  
\\	春名は出来ないとドンドン振り付け変えちゃうからわけ分かんなくなちゃう.   
\\	やまない (止ま無い・已ま無い)	やまない	
\\	子供たちは親の愛を求めてやまない。  
\\	空模様からすると、雨はしばらくはやまないだろう。   
\\	ご健康を心から願ってやみません。  
\\	通じて	つうじて	
\\	仕事を通じて会いました  
\\	会話を通じて日本語覚えてます  
\\	友達を通じて、聞きました  
\\	カタログを通じて郵便で買い物をすることによって、人々は幅広い商品を選ぶことが出来る。   
\\	トムとマイクはお互いの友人を通じて知り合った。   
\\	貨幣は銀行制度を通じて流通する。   
\\	教育は若い時代に限られてはならず、われわれの全生涯を通じて継続して行われるものでなければならない。   
\\	言葉を通じて多くの言葉が習得される。   
\\	知識を通じてのみ、われわれは、一つの民族として、われわれの社会を脅かす危険に対処することができるからである。   
\\	彼はその事柄に多かれ少なかれ通じているようだ。  
\\	あの地方は、1年を通して雨が多い。  
\\	彼は一生を通して日本との友好のために働いた.  
\\	今はインターネットを通して世界中の情報が手に入る。  
\\	トムとメリーはお互いの友人を通じて知り合った。   
\\	インターネットを通じて本を注文しました。   
\\	〜からして	〜からして	
\\	田舎は天気からして違う。  
\\	向き	むき	
\\	この小説を子供向きに書き改めるのは難しい。  
\\	こういう本は若い読者向きではない。   
\\	このとおりやさしい英語で書かれているので、その本は初心者向きである。   
\\	このテキストは初心者向きにできている。   
\\	この派手な服は私には向きません。   
\\	この物語は成人向きで子供が楽しむものではない。   
\\	ひとつの雑誌は男の子向きで、また別のものは女の子向きである。   
\\	やさしい英語で書いてあるので、この本は初心者向きだ。  
\\	向けの一番太いパイプ 
\\	このスキー場は初心者向きです。  
\\	この別荘は夏向きにできているので、冬は寒いです。  
\\	これは体力が必要なので、どちらかというと若い人向きの仕事です。  
\\	させる	させる	
\\	田中さんは娘を大学へ行かせた  
\\	そんなことさせるなんて、あんたも強情だね。  
\\	あなたの息子さんを親離れさせるべきです。  
\\	あなたに心配させるつもりはない。  
\\	どうしても私の言葉を彼に信じさせる事ができなかった。  
\\	サリーは弟にバスルームを掃除させた。  
\\	私は彼に計画を変更させた。  
\\	私は彼に戸を開けさせた。  
\\	先生は彼を帰宅させた。  
\\	未だに	いまだに	
\\	予想や期待とは違って、現在もその状態が続いているさま。今になってもまだ。今なお。	
\\	御爺さんは未だに古いペンを愛用している。  
\\	彼女は未だに帰宅していない。  
\\	その祖父が買った時計は未だに調子がいいです。  
\\	その文化人類学者は「その州では未だに古い習慣が根強い。」だと言いました。  
\\	その傷は未だに治らない。  
\\	〜ものだから	〜ものだから	
\\	家が狭いものですから大きい家具は置けません  
\\	一人っ子なものだからわがままに育ててしまいました。  
\\	遅くなってごめんね。道路が込んでいたもんだから。  
\\	上着を脱いでもいいですか。暑いものですから。   
\\	僕は何でも嗅げれない。どうやら風邪を引っちゃったよ。ものだから。  
\\	ものだから 
\\	~出す	〜だす	
\\	地震が収まると、みんな家を飛び出した  
\\	泥棒が警察を見て、突然走り出しました。  
\\	彼女は笑い出した。  
\\	気圧計は昨夜下がり出したが朝には止まった。  
\\	朝から雨が降り出した。  
\\	先週から疲れを感じ出した。  
\\	からすると	からすると	
\\	部長のあの表情からすると、この企画はあんまり進めたくないようだ。  
\\	日本人一般の考え方からすると彼の態度は非常識だと言われるかもしれない。  
\\	彼のメールから判断すると馬鹿な人です。   
\\	この空模様からすると、雨になりそうだ。   
\\	この見地からすると、君の言うことは正しい。   
\\	みんなの話からすると、それは真実だろう。   
\\	空模様からすると、雨はしばらくはやまないだろう。   
\\	空模様からすると、今夜あたり雪になるかもしれない。   
\\	私の美的観点からすると。  
\\	周りの事情からすると彼女の話は本当らしかった。   
\\	買い手の立場からするとこれらのCDプレーヤーの値段は高すぎる。   
\\	アメリカ人からすると、日本人の働きすぎは理解できません。  
\\	(赤毛)
\\	話し方からすると、かれは、東京の人ではないようだ。  
\\	(赤毛)
\\	さっきの態度からすると、あの人はあやまる気は全然なさそうだ。  
\\	(赤毛)
\\	たい	たい	
\\	ボンドの映画を見たいな〜。  
\\	この車を買いたいんですが、お金が無い。  
\\	あなたは何をしたいのですか。  
\\	彼女は私のすることをいつみ知りたがる。  
\\	条件形	えば	
\\	行けばわかる。  
\\	この本をもう一度読めば、5回読んだことになります。  
\\	走れば、すぐに彼に追いつくだろう。  
\\	しばらく眠れば、ずっと気分がよくなりますよ。  
\\	このクラブに入会すれば、その施設の全てを使用する権利が与えられる。  
\\	すぐに出発すれば電車に間に合うだろう。  
\\	そのとおりにすれば間違いない。  
\\	その本は、安ければ買うかもしれない。  
\\	この薬を飲めばよくなります。(このくすりをのめばよくなります。)えば  
\\	練習すればもっと上手になります。  
\\	わけではない	わけではない	
\\	失敗というわけではない  
\\	全然見込みがないわけではないが、あまり期待はしない方がいいでしょう  
\\	あいつは馬鹿というわけではないがそれにしても、やってることはどうかなあ。  
\\	世界中の人が同じ言語を話すわけではないので、現在、翻訳に多くの努力が費やされている。  
\\	あの男しかいないわけではないよ。  
\\	あれは全く誤りというわけではない。  
\\	今日はとりわけ暑いというわけではない。  
\\	この規則はあらゆる場合に適用されるわけではない。  
\\	これらの卵が全部新鮮というわけではない。  
\\	すべての人が若い頃の夢を実現できるわけではない。  
\\	知り合いになれて嬉しくないってわけじゃないんだよ。   
\\	人間は働くために生まれてきたわけではない。  
\\	退院しましたが、病気が治って退院したわけではないんです。   
\\	梅雨には雨が降るけれど、ザーザー強く降るわけではありません。  
\\	すべてが私の責任というわけではありません。   
\\	彼女は必ずしもすべての質問に答えたわけではない。   
\\	金がすべてというわけではない。   
\\	その会は特に出席がよいわけではなかった。   
\\	これ以上〜ば	これいじょう〜ば	
\\	これ以上雨が降れば、川が氾濫してしまう  
\\	皆さんからお褒めの言葉を頂きこれ以上の光栄はございません。  
\\	これ以上待つのは時間の浪費だ。  
\\	これ以上何を失えば 心は許されるの  
\\	せめて	せめて	
\\	せめてなべでもあれば、ご飯が炊けるんですが  
\\	せめてつまらないものですが、、  
\\	せめて両親には分かって貰いたいんですが  
\\	じゃ、せめてテーブルの準備でもしましょう。  
\\	せめて後十分待ってくれませんか。  
\\	せめて謝ってくれてもいいのに。  
\\	せめて日常会話ができるくらいになりたい。  
\\	せめて病気のときぐらいは、お酒を飲まないぐらいの分別は持つべきだ。  
\\	せめてもう一日あれば  
\\	せめてもう一度だけ、笑顔を見せて。  
\\	(マリ)
\\	〜ね	〜ね	
\\	これはいいですね。
\\	ええと、映画館は駅の近くにありますね。
\\	突然寒くなったね。  
\\	「彼女は音楽が好きだね。」「そうだね。」  
\\	9時でよろしいですね。  
\\	あなたのお母さんは、とても若いですね。  
\\	あなたは一度も沖縄に行ったことがありませんね。  
\\	あなたはおもしろい場所をたくさん知っていますね。  
\\	あなたはその計画に賛成ですね。  
\\	あなたはタバコを吸わないんですね。  
\\	あなたは泳げますね。  
\\	あなたは昨日家にいましたね。  
\\	「一杯飲もうよ、ね。」「うん、そうしよう。」  
\\	〜か	〜か	
\\	私が書いたかを問わず、(全て)「作者:あまつか」です。  
\\	おまえは・・・・俺が誰か知ってるのか。  
\\	今度の出張はいつかまだ分かりません。  
\\	誰が一緒に出掛けたか教えてください。 
\\	書類どこにあるか(と)聞きました。   
\\	飛行機がいつ到着するかわからない。  
\\	友達にどんなプレゼントがいいか尋ねた。  
\\	~としたところで, ~としたって, ~にしたところで	~としたところで, ~としたって, ~にしたところで	
\\	としても	
\\	全員が参加するとしたところで、せいぜい20人位だ.  
\\	彼は日本語を20年も勉強しているが、その彼にしたところで、まだ分からない文法に時々出くわすそうだ。  
\\	三島由紀夫は産湯に浸かった記憶があると書いているが、彼にしたところで、死ぬまでのすべての記憶がある訳ではない。  
\\	彼がどんなに歌が得意としたって、素人の域を出ていないよ。  
\\	わたしにしたって、その問題については、どうしたりいかわからにんです。  
\\	〜もの	〜もの	
\\	電話はあしたにしたほうがいいんじゃない、もう遅いもの。  
\\	しょうがないよ。子供なんだもん。   
\\	どうして食べないの? だって、まずいもの。  
\\	警察はあの人を抑えたよ!。毀れるもん。   
\\	と相まって	とあいまって	
\\	といっしょになって	
\\	多くの国々では、水や空気の汚染と相まって、環境破壊が進んでいる  
\\	実力と運とが相まって、彼を成功に導いた。  
\\	そのゲームはかわいいキャラクターと魅力的なストーリーがあいまって、多くのユーザーに支持されたヒット作品となった。  
\\	コーチの指導力と選手のやる気が相まって、優勝できた。  
\\	たまらない	たまらない	
\\	今日は暑くてたまらない.  
\\	「ほじくられてはたまりません」と柿のたねは言いました。  
\\	このにおいはたまらなく嫌だ。  
\\	あのカメラがほしくてたまりません。  
\\	あまりの暑さに一休みしたくてたまらない。   
\\	彼女は車が欲しくてたまらなかった。   
\\	私たちは行きたくてたまらないのだ。   
\\	〜に越したことは無いが	〜にこしたことはないが	
\\	常識から言うと〜の方がいい	
\\	用心に越したことはないんですが
\\	この仕事は経験がなくてもかまわないが、もちろんある に越したことは無い。  
\\	明日、雨が降らないと言っても、傘を持つに越したことは無い。  
\\	病気の時は寝るに越したことはない。  
\\	日本語が話せるようになりたかったら、日本へ行くに越したことはないと思うけれど、まずアメリカでちゃんと勉強してから行った方がいいと思います。  
\\	〜だ	〜だ	
\\	犬だ!
\\	イギリス人です。
\\	それはなんですか 
\\	あの人は誰でしたか。 
\\	田中さんではありませんでした。
\\	猫だった。
\\	犬じゃない。
\\	きつねじゃなかった?
\\	まだプロじゃないです。
\\	イギリス人でロンドンに住んでます。
\\	田中さんがこの問題は難しすぎるだと言った。  
\\	この本は新しいだ。  
\\	の極み	のきわみ	
\\	遺憾の極みである   
\\	光栄の極みである   
\\	混乱の極みである   
\\	混乱の極みにある理由を説明する   
\\	興奮の極みに   
\\	ぜいたくの極み   
\\	混乱の極み   
\\	めく	めく	
\\	社長の皮肉めいた言い方を、社員は不快に感じている  
\\	きらめく太陽に青い海、考えだけでも胸がときめくわ。  
\\	俺たちは十年来の友人じゃないか、そんな他人めいた言い方をするなよ。  
\\	依頼	いらい	
\\	頼むこと	無し	
\\	引き続き、ご指導、ご鞭撻の程をどうぞ宜しくお願いもうし上げます。  
\\	さながら	さながら	
\\	夜のトイレを恐れる駄々っ子さながらの、言い訳にもならないセリフでお茶を濁した。  
\\	さまざまなトラブルを想定した本番さながらの訓練  
\\	本場ニューヨークさながらのジャズナイト  
\\	石器時代さながらの着火方法  
\\	さながらの
\\	な雰囲気   
\\	テストさながらの出題形式   
\\	本番さながらのシミュレーション   
\\	実際の試験さながらの模擬テスト   
\\	強盗団は近くの家から110メートルに及ぶトンネルを掘って侵入した。映画さながらの強盗だ。  
\\	〜て	〜て	
\\	この赤くて大きい車は田中さんのです。 
\\	本屋に行って買います。
\\	茶碗を口のところまで持ってきて、箸でごはんをかきこむのは行儀が悪い。  
\\	メロンを食べて、コーヒーを飲ました。
\\	歩いて来ます。
\\	妹は遊そんでいます。
\\	社長はたいへん怒っています。
\\	喜んで手伝います。
\\	手紙をもって行きました。
\\	田中さんは出掛けています。
\\	た形ところで	たところで	
\\	〜してもよい結果にはならない; 期待できない	
\\	友人に相談したところで、同じだろう。  
\\	喧嘩したところでどうにもならないだろ。  
\\	癌(がん)が肺まで広がっている。手術したところで、助からない。  
\\	タクシーで行ったところで、もう間に合わない.  
\\	君がいくら話をしたところで私を納得させることはできないよ。  
\\	いくら話し合ったところで、この問題を解決することはできません。  
\\	(赤毛)
\\	もともと勉強する気がないのなら、大学を受けてみたところで何の意味があるのか。  
\\	(赤毛)
\\	どんなに一生懸命働いたところで、生活は楽にならないだろうと思う。  
\\	(赤毛)
\\	まさかの〜	まさかの〜	
\\	まさかの場合、ここに電話してください。  
\\	(誠)
\\	まさかの時、すぐお知らせ下さい。  
\\	(誠)
\\	まさかのときには貯蓄に頼ればよい。  
\\	まさかのときに備えて貯蓄せよ。  
\\	まさかのときの友こそ真の友。  
\\	まさかの時にはあなたを助けてやろう。  
\\	からすると	からすると	
\\	メールを受け取った方からすると、何のわるぎもなくついうっかり転送してしまったということでも、発信者からすると自分が丹精こめて書いたものを、無断で第三者に転送されてしまったということにもなりかねません。  
\\	知っている側からすると  
\\	この見地からすると、君の言うことは正しい。  
\\	空模様では、からすると、雨はしばらくは止まないだろう。  
\\	空模様からすると、今夜あたり雪になるかもしれない。  
\\	私の美的観点からすると,これは醜いです  
\\	周りの事情からすると彼女の話は本当らしかった。  
\\	買い手の立場からするとこれらの
\\	プレーヤーの値段は、高すぎる。  
\\	彼の案は私の見方からすると、とてもいいようにみえる。  
\\	法律の観点からすると、彼は自由だ。  
\\	〜を	〜を	
\\	出発点	
\\	部屋を出た。  
\\	大阪を出発する。  
\\	去年日本を離れた。  
\\	1900年に彼はイギリスを去った。  
\\	のだ	のだ	
\\	どうしたんですか。顔色が悪いですよ。   
\\	どうしたのですか。頭が痛いのです。   
\\	〜にしては	〜にしては	
\\	年寄りにしては若く見える。  
\\	お金持ちにしては古い車に乗っている。  
\\	6月にしては寒い朝だ。   
\\	70歳にしては、彼はいまだに驚くほど元気である。   
\\	このホテルはこの町にしては比較的高い。   
\\	こんな暖かい陽気は2月にしては異常だ。   
\\	彼は、タクシーの運転手にしては道を知らない。  
\\	そんなに乱暴ふるまうなんて、クリスにしては普通ではありませんでした。   
\\	ほかの人の助けを当てにしてはいけない。   
\\	初めてケーキを作ったにしては、上手にできましたね。  
\\	ビリーは彼の年にしてはとても背が高い。   
\\	外見はあまり当てにしてはいけない。   
\\	この子は小学生にしてはずいぶんしっかりしている。  
\\	駆け出しの新聞記者にしては腕がいい。   
\\	彼は20代にしては、ふけてるな。  
\\	君は物事を中途半端にしてはいけないよ。   
\\	言うまでもないことだがローマは1日にしては成らず。   
\\	今ごろにしては暖かい。   
\\	今日はこれでおしまいにしてはどうでしょう。   
\\	中古車にしては、値段が幾分高い。  
\\	外国人にしては日本語が上手だ。   
\\	洗ったの? それにしてはきれいじゃないね。  
\\	て形
\\	はいけない (ては行け無い)	てはいけない	
\\	美術館で写真を撮ってはいけない。/
\\	美術館で写真を撮っては行けない。  
\\	(赤毛)
\\	どうぶつをいじめてはいけません。  
\\	(赤毛)
\\	ここでたばこを吸うってはいけません。禁煙ですから。  
\\	(赤毛)
\\	「先生、ここで遊んでもいいですか。」 「いいえ、[遊んでは]いけません。」  
\\	(赤毛)
\\	その顧客に応対してこれ以上時間を無駄にしてはいけない。  
\\	ほかの人の助けを当てにしてはいけない。  
\\	ドアを開けっ放しにしてはいけません。  
\\	外見はあまり当てにしてはいけない。  
\\	緊急の場合以外は私の援助をあてにしてはいけません。  
\\	君は失敗を不運のせいにしてはいけない。  
\\	君は物事を中途半端にしてはいけないよ。  
\\	お酒を飲んではいけません。  
\\	(赤毛)
\\	ここに自転車を止めてはいけない。  
\\	(赤毛)
\\	「あなたは塩辛いものを食べてはいけないよ」と言ったけど、彼は食べ物にたくさんの塩を加えた。   
\\	くらい	くらい	
\\	人に会ったら、あいさつぐらいして欲しい  
\\	子どもでもそれくらいのことはできる  
\\	駄目だったら!こっちが手伝ってもらいたいくらい。  
\\	どのくらいやったら認めてくれるの。  
\\	あなたが駅に着くのにどのくらいかかりますか。  
\\	「ここのバスは1時間に何回くらい出るのですか」「30分ごとに出ます」   
\\	ウィーンまでは歩いてどのくらいかかりますか。  
\\	「バスは後どのくらいで来ますか」「5分で来ます」  
\\	「会合はどのくらい続きそうですか」「2時間くらいです。」  
\\	100ドルでどのくらいの間やっていけますか。  
\\	1年にどのくらい頻繁にスキューバダイビングに行きますか。  
\\	1泊50ドルくらいのダブルの部屋を一室、今晩から4泊お願いします。  
\\	2月27日の3時から1時間くらい会うことにしませんか。  
\\	2番目の文も丁度同じくらいの長さだった。  
\\	50メートルくらい行くと、郵便局があります。  
\\	彼女は40歳ぐらいだ。   
\\	私のとった魚はこれくらいの大きさだった。   
\\	昼くらい雨がやんだ。   
\\	あの橋はどれくらいの長さですか   
\\	そのビールは建ってからどのくらい経っていますか   
\\	その小包はどれくらいの重さですか   
\\	このびんに油がどれくらい入るだろうか   
\\	彼は私と同じくらい金を持っている。   
\\	先生は私をしかるのと同じくらいよくテリーをしかる。   
\\	彼は速く泳ぐが、私もまったく同じくらい速く泳げる。   
\\	彼は英語を話すが、彼の妹も同じくらいうまく話す。   
\\	きっかけ	きっかけ	
\\	イタリアンルネッサンスを開花させるきっかけを作ったのはジョットの功績だ。   
\\	私の意見がそのグループでの議論のきっかけになった。   
\\	事のきっかけとなったのは誰だ。   
\\	事のきっかけを作ったのは誰だ。   
\\	彼女と付き合うきっかけになったのは、なんとなくフィーリングが合ったからだ。   
\\	彼等は彼に逃げるきっかけを与えた。   
\\	開発のきっかけ   
\\	恋人募集広告がきっかけで(人)と出会う   
\\	どういうきっかけでそんな考えを思い付いたの?   
\\	何がきっかけでギターを始められたのですか?   
\\	がきっかけで起こった激しい論争   
\\	彼の支持率は金融不祥事がきっかけで急に落ちている   
\\	仕事がきっかけで知り合いになる   
\\	スキャンダルがきっかけで浮かび上がる   
\\	ほんのちょっとしたきっかけで変わる   
\\	どのようなきっかけで野球に興味を持ったのですか?   
\\	(人)が_年以上勤務した会社を(主語)がきっかけで離れることになる  
\\	キャリアの成功へのきっかけとしてプログラムを活用する   
\\	ジャズにのめり込むきっかけとなったアルバム   
\\	人種問題がきっかけとなった暴動。  
\\	近ごろ国際交流がますますさかんになっている。外国文化の情報が増える一方だが、それは自分の国のことについて考えるきっかけにもなっている。  
\\	(赤毛)
\\	インフレが失業を生むきっかけとなった。  
\\	何がきっかけだったが忘れたが。。。   
\\	英会話を始めたきっかけは何ですか   
\\	ならでは	ならでは	
\\	あの人ならではこのような文章を書けなかっただろう。  
\\	あの人ならではとてもこの仕事はできない。  
\\	こんな美味しい料理は母親ならではのものです。  
\\	その辞書ならではの特徴   
\\	この季節ならではの感じなのかもしれない。   
\\	地方ならではの魅力的な仕事   
\\	うるさいマスターならではの料理は、どれも新鮮で、うまいっ   
\\	限定・入手困難】の入荷に強い当社ならではの商品力とプライスに必ずご満足頂けると思います。   
\\	こんな新しい発想はあのデザイナーならではのものです。  
\\	この大学の法律図書館には、この大学ならではの充実した資料が揃っている。  
\\	にする	にする	
\\	授業を面白くします。  
\\	を 3日でものにする!!  
\\	日本語を自家薬籠中のものにする  
\\	にする	にする	
\\	うちは毎年9月末の日曜日にすると決まっているのです。  
\\	そのようにすると後で困らない?  
\\	ないばかりか	ないばかりか	
\\	ないだけでなく	
\\	彼は会社へ来ても仕事をしないばかりか女性社員としゃべってばかりいる。  
\\	彼女は歌をよく歌えないばかりかピアノもよく弾けます。  
\\	〜おうとすると	〜おうとすると	
\\	バスを降りようとすると、後ろから押された。   
\\	出かけようとすると、父に呼ばれた。  
\\	お風呂に入ろうとしたときに、電話が鳴りました。  
\\	一息	ひといき	
\\	我慢することです	
\\	あと一息だ。   
\\	もう一息というところです。   
\\	もう一息努力すれば成功するだろう。 
\\	一息いれようよ。 
\\	講師は一息入れてから、また話し続けた。 
\\	座って、一息いれましょう。 
\\	彼女は新鮮な空気を一息吸いに浮ヨ出た。 
\\	ついでに・を契機に・をきっかけに	ついでに・をけいきに・をきっかけに	
\\	用事で都心へ出たついでに、美術館をのぞいてみた。  
\\	(コピーをしている人に)すみません、ついでにこれもお願いします。  
\\	大学入学をきっかけに、一人暮らしを始めた。  
\\	病気で 入院したのをきっかけに、タバコをやめることにした。  
\\	(赤毛)
\\	松川先生との出会いをきっかけに、私の人生は変わりました。  
\\	(赤毛)
\\	旅行をきっかけにして木村さんと親しくなりました。  
\\	〜まじき	〜まじき	
\\	許すまじきことであれ。  
\\	すまじきものは宮仕え。  
\\	許すまじき暴言を吐いた。  
\\	に応えて	にこたえて	
\\	わが社は消費者のニーズに応えて新製品を開発しました。  
\\	社員の要求にこたえて労働時間を短縮した。  
\\	「医者になってほしい。」という親の期待に応えて、彼は医者になった。   
\\	学生の希望 にこたえて,図書館は夜10時まで開けられることになった。  
\\	なる	なる	
\\	な 
\\	悪魔め!俺様は聖なる剣で貴様を倒す!  
\\	スター・ウォーズのエピソード4の副題は、「新たなる希望」である。  
\\	〜ては	〜て は	
\\	死んじゃ、ダメ!  
\\	落としてはまずい。  
\\	じゃ、また来週!  
\\	ことにする	ことにする	
\\	今年日本へ行くことにしました。	
\\	最近ちょっと太ったから毎日ハンバーガーを食べないことにしました。  
\\	合格するようにもっと勉強をすることにしました。  
\\	ほどだ; ほどの	ほどだ; ほどの	
\\	くらい; くらいの	
\\	山から流れてくる川の水は冷たくて、手を入れると痛いほどだった。  
\\	ラッシュの電車はとても込んでいますよ。ドアが閉まらないほどです。  
\\	その子は大人もびっくりするほどの作品を仕上げた。  
\\	そのロープは長さが5メートルほどあります。  
\\	旅は5時間ほどかかるでしょう。  
\\	彼は200万円ほども手に入れた。  
\\	彼は歩けないほど弱ってはいなかった。  
\\	唯でさえ	ただでさえ	
\\	普通の状態でも、それでなくてもいい。もうこれだけで十分だって意味	
\\	もうこの近所はただでさえうるさいところが、今週は近くで工事が始まったため、夜は一睡も出来なくなった。  
\\	もうただでさえ迷惑をかけているのに〜  
\\	これは ただでさえおいしいものが、さらに我が隠し味を入れたら美味しくてたまらなくなるよ。  
\\	今日はただでさえ忙しいのに、母に買い物に行かされた  
\\	〜ようがない	〜ようがない	
\\	スプーンがないことに気がついた。これでは食べようがない。  
\\	道は通行止めだし。これではそこへ行きようがない。  
\\	その会社はまた赤字になった。救いようがない。   
\\	ばかばかしいとしか言いようがない。   
\\	何とも言いようがないなあ。   
\\	私はあなたにお礼の申しようがない。  
\\	誕生と死との間を享楽する以外には、この両者に対して回復の手の施しようがない。   
\\	彼の発言は他に解釈のしようがない。   
\\	質問の意味はわからなくて、答えようがなかった。  
\\	このカメラ、ちょっと直しようがないですね。  
\\	(赤毛)
\\	時間がぎりぎりで、練成しようがない。  
\\	せいにする	せいにする	
\\	ナチスは、この世の全ての悪をユダヤ人のせいにした。   
\\	他の人のせいにしないでくれない?  
\\	彼は貧乏なのを不運のせいにする。   
\\	彼女は失敗を私のせいにした。   
\\	上で	じょうで	
\\	じょうで、 
\\	が立ち上がりません。.   
\\	いろいろな動物が地球上から姿を消した。  
\\	空気が無ければ、我々は地球上で生きられないだろう。  
\\	〜よ	〜よ	
\\	バスが来るよ。
\\	それは大変なことだよ。  
\\	あなたが好きなんだよ  
\\	好い加減にしなさいよ。  
\\	によれば	によれば	
\\	さっきの地震はテレビの速報によれば震度3だそうだ。   
\\	天気よほうによれば、明日は晴れるらしい。   
\\	つつ	つつ	
\\	コマーシャルのマーケティングの仕事をやりつつ、音楽の仕事もやりました。  
\\	食べつつパソコンも使う  
\\	道を歩きつつ本を読む。  
\\	事態は、改善されつつある。  
\\	ミルクパンをほおばりつつ、チョコパイにも手を伸ばす。  
\\	昨夜、久しぶりに会った友達と酒を飲みつつ語り合った。   
\\	目の前の美しい山を眺めつつ、自分に絵の才能のないことを残念に思った。  
\\	このレポートでは、調査の結果を分析しつつ、今後の問題点を明らかにしてみたいと思う。   
\\	運動不足は体に悪いと知れつつ、時間がなくて運動ができない。   
\\	もらった手紙に返事を書かなければと思いつつ、もう10日も経ってしまった。  
\\	試験の準備を早くはじめようと思いつつも、まだはじめていない。  
\\	今夜漢字を勉強しようと思いつつ、テレビゲームをやってしまいました。   
\\	もう起き なければと思いつつも, なかなか起きられない.  
\\	その病気は蔓延しつつあるそうだ。   
\\	地球は温室効果の増大によりしだいに暖かくなりつつある。   
\\	これ以上はやめておくべきだと思いつつ、彼はビールをもう一杯注文しました。   
\\	立ち止まりつつ涼みます。  
\\	先生に臨みたいと思いつつ、先生を避けました。  
\\	〜と	〜と	
\\	「ロミオとジュリエット」を見ると必ず涙ぐむ。  
\\	家を出るとすぐに雨が降り始めた。  
\\	ここだと2万円前後の テレビ が 買うことができる。  
\\	冬になると雨が降っています。  
\\	冬になると雨が降っています 
\\	立川駅に着くと電話をしなくてはいけない  
\\	立川駅に着くと電話をしなくてはいけない 
\\	雨が降ると、送別会パーティに行けません。  
\\	3級の能力試験を合格できると、来年2級の試験を参加してみる嵐閧ナす。  
\\	〜そびれる	〜そびれる	
\\	ある活動のすることを失敗する	
\\	借りたお金を返しそびれた。  
\\	飲みすぎて終電に乗りそびれた。  
\\	〜やら〜やら	〜やら〜やら	
\\	帰国する前は飛行機のよやくをするやらおみやげを買うやらで忙しい。  
\\	宿題やらレポートやらで、のんびりする暇もない。  
\\	彼女には、宝石やらバックやら、たくさんのプレゼントを買いましたよ。  
\\	(赤毛)
\\	ポケットにはハンカチやらガムやらが入っている。  
\\	(赤毛)
\\	料理やら洗濯やら掃除で彼女の母は大変忙しかった。   
\\	暑いやらむしむしするやらで熟睡出来なかった  
\\	〜の	〜の	
\\	フレッド君の車はどこにありますか。
\\	フレッド君のですか。
\\	「これは誰のペン?」「私の。」
\\	リリーの靴はどこにあるか知りません。  
\\	といったら	といったら	
\\	息子の部屋の汚いことといったら、ひどいものです。   
\\	これを見たときの私の驚きといったら。  
\\	メアリーは彼に何といったら良いのか途方に暮れている。   
\\	私にできることといったら彼女に忠告してやるくらいのことです。   
\\	私には彼に何といったら良いか分からなかった。   
\\	私はなんといったらいいかわからなくてだまっていた。   
\\	私は彼に何といったら良いか分からなかった。   
\\	私は母に何といったらいいのか困ってしまった。   
\\	彼の食べ方といったら、胸が悪くなるね。   
\\	御	ご、お、み、おん	
\\	尊敬を示す語	
\\	ご主人はお元気ですか。
\\	あなたはどちらのご出身ですか。
\\	あなたは御両親のどちらに似ていると思いますか。  
\\	ご家族はいかがお過ごしですか。  
\\	神の御名  
\\	「体」と言う言葉は尊敬語では「御身・おんみ」です。  
\\	まして〜はずがない	まして〜はずがない	
\\	大人でさえ知らない事は、 まして子供は知っているはずがない。  
\\	(誠)
\\	大人でさえ知らないし、まして子供は知っているはずがない。  
\\	(誠)
\\	出す、出して、出した	だす、だして、だした	
\\	まど から くび を だす。  
\\	わたし の かいしゃ は ほん を 出した。  
\\	ごみ を 出す。  
\\	どころか	どころか	
\\	食べるどころか  
\\	私は彼があまり好きでなかった、それどころか憎んでさえいた。  
\\	彼の奥さんは、やさしいどころか鬼のような人だ。  
\\	「彼女は、まだ若いですね」 「いえ、若いどころか、もう孫が三人いるんですよ」  
\\	間違いを指摘すると、感謝されるどころか、怒られた。  
\\	ダイエットしていたのに、痩せるどころか、太ってしまいました。  
\\	「おわったの」「それどころかまだ始めていないよ」   
\\	「終わったの」「それどころかまだ始めてもいないよ」   
\\	1ドルどころか、1セントも持っていない。   
\\	の低さの原因を人種に求めるという議論を論駁するどころか、リンのデータはそれを事実上補強することになっている。   
\\	あなたは、私はすることが何もないと思っているが、それどころかたいへん忙しいのです。   
\\	あなたは仕事が気に入ってると思いますが。それどころか嫌いだ。   
\\	この辞書は初級者に、いやそれどころか上級者にも役に立つ。   
\\	その絵は醜いとは思わない。それどころか、むしろ美しいと思う。   
\\	その手紙を読むどころか、彼は封も切らなかった。   
\\	それは、失敗どころか、大成功だった。   
\\	それもう終わった。それどころか、いまからだよ。   
\\	ためになるどころか、その雨は収穫にひどい害を与えた。   
\\	ちゅうちょするどころか彼女は喜んで私を援助してくれると言った。   
\\	ジャックが音楽が苦手なんてとんでもない、それどころか、ピアノを弾くのがうまい。   
\\	トムは怠け者の少年ではありません。それどころかよく働きます。   
\\	メアリーは貧しくはない。それどころか、かなり裕福だ。   
\\	雨はやむどころかいよいよひどくなった。   
\\	我々の交渉は失敗どころか、大成功だった。   
\\	休養するどころかかえっていつもより仕事に励んだ。   
\\	今度の上司?個性的どころか、ありがちなタイプね。   
\\	彼は料理どころか音楽もスポーツもできる。  
\\	スミスさんは日本語の新聞が読めるどころか、平仮名も知らない。   
\\	ジョンャ唐ウんは漢字が書けないどころか、平仮名も書けない。   
\\	スミスさんは日本語が書けないどころか、日本語で小説が書けるぐらいだ。   
\\	クラークさんは刺し身が食べられるどころか、納豆まで食べられる。  
\\	けさの天気藍
\\	ナは午後から晴れると言っていた。ところが、藍
\\	ノ反して雨が降り続いている。  
\\	警察の必死の捜索にも関わらず、生存者 どころか、死体すら発見されていない。  
\\	患者の体は立ち上げるどころか、座ることもできません。  
\\	〜も	〜も	
\\	コーヒーも紅茶も飲みません。
\\	これも会社の車じゃありません。
\\	「ぼくの父は酒を飲みません。」「私の父もなんです。」  
\\	あなたがスキーに行かないのなら、私も行きません。  
\\	がたい	がたい	
\\	その景色の美しさは筆舌に尽くしがたい。  
\\	抑え難い衝動  
\\	人生にはおうおうにして即決し難い事柄に出くわすことがあります。  
\\	この勝負は白黒つけがたい。   
\\	彼女は,とてもえらくなってしまって、近寄りがたい。   
\\	彼が宝くじで1億円当てたなんて、すぐには信じがたい話だ。  
\\	これは正解ではないが、間違っているとも言いがたい。   
\\	日本に来て色々と得がたい経験をした。   
\\	あんな気の弱そうな人が ""人殺し”なんて信じがたい。   
\\	親が子を、若者が老人を虐待するような犯罪は許しがたい。  
\\	彼女が1等をとったなんて信じがたい。  
\\	事無く	ことなく	
\\	ライセンスを失うことなく保護されたファイルを別のコンピュータに移動するには  
\\	ためらうことなく、その陰謀に対する徹底した対抗処置をとった。  
\\	雨は途切れることなく一日中降り続いた。  
\\	学部長は、ためらうことなく永遠の英知を選んだ。  
\\	警察が墜落事故の原因を昼夜休むことなく調査している。  
\\	私は気づかれることなく彼の後ろを通り過ぎた。  
\\	彼女は止めることもなく何時間も泣き続けた。  
\\	人は休んでも、工場の機械は365日休むことなく働いている。  
\\	犯人は、だれにも気づかれることなく、その場所を離れた。  
\\	贅沢などすることなく、貯金をしています。  
\\	(赤毛)
\\	山川さんは二十年間休むことなく会社に通った。  
\\	(赤毛)
\\	彼女は思うことなくコーヒーショップを出て行きました。でも、払って忘れてしまいました。  
\\	私は理由がある事無く、彼女が大好きです。  
\\	ロポットは24 時間休むことなく働いている。  
\\	あの子はあきることなく、
\\	に熱中している。  
\\	完全な糧食を持っていく事無く、登山します。   
\\	ところで (所で・処で)	ところで	
\\	それはそうと。話題を変えるときに使う語。	
\\	雨が降ってきたなあところで飯にしないか  
\\	「ところで」の他の意味 
\\	私が説得したところであいつは耳をかさないだろう.  
\\	ところで、あなたの仕事は何ですか。
\\	ところで、あなたは宿題をしましたか。  
\\	ところで、ちょっと君に話がある。
\\	ところで、あなたは今晩暇ですか。
\\	まいか	まいか	
\\	先生にこの話を君からしてよろしく相談してくれまいか  
\\	だらけ	だらけ	
\\	この本は間違いだらけだ。   
\\	この公園は、犬だらけだ。  
\\	うちの庭は雑草だらけだった。  
\\	うわぁ、観光客だらけ。  
\\	この制度は欠陥だらけだ。  
\\	石けんだらけの手で私に触らないで。  
\\	そのテーブルは一面にほこりだらけだった。  
\\	その部屋は紙くずだらけだった。  
\\	それは矛盾だらけだ。  
\\	ラグビーの試合の後、全身打ち身だらけだった。  
\\	今は虫歯だらけだ。  
\\	彼女の顔はにきびだらけだった。  
\\	妻の趣味で、僕の服は、ブランド物だらけになってしまいました。  
\\	(赤毛)
\\	昨日からの雨がようやくやんだが、運動場はまだぬれていた。試合を終えたサッカー三種の顔はみんな泥だらけだ。  
\\	(赤毛)
\\	私には、世界のあちこちに友達がいて、絵葉書を送ってくれる。そういう絵葉書を壁にはり始めたところ、それが、いつのまにかものすごい数になった。今や壁という壁が絵葉書だらけになってしまった。  
\\	(赤毛)
\\	この紙は字だらけです。  
\\	彼女の生涯は悲しいことだらけだった。  
\\	彼女は不平だらけだ。  
\\	彼の手は泥だらけだった。  
\\	せいぜい (精精)	せいぜい	
\\	できるだけ。精一杯。最大限。	
\\	今度は徹夜になりそうだ。今のうちにせいぜい休んでおこう。  
\\	この仕事だったらせいぜい三日もあればできる。  
\\	この仕事の完成にせいぜい努力します。  
\\	あの婦人はせいぜい40歳だ。  
\\	ここから歩いてせいぜい10分のところです。  
\\	この時計はせいぜい10ドルだ。  
\\	この辞書はせいぜい2万語くらいしか載っていない。  
\\	せいぜい20人しかそのパーティーには来ないだろう。  
\\	せいぜい40
\\	の高校生しか大学に進学しない。  
\\	せいぜい自分の能力を活用しなさい。  
\\	それはせいぜい二流のホテルだ。  
\\	彼はせいぜい二流の学者だ。  
\\	致します	いたします	
\\	致すと言う言葉はするの意味と一緒です	
\\	あなたの申し出をお受け致します。  
\\	また何か他の募集がありましたらお知らせ致します。  
\\	少々お待ち下さい。おつなぎ致します。  
\\	この場合、未開封のメディアパッケージを速やかに
\\	社に返送して頂ければ、商品代金を全額払い戻し致します。  
\\	お手伝い致しましょうか  
\\	それで	それで	
\\	そして、その後;それだから、そういうわけで	
\\	朝寝坊した、それで仕事に遅れたんだ。  
\\	3時のバスで行こうよ。それでいいかい?  
\\	それで,どうなりましたか?  
\\	それでこのカードに記入してください。   
\\	それでこれ以上の義務を負わなくてよくなる。   
\\	それでしたら今日いらしてください。   
\\	それでそのうわさが本当である。   
\\	それでだいぶ手間が省ける。   
\\	それでは、あなたは来ないということですか。   
\\	それでは、またの機会ということに。   
\\	それでは、失礼いたします。   
\\	それでは、習慣は人間生活の偉大な道しるべである。   
\\	「それではおまえにねこを一匹とってきてあげよう」と、おじいさんはいいました。   
\\	それではそろそろ失礼します。   
\\	それではまるで話がちがう。   
\\	それではテキストの10ページを開いて下さい。   
\\	それでは音楽をお楽しみください。   
\\	それでは何も起こらないよ。   
\\	病気の母が泣いて止めたけれど、それでも留学したい気持ちは変わらなかった。  
\\	(赤毛)
\\	「それで彼方は6時にこの部屋にいましたね」 「そうです」   
\\	それで彼は私のうちへ来たのね。   
\\	彼女は寝過ごした、それで列車に乗り遅れたんだ。   
\\	たび、たびに	たび、たびに	
\\	にサインインするたびに、プログラムをアップグレードするように求められます。 
\\	あの日の出来事を思い出すたびに、涙を止められない。   
\\	〜に	〜に	
\\	その車に犬 が います 。
\\	10時に約束があります。  
\\	きっかり9時に来なさい。  
\\	10人の学生が同時に立ち上がった。  
\\	2時に空港に見送りに行きます。  
\\	彼は休暇でコロラドにいる。  
\\	だれか玄関にいる。  
\\	その町はロンドンのすぐ北にある。  
\\	その店は通りの左側にある。  
\\	アルプス山脈はヨーロッパの中央にある。  
\\	サンフランシスコはアメリカの西海岸にある。  
\\	テレビのリモコンがソファーの下にある。  
\\	バス停は学校の近くにある。  
\\	ローマはイタリアにある。  
\\	空は私たちの頭上にある。  
\\	午後6:00にビデオを作りました。  
\\	ありうる, ありえる (有り得る)	ありうる, ありえる	
\\	ある可能性がある。あるはずである。	
\\	日本においても充分ありえる話だと思います。  
\\	安全という状態はどこにもありえない。  
\\	冗談を言うほど賢いコンピューターがありえるだろうか。   
\\	彼女の成功はありえることだ。   
\\	一方だ	いっぽうだ	
\\	携帯電話を使う人の数は増える一方です  
\\	本屋で英語関係の本を見つけると、次々に買っていますが、読まずにいるので、溜まる一方です。  
\\	物価は上がる一方だ。  
\\	近ごろ国際交流がますますさかんになっている。外国文化の情報が増える一方だが、それは自分の国のことについて考えるきっかけにもなったいる.  
\\	アメリカに帰って以来、日本語が使えないので、忘れる一方で困っている。 アメリカに帰って以来、日本語が使えないので、忘れる一方で困っている。  
\\	彼の体重は増える一方だ。  
\\	〜なのに	〜なのに	
\\	夏なのに、寒い。	
\\	このコンピューター1年前に買ったばかりなのに、もうすっかり時代遅れだわ。   
\\	この情報は当然最新であるべきなのに、実際はそうでない。   
\\	これは新しい靴なのにもう修理する必要がある。   
\\	ずっと歩いて来たとは、あなたは足が丈夫なのにちがいない。   
\\	ジムは目覚めると、自分が部屋に1人なのに気がついた。   
\\	ビルはただモニカを慰めたかっただけなのに、彼女は彼が自分に気があるのだと判明した。   
\\	ベンチに腰かけたとたんに、ペンキが塗りたてなのに気づいた。   
\\	一日風邪で休んだだけなのに、机の上に書類が山積みになっている。   
\\	兄は全く不器用なのに、彼は優秀な外科医だ。   
\\	残念だわ。昼食の後でケーキを食べるつもりなのに。   
\\	彼って長野に住んでいるはずなのに、なんで東京駅にいる?   
\\	彼は、そこにいたら幸せなのに、と言った。   
\\	彼は38歳なのに両親に頼りきりです。   
\\	彼はそれを知っている、なのに少しも気にしていない。   
\\	彼はもうここに着いているはずなのに。   
\\	彼は金持ちなのに、幸福ではない。   
\\	彼は若い、それなのに分別がある。   
\\	彼を見るのさえ嫌なのに、デートするなんてとんでもないことです。   
\\	彼女の利口なのには舌を巻く。   
\\	不景気なのに依然物価は高い。   
\\	僕はその少年の利発なのに驚いた。   
\\	というものではない; というものでもない	というものではない; というものでもない	
\\	どんなこともお金で解決できる、というものではありません。  
\\	学歴が高いからといって、その人が有能というものでもない。  
\\	商品は安ければいいというものでもない。「安物買いの銭失い」と言う諺もある。  
\\	着ている服だけが、人間の個性を表すというものではありません。  
\\	(赤毛)
\\	失敗は成功じゃないというものではない。   
\\	だけに	だけに	
\\	苦難の途を歩んできただけにできたひとだ。  
\\	やっぱり東京だけに(だけあって)世界各地の食べ物が揃っている。  
\\	火星は地球とよく似ているだけにひとしお興味をひく。  
\\	フランスで勉強しただけに、彼女のフランス語は発音がきれいだ。  
\\	両親は、田舎で二人だけで暮らしている。年をとっているだけに、娘としては心配だ。  
\\	一生懸命働いただけに、彼に促成をしました   
\\	トラもウマも私は怖いです。トラウマだけに  
\\	サメの写真を上手に撮ってきましたよ。ジョーズだけに。  
\\	からいうと	からいうと	
\\	価格から言えば日本の方が高い。  
\\	エコロジーの視点からいうと、南極は観光や商業的な探索ではなく、研究のみに利用されるべきである。  
\\	建築デザインの立場からいうと、このアプローチにはもっと多くの代案が考えられる。   
\\	文法からいうと、ロシア語ほど難しい言葉はない。  
\\	現状から言って、直ちにその計画を実行するのは無理だ。  
\\	(赤毛)
\\	顔からいうと、彼女がミス日本になると思います。  
\\	(赤毛)
\\	故に, 故の	ゆえに, ゆえの	
\\	動かぬ証拠があるゆえに、有罪が確定した  
\\	彼女はその美貌の故に、悩みも多い。  
\\	結果にこだわりすぎるがゆえに、小さなことに心を奪われることになる。  
\\	〜な	〜な	
\\	私はきっと変な人だと思われてる。  
\\	彼女はきれいな足をしている。  
\\	静かなレストランに行きましょう。  
\\	もさることながら (も然る事乍ら)	もさることながら	
\\	〜もそうだが、そればかりでなく。前のことも勿論だが、後のことの方がもっと度合いが強いということを表す。	
\\	路上駐車の問題解決には、自治体の対応もさることながら、地域住民の態度も大切な要素となる。  
\\	アメリカではバスケもさることながらアメフトも人気だ  
\\	両親の意向もさることながら、子供自身が有名校にあこがれている。  
\\	この図書館は蔵書の多さもさることながらサービスが実にすばらしい。   
\\	高齢化が進むにつれ、認知症老人の介護が問題になっているが、これかれは、介護のこともさることながらぼけをいかに予防するかがさらに重要だ。  
\\	歌のうまさもさることながら彼女はピアノの名手なんです。   
\\	会話の練習では、話すこともさることながら、確実に聞き取ることが重要です。   
\\	この合作映画は、作品の完成度の高さもさることながら、文化の異なるにこくが協力して一つの映画を完成させたという事実が非常に意義深い。   
\\	よう	よう	
\\	雨の降るような音がしました。  
\\	そして、ウッドは「負けたら承知しないぞ」と言うように全員をにらみつけた。  
\\	星のように輝いてる瞳。  
\\	そのようなこともあるのでしょうね。  
\\	てならない	てならない	
\\	戦争が始まるかもしれない。国民は不安でならない様子だ  
\\	彼は全然仕事をしていないのに、どうしてあんなに金回りがいいのか不思議に思えてならない。  
\\	毎日が楽しくはずなのに、気分が滅入ってならない  
\\	明日のプレゼンのことが心配でならない。眠れないよ〜。   
\\	私には父がまだ生きていると思えてならない。   
\\	日本の歴史をもっと深く知りたくてならない。   
\\	夫が単身赴任しているので、寂しくてなりません。   
\\	東京の夏は暑くてならない。  
\\	彼のいびきにはまったく我慢ならない。   
\\	彼女がなぜあんなことを言ったのか不思議でならない。   
\\	と思いきや	ともいきや	
\\	想像したろうか、いや、しなかった。思ったろうか、いや、思わなかった。
\\	昼間だから絶対込んでいると思いきや、一人もいなかった。  
\\	このレストランは安いと思いきや、会計は5千円以上だった!  
\\	雪国で育った彼のことだから、さぞかしスキーがうまいだろうと思いきや、滑っては転び、滑っては転びの繰り返しだった  
\\	老後は退職金の蓄えで安泰だと思いきや、インフレで預金が目減りしてしまった。  
\\	チームの勝利で試合終了かと思いきや、最後に大逆転が起こった。  
\\	と言わんばかりに	といわんばかりに	
\\	私がうそをついたと言わんばかりにののしった  
\\	「あなたを愛しているのよ」と言わんばかりに彼女は僕にウィンクした。  
\\	これ以上の議論は無駄だと言わんばかりに、
\\	国の代表は席を立った。   
\\	彼は同意すると言わんばかりにうなずいた。  
\\	彼らは「かわいそうな奴」といわんばかりに私達をじっと見た。  
\\	彼女は万事承知と言わんばかりに私に目くばせした。  
\\	友人は「不可能」と言わんばかりに首を横に振った。   
\\	丸で	まるで	
\\	まるで猿のような顔だ  
\\	これはまるで戦争だ。  
\\	まるで生きているようだ  
\\	まるでつい昨日のことのようにはっきりとあの出来事を覚えている。   
\\	まるでなんでも知っているような口ぶりだね。   
\\	まるでひとりで笑っているみたいだ。   
\\	まだ三月の初めなのにとても暑い。まるで夏になったかのようだ。  
\\	まるでボスのような話し方をする。   
\\	まるでワインで口の滑りがよくなったように。   
\\	まるで悪魔にでも追いかけられてるみたいだ。   
\\	彼女はまるで一人で笑ってるみたいだ。   
\\	まるで雨が降り出しそうに見える。   
\\	まるで化石みたいな人間だわ。   
\\	まるで革命前夜のような騒ぎだった。   
\\	まるで空全体が燃えているようである。   
\\	まるで君は僕の鏡みたいだ。   
\\	まるで月面を歩いているみたい。   
\\	まるで腫れ物に触るように扱った。   
\\	彼女はまるで酔っているようだ。   
\\	まるで太陽が沈んでしまったかのように空は暗くなった。   
\\	まるで頭のような話し方をする。   
\\	まるで彼の手にふれるものはみんな金に変わるみたいだ。   
\\	まるで夢を見ているか気分だった。   
\\	僕はまるで夢を見ているようだ。   
\\	ケンはまるで何でも知っているかのように話す。   
\\	ジョディはまるで幽霊でも見たような顔をしている。   
\\	ジョンが戻ってきたとき、彼はまるで幽霊でもみたように青白い顔つきだった。   
\\	ジョンソンさんはまるで生き字引です。   
\\	この湖は水がきれいで、まるで透明なガラスのようだ。  
\\	ティムはまるで左利きのような書き方をする。   
\\	ハムレットはまるで気が狂ったかのようにふるまう。   
\\	ブレントはアメリカ人であるが、彼はまるで自分の母国語であるかのように日本語を話せる。   
\\	ベティーはまるで何でも知っているみたいに話す。   
\\	ライトさんは、まるで母国語の様に日本語を話します。   
\\	英語はまるで海のように私達を取り囲んでいる。   
\\	我々は山道をまるでアリがはうように進んだ。   
\\	かれはまるでおさけを飲んだようなかおをしています。  
\\	君はまるで何も起こらなかったような顔をしている。   
\\	君はまるで何も起こらなかったような顔をしているね。   
\\	君はまるで人ごとみたいな顔をしているね。   
\\	群衆がまるでハエのようにその車の回りに集まった。   
\\	て形
\\	くる (て来る)	てくる	
\\	新聞を買ってくる。  
\\	医者に行って薬をもらってきました。  
\\	日本語が少しずつ分かってきた  
\\	物価が下がってくる。  
\\	〜な	〜な	
\\	さっさと起きな!  
\\	何か質問があったら、先生に聞いてみな。  
\\	とても〜ない	とても〜ない	
\\	無理と言う意味	
\\	たくさんあって、とても食べきれません   
\\	東京ではとても家なんか買えません。  
\\	貰う	もらう	
\\	私は父にカメラを買って貰った  
\\	私は友達に日本語を教えて貰っている  
\\	コンビニで、何か買って貰いたいですか?  
\\	4時に医者に診てもらう予約をした。  
\\	あなたはグリーン先生に英語の発音を直してもらうべきだ。  
\\	今すぐ髪を切ってもらう必要はないよ。  
\\	ことによると、きみにも一緒にきてもらう。  
\\	このヒーターは故障しているようなので、彼に修理してもらう予定だ。  
\\	この椅子を修理してもらうのにどれくらいかかるのでしょうか。  
\\	この仕事は誰にもしてもらうわけにはいかない。  
\\	この車は修理してもらう必要がある。  
\\	じゃあ、言わせてもらうけど。  
\\	〜をおいて (を措いて)	〜をおいて	
\\	〜を除いて。〜をさしおいて。	
\\	あなたをおいて、この会社の社長の適任者はいない   
\\	日本のミサから:「主よ、あなたは神の子キリスト、永遠の命の糧。あなたをおいて誰のところに行きましょう。」  
\\	最近は職場だけでなく家庭においてもパャ
\\	ンが使われている. 
\\	によると	によると	
\\	友達の話しによればあの映画はとても面白いということです。  
\\	米国
\\	、ロイターなどの報道によると、ブッシュ米大統領が、新しい宇宙探査計画の発表を準備しているとのことだ。  
\\	電灯の発明はエジソンによるとされている。によるとされている。   
\\	彼の概算によると家の新築費用は3000万円です。   
\\	彼は自分の成功を、良い教育を受けたことによると考えた。  
\\	彼らが行った調査によると、アメリカ人は日本に興味を持っていないことがわかる。   
\\	彼らの話によると、あの男の子は歌がうまいらしい。   
\\	副社長によると、会社はいまだに外国の会社から契約をもらっていない。   
\\	聞いたところによると彼らの結婚は破綻をきたしているらしいよ。   
\\	医者の話によると、この病気はすぐによくなるというものではないらしい。  
\\	(赤毛)
\\	報告書によると、この病院の患者の3分の1は12歳以下だった。  
\\	から・ので	から・ので・	
\\	お魚が安かったからたくさん買いました。  
\\	彼は寝過ごしたから (ので)学校に遅刻した。   
\\	「どうして行かないの」 「忙しいからです」   
\\	そうしたのは腹が立っていたからだ。   
\\	彼女が帰ってきたのは雨が降ったからではない。   
\\	身なりが貧しいからといって人を軽蔑してはいけない。   
\\	私がどうしても手助けをしないので彼は怒った。   
\\	お金の価値がなくなってきたのでためる張り合いがなくなかった。   
\\	風邪を引いているので何の味もわかりません。   
\\	切手がなかったので、すぐに手紙を出すことができなかった。   
\\	など; なんか; なんて	など; なんか; なんて	
\\	台風の日に風雨の中でテニスをするなど、考えられない。   
\\	20枚のレポートを1日で書いたなんて、すごいねえ。私なんか3枚のレポートも書けない。   
\\	彼女、大人っぽいね。まだ15歳だなんて信じられないよ。  
\\	お化粧なんかしてはいけません。   
\\	納豆なんてきらいだ。   
\\	日本語でスピーチなどできません。   
\\	なんてなっ! 死んだふり!  
\\	文法なんかくそくらえだ。   
\\	犬なんかちっとも怖くない。   
\\	今ではもう彼女はなんか愛していない。   
\\	君がそんなことをやるなんて考えられない。   
\\	お金を持って来なかったなんておかしいよ。   
\\	嵐の中を出かけるなんてどうかしてういるよ。   
\\	人生なんてそんなもんだ。   
\\	彼がまだ生きているなどということはありえない。   
\\	彼女はテレビで野球などは見えない。   
\\	〜であれ	〜であれ	
\\	社長 であれ規則はまもらないといけない。  
\\	投票は海外に住んでいる人 であれするべきだ。  
\\	あなたの職業がなんであれ、また、その職業にどんなに満足していても、何かほかの仕事を選べばよかったと思うときがあるものだ。   
\\	たとえ天候がどうであれ、私は行きます。  
\\	ドアの所にいる人が誰であれ、待つように言ってください。  
\\	通勤がいかに不便であれ、今の家を変わるつもりはない。  
\\	たとえ国王であれ、国民すべてを従わせることができるわけではない。  
\\	当時の軍隊では上官の命令がいかに理不尽であれ、下級兵はそれに従うしかなかった。  
\\	通り	とおり	
\\	予想通り日本チームが優勝した。  
\\	彼女はもと通り元気になった。  
\\	私が言った通りになったでしょう!  
\\	その通り!  
\\	その問題には三通りの答えがあります。  
\\	先ず一通り見せて頂けますか?  
\\	大丈夫。マニュアルに書いてあるとおりにすれば、うまくいきます。   
\\	昨日のサッカーの試合は、期待された とおりにA国が勝った。  
\\	天気予報とおり、午後から雨になった。  
\\	医者の指示どおり、毎日運動をすることにした。  
\\	その規則は数通りに解釈できる。   
\\	わけには いかない	わけには いかない	
\\	しなければならないことについて	
\\	一人でやるのはたいへんですが、みんな忙しそうなので手伝ってもらうわけにはいきません。  
\\	普通のアメリカのパーティーで日本語を話すわけにはいきません。   
\\	これは先生の本だから、あなたに貸すわけにはいきません。  
\\	会を欠席するわけにはいきません。(理由があって、できない)  
\\	(赤毛)
\\	あなたにおごってもらうわけにはいきません。(しなければならない)  
\\	(赤毛)
\\	我々は、彼のした無責任な行動を見逃すわけにはいかないのである。  
\\	(赤毛)
\\	もう9時を過ぎているが、この仕事を終えるまでは帰るわけにはいかない。  
\\	(赤毛)
\\	あしたはほかの仕事をしなければならないのだから、この仕事をやりかけのまま、帰るわけにはいかない。  
\\	(赤毛)
\\	多くの人が台風に家をとられたので、腕組で見るわけにはいかない。  
\\	彼の垂オ出を簡単に断るわけにはいかない。   
\\	君のことを心配しないわけにはいかない。  
\\	〜べからず	〜べからず	
\\	ここにゴミを捨てるべからず。  
\\	公園の掲示に芝生にはいるべからず。とかいてあった。   
\\	好機逸すべからず。この機に一気に仕事を仕上げてしまおう。   
\\	芝生に入るべからず。   
\\	書籍が学問に従うべく、学問が書籍に従うべからず。   
\\	鉄は熱いうちに打て;後期逸すべからず。   
\\	流言飛語に惑わされるべからず、とはいうものの、言うは易く行うは難し、と思わない?   
\\	ここにゴミ捨てるべからず。   
\\	その混乱実に名状すべからず。   
\\	働かざる者は食う べからず。  
\\	わりに・にしては・向きに	わりに・にしては・向きに	
\\	彼女は日本語が上手だ、外人にしてはね。   
\\	ブラピの映画にしては、なかなか面白いよ。   
\\	外人 
\\	外国人. 
\\	一生懸命やっているわりには、対象者に伝わっていない。  
\\	南向きに置いた。  
\\	なりに	なりに	
\\	私は私なりに人生観を持っています。  
\\	動物は動物なりのコミュニケーションが出来る  
\\	子供は子供なりに自分の世界を持っているものだ  
\\	そうなら、それなりにやってみてください。  
\\	言うなりになる。  
\\	あの国では、私は外国人だったのでそれなりに扱われた。   
\\	このプッシュホンもそれなりに役に立つのが分かった。   
\\	すべての幸福な家庭という物はお互いに似通っているが、不幸な家庭という物はそれぞれそれなりに違った不幸があるものだ。  
\\	その絵はそれなりによくできている。   
\\	タケオは彼なりに良いやつだ。   
\\	テレビの漫画は、それなりに教育的だと言う人もいる。   
\\	私は若い頃にたくさん本を読んだので、それなりに物知りだ。   
\\	人の価値は持ち物というより人となりにある。   
\\	彼にもそれなりに恐いものがある。   
\\	彼の言いなりになるくらいなら、独りで暮らした方がましだ。   
\\	収入が増えれば増えたなりに、支出も多くなっていく。  
\\	〜は愚か・〜は疎か	〜はおろか	
\\	は 言うまでもない; はもちろん; 
\\	はおろか
\\	も/さえ/まで」のように強調して表現されることが多い。	
\\	その赤ちゃんは走る事はおろか、歩くことさえできない。   
\\	私の連れは疲れすぎていて走るのはおろか、歩くこともできない、と言った。   
\\	私はバイオリンはおろかギターも弾けない。   
\\	僕はイタリア語を書くのはおろか読めもしない。   
\\	話すことはおろか、ささやくなんてとんでもないことです。   
\\	彼女は漢字はおろか、ひらがなも正確に読めません  
\\	こんな成績では進学はおろか、卒業だって危ないよ。  
\\	弟は内気で、人前でスピーチはおろか簡単な挨拶さえできない   
\\	アメリカでは隣町まで100マイル、途中途中ガソリンスタンドはおろか、民家さえないという場所がたくさんある。  
\\	最近は大学はおろか、中学生でもかなりの者が海外旅行を経験している。   
\\	戦争はおろか食糧難すら経験したことがない者が口先だけで戦争を議論しているのは笑止だ。   
\\	きっかけに	きっかけに	
\\	引っ越しをきっかけに家具を買いかえた。  
\\	子供が生まれたのをきっかけにタバコを止めました  
\\	この事件をきっかけに   
\\	日本に住んでいるきっかけに生け花の伝統的な芸道を勉強しています。  
\\	そのことをきっかけに彼は写真に夢中になった。  
\\	妻の病気をきっかけに、家族の結束は再び固まった。   
\\	この劇では導入部をきっかけにストーリーが展開する。  
\\	その事件をきっかけとして、第二次世界大戦がはじまったといわれる。  
\\	それをきっかけとしてけんかに加わっていた他の人たちもお互いにトマトを投げ合い始めました。  
\\	多くのひとが新年を禁煙のきっかけにする。  
\\	なり	なり	
\\	彼は帰ってくるなり食事もせずに寝てしまいました。  
\\	彼はドアを開けるなり、靴をはいたまま部屋に入ってきた。  
\\	パーティー会場に入るなり、サムは食べ物めがけて突進した。  
\\	ファンは、有名人と出くわすなり、彼にサインを求めた。  
\\	教室に入るなり、先生は突然怒ったような口調で話しはじめた。  
\\	入ってくるなり、彼は倒れた。  
\\	おかげで	おかげで	
\\	そんな中、ホストファミリーが陽気に話しかけ、あちらこちらへ連れ出してくれたお蔭で、徐々に英語にも慣れ理解できるようになっていきました。  
\\	あいつのお陰で事故に巻き込まれた。  
\\	両親の お陰で、私はアルバイトをしないで勉強することができる。  
\\	私がアルバイトせずに勉強できるのは、両親のお陰だ。  
\\	君のおかげで、お金を全部使ってしまい歩いて帰宅せねばならなかった。   
\\	私は彼の援助のおかげで成功した。   
\\	私が成功したのはあなたのおかげです。   
\\	私が逃れられたのはあなたのおかげだ。   
\\	たものだ	たものだ	
\\	子供のとき、よく川で遊んだものだ。	
\\	子供のとき、よく川で遊んだものだ。  
\\	かつてはここからあの教会が見えたものだ。   
\\	かつては私は毎朝ジョギングしていたものだ。   
\\	日曜日にはいつも釣りに行ったものだ。   
\\	ことになる	ことになる	
\\	このレポート、いずれやることになるのだから、今すぐやろう。  
\\	すぐに使うことになるシステム。  
\\	私は来年大阪へ転勤することになった  
\\	あすで1週間雨が降りつづくことになる。   
\\	あなたにはこの工場の女子職員の管理をしていただくことになるでしょう。   
\\	主人が仕事で東京にいくことになった。  
\\	留学することになりました。  
\\	やすい	やすい	
\\	〜が使ってる動詞をする場合が簡単だと言う意味です	
\\	雨がよく降ってる日が 家の中に 遊びやすいです。  
\\	もしかしたら日本語で毎日、読んだり、書いたりしたら、日本語が分かりやすくなると思います。  
\\	このくつはあのくつよりはきやすいです。  
\\	(赤毛)
\\	このガラスの人形は壊れ易いので、触らないでください。  
\\	(赤毛)
\\	かいだんがぬれています。すべりやすいので、ちゅういしてくたさい。  
\\	(赤毛)
\\	このもんだいはまちがえやすい。  
\\	(赤毛)
\\	東京は生活しやすいところだと思いますか。  
\\	(赤毛)
\\	このペンは書きやすいですね。  
\\	(赤毛)
\\	みせる	みせる	
\\	今年こそは修士論文を書き上げてみせる。  
\\	読んでみせる  
\\	どんな手を使ってでも勝ちとってみせる。   
\\	私はどんな困難にも耐えてみせる。   
\\	石にかじりついても成功してみせる。  
\\	恐れがある	おそれがある	
\\	戦争になるおそれがある  
\\	この販売方法は独禁法に触れるおそれがある  
\\	日本産のトキは絶滅のおそれがある。  
\\	この車は、予算より高いおそれがある。  
\\	気が付いたときには遅すぎるおそれがある。  
\\	この小屋は倒壊のおそれがある。   
\\	そういう事故は再発するおそれがある。   
\\	すごく重くなるおそれがあるけど、試してみたい   
\\	今、対策をたてないと大事故につながるおそれがある。  
\\	(赤毛)
\\	知人の会社は経営が相当苦しそうで、このままではつぶれるおそれがある。  
\\	(赤毛)
\\	雨が降っている夜には、できるだけ運転しない方がいい。事故を起こすおそれがある。  
\\	(赤毛)
\\	そのプランでは、失敗するおそれがありますよ。  
\\	(赤毛)
\\	早く手術じゃないと手遅れになる恐れがある。  
\\	この病気は伝染の恐れがありません.  
\\	地震による津波が沿岸を襲う恐れがある。   
\\	ここにいれば雨にぬれる恐れがありません。   
\\	はともかく	はともかく	
\\	あの店は雰囲気はともかく味はいい。  
\\	いつにするかはともかく、どこにしましょうか?  
\\	冗談はともかく、頭が痛いのなら医者にみせるべきだ。  
\\	私は話すことはともかく、 書くことは苦手だ。  
\\	数学はともかく、テストはまあまあだった。  
\\	掛かる 
\\	かかる 
\\	〜	犬が子供に飛び掛った。
\\	見知らぬ人に吠え掛かるのは多くの犬に共通の習慣です。   
\\	いかなる〜でも (如何なる〜でも)	いかなる〜でも	
\\	いかなる意見でも、必ず聞き入れます。  
\\	いかなる人でも入ることはできません。  
\\	いかなる状況でも部屋を離れてはならない。  
\\	いかなる状況のもとでも、王女が王位を継承することはあるまい。  
\\	いつ、いかなる時でも。  
\\	いかなる彼の態度でも彼をまだ愛してる。  
\\	〜が最後	〜がさいご	
\\	〜したら 
\\	いったん〜たらそのまま 
\\	いったん〜たら必ず	
\\	食べ出したが最後、おいしくて止められなかった  
\\	見つかったが最後、真珠は取上げられてしまいますぞ。  
\\	入った が最後、二度と出てこられなくなるトイレ  
\\	あいつはマイクを握ったが最後、離そうとしないカラオケ狂だ。  
\\	冒頭の一文がいい。「これからする話を聞いてほしいんだ。」素直に耳を傾け始めたが最後、そのまま止まらなくなること必定である。  
\\	男は踏み入れたが最後、生きては帰られない地獄寺で噴出する女の怨念  
\\	""牧口雄二の世界
\\	どんな	どんな	
\\	どんなことがあっても  
\\	どんなことを言ったとしても.....  
\\	どんな人も、入ることはできません  
\\	どんな意見でも聞き入れるようにします。  
\\	どんな風に説明したの?  
\\	どんなに飲んでももっとほしい  
\\	あなたはその事とどんな関係があるのですか。  
\\	いかんによらず	いかんによらず	
\\	理由のいかんによらず、殺人は許されないことだ。  
\\	国籍のいかんを問わず、採用試験を受けることのできる自治体が増えている。  
\\	納入した学費は理由のいかんに関わらず返却しません。  
\\	このコンテンツの複製は方法、媒体のいかんを問わず禁止されている。  
\\	だけあって	だけあって	
\\	さすが毎日トレーニングしているだけあってすごい体力。  
\\	さすが評判のレストランだけあって料理もサービスもすばらしかった。  
\\	彼はずっと運動していないだけにやっぱり太った  
\\	さすがに偉大な学者だけあって、彼はその問いに容易に答えた。  
\\	専門家だけあって彼はその分野に詳しい。  
\\	彼はボディビルが趣味というだけあって体がガシッとしている。  
\\	が作った映画だけあって、面白かった。   
\\	彼女は、美術学校を出ただけあって、絵が上手です。  
\\	あの先生は、経験が長いだけあって、授業がわかりやすくておもしろいですね。  
\\	ながらに	ながらに	
\\	〜ままの状態で	
\\	すべての人は生まれながらに平等である。  
\\	彼は生まれながらに知性と美に恵まれていた。  
\\	彼は生まれながらにして偉大な才能に恵まれていた。  
\\	昔 ながらのやり方。  
\\	古いものを昔ながらに使う。  
\\	インターネットの活用すれば、居ながらにして世界中の情報が手に入る   
\\	その少年は生まれながらに動物たちと交流できる不思議な迫ヘを身につけていた。   
\\	薬害の被害者たちは、これまでの苦しい経験を涙ながらに語った。  
\\	いつもながらの心遣いに、心から感謝垂オ上げます。  
\\	人間には、生まれながらにして貧富の差がある。それは現実だ。   
\\	に相違ない	にそういない	
\\	あの「女」の人は男に相違ない。   
\\	時間も予算も十分ではない。この計画の実行は困難に相違ない。  
\\	彼は有能であるから、この選挙で当選するに相違ない。  
\\	彼は天才であるに相違ない。  
\\	仰ることは、事実に相違ありません。  
\\	(赤毛)
\\	彼女はこの惑星に一番かわいい女であるに相違ない。  
\\	まじめなあの人のことだから、時間どおりに来るに違いない.  
\\	上記の通り相違ありません。   
\\	きっと彼が告げ口をしたに相違ない。   
\\	に即して	にそくして	
\\	日本の習慣に即して  
\\	彼の伝記は全くの事実に即して書かれたものだ。  
\\	現実に即して(人)を判断する   
\\	その仕事のルールに即していないことをする   
\\	業種に即している   
\\	事実に即して言えば   
\\	勧告に即して作成される   
\\	を事実に即して伝える   
\\	現実に即して物事を考える   
\\	耐え難いストレスに対して現実に即して立ち向かう   
\\	違反者は、法律に則して処罰されます  
\\	実情に即した対応策を検討致します。  
\\	時代に即した教育  
\\	現実に即して行動する   
\\	彼の伝記は全くの事実に即して書かれたものだ。   
\\	〜極まる	〜きわまる	
\\	極限に達する	
\\	失礼極まりない   
\\	その映画を見て、感極まって泣いた   
\\	その映画をみて、感極まった様子だった   
\\	不愉快極まる会話だった   
\\	彼の考えは不愉快極まる。  
\\	彼女はその知らせを聞いて感極まって泣き出した。  
\\	進退窮まる.  
\\	〜べからざる	〜べからざる	
\\	成人指定映画は青少年の見るべからざるものだ。  
\\	許すべからざる行為。  
\\	これは将に無視すべからざる犯罪ではないでしょうか。  
\\	今日の社会では、パソコンは欠くべからざるものとなっている。  
\\	この荷物には「危険!触るべからず」と書いてある  
\\	(誠)
\\	にもかかわらず	にもかかわらず	
\\	雨にもかかわらず大勢の人が集まって  
\\	私は雨にもかかわらず出かけた。   
\\	私は疲れていたが、それにもかかわらず眠れなかった。   
\\	私達は苦痛や死があるにもかかわらず喜びを見いだせるでしょう。   
\\	資金不足にもかかわらず、概して彼の実験はうまくいった。   
\\	社会党の猛烈な反対にもかかわらず、その議案は過半数で可決された。   
\\	社会党の猛烈な反対にもかかわらず、その知識の深さは驚くばかりである。   
\\	渋滞にもかかわらず、私はどうにか空港に遅れずに到着することができた。   
\\	所属していた組織に対する、長期間にわたる忠勤ぶりにもかかわらず、かれが報われたのは僅かな金額にすぎなかった。   
\\	小さなレストランにもかかわらず、彼らは何とか不況を通りぬけた。   
\\	少女は病気にもかかわらず学校へいった。   
\\	晴雨にもかかわらず、私は行きます。   
\\	晴天にもかかわらず、空気は冷え冷えとしていた。   
\\	先生の注意にもかかわらず、その怠け者の少年は再々遅れてくる。   
\\	大幅な貿易黒字にもかかわらず、貿易収支外の赤字のために計上収支黒字はそれほど大きくない。   
\\	天才にもかかわらず、彼は相変わらずうだつがあがらない。   
\\	努力したにもかかわらず、彼はその試験に失敗した。   
\\	日本は資源が乏しい国にもかかわらず、国際貿易のお陰で経済大国に発展した。   
\\	日本文学は、その美しさと豊かさにもかかわらず、西欧ではまだ不助ェにしか知られていない。   
\\	日本文学は、其の美しさと豊かさにもかかわらず、西欧ではまだ不助ェにしか知られていない。   
\\	年にもかかわらず彼は健康だ。   
\\	彼には欠点があるにもかかわらず、私は彼を立派だと思う。   
\\	彼には行かないように頼んだ。にもかかわらず彼は去った。   
\\	彼には美しい奥さんがあるにもかかわらず、その割りには喜んでいない。   
\\	彼の欠点にもかかわらずサムはとても好ましい人物だ。   
\\	彼の欠点にもかかわらず私は彼を愛する。   
\\	兼ねて	かねて	
\\	大阪の仕事を兼ねて京都見物に行く  
\\	総理大臣(首相)と外務大臣(外相)とを兼ねる。  
\\	私は勉強と遊びを兼ねている仕事につきたい。   
\\	作文の練習を兼ねて、日本のペンフレンドと文通をしています。まあ、一挙両得というわけです。  
\\	趣味と実益を兼ねて日本語の勉強をしたが、いつの間にか、通訳が本業になってしまった。  
\\	名詞が名詞に代わって	めいしがめいしにかわって	
\\	が 
\\	に これがそれに代わって	
\\	ここでは、人間にかわってロボットが作業をしている。  
\\	社長にかわって私が会議に参加しています。   
\\	ジムは、ベティーにかわって手紙を書いた。  
\\	私があなたにかわって全て行いますから、あなたはただここで待っているだけでいいのです。  
\\	戦後になってガラスにかわってプラスチックが出てきた。  
\\	マッチにかわって、ライターが使われています。  
\\	(赤毛)
\\	[私が]課長にかわって、お詫びいたします。  
\\	(赤毛)
\\	入院した先生にかわって、今は新しい先生が授業をしている。  
\\	(赤毛)
\\	ここでは、人間にかわってロボットが作業をしている。   
\\	首相にかわり、外相がアメリカを訪問した。  
\\	かわきりに, 皮切りに	かわきりに	
\\	冗談をかわきりに話をする。  
\\	「私の経験では」のような言葉を皮切りに話しを始めなさい。  
\\	札幌をかわきりに全国5箇所でセミナーを開催いたします。  
\\	朝の連続ドラマに主演したのを皮切りに、彼女はスターへの道を歩み始めた  
\\	競技場では、100メートル競走を皮切りとして、次々に熱戦が繰り広げられた。  
\\	今度の演奏会は、大阪を皮切りに、全国10カ所で行われる。   
\\	ことになる	ことになる	
\\	主人が仕事で東京にいくことになった。   
\\	留学することになりました。   
\\	にもまして	にもまして	
\\	にもまして
\\	は
\\	もそうだが、それ以上
\\	の意味	
\\	トムはいまや以前にもまして勤勉に働いている。   
\\	何ものにもまして私は健康になりたい。   
\\	彼は以前にもまして勉強しようと決めたのです。   
\\	彼は何物にもまして名誉を重んじる。  
\\	その花は色が美しいのにもまして、香りが素晴らしい。  
\\	新しい会社では、覚えなければならないことが多くて大変だが、それにもましてたいへんなのは、人間関係だ。  
\\	試合に勝ったのにもまして、全力を出し切れたことを誇りに思う   
\\	通訳の試験に合格し、嬉しいのにもまして、今は果たしてうまくやれるか心配だ。   
\\	一方	いっぽう	
\\	その一方で内政要素はなく、基本的に手持ちの戦力とスペルのみで敵を駆逐していく必要がある。  
\\	その一方で,ネットワークを取り巻く新たな問題を生み出しています
\\	しかし、その一方、ラテン語はその時までに「死語」となっていた。  
\\	その仕事はあまりおもしろくなかったが、その一方で給与は良かった。  
\\	コミュニケーションの方法が多種にわたって増加する一方で、人間関係が希薄なものになりつつある。まさに本末転倒だ。  
\\	ジョンは学校で一生懸命勉強した。一方家では母の仕事を手伝った。  
\\	一方、アメリカ人は大きな成功を夢見て、危険を冒す傾向が強い。  
\\	一方で我々は大損害を被ったが、他方その経験から学んだものも大きかった。  
\\	彼ら二人のうち、どちらか一方が先に謝らなければ、けんかは終わらない。   
\\	学芸会で、子供たちは二組に分かれて、一方は歌を歌い、もう一方はダンスを踊った。  
\\	私の仕事は夏は非常に忙しい一方、冬は暇になる。  
\\	日本は工業製品の輸出国である一方、原材料や農産物の輸入国でもある。  
\\	彼女は女優として活躍する一方で,親善大使として貧しい子供だちのために世界中を回っている。   
\\	国は天然資源が豊かな一方で,それを活用できるだけの技術かない。  
\\	地球上には豊かな人がいる一方では、明日の食べ物もない人がいる。  
\\	一方で私はレバーが嫌いだが;他方ではレバーは私のためによいかもしれない。   
\\	くらい、ぐらい (位)	くらい、ぐらい	
\\	今晩は暖かいので、ストーブが要らないくらいだ。   
\\	私たちは一歩も歩けないくらい疲れていた。   
\\	山田さんくらい物を忘れる人はいない。  
\\	私は料理は下手ですが、ご飯くらい(は)炊けます。   
\\	今この仕事が出来るのは彼くらいのものだ。   
\\	そんなことをするくらいなら、死んだほうがましだ。  
\\	合格の知らせを受け取ったときは、うれしくて、踊りあがったくらいだ。  
\\	〜の至り	〜のいたり	
\\	赤面の至りだ。  
\\	若気の至りで   
\\	喜びの至り   
\\	冥加の至り   
\\	憧れの女優に握手してもらって、もう感激の至りだ。  
\\	つつある	つつある	
\\	、
\\	日本語学校の数は減りつつある。 
\\	ゲーム用
\\	としても 進化を 遂げつつある
\\	いい仕事につくのは難しくなりつつあるようだ。   
\\	ここでの私の関心は、ウーマン・リブ運動が消えつつあるのか否かということである。   
\\	ここで明確にしなければならない点は、スポーツにおいてアマチュアとプロフェッショナルの境界線が消失しつつあるということだ。   
\\	このことは家が沈み始め、道路が裂けて街灯が途方もない角度に傾きつつあることを意味する。   
\\	さまざまな種類の生物が絶滅しつつある。   
\\	しかし老人の現在のこうした低い地位も変わりつつあるのかもしれない。   
\\	その運動は急速に進みつつある。   
\\	その学生は猛勉強したので、急激に進歩しつつある。   
\\	その結果、同一労働に対し同一賃金を得ている女性が増えつつある。  
\\	その国は高等教育社会に変わりつつある。   
\\	その子の生存に対する希望は急速に消えつつある。   
\\	その庭園は荒れ地に変わりつつある。   
\\	その病気の子供は日々よくなりつつある。   
\\	その病気は蔓延しつつあるようだ。   
\\	その両国間に緊張が高まりつつあるきざしがある。   
\\	だんだん暗くなりつつある。   
\\	アメリカの都市人口は増加しつつある。   
\\	コンピューターの使用は急速に増加しつつある。   
\\	ジョンソンさんは不注意な管理のせいで失われつつあるお金の額を心配していた。  
\\	一列に立ち並んだ家が新しいアパートに席を譲るためにとりこわされつつある。  
\\	父は退院してから、元気になりつつある。  
\\	(赤毛)
\\	都心では子供が少なくなってきていることから、学生の数もへりつつある。  
\\	(赤毛)
\\	多いの国で公害が年々ひどくなっているが、一方では、それをなくすために、技術協力をする国々も増えつつある。  
\\	(赤毛)
\\	生きれば生きるほど、年が早く経ちつつあること気付くよね。   
\\	3/11から時間がたち、被災地の多くの自治体では復興計画がつくられ、多少の異論があっても具体化する方向に進みつつあります。   
\\	意向形ではないか	〜ようではないか	
\\	一緒に勉強 しようではないか。/ 
\\	みんなで力を合わせて、平和で住みやすい社会を作ろうではないか。  
\\	一緒に勉強 しようではないか。  
\\	この論議はやめようではないか。   
\\	もう勝負は決まった。男らしく罰を受けようではないか。   
\\	元気を出そうではないか。  
\\	(赤毛)
\\	今度祭りにはいっしょに踊ろうではないか。  
\\	この課題は両親に頼もうではないか。  
\\	うさぎは可愛いですよ!。養おうではないか。   
\\	ならともかく	ならともかく	
\\	君ならともかく、俺はそんなことをしない。  
\\	外人ならともかく、日本人はそんなことは絶対にしない。  
\\	さえ	さえ	
\\	食べ過ぎて、イチゴさえ食べられない。  
\\	食べ物がなくなれば、人間はネズミさえ食べる。  
\\	ほとんどの人がその存在さえ知らないようなこと  
\\	ソースコードを見つけられさえしないのに、開発するなんてもってのほかだ。  
\\	その手紙を(読んだことはともかくとして、)見たかさえ分からない  
\\	見えさえしない、絶対読めない。  
\\	定規を 使ってさえ、まっすぐに線が引けない。  
\\	あなたさえよければ、私はかまいませんよ。   
\\	あなたはいわれた通りにしさえすればいい。  
\\	あなたの欠点でさえも私のあなたに対する尊敬の気持ちを減じはしないし、友情においてはこれが大切なことです。   
\\	あなたは、彼に参加したいかどうか尋ねさえすればいい。   
\\	あなたはここで待ってさえいればいいのです。   
\\	あなたはここに来さえすればよい。   
\\	あなたはこのボタンを押しさえすればいいのです。   
\\	あなたはそれを求めさえすればいいそうすれば、与えられるでしょう。   
\\	あなたはもっと一生懸命勉強しさえすればいいのです。   
\\	あなたは意見を述べさえすればよい。   
\\	あなたは一所懸命勉強しさえすればよい、そうすれば試験に合格するだろう。   
\\	あなたは私についてきさえすればよい。   
\\	彼は英語を勉強したことがなく、アルファベットさえ読めない。  
\\	この数日、歯が痛くて、固いものはもちろん、豆腐さえ食べられなくて困った。  
\\	食べ物がなくなれば、人間はネズミさえ食べる。  
\\	彼女は酒が好きで、料理に使う安い酒さえ飲んでしまう。  
\\	今でさえ彼女を信じられない。   
\\	彼女は食費さえ惜しんだ。   
\\	子供ばかりでなく大人さえも悲鳴をあげた。   
\\	知ってさえいたならなあ。   
\\	に あたらない	に あたらない	
\\	天気藍
\\	ヘめったにあたらない。   
\\	それは驚くに当たらない。  
\\	同情するにあたらない、飲酒運転したんだから、自業自得だ  
\\	〜わ	〜わ	
\\	いいわよ。  
\\	この本おもしろいわね。  
\\	(レジで)あら、どうしよう。お財布にお金が入ってないわ。困ったわね。このにんじんは要らないわ。  
\\	にわたる,にわたって	にわたる, にわたって	
\\	オリンピックが16日間にわたって開かれた.	
\\	にわたる  
\\	保証は、配線および配管工事の範囲にわたる。 
\\	上で	うえで	
\\	よく考えた上でご返事いたします。  
\\	メールを使ううえで注意すべきマナーって?  
\\	手続きの書類は、必要事項をご記入の上、ご返送下さい。  
\\	(誠)
\\	みんなで十分話し合った上の結論なのだから、全員従ってほしい。  
\\	(誠)
\\	家族と相談したうえで、ご返事します。  
\\	(赤毛)
\\	アパートは部屋の中を見たうえで借りるかどうか決めたい。  
\\	(赤毛)
\\	それぞれの説明をよこ聞いたうえで、旅行のコースを遊びたいと思います。  
\\	(赤毛)
\\	私が皆様のご意見をうかがったうえで、来週ご報告いたします。  
\\	(赤毛)
\\	テレビは外国語の勉強の上でかなり役に立つ。  
\\	熟慮の 上の結論です.  
\\	皆の意見を聞いた上で決めました。  
\\	内容をご確認の上で,サインをお願いいたします。  
\\	すべて分かった上でやったことだ。   
\\	まいか	まいか	
\\	〜しないだろう	
\\	しかし、学校に行こうと思うと思はまいと、子どもたちは学校に行かされます。  
\\	行こうか行くまいか知ったことじゃない。  
\\	私は彼女にそれについて聞こうか聞くまいか思案した。  
\\	彼のふるまいから彼が嘘をついているのは明らかだ。   
\\	彼女の異常なふるまいから何を推測しますか。   
\\	彼女は招待を受けようか受けまいか決めかねていた。   
\\	問わず	とわず	
\\	その小説は、年齢や性別を問わず,多くの人々に読まれている.  
\\	彼女は問わず語りに自分の経歴を話した。   
\\	小父さんは問わず語りに身の上話をした。  
\\	かと思うと	かとおもうと	
\\	田中さんはいつも忙しそうだ。今来たかと思うともう帰ってしまっている。  
\\	雨が降ってきたかと思うともうやんだ。  
\\	一つが通り過ぎたかと思うと、すぐに次の台風が接近する。  
\\	空が暗くなったかと思ったら、大粒の雨が降ってきた。   
\\	気になる	きになる	
\\	心配する	
\\	何かきになることがある?   
\\	なんでその変なスタイルにしたの? 
\\	何か、気になるよ   
\\	何か、気になるよ 
\\	ろくに〜ない	ろくに〜ない	
\\	ろくに彼女と話をする暇もない.  
\\	どうやら	どうやら	
\\	急いだので、どうやら間に合った。  
\\	どうやら、彼は知らないらしい。  
\\	どうやらあの人が判らないらしい。  
\\	「あら ごめんなさい。どうやら間違え電話をしてしまったようだね。」  
\\	どうやら、そのみすぼらしいアパートは空き家のようだ。  
\\	どうやらおっしゃるとおりだと思います。  
\\	どうやらひどい風邪にかかったようだ。  
\\	どうやら誤解があるようだ。  
\\	どうやら私はバスの中で財布を盗まれたらしい。  
\\	どうやら食べ過ぎたみたいだ。  
\\	どうやら明日は雨のようだ。  
\\	天気はどうやら、回復しそうだ。  
\\	危機をどうやら脱したようだ。  
\\	一個のクッキーを食おうとしたんだが、どうやら全部食われちゃった。   
\\	ながらも	ながらも	
\\	少し悪いと思いながらも、彼は笑うのを抑えることはできなかった。   
\\	彼は、結婚するためにイランに帰るかもしれないと言いながらも、日本から帰った後の彼の計画は依然としてめどが立っていない。   
\\	彼は少ないながらも持ち合わせていた金を全部私にくれた。  
\\	トムは元気を取り戻した。そして、やっぱり、大晦日にはびっこを引きながらもパーティーに参加できた。   
\\	彼は少ないながらも彼の持ち物すべてを私にくれた。  
\\	彼女は女ながらも勇敢に戦った。   
\\	彼女は少ないながらも持っていた全ての硬貨をそのものもらいに与えました。   
\\	彼女は不本意ながらもそうした。  
\\	幼いながらも、その子供は勇敢だった。  
\\	その職人は古い伝説を守りながらも、新しい工夫を重ねている  
\\	田中君は若いながらも古い習慣をよく知っている。  
\\	でも	でも	
\\	品質はいい。でも、値段は高いです。
\\	でも、あなたにあげるりんごはありません。
\\	日本が好きです、でもときどき日本が好きではありません。  
\\	彼はアニメが好きです。でもアニメが好きではありません。  
\\	〜尽くめ	〜ずくめ	
\\	ばかり, 全部、全体がそればかりの状態であることを表す。	
\\	その学校は規則ずくめだ。  
\\	幸せずくめに見えた彼女にも人知れぬ悩みがあったんだね。  
\\	彼は全身黒ずくめだった。  
\\	日本語能力試験の一級も無事合格したし、今年はいいことずくめだった。  
\\	結構ずくめめな話である  
\\	彼女は赤ずくめの服装だった  
\\	小言ずくめでうんざりした  
\\	向けに	むけに	
\\	これは幼児向けに書かれた本です。 
\\	このマンションは一人暮らしの高齢者向けに設計されています。  
\\	きらいがある	きらいがある	
\\	人は自分の聞きたくないことは耳に入れないというきらいがあるのではないか。   
\\	彼女は優しいですが、知らずに人を傷付けることを言うきらいがある。  
\\	人間は年をとると、ほかの人の話しを聞かなくなるきらいがある。  
\\	である	である	
\\	だ 
\\	医者である田中さん  
\\	そうであるとしても、この場合は別でしょう。  
\\	いつも正直であるということは、容易なことではない。  
\\	我輩は猫である。 
\\	〜にしても	〜にしても	
\\	グループ・ポリシーの設定を無効にしても設定内容が元に戻らない。  
\\	国民にしても考えは同じです。  
\\	この漢字は難しい。大人にしても読みにくい。  
\\	日本人にしても敬語は難しい。  
\\	それにしても最近雨ばかり。  
\\	交通渋滞にしても遅くちゃいけない   
\\	たとえ冗談にしても、それは失礼だ。  
\\	日本の夏はいつも蒸し暑いですよ。それにしても今年は随分蒸し暑いですね。  
\\	冬に しては結構暑いよね。  
\\	あなた,道は全然分からないな。タクシードライバーのくせに!  
\\	〜(で)すら	〜(で)すら	
\\	〜さえ	
\\	この天才の私ですらわからなかった  
\\	私は緊張しすぎて、ちらっと見ることすら出来ませんでした   
\\	「人」の漢字すら知らない生徒は、いないでしょ!  
\\	「人」 
\\	この大学の学生の中には教師に対する口の聞き方すら知らないものがいる。   
\\	これらの言葉ですらいつか消えてしまいます。   
\\	こんなケーブル線ですら、6000円もした。   
\\	しっと深い人は、友達についてすら、いじわるをして、満足感をうる。   
\\	その生徒は先生に返事すらしませんでした。   
\\	もちろん品も、そして郵送料すらも返ってこなかった。   
\\	海外で育つ日本人の子供の場合、日本語が完璧にできる場合ですら、帰国後日本の学校に適応するのに大変苦労することが時々ある。   
\\	最も強大な帝国ですら崩壊する。   
\\	社長は我々の給料を少しあげることすら惜しんだ。   
\\	想像すらできない。   
\\	中には月を訪れたことのあるものすらいる。   
\\	虫ですら向かってくるものだ。   
\\	彼ですらそれが上手にできるなら、我々ならなおさらだ。   
\\	彼には日常必需品すらない、まして贅沢品はあるわけがない。   
\\	彼の息子は自分の名前を書くことすらできない。   
\\	彼は、オペラはもちろんのこと、童謡すら歌えない。   
\\	彼は自分の家族すら取りしきれないのに、まして一国を治めるなんて!   
\\	彼女はフランス語は言うに及ばず母国語すらろくに話せない。   
\\	た形 積もりで	たつもりで	
\\	運動したつもりで、歩いて行った。/ 
\\	ボナスは買いに使ったつもりで銀行にお金を入れた。  
\\	(赤毛)
\\	運動したつもりで、歩いて行った。  
\\	(赤毛)
\\	特別料金は五百円だから、コーヒーを一杯飲んだつもりで、特急電車に乗った。  
\\	(赤毛)
\\	死んだつもりで決心した。  
\\	(赤毛)
\\	父親になったつもりで学生の世話をした。  
\\	(赤毛)
\\	有名な人になったつもりで、道で写真を撮らせた。  
\\	(赤毛)
\\	始末だ	しまつだ	
\\	あの二人は犬猿の仲で、ちょっとしたことでも、すぐ口論になる始末だ。  
\\	友達に借金までするしまつだ。  
\\	この始末は、今まで述べたとおりです。  
\\	離婚したうえに、首にまでなるしまつだ。  
\\	〜次第で	〜しだいで	
\\	行動次第で物語は意外な方向に進むかもしれません。  
\\	天気次第でゆくかどうか決めます。  
\\	わたしがご一緒するかどうかはすべてあなた次第ですよ。  
\\	お暇なわけじゃないけど、 お礼次第では、手伝ってあげないものでもないね。  
\\	食事はワイン次第である。  
\\	裸足で歩いているので、足が傷になっている次第だ。  
\\	落し物次第で、警官に報告することだ。  
\\	むしろ	むしろ	
\\	あの人は学者というよりむしろ政治家だ。  
\\	金持ちになるよりむしろ貧乏でいる方がいい。  
\\	とてもすばらしい天気だったのでむしろ歩きたいと思った。  
\\	彼女は女優というよりもむしろ歌手だ。   
\\	彼らは外車よりむしろ日本の車を買う。   
\\	公園へ行くよりむしろ動物園へ行きたい。   
\\	頂く	いただく	
\\	今日頂いたメールに基づいて、、  
\\	頂きます  
\\	社長は手紙を
\\	から弁護士に送って貰いたいのでしょうか?  
\\	あなたの許可を頂くべきだったでしょうか。  
\\	お返事をお待ちしております。またお返事をいただくことにより、私どもはより効率よく会議の準備をすることができます。  
\\	いる (居る)	いる	
\\	その車に犬がいます。
\\	あしたは家にいるつもりだ。  
\\	彼はまだ休暇でハワイにいる。  
\\	その時私は家にいました。  
\\	あなたはどのくらいカナダにいたのですか。  
\\	あなたはニューヨークにいたことがありますか。  
\\	70ぴきぐらいねこが近所にいた。  
\\	たなかさんはいますか?    
\\	はい、います。   
\\	猫は何処にいますか? ねこはどこにいますか?   
\\	メリさんは来月日本にいるでしょう。メリさんはらいげつにほんにいるでしょう。    
\\	〜たい	〜たい	
\\	〜たい	
\\	日本に行きたいです。  
\\	君のところへ行きたいよ。  
\\	日本で英語をおしえたいです。  
\\	ものだ	ものだ	
\\	だってクリスマスは家族が一緒になって祝うものだろう。  
\\	会社はつぶれるものだ  
\\	人間の命ははかないものだ。  
\\	約束したら、きちんと守るものだ  
\\	(誠)
\\	自分のことだけではなく、相手の立場に立って考えることのできる人、それが大人という ものです。  
\\	(赤毛)
\\	までもない	までもない	
\\	〜しなくてもいい	お金よりも健康が大切なことを言うまでもない 
\\	電話ですむことですから、わざわざ行くまでもありません。  
\\	ただの風邪ですから、薬を飲む までもありません。  
\\	いろいろな意味で、正直が最善の策であることは言うまでもない。  
\\	お金がすべてではないのは言うまでもない。  
\\	お金よりも健康が大切なことは言うまでもない。  
\\	すべての人に平等の話す権利が与えられているのは言うまでもない。  
\\	たばこは健康に悪いことは言うまでもない。  
\\	たゆまぬ努力が成功の鍵であることは言うまでもない。  
\\	だれも私たちの間に割り込めないことは言うまでもない。  
\\	疲れているならば、休みをとるべきなのは、言うまでもないことだ。  
\\	スレ立てるまでもない話題はこちらへ   
\\	ことがある	ことがある	
\\	動作・状態がときどき起こるという意味	
\\	わたしの家では父が夕食を作ることがある。  
\\	たいてい私が犬を散歩させていた。しかし最近、妻が散歩させることがある。  
\\	チョコレートを食べ過ぎると、鼻血が出ることがある。  
\\	弟は、バイオリンのコンクールで三位になったことがある。  
\\	(赤毛)
\\	考えれば考えるほど解決方法、わからなくなることがある。  
\\	(赤毛)
\\	(教室で)ご相談したいことがあるんですが、先生の研究室にうかがってもよろしいですか。  
\\	(赤毛)
\\	彼女は親切な人ですが、怒ることもあります。  
\\	私が眠られないことがある。  
\\	朝まで友達と話すことがあります。  
\\	すぎる	すぎる	
\\	その人はちょっと食べすぎましたね。   
\\	このりんごは甘すぎて、食べられない。  
\\	あそこは少し高すぎると思います。  
\\	あの男はおしゃべりすぎる。  
\\	いまからでは遅すぎる。  
\\	このイスは私には低すぎる。  
\\	このウイスキーは強すぎる。  
\\	このカレーはからすぎる。  
\\	このズボンはわたしには派手すぎる。  
\\	この問題は 単純すぎる。  
\\	彼は飴を食べすぎます。   
\\	に依存	にいぞん, にいそん	
\\	オーサリングツールとなると結局ツールメーカーに依存する  
\\	その組織の運営は自発的な寄付 に依存している。  
\\	その国の経済は、農業に依存している。  
\\	皆の成功は一郎の動力に依存しています。  
\\	合衆国が日本の先端技術に依存していること。   
\\	日本は石油を外国に依存している。   
\\	彼女の援助に依存はできないよ。   
\\	君はいつも彼女の援助に依存ばかりしている。   
\\	この国は石油とガスを輸入に依存している。   
\\	私たちの成功は方法に依存している。   
\\	〜末に	〜すえに	
\\	アップダウンの末に手に入れた安定感。  
\\	一晩中書いた末に、エッセイを終えた。  ←終わった。  
\\	いろいろ考えた末に、会社を辞めることにした。  
\\	苦労した末の成功は、何よりもうれしいものです。  
\\	一週間熟考した末に新しい計画を考えついた。  
\\	長々と協議した末に、売り手と買い手は結局折り合った。  
\\	ジョージは、最近3時間の格闘の末に釣った重さ30ポンドのバスの話をしていた。  
\\	山の方では、4月の末になって初めて地面から雪が消える。  
\\	何もおっしゃらないでください。私なりによく考えたすえに出した結論なのです。  
\\	(赤毛)
\\	苦労した末の成功は,何よりもうれしいものだ。   
\\	夏の末に咲くバラもある。   
\\	接戦の末に彼女が当選した。   
\\	ようになる	ようになる	
\\	彼は日本に住んでいたくせに、日本語を話せる ようにならなかったよ。  
\\	子供は, 自分で部屋を片付けるようになりました。  
\\	赤ちゃんは, もうすぐ笑うようになる。  
\\	日本に来たから、豆腐を食べるようになりました。  
\\	レンズさんは日本語が話せるようになった。  
\\	パットは私と話さないようになった。  
\\	以上	いじょう	
\\	先生である以上は  
\\	学生である以上は勉強すべきだ。  
\\	これだけ禁煙、禁煙と言われる以上、喫煙というのは公序良俗に反する行為なんだろうね。  
\\	公僕も人間であること、また人間である以上過ちを犯す可能性があることを、彼は認識している。  
\\	契約書に書かれている以上、期日までにこの仕事を完成させなければならない。  
\\	いったん仕事を引き付けた以上は,途中で止めることはできない.  
\\	彼を信頼できない以上雇うべきではない。  
\\	彼と戦った以上、今まで私にまだ話し掛けていない。  
\\	ことだ	ことだ	
\\	自由とは人に迷惑をかけないことだと言ったJ.S.ミル  
\\	大学に入りたければ一生懸命勉強することだ  
\\	風邪気味なら、早く寝ることだ  
\\	あなたはこれをよむことだ  
\\	上手になるために、繰り返して練習をすることだ。  
\\	一人で悩んでいないで、家族と友達に相談してみることですよ。何かいいアドバイスがもらえるでしょう。  
\\	お金をためたいのでしょう。それなら、無駄づかいをしないことです。  
\\	暖かくして、ゆっくり休むことだ。   
\\	無理をしないことです。  
\\	発言する前に、この問題の要点は理解することだ。  
\\	発言する前に、この問題の要点は理解することだ。  
\\	条件形〜ほど	ば〜ほど	
\\	日本語は勉強すれば勉強するほど面白くなります。   
\\	そこに居れば居るほどその土地が好きになる。  
\\	やればやるほど悪くなる  
\\	早ければ早いほどありがたいです  
\\	「急げば急ぐほど遅くなる」は逆説である。  
\\	の角度が広ければ広いほど、スピードが緩やかになる。  
\\	たくさん本を読めば読むほど、たくさんの事を知るだろう。  
\\	そのことを考えれば考えるほど、私はますます嫌になる。  
\\	その質問が難しくなればなるほど答えることができなくなりそうだ。  
\\	その本は、読めば読むほど分からなくなる。  
\\	その問題について考えれば考えるほど、一層難しくなるように思えた。  
\\	たくさん持てば持つほど、それだけいっそう多く欲しくなる。  
\\	われわれが他人を疑えば疑うほど、他人もわれわれを疑うであろう。  
\\	ポップコーンは食べれば食べるほどもっと食べたくなる。  
\\	飲めば飲むほど注意力が鈍る。  
\\	暇な時間が多ければ多いほど、それだけ彼は楽しい。  
\\	各人が個性的であればあるほど、それだけ他人の英知に寄与する。  
\\	ハンサムならばハンサムなほど恋人を探しやすいです。  
\\	多ければ多いほどよい。  
\\	考えれば考えるほど解決方法、わからなくなることがある。  
\\	(赤毛)
\\	プレゼンテーションに準備すればするほど当時に安心でしょう。  
\\	歌えば歌うほど、声が出るようになったんです。  
\\	(赤毛)
\\	はもとより	はもとより	
\\	低賃金、劣悪な労働条件、一方的な首切りなどの経済・労働問題はもとより、各種の人道的問題までも引き起こしている。  
\\	彼女は英語はもとより、フランス語も中国語も出来る。  
\\	低賃金、劣悪な労働条件、一方的な首切りなどの経済・労働問題はもとより、各種の人道的問題までも引き起こしている。   
\\	マラソンはもとより、100mも走れないよ!  
\\	まんがの種類が増えている。子どものためのものはもとより大人が読むための歴史や経済のまんがもよく見かける.   
\\	彼は料理はもとより食器も洗う。  
\\	つもり (積もり・積り)	つもり	
\\	その意志	
\\	会議には出ないつもりです  
\\	そのつもりは全然ありません  
\\	1時間後に出発するつもりだ。  
\\	3月末に私たちは結婚するつもりです。  
\\	明日は家に居るつもりだ。  
\\	あす(明日)一番にそれをするつもりです。  
\\	あなたに結果を言うつもりはありません。  
\\	彼は日曜に農場を訪れるつもりだ。  
\\	彼女は、30才(歳)になるまで子供は生まないつもりだと言っている。  
\\	彼女は銀行から貯金を全部おろすつもりだった。  
\\	彼女は彼とけんかをするつもりはなかった。  
\\	お前指揮者にでもなるつもりか?  
\\	わたしはいしゃになるつもりです。  
\\	(赤毛)
\\	出かけないつもりです。  
\\	(赤毛)
\\	子供に財産をわけてやるつもりはない。  
\\	(赤毛)
\\	今度ともなお一層の努力を続けて行く所存でございますので、何卒宜しくご指導下さいますようお願い申し上げます。  
\\	(赤毛)
\\	上	じょう	
\\	この映画は教育上良くない。  
\\	健康上の理由で、引退することにした。  
\\	計算上はこれで正しい。  
\\	2国間の貿易上のアンバランスを改善しなければならない。  
\\	あくまで理論上では、その計画は可能である。  
\\	結果の不正確さは言うまでもなく、方法論上の問題が数多く存在している。  
\\	この場合は、翻訳は事実上不可能だ。  
\\	私たちは、一緒に暮らしているけれども、法律上は夫婦ではないんです。  
\\	法律上  
\\	実際上は   
\\	この計画は理論上は問題ない。  
\\	隙に	すきに	
\\	時間に少し開いている隙間を最高の機会とし、そのチャンスを逃さないうちにずるかしこい行為をする。	
\\	皆さん、ジョンさんがいないすきに、サプライズパーティーを整えましょう。  
\\	皆が前を向いているすきにぎゅっと手を引いてほしいんだ。  
\\	先生が余所見をしているすきに、ラッブレターをきみが渡してくれたんだな。  
\\	彼女は気がつかないすきに、彼女にキスした。  
\\	〜につけ	〜につけ	
\\	私が料理すると、味が濃いにつけ、薄いにつけ家族から文句がでる。  
\\	彼を見るにつけ、私の兄を思い出す  
\\	祖母が編んだセーターを見るにつけ、やさしかった祖母を思い出す。   
\\	親は、暑いにつけ寒いにつけ、離れて暮らす子供を思い出し、心配するものだ。  
\\	私が料理すると、味が濃いにつけ薄いにつけ、家族から文句が出る。  
\\	試験での失敗を思い出すにつけ、悔しさがよみがえる。  
\\	彼女は昨日何かにつけついていた。   
\\	相談	そうだん	
\\	(〜ください、〜ちょうだい、〜なさい、
\\	物事を決めるために他の人の意見を聞いたり、話し合ったりすること。また、その話し合い。	
\\	できれば明日までに決めて頂けますか?  
\\	すぐにでもご回答頂けると助かるのですが、お願いできますか。  
\\	さんが参加ができるかどうか、早めに決めなければいけない。  
\\	中国の
\\	社より下記の通り依頼を受けました。  
\\	社への確認をお願いします  
\\	すぐに回答を貰えると、助かるのですが  
\\	ポリシーにより/ポリシーに沿い、辞退することを ご希望ですか?←ポリシーによりご辞退を  
\\	社長は手紙を
\\	から弁護士に送って貰いたいのでしょうか?  
\\	担当者がわかりません。下記の件に関して、誰に連絡をしたらいいのか火曜日までに教えていただけませんか。  
\\	速やかに調達を進めるためには、どなたと意思疎通を図ればいいでしょうか。  
\\	ママ、もっとお魚をちょうだい。   
\\	結高ネものをちょうだいいたしましてありがとうございます。   
\\	十分な食事を ちょうだいいたしました。  
\\	「もっとケーキを召し上がりませんか」「ありがとう。もう助ェちょうだいしました」   
\\	ラジオの音を小さくしてちょうだい。   
\\	ところに、ところへ、ところを	ところに、ところへ、ところを	
\\	これから寝ようとしたところへ友達が訪ねてきた。  
\\	いいところへ来ましたね。今ちょうどスイカを切ったんです。一緒に食べましょう。  
\\	こっそりたばこを吸っているところを妹に見られた。  
\\	ぐっすり寝ているところに地震が起きて、はっと目が覚めた。   
\\	彼女に電話をかけようとしたところへ、彼女のほうからかかってきた。   
\\	犯人は、店から出てきたところを、警官に捕らえられた。   
\\	お忙しいところを来てくださって、ありがとうございます。   
\\	本日は、休日のところを、おいでいただき恐れ入ります。  
\\	家を出ようとしたところに電話があり、打ち合わせはキャンセルになりました。  
\\	(赤毛)
\\	あわてたところをみると、彼は事実を知っているにまちがいない。  
\\	(赤毛)
\\	かどうか	かどうか	
\\	知り合いであるかどうかは関係ない  
\\	あなたはルーシーが日本語を話すことができるかどうか知っていますか。  
\\	あの人に彼女がいるかどうか知ってる?  
\\	きのう彼はあなたがお元気かどうか尋ねていました。   
\\	このくつを試しにはいて、合うかどうかみてみなさい。   
\\	このバスが空港へ行くかどうか私にはわかりません。   
\\	このワインがお口にあうかどうか試しに飲んでみてください。   
\\	この機械がうまく動くかどうかは疑わしい。   
\\	この靴を試しに履いて、あうかどうかみてください。   
\\	この計画がうまくいくかどうかはまだわからない。   
\\	彼は日本人かどうか知りませんか。  
\\	ついでに	ついでに	
\\	実家に帰ったついでに昔の学校に行ってみた。  
\\	仕事へ行くついでに、郵便局で手紙を出した  
\\	出て行くついでに頼むのは悪いけど、ゴミを捨ててくれる?  
\\	パリへ行くついでにロンドンに寄る  
\\	デパートへ行ったついでに、そこで開かれていた展覧会を見た。   
\\	郵便局へ行くの。それなら、ついでに切手を買ってきてくれない。  
\\	そこに着くついでに、マナさんにお金を返します.  
\\	彼を迎えに行って、ついでにこの手紙をポストに入れておいて。   
\\	彼は話のついでに自分の最新作に言及した。   
\\	銀行へ行くついでに手紙を出しておこう。   
\\	ついでに言いますと、彼は食事をしていないと言ってしまった。   
\\	より	より	
\\	よりいい方法知りませんか。  
\\	あなたと会って、より一層あなたのことが好きになりました。  
\\	より早く着く電車はないのですか?  
\\	なさい	なさい	
\\	「一人で間違えずに歌える自信のある人だけ座りなさい。」と指示しました  
\\	早くしなさい。  
\\	好い加減にしなさいよ。  
\\	野菜を食べなさい!   
\\	座りなさい。  
\\	よりもさらに	よりもさらに	
\\	ここから頂上までの道は今までよりもさらに険しいに違いない。  
\\	やっぱり(やはり)	やっぱり(やはり)	
\\	貧乏でいるよりもやっぱり金持ちの方がいい。  
\\	あの娘は結構悩んでいたがやっぱり彼とは結婚しなかった。  
\\	やっぱりそうだったか!  
\\	やっぱり!  
\\	やっぱりあいつは来なかった。  
\\	ヤッパリ、ずーとあったんだよ!  
\\	やっぱりあの娘(こ)に惚れてんだ!  
\\	ヤッパリ、俺は掃除すりゃよかったんだ。  
\\	電車でいきますか?飛行機で行けばいいのに。 
\\	でもちょっと高いですね。やっぱり電車で行きます。  
\\	週末の勉強だけじゃ、全然進歩しないよ。やっぱり、毎日やらないと意味がない。  
\\	日本にあるインドカレー屋のカレーの味って、やっぱり日本人の口に合わせて作られているよね。  
\\	冬はやっぱり雪がないとね。  
\\	「このねこは、やっぱりとてもきれいですよ!」と、とても年取ったおばあさんが言いました。  
\\	やはり、ここのタイ料理の味付けは、日本人好みにしてあるんだね。  
\\	あなたがなんと言おうと、私はやはりその説が正しいと思う。  
\\	疲れているがやっぱり行くつもりだ。   
\\	やっぱりお会いましたね。   
\\	彼女には欠点があるが、やはり彼女が好きだ。   
\\	バターは高価だが、私はやはりマーガリンよりその方が好きだ。   
\\	彼女はやはり元気だ。   
\\	彼らはやはり来年結婚するのよ。   
\\	に至って	にいたって	
\\	程度が普通の状態をこえているさま。大変。非常に。	
\\	破産しそうで破産に至っていない人   
\\	学生達も受験を目前にするに至って、やっとやる気になった。   
\\	いじめによる自殺が社会問題化するに至って、やっと文部省は重い腰を上げた。   
\\	大地震で何千人もの死傷者を出すに至って、やっと危機管理の重要性が叫ばれるようになった。  
\\	腹心の部下が辞めてしまうに至って、社長は自分の非に気づいた。  
\\	わけがない、わけはない	わけがない、わけはない	
\\	あんな下手な絵が売れるわけがない。  
\\	あんなやせた人が相撲とりのわけがない。  
\\	こんな難しい問題、私にできるわけがない。   
\\	こんなにかたい肉、おいしいわけがない。   
\\	うちの会社は人手が足りないから、ひまなわけがない。  
\\	こんなやさしい仕事が君にできないわけがない。  
\\	日本語そんなに速くマスターできるわけがない。  
\\	チャンさんは中国人だから漢字が書けないわけがありません。  
\\	をはじめ	をはじめ	
\\	歌手は最新のヒット曲をはじめ25の新曲を歌った。  
\\	本日、退社することになりました社長を初め皆様には本当にお世話になりました  
\\	メルセデスベンツをはじめとして、輸入車を販売しています。  
\\	携帯電話をはじめ、あらゆる組み込み機器の
\\	を
\\	により実現する。  
\\	今日の祝賀会には首相をはじめ多くの政界人が出席した   
\\	山本さんが難しい試験に合格した。ご両親 をはじめ,先生がたも喜んでいらっしゃる。   
\\	その詩を初めから終わりまで暗記する。   
\\	意向形と思う	(しよう)とおもう	
\\	今から銀行へ行こうと思います。/ 
\\	今から銀行へ行こうと思います。  
\\	(赤毛)
\\	週末は海に行こうと思っています。  
\\	(赤毛)
\\	彼は外国で働こうと思っています。  
\\	(赤毛)
\\	将来自分で会社を作ろうと思っています。  
\\	(赤毛)
\\	「レポートはもうできましたか。」「いいえ、まだ書いていません。金曜日までにまとめようと思っています。」  
\\	(赤毛)
\\	わたしは手紙のへんじをまだ書いていません。いますぐ書こうと思います。  
\\	(赤毛)
\\	わたしは弟に読ませようと思って、この本を買いました。  
\\	(赤毛)
\\	こんばんはこの本をよもうと思っています。  
\\	(赤毛)
\\	お礼を言おうと思いながら、かなり時間がたってしまいました。  
\\	(赤毛)
\\	がんばろうと思いつつ、怠けてしまうんですよ。  
\\	(赤毛)
\\	を聞いた	をきいた	
\\	あなたがフランスに行きたいということをトムから聞いた  
\\	あなたは酒嫌いだと聞いた  
\\	トムが金持ちの女性と結婚したことを聞いた。  
\\	後ろから私の名前が呼ばれるのを聞いた。  
\\	昨日になって初めて、その事故のことを聞いた。  
\\	私たちは雷鳴を聞いた。  
\\	私は子供たちが一緒に歌っているのを聞いた。  
\\	私は彼がぶつぶつ独り言を言うのを聞いた。  
\\	私は彼が新製品について説明するのを聞いた。  
\\	私は彼が犯罪に関わったことを聞いた。  
\\	私は彼女が「いいえ」というのを聞いたことがない。  
\\	っぽい	っぽい	
\\	子どもっぽい表現だという印象があります  
\\	かわいいっぽい上原さん。   
\\	なにっぽいって!  
\\	ローラちゃんは色っぽいよ。  
\\	あの黒っぽい服を着た人はだれですか。  
\\	ジョンは怒りっぽい。   
\\	彼女は、飽きっぽい性格で、今までに職場を何回か変えている。   
\\	怒りっぽい人は、友達が少ない。  
\\	あの子供は、大人っぽい。   
\\	年を取ると、忘れっぽくなる。  
\\	熱っぽい状態で   
\\	君の欠点は忘れっぽいことだ。   
\\	いたずらっぽい微笑   
\\	あの話は嘘っぽい。  
\\	せる・させる	せる・させる	
\\	娘にアイスを 食べさせてあげた。  
\\	僕に払わせてください。  
\\	彼女は彼に早く寝させました。  
\\	うっかり	うっかり	
\\	うっかりミスをする  
\\	うっかり間違いを犯した。   
\\	うっかり口を滑らせると思わぬ結果を招くことが多い。   
\\	降りるところをうっかり通り越した。  
\\	ついうっかりしてバスを乗り間違えた。   
\\	ついうっかりして花瓶を落としてしまった。   
\\	ついうっかり言わないでおけばいいことを言ってしまった。   
\\	なんだか焦げ臭いね。うっかりパンを黒こげにしちゃったの。   
\\	ぼくはうっかり受話器をはずさないでダイヤルを回した。   
\\	タクシーの中にカメラを置き忘れてくるなってうっかりしてましたね。   
\\	なし	なし	
\\	とても寒いけれど、彼女はコートなしで出かけました。  
\\	彼女はコートなしで出かけました   
\\	に基づいて	にもとづいて	
\\	〜を基本にして	
\\	事実に基づいてお話します  
\\	ひらがな、かたかなの形は漢字に基づいている  
\\	は、小説にもとづいて作られた映画です。  
\\	国家である以上、法律に基づく制度が定められているはずだ。  
\\	これまでお研究にもとづいてレポートをまとめた  
\\	彼女の理論は入念な調査に基づいている。   
\\	彼は事実に基づいた記述をした。   
\\	彼女の理論は事実に基づいていなかった。   
\\	彼の言い訳はほんど事実に基づいていませんでした。   
\\	キリスト教的信条に基づいて大学を設立する。   
\\	ものを	ものを	
\\	〜のに!	
\\	知ってるのであれば、教えてくれてもいいものを。  
\\	人手が足りなかったんですって?ちょっと連絡してくれれば、手伝ってあげたものを。  
\\	インターネットで予約すれば簡単にすむものを、彼女は国際電話をかけたんだって。  
\\	本当のことを言えばいいものを、彼は嘘をついて信用を失ってしまった。  
\\	すぐ医者に診(み)てもらったら簡単に治っていた
\\	ものを。手遅れになってしまった。  
\\	父が後1年生きていたくれたら私の結婚式に出てもらえた
\\	ものを、私がぐずぐずしていたばかりに、父に私の晴れ姿を見てももらえず、とても残念です。  
\\	彼は日本語を続けて勉強すればいいものを1年間やっただけでやめてしまった。  
\\	〜こなす, こなす	〜こなす	
\\	このシャンプーは使いこなした。  
\\	この本を読みこなします。  
\\	一日でこの仕事を全部こなすのは難しいと思います。   
\\	彼にはその仕事をきちんとこなす能力がない。   
\\	彼にはその仕事をこなす能力がある。   
\\	彼の英語を使いこなす力は私とほぼ等しい。  
\\	彼はフランス語を自由に使いこなすことにかけてはだれにも劣らない。   
\\	彼は英語をうまく使いこなす。   
\\	日本語ちゃんと使い こなせてるんですね!!  
\\	彼がそのゲームをやりこなしました  
\\	このシステムは、ほとんどすべての仕事をこなす   
\\	(服装は)個性的でもないし、洗練された着こなしというのでもない。  
\\	彼は新しいパャ
\\	ンを使いこなせなかった。   
\\	彼は主人役を見事にこなした。   
\\	自分の番になると彼女はうまくこなした。   
\\	ばかりに	ばかりに	
\\	あの日たまたま休んでいたばかりに、大事な連絡を開きそびれた。  
\\	新車を買ったばかりにお金がありません。  
\\	渡し船があることを知らなかったばかりに歩いて湖を半周してしまった。   
\\	ドイツ語の試験に落ちたばかりに、来月に卒業できない。  
\\	うそをついた ばかりに彼に嫌われた。  
\\	パーティー場所を聞いたばかりに,手伝いを頼まれてしまった。  
\\	寝過ごしたばかりに学校に遅刻した。   
\\	からには	からには	
\\	やると決めたからには最後までやれ  
\\	パリに行ったからには絶対エッフェル塔に昇りたい  
\\	他国を侵略したからにはその責任をとるべきだ。  
\\	女として生まれたからには綺麗でいたい  
\\	約束したからには、守っていただきます。守らなければ死んでもらいます。  
\\	試合に出るからには、勝ちたい。  
\\	約束したからには、守らなければいけない。  
\\	〜にたえない	〜にたえない	
\\	これは芸術作品どころか、見るにたえないほどのできだ。  
\\	聞くに耐えない歌いぶり   
\\	最近のテレビには、見るにたえないほどひどい番組がある   
\\	見るに耐えない   
\\	言うに耐えない   
\\	ああいう陰口は聞くにたえません。  
\\	国会議員と言う要職にある彼が酒に酔って魔黷トいる状態は、全く見るにたえない姿だ。   
\\	に限らず	にかぎらず	
\\	母親に限らず父親も育児をする時代になった。  
\\	ここ数年、コンピューターは仕事に限らず広く利用されるようになりました。   
\\	デズニーランドは子供に限らず大人にも人気がある。  
\\	女性に限り、クラブの入場料は半額です。   
\\	ビザを持っている人に限り入国することができます。   
\\	原子力発電所の周辺住民に限らず遠くの住民も原子力発電の反対運動に参加した。  
\\	もう3時間ぐらいこの姿勢で座っているのは退屈に限らず、足も痺れた。  
\\	〜をよそに (を余所に)	〜をよそに	
\\	〜を考えず・〜と関わりなく・〜を無視して	
\\	彼は大学入試をよそに、毎日パソコンに興じてる。  
\\	親の心配をよそに遊んでばかりいる。  
\\	中国人の反対の声をよそに、小泉総理は靖国神社に参拝してしまいました。  
\\	首相は国民の批判をよそにイラク戦争支持を楓セした。  
\\	ふと	ふと	
\\	電車の中でふと木村さんを見かけた  
\\	ふと新しい考えが浮かんだ。  
\\	テレビを見ていて学生時代の友達をふと思い出した。  
\\	ふとした事が人生の分かれ目になることがある。  
\\	うまい考えがふと浮かんだ。  
\\	素晴らしい考えがふと心に浮かんだ。  
\\	その言葉はふと彼の口から漏れた。  
\\	その老人はふと立ち止まって振り返った。  
\\	それは秘密にしておくべきではないと私はふと思った。  
\\	ふとしたことからその噂は嘘だと分った。  
\\	財布を家に忘れてきたことをふと思い出した。  
\\	私の見たことを話そうかなとふと思った。  
\\	私はふとその男が私をだまして金を奪い取ろうとしているのだと思った。  
\\	私はふと彼女が何かを隠そうとしていると感じた。  
\\	私は列車の中に鞄を忘れたことをふと思い出した。  
\\	試行錯誤の末、彼はふと正しい答えを思いついた。  
\\	彼こそ適任者であるという考えがふと心に浮かんだ。  
\\	彼はそれを見るとふと魔がさして盗んだ。  
\\	彼はわざとそうしたんだとふと私は思った。  
\\	彼は何かを隠そうとしていると、私はふと思った。  
\\	彼は真実を何も言わないかもしれないということがふと頭に浮かんだ。  
\\	彼らのあざ笑いをふと耳にした。  
\\	彼女はふと名案を思いついた。  
\\	僕はふと足を止めた。
\\	〜ことか	〜ことか	
\\	彼が帰って来るのを何年待ったことか。  
\\	3人の子供を教育するには、どれだけお金が掛かることか。  
\\	なんということか!  
\\	この花はなんときれいなことか。   
\\	彼は時がたつのはなんと早いことかとつくづく考えた。   
\\	ずっと一緒だった仲間が別れなければならないのは、なんと残念なことか。  
\\	平和が戻ってきたということは、国民にとってどんなにうれしいことか。  
\\	合格できたら、どんなにうれしいことか。   
\\	何度注意したことか。  
\\	今年は要求したことか。  
\\	〜しかない	〜しかない	
\\	やるしかないじゃん!  
\\	彼女と結婚するしかなかった。  
\\	卒業出来なかったからバイトをするしかない。  
\\	2つの物の間にはごくわずかな差しかない。   
\\	試験を受けるからには、がんばるしかない。  
\\	漢字は、一つ一つ覚えるしかありません。とにかく覚えることです。  
\\	最終電車が出てしまった後だったので、タクシーに乗るしかなかった。  
\\	努力しないで成功できるわけがない。努力しかありませんよ。  
\\	とりあえず我慢するしかないみたいだね。  
\\	(マリ)
\\	ここまできたら、やるしかない。  
\\	(赤毛)
\\	修理が無理だとしたら、買いなおすしかないですね。  
\\	(赤毛)
\\	息子は桜大学に進学することを希望している。しかし、今の息子の学力ではとても無理だから、別の大学を受験させるしかない。本人の希望とは違ってもしかたがないことだ。  
\\	(赤毛)
\\	美人と結婚したいですから、たくさんお金を稼ぐしかない。  
\\	彼が怒っているから、静かにするしかない。  
\\	ここまで来たらもうやるしかないのに、君はまだ迷っているのか。  
\\	(赤毛)
\\	直せないから、新しいのを買うしかなかった。  
\\	残念ながら、時間はほんのわずかしかない。   
\\	からして	からして	
\\	ひらがなからして読めない。もちろん漢字も読めない。  
\\	国は食料からして、不足している。(暗示:だからガソリンなど、ほとんどないだろう)  
\\	彼の健康状態からしてスポーツは無理でしょう。  
\\	態度からしてけしからん  
\\	値段からして、
\\	の機種は
\\	とは比べものにならないと思っていた。  
\\	もうすぐ結婚するというのに、準備がぜんぜんできていません。住む所からして、まだ決まっていないんです。  
\\	あの人の挨拶からして我慢できない。   
\\	プロジェクトチームの名前からして独特だ。   
\\	ではあるまいし	ではあるまいし	
\\	〜ではないのだから 理由を表す。
\\	冬山登山をするのではあるまいし、そんな大げさ格好は要りません  
\\	こどもではあるまいし、暗い所が怖いなんて、おかしいですね。  
\\	幽霊が現れたんじゃあるまいし、そんな驚いた顔をするなよ。  
\\	に於いて	において	
\\	この場合において、乙は改変部分に関する著作権等の権利主張を行なわない。   
\\	英語は商業において役に立つ。   
\\	科学技術において、大きな革命が起こった。   
\\	科学技術において革命が起きた。   
\\	過去において高齢者たちは人生の様々な問題を解決するときのエキスパートと考えられていた。   
\\	過去において女性は他の選択の余地はほとんどなかった。   
\\	概して、トムはやることほとんど全てにおいてのんびりした人間だ。   
\\	気候の違いのため、同種の穀物が国の北部、東部においては収穫されていない。   
\\	教育において試験が大きな役割を果たしている。   
\\	極端に走ってはいけない。適度であることは何ごとにおいても大切である。   
\\	空気の質は過去数年において悪化した。   
\\	私の責任においてこの計画を進めます。   
\\	私の知る限りにおいては、湖はここが一番深い。   
\\	私はみな、人生において成功を収めようとするが、成功するのは、きわめて数少ない人たちである。   
\\	試みはだいたいにおいて失敗だった。   
\\	自分の責任において何でも行いなさい。   
\\	自由社会においては個人の権利は重要である。   
\\	実質上、我々が関わるすべての活動において、学習が行われるのだろう。   
\\	郵便局に何度も足を運ばなくてもよいように切手を沢山手元においてある。   
\\	恋と戦争においてはすべてが正当である。   
\\	当時においては、海外旅行など夢のようなことだった。  
\\	(赤毛)
\\	最近は職場だけではなく家庭においてもパャ
\\	ンが使われている。  
\\	(赤毛)
\\	アメリカは宇宙技術において進んでいる。  
\\	〜てはいられない・〜てばかりはいられない	〜てはいられない・〜てばかりはいられない	
\\	もうすぐ試験なので、遊んでばかりはいられないのです。  
\\	(誠)
\\	お客様がそろそろお着きになって大変なので、のんびりお茶など飲んではいられない  
\\	(誠)
\\	をもとに	をもとに	
\\	ファンの人気投票をもとに審査し、今年の歌のベストテンが決まります。  
\\	この映画は小説をもとにしている。   
\\	自分が集めた事実をもとにして、科学者は、その事実に意味と秩序と価値を与える論理の模様、すなわち理論を織りあげるのである。  
\\	山川民の調査結果をもとに、今後の方針を決めようと思う。  
\\	(赤毛)
\\	日本語のクラスは、テストの点数と今までの学習期間をもとに決定される。  
\\	(赤毛)
\\	我々の販売予測をもとにして。  
\\	新しい筋をもとに作られた物語でした。   
\\	父から聞いた話をもとにして童話を書いて見た。  
\\	到底	とうてい	
\\	とても	
\\	彼女は彼氏を到底殺すことはできない。  
\\	は賢いとは到底言えない。  
\\	到底そうとは思えない。  
\\	あなたの話は到底真実だとおもえない。   
\\	彼に会う事などは到底思いもよらない。   
\\	彼の部屋はこぎれいだとは到底言えない。   
\\	彼は大事な犬を到底手放すことはできなかった。   
\\	彼はバッハには到底かなわない。  
\\	ないまでも	ないまでも	
\\	この公演は大成功とは言えないまでも、それなりの成果をもたらした。  
\\	彼の論文は決して多くないまでも、その質は非常に高い。   
\\	彼が犯人だと断定できないまでも、いろいろと怪しいところがある   
\\	この彫刻は完璧とは言えないまでも、非常に可柏ォを感じさせる作品だった。   
\\	〜ようとする	〜ようとする	
\\	試してみるだけで結果がともなわないニュアンスがある。	
\\	このコンピューターを使おうとしたが、ダメだった。  
\\	電車に乗ろうとしたら、ドアが閉まってしまいました。  
\\	彼が彼女にキスしようとした時、,お母さんが 部屋に入って来た(のでキスできなかった)。  
\\	私は彼に真実な気持ちを言おうとしたが、彼の携帯電話が鳴ってしまいました。言おうとした  
\\	や否や	やいなや	
\\	動詞の終止形に付いて、同時にまたは引き続いてすぐに物事が行われる意を表す。
\\	するとすぐに。
\\	するが早いか。	
\\	赤ん坊は私の顔をみるやいなや、泣き出した。  
\\	そこに出かけるやいなや彼女は病気になった。   
\\	その一行が出発するやいなや、雨が降り出した。   
\\	その音を聞くやいなや、犬は逃げていきました。   
\\	その犬は私の姿を見るやいなや逃げて行った。   
\\	その子はお母さんを見るやいなやワッと泣き出した。   
\\	その女の子は母親を見るやいなやわっと泣き出した。  
\\	その少女は怪物の姿を見るやいなや逃げ出した。   
\\	その知らせを聞くやいなや彼女は青ざめた。   
\\	その部屋にはいるやいなや私は、たばこの臭いのほかにガスの臭いがするのに気がついた。   
\\	その部屋に入るやいなや彼らは話をやめた。   
\\	は発売されるや、すべて売り切れた。  
\\	私が檻を開けるや否やその猫は飛び出して行った。  
\\	出発するや否や雨が降り出した。  
\\	帰宅するや否や雨が降り出した。  
\\	〜際	〜さい	
\\	このサイトをご利用いただく際にご了解いただきたい事項  
\\	この際だからはっきりさせておきたい。  
\\	入院の際は、いろいろお世話になりました。  
\\	あなたの街へ今度旅行する際、あなたにお会いするのが楽しみです。   
\\	お礼を申し上げるのが遅くなりましたが、貴国訪問の 際にはお世話になりました。  
\\	こちらにおいでの際にはぜひお立ち寄りください。   
\\	電車を降りの際、足元に注意して下さい。  
\\	入会を申し込みをする 際、何が必要でしょうか。  
\\	この絵は、社長が中国を訪問した際に北京で買ったものです。   
\\	非常の際は、この階段を使って外へ出てください。   
\\	結婚するのだから、この際、借金をきれいに全部反してしまおう。  
\\	受験の際に、写真が必要です。   
\\	申し込んだ 際、住所を間違って書いてしまった。  
\\	彼は勉強がきらいだった。授業中はあくびばかりしていた。しかし卒業に際して先生との別れをいちばん悲しんだのは彼だった。  
\\	(赤毛)
\\	館内を見学する際には、写真さつえいはご遠りょください。  
\\	(赤毛)
\\	切符を垂オ込まれる際は代替日もご指定ください。   
\\	地震の際はガスの元栓を閉めること。   
\\	危急の際には非常ベルを鳴らすこと。   
\\	職業選択の際に先生が相談に乗ってくれた。   
\\	臨終の際に幸せなことを語るkとですよ!。  
\\	がする	がする	
\\	「ちょっと目まいがするわ。」「顔色が悪いぞ。一日休んだほうがいいよ。」  
\\	このスープは魚の味がする。  
\\	この部屋はガスのにおいがする。  
\\	あなたも血のにおいがするな.  
\\	今から思えば	いまからおもえば	
\\	今から思えばあの時に会社を辞めたことは間違いではなかった。  
\\	今から思えば両親に言われた通りにしておけば良かった。  
\\	今から思えばまだ両親に手伝うことをしなかった。  
\\	〜中に、〜上に、〜下に、など	〜なかに、〜うえに、〜したに、など	
\\	ある物について立地を説明するために	
\\	箱の中に、なにがあるの? 
\\	スポーツ の中に、どちらが一番好き?  
\\	その建物に銀行がありますか。   
\\	ドアの近いにあるかばんは誰のですか。   
\\	車が家の左に止まっておいたよ。   
\\	チキンは冷凍庫の中です。   
\\	郵便局はまっすぐに行って、右側にあります。   
\\	授業は2階にあります。  
\\	〜始める	〜はじめる	
\\	ある活動を始める	
\\	本を読み始めた  
\\	本格的に本を読み始めたのは大学生になってからである。  
\\	6時を過ぎると従業員は帰り始めた。  
\\	あまりにも彼の手紙が優しかったので、彼女は感動して泣き始めた。  
\\	お母さんはコーヒーカップの歴史について調べ始めた。  
\\	お母さんが突然歌い始めた。  
\\	この新聞は最も人気のある筆者をはずしてから、読者数を減らし始めた。  
\\	さらに悪いことに、激しく雷が鳴り始めた。  
\\	げ	げ	
\\	(〜そう)	
\\	とても寂しげな表情で笑う。  
\\	あの子供は楽しげだね?  
\\	おとなげがない。  
\\	彼は得意げな顔で、みんなに新しいカメラを見せた。  
\\	彼は寂しげに一人で公園のベンチに座っていた。  
\\	遠足に来た子供たちが楽しげにお弁当を食べていた。  
\\	彼は何か言いたげだった。   
\\	あの人はさびしげな目をしている。  
\\	そのニュースを聞いたら、彼は嬉しげになった。  
\\	における・おいて	における・おいて	
\\	来年の総会はロンドンにおいて行われる。  
\\	医学における進歩は素晴らしい  
\\	試験は会議室において行われた  
\\	この点において私と彼とは意見が食い違っています  
\\	海外における諸情勢  
\\	極限状態においては、すべての人が協力し合うという同じ反応が起きると、私は信じている。  
\\	マーチン・ルーさー・キングは英国における黒人の指導者であった。  
\\	かなわない	かなわない	
\\	隣の工場がうるさくてかなわない。   
\\	私の家は,駅から遠くて不便でかなわない   
\\	3時間も待つなんてかなわない。  
\\	〜を	〜を	
\\	(通過点)	
\\	公園を歩きました。  
\\	12歳のとき、私は川を泳いで渡ることができた。  
\\	この列車はニューヨークとボストン間を走っている。  
\\	お忙しいところをお出で頂きありがとうございます。  
\\	ようもない	ようもない	
\\	アイスクリームを食べようとして、スプーンがないことに気がついた。これでは食べようがない。どうしよう。  
\\	ひどい虫歯ですね。治しようがありません。もう抜くしかありません。  
\\	コンピュータのデータがすっかり消えてしまった。どうしようもない。困った。  
\\	お礼の垂オ上げようもない。  
\\	そのニュースが彼女に大きなショックを与えた事は否定しようもない。   
\\	どこに行ってもどうしようもないという感じになっていますね。   
\\	君って本当にどうしようもないね。   
\\	君はそれをもっと早くやるべきだったのに。今となってはどうしようもない。   
\\	私としてはどうしようもない。   
\\	親がいないとどうしようもないほど悪さをする子供たちもいるものだ。   
\\	彼が怠け者なのは私にはどうしようもない。   
\\	彼の作品はたとえようもないほど素晴らしい。   
\\	彼女の美しさはたとえようもない。   
\\	分かってるんだけど、どうしようもないのよ。   
\\	万事休すだ。もうどうしようもない。   
\\	ゴミがこれほど散らかっていたら、一人で全部集めようもない。  
\\	(赤毛)
\\	潰れたら熱く語ろうもない。  
\\	(赤毛)
\\	うちに	うちに	
\\	みるみるうちに太った。/ 
\\	若いうちに本をたくさん読みなさい。  
\\	暇なうちに、遊びに来てください。  
\\	暗黙のうちに承諾する。  
\\	みるみるうちに太った。  
\\	どうぞ、熱いうちに、食べて下さい。   
\\	学生のうちに、よく勉強しておきなさい。  
\\	先生の話を聞いているうちに、眠くなってきた。  
\\	ふるさとの母から届いた手紙を読んでいるうちに、思わず涙がこぼれた。  
\\	日本語を勉強して行くうちに日本に関する色々なことが分かって来た。  
\\	(赤毛)
\\	時間があるうちに、やりなさい  
\\	ので	ので	
\\	彼の話はあまりにも馬鹿げていたので誰も信じなかった。  
\\	このように平易な英語で書かれているので、この本は初心者に役立つ。  
\\	彼はお金に困っていたので、昨日スーパーマーケットで万引きを働いた。  
\\	風邪をひいたので学校に行かない  
\\	〜に	〜に	
\\	これは母に貰いました。 
\\	そこに着いたら私に教えてください。  
\\	私はともだちに電話をします。  
\\	私はあなたにりんごをあげる。  
\\	彼は先生にしかられた。  
\\	といえども	といえども	
\\	けれども、〜ても、〜であっても (文語体の逆説の表現。後ろには、前の内容から当然予想されることと反対の内容を表すぶんがくることが多い。「いかに〜といえども」や「たとえ〜といえども」のような形をとると、特別な状況を強調する表現になる。)	
\\	近年、医学が目覚ましく進歩したといえども、病人の数は減少傾向にはない  
\\	いかに多忙といえども、健康管理を怠ってはならない  
\\	たとえ外国人といえども「郷に入って、郷に従え」です。  
\\	老いたりといえども、まだまだ若いものには負けないつもりだ。  
\\	いかに日本での生活が長いといえども、言葉の問題には常に遭遇する。  
\\	つい	つい	
\\	秘密でしたか?ついうっかり喋ちゃいました。   
\\	つい買っちゃった!  
\\	つい「コンピュータがおかしい」と思ってしまいます
\\	でも,コンピュータ側がおかしいことはまずありません
\\	つい先ほど時雨れた  
\\	つい必要なワードフィルを削除してしまった。  
\\	彼はつい自分の名前を言ってしまった。   
\\	彼は大変まずいことをつい言ってしまった。   
\\	彼女は貧しさのあまりつい盗みを働いた。   
\\	というもの	というもの	
\\	これこそ「無法外人」というものでしょう。  
\\	あなたが彼に援助を求めても、無駄というものだ。   
\\	あなたが彼に援助を求めても無駄 というものです。   
\\	ある時代の特徴となる流行の顔と撫というものが、常に存在した。   
\\	あんたは忍耐というものを知っているのか。   
\\	いわゆる『観念連想』というものがなければ、考える事ができない。   
\\	この品物はどこにでもあるというものではない。   
\\	これがいわゆる「天ぷら」というものです。  
\\	これが速記術というものだ。   
\\	これが速記術というもので、こうすれば話す人に付いて行けるのです。   
\\	これは、すでに教科書に書かれている結果が得られるようにお膳立てをするというものだった。  
\\	これ以上待つのはそれこそ時間の浪費というものだ。   
\\	そいつは衰退のレトリックというものだ。   
\\	人から言われたことだけをする。自分から行動を起こさない。それは無気力というものだ。  
\\	本当に困ったときにこそ助け合うのが友達というものだ。  
\\	人のことを考えないで、自分の都合だけで行動する彼女のやり方、あれはまさに自分勝手というものだ。  
\\	全部ものを集めていたことは欲張りというものだ。  
\\	ざるをえない	ざるをえない	
\\	行かざるをえない。  
\\	みんなで決めた規則だから、守らざるを得ない。  
\\	成績が悪くて、留年せざるをえない。  
\\	この考えには反対せざるをえない。  
\\	我々は妥協せざるをえない。  
\\	この分では予定を変更せざるをえないだろう。  
\\	何かをせざるをえないという場合、それを妨げたり避けたりできないという意味である。  
\\	私はそう考えざるをえない。  
\\	社長が反対したら、我々は計画を変えざるをえない。  
\\	同情の涙を禁じえない。  
\\	彼の招待は受けざるをえない。  
\\	彼は死んでしまったと考えざるをえない。  
\\	僕は日曜日に働かざるをえない父に同情する。[僕は
\\	行きたくないけれど、部長の命令だから、行かざるをえない。   
\\	金額が大きくて私には支払えないので、父に援助を頼まざるをえない。  
\\	大切な人を擁護せざるを得ない。  
\\	と言えば	といえば	
\\	春と言えば、卒業の季節。  
\\	果物と言えばバナナだ。  
\\	秋葉原といえば、アニメの聖地です。  
\\	初めてと言えば、あなたの料理を食べるのはこれが初めてです。  
\\	春といえば、思い浮かべるものは何ですか?  
\\	日本と言えば、島国だ。  
\\	お金と言えば、彼はまたお金を借りたらしいよ。  
\\	ニューヨークといえば自由の女神像が有名です。  
\\	ブラジルと言えばサッカーが有名だ。   
\\	2008年のオリンピックと言えば北京です。  
\\	侍と言えば、刀だ。  
\\	ロシアと言えば、寒さで有名です。  
\\	沖縄と言えば美しい浜辺である。  
\\	を中心に	をちゅうしんに	
\\	駅を中心にたくさんの商店が集まっている。  
\\	中国を海外展開計画の中心に議論しましょう  
\\	今日の会議でこの議題を中心に議論をしましょう。  
\\	この地方を中心として、山地は、本州の東北では三列に、同じく西南では二列に分かれて連なっています。  
\\	その家のことはすべてメアリーおばさんを中心に動いていた。   
\\	世界経済はアメリカ経済を中心に動いているという事実は誰も否定できない。   
\\	非核武装を中心にした話し合い。   
\\	午前中は文法を勉強する。午後は会話を中心に勉強することになったいる。  
\\	私は家族を中心に物事を考えている。   
\\	有る	ある	
\\	台所にテレビがあります。
\\	旅に出るまで三日有る(ある)。  
\\	丘の上に家が在り(あり)ます。  
\\	ここに5ドルある。
\\	「今日の新聞にいいニュースがある?」「いや、特にないね。」
\\	57便の荷物はどこにあるのか。
\\	50円ある(有る)?かして。
\\	富士山は日本に有ります。 ふじさんはにほんにあります。   
\\	銀座に、スーパーが有りますか? ぎんざに、スーパーがありますか?   
\\	駅は、何処に有りますか? えきは、どこにありますか?   
\\	辞書が有りますか? じしょがありますか?   
\\	いいえ、有りません。いいえ、ありません。   
\\	ここに、公園が有りました。 ここに、こうえんがありました。  
\\	必ずしも	かならずしも	
\\	善人が必ずしも成功するとは限らない。  
\\	パリに行ったとは言え必ずしもエッフェル塔を登らなくてはならないわけではない。  
\\	身体の大きな男が必ずしも強い男とは限らない  
\\	渋谷といっても、必ずしもにぎやかな店ばかりだとは限らない  
\\	東京だといっても必ずしもなんでも高い訳ではない  
\\	お金持ちが必ずしも幸せとは限らない  
\\	語学が得意だからといって、必ずしも就職に有利だとは限らない  
\\	日本人は礼儀正しい人々だと言う人もいるようだが、実態は必ずしもそうではないと私は思っている。  
\\	金持ちは必ずしも幸福とは限らない。  
\\	必ずしも君の言う通りではない。   
\\	金はそれ自体では必ずしも幸福をもたらすとは限らない。   
\\	偉大が必ずしも賢いとは限らない。  
\\	てやがる	てやがる	
\\	何をかんがえやがったんだ   
\\	こぼしやがった   
\\	ぼくをおいて行ってやがったんだ!  
\\	に関わる	にかかわる	
\\	〜に影響を呼ばす。〜ノブ分には大切なことや重大な内容がくることが多い。単に「〜に関係する」の意味で用いることも多い	
\\	ああいう行動は君の名誉にかかわる。   
\\	こういう行いは君の威信にかかわる。   
\\	これは我が校の名誉にかかわる問題だ。   
\\	それは生死にかかわる問題だ。   
\\	よく注意して運転しないと、命にかかわるよ。   
\\	酸素が不足することは大部分の動物にとって命にかかわることである。   
\\	私たちは狂気のテロ活動にかかわる者に寛容でいるつもりはない。   
\\	製品仕様にかかわる一切の工業所有権はABC社に帰属するものとする。   
\\	戦争は我々すべてにかかわることだ。   
\\	名誉にかかわる問題なので、彼は怒った。   
\\	命にかかわるような病気ではありません。   
\\	負ければ大国の威信にかかわるとあって、無意味な戦いが続けれた。  
\\	〜だけの	〜だけの	
\\	お好きな だけどうぞ。  
\\	できる だけのことはした。  
\\	12人座れるだけの椅子がありますか。  
\\	彼女は稼ぐのと同じだけのお金を使う。  
\\	彼に必要なだけのお金を貸してやりなさい。  
\\	世界に蚊が何匹いるかのだけのお金を作りたい。  
\\	とにかくやってみるだけのことだ。  
\\	できるだけの援助をしましょう。  
\\	彼は手に入るだけの金は受け取るだろう。  
\\	彼女はその訴訟に勝てるだけの有能な弁護士である。  
\\	私は家を買えるだけの金を積み立ている。  
\\	君たち全員に足りるだけの分がある。  
\\	どうして	どうして	
\\	どうしてそんなことが起こったのか。
\\	どうして来なかったのですか。
\\	どうしてこんなに暑いのですか。
\\	どうしてそんなに怒っているの。
\\	何	なに	
\\	何にしますか。
\\	何の用ですか。
\\	この花は英語で何(なん)と言いますか。  
\\	この箱の中に何があるのか。  
\\	あなたの好きな食べ物は何ですか。  
\\	お名前は何ですか。(なん) おなまえはなんですか 
\\	あれは何ですか? (なん)あれはなんですか。
\\	あなたが本当に言いたいことは何ですか。  
\\	何を見ましたか。  
\\	邪魔する	じゃまする 
\\	彼のいびきが一晩中邪魔になった   
\\	(ラブレス)
\\	あの子はいつも付きまとって邪魔だ   
\\	(ラブレス)
\\	それが彼の成功の邪魔になる   
\\	(ラブレス)
\\	明日お邪魔してもよろしいですか。   
\\	おじゃまします。  
\\	くせに	くせに	
\\	田中さんは日本人のくせにあまり漢字を知らない。  
\\	背が高いくせに早く走れない  
\\	いい年をしているくせにこんなことも判らないのか。  
\\	できるくせにやろうとしない。  
\\	なにも知らないくせに威張るんじゃねえ。  
\\	私は日本人のくせして日本語の文法が大嫌いだ。  
\\	警官のくせに泥棒を怖がるなんて。  
\\	新米のくせに大きな顔をするな。  
\\	彼は金持ちのくせに乞食のような生活をしている。  
\\	彼は太っているくせに声は細い。  
\\	知っているくせに、何も教えてくれないんですか。ひどいなあ。  
\\	彼は、彼女のことが好きなくせに、それを彼女に言えない。  
\\	知っているくせに、教えてくれない。   
\\	元気なくせに、病気のふりをしている。  
\\	コムスメのくせに大した怪力だ。  
\\	彼は私の本を無くしたくせに、うそをついた。  
\\	お金もないくせに高い物ばかり買う人は困ります。  
\\	貧乏のくせに彼は気前がいい。   
\\	彼女は足が痛いくせにピクニックに行った。  
\\	ないではいられない	ないではいられない	
\\	買い物せずにはいられない。   
\\	ストレスが溜まった時は酒を飲まないではいられない。  
\\	あなたにキスをしないでいられません。   
\\	あの映画を見れば、だれでも感動せずにはられないだろう。  
\\	ふられた後で、泣かずにはいられなかった。  
\\	彼のもの真似を見るとおかしくて、笑わないではいられない。  
\\	私は、捨てられた子犬や子猫を見ると、助けたあげないではいられない。   
\\	ダイエットをしているのに、食後には甘いものを食べないではいられない。  
\\	困っている人を見たら、助けずにはいられない。  
\\	(赤毛)
\\	決して夢をあきらめない彼の生き方をみていると、私は感動しないではいられない。  
\\	(赤毛)
\\	失礼ことがあった時に、詫びずにはいられない。  
\\	とあって	とあって	
\\	ボーナス後の連休とあって、観光地はどこも人でいっぱいだ。  
\\	夏休みが始まったとあって、子供たちはみんな嬉しそうだ。  
\\	人気スターがやって来るとあって、空港は大勢の人たちが待ち受けていた  
\\	掃除が終わったらお菓子がもらえるとあって、子供たちは一生懸命手伝っている。  
\\	おく	おく	
\\	会議のために、準備 しておきました  
\\	もっと早く検診を受けておけばよかった  
\\	2本も飲めなかったから、1本はあなたに残しておきました。   
\\	あなたの報告書に目を通し、コメントを書いておきました。   
\\	それは引き出しに入れておきました。   
\\	ボブは彼が集めていた切手をほとんど全部ティナにあげて、ほんのすこしだけ自分のためにとっておきました。   
\\	君は、窓を開けたままにしておきましたか。  
\\	状況を改善するための提案を付け加えておきました。   
\\	彼に君の名を言っておきました。   
\\	彼に助言をしておきました。   
\\	彼女は私を2時間もたたせておきました。   
\\	いや。。ちゃんと聞いておこう。  
\\	わたしが払っておきます。  
\\	〜かな	〜かな	
\\	か 
\\	この靴は高いかな   
\\	締め切りに間に合わなかったらどうなるかな。  
\\	「彼の病気は重いのかな。」「そうでないといいね。」  
\\	「肘が痛むんだよ。病院に行こうかな。」  
\\	7月の上旬に、海に行くのは早いかな。  
\\	ずに	ずに	
\\	朝ごはんを食べずに学校へ来ました。  
\\	勉強せずにテストを受けた。  
\\	に等しい	にひとしい	
\\	火星は地球とよく似ているだけにひとしお興味をひく。   
\\	3角形のすべての角の和は180度に等しい。   
\\	あなたの返答はほとんど脅迫に等しい。   
\\	あの会社との契約はあってなきに等しいものです。   
\\	この条件では拒絶に等しい。   
\\	この地域の失業はほぼないに等しい。   
\\	滑り出しが好調なら事は半ば成就したのに等しい。   
\\	君の言葉はほとんど屈辱に等しい。   
\\	真実と科学者の関係は言葉と詩人の関係に等しいとしばしば言われてきた。   
\\	水と魚の関係は、空気と人間の関係に等しい。   
\\	読書と精神の関係は、食物と肉体の関係に等しい。   
\\	彼を雇うことは誰も雇わないことに等しい。   
\\	本の心に対する関係は、食糧の体に対する関係に等しい。   
\\	葉と植物との関係は、肺と動物との関係に等しい。  
\\	はもちろん	はもちろん	
\\	国が違うと、言葉や文化はもちろん、人々の考え方まで違う。  
\\	あのレストランはサービスはもちろん味も悪くない。  
\\	彼は私たちに着物はもちろんお金もくれた。   
\\	彼は書くことはもちろん読むことも出来ない。  
\\	彼は走ることはもちろん、歩くこともできない。   
\\	彼は知識はもちろん人格も備えている。   
\\	彼女はピアノはもちろんのこと、バイオリンも弾ける。   
\\	彼女は英語はもちろん、フランス語も話せる。   
\\	彼女は言語はもちろん、文学も研究している。   
\\	彼女は自転車はもちろんオートバイにも乗れない。  
\\	疲れきっていたので勉強するのはもちろん考えるのも嫌だった。   
\\	彼は、勉強はもちろんスポーツもよくできる。   
\\	うちには、車はもちろん自転車もないんです。  
\\	私が生まれた村は,電車はもちろん 
\\	バスも通っていない。  
\\	何があったのか教えてださい、もちろんここだけの話です。  
\\	「一緒に行ってもいいですか」 「もちろんです。」  
\\	もちろん二人は結婚するものと思っている。   
\\	「本当にこのネクタイをするのかい」「うん、もちろんだよ。」  
\\	にしろ; にせ(も)よ; にしても	にしろ; にせ(も)よ; にしても	
\\	日本にしろ、中国にしろ、アジアの国へ行くのは面白い。	
\\	日本、にしろ中国にしろ、アジアの国へ行くのは面白い。  
\\	引き受けるにしろ引き受けないにしろ、なるべく早く決めたほうがいい。  
\\	(赤毛)
\\	本当にしろうそにしろ。 
\\	ビールにしろ、日本酒にしろビール飲んだら運転できない。  
\\	好きにせよ嫌いにせよ、彼女が優れた歌手であることはみんなが認めている。  
\\	〜が早いか	〜がはやいか	
\\	〜するや否(いな)や。〜するとすぐに。	
\\	社長が会議室に入ったがはやいか、早速会議が始まった。  
\\	その朝彼がキッチンに入って来るがはやいか、彼女は彼と些細なことで口論を始めた。  
\\	その力士はまわしを取るが早いか、相手を投げ飛ばした。  
\\	新刊を出版したのですが、出版するが早いか腹を立てた人から手紙が届き始めました。  
\\	ことによるとあの女は、私が太刀打ちを始めるが早いか、人の助けでも呼ぶために、藪をくぐって逃げたかもしれない。  
\\	言うが早いかやってしまった。  
\\	ベルが鳴ったが早いか、生徒たちは教室を飛び出して行った。  
\\	弟は晩ごはんを食べ終わるが早いか宿題をし始めた。  
\\	弟は晩ごはんを食べ終わるが早いか宿題をし始めた。 
\\	攻めに	せめに	
\\	広告攻めにされた。  
\\	質問攻めにする。   
\\	毎日Eメール攻めに遭っている。  
\\	のもとで	のもとで	
\\	先生の指導のもとに試験に受かる事が出来ました。  
\\	国連から派遣された医師団のもとで救援活動が続けられた  
\\	法のもとではすべての人は平等だ。   
\\	好奇心が知識のもとである。  
\\	不調法 
\\	無調法	ぶちょうほう	
\\	「一杯如何?」「ごめんね。まったく不調法なの」   
\\	(キャプテン)
\\	口は不調法だが、腕は確かだ  
\\	(キャプテン)
\\	其れが	それが	
\\	それなのに	
\\	朝は台風の影響で大雨でした。それが、昼前から急に晴れてきて、今はすっかり快晴だ。  
\\	ほど-2	ほど-2	
\\	私はレンズほど強くありません。	
\\	僕は君ほど速く歩けません。   
\\	ジェニファーほどよく勉強する学生はいない。   
\\	私は西田さんがあれ ほどピアノが弾けるは思わなかった。  
\\	ルイスさんの日本語は太田先生も外国人だと気づかなかったほど上手です。   
\\	〜っけ	〜っけ	
\\	か	
\\	誰だっけ?  
\\	お名前は何とおっしゃいましたっけ。  
\\	この前、僕が君に会ったのはいつだっけ。  
\\	この前、私があなたに会ったのはいつだっけ。  
\\	何の話をしていたっけ?  
\\	私の新しいムスタングを見せてなかったっけ?  
\\	私達は何をするように言われたのだっけ。  
\\	次の時間(授業)は何だっけ?  
\\	車酔いするほうだっけ?  
\\	今晩、山田さんも来るんだっけ。   
\\	君は去年日本へ来たんだっけ。   
\\	この本、おもしろかったっけ。   
\\	お宅はどちらでしたっけ。  
\\	あしたのパーティーのこと、話したっけ?  
\\	得る	える	
\\	ここでは、より良い笑顔をえるための方法を紹介しています。  
\\	あなたが教えてくれたことから大いに得るところがあった。  
\\	あの青年は大金を得るため一生懸命働くのをいとわなかった。  
\\	お客様の信頼と尊敬を得ることが当社の目標です。  
\\	この建物に入るには当局の許可を得る必要がある。  
\\	その男は権力を得るために多額の金を使った。  
\\	その本からたくさんの貴重な情報を得ることができる。  
\\	もしアメリカで勉強するなら学生ビザを得る必要がある。  
\\	今の職を得るために彼女は父親のコネを利用した。  
\\	事故の後、彼は公式試合に出場する機会を得ることは二度となかった。  
\\	彼女に話して得るところがあった。  
\\	かれは自分の車を売って15万円を得た。  
\\	知識を得る一番よい方法の一つは経歴の異なる人たちと話すことです。  
\\	ジョージは真っ正直なので顧客の信頼を得た。  
\\	彼は医者としての評判を得ている。   
\\	あのずるい人のようには会社に裏切る。   
\\	い形容詞変化	いけいようしへんか	
\\	い形容詞の変化し方
\\	~ず もがな	~ず もがな	
\\	サッカーで日本が勝って、韓国人の金さんがどんな気持ちかなんて、そんなこと、言わずもがなのことだよ。  
\\	新しい電気製品なんて、買わずもがなで、買うとろくなものがない。  
\\	あの人の書く文章には無くもがなといえる言葉が多い。  
\\	〜程	〜ほど	
\\	寒ければ寒い
\\	冷たければ冷たい
\\	ほどお酒がおいしくなると言われている  
\\	彼は口も利けないほど腹を立てていた。  
\\	ドアが閉まらないほど込んでいます。  
\\	一日では食べきれないほどのリンゴがある。  
\\	この英語の小説は、君が一週間で読めるほどやさしくない。  
\\	この家具のほうが比較にならないほど上等である。  
\\	この果物は摘み取られるほど成長していない。  
\\	この仕事は死ぬほどしんどい。  
\\	愛するものを失う(こと)ほどつらいことはない。  
\\	この店は僕が望んでいるほど遅くまで開いていない。  
\\	この薬は鼻水に不思議なほどよく効く。  
\\	これまで訪れた中で、エベレストほど美しい山を見たことがない。  
\\	そう言う訳で、私は彼を死ぬほど憎んでいる。  
\\	そのコーヒーは私が飲めないほど熱かった。  
\\	あの店はこちらの都合に合わせるほど遅くまでやっていない。  
\\	私はあなたほど忙しいく有りません。   
\\	相撲では、太っているほど有利だ。  
\\	かどうかのうちに	かどうかのうちに	
\\	電車のドアが開いたかどうかのうちに  
\\	授業ベルが鳴らすかどうか教室から学生が出ました。  
\\	とも限らない	ともかぎらない	
\\	将来海外に転勤しないとも限らないから、英語を勉強しておこう。  
\\	試験の日に、風邪をひかないとも限らない。  
\\	彼らが、明日の試合に絶対勝てるとは限らない。  
\\	可能性	かのうせい	
\\	再発の可能性が少しあります。  
\\	生物兵器テロの可能性が高い感染症について  
\\	説は少し可能性が低いとされている。  
\\	ほとんど可能性がないといってもいいのですが,  
\\	そのことをきっかけにして事件に巻き込まれる可能性がなきにしもあらずだと思います。  
\\	とか	とか	
\\	お嬢さんが結婚なさったとか。おめでとうございます。   
\\	銀行は経営が危なくなっているとか。預けた金は大丈夫だろうか。   
\\	彼は大金持ちだとか。うらやましいなあ。  
\\	田中さんの犬がおじさんを噛んだとか。   
\\	にしろ; にせ(も)よ; にしても	にしろ; にせ(も)よ; にしても	
\\	たとえ〜ても ・ 〜でも	
\\	たとえ貧乏にせよ盗みはいけないよ!  
\\	たとえ社長になったにせよ謙虚な生活をした方がいい。  
\\	本当にしろうそにしろ  
\\	どんな金持ちにしろ地獄に財産を持って行けない  
\\	いずれにしろ  
\\	いくら忙しいに(も)せよ食事をしないのはよくない。  
\\	どんな人に会うにしろ、失礼はだめです。  
\\	彼に同意するにせよしないにせよ、考えさせられる話だ。   
\\	どういう結果になるにせよ、やるだけやってみるべきだ。   
\\	どんな理由があるにせよ人を殺す権利はない。  
\\	いくら有能があるにせよ、努力なしに成功できるとは思わない。  
\\	少年にしろ、成人にしろ、悪いことをしたら、厳しい罰を受けるべきです。   
\\	旅行に行くにせよ、行かないにせよ、早く決めたほうがいい。  
\\	たった三日の旅行にせよ、準備は必要だ。  
\\	(赤毛)
\\	いくら忙しいにもせよ食事をしないのはよくない。  
\\	かけだ・かけの・かける	かけだ・かけの・かける	
\\	雑誌を読みかけて、そのままうとうと寝てしまった。  
\\	何か言いかけてやめるのはよくない。  
\\	やっと風邪が治りかけたと思ったら、今度は歯が痛くなってきた。   
\\	彼は帰りかけたが、用事を思い出したらしく、すぐに戻ってきた。 
\\	父は新聞を読みかけたままにして、どこかへ行ってしまった。 
\\	おとといから今日にかけてずっと雨が降り続いている. 
\\	これは、食べかけですが、どうぞ召し上がってください。
\\	かわりに	かわりに	
\\	今朝はコーヒーの代わりにココアを飲んだ  
\\	話す代わりに  
\\	高い代わりに  
\\	先生のかわりに私が教えた。  
\\	2階に私のかわりにもっていってください。   
\\	かわりに君にそこへ行って欲しい。   
\\	これらの労働者を一時解雇するかわりに、彼等の労働時間を短縮すればよいじゃないか。   
\\	これを私のかわりにしていただけたらとてもありがたいのですが。   
\\	しばらく私のかわりに子どもから目を離さないでください。   
\\	そのかわりに最前定義した二つの経済学的変数に関する議論を行うことにしよう。   
\\	まことにすみませんが、私のかわりにそれをやっていただけませんか。  
\\	音楽会に行くかわりにCDを三枚買うほうがいいと思う。  
\\	ビールを切らしてしまった。代わりにウイスキーを飲めよ。  
\\	お辞儀の代わりに、アメリカ人は握手をする。   
\\	この石は金づちのかわりになる。  
\\	(よ)うと〜まい, (よ)うが〜まい, ようと〜まい, よう	(よ)うと〜まい, (よ)うが〜まい, ようと〜まい, よう	
\\	雨が降ろうが降るまいが、私は出かけます  
\\	パーティに参加しようがするまいが、会費は全員払わなければならない。   
\\	昔の恋人が結婚しようとしまいと、今の私に関係ないことです。  
\\	あの教授は学生がわかろうと分かるまいとかまわず授業を進めて行く   
\\	込む	こむ	
\\	あまり暑いので、プールに飛び込んだ。   
\\	心を込めて作りました。  
\\	てから	てから	
\\	以下のプルダウンメニューから、言語およびプラットフォームご利用いただけるバージョンが表示されます。   
\\	先週の木曜日からずっと雨が降っている。   
\\	それをここから持って行け!  
\\	「オーストラリアからですか」とフィリピン人はたずねました。  
\\	「結婚してからも私働いて高
\\	槓いわよ」と彼女は言った。  
\\	私は火星から来ました。 わたしはかせいからきました。  
\\	この本は図書館から借りました。このほんはとしょかんからかりました。  
\\	そう	そう	
\\	雪が降りそうですね。  
\\	美味しそうなデザートですね。  
\\	「起こしてくれればよかったのに」翌朝ロンが不機嫌そうにいった。  
\\	この本は高そうです。  
\\	饂飩が美味しそう!何時食べられるの?  
\\	これは、食べられそうではありません。  
\\	教室は静かそうだ。   
\\	雨が降りそうだ。  
\\	どうも	どうも	
\\	この空模様では、どうも雨になりそうだ。  
\\	どうもうまくゆかない。  
\\	どうも済みません。  
\\	どうもよくわからん。  
\\	「あの人たちは、またストをやるんだろうか」「どうもそうらしいね」  
\\	どうも行く必要はないようです。  
\\	いやどうも、どうも。  
\\	とは限らない	とはかぎらない	
\\	高い物が必ずしも安いものより良いとは限らない。  
\\	あらゆる真実がいつでも語られるとは限らない。  
\\	いつでも自分の思い通りにできるとは限らないよ。   
\\	うまい冗談を言うことはいつもたやすいとは限らない。   
\\	おいしそうに見える食べ物が必ずしもおいしいとは限らない。  
\\	お金が幸福をもたらすとは限らない。   
\\	お金持ちが必ずしも幸せとは限らない。   
\\	この法則はすべての場合に適用されるとは限らない。   
\\	たとえ自分の思い通りにしたとしても、必ずしも成功するとは限らない。   
\\	われわれは必ずしもあの星を肉眼で見られるとは限らない。   
\\	偉大な学者が必ずしも良い先生であるとは限らない。   
\\	科学の進歩は必ずしも人類に利益をもたらすとは限らない。   
\\	科学の進歩は必ずしも人類に利益を与えるとは限らない。   
\\	科学技術が必要な物を何でも与えてくれるとは限らない。   
\\	科学上の発見が、世の中を今までよりも良い場所にするとは限らない。   
\\	近ごろでは、結婚の動機は必ずしも純粋とは限らない。例えばジェニファーを見てごらんよ。   
\\	どの子供もリンゴが好きだとは限らない。   
\\	金はそれ自体では必ずしも幸福をもたらすとは限らない。   
\\	金持ちは必ずしも幸福とは限らない。   
\\	上に	うえに	
\\	別に心配するほどのものでもないと思ったうえに、有益な情報や新しい提案があるなら聞いてはみたいと思ったのでしょう。  
\\	何といっても以前のバージョンと比べて処理がとにかく速いうえにとても管理しやすくなっています。  
\\	寒かったうえに、風が強かった。   
\\	道に迷ったうえに、雨にまで降られた。   
\\	彼はウイスキーを一本あけたうえに、ビールを少し飲んだ。  
\\	彼は私に助言してくれたうえにお金までくれた。   
\\	彼は詩人であるうえに学者でもある。   
\\	彼女はとても頭がいいうえに一生懸命勉強する。   
\\	彼女は性格が悪いうえに会った女性の中で一番醜いです。  
\\	その日の帰宅後、僕はママとパパに髪を切らされてしまい、全くうんざりだった上に2人とも僕の進学適性試験のことでうるさかった。  
\\	この手続きは、面倒なうえに時間もかかるので、皆がいやがっている。  
\\	(赤毛)
\\	私の彼はかっこいいうえに、お金持ちなんです。  
\\	(赤毛)
\\	テーブルの上においた読みかけの本を母がかたづけてしまった。  
\\	遅刻した上に教科書まで忘れた。   
\\	まだしも	まだしも	
\\	だったら、ほかのよりいい	
\\	一度ならまだしもここまで六回その言葉を間違えって書いた。  
\\	この方がまだしもあれよりよい。   
\\	こんなひどい目に合うよりまだしも死んだほうがよい。   
\\	すうじつならまだしも10日も滞納している。   
\\	〜の	〜の	
\\	結婚するのは大変だ。  
\\	「さあ、皆さん、正々堂々戦いましょう」全選手が周りに集まるのを待って先生が言った。  
\\	夢に耽ったり、生きることを忘れてしまうのはよくない。   
\\	私も買い物に行くのが好きです。  
\\	アルバイトをしながら学校に行くのは大変ですね。  
\\	兎も角	ともかく	
\\	「それはともかく」、「ともかく無事でよかった」など、“	
\\	試合の結果はともかく,最後までみんなよくがんばった。  
\\	彼女の作ったケーキは、 ともかく見た目は味は大変いい。  
\\	勝てるかどうかはともかくとして、試合に出られるだけでうれしい。  
\\	ともかくお電話します。   
\\	ともかく私について来なさい。   
\\	彼は遅刻はしたが、ともかくやって来た。   
\\	ともかくすぐ出発する。   
\\	10年前ならともかく、今はそんな服は着られない。  
\\	はかどる (捗る)	はかどる	
\\	仕事がはかどってるようですね  
\\	どの程度はかどっているのですか?  
\\	あの人は学校の勉強のほうははかどっているのですか。  
\\	仕事は非常にはかどっている。  
\\	〜れる・〜られる	〜れる・〜られる	
\\	水が子供に飲まれました。  
\\	このビルは二年前に建てられた。  
\\	携帯コンテンツの海外事業開発を任されています。   
\\	長い間待ってやっと私の名前が呼ばれた。  
\\	あいつにしてやろうと思ったのに、逆にあいつにやられてしまった。  
\\	彼は光に吸い寄せられた。  
\\	言われなくても、分かってるよ.   
\\	日本では日本語が話されています。   
\\	佐藤さんは田中さんにパーティーに招待されました。   
\\	日本ではこの歌はみんなに知られています。  
\\	ひとり〜だけでなく, ひとり〜のみならず	ひとり〜だけでなく, ひとり〜のみならず	
\\	最近のロンドン爆破テロは、ひとりイギリスだけではなく世界の問題です。  
\\	いじめや不登校問題は、ひとり教育の現場だけでなく大きな社会問題となっております。  
\\	義務兵役制度は ひとり台湾、ドイツだけでなくデンマーク、オーストリアなど全世界20カ 国で長い歴史を持っています。  
\\	失業はひとり本人の問題だけでなく、家族にも心配をかける。  
\\	公共の場所でタバコを吸うのはひとり本人に有害であるだけでなく、周囲の人にとっても迷惑なものである。  
\\	就職難はひとり上海のみならず、ほかのところにおいても同様に見られる傾向である。  
\\	出生数の減少傾向は、ひとり日本のみならず、諸外国においても同様に見られる。  
\\	ただ〜のみならず	ただ〜のみならず	
\\	のみならず
\\	(まで)(も)」の形でも用い、「ただ」がつくと、より強調する阜サとなる。	
\\	富士山はただ高いのみならず、姿も美しいので、日本の象徴として愛されている。  
\\	彼はただ勇敢であるのみならず、優しい心の持ち主である。  
\\	これからの企業は単に利益を追求するのみならず、環境に配慮した生産活動を行わなければならない。  
\\	結果	けっか	
\\	かれと相談した結果決めた  
\\	ある研究の報告によれば、二次喫煙の結果、53,000人のアメリカ人が毎年死亡している。  
\\	君のせきはタバコを吸う結果だよ。  
\\	この長い不況の結果、社会不安が起こるかもしれない。  
\\	この政策の結果、物価が大幅に上昇した。  
\\	結果は次の通りである。  
\\	結果をグラフで示しなさい。   
\\	彼女は息子のテスト結果がどうなるかと心配している。   
\\	核戦争の結果はどんな事態になるのだろうか  
\\	その地震は多数の死者を出す結果になった。  
\\	あの	あの	
\\	あの 
\\	その 
\\	あの 
\\	あの人	
\\	あの林檎は緑。あのりんごはみどり。   
\\	あの本は長いです。  
\\	あの男の子と女の子はとても小さいです。  
\\	あの古い学校は私の学校です。  
\\	あの若い女の子はカナダに行くらしいです。   
\\	〜ところ(だ) (所だ・処だ)	〜ところ(だ)	
\\	私は息子がクッキーを盗み食いしようとしているところを見つけた。  
\\	バスは発車するところだった。   
\\	今、食べる 所です。  
\\	これから、部屋を掃除する ところです。  
\\	食事がおいしくないといってくれて、ありがとう、危うく食べる 所でした。  
\\	私はもう少しで宿題を忘れるところだった。  
\end{CJK}
\end{document}