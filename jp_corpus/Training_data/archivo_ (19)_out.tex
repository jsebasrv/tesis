\documentclass[8pt]{extreport} 
\usepackage{hyperref}
\usepackage{CJKutf8}
\begin{document}
\begin{CJK}{UTF8}{min}
\\	愛	あい	
\\	相変わらず	あいかわらず	
\\	彼女は相変わらず彼に好意を持っていた 
\\	厳しく叱(しか)られたのに彼は相変わらず犬をいじめる 
\\	彼は相変わらず不機嫌(ふきげん)な顔をしていた 
\\	祭日(さいじつ)だというのに父は相変わらず忙しい 
\\	父の症状(しょうじょう)は相変わらずだ 
\\	愛情	あいじょう	
\\	合図	あいず	
\\	前進の合図に手を振った 
\\	巡回(じゅんかい)中の警官が車を止めるよう合図した 
\\	コーチからの合図で 
\\	彼らに黙るように合図した 
\\	合図をしたらそのせりふを言ってください 
\\	笛(ふえ)を合図に 
\\	正午(しょうご)を合図に黙祷(もくとう)した 
\\	愛する	あいする	
\\	相手	あいて	
\\	〔物事を一緒にする人〕
\\	遊び相手 
\\	ダンスの相手 
\\	老婦人(ろうふじん)の相手をした 
\\	彼女の相手をするのは楽しい 
\\	誰かお客様の相手をしてあげてください 
\\	相手かまわず話しかけた 
\\	相談相手(そうだんあいて) 
\\	話し相手がいなくて寂(さび)しい 
\\	手紙の相手はだれだ 
\\	それは相手によりけりだ 
\\	だれも彼をまともに相手にしない 
\\	〔商売・取引などの対象となる人〕 これは若い女の子相手の店だ 
\\	〔結婚の対象となる人〕 妹がいい相手を見付ければと願(ねが)っています 
\\	〔相対して争う人〕
\\	〔競争相手〕
\\	〔訴訟などの〕
\\	彼は私の相手として不足はない 
\\	とても私たちでは相手にはなれなかった 
\\	彼らを相手にダブルスの試合をした 
\\	その賞を張り合う相手同士だった 
\\	相手は何を望んでいるのか 
\\	相手方 
\\	相手次第 相手次第でこちらの態度を決めることにした 
\\	相手役 彼女は主人公(しゅじんこう)の相手役に選ばれた 
\\	あいにく	あいにく	
\\	あいにく先約があります 
\\	あいにく一張羅(いっちょうら)はクリーニング屋に出してあった 
\\	あいにく母は出掛けています 
\\	それはおあいにくさま 
\\	語調によっては皮肉になる)/
\\	あいにくの天気 
\\	あいにくの雨で野球の試合は流れた 
\\	遭う	あう	
\\	明かり	あかり	
\\	空き	あき	
\\	明らか	あきらか	
\\	諦める	あきらめる	
\\	飽きる	あきる	
\\	飽きさせる 
\\	/((口)) 
\\	勉強はもう飽きた 
\\	映画は飽きた 
\\	彼の長話には飽きた 
\\	彼らのけんかにはもう飽きた 
\\	飽きるほどてんぷらを食べた 
\\	彼の権力欲は飽きることを知らない 
\\	握手	あくしゅ	
\\	悪魔	あくま	
\\	明ける	あける	
\\	〔新しい年になる〕 年が明けた 
\\	明けましておめでとうございます 
\\	〔朝になる〕 5時に夜が明ける 
\\	目がさめたら夜はすっかり明けていた 
\\	明けそめた空 
\\	初める=そめる 
\\	明けて5月1日(ついたち、いちにち、いちじつ),我々は横浜港(こう)を出た 
\\	〔ある期間が終わる〕
\\	梅雨(つゆ)が明けた 
\\	年季(ねんき)が明けた 
\\	明けても暮れても 
\\	彼は明けても暮(く)れても本ばかり読んでいる 
\\	揚げる	あげる	
\\	(旗を)
\\	(揚げ物を)
\\	(陸揚げする)
\\	〔掲げる〕
\\	〔空中に飛ばす,翻す〕
\\	旗(はた)を揚げる 
\\	するすると旗を揚げる 
\\	凧(たこ)を揚げる 
\\	花火を揚げる 
\\	〔揚げ物を〕
\\	えび[魚]を揚げる 
\\	〔陸に移す〕
\\	これらの箱はすぐに陸(りく)に揚げる必要がある 
\\	〔芸者を呼ぶ〕 芸者を揚げてどんちゃん騒(さわ)ぎをする 
\\	挙げる	あげる	
\\	〔上の方へ伸ばす〕
\\	賛成(さんせい)の人は手を挙げてください 
\\	後ろで「手を挙げろ」という声がした 
\\	〔示す〕 似た例をもう2,3挙げる 
\\	[((文))
\\	警察は状況(じょうきょう)証拠(しょうこ)しか挙げられなかった 
\\	国会議員の名前を挙げた 
\\	〔推挙(すいきょ)する〕 彼を会長候補(こうほ)に挙げることが決まった 
\\	〔検挙(けんきょ)する〕
\\	殺人犯はその場で挙げられた 
\\	[((文))
\\	麻薬(まやく)密輸(みつゆ)団はまだ挙げられていない 
\\	〔式を行う〕 
\\	10月に結婚式を挙げる予定です 
\\	〔大勢の人を動かす〕 
\\	彼らはついに政府に対して兵(へい)を挙げた 
\\	〔世間に知らせる〕 第2作目の小説で彼は名を挙げた 
\\	預かる	あずかる	
\\	〔引き受けて保管する〕
\\	この旅行かばんを預かってください 
\\	このお金を預かって欲しい 
\\	〔銀行で〕
\\	彼は学校の鍵を預かっている 
\\	〔引き受けて世話をする〕 
\\	昼間(ひるま)子供を5人預かっている 
\\	私は彼の別荘(べっそう)を預かっている 
\\	夫婦の旅行中は私が留守を預かっていた 
\\	彼女は一家の家計(かけい)を預かっている 
\\	〔保留する〕
\\	審判(しんぱん)は競技を中止(ちゅうし)し勝負(しょうぶ)を預かった 
\\	君の辞表(じひょう)はしばらく預かっておく 
\\	/((米・文)) 
\\	預ける	あずける 
\\	〔人に保管を頼む〕
\\	〔金を銀行に〕
\\	預金(よきん)通帳(つうちょう)は彼に預けてある 
\\	駅で手荷物を預けた ((米)) 
\\	/((英)) 
\\	ホテルのクロークに携帯品を預けた 
\\	[((米))
\\	お金を全部彼に預けた 
\\	私のお金は銀行に預けてある 
\\	私は100万円銀行に預けてある 
\\	〔人に世話を頼む〕 子供を姉に預けた 
\\	彼は帳簿(ちょうぼ)を預けられた 
\\	人に命を預ける 
\\	〔勝負などの決定を一任する〕 この勝負(しょうぶ)は彼に預けよう 
\\	汗	あせ	
\\	与える	あたえる	
\\	母は彼に相当(そうとう)額の金を与えた 
\\	勝者(しょうしゃ)には金メダルが与えられた 
\\	博士号(はくしご)がオックスフォード大学から与えられた 
\\	私に与えられた賞賛(しょうさん)は身に余(あま)るものです 
\\	〔供給する〕
\\	((文)) 
\\	〔支給する〕
\\	〔あてがう〕
\\	読書は楽しみを与える 
\\	被災者(ひさいしゃ)は毛布(もうふ)を与えられた 
\\	研究所は私の研究にあらゆる便宜(べんぎ)を与えてくれた 
\\	重要な地位を与えられた 
\\	円周(えんしゅう)上の与えられた1点 
\\	〔影響などを及ぼす〕
\\	台風が稲作(いなさく)に大損害を与えた 
\\	友人が死去(しきょ)したという知らせは私にショックを与えた 
\\	温かい	あたたかい	
\\	暖まる	あたたまる	
\\	温まる	あたたまる	
\\	暖める	あたためる	
\\	火にあたって体を暖めなさい 
\\	手をこすって暖める 
\\	空気は暖めると膨張(ぼうちょう)する 
\\	部屋をストーブで暖めた 
\\	温める	あたためる	
\\	辺り	あたり	
\\	当たり前	あたりまえ	
\\	〔当然〕 当たり前の 〔自然な〕
\\	〔妥当な〕
\\	あなたが怒るのは当たり前だ 
\\	コートも着ずに寒い戸外にいたのだから風邪をひくのは当たり前だ 
\\	借金(しゃっきん)を返すのは当たり前じゃないか 
\\	彼が言ったことは至極(しごく)当たり前のことです 
\\	彼はしかられても当たり前だ 
\\	彼は当たり前のこととして許可をもらいに来た 
\\	「お母さんに謝(あやま)ったかね」「当たり前だよ」 
\\	彼がそう思うのは当たり前だ 
\\	女が家事をすることはまだ当たり前と思われている 
\\	〔普通〕 当たり前の 〔普通の〕
\\	〔いつもの〕
\\	〔標準的〕
\\	彼女は当たり前のアメリカの少女です 
\\	それが当たり前のやり方である 
\\	良家(りょうけ)の子女(しじょ)が日暮れ後の一人歩きを禁じられるのは当たり前のことであった 
\\	当たる	あたる 
\\	(命中する)
\\	(的中する)
\\	(くじなどに)
\\	(成功する)
\\	(触れる)
\\	(雨などに)
\\	(光が)
\\	(暖まる)
\\	(対抗する)
\\	(方角が)
\\	(相当する)
\\	(毒などに)
\\	(暑気に)
\\	(探る)
\\	(試す)
\\	(役割を担う)
\\	〔ぶつかる〕
\\	〔軽くぶつかる〕
\\	ボールが頭に当たった 
\\	矢(や)は的(まと)の真ん中に当たった 
\\	石が窓ガラスに当たった 
\\	雨[風]が勢(いきお)いよく窓に当たった 
\\	満員電車の中でだれかのぬれた傘が背中に当たった 
\\	〔光・風などを受ける〕 火に当たって暖まる 
\\	日に当たるのが好きだ 
\\	この部屋はよく日が当たる[当たらない] 
\\	テープには日が当たらないようにしてください 
\\	雨に当たってペンキがはげている 
\\	夜霧(よぎり)に当たって風邪をひいた 
\\	〔野球で〕 彼は今日はよく当たっている 
\\	〔うまくいく,成功する〕 彼の芝居(しばい)は当たった 
\\	商売(しょうばい)が当たった 
\\	〔くじなどで〕
\\	くじで100万円[一等賞(いっとうしょう)]が当たった 
\\	大穴(おおあな)が当たった 
\\	〔的中する〕
\\	私の推測(すいそく)が当たった[当たらなかった] 
\\	/((口)) 
\\	予測が当たった 
\\	彼の予言は当たった[当たらなかった] 
\\	〔ひどい仕打ちをする〕 
\\	ことがうまくいかないと周(まわ)りの誰にでも当たる 
\\	彼はいつも私につらく当たる 
\\	〔仕事・役割を引き受ける〕
\\	私がそのクラスの担任(たんにん)に当たっている 
\\	私が審判(しんぱん)に当たった 
\\	私がその任(にん)に当たることになった 
\\	〔立ち向かう〕
\\	全力を挙げて敵に当たった 
\\	みんなでこの危機に当たらなければならない 
\\	〔指名される〕 
\\	英語の時間に読みが当たった 
\\	主役に当たった 
\\	〔探(さぐ)りを入れる〕 
\\	同僚の意向も当たってみなくては 
\\	太郎がだめなら,正男に当たってみなさい 
\\	どんな様子か当たってみよう 
\\	〔照らし合わせて調べる〕 単語を辞典に当たって調べる 
\\	〔身体(からだ)に障(さわ)る〕
\\	河豚(ふぐ)に当たる 
\\	暑さに当たった 
\\	/〔日射病になる〕
\\	きっと食べた魚に当たったのだ 
\\	〔傷む〕 
\\	この桃は所々当たっている 
\\	〔相当する〕
\\	彼は私の曾(そう)祖父(そふ)に当たる 
\\	彼女は私のいとこに当たる 
\\	誕生日が祝日(しゅくじつ)に当たった 
\\	1ドルは140円に当たった 
\\	「家」に当たるドイツ語は何か 
\\	その語句に当たる英語はない 
\\	国務長官は日本の外務大臣に当たる 
\\	この場合は第3条に当たる 
\\	〔その方角にある〕 
\\	その町は東京の西北(せいほく)に当たる 
\\	〔ある場合に出会う〕 
\\	彼の門出(かどで)に当たり乾杯しよう 
\\	〔否定語(ひていご 
\\	と共に用いて,…するには及ばない〕 
\\	〜するには当たらない, 
\\	感心するには当たらない 
\\	彼を煩(わずら)わせるには当たらない 
\\	遠慮(えんりょ)するには当たらない 
\\	驚くには当たらない 
\\	彼女が未婚(みこん)の母だったからって驚くには当たらない 
\\	〔髭をそる〕⇒そる(剃る) 当たらずといえども遠からず 
\\	/((口)) 
\\	〜するには当たらない, 
\\	当たっている 
\\	〜の〜に当たる
\\	あちこち	あちこち	
\\	扱う	あつかう	
\\	(手で) 使う〕
\\	〔操作する〕
\\	壊れ物(こわれもの)は丁寧(ていねい)に扱ってくれ 
\\	この機械(きかい)は扱いやすい[にくい] 
\\	〔待遇する〕
\\	〔世話する〕
\\	〔あやつる〕
\\	〔もてなす〕
\\	部下をもっと公平(こうへい)に扱いなさい 
\\	新人は彼が扱う 
\\	彼は扱いにくい 
\\	/((俗)) 
\\	彼女はお客を扱うのが上手だ 
\\	冷(つめ)たく扱われた 
\\	〔仕事として処理する〕 
\\	ここでは書留(かきとめ)は扱いません 
\\	事務をもっと能率(のうりつ)よく扱えないのか 
\\	この問題は次の章で扱う 
\\	彼のけがは労働災害(ろうどうさいがい)として扱われた 
\\	労働災害補償 
\\	〔店で売る〕 
\\	当店ではお米は扱っておりません 
\\	〔報道する〕
\\	新聞は彼の発見を大きく[3段抜きで]扱った 
\\	集まり	あつまり	
\\	当てる	あてる	
\\	〔ぶつける〕
\\	ボールにバットをまっすぐに当てる 
\\	車を門に当ててしまった 
\\	〔命中(めいちゅう 
\\	させる〕
\\	的(まと)の真ん中に当てた 
\\	〔さらす〕
\\	フィルムを日に当てないように 
\\	植木鉢(うえきばち)を日に当てた 
\\	〔あてがう,つける〕
\\	口に手を当ててあくびをかみ殺した 
\\	氷嚢(ひょうのう)を病人の額に当てた 
\\	戸に耳を当てた 
\\	どうぞ座布団をお当てください 
\\	ズボンのひざに継(つ)ぎを当てた 
\\	〔いい結果・賞などを得る〕 くじを引いて2等賞(とうしょう)を当てた 
\\	軽量レンズを発明して一山(いっさん)当てた 
\\	当てたね
\\	今日はハイキング日和(にってい)だ 
\\	〔正しく推量する〕
\\	〔予測する〕
\\	当ててごらん 
\\	うまく当てた 
\\	当て損(そこ)なう 
\\	まぐれで答えを当てた 
\\	彼は試合の行方を当てた 
\\	〔指名してやらせる〕 田中君を当てて第1段落(だんらく)を読ませた 
\\	跡	あと	
\\	〔何かが通ったあと〕
\\	〔獲物などの残した跡〕
\\	雪の上についたスキーの跡 
\\	車が通った跡 
\\	牛が通った跡 
\\	きつねの跡を追う[失う] 
\\	〔何かが行われたあと〕
\\	鉛筆で付けた跡 
\\	指の跡 
\\	/〔指紋〕
\\	爪の跡 〔ひっかいた跡〕
\\	/〔動物の〕
\\	歯の跡 
\\	血の跡 
\\	/〔かすかな跡〕
\\	インクの跡 
\\	蚊に刺された跡 
\\	〔以前に何かがあった印〕
\\	ここに (退化した) ひれの跡がある 
\\	進歩[改善]の跡が見えない 
\\	努力の跡が著しい 
\\	〔遺跡(いせき)〕
\\	〔遺物〕
\\	〔廃墟(はいきょ)〕
\\	石器(せっき)時代の跡が発見された 
\\	城の跡を保存する 
\\	〔家督(かとく)〕 
\\	次男が彼の跡を継いで社長になった 
\\	彼の死で一家は跡が絶(た)えた 
\\	跡を追う 
\\	〔追跡する〕⇒ついせき(追跡)1 
\\	〔続いて死ぬ〕 母は父の跡を追うように亡くなった 
\\	跡をくらませる 
\\	((文)) 
\\	跡を絶つ 
\\	彼の墓には今でも訪れる人が跡を絶たない 
\\	万引きが跡を絶(た)たない 
\\	穴	あな	
\\	油	あぶら	
\\	脂	あぶら	
\\	誤り	あやまり	
\\	粗	あら	
\\	〔魚を調理したあとの骨など〕
\\	〔精米(せいまい)したときに出るくず〕
\\	〔欠点〕
\\	他人の粗はよく目につく 
\\	〔接頭語(せっとうご)として,未完成の〕
\\	あら訳 
\\	あら乾(かわ)きの 
\\	嵐	あらし	
\\	争う	あらそう	
\\	〔競争する〕
\\	彼はほかの3人の男と賞を争った 
\\	日本は自動車産業で他国(たこく)と争っている 
\\	彼らは議会の1議席を争った 
\\	彼らは主導権を争った 
\\	スタジアムに入ろうと先を争って走った 
\\	〔けんかする〕
\\	〔論争する〕
\\	〔口論する〕
\\	隣室(りんしつ)から争う声が聞こえた 
\\	裏庭(うらにわ)で2人の男が争っているのが見えた 
\\	科学者はその問題について争っていた 
\\	〔主張する〕 最高裁まで争うつもりだ 
\\	新た	あらた	
\\	あらゆる	あらゆる	
\\	華道(かどう)のあらゆる流派(りゅうは) 
\\	あらゆる角度(かくど)から検討(けんとう)する 
\\	あらゆる手段(しゅだん)を尽(つ)くす 
\\	表す	あらわす	
\\	〔はっきり表に出して示す〕
\\	彼女の態度(たいど)は不満を表している 
\\	彼は満面(まんめん)に喜びの色を表していた 
\\	本の題名が内容を表すとは限らない 
\\	彼の咳(せき)は明らかに喘息(ぜんそく)の症状(しょうじょう)を表している 
\\	〔表現する〕
\\	考えを言葉に表す 
\\	この深い感謝の気持ちは言葉では表せません 
\\	その壮大(そうだい)な眺めは筆では表せない 
\\	〔意味する〕
\\	〔記号として示す〕
\\	〔象徴する〕
\\	この場合「赤」は「危険」を表す 
\\	は
\\	を表す 
\\	メダルの月桂樹(げっけいじゅ)の葉は勝利を表す 
\\	現す	あらわす	
\\	〔出現させる〕 角を曲がると高い建物(たてもの, 
\\	見物(けんぶつ)
\\	が姿を現した 
\\	彼はついに正体を現したな 
\\	〔発揮(はっき)する〕
\\	彼はサッカー界で名を現しはじめた 
\\	その薬は1時間で効果を現すだろう 
\\	著す	あらわす	〔書く〕
\\	〔出版する〕
\\	彼は10冊の小説を著した 
\\	現れ	あらわれ	
\\	〔姿が見えてくる〕
\\	夢に現れる 
\\	満月が山の端(はし)に現れた 
\\	車を走らせて行くと教会の尖塔(せんとう)が行く手に現れた 
\\	女性の姿が霧の中に現れた 
\\	〔出現する〕 数年前会員が女性だけというクラブが現れた 
\\	彼は19世紀に現れた偉人の一人である 
\\	その家にはお化けが現れる 
\\	〔そこへ来る〕 彼が現れたときには食事は終わっていた 
\\	[((口))
\\	/((口))
\\	〔目立つようになる〕 彼が世に現れたのは40代の時であった 
\\	景気回復の兆しが現れはじめた 
\\	現れる	あらわれる	
\\	〔姿が見えてくる〕
\\	夢に現れる 
\\	満月が山の端(はし)に現れた 
\\	車を走らせて行くと教会の尖塔(せんとう)が行く手に現れた 
\\	行く手 
\\	女性の姿が霧(きり)の中に現れた 
\\	〔出現する〕 数年前会員が女性だけというクラブが現れた 
\\	彼は19世紀に現れた偉人の一人である 
\\	その家にはお化(ば)けが現れる 
\\	お化け 
\\	〔そこへ来る〕 彼が現れたときには食事は終わっていた 
\\	[((口))
\\	/((口))
\\	〔目立つようになる〕 彼が世に現れたのは40代の時であった 
\\	景気回復の兆(きざ)しが現れはじめた 
\\	有る	ある	
\\	(存在する)
\\	(実在する)
\\	(位置する)
\\	(建物が)
\\	(島・都市などが)
\\	(所有する)
\\	(起こる)
\\	(行われる)
\\	(度量が)
\\	(身長が)
\\	(目方が)
\\	(面積が)
\\	(含む)
\\	〔持っている,備わっている〕
\\	彼は金と暇がある 
\\	田舎に農場がある 
\\	我々には投票権がある 
\\	彼には天賦の才能がある 
\\	その仕事をする資格が十分ある 
\\	〔行われる,起こる〕 試験は3月にある 
\\	昨夜地震があった 
\\	オリンピックは4年に1度ある 
\\	あのスーパーで大安売りがある 
\\	〔経験する〕 富士山に二度登ったことがある 
\\	怖い夢を見ることがよくある 
\\	〔数量がある〕 彼の身長は6フィートある 
\\	体重は60キロある 
\\	〔存する〕 この辞書の長所は使いよさにある 
\\	[((文))
\\	〔「…してある」の形で,行為の完了〕 この件は報告してある 
\\	食卓の用意はしてある 
\\	〔「…である」の形で,断定〕 彼は学生である 
\\	彼の態度は友好的である 
\\	有る事無い事 彼女は有る事無い事なんでも言い触らす 
\\	或	ある	
\\	ある所に 
\\	ある教授がそう言った 
\\	ある日 
\\	ある時 
\\	ある場合には 
\\	私はそう思うが,ある者はそう思わない 
\\	ある程度目的を達した 
\\	ある意味ではそれは正しい 
\\	あるいは	あるいは	
\\	〔または〕
\\	〔…かまたは…〕
\\	これかあるいはそれ 
\\	新幹線で行こうとあるいは飛行機に乗ろうと時間はあまり違わない 
\\	英語かあるいはフランス語かどちらかが必修(ひっしゅう)です 
\\	〔ある者[物]は,ある時は〕 人々はあるいは海へあるいは山へと出かけた 
\\	バイオリンはあるいは強くあるいは弱[よわ]く聞こえた 
\\	〔もしかすると〕
\\	あるいはそうかもしれない 
\\	泡	あわ	
\\	合わせる	あわせる	
\\	〔一つにする〕
\\	2枚の紙を合わせる 
\\	彼は手を合わせて祈(いの)っていた 
\\	力を合わせて共通の利益を守った 
\\	声を合わせて歌った 
\\	〔加える,合計する〕
\\	これも勘定(かんじょう)に合わせてください 
\\	合わせて15個ある 
\\	合わせていくらになりますか 
\\	老若(ろうにゃく)合わせて5万人が会場に詰めかけた 
\\	四国と九州を合わせたほどの面積(めんせき)である 
\\	〔機械などを調整する〕
\\	〔楽器などの調子を合わせる〕
\\	メーターをゼロに合わせてくれ 
\\	時計は毎朝ラジオの時報(じほう)に合わせる 
\\	バイオリンの調子をピアノに合わせた 
\\	ラジオを
\\	の
\\	放送(ほうそう)に合わせてくれ 
\\	〔順応する〕
\\	都会の生活に合わせられなかった 
\\	話にうまく合わせた 
\\	彼はだれかれかまわず調子を合わせる 
\\	誰彼(だれかれ) 
\\	彼女は靴に合わせたバッグを持っていた 
\\	〔ぴったりさせる〕
\\	この場所に合わせてテーブルを作ってください 
\\	体に合わせてドレスを詰める 
\\	〔照合する〕
\\	在庫をリストと合わせる 
\\	私たちは書き留めたものを合わせた 
\\	帳尻(ちょうじり)を合わせる 
\\	〔そろえる〕 音楽に合わせて踊った 
\\	ピアノに合わせてフルートを吹いた 
\\	〔混ぜる〕 酢と醤油(しょうゆ)を合わせる 
\\	慌てる	あわてる	
\\	慌てて 
\\	慌てず 
\\	慌てるな 
\\	慌てるにはおよばない 
\\	慌ててコップを割ってしまった 
\\	火災警報(かさいけいほう)を聞いて慌てて飛び出した 
\\	どんな時でも彼は慌てない 
\\	たばこを吸っているのが見つかって少年は慌てた 
\\	哀れ	あわれ	
\\	〔かわいそうに思う気持ち〕
\\	哀れな 
\\	〔みじめな〕
\\	〔感動的な〕
\\	哀れを催(もよお)す光景(こうけい) 
\\	哀れを感じる 
\\	戦争で親を失った子供達は人々の哀れを誘(さそ)った 
\\	哀れな娘 
\\	哀れな姿 
\\	哀れな話 
\\	〔しみじみとした思い〕 ものの哀れを感じる人でなくてはこの詩は理解できない 
\\	世の哀れをしみじみ思う 
\\	〔感嘆〕
\\	((文)) 
\\	哀れ,その子は3歳にして両親に死に別れた 
\\	案	あん	
\\	〔計画〕
\\	〔提案〕
\\	ハイキングの案を立てる 
\\	図書館建設の案は流れた 
\\	山田氏がその制度をやめたらどうかという案を出した 
\\	組合の総会で太田氏の案が採択(さいたく)された 
\\	〔考え,意見〕
\\	案を練(ね)る 
\\	何かいい案はありませんか 
\\	〔予期〕 案にたがわず事はうまくいっている 
\\	案に相違(そうい)して資金が十分集まらなかった 
\\	案の定 
\\	((口)) 
\\	案の定,彼らは結婚した 
\\	彼は拒否するだろうと思っていたら案の定拒否した 
\\	案外	あんがい	〔思いがけなく〕
\\	〔驚いたことに〕
\\	⇒いがい(意外) 試験は案外難しかった[易しかった] 
\\	案外な結果だった 
\\	利益は案外少ない 
\\	暗記	あんき	
\\	安定	あんてい	
\\	安定したいす(椅子) 
\\	安定した社会 
\\	安定した心 
\\	あんなに	あんなに	
\\	余り	あまり	あまり【余り】 
\\	〔余った物,残り〕
\\	〔剰余〕
\\	彼が一番大きい部分を取り,余りを弟たちに分けた 
\\	給料(きゅうりょう)の余りで本を買った 
\\	食事の余りは捨てた 
\\	10を3で割ると商(しょう)は3で余りは1である 
\\	〔接尾辞(せつびじ, 
\\	として,以上〕 5年余り前に 
\\	余りある 彼女の長所(ちょうしょ)は欠点を補(おぎな)って余りある 
\\	憎んでもなお余りあるやつだ. 
\\	彼の死は惜しみてもなお余りある 
\\	〔度を越えること〕 あまりの暑さに皆,水ばかり飲んでいた 
\\	うれしさのあまり彼女は踊りだした 
\\	あまりあせるからへまをやったのだ 
\\	焦(あせ)る 
\\	へま 
\\	あまりといえばあんまりだ 
\\	あまりと言えば 
\\	あんまりだ 
\\	〔打ち消しを伴って,それほど,たいして〕 彼の容体(ようたい)はあまりよくない 
\\	彼女はあまり本を読まない 
\\	胃	い	
\\	胃の 
\\	胃が痛い 
\\	胃が丈夫だ[弱い] 
\\	胃が悪い 
\\	胃を壊(こわ)した 
\\	胃を洗浄(せんじょう)する 
\\	ゆうべの食事で胃の具合が悪くなった 
\\	芋(いも)は胃にもたれる 
\\	胃アトニー 
\\	胃液 
\\	胃炎 
\\	胃潰瘍(かいよう) 
\\	胃拡張(いかくちょう) 
\\	胃下垂 
\\	胃カタル 
\\	胃カメラ 
\\	胃カメラを飲む 
\\	胃癌(がん) 
\\	胃鏡 
\\	胃痙攣(けいれん) 
\\	胃切除 
\\	胃洗浄 
\\	胃洗浄をする 
\\	胃痛 ⇒詳細を見る 胃壁(いへき) 
\\	委員	いいん	
\\	意外	いがい	
\\	行き	いき 
\\	ゆき	
\\	息	いき	〔呼吸する空気〕
\\	〔呼吸〕
\\	息をする 
\\	息を吐く 
\\	息を吸う 
\\	息を止める 
\\	息が詰まる 
\\	彼らはすぐに息を切らした 
\\	あまり急いだので息が切れた 
\\	彼はすぐ息が切れる 
\\	息が荒く[速く]なった 
\\	彼は息が長く続く 
\\	発見されたとき彼女はまだ息があった 
\\	私の息がある限り 
\\	衿がきつくて息が苦しい 
\\	この部屋は換気が悪くて息が詰まりそうだ 
\\	息が合う 
\\	あの二人の俳優はぴったり息が合っている 
\\	二人はぴったり息が合っている〔仕事・スポーツで〕 
\\	息がかかる 
\\	あの人には山田氏の息がかかっている 〔庇護の下にある〕
\\	/〔影響下にある〕
\\	息が切れる 
\\	この事業はあまり金が掛かるので息が切れてしまいそうだ 
\\	息の下 
\\	苦しい息の下から彼は一人息子のことを私に頼んだ 
\\	息の長い 〔長期にわたる〕
\\	〔長い間続いている〕
\\	この地域の開発には息の長い計画が必要である 
\\	これは息の長い本である 
\\	教育には息の長い努力が必要だ 
\\	息の根(ね) 
\\	息の根を止める 
\\	あいつの息の根を止めてやる 
\\	息もつかせぬ 息もつかせぬ早業 
\\	息もつかせぬ早業を見せた 
\\	息を凝(こ)らす カーテンの後ろに隠れて,じっと息を凝らした 
\\	[((文))
\\	息を殺す 
\\	⇒息を凝らす 5歳の息子が屋根に立っているのを見て息を殺した 
\\	息をつく 
\\	〔息をする〕
\\	〔休む〕
\\	[((口)) 
\\	彼は息もつかずにしゃべり続けた 
\\	息をつく間もなかった 
\\	〔ほっとする〕 娘が結婚してほっと息をついた 
\\	息を抜く 息抜き 
\\	息を抜く暇もない忙しさだった 
\\	息を呑(の)む 
\\	息をのむような体操の演技 
\\	彼女は息をのむほどの美人である 
\\	/((口)) 
\\	息を引き取る 彼は今朝息を引き取った 
\\	/((文))
\\	息を吹き返す 
\\	〔生き返る〕
\\	[((英)) 
\\	〔意識を取り戻す〕
\\	15分ほど人工呼吸をすると溺れた子は息を吹き返した 
\\	〔よみがえる〕
\\	アールデコが息を吹き返した 
\\	勢い	いきおい	
\\	〔動きに伴う力〕
\\	風の勢い 
\\	あらしの勢いは全然衰(おとろ)えない 
\\	流れの勢いが強い 
\\	勢いよく蛇口(じゃぐち)から水がほとばしり出た 
\\	迸(ほとば)しり出る 
\\	〔元気,活気〕
\\	((英)) 
\\	大した勢いだね 
\\	彼は昔の勢いがなくなった 
\\	彼は勢いよく漕(こ)ぎ出した 
\\	漕ぎ出す 
\\	彼女のちょっとした一言で,彼らは勢いをそがれてしまった 
\\	そがれる  
\\	〔はずみ 
\\	調子〕 事の勢いで仕方がなかった 
\\	酔った勢いで彼は上役にねじ込んだ 
\\	その勢いなら3時までには終わるだろう 
\\	〔勢力〕
\\	その一派の勢いは衰えかけている 
\\	[((口))
\\	当時,豪族(ごうぞく)武田氏がこの地方で勢いを振るっていた 
\\	その地方のコレラの勢いはまだ衰えていない 
\\	〔自然の成り行きで〕 君がその気なら,勢い僕だって一生懸命やるぞ 
\\	生き物	いきもの	
\\	いけない	いけない	
\\	〔よくない〕
\\	⇒だめ(駄目) 私のどこがいけないのでしょうか 
\\	何ていけない子だろう 
\\	お風邪ですか,それはいけませんね 
\\	その上彼は無免許(むめんきょ)ときてるからますますいけない 
\\	君がそんなことを言うからいけないのだ 
\\	「今度の計画はどうですか」「いけないね」 
\\	胃を壊(こわ)しているから油っこいものはいけない 
\\	この封筒(ふうとう)は小さすぎていけない 
\\	あ,いけない,どうしよう 
\\	!は女性がよく用(もち)いる) いけない 〔しまった〕
\\	!/〔ちくしょう〕((卑)) 
\\	〔酒が飲めない様子〕 「いける口ですか」「いや,まったくいけません」 
\\	いける口 
\\	〔望み・見込みがない様子〕⇒だめ(駄目) あの病人はもういけないようです 
\\	試験はやっぱりいけなかった 
\\	2回に5点とられて,もういけないと分かった 
\\	〔「…するといけないから」の形で〕 列車に遅れるといけないので彼は早く発(た)った 
\\	子供が泣き出すといけないからもう帰ります 
\\	雨が降るといけないから傘をお持ちなさい 
\\	〔禁止〕
\\	この塀(へい)に張り紙をしてはいけない 
\\	/〔掲示〕
\\	芝生に入ってはいけない〔掲示〕 
\\	「テレビを見てはいけないかい」「いけないね」 
\\	〔必要〕
\\	「行かなくてはいけませんか」「いけませんね」 
\\	もっと早く申し込まなければいけなかったのだ 
\\	医師	いし	
\\	意思	いし	
\\	〔考え〕
\\	〔心づもり〕
\\	〔望み〕
\\	意思が通じる[疎通(そつう)する] 
\\	意思を変える 
\\	私は教師になる意思は毛頭(もうとう)ない 
\\	両者の意思の疎通(そつう)を欠(か)いたためその計画は挫折(ざせつ)した 
\\	彼は両親の意思に背(そむ)いて結婚した 
\\	彼は他人の意思に従ってばかりいる 
\\	/〔言い なりになる〕
\\	意思表示 
\\	〔法律用語〕
\\	意思表示する 
\\	意志	いし	
\\	((文)) 
\\	意志の強い[弱い]人 
\\	彼は不屈(ふくつ)の意志を持っている 
\\	彼女は自分の意志でこの計画に参加した 
\\	意志に反してそれをせざるを得なかった 
\\	せざるを得ない 
\\	疲れきっていたが意志の力で目的地にたどり着いた 
\\	たどり着く 
\\	彼女はいつも自分の意志を通した 
\\	彼の辞任(じにん)の意志は固い 
\\	意志決定 
\\	意志決定者 
\\	意志薄弱(はくじゃく) 
\\	維持	いじ	
\\	現状[世界平和]を維持する 
\\	面目(めんもく/めんぼく)を維持する 
\\	健康を維持する 
\\	生命(せいめい)を維持する 
\\	患者の生命を維持するための装置(そうち)が取り付けられた 
\\	会を維持する 
\\	その協会の維持はほとんど寄付金(きふきん)による 
\\	維持費 
\\	この古寺(こじ)の維持費に年額300万円かかる 
\\	意識	いしき	
\\	〔知覚〕
\\	意識を失う 
\\	/〔一時的に〕
\\	彼は意識がない 
\\	彼は3日間意識不明だった 
\\	意識を回復させる 
\\	[((英)) 
\\	彼は3時間後に意識を回復した 
\\	意識は最後まではっきりしていた 
\\	〔認識,自覚〕 意識する 
\\	意識的 (に) 
\\	彼女はだれかに後をつけられているのを意識していた 
\\	彼には罪の意識が全くない 
\\	意識の流れの手法(しゅほう) 
\\	彼が意識的にしたことだ〔故意に〕 
\\	〔個人や集団の思想・感情〕 階級[民族]意識 
\\	意識調査 大学生の政治に対する意識調査 
\\	意識不明 彼は意識不明で病院に運び込まれた 
\\	異常	いじょう	
\\	異常な 〔普通でない〕
\\	〔正常でない〕
\\	〔特異な〕
\\	((文)) 
\\	〔非凡な〕
\\	〔著しい〕
\\	異常に 
\\	その申し出に彼は異常な熱意をみせた 
\\	子供への打ち込み方は異常とも言える 
\\	異常な目つき 
\\	異常な才能 
\\	異常な進歩 
\\	異常乾燥注意報 
\\	異常気象(きしょう) 
\\	異常心理 
\\	異常心理学 
\\	異常性格 彼は異常性格者だ 
\\	異常接近 
\\	異常増殖 〔医学用語〕
\\	異常反応(はんのう) 
\\	異常隆起(りゅうき) 〔土地の〕
\\	意地悪	いじわる	
\\	そんなことをしたのは意地悪からだ 
\\	家主(やぬし/いえぬし)は意地悪そうな女だ 
\\	彼は意地悪く私を問い詰めた 
\\	問い詰(つ)める=
\\	意地悪には見えないけど 
\\	自分が酒が飲めないので妻にも飲ませないとは全く意地悪だ 
\\	泉	いずみ	
\\	いずれ	いずれ	
\\	〔二者のうちで〕
\\	〔たくさんのうちで〕
\\	〔両方〕
\\	あの二人はいずれも帰国子女(しじょ)だ 
\\	(帰国子女, 
\\	(海外子女, 
\\	いずれも彼の気に入らなかった 
\\	は二つのうち,
\\	は三つ以上) この子らはいずれも田舎(いなか)育ちだ 
\\	その実験はいずれの場合も失敗だった 
\\	〔どのみち〕 いずれにせよ 
\\	/((口)) 
\\	いずれにせよそれはばかげた話だ 
\\	いずれにしても彼がその列車に乗らなかったという事実は動かせない 
\\	〔そのうち〕
\\	((文)) 
\\	〔早晩〕
\\	((文)) 
\\	いずれ彼から連絡(れんらく)があるでしょう 
\\	は後で) いずれ加藤氏に詳細(しょうさい)を報告(ほうこく)いたします 
\\	いずれまたゆっくり話そう 
\\	いずれまたお便りいたします 
\\	いずれ釈放(しゃくほう)されるだろう 
\\	いずれの日にか故郷(こきょう)に錦(にしき)を飾(かざ)ることだろう 
\\	〔どこ〕
\\	⇒いずこ(何処) 昔の栄光(えいこう)今いずこ ((文))
\\	いずこへともなく立ち去った 
\\	/((口)) 
\\	以前	いぜん	
\\	〔その時より前に〕
\\	彼女は結婚以前には夫に一度会ったきりだった 
\\	朝8時以前には電話をしないでください 
\\	〔今より前〕 以前は (に) 
\\	以前の 
\\	以前の通り 
\\	以前の住所(じゅうしょ) 
\\	以前にお会いしませんでしたか 
\\	彼は以前とは全く人が変わってしまった 
\\	彼は以前ほど怒りっぽくなくなった 
\\	おうわさは以前から伺(うかが)っておりました 
\\	おうわさ 
\\	お噂
\\	板	いた	
\\	偉大	いだい	
\\	偉大な人 
\\	偉大な計画 
\\	彼は応用科学の分野で偉大な足跡を残した 
\\	抱く	いだく 
\\	だく	いだく 
\\	反感(はんかん)を抱く 
\\	不審(ふしん)を抱く 
\\	恨(うら)みを抱く 
\\	希望を抱く 
\\	私は彼に殺意を抱いた 
\\	だく 
\\	〔腕にかかえる〕
\\	彼女は人形を抱いていた 
\\	赤ん坊は母親に抱かれて眠っている 
\\	〔抱擁(ほうよう)する〕
\\	いたずら	いたずら	
\\	いたずら【▲悪▲戯】 〔悪さ〕
\\	〔悪ふざけ〕
\\	いたずらな 
\\	〔わんぱくな〕
\\	いたずらする 
\\	〔性的な〕
\\	いたずら盛(ざか)りの子供 
\\	いたずらっぽい目付きで 
\\	子供らがいたずらをしないよう見張っていなさい 
\\	呼(よ)び鈴(りん)にいたずらをしてはいけません 
\\	彼はいたずらが過ぎて父の不興を買った 
\\	私はほんのいたずら半分にそのつぼを焼いたのです 
\\	その破廉恥(はれんち)な教師は女生徒(じょせいと)にいたずらをしようとした 
\\	いたずら書き ⇒らくがき(落書き) 
\\	いたずら心 いたずら心を起こす 
\\	いたずらっ子 
\\	〔主に男の子〕
\\	いたずら電話 
\\	〔わいせつな〕
\\	いたずら【▲徒ら】 
\\	いたずらに時間を費(つい)やす 
\\	いたずらに見栄(みえ)を張る 
\\	我々はいたずらに奇跡(きせき)を待っていたのだ 
\\	痛み	いたみ	
\\	至る	いたる	
\\	〔達する〕⇒たっする(達する),とうちゃく(到着) この道は白根を経て万座に至る 
\\	至る渋谷〔地図などで〕 
\\	〔やってくる〕 実行の時至る 
\\	悲喜(ひき)こもごも至った 
\\	〔及ぶ〕
\\	社長から平社員に至るまで 
\\	夜半(やはん)に至るまで 
\\	今に至るまで彼から連絡がない 
\\	10月に至っても交渉(こうしょう)は妥結(だけつ)しなかった 
\\	事ここに至っては何とも致(いた)し方ない 
\\	〔…するようになる〕
\\	〔結果が…となる〕
\\	彼の誠意(せいい)を疑うに至った 
\\	些細(ささい)な意見の相違から夫妻は長く別居(べっきょ)するに至った 
\\	彼の優柔不断(ゆうじゅうふだん)から憂慮(ゆうりょ)すべき事態に至った 
\\	新しい職場(しょくば)でのストレスのため健康を害(がい)するに至った 
\\	ぼや(小火)は大事に至らず消(け)し止められた 
\\	市	いち 
\\	し	いち 
\\	蚤(のみ)の市 
\\	作物(さくぶつ)を市へだす 
\\	国際見本市 
\\	古本市で掘(ほ)り出(だ)し物をした 
\\	市の立つ日[町] 
\\	この通りは毎週一度市が立つ 
\\	[((英)) 
\\	この易者(えきしゃ)の評判は大したもので彼の家は常に門前市(もんぜんいち)を成すありさまだ 
\\	門前市を成す 
\\	し 
\\	〔行政区画〕
\\	市の 
\\	長野市 
\\	市債 
\\	市議会 ⇒詳細を見る 市葬 
\\	位置	いち	
\\	〔場所〕
\\	北極星(ほっきょくせい)は位置が変わらない 
\\	あの店は位置がいい 
\\	その道路標識(ひょうしき)は位置が悪い 
\\	位置につく 
\\	位置について〔競技の合図〕 
\\	静岡は東京と名古屋の中間に位置する 
\\	〔地位〕
\\	彼はその会社で高い位置にある 
\\	〔立場〕⇒たちば(立場) 位置エネルギー 
\\	一時	いちじ	
\\	一度に	いちどに	
\\	一度に二つの事に集中することはできない 
\\	みんな一度に入ろうとしては駄目だ 
\\	一度にどっと疲れが出た 
\\	市場	いちば	
\\	いつか	いつか	
\\	〔そのうちに〕
\\	((英)) 
\\	((英)) 
\\	いつか近い将来(しょうらい)にアフリカ旅行をするつもりだ 
\\	彼もいつかは目が覚めるだろう 
\\	いつかきっと伺(うかが)います 
\\	またいつか電話します 
\\	今回の損失(そんしつ)はいずれいつか埋め合わせがつくだろう 
\\	うそはいつか (は) ばれる 
\\	/((諺)) 
\\	いつか (は) その本当の意味が分かる時が来る 
\\	〔以前〕
\\	いつかお目に掛かりましたね 
\\	これがいつかお話しした本です 
\\	〔知らないうちに〕 いつかしら
\\	一家	いっか	
\\	一種	いっしゅ	
\\	一瞬	いっしゅん	
\\	一生	いっしょう	
\\	一生に一度の好機(こうき)だ 
\\	そんな好運(こううん)は一生に二度とない 
\\	人の一生は短い 
\\	彼はその仕事に一生を捧(ささ)げた 
\\	一生を安楽におくる 
\\	泣いて暮らすも一生,笑って暮らすも一生 
\\	彼女は一生独身で過ごした 
\\	一生の仕事 
\\	/〔一生かかる仕事〕
\\	一生のお願いだ 
\\	あなたのご親切は一生忘れません 
\\	[((米)) 
\\	彼はこの町で一生を終えた 
\\	一層	いっそう	
\\	いっそう悪くなる 
\\	大学に入ってからいっそう勉強するようになった 
\\	おなかがすいていたのでいっそうおいしかった 
\\	彼が参加してくれるならいっそう好都合(こうつごう)だ 
\\	今後一層のご支援(しえん)をお願いします 
\\	一体	いったい	
\\	〔一団〕
\\	〔統一体〕
\\	人々は上下(うえした/じょうげ)一体となって働いた 
\\	全市民は一体となって町の美化(びか)に努めた 
\\	〔「一体に」の形で,全般に〕⇒がいして(概して)
\\	〔強い疑問〕⇒いったいぜんたい(一体全体) 一体なぜ昨日来なかったのか 
\\	一体化 その地方の部族(ぶぞく)を一体化する 
\\	一体感 両者の間に一体感が生まれた 
\\	長い議論のあと,我々は皆一体感を感じた 
\\	一致	いっち	
\\	〔合致(がっち)〕
\\	〔調和〕
\\	一致する 
\\	これに関して彼の考えと一致した 
\\	彼らの見解(けんかい)は完全に一致している 
\\	私の意見はあなたのと一致しています 
\\	君は言行(げんこう)が一致していない 
\\	満場(まんじょう)一致で会長に選ばれた 
\\	父と母は私の進学先について意見が一致しない 
\\	市民は一致協力して大気汚染(たいきおせん)の排除(はいじょ)に努(つと)めた 
\\	〔偶然の符合〕
\\	〔呼応〕
\\	彼の引退と彼の息子の帰国とは時期が一致した 
\\	日本の県は米国の州と完全に一致するものではない 
\\	一致団結(いっちだんけつ) 
\\	一致団結して 
\\	一致点 
\\	両者の間に一致点を見出すのは困難(こんなん)だ 
\\	いつでも	いつでも	
\\	〔どんな時でも〕 太郎はいつでも出発できるように支度(したく)している 
\\	いつでも好きな時にいらっしゃい 
\\	いつでも結構(けっこう)です 
\\	〔常に〕
\\	彼女はいつでも勉強している 
\\	いつでも電車に乗り遅(おく)れるわけではない 
\\	いつのまにか	いつのまにか	
\\	一般	いっぱん	
\\	一方	いっぽう	
\\	〔一つの方面〕
\\	〔他の方面〕
\\	一方的 ⇒詳細を見る 住宅地(じゅうたくち)は町の一方に片寄っている 
\\	この家はもう一方にも入り口がある 
\\	彼の意見は一方に片寄っている 
\\	政府は一方では財政(ざいせい)再建(さいけん)を,もう一方では減税(げんぜい)を迫(せま)られている 
\\	〔立場の違う二者の片方〕
\\	〔相手方〕
\\	この件では一方を加害者,一方を被害者とは必ずしも言えない 
\\	〔それだけ〕 あの男はまじめ一方だ 
\\	川の水は増(ま)す一方である 
\\	氷は厚(あつ)くなる一方だった 
\\	〔他方では,話変わって〕 長男は社交的だが,一方,弟のほうは引っ込み思案(しあん)である 
\\	一方通行 
\\	〔掲示〕
\\	この道は一方通行(つうこう)だ 
\\	いつまでも	いつまでも	
\\	移動	いどう	
\\	民族の移動 
\\	遊牧民(ゆうぼくみん)たちは草を求(もと)めて絶(た)えず移動した 
\\	車を移動してください 
\\	違法駐車の車が警察官によって移動された 
\\	移動劇団 
\\	移動撮影(場面) 
\\	移動式
\\	移動証明 
\\	移動申告(しんこく) 
\\	移動診療所(しんりょうしょ) 
\\	移動性高気圧 
\\	移動大使 
\\	移動体通信事業 
\\	移動展覧会 
\\	移動図書館 
\\	((米)) 
\\	移動病院 
\\	従兄弟	いとこ	
\\	従姉妹	いとこ	
\\	稲	いね	
\\	居眠り	いねむり	
\\	居眠りする 
\\	〔こっくりする〕
\\	手紙を書きながら居眠りする 
\\	電車の中で居眠りをする 
\\	午後の授業で生徒たちは居眠りを始めた 
\\	居眠り運転 居眠り運転する 
\\	命	いのち	
\\	違反	いはん	
\\	条約(じょうやく)に違反する行為 
\\	交通違反は徹底的(てっていてき)に取り締(し)まらねばならない 
\\	君の行為は校則(こうそく)違反である 
\\	駐車違反で罰金(ばっきん)を払わされた 
\\	違反者 
\\	衣服	いふく	
\\	居間	いま	
\\	今に	いまに	
\\	〔今になっても〕 あれだけは今に忘れられない出来事だ 
\\	〔やがて,間もなく〕 今に分かるだろう 
\\	今に彼も後悔するだろう 
\\	今に手紙をくれるだろう 
\\	今にみていろ 
\\	今にも	いまにも	
\\	今にも降り出しそうな空模様(もよう)だ 
\\	彼女は今にも泣き出しそうだった 
\\	否	いや	
\\	以来	いらい	
\\	〔その後〕
\\	あれ以来彼に会っていない 
\\	彼女は出産以来ずっと健康がすぐれない 
\\	初めて彼女に会って以来,すっかり魅(ばか)せられてしまった 
\\	彼が総督(そうとく)になって以来5年になる 
\\	依頼	いらい	
\\	〔願い,頼み〕
\\	依頼する 
\\	一つ依頼のすじがあって伺(うかが)いました 
\\	推薦状(すいせんじょう)を書いていただくようご依頼申し上げたいのですが 
\\	ご依頼申し上げたいことがあるのですが 
\\	依頼により 
\\	品物は依頼どおりに航空便で送った 
\\	〔頼ること〕
\\	依頼する 
\\	あなたは親に依頼しすぎている 
\\	〔委任〕 人に物事を依頼する 
\\	歌詞を詩人に依頼する 
\\	その事は弁護士に依頼してあります 
\\	依頼状 
\\	依頼心 
\\	あの男は依頼心が強すぎる 
\\	依頼人 〔弁護士などの〕
\\	いらいら	いらいら	
\\	⇒いらだつ(苛立つ) 交通渋滞(じゅうたい)に巻き込まれていらいらした 
\\	隣家(りんか)のピアノの音で彼女はいらいらしていた 
\\	彼のかん高い声を聞くといらいらする 
\\	人を馬鹿にしたような彼女の笑いはいらいらする 
\\	彼女はいらいらしながら夫の帰りを待っていた 
\\	あの人はすぐいらいらする 
\\	いらっしゃい	いらっしゃい	
\\	医療	いりょう	
\\	岩	いわ	
\\	祝い	いわい	
\\	祝う	いわう	〔人を〕
\\	〔事柄を〕
\\	誕生日を祝って 
\\	ひな祭を祝う 
\\	写真コンクール一等賞のご受賞を心からお祝い申し上げます 
\\	ご結婚をお祝い申し上げます 
\\	昨年学校の創立50周年記念日を祝った 
\\	君の門出(かどで)を祝って乾杯だ 
\\	言わば	いわば	
\\	彼は言わば成人した赤ん坊だ 
\\	いわゆる	いわゆる	
\\	〔軽べつ的に〕
\\	これがいわゆる二・二六事件である 
\\	彼らはいわゆる暴走族(ぼうそうぞく)だ 
\\	印刷	いんさつ	
\\	印象	いんしょう	
\\	よい[悪い]印象 
\\	第一印象 
\\	彼の印象はよかった 
\\	その本は誤った日本の印象を与えている 
\\	京都の印象はいかがでしたか 
\\	その一家は困窮(こんきゅう)しているという印象を受けた 
\\	印象的な光景 
\\	彼女の優雅(ゆうが)な身のこなしが印象深かった 
\\	印象主義 
\\	印象派 
\\	印象批評 
\\	引退	いんたい	
\\	引退する 
\\	引退している 
\\	彼は28歳でプロテニスを引退すると公表した 
\\	彼が現役を引退したのは何歳の時でしたか 
\\	引退興行 〔相撲の〕
\\	〔舞台などの〕
\\	引用	いんよう	
\\	引用する 〔例示〕
\\	〔引証〕
\\	この句はミルトンからの引用です 
\\	彼はしばしばヘンリー・スウィートから引用する 
\\	引用書 
\\	引用書目 
\\	引用符(ふ) 
\\	引用文 
\\	魚	うお	
\\	うがい	うがい	
\\	うがいする 
\\	〔口をすすぐ〕
\\	うがい薬 
\\	受け取る	うけとる	
\\	〔受領する〕
\\	〔もらう〕
\\	〔よろこんで〕
\\	お手紙はまだ受け取っていません 
\\	金百万円正(しょう)に受け取りました 
\\	ご好意は喜んで受け取りますが,お金は受け取れません 
\\	それに対して会社からなにがしかの謝礼(しゃれい)を受け取っている 
\\	何某(なにがし)かの金 
\\	受験票(じゅけんひょう)は願書(がんしょ)を受け取り次第発送(はっそう)します 
\\	受け取られたらご一報(いっぽう)ください 
\\	〔解釈する〕 彼の言うことはそのままには受け取れない 
\\	そうとしか受け取れないね 
\\	私のこと[したこと]を悪く受け取らないで欲しい 
\\	ジェイは私の言葉を文字どおりに[割り引いて]受け取ったようだ 
\\	人の言葉を間違って受け取る 
\\	動かす	うごかす	
\\	兎	うさぎ	
\\	牛	うし	
\\	失う	うしなう	
\\	疑う	うたがう	
\\	宇宙	うちゅう	
\\	討つ	うつ	
\\	〔敵(かたき)をとる〕 いつかは敵を討つぞ 
\\	〔討伐(とうばつ)する〕
\\	反逆者(はんぎゃくしゃ)を討つ 
\\	天に代わって不義を討つ 
\\	撃つ	うつ	
\\	〔発射・射撃(しゃげき)する〕
\\	銃を撃つ 
\\	鳥を撃つ 
\\	脚を撃たれた 
\\	撃て 
\\	3発撃って1発当たった 
\\	〔攻撃する〕
\\	敵の背後(はいご)を撃つ 
\\	うっかり	うっかり	
\\	〔ぼんやり〕 うっかり約束を忘れてしまった 
\\	うっかりして駅を3つも乗り過ごしてしまった 
\\	〔不注意〕 すりが多いからうっかりできません 
\\	うっかり兄の本を持ってきてしまった 
\\	うっかりして招待状(しょうたいじょう)をなくしてしまった 
\\	うっかり本心(ほんしん)をしゃべってしまった 
\\	映す	うつす	
\\	〔反射するものに姿・形を〕
\\	彼女は鏡に顔を映した 
\\	濠(ほり)は城(しろ)を映していた 
\\	子供は親を映す鏡だ 
\\	〔投映する〕
\\	スライドを映す 
\\	映画を映す 
\\	訴える	うったえる	
\\	〔告訴する〕〔裁判所に〕
\\	〔警察などに〕
\\	私は阿部氏を債務(さいむ)不履行(ふりこう)で訴えた 
\\	彼は暴行(ぼうこう)罪で訴えられた 
\\	被害者が警察に訴え出たことから問題が明るみに出た 
\\	〔不平などを言う〕
\\	患者は苦痛(くつう)を訴えた 
\\	人々は窮状(きゅうじょう)を訴えた 
\\	窮状=
\\	この本は何を訴えようとしているのか 
\\	〔解決の手段として頼る〕
\\	腕力(わんりょく)に訴える 
\\	法の力に訴えるよりほかない 
\\	〔人の心を動かす〕
\\	心に訴える歌 
\\	この写真は見る者の胸に訴えるだろう 
\\	この演説は国民に訴える力がない 
\\	少女は訴えるように黙って私を見た 
\\	写る	うつる	
\\	〔透(す)けて見える〕 怪(あや)しい影が障子に写った 
\\	〔写真などに撮られる〕 真ん中に写っている人は私の父です 
\\	この写真では赤ん坊が実にかわいらしく写っている 
\\	昨日テレビに僕たちの学校が写っていた 
\\	〔カメラ・写真が撮れる〕 これはよく写るカメラだ 
\\	露出(ろしゅつ)不足で写真がよく写らなかった 
\\	映る	うつる	
\\	〔鏡・水面などに反射して見える〕
\\	藤(ふじ)の花が水に映っている 
\\	彼女は鏡に映った自分の姿ににっこりと笑った 
\\	目に映るものは皆美しかった 
\\	彼の目には彼女が天使のように映った 
\\	〔映像として見える〕 このテレビはうまく映らない 
\\	〔色などが調和する〕
\\	深紅(しんこう)のばらがその花瓶によく映っている 
\\	あなたは色が白いからどんな色でもよく映る 
\\	うなる	うなる	
\\	〔猛獣・犬などが低い声を出す〕
\\	おおかみは低くうなった 
\\	〔うめく〕
\\	彼らは傷(きず)の痛みでうなっている 
\\	〔うなるように響く〕 弾丸(だんがん)がうなる中を私は駆け抜けた 
\\	機械がぶーんとうなっている 
\\	風がうなりながら木々の間を吹き抜けた 
\\	〔うたう〕 浪曲(ろうきょく)をうなる 
\\	〔驚嘆(きょうたん)する〕
\\	良子は英語ですばらしいスピーチをやってのけて全校をうならせた 
\\	ファインプレイに観衆(かんしゅう)はうなった 
\\	〔あり余る〕 
\\	彼女はうなるほどお金を持っている ((口)) 
\\	/((口)) 
\\	奪う	うばう	
\\	〔力ずくで〕
\\	特に戦争や動乱の折に) 財産(ざいさん)を奪われた 
\\	王位(おうい)はおじに奪われた 
\\	飛行機の墜落(ついらく)で50名の命が奪われた 
\\	そのストライキで2万人の通勤の足が奪われた 
\\	相手チームの投手からヒット15本を奪う 
\\	〔注意・心を〕
\\	〔人の目を眩惑(げんわく)する〕
\\	彼は孫のしぐさ(仕草)に注意を奪われていた 
\\	観客は彼女のみごとな演技に心を奪われた 
\\	彼はその少女の美しさに心を奪われた 
\\	馬	うま	
\\	生まれ	うまれ	
\\	有無	うむ	
\\	〔有る無し〕 経験の有無によって給料(きゅうりょう)が違う 
\\	欠席者(けっせきしゃ)の有無を確(たし)かめる 
\\	有無相通じる あの夫婦は有無相通(あいつう)じるところがあるらしい 
\\	有無を言わさず 
\\	彼は有無を言わさず連れて行かれた 
\\	有無を言わさず彼女は部屋から追い出された 
\\	梅	うめ	
\\	埋める	うめる	
\\	〔うずめる〕
\\	〔おおう〕
\\	瓶(かめ)を地中(ちちゅう)に埋めた 
\\	マフラーに顔を半分埋める 
\\	〔空きをふさぐ〕
\\	壁の穴をセメントで埋める 
\\	余白(よはく)は挿(さ)し絵(え)で埋めよう 
\\	〔一杯にする〕 本棚を漫画本で埋める 
\\	〔補(おぎな)う〕
\\	赤字を埋めるために時計を売った 
\\	〔ぬるめる〕 
\\	風呂をうめる 
\\	裏切る	うらぎる	
\\	羨ましい	うらやましい	
\\	〔ねたみを抱かせるような〕
\\	〔うらやましく思う〕
\\	うらやましそうに 
\\	うらやましい地位 
\\	君がうらやましいなあ 
\\	彼の成功がうらやましかった 
\\	彼の幸運もうらやましくなかった 
\\	うらやましがらせないで 
\\	がる 
\\	うらやましくて君の悪口を言っている人がいる ((口)) 
\\	売れる	うれる	
\\	噂	うわさ	
\\	運	うん	
\\	運転	うんてん	
\\	〔機械などを動かすこと〕 運転する 〔機械を〕
\\	〔車を〕
\\	〔機械・バスや電車の便などを〕
\\	機械を運転する 
\\	自動車の運転を習う〔教習を受ける〕 
\\	機械は調子よく運転している 
\\	その機械は運転中である 
\\	新工場の運転開始は9月だ 
\\	エレベーターは運転休止です 
\\	8月は臨時(りんじ)列車[バス]が運転される 
\\	バスは15分間隔(かんかく)で運転されている 
\\	酔っ払い運転でつかまった 
\\	名古屋で運転を代わってやった 
\\	〔資金(しきん)などを回転すること〕 資金の運転は効率よくすべきだ 
\\	運転系統 
\\	運転士[手,者] 〔自動車の〕
\\	〔お抱えの〕
\\	〔列車の〕
\\	〔電車の〕
\\	〔機械の〕
\\	〔船の〕
\\	運転資金 
\\	運転室 
\\	運転台[席] 〔自動車の〕
\\	〔電車の〕
\\	運転費 
\\	運転免許試験 
\\	運転免許証 
\\	((英)) 
\\	永遠	えいえん	
\\	永久	えいきゅう	
\\	影響	えいきょう	
\\	営業	えいぎょう	
\\	〔業務の運営〕
\\	営業する 〔事業を行う〕
\\	〔営業している〕
\\	個人営業の店 
\\	風俗(ふうぞく)営業 
\\	営業を始める 
\\	人の営業を妨害(ぼうがい)する 
\\	日曜日を除(のぞ)き毎日営業 
\\	営業中〔掲示〕 
\\	営業案内 
\\	営業外収益(がいしゅうえき) 
\\	営業許可 
\\	営業権 
\\	営業時間 
\\	銀行の営業時間は9時から3時である 
\\	営業実績(じっせき) 
\\	営業収入 
\\	営業主任 
\\	営業所 
\\	営業スタッフ 
\\	営業停止 
\\	レストランが営業停止になった 
\\	営業費 
\\	営業品目(しなめ) 
\\	〔サービス業〕 
\\	営業部 
\\	営業部長 
\\	営業不振 会社は営業不振(ふしん)で20人の人員(じんいん)削減(さくげん)を行った 
\\	は一時的な削減) 営業妨害 
\\	営業報告 
\\	営業方針 
\\	営業利益 
\\	衛星	えいせい	
\\	栄養	えいよう	
\\	笑顔	えがお	
\\	描く	えがく	
\\	〔鉛筆・ペンなどで〕
\\	〔絵筆で〕
\\	びょうぶには牡丹(ぼたん)が描かれていた 
\\	〔形づくる〕
\\	コンドルが円を描いて飛んでいる 
\\	稜線(りょうせん)がゆるやかな弧(こ)を描いていた 
\\	〔表現する〕
\\	((文)) 
\\	バルザックは当時の社会の多様(たよう)な現象(げんしょう)を生き生きと描いた 
\\	〔思い浮かべる〕 美しい風景を心に描く 
\\	餌	えさ	
\\	得る	える	
\\	〔手に入れる〕
\\	〔不断の努力で得る・習得する〕
\\	〔遂げる,達する〕
\\	〔努力・要求・懇願で獲得する〕
\\	〔調達する〕((文)) 
\\	〔首尾よく勝ち得る〕
\\	〔確保する〕
\\	金を得る 〔もうける〕
\\	/〔生活費を〕
\\	賞を得る 
\\	人の好意を得る 
\\	地位[富(とみ)]を得る 
\\	名声(めいせい)を得る 
\\	資格(しかく)を得る 
\\	権力を得る 
\\	彼はやっと両親の許可を得た 
\\	なんとか確証(かくしょう)を得ることができた 
\\	投票者の支持を得た 
\\	彼の知己(ちき)を得たのは幸いだった 
\\	彼の話は何も得る所がない 
\\	〔できる〕
\\	⇒できる(出来る)7 敵の所在を探知(たんち)し得た 
\\	…せざるを得ない 
\\	ざるをえない
\\	ざるを得ない) やむを得ない 
\\	延期	えんき	
\\	〔会議などの休会,中断〕
\\	延期する 〔開始を遅らせる〕
\\	〔休会する〕
\\	遠足(えんそく)は雨のため1日延期された 
\\	手紙の返事を書くのを数日(すうじつ)延期した 
\\	刑(けい)の執行(しっこう)が延期された 
\\	会議は明後日(あさって)まで延期になった 
\\	演技	えんぎ	
\\	〔劇などの〕
\\	体操やフィギュアスケートの場合にも使える) 彼は立派な演技を見せた 
\\	彼は演技がうまい[まずい] 
\\	彼の演技が下手なため芝居はだいなしだった 
\\	〔見せ掛け〕 彼女はどこまでが演技なのか分からない 
\\	首相の親米政策は単なる演技だ 
\\	演技者 
\\	演技派 演技派女優 
\\	演技力 
\\	援助	えんじょ	
\\	演説	えんぜつ	
\\	演 
\\	説 
\\	〔公式の〕
\\	〔厳粛・荘重な〕
\\	〔演説すること〕
\\	演説する 
\\	彼は原稿(げんこう)を見ながら演説した 
\\	彼女は演説がうまい[まずい] 
\\	演説口調(くちょう)で話す 
\\	演説家 
\\	演説会 
\\	演説会場 
\\	演奏	えんそう	
\\	老い	おい	〔老齢〕
\\	老いも若きも 
\\	老いを忘れる 
\\	老いのくり言 
\\	老いの一徹(いってつ) 老いの一徹で 
\\	追い付く	おいつく	
\\	バス停で彼女に追い付いた 
\\	〔他と同じ状態・能力に達する〕 西欧の水準に追い付く 
\\	この製品には生産が需要(じゅよう)に追い付く見通しがない 
\\	相手チームと同点に追い付く 
\\	(能力の上で) 彼に追い付く者はいない 
\\	いまさら後悔しても追い付かない 
\\	王	おう	
\\	追う	おう	
\\	〔追い掛ける〕
\\	〔犯人・獲物などを〕
\\	獲物(えもの)を追う 
\\	泥棒[犯人]を追う 
\\	追う者と追われる者 
\\	彼は母親の後を追うように死んだ 
\\	彼は警察に追われている 
\\	追いつ追われつの接戦に観客はわいた 
\\	〔駆り立てる〕 
\\	家畜(あちく)を追う 
\\	〔追い払う〕⇒おいはらう(追い払う) 
\\	〔遠くに追いやる〕 うるさい酔っ払いを追い払った 
\\	猫をしっと追い払った 
\\	彼はセールスマンを追い払った ((口)) 
\\	彼女は彼から悪霊を追い払えると言った 
\\	祈祷などで) 
\\	〔邪魔なものを追い出す〕 彼をその役[地位]から追い払った 
\\	邪悪な考えを追い払った 
\\	〔追放する〕⇒おいだす(追い出す),ついほう(追放) 地位を追われる 
\\	故郷を追われた 
\\	国を追われる 
\\	〔受身の形で,せかされる〕せかす 
\\	仕事[時間]に追われる 
\\	予習(よしゅう)に追われる 
\\	〔追い求める〕
\\	理想を追う 
\\	[((文))
\\	流行を追う 
\\	〔順序などに従う〕 先例を追う 
\\	順を追って説明します 
\\	日を追って暖かくなってきた 
\\	年を追って成熟(せいじゅく)してきた 
\\	応援	おうえん	
\\	王様	おうさま	
\\	王子	おうじ	
\\	応じる	おうじる	
\\	〔答える〕
\\	ご質問に応じて 
\\	彼の呼び掛けに応じて全国から同志(どうし)が集まった 
\\	〔従う〕 
\\	社員募集(ぼしゅう)に応じる 
\\	提案に二つ返事で応じる 
\\	二つ返事 
\\	先方(せんぽう)の条件に応じる 
\\	注文[招待/挑戦]に応じる 
\\	呼び出しに応じる 
\\	ご希望に応じて 
\\	時に応じて 
\\	必要に応じて 
\\	〔ふさわしくする〕(ふさわしい 
\\	需要(じゅよう)に応じる 
\\	身分[能力]に応じて 
\\	収入に応じて税金が課(か)される 
\\	勤務年限(ねんげん)に応じて退職金が異なる 
\\	年限 
\\	横断	おうだん	
\\	〔渡ること〕 横断する 
\\	((文)) 
\\	道路を横断する 
\\	交通規則を無視して道路を横断する 
\\	ここで道路が線路を横断している 
\\	〔東西に通り抜けること〕 大陸横断鉄道 
\\	[((米)) 
\\	谷[砂漠]を横断する 
\\	飛行機で太平洋を横断する 
\\	横断線 
\\	横断歩道 
\\	((米)) 
\\	((英)) 
\\	横断幕 
\\	横断面 
\\	終える	おえる	
\\	大いに	おおいに	
\\	大変 (たいへん)), 
\\	覆う	おおう 【覆う・▲被う・×蓋う・×蔽う】	
\\	〔上にかぶせる〕
\\	雪が地面を覆った 
\\	ベールで顔を被(おお)う 
\\	目を覆う惨状(さんじょう)であった 
\\	〔一面に広がる〕 雲が空を覆った 
\\	〔包み隠す〕
\\	⇒おおいかくす(覆い隠す) 
\\	〔広くゆきわたる〕 反戦気運が全国を覆った 
\\	大家	おおや	
\\	丘	おか	
\\	沖	おき	
\\	〔公海〕
\\	下田沖で 
\\	1キロ沖に漂流物(ひょうりゅうぶつ)がある 
\\	沖に白波(しらなみ)が立ち始めた 
\\	奥	おく	
\\	贈る	おくる	
\\	〔物をあげる〕
\\	〔賞・記念品などを与える〕
\\	人に時計を贈る 
\\	学校にピアノを贈った 
\\	〔称号・栄誉などを授ける〕 大学から彼に博士号(はくしごう)が贈られた 
\\	彼の絵は金賞を贈られた 
\\	起こる	おこる	
\\	(生じる)
\\	(戦争・火事が)
\\	(…の結果として)
\\	(…によって引き起こされる)
\\	(起因する)
\\	〔生じる〕
\\	何事が起ころうと慌(あわ)ててはいけない 
\\	恐ろしい事故が起こった 
\\	校内で暴力事件が起こった 
\\	聴衆(ちょうしゅう)から笑いが起こった 
\\	隣人(りんじん)の間にいざこざ[論争(ろんそう)/けんか(喧嘩)]が起こった 
\\	群衆の間に驚きの声が起こった 
\\	波[風]が起こってきた 
\\	この町に大きな変化が起こった 
\\	この市に新しい産業が起こった 
\\	〔起因(きいん)する〕
\\	それはすべて私の誤解(ごかい)から起こった 
\\	紛争(ふんそう)は住民(じゅうみん)エゴから起こった 
\\	今度の失敗(しっぱい)は僕の過信(かしん)から起こった 
\\	〔起源(きげん)を持つ〕
\\	この儀式は平安時代に起こった 
\\	インダス川はヒマラヤ山脈(さんみゃく)に起こる 
\\	〔熱・電気が発生する〕
\\	木をこすると熱が起こった 
\\	化繊(かせん)の下着(したぎ)は静電気(せいでんき)が起こりやすい 
\\	〔病気が出てくる〕 頭痛が起こった 
\\	ヒステリーの発作が起こった 
\\	押さえる	おさえる	
\\	(動かぬようにする)
\\	(逮捕する)
\\	(予約する)
\\	(理解している)
\\	〔動かぬようにする〕
\\	文鎮で紙を押さえる 
\\	はしごを押さえていてくれ 
\\	風に飛ばされないように戸をしっかりと押さえていた 
\\	馬を押さえる 
\\	はくつわをつけて) 犬を押さえる 
\\	動けないようにしっかりと彼の手足を押さえた 
\\	〔ふさぐ〕 傷口を押さえる 
\\	耳を押さえる 
\\	手[ハンカチ]で口を押さえる 
\\	目頭(めがしら)を押さえる 
\\	〔つかまえる〕
\\	犯人を押さえる 〔捕える〕
\\	/〔逮捕する〕
\\	すりの現場を押さえる 
\\	彼の脱税の証拠を押さえた 
\\	重要書類が捜査当局の手に押さえられた 
\\	問題の要点を押さえる 
\\	差し押さえる 
\\	〔法律用語〕
\\	彼は財産を差し押さえられた 
\\	警察は密輸された銃を差し押さえた 
\\	幼い	おさない	
\\	〔幼少の〕
\\	幼い子供 
\\	幼いころ 
\\	〔幼稚な〕
\\	〔未熟な〕
\\	幼い考え 
\\	収める	おさめる	
\\	〔きちんと入れる〕
\\	元の所に収めておく 
\\	金は金庫(きんこ)に収めた[収めてある] 
\\	刀をさやに収めた 
\\	彼の詩はその詩選集(せんしゅう)に収められている 
\\	〔得る〕
\\	満足な結果を収める 
\\	勝利(しょうり)を収める 
\\	投手の交替(こうたい)が効果を収めた 
\\	彼は権力(けんりょく)を一手(いって)に収めた 
\\	一手に 
\\	彼は海外貿易(ぼうえき)で大きな利益(りえき)を収めた 
\\	[((文))
\\	納める	おさめる	
\\	〔受け取る〕 
\\	つまらぬものですが感謝の志(こころざし)をお納めください 
\\	〔金・品物を納入(のうにゅう)する〕 
\\	部品は水曜までに納めてください 
\\	税金を納める 
\\	授業料を納める 
\\	〔終える〕
\\	舞(ま)い納めると彼女は舞台から消(き)えた 
\\	〔しまう〕 腹に納めかねることだ 
\\	かねる 
\\	これは私の胸に納めておこう 
\\	治める	おさめる	
\\	〔乱れを鎮める〕 暴動を治める 
\\	なんとか丸く治めたいのですが 
\\	〔統治する〕 国を[家を]治める 
\\	チャールズ一世は1600年から1649年までイギリスを治めた 
\\	お辞儀	おじぎ	
\\	お洒落	おしゃれ	
\\	おしゃれな人 
\\	彼女はたいそうおしゃれだ 
\\	彼女はおしゃれをして現れた 
\\	お喋り	おしゃべり	
\\	汚染	おせん	
\\	恐らく	おそらく	〔ほとんど確実に〕
\\	〔たぶん〕
\\	恐らくそれは見つからないだろう 
\\	恐れる	おそれる	
\\	恐ろしい	おそろしい	
\\	〔怖い〕
\\	〔どうもうな〕
\\	〔相手として手ごわい〕
\\	恐ろしい飛行機事故 
\\	恐ろしい目にあった 
\\	/〔ぎょっとした〕((口)) 
\\	目に遭(あ)う 
\\	恐ろしくてがたがた震えた 
\\	恐ろしい猛獣(もうじゅう) 
\\	そんな恐ろしい目で見ないで 
\\	恐ろしい相手 
\\	その辺は恐ろしい所だから近寄らないように 
\\	〔非常な〕
\\	((口)) 
\\	恐ろしい速度で運転する 
\\	恐ろしく酒を飲む男だ ((口)) 
\\	恐ろしく寒い 
\\	教わる	おそわる	
\\	穏やか	おだやか	
\\	〔静かで平穏(へいおん)な〕 穏やかな 
\\	明日は穏やかな天気になるでしょう 
\\	穏やかな処置をとる 
\\	それは穏やかな解決法ではない 
\\	穏やかに話し合おう 
\\	穏やかに暮らしたいものだ 
\\	あの問題は穏やかに解決された 
\\	[((文))
\\	心中穏やかならぬものがあった 
\\	辞表(じひょう)を出すとは穏やかでない 
\\	[((口)) 
\\	〔態度が落ち着いて,もの静かな〕 穏やかな 
\\	あの人は穏やかな人だ 
\\	男の人	おとこのひと	
\\	大人しい	おとなしい	
\\	〔静かにして騒がない〕
\\	〔行儀のよい〕
\\	おとなしいお嬢(じょう)さん 
\\	この猫は随分(ずいぶん)おとなしいね 
\\	おとなしくしていらっしゃい 
\\	おとなしくしないと怒るぞ 
\\	〔はで(派手)でない〕 これはおとなしい柄(がら)だ 
\\	彼女はおとなしい色が似合う 
\\	劣る	おとる	
\\	この点で私は彼に劣る 
\\	暗算(あんざん)ではだれにも劣らなかった 
\\	彼は獣(けもの)にも劣る 
\\	彼は亡くなった父親に劣らず親切な人だ 
\\	鬼	おに	
\\	帯	おび	
\\	お昼	おひる	
\\	おぼれる	溺れる	
\\	〔溺死(できし)する〕
\\	湖でおぼれ死んだ 
\\	おぼれかかったところを助かった 
\\	おぼれかけた人を助けた 
\\	はおぼれて死んでしまった人) おぼれる者はわらをもつかむ ((諺)) 
\\	〔夢中になる〕 女の愛におぼれる 
\\	ギャンブルにおぼれる 
\\	酒におぼれて暮らす 
\\	/((口)) 
\\	おめでとう	おめでとう	
\\	思い付く	おもいつく	
\\	思い出	おもいで	
\\	主に	おもに	
\\	思わず	おもわず	〔気が付かずに〕
\\	〔自分の意志に反して〕
\\	〔本能的に〕
\\	思わずため息が出る 
\\	思わず笑い出す 
\\	車がぶつかりそうになって思わず目をつぶった 
\\	泳ぎ	およぎ	
\\	およそ	およそ	
\\	及ぼす	およぼす	
\\	台風は田畑に甚大(じんだい)な被害を及ぼした 
\\	それはよい[悪い]影響を彼らに及ぼした 
\\	たばこは健康に悪い影響を及ぼす 
\\	下ろす	おろす	
\\	下(くだ)す 
\\	(高所から)
\\	(ブラインド・幕などを)
\\	(荷物を)
\\	(船荷を)
\\	(錨{いかり}を)
\\	(車から)
\\	(地位から)
\\	(胎児を)
\\	(新品を)
\\	(預金を)
\\	(錠を)
\\	(大根などを)
\\	(魚を)
\\	〔上から下に移す〕 (上げた) 手を下ろす 
\\	棚(たな)から辞書を下ろす 
\\	やかんを火から下ろす 
\\	ボートを下ろす 
\\	は水面に浮かべる) 幕(まく)を下ろす 
\\	シャッターを下ろす 
\\	上げを下ろす 
\\	いかりを下ろす 
\\	木が根(ね)を下ろした 
\\	いすに腰を下ろす 
\\	〔掲げてあるものを外す〕 表札(ひょうさつ)を下ろす 
\\	かんばんをおろす【看板を下ろす】 
\\	〔閉店する〕
\\	〔店を畳む〕
\\	〔風が高い所から吹く〕 山を下ろす冷たい風 
\\	〔そったり切ったりして落とす〕 髪を下ろす 
\\	〔仏門に入る〕
\\	[〔尼〕
\\	枝を下ろす 
\\	〔錠をかける〕 戸の錠を下ろす 
\\	〔魚肉(ぎょにく)を切る〕 
\\	魚を三枚(さんまい)に下ろす 
\\	〔下ろし金で〕
\\	大根を下ろす 
\\	〔新品(しんぴん)を使う〕 
\\	新しいタオルを下ろす 
\\	新しい背広を下ろす 
\\	下ろし立ての帽子 
\\	〔引き出す〕 
\\	貯金を下ろす 
\\	〔妊娠中絶(にんしんちゅうぜつ)をする〕 子を下ろす 
\\	降ろす	おろす	
\\	〔乗物から人を出す〕 どこで降ろしましょうか 
\\	駅の前で降ろしてください 
\\	彼は手を貸して老人をバスから降ろしてあげた 
\\	運転手は乗客をみんなバスから降ろしてタイヤを調べた 
\\	〔荷物を出す〕 車から荷(に)を降ろす 
\\	船の荷を降ろしている 
\\	肩の荷を降ろす ⇒に(荷) 
\\	〔旗などを下げる〕 旗を降ろす 〔下げる〕
\\	/〔降参の合図・敬礼として〕
\\	〔地位を退かせる〕 彼は主役から降ろされた 
\\	卸す	おろす	
\\	小売値(こうりね)の7掛けで卸す 
\\	音	おん	
\\	恩	おん	〔受けた好意〕
\\	((英)) 
\\	〔恩義〕
\\	ご恩は一生忘れません 
\\	恩に着せる 
\\	あの人は何かやってくれるとすぐ恩に着せる 
\\	恩に着る 
\\	ありがたい,恩に着ます 
\\	恩を仇(あだ)で返す 
\\	恩を売る 
\\	あいつには昔から恩を売ってあるから私の頼みを断れないさ 
\\	温暖	おんだん	
\\	温度	おんど	
\\	蚊	か	
\\	課	か	
\\	貝	かい	
\\	害	がい	
\\	会員	かいいん	
\\	絵画	かいが	
\\	海外	かいがい	
\\	会計	かいけい	
\\	〔金銭の出し入れの仕事〕
\\	一般会計 
\\	特別会計 
\\	一般会計予算 
\\	彼はあの店で会計を担当(たんとう)している 
\\	〔勘定〕
\\	〔勘定書〕
\\	((米)) 
\\	〔支払い〕
\\	会計をお願いします〔飲食店などで〕 
\\	会計を済ます 
\\	会計課 
\\	会計課長 
\\	会計係(がかり) 
\\	〔団体・クラブなどの〕
\\	〔現金出納係〕
\\	会計学 
\\	会計監査(かんさ) 
\\	会計監査をする 
\\	会計検査院(けんさいん) 
\\	会計検査官 
\\	会計士 
\\	〔公認会計士〕((米)) 
\\	((略
\\	((英)) 
\\	会計帳簿(ちょうぼ)
\\	会計年度 ((米)) 
\\	((英)) 
\\	会計報告 
\\	解決	かいけつ	
\\	会合	かいごう	
\\	外交	がいこう	
\\	開始	かいし	
\\	解釈	かいしゃく	
\\	〔語句・法律などの〕
\\	解釈する 
\\	憲法[古典]の解釈 
\\	法を厳密(げんみつ)に解釈する 
\\	それはいろいろに解釈できる 
\\	/((文)) 
\\	内気(うちき)なのが傲慢(ごうまん)と解釈されたらしい 
\\	彼は彼女の言葉にユーモラスな解釈を加えた 
\\	君の解釈は間違っているようだ 
\\	私は彼の行動を善意(ぜんい)に解釈した 
\\	この一節(いっせつ)を解釈しなさい 
\\	日本語の解釈力がついてきた 
\\	外出	がいしゅつ	
\\	彼は外出中だ 
\\	彼の外出中の出来事だ 
\\	彼の外出先を知っていますか 
\\	外出嫌いの人 
\\	外出着 
\\	外出禁止令 
\\	改善	かいぜん	
\\	快適	かいてき	快適な 
\\	快適なベッド 
\\	新車の乗り心地はすこぶる快適だ 
\\	快適な船旅 
\\	こじんまりとした快適な家 
\\	回復	かいふく	
\\	〔元通りになること・すること〕
\\	回復する 
\\	社会秩序(ちつじょ)を回復する 
\\	平和の回復を望(のぞ)む 
\\	名誉(めいよ)を回復する 
\\	国の経済を回復する 
\\	景気が回復した 
\\	天気が回復した 
\\	失地(しっち)を回復する 
\\	〔病気がよくなること〕
\\	回復する 
\\	彼の回復はおぼつかない 
\\	おぼつかない 
\\	彼は急速に回復している 
\\	元気を回復した 
\\	健康[視力]を回復する 
\\	とうとう意識を回復することがなかった 
\\	回復期 
\\	飼う	かう	
\\	帰す	かえす	
\\	〔元の所へ行かせる〕
\\	客を帰す 
\\	〔野球で,走者を本塁に戻す〕 二塁打(るいだ)で2走者(そうしゃ)をホームに帰した 
\\	代える	かえる	
\\	〔代用させる〕
\\	バターをマーガリンに代えてみたらどうですか 
\\	書面をもって挨拶に代えさせていただきます 
\\	をもっ(以)て 
\\	私に代えて代理人を出席させます 
\\	命には代えられません 
\\	背に腹は代えられぬ ((諺)) 
\\	替える	かえる	
\\	古い機械を新しいのに替える必要がある 
\\	畳を替える 
\\	換える	かえる	
\\	〔交換(こうかん)する〕
\\	円をドルに換える 
\\	この指輪を金に換えたい 
\\	あの女の人に席を換えてもらいなさい 
\\	千円札(さつ)を小銭(こぜに)に換えてもらった 
\\	彼女は命に換えてもそれを手放(てばな)さなかっただろう 
\\	かえる【▲反る】	かえる	
\\	かえる【▲反る】 着物の裾(すそ)がかえった 
\\	行司(ぎょうじ)の軍配(ぐんばい)がかえった 
\\	香り	かおり	
\\	画家	がか	
\\	抱える	かかえる	
\\	〔手・腕などで〕 
\\	小脇(こわき)に抱える 
\\	頭を抱える 
\\	すきっ腹(すきっぱら)を抱えてうろつき回った 
\\	〔厄介(やっかい)な物を持つ〕 
\\	病人を抱えている 
\\	その国は紛争(ふんそう)の火種(ひだね)を抱えている 
\\	火種 
\\	沢山の注文を抱えて毎日残業(ざんぎょう)だ 
\\	あの会社は巨額(きょがく)の借金を抱えている 
\\	彼は体に時限爆弾(じげんばくだん)を抱えている 
\\	〔雇う〕 
\\	運転手を抱える 
\\	価格	かかく	
\\	引き渡し価格 
\\	市場価格 
\\	国内[国際]価格 
\\	公正価格 
\\	公定価格 
\\	協定価格 
\\	最高価格 
\\	最低価格 
\\	生産者[消費者]価格 
\\	適正価格 
\\	価格を上げる 
\\	価格を下げる 
\\	価格を維持する 
\\	価格をつり上げる 
\\	法外な[5万円の]価格で売る 
\\	安い価格で 
\\	その土地の価格を5千万円と見積もった 
\\	その価格では安い 
\\	価格安定 
\\	価格カルテル 
\\	価格競争 
\\	価格協定 
\\	価格統制 
\\	価格操作(そうさ) 
\\	価格破壊(はかい) 
\\	価格表 
\\	価格変動 
\\	化学	かがく	
\\	輝く	かがやく	
\\	〔光る〕
\\	は光を出す,光って見える
\\	は星や遠い光などがちかちか光る
\\	はぴかぴかきらめく
\\	はきらきらと光る,また 目などが生き生きと輝くときにも用いる) 太陽は明るく輝いている 
\\	星が輝いていた 
\\	宝石が彼女の指にきらきら輝いていた 
\\	それを見ると,彼の目は興奮(こうふん)に輝いた 
\\	〔晴れやかに見える〕 これは喜びに輝く一家の写真です 
\\	彼は名誉(めいよ)に輝く業績を成(な)し遂(と)げた 
\\	係	かかり	
\\	受付係(うけつけがかり) 
\\	/〔ホテルの〕
\\	出納係(すいとうがかり) 
\\	私はパーティーの準備係になった 
\\	これの係は誰だ 
\\	係員 
\\	係官 
\\	係長 
\\	罹る	かかる	
\\	〔病気に〕
\\	[((米)) 
\\	〔特定の病気に〕
\\	がんにかかって死ぬ 
\\	私は以前その病気にかかったことがあるのでもうかからない 
\\	子供は病気にかかりやすい 
\\	彼は気管支炎(きかんしえん)にかかっている 
\\	限る	かぎる	
\\	〔制限する〕
\\	我々の予算は10万円以内に限られている 
\\	入場者は女性に限ります 
\\	/〔掲示〕
\\	日を限って願書を受け付けます 
\\	日を限る 
\\	今回に限り教えてあげよう 
\\	正当(せいとう)な理由のある場合に限り考慮する 
\\	音楽が好きなのは君に限ったことではない 
\\	〔最上である〕 辞書はこれに限る 
\\	夏は冷たいビールに限る 
\\	疲れをいやすには眠るに限る 
\\	〔否定語を伴い,かならずしも…ではない〕 漫画がすべて俗悪(ぞくあく)であるとは限らない 
\\	火山が噴火しないとも限らない 
\\	かく
\\	掻く】	かく	掻く 
\\	〔爪で〕
\\	かゆい[虫に刺された]所をかく 
\\	頭をかく 
\\	耳をかく 
\\	馬が前脚(まえあし)で地面をかいている 
\\	〔削(けず)る,取る〕 かつおぶし[氷]をかく 
\\	むだな新芽(しんめ)をかく 
\\	〔押(お)し退(の)ける〕
\\	熊手(くまで)で芝生の落葉(らくよう)をかく 
\\	道の雪をかく 
\\	オールで水をかく 
\\	描xく	かく	
\\	〔鉛筆・クレヨンなどで〕
\\	〔彩色して〕
\\	富士山の水彩画を書いている 
\\	嗅ぐ	かぐ	
\\	家具	かぐ	
\\	学	がく	
\\	額	がく	
\\	〔金額〕
\\	⇒きんがく(金額) 多[少]額の金 
\\	買い物は予定していた額を上回ってしまった 
\\	〔額縁〕
\\	絵を額に入れる 
\\	額を神社に奉納(ほうのう)する 
\\	覚悟	かくご	
\\	〔心に用意のあること〕
\\	覚悟する 
\\	〔気持ちを引きしめる〕
\\	それは覚悟の上だ 
\\	覚悟はよいか 
\\	断られるのを覚悟で頼んでみるつもりだ 
\\	君が理不尽なことをやめないのならこっちにも覚悟がある 
\\	〔決心〕
\\	覚悟する 
\\	最後までやり抜く覚悟だ 
\\	〔あきらめ〕
\\	この場合どうもしようがないと覚悟をきめた 
\\	確実	かくじつ	
\\	学者	がくしゃ	
\\	学習	がくしゅう	
\\	隠す	かくす	
\\	〔見つからないようにする〕
\\	少し形式張った語) 大事な物を屋根裏(やねうら)部屋に隠した 
\\	素早(すばや)くドアの陰に[木立(こだち)の中に]身を隠した 
\\	ほとぼりが冷(さ)めるまで身を隠しているつもりだ 
\\	/((口))
\\	悲しみを隠そうとして無理に笑った 
\\	彼女はハンカチで口元を隠した 
\\	〔知られないようにする〕
\\	年齢を隠す 
\\	名前を隠して新聞に投書した 
\\	サングラスで顔を隠していた 
\\	私たちは君に何も隠していない 
\\	会社は不祥事(ふしょうじ)を隠し切れなかった 
\\	たいていの家には他人に隠したい事情があるものだ 
\\	[((英)) 
\\	何を隠そう,私がその金を盗(ぬす)んだのです 
\\	問題を隠した 
\\	拡大	かくだい	
\\	は実物より大きく見せること
\\	は大きさ・量・範囲を増大) 拡大する 
\\	この顕微鏡(けんびきょう)で見るとバクテリアが千倍(せんばい)に拡大される 
\\	バクテリアは普通複数形を用いる) 対米(たいべい)貿易(ぼうえき)の拡大 
\\	当時英国は領土(りょうど)を拡大していた 
\\	戦火(せんか)は半島全域(ぜんいき)に拡大した 
\\	核軍備(かくぐんび)はさらに拡大される模様(もよう)だ 
\\	拡大委員会 
\\	拡大解釈(かいしゃく) 拡大解釈する 
\\	拡大鏡(きょう) 
\\	拡大家族 
\\	拡大コピー 
\\	拡大再生産 
\\	拡大図[写真] 
\\	拡大率 
\\	確認	かくにん	
\\	〔法律などの〕
\\	確認する 
\\	それは事実であることが確認された 
\\	必要な物は全部買ったかどうか確認しなさい 
\\	その男の身元を確認するものは何もなかった 
\\	確認事項 
\\	確認書 
\\	学問	がくもん	〔学ぶこと〕
\\	〔学識〕
\\	〔学業〕
\\	学問的 〔学術的〕
\\	〔科学的〕
\\	学問をする 
\\	あの人は学問がある 
\\	[((文)) 
\\	彼の父親は学問のない男だった 
\\	彼は息子に学問をさせた 
\\	兄は学問好きです 
\\	哲学は高尚(こうしょう)な学問だ 
\\	彼女は学問が出来る 
\\	学問だけでは立派な人間にはなれない 
\\	耳学問も時には役に立つ 
\\	彼は学問的な批評書を出した 
\\	彼の考え方は学問的だ 
\\	学問の自由を守っていかねばならない 
\\	隠れる	かくれる	
\\	〔見えなくなる〕
\\	湖は霧に隠れて見えなかった 
\\	山の頂は雲に隠れている 
\\	月が雲の陰に隠れた 
\\	人込みに隠れて見えなくなった 
\\	〔潜(ひそ)む〕
\\	あわてて木の陰に隠れた 
\\	野原には隠れる所が一つもなかった 
\\	脱獄囚(だつごくしゅう)は山の中に隠れている 
\\	彼らはまた隠れて酒を飲んでいる 
\\	〔隠遁(いんとん)する〕 
\\	世の中から隠れて暮らす 
\\	〔人目につかずにいる〕 隠れた人材を探す[掘り起こす] 
\\	彼は隠れた逸材(いつざい)だ 
\\	隠れた善行を重ねてきた 
\\	〔口実とする〕 彼らは開発の名に隠れて自然を破壊(はかい)した 
\\	影	かげ	
\\	〔光〕
\\	月影(つきかげ) 
\\	星影 
\\	〔影法師〕
\\	建物(たてもの)は舗道(ほどう)に影を落としていた 
\\	木の柵(さく)が芝生の上に影を投げかけていた 
\\	障子に人の影が映った 
\\	〔映像〕
\\	太鼓橋(たいこばし)の影が池に映っていた 
\\	〔姿〕
\\	逃げていく人の影が見えた 
\\	〔暗い面〕 死の影におびえる 
\\	彼らの幸せな生活にも暗い影が忍(しの)び寄(よ)っていた 
\\	影が薄(うす)い 
\\	あの人は影が薄い 〔元気がない〕
\\	/〔存在が目立たない〕
\\	最近,あいつは影が薄(うす)くなった 
\\	影がさす 
\\	うわさをすれば影がさす ((諺)) 
\\	影の形(かたち)に添うよう 影の形にそうように彼のそばにはいつも夫人がいる 
\\	影も形もない 
\\	戻ってきたら荷物が影も形もなくなっていた 
\\	影を潜(ひそ)める 
\\	最近では暴力事件が影を潜めている 
\\	見る影もない 
\\	彼は見る影もなくやせ衰(おとろ)えていた 
\\	影の内閣 
\\	陰	かげ	
\\	〔光の当たらない所〕
\\	木の陰で 
\\	私の家はあのアパートの陰になっている 
\\	庭は半分陰になっていた 
\\	〔後ろ〕 戸の陰にだれかいる 
\\	塀(へい)の陰から男が飛び出した 
\\	船が島の陰に隠(かく)れた 
\\	〔見えないところ〕 陰の声 
\\	クイズのときなどの裏の声) 陰の実力者 
\\	陰で悪口を言う 
\\	陰で色々な工作(こうさく)が行われていた 
\\	陰であやつっているのはだれだ 
\\	父にはよく怒られていたが,母が陰でかばってくれた 
\\	かばう 
\\	今ごろは陰で舌を出していることだろう 
\\	母は気丈(きじょう)な女だったが時々陰で泣いていた 
\\	〔暗い面〕 あの人にはどこかしら陰がある 
\\	陰で糸を引く 陰で糸を引く者がいるらしい 
\\	陰ながら 
\\	陰ながら成功をお祈りします 
\\	陰になり日向になり 彼は陰になり日向(ひなた)になり彼らを助けてやった 
\\	欠ける	かける	
\\	〔一部分が壊れる〕
\\	この急須(きゅうす)は口が欠けている 
\\	コップの端(はし)が欠けた 
\\	ふちの欠けた皿 
\\	この刀は刃(は)が欠けている 
\\	〔月などが〕
\\	月が欠け始めた 
\\	日食(にっしょく)で太陽の端が欠け始めた 
\\	〔不足する〕
\\	彼は忍耐力(にんたいりょく)が欠けている 
\\	欲しいもので欠けているものは何もない 
\\	この本は2ページ欠けている 
\\	彼は90点に少し欠ける点数を取った 
\\	この委員会は3人メンバーが欠けている 
\\	加減	かげん	
\\	〔足し算と引き算〕
\\	〔調節すること〕
\\	〔手加減〕
\\	加減する 〔調節する〕
\\	〔手加減する〕
\\	いすの高さは少年の身長(しんちょう)に応じて加減してある 
\\	物事をほどよく加減してする 
\\	ほどよく 
\\	生徒の精神年齢に応じて叱(しか)り方を加減しなさい 
\\	〔物事の調子や具合〕 
\\	パンを焼くには火加減が大事だ 
\\	彼女の口のうまさ加減には驚(おどろ)く 
\\	ちょうどいい味加減です 
\\	お吸物(すいもの)の加減はいかがですか 
\\	肉の焼き加減はいかがいたしましょう 
\\	答えは
\\	あるいは
\\	陽気(ようき)の加減かどうも体がだるい 
\\	どうした加減か実験がうまくいかなかった 
\\	〔健康状態〕 
\\	お加減はいかがですか 
\\	病人の加減はよくない 
\\	少し腹(はら)の加減が悪い 
\\	〔少し…の感じ〕 彼女はうつむき加減に話し出した 
\\	うつむき 
\\	加減乗除(じょうじょ) 
\\	過去	かこ	
\\	かご	かご 
\\	籠】	
\\	〔鳥かご〕
\\	〔ニワトリ・ウサギなどを入れる〕
\\	〔運送・荷作り用の〕
\\	竹[籐(とう)]のかご 
\\	かご一杯のいちご 
\\	紙くずかご 
\\	買物(かいもの)かご 
\\	かごを編(あ)む 
\\	鶏の伏(ふ)せかご 
\\	籠細工(かごさいく) 
\\	籠の鳥 かごの鳥のようなみじめな生活を送っていた 
\\	囲む	かこむ	
\\	〔取り巻く〕
\\	〔壁・塀で〕
\\	〔柵・塀で〕
\\	彼は新聞記者に囲まれた 
\\	高い塀(へい)に囲まれた家 
\\	暖炉(だんろ)[食卓]を囲む 
\\	海に囲まれた国 
\\	番号を
\\	で囲む 
\\	100人の警官が爆発現場を囲んだ 
\\	〔包囲(ほうい)して攻(せ)める〕
\\	城は敵に囲まれていた 
\\	敵軍が城を囲んだ 
\\	火災	かさい	
\\	重なる	かさなる	
\\	〔物の上に物が乗る〕
\\	〔積み重なる〕
\\	紙が2枚重なっている 
\\	箱がたくさん重なっている 
\\	〔さらに加わる〕 重なる不幸 
\\	悪いことは重なるものだ 
\\	仕事が重なって悲鳴(ひめい)を上げています 
\\	寝不足が重なって病気になった 
\\	〔かち合う〕 今年は秋分(しゅうぶん)の日と日曜日とが重なる 
\\	たまたま二つの事故が重なって起きた 
\\	重ねる	かさねる	
\\	飾り	かざり	
\\	貸し	かし	〔貸し付け〕
\\	〔売り掛け金〕
\\	彼に5万円貸しがある 
\\	困ったところを助けてやって彼に貸しを作った 
\\	貸し衣装 ⇒詳細を見る 貸し金庫 
\\	貸し事務所 ((米)) 
\\	((英)) 
\\	貸し賃(ちん)[料] ((米)) 
\\	((英)) 
\\	〔貸し本などの〕
\\	ボートの貸し賃はいくらですか ((米)) 
\\	/((英)) 
\\	貸し布団 ((米)) 
\\	((英)) 
\\	貸しボート[自転車] ((米)) 
\\	((英)) 
\\	菓子	かし	
\\	家事	かじ	
\\	家の仕事 (いえのしごと))
\\	賢い	かしこい	
\\	歌手	かしゅ	
\\	数	かず	
\\	稼ぐ	かせぐ	
\\	〔働いて手に入れる〕 生活費を稼ぐ 
\\	月に20万円稼ぐ 
\\	稼ぐに追いつく貧乏なし 
\\	〔自分の有利になるよう行動する〕 時間を稼ぐ 
\\	点を稼ぐため上役(うわやく)にお世辞(せじ)を言った 
\\	彼はホームランで1点稼いだ 
\\	数える	かぞえる	
\\	型	かた	
\\	〔細工品を作る原型〕
\\	〔鋳型〕((米)) 
\\	((英)) 
\\	ドレスの型を取る 
\\	歯の型を取る 
\\	鉛(なまり)を型に流し込む 
\\	〔スポーツなどで動作の決まり〕 バレー[剣道]の型 
\\	〔伝統的な形式〕
\\	彼は型のごとく8時半の列車に乗った 
\\	型にはまった言い回し 
\\	型にはまる 
\\	彼の型にはまらないアイデアがいい 
\\	型を破る 
\\	〔特有の様式〕 彼女らは同じ型のドレスを着ていた 
\\	僕の車は1998年型だ 
\\	古い型の人間 
\\	ああいう型の人は嫌いだ 
\\	肩	かた	
\\	堅い	かたい	
\\	〔堅牢(けんろう)な〕
\\	〔密で堅い〕
\\	堅い材木 
\\	「堅材」の意味では
\\	堅い地面 
\\	〔堅実な〕
\\	〔信用できる〕
\\	堅い商売 
\\	堅い投資 
\\	堅い人物である 
\\	堅い証拠(しょうこ) 
\\	堅い女 
\\	〔まじめな〕
\\	〔形式ばった〕
\\	⇒かたくるしい(堅苦しい) 堅い読物(どくぶつ) 
\\	そう堅く考えないでください 
\\	堅い話はよそう 
\\	止す(よす) 
\\	堅いお役所言葉 
\\	堅いことは抜きにしよう 
\\	硬い	かたい	
\\	〔力を加えても形が変わらない〕
\\	硬い毛 
\\	硬い歯ブラシ 
\\	硬いマットレス 
\\	この鉛筆は硬い 
\\	鉄を硬くする 
\\	金剛石(こんごうせき)のように硬い 
\\	〔こわばった〕
\\	寒くて手の指が硬くなった 
\\	猫は死んで硬くなっていた 
\\	死体はすでに硬くなっていた 
\\	〔文体などが練れていない〕
\\	〔文語的な〕
\\	硬い文体 
\\	硬い言い回し 
\\	この絵は硬い 
\\	〔表情などがきつい〕 硬い顔つきで 
\\	それを聞くと表情が硬くなった 
\\	片付く	かたづく	
\\	(整頓される)
\\	(処理される)
\\	(完結する)
\\	(嫁ぐ)
\\	〔整頓される〕
\\	部屋がすっかり片付いていた 
\\	〔解決する〕
\\	今度の仕事が片付いたら旅行に行きます 
\\	〔娘が結婚する〕 娘が3人とも片付いた 
\\	刀	かたな	
\\	語る	かたる	
\\	〔話す〕
\\	〔物語る〕
\\	彼は真相(じっそう)を語った 
\\	彼は経験談を語った 
\\	彼は民話を語るのがうまい 
\\	その出来事を見てきたように語った 
\\	彼は過去の事はいっさい語らない 
\\	〔朗読する〕
\\	義太夫を語る 
\\	〔おのずと示す〕 いすのぬくもりは彼がさっきまでそこに座っていたことを語っている 
\\	語るに落ちる 
\\	勝ち	かち	
\\	価値	かち	
\\	価値のある 
\\	健康の価値 
\\	文学的価値のある本 
\\	人間の価値 
\\	この花瓶は100万円の価値がある 
\\	あの芝居は見る価値がある 
\\	彼の勤勉(きんべん)さは賞讃(しょうさん)の価値がある 
\\	この絵は私には何の価値もない 
\\	価値観 
\\	価値観は人によって異なる 
\\	価値判断 
\\	がっかり	がっかり	
\\	〔失望する様子〕 がっかりする 
\\	彼女はとてもがっかりして食物(しょくもつ)がのどを通らなかった 
\\	がっかりするな 
\\	/〔元気を出せ〕
\\	それを聞いたときはがっかりしました 
\\	彼が来なくてがっかりした 
\\	お金をなくして彼はすっかりがっかりしてしまった 
\\	〔疲れて気が抜ける様子〕 会議が終わると急にがっかりした 
\\	活気	かっき	
\\	楽器	がっき	
\\	学期	がっき	
\\	活動	かつどう	
\\	活躍	かつやく	活躍する 
\\	彼は政界で活躍している 
\\	この分野では女性の活躍が目立っている 
\\	彼は活躍の場を広げたがっている 
\\	活用	かつよう	
\\	〔応用〕
\\	ここでは英語が活用できない 
\\	外国語を学ぶには辞書を大いに活用する必要がある 
\\	新しい方法を産業に活用する 
\\	機会をせいぜい活用するよう努(つと)めなさい 
\\	銅は種々(しゅじゅ)の用途(ようと)に活用される 
\\	〔文法で動詞の〕
\\	活用する[させる] 
\\	動詞 
\\	の活用を示しなさい 
\\	活用形 
\\	活用語 
\\	活用語尾 
\\	仮定	かてい	
\\	〔仮説〕
\\	((複-
\\	〔数学で〕
\\	仮定する 
\\	〔数学で〕
\\	仮定的 
\\	それが事実と仮定してみよう 
\\	今,大地震が起きたと仮定したらまずどうしますか 
\\	と
\\	が等しいと仮定せよ 
\\	それはあくまでも仮定に過ぎない 
\\	彼がやったと仮定しても証拠(しょうこ)は一つもない 
\\	仮定法 〔文法で〕
\\	過程	かてい	
\\	そのころ会社は発展(はってん)の過程をたどっていた 
\\	包装(ほうそう)過程で何か手違いがあったらしい 
\\	ここに至るまでの過程をたどってみよう 
\\	課程	かてい	
\\	彼は高校の課程を終えた 
\\	修士(しゅうし)課程を終了して博士課程へ進んだ 
\\	悲しむ	かなしむ	
\\	必ずしも	かならずしも	
\\	かなり	かなり	
\\	は望ましいことに用いる); 〔相当に〕
\\	このパイはかなりおいしい 
\\	テストでかなり悪い点を取った 
\\	彼は英語をかなりよく話す 
\\	かなり長い間待った 
\\	彼はかなりいい絵を描く 
\\	かなりの距離 
\\	かなりの金額 
\\	かなりの成功 
\\	彼らはかなりの暮しをしている 
\\	かなりの収入[財産] 
\\	かなりの家に住んでいる 
\\	かなりのインテリだ 
\\	金	かね	
\\	鐘	かね	
\\	可能	かのう	
\\	株	かぶ	
\\	かぶる	被る	
\\	〔覆うようにして乗せる〕
\\	は動作,
\\	は状態を表す) あわてて帽子をかぶった 
\\	ベレー帽をかぶった男 
\\	子供は帽子[面(おもて)]をかぶっていた 
\\	少年は帽子をかぶらずに雨の中を歩いた 
\\	彼女は毛布(もうふ)をすっぽりかぶっていた 
\\	〔上から浴びる〕 
\\	毎朝水をかぶる習慣(しゅうかん)だ 
\\	田畑が水をかぶった 
\\	棚がほこりをかぶっていた 
\\	火の粉(こ)を頭からかぶりながら逃げた 
\\	車の通った時に泥水(どろみず/でいすい)をかぶった 
\\	〔身に引き受ける〕 
\\	彼は親分(おやぶん)の罪をかぶった 
\\	彼はどうして弟(おとうと)の負債(ふさい)をかぶることになったのか 
\\	我慢	がまん	〔耐え忍ぶこと〕
\\	〔頑張ること〕
\\	〔自制すること〕
\\	我慢する 
\\	と
\\	は否定文で使うことが多い
\\	が最も強い程度を表す); 
\\	他人の言動を大目に見る) 怒りたいのを我慢した 
\\	お金があまりなかったので昼食はかけそば1杯で我慢した 
\\	今回だけは我慢してやる 
\\	ここが我慢のし所だ 
\\	我慢できる 
\\	我慢できない痛み 
\\	あんな人には我慢できません 
\\	あの厚かましさは我慢ができない 
\\	私はついに我慢しきれなくなった 
\\	今夜ぐらい酒を我慢できないのか 
\\	我慢強い 
\\	我慢強く歩き続けた 
\\	我慢比べ 
\\	神	かみ	
\\	雷	かみなり	
\\	髪の毛	かみのけ	
\\	科目	かもく	
\\	〔項目〕
\\	勘定科目 
\\	請求書(せいきゅうしょ)を科目別にする 
\\	〔学科〕
\\	〔課程〕
\\	必修(ひっしゅう)科目 
\\	選択科目 
\\	全科目 
\\	今学期は何科目取っていますか 
\\	フランス語の科目は履修(りしゅう)した 
\\	かもしれない	かもしれない	
\\	明日は雨になるかも知れない 
\\	彼は来ないかも知れない 
\\	いつ惨事(さんじ)が起こるかも知れない 
\\	あの人は忘れたのかも知れない 
\\	急行に間に合うかも知れないと思って走った 
\\	かゆい	痒い	
\\	背中がかゆい 
\\	かゆい所をぼりぼりかいた 
\\	痒い所に手が届く 
\\	かゆい所に手が届くように僕の面倒(めんどう)を見てくれた 
\\	歌謡	かよう	〔歌曲〕
\\	〔伝承的な物語詩〕
\\	〔総称〕
\\	中世の歌謡 
\\	歌謡曲 
\\	空	から	
\\	殻	から	
\\	〔穀物などの〕
\\	〔豆類の〕
\\	〔貝・くるみなどの〕
\\	〔えんどう豆の〕
\\	もみの殻を取る 
\\	かたつむりの殻 
\\	卵の殻 
\\	豆の殻を取る 
\\	/((英)) 
\\	かきの殻を取る 
\\	ピーナッツの殻をむく 
\\	[((米)) 
\\	〔中身の無くなった物〕 弁当の殻 
\\	せみの殻 
\\	〔自分の世界〕 自分の殻に閉じこもる 
\\	自分の殻から脱(ぬ)け出す 
\\	若い芸術家たちは因習の殻を打ち破(やぶ)ろうと努(つと)めた 
\\	柄	がら 
\\	え	がら 
\\	〔模様〕
\\	は小さな繰り返し模様); 
\\	形や色の扱い方や配列などを強調する) 縞(しま)柄のドレス 
\\	大柄 
\\	小さな柄のプリント地 
\\	複雑な柄 
\\	〔体格〕
\\	柄の大きい少年 
\\	柄の小さい子供 
\\	小柄[大柄]の婦人 
\\	〔性質・状態〕 あんな柄の悪い奴と付き合うな 
\\	彼は指導者といった柄ではない 
\\	全く柄にもないことを父は言った 
\\	この辺は柄が悪い 
\\	〔立場(たちば)〕
\\	文句を言える柄か 
\\	柄物(がらもの) 
\\	え 
\\	傘の柄 
\\	刈る	かる	
\\	〔庭木などを〕
\\	は形を整える,
\\	は余分な枝を切る); 〔草などを〕
\\	〔穀物を〕
\\	生け垣(がき)を刈る 
\\	2週間ごとに芝(しば)を刈る 
\\	稲(いね)[小麦]を刈る 
\\	〔毛を〕
\\	〔短い毛を〕
\\	〔羊毛(ようもう)などを〕
\\	頭を刈る 
\\	髪を短く刈ってもらった 
\\	は非常に短く) プードル犬の毛を刈る 
\\	羊(ひつじ)の毛を刈る 
\\	河	かわ	
\\	皮	かわ	
\\	革	かわ	
\\	可愛そう	かわいそう	
\\	〔いたわしい〕
\\	〔悲しい〕
\\	母のないかわいそうな子馬(こうま) 
\\	かわいそうな物語 
\\	かわいそうな男だ 
\\	あの独りぼっちの老女がかわいそうだった 
\\	かわいそうに 
\\	本人に向かって 
\\	とは決して言わない) かわいそうに,あの子は靴をはいていなかった 
\\	かわいそうなのはその幼い姉妹だった 
\\	虫を殺してはかわいそうだ 
\\	かわいそうだと思って許してやれ 
\\	彼は子犬をかわいそうに思って連れて帰った 
\\	可愛らしい	かわいらしい	
\\	乾かす	かわかす	
\\	洗濯物を日光で乾かす 
\\	ぬれた衣服を火で乾かす 
\\	渇く	かわく	
\\	〔のどが〕 のどが渇いた 
\\	のどが渇き切っていた 
\\	川の水で渇いたのどを潤(うるお)した 
\\	〔ひどく欲しがる〕⇒かつぼう(渇望) 兵士たちは血に渇いていた 
\\	代る	かわる	
\\	〔一時的または永久に〕
\\	〔一時的に〕
\\	外遊中(がいゆうちゅう)だれが彼に代わるのか 
\\	機械は必ずしも人力に代わるものではない 
\\	彼に代わるべき人は見当たらない 
\\	協会一同(いちどう)に代わってごあいさついたします 
\\	缶	かん	
\\	勘	かん	
\\	〔直観〕
\\	〔第六感〕
\\	そのことは勘で分かった 
\\	勘が当たった[外れた] 
\\	私の勘によるとこの計画は失敗しそうだ 
\\	彼女は勘がいい 
\\	私はどうも勘が悪い 
\\	その場の状況が勘で分かった 
\\	勘を働かせる 
\\	考え	かんがえ	
\\	感覚	かんかく	
\\	〔知覚〕
\\	感覚的 〔感覚に訴える〕
\\	平衡(へいこう)感覚 
\\	方向感覚 
\\	気温の変化に対して感覚が鋭い[鈍い] 
\\	あまり寒いので手足の感覚が無くなった 
\\	彼女の手に触(ふ)れたときの暖かい感覚がまだ残っていた 
\\	感覚的に彼が嫌いだ 
\\	/((口)) 
\\	主に若い人が使う) 
\\	〔心の働き〕
\\	芸術的[音楽的]感覚 
\\	彼には道徳的(どうとくてき)感覚が欠(か)けている 
\\	彼は若者(わかもの)らしいみずみずしい感覚で現代の社会にメスを入れている 
\\	メス 
\\	メスを入れる 
\\	感覚器官 
\\	感覚主義 
\\	感覚神経 
\\	感覚論 〔哲学で〕
\\	間隔	かんかく	
\\	〔距離の〕
\\	一定の間隔をおいて 
\\	壁と壁の間は少なくとも5メートルの間隔が欲しい 
\\	行(ぎょう)と行との間隔を詰めて[あけて]作文を書きなさい 
\\	柱と柱の間隔はどのくらいありますか 
\\	〔時間の〕
\\	電車は10分(じっぷん ー じゅっぷん)間隔で発車する 
\\	観客	かんきゃく	
\\	環境	かんきょう	
\\	知的(ちてき)な[幸せな]環境に育った 
\\	人は遺伝(いでん)と環境によって左右される 
\\	開発事業のためこの辺りでは自然環境が破壊(はかい)されている 
\\	彼は環境に順応(じゅんのう)するのが遅い 
\\	家は気に入ったが環境がいやだ 
\\	この山荘(さんそう)はすばらしい環境にある 
\\	近くにゲームセンターができて周囲(しゅうい)の環境が変わった 
\\	環境アセスメント 
\\	環境アセスメント報告 
\\	環境医学 
\\	環境衛生 
\\	環境汚染 
\\	環境汚染測定法 
\\	環境汚染物質 
\\	環境基準 
\\	環境基本計画 
\\	環境基本法 
\\	環境教育 
\\	環境共生住宅 
\\	環境権 
\\	環境工学 
\\	環境コスト 
\\	環境税 
\\	環境対策車 
\\	環境省(大臣) 〔日本の〕
\\	環境庁(長官) 〔米国の〕
\\	((略
\\	環境適応 
\\	環境難民 
\\	環境破壊 
\\	〔生態系破壊〕
\\	環境ビジネス 
\\	環境保護 
\\	環境保護主義者 
\\	環境保全 
\\	環境保全運動 
\\	環境保全主義者 
\\	環境ホルモン 
\\	環境問題 
\\	環境倫理 
\\	歓迎	かんげい	
\\	観光	かんこう	
\\	観察	かんさつ	
\\	観察する 〔特別の目的のために〕
\\	〔じっと見守る〕
\\	天体の動き[魚の生態]を観察する 
\\	会話を交(か)わしながら彼女をよく観察した 
\\	人の服装(ふくそう)に観察が鋭い 
\\	観察眼(かんさつがん) 
\\	観察眼が鋭い 
\\	観察者 
\\	野鳥(やちょう)観察者 
\\	((口)) 
\\	観察点 
\\	観察力 
\\	感じ	かんじ	
\\	感謝	かんしゃ	
\\	患者	かんじゃ	
\\	勘定	かんじょう	
\\	〔計算〕
\\	は通例複雑な計算を行うこと
\\	は一つ一つ数えること
\\	は簡単な計算,またはざっと計算すること) ⇒けいさん(計算)
\\	勘定する 
\\	彼は指を折って日数を勘定した 
\\	勘定が上手[下手]だ 
\\	勘定がどうしても合わない 
\\	勘定するといくらになりますか 
\\	勘定を間違えたらしい 
\\	勘定合って銭足(ぜにた)らず 
\\	〔支払い〕
\\	〔決算〕
\\	〔会計〕
\\	〔勘定書〕
\\	[((米)) 
\\	勘定を払う[済ます] 
\\	勘定は私の方に付けておいてください 
\\	勘定はいくらですか 
\\	お勘定! 
\\	[((米))
\\	お勘定は8,000円になります 
\\	毎月1回勘定を取りに回る 
\\	引っ越しの勘定は全部私が持とう 
\\	勘定は勘定だ 
\\	〔考慮(こうりょ)〕
\\	彼らが来ることは勘定に入れていなかった 
\\	そのことは勘定に入れなくてよろしい 
\\	勘定係 
\\	勘定書 
\\	((米)) 
\\	勘定ずく 彼は何事も勘定ずくでやる 
\\	勘定台 
\\	〔スーパーなどの〕
\\	勘定高い 〔打算的な〕
\\	〔けちな〕
\\	勘定取り 〔人〕
\\	〔仕事〕
\\	勘定日 〔給料日〕
\\	ただし
\\	では無冠詞); 〔株式取り引きなどの〕
\\	[((英)) 
\\	〔四半期支払い日〕((英)) 
\\	感情	かんじょう	
\\	感じる	かんじる	
\\	感心	かんしん	
\\	〔立派だと感じ入ること〕 感心する 〔感嘆する〕
\\	〔感銘を受ける〕
\\	彼の態度には皆感心した 
\\	彼らはすっかり感心して体操選手の妙技(みょうぎ)を眺めていた 
\\	〔褒(ほ)めるべきこと〕
\\	感心,感心
\\	よくやったね 
\\	愚痴(ぐち)一つこぼさないとは全く感心だ 
\\	彼の感心な点は誠実(せいじつ)さだ 
\\	[((文)) 
\\	そういうタイプの絵はあまり感心しません 
\\	この家で感心しない点は天井(てんじょう)が低(ひく)いということだ 
\\	関心	かんしん	〔興味〕
\\	〔気にすること〕
\\	私は政治にはあまり関心がない 
\\	彼女は環境(かんきょう)問題に関心を持っている 
\\	少年のテレビ出演はクラスの関心の的となった 
\\	あの会社に関心がある 
\\	彼は子供のことに関心がない 
\\	彼は食物(しょくもつ)に関心がないように見える 
\\	今度出た製品に関心がある 
\\	関心事(じ) 
\\	私の最大の関心事は彼女が私をどう思っているかだ 
\\	関する	かんする	
\\	政治に関する講演 
\\	防衛(ぼうえい)に関する論争(ろんそう) 
\\	彼に関する限り問題はない 
\\	この報告ではその件に関して何も言っていない 
\\	彼の辞職(じしょく)に関していろいろ取り沙汰(ざた)されている 
\\	取り沙汰 
\\	彼はいつも我関(われかん)せずという態度を取っている 
\\	我関せず 
\\	彼は少し内気(うちき)だが勉強に関しては申し分(ぶん)ない 
\\	申し分無い 
\\	完成	かんせい	
\\	完全	かんぜん	
\\	乾燥	かんそう	
\\	感想	かんそう	
\\	彼は選挙の結果に関する感想を述べた 
\\	「ご感想はいかが?」「別にありません」 
\\	感想文[録] 生徒たちは読後の感想文を書かされた 
\\	[((米))
\\	感動	かんどう	
\\	監督	かんとく	
\\	〔取り締まること〕
\\	監督する 
\\	従業員の監督を厳しくする必要がある 
\\	社長の監督の下に会社再建に励(はげ)んだ 
\\	その機関は政府の監督下に置かれた 
\\	試験の監督を頼まれた 
\\	[((米))
\\	/((英))
\\	黒沢明監督の映画 
\\	〔監督する人〕
\\	野球の監督 
\\	舞台監督 
\\	現場監督 
\\	映画監督 
\\	試験の監督 ((米)) 
\\	/((英)) 
\\	売り場の監督 
\\	工事監督 
\\	監督官庁 
\\	管理	かんり	〔経営(けいえい)〕
\\	〔経営,運営(うんえい)〕
\\	〔統制(とうせい)〕
\\	〔監督(かんとく)〕
\\	管理する 
\\	生産(せいさん)管理 
\\	工場(こうじょう)の管理 
\\	業務管理 
\\	遺産管理〔法律で〕 
\\	品質管理 
\\	ファイル管理 
\\	鉄鋼業(てっこうぎょう)は国家の管理下に置かれた 
\\	彼は森林の管理をしている 
\\	彼の財産の管理を頼まれた 
\\	その公園は管理がよく行き届いている 
\\	行き届く 
\\	管理価格 
\\	管理権 
\\	管理工学 
\\	((略
\\	管理社会 
\\	管理主義教育 
\\	管理職 
\\	〔人〕
\\	〔総称〕
\\	中間管理職の人々 
\\	管理職手当(てあて) 
\\	手当 
\\	管理人[者] 
\\	〔家屋・建物などの〕
\\	((米)) 
\\	((口)) 
\\	((米)) 
\\	遺産管理人 〔裁判所が任命した〕
\\	/〔遺言者が指定した〕
\\	マンションの管理人 
\\	劇場の管理人 
\\	別荘の管理人 
\\	管理能力 
\\	管理部 
\\	管理法 
\\	管理貿易 
\\	完了	かんりょう	
\\	関連	かんれん	
\\	関連する 
\\	これはそれとは何の関連もありません 
\\	植物の成長は天候と深い関連がある 
\\	今言われたことに関連して思い当たる節があります 
\\	思い当たる 
\\	思い当たるふしがある 
\\	関連会社 
\\	関連記事 
\\	関連産業 
\\	自動車関連産業 
\\	関連事項 
\\	関連質問 
\\	[〔議会での〕
\\	議員	ぎいん	
\\	記憶	きおく	
\\	気温	きおん	
\\	機械	きかい	
\\	〔総称〕
\\	〔仕掛け〕
\\	⇒きき(機器) 工作機械 
\\	精密機械 
\\	印刷機械 
\\	機械を動かす 〔始動〕
\\	/〔稼働〕
\\	機械製の 
\\	機械的な動作 
\\	/〔人の〕
\\	機械的に動く 
\\	機械油 
\\	機械編み 機械編みの 
\\	機械折り 〔製本の〕
\\	機械化 
\\	機械化する 
\\	機械化部隊 
\\	機械化農業 
\\	機械学 
\\	機械技師 
\\	機械語 〔コンピュータで〕
\\	機械工 
\\	機械工学 
\\	機械工業 
\\	機械工場 
\\	機械室 
\\	機械仕掛け 機械仕掛けの玩具 
\\	機械文明 
\\	機械翻訳 〔コンピュータで〕
\\	機械翻訳ソフト 
\\	機械力 
\\	機械論 〔哲学で〕
\\	器械	きかい	
\\	〔一式の〕
\\	光学器械 
\\	器械体操 
\\	議会	ぎかい	
\\	〔国の〕
\\	〔日本の〕
\\	〔英国の〕
\\	〔米国の〕
\\	⇒こっかい(国会) 今[第40]議会 
\\	議会を解散する[招集(しょうしゅう)する] 
\\	〔地方の〕 都[県/市]議会 
\\	議会主義 
\\	議会政治 
\\	議会制度 
\\	議会制民主主義 
\\	期間	きかん	
\\	機関	きかん	
\\	〔機械を運転する装置〕
\\	内燃(ないねん)機関 
\\	往復機関 
\\	〔組織〕
\\	〔施設〕
\\	教育機関 
\\	行政(ぎょうせい)機関 
\\	交通機関 
\\	在外機関 
\\	テレビは今や重要な報道機関である 
\\	機関紙(きかんし)[誌] 
\\	機関士(きかんし) 
\\	機関室 
\\	機関車 
\\	電気機関車 
\\	機関長 
\\	機関投資家 
\\	企業	きぎょう	
\\	企業を起こす 
\\	企業を合理化(ごうりか)する 
\\	公営(こうえい)企業 
\\	民間企業 
\\	/〔総称(そうしょう)〕
\\	零細(れいさい)企業 
\\	中小企業 
\\	大企業 
\\	/〔総称〕
\\	多国籍(たこくせき)企業 
\\	外資系企業(がいしけいかいぎょう) 
\\	独占(どくせん)企業 
\\	複合企業 
\\	企業イメージ 
\\	企業家 
\\	企業化 
\\	〔製品化〕
\\	企業化する 
\\	〔製品化する〕
\\	企業化調査 
\\	企業合併(がっぺい) 
\\	企業間信用 
\\	企業ぐるみ選挙 
\\	企業系列 
\\	企業広告 
\\	企業心 
\\	企業心に富んだ男 
\\	企業整備 
\\	企業体 
\\	企業内組合 
\\	企業内失業 
\\	企業年金制度 
\\	企業秘密 
\\	企業閉鎖(へいさ) 
\\	企業別組合(べつくみあい) 
\\	企業年金 
\\	企業連合 〔カルテル〕
\\	〔共同販売のための〕
\\	効く	きく	
\\	この薬は頭痛によく効く 
\\	テレビのコマーシャルが効いた 
\\	アスピリンは効かなかった 
\\	彼におだては効かない 
\\	ブランディーが効いてきた 
\\	期限	きげん	
\\	機嫌	きげん	
\\	機  
\\	嫌 
\\	〔安否(あんぴ)〕
\\	機嫌を伺(うかが)う 〔安否を問う〕
\\	/〔訪問する〕
\\	〔気分〕
\\	((英)) 
\\	〔いい気分〕
\\	機嫌よく 
\\	機嫌がよい 
\\	機嫌が悪い 
\\	機嫌を損(そん)じる[損(そこ)ねる] 
\\	/((文)) 
\\	機嫌を直す 〔自分で〕
\\	/〔人の〕
\\	機嫌を取る 
\\	〔取り入る〕
\\	((文)) 
\\	((口)) 
\\	((口)) 
\\	〔へつらう〕
\\	機嫌の取りにくい人だ 
\\	人の機嫌を取る人 
\\	機嫌買い 
\\	〔感情の変わりやすい人〕
\\	〔性格〕
\\	⇒機嫌を取る
\\	気候	きこう	
\\	〔天候〕
\\	は日々変わるもの,
\\	はある土地の風土を指し,長年にわたってほとんど変化しないものをいう); 〔時候〕
\\	海洋性気候 
\\	温和(おんわ)な気候 
\\	気候不順(ふじゅん)なおりから 
\\	おりから 
\\	気候の変わり目に 
\\	気候学 
\\	気候帯 
\\	岸	きし	
\\	生地	きじ	
\\	〔布地(ぬのじ)〕
\\	〔材料としての布〕
\\	目の粗い[柔らかい]生地 
\\	ワイシャツの生地 
\\	ドレスの生地 
\\	紳士服(しんしふく)の生地 
\\	生地の見本 
\\	/((米)) 
\\	〔パンなどの〕
\\	〔素焼きの陶磁器〕
\\	〔素焼きしないもの〕
\\	〔地のままの性質〕 生地のままで人と付き合う 
\\	記事	きじ	
\\	技師	ぎし	
\\	記者	きしゃ	
\\	傷	きず	
\\	期待	きたい	
\\	…を期待して 
\\	彼の成功を期待している 
\\	昇進(しょうしん)を期待して一生懸命働いた 
\\	あの人にはあまり期待しないほうがいいよ 
\\	彼は愛弟子(まなでし)に期待をかけていた 
\\	新しい療法(りょうほう)で彼の容体(ようたい)の著(いちじる)しい好転(こうてん)が期待される 
\\	彼は父の期待に添(そ)うだろう 
\\	両親の期待を裏切る 
\\	期待に反して落選(らくせん)した 
\\	昇給(しょうきゅう)は期待薄(うす)だ 
\\	残念ながらご依頼(いらい)のご期待に添うことはできません 
\\	それは期待外れの結果に終わった 
\\	[((口)) 
\\	期待外れ 
\\	ヘンリーについてたくさんいいことを聞かされて、彼に会える日を楽しみにしていたのに、会ってみたらとんだ期待外れだった 
\\	そのパーティはとても面白そうに思えたのだが、期待外れだった 
\\	気体	きたい	
\\	〔ガス〕
\\	〔蒸気〕
\\	((英)) 
\\	気体燃料(ねんりょう) 
\\	気体化 
\\	帰宅	きたく	
\\	貴重	きちょう	貴重な 
\\	は金銭的にまた有用性の上で価値があるものに,
\\	は本来高い価値のあるもの,大切なものなどに用いる) 貴重な経験 
\\	貴重なお時間 
\\	貴重品 
\\	貴重品を金庫にしまっておく 
\\	議長	ぎちょう	
\\	きちんと	きちんと	
\\	〔整った様子〕 きちんとした 
\\	きちんとした服装の人 
\\	きちんとした部屋 
\\	〔正確に〕
\\	きちんと書き留めてください 
\\	きちんと約束の時間に着く 
\\	きちんと公私のけじめをつける 
\\	けじめ 
\\	〔規則正しく〕
\\	〔秩序正しい様子〕
\\	きちんとした 
\\	借金(しゃっきん)を毎月きちんと払う 
\\	きちんとした生活をする 
\\	本をきちんと整頓(せいとん)する 
\\	しきたりをきちんと守る人だ 
\\	〔適切に〕
\\	すべての面できちんとした人 
\\	きちんと座る 
\\	きつい	きつい	
\\	〔苛酷(かこく)なまでに厳しい〕
\\	〔つらい〕
\\	〔規則などが厳しい〕
\\	きつい仕事 
\\	きつい日程 
\\	暑さがきつい 
\\	当時は労働運動の取り締(しま)りがきつかった 
\\	きつくしかる 
\\	きついことを言うなあ 
\\	〔強烈(きょうれつ)な〕 きつい酒 
\\	〔気が強い〕 
\\	きつい顔付き 
\\	なかなかきつい子だった 
\\	〔窮屈(きゅうくつ)な〕
\\	靴がきつい 
\\	ベルトをきつくする 
\\	ひもをきつく縛(しば)る 
\\	気付く	きづく	
\\	彼はだれかが後について来るのに気付いた 
\\	彼女は近づく危険に気付いていた 
\\	彼がいらいらしていることにすぐ気付いた 
\\	財布(さいふ)をなくしたのに気付いた 
\\	彼の容態(ようだい)の悪さに気付かなかった 
\\	容態 
\\	あまり背が高くなったので彼だとは気付かなかった 
\\	彼は彼女を傷付けているとは気付かなかった 
\\	彼の間違いはだれにも気付かれずに済んだ 
\\	彼に気付かれずに机の中を調べた 
\\	気に入る	きにいる	
\\	記入	きにゅう	
\\	記入する 
\\	((米)) 
\\	((英)) 
\\	名簿に名前を記入する 
\\	申込書(もうしこみしょ)に名前を記入しなさい 
\\	記入済み 
\\	記入漏(も)れのないように皆の名前を確(たし)かめてくれ 
\\	記念	きねん	
\\	機能	きのう	
\\	機能する 
\\	機能的 (に) 
\\	消化(しょうか)機能 
\\	脳の機能 
\\	スプリンクラーは出火(しゅっか)とともに機能した 
\\	いっさいの機能が停止した 
\\	この机はなかなか機能的にできている 
\\	機能キー 〔コンピュータなどの〕
\\	機能障害 
\\	機能設計(せっけい) 
\\	機能不全 
\\	計器(けいき)が機能不全に陥(おちい)った 
\\	気の毒	きのどく	
\\	寄付	きふ	
\\	寄付する 
\\	((口)) 
\\	は他の人々と出し合う
\\	は署名して寄付を約束する) 寄付を募(つの)る 
\\	バザーに衣類を寄付した 
\\	コンピュータ5台が寄付された 
\\	地元の会社数社が100万円ずつ寄付すると申し出た 
\\	寄付の申し出がとても少ない 
\\	寄付金 
\\	多額の金を寄付しその利息を運用してもらう場合) 寄付者 
\\	希望	きぼう	
\\	基本	きほん	
\\	決まり	きまり	
\\	〔規則〕
\\	⇒きそく(規則) 会の決まりを守る 
\\	決まりを破る 
\\	〔解決〕
\\	〔決定〕
\\	〔合意〕
\\	〔取り決め〕
\\	この件はなかなか決まりがつかない 
\\	よし,それで話は決まりだ 
\\	話は決まりがついた 
\\	早く仕事に決まりをつけよう 
\\	〔習慣〕
\\	会社の帰りに一杯やるのが彼の決まりだ 
\\	また彼のお決まりのぼやきが始まった 
\\	〔けじめ〕
\\	生活にきちんと決まりをつけたらどうだ 
\\	気味	きみ	
\\	彼女は自信過剰(かじょう)の気味がある 
\\	二日酔(ふつかよ)いの気味がある 
\\	奇妙	きみょう	
\\	義務	ぎむ	
\\	疑問	ぎもん	
\\	〔疑い〕
\\	〔質問〕
\\	⇒うたがい(疑い) 疑問に答える 
\\	そんなことが可能かどうか疑問だ 
\\	その数字については疑問がある 
\\	彼の信頼性については疑問の余地(よち)がある 
\\	彼がスパイであることは疑問の余地がない 
\\	疑問詞 
\\	疑問符(ふ) 
\\	疑問文 〔文法で〕
\\	逆	ぎゃく	
\\	旧	きゅう	
\\	級	きゅう	
\\	球	きゅう	
\\	休暇	きゅうか	
\\	休憩	きゅうけい	
\\	10分間の休憩をする 
\\	ちょっと休憩しよう 
\\	[((口)) 
\\	は5分ほど休憩するという意味から) 休憩のための停車〔バス旅行で〕 
\\	休憩時間 ((米)) 
\\	((英)) 
\\	休憩室 〔ホテル・クラブ・船などの〕
\\	休憩所 
\\	急激	きゅうげき	急激な 〔突然の〕
\\	〔速い〕
\\	〔急ではげしい〕
\\	急激な政策の変化 
\\	車の事故による死亡者は急激に増加している 
\\	株は急激な上下動(じょうげどう)を見せている 
\\	吸収	きゅうしゅう	
\\	〔吸い取ること〕
\\	吸収する 
\\	この酵素(こうそ)は蛋白(たんぱく)質の消化吸収を促す 
\\	この布は汗をよく吸収する 
\\	〔同化〕
\\	〔同化,併合〕
\\	知識の吸収 
\\	彼らは新知識の吸収に没頭(ぼっとう)した 
\\	/((口)) 
\\	小企業は大企業に吸収された 
\\	吸収合併(がっぺい) 
\\	吸収合併される 
\\	吸収剤 
\\	吸収スペクトル 
\\	吸収力 
\\	吸収力のある 
\\	若者は吸収力がある 
\\	救助	きゅうじょ	
\\	急速	きゅうそく	
\\	休息	きゅうそく	
\\	急に	きゅうに	
\\	給料	きゅうりょう	
\\	注 
\\	供給(きょうきゅう) 
\\	器用	きよう	
\\	〔上手な様子〕 器用な大工(だいく) 
\\	彼女は指先(ゆびさき
\\	が器用だ 
\\	彼は器用に道具(どうぐ)を使う 
\\	彼はなんでも器用にする 
\\	彼女はたくさんの仕事を器用にこなしている 
\\	こなす 
\\	少年は器用に人込みをくぐり抜けた 
\\	〔抜け目のない様子〕 商(しょう)取り引きにかけては彼ほど器用な者はいない 
\\	彼は器用に世渡(よわた)りをする 
\\	器用貧乏 彼は器用貧乏だ 
\\	教科書	きょうかしょ	
\\	競技	きょうぎ	
\\	行儀	ぎょうぎ	彼は行儀を知らない 
\\	お行儀はどうしたの 
\\	なんて行儀のよい[悪い]子だろう 
\\	食べ物を口に入れてしゃべるのは行儀の悪いことだ 
\\	行儀よくしなさい 
\\	行儀よく座りなさい 
\\	行儀作法 
\\	〔礼法〕
\\	行儀見習(みなら)い 行儀見習いをする 
\\	供給	きょうきゅう	
\\	電力の供給を止める 
\\	これが町にガスを供給する管(くだ)です 
\\	需要(じゅよう)と供給のバランス 
\\	供給源 
\\	供給者 
\\	教授	きょうじゅ	
\\	〔教えること〕
\\	教授する 
\\	伊藤先生の教授を受けている 
\\	加藤先生からピアノの個人教授を受けています 
\\	〔大学の教員〕
\\	指導教授 
\\	/((英)) 
\\	小野教授 
\\	助教授 
\\	準教授 
\\	正教授 
\\	九州大学言語学教授 
\\	当大学に経済学教授の席が設(もう)けられた 
\\	教授会 〔組織〕
\\	〔集会〕
\\	教授陣 
\\	教授法 
\\	強調	きょうちょう	強調する 
\\	このことを特に強調して言いたいのです 
\\	この柄(がら)は特に斜(なな)めの線を強調しました 
\\	共通	きょうつう	共通の 
\\	共通の利害 
\\	彼は我々の共通の友人である 
\\	離婚の急増は世界共通の現象だ 
\\	これが彼ら全員に共通する欠点だ 
\\	共通語 
\\	共通点 
\\	二人には全く共通点がない 
\\	共同	きょうどう	
\\	〔力を合わせること〕
\\	〔一緒にすること〕
\\	共同する 
\\	彼と共同で執筆(しっぴつ)した 
\\	共同で店を経営(けいえい)する 
\\	車を共同で使用する 
\\	我々は共同で責任(せきにん)を負(お)わねばならない 
\\	共同で炊事(すいじ)をしている〔農家などが〕 
\\	共同アンテナテレビ 
\\	共同一致 共同一致で 
\\	共同運航 〔飛行機の〕
\\	共同覚書 
\\	共同会見 共同会見を行う 
\\	共同開催 
\\	2002年の日韓共同開催(かいさい)のワールドカップ 
\\	共同加入電話 
\\	共同管理 
\\	共同管理の下におく 
\\	共同企業体 
\\	共同記者会見 
\\	共同経営 
\\	共同経営する 
\\	共同経営者 
\\	共同決議 〔衆参両院の〕
\\	共同研究 
\\	共同財産 
\\	共同事業[企業] 
\\	共同市場 
\\	共同執筆 
\\	共同社会 
\\	共同出資(しゅっし) 
\\	共同出資する 
\\	共同生活 
\\	友達と共同生活をする 
\\	共同制作 
\\	共同制作品 
\\	共同正犯 
\\	共同声明 共同声明を出す 
\\	共同宣言 共同宣言をする 
\\	共同戦線 共同戦線を張る 
\\	共同体 
\\	共同体意識 
\\	共同販売 
\\	共同販売する 
\\	共同販売店 
\\	共同便所 
\\	共同謀議 
\\	共同募金 
\\	共同募金をする 
\\	共同墓地 
\\	恐怖	きょうふ	
\\	協力	きょうりょく	
\\	強力	きょうりょく	
\\	許可	きょか	
\\	局	きょく	
\\	巨大	きょだい	
\\	嫌う	きらう	
\\	霧	きり	
\\	切れ	きれ	
\\	切り 
\\	〔断片〕
\\	〔薄片〕
\\	〔紙などの細長い1片〕
\\	〔切れ端〕
\\	木の切れ 
\\	パン2切れ 
\\	肉の1切れ 
\\	/〔豚・子羊の〕
\\	は通例骨付きの)/〔牛の〕
\\	紙切れ 
\\	〔布〕
\\	もめんのきれ 
\\	〔切れあじ〕 
\\	切れのよいカーブ 
\\	〔水分などがなくなること〕 この布は水の切れがよい 
\\	切れる	きれる	
\\	(切断される)
\\	(刃物が)
\\	(鋭利である)
\\	(堤が)
\\	(関係が)
\\	(電話が)
\\	(期限が)
\\	(品切れになる)
\\	(使い果たす)
\\	(電池が)
\\	(頭が)
\\	(かっとなる)
\\	〔切断される〕
\\	糸が切れた 
\\	ヒューズが切れた 
\\	雲が切れた 
\\	電球が切れた 
\\	土手(どて)が切れた 
\\	〔傷つく〕 指が切れた 
\\	手の切れるような札 
\\	その子の手は寒さでひびが切れていた 
\\	ひび 
\\	〔切れ味がよい〕
\\	このナイフはよく切れる 
\\	このナイフは切れなくなった 
\\	〔擦り切れる〕
\\	彼の洋服は切れている 
\\	〔関係(かんけい)がなくなる〕 彼女とは手が切れた 
\\	彼女とは音信が切れてしまった 
\\	音信(おんしん) 
\\	〔とぎれる〕 
\\	私は息が切れてもう走れない 
\\	電話が切れた 〔相手が切った〕
\\	/〔通じなくなった〕
\\	〔尽きる〕
\\	その品は切れている 
\\	砂糖が切れた 
\\	自転車の油が切れた 
\\	薬が切れたらまた熱が出てきた 
\\	電池が切れた 
\\	〔終わりになる〕
\\	協定の期限は先月切れた 
\\	リポートの提出期限が切れている 
\\	時間が切れましたよ 
\\	〔下回る〕
\\	これは1キロに1グラム切れる 
\\	〔敏腕(びんわん)である〕
\\	彼はなかなか切れる 
\\	〔しびれが切れる〕 痺れを切らす 
\\	店員が出て来ないので客はしびれを切らして出て行った 
\\	しびれ
\\	痺れ】 
\\	〔曲がる〕 球は右に切れた 
\\	記録	きろく	
\\	議論	ぎろん	
\\	金	きん	
\\	銀	ぎん	
\\	禁煙	きんえん	
\\	金額	きんがく	
\\	金庫	きんこ	
\\	禁止	きんし	
\\	金銭	きんせん	
\\	金属	きんぞく	
\\	近代	きんだい	
\\	緊張	きんちょう	
\\	〔心身が引き締まること〕
\\	緊張する 
\\	緊張した面(おも)もち[声] 
\\	社長の前で緊張してろくろく意見も言えなかった 
\\	彼女の笑顔で私たちの緊張がほぐれた 
\\	ほぐれる 
\\	〔今にも争いが起こりそうなこと〕
\\	通例複数形) 東西の緊張緩和(かんわ) 
\\	緊張する国境地帯 
\\	緊張緩和 〔国際関係の〕
\\	筋肉	きんにく	
\\	金融	きんゆう	
\\	〔資金の貸し借り〕
\\	金融が緩慢(かんまん)である[逼迫(ひっぱく)している] 
\\	金融の引き締(し)めを行う 
\\	金融の引き締(し)めがだいぶ緩(ゆる)やかになった 
\\	小口(こぐち)金融 
\\	金融界 
\\	金融界の不況 
\\	金融会社 
\\	金融緩慢(かんまん) 
\\	金融緩和[引き締め] 
\\	金融緩和政策 
\\	金融機関 
\\	金融恐慌(きょうこう) 金融恐慌に見舞われる 
\\	金融業 
\\	金融業者 
\\	金融公庫(こうこ) 
\\	金融コングロマリット 
\\	金融債(きんゆうさい) 
\\	金融先物取り引き 
\\	金融政策 
\\	金融市場 〔狭義の〕
\\	〔広義の〕
\\	金融市場から締め出される 
\\	金融資本 
\\	金融資本家 
\\	金融自由化 
\\	金融情勢 
\\	金融筋 
\\	金融組織 
\\	金融統制(とうせい) 
\\	金融引き締め政策 
\\	金融ビッグバン 
\\	1986年の英国の金融自由化) 金融逼迫(ひっぱく) 
\\	金融不安 
\\	((口)) 
\\	句	く	
\\	〔言葉・表現の一区切り〕
\\	〔2語以上の語の集まり〕
\\	その詩を一字一句間違いなく暗記した 
\\	⇒にのく(二の句) 
\\	〔俳句〕
\\	句を作る 
\\	この情景は句になる 
\\	句動詞 
\\	食う	くう	
\\	(食べる)
\\	(魚・虫などが)
\\	(生計を立てる)
\\	(時間・費用を)
\\	(消費する)
\\	⇒たべる(食べる) 
\\	〔生活をする〕
\\	食うに困らないだけの収入がある 
\\	食うや食わずの生活だった 
\\	〔虫などが〕 脚を蚊に食われた 
\\	コートを虫に食われた 
\\	〔激しく相手に迫る〕
\\	⇒くってかかる(食って掛かる) 
\\	彼女の作品をけなしたといって私に食って掛かった 
\\	〔相手の勢力・領分を侵す〕⇒くいこむ(食い込む) スーパーに食われて商店街の売り上げが減った 
\\	食うか食われるかの死闘が続いた 
\\	主役が脇役(わきやく)に食われた 
\\	〔強い相手を負かす〕 新人が昨年のチャンピオンを食った 
\\	〔時間・費用がかかる〕 これは時間と金を食う仕事だ 
\\	大型車はガソリンを食う 
\\	この仕事は手間ひまを食う 
\\	〔年をとる〕 年を食った男 
\\	〔被(もうむ)る〕 小言(こごと)を食った 
\\	偶然	ぐうぜん	
\\	偶然の 
\\	偶然に 
\\	偶然性 
\\	〔哲学〕
\\	偶然の出来事 
\\	偶然の出会い 
\\	不思議な偶然の一致で 
\\	彼に空港で会ったのは全くの偶然だった 
\\	偶然にも彼らは現場にいた 
\\	偶然ある考えが浮かんだ 
\\	彼は答が一致したのは偶然だと強調した 
\\	臭い	くさい	
\\	鎖	くさり	
\\	腐る	くさる	
\\	癖	くせ	
\\	〔習癖(しゅうへき)〕
\\	彼は教室で後ろに座る癖がある 
\\	彼は深刻((しんこく)に考え過ぎる癖がある 
\\	彼に一度金を貸すと癖になるからやめておいたほうがいい 
\\	夜更(よふ)かしの癖がついた 
\\	シンナーをやる癖がついた 
\\	怠け癖を直す 
\\	息子の爪を噛(か)む癖を直そうとした 
\\	いつもの癖でネクタイをいじり始めた 
\\	〔普通と変わった特徴〕 癖のある文章を書く 
\\	癖のある物の言い方をする 
\\	アメリカ人の友人は癖のない日本語をしゃべる 
\\	〔普通と変わった状態〕 髪の癖 
\\	癖のある髪の毛 
\\	ズボンにへんな癖がついた 
\\	無くて七癖 
\\	はちょっとした癖)
\\	具体	ぐたい	
\\	具体的 (に) 
\\	具体的な例をあげる 
\\	具体的に言えば 
\\	君の考えを具体的に言い表(あらわ)しなさい 
\\	具体的なことはまだ何も聞いていない 
\\	具体案 
\\	具体化 
\\	〔実現化〕
\\	具体化する 
\\	〔実現する〕
\\	私の長年(ながねん)の夢が具体化した 
\\	具体性 
\\	下り	くだり	
\\	〔下降(かこう)〕
\\	ここから道は下りになる 
\\	〔低い方へ行くこと〕 川下りをする 
\\	下りのエレベーターに乗る 
\\	〔中心地から地方へ行くこと〕 下り列車 
\\	午後8時の下り急行 
\\	下り坂 
\\	〔坂の〕
\\	〔状態の〕 このところ物価[景気]は下り坂(さか)だ 
\\	天気が下り坂だ 
\\	[((口))
\\	暑さも下り坂になった 
\\	[((文))
\\	彼の人気も下り坂に向かった 
\\	下り線 
\\	下る	くだる	
\\	(下回る)
\\	(命令などが)
\\	(降参する)
\\	(おなかが)
\\	〔低い所に移る〕
\\	⇒おりる(降りる) 私たちは急(いそ)いで山を下った 
\\	船はゆるやかに川を下って行った 
\\	坂を下ったところに学校がある 
\\	〔下回(したまわ)る〕
\\	球場(きゅうじょう)には5万を下らない観衆(かんしゅう)が詰めかけた 
\\	〔中心地から地方へ行く〕 私は新幹線で博多まで下った 
\\	九州へ下る 
\\	〔申し渡される〕
\\	被告(ひこく)に懲役(ちょうえき)2年の判決が下った 
\\	撤退(てったい)せよとの命令が下った 
\\	この選挙によって政府に対する国民の審判が下った 
\\	〔時代が後になる〕 下って明治になるとこの地方の開発(かいはつ)は盛んとなった 
\\	〔官職を辞める〕
\\	野に下る ⇒や(野) 
\\	〔下痢(げり)をする〕 ここ2,3日腹(はら)が下っている 
\\	〔降参(こうさん)する〕
\\	((文)) 
\\	我々は力尽(ちからつ)きて敵の軍門(ぐんもん)に下った 
\\	苦痛	くつう	
\\	ぐっすり	ぐっすり	
\\	ぐっすり眠る 
\\	ぐっすり眠っている 
\\	ぐっすり眠って気分爽快(きぶんそうかい)だ 
\\	区別	くべつ	
\\	〔差異〕
\\	区別する 
\\	は特徴により,
\\	は微妙な違いや価値判断により区別する) 人種(じんしゅ)・年齢(ねんれい)・性の区別なく 
\\	この鳥は色や大きさでは雌雄(しゆう)の区別がつけられない 
\\	旅館とホテルにはどんな区別がありますか 
\\	彼女は実子(じっし)と継子(ままこ)の区別なく慈(いつく)しんだ 
\\	君はもう善悪(ぜんあく)の区別のつかない年ではない 
\\	親切と単なるお節介(せっかい)を区別する 
\\	彼らは公私の区別を明確(めいかく)にすべきだ 
\\	〔分類〕
\\	区別する 
\\	組	くみ	
\\	組合	くみあい	
\\	〔労働者の〕
\\	〔同業組合〕
\\	⇒きょうどうくみあい(協同組合),ろうどうくみあい(労働組合) 組合を組織する 
\\	組合に加入する 
\\	消費者組合を組織する 
\\	組合員 
\\	〔労働組合の〕
\\	組合活動 
\\	組合幹部 
\\	組合規約 
\\	組合専従者 
\\	組合組織 
\\	組合長 
\\	組合費 
\\	組む	くむ	
\\	〔組み立てる〕
\\	彼は丸太(まるた)を組んで小屋を作った 
\\	足場(あしば)を組む 
\\	〔交差させて組む〕
\\	脚を組む 
\\	彼は腕を組んで戸口(とぐち)に立っていた 
\\	労働者が腕を組んで市中を行進した 
\\	〔活字を〕
\\	9ポイントで[小文字で]組む 
\\	見出しを組む 
\\	〔仲間になる,組になる〕
\\	〔競技で〕
\\	青木氏は小野氏と組んで事業を始めた 
\\	次のゲームでは僕と組まないか 
\\	彼がジムと組んでも勝ち味はない 
\\	バレーのチームを組む 
\\	執事(しつじ)は家政婦と組んで主人の金品をくすねていた 
\\	〔取り組む,対戦する〕
\\	〔つかみ合う〕
\\	両軍は四つに組んだ 
\\	〔為替(かわせ)を〕
\\	彼あてに50万円の為替を組んだ 
\\	汲む	くむ	
\\	〔液体をすくう〕
\\	〔バケツなどで〕
\\	ひしゃくでバケツに水を汲む 
\\	風呂に水を汲んでおいてくれ 
\\	〔摂取(せっしゅ)する〕
\\	彼は宝生流(ほうしょうりゅう)の流れを汲む能楽師(のうがくし)だ 
\\	流れを汲む 
\\	芭蕉(ばしょう)の句は汲めども尽(つ)きぬ味わいがある 
\\	〔推察する〕
\\	〔推量する〕
\\	〔考慮に入れる〕
\\	彼の気持ちを汲んで事を穏便(おんびん)に取り計らった 
\\	酌む	くむ	
\\	〔注ぐ〕
\\	〔飲む〕
\\	私たちは雪を見ながら酒を酌んだ 
\\	悔しい	くやしい	〔屈辱(くつじょく)を感じさせる〕
\\	〔残念な〕
\\	〔腹立(はらだ)たしい〕
\\	((口)) 
\\	もっと気を付ければよかったと思うと悔しい 
\\	人前でののしられるなんて悔しいことだ 
\\	罵(ののし)る 
\\	あんな下手な選手に負けて悔しかった 
\\	それを実行する勇気がないのがわれながら悔しい 
\\	我ながら 
\\	今さら悔しがったってしようがない ((諺)) 
\\	[((英)) 
\\	位	くらい	
\\	〔地位〕
\\	〔位階〕
\\	位の高い人 
\\	高い位に上る 
\\	位が高い[低い] 
\\	位を上げる[下げる] 
\\	彼は私より位が上だ 
\\	彼は位人臣(じんしん)を極めた 
\\	〔王位〕
\\	位につく[を退く] 
\\	ヘンリー8世(せい)は何年間位についていたか 
\\	〔数値のけた〕 
\\	の位 
\\	位を間違える 
\\	位する 
\\	〔位置〕 
\\	〔地位を占める〕 その会社は大手の下位(かい)に位している 
\\	彼女の家は中流の上に位する 
\\	〔位置を占める〕 その国の首都は少し北の方に位している 
\\	暮らし	くらし	
\\	暮らす	くらす	
\\	〔生活する〕
\\	〔生計を立てる〕
\\	彼女は幸せに暮らしている 
\\	生まれた時からずっとここで暮らしている 
\\	いかがお暮らしですか 
\\	月10万で暮らせますか 
\\	安月給(やすげっきゅう)でどうにか暮らしている 
\\	牧畜(ぼくちく)で安楽に[やっと]暮らしている 
\\	〔時を過ごす〕 夏休みは本を読んで暮らした 
\\	ぶらぶらと遊んで暮らす 
\\	子供をなくしてから彼女は毎日泣き暮らした 
\\	繰り返す	くりかえす	
\\	狂う	くるう	
\\	〔心の動きがおかしくなる〕
\\	失恋(しつれん)して気が狂った 
\\	くやしさで気も狂わんばかりだった 
\\	かわいそうにとうとう気が狂ったのだ 
\\	気が狂ったようにわめきちらした 
\\	[((口))
\\	喚き散らす 
\\	〔正常でなくなる〕 この体重計は狂っている 
\\	この時計は狂わない 
\\	世の中どこか狂っていると思わないか 
\\	投手は第1球を打たれてから調子が狂った 
\\	彼の歌は調子が狂っている 
\\	踏切(ふみきり)事故のため電車のダイヤが狂っている 
\\	ダイヤ 
\\	索引(さくいん)カードの順序(じゅんじょ)が狂っている 
\\	〔予想しない状態になる〕 ねらいが狂った 
\\	病気のため計画が狂った 
\\	〔夢中になる〕 女に狂っている 
\\	苦しい	くるしい	
\\	(痛い)
\\	(困難な)
\\	(厄介な)
\\	(家計が)
\\	(無理な)
\\	〔体が〕 食べすぎておなかが苦しい 
\\	苦しそうな足取りで 
\\	病人は苦しそうに息をしていた 
\\	階段を駆け上がったら息が苦しくなった 
\\	胸が苦しい 〔痛い〕
\\	/〔締め付けられるようだ〕
\\	〔精神的に〕⇒つらい(辛い) 苦しい仕事 
\\	苦しい一年だった 
\\	〔困窮している〕 彼らは苦しい生活をしている 
\\	/((文)) 
\\	この苦しい財政を立て直さなければならない 
\\	〔困惑させられる〕 苦しい立場におかれた 
\\	/((口)) 
\\	〔無理な〕 苦しい言い訳をした 
\\	苦しい笑いを浮かべた 
\\	苦しい時の神頼み ((諺)) 
\\	苦しむ	くるしむ	
\\	〔肉体的に〕
\\	歯痛(しつう)に苦しんでいる 
\\	〔悩む〕 彼は息子の非行に苦しんでいた 
\\	彼女は自分の将来のことで苦しんでいる 
\\	彼は失敗したことでまだ苦しんでいる 
\\	〔困る〕 彼の論点は理解に苦しむ 
\\	〔骨折る〕
\\	この数学の問題を解くのに苦しんだ 
\\	暮れ	くれ	
\\	〔日暮(ひぐ)れ〕
\\	日暮れに 
\\	暮れの鐘(かね) 
\\	〔季節の終わり〕
\\	秋の暮れのある夕べに 
\\	年の暮れに 
\\	暮れの売り出し[大掃除(おおそうじ)] 
\\	苦労	くろう	〔辛苦〕
\\	〔骨折り〕
\\	〔心配〕
\\	親には大変苦労をかけた 
\\	病弱な息子が苦労の種だった 
\\	苦労の多い一生だった 
\\	彼は何の苦労も知らずに育った 
\\	君はまだ苦労が足りないね 
\\	彼女を喜ばせるためならなんの苦労もいとわなかったろう 
\\	苦労のかいがあった 
\\	どれほど苦労してきたか誰にもわからない 
\\	ご苦労をかけてすみません ⇒ごくろう(御苦労) 苦労して彼の機嫌をとる必要はない 
\\	未亡人はいろいろ苦労して子供たちを育てた 
\\	薄暗がりで苦労してその手紙を読んだ 
\\	彼らは苦労して大雪の中を進んだ 
\\	当時は金で苦労していた 
\\	苦労性 彼は苦労性だ 
\\	苦労人 
\\	加える	くわえる	
\\	〔加算する〕
\\	3に4を加えると7だ 
\\	子供を加えて一行は50名であった 
\\	〔増す〕⇒ます(増す)1 
\\	〔仲間に入れる〕 彼を仲間に加えたらどうだろうか 
\\	〔与える〕 彼は男に一撃(いちげき)を加えた 
\\	危害(きがい)を加える 
\\	敵(てき)は町に爆撃(ばくげき)を加えた 
\\	医者はすぐに治療(ちりょう)を加えたが手遅れだった 
\\	くわえる	くわえる	咥える 
\\	パイプをくわえる 
\\	猫が魚をくわえて逃げていった 
\\	子供が指をくわえてケーキを見つめていた 
\\	詳しい	くわしい	
\\	〔詳細(しょうさい)な〕
\\	詳しいことは知りません 
\\	残念ながら今詳しいことはお話しできません 
\\	その問題を詳しく説明してください 
\\	詳しく調べる 
\\	後(のち)ほどもっと詳しくお知らせします 
\\	〔精通(せいつう)している〕 
\\	彼は銀座に詳しい 
\\	彼は中近東の情勢(じょうせい)に詳しい 
\\	/((文)) 
\\	彼は会社の内部事情に詳しい 
\\	[((口))
\\	加わる	くわわる	
\\	〔増(ま)す〕
\\	ほめられて彼女のほおに赤みが加わった 
\\	今回の選挙で与党(よとう)には更(さら)に15議席(ぎせき)が加わった 
\\	夕方(ゆうがた)になって風速が加わった 
\\	日増(ひま)しに暑さが加わった 
\\	第1戦に勝ってチームに活気(かっき)が加わった 
\\	〔付加(ふか)される〕
\\	娘の結婚に息子の昇進(しょうしん)が加わって両親は二重の喜びに包(つつ)まれた 
\\	彼女は年とともに柔軟(じゅうなん)さが加わった 
\\	〔仲間入りする〕
\\	討論(とうろん)に加わる 
\\	あの仲間には加わってはいけない 
\\	セミナーに加わる 
\\	田中氏が味方(みかた)に加わったのはたいへんありがたい 
\\	訓	くん	
\\	軍	ぐん	
\\	郡	ぐん	
\\	西多摩郡 
\\	郡部 
\\	軍隊	ぐんたい	
\\	訓練	くんれん	
\\	計	けい	
\\	敬意	けいい	
\\	経営	けいえい	
\\	経営する 
\\	多角経営 
\\	事業を経営する 
\\	クラブを経営する 
\\	店を経営する 
\\	ホテル[学校]を経営している 
\\	父親は彼に農場の経営を任せた 
\\	彼には経営の才がない 
\\	経営学 
\\	経営学修士 
\\	経営管理 
\\	経営工学 
\\	経営合理化 
\\	経営合理化を行う 
\\	経営コンサルタント 
\\	経営参加 
\\	経営者 
\\	〔総称〕
\\	〔店・ホテルなどの主人〕
\\	経営者側と労働者側の調停をする 
\\	経営情報システム 
\\	((略
\\	経営陣 
\\	経営多角化 
\\	この会社は経営多角化をめざしている 
\\	経営多角化する 
\\	経営難 会社は経営難に陥っている 
\\	経営費 ((米)) 
\\	((英)) 
\\	経営方針 
\\	景気	けいき	
\\	円高不況(えんだかふきょう) 
\\	恐慌(きょうこう) 
\\	景気循環(じゅんかん)の谷間(たにま) 
\\	景気動向 
\\	景気動向指数 
\\	景気の悪化(あっか) 
\\	景気の一服 
\\	景気の回復 
\\	景気の拡大 
\\	景気の過熱(かねつ) 
\\	景気の後退(こうたい) 
\\	景気の好転 
\\	景気の循環 
\\	((米))
\\	((英))
\\	景気の底 
\\	景気の頂上 
\\	景気の停滞(ていたい) 
\\	景気の中だるみ 
\\	中弛み 
\\	景気の不透明感(ふとうめいかん) 
\\	景気の変動 
\\	景気の冷却(れいきゃく) 
\\	景気見通し 
\\	傾向	けいこう	
\\	英文学の最近の傾向 
\\	物価は上昇の傾向を示している 
\\	世論の傾向は戦争[戦争反対]に向かっている 
\\	私は太る傾向がある 
\\	〔性向〕
\\	は特に自分の好みへ,
\\	はしばし ば好ましくない方向へ向かう性質に用いる) 彼女は大げさに言う傾向がある 
\\	彼らは保守的(ほしゅてき)な傾向がある 
\\	彼は仕事を怠ける傾向がある 
\\	〔左翼(さよく)思想の〕 傾向文学 
\\	警告	けいこく	
\\	計算	けいさん	
\\	掲示	けいじ	
\\	刑事	けいじ	
\\	私服刑事(しふくけいじ) 
\\	刑事上の責任を問(と)われる 
\\	問う 
\\	刑事裁判所 
\\	刑事事件 
\\	刑事事件を起こした 
\\	刑事責任 
\\	子供だから刑事責任はない 
\\	刑事訴訟(そしょう) 
\\	刑事訴訟法(そしょうほう) 
\\	刑事犯 
\\	刑事犯の容疑 
\\	刑事補償(ほしょう) 
\\	刑事補償法 
\\	芸術	げいじゅつ	
\\	契約	けいやく	
\\	経由	けいゆ	
\\	ホノルル経由でメキシコへ行く 
\\	その情報はフランス大使館経由で入手(にゅうしゅ)した 
\\	劇	げき	
\\	劇場	げきじょう	
\\	化粧	けしょう	
\\	けち	けち	
\\	〔しみったれ〕
\\	〔けちな人〕
\\	((口)) 
\\	けちな 
\\	金にけちである 
\\	なんというけちだ 
\\	[((米))
\\	〔みすぼらしい〕
\\	けちな外套(がいとう) 
\\	けちな家 
\\	〔くだらない〕
\\	〔卑しい〕((文)) 
\\	〔心などが狭い〕
\\	けちな心配事 
\\	けちな野郎だ 
\\	けちな根性(こんじょう)をしている 
\\	けちな了見(りょうけん)を起こすな 
\\	けちがつく 
\\	《事が主語》 
\\	《口語》 
\\	この企画(きかく)は最初からけちがついた 
\\	けちをつける  
\\	私の案にけちをつけた 
\\	あいつは僕のすることに一々けちをつける 
\\	血液	けつえき	
\\	結果	けっか	
\\	欠陥	けっかん	〔重大な〕
\\	〔完全さを損なう比較的小さな〕
\\	欠陥のある 
\\	精神的[身体的]に欠陥がある 
\\	彼は性格上,何か欠陥がある 
\\	制度の欠陥 
\\	欠陥車 
\\	((俗)) 
\\	欠陥車回収(かいしゅう) 
\\	会社は欠陥車を回収した 
\\	欠陥商品 
\\	結局	けっきょく	
\\	この問題は結局だれにも分からなかった 
\\	結局会社の役に立ったじゃないか 
\\	結局実力が勝つ 
\\	結局はもうけのほうが多いと思う 
\\	いろいろ努力したが結局駄目だった 
\\	決心	けっしん	
\\	欠席	けっせき	
\\	決定	けってい	
\\	私は行くことに決定しました 
\\	旅行の日取(ひど)りは10月5,6日と決定した 
\\	僕の後任(こうにん)が決定した 
\\	彼のその一言で私の進路が決定した 
\\	決定的勝利(しょうり) 
\\	決定的な証拠(しょうこ) 
\\	カメラマンはミサイル発射の決定的瞬間をとらえた 
\\	わがチームの勝利は決定的となった 
\\	決定権 
\\	私には決定権はない 
\\	/((口)) 
\\	決定権を持つ人 
\\	決定打 〔野球で〕
\\	〔ボクシングなどで〕
\\	決定戦 優勝決定戦 
\\	3位決定戦 
\\	決定投票 
\\	決定版 
\\	和英辞典の決定版 
\\	これこそ水虫(みずむし)治療薬(ちりょうやく)の決定版だ 
\\	は強調のため
\\	と発音する) 決定票 〔選挙などの〕
\\	〔賛否同数の時の議長の〕
\\	〔その1票で可否が決まるとき〕
\\	決定論 〔哲学〕
\\	非決定論 
\\	決定論者 
\\	欠点	けってん	
\\	〔欠けている所〕
\\	〔弱点〕
\\	人の欠点を捜(さが)す 
\\	欠点だらけである 
\\	欠点のない人はいない 
\\	この設備の欠点は大きさだ 
\\	その説の欠点は文化の違いを無視していることだ 
\\	結論	けつろん	
\\	煙	けむり	
\\	蹴る	ける	
\\	券	けん	〔切符〕
\\	〔切り取り券〕
\\	会員券 
\\	食券 
\\	券を切り離す 
\\	券がなければ入場できません 
\\	県	けん	
\\	見解	けんかい	
\\	それは見解の相違(そうい)である 
\\	彼とは見解を異にする 
\\	彼とは見解を一にする 
\\	官房長官が政府見解を発表した 
\\	限界	げんかい	
\\	現金	げんきん	
\\	言語	げんご	
\\	は一国,一民族の言葉など,一般的に言葉による伝達手段を表し,
\\	は現に用いられている話言葉を表す) 言語 (上) の 
\\	⇒ことば(言葉) ヨーロッパの言語 
\\	言語に絶(ぜっ)する 言語に絶する苦しさだった 
\\	言語学 
\\	言語学的[上の] 
\\	言語学者 
\\	言語矯正(きょうせい) 
\\	言語形態学 
\\	言語社会学 
\\	言語障害 
\\	〔失語症〕
\\	言語障害者 
\\	〔失語症患者〕
\\	言語心理学 
\\	言語中枢(ちゅうすう) 
\\	言語地理(学) 
\\	言語療法 
\\	言語療法士 
\\	健康	けんこう	
\\	検査	けんさ	〔主に公式の〕
\\	〔詳しく調べること〕
\\	〔試すこと〕
\\	〔会計の〕
\\	検査する 
\\	〔分解点検〕
\\	その器具は検査に合格しなかった 
\\	役人が食堂の衛生状態の検査に来た 
\\	検査済み〔表示〕 
\\	所持品の検査を受けた 
\\	井戸の水質検査をする 
\\	知能検査 
\\	糖尿病の検査をする 
\\	検査官 
\\	〔税関の〕
\\	〔会計の〕
\\	検査器 
\\	検査室 
\\	検査証明書 
\\	検査役 
\\	現在	げんざい	
\\	今 (いま)), 
\\	現実	げんじつ	
\\	現実の問題として 
\\	現実に目覚める 
\\	現実の厳しさを正視(せいし)する 
\\	現実は非常に厳しい 
\\	現実に即(そく)して計画を立てる 
\\	その解決法は現実にそぐわない 
\\	現実的な人 
\\	彼の考え方は現実的でない 
\\	彼女は現実的な話し方をする 
\\	現実化 
\\	計画を現実化した 
\\	現実主義 
\\	現実主義的 
\\	現実主義者 
\\	現実性 彼の計画は現実性がない 
\\	その計画は次第に現実性を帯(お)びてきた 
\\	現実逃避(とうひ)主義(者) 
\\	現象	げんしょう	
\\	〔現れた形・姿〕
\\	((複-
\\	〔出来事〕
\\	〔局面〕
\\	自然[社会]現象 
\\	不思議な現象が起きた 
\\	それは思春期(ししゅんき)の一時的な現象だ 
\\	今度の事件は金万能の時世(じせい)が生(う)んだ現象だ 
\\	万能 
\\	〔うわべだけの現れ〕 単なる現象にとらわれず,物の本質を見極(みきわ)める 
\\	物事を現象的に見る 
\\	現象界 
\\	現象論 
\\	現象学 
\\	現状	げんじょう	
\\	建設	けんせつ	
\\	〔建造〕
\\	建設する 
\\	道路を建設する 
\\	新しいビルが今,建設中だ 
\\	〔設立〕
\\	建設する 
\\	福祉国家を建設する 
\\	建設会社 
\\	建設局 
\\	建設業 
\\	建設工事 
\\	建設工事人 
\\	建設者 
\\	〔設立者〕
\\	建設省 
\\	建設大臣 
\\	建設的 建設的な意見を出す 
\\	建設費 
\\	建設用地 
\\	現代	げんだい	
\\	〔当世〕
\\	現代の 
\\	現代的 
\\	〔モダニズムの〕
\\	現代において 
\\	現代の重要な課題 
\\	現代的な絵画 
\\	現代的な若者 
\\	現代的なデザインの椅子(いす) 
\\	〔歴史上の一区分〕
\\	現代史 
\\	現代英語 
\\	現代化 
\\	現代化する 
\\	現代劇 
\\	現代語 
\\	現代思想 
\\	現代人 
\\	現代生活 
\\	現代っ子 
\\	現代版 
\\	それはいわばトロイの木馬の現代版だ 
\\	現代風[式] 
\\	現代文 日本語の現代文 
\\	現代文学 
\\	建築	けんちく	〔建造〕
\\	〔建物〕
\\	〔総称〕
\\	建築する 
\\	家を建築する 
\\	/〔建てさせる〕
\\	木造[コンクリート]建築 
\\	高層建築 
\\	アパートは今建築中だ 
\\	建築請負(うけおい)業者 
\\	請負 
\\	建築請負人 
\\	建築家 
\\	建築学 
\\	建築技師 
\\	建築基準法 
\\	建築業 
\\	建築業者 
\\	建築許可 
\\	建築現場 
\\	建築工事 
\\	建築材料 
\\	建築士 
\\	建築制限 
\\	建築費 
\\	建築物 
\\	建築面積 
\\	建築様式 
\\	ゴシック[ルネサンス]式建築様式の教会 
\\	見当	けんとう	
\\	検討	けんとう	
\\	検討する 〔詳細に調べる〕
\\	〔系統立てて調べる〕
\\	〔熟考する〕
\\	再検討 
\\	更に検討してから報告します 
\\	予算案の検討を任(まか)せられた 
\\	検討中の法案 
\\	その問題を会議に出すかどうか今検討中だ 
\\	現場	げんば	
\\	憲法	けんぽう	
\\	権利	けんり	
\\	〔要求する権利〕
\\	正当な権利 
\\	当然の権利 
\\	平等の権利 
\\	法律[宗教]上の権利 
\\	権利を得る[失う] 
\\	権利を奪う 
\\	/〔不当に〕
\\	権利を放棄(ほうき)する 
\\	権利を要求する 
\\	権利を主張する 
\\	権利を乱用する 
\\	権利を侵害(しんがい)する 
\\	いかなる人も生きる権利を奪われてはならない 
\\	私はその手紙を読む権利がある[ない] 
\\	私たちはだれもその地所(じしょ)に対する所有の権利はない 
\\	権利章典(しょうてん) 
\\	権利宣言〔英史で〕 
\\	権利の請願〔英史で〕 
\\	権利落ち 〔証券で〕
\\	権利金 〔土地・営業などの〕
\\	〔貸室・貸家などの〕
\\	権利金5万円の貸室(かししつ) 
\\	権利者 
\\	権利書 〔不動産の〕
\\	権利付き 〔証券〕
\\	碁	ご	
\\	恋	こい	
\\	濃い	こい	
\\	〔色が強い〕 濃い赤 
\\	濃い化粧をする 
\\	〔濃度・密度が高い〕 濃い霧(きり) 
\\	濃いスープ 
\\	その子はまゆ[髪の毛]が濃い 
\\	〔味が強い〕 お茶が濃い 
\\	このシチューは味付けが濃い 
\\	〔度合いが強い〕 敗色(はいしょく)が濃い 
\\	敗色 
\\	作業員の疲労(ひろう)の色が濃くなってきた 
\\	あの男が彼を殺した疑いが濃い 
\\	〔関係が深い〕 彼女は濃い血縁(けつえん)の者です 
\\	血は水よりも濃い ((諺)) 
\\	恋人	こいびと	
\\	幸運	こううん	
\\	講演	こうえん	
\\	効果	こうか	
\\	効果的/効果のある 
\\	効果的に 
\\	それはかなりの効果があった 
\\	それは効果のありそうな計画だ 
\\	これは劇的な効果を生むだろう 
\\	友人の忠告(ちゅうこく)も彼には効果がない 
\\	彼の努力は効果がなかった 
\\	彼の発言は意図(いと)した効果をもたらさなかった 
\\	もたらす 
\\	即座(そくざ)に薬の効果が現れた 
\\	効果が現れるまでには多少時間がかかる 
\\	彼の行動は選挙区(せんきょく)向けの効果をねらったものだ 
\\	狙う(ねらう) 
\\	紫色(むらさきいろ)の効果的な使い方がこの絵に特殊(とくしゅ)な魅力(みりょく)を与(あた)えている 
\\	〔柔道で〕 効果あり 
\\	効果係 〔芝居などの〕
\\	効果てき面 彼の威嚇的(いかくてき)な言葉は少年たちに効果てき面だった 
\\	この薬はせきに効果てき面だ 
\\	硬貨	こうか	
\\	高価	こうか	
\\	豪華	ごうか	
\\	合格	ごうかく	合格する 〔試験に〕
\\	〔一定の条件・資格に〕
\\	合格おめでとう 
\\	大学入試に合格した 
\\	合格者最低点 
\\	合格者 
\\	合格証 
\\	合格通知 
\\	合格点 
\\	交換	こうかん	
\\	交換する 
\\	文化の交換 
\\	物々(ぶつぶつ)交換 
\\	そのボールペンと交換にノートをあげよう 
\\	/((口)) 
\\	両大学間で教授の交換を行っている 
\\	我々は名刺を交換した 
\\	率直(そっちょく)に意見を交換した 
\\	彼女は彼と席を交換した 
\\	古いタイヤを新しいのと交換した 
\\	彼らは塩を茶と交換した 
\\	手形(てがた)を交換する 
\\	交換学生[教授] 
\\	交換局 
\\	〔交換手〕
\\	交換所 〔手形の〕
\\	交換条件 
\\	交換台 
\\	交換法則 〔数学で〕
\\	交換レンズ 〔写真で〕
\\	航空	こうくう	
\\	光景	こうけい	
\\	衝突事故の現場は惨憺(さんたん)たる光景を呈(てい)していた 
\\	マンションの階上(かいじょう)からはすばらしい光景が見られる 
\\	合計	ごうけい	
\\	攻撃	こうげき	
\\	貢献	こうけん	
\\	貢献する 
\\	彼は医学界に大いに貢献した 
\\	彼は村における福祉(ふくし)の推進(すいしん)にいささか貢献した 
\\	いささ=
\\	における=
\\	広告	こうこく	
\\	交際	こうさい	
\\	〔付き合い〕
\\	((文)) 
\\	〔友人関係〕
\\	交際する 
\\	((口)) 
\\	悪い[良い]人と交際をする 
\\	彼との交際は楽しい 
\\	彼と親しく交際している 
\\	彼女との交際は浅(あさ)い[長い] 
\\	政治家との交際はない 
\\	父は彼との交際をやめるように忠告(ちゅうこく)した 
\\	/〔特に娘に対して〕
\\	彼女は交際好きだ 
\\	彼女は交際嫌いだ 
\\	彼は交際が広い 
\\	〔男女の〕
\\	彼に交際を申し込まれた 
\\	彼女との交際を断った 
\\	交際家 
\\	交際費 
\\	〔接待費〕
\\	〔会社の〕
\\	食事代を交際費につけておいた 
\\	校舎	こうしゃ	
\\	後者	こうしゃ	
\\	後者の場合は 
\\	カナダとアメリカ合衆国(がっしゅうこく)は北アメリカ大陸にあり,前者は後者の北にある 
\\	工場	こうば	
\\	公正	こうせい	
\\	公正な (に) 
\\	公正な取引 
\\	[((口))
\\	公正の観念 
\\	公正な処置 
\\	すべての人を公正に扱う 
\\	公正を期(き)するために記名投票にした 
\\	彼の判断は公正を欠く 
\\	公正価格 
\\	公正価格変動 
\\	公正証書 
\\	公正取引委員会 
\\	構成	こうせい	構 
\\	〔構成の要素〕
\\	〔構成の仕方〕
\\	〔組織〕
\\	社会を構成する 
\\	犯罪を構成する 
\\	我がクラブは30人の学生で構成されている 
\\	この物語の構成はよく出来ている 
\\	文の構成がおかしい 
\\	構成員 
\\	〔一人〕
\\	構成概念(がいねん) 
\\	構成団体 
\\	構成物 
\\	構成分子(ぶんし) 
\\	構成要素(ようそ) 
\\	〔文や句の〕
\\	高速	こうそく	
\\	行動	こうどう	
\\	強盗	ごうとう	
\\	後輩	こうはい	
\\	大学では彼より2年後輩だった 
\\	最上級の学生たちは後輩に威張(いば)りちらした 
\\	幸福	こうふく	
\\	公平	こうへい	
\\	公平に 
\\	公平な定価(ていか) 
\\	公平に分配(ぶんぱい)する 
\\	公平な判断(はんだん)を下(くだ)す 
\\	あの判定(はんてい)は公平でなかった 
\\	公平に見て彼の勝ちだ 
\\	候補	こうほ	〔立候補すること〕((米)) 
\\	((英)) 
\\	〔人〕
\\	⇒こうほしゃ(候補者) 党の (非) 公認候補 
\\	当選[落選]候補 
\\	次期市長候補 
\\	委員長の候補を辞退する 
\\	彼は学長の後任として第一候補である 
\\	候補に立つ ((米)) 
\\	/((英)) 
\\	次の大統領戦ではだれが候補に立つのでしょう 
\\	我々は次期会長の候補に田川氏を推(お)す 
\\	入賞候補の作品 
\\	花嫁(はなよめ)候補 
\\	候補地 児童公園の候補地 
\\	考慮	こうりょ	
\\	〔よく考えること〕
\\	考慮する 
\\	〔熟慮する〕
\\	十分考慮の上 
\\	これらは特に考慮すべき点である 
\\	彼は私の年齢を考慮してはくれなかった 
\\	新しい企画を考慮中である 
\\	あらゆる点を考慮してみるとそれは高い買い物ではない 
\\	〔酌量(しゃくりょう)する〕 
\\	彼が未経験(みけいけん)であることを考慮する必要がある 
\\	交通渋滞の可能性を考慮して1時間早く家を出た〔見越して〕 
\\	越える	こえる	〔乗り越えて向こう側へ行く〕
\\	峠[国境]を越える 
\\	球はライトの頭上を越えた 
\\	⇒こえる(超える)
\\	超える	こえる	
\\	〔上回る〕
\\	彼は2メートルを超える大男だった 
\\	支出が収入(しゅうにゅう)を超えた 
\\	その人は70歳を超えていた 
\\	それは人知(じんち)を超えている 
\\	彼は子供のころから人を超えた才能を示した 
\\	〔超越(ちょうえつ)する〕
\\	人々は主義や利害関係を超えて反核運動に結集(けっしゅう)した 
\\	彼の作品は世代(せだい)を超えて愛読されている 
\\	氷	こおり	
\\	凍る	こおる	
\\	誤解	ごかい	
\\	語学	ごがく	
\\	呼吸	こきゅう	
\\	故郷	こきょう 
\\	ふるさと	〔生まれた町〕
\\	〔生まれた村〕
\\	〔生まれた所〕
\\	第二の故郷 
\\	彼は10歳の時に故郷を出た 
\\	故郷を恋しく思った 
\\	故郷はどちらですか 
\\	故郷は岐阜です 
\\	故郷に錦を飾る 
\\	ごく	ごく	
\\	ごく面白い本 
\\	そんなことはごく当たり前だ 
\\	病気のごく初期に 
\\	国語	こくご	
\\	国籍	こくせき	
\\	((米))
\\	二重国籍者 
\\	無国籍者 
\\	「あなたの国籍はどこですか」「ドイツです」 
\\	彼は国籍はアメリカだがイタリア系である 
\\	国籍不明の飛行機 
\\	日本国籍の船 
\\	国籍を偽(いつわ)る 
\\	その船は国籍を偽(いつわ)った旗を揚(あ)げている 
\\	国籍を失う[回復する] 
\\	彼は日本国籍を離脱(りだつ)した 
\\	国籍を剥奪(はくだつ)される 
\\	彼は日本の国籍を取得(しゅとく)した 
\\	アメリカの国籍を得る[与えられる] 
\\	黒板	こくばん	
\\	克服	こくふく	
\\	克服する 
\\	悪い習慣(しゅうかん)を克服する 
\\	種々(しゅじゅ)の困難(こんなん)を克服した 
\\	国民	こくみん	
\\	穀物	こくもつ	
\\	((英)) 
\\	((英)) 
\\	穀物倉(ぐら) 
\\	((米)) 
\\	穀物商 ((米)) 
\\	((英)) 
\\	穀物畑(ばたけ) 
\\	((英)) 
\\	腰	こし	
\\	〔体の〕
\\	〔胴のくびれた部分〕
\\	〔尻〕
\\	腰の細い女の子 
\\	腰がほっそりとくびれている 
\\	腰までくるセーター 
\\	腰が張っている 
\\	倒れて腰を打った 
\\	年をとるとどうして腰が曲がるのでしょう 
\\	レシーブするときはもっと腰を落としてごらん 
\\	腰の痛み 
\\	腰をかけてお待ちください 
\\	腰を上げる 
\\	彼が入って来ると皆腰を浮かしてあいさつした 
\\	腰をかがめる 
\\	腰を伸ばす 
\\	腰にピストルをさげている 
\\	〔障子などの下の部分〕
\\	〔和歌の〕
\\	腰がある この餅は腰がある 
\\	この生地はなかなか腰がある 
\\	腰が重い 父は腰が重いのでなかなか上京しそうにもない 
\\	腰が折れる 〔途中でつづけられなくなる〕 あの人はいつも途中で腰が折れる 
\\	腰が軽い 
\\	〔気軽に動く〕
\\	〔軽率に行動する〕 腰が軽すぎて失敗しがちだ 
\\	腰が砕ける ⇒こしくだけ(腰砕け) こちらの腰が砕けて交渉は失敗に終わった 
\\	腰が高い あの人は腰が高いから嫌われる 
\\	/((口)) 
\\	腰が立たない 
\\	腰が強い 
\\	〔気が強くて負けない〕 彼は腰の強い男だ 
\\	〔しなやかで折れにくい〕 腰の強いそば 
\\	腰が低(ひく)い だれに対しても腰が低い 
\\	腰が弱い 
\\	〔がんばりがきかない〕 君は腰が弱くてしょうがない 
\\	〔ねばりけが少ない〕 この糊(のり)は腰が弱い 
\\	腰の弱い餅(もち) 
\\	〔折れやすい〕 この筆は腰が弱い 
\\	腰を上げる やっと腰をあげる気になった 
\\	腰を入れる 
\\	〔腰に力を入れる〕 もっと腰を入れて押しなさい 
\\	〔本気になる〕 もっと腰を入れてやってもらいたい 
\\	[((口))
\\	腰を折る 話の腰を折らないでくれ 
\\	腰を据える[落ち着ける] 今度の仕事は腰を据えてやってほしいものだ 
\\	腰を抜かす[が抜ける] 
\\	〔立てなくなる〕
\\	〔びっくりして動けなくなる〕 腰を抜かす[が抜ける]ほど恐ろしかった 
\\	腰を割る 〔相撲で〕
\\	胡椒	こしょう	
\\	胡 
\\	椒 
\\	個人	こじん	個人的な (に) 〔個人の〕
\\	〔個人に関する〕
\\	〔私的な〕
\\	個人の尊厳[自由] 
\\	その仕事に私は個人の資格で参加した 
\\	個人としては私はこの政策は好(す)かない 
\\	私個人の意見では 
\\	個人専用(せんよう)のロッカー 
\\	個人用の電話 
\\	個人的な趣味(しゅみ)の問題だ 
\\	その会社とは個人的な関係は何もない 
\\	それは個人的な問題だ 
\\	あの方は個人的には知りません 
\\	個人企業 
\\	個人教授 フランス語の個人教授を受ける[する] 
\\	個人経営 個人経営の会社 
\\	個人献金 
\\	個人攻撃 
\\	個人差 体力の個人差 
\\	個人主義 
\\	個人主義の 
\\	個人主義者 
\\	個人種目 〔競技の〕
\\	個人消息欄 
\\	〔新聞の〕
\\	個人消費 
\\	個人消費支出 
\\	個人情報 
\\	学生の個人情報の取り扱いには注意してください 
\\	個人情報保護法 
\\	個人所得 
\\	個人崇拝(すうはい) 
\\	個人戦 
\\	個人総合競技 〔体操の〕
\\	個人タクシー 
\\	〔その運転手〕
\\	個人優勝 〔競技の〕
\\	個人輸入 
\\	越す	こす	
\\	超す 
\\	〔横切(よこぎ)る〕
\\	〔通り過ぎる〕
\\	川を越す 
\\	日が中天を越した 
\\	山を一つ越せば行軍(こうぐん)は終わりだ 
\\	〔時を過ごす〕
\\	箱根で夏を越した 
\\	この冬を越せば病人は大丈夫だ 
\\	〔絶頂を過ぎる〕 あらしも峠(とうげ)を越したらしい 
\\	交渉(こうしょう)は峠を越した 
\\	ここまで来たんだから仕事の峠は越した 
\\	[((口))
\\	〔追い抜く〕
\\	先(せん)を越す 
\\	先を越す 
\\	こっちから手を出そうと思ったら先を越された. 
\\	〔引っ越す〕
\\	⇒ひっこす(引っ越す) 一家は大阪に越した 
\\	新しく隣に越してきた人 
\\	〔来る,行く〕 ぜひまたお越しください 
\\	例文 「さっき多摩川を越したよ」「もう東京か。3時間なんてすぐだよな」 
\\	「あの山を越したところが今日の目的地だよ」「まだそんなにあるんですか」 
\\	「俺も60の坂を越してしまったか」「何言ってんだい。人生80年の時代だぜ」 
\\	「応募者の状況はどうだい」「1万人を超すのも時間の問題でしょう」 
\\	「君たち、5年越しの付き合いだって?」「いや、もうかれこれ7年になるんだ」 
\\	「ダイエットも度を超すと危ないって言うわよ」「大丈夫。そこまで行きゃあしないわよ」 
\\	「海の見えるところに越したんだって?」「その代わり、通勤2時間半だよ」 
\\	「何時頃、お越しになりますか」「明日の3時に伺います」 
\\	「あら、どちらへお越しですか」「ちょっと主人の実家まで」 
\\	「彼女の人気も、ひと頃ほどじゃないね」「もう峠を越したね」 
\\	「この仕事も山を越したな」「あと、もうひと頑張りですね」 
\\	「いくらぐらいの部屋をお探しですか」「安いに越したことはないけど、あんまり古くても困るなあ」 
\\	超す	こす	
\\	〔超過(ちょうか)する〕
\\	出費が予算を超している 
\\	彼女は30を超しているに違いない 
\\	気温が30度を超している 
\\	〔勝(まさ)る〕 君が行かれるのなら,それに超したことはない 
\\	慎重(しんちょう)であるに超したことはない 
\\	国家	こっか	
\\	国会	こっかい	
\\	国境	こっきょう	
\\	骨折	こっせつ	
\\	小包	こづつみ	
\\	琴	こと	
\\	異なる	ことなる	
\\	諺	ことわざ	
\\	断る	ことわる	
\\	〔丁重に辞退する〕
\\	〔拒絶する〕
\\	〔はねつける〕
\\	彼は就職を丁重(ていちょう)に断った 
\\	にべもなく断る 
\\	(膠も無い)にべもない 
\\	そんなばかげた提案は断る 
\\	僕は彼女だと何も断れないんだ 
\\	3か所応募(おうぼ)していたが全部断られた 
\\	どうか断らないでください 
\\	セールスマンが来てもみんな断って頂戴(ちょうだい) 
\\	あのコックはもう断れ 
\\	〔予告する〕
\\	〔警告する〕
\\	〔許可を求める〕
\\	辞める場合は1か月前に断ってください 
\\	断っておくがもう一度やったら首だ 
\\	〔わびる〕
\\	〔釈明(しゃくめい)する〕
\\	それで彼は何で来られないと断ってきたんだ 
\\	粉	こな	
\\	好み	このみ	
\\	〔好むこと〕
\\	〔趣味〕
\\	色の好み 
\\	好みの着物 
\\	彼女は好みが難しい 
\\	このハンドバッグは彼女の好みに合うだろう 
\\	すべて彼女の好みどおりに用意された 
\\	〔望み〕
\\	〔選択〕
\\	〔えり好み〕
\\	お好みならば… 
\\	特別好みはありません 
\\	私の好みは鳥肉より牛肉です 
\\	ごのみ
\\	好み】 政治家好みの料亭 
\\	花子好みの男 
\\	好む	このむ	
\\	こぼす	こぼす	
\\	〔水・ご飯つぶなどを〕
\\	〔涙を〕
\\	塩をこぼす 
\\	〔不平を言う〕
\\	いつも何かかんかこぼしている 
\\	こぼれる	こぼれる	
\\	塵ごみ	塵 
\\	芥	〔廃棄物一般〕
\\	〔一般に,または台所の〕
\\	((米)) 
\\	((主に英)) 
\\	〔ちり〕
\\	ごみだらけの路地 
\\	床のごみを掃く 
\\	ごみを捨てる 
\\	ごみを捨てるな〔掲示〕 
\\	燃える[燃えない]ごみ 
\\	ごみ集めの車 
\\	行楽地はごみの山と化した 
\\	ごみ収集 
\\	ごみ収集作業員 
\\	ごみ収集有料化 
\\	ごみ固形燃料 
\\	ごみ捨て場 
\\	((英)) 
\\	ごみ取り 
\\	ごみ箱 
\\	((英)) 
\\	「ごみはどこへ捨てればいいのかしら」「ああ、各階にトラッシュシュートがありますからね。そこに投げ込めばいいんですよ」 
\\	上げぶたをこういうふうに開けて、ごみを放り込めばいいんです。地下の焼却炉まで行っちゃいますから 
\\	大きすぎてシュートに入らないようなごみの場合には、床のここに置いといてください。だれかが集めに来ますから 
\\	小麦	こむぎ	
\\	ごめんなさい	ごめんなさい	
\\	小屋	こや	
\\	これら	これら	
\\	殺す	ころす	
\\	転ぶ	ころぶ	
\\	〔倒れる〕
\\	子供は滑(すべ)って転んだ 
\\	石につまずいて前に転んだ 
\\	子供たちは転ぶように走っていった 
\\	階段を一段踏み外して転んだ 
\\	〔節を曲げて他方へ行く〕 彼は私たちを裏切って敵方に転んだ 
\\	〔物事の成り行きが変わる,ある事態になる〕 この問題の解決はどっちに転んでも同じである 
\\	転ばぬ先の杖(つえ) ((諺)) 
\\	/((諺)) 
\\	転んでもただでは起きない 
\\	今回	こんかい	
\\	今回の事故では負傷者はなかった 
\\	今回左記(さき)に転居(てんきょ)しました 
\\	今回に限り特例(とくれい)を認(みと)めよう 
\\	今回は申し込み者が少なかった 
\\	今後	こんご	
\\	〔これから〕
\\	〔将来〕
\\	((英))では
\\	はつけない) 今後はもっと気をつけます 
\\	今後の成り行きを見守る 
\\	今後の検討(けんとう)が必要だ 
\\	退院(たいいん)したが今後数週間は家でおとなしくしていなければならない 
\\	トムが今後どうなるか分からない 
\\	時制で未来のことと分かるので
\\	はなくてもよい)
\\	混雑	こんざつ	〔雑踏〕
\\	〔押し合いへし合い〕
\\	混雑する 
\\	〔過度に〕
\\	混雑を緩和[整理]する 
\\	混雑した師走(しわす)の町 
\\	夕方の混雑する時間に 
\\	冬になると朝の電車の混雑は大変なものだ 
\\	にぎやかな新宿の街の混雑に彼を見失った 
\\	道路は色々な車で混雑していた 
\\	特売場は大混雑だった 
\\	こんなに	こんなに	
\\	困難	こんなん	
\\	婚約	こんやく	
\\	混乱	こんらん	
\\	差	さ	
\\	際	さい	
\\	⇒とき(時) 出発の際,彼は元気そうだった 
\\	こういう際だからあまり贅沢(ぜいたく)にするべきでない 
\\	あの際仕方がなかった 
\\	必要の際にはどうぞ電話してください 
\\	緊急の際はこのボタンを押してください 
\\	最高	さいこう	
\\	財産	ざいさん	
\\	私有(しゆう)[公有]財産 
\\	国有財産 
\\	人の財産の管理をする 
\\	財産目当てに結婚する 
\\	財産目当(めあ)ての求婚者(きゅうこんしゃ) 
\\	求婚者 
\\	父は株で財産を築(きず)いた[失った] 
\\	田中氏は10億円の財産がある 
\\	誠実(せいじつ)は最大の財産だ 
\\	財産家 
\\	財産権 
\\	財産差し押さえ 
\\	〔法律用語〕
\\	財産取得税 
\\	財産税 
\\	((英)) 
\\	財産相続 
\\	財産分与(ぶんよ) 
\\	財産目録(もくろく) 
\\	最終	さいしゅう	
\\	最中	さいちゅう	
\\	最低	さいてい	
\\	才能	さいのう	
\\	裁判	さいばん	
\\	材料	ざいりょう	
\\	幸い	さいわい	
\\	〔幸せ〕 折り返しお返事をいただければ幸いです 
\\	お役に立てば幸いです 
\\	〔都合のよいこと〕 幸い (に) けが人はなかった 
\\	私が訪ねたとき幸い彼は在宅(ざいたく)だった 
\\	幸い (に) 彼に会うことができた 
\\	彼が動揺(どうよう)したのを幸いにスマッシュを決めた 
\\	天候が幸いして美しい富士山を眺めることができた 
\\	世の中は何が幸いするか分からない 
\\	境	さかい	
\\	〔国・州などの境界線〕
\\	〔境界線〕
\\	⇒きょうかい(境界) 州の境 
\\	市の境 
\\	境を決める ((文)) 
\\	彼らは土地(とち)の境を定めた 
\\	フィンランドは西はスウェーデンと境を接(せっ)している 
\\	山の尾根(おね)が県の境になっている 
\\	この川が奥沢村と緑が丘村の境を成(な)している 
\\	〔分かれ目〕 哀れみと同情との境 
\\	生死の境をさまよう 
\\	その事件を境に彼の生活はがらりと変わった 
\\	境目 
\\	及落(きゅうらく)の境目にいる 
\\	逆らう	さからう	
\\	盛り	さかり	
\\	(全盛期)
\\	(雄の発情期)
\\	(雌の発情期)
\\	〔一番勢(いきお)いのよい時期〕 暑い夏の盛りに 
\\	今はつつじの花が盛りだ 
\\	もうぶどうは盛りを過ぎた 
\\	私が到着(とうちゃく)したときには宴(えん)も盛りであった 
\\	〔一生で一番充実した時期〕 人生の盛りを過ぎた人々 
\\	盛りをすぎた野球選手 
\\	[((口)) 
\\	〔発情,雌の〕
\\	〔雄の〕
\\	盛りがつく 
\\	猫に盛りがついている ((米)) 
\\	/((英)) 
\\	作業	さぎょう	
\\	〔仕事〕
\\	〔操業〕
\\	作業する 
\\	一斉(いっせい)に作業を開始した 
\\	工場は作業を停止中である 
\\	作業中出入り禁止 
\\	作業員 
\\	作業時間 
\\	作業場 
\\	作業条件 〔環境〕
\\	〔作業に必要な条件〕
\\	作業効率(こうりつ) 
\\	そのやりかたでは作業効率が悪い 
\\	作業長[監督(かんとく)] 
\\	作業服 
\\	作業療法(りょうほう) 
\\	裂く	さく	
\\	昨	さく	
\\	作品	さくひん	
\\	作物	さくもつ	
\\	桜	さくら	
\\	酒	さけ	
\\	叫ぶ	さけぶ	
\\	避ける	さける	
\\	〔望ましくない物事をよける〕
\\	私たちは危険を避けようとした 
\\	日光(にっこう)を避けて日陰(ひかげ)に座った 
\\	私は (身をかわして) 倒れてくる木を避けた 
\\	これは避けて通れない問題だ 
\\	両者の意見の衝突(しょうとつ)は避けられない 
\\	〔責任などを逃れる〕
\\	社長は明言(めいげん)を避けた 
\\	首相は質問に答えるのを避けた 
\\	〔人に会わないようにする〕
\\	悪友(あくゆう)は避けたほうがよい 
\\	彼は父親を避けた 
\\	人目を避けて裏通りを歩いた 
\\	〔遠慮する〕
\\	人前(ひとまえ)での批評は避けたほうがよい 
\\	支える	ささえる	
\\	〔つっかいをする〕
\\	棒で塀(へい)を支える 
\\	杖(つえ)で体を支える 
\\	〔防(ふせ)ぎ止める〕 
\\	辛(かろ)うじて敵を支えている 
\\	辛うじて  
\\	〔維持(いじ)する〕
\\	彼は大家族を支えて行かねばならない 
\\	*勢(いきお)いがすさまじくて防(ふせ)ぎ止めることができないもの 
\\	すさまじい 
\\	刺さる	ささる	
\\	指にとげが刺さった 
\\	刺す	さす	
\\	〔突き刺す〕
\\	〔針などで〕
\\	男ののどを刺した 
\\	指にとげを刺した 
\\	風が刺すように冷たい 
\\	舌を刺すような味 
\\	鼻を刺すような臭い 
\\	胸を刺すような言葉 
\\	〔虫が〕
\\	蚊[はち]に刺された 
\\	〔縫う〕
\\	〔野球で,走者を〕
\\	例文 彼はナイフで人を刺した 
\\	彼女はバラのとげで指をちくりと刺した 
\\	彼女は肉の焼け具合を見るために串で刺してみた 
\\	ハチに腕を刺された 
\\	蚊に刺されたところがかゆい 
\\	蚊に刺されたところがかゆい 
\\	外は肌を刺すような寒さだ 
\\	刺すような痛みが胃に走った 
\\	彼の刺すような視線にぞっとした 
\\	「そろそろバーベキューの準備始めようよ」「そうね。じゃあ、この串(くし)に材料を刺していってちょうだい」 
\\	「彼、腕に包帯してたけど、どうしたの」「飲み屋でヤクザとけんかして刺されたんだってさ」 
\\	「キャンプはどうだった?」「あちこち虫に刺されて、かゆくてたまらないわ」 
\\	指す	さす	(指示する)
\\	(意味する)
\\	(言及する)
\\	(指名する)
\\	(密告する)
\\	〔指し示す〕
\\	その子は自分の家の方を指した 
\\	時計の針は10時5分を指していた 
\\	〔名を示す〕 だれのことを指して言っているのですか 
\\	先生に指された 
\\	〔目指す〕⇒めざす(目指す) 
\\	〔将棋をする〕 一局指しませんか 
\\	駒(こま)を指す 
\\	もう指す手がない 
\\	〔密告する〕
\\	((口)) 
\\	仲間を指す 
\\	挿す	さす	
\\	〔さしはさむ〕
\\	髪にかんざしを挿す 
\\	〔生ける〕
\\	花瓶(かびん)にばらの花を挿す 
\\	〔挿し木をする〕
\\	椿(つばき)は梅雨時(つゆどき)に挿すとよく根付(ねづ)く 
\\	【差す・▲注す】	さす	さす【差す・▲射す】 
\\	〔光などが〕
\\	屋根に朝日が差している 
\\	この部屋はよく日が差す 
\\	障子にだれかの影(かげ)が差した 
\\	〔表面に出てくる〕 ほおに赤みが差した 
\\	〔気持ちがおこる〕 嫌気(いやき)が差す 〔いやになる〕
\\	/〔飽きる〕
\\	あの時は魔(ま)が差したのだろう 
\\	〔そそぐ〕
\\	コップに水を差す 
\\	父は毎朝鉢植(はちう)えに水を差している 
\\	この薬を3滴,両眼に差しなさい 
\\	〔加える〕
\\	沸騰(ふっとう)しているやかんに水を差す 
\\	〔つける〕
\\	口紅を差す 
\\	芝刈(しばか)り機に油を注す 
\\	〔かざす〕
\\	傘を差す 
\\	〔さしはさむ〕 
\\	さむらいは腰に刀を差していた 
\\	〔舟を動かす〕 棹(さお)をさす 
\\	〔酒をすすめる〕 
\\	彼に杯を差した 
\\	座席	ざせき	
\\	誘う	さそう	
\\	〔勧誘(かんゆう)する〕
\\	友人数人をお茶に誘った 
\\	〔連れ出す〕 
\\	途中私を誘って車で海岸まで連れて行ってくれた 
\\	〔促(うなが)す〕
\\	今日の暖かさには眠りを誘われる 
\\	その光景(こうけい)は観衆(かんしゅう)の涙を誘った 
\\	酒は眠気(ねむけ)を誘う 
\\	〔誘惑(ゆうわく)する〕
\\	彼は悪い仲間に誘われるままに悪の道に入った 
\\	ままに 
\\	札	さつ	
\\	千円札 
\\	千円札で2万円 
\\	大きな札を細かくする 
\\	作家	さっか	
\\	作曲	さっきょく	
\\	ざっと	ざっと	
\\	〔大略(たいりゃく)〕
\\	ざっと見積(みつも)ってもらった 
\\	この図書館にはざっと50万冊の蔵書がある 
\\	ざっとそんな男だ 
\\	〔手短に〕
\\	〔軽く〕
\\	ざっと説明する 
\\	嘆願書(たんがんしょ)の主旨(しゅし)をざっと述べなさい 
\\	私はざっと新聞に目を通した 
\\	彼はその問題にはざっと触れただけだ 
\\	さっぱり	さっぱり	
\\	〔こぎれいな様子〕 さっぱりした 
\\	質素(しっそ)だがさっぱりとした服装をしている 
\\	〔爽快(そうかい)な様子〕 さっぱりした気分になる 
\\	/〔重荷が降りて〕
\\	お風呂に入るとさっぱりしますよ 
\\	宿題を済ませたらさっぱりした 
\\	〔あっさりした様子〕 
\\	さっぱりした気性の人 
\\	彼はさっぱりとした調子(ちょうし)で答えた 
\\	〔味が淡白(たんぱく)な様子〕 さっぱりした味の食物 
\\	〔すっかり〕
\\	さっぱりとあきらめる 
\\	彼とはきれいさっぱりと縁(えん)を切った 
\\	〔一向に,まるで〕 
\\	さっぱり思い出せない 
\\	彼が何を言っているのかさっぱり分からなかった 
\\	彼は近ごろさっぱり姿を見せない 
\\	〔全く駄目な様子〕 
\\	ここのところ英語の成績(せいせき)はさっぱりだ 
\\	不景気で売り上げはさっぱりだ 
\\	さて	さて	
\\	〔ところで〕
\\	話題を変えるのではなく前に述べたことを受けて) 
\\	〔さあ〕 さて,どうしよう 
\\	さて,これは困った 
\\	「彼女は約束したの」「さてね」 
\\	〔ところが〕 さてやってみるとなかなかうまくいかなくって 
\\	砂漠	さばく	
\\	差別	さべつ	
\\	作法	さほう	
\\	家庭の礼儀作法 
\\	食卓での作法を守る 
\\	よい[悪い]作法 
\\	作法にかなう[反する] 
\\	[((文))
\\	彼は全く作法を心得ていない 
\\	彼の作法はよい 
\\	様々	さまざま	
\\	冷ます	さます	
\\	〔冷たくする〕
\\	吸い物を冷ます 
\\	息を吹いて茶を冷ます 
\\	熱を冷ます 
\\	〔感情・興味などをなくす〕 興(きょう)をさます 
\\	彼は人の興を冷ますだけだ 
\\	覚ます	さます	
\\	目を覚ます 
\\	冷める	さめる	
\\	〔冷たくなる〕
\\	お茶がすこし冷めるまで待ちなさい 
\\	スープが冷めてしまった 
\\	冷めないうちに召し上がれ 
\\	〔感情や興味がうすらぐ〕
\\	彼の怒りがすこし冷めるまで待ったほうがいい 
\\	彼の競馬熱もちょっと冷めてきたようだ 
\\	彼の言葉で興が冷めてしまった 
\\	それを見たら彼女に対する愛情が一時に冷めてしまった 
\\	覚める	さめる	
\\	〔眠りから〕 目が覚める 
\\	麻酔(ますい)から覚める 
\\	娘は彼の美(うつく)しい顔を見て目が覚める思いであった 
\\	目の覚めるような色 
\\	〔迷いから〕
\\	彼も間もなく目が覚めるだろう 〔幻想(げんそう)から覚める〕
\\	/〔正気を取り戻す〕
\\	目を覚ませ
\\	お前はだまされているのだよ ((口)) 
\\	〔酔いから〕 酔いがさめる 
\\	左右	さゆう	
\\	〔左と右〕
\\	左右をよく見て道を渡(わた)りなさい 
\\	橋のところで我々は左右に別れた 
\\	〔両側〕 左右にお供(とも)を従(したが)えて 
\\	通(とお)りの左右に 
\\	〔あいまいなこと〕 言(げん/こと)を左右にする 
\\	〔支配する〕
\\	一国の運命を左右する 
\\	感情に左右される 
\\	〔影響する〕
\\	友達に左右されやすい人だ 
\\	左右相称(そうしょう) 
\\	左右相称の 
\\	皿	さら	
\\	さらに	更に	
\\	〔その上〕
\\	さらに友人に感謝していると言った 
\\	さらに悪いことには彼は年齢をごまかしていた 
\\	誤魔化す 
\\	〔なおいっそう〕
\\	さらに多くの人を雇わねばならない 
\\	さらに検討(けんとう)する必要がある 
\\	〔少しも〕
\\	⇒さらさら(更更) それを信じようという気はさらになかった 
\\	反省(はんせい)の色がさらにない 
\\	去る	さる	
\\	〔ある場所から離れる〕
\\	再び戻らないつもりで彼はパリを去った 
\\	[((文))
\\	去る者は追(お)わずというのが私の主義だ 
\\	去る者は日々に疎(うと)し ((諺)) 
\\	〔時が過ぎる〕
\\	夏が去って秋が来た 
\\	5年の幸福な年月が瞬(またた)く間に去った 
\\	〔死ぬ〕 2年前に父はこの世を去った 
\\	〔なくなる〕
\\	痛みが去った 
\\	(病気の) 危機は去った 
\\	〔隔たっている〕 大阪を去ること30キロの地点 
\\	/((文))
\\	〔過ぎ去った,へだたる〕 彼は去る1月結婚した 
\\	今を去ること15年昔 
\\	〔取り除く〕 雑念を去る 
\\	〔職などを〕
\\	教壇(きょうだん)を去った 
\\	は定年で去ること) 彼女はそのショーを最後に舞台を去った 
\\	〔縁を切る〕 妻を去る 
\\	猿	さる	
\\	騒ぎ	さわぎ	
\\	参加	さんか	
\\	参考	さんこう	
\\	本を参考にする 
\\	森氏の注釈を参考にした 
\\	参考になる 
\\	この本は大いに参考になった 
\\	参考のため 
\\	この書類は後の参考のために取っておいた 
\\	参考のために教えてほしいのですが,このコンピュータはいくらでしたか 
\\	ご参考までにここに過去5年間の統計があります 
\\	参考書 
\\	参考書をいくつか調べたが出ていなかった 
\\	参考書目(しょもく) 〔文献〕
\\	参考資料 
\\	参考人(にん) 
\\	重要参考人として任意出頭(にんいしゅっとう)する 
\\	任意出頭 
\\	参考人招致(しょうち) 参考人招致を行う 
\\	賛成	さんせい	
\\	〔動議の〕
\\	賛成する 
\\	〔動議に〕
\\	その計画に賛成である 
\\	彼に賛成だ 
\\	社長の賛成が必要だ 
\\	議案に賛成する 
\\	/〔投票で〕
\\	動議に賛成の意思表示をする 
\\	投票は賛成10票,反対5票だった 
\\	提案に賛成のかたは挙手(きょしゅ)願います 
\\	賛成 
\\	不賛成 
\\	賛成多数により可決いたします〔議長の言葉〕 
\\	議案は過半数の賛成により可決された 
\\	賛成演説 
\\	賛成演説をする 
\\	賛成者 
\\	賛成投票 
\\	酸性	さんせい	
\\	酸素	さんそ	
\\	氏	し 
\\	うじ	し 
\\	〔氏族〕
\\	藤原氏 
\\	〔男性に付けて〕
\\	((複
\\	〔女性に付けて〕
\\	((複
\\	は
\\	や
\\	の区別を嫌う女性に付ける.((英))ではすべてピリオドを省く傾向がある) 木田,谷の両氏 
\\	某氏(ぼうし) 
\\	〔特定の人を指して〕 氏は京都の出身です 
\\	ーーーーーー うじ 
\\	〔名字〕
\\	〔家柄〕
\\	〔血統〕
\\	彼は氏も育ちもよい 
\\	彼は氏も育ちも分からない 〔身元不明〕
\\	/〔経歴不明〕
\\	氏より育ち 
\\	詩	し	
\\	幸せ	しあわせ	
\\	四角	しかく	
\\	じかに	じかに	
\\	〔人から人へ〕
\\	これは彼からじかに聞いた話だ 
\\	手紙をじかに渡す 
\\	本人がじかに (出向いて) 申し込む 
\\	〔間に隔てるものが無いこと〕 手でじかにつかむ 
\\	肌(はだ)にじかに着る 
\\	床の上にじかに置かないでください 
\\	しかも	しかも	
\\	〔その上〕
\\	彼女は美しくて賢(かしこ)く,しかもとても親切だ 
\\	彼女は病気にもかかわらず安産(あんざん)した
\\	しかも元気な赤ん坊だった 
\\	彼は未熟練(みじゅくれん)でしかも怠け者ときている 
\\	は通常,文の終わりに置き,悪いことに使う) 
\\	〔それでもなお〕 彼はせきと高熱(こうねつ)で苦(くる)しんでいながら,しかも薬を飲まなかった 
\\	彼は父親にひどくしかられて,しかも全く後悔(こうかい)していない 
\\	四季	しき	
\\	じき	じき	
\\	〔間もなく〕
\\	じき帰ってくるでしょう 
\\	雨もじき上がるでしょう 
\\	クリスマスももうじきだ 
\\	もうじき60歳になります 
\\	じきに済みます 
\\	〔すぐ近くに〕⇒すぐ4
\\	時期	じき	
\\	支給	しきゅう	
\\	〔一定の量を〕
\\	貧(まず)しい子供たちに食物(しょくもつ)と衣服を支給した 
\\	ヘリコプターで島民(とうみん)に食料品(しょくりょうひん)が支給された 
\\	会社は通勤費をどれだけ支給しますか 
\\	支給額 
\\	支給品 
\\	至急	しきゅう	
\\	至急の場合に 
\\	至急の商用で 
\\	至急便で送る 
\\	至急おいで願いたい
\\	お出で (おいで) 
\\	至急お返事をお願いします 
\\	しきりに	しきりに	
\\	〔たびたび〕
\\	〔繰り返し〕
\\	彼はしきりに亡くなった母親の話をした 
\\	彼はしきりに会いに来た 
\\	午前中しきりに電話が鳴った 
\\	〔絶えず〕 しきりに雨が降っている 
\\	〔ひどく〕
\\	頭がしきりに痛んだ 
\\	〔熱心に〕
\\	〔強く〕
\\	彼はしきりにそこへ行きたがっている 
\\	彼はしきりに反対した 
\\	刺激	しげき	
\\	〔感覚器官に対する外部からの働き〕
\\	((複-
\\	〔ちくちく,ぴりぴりした感じ〕
\\	刺激する 
\\	強烈(きょうれつ)な日光(にっこう)は目を刺激する 
\\	この果物は舌(した)を刺激するような味だ 
\\	刺激のある食べ物〔香辛料(こうしんりょう)の効(き)いた〕 
\\	〔食欲をそそること〕 ぶどう酒が食欲を刺激した 
\\	〔神経にさわること〕 彼女のきんきん声は彼の神経を刺激した 
\\	〔励みとなるもの〕
\\	〔誘因〕
\\	〔鼓舞するもの〕
\\	刺激する 
\\	刺激的 
\\	彼の成功が勉強の刺激となった 
\\	兄の成功に刺激されて彼もいっそう努力(どりょく)した 
\\	〔興奮・わくわくさせること〕 刺激的 
\\	病人をあまり刺激しないで欲しい 
\\	オートバイに刺激を求めた 
\\	刺激のない生活をしている 
\\	刺激的な映画 
\\	〔挑発(ちょうはつ)〕
\\	刺激的な態度で 
\\	彼はおこりっぽいからあまり刺激しないでほしい 
\\	首相の発言は相手国の感情を刺激した 
\\	彼女の嘲笑(ちょうしょう)に刺激されてそういう行動に出た 
\\	刺激剤(ざい) 
\\	刺激臭(しゅう) 
\\	刺激性 刺激性の 
\\	刺激物 〔コーヒー・紅茶など〕
\\	資源	しげん	
\\	地下資源 
\\	天然[物的]資源に富(と)む 
\\	資源を開発[保護]する 
\\	戦争は国の資源をからしてしまうだろう 
\\	工場における人的資源の不足 
\\	資源エネルギー庁 
\\	資源供給国 
\\	資源ごみ 
\\	資源産出国 
\\	資源探査衛星 
\\	資源ナショナリズム 
\\	資源リサイクルセンター 
\\	事件	じけん	
\\	時刻	じこく	
\\	⇒じかん(時間) 約束の時刻 
\\	出発[到着]時刻 
\\	正確(せいかく)な時刻 
\\	この時計は1時間毎(ごと)に時刻を知らせる 
\\	彼は時刻をたずねた 
\\	予定の時刻に開演(かいえん)した 
\\	時刻が遅い 
\\	もう帰る時刻だ 
\\	時刻表 
\\	列車[船]の時刻表 
\\	日本では列車は時刻表どおりに走る 
\\	自殺	じさつ	
\\	事実	じじつ	
\\	支出	ししゅつ	〔支払うこと〕((文)) 
\\	((文)) 
\\	〔支払う金額〕
\\	((米))では複数形がよく使われる) 
\\	〔経費〕
\\	支出する 
\\	((文)) 
\\	予算外支出 
\\	国庫(こっこ)からの不正支出 
\\	支出が多い[少ない] 
\\	支出を抑(おさ)える 
\\	軍備(ぐんび)に多額(たがく)の支出をする 
\\	支出が収入を上回(うわまわ)った 
\\	支出係 
\\	支出額 
\\	50万円の支出額 
\\	事情	じじょう	
\\	⇒じょうせい(情勢); 〔理由〕
\\	事情によっては 
\\	[((文)) 
\\	このような事情で 
\\	やむを得ぬ事情で 
\\	家庭の事情で 
\\	中東の事情に明るい 
\\	住宅事情が全く変(か)わった 
\\	ここでは事情が全く違う 
\\	そういう事情ならあきらめます 
\\	どういう事情になっているのか訳が分からない 
\\	事情聴取(ちょうしゅ) 
\\	事情聴取を受ける 
\\	事情通 
\\	彼は証券市場の事情通である 
\\	詩人	しじん	
\\	自身	じしん	それ自身のために 
\\	自身で 〔一人で〕
\\	/〔自ら〕
\\	文部大臣自身が来られた 
\\	私自身はこの絵のほうが好きです 
\\	君自身の意見はどうなんだ 
\\	私自身はインドへ行きたい 
\\	沈む	しずむ	
\\	(太陽が)
\\	(気分が)
\\	〔水中に隠れる〕
\\	ボートは見る見るうちに沈んだ 
\\	船は台風のために沈んだ 
\\	海に沈む 
\\	〔地平線などに隠れる〕
\\	太陽が西に沈んだ 
\\	沈みかけた太陽 
\\	〔低くなる〕 地盤が5センチ沈んだ 
\\	〔陰うつになる〕 彼女は沈んだ顔をしている 
\\	沈んだ声で 
\\	悲しみに沈んでいる 
\\	〔落ちぶれる〕 不運に沈んだ一生だった 
\\	自然	しぜん	
\\	〔天然の姿〕
\\	冠詞をつけない) 自然 (のまま) の 
\\	自然を友とする 
\\	自然に親しむ 
\\	自然の懐(ふところ)に抱(だ)かれて休息(きゅうそく)した ((文)) 
\\	渓谷(けいこく)のすばらしい自然の景観(けいかん)を楽しんだ 
\\	自然の威力(いりょく)に打たれた 
\\	小さな虫の世界にも自然の営(いとな)みがある 
\\	子孫(しそん)のために美しい自然を保護してゆかねばならない 
\\	このあたりにはまだ自然が残っている 
\\	〔ありのまま,無理がない様子〕 君の話し方は自然だ 
\\	彼の立ち居(い)振る舞いは自然でない 
\\	気取らず自然に話す 
\\	自然に振る舞う 
\\	〔当然〕 彼女が母を慕(した)うのは自然だ 
\\	無口だから自然と友人も少ない 
\\	〔ひとりでそうなる様子〕 自然の 〔自然発生的,自発的〕
\\	〔自動的,反射的〕
\\	自然に 
\\	自然にわく思想 
\\	我々の呼吸は自然になされるものだ 
\\	傷は自然になおった 
\\	自然の成り行きに任せる 
\\	自然の結果として 
\\	あの人の前に出ると自然と礼儀正しくなる 
\\	それを聞いて自然とほほが緩(ゆる)んだ 
\\	あとで自然と分かることだ 
\\	最年長者なので自然とまとめ役になった 
\\	自然界 
\\	自然改造 
\\	自然科学 
\\	自然観 
\\	自然換気 
\\	自然環境保全地域 
\\	自然環境保全法 
\\	自然休会 
\\	自然現象 
\\	((複-
\\	自然公園 
\\	自然災害 
\\	自然死 
\\	私の祖父は自然死で亡くなった 
\\	自然死する 
\\	自然主義 〔文学,哲学で〕
\\	自然主義的 
\\	自然主義者 
\\	自然食品 
\\	自然人 
\\	自然数 〔数学で,正の整数〕
\\	自然崇拝 
\\	自然崇拝者 
\\	自然生態系 
\\	自然葬 
\\	自然増加 〔法律で,財産の〕
\\	人口の自然増加率 
\\	自然増収 〔経済で〕
\\	自然体 
\\	自然体で立つ[構える] 
\\	自然対数 〔数学で〕
\\	自然淘汰(とうた) 
\\	弱いものは自然淘汰されてしまう 
\\	自然治癒 
\\	自然治癒する 
\\	自然地理学 
\\	自然農法 
\\	自然発火 
\\	自然発生 
\\	〔生物で〕
\\	山火事の自然発生 
\\	この会は自然発生的にできたものだ 
\\	自然美 
\\	自然描写 
\\	自然法 
\\	自然法則 
\\	自然保護 
\\	自然保護運動 
\\	自然保護団体 
\\	自然遊歩道 
\\	自然療法 
\\	自然力 
\\	思想	しそう	
\\	舌	した	
\\	次第	しだい	
\\	〔順序〕
\\	〔事情〕
\\	次第によっては 
\\	〔成り行きにまかせること〕 
\\	言いなり次第になる 
\\	〔それによって決まる様子〕 努力(どりょく)次第で成功します 
\\	〔…するや否や〕 品物が着き次第送金します 
\\	⇒だんだん(段段) 私は彼のよさが次第に分かってきた 
\\	〔順序〕
\\	式(しき)の次第 
\\	〔事情〕
\\	次第によっては 
\\	こういう次第です 
\\	そのような次第ですから行かれません 
\\	[((文))
\\	事と次第によっては私が直接(ちょくせつ)出向(でむ)かなければなるまい 
\\	お恥(はずか)しい次第です 
\\	〔成り行きにまかせること〕 言いなり次第になる 
\\	手当り次第に物を投げる 
\\	〔それによって決まる様子〕 努力(どりょく)次第で成功します 
\\	お天気次第でどこへ行くか決めよう 
\\	決めるのは君次第だ 
\\	地獄の沙汰(さた)も金次第 ((諺)) 
\\	〔…するや否や〕 計画が立ち次第実行することにしよう 
\\	品物が着き次第送金します 
\\	ご都合がおつきになり次第いらして下さい 
\\	都合がつき次第お訪ねします 
\\	⇒だんだん(段段) 彼の姿は次第に闇(やみ)の中に消えていった 
\\	私は彼のよさが次第に分かってきた 
\\	従う	したがう	
\\	したがって	したがって	
\\	彼は大層親切だった,従って皆に愛されていた 
\\	親しい	したしい	
\\	〔親密な〕
\\	〔仲のよい〕
\\	親しい友人 
\\	親しい間柄である 
\\	親しき仲にも礼儀あり ((諺)) 
\\	〔なじみが深い〕 それは我々の耳に親しい言葉だ 
\\	質	しつ	
\\	〔資質〕
\\	〔気質〕
\\	天性の質 
\\	教師の質の低下が最近著しい 
\\	あの人は蒲柳(ほりゅう)の質だ 
\\	蒲柳 
\\	あのピアニストは質がいい 
\\	とびきり質のよい学生 
\\	〔成分となる物質〕
\\	動物[植物]質 
\\	〔品質〕
\\	この紙は質がよい[悪い] 
\\	質をよくする[落とす] 
\\	質的にも量的にも 
\\	量より質 
\\	失業	しつぎょう	
\\	湿気	しっけ	
\\	湿気を含んだ 
\\	/〔空気が〕
\\	湿気を取る 
\\	湿気を吸う 
\\	湿気をおびる 
\\	湿気で金具がさびている 
\\	この電気器具は湿気を嫌(きら)う 
\\	実験	じっけん	
\\	実現	じつげん	
\\	実現する 
\\	〔夢・希望が〕
\\	実現 (不) 可能な 
\\	彼の夢は実現した 
\\	彼の偉大な計画[夢]はついに実現した 
\\	実行	じっこう	
\\	〔実際に行うこと〕
\\	〔遂行〕
\\	実行する 
\\	実行可能の 
\\	実行不可能の 
\\	我々の考えを実行に移す 
\\	約束を実行する 
\\	不言実行(ふげんじっこう) 
\\	実行委員 
\\	実行予算 
\\	実行力 実行力のある男 
\\	実際	じっさい	
\\	実施	じっし	施 
\\	実施する 
\\	〔法律などを〕
\\	条約を実施する 
\\	実施されている 
\\	その法律(ほうりつ)は来年1月から実施される 
\\	計画通り実施します 
\\	実施要項 
\\	湿度	しつど	
\\	じっと	じっと	
\\	〔静かにしている様子〕 
\\	たいていの子供は30分もじっとしていられない 
\\	芝居が終わるまでじっと見ていた 
\\	それを考えるとじっとしていられない 
\\	みんなが働いているのにどうしてじっとしていられるか 
\\	〔心を集中する様子〕 
\\	じっと見つめる 
\\	彼はその絵をじっと見つめた 
\\	彼は彼女が見えなくなるまでじっと見送った 
\\	子供たちは先生の話をじっと聞いていた 
\\	じっと考え込んでいた 
\\	〔耐(た)え忍(しの)ぶ様子〕
\\	泣き出したいのをじっと我慢(がまん)した 
\\	痛(いた)さをじっとこらえた 
\\	夫の帰りをじっと待った 
\\	実に	じつに	
\\	実は	じつは	
\\	失望	しつぼう	
\\	実力	じつりょく	
\\	〔力量・腕前〕
\\	実力のある 
\\	実力で 
\\	実力を養(やしな)う[発揮する] 
\\	英語の実力がある 
\\	〔武力,腕力〕
\\	実力に訴(うった)える 
\\	実力で門を開けた 
\\	実力行使 
\\	実力行使に出る 
\\	/〔ストライキをする〕
\\	実力者 
\\	政界の実力者 
\\	財界の実力者 
\\	本当の実力者は彼である 
\\	実力主義 
\\	この会社は実力主義だ 
\\	実力テスト 
\\	支店	してん	
\\	指導	しどう	
\\	自動	じどう	
\\	児童	じどう	
\\	品	しな	
\\	支配	しはい	
\\	〔統治〕
\\	〔統制〕
\\	支配する 
\\	チャールズ王支配下(しはいか)の英国 
\\	これらの地方は皆ローマ帝国の支配の下にあった 
\\	〔指図〕
\\	〔管理〕
\\	支配する 
\\	この仕事は彼の支配で完成(かんせい)した 
\\	実際この会社を支配しているのは社長の息子なんだ 
\\	[((口))
\\	〔優勢(ゆうせい)〕
\\	白人支配 
\\	彼は私たちのグループで支配的な立場(たちば)にいる 
\\	先入観(せんにゅうかん)に支配されて物を見てはいけない 
\\	この国では保護主義の考えが支配的である 
\\	〔左右すること〕
\\	〔影響を与えること〕
\\	憎(にく)しみに支配される 
\\	彼はとかく感情に支配されやすい 
\\	とかく 
\\	この国の農業は以前は全く天候に支配されていた 
\\	生物(せいぶつ)はすべて自然の法則に支配されている 
\\	支配階級 
\\	支配権 
\\	支配者 
\\	支配人 
\\	総支配人 
\\	支配力 
\\	芝居	しばい	
\\	しばしば	しばしば	
\\	⇒たびたび(度度) この地方ではそのようなことがしばしば起こる 
\\	来ないと思っているときにしばしば現れるんだ 
\\	芝生	しばふ	
\\	支払	しはらい	
\\	支払う	しはらう	
\\	死亡	しぼう	
\\	資本	しほん	
\\	〔元手,資金〕
\\	準備[流動/流通]資本 
\\	固定[外国]資本 
\\	資本の蓄積(ちくせき) 
\\	資本の自由化 
\\	資本の不足に苦しむ 
\\	資本を投(とう)ずる 
\\	資本を運転する[回転させる] 
\\	資本をねかせておく 
\\	彼は2000万円の資本で本屋を始めた 
\\	父が私の企画の資本を出してくれた 
\\	〔かんじんなもの〕 健康[体]は資本です 
\\	資本家 
\\	資本家と労働者 
\\	資本家陣営[階級] 
\\	資本金 
\\	あれは資本金1億円の会社だ 
\\	資本構成 
\\	資本財 
\\	資本支出 
\\	資本主義 
\\	修正[独占]資本主義 
\\	資本主義経済 
\\	資本主義国家 
\\	資本取り引き 
\\	姉妹	しまい	
\\	自慢	じまん	〔誇り〕
\\	〔大言〕
\\	自慢する 〔誇りに思う〕
\\	〔ほらをまじえて大げさに〕
\\	のど自慢 
\\	お国(くに)自慢をする 
\\	お国自慢に花を咲(さ)かせた 
\\	花を咲かせる 
\\	自慢顔に 
\\	これは彼の自慢の絵です 
\\	あの娘は両親の自慢の種だ 
\\	彼女はよく息子の自慢をする 
\\	おじいさんはいつも若いころのことを自慢して話す 
\\	自慢じゃないが私の講演は成功だった 
\\	自慢話(じまんばなし) 
\\	自慢話をする 
\\	((口)) 
\\	自慢屋 
\\	地味	じみ	
\\	〔質素(しっそ)なこと〕 地味な暮らし 
\\	彼の生活はきわめて地味であった 
\\	〔くすんで落ち着いていること〕 地味な色 
\\	地味な服装をする 
\\	〔おとなしくて目立(めだ)たないこと〕 彼の演技はあまりに地味だ 
\\	彼は地味な性格で社交は苦手だ 
\\	苦手 
\\	示す	しめす	
\\	〔見せる〕
\\	身分証明書(みぶんしょうめいしょ)を示した 
\\	実例(じつれい)を示しましょう 
\\	〔発揮する〕
\\	〔気持ちなどを表す〕
\\	彼は私が言ったことに興味を示さなかった 
\\	なかなかりっぱな統率力(とうりつりょく)を示した 
\\	彼はその案に難色(なんしょく)を示した 
\\	誠意(せいい)は行動で示してもらいたい 
\\	〔指し示す〕 カレンダーの丸をつけた日付けを指で示した 
\\	〔時計・計器が目盛りを指す〕
\\	は時計には使わない) 寒暖計(かんだんけい)は零下(れいか)2度を示した 
\\	占める	しめる	
\\	彼は銀行で重要な地位を占めている 
\\	彼は高校の3年間首席(しゅせき)を占めた 
\\	だれが第1位を占めているか 
\\	彼の党は国会で絶対多数を占めている 
\\	賛成者(さんせいしゃ)が3分の1を占めた 
\\	輸出の3割を電気製品が占めている 
\\	湿る	しめる	
\\	湿った洗濯物 
\\	草が露で湿っている 
\\	霜	しも	
\\	借金	しゃっきん	
\\	借金する 
\\	借金を申し込む 
\\	借金を全部返す 
\\	借金がかさむばかりだった 
\\	借金で首が回らない 
\\	彼は借金を踏み倒した 
\\	/((口)) 
\\	叔父に8万円借金がある 
\\	ありがたいことに今は借金がない 
\\	借金で暮らす 
\\	借金取り 
\\	借金取りに歩き回った 
\\	しゃべる	しゃべる	
\\	話す (はなす))
\\	週	しゅう	
\\	州	しゅう	
\\	銃	じゅう	
\\	周囲	しゅうい	
\\	収穫	しゅうかく	
\\	〔農作物などの〕
\\	収穫する 
\\	〔刈り入れる〕
\\	去年は収穫が多かった[少なかった] 
\\	農家はいもの収穫で多忙(たぼう)だ 
\\	〔得たよい結果〕 大学四年間の収穫は計り知れない 
\\	彼と長い間話し合ったがこれといった収穫はなかった 
\\	これといった 
\\	収穫時期 
\\	収穫高 
\\	収穫予想 
\\	宗教	しゅうきょう	
\\	重視	じゅうし	
\\	量より質を重視する 
\\	私は人柄(ひとがら)を最も重視して人を雇(やと)う 
\\	彼の発言は重視された 
\\	就職	しゅうしょく	
\\	修正	しゅうせい	
\\	修正する 
\\	予算の大幅な修正 
\\	軌道(きどう)修正 
\\	憲法修正第一条〔米国で〕 
\\	議案を修正する 
\\	字句を修正する 
\\	原稿(げんこう)に修正を加える 
\\	修正案 〔原案〕
\\	〔修正された案〕
\\	修正液 
\\	修正資本主義 
\\	修正主義 
\\	修正主義者 
\\	所得税の修正申告(しんこく) 
\\	修正予算 
\\	重体	じゅうたい	
\\	渋滞	じゅうたい	
\\	重大	じゅうだい	
\\	住宅	じゅうたく	
\\	集団	しゅうだん	
\\	集中	しゅうちゅう	
\\	集中的 
\\	集中的に 
\\	人の気持ちの集中を妨(さまた)げる 
\\	彼は注意をそれに集中した 
\\	議論はその点に集中した 
\\	彼に視線が集中した 
\\	集中豪雨(ごうう) 
\\	集中講義(こうぎ) 
\\	中国語の集中講義 
\\	集中攻撃 〔軍事〕
\\	集中攻撃する 
\\	その問題で彼は集中攻撃を受けた 
\\	集中治療室(ちりょうしつ) 
\\	((略
\\	集中データ処理システム 
\\	集中排除(はいじょ) 
\\	集中砲火(ほうか) 
\\	集中力 
\\	集中力に欠ける 
\\	集中暖房 
\\	収入	しゅうにゅう	
\\	住民	じゅうみん	
\\	重要	じゅうよう	
\\	修理	しゅうり	
\\	修理する 
\\	((米)) 
\\	この古い自動車は修理が行き届いている[いない] 
\\	行き届く= 
\\	この家は修理が必要だ 
\\	家の古い鳩(はと)時計はもう修理ができない 
\\	時計[ハンドバッグ]を修理してもらった 
\\	屋根を修理中だ 
\\	この梯子(はしご)を修理できますか 
\\	修理工場 
\\	修理人 
\\	時計修理人 
\\	主義	しゅぎ	
\\	〔行動の指針〕
\\	〔教義〕
\\	〔信念〕
\\	主義を通す[曲げる] 
\\	彼は主義に基づいて拒絶(きょぜつ)した 
\\	それは私の主義に反する 
\\	私には主義などはない 
\\	彼は最後まで主義を貫(つらぬ)きとおした 
\\	主義に殉(じゅん)ずる 
\\	安全第一主義でいこう 
\\	私は陽気であることを主義としている 
\\	主義主張 
\\	宿泊	しゅくはく	
\\	手術	しゅじゅつ	
\\	大[小]手術 
\\	盲腸炎(もうちょうえん)の手術を受ける 
\\	ひざの手術を受ける 
\\	手術中〔掲示〕 
\\	患者は今,手術中だ 
\\	手術にはもう手後れだ 
\\	手術前[後]の手当て 
\\	手術医 
\\	手術衣 
\\	手術室 
\\	[((英)) 
\\	手術台 
\\	首相	しゅしょう	
\\	手段	しゅだん	
\\	問題解決のために積極的(せっきょくてき)な手段を講(こう)じよう 
\\	講じる 
\\	手段を誤る 
\\	[((文)) 
\\	目的達成(たっせい)のための手段 
\\	生活を楽しむ手段 
\\	あらゆる手段を尽くす 
\\	彼は目的のためには手段を選ばない人間だ 
\\	手段を選ばない 
\\	その金は正当(せいとう)な[不当な]手段によって得たものだ 
\\	申し入れを聞いてもらえなければ最後の手段に訴(うった)えます 
\\	主張	しゅちょう	
\\	彼はひどく自己主張が強い 
\\	権利を主張する 
\\	無罪を主張する 
\\	彼は作品の中で自由の重要性を主張した 
\\	彼は自分がその土地の所有者(しょゆうしゃ)であると主張した 
\\	彼はあくまで主張を曲げなかった 
\\	彼の主張が通った 
\\	彼は改革(かいかく)を主張した 
\\	出場	しゅつじょう	
\\	彼は100メートル競走(きょうそう)に出場した 
\\	不正行為(こうい)のためその野球チームは出場停止になった 
\\	その競走に20人以上の出場申し込み者があった 
\\	出場する 
\\	出身	しゅっしん	
\\	〔生まれた土地〕 大阪のご出身ですか
\\	いいえ京都です 
\\	〔卒業した学校〕 官僚(かんりょう)にはその大学の出身 (者) が多い 
\\	どの大学のご出身ですか 
\\	〔所属政党など〕 新しい知事(ちじ)は何党の出身ですか 
\\	出版	しゅっぱん	
\\	首都	しゅと	
\\	主婦	しゅふ	
\\	主要	しゅよう	
\\	需要	じゅよう	
\\	需要と供給(きょうきゅう) 
\\	語順(ごじゅん 
\\	注意) 需要を満(み)たす 
\\	ビールの需要が多い[少ない] 
\\	需要インフレ 
\\	種類	しゅるい	
\\	種類に分ける 
\\	いろいろな種類の果物 
\\	いろいろな種類の花 
\\	どんな種類の人と付き合っているのか 
\\	あらゆる種類の人が集まっていた 
\\	こういう種類の研究は時間が掛かる 
\\	こういう種類の機械は長持ちする 
\\	順	じゅん	
\\	⇒じゅんばん(順番),じゅんじょ(順序) 
\\	[番号/年代]順に 
\\	背[年齢]の順に 
\\	申込(もうしこみ)受け付けは先着順になります 
\\	順を追う 
\\	順を追って説明しましょう 
\\	繰り上げる 
\\	順を繰り上げる[下げる] 
\\	順を追って経過(けいか)を説明する 
\\	順送り 
\\	この通知は読んだら順送りにしてください 
\\	瞬間	しゅんかん	
\\	順調	じゅんちょう	順調な 〔申し分のない〕
\\	〔都合のよい〕
\\	((英)) 
\\	順調に 
\\	〔円滑(えんかつ)に〕
\\	手術後は順調だ 
\\	このところ陽気(ようき)は順調だ 
\\	事は順調に運んでいる 
\\	運ぶ 
\\	新製品は順調に売れている 
\\	順番	じゅんばん	
\\	〔順序〕
\\	カードは順番が狂(くる)っている 
\\	順番を狂わせる 
\\	〔何かをする番〕 順番を待つ 
\\	やっと順番が来た 
\\	じっと立っていないと順番が分からなくなるよ 
\\	順番に歌を歌う 
\\	使用	しよう	
\\	使うこと (つかうこと)), 
\\	使用中〔部屋・便所などの掲示〕 
\\	使用禁止[できない]〔表示〕 
\\	電話は今全部,使用中です 
\\	使用済(す)みの切手 
\\	ボールペンの使用は許(ゆる)されません 
\\	余暇(よか)を有効(ゆうこう)に使用する 
\\	この表現は今では広く使用されている 
\\	使用されていないトラクターが3台ある 
\\	コンピュータは何台使用できるかね 
\\	使用価値 
\\	使用権 
\\	〔法律用語〕
\\	使用者 〔利用者〕
\\	〔消費者〕
\\	〔雇い主〕
\\	使用済み燃料 
\\	使用済み核燃料(かくねんりょう) 
\\	使用人 
\\	使用法 
\\	使用料 技術使用料 
\\	講演(こうえん)会場(かいじょう)の使用料 
\\	スキーの使用料 
\\	港湾(こうわん)使用料 
\\	埠頭(ふとう)使用料 
\\	小	しょう	
\\	章	しょう	
\\	〔文章などの段落〕
\\	第2章 
\\	この問題は章を改(あらた)めて検討(けんとう)しよう 
\\	〔記章〕
\\	会員章 
\\	賞	しょう	
\\	障害	しょうがい	
\\	奨学金	しょうがくきん	
\\	乗客	じょうきゃく	
\\	上京	じょうきょう	
\\	状況 ー 情況	じょうきょう	〔有り様〕
\\	〔おかれた立場〕
\\	〔周囲の情勢〕
\\	⇒じょうせい(情勢) 生活状況 
\\	状況を把握(はあく)する 
\\	目下(もっか)の状況では計画は成功しそうだ 
\\	目下 
\\	あの微妙(びみょう)な情況下での私の発言は誤解された 
\\	状況証拠 
\\	状況判断 
\\	状況判断をする 
\\	状況判断を誤る 
\\	状況分析 
\\	条件	じょうけん	〔必要な事柄〕
\\	((文))〔前提条件〕
\\	〔契約・協定などの付帯条件〕
\\	通例複数形); 〔資格として必要な条件〕
\\	必要十分条件 
\\	付帯(ふたい)[必須(ひっす)]条件 
\\	第一条件 
\\	講和(こうわ)[降伏(こうふく)]条件 
\\	労働条件 
\\	売買(ばいばい)[契約]条件 
\\	立地(りっち)条件 
\\	…という条件で 
\\	それは私の条件に合わない 
\\	彼女は条件に合わず採用(さいよう)されなかった 
\\	かれらは支払い条件について一致した 
\\	…に条件をつける 
\\	有利(ゆうり)な[相応な/無理のない/対等の]条件で交渉する 
\\	前払(まえばら)いを条件とする 
\\	条件によっては援助しないでもないさ 
\\	健康が成功の条件である 
\\	条件付き 条件付きの (で) 
\\	条件付き契約 
\\	条件付き採用 
\\	彼の提案に条件付きで賛成する 
\\	政府の補助金(ほじょきん)はいつも何かの条件付きである 
\\	条件付け 〔心理学で〕
\\	条件付けする 
\\	条件闘争(とうそう) 
\\	条件反射 〔生理学で〕
\\	条件文[節] 〔文法で〕
\\	条件法 〔文法で〕
\\	正午	しょうご	
\\	正直	しょうじき	
\\	〔率直〕
\\	〔誠実〕
\\	正直な 
\\	〔率直な〕
\\	正直に言ってあの時は穴があったら入りたいぐらいだった 
\\	ばか正直な人 
\\	正直は最良の策 
\\	正直なところ君は間違っている 
\\	君の正直な気持ちが聞きたい 
\\	正直に言いなさい 
\\	カメラは正直である 
\\	正直者 
\\	これは正直者がばかを見るいい例だ 
\\	常識	じょうしき	
\\	少女	しょうじょ	
\\	少々	しょうしょう	
\\	少し (すこし))
\\	症状	しょうじょう	
\\	生じる	しょうじる	
\\	⇒はえる(生える) 
\\	庭に名前のわからない植物が生えてきた 
\\	〔起こる〕⇒おこる(起こる) 
\\	危険な事態が生じた  
\\	〔生まれる〕  
\\	資本(しほん)から生じる利子(りし) 
\\	思わぬ結果が生じた 
\\	彼らの不和(ふわ)は何から生じたのですか 
\\	暴力(ぼうりょく)は恐怖(きょうふ)から生じる 
\\	状態	じょうたい	
\\	通例単数形); 〔状況〕
\\	通例複数形) 健康[精神]状態 
\\	その国の現在の状態 
\\	生活状態 
\\	今の状態では 
\\	橋は危険な状態だ 
\\	彼女はとても通学できるような状態ではない 
\\	体の状態を整(ととの)える 
\\	健康状態がよい[悪い] 
\\	上達	じょうたつ	
\\	英語が少しずつ[ぐんぐん]上達している 
\\	フランス語がすっかり上達した 
\\	驚(おどろ)くほど絵が上達した 
\\	冗談	じょうだん	
\\	上等	じょうとう	上等な絹織物(きぬおりもの) 
\\	上等な機械 
\\	上等の酒 
\\	最上等のカメラ 
\\	「これでいいですか」「ああ,上等だよ」 
\\	衝突	しょうとつ	
\\	〔ぶつかること〕
\\	衝突する 
\\	2台の車が正面衝突した 
\\	店から飛び出して来た男と衝突した 
\\	〔相反すること〕 利害の衝突 
\\	彼と意見が衝突した 
\\	衝突事故 
\\	商人	しょうにん	
\\	承認	しょうにん	
\\	承認する 
\\	それは知事の承認を要する 
\\	議案が承認された 
\\	親の承認なしで 
\\	先生は生徒たちの申し出を承認した 
\\	国家の独立を承認する 
\\	証書(しょうしょ)を承認する 
\\	少年	しょうねん	
\\	商売	しょうばい	
\\	消費	しょうひ	
\\	商品	しょうひん	
\\	賞品	しょうひん	
\\	消防	しょうぼう	
\\	情報	じょうほう	
\\	証明	しょうめい	
\\	省略	しょうりゃく	
\\	女王	じょおう	
\\	職	しょく	
\\	職業	しょくぎょう	〔一般に〕
\\	〔熟練を要する〕
\\	〔専門知識を要する〕
\\	職業的 
\\	ご職業は 
\\	教師の職業 
\\	職業を身につける 
\\	父の職業は大工[医者]でした 
\\	職業安定所 ((米)) 
\\	((口)) 
\\	((英)) 
\\	職業案内欄 
\\	職業意識 彼は職業意識がない 
\\	職業教育 
\\	職業柄 
\\	彼は職業柄,計算が速い 
\\	職業軍人 
\\	職業訓練 
\\	職業指導 
\\	職業指導係 
\\	職業紹介所 
\\	職業選手 
\\	((複 〜
\\	職業病 
\\	職業婦人 
\\	〔知的職業の〕
\\	〔専門職の〕
\\	職業別電話帳 
\\	〔1冊の中の職業欄〕((米)) 
\\	食卓	しょくたく	
\\	食品	しょくひん	
\\	植物	しょくぶつ	
\\	食物	しょくもつ	
\\	食欲	しょくよく	
\\	食料	しょくりょう	
\\	食糧	しょくりょう	
\\	書斎	しょさい	
\\	斎 
\\	女子	じょし	
\\	助手	じょしゅ	
\\	研究助手 
\\	山田教授の助手を務める 
\\	運転助手 
\\	助手席 〔自動車などの〕
\\	((口))
\\	徐々に	じょじょに	
\\	〔ゆっくり〕
\\	徐々に前進する 
\\	〔段々に〕
\\	女性の地位は徐々に向上した 
\\	署名	しょめい	
\\	書類に署名捺印(なついん)する 
\\	署名のない手紙 
\\	著者の署名のある本 
\\	ここに署名してください 
\\	署名入り記事 
\\	記事の前に記者の名前が出る) 署名運動 署名運動をする 
\\	署名国 
\\	署名者 
\\	〔法律で〕
\\	署名本 
\\	書物	しょもつ	
\\	女優	じょゆう	
\\	処理	しょり	
\\	〔始末すること〕
\\	処理する 
\\	ごみの処理 
\\	下水(げすい)処理 
\\	廃棄物(はいきぶつ)を処理する 
\\	〔化学的な処置〕
\\	物質を熱[酸]で処理する 
\\	〔事件・事務などの始末〕 事は簡単に処理できない 
\\	問題の処理がまずかった 
\\	/((口)) 
\\	それは事務的に処理すべきだ 
\\	処理場(しょりじょう) 下水処理場 
\\	処理速度 〔コンピュータで〕
\\	書類	しょるい	
\\	知らせ	しらせ	
\\	尻	しり	しり
\\	尻・×臀】 ーーー 尻が大きい 
\\	糸(いと)の尻を結(むす)ぶ 
\\	尻はしょりする 
\\	彼はクラスのどん尻にいる 
\\	尻が暖まる暇もないほど忙しい 
\\	尻が軽(かる)い 
\\	《俗語》 
\\	尻が割(わ)れる 
\\	尻の穴が小さい 
\\	尻をたたく 〔励ます〕
\\	〔催促する〕
\\	〔人間の腰〕
\\	は一方をさす); 〔背部の隆起した部分〕
\\	((文)) 
\\	((口)) 
\\	((口)) 
\\	((米口))
\\	((英俗))
\\	尻が大きい 
\\	彼は息子の尻を平手(ひらて)でたたいた〔罰として〕 
\\	は数回,
\\	は一回) ダンサーはお尻を振って踊った 
\\	〔牛などのしり肉〕
\\	〔底,端〕 徳利(とくり)の尻 
\\	糸(いと)の尻を結(むす)ぶ 
\\	ズボンの尻が抜(ぬ)けそうだ 
\\	[((米)) 
\\	〔衣服のすそ〕 尻はしょりする 
\\	〔後〕 彼はクラスのどん尻にいる 
\\	100メートル競走(きょうそう)で尻から2番目だった 
\\	[((英))
\\	列の尻にいる 
\\	その子は母親の尻に付いてまわった 
\\	尻が暖まる 彼は尻の暖まる暇もなく転勤(てんきん)になった 
\\	尻が暖まる暇もない 
\\	尻が落ち着かない 彼女は尻が落ち着かない 
\\	尻が重い 
\\	あの人は尻が重い 
\\	尻が軽(かる)い 
\\	《俗語》 
\\	彼女は尻が軽い 〔軽薄〕
\\	/〔移り気〕
\\	/〔身持ちが悪い〕
\\	/((俗)) 
\\	尻が据わる 仕事に尻が据(す)わってきた 
\\	尻が長い 
\\	あいつは尻が長い 
\\	尻が割(わ)れる 
\\	尻が割れて町にいられなくなった 
\\	尻に敷く 
\\	亭主(ていしゅ)を尻に敷(し)いている 
\\	女房(にょうぼう)の尻に敷かれた男 
\\	尻に火がつく 尻に火がつくまで仕事をやらないんだから 
\\	尻に帆を掛ける 
\\	連中(れんちゅう)は尻に帆(ほ)を掛けて逃げていった 
\\	連中 
\\	尻の穴が小さい 
\\	あいつは尻の穴が小さいやつだ 〔金銭的に〕
\\	/〔心が狭い〕
\\	尻を追い回す 
\\	女の尻を追い回す 
\\	尻を落ち着ける 〔定住する〕
\\	〔腰を据える〕
\\	⇒おちつく(落ち着く) 尻を据(す)える 
\\	じっくりと尻を据えて仕事をする 
\\	尻をたたく 〔励ます〕
\\	〔催促する〕
\\	しりぬぐい 
\\	君のへまの尻ぬぐいは断る 
\\	保証人だったので彼の借金の尻ぬぐいをさせられた 
\\	尻をまくる 
\\	自分の失敗なのに彼は尻をまくって居直(いなお)った 
\\	尻を向ける 
\\	知合い	しりあい	
\\	印	しるし	
\\	〔目印〕
\\	印の付いた語 
\\	と読む) 単語に星の印を付ける 
\\	〔記章〕
\\	柏(かしわ)はその学校の印だ 
\\	〔商標(しょうひょう)〕
\\	〔銘柄(めいがら)〕
\\	〔証拠(しょうこ)〕
\\	⇒しょうこ(証拠); 〔気持ちなどの表れ〕
\\	感謝の印 
\\	旅行の記念の印に 
\\	ほんの印ばかりですがお気に召したらうれしいです 
\\	ほんの 
\\	岩(いわ)の上に氷河(ひょうが)の移動(いどう)の印を見た 
\\	城	しろ	
\\	進学	しんがく	
\\	神経	しんけい	
\\	〔器官〕
\\	神経の 
\\	視神経(ししんけい) 
\\	神経を殺す[抜く] 
\\	〔過敏な心の働き〕
\\	〔感受性〕
\\	神経が細(こまか)い 
\\	神経が太い 
\\	/((口)) 
\\	神経が鈍い 
\\	/〔愚鈍な〕
\\	神経が高(たか)ぶっている 高ぶる 
\\	/((口)) 
\\	何という神経だ〔厚かましい〕 
\\	!/((英)) 
\\	神経に触(さわ)る騒音 
\\	あの金属音は神経に触る 
\\	私の言った事が彼の神経に触ったらしい 
\\	それは君の神経のせいさ 
\\	神経をすり減らす 私は社内の人間関係に神経をすり減(へ)らした 
\\	神経を尖(とが)らせる 車の音に神経を尖らせている 
\\	彼は私の小さなミスにも神経を尖(とが)らせる 
\\	神経炎 〔医学用語〕
\\	神経科 
\\	神経科医 
\\	神経過敏(かびん) 神経過敏の 
\\	((口)) 
\\	((英口)) 
\\	神経過敏になる 
\\	/((口)) 
\\	神経学 
\\	神経ガス 
\\	神経系統 
\\	神経細胞 
\\	神経質 
\\	神経質の 
\\	〔神経を張り詰めた〕
\\	((口)) 
\\	神経症 
\\	神経症の 
\\	神経症患者 
\\	神経障害 
\\	神経衰弱(ずいじゃく) 
\\	〔医学用語〕
\\	神経戦 
\\	神経繊維(せんい) 
\\	神経組織 
\\	神経中枢 
\\	神経痛 
\\	神経病 
\\	真剣	しんけん	
\\	〔本当の剣〕
\\	〔本気で取り組む様子〕 真剣な 
\\	真剣に 〔本気で〕
\\	〔熱心に〕
\\	人生について真剣に考える 
\\	私は真剣です 
\\	/((口))
\\	[((口)) 
\\	いつも真剣さが足りない 
\\	真剣勝負(しょうぶ) 〔真剣を使っての勝負〕
\\	〔本気で争うこと〕
\\	信仰	しんこう	
\\	信号	しんごう	
\\	人工	じんこう	
\\	深刻	しんこく	深刻な 〔重大な〕
\\	〔きわめて重大な〕
\\	深刻な表情で 
\\	事態は一段と深刻になった 
\\	深刻な人口問題 
\\	診察	しんさつ	
\\	人種	じんしゅ	
\\	共通の習慣・文化などをもつ集団) 人種の 
\\	黄色(おうしょく)[白色(はくしょく)]人種 
\\	人種の別なく 
\\	この町の住民は三つの人種から成(な)り立っている 
\\	人種学 
\\	人種学者 
\\	人種差別 
\\	〔人種隔離策(かくりさく)〕
\\	〔南アフリカの〕
\\	1993年全廃) 人種差別主義[者] 
\\	人種差別撤廃(てっぱい) 
\\	学校における人種差別撤廃 
\\	人種的偏見(へんけん) 
\\	人種問題 
\\	人種問題から起こる暴動(ぼうどう) 
\\	信じる	しんじる	
\\	人生	じんせい	
\\	〔一生〕
\\	人生を楽しむ 
\\	それが人生なのさ 
\\	人生観 
\\	人生行路 
\\	人生哲学 
\\	親戚	しんせき	
\\	新鮮	しんせん	
\\	心臓	しんぞう	
\\	身体	しんたい	
\\	身長	しんちょう	
\\	慎重	しんちょう	
\\	審判	しんぱん	
\\	〔事件などの〕
\\	⇒はんけつ(判決) 審判する 
\\	裁判官は彼を有罪(ゆうざい)と審判した 
\\	〔競技の〕 審判する 
\\	ボクシングの試合の審判をする 
\\	〔キリスト教で〕
\\	最後の審判の日 
\\	審判員 〔野球・テニスなどの〕
\\	〔ラグビーなどの〕
\\	人物	じんぶつ	
\\	〔人〕
\\	((口)) 
\\	〔重要な人〕
\\	〔小説などに描かれた人間〕
\\	危険人物 
\\	要注意人物 
\\	重要人物 
\\	の略) 歴史上の人物 
\\	小説[劇]の登場(とうじょう)人物 
\\	実業界(じつぎょうかい)きっての人物 
\\	切っての 
\\	彼は相手国(あいてこく)から外交官として好(この)ましからざる人物とされた 
\\	好ましからざる 
\\	好ましからざる人物 
\\	〔人柄,性格〕
\\	どんな人物ですか 
\\	立派な人物だ 
\\	変わった人物だ 
\\	人物は申し分(ぶん)ない 
\\	申し分無い 
\\	人物を見る目がある[ない] 
\\	彼は人物が大きい[小さい] 
\\	〔能力のある人〕
\\	人物画 
\\	人物画家 
\\	人物試験 
\\	人物描写(びょうしゃ) 
\\	人物評[論] 
\\	進歩	しんぽ	
\\	〔前進〕
\\	〔改善〕
\\	進歩する 
\\	進歩的 
\\	〔進んだ〕
\\	進歩が速い[遅(おそ)い] 
\\	進歩が速い[遅い]生徒 
\\	科学の進歩は著しい 
\\	進歩的な思想 
\\	作文がたいへん進歩した 
\\	彼の考えは当時の人より進歩していた 
\\	進歩主義 
\\	進歩主義者の 
\\	進歩派 〔人〕
\\	〔総称〕
\\	親友	しんゆう	
\\	信用	しんよう	
\\	〔信頼〕
\\	は経験や証拠に基づく信用に,
\\	は尊敬など主観に基づく信用に使われることが多い
\\	は理性を越えた強い信用
\\	は頼みにすること) 信用する 
\\	彼は社長に十分信用がある 
\\	[((文)) 
\\	信用を得る[失う] 
\\	国民は現政府をあまり信用していない 
\\	私は彼女[彼女の能力]を完全(かんぜん)に信用している 
\\	そんなことを続けたらだれも彼を信用しなくなる 
\\	〔信じること〕 彼の言うことを信用します 
\\	[((文))
\\	彼は信用の出来る人物だ 
\\	娘は人の言葉を信用しやすいので心配だ 
\\	彼の言うことは全然信用しない 
\\	医者の判断を100パーセント信用している 
\\	〔評判〕
\\	〔取り引き相手の信用度〕
\\	店の信用を傷つける[にかかわる]ことはしたくない 
\\	信用のある商社に掛け合ったほうがよい 
\\	掛け合う 
\\	信用買い 
\\	信用貸し 
\\	信用貸しで 
\\	信用金庫 
\\	信用組合 
\\	信用限度 
\\	信用詐欺師 
\\	[((口)) 
\\	信用状 
\\	((略
\\	信用照会 
\\	信用状態 
\\	信用調査 
\\	信用調書 
\\	信用取り引き 〔証券で〕
\\	〔商業で〕
\\	信用取り引き勘定 
\\	信用販売 
\\	信用銘柄 
\\	信頼	しんらい	
\\	心理	しんり	〔精神状態〕
\\	〔精神活動・物の考え方〕
\\	〔心理状態〕
\\	心理的 (に) 
\\	群集心理 
\\	彼には女性の繊細(せんさい)な心理は分からない 
\\	それが大衆(たいしゅう)心理というものだ 
\\	心理的に疲れ切っていた 
\\	心理学 
\\	児童(じどう)[教育/犯罪]心理学 
\\	心理学者 
\\	心理言語学 
\\	心理作戦 
\\	心理小説 〔戦術〕
\\	〔比喩的(ひゆてき)に〕
\\	心理状態 その時の心理状態は異常だった 
\\	心理戦争 
\\	心理描写(びょうしゃ) 
\\	心理療法 
\\	心理療法を受けている 
\\	心理療法士 
\\	人類	じんるい	
\\	人類の幸福 
\\	人類の歴史 
\\	「差別・人権」 人類愛 
\\	人類学 
\\	人類学者 
\\	巣	す	
\\	〔鳥・小動物・昆虫などの〕
\\	〔獣の〕
\\	主に大形の野獣の) 蜂(はち)の巣 〔雀蜂などの〕
\\	/〔蜜蜂の〕
\\	くもの巣 
\\	巣を作る 
\\	〔すみか〕 悪の巣 
\\	/〔犯罪の温床〕
\\	愛の巣 
\\	ばい菌の巣 
\\	病気の巣 
\\	酢	す	
\\	図	ず	
\\	水準	すいじゅん	〔標準〕
\\	〔程度〕
\\	水準に達する
\\	水準に達しない
\\	/((口)) 
\\	生活の水準を高める[低くする]
\\	彼の学力は水準以上[以下]だ
\\	数学の水準では我々のほうが高い[彼らと同じである]
\\	水準器 
\\	水準儀 
\\	水準測量 
\\	水準面 
\\	推薦	すいせん	
\\	推薦する 
\\	校長の推薦で彼を採用(さいよう)した 
\\	彼女をあなたの秘書(ひしょ)に推薦します 
\\	高山氏を会長候補(こうほ)に推薦したい 
\\	推薦候補 
\\	党推薦候補 
\\	推薦者 
\\	〔身元照会先〕
\\	((英)) 
\\	就職申し込みをするとき推薦していただけますか 
\\	被推薦者 
\\	推薦状 
\\	〔身元などの照会状〕
\\	推薦図書 
\\	推薦入学 
\\	この大学は推薦入学を認(みと)めている 
\\	睡眠	すいみん	
\\	数字	すうじ	
\\	姿	すがた	
\\	(体付き)
\\	(格好)
\\	(外見)
\\	(様子)
\\	〔体の格好〕
\\	彼女は優雅(ゆうが)な姿をしている 
\\	トムは後ろ姿がジムにそっくりだ 
\\	彼女の目には父親の姿が今でもありありと浮かぶ 
\\	〔外見,身なり〕
\\	托鉢(たくはつ)僧に姿をやつす 
\\	男の子はみすぼらしい姿をしていた 
\\	その子は変わり果てた姿で親のもとに帰って来た 
\\	〔ありさま,様子〕 若者が去って,この村の昔の姿はもはやない 
\\	それが民主国家の本当の姿だ 
\\	〔物の形〕 穂高が雲の間から姿を現した 
\\	間もなく湖が姿を現した 
\\	太陽が何日も姿を見せなかった 
\\	〔人の目に映る体,その人〕 ゆうべの集まりでティムの姿は見かけなかった 
\\	ジムはまだ姿を見せない 
\\	彼の姿はどこにも見当たらなかった 
\\	あれっきり彼は姿を見せない 
\\	彼は姿を隠した[くらました] 
\\	姿焼き 鯛の姿焼き 
\\	例文 「彼、とうとう芥川賞を取ったね」「作家になるのが夢だったからなあ。あいつの喜ぶ姿が目に浮かぶよ」 
\\	「彼女、今日のパーティに来るって言ってたのに、姿が見えないねえ」「変だなあ。場所がわからないのかなあ」 
\\	「彼なら詳しい事情を知っているだろう」「それが、この1週間姿を消していて行方がつかめないんです」 
\\	「着物姿の君もなかなか素敵だね」「ありがとう。こう見えても着物は好きなの」 
\\	救う	すくう	
\\	すくなくとも	すくなくとも	
\\	少なくとも5万円[10時間]はかかる 
\\	1日に少なくとも20ドル要る 
\\	彼は少なくとも3千冊の本を持っている 
\\	優れる	すぐれる	
\\	〔勝(まさ)る〕
\\	雄弁家(ゆうべんか)として優れている 
\\	彼は学力では他の人々より優れている 
\\	数学では彼より優れている者はいなかった 
\\	少しも 
\\	ない	すこしも	((否定を伴って))
\\	彼は少しもひるまなかった 
\\	[((米口)) 
\\	ひるむ 
\\	そんなことは少しも気にならない 
\\	少しも知らなかった 
\\	少しも疑わしい点はない 
\\	彼が不誠実(ふせいじつ)だとは少しも知らなかった 
\\	「朝食(ちょうしょく)をどれくらい食べた」「少しも」 
\\	過ごす	すごす	
\\	〔時を〕
\\	テレビを見ながら (時を) 過ごした 
\\	彼の家で楽しい時を過ごした 
\\	軽井沢で1か月を過ごした 
\\	彼は平凡な一生を過ごした 
\\	時を無駄に過ごすな 
\\	いかがお過ごしでしょうか 
\\	〔度を越す〕 彼はしばしば酒を過ごす 
\\	冗談が度を過ごしている 
\\	筋!!!	すじ	
\\	〔筋肉の繊維(せんい)〕
\\	〔腱(けん)〕
\\	筋張(は)った手 
\\	〔線〕
\\	〔縞(しま)〕
\\	紙の真ん中に上から筋を引きなさい 
\\	〔血管〕
\\	⇒あおすじ(青筋) 
\\	〔繊維〕
\\	筋の多い肉 
\\	〔道すじ〕 
\\	甲州街道(かいどう)筋の宿屋(やどや) 
\\	〔血統〕((文)) 
\\	彼は筋がよい 
\\	〔素質〕
\\	彼の習字(しゅうじ)は筋が良いようだ 
\\	〔道理〕
\\	筋を通す 
\\	〔物語の〕
\\	物語はどういう筋ですか 
\\	〔方面〕
\\	確かな筋からの情報では戦争は終わりに近付いている 
\\	〔筋肉の繊維〕
\\	〔腱(けん)〕
\\	筋張った手 
\\	肩の筋が凝っている 
\\	首の筋を寝違えた 
\\	ねちがえる 
\\	くるぶしの筋を違えた 
\\	〔線〕
\\	〔縞(しま)〕
\\	紙の真ん中に上から筋を引きなさい 
\\	緑の筋の入ったユニフォーム 
\\	〔血管〕
\\	⇒あおすじ(青筋) 
\\	〔繊維〕
\\	いんげんは筋を取ってください 
\\	筋の多い肉 
\\	〔道すじ〕 甲州街道筋の宿屋 
\\	〔血統〕((文)) 
\\	彼は筋がよい 
\\	彼は源氏の筋を引いている 
\\	〔素質〕
\\	彼の習字は筋が良いようだ 
\\	〔道理〕
\\	筋を通す 
\\	筋が通った話なら聞きましょう 
\\	君の言うことは筋が通らないよ 
\\	筋が違う ⇒すじちがい(筋違い)
\\	〔物語の〕
\\	物語はどういう筋ですか 
\\	〔方面〕
\\	確かな筋からの情報では戦争は終わりに近付いている 
\\	その筋からのお達しです 
\\	モスクワにおける西側の外交筋 
\\	すじ
\\	筋】 【例文】 二筋の涙がほおを伝った 
\\	一筋の白い道が畑の中を通っていた 
\\	進める	すすめる	
\\	〔前進させる〕 将軍は兵を国境まで進めた 
\\	彼は (将棋の) 歩(ふ)を進めた 
\\	車をこの線まで進めて下さい 
\\	〔進展させる〕 彼は交渉を進めた 
\\	この計画(けいかく)を進めてください 
\\	新工法(こうほう)は工事を進めるのに役立(やくだ)った 
\\	この縁談(えんだん)を進めていいね 
\\	〔昇進させる〕 功績(こうせき)により彼は位(い)一級(いっきゅう)を進められた 
\\	〔食欲を盛んにする〕 この調味料(ちょうみりょう)は食欲を進める 
\\	〔時刻を早くする〕 時計を5分進めた 
\\	〔程度を高める〕 もう一歩進めて考えてみなさい 
\\	勧める	すすめる	
\\	〔忠告する〕
\\	〔提案する〕
\\	〔励ます〕
\\	担任(たんにん)教師に勧められてこの大学に進んだ 
\\	教授は私に留学を勧めてくれた 
\\	あの人に生命保険(せいめいほけん)に入るよう勧めても無駄(むだ)だ 
\\	その旅行を取りやめるよう彼に勧めた 
\\	[((文)) 
\\	人に勧められてご相談にうかがいました 
\\	遊びに来るよう彼に勧めてください 
\\	遊びに来る 
\\	店員は私にそのコートを試着(しちゃく)してみるようにと勧めた 
\\	〔差し出す〕
\\	スチュワーデスはキャンデーを客に勧めた 
\\	彼は私に葡萄酒(ぶどうしゅ)を無理に勧めた 
\\	〔奨励(しょうれい)する〕
\\	産業[学問]を勧める 
\\	頭痛	ずつう	
\\	ずっと	ずっと	
\\	〔時間・距離が離れて〕 
\\	ずっと以前に君に会ったことがある 
\\	ずっと遠くに富士山が見える 
\\	沖縄はずっと南の方だ 
\\	〔長い間続けて〕
\\	今までずっとここにいたのか 
\\	京都からずっと立ち通しだった 
\\	立ち通し 
\\	このところずっと外出していない 
\\	一晩中ずっと本を読み続けた 
\\	彼女は授業中ずっと眠っていた 
\\	彼は一生ずっと独身で通した 
\\	〔まっすぐ〕
\\	この道をずっと行くと公会堂に着きます 
\\	ずっと入ってください 
\\	〔はるかに〕
\\	君の方が彼女よりずっと美しい 
\\	母はずっとよくなっている 
\\	この本は他の本よりもずっと面白い 
\\	すっぱい	すっぱい	
\\	梅干(うめぼ)しは酸っぱい 
\\	酸っぱいりんご[すもも] 
\\	酸っぱくて顔がゆがんだ 
\\	この豆腐は少し酸っぱくなっている 
\\	口を酸っぱくして 両親が口を酸っぱくしてもっと勉強に時間を使うようにと説得(せっとく)したので,彼はついにサッカーをやめた 
\\	そんな事をするなと口を酸っぱくして言ったじゃないか 
\\	すてき	すてき	
\\	既に	すでに	
\\	もう) 
\\	〔それまでに〕
\\	私が着いた時には彼らは既に出発していた 
\\	既に述べたように 
\\	〔今ではもう〕 既に手術できる状態ではなかった 
\\	時既に遅(おそ)しだ 
\\	すなわち	すなわち	
\\	1ポンド,すなわち100ペンス 
\\	日本の中等(ちゅうとう)教育は二つの学校,すなわち中学校と高等学校で行われる 
\\	彼の祖父(そふ),すなわち故山本大将 
\\	全て	すべて	
\\	済ませる	すませる	
\\	〔終らせる〕
\\	早く仕事を済ませてしまいなさい 
\\	〔払う,返す〕
\\	勘定(かんじょう)はもう済ませた 
\\	〔間に合わせる〕
\\	月10万円では済まされない 
\\	水がなくては1日も済まされない 
\\	〔解決する〕
\\	問題はこのままでは済まされない 
\\	笑って済まされる問題ではない 
\\	金で済むなら済ませたい 
\\	隅	すみ	〔囲まれた部分の角〕
\\	((文)) 
\\	私は写真の左下(ひだりした)の隅にいます 
\\	いつも隅に引っ込んでいてはだめよ 
\\	引っ込む 
\\	隅から隅まで 隅から隅まで捜したのか 
\\	そのことを隅から隅まで知っている 
\\	隅に置けない 
\\	あんな美人の彼女がいるなんてあの男も隅に置けないな 
\\	墨	すみ	
\\	すみません (かん)	すみません (かん)	
\\	澄む	すむ	
\\	〔液体などが〕
\\	澄んだ水 
\\	/((文))
\\	〔空などが〕
\\	澄んだ空 
\\	星が澄んでいる 
\\	澄んだ色 
\\	澄んだ目 
\\	〔音が〕
\\	澄んだ声 
\\	〔心が〕
\\	澄みきった心境(しんきょう) 
\\	【澄む・▲清む】	すむ	
\\	〔液体などが〕
\\	澄んだ水 
\\	/((文))
\\	〔空などが〕
\\	澄んだ空 
\\	星が澄んでいる 
\\	澄んだ色 
\\	澄んだ目 
\\	〔音が〕
\\	澄んだ声 
\\	〔心が〕
\\	澄みきった心境 
\\	刷る	する	
\\	鋭い	するどい	
\\	すれ違う	すれちがう	
\\	ずれる	ずれる	
\\	〔正しい位置からそれる〕 壁の絵が少しずれている 
\\	机の位置がずれている 
\\	コースが北にずれたようだ 
\\	印刷(いんさつ)がずれた 
\\	布団がベッドからずれていた 
\\	〔基準からはずれる〕 常道(じょうどう)[正しい行為]からずれる 
\\	会の予定が1週間ずれた[ずれて1週間早まった] 
\\	君の言うことは少しずれているようだ 
\\	彼の時代感覚(じだいかんかく)はずれている 
\\	正	せい	
\\	〔正しいこと〕 正邪(せいじゃ) 
\\	〔真の〕 正三角形(せいさんかくけい) 
\\	正四面体(せいしめんたい) 
\\	〔副に対する〕
\\	正副2通 
\\	〔正規の〕 正会員 
\\	〔数学で,負に対する〕
\\	正数 
\\	正負(せいふ) 
\\	性	せい	《性・性別》 性,性別 
\\	は生物学的な両性の差異, 
\\	は社会的・文化的な両性の意識,役割などの差異を指すとされる) 異性(の人) 
\\	女性 
\\	女性性 〔女性であること〕
\\	性的アイデンティティ 
\\	男性 
\\	男性性 〔男性であること〕
\\	同性(の人) 
\\	《性行動》 異性愛 
\\	異性愛の 
\\	差別的と取られることがある) 異性愛者 
\\	後の2つは差別的と取られることがある) オルガスム 
\\	オルガスムに達する 
\\	婚前交渉 
\\	サディズム 
\\	自慰(じい) 
\\	自慰行為をする 
\\	性行動 
\\	性的衝動,性欲 
\\	〔さかり〕
\\	性交 
\\	〔医学用語〕
\\	性交する 
\\	性転換 
\\	性転換者 
\\	性転換手術 
\\	セックスレス夫婦 
\\	同性愛 
\\	両性ともに用いる); 〔女性〕
\\	同性愛の 
\\	〔女性〕
\\	同性愛者 
\\	服装倒錯,異装趣味 
\\	服装倒錯者 
\\	((米俗))
\\	フリーセックス 
\\	ペッティング 
\\	マゾヒズム 
\\	無性愛 
\\	無性愛の 
\\	は「雌雄の別がない,無性の」の意もある. 
\\	は「性とは無関係の」の意がある) 無性愛者 
\\	乱交 
\\	両性愛 
\\	両性愛の 
\\	両性愛者 
\\	《生殖・出産》 生殖,出産 
\\	顕微授精 
\\	私生児 〔未婚の母の子〕
\\	人工授精 
\\	シングルマザー 
\\	精子 
\\	((複〜, 〜
\\	精子銀行 
\\	生殖器,性器 
\\	体外受精 
\\	((略
\\	胎児性別の判定法 
\\	代理出産 
\\	代理母 
\\	多胎減数 
\\	多胎減数手術 
\\	男女産み分け法 
\\	中絶 
\\	妊娠 
\\	妊娠する 
\\	妊娠させる 
\\	妊娠中の 
\\	計画外妊娠 
\\	意に反した妊娠 
\\	妊娠診断薬 
\\	認知訴訟 〔私生児の〕
\\	妊婦 
\\	配偶者間人工授精 
\\	((略
\\	不妊 
\\	不妊手術 
\\	非配偶者間人工授精 
\\	((略
\\	未婚の母 
\\	ラマーズ法 
\\	卵子 
\\	《性をめぐる病気》 クラミジア感染症 
\\	性感染症 
\\	((略
\\	性的不能,インポテンス 
\\	性器ヘルペス症 
\\	バイアグラ 
\\	商標名) 性同一性障害,性転換願望症 
\\	《避妊方法・用具・薬・関連法》 オギノ式 
\\	経口避妊薬,ピル 
\\	コンドーム 
\\	((米口))
\\	殺精子剤 
\\	子宮内避妊器具,避妊リング 
\\	((略
\\	人工妊娠中絶法 
\\	断種 
\\	男性用ピル 
\\	腟内リング 
\\	避妊手段[用具] 
\\	避妊ワクチン 
\\	母体保護法 
\\	《性産業》 性産業 
\\	アダルトビデオ 
\\	アダルト漫画 
\\	援助交際 
\\	主婦売春 
\\	少女売春 
\\	ストリッパー 
\\	ストリップショー 
\\	セックスの産業化 
\\	売春 
\\	買春 
\\	売春婦 
\\	売春夫 
\\	売春宿 
\\	風俗 
\\	ポルノ映画 
\\	ポルノグラフィー 
\\	ポルノ雑誌 
\\	ポルノショップ 
\\	《性をめぐる運動・主義・思想・研究》 ジェンダー研究 
\\	女性研究,女性学 
\\	性教育 
\\	性差別主義 
\\	性平等主義 
\\	性平等主義の 
\\	男女同一化 
\\	男性研究,男性学 
\\	男性優位主義者 
\\	男尊女卑,男性優位主義 
\\	フェミニスト 
\\	フェミニズム 
\\	同性愛研究 
\\	同性愛文化,同性愛風俗 
\\	ピルを服用している 
\\	避妊薬を常用している 
\\	そのカップルは,現在避妊用具を使用している 
\\	先進国のうち,ピルや新しい子宮内避妊器具の販売が許可されていないのは日本だけであった 
\\	現在20種を越える性感染症が確認されている 
\\	1998年4月,アメリカで男性のインポテンスを治療する初の経口薬が販売された 
\\	売春防止法はいまや,男性の買春(かいしゅん)にも処罰を適用している 
\\	セックスレス夫婦が増えているという報告がある 
\\	最近は髪型も服装も性別がなくなってきた 
\\	あの子は計画外の子です 
\\	医者は性病の感染予防にコンドームを勧めている 
\\	性同一性障害に対する公認の手術は日本では1998年に初めて行われた 
\\	この国では同性愛者の多くはまだ身を潜めている 
\\	姓	せい	
\\	彼の姓は山田です 
\\	田中の姓を名乗(なの)る 
\\	せい	所為	
\\	列車に乗り遅れたのは交通渋滞(じゅうたい)のせいです 
\\	友達はその失敗を彼のせいにした 
\\	彼は同僚の成功を幸運のせいにした 
\\	僕のせいじゃない 
\\	これは長い習慣(しゅうかん)のせいだ 
\\	気のせいだよ
\\	地震なんてなかったよ 
\\	薬のせいで眠くて仕方がなかった 
\\	税	ぜい	
\\	性格	せいかく	
\\	正確	せいかく	正確な (に) 
\\	は誤りがないという一般的な語
\\	は寸分の違いもない,
\\	は決まった水準に忠実な,
\\	は細部まで厳密に正確なという意味を含む) 正確な答えを出す 
\\	彼は正確な英語を話す 
\\	正確な日付けは何日でしたか 
\\	正確な円 
\\	記憶が正確である 
\\	私の時計は正確だ 〔現在合っている〕
\\	/〔いつも合っている〕
\\	正確に言えば5ミリ足りない 
\\	私の言った事を正確に彼に伝えて欲しい 
\\	彼女は約束(やくそく)通り正確に7時にやって来た 
\\	/((口)) 
\\	ニュースは正確さが第一である 
\\	時計仕掛けのような正確さで彼は爆弾の信管を外した 
\\	世紀	せいき	
\\	請求	せいきゅう	〔強い要求〕
\\	〔要望〕
\\	〔権利としての要求〕
\\	請求する 
\\	〔代金などを〕
\\	⇒ようきゅう(要求) 支払いを請求する 
\\	/〔請求書を送って〕
\\	請求があり次第支払います 
\\	請求に応じる 
\\	損害賠償(そんがいばいしょう)を請求する 
\\	本のカタログは請求があれば送付(そうふ)します 
\\	見本[願書用紙]を請求する 
\\	この本の請求は私に付けておいてください 
\\	請求額 
\\	請求権 
\\	請求者 
\\	請求書 
\\	請求書を出す 
\\	請求払い 
\\	((略
\\	税金	ぜいきん	
\\	清潔	せいけつ	
\\	潔 
\\	〔きれいな状態〕
\\	清潔な 
\\	清潔なタオル 
\\	清潔さの点ではこのホテルの方がよい 
\\	体[手足]を清潔にしておく 
\\	あの人は清潔な感じがする 
\\	〔後ろ暗い所がないこと〕 清潔な政治 
\\	清潔な人柄 
\\	制限	せいげん	
\\	制限する 
\\	電力制限 
\\	輸入制限 
\\	年齢制限 
\\	速度制限 
\\	産児(さんじ)制限 
\\	聴講者(ちょうこうしゃ)は10人に制限されている 
\\	彼女は高血圧(けつあつ)のため食事が制限されている 
\\	参加に男女の制限は無い 
\\	会員資格(しかく)の制限を緩和(かんわ)した 
\\	彼らは最高価格と最低価格の制限を設けた 
\\	輸入に厳しい制限を加えた 
\\	制限酵素(こうそ) 
\\	制限時間 
\\	制限時間一杯となった 
\\	制限速度 
\\	制限速度は60キロです 
\\	制限速度を越える 
\\	[((文)) 
\\	成功	せいこう	
\\	正式	せいしき	正式な 〔改まった〕
\\	〔公式な〕
\\	〔法にかなった〕
\\	正式な会合 
\\	正式に訪問する 
\\	役所から正式な通達(つうたつ)があった 
\\	正式に結婚する 
\\	性質	せいしつ	
\\	〔生まれつきの気質〕
\\	⇒きしょう(気性) 性質がよい[悪い] 
\\	明るい性質の娘 
\\	もともと彼は性質が傲慢(ごうまん)だ[おとなしい] 
\\	娘は父に似て芸術家肌(はだ)の性質だ 
\\	〔物の本来の特性〕
\\	ある種(たね)の野草(やそう)は病気を治す性質がある 
\\	問題の性質上注意深い扱(あつか)いが必要だ 
\\	精神	せいしん	
\\	成人	せいじん	
\\	せいぜい	精々	
\\	〔できるだけ〕 せいぜい勉強しなさい 
\\	せいぜいおまけしておきますよ 
\\	せいぜいお体をお大事に 
\\	〔多く見積もって,高々〕
\\	少女はせいぜい10歳(じゅっさい)ぐらいだったと思う 
\\	出願者(しゅつがんしゃ)はせいぜい500人ぐらいだろう 
\\	後で地団太(じだんだ)を踏むのがせいぜいだろう 
\\	地団駄
\\	太)を踏(ふ)む 
\\	この機械が持つのはせいぜい3年くらいでしょう 
\\	[((口))
\\	持つ 
\\	せいぜい60点しかあげられない 
\\	成績	せいせき	
\\	〔学業の記録〕
\\	〔試験などの結果〕
\\	〔成績の評価〕((米)) 
\\	((英)) 
\\	試験の成績 
\\	化学ではいい成績を取った 
\\	学校の成績が上がってきた[下がってきた] 
\\	彼は学校の成績がよい[悪い] 
\\	クラスではトップの成績だ 
\\	今年の成績は
\\	が5つだった 
\\	成績をつける 
\\	〔仕事などの成果〕
\\	〔出来栄え〕
\\	営業(えいぎょう)成績を上げる 
\\	彼の営業成績が下がった 
\\	彼は仕事の成績が良かった 
\\	成績証明書 ((米)) 
\\	成績表 〔学校の〕
\\	〔通知表〕
\\	((英)) 
\\	製造	せいぞう	
\\	贅沢	ぜいたく	
\\	〔豪華〕
\\	ぜいたくな 
\\	彼らはぜいたくに暮らしていた 
\\	/((口)) 
\\	彼女はぜいたくに育った 
\\	そんなぜいたくは出来ない 
\\	今日はぜいたくな食事をした 
\\	〔浪費(ろうひ)〕
\\	ぜいたくな 
\\	たまにはぜいたくをするのも楽しいものですね 
\\	たまに【▲偶に】 
\\	別荘(べっそう)なんてぜいたく過ぎる 
\\	「ダイヤのブローチですって
\\	ぜいたく過ぎるわ」 
\\	〔過度の〕 あなたはあれこれぜいたくを言う 
\\	これ以上望(のぞ)むのはぜいたくだ 
\\	ぜいたくを言えばきりがない 
\\	贅沢三昧 ぜいたくざんまいに暮らす 
\\	三昧 
\\	成長	せいちょう	
\\	〔伸び育つこと〕
\\	成長する 
\\	子供は成長が早い 
\\	成長が止まる 
\\	成長して洋服が小さくなった 
\\	少年時代の彼の知的(ちてき)成長は実に目覚ましかった 
\\	〔成熟すること〕 成長する 
\\	成長して立派な医者になった 
\\	彼は成長した子供が3人もある 
\\	成長した犬 
\\	今度のことで彼は人間として成長したと思う 
\\	〔発展〕 高度成長 
\\	経済成長 
\\	成長因子 
\\	成長株 
\\	〔株で〕
\\	〔将来性のある人〕
\\	あの選手は今年の成長株だ 
\\	成長期 
\\	成長曲線 
\\	成長産業 
\\	成長点 〔植物の〕
\\	成長ホルモン 
\\	成長率 
\\	名目成長率は3パーセントだが実質成長率は1パーセントに過ぎない 
\\	成長・生長	せいちょう	
\\	〔伸び育つこと〕
\\	成長する 
\\	子供は成長が早い 
\\	成長が止まる 
\\	成長して洋服が小さくなった 
\\	少年時代の彼の知的成長は実に目覚ましかった 
\\	〔成熟(せいじゅく)すること〕
\\	成長する 
\\	成長して立派な医者になった 
\\	彼は成長した子供が3人もある 
\\	成長した犬 
\\	今度のことで彼は人間として成長したと思う 
\\	〔発展〕 高度成長 
\\	経済成長 
\\	成長因子(いんし) 
\\	成長株 
\\	〔株で〕
\\	〔将来性のある人〕
\\	あの選手は今年の成長株だ 
\\	成長期 
\\	成長曲線 
\\	成長産業 
\\	成長点 〔植物の〕
\\	成長ホルモン 
\\	成長率 
\\	名目(めいもく)成長率は3パーセントだが実質成長率は1パーセントに過ぎない 
\\	制度	せいど	
\\	〔体系的な〕
\\	〔慣習的な〕
\\	〔機構〕
\\	社会[教育]制度 
\\	現行(げんこう)の制度では 
\\	制度を設(もう)ける 
\\	制度を改(あらた)める〔廃止(はいし)する〕 
\\	一夫一婦(いっぷいっぷ)制度 
\\	軍隊の制度 
\\	制度化 
\\	制度化する 
\\	青年	せいねん	
\\	生年月日	せいねんがっぴ	
\\	製品	せいひん	
\\	政府	せいふ	
\\	生物	せいぶつ	
\\	〔総称〕
\\	森の生物たち 
\\	地球上の生物 
\\	生物界(せいぶつかい) 
\\	生物化学 
\\	生物(化学)兵器 
\\	生物学 
\\	生物学的 
\\	生物学上 
\\	生物学者 
\\	生物活性物質 
\\	生物共同体 
\\	生物圏 
\\	生物工学 
\\	生物指標 
\\	生物地球化学 
\\	生物地理学 
\\	生物統計学 
\\	生物濃縮 
\\	生物物理学 
\\	生物兵器 
\\	生命	せいめい	
\\	整理	せいり	
\\	〔乱(みだ)れているものを整えること〕 整理する 
\\	交通整理 
\\	交通整理をする 
\\	残務整理〔解散した会社などの〕 
\\	残務 
\\	残務整理 
\\	残務(ざんむ)整理をする 
\\	行政整理 
\\	区画(くかく)整理 
\\	引き出しを整理する 
\\	資料を整理する 
\\	書類を整理する 
\\	帳簿(ちょうぼ)を整理する 
\\	本を整理する 
\\	警官が見物人を整理した 
\\	あの通りはマラソンのために交通整理が行われている 
\\	みなさんの発言を整理すると次のようになります 
\\	〔無駄なものを処分(しょぶん)すること〕 整理する 
\\	〔不要な人員を減らす〕
\\	要(い)らないものを整理する 
\\	会社の人員(じんいん)整理で彼は解雇(かいこ)された 
\\	身辺(みじか)を整理して田舎(いなか)に引き上げた 
\\	〔借金などを支払うこと〕
\\	整理する 
\\	債務(さいむ)を整理する 
\\	整理会社 
\\	〔破産清算中の〕
\\	整理株 
\\	整理券 
\\	整理公債(こうさい) 
\\	公債 
\\	整理だんす 
\\	整理番号 
\\	整理部 
\\	整理部員 ((米)) 
\\	((英)) 
\\	咳	せき	
\\	石炭	せきたん	
\\	責任	せきにん	
\\	石油	せきゆ	
\\	世間	せけん	
\\	説	せつ	
\\	〔意見,主張〕
\\	自分の説に固執(こしつ)する 
\\	その点はお説の通りです 
\\	〔学説〕
\\	説を立てる 
\\	〔風評〕
\\	((米)) 
\\	((英)) 
\\	彼女は離婚したという説がある 
\\	積極的	せっきょくてき	
\\	〔活動的な (に) 
\\	積極的な態度を取る 
\\	積極的な男 
\\	企画に積極的に参加する 
\\	設計	せっけい	
\\	〔建物などの〕
\\	〔形・構造・外観などの〕
\\	設計する 
\\	教会を設計する 
\\	庭を設計する 
\\	車の設計 
\\	この駅の設計はよくない 
\\	〔人生・生活などの〕
\\	元旦(がんたん)に一年の生活設計をすることにしている 
\\	設計技師 
\\	〔建築〕
\\	設計基準 
\\	((通例複数形 
\\	設計支援システム 
\\	((略
\\	設計者 
\\	設計(仕様)書 
\\	[((口)) 
\\	((通例複数形)) 設計図 
\\	絶対	ぜったい	
\\	設備	せつび	〔備品・機械など〕
\\	集合的に); 〔施設〕
\\	〔宿泊・収容設備〕
\\	((米))ではしばしば複数形で) 家庭用暖房設備(かていようだんぼうせつび) 
\\	衛生設備 
\\	娯楽設備 
\\	宿泊(しゅくはく)設備 
\\	生産設備 
\\	近代的な設備を施(ほどこ)したホテル 
\\	この病院は設備がよい[悪い] 
\\	湯沸(ゆわ)かし器を設備する 
\\	この辺は下水設備が整(ととの)っている 
\\	/((英)) 
\\	設備投資 
\\	設備費 
\\	絶滅	ぜつめつ	
\\	節約	せつやく	
\\	電気を節約する 
\\	厳しい節約をする 
\\	改まった表現)/〔やりくりを上手に〕
\\	これは大いに時間[労力]の節約になる 
\\	経費の節約 
\\	それは経費の節約になる 
\\	節約家 
\\	攻める	せめる	
\\	敵を攻める 
\\	激しく攻めて要塞(ようさい)を占領(せんりょう)する 
\\	責める	せめる	
\\	〔非難する〕
\\	〔しかる〕
\\	は口に出して責めることだが,
\\	は相手に責任があると思うこと) 人の職務怠慢を責める 
\\	我が身を責める 
\\	人の違約(いやく)を責める
\\	違約  
\\	〔厳しく要求する〕
\\	〔ねだる〕
\\	金を払えと責められる 
\\	息子に責められて海辺(うみべ)へ連れて行った 
\\	〔苦痛を与える〕
\\	善	ぜん	
\\	善と悪をはっきり区別する 
\\	善は急げ 
\\	全	ぜん	
\\	全員	ぜんいん	
\\	専攻	せんこう	
\\	((英))
\\	((米))
\\	専攻する 
\\	[((米)) 
\\	((英)) 
\\	彼は大学では何を専攻しましたか ((米)) 
\\	/((英)) 
\\	彼は歴史を専攻している ((米)) 
\\	/((英)) 
\\	専攻科 ((米)) 
\\	((英)) 
\\	専攻科目(かもく) 
\\	((米))
\\	専攻分野 
\\	全国	ぜんこく	
\\	洗剤	せんざい	
\\	合成洗剤 
\\	中性洗剤 
\\	先日	せんじつ	
\\	先日の約束をお忘れなく 
\\	先日来彼に会っていない 
\\	先日来 (せんじつらい) 
\\	前者	ぜんしゃ	
\\	選手	せんしゅ	
\\	全身	ぜんしん	
\\	全身に火傷(やけど)を負う 
\\	[((文)) 
\\	全身が痛む 
\\	全身の力を込めて引っ張った 
\\	彼は全身これ革命家だった 
\\	全身運動 全身運動をする 
\\	全身全霊(ぜんれい) 彼は医学の研究に全身全霊を傾(かたむ)けた 
\\	全身像 
\\	全身不随 
\\	彼は全身不随(ふずい)だ 
\\	全身麻酔 
\\	全身麻酔(ますい)をかける 
\\	前進	ぜんしん	
\\	前進する 
\\	前進!〔号令〕 
\\	彼らは国境を越えて前進した 
\\	人類(じんるい)は病気の絶滅(ぜつめつ)に向かって少しずつ前進している 
\\	今年になって我々の運動は大きく前進した 
\\	前進基地 
\\	全体	ぜんたい	
\\	〔物事の全部〕
\\	((文)) 
\\	全体の 
\\	その提案全体を却下(きゃっか)した 
\\	町全体に悪臭(あくしゅう)が立ち込めた 
\\	その旅行全体にわたって 
\\	国全体にわたって 
\\	クラス全体から見るとこの子は決して出来が悪くない 
\\	全体として見ればこの催しは成功だった 
\\	旅費は全体で30万円かかった 
\\	この地方の人々は全体 (的) に勤勉(きんべん)だ 
\\	〔いったい,そもそも〕 全体あなたは公徳心が欠けています 
\\	一体全体 
\\	一体全体どうしたというんだ 
\\	一体全体彼は何が望みなのか私には分からん 
\\	[((俗))
\\	全体会議 
\\	[((文)) 
\\	全体集合 〔数学で〕
\\	全体主義 
\\	全体主義の 
\\	全体主義国家 
\\	選択	せんたく	〔選ぶこと〕
\\	〔選抜〕
\\	選択する 
\\	選択に迷う 
\\	選択を誤る 
\\	これ以上選択の余地(よち)はない 
\\	選択は私の自由だ 
\\	二人とも優秀(ゆうしゅう)なので代表としての選択に苦しんだ 
\\	会議の日時の選択はあなたに任せます 
\\	選択科目 ((米)) 
\\	((英)) 
\\	選択権 
\\	選択権を放棄(ほうき)する 
\\	選択肢(し) 
\\	選択売買権 
\\	宣伝	せんでん	
\\	象	ぞう	
\\	騒音	そうおん	
\\	増加	ぞうか	
\\	操作	そうさ	
\\	〔機械などの〕
\\	操作する 
\\	機械を操作する 
\\	遠隔(えんかく)操作 
\\	この機械のギア操作は複雑だ 
\\	〔資金などのやりくり〕 たくみに株の操作をして巨額(きょがく)の富(とみ)を得た 
\\	金融(きんゆう)[市場]操作 
\\	/〔巧妙な〕
\\	想像	そうぞう	
\\	想像する 
\\	〔仮定する〕
\\	〔推測する〕
\\	想像上の 
\\	想像力のある 
\\	想像をたくましくする 
\\	逞しくする 
\\	あとはご想像にお任(まか)せします 
\\	想像を絶(ぜっ)する 
\\	不死鳥(ふしちょう)は想像の鳥にすぎない 
\\	想像し得るあらゆる方法をやってみた 
\\	想像もできない窮地(きゅうち)に追い込まれた 
\\	私の想像が当たった[外れた] 
\\	どんなことが起こるか想像もつきません 
\\	自分が大富豪(ふごう)だと想像してみた 
\\	あの憶病(おくびょう)な娘が馬に乗っているのを想像できますか 
\\	きみが彼の立場に居ると想像してみたまえ 
\\	想像画 子供の描いた氷河(ひょうが)時代の想像画 
\\	想像妊娠 
\\	想像力 想像力の豊(ゆた)かな[乏しい]人 
\\	想像力を働かせる 
\\	相続	そうぞく	
\\	〔継承〕
\\	相続する 
\\	彼の未亡人(みぼうじん)が田畑を相続した 
\\	彼女は莫大(ばくだい)な資産(しさん)を相続した 
\\	長男が父の爵位(しゃくい)を相続した 
\\	彼女は叔父(おじ)の跡目(あとめ)を相続して取締役(とりしまるやく)に就任(しゅうにん)した 
\\	相続争い 
\\	相続財産 
\\	相続税 
\\	相続人 〔男〕
\\	〔女〕
\\	通例の表現,および法律では,女相続人も
\\	を用いる); 〔後継者〕
\\	彼は叔母(おば)の財産の相続人となった 
\\	装置	そうち	
\\	は集合名詞で,設備された機器
\\	は機械的な仕掛け,
\\	は特定の目的をもった一組の機械で,複数形はまれ) 防火(ぼうか)装置 
\\	無電装置 
\\	安全装置 
\\	暖房装置 
\\	舞台装置 
\\	冷房(れいぼう)装置 
\\	実験室にはあらゆる器具が装置してある 
\\	建物は出来たが器具の装置はまだだ 
\\	相当	そうとう	
\\	〔釣り合うこと〕
\\	それは相当の処置(しょち)であった 
\\	彼に相当した仕事だ 
\\	彼は収入に相当した生活をしている 
\\	〔当てはまること〕
\\	5万円相当の贈(おく)り物 
\\	彼の罪は死に相当する 
\\	この会議は我が国の国会に相当する 
\\	1ドルは今日本円でいくらに相当しますか 
\\	その言葉に相当する英語が見当たらない 
\\	〔かなりの程度〕 彼は相当な暮らしをしている 
\\	相当の金額だ 
\\	相当の出来栄(ば)えだった 
\\	ここから相当の距離がある 
\\	道は相当に険(けわ)しい 
\\	それは相当難しそうだ 
\\	彼はそんなことを言ったのか,相当なものだね 
\\	速度	そくど	
\\	両者はほとんど同義だが,科学用語としては後者を用いるのが普通) 最高速度 
\\	初[終]速度 
\\	法定速度 
\\	制限速度を守る 
\\	猛烈(もうれつ)な速度で 
\\	車は時速80キロの速度で通過した 
\\	交通渋滞(じゅうたい)を抜け出て車は速度を増した 
\\	速度落とせ〔掲示〕 
\\	船の速度ががくっと落ちた 
\\	今,飛行機はどれくらいの速度で飛んでいるのか 
\\	速度の遅い車を追い越した 
\\	この速度では9時までに着かないね 
\\	彼は読む速度が速い 
\\	彼は飲む速度が速い 
\\	速度違反 
\\	⇒スピードいはん(スピード違反) 速度計 
\\	速度制限 
\\	速度制限をする 
\\	底	そこ	
\\	そこで	そこで	
\\	〔それゆえ〕
\\	そこでぜひ君に頼みたいということになったのだ 
\\	〔さて〕
\\	そこで次の問題に移ろう 
\\	組織	そしき	
\\	〔統一体,結成されたもの〕
\\	〔結成すること〕
\\	組織する 
\\	組織替えをする 
\\	麻薬密売(まやくみつばい)組織 
\\	組織のボス 
\\	スミス氏が内閣を組織した 
\\	労働組合を組織する 
\\	〔構成〕 この図は我々の協会の組織を示したものです 
\\	理事会(りじかい)は11名で組織されている 
\\	〔系統〕 組織的な (に) 
\\	組織化する 
\\	組織的研究 
\\	彼のやり方は組織的だ 
\\	彼らはそれらの事件を組織的に検討している 
\\	彼の陳述(ちんじゅつ)は組織立っていない 
\\	〔生物の〕
\\	細胞(さいぼう)組織 
\\	神経[筋肉]組織 
\\	組織委員会 
\\	組織家 
\\	組織工学 
\\	組織培養 〔生物〕
\\	組織バンク 
\\	組織票 
\\	組織暴力 
\\	組織網 後援会の組織網 
\\	組織力 
\\	〔組織する力〕 彼には組織力がある 
\\	〔結成した力〕 あの組合は大きな組織力を誇っている 
\\	組織労働者 
\\	は集合名詞) 未組織労働者 
\\	そして	そして	
\\	〔それから〕
\\	そしてどうしました? 
\\	そして次は私の番でした 
\\	そしてなお彼らは要求を続けた 
\\	そして我々は目的を果たしたというわけだ 
\\	育つ	そだつ	
\\	そっくり	そっくり	
\\	〔全部〕 宝石(ほうせき)をそっくり盗(ぬす)まれた 
\\	この魚は骨ごとそっくり食べられる 
\\	〔そのまま〕 そっくりしまっておきなさい 
\\	〔酷似(こくじ)〕
\\	父親そっくりだ 
\\	/((口)) 
\\	肖像画(しょうぞうが)は本人そっくりに出来ている 
\\	モンローのそっくりさん 
\\	/((口))
\\	そっと	そっと	
\\	〔音を立てないように〕
\\	病人を起こさないようにそっと歩いた 
\\	〔壊れないように〕
\\	壊(こわ)れないようにそっと降ろしてください 
\\	〔こっそり〕
\\	そっと父の机の引き出しを開けた 
\\	彼はそっと私の部屋に入ってきた 
\\	彼らはそっと事を運んだ 
\\	/((口))
\\	事を運ぶ 
\\	〔触れないでおく〕 彼女をそっとしておいてやりましょう 
\\	この問題はそっとしておくことにした 
\\	藪蛇(やぶへび)にならないようにそっとしておきなさい 
\\	藪蛇  
\\	袖	そで	
\\	備える	そなえる	
\\	〔準備する〕
\\	将来[万一]に備える 
\\	地震に備える 
\\	老後に備える 
\\	この冬の灯油(とうゆ)不足に備えていますか 
\\	コンテストに備えて毎晩スピーチの練習をしている 
\\	〔そろえておく〕
\\	はある目的のために必要なものを備え付ける) このホテルは各室(かくしつ)に冷蔵庫が備えてある 
\\	この事務所には机といすが備えてある 
\\	敵軍は強大な武器を備えていた 
\\	〔ある才能や性質をもつ〕 彼女は法律の知識を十分備えている 
\\	彼は日本人の長所と短所を遺憾(いかん)なく備えている 
\\	彼は声楽(せいがく)の才を備えている 
\\	彼は教師の資格(しかく)を十分(じゅうぶん)備えている 
\\	そのうえ	そのうえ	
\\	その上あわて者ときている 
\\	その上悪いことに[不幸にも]子供が死んでしまった 
\\	その上まずいことに大雨になった 
\\	その上に賞金(しょうきん)50万円をもらった 
\\	彼は大酒飲(おおざけの)みでその上ギャンブル狂(きょう)だ 
\\	はいつも文末に置く)
\\	そのうち	そのうち	
\\	〔近日(きんじつ)〕 
\\	そのうちお伺(うかが)いします 
\\	〔他日〕
\\	そのうち貴国(きこく)を訪問したいと思います 
\\	〔やがて〕 そのうち雨がやんだ 
\\	[((文))
\\	そのうちに帰って来ますよ 
\\	〔その間に〕 そのうちに彼のほうでも気が変わった 
\\	〔該当(がいとう)する事物・人の中で〕 10人の登山者があったが,そのうち3人は女性だった 
\\	落伍者(らくごしゃ)は8名いたが,彼もそのうちの1人だった 
\\	そのまま	そのまま	
\\	〔元通りの状態〕 両親は亡くなった兄の部屋を長年そのままにしている 
\\	父の蔵書(ぞうしょ)はそのまま残(のこ)されています 
\\	その文章はそのままでよい 
\\	はい,そのまま動かないで〔写真を撮るときなど〕 
\\	そのまま練習を続けなさい 
\\	言われた事をそのままやったまでです 
\\	〔次の動作に移る様子〕 彼はアメリカへ行ってそのまま帰って来なかった 
\\	他に何もしないでそのまま寝てしまった 
\\	〔そっくり〕 母親そのままの明るい子だ 
\\	蕎麦	そば	
\\	粗末	そまつ	
\\	〔上等でない様子〕⇒おそまつ(御粗末) 粗末な 〔みすぼらしい〕
\\	〔粗悪な〕
\\	〔質素な〕
\\	粗末な家に住んでいる 
\\	粗末な紙 
\\	作りが粗末な家具 
\\	彼らは粗末な食べ物を食べていた 
\\	はろくな食べ物がないこと) 粗末なお菓子ですが,どうぞお召し上がりください 
\\	英語では謙遜して「粗末な」とは言わない) 粗末な物ですが,どうぞ 
\\	/((口)) 
\\	〔おろそかにする様子〕 親を粗末にする 
\\	近ごろの子供は物を粗末にする 〔無駄にする〕
\\	/〔乱暴に扱う〕
\\	命を粗末にするな 
\\	店員に粗末にされて怒っている 
\\	それぞれ	それぞれ	
\\	それぞれの 
\\	少年たちはそれぞれ犬を飼(か)っている 
\\	子供たちはそれぞれの席についた 
\\	最近では
\\	を
\\	で受けることが多い) 人はそれぞれの長所がある 
\\	トムと太郎とフィリップはそれぞれ英語,日本語,フランス語を話した 
\\	彼らはそれぞれの責任を極(きわ)めて慎重(しんちょう)に果たしている 
\\	は通例あとに複数形の名詞を伴う)
\\	それでも	それでも	
\\	彼女は優しくて親切だが,それでも私は彼女が好きになれない
\\	だれもその絵をほめないがそれでも立派だと私は思う
\\	それとも	それとも	
\\	コーヒーですか,それとも紅茶にしますか 
\\	引き受けるべきか,それとも断(ことわ)るべきか分からない 
\\	そろう	揃う	
\\	〔一か所に集まる〕 これで全員そろった 
\\	そろって先生のところへ行った 
\\	この図書館には生物関係の本がそろっている 
\\	あの店には子供のものが何でもそろっている 
\\	〔全部整(ととの)う〕 これでディナーセットが全部そろう 
\\	美しくて賢(かしこ)くて健康で,三拍子(さんびょうし)そろった女性だ 
\\	三拍子揃う 
\\	客の数だけコップがそろわない 
\\	この百科事典はそろっていない 
\\	条件がそろうのを待っていたらいつまでも職につけない 
\\	〔同じ状態になる〕 真珠(しんじゅ)の粒(つぶ)がそろっていない 
\\	そろいもそろってぼんくらばかりだ 
\\	揃いも揃って 
\\	盆暗=ぼんくら 
\\	全員の足並(あしな)みがそろうと行進(こうしん)が美しく見える 
\\	各組合(かくくみあい)の足並みがそろわない 
\\	足並みが揃わない 
\\	左右がそろっていない手袋をはめていた 
\\	この辺は道幅(みちはば)がそろっている 
\\	そろえる	そろえる	
\\	〔一か所に集める〕 若い者を10人そろえる必要がある 
\\	あそこはしつけのよい店員をそろえている 
\\	躾(しつけ) 
\\	あの店は一流の品物をそろえている 
\\	〔全部を備える〕 良い包丁をそろえる 
\\	バッハの
\\	をそろえたい 
\\	〔きちんと並べる〕 靴はそろえて下駄(げた)箱に入れなさい 
\\	道具はすぐ使えるようにそろえておきなさい 
\\	〔同じ状態にする〕 木の高さをそろえる 
\\	大きさをそろえて人参(にんじん)を切りなさい 
\\	彼らは口をそろえて彼を知らない人だと言った 
\\	損	そん	
\\	損害	そんがい	
\\	〔損傷〕
\\	〔損失〕
\\	爆撃(ばくげき)[洪水(こうずい)]による損害 
\\	2千万円の損害 
\\	損害を与える 
\\	大損害を受ける 
\\	損害を償(つぐな)う 
\\	そのために約5千万円の損害が出た 
\\	台風で米作(べいさく)に大損害を受けた 
\\	損害額 
\\	損害証明 
\\	損害賠償(ばいしょう) ⇒詳細を見る 損害保険 
\\	〔財産損害保険〕
\\	尊敬	そんけい	
\\	存在	そんざい	
\\	尊重	そんちょう	
\\	田	た	
\\	他	た	
\\	ほか =【外・▲他】 
\\	〔ほかの人・事・物〕⇒ほか 他を利する 
\\	他にもらしてはいけない 
\\	彼の陶芸(とうげい)の術は他の追随(ついずい)を許(ゆる)さない 
\\	〔もう一方のもの〕
\\	に対する残りが一つならば
\\	二つ以上ならば
\\	〔残りのもの〕
\\	一つは大きかったが,他は小さかった 
\\	他は推(お)して知るべし 
\\	対	たい	
\\	〔対等〕
\\	対で勝負(しょうぶ)する 
\\	〔数の対照(たいしょう)〕
\\	早稲田が5対3で勝った 
\\	5-3は
\\	と読む) 法案は380対120で否決(ひけつ)された 
\\	ジンとライムを1対3の割合(わりあい)で混ぜる 
\\	〔勝負の組み合わせ〕 巨人対阪神戦 
\\	スミス対ジョーンズ訴訟(そしょう)事件 
\\	は
\\	の略)
\\	大	だい	
\\	題	だい	
\\	〔表題〕
\\	〔題目〕
\\	彼は自分の小説に『緑の大地』という題をつけた 
\\	『スターウォーズ』という題の映画 
\\	レポート[討論(とうろん)]の題を出す 
\\	彼は「人間と科学」という題で講演した 
\\	〔問題〕 数学のテストで2題しか解けなかった 
\\	体育	たいいく	
\\	体温	たいおん	
\\	大会	たいかい	
\\	大気	たいき	
\\	代金	だいきん	
\\	〔買った物に払う〕
\\	法外な代金を請求(せいきゅう)された 
\\	代金は前払いでお願いします 
\\	車の代金として300万円払った 
\\	現物は代金と引き換(か)えにお渡しいたします 
\\	彼は代金を取り立てに出掛けた 
\\	代金引替(ひきかえ) 
\\	((略
\\	代金引替郵便で送る 
\\	退屈	たいくつ	退屈な 〔冗長(じょうちょう)で〕
\\	〔内容がなくて〕
\\	退屈する 
\\	彼の話はいつも退屈だ 
\\	彼の講演は退屈至極(しごく)だった 
\\	退屈しのぎに散歩に出掛けた 
\\	退屈な人 
\\	滞在	たいざい	
\\	((文)) 
\\	滞在する 
\\	英国滞在中に 
\\	スイスのホテルに2週間滞在した後,イタリアに向かった 
\\	ロンドンには長く[短期間]滞在します 
\\	叔父の所に滞在しています 
\\	いつまでご滞在ですか 
\\	滞在期間 
\\	滞在期間は延長できません 
\\	滞在客 
\\	滞在地 
\\	((文)) 
\\	滞在費 滞在費はこちらで持ちます 
\\	/〔宿泊費を〕
\\	大使	たいし	
\\	大した	たいした	
\\	〔程度がはなはだしい〕 大した人出でした 
\\	大した心臓だ 
\\	小学生にしては大したものだ 
\\	あの人は大した芸術家だ[ではない] 
\\	5千万円貯金したって
\\	大したもんだね 
\\	〔「大した…ではない」の形で,それほどではない〕 金を落としたが大した額ではない 
\\	大した損害ではなかった 
\\	それは大した違いにはならない 
\\	それは大したことではない 
\\	彼の病気は大したことはない 
\\	体重	たいじゅう	
\\	対象	たいしょう	
\\	哀(あわ)れみ[羨望(せんぼう)]の対象 
\\	課税(かぜい)対象の財産 
\\	研究の対象 
\\	彼女の小説は厳しい批評の対象となった 
\\	20代の女性を対象にアンケート調査を行う 
\\	この献立(こんだて)はお年寄りを対象としたものです 
\\	対照	たいしょう	
\\	〔対比〕
\\	対照する 
\\	白い壁が森林と美しい対照をなしている 
\\	この二つは好(こう)対照をなす 
\\	彼は私の夫とはいい対照だ 
\\	ていねいな言葉遣いとは対照的に彼女の態度はひどく無作法だった 
\\	〔比較〕
\\	対照する 
\\	訳文と原文を対照する 
\\	対照言語学 
\\	対照実験 
\\	対照分析 
\\	大臣	だいじん	
\\	対する	たいする	
\\	〔向かい合う〕 学校と教会が道をはさんで対している 
\\	挟(はさ)む 
\\	〔ある対象に向かう〕 質問に対する答え 
\\	それに対して全く無関心だった 
\\	友人に対する思いやり 
\\	あの方の学識(がくしき)に対して十分敬意を抱(いだ)いております 
\\	あの方 
\\	隣国が我が国に対して宣戦(せんせん)を布告(ふこく)してきた 
\\	あなたに対して腹を立てているのではありません 
\\	子供たちは家庭教師に対して反抗的な態度を示した 
\\	児童(じどう)心理学に対して非常に興味をもっている 
\\	〔応対する〕⇒おうたい(応対),せっする(接する)
\\	〔対抗する〕 ベテラン力士(りきし)に対するは新進気鋭(きえい)の力士である 
\\	〔対照(やいしょう)となる〕 「大」に対する語は何か 
\\	彼が理性的であるのに対して,彼女は情緒的(じょうちょてき)だ 
\\	先月の交通事故10件に対し,今月は18件だった 
\\	大戦	たいせん	
\\	態度	たいど	
\\	〔素振り,様子〕
\\	単数形のみで); 
\\	((英)) 
\\	警官はその男の態度が怪(あや)しいと思った 
\\	彼らの態度は実に気持ちがいい 
\\	高慢な態度 
\\	態度が大きいぞ 
\\	彼の態度はそわそわしていた 
\\	授業中の態度が悪い 〔いたずらをする〕
\\	/〔だらけた,注意散漫な〕
\\	〔物事に対する心構え〕
\\	彼らに対して反抗的(はんこうてき)な[厳然(げんぜん)たる]態度に出た 
\\	それを聞いて彼女は態度をがらりと変えた 
\\	この件に対する態度を早急(さっきゅう)に決めてもらいたい 
\\	君の態度がはっきりしないので困る 
\\	日本政府はアメリカ政府に対して強い態度を取った 
\\	大統領	だいとうりょう	
\\	大半	たいはん	
\\	住民の大半は川の西岸(せいがん)に住んでいる 
\\	ここの労働者の大半は地方出身者(しゅっしんしゃ)である 
\\	仕事の大半は片付いた 
\\	工事は大半完成した 
\\	代表	だいひょう	
\\	〔多数の人の代わりをすること〕
\\	〔人〕
\\	代表する 
\\	地域(ちいき)[比例(ひれい)]代表制 
\\	会議に2名の代表を送った 
\\	彼はクラスの代表として自治会(じちかい)に出た 
\\	グループを代表して言わせてください 
\\	〔全体の性質・役目などを表すこと〕 代表的 
\\	我が校の代表的な学生 
\\	代表的な例 
\\	この絵はピカソの初期の作品の代表的なものです 
\\	彼の提案は選挙人の希望を代表している 
\\	代表作 
\\	代表者 
\\	〔政治的な会議などに派遣する〕
\\	〔代理役,代議員〕
\\	代表社員 
\\	代表質問 総理大臣に野党(やとう)各党から代表質問が行われた 
\\	代表団 
\\	代表取締役(とりしまりやく) 
\\	代表番号 〔電話の〕
\\	大部分	だいぶぶん	
\\	逮捕	たいほ	
\\	題名	だいめい	
\\	太陽	たいよう	
\\	平ら	たいら	
\\	代理	だいり	
\\	〔取引などの〕
\\	〔人〕⇒だいりにん(代理人) 代理の 
\\	代理 (を) する 
\\	校長[社長]代理 
\\	議長代理 
\\	彼は1週間先生の代理をつとめた 
\\	大統領の入院中副大統領が代理をつとめた 
\\	父の代理で[として]会議に出席した 
\\	彼の代理で投票した 
\\	代理業 
\\	代理業者 
\\	代理出産 
\\	代理署名 
\\	知事は民有地使用継続のための代理署名を拒否した 
\\	代理戦争 
\\	代理大使 
\\	((複
\\	代理店 
\\	広告[保険]代理店 
\\	代理店手数料 
\\	代理人 
\\	投票・株主総会などで委任された) 
\\	ある職務を代行する); 〔取引などの〕
\\	〔法廷の〕
\\	代理人となる 
\\	代理人を立てて応訴することにした 
\\	代理母(だいりぼ) 
\\	代理販売 
\\	大陸	たいりく	
\\	倒す	たおす	
\\	〔横にする〕 はしごを倒して置く 
\\	昨夜(さくや)の嵐がたくさんの木を倒した 
\\	うっかりして花瓶を倒した 
\\	〔負(ま)かす〕
\\	彼は強敵(きょうてき)を倒した 
\\	〔滅(ほろ)ぼす〕
\\	現政府は倒すべきだ 
\\	だが	だが	
\\	互い	たがい	
\\	互いの利益(りえき)[理解] 
\\	互いに罪をなすり合った 
\\	少年たちは互いに顔を見合わせた 
\\	は二者に,
\\	は三者以上の間に用いると言われるが,三者以上の間でも
\\	を使うことがある) 三者は互いに助け合っている 
\\	互い先〔囲碁で〕 互い先で打つ 
\\	高める	たかめる	
\\	〔高くする〕
\\	〔程度を強める〕
\\	士気(しき)を高める 
\\	原色の大胆(だいたん)な使い方が効果を高めている 
\\	国民の文化水準を高める 
\\	女性の地位を高める 
\\	宝	たから	
\\	宅	たく	
\\	炊く	たく	
\\	ご飯を炊く 
\\	ご飯は炊けたの 
\\	焚く	たく	
\\	〔燃料などをもやす〕
\\	〔火をつける〕
\\	石炭(せきたん)をたく 
\\	線香(せんこう)をたく 
\\	ストーブをたく 
\\	火をたく 
\\	〔ふろをわかす〕 ふろをたく 
\\	〔フラッシュなどを〕 フラッシュ[ストロボ]をたく 
\\	/((口)) 
\\	だけど	だけど	
\\	[接]「だけれど」のくだけた言い方。→だけれども ーーー だけれども [接]前に述べたことと相反することをいうときに用いる。そうではあるが。だけど。だけども。だけれど。「話はよく分かった。―承諾するわけにはいかない」
\\	たしか	たしか	
\\	〔確実〕 確かな 
\\	と
\\	はほぼ同意だが,
\\	の方が客観的で確実性が高いことが多い) 確かな証拠 
\\	彼が早い汽車に乗ったのは確かだ 
\\	その病気の確かな治療法(ちりょうほう) 
\\	この問題が試験に出るのは確かだ 
\\	確かにお便りを拝見しました 
\\	彼は確かに頭はよいが,あまりに利己主義だ 
\\	確かに彼はそれを知っている 
\\	確かに一万円札はくれたが,にせ札だった 
\\	〔信頼しうる様子〕 確かな 
\\	この情報は確かな筋からのものだ 
\\	彼女の料理の腕は確かだ 
\\	あの医者の腕は確かだ 
\\	〔健全・正常な様子〕 確かな 
\\	頭はまだ確かだが目と耳が衰(おとろ)えてきた 
\\	君,気は確かかね 
\\	〔たぶん〕 入荷(にゅうか)は確か明日になるはずです 
\\	確か5月末のことだった 
\\	確かめる	たしかめる	
\\	多少	たしょう	
\\	少し (すこし)) 
\\	〔数の大小〕
\\	〔量の〕
\\	参加者の多少によって予定は変わるかも知れない 
\\	ごみは多少にかかわらず自分で始末(しまつ)してください 
\\	この際費用の多少は問題ではない 
\\	目方(めかた)の多少に関係なく無料で配達いたします 
\\	〔いくらか〕
\\	これは君には多少難しいかもしれない 
\\	フランス語を多少話せます 
\\	私だって多少の貯金はある 
\\	多少医学の心得(こころえ)がある 
\\	彼は多少は名の知れた画家だ 
\\	世の中を多少は知っている 
\\	彼は多少いらいらしているようだった 
\\	助かる	たすかる	
\\	〔命拾(いのちびろ)いする〕
\\	〔救助される〕
\\	バスが転覆(てんぷく)したが全員命は助かった 
\\	少年は川に落ちたが助かった 
\\	彼は家族の手厚(てあつ)い看病(かんびょう)で助かった 
\\	彼は助かる見込みがほとんどない 
\\	〔被害を免れる〕 留守中に泥棒が入ったが宝石(ほうせき)は助かった 
\\	〔費用・苦労などが少なくてすむ〕 費用が助かった 
\\	この地方は物価が安いので助かっている 
\\	この辞書でおおいに助かっている 
\\	助ける	たすける	
\\	ただ	ただ	
\\	〔普通〕 ただの 
\\	ただの人 
\\	飲んでみたらただの砂糖水だった 
\\	二人はただの仲ではないらしい 
\\	〔何でもないこと,そのまま〕 生意気(なまいき)言うとただでは済まないぞ 
\\	〔無料〕 ただの[で] 
\\	配達料はただです 
\\	見本はただで進呈(しんてい)いたします 
\\	は請求しただけでの意) ただで差し上げます 
\\	ただより高いものはない 
\\	転んでもただでは起きない 
\\	ただ同然で 
\\	この古いいすはただ同然で手に入れた 
\\	只	ただ	
\\	〔普通〕 ただの 
\\	ただの人 
\\	飲んでみたらただの砂糖水だった 
\\	二人はただの仲ではないらしい 
\\	〔何でもないこと,そのまま〕 生意気(なまいき)言うとただでは済まないぞ 
\\	〔無料〕 ただの[で] 
\\	配達料はただです 
\\	見本はただで進呈(しんてい)いたします 
\\	は請求しただけでの意) ただで差し上げます 
\\	ただより高いものはない 
\\	転(ころ)んでもただでは起きない 
\\	ただ同然で 
\\	この古いいすはただ同然で手に入れた 
\\	ただ【▲唯】 
\\	〔ひたすら,もっぱら〕
\\	ただ笑うばかりだった 
\\	ただ命令に従うほかなかった 
\\	ただ言われたとおりにすればよい 
\\	〔単に〕
\\	その手紙はただちらと見ただけです 
\\	ちら(っ)と 
\\	理由はただそれだけです 
\\	ただ来たいから来たのです 
\\	それはただ憶測(おくそく)にすぎない 
\\	ただ一人で難事を成しとげた 
\\	戦い	たたかい	
\\	戦う	たたかう	
\\	(困難と)
\\	(争う)
\\	(競技する)
\\	〔戦争をする〕
\\	〔戦闘をする〕
\\	当時日本はロシアと戦っていた 
\\	〔競争する〕
\\	田中と鈴木は優勝を懸(か)けて戦った 
\\	正々堂々と戦おう 
\\	!たたく	たたく	
\\	(続けざまに)
\\	(拳などで)
\\	(コツコツと)
\\	(軽く)
\\	(ドンドンと)
\\	(ドンと)
\\	(平手で)
\\	(非難する)
\\	〔英〕
\\	(攻撃する)
\\	(値切る)
\\	たたく
\\	叩く・×敲く】 
\\	〔打つ〕
\\	〔軽く〕
\\	太鼓をたたく 
\\	じゅうたんをたたいてほこりを出した 
\\	あられ(霰)が窓をたたいている 
\\	だれかが戸をたたいている 
\\	ドアを3度たたく音が聞こえた 
\\	だれかが私の肩をたたいた 
\\	母親はいたずら息子のお尻(しり)をたたいた 
\\	父親はその子の頬(ほお)をぴしゃりとたたいた 
\\	新聞紙を折って蝿(はえ)をたたいた 
\\	が入ると,蝿を目掛けてたたいたが当たらなかったという意味になる) 彼は机をたたきながら演説した 
\\	聴衆(ちょうしゅう)はいっせいに手をたたいた 
\\	〔打って柔らかくする〕 肉をたたく 
\\	〔非難する〕
\\	((文)) 
\\	彼女の不祥事(ふしょうじ)を新聞がたたいた 
\\	彼は過(あやま)ちをたたかれても当然だ 
\\	〔打診(だしん)する〕 
\\	上司の意見をたたいたほうがよい 
\\	〔値切る〕
\\	000円までたたいてやっと買うことにした 
\\	かばんをうんとたたいて買った 
\\	うんと 
\\	〔多く「…口をたたく」の形で,さかんに言うの意〕 減(へ)らず口をたたく 
\\	門をたたく 彼は老師の門をたたいた 
\\	たたけばほこりが出る 彼もたたけばほこりが出る体だ 
\\	はたく
\\	叩く】 
\\	〔ちりをはたく〕
\\	いすの背をはたく 
\\	じゅうたんをはたく 
\\	〔平手などで打つ〕 彼は怒って私の手から本をはたき落とした 
\\	はえをはたく 
\\	〔使い果たす〕 有り金をはたいてコンピュータを買った 
\\	有り金 
\\	身代(しんだい)をはたいてしまった 
\\	身代 
\\	財布をはたいて宝(たから)くじを買った 
\\	直ちに	ただちに	
\\	⇒すぐ(直ぐ)1 ただちにファックスを送った 
\\	病人はただちに手術を必要とする 
\\	彼はただちに決意した 
\\	〔そのまま〕 努力がただちに成功につながるとは限らない 
\\	畳む	たたむ	
\\	〔折り重ねる〕
\\	シーツをきちんと畳んだ 
\\	テントを畳む 
\\	傘を畳む 
\\	〔商売などをやめる〕 店を畳む 
\\	彼は田舎の家を畳んで上京した 
\\	〔内に秘める〕 彼女はこの悲しい出来事を長く胸に畳んでおいた 
\\	〔殺す〕 畳んでしまえ ((俗))
\\	立ち上がる	たちあがる	
\\	〔起立する〕
\\	老人に席を譲(ゆず)るために立ち上がった 
\\	彼はよろよろと立ち上がった 
\\	〔行動を開始する〕 彼らは飢(う)えた人を救(すく)うために立ち上がった 
\\	今こそ武器を取って立ち上がるときだ 
\\	〔回復する〕 彼は一人息子を亡くした悲しみから立ち上がれなかった 
\\	立場	たちば	
\\	〔置かれた状況〕
\\	我が国は今難しい[微妙(びみょう)な]立場にある 
\\	君の不用意な言葉で僕はまずい立場に立たされてしまった 
\\	私はこの件についてははっきりしたことが言えない立場にある 
\\	あなたのような立場にはなりたくない 
\\	私があなたの立場ならその金は取らないね 
\\	〔見地,観点〕
\\	異なる立場から彼はその件を論じた 
\\	彼は立場を明らかにしなかった 
\\	この法案は私の政治的立場から反対しなければならない 
\\	彼の立場からするとそれは時間の浪費(ろうひ)だ 
\\	建つ	たつ	
\\	たつ	たつ	
\\	彼がここに移り住んでから3年たった 
\\	時がたつにつれて痛みは消(き)えた 
\\	時のたつのは実に早い 
\\	話に夢中になって時のたつのを忘れた 
\\	10年たてば 
\\	5年もたたないうちに 
\\	達する	たっする	
\\	〔行き着く〕
\\	目的地に達する 
\\	その道は国境に達している 
\\	〔ある状態になる〕 成熟期(せいじゅくき)[高齢]に達する 
\\	息子が成年に達した 
\\	結論[妥協点(だきょうてん)]に達する 
\\	彼の技法はまだ完成の域(いき)に達していない 
\\	まもなく彼の仕事も水準に達するだろう 
\\	恐怖は極度に達した 
\\	〔ある数量になる〕 その都市の人口は500万人に達した 
\\	損害は1億円[億単位]に達した 
\\	〔深く通じる〕 一芸に達した人 
\\	〔なし遂げる〕 希望を達する 
\\	だって	だって	
\\	たって 
\\	〔なぜかというと〕 「まだその小説読んでるの」「だって500ページもあるんだもの」 
\\	[((口)) 
\\	〔しかし〕 「お皿のものはみんな食べなさい」「だっておなかが一杯なんだもん」 
\\	〔…でさえ〕
\\	先生だって間違えることはある 
\\	〔…もまた,…でも〕 今からだって間に合う 
\\	だれだってそんな批評をされたら怒るさ 
\\	[((口)) 
\\	君だって知っていたんだろう 
\\	物忘れするのは君だって同じだ 
\\	遅いといえば今だってもう遅い 
\\	「雨降りはいやだね」「僕だっていやだよ」 
\\	「そんな苦しい思いはしたくない」「だれだってそうだよ」 
\\	〔…だそうだ〕 あの人お金持ちなんだって 
\\	彼女結婚したんだって? 
\\	あの男が詩人だって?まさか ((口)) 
\\	〔たとえ…しても〕 助けを呼んだって無駄だ 
\\	たとえいやだってやらなきゃならないんだ 
\\	いくら利口だってこれには気付くまい 
\\	例え~だとしても 
\\	日がくれてから彼女が外出することは例えあるにしても極めてまれだ。 
\\	たっぷり	たっぷり	
\\	〔十分〕 皮肉たっぷりな言葉 
\\	この瓶はたっぷり3リットル入る 
\\	たっぷり召し上がれ 
\\	ここから駅まで歩いてたっぷり1時間かかる 
\\	パンにジャムをたっぷり塗る 
\\	食べ物[時間]はたっぷりある 
\\	愛嬌(あいきょう)たっぷりだった 
\\	〔ゆったりした〕
\\	たっぷりした服 
\\	たとえ	たとえ	
\\	たとえくびになっても真実(しんじつ)を言うつもりだ 
\\	たとえ本当だとしても証拠(しょうこ)がない 
\\	たとえ間違いがあるにしてもごく少しだ 
\\	たとえどんなに貧(まず)しくても決してくじけるな 
\\	たとえ彼がなんと言っても言うことを聞くな 
\\	ても 
\\	たとえ
\\	譬え・×喩え】 
\\	〔直喩(ちょくゆ)〕
\\	〔隠喩(いんゆ)〕
\\	〔比喩(ひゆ)〕
\\	いろいろなたとえを使って話をした 
\\	〔例〕
\\	たとえを引けば説明は簡単だ 
\\	〔ことわざ〕
\\	たとえに言う通り,光陰矢(こういんや)の如(ごと)しだ 
\\	光陰 
\\	の如し 
\\	たとえ話 
\\	谷	たに	
\\	他人	たにん	
\\	種	たね	
\\	〔種子(しゅし)〕
\\	みかん[りんご]の種 
\\	桃(もも)[梅(うめ)]の種 
\\	くるみの種 
\\	種なしぶどう 
\\	庭に種をまいた 
\\	畑(はたけ)に穀物(こくもつ)の種をまく 
\\	ぶどうの種を取り除(のぞ)く 
\\	まかぬ種は生えぬ 
\\	〔血統(けっとう),産種〕
\\	〔子〕
\\	家畜(かちく)の種のよいのが彼の自慢(じまん)だ 
\\	一粒種(ひとつぶだね) 
\\	〔原因,材料〕
\\	心配の種 
\\	話[新聞記事]の種 
\\	小説の種 
\\	争(あらそ)いの種 
\\	災(わざわ)いの種 
\\	彼は物笑(ものわら)いの種だった 
\\	彼はいつも災いの種をまく男だ 
\\	思い出すも涙の種 
\\	自分でまいた種は自分で刈り取れ 
\\	〔料理の材料〕
\\	あのすし屋は種がいい 
\\	〔手品などの仕掛け〕
\\	手品の種 
\\	種も仕掛けもありません 
\\	手品の種を明(あ)かしましょう 
\\	種を宿す 彼女は彼の種を宿(やど)した 
\\	種を宿す 
\\	束	たば	
\\	足袋	たび	
\\	度	たび	
\\	〔回数〕 三度 
\\	七度捜(さが)して人を疑え 
\\	〔「…する度に」の形で,…するごとに〕 彼女は会う度にわび言を言う 
\\	会う度にその子は大きくなっていた 
\\	旅	たび	
\\	たびたび	たびたび	
\\	玉	たま	
\\	〔球形のもの〕
\\	目の玉 
\\	パチンコの玉 
\\	毛糸の玉 
\\	ガラス玉 
\\	ビー玉 
\\	100円玉 
\\	〔水・露の〕
\\	玉のような汗が彼の顔を流れた 
\\	〔レンズ〕
\\	〔うどん・そばの〕
\\	〔人物〕((米口)) 
\\	〔美しい女〕((俗)) 
\\	((英)) 
\\	玉の輿(こし) 
\\	彼女は玉の輿にのった 
\\	は身分の上の人と) 玉のような 玉のような赤ん坊が生まれた 
\\	玉を転(ころ)がすような 玉を転がすようなソプラノ 
\\	球	たま	
\\	弾	たま	
\\	〔ピストル・ライフルの〕
\\	〔鉄砲の〕
\\	〔大砲の〕
\\	彼は弾に当たって死んだ 
\\	そのピストルは弾が込めてある 
\\	〔ホッチキスの〕
\\	だます	騙す	
\\	〔あざむく〕
\\	通例 
\\	は不正な手段,詐欺などで,
\\	は巧妙な策略で,
\\	はうそなどで人を惑わし,だます) 彼らはその男をだまして5万円巻き上げた 
\\	巻き上げる 
\\	女をだまして高い品物(しなもの)を買わせた 
\\	彼はまんまとだまされて彼女に金をやった 
\\	だまされやすい人 
\\	/((口)) 
\\	〔(宥める)なだめる〕
\\	子供をだましてねかせた 
\\	〔上手にあやつる〕 がたがたいう車をだましだまし運転した 
\\	たまたま	偶々	
\\	〔偶然〕
\\	たまたま通り掛かって老人が倒れるのを見た 
\\	その会でたまたま彼の隣に座っただけだ 
\\	たまらない		
\\	〔我慢できない〕
\\	〔感情などを抑えられない〕
\\	今日は暑くてたまらない 
\\	彼のあの傲慢(ごうまん)な態度はたまらない 
\\	娘たちはおかしくてたまらなかった 
\\	頭が痛くてたまらない 
\\	彼に会いたくてたまらない 
\\	〔やりきれない〕 こんなことで不平を言われてはたまらない 
\\	そんなに金がかかるんではたまらない 
\\	〔我慢できないほどいい〕 この味がなんともたまらない 
\\	たまる	たまる	
\\	〔一つの所に集まる〕
\\	テレビのスクリーンにほこりがたまった 
\\	雨が上がって道路のあちこちに水がたまっていた 
\\	溝(みぞ)にごみがたまった 
\\	〔とどこおる(滞る)〕
\\	仕事がたまっていたのでその会合(かいごう)は欠席(けっせき)した 
\\	たまった仕事を処理(しょり)する 
\\	2か月分も家賃がたまっている 
\\	黙る	だまる	
\\	試し	ためし	
\\	試す	ためす	
\\	機械がうまく動くかどうか試してみた 
\\	人を試すのは好まない 
\\	最初はだめでももう一度試してごらん 
\\	ためる	ためる	
\\	〔蓄(たくわ)える〕
\\	〔蓄積(ちくせき)する〕
\\	〔集める〕
\\	〔節約(せつやく)してためる〕
\\	りすは木の実をためる 
\\	彼は珍(めずら)しい切手を多くためている 
\\	その子はきれいな小石をためるのが好きだった 
\\	若いうちに知識をためておきなさい 
\\	彼女は目に一杯涙をためていた 
\\	〔滞(とどこお)らせる〕
\\	仕事をため過ぎた 
\\	支払いはためないように 
\\	便り	たより	
\\	〔消息〕
\\	((文)) 
\\	〔手紙〕
\\	その後,彼から何の便りもない 
\\	時々便りをください 
\\	便りのないのはよい便り ((諺)) 
\\	だより
\\	便り】 【例文】 パリ便り 
\\	花便り 
\\	((文)) 
\\	国連便り 
\\	頼る	たよる	
\\	彼らは収入(しゅうにゅう)を牧羊(ぼくよう)に頼っていた 
\\	叔母(おば)は父を頼ってわが家に来た 
\\	頼るべき財産(ざいさん)が[人が他に]ない 
\\	彼以外には頼れる友は一人もいない 
\\	彼女は一家の柱と頼る夫を失った 
\\	頼るべき証拠(しょうこ)を出しなさい 
\\	段	だん	
\\	単位	たんい	
\\	〔数量をはかる基準となるもの〕
\\	貨幣(かへい)の単位 
\\	速力(そくりょく)[重量]単位 
\\	熱[電気]単位 
\\	単位制 
\\	表は数字を千単位で示している 
\\	〔基本的まとまり〕
\\	夫婦が一単位となって社会生活を行う 
\\	〔大学などの一定の学習量〕
\\	((米)) 
\\	必修(ひっしゅう)単位 
\\	英語で8単位とる 
\\	数学の単位を落とした 
\\	/((口)) 
\\	単語	たんご	
\\	〔語彙(い)〕
\\	基本単語 
\\	/〔総称〕
\\	知っている英語の単語数 
\\	英語の単語をあまり知らない 
\\	単語集 
\\	男子	だんし	
\\	単純	たんじゅん	
\\	誕生	たんじょう	
\\	団体	だんたい	
\\	〔人の集まり〕
\\	〔一行〕
\\	〔一団〕
\\	団体で旅行する 
\\	20人の団体を作る 
\\	10人ずつ団体になった 
\\	ずつ 
\\	〔組織体〕
\\	婦人団体 
\\	政治団体 
\\	宗教(しゅうきょう)団体を組織する 
\\	団体客 
\\	〔旅行者の〕
\\	団体競技 
\\	団体協約 
\\	団体交渉 
\\	経営(けいえい)側に団体交渉をする 
\\	団体交渉権 
\\	団体乗車券 
\\	団体信用生命保険 
\\	団体生活 
\\	団体精神 
\\	団体保険 
\\	団体旅行 
\\	団体割引 
\\	〔旅行の〕
\\	20名以上には団体割引があります 
\\	担当	たんとう	
\\	この地区担当のセールスマン 
\\	その事件担当の弁護士 
\\	私は2年生を担当している 
\\	彼が会計を担当した 
\\	彼は高校で歴史の授業を担当している 
\\	担当教師 物理の担当教師 
\\	担当者 
\\	だれがこの課の担当者ですか 
\\	単なる	たんなる	
\\	彼は単なる文士(ぶんし)ではない 
\\	単に	たんに	
\\	地	ち	
\\	地位	ちい	
\\	地域	ちいき	
\\	は地理的・文化的などの特徴により
\\	は行政などの目的により区分された区域をさす) 地域的 
\\	地域別に 
\\	住宅地区 
\\	工業地域 
\\	降水量の多い地域 
\\	地域医療 
\\	過疎地(かそち)では地域医療は重要だ 
\\	地域格差 収入に地域格差がある 
\\	地域給 
\\	地域開発 
\\	地域研究 
\\	アメリカを対象とする地域研究 
\\	地域差 
\\	地域社会 
\\	地域主義 
\\	地域代表 
\\	地域団体 
\\	地域暖房 
\\	知恵	ちえ	
\\	地下	ちか	
\\	違い	ちがい	
\\	違いない	ちがいない	
\\	近頃	ちかごろ	
\\	地球	ちきゅう	
\\	地区	ちく	
\\	は特殊な目的で区切られた区域) 風致地区 
\\	住宅地区 
\\	の方は工場などの許可されない地区) 文教地区 
\\	遅刻	ちこく	
\\	知事	ちじ	
\\	知識	ちしき	
\\	父親	ちちおや	
\\	知能	ちのう	
\\	地平線	ちへいせん	
\\	地方	ちほう	
\\	〔広い地域〕
\\	⇒ちいき(地域) 九州地方 
\\	京阪(けいはん)地方 
\\	山岳(さんがく)地方 
\\	この地方では 
\\	〔中央に対して〕
\\	地方的 
\\	は「田舎の」という意味で使われるが,
\\	は都市も含めてある特定地域に関係あるという意味で用いられる) 地方に住む 
\\	地方の人 
\\	地方では有名な詩人 
\\	この地方の名物(めいぶつ)は何ですか 
\\	地方化 
\\	地方官庁 
\\	((英)) 
\\	地方機関 
\\	地方(自治体)議会 
\\	地方議会議員 
\\	地方記事 
\\	地方行政 
\\	地方銀行 
\\	地方検察庁 
\\	地方交付税 
\\	地方公務員 
\\	地方債 
\\	地方裁判所 
\\	地方財政 
\\	地方紙 〔新聞〕
\\	〔総称〕
\\	地方時 
\\	地方自治 
\\	地方自治体 
\\	地方事務所 
\\	地方主義 
\\	地方巡業 
\\	地方消費税 
\\	地方色 
\\	地方色豊かな産物 
\\	地方税 ((米)) 
\\	((英)) 
\\	地方選挙 
\\	地方なまり 
\\	地方版 
\\	地方病 
\\	地方分権 
\\	茶	ちゃ	
\\	ちゃんと	ちゃんと	
\\	〔きちんと〕
\\	〔ゆがまずに〕
\\	ちゃんとした身なり 
\\	ちゃんとした服装をしている 
\\	ちゃんと仕事をする 
\\	家賃をちゃんと支払う 
\\	部屋をちゃんと片付ける 
\\	子供たちはなかなかちゃんと並ばなかった 
\\	ちゃんと腰(こし)かけなさい 
\\	〔間違いなく〕 ちゃんと覚えています 
\\	秘密はちゃんと知ってるぞ 
\\	ちゃんと聞いているよ 
\\	ちゃんとここに置いたんだ 
\\	〔身元などが確かな様子〕 あの娘はちゃんとした家の出(しゅつ)だ 
\\	ちゃんとした会社に勤めている 
\\	注	ちゅう	〔全体〕
\\	〔個々の〕
\\	注を付ける 
\\	注付きの 
\\	注付きのハムレットの新版(しんぱん) 
\\	その本には詳(くわ)しい注が付いている 
\\	中央	ちゅうおう	
\\	中古	ちゅうこ	
\\	〔中世〕
\\	中古の 
\\	〔やや古いこと〕 中古の 
\\	中古車 
\\	中古品 
\\	中止	ちゅうし	
\\	駐車	ちゅうしゃ	
\\	昼食	ちゅうしょく	
\\	昼ご飯 (ひるごはん))
\\	中心	ちゅうしん	
\\	〔中央〕
\\	((英)) 
\\	中心の 
\\	彼の家は市の中心にある 
\\	軸を中心にして回転する 
\\	〔最も重要な位置にある物や人〕 東京は政治の中心だ 
\\	その物語は若い男女の恋愛を中心に展開する 
\\	中心となって働く 
\\	運動の中心は右翼の青年たちであった 
\\	人々は彼を中心に集まってきた 
\\	〔大事な点〕
\\	あなたの発言は問題の中心を外れている 
\\	中心角 
\\	中心気圧 
\\	中心人物 
\\	中心線 
\\	中心地 
\\	商業[工業]の中心地 
\\	中心点[部] 
\\	注目	ちゅうもく	
\\	世人(せじん)の注目の的(まと)となる 
\\	彼の作品は注目に値(あたい)する[値しない] 
\\	注目すべき現象(げんしょう) 
\\	その書物(しょもつ)はだれからも注目されなかった 
\\	公害問題は最近広く注目を集めている 
\\	日本の工業は全世界の注目を浴びている 
\\	注目株 
\\	注文	ちゅうもん	
\\	(要求)
\\	(条件)
\\	〔あつらえ〕
\\	注文する 
\\	大口(おおぐち)[小口(こぐち)]注文 
\\	急ぎの注文 
\\	追加注文をする 
\\	雑誌の注文を取る[取り消す] 
\\	電話でのご注文を承(うけたまわ)ります 
\\	インターネットで注文をする 
\\	注文に応じきれなかった 
\\	その品物に注文が殺到(さっとう)した 
\\	私は靴を全部注文で作らせる 
\\	注文したものがやっと手に入った 
\\	英国[外国]に本を注文する 
\\	ウエートレスにコーヒーを注文した 
\\	それはあの会社に注文した 
\\	百科事典は注文中だ 
\\	〔要求〕
\\	〔条件〕
\\	注文をつける 
\\	彼は無理な注文をつけてきた 
\\	居間は南向きにするという注文を一つだけつけて家を建てた 
\\	すべてが注文通りにゆくものではない 
\\	「一人でやれ」「それは無理な注文です」 
\\	注文建築 
\\	注文先 〔注文した人〕
\\	〔注文を受けた人〕
\\	注文書 
\\	注文取り 
\\	注文流れ 
\\	注文流れの品 
\\	注文主 
\\	注文主は倒産した 
\\	注文主は丸山商事だ 
\\	注文品 
\\	注文服 
\\	長期	ちょうき	
\\	調査	ちょうさ	〔事実・原因などに関する組織的な調査〕
\\	〔事実などの探求〕
\\	〔学術的な研究〕
\\	調査する 
\\	((文)) 
\\	〔照合などにより真偽を確かめる〕
\\	現地(げんち)調査 
\\	人の身元調査をする 
\\	市場調査 
\\	事故の原因は目下(もっか)調査中 
\\	目下 
\\	当局の調査によれば 
\\	彼の背後(はいご)関係を調査中である 
\\	委員会は調査結果を発表した 
\\	(会計の) 帳簿を調査する 
\\	人物を調査する 
\\	人口を調査する 
\\	土地を調査する 
\\	調査委員会 
\\	調査員[官] 
\\	財務調査官 
\\	調査課 
\\	〔新聞社などの〕
\\	調査事項 
\\	調査書 
\\	〔内申書〕
\\	調査団 
\\	調査部 〔外務省の〕
\\	〔国税局の〕
\\	調査報告 
\\	調査捕鯨 
\\	調子	ちょうし	
\\	〔音調〕
\\	〔高低〕
\\	調子が合っている[外れている] 
\\	バイオリンの調子を合わせた 
\\	彼女の声は調子が高い[低い] 
\\	彼らは声の調子を上げた[下げた] 
\\	役者(やくしゃ)は声の調子をたやすく変えられる 
\\	〔拍子(ひょうし)〕
\\	調子のよい歌 
\\	指で調子をとる 
\\	彼女は規則(きそく)正しい調子で歩いた 
\\	〔語調〕
\\	一本調子で 
\\	彼は怒った調子で否定した 
\\	彼女は急に声の調子を変えた 
\\	〔字句の言い回し〕 大衆向きに調子を下げて書いた 
\\	原文(げんぶん)の調子を生(い)かして詩を訳す 
\\	〔やり方〕
\\	こういう調子で書きなさい 
\\	この調子では成功できないね 
\\	そうそう,その調子 
\\	〔具合〕
\\	体の調子を整(ととの)える 
\\	/〔運動選手が〕
\\	彼は体の調子が大変よい[悪い] 
\\	腹の調子が変だ 
\\	調子よく事が運んでよかった 
\\	調子はどうだい 〔体の具合〕
\\	/〔仕事などの具合〕
\\	機械の調子が狂(くる)っている 
\\	〔勢い〕 この調子なら早めに目的地に着きそうだ 
\\	調子が出る やっと調子が出てきた 
\\	クォーターバックが今日はいつもの調子が出なかった 
\\	調子に乗る 調子に乗ってしゃべりまくった 
\\	調子に乗るな 
\\	/〔はしゃいでいる子供たちに〕
\\	調子のいい 彼は調子のいい男だ〔口先のうまい〕 
\\	調子のいい返事をして,いつも後で後悔するんだ 
\\	調子はずれ 調子はずれに歌う 
\\	調子を合わせる 時には人に調子を合わせるのもよい 
\\	彼女が自分の意見に固執するので適当に調子を合わせておいた 
\\	謀略者(ぼうりゃくしゃ)と調子を合わせて相手の意図を探った 
\\	調子者 
\\	頂上	ちょうじょう	
\\	〔山の〕
\\	〔切り立つ山頂〕
\\	山の頂上に 
\\	遂(つい)に彼らは頂上を極めた 
\\	〔極点〕
\\	彼は当時栄華(えいが)の頂上にあった 
\\	物価が頂上に達した 
\\	頂上会談 
\\	ちょうだい	ちょうだい	
\\	ください; 
\\	貯金	ちょきん	
\\	直接	ちょくせつ	
\\	直接の (に) 
\\	この事件はそれと直接の関係はない 
\\	紛争(ふんそう)の直接の原因は何だったのか 
\\	その件につき彼と直接交渉(こうしょう)した 
\\	自分で直接行ったほうがいいですよ 
\\	直接得た情報 
\\	直接教授法 〔語学の〕
\\	直接行動 
\\	直接照明(しょうめい) 
\\	直接税 
\\	直接選挙 
\\	直接談判 
\\	直接販売 
\\	直接費 
\\	直接民主制 
\\	直接話法 〔文法で〕
\\	著者	ちょしゃ	
\\	散らす	ちらす	
\\	〔まき散らす〕
\\	突風が桜の花を散らした 
\\	天井から舞台に紙吹雪(かみふぶき)を散らした 
\\	火打ち石が火花を散らした 
\\	〔追い散らす〕
\\	警官は群衆を散らした 
\\	〔心を他に向ける〕 仕事をしている人の気を散らす 
\\	あの人の気を散らしてやろうと思った 
\\	〔はれ物などを〕
\\	この薬ははれ物を散らすのに効く 
\\	散る	ちる	
\\	〔ばらばらに落ちる〕
\\	桜の花が散ってしまった 
\\	椿(つばき)の花が地面に散っていた 
\\	波(なみ)が岩(いわ)に当たって散った 
\\	〔離れ離れに去る〕
\\	見物人(けんぶつにん)もやがて散って行った 
\\	パトカーが来たので群衆は散った 
\\	〔きれぎれに広がって消える〕 雲が散った 
\\	霧(きり)が散り始めた 
\\	〔染みて広がる〕 インキが散った 
\\	手紙のインキが散っていた 
\\	〔はれ物が〕 はれものが散った 
\\	〔心が他に向かう〕 音がすると気が散る 
\\	〔知れ渡る〕 うわさが町中(まちなか、じゅう)に散った 
\\	〔いさぎよく死ぬ〕 戦(たたか)いに散った兵士(へいし)たち 
\\	遂に	ついに	〔とうとう〕
\\	〔結局〕
\\	〔ようやく〕
\\	橋の建設はついに完成した 
\\	彼らはついにストライキを起こした 
\\	彼はついに再起(さいき)不能となった 
\\	彼とはついに会わずじまいだった 
\\	通過	つうか	
\\	〔通り過ぎること〕
\\	通過する 
\\	間もなくトンネルを通過する 
\\	大島の上空(じょうくう)を通過中だ 
\\	車は橋を通過した 
\\	列車が通過した 
\\	満員のためバスは止まらずに通過した 
\\	〔可決されること〕
\\	この法案は楽々(らくらく)と通過した 
\\	この議案は満場(まんじょう)一致で議会を通過した 
\\	彼は国会でその議案を通過させた 
\\	は強引に) 
\\	〔許可されて通ること〕 彼女は第一次(いちじ)審査(しんさ)を通過した 
\\	それは税関(ぜいかん)を通過した 
\\	通過駅 
\\	通過税 〔貨物の〕
\\	通過儀礼 
\\	通貨	つうか	
\\	〔金〕
\\	通貨の安定 
\\	通貨制度の改革(かいかく)を行う 
\\	フランスでは通貨としてフランが使われている 
\\	その国の通貨で支払う 
\\	通貨の切り下げ[切り上げ]を行った 
\\	その国の通貨名を入れる) 切り下げ 
\\	切り上げ 
\\	通貨安定策 
\\	通貨危機 
\\	通貨切上げ 
\\	通貨切り上げで輸出業界は減益になった 
\\	その国は通貨切り上げを行った 
\\	通貨切下げ 
\\	通貨切り下げが国の貿易赤字削減を助けた 
\\	その国は通貨切り下げを行った 
\\	通貨再膨張(ぼうちょう) 
\\	通貨収縮 
\\	通貨スワップ取り引き 
\\	通貨単位 
\\	通貨膨張 
\\	通貨流通高 
\\	通貨流通高は10兆円に達した 
\\	通学	つうがく	
\\	通勤	つうきん	
\\	通行	つうこう	
\\	〔交通,往来(おうらい)〕
\\	通行する 
\\	この道は通行出来ない 
\\	この先,車は通行禁止だ 
\\	ここは一方通行だ 
\\	/〔掲示〕
\\	右側通行〔掲示〕 
\\	この国では車は左側通行だ 
\\	この道路は通行量が多い[少ない] 
\\	駐車中の車が歩行者(ほこうしゃ)の通行を妨(さまた)げている 
\\	通行(優先)権 
\\	通行税 
\\	〔貨物の〕
\\	通行止め 車両通行止め〔掲示〕 
\\	この道路は修理のため一時通行止めになっている 
\\	通行人 
\\	((複
\\	通行妨害 
\\	〔法律用語〕
\\	通行料金 
\\	この道路は通行料金を取られる 
\\	通じる	つうじる	
\\	〔道路などが〕
\\	〔鉄道・バス・道などが〕
\\	この道は隣の町に通じている 
\\	この道は新宿から青梅に通じる 
\\	二つの村の間にトンネルが通じていた 
\\	半島の突端(とったん)までバスが通じている 
\\	やっと地下鉄が郊外(こうがい)まで通じた 
\\	〔電話が〕 話の途中で電話が通じなくなった 
\\	話し中で電話が通じません 
\\	〔電流が〕 電流を通じると機械は動きだした 
\\	〔連絡する〕 訪問する前にあらかじめ話を通じておいた 
\\	あらかじめ(予め) 
\\	〔相手に理解される〕 意味の通じない文 
\\	パリでは言葉が通じなくて困った 
\\	彼には冗談(じょうだん)が通じない 
\\	やっと彼女に僕の気持ちが通じたようだ 
\\	彼の考えは上司にうまく通じなかった 
\\	私の一念(いちねん)が天に通じたのだろう,ついに試験に合格した 
\\	〔精通する〕
\\	〔こっそり
\\	関係する〕 彼はライバル会社の重役と通じている 
\\	人妻(ひとづま)と通じる 
\\	通信	つうしん	
\\	通訳	つうやく	
\\	捕まる	つかまる	
\\	(捕えられる)
\\	(逮捕される)
\\	(すがる)
\\	〔捕らえられる〕
\\	〔逮捕される〕
\\	少年2人が万引きで捕まった 
\\	追っ手に捕まった 
\\	20キロオーバーで捕まった 
\\	逃げたライオンはまだ捕まらない 
\\	〔引き留められる〕 彼女の所へ行くと捕まってなかなか帰れないよ 
\\	あの人に捕まると嫌だから別の道を行こう 
\\	〔見つかる〕 この時間はタクシーがなかなか捕まらない 
\\	彼は多忙なのでいつ訪ねても捕まらない 
\\	例文 「疑いが晴れてよかったね」「うん。でも真犯人が捕まるまでは何だかすっきりしないよ」 
\\	「何しょげてるの?」「スピード違反で捕まってさ。ついてないよ」 
\\	「緊急の用事なんで、何とか彼に連絡つきませんかねえ」「ちょっとお待ちください。捕まりそうなところを今2、3当たってみますから」 
\\	「遅かったわねえ。何してたのよ」「ごめんごめん。課長に捕まって一杯付き合わされてたんだ」 
\\	「この辺はなかなかタクシーが捕まらないねえ」「駅前のタクシー乗り場に行きましょうか」 
\\	「このバス、いやに揺(ゆ)れるね」「危ないから、釣り革につかまっていた方がいいよ」 
\\	「こんな大きいバイクの後ろに乗せてもらったのは初めてよ」「しっかりつかまって。スピード出すよ」 
\\	つかむ	掴む	
\\	〔物を捕まえる〕
\\	〔急に,力ずくで〕
\\	〔握りしめる〕
\\	〔しっかりつかむ〕
\\	警官は強盗(ごうとう)の腕をつかんだ 
\\	この鎖(くさり)をつかみなさい 
\\	彼はそれをつかんでぎゅっと握(にぎ)りしめた 
\\	猿(さる)はボールをつかんで離(はな)さなかった 
\\	父は歓迎(かんげい)して彼の手をつかんだ 
\\	船(ふね)が揺(ゆ)れたのであわてて手すりをつかんだ 
\\	溺(おぼ)れる者はわらをもつかむ ((諺)) 
\\	〔手に入れる〕 新しいネタ[証拠]をつかんだらしい 
\\	主役(しゅやく)に抜擢(ばってき)される幸運をつかんだ 
\\	抜擢 
\\	思い掛けず大金をつかんだ 
\\	聴衆(ちょうしゅう)の心をつかむのがうまい 
\\	機会をつかんで世界一周(いっしゅう)をした 
\\	要点をつかむ 
\\	彼は真相(しんそう)をつかんだ 
\\	彼は算数をよくつかんでいる 
\\	疲れ	つかれ	
\\	付き合い	つきあい	
\\	〔交際〕 あの人は付き合いが広い[狭い] 
\\	付き合い上手の[下手な]人 
\\	彼とは兄弟の付き合いをしている 
\\	私たちはおれ,おまえの付き合いをしている 
\\	私たちは長い付き合いです 
\\	彼とは会釈(えしゃく)をする[話を交わす]程度の付き合いだ 
\\	近所付き合いはうまくいっていますか 
\\	彼とは付き合いがない 
\\	なぜ彼との付き合いをやめたの 
\\	彼は付き合いが悪い 
\\	/((口)) 
\\	〔一緒に行動すること〕 お付き合いでバーに行った 
\\	付き合いきれない 
\\	付合う	つきあう	
\\	〔交際する〕
\\	悪い者と付き合うな 
\\	初めての人たちと付き合うのが苦手(にがて)だ 
\\	彼女は付き合っていて気持ちのいい人だ 
\\	彼は付き合いにくい人だ 
\\	彼女とはとてもつきあいきれない 
\\	彼は彼女とこの数か月付き合っている 
\\	彼女はいかがわしい男と付き合っているらしい 
\\	〔一緒に行動する〕 上役(うわやく)の夕食に付き合った 
\\	妹の買い物に付き合った 
\\	お義理で彼に酒を付き合った 
\\	次々	つぎつぎ	
\\	付く	つく	
\\	(付着する)
\\	(汚れ・染みが)
\\	(跡が残る)
\\	(付属する)
\\	(従う)
\\	(味方に)
\\	(根を下ろす)
\\	(幸運である)
\\	〔付着する〕
\\	〔汚れなどが〕
\\	靴に泥が付いていますよ 
\\	洋服にインキの染みが付いた 
\\	容疑者のシャツに血痕(けっこん)が付いていた 
\\	〔跡が残る〕 窓の下に大きな足跡が付いていた 
\\	〔付属(ふぞく)する,備わる〕 この書棚には二つ引き出しが付いている 
\\	この商品には景品が付いている 
\\	この定食にはコーヒーが付いている 
\\	〔設(もう)けられる〕 この町にも高速道路が付いた 
\\	〔加わる〕 トマトに赤みが付いてきた 
\\	この定期預金には6分の利子が付く 
\\	この料理は味が付いてない 
\\	〔自分の物になる〕 あの子はいくらか知恵(ちえ)が付いてきた 
\\	英語の力が付いてきた 
\\	〔根を下ろす〕 つつじの挿し木が付いた 
\\	〔そばにいる〕 首相には護衛が付いていた 
\\	国会議員には2人秘書が付く 
\\	看護師が病人に付いた 
\\	〔一緒に行く〕 おばさんに付いて買物に行った 
\\	〔味方する〕 彼はいつも勝ちそうな側に付く 
\\	彼は母親の側に付いた 
\\	〔親しくする〕 付かず離(はな)れず 
\\	二人の間はつかず離れずというところだ 
\\	〔ある値段になる〕 この卵は1個16円に付く 
\\	ダースで買えば安く付く 
\\	この買い物は結局高いものに付いた 
\\	〔名前になる〕 この新商品にはまだ名前が付いていない 
\\	〔燃え移る〕⇒つき(付き)1 このライターは付かない 
\\	木造(もくぞう)の家は火が付きやすい 
\\	〔決まる〕 なかなか決心が付かない 
\\	まだ彼らとの話が付かない 
\\	今のポイントで勝負が付いた 
\\	片が付いた 
\\	/〔終わる〕
\\	〔運がいい〕⇒つき 付いていた 
\\	就く	つく	
\\	〔仕事・地位に身を置く〕 
\\	彼はやっと職に就いた 
\\	彼は会長の地位に就いた 
\\	[((文))
\\	彼女は教職(きょうしょく)に就くつもりだ 
\\	彼はゴルフ日本一の座(ざ)に就いた 
\\	〔ある動作(どうさ)を始める〕 
\\	眠りに就く 
\\	彼らは帰路(きろ)に就いた 
\\	彼は欧州旅行の途(と)に就いた 
\\	〔師事する〕 
\\	彼女は彼に就いてピアノを習った 
\\	彼は加藤教授に就いてフランス文学を研究中だ 
\\	〔従(したが)う〕
\\	人は易(やす)きに就く 
\\	突く	つく	
\\	〔とがった物で突き刺す〕
\\	〔刃物で〕
\\	〔角・牙(きば)で〕
\\	誤(あやま)って指を針(はり)で突いてしまった 
\\	暴漢(ぼうかん)が短刀(たんとう)で彼の背中を突いた 
\\	荒(あ)れ狂(くる)う波が天を突かんばかりであった 
\\	闇(やみ)を突いて稲妻(いなずま)が走った 
\\	〔強く押す〕 背中を強く突かれて私は前に倒れた 
\\	わき腹(ばら)を突かれて息が詰(つ)まった 
\\	脇腹(わきばら) 
\\	鹿(しか)が頭で私の腹を突いた 
\\	書類に判(はん)をつく 
\\	まりをつく 
\\	〔細長いものを押し立てて支えとする〕 松葉杖(まつばづえ)を突けば歩けますよ 
\\	彼はひざを突いて謝(あやま)った 
\\	よろめいて思わずひざを突いた 
\\	〔鋭く攻める〕 敵陣(てきじん)を突く 
\\	相手(あいて)は私の理論の弱点を突いてきた 
\\	怠(なま)けていることを突かれて申し開(ひら)きが出来なかった 
\\	事件の核心(かくしん)を突くような証拠(しょうこ)を見付けた 
\\	〔ものともしない〕 実験は風雨(ふうう)を突いて決行された 
\\	飛行機は嵐を突いて飛んだ 
\\	〔心・感覚を強く刺激する〕 その悲しい光景(こうけい)が見る者の胸を突いた 
\\	辺りにはつんと鼻を突くアンモニアのにおいが立ち込めた 
\\	次ぐ	つぐ	
\\	〔次に位する〕
\\	彼女は彼に次ぐ地位にいる 
\\	彼は知事に次ぐ実力者だ 
\\	英語に次いで数学が好きだ 
\\	〔続く〕 大雨に次いで台風が来た 
\\	地震に次いで火災が発生した 
\\	[((文)) 
\\	一家は不幸に次ぐ不幸に見舞われた 
\\	注ぐ	そそぐ	そそぐ 
\\	〔流れ込む〕
\\	荒川は東京湾に注ぐ 
\\	〔振り掛かる〕 窓から雨が降り注いでいた 
\\	〔掛ける〕
\\	〔振り掛ける〕
\\	植木(うえき)に水を注ぐ 
\\	そんな事をしたら火に油を注ぐようなものだ 
\\	〔注ぎ入れる,つぐ〕
\\	コップに牛乳を注ぐ 
\\	〔灌漑(かんがい)する〕
\\	春には田に水を注ぐ 
\\	〔集中する〕
\\	その仕事に全力を注いだ 
\\	これは私が心血を注いで作り上げた作品だ 
\\	いつも前方(ぜんぽう)に注意を注いでいなさい 
\\	彼は急に私の方に目を注いだ 
\\	付ける	つける	
\\	(取り付ける)
\\	(塗る)
\\	(張り付ける)
\\	(値を)
\\	(尾行する)
\\	(火を)
\\	(染みなどを)
\\	〔密着させる〕 両膝(りょうひざ)をぴったり付けて立つ 
\\	本箱(ほんばこ)を壁(かべ)に付けて置いてください 
\\	〔付着させる〕 スーツケースに荷札(にふだ)を付ける 
\\	コートにペンキを付けてしまった 
\\	〔塗る〕 パンにバターを付ける 
\\	おしろいを付ける 
\\	けがをした膝に薬を付けた 
\\	〔取り付ける〕 窓にカーテンを付ける 
\\	テレビにアンテナを付けた 
\\	〔添える,加える〕 履歴書(りれきしょ)に写真を付ける 
\\	本には用語解説(ようごかいせつ)が付けてある 
\\	利子(りし)を付けて払い戻(もど)す 
\\	肉に塩味(しおあじ)を付ける 
\\	上の句に下の句を付ける 〔上の句と下の句を合わせる〕
\\	/〔下の句を作って付ける〕
\\	景品(けいひん)を付けて品物を売った 
\\	/((口)) 
\\	彼が禁煙(きんえん)するという条件(じょうけん)を付けて彼女は婚約した 
\\	〔付き添わせる〕 太郎をおじいさんのお供に付けよう 
\\	〔跡を残す〕 足跡を付ける 
\\	家具に傷を付ける 
\\	〔あることをし始める〕 食事に箸(はし)を付ける 
\\	まだ手を付けるな 
\\	〔記入する〕 日記を付ける 
\\	来月の勘定に付けておく 
\\	帳簿(ちょうぼ)に付ける 
\\	〔自分の物にする〕 語学力を付ける 
\\	夜更(よふか)しの悪い癖を付けた 
\\	ステーキを食べて元気を付けた 
\\	彼を味方に付ければ安心だ 
\\	〔食べ物を用意する〕 熱燗(あつかん)を1本付ける 〔温める〕
\\	/〔出す〕
\\	御飯を付ける 
\\	/〔盛る〕
\\	ライスをお付けしましょうか〔レストランで〕 
\\	〔尾行する〕 こっそり人 (の後) を付ける 
\\	/((口))
\\	〔注意を向ける〕 かわいい女の子に目を付ける 
\\	車に気を付ける 
\\	〔まとめる〕 うまく話を付ける 
\\	仕事に片を付ける 
\\	〔定める〕 値を付ける 
\\	/〔値踏みする〕
\\	⇒つけね(付け値) 見当を付ける ⇒けんとう(見当) 
\\	〔与える〕 そのキリンに長太郎という名を付けた 
\\	〔火をうつして燃やす〕 家に火を付ける 
\\	着ける	つける	
\\	〔体に帯びる〕
\\	は着けている状態,
\\	は着ける動作を示す) 黒いドレスを身に着けた女 
\\	髪にリボンを着ける 
\\	ブローチを着ける 
\\	イヤリングを着ける 
\\	は
\\	の場合には使えない) 
\\	〔ある場所へ寄せる〕 ボートを岸(きし)に着ける 
\\	運転手は車を門前(もんぜん)に着けた 
\\	〔届かせる〕 前にかがんで床に手を着ける 
\\	伝わる	つたわる	
\\	〔伝導・伝達される〕
\\	この金網(かなあみ)には電気が伝わっている 
\\	列車の音はここ[この高さ]までは伝わって来ない 
\\	彼の緊張(きんちょう)が彼女にも伝わったようだ 
\\	彼の悲しみがひしひしと伝わって来る 
\\	ひしひし 
\\	〔知れ渡る〕
\\	大地震があるといううわさが口から口へと伝わった 
\\	彼が離婚するといううわさが伝わった 
\\	〔沿って行く〕
\\	⇒つたう(伝う) 涙が彼女のほおを伝わって流れた 
\\	〔伝承(でんしょう)される〕
\\	この掛け軸(じく)は祖父の代から伝わった家宝(かほう)だ 
\\	〔伝来する〕
\\	漢字は中国から日本に伝わった 
\\	土	つち	
\\	続き	つづき	
\\	包み	つつみ	
\\	勤め	つとめ	
\\	務め	つとめ	〔任務,義務〕
\\	彼は警官としての務めを果たした[怠った] 
\\	務めとしてやったまでですからお礼には及びません 
\\	つながる	繋がる	
\\	〔物が結(むす)びつく〕 舟は綱(つな)でつながっている 
\\	この村は鉄道で町につながっている 
\\	この道は100メートル先で高速道路とつながっている 
\\	自動車がつながって走り過ぎた 
\\	〔関係する〕 もう一人事件につながる人がいる 
\\	この仕事は金もうけにつながらない 
\\	儲(もう)け 
\\	候補者(こうほしゃ)の清潔(せいけつ)なイメージが当選につながった 
\\	〔血筋(ちすじ)が同じである〕 彼女は私と血がつながっている 
\\	〔電話が〕 ロンドンにつながりました 
\\	電話がなかなかつながらない 
\\	首がつながる ⇒くび(首)
\\	つなぐ	繋ぐ	
\\	〔ひもでゆわえる〕
\\	馬を木につなぐ 
\\	波止場(はとば)に舟をつなぐ 
\\	ブイにボートをつなぐ 
\\	犬を皮ひもでつないでおく 
\\	罪人を獄(ごく)につなぐ 
\\	〔連結する〕 2本の糸をつなぐ 
\\	手をつないで歩く 
\\	選挙運動で彼らは手をつないだ 
\\	冷蔵庫をコンセントにつなぐ 
\\	(電話を) 3号室(ごうしつ)につないでください 
\\	内線5021番につないでください 
\\	と読む) 
\\	〔切れないように保たせる〕 露命(ろめい)をつなぐ 
\\	露命 
\\	料理屋の残飯(ざんぱん)で命をつないでいた 
\\	座をつなぐために即興(そっきょう)でピアノを弾いた 
\\	彼女は息子の成功に一縷(いちる)の望(のぞ)みをつないでいる 
\\	つなげる	繋げる	
\\	[動ガ下一][文]つな・ぐ[ガ下二]ひもや綱などを結び合わせてひと続きにする。「短いひもを―・げて長くする」
\\	常に	つねに	
\\	いつも) 
\\	彼は常に大ぶろしきを広げる 
\\	警官が常に首相官邸を見張っている 
\\	外へ出る時は常に服装に気をつける 
\\	彼は常にポケットに手を入れて歩いている 
\\	彼女は常に文句を言っている 
\\	翼	つばさ	
\\	翼の大きな鳥 
\\	鳥は翼を広げた[畳んだ] 
\\	はとは電線の上で翼を休めた 
\\	その鳥の翼は広げると90センチある 
\\	つまり	つまり	
\\	〔物が詰まること〕 枕(まくら)のパンヤの詰(つ)まり具合(ぐあい)がよくない 
\\	パンヤ 
\\	〔要するに,すなわち 
\\	つまり彼は英雄(えいゆ)ではないんだ 
\\	つまり私はあなたをもう信用出来ないということです 
\\	彼は純粋(じゅんすい)過ぎるんだ
\\	つまり世間知(せけんし)らずなんだ 
\\	罪	つみ	
\\	〔道徳・宗教上の〕
\\	自分の罪を告白する 
\\	罪を犯(おか)す 
\\	〔法律上の〕
\\	((英)) 
\\	はごく軽い犯罪も含む) 彼は罪を犯して逮捕された 
\\	罪の意識 
\\	何の罪も犯していない 
\\	それは人道にもとる罪だ 
\\	〔処罰〕 罪に服(ふく)する 
\\	彼は反逆の罪に問われた 
\\	〔悪い結果や過ちに対する責任〕 この事故は彼の罪ではない 
\\	〔無慈悲(むじひ)なこと〕
\\	夫婦の仲を裂(さ)くなんて罪なことだ 
\\	何て罪な女だ 
\\	罪のない 罪のない戯(たわむ)れ 
\\	子供は罪のない顔をして眠っていた 
\\	罪のない人々をいじめる 
\\	罪を着せる 彼は罪を同僚(どうりょう)に着せた 
\\	積む	つむ	
\\	〔だんだん高く〕
\\	〔雑然と〕
\\	〔干し草などを整然と〕
\\	棚の上に本を積んだ 
\\	トラックに材木が山と積んであった 
\\	[((英)) 
\\	干し草をうず高く積む 
\\	〔荷を載せる〕
\\	〔船に〕
\\	車をトレーラーに積む 
\\	荷車は石炭を積んでいた 
\\	船は200トンの荷を積んでいた 
\\	〔重ねる〕 もっと練習を積むべきだ 
\\	経験を積む 
\\	〔蓄(たくわ)える〕
\\	彼は巨万(きょまん)の富(とみ)を積んだ 
\\	詰める	つめる	
\\	〔物を一杯に入れる〕
\\	瓶に油を詰めた 
\\	この布団には羽毛(うもう)が詰めてある 
\\	ダンボールに本を詰める 
\\	小さいバスにいっぱい人を詰める 
\\	袋に食料を詰める 
\\	〔透(す)き間がないようにする〕 
\\	もう少し詰めてください 
\\	奥へお詰め願います 
\\	そんなに詰めて書くと読みにくいよ 
\\	〔通じなくする〕 息を詰めてけんかを見守った 
\\	〔縮める〕
\\	スカート丈(たけ)を5センチ詰める 
\\	〔倹約(けんやく)する〕
\\	人件費を半分に詰めても事業はまだ赤字だ 
\\	詰めた生活をすれば年金で暮らせる 
\\	〔将棋などで〕
\\	〔そこに出向いて勤務する〕 
\\	月曜日は本店に詰めています 
\\	そこには警官が常時(じょうじ)詰めている 
\\	〔一心に
\\	続ける〕 細かい仕事を詰めてやると目を悪くする 
\\	根を詰める ⇒こん(根) 詰めなくては。
\\	このブラウスをつめていただけますか。
\\	根を詰める, 
\\	根を詰める 
\\	指を詰める, 
\\	積もる	つもる	
\\	〔重なって高くなる〕
\\	〔場所が〕
\\	この地方では雪が1メートルも積もるのは珍しくない 
\\	テーブルにはほこりが積もっていた 
\\	〔たまる〕 彼女は積もる恨(うら)みを述べたてた 
\\	立てる 
\\	夜が更(ふ)けるまで積もる話をした 
\\	つらい	つらい	
\\	〔堪えがたい,難儀な〕 つらい役目 
\\	つらい仕事 
\\	/((口))
\\	そのことは思い出すだけでもつらい 
\\	一人息子を亡くしたことは彼には生涯(しょうがい)でもっともつらい経験であった 
\\	つらいことばかりの一生だった 
\\	長年住み慣れた家を売るのはつらいことだった 
\\	どんなにつらい時でも彼女は愚痴(ぐち)をこぼさなかった 
\\	〔冷酷(れいこく)な〕
\\	彼女は私につらく当たる 
\\	釣	つり	
\\	連れ	つれ	
\\	〔同伴者,仲間〕
\\	人込みで連れを見失った 
\\	お連れ様はいらっしゃらないのですか 
\\	〔一緒にいること〕 あの時の連れの女が怪(あや)しい 
\\	旅行中にドイツ人夫婦と連れになった 
\\	〔能の〕
\\	出会い 。 出合い	であい	
\\	運命的な出会い 
\\	その本との出会いが私の人生観(かん)を変えた 
\\	彼との出会いの場所は教会だった 
\\	出会い頭(がしら) 
\\	出会い頭になぐられた 
\\	角を曲がったところで出会い頭にトラックと衝突(しょうとつ)した 
\\	出会い系(けい)サイト 
\\	彼と出会い系サイトで知り合った 
\\	出会う	であう	
\\	は事故・災難等に遭った場合に用いる) 道でばったり旧友(きゅうゆう)に出会った 
\\	[((口))
\\	出勤(しゅっきん)の途中で事故に出会った 
\\	ああした事件にはめったに出会わない 
\\	思わぬ困難に出会った 
\\	[((文))
\\	提案	ていあん	
\\	提案する 
\\	両国間の貿易(ぼうえき)拡大の提案が可決(かけつ)された 
\\	青木氏から動議の提案があった 
\\	早く出発することを提案します 
\\	議長の選出(せんしゅつ)方法について提案させてください 
\\	提案者 
\\	〔動議の〕
\\	提案理由 
\\	定期	ていき	
\\	定期 (的) の 〔周期的な〕
\\	〔規則正しい〕
\\	検針係(けんしんがかり)は定期的にガスのメーターを調べに来る 
\\	定期入れ 
\\	定期刊行物 
\\	定期刑 
\\	定期券 〔定期乗車券〕
\\	((英)) 
\\	〔音楽会などの定期入場券〕
\\	定期検診(けんしん) 定期検診を受ける 
\\	定期昇給 
\\	定期船 
\\	遠洋航路の定期船 
\\	定期戦 
\\	定期点検 
\\	定期取引 
\\	定期便 
\\	〔バス・トラックなどの〕
\\	〔航空便の〕
\\	定期保険 
\\	定期預金(よきん) 
\\	100万円を定期預金にした 
\\	抵抗	ていこう	
\\	〔逆らうこと〕
\\	抵抗する 
\\	彼らは敵軍に頑強に抵抗した 
\\	激しい[弱い/非暴力の]抵抗 
\\	運命に抵抗しても無駄だ 
\\	/((米口)) 
\\	〔反発(はんぱつ)〕 彼のあの態度に私は少々抵抗を感じる 
\\	[((口))
\\	〔物理で,反対に作用する力〕 空気抵抗 
\\	電気抵抗 
\\	この種の金属は熱に抵抗する力がある 
\\	抵抗運動 
\\	第二次大戦中の地下組織などによる) 抵抗器 〔物理で〕
\\	抵抗値(ていこうち) 
\\	提出	ていしゅつ	
\\	提出する 
\\	〔議案などを〕
\\	〔付託(ふたく)する〕
\\	〔手渡す〕
\\	彼らは会議に提出する書類を作成中だ 
\\	作成 
\\	願書の提出は2月5日(いつか)まで 
\\	時間です
\\	答案を提出しなさい 
\\	[((米))
\\	彼は証拠(しょうこ)の提出を求(もと)められた 
\\	野党は不信任案(ふしんにんあん)の動議を提出した 
\\	彼は辞表(じひょう)を提出した 
\\	提出者 〔書類の〕
\\	〔案の〕
\\	提出物 〔書類〕
\\	〔学生が教師に出す〕
\\	〔論文〕
\\	停電	ていでん	
\\	程度	ていど	
\\	〔度合い〕
\\	と
\\	との違いは程度の違いで種類の違いではない 
\\	〔標準〕
\\	〔高さ〕
\\	この学校は彼には程度が高過ぎる 
\\	高校卒業程度の学力 
\\	程度の低い話題 
\\	お前の頭の程度が知れるよ 
\\	はこの場合皮肉) しれる 
\\	〔強弱の度合〕
\\	〔範囲〕
\\	我慢できない程度まで 
\\	程度の差こそあれ二人ともゴルフに熱心だ 
\\	こそあれ 
\\	程度の差はあっても 
\\	危機当事国は、多くの場合程度の差はあっても構造(こうぞう)的な脆弱性(ぜいじゃくせい)を抱(かか)えており、この根源的(こんげんてき)な問題に取り組むことが重要である。 
\\	それはある程度本当だ 
\\	ある程度の損害はやむを得(え)まい 
\\	どの程度まで彼を信用すべきかしら 
\\	〔およそ〕 
\\	000円程度の品物 
\\	程度問題 子供をしかることのよしあしは程度問題だ 
\\	善し悪し 
\\	停留所	ていりゅうじょ	
\\	敵	てき	
\\	出来事	できごと	
\\	日常のできごと 
\\	思いがけないできごと 
\\	今日のできごと 
\\	その日は大したできごともなかった 
\\	適する	てきする	
\\	(適任である)
\\	〔よく合う〕
\\	その服装は山登りに適していない 
\\	彼の指示はその場に適さなかった 
\\	この本は初心者に適している 
\\	それは自分に適した仕事ではない 
\\	ここは静かに暮らすには適した所ではない 
\\	この辺の気候は私に適している 
\\	〔ふさわしい能力がある〕 彼は医者には適さない 
\\	⇒てきせい(適性)
\\	適切	てきせつ	適切な 〔目的・機会などに適合した〕
\\	〔目的・状況・規準・礼儀に合った〕
\\	〔ふさわしい〕
\\	〔ぴったりの〕
\\	適切な例 
\\	適切な処置をとる 
\\	この感動を表す適切な言葉が見当たらない 
\\	この題は詩の内容から見て適切だ 
\\	批評は適切だった 
\\	彼の言うことはいつも適切だ 
\\	適度	てきど	⇒ほどほど(程程) 適度に運動する 
\\	飲食は適度にしなさい 
\\	今後はたばこも適度にしよう 
\\	適用	てきよう	
\\	彼らは生活保護法の適用を受けている 
\\	この語の適用範囲(はんい)は非常に広い 
\\	法規(ほうき)第2条(じょう)がこの件に適用される 
\\	同じことが次のような場合にも適用される 
\\	手品	てじな	
\\	ですから	ですから	
\\	鉄	てつ	
\\	哲学	てつがく	
\\	手伝い	てつだい	
\\	徹底	てってい	徹 
\\	底 
\\	徹底した 〔徹底的にやる〕
\\	〔根っからの〕
\\	〔度し難い〕
\\	〔どうしようもない〕
\\	徹底した無神論者(むしんろんしゃ) 
\\	徹底した利己主義者 
\\	徹底した楽天家 
\\	徹底した怠け者 
\\	彼は徹底して遊興(ゆうきょう)にふけった 
\\	徹底的な調査が必要だ 
\\	徹底的な改革(かいかく)が望(のぞ)まれる 
\\	徹底的な変革は時期尚早(しょうそう)だ 
\\	尚早 
\\	問題の徹底的な研究が始められた 
\\	彼は何かやり始めたら徹底的にやる 
\\	相手を徹底的に負かしてやろう 
\\	命令を徹底させる 
\\	連絡(れんらく)が徹底していない 
\\	会社の窮状(きゅうじょう)をみんなに徹底させたい 
\\	鉄道	てつどう	
\\	徹夜	てつや	
\\	手間	てま	
\\	〔掛かる時間〕
\\	〔労力〕
\\	この子は手間が掛かる[掛からない] 
\\	この機械のおかげで大いに手間が省(はぶ)けた 
\\	あなたのおかげで毎日買い物をする手間が省けます 
\\	彼女は手間の掛かる料理を作ることを苦(く)にしない 
\\	手間仕事 〔手間の量により支払われる仕事〕
\\	〔手間の掛かる仕事〕
\\	手間賃(てまちん) 
\\	手間賃を払う 
\\	手間暇(てまひま) 
\\	そんなつまらぬ仕事に手間暇かけてはいられない 
\\	典型	てんけい	
\\	〔模範(もはん)〕 
\\	女性の典型 
\\	〔代表的なもの〕 彼女は戦前の道徳的(どうとくてき)日本女性の典型だった 
\\	彼は戦後の作家の典型だ 
\\	典型的な英国紳士(しんし) 
\\	天候	てんこう	
\\	電子	でんし	
\\	伝統	でんとう	
\\	〔因習〕
\\	伝統的 (に) 
\\	長く培(つちか)われて来た伝統 
\\	伝統に従(したが)う 
\\	伝統を守る 
\\	伝統を重(おも)んじる人々 
\\	伝統を破る 
\\	我が社は80年の伝統を持つ 
\\	この祭はその国の伝統的な行事に数えられる 
\\	この学校では伝統的に規律がやかましい 
\\	伝統主義 
\\	伝統主義者 
\\	天然	てんねん	
\\	⇒しぜん(自然) 天然の 
\\	天然の美 
\\	天然のままの岩 
\\	天然ガス 
\\	天然ガス(自動)車 
\\	天然記念物 
\\	〔動植物の場合〕
\\	天然魚 
\\	天然資源 
\\	天然色 
\\	〔映画の〕
\\	天然パーマ 
\\	彼女の髪は天然パーマだ 
\\	問い	とい	
\\	党	とう	
\\	塔	とう	
\\	答案	とうあん	〔解答〕
\\	〔用紙〕
\\	答案を出す 
\\	白紙答案 
\\	答案に赤い鉛筆で印をつける 
\\	答案はペンで書きなさい 
\\	同一	どういつ	
\\	二者を同一とみる 
\\	それは僕が無くした万年筆と同一のものだ 
\\	しかし,無くしたそのペンではない) それは私が探していたのと同一の女性だと判明した 
\\	同一水準に 
\\	同一視 
\\	と
\\	を同一視する 
\\	同一人物 
\\	同一性 
\\	銅貨	どうか	
\\	当時	とうじ	
\\	そのころ)
\\	動詞	どうし	
\\	同時	どうじ	
\\	どうしても	どうしても	
\\	〔必ず〕 どうしても今日じゅうに終えなければならない 
\\	どうしても他人のことより自分のことを先に考えてしまう 
\\	〔どういうふうにしても〕 どうしてもあの人の名前を思い出せない 
\\	どうしても避けられない運命だった 
\\	あの人はどうしてもうそが言えない 
\\	彼はどうしても譲(ゆず)らなかった 
\\	どうぞよろしく	どうぞよろしく	
\\	到着	とうちゃく	
\\	道徳	どうとく	
\\	投票	とうひょう	
\\	東洋	とうよう	
\\	同様	どうよう	
\\	童謡	どうよう	
\\	同僚	どうりょう	
\\	道路	どうろ	
\\	通す	とおす	
\\	〔通行させる〕
\\	走行(そうこう)中の車は道をあけて救急車を通した 
\\	ちょっと通してください 
\\	〔中へ入れる〕 我々は客間に通された 
\\	秘書は大臣(だいじん)の部屋に通してくれなかった 
\\	〔突き抜けさせる〕 針に糸を通す 
\\	穴に綱(つな)を通す 
\\	物置に風を通す 
\\	〔向う側まで届かせる〕 婦人(ふじん)は薄いカーテンを通して外(そと)を眺めていた 
\\	光が壁の透(す)き間を通して差し込んだ 
\\	砂(すな)はよく水を通す 
\\	このコートは雨を通さない 
\\	ビニールは電気を通さない 
\\	グラスファイバーは熱を通さない 
\\	火を通す 
\\	/〔煮る〕
\\	/〔焼く〕
\\	/〔オーブンで〕
\\	牛肉にはよく火を通して下さい 
\\	この肉にもう一度よく火を通してください 
\\	消毒(しょうどく)のためふきんを熱湯(ねつゆ)に通す 
\\	〔終わりまで見る〕 本に目を通す 
\\	請求書(せいきゅうしょ)[手紙]に目を通す 
\\	〔合格させる〕
\\	先生はお情(なさ)けで僕を通してくれた 
\\	〔続ける〕 私は3日間泣き通した 
\\	彼は3年間皆勤(かいきん、
\\	で通した 
\\	会議は3時間通して行われた 
\\	夜を通して彼と話し合った 
\\	彼は一生独身で通した 
\\	〔議案などを通過させる〕 国会はその議案を通した 
\\	〔経由する〕 人を通して聞いたところ彼は元気でやっている 
\\	それが世論調査を通して見た若い人たちの傾向(けいこう)である 
\\	〔やり抜く,貫く〕 自分の主義を通す 
\\	自分の意見を通す 
\\	彼は無知で通した 
\\	彼は聖人君子(せいじんくんし)で通した 
\\	彼女は我意(がい)を通そうとする 
\\	無理を通すことは出来ない 〔無理なことを押し通す〕
\\	/〔精神的・肉体的に〕
\\	筋を通して話せば分かる 
\\	辛抱(しんぼう)し通すことは出来なかった 
\\	通り過ぎる	とおりすぎる	
\\	都会	とかい	
\\	田舎に対して) 都会の 
\\	都会的 〔洗練(せんれん、 
\\	された〕
\\	都会の人口 
\\	大都会の魅力 
\\	彼の話し方は都会的だ 
\\	彼は都会育ちだ 
\\	都会化 
\\	都会人 
\\	都会生活 
\\	都会地 
\\	溶く	とく	
\\	塩を水で溶く 
\\	解く	とく	
\\	〔ほどく〕 
\\	縄(なわ)を解く 
\\	包みを解く 
\\	靴ひもを解く 
\\	もつれを解く 
\\	彼女は彼の額の包帯(ほうたい)を解いてやった 
\\	旅装(りょそう)を解く 
\\	〔問題を解決する〕
\\	三次(さんじ)方程式を解く 
\\	〔誤解などをなくす〕 誤解を解く 
\\	彼の怒りを解くのに苦労した 
\\	〔解除する〕 契約を解く 
\\	社長は彼の兼務(けんむ)を解いた 
\\	兼務 
\\	---彼は取締役(とりしまりやく)と販売部長を兼務している 
\\	停年により職を解かれた 
\\	戒厳令(かいげんれい)を解く 
\\	禁令を解く 
\\	砦(とりで)の包囲(ほうい)を解く 
\\	どく		退く 
\\	〔わきへ〕
\\	〔後ろへ〕
\\	じゃまにならないようにわきへどきなさい 
\\	どいてくれ 
\\	毒	どく	
\\	得意	とくい	
\\	〔満足,自慢〕 彼女は得意の絶頂(ぜっちょう)だ 
\\	クーデターは首相(しゅしょう)を得意の絶頂から失意(しつい)の底(そこ)に突き落とした 
\\	ちょっとうまく行ったからといって得意になってはいけない 
\\	うまく行く=
\\	彼は見え透(す)いたお世辞ですぐ得意になる 
\\	彼は得意になって父親の手柄話(てがらばなし)をしたものだ 
\\	手柄話=
\\	彼は有名人と並んで写真をとっては得意になっている 
\\	〔優れていること〕 彼は算数(さんすう)が得意だ 
\\	だれでも一つぐらいは得意なものがあるものだ 
\\	英語は私の得意な学科(がっか)ではなかった 
\\	この画家(がか)は静物(せいぶつ)を得意とする 
\\	彼は得意の弁舌(べんぜつ)で相手を煙(けむり)に巻(ま)いた 
\\	〔顧客(こきゃく)〕
\\	店のお得意 
\\	この店は芸術家の間にお得意が多い 
\\	得意顔 少年は10キロの道を歩き通したと得意顔だった 
\\	先生が一言褒(ほ)めると彼はたちまち得意顔をした 
\\	得意客[先] 
\\	この店は得意先が多い 
\\	得意回り 彼は勤め先の銀行の得意回りをした 
\\	[((英))
\\	得意満面 彼は得意満面(まんめん)で釣り上げた魚を見せた 
\\	読書	どくしょ	
\\	独身	どくしん	
\\	特徴	とくちょう	
\\	日本の地形(ちけい)の著(いちじる)しい特徴 
\\	日本の夏の特徴は高温多湿(こうおんたしつ)なことだ 
\\	マンクス猫は尾がないのが特徴だ 
\\	忍耐(にんたい)と勤勉(きんべん)はミラー家の特徴だ 
\\	は個人・民族などの持続的特質) 
\\	彼の表情には際立(きわだ)った特徴がある 
\\	彼女の笑い方には特徴がある 
\\	彼は特徴のある人物だ 
\\	特徴のない荒野(あれの) 
\\	この町には特徴がない 
\\	まゆ毛が濃いことがこの人々を特徴づけている 
\\	現代の特徴的な一つの傾向(けいこう)は科学の信仰(しんこう)である 
\\	彼の音楽の特徴的な牧歌性(ぼっかせい) 
\\	牧歌 
\\	特長	とくちょう	
\\	燃料費のかからないのがこの車種(しゃしゅ)の特長だ 
\\	すぐ仕事にかかるのが彼らの特長だ 
\\	バロック音楽がこの四重奏団(しじゅうそうだん)の特長です 
\\	独特	どくとく	独特の 〔唯一無二の〕
\\	〔特有の〕
\\	彼独特の技巧 
\\	その地方独特のアクセント 
\\	彼の演技には独特の風格がある 
\\	独立	どくりつ	
\\	溶ける	とける	
\\	解ける	とける 
\\	〔ほどける〕
\\	⇒ほどける(解ける) 結(むす)び目が解けている 
\\	〔雪・氷が〕
\\	は解けて液体になること,
\\	は熱によってやわらかくなることを言う) アイスクリームが解けないうちに召し上がれ 
\\	あっ,角氷(かくごおり)がみんな解けてしまった 
\\	ツンドラが解けはじめた 
\\	小川にはった氷が解けた 
\\	〔怒りなどが消える〕 彼の怒りは解けた 
\\	彼の誤解はなかなか解けない 
\\	彼らの不仲(ふなか)はまだ解けない 
\\	〔答えが出る〕
\\	やっとそのなぞ[数学の問題]が解けた 
\\	〔制限などが除かれる〕 首都の戒厳令(かいげんれい)はまだ解けない 
\\	どこか	どこか	
\\	〔ある場所〕 どこか遠い国へ行きたい 
\\	どこか他のところを探そう 
\\	彼は東京のどこかに家を買った 
\\	〔何となく〕 彼にはどこか変わったところがある 
\\	ところが	ところが	
\\	息子は出掛けたと思っていた
\\	ところが2階で昼寝していたんだ 
\\	親切心から忠告(ちゅうこく)したところが憎(にく)まれてしまった 
\\	絵を試(ため)しに出品(しゅっぴん)してみたところが,入選してしまった 
\\	ところで	ところで	
\\	車で行ったところで式には間に合わないよ 
\\	泣いたところでどうにもならない 
\\	いくら考えたところで名案は出てきそうにもない 
\\	、、、、、、、 ところでどこで食事をしますか 
\\	ところで希望のカメラは手に入りましたか 
\\	ところでこの前の記事は一つ間違いがあったよ 
\\	登山	とざん	
\\	都市	とし	
\\	図書	としょ	
\\	年寄	としより	
\\	閉じる	とじる	
\\	⇒しめる(締める) 本を閉じる 
\\	会は6時に閉じた 
\\	店を閉じる 
\\	その日の営業を終わること,営業中止の両方に用いる) 彼は門を閉じて報道関係者を入れようとしなかった 
\\	目を閉じて音楽を聴く 
\\	途端	とたん	
\\	彼が出て行った途端に彼女が帰って来た 
\\	自転車をよけようとした途端に転(ころ)んだ 
\\	よける 
\\	避(さ)ける その途端に電話のベルが鳴った 
\\	ドアを開けた途端に猫が家に飛び込んできた 
\\	買った途端に後悔した 
\\	土地	とち	
\\	(地所)
\\	(区画)
\\	(土壌)
\\	(地域)
\\	〔大地〕
\\	人跡未踏(じんせきみとう 
\\	人跡未到)の土地 
\\	土地が肥(こ)え[やせ]ている 
\\	土地を耕(たがや)す 
\\	/〔すきなどで〕
\\	やせた土地を肥沃(ひよく)にした 
\\	〔地所,地面〕
\\	〔広い地所〕
\\	土地を買う 
\\	土地に投資(とうし)する 
\\	広大な土地を開発する 
\\	〔その地方〕
\\	土地の者 
\\	土地の名産 
\\	私はこの土地は不案内だ 
\\	ここは初めての土地です 
\\	彼はこの土地に明るい 
\\	彼は土地のなまりがある 
\\	土地改革 
\\	土地改良 
\\	土地家屋 
\\	土地家屋周旋業 
\\	土地価格 
\\	土地柄 
\\	土地鑑 土地鑑がある 
\\	土地転がし 
\\	土地収用 
\\	土地収用権 
\\	土地収用法 
\\	土地所有権 
\\	土地所有者 
\\	土地信託 
\\	土地台帳 
\\	((英)) 
\\	土地付き家屋 
\\	((米)) 
\\	土地っ子 
\\	土地売買 
\\	土地保有税 
\\	土地利用 
\\	形容詞(相当語句) 
\\	土地 移転先の土地の確保 
\\	土地(の)開発 
\\	被災した土地 
\\	土地 
\\	名詞 土地(の)価格 
\\	土地所有(権) 
\\	土地所有者 
\\	土地取引 
\\	土地の販売価格 
\\	土地の販売による収入 
\\	土地不足 
\\	土地利用 
\\	土地 
\\	動詞 その土地を(無償で)貸す 
\\	土地を借りる 
\\	土地を寄付する 
\\	土地を購入する 
\\	その他 土地が高価だ 
\\	突然	とつぜん	
\\	届く	とどく	
\\	〔到着する〕 
\\	8月4日(よっか)付けのお手紙が昨日届きました 
\\	[((英))4 
\\	彼女が出掛けた後で伝言が届いた 
\\	〔達する〕 
\\	病人のすぐ手の届く[届かない]所にラジオを置いた 
\\	⇒て(手) 天井まで届くほどの大男(おおおとこ)だ 
\\	彼の髪は肩まで届いている 
\\	彼女の声は遠くまで届く 
\\	目の届く限り一面の砂浜(すなはま)だった 
\\	彼女は30歳に手が届く 
\\	次の誕生日で) 
\\	〔行き渡る〕
\\	手入れのよく届いた庭 
\\	両親は一日中不在なので子供に目が届かない 
\\	〔願いなどが通じる〕 ついに彼の長年の思いが届いた 
\\	私の好意(こうい)は彼に届かなかった 
\\	彼女の誠意(せいい)が彼に届いた 
\\	とにかく	とにかく	
\\	〔何にせよ〕 とにかく彼に聞いてみましょう 
\\	とにかく途中で計画(けいかく)の変更(へんこう)はできない 
\\	私はとにかく最後までやります 
\\	〔いざしらず〕
\\	大昔ならとにかく,この時代にそんな迷信(めいしん)を信じる人はいないよ 
\\	飛ばす	とばす 
\\	〔飛ぶようにする〕
\\	子供は紙飛行機を飛ばした 
\\	帽子が風に飛ばされた 
\\	山田は二塁打(にるいだ)を飛ばした 
\\	[((口))
\\	〔勢いよく出す〕 矢(や)を飛ばす 
\\	弾(たま)を飛ばす 
\\	つばを飛ばす 〔吐いて〕
\\	〔しゃべりながら〕
\\	内掛けを飛ばした 
\\	うちがけ 
\\	〔飛散(ひさん)させる〕 車が泥水のしぶきを飛ばして走り去った 
\\	しぶき 
\\	〔速く走らせる〕
\\	全速力で車を飛ばす 
\\	編集長は記者を現場に飛ばして事故の取材に当たらせた 
\\	〔途中を省く〕
\\	難しいところは飛ばしてもよい 
\\	〔広める,配る,言い触らす〕 檄(げき)を飛ばす 
\\	デマを飛ばす 
\\	彼はいつも冗談を飛ばしている 
\\	〔転勤させる〕 彼は地方に飛ばされた 
\\	飛び出す	とびだす	
\\	〔急に出て来る〕 うさぎが穴から飛び出した 
\\	変な男が飛び出して来た 
\\	虎がおり(檻)から飛び出して逃げた 
\\	〔急に出て行く〕 部屋から飛び出す 
\\	憤然(ふんぜん)と部屋から飛び出した 
\\	〔突き出る〕
\\	壁から鉄骨(てっこつ)が飛び出していた 
\\	目の玉が飛び出すほど高い 
\\	目の玉が飛び出す 
\\	〔突然出現する〕 思い掛けない証言が証人の口から飛び出した 
\\	〔去る〕 ついに少年は家を飛び出してしまった 
\\	意見が合わず会を飛び出した 
\\	留める	とめる	
\\	〔固定する〕
\\	クリップでレポートを留める 
\\	ホッチキスで書類を留める 
\\	掲示板に掲示をピンで留める 
\\	2枚の板(いた)をくぎで留めた 
\\	上着のボタン[ホック]を留める 
\\	〔引き留める〕
\\	彼は警察に留められた 
\\	これ以上お留めはいたしません 
\\	〔記憶・注意する〕 それを心に留めておきなさい 
\\	そんなことを気に留める人ではない 
\\	刑事はじゅうたんの染(し)みに目を留めた 
\\	泊める	とめる	
\\	友	とも	
\\	共に	ともに	
\\	〔一緒に〕
\\	兄と共に家を出た 
\\	寝起(ねお)きを共にしている 
\\	運命を共にする 
\\	妻とは長い間苦労を共にしてきた 
\\	男女共に 
\\	本の送料(そうりょう)を代金と共に送ってください 
\\	〔そろって〕
\\	英夫も僕も共に15歳だ 
\\	2人は共に試験に落ちた 
\\	3人は共に疲れてしまった 
\\	〔同時に,…につれて〕 テレビの普及(ふきゅう)と共に方言の標準語化(ひょうじゅんごか)が促進(そくしん)されてきた 
\\	標準語 
\\	年と共に人口が増え続けている 
\\	虎	とら	
\\	取り上げる	とりあげる	
\\	〔手に取る〕
\\	受話器を取り上げる 
\\	〔採用する〕
\\	こういったトピックも時には取り上げてほしい 
\\	彼らの提案は全く取り上げられなかった 
\\	〔没収(ぼっしゅう)する〕
\\	〔職権で〕
\\	先生は生徒の漫画の本を取り上げた 
\\	彼は弁護士の資格(しかく)を取り上げられた 
\\	ひき逃げ事件を起こして運転免許を取り上げられた 
\\	〔赤ん坊を〕 看護師が赤ちゃんを取り上げた 
\\	努力	どりょく	
\\	取れる	とれる 
\\	〔離れる〕
\\	上着のボタンが取れた 
\\	シールが取れた 
\\	この染(し)みはなかなか取れない 
\\	〔消え去る〕 
\\	いつのまにか熱が取れた 
\\	この薬で痛みが取れた 
\\	彼女の一言で一座(いちざ)の緊張(きんちょう)が取れた 
\\	〔得られる〕 
\\	この地方はじゃがいもがよく取れる 
\\	この畑から麦が取れる 
\\	よい給料が取れる仕事に就(つ)きたい 
\\	〔解釈される〕 
\\	そういう意味にも取れる 
\\	それは幾通(いくとお)りにも取れる 
\\	泥	どろ	
\\	とんでもない	とんでもない	
\\	〔思いがけない〕
\\	〔途方もない〕
\\	〔不快な,無礼な〕
\\	それはとんでもない誤解だ 
\\	とんでもない値段(ねだん) 
\\	とんでもない要求 
\\	とんでもない人に通りで出(で)くわしてしまった 
\\	出くわす 
\\	彼はとんでもない策士(さくし)だ 
\\	〔相手の言ったことを否定して〕 
\\	彼に助けを求めるなんてとんでもない 
\\	彼が学者ですって
\\	とんでもない 
\\	彼が知らなかったって
\\	とんでもない 
\\	「待たせたな」「とんでもない,僕も今来たところだ」 
\\	どんなに	どんなに	
\\	〔いかに〕
\\	どんなに急いでももう間に合わない 
\\	仕事がどんなにつらくても全力を尽くします 
\\	〔どれほど〕
\\	どんなに悲しかったでしょう 
\\	ご両親がどんなに心配しておられるか分からないのか 
\\	内容	ないよう	
\\	〔中に入っているもの〕
\\	その箱の内容は何ですか 
\\	胃の内容物 
\\	〔著作物などで表現されたこと〕
\\	具体的な個々の内容を指すときは複数形) 雑誌[彼の講演]の内容 
\\	その本の内容は2ページ目に示してある 
\\	この手紙の内容を写(うつ)しなさい 
\\	伝言の内容は何ですか 
\\	事件の内容が分からない 
\\	内容の豊富(ほうふ)な議論 
\\	内容のない演説 
\\	内容のある[ない]人 
\\	この集まりを内容あるものとしたい 
\\	よくしゃべるが内容がない人だ 
\\	〔実質〕
\\	形式と内容 
\\	内容証明郵便 
\\	内容見本 〔本の〕
\\	なお	なお	
\\	〔まだ〕
\\	出発(しゅっぱつ)の時が来てもなお彼は迷(まよ)っていた 
\\	春なお浅(あさ)い山々 
\\	〔更に〕
\\	このほうがなおよい 
\\	なお一層悪いことに 
\\	憎(にく)んでもなおあまりある敵 
\\	ワープロも使えないしコンピュータはなおのこと全然だめです 
\\	なお申し添(そ)えますが私は彼とは何の関係もございません 
\\	申し添える 
\\	なお議論の余地(よち)がある 
\\	仲	なか	
\\	仲を取り持つ 
\\	/〔媒酌する〕
\\	仲がよい 
\\	仲が悪い 
\\	彼とは冗談を言い合える仲だ 
\\	二人の少年は親しい仲だ 
\\	夫婦の仲を裂(さ)こうと画策(かくさく)している 
\\	私たちの仲はだれも裂くことは出来ない 
\\	しゅうとめ(姑)との仲はうまくいっている 
\\	二人はいい仲になった 
\\	流す	ながす	
\\	(液体を)
\\	(排水を)
\\	(トイレの水を)
\\	(涙を)
\\	(流し去る)
\\	〔液体を〕 バケツの水を流す 
\\	水を流しっぱなしにする 
\\	水洗トイレの水を流す 
\\	水を流して手足を洗(あら)った 
\\	樋(とい)で雨水(あまみず)を流している 
\\	悔恨(かいこん)の涙を流した 
\\	血を流さずに暴徒(ぼうと)を鎮圧(ちんあつ)した 
\\	〔水の力で押し流す〕 
\\	材木(ざいもく)を川下(かわしも)に流した 
\\	洪水(こうずい)が橋を流してしまった 
\\	川にごみを流さないでください 
\\	〔洗い落とす〕 背中を流してあげましょう 
\\	風呂に入って汗を流しなさい 
\\	〔広める〕
\\	だれがデマを流したのか 
\\	若いころは浮(う)き名を流したものだ 
\\	浮名 
\\	日曜日には宗教(しゅうきょう)番組を流している 
\\	病院は心の安(やす)まる音楽を館内放送で流した 
\\	〔流罪(るざい)にする〕 
\\	絶海(ぜっかい)の孤島(ことう)に流された 
\\	絶海 
\\	〔あちこちを回る〕 タクシーがたくさん流していた 
\\	ギターをかかえて裏町を流していた 
\\	貧(まず)しい芸人は一軒一軒流して歌を歌った 
\\	〔所有権を失くす〕 質草(しちぐさ)を流す 
\\	〔流用する〕 
\\	公金が知事の私邸建築(していけんちく)に流されていたのが発覚(はっかく)した 
\\	〔会などを中止する〕 
\\	来週の集まりは流そう 
\\	〔気楽にやる〕 軽く100メートル流した 〔泳いだ〕
\\	/〔走った〕
\\	〔身を入れて…しない〕 聞き流す 
\\	読み流す 
\\	書き流す 
\\	例文 「何だか水道がにごっているみたいね」「本当だ。しばらく水を流しっ放(ぱな)しにしてみよう」 
\\	「毎朝ジョギングをしてるんだって?」「ああ、朝から汗を流すのは気持ちいいよ」 
\\	「昨日の豪雨(ごうう)はすさまじかったね」「川が氾濫(はんらん)して、家が流されたところもあるそうだよ」 
\\	「帽子、なくしちゃったの?」「河原(かわら)で遊んでるときに、水に流しちゃったの」 
\\	「家にいるときはいつも音楽を聞いてるの?」「聞いてるわけじゃないけど、音楽を流しておかないと何だか落ち着かなくてね」 
\\	「仕事場にこもりっきりだから、世情(せじょう)にうとくなってね」「ラジオでも流しながら仕事してみたら?」 
\\	「君、会社辞めるって本当かい?」「誰がそんな噂を流してるんだい? 困(こま)るなあ」 
\\	「運転手さん、お盆は都内じゃお客もあんまりいないでしょ」「そうさねえ、お昼から流してるけど、お客さんでやっと三人目ですよ」 
\\	「今年は黒がはやりなんですって。黒のコート買おうかな」「君ってほんとに流行に流されやすいね」 
\\	「あなたって、しっかりしていてうらやましいわ」「ただ、人の言うことに流されないようにしているだけよ」 
\\	「やっぱり温泉は気持ちいいわね」「背中を流しましょうか」 
\\	「今までのことは水に流して、これからは仲よくやりましょう」「そうだな。いつまでもこだわるのも大人げないね」 
\\	半ば	なかば	
\\	〔半分〕 半ば冗談にそう言ったのだ 
\\	仕事は半ばすんだ 
\\	彼の話の半ばはうそだった 
\\	この棚(たな)の本の半ばは辞典だ 
\\	〔ある程度〕 彼女の病気は半ば神経から来ている 
\\	この作品は半ば小説で半ば詩だ 
\\	半ばあきらめていた 
\\	〔真ん中あたり〕 橋の半ばに立つ 
\\	5月の半ばから 
\\	60代の半ばに 
\\	9月の半ばごろ 
\\	彼は夕食会の半ばに意識を失った 
\\	半ばで引き返(かえ)す 
\\	僕は学業半ばで退学した 
\\	仲間	なかま	
\\	中身	なかみ	
\\	〔中のもの〕
\\	瓶(かめ)の中身をからにする 
\\	この箱の中身は何ですか 
\\	〔刀身(とうしん)〕
\\	〔表現されたこと〕
\\	⇒ないよう(内容)2 この本は表題は良いが中身が貧相(ひんそう)だ 
\\	/((口)) 
\\	中身の薄(うす)い[濃い]話だった 
\\	食事の中身が足(た)りなかった 
\\	眺め	ながめ	
\\	丘の頂からの眺めはすばらしい 
\\	このホテルは湖の眺めがよい 
\\	眺める	ながめる	
\\	〔見つめる〕
\\	天井[床]を眺める 
\\	人の顔を眺める 
\\	一点をじっと眺める 
\\	彼は私をしげしげと眺めた 
\\	子供たちが庭で遊んでいるのを眺めた 
\\	〔見渡す〕
\\	先生は教室を眺め回した 
\\	この部屋からすばらしい夜景を眺めることができる 
\\	この窓から海が眺められる 
\\	この問題を別の角度から眺めてごらんなさい 
\\	流れ	ながれ	
\\	流れる	ながれる	
\\	亡くす	なくす	
\\	両親を一度に亡くした 
\\	両親を亡くした子供たち 
\\	父を亡くしてから3年たった 
\\	親友を亡くして悲嘆(ひたん)にくれた 
\\	殴る	なぐる	
\\	〔連続して〕
\\	〔平手で〕
\\	〔棒で〕
\\	顔を殴る 
\\	/〔鼻を〕
\\	頭を殴る 
\\	あざができるほど殴る 
\\	あざ(痣) 
\\	殴られて気絶(きぜつ)した 
\\	あごをいやというほど殴られた 
\\	嫌というほど 
\\	父親は言うことをきかない息子の横っ面(よこっつら)を殴った 
\\	言う事を聞かない 
\\	こっぴどく殴ってやれ 
\\	こっぴどい 
\\	[形]非常にひどい。手厳しい。「―・くしかられる」「―・い批評」
\\	無し	なし	
\\	なぜなら	なぜなら	
\\	あの二人は兄弟かも知れない
\\	なぜならとてもよく似ているから 
\\	謎	なぞ	
\\	納得	なっとく	
\\	〔同意〕
\\	納得する ((文)) 
\\	君の提案は納得できない 
\\	[((文))
\\	外国行きをあきらめるように納得させた 
\\	/〔文〕
\\	私たちは納得ずくで離婚した 
\\	〔理解〕 納得する 
\\	〔得心がいく〕
\\	そのことで母性愛(ぼせいあい)の力を納得した 
\\	具体的な例をあげて説明しても彼はなかなか納得しないだろう 
\\	納得がいくように説明してやった 
\\	この説明で納得がいきますか 
\\	何か	なにか	
\\	なにも	なにも	
\\	〔どんな事・物も〕 私は関係ないから何も言えません 
\\	暑さも何も忘れて仕事に精(せい)を出した 
\\	彼は肉類(にくるい)は何も食べない 
\\	何も心配することはない 
\\	〔特に〕 何もこんな遅(おそ)くに外出する必要はないだろう 
\\	何も私だけが悪いんではない 
\\	鍋	なべ	
\\	生	なま	
\\	〔料理する前の状態〕 生の 〔火を通していない〕
\\	〔生煮え・生焼けの〕
\\	名詞の前ではハイフンをつける); 〔塩をしていない〕
\\	生の牛肉 
\\	生のキャベツ 
\\	魚を生で食べる 
\\	この豚肉(ぶたにく)はまだ生だ 
\\	〔乾燥していない〕 生の薪(まき)がいぶっている 
\\	燻る 
\\	いぶる 
\\	〔放送(ほうそう)などが録画(ろくが) (音) でないこと〕 生の 
\\	生の放送[演奏(えんそう)] 
\\	生の音楽 
\\	開会式の模様(もよう)が生で放送された 
\\	〔手書き〕 生の原稿(げんこう) 
\\	〔作為を施さない〕 読者の生の声を聞きたい 
\\	その話は目撃者(もくげきしゃ)から生で聞いた 
\\	〔未熟練(みじゅくれん)〕 
\\	あの石工(いしく)は腕が生だ 
\\	この作品はまだ生だ 
\\	〔鈍いこと〕 生の 
\\	この刃(は)は生だ 
\\	〔生ビール〕 生を一杯やる 
\\	[((英))
\\	〔生意気〕 
\\	生を言うな 
\\	生コンクリート 
\\	生卵 
\\	生物 ⇒詳細を見る 生野菜 
\\	生ワクチン 
\\	怠ける	なまける	
\\	〔怠惰である〕
\\	〔怠(おこた)る〕
\\	彼は学業を怠けている 
\\	仕事[宿題]を怠けてはいけない 
\\	怠けている暇などない 
\\	一日中怠けていてほしくない 
\\	〔ずる休みする〕 学校を怠ける 
\\	会社を怠ける 
\\	波	なみ	
\\	涙	なみだ	
\\	悩む	なやむ	
\\	〔心を痛める〕
\\	良心(りょうしん)の呵責(かしゃく)に悩んでいる 
\\	彼は事業の失敗をひどく悩んでいる 
\\	恋に悩む若者 
\\	貧困(ひんこん)に悩む家族 
\\	仕事が見つかるかどうか悩んでいる 
\\	〔困る〕
\\	借金に悩んでいる 
\\	近くの工場の騒音に悩んでいる 
\\	会社は資金不足に悩んでいる 
\\	〔肉体的に苦しむ〕
\\	リューマチに悩んでいる 
\\	鳴らす	ならす	
\\	なる(生る)	なる	
\\	この木は実(み)がなりますか 
\\	オレンジがたわわになっていた 
\\	[((文)) 
\\	撓(たわ)わ 
\\	馴れる	なれる	
\\	よくなれた[なれない]猿 
\\	なれた馬 
\\	縄	なわ	
\\	何で	なんで 
\\	なぜ	
\\	〔疑問〕
\\	なぜ遅刻(ちこく)したか言いなさい 
\\	なぜ彼女を疑うのですか 
\\	なぜ悲しそうな顔をしているのか 
\\	なぜそんな遠くへ引っ越すことになったのか 
\\	〔理由〕 あの二人は兄弟かも知れない
\\	なぜならとてもよく似ているから 
\\	なぜって言えないけど,あの人は嫌いなの 
\\	何でも	なんでも	
\\	〔どれでも〕
\\	何でも好きなものを召し上がれ 
\\	彼女は何でも出来る人だ 
\\	本なら何でもいい 
\\	〔どんなことでも〕
\\	何でも一番よいと思う処置(しょち)をとれ 
\\	何でもいいから,すぐここに来(き)なさい 
\\	〔聞くところによると〕  
\\	何でも彼女はフランスへ行くそうですよ 
\\	何でも彼は道で大金を拾(ひろ)ったという話だ 
\\	何でもかんでも 
\\	何でもかんでもやりたがる 
\\	何でもかんでもそれを手に入れるんだ 
\\	何とか	なんとか	
\\	〔どうにか〕 何とか間に合った 
\\	何とかして手に入れたい 
\\	何とか宿題を仕上げた 
\\	少ない年金で何とか暮らしている 
\\	狭(せま)い木戸(きど)を何とかくぐり抜けた 
\\	かろうじてなんとかやっていける〔経済的に〕 
\\	かろうじて 
\\	やって行く 
\\	〔なんでもよいからなにか〕 何とかしていただけませんか 
\\	何とか忠告(ちゅうこく)してあげなさい 
\\	何とかかんとか口実(こうじつ)をつけて金を払わなかった 
\\	何とかかんとか 
\\	〔名前をはっきり言わないとき〕 
\\	何とかいう人 
\\	あの何とかというまぬけな野郎 
\\	間抜(まぬ)けな 
\\	あの人は木村何とかさんといいましたね 
\\	何とかいう物〔名前を忘れたり,知らなかったりする物〕 
\\	の発音がつまってできた語)/
\\	あの何とかいう物を持ってきて 
\\	似合う	にあう	
\\	〔適する〕
\\	〔つり合う〕
\\	その上着は彼によく似合う 
\\	君は赤が似合う 
\\	彼は口ひげがとてもよく似合う 
\\	そのネクタイはこの場合には似合わない 
\\	帽子がドレスによく似合っていた 
\\	女の子に似合わずフェンシングが上手だ 
\\	悪口を言うなんて君にも似合わない 
\\	舞台で歌うのが彼女に似合っている 
\\	それは裁判官に似合わぬ行いだ 
\\	煮える	にえる	
\\	豆(まめ)がよく[軟らかく]煮えた 
\\	いもが煮えている 
\\	このにんじんはすぐ煮える 
\\	このかぼちゃはまだよく煮えていない 
\\	スープが煮えてきた 
\\	苦手	にがて	
\\	〔いやな相手〕 あいつは苦手だ 
\\	〔不得意なもの〕
\\	人とうまく接するのが苦手だ 
\\	私の苦手は物理(ぶつり)だ 
\\	人に物を頼むのは苦手だ 
\\	そのような役は苦手だ 
\\	握る	にぎる	
\\	〔指でつかむ〕
\\	はともに握る意味で,その度合は
\\	が一番強く,
\\	が一番弱い
\\	は一般的な語で,
\\	は急に強くつかむ) 手を握る 〔握手する〕
\\	/〔つかむ〕
\\	/〔こぶしを作る〕
\\	車のハンドルを握る 
\\	彼の100メートル競走(きょうそう)には手に汗を握った 
\\	手に汗を握る 
\\	〔握りずしなどを作る〕 
\\	すし[おにぎり]を握る 
\\	〔自分のものとする〕 
\\	権力(けんりょく)を握っている 
\\	来たる選挙で保守党(ほしゅとう)が権力を握るだろう 
\\	何か私たちの秘密を握っているようだ 
\\	日	にち	
\\	日常	にちじょう	
\\	日光	にっこう	
\\	日中	にっちゅう	
\\	にっこり	にっこり	
\\	入場	にゅうじょう	
\\	〔観客などの〕
\\	入場する 
\\	〔観客が〕
\\	入場は成年者に限る 
\\	12歳未満は入場お断り〔掲示〕 
\\	無用の者入場を禁ず〔掲示〕 
\\	入場無料〔掲示〕 
\\	入場券 
\\	〔駅の〕
\\	無料入場券 
\\	入場券売場 
\\	/((英)) 
\\	入場行進 
\\	入場式 
\\	入場者 〔会場に来た人〕
\\	〔見物人〕
\\	〔総称〕
\\	多数の[500名の]入場者があった 
\\	入場料 
\\	〔入場料総額〕
\\	入場料を1,000円取る 
\\	000円の入場料を払う 
\\	その映画の入場料はいくらか 
\\	煮る	にる	
\\	〔とろ火で〕
\\	じゃがいもを煮る 
\\	野菜をとろ火で煮る 
\\	かゆをぐつぐつ煮る 
\\	いためてから煮た肉 
\\	煮て食おうと焼いて食おうと あいつは煮て食おうと焼いて食おうとおれの勝手だ 
\\	煮ても焼いても食えない 煮ても焼いても食えないやつ 
\\	煮るなり焼くなり 煮るなり焼くなり勝手にしろ 
\\	人気	にんき	
\\	人間	にんげん	
\\	抜く	ぬく	
\\	(引き抜く)
\\	(栓を)
\\	(除(のぞ)く)
\\	(省略する)
\\	(追い越す)
\\	(走って)
\\	-ぬく 
\\	彼女は一生働き抜いた 
\\	これは迷い抜いたあげく決めたことです 
\\	〔引いて取る〕
\\	刀を抜く 
\\	歯を抜いてもらう 
\\	とげを抜く 
\\	ぶどう酒の栓(せん)を抜く 
\\	ラジオのプラグを抜く 
\\	〔選び出す〕
\\	棚の本の中から1冊を抜いた 
\\	彼の論文の一節を抜いて引用した 
\\	〔除き去る〕
\\	しみを抜く 
\\	タイヤの空気を抜く 
\\	風呂の水を抜く 
\\	彼女は籍(せき)を抜いた 
\\	〔こっそり盗み取る〕 人ごみで財布を抜かれた 
\\	〔省く〕
\\	朝食(ちょうしょく)を抜く 
\\	これは抜かしましょう 
\\	この品は製造に手が抜いてある 
\\	〔追い越す〕
\\	後ろの車に抜かれた 
\\	はレースを想定する) 我々のチームはラストスパートで相手チームを抜いて勝った 
\\	両者は抜きつ抜かれつの大接戦(せっせん)を演(えん)じた 
\\	抜きつ抜かれつ 
\\	追い越したり、追い越されたりして、激しく先後を争うさま 
\\	〔相手より優れる〕⇒まさる(勝る) フェンシングでは群を抜いていた 
\\	抜ける	ぬける	
\\	〔離れて取れる〕
\\	歯が抜けた 
\\	箱の底(そこ)が抜けてしまった 
\\	最近髪の毛がよく抜ける 
\\	取っ手が抜けた 
\\	〔抜け出る〕
\\	寝床(ねどこ)から抜ける 
\\	午前中は会社から抜けられません 
\\	〔あるべきものが洩(も)れる〕 3語抜けていた 
\\	名簿(めいぼ)から3人の名が抜けていた 
\\	〔通り抜ける〕
\\	打球(だきゅう)は一塁(いちるい)手の足元を抜けた 
\\	トンネルを抜けると目の前に富士山が見えた 
\\	台風は日本海に抜けた 
\\	〔やめる〕 その会から抜けた 
\\	〔なくなる〕 一度くせがつくとなかなか抜けないものだ 
\\	一風呂(ふろ)浴びたら疲れがすっかり抜けた 
\\	息子に死なれてすっかり張り合いが抜けてしまった 
\\	張り合い 
\\	このビールは気が抜けている 
\\	〔知恵(ちえ)が足りない〕 抜けた男 
\\	彼はちょっと抜けている ((口)) 
\\	/((口)) 
\\	彼は大学者だが日常のこととなると抜けている 
\\	[((口))
\\	抜けるよう 抜けるような青空 
\\	抜けるように肌(はだ)の色が白い 
\\	布	ぬの	
\\	ぬらす	濡らす	
\\	〔湿らす〕
\\	スポンジを水でぬらす 
\\	涙にほおをぬらしながら謝(あやま)った 
\\	雨で靴をぬらしてしまった 
\\	髪をぬらさないようにする 
\\	ふきんをぬらしてテーブルを拭く 
\\	根	ね	
\\	願い	ねがい	
\\	願う	ねがう	
\\	〔望む〕
\\	〔欲する〕
\\	〔期待する〕
\\	世界平和を願う 
\\	ご成功を願っております 
\\	今後ともよろしくお願いします ((文)) 
\\	英語では改まった商業文ぐらいにしか用いない) 
\\	〔依頼する〕
\\	〔懇願する〕
\\	((文)) 
\\	お願いしたいことがあります 
\\	お手伝いをお願いできますか 
\\	遅刻しないように願います 
\\	ちょっと荷物[子供]をお願いします 
\\	田中さんをお願いします 〔電話で〕
\\	/〔受付で〕
\\	もっと大きい声でお願いします 
\\	この手紙を読んで下さるようお願いします 
\\	ご助言を願いたいのですが 
\\	願ったり叶(かな)ったり それは願ったりかなったりだ 
\\	願ってもない 願ってもない幸せです 
\\	願ってもないチャンスだ 
\\	鼠	ねずみ	
\\	熱帯	ねったい	
\\	熱中	ねっちゅう	
\\	熱中する[している] 
\\	〔専念する〕
\\	⇒ねつ(熱)
\\	研究に熱中している 
\\	このごろは切手集めに熱中しなくなった 
\\	ロックに熱中している ((口)) 
\\	かけ事に熱中して家庭を顧(かえり)みなかった 
\\	革命思想に熱中したころもあった 
\\	あまり熱中しないうちにやめておきなさい 
\\	年間	ねんかん	
\\	年月	ねんげつ 
\\	としつき	
\\	年中	ねんじゅう	
\\	〔一年中〕 この島は年中暖かい 
\\	〔始終〕
\\	彼は年中忙しいと言っている 
\\	年中無休 当店は年中無休です 
\\	年中無休で働いている 
\\	年代	ねんだい	
\\	年齢	ねんれい	
\\	〔個々の年齢〕
\\	平均年齢 
\\	精神年齢 
\\	ジャックとビルは同年齢だ 
\\	年齢不詳の女性 
\\	いろいろな年齢の子供たちがいた 
\\	あの年齢にしては若く見える 
\\	彼らと話をすると年齢の差を感じる 
\\	[((文)) 
\\	そのマラソンレースは年齢に関(かか)わりなく参加出来る 
\\	我々は年齢別に分けられた 
\\	60歳だが肉体的年齢は40歳だと医者に言われた 
\\	年齢給 
\\	年齢制限 
\\	あなたは年齢制限(せいげん)を越えている 
\\	年齢層(ねんれいそう) 
\\	彼の年齢層の人たちはもう古い 
\\	この映画は20歳(はたち)から30歳の年齢層に受けている 
\\	野	の	
\\	能	のう 
\\	⇒のうりょく(能力),さいのう(才能) 人の上に立つ能がない 
\\	あてがわれた仕事をするよりほかに能のない男だった 
\\	宛(あて)がう 
\\	彼は何の能もない男だ 
\\	言葉を飾(かざ)るだけが能ではない 
\\	能がない 
\\	〔効能 
\\	今さら古い議論を蒸し返しても能のないことだ 
\\	能書き ⇒詳細を見る 
\\	〔能楽〕
\\	能の舞台 
\\	⇒のうがく(能楽),のうきょうげん(能狂言) 能ある鷹(たか)は爪を隠す ((諺)) 
\\	能装束 
\\	能舞台 
\\	農家	のうか	
\\	農業	のうぎょう	
\\	農民	のうみん	
\\	能力	のうりょく	〔物事を成し遂げる力〕
\\	〔特定のことをする能力〕
\\	〔適性〕
\\	〔潜在的能力〕
\\	支払い能力 
\\	生産能力 
\\	知的能力 
\\	能力のある男 
\\	彼にはその仕事をやる能力がない 
\\	貧しい父親に賠償能力はなかった 
\\	私にはその問題を解く能力はない 
\\	そのホールは2,000名の収容能力がある 
\\	彼に能力以上のことを期待しても無駄だ 
\\	能力開発 
\\	能力給 
\\	⇒のうりつきゅう(能率給) 能力主義 
\\	能力テスト 
\\	軒	のき	
\\	残す	のこす	
\\	〔後に置いておく〕
\\	彼は妻と二人の幼い子供を残して死んだ 
\\	残された家族 ((文)) 
\\	あなたにこの包みを置いて行きました 
\\	彼は財産の半分を次男に残した 
\\	妻子(さいし)を故郷に残して東京に出た 
\\	〔後に留める〕 先生は罰(ばつ)として二人の生徒を残した 
\\	〔済(す)んでいない〕 まだ問題が2,3残されている 
\\	〔後世(こうせい)に伝える〕
\\	偉大な音楽家として名を残した 
\\	戦争の悲惨(ひさん)さを物語る記録を後世(こうせい)に残す 
\\	〔余らす〕
\\	〔とっておく,金をためる〕
\\	総会まであと1週間を残すのみとなった 
\\	仕事を明日に残しておこう 
\\	何も残す余裕(よゆう)がない 
\\	老後(ろうご)のために金を残した 
\\	紙を1枚も残さず使ってしまった 
\\	10万円のうち3万円残して後は本を買った 
\\	〔相撲(すもう)で〕
\\	土俵際(どひょうぎわ)で辛(かろ)うじて残した 
\\	残り	のこり	
\\	乗せる	のせる	
\\	乗せる 
\\	載せる 
\\	〔手などを貸して乗物などに〕
\\	〔自分の車などに〕
\\	〔途中で拾う〕
\\	子供に手を貸(か)してスキーリフトに乗せてやった 
\\	彼の車に乗せてもらった 
\\	うちまで乗せてあげよう 
\\	タクシーはホテルでアメリカ人を二人乗せた 
\\	その飛行機は首相の一行を乗せていた 
\\	彼のお母さんを大阪行きの列車に乗せてあげるよう頼まれた 
\\	〔伝達手段によって伝える〕 大統領の教書(きょうしょ)を電波に乗せて送る 
\\	〔調子に合わせる〕 カラオケに乗せて歌う 
\\	〔順調に進める〕 計画を早く軌道(きどう)に乗せたい 
\\	〔欺(あざむ)く〕
\\	まんまと口車(くちぐるま)に乗せられた 
\\	〔仲間に入れる〕 私もその仕事に一口(ひとくち)乗せてくれ 
\\	/((口)) 
\\	載せる	のせる	
\\	〔上に置く〕 植木鉢(うえきばち)を花台(かだい)に載せる 
\\	子供をひざに載せてあやしていた 
\\	〔車などに積む〕
\\	旅行者の荷物を載せた車 
\\	スキーを車の屋根に載せて出掛けた 
\\	荷物をカートに載せて税関(ぜいかん)の方へ行った 
\\	〔記事などを掲載する〕
\\	今日の新聞は面白い記事を載せている 
\\	それは記録に載せるほどの出来事ではない 
\\	この辞書は用例を沢山載せている 
\\	新聞はわざとその事件を載せなかった 
\\	のぞく	のぞく	のぞく
\\	覗く・×覘く】 
\\	〔すき間などから見る〕
\\	部屋をのぞく 
\\	隙間[すきま]からのぞく 
\\	だれかが窓から外をのぞいている 
\\	顕微鏡(けんびきょう)をのぞく 
\\	彼女は日に何度も鏡(かがみ)をのぞく 
\\	〔一部が見える〕
\\	スカートからスリップがのぞいている 
\\	ポケットから札束(さつたば)がのぞいていた 
\\	〔立ち寄って様子を見る〕
\\	お帰りにちょっと私のところをのぞいてください 
\\	店をのぞいて見ようよ 
\\	〔少しだけ知る〕 スペイン語はのぞいただけだ 
\\	この本はちょっとのぞいて見たことがある 
\\	除く	のぞく	
\\	〔取り去る〕
\\	障害を除く 
\\	芝生の雑草を除く 
\\	〔除外する〕
\\	〔省く〕
\\	最初の100ページは除いてよろしい 
\\	無関係の項目(こうもく)を除く 
\\	1題を除いて全部できていた 
\\	望み	のぞみ	
\\	望む	のぞむ	
\\	〔願望する〕
\\	((文)) 
\\	彼は君がそれをすることを望んでいる 
\\	画家になりたいと望んでいる 
\\	君が望むなら一緒に行きましょう 
\\	君が望むようにしたらよい 
\\	それこそ望むところです 
\\	諸君(しょくん)が全力を尽(つ)くすことを望みます 
\\	娘の望む人と結婚させた 
\\	名声など望んでいません 
\\	望むらくは我がチームが優勝するように 
\\	望むらくは 
\\	〔期待する〕
\\	息子に望み過ぎる 
\\	彼の回復は望めそうもない 
\\	無駄と知りながらあの人に帰って来てほしいと望んでしまうんです 
\\	〔眺める〕
\\	〔見渡す〕((場所を主語にして)) 
\\	湖(みずうみ)はここから望むと実に美しい 
\\	この部屋から松島の素晴らしい景色が望まれる 
\\	臨む 
\\	後 
\\	のち 
\\	〔あと〕 後に 
\\	その後20年にわたって 
\\	10年の後に 
\\	後になって分かったのだが彼は大金持ちだった 
\\	今日の天気は晴れ後曇りである 
\\	〔将来〕 10年の後にはこの土地の価値も随分(ずいぶん)上がるだろう 
\\	後になって間違っていたことを悟(さと)るでしょう 
\\	そのことはまた後に相談しましょう 
\\	後の世 
\\	/〔死後〕
\\	後の人々のために 
\\	後の世まで語り草になるだろう 
\\	かたりくさ 
\\	、、、、、、、、、、、、、、、、、、、、、、、、、、 あと 
\\	〔物の後部(こうぶ)〕 後の 
\\	⇒うしろ(後ろ) 後の2両を切り離す 
\\	デモ隊の一番後にいた 
\\	〔後方〕 後に 〔後の方向に〕
\\	〔後方に,おくれて〕
\\	後を振り向く 
\\	後に下がる 
\\	競走(きょうそう)の後半(こうはん)で私は他の選手のずっと後になってしまった 
\\	荷車(にぐるま)を後から押す 
\\	京都を後にして大阪に向かった 
\\	そっと彼の後をつけた 
\\	/〔尾行する〕
\\	〔その後〕 彼らが帰った後で金がなくなっているのに気付いた 
\\	1週間たった後で彼から手紙が届いた 
\\	〔今より後〕 ではまた後で 
\\	後で電話します 
\\	/((英)) 
\\	〔次〕 お風呂は君の後に入る 
\\	後から後から問い合わせがきた 
\\	お後は何にいたしましょうか〔店で〕 
\\	後の列車は何時ですか 
\\	一番後に事務所を出たのはだれだ 
\\	〔人がいなくなったのち,死後〕 後はよろしく頼みます 
\\	後には幼(おさな)い子供が3人残された 
\\	後に残された家族は途方に暮れた 
\\	後を盛大に弔った 
\\	〔後任〕
\\	後を決める 
\\	〔残り〕
\\	後は来月に支払っていただきます 
\\	もういい,後は分かる 
\\	後 (のこと) は私に任せなさい 
\\	〔それから先,さらに〕 後1週間で終わる 
\\	りんごは後2個しかない 
\\	出発まで後まだ3日ある 
\\	後2人人手(ひとで)がほしい 
\\	あと三振を一つとれば新記録だ 
\\	百歳の誕生日まであと一日というところで他界した 
\\	後がない もう後がない 〔これで終わりだ〕
\\	/〔追い詰められて〕
\\	後の祭り ⇒詳細を見る 後は野となれ山となれ 
\\	後へ引く もう後へは引けない 
\\	後を引く 
\\	〔影響が残る〕 傷の痛みがいつまでも後を引いている 
\\	あの事件の後遺症がいまだに後を引いている 
\\	〔次々に欲しくなる〕 このクラッカーは食べ出すと後を引く 
\\	伸ばす	のばす	
\\	〔長くする〕 髪を伸ばす 
\\	ぶどうがつるを伸ばし始めた 
\\	(蔓)つる 
\\	ひもをつないで3メートルに伸ばす 
\\	〔真っすぐにする〕 背筋(せすじ)を伸ばす 
\\	足を伸ばす 〔真っすぐにする〕
\\	/〔投げ出す〕
\\	腕を伸ばしたが電灯(でんとう)に届かなかった 
\\	彼は本を取ろうと手を伸ばした 
\\	ハンカチのしわを伸ばす 〔手で〕
\\	/〔アイロンで〕
\\	〔程度を高める〕 子供の才能を伸ばす教育 
\\	勢力(せいりょく)を伸ばす 
\\	売り上げを伸ばす 
\\	〔なぐり倒す〕 彼を袋だたきにして伸ばした 
\\	袋だたきにする 
\\	延ばす	のばす	
\\	〔距離・期間を長くする〕
\\	期限[営業時間]を延ばす 
\\	日本滞在を1週間延ばす 
\\	試験の時間を1時間延ばした 
\\	地下鉄を空港(くうこう)まで延ばす 
\\	〔延期する〕
\\	〔遅らせる〕
\\	出発を2日延ばした 
\\	支払いを延ばしてください 
\\	いつまでも延ばしてはいられないよ 
\\	〔平らに広げる〕
\\	パンにバターを付けて延ばす 
\\	パイ皮(がわ)を延ばす 
\\	〔薄める〕 塗料(とりょう)を水で延ばす 
\\	濃縮(のうしゅく)ジュースを5倍に延ばす 
\\	伸びる	のびる	
\\	〔長くなる〕 髪が伸びた 
\\	木の枝が伸びた 
\\	背が伸びる 
\\	背がのびたねえ 
\\	〔真っ直ぐになる〕 ゴムひもが伸びてしまった 
\\	ラーメンが伸びてしまった 
\\	〔よくなる〕 英語の成績が伸びた 
\\	売り上げが伸びた 
\\	〔体が参ってしまう〕 なぐられて伸びてしまった 
\\	そんなに働いたら伸びてしまうよ 
\\	延びる	のびる	
\\	〔距離・期間が長くなる〕
\\	申し込みの期間が延びた 
\\	医学の進歩のおかげで寿命(じゅみょう)が延びている 
\\	このごろ日が延びてきた 
\\	道がどこまでも延びていた 
\\	山並みは南北(なんぼく)に延びていた 
\\	〔延期される〕
\\	〔おくれる〕
\\	彼らの出発が3日[来月まで]延びた 
\\	〔薄く広がる〕 このコールドクリームはよく延びる 
\\	述べる	のべる	〔言明する〕
\\	〔話す〕
\\	〔表現する〕
\\	上に述べたようにそれは不可能です 
\\	私は真実を述べているのです 
\\	彼は意見を述べる機会がなかった 
\\	上る	のぼる	
\\	〔上の方へ行く〕 坂[階段]を上る 
\\	石段(いしだん)を上ったところに寺があった 
\\	温度はうなぎ登(のぼ)りに上った 
\\	その話で頭に血が上ったらしい 
\\	〔さかのぼる〕 川を上る 
\\	〔高い地位に就く〕 彼はついに王位に上った 
\\	[((文))
\\	彼は50歳で首相の位に上った 
\\	〔地方から首府へ行く〕 20歳(はたち)で都(みやこ)へ上った 
\\	〔ある数量になる〕 暴徒(ぼうと)は1万人にも上った 
\\	建築費(けんちくひ)は5,000万円に上るだろう 
\\	〔取り上げられる〕 それは話題に上ったのですか 
\\	いったん人の口に上ると消すのは難しい 
\\	〔食膳に出される〕 新鮮(しんせん)な魚が食卓(しょくたく)に上った 
\\	昇る	のぼる	
\\	日が昇る前に 
\\	天にも昇る心地(こころち)だった 
\\	載る	のる	
\\	〔上に置かれる〕 ピアノの上に一輪挿(いちりんざ)しが載っていた 
\\	〔掲載(けいさい)される〕
\\	彼の短編(たんぺん)が雑誌に載った 
\\	「着物」という語は英語の辞書にも載っている 
\\	その事件は新聞に載った 
\\	彼の名が乗客名簿(めいぼ)に載っていた 
\\	その町はどの地図にも載っている 
\\	のんびり	のんびり	
\\	のんびりとした 〔心配のない〕
\\	〔楽天的な〕
\\	〔穏やかな〕
\\	〔ゆうゆうとした〕
\\	〔のんきな〕
\\	いつものんびりとした顔をしている 
\\	のんびりとした人 
\\	この山荘(さんそう)にいるとのんびりとした気分になる 
\\	退職(たいしょく)したらのんびりと暮らそう 
\\	彼女は一日中のんびりと絵を描いている 
\\	彼はのんびりと漫画本を読んでいた 
\\	ここにいる間はのんびりしなさい 
\\	心配事(しんぱいごと)なんか忘れてのんびりできたらいいなあ 
\\	場	ば	
\\	(地点)
\\	(場面)
\\	(場合)
\\	(株の)
\\	〔場所〕
\\	⇒ばしょ(場所) たまたまその場に居合わせた 
\\	居合(いあ)わせる 
\\	その場で申し出を断った 
\\	そっと場をはずした 
\\	〔空間〕⇒ばしょ(場所) 部屋はちらかっていて足の踏(ふ)み場もなかった 
\\	〔行動の場面〕 彼は外交官として場を踏んでいる 
\\	ここが彼の活動の場だ 
\\	〔場合・状況〕 この場に及(およ)んで逃げだすとは彼もひきょうだ 
\\	この話はこの場限りにしてください 
\\	その場に必要なことなら何でもやろう 
\\	〔株の立会い場〕
\\	午前の場が立った 
\\	〔劇・舞台の場面〕
\\	3幕(まく)7場の劇 
\\	第3幕第2場で 
\\	殺しの場 
\\	〔物理〕
\\	重力[磁力]の場 
\\	場の理論 
\\	灰	はい	
\\	梅雨	ばいう/つゆ	
\\	配達	はいたつ	
\\	配達する 
\\	1日2回郵便が配達される 
\\	配達したときお金を払ってください 
\\	市内は無料配達です 
\\	配達区域 
\\	配達先 
\\	配達証明書 
\\	配達トラック 
\\	配達人 
\\	〔郵便の〕
\\	〔牛乳の〕
\\	〔新聞の〕
\\	配達不能 
\\	配達不能の手紙 
\\	配達料 
\\	俳優	はいゆう	
\\	生える	はえる	
\\	〔芽などを出す〕
\\	庭に名前のわからない植物が生えてきた 
\\	根が生える 
\\	まいた種から芽(め)がまだ生えてこない 
\\	〔植物が生長している〕
\\	この植物(しょくぶつ)は高山地帯に生える 
\\	庭一面に雑草(ざっそう)が生えている 
\\	苔(こけ)の生えた庭 
\\	〔歯や羽などが〕 この子は1本目の歯が生えた 
\\	この子は歯が生えかかっている 
\\	少年にひげが生えてきた 
\\	羽が生えそろった[まだ羽の生えていない]鳥 
\\	鹿(しか)の角は春に落ちてまた生えてくる 
\\	墓	はか	
\\	馬鹿	ばか	
\\	博士	はかせ	
\\	〔物知り〕
\\	((文)) 
\\	〔達人〕
\\	物知り博士 
\\	お天気博士 
\\	はくし【博士】 
\\	〔称号(しょうごう 
\\	((略
\\	博士の 
\\	佐々木博士 
\\	医学博士 
\\	/〔称号〕
\\	((略
\\	文学博士 
\\	((略
\\	理学博士 
\\	((略
\\	博士課程 
\\	博士課程に在籍(ざいせき)する 
\\	[((米))
\\	博士号 
\\	経済学で博士号を取る 
\\	博士論文 
\\	計る	はかる	
\\	〔数・時間などを数える〕 湯がわくまでの時間を計ってみよう 
\\	レース[走者(そうしゃ)]のタイムを計る 
\\	3分計ってくれないか 
\\	体温[血圧]を計る 
\\	〔評価する〕
\\	〔推測する〕
\\	〔予測する〕
\\	自分を中心にして物事を計るくせがある 
\\	金銭では計れない物がある 
\\	彼らの考えは計れない 
\\	今後の成り行きを計るのは難しい 
\\	ころあいを計って話を切りだした 
\\	〔だます〕
\\	まんまと計られてしまった 
\\	量る	はかる	〔数量を〕
\\	〔目方を〕
\\	その地域の雨量を量る 
\\	はかりで体重を量った 
\\	測る	はかる	
\\	〔水深を〕
\\	棒(ぼう)の長さ[家の高さ]を測る 
\\	川の水深(すいしん)を測る 
\\	土地を測る 
\\	物差しで箱の寸法(すんぽう)を測った 
\\	掃く	はく	
\\	〔ほうきで〕
\\	床を掃く 
\\	ごみを掃いて捨てる[外へ出す] 
\\	歩道を掃く 
\\	落葉を掃いて集める 
\\	〔さっと塗る〕
\\	眉(まゆ)を掃く 
\\	〔蚕の掃き立てをする〕
\\	掃いて捨てるほど 大学出は掃いて捨てるほどいる 
\\	は((米)),
\\	は((英)))
\\	吐く	はく	
\\	拍手	はくしゅ	
\\	莫大	ばくだい	
\\	爆発	ばくはつ	
\\	博物館	はくぶつかん	
\\	激しい	はげしい	〔勢いが強い〕
\\	〔はなはだしい〕
\\	〔きびしい〕
\\	〔猛烈な〕
\\	激しさ 
\\	激しい嵐[攻撃] 
\\	激しい気性 
\\	激しい言葉 
\\	激しい愛 
\\	激しい暑さ[寒さ] 
\\	激しい痛み 
\\	激しい反対 
\\	激しい競争 
\\	このあたりは交通が激しい 
\\	激しい調子で私たちにどなりちらした 
\\	激しく責める ((文)) 
\\	激しく議論する 
\\	選挙戦はいよいよ激しさを増(ま)した 
\\	はさみ	はさみ	
\\	破産	はさん	
\\	端	はし	
\\	〔末端(まったん)〕
\\	〔先端(せんたん)〕
\\	〔縁(ふち)〕
\\	ひもの端 
\\	ページの端を折(お)る 
\\	敷地(しきち)の北の端に 
\\	道(みち)の端に立っていた 
\\	いすの端に腰をかける 
\\	新聞を端から端まで読む 
\\	表(おもて)の部屋は家の端から端まであった 
\\	〔物事の一部分〕 彼の言葉の端にはとげがあった 
\\	言葉の端を捕(とら)える 
\\	〔「…する端から」の形で,…するものを次々に〕 料理する端から食べた 
\\	始まり	はじまり	
\\	外す	はずす	
\\	〔取り外す〕
\\	〔ボタンなどを〕
\\	眼鏡を外す 
\\	上着のボタンを外す 
\\	犬の鎖を外す 
\\	箱のふたを外した 
\\	ドアの鍵(かぎ)[留(と)め金(がね)]を外しなさい 
\\	彼の名は会員リストから外された 
\\	〔席などを離れる〕 席を外す 
\\	彼はいつの間にか席を外していた 
\\	彼はただいま席を外しております〔電話などで〕 
\\	〔そらせる〕
\\	相手の打撃(だげき)を巧(たく)みに外した 
\\	彼女は私の質問を外した 
\\	投手(とうしゅ)が次の一球(いっきゅう)を外した 
\\	投手(とうしゅ)はゆるい球で打者(だしゃ)のタイミングを外した 
\\	〔のがす〕 
\\	機会[的]を外す 
\\	外れる	はずれる	
\\	〔留めてあるもの・掛けてあるものが離れる〕 取っ手が外れた 
\\	ボタンが一つ外れているよ 
\\	ホックが外れた 
\\	引き戸が外れた 
\\	門扉(もんぴ)のかんぬきが外れていた 
\\	〔外へそれる〕 市街(しがい)を外れると麦畑が広がっていた 
\\	その質問はピントが外れている 
\\	ピント 
\\	彼女の歌は調子が外れている 
\\	〔当たらない〕 矢が的(まと)を外れた 
\\	くじに外れた 
\\	天気予報が外れた 
\\	期待(きたい)が外れた 
\\	[((口)) 
\\	あてが外れた 
\\	〔反する〕
\\	それは規則に外れる 
\\	彼の振る舞いは礼儀作法から外れている 
\\	道義に外れた行い 
\\	旗	はた	
\\	肌	はだ	【肌・▲膚】 
\\	〔皮膚〕
\\	彼女は肌が白い 
\\	私の肌は荒れやすい 
\\	風は肌を刺すように冷たかった 
\\	〔物の表面〕
\\	〔きめ〕
\\	黒い山肌 
\\	肌の細かい[肌の粗い]材木 
\\	〔気質〕 彼は芸術家肌である 
\\	肌が合わない 彼とは肌が合わない 
\\	/((口)) 
\\	肌で感じる[知る] 彼はベトナム戦争に従軍して戦争の悲惨さを肌で感じた[知った] 
\\	肌を許す 彼女は彼に肌を許そうとしなかった 
\\	例文 「君、肌が白いね」「日に焼けてもすぐ取れちゃうんだよ」 
\\	「最近、仕事で毎日遅いみたいね」「そうなの。睡眠不足で肌が荒れちゃって」 
\\	「どうしたの? 腕が真っ赤じゃない」「山に行ったらうるしにかぶれちゃって。肌が弱いんだよ」 
\\	「あの山、木がたくさん切られてるね」「山肌が見えちゃってるもの」 
\\	「この白木の机、素敵ね」「木肌がなめらかで、すごく気に入ってるの」 
\\	「何で彼が苦手なの? 気さくなやつじゃないか」「何となく肌が合わないんだ」 
\\	「この仕事はあまり彼に向かないんじゃないの?」「うん。彼は学者肌の人だから、交渉よりは研究に向いてるな」 
\\	「北海道は寒かっただろうな」「空港に降り立ったとたん、肌を刺すような寒さだったよ」 
\\	「彼、喫茶店を始めたのかい?」「ああ。長年の友人なんで、僕もひと肌脱いで資金援助をしたんだ」 
\\	「娘は夏休みの間、アメリカのおじのところに行ってるんですよ」「外国生活に肌で触れることができて、いい経験になりますね」 
\\	裸	はだか	
\\	畑	はたけ	
\\	働き	はたらき	
\\	発見	はっけん	
\\	発行	はっこう	
\\	〔図書・新聞などの〕
\\	発行する 
\\	今日発行の朝日新聞 
\\	各週[各月/年4回]発行の雑誌 
\\	本[雑誌/新聞]を発行する 
\\	本の発行を中止[禁止]する 
\\	〔貨幣(かへい)・債券(さいけん)などの〕
\\	発行する 
\\	切手[貨幣]の発行 
\\	貨幣[債券/証明書]を発行する 
\\	発行価格 
\\	発行銀行 
\\	発行市場 
\\	発行者[人] 
\\	発行所 
\\	発行済株式 
\\	発行高 〔紙幣・公債などの〕
\\	発行高限度 
\\	発行停止[禁止] ⇒はっきん(発禁) 発行日 
\\	発行部数 
\\	発行部数100万部の雑誌 
\\	この新聞は発行部数が多い 
\\	発行余力 〔貨幣の〕
\\	発車	はっしゃ	
\\	発射	はっしゃ	
\\	罰する	ばっする	
\\	〔スポーツなどで〕
\\	彼は強盗のかどで罰せられた 
\\	汚職(おしょく)官吏を法に照(て)らして罰しようではないか 
\\	そんなことをすれば必ず罰せられる 
\\	発達	はったつ	
\\	〔成長〕
\\	〔発展・進化〕
\\	発達する 
\\	ビタミンの不足は体の発達を妨(さまた)げる 
\\	この動物は牙(きば)[聴覚(ちょうかく)]が非常に発達している 
\\	沖合(おきあい)に発達中の低気圧(ていきあつ)は東に移動するでしょう 
\\	〔進歩〕
\\	発達する 
\\	近代言語学の発達 
\\	この国の教育は急速な発達を遂(と)げた 
\\	発達心理学 
\\	発達段階 
\\	ばったり	ばったり	
\\	〔急に倒れる様子〕 
\\	ばったり倒れる 〔音を立てて〕
\\	/〔平らに〕
\\	〔不意に出会う様子〕 
\\	公園で彼にばったり出会った 
\\	〔急に途絶(とだ)える様子〕 
\\	音はばったりとやんだ 
\\	その後彼女から便りがばったり来なくなった 
\\	発展	はってん	
\\	〔発達〕
\\	〔拡大〕
\\	発展する 
\\	この地域の発展は目を見張(みは)るばかりだ 
\\	我が国は友好国(ゆうこうこく)の発展に助力(じょりょく)するつもりだ 
\\	事態は我々に有利に発展した 
\\	その商社は海外へ発展していった 
\\	話が思いがけない方向に発展した 
\\	〔いっそう栄(は)えること〕 最近この町は非常に発展した 
\\	発展家 〔異性との交際に関して〕
\\	〔女〕
\\	発展性 〔将来性〕
\\	〔潜在能力〕
\\	発展的解消(かいしょう)
\\	この会は発展的解消をして新しい組織(そしき)となった 
\\	発展途上国(とじょうこく) 
\\	発表	はっぴょう	
\\	発明	はつめい	
\\	発明する 
\\	彼には発明の才がある 
\\	必要は発明の母である ((諺)) 
\\	この装置は彼が発明したものだ 
\\	発明家 
\\	発明対価 
\\	発明対価をめぐる訴訟を起こす 
\\	発明品 
\\	蓄音器(ちくおんき)はトマス・エジソンによって発明された 
\\	発明の才能はだれにでもあるわけではない 
\\	コンピューターの発明で現代社会は大きく変わった 
\\	派手	はで	
\\	〔はなやかな様子〕 派手な 〔人目を引く〕
\\	〔けばけばしい〕
\\	派手好みである 
\\	/〔服装が〕
\\	派手に立ち回る 
\\	派手なことをする 
\\	派手な服を着る 
\\	この服は私には派手過ぎます 
\\	派手な文体 
\\	会場は派手な飾(かざ)り付けがしてあった 
\\	〔大げさな様子〕 彼は金遣(かねづか)いが派手だ 
\\	その子は派手に泣き出した 
\\	話し合う	はなしあう	
\\	離す	はなす	
\\	〔分離する〕
\\	けんかしている男の子を離した 
\\	その男から目を離すな 
\\	その子はいたずらで目が離せない 
\\	私はこの電卓(でんたく)をいっときも離せない 
\\	いま手が離せない 
\\	/((米)) 
\\	この問題はそれとは離して考えよう 
\\	〔間をおく〕 3メートルずつ離して10本の木を植えた 
\\	もっと距離を離しておけ 
\\	テーブルを壁から離して置く 
\\	2着を2メートル離してゴールインした 
\\	その2匹の犬はけんかするから離しておくのです 
\\	放す	はなす	
\\	〔つかんでいるのをやめる〕 彼は綱(つな)を手から放した 
\\	拳銃(けんじゅう)を放せ 
\\	〔自由にする〕 犬を放す 
\\	小鳥を放す 
\\	牛を牧場(ぼくじょう)に放す 
\\	放してくれ 
\\	離れる	はなれる	
\\	〔間隔があく〕 彼は東京から100キロ離れた所に住んでいる 
\\	木が2メートルずつ離れて立っている 
\\	この絵は少し離れて見たほうがよく見える 
\\	ガソリンスタンドはここから遠く離れている 
\\	彼は両親と離れて暮らしている 
\\	兄とは大分年が離れている 
\\	〔ある場所から去る〕
\\	列車が駅を離れた 
\\	食卓を離れるときは「失礼」と言いなさい 
\\	家[部署]を離れる 
\\	列を離れた 
\\	床(とこ)を離れたのは昼過ぎだった 
\\	〔くっついていたものが分離する〕
\\	この2枚の切手はくっついてしまってどうしても離れない 
\\	〔関係がなくなる〕 その事業は私の手を離れている 
\\	彼は職を離れた 
\\	話が本筋(ほんすじ)を離れてしまった 
\\	大人になった子供は親の監督を離れる 
\\	金がなくなると友人は彼から離れていった 
\\	あの光景がいつも頭を離れない 
\\	放れる	はなれる	
\\	馬が杭(くい)から放れて逃げ出した 
\\	羽	はね	
\\	〔鳥の翼,昆虫の羽〕
\\	鳥が羽を広げた[たたんだ/はばたいた] 
\\	〔羽毛(うもう)〕
\\	〔装飾用〕
\\	〔総称(そうしょう)〕
\\	〔柔毛〕
\\	羽の生えそろった鳥 
\\	羽をむしった鳥 
\\	あの鳥は羽が抜けた[生え変わった] 
\\	これは羽のように軽い 
\\	くじゃくは羽がきれいだ 
\\	羽のついた帽子をかぶっていた 
\\	赤い羽根運動 
\\	〔機械などの〕 飛行機の羽 
\\	扇風機の羽 
\\	プロペラの羽 
\\	羽が生えたよう 羽が生えたように売れた 
\\	羽が生えたように金がなくなった 
\\	羽を伸ばす 先生のいない間に彼らは羽を伸ばしていた 
\\	羽布団 〔掛け布団〕
\\	羽ぼうき 
\\	羽まくら 
\\	はね【羽|羽根】 
\\	(全体)
\\	(羽毛)
\\	羽根	はね	
\\	幅	はば	
\\	〔横の長さ〕
\\	((文)) 
\\	幅が1メートルある 
\\	幅の広い[狭い]道路 
\\	川の幅は30フィートある 
\\	幅120センチ奥行(おくゆき)40センチ高さ185センチの戸棚(とだな) 
\\	〔余裕,ゆとり〕 規則の運用にはある程度の幅を持たせるべきだ 
\\	〔差〕 利益の幅は小さい 
\\	幅がきく 彼はこの辺では幅がきく 
\\	/((米)) 
\\	幅をきかす 軍人が幅をきかす国 
\\	母親	ははおや	
\\	省く	はぶく	
\\	〔取り除く〕
\\	その部分は映画では省いてあった 
\\	この項目(こうもく)を表から省いた 
\\	〔少なくする〕 
\\	出費(しゅっぴ)を省く 
\\	あなたのおかげで手数(てすう)が省けました 
\\	手数 
\\	時間を省く 
\\	場面	ばめん	
\\	〔舞台のシーン〕
\\	場面が変わった 
\\	〔光景〕
\\	とても楽しい場面に出会った 
\\	はやる	流行る	
\\	〔流行する〕
\\	〔人気がある〕
\\	革のコートがはやってきた[はやらなくなった] 
\\	この種の音楽が今はやっている 
\\	そのような意見が若者の間にはやっている 
\\	〔病気が広まる〕 赤痢(せきり)がはやってきた 
\\	去年は全国で流感(りゅうかん)がはやった
\\	流感 
\\	流行性感冒 疫病(えきびょう)がはやっている 
\\	猛威をふるって) 
\\	〔繁盛(はんじょう)する〕 
\\	あの医者ははやっている 
\\	あの店ははやっている 
\\	腹	はら	
\\	(胃)
\\	(腹部)
\\	【解剖】
\\	(おなか)
\\	(子宮)
\\	(本音)
\\	(度量)
\\	〔腹部〕
\\	〔胃〕
\\	〔腸〕
\\	〔子宮〕
\\	彼は腹が出ている 
\\	腹が痛い 
\\	腹が減った 
\\	腹が張る 
\\	腹が一杯だ 
\\	腹が下っている 
\\	食べすぎて腹をこわした 
\\	彼女は大きな腹をしていた〔妊娠していた〕 
\\	〔膨れた所〕 親指の腹 
\\	〔心の中〕
\\	〔意図〕
\\	〔胆力〕
\\	彼女の腹は見抜けない 
\\	腹の中はいい人だ 
\\	腹の中で笑う 
\\	彼の腹の中は見え透いている 
\\	彼は私を人々の面前でなぶりものにする腹だ 
\\	腹が癒(い)える 
\\	彼が謝るまでは私の腹は癒えない 
\\	腹が据わる 
\\	初めのうちは狼狽(ろうばい)したが,しばらくすると腹が据わってきた 
\\	腹の据わった人〔度胸のある〕((口)) 
\\	/〔意志の堅い〕 
\\	腹が立つ 腹が立って仕方がなかった 
\\	腹が膨れる 〔食べて〕
\\	〔妊娠して〕
\\	〔言いたいことを言わないで〕
\\	腹が太い 
\\	〔度量がある〕
\\	〔大胆な〕
\\	腹が減っては戦はできぬ 
\\	腹に一物 腹に一物ある 
\\	腹に収める 
\\	腹に据えかねる 彼の無礼な態度は腹に据えかねる 
\\	腹の皮がよじれる 腹の皮がよじれるほど笑った 
\\	腹の虫 
\\	腹の虫が治まらない 
\\	あれじゃ彼女の腹の虫が治まらないだろう 
\\	腹八分目 
\\	腹八分に医者いらず 
\\	腹も身の内 
\\	腹を合わせる 
\\	彼らは腹を合わせて悪事を企んでいるのではないか 
\\	腹を痛める 
\\	〔産む〕 あの女の子は私の腹を痛めた子です 
\\	〔自分の金を出す〕 あいつは絶対自分の腹を痛めるようなことはしない 
\\	腹を抱える 
\\	皆腹を抱えて笑った 
\\	腹を決める 腹を決めかねていた 
\\	腹を切る 
\\	⇒せっぷく(切腹) 
\\	〔責任をとって辞任する〕 もし計画が失敗したら腹を切らねばなるまい 
\\	腹をくくる 
\\	腹をこしらえる ⇒はらごしらえ(腹拵え)  
\\	腹を探る 
\\	痛くもない腹をさぐられた 
\\	腹を据える 
\\	食べるためならどんなつらい仕事でもやろうと腹を据えた 
\\	腹を立てる 彼はつまらないことに腹を立てる 
\\	[((口))
\\	彼が腹を立てたのを見たことがない 
\\	((英))では不定詞の
\\	はなくてもよい) 腹を割る  
\\	腹を割って話し合おう 
\\	原	はら	
\\	針	はり	
\\	範囲	はんい	〔活動・見通しなどの範囲〕
\\	〔勢力・知識などの領域〕
\\	〔力などの及ぶ限度〕
\\	〔広がりの程度〕
\\	勢力(せいりょく)範囲を広げる 
\\	警察は広範囲(こうはんい)な捜査を始めた 
\\	大統領の権限(けんげん)の範囲は法律で制限(せいげん)されている 
\\	自然界には我々の理解力の範囲を越えるものがある 
\\	彼のエッセーの内容(ないよう)は広い範囲に渡っている 
\\	予算の範囲内でやらねばならない 
\\	彼とは交際(こうさい)範囲が異なる 
\\	試験範囲は1章から5章まで 
\\	反抗	はんこう	反抗する 〔抵抗する〕
\\	〔反対する〕
\\	〔権力に挑戦する〕
\\	〔命令に従わない〕
\\	占領軍(せんりょうぐん)に反抗した 
\\	一部学生は体制側(たいせいがわ)に反抗しようと試みた 
\\	反抗的な若者 
\\	彼は先生に対して反抗的な態度を取った 
\\	反抗期 
\\	彼は反抗期だ 
\\	反抗心 
\\	犯罪	はんざい	
\\	反省	はんせい	
\\	〔内省〕
\\	〔再考〕
\\	反省する 
\\	反省してみると私は不親切だったような気がする 
\\	寝る前にその日の行(おこな)いを反省することにしている 
\\	反省しなさい 
\\	このことに関して君の反省を促(うなが)したい 
\\	〔遺憾(いかん)に思うこと〕
\\	彼は反省の色が濃い 
\\	あんなことをしておきながら反省の色が全く見えない 
\\	自分が間違っていたと深く反省しています 
\\	反省会 
\\	判断	はんだん	
\\	〔判定〕((米)) 
\\	((英)) 
\\	判断する 
\\	ご判断にまかせます 
\\	公正な判断を下(くだ)す 
\\	人を見掛けで判断するものではない 
\\	報告から判断すると被害(ひがい)は相当(そうとう)大きいらしい 
\\	どちらがいいか判断がつかない 
\\	〔占(うらな)い〕 姓名判断 
\\	身の上判断 
\\	夢判断をしてみよう 
\\	判断力 
\\	子供には判断力はない 
\\	犯人	はんにん	
\\	販売	はんばい	
\\	灯	ひ	〔明かり〕
\\	灯を付ける[消(け)す] 
\\	ランプの灯をともす 
\\	実験室には赤々と灯がともっていた 
\\	一つ一つ灯が消(き)えていった 
\\	灯が絶(た)える 戦後,古い伝統の灯は絶えてしまった 
\\	被害	ひがい	
\\	比較	ひかく	
\\	比較する 
\\	比較的 
\\	この二つは比較にならない 
\\	祖母のスケッチを後年(こうねん)の写真と比較した 
\\	その絵は比較にならないほど美しい 
\\	これはそれとは比較にならない〔それより劣る〕 
\\	太田と比較すると鈴木はまだまだだ 
\\	まだまだ 
\\	両社の業績を比較すれば
\\	社のほうがはるかに将来性があることが分かる 
\\	ここは比較的涼しい 
\\	比較級 〔文法で〕
\\	比較研究 
\\	比較広告 
\\	比較文学 
\\	悲劇	ひげき	
\\	飛行	ひこう	
\\	膝	ひざ	
\\	〔ひざがしら〕
\\	彼は祭壇(さいだん)の前に膝をついた 
\\	/((文)) 
\\	膝を曲げる[立てる] 
\\	膝まで水につかってしまった 
\\	丘(おか)の斜面(しゃめん)には雑草(ざっそう)が膝まで生えていた 
\\	このズボンは膝が抜けている 
\\	〔座ったときの腰からひざまで〕
\\	彼女は大きな包みを膝にのせていた 
\\	膝が笑う 〔膝ががくがくする〕 一日中歩きづめで膝が笑っている 
\\	膝を打つ 〔考えがひらめいて〕
\\	そう言われてはたと膝を打った 
\\	はたと 
\\	膝を崩す 
\\	無礼講(ぶれいこう)ですからどうぞ膝を崩(くず)してください 
\\	膝を屈する 権力の前に膝を屈(くっ)した 
\\	膝を組む 彼は膝を組んで座っていた 
\\	膝を進める 
\\	〔近寄る〕 彼の前に膝を進めた 
\\	〔乗り気になる〕
\\	彼の話に膝を進めた 
\\	膝を正す 上役を見て,彼は膝を正した 
\\	膝を突き合わせる 膝を突き合わせて相談する 
\\	膝を交える 我々は膝を交(まじ)えて話し合った 
\\	膝掛け ((米)) 
\\	((英)) 
\\	膝頭 〔膝小僧〕
\\	〔膝蓋骨〕
\\	〔膝関節〕
\\	膝とも談合 膝とも談合する 
\\	膝詰め談判 ⇒詳細を見る 膝枕 妻の膝枕(ひざまくら)でぐっすり眠ってしまった 
\\	膝枕 
\\	非常	ひじょう	
\\	非常に 
\\	非常の場合に備える 
\\	彼は非常に際して沈着に振る舞った 
\\	非常の際は110番すること 
\\	非常階段 
\\	非常管制 
\\	非常口 
\\	非常警戒 
\\	非常警報 
\\	非常コック 
\\	非常時 ⇒詳細を見る 非常事態 
\\	非常事態を宣言する 
\\	非常手段 彼は非常手段を講じた 
\\	非常上告 
\\	非常招集 
\\	非常食 
\\	非常線 ⇒詳細を見る 非常措置 
\\	非常報知器[ベル] 
\\	非常持ち出し 〔表示〕
\\	美人	びじん	
\\	額	ひたい	
\\	日付	ひづけ	
\\	日付を入れる 
\\	5月10日の日付の手紙 
\\	日付のない手紙 
\\	4月3日の日付で手紙を出す 
\\	日付変更線 
\\	引越し	ひっこし	
\\	必死	ひっし	
\\	〔必ず死ぬこと〕
\\	必死の覚悟で戦地におもむいた 
\\	〔死に物狂い〕 必死な (に) 
\\	必死の攻撃 
\\	必死の努力をする 
\\	彼らは必死に戦った 
\\	必死にもがいた 
\\	ぴったり	ぴったり	
\\	〔くっつく様子〕
\\	〔きちんと合う様子〕
\\	〔急に中止する様子〕 
\\	〔くっつく様子〕
\\	ぬれたシャツが背中にぴったりとくっついた 
\\	窓はぴったり閉ざされていた 
\\	恋人同士はぴったりと寄り添(そ)って座っていた 
\\	寄り添う=
\\	〔きちんと合う様子〕
\\	この本は5歳の子供にぴったりだ 
\\	このスーツは私にぴったりだ 
\\	二人は呼吸(こきゅう)がぴったりと合っている 
\\	呼吸がぴったり合う 
\\	彼女の予言(よげん)はぴったり当たった 
\\	計算がぴったり合った 
\\	彼はその役(やく)にぴったりの人だ 
\\	〔急に中止する様子〕 彼はその日からぴったり来なくなった 
\\	酒もたばこもぴったりやめた 
\\	引っ張る	ひっぱる	
\\	〔引き寄せる〕
\\	人の手[耳]を引っ張る 
\\	襟首(えりくび)を引っ張る 
\\	綱(つな)を引っ張る 
\\	ぐいと引っ張る 
\\	引っ張り合う 〔互いに〕
\\	/〔両側から〕
\\	〔ぴんと張る〕
\\	川に綱を引っ張る 
\\	〔連行(れんこう)する〕 
\\	警察に引っ張って行く 
\\	罪人(ざいにん)を引っ張ってくる 
\\	〔打球を〕 彼はボールを左に引っ張った 
\\	〔誘う〕⇒かんゆう(勧誘),さそう(誘う) クラブに引っ張る 
\\	否定	ひてい	
\\	((文)) 
\\	否定の,否定的 
\\	否定する 
\\	否定できない事実 
\\	そのことは否定できない 
\\	彼はその非難をはっきり否定した 
\\	彼はそんなことは言わなかったと否定した 
\\	彼は否定的な答えをした 
\\	否定訴求 〔広告で〕
\\	否定文 
\\	一言	ひとこと	
\\	人込み	ひとごみ	
\\	等しい	ひとしい	
\\	〔数量が同じ〕
\\	この2本の線は長さが等しい 
\\	は
\\	に等しい 
\\	を
\\	と等しくせよ 
\\	〔同然である〕 彼の頼みは脅迫(きょうはく)に等しかった 
\\	君の言うことは僕に対する侮辱(ぶじょく)に等しい 
\\	ほとんど何も無いに等しい 
\\	独り	ひとり	
\\	一人一人	ひとりひとり	
\\	批判	ひはん	
\\	批判的 
\\	論文を批判する 
\\	この絵について先生のご批判をあおぎたい 
\\	彼は私を不注意だと批判した 
\\	私には文学を批判する力はない 
\\	彼は同僚の批判ばかりしている 
\\	不公平税制に対する国民の批判 
\\	国民の批判は謙虚(けんきょ)に受け止めたい 
\\	快(こころよ)く批判の一つ一つを受け入れてくれた 
\\	批判をたいへん気にする 
\\	その問題については彼は批判的な立場にある 
\\	彼は私のすることに批判的である 
\\	批判者 
\\	批判票 
\\	批判票を投じる 
\\	戦争反対者たちが批判票を集めようと奮闘(ふんとう)している 
\\	批判文 
\\	批判力 彼の批判力を疑う 
\\	批判力のある読者 
\\	批評	ひひょう	
\\	この作品は批評の価値(かち)がない 
\\	この本の批評を頼まれた 〔意見を〕
\\	/〔書評を〕
\\	この小説は暗すぎると批評された 
\\	みなあれこれと私の帽子を批評した 
\\	社長は計画に何の批評もしなかった 
\\	外国人は日本の家屋(かおく)が狭(せま)いとよく批評する 
\\	批評家 
\\	〔書物の〕
\\	批評眼 
\\	秘密	ひみつ	
\\	微妙	びみょう	微妙な 
\\	((文)) 
\\	は識別しにくい微妙さ,
\\	は細心の注意を要する感じを持つ) 微 
\\	微妙な区別 
\\	微妙な意味合い 
\\	意味合い 
\\	これとそれとは微妙な点で違っている 
\\	彼は今,微妙な立場にある 
\\	日本の微妙な立場を理解すべきだ 
\\	情勢(じょうせい)は微妙だ 
\\	紐	ひも	
\\	冷やす	ひやす	
\\	〔温度を下げる〕
\\	〔氷で〕
\\	氷で冷やした紅茶 
\\	牛乳を冷蔵庫で冷やしておく 
\\	シャンペンは氷で冷やしてある 
\\	病人の頭を氷で冷やした 
\\	足を冷やさないようにしなさい 
\\	〔頭を〕 頭を冷やせ 
\\	/((口)) 
\\	〔肝を〕 それを見て肝(きも)を冷やした 
\\	費用	ひよう	
\\	〔支出〕
\\	〔ある物事に出費する金額〕
\\	〔ある物事にかかる金〕
\\	〔経費〕
\\	用途を表す語を伴う
\\	複数形) 旅行の費用 
\\	負担費用 
\\	費用は少なくとも20万円くらいになるだろう 
\\	費用を出しあって会を開いた 
\\	費用を切り詰める 
\\	費用倒れになる 
\\	生活の費用 
\\	生産費用 
\\	それはひどく費用が掛かる 
\\	費用はいくら掛かってもかまわない 
\\	10万円の費用で旅行した 
\\	わずかな費用で一部屋増築した 
\\	彼はわずかな費用で店を出した 
\\	費用効率 
\\	費用効率が高い 
\\	表	ひょう	
\\	秒	びょう	
\\	評価	ひょうか	
\\	表現	ひょうげん	〔主として,言葉で表すこと〕
\\	〔具体的な形に表すこと〕
\\	表現する 〔言葉などで〕
\\	〔具体的な形に〕
\\	表現の自由 
\\	イソップ物語の絵画による表現 
\\	自分の考えを表現する 
\\	その気持ちは何とも表現しがたい 
\\	適切な言葉で表現している 
\\	彼の絵には自然を愛する心がよく表現されている 
\\	喜びを体中(からだじゅう)で表現した 
\\	表現主義 〔芸術思潮〕
\\	表現派 〔人〕
\\	〔総称〕
\\	表現力 
\\	彼には豊(ゆた)かな表現力がある 〔文章の〕
\\	/〔話す言葉に〕
\\	表情	ひょうじょう	
\\	平等	びょうどう	平等の 
\\	平等に 〔均等に〕
\\	〔差別なく〕
\\	男女平等 
\\	我々はすべて法の前には平等である 
\\	収益(しゅうえき)を皆に平等に分けた 
\\	男女の比率を平等にする 
\\	子供たちに平等に小遣いを与えた 
\\	彼は従業員を平等に扱った 
\\	平等主義 
\\	平等主義者 
\\	評判	ひょうばん	
\\	〔世評〕
\\	〔名声〕
\\	〔人気〕
\\	評判の 
\\	評判がよい[悪い] 
\\	評判がよい小説 〔よいと認められた〕
\\	/〔大衆に受けた〕
\\	新市長は世間(せけん)の評判がよい 
\\	彼女は友人の間で評判がよい[悪い] 
\\	その週刊誌はスキャンダル記事で評判が悪い 
\\	彼は学者の間で評判が高く[悪く]なった 
\\	評判を高める 
\\	評判を落とす 
\\	評判を取り戻す 
\\	彼は有能(ゆうのう)な実業家だという評判を取った 
\\	その川堤(かわづつみ)は花見の名所として評判になった 
\\	その小説は主婦(しゅふ)たちに評判になった 
\\	彼は評判倒れだった[評判ほどではなかった] 
\\	/((口)) 
\\	〔うわさ〕
\\	((英)) 
\\	評判の 
\\	評判の人 
\\	彼女は評判の美人だ 
\\	彼が知事に当選するだろうという評判がたった 
\\	その結婚は町中(まちなか)の評判になった 
\\	誰が評判を立てたか分からない 
\\	彼と事件のかかわりあいについていろいろ評判されている 
\\	かかわりあい 
\\	〔名声〕
\\	〔悪名〕
\\	評判の 〔有名な〕
\\	〔悪名高い〕
\\	表面	ひょうめん	
\\	〔物の外側の部分〕
\\	表面の 
\\	地球の表面に 
\\	ガラスの表面に傷がついている 
\\	〔裏側に対して,表側〕 紙の表面はすべすべしているが,裏面(りめん)はざらざらしている 
\\	〔上辺(うわべ)〕 
\\	表面だけの親切 
\\	表面は親切そうだが本心は分からない 
\\	表面は病気ということにして欠勤(けっきん)した 
\\	表面はいかにも専門家のように見えた 
\\	いかにも 
\\	表面上は何事も無いように見える 
\\	〔目立つ部分〕 彼は表面に立つのをうれしがる 
\\	彼は表面に立つ[出る]のが嫌いだ 
\\	隠しても事実は次第に表面に出るようになる 
\\	ても 
\\	〔皮相(ひそう)〕 
\\	彼はその事柄(ことがら)について表面的な知識しかない 
\\	表面化 内部抗争(こうそう)がついに表面化した 
\\	事が表面化しないうちに処理したい 
\\	表面処理[加工] 〔洋服の〕
\\	表面張力 
\\	広がる	ひろがる	
\\	〔広くなる〕
\\	火事は隣町(となりまち)に広がった 
\\	洪水(こうずい)は全村(ぜんそん)に広がった 
\\	道路いっぱいに広がって行進した 
\\	雨雲(あまぐも)が空一面に広がった 
\\	見渡す限り花畑(はなばたけ)が広がっていた 
\\	〔幅が大きくなる〕
\\	そこで道幅(みちはば)が急に広がっている 
\\	すその広がったスカート 
\\	〔広くのびる〕 青々とした麦畑がふもとまで広がっていた 
\\	〔広く行き渡る〕
\\	〔伝わる〕
\\	うわさがぱっと広がった 
\\	インフルエンザは全国に広がっている 
\\	うわさは学生の間に広がった 
\\	〔状態が大きくなる〕 企業はますます広がっていく 
\\	両チームの実力の差はますます広がった 
\\	調査が進むにつれて被害はさらに広がるであろう 
\\	広げる	ひろげる	
\\	〔畳んだものを開く〕
\\	〔開ける〕
\\	鳥が翼(つばさ)を広げた 
\\	本を広げる 
\\	包みを広げる 
\\	弁当を広げる 
\\	傘[新聞]を広げる 
\\	彼女は両手を広げて私たちを出迎(でむか)えた 
\\	足を広げて座る 〔脚を伸ばして〕
\\	[〔股を広げて〕
\\	〔幅・範囲を広くする〕
\\	〔大きくする〕
\\	川幅(かわはば)を広げる 
\\	運動場[領土]を広げる 
\\	家を広げる 
\\	〔一杯に並べる〕 床の上におもちゃを広げて遊んでいた 
\\	部屋中に資料を広げて論文を書いている 
\\	〔状態を大きくする〕 事業の手を広げる 
\\	彼は活動範囲を広げた 
\\	海外旅行をして視野(しや)を広げたい 
\\	彼は2位との差を大きく広げた 
\\	広める	ひろめる 
\\	〔広くする〕
\\	見聞(けんぶん)を広める 
\\	その国はどんどん勢力(せいりょく)を広めた 
\\	〔普及させる〕
\\	彼は学問を広めることに身をささげた 
\\	[((文))
\\	彼は仏教を広めた 
\\	その曲は彼の名声を全世界に広めた 
\\	うわさを広めたのは彼女だ 
\\	瓶	びん	
\\	便	びん	
\\	〔郵便〕((米)) 
\\	((英)) 
\\	詳細(しょうさい)は次の便でお知らせします 
\\	〔交通などの〕 
\\	鉄道[船]便で 
\\	航空会社は今日の第8便を欠航とした 
\\	当時中国に渡る便はこれ以外になかった 
\\	〔都合のよい機会〕
\\	便のあり次第知らせてください 
\\	不安	ふあん	
\\	風景	ふうけい	
\\	夫婦	ふうふ	
\\	笛	ふえ	
\\	不可	ふか	
\\	〔よくないこと〕 可もなく不可もない 
\\	採決(さいけつ)の結果,可とする者18名,不可とする者3名であった 
\\	〔不合格(ふごうかく)〕 化学の試験で不可をとった 
\\	[((口)) 
\\	深まる	ふかまる	
\\	彼らの愛情は日増(ひま)しに深まった 
\\	それに対する興味は深まるばかりだった 
\\	武器	ぶき	
\\	ふく	ふく	
\\	〔拭(ぬぐ)う〕
\\	〔水分を取る〕
\\	手ぬぐいで手をふく 
\\	顔の汗をふく 
\\	ハンカチで涙をふく 
\\	ふきんで皿をふく 
\\	ふろから上がって彼は体をタオルでよくふいた 
\\	服装	ふくそう	
\\	含む	ふくむ	
\\	〔中に持っている〕
\\	〔全体の一部として〕
\\	鉄分を含んだ水 
\\	この本には有益(ゆうえき)な情報が含まれている 
\\	その団体には女性2人が含まれている 
\\	〔口に入れている〕
\\	〔心に留める〕
\\	この事情を含んでおいてください 
\\	〔恨みなどをいだく〕 含むところがあってこんなことをしたのだ 
\\	〔様子を帯びる〕 怒りを含んだ目付きで私を見た 
\\	憂(うれ)いを含んだ顔 
\\	〔意味を含む〕
\\	彼の言葉は拒否(きょひ)の意味を含んでいた 
\\	含める	ふくめる	
\\	〔含ませる〕
\\	⇒ふくむ(含む) 彼らの名をリストに含めた 
\\	子供を含めて[含めずに]300人だ 
\\	郵送料(ゆうそうりょう)を含めて5,000円だった 
\\	私は税金を含めて月40万円の収入がある 
\\	〔言い聞かせる〕 いいふくめる【言い含める】 
\\	彼には十分言い含めておくがよい 
\\	彼女は男をあきらめるよう言い含められた 
\\	かんでふくめる
\\	噛んで含める】 
\\	かんで含めるように説明する 
\\	袋	ふくろ	
\\	不幸	ふこう	
\\	〔不幸せ〕
\\	〔不運〕
\\	不幸な 
\\	不幸に見舞われる 
\\	彼は不幸にも失明した 
\\	事故にあったが骨折だけですんだのは不幸中の幸いだった 
\\	済(す)む 
\\	けがをしなかったのは不幸中の幸いだった 
\\	〔人の死去〕
\\	最近親類に不幸があった 
\\	お父上のご不幸をお悔(く)やみ申し上げます ((文)) 
\\	無事	ぶじ	
\\	〔安全〕
\\	無事東京駅に着いた 
\\	昨夜は無事にお帰りでしたか 
\\	どうぞご無事で〔旅(たび)する人に〕 
\\	全員無事に避難(ひなん)した 
\\	小包が無事にあて先に着いたらしい 
\\	〔健康〕
\\	無事なお顔を見てうれしく存じます 
\\	ご両親は無事にお過ごしですか 
\\	家族一同無事に暮らしています 
\\	〔平穏〕
\\	今は太平無事の世の中だ 
\\	無事に暮らす 
\\	その日は無事に過ぎた 
\\	首脳会談は無事に済んだ ((口)) 
\\	不思議	ふしぎ	
\\	不思議な 〔奇妙な〕
\\	〔神秘的な〕
\\	〔奇跡(きせき)的な〕
\\	〔奇異(きい)な〕
\\	世界の七不思議 
\\	不思議な物[事/人] 
\\	不思議な事件が相次いだ 
\\	不思議なことに 
\\	彼の不思議な失踪(しっそう) 
\\	彼が来なかったとは全く不思議だ 
\\	彼が急にロンドンへ発ったとは不思議だ 
\\	彼の失敗に不思議はない 
\\	不自由	ふじゆう	
\\	〔不便〕
\\	不自由な 
\\	不自由な生活をする 〔不便で〕
\\	/〔金銭的に〕
\\	人に不自由をさせる 
\\	金に不自由する[しない] 
\\	何不自由なく育てられた 
\\	〔自由がきかないこと〕 目が不自由である 〔見えない〕
\\	/〔視力が弱い〕
\\	耳が不自由である 
\\	足が不自由である 
\\	体が不自由である 
\\	夫人	ふじん	
\\	〔他人の妻〕
\\	この会にはどうぞ夫人ご同伴(どうはん)でご出席ください 
\\	〔敬称(けいしょう)〕
\\	通例
\\	を用いるが,改まったときに
\\	を用いる
\\	また大統領夫人,首相夫人,大使夫人あるいは音楽家などは
\\	と呼ばれる) 佐藤夫人 
\\	婦人	ふじん	
\\	女性 (じょせい)、女の人 (おんなのひと))
\\	不正	ふせい	
\\	防ぐ	ふせぐ	
\\	〔防御する〕
\\	〔防護する〕
\\	泥棒を防ぐために戸締(とじま)りを厳重(げんじゅう)にする 
\\	風雨(ふうう)を防ぐものは何もなかった 
\\	〔防止する〕
\\	〔予防する〕
\\	蝿(はえ)を防ぐ 
\\	すきま風を防ぐ 
\\	川の氾濫(はんらん)を防ぐ 
\\	横揺(よこゆ)れを防ぐ 
\\	青少年の非行を未然(みぜん)に防ぐ 
\\	未然 
\\	この薬は吐(は)き気(け)を防いでくれる 
\\	塩と砂糖は食品の腐敗(ふはい)を防ぐ 
\\	ミスを防ぐようにしなさい 
\\	誤解を防ぐ 
\\	不足	ふそく	
\\	舞台	ぶたい	
\\	双子	ふたご	
\\	再び	ふたたび	
\\	普段	ふだん	ふだん (は) 〔平常〕
\\	〔常に〕
\\	ふだんの 〔いつもの〕
\\	〔習慣的な〕
\\	〔通常の〕
\\	ふだんは6時に起きる 
\\	ふだんから家をきれいにしている 
\\	ふだんの通り散歩に出掛けた 
\\	それはふだんよくある間違いだ 
\\	娘にはふだんと変わったところがなかった 
\\	君はふだんの心掛けが悪い 
\\	縁(ふち)	ふち	
\\	〔へり〕
\\	川[池]の縁にたたずむ 
\\	湖の縁に 
\\	この皿は縁がかけている 
\\	布の縁 
\\	カーテンの縁を縫(ぬ)う 
\\	ビールがコップの縁からあふれた 
\\	スープを茶わんの縁まで入れた 
\\	〔帽子のつば〕 帽子の縁 
\\	縁の広い帽子 
\\	〔わく〕 めがねの縁 
\\	銀縁のめがね 
\\	縁なし眼鏡 
\\	へり【▲縁】 
\\	〔ふち〕
\\	⇒ふち(縁) テーブル[崖]の縁 
\\	縁にすみれを植(う)えた花壇(かだん) 
\\	川の縁に住んでいる 
\\	〔畳などのへり布〕
\\	畳の縁 
\\	打つ	うつ	
\\	〔たたく〕
\\	うまく打ち当てる意味を含む); 
\\	ある意図をもって続けざまに打つ,対象が人の場合には「痛めつける」という意味を含む); 〔平手で打つ〕
\\	〔げんこつでなぐる〕
\\	〔こつこつとたたく〕
\\	子供の頭を打つ 
\\	ぴしゃりと彼のほおを打った 
\\	無法者をステッキで打った 
\\	転んでひじを打った 
\\	雷に打たれた 
\\	〔打ちつける〕 車の屋根を激しく打つ雨 
\\	波が岸を打っていた 
\\	あられがぱらぱらと屋根を打った 
\\	〔たたき込む〕 新しい杭(くい)を打った 
\\	ほうきをつるすために釘を打った 
\\	〔打って飛ばす〕 ヒットを打つ 
\\	本塁打(ほんるいだ)を打つ 
\\	我がチームはよく打った 
\\	〔たたいて鳴らす〕 太鼓を打つ 
\\	手を打って喜んだ 
\\	時計が12時を打った 
\\	〔キーをたたく〕 この書類をワープロで打ってくれ 
\\	彼に祝電を打った 
\\	〔針を刺す〕 予防注射を打ってもらった 
\\	鍼(はり)を打っている 
\\	〔付ける〕 
\\	の点を打ち忘れた 
\\	漢字に振り仮名を打つ 
\\	〔こねる〕
\\	うどんを打つ 
\\	〔撒く〕 庭に水を打つ 
\\	会場は水を打ったように静まり返った 
\\	〔興行する・行う〕 芝居を打つ 
\\	碁を打つ 
\\	〔手段を講ずる〕 彼はうまい手を打ったもんだ 
\\	〔感動させる〕
\\	胸を打つ話 
\\	彼女の誠実さが私の心を打った 
\\	畏敬(いけい)の念に打たれて立ち尽くしていた 
\\	打てば響く あの子は打てば響くような子だ 
\\	打てば響く応答 
\\	打てば響くように答える 
\\	打って一丸となる 全員打って一丸となって事に当った 
\\	不通	ふつう	
\\	物価	ぶっか	
\\	この辺は物価が高い[低い] 
\\	物価の変動が大きい 
\\	今月は物価が上がった[下がった] 
\\	消費者物価 
\\	消費者 
\\	消費者物価に合わせた昇給 
\\	ぶつかる	ぶつかる	
\\	〔衝突する〕 壁にぶつかる 
\\	岩にぶつかる波 
\\	通りで知らない人にぶつかった 
\\	[((米)) 
\\	船が岸壁(がんぺき)にぶつかった 
\\	は波などのため,
\\	は操縦者のミスなどのため) 2台の車が正面からぶつかった 
\\	〔一緒になる〕 そこで本流と支流がぶつかっている 
\\	〔出くわす〕 思わぬ困難にぶつかった 
\\	企業はついに壁にぶつかった 
\\	小説の中でよくぶつかるタイプの人間だ 
\\	〔まじめに取り組む〕 勇敢(ゆうかん)にも危険にぶつかって行った 
\\	実地にぶつかってみよう 
\\	〔かち合う〕
\\	祭日と日曜がぶつかった 
\\	予定が二つぶつかってしまった 
\\	〔反発し合う〕 仕事のことで父とぶつかった 
\\	[((米口))
\\	ぶつける	ぶつける	
\\	〔投げつける〕
\\	怒って荒々(あらあら)しく本をぶつけた 
\\	〔衝突させる〕 彼は車を電柱にぶつけてしまった 
\\	頭を戸にぶつけた 
\\	〔難問・激しい感情などを投げかける〕 わざと難問(なんもん)をぶつけて試(ため)してみた 
\\	やり場のないくやしさを夫にぶつけた 
\\	彼は激しい言葉をぶつけてきた 
\\	物質	ぶっしつ	
\\	〔精神に対して〕
\\	〔材料〕
\\	硬い[堅い]物質 
\\	それは何か半透明(はんとうめい)な物質で出来ていた 
\\	彼は物質的に恵(めぐ)まれていた[恵まれていなかった] 
\\	彼には物質的な援助が必要だ 
\\	彼は物質的快楽(かいらく)のみ追っている 
\\	物質界 
\\	物質主義 
\\	物質文明 
\\	物質名詞 〔文法で〕
\\	物理	ぶつり	
\\	筆	ふで	
\\	ふと	ふと	
\\	〔突然〕
\\	〔偶然〕
\\	〔不意に〕
\\	ふと立ち止まる 
\\	彼にふと出会った 
\\	/((口)) 
\\	その考えがふと思い浮かんだ 
\\	彼とはふとしたことから知り合った 
\\	ふとした出来心(できごころ)で盗(ぬす)みを働いた 
\\	部分	ぶぶん	
\\	不平	ふへい	
\\	住民たちの間には不平の声が高い 
\\	この取り決めに対して不平がありますか 
\\	不平をもらす 
\\	/〔ぶつぶつ言う〕
\\	彼は校長にさまざまな不平を並べたてた 
\\	君は今の仕事について少しも不平を言わないね 
\\	[((口))
\\	彼女はその金を不平そうに受け取った 
\\	不平家 
\\	不平分子 
\\	不満	ふまん	
\\	〔不平〕
\\	不満である 
\\	不満の種 
\\	彼が度々海外出張するのが妻の不満の種だった 
\\	この取り決めに対しなんの不満もありません 
\\	私のどこが不満なのか言ってくれ 
\\	市民は市長の決定に不満を表明した 
\\	不満を口にする 
\\	/〔不機嫌そうに〕
\\	寮の生活に不満はありませんか 
\\	彼は食事の不満をしょっちゅう言っている 
\\	彼は不満気にはいと言った 
\\	増やす	ふやす	【増やす・殖やす】 〔増大させる〕
\\	〔さらに増す〕
\\	財産を殖やす 
\\	俸給(ほうきゅう)を増やす 
\\	俸給を増やすように組合が要求してきた 
\\	子供は毎日語彙(ごい)を増やしていく 
\\	会社は事業の拡張に伴って人を増やした 
\\	あの大学は来年から入学定員を増やすそうだ 
\\	電車の本数をもっと増やして欲しい 
\\	「このラッシュ、何とかならないかしら」「せめて朝のピークの時間くらい、電車の本数を増やしてほしいわね」 
\\	「部長、プロジェクトの予算を増やしてもらうよう折衝(せっしょう)していただけませんか」「仕方ないな。もう一回、きちんとした試算(しさん)を出したまえ」 
\\	「お小遣い、増やしてくれない?」「成績が上がったら、お父さんに聞いてみましょう」 
\\	「今時(いまどき)、財産を殖やすんだったら、不動産くらいしか当てにならないからなあ」「そうは言っても、投機(とうき)のために土地を買うなんて間違ってるよ」 
\\	「人手が足りなくて、このままじゃ出荷(しゅっか)の手配も間に合いませんよ」「じゃ、アルバイトを増やした方がいいな」 
\\	【増やす・殖やす】	ふやす	ふやす【増やす・殖やす】 増 
\\	殖 
\\	〔増大させる〕
\\	〔さらに増す〕
\\	財産を殖やす 
\\	俸給(ほうきゅう)を増やす 
\\	俸給を増やすように組合が要求してきた 
\\	子供は毎日語彙(ごい)を増やしていく 
\\	不利	ふり	
\\	不利な 〔不都合な〕
\\	不利は覚悟(かくご)の上であった 
\\	試合の形勢(けいせい)は我々に不利だった 
\\	我々の不利な点は資金不足だ 
\\	彼は被告(ひこく)に不利な証言(しょうげん)をした 
\\	選挙は我々の党に不利な結果となった 
\\	振る	ふる	
\\	〔振り動かす〕
\\	の方が軽く振る感じ,合図などにも用いる); 〔一端(いったん)が固定した物などを〕
\\	瓶[体]を振る 
\\	さいころを振る 
\\	よく振ってからお飲みください 
\\	首を横に振る 
\\	首を縦(たて)に振る 
\\	/〔比ゆ的〕
\\	首を縦(たて)に振る 
\\	腕[バット]を振る 
\\	振り子を振る 
\\	犬が尾を振った 
\\	人に手を振る 
\\	彼女はさよならと手を振った 
\\	警官が手を振って人波(ひとなみ)を前進させた 
\\	〔振り掛ける〕 野菜に塩を振る 
\\	〔割り当てる〕 役を振る 
\\	〔仕事上の〕
\\	〔捨てる〕
\\	〔拒絶(きょぜつ)する〕
\\	の方が強い調子); 〔失う〕
\\	〔取り消す〕
\\	彼女に振られた 〔デートを〕
\\	/〔今まで付き合っていたのに〕
\\	/〔約束していて来なかった〕
\\	/〔婚約をしてから〕
\\	/〔結婚を〕
\\	仕事の口を振った 
\\	約束[試験]を振る〔無断(むだん)で 
\\	彼は社長のいすを振って遠征隊(えんせいたい)に加わった 
\\	汚職(おしょく)のために大臣(だいじん)のいすを振ってしまった 
\\	棒に振る 
\\	失言して大臣の椅子を棒に振ってしまった 
\\	つまらない男と結婚して一生を棒に振ってしまった 
\\	折角(せっかく)のチャンスを棒に振った 
\\	[((口))
\\	震える	ふるえる	
\\	触れる	ふれる	
\\	(反する)
\\	(言及する)
\\	〔触る〕
\\	〔触ってみる〕
\\	何かが足に触れた 
\\	手を触れないでください〔掲示〕 
\\	ちょっと触れただけで壊れる 
\\	擦れ違ったとき,彼の腕に触れた 
\\	この布に触れて確かめてごらんなさい 
\\	触れると柔らかい 
\\	〔ぶつかる〕 機雷(きらい)[岩]に触れる 
\\	〔知覚(ちかく)する〕
\\	((文)) 
\\	〔経験する〕
\\	目に触れるものはみな面白かった 
\\	彼らの悲惨な状態に触れた 
\\	〔人の感情に影響を及ぼす〕 怒りに触れる 
\\	[((文)) 
\\	音楽の何かが彼の心に触れたのだろう 
\\	逆鱗(げきりん)に触れる 
\\	逆鱗(げきりん) 
\\	〔言及(げんきゅう)する〕
\\	この本はそれに触れていない 
\\	ここではその件に触れないことにする 
\\	その問題には触れる時間がない 
\\	傷口(きずぐち)に触れないでおいてやれ 
\\	過去のことには触れてもらいたくない 
\\	〔抵触(ていしょく)する〕 
\\	法[規則]に触れる 
\\	版権(はんけん)に触れる 
\\	〔機会・物事に出会う〕 折(おり)に触れて 
\\	触れ回る 
\\	村中にそのうわさを触れ回った 
\\	自分の才を触れ回る 
\\	触れ込む 
\\	〔自称(じしょう)する〕 
\\	弁護士だと触れ込んだ 
\\	〔宣伝する〕 
\\	彼はその少女をスペインの踊り子だと触れ込んだ 
\\	文	ぶん	
\\	雰囲気	ふんいき	〔感じ〕
\\	雰囲気を醸(かも)し出す[壊す] 
\\	醸し出す 
\\	醸 
\\	学問的[宗教的]雰囲気の中に育った 
\\	芸術家的な雰囲気の人 
\\	雰囲気のある女の人 
\\	くつろいだ雰囲気の喫茶店 
\\	くつろぐ 
\\	分析	ぶんせき	
\\	((複-
\\	分析する ((米)) 
\\	((英)) 
\\	分析的 
\\	スペクトル分析 
\\	精神分析 
\\	定量[定性]分析 
\\	分析によりそれは三つの要素(ようそ)から成り立っていることが分かった 
\\	彼は心理を分析して患者を幾つかのタイプに分けた 
\\	鉱石(こうせき)見本を分析する 
\\	分析化学 
\\	分析学 
\\	分析学者 
\\	文明	ぶんめい	
\\	西洋文明 
\\	機械[物質]文明 
\\	古代文明の発祥地(はっしょうち) 
\\	[((文)) 
\\	この文明の世にも治らない病気はある 
\\	我々は文明の利器に囲まれている 
\\	文明開化 〔日本の〕
\\	開化 
\\	開化する 
\\	文明国 
\\	文明社会 
\\	分野	ぶんや	
\\	彼女は女性の活動に新分野を開いた 
\\	数学の諸分野 
\\	彼は彫刻(ちょうこく)の分野では第一人者(だいいちにんしゃ)だ 
\\	それは私の専門(せんもん)分野ではない 
\\	/((口)) 
\\	国会内の勢力(せいりょく)分野に異動があった 
\\	塀	へい	
\\	〔垣〕
\\	塀を建てる 
\\	塀を乗り越える 
\\	平均	へいきん	
\\	平和	へいわ	
\\	別に	べつに	
\\	〔とりたてて〕
\\	⇒とくに(特に) 今日は別に忙しくない 
\\	「何か欲しいものがありますか」「いえ,別に」 
\\	それ以外別に理由はありません 
\\	減らす	へらす	
\\	〔数量・大きさ・程度などを少なくする〕
\\	〔数量を少なくする〕
\\	〔出費を切り詰める〕
\\	数を減らす 
\\	重荷(じゅうか)を減らす 
\\	減食(げんしょく)して体重を減らしているところだ 
\\	睡眠時間を減らす 
\\	出費を減らす 
\\	医者に酒を減らせと言われた 
\\	〔おなかを空かす〕 腹を減らす 
\\	何日も腹を減らしていた 
\\	減る	へる	
\\	〔数・分量などが〕
\\	会員数が減った 
\\	収益(しゅうえき)が減った 
\\	灯油(とうゆ)の貯(たくわ)えが減ってきた 
\\	川の水かさが減っている 
\\	体重が5キロ減った 
\\	靴が減ってしまった 
\\	農業人口が目立って減っている 
\\	〔おなかが空く〕 腹が減った 
\\	変化	へんか	
\\	〔変わること〕
\\	変化する 
\\	⇒かわる(変わる) 境遇(きょうぐう)の変化 
\\	その上変化を加える余地はない 
\\	時代の変化に応じる 
\\	著しい変化を見せる ((文)) 
\\	舞台の照明は目まぐるしく変化した 
\\	目まぐるしい 
\\	毛虫(けむし)が変化して蝶になる 
\\	明治初期は女性の地位が変化した時期だった 
\\	秋は天候が変化しやすい 
\\	彼の意見は変化しやすい 
\\	変化のない政策 
\\	変化のない毎日の生活に飽きていた 
\\	〔多様なこと〕 
\\	この辺の山々は変化に富んでいる 
\\	変化の多い[変化に富んだ]生涯(しょうがい)を送った 
\\	彼の小説に登場する女性は変化に乏しい 
\\	〔文法で語尾の変化〕
\\	〔動詞の〕
\\	〔格の〕
\\	ラテン語は変化が多い 
\\	この動詞の変化を言えますか 
\\	名詞[代名詞]を格変化させる 
\\	一人称単数の代名詞の変化 
\\	変化球 
\\	変化球を投げる〔野球で〕 
\\	変化球サーブをする〔バレーボールで〕 
\\	変更	へんこう	
\\	予定に変更はない 
\\	食費は予告(よこく)なしに変更することがある 
\\	年を取るにつれて彼も主義を徐々(じょじょ)に変更した 
\\	弁当	べんとう	
\\	方	ほう	
\\	法	ほう	
\\	棒	ぼう	
\\	冒険	ぼうけん	
\\	冒険的 
\\	冒険の好きな人 
\\	命がけの冒険をする 
\\	命を懸ける 
\\	命がけ 
\\	それはちょっと冒険ではないかと気に掛かる 
\\	どうしても冒険する気になれなかった 
\\	冒険家 
\\	冒険小説 
\\	冒険心 
\\	冒険談 
\\	方向	ほうこう	
\\	報告	ほうこく	
\\	〔報告された情報〕
\\	不可算名詞で不定冠詞をつけない) 報告する 〔報じる〕
\\	〔告げ知らせる〕
\\	〔学会などで〕
\\	年次[中間]報告 
\\	私は事件についてかいつまんで報告した
\\	掻(か)い摘(つま)む 
\\	彼は道路の状況について何も報告を受けていなかった 
\\	報告者 
\\	〔学会・ゼミなどの〕
\\	報告書 
\\	〔公式の答申などの〕
\\	〔統計の〕
\\	通例複数形); 〔学会の〕
\\	この意味では複数形) 選挙報告書 
\\	報告書を作成[提出]する 
\\	その事故の報告書 
\\	宝石	ほうせき	
\\	包装	ほうそう	
\\	〔荷造り〕
\\	包装する 
\\	オレンジはしっかり包装してあった[してなかった] 
\\	〔うわ包み〕
\\	包装する 
\\	商品を売るには包装が大切だ 
\\	割れ物を荷造(にづく)りするときは十分包装するように 
\\	包装紙 
\\	豊富	ほうふ	
\\	豊富な 
\\	〔あり余るほどの〕
\\	豊富に 
\\	((文)) 
\\	この地方は石油が豊富だ 
\\	[((文))
\\	知識を豊富にする 
\\	天然資源(てんねんしげん)の豊富な国 
\\	経験の豊富な人 
\\	彼は話題の豊富な人だ 
\\	商品の豊富さ 
\\	その店は在庫(ざいこ)が豊富だ 
\\	[((文)) 
\\	方法	ほうほう	
\\	方々	ほうぼう	
\\	方々から手紙が来た 
\\	いなくなった猫を方々たずね歩いた ((口)) 
\\	方々に飛び火(ひ)した 
\\	彼は方々に敵がいる 
\\	その思想は方々に広まった 
\\	ーーー かたがた 
\\	ご来場(らいじょう 
\\	のかたがた〔呼び掛け〕 
\\	⇒みなさん(皆さん)
\\	訪問	ほうもん	
\\	訪問する 
\\	家庭訪問 
\\	彼は日曜日には訪問を受けつけない 
\\	大統領は東南アジア諸国訪問の旅に出た 
\\	私はスミス氏を自宅に訪問した 
\\	訪問記 
\\	訪問着 
\\	訪問客[者] 
\\	訪問販売 
\\	今日、先生の家を訪問した 
\\	アメリカを訪問中の首相は、昨日大統領と会談を行った 
\\	彼は訪問販売のセールスマンだ 
\\	吠える	ほえる	
\\	〔犬が〕
\\	〔猟犬が〕
\\	〔犬・おおかみなどが〕
\\	〔ライオン・とらなどが〕
\\	猟犬(りょうけん)の吠える声が遠くで聞こえる 
\\	吠える犬はかみつかぬ ((諺)) 
\\	〔人が〕⇒どなる(怒鳴る) そうほえるなよ 
\\	ほか【外/他】	ほか	
\\	〔よそ〕((肯定文で)) 
\\	((疑問・否定・条件文で)) 
\\	これからまだほかに行くんですか 
\\	これはほかでは買えないよ 
\\	どこかほかを捜すことにしよう 
\\	本学の夏期(かき)講習会(こうしゅうかい)にはほかからの講師も数名来る 
\\	〔その人以外〕 だれかほかの人に聞いてごらん 
\\	ほかに行きたい人はいますか 
\\	山田氏ほか2名 
\\	演奏は佐藤氏ほかの方々でした 
\\	ほかの人はともかく私は小野氏を支持する 
\\	〔それ以外〕 何かほかの品を見せてください 
\\	ほかにはなにも文句はない 
\\	ほかに質問はありませんか 
\\	このほかにはなにも言わなかった 
\\	彼はフランス語のほかにイタリア語も話す 
\\	この上着は小さすぎるからほかのを試してみよう 
\\	〔「…よりほかに…ない,…するほかはない」の形で,限定を表して〕 彼よりほかに知人はいない 
\\	そうするよりほかしかたがない 
\\	それを聞いて苦笑(くしょう)するほかはなかった 
\\	ほかでもない ほかでもないが,人探しの件はどうなっているのかね 
\\	ほかならない 
\\	ほかならないあなたの頼みだから聞いてあげよう 
\\	それは市長にほかならなかった 
\\	これこそ探していたものにほかならない 
\\	彼のしたことは詐欺にほかならない 
\\	今度の成功は皆の努力の結果にほかならない 
\\	誇り	ほこり	
\\	誇り高い 
\\	誇りをひどく傷つける 
\\	彼の許しを請(こ)うなんて私の誇りが許さない 
\\	彼は自分の家柄を誇りとしている 
\\	彼女は当校の誇りだ 
\\	彼は我が校の誇りです 
\\	彼は自分の仕事に誇りを持っている 
\\	彼は助けが欲しかったが、誇りが許さなかった 
\\	彼は誇りが高すぎてそんなに易しい曲は演奏しない 
\\	彼の言葉は彼女の誇りを傷つけた 
\\	彼女は5ヵ国語が話せるのを誇りに思っている 
\\	私は意欲的で、信頼でき、プロフェッショナルであり、そしてどの地位にある人々とも良好(りょうこう)なコミュニケーションが取れることに誇りを持っています 
\\	埃	ほこり	
\\	保証	ほしょう	
\\	〔品質などの〕
\\	〔主として担保〕
\\	保証する 
\\	3年間の保証つき製品 
\\	それは純毛(じゅんもう)だと保証できます 
\\	借金(しゃっきん)の保証として家を担保(たんぽ)に入れる 
\\	〔確かだと請け合うこと〕
\\	保証する 
\\	頭がいいからと言って出世(しゅっせ)するという保証はない 
\\	彼には社長の地位が保証されている 
\\	息子さんはきっと帰ってくる
\\	保証します 
\\	彼がまじめなことは私が保証します 
\\	保証期間 
\\	保証期間内に故障(こしょう)した 
\\	保証期間が切れている 
\\	保証金 ⇒詳細を見る 保証小切手 
\\	保証書 
\\	〔人物・技量などの〕
\\	〔商売の〕
\\	保証責任 
\\	保存	ほぞん	
\\	歩道	ほどう	
\\	仏	ほとけ	
\\	骨	ほね	
\\	炎	ほのお	
\\	頬	ほお 
\\	ほほ	
\\	ほぼ	ほぼ	
\\	⇒ほとんど(殆ど)1 その工事(こうじ)はほぼ完了した 
\\	この土地(とち)1エーカーの価格はその土地の10エーカー分にほぼ等(ひと)しい 
\\	当時の父は今の私とほぼ同年配(ねんぱい)だった 
\\	学生のレポートはどれもほぼ似(に)たような物だ 
\\	どれも, いずれも 
\\	微笑む	ほほえむ	
\\	堀	ほり	
\\	濠	ほり	
\\	本当	ほんと	
\\	本人	ほんにん	
\\	本物	ほんもの	
\\	〔真のもの〕
\\	本物の 
\\	〔典拠のある〕
\\	本物の真珠(しんじゅ)〔養殖・模造でない〕 
\\	本物の16世紀の古文書(こもんじょ) 
\\	本物と模造品(もぞうひん)を見分ける 
\\	〔実物〕 本物そっくりの肖像画(しょうぞうが) 
\\	〔本格的なこと〕 あの人のピアノは本物です 
\\	この味噌汁(しる),これこそ本物の味だ 
\\	ぼんやり	ぼんやり	
\\	〔不明瞭(ふめいりょう)〕 ぼんやりした (と) 〔輪郭・色などが〕
\\	〔考えなどが〕
\\	ぼんやりした明かり 
\\	遠くに島がぼんやり見える 
\\	そのことをぼんやりとしか覚(おぼ)えていない 
\\	寝不足で今朝は頭がぼんやりしている 
\\	〔放心,不注意〕 ぼんやりした (と) 
\\	ぼんやりと雲を眺めていた 
\\	彼女はぼんやりと戸(と)のそばに立っていた 
\\	ぼんやりするな 
\\	ぼんやりしていて鞄(かばん)を間違えて持って来てしまった 
\\	〔何もしない様子〕 ぼんやりと暮らす 
\\	乱闘(らんとう)が始まっても審判(しんぱん)はぼんやりつっ立っていた 
\\	つったつ【突っ立つ】 
\\	〔鈍感(どんかん)〕 ぼんやりした 
\\	ぼんやりした人 
\\	間	ま 
\\	あいだ	ま 〔時間〕 休む間もない 
\\	そうこうしている間に 
\\	そうこう【▲然う×斯う】 
\\	1万円なんかあっという間に使ってしまう 
\\	あっと 
\\	間をおいて答えた 
\\	彼らは結婚して間がない 
\\	〔空間〕 間をあけて机を置く 
\\	10メートルの間をおいて木が立っている 
\\	〔部屋〕
\\	6畳の間 
\\	7間の家 
\\	〔ころあい〕
\\	間を見計らってその問題を切り出そうとしていた 
\\	[((文))
\\	〔邦楽(ほうがく 
\\	で,音と音の間の休止時間〕 間が合っている[いない] 
\\	〔せりふや動作の中断の時間〕
\\	間〔脚本のト書きで〕 
\\	間をとる 
\\	間のとり方がまずくて彼は芝居を台無(だいな)しにした 
\\	この種の音楽は間を大切にする 
\\	間がいい 何て間がいいんだろ 〔運がいい〕
\\	!/〔タイミングがいい〕
\\	間が抜ける 間が抜けた顔 
\\	彼がすることはすべて間が抜けている 
\\	間が悪い 〔きまりが悪い〕
\\	〔運が悪い〕
\\	〔折りが悪い〕
\\	間が悪く 
\\	間を持たす 彼が無口なので間を持たすのに苦労した 
\\	演説者の来るのが遅れたので司会者(しかいしゃ)は何とかして間を持たせようとした 
\\	あいだ 
\\	〔二つの物・場所にはさまれた部分〕
\\	間にテニスコートのある二つの建物 
\\	彼は木と木の間から首を出した 
\\	行と行の間をつめなさい[少しあけなさい] 
\\	東京と名古屋の間をバスが通っている 
\\	道を間にはさんで商売がたきの銀行が向き合っている 
\\	木の葉の間から陽がもれていた 
\\	2メートルの間をおいて柱が立っている 
\\	彼女は間をつめてノートに書く癖がある 
\\	ここから5キロの間は道が悪い 
\\	美術館に行くまでの間ずっと桜並木(なみき)だった 
\\	〔ある時からある時までの時間〕〔…の間ずっと〕
\\	〔…の期間中に〕
\\	9時間の間 
\\	しばらくの間 
\\	彼は規則(きそく)的な[3秒の]間をおいて太鼓を打った 
\\	10月10日から11月5日までの間 
\\	[((米))
\\	5日の間 (以内) に 
\\	夜の間ずっと強風が吹いていた 
\\	彼女はその間泣き続けた 
\\	そうしている間に事態は悪化した 
\\	留守(るす)の間に部屋に泥棒が入った 
\\	若い間に好きなことをしなさい 
\\	私が生きている間はあなたのお世話をします 
\\	〔中間〕 彼の事務所は駅と郵便局の間にある 
\\	食事と食事の間には何も食べない 
\\	私は青と藍(あい)の間の色が好きだ 
\\	この案は
\\	案と
\\	案の間をとったものである 
\\	〔関係,間柄〕 私たちの間を裂こうとしてもだめだ 
\\	その論争(ろんそう)は君たち二人の間で決着(けっちゃく)をつけなさい 
\\	通例二者間は
\\	三者以上は
\\	を用いるが,次例のように三者以上が個別的に関係しているときは
\\	が しばしば用いられる) 三社の間に
\\	の間に]協定が結ばれた 
\\	4枚(まい)の絵の間には類似点(るいじてん)が幾つか認められた 
\\	彼らの間はうまくいっている 
\\	[((口)) 
\\	彼との間には連帯責任がある 
\\	彼はギリシア人とフランス人の間に生まれた 
\\	〔あるグループの人たちの範囲〕 その男たちの間に女が2人混じっていた 
\\	その歌手は十代の少年少女の間で人気がある 
\\	議長は正会員の間から選ばれた 
\\	まあ	まあ	
\\	迷子	まいご	
\\	任せる	まかせる	
\\	〔委(ゆだ)ねる〕
\\	あの男にはこの仕事は任せられない 
\\	万事(ばんじ)あなたにお任せします 
\\	子供を母に任せる 
\\	少年の後見役(こうけんやく)は彼に任された 
\\	〔そのもののなすままにする〕 
\\	運(うん)を天に任せる 
\\	その件は自然の成り行きに任せよう 
\\	それはご想像に任せます 
\\	彼女はとうとう彼に身を任せた 
\\	[((文))
\\	足に任せて歩く 
\\	〔存分(ぞんぶん)に使う〕 存分 
\\	金に任せて 
\\	力に任せて 
\\	暇に任せて 
\\	暇に任せて今週は5冊も小説を読んだ 
\\	巻く	まく	
\\	蒔く	まく	牧草地(ぼくそうち)にクローバーの種(たね)をまく 
\\	庭に種をまいた 
\\	とうもろこし(玉蜀黍)の種を畑にまいた 
\\	蒔いた種は刈(か)らねばならぬ ((諺)) 
\\	蒔かぬ種は生えぬ ((諺)) 
\\	まく	
\\	撒く	
\\	〔あちこちに散らす〕
\\	凍った道に砂をまく 
\\	花壇(かだん)[芝生]に水をまく 
\\	ビラをまく 
\\	〔連れなどをはぐらかす〕
\\	うまく彼女をまいた 
\\	尾行(びこう)をまく 
\\	幕	まく	
\\	負け	まけ	
\\	孫	まご	
\\	まさか	まさか	
\\	まさか君がやったのではないだろうね 
\\	まさか本人に向かってばかだとも言えやしないしね 
\\	まさか一等(いっとう)になるとは思わなかった 
\\	「僕が殴ったのです」「まさか」 
\\	まさかの時 まさかの時には伯父の所へ行け 
\\	まさかの時の備えは出来ている 
\\	まさに	まさに	まさに【▲将に】 
\\	列車はまさに発車しようとしていた 
\\	まさに沈もうとするところを救われた 
\\	まさに【正に】 〔確かに〕⇒まさしく(正しく)〔ちょうど〕
\\	正にその通りです 
\\	それは正に私の言いたいところだ 
\\	彼は正に日本のピカソだ 
\\	交ざる	まざる	
\\	男の中に女が1人交じっていた 
\\	現地人に交じって暮らした 
\\	髪の毛に白いものが交じっていた 
\\	老人は若者に交じって踊った 
\\	ところどころ外国語の交じった日本文 
\\	混じる	まじる	
\\	麻(あさ)が混じった綿(めん) 
\\	あの少女には日本人とイタリア人の血が混じっている 
\\	この言語は色々の要素(ようそ)が混じっている 
\\	何も混じっていない水 
\\	交じる	まじる	
\\	男の中に女が1人交じっていた 
\\	現地人に交じって暮らした 
\\	髪の毛に白いものが交じっていた 
\\	老人は若者に交じって踊った 
\\	ところどころ外国語の交じった日本文 
\\	増す	ます	
\\	増(ふ)やす 
\\	〔数量・程度が多くなる〕
\\	詩に対する興味が増してきた 
\\	少年犯罪は増しつつある 
\\	体重が5キロ増した 
\\	車は速さを増した 
\\	川の水量が増してきた 
\\	彼の演技(えんぎ)は円熟味(えんじゅくみ)が増した 
\\	コンクリート造りにすると費用が50パーセント増す 
\\	それにも増して困ったことには 
\\	〔数量・程度を多くする〕
\\	⇒ふやす(増やす) 月給(げっきゅう)を5パーセント増す 
\\	低(ひく)い声が彼の威厳(いげん)を増している 
\\	貧しい	まずしい	
\\	ますます	ますます	
\\	もっと), 
\\	〔いっそう多く〕
\\	〔いっそう少なく〕
\\	勉強に対する興味がますます強くなった[薄(うす)れた] 
\\	見れば見るほどますますほしくなった 
\\	そこへ彼が入って来てますます困ったことになった 
\\	そこへ持ってきて 
\\	彼の行状はますます悪くなるばかりだった 
\\	その話を聞いてますます不愉快(ふゆかい)になった 
\\	革命の思想は国民の間にますます広まっていった 
\\	混ぜる	まぜる	
\\	〔加え入れる〕
\\	〔分離出来ないほどに〕
\\	砂糖とバターを混ぜる 
\\	数種(すうしゅ)のコーヒーを混ぜる 
\\	ウイスキーに水を混ぜる 
\\	二つの薬を混ぜる 
\\	〔かきまぜる〕
\\	コーヒーに牛乳を入れて混ぜた 
\\	卵をよくまぜる 
\\	交ぜる	まぜる	
\\	漢字に仮名を交ぜて書く 
\\	フランス語をよく交ぜて話す人だ 
\\	街 
\\	町	まち	〔都会〕
\\	〔商店・娯楽街の多い区域〕
\\	〔街路〕
\\	土曜日には必ず町に出掛ける 
\\	町を歩く 
\\	町医者 
\\	町方 町方の人 
\\	複数扱い) 町工場 
\\	町役人 
\\	町役場 
\\	間違い	まちがい	
\\	松	まつ	
\\	真っ赤	まっか	
\\	全く	まったく	
\\	祭	まつり	
\\	まとまる	まとまる
\\	纏まる】	
\\	〔集まる〕
\\	人数がまとまらない 
\\	生徒はまとまって乗車(じょうしゃ)するように 
\\	〔整う〕
\\	〔統一がある〕
\\	やっと考えがまとまった 
\\	その論文はまとまっている 
\\	このチームはまとまっている 
\\	党内がまとまっていない 
\\	〔決着がつく〕
\\	明日出発することで話がまとまった 
\\	交渉(こうしょう)がまとまった 
\\	彼の努力(どりょく)で条約(じょうやく)はまとまった 
\\	二人の縁談(えんだん)がまとまった 
\\	まとめる	まとめる	
\\	〔集める〕
\\	彼は数編(すうへん)の随筆(ずいひつ)をまとめて1冊の本にした 
\\	全部まとめて送ったほうがよい 
\\	1週間分の食料品をまとめて買う 
\\	まとめて金を払った 
\\	〔整える〕
\\	〔統一する〕
\\	データをまとめる 
\\	考えをまとめる 
\\	左派(さは)勢力(せいりょく)をまとめる 
\\	クラスをまとめるのが君の役目だ 
\\	〔決着をつける〕
\\	〔調停する〕
\\	縁談(えんだん)をまとめる 
\\	交渉(こうしょう)をまとめる 
\\	論争(ろんそう)をまとめる 
\\	まとめ役 〔調停者〕
\\	〔争いの〕
\\	〔調整者〕
\\	学ぶ	まなぶ	
\\	真似	まね	
\\	〔模倣〕
\\	〔物まね〕
\\	〔さるまね〕
\\	〔手本としてまねること〕
\\	おうむ(鸚鵡)は人間の声色(こわいろ)のまねをする 
\\	彼は人のまねがうまい 
\\	死んだまねをする 
\\	/((文))
\\	これは私には真似の出来ない作品です 
\\	〔振る舞い〕
\\	((英)) 
\\	ばかなまねはするな 
\\	私にはあんな真似は出来ない 
\\	招く	まねく	
\\	まぶしい	まぶしい	
\\	〔目をくらませるような〕
\\	〔ぎらぎらする〕
\\	太陽がまぶしかった 
\\	まぶしい光に目がくらんだ 
\\	〔まともに見られない〕 昨日の失態(しったい)で,今日はみんなの顔がまぶしい 
\\	豆	まめ	
\\	守る	まもる	
\\	〔防御する〕
\\	〔警護する〕
\\	〔保護する〕
\\	敵から国を守る 
\\	門を守る 
\\	日光から目を守る 
\\	危害(きがい)から身を守る 
\\	名誉を守る 
\\	〔大事に保(たも)つ〕 節操(せっそう)を守る 
\\	沈黙(ちんもく)を守る 
\\	伝統を守る 
\\	我がチームは今まで1敗を守ってきた 
\\	〔決められたことに従う〕
\\	約束を守る 
\\	規則を守る 
\\	(防御する)
\\	(保護する)
\\	(警護する)
\\	(約束を)
\\	(規則などを)
\\	【野球】
\\	迷う	まよう	
\\	〔道が分からなくなる〕
\\	〔はぐれる〕
\\	知らない場所に迷い込んだ 
\\	とんぼが部屋に迷い込んだ 
\\	〔思い惑う〕
\\	〔決断できない〕
\\	行くべきかどうか迷っている 
\\	希望と不安の間を迷っている 
\\	彼は決断(けつだん)に迷った 
\\	彼に話すべきかどうか迷っている 
\\	[((米))
\\	/((英))
\\	〔魅(み)せられる〕 彼は女に迷った 
\\	恋に迷う 
\\	巧言令色(こうげんれいしょく)に迷う 
\\	〔霊魂(れいこん)が浮かばれない〕 彼の亡霊がまだ迷っている 
\\	迷わず成仏(じょうぶつ)するようにと祈った 
\\	丸	まる	
\\	〔円〕
\\	丸を書く 
\\	該当(がいとう)する項目(こうもく)の番号を丸で囲(かこ)め 
\\	〔正しいことを示す印〕 答案に丸[二重丸]をもらった 
\\	〔完全,全体〕⇒まるごと(丸ごと) りんごを丸のまま食べた 
\\	〔城の一部〕 
\\	西の丸 
\\	彼が死んでから丸5年になる 
\\	その仕事は丸1日かかる 
\\	丸12時間眠った 
\\	まるで	まるで	
\\	〔通例,否定語を伴って,全く…ない〕⇒まったく(全く) それが何かまるで見当がつかない 
\\	そんなことはまるで気に掛けなかった 
\\	[((俗)) 
\\	試験の結果はまるで駄目でした 
\\	そんな申し出ではまるで話にならない 
\\	話にならない 
\\	〔あたかも〕
\\	⇒あたかも(恰も) あの人はまるでドイツ人のようにドイツ語を話す 
\\	万一	まんいち	
\\	〔非常の場合〕 万一の場合にはこれを持って逃げなさい 
\\	万一に備(そな)える 
\\	万一のことがあっても私がいるから大丈夫だよ 
\\	万一に備えて少しばかり貯金をしている 
\\	〔わずかな望み〕 万一を頼む 〔見込みがないのに望む〕
\\	/〔運をあてにする〕
\\	彼は万一を頼んで大事業に乗り出した 
\\	〔もしも〕 万一列車に乗り遅れたら先に行ってください 
\\	満足	まんぞく	
\\	身	み	
\\	(体)
\\	(自分)
\\	(心)
\\	(立場)
\\	(肉)
\\	(生命)
\\	(刀身)
\\	〔体,自分の体〕
\\	茂(しげ)みに身を隠(かく)す 
\\	川に身を投げる 
\\	彼女は清潔(せいけつ)なブラウスを身につけていた 
\\	貴重品(きちょうひん)はいつも身につけている 
\\	身を寄せあって寒さをしのいだ 
\\	〔自分自身〕 浪費(ろうひ)が身の破滅(はめつ)を招いた 
\\	ただちに出ていったほうが身のためだぞ 
\\	/((俗)) 
\\	身を引き締めて困難に立ち向かう 
\\	〔立場,身分〕
\\	私の身にもなってごらんなさい 
\\	〔皮・骨に対して,肉〕
\\	魚の骨を身から外す 
\\	骨ばかりで身がない 
\\	〔樹皮に包まれた部分〕
\\	木の皮をはいで身を調べる 
\\	身が縮む 身の縮(ちぢ)む思いをする 〔恥じ入る〕
\\	/〔すくむ〕
\\	身が入る 勉強に身が入らない 
\\	⇒みをいれる(身を入れる) 身が持たない こう忙しくては身が持たない 
\\	身から出た錆(さび) 身から出た錆だ 
\\	聖書の言葉から) 身に余る 身に余る光栄です 
\\	身に覚えがある 身に覚えがあるでしょう 
\\	全く身に覚えがない 
\\	身を誤る 
\\	身を売る 
\\	身を起こす 一介(いっかい)のセールスマンから身を起こし大会社の重役になった 
\\	身を落とす やくざに身を落とすまでになった 
\\	身を固める 
\\	身をかわす ⇒かわす(躱す) 身を切られる 身を切られる思いがする 
\\	/((口)) 
\\	身を切る 身を切るような北風 
\\	身を殺す 
\\	実	み	
\\	見送り	みおくり	
\\	味方	みかた	
\\	〔友人〕
\\	〔支持者〕
\\	〔同盟〕
\\	味方する 
\\	敵も味方も 
\\	人を味方に引き入れる 
\\	いつも君の味方だ 
\\	あなたはどちらの味方ですか 
\\	見事	みごと	
\\	〔すばらしいこと〕 見事な 〔優れた〕
\\	〔立派な〕
\\	〔最高の〕
\\	〔あっぱれな〕
\\	見事な出来栄えのつぼ 
\\	見事なはすの花 
\\	彼の勇気は見事だ 
\\	見事にやってのける 
\\	空は見事に晴れわたっていた 
\\	見事見事 
\\	!/〔スポーツで〕
\\	〔完全に〕 見事にだまされた 
\\	見事にやっつけられた 
\\	満ちる	みちる	
\\	〔あふれるばかりになる〕
\\	バケツには水が満ちている 
\\	希望に満ちている 
\\	活気(かっき)に満ちた若者たち 
\\	〔満潮(まんちょう, みちしお)になる〕 潮(しお)が満ちている 
\\	〔満月になる〕 月が満ちている 
\\	〔期限になる〕
\\	⇒まんき(満期) 月が満ちて子供が生まれた 
\\	密	みつ	
\\	〔多く集まっていること〕 この国は人口が密である 
\\	〔秘密〕 はかりごとは密なるをよしとす 
\\	⇒ひみつ(秘密)
\\	認める	みとめる	
\\	〔見える〕
\\	〔気付く〕
\\	そこには人影(ひとかげ)は認められなかった 
\\	〔判断する〕
\\	〔みなす〕
\\	法廷(ほうてい)は彼を犯人と認めた 
\\	彼は正直(しょうじき)な人物と認めます 
\\	〔真実・有効性などを認める〕
\\	〔適当・正常だと認める〕
\\	〔よいと評価する〕
\\	〔許可する〕
\\	彼は自分の誤(あやま)りを[自分が誤っていると]認めた 
\\	欠席(けっせき)理由を認める 
\\	だれも彼女の行いを認めなかった 
\\	彼の偉業(いぎょう)は世に認められた 
\\	先生は試験に鉛筆の使用を認めた 
\\	犯行を認めた 
\\	学校は二通(にどお)りの制服を認めている 
\\	見舞い	みまい	
\\	〔見舞うこと〕
\\	〔見舞いの言葉〕
\\	⇒みまう(見舞う) 負傷者(ふしょうしゃ)を見舞いに行った 
\\	暑中見舞い 〔葉書〕
\\	英米には暑中見舞いを出す習慣はない) 病人に見舞いの言葉を述べる 
\\	〔悪い物事(ものごと)にあうこと〕 父のげんこつのお見舞いを受けた 
\\	/((口)) 
\\	見舞い客 
\\	見舞い金 
\\	〔慰謝料(いしゃりょう)〕
\\	((複-
\\	火事の見舞い金をたくさんもらった 
\\	見舞い状 
\\	見舞い品 
\\	土産	みやげ	
\\	都	みやこ	
\\	妙	みょう	
\\	〔巧(たく)みさ〕 造化の妙 
\\	妙を心得(こころえ)ている 
\\	演技の妙を見せてくれた 
\\	言い得て妙である 
\\	〔不思議な様子〕 妙な 
\\	はなじみがなく不思議な,
\\	は好奇心をそそる,
\\	は普通と違って奇妙な) 妙に 妙に青ざめていた 
\\	今日は彼は妙に遅いなあ 
\\	このごろは妙に事故が多い 
\\	妙な話 
\\	妙なことに靴が片方(かたほう)なくなっていた 
\\	妙なうわさを耳にした 
\\	その地方には妙な風習がある 
\\	それを聞くと彼は妙な顔をした 
\\	彼がそんなこと言うとは妙だ 
\\	明後日	みょうごにち	
\\	未来	みらい	
\\	〔将来〕
\\	未来の 
\\	未来を予測する 
\\	彼女は未来のスターだ 
\\	これが未来の夫です 
\\	彼は建築家(けんちくか)として大きな未来がある 
\\	〔仏教で,来世〕
\\	〔文法で時制の〕
\\	未来永劫(えいごう) 
\\	未来永劫(えいごう)忘れるな 
\\	未来学 
\\	未来学者 
\\	未来像 この本は社会の未来像を描いている 
\\	未来派 
\\	〔人〕
\\	魅力	みりょく	
\\	は引きつける力,
\\	は心に訴える力) 魅力的 
\\	魅力がない 
\\	性的魅力 
\\	独特の魅力をもった文体 
\\	彼女の美しい目に魅力を感じた 
\\	その提案は魅力がない 
\\	彼女は魅力的な女性だ 
\\	ああいうタイプの男性には何の魅力も感じない 
\\	ロックは特に若者にとって魅力がある 
\\	彼の小説の何が魅力なんですか 
\\	「やっぱり腹いっぱい食いたいときはバイキングがいいね」「どれだけ食べても値段が同じってのが魅力だよな」 
\\	「この仕事でこのギャラじゃ、あまり魅力を感じないな」「うん。やっぱり断ることにするか」 
\\	ギャラ 
\\	〔保証出演料〕
\\	「君は彼女のどんなところに魅力を感じたの?」「やっぱり気さくで話しやすいところかな」 
\\	「彼のあの笑顔が魅力的なのよね」「うん、怖い顔して考え事しているときもいいけどね」 
\\	みちのみりょく【未知の魅力】 
\\	この
\\	は「誘惑」「魅力」の意。英語も一つの決まった言い方である。
\\	を
\\	に換えて
\\	としても同様。
\\	(未知への魅力が宇宙探検の強い動機となっていることは明白である)。
\\	のようなかたちで言うこともできる。
\\	(未知の魅力に多くの人々はその命運を賭ける)。 「ブラジルに来られた動機は?」「まあ、未知の魅力といったところでしょうか」 
\\	未知への魅力が宇宙探検の強い動機となっていることは明白である 
\\	未知の魅力に多くの人々はその命運を賭ける 
\\	診る	みる	
\\	医者に診てもらう 
\\	医者が脈(みゃく)[患者]を診た 
\\	《眼科》もう少しよく診てみましょう。(・・・)ここにあごを載せて・・・(目を)大きく開いて・・・肩の力を抜いて・・・心配することはありませんよ 
\\	向かい	むかい	
\\	向かいの 
\\	向かいの家 
\\	私たちは消防署(しょうぼうしょ)の向かいに住んでいる 
\\	お向かいさん 
\\	向かい風 向かい風を受けて走る 
\\	向かい火 
\\	山火事(やまかじ)を食い止めるために向かい火をたいた 
\\	焚く (たく) 
\\	迎え	むかえ	
\\	〔迎えること〕
\\	〔迎える人〕
\\	駅まで迎えに行く 
\\	迎えに来る 
\\	医者を迎えにやる 
\\	お迎えが参りました 
\\	10時にお迎えの車がそちらに参ります 
\\	〔仏が浄土に呼びよせること〕 お迎えが来るのを待っています 
\\	迎え酒 
\\	((俗)) 
\\	迎え火 
\\	迎え飛車 
\\	迎え水 
\\	向く	むく	
\\	(向きを変える)
\\	(顔を向ける)
\\	(面する)
\\	(指す)
\\	(適する)
\\	(気持ちが)
\\	〔体の向きをかえる〕
\\	〔顔を向ける〕
\\	右を向く 
\\	後ろを向く 〔人に対して〕
\\	/〔振り返る〕
\\	風向計が北を向いた 
\\	上[下]を向く 
\\	そっぽを向く 
\\	どこを向いても美しい花ばかりだった 
\\	下を向いたまま答えた 
\\	〔面する〕
\\	東に向いた家 
\\	庭に向いた窓 
\\	〔合う〕
\\	⇒むき(向き)2 若い人に向いた仕事 
\\	彼はそんな役割に向いている 
\\	この川岸(かわぎし)はキャンプに向いていない 
\\	〔気持ちなどが傾く〕 気が向いたら行きます 
\\	気の向かない日は能率が上がらない 
\\	気の向くままにしてよろしい 
\\	運が向いてきたようだ 
\\	(皮を)むく	(かわを)むく	
\\	むく
\\	剥く】 
\\	ははがすように,
\\	はナイフでそぎとるように) みかんの皮(かわ)をむく 
\\	じゃがいもの皮をむく 
\\	エンドウ豆の皮をむく 
\\	木の皮をむく 
\\	うさぎの皮をむく 
\\	向ける	むける	
\\	〔向かせる〕
\\	窓の方に顔[目]を向ける 
\\	銃を男に向けた 
\\	顔を上に向けて 
\\	戸口(とぐち)に背(せ)を向けて立つ 
\\	機首(きしゅ 
\\	を東に向けた 
\\	北海道へと足を向ける 
\\	新しい事態に皆の注意を向けた 
\\	計画(けいかく)に背を向ける 
\\	〔目指す〕 イギリスに向けて出発した 
\\	その皮肉は私に向けられたものだった 
\\	問題解決に向けて努力した 
\\	〔行かせる〕
\\	使者を向ける 
\\	〔充(あ)てる 
\\	研究に心を向けた 
\\	その金は旅費に向けた 
\\	一日のうち2時間をピアノの練習に向けている 
\\	無視	むし	無視する 
\\	規則を無視する 
\\	個人の意思を全く無視して 
\\	理不尽(りふじん)な要求を無視する 
\\	事故は信号無視が原因である 
\\	無視していいほどの微量(びりょう) 
\\	蒸し暑い	むしあつい	
\\	蒸し暑い天気 
\\	はじめじめして,
\\	は風がなくてうだるような) 今日は蒸し暑くてやり切れない 
\\	虫歯	むしば	
\\	むしろ	むしろ	
\\	この絵は写実派的(しゃじつはてき)というよりむしろ印象派的だ 
\\	彼を裏切るぐらいならむしろ死んだほうがいい 
\\	彼女は歌手というよりむしろ女優だ 
\\	保守派(ほしゅは)か
\\	いや,むしろ実用主義者だ 
\\	蒸す	むす	
\\	〔ふかす〕
\\	いもを蒸す 
\\	〔蒸し暑い〕 今夜はひどく蒸す 
\\	結ぶ	むすぶ	
\\	無駄	むだ	
\\	夢中	むちゅう	
\\	胸	むね	
\\	無料	むりょう	
\\	芽	め	
\\	明確	めいかく	明確な 〔はっきりした〕
\\	〔確実な〕
\\	〔的確な,正確な〕
\\	明確に 
\\	明確な答え 
\\	明確な発音 
\\	この点を明確にする必要がある 
\\	公(おおやけ/こう)と私(し)を明確に区別する 
\\	彼の説明は明確さを欠(か)いている 
\\	名刺	めいし	
\\	名詞	めいし	
\\	命じる	めいじる	
\\	〔言いつける〕
\\	命じられるままそこに行った 
\\	上司は彼にすぐ出発するように命じた 
\\	〔任命する〕
\\	調停(ちょうてい)委員を命じられた 
\\	名人	めいじん	
\\	命令	めいれい	
\\	迷惑	めいわく	
\\	目上	めうえ	
\\	飯	めし	
\\	めった	滅多	
\\	〔無分別(むふんべつ 
\\	むやみやたらなこと〕 あの人にはめったなことは言えない 
\\	めったなまねをするんじゃない 
\\	〔「めったにない」の形で,ほとんどない〕 
\\	めったにないチャンスだ 
\\	/〔一生一度の〕
\\	そんなことはめったにない ((口)) 
\\	〔「めったに…ない」の形で〕 日曜日はめったに家にいない 
\\	めったに彼に会わない 
\\	めったにだまされはしない〔簡単に〕 
\\	めった打ち 
\\	村人(むらびと)は男をめった打ちにした 
\\	めった斬[切]り 
\\	人をめった斬(き)りにする 
\\	めった斬りの死体 
\\	「部長、やっぱりうちの会社、危ないんですか?」「この店は業界の人間が多いから、めったなことは言わない方がいいぞ」 
\\	「おたくのご主人、夜は遅いの?」「そうね。平日に家で晩ご飯を食べることは、めったにないわね」 
\\	「今度、うちの学校出身のプロ選手がテニスの指導に来てくれるんだって」「そうか。めったにない機会だから、よく見て技を盗むんだぞ」 
\\	「珍しい型のカメラだね」「限定品だから、めったに手に入らない型なんだよ」 
\\	面	めん	
\\	(仮面)
\\	(剣道の)
\\	(表面)
\\	(前面)
\\	(局面・側面)
\\	(新聞の)
\\	〔顔〕
\\	(お) 面のいい女の子 
\\	〔仮面〕
\\	〔剣道などの〕
\\	般若(はんにゃ)の面 
\\	面をかぶる[はずす] 
\\	面をつける[つけている]〔剣道などで〕 
\\	〔剣道のわざ〕
\\	面を1本取られた 
\\	面を取った 
\\	〔新聞の〕
\\	第1面 
\\	〔数学で〕
\\	〔平面〕
\\	水平面 
\\	面対称(たいしょう) 
\\	〔角材などの角を45度に面取りした面〕
\\	面を取る 
\\	〔局面〕
\\	〔方面〕
\\	人生の暗い面 
\\	彼はあらゆる面ですばらしい 
\\	電子工学の面での優れた業績 
\\	資金の面で援助しよう 
\\	面が割(わ)れる 
\\	〔写真から身元が分かる〕 被害者の面が割れた 
\\	〔顔を知られる〕 私は面が割れているから,君が行ってくれ 
\\	面と向かって 
\\	〔向き合って〕 彼は容疑者と面と向かって座った 
\\	〔ずけずけと
\\	面と向かってよくそんな事が言えるもんだ 
\\	免許	めんきょ	
\\	面接	めんせつ	
\\	⇒めんかい(面会) 面接する 
\\	一人一人に面接してアンケート調査をした 
\\	面接試験 
\\	((口)) 
\\	面接試験を行う[受ける] 
\\	面接室 〔会見室〕
\\	〔面接試験の〕
\\	面接者 〔行う側〕
\\	〔受ける側〕
\\	例文 選考に残られた方に面接のご連絡をいたします 
\\	弊社(へいしゃ)役員秘書職の面接にお越しいただき、ありがとうございました
\\	お越(こ)しいただき 
\\	相手に自分のもとへ来てもらうことを丁寧に言う表現。「お越しいただきありがとうございます」などのように、他の語に繋(つな)げる言い方。 面接で話し合ったとおり、9月1日から出勤してください 
\\	ぜひグローバル社へ面接におうかがいしたいと思いますが、1月17日から22日までは外出しておりますので、 その間はうかがえないことを記しておきます 
\\	面接でお話しをしたいと思います 
\\	ぜひ面接にお越しください 
\\	応募書類を拝見しました。面接にお越しいただければと存じます 
\\	先週の金曜日は面接してくださり、ありがとうございました 
\\	10月1日の面接にお越し下さいましてありがとうございました 
\\	あなた、面接においでになれます? 
\\	御出でになる 
\\	面倒	めんどう	
\\	〔手数〕
\\	面倒な 
\\	面倒な仕事 
\\	彼らにいちいちそれを送り返すのが面倒になった 
\\	毎回書類に書き込むのは本当に面倒だ ((口)) 
\\	これ以上あなたにご面倒はかけません 
\\	ご面倒をかけてすみませんがコピー機の使い方を教えていただけませんか 
\\	彼は面倒がって床屋(とこや)へ行こうとしなかった 
\\	彼女は面倒がらずにしゅうとめの世話をする 
\\	〔困難〕
\\	〔いざこざ〕
\\	面倒な 〔厄介な〕
\\	〔困難な〕
\\	〔込み入った〕
\\	〔ゆゆしい〕
\\	面倒な事件 
\\	事態は面倒なことになった 
\\	彼は警察と面倒を起こしている 
\\	いつも面倒ばかり起こしている 
\\	そんなことをいつまでもやっていると面倒なことになるよ 
\\	〔世話〕
\\	孤児(こじ)の面倒を見ている 
\\	老母(ろうぼ)は息子に面倒を見てもらっている 
\\	私の言うことを聞こうとしないのでもうあいつの面倒は見切れない 
\\	面倒見(み) 
\\	あの先生は学生の面倒見がいい[悪い] 
\\	申し込む	もうしこむ	
\\	〔申し入れる〕
\\	〔依頼する〕
\\	彼はメアリーに結婚を申し込んだ 
\\	彼らは社長に面会(めんかい)を申し込んだ 
\\	わがチームは彼らに野球の試合を申し込んだ 
\\	〔出願する〕
\\	〔出資・寄付を約束する〕
\\	〔予約する〕
\\	直接または手紙で申し込んでよい 
\\	その雑誌の定期購読(ていきこうどく)を申し込んだ 
\\	図書館建設(けんせつ)に10万円の寄付を申し込んだ 
\\	ホテルで一番いい部屋を申し込んだ 
\\	申し訳	もうしわけ	
\\	毛布	もうふ	
\\	燃える	もえる	
\\	目的	もくてき	
\\	目標	もくひょう	
\\	〔ねらい〕
\\	〔最終目的〕
\\	〔達成目標〕
\\	学習の目標 
\\	人生の目標をきめる 
\\	何をするにしても目標を高く定めるべきだ 
\\	売り上げ目標 
\\	昨年の政策目標は全部達せられた 
\\	〔目印〕
\\	〔飛行機の〕
\\	〔射撃の〕
\\	私の家は学校を目標にして来ればすぐ分かります 
\\	目標に向けて銃の狙(ねら)いを定める 
\\	目標額 
\\	文字	もじ	
\\	もしかすると	もしかすると	
\\	もしかすると彼が正しいのかもしれない 
\\	もしかしてあの人は田中さんではないかしら 
\\	もしかして彼に会えたら,私が会いたがっていると伝えてくれ 
\\	もしも	もしも	
\\	⇒もし(若し) もしものこと もしものことがあったら電話しなさい 
\\	持ち上げる	もちあげる	
\\	〔持って上げる〕
\\	片手(かたて)でいすを持ち上げる 
\\	帽子を持ち上げてあいさつする 
\\	〔起こす〕 頭を持ち上げる 
\\	〔おだてる〕
\\	彼を持ち上げてその役目(やくめ)を引き受けさせた 
\\	あの人は持ち上げてもだめだ 
\\	用いる	もちいる	
\\	〔使う〕
\\	〔利用する〕
\\	〔採用する〕
\\	腕力を用いるな 
\\	その機械は広く用いられている 
\\	この方法には重力の法則が用いられている 
\\	今度は新しい方法を用いよう 
\\	〔任用する〕 
\\	彼は重く用いられるようになった 
\\	もったいない	もったいない	
\\	〔惜(お)しい〕
\\	水をそんなに流してはもったいない 
\\	〔過分(かぶん)である〕 
\\	私のような者にはもったいないおほめの言葉です 
\\	彼女の立派な料理は彼にはもったいない 
\\	〔恐れ多い〕 そんな所に仏像(ぶつぞう)を置いてはもったいない 
\\	もっとも	尤も 最も	尤も 
\\	〔道理にかなう様子〕 
\\	あなたがそう言うのももっともだ 
\\	妻としてはもっともな行動だ 
\\	君が不平を言うのももっともだ 
\\	彼女が彼を避(さ)けるのにはもっともな理由がある 
\\	〔ただし,そうは言うものの〕 
\\	彼は力が強い
\\	もっとも体も大きいがね 
\\	女性にはこの仕事はできない
\\	もっとも例外はあるが 
\\	彼女は英語が上手だ
\\	もっともアメリカ育ちだからね 
\\	最も 最も印象(いんしょう)的だったこと 
\\	世界で最も高い山 
\\	この川はここが最も広い 
\\	今までに見た映画の中で最も好きなのはどれですか 
\\	彼は私が最も嫌いなタイプだ 
\\	元	もと	
\\	〔起こり〕
\\	元は 
\\	事件の元をたどる 
\\	この計画(けいかく)は元は彼が考え出したものである 
\\	失敗は成功の元 
\\	口は禍(わざわい)の元 
\\	禍 
\\	災い 
\\	〔原因,根源〕
\\	けんか(喧嘩)の元 
\\	あの大火(たいか)の元はたばこの不始末(ふしまつ)だった 
\\	使い込みが元で彼は会社を首になった 
\\	使い込み 
\\	〔以前〕 元首相 
\\	元の同僚(どうりょう) 
\\	大スターも元は田舎(いなか)の小娘(こむすめ)に過ぎなかった 
\\	読んだら元の所に戻しておいてください 
\\	ここは元は荒(あ)れ地だった 
\\	命はとりとめたが,元の体にはならなかった 
\\	〔原料〕
\\	この薬は樹皮(じゅひ)を元にして作ったものである 
\\	〔元手〕
\\	元がかかる事業 
\\	この商売は元がとれない 
\\	元の鞘(さや)に収まる 
\\	あの夫婦は元の鞘に収(おさ)まった 
\\	元の木阿弥(もくあみ) 
\\	株の大暴落(だいぼうらく)で元の木阿弥となった 
\\	結局元の木阿弥だった 
\\	元も子もない 
\\	これだけしか払ってもらえないのでは,元も子もない 
\\	基	もと	
\\	この小説は事実を基にして書かれたものだ 
\\	何を基にして私を疑(うたが)うのか 
\\	戻す	もどす	
\\	〔元に返す〕
\\	本を読んだら元の位置[棚]へ戻しなさい 
\\	時計を2時間戻す 
\\	わかめを水で戻す 
\\	太郎は和子とよりを戻した 
\\	〔吐く〕
\\	彼は食べた物を全部戻してしまった 
\\	車の中で戻しそうになった 
\\	のほうが上品な言い方)
\\	基づく	もとづく	
\\	求める	もとめる	
\\	〔探す〕
\\	富(とみ)を求める 
\\	職を求めている 
\\	事件の解決の手がかりを求めて歩き回っている 
\\	手がかり 
\\	販売員を求む〔掲示〕 
\\	の値を求めよ〔問題で〕 
\\	〔要求する〕
\\	〔頼む〕
\\	〔依頼する〕
\\	組合は,会社側に給料を上げるように求めた 
\\	[((口)) 
\\	あの人に援助を求めてごらんなさい 
\\	社長に面会を求めた 
\\	声を上げて助けを求めた 
\\	企画の説明を求められた 
\\	〔買う〕
\\	⇒かう(買う)
\\	者	もの	
\\	人 (ひと))
\\	物音	ものおと	
\\	物語	ものがたり	
\\	物事	ものごと	
\\	物事を苦(く)にする[しない] 
\\	物事をよくわきまえている 
\\	燃やす	もやす	
\\	模様	もよう	
\\	〔図案〕
\\	は図柄の繰り返しを示唆し
\\	は模様全体をまとまりとしてとらえた言葉) 水玉(みずたま)模様のプリント地 
\\	花模様のあるじゅうたん 
\\	手書き模様を施(ほどこ)した茶碗 
\\	蝶を羽(はね)の模様で見分ける 
\\	〔有り様,様子〕
\\	この空模様では雪が降りそうだ 
\\	模様によっては予定(よてい)を変更することもある 
\\	この模様では会議は長引(ながび)きそうだ 
\\	試合の模様はどうですか 
\\	どうも彼は来ない模様だ 
\\	[((米))
\\	火星(かせい)には生物(せいぶつ)はいない模様である 
\\	模様替え 部屋を模様替えする 〔内装をかえる〕
\\	/〔家具を並べ変える〕
\\	市の公会堂はすっかり模様替えされた 
\\	その土地の開発計画は市民公園の建設(けんせつ)に模様替えになった 
\\	文句	もんく	
\\	〔語句〕
\\	〔表現〕
\\	こういう文句で書き出したらどうだ 
\\	その一節の文句は覚えていないが,とても感動的だった 
\\	きまり文句 
\\	あいさつの文句 
\\	気のきいた[しゃれた]文句を使う 
\\	〔不満,言い分〕 文句を言う 〔苦情を言う〕
\\	/((口)) 
\\	/〔異議をと なえる〕
\\	/〔非難する〕
\\	食事のことで妻に文句を言った 
\\	この案に文句があるなら早く言いたまえ 
\\	彼女は嫁(よめ)のすることにいちいち文句をつけた 
\\	文句はない 
\\	[((口))
\\	今の環境に何も文句はありません 
\\	この絵は文句なしに彼の代表作である 
\\	文句なしに彼の勝ちだった 
\\	彼の演奏は文句なしだった 
\\	やがて	やがて	
\\	〔間もなく〕 日は沈み,やがて月が出てきた 
\\	/((文))
\\	やがて城が見えてきた 
\\	やがて (は) 彼にも分かる時が来るでしょう 
\\	〔およそ,ほとんど〕 彼が行ってからやがて1年になる 
\\	〔結局〕 そのためやがては自分で自分の首を絞(し)めることになった 
\\	自然を守ることはやがて人間社会を救(すく)う道につながるだろう 
\\	役	やく	
\\	〔任務,役割〕
\\	彼らはめいめいの役を立派にこなした 
\\	彼は何も役がない 
\\	私は献立(こんだて)を決める役を引き受けた 
\\	私には役が重過ぎる 
\\	彼女は団長秘書の役で旅行団に参加した 
\\	いざという時には傘でも武器の役を果たす 
\\	〔地位〕
\\	〔責任ある地位・公職〕
\\	役を退(しりぞ)く 
\\	ただ1回の失策(しっさく)で彼はお役御免(ごめん)となった 
\\	彼は次官(じかん)の役に任(にん)ぜられた 
\\	彼が市長の役にある間は汚職(おしょく)事件がなかった 
\\	彼女の役は何ですか 
\\	〔芝居の〕
\\	シーザーの役を演じる 
\\	母と娘の二役を務める 
\\	彼女にカルメンの役が振り当てられた 
\\	〔トランプなどの〕 役が付く 
\\	はマージャンで) 役に立つ 
\\	役に立たない 
\\	/((口)) 
\\	あの青年はなかなか役に立つ 
\\	私で何かお役に立ちましょうか 
\\	そのような訓練(くんれん)もいつかは役に立つでしょう 
\\	この古い自転車が何かの役に立つだろうか 
\\	泣いたところで何の役に立つのか 
\\	約	やく	
\\	約3時間 
\\	約2万人の人々 
\\	損害は約1億円と見積(みつ)もられている 
\\	約2マイルほど徒歩(とほ)で進んだ 
\\	約500万円かかると思います 
\\	この町の人口は約3万人です 
\\	訳	やく	
\\	訳す	やくす	
\\	役割	やくわり	
\\	光は植物(しょくぶつ)の生育(せいいく)に重要な役割を果たす 
\\	演技者(えんぎしゃ)の中には自分の役割が気に入らない者もいた 
\\	彼は自分の役割を果たしていない 
\\	家賃	やちん	
\\	厄介	やっかい	
\\	〔面倒なこと〕
\\	厄介な 
\\	彼が厄介な仕事を持ち込んで来た 
\\	厄介なことに彼は警官をなぐってしまった 
\\	子供5人の世話をするのは本当に厄介だ 
\\	彼の不用意な発言が事態をさらに厄介にした 
\\	たびたびご厄介をかけて申し訳ありませんでした 
\\	人に厄介をかけるようなことはしてはなりません 
\\	〔世話〕 
\\	彼はいい年をしてまだ親の厄介になっている 
\\	年をとっても息子の厄介になるつもりはない 
\\	1年に一度は医者の厄介になる 
\\	たびたび睡眠薬のご厄介になる 
\\	時々警察の厄介になる 
\\	息子は先週から友だちの家に厄介になっている 
\\	京都に滞在中は大変ご厄介になりました 
\\	厄介払い 
\\	厄介払いするように彼を追い出した 
\\	厄介者 
\\	〔居候〕
\\	((複
\\	〔扶養家族〕
\\	うちには厄介者が二人いる 
\\	今兄の家の厄介者になっています 
\\	〔困り者〕
\\	彼は家族に厄介者扱いされている 
\\	彼は何かというと問題を起こす厄介者だ 
\\	何かというと 
\\	宿	やど	
\\	雇う	やとう	
\\	屋根	やね	
\\	破る	やぶる	
\\	〔引き裂く〕
\\	が一般的な語,
\\	は裂き易い方向,縫目などにそって勢いよく引き裂く) スカートをくぎに引っ掛けて破いてしまった 
\\	彼は手紙を二つに破いた 
\\	だれかがこの本の一部を破り取った 
\\	〔壊す〕
\\	窓ガラスを破る 
\\	警官はドアを破って入った 
\\	〔事態をだめにする〕 平和を破る 
\\	少年の夢は破られた 
\\	黄色のカーテンがその部屋の調和(ちょうわ)を破っていた 
\\	〔約束事にそむく〕 約束を破ってはならない 
\\	ならない 
\\	それは古来(こらい)の伝統を破ることになる 
\\	は「犯す」,
\\	は「逸脱する」意) 
\\	〔牢(ろう)などを〕 囚人(しゅうじん)が牢を破った 
\\	関所(せきしょ)を破る 
\\	〔突き抜ける〕 サイレンが夜の静けさを破った 
\\	〔改める〕 鈴木は世界記録を破った 
\\	破れる	やぶれる	
\\	〔裂ける〕
\\	〔縫い目などにそって〕
\\	上着がくぎに引っ掛かって破れた 
\\	ズボンのおしりが破れた 
\\	あちこち破れた服を着ていた 
\\	この布はすぐ破れる 
\\	[〔すり切れる〕 
\\	〔壊れる〕
\\	風船(ふうせん)が破れた 
\\	破れた靴 
\\	〔成り立たなくなる〕 和平交渉は破れた 
\\	彼女の夢は破れた 
\\	両国間の均衡(きんこう)が破れた 
\\	〔失敗して傷付く〕 彼女は恋愛(れんあい)[結婚]に破れて故郷(さと)に帰った 
\\	辞める	やめる	
\\	やや	やや	
\\	〔少し〕
\\	〔いくらか〕
\\	やや太りぎみだ 
\\	こちらの本のほうがやや厚い 
\\	やや驚いた様子だった 
\\	彼にはややこっけいなところがある 
\\	彼の態度はやや軟化(なんか)したようだ 
\\	やや失望した様子だった 
\\	〔しばらく〕
\\	ややあって彼女が現れた 
\\	唯一	ゆいいつ	
\\	勇気	ゆうき	
\\	友好	ゆうこう	
\\	有効	ゆうこう	
\\	〔効力があること〕 有効な 〔効果がある〕
\\	〔ある期間・条件のもとで〕
\\	有効な手段を取る 
\\	この薬は頑固(がんこ)な頭痛に有効である 
\\	この法規(ほうき)は交通事故をなくすのに有効だろう 
\\	この契約は5年間有効だ 
\\	その法律はまだ有効である 
\\	不可侵(ふかしん)条約(じょうやく)が有効な間,わが国は安全である 
\\	〔役に立つこと〕 彼は研究休暇(きゅうか)を有効に使った 
\\	有効期間[期限] 
\\	切符の有効期間 
\\	有効求人倍率(きゅうじんばいりつ) 
\\	今年の求人倍率は2.1倍に回復した 
\\	有効距離(きょり)[射程(しゃてい)] 
\\	有効需要(じゅよう) 
\\	有効数字 〔数学で〕
\\	有効性 
\\	有効成分 
\\	〔薬の〕
\\	有効票 
\\	優秀	ゆうしゅう	
\\	優勝	ゆうしょう	
\\	〔第一位になること〕
\\	〔勝ち得た選手権〕
\\	2チームが優勝を争っている 
\\	シングルスに優勝した 
\\	2秒の差で優勝を逸(そら)した 
\\	今年は彼らが優勝すると思う 
\\	は野球で,
\\	は一般にリーグ戦で) 優勝カップ[杯] 
\\	((英)) 
\\	優勝カップを獲得(かくとく)する 
\\	優勝旗 
\\	優勝決定戦 
\\	優勝者 
\\	〔優勝チーム〕
\\	優勝劣敗 〔生存競争で〕
\\	優勝する 
\\	友情	ゆうじょう	
\\	彼との友情を厚(あつ)くした 
\\	彼には厚い友情を持っている 
\\	彼女は友情に厚い[のない]人だ 
\\	友人	ゆうじん	
\\	有能	ゆうのう	
\\	郵便	ゆうびん	
\\	有利	ゆうり	
\\	〔得なこと〕 有利な 
\\	有利な取引 
\\	有利な事業 
\\	〔都合がいいこと〕 有利な 
\\	((英)) 
\\	その申し出は有利な条件であった 
\\	そのアパートの有利な点は駅に近いということである 
\\	対立候補は資金の面で彼より有利な立場にあった 
\\	情況は我々に有利に展開した 
\\	陪審(ばいしん)の評決は被告に有利であった 
\\	床	ゆか	
\\	愉快	ゆかい	
\\	愉快な 
\\	とても愉快なひとときでした〔辞去する時に言う言葉〕 
\\	こうして時々お会いできるのは大変愉快なことです 
\\	あの人は愉快な人ですね 
\\	彼は愉快な話し相手です 
\\	彼と愉快に語り合った 
\\	大いに愉快にやりましょう 
\\	愉快げ[愉快そう]に笑う 
\\	彼は友人にあだ名を付けて愉快がっている 
\\	愉快犯 
\\	譲る	ゆずる	
\\	〔譲渡する〕
\\	会社の管理(かんり)を息子に譲った 
\\	財産(ざいさん)を孫に譲った 
\\	秘伝(ひでん)は父から子へと代々譲られた 
\\	秘伝 
\\	王は位(くらい)を甥(おい)に譲った 
\\	〔希望者に売る〕
\\	〔分けてやる〕
\\	希望者 
\\	彼はカメラを譲って欲しいと言った 
\\	1冊譲ってくださいませんか 
\\	〔人を先にする〕 老婦人(ろうふじん)に席を譲りなさい 
\\	順番(じゅんばん)を譲ってくれませんか 
\\	どの車も救急車に道を譲った 
\\	〔譲歩(じょうほ)する〕
\\	〔屈(くっ)する〕
\\	彼はその議論で一点以外は譲ろうとはしなかった 
\\	彼は一歩も譲ろうとはしない 
\\	彼らのしつこい要求に負けてとうとう譲ってしまった 
\\	双方(そうほう)自分の意見を固守(こしゅ)して譲らなかった 
\\	彼は計算にかけては何人にも譲らない 
\\	〔保留する〕
\\	〔先へのばす〕
\\	この続きはまたの機会に譲ろう 
\\	その件は他日(たじつ)に譲りましょう 
\\	豊か	ゆたか	
\\	ゆでる	茹でる	
\\	卵をゆでる 
\\	許す	ゆるす	
\\	(容赦する)
\\	(許可する)
\\	(心などを)
\\	(気を)
\\	〔罪などを免じる〕
\\	3語とも大体同じだが,
\\	は個人的な感情の上で心からその恨みを無くすこと
\\	宗教的な意味合いを含むこともある
\\	は立場が上の者が寛容に法・道徳上の罪を許すこと
\\	は特に小さな過失を許すときに用いる) 彼の罪は許された 
\\	彼は彼女の過失を許した 
\\	ご無礼をお許しください 
\\	彼は借金の返済を許してもらえた 
\\	[((文))
\\	〔許可する〕
\\	教室では喫煙は許されていない 
\\	帰宅を許された 
\\	母は私が外泊するのを許してくれなかった 
\\	両親はついに娘の結婚を許した 
\\	法律の許す範囲で 
\\	許可なしに入室は許されない 
\\	〔自由がきく〕 天候が許せば 
\\	時間[事情]の許す限り 
\\	紙面の許す限り 
\\	そんな高価な時計は僕の予算が許さない 
\\	〔受け入れる〕 50名がその大学に入学を許された 
\\	一切の弁解は許されない 
\\	神は私の願いを許し給うた 
\\	〔相手の自由に任せる〕 彼女はその男に肌を許した 
\\	〔気を緩める〕 私には心を許せる友が二人いる 
\\	彼に心を許した 
\\	他人に気を許すな 
\\	気を許しているすきに財布をとられた 
\\	〔相手に取らせる〕 相手に先取点を許した 
\\	〔認める〕
\\	彼は自他ともに許す我が国最高の彫刻家である 
\\	例文 上司は私の失敗を許してくれた 
\\	うそをつくなんて許せないよ 
\\	風邪をひいているので、今日の練習は許してください 
\\	父は大学進学を許してくれた 
\\	両親は私たちの結婚を許してくれない 
\\	事情が許せばすぐフランスへ出発します 
\\	君には心を許せる友人がいますか 
\\	君には心を許せる友人がいますか 
\\	この交渉は最後まで気を許せない 
\\	事態は予断を許さない 
\\	父の病状は楽観を許さない 
\\	彼は自他ともに許す経済界の第一人者だ 
\\	「課長に絞られたんだって?」「おまけに始末書まで書かされて、やっと許してもらったよ」 
\\	「いい加減に仲直りしたら?」「彼の嘘が許せなくて、謝られても当分会いたくないの」 
\\	「進路のことでご家族と話し合ったかね?」「はい、父が大学院への進学を許してくれました」 
\\	社との契約も何とかなりそうじゃないか」「いや、K社もプロポーザルを出していますから、最後まで気を許せないよ」 
\\	岳の噴火、週末には収まりそうだから、社員旅行は決行だな」「いや、まだ予断を許さない状況だから、延期した方がいいよ」 
\\	「お父さんの容体はどうなの?」「それが、楽観を許さない状態がもう1週間も続いているんです」 
\\	「フランス留学を延期するんだって?」「家庭の事情さえ許せば、すぐにでも出発したいのですが」 
\\	あいにく、日本政府とのスポンサー契約がこの研究にも適用されるため、それは許されていないのです 
\\	当社は、どうしても御社を許すことができません。法的手段に訴えさせていただきます 
\\	夜	よ	
\\	夜明け	よあけ	
\\	酔う	よう	
\\	容易	ようい	
\\	容器	ようき	
\\	陽気	ようき	
\\	〔朗らかなこと〕 陽気な 
\\	陽気な音楽 
\\	あの人は陽気だ 
\\	子供たちは陽気に遊んでいた 
\\	陽気な踊り 
\\	〔時候,天候〕
\\	今日は陽気がいいので散歩に行こう 
\\	陽気のせいか眠くてたまらない 
\\	要求	ようきゅう	〔権利としての,あるいは強い〕
\\	〔当然の権利としての〕
\\	〔要請〕
\\	〔必要〕
\\	要求する 
\\	労働者の賃金(ちんぎん)引き上げの要求は拒否された[受け入れられた] 
\\	その家は自分のだという要求は無効だった 
\\	あなたの要求に応じてここへ来た 
\\	差し当たりの要求を満(み)たしてくれる家であれば我慢しましょう 
\\	差し当たり 
\\	そんな考え方は時代の要求にそぐわない 
\\	それは無理な要求だ 
\\	社長は退陣(たいじん)せよと要求した 
\\	法の保護(ほご)を要求する 
\\	その土地が本当に君のものならなぜその権利を要求しないのだ 
\\	社会は人材(じんざい)を要求している 
\\	用心	ようじん	〔注意〕
\\	〔警戒〕
\\	用心する 
\\	この家は用心が良い 
\\	用心に傘(かさ)を持って来ました 
\\	火災だけは起こさないようにと用心に用心を重ねていた 
\\	病気にならないよう用心しなさい 
\\	落石(らくせき)にご用心 
\\	あの男には用心しなさいよ 
\\	あまり強引(ごういん)にならないように用心した 
\\	強引 
\\	すりにご用心 
\\	足下(あしもと)にご用心 
\\	火の用心 
\\	用心深い 〔注意している〕
\\	〔警戒している〕
\\	〔慎重な〕
\\	用心深く 
\\	彼はたいへん用心深い答え方をした 
\\	用心棒 
\\	〔やくざの〕
\\	〔酒場・ホテルなどの〕
\\	様子	ようす	
\\	〔ありさま,状態〕
\\	彼の家庭の様子は今どんな具合ですか 
\\	この様子では彼は二,三日持ちそうだ 
\\	彼の様子ではとても旅行は無理だ 
\\	もう少し様子を見よう 
\\	男はあたりの様子をうかがってから家の中に入った 
\\	〔外観〕
\\	〔態度〕
\\	みすぼらしい様子の男 
\\	彼は終始浮かぬ様子でした 
\\	たまげた様子で彼女は僕の顔を見つめた 
\\	母は怒っている様子だった 
\\	彼女はまるで狂った様子で何事かをわめいていた 
\\	彼は朝から様子がおかしい 
\\	〔気配〕
\\	彼女には私に感謝している様子は全然見えなかった 
\\	だれかが起きている様子だった 
\\	雪になりそうな様子だ 
\\	ベッドに人が寝た様子はなかった 
\\	〔特別な事情〕 彼女は私の顔を様子ありげに見た 
\\	要するに	ようするに	
\\	要するに彼は日和見主義者(ひよりみしゅぎしゃ)だ 
\\	要するに彼女は彼に参(まい)ってきたのだ 
\\	要するに彼は学長の器(うつわ)ではなかったのだ 
\\	要素	ようそ	
\\	〔成分〕
\\	〔構成(こうせい)要素〕
\\	〔要因〕
\\	不注意が失敗の主な要素の一つだった 
\\	健康は美しさの重要な要素の一つである 
\\	要点	ようてん	
\\	曜日	ようび	
\\	予期	よき	
\\	予期に反して 
\\	特別寒い冬を予期して厚いコートを買った 
\\	彼女が来てくれるとは予期していなかった 
\\	津波は予期したよりも早く打ち寄せた 
\\	予期していたとおり彼は留守だった 
\\	父母(ふぼ)の死は彼の人生に予期しなかった変化をもたらした 
\\	横切る	よこぎる	
\\	踏切(ふみきり)[道路]を横切る 
\\	川を横切る橋 
\\	野山を横切る 
\\	並んでいる人の列を横切らないと向こう側に行けない 
\\	汚す	よごす	
\\	予算	よさん	
\\	〔国家などの〕
\\	暫定(ざんてい)[総(そう)]予算 
\\	国の予算編成(へんせい) 
\\	予算の均衡(きんこう)を計る 
\\	予算を編成する 
\\	予算外の支出 
\\	予算の復活(ふっかつ)折衝(せっしょう) 
\\	社会福祉の予算が大幅に削減(さくげん)された 
\\	〔予定経費〕
\\	〔見積もり〕
\\	パーティーに100万円の予算を立てた 
\\	外国旅行をする予算はない 
\\	その計画は予算の範囲内(はんいない)では無理だ 
\\	限られた予算で家の切り盛りをしなければならなかった 
\\	実際の支出は予算をはるかに上回った 
\\	お宅の建築の予算はどれくらいですか 
\\	予算案 
\\	予算案を審議する 
\\	予算委員 
\\	予算委員会 
\\	予算調整 
\\	予算復活折衝 
\\	寄せる	よせる	
\\	(動かす)
\\	(脇に)
\\	(近付ける)
\\	(送る)
\\	(贈る)
\\	(寄稿する)
\\	(加える)
\\	(集める)
\\	〔近付ける〕
\\	〔ある方向に動かす〕
\\	〔引き寄せる〕
\\	あんな乱暴者を寄せるな 
\\	机を隅(すみ)に寄せなさい 
\\	おもちゃを脇へ寄せ彼の座る場所を作った 
\\	ベッドのそばにいすを寄せた 
\\	車をポーチに寄せた 
\\	彼らは額を寄せて話し合っていた 
\\	〔集める〕 枯(か)れ葉(は)を寄せて火を付けた 
\\	額にしわを寄せる 
\\	しわ 
\\	〔加える〕
\\	2と5を寄せれば7 
\\	〔送(おく)る,贈(おく)る〕
\\	多くの励(はげ)ましの手紙が寄せられた 
\\	投稿(とうこう)をお寄せください 
\\	母に寄せる歌 
\\	春によせて〔題名〕 
\\	〔波などが〕
\\	〔大波が〕
\\	〔敵などが近付く〕
\\	敵が寄せてきた 
\\	寄せ来る敵 
\\	〔頼(たよ)る〕
\\	彼は伯父(おじ, 
\\	叔父)の家に身を寄せた 
\\	身を寄せる所がない 
\\	〔訪問する〕
\\	また寄せていただきます 
\\	〔かこつ(託)ける〕 
\\	他人のことに寄せて皮肉を言う 
\\	〔思いなどを〕 彼女は彼に心を寄せていたらしい 
\\	彼女に思いを寄せた 
\\	信頼(しんらい)を寄せていた秘書に裏切られた 
\\	彼女に多くの同情が寄せられた 
\\	予測	よそく	
\\	未来を予測できていたらこんなことはしなかったはずだ 
\\	予測を誤(あやま)った 
\\	実験は2か月かかると予測している 
\\	最終(さいしゅう)的にいくら掛かるか予測がつかない 
\\	夜中	よなか	
\\	世の中	よのなか	
\\	余分	よぶん	
\\	〔余った分〕
\\	教科書の余分が1冊あります 
\\	100万トンの米の余分が出た 
\\	〔余計〕 もし余分の金があったら,その帽子を買ったのに 
\\	少し余分に金を払う 
\\	余分な金は持ち歩かないことにしています 
\\	彼は余分な服を1着旅行に持って行った 
\\	予報	よほう	
\\	天気予報 
\\	長期予報 
\\	今夜の予報は雨である 
\\	予報は当たった[外れた] 
\\	天気は今夜から悪くなるという予報だ 
\\	予防	よぼう	(防止)
\\	(保護)
\\	(病気の)
\\	(用心)
\\	予防する 
\\	火災予防 
\\	この病気は予防できる 
\\	子供が風邪をひかないように予防する 
\\	ばらのうどん粉病(こびょう)の予防にはこの散布剤(さんぷざい)がよい 
\\	予防医学 
\\	予防策 
\\	通例複数で用いる) 予防接種(せっしゅ) 
\\	予防接種証明書 
\\	予防線 
\\	叱(しか)られないように,できるだけ予防線を張っておいた 
\\	予防注射 ⇒よぼうせっしゅ(予防接種) 昨日,腸(ちょう)チフスの予防注射をしてもらった 
\\	予防点検 
\\	予防保全 
\\	虫歯の予防には歯磨きがいちばんです 
\\	今、村では水害を予防するための工事を進めている 
\\	うちの子はインフルエンザ〔はしか〕の予防注射を受けてあります 
\\	彼は予想される非難に対して予防線を張った 
\\	抗真菌薬(こうしんきんやく)はまた、移植(いしょく)患者の(免疫反応(めんえきはんのう)の)予防に広く使われている 
\\	免疫反応 
\\	読み	よみ	
\\	嫁	よめ	
\\	〔花嫁〕
\\	〔妻〕
\\	〔息子の妻〕
\\	((複
\\	嫁をいびる 
\\	彼は嫁をもらったそうだ 
\\	去年(きょねん)長女を嫁にやりました 
\\	彼女は商家に嫁に行った 
\\	嫁姑(よめしゅうとめ) 
\\	余裕	よゆう	
\\	〔空間のゆとり〕 車にうちの子供を乗せてもらう余裕がありますか 
\\	〔時間のゆとり〕 時間の余裕がない 
\\	コーヒーを飲んでいる余裕はない 
\\	約束には少し時間の余裕をみて出掛けた 
\\	日取りを決める時はちょっと余裕をみておいた方がよいだろう 
\\	〔金銭のゆとり〕 彼らには金はあり余(あま)るほど余裕があった 
\\	娘にピアノを習わせてやる余裕はなかった 
\\	〔気持ちのゆとり〕 他の人たちのことを考える余裕はなかった 
\\	心の余裕 
\\	〔余地(よち)〕
\\	後で工作(こうさく)する余裕があるようにあまり細かく決めないことにした 
\\	[((口))
\\	工作 
\\	余裕綽綽(しゃくしゃく) 彼は入学試験の前日でも余裕綽々(しゃくしゃく)としていた 
\\	より
\\	縒り】	より	
\\	〔糸の〕
\\	糸に強いよりをかける 
\\	糸のよりを戻(もど)す 
\\	3本よりの綱(つな) 
\\	縒りをかける 腕によりをかけた料理 
\\	ここの家具は一流の職人(しょくにん)が腕によりをかけて作ったものである 
\\	縒りを戻す 
\\	二人はよりを戻した 
\\	因る	よる	【因る・▲由る・▲依る】 
\\	よる【寄る】 
\\	よる【▲拠る】 
\\	〔したがう,応じる〕
\\	重さによって値段が違う 
\\	彼の出方(でかた)によっては実力行使も辞(じ)さない 
\\	それは場合による 
\\	場合によってはダムが決壊(けっかい)するかもしれない 
\\	人によってはアレルギーを起こす 
\\	風習(ふうしゅう)は国によって異なる 
\\	〔基(もと)づく〕
\\	実験結果による結論(けつろん) 
\\	労働による所得(しょとく) 
\\	飲酒運転は法律(ほうりつ)によって禁じられている 
\\	彼は規則により免職(めんしょく)になった 
\\	その番組は視聴者(しちょうしゃ)の要望により再放送(さいほうそう)された 
\\	その申し立ては何によっているのか 
\\	新聞によると天然痘(てんねんとう)は撲滅(ぼくめつ)されたそうだ 
\\	聞くところによると彼は辞職(じしょく)するとのことだ 
\\	〔起因(きいん)する〕
\\	天候(てんこう)不良により飛行機は飛べなかった 
\\	その事故は彼の不注意運転によるものだった 
\\	漏電(ろうでん)による火災(かさい)に注意 
\\	癌(がん)による死亡率が増加している 
\\	彼女の助力(じょりょく)により成功した 
\\	濃霧(のうむ)による欠航(けっこう)が続いた 
\\	〔手段とする〕
\\	音によって感情を表現する 
\\	多数決によって事を決する 
\\	武力によらざるをえなかった 
\\	情報はすべてコンピュータにより処理(しょり)される 
\\	詩作によって食べてゆくことは難しい 
\\	〔行為者(こういしゃ)を表して〕
\\	この詩はミルトンによって書かれたものだ 
\\	ブラームスによる交響曲(こうきょうきょく) 
\\	この絵は18世紀のフランスの画家によるものとされている 
\\	喜び	よろこび	
\\	よろしく	【▲宜しく】	
\\	(初めまして)
\\	(ぜひとも…すべきである)
\\	〔適当に,具合よく〕 息子は課長に気に入られて職場でよろしくやっている 
\\	今ごろあいつは彼女とよろしくやっているんだろう 
\\	この件はよろしく処理してください 
\\	〔伝言として〕 お父様によろしくお伝えください 
\\	父からよろしくとのことです 
\\	おばあちゃんによろしく〔親戚(しんせき)などの間で〕 
\\	トムによろしく〔親しい間で〕 
\\	〔あいさつ・依頼の言葉で〕 初めまして
\\	どうぞよろしく 
\\	「どうぞよろしく」に当たることは言わない) 娘をどうぞよろしく 
\\	では,その件はよろしく 
\\	ひとつここはよろしくお願いしますよ 
\\	英語圏ではあまりこういうことを言わない) 今度の選挙では中野氏をよろしくお願いします 
\\	〔…のように〕 彼は歌舞伎(かぶき)役者よろしく見得を切った 
\\	四	よん	
\\	来	らい~	
\\	楽	らく	
\\	〔心身の安らかさ〕 楽な 
\\	楽な姿勢(しせい)で座る 
\\	このいすは楽だ 
\\	どうぞお楽になさってください 
\\	はい,楽にして〔患者に対して〕 
\\	少し楽になりましたか 
\\	死ねば楽になる 
\\	大役(おおやく)を果たして気が楽になった 
\\	楽あれば苦(く)あり ((諺)) 
\\	〔裕福なこと〕 年金で楽に暮らしている 
\\	生活は楽ではない 
\\	〔たやすいこと〕 楽な 
\\	楽に 
\\	楽な仕事 
\\	思ったほど楽ではない 
\\	この問題は楽に解ける 
\\	彼なら楽にその役がつとまるだろう 
\\	彼はフランス語が楽に話せる 
\\	利益	りえき	
\\	〔もうけ; 得(とく)〕
\\	隠(かく)し利益 
\\	取引で[株を売って]利益を得る 
\\	利益がある 
\\	利益のない 
\\	それを売れば2,000ドルの利益になる 
\\	それでは大した[全然]利益にならない 
\\	社会に利益を還元する 
\\	〔役に立つこと〕
\\	…の利益のため 
\\	社会の利益のために尽くす 
\\	そんな本を読んだってなんの利益にもならないよ 
\\	それは私の利益になる 
\\	彼は自分の利益ばかり考えている 
\\	利益金 
\\	利益社会 
\\	((ドイツ語))] 利益代表部 
\\	利益配当 
\\	利益配当金 〔株・保険の〕
\\	利益誘導型政治 
\\	((米)) 
\\	利益率 
\\	理解	りかい	
\\	陸	りく	
\\	(岸)
\\	陸にすむ動物 
\\	陸が見えてきた 
\\	2か月ぶりで陸に上がった 
\\	利口	りこう	
\\	〔頭がよいこと〕 利口な 〔聡明な〕
\\	〔賢い〕
\\	利口な少年 
\\	利口な犬 
\\	いかにも利口そうだ 
\\	先生に相談したのは利口だった 
\\	〔子供が聞き分けがよいこと〕 聞き分け 
\\	お利口さんね 
\\	お利口にしているんですよ 
\\	〔要領(ようりょう)がいい〕 
\\	利口な 
\\	あの人は利口だから自分が傷付(きずつ)くようなことはしない 
\\	利口者 
\\	離婚	りこん	
\\	理想	りそう	
\\	理想の[的] 
\\	高い理想を抱く 
\\	[((文)) 
\\	理想にかなう 
\\	理想が実現した 
\\	この道は散歩するのに理想的だ 
\\	理想化 
\\	理想化する 
\\	理想家 
\\	理想型 
\\	理想郷(きょう) 
\\	理想主義 
\\	理想主義的 
\\	理想像 私の考える女性の理想像に近い 
\\	理想論 
\\	率	りつ	
\\	〔割合〕
\\	は100に対する率,
\\	は割合) 出生率 
\\	死亡率 
\\	高い[低い]率 
\\	平均率 
\\	今年の卒業生の就職率はどのくらいですか 
\\	〔労力などに対する報酬の程度〕 率のよい仕事 
\\	は一時的な仕事や取引に用いる) 率の悪い仕事 
\\	留学	りゅうがく	
\\	流行	りゅうこう	
\\	〔服装などの〕
\\	〔一時の熱狂的な〕
\\	〔人気〕
\\	流行を追う 
\\	流行する[しなくなる] 
\\	フォークの流行はすたれた 
\\	最新流行の服装をしている 
\\	今は長いスカートが流行だ 
\\	テレビゲームが子供たちの間で大流行だ 
\\	流行遅れの帽子 
\\	一時学生の間にマルキシズムが流行していた 
\\	無感動が今の流行のようだ 
\\	〔病気の〕 その地域ではコレラがまだ流行していますか 
\\	流行歌 
\\	流行歌手 
\\	流行語 
\\	「国際化」が80年代の流行語であった 
\\	流行作家 
\\	流行児 (りゅうこうじ) 
\\	彼は当時の画壇の流行児だった 
\\	流行性感冒(かんぼう) 
\\	流行性肝炎(かんえん) 
\\	流行性結膜炎(けつまくえん) 
\\	流行性耳下腺炎(じかせんえん) 
\\	流行性脳炎 
\\	流行病 
\\	量	りょう	
\\	寮	りょう	
\\	両替	りょうがえ	
\\	両替する 
\\	1万円を千円札10枚に両替する 
\\	空港でドルをポンドに両替した 
\\	100ドル札を両替してくださいませんか 
\\	両替機 
\\	千円札両替機 
\\	両替所 
\\	両替屋 〔人〕
\\	〔店〕
\\	両替料 
\\	料金	りょうきん	
\\	例	れい	
\\	〔先例〕
\\	そういう例はこれまでない 
\\	彼らはこれまでの例を破(やぶ)って平服で式に出席した 
\\	例のない失策 
\\	例にならう 
\\	〔慣例(かんれい)〕
\\	朝食(ちょうしょく)に雑煮(ぞうに)を食べるのが新年の例だ 
\\	〔いつもと同じこと〕 彼女は例の話を繰り返した 
\\	例の通り6時に起きた 
\\	例によって例のごとき来賓(らいひん)のあいさつが続いた 
\\	例によって彼は会合に遅刻(ちこく)した 
\\	それもこの例に漏(も)れない 
\\	漏る 
\\	その朝,例になくよく食べた 
\\	〔代表的な実例〕
\\	〔一つの実例〕
\\	〔事実などを表す例〕
\\	彼の例にならう 
\\	彼の正直(しょうじき)さの例を2,3挙げよう 
\\	若くして世を去った芸術家は多い
\\	一例をあげればモーツアルトは35歳で他界した 
\\	フランス人を例にとってみよう 
\\	〔分かっているものを指して〕 例の女 
\\	礼	れい	
\\	例外	れいがい	
\\	礼儀	れいぎ	
\\	冷静	れいせい	冷静な (に) 
\\	冷静になる 
\\	間もなく彼は冷静さを取り戻した 
\\	彼女は努(つと)めて冷静を装(よそお)った 
\\	彼は常(つね)に冷静だ 
\\	/((口)) 
\\	危機に当たって冷静を保つのは重要だ 
\\	あの人はすぐに冷静さを失う 
\\	冷静な態度で 
\\	冷静な判断力 
\\	列	れつ	
\\	列車	れっしゃ	
\\	連想	れんそう	
\\	通俗的(つうぞくてき)な連想を伴(ともな)わない純粋(じゅんすい)に学術的な用語 
\\	4月というと桜を連想する 
\\	これを見て何を連想しますか 
\\	連続	れんぞく	
\\	老人	ろうじん	
\\	労働	ろうどう	
\\	論じる	ろんじる	
\\	論争	ろんそう	
\\	原子力潜水艦(せんすいかん)に関する論争 
\\	それは論争されている説だ 
\\	彼らの間には論争が絶(た)えない 
\\	彼は論争が好きだ 
\\	その件は論争中だ 
\\	ある問題について人と論争する 
\\	[((文))
\\	論争者 
\\	論争点 
\\	論文	ろんぶん	
\\	〔一般に〕
\\	文学的な論文を指すことが多い); 〔学術的な〕
\\	((複-
\\	〔学位論文・博士論文〕
\\	〔卒論・修論〕
\\	〔新聞・雑誌の論説〕
\\	遺伝子について論文を書く 
\\	大学に学位論文を提出する 
\\	輪	わ	
\\	わがまま	わがまま	
\\	〔身勝手〕
\\	((英)) 
\\	〔強情〕
\\	((英)) 
\\	わがままな 〔利己的な〕
\\	〔強情な〕
\\	((英)) 
\\	わがままに育てられた子供 
\\	親の言うことを聞かないわがままな少年 
\\	わがままを通す 
\\	わがままを言うんじゃない 
\\	別れ	わかれ	
\\	〔離別〕
\\	〔いとまごい〕
\\	別れの悲しみ 
\\	別れの杯(はい、さかずき)をくみかわす 
\\	くみかわす 
\\	別れにのぞんで一言も物が言えなかった 
\\	あれが父との最後の別れとなった 
\\	友人に別れを告(つ)げた 
\\	[((文))
\\	先生と別れの挨拶をかわした 
\\	彼は手を振って皆に別れを告げた 
\\	お別れパーティーを開く 
\\	/((口)) 
\\	分かれる	わかれる	
\\	〔分岐(ぶんき)する〕
\\	〔またになる〕
\\	公道から分かれた狭(せま)い道 
\\	道はそこで二またに分かれている 
\\	中央高速は大月で甲府方面と河口湖方面に分かれる 
\\	〔別々になる〕
\\	〔分裂する〕
\\	警官は二手(ふたて)に分かれて犯人を追った 
\\	3台の車に分かれて乗った 
\\	党が二つに分かれた 
\\	この一派(いっぱ)は主流(しゅりゅう)から分かれたものだ 
\\	評価が分かれた 
\\	〔区分される〕
\\	この本は10章に分かれている 
\\	商品(しょうひん)は色別(しょくべつ)に分かれている 
\\	本は分野別に分かれている 
\\	その社会はいくつかの階層に分かれていた 
\\	〔解散する〕
\\	デモ隊は目的地で分かれた 
\\	脇	わき	
\\	〔腕の付け根の下〕
\\	⇒わきのした(脇の下) 右脇が痛い 
\\	〔すぐそば〕 
\\	彼は妻の脇に立っていた 
\\	〔横の邪魔にならない所〕 彼らはそれを脇へ持って行った 
\\	〔別の方向〕 脇を向く 
\\	〔能楽の〕⇒ワキ 脇が甘(あま)い あの力士(りきし)は脇が甘い 
\\	〔腕の付け根の下〕
\\	⇒わきのした(脇の下) 右脇が痛い 
\\	本を数冊脇に抱(かか)えていた 
\\	ブラウスの脇がほころびた 
\\	〔すぐそば〕 
\\	彼は妻の脇に立っていた 
\\	彼はすぐ脇にやって来た 
\\	門の脇を通って行った 
\\	彼は脇から口を出した 
\\	〔横の邪魔にならない所〕 彼らはそれを脇へ持って行った 
\\	脇へどきなさい 
\\	その話は今は脇へ置いておこう 
\\	〔別の方向〕 脇を向く 
\\	話が脇にそれてしまった 
\\	[((英))
\\	逸れる 
\\	それる 
\\	〔能楽の〕⇒ワキ 脇が甘(あま)い あの力士(りきし)は脇が甘い 
\\	脇が甘い 
\\	湧く	わく	
\\	〔わき出る〕
\\	彼の畑から温泉がわいた 
\\	岩の間から清水(しみず)がわいている 
\\	〔発生する〕 水たまりに蚊(か)がわく 
\\	〔急に起こる〕 西の空に雲が湧いてきた 
\\	〔心の中に起こる〕 あなたの話を聞いて勇気[希望]がわいた 
\\	恐怖心が湧いた 
\\	見る物すべてに興味がわいた 
\\	それには全く好奇心(こうきしん)がわかなかった 
\\	分ける	わける	
\\	〔分割する〕
\\	コーチは選手を二つのチームに分けた 
\\	髪を真ん中から[左側で]分ける 
\\	ピアノの代金は3回に分けて支払った 
\\	その話は5回に分けて掲載(けいさい)された 
\\	〔分離する〕
\\	穀物(こくもつ)をもみ殻(がら)と分けた 
\\	彼がやって来てけんかを分けた 
\\	その事件は二人のその後の人生を分けた 
\\	〔分類する〕
\\	〔区別する〕
\\	彼らはいくつかのタイプに分けられる 
\\	これらのビデオはジャンル別に分けてある 
\\	洋書と和書を分ける 
\\	過失(かしつ)と犯罪は分けて考えねばならない 
\\	〔分配する〕
\\	りんごを妹(いもうと)に分けてあげた 
\\	彼は財産を3人[2人]の息子に分けた 
\\	卵を1個分けてくださいませんか 
\\	〔売る〕
\\	ハガキを5枚分けてください 
\\	〔邪魔な物を左右にどける〕 彼は人込みを分けて進んだ 
\\	〔理非を区別する〕 事を分けて話せば分かる 
\\	わざと	わざと	
\\	〔故意に〕
\\	それはわざとしたのだ 
\\	わざとうそをついたのです 
\\	わざとらしい 
\\	彼女はわざとらしい微笑(びじょう)を浮かべた 
\\	彼はわざとらしいお辞儀(じぎ)をした 〔気取って〕
\\	/〔無理に〕
\\	彼はわざとらしく振る舞った 〔不自然に〕
\\	/〔芝居がかって〕
\\	わずか	わずか	僅か 
\\	〔数が少ない〕
\\	〔量が少ない〕
\\	部屋にはわずかな本しかなかった 
\\	食料[弾薬(だんやく)]はわずかしか残っていない 
\\	わずかな金でこの家を買った 
\\	彼はほんのわずかの常識も持ち合わさなかった 
\\	〔度合が少ない〕 わずかな痛み 
\\	まだわずかな可能性が残されている 
\\	彼のわずかな楽しみを奪(うば)うものではない 
\\	わずかな差で2位だった 
\\	わずかなことで彼は機嫌(きげん)を損(そこ)ねた 
\\	英語をわずかばかり話します 
\\	〔たった〕 
\\	それをわずか5,000円で買った 
\\	わずか1年で昇進(しょうしん)した 
\\	〔「わずかに」の形で,かろうじて〕 彼の家だけがわずかに焼け残った 
\\	彼のことをわずかに覚えている 
\\	わずかに武蔵野の名残(なごり)をとどめていた 
\\	綿	わた 
\\	めん	わた 〔綿花〕
\\	〔外科用・布団用の〕
\\	[((英)) 
\\	〔原綿〕
\\	綿の木 
\\	綿の実 
\\	掛け布団に綿を入れる 
\\	夕方までには綿のように疲れてしまった 
\\	[((口)) 
\\	綿油 
\\	綿入れ ⇒詳細を見る 綿打ち 
\\	綿打ちする 
\\	綿打ち機 
\\	綿雲 ⇒詳細を見る 綿繰り機 
\\	綿摘み 
\\	ーーー めん 
\\	綿100パーセント 
\\	綿製品 
\\	これは綿製品です 
\\	話題	わだい	
\\	笑い	わらい	
\\	〔笑うこと〕
\\	〔くすくす笑い〕
\\	独り楽しんで) 
\\	主に女の子が); 〔微笑〕
\\	私たちは大笑いをした 
\\	そんなばか笑いをするな 
\\	彼は卑(いや)しい笑いを浮かべていた 
\\	笑いを抑える 
\\	彼のひょうきんなしぐさが観衆(かんしゅう)の笑いを誘(さそ)った 
\\	彼は笑いに紛らわせてちくりちくりとやった 
\\	〔あざけり〕 彼は私を笑いの種にした 
\\	皆の笑いを買った 
\\	笑いが止まらない すべてがうまくいって笑いが止まらなかった 
\\	割る	わる	
\\	(粉々に)
\\	(砕く)
\\	(裂く)
\\	(分割する)
\\	【数学】(割り算する)
\\	(薄める)
\\	(水で薄める)
\\	(ある数量以下になる)
\\	〔壊す〕
\\	ボールをぶつけて窓ガラスを割ってしまった 
\\	卵を割る 
\\	腹立ちまぎれに鏡を粉々(こなごな)に割った 
\\	〔たたき切る〕
\\	〔裂く〕
\\	薪(まき)を割る 
\\	板を二つに割る 
\\	竹を細く割ってかごを作った 
\\	〔裂傷(れっしょう、
\\	を付ける〕 壁にぶつかって額を割った 
\\	〔押し分ける〕
\\	〔引き離す〕
\\	人垣(ひとがき)を割って中へ入った 
\\	二人の間に割って入った 
\\	〔分割(ぶんかつ)する〕
\\	メロンを6人で[6つに]割る 
\\	金を頭数(とうすう)で割って分けた 
\\	〔割り算をする〕
\\	50を6で割ると8がたって2余る 
\\	〔関係を裂く〕
\\	〔分裂させる〕
\\	二人の仲を割る 
\\	彼は党を2派に割ろうとたくらんでいる 
\\	〔介入する〕
\\	ずうずうしくも彼らの話に割って入った 
\\	〔ある数量以下になる〕
\\	株価(かぶか)が200円を割った 
\\	1ドルが110円台を割った 
\\	社員が100人を割った 
\\	与党が過半数(かはんすう)を割った 
\\	定員を割る大学が出てきた 
\\	〔液体を薄める〕
\\	ウイスキーを水で割る 
\\	〔ある線を越える〕
\\	土俵(どひょう)を割る 
\\	土俵 
\\	最後のスクラムでゴールラインを割った 
\\	例文 「このクルミ、割れないよ」「不器用だなあ。貸してごらん」 
\\	「大輔のやつ、何しょげてるの?」「野球してて、ご近所の窓ガラスを割って叱られたのよ」 
\\	「それ、大切なお皿だから、割ったりしないように気をつけてね」「珍しい色合いなのね。どこで買ったの?」 
\\	「あのおばさん、図々しかったな」「ああ。みんなが並んでるのに、平然と割り込んでくるんだもんな」 
\\	「今年の売上(うりあげ)は10億を割りそうだな」「ボーナスに響かないことを祈るばかりだよ」 
\\	「この間の都知事(とちじ)選は低調(ていちょう)だったな」「投票率も、50パーセントを割ったらしいですよ」 
\\	「ウィスキー、水で割る?」「いや、僕はロックにするよ」 
\\	「俺、暗算は得意なんだ」「じゃあ、335割る27は?」 
\\	「お勘定、2万5千円だってさ」「とすると、頭数で割るといくらになる?」 
\\	悪口	わるくち	
\\	我々	われわれ	
\\	湾	わん	
\\	より大きく奥行きが広い); 〔入り江〕
\\	東京湾 
\\	冠詞を付けない)/
\\	メキシコ湾 
\\	椀	わん	
\\	椀にみそ汁を入れて出す 
\\	碗	わん	
\\	わん
\\	椀・×碗】 
\\	椀にみそ汁を入れて出す 
\\	味噌汁 
\\	みそしる
\\	悪	あく	
\\	当り	あたり	
\\	〔命中(めいちゅう) 
\\	射撃(しゃげき)で当たり3発,外れ2発だった 
\\	〔光・風などが当たること〕 この辺は風の当たりが強い 
\\	〔打球(だきゅう)の勢(いきお)い〕 第1球を見事な当たりで打った 
\\	〔計画の成功〕
\\	その番組は大当たりだった 
\\	〔当選,的中(てきちゅう) 
\\	くじを開けたら「当たり」と出た 
\\	〔手掛かり,見当〕 当たりをつける 
\\	〔人当たり〕 
\\	当たりのいい人 
\\	〔…につき〕 1人当たり 
\\	〔体に害になること〕 むかむかする
\\	たぶん食あたりだろう 
\\	食あたり 
\\	〔命中(めいちゅう) 
\\	当たり 
\\	射撃(しゃげき)で当たり3発,外れ2発だった 
\\	〔光・風などが当たること〕 この辺は風の当たりが強い 
\\	部屋はだんだん日の当たりがよくなる 
\\	〔打球(だきゅう)の勢(いきお)い〕 猛烈(もうれつ)な当たりのライナーだった 
\\	第1球を見事な当たりで打った 
\\	今季節(きせつ)の彼の当たりはかなりよい 
\\	〔計画の成功〕
\\	その番組は大当たりだった 
\\	その俳優(はいゆう)はこの前の芝居で当たりをとった 
\\	[((口)) 
\\	〔当選,的中(てきちゅう) 
\\	当たりだ 〔なぞ・クイズで〕
\\	/〔くじで〕
\\	くじを開けたら「当たり」と出た 
\\	〔手掛かり,見当〕 当たりをつける 
\\	彼を探すなら当たりをつけてからしなさい 
\\	その結果に当たりをつけてみた 
\\	〔人当たり〕 当たりのいい人 
\\	当たりの柔らかい態度 
\\	〔舌ざわり〕 したざわり 飲食物などの舌に触れる感じ。舌の感触。「―のやわらかい酒」 このぶどう酒は当たりが柔らかい 
\\	〔釣で,魚のくいつき〕
\\	たくさん当たりがあった 
\\	〔…につき〕 1人当たり 
\\	成人は1人,1日当たり約120グラムの米を消費する 
\\	砂糖はキロ当たりで売られる 
\\	1ページ当たりの語数は約350である 
\\	〔体に害になること〕 むかむかする
\\	たぶん食あたりだろう 
\\	暑気当たりした 
\\	宛てる	あてる	
\\	父に宛てた手紙 
\\	あて
\\	宛)
\\	異	い	
\\	〔異議(いぎ)〕 
\\	彼の提案に数名の者が異を唱(とな)えた 
\\	〔不思議なこと〕 異な 
\\	意	い	
\\	〔気持ち〕 私は遺族(いぞく)に哀悼(あいとう)の意を表した 
\\	〔考え,意志〕 意のままにする 
\\	すべて自分の意のままになると思ったら間違いである 
\\	彼女は両親の意に反してその男と結婚した 
\\	⇒いみ(意味) 読書百遍(どくしょひゃっぺん)意おのずから通ず 
\\	読書百遍 
\\	意余(あま)って言葉足(た)らず 
\\	意とするに足らない 彼らの妨害など意とするに足らない 
\\	意に介(かい)する 
\\	彼が私のことをなんと言おうと意に介(かい)さない 
\\	彼は母親の忠告(ちゅうこく)など少しも意に介さなかった 
\\	意に叶(かな)う 
\\	私の意に叶う家は一つもなかった 
\\	彼自身の意に叶った作品はほとんどない 
\\	意に染(そ)まぬ 彼女は意に染まぬ結婚をした 
\\	意に満(み)たない 
\\	最近の作品はまったく意に満(み)たない 
\\	意を得る 
\\	彼の言葉を聞いて我が意を得たりと喜んだ 
\\	意を汲(く)む 
\\	彼女の意を汲んで離婚の理由はきかなかった 
\\	意を決する 
\\	社長に昇給(しょうきゅう)を頼もうと意を決した 
\\	彼はまだ意を決しかねていた 
\\	彼は意を決して社長の決定に抗議した 
\\	意を尽くす 彼女の手紙は簡単だが意を尽くしている 
\\	意を強くする 彼の言葉を聞いて意を強くした 
\\	調査の結果は我々の意を強くするに足(た)るものだった 
\\	意を体する 現場教師の意を体して,校長は学校規則の改善に取り掛かった 
\\	意を迎(むか)える 
\\	彼は上役の意を迎えるのに汲々(きゅうきゅう)としていた 
\\	意を用いる 
\\	彼は学生たちの健康に大いに意を用いている 
\\	医院	いいん	
\\	((英)) 
\\	山田医院 
\\	医院長 
\\	〔外科医〕
\\	怒り	いかり	
\\	粋	いき	
\\	〔垢抜(あかぬ)けした様子〕 いきな 
\\	〔特に服装が〕
\\	いきな造りの家 
\\	いきな人 
\\	彼女はいきななりをしている 
\\	なり 
\\	黒い服でかえっていきに見えた 
\\	〔人情に通じている様子〕
\\	上司のいきな計(はか)らいで彼らはまる二週間新婚旅行に行くことができた 
\\	計らい 
\\	粋事(いきごと) 
\\	粋筋(いきすじ) 〔花柳界(かりゅうかい)〕
\\	意地	いじ	
\\	〔気立て〕
\\	意地の悪い ⇒いじわる(意地悪) 意地が汚(きたな)い ⇒いじきたない(意地汚い) 
\\	〔意気地〕
\\	〔誇り〕
\\	〔片意地〕
\\	彼には全く意地というものがない 
\\	男[女]の意地 
\\	意地になって反対する 
\\	意地を通す 
\\	/((口)) 
\\	いい意味に用いる) つまらないことで意地を張る 
\\	自分が始めたことだから意地でもやり遂(と)げてみせる 
\\	やり遂げる 
\\	依然	いぜん	
\\	彼は依然として若いころからの朝の散歩を欠(か)かさない 
\\	あの子は依然として病弱(びょうじゃく)だ 
\\	政局は依然不安定だ 
\\	傷める	いためる	
\\	〔物を損傷(そんしょう)する〕
\\	借りていた本をうっかりして傷めてしまった 
\\	〔食物を腐らせる〕 忘れていて魚を傷めてしまった 
\\	〔病気の状態にする〕 冷たい物をあまり沢山飲むとおなかを傷める 
\\	炒める	いためる	
\\	玉ねぎをいためた 
\\	一帯	いったい	
\\	一帯の麦畑(むぎばたけ) 
\\	警官が付近(ふきん)一帯を捜索(そうさく)した 
\\	その辺り一帯は湿地(しっち)である 
\\	昨夜丸の内一帯が停電した 
\\	異動	いどう	
\\	人事異動 
\\	今回の異動で彼は九州へ転勤(てんきん)になった 
\\	役員に大異動があった 
\\	衣料	いりょう	
\\	渦	うず	
\\	〔川や海などの渦巻き〕
\\	は
\\	よりも小さい感じで,煙・風・砂塵などにも用いる); 〔大渦巻き〕
\\	渦に巻き込まれる 
\\	潮流(ちょうりゅう)は渦を巻いていた 
\\	煙が渦を巻いて立ち昇った 
\\	群衆は競技場(きょうぎじょう)の周りに渦を巻いた 
\\	〔激しい状態〕 彼の作品は賛否(さんぴ)の渦を巻き起こした 
\\	彼らは革命の渦に巻き込まれた 
\\	埋まる	うまる 
\\	〔うずまる〕
\\	〔穴などが一杯になる〕
\\	地中深くに管が埋まっている 
\\	穴は土砂(どしゃ)で埋まっている 
\\	〔一杯になる〕 公園は人で埋まっている 
\\	生徒たちで席が埋まった 
\\	〔不足分が補われる〕 赤字はまだ埋まっていない 
\\	欠員(けついん)が埋まった 
\\	産む	うむ	
\\	〔出産する〕
\\	((文)) 
\\	〔動物が〕
\\	〔卵を〕
\\	彼女はいつ赤ちゃんを産むのだろうか 
\\	妊娠中の女性について言う時は
\\	そうでなければ
\\	彼女は男の子を産んだ 
\\	うちの猫が子供を産んだ 
\\	この鶏はもう卵を産まなくなった 
\\	豚は年に何回子を産むのか 
\\	魚は一般にたくさんの卵を産む 
\\	〔作り出す,もたらす〕 日本が生んだ最大の建築家(けんちくか) 
\\	よい結果を生む 
\\	そのニュースは色々なうわさを生んだ 
\\	彼の行動は疑惑(ぎわく)を生みやすい 
\\	この投資は9パーセントの利子を生む見込みだ 
\\	金が金を生む ((諺)) 
\\	案ずるより生むが易(やす)し 
\\	縁(えん)	えん	
\\	〔巡り合わせ〕
\\	〔因縁〕
\\	縁があったらまた会いましょう 
\\	不思議な縁で彼とは往復(おうふく)とも同じ飛行機だった 
\\	ふとした縁で共同の事業を始めた 
\\	これは前世(ぜんせ)の縁である 
\\	前世 
\\	〔人との関係,結びつき〕 夫婦の縁を結(むす)ぶ 
\\	彼は私の妹と夫婦の縁を結んだ 
\\	夫婦の縁を切る 
\\	親子の縁を切る 
\\	これも他生(たしょう)の縁と思って最後まで世話をした 
\\	私共(わたしども)には縁もゆかりもない人です 
\\	ゆかり 
\\	〔物事との関係〕 金には縁のない男だ 
\\	化学(かがく、
\\	とは縁のない科学(かがく、
\\	えんがわ【縁側】 
\\	縁側でひなたぼっこをしましょう 
\\	ぬれ縁 
\\	縁なき衆生(しゅじょう)は度し難し 〔仏縁のない人〕
\\	/〔人の言葉を聞き入れない人〕
\\	尾	お	
\\	〔動物一般の〕
\\	リス[きつね]の尾 
\\	鹿[うさぎ]の尾 
\\	は短い尾) くじゃくの尾 
\\	尾のない 
\\	犬が尾を振った 
\\	馬がひゅっと尾を振った 
\\	犬は尾を垂れて主人に付いて行った 
\\	〔彗星などの〕
\\	ほうき星の尾 
\\	ジェット機が長く尾を引いた 
\\	〔山のすそ〕
\\	尾を引く 両家の争いはまだ尾を引いている 
\\	負う	おう	意味 おう【負う】 
\\	⇒せおう(背負う) 彼女は子供を負ってやって来た 
\\	〔責任などを引き受ける〕 責任(せきにん)を負う 
\\	彼は重要な任務(にんむ)を負っていた 
\\	手に負えない 
\\	〔おかげ・恩などをこうむる〕
\\	君の援助(えんじょ)に負うところが大だ 
\\	国の財政(ざいせい)は国債(こくさい)に負うところが大きい 
\\	〔傷を受ける〕 傷を負う 
\\	手傷(てきず)を負った犬 
\\	負うた子に教えられて浅瀬(あさせ)を渡る ((諺)) 
\\	負うた子より抱いた子 
\\	遅れ	おくれ	
\\	遅(おく)れる 
\\	遅(おそ)い 夏休みに勉強の遅れを取り戻したい 
\\	郵便に二日の遅れが出ている 
\\	この時計は遅れがちだ 
\\	月遅れのお盆に帰省します 
\\	月遅れの雑誌 
\\	私の郷里では月遅れの正月を祝う[お盆をする]. 
\\	教え	おしえ	
\\	〔教えること〕
\\	教えを受ける 
\\	教えを仰(あお)ぐ 
\\	〔教訓〕
\\	〔行動の規範〕
\\	〔教義〕
\\	仏の教え 
\\	父の教えに従(したが)った 
\\	驚き	おどろき	
\\	織る	おる	
\\	機(はた)を織る 
\\	織機(しょっき) 
\\	絹糸(けんし)でブラウス布地(きれじ)を織る 
\\	この布はウールで織ってある 
\\	欠く	かく	
\\	〔損(そこ)なう〕
\\	〔一部を壊す〕
\\	皿を欠く 
\\	刃物(はもの)を欠く 
\\	/〔先端を〕
\\	転んで前歯を欠いた 
\\	氷を小さく欠く 
\\	〔あるべきものがない〕
\\	礼儀を欠く 
\\	勇気(ゆうき)を欠く 
\\	[((文))
\\	義理を欠く 
\\	あれは常識(じょうしき)を欠く行動だった 
\\	彼の仕業(しわざ)だと断定(だんてい)するには決(き)め手を欠く 
\\	仕業 
\\	全集の第3巻(まき)を欠く 
\\	欠くべからざる 
\\	成功に欠くべからざる条件 
\\	角	
\\	かく 
\\	⇒かくど(角度)
\\	〔方形〕
\\	にんじんを角に切る 
\\	10センチ角の柱 
\\	〔将棋の〕
\\	角速度 
\\	かど 
\\	〔物の隅の突き出た部分〕
\\	〔へり〕
\\	机の角 
\\	とがった岩角 
\\	〔曲り角〕
\\	角の本屋 
\\	角を曲がる 
\\	3つ目の角を左に曲がりなさい 
\\	[((英)) 
\\	角を曲がって2軒目が山田さんの家です 
\\	町角に電話ボックスがある 
\\	〔人を刺激するような点〕 彼の言葉には角がある 
\\	角が立つ 
\\	そう言っては角が立つ 
\\	せっかくの申し出を断ったら角が立つよ 
\\	角が取れる 
\\	角が取れた人 
\\	彼女は結婚してから角が取れた 
\\	つの 
\\	〔牛・やぎなどの〕
\\	〔しかの枝角〕
\\	通例複数形) 角のある[ない]やぎ 
\\	子牛(こうし)に角が生えてきた 
\\	しかが角を落とした 
\\	雄牛(おうし)に角で突かれて死んだ 
\\	〔形が角状のもの〕 かたつむりが角を出した[角を引っ込めた] 
\\	角突き合わせる 彼の妻と母親はいつも角突き合わせている 
\\	角を折る 彼はついに角を折って私たちの提案に同意した 
\\	角を出す[生やす] 
\\	彼が遅く帰ると妻君(めぎみ)が角を生(は)やして玄関に立っていた 
\\	角を矯(た)めて牛を殺す 
\\	中小企業をあまり締め付けると,角をためて牛を殺すことになりかねない 
\\	角細工 
\\	〔細工品〕
\\	核	かく	
\\	〔果実の〕
\\	((米)) 
\\	〔細胞の〕
\\	((複
\\	〔原子核〕
\\	((複
\\	核の 
\\	原子核 
\\	〔核兵器〕
\\	我が国は核の持ち込みを禁止している 
\\	〔物事の最重要点〕
\\	この市民団体がその運動の核になっている 
\\	核アレルギー 
\\	核医学 
\\	核移植 
\\	核エネルギー 
\\	核拡散 
\\	核拡散防止条約 
\\	((略
\\	核家族 ⇒詳細を見る 核クラブ 
\\	核軍縮 
\\	核シェルター 
\\	核施設 
\\	核実験 
\\	地下核実験 
\\	核実験禁止条約 
\\	臨界前核実験 
\\	核戦争 
\\	核大国 
\\	核弾頭 
\\	核燃料 
\\	核燃料サイクル 
\\	核燃料産業 
\\	核燃料再処理工場 
\\	核の傘 
\\	核廃棄物 
\\	核爆弾 
\\	核爆発 
\\	核反応 
\\	核武装 
\\	核武装する 
\\	核武装禁止地域 
\\	核物理学 
\\	核分裂 ⇒詳細を見る 核兵器 
\\	((米俗)) 
\\	核兵器拡散 
\\	核兵器廃絶 
\\	核保有国 
\\	核融合 
\\	核抑止力 
\\	格	かく	
\\	〔格式,身分〕
\\	彼の方が格が上だ 
\\	格が上がる[下がる] 
\\	彼は我々とは格が違う 
\\	/((口)) 
\\	親分格の人 
\\	〔文法〕
\\	主格 
\\	目的[所有]格 
\\	格変化 
\\	学歴	がくれき	
\\	駆ける	かける	
\\	坂(さか)を駆け上がる[下りる] 
\\	駅まで駆け通した 
\\	家に駆けて帰った 
\\	賭ける	かける	
\\	競馬に金をかける 
\\	どちらが勝つかかけようじゃないか 
\\	あの男がやったにきまっている
\\	かけてもいい 
\\	彼女は今仕事に賭けている 
\\	課題	かだい	
\\	〔題〕
\\	都会生活という課題で作文を書く 
\\	〔練習問題〕
\\	〔宿題〕
\\	((米)) 
\\	今日は文法の課題をたくさん出します 
\\	夏休みの課題 
\\	〔解決すべき問題〕
\\	これが我々の当面する課題だ 
\\	課題曲 
\\	片付け	かたづけ	
\\	加味	かみ	
\\	借り	かり	
\\	〔借金〕
\\	彼に借りがある 
\\	君に1万円借りがある 
\\	1か月以内にきっと借りを返します 
\\	あのバーに借りがたまった 
\\	これで彼には何も借りは無くなった 
\\	〔簿記(ぼき 
\\	で〕
\\	⇒かりかた(借り方) 私の借りに付けておいてください 
\\	〔人から受けた恩・恨み〕 あなたにはたいへん借りがあります 
\\	あの件で彼にたいへんな借りを作った 
\\	恩人(おんじん)に借りを返さなければならない 
\\	この借りはきっと返してやるぞ 
\\	狩り	かり	
\\	管	かん 
\\	くだ	
\\	くだ 
\\	管を巻(ま)く 
\\	管を巻くのはもうよせ 
\\	かん 
\\	〔くだ〕
\\	〔導管〕
\\	水道[ガス]管を引く 
\\	〔管楽器〕
\\	~観	かん	
\\	〔有り様(さま)〕
\\	彼は以前とは別人の観がある 
\\	いよいよ大詰(おおづ)めが迫(せま)ったという観がある 
\\	〔見方〕
\\	癌	がん	
\\	刊行	かんこう	
\\	刊行する 
\\	茶の湯の本が英語で刊行された 
\\	会報は年2回刊行される 
\\	刊行物 
\\	定期刊行物 
\\	慣行	かんこう	
\\	⇒かんれい(慣例) 慣行の 
\\	慣行を守る 
\\	慣行に従(したが)って 
\\	鉄道の運賃(うんちん)値上げは毎年の慣行であった 
\\	歓声	かんせい	
\\	官僚	かんりょう	
\\	官僚的 
\\	高級官僚 
\\	官僚的形式(けいしき)主義 
\\	官僚化する 
\\	官僚主義 
\\	官僚主義者 
\\	官僚政治 
\\	器官	きかん	消化器官 
\\	発声(はっせい)器官 
\\	季刊	きかん	
\\	起源	きげん	
\\	生命(せいめい)の起源 
\\	…を起源とする 
\\	/((文)) 
\\	/〔ある年代から始まる〕
\\	起源を尋ねる 
\\	この行事(ぎょうじ)の起源はギリシア (人) にある 
\\	この風習(ふうしゅう)の起源は15世紀にさかのぼる 
\\	さかのぼる 
\\	機構	きこう	
\\	構 
\\	〔組織の仕組み〕
\\	〔組織〕
\\	〔構造〕
\\	機構を改革(かいかく)する 
\\	北大西洋条約機構 
\\	((略
\\	国連の機構 
\\	行政機構 
\\	社会機構 
\\	経済機構 
\\	流通機構 
\\	人体の機構を調べる 
\\	〔機械の内部の仕組み〕
\\	この時計の機構は簡単だ 
\\	機構改革 
\\	築く	きずく	
\\	〔土や石を固めて作る〕
\\	城を築く 
\\	石垣(いしがき)[堤(つつみ)]を築く 
\\	〔しっかりと作り上げる〕 彼は会社のために確固(かっこ)たる基礎(きそ)を築いた 
\\	確固たる 
\\	彼は徒弟(とてい)から現在の地位を築き上げた 
\\	富(とみ)と名声(めいせい)を築く 
\\	幸せな家庭を築く 
\\	身代(しんだい)を築く 
\\	規制	きせい	
\\	規制する 
\\	〔統制する〕
\\	交通規制 
\\	排(はい)ガス規制 
\\	ストは法で規制されている 
\\	自主規制 ⇒じしゅ(自主) 規制解除[緩和/撤廃] 
\\	排ガス規制 ⇒はいガス(排ガス)
\\	丘陵	きゅうりょう	
\\	協議	きょうぎ	
\\	享受	きょうじゅ	
\\	この国は自然の恵(めぐ)みを享受している 
\\	この国の人々は言論(げんろん)の自由を享受している 
\\	協調	きょうちょう	
\\	協調する 
\\	[((文)) 
\\	協調的 
\\	協調の精神 
\\	労資協調 
\\	協調性 協調性がある[ない] 
\\	協調介入 
\\	切り	きり	
\\	切れ 
\\	〔区切り〕
\\	切りをつける 
\\	/((口)) 
\\	ここは切りがよい 
\\	切りのいいところまで読む 
\\	議論に切りをつけた 
\\	その問題は早く切りをつけなさい 
\\	〔際限〕
\\	欲(よく)には切りがない 
\\	彼の野望(やぼう)には切りがない 
\\	彼が決心するのを待っていたら切りがない 
\\	親の借金を返していたら切りがない 
\\	菌	きん	
\\	〔細菌〕⇒さいきん(細菌) 菌の培養(ばいよう) 
\\	赤痢菌(せきりきん) 
\\	菌の保有者 
\\	⇒きのこ(茸),かび(黴) 菌類 
\\	近視	きんし	
\\	苦	く	
\\	〔苦しみ〕
\\	⇒くつう(苦痛) 苦あれば楽あり/苦は楽の種 ((諺)) 
\\	三重苦 
\\	/〔比喩的に〕
\\	〔心配,悩み〕
\\	借金を苦にしていた 
\\	貧乏なんか一向(いっこう)苦にしない 
\\	暑さなど苦にならない 
\\	彼女は物事を苦にしない 
\\	〔骨折り〕
\\	((文)) 
\\	苦もなく 
\\	彼女は苦もなくその迷路(めいろ)を通り抜けた 
\\	群	ぐん	
\\	⇒むれ(群れ),あつまり(集まり) 群を抜く 
\\	あの子は数学では群を抜いている 
\\	〔数学で〕
\\	刑	けい	
\\	経緯	けいい	
\\	〔縦糸と横糸〕
\\	〔経度と緯度〕
\\	〔細かいいきさつ〕 彼の離婚の経緯を聞かせてくれ 
\\	経緯儀 
\\	計器	けいき	
\\	契機	けいき	
\\	これを契機として心を改(あらた)めなさい 
\\	新校舎ができたのを契機にクラブ活動が盛んになった 
\\	病気になったのを契機に酒もたばこもやめた 
\\	携帯	けいたい	携帯する 
\\	双眼鏡(そうがんきょう)携帯のこと 
\\	銃砲の携帯は禁止されている 
\\	免許証不携帯で罰金をとられた 
\\	パスポートは常時携帯のこと 
\\	携帯依存症 
\\	〔中毒〕 
\\	携帯依存症になる中学生がふえている 
\\	携帯食料 〔非常用〕
\\	〔米軍の〕
\\	携帯手荷物 ((米)) 
\\	[((英)) 
\\	〔機内持ち込みの〕
\\	[((英)) 
\\	携帯電話 
\\	[((口)) 
\\	携帯電話の電源を切る 
\\	プリペイド式携帯電話 
\\	携帯燃料 
\\	((米)) 
\\	((英)) 
\\	携帯品 
\\	((米)) 
\\	((英)) 
\\	携帯品預り所 ((米)) 
\\	((英)) 
\\	携帯用 
\\	携帯用ラジオ 
\\	携帯用の辞書 
\\	[参考]「携帯する 
\\	という動詞で、また「携帯(用)ラジオ 
\\	のように形容詞的に用いられるのが普通。特に最近では「携帯電話」のことを単に「携帯[ケータイ] 
\\	ということが多い。
\\	形態	けいたい	
\\	⇒かたち(形) 形態学 〔生物学で〕
\\	形態素 
\\	形態論 〔言語学で〕
\\	血管	けっかん	
\\	決行する	けっこう	決行する 
\\	隊は明朝(みょうちょう)登頂(とうちょう)を決行することにした 
\\	試合は雨天(うてん)にもかかわらず決行された 
\\	従業員(じゅうぎょういん)はスト決行中だ 
\\	件	けん	急を要する件で 
\\	その件については何も知りません 
\\	規律(きりつ)の件ではうちの学校はなかなか厳しい 
\\	先月殺人が2件あった 
\\	減少	げんしょう	
\\	公演	こうえん	
\\	後悔	こうかい	〔残念に思う気持ち〕
\\	〔悔い改める気持ち〕
\\	〔自責の念〕
\\	後悔する 
\\	は改まった語) 彼は自分のしたことを後悔している 
\\	もっとよく考えればよかったと後悔している 
\\	罪を後悔しています 
\\	/((文)) 
\\	後悔先に立たず 
\\	/((諺)) 
\\	航海	こうかい	〔航行〕
\\	〔船旅(ふなたび)〕
\\	〔巡航(じゅんこう)〕
\\	航海する 〔人や船が〕
\\	〔人が〕
\\	〔巡航する〕
\\	遠洋航海 
\\	処女航海 
\\	その船は航海中である 
\\	豪華客船で太平洋を航海中 
\\	沖縄までゆっくり航海した 
\\	真冬に航海に出た 
\\	大荒れの[静かな]航海でした 
\\	航海の無事を祈ります 
\\	航海権 
\\	航海士 
\\	一等航海士 
\\	航海時間 
\\	航海術 
\\	航海図 
\\	航海長 
\\	〔ヨットなどの〕
\\	〔軍艦の〕
\\	航海灯 
\\	航海日誌 
\\	拘束	こうそく	
\\	〔束縛(そくばく)〕 拘束する 〔制限する〕
\\	〔規制する〕
\\	〔束縛する〕
\\	現体制(げんたいせい)では言論(げんろん)の自由は拘束されている 
\\	そんな理不尽(りふじん)な制約に拘束されたくない 
\\	〔監禁〕
\\	〔拘留〕
\\	彼は1週間身柄(みがら)を拘束された 
\\	拘束を解(と)く 
\\	拘束時間 
\\	この仕事は楽だが拘束時間が長い 
\\	拘束服[衣] 
\\	拘束名簿式比例代表制(こうそくめいぼしきひれいだいひょうせい) 
\\	拘束力 
\\	総会の決議(けつぎ)は全会員に対し拘束力を持つ 
\\	荒廃	こうはい	
\\	降伏	こうふく	
\\	((文)) 
\\	条件付きで) 降伏する 
\\	〔屈する〕
\\	…という条件で降伏する 
\\	無条件降伏を強(し)いられた 
\\	降伏条件 
\\	降伏文書 
\\	興奮	こうふん	〔快(こころよ)い〕
\\	〔不快(ふかい)な〕
\\	興奮する 
\\	((口)) 
\\	興奮させる 
\\	/〔器官などを〕
\\	興奮して 
\\	興奮しやすい 
\\	興奮はなかなか冷(さ)めなかった 
\\	昨夜は興奮して眠れなかった 
\\	彼は何をあんなに興奮しているのかね 
\\	興奮するなよ 
\\	興奮剤(ざい) 
\\	語句	ごく	語句の用法 
\\	個々	ここ	
\\	個々に 〔個別に〕
\\	〔一人・一個ずつ〕
\\	〔別々に〕
\\	時々,先生は私たちを個々に指導される 
\\	それは我々個々の問題だ 
\\	少年たちは個々に部屋をあてがわれた 
\\	彼らは個々別々の方法でそれをした 
\\	君たちは個々に決めなければならない 
\\	故人	こじん	
\\	小銭	こぜに	
\\	ことによると	ことによると	
\\	事によると事態は悪くなるかもしれない 
\\	、、、、、、、、 ことによるともう二度と彼らに会えないかもしれないという気がした 
\\	採集	さいしゅう	
\\	作	さく	
\\	〔作品〕⇒さくひん(作品) 苦心(くしん)の作 
\\	この彫刻(ちょうこく)はロダンの作だ 
\\	この曲はだれの作ですか 
\\	〔作物の出来具合〕
\\	平年作 
\\	策	さく	〔計画〕
\\	〔方策〕
\\	〔政策〕
\\	〔方法,手段〕
\\	事故防止に万全の策を講じる必要がある 
\\	彼は税金をごまかそうといろいろ策をめぐらした 
\\	策を弄(ろう)する 
\\	策を誤る 
\\	策が尽きた 
\\	彼はなかなか策に富む人だ 
\\	策の施しようがない 
\\	柵	さく	
\\	裂ける	さける	
\\	〔ぎざぎざに〕
\\	〔縦にすっぱりと〕
\\	雷(かみなり)が落ちて木が二つに裂けた 
\\	この布は裂けやすい 
\\	詰め込み過ぎて袋が裂けてしまった 
\\	それは口が裂けても言えない 
\\	口が裂けても言えない 
\\	裁く	さばく	
\\	事件を公平に裁く 
\\	けんかを裁く 
\\	天が彼の罪を裁いた 
\\	酸化	さんか	
\\	死	し	
\\	資格	しかく	
\\	〔身分,地位〕
\\	彼は何の資格でその会議に出席したのか 
\\	〔権利〕
\\	国民はすべて法律の保護を受ける資格がある 
\\	私には君を責(せ)める資格はない 
\\	〔必要な条件〕
\\	〔証明〕
\\	〔免許〕
\\	((英)) 
\\	有[無]資格者 
\\	人に資格を与える 
\\	資格を奪(うば)う 
\\	彼は弁護士の資格を取った 
\\	私は英語を教える資格を持っている 
\\	彼はその研究会のメンバーになる資格がある 
\\	[((英)) 
\\	会員になるのに特別な資格はいりません 
\\	応募(おうぼ)[入学]資格として高校卒業が必要である 
\\	資格試験 
\\	資格審査(しんさ) 
\\	資格審査委員会 
\\	資格喪失(そうしつ) 
\\	資格認定試験 
\\	視覚	しかく	
\\	指揮	しき	
\\	〔軍隊などの司令〕
\\	〔指図〕
\\	指揮する 
\\	阿川氏の指揮の下に 
\\	彼はその兵団(へいだん)の指揮を取った 
\\	〔楽団の〕 オーケストラを指揮する 
\\	小川氏の指揮で 
\\	指揮官 
\\	指揮系統(けいとう) 
\\	この会社は指揮系統がしっかりしている 
\\	指揮権 
\\	指揮権を発動する 
\\	指揮者 〔指導者〕
\\	〔音楽で〕
\\	指揮台 〔音楽で〕
\\	((複-
\\	指揮棒 〔音楽で〕
\\	磁気	じき	
\\	磁器	じき	
\\	自己	じこ	
\\	自己を知る 
\\	自己の利益のみをはかる 
\\	自己暗示(あんじ) 
\\	自己暗示にかかる 
\\	自分は出来るのだと自己暗示にかけた 
\\	自己感染 
\\	自己欺瞞(ぎまん) 
\\	自己血輸血 
\\	自己嫌悪 
\\	彼は自己嫌悪に陥ってどうしようもない 
\\	自己顕示(けんじ) 自己顕示をする 
\\	自己顕示欲(よく) 自己顕示欲の強い人 
\\	自己顕示欲が旺盛(おうせい)だ 
\\	自己抗体 
\\	自己資金 
\\	自己主張 
\\	自己主張する 
\\	自己資本 
\\	自己資本比率 
\\	自己紹介 
\\	自己紹介をする 
\\	右端から順に自己紹介して下さい 
\\	自己紹介をさせていただきます
\\	私は奥田健と言います 
\\	自己申告制度 
\\	自己宣伝 
\\	自己宣伝をする 
\\	自己疎外(そがい) 
\\	自己中(心) 
\\	彼は自己中だ 
\\	自己中心的 
\\	自己陶酔(とうすい) 
\\	((口)) 
\\	自己破産 自己破産を申請(しんせい)する 
\\	自己批判 
\\	自己批判する 
\\	自己負担(ふたん) 
\\	費用は自己負担(ふたん)だ 
\\	酒代(さかだい)は自己負担にする 
\\	酒代 
\\	自己弁護 
\\	自己弁護する 
\\	君の自己弁護にはあきあきだ 
\\	自己保存 
\\	自己本位 自己本位の人 
\\	自己満足 
\\	自己満足の 
\\	自己満足する 
\\	そんなことで自己満足するべきではない 
\\	自己流 自己流で 
\\	自己流のピアニスト 
\\	自己流ですよ 〔先生について学んでいない〕
\\	/〔我流〕
\\	自己憐憫(れんびん) 
\\	自己憐憫に陥(おちい)る 
\\	字体	じたい	
\\	辞退	じたい	
\\	辞退する 
\\	経済援助の申し出を辞退する 
\\	彼はその賞を辞退した 
\\	(招待(しょうたい)の) 辞退の返事を出した 
\\	辞退の方のみお返事ください 
\\	招待カードの左下隅(ひだりしたすみ 
\\	に書く)
\\	視点	してん	
\\	視点を変えて考えてみなさい 
\\	脂肪	しぼう	
\\	志望	しぼう	
\\	衆	しゅう	
\\	修飾	しゅうしょく	
\\	〔飾ること〕
\\	修飾する 
\\	〔潤色(じゅんしょく 
\\	する〕
\\	文章を美辞麗句(びじれいく)で修飾する 
\\	彼の話は修飾が多い 
\\	ごてごてと修飾の多い文章を書く 
\\	〔文法で〕
\\	この副詞はどの語を修飾するか 
\\	修飾語 
\\	名詞修飾語 
\\	私用	しよう	
\\	仕様	しよう	
\\	〔方法,手段〕
\\	どうしようもない 
\\	なんとかしようがありそうなものだ 
\\	父に経済的援助を頼むよりほかしようがなかった 
\\	涙が出てきてしようがなかった 
\\	しようのないいたずらっ子だね 
\\	彼はそのことを友だちに言いたくてしようがなかった 
\\	今日は寒くてしようがない 
\\	〔物事の内容の規定〕
\\	通例複数) 仕様を変更(へんこう)する 
\\	仕様書 
\\	情	じょう	
\\	〔感情,気持ち〕
\\	〔心情〕
\\	彼は情が激して絶句(ぜっく)してしまった 
\\	情をこめて話す 
\\	懐旧(かいきゅう)の情 
\\	懐旧 
\\	情において彼を解雇(かいこ)できなかった 
\\	〔情(なさ)け,情愛〕
\\	情が深い[薄い]人 
\\	彼女から情のこもった贈り物をもらった 
\\	彼はたいへん情にもろい 
\\	[((文)) 
\\	は感動しやすい,
\\	は感じやすい) 情にもろい 
\\	脆(もろ)い 
\\	〔愛情〕
\\	〔異性への〕
\\	夫婦の情 
\\	情が移(うつ)る 
\\	一緒に仕事をしているうちにだんだん彼女に情が移ってきた 
\\	情を通じる 
\\	〔内通する〕 
\\	敵に情を通じる 
\\	〔密通する〕
\\	二人は情を通(つう)じるようになった 
\\	生涯	しょうがい	
\\	〔一生〕
\\	〔職業上の〕
\\	公的な生涯 
\\	生涯の事業 
\\	生涯の友 
\\	彼は生涯に小説を20冊書いた 
\\	幸福な生涯を送った 
\\	がんの研究に生涯をささげる 
\\	医者としての生涯を送る 
\\	彼はこの町で生涯を閉じた 
\\	〔一生ずっと〕 これは生涯忘れられない経験だ 
\\	彼は生涯貧乏だった 
\\	生涯名誉(めいよ)市民の称号(しょうごう)を与えられた 
\\	生涯学習 
\\	生涯教育 
\\	上司	じょうし	
\\	正体	しょうたい	
\\	〔真の性格〕
\\	正体を現す 
\\	/〔うっかりして〕
\\	彼の正体がつかめない 
\\	それが彼女の正体だ 
\\	がたがた言う音の正体は何だろう 
\\	あのいかさま師の正体を暴(あば)いてやろう 
\\	〔身元〕
\\	正体を隠す 
\\	その俳優はハワイで正体を隠して旅行しようとした 
\\	正体を現す 
\\	〔正気〕 
\\	正体もなく酔う 
\\	正体もなく眠ってしまった 
\\	照明	しょうめい	〔光を当てて明るくすること〕
\\	〔舞台照明〕
\\	照明する 
\\	直接[間接]照明 
\\	照明の効果 
\\	照明 佐藤明〔映画・舞台などで〕 
\\	建物を照明で照(て)らし出す 
\\	この部屋は照明がよい[悪い] 
\\	人に局部(きょくぶ)照明を当てる 
\\	舞台全体に青い照明を当てる 
\\	照明係(がかり) 
\\	照明器具 
\\	〔取り付けの〕
\\	照明弾 
\\	女史	じょし	〔未婚者(みこんしゃ)〕
\\	〔既婚者(きこんしゃ)〕
\\	は未婚・既婚の別をつけないときに用いるが,当人が望む場合にのみ用いるのが無難) 中田女史 
\\	助詞	じょし	
\\	進行	しんこう	
\\	〔前に進んでいくこと〕
\\	進行する 
\\	進行中の列車から飛び下りた 
\\	〔物事がはかどること〕
\\	(急速に) 進行する 
\\	仕事は順調に進行している 
\\	事件の調査が進行中である 
\\	論文の進行はいかがですか 
\\	議事の進行を図る[妨げる] 
\\	〔病気が悪化すること〕 進行する 
\\	進行性の 
\\	彼の病気はかなり進行している 
\\	進行係 
\\	進行形 〔文法で〕
\\	進行方向 この道を行くと進行方向に向かって右側に銀行があります 
\\	進行方向に雷雲(らいうん)が見える 
\\	骨患部(ほねかんぶ)の外科切除(せつじょ)は、治癒(ちゆ)の促進および病気進行の抑制に役立つものと思われる 
\\	2つの一般的な鎮痛剤では、アルツハイマー病の進行を遅らす効果がみられない 
\\	新興	しんこう	
\\	新興の 〔発展中の〕
\\	〔新しい〕
\\	新興階級 
\\	〔台頭中の〕
\\	新興国 
\\	新興産業 
\\	新興宗教 
\\	新興成長産業 
\\	新興都市 
\\	〔にわかブームで急速に発展した〕
\\	振興	しんこう	
\\	振興する 
\\	貿易(ぼうえき)[産業]の振興を図(はか)る 
\\	その協会(きょうかい)は科学の振興を目的として設立(せつりつ)された 
\\	申告	しんこく	
\\	は主に収入について使うが
\\	は関税品にも収入にも使える) 申告する 
\\	青色申告 
\\	確定(かくてい)申告 
\\	所得税(しょとくぜい)の申告 
\\	所得税の申告をする 
\\	税関にカメラを申告する 
\\	申告締め切り日 
\\	申告者 〔税金の〕
\\	申告書 
\\	申告納税額 
\\	は還付されるとき) 申告納税者 
\\	申告納税制度 
\\	申告用紙 
\\	真理	しんり	
\\	真理を探求(たんきゅう)する 
\\	彼の言うことにも一面の真理がある 
\\	水洗	すいせん	
\\	水洗トイレの水を流す 
\\	擦る	する	
\\	〔こすりあわせる〕
\\	〔こすってすりむく〕
\\	彼はマッチを擦った 
\\	〔使い果たす〕
\\	彼は競馬(けいば)で身代(しんだい)を擦った 
\\	彼は金相場(きんそうば)で大分擦った 
\\	正規	せいき	正規の[に] 〔通常の〕
\\	〔正常の〕
\\	正規の授業 
\\	正規の手続きを踏(ふ)む 
\\	正規のルートを経る 
\\	正規軍 
\\	正規採用 
\\	正規従業員 
\\	正規分布(ぶんぷ) 
\\	精巧	せいこう	精巧な 〔念入りに仕上げた〕
\\	〔優美に洗練された〕
\\	これは精巧な出来栄(ば)えだ 
\\	精巧な機械 
\\	精算	せいさん	
\\	切符の精算をする 
\\	私がまとめて払っておくから後でみんなで精算しよう 
\\	精算額 
\\	精算書 
\\	精算所 〔駅の〕
\\	成年	せいねん	
\\	成年に達する 
\\	彼はまだ成年に達していない 
\\	声明	せいめい	
\\	姓名	せいめい	
\\	生理	せいり	
\\	〔生物の生活する原理〕
\\	生理的 
\\	生理 (学) 上 
\\	生理的欠陥(けっかん) 
\\	生理的現象 
\\	動物は火に対して生理的に恐怖感を持っている 
\\	あの人とは生理的に合わない 
\\	〔月経〕
\\	生理の 
\\	生理になる 
\\	今生理なの 
\\	生理学 
\\	生理学者 
\\	生理休暇 
\\	生理作用 
\\	生理痛 
\\	((口))
\\	生理的食塩水 
\\	生理日 
\\	生理用ショーツ 
\\	生理用ナプキン ((米))
\\	((英)) 
\\	生理用品 
\\	節	せつ 
\\	ふし	せつ 
\\	(一節)
\\	(段落)
\\	【文法】
\\	(詩)
\\	(信条)
\\	(時)
\\	〔時,ころ〕
\\	この節 
\\	この節の若者は礼儀に欠ける 
\\	その節はご親切にありがとうございました 
\\	こちらへお出掛けの節はぜひお立ち寄りください 
\\	西洋人はあいさつとしてこういうことは言わない) 
\\	〔主義,信条〕
\\	彼は自分の節を守った 
\\	[((口)) 
\\	彼はとうとう節を曲げた 
\\	〔文章などの区切り〕
\\	〔詩節〕
\\	第1章第1節を訳す 
\\	〔文法で〕
\\	、、、、、 ふし 
\\	(木の)
\\	(関節・竹の)
\\	(指の)
\\	(旋律)
\\	(箇所)
\\	〔竹などの〕
\\	〔木のこぶ〕
\\	〔関節〕
\\	〔指の関節〕
\\	〔糸などのこぶ〕
\\	竹の節 
\\	節の多い板 
\\	〔メロディー〕
\\	節回し 昔の歌のなつかしい節回し 
\\	歌に節をつける 
\\	節をつけて歌う 
\\	節をつけて読む 
\\	〔…と思われる点〕
\\	彼の説にはどうかと思われる節が二,三ある 
\\	膳	ぜん	
\\	〔食物を載せる台〕
\\	〔台に載せた料理〕
\\	膳を運ぶ 
\\	全員膳に着いた 
\\	膳を調(ととの)えた 
\\	客に膳を出す 
\\	家では焼き魚が膳に載ることが多い 
\\	【例文】 
\\	ご飯1膳 
\\	はし1膳 
\\	禅	ぜん	
\\	禅を実践(じっせん)する 
\\	禅僧 
\\	禅道場 
\\	禅問答 
\\	〔難解なやりとり〕 
\\	選挙	せんきょ	
\\	先行	せんこう	
\\	〔先に行くこと〕 先行する 
\\	先行の3人は間もなく引き返した 
\\	〔順序(じゅんじょ)などが先立つこと〕 先行する 
\\	この名詞に先行する形容詞 
\\	彼の思想は時代に先行していた 
\\	我々のチームが2点先行した 
\\	は野球の場合) 
\\	〔先にあること〕 先行文献(ぶんけん) 
\\	先行権 
\\	先行詞 〔文法で〕
\\	先行指数 
\\	先行投資 
\\	先行法規 
\\	選考	せんこう	
\\	相	そう	
\\	〔ありさま〕 
\\	人生の色々な相 
\\	〔人相〕
\\	水難の相がある 
\\	彼は貴人(きじん)の相がある 
\\	〔吉凶(きっきょう)の印〕
\\	この家には不吉(ふきつ)の相がある 
\\	沿う	そう	
\\	〔離れずにある〕 海岸に沿ってホテルが並んでいる 
\\	海岸に沿って新しい道路が出来た 
\\	鉄道(てつどう)はしばらく川に沿って走っていた 
\\	〔決まった事に従う〕 政府の方針に沿いエネルギーの節約を心掛けている 
\\	添う	そう	
\\	〔そばに付き従う〕
\\	よりそう【寄り添う】 
\\	子供は母親に寄り添っていた 
\\	夫婦は寄り添って歩いた 
\\	つきそう【付き添う】 
\\	病人に付き添う 
\\	母親が付き添ってその子は学校に行った 
\\	〔夫婦になる〕
\\	二人を添わせる 
\\	〔合う,適(かな)う〕 人の要望に添う 
\\	目的に添う 
\\	親の期待(きたい)に添わない子供だった 
\\	今回はご希望に添えなくてすみません 
\\	彼は父の期待に添うだろう 
\\	僧	そう	
\\	像	ぞう	
\\	〔似姿〕
\\	〔彫像〕
\\	大理石のマリア像 
\\	自由の女神像(めがみぞう) 
\\	兵士の像が壁に刻(きざ)まれていた 
\\	〔映像〕
\\	鏡に映る像 
\\	レンズを通った光が像を結(むす)んだ 
\\	捜査	そうさ	
\\	捜査する 
\\	科学捜査 
\\	殺人事件の捜査 
\\	犯人の捜査は1年で打ち切られた 
\\	事件は捜査中である 
\\	駅周辺で容疑者を捜査している 
\\	捜査員 
\\	捜査主任(しゅにん) 
\\	捜査陣(そうさじん) 
\\	捜査線 
\\	捜査線上に彼の名前が浮かんだ 
\\	捜査本部 
\\	特別捜査本部が設(もう)けられた 
\\	捜査網(そうさもう) 全体に捜査網が敷(し)かれた 
\\	捜査令状(れいじょう) 捜査令状が発行された 
\\	操縦	そうじゅう	
\\	〔機械・乗り物の〕
\\	機械を操縦する 
\\	飛行機を操縦する 
\\	船を操縦する 
\\	〔巧みに操る〕
\\	((英)) 
\\	彼女は夫を思いどおり操縦できる 
\\	思いどおり 
\\	彼を操縦するなどわけはない
\\	[((英)) 
\\	わけはない 
\\	わけがない 
\\	彼が陰で操縦したのだ 
\\	操縦桿(かん) 
\\	操縦士 〔飛行機の〕
\\	副操縦士 
\\	操縦席 〔飛行機の〕
\\	操縦性 
\\	操縦装置 
\\	〔船・車の〕
\\	創造	そうぞう	
\\	創造する 
\\	創造的 
\\	天地創造 
\\	創造者 
\\	〔神〕
\\	創造性 
\\	創造性に富(と)む作品 
\\	創造物 
\\	すべての創造物 
\\	創造力 
\\	隊	たい	
\\	退学	たいがく	
\\	高まる	たかまる	
\\	〔高くなる〕
\\	〔程度を増す〕
\\	この作品のおかげで彼の名声(めいせい)は高まった 
\\	彼の悪評(あくひょう)は高まるばかりだった 
\\	新政府を求(もと)める声は日ましに高まった 
\\	石油株に投資家(とうしか)の関心(かんしん)が高まっている 
\\	国中に民主化の気運(きうん)が高まっている 
\\	断つ	たつ	
\\	〔切断する〕
\\	〔全体から〕
\\	鎖(くさり)をばらばらに断つ 
\\	綱(つな)を断つ 
\\	有刺鉄線(ゆうしてっせん)を断つ 
\\	〔習慣などを止める〕
\\	〔控える〕
\\	一か月(いっかげつ)前から酒[たばこ]を断った 
\\	〔さえぎる〕
\\	電流を断つ 
\\	大雪のために軍隊は糧食(りょうしょく)を断たれた 
\\	盾	たて	
\\	挑戦	ちょうせん	
\\	〔権威などに対する反抗〕
\\	挑戦する 
\\	挑戦的 (に) 
\\	〔挑発的〕
\\	無言の挑戦 
\\	挑戦に応じる 
\\	彼はジョンに100メートル競走(きょうそう)で挑戦した 
\\	彼は単身(たんしん)8,000メートルの山に挑戦した 
\\	彼は世界新記録に挑戦した 
\\	世論(せろん)に挑戦した 
\\	彼は挑戦的な態度に出た 
\\	挑戦者 
\\	挑戦状(じょう) 彼らに挑戦状を突(つ)きつけた 
\\	治療	ちりょう	
\\	接ぐ	つぐ	
\\	枝(えだ)を台木(だいぎ)に接いだ 
\\	病院で骨を接いでもらった 
\\	それは木に竹を接いだような話だ 
\\	[((口)) 
\\	継ぐ	つぐ	
\\	(仕事などを)
\\	(相続する)
\\	〔受け継ぐ〕 弟が家業を継いでいる 
\\	彼の遺志(いし)を継いで運動を続けた 
\\	彼はすぐに言葉を継いだ 
\\	〔繕う〕
\\	⇒つくろう(繕う) 
\\	〔足す〕
\\	火に炭(すみ)を継ぐ 
\\	摘む	つむ	
\\	〔集める〕
\\	⇒つみとる(摘み取る) 花[いちご/茶]を摘む 
\\	花を摘みに行く 
\\	ばらのつぼみを摘んではいけない 
\\	母親は構(かま)い過ぎて子供の独立心(どくりつしん)の芽(め)を摘んでしまった 
\\	構う 
\\	露	つゆ	
\\	〔露のしずく〕
\\	草の上に露が降りていた 
\\	〔はかないもの〕 断頭台(だんとうだい)の露と消える 
\\	〔否定語を伴って,少しも (…ない) 
\\	そんな策略(さくりゃく)とは露知らず彼の要求を承諾(しょうだく)した 
\\	露の命[身] 
\\	露の間 
\\	露の間もあなたのことを忘れません 
\\	露の世 
\\	強まる	つよまる	あらしが強まって来た 
\\	政府に抗議する声はますます強まった 
\\	彼に対する疑いは強まった 
\\	不況が長びいて大統領に対する不信の念は強まった 
\\	長引く 
\\	演説を聞いて彼が大人物だという印象が強まった 
\\	/((文))
\\	強める	つよめる	ガスの火を強めた 
\\	この語は第2音節(おんせつ)を強めて発音する 
\\	彼は語気を強めて我々の怠惰(たいだ)をなじった 
\\	かぜに対する抵抗力を強めるよういつも心掛けています 
\\	彼は党内での発言権を強めようとしている 
\\	政府は言論の統制を強める腹づもりだ 
\\	国の防衛力を強めるにはばく大な金がいる 
\\	/((口))
\\	転校	てんこう	
\\	伝言	でんごん	
\\	彼からの伝言を頼まれています 
\\	伝言を奥様のところへ置いてきました 
\\	お父さんに伝言をお願いできますか〔電話で〕 
\\	君によろしくと伝言を頼まれた 
\\	翌日(よくじつ)来るという伝言を彼女は息子に届けさせた 
\\	伝言板 
\\	問う	とう	
\\	〔尋ねる〕
\\	人の安否を問う 
\\	その問題について賛否を問うべきだ 
\\	彼の意図を問う 
\\	〔追及する〕 過去のことを問われても困る 
\\	事故の責任を問われて社長は辞職した 
\\	男は傷害罪に問われた 
\\	〔問題にする〕((否定文で用いる)) 経験は問いません〔広告で〕 
\\	年齢を問わずだれでもできるスポーツです 
\\	[((文))
\\	値段は問わないからぜひ手に入れてくれ 
\\	気立てさえよければ美醜は問わない 
\\	問うに落ちず語るに落ちる 彼は問うに落ちず語るに落ちた 
\\	棟	とう	
\\	〔建物〕
\\	〔建物を数える語〕 3棟 
\\	1号棟の2階 
\\	[((英)) 
\\	むね 
\\	棟上(むねあ)げ 
\\	別棟(べつむね) 
\\	倒産	とうさん	
\\	同士	どうし	
\\	仲間同士じゃないか,口げんかは止(よ)せよ 
\\	兄弟同士のけんか 
\\	気の合った同士でグループを作った 
\\	お互いに人間同士だ 
\\	乗客同士がけんかを始めた 
\\	同士打(う)ち 
\\	同士打ちをした 
\\	暗(くら)やみで同士打ちになった 
\\	同志	どうし	
\\	((文)) 
\\	〔同じ政党などの〕
\\	同志を募(つの)る 
\\	当然	とうぜん	彼は当然解雇(かいこ)された 
\\	彼は当然電話してくるべきだった 
\\	あんなことを言ったら嫌(きら)われて当然だ 
\\	彼女が子供たちの成功に鼻高々(はなたかだか)なのも当然のことだった 
\\	彼がその土地の所有権を主張するのは理の当然だ 
\\	当然の処置(しょち)をとったまでだ 
\\	当然のことながら断られた 
\\	動揺	どうよう	
\\	車がひどく動揺する 
\\	船が動揺しはじめた 
\\	は横揺れ,
\\	は上下動) 
\\	〔不安な状態〕 物価(ぶっか)の動揺 
\\	今年は政界(せいかい)の動揺が続いた 
\\	大統領の不祥事(ふしょうじ)は国民の間に動揺を引き起こした 
\\	は特に社会的・政治的不穏状態) その知らせで彼女はひどく動揺した 
\\	彼は動揺することなく目的に向かって進んだ 
\\	説く	とく	
\\	〔説明する〕
\\	((文)) 
\\	先生はこの点を詳細(しょうさい)に説いた 
\\	この原理を説いた本 
\\	〔教え諭(さと)す〕
\\	〔提唱(ていしょう)する〕
\\	人の道を説く 
\\	彼は息子の誤(あやま)りをこんこんと説いた 
\\	学者たちは新しい外交政策の必要性を説いている 
\\	〔説得(せっとく)する〕
\\	彼を説いて辞意(じい)を翻(ひるがえ)させた 
\\	[((文))
\\	社長を説いて新方式(しんほうしき)を採用(さいよう)してもらった 
\\	とじる (綴じる)	とじる	綴じる 
\\	この本はしっかりとじてある 
\\	新聞をとじておく 
\\	ほころびをとじる 
\\	供	とも	
\\	〔従者〕
\\	〔随行員,集合的〕
\\	皇太子殿下(こうたいしでんか)のお供をしてタイに行った 
\\	供を連れて行く 
\\	供を連れずに行く 
\\	妻のお供をしてコンサートに行った 
\\	あの歌手はいつも大勢のお供をつれて旅行する 
\\	駅までお供しましょう 
\\	供ぞろい・供回り 
\\	並	なみ	
\\	《動詞「なむ」の連用形から》 
\\	並んでいること。並んだもの。ならび。 
\\	よくも悪くもないこと。ごく一般的であること。また、そのもの。中ぐらい。普通。「―の品」「―の選手」 
\\	商品などの等級で、上と下の中間のもの。中程度。「にぎりの―」 
\\	同程度・同類であること。同等。 「わが―の人にはあらじと思ひて」〈源・玉鬘〉 
\\	名詞の下に付いて接尾語的に用いられる。 ㋐そのものと同じ、そのものと同じ程度などの意を表す。「例年―の気候」「世間―の生活」 ㋑その物ごと、その時ごとの意を表す。「軒―」「月―」 ㋒それが並んでいること、いくつも重なっている意を表す。「歯―」「家―」
\\	慣らす	ならす	
\\	〔なじませる〕
\\	私の使い慣らしたペン 
\\	〔鍛(きた)える〕
\\	いい音楽に耳を慣らす必要がある 
\\	競走に備(そな)えて足を慣らす 
\\	はレース直前,
\\	はレースに向けての長い練習)
\\	ならす	ならす	馴らす 
\\	〔手なづける〕
\\	〔訓練する〕
\\	〔家畜化する〕
\\	ゴリラをならす 
\\	いるか(海豚)をならして芸(げい)をさせる 
\\	馬をならす 
\\	難	なん	
\\	〔困難〕
\\	⇒こんなん(困難); なんなく(難無く) 生活難 
\\	深刻(しんこく)な住宅[食糧]難に悩んでいる 
\\	敢然(かんぜん)と難にあたる 
\\	〔災い〕
\\	⇒わざわい(災い); さいなん(災難) かろうじて難を免(まぬが)れた 
\\	近くの小屋に難を避けた 
\\	難を避けて伯父の所に身を寄せた 
\\	一難去ってまた一難 
\\	〔欠点〕
\\	難を言えば,彼女は少し色が黒い 
\\	彼の技法(ぎほう)は難がない 
\\	難は資金が足りないことだ 
\\	〔非難〕
\\	怠慢だったという難は免れない 
\\	音	ね	〔音〕
\\	〔音楽〕
\\	鈴の音 
\\	虫の音 
\\	妙(たえ)なる楽の音 
\\	このバイオリンは音がよい 
\\	ぐうの音も出ない 急所を突かれてはぐうの音もでなかった 
\\	彼の筋道(すじみち)立った議論にはぐうの音も出なかった 
\\	音を上げる 〔降参する〕
\\	〔弱音をはく〕
\\	あまりの重労働に音を上げてしまった 
\\	年鑑	ねんかん	
\\	歴史年鑑 
\\	脳	のう	
\\	臨む	のぞむ	
\\	〔面する〕 
\\	海に臨む丘(おか) 
\\	私の家は湾に臨んでいる 
\\	〔出掛けて行く〕 
\\	戦場(せんじょ)に臨む 
\\	試験に臨む 
\\	[((英))
\\	〔事に当たる〕 
\\	落ち着いて難局(なんきょく)に臨む 
\\	父は平然(へいぜん)と死に臨んだ 
\\	彼の学生に臨む態度は立派だった 
\\	望む 
\\	肺	はい	
\\	~敗	はい	
\\	映える	はえる	はえる【映える・栄える】 
\\	〔照らされて輝く〕 山が夕日(ゆうひ)に映えている 
\\	〔目立つ〕 映える[映えない]色 
\\	〔引き立つ〕
\\	あの赤いコートを着ると彼女は映える 
\\	石1つで庭がぐんと映えてくるだろう 
\\	諮る	はかる	
\\	課長に諮ってお返事します 
\\	この問題は教授会に諮らなくてはならない 
\\	委員全員に諮ったのか 
\\	図る	はかる	
\\	〔計画する〕
\\	〔試みる〕
\\	市場(しじょう)の拡大を図っている 
\\	彼は再挙(さいきょ)を図った 
\\	再挙を図る 
\\	/((口)) 
\\	彼女は自殺を図った 
\\	公益(こうえき)を図る 
\\	私利(しり)を図る 
\\	事件の早期解決(そうきかいけつ)を図る 
\\	〔取り計らう〕⇒はからう(計らう),はからい(計らい) できるだけ便宜(べんぎ)を図りましょう 
\\	便宜 
\\	〔考える〕 あに図らんや,土壇場(どたんば)で大逆転が起こった 
\\	あにはからんや、豈計らんや 
\\	土壇場 
\\	生やす	はやす	
\\	あごひげを生やし始めた 
\\	花壇(かだん)に雑草(ざっそう)を生やしたままにする 
\\	班	はん	〔組〕
\\	〔軍隊などの〕
\\	少年たちを3班に分けた 
\\	第2中隊第6班 
\\	判	はん	
\\	大判(おおばん)の紙 
\\	二つ折り[四つ折り/八つ折り]判 
\\	菊判 
\\	4判 
\\	4判 
\\	4は英国の紙のサイズ
\\	米国は
\\	だが大きさは少し違う)
\\	版	はん	
\\	碑	ひ	〔墓碑(ぼひ)〕
\\	〔記念碑〕
\\	文豪(ぶんごう)を記念して碑を立てる 
\\	非行	ひこう	
\\	青少年の非行 
\\	非行に走る 
\\	非行少年[少女] 
\\	票	ひょう	
\\	〔投票〕
\\	固定票 
\\	浮動(ふどう)票 
\\	組織票 
\\	女性票 
\\	票を投(とう)じる 
\\	彼に票[反対票]を投じた 
\\	全投票数 
\\	彼は20票対5票で委員長に選ばれた 
\\	わずか6票で彼に敗(やぶ)れた 
\\	今回の選挙で彼は約50,000票取った 
\\	10票は無効とされた 
\\	今回の選挙で我が党は票を伸ばした 
\\	〔札〕 認識票〔兵士の〕 
\\	/((米口))
\\	申し込み票 
\\	票が割れる 保守系(ほしゅけい)2候補に票が割れた 
\\	票を読む 〔得票数を見積もる〕
\\	〔投票数を数える〕
\\	票集め 
\\	票集めをする 
\\	票固(かた)め 各候補は票固めに躍起(やっき)になっている 
\\	票数 
\\	票田(ひょうでん) 
\\	労組(ろうそ)は彼の票田だ 
\\	票読み ⇒詳細を見る
\\	広まる	ひろまる	
\\	⇒ひろがる(広がる) 教育が全国民に広まった 
\\	野球が少年の間に広まっている 
\\	その髪型(かみがた)が若い女性の間に広まっている 
\\	深める	ふかめる	
\\	知識を深める 
\\	今度の旅行でその国に対する理解を深めた 
\\	友情を深める 
\\	その事件は二人の溝(みぞ)を更(さら)に深めた 
\\	福	ふく	
\\	振り	ふり	
\\	(振動)
\\	(様子)
\\	(見せ掛け)
\\	〔振動〕
\\	バットの振りが鈍(にぶ)い 
\\	〔様子,なり〕
\\	へんなふりをした男 
\\	人のふり見て我がふり直せ 
\\	〔見せかけ〕
\\	見えないふりをする 
\\	そのことは知らないふりをした 
\\	/((文)) 
\\	危険な時にくも(蜘蛛)はしばしば死んだふりをする 
\\	少年たちはけんかのふりをして騒(さわ)いでいる 
\\	〔なじみでないこと〕 なじみ 
\\	〔店の〕ふりの客 
\\	このホテルはふりの客は泊めない〔予約なしの〕 
\\	〔仕方〕 舞踊(ぶよう)のふりをつける 
\\	⇒ふりつけ(振り付け) ーーー 
\\	振り 
\\	〔…の様子〕 彼の話しぶりが気にくわない 
\\	彼女の客に対する応対ぶりは非の打ち所がない 
\\	その会社の営業ぶりは堅実だ 
\\	彼のあわてぶりを見せたかった 
\\	彼の役者としての成長ぶりには目を見張るものがあった 
\\	〔時間が経過した程度〕 彼に会うのは5年ぶりだった 
\\	3年ぶりに便りをよこした 
\\	兄が10年ぶりで故郷へ帰って来た 
\\	これは7年ぶりの豊作だ 
\\	経る	へる	
\\	〔時がたつ〕
\\	((文)) 
\\	⇒たつ(経つ) 1か月を経ても音沙汰(おとさた)がない 
\\	音沙汰 
\\	なす事もなく日を経る 
\\	事故以来3年を経た 
\\	2年を経て彼らは結婚した 
\\	〔通過する〕
\\	京都を経て大阪へ行く 
\\	インドを経て中近東へ行く 
\\	〔過程を通る〕 書類は課長を経て重役に渡される 
\\	許可が下(お)りるまでには複雑な手続きを経なければならない 
\\	〔経験する〕 多くの困難を経てようやく成功した 
\\	保護	ほご	
\\	保障	ほしょう	
\\	自由の保障 
\\	平和を保障する 
\\	航路の安全を保障する 
\\	安全保障条約 
\\	日米相互協力及び安全保障条約 
\\	安全保障理事会 〔国連の〕
\\	補償	ほしょう	
\\	((文)) 
\\	補償する 
\\	((文)) 
\\	国は原告(げんこく)の訴訟(そしょう)費用を補償すべきである 
\\	作業中のけがに対し,会社に補償を請求した 
\\	補償作用 〔生物の〕
\\	補償金 
\\	その損害に対して補償金を請求した 
\\	彼は立退(たちの)き料として1千万円の補償金をもらった 
\\	ほっと	ほっと	
\\	ほっと息をつく 〔ため息をつく〕
\\	/〔安心して〕
\\	父が無事帰ったのでほっとした 
\\	無事ですって,ほっとした 
\\	論文を書き終えてほっとした 
\\	前もって	まえもって	
\\	前もってお知らせします 
\\	膜	まく	
\\	〔生物学で〕
\\	膜状の 
\\	表面に膜が張った 
\\	牛乳を温めたら膜が出来た 
\\	股	また	
\\	股を広げる 
\\	股を広げて立つ 
\\	股に掛ける 
\\	世界を股に掛ける 
\\	日本国中を股に掛けて手広く商売をする 
\\	股ぐら 
\\	股ずれ 
\\	股ずれがする 
\\	股火鉢(またひばち) 股火鉢をする 
\\	見掛ける	みかける	
\\	あの人はよく駅で見掛ける 
\\	昨日図書館で彼女を見かけた 
\\	よく見掛ける光景(こうけい) 
\\	よく見掛ける顔 
\\	めくる 
\\	まくる	めくる 
\\	まくる	
\\	捲る 
\\	〔巻き上げる〕 ズボンのすそ[シャツのそで]をまくる 
\\	スカートをまくる 
\\	彼女は着物のすそをまくって波打(なみう)ち際(ぎわ)を歩いた 
\\	〔盛(さか)んに行う〕 
\\	彼女はしゃべりまくった ((口)) 
\\	彼は物語を書きまくった 
\\	/〔次から次へと〕
\\	彼は働きまくった 
\\	[((口)) 
\\	〔裏返す〕
\\	本[ページ]をぱらぱらとめくる 
\\	カードをめくる 
\\	5ページまでめくった 
\\	カレンダーをめくる 
\\	/〔1枚引きはがす〕
\\	〔はがす〕
\\	床板(ゆかいた)をめくる 
\\	壁紙(かべがみ)をめくる 
\\	工事人が瓦(かわら)をめくっている 
\\	毛布をめくる 
\\	野党	やとう	
\\	優	ゆう	
\\	有機	ゆうき	有機の[的] 
\\	有機的に 
\\	有機化学 
\\	有機化合物 
\\	有機栽培(さいばい) このレタスは有機栽培による 
\\	有機栽培の野菜 
\\	有機水銀(すいぎん) 
\\	有機水銀中毒 
\\	有機体 
\\	有機体の 
\\	有機農業 
\\	有機肥料(ひりょう) 
\\	有機物 
\\	有機物質 
\\	世	よ 
\\	-せい	よ 
\\	〔世間,世の中〕
\\	世に長らえる限りは 
\\	彼の名は世に忘れられている 
\\	世に知られた科学者 
\\	これが世に言う「名誉革命」だ 
\\	私は世におもねることはしたくない 
\\	世のため人のために尽くす 
\\	〔人生〕
\\	世をはかなむ 
\\	世の荒波を切り抜ける 
\\	〔過去・現在・未来のいずれか〕 あの世 
\\	世を去る 
\\	この世で再びお目にかかることはありますまい 
\\	〔時勢〕
\\	世に後れる 
\\	世に後れないようにする 
\\	世に逆らう 
\\	彼は世に先んじてこの薬の副作用を警告した 
\\	〔時代〕
\\	〔治世〕
\\	世はまさにコンピュータ時代だ 
\\	明治天皇の世に 
\\	徳川の世には厳格な階級制度があった 
\\	世が世なら 世が世なら私も今ごろは楽隠居をしているはずだ 
\\	世に合う 彼は世に合うことなく失意のうちに死んだ 
\\	世に入れられる 
\\	世に処する 
\\	世に背く 
\\	〔隠遁する〕
\\	〔出家する〕
\\	世に出る 
\\	〔世間に知られる〕
\\	〔出世する〕
\\	彼は20代で早くも世に出た 
\\	〔出版される〕 この本は10年前に世に出た 
\\	世に問う 
\\	研究の成果を世に問うことにした 
\\	世の聞こえ 
\\	世の定め 
\\	世の常 それは世の常だ 
\\	世の習い 「おごれる者久しからず」とは世の習いである 
\\	世を忍ぶ あの旅の僧は領主の世を忍ぶ仮の姿です 
\\	世を捨てる 
\\	世をはばかる 世をはばかってひっそり暮らす 
\\	世を渡る 彼はうまく世を渡った 
\\	-せい 
\\	〔一族の…代目〕 ヘンリー四世 
\\	は
\\	と読む) トム・スミス二世 
\\	は
\\	と読む) アメリカ日系三世 
\\	〔地質の年代〕 洪積[沖積]世 
\\	良い	よい	
\\	予想	よそう	
\\	〔予期〕
\\	〔予測〕
\\	〔見積もり〕
\\	予想する 〔予期する〕
\\	〔見積(みつ)もる〕
\\	我がチームの得点(とくてん)は予想どおり[予想はずれ]だった 
\\	業績(ぎょうせつ)の不振(ふしん)は予想どおりだ 
\\	実際の費用は予想を上回(うわまわ)った 
\\	売り上げは予想以上[以下]だった 
\\	さばの水揚(みずあ)げ予想が当たった[外れた] 
\\	混雑を予想して30分早く家を出た 
\\	明日は好天気が予想されている 
\\	人生(じんせい)では次に何が起こるか予想できない 
\\	専門家は景気の回復を予想した 
\\	控(ひか)え目(め)に予想しても2万人の人が参加するだろう 
\\	結果は全く予想がつかない 
\\	予想外 彼女は入試で予想外の成績をおさめた 
\\	事態は予想外の展開となった 
\\	話し合いから予想外の収穫(しゅうかく)を得る 
\\	予想配当 〔株・競馬などの〕
\\	予想屋 〔競馬などの〕
\\	〔選挙などの〕
\\	弱まる	よわまる	
\\	弱める	よわめる	
\\	〔衰弱させる〕
\\	過労(かろう)は体を弱める 
\\	[((文)) 
\\	睡眠不足が彼の体を弱めた 
\\	〔程度を弱くする〕 彼は演説の調子を少し弱めた 
\\	現在の状況は少年の感受性を弱める 
\\	民族主義者の力を弱める 
\\	〔ガスの出などを〕
\\	〔薄める〕
\\	ガスの火を弱めないと肉が焦(こ)げ付きますよ 
\\	この酸(さん)は (水で) 弱めてある 
\\	枠	わく	
\\	〔木などによる囲い〕
\\	窓[戸]の枠 
\\	刺繍(ししゅう)枠 
\\	〔四方を囲む線〕 黒い枠で囲(かこ)まれた死亡広告 
\\	〔制限範囲〕
\\	法規(ほうき)の枠内で 
\\	出費(しゅっぴ)は予算の枠を超えそうだ 
\\	生徒のクラブ活動に枠をはめる 
\\	枠にはまった考え 
\\	も	((係助詞)) 
\\	〔もまた〕
\\	((否定文で)) 
\\	私も彼を知っている 
\\	彼はその子に小遣いもやった 
\\	強い風が吹いていたが雨も降り始めた 
\\	「切手を集めたいな」「僕もだ」 
\\	「ちっともうれしくないよ」「僕もだよ」 
\\	私も彼を知らない 
\\	〔同種の事物の列挙〕
\\	〔二つの物のとき〕
\\	彼は昼も夜も働いた 
\\	ドイツ語もフランス語も話せる 
\\	彼も私も特に利口ではない 
\\	〔さえ〕 猿も木から落ちる 
\\	彼はありがとうも言わずにそれを取った 
\\	〔すべて〕 だれもが彼の勝利を信じていた 
\\	町はどこもかしこも雪景色だった 
\\	何も見えなかった 
\\	何も悪いところはない 
\\	〔強調〕 この品物は1万円もした 
\\	彼はそれを完成するのに10年もかかった 
\\	10人もの人が皆同じ間違いをした 
\\	この村の者は一人も知らない 
\\	冗談もいいところだ 
\\	こうも暑いとぐったりしてしまう 
\\	そんなことをするとは非常識もはなはだしい 
\\	今まで黙っているなんて君も君だよ 
\\	歌いも歌ったり,その数は50曲に及んだ 
\\	3時間も待った 
\\	〔含みを持たせて〕 そのような訳でこの説は間違っているとも言える 
\\	((接続助詞))〔たとえ…でも〕 遅くとも7時には帰るでしょう 
\\	高くても高くなくても買います 
\\	行っても行かなくてもよい 
\\	家族がいるので仕事をやめたくもやめられない 
\\	電車はとても混んでいて体を動かそうにも動かせなかった 
\\	どこを見てもがれきの山だった 
\\	いつ行っても留守だった 
\\	在る	ある 
\\	〔存在する〕
\\	〔生きる〕
\\	かごの中にりんごがある 
\\	君は神が本当にあると思うか 
\\	これは今ある船でいちばん大きいものです 
\\	この世にある限り君を忘れない 
\\	〔位置・地位を占める〕
\\	((米)) 
\\	学校は東京の郊外にある 
\\	記念塔は公園にある 
\\	川の向こうに牧場(ぼくじょう)があった 
\\	この村には城址(じょうし)がある 
\\	彼は政府の要職にあった 
\\	出来る	
\\	〔仕上がる〕
\\	〔準備が整う〕
\\	⇒できあがる(出来上がる) 仕事が出来た 
\\	食事が出来た 
\\	準備が出来次第(できしだい)出かけよう 
\\	出来次第 
\\	試験勉強がまだ出来ていない 
\\	どうなっても覚悟は出来ている 
\\	〔作られる〕
\\	バターは牛乳から出来る 
\\	は原料を示す) この机は木で出来ている 
\\	は材料を示す) この布でスカートが2枚出来る 
\\	この橋は頑丈に出来ている 
\\	大雨で庭に大きな水たまりが出来た 
\\	この玩具(おもちゃ)は精巧(せいこう)に出来ている 
\\	このスープは上手に出来ている 
\\	どんなピザが出来ますか〔店などで〕 
\\	〔産出される〕
\\	このあたりでは桃がよく出来る 
\\	この農場では促成栽培(そくせいさいばい)の野菜がよく出来る 
\\	促成栽培  
\\	〔生まれる〕
\\	〔生じる〕
\\	〔形を成す〕
\\	二人の間に女の子が出来た 
\\	来年もう一人子ども(供)が出来ます 
\\	新しい憲法が出来た 
\\	胃に潰瘍(かいよう)が出来たらしい 
\\	彼女の手にあかぎれが出来ていた 
\\	背中に湿疹(しっしん)が出来た 
\\	急用ができて帰って行った 
\\	〔生まれつく〕 
\\	君はよほど意気地(いくじ)なしに出来ているのだな 
\\	〔組織される〕
\\	新しい政党が出来た 
\\	新しい制度がやがて出来るだろう 
\\	この会社が出来てから50年になる 
\\	〔可能である〕((人が主語で)) 
\\	((事が主語で)) 
\\	彼なら出来る 
\\	その時は許すことが出来なかった 
\\	それは出来ない相談だ 
\\	難しくて私には出来ない 
\\	悪天候(あくてんこう)のため船は出航(しゅっこう)出来なかった 
\\	出来たら10日までに仕上げてください 
\\	〔能力がある〕 よく出来る生徒 
\\	出来ない生徒 
\\	学校の勉強が出来る[出来ない] 
\\	数学が出来る[出来ない] 
\\	あの男は出来る 
\\	〔人格のある〕 彼はよく出来た人物だ 〔円熟した人〕
\\	/〔人格者〕
\\	〔男女が深い仲になる〕 
\\	あの二人は出来ているらしい 〔相愛の仲〕
\\	/〔肉体関係がある〕
\\	できあがる【出来上がる】 
\\	〔完成する〕
\\	その家は来月出来上がる 
\\	ワンピースが出来上がるのにあと何時間掛かりますか 
\\	〔酔う〕 もう出来上がっている 
\\	あいつ出来上がっちまったらしいな 
\\	知る	
\\	〔認識する〕
\\	どこで会があるか知ってますか 
\\	私は何も知りません 
\\	知らないうちに会から除名されていた 
\\	世界的に知られた学者 
\\	/〔有名な〕
\\	私の知る限りでは 
\\	知る権利 
\\	そのことは今朝まで知らなかった 
\\	事件はたちまち人々の知るところとなった 
\\	〔気付く〕
\\	〔悟る〕
\\	道を間違えたことを知った 
\\	私は健康が大切だということを知るようになった 
\\	[((文))
\\	彼は自分の結婚申し込みが相手にはありがたくないのを知った 
\\	僕はそれがごまかしだと知っていた 
\\	彼は自分の弱点を知らない 
\\	それはうすうす知っていた 
\\	〔見分ける〕 うわさか事実かを知るのは難しい 
\\	あざらし(海豹)とせいうち(海象)の区別を知らない 
\\	〔内容を理解する〕 ロシア語を多少知っている 
\\	あの会社の事情をよく知っている 
\\	こつを知っている 
\\	妻は僕の意図をちゃんと知っている 
\\	一を聞いて十を知る男だ 
\\	知った風な口をきくな 
\\	京都を知ってますか 
\\	〔感じる〕 恥(はじ)[憐(あわ)れみ]を知らない男 
\\	〔予見する〕 こんな不幸が来るとは知らなかった 
\\	〔経験する〕 私の全く知らない世界のことだ 
\\	彼女は世の中の苦労を知らない 
\\	酒の味を知った 
\\	本当の空腹(くうふく)を知っていますか 
\\	〔面識がある〕 鈴木君をよく知っている 
\\	知らない人たちの中ではいつも気後れがしてしまう 
\\	あの人の顔[名前]は知っている 
\\	〔関係する〕 私の知ったことではない 
\\	最後の表現が最もぶっきらぼう) 知る人ぞ知る そのバイオリンは知る人ぞ知る名器だ 
\\	分かる	
\\	〔理解する〕
\\	〔洞察する〕
\\	〔よさを認める〕
\\	彼の言っていることは分からない 
\\	私の言うことが分かってもらえなかった 
\\	君の理屈(りくつ)はちっとも分からない 
\\	それで訳が分かりました 
\\	私には到底(とうてい)分からない 
\\	何をやらかそうとしているのか皆目(かいもく)分からない 
\\	/〔行動が不可解だ〕
\\	一体何が不満なのか分からない 
\\	病人が何かささやいたが何と言ったのか分からなかった 
\\	講演を聴いて北京の事情がだいぶ分かった 
\\	分かりました
\\	それでは別の人に当たってみましょう 
\\	これでやっと世の中が分かった 
\\	あの人が自殺するとは,分からないものですね 
\\	人の命とは分からないものだ 
\\	彼には詩が分からない 
\\	私は音楽は分かりません 
\\	外交交渉のことが分かっている上田氏は代表団に加わることを断った 
\\	〔知る〕
\\	〔見分ける〕
\\	〔発見する〕
\\	時がたてば分かる 
\\	今日は行かないほうがよいことはよく分かっている 
\\	薄々(うすうす)分かっていた 
\\	ついに彼の居所が分かった 
\\	答の分かった人は手を挙げなさい 
\\	してよいことと悪いことが分からないのか 
\\	どうしてよいのか分からない 
\\	災害はいつ起こるか分からない 
\\	声だけではどっちがどっちか分からなかった 
\\	この計画がうまくいくかどうかはまだ分からない 
\\	〔認知する〕
\\	〔判断する〕
\\	一瞬誰だか分からなかった 
\\	被害者の身元はまだ分かっていない 
\\	私には瀬戸物(せともの)の良し悪しは分からない 
\\	〔判明する〕
\\	殺したのは被害者の夫と分かった 
\\	これで彼女の言ったことが本当だと分かる 
\\	選挙の結果はいつ分かりますか 
\\	真相(しんそう)がだんだん分かってきた 
\\	〔道理が分かる〕 彼はもう少しものの分かる人かと思っていた 〔常識がある〕
\\	は理解が ある)/
\\	/〔融通性がある〕
\\	もう	
\\	〔すでに〕
\\	〔今や〕
\\	もう昼だ 
\\	もう読み終わったのですか 
\\	もう彼はだめだ 
\\	/((口)) 
\\	病気はもう治りましたか 
\\	もうおいとましなくては 
\\	手紙はもうとっくに着いているはずだ 
\\	〔間もなく,やがて〕
\\	もうじき3時だ 
\\	もう母も来るでしょう 
\\	〔さらに〕
\\	もう一度調べてみましょう 
\\	もう2週間で期限が切れる 
\\	もう1杯コーヒーをください 
\\	もう一言付け加えたい 
\\	もうちょっと歩けば頂上だ 
\\	もう何も言うな 
\\	もうがまんできない 
\\	-と言って	
\\	といっても 
\\	といって だからと言って簡単に出来ることではない 
\\	今これと言ってすることがない 
\\	きれない	
\\	使いきれないほどお金を持っていた 
\\	今日中に仕事を全部はやりきれない 
\\	このホールに500人は入りきれない 
\\	いくら感謝してもしきれない 
\\	引き上げる、引き揚げる	
\\	〔引っ張り上げる〕
\\	〔難破船を〕
\\	〔遺体を〕
\\	〔網を〕
\\	沈没船(ちんぼつせん)を引き上げた 
\\	〔値段を〕
\\	運賃(うんちん)を5パーセント引き上げる 
\\	〔昇進させる〕
\\	彼を課長に引き上げた 
\\	〔撤退(てったい)する〕
\\	〔立ち去る〕
\\	軍隊は占領地(せんりょうち)から引き上げた 
\\	ここを引き揚げて東京へ行きます 
\\	[((口))
\\	〔外地から帰る〕
\\	終戦(しゅうせん)後中国から引き揚げて来ました 
\\	どうも	
\\	〔どうしても〕
\\	どうも分からない 
\\	どうも騒(さわ)がしくて眠れない 
\\	〔なんだか〕
\\	どうもあの人の言うことはおかしい 
\\	どうも事実とは思えない 
\\	どうも明日は雨らしい 
\\	どうも受け取った覚えはない 
\\	[((口)) 
\\	〔まったく〕 
\\	どうもやっかいな人だ 
\\	どうも困ったことだ 
\\	〔あいさつに〕
\\	どうもありがとう 
\\	どうもすみません 
\\	やあ先日(せんじつ)はどうも 
\\	どうもご苦労さま 〔帰るとき同僚などに〕
\\	/〔仕事をしてくれた人に〕
\\	こなす	
\\	〔細かく砕く〕
\\	〔押しつぶして〕
\\	土の塊(かたまり)をこなして細かい土にする 
\\	〔消化する〕
\\	胃は食べた物をこなす働きをする 
\\	〔うまく扱(あつか)う,なし遂(と)げる〕 
\\	彼でもこの仕事を3日でこなすことはできまい 
\\	彼はシャイロックの役をなかなかうまくこなしている 
\\	質で劣(おと)るところは量(りょう)でこなそう 
\\	あの店では商売を数でこなしている 
\\	数でこなす 
\\	やる	
\\	〔送る,行かせる〕
\\	医者を呼びにやった 
\\	〔他の場所へ移す〕 
\\	辞書をどこへやった 
\\	〔向ける〕
\\	彼の方に顔をやった 
\\	〔進ませる〕 
\\	車に乗り込むと,急いでやってくれと運転手に頼んだ 
\\	〔与える〕
\\	金魚にえさをやりましたか 
\\	〔行う〕
\\	やるべきことはちゃんとやりなさい 
\\	〔開催(かいさい)する〕
\\	〔上演する〕
\\	その展覧会(てんらんかい)はいつやるのですか 
\\	〔経営する〕
\\	〔従事する〕((文)) 
\\	父は薬屋をやっています 
\\	〔飲む,食う〕 
\\	まあ,一杯おやりなさい 
\\	〔暮らす〕 
\\	そんなわずかな収入ではやっていけない 
\\	〔「…してやる」の形で,他人のために行う〕
\\	子供たちに本を読んでやった 
\\	〔積極的にする〕
\\	絶対に優勝してやる 
\\	〔送る,行かせる〕
\\	医者を呼びにやった 
\\	秘書をやって書類を取って来(きた)させた 
\\	2人の息子を大学へやるのは大変だった 
\\	時計は直しにやってあります 
\\	娘を弁護士のところへ嫁(よめ)にやった 
\\	〔他の場所へ移す〕 辞書をどこへやった 
\\	〔向ける〕
\\	彼の方に顔をやった 
\\	入ってきた男に目をやった 
\\	〔進ませる〕 車に乗り込むと,急いでやってくれと運転手に頼んだ 
\\	〔与える〕
\\	金魚にえさをやりましたか 
\\	あの子にお駄賃(だちん)をいくらやったらいいだろうか 
\\	死んでもおまえにやるものは何もないよ 
\\	欲しかったらみんなやるよ 
\\	毎朝植木(うえき)に水をやる 
\\	〔行う〕
\\	やるべきことはちゃんとやりなさい 
\\	よくやったね 
\\	今宿題をやっている 
\\	やってみたらどうですか 
\\	どんどんやれ 
\\	!/((口)) 
\\	うん,やっているね 
\\	チェスをやろう 
\\	この役は誰がやるのか 
\\	3日もあればやれるでしょう 
\\	ドイツ語を週2回やっている 
\\	大学では数学をやりました〔専攻した〕 ((米)) 
\\	/((英)) 
\\	あいつならやりかねない 
\\	かねない 
\\	〔開催(かいさい)する〕
\\	〔上演する〕
\\	その展覧会(てんらんかい)はいつやるのですか 
\\	明日から歌舞伎座で「忠臣蔵」をやります 
\\	あの映画館で『風と共に去りぬ』をやっている 
\\	〔経営する〕
\\	〔従事する〕((文)) 
\\	父は薬屋をやっています 
\\	何の商売をやっている人か 
\\	弁護士[医者]をやっている 
\\	〔飲む,食う〕 まあ,一杯おやりなさい 
\\	酒もたばこもやりません 
\\	〔暮らす〕 そんなわずかな収入ではやっていけない 
\\	新しい同僚となんとかうまくやっているらしい 
\\	〔「…してやる」の形で,他人のために行う〕 子供たちに本を読んでやった 
\\	面白い話を聞かせてやろう 
\\	子供をいすに掛けさせてやった 
\\	そいつをたたき出してやれ 
\\	好きなようにさせてやれ 
\\	〔積極的にする〕 絶対に優勝してやる 
\\	どうか	
\\	〔なにとぞ 
\\	⇒どうぞ どうかそれは忘れてくださいませんか 
\\	〔なんとか〕 心配するな
\\	どうかなるさ 
\\	〔どうであるか〕 新しいかどうかは知りません 
\\	彼が私の提案を受け入れるかどうかは難しい問題だ 
\\	節が主語や補語である場合,
\\	のように名詞の後にくる場合,前置詞のすぐ後にくる場合などには,
\\	は
\\	で置き換えられない) 
\\	〔普通でない,おかしい〕 頭がどうかなってしまいそうだ 
\\	/((口)) 
\\	それはどうかと思う 
\\	どうかと思われる服装で歩き回っている 
\\	どうかする 
\\	〔何とかする〕 
\\	そのことはどうかしなければならない 
\\	どうかしてこの本を書き上げなければならない 
\\	どうかして会うだけでも会いたい 
\\	わずかな給料で何とかする. 
\\	〔普通でなくなる〕 このごろどうかしていますね 
\\	今日は朝からどうかしている 
\\	あの人どうかしているんじゃないか 
\\	陽気(ようき)がどうかしている 
\\	エンジンがどうかしてしまった 
\\	どうかしましたか 
\\	〔たまたま〕 どうかして遅刻することがある 
\\	どうかすると 
\\	〔ときどき〕
\\	どうかすると彼は返事をしない 
\\	どうかすると怠けがちだ 
\\	〔下手をすると〕 
\\	どうかするとあと1か月病院にいることになる 
\\	開! 
\\	ひらく 
\\	あく ひらく 
\\	〔あける,あく〕
\\	⇒あける(開ける) ふたを開く 
\\	包みを開く 
\\	扇子(せんす)を開く 
\\	本の38ページを開きなさい 
\\	戸はすぐに開いた 
\\	口を開く 
\\	比喩的にも用いる) 彼が美術について最初に私の目を開いてくれた 
\\	開かれた大学にしなければならない 
\\	土地の人はよそ者になかなか心を開かない 
\\	〔咲く〕 花が開いた 
\\	つぼみが開いた 
\\	〔差がつく〕 二人の意見はだいぶ開いている 
\\	彼と2着の差は2メートルに開いた 
\\	年の開いた姉妹 
\\	〔始める〕
\\	〔創設する〕
\\	店を開く 
\\	彼は新橋で店を開いている 
\\	彼はそこで病院を開いた 
\\	新しい事務所を開く 
\\	花屋を開く 
\\	口座を開く 
\\	〔開催する〕
\\	舞踏会を開く 
\\	討論会を開く 
\\	展覧会は明日開かれる 
\\	理事会は毎月開かれる 
\\	国会が開かれている 
\\	〔開拓する〕
\\	土地を開いて畑にする 
\\	鉱山を開く 
\\	荒野に道を開く 
\\	後進のために道を開く 
\\	その業績で彼は出世の道を開いた 
\\	〔広くなる〕 裾の開いたドレス 
\\	〔数学で,平方根の値を出す〕
\\	平方[立方]に開く 
\\	あく 
\\	⇒ひらく(開く) 戸(と)が開かない 
\\	開いた窓 
\\	店は10時に開く 
\\	幕は午後7時に開きます 
\\	開いた口がふさがらない 
\\	/((口)) 
\\	ひらける 
\\	〔広がる〕
\\	広々とした眺めが眼下に開けた 
\\	〔開化する〕
\\	〔発展する〕
\\	開けた国 
\\	この辺も近ごろずいぶん開けてきた 
\\	〔開通する〕⇒かいつう(開通) 
\\	〔人情に通じる〕 なかなか開けた人だ 〔人づき合いのよい〕
\\	[〔世なれた〕
\\	/〔物分かりのよい〕
\\	〔進んだ〕 年寄りだがなかなか開けた考えを持っている人だ 
\\	〔よい方向へ向かう〕 運が開けてきた 
\\	少しも運が開けない 
\\	あける 
\\	〔開く〕
\\	〔包装などを〕
\\	ドアを開ける 
\\	/〔鍵を使って〕
\\	ドアを乱暴に開ける 
\\	箱のふたを開ける 
\\	(本の) 25ページを開けなさい 
\\	窓を壊して開けた 
\\	窓は開けておいてください 
\\	包みを開ける 
\\	夜中の12時まで鍵は開けておく 
\\	〔穴を〕
\\	〔くり開ける〕
\\	〔きりなどで〕
\\	〔パンチで〕
\\	入る	
\\	(到着する)
\\	(加入する)
\\	(含む)
\\	(収まる)
\\	(収容する)
\\	(入手する)
\\	(点が)
\\	(押し入る)
\\	(入学する)
\\	(就職する)
\\	(収入がある)
\\	(始まる)
\\	はいる 
\\	いる はいる 
\\	〔中へ〕
\\	入っている 
\\	お入りください 
\\	[((米))
\\	入るべからず〔掲示〕 
\\	「入ってます」〔トイレで〕 
\\	窓から入る 
\\	水[プール]に入る 
\\	目にごみ[何か]が入った 
\\	入りそうです
\\	入りました
\\	ホームランです〔野球で〕 
\\	こっそり部屋に入る 
\\	こんだ会場に無理に入る 
\\	人込みの中に割って入る 
\\	傘に入りませんか 
\\	傷口にばい菌が入った 
\\	家は大通りから少し入った所にある 
\\	ここはすきま風が入る 
\\	〔到着する〕
\\	列車は10時に入る予定 
\\	港に入る 
\\	ご注文の本が入りました 
\\	〔団体・組織などに加入する〕 6歳の時学校に入った 
\\	この春都立高校に入った 
\\	彼は今度財務省に入った 
\\	彼は実業界[政界]に入った 
\\	彼は軍隊に入っている 
\\	君もトランプに入らないか 
\\	生命保険に入った 
\\	〔包含(ほうがん)される〕 
\\	この錠剤(じょうざい)にはビタミン
\\	が入っている 
\\	砂糖はたくさん入っていますか 
\\	財布には金がたくさん入っていた 
\\	私もその中に入っています 
\\	全費用が勘定に入っている 
\\	それはこの項目に入る 
\\	手紙の中に写真が入っていた 
\\	〔収容能力がある〕 この瓶はどのくらい入るかしら 
\\	この会館は1,500人入る 
\\	全員は入れません 
\\	〔居住する〕 その家は今だれも入っていない 
\\	(ある時期に) なる〕 明日から新学期に入る 
\\	梅雨に入った 
\\	夜に入って雨が降りだした 
\\	20世紀に入って間もなく 
\\	〔自分のものになる〕 月に30万円入ります 
\\	ニュースが入った 
\\	彼から電話が入った 
\\	〔獲得される〕 5回に3点入った〔野球で〕 
\\	彼に20票入った 
\\	〔利用できる状態になる〕 テレビの電源が入っている[いない] 
\\	仕事場に電話[コンピュータ]が入った 
\\	〔飲む〕 彼女はアルコールが入ると急におしゃべりになる 
\\	大分入っているらしいね 
\\	〔目や耳に〕 社内で起きたことは何でも耳に入る 
\\	目に入るものは何でも珍しかった 
\\	〔加わる〕 この絵には他人の手が入っているようだ 
\\	この報道には当局の手が入っている 
\\	先生の手が入った作文 
\\	いる 
\\	家に入る 
\\	入口 
\\	り),入江 
\\	入る 
\\	入り
\\	きにいる、手に入る 
\\	-入る 
\\	恐れ入る 
\\	-入り 
\\	箱入り(はこいり) 
\\	いれる 
\\	入れる	
\\	(物を)
\\	(人を)
\\	(入れておく)
\\	(収容(しゅうよう)する)
\\	(含(ふく)める)
\\	(雇用する)
\\	(聞き入れる)
\\	(受け入れる)
\\	(投票する)
\\	(茶を)
\\	いれる 
\\	〔中に納める〕 帽子を箱に入れる 
\\	たらいに湯を入れる 
\\	旅行かばんに衣類を入れる 
\\	小箱に宝石類を入れておく 
\\	[((英)) 
\\	スナップ写真を手紙に入れて送る 
\\	新しい図面を入れておきます〔手紙で〕 
\\	この車庫ならバスでも入れられる 
\\	コーヒーにクリームを入れますか 
\\	切手は二番目の引き出しに入れてあります 
\\	〔中へ入らせる〕 風[空気]を入れる 
\\	あの男を屋敷内に入れるな 
\\	カーテンを閉めて日を入れないようにする 
\\	客を (居間に) 入れる 
\\	〔収容する〕 この劇場には800人入れられる 
\\	この病院には150人 (の患者を) 入れられる 
\\	〔はめ込む〕 窓枠にガラスを入れる 
\\	彼女は真珠を入れた指輪をしている 
\\	この文を5ページの2行目と3行目の間に入れなさい 
\\	〔仕入れる〕
\\	〔品物を納入する〕
\\	大木病院は山田商店から石油を入れている 
\\	青のスラックスを入れていますか 
\\	このサイズしか入れていません 
\\	当店はその学校に文房具を入れています 
\\	〔雇用・採用する〕
\\	((文)) 
\\	〔弁護士など〕
\\	この会社は大学卒業生しか入れない 
\\	〔外から加える〕 疑いを入れる余地がない 
\\	この絵には後から誰かが手を入れた形跡(けいせき)がある 
\\	形跡 
\\	〔承認・容認する〕 人の願いをいれる 
\\	少数意見をいれる 
\\	彼の要請をいれる 
\\	小田さんの助言をいれた 
\\	彼の説は世にいれられない 
\\	我々は彼の案をいれた 
\\	彼には意見の異なる人をもいれる雅量(がりょう)がある 
\\	〔仲間に加える〕 彼を仲間に入れる 
\\	彼はその社会に入れられない 
\\	〔学校・病院に入れる〕
\\	私は娘をカトリックの学校に入れた 
\\	医師は彼に父親を病院に入れるように勧めた 
\\	[((米)) 
\\	米国では
\\	に
\\	をつける) 
\\	〔含める〕
\\	うちは祖父母を入れて10人の大家族だ 
\\	サービス料を入れて1泊8,000円 
\\	〔入金する〕 今月の家賃は入れましたか 
\\	娘は食費を家に入れている 
\\	彼は給料の3分の1を毎月銀行に入れる 
\\	〔投票する〕
\\	だれに入れてよいか分からない 
\\	〔念頭におく〕 考慮に入れる 
\\	交通渋滞(じゅうたい)を計算に入れて早めに出発した 
\\	〔集中する〕 彼はその企画に身を入れた 
\\	彼はその事業に身を入れて乗り出した 
\\	この学校は生徒の健康に特に力を入れている 
\\	〔茶やコーヒーを〕 茶をいれる 
\\	〔機能させる〕 冷房を入れる 
\\	機械のスイッチを入れる 
\\	〔電話などを〕 ファックスを入れる 
\\	携帯電話を入れます 
\\	〔耳に〕 お耳に入れたいことがあります 
\\	⇒みみ(耳)
\\	出掛ける	
\\	〔外出する〕
\\	〔出発する〕
\\	出掛けるのが好きだ 
\\	彼は買物に出掛けている 
\\	出張で九州に出掛けている 
\\	旅に出掛ける 
\\	散歩に出掛ける 
\\	5時に出掛ける予定だ 
\\	そろそろ出掛けなければなりません 
\\	〔訪れる〕たまにはお出掛けください 
\\	[((米))
\\	〔出ようとする〕ちょうど出掛けるところへ彼から電話が来た 
\\	どうにか	
\\	〔何とか〕 この騒ぎをどうにかしてください 
\\	どうにかしてもう少し早く来られませんか 
\\	〔やっとのことで〕 困難をどうにか (こうにか) 切り抜けた 
\\	どうにか (こうにか) 大学は出た 
\\	送る	おくる	
\\	〔発送する〕
\\	〔商品などを船・鉄道・飛行機などで〕
\\	〔金を〕
\\	人に小包を送る 
\\	我が社は世界各地に製品を送ります 
\\	部品を至急(しきゅう),車[鉄道/郵便/航空便]でお送りします 
\\	至急金を送ってくれ 
\\	新聞社には多くの投書が送られてきた 
\\	〔派遣(はけん)する〕
\\	特使[使いの者]を送る 
\\	〔見送る〕
\\	戸口(とぐち)まで送る 
\\	家[駅]までお送りしましょう 
\\	/〔車で〕
\\	〔死者をとむらう〕 
\\	死者を送る人たちの長い列が続いた 
\\	父をあの世に送った後 
\\	〔過ごす〕 
\\	彼は静かな晩年を送った 
\\	多忙(たぼう)な日を送っている 
\\	〔順々に移す〕
\\	この通達(つうたつ)を順(じゅん)に送ってください 
\\	順に 
\\	解答用紙を1枚ずつ取って後ろに送ってください 
\\	〔そちらに向ける〕 彼女にウィンクを送った 
\\	彼女の名演技にみな拍手を送った 
\\	〔送り仮名を〕 活用語尾を送る 
\\	呈する	ていする	
\\	〔進呈する〕
\\	((文)) 
\\	君の将来を思ってこそあえて苦言(くげん)を呈したい 
\\	彼らの努力に賛辞(さんじ)を呈しましょう 
\\	〔示す〕 魚市場(うおいちば)は盛況(せいきょう)を呈していた 
\\	爆発現場は惨状(さんじょう)を呈していた 
\\	その問題はますます泥沼(どろぬま)の様相を呈している 
\\	運動	
\\	〔物体の動き〕
\\	運動の法則 
\\	振り子の運動が止まった 
\\	惑星の運動 
\\	〔体を動かすこと〕
\\	〔スポーツ〕
\\	⇒スポーツ; 〔運動競技〕
\\	運動する 
\\	屋内[屋外]運動 
\\	運動不足で太ってきた 
\\	犬は毎日運動させる必要がある 
\\	〔働きかけ〕
\\	〔集団による〕
\\	〔特別の目的をもつ組織的活動〕
\\	募金(ぼきん)運動 
\\	学生[政治/労働]運動 
\\	禁酒運動 
\\	禁煙運動 
\\	女性解放運動 
\\	交通安全運動 
\\	選挙運動 
\\	就職運動 
\\	就職運動を始めた 
\\	国会議員にさかんに運動した 
\\	運動員 
\\	〔投票を勧誘する者〕
\\	運動家 〔スポーツマン〕
\\	婦人解放運動家 
\\	((口)) 
\\	運動会 〔学校などの〕
\\	〔競技会〕
\\	運動界 
\\	運動学 
\\	運動着 
\\	運動器具 
\\	運動靴 〔スポーツ用の〕
\\	〔ゴム底ズック靴〕
\\	運動場 
\\	〔主に子供の〕
\\	〔屋内の〕
\\	((口)) 
\\	運動神経 
\\	〔反射神経〕
\\	彼は運動神経が発達している[鈍い] 
\\	運動選手 
\\	運動費 
\\	〔資金〕
\\	運動量 
\\	〔物理で〕
\\	捕らわれる	とらわれる	
\\	〔捕まえられる〕
\\	〔物事のとりこになる〕 恐怖に捕らわれて人々は逃げた 
\\	感情に捕らわれて物事を判断すべきではない 
\\	外観に捕らわれて物事[人]を判断すべきでない 
\\	目先のことに捕らわれるな 
\\	目先 (近い将来)
\\	(先見)
\\	【証券】
\\	因襲(いんしゅう 因習)[先入観]に捕らわれる 
\\	過去に捕らわれるのは無意味(むいみ)だ 
\\	真(しん)の礼儀は形式に捕らわれるべきでない 
\\	だけ	だけ	
\\	〔限定,最低の限度〕
\\	これができるのは君だけだ 
\\	午前中だけそれを借りていいですか 
\\	新聞だけでなく週刊誌にも出ていた 
\\	東京だけでもそのような事件は10件あった 
\\	「ごめん」と一言だけでも言ってほしかった 
\\	言ってみただけだ 
\\	一目(いちめ)見ただけでその家が気に入った 
\\	そう言った[それを見た]だけで彼は怒った 
\\	一回見せてくれるだけでいい 
\\	やるだけはやってみなさい 
\\	〔程度〕 
\\	彼はかせいだだけ使う 
\\	稼ぐ 
\\	言うだけのことは言ったほうがよい 
\\	〔限定された分量〕 
\\	私は月に2万円だけは貯金したい 
\\	10センチだけ余(あま)る[足りない] 
\\	りんごは一つだけ余る 
\\	私より二つだけ若い 
\\	みそを500円分だけください 
\\	〔「…だけの…はある,…だけあって,…だけに」の形で,…にふさわしい〕 
\\	それは言う[読む]だけの価値がある 
\\	読むだけの価値がある 
\\	努力しただけのことはあった 
\\	だけのことはある 
\\	だけあって 
\\	さすが彼女だけあって立派にやった 
\\	だけに 
\\	苦労しただけに人生経験を積(つ)んでいる 
\\	自慢するだけあって見事だ 
\\	〔「…だけに」の形で,だから,なおさら〕 予想しなかっただけに一層うれしかった 
\\	〔比例を表して,…につれて,…するほど〕 安ければ安いだけよい 
\\	生活水準が高くなるだけ出費もかさむ 
\\	嵩む 
\\	かさむ 
\\	ところ	所・▲処】	
\\	〔場所〕
\\	〔特定の〕
\\	〔現場〕
\\	〔所在地〕
\\	なんて気持ちのいい所だ 
\\	遠足に好適の所が見つかった 
\\	事故のあった所はここですか 
\\	首府のある所は地図の上に赤い丸がついている 
\\	もう少し近い所から撃ってみたらどうだ 
\\	元あった所に返しなさい 
\\	駅からバスでどれぐらいの所にありますか 
\\	彼はドアの所に歩いて行った 
\\	あらゆる所にポスターが貼ってあった 
\\	通勤に便利な所に住みたい 
\\	〔地域〕⇒ちほう(地方),ちいき(地域) 所によって習慣が違う 
\\	みかんがたくさん取れる所 
\\	今日は所により雨でしょう 
\\	〔家〕 山田さんの所へ預けて来た 
\\	私の所はこの近くです 
\\	今友達の所に泊まっています 
\\	〔住所〕 石田さんの所を教えてください 
\\	〔余地〕
\\	皆が座る所はない 
\\	このページはもう書く所が無い 
\\	〔部分〕
\\	初めのところを読んだだけです 
\\	壊れたところを修理した 
\\	〔点〕 それが彼のよい所なんだ 
\\	そこが彼が君に勝(まさ)る所なんだ 
\\	彼にはどこかこっけいな所がある 
\\	〔ちょうどその時,場合〕 手紙を書くところだ 
\\	皆そろったところで写真を撮りますよ 
\\	ちょうど顔を洗っていたところへ電話が鳴った 
\\	もう少しで忘れるところだった 
\\	笑っているどころではありません 
\\	危ないところを助かった 
\\	今のところは健康です 
\\	このところ暑い日が続いている 
\\	よいところへ来た,上がりなさい 
\\	今日のところは勘弁してやる 
\\	〔範囲〕 私の知っているところではよい学校だ 
\\	見たところ何でもなかったが切ってみると中が腐っていた 
\\	彼の話してくれたのは大体そんなところだ 
\\	せいぜいのところで80点かな 
\\	〔…すること〕 聞くところによると,彼は再選を目指すそうだ 
\\	彼の言うところではそれはうそだ 
\\	何も言わなかったところを見るとまだ知らないらしい 
\\	やりたくないというのが本当のところだ 
\\	〔立場〕
\\	攻守(こうしゅ)ところを変えた 
\\	攻守 
\\	所構(ところかま)わず 
\\	彼は所構わず寝転がる 
\\	壁に所構わず彼女の写真をピンでとめた 
\\	所変われば品変わる 
\\	所へ[に]持ってきて 失業したところへ持ってきて,妻が病気になってしまった 
\\	所を得る[得ない] 
\\	彼はここに来てから所を得たように元気になった 
\\	家具はどれもこれも所を得ていなかった 
\\	移す	うつす	
\\	〔場所を変える〕
\\	重心を右足に移す 
\\	花瓶をテーブルから移す 
\\	住まいを移したばかりです 
\\	住まい 
\\	菊(きく)を苗床(なえどく)から鉢(はち)に移した 
\\	酒を徳利(とっくり)へ移す 
\\	〔転任・転校させる〕 山田は本社に移された 
\\	〔心を他に向ける〕 彼は新入りに注意を移した 
\\	他の男に心を移したらしい 
\\	〔移管(いかん)する〕 
\\	その事件は高等裁判所へ移された 
\\	〔状態などを変える〕 約束を行動に移してほしい 
\\	計画を実行に移すべきだ 
\\	〔病気などを伝染(でんせん)させる〕 赤ん坊に風邪を移さないように気をつけている 
\\	肝	きも	
\\	〔肝臓〕
\\	〔家畜の内臓〕
\\	((口)) 
\\	〔鳥の〕
\\	〔勇気〕
\\	((口)) 
\\	肝が据(す)わる 若いのに彼は肝が据わっている 
\\	肝が小さい 
\\	肝が太い 
\\	((口)) 
\\	肝に銘じる 
\\	私は先生の忠告(ちゅうこく)を肝に銘じた 
\\	肝を据える 
\\	肝を据えて難問に取り組んだ 
\\	肝を潰(つぶ)す 突然の大音響(だいおんきょう)に肝をつぶした 
\\	肝を冷やす あわや墜落(ついらく)かと肝を冷やした 
\\	据わる	すわる	
\\	据える-
\\	-据わる 
\\	〔しっかりと位置を占める〕 赤ちゃんの首が据わった 
\\	〔じっとして動かない〕 彼は飲みすぎだ
\\	目が据わっているよ 
\\	〔気持ちが落ち着く〕 彼は度胸(どきょう)が据わっている 
\\	度胸 
\\	彼は腰が据わらないのが問題なんだ 
\\	ぱらぱら	ぱらぱら	
\\	〔雨などが降る音〕 雨がぱらぱらと降り出した 
\\	あられがぱらぱら降っている 
\\	〔手早くめくる様子〕 
\\	ぱらぱらとページをめくっていった 
\\	〔乱(みだ)れ散る様子〕 ぱらぱらと小銭が道路に転(ころ)がり落ちた 
\\	鳥肉にぱらぱらと塩こしょうを振り掛ける 
\\	〔まばらな様子〕 
\\	彼はぱらぱらの観客(かんきゃく)に向かって熱心に話し掛けた 
\\	疎通	そつう	
\\	意思の疎通 
\\	二人は意思の疎通を欠いた 
\\	他の会員との意思の疎通を図った 
\\	ひしひし		
\\	〔緩(ゆる)みなく迫(せま)る様子〕 敵軍が我が陣地(じんち)にひしひしと押(お)し寄(よ)せた 
\\	私はひしひしと迫(せま)ってくる危険を感じた 
\\	〔強く感じる様子〕 その言葉はひしひしと胸にこたえた 
\\	健康のありがたみがひしひしと身にしみた 
\\	みにしみる ー
\\	べたべた		
\\	〔粘(ねば)りつく様子〕 飴(あめ)が手にべたべたくっついた 
\\	[((口))
\\	汗でシャツがべたべた体についていた 
\\	〔たくさん塗る・張る様子〕 
\\	壁にペンキをべたべた塗る 
\\	顔におしろいをべたべた塗っていた 
\\	ポスターを壁にべたべた張った 
\\	〔人にまつわりつく様子〕 
\\	ベスはいつも母にべたべたしていた 
\\	彼女は人前でもボーイフレンドとべたべたする 
\\	ごてごて		
\\	〔濃厚な様子〕⇒こてこて ごてごてと化粧する 
\\	ごてごてした装飾 
\\	安物(やすもの)のアクセサリーをごてごてつけている 
\\	⇒くどくど
\\	こてこて		
\\	〔濃密に,厚く〕
\\	〔あり余るほど〕
\\	⇒こってり 彼女はマフィンにマーマレードをこてこてに塗った 
\\	そこ		
\\	〔その場所〕 そこに 
\\	本はそこに置いていってください 
\\	そこのところをもう一度読んでみなさい 
\\	ところ 
\\	〔部分〕
\\	初めのところを読んだだけです 
\\	そこの窓を開けてください 
\\	そこから急な坂になる 
\\	〔その時〕 そこへ彼が飛び込んできた 
\\	そこで幕が下りた 
\\	夕立が来たと思ったらそこへ運よくタクシーが通り掛かった 
\\	夕立(ゆうだち) 
\\	〔その段階・範囲〕 そこまで歩けるか 
\\	そこまでは賛成(さんせい)だ 
\\	そこまで言うものではない 
\\	そこまでは理解できない 
\\	〔その点〕 そこが大事な点だ 
\\	そこをなんとか頼む 
\\	そこへ持ってきて 〔さらにその上〕
\\	〔悪いことに〕
\\	そこへ持ってきて雪まで降りだした 
\\	そこへ行くと 〔その点から考えると〕
\\	にかけては		
\\	〔…に関しては〕 碁にかけては彼は一流だ 
\\	金にかけては心配はいらない 
\\	起こす	おこす	
\\	(目を覚まさせる)
\\	(立たせる)
\\	(引き起こす)
\\	(設立する)
\\	(開始する)
\\	〔横になったものを立てる〕
\\	彼はベッドで上体を起こした 
\\	倒れた椅子を起こした 
\\	転んだ子供を起こしてやった 
\\	〔目を覚まさせる〕
\\	明朝(みょうちょう)7時に起こしてください 
\\	/〔ホテルなどで〕
\\	地震のために起こされた 
\\	〔土などを掘り返す〕 
\\	畑を起こす ((米))
\\	/((英))
\\	土を起こす 
\\	敷石(しきいし)を起こす 
\\	〔設立する〕
\\	学校[会社]を起こす 
\\	組合[協会]を起こす 
\\	〔始める〕
\\	彼女は知事を相手に訴訟(そしょう)を起こした 
\\	会社が新しい事業を起こした 
\\	〔引き起こす〕
\\	社会不安が暴動を起こした 
\\	君の不注意が事故を起こしたのだ 
\\	犯罪事件を起こす 
\\	車はエンストを起こした 
\\	〔病気を〕 
\\	彼はひきつけ[心臓発作/腹痛(ふくつう)]を起こした 
\\	〔生(しょう)じさせる〕 彼はかんしゃくを起こした 
\\	また例の気まぐれを起こしている 
\\	それが彼に疑いの念を起こさせた 
\\	〔熱・電気を発生する〕 
\\	(静) 電気を起こす 
\\	熱を起こす 
\\	〔テープの音声などを文字にする〕 
\\	会議の録音テープを起こす 
\\	ぐんと	ぐんと	
\\	〔力を入れる様子〕 ぐんと踏ん張る 
\\	ぐんと腕を伸ばして背筋(せすじ)を伸ばしましょう 
\\	重い扉(とびら)をぐんと押(お)した 
\\	〔一段と〕 
\\	ちょっと見ない間にぐんと大きくなったね 
\\	彼の発音は前よりぐんとよくなった 
\\	遣う	つかう	
\\	〔金,時間などを費やす〕
\\	〔消費する〕
\\	ほとんどの金を書物(しょもつ)に遣ってしまう 
\\	時間を有効に遣う 
\\	金銭をつまらないことに遣う 
\\	毎月お米はどのくらい遣いますか 
\\	その仕事は体力を遣う 
\\	〔心をわずらわせる〕 煩(わずら)わせる 
\\	これはずいぶん神経を遣う仕事だ 
\\	あまり気を遣うな 
\\	彼女はしゅうとめにとても気を遣っている 
\\	〔操る〕 人形を遣う 
\\	大穴	おおあな	
\\	〔大きな穴〕
\\	〔大きな損失〕
\\	彼は会社の経理に大穴をあけた 
\\	妻の大病で家計に大穴があいた 
\\	〔競馬などの大番狂わせ〕 レースで大穴をねらう 
\\	済む	すむ	
\\	〔終わる〕
\\	〔借金,勘定が〕
\\	話は済んだ 
\\	試験が済んだ 
\\	手術が済んでから2日間は何も食べさせてくれない 
\\	準備は1日で済んだ 
\\	済んだことをくよくよするな ((諺)) 
\\	済んだことは仕方がない ((諺)) 
\\	ローンは今月で済む 
\\	〔間に合う〕 11月はまだオーバーなしで済む 
\\	君のノートを借りないで済みそうだ 
\\	本は彼に借りれば買わなくて済む 
\\	外国から注文するのに
\\	メールを使えば出かけずに済む 
\\	社長が出席しなくても済むだろう 
\\	〔解決できる〕 これは金で済む問題ではない 
\\	示談(じだん)で済ませられないのか 
\\	示談 
\\	罰金(ばっきん)で済んだ 
\\	軽いけがで済んだ 
\\	謝れば済むと思ったら大間違いだぞ 
\\	気が済むまで殴(なぐ)ってくれ 
\\	〔申し訳ない〕⇒すまない(済まない),すみません(済みません)
\\	大詰め	おおづめ	
\\	〔芝居の最後〕
\\	〔物事の最終〕
\\	交渉は大詰めに近付いている 
\\	踏む	ふむ	
\\	(手続きを)
\\	(評価する)
\\	〔足で〕
\\	〔踏みつける〕
\\	人の足を踏む 
\\	ペダルを踏む 
\\	〔現場に行く〕 イギリスの地を踏む 
\\	イタリアの地を踏んだことがありますか 
\\	〔経験する〕
\\	彼のフランス語は本場を踏んでいるから堂(どう)に入っている 
\\	堂にいる 
\\	場数を踏む ⇒ばかず(場数) 
\\	〔手続きなどを経る〕
\\	正規(せいき)の課程を踏む 
\\	〔評価する〕
\\	これをいくらに踏むか 
\\	これはどう安く踏んでも1万円はする品だ 
\\	彼が簡単に許してくれるものと踏んでいた 
\\	〔韻(いん)を〕 韻を踏む 
\\	見張る	みはる	
\\	〔警戒して守る〕
\\	〔警察が場所などを〕((口)) 
\\	店員が万引きを見張っている
\\	2人の男が門前で見張っていた
\\	[((文))
\\	警官が容疑者の家を見張っている
\\	〔目を見開く〕⇒みひらく(見開く)目を見張るような光景
\\	(驚きの) 目を見張って
\\	張り込む	はりこむ	
\\	〔犯人などを見張る〕
\\	〔場所を〕
\\	犯人は張り込んでいた警官を出し抜いた
\\	3人の刑事が向かいのアパートに張り込んでいた
\\	〔大金を掛ける〕 スイス製の時計を張り込んだ
\\	チップを張り込む
\\	繰る	くる	
\\	〔巻き取る,たぐる〕
\\	〔紡ぐ〕
\\	手早く釣糸(つりいと)を繰った 
\\	〔順におくる〕 雨戸を繰る 
\\	じゅずを操る 
\\	〔順に数える〕 それがいつだったか日を繰って確かめた 
\\	〔紙などをめくる〕 ページを繰る 
\\	辞書を繰る 
\\	その書物(しょもつ)を繰って230ページを開けた 
\\	折から	おりから	
\\	〔ちょうどその時〕 折からの雨で野球の試合は中止になった 
\\	折から彼が来合(きあ)わせた 
\\	〔…の時であるから〕 寒さの折からご自愛専一(じあいせんいつ)に 
\\	置く	おく	
\\	(横たえる)
\\	(下ろす)
\\	(残す)
\\	(放置する)
\\	(据え付ける)
\\	(立てる)
\\	(設立する)
\\	(保存する)
\\	(任命する)
\\	(雇う)
\\	(飼う)
\\	(預ける)
\\	(質屋に)
\\	〔物を据(す)える〕
\\	どこかこれを置くよい場所はないか 
\\	本を元の棚に置いておきなさい 
\\	花瓶を卓(たく)のまん中に置いた 
\\	〔置いておく〕
\\	車を置いておく場所がない 
\\	ここに置いといて 
\\	〔残しておく〕
\\	家族を東京に置いて仙台に来ている 
\\	〔仕入れて売る〕 男物(おとこもの)は置いておりません 
\\	あの店は輸入品をたくさん置いている 
\\	〔設置する〕 丸の内に事務所を置くことにしました 
\\	世界各地に支店を置いている 
\\	〔任命する〕
\\	〔雇う〕
\\	監査役(かんさやく)を置くことにした 
\\	新しい秘書を置いた 
\\	その国は東京にスパイを置いているといううわさだ 
\\	〔家に入れる〕 下宿人(げしゅくにん)を二人置いている 
\\	/((米)) 
\\	伯父の家に置いてもらっています 
\\	〔預ける〕
\\	〔質屋に預ける〕
\\	((俗)) 
\\	そのコート,クロークに置いてきたら 
\\	〔書くのをやめる〕 
\\	ペンを置いてください 
\\	〔保存する〕
\\	この種のソーセージは長く置けます 
\\	〔隔(へだ)てる〕 
\\	1メートル置いて隣に家が建った 
\\	2軒置いて隣の家 
\\	それから1日置いてまた彼と出会った 
\\	〔ある状態にとどめる〕 日本は今難しい立場に置かれている 
\\	そのころは絶望的な状況に置かれていた 
\\	君をいつまでもそんな状況に置いておくわけにはいかない 
\\	〔「…しておく」の形で〕 
\\	〔前もってする〕 お部屋を掃除しておきました 
\\	彼が席を取っておいてくれた 〔確保してくれた〕
\\	/〔予約してくれた〕
\\	〔そのままの状態にする〕
\\	〔させておく〕
\\	ドアを開けっ放(はな)しにしておく 
\\	そのままにしておけ 
\\	/〔触れるな〕
\\	混ぜたら一晩そのままにしておきなさい 
\\	言いたいことを言わせておいた 
\\	彼を長い間待たせておいた 
\\	部屋の温度を一年中一定にしておく必要がある 
\\	〔さしあたりそうする〕 聞かなかったことにしておこう 
\\	おく	【▲措く】	
\\	〔やめる〕 それはさておいて ⇒さておき そのようなことを言えば彼らの失笑(しっしょう)を買わずにはおかないだろう 
\\	〔除く〕 頼(たよ)れるのは彼をおいてほかにない 
\\	これをおいてほかに方法はなかった 
\\	仕掛け	しかけ	
\\	〔やりかけ〕 仕事をしかけたまま出掛けてしまった 
\\	〔装置〕
\\	((文)) 
\\	彼はねずみ取りのうまい仕掛けを拵(こしら)えた 
\\	電話の会話をテープにとる仕掛け 
\\	テープの速度を調節(ちょうせつ)する仕掛け 
\\	この時計は電気仕掛けです 
\\	種も仕掛けもございません 
\\	〔仕組み〕 貧乏人はますます貧乏になる仕掛けになっている 
\\	仕掛け人 
\\	仕掛け花火 
\\	はらむ	
\\	孕む】	
\\	〔身ごもる〕
\\	その女は子をはらんでいた 
\\	雌牛(めうし)がはらんだ 
\\	〔膨(ふく)らむ〕 
\\	帆(ほ)が風をはらんだ 
\\	〔含(ふく)む〕 あらしをはらんだ情勢(じょうせい) 
\\	政局は危機をはらんでいた 
\\	会場の空気は殺気(さっき)をはらんでいた 
\\	殺気 
\\	挨拶	あいさつ	
\\	〔会釈〕
\\	〔敬礼〕
\\	〔軍人の〕
\\	あいさつを交わす 
\\	/〔頭を下げて〕
\\	/〔軽く会釈する〕
\\	帽子をとってあいさつする 
\\	帽子に手をやってあいさつする 
\\	手を振って人にあいさつする 
\\	笑顔であいさつする 
\\	彼はあいさつ抜きでいきなり用件を切り出した 
\\	〔会合で述べる言葉〕 卒業式であいさつをした 
\\	[((米)) 
\\	都知事(とちじ)に一言歓迎(かんげい)のごあいさつを申し上げます 
\\	〔返答・わびなどの言葉〕 迷惑(めいわく)をかけておいてなんのあいさつもない 
\\	〔儀礼的な言葉〕 新年のあいさつを送る 
\\	季節のあいさつ〔特にクリスマス・新年の〕 
\\	ほんのあいさつ代わりです 
\\	あいさつ回りをする 
\\	明日引っ越すのでちょっと近所にあいさつして来ます 
\\	〔失礼な言葉〕 それはごあいさつだね 
\\	〔仕返し〕 
\\	刑務所を出たらあいさつに行くから覚えていろ 
\\	挨拶状 〔暑中見舞いなどの〕
\\	移転のあいさつ状 
\\	得たり	えたり	
\\	ーーー 彼は得たりとばかりに自分を売り込んだ 
\\	ばかりに 
\\	得たりや応(おう)とばかり彼は1歩前に進み出た 
\\	応 
\\	得たり顔で 
\\	彼はその申し出に得たり賢しと飛びついた 
\\	取る	とる	
\\	〔手に持つ〕
\\	子供の手を取った 
\\	お手に取ってご覧ください 
\\	その男が私の胸ぐらを取った 
\\	〔取って渡す〕 紙を取ってくださいますか 
\\	その辞書を取ってください 
\\	からしを取ってください 
\\	〔手に持って使う〕 彼の代りに私がハンドルを取った 
\\	ここ数年絵筆を取ったことがない 
\\	剣を取らせたら右に出る者はない 
\\	〔除く〕
\\	〔脱ぐ〕
\\	〔校正で〕
\\	りんごの腐ったところを取る 
\\	スカートの汚れを取る 
\\	箱のふたを取る 
\\	庭の雑草を取る 
\\	帽子を取る 
\\	サングラスを取ったら可愛い女の子だった 
\\	痛み[悩み]を取る 
\\	〔奪う〕
\\	〔盗む〕
\\	新しいスーパーに客を取られた 
\\	命を取る 
\\	彼女のハンドバッグを取って逃げた 
\\	金を取られた 
\\	〔権力を握る〕 天下を取る ⇒てんか(天下) 
\\	政権を取る 
\\	〔採集する〕
\\	薪を取りに行く 
\\	その貝殻はどこで取ったの 
\\	木からりんごを取る 
\\	ビートから砂糖を取る 
\\	〔受ける,得る〕
\\	2日間の暇を取る 
\\	わいろを取る 
\\	博士号を取る 
\\	数学で80点を取った 
\\	[((英))80 
\\	彼はいつも良い点を取る ((米)) 
\\	[((英)) 
\\	1等賞を取る 
\\	高給を取る 
\\	注文を取る 
\\	情報を取るばかりでちっとも出さない 
\\	〔食物・睡眠などを取る〕 もっと滋養物を取りなさい 
\\	遠慮なく取って召し上がってください 
\\	君は何時間睡眠を取るのかい 
\\	〔選ぶ〕 好きな方を取って構(かま)わないよ 
\\	私なら安いテーブルより高い方を取る 
\\	どちらを取ろうか迷っている 
\\	〔時間・空間を占める〕 これはひどく時間を取る 
\\	グランドピアノはひどく場所を取る 
\\	新築の家には書斎を広く取った 
\\	〔料金などを徴収する〕 食費として月6万円取る 
\\	家を買うと税金を取られる 
\\	彼は駐車違反で罰金を取られた 
\\	あの人から会費が取れない 
\\	金を貸せば利息が取れる 
\\	〔記録などをつける〕 講演のメモ[ノート]を取る 
\\	彼の話をテープに取ろう 
\\	音楽番組をビデオに取った 
\\	〔理解する〕 彼はその意味を文字どおりに取った 
\\	私の皮肉を褒められたと取った 
\\	私のやる事すべてどうしてそんなに悪く取るの 
\\	私の言ったことを人がどう取ろうと勝手だが君にはそんな風に取られたくない 
\\	〔確保する〕⇒とっておく(取って置く) あなたの席を取ってあります 
\\	宿を取る ⇒やど(宿) 旅行の費用は別に取ってある 
\\	明日は君[テニス]のために取ってあるんだ 
\\	〔保存する〕 君が取っといてくれ 
\\	子供のころ描いた絵がまだ取ってある 
\\	〔購読する〕
\\	うちは読売を取っています 
\\	美術雑誌を取る 
\\	〔注文する〕 私は何を取ろうかしら 
\\	レストランでカレーライスを取った 
\\	〔注文して持ってこさせる〕 近所のすし屋からすしを取ろう 
\\	〔年齢を重ねる〕 年を取る ⇒とし(年) 当年取って25歳 
\\	〔引き受ける〕 責任を取って辞職した 
\\	〔手段を講じる〕 騒音に対してもっと有効な措置(そち)は取れないものか 
\\	思い切った手段を取ってはどうだ 
\\	〔計る,数える〕 脈を取る 
\\	〔入れる〕 天窓から明かりを取る 
\\	〔敷く〕 床を取る 
\\	取るに足りない 取るに足りないやつ 
\\	取るに足りないことに悩む 
\\	その説は取るに足りない 
\\	取る物も取りあえず 取る物も取りあえず病院に駆けつけた 
\\	取って付けたような 取って付けたようなお世辞 
\\	ちょっと	ちょっと	
\\	〔わずか〕
\\	もうちょっと右へ押してみてくれ 
\\	000円ちょっとで買えます 
\\	ちょっとは人のことを考えろ 
\\	これをちょっと見てくれ 
\\	ちょっと拝見 
\\	彼は写真をちょっと見てうなずいた 
\\	ちょっとそこまでご一緒しましょう 
\\	ちょっとお尋ねしたいのですが 
\\	ちょっと言いにくいのですが… 
\\	ちょっと気掛りなので伺います 
\\	〔しばらく〕
\\	ほんのちょっとの間です 
\\	ちょっとの間に村の様子がすっかり変わった 
\\	ちょっと待ってください 
\\	ちょっと忘れました 
\\	ちょっとお目に掛かりたいのですが 
\\	もうちょっとでバスに遅れるところだった 
\\	〔ほんのついでに〕 ちょっとパーティーに顔を出してみないか 
\\	ちょっとお寄りになりませんか 
\\	〔かなり〕 最後の問題はちょっと難しかった 
\\	あのバーにはちょっとかわいい女の子がいる 
\\	〔呼び掛け〕((米)) 
\\	((英)) 
\\	ちょっとすみません
\\	今何時でしょう 
\\	〔打ち消しを伴って,容易には〕 彼がどこにいるか,ちょっと見当もつかない 
\\	そのような条件ではちょっとお引き受けできません 
\\	ちょっとやそっとの 
\\	彼はちょっとやそっとのことでは驚かない 
\\	ずく	
\\	尽く】	
\\	力[腕]ずくで 
\\	親切ずくで 
\\	金ずくで 
\\	欲得(よくとく)ずくで 
\\	頼みを力ずくでも承知させよう 
\\	私たちは納得ずくで離婚した 
\\	寝起き	ねおき	
\\	〔目覚め〕 この子はいつも寝起きが悪い[よい] 〔起きられない[起きられる]〕
\\	/〔機嫌が悪い[よい]〕
\\	〔居住〕 彼女と同じ家に寝起きしている 
\\	東京では叔母の家に寝起きしていた 
\\	代	だい	
\\	〔継承の順位,その期間〕
\\	〔治世〕
\\	親[子]の代 
\\	何代も続いている名家 
\\	5代目の大統領 
\\	ルーズベルトは何代目の大統領か 
\\	あの店は今7代目の主人がやっている 
\\	その店は息子に代が変わった 
\\	田中家は彼の祖父の代に没落(ぼつらく)した 
\\	ジェームズ一世の代に 
\\	と読む) 
\\	〔年齢の10歳単位〕 10代の子供 
\\	13歳から19歳までを言う) 彼女は20代[10代]です 
\\	彼は50代の終わり[30代の半ば]です 
\\	〔時代の10年単位〕 1960年代に 
\\	1980年代の初め[半ば]に 
\\	〔代わりの人〕 剣道道場の師範代 
\\	〔代金〕 部屋代 
\\	郵便代 
\\	お代はいりません 
\\	お代は後で結構です 
\\	〔地質学上の年代〕
\\	古生代 
\\	揚がる	あがる	
\\	〔空中に高く上る〕 夜空に花火が揚がった 
\\	たくさんの凧(たこ)が揚がっている 
\\	〔揚物(あげもの)ができる〕 えびがからっと揚がった 
\\	天ぷらが揚がったらすぐに食べよう 
\\	〔高まる,強くなる〕 どっと笑い声が揚がった 
\\	計画(けいかく)に対して反対の声が揚がった 
\\	〔上陸する〕
\\	水夫(すいふ)たちが船から揚がってきた 
\\	乗客たちは陸(りく)に揚がった 
\\	先方	せんぽう	
\\	先方の言い分を聞きましょう 
\\	こちらの決定を先方に伝えました 
\\	営み	いとなみ	
\\	〔仕事〕
\\	日々の営みに忙しい 
\\	〔準備〕
\\	森のけものたちは冬の営みを急いでいる 
\\	〔作用〕
\\	自然の営みは不可思議である 
\\	在籍	ざいせき	その学校には外国人学生が多数在籍している 
\\	その学生はもう在籍していません 
\\	在籍者 〔団体などの〕
\\	〔学生〕
\\	その会の在籍者は2千名だ 
\\	在籍証明書 〔学校の〕
\\	〔団体の〕
\\	在籍専従者 ⇒せんじゅう(専従)
\\	浮かす	うかす	
\\	〔水面などに〕
\\	⇒うかべる(浮かべる)
\\	〔不安定な状態にする〕 自分の名前が呼ばれたと思って腰を浮かせた 
\\	〔余りを出す〕 僕は出張旅費(りょひ)を浮かせて彼女へのプレゼントを買った 
\\	やりくりして10万円浮かせた 
\\	やりくり-やり繰り 
\\	いとう	
\\	厭う】	
\\	〔いやがる〕 どんな仕事もいといません 
\\	寒風(かんぷう)をいとわず彼は毎朝ジョギングをする 
\\	それを手に入れるためなら労苦(ろうく)[出費]をいとわない
\\	労苦 
\\	苦労 
\\	〔いたわる〕 
\\	どうぞお体をおいといください 
\\	働く	はたらく	
\\	〔仕事をする〕
\\	((英)) 
\\	は長くて激しい仕事・肉体的労働をすること) 金[生計]のために働く 
\\	兄は工場で働いている[に勤めている] 
\\	働きすぎて体をこわした 
\\	〔活動する,作用する〕 頭のよく働く人 
\\	朝は頭がよく働かない 
\\	この場合この単語は動詞として働く 
\\	〔悪いことなどを行う〕 盗みを働く 
\\	彼は数々の悪事を働いた 
\\	彼は隣人に乱暴を働いた 
\\	[((口)) 
\\	講じる	こうじる	
\\	〔講義をする〕 日本史を講じる 
\\	〔手段を取る〕 あらゆる手段を講じたが駄目だった 
\\	政府は何の手段も講じていない 
\\	その事態に対して善後策(ぜんごさく)を講じたのはスミス氏であった 
\\	〔交渉する〕 和を講じる 
\\	現役	げんえき	
\\	〔軍務の〕 現役の軍人 
\\	現役に服(ふく)する[を退く] 
\\	〔現職の〕 現役の政治家 
\\	現役の投手 
\\	父はまだ現役で働いている 
\\	〔在学中〕 息子は現役で大学に入った 
\\	かど【▲廉】	かど	〔理由〕
\\	〔罪科〕
\\	〔容疑〕
\\	職務不履行(ふりこう)のかどをもって退職を命ずる 
\\	文書偽造(ぎぞう)のかどで取り調べられた 
\\	殺人のかどで逮捕された 
\\	能書き	のうがき	
\\	〔薬の効能書〕
\\	〔薬の箱などに書かれた〕
\\	能書きには頭痛に効(き)くと書いてある 
\\	〔宣伝文句〕
\\	〔大げさな〕
\\	能書き程(ほど)ではない 
\\	彼は能書きを並べたてた 
\\	/((文)) 
\\	ごとき	
\\	▲如き】	
\\	秀吉の如き英雄はたんと無い 
\\	《比況(ひきょう)の助動詞「ごとし」の連体形》活用語の連体形、体言、助詞「の」「が」に付いて、比喩 (ひゆ) ・例示を表す。…のような。 「彼が言うごとき結果になる」 「彼のごとき人物はこの世に二人とはいない」 「眠るがごとき大往生(だいおうじょう)」 [補説]現代では文章語的表現、または改まった表現をする場合に用いられる。 
\\	ごとき 
\\	様 
\\	行われる	おこなわれる	
\\	〔実行される〕 それは彼の同意の下で行われた 
\\	住民の間に詳しい調査が行われた 
\\	彼の家の家宅捜索が行われた 
\\	ここで天下分け目の戦いが行われた 
\\	天下分け目(てんかわけめ) 
\\	手術は朝のうちに行われた 
\\	結婚式は日曜日に行われた 
\\	〔広くはやる〕 その儀式はこの国では今でも行われている 
\\	思潮	しちょう	
\\	現代思潮 
\\	時代思潮 
\\	現代文芸思潮 
\\	行く手	ゆくて	
\\	〔行く方向〕 行く手に熊が現れた 
\\	行く手を遮(さえぎ)るものは今や何もない 
\\	行く手に人家(じんか)の屋根が見えてきた 
\\	〔今後〕 行く手には多くの困難が待ち受けているだろう 
\\	演習	えんしゅう	
\\	〔継続的で規則正しい〕
\\	〔反復して行う〕
\\	射撃(しゃげき)の演習 
\\	英作文演習 
\\	予行演習 
\\	〔大学での〕
\\	高等代数学の演習 
\\	〔軍隊での〕((米)) 
\\	((英)) 
\\	演習を行う 
\\	防空演習 
\\	形容詞(相当語句) 
\\	演習 環太平洋合同演習[リムパック] 
\\	《米海軍主催の》 机上[図上](軍事)演習 
\\	軍事演習 
\\	合同(軍事)演習 
\\	災害救援演習 
\\	実弾演習 
\\	対潜水艦演習 
\\	大規模演習 
\\	野外演習 
\\	予行演習 
\\	演習 
\\	名詞 演習区域 
\\	演習場 
\\	ぶつ	【▲打つ・▲撃つ】	
\\	〔なぐる〕 子供の頭をぶってはいけない 
\\	平手(ひらて)で頬をぶたれた 
\\	〔演説を〕 
\\	一席(いっせき)ぶつ 
\\	彼は得意の演説をぶった 
\\	社長は社員の士気を鼓舞(こぶ)するために一席ぶった. 
\\	遺憾	いかん	遺憾な 
\\	遺憾に思う 
\\	遺憾ながらこの番組は打ち切られます 
\\	従業員の過失に対し心から遺憾の意を表します 
\\	彼が受賞しなかったことは遺憾です 
\\	本会議に於(お)ける彼の欠席を遺憾とする 
\\	は残念という気持ちだけでなく非難をも意味する) 遺憾なく 
\\	〔十分に〕
\\	準備万端(じゅんびばんたん)遺憾なく整(ととの)った 
\\	彼の特質はこの詩に遺憾なく発揮(はっき)されている 
\\	発動	はつどう	
\\	〔動かすこと〕 動する 
\\	〔権限の行使〕 大統領は強権を発動してストを中止させた 
\\	特別条例が発動された 
\\	さや	
\\	鞘】	
\\	〔刀の〕
\\	〔小刀の〕
\\	〔鉛筆などの〕
\\	刀の鞘を払う 
\\	刀を鞘に収める 
\\	〔差額〕 
\\	鞘を取る 
\\	いくらの鞘になりますか 
\\	折る	おる	
\\	〔曲げて重ねる〕
\\	紙を二つに折る 
\\	手紙を折る 
\\	ページを折る 
\\	折り鶴(つる)を折る 
\\	〔強く曲げて切り離す〕
\\	桜の枝を折る 
\\	木の枝を折るな 
\\	右腕を折った 
\\	〔曲げる〕
\\	ひざを折る 
\\	ひざを折って許(ゆる)しを乞(こ)う 
\\	1から10まで指を折って数える 
\\	胎盤	たいばん	
\\	((複-
\\	胎盤の 
\\	胎盤形成 
\\	胎盤剥離(はくり) 
\\	要項	ようこう	
\\	募集(ぼしゅう)要項 
\\	がる	がる	
\\	私が早く去るので彼らは残念がった。 
\\	〔思う〕 うれしがる[寂しがる] 
\\	哀れがる 
\\	彼女は友達が持っているものを何でも欲しがる 
\\	[((文))
\\	彼らは閉めきった部屋で暑がっていた 
\\	〔様子をする〕 強がる 
\\	あわや		
\\	あわやと思う間もなく 
\\	あわやおぼれるところだった 
\\	あわやというところで助けてもらった 
\\	/((口)) 
\\	ひょうきん	
\\	剽▲軽】	ひょうきんな 
\\	ひょうきんな仕草(しぐさ) 
\\	ひょうきんないたずら 
\\	ひょうきんなことをして[言って]よく人を笑わす 
\\	剽軽者 〔陽気で茶目な〕
\\	〔ユーモラスな〕
\\	〔おどける〕
\\	〔機知に富んだ〕
\\	体する	たいする	
\\	大使は政府の訓令を体して新提案をした 
\\	[動サ変][文]たい・す[サ変]命令や教えなどを心にとどめて守るようにする。「遺訓を―・する」「首脳部の意を―・して交渉にあたる」
\\	切り出す	きりだす	
\\	(石を)
\\	(木材を)
\\	(話を)
\\	〔切って運び出す〕 山から木材を切り出す 
\\	山から切り出したばかりの材木 
\\	石材(せきざい)を切り出す 
\\	〔話し始める〕 その客は座るなり用件を切り出した 
\\	だれも幹部(かんぶ)に思いきってその話題を切り出すものはいなかった 
\\	切り出し 
\\	〔木材の〕
\\	〔小刀〕
\\	〔言い出し 
\\	何事	なにごと	
\\	〔どんなこと〕
\\	何事ですか 
\\	/((口))
\\	この静けさはいったい何事だろう 
\\	何事が起ころうと取り乱(みだ)すな 
\\	〔何か〕
\\	何事もなく散会した 
\\	〔万事〕
\\	何事をおいても聖書(せいしょ)を読みなさい 
\\	何事によらず注意深くやりなさい 
\\	〔詰問して〕 女の子を泣かせるとは何事か 
\\	ありげ		
\\	いわくありげ[意味ありげ]な様子で彼女は私を見た 
\\	彼はとても自信ありげだった 
\\	下す	くだす	
\\	おろす 
\\	⇒おろす(下ろす) 
\\	〔判断・命令などを与える〕 命令を下す 
\\	判決を下す 
\\	結論を早く下し過ぎた 
\\	時が評価を下すだろう 
\\	〔行う〕 彼自ら手を下した 
\\	〔下痢をする〕 昨夜からおなかを下している 
\\	〔お腹から出す〕 虫を下す 
\\	〔負かす〕 敵を下す 
\\	2対0で相手チームを下した 
\\	座	ざ	
\\	〔座る場所,席〕
\\	⇒せき(席) 全員が座についた 
\\	座を外す 〔席を〕
\\	/〔部屋から出る〕
\\	/〔しりぞく〕
\\	食事のあと女性は座を外した 
\\	〔地位〕
\\	権力の座につく 〔会社などで〕
\\	〔独裁者などが〕
\\	彼女は妻の座だけは守りたいと思った 
\\	〔大勢集まった場所〕 彼の言葉で座が白けてしまった 
\\	彼女がいると座が白ける 
\\	座を(取り)持つ 
\\	慈しむ	いつくしむ	
\\	子供を慈しみ育てる 
\\	白ける	しらける	
\\	白ける冗談 
\\	白けた顔をする 
\\	彼の言葉で座(ざ)が白けた 
\\	彼女の出現で楽しい座が白けてしまった 
\\	座を白けさせるなよ 
\\	/((俗))
\\	ほぐれる	ほぐれる	
\\	〔もつれたものがほどける〕
\\	もつれた毛糸の玉がなかなかほぐれない 
\\	〔解ける〕 やっと解決の糸口(いとぐち)がほぐれてきた 
\\	糸口を開く 
\\	〔固いものが柔らかくなる〕 一ふろ浴びて肩の凝(しこ)りがほぐれた 
\\	〔緊張などが和(やわ)らぐ〕 その一言で彼の気分もほぐれた 
\\	両者の間の張りつめた感情が次第にほぐれてきた 
\\	経理	けいり	
\\	彼はその会社で経理を担当している 
\\	経理部 
\\	身を寄せる	みをよせる	
\\	両親を失って彼はおじの家に身を寄せた 
\\	身を持ち崩す	みをもちくずす	
\\	〔身を〕
\\	彼は放蕩(ほうとう)に身を持ち崩した 
\\	〔財産を〕 身代を持ち崩す 
\\	身を以て	みをもって	
\\	〔みずから〕
\\	身を以て範(はん)を垂(た)れる 
\\	範を垂れる 
\\	〔体一つで〕 身を以て逃れる 
\\	身をやつす		身をやつす 
\\	〔身なりを変える〕 こじきに身をやつす 
\\	〔一生懸命やる〕 おしゃれに身をやつす 
\\	身を任せる	身をまかせる	身を任(まか)せる 
\\	〔なすがままになる〕
\\	なすがまま 
\\	絶望に身を任せる 
\\	〔男に抱かれる〕
\\	身を持ち崩す 
\\	身を引く		
\\	第一線から身を引く 
\\	事がおもてざたにならないうちに身を引いた 
\\	表沙汰 おもてざた 
\\	身を立てる		
\\	作家として身を立てることにした 
\\	身を粉(こな)にする		
\\	((俗)) 
\\	身を焦がす	みをこがす	
\\	身を削る	みをけずる	〔苦労する〕
\\	〔心を痛める〕
\\	身を砕く	みをくだく	
\\	((文)) 
\\	大変な苦労・努力をする。身を粉 (こ) にする。「―・いて研究にうち込む」 
\\	ひどく悩み苦しむ。「命も堪ふまじく、―・きておぼし惑ふ」
\\	身を入れる	みをいれる	仕事[勉強]に身を入れる 
\\	身も世もない		
\\	彼女は身も世もなく泣いた 
\\	身も細る	みもほそる	〔やせる〕
\\	⇒やせる(痩せる); 〔減じる〕
\\	食が細った 
\\	きまり悪くて身が細る思いでした 
\\	彼女の声は細っていった 
\\	身も蓋もない	身もふたもない	
\\	蓋 
\\	そう言ってしまっては身も蓋もない 
\\	身も蓋もないことを言うね 
\\	身の振り方		
\\	身の振り方について相談する 
\\	身の置き所がない		
\\	恥ずかしくて身の置き所がなかった 
\\	身につまされる		
\\	その娘の苦労が身につまされる 
\\	彼の話には身につまされて皆泣いた 
\\	身に付ける		〔知識・技術などを習得する〕 
\\	一芸(いちげい)を身に付ける 
\\	身に付いた芸 
\\	英語が身についている 
\\	なかなかコンピュータの技術が身に付かない 
\\	身にしみる		
\\	身にしみて有り難く思う 
\\	彼の言葉は身にしみた 
\\	やってのける	▲遣って▲退ける	
\\	彼はその大事業を見事にやってのけた 
\\	/((口))
\\	彼女は店の拡張をなんとかやってのけた 
\\	は店の床面積を増すこと
\\	は営業の範囲を広げること)
\\	もらう	
\\	貰う】	(受ける)
\\	(付与される)
\\	(賞などを)
\\	(病気を)
\\	〔物を受け取る〕
\\	父からもらった辞書 
\\	毎月5千円小遣いをもらう 
\\	彼はノーベル賞をもらった 
\\	コーヒーを一杯もらえませんか 
\\	〔家に迎える〕 子供をもらう 
\\	美人の奥さん[金持ちの嫁]をもらった 
\\	〔買う〕 このネクタイをもらおう 
\\	2キロもらおうか 
\\	〔自分のものにする〕 この勝負はもらった 
\\	命はもらったぞ 
\\	〔うつされる〕 どこかで結膜炎(けつまくえん)をもらった 
\\	〔自分で引き受ける〕 よし,このけんかは私がもらった 
\\	〔「…してもらう」の形で〕 彼は論文を妹にワープロで打ってもらった 
\\	すぐに出て行ってもらいたい 
\\	部屋代はきちんと払ってもらいます 
\\	泳ぎ方を彼に教えてもらった 
\\	例文 「この敷物(しきもの)、珍しいデザインなのね」「友達からインドのおみやげでもらったのよ」 
\\	「あいつ、アメリカでうまくやってるのかな?」「先月、手紙をもらったけど、元気そうなこと書いてあったよ」 
\\	「あの人、数学の世界ではすごく有名なんですってね」「いくつもその世界の賞をもらってる人らしいよ」 
\\	「彼、プログラマーなんだってね」「ああ、けっこういい給料もらってるらしいよ」 
\\	「君もそろそろ嫁さんをもらったらどうだい?」「自分でもそうは思うんですが、なかなかいい相手が見つからなくて」 
\\	「困ったな。これから会議なんだけど、
\\	社がすぐに品物を届けてほしいって言うんだ」「石野さんに行ってもらえば? 彼、今日暇そうにしていたから」 
\\	「素敵なネックレスね」「これね、彼に買ってもらったのよ」 
\\	「この書類、いつまでに整理すればいいんですか?」「できれば今日中にやってもらいたいんだがね」 
\\	いられない	
\\	笑(わら)わずにはいられない 
\\	泣かずにはいられなかった 
\\	ぐずぐずしてはいられない 
\\	その問題についてもう黙ってはいられない 
\\	うごめく	
\\	蠢く】	
\\	下に見える人たちがうごめく虫のように見えた 
\\	見聞	けんぶん	(知識)
\\	(経験)
\\	(観察(かんさつ))
\\	見聞する 
\\	見聞を広める 
\\	見聞が広い 
\\	見聞記[録] 
\\	見栄	みえ	
\\	(虚栄(きょえい))
\\	〔虚栄〕
\\	見栄でやった 
\\	見栄を張る 〔見せびらかす〕
\\	〔気取る〕
\\	見栄をはって彼女に高価な物を贈った 
\\	[((口))
\\	見栄坊 ⇒みえっぱり(見栄っぱり) 〔人〕
\\	「おしゃれにかけては誰にも負(ま)けたくないんだよ」「金もないくせに見栄張って」 
\\	「お前、ずいぶんいい車に乗ってるな」「見栄だよ、見栄。お金もないのに無理しちゃったんだよ」 
\\	「お前、本当にT大受けるの?」「先生に大見栄を切った
\\	手前、引っ込みがつかなくてね」 
\\	切る 
\\	彼は見栄を張って高い車を買った 
\\	彼女は見栄を張って毛皮のコートを買った 
\\	彼女は見栄でそんなうそをついたんだ 
\\	すする	
\\	啜る】	
\\	(音を立てて)
\\	(鼻水を)
\\	〔ずるずると音を立てて〕
\\	〔ちびちびと〕
\\	そばをすする 
\\	熱いコーヒーをすすった 
\\	彼らは茶をすすりながら昔話をした 
\\	〔鼻を〕 あの子はすぐ鼻をすする 
\\	場数	ばかず	
\\	場数を踏む 
\\	君はまだ場数を踏んでいないね 
\\	彼は講演者として場数を踏んでいる 
\\	門出	かどで	
\\	人生の門出に際して 
\\	君の門出を祝(いわ)おう 〔旅立ち〕
\\	!/〔新生活〕
\\	折り返し	おりかえし	
\\	(襟の)
\\	(ズボンの)
\\	(水泳の)
\\	(歌の)
\\	〔コートの襟の〕
\\	〔ズボンのすその〕((米)) 
\\	((英)) 
\\	〔水泳の〕
\\	〔歌の〕
\\	〔マラソンなどの〕
\\	〔列車・バスなど〕
\\	折り返し列車 
\\	10時半の折り返し便でニューヨークへ行った 
\\	〔…したらすぐに〕 折り返しお返事ください 
\\	戻りましたら折り返しお電話させます 
\\	折り返し運転 事故のため新宿・立川間を折り返し運転している 
\\	折り返し地点 
\\	例文 注文確認を折り返しご連絡下さい 
\\	営業担当に折り返し電話させます 
\\	折り返し電話をしているのですが 
\\	相棒	あいぼう	
\\	((口)) 
\\	〔犯罪の〕
\\	商売の相棒 
\\	スミス氏を相棒に商売をした 
\\	けじめ		
\\	⇒くべつ(区別) よいことと悪いことのけじめをつける 
\\	公私のけじめがきちんとした人だ 
\\	夫婦の間といえども、プライバシーを守るためのけじめは、つけなくてはならない 
\\	ものごとにはけじめが大切だ 
\\	自分個人と大統領の地位を切り離しておかなければ何かにつけ面倒になる 
\\	「お前、彼女との付き合い長いなあ」「うん、そろそろけじめをつけなきゃって親からも言われてるんだ」 
\\	「また昼休みに旅行の相談かい」「でも、ちゃんと仕事と遊びのけじめはつけてまーす」 
\\	「いいよ、借用証なんて書かなくても」「いくら友達でも、金銭のけじめはきちんとしとかないとね」 
\\	公私のけじめをつけなさい 
\\	公私のけじめをつけなさい 
\\	公私のけじめをつけなさい 
\\	最近うちの部署では仕事と休憩時間のけじめがなくなってきている 
\\	この辺でけじめをつけて、ちゃんと就職〔結婚〕しなさい 
\\	僕たち、もうだらだら会うのはやめて、けじめをつけよう 
\\	今回の件で、おたくはどうけじめをつけてくれるのかね? 
\\	今回の件で、おたくはどうけじめをつけてくれるのかね? 
\\	僕自身のけじめとして、先方にきちんと謝りに行きます 
\\	国会議員は再選され、「これで疑惑のけじめはついた」と語った 
\\	ほっぺ(た)	
\\	頬っぺ(た)】	
\\	ほっぺたが落ちる 
\\	このケーキはほっぺたが落ちそうだ 
\\	/((口)) 
\\	後の祭り	あとのまつり	
\\	から見ると		
\\	しかし、外から見ると、なぜこんなに非効率な作業をしているのだろう、と疑問に思うことが少なくない 
\\	私の立場からみると当然でしょう。 
\\	月から見ると、地球はボールのように見える。 
\\	小屋の様子から見ると、そこには誰も住んでいないようだ。 
\\	土壇場	どたんば	
\\	土壇場で 
\\	土壇場に追い込む 
\\	土壇場で逃げる 
\\	土壇場の和平提案 
\\	土壇場になって彼は翻意(ほんい)した 
\\	気さく	きさく	気さくな 
\\	〔気取らない〕
\\	〔うちとけた〕
\\	気さくな態度 
\\	気さくな男 
\\	叔父は気さくな人だ 
\\	この大学の学長は気さくな人柄で学生に親しまれている 
\\	彼はあまり親しくない人にも気さくに話しかける 
\\	命運	めいうん	
\\	翻意	ほんい	
\\	翻意を促(うなが)す 
\\	書面	しょめん	
\\	〔書類〕
\\	要請を書面にする 
\\	書面で申し込む 
\\	〔手紙〕
\\	書面で知らせる 
\\	〔文面〕
\\	書面だけでははっきりしない 
\\	書面審理(しんり) 〔裁判の〕
\\	いざというとき		
\\	会社の経営(けいえい)がはかばかしくないそうだが、いざというときには力になるよ 
\\	来年、アメリカに留学なさるそうですが、いざというときは、いつでもカリフォルニアにいる私の息子に相談してください 
\\	金のことでくよくよするな。いざというときは俺のところへ来い 
\\	いつ病気になるかわからないんだから、いざというときのために健康保険に入った方がいいよ 
\\	しかし、いざ計画をスタートさせる段になると彼は手を引いてしまった 
\\	力になる	ちからになる	
\\	喜んでお力になりますから。私たち、そのために、ここにいるんですからね 
\\	万一、もめたりするようでしたら、むろん私たちも力になります 
\\	踏ん張る	ふんばる	
\\	(足を)
\\	(努力する)
\\	(持ちこたえる)
\\	〔足を〕
\\	〔頑張る〕 その婦人団体は賃上げを要求して踏ん張った 
\\	任ずる	にんずる	
\\	〔任命する〕
\\	⇒にんめい(任命) 
\\	〔自分で思い込む〕 自ら詩人をもって任じている 〔詩人ぶる〕
\\	/〔自称する〕
\\	/〔詩人のつもりでいる〕
\\	こういう		
\\	こういう絵は見たことがない 
\\	こういう風にやるんです 
\\	これこれこういう理由で 
\\	こういう人が事故を起こしやすい 
\\	こういう事は二度と起ってはならない 
\\	たまえ		
\\	来たまえ、こっちに来たまえ! 
\\	入りたまえ. 
\\	《補助動詞「たまう」の命令形》 
\\	恩恵をお授けください、与えてくださいの意を表す。現代では文語的な文に用いる。「神よ、恵(めぐ)みを垂(た)れ―」 
\\	(給え)友人または目下の者に対する、穏やかな命令の意を表す。明治時代の書生言葉から。「まあ、席につき―」「君も一杯やり―」
\\	突っ立つ	つったつ	
\\	ぼうぜんとして彼女はそこに突っ立っていた 
\\	毛頭	もうとう	
\\	何を言われても毛頭驚かなかった 
\\	そんな気は毛頭ありませんでした 
\\	私の言うことにいつわりなど毛頭ありません 
\\	私は毛頭気にしていません 
\\	彼には親切心なんて毛頭ない 
\\	愛▲敬・愛×嬌	あいきょう	
\\	〔魅力があること〕 愛敬のある笑顔 
\\	愛敬のない女 
\\	〔お世辞などのうまいこと〕 あの店の売り子は愛敬がない 
\\	彼女の返答は愛敬のないものだった 
\\	彼女は客に愛敬を振りまいていた 
\\	店主は愛敬たっぷりに私を迎えた 
\\	〔こっけいなこと〕 彼の失策はいつも愛敬があって全く憎(にく)めない 
\\	〔座興(ざきょう)〕 彼はご愛敬でしくじってみせた 
\\	とんだご愛敬さ 
\\	愛敬者 〔こっけいな人〕
\\	猿の太郎は愛敬者だ 
\\	例文 彼女は愛敬たっぷりだ 
\\	彼女は目に愛敬がある 
\\	あの娘は愛敬がない 
\\	ウエートレスが客に愛敬を振りまいている 
\\	こんな失敗はご愛敬だよ。気にするな 
\\	ぼうぜん		
\\	茫然と 
\\	茫然とする 
\\	その知らせを聞いて彼はしばらく茫然としていた. 
\\	茫然自失(じしつ) 茫然自失する 
\\	彼女は息子が自分の誕生日に自殺したという知らせを聞いて茫然自失した. 
\\	ぼうぜん. 
\\	ぼう然 「それにしても首席の彼が落ちるとはねえ」「合格間違いなしって言われてたから、本人もぼう然としてるらしいね」 
\\	「不景気だから、ボーナスは現物支給って本当かい?」「ああ、俺なんかぼう然としてしまって、昨日一日、仕事が手につかなかったよ」 
\\	呆然 驚いてぼう然としてしまった 
\\	ぼう然と彼の顔を見詰めた 
\\	ぼう然とその場に立ち尽くした 
\\	茫然 ぼう然と 
\\	〔特に表情が〕
\\	焼け跡にぼう然と立ち尽くしていた 
\\	ぼう然としてどうしたらよいかわからなかった 
\\	ぼう然とした顔つきで 
\\	茫然自失 父の死の知らせに茫然自失した 
\\	その恐ろしい知らせに皆茫然とした 
\\	その恐ろしい知らせに皆茫然とした 
\\	彼はその光景に茫然として立ちつくした 
\\	事故のあと彼女は茫然としていた 
\\	事故のあと彼女は茫然としていた 
\\	といっても		
\\	私は子供がいます。子供と言ってももう大学生てすが。 
\\	偽り	いつわり	〔人をだますためのうそ〕
\\	〔罪のないうそ〕
\\	〔真実でないこと〕
\\	偽りの 
\\	〔人をだますような〕
\\	偽りのない 〔真実の〕
\\	/〔率直な〕
\\	これこそ偽りのない今の気持ちです 
\\	偽りの申し立てをする 
\\	偽りのないところ 
\\	その話は全くの偽りだった 
\\	立ち尽くす	たちつくす	
\\	遺族は雨の中でじっと立ち尽くしていた 
\\	払う		
\\	(支払う)
\\	(返済する)
\\	(ほこりを)
\\	(枝などを)
\\	(除去する)
\\	(追い払う)
\\	〔じゃまなものを取り除く〕 上着のちりをブラシで払う 
\\	本棚のほこりを払った 
\\	ズボンについたどろを払った 
\\	彼は食物に群(むら)がるはえを手で払った 
\\	庭木(にわき)の枝を払ってもらった 
\\	〔横に勢いよく動かす〕 相手の切っ先を払った 
\\	不意に彼に杖で足を払われた 
\\	〔売却する〕 ぼろをくず屋に払った 
\\	古い家具を払う 
\\	〔立ち退く〕 下宿を払う 
\\	〔支払う〕
\\	あの本にいくら払いましたか 
\\	現金[品物]で払う 
\\	〔注意・関心などを向ける〕 それには特別の注意を払うべきだ 
\\	誰も彼に敬意を払わなかった 
\\	〔そろばんの玉を〕
\\	例文 彼は部屋代を払った 
\\	今度の給料日には借金を払ってくれよ 
\\	高い月謝を払っているんだからしっかりピアノのレッスンをしてちょうだい 
\\	二人はレストランの代金をどう払うか話がまとまらなかった 
\\	彼はズボンについたほこりを払った 
\\	庭の木の枝を払ったら明るくなった 
\\	彼は敬意を払って先生に挨拶した 
\\	「梅の枝が伸びすぎたみたい」「次の日曜に払っておくよ」 
\\	「物置きがくもの巣だらけだから払っておいたわ」「あそこも時々掃除(そうじ)しなくちゃね」 
\\	「上着の背中にゴミがついてるわよ」「悪いけど払ってくれる?」 
\\	「本にほこりが積もってるよ」「たまには自分でほこりを払ってよ」 
\\	「娘さん、ピアノが上手ね」「そりゃあ高い月謝を払っているもの」 
\\	「お給料が出るとホッとするわね」「今月はまだ部屋代を払ってなかったから、すぐ払おう」 
\\	「仕事一本やりの彼を奥さんが嫌になって離婚さ」「そんな犠牲を払ってまで出世してもしょうがないのにね」 
\\	「最近の若い人はお年寄りを尊重してないな」「そんなことない。ちゃんと敬意を払っていますよ」 
\\	「炎天下の体育でうちの娘が失神しちゃったのよ」「先生ももう少し注意を払ってほしいね」 
\\	確固たる	かっこたる	(信念・主義などが)
\\	(意志などが)
\\	(地位などが)
\\	確固(かっこ)たる決心 
\\	確固たる信念 
\\	いびる		〔いじめる〕
\\	のほうが程度がひどい); 〔つらく当たる〕
\\	⇒いびりだす(いびり出す) いびる 
\\	いびりだす 
\\	彼女は辞職したが実は上司にいびり出されたのだ 
\\	彼は嫁をいびり出した 
\\	あの子をいびるのはいい加減にしなさい 
\\	程よい	ほどよい	
\\	程よい運動 
\\	程よい時分に 
\\	程よい湯かげんです 
\\	長い散歩をして程よく疲れた 
\\	鳥の丸焼きは程よくでき上がっている 
\\	は特に料理について用いる) 
\\	履修	りしゅう	
\\	《履修登録》 
\\	《課程修了》 履修する 〔学科・課程をとる〕
\\	〔修了する〕
\\	全履修科目が優(ゆう)だ 
\\	去年は何科目履修しましたか 
\\	彼女は学位取得に必要な単位をすべて履修した 
\\	[((米))
\\	彼は3年で全必須(ひっす)科目を履修した 
\\	履修カード 
\\	必須	ひっす	必須の 
\\	必須アミノ酸 
\\	必須科目 
\\	必須条件 
\\	請う|乞う	こう	
\\	アメリカ側の協力[援助]を請うために出掛けた 
\\	その部屋の使用許可を請うた 
\\	乞う,ご期待 
\\	おおぶろしき	大風×呂敷	
\\	大風呂敷を広げる 
\\	引く	ひく	
\\	(引っ張る)
\\	(引きずる)
\\	(差し引く)
\\	(減ずる)
\\	【数学】(引き算する)
\\	(線などを)
\\	(設置する)
\\	(退く)
\\	(引用する)
\\	(治(おさ)まる)
\\	(くじを)
\\	(カードを)
\\	(たじろぐ)
\\	〔引っ張る〕
\\	綱を引く 
\\	ボートを綱で引く 
\\	荷車を引く 
\\	人のそでを引く 
\\	いすを炉の側に引く 
\\	カーテンを引いて開ける 
\\	引き金を引く 
\\	〔抜き出す〕
\\	くじ[カード]を引く 
\\	〔引っ込める〕 さっと体を引く 
\\	あごを引く 
\\	さっと頭を引いて石をよけた 
\\	〔すそなどを引きずる〕 すそを長く引いたイブニングドレス 
\\	〔導く〕
\\	子供の手を引く 
\\	馬を引く 
\\	〔数を減らす〕
\\	⇒ひきざん(引き算); 〔費用・点数などから差し引く〕
\\	10から5を引く 
\\	10引く3は7 
\\	給料から税金を引く 
\\	値段を引く 
\\	あの店では全商品を1割引いてくれる 
\\	5分引けませんか 
\\	〔電気・水道などを引き入れる〕
\\	((英)) 
\\	この家にはまだ電気[ガス]が引いてない 
\\	家に電話を引いた 
\\	ガスと水道が引いてある 
\\	田に水を引く 
\\	〔線などを〕
\\	線を引く 
\\	間違った語を線を引いて消す 
\\	〔糸・筋などを〕 納豆が糸を引いている 
\\	〔引くようにして塗る〕
\\	口紅を引く 
\\	眉を引く 
\\	フライパンに油を引く 
\\	床にワックスを引く 
\\	〔あったものがなくなる〕 朝になったら熱が引いていた 
\\	洪水が引いたあと家に帰された 
\\	〔引き付ける,誘う〕
\\	注意を引く 
\\	同情を引く 
\\	[((文))
\\	女心を引く 
\\	客を引く 〔店の前などで〕
\\	/〔売春などのために〕
\\	〔例として挙げる〕
\\	〔引用する〕
\\	ミルトンから引いた句 
\\	〔辞書などで見付ける〕 辞書を引く 
\\	〔血筋・系統にある〕 貴族の血を引いている 
\\	〔風邪などを〕 風邪を引く 
\\	風邪を引いている 
\\	仕草	しぐさ 【仕草・仕▲種】	
\\	〔動作〕
\\	((英)) 
\\	あいつのしぐさが気にくわない 
\\	彼女は優雅(ゆうが)なしぐさで茶を注いだ 
\\	女の子の仕草がかわいい 
\\	あの気取った仕草 
\\	〔演技〕
\\	〔身振り〕
\\	劇のしぐさ 
\\	子供はしぐさで母親にのどの渇きを訴えた 
\\	その俳優は女のしぐさがうまい 
\\	あまた		
\\	数量の多いさま。たくさん。多く。名詞的にも用いる。「―の判例(はんれい)を集積(しゅうせき)する」「引く手―」 
\\	程度のはなはだしいさま。非常に。はなはだしく。 「たぶてにも投げ越しつべき天の川隔てればかも―すべなき」
\\	ばかり		
\\	((副助詞)) 
\\	〔おおよそ〕 10分ばかり待ってくれないか 
\\	20人ばかりの人がたまたまそこにいあわせた 
\\	〔だけ〕 彼は読むものといえば漫画ばかりだ 
\\	朝から晩まで文句ばかり言う 
\\	人の本ばかりを借りる 
\\	この国の民主主義とは名ばかりだ 
\\	彼は名ばかりの画家だ 
\\	金ばかりか家までだましとった 
\\	いくら食べてもやせるばかりだった 
\\	人間堅いばかりが能じゃないよ 
\\	黙って耐えるばかりだった 
\\	〔意味を強めて〕 こればかりの人数ではどうにもならない 
\\	何でもするがそればかりは勘弁してくれ 
\\	〔今にも…しそう〕 今にも泣き出さんばかりだった 
\\	今にも降り出さんばかりだった 
\\	出発するばかりになっている 
\\	飛び上がらんばかりに喜んだ 
\\	〔…して間もない〕 今終えたばかりだ 
\\	結婚したばかりの夫婦 
\\	〔さも…のように〕 そんなものは見あきたと言わんばかりの顔つきだった 
\\	〔それだけが原因で〕 その1題ができなかったばかりに落第した 
\\	〔…とばかりの形で、というほどの勢いで〕 負けじとばかりゴールを目ざして走った 
\\	例文 あいつばかりが男じゃない 
\\	あいつばかりが男じゃない 
\\	見渡しても見えるのは海ばかりだ 
\\	コップに半分ばかりの水が入っている 
\\	彼女に泣かんばかりに頼まれた 
\\	私も来たばかりです 
\\	電車は駅に入るばかりのところで止まってしまった 
\\	えいっとばかりに水たまりを飛び越えた 
\\	彼は油断したばかりに事故を起こしてしまった 
\\	「ここのところ見かけなかったけど、どこに行ってたの」「出張で1週間ばかり九州に行ってたんだ」 
\\	「加藤さん、いますか?」「今お昼を食べに出たばかりだよ」 
\\	「もう旅行の仕度はできたの?」「あとは出発するのを待つばかりよ」 
\\	「ディズニーランドに行きたいって、娘が泣かんばかりに頼むんですよ」「あそこは大人も楽しめますよ」 
\\	「何だ、客は女性ばかりじゃないか」「彼は女性に人気のあるピアニストだからねえ」 
\\	「あの仕事で失敗したばかりに、すっかり課長に信用されなくなっちゃって」「また挽回する機会があるよ」 
\\	「会議に遅刻したばかりか、皆に配る書類まで忘れちゃってね」「失敗が続くときってあるんだよね」 
\\	「僕がワープロを打ちましょう」「君ばかりにやらせて悪いけど、よろしく頼むよ」 
\\	トランスワールド研究所のハーマン・ヨークと話したばかりです 
\\	本数		
\\	大阪行き列車の1日の本数 
\\	あの線は本数が多い[少ない] 
\\	綺麗	きれい	
\\	〔美しい〕 きれいな 〔美しい〕
\\	〔美しく魅力的な〕
\\	〔かわいくて見た感じのよい〕
\\	きれいな服 
\\	家の後ろにきれいな庭がある 
\\	いつもおきれいですこと 
\\	きれいな少女 
\\	きれいな顔立ちの青年 
\\	きれいな声 
\\	〔汚れがない〕 きれいな水 〔清潔な〕
\\	/〔澄んだ〕
\\	部屋をきれいに掃除(そうじ)する 
\\	/〔整頓する〕
\\	手をきれいにしましょう 
\\	このふきんはきれいですか 
\\	〔手際がよい〕 彼女はきれいに相手の球を打ち返した 
\\	毛布をきれいに畳んだ 
\\	〔不正がない〕 きれいな態度 
\\	きれいな勝負 
\\	きれいな勝負にしよう 
\\	彼の腹の中はきれいだ 
\\	彼の経歴はきれいなものです 
\\	〔完全な〕 借金をきれいに返した 
\\	彼はバーをきれいに跳び越えた 
\\	皿の上のパンをきれいに食べてしまった 
\\	やくざ仲間からきれいさっぱりと足を洗いたい 
\\	過去のことはきれいに忘れなさい 
\\	〔核兵器が放射性物質を出さない〕 きれいな水爆 
\\	奇麗事 きれい事を並べた話 
\\	もんちゃくをきれい事で済ませる 
\\	そのような言い訳はきれい事にすぎない 
\\	奇麗好き きれい好きな主婦 
\\	奇麗どころ 
\\	例文 「社長の奥さん、どんな人だった?」「いやあ、きれいな人でびっくりしたよ」 
\\	「まあ、きれいな花。ありがとう」「花屋さんの店先で、あんまりきれいだから、つい買ってきちゃったの」 
\\	「字のきれいな人ってうらやましいね」「本当だね。だから最近、仕事関係の文書はだいたいワープロにしてるんだ」 
\\	「彼の机はいつもきれいに片づいてるね」「そりゃそうさ。そんなに仕事していないもの」 
\\	「マンション買ったんだって?」「うん。中古だけど、わりにきれいな出物があってね」 
\\	「魚をきれいに食べる人なんて、今時珍しいね」「私、骨以外はだいたい食べられるんです」 
\\	「また前の恋人のことを考えてるの?」「いや、彼女のことはきれいさっぱり忘れたよ」 
\\	「これでローンもきれいに片づいたね」「ほんと、長かったわね」 
\\	桜の花がきれいだ 
\\	今日は空がきれいだね 
\\	彼の恋人はきれいな人だよ 
\\	君の字はきれいだね 
\\	手をきれいに洗いなさい 
\\	彼の部屋はいつ行ってもきれいだ 
\\	部屋をもう少しきれいにしなさい 
\\	机の上をきれいにしなさい 
\\	よほど空腹だったのか彼は料理を全部きれいに食べた 
\\	彼女のことはきれいさっぱりあきらめた 
\\	借金を全部きれいに返した 
\\	きれいごとを言っている場合じゃない 
\\	商用	しょうよう	
\\	商用で訪問する 
\\	商用で出掛けている 
\\	担任	たんにん	
\\	担任する 
\\	このクラスは私の担任です 
\\	彼は新任の先生を2年の担任にした 
\\	担任学級 
\\	担任教師 
\\	((英)) 
\\	[((女性)) 
\\	そんなに		
\\	(数が)
\\	(量が)
\\	そんなにたくさんは無理だと思いますけど、週に1つか2つでも大助かりだわ 
\\	「妻は、食事の仕度をしてお待ちしています」「おやまあ、そんなご心配なさらなくてもいいのに」 
\\	「この部屋、冷房が少しききすぎじゃない? ぞくぞくしてきた」「そんなに寒いなら、カーディガン貸してあげましょうか」 
\\	「私一日で10万円も使っちゃった」「そんなにたくさん!?」 
\\	「君はいつも待ち合わせに遅れるから、うんざりだよ」「そんなに怒らなくてもいいでしょ。バスが遅れたんだからしょうがないじゃない」 
\\	「どうしてもこれだけはお受け取りください」「そんなにおっしゃるなら、いただいておきましょう」 
\\	「この本、面白かったよ」「私はそんなに面白いとは思わなかったけど」 
\\	「寒いと思わない?」「何言ってんだ。今日はそんなに寒くないよ」 
\\	支度・仕度	したく	
\\	〔準備〕
\\	〔段取り〕
\\	式典(しきてん)の支度はすべてできた 
\\	団体旅行の支度は皆旅行社がやってくれる 
\\	夕食の支度はまだかね 
\\	〔装備〕
\\	登山の支度 
\\	探検隊(たんけんたい)は支度が十分でなかった 
\\	花嫁の支度 
\\	〔身なりを整えること〕 彼女は支度が長いんだ 
\\	外出の支度をする 
\\	支度金 
\\	〔移転の〕
\\	海外出張の支度金 
\\	支度部屋 
\\	〔相撲の〕
\\	例文 食事の支度を大急ぎでやった 
\\	彼女は夫の旅行の支度を手伝った 
\\	彼女は夫の旅行の支度を手伝った 
\\	彼女は外出の支度に時間がかかる 
\\	(出かける)支度は出来ましたか 
\\	帰ると夕食の支度が出来ていた 
\\	支度はすっかり整いました 
\\	あか抜	あかぬけ
\\	垢抜け】	あか抜けのした 
\\	/〔衣服などが粋な〕
\\	/〔都会風な〕
\\	あか抜けのしない 
\\	あか抜けした物腰(ものごし) 
\\	あか抜けのした服装をしている 
\\	いくら	幾ら	"いくらも 
\\	いくら何でも 
\\	幾らでも 
\\	幾らか 
\\	いくら 
\\	ーーー (数が)
\\	(量・金額が)
\\	(時間が)
\\	(距離が)
\\	〔金額〕
\\	それをいくらで買いましたか 
\\	みんなでいくらになりますか 
\\	〔数量〕⇒どれくらい(何れ位) この水槽には水がいくら入りますか 
\\	この旅行かばんの重さはいくらありますか 
\\	〔概数〕 それを3千いくらかで買った 
\\	〔単位の数字の後につけて,…いくらで〕 1時間いくらで 
\\	この商品は1ダースいくらで売られている 
\\	私はガイドを1日いくらで頼んだ 
\\	〔どんなに…でも〕 いくら金持ちでも永遠の若さは買えない 
\\	いくら悪くても50キロの収穫(しゅうかく)はあると見積もられる 
\\	いくら遅くても8時には着くでしょう 
\\	いくら経験豊かな彼でもこの難事業には歯が立つまい 
\\	いくらうまくいっても収支とんとんだ 
\\	いくらなんでもそれはあんまりだ 
\\	いくらなんでもこれは高過ぎる 
\\	いくらなんでもあの方にそんなことまでお願いできません 
\\	例文 「このパン5個でいくらになりますか?」「おまけして500円ちょうどでいいよ」 
\\	「このスカーフはいくらで買ったの?」「150ドルくらいだったと思うけど」 
\\	「このビルの建設にはいくらぐらいかかる見込みかね」「ざっと5億円は下らないと思われます」 
\\	「いろいろ取りそろえて花束をつくってほしいんですが」「おいくらぐらいのにしましょうか」 
\\	「今月はおこづかい、まだいくらか残ってる?」「うん、給料日まではもちそうだよ」 
\\	「今日は私がおごるわよ」「でも、私にもいくらか払わせてよ」 
\\	「この商品、まだ在庫ありますか?」「いくらか倉庫に残ってるはずですよ」 
\\	「このごろ仕事は忙しいの?」「以前と比べるといくらか楽になったよ」 
\\	「金ならいくらでも出すよ」「これは金の問題じゃないことくらい、君にもわかってるだろう」 
\\	「もうS社との契約は諦めた方がいいですね」「いや、まだいくらでも打つべき手があるだろう」 
\\	「疲れただろう。喫茶店にでも入って休もうか」「いや、大丈夫。まだいくらでも歩けるから」 
\\	「今日はお金はいくらも持っていないんです」「お金のことなら心配いりませんよ。今日は私がおごりますから」 
\\	「もうお腹がいっぱいだよ」「まだいくらも食べてないじゃないの。せっかく作ったんだから、もっと食べて」 
\\	「発車まで時間はいくらもないよ」「うん、とにかく急ごう」 
\\	「この企画の実現にはかなりの費用がかかりますよ」「いくらお金がかかっても構わないから、進めてみなさい」 
\\	「約束の時間に伺えそうもないのですが」「私の方は、あなたがいくら遅くなっても構いませんが」 
\\	「僕はいくら飲んでも酔わないんだよ」「僕なんか、ほんの少し飲んだだけで酔っちゃうから、経済的でいいって女房が言ってるよ」 
\\	「いくら働いてもお金ってたまらないね」「本当。ローンのために働いてるようなものさ」 
\\	「いくら言っても、親には僕の夢をわかってもらえないんだ」「だからって諦めるのか?」 
\\	「借金の保証人になっていただけませんか」「いくらあなたの頼みでも、お金がからむことは引き受けたくないなあ」 
\\	「彼は少し強引すぎるね」「いくら仕事ができても、あのやり方は反感を買うよ」 
\\	「いくら課長とはいえ、あの発言は許せないわ」「本当。女をばかにしてるわよね」 
\\	あといくら必要か? 
\\	増加はいくらか? 
\\	15の2倍はいくらか? 
\\	5はいくら? 
\\	「今日のところはおいくらですか」「10ドル35セントになります」 
\\	米5キロでいくら(値段)ですか 
\\	いくらで買ったんですか 
\\	この指輪をお金にするといくらぐらいになりますか 
\\	小麦粉はいくら買いましたか 
\\	いくら(値段)でもいい、買うよ 
\\	金はいくらでも出しましょう 
\\	米ならいくらでもある 
\\	まだいくらでも方法はある 
\\	米ならいくらでも持って行きなさい 
\\	金はいくらも持っていない 
\\	面白い本はいくらもない 
\\	コーヒーはいくらも残っていない 
\\	時間はもういくらもない 
\\	私がいくらか払うよ 
\\	いくらかお金を貸してくれませんか 
\\	「米はいくらか残っていますか」「はい、(いくらか)残っています」 
\\	いくらお金を出してもらってもこの絵は売りません 
\\	いくら時間がかかってもよいからいいものを作りなさい 
\\	いくら水を飲んでものどが渇く 
\\	いくら遅く〔少なく〕てもかまわない 
\\	いくら働いてもお金がたまらない 
\\	いくら彼が頭が良くてもこの問題は解けないだろう 
\\	いくらあなたの頼みでも私にはできません 
\\	いくら早く来ても早過ぎるということはない 
\\	いくら厳重に警戒してもし過ぎるということはない 
\\	いくら厳重に警戒してもし過ぎるということはない 
\\	いくら何でもその言い方はひど過ぎる 
\\	物腰	ものごし	
\\	((英)) 
\\	彼女は静かな物腰をしている 
\\	やさしい物腰で 
\\	しとやかな物腰 
\\	/((文))
\\	穏便	おんびん	穏便な 
\\	穏便な処置を頼む 
\\	は「示談による」の意味) 穏便に事を運ぶ 
\\	穏便にすます 
\\	和英辞書の中には
\\	をあげているものもあるが、これは「秘密のうちに処理する」とか「揉み消す」ということ。やや意味が異なる。ここは、サラリと
\\	とするのが無難であろう。 おい、そんなに怒るなよ。何か穏便にすます法があると思うがね 
\\	運ぶ	はこぶ	
\\	(移動する)
\\	(輸送する)
\\	(進展する)
\\	(進展させる)
\\	〔運搬する〕
\\	((文)) 
\\	〔輸送する〕
\\	と
\\	は運んで先方に渡すまでを言う) ⇒うんそう(運送),ゆそう(輸送) ピアノを家の中に運び込む 
\\	テーブルを部屋から運び出す 
\\	次の駅へ乗客を運ぶ 
\\	飛行機[船]でその機械を運んだ 
\\	テーブルをここに運んできてくれ 
\\	〔物事が進む〕
\\	仕事は順調に運んでいる 
\\	万事滞(とどこお)りなく運んでいる 
\\	滞りなく 
\\	事はそこまで運んでいない 
\\	〔事を進める〕 彼女はいつも自分に都合のよいように事を運ぶ 
\\	試合をうまく運ぶ 
\\	〔動かし進める〕 筆を運ぶ 
\\	足を運ぶ 
\\	隣町に足を運ぶ 
\\	例文 「奥の部屋まで、さっき届いた小包を運んでくれない?」「今ちょっと手が離(はな)せないんだ」 
\\	「交通事故で病院に運び込まれたんだって?」「うん。でもたいしたけがじゃなかったんだ」 
\\	「工事は進んでいますか?」「今のところ、順調に運んでいますよ」 
\\	「あの二人、忘年会ですっかり意気投合して、この秋結婚だってね」「こんなに簡単に話が運ぶとは思ってなかったよ」 
\\	「夜中に原稿を書くんですか」「その方が筆が運ぶもんだから」 
\\	そのトランクは、一人では運べないほど重たかった 
\\	石油はタンカーで運ばれてきます 
\\	風が春の便りを運んできた 
\\	事はうまく運んでいる 
\\	計画が簡単に運ぶとは思わない 
\\	工事は順調に運んでいます 
\\	出る	でる	
\\	(出て来る)
\\	(出て行く)
\\	(去る)
\\	(卒業する)
\\	(出席・参加する)
\\	(現れる)
\\	(日・月が)
\\	(芽が)
\\	(露出する)
\\	(流出する)
\\	(発生する)
\\	(産出する)
\\	(発売する)
\\	(果物・野菜が)
\\	(料理が)
\\	(与えられる)
\\	(超過する)
\\	〔外に行く〕
\\	庭に出ようか 
\\	母は買い物に[用事で]出ています 
\\	部屋から出て来なさい 
\\	出て行け 
\\	パパはもうお風呂から出た 
\\	その男は先月刑務所を出たばかりだ 
\\	〔進み出る〕 一歩前へ出なさい 
\\	〔出発する〕 旅行に出る 
\\	今度の急行は5番ホームから出る 
\\	何時に家を出ようか 
\\	彼女は15歳の時に東京に出て来た 
\\	(((英))
\\	駅前から公園行きのバスが出ています 
\\	〔現れる〕 日が出た 
\\	今夜は月が出ている 
\\	この家には幽霊が出る 
\\	その家はごきぶりがたくさん出る 
\\	畑から小判がたくさん出てきた 
\\	彼はよくテレビ[舞台]に出る 
\\	すいかが市場に出ている 
\\	クロッカスの芽が出た 
\\	私は酒を飲むとすぐ顔に出る 
\\	驚きの色が顔に出た 
\\	〔突き出る〕 くぎの頭が出ている 
\\	スリップがスカートの下から出ていますよ 
\\	腹が出ている 
\\	〔流れ出る〕 目から涙が出た 
\\	鼻水が出ている 
\\	腕から血が出ている 
\\	よだれが出た 
\\	額に玉のような汗が出ていた 
\\	この谷は温泉が出る 
\\	〔出席する〕
\\	〔参加する〕
\\	〔出馬する〕((米)) 
\\	((英)) 
\\	会合[講義]に出る 
\\	式に出た 
\\	法廷に出る 
\\	今日は会社に出ている 
\\	試合[弁論大会]に出た 
\\	人前に出ると気後れがする 
\\	〔電話に答える〕 電話に出てください 
\\	はい,お出になりました〔交換手が〕 
\\	母は今電話に出ている 
\\	〔到達する〕 間もなく村へ出た 
\\	この道は海に出る 
\\	20分で街道へ出た 
\\	〔超過する〕⇒こえる(越える) 3日を出ないうちに仕上がるだろう 
\\	彼は40を出ている 
\\	費用は2,000円を少し出た 
\\	〔出版・掲載される〕 私の本はもうじき出る 
\\	最近出た本 
\\	特別号はいつ出るのですか 
\\	そのニュースは明日の新聞に出る 
\\	あの事件が週刊誌に詳しく出ている 
\\	彼は電話番号を電話帳に載せていない 
\\	[((米)) 
\\	/((英)) 
\\	〔見つかる〕 これは彼の小説によく出て来る女性だ 
\\	盗まれた指輪はじきに出た 
\\	無くなったペンは出て来ましたか 
\\	〔与えられる〕 1週間の暇が出た 
\\	許可が出た 
\\	その仕事には5万円の報酬が出る 
\\	食事が出た 
\\	ごちそうが山と出た 
\\	〔提出・提示される〕 その問題は会議に出た 
\\	宿題が出た 
\\	論文は全部出ている 
\\	彼の辞表が出た 
\\	[((文))
\\	試験ではどんな問題が出ましたか 
\\	街角に大きな看板が出ている 
\\	〔起こる〕 火事は隣家から出た 
\\	町に大水が出た 
\\	風が出て来た 
\\	その事故で死者が5名出た 
\\	白髪が出て来た 
\\	あの人たちの顔を見たら元気が出た 
\\	ゆうべから高い熱が出ている 
\\	食物の匂いをかいだら食欲が出た 
\\	いざとなると勇気が出なかった 
\\	今日はどうも調子が出ない 
\\	〔産出される〕 この辺は石油が出る 
\\	この町からは多くの偉人が出た 
\\	〔由来する〕 この語はギリシア語から出ている 
\\	その話は彼から出たものだ 
\\	山田から出たうわさだ 
\\	彼の家は源氏から出ている 
\\	費用は彼から出たらしい 
\\	[((口))
\\	それは好奇心から出た行為だ 
\\	〔もたらされる〕 ようやく結論が出た 
\\	なかなかいい案が出て来ない 
\\	〔乗り出す〕 世に出る 
\\	[((文))
\\	/〔出世する〕
\\	実業界に出る 
\\	〔卒業する〕 大学を出る 
\\	小学校もまともに出ていない 
\\	〔支出される〕 先月はずいぶん金が出た 
\\	出る方が入るより多い 
\\	〔売れる〕 この本はよく出る 
\\	今年はガスストーブがよく出た 
\\	〔態度をとる〕⇒でかた(出方) 高飛車に出る 
\\	彼らには強く出たほうがいい 
\\	すべては先方がどう出るかに掛かっている 
\\	「僕に出来ないことはないさ」「大きく出たね」 
\\	〔増す〕 途中からスピードが出た 
\\	〔色や味などが〕 この昆布はよくだしが出る 
\\	このお茶はよく出る 
\\	出る杭(くい)は打たれる ⇒くい(杭) 出る所 出る所へ出て決着を付けよう 〔法廷で〕
\\	/〔警察に行って〕
\\	出る幕 君の出る幕ではない 
\\	例文 「この映画、面白くないね」「うん、もったいないけど出ましょうか」 
\\	「明日は早いの?」「朝6時には家を出ないと間に合わないんだ」 
\\	「娘さんはお元気?」「去年結婚して、家を出たんですよ」 
\\	「駅から直通のバスは出てるんですか?」「朝の混雑時は10分おきに出てますよ」 
\\	「駅はどこですか?」「この角を右に行けば出ますよ」 
\\	「痛っ。あら、ここくぎが出てるわ」「打っておかなくちゃ。カナヅチはどこだったかな」 
\\	「あの木の枝、車道の方に出すぎていて、危ないわね」「市役所の方で、もっとまめに手入れしてほしいよね」 
\\	「今、4年生よね。大学出たらどうするの?」「さあ。その前に出してもらえるかどうか…」 
\\	「お茶を習うのに費用はどのくらいかかるの?」「ひと月で2万円をちょっと出るぐらいよ」 
\\	「雲が出てきたみたいね」「せっかくのお月見なのに、残念だわ」 
\\	「あったかくなってきたわね」「そろそろゴキブリが出る頃だな」 
\\	「あら、もうすいかが出ているわ」「うん。ひとつ買ってみようか」 
\\	「なくした手袋、出てきたの?」「うん、スキーストックについたままになってたの」 
\\	「おたくのお子さん、朝からマラソンして張り切ってるわね」「今度の区の陸上大会に出るんですって」 
\\	「法学の講義に出るの?」「今日は休講にしましょうよ。どうせ大教室の授業ですもの」 
\\	「お前の親父、政治家になるって言い出したんだって?」「ああ、選挙に出るって張り切っちゃって、家族は大弱りだよ」 
\\	「そのニュースどこで知ったの?」「何言ってるんだ。どの新聞にも出てるよ」 
\\	「近頃、彼、ちらほらテレビに出てるわね」「やつも、やっと売れてきたな」 
\\	「今年は台風がひどかったなあ」「水が出て被害を受けたところも多かったそうだよ」 
\\	「あっ、花粉症でしょ」「鼻は出るわ、涙は出るわ、くしゃみは出るわで、仕事なんかしてられないよ」 
\\	「このお茶あまり出ないわね」「それ、もう出がらしよ。葉っぱを替えなくちゃ」 
\\	「熱が出て、ちょっと起き上がれなかったんです」「今、季節の変わり目だから疲れが出やすいんだよね」 
\\	「あの車、かなりスピードが出そうね」「でも東京じゃ、ぶっ飛ばすのは無理だよ」 
\\	「昨日のクラス会で君の話が出たよ」「どうせまた悪口を言ってたんだろうよ」 
\\	「物理の試験は終わったの?」「ヤマをかけた問題が全然出なくて、まいったよ」 
\\	「僕は元々、山口県の出身なんです」「山口県からは偉い人がたくさん出ているね。君も頑張るんだな」 
\\	「あなたの企画、会議で評判よかったみたいよ」「お世辞でもそう言われると、やる気が出るなあ」 
\\	「今日はさんざん歩きまわって、クタクタだ」「ゆっくり風呂にでもつかって、食事すれば、元気も出てくるよ」 
\\	「ボーナス出た?」「すずめの涙ほどだけどね」 
\\	「お昼はサンドイッチでも作って持っていく?」「アルバイト先でお弁当が出ると思うよ」 
\\	「夢のような結婚式だったわね。純白のウェディングドレスも素敵だったし…」「豪華な食事も出たしね」 
\\	「歴史シリーズの最新刊はいつ出るんですか?」「作業が遅れてまして、2、3カ月先になりそうです」 
\\	あたかも	
\\	恰も	
\\	〔ちょうど〕
\\	⇒まるで(丸で)2 あたかも雷のような音だった 
\\	端末(たんまつ)がデバイスを、あたかも一般のサーバであるかのように利用できる 
\\	〔まさに〕
\\	時あたかも関東大震災の年であった 
\\	あたかもよし あたかもよし船は翌日出港することになっていた 
\\	差し当たり	さしあたり	〔目下〕
\\	〔当分〕
\\	差し当たりこれ以上は要らない 
\\	差し当たりここに滞在することにしよう 
\\	差し当たりの資金は確保した 
\\	会釈	えしゃく	
\\	〔軽いお辞儀〕
\\	〔うなずき〕
\\	会釈する 
\\	〔思い遣り〕 
\\	遠慮会釈なく 遠慮会釈なくつばを吐き散らす〔気のとがめなしに〕 
\\	遠慮会釈なく相手をやっつけた〔容赦なく〕 
\\	彼は遠慮会釈なく私の論文を酷評(こくひょう)した〔あからさまに〕 
\\	ともなく		
\\	だれにともなく話し掛けた 
\\	鳥はどこへともなく飛び去った 
\\	ラジオから流れる音楽を聞くともなく聞いていた 
\\	怒る	おこる 
\\	いかる	いかる 
\\	⇒おこる(怒る) 
\\	〔角(かく)ばる〕 -角張る 
\\	彼は肩が怒っている 
\\	(堪忍袋の緒が切れる)
\\	(しかる)
\\	〔腹を立てる〕
\\	((口)) 
\\	彼女の言葉を聞いて怒った 
\\	そのことで私に怒っているのか 
\\	かんかんに怒っている 
\\	口をすべらせて彼女を怒らせてしまった 
\\	彼は怒って私になぐりかかった 
\\	〔叱る〕
\\	あの子はよく先生に怒られる 
\\	そんなことをすると怒られるぞ ((口)) 
\\	例文 「子供のいたずらにいちいち怒るなよ」「大人気ないことはわかってるんだけど、ついカッとなってしまうのよ」 
\\	「僕のこと、怒ってる?」「当たり前でしょ。約束すっぽかしといて」 
\\	「君って怒った顔も素敵だね」「お世辞言ってもダメよ。浮気(うわき)の相手は誰なの!?」 
\\	「うちの子、何にも悪いことしてないのに先生に怒られたってむくれてるわ」「子供を叱(しか)るときは慎重(しんちょう)にやってほしいわね」 
\\	寄る	よる	
\\	よる【因る】 
\\	よる【▲拠る】 
\\	(近付く)
\\	(寄り道する)
\\	(訪問する)
\\	(脇に)
\\	(集合する)
\\	(しわが)
\\	〔近付く〕
\\	彼はそばへ寄って来た 
\\	冬が忍び寄ってきた 
\\	〔端に近付く〕 わきへ寄れ 
\\	もう少し右に寄ってください 
\\	彼らはわきに寄って警官を通した 
\\	〔集まる〕 彼女らが寄ると亭主の悪口を言う 
\\	彼らはストーブの前に寄った 
\\	三人寄れば文殊(もんじゅ)の知恵 ((諺)) 
\\	〔訪れる〕 後で[帰りに]君の家に寄るよ 
\\	私は仙台に寄ってから青森に行く 
\\	近くにお越しの節はお寄りください 
\\	英語では単なる挨拶でなく本気で言っているととられることに注意) 
\\	〔重なり増える〕 笑うと顔にしわが寄る 
\\	寄る年波には勝てなくなった 
\\	〔もたれる〕 手すりに寄って考えごとをしていた 
\\	〔相撲で〕
\\	寄らば大樹の陰 
\\	寄ると触(さわ)ると 
\\	寄ると触るとその事件のことばかり言っている 
\\	例文 「全員、カメラに入るかな」「坂本さんが入らないよ。もう少し右に寄って」 
\\	「この道路は大きなトラックがびゅんびゅん通ってるね」「危ないから、もう少し端に寄ろう」 
\\	「お父さん、そばに寄らないでよ。お酒くさーい」「何だい。せっかくおみやげ買ってきてやったのに」 
\\	「ピザ持ってきてくれたの? ありがとう」「あらあら、急いで歩いてきたから、中身が寄っちゃったわね」 
\\	「このタレントは若い子に絶大な人気だな」「うちの女子社員たちも、寄るとさわると彼の噂をしているぜ」 
\\	「子供のいじめ問題もなかなか収まらないね」「クラス中が寄ってたかって弱い者いじめをするんだってね」 
\\	「なかなかいい案じゃないか。みんなで考えを出し合ったのかね」「ええ、三人寄れば文殊の知恵ですよ」 
\\	「家を建て直したんだって?」「狭い家ですけど、近くに来たときはぜひ寄ってください」 
\\	「この近くに林さんの家があるんだ。ちょっと寄ってみないか」「連絡なしに行ったりして、迷惑じゃないの?」 
\\	拠る	よる	
\\	よる【因る】 
\\	よる【寄る】 
\\	〔頼る〕 信仰(しんこう)によらなければ私は生きてゆけなかったろう 
\\	この党のよって立つ地盤は労働者だ 
\\	〔かかる〕 成否(せいひ)は君の努力いかんによる 
\\	「社会的規範などを持つ、~などに拠る」という意味で、社会や周囲の環境から自然に身についた状態を表す。
\\	(辞めたかったんだけど、契約を盾にとられてしまったんだ)のように、「契約などにとらわれる」という意味もある。一方、
\\	という能動態で使うと、訳は同じでも自ら選び取るという自発的なニュアンスが強く出る。 ビクターさん、彼らの考え方の基準はあなたたちと違ってるように思われますか――ビジネスマンに対するのと報道関係者に対するのとでは基準が違うのでしょうか? 
\\	男と女では、求められる道徳が違うな 
\\	辞する	じする	
\\	じ・す【辞す】 
\\	〔辞去する〕
\\	〔辞職する〕
\\	〔辞退する〕
\\	〔遠慮せず…する〕 彼があくまで譲歩(じょうほ)しないなら裁判に持ち込むことも辞さない 
\\	出方	でかた	
\\	〔水などの〕
\\	⇒で(出) 水の出方がよい[悪い] 
\\	〔態度〕
\\	〔やり方〕
\\	こちらが友好的な出方をすれば彼も和(なご)むだろう 
\\	相手の出方次第だ 
\\	彼の出方を見守ろう 
\\	見守る	みまもる	
\\	これから私のすることを見守っていなさい 
\\	小鳥がひなにえさをやるのをじっと見守っていた 
\\	子供を見守る母 
\\	その事業の発展を見守ることにした 
\\	雅量	がりょう	
\\	は人の言動を許容できること
\\	は大らかなこと) 雅量のある 
\\	雅量のない 
\\	観念	かんねん	
\\	(意識)
\\	(諦め)
\\	〔ある事に関する考え〕
\\	平等(びょうどう)の観念 
\\	それは誤った観念だ 
\\	彼女はアメリカ人に関して固定観念を抱いていた 
\\	戦争の悲惨さは観念として知っているだけだ 
\\	この際,観念的でなく実際的な計画を立てるべきだ 
\\	〔物事についての意識〕
\\	彼には時間[距離]の観念がない 
\\	〔あきらめ〕
\\	〔覚悟〕
\\	死を観念する 
\\	観念のほぞを固める 
\\	ほぞを固める 
\\	病気は治らないものと観念した 
\\	犯人は観念してピストルを投げ捨てた 
\\	くじける	
\\	挫ける】	
\\	〔手足などが〕⇒くじく(挫く) 
\\	〔熱意などが弱まる〕 心がくじける 
\\	この失敗で彼の勇気はくじけた 
\\	この災難にも彼は少しもくじけなかった 
\\	彼の返事で彼女の望みはくじけた 
\\	は「希望[元気]を失わせる」。ほかに
\\	(士気をくじく)を使って
\\	としてもよい。能動形で
\\	(失敗して彼はくじけてしまった)のような言い方もできる。 3度目の大学入試にも失敗して、彼はすっかりくじけてしまった 
\\	失敗して彼はくじけてしまった 
\\	くじく	
\\	挫く】	
\\	(足首などを)
\\	(落胆(らくたん)させる)
\\	(挫折(ざせつ)させる)
\\	〔捻挫(ねんざ)する〕
\\	転んで足首をくじいた 
\\	〔勢いをそぐ〕
\\	彼の冷ややかな応対で私の意気込みはすっかりくじかれた 
\\	弱きを助け強きをくじく 
\\	私のすべての望みはくじかれた 
\\	彼らの計略(けいりゃく)をくじいてやった 
\\	あの人はいつも人の出端をくじく 
\\	事故のため計画は出端をくじかれた 
\\	付き	つき	1〈付着すること, またその具合〉 このインキは付きがいい[悪い]. 
\\	2〈火の付き具合〉 乾いた木は火の付きがいい. 
\\	3〈幸運〉 
\\	つきが回ってきた. 
\\	《口語》 
\\	〔点火〕 このマッチは付きが悪い 
\\	乾(かわ)いた材木は火の付きが早い 
\\	〔乗り〕 この口紅は付きが悪い 
\\	この光沢紙(こうたくし)はインクの付きがよくない 
\\	つき		
\\	いよいよつきが回ってきた 
\\	つきに見放(みはな)された 
\\	ありあり		〔はっきりと〕
\\	〔他と区別できるほど明瞭(めいりょう)に〕
\\	〔生き生きと〕
\\	月の光で背の高い人影(ひとかげ)をありありと見ることができた 
\\	[((文)) 
\\	彼の顔をありありと憶(おぼ)えている 
\\	見たことをありありと描写(びょうしゃ)する 
\\	彼女が喜んでいるのがありありとわかった 
\\	ありそう	有りそうな	
\\	経済	けいざい	
\\	(財政)
\\	(節約)
\\	〔生産,流通,消費などの活動〕
\\	〔その仕組〕
\\	経済の 
\\	日本[国民]経済 
\\	ブロック[自立/自由/統制/計画]経済 
\\	経済危機 
\\	経済の安定[安定化] 
\\	経済のひずみ 
\\	経済的相互依存 
\\	〔財政〕
\\	家庭経済 
\\	それではわが家の経済が成り立たない 
\\	それは私の経済が許さない 
\\	経済援助 
\\	〔節約〕
\\	経済的な (に) 
\\	労力の経済 
\\	このやり方をすれば時間の経済になる 
\\	天然資源を経済的に使う 
\\	いつも経済的にやってきた 
\\	落とす	おとす	
\\	(落下(らっか)させる)
\\	(汚れなどを)
\\	(漏(も)らす)
\\	(下落させる)
\\	(失う)
\\	(試験・単位などを)
\\	(減じる)
\\	(低下させる)
\\	〔落ちさせる〕
\\	橋の上から石を落とす 
\\	木々はすっかり葉を落とした 
\\	木立が地上に影を落としていた 
\\	月が湖面に影を落としていた 
\\	川の水を貯水池に落とす 
\\	〔取り去る〕
\\	ペンキ[さび]を落とす 
\\	ひげを落とす 
\\	垢(あか)[化粧]を洗い落とす 
\\	服[雨靴]の泥を落とす 
\\	セーターの汚れをきれいに洗い落とす 
\\	〔落としてなくす〕
\\	お金[財布]を落とした 
\\	〔程度を下げる〕 スピードを落とす 
\\	声を落とす 
\\	高度を落とす 
\\	ステレオ[テレビ]の音量を落とす 
\\	品質を落とす 
\\	話の程度を落としてもっと分かりやすくした 
\\	彼女は調子を落としている 
\\	はスポーツで)/
\\	選手を二軍に落とす 
\\	乞食(こじき)に身を落とす 
\\	今度のことですっかり信用を落としてしまった 
\\	彼女は麻薬事件に巻き込まれて評判を落とした 
\\	〔悪い状態に陥らせる〕 彼は私を罪に落とした 〔仕向けた〕
\\	/〔計略的に〕
\\	/〔罪を転嫁(てんか)した〕
\\	〔金などを使う〕 観光客がたくさん金を落としていった 
\\	〔落第させる〕 50点未満は落とす 
\\	〔負ける〕
\\	日本シリーズの第1戦を落とした 
\\	〔攻め取る〕 要塞(ようさい)を落とす 
\\	〔口説いて自分の意に従わせる〕 あの女きっと落としてみせるぞ 
\\	〔「…し落とす」の形で,省く〕 1語読み落とす 
\\	言い落としたことがある 
\\	報告書に重要な事項を書き落としてしまった 
\\	〔競(せ)り落とす〕 テーブルを3万円で落とした 
\\	道路工事を入札して落とした 
\\	〔処理する〕 航空運賃は会社の伝票で落とした 
\\	交際費を経費で落とした 
\\	〔柔道などで〕 締め技で相手を落とす 
\\	参る	まいる	
\\	(行く)
\\	(来る)
\\	(負ける)
\\	(当惑する)
\\	(精神的に)
\\	(墓に)
\\	(神社に)
\\	〔「行く,来る」の謙譲語〕
\\	はい,ただいま参ります 
\\	〔「…してくる」の丁寧語〕 春めいて参りました 
\\	〔参詣する〕
\\	〔降参する〕
\\	〔閉口する〕((口)) 
\\	これは[一本]参った 
\\	/((米口)) 
\\	あのなぞなぞには参った 
\\	[((英))
\\	どうだ,参ったか 
\\	「離してくれ」「参ったと言え」「参った」 〔子供のけんかで〕 
\\	米国では
\\	英国では
\\	いつものばか話で参ったよ 
\\	部屋の暑いのには参った 
\\	〔心を奪われる〕
\\	彼は三田嬢にすっかり参っているらしい ((口)) 
\\	理不尽	りふじん	
\\	理不尽な値段 
\\	/((口)) 
\\	あんな小さな子を働かせるなんて理不尽だ 
\\	春めく	はるめく	
\\	日ましに春めいてまいりました 
\\	吹く風がどことなく春めいていた 
\\	どことなく 
\\	投じる	とうじる	
\\	(投げる)
\\	(費やす)
\\	(投資する)
\\	〔投げる〕
\\	彼女は海に身を投じた 
\\	死んだ場合は
\\	とも言う) 獄に投じられた 
\\	筆を投じる 
\\	夕日が海に燃えるような光を投じた 
\\	〔仲間に入る〕 彼女は女性解放運動に身を投じた 
\\	少年は革命グループに身を投じた 
\\	〔つぎ込む〕 彼女は私財を投じて学校を建てた 
\\	父は全財産をその事業に投じた 
\\	トンネル建設のために投じられた労力ははかりしれない 
\\	〔うまく合う,一致する〕 時流に投じる 
\\	その標語(ひょうご)はうまく人気に投じた 
\\	政府は機(き)に投じた策を立てるべきだ 
\\	〔投票する〕 清き一票を投じる 
\\	彼に1票を投じた 
\\	〔投降する〕 敵に投じる 
\\	〔泊まる〕 一夜を湖畔の宿に投じた 
\\	狭い	せまい	(幅が)
\\	(面積が)
\\	(限られた)
\\	〔幅が〕
\\	〔面積が〕
\\	狭い道 
\\	狭い庭[部屋] 
\\	日本は狭い国だ 
\\	その部屋には所狭(ところせま)しと本が散乱していた 
\\	所狭し 
\\	〔考えなどが〕 了見が狭い 
\\	肩身が狭い思いをする 
\\	彼は視野が狭い 
\\	彼の外国文学の知識は狭い 
\\	ともす	【▲点す・▲灯す】	
\\	電灯をともす 
\\	爪(つめ)に火を点すような暮らしをする 
\\	なすりあい	【▲擦り合い】	
\\	責任のなすり合いはやめよう 
\\	切り抜ける	きりぬける	
\\	切り抜く 
\\	(克服(こくふく)する)
\\	(苦闘(くとう)して)
\\	〔敵の囲(かこ)みなどを〕
\\	〔困難などを〕 彼は苦境を切り抜けた 
\\	何とか切り抜けられますよ 
\\	私は苦境を楽々と切り抜けた 
\\	私たちはいくつものあらしを切り抜けてきた 
\\	彼の援助のおかげで難局を切り抜けた 
\\	母はきっと切り抜けられると医者は言った 
\\	切り抜く	きりぬく	
\\	切り抜ける 
\\	新聞の美術欄を切り抜いた 
\\	おっしゃられた通り切り抜いて、ほかのものと一緒にファイルするために、ゼロックスしておきました 
\\	切り抜き 〔新聞雑誌の〕
\\	((英)) 
\\	新聞の切り抜き 
\\	切り抜き細工 〔色紙の〕
\\	通例複数) 切り抜き帳 
\\	切り抜き人形 
\\	過敏	かびん	過敏な 〔神経が鋭すぎる〕
\\	〔人が神経質な〕((口)) 
\\	((英)) 
\\	抗生物質(こうせいぶっしつ)に過敏である 
\\	オートバイの騒音には皆過敏になっていた 
\\	過敏症 
\\	過敏症の 
\\	引き立つ	ひきたつ	
\\	(良く見える)
\\	(元気になる)
\\	〔よく見える〕 黒い背広で彼の長身が引き立った 
\\	松の緑が一面の桜の桃色の中で見事に引き立っている 
\\	〔元気づく〕 春とともに気持ちが引き立ってきた 
\\	[((口))
\\	市況は次第に引き立ってきた 
\\	[((口))
\\	例文 「お前の部の女の子、なかなかかわいいじゃないか」「男ばかりの職場だからよけい引き立つのかな」 
\\	「たまには黒のワンピースもいいかなと思ったんだけど、どう?」「あなたの色の白さが引き立ってなかなか似合うわよ」 
\\	懸かる	かかる	
\\	〔高い所に浮かぶ〕
\\	中天に虹が懸かっている 
\\	月が空に懸かっている 
\\	〔賞などが付いている〕
\\	このレースには500万円の賞金が懸かっている 
\\	〔左右される〕 優勝が懸かった試合 
\\	国の運命が彼らの双肩に懸かっている 
\\	店の興亡(こうぼう)は彼の商才(しょうさい)に懸かっている 
\\	弄する	ろうする	
\\	駄弁(だべん)を弄する 
\\	詭(き)弁を弄する 
\\	卑劣(ひれつ)な策を弄して情報を入手した 
\\	策を弄してその女からすっかり金を巻き上げた 
\\	労する	ろうする	
\\	⇒くしん(苦心),ほねおる(骨折る) 労せずして巨額の富を得る 
\\	〔煩(わずら)わせる〕 その仕事に彼は心身を労した 
\\	そんなことに人手を労することはない 
\\	ほとぼり	熱り	
\\	(余熱(よねつ))
\\	(興奮(こうふん))
\\	〔余熱〕
\\	〔ぬくみ〕
\\	〔興奮・関心などのなごり〕 事件のほとぼりが冷めるまで家に閉(と)じこもっていなさい 
\\	思う		
\\	(考える)
\\	(想像する)
\\	(…かしらと思う)
\\	(心配する)
\\	(気にする)
\\	〔願う〕
\\	〔慕う〕
\\	〔考える〕
\\	〔考慮する〕
\\	あなたのおっしゃるとおりだと思います 
\\	思っていることは何でも言いなさい 
\\	思っていることがうまく言えない 
\\	それを思うとわくわくした 
\\	両親の気持ちを思って家業に就くことにした 
\\	彼は党の将来の指導者と思われている 
\\	彼なら信用できると思っていた 
\\	思うところを実行すればよい 
\\	友人が遅れるものと思い,ゆっくり昼食をした 
\\	上司に話をつけようと思っている 
\\	今年は南米へ行こうと思っている 
\\	まっすぐ家に帰ろうと思った 
\\	〔あやしむ〕
\\	〔いぶかる〕
\\	〔懸念する〕
\\	彼がやったのではないかと思っている 
\\	彼が本当に理解するかしらと思った 
\\	父はもう長くはないと思う 
\\	明日彼女が来るとは思わない 
\\	〔推量する〕
\\	このペンいくらしたと思う 
\\	彼女はどう言うと思う 
\\	〔想像する〕
\\	思ったほどきれいな人ではなかった 
\\	彼は自分を偉い政治家だと思っている 
\\	こんなことが起ころうとは夢にも思わなかった 
\\	〔心に感じる〕 無事に帰れてうれしく思います 
\\	あのころをなつかしく思う 
\\	今日は昨日より寒いと思われる 
\\	心臓が止まったのではないかと思った 
\\	〔願う〕
\\	もう10年若かったらなあと思う 
\\	いつも大きな家が欲しいと思っていた 
\\	思うように言えません 
\\	思うようにならないからといってヒステリーを起こすことはないでしょう 
\\	〔心にかける〕
\\	暑さなんかなんとも思わない 
\\	僕の健康のことを思ってくれてありがとう 
\\	子を思う親の気持ち 
\\	〔慕う〕
\\	二人は出会ったとたん思い思われる仲になった 
\\	パーティーで会って以来彼女のことを思うようになった 
\\	〔思い出す〕
\\	3年前に死んだ妻を思うと胸が締め付けられるようだ 
\\	思えばあれは18の時だった 
\\	あのころを思うと隔世(かくせい)の感がある 
\\	隔世の感がある 
\\	常時	じょうじ	〔いつも〕
\\	〔習慣的に〕
\\	警官が常時見張っている 
\\	常時購読者 
\\	被る	こうむる	
\\	(損害を)
\\	(利益を)
\\	〔受ける〕
\\	〔損害を〕
\\	〔屈辱(くつじょう)を〕
\\	大きな損害を被る 
\\	社長の不興を被った ((文)) 
\\	彼女には多大の恩義を被っている 
\\	〔いただく〕 ちょっとご免を被って一眠りさせてもらいましょう 
\\	そんな仕事はご免被りたい 
\\	[((口)) 
\\	張り合う	(競争する)
\\	隣の店と張り合うことになった 
\\	彼らは優勝を張り合っている 
\\	私たちは女を張り合った仲だった 
\\	「あの二人は営業成績で張り合っているね」「まわりへのいい刺激になるよ」 
\\	「あの二人、彼女を張り合っていたけど、どうなったの?」「結局二人ともふられたんだ」 
\\	「いやあ、完敗(かんぱい)です」「囲碁で私に張り合おうなんて、10年早いよ」 
\\	頼む	たのむ	
\\	(依頼する)
\\	(当てにする)
\\	(注文する)
\\	〔依頼する〕
\\	彼にそれをするように頼んだ 
\\	彼に手紙[伝言]を頼んだ 
\\	彼に頼まれれば嫌とは言えないんだ 
\\	彼女には何も頼む気はしない 
\\	頼むから独りにしておいてくれ 
\\	頼む,頼むよ 
\\	トムに頼んでお茶を届けます 
\\	彼女からあなたに伝言を頼まれました 
\\	あの事は万事よろしく頼む 
\\	判断に任せる)]. 
\\	〔注文する〕
\\	〔予約する〕
\\	寿司を頼んだ 〔店で〕
\\	/〔出前を〕
\\	音楽会の切符を頼んでおいた 
\\	[((主に英))
\\	飛行機の席は頼んだかね 
\\	〔何かをしてもらうために来てもらう〕
\\	タクシー[医者]を頼みましょうか 
\\	この子に家庭教師を頼もう 
\\	〔任せる〕 このことは弁護士に頼みます 
\\	警察[専門家]に頼んだほうがよい 
\\	叔父に頼んであげよう 
\\	〔頼る〕
\\	彼は頼むに足りる人物だろうね 
\\	父親とも頼む阿部氏に死なれた 
\\	こうなれば彼をおいて他に頼む人がいない 
\\	数を頼んでやりたい放題のことをした 
\\	彼は万一を頼んでそんな行動に出たのだ 
\\	いえども	
\\	雖も】	
\\	〔…といっても〕
\\	当たらずといえども遠からず
\\	遠からず 
\\	[副]遠くない将来に。近いうちに。ほどなく。「遠からず実現するだろう」 貧(ひん)なりといえどもむさぼらず 
\\	〔…ですら〕
\\	幼児(ようじ)といえども彼の名は知っている 
\\	英雄といえども恐怖を覚える瞬間はある 
\\	とも		((接続助詞))
\\	ても 
\\	〔たとえ…でも〕 だれが来ようとも門は閉めておきなさい 
\\	何度尋ねられようとも答えは同じです 
\\	行かなくともよろしい 
\\	進歩は遅くとも着実だ 
\\	〔程度・見積もりを示して〕 少なくとも3回は電話した 
\\	いくら長くとも10日以上は掛からないでしょう 
\\	〔判断の迷いを示して〕 手紙には来るとも来ないとも書いてない 
\\	まだどっちとも答えていませんよ 
\\	家にはうんともすんとも言って来ない
\\	うんともすんとも 
\\	〔強め〕 そんな所に行きたいとも思わない 
\\	着実	ちゃくじつ	着実な 〔確実な,まじめな〕
\\	〔安全な〕
\\	〔信頼できる〕
\\	着実な男 
\\	研究に着実な歩(あゆ)みを続ける 
\\	着実なやり方[投資] 
\\	着実な実業家 
\\	彼は目標に向かって着実に進んでいる 
\\	打ち解ける	うちとける	(心を開く)
\\	(率直な)
\\	打ち解けた態度で 
\\	彼らはすぐに打ち解けた 
\\	彼女はあまり打ち解けない 
\\	彼はどうしても私たちに打ち解けようとしなかった 
\\	際して	さいして	(危機に)
\\	(出発に)
\\	出発に際して 
\\	困難に際して沈着(ちんちゃく)であれ 
\\	この百科事典の刊行に際して 
\\	沈着	ちんちゃく	
\\	沈着な 
\\	沈着な面持(おもも)ち 
\\	彼は沈着な態度を失わなかった 
\\	彼は沈着に行動した 
\\	地震の際は沈着であることが肝心だ 
\\	[((口))
\\	しばらく	【▲暫く】	
\\	(久しく)
\\	(当分)
\\	(待った!)
\\	〔少しの間〕
\\	しばらくお待ちください 
\\	しばらくして彼が現れた 
\\	は((米))ではやや堅い感じ) しばらくすれば彼が来るだろう 
\\	〔当分〕
\\	しばらくは引っ越さないことにした 
\\	まだしばらくは値下がりしないでしょう 
\\	〔少し長い間〕 やあ,しばらくだったね 
\\	しばらくぶりで東京に帰った 
\\	しばらく晴天(せいてん)が続いた 
\\	自失	じしつ	
\\	名](スル)自己を見失ってぼんやりすること。「突然の出来事に―する」「茫然 (ぼうぜん) 
\\	あまりの驚きに自失した 
\\	立ち退く	たちのく	
\\	(明け渡す)
\\	家を立ち退く 
\\	洪水警報が出たので村人は立ち退いた 
\\	20年間住んでいたアパートを立ち退かされた 
\\	捗々しい	はかばかしい	
\\	はかばかしくない 
\\	彼の病気ははかばかしくありません. 
\\	注釈	ちゅうしゃく	
\\	⇒ちゅう(注) これには多少の注釈が必要である 
\\	[((文))
\\	漢詩に注釈をつける 
\\	注釈つきの本 
\\	注釈者 
\\	あさぼらけ	【朝ぼらけ】	
\\	逞しくする	たくましくする	
\\	ボディービルという筋肉を逞しくする運動で筋肉を鍛えること 
\\	段取り	だんどり	
\\	(手順)
\\	(計画)
\\	(構想)
\\	〔進行計画〕
\\	〔手はず〕
\\	段取りを決める 
\\	祝賀会(しゅくがかい)の段取りを決めた 
\\	段取りがまずかった 
\\	会議は月曜日に開く段取りだ 
\\	おれが奴を押さえるとジョージが金を取るという段取りになっているのさ 
\\	当面	とうめん	
\\	〔面と向かうこと〕 当面する課題は山のようにある 
\\	〔さしあたり〕 当面の急務 
\\	当面の問題は日本の経済だ 
\\	当面の解決策としては500万円借り入れることしかない 
\\	手はず	てはず	【手×筈】 
\\	彼は休暇(きゅうか)をウェールズで過ごそうと手はずを整えた 
\\	(手はずを)整える 
\\	私たちは駅で出会う手はずだ. 
\\	引き締める	ひきしめる	
\\	(気を)
\\	〔強く締める〕
\\	切れんばかりに縄を引き締める 
\\	〔緊張させる〕
\\	気を引き締める 
\\	心配して顔を引き締める 
\\	部下を引き締めて統率(とうそつ)した 
\\	〔無駄な費用をなくする〕
\\	金融を引き締める 
\\	市場を引き締める 
\\	(銀行が) 信用貸しを引き締める 
\\	気鋭	きえい	気鋭の 
\\	気鋭の新人 
\\	新進気鋭のジャーナリスト 
\\	非の打ち所がない	ひのうちどころがない	
\\	演奏は非の打ち所がない出来栄えだった 
\\	表(おもて)		
\\	(表面)
\\	(貨幣の)
\\	(着物の)
\\	(前面)
\\	(玄関)
\\	(戸外)
\\	(外見)
\\	(外面)
\\	【スポーツ】(イニングの)
\\	〔表面〕
\\	カードの表を出す 
\\	封筒の表に航空便と朱書(しゅしょ)してください 
\\	これはどっちが表ですか 
\\	/〔布地・紙などで〕
\\	〔家の正面〕
\\	表に回ってください 
\\	表に忌中(きちゅう)の札(ふだ)が出ていた 
\\	〔屋外,戸外〕
\\	表で遊んでおいで 
\\	〔野球で各回の前半〕
\\	7回の表を終わって,依然両チーム無得点のままだった 
\\	〔外見,うわべ〕
\\	あの家族はどうも表を飾りたがる 
\\	彼女は表はほがらかだが,実際は悲観的(ひかんてき)です 
\\	表編み 
\\	表構え 
\\	表玄関 
\\	表地 
\\	表通り 
\\	表門 
\\	表(ひょう)		(目録)
\\	(一覧表)
\\	(図表)
\\	〔一覧表,目録〕
\\	〔目録・名簿など〕
\\	〔図表〕
\\	株価の一覧表 
\\	年表 
\\	換算[統計]表 
\\	定価表 
\\	免税品目表 
\\	予定表 
\\	一部[全部]の表 
\\	表にする 
\\	表に示したように 
\\	レモンは次表の3番目にある 
\\	この表でこの10年間の地価の変動が分かる 
\\	会員の名前がこの表に載っている 
\\	やり抜く	やりぬく	【▲遣り抜く】 (成し遂げる)
\\	(こつこつやる)
\\	彼はとうとうそのむずかしい仕事をやり抜いた 
\\	取り掛かる	とりかかる	
\\	今,仕事に取り掛かったところだ 
\\	6月から会の準備に取り掛かった 
\\	忙しくてまだそれには取り掛かっていません 
\\	ただちに仕事に取り掛かれ 
\\	しみったれ		
\\	うちの亭主はしみったれで、これまで買ってくれた一番高いものはキャンディー1箱だけなんだから 
\\	6カ月間仕事が見つからないので彼はしみったれた様子になってきた 
\\	亭主	ていしゅ	
\\	(夫)
\\	(店主)
\\	(茶会の)
\\	〔店の主〕
\\	宿屋(やどや)の亭主 
\\	〔夫〕
\\	亭主持ち 
\\	〔茶の席の〕
\\	茶会の亭主役をつとめた 
\\	亭主関白 あの家は亭主関白だ 
\\	[((米口)) 
\\	/((英口)) 
\\	関白		
\\	【日本史】 
\\	〈いばる人〉 ーいばる【威張る】 
\\	(自慢する)
\\	〔日本史で〕
\\	関白太政大臣 
\\	〔いばっている人〕 亭主関白 
\\	すこぶる	
\\	頗る】	
\\	⇒ひじょうに(非常に) 彼はすこぶる喜んでいる 
\\	すこぶるおもしろい芝居だった 
\\	すこぶる付きの秀才(しゅうさい) 
\\	小ぢんまり	こじんまり 
\\	こぢんまり	
\\	芸術の小ぢんまりした作品 
\\	抑える	おさえる	
\\	〔抑制(よくせい)する〕
\\	暴動を抑える 
\\	非難[世論]を抑える 
\\	個性を抑える 
\\	報道を抑える 
\\	/〔差し止める〕
\\	人口の増加を抑える 
\\	出費を抑える 
\\	赤字を低く抑える 
\\	輸出を抑える 
\\	増資額を30パーセントに抑えた 
\\	うわさの広まるのを抑える 
\\	〔圧倒する〕 相手チームを無得点に抑える 
\\	敵を抑える 
\\	〔感情をこらえる〕
\\	〔抑制する〕
\\	激情を抑える 
\\	涙を抑える 
\\	笑い[あくび]を抑えるのに苦労した 
\\	抑え難い悲しみ 
\\	怒りを抑える 
\\	自分の感情を抑えることができなかった 
\\	喜びを抑えることができなかった 
\\	〔程度を下げる〕 声[色調]を抑える 
\\	声を抑えて話した 
\\	彼は調子を抑えて走っている 
\\	発作を抑えるために注射した 
\\	傲慢	ごうまん	
\\	傲慢な 
\\	あいつの傲慢さをくじいてやろう
\\	くじく 
\\	注意するといつも傲慢な態度をとる 
\\	傲慢無礼 
\\	彼は誰に対しても傲慢無礼だ 
\\	非常に	ひじょうに	
\\	非常 
\\	彼は非常にうまくピアノを弾いた 
\\	それが非常に好きだ 
\\	非常に驚いた 
\\	非常に高価な絵 
\\	非常に長いスピーチ 
\\	非常に疲れている 
\\	[((口)) 
\\	非常にのどが渇いた 
\\	それを聞いて非常にうれしい 
\\	これは非常に重大なことです 
\\	騒音に非常に悩まされている 
\\	引き入れる	ひきいれる	
\\	(中へ入れる)
\\	(味方に)
\\	〔引っ張り入れる〕 彼は私を部屋に引き入れた 
\\	〔誘い入れる〕 彼を仲間に引き入れた 
\\	この条約は日本を戦争に引き入れる恐れがある 
\\	陰謀(いんぼう)に引き入れられてしまった 
\\	堪忍袋の緒が切れる	かんにんぶくろのおがきれる	
\\	酷評	こくひょう	
\\	酷評する 
\\	((文)) 
\\	((口)) 
\\	彼の小説は酷評を受けた 
\\	あからさま		(明白(めいはく)に)
\\	(率直に)
\\	(あけっ広げに)
\\	あからさまな (に) 〔ありのままの〕
\\	〔率直な〕
\\	〔単刀直入な〕
\\	〔露骨(ろこつ)な〕
\\	彼は不満をあからさまに顔に出した 
\\	彼女のあからさまな詰問(きつもん)にたじろいだ 
\\	そんなあからさまな言い方をするものじゃない 
\\	「そんなにあからさまに会社の悪口を言っていいの?」「いいんだよ。もう辞めることに決めたんだから」 
\\	「あの新人に残業を頼んだら、あからさまに嫌な顔をされたよ」「まったく最近の若い子は扱いにくいな」 
\\	級数	きゅうすう	
\\	等差[等比]級数 
\\	有限[無限]級数 
\\	万事	ばんじ	
\\	万事好都合である 
\\	万事お任せします 
\\	万事休(きゅう)す 
\\	万事休す 
\\	/((口)) 
\\	直訳すれば「出口なし」ということ。
\\	でもよい。 会社を破産から救うため、債券(さいけん)を売ったり顧問(こもん)を雇(やと)ったり、要するにできることは何でもやったが、万事休してしまった 
\\	万事こんな調子だった 
\\	一事が万事 
\\	万事心得ています 〔今やっていることを〕
\\	/〔何をやるべきか〕
\\	/〔任せておきなさい〕
\\	「彼が来たらどうするかわかっているね」「万事心得ています」 
\\	悔恨	かいこん	〔後悔〕
\\	〔残念さ〕
\\	悔恨の涙を流す 
\\	犯(おか)した罪に対する悔恨の情に駆(か)られる
\\	駆られる 
\\	胸にこたえる	むねにこたえる	
\\	彼の苦言は胸にこたえた 
\\	/((口)) 
\\	便乗	びんじょう	
\\	〔人が乗るのに乗(じょう)じること〕 新宿まで友達が車に便乗させてくれた 
\\	〔機会を利用すること〕 公共料金の値上げに便乗して業者は価格を上げた 
\\	便乗商法 
\\	便乗値上げ 
\\	便乗値上げする 
\\	例文 「明日、車で甲府(こうふ)まで帰省(きせい)するんだ」「じゃあ、途中の大月まで便乗させてくれないか」 
\\	「また石油不足が騒がれているね」「この機に便乗して値上げをする企業が続出するんじゃないかな」 
\\	「この広告、わが社は地球に優しい企業ですだってさ」「環境問題の時流に便乗したみたいで、かえっていやらしいね」 
\\	やって来る		
\\	彼は駆け足でやって来た 
\\	窮する	きゅうする	
\\	(言葉に)
\\	(金に)
\\	〔行き詰まる〕 言葉に窮する 
\\	〔足りなくて困る〕 金に窮する 
\\	窮すれば通ず 
\\	すかっと		
\\	思っていることをはっきり言ったらすかっとした 
\\	この飲物はすかっとした味がする 
\\	戸外で体操をすると気分がすかっとしますよ 
\\	鼻息	はないき	
\\	〔鼻でする息〕
\\	馬が荒い鼻息をした 
\\	病人の鼻息が荒い 
\\	〔意気込み〕 大した鼻息だね 
\\	鼻息の[が]荒い  ー
\\	鼻息の荒い若手(わかて)官僚 
\\	彼に1回勝ったぐらいでそんなに鼻息が荒いのか 
\\	[((俗))
\\	鼻息を伺う  
\\	上役の鼻息を伺う 
\\	/((文))
\\	いちゃつく		
\\	[((俗)) 
\\	彼らは人前でいちゃついていた 
\\	「あなたたち、いちゃつくんだったら会社の外でやってよね」「変なこと言わないでよ。肩のほこりを払ってあげてただけよ」 
\\	「この頃、電車の中で人目もはばからずいちゃついてるカップルが増えたね」「まったく、こっちは目のやり場に困るよ」 
\\	はばかる	
\\	憚る】	
\\	〔ためらう,遠慮する〕
\\	〔恐れる〕
\\	金の借り倒しをはばからなかった
\\	借り倒し 
\\	はばからずに意見を述べる 
\\	過(あやま)ちを改(あらた)めるのにはばかることはない 
\\	人目をはばかることはない 
\\	他聞をはばかることですので内密に願います 
\\	世間体(せけんてい)をはばかって事件は公表されなかった 
\\	〔幅をきかせる〕 憎まれっ子世にはばかる 
\\	/((諺)) 
\\	腹の虫が治まらない	はらのむしがおさまらない	腹の虫が治まらない 
\\	あれじゃ彼女の腹の虫が治まらないだろう 
\\	虫がいい		
\\	それではあまり虫がよすぎる 
\\	虫が起こる		
\\	〔子供がひきつけたりする〕
\\	〔何かを始める〕 彼はまた浮気の虫が起こった 
\\	虫が知らせる		
\\	虫が知らせたがやはり彼は急病になっていた 
\\	虫が好かない		
\\	虫の好かないやつ 
\\	彼は虫が好かない 
\\	虫の居所が悪い	いどころ	
\\	彼は今日虫の居所が悪い 
\\	弁ずる	べんずる	
\\	弁ず 
\\	是非を弁ずる 
\\	とうとうと弁ずる 
\\	この長椅子で寝台の用が弁ずる 
\\	多々ますます弁ず〔多いほどいい〕 
\\	不定	ふてい	不定の 〔不安定な〕
\\	〔不明確な〕
\\	不定の収入 
\\	住所不定の人 
\\	/〔総称〕
\\	不定冠詞 
\\	不定詞 
\\	不定愁訴(しゅうそ) 
\\	不定愁訴を訴(うった)える 
\\	不定積分 
\\	不定代名詞 
\\	不定法 〔文法で〕
\\	不定方程式 
\\	頼りない	たよりない	
\\	頼りない人 〔信頼性(しんらいせい)がない〕
\\	/〔心もとない〕
\\	/((口)) 
\\	あの人が社長では頼りないな 
\\	彼女から頼りない返事が来た 
\\	この分	このぶん	この分では 〈この調子では〉 
\\	〈今のありさまでは〉 
\\	この分でいくと 
\\	この分では 
\\	月中に我々の仕事は片づくまい. 
\\	この分でいくと今週末までには仕事は出来上がる 
\\	この分では何もかもだめになるだろう 
\\	この分では山には雪が降っているだろう 
\\	色よい	いろよい	
\\	((英)) 
\\	色よい返事 
\\	どっちつかず	【▲何▲方付かず】	
\\	どっちつかずの返事をする 
\\	「そんなどっちつかずの返事じゃ困りますねえ」「すみません。会社のトップの決裁がまだ下りないんです」 
\\	「彼は何て言ってるの?」「あいつはいつもどっちつかずで、最後に決まった方につくんだよ」 
\\	手付き	てつき	
\\	慣(な)れた手付きで 
\\	器用な手付きで花を生けた 
\\	不器用な手付きで赤ん坊を抱き上げた 
\\	あの子は妙な手付きではしを持つ 
\\	かたぎ	【▲気▲質】	
\\	学者かたぎの人だ 
\\	職人かたぎの男 
\\	芸術家かたぎの人 
\\	昔かたぎの人 
\\	早々	そうそう 
\\	はやばや(と)	
\\	と 
\\	はやばや, 
\\	そうそう 
\\	〔…してすぐ〕 社長に就任早々人員の削減(さくげん)に取り掛かった 
\\	新年早々の大惨事でおとそ気分も吹っ飛んでしまった
\\	屠蘇気分(とそきぶん) 
\\	屠蘇 
\\	二人は新婚早々だ 
\\	〔急ぐ様子〕 早々に 
\\	ボスのきげんが悪かったので早々に引き上げた 
\\	はやばや(と) 〔大変早く〕
\\	〔直ちに〕
\\	第一陣が早々と到着した 
\\	早々とお返事ありがとう 
\\	元通り	もとどおり	
\\	元どおりに置く 
\\	元どおりになる 〈事物が〉 
\\	〈人が〉 
\\	傷ついた彫刻を元どおりにする 
\\	こんな衝撃的な経験をしては元どおりにはなれるはずがない 
\\	町は元どおりの静けさにかえった 
\\	「こんなにお父さんの大工道具散らかしちゃって。元どおりにしておきなさいよ」「でも、お母さん。どんなふうになってたか忘れちゃった」 
\\	「あの二人、離婚は思いとどまったらしいね」「でも、あれだけけんかしたんだから、元どおりってわけにはいかないだろうな」 
\\	筋道	すじみち	〔道理〕
\\	〔話や行動などの順序〕
\\	筋道の通った[通らぬ]話 
\\	筋道の立った 
\\	筋道の立たない 
\\	筋道を立てて話す 
\\	話の筋道をたどる 
\\	彼のやることには彼なりに筋道がある 
\\	一定の筋道を踏む 
\\	山をかける		
\\	試験で山をかける 
\\	無茶な山をかける 
\\	山をかけてその株を買った 
\\	山を当てる		
\\	〔鉱山を掘り当てる〕
\\	〔予想が当たる〕
\\	試験で山を当てた 
\\	出頭	しゅっとう	
\\	裁判所に出頭する 
\\	警察に出頭した 
\\	明日税務署に出頭する 
\\	任意出頭 警察署に任意出頭を求められた 
\\	出頭命令 
\\	〔法廷への召喚状〕
\\	決め手	きめて	
\\	〔勝負の〕
\\	あのボクサーは決め手を欠いている 
\\	〔事件などの〕
\\	((口)) 
\\	弁護士は無罪の決め手となる証拠を提出した 
\\	問題解決の決め手がつかめない 
\\	科学者はこの発見が癌(がん)制圧の決め手になるものと信じている 
\\	検事は被告を不利にする有力な決め手を欠いていた 
\\	こき使(つか)う	
\\	扱き使う】	
\\	(あごで使う) 社長が彼らをこき使った 
\\	雇主(やといぬし)に安い賃金でこき使われた 
\\	ぐるり		
\\	〈まわり〉 
\\	【形式ばった表現】 
\\	〈一回転・一周するようす〉 《主に米》 
\\	《主に英》 
\\	〔周囲〕 家のぐるりに垣根(かきね)をめぐらす 
\\	警官が家のぐるりを固めた 
\\	敵がぐるりから攻(せ)め寄せてきた 
\\	〔回る様子〕 首すじをぐるりと回す運動 
\\	ぐるりと辺りを見回した 
\\	あっという間に数人の男がぐるりと彼を取り巻いた 
\\	庭は高い塀にぐるりと囲まれていた 
\\	出所	しゅっしょ 
\\	でどころ	しゅっしょ 
\\	〔でどころ〕
\\	そのニュースの出所は不明だ 
\\	この句の出所はシェークスピアである 
\\	このうわさの出所はだれだろう 
\\	〔刑務所から出ること〕 彼は刑期が終わって出所した 
\\	かりしゅっしょ【仮出所】 
\\	囚人(しゅうじん)を仮出所させる 
\\	仮出所を許される 
\\	仮出所中だ 
\\	ーーー でどころ 
\\	〔出所(しゅっしょ)〕
\\	うわさの出所をつきとめた 
\\	この流行語の出所はよく分かっていない 
\\	ところで金の出所はあるのか 
\\	〔出口〕
\\	〔潮時(しおどき)〕 今が出所だ 
\\	野営	やえい	
\\	〔野営所〕
\\	野営する 
\\	〔兵士が〕
\\	野営地 
\\	ぶざまな	【無様な|不様な】	(見苦しい)
\\	(だらしのない)
\\	(不器用な)
\\	(ずさんな)
\\	(結果が)
\\	〔不体裁な〕
\\	〔不器用な〕
\\	夜半(やはん)に起こされて彼女はぶざまな身なりで戸口に現れた 
\\	ぶざまに地面にひっくり返った 
\\	ぶざまな歩き方で去って行った 
\\	衝突した自動車がぶざまな残骸(ざんがい)をさらしていた 
\\	その彫り物はぶざまな細工だった 
\\	手を入れる	てをいれる	
\\	原稿に三回も手を入れた 
\\	帳簿に手を入れた〔不正に修正する〕 
\\	くらい・
\\	ぐらい		
\\	〈およそ〉 
\\	どのくらい 
\\	どのくらいかかったか. 〈費用が〉 
\\	〈時間が〉 
\\	〈程度が軽いことを表わす〉 このくらい 
\\	お茶の一杯ぐらいは出してもいいのに. 
\\	〈動作・状態の程度を表わす〉 
\\	日本人であれくらい英文を書く人は少ない. 
\\	〈いっそのこと〉 
\\	そんなことをするくらいならいっそ死んだ方がましだ. 
\\	〔大体の数や量〕 30人くらい 
\\	40歳くらいだろう 
\\	20日くらいかかるだろう 
\\	町までここから5マイルくらいだ 
\\	その動物は猟犬くらいの大きさだった 
\\	1年に何回ぐらいハイキングに行きますか 
\\	どれぐらいの長さ[幅/高さ/深さ]ですか 
\\	値段はどのくらいしますか 
\\	どのくらいかかりますか 〔時間〕
\\	/〔費用〕
\\	〔軽く見る気持ちで,その程度の〕 それくらいでは駄目だ 
\\	それくらいであわてるな 
\\	それぐらいは私でも知っている 
\\	あれくらいはなんでもなかった 
\\	それを買うぐらいの金なら持っている 
\\	返事ぐらいしたらどうだ 
\\	0点を取ったのは君ぐらいのものだ 
\\	ほめるのはこれくらいにしてよ
\\	きまりが悪くなる 
\\	〔…ほど〕 声も出ないくらい驚いた 
\\	気絶するくらいのショックだった 
\\	彼くらい背の高い人はめったにいない 
\\	不思議なくらい彼女は落ち着いていた 
\\	この映画くらい長期にわたって上映されたものはない 
\\	これぐらいいい音楽なら何度聞いても飽きない 
\\	これぐらい面白い本は読んだことがない 
\\	りんごは全部と言っていいぐらい腐っていた
\\	と言ってもいい 
\\	言ってもいいくらいだ 
\\	資源は無尽蔵(むじんぞう)と言ってもいいぐらいあった 
\\	〔「…するくらいなら」の形で,いっそのこと〕 こじきをするくらいなら飢え死(うえじ)にしたほうがましだ 
\\	途中で投げ出すくらいならなぜ始めるんだ 
\\	しぶき	【▲飛×沫】	〔水煙〕
\\	〔はね水〕
\\	彼はしぶきをあげて水中に飛び込んだ 
\\	[((米)) 
\\	水しぶきがかかった 
\\	モーターボートがしぶきを立てて走っている 
\\	覆る	くつがえる	
\\	〔引っくり返る〕
\\	盆が覆った 
\\	〔打ち倒される〕 政権が覆った 
\\	〔根本(こんぽん)から否定される〕 判決が覆った 
\\	彼の主張は覆った 
\\	体得	たいとく	体得する 
\\	〔経験によって〕
\\	スキーの要領をやっと体得した 
\\	たくらみ	【▲企み】	〔人をあざむく策略〕
\\	〔陰謀〕
\\	〔共同謀議〕
\\	彼になにか悪い企みがありそうだ 
\\	腹黒(はらぐろ)い企み 
\\	苦境	くきょう	
\\	〔難局〕
\\	〔逆境〕
\\	苦境に立つ 
\\	/((口)) 
\\	苦境にある ((口)) 
\\	多少ユーモラスに) 苦境を乗り越える 
\\	財政的苦境に直面している 
\\	妻のとっさの気転が私を苦境から救ってくれた 
\\	辺ぴな	へんぴ	【辺×鄙】な 
\\	もほとんど同じに使える。これらの表現には「辺ぴな」と同じく、
\\	(遠く離れた)、
\\	(人里離れた)、
\\	(めったに人の行かない)、
\\	(不便な)のニュアンスが含まれている。 うちの夏の別荘はちょっと辺ぴなところにあるけど、空気は最高にきれいだ 
\\	分際	ぶんざい	
\\	私のような分際ではそんなことは望めません 
\\	私のような田舎者の分際ではそれがちょうどよい
\\	ちょうどいい 
\\	子供の分際で口出(くちだ)しするな 
\\	すると		
\\	〔その時〕
\\	((文)) 
\\	するとそこに警官が通り掛かった 
\\	〔…であるとすると,それならば〕 すると彼はその場にはいなかったということだ 
\\	例文 私は箱を開けた。すると中に1匹のかわいい猫がいた 
\\	私は寝ようとした。すると電話がかかってきた 
\\	容疑者は全員白だ。すると犯人はだれだ? 
\\	「するとどうなった?」とみんなは一斉に私に聞いた 
\\	たらたら		
\\	〔したたり落ちる様子〕 顔には汗がたらたらと流れていた 
\\	手の平から血がたらたら流れていた 
\\	赤ん坊はよだれをたらたら流していた 
\\	〔長々と言う様子〕 彼女はいつも不平たらたらだ 
\\	彼のお世辞たらたらは我慢ならない 
\\	とうとう		とうとう【到頭】 (ついに)
\\	(結局)
\\	は肯定にも否定にも用いられる) 彼はとうとう考えを変えた 
\\	実験を繰り返してとうとう予期した結果を得た 
\\	とうとう来なかったんですね 
\\	とうとう諦めて引き上げた 
\\	とうとう
\\	滔×滔】 
\\	〔盛んに流れる様子〕 川の水はとうとうと流れていた 
\\	大雨の後のとうとうたる濁流(だくりょう) 
\\	〔弁舌のよどみない様子〕 とうとうとしゃべる 
\\	とうとうとまくし立てられて我々はあっけにとられた 
\\	〔世間の風潮が一方に進む様子〕 世はとうとうとして女性解放に向かっていた 
\\	まい		
\\	〔否定的推量〕 まだ雨は降るまい 
\\	彼は成功すまい 
\\	事態が悪化することはあるまい 
\\	そんなことはあるまい 
\\	〔打ち消しの意志〕 二度と行くまい 
\\	我々は彼を行かせまいとした 
\\	〔「…まいし」の形で,…ではないのに,…でないから〕 子供ではあるまいし,ばかなまねはするな 
\\	推量	すいりょう	〔推測〕
\\	〔推論〕
\\	⇒すいそく(推測) 推量する 〔推測する〕
\\	〔推論する〕
\\	推量で答える 
\\	君の推量は当たった 
\\	推量がはずれた 
\\	お言葉から推量しますとこの計画にご不満のようですね 
\\	推量によってこの結論を導(みちび)きだした 
\\	あてずいりょう【当て推量】 
\\	⇒あてずっぽう(当てずっぽう) 当て推量をする[言う] 
\\	〔あえて〕
\\	((文)) 
\\	君の言っていることは当て推量に過ぎない 
\\	当てずっぽう	あてずっぽう	
\\	当てずっぽうに歩いていたらいつの間にか出口に出ていた 
\\	彼は当てずっぽうを言っているに過ぎない 
\\	当てずっぽうに答えたら,なんと正解だった 
\\	「試験の成績よかったじゃないか、お前」「ほとんど当てずっぽうだったのに、運がよかったんだな」 
\\	「ねえ、課長、私たちのこと知ってるみたいな口ぶりだったわよ」「いや。当てずっぽうでカマかけられたんじゃない?」
\\	鎌(かま)を掛ける 
\\	あやす		
\\	(面倒を見る)
\\	(慰める)
\\	(寝かし付ける)
\\	〔抱いたり膝の上で揺らして〕
\\	〔寝かしつけるように〕
\\	〔物を与えるなどで機嫌をとる〕((米)) 
\\	((英)) 
\\	母親は赤ん坊をあやして寝かしつけた 
\\	彼は赤ん坊を膝の上であやした 
\end{CJK}
\end{document}