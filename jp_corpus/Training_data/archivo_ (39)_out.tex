\documentclass[8pt]{extreport} 
\usepackage{hyperref}
\usepackage{CJKutf8}
\begin{document}
\begin{CJK}{UTF8}{min}
\\	木星(もくせい) あの上の方で光っている星は木星だ
\\	申し訳ないんですが (もうしわけないんですか) 眞子さん、申し訳ないんですがお金を少し貸してもらえませんか
\\	国民(こくみん) 人民(じんみん)
\\	日本国民 国民の祝日
\\	廃墟(はいきょ) 廃墟になる/と化す(かする) 町は今,廃墟になっている
\\	明け方(あけがた) 赤ん坊は明け方近くに生まれた
\\	盆踊り(ぼんおどり)
\\	海流(かいりゅう) 日本海流
\\	垂直(な/の)(すいちょく) 水平(な/の)(すいへい) その崖(がけ)はほとんど垂直です
\\	読み物(よみもの) この本は面白い読み物です
\\	上る(のぼります)
\\	一言(ひとこと) 一言も言わずに 一言で言うと 終わりに一言
\\	警察署(けいさつしょ) 交番(こうばん) 警察署長
\\	基本的(きほんてき) 犬を飼うための基本的ルールは何ですか
\\	足元(あしもと) あなたは途中ずっと足元を見なければなりません
\\	~と同じぐらい 古代ギリシアは太陽系について私達と同じぐらいよく知っている
\\	葬る(ほうむります) 彼は海に葬られた その国は過去を葬りたかったのだ
\\	エアロビクス
\\	ユニーク(な) ユニークな芸術作品
\\	要約する(ようやく) 彼の演説の要約が新聞に載せられた
\\	客席(きゃくせき) この劇場には500人分の客席がある
\\	(~に)賛成する(さんせい) 私もあなたの意見に賛成です
\\	ユダヤ大虐殺(だいぎゃくさつ)
\\	環状道路(かんじょうどうろ) 環状の(かんじょう)
\\	忙しい仕事(しごと)
\\	所々(ところどころ) あちこち あちらこちら 雪がところどころ残っている
\\	国外(こくがい) 海外(かいがい) 上田さんは主に国外で活躍(かつやく)している
\\	埃(ほこり) 誇り(ほこり)
\\	地上(ちじょう) 地上の事 地上の楽園 地上に/で
\\	鮪(まぐろ)
\\	作成(さくせい) 契約(けいやく)を作成する
\\	どれ 動物でどれが一番好きですか
\\	狂信者(きょうしんしゃ) 狂信(きょうしん) 狂信する
\\	サービス業
\\	静脈(じょうみゃく) 動脈(どうみゃく)
\\	国籍(こくせき) 市民権(しみんけん)
\\	二重国籍者 (にじゅうこくせきしゃ)
\\	値(あたい) 値段(ねだん)
\\	この本はもっと安い値でかった
\\	広い(ひろい) 幅広い(はばひろい)
\\	八百屋(やおや)
\\	ボタン シャツのボタンを掛ける
\\	和菓子(わがし)
\\	学生時代 学生時代、イギリスに半年留学した
\\	追跡する(ついせき) その警官は泥棒を追跡した
\\	パンクする タイヤがパンクした 僕の自転車はパンクしている
\\	分子(ぶんし) 水分子(みずぶんし)は2個の水素原子と1個酸素原子からなる
\\	上昇(じょうしょう) 上がる(あがります) 物価はますます上昇しています 海水の温度が年々上昇している
\\	潜水する(せんすい) 潜る(もぐります)
\\	湿度(しつど) 湿度が高い 多湿(たしつ)です
\\	フード 風土(ふうど) 
\\	ずきん ずきんをかぶる
\\	宿泊(しゅくはく) 旅館に宿泊する 旅館に泊まる(とまります)
\\	証拠(しょうこ) 証拠は明らかだ
\\	暫くですね 久しぶり(ひさしぶり) 久々(ひさびさ)
\\	凍える(こごえます) 凍る(こおります)
\\	指が凍えてしまった 凍えた足
\\	資本主義(しほんしゅぎ) 共産主義(きょうさんしゅぎ)
\\	ぬれた ぬれたタオル
\\	思い付き アイデア 考え 何てよい思い付きなのでしょう!
\\	血(ち) 血液(けつえき)
\\	地名(ちめい) その地名は伝説に由来している
\\	受精する(じゅせい) 人工授精(じんこうじゅせい) 体外受精(たいがいじゅせい)
\\	首相(しゅしょう) 総理大臣(そうりだいじん)
\\	果たして(はたして) やはり やっぱり 彼は果たして成功した
\\	お気に入り(の)(おきにいり) 好き(な) これはあなたのお気に入りの曲ですか
\\	若い時(わかいとき)
\\	乏しい(とぼしい) 
\\	貧乏(な)(びんぼう)
\\	貧しい(まずしい)
\\	金が乏しい 乏しい学歴
\\	さて さて始めよう
\\	面倒くさい(めんどうくさい) 面倒(な)(めんどう)
\\	うざい
\\	まったく面倒臭い仕事だ
\\	熱心(な)(ねっしん) ワットさんは熱心だし、まじめだし、それに経験もある
\\	爪楊枝(つまようじ)
\\	垣根(かきね) 彼はその垣根を飛び越えた
\\	太子(たいし) 皇太子(こうたいし) 大使(たいし) 
\\	貯金する(ちょきん)
\\	倹約する(けんやく) 
\\	節約する(せつやく) 
\\	貯金がある
\\	半ば(なかば) 七月の半ば
\\	担架(たんか) 短歌(たんか)31-
\\	担架で運ぶ(はこびます) 担架に乗せる
\\	一度も(いちども) 一回も(いっかいも)
\\	蛋白質(たんぱくしつ) たんぱく質に富む(とみます)
\\	四字熟語 
\\	~に 悪戦苦闘する(あくせんくとう) 資金集めに悪戦苦闘する
\\	まな板(まないた) まな板の上の鯉(こい)のようなものだ
\\	谷(たに) その谷で金が見つかったといううわさがある
\\	動機(どうき) 殺害の動機は分からない
\\	バッテリー バッテリーが上がった(あがった) バッテリーを充電する
\\	魔法使い
\\	活発(な)(かっぱつ) 生き生き(いきいき) うちの母は活発です
\\	爪(つめ) 常用外
\\	アイスランド
\\	精巣(せいそう) 金玉(きんたま)
\\	敬老(けいろう) 敬老の日(けいろうのひ) (9月の第3月曜日)
\\	有罪(ゆうざい) すべての証拠は彼の有罪を示している
\\	インフルエンザ 風邪(かぜ)
\\	インフルエンザにかかりました 風邪をひく
\\	プディング
\\	関係(かんけい) 犯罪に関係する
\\	鮫(さめ) 常用外
\\	少なくとも(すくなくとも) この仕事を完成するには、少なくとも一ヶ月はかかります
\\	お経(きょう) お経をあげる[読む]
\\	初めての人 初心者(しょしんしゃ)
\\	南極(なんきょく)
\\	言い訳(いいわけ)
\\	口実(こうじつ) したことを言い訳するな
\\	食品店(しょくひんてん)
\\	する事が出来る
\\	(お)湯(ゆ) 薬缶の湯が沸いている
\\	有害(な)(ゆうがい) タバコは健康に有害です
\\	地理学(ちりがく) 日本の地理
\\	建築家(けんちくか) 建築家はモダンな家を建てる
\\	成績(せいせき) 点数(てんすう) 学校の成績は普通でした
\\	北極(ほっきょく) 南極(なんきょく)
\\	~を 聞き取る(ききとります) 人の話を聞き取る 英語を聞き取る力
\\	面接時間(めんせつじかん)
\\	キャットウォーク
\\	取り替える(とりかえます) ラジオの部品が壊れたので、新しいのに取り替えました
\\	メリークリスマス!
\\	司書(ししょ) 図書館(としょかん)
\\	乾かす(かわかす) 干す(ほします) 
\\	乾く(かわきます) 洗濯物を日光で乾かす/干す
\\	気が狂う(くるいます) 失恋して気が狂った
\\	書類(しょるい) 資料(しりょう)
\\	流砂(りゅうさ/りゅうしゃ) 
\\	よく、いらっしゃいました。
\\	氷河(ひょうが) 氷河時代
\\	集金(しゅうきん) 彼はガス料金の集金に回っています 集金人(しゅうきんにん)
\\	(略
\\	性感染症 (せいかんせんしょう) 現在20種を越える性感染症が確認されている
\\	愛読書(あいどくしょ)
\\	オリジナル(な)
\\	裏口(うらぐち) 
\\	彼は裏口入学した 裏口契約(うらぐちじょうやく)
\\	余り(あまり) 魚の余りを猫にやりました
\\	上げます(あげます) 売り手は値段を上げます
\\	悪者(わるもの) 
\\	腕白(わんぱく) 悪人(あくにん) 
\\	往復(おうふく) 東京から大阪までのチケット往復で一枚ください
\\	~を 番する(ばん) この店は、いつも子供が番をしている
\\	メインコース メインコースの後のデザート
\\	朱に交われば赤くなる (しゅにまじわればあかくなる)
\\	大陸(たいりく) 世界には7大陸がある
\\	光るものすべてが金ではない
\\	筋骨たくましい (きんこつたくましい) 筋骨たくましい大男 (おおおとこ)
\\	出演(しゅつえん) 映画に出演する
\\	頼む(たのみます) 注文する(ちゅうもん) レストランでスープしか頼みませんでした
\\	立春(りっしゅん)
\\	画家(がか) 私はこの絵を描いた画家を知っている
\\	受身(な)(うけみ) 受身(形) 受け身に回る
\\	我がまま(な)(わがまま)
\\	我利我利(な)(がりがり)
\\	利己的(な)(りこてき)
\\	窓口(まどぐち) カウンター 
\\	先にお金を払ってから、三番の窓口に行ってください
\\	黒死病(こくしびょう) ペスト
\\	分子(ぶんし) 
\\	分数(ぶんすう) 分母(ぶんぼ)
\\	発表する(はっぴょう) あした、試験の結果が発表される
\\	招待する(しょうたい) 上司が夕食に招待してくれた
\\	広める(ひろめます) 彼らは福音を全世界に広めた うわさを広める
\\	の
\\	あまり, 
\\	あまり (すぎる 
\\	この教科書は文法を重視(じゅうし)するあまり面白くないです。 秋子は恐怖のあまり声も出なかった。
\\	クリスマスツリー
\\	恋する(こい) 
\\	愛する(あい) 
\\	親子, 
\\	上田さんは田中さんの妹に恋している
\\	擬態語	ざあざあ 擬態語
\\	画像(がぞう) テレビの画像はぼやけて見える
\\	レーダー 魚群の位置はこのレーダー上に雲のような形で表される
\\	継父(ままちち)
\\	~ほど 猫は犬ほど好きじゃない
\\	キャンパス
\\	ご注文は?(ごちゅうもん)
\\	暴動(ぼうどう) 暴動を鎮圧する
\\	蚕 (カイコ) 蚕の祖先は東アジアに生息したと考えられている
\\	陥る(おちいります) 昏睡状態(こんすいじょうたい)に陥る 誘惑(ゆうわく)に陥る
\\	ハシバミの実 (み) ヘイゼルナッツ
\\	ベロ (俗) 
\\	舌(した) ベロを出す 靴のべろ
\\	どうも この文の意味がどうもよく分からない 鈴木さんはどうも京都大学に入りたいらしい このごろどうも体の調子(ちょうし)がよくないんです
\\	縁側(えんがわ) ポーチ 
\\	縁側から家に上がった
\\	電話が鳴っている(なる)
\\	約束(やくそく) 
\\	教員(きょういん) 教師(きょうし) 先生(せんせい) 数学の教員
\\	タンゴ 単語(たんご)
\\	占領(せんりょう) その都市は1662年にイギリス人に占領された
\\	素直(な)(すなお) 素直な子だ
\\	意地悪(な)(いじわる) 意地悪をする 意地悪には見えないけど
\\	おなかがいっぱいです。
\\	家へ行く道が分からない
\\	青年(せいねん) 若者(わかもの)
\\	青春(せいしゅん)
\\	果汁しぼり器 (かじゅうしぼりき) 果汁(かじゅう)
\\	ヒレ
\\	この頃(このごろ) 最近(さいきん)
\\	擦り切れる(すりきれます) 破れる(やぶれます) 彼のオーバーは擦り切れている
\\	十二指腸(じゅうにしちょう)
\\	食卓(しょくたく) 食卓から皿を片付ける
\\	投げる(なげます) 子供がボールを投げている
\\	コンセント
\\	産業(さんぎょう) 工業(こうぎょう)
\\	彼は自動車産業に従事(じゅうじ)している
\\	砂糖を入れる
\\	青春(せいしゅん) 青春の夢
\\	けいこと一緒に
\\	索引(さくいん) その本には索引が付いていますか
\\	仕事(しごと) まだ仕事をしてるんですか。
\\	目が覚める(さめます) 起きる(おきます) すると、目が覚めました
\\	大半(たいはん) 過半数(かはんすう)
\\	社員の大半 一日の大半を読書して過ごした 大半は/が
\\	屋上(おくじょう) 
\\	屋根(やね) 屋上庭園(おくじょうていえん)
\\	プレーボーイ 女垂らし(おんなたらし)
\\	革命(かくめい) その発見は建設的であるとともに革命的だった
\\	命を落とす(おとします) 死ぬ(しにます) 亡くなる(なくなります) 死亡する(しぼう)
\\	魚卵(ぎょらん) イクラ
\\	古文(こぶん)
\\	期待する(きたい) 気体(きたい) 
\\	望む(のぞみます)
\\	を期待して 昇進を期待して一生懸命働いた
\\	借り(かり) 借金(しゃっきん) 私は君にいくら借りがありますか 彼の借金はドルに達した
\\	死体(したい)
\\	高校生(こうこうせい)
\\	雛鳥(ひなどり) 
\\	小鳥(ことり) 公園をあるいているとき、私はひな鳥を見つけた
\\	間隔(かんかく) 感覚(かんかく) 
\\	バスは15分間隔で運転される
\\	未だに(いまだに) 今でも(いまでも)
\\	勉強中(べんきょうちゅう) 勉強している
\\	息をする(いき) 
\\	呼吸する(こきゅう)
\\	クレーン
\\	パン粉(こ)
\\	学問する(がくもん) 
\\	あの人は学問がある
\\	様々(な)(さまざま)
\\	へとへと(です/になる) 疲れた(つかれた) 二時間の通勤すると、会社に着く頃にはもうへとへとでうよ
\\	幾ら(いくら)
\\	(俗)	げじげじ 
\\	百足(むかで) 
\\	げじげじ眉(まゆ)
\\	意識を失う (いしきをうしなう) 失神する(しっしん) 気絶する(きぜつ)
\\	(日が) 暮れる(くれます) 
\\	列車は日が暮れてからローマに着いた 日が暮れる前に 秋が暮れた
\\	西日(にしび)
\\	濃い(こい) 濃いお茶(こい) 濃いスープ 濃い赤
\\	先史の(せんし) 先史時代 先史時代の日本
\\	箱(はこ) ゴミ箱(ごみばこ)
\\	黒海(こっかい) 国会(こっかい)
\\	時計屋 
\\	時計屋さん 
\\	品質(ひんしつ) 品質はいいでも値段は高い
\\	くすくす笑う
\\	作り上げる(つくりあげます) 
\\	でっち上げる その話はだれかが作り上げたに違いない
\\	うに
\\	~を 捕食する(ほしょくしゃ) 
\\	~を 狩る(かります)
\\	変身(へんしん) 返信(へんしん)
\\	サナギは蝶へと変身した 彼はスーパーマンに変身した
\\	満開(まんかい) その植物園はバラの花が満開だった
\\	内科(ないか) 内科はどこですか
\\	出発(しゅっぱつ)
\\	支える(ささえます)
\\	支持する(しじ) 
\\	杖で体を支える 彼は大家族を支えて行かねばならない
\\	官僚(かんりょう) 官僚主義(かんりょうしゅぎ) 官僚(かんりょう) 官僚主義者(かんりょうしゅぎしゃ)
\\	石油(せきゆ) 日本はアラブから石油を買った
\\	以外(いがい) 意外(いがい)
\\	ジョン以外はみんな到着した
\\	直接接触 (ちょくせつせっしょく)
\\	生物(せいぶつ) 生き物(いきもの)
\\	(普通形) ~などと言う お金がないから、日本に行くなどということは夢だ
\\	農場(のうじょう)
\\	インド洋(よう)
\\	離婚(りこん) 二人が離婚するといううわさが流れる
\\	許可(きょか) 許可を取る 許可を与える 許可がなければ、ここで写真が取れません
\\	教え子(おしえご) だれが山田先生の教え子ですか
\\	カフェテリア
\\	返事(へんじ) 
\\	答え(こたえ) 回答(かいとう)
\\	~と~ 天気が悪いと、私は眠くなる
\\	副鼻腔(ふくびこう)
\\	貧乏人(びんぼうにん)
\\	臭い(におい) 
\\	香り(かおり) 
\\	バラはいい臭いがする
\\	機嫌(きげん) 社長は今日機嫌がいいですか
\\	図案(ずあん) 彼は象牙に図案を彫った 図案家
\\	王子(さま)(おうじ)
\\	おかげさまで。
\\	仕上がる(しあがります) 出来上がる(できあがります) 洗濯は仕上がっていますか 仕事は途中まで仕上がっている
\\	お札(おさつ)
\\	想像(そうぞう) 創造(そうぞう)
\\	空想(くうそう) 自分が富豪(ふごう)だと想像してみた
\\	カジノ
\\	牛乳(ぎゅうにゅう) ミルク
\\	戸籍(こせき)
\\	リーキ 西洋ニラネギ(せいよう)
\\	~階建ての(かいだて) 3階建ての建物 11階建てのマンション
\\	学園(がくえん) 学校では学園音楽祭の準備しています 学園都市(がくえんとし)
\\	のど 咽喉(いんこう)
\\	サタン 悪魔(あくま)
\\	針金(はりがね) ワイヤ 針金のような毛〔犬などの〕
\\	免疫系(めんえきけい)
\\	邪魔する(じゃま) 仕事の邪魔をする ラジオの音が邪魔になって、勉強が出来ません 午後、お邪魔してもよろしいでしょうか
\\	和平条約(わへいじょうやく) 和平条約の調印(ちょういん)
\\	なんとなく 暗い色はなんとなく重く感じるようです
\\	プチ 
\\	プチ旅行
\\	走者(そうしゃ)
\\	心臓(しんぞう)
\\	テーブルクロス
\\	いぼ 首にいぼができた
\\	住所(じゅうしょ)
\\	せいぜい 
\\	私にはせいぜいあなたの半分しができない
\\	細い(ほそい) 私は目が細い。
\\	リンス
\\	放送する(ほうそう) ニュースは毎時放送している
\\	道の向こう側(みちのむこうがわ) 道の向こう側をセクシーな女の人歩いている
\\	~が 凍る(こおります) ~を 凍らせる(こおらせます) 11月には湖が凍ってしまう
\\	海星(ひとで) 人出(ひとで)
\\	辺(へん) この辺だと思う
\\	販売法(はんばいほう)
\\	家庭的(な)(かていてき) 家庭的な雰囲気 家庭的な男性 あの旅館は家庭的だ
\\	一朝一夕(いっちょういっせき) 
\\	ない これは一朝一夕に解決できる問題じゃない
\\	ハネムーン 新婚旅行(しんこんりょこう) その新婚夫婦は新婚旅行でハワイに出かけて行った
\\	送料(そうりょう) 外国郵便の送料
\\	~に 関連する(かんれん) ~と 関係する(かんけい)
\\	いびきとタバコのすい過ぎに関連がある
\\	味噌っ歯(みそっぱ) 虫歯(むしば)
\\	味噌っ歯になる うちの子供は甘いものの食べ過ぎで、味噌っ歯になってしまった
\\	症状(しょうじょう) 症状がひどくなったら、医者に相談してください 徐々に(じょじょ)症状が悪化した
\\	接近する(せっきん) 近づける(ちかづけます) 台風が九州に接近していると発表された
\\	虐殺(ぎゃくさつ) 彼は虐殺された 村中の人が虐殺された ユダヤ大虐殺 大量虐殺(たいりょうぎゃくさつ)
\\	戸棚(とだな) 
\\	工業(こうぎょう)
\\	産業(さんぎょう)
\\	重工業(じゅう) 軽工業(けい) この国は工業が発達している
\\	静物画(せいぶつが)
\\	テニスコート
\\	季節(きせつ)
\\	ひょう窃(ひょうせつ) ひょう窃する
\\	異常(な)(いじょう) 以上(いじょう)
\\	異常な才能 こんな暖かい天気は二月にしては異常だ
\\	何してんの?
\\	灯台(とうだい) 遠い灯台のかすかな光を見た
\\	自首する(じしゅ)
\\	自主(じしゅ)
\\	降参する(こうさん)
\\	降伏する(こうふく)
\\	その犯人は警察に自首した
\\	戦う(たたかいます) 敵と戦う
\\	左右する(さゆう) 影響する(えいきょう) 
\\	支配する(しはい) 
\\	友達に左右されやすい人だ 一国の運命を左右する
\\	召し上がれ(めしあがれ) 召し上がる (めしあがります) 召し上がり物 (めしあがりもの)
\\	血を流せる(ながせる)
\\	魚群(ぎょぐん) 魚群の位置はこのレーダー上に雲のような形で表される
\\	民族(みんぞく)
\\	狭い(せまい)
\\	池(いけ) 湖(みずうみ)
\\	日夜(にちや) 日夜働き続ける
\\	海賊(かいぞく) 海賊たちには降参する以外に道はなかった
\\	歌(うた)
\\	地域(ちいき) 地方(ちほう)
\\	その地域の方言 
\\	の国と地域から 
\\	名の若者が集まった
\\	オリエンテーション 新入生向けのオリエンテーションを行う
\\	鯨(くじら)
\\	より 猫のほうが犬より好きです。
\\	スウプを飲む(のむ)
\\	ブラックボックス
\\	塩入れ(しおいれ)
\\	背広(せびろ)
\\	スーツ 
\\	未満(みまん) 18歳未満の方の入場は禁じます
\\	おかまいなく。
\\	脛(すね) 弁慶の泣き所 (べんけいのなきどころ)
\\	吹き替え(ふきかえ) 映画を吹き替える(ふきかえます)
\\	振り返る(ふりかえります)
\\	日本語に吹き替えた『ハムレット』 アメリカ映画を日本語に吹き替える
\\	むしろ 私は甘いものよりむしろ辛い物のほうが空きだ
\\	学園都市(がくえんとし) 学園(がくえん)
\\	自発的(な)(じはつてき) 彼はそれを自発的に行った 生徒は自発的に教室の掃除をした
\\	炎(ほのお) 家々は次々と炎を噴き出して燃えた
\\	ヘッドホン ヘッドホンから音が漏れているのは彼です
\\	豆をまく
\\	課(か) 六課 (ろっか) 八課(はっか) 
\\	三課 (さんか)
\\	冥王星(めいおうせい)
\\	巨人(きょじん)
\\	人工知能(じんこうちのう) 
\\	は人工知能の略だ
\\	年代順に(ねんだいじゅん) 年代順の
\\	家事手伝い(かいてつだい) お手伝いさん 
\\	得る(えます) 手に入れる もらう その男は権力を得るために多額の金を使った
\\	目指す(めざします) 教育はテストに合格することを目指すべきではない
\\	愛読する(あいどく) 彼は太宰治を愛読している 私は彼の小説を愛読した
\\	鯉のぼり(こいのぼり) 鯉のぼりを立てる(こいのぼり)
\\	短い(みじかい)
\\	知識(ちしき) 幅広い知識(はばひろい)
\\	コレステロール コレステロール値(ね)が上がってきています
\\	当然(とうぜん) 明らか(な)(あきらか)
\\	彼女は当然私の見方になると思った
\\	滴(しずく) 一滴 (いってき) 一滴の涙 (いってきのなみだ) ぶどう酒のグラスを最後の一滴まで飲み干した (のみほす)
\\	アメリカ豹(ひょう) 豹(ひょう)
\\	幼稚園(ようちえん) 保育園(ほいくえん)
\\	むりもありません 
\\	支社(ししゃ) 死者(ししゃ)
\\	支店(してん)
\\	この書類を支社に届けてくれませんか
\\	一日おきに 五メートルおきに
\\	森林(しんりん) 森(もり) 私は森林の中で道に迷った(まよう)
\\	セクハラ
\\	おやすみ。 おやすみなさい。
\\	ルネサンス ルネサンスの芸術(げいじゅつ)
\\	海草(かいそう)
\\	コーチ トレーナー
\\	切り裂く(きりさきます) 布を切り裂く 落雷で木はまっぷたつに切り裂かれた 切り裂きジャック(きりさき)
\\	徐々に(じょじょ) ゆっくり 
\\	段々に(だんだん)
\\	次第に(しだい)
\\	列車は徐々にスピードを上げた
\\	おしゃべりや
\\	じゃ、またね
\\	ネパール
\\	丁度(ちょうど) ちょうどいいです
\\	脇(わき) 右脇が痛い 雑誌はデーブルの脇に積んである
\\	~が 下手(な)(へた) ~が 苦手です(にがて) 
\\	~を 弄る(いじります) 私のコンピュータをいじらないでくれ ペンを弄る
\\	現代的(げんだいてき) 近代(きんだい)
\\	現代日本文学
\\	アーモンド
\\	蝋人形(ろうにんぎょう) 蝋人形館(ろうにんぎょうかん)
\\	開戦する(かいせん) 戦争の勃発(ぼっぱつ) 終戦(しゅうせん) 開戦の理由
\\	心臓発作(しんぞうほっさ) 彼は心臓発作した
\\	孔子(こうし) 孔子の言説は有名です
\\	胴(どう) 銅(どう)
\\	ダックスフントは、非常に長い胴と短い足をした犬です
\\	浮き輪(うきわ)
\\	~倍(ばい) 5の倍は10だ この木はあの木の2.5倍の高さだ
\\	討論(とうろん) 議論(ぎろん)
\\	私は彼とその問題を討論した 討論者(とうろんしゃ)
\\	頼り(たより) 便り(たより)
\\	頼りにして大丈夫だよ 私が知る限りでは、彼は頼りになる人だ 彼は頼りにできる男だよ
\\	叱る(しかります) 小言する(こごと) 汚い手で着物に触ったら、叱られました
\\	赤道ギニア (せきどう)
\\	取る(とります) 渡す(わたします) 手渡す(てわたします) すみませんが、塩を取って下さい
\\	掃除婦(そうじふ)
\\	西洋料理(せいようりょうり)
\\	撤兵(てっぺい) イラクから撤兵する
\\	雄牛(おうし) 雄(おす) 雌(めす)
\\	滝(たき)
\\	家具(かぐ) ファーニチャー
\\	経済(けいざい) 経済成長(せいちょう)
\\	消防車(しょうぼうしゃ) 群集がすぐに消防車の回りに集まった
\\	脅える(おびえます) ~に びくびくする ~を 脅かす(おびやかします)
\\	鶫(つぐみ) 常用外
\\	宗教(しゅうきょう)
\\	~を 助ける(たすけます) ~を 救う(すくいます) おぼれている人を助ける
\\	急速(な)(きゅうそく) 速い(はやい) 急速な進歩を遂げる
\\	しんどい  
\\	疲れた へとへとになった
\\	コーヒーミル 受け取ったコーヒーミルが動きません
\\	酒場(さかば) バー
\\	オリーブ油(ゆ) オリーブオイル
\\	判断する(はんだん) 見掛けで人を判断してはいけない
\\	凸凹(な)(でこぼこ) 凸凹の道路 月の表現は凸凹だ
\\	~を 持ち上げる (もちあげる)
\\	ハンモック 木と木の間にハンモックを吊った(つる)-
\\	肛門(こうもん) 校門(こうもん)
\\	人質(ひとじち) 人質はクリスマスの前に解放(かいほう)されるだろう
\\	神経(しんけい)
\\	にっこりする ニコニコする
\\	頬笑む(ほほえみます) こういう風に彼女は私を見てにっこりした
\\	良さ(よさ) 
\\	長所(ちょうしょ)
\\	この制度の良さは今に分かる
\\	一晩中(ひとばんじゅう)
\\	~あります 肉は焼いてあります
\\	蛍(ほたる) 蛍が光る
\\	消す(けします)
\\	スクリーンセーバー
\\	カウンター 
\\	家事(かじ) 火事(かじ)
\\	家事をしながら働き出る女性がたくさんいる
\\	~に 富む(とみます) ~が 豊富です(ほうふ) ~が 豊か(ゆたか) カナダは木材に富む
\\	鯨(くじら) 鯨は独自の言語があると信じられている
\\	園芸(えんげい) 父は園芸の基本を教えてくれた
\\	踏まえる(ふまえます) 基づく(もとづきます) 事実を踏まえた 経験を踏まえた助言
\\	栗(くり) 彼は通りから栗を取り除いた(とりのぞいた) 栗のいが いが 
\\	鎖(くさり) その犬は鎖でつないでおくべきだ
\\	職業安定所(しょくぎょうあんていじょ) 職業安定所に登録する
\\	職場(しょくば) 仕事場(しごとば) 私たちはコンピューターを職場へ運んだ
\\	一期一会(な) (いちごいちえ)
\\	薬草(やくそう)
\\	マスク マスクをつける 防毒マスク(ぼうどく)
\\	水彩絵の具 (すいさいえのぐ) 水彩画(すいさいが) 水彩画をかく
\\	言うなれば(いうなれば) 言わば あの男は言うなれば大きくなった子供です
\\	葡萄(ぶどう) 武道(ぶどう) 
\\	寒い(さむい)ですよ。
\\	明日(あした) 明日(あす) 明日(みょうにち)
\\	自分(じぶん) 私は自分で数えた
\\	人参(にんじん)
\\	と言う、と思う 
\\	~って 今、時間がないって、本当?
\\	言語学者(げんごがくしゃ)
\\	上がる(あがります) 二階に上がる
\\	帰宅する(きたく) 家に帰る 夜になったので、帰宅する
\\	わいろを受け取る
\\	ちょっとお願いがあるんですが
\\	提出する(ていしゅつ)
\\	出す(だします) 時間です
\\	答案を提出しなさい
\\	ルームメート
\\	またがる(またがります)
\\	北アルプスは 
\\	県にまたがる. これは 
\\	年にまたがる計画である
\\	捻る(ひねります) 
\\	回す(まわします)
\\	水道の栓(せん)を捻っても水が出ません 口ひげをひねる 腰を左へひねる
\\	おむつ おむつを外す おむつを替える[取り替える]
\\	「我輩は猫である」 夏目漱石 我輩 (わがはい) 
\\	電動の(でんどう) 電動タイプライター この機械は電動だ
\\	賑やかなパーディー
\\	乾いた (かわいた) 乾燥した(かんそう)
\\	乾いたタオル
\\	消去する(しょうきょ)
\\	消す(けします)
\\	削除(さくじょ)
\\	過去の思い出を消去する
\\	サービス
\\	チンパンジー
\\	親切(な)(しんせつ) 優しい(やさしい)
\\	旅行者(りょこうしゃ) 旅人(たびびと)
\\	うざい 面倒くさい(めんどうくさい) 面倒(な)(めんどう)
\\	考古学(こうこがく)
\\	飼う(かいます) このマンションでペットを飼えません
\\	社員旅行(しゃいんりょこう)
\\	尋ね(たずね) 行方不明(ゆくえふめい) 尋ね人(たずねびと) たずね犬
\\	偏食(へんしょく) 偏った食事(かたよった) 偏食する
\\	退職(たいしょく) たいていの人は60歳で退職する
\\	乞食(こじき) 私は乞食に持ちかね全部を与えた
\\	微笑む(ほほえみます) 
\\	笑む(えみます) 笑み(えみ)
\\	笑顔(えがお)
\\	微笑み(ほほえみ)
\\	ユーモアの感覚(かんかく) 彼には素晴らしいユーモアの感覚がある
\\	魅惑(みわく) 魅力(みりょく)
\\	日本の魅惑
\\	台湾(たいわん) 台北は台湾の首都です。
\\	とうとう やっと いよいよ 遂に(ついに)
\\	カルダモン
\\	眠り薬(ねむりぐすり)
\\	副詞(ふくし)
\\	発育(はついく) 赤ん坊は正常な発育を示す
\\	動かす(うごかす)
\\	姪 (めい)
\\	姪御さん (めいごさん)
\\	行き(ゆき) 大阪行きの新幹線
\\	大好物(だいこうぶつ)
\\	拝見する(はいけん) パスポートを拝見してよろしいですか
\\	怠る(おこたります) 義務を怠ってはならない
\\	本当ですか
\\	破く(やぶきます) 破る(やぶります) 彼は手紙を二つに破いた
\\	全体(ぜんたい) 全体に山が多いです
\\	船便(ふなびん) 船便で出す
\\	編む(あみます) セーターを編むことが出来る
\\	芝居(しばい) 演劇(えんげき) 芝居に出る
\\	それに
\\	先程(さきほど) さっき 最近(さいきん)
\\	そのことは先程聞きました
\\	油絵(あぶらえ) この油絵は17世紀のものです
\\	協定(きょうてい) 協定を破る(やぶります)
\\	人食いの(ひとくい) 人食い鮫(ひとくいざめ) 人食い
\\	女王(じょおう)
\\	底(そこ) 心の底から感謝した 鍋(なべ)の底に穴があいている
\\	神父(さん)(しんぷ) 
\\	新婦(しんぷ) 聖イグナシオ教会の神父
\\	ダブルベッド シングルベッド キングサイズベッド
\\	私立(大学)(しりつ)
\\	はじめまして。
\\	クリーム クリームと砂糖をすこし入れてください
\\	巧い(うまい) 旨い/美味い/甘い/上手 
\\	処刑する(しょけい) チャールズ1世は断頭台で処刑された
\\	海豚(いるか)
\\	ハワイ諸島(しょとう)
\\	喉(のど) 咽喉(いんこう) のどが渇く(かわきます)
\\	カルシウム 彼の食事にはカルシウムが欠けている
\\	人体(じんたい) 人体解剖学 (じんたいかいぼうがく) 人体に無害である
\\	情熱(じょうねつ) 作家は恋人に対する情熱を表現している
\\	ストーブ
\\	湿す(しめす) ぬらす 
\\	示す(しめします)
\\	タオルを湿らす 水に湿したタオルを額に当てる
\\	月の終わりごろ
\\	部品(ぶひん) 部分(ぶぶん)
\\	テレビの部品
\\	愚か(な)(おろか) 馬鹿(ばか) 愚かさ(おろかさ) 愚か者(おろかもの)
\\	群がる(むらがります) 集まる(あつまります)
\\	その車の周りに人が群がっている ごみ箱にはえが群がっている
\\	サバンナ 食後にサバンナ行ったとき
\\	~ば お金があればいいね
\\	翼竜(よくりゅう)
\\	多忙(な)(たぼう) 忙しい(いそがしい) 
\\	先生は試験の採点(さいてん)で多忙だ
\\	無理するな
\\	星影(ほしかげ)(書)
\\	火事(かじ) 家事(かじ) 
\\	火災(かさい)
\\	語り手(かたりて) ナレーター
\\	飛ばす(とばします) 子供は紙飛行機を飛ばした 矢を飛ばす
\\	(~を)隠す(かくします) (~が)隠れる(かくれます)
\\	十三日の金曜日(じゅうみっか)
\\	ジュークボックス
\\	床(ゆか) フロア
\\	北極海(ほっきょくかい)
\\	血脈(けつみゃく) 
\\	兄弟(きょうだい)
\\	~っぽい (俗) ~みたい ~よう(な) 黒っぽい 子供っぽい サングラスをかけたヤクザっぽい男が街角にたっていた
\\	悪趣味(な)(あくしゅみ) 彼女の着る物は悪趣味だ
\\	愛犬(あいけん) 愛犬家(あいけんか)
\\	行っていらっしゃい
\\	興奮(こうふん) 私はその晩興奮のあまり寝られなかった
\\	風来坊(ふうらいぼう) 浮浪者(ふろうしゃ)
\\	人柄(ひとがら) 
\\	性格(せいかく) 
\\	人格(じんかく) 
\\	人は交わる友によってその人柄が分かる
\\	横入りする(よこいり)
\\	チョコレート(の)詰め合わせ (つめあわせ) 詰め合わせ(つめあわせ) 
\\	新築する(しんちく) 新築の家  新築中の家
\\	出生(しゅっしょう) お産(おさん) 出産(しゅっさん) 出生する 生まれる(うまれます) 出生率(しゅっしょうりつ) 出生地(しゅっしょうち)
\\	方法(ほうほう) この方法で
\\	脹脛(ふくらはぎ)
\\	忠誠(ちゅうせい) 忠実(ちゅうじつ)
\\	忠誠を誓う 忠誠を尽くす 犬は飼い主に忠誠です
\\	環状の(かんじょう) 感情(かんじょう) 
\\	勘定(かんじょう)
\\	環状道路(かんじょうどうろ)
\\	~んです・なのです ベジタリアンなんです
\\	文字(も(ん)じ)
\\	文学(ぶんがく)
\\	胸焼け(むねやけ) 胸焼けがする
\\	フォーク
\\	燃料(ねんりょう) 石油はもはや安い燃料ではない
\\	お待ちどうさま お待たせしました
\\	包皮(ほうひ) 包皮を割く(さきます)/切る 割礼(かつれい) 包皮切断(ほうひせつだん)
\\	飲み込む(のみこみます) 食べ物を飲み込むとのどが痛みます
\\	(星座)	魚座(うおざ)
\\	ぬれる(ぬれます) ぬれた ~を ぬらす 彼女の目は涙でぬれていた
\\	ぼったくり ぼったくりの値段
\\	有利(ゆうり) バスケットボールでは背が高い選手は有利だ 有利な立場にある
\\	用語(ようご) 科学用語 その手紙は商業用語で書かれた
\\	予測する(よそく)
\\	予知する(よち)
\\	人の運命を予知することは困難です 予知能力者(よちのうりょくしゃ)
\\	家庭(かてい) 今でもまだほとんどの家庭でいろいろな行事します
\\	ビーバー
\\	乳癌(にゅうがん) 肺癌(はいがん) 前立腺癌(ぜんりつせんがん)
\\	熱がある(ねつ)
\\	ピラミッド その屋根の格好はピラミッドに似ている
\\	雑種の犬(ざっしゅのいぬ) 雑種犬(ざっしゅけん)
\\	知恵(ちえ) 懸命(けんめい) 知恵のある人 知恵を働かせる
\\	向こう(むこう) 無効(むこう)
\\	向こう側(むこうがわ) 
\\	向こうに見えるのが富士山です 向こうの生活はいかがでしたか
\\	インタービュー 面接(めんせつ)
\\	膣(ちつ)
\\	ヴァギナ マンコ
\\	乙(おつ) 乙に あの人は乙なことを言う これは中々乙な味がする
\\	吃もる(どもります) どもり 彼はどもりながら礼を言った
\\	出る(でます) 学校を出る
\\	憂鬱(ゆううつ) その失敗で彼は憂鬱になった
\\	他にだれかが手伝いに行きますか
\\	既婚の(きこん) 未婚の(みこん) 既婚女性
\\	純情(な)(じゅんじょう) 純情な乙女(おとめ)
\\	建坪(たてつぼ) この家は建坪100平方メートルだ
\\	うちの鍵が掛かっている(かぎ)	
\\	(お)花見 (はなみ)
\\	東南アジア(とうなんあじあ)
\\	~を 引き受ける(ひきうけます) そんな難しいことを引き受けて大丈夫ですか
\\	水滴(すいてき) 木の葉から水滴がしたたり落ちた
\\	窮屈(な)(きゅうくつ) 窮屈な靴
\\	後半(こうはん) 前半(ぜんはん) 
\\	世紀の後半に
\\	咳をする(せき) デッドの咳は、きっと喫煙(きつえん)のせいだ
\\	海王星(かいおうせい) 海王星は太陽系の8番目の惑星です
\\	だちょう
\\	奈良(なら) 奈良へ行くことにした
\\	洗濯(せんたく) 選択(せんたく)
\\	登校(とうこう) 今朝、登校の途中で外人の一団に会った
\\	放射性(の)(ほうしゃせい)
\\	忘年会(ぼうねんかい) 大晦日(おおみそか)
\\	地元の(じもと) 地元のニュースは次に放送されます 地元チーム
\\	風呂桶(ふろおけ)
\\	見所(みどころ) 見所のある青年
\\	雨降り(あめふり) 雨天(うてん) 雨降りの こんな雨降りに出掛けるのですか
\\	滑る(すべります) 
\\	滑って転んだ 凍った道で車が滑った
\\	交流(こうりゅう)
\\	交換(こうかん) 文化の交流が活発になった
\\	分け合う(わけあう)
\\	物音(ものおと) 屋根で変な物音がする 何か物音がする
\\	(お)寺(てら) お寺に参ります(まいる)
\\	強姦(ごうかん) 性的暴行(せいてきぼうこう) 強姦者(ごうかんしゃ)
\\	欠席(けっせき) 留守(るす)
\\	出席(しゅっせき) 会議を欠席する
\\	同格(どうかく) 彼は田中氏と同格である 彼と同格の人
\\	終了する(しゅうりょう) 終わる(おわります) そのパーティーは真夜中に終了した
\\	夫人(ふじん) クリー夫人
\\	昨晩(さくばん) 昨晩は、寝ないで勉強した
\\	砂(すな)
\\	大使館(たいしかん) 大使(たいし)
\\	漬物(つけもの)
\\	太い(ふとい) 眉が太い(まゆ) げじげじまゆ 
\\	ほら!
\\	群集心理(ぐんしゅうしんり)
\\	余り(あまり) あんまり 私はテレビをあまり見ない
\\	山田夫妻(ふさい)
\\	火星(かせい)
\\	デザイン
\\	シルバーシート (直訳: 
\\	無い(ない) お金がない
\\	親知らず(おやしらず)
\\	連れて来る(つれてきます) 母は彼を私の部屋に連れてきた
\\	食費(しょくひ) 月々の生活費の中では食費が一番高くつく
\\	野鳥(やちょう) 野鳥観察(かんさつ)
\\	振り返る(ふりかえります) 
\\	振り替える(ふりかます) 
\\	秋子さんは何度も振り返って手を振った
\\	曇る(くもります)
\\	各々(おのおの) 夫々(それぞれ) 人はおのおの考えが異なる
\\	多国語(たこくご) 彼は多国語が話せる
\\	日本化(にほんか)
\\	吸殻(すいがら) 葉巻の吸い殻を捨てる
\\	ホチキス 
\\	ホッチキス ホチキスの針 書類をホチキスで留める
\\	~に 火を点す/灯す(ともす)
\\	火をつける 
\\	燃やす(もやします) 
\\	電灯をともす 灯籠に火をともす 爪に火を点すような暮らしをする
\\	(ココ)椰子の実 (ここやしのみ) ココナッツ 椰子 (やし)
\\	諺(ことわざ) 常用外
\\	日時計(ひどけい)
\\	先に行く(さきにいく)
\\	売春婦(ばいしゅんふ) 売春する 彼らは女に売春をさせて暮らしている
\\	ボウリング 
\\	ボーリング 
\\	あなたはボウリングしますか
\\	フィルム
\\	デジタル時計(どけい)
\\	まぐれ まぐれで
\\	破る(やぶります) 平和を破る ドアを破って開ける 約束を破る 規則を破った
\\	原発(げんぱつ) 原子発電所(げんしはつでんしょ)
\\	絶対零度(ぜったいれいど)
\\	さばを読む 彼は101歳だというがいくらかさばを読んでいるさ
\\	磨く(みがく) 歯を磨きます
\\	姉妹(しまい)
\\	図書(としょ) 本(ほん) あなたはもう推薦図書(すいせん)を読み終えましたか
\\	映画館(えいがかん)
\\	実験(じっけん) キュリー夫人が行った実験 動物で実験する 実験室(じっけんしつ)
\\	参った!(まいった) これは[一本]参った どうだ,参ったか 部屋の暑いのには参った
\\	無断(で)(むだん) 許可なく この池の魚を無断で捕ってはいけない
\\	~ば と思う 学生の時にもっと勉強していればと思った
\\	~に 責任がある(せきにん) ~を 責任をとる あなたはその事故に責任がないよ
\\	~を 誇る(ほこります) 得意する(とくい) ~を 自慢する(じまん) 
\\	私は誇るべきものが何もない
\\	なべ
\\	まずい
\\	また今度お願いします (またこんどおねがいします)
\\	皆に(みんなに) 皆さんに(みなさんに)
\\	~に うんざり(です/する) この悪天候にはうんざりだ/している 彼のひとりよがりの考えにはうんざりする
\\	文法(ぶんぽう) 文典(ぶんてん) 文法書のこと, 分点(ぶんてん)
\\	語法(ごほう) 
\\	社交的(しゃこうてき) 社交好き(しゃこうずき)
\\	社交ダンス
\\	複雑(な)(ふくざつ)
\\	~に 入る(はいります)
\\	尾(び)
\\	違う(ちがいます) 異なる(ことなります)
\\	旅行の仕方が違う
\\	気体(きたい) 期待(きたい)
\\	液体は気体より重い
\\	引き出し(ひきだし)
\\	出来上がる(できあがります) 計画が出来上がった
\\	目まい(めまい) ―あっ、目まいがします -すぐ、病院へ行ったほうがいいです
\\	環境(かんきょう) 人は遺伝と環境によって左右(さゆう)される
\\	鉄橋(てっきょう) 鉄橋を通過する
\\	請求する(せいきゅう) 要求する(ようきゅう)
\\	支払いを請求する 損害賠償を請求する
\\	険しい(けわしい) 
\\	厳しい(きびしい) 厳重(げんじゅう) 厳格(げんかく) 険しい顔付き 険しい声で
\\	畳(じょう) 10畳 四畳半の部屋
\\	生かす(いかす)
\\	活用する(かつよう)
\\	自分の才能を十分生かす スペイン語の知識を生かしてラテンアメリカとの商売を始めた
\\	死火山(しかざん)
\\	ポリ公(こう) ポリ 
\\	ポリス
\\	客(きゃく) お客様(おきゃくさま)
\\	怪我人(けがにん) 負傷者(ふしょうしゃ) 車が電車ぶつかって怪我人が出た
\\	栗鼠(りす) 常用外 栗(くり)
\\	敬う(うやまいます) 尊敬する(そんけい) 私は、年上の人を敬うように教えられた
\\	プリクラ 
\\	プリント倶楽部 (くらぶ)
\\	月食(げっしょく) 皆既月食(かいき) 部分月食
\\	左の方(ひだりのほう) 駅の左の方で今もうちが燃えている
\\	愛称(あいしょう) 私の両親はめったに「ハニー」などの愛称を用いなかった
\\	持ち物(もちもの) 
\\	私物(しぶつ) 
\\	四字熟語	忠言逆耳 (ちゅうげんぎゃくじ)
\\	口笛を吹く(くちぶえをふく) 口笛を吹いて犬を呼ぶ ジムは楽しそうに口笛を吹きながら運転した
\\	厳密に(げんみつ) 厳密に言えば
\\	額(がく) 額縁(がくぶち) 写真を額に入れる(いれる)
\\	つわり つわりになる 長男のときはつわりがひどかった
\\	長方形(ちょうほうけい) 長方形の
\\	体(からだ) 肉体(にくたい) 身体(しんたい)
\\	日光浴をする(にっこうよく)
\\	発明する(はつめい) 発明品 発明家/発明者
\\	笑う(わらいます) ~を
\\	ユダヤ教会(きょうかい)
\\	怪しい(あやしい) 彼が怪しいと思う
\\	野菜(やさい)
\\	燃やす(もやします) ろうそくを燃やす ろうそくに火をつける
\\	足す(たします) 加える(くわえます) 追加する(ついか)
\\	6と4を足すと10になる 6に4を加えると10になる
\\	分数(ぶんすう) 少数(しょうすう) 分子(ぶんし) 分母(ぶんぼ)
\\	細やか(な)(こまやか) 友好的(ゆうこうてき) 細やかな愛情
\\	当局(とうきょく) 学校当局 政府当局 当局者
\\	集まり(あつまり) 会合(かいごう)
\\	アメリカ大使がその集まりに招待された
\\	吸う(すいます) タバコを吸う
\\	僕んとこ 
\\	僕のところ
\\	宣伝(せんでん) 広告する(こうこく)
\\	テレビで製品を宣伝する会社が多い その宣伝に乗る
\\	句作する(くさく) 俳句を作る 俳人(はいじん)
\\	ニュージーランド
\\	扇風機 (せんぷうき) 風車小屋(ふうしゃごや)
\\	公用語(こうようご) ブラジルの公用語はポルトガル語です
\\	異郷(いきょう) 異教徒(いきょうと)
\\	進める(すすめます) 推薦する(すいせん) おすすめは何ですか。
\\	門番(もんばん) 彼に門番をたのんだ
\\	鈎針(かぎばり) 鉤針編み(かぎばりあみ)
\\	必要性(ひつようせい)
\\	平静な(へいせい) 平成(へいせい) 
\\	静かな(しずか) 町は平静であった 彼の顔は最後まで平静であった
\\	組織(そしき)
\\	筋肉組織は無数の細胞からなっている
\\	響き(ひびき) 爆発の響き 大砲(たいほう)の響き
\\	なければ 水がなければ生きられない
\\	身体が不自由である (しんたい) 身障者(しんしょうしゃ)
\\	意味(いみ) 意義(いぎ) その意味が分からない
\\	水巻ホース(みずまき) (ガーデン)ホース 彼はホースを使って庭に水をまいてた (撒く(まく)- 
\\	いろいろお世話になりました
\\	渦巻き(うずまき) たばこの煙の渦巻き
\\	原住民(げんじゅうみん) 原住民たちは初めて飛行機をみた
\\	ビールでも飲む
\\	写真(しゃしん) 桜の写真を撮る
\\	面積(めんせき) 日本の面積は37万平方キロです
\\	確認する(かくにん) 確認ボタン
\\	別の(べつ) 食事は別の部屋でするんですか
\\	あざらし
\\	喫煙者(きつえんしゃ)
\\	学科(がっか) 科目(かもく) 言語学科(げんごがっか) ども学科が一番好きですか
\\	アクセント その語のアクセントは第2音節にある
\\	刑事(けいじ) 掲示(けいじ) 
\\	その泥棒は刑事を見ると走って逃げた
\\	病は口から
\\	上達する(じょうたつ) アメリカであなたの英語は上達しましたか
\\	酸っぱい(すっぱい)
\\	女史(じょし) 女子(じょし)
\\	助詞(じょし)
\\	中田女史
\\	砂糖(さとう) 砂糖を入れる
\\	共用する(きょうよう) 私は妹とアパートを共用している 共用の台所
\\	飯(めし)
\\	インターホン インターホンで話す
\\	試着室(しちゃくしつ)
\\	落とす 
\\	とす	送り仮名 (おくりがな)
\\	排気ガス(はいき) 排ガス(はい)
\\	毛(け) 犬の毛はいわばセーターだ 彼は羊の毛を刈った
\\	庶民(しょみん) 一般人(いっぱんじん) 庶民の暮らし
\\	専門(せんもん)
\\	外れる(はずれます) 外す(はずします) ボタンが外れた 取っ手が外れた/抜けた
\\	ダイヤモンド ダイヤ 
\\	新年(しんねん)
\\	こちらこそ。
\\	~まで ~までに 日曜日までに払いたいです
\\	トピック テーマ 話題(わだい)
\\	(警護) いる, 来る, 行く 
\\	(いらっしゃる) いらっしゃいます どのくらい日本にいらっしゃいますか いらっしゃってください
\\	飲酒(いんしゅ) 私は飲酒の習慣を治してもらった
\\	ペスト 黒死病(こくしびょう)
\\	荷造りする(にづくり) 荷造りはもう終わりましたか
\\	空腹(くうふく) 飢え(うえ)
\\	空腹で苦しむ(くるします)
\\	留守番する(るすばん) 母が留守番をすることになっている
\\	日の丸(ひのまる)
\\	日章旗(にっしょうき)
\\	全世界に(ぜんせかい) 世界中(せかいじゅう)
\\	オマーン
\\	迷信(めいしん) 私は愚かな迷信など信じらない 君は迷信を信じるかい
\\	明るい(あかるい) 朗らか(な)(ほがらか) 陽気(な)(ようき) 中村さんは明るい人です
\\	利用する(りよう)
\\	使用する(しよう)
\\	使う(つかいます) あなたの時間をうまく利用しなさい
\\	結跏趺坐する(けっかふざ)
\\	患者(かんじゃ) あの看護婦さんは患者にはまさに天使だ   患者を入れる(いれる)
\\	うがいする うがいをしないほうがいい
\\	執筆する(しっぴつ) 執筆者(しっぴつしゃ) 著者(ちょしゃ) 作家(さっか) 小説家(しょうせつか) 次作を執筆中です 筆跡(ひっせき)
\\	狐(きつね) 常用外
\\	干す(ほします) 乾かす(かわかす) 洗たく物を干す 干した魚
\\	弁明する(べんめい) 私にも弁明させてくれ
\\	もちろん ええ、もちろん行きました
\\	西瓜(すいか) 瓜(うり) 常用外 メロン
\\	別々に(べつべつ)
\\	部首(ぶしゅ) 漢字部首リスト
\\	口調(くちょう) 彼女は穏やかな口調で話した
\\	箒(ほうき) 新しい箒はきれいに掃ける(はきます)
\\	スペアタイヤ
\\	~を 後悔する(こうかい) 公開する(こうかい) 
\\	彼は怠惰であったことを後悔している
\\	悪意のある(あくい) 意地悪(な)(いじわる)
\\	悪意(あくい) 彼女はあなたに悪意など持っていなかった
\\	独身(どくしん) 独身じゃなくて、結婚している
\\	非武装化(ひぶそうか) 国を非武装化する
\\	お出かけですか
\\	同じぐらい(おなじぐらい) どちらも 犬も猫も同じぐらい好きです。
\\	応用する(おうよう) 実施(じっし)
\\	彼女は授業で学んだことをその実験に応用した
\\	~を 省く(はぶきます) ~を 省略する(しょうりゃく) 渋る(しぶります) その部分は映画では省いてあった
\\	主食(しゅしょく) ほとんどの日本人は米を主食にしている このあたりではジャガイモを主食としている
\\	そばかす 彼女はそばかすだらけだ
\\	長所(ちょうしょ) 全ての人は長所と短所を持っている 短所(たんしょ)
\\	罰(ばつ) うそをついたので彼は罰を受けた 生徒を罰する
\\	芝 (しば) 
\\	芝生(しばふ)
\\	芝を刈る(しばをかります)
\\	先に(さきに)
\\	前に 
\\	もっと早く 
\\	その情報は発表するより先に新聞に漏れた
\\	くぐもる(くぐもります) くぐもった声で
\\	契約(けいやく) この契約は一年間有効です
\\	トガリネズミ 尖る(とがります)
\\	遅れてどうもすみません 遅くなってすみません
\\	コーヒー 喫茶店へコーヒーを飲みに行く
\\	痛い目(いたいめ) つらい思い(おもい)
\\	~か ~か それとも 
\\	私は電車かバス(か)でいく 肉が高かったか一郎が肉が嫌いだったかどちらかだ
\\	親しい(したしい) 親しい友人 親しい間柄である
\\	俗語(ぞくご)
\\	口語(こうご)
\\	俗語の表現 俗語を使う テレビ業界の俗語
\\	お湯が沸く(わきます) お湯が沸騰する(ふっとう)
\\	父方(ちちかた) 父方の祖母が亡くなった 父方(ちちかた)の親類
\\	~に 従事する(じゅうじ) 十字(じゅうじ)
\\	ダム建設に従事する 父は印刷業に従事している
\\	生き生き(いきいき) 活発(かっぱつ)
\\	~のように ノヴァクさんは日本人のサラリーマンのように遅くまで働きます
\\	焼餅を焼く(やきもちをやく)
\\	嫉妬する(しっと)
\\	羨ましい(うらやましい) 彼は妻に焼餅を焼く 彼は妻に嫉妬をする 彼は彼女の才能に焼餅を妬いている(やく)
\\	ずさん(な) 彼はやることがずさんだ この計画はずさんだ
\\	怒り(いかり) 錨(いかり)
\\	怒る(おこります) 怒りに燃える
\\	自国語(じこくご)
\\	母国語(ぼこくご)
\\	彼らの自国語
\\	両替する(りょうがえ) お金を替える この円の現金を、ドルに両替してください 両替機
\\	頭を捻る(ひねります) 頭を捻って考えましたがいい考えが浮かびませんでした
\\	作り上げる(つくりあげます) 
\\	完成する(かんせい) この橋を造り上げるのに5年かかった
\\	温暖(おんだん) 温かみ(あたたかみ) 温もり(ぬくもり) 四方を海に囲まれているので、日本の気候は温暖です 温暖化 温暖化ガス
\\	キャラメル
\\	宅急便(たっきゅうびん)
\\	楕円(だえん)
\\	仲間(なかま) こっち来て仲間に入りなさい(はいる) クラスの仲間
\\	残忍(な)(ざんにん) 残酷(な)(ざんこく) 残忍な人間
\\	社交好きな人(しゃこうずき) 彼は社交好きだ
\\	大阪城(おおさかじょう) 城(しろ)
\\	ボール
\\	断る(ことわります) 彼女の招待を断るはあなたも不作法(ぶさほう)な人だ
\\	孝行(こうこう) 高校(こうこう)
\\	孝行息子 親に孝行する
\\	〔危険な〕
\\	やばい 見つかったらやばいぞ やばかったな やばい仕事だ
\\	大脳皮質(だいのうひしつ)
\\	野生(の)(やせい)
\\	野蛮(な)(やばん)
\\	着く(つきます) 到着(とうちゃく) 名古屋に着いた。
\\	窓が閉まっている
\\	スケート スケートをする
\\	国際法(こくさいほう)
\\	倍増する(ばいぞう) 2020年までに、この市の人口は倍増しているだろう
\\	費用(ひよう) 旅行の費用 費用を計算してみましょう
\\	気管(きかん) 機関(きかん)
\\	器官(きかん)
\\	期間(きかん)
\\	没落する(ぼつらく) 滅亡する(めつぼう)
\\	滅びる(ほろびます)
\\	破産(はさん)
\\	炭鉱業(たんぎょうこう)の没落 さしも栄えた王国もついに没落した
\\	ばく大(な)(ばくだい) ばく大な金額 株でばく大な損をした
\\	社員食堂 (しゃいんしょくどう)
\\	証言する(しょうげん) 目撃者の証言(もくげきしゃ)
\\	蔦(つた) つたにおおわれた家
\\	地形(ちけい) 地形図(ちけいず) 彼はこの辺の地形に詳しい
\\	~を 連想(れんそう) どこの国でも赤いから「危険」を連想する
\\	色盲(しきもう) 色盲のために、色彩の区別ができない人もいる
\\	移民(いみん) その国はヨーロッパからの移民が多い
\\	ミサ ミサを行う
\\	囲む(かこみます)
\\	計る/測る(はかります)
\\	表す(あらわす) 意味をする 
\\	表示する(ひょうじ)
\\	この印は何を表していますか
\\	学生証(がくせいしょう)
\\	多くの時間 多くの時間を無駄(むだ)に過ごす
\\	薬指(くすりゆび)
\\	有望な(ゆうぼう) 前途有望の(ぜんとゆうぼう) 前途洋洋の(ぜんとようよう) 有望な研究者 彼女の前途は有望である
\\	咳き(せき) 席(せき)
\\	咳きが出る
\\	檻 (おり) ライオンの檻
\\	セロテープ
\\	有力(な)(ゆうりょく) 影響力のある/影響が強い 
\\	有力な政治家 有力な証拠が見つかった 中国は武装の有力運出国だ
\\	工作(こうさく) 工作の時間〔学校の〕
\\	衰える(おとろえます) 悪化する(あっか)
\\	彼の健康がめっきり衰えた 年をとって記憶力[視力]が衰えてきた
\\	部下(ぶか) 目下(めした)
\\	どの上司にもお気に入りの部下がいるものだ
\\	オーロラ
\\	手こき(てこき)
\\	遭う(あいます) 田中さんは交通事故に遭ったそうです
\\	雨具(あまぐ) 彼は雨具を用意して出かけた
\\	改正する(かいせい) 憲法を改正したいと考えている人たちがいる
\\	と共に(とともに) 一緒に(いっしょに) 揃って(そろって) お正月は家族と共に過ごしたい
\\	大枝(おおえだ) 小枝(こえだ)
\\	メッセージ 彼にメッセージを頼みます
\\	暴力(ぼうりょく) 家庭内暴力(かていないぼうりょく)
\\	安価(あんか) 高価(こうか) 安価に 安価の 値段が高い 安い/安っぽい
\\	クロワッサン
\\	鼻(はな) 鼻が高い/低い
\\	洗礼(せんれい) 彼は洗礼を受けてカトリック教徒となった
\\	白かんばの木(しらかんば)
\\	朝早く(あさはやく) 早朝に(そうちょう) 朝方に(あさがた) 朝早くに、鳥が歌う
\\	結婚式(けっこんしき)
\\	王冠(おうかん)
\\	冠(かんむり) 王冠をかぶる
\\	高齢化社会(こうれいかしゃかい)
\\	モンゴロイド
\\	本当(ほんとう) 本当です
\\	接頭辞(せっとうじ) 
\\	は多くの動詞の接頭辞になっている 接尾辞(せつびじ)
\\	他人(たにん) 他人のことを気にするな
\\	検索する(けんさく) 調べる(しらべます)
\\	探す(さがします)
\\	言葉を辞書で検索する
\\	視覚(しかく) 四角(しかく)
\\	視力(しりょく) 
\\	視覚を失う
\\	真夜中(まよなか) 真夜中に
\\	嘆く(なげきます) 「財布を無くした」ジョンは嘆いた
\\	格好をつける(かっこう)
\\	渡り鳥(わたりどり)
\\	ありえない 不可能(ふかのう)
\\	無理(むり)
\\	ありえる(ありえます) ありうる
\\	二国語(にこくご)
\\	動脈(どうみゃく) 静脈(じょうみゃく) 大動脈(だいどうみゃく)
\\	ただいま。
\\	遊歩道(ゆうほどう) 散歩道(さんぽみち)
\\	きれいでなくてもいいです
\\	屈伸(くっしん) 屈伸する (ひざの)屈伸運動をする
\\	補強する(ほきょう) チームを補強する
\\	自己紹介する (じこしょうかい)
\\	と言う ケンさんはパーティーが面白かったと言っていました
\\	縄跳び(なわとび) 縄跳びをする アメリカの子には縄跳びが人気がある
\\	酔っ払い(よっぱらい) 酔っ払いが階段から落ちた
\\	今日は、あいてます?
\\	巣(す) 鳥は小枝で巣を作る
\\	値切る(ねぎります) その鞄を8千円に値切った
\\	ペパーミント
\\	扱う(あつかいます) 
\\	部下をもっと公平に扱いなさい 国連はその国際問題を扱う
\\	根(ね) 木の根
\\	行き止まり(いきどまり) その道は行き止まりだった
\\	憎悪(ぞうお) 憎しみ(にくしみ) 嫌い(きらい) 憎悪すべき行為
\\	推理小説(すいりしょうせつ) 推理(すいり)
\\	いよいよ 遂に(ついに) やっと いよいよ演奏会の日となった
\\	下車する(げしゃ) 車を出る
\\	零下(れいか) 温度計は零下3度を示している
\\	映る(うつります) 移る(うつります)
\\	花が水に映っている 湖に富士山の影が映っている
\\	プラチナ
\\	白金(はっきん)
\\	立派(な)(りっぱ) 素晴らしい(すばらしい)
\\	~に 油をさせる 機械に油にさせてください
\\	クロスボウ
\\	無限(むげん) 無限の空間
\\	メーキャップ 化粧(けしょう)
\\	軒下(のきした) 軒下に 店の軒下で雨宿りをする
\\	日焼け(ひやけ) 日焼けした顔 彼女の皮膚は日焼けしやすい
\\	勃発(ぼっぱつ)する 第一次世界大戦は1914年に勃発した
\\	スチュワーデス
\\	外食する(がいしょく)
\\	知的(な)(ちてき) 知的生活 知的障害者(ちてきしょうがいしゃ)
\\	叫ぶ(さけぶ)
\\	わめく 
\\	助けてくれと叫ぶ
\\	循環器(じゅんかんき) 循環(じゅんかん) 血液が体内を循環する 循環器学(じゅんかんきがく)
\\	被る(かぶります) 帽子・ヘルメットを被る
\\	土星(どせい) 土星の輪(わ)
\\	内陸国 (ないりくこく)
\\	預ける(あずけます) 子供を姉に預けた お金を全部彼に預けた
\\	所で(ところで) ちなみに
\\	速達(そくたつ) その手紙を速達で出した
\\	小人(こびと)
\\	成年(せいねん) 青年(せいねん)
\\	息子が成年に達した
\\	レバー 
\\	サイドブレーキ レバーを引く
\\	全然(ぜんぜん)
\\	全然分からない
\\	脆い(もろい) 彼らの友情は実にもろかった 脆い人
\\	~を 広める(ひろめます) その国はどんどん勢力を広めた 広げる(ひろげます)
\\	お袋(おふくろ) 親父(おやじ) これはお袋の味です
\\	突然(に)(とつぜんに) とっさに 2時間仕事を続けていたら、突然気分が悪くなった
\\	もう、家に帰っている
\\	休憩(きゅうけい) ちょっと休憩しよう
\\	菓子屋(かしや)
\\	黙る(だまります) 黙れ!
\\	バニラ バニラ・ビーンズ バニラ豆(まめ) 
\\	利用者(りようしゃ) 使用者(しようしゃ) ユーザー 機会を利用する
\\	映画スター
\\	乗っ取る(のっとります) 
\\	彼は会社を乗っ取る 社長の地位を乗っ取った
\\	乗組員 (のりくみいん) 乗組員たちは宇宙への航海(こうかい)のために準備した
\\	変わる(かわります) 毛虫が蝶々に変わった
\\	バス乗り場(のりば) バス停(てい) 乗り場(のりば)
\\	ティーポット
\\	糸(いと)
\\	まぶた
\\	小柄の/な(こがら) 小柄な女性 小柄の花模様のブラウス
\\	(学)	解剖学(かいぼうがく)
\\	以内(いない) 三日以内に返事します
\\	給水塔(きゅうすいとう) 給水(きゅうすい)
\\	ちょっと高い
\\	朝日(あさひ)
\\	~と~ たくさん食べると太るよ
\\	同じ(おなじ) 同じ部屋で
\\	着く(つきます) 到着する(とうちゃく) どこかに着く
\\	仮説(かせつ) 仮説的 私は彼の仮説を支持する
\\	停学する(ていがく) 停学にされた学生
\\	青信号(あおしんごう) 赤信号(あかしんごう)
\\	家畜(かちく) 家畜はみんな丸々している
\\	平服(へいふく) カジュアルウエア 式服(しきふく) フォームウエア
\\	マンコ
\\	かき回す(かきまわします) スプーンで紅茶をかき回す
\\	半分分ける(はんぶん)
\\	運転する(うんてん)
\\	不平(ふへい) 不満(ふまん)
\\	~の 不平(ふへい)する 母はいつも私の不平ばかり言ってる
\\	警察官(けいさつかん) 警官(けいかん)(略語) お巡りさん(おまわりさん)
\\	孔雀(くじゃく) 孔雀は羽は美しいが足が汚い
\\	何というものですか これは日本語でなんと言いますか
\\	果物(くだもの)
\\	ぺろぺろキャンデー ペろキャン
\\	一般に(いっぱんに) 最近の映画は一般にあまり面白くない
\\	面接(めんせつ) 面接に行く
\\	大半は/が(たいはん) ほとんど 
\\	大抵(たいてい)
\\	生徒は大半はフランス人だ
\\	五輪(ごりん) オリンピック オリンピック記録を出す
\\	温もり(ぬくもり) 温暖(おんだん) 灰にまだ温もりが残っている
\\	説明する(せつめい) それは説明の要がない
\\	ヘルニア
\\	準備はもうしてある ~てある 
\\	自由(な)(じゆう) 自由
\\	観光する(かんこう) 見物する(けんぶつ) 見学する(けんがく)
\\	交換(こうかん)
\\	接触(せっしょく)をとる 連絡(れんらく)をとる
\\	キノコ雲(きのこぐも) 
\\	経営者(けいえいしゃ) マネージャー
\\	直訳(ちょくやく)
\\	恐怖症(きょうふしょう)
\\	刀(かたな) 鉄で刀を作ります
\\	~に 入居する(にゅうきょ)
\\	かぎを開ける 玄関の鍵をかけ忘れた
\\	肋骨(ろっこつ) あばら骨(あばらぼね)
\\	講義(こうぎ) 
\\	抗議(こうぎ) 講演(こうえん)
\\	公園(こうえん) 講義に出る 鈴木さんの講義では、いつも面白い話が聞けます
\\	雨天(うてん)
\\	驚く(おどろきます) ポーランドの柔道チームの強さに、日本人は驚く
\\	漏斗(じょうご) 漏斗形の(じょうごけいの)
\\	散らばる(ちらばります) 散らばった その銀行の支店は日本全国に散らばっている
\\	生々しい(なまなましい) 鮮やか(な)(あざやか)
\\	活気(かっき)
\\	活発(かっぱつ)
\\	生き生きとする(いきいき)
\\	生々しい思い出 生々しい事件の描写
\\	反感(はんかん) 少年は厳しい父親に反感を抱いていた(いだく)
\\	砥石(といし) 砥石で包丁(ほうちょう)を磨く/研ぐ
\\	静まる(しずまります) 嵐は静まった 音楽を聞いて彼女の神経が静まりました
\\	大口(おおぐち) 大口で食べる 大口の寄付 大口をたたく
\\	文章(ぶんしょう) 文(ぶん) この文章は意味をなさない 彼は文章がうまい
\\	例(れい)/一例(いちれい) 先生は例を挙げて説明してください
\\	大地の母(だいち)
\\	解散(かいさん) 警察がデモ隊を解散させた 一人,二人と去って会は自然解散となった
\\	大鎌(おおがま)
\\	軍人(ぐんじん) 
\\	兵士(へいし) 
\\	戦士(せんし) 
\\	ルーマニア
\\	レバー 
\\	収集(しゅうしゅう)する コレクションする 切手の収集
\\	暗証番号 (あんしょうばんごう)
\\	人でなし(ひとでなし) この人でなし
\\	ココア 熱いココア1 杯
\\	言及(げんきゅう) 彼女は試験の結果について言及したの
\\	経営する(けいえい) 彼女は美容院を経営している 会社を経営する 経営者(けいえいしゃ)
\\	大蔵省(おおくらしょう) 大蔵大臣(おおくらだいじん)
\\	番犬(ばんけん)
\\	開く(ひらきます)
\\	腕時計(うでどけい)
\\	音声器官(おんせいきかん)
\\	怖い(こわい)
\\	宇宙人(うちううじん) 宇宙(うちゅう)
\\	旧型(きゅうがた) 新型(しんがた) 新コンピューターは旧型よりも10倍速いです
\\	摘出する(てきしゅつ) 除く(のぞきます) 取り除く(とりのぞきます) 傷口からガラスの破片を摘出する 腫瘍を摘出する 胆嚢摘出(術)
\\	運が悪い(うんがわるい)
\\	外務省(がいむしょう) 内務省(ないむしょう)
\\	転勤(てんきん) 彼は千葉の事務所に転勤した
\\	イントネーション 英語で話すとき、もっとイントネーションに注意しない
\\	百姓(ひゃくしょう) 農家(のうか)
\\	百姓をする 彼の両親はお百姓でした
\\	クイズ クイズを解く
\\	獲物(えもの) 狼(おおかみ)は獲物が豊富(ほうふ)な地域に住んでいる
\\	上等(な)(じょうとう) 上等の酒 最上等のカメラ
\\	仮に(かりに) (~したら) 仮に君が僕の立場だったとして,何が出来るかね 仮に彼が来なかったらどうするか
\\	きゅうり
\\	配達する(はいたつ) ここへ手紙は正午頃配達される
\\	現実する(げんじつ) いつか私の夢は現実する
\\	~た/だって 
\\	~て/でも あの人には話したって分からない 僕はその切符を高くたって買います
\\	ばかりか (さえ) 
\\	ばかりか な 
\\	な
\\	ばかりか 
\\	ばかりか 日本では子供ばかりか大人さえ漫画を読んでいる あの人は絵を見て楽しむばかりか、自分でも絵を描く
\\	世界一周の旅 (せかいいっしゅうのたび) 世界旅行(せかいりょこう)
\\	良し(よし)
\\	~てもいい? 借りてもいいですか。
\\	万国(ばんこく) 
\\	軍基地(ぐんきち)
\\	足を引きずる (ひきずります) 足を引きずって歩く けがをした選手は足を引きずってベンチに引き下がった
\\	片仮名(かたかな)
\\	卒業する(そつぎょう) 大学を出る 卒業生(そつぎょうせい)
\\	オペラ
\\	キロ(メートル)
\\	宇宙船(うちゅうせん)
\\	賑やか(な)(にぎやか) 常用外
\\	イグルー
\\	方針(ほうしん) 
\\	政策(せいさく) 日本の対ロ方針  方針がはっきりしていない 方針を立てる
\\	のっぽ 
\\	のっぽ(な/の) のっぽの木
\\	おまちください。
\\	だれかを 許す(ゆるす) どんな言い訳をしようとも、彼を許すことはできない
\\	委員会(いいんかい) その委員会は四人の委員からなる
\\	大きすぎる 静か過ぎる
\\	解毒剤(げどくざい) 解毒する
\\	間抜け(な)(まぬけ) 馬鹿(な)(ばか) 愚か(な)(おろか)
\\	愚か者(おろかもの) 間抜け面(まぬけづら) そんな間抜けな話は聞いたことがない この間抜けが
\\	航空便(こうくうびん) エアメール 航空(こうくう)
\\	潟(かた) 方(かた)
\\	肩(かた)
\\	型(かた)
\\	求婚者(きゅうこんしゃ) 求婚する(きゅうこん) プロポーズする 彼はスーに求婚した
\\	一長一短(いっちょういったん) 長短(ちょうたん) これらの計画はみな一長一短がある
\\	お化け(おばけ)
\\	ジレンマ ジレンマに陥る
\\	どうやって どのように 博物館までどうやって行きますか
\\	報告する(ほうこく) レポートする 上司に結果を報告する 会議の報告は次のとおり
\\	恋人(こいびと)
\\	血脈(けつみゃく) 
\\	雑種(ざっしゅ) うちの犬は雑種です 彼はばらのいろいろな雑種を作っている
\\	パンフレット
\\	頭足動物 (とうそくどうぶつ) その頭足動物は生きている化石だ 頭足類(とうそくるい)
\\	踊り(おどり) ダンス 舞踊(ぶよう)
\\	筆跡(ひっせき) あなたの筆跡は私のに似ている
\\	就く(つく) 政府の高い地位(ちい)に就く
\\	とっくに 昔(むかし) 彼はその大学をとっくに卒業しました 宿題はとっくに(=すでに)終わったよ
\\	(大きな神社)	大社(たいしゃ) 
\\	三嶋大社 (みしまたいしゃ)
\\	求める(もとめます) 彼はほかに情報を求めることにした
\\	温度(おんど) 氷の温度は何度ですか
\\	笑窪(えくぼ) 彼は笑うとえくぼができる
\\	蟹(かに)
\\	直接だれかに会う
\\	宴会(えんかい) 彼は宴会の事について触れた
\\	血管(けっかん) 血脈(けつみゃく) 血は血管の中を流れる
\\	~を 探す(さがす)
\\	盛る(もります) 
\\	積み上げる(つみあげる) 盆にオレンジを盛る 砂を盛る
\\	広大(こうだい)
\\	巨大(きょだい)
\\	太平洋は非常に広大です
\\	オルゴール
\\	蜂の巣(はちのす)
\\	歯垢(しこう) 思考(しこう)
\\	志向(しこう)
\\	歯垢をとらなけれべなりません
\\	記号(きごう)
\\	など) 印(しるし) 
\\	シンボル 
\\	元素記号
\\	は水素を表す 数学の記号
\\	とぼとぼ歩く 老人が道をとぼとぼ歩いていった
\\	渡る(わたります) 横断する(おうだん) 犬は通り(とおり)を渡った
\\	覆う(おおいます) スカーフで顔を覆う 雪が地面を覆った
\\	~に 奥(おく)
\\	レポート
\\	痛み(いたみ)
\\	一時間後(いちじかんご)
\\	短時間(たんじかん) 検査は短時間ですみますからね
\\	のろのろ歩く
\\	罪(つみ)を犯す(おかす)
\\	ムース
\\	印(いん) 印を押す
\\	防弾の(ぼうだん) 防弾ガラス  防弾チョッキ(ぼうだん)
\\	芽キャベツ(めきゃべつ)
\\	がーがー鳴く(なきます) アヒルはがーがー鳴く ガーガー言う 
\\	正解(せいかい) 正解を丸で囲みなさい(かこむ)
\\	ガラス ガラスが割れている
\\	樫の木(かしのき) オーク 私は樫の木の森へ散歩した
\\	お使いをする/~に行く あの子はよくお使いをする 用事がある
\\	ローマ帝国(ていこく)   帝国主義的(ていこくしゅぎてき)
\\	かみそりの刃(は)
\\	革命(かくめい) バイオテクノロジーは農業に革命をもたらすだろう
\\	自己批判(じこひはん)
\\	過去(かこ) 過去のことを思う 過去のある女性
\\	友好的 (な)(ゆうこうてき) 友好国(ゆうこうこく)
\\	迎える(むかえます) 私たちは温かく迎えました
\\	操る(あやつります) 人形を操る 世論を操る 馬を操る
\\	工場(こうじょう) ビールの工場
\\	浮気する(うわき) 浮気者(うわきしゃ) 裏切る(うらぎります) 彼女は上司と浮気をしている
\\	遠慮なく(えんりょ) ご遠慮なくお菓子を召し上がれ 分からないところは遠慮なく質問してください
\\	交渉(こうしょう) 両国は平和交渉を開始した
\\	~なくちゃ ~なきゃ 勉強しなくちゃ 勉強しなきゃ
\\	控える(ひかえます) 
\\	待つ(まちます) 彼が次の間に控えている
\\	何時でも(いつでも)
\\	猪 (いのしし)
\\	恋人(こいびと)
\\	杖(つえ) 杖をついて歩く
\\	前の日(まえのひ) 前日(ぜんじつ)
\\	疲れ(つかれ) 疲れる(つかれます) 疲れが出る へとへとになる 
\\	疲れを休める(やすめる)
\\	流れ星(ながれぼし) 流星(りゅうせい)
\\	予定です(よてい) ~ようと思う 
\\	会議は何日ですか 火曜日の予定です 飛行機は何時に着きましたか 5時半の予定です
\\	元気ですか。(げんき) はい、元気です。
\\	身分(みぶん)
\\	地位(ちい)
\\	江戸時代は主な身分が四つあった
\\	癌(がん) 看護婦は癌で死んだ
\\	営業中(えいぎょうちゅう) 日曜日を除き毎日営業
\\	水が漏れる(もれます)
\\	夏物(なつもの)
\\	夏服(なつふく) 冬物(ふゆもの)
\\	冬服(ふゆふく)
\\	目が釘付け(くぎづけ)になる
\\	ひどい 残酷(な)(ざんこく)
\\	幸運を祈る(こううん) ~よう
\\	学校に通う(かよいます) 通学する(つうがく) 学生時代、私はアルバイトをしながら、日本語学校に通っていた
\\	誘う(さそいます) 招待(しょうたい) 招く(まねきます) 友人数人をお茶に誘った ディズナイレンドへ行こうと誘いました
\\	食生活(しょくせいかつ)
\\	株式市場(かぶしきしじょう) 株式市場は今日、下落した
\\	損する(そん)
\\	損害する(そんがい) 
\\	喪失する(そうしつ) 
\\	失う(うしないます)
\\	歩行(ほこう) 補講(ほこう)
\\	この患者さんたちは歩行が困難です
\\	今でも(いまでも) 未だに(いまだに) このテーブルは今でも使われている
\\	おねがいします。
\\	聴解(ちょうかい)
\\	~て 京都へ行って、何をしましたか
\\	地雷(じらい) 彼は地雷の上を運転して、彼のジープは爆発した
\\	ぶらぶら歩く 腕をぶらぶらさせて歩いて行った
\\	水車小屋(すいしゃごや) 水車(すいしゃ)
\\	皮膚(ひふ)
\\	肌(はだ) 皮膚病(ひふびょう) 弱い皮膚
\\	(在日韓国人) 
\\	パンチョッパリ
\\	蜃気楼(しんきろう) 蜃気楼が現れた
\\	宝石店 (ほうせきてん)
\\	ううん
\\	ごめんください。
\\	化石(かせき)
\\	モンゴル(国) 蒙古(もうこ)
\\	くしゃみする
\\	酔う(よう) 酔っ払う(よっぱらう)
\\	冬物(ふゆもの)
\\	冬服(ふゆふく) 夏物(なつもの)
\\	夏服(なつふく)
\\	鉛(なまり) なまり 
\\	モーターボート
\\	登山家(とざんか)
\\	話題が多い人(わだい)
\\	~ければ、 ~ほど 
\\	~なら、 ~ほど 
\\	野菜は新しければ、新しいほどおいしいです
\\	たくさん(沢山)
\\	~結果(けっか) 長い間努力した結果、ついに実験に成功した
\\	達する(たっする) 息子が成年に達した 目的地に達する
\\	回想する(かいそう) 思い出す(おもいだす) 
\\	振り返る(ふりかえます) 彼女は子供の時代を回想した
\\	職業(しょくぎょう) ご職業は
\\	摂氏(せっし) 摂氏10度2分 摂氏5度の水 華氏(かし)
\\	プール
\\	返す(かえす)
\\	お玉(おたま)
\\	最北端(さいほくたん) その町は日本の最北端に位置する
\\	関係者(かんけいしゃ) 関係者以外立ち入り禁止
\\	母校(ぼこう) あなたの母校はどこですか
\\	侵入(しんにゅう) ドイツ軍はベルギーに侵入した
\\	秋(あき) 秋期(しゅうき)
\\	~に 腰をかける(こし) ~に 座る(すわります) ~に 席をとる 
\\	椅子に腰をかけてください
\\	沸騰する(ふっとう) 沸く(わきます) 牛乳は水よりも高い温度で沸騰する
\\	飛行士(ひこうし) パイロット 操縦士(そうじゅうし)
\\	銀行(ぎんこう)
\\	鷲(わし) 和紙(わし)
\\	私(わし)
\\	助教授(じょきょうじゅ) 私は助教授です、いやもっと正確に言えば、助教授です
\\	禁煙(きんえん) 禁煙する(きんえん) 今年こそは禁煙するぞ
\\	紅葉する(こうよう) 木の葉は秋に紅葉する
\\	サファイア
\\	何百~もの~(なんびゃく) 何百人もの人(なんびゃくにん) 何百冊もの本 何百人ものの少年達がグランドで遊んでいる
\\	忍者(にんじゃ)
\\	ハイジャック犯(はん)
\\	新陳代謝(しんちんたいしゃ) 代謝(たいしゃ)(略) 新陳代謝が激しい[緩慢である]
\\	請う/乞う(こいます) 乞食(こじき) その部屋の使用許可を請うた
\\	失望(しつぼう) おまえには失望したよ 彼女が写真ほどきれいではなかったので失望した
\\	題名(だいめい) 題(だい) 話題(わだい)
\\	タイトル 
\\	この劇の題名は「オセロ」です
\\	防ぐ(ふせぎます) 彼女は病気を防ぐためにコートを着ていた
\\	クミン クミンシード クミンパウダー
\\	そんなに+ない 私はそんなに飲みません
\\	ふけ ふけだらけの頭
\\	多くの人 遠くの町 近くの店
\\	回帰線(かいきせん) 北回帰線(きた) 南回帰線(みなみ)
\\	受け取る(うけとります) 手に入れる(てにい)ます) 貰う(もらいます) 得る(えます) われわれは直接手紙の返事を受け取った
\\	利用する(りよう) 使用する(しよう) 
\\	使う(つかいます) 使用する前にレンガを振ってください
\\	肌(はだ) 皮膚(ひふ)
\\	早期の(そうき) 初期(しょき)
\\	早期診断(そうきしんだん) 早期治療を受ける 早期発見(そうきはっけん)
\\	精神分析(せいしんぶんせき)
\\	結婚(けっこん) 誰かと結婚する
\\	厳格(げんかく)
\\	幻覚(げんかく) 
\\	厳重(げんじゅう) 厳しい(きびしい) 吉田さんは子供に対して厳格すぎだ
\\	価格(かかく) 値段(ねだん)
\\	価値(かち)
\\	価格を下げる
\\	釘(くぎ) 何かに釘を打つ
\\	ブランコ ぶらんこに乗る パパにぶらんこを押してもらう  ぶらんこしよう
\\	アカシア
\\	ハンドの操り人形 (あやつりにんぎょう) 指人形(ゆびにんぎょう)
\\	可笑しい(おかしい) 何がおかしいんだ
\\	足(そく)
\\	婚約(こんやく) ~と 婚約している
\\	すたすた(と)歩く
\\	こんにちは。
\\	未熟児(みじゅくじ)
\\	落下枝に帰らず (らっかえだにかえらず)
\\	ファン 相撲の大ファン(だい)
\\	合図(あいず) 手を挙げて合図しても、タクシーは止まりませんでした
\\	照る照る坊主(てるてるぼうず) 坊主(ぼうず)
\\	その場で(ば) そのお巡りさんは、その場で、その男を逮捕した
\\	亀頭(きとう)
\\	紅葉(もみじ) 紅(べに)
\\	文明(ぶんめい) 西洋文明 中東は文明の発祥地である
\\	圧縮する(あっしゅく) 圧縮したファイル 5巻の著書を1巻に圧縮する
\\	サンダル
\\	誰かに 小言を言う(こごと) 叱る(しかります) 不平する(ふへい)
\\	小言はもうたくさんだ 彼女の小言にうんざりしている
\\	十行目(じゅうぎょうめ) 十行目の文を読んでください
\\	植物学(しょくぶつがく)
\\	飲む(のみます)
\\	迎える(むかえます)
\\	鉱夫(こうふ)
\\	ウーマン 女の人/方 ウーマンパワー
\\	具体的な(ぐたいてき) 彼の考え方は具体的でも中傷的(ちゅうしょうてき)でもなかった
\\	樹皮(じゅひ)
\\	武士(ぶし) 侍(さむらい)
\\	かび カビが生える(はえます)
\\	こう 
\\	その漢字はこう書くんです こう忙しくては、ゆっくり新聞も読めません
\\	おじゃまします。
\\	黄昏(たそがれ) 夕暮れ(ゆうぐれ)
\\	文明の黄昏 たそがれになった 黄昏の街(まち)
\\	同音異義語(どうおんいぎご)
\\	常識(じょうしき) 彼はとても学問があるが、常識に欠けている
\\	腱(けん) アキレス腱
\\	(警護) 食べる、飲む 
\\	(召し上がる) 召し上がります いただきます もっと召し上がりますか いいえ、十分いただきました
\\	結構(けっこう) 素晴らしい(すばらしい) 結構な贈り物 結構なお味です
\\	オーバー コート 彼のオーバーは擦り切れている
\\	瞬く(またたきます) ウインクする
\\	瞬く間に(まに) 時は瞬く間に過ぎ去った
\\	ワイシャツ
\\	文脈(ぶんみゃく) 言葉の意味は文脈によって変わる
\\	何かに 没頭する(ぼっとう) ふける(ふけります) 夢中に(むちゅう) 
\\	彼女は研究に没頭します
\\	これまで
\\	(警護)食べる、飲む 
\\	上がる(あがります) どうぞ、お茶をお上がりください
\\	それから
\\	瑪瑙(メノウ)
\\	モニター
\\	炊飯器(すいはんき)
\\	帰省(きせい) 規制(きせい) 
\\	帰宅する(きたく) 帰る(かえります) 彼は休暇で帰省中です 帰省列車[バス] 帰省ラッシュ
\\	辞書を貸してくださいませんか
\\	目玉(めだま) 眼球(がんきゅう)
\\	四日だけ(よっか)
\\	ガソリンスタンド
\\	カジュアル カジュアルウェア カジュアルシューズ
\\	糖尿病 (とうにょうびょう) 糖尿病患者 (とうにょうびょうかんじゃ)
\\	気圧計(きあつけい) 気圧(きあつ)
\\	吹雪(ふぶき) ひどい吹雪のために私達は外出できなかった
\\	新暦(しんれき) グレゴリオ暦(れき)
\\	~を 廃止(はいし) 死刑は廃止すべきである
\\	風呂場(ふろば) 浴室(よくしつ) お手洗い(おてあらい) トイレ
\\	中華人民共和国(ちゅうかじんみんきょうわこく) 中国(ちゅうごく)
\\	待合室(まちあいしつ) 待合室に五人の患者がいる
\\	延期する(えんき) 私たちは会合を次の日曜日に延期することに決めた 試験は延期された
\\	"神奈川沖浪裏 (かながわおきなみうら) 
\\	葛飾 北斎(かつしか ほくさい)
\\	お宅(おたく)
\\	高速道路(こうそくどうろ) 高速道路を走る(はしります)
\\	お寝小する(おねしょう) 寝小便する(ねしょうべん)
\\	~を 設計する(せっけい) その庭は日本式に設計されている この建物は有名な建築家に設計された
\\	電池(でんち) 太陽電池(たいようでんち) 電池で動く時計(うごく)
\\	執事(しつじ)
\\	原子(げんし) 原子核(げんしかく) 原爆(げんばく) 原子爆弾(げんしばくだん)
\\	こうもり 空を飛んでいるこうもりは蝶のように見える
\\	封印(ふういん) 
\\	判子(はんこ) 
\\	印(いん) 
\\	封印する 封印を切る
\\	ポーチ
\\	排水管(はいすいかん) 水道管(すいどうかん)
\\	排水(はいすい) 排水管が詰まっている
\\	ワゴン車(しゃ) バン
\\	旅行(りょこう) 旅(たび)
\\	アーチ 弓(ゆみ) アーチ型の門
\\	~源(みなもと) 川の源を尋ねる エネルギー源 再生可能エネルギー源(さいせいかのう)
\\	楽しむ(たのしむ) 人生を楽しむ
\\	私鉄(してつ)
\\	内部(の)(ないぶ)
\\	内側(うちがわ)
\\	室内(しつない)
\\	家の内部  内部から外をうかがう
\\	天国(てんごく)
\\	恐竜(きょうりゅう) かつては恐竜が地球を支配していた
\\	重量(じゅうりょう) 重さ(おもさ)
\\	体重(たいじゅう)
\\	重量が3ポンドある 正味重量 総重量
\\	クラシック音楽(おんがく)
\\	適切(な)(てきせつ) 適当(な)(てきとう) 「猫」といのは適切な言葉です
\\	ザリガニ
\\	精神(せいしん) 精神不安の 精神病(せいしんびょう) 体は精神と共に発達しなければならない
\\	築山(つきやま) 築山を作る
\\	拳(こぶし) 拳骨(げんこつ)
\\	拳/拳骨を握る(にぎります) 彼はデープルを拳骨でどんどんたたいた
\\	和服(わふく)
\\	七面鳥(しちめんちょう) 七面鳥を食べたことがある?
\\	足跡(あしあと) 我々は砂の上に足跡を見つけた
\\	元々(もともと) 初めから 元々彼女は彼なんか好きではなかった 人間は元々社会的動物だ
\\	いってきます 
\\	いってまいります。
\\	何回も(なんかいも) 私も何回もダイエットをした
\\	高温(こうおん 低温(ていおん)
\\	多少(たしょう) フランス語を多少話せます 私だって多少の貯金はある
\\	応用~(おうよう) 応用心理学 応用物理学
\\	無責任(な)(むせきにん)
\\	私も同じ意見です
\\	~に 目がない 彼は寿司に目がない
\\	それにします   カプチーノにする
\\	開店する(かいてん) その新しいスーパーは先月開店した
\\	ハンガー 衣文掛け(えもんかけ)
\\	一輪車(いちりんしゃ)
\\	塵(ちり) 
\\	埃(ほこり) 
\\	垢(あか) 
\\	ちりにまみれる ちりの山 ちりを払う
\\	虹(にじ) なんて美しい虹だろう!
\\	横隔膜(おうかくまく)
\\	防毒マスク(ぼうどく) 防毒面(ぼうどくめん) 防毒(ぼうどく)
\\	置く(おきます) 乗せる(のせます)
\\	~する 
\\	~経つ(たちます)
\\	~として 先生として働きます
\\	失う(うしないます) 損する(そん)
\\	損害する(そんがい) 
\\	喪失する(そうしつ) 
\\	私たちは命を失う危険が合った 信用を失う
\\	踏む(ふみます) アクセルを踏んだ 踏み絵(ふみえ)
\\	原子核(げんしかく)
\\	仲直りする(なかなおり) 夫婦はけんかしたがすぐに仲直りした
\\	うんと どっさり 
\\	たくさん うんと暖かくなった 彼はお金がうんとあって使えません あいつには金がうんとある
\\	長江(ちょうこう) 
\\	黄河(こうが)
\\	大学の寮(りょう)
\\	見渡す限り(みわたすかぎり) 見渡す(みわたします)
\\	横断歩道(おうだんほどう) 彼女は横断歩道を渡った(わたった)
\\	棒(ぼう) スティック 
\\	犬は棒の臭いを嗅ぐ(がきます) 棒でたたく
\\	空く(あく) 開く(あく)
\\	高速道路が空いています 明日の午後は空いていますか
\\	脊髄(せきずい) 脊髄液(せきずいえき)
\\	人前結婚式 (じんぜんけっこんしき)
\\	模範囚(もはんしゅう) 模範(もはん)
\\	工業地帯(こうぎょうちたい)
\\	尿道(にょうどう) 尿道炎(にょうどうえん)
\\	~たら どう ~ば どう 銀行に言ったらどうですか 銀行に行けばどうですか
\\	歯ブラシ(は)
\\	背が低い(せがひくい)
\\	二人とも
\\	~に つまずく 石につまずいて転ぶ
\\	~ておく ~とく 
\\	晩御飯を作っておく
\\	日章旗(にっしょうき)
\\	日の丸(ひのまる)b
\\	~を 落とす(おとす)
\\	水素(すいそ)
\\	推薦する(すいせん) 勧める(すすめます) あなたが推薦する場所がありますか
\\	アパート 
\\	住居(じゅうきょ)
\\	自宅(じたく) 
\\	お宅(おたく)
\\	泊まるところがあって、よかった
\\	買い物に行く
\\	個性(こせい) 個性的(こせいてき) みんなは独自の個性をもっています
\\	よかったら
\\	瓶(びん) ボトル 瓶には少し牛乳がある
\\	水も滴る(したたる) 水もしたたる美男
\\	寝る(ねます) 眠る(ねむります)
\\	無鉛ガソリン (むえんガソリン)
\\	チャックが開いている 社会の窓が開いている 
\\	家出する(いえで) 家出人(いえでにん)
\\	広告する(こうこく) 広告を出す 彼は家を売る広告を出した
\\	判子(はんこ) 封印(ふういん) 
\\	印(いん) 
\\	水道料金(すいどうりょうきん) 水道料金がずいぶん高い
\\	クリスマスキャロルを歌う
\\	労働(ろうどう) 労働者 肉体労働 頭脳労働
\\	包丁(ほうちょう)
\\	ようになる 日本語のニュースがわかるようになった
\\	注文する(ちゅうもん) 頼む(たのみます)
\\	病気(びょうき) 病(やまい)
\\	(ご)家族(かぞく)
\\	不用心(な)(ぶようじん) 危険(な)(きけん) 危ない(あぶない) 深夜の女の一人歩きは不用心だ
\\	霊柩車(れいきゅうしゃ)
\\	韓国(かんこく) 韓国の町の中でどこが一番好き?
\\	…、どうなるだろう。
\\	身分が高い(みぶん) 身分が低い
\\	~に適切(な)(てきせつ) ~に適当(な)(てきとう) ~にぴったり 
\\	その仕事は若い女性には適当ではない
\\	声高に(こわだかに) 大きな声で 
\\	声高に批判する
\\	家内(かない) 妻(つま)
\\	黒ミサ(くろみさ)
\\	険しい(けわしい) 急(な)(きゅう)
\\	険しい坂(さか)
\\	誇張する(こちょう) 彼女は自分の苦しみを誇張して話した
\\	独立(な)(どくりつ) 独立を宣言する(せんげん)
\\	乾燥機(かんそうき) 乾燥機で洗濯物を乾かす
\\	契約(けいやく) 協定(きょうてい)
\\	契約を結ぶ 契約を守る 契約を破る 契約書(けいやくしょ) 間もなく契約 (の期限) が切れる
\\	胃(い) 胃袋(いぶくろ)
\\	傘(かさ) 傘をさして歩く   傘を指す(さす)/広げる(ひろげます) 傘をすぼめる(すぼめます)
\\	恩赦(おんしゃ) 恩赦を与える(あたえます)
\\	ふらふらする ふらふらと店に入った 気持ちがふらふらしている
\\	従う(したがいます) 判決に従うほかない
\\	肩をすくめる (かたをすくめます)
\\	新人(しんじん) 彼は、まだ新人です
\\	申し込み(もうしこみ) この仕事に対する申し込みは3月1日より受け付けます
\\	終点(しゅうてん) 私は彼女の顔にカメラの終点を合わせた
\\	漢語(かんご) 看護(かんご)
\\	パトカー パトカーが速度で走っていた
\\	~せずに ~しないで
\\	ちなみに ところで
\\	ちなみに言う
\\	遅れる(おくれます) 遅刻する(ちこく) 遅くなる(おそくなります) バスに遅れたので、授業に遅刻した
\\	入場(にゅうじょう) 子供は入場できない
\\	算盤(そろばん)
\\	同情する(どうじょう) 彼に同情した
\\	負傷(ふしょう)
\\	傷(きず)
\\	怪我(けが)
\\	その医者は事故で負傷した四人を救った(すくった)
\\	伝記(でんき) 電気 
\\	偉大な天才は最短伝記を有する 自伝(じでん)
\\	裏切り者 (うらぎりもの)
\\	~を 嫌う(きらう) ~が 嫌いです
\\	剃る(そります) ひげを剃る
\\	~を 禁止する(きんし) ~を 禁じる(きんじます) 駐車禁止 立ち入り禁止
\\	皿(さら) 灰皿(はいざら)
\\	泡立て器(あわだてき) 卵を泡立てる
\\	唾を吐く(つばをはきます) シンガプルでは道路につばを吐くのは犯罪とされる
\\	フィジー
\\	鼻声で話す(はなごえ)
\\	リスト 彼の名前は候補者リストにある
\\	キリスト教(きょう) またキリスト教を信じてはいけなかった
\\	わら 溺れる(おぼれます)者はわらをも掴む
\\	刃物(はもの)
\\	手術(しゅじゅつ) 手術する 手術を行う 手術を施す(ほどこします) 手術を受ける
\\	海辺(うみべ) 海岸(かいがん)
\\	税金(ぜいきん) 税金を取り立てる
\\	一軒家(いっけんや) その家は一軒家だ
\\	譲る(ゆずります) 私は年寄りにいつも席を譲ってあげます
\\	内出血(ないしゅっけつ) 内出血を起こす
\\	二度と
\\	彼は二度と遅刻しないと約束した
\\	列島(れっとう) 日本列島 千島列島 (ちしまれっとう)
\\	ミキサー
\\	以前(いぜん) ~前に 彼女は結婚以前には夫に一度会った
\\	左翼(さよく) 右翼(うよく) 左翼的 彼らは政治的には左翼である
\\	紺(こん)の 医者は紺のスーツの上に白衣(はくい)を着ていた
\\	もはや 彼はもはや立ち上がる気力を失っている
\\	家屋(かおく) 家(いえ) ほとんどの家屋が粉々に破壊された 家屋税
\\	~を 省略する(しょうりゃく) ~を 省く(はぶきます) 履歴を省略せずに書いてください その文からこの言葉を省略すべきです
\\	慈善(じぜん) 慈善を行う
\\	半死半生の(はんしはんしょう) 半死半生の目にあわされる
\\	熱心(ねっしん) 熱心(な)(ねっしん) 熱心な仏教徒
\\	香り(かおり)
\\	臭い/匂い(におい)
\\	この花はよい香りがする
\\	影(かげ)
\\	簡単(な)(かんたん) 易しい(やさしい) 単純(な)(たんじゅん)
\\	同時に(どうじ) 二つの動作を同時に行う
\\	向かう(むかいます) その船は合衆国へ向かう何百人(なんびゃくにん)もの移民を運ぶ
\\	糞する(くそ)
\\	うんち
\\	うんこ
\\	忘れっぽい(わすれっぽい)
\\	自業自得だ(じごうじとく) 自業自得の目にあう 自業自得とあきらめなさい 勉強せずにファミコンばかりしてたんだから、試験に落ちても自業自得だよ
\\	浜(はま)
\\	岸辺(きしべ)
\\	海辺(うみべ)
\\	海岸(かいがん)
\\	好き? はい、好きだ だ 
\\	少しずつ(すこしずつ)
\\	祖先(そせん) 先祖(せんぞ)
\\	築く(きずきます) 城を築く 基礎(きそ)/礎(いしずえ)を築く 幸せな家庭を築く
\\	アレルギー アレルギー性の(せい) 卵アレルギーがある 牛乳アレルギーになる ほこりでアレルギーをおこした
\\	狂人(きょうじん) 彼は狂人のように振舞う(ふるまう)
\\	供給する(きょうきゅう) 
\\	電力の供給を止める 需要と供給(じゅようときょうきゅう) 
\\	ベンチ
\\	(仏教)
\\	金の亡者(かねのもうじゃ) 亡者(もうじゃ) 権力の亡者(けんりょく)
\\	(お)昼前に(ひるまえ)
\\	締め切り(しめきり) 締め切りに間に合う
\\	もたらす 持ってくる 
\\	お金が幸福をもたらすと一般に信じられている 彼からもたらされた情報によれば あらしは作物に大きな被害をもたらした
\\	出世頭(しゅっせがしら) 彼が高校時代のクラスの出世頭だ
\\	注意(ちゅうい) 道を渡る時、車に注意してください
\\	初めての人 初心者(しょしんしゃ)
\\	ファースト・フード
\\	戦闘(せんとう) 戦い(たたかい)
\\	先頭(せんとう)
\\	多くの兵士はその戦闘で負傷した
\\	一歩一歩 (いっぽいっぽ)
\\	ごみの分別(ぶんべつ) ごみを分別する
\\	おめでとう
\\	創造者(そうぞうしゃ) 創造する
\\	低空(ていくう) 頭上(ずじょう)
\\	裏(うら) 私の家の裏に小さい池がある
\\	マリワナ
\\	補講(ほこう) 歩行(ほこう)
\\	企業(きぎょう) 企業を起こす(おこす)
\\	共学(きょうがく)
\\	と思う この辞書便利だと思います
\\	スニーカー
\\	オランウータン
\\	会場(かいじょう) 今、会場を用意をしているところです
\\	樺太(からふと)
\\	便箋(びんせん) 便箋と封筒(ふうとう)のセット
\\	花びら(はなびら) バラの花びら
\\	仄暗い(ほのぐらい) ほの暗い光  まだほの暗い午前 
\\	時であった
\\	~難い(にくい) ~難い(かたい) やりにくい
\\	もう一度チャンスをくれ
\\	シャム双生児 (そうせいじ) 
\\	陽気(ようき)
\\	天気(てんき) 天候(てんこう) このところの陽気で春の花が咲き始めた いやな陽気 今日は陽気がいいので散歩に行こう
\\	狸(たぬき) 常用外
\\	空席(くうせき) 空席を埋める(うめます) 校長の地位が空席になった
\\	正門(せいもん) 声門(せいもん)
\\	それは声門です
\\	予約する(よやく) 予約はできますか
\\	被害(ひがい)を与える 痛みを与える
\\	姑(しゅうとめ) 私は姑と仲良く暮らしている
\\	感動する(かんどう) ~を 勘当する(かんどう)
\\	彼の話で僕は感動した
\\	切っ掛け(きっかけ)
\\	原因(げんいん)
\\	日本語を習い始めた切っ掛け
\\	鳥かご(とりかご) 檻 (おり)
\\	新しい本ができました 昼ご飯ができました
\\	ざけんなよ!
\\	利己的(な)(りこてき) 我がまま(な)(わがまま)
\\	我利我利(な)(がりがり)
\\	利己主義者(りこしゅぎしゃ) 利己心(りこしん) あいつは利己心のかたまりみたいなやつだ
\\	口紅(くちべに) 口紅をつける 口紅を直す
\\	母語(ぼご)
\\	遺書(いしょ) 遺書を書く
\\	ジム 私は週に2,3回、ジムで運動している
\\	花束(はなたば) ブーケ 女優はバラの花束を贈呈された
\\	パントマイム
\\	グルジア
\\	笑い(わらい)
\\	兵士(へいし) 軍人(ぐんじん) 戦士(せんし)
\\	川沿い(かわぞい) 川沿いにきれいな散歩道(さんぽみち)がある
\\	国交(こっこう) 国交を結ぶ 日本はあの国とはまだ国交がない. 日米間の国交
\\	繰り返し(くりかえし) 言葉の学習には繰り返しが必要だ
\\	博士(はくし) 博士(はかせ) 田中博士の講義は面白いです
\\	十代(じゅうだい) その小説家は十代の若者の間でかなり人気がある 彼女は十代で結婚した
\\	夏休みぼけ
\\	危地(きち) 危地を脱する(だっする) 危地に陥る(おちいります)
\\	別に(べつ) 
\\	ない 今日は別に忙しくない 「何か欲しいものがありますか」「いえ,別に」
\\	中米(ちゅうべい)
\\	ペナルティーキック   コーナーキック フリーキック
\\	有限責任会社 (ゆうげんせきにんか/がいしゃ) 
\\	(エルエルシー) 
\\	民主主義(みんしゅしゅぎ)
\\	扁桃腺(へんとうせん)  扁桃腺炎
\\	句動詞 (くどうし)
\\	精神科病院 (せいしんびょういん)
\\	~たらどうですか ~たらいかがですか 先生に相談したら いかがですか/どうですか
\\	(新聞)記者 (しんぶんきしゃ) 医者と記者と、どちらが大変だと思う
\\	蚯蚓(みみず) 常用外
\\	袋(ふくろ)
\\	童話(どうわ) 物語(ものがたり) 昔話(むかしばなし)
\\	彼氏(かれし) 彼 ボーイフレンド
\\	水中花(すいちゅうか)
\\	代名詞(だいめいし) 
\\	彼の名は冒険の代名詞となった ハリウッドは映画の代名詞だ
\\	出獄する(しゅつごく) 出獄者(しゅつごくしゃ) 彼は終戦直後に出獄した
\\	ゼロ 零(れい)
\\	手数を掛ける(てすう) この子供はちっとも手数が掛からない
\\	紙一重(の差) (かみひとえ(のさ) 紙一重の差で勝った
\\	危機一髪(ききいっぱつ)
\\	断食(だんじき) 断食する 断食を守る
\\	掃除する(そうじ) 
\\	除く(のぞきます)
\\	に対する影響(えいきょう) ~に 影響を与える 社会に対する科学の影響は大きいです
\\	改善(かいぜん) この技術は大いに改善の余地がある 暮らしを改善する
\\	オギノ式(しき)
\\	連続殺人者(れんぞくさつじんしゃ) 切り裂きジャックは連続殺人者だった 連続(れんぞく)
\\	余地(よち) 選択の余地がありません 妥協(だきょう)の余地がない
\\	貸し出す(かしだす) 貸す(かします) 本を貸し出す 本を借りる
\\	方針(ほうしん) 
\\	磁針(じしん)
\\	事実(じじつ) うわさは事実(じじつ)だと分かった
\\	食道(しょくどう) 食堂(しょくどう)
\\	陰陽(いんよう) 引用(いんよう) 
\\	暖房(だんぼう) お宅の暖房はどのようにしていますか
\\	手に入れる(てにいれます) 受け取る(うけとります) 貰う(もらいます) 得る(えります) 私は直接にその情報を手に入れた
\\	嫌(な)(いや) 嫌な女 嫌なにおい たばこの煙が嫌だ
\\	が、
\\	けど、
\\	これは小さいですが、重いです
\\	釣り糸(つりいと) 釣りざおと釣り糸 釣り糸を垂れる(たれます)
\\	さあ
\\	天井(てんじょう)
\\	公務員(こうむいん) 彼は20年前、大学を卒業するとすぐに公務員になった
\\	郊外(こうがい)
\\	捜索(そうさく) 
\\	探す(さがします) 
\\	行方不明の飛行機を捜索する 彼は家出した娘の捜索願を出した
\\	模範生(もはんせい)
\\	上がる(あがる)
\\	増減(ぞうげん) 大都市の人口は増減が少ない
\\	用意(ようい) 準備(じゅんび) 支度(したく) 母は夕食の用意をしていた
\\	下りる(おります) 下がる(さがります) 降りる(おります) 階段を下りる 早朝に露(つゆ)が下りた
\\	登山(とざん) 山登り(やまのぼり) こんなに悪い天候の中で登山するべきではない
\\	上旬(じょうじゅん) 中旬(ちゅうじゅん) 下旬(げじゅん)
\\	電流(でんりゅう) 電気(でんき) 電流の強さ 電流が切れている
\\	ストッキング
\\	内緒(ないしょ) ひそか(な)
\\	秘密(ひみつ)
\\	神秘(しんぴ)
\\	この話は内緒にしておこう
\\	ヘアピン
\\	道路(どうろ) 道(みち) 道路で遊んではいけない
\\	手入れする(ていれ) よく手入れされた庭 肌の手入れ
\\	乾癬(かんせん) 
\\	感染(かんせん)
\\	モグラ塚(づか)
\\	エレガント(な) スマート(な)
\\	記事(きじ) 雉(きじ)
\\	生地(きじ)
\\	彼は夢中になってその記事を読んでいるように見える
\\	登録する(とうろく) 職業安定所に登録する
\\	秋(あき)
\\	うまくいく 君のおかげで仕事はうまく行った
\\	スカーフ 
\\	肥える(こえます) 太る(ふとります) 超える(こえます)
\\	吠える(ほえます) うなる(うなります) 吠え声(ほえごえ) 彼女は吠える犬を怖がる(こわがります)
\\	さよなら|さようなら!
\\	不快(ふかい) 私たちはそこで不快な経験(けいけん)をした
\\	出っ歯(でっぱ) 出っ歯の男の子
\\	物質(ぶっしつ)
\\	イェスの弟子(でし) 弟子(でし)
\\	口移し法 (くちうつしほう)
\\	マラリア マラリア熱(ねつ) マラリアにかかる
\\	真っ青(な)(まっさお) 青白い(あおじろい)
\\	あなたは顔が真っ青です 真っ青な海
\\	同性婚(どうせいこん)
\\	分母(ぶんぼ) 分数(ぶんすう) 分子(ぶんし)
\\	ピカピカ光る(ひかります) 輝く(かがやきます) ぎらぎらする 
\\	ぴかぴかの靴 無数の星がぴかぴか輝いていた
\\	振りをする(ふり) その子供は眠っている振りした その神父は人前で真面目な(まじめ)振りをする
\\	自宅電話番号
\\	寝かす(ねかします) 子供を寝かす
\\	後(のち)(の) 後(あと) もっとのちそれを説明する
\\	チェス 将棋(しょうぎ)
\\	また別のときにチェスしましょう
\\	恐ろしい(おそろしい) ひどい あの吹雪の中で車を運転は恐ろしい経験だった
\\	空き(あき) 空っぽ(からっぽ)
\\	空き家(あきや) 空き瓶(あきびん)
\\	冬の山(ふゆのやま)
\\	金庫(きんこ) 禁固(きんこ)
\\	金庫に入れる
\\	衣装(いしょう) 衣装をたくさん持っている 芝居の衣装
\\	女神(じょしん) 他の読み方: 女神(めがみ) 自由の女神の像(ぞう)
\\	股(また) 誰かの股をけりあげる
\\	バンカー
\\	バイオリン バイオリンを弾く
\\	丘(おか)
\\	群(むれ)
\\	群(ぐん)
\\	虫の群れ
\\	浮気っぽい (うわきっぽい)
\\	見地(けんち) 見方(みかた) 厳密な科学的な見地からは、歴史は科学と言えない
\\	~ばいいですか だれを呼べばいいですか
\\	素敵(な)(すてき)
\\	まだ ~ていません レポートはもう出来ましたか いいえ、まだ書いていません
\\	でしょう/だろう あれは鈴木さんの車でしょう
\\	頭上に(ずじょうに) 低空に(ていくうに)
\\	蛭(ひる) 昼(ひる)
\\	ひるで血を取る
\\	抗生物質(こうせいぶっしつ) 抗生物質はなくなるまで飲んで下さい
\\	望み(のぞみ) 希望(きぼう) 成功の望みはない
\\	東洋的(な)(とうようてき) 西洋的(な)(せいようてき)
\\	市長(しちょう) 市長選挙
\\	行方不明(ゆくえふめい) 尋ね(たずね)
\\	行方不明者 村人たちは皆、行方不明になった猫を探す
\\	蛹(さなぎ) 蛹は繭を作る
\\	折り紙(おりがみ)
\\	土竜(もぐら) 義訓
\\	手段(しゅだん) 生活を楽しむ手段 あらゆる可能な手段を選ぶ
\\	書き込む(かきこむ) 記入する(きにゅう)
\\	この用紙に書き込むだけです
\\	記念(きねん) 記念碑(きねんひ) 20周年記念に乾杯(かんぱい)
\\	凍らせる (こおらせます) 冷凍する(れいとう) 水を凍らせた
\\	月給(げっきゅう) 時給(じきゅう)
\\	鳥類学(ちょうるいがく)
\\	喘ぐ(あえぎます) 喘ぎながら言う 祖母は喘ぎながら坂を登る
\\	気(き) 気のいい人 彼と一緒だと気が楽である
\\	~を 自慢する(じまん) ~を 誇る(ほこります) 
\\	~が 得意になる(とくい) 父はハンサムなのを自慢している 自慢話(じまんばなし)
\\	形成する(けいせい) 人柄(ひとがら)の形成(けいせい) 結成(けっせい) イメージはマスコミの情報に形成された
\\	通過する(つうか) 通貨(つうか) 
\\	間もなくトンネルを通過する
\\	コンサート 演奏会(えんそうかい) 音楽会(おんがくかい)
\\	重さ(おもさ) 重量(じゅうりょう) 体重(たいじゅう)
\\	痛み止め(いたみどめ) 鎮痛剤(ちんつうざい)(医) 痛くなったら鎮痛剤を飲んでください
\\	新緑の(しんりょく) 新緑の山々
\\	~に 屈み込む(かがみこむ)
\\	銀(ぎん) 白金/銀(しろがね)
\\	尖る(とがります)
\\	鋭い(するどい) 尖った鉛筆
\\	科学者(かがくしゃ)
\\	ヨウ素(ようそ) 要素(ようそ)
\\	切手(きって) 小切手(こぎって)
\\	依頼(いらい) お願い(おねがい) 頼み(たのみ) 彼の依頼は命令に等しい(ひとしい)
\\	発音(はつおん) 発音する
\\	真珠湾(しんじゅわん)
\\	~が 治る(なおります)
\\	解放する(かいほう) 放す(はなします) 王はその囚人を解放するように命令した
\\	玄人(くろうと) 素人(しろうと) 彼は玄人に近い 料理の玄人
\\	殺人(さつじん)
\\	殺害(さつがい) その男は殺人を犯した(おかした) 殺人未遂(さつじんみすい)
\\	写真家(しゃしんか) 撮影者(さつえいしゃ) フォトグラファー
\\	皮膚科学(ひふかがく)
\\	一方で(は)(いっぽう) ~他方で(は)(たほう) この薬は、一方では症状を軽くするが、他方では強い副作用がある
\\	コップ
\\	通信販売(つうしんはんばい) 通信販売で買った
\\	~を 中断する(ちゅうだん) 裁判を中断するのは不可能だ その試合は一時中断された
\\	前へ行く 先に行く
\\	約~(やく) 約269年かかった
\\	ぶつかる(ぶつかります) 2台の車が正面(しょうめん)からぶつかった
\\	適度(な)(てきど) 適度に運動する 適度の飲酒は健康によい
\\	私もそう思います
\\	才能豊か(な) (さいのうゆたか)
\\	振る(ふります) 犬は尾を振る
\\	タバコ
\\	バケツ
\\	固有(こゆう)(な) 羽は鳥に固有のものだ
\\	四角(しかく)
\\	番(ばん) 次は山田さんの読む番です
\\	解決(かいけつ) 解決する(かいけつ) 解く(ときます)
\\	重大(な)(じゅうだい) 国際情勢は重大になりつつある 重大な問題
\\	迷路(めいろ) この迷路は抜け出すのがとても難しい
\\	靴べら(くつべら)
\\	運命(うんめい) 運命のいたずら 運命を支配する
\\	また明日!(あした)
\\	残念ながら
\\	容器(ようき) 陽気(ようき)
\\	空きの容器は一番音を立てる
\\	光(ひかり) 電気が消えたので、ろうそくの光で勉強した
\\	単なる(たんなる) ただの 田中さんは単なる友人だ 彼は単なる文士ではない
\\	淡水魚(たんすいぎょ)
\\	暖炉(だんろ) 煙突(えんとつ)
\\	服飾品(ふくしょくひん) アクセサリー
\\	梅毒(ばいどく)
\\	全焼する(ぜんしょう) 燃える(もえる)
\\	10戸(こ)が全焼した
\\	~に 属する(ぞく) ~に 所属する(しょぞく) 彼は上流階級に属する
\\	ずんぐりした ずんぐりした女だ
\\	ロシアンルーレット
\\	まさかの時 私達はまさかの時のために貯金しなきゃ
\\	温暖化現象 (おんだんかげんしょう) 温室効果 (おんしつこうか)
\\	このくらいの…
\\	カルト 新宗教(しんしゅうきょう) セクト 
\\	そういえば、そうですね
\\	緯線(いせん) 子午線(しごせん)
\\	植物(しょくぶつ)
\\	激しい(はげしい) 3時間以上も激しい雨が降り続いている
\\	逆にする アルファベットの順を逆にしなさい
\\	具体化する(ぐたいか) 実現する(じつげん) 
\\	その計画は具体化された 彼の理論はこの本の中で具体化されている
\\	~に 勝つ(かちます) 戦いに勝つ 敵に勝つ
\\	画数(かくすう)
\\	しわ 額にしわを寄せる しわが寄る(よります)
\\	いいえ、結構です(けっこう)
\\	市民権(しみんけん)
\\	国籍(こくせき)
\\	彼にアメリカの市民権が与えられるだろう
\\	獣姦(じゅうかん)
\\	相手(あいて) 結婚の相手を探してる
\\	異論(いろん) 私としては異論はありません この問題に関してはいろいろ異論がある
\\	専業主婦(せんぎょうしゅふ) (職に就かないで、家事に専念する主婦) 主婦(しゅふ)
\\	~に 劣る(おとります) この点で私は彼に劣る 暗算ではだれにも劣らなかった
\\	十字路(じゅうじろ) 
\\	交差点(こうさてん) 
\\	十字路に立つ
\\	洗濯物(せんたくもの) 雨が降り始めると、彼女は息子に洗濯物を取り込むように言った
\\	全国民(ぜんこくみん) 全国民が選挙で投票した(とうひょう)
\\	旬の野菜(しゅんのやさい)
\\	ダックスフント 
\\	いってまいります
\\	アーチスト 芸術家(げいじゅつか) 芸術的(な)
\\	派遣する(はけん) アフガニスタンに軍隊を派遣する
\\	名称(めいしょう) 名所(めいしょ)
\\	名称を与える この会の名称は
\\	とする
\\	キッズ
\\	回転木馬(かいてんもくば)
\\	同性カップル
\\	~を 容認する(ようにん) ~を 承知する(しょうち)
\\	~に 寛容になる(かんよう)
\\	案外(あんがい) 意外(いがい)    以外(いがい)
\\	この本は案外に安いですね
\\	どうしましたか
\\	解明する(かいめい)
\\	説明する(せつめい)
\\	秘密を解明した 動機を解明する
\\	祈る(いのります) お祈りする(おいのり)
\\	非難すべき(ひなん)
\\	断片(だんぺん) 破片(はへん)
\\	このパズルに500の断片がある 事故の断片的ニュース
\\	人格(じんかく)
\\	性格(せいかく)
\\	人柄(ひとがら)
\\	人格の分裂 【心理】 彼には人格が備わっている(そなわる) 備わる(そなわる)
\\	中退する(ちゅうたい) 
\\	中途退学(ちゅうとたいがく) 大学を中退した人 中退者(ちゅうたいしゃ)
\\	届ける(とどけます) 盗難を警察に届ける 彼は彼女に花束を届けた
\\	やがて やがて城が見えてきた
\\	ベッド 
\\	寝台(しんだい)
\\	それなら それでは 
\\	そうか, それなら話は別だ
\\	図書館(としょかん) ライブラリー
\\	三角(形)(さんかく「けい」)
\\	雪合戦(ゆきがっせん) 雪合戦をする
\\	二分の一(にぶんのいち)
\\	耳が聞こえない 耳が聞こえなくなる ろう者(ろうしゃ)
\\	(星座) 
\\	水瓶座(みずがめざ) 水瓶(みずがめ)
\\	痙攣(けいれん) 足にけいれんを起こす 泳いでいてけいれんを起こした まぶたがけいれんしている
\\	竹(たけ)
\\	~を 着く(はきます)
\\	じゃんけん じゃんけんポン!
\\	しつれいしました。 しつれいします。
\\	いくつか(の) いくつかのシーン
\\	~を 指を折って数える ~を 指で数える
\\	服飾デザイナー(ふくしょく)
\\	感嘆符(かんたんふ) ~に感嘆する
\\	引き金(ひきがね) 引き金を引く
\\	品(ひん) 品物(しなもの)
\\	商品(しょうひん) 
\\	製品(せいひん)
\\	泣き虫(なきむし) やーい, 泣き虫!
\\	鶴(つる) 鶴を折ります
\\	告白(こくはく) 愛を告白する 私は罪(ざい)を告白した
\\	争いする(あらそい) けんかする ~誰かと ~について
\\	易々(やすやす) 彼は私の質問に易々と答えた
\\	ごくろうさま。
\\	指導者(しどうしゃ)
\\	統率者(とうそつしゃ)
\\	リーダー 先生の指導を受ける
\\	~に 泊まる(とまります) 宿泊する(しゅくはく) ホテルに宿泊する
\\	交わる(まじわります) 
\\	交差する(こうさ) そこで二つの高速道路が交わっている 朱に交われば赤くなる
\\	習得する(しゅうとく) その技術は簡単に習得できる
\\	内容(ないよう) 本文(ほんぶん)
\\	彼のスピーチ内容はとてもよかった
\\	~て/で 仕方がない(しかた) そんな本は読んでも仕方がない
\\	手回しオルガン
\\	創造する(そうぞう) 想像(そうぞう)
\\	想像しにくい 創造的 創造力(そうぞうりょく)
\\	盲人(もうじん) 盲目的(な)(もうもくてき)
\\	相続人(そうぞくにん) 跡継ぎ(あとつぎ)
\\	彼はばく大な財産(ざいさん)の相続人である
\\	手品師(てじなし) 彼は手品で帽子からうさぎを出した
\\	留学生(りゅうがくせい)
\\	見解(けんかい) 意見(いけん) この件についての我々の見解は一致している
\\	お大事に(おだいじに)
\\	(漢字!!!) 
\\	熱い(あつい) 
\\	厚い(あつい) 
\\	暑い(あつい) 
\\	新型(しんがた) 旧型(きゅうがた) 新型の自動車
\\	更に(さらに) さらに友人に感謝していると言った さらに50円安いです
\\	観察(かんさつ) 観測(かんそく) その男の子は一日中鳥を観察していた
\\	フッ素(ふっそ)
\\	黒板(こくばん) 黒板をふく
\\	食事する(しょくじ)
\\	昔々(むかしむかし)
\\	養子(ようし) 用紙(ようし)
\\	養子をとる
\\	だけで 私は筆跡(ひっせき)を見ただけで誰が書いたが分かった 君はここに座っているだけでよい 読む/読んだだけで
\\	カメレオン カメレオンはどんな色にでもなることが出来る
\\	時間がかかる
\\	バランス 彼はバランスを失って自転車から落ちた
\\	固い(かたい) 難い(かたい)
\\	下から二段目 (したからにだんめ)
\\	あちこち あちらこちら 所々(ところどころ) 老人はあちらこちら帽子を探し回った
\\	多い(おおい)
\\	これで これで結構です これで自分がばかだったことが分かった
\\	これから
\\	引き出す(ひきだします) お引き出しですか
\\	悪口(わるくち) 他人の悪口を言うもんじゃない
\\	しつこい
\\	しつこいせき あなたは英語を学ぶ努力をしつこいくらいにやるべきだ いつまでもしつこく言うなよ
\\	倹約する(けんやく) 
\\	節約する(せつやく) 
\\	貯金する(ちょきん)
\\	食費を倹約する 収入が低いから倹約しなきゃ
\\	世界は狭いね
\\	~号室(ごうしつ) 213号室の前でお待ちください
\\	健康的(な)(けんこうてき) 健康にいい 海辺の空気は健康的だ
\\	~料(りょう) 料金(りょうきん) 代金(だいきん)
\\	入場料(にゅうじょうりょう) 電話料
\\	労働省(ろうどうしょう)
\\	手すり(てすり) 手すりに捕まりなさい  (つかまる)
\\	引越しする(ひっこし) 家に変わる 彼は新しいに変わります
\\	蹄鉄(ていてつ) 馬に蹄鉄を打つ 蹄(ひづめ)
\\	ぽつぽつ)	擬態語(ぎたいご)
\\	道がこんでこんで 道が込んでいる 道が渋滞している
\\	電源(でんげん) 電源を開
\\	~を 承知する(しょうち) そのことは承知しています 父が私達の結婚を承知しないので、困っています
\\	ける(けります) 彼はわざと私をけった
\\	百合(ゆり)
\\	死海(しかい)
\\	ひげ あごひげ 口ひげ(くちひげ)
\\	縞馬(しまうま)
\\	ノーベル賞(しょう)
\\	孤独感(こどくかん)
\\	安全ピン ピン
\\	やかましい 騒がしい(さわがしい) 煩い(うるさい) やかましいラジオの音 それはやかましい社会問題である
\\	祝う(いわいます) 祝いする(いわい) 私達は結婚の10周年の祝いをする
\\	国境(こっきょう)
\\	境界(きょうかい)
\\	境(さかい)
\\	彼は国境を越えて言った
\\	中指(なかゆび)
\\	泉(いずみ) トレヴィの泉(いずみ)
\\	仕方がない(しかた)
\\	花吹雪(はなふぶき)
\\	岩(いわ) 岩石(がんせき)
\\	ずるい ずる賢い(ずるがしこい) あいつはずるい事をする
\\	個人情報(こじんじょうほう)
\\	運動会(うんどうかい)
\\	便り(たより)
\\	頼り(たより)
\\	連絡(れんらく)
\\	便りをください 彼から便りがありましたか 便りのないのはよい便り
\\	樽(たる) ビール樽 樽詰めのビール
\\	さんご礁(さんごしょう)
\\	~に 代わって(話/書) /~に 代わり (書) ~代りに 
\\	病気の山田先生に代わって、鈴木先生が教えて下さった
\\	~どうしたらいいですか ファックスが故障なんですが、どうしたらいい
\\	ボイコットする その店のボイコットを決めた
\\	降伏する(こうふく) 幸福(こうふく) 
\\	降伏するより死んだほうがします
\\	真ん中に(まんなか)
\\	亜鉛(あえん) 真鍮は銅と亜鉛の合金である
\\	水夫(すいふ) 彼は水夫になることを望んだ(のぞんだ)
\\	汗が出る(あせ) 汗をかく ハンカチで汗を拭いた(ふく)
\\	出張(しゅっちょう) 出張をキャンセルする
\\	茶わん(ちゃわん)
\\	身代金(みのしろきん) そのハイジャック犯は200万ドルの身代金を要求した
\\	王朝(おうちょう) ブルボン王朝
\\	チュニジア
\\	条約(じょうやく) 和平条約(わへいじょうやく) 条約を結ぶ
\\	そんなに そなに大きくない
\\	水筒(すいとう) 水筒にはほとんど水が残っていない
\\	朝方(あさがた) 早朝(そうちょう)
\\	朝方に/早朝に 朝方降った雨のおかげで今日は緑がいきいきしている
\\	傾げる(かしげます) 傾ぐ(かしぎます) 首をかしげる
\\	乾く(かわきます) 乾かす(かわかす) 乾いた シャツが乾いた
\\	木こり(きこり) 
\\	杯(はい)
\\	~易い(やすい) 分かりやすい
\\	義訓(ぎくん) 土竜(もぐら)
\\	垢(あか) 赤(あか) 
\\	垢を落とす
\\	炒める(いためます) 玉ねぎをいためた
\\	中央アフリカ
\\	大口(おおくち)をたたく 
\\	~を 自慢する(じまん)
\\	~ということ を 誇る(ほこります)
\\	虫食いの(むしくい) 虫食いのきゅうり 虫食いのセーター
\\	管理人(かんりにん) 管理(かんり)
\\	傾向(けいこう) 私は太る傾向がある 英文学の最近の傾向
\\	嫉妬(しっと) 焼餅(やきもち) 嫉妬の余り妻を殺した 彼女は美人の妹をいつも嫉妬していた
\\	名詞(めいし) 名刺(めいし) 
\\	マネキン
\\	狩猟(しゅりょう) 森へ狩猟に行った 狩る(かります)
\\	有機(な)(ゆうき) 勇気 
\\	有機化学
\\	暗記する(あんき) そらで覚える(おぼえます) 
\\	子供たちは掛け算を暗記している
\\	煙(けむり) 火事の煙
\\	最後(の)(さいご) 
\\	最終(の)(さいしゅう) 
\\	最後の最後まで
\\	雪かきする(ゆきかき) 道の雪かきをする
\\	悪因悪果 (あくいんあっか)
\\	三つ子(みつご) 
\\	四つ子(よつご)
\\	個(こ) ~つ
\\	引きつる(ひきつります) 足が引きつる 彼の顔は怒りで引きつった
\\	黄河(こうが) 
\\	長江(ちょうこう)
\\	おきのどくに。
\\	ドライヤー
\\	お久しぶり 久々(ひさびさ)
\\	危険人物 彼は社会に危険人物だ 彼はわが社にとって危険人物になるかもしれない
\\	アイデアを考える
\\	少年(しょうねん) 男の子(おとこのこ)
\\	トレーナー 
\\	そのぎらぎらしたアクセサリーは君のトレーナーに合わないよ
\\	マナー 礼儀(れいぎ) 
\\	作法(さほう) 行儀(ぎょうぎ) 
\\	エチケット 
\\	最近の若い人は電車の中でマナーが悪いです
\\	ともかく ともかく彼は嫌いだ ともかく夕食にしよう
\\	3つ星のレストラン
\\	平日(へいじつ)
\\	時期(じき) 一年中での楽しい時期です
\\	待ち伏せ(まちぶせ) 待ち伏せする/待ち伏せる 待ち伏せを食う(くいます)
\\	男性より女性が先(さき)
\\	滞在(たいざい) いつまでご滞在ですか
\\	(時間)ぎりぎりに ぎりぎり間に合った
\\	作法(さほう) 行儀(ぎょうぎ) マナー 彼の貴族的な作法を感心する
\\	アジア
\\	道具(どうぐ)
\\	~過ぎ(すぎ) 二十歳過ぎの女の人(はたち) 食べすぎ
\\	孤立(こりつ) 孤立した 人を仲間から孤立させる その国は経済的にも政治的にも孤立している
\\	理科(りか)
\\	科学(かがく) 
\\	理科系に進むつもりだ 理科大学
\\	無秩序(の)(むちつじょ) 教室の中は無秩序の状態だった
\\	呆れる(あきれます) あきれるほどの  彼はあきれた人間だ 彼の不注意にあきれてしまった
\\	つなぐ(つなぎます) つなげる(つなげます) 接続する(せつぞく) 結ぶ(むすびます)
\\	このプリンターはそのコンピューターにはつなげられない
\\	昨夜(さくや) 昨夜はよくおやすみになれましたか
\\	臆病者(おくびょうもの) 一言で(いちごん)言えば彼は臆病者だ
\\	天道虫(てんとうむし)
\\	前書き(まえがき) 手紙の前書き
\\	口癖(くちぐせ) 最近「横入りしないでください」という表現が口癖になった 五歳の子がお母さんの口癖をまねして皆を笑わせた
\\	歩む(あゆみます) 歩く(あるきます) 道を歩む
\\	神主(さん)(かんぬし)
\\	貴族(きぞく) 彼は貴族の出である(で)
\\	理想(りそう) 理想的(りそうてき) 絶好の(ぜっこう) 彼は私にとって理想の夫です
\\	愛は盲目(あいはもうもく) 桃子の新しい彼見た?いったいどこがいいのかしら、、「愛は盲目」というけど、ほんとね
\\	働き手(はたらきて)
\\	入試(にゅうし) 
\\	入学試験(にゅうがくしけん) 私は入試の準備をしている
\\	唯一の(ゆいいつ) 彼は一人っ子だので、唯一の相続人(そうぞうくにん)だった
\\	おっぱい飲む おっぱい 
\\	ぱいおつ
\\	怒る(おこります) 起こる(おこります)
\\	~を
\\	人脈(じんみゃく)
\\	(お)金持ち(かねもち)
\\	柔らかい(やわらかい) その靴は革に似た何か柔らかい材料で出来ていた
\\	焼き鳥(やきとり) 焼き魚(やきさかな) 焼肉(やきにく)
\\	倒す(たおす) ビール瓶をひじで倒す
\\	豊胸手術 (ほうきょうしゅじゅつ) 豊胸(ほうきょう) 美容整形(びようせいけい)
\\	毒(どく) 誰かに毒を盛る(もります)
\\	前世(ぜんせ) 前世に
\\	(お金)	細かいお金 (こまかい)
\\	焦がす(こがします) 焦げる(こげます) たばこの火でカーペットを焦がした
\\	手荷物(てにもつ) 彼はトラックに手荷物を積んだ(つむ)
\\	~を 始める(はじめる) ~が 始まる
\\	吸入器(きゅうにゅうき) 吸入(きゅうにゅう)
\\	毛糸(けいと
\\	モダン(な)
\\	粘液(ねんえき)
\\	滑り(ぬめり)
\\	滑りのある 粘液の 粘液を分泌(ぶんぴつ)する
\\	(ご)都合(つごう) いつご都合よろしいですか
\\	昼間(ひるま)
\\	故郷(ふるさと) 私は故郷が恋しい/懐かしい
\\	続く(つづきます) 交通渋滞が一時間続いた
\\	形式(けいしき) 
\\	この書類は実に複雑な形式である 注意書きは質疑応答の形式で書いてある
\\	(練り)歯磨き(ねりはみがき)
\\	留守にする/です(るす) しばらく日本を留守にします 彼は留守です
\\	礼儀(れいぎ) 作法(さほう) 行儀(ぎょうぎ) 
\\	マナー 
\\	エチケット 
\\	礼儀に気をつけなければならない
\\	急性(の)(きゅうせい) 旧姓(きゅうせい)
\\	急性の痛み 急性アルコール中毒
\\	初夢(はつゆめ) どんな初夢を見た
\\	~と 確信する(かくしん) 確信はありません 成功すると確信しているのか
\\	物体(ぶったい) これらの物体は本来は有毒ではない
\\	中華鍋(ちゅうかなべ)
\\	高速道路(こうそくどうろ) 高速道路が込んでいます
\\	角(つの)
\\	盾(たて) 縦(たて)
\\	番線(ばんせん) 9番線の急行に乗りなさい
\\	昨日(きのう) 昨日(さくじつ)
\\	水ぶくれ(みずぶくれ)
\\	バザー 市(いち)
\\	粗末にする(そまつ) 親を粗末にする 命を粗末にするな
\\	ショップ ストア 店(みせ) 店屋(みせや)
\\	歯(は) 
\\	葉(は)
\\	(星座)	山羊座(やぎざ)
\\	しゃがむ
\\	大学に入る 入学する(にゅうがく)
\\	不治の(ふじ) 彼は不治の病にかかります
\\	甘党(あまとう)
\\	甘口(あまくち)
\\	ぼくは甘党だ
\\	四字熟語(よじじゅくご)
\\	印刷(いんさつ) この機械は1分間に60ページ印刷できる
\\	立ち寄る(たちよります) 
\\	途中で立ち寄る 京都に 
\\	日立ち寄る
\\	一人歩きする (ひとりあるき)
\\	木材(もくざい)
\\	早速(さっそく) 直ぐ(すぐ) 早速のお返事ありがとう 早速…する
\\	ヨーロッパ 欧州(おうしゅう) ヨーロッパで行ったことがある国はどこですか
\\	平和(へいわ)
\\	和平(わへい)
\\	角(かど)
\\	冷暖房(れいだんぼう) 
\\	冷房(れいぼう) 
\\	その部屋は冷房がついていますか
\\	小枝(こえだ) 大枝(おおえだ)
\\	別の時に(べつのときに) また別のときにチェスしましょう
\\	ソ連(それん) ソビエト連邦(れんぽう)
\\	見渡す(みわたします) 聴衆を見渡す
\\	口語(こうご)
\\	飴(あめ) 常用外 雨(あめ) 
\\	オーストラリア
\\	明かす(あかす) 暴く(あばく) 秘密を明かす 彼女は自分の秘密を明かすことには気が進まなかった
\\	先週に映画を見た人は誰?
\\	運動(うんどう) 
\\	反戦運動(はんせんうんどう) 世界政府をつくろうとする運動
\\	エッチ(な) エッチな話 いやっ,触らないでよ,エッチな人ね
\\	退化(たいか) 大家(たいか)
\\	その器官は使用しないので退化した
\\	個人(こじん)
\\	不道徳(な)(ふどうとく) 不道徳な行為
\\	小刻みに(こきざみ) 少しずつ 
\\	段々(だんだん)
\\	次第に(しだい)
\\	小刻みに歩く ニンジンを小刻みに切る
\\	閉所恐怖症 (へいしょきょうふしょう)
\\	野球(やきゅう) サッカーと野球とどちらが面白い?
\\	コピー機(き)
\\	限る(かぎります) 
\\	我々の予算は10万円以内に限られている 入場者は女性に限ります
\\	新しい 新しい型の自動車
\\	モスク 回教寺院(かいきょうじいん)
\\	~が 上手です(じょうず) ~が得意です(とくい)
\\	路上で(ろじょう) 路上生活者
\\	リモコン
\\	~につれて/つれ 日本語が上達するにつれて、日本人の友達が増えた 気温の変化につれて
\\	車道(しゃどう) 車道の向こう側にあるバリケードは、「入れない」ということである
\\	大晦日(おおみそか)
\\	しっかり しっかりした友達 信用できる友達 しっかり勉強する
\\	探知する(たんち) 探知機 魚群を探知する
\\	主語(しゅご) 文には普通、主語と動詞がある
\\	ただで 無料(な)(むりょう)
\\	天才(てんさい) 彼は数学の天才だった
\\	世界中(せかいじゅう) これは世界中で一番長い橋です
\\	林業(りんぎょう)
\\	一輪社(いちりんしゃ)
\\	ノックする 叩く(たたきます) ドアをノックしなさい
\\	床擦れ(とこずれ)
\\	咥える(くわえます) 加える(くわえる) 
\\	犬がボールをくわえて戻ってきた 猫が魚をくわえて逃げていった
\\	切り株(きりかぶ)
\\	ものすごい ものすごく金持ちの ものすごいあらし
\\	"複数(ふくすう) 単数(たんすう) 複数の 
\\	の複数形は何ですか
\\	塗り絵(ぬりえ)
\\	電卓(でんたく) 計算機(けいさんき)
\\	伝統的な見方 (でんとうてきなみかた)
\\	こっそり 泥棒はこっそり家に入った
\\	無理(な)(むり) 無理な要求
\\	編み物(あみもの) 彼女は熱心に編み物をしていた
\\	発生する(はっせい) 起こる(おこります) 地震は時を選ばず発生する 事故が発生する
\\	失恋(しつれん) あの男は失恋している
\\	世代(せだい) 世代差(せだいさ) その習慣は世代から世代へと伝えられた
\\	四方八方(しほうはっぽう) うわさは四方八方に広がった
\\	なるほど
\\	記者会見(きしゃかいけん) 会見(かいけん) 
\\	ポーランド共和国(きょうわこく)
\\	トースト
\\	(悪くない)	悪くない ー 悪くありません
\\	健康にいい(けんこう) 健康的(けんこうてき)
\\	電源を切る(でんげん) 携帯電話の電源を切るべきである
\\	野山(のやま) 春が来ると野山の全てが生き生きとする
\\	恋心(こいごころ) 彼女は恋心を知るようになる
\\	掃除機(そうじき) この掃除機は非常にうるさい音がする
\\	三次元の(さんじげん) 三次元映画
\\	お疲れ様です(おつかれさま) お疲れ(略)
\\	用紙(ようし) この用紙に書き込むだけです 申し込み用紙
\\	~た 途端に(とたんに) 青信号が出たとたんに動き出す
\\	体育(たいいく) 体育の先生
\\	あばずれ(女)(おんな)
\\	オムレツ
\\	地中海(ちちゅうかい) 彼は休暇で地中海へ行く
\\	~に 大賛成する (だいさんせい)
\\	主人公(しゅじんこう) 有名な力士は落語の主人公になった
\\	情熱的(じょうねつてき) 星の下で情熱的なセックスをする
\\	エトセトラ
\\	ちっちゃ(な) ちっぽけ(な)
\\	曲(きょく) 覚えやすい曲 歌を1曲歌う
\\	修繕する(しゅうぜん) 修理する(しゅうり) 治す(なおします) 私は彼に靴を修繕するように頼んだ
\\	似合う(にあいます) このドレスは君によく似合う あなたはあの赤いセーターが似合う
\\	彗星(すいせい) 水星(すいせい)
\\	水星は小天体(しょうてんたい)
\\	セイウチ
\\	式典(しきてん) 式(しき) 式典は彼の話が始まった
\\	一面(いちめん) 
\\	空一面に星がまたたいていた
\\	有り難い (ありがたい) 本当に有り難いと思っています
\\	死亡(しぼう)
\\	死(し) 脂肪(しぼう)
\\	彼は母親が死亡したという電報を受け取った
\\	群集(ぐんしゅう) 人出(ひとで) 人手(ひとで)
\\	群(むれ)
\\	夥しい人手(おびただしい)
\\	射手(いて)
\\	ジーンズ ジーンズで行ってもいいかな
\\	経済学(けいざいがく)
\\	対称(たいしょう) 対象(たいしょう)
\\	対照的(な)
\\	柱(はしら) 彼は我が社の柱だ これらの柱が舞台を支えている
\\	天文台(てんもんだい) プラネタリウム
\\	贈呈する(ぞうてい) 彼女は幼稚園にピアノを贈呈した
\\	前途有望の(ぜんとゆうぼう) 前途洋洋の(ぜんとようよう) 前途有望の青年だ 有望(ゆうぼう)
\\	シャンプー
\\	キャンプ 私達は川にキャンプに行った
\\	神話学(しんわがく) 神話(しんわ)
\\	年輪(ねんりん) 木の年輪
\\	隠れ家(かくれが) その洞窟が私たちの隠れ家になった
\\	えんどう豆(えんどうまめ)
\\	台無し(だいなし)にする 壊す(こわします) 雨で桜が台無しになった あの看板のおかげで景色が台無した
\\	はげたか
\\	きちんと ちゃんと 
\\	きちんと座りなさい 彼の部屋はいつもきちんとしている
\\	救急箱(きゅうきゅうばこ)
\\	紀元前(きげんぜん) 紀元前一世紀(きげんぜんいっせいき)
\\	数字(すうじ) 数(かず) ローマ数字 2桁(けた)の数字
\\	吹き矢(ふきや) 吹き矢を吹く
\\	外交官(がいこうかん) 私は日本の外交官だ
\\	遅い(おそい)
\\	暗算する(あんざん) 彼は暗算するほど頭がよくない
\\	夫(おっと) (ご)主人(しゅじん) 囚人(しゅうじん)
\\	中止する(ちゅうし) キャンセルする にわか雨で試合は中止になった
\\	洋食(ようしょく) 和食(わしょく)
\\	水銀(すいぎん) 水銀温度計 (すいぎんおんどけい)
\\	初恋(はつこい)
\\	まとめる(まとめます) データをまとめる 月曜日までにこのレポートをまとめておいてください
\\	臭い(くさい) 君は息(いき)が臭い
\\	出版する(しゅっぱん) 発行する(はっこう) その会社は新しい雑誌を出版した
\\	象徴する(しょうちょう) 象徴的 赤いばらは愛を象徴している
\\	(書) 
\\	万人(ばんにん)(書) みんな 番人(ばんにん)
\\	妊婦(にんぷ)
\\	回転椅子(かいてんいす)
\\	大鋸屑(おがくず)
\\	マンション 高級マンション
\\	日(ひ) お日様(おひさま)
\\	太陽(たいよう)
\\	大洋(たいよう)
\\	に 精通する(せいつう) に 詳しい(くわしい)
\\	~を 通じる(つうじます) 彼は日本の宗教に精通している
\\	ビニ本(びにぼん) 弟の部屋を掃除していたら、ビニ本が枕の下に隠してあったのよ
\\	排気管(はいきかん) エキパイ
\\	一目で(ひとめで)
\\	信号(しんごう) あの信号を右へ曲がってください
\\	スリップ パンツ 
\\	瀬戸物(せともの)
\\	陶磁器(とうじき)
\\	瀬戸物屋(せとものや) 陶磁器店(とうじきてん)
\\	パイロット 操縦士(そうじゅうし) 副操縦士
\\	不滅(な・の)(ふめつ)
\\	不死(ふし) 
\\	不滅の魂 世の中に不滅なものはない
\\	2対0 (たい) 試合は4対3で、私達のチームが勝った 日本対アメリカのバレーボールの試合はあした行う
\\	下る(くだります) 
\\	上る(のぼります) 
\\	手当て(てあて) 手当てする その男はこの種の傷の手当てに慣れていた
\\	~つつ(も)(書) ~ながら 思いつつ 悪いことと知りつつ、友達にうそをついてしまった
\\	中東(ちゅうとう)
\\	タイトル 小説のタイトル
\\	道徳(どうとく) 不道徳(な) 道徳を守る 道徳家
\\	蝶結び(ちょうむすび) リボンを蝶結びにする
\\	金銭(きんせん) お金(おかね) 金銭的問題
\\	入院する(にゅういん) 母は入院している
\\	シャボン玉(だま) シャボン玉がこわれた シャボン玉を飛ばす[吹く]
\\	代謝(たいしゃ)(略) 退社(たいしゃ)
\\	新陳代謝(しんちんたいしゃ) 代謝する
\\	内戦(ないせん) 多くの国々で内戦が起きている
\\	誰かと 意見が合わない 私たちは意見がめったに合わない
\\	群(ぐん) 群れ(むれ) 群を抜く あの子は数学では群を抜いている
\\	帰り(かえり) 帰宅(きたく) 母は勤めの帰りに買い物をします
\\	~ごと 日曜日ごとに釣りに行く
\\	ろう者(ろうしゃ) 聾者は手話で会話ができる
\\	下がる(さがります)
\\	暴言(ぼうげん) 暴言を吐く
\\	法則(ほうそく) 法律(ほうりつ)
\\	あらゆるものは自然の法則に従う
\\	~が 苦手です(にがて) ~が 下手(な)(へた) 彼はスポーツが苦手だ
\\	ポスター
\\	尻取り(遊び)(しりとり) しりとり(遊び)をする
\\	~ないとだめ ~なくてはならない ~なければいけない 毎日学校に行かないとだめ 毎日学校に行かなくてはならない 毎日学校に行かなければいけない
\\	川獺(カワウソ)
\\	遠視(えんし) 私はかなりの遠視です
\\	荷物(にもつ)
\\	自宅(じたく) 自宅にいる 自宅の電話番号と職場の電話番号をお教えします
\\	封筒(ふうとう)
\\	税務署員(ぜいむしょいん) 税務署
\\	ひどい目に遭う(あいます) 雨に降られて、ひどい目に遭った
\\	着ける(つけます) 指輪を着ける
\\	鴎(かもめ) 鴎が低空に飛んでいる
\\	ウェートレス
\\	クラシックカー
\\	後で(あとで) それから
\\	年度(ねんど) 粘土(ねんど)
\\	2003年度 2004年度入学試験
\\	猿(さる)
\\	散らかす(ちらかします) 散らばる(ちらばります) 子供は部屋におもちゃを散らかした 新聞を散らかしたままにする 部屋は散らかしっぱなしだ
\\	よう くる する おぼえる まつ かう なおす	意向形(いこうけい) くる 
\\	こよう する 
\\	しよう おぼえる 
\\	おぼえよう まつ 
\\	まとう かう 
\\	かおう なおす 
\\	なおそう
\\	会社の仲間 (なかま) 会社の人
\\	感謝する(かんしゃ) 感謝の言葉もありません あなたの親切に感謝する
\\	基金(ききん) 病院に基金を寄付する 国際交流基金(こくさいこうりゅうききん)
\\	祇園祭り(ぎおんまつり)
\\	騎乗位(きじょうい)
\\	むく(むきます) みかんの皮(かわ)をむく
\\	でしゃばり でしゃばりの 君のような出しゃばりは嫌いだ
\\	接触する(せっしょく) 車は塀に接触した
\\	ためいきを吐く(つきます)
\\	不運(な)(ふうん) 運が悪い(うんがわるい) 不運に 不運にもブライアンは悪天候に遭った
\\	(警護) 知っている 
\\	ご存知です(ごぞんじ) ご存知のように、
\\	とは国際連合のことです(こくさいれんごう)
\\	スリッパ スリッパを履く(はきます)
\\	本屋(ほにゃ)
\\	書店(しょてん)
\\	舌足らず(したたらず) その子供は舌足らずに話す/しゃべる 舌足らずなしゃべり方
\\	一ヶ月(いっかげつ) 一月(ひとつき)
\\	案内人(あんないじん) 案内書(あんないしょ) 案内所(あんないじょ) 道を案内する
\\	応募(おうぼ) 申し込む(もうしこむ) 私は仕事に応募する 応募して兵士になった
\\	様子(ようす) 
\\	母は怒っている様子だった
\\	人文科学(じんぶんかがく) 哲学研究は人文科学のひとつだ 理学(りがく)
\\	子供が出来る 結婚の後で、すぐ子供が出来ました
\\	骨盤(こつばん)
\\	学校をサボる(さぼります)
\\	ミクロ経済学(けいざいがく) マクロ経済学
\\	前もって(まえもって) 予め(あらかじめ) あなたは前もって予約をしなきゃ
\\	シロナガスクジラ (白長須鯨)
\\	押す(おします)
\\	比較的に(ひかくてき) コンピューターは比較的最近の発明品だ
\\	鼻をほじる(ほじります)
\\	白衣(はくい) 白衣を着ている 白衣の天使
\\	身長(しんちょう) 背丈(せたけ)
\\	身長は何センチですか。
\\	奇妙(な)(きみょう) 珍しい(めずらしい)
\\	奇妙な経験 奇妙なことに
\\	活発(かっぱつ) スミス夫人は活発にボランティアの仕事に従事(じゅうじ)している
\\	乗り物(のりもの) 車やバスやトラックはすべて乗り物だ
\\	身近(な)(みぢか) 身近な例をあげよう
\\	嵐(あらし)
\\	弦(げん) 弦楽器(げんがっき)
\\	おかけください。
\\	ズボン
\\	ネックレス 首飾り(くびかざり) 母は真珠(しんじゅ)のネックレスをくれた
\\	阪神地区(はんしんちく)
\\	共感(きょうかん)
\\	同情する(どうじょう)
\\	彼の言葉は多数の聴衆の共感を呼んだ
\\	沖縄 沖縄は九州の南にある
\\	キャリアウーマン 
\\	預かる(あずかります) この旅行かばんを預かってください 昼間子供を5人預かっている
\\	チョンガー 
\\	(朝鮮語)
\\	独身(どくしん)
\\	指名する(しめい) 誰かを議長(ぎちょう)に指名した
\\	大座(おうざ) 王座に就く(つく)
\\	歳/才(さい) 1歳 いっさい 八歳(はっさい)
\\	物質主義(ぶっしつしゅぎ) 物質主義者
\\	羊毛(ようもう) ウール
\\	発覚する(はっかく) 暴露する(ばくろ)
\\	スキャンダルが発覚し
\\	羽布団(はねぶとん)
\\	どれぐらい
\\	地帯(ちたい) 安全地帯 砂漠地帯
\\	棺(ひつぎ)・(かん)
\\	額(ひたい) お凸(おでこ)
\\	小雨(こさめ) にわか雨(にわかあめ)
\\	平和的(な)(へいわてき) 平和(へいわ)
\\	泣き言(なきごと) 泣き言を言っても始まらない
\\	サンプル 
\\	見本(みほん) 
\\	のサンプル
\\	肥大名土地(ひだいなとち) 農業には肥大な土地が不可欠だ
\\	かな 
\\	かしら 
\\	「彼の話は本当かな」 「本当ではないでしょう」
\\	(お)菓子(かし) 甘いもの(あまいもの)
\\	じゃがいもを下ろす(おろす) お金を下ろす
\\	今度(こんど) 今度は女の人に聞きました
\\	統治する(とうち)
\\	支配する(しはい) 
\\	スペインは昔フィリピン諸島統治していた
\\	図形(ずけい)
\\	図(ず) 
\\	書籍(しょせき) 本(ほん) ここのところ書籍の価格が上がっている
\\	胎児(たいじ) 胎児の権利
\\	台(だい)
\\	湿気(しっけ) この電気器具は湿気を嫌う
\\	意見交換(いけんこうかん)
\\	まさか まさか一等になるとは思わなかった まさかの時(とき)
\\	処女幕(しょじょまく) 処女を失う
\\	アラブ首長国連邦 (アラブしゅちょうこくれんぽう) 首長国 (しゅちょうこく)
\\	~何より 一番 人間の命は何より大切です
\\	足りる(たります) 彼女の経験は足りない
\\	景色(けしき)
\\	風景(ふうけい) 
\\	田舎の景色
\\	拝む(おがみます) 金を貸してくれと拝まれた 助けを求めて神を拝んだ
\\	痛み止め(いたみどめ) 鎮痛剤(ちんつうざい)
\\	典型(的)(てんけい(てき) 女性の典型 典型的な英国紳士
\\	洗脳(せんのう) 彼は洗脳されて秘密を漏らしてしまった(もらす)
\\	こはく こはくいる(の)
\\	外科学(げかがく) 外科医 外科医をする
\\	気が進まない(すすまない) 知らない人ばかりの所に行くのは気が進まないです
\\	織姫(おりひめ)
\\	意見(いけん) 見解(けんかい) 意見を述べる(のべます)/伝える(つたえます)
\\	タバコをやめる 禁煙する(きんえん) タバコをやめてから、体重が急に増えた
\\	宣教師(せんきょうし) 宣教(せんきょう)
\\	言うは易く行うは難し (いうはやすくおこなうはかたし)
\\	ずる賢い(ずるがしこい) ずるい あいつはずるい事をする
\\	いらっしゃい・ませ。
\\	あら!(女) おや! 
\\	あら、よく聞こえる
\\	沖縄(おきなわ) 沖縄は本州の南にある
\\	酸(さん)
\\	社交生活(しゃこうせいかつ)
\\	結婚指輪(けっこんゆびわ) 指輪(ゆびわ)
\\	湖(みずうみ) 琵琶湖(びわこ)
\\	安全器(あんぜんき)
\\	妨げる(さまたげる)
\\	邪魔する(じゃま) 進歩を妨げる 悪天候に妨げられてそこに行けなかった
\\	ロケット 月ロケット(つき)
\\	ささやき
\\	神秘(しんぴ) この神秘を解決しましたものはいますか
\\	特急(とっきゅう) 特急で 仕事を特急で仕上げる
\\	~ように 春子さんは男のような声をしている
\\	リラックスする のんびりする 
\\	日曜日はうちでのんびりします
\\	急行(きゅうこう) 休校(きゅうこう) 
\\	エキスパート 大家(たいか) 退化(たいか) 
\\	大家(おおや)
\\	大屋(おおや) 
\\	~に 興味がある(きょうみ) ~に 関心がある(かんしん) 彼は科学には全然興味を示さない 私は化学にまったく興味がない
\\	創造物(そうぞうぶつ)
\\	悪意のない嘘 (あくいのないうそ)
\\	国王(こくおう) 王様(おうさま)
\\	でも
\\	堀(ほり) 堀を掘る 城に堀をめぐらす
\\	~に関する 文献(ぶんけん) 専門の図書館が芸術に関する文献を収集している 日本に関する文献
\\	運輸(うんゆ)
\\	運搬(うんぱん) 運輸省(うんゆしょう) 運輸機関(うんゆきかい)
\\	いったい 一体なぜ昨日来なかったのか 一体どうしてその事を知っていたの
\\	未だ未だ (まだまだ) 私の日本語は未だ未だです
\\	コアラ
\\	人類学(じんるいがく) 人類学者
\\	ジプシー 彼女は慈善の気持ちからそのジプシーに衣服を与えた(あたえた)
\\	奥さん(おくさん)
\\	上流階級 (じょうりゅうかいきゅう) エリート
\\	少数(しょうすう) 小数第2位 分数(ぶんすう)
\\	口蓋垂(こうがいすい) のどちんこ 
\\	口蓋(こうがい)
\\	屈む (かがみます)
\\	~を 包む(つつみます) 包む必要はありません
\\	次元(じげん) 三次元の 蜘蛛はいつも自分たちの巣を三次元に織る(おります)
\\	憲法(けんぽう)
\\	ふっくら(とした/として) ふっくらとした枕 ほおがふっくらとしている ご飯がふっくらと炊き上がった
\\	発明する(はつめい) 発明品
\\	カラット 14カラットの金
\\	濁る(にごります) 濁った川 「が」は濁った(にごった)音だ
\\	大動脈(だいどうみゃく) 動脈(どうみゃく)
\\	飲み水(のみみず)
\\	名人(めいじん)
\\	大家(たいか)
\\	エキスパート 料理にかけては、彼女は名人だ
\\	調べる(しらべます) チェックする
\\	肺炎(はいえん) 肺炎が治るのに長い時間かかった
\\	~に 面している(おも) 私たちの家は海岸に面している
\\	闘争する(とうそう)
\\	戦い(たたかい) 蛇とマングースの激しい闘争 自由を求めて闘争する
\\	医者(いしゃ) 医師(いし)
\\	参議院(さんぎいん)
\\	花婿(はなむこ) 新郎(しんろう)
\\	いろいろ調べて、また電話をする
\\	マヨネーズ
\\	アイスホッケー
\\	雨垂れ(あまだれ) 軒から落ちる雨だれ
\\	十人十色(じゅうにんといろ)
\\	まぶい
\\	錨(いかり) 怒り(いかり) 
\\	錨を降ろす
\\	防腐剤(ぼうふざい) 添加物(てんかぶつ)
\\	浮かぶ(うかびます) ビニール袋が川面に浮かんでいた 空に雲が浮かんでいた いい考えが頭に浮かんだ
\\	布団(ふとん)
\\	必要(な)(ひつよう)
\\	傷(きず) 負傷(ふしょう)
\\	怪我(けが)
\\	傷は未だに(いまだに)治らない
\\	遅い(おそい)
\\	運ぶ(はこびます) このエレベーターは10人を運ぶことが出来る
\\	統率する(とうそつ)
\\	指導する(しどう)
\\	海軍を統率する 彼女は統率力がある
\\	投石器(とうせきき) 投石する(とうせき)
\\	狙う(ねらいます) 
\\	目指す(めざします)
\\	鳥をねらって撃った(うつ)
\\	保育園(ほいくえん) 茂美ちゃんを保育園に迎えに行ってもらえる?
\\	狩人(かりゅうど) 男は狩人で、女はその獲物である
\\	スーツケース
\\	水平線(すいへいせん) 地平線(ちへいせん) 月が水平線上にでた 太陽が)地平線上に上る 
\\	地平線(ちへいせん) 水平線(すいへいせん) 月が水平線上にでた 太陽が地平線上に上る 
\\	水平線(すいへいせん) 地平線(ちへいせん) 月が水平線上にでた 太陽が地平線上に上る
\\	日焼け止め(ひやけどめ) 日焼けオイル 日焼け(ひやけ)
\\	歯車(はぐるま) 歯車の歯
\\	花嫁(はなよめ) 新婦(しんぷ) 神父(しんぷ)
\\	甲虫 (かぶとむし)
\\	心配する(しんぱい) 心配をさせる 心配事(しんぱいごと)
\\	貿易(ぼうえき) 自由貿易 最近、日本とアジアの国々の貿易が盛んになった
\\	壁紙(かべがみ) デスクトップの壁紙をペットの写真にした
\\	飼育(しいく) 飼う(かいます)
\\	牛を飼育する 牧畜する(ぼくちく)
\\	移植(いしょく) 臓器(ぞうき)の移植 外科医(げかい)は私を説得して、臓器の移植手術を受けることに同意させた
\\	パジャマ 寝間着(ねまき)
\\	低気圧(ていきあつ) 高気圧(こうきあつ)
\\	バルト海(かい)
\\	立ち話する(たちばなし) 店の前で女たちが立ち話をしていた
\\	髪(の毛(かみのけ) 紙(かみ)
\\	神(かみ)
\\	立つ(たちます)
\\	両生類(りょうせいるい) 両生動物
\\	引きこもり (ひきこもり) 引きこもる (ひきこもります)
\\	商業(しょうぎょう) 商売(しょうばい)
\\	貿易(ぼうえき)
\\	この都市は国の商業の中心だ 商業化(しょうぎょうか)
\\	掛かる(かかります) どれぐらい掛かりますか。
\\	聞こえる(きこえます) 音楽が聞こえる
\\	美しい(うつくしい) 綺麗(な)(きれい) 常用外
\\	コンドーム
\\	抽選(ちゅうせん)
\\	宝くじ(たからくじ)
\\	抽選する 抽選で決める 抽選に当たる
\\	富(とみ) 教育の目標は、富や地位ではなく人柄(ひとがら)の形成(けいせい)にある
\\	文語(ぶんご) この語は文語としてしか用いられない
\\	整形手術(せいけいしゅじゅつ) 美容整形(びようせいけい)
\\	潜水艦(せんすいかん) 私はこれまで黄色い潜水艦を見たことがない
\\	徒歩で行く(とほ) 歩いていく そこまで徒歩で行った
\\	リハビリ リハビリにはいる
\\	キューバ
\\	例外(れいがい) 例外なく は例外として
\\	段ボール箱(だんボールばこ)
\\	歩道橋(ほどうきょう)
\\	出生証明書 (しゅっしょうしょうめいしょ)
\\	伝聞(でんぶん) うわさ 
\\	デマ 
\\	伝聞する 伝聞するところによれば
\\	婦人(ふじん) その婦人の葬式は地元の教会で行われた 婦人解放運動
\\	根気(こんき) 忍耐(にんたい) 我慢(がまん) 持久(じきゅう) 根気よい (よく) セーターを編むには根気がいる
\\	合金(ごうきん) 金と銅を合金する
\\	飢え(うえ) 上(うえ)
\\	アフリカでは飢えで死ぬ子供が多い
\\	いななく 馬がいなないた
\\	可能(な)(かのう) 可能性(かのうせい)
\\	小物(こもの) 
\\	彼は政治家仲間では小物とされている
\\	席(せき) 咳き(せき) 
\\	バスは止まってから、席を立ってください
\\	中央同盟国(ちゅうおうどうめい国) 連合国(れんごうこく)
\\	叔父・伯父(おじ)
\\	名声(めいせい) 名声を得る 世界的な名声のある科学者
\\	宇宙飛行士 (うちゅうひこうし)
\\	赤字(あかじ) 
\\	会社はまた赤字だ
\\	~たことがない 私は横浜へ行ったことがない
\\	火傷する(やけど) 火傷(やけど) 私は手に火傷をした
\\	錠剤(じょうざい) これらの錠剤を飲めば腹痛(ふくつう)は治るでしょう
\\	草(くさ)
\\	恥(はじ) 恥を知れ あなたはカンニングをすることを恥と思うことである
\\	月刊誌(げっかんし)
\\	次に(つぎに)
\\	染み一つない(しみひとつない) 染み(しみ)
\\	彼は苦しい生活に耐えた この騒音には耐えられない	耐える(たえます) 我慢する(がまん) 彼は苦しい生活に耐えた この騒音には耐えられない
\\	先頭(せんとう)に立つ 戦闘(せんとう)
\\	我々の車は速く、じきにほかの車の先頭に立った 彼は走者(そうしゃ)の先頭に立っている
\\	~誰かの ~何を 手伝う(てつだう) 手伝ってあげる 仕事を手伝う 子供の宿題を手伝う
\\	怠ける(なまけます) 怠け者(なまけもの) 怠惰(たいだ)
\\	ぴったり このスーツは私にぴったりだ 彼はタバコをぴったりやめた
\\	弾丸(だんがん) 弾(たま) 球(たま)
\\	その犬は弾丸のように去った 弾丸列車
\\	キャンドルスティック ろうそく立て(たて)
\\	自主的(な)(じしゅてき) 独立(どくりつ)
\\	自主の映画 自主独立の精神がある 自主性がある
\\	突風(とっぷう) 突風にあおられる
\\	オレンジ色(の) 橙色(の)(だいだいいろ) 橙(だいだい)
\\	常用外
\\	自然に(しぜん) 火が自然に消えた
\\	編集する(へんしゅう) 新聞記事を編集する
\\	収入が多い/少ない(しゅうにゅう) 私は一ヶ月の収入は100円だ
\\	国字(こくじ)
\\	身分証明書 (みぶんしょうめいしょ)
\\	脾臓(ひぞう)
\\	ばらばら歩く 別々に(べつべつに) ばらばらで売る ばらばらの髪 時計をばらばらにす
\\	鎖骨(さこつ)
\\	穀物(こくもつ) 麦(むぎ)
\\	~に 脅す(おどします) 彼は私に刃物で脅した 殺すぞと脅す
\\	忘れな草 (わすれなぐさ)
\\	ホテル
\\	暑い盛り (あついさかり) 午後の暑い盛りに2時間歩いた
\\	鶏肉(けいにく) 鳥肉(とりにく)
\\	料金(りょうきん) 代金(だいきん)
\\	~料(りょう) 
\\	入場料金 電気料金を払う
\\	皮むき(かわむき)
\\	~に 心配を掛ける 親に心配を掛ける
\\	ちっぽけ(な) ちっちゃ(な) ちっぽけな村
\\	気温(きおん) 最高気温は20度です
\\	多分(たぶん)
\\	暮らす(くらします) 
\\	生きる(いきます) 
\\	生まれた時からずっとここで暮らしている
\\	小遣い(こづかい) あなたの小遣いは一ヶ月くりですか
\\	茶番(ちゃばん)
\\	(ので) 
\\	ので から 正月なので、 暇なので、
\\	病む(やみます) 病気です(びょうき)
\\	団体(だんたい) サークル 僕は文学の団体に所属している
\\	恋愛結婚(れんあいけっこん) 恋愛(れんあい)
\\	献立(こんだて) メニュー それほど多くの人々のための料理の献立を考えるのは難しい 夕食の献立は何ですか 献立表(こんだてひょう)
\\	平方キロメートル(へいほう)
\\	週刊誌(しゅうかんし)
\\	一家(いっか) 家族(かぞく) 家庭(かてい) 一家の主人 父親の死後長女が一家を支えた
\\	双子(ふたご) 三つ子(みつご)
\\	大家(さん) (おおや)
\\	不足する(ふそく) ~に かかる われわれは石油が不足している 水不足(みずぶそく)
\\	擬態語 
\\	ぽつぽつ 雨がぽつぽつ降りだした
\\	子供の日(こどものひ)
\\	文末(ぶんまつ)
\\	どちら(どっち) 犬と猫とどちらが好きですか。
\\	サイドブレーキ
\\	堅表紙の(かたびょうし) ハードカバーの 堅表紙の本
\\	仕事中(しごとちゅう) 彼は仕事中です
\\	八月の初めごろ
\\	香水(こうすい)
\\	距離(きょり) 間隔(かんかく)
\\	タクシーの料金は距離によって計算される
\\	表紙(ひょうし) プラスチックの表紙
\\	交番(こうばん) 警察署(けいさつしょ)
\\	去勢(きょせい) 去勢する 去勢した動物
\\	区切る(くぎります) 言葉を区切って話した
\\	~を 確立する(かくりつ)
\\	制定する(せいてい)
\\	ニュートンは重力の法則を確立した 名声を確立する
\\	売春宿(ばいしゅんやど)
\\	正しい(ただしい) この文は文法的に正しい
\\	カウンター 
\\	窓口(まどぐち)
\\	年をとると共に (としをとるとともに) 年をとると共に、体力が衰える(おとろえます)
\\	効果(こうか) 硬貨(こうか)
\\	連合国(れんごうこく) 中央同盟国(ちゅうおうどうめい国)
\\	ヒント 暗示(あんじ) ヒントを与える 暗示する
\\	後ろ姿(うしろすがた) 彼は後ろ姿が私のおじさんに似ている
\\	人民(じんみん) 国民(こくみん)
\\	人民の権利
\\	後 
\\	後で 
\\	祭りの後で、ごみがいっぱいだった 祖父はみんなとお茶を飲んだ後、ずっと部屋で本を読んでいます
\\	樅の木(もみのき)
\\	宿題(しゅくだい)
\\	そろそろ失礼します
\\	表す(あらわします) 彼女の微笑みが喜びを表していた
\\	富士山(ふじさん)
\\	十年ぐらい 十年ぐらい前はもっと静かな町だった
\\	~きり 二人きり 一人きり
\\	この辞書はどうしましょうか
\\	風俗(ふうぞく)
\\	習慣(しゅうかん) 
\\	土地の風俗
\\	サングラス
\\	(警護) 
\\	おる (おります)
\\	ズボン ズボンを履く(はきます)
\\	王国(おうこく)
\\	~を 閉める(しめます) ~が 閉まる(しまります)
\\	化学療法(かがくりょうほう)
\\	沸かす(わかします) やかんで湯を沸かす 風呂を沸かす
\\	がらがらへび
\\	所(ところ) 場所(ばしょ)
\\	長女(ちょうじょ)
\\	牧畜(ぼくちく) 牧畜する
\\	羽(わ)
\\	間(あいだ)
\\	ぼやける(ぼやけます) その時の記憶はすっかりぼやけている かすみで遠くの山がぼやけていた
\\	温泉(おんせん)
\\	(警護) 
\\	伺う (うかがいます) 失礼ですが、お名前を伺ってもよろしいですか
\\	ちっとも 
\\	ない 少しも 
\\	ない 
\\	全然 
\\	ない 
\\	まったく 
\\	ない ちっとも寒くない ちっともかまわない
\\	人類(じんるい) 癌は人類の大敵です
\\	ぬるい
\\	タクシー タクシーを呼ぶ
\\	あの髪が暗い色の女の人
\\	リール 
\\	釣り糸をリールに巻き込む(まきこきます) 釣り糸をリールから繰り出す(くりだします)
\\	尽きる(つきます) 私はお金が尽きるまでここにいるつもりです
\\	いたずらする お母さんがいたずらした子供を叱っています
\\	スピーチ 演説(えんぜつ) スピーチの下書きを書いてる スピーチする
\\	喫茶店(きっさてん)
\\	(警護) します 
\\	なさいます いたします 
\\	喜んでそういたしましょう
\\	東部(とうぶ) 西部(せいぶ) 北部(ほくぶ) 南部(なんぶ) 彼は町を出て東の方へ行ったらしい
\\	消化する(しょうか) 消火する(しょうか)
\\	チーズは簡単に消化しない
\\	二番目(にばんめ) 二番目の道
\\	唾液腺(だえきせん)
\\	酔っ払い(よっぱらい)
\\	課外(かがい) 課外活動(かがいかつどう) 課外授業
\\	長生き(ながいき) ソウルの女の人が一番長生きします
\\	楽観(らっかん) 楽観的 未来のことをどうしてそんなに楽観できるんですか
\\	毛深い(けぶかい) 毛深い眉毛が彼の顔の特徴だった
\\	賢明(けんめい) 知恵(ちえ)
\\	すぐに謝ったほうが賢明だと思う
\\	団地(だんち)
\\	~を 達成する(たっせい) ~を 遂げる(とげます) 目的を達成する そのような目標はとても達成できない
\\	もうすぐ 復活祭はもうすぐだ
\\	早朝(そうちょう) 朝方(あさがた) 早朝、ドアをどんどんたたく音が聞こえた
\\	郵便受け(ゆうびんうけ) 郵便箱(ゆうびんばこ) ポスト
\\	婦人科医(ふじんかい)
\\	様子(ようす) 
\\	状態(じょうたい)
\\	彼の家庭の様子は今どんな具合ですか
\\	短文(たんぶん)
\\	薄い(うすい) 薄いお茶
\\	核家族(かくかぞく)
\\	かかと
\\	古本(ふるほん)
\\	コンタクト(レンズ) コンタクトをする
\\	丼(どんぶり)
\\	粘々する(ねばねば) 粘る(ねばります) 
\\	クモはねばねばする/粘る糸を出す
\\	鉱物(こうぶつ) 好物(こうぶつ)
\\	オーストラリアは鉱物に富みます(とむ)
\\	親父(おやじ) お袋(おふくろ) 頑固おやじ(がんこ)
\\	プレイスメントテスト
\\	孤児院(こじいん)
\\	服飾センス(ふくしょく)
\\	同級生(どうきゅうせい)z 
\\	学友(がくゆう) クラスの仲間(なかま) 同級生に伝える
\\	プラン 週末のプランは?
\\	学年(がくねん) 新学年は日本では4月に始まる
\\	息をする(いき) 
\\	呼吸する(こきゅう)
\\	息をのむ 息を止める 息が切れる
\\	顔(かお)
\\	~省(しょう) 運輸省(うんゆしょう)
\\	うなり/唸り声(うなりごえ) 吠える(ほえます) 犬がうなる(うなります)
\\	同性愛者(どうせいあいしゃ) 異性愛者(いせいあいしゃ) 両性愛者(りょうせいあいしゃ)
\\	道を通った人 歩行者(ほこうしゃ)
\\	金額(きんがく) 金額を押します
\\	デート
\\	満員(まんいん) 込んでいる(こむ) 寿司詰め(すしづめ)
\\	推薦状(すいせんじょう) 先生は私ために推薦状を書いた
\\	灯篭(とうろう)
\\	雪洞(ぼんぼり)
\\	茶室, 
\\	灯籠に火をともす
\\	車を止める(とめる) 駐車する(ちゅうしゃ) ここに車を止めろ
\\	プラグイン
\\	ガムを噛む(かみます)
\\	研究室(けんきゅうしつ)
\\	足の指(あしのゆび)
\\	雪を丸める(まるめます)
\\	石(いし) 医師(いし)
\\	意思(いし)
\\	辛い(つらい) 辛い(からい)
\\	辛い思い つらい仕事 そのことは思い出すだけでもつらい
\\	賃金(ちんぎん)
\\	原因(げんいん) 今、事故の原因を調べている 火事の原因はタバコの火でした
\\	座剤(ざざい) 座薬(ざやく)
\\	肝細胞(かんさいぼう)
\\	筆順(ひつじゅん) 書き順(かきじゅん)
\\	追加(ついか) ビールを二本追加して頼みました
\\	未婚の(みこん) 既婚の(きこん) 未婚の男性
\\	低下する(ていか) 下がる(さがります) 夜は気温が零下に低下する
\\	言わぬが花
\\	牢屋(ろうや)
\\	ソース チリ・ソース
\\	外で(そとで) タバコは外で吸ってください
\\	~あたり ~ぐらい 約~ 去年あたり ポーランドあたりで
\\	耳が遠くなる
\\	そのことは思い出すだけ… そのことは思い出すだけでもつらい
\\	過半数(かはんすう) 大半(たいはん)
\\	大統領は投票の過半数を持って選ばれる ぎりぎりの過半数
\\	せめて じゃ、せめてテーブルの準備でもしましょう
\\	倹約家(けんやくか) 節約家(せつやくか)
\\	不死(ふし) 
\\	不滅(な・の)(ふめつ)
\\	不死の人
\\	他(ほか)(に/の) その他/外(ほか)
\\	快楽(かいらく) 楽しみ(たのしみ) 喜び(よろこび) 快楽を求めて(もとめます)血を流せ(ながす)
\\	凄い(すごい) 常用外
\\	発祥(はっしょう) 出身(しゅっしん)
\\	古代中国に発祥を持つ
\\	クリスマスツリー を飾る(かざります)
\\	サミット サミット参加者
\\	尊重(そんちょう)
\\	尊敬(そんけい)
\\	人権の尊重 彼はいつも奥さんの意見を尊重している
\\	近付き(ちかづき) 知り合い(しりあい) お父様とは5年前からお近付きになりました
\\	学長(がくちょう)
\\	前途(ぜんと) 彼女は前途を悲観している
\\	うちの鍵が掛かっている(かぎ)
\\	そして
\\	自己尊重(じこそんちょう)
\\	流行(りゅうこう) ミニスカートがまた流行している
\\	真ん前(まんまえ) 彼は私の真ん前に立っていた
\\	すごいいい気分だ。
\\	恩(おん) 恩恵(おんけい)
\\	恩を返す 人に恩を受けている ご恩は一生忘れません
\\	望月(もちづき) 三日月(みかづき)- 
\\	みっかづき!!!
\\	外交(がいこう) 国際問題は戦争ではなく外交で解決されなければならない 外交官(がいこうかん)
\\	声門 (せいもん)  正門(せいもん) 
\\	声門閉鎖音(せいもんへいさおん)
\\	工事(こうじ) その工事の完成には約 
\\	年を要した 工事中(こじちゅう)
\\	吹く(ふきます) 風は南へ吹く
\\	特技(とくぎ) 才能(さいのう) 
\\	人の名前を覚えることが彼女の特技だ
\\	アナウンサー アナウンサーは早口ができる
\\	荒地ねずみ 荒地(あれち)
\\	歯医者(はいしゃ) 廃車 (はいしゃ)
\\	給料(きゅうりょう) 月給(げっきゅう) 自給(じきゅう)
\\	お坊さん(おぼうさん) 僧(そう)
\\	豆(まめ)
\\	生かす(いかす) 
\\	誰かを生かす(いかす) 釣った魚を生かしておく
\\	南極圏(なんきょくけん) 北極圏(ほっきょくけん)
\\	雄鶏(おんどり) (鳥の雄) 雌鳥(めんどり) (鳥の雌) 鶏(にわとり)
\\	活気(かっき)
\\	活発(かっぱつ)
\\	鮮やか(な)(あざやか)
\\	生々しい(なまなましい)
\\	生き生きとする(いきいき)
\\	私は活気がある人です
\\	語源(ごげん) これらの語は同じ語源から出る
\\	幸せ(な)(しあわせ)
\\	幸福(な)(こうふく)
\\	幸運(こううん)
\\	幸い(な)(さいわい)
\\	(副詞)
\\	のみ 蚤の市(のみのいち)
\\	治る(なおります) 風邪が治る
\\	おだいじに。
\\	存続(そんぞく)
\\	その地方ではその古い風俗がお存続している
\\	歯が悪い   歯が悪くなる
\\	差し当たり(さしあたり) 差し当たり、あれで十分でしょう
\\	技術(ぎじゅつ) 
\\	腕前(うでまえ)
\\	能力(のうりょく) 
\\	私達に必要なのは頭の技術であって、手の技術ではない 水泳は大変役に立つ技術である
\\	街灯(がいとう) 街灯の柱
\\	四方に(しほう) 四方八方に(しほうはっぽう) 町の中央から四方に道が延びている(のびる)
\\	下し金(おろしがね)
\\	騒がしい(さわがしい) やかましい 煩い(うるさい) こんな騒がしい部屋は話が続けられない
\\	忠実さ(ちゅうじつさ) 忠実(な)(ちゅうじつ)
\\	両手に花(りょうてにはな)
\\	ぼ 
\\	濁点(だくてん) 濁点を打つ  「が」は濁った(にごった)音だ
\\	ナイル川(がわ) アマゾン川(がわ)
\\	~対して 無力(むりょく) 人間は自然の前には無力だ
\\	荒れる(あれます) 海が荒れている この家はずいぶん荒れています
\\	格好(かっこう) 郭公(かっこう)
\\	格好のいい女の人
\\	ぐらい 二人ぐらい
\\	訪問する(ほうもん) 訪ねる(たずねます) フランス大統領は沖縄を訪問した
\\	子犬(こいぬ)
\\	頑固(な)(がんこ) 父はとても頑固なので、自分の誤りを認めることができない
\\	選ぶ(えらびます) 選択する(せんたく) 洗濯(せんたく) 
\\	記す(しるす) 書き記す(かきしるす) その日の出来事を日記に記す
\\	ラトビア リトアニア
\\	睡蓮 (すいれん) 蘭(らん)
\\	風土(ふうど) フード 
\\	気候(きこう) 
\\	日本の風土
\\	真っ黒(な)(まっくろ) ああ、トーストが真っ黒に焦げて(こげる)しまっているよ
\\	弓(ゆみ) 
\\	育児休暇(いくじきゅうか) 育児(いくじ) 彼女は育児に専念している
\\	キプロス
\\	明々後日(しあさって)
\\	正義(せいぎ) 公表(こうひょう)
\\	彼らは正義のために戦った
\\	エプロン エプロンのひもを結んでください(むすぶ)
\\	千金(せんきん) 彼の友情は千金に値する
\\	のり
\\	反吐(へど)を吐く 嘔吐(おうと)
\\	ゲロを吐く 
\\	床に反吐を吐いた 顔を見ただけ反吐が出る
\\	縫いぐるみ(ぬいぐるみ) テディベアは熊の縫いぐるみだ
\\	抱く(いだく) 
\\	抱く(だく) 恨みを抱く 反感を抱く
\\	邦楽(ほうがく) 法学(ほうがく)
\\	邦楽バンド
\\	ステンドグラス (語源:
\\	公平(こうへい) 
\\	正義(せいぎ)
\\	公平に見て彼の勝ちだ
\\	~に 専念する(せんねん) 彼はボランティア活動に専念した
\\	銃剣(じゅうけん) 銃剣を着ける[外す]
\\	雨雲(あまぐも)
\\	頬紅(ほおべに) 頬紅をつける
\\	密度(みつど) 人口密度(じんこうみつどう)
\\	~ほど ~ば ~ほど 子供は小言(こごと)を言うはほど反発(はんぱつ)するものだ
\\	珍しい(めずらしい)
\\	肥満(ひまん) 彼は肥満している
\\	徹夜で/して (てつや) 一晩中 (ひとばんじゅう) ゆうべは徹夜して小説を読んだ 彼らは徹夜でマージャンをした
\\	頑張って(がんばって) 頑張れ!
\\	親子(おやこ) 親子の関係
\\	異なる(ことなります) 他の(ほかの) 十五歳までに、彼女は異なる五つの国で暮らしてきた
\\	急ぐ(いそぎます)
\\	二十歳(はたち) 今年20歳になった
\\	履物(はきもの) 履物を履く
\\	回復する(かいふく) 治る(なおる) 君が早く回復することを望みます
\\	頭字語(とうじご)
\\	天皇の誕生日 (てんのうのたんじょうび) 12月23日
\\	滅びる(ほろびます) 滅亡する(めつぼう)
\\	没落する(ぼつらく)
\\	国が滅びた 西ローマ帝国は西暦 
\\	年に滅んだ 滅びようとしている その動物は日本でもう滅びてしまった
\\	家賃(やちん) 大家さんは家賃をあげたいそうだ
\\	特権(とっけん) 私達は図書館利用の特権を与えられた
\\	楽団(がくだん) バンド(ばんど) オーケストラの楽団員はみんな成功を喜んだ
\\	将来(しょうらい)
\\	未来(みらい)
\\	歩行者(ほこうしゃ)
\\	大ざっぱ(な)(おおざっぱ) 大ざっぱな計算で約100ドルかかる
\\	骸骨(がいこつ) 骸骨のような人
\\	~を 別にとる(べつ) ~以外に(いがい)
\\	利口な(りこう) 10歳の少年としては彼女は利口だ
\\	足袋(たび) 旅(たび)
\\	既に(すでに) もう 
\\	既に宿題を終えてしまった すでに述べたように
\\	平凡(な)(へいぼん) ありふれた
\\	人並み(な)(ひとなみ) 
\\	平均的(へいきんてき) 
\\	普通(な)(ふつう)
\\	平凡な小説 平凡だね
\\	よちよち歩く ヨチヨチ歩き始める
\\	蝶番(ちょうつがい) ドアのちょうつがいがはずれている
\\	中心(ちゅうしん) 中央(ちゅうおう) 
\\	彼の家は市の中心にある 東京は政治の中心だ
\\	微笑む(ほほえむ)
\\	卵を割る(わります)
\\	左右(さゆう) 
\\	左右をよく見て道を渡りなさい
\\	指導する(しどう) このチームはよいコーチに指導されている
\\	子供(こども) 子(こ)
\\	カーテン カーテンを引く/開く
\\	中身(なかみ) 彼はポケットから中身を出した
\\	どこか
\\	翻訳者(ほんやくしゃ)
\\	一生懸命(いっしょうけんめい)
\\	肉体(にくたい)
\\	身体(しんたい)
\\	体(からだ)
\\	人体(じんたい)
\\	肉体的(にくたいてき)
\\	明けましておめでとうございます 今年もよろしくおねがいします
\\	ありそう(な) 事故の最もありそうな原因 それはありそうなことだ
\\	コントラバス
\\	方言(ほうげん) ~弁(べん) 田舎の方言で話す
\\	年上(としうえ) その俳優は私より2歳年上だ
\\	探知機(たんちき)
\\	獄中で(ごくちゅうで) 
\\	刑務所で(けいむしょ)
\\	いつごろ いつごろから米の消費が減り始めたのだろう
\\	~に 限る(かぎります) 一番(いちばん) 夏は冷たいビールに限る 辞書はこれに限る
\\	物差し(ものさし) どのくらいの長いさか、物差しで計ってごらんなさい
\\	動詞の変化 日本語の動詞の変化を習った
\\	尿(にょう) 以前尿にアルブミンが出ました
\\	ペルー
\\	看護士(かんごし) 看護婦(かんごふ)
\\	ロボット
\\	パーキングメーター
\\	スキー場(じょう)
\\	アスピリン 頭痛薬 (ずつうやく) 頭が痛いんです。頭痛薬がありますか
\\	琴(こと)
\\	わめく わめくのを止めて
\\	只今(ただいま)
\\	添削する(てんさく) 直す(なおす) 私の作文を先生に添削してもらった
\\	操縦する(そうじゅう) 船を操縦する 飛行機を操縦する 機械を操縦する
\\	代金(だいきん) ~料(りょう)
\\	料金(りょうきん)
\\	車の代金として300万円払った
\\	鰭(ひれ)  ヒレ
\\	小川(おがわ)
\\	魚市場(うおいちば)
\\	性的不能 (せいてきふのう) インポテンス インポ 
\\	誰かを絞め殺す(しめころす)
\\	合計(ごうけい) 我は投票数を合計した
\\	肥えた土地(こえた)
\\	サルモネラ
\\	通学する(つうがく) 学校に通う(かよいます) 通勤する(つうきん)
\\	近所(きんじょ) 隣人(りんじん) 
\\	彼女は近所の全ての若者の心を引いた
\\	偏見(へんけん) ~に 偏見を持つ 彼の意見には偏見がない
\\	応じる(おうじります) 
\\	答える(こたえます) 回答する(かいとう) ご質問に応じて
\\	目茶苦茶(めちゃくちゃ) 大雨で盆栽が目茶苦茶になった
\\	噂話する(うわさばなし)
\\	勝利(しょうり) 優勝(ゆうしょう) 
\\	私は彼の勝利を確信している
\\	物理学(ぶつりがく)
\\	フォーク フォークがテーブルから落ちた
\\	怒鳴る(どなります) 叫ぶ(さけびます) わめく 警官に怒鳴られた
\\	あだ名(あだな)
\\	呼び名(よびな)
\\	愛称(あいしょう) 
\\	補聴器(ほちょうき) 補聴器をつければよく聞こえます
\\	ひんやりした 涼しい(すずしい) ひんやりする ひんやりした風 木陰はひんやりと心地よい
\\	特許(権) (とっきょ(けん)) 会社へ特許権を売る
\\	耳栓(みみせん)
\\	マスコミ 二十一世紀のマスコミ
\\	肉食(にくしょく) 肉食動物
\\	豹(ひょう) 表(ひょう) 
\\	雹(ひょう) 
\\	アメリカ豹
\\	(銀行口座から) 引き落とす(ひきおとします) 電気代は毎月私の口座から引き落とされる
\\	定例(ていれい) 定例の 定例(集)会 定例によって集会は毎月がある
\\	綿毛(わたげ) タンポポの綿毛(わたげ) 桃の綿毛
\\	しかし
\\	だれかに魔法をかける
\\	塗る(ぬります) 蚊に刺されたところに軟こうを塗る 壁を白く塗る
\\	活動(かつどう) この町の人は皆いつも忙しく活動しているようだ
\\	レバノン
\\	黒苺(くろいちご) クロイチゴ摘みに行く(つまむ)
\\	現象(げんしょう) 減少(げんしょう)
\\	月食は珍しい現象だ
\\	主食(しゅしょく) たいていの日本人は米を主食にしている
\\	挿入する(そうにゅう) 置く(おきます) 2語の間にコンマを挿入する 挿入キー
\\	駄目です(だめ)
\\	(警護) 言う 
\\	(おっしゃる) おっしゃいます 私はあなたのおっしゃることが分かりません
\\	冠(かんむり) 王冠(おうかん)
\\	先生は生徒の作った花の冠をかぶった
\\	蛇口(じゃぐち) 水が壊れた蛇口から吹き出した
\\	夕飯(ゆうめし) 夕飯(ゆうはん) 夕食(ゆうしょく) 晩御飯(ばんごはん)
\\	以前の(いぜん) 前の(まえ) 以前の住所 以前にお会いしませんでしたか
\\	湖沼(こしょう) 故障(こしょう)
\\	逆(ぎゃく) あの人の考えはその逆だ
\\	聴診器(ちょうしんき) 医者は患者の胸に聴診器を当てた
\\	田園(でんえん)
\\	田舎(いなか)
\\	村(むら) 
\\	田園生活 田園風景
\\	制度化する(せいどか) システム化する
\\	周年(しゅうねん) 12周年記念を祝う(いわう)
\\	流暢に(りゅうちょう) ぺらぺら その言語学者はいくつかの中国語の方言を流暢にはなす
\\	計算する(けいさん) 数える(かぞえます) 私は先月いくたお金を使うか計算しなければならなかった
\\	先ず一月には正月の行事(ぎょうじ)がある
\\	直訳(ちょくやく)
\\	美味しそう(おいしそう)
\\	トラックスーツ
\\	めちゃくちゃ(な)
\\	無秩序の(むちつじょ) あらしで村がめちゃくちゃになった
\\	少ない(すくない) 人は少なかったですか いいえ、大変多かったです
\\	ヘアスプレー ヘアスプレーをかける
\\	~(ないで)てほしいです 彼は私があそこへ行ってほしい
\\	~を 挑発する(ちょうはつ) その事件は戦争を挑発した 発に乗る 挑発的(な)
\\	(お)誕生日 母の誕生日のプレゼントを買います
\\	スケジュール スケジュールを自分で作る
\\	受話器(じゅわき) 電話が鳴る。スーザンは受話器を取り上げた
\\	独楽(こま) こまを回す
\\	意見を伝える(つたえる) 意見を述べる(のべます) 彼は手紙でニューズウィック誌に自分の意見を伝えた
\\	速い(はやい) 早い(はやい)
\\	骨(ほね)
\\	輸血(ゆけつ) 輸血を受ける
\\	明後日(あさって)
\\	潜る(もぐります)
\\	潜水する(せんすい)
\\	深く潜る
\\	万年筆(まんねんひつ)
\\	危害(きがい) あなたに危害を加えることはありません 危害を加える
\\	消える(きえます) 電気が消えた
\\	各地(かくち) 私はイギリス各地を見物した インフルエンザが各地に広がった
\\	水切りする(みずきり) 
\\	水切り遊び(みずきりあそび)
\\	観測(かんそく) 観察(かんさつ) 星を観測する
\\	冷凍庫(れいとうこ)
\\	馬力(ばりき) 馬力のある人 100馬力のモーター
\\	ステロイド
\\	外車(がいしゃ) 彼は外車を運転する
\\	水泳帽(すいえいぼう)
\\	爆笑する(ばくしょう) 笑い出す(わらいだします) 彼がそう言うと聴衆から爆笑が起こった
\\	いもり
\\	返信(へんしん) 
\\	変身(へんしん) 答え(こたえ) 返事(へんじ) 彼の手紙に返信を出すべきだった
\\	部長(ぶちょう) 課長(かちょう)
\\	ドーム ドーム球場
\\	大したもん(もの) 大したもんではない
\\	呼ぶ(よびます)
\\	眉(まゆ) 眉毛(まゆげ) 繭 (まゆ)
\\	(俗) 
\\	大根足(だいこんあし) 大根足だから、白いストッキングはやめといたほうがいいわね
\\	海峡(かいきょう) 回教(かいきょう) 
\\	イギリス海峡 この船はジブラルタル海峡を通過した
\\	選挙(せんきょ) 総選挙(そうせんきょ) 彼は来年の選挙に立候補するだろう 全国民が選挙で投票した(とうひょう)
\\	ほおずき
\\	実習する(じっしゅう) 料理の実習をする 実習生
\\	初歩(しょほ) 基礎(きそ)
\\	基本(きほん)
\\	化学の初歩 初歩の英語 ドイツ語を初歩から習う
\\	肥料(ひりょう) 肥料をやる
\\	単細胞生物(たんさいぼうせいぶつ) 多細胞生物(たさいぼうせいぶつ)
\\	ブローチ
\\	始めに まず
\\	混ぜる(まぜます) 砂糖に塩を混ぜないでください 日本語に英語を混ぜて話す
\\	温暖化ガス(おんだんか)
\\	実験室(じっけんしつ) その実験室には最新式の設備がある
\\	復活祭(ふっかつさい) イースター
\\	るーろ・うーえ 
\\	しろ 
\\	こい 
\\	くれ 
\\	買え 
\\	寝ろ
\\	正式(な)(せいしき) 息子は正式の相続人だった
\\	社会人(しゃかいじん)
\\	牙(きば) 牙をむく
\\	ドアを押し開ける(おしあける) ドアを破ってあける(やぶってあけます)
\\	牧師(ぼくし) 40代になるまでは牧師になるべきじゃない
\\	元年(がんねん) 平成元年
\\	不妊症(ふにんしょう) 不妊症です
\\	お兄さん (おにいさん) 兄 (あに)
\\	~を 渋る(しぶります) 返事を渋る 出席を渋る
\\	屋外(おくがい) 屋外で ラグビーは屋外スポーツである
\\	濯ぐ(すすぐ) 濯ぐ(ゆすぐ)
\\	口をゆすぐ 洗濯物をゆすぐ
\\	曾(そう)~ 曾祖母(そうそぼ)
\\	アルバイト
\\	血圧(けつあつ) 血圧が低い 血圧を計りましょう
\\	誰かを 訪ねる(たずねます)
\\	破片(はへん) 割れた花瓶の破片 ガラスの破片が指に刺さった
\\	売り手(うりて)
\\	安静(あんせい) 安静にする 医者は絶対安静を命じた
\\	別荘(べっそう) 彼女は海辺に別荘を持っている
\\	自信(じしん) 地震(じしん) 
\\	自身(じしん) 
\\	ケンは自信を持って試験を受けた
\\	アルマジロ
\\	グラフ 図表(ずひょう)
\\	グラフにする
\\	子守歌(こもりうた)
\\	寄り道する(よりみち) 寄り道して帰る 寄り道しないですぐお帰り 途中寄り道するところがあるから, 少し帰りが遅くなるよ
\\	カレー タイなどの暑い国には、カレーのような辛い食べ物がたくさんある
\\	皇室(こうしつ) 王室(おうしつ)
\\	検定試験(けんていしけん) 教員検定試験 検定する(けんてい)
\\	自動車(じどうしゃ) 車(くるま)
\\	砂糖の衣(さとうのころも) 衣(ころも) 
\\	操舵輪(そうだりん) 操舵する(そうだ)
\\	操縦する(そうじゅう)
\\	一環(いっかん)
\\	部分(ぶぶん) 都市計画の一環としてごみ処理場が造られた
\\	接続(せつぞく) インターネットの接続 2本のパイプを接続させる
\\	住宅ローン(じゅうたく)
\\	酒(さけ) 私はいつも酒を飲みながら勉強してる
\\	西(にし)
\\	オゾン オゾン層(そう) オゾンホール
\\	まつげ
\\	財政危機 (ざいせいきき) 財政(ざいせい) 国の財政は豊かだ
\\	好かれる(すかれます) 子供は母に好かれます
\\	バカンス 休暇(きゅうか)
\\	休日(きゅうじつ)
\\	(夏)休み
\\	気違い(きちがい) デルバートは気違いだと思う
\\	推定する(すいてい) 彼女は30歳過ぎだと推定する
\\	イスラエル
\\	テントを張る(はります) 張る(はります)
\\	持ち主(もちぬし) 主人(しゅじん) 彼があの車の持ち主ですか
\\	大きいお世話
\\	遺伝子(いでんし) 遺伝学(いでんがく)
\\	室内(しつない)
\\	内部(の)(ないぶ)
\\	内側(うちがわ)
\\	室内に入るときは帽子をとりなさい
\\	たどる(たどります) 田舎道をたどって行く 観光ルートをたどる 物語の筋(すじ)をたどる
\\	(野球) 
\\	カーブ 彼は右に急カーブした
\\	一刻千金 (いっこくせんきん) 時は金なり (ときはかねなり)
\\	リムジン
\\	のために 病気の母のために、スープを作った
\\	供給者(きょうきゅうしゃ)
\\	流れる(ながれます) 速く流れる雲
\\	のどが渇いている(かわいています) 乾く(かわく)
\\	絹(きぬ) シルク 
\\	歌手(かしゅ)
\\	膀胱(ぼうこう)
\\	~なくて~ 彼女は学生ではなくて、先生です
\\	栓抜き(せんぬき) コルク栓抜き 瓶の栓をする/抜く(ぬきます)
\\	~ないで/ずにはいられない 買い物せずにはいられない 買い物しないではいられない
\\	金魚鉢(きんぎょばち)
\\	馴らす(ならします) 鳴らす(ならす)
\\	慣らす(ならす)
\\	ゴリラをならす いるかをならして芸をさせる
\\	~式(しき) 式典(しきてん)
\\	入学式 卒業式
\\	話題(わだい) 話題の人 話題を変えよう
\\	哺乳動物(ほにゅうどうぶつ) 哺乳類(ほにゅうるい) 鯨は哺乳類です
\\	折る(おります)
\\	巨大(な)(きょだい) 
\\	莫大(な)(ばくだい) 
\\	巨木(きょぼく)
\\	血脈(けつみゃく) 
\\	血管(けっかん)
\\	(諺) 
\\	百聞は一見に如かず (ひゃくぶんはいっけんにしかず) それは正に(まさに)「百聞は一見にしかず」というところである
\\	事件(じけん) 出来事(できごと)
\\	その作家は殺人事件を生々しく描写した 歴史的事件
\\	病人(びょうにん)
\\	第一人者(だいいちにんしゃ) 彼は心臓外科(げか)の第一人者だ
\\	段々(だんだん) 次第に(しだいに)
\\	罪人(つみびと)
\\	弱火で(よわび) 強火で(つよび)
\\	義務教育(ぎむきょういく)
\\	社長(しゃちょう)
\\	統率力(とうそつりょく) 彼女は統率力がある
\\	埃を払う (はこりをはらう)
\\	見つかる(みつかります) 僕は大学で勉強したいことが見つかりませんでした
\\	究明(きゅうめい) 明らかにする(あきらか)   原因を究明する 科学者たちは癌(がん)の原因を究明中だ
\\	実る(みのります) 長い間の研究がやっと実った
\\	使い捨て(の)(つかいすて) 使い捨てのオムツ
\\	魚雷(ぎょらい)
\\	並木(なみき) 並木のある歩道 ポプラ並木を散歩する
\\	施す(ほどこします) 行う(おこないます) 与える(あたえます) 人に恩恵(おんけい)を施す 手術を施す 子供たちにスパルタ教育を施す
\\	つまみ このつまみを左へ回すと音が小さくなった
\\	(俗)
\\	ぺチャパイ パイ: 
\\	おっぱい ぺチャ: 
\\	ぺちゃんこ 
\\	胸がぺちゃんこ あたしみたいなぺチャパイには、ブラジャーなんか要らないの
\\	笑み(えみ) 微笑み(ほほえみ)
\\	前線(ぜんせん) 桜前線(さくらぜんせん) 彼らは兵士として前線に行った
\\	腐肉(ふにく) 死肉(しにく) 歯肉(しにく)
\\	腐肉食(ふにくしょく) 腐肉を食べる鳥
\\	バニラ豆(まめ) 
\\	バニラ・ビーンズ
\\	市場(いちば)
\\	行動する(こうどう) 行動を取る 理論を行動に移す 上田さんは大変行動的な人です
\\	足首(あしくび)
\\	鈴(すず) 鈴が鳴る(なります)
\\	輪ゴム(わゴム) ゴムバンド 文房具(ぶんぼうぐ)
\\	アスパラガス
\\	道に迷う(まよいます)
\\	象牙(ぞうげ) 彼は象牙に図案を彫った
\\	米(こめ)     稲(いね)   ご飯(ごはん)
\\	松かさ(まつかさ)
\\	金曜日に宿題が出る
\\	計量器(けいりょうき) ガスの計量器 計量する(けいりょう)
\\	一日千秋(の思いで) (いちじつせんしゅう) あなたのお帰りを一日千秋の思いでお待ちしています
\\	何の為に(ために) このキーは何のための物ですか
\\	墜落(ついらく) 墜落事故(ついらくじこ) 墜落した飛行機は急に燃え上がった
\\	門限(もんげん) 彼は門限に遅れた
\\	建造する(けんぞう) 建設する(けんせつ)
\\	建築する(けんちく)
\\	建てる(たてます)
\\	その工場はコンクリートで建造された
\\	笛 
\\	フルート 
\\	俳人(はいじん) 句作する(くさく)
\\	インチキ(な) いんちきな医者 いんちきをやる
\\	大勢(おおぜい) 私の家族は大勢です
\\	音節(おんせつ) 
\\	という語ではアクセントは2番目の音節にあります
\\	好きになる
\\	風邪(かぜ) 風邪をひく
\\	こちらーここーこっち そちらーそこーそっち あちらーあそこーあっち
\\	アイデア アイデアを考える
\\	線香花火(せんこうはなび)
\\	汁(しる) 彼女はお茶にレモンの汁をたらした
\\	分(ぶん) 
\\	分 (の
\\	分 三日分の仕事 十ドリル分のガソリン
\\	パック 
\\	小包(こづつみ) 
\\	荷物(にもつ) 
\\	女の方(おんなのかた) 男の方
\\	怠け者(なまけもの) 怠ける(なまけます) 怠惰(たいだ)
\\	人生(じんせい) 人生の目的は何ですか 人生を楽しむ
\\	何か(なにか)
\\	利子(りし) 彼はそのローンに7%の利子を払った
\\	小便所(しょうべんじょ)
\\	届く(とどきます) 着く(つきます) お手紙が昨日届きました 彼女の声は遠くまで届く
\\	九九(くく) 掛け算(かけざん)
\\	九九を言う
\\	~に 直面する(ちょくめん) 彼は危険に直面しても勇気があった 厳しい現実に直面する
\\	湿る(しめります) ぬれる 
\\	湿ったタオル
\\	昼間(ひるま) 夜間(やかん) 薬缶(やかん) 
\\	浮き(うき) 釣り糸に浮きを付ける
\\	うろこ 目からうろこがおちた 魚のうろこを取る
\\	教会(きょうかい)
\\	下(しも) 霜(しも)
\\	下半期(しもはんき) 下の句
\\	くちばし きつつきのくちばし
\\	膵臓(すいぞう)
\\	~に 席をとる 
\\	~に 腰をかける(こし) ~に 座る(すわります) 中ほどの列に席をとる
\\	摩擦(まさつ) 木片(もくへん)を摩擦して火を起こす  米国人とイギリスとの間の摩擦が高まった
\\	(お)へそ
\\	うなずく 賛成してうなずく
\\	大変(たいへん)
\\	市役所(しやくしょ)
\\	政治家(せいじか) あの政治家は野心に満ちている
\\	非常口(ひじょうぐち) 非常時(ひじょうじ)
\\	体重計(たいじゅうけい) 体重計に乗りなさい
\\	扇ぐ(あおぎます) 帽子で自分を扇ぐ
\\	詩人(しじん) 詩を作る
\\	そっと そっと触れる 彼はそっと入ってきた
\\	挨拶(あいさつ) 誰かに挨拶する
\\	時計はチクタク鳴る(なります)
\\	期限(きげん) 締め切り(しめきり)
\\	紀元(きげん)
\\	機嫌(きげん)
\\	一定の期限 返事の期限
\\	流星雨(りゅうせいう) 流れ星(ながれぼし)
\\	誰か(だれか)
\\	液体(えきたい)
\\	決定(けってい) 決心(けっしん) 決まり(きまり)
\\	差し当たり私たちは決定することを延期したいと思う
\\	磨く(みがく)
\\	かなづち ハンマー 犬小屋を作るために、父のハンマーを借りた
\\	難関(なんかん) 越えられない難関にぶつかる
\\	ゾウリムシ(草履虫)
\\	星占い(ほしうらない) 彼は雑誌を開くと、たいていまず自分の星占いを読みます
\\	ここから遠いですか。
\\	垂らす(たらします) ハンカチに香水を垂らす 屋根から地面までロープを垂らす 彼は額に汗を垂らしていた
\\	認める(みとめます) 
\\	結婚を認める 誤りを認める 子供をわが子と認める
\\	掘る(ほります) その子供は砂場にトンネルを掘った
\\	麻(あさ) 朝(あさ)
\\	朝寝坊(あさねぼう)する 朝寝坊
\\	ずつ 男の人と女の人が3人ずつです 一人ずつバスに乗りなさい
\\	どんなに ~た/だって どんなに ~て/でも どんなに考えたって分からないよ
\\	像(ぞう) 偶像(ぐうぞう) 胸像(きょうぞう) 自由の女神の像 獅子の像 マリア像
\\	巡査(じゅんさ)
\\	警官(けいかん)/警察官 お巡りさん(おまわりさん)
\\	山田巡査 交通巡査
\\	愛くるしい(あいくるしい) 可愛い(かわいい)
\\	(女性語)	おやまあ!
\\	速読(そくどく)
\\	~強(きょう) ~ちょっと ~弱(じゃく) 一杯強 
\\	000円強
\\	ヴィオラ ビオラ
\\	養子縁組(ようしえんぐみ)
\\	緑(の)(みどり)
\\	交差点(こうさてん) 十字路(じゅうじろ)
\\	着信音(ちゃくしんおん) 着メロ 
\\	着信メロディー)
\\	発達する(はったつ) 成長(せいちょう)
\\	進歩(しんぽ)
\\	ビタミンの不足は体の発達を妨げる この国の教育は急速な発達を遂げた
\\	困惑(こんわく) 彼の質問に困惑する
\\	いってきます。 いってらっしゃい。
\\	黄身(きみ) 君(きみ)
\\	黄身の2 つある卵 黄味がかった茶色
\\	声明する(せいめい) 言明する(げんめい)
\\	宣言する(せんげん)
\\	支持を声明する 声明を出す/発表する
\\	明ける(あけます) 明けましておめでとう!
\\	舌(した) 舌を出す(だす)
\\	~に がっかりする ~をがっかりさせる がっかりするな! 私は彼の死を聞いてがっかりした
\\	掲示(けいじ)
\\	刑事(けいじ)
\\	伝言(でんごん)
\\	メッセージ 
\\	掲示は「芝生に入らないでください」と書いてあります
\\	缶切り(かんきり) 栓抜き(せんぬき) 私は時々はさみを缶切りとして使う
\\	甘やかす(あまやかします) 甘える(あまえます) 子供を甘やかすな
\\	(どんなに)たって 
\\	ても あの人には話したって分からない どんなに寒くたって大丈夫 静だって
\\	動物(どうぶつ) あの鼻が長い動物は何ですか
\\	トウガラシスプレー 携帯用トウガラシスプレー(けいたいよう)
\\	~を 終える(おえます) ~が 終わる(おわります)
\\	シロップ
\\	削る(けずります) 鉛筆をけずる
\\	する、来る、調べる、死ぬ	する 
\\	できる 来る 
\\	こられる 調べる 
\\	調べられる 死ぬ 
\\	死ねる
\\	いってきます いってらっしゃい
\\	学説(がくせつ) 学説を立てる 彼はついにその学説を理解するにいった
\\	やっぱり やはり ああ,やっぱり思ったとおりだ やっぱり行くのはやめる
\\	~に 無関心(な)(むかんしん) わが国の青年男女に政治に無関心です
\\	湿っぽい(しめっぽい)
\\	ぬれた 
\\	湿っぽい空気
\\	塚(つか) 丘(おか)
\\	揺れる(ゆれます) 列車が通ると家が揺れる 家が少し揺れるのを感じた
\\	ペットボトル
\\	リン
\\	爪をかむ(つめ)
\\	割れる(われます) 砕ける(くでけます)
\\	ガラスが割れている 風船が割れた
\\	アスファルト アスファルトの[を敷いた]道路
\\	おめでとう。 おめでとうございます。
\\	万葉仮名(まんようがな)
\\	割れ易い(われやすい)
\\	入れ替わり立ち替わり (いれかわりたちかわり) 次々(つぎつぎ)
\\	店には入れ替わり立ち替わり客がある
\\	効く(ききます) 聞く(ききます) 
\\	動く(うごきます) 
\\	この薬は頭痛によく効く テレビのコマーシャルが効いた
\\	困ったことに、... 困ったことに彼はどこが間違っているのかが全く(まったく)分からない
\\	文書(ぶんしょ) 書類(しょるい)
\\	資料(しりょう) 
\\	文書でお申し込みください 父はフランス語の文書を日本語に翻訳した
\\	~の上に 横になる(よこ) 芝生の上に横になっている
\\	ハンカチ
\\	縄(なわ)
\\	麺類(めんるい) うどんやラーメンなどの麺類もよく食べる
\\	セクシー(な)
\\	睡眠(すいみん) 眠り(ねむり) 寝(ね) このところ睡眠が足りない もう少し睡眠をとる必要がありますね
\\	犯す(おかします) 罪を犯す 規則を犯す
\\	肩(かた)
\\	消毒する(しょうどく) 熱湯消毒する(ねっとう) 傷口をアルコールで消毒する
\\	崇拝(すうはい) 崇拝者(すうはいしゃ) 彼らは彼を英雄として崇拝した
\\	冷える(ひえます) 冷やす(ひやします) 今朝はいくぶん冷えますね
\\	ふくろう ふくろうがほうほうと鳴いた
\\	のれん のれんを押し分けて中へ入る のれんを築く あの店は古いのれんを誇っている(ほこる)
\\	風疹(ふうしん) 風疹にかかる
\\	いっらしゃい
\\	視線(しせん) 四川(しせん)
\\	彼女の視線を背中に感じた 一瞬二人の視線が合った ~に 視線を向ける
\\	ランキング ランキン
\\	専攻する(せんこう) 専門(せんもん) あなたは何の科目(かもく)を専攻しているのですか
\\	食器(しょっき)
\\	恋煩い(こいわずらい)する かわいそうな娘だよ. 恋煩いをしているんだ
\\	杏(あんず) 桃(もも) あんずの花が満開だ
\\	大西洋(たいせいよう) 大洋(たいよう) 太陽(たいよう) 
\\	県(けん)
\\	非常口(ひじょうぐち)
\\	未だ(まだ) まだ分からない
\\	休業(きゅうぎょう) 本日休業 その店は月曜日には休業です
\\	べきだ 
\\	であるべきだ す(る)べき 今、家を買うべきじゃないよ
\\	代わりに(かわりに)   ココアの代わりに 速く代わりに
\\	あら筋(あらすじ) 前回までのあら筋
\\	処理する(しょり) あの問題をどうやって処理しますか 問題の処理を誤る
\\	~料金(りょうきん) 電気料金
\\	釣り(つり) 釣りする
\\	添加する(てんか) 甘味(あまみ)を添加する この食品は防腐剤を添加していない
\\	波長(はちょう) 私は彼と波長が合う 
\\	に波長を合わせる
\\	精神科医(せいしんかい)   心理学者(しんりがくしゃ)
\\	思春期(ししゅんき) 僕は自分の思春期を振り返ると、いつも憂鬱な気分になる 思春期の少年
\\	的を射った(まとをいった) 的を射る 的を射た言葉 彼は射るような目で私を見た
\\	平城京(へいじょうきょう)
\\	サンタクロース
\\	高齢者(こうれいしゃ) お年寄り 老人(ろうじん) 年を取った人
\\	目的(もくてき) 目標(もくひょう)
\\	彼の目的はそのテストに受けることです
\\	砂糖黍(さとうきび)
\\	持続(じぞく) 
\\	続く(つづきます) 継続(けいぞく) 
\\	続ける(つづけます)
\\	薬品(やくひん) 薬(くすり)
\\	サイレン サイレンが鳴った
\\	お昼(おひる) 昼ごはん 昼食(ちゅうしょく) お昼、どうしょうか
\\	出典(しゅってん) 引用は出典を明らかにすること この句の出典は聖書である
\\	宇宙(うちゅう)
\\	強烈(な)(きょうれつ)
\\	強い(つよい)
\\	彼に鼻へ強烈なパンチをした その映画は強烈な印象を彼に与えた
\\	溶岩(ようがん) 火山は炎と溶岩を吹き出す
\\	鰻(うなぎ) 常用外
\\	痩せる(やせます)
\\	教えてもらえませんか この漢字の読み方を教えてもらえませんか
\\	云々(うんぬん) エトセトラ
\\	親友(しんゆう)
\\	仲良し(なかよし)
\\	私は彼を親友の一人に数えている
\\	国土(こくど) 国土の80パーセント近くが山だ
\\	痴漢(ちかん) 込んだ電車の中の痴漢
\\	漁業(ぎょぎょう) 村人たちは主として漁業に従事している
\\	上手な字
\\	発売する(はつばい) 発売中である その新型車は五月に発売される
\\	安眠(あんみん) 安眠できない人
\\	チェロ
\\	~にまったく耳を貸さない
\\	目覚し時計 (めざましどけい)
\\	ちんぷんかんぷん(な)
\\	でたらめ(な)
\\	それは全くちんぷんかんぷんだ ちんぷんかんぷんなことを言う でたらめを言う
\\	プリンター
\\	犬小屋(いぬごや)
\\	暗やみ(くらやみ) 
\\	暗黒(あんこく) 
\\	ふくろうは暗闇が見える
\\	風鈴(ふうりん)
\\	歌が上手(うたがじょうず) あの歌が上手じゃない男の人は誰ですか
\\	秘密で/に話す
\\	ダイナマイト ~を ダイナマイトで 爆破する(ばくは)
\\	水準(すいじゅん)
\\	標準(ひょうじゅん) 
\\	基準(きじゅん)
\\	賃金水準(ちんぎん) 生活の水準を高める(たかめます)
\\	車輪(しゃりん) ホイール 一輪車は車輪がひとつしかない
\\	稲(いね) 米(こめ)      ご飯(ごはん)
\\	針鼠(はりねずみ)
\\	~の話だと 
\\	~によると タクシーの運転手の話だと、今日天気がいい
\\	会計係(かいけいがかり)
\\	不便(な)(ふべん) 電話がないのは不便なことです
\\	進む(すすみます) 大学院に進むつもりですか
\\	電子レンジ(でんし)
\\	こぶ せむし
\\	洗う(あらう)
\\	クラクションの音 クラクションを鳴らす(ならす)
\\	涙(なみだ)
\\	夕暮れ(ゆうぐれ) 夕方(ゆうがた)
\\	晩(ばん)
\\	夕暮れに 夕暮れ時になるとこの辺に人々が集まってくる
\\	筆(ふで)
\\	~に 値する(あたい) この本は一読(いちどく)に値する この問題はこれ以上の論議に値しない 彼の友情は千金に値する
\\	戻る(もどります)
\\	艶(つや) 宝石を磨いてつやを出す つやのある つやのない皮 話につやを付ける
\\	ステレオ
\\	サフラン
\\	汚れる(よごれます) ~を 汚す(よごします)
\\	東京 東京はニューヨークよりひとが多いです はい、ずっと多いです
\\	枕カバー (まくら)
\\	数える(かぞえます) うちの息子は百まで数えられる
\\	卑怯(な)(ひきょう) 
\\	彼は私にひきょうなことをした そりゃひきょうだよ
\\	正常(せいじょう) 通常(つうじょう) 普通(ふつう) 君の脈(みゃく)は正常だ
\\	これどうしますか
\\	乳首(ちくび)
\\	光合成(こうごうせい) 光合成する/を行う
\\	一日中(いちにちじゅう)
\\	中毒(ちゅうどく) 彼は麻薬中毒だ 麻薬中毒者
\\	裸足(はだし) 裸足で歩く
\\	はち みつばち
\\	俳優(はいゆう) 男優(だんゆう) 女優(じょゆう)
\\	耳垢(みみあか) 耳あかを取る
\\	十字(じゅうじ) 十字を切る
\\	で この翻訳は一ページ(いちページ)二千円お払いします このりんごは一山(ひとやま)で二百円です
\\	疲れる(つかれます)
\\	病院(びょういん)
\\	葉(は) 歯(は)
\\	葉っぱ(はっぱ) 葉が木から落ちた
\\	美術館(びじゅつかん)
\\	マイペース 彼はマイペースで仕事をする
\\	端子(たんし) 
\\	端子
\\	すると…ことが多くなった お金を払うとカードを利用することが多くなった 用があると、電話ですることが多くなった
\\	倫理(りんり) 道徳(どうとく)
\\	倫理的(りんりてき)
\\	居心地(いごこち)がいい 居心地のよい部屋 彼の前ではいつも居心地が悪い
\\	手袋(てぶくろ)
\\	黄疸(おうだん) 横断(おうだん)
\\	手錠(てじょう) 私は彼に手錠をかけた
\\	残念です(ざんねん)
\\	紫禁城 (しきんじょう)
\\	(~を)裏切る(うらぎります) 裏切り 裏切り者(うらぎりもの)
\\	久しぶり(ひさしぶり) ほんとうに久しぶりに会いましたね 久しぶり(ひさしぶり) 久々(ひさびさ) 暫くですね
\\	~られるだけ 食べられるだけ食べたい できるだけ
\\	帽子(ぼうし) 帽子を亡くしてしまいました
\\	手を打つ(うちます)
\\	クリスマスイヴ
\\	愛読者(あいどくしゃ) 漱石の愛読者 朝日新聞の愛読者だ
\\	熱波(ねっぱ)
\\	作曲家(さっきょくか) 作曲者(さっきょくしゃ) コンポーザー 作曲家の中で誰が一番好きですか
\\	階段(かいだん)
\\	先回(せんかい) 旋回(せんかい)
\\	染める(そめます) 白髪を黒く染めた 染めた髪の毛の女の人
\\	挑戦(ちょうせん)  朝鮮(ちょうせん)
\\	チャレンジ ~に 挑戦する 難問に挑戦する 世界記録に挑戦する
\\	泡(あわ) 泡が立つ 泡風呂(あわぶろ)
\\	転ける(こけます)(関西弁) 転ぶ(ころびます)
\\	倒れる(たおれる)
\\	玄関(げんかん)
\\	軽蔑する(けいべつ) あなたは老人を軽蔑してはいけない
\\	ポスト 郵便箱(ゆうびんばこ)
\\	あふれる(あふれます) ビールがコップからあふれてしまった
\\	水切りボール(みずきり) きざんだキャベツの水切りをする
\\	インコ 
\\	オウム 
\\	犬と猫とインコとどれが一番好きですか。
\\	ごえんりょなく。
\\	崩壊する(ほうかい) その家は地震で崩壊した この事件は運動を崩壊させた
\\	咲く(さきます) 開花する(かいか) 山に花が咲いています
\\	麓(ふもと) それは富士山の麓にある
\\	元首(げんしゅ) 一国の元首
\\	椰子(やし) ココヤシ 椰子油(やしゆ)
\\	天然の(てんねん) 自然の(しぜん) 天然の美
\\	~が 見える(みえます) まだ、富士山は見えませんか。
\\	俗(な)(ぞく) 俗なことを言う  あいつは俗な男だよ 俗な読物  
\\	俗に言えば/いわゆる
\\	読書室(どくしょしつ)
\\	鉛筆削り(えんぴつけずり)
\\	デザート 食後(しょくご) 
\\	食後は何にしましょうか
\\	不親切(ふしんせつ)
\\	吸盤(きゅうばん) タコの吸盤
\\	~を ぬらす ぬれる(ぬれます) 雨で靴をぬらしてしまった
\\	浴室(よくしつ) お手洗い(おてあらい) トイレ この家には浴室が二つ付いている
\\	ぶどう酒(しゅ)
\\	ワイン 
\\	生物(なまもの) 生物(せいぶつ)
\\	生物が食べられない
\\	要望する(ようぼう) 要求する(ようきゅう)
\\	請求する(せいきゅう)
\\	願う(ねがいます)
\\	この本はあなたの要求にぴったりです 要望に応じて
\\	彫刻家(ちょうこくか) ルネサンスの彫刻家
\\	さば さばを読む 彼は101歳だというがいくらかさばを読んでいるさ
\\	シーツ シーツを敷く
\\	通訳(つうやく) ホワイト博士が我々の通訳をして下さった
\\	副作用(ふくさよう) 抗生物質の副作用で食欲がなくなった
\\	平仮名(ひらがな)
\\	鉄砲(てっぽう) 銃(じゅう) 鉄砲を撃つ(うつ) 人に鉄砲を向ける
\\	従姉妹(いとこ) 再従姉妹(はいとこ)
\\	授業(じゅぎょう)
\\	林(はやし) 林の中でいっぱい虫に刺された
\\	特別(な)(とくべつ) そも店は夏の間特別割引をやった
\\	副産物(ふくさんぶつ)
\\	北欧(ほくおう) 南欧(なんおう)
\\	選出する(せんしゅつ)
\\	選挙する(せんきょ)
\\	選ぶ(えらびます) 
\\	選択する(せんたく) 
\\	彼は会長に選出された 選挙(せんきょ) 投票(とうひょう)
\\	去る(さります) 出発する(しゅっぱつ)
\\	出る(でます)
\\	再び戻らないつもりで彼はパリを去った 世を去る
\\	雲雀(ひばり)
\\	傷つける(きずつける) 君の心を傷つけるつもりはなったんだ
\\	~が (構わない)かまわない ~を 気にしない 私のことはかまわないでください 鉛筆で書いても構いませんか 構うものか
\\	支度(したく)
\\	準備(じゅんび)
\\	用意(ようい)
\\	母は夕食の支度をしています 式典の支度はすべてできた
\\	輝く(かがやきます) 猫の目はなぜ暗闇で輝くのですか
\\	創造力(そうぞうりょく) 想像力(そうぞうりょく)
\\	図々しい(ずうずうしい) 何とずうずうしいやつだろう 彼はずうずうしくまたやって来た
\\	炭(すみ) 墨(すみ)
\\	隅(すみ)
\\	石炭(せきたん)
\\	辞する(じする) 辞任する(じにん) 辞める(やめます)
\\	術語(じゅつご) 述語(じゅつご)
\\	医学上の術語
\\	雷(かみなり) 雷がその塔(とう)に落ちた
\\	知恵遅れの(ちえおくれ)
\\	雨宿りする(あまやどり) 雨宿りに格好の大木 店の軒下で雨宿りをする
\\	選手(せんしゅ) このチームの選手たちはみんな大男だ
\\	タンポポ タンポポの綿毛(わたげ)
\\	定年(ていねん) 私たちの会社で定年は60歳です
\\	農薬(のうやく)
\\	厳重(な)(げんじゅう) 厳しい(きびしい) 厳格(げんかく) 
\\	交通規則は厳重に守らなければならない
\\	にわたって 13年間にわたって
\\	金属(きんぞく) メタル
\\	オゾンホール
\\	用意(ようい) 支度(したく)
\\	準備(じゅんび)
\\	6時半までには、夕食の用意ができているだろう
\\	危機(きき) 消滅の危機
\\	案山子(かかし)
\\	発電所(はつでんしょ) 水力発電所 (すいりょくはつでんしょ)
\\	本間に(ほんま)(関西弁) 本当に(ほんとう)
\\	薬(くすり)を飲む
\\	愛国心(あいこくしん)
\\	嫌がる(いやがります)
\\	嫌う(きらいます) 嫌いです 嫌です(いや) アンは外国で一人で暮すのを嫌がらなかった 彼はクラスの皆に嫌がられている
\\	烏賊(いか)
\\	(フレンチ)フライドポテト
\\	ジムのだ ジムなのだ・ジムなんだ
\\	花瓶(かびん) 瓶(びん)
\\	勇敢(な)(ゆうかん) 勇気(な)(ゆうき) 勇ましい(いさましい)
\\	太陽系(たいようけい)
\\	遮る(さえぎります) 邪魔する(じゃま)
\\	道を遮る 人の言葉を遮る 倒れた木が交通を遮った
\\	早産(そうざん) 先回は早産でした
\\	かまくら イグルー
\\	お茶を入れる(いれます)
\\	新聞社(しんぶんしゃ)
\\	幾つ(いくつ)
\\	四人分(よにんぶん)
\\	テレビの音をもっと大きくしてください
\\	鉄格子(てつごうし) 鉄格子のはまった窓
\\	見事(みごと)な 公園の桜が見事に咲いた(さく)
\\	非難する(ひなん) 避難する(ひなん) 
\\	批判する(ひはん) 
\\	批評(ひひょう) 彼らは彼を無能だと非難した
\\	胸像(きょうぞう) 
\\	カシューナッツ
\\	鰐 (わに) 常用外
\\	機構(きこう) 気候(きこう)
\\	組織(そしき)
\\	北大西洋条約機構 (きたたいせいようじょうやくきこう) 機構を改める
\\	トナカイ
\\	携帯電話(けいたいでんわ)
\\	牧羊犬(ぼくようけん) 牧羊犬が、羊の群を牧場へ連れていく
\\	ありうる ありえる
\\	明治時代 
\\	生(の)(なま) 生の魚を食べたことがある?
\\	迂回(うかい)
\\	回り道(まわりみち)
\\	迂回路(うかいろ) バイパス 迂回する 回り道する その通りは工事中だったので迂回した
\\	守る(まもります) 犬は主人を危害から守った
\\	階級(かいきゅう) 身分(みぶん)
\\	武士階級(ぶしかいきゅう) ブルジョア階級 労働階級(ろうどう)
\\	小学校(しょうがっこう)
\\	~に 触れる(ふれます) 触る(さわります) 触ると冷たい
\\	外れ(はずれ) 町・村外れ(むらはずれ)
\\	お嬢さん(おじょうさん) 娘(むすめ) 
\\	あの人はいい所のお嬢さんです
\\	本社(ほんしゃ) 本店(ほんてん)g
\\	支社(ししゃ) 本社勤務を希望した
\\	機関車(きかんしゃ)
\\	怖がる(こわがります) 恐れる(おそれます) びくびくする 彼女はしかられるのを怖がって黙っていた だれかを怖がらせる
\\	退屈(たいくつ) 退屈(な) つまらない
\\	芸術院(げいじゅついん) 芸術院賞(げいじゅついんしょう)
\\	~に 感心する(かんしん) 関心する (かんしん) 
\\	私はその小学生の作文にほとほと感心した
\\	初演(しょえん) そのオペラは1960年に初演された
\\	バレー
\\	聖地(せいち)
\\	とか や 「乱」とか「七人の侍」とかを見た
\\	周り(まわり) 家の周りに塀が立っている
\\	右利き(の)(みぎきき) 左利き(ひだりきき) ぎっちょ 
\\	純愛(じゅんあい)
\\	櫛(くし) 常用漢字以外 櫛で髪の毛を解かす(とかす)
\\	地球(ちきゅう) 土(つち) 
\\	土地(とち) 
\\	センチ
\\	オープニング・パーティー
\\	算数(さんすう) 算数では正確さが重要だ
\\	この前の日曜日 先週の日曜日 この前の日曜日どこへ行った?
\\	ストレス ストレスの多い ストレスを受ける
\\	お腹がペコペコです
\\	淡水(たんすい) 海水(かいすい) 淡水魚(たんすいぎょ)
\\	赤の他人(あかのたにん) あのね、日本人はそうやって赤の他人に声をかけたりしないの
\\	痛烈 (つうれつ) 痛烈な批判 あんな痛烈なことを言う女は初めてだ
\\	言説(げんせつ) 孔子の言説は有名です
\\	短大(たんだい) 小学校 中学校 高校, 高等学校(こうとうがっこう) 大学
\\	誰かと約束がある(やくそく) ですから、妻と約束があります
\\	余る(あまります) 
\\	残る(のこります)
\\	まだお金が余っている 人が余っている職業
\\	餓鬼(がき) この餓鬼
\\	止む(やむ) 病む(やむ)
\\	風がやんだ 痛みがやんだ
\\	インドネシア
\\	支援する(しえん) 支持する(しじ)
\\	支える(ささえます) その事件で彼は我々を支援してくれた 支援の手を差し伸べる
\\	角隠し(つのかくし)
\\	段々(だんだん) 次第に(しだいに) 彼女はだんだんきれいになちます
\\	未だ(まだ)
\\	発揮する(はっき)
\\	示す(しめます)
\\	見せる(みせます)
\\	彼はサラリーマンとしての腕前を発揮した
\\	背が高い(せがたかい)
\\	~も ~ば、~も~ ~である と共に ~も ~だし 上田さんの部屋にはラジオもあれば、テレビもある 中村さんはポーランド語も話せば、日本語も話す
\\	うん
\\	催促する(さいそく) 彼は家賃の催促をした
\\	~を 減らす(へらします) 減る(へります)
\\	私たちはエネルギーの使用を減らすことができる
\\	殺虫剤(さっちゅうざい)
\\	~けど ~が ~けれど 
\\	~けれども 友達に聞いたけど知らなかった
\\	雪洞(ぼんぼり)
\\	~に 火をつける ~を 燃やす(もやします)
\\	ライターでタバコに火をつけた
\\	給食(きゅうしょく)
\\	とうもろこし
\\	博物館(はくぶつかん)
\\	言わば 言うなれば 犬の毛はいわばセーターだ
\\	侵害(しんがい) 人権を侵害する
\\	ねじ ねじ回し(ねじまわし) ねじを留める(とめます)
\\	光る(ひかります) 輝く(かがやきます) ぎらぎらする 
\\	あの上の方で光っている星は木星だ
\\	抜ける(ぬけます) 外れる(はずれます) 髪が抜ける 取っ手が抜けた
\\	真っ直ぐ(まっすぐ)
\\	敷く(しきます) 布団(ふとん)を敷く
\\	漁業(ぎょぎょう)
\\	またがる(またがります) 
\\	馬にまたがる 手すりにまたがって階段をすべりおりる
\\	ますます 二学期になったら、勉強がますます難しくなってきました
\\	店主(てんしゅ)
\\	どうしても どうしても500 万円必要なんです 娘はどうしても留学したいと言ってきかない
\\	戻す(もどします) 本棚に戻しておいてください
\\	常用漢字(じょうようかんじ)
\\	どれも この計画はどれも一長一短がある そのどれも気に入らない
\\	バイオテクノロジー
\\	イタリア イタリア料理
\\	裏口(うらぐち) 
\\	裏口から出る
\\	翻訳家 (ほんやくか) 
\\	翻訳者(ほんやくしゃ)
\\	読者(どくしゃ) 読書家(どくしょか)
\\	夫々、其々(それぞれ) 学生は夫々順番に卒業証書(そつぎょうしょうしょ)を受け取った
\\	劇団(げきだん) 劇団に入る
\\	対比する(たいひ)
\\	比較する(ひかく) 比べる(くらべます) 私はよく兄と対比される 色の対比
\\	富豪(ふごう) お金持ち(おかねもち) 大富豪(だいふごう)
\\	掛かる(かかります) 暖炉の上に時計が掛かっている
\\	国会(こっかい) 
\\	議会(ぎかい) 国会議員(こっかいぎいん)
\\	スマートな 
\\	ほっそりした 
\\	老人(ろうじん) お年寄り(おとしより) 年を取った人
\\	真空(しんくう) 自然は真空を嫌う
\\	~くて/~で たまらない 非常に(ひじょう) 今日は暑くてたまりません 数学がきらいでたまらない 僕はあの子が好きでたまらない
\\	口実(こうじつ) 言い訳(いいわけ) 彼は遅刻の口実をでっち上げた
\\	弓(ゆみ) 
\\	一つ一つ (ひとつひとつ) 彼は言葉を一つ一つ注意深く選んだ
\\	靭帯(じんたい) 人体(じんたい)
\\	靭帯を切る
\\	指紋(しもん) 指紋を残す
\\	定義(ていぎ) あなたは「幸福」をどのように定義しますか 単語に定義を与える
\\	農業(のうぎょう) 彼らはもともと農業をやっている
\\	財政(ざいせい) 財産(ざいさん)
\\	国の財政は豊かだ
\\	胸毛 (むなげ) 胸毛のある男
\\	性欲(せいよく) 性欲がない
\\	これ以上(いじょう) これ以上は働けません
\\	話しかける 彼女は彼が他の女の子に話しかけると嫉妬した(しっと)
\\	書道(しょどう)
\\	習字(しゅうじ)
\\	書道を習うのは楽しい
\\	接尾辞(せつびじ) 接頭辞(せっとうじ)
\\	かじる(かじります) 噛む(かみます)
\\	板チョコをかじる パンをかじる
\\	アメシスト 紫水晶 (むらさきすいしょう)
\\	ランドセル ランドセルを背負った(せおった)子供
\\	チューリップ チューリップは見事に咲いた
\\	離陸する(りりく) 二機のジェット機が同時に離陸した 着陸する(ちゃくりく)
\\	理由(りゆう) あなたが田舎に住みたい理由を教えてください
\\	爆破する(ばくは) 私たちは巨大な岩をダイナマイトで爆破した
\\	くそ面白く(も)ない
\\	海千山千(うみせんやません) やつは女にかけては海千山千のプレイボーイだ
\\	利益(りえき) 利益のない …の利益のため 一郎は利益よりも友情を大切にする
\\	~にかけて 料理にかけては、彼女は名人だ
\\	トラム 路面電車(ろめんでんしゃ) 市電(しでん) 
\\	市街電車(しがいでんしゃ)
\\	生殖する(せいしょく) これらの雑種は生殖能力を持つ
\\	カンガルー カンガルーの雌は子供を腹の袋に入れて動く
\\	かみそり 電気かみそり
\\	うるう年(うるうどし)
\\	満州(まんしゅう) 満州国(まんしゅうこく)
\\	世代差(せだいさ)
\\	チャック 彼女はかばんのチャックを閉めた
\\	男色(だんしょく) おかま 
\\	ゲイ 男色を行う
\\	祝いの言葉を言う(いわい)
\\	拾う(ひろいます) (持ち上げる) 
\\	(見つける) 
\\	通りで財布を拾った ぎんなんを拾っている
\\	閉鎖する(へいさ)
\\	閉じる(とじります) 閉める(しめます) 彼らは劇場を閉鎖した
\\	ブーブー鳴く(なく) 豚はブーブー鳴く
\\	ビデオ
\\	無実の罪の人 無罪(むざい) 無実(むじつ)
\\	運動(うんどう) 
\\	体操(たいそう) 
\\	運動 
\\	新鮮な空気と運動は健康によい
\\	それを食べるな!
\\	手数料(てすうりょう) 登録手数料
\\	マイホーム 自宅(じたく) 東京でマイホームを持つことは難しい
\\	回数(かいすう) 最近遅刻の回数が増えている 彼が跳んだ回数を指を折って数えた
\\	口唇裂(こうしんれつ) 兎唇(としん) 口唇(こうしん)
\\	唇(くちびる)
\\	縁談(えんだん) 縁談がある 縁談をまとめる お見合い(おみあい)
\\	回り道(まわりみち)
\\	暫く寝る(しばらくねる)
\\	層(そう)   僧(そう)
\\	オゾン層(そう)
\\	梅雨(つゆ) 露(つゆ)
\\	梅雨に入った(いった)/が明けた(あけた) 梅雨入り(つゆいり)・梅雨明け
\\	校門(こうもん) 肛門(こうもん) 
\\	校門を出る 卒業する 大学を出る
\\	見本(みほん) サンプル 
\\	見本を見て、買うか買わないか決めます
\\	それはいけませんね。
\\	礎(いしずえ)
\\	助言(じょげん) アドバイス 人に助言を求める 適切な助言 その株を買えと助言してくれた
\\	指貫(ゆびぬき)
\\	定期航空機(ていきこうくうき)
\\	直に(じきに) 直に雨が降るかもしれない
\\	骨折(こっせつ) 腕を骨折する 腕の骨を折れる
\\	集会(しゅうかい) ミーティング 会合(かいごう) 集会の自由 政治集会
\\	作文(さくぶん) 作文の長さを半分にしてください
\\	金星(きんせい) 近世(きんせい) 
\\	チーズケーキ
\\	終戦(しゅうせん) 開戦(かいせん) 終戦になった
\\	汚い(きたない)
\\	えら
\\	失神する(しっしん)
\\	気絶する(きぜつ) 
\\	意識を失う(いしきをうしなう)
\\	彼女は恐怖で失神した
\\	くそ食らえ(くらえ) ~なんてくそくらえだ 試験なんてくそくらえだ あいつなんかくそくらえだ
\\	首都(しゅと) 都(みやこ)
\\	時間通り(の)(じかんどおり)
\\	古語(こご) この言葉はもう古語になっている
\\	~できるようになる 車の免許を取って、運転できるようになった
\\	公立学校 (こうりつがっこう)
\\	ナチズム
\\	心優しい(こころやさしい) 昔々、ある所に、貧乏だけど心優しいお祖父さんとお祖母さんがいました
\\	信じる(しんじます) 信用する(しんよう) 
\\	信仰する(しんこう) 
\\	もう彼を信じたりしない
\\	秋口(あきぐち)
\\	落ち込む(おちこみます) 穴に落ち込む
\\	モデル
\\	斜視(しゃし) 彼女は斜視だ 斜視の
\\	遊牧(ゆうぼく) 遊牧民(ゆうぼくみん) 遊牧の生活を送る
\\	靴下(くつした)
\\	~学部(がくぶ) 学科(がっか) 農学部(のうがくぶ)
\\	国富(こくふ)
\\	浴衣(ゆかた)
\\	カテーテル
\\	イースター島(とう) イースター島へ十日間(とおか)の旅行をした
\\	夥しい(おびただしい) 夥しい人出(ひとで) 彼らは騒々しいことおびただしい
\\	噴火(ふんか) その活火山は噴火した
\\	西洋(せいよう)
\\	欧米(おうべい)
\\	西欧(せいおう) 西洋諸国(せいようしょこく)
\\	第一歩(だいいっぽ) 第一歩を踏み出す 社会人としての第一歩を踏み出す
\\	【テレビ】	チャンネル 6チャンネルにする
\\	~号車(ごうしゃ) ビュッフェ、またはカフェテリアは9号車にあります
\\	母方(ははかた) 母方(ははかた)の親類 父方(ちちかた)の親類
\\	フロッピー(ディスク) このフロッピーをフォーマットしないでもいい
\\	ろうそく ろうそくを燃やす/に火をつける
\\	で?
\\	一定の(いってい) 一定の時間に 一定の職業
\\	(お)風呂(ふろ) 風呂に入ります
\\	紙袋(かみぶくろ) ペーパーバッグ
\\	掛け金(かけきん) 賭け金(かけきん) 
\\	ドアに掛け金がかかっていた
\\	東欧(とうおう) 東欧諸国(とうおうしょこく)
\\	視聴者(しちょうしゃ)
\\	卑怯(な)(ひきょう) 
\\	臆病(おくびょう)
\\	ひきょうな振る舞いをする 臆病者(おくびょうもの) 卑怯者(ひきょうもの)
\\	境目(さかいめ) 境(さかい)
\\	境界(きょうかい)
\\	古代と中世の境目  夏と秋の境目がはっきりしない
\\	災難(さいなん) それはあなたの家に災難をもたらすでしょう 災難にあう
\\	トラック 貨物自動車(かもつじどうしゃ) 貨物(かもつ)
\\	気が弱い(きがよわい) 気が弱いのが彼の短所だ
\\	合う(あいます) 似合う(にあいます)
\\	この靴はパーティーに合わない あなたの答えは合っていない
\\	すごく
\\	音色(ねいろ)  バイオリンの音色はとても美しい このギターは音色があってる
\\	領事館(りょうじかん) 領事(りょうじ)
\\	焼酎(しょうちゅう)
\\	北(きた) 北京(ぺきん)
\\	法務省(ほうむしょう)
\\	靴拭き(くつふき) ドアマット
\\	黒人(こくじん) 白人(はくじん)
\\	売り切る(うりきります) その雑誌は売り切れでした
\\	疑似~(ぎじ) 疑似家族 疑似科学
\\	ペン 
\\	ボウルペン 
\\	和平(わへい)
\\	平和(へいわ)
\\	和平交渉(へいわこうしょう) 和平会議 和平条約の調印(ちょういん)
\\	コート オーバー
\\	割礼(かつれい) 割礼を受ける 割礼を施す(ほどこします) その国では男子が生まれるとすぐ割礼を行う習慣がある
\\	高地(こうち) 高知(こうち)
\\	平野(へいや)
\\	よく知っている
\\	電球(でんきゅう) この電球は暗いから、もっと明るいのに取り替えたほうがいいですよ
\\	錯誤(さくご) (書) 間違い(まちがい)
\\	誤り(あやまり)
\\	錯誤に陥る
\\	朝型の人間 (あさがた) 早起き(はやおき)
\\	洗車場(せんしゃじょう) 洗車する(せんしゃ) あなたは、一月にどのくらい洗車しますか
\\	数詞(すうし)
\\	赤ちゃん(あかちゃん) 赤ん坊(あかんぼう)
\\	作品(さくひん) 著書(ちょしょ)
\\	彼の最新の作品がその広場にある
\\	後(あと) 後ろ(うしろ)
\\	私の後から犬がついてきた 後を振り返る
\\	ようこそ
\\	独り言を言う (ひとりごと) 「次はどうしようか」と彼女は独り言を言った
\\	つづり 
\\	彼の名前のつづりが分からない 単語のつづりを間違える
\\	水蒸気(すいじょうき)
\\	蒸気(じょうき)
\\	蒸気で動く 蒸気機関(きかん)
\\	このことが気に入ります (きにいります)
\\	ブーブークッション 
\\	ナメクジ カタツムリ
\\	不可欠(ふかけつ) 勤勉は人生の成功に不可欠である
\\	~も~し~も~ ~ばかりか~さえ~ このセーターは色もきれいだし、形もいいでう
\\	便利(な)(べんり) 便利さ(べんりさ) 役に立つ(やくにたつ)
\\	郵便番号 (ゆうびんばんごう)
\\	金閣寺(きんかくじ)
\\	襟(えり) 酒に酔った男が私の襟を掴んで(つかんで)汚い言葉を吐いた(はく)
\\	茄子(なす) なすび
\\	何と言いましたか
\\	しわになった本
\\	漁師(りょうし)
\\	1950年代(ねんだい) この歌は1970年代に流行した
\\	ぎっちょ (俗) 左利き(ひだりきき) あたしは、食べるときだけぎっちょになるの。ほかのときは右よ。
\\	アウト アウト!
\\	単数(たんすう) 複数(ふくすう) 単数形 この語はこの意味では単数です
\\	用事(ようじ) 幼児(ようじ)
\\	巡航(じゅんこう) 船旅(ふなたび)
\\	巡航する 巡航船
\\	安易(な)(あんい) 安易な生き方をする 安易な態度(たいど)
\\	軟骨(なんこつ)
\\	詩(し) 詩を作る
\\	手順(てじゅん) どんな手順でビザをとるのか教えてください
\\	~に/で 悩む(なやみます) ~を 心配する(しんぱい) ~を 気にする 
\\	そのことで悩むな!
\\	振り込む(ふりこみます) 振込み 自動振込み(じどうふりこみ)
\\	ネオンサイン
\\	大事(な)(だいじ) 重大(な)(じゅうだい)
\\	それは大事だ このままでは大事になる
\\	画面(がめん) スクリーン スクリーンに出る
\\	莢(さや) 莢をむく さや入り[なし]の豆
\\	沼(ぬま) 沼の多い
\\	電報(でんぽう) 電報を打つ(うつ)
\\	寒気(さむけ) 風邪を引いたので、寒気がする
\\	支払う(しはらいます) 払う(はらいます) 料金を支払う
\\	二酸化炭素(にさんかたんそ) 二酸化炭素の排出(はいしゅつ)
\\	一昨日(おととい)
\\	~に 沿って(そいます) 海岸に沿って新しい道路が出来た 鉄道はしばらく川に沿って走っていた
\\	煉瓦(れんが) 常用外
\\	ハイジャック 
\\	乗っ取る(のっとります)
\\	誘拐(ゆうかい)
\\	ハイジャック犯(はん) ハイジャックにあったらどうしますか
\\	またたく間に(まに)
\\	姫(ひめ) お姫様(おひめさま)
\\	用件(ようけん) 用意(ようい)
\\	ご用件は? その用件は明日まで待てますか
\\	ささやく 彼は彼女に何かささきました
\\	菊(きく) 菊の紋(きくのもん)
\\	合間(あいま) 仕事の合間に 読書と執筆の合間に
\\	面長の(おもなが) 卵型の(たまごがたのかお)
\\	面長の美人
\\	社説(しゃせつ) 今日の朝日新聞の社説は予算案に関するものだ
\\	~に 貢献する(こうけん) それらの人々は世界平和に大きに貢献した
\\	チョーク 白いチョークを一本取ってください
\\	貝殻(かいがら)
\\	殺す(ころします)
\\	蚤の市(のみのいち)
\\	白内障(はくないしょう) 白内障になる[なっている]
\\	助詞(じょし)
\\	平均(へいきん) 平均点(へいきんてん) 平均して一日何時間勉強する
\\	舞台負け(ぶたいまけ) 舞台負けする
\\	逃亡(とうぼう) 逃亡者(とうぼうしゃ) 逃亡中の犯人
\\	的(まと) 対象(たいしょう)
\\	その歌手は注目の的だった
\\	クロム
\\	浪費する(ろうひ) 無駄遣いする (むだづかい) 
\\	金銭を浪費する 資源を浪費する 時間を浪費する
\\	~を貸してくれませんか ちょっとペンを貸してくれませんか
\\	沈黙(ちんもく)
\\	静けさ(しずけさ) 彼の沈黙は何を意味するのか
\\	日食(にっしょく) 先月,日食があった
\\	ワックス 床にワックスをかける
\\	大勢(の)(おおぜい) 大勢の乗客が怪我をしている
\\	ぎらぎらする 
\\	光る(ひかります) 輝く(かがやきます) 彼の目は怒りでぎらぎらした
\\	シンデレラ 祖母は私にシンデレラの物語を語ってくれた
\\	風船(ふうせん) 彼女の風船に穴を空けた(あけた)
\\	軟膏(なんこう) 何に軟膏を塗る
\\	調べる(しらべます) 辞書でその単語を調べてみなさい
\\	予習(よしゅう) 金曜日の予習に第四章を読んでおくこと
\\	嘔吐(おうと)
\\	ゲロを吐く 
\\	反吐(へど)を吐く 
\\	へどを吐く(はきます) 嘔吐する 夕食に食べたものを吐く
\\	空港(くうこう) 飛行場(ひこうじょう)
\\	褒美(ほうび)を与える
\\	楽器(がっき) 学期(がっき) 
\\	天気(てんき) 晴天(せいてん) いい天気
\\	プードル
\\	共産主義(きょうさんしゅぎ) 共産主義の 共産主義者 資本主義(しほんしゅぎ)
\\	家にいて
\\	ホームステイする
\\	四捨五入する (ししゃごにゅう) 
\\	758を小数第3位で四捨五入すると45.76になる
\\	札幌(さっぽろ) 札幌は緑が多い
\\	赤面する(せきめん) 赤らむ(あからみます) 彼の言葉に私は赤面した
\\	本人(ほんにん) それは父本人の望みだ(のぞみ)
\\	随筆(ずいひつ) 随筆家
\\	豪雨(ごうう) 豪雨だった
\\	今日(きょう) 本日(ほんじつ)
\\	安定(あんてい) 日本の円は安定した通貨です
\\	衰弱(すいじゃく) 悪化(あっか) 病気で衰弱する 神経衰弱(しんけいすいじゃく)
\\	態と(わざと) わざとうそをついたのです
\\	食べなさすぎた 面白くなさ過ぎる
\\	親馬鹿(おやばか)
\\	粒(つぶ) 種(たね)
\\	薬を2粒飲んだ 1粒の米
\\	緩める(ゆるめます) 緩い(ゆるい) 負傷者の襟を緩める 彼は握った手を緩めた 車のスピードを緩める 政府は税制を緩めた
\\	~も ~でも ハリスさんはこんな易しい漢字も読めない
\\	虫歯(むしば)
\\	教育(きょういく)
\\	操作(そうさ) 車は操作が簡単なら簡単ほど運転が楽だ
\\	(妊娠)中絶 (にんしんちゅうぜつ)
\\	~を 続ける(つづけます) 柔道を続ける
\\	止める(とめます) クーラーを止める
\\	どっさり お金がどっさりある 宿題がどっさりある
\\	安心(あんしん) 安心する
\\	金魚(きんぎょ)
\\	風呂敷(ふろしき) ふろしき包み
\\	マザコン 
\\	あいつはマザコンなんだ
\\	体温計(たいおんけい) 温度計(おんどけい) 看護婦は体温計で彼の体温を計った
\\	(語尾) 
\\	かしら(女性語) かな 松本さんは来るかしら
\\	立てる(たてます) 「芝生に入たないで下さい」という札が立ててあります
\\	人並み(な)(ひとなみ) 
\\	普通(な)(ふつう) 
\\	平凡(な)(へいぼん) 
\\	平均的(へいきんてき) 
\\	私も人並みに大学に進んだ
\\	漏れる(もれます) 蛇口から水がちょろちょろ漏れている 木の間を漏れる日の光
\\	鳩(はと)
\\	喉ちんこ(のどちんこ)
\\	口蓋垂(こうがいすい)
\\	気管支炎のせいで、喉ちんこが真っ赤に晴れ上がってるんです(はれあがる)
\\	噛み切る(かみきります) 舌をかみ切る 肉が硬くてかみ切れない
\\	添加物 (てんかぶつ) 
\\	防腐剤(ぼうふざい)
\\	食品添加物 人工添加物を含まず
\\	お辞儀(おじぎ)
\\	会釈(えしゃく) この国の人々は会うとお辞儀することを習慣にしている
\\	揺りかご(ゆりかご) あの赤ん坊は揺りかごの中ですぐ眠る
\\	子供を出産する(しゅっさん) 子供を生む(うみます)
\\	電話を掛ける(かける) 電話をする
\\	高校生(こうこうせい) 高等学校の生徒 (こうとうがっこうのせいと)
\\	敷金(しききん) 二ヶ月の敷金を入れて、いただきます
\\	を 狩る(かります) 刈る(かります) 
\\	きじを狩ります
\\	実施(じっし)
\\	応用(おうよう)
\\	条約を実施する
\\	お巡り(さん)(おまわりさん) 警官(けいかん)
\\	ポンプ ポンプはちゃんと動かなかった
\\	商人(しょうにん) 
\\	証人(しょうにん)
\\	絶対に ~ない (ぜったいに) 携帯電話を絶対に使ってはいけない
\\	名所(めいしょ) 京都の名所はいつも人がおおい
\\	~を 空を飛ぶ
\\	ベテラン 
\\	エキスパート 名人(めいじん) 大家(たいか) 彼は中東問題のベテランだ
\\	塵取り(ちりとり) 塵(ちり) 埃(ほこり)
\\	失業(しつぎょう) その国ではたくさんの若い人々が失業している
\\	浅い(あさい) この池は浅くて、底(そこ)が見える
\\	~が 広がる(ひろがります) 火事は隣町(となりまち)に広がった
\\	修羅(しゅら)
\\	結合(けつごう) 二つの物を結合する これは二つの元素が結合して出来たものである
\\	枚(まい)
\\	鮮魚(せんぎょ) 新しい魚
\\	神経衰弱(しんけいすいじゃく)
\\	写す(うつす) 移す(うつす) 
\\	映す(うつす)
\\	彼は友達のノートを写すのに忙しかった
\\	水星(すいせい) 彗星(すいせい)
\\	アルコール中毒(ちゅうどく) アルコール中毒者 アル中 
\\	酔っ払い(よっぱらい)
\\	(お)年寄り(としより)
\\	華氏(かし) 華氏零下5度 摂氏100度は華氏では212度だ
\\	ピン 針(はり)
\\	荒い(あらい) 波(なみ)が荒い
\\	すると するとあなたは幸せでしょう
\\	悪天候(あくてんこう) 悪天候で飛行機が遅れた
\\	金玉(きんたま) 人の金玉を握る(にぎります)
\\	茶道(さどう) 私は茶道にも華道にも興味がある
\\	簡易(な)(かんい)
\\	簡単(かんたん)
\\	易しい(やさしい)
\\	単純(たんじゅん) 
\\	簡易な方法
\\	お釣り(おつり)
\\	おげんきですか。
\\	存在する(そんざい) 神の存在を信じる?
\\	最近(さいきん) この頃(このごろ) その鶏(にわとり)最近卵を産んでいない(うむ)
\\	基準(きじゅん) 標準(ひょうじゅん)
\\	判断の基準 この学校の入学基準はとても高い
\\	甘口の(あまくち) 甘い(あまい) 甘口がほしい 甘口のシェリー
\\	動物病院 (どうぶつびょういん)
\\	あちら様(さま) こちら様 このことについて、あちら様では何とおっしゃっていますか
\\	表現(ひょうげん) 言い回し(いいまわし)
\\	語句(ごく) 
\\	どう表現すればいいでしょうか これは表現できない
\\	どうやら... (~そうだ/らしい) どうやら道を間違えたらしいね
\\	一番上の棚です
\\	科目(かもく) 学科(がっか) どの科目が一番好きですか
\\	クローブ
\\	レタス
\\	暖房機(だんぼうき)
\\	工業専門学校  (こうぎょうせんもんがっこう)
\\	盲目(もうもく) 盲目的(もうもくてき) 盲人(もうじん) 彼らは盲目的に指導者(しどうしゃ)に従う(したがいます)
\\	殺す(ころす)
\\	クラスの仲間 学友(がくゆう)
\\	同級生(どうきゅうせい)
\\	回答(かいとう) 答え(こたえ) 返事(へんじ) 返信(へんしん)
\\	それは我々がまったく回答できない問題です
\\	ゴキブリ
\\	腿(もも)常用漢字以外 桃(もも)
\\	日曜大工(にちようだいく) 日曜大工が趣味なんです
\\	宿無し(の)(やどなし) 家のない ホームレス(の) 路上生活者(ろじょうせいかつしゃ) 村から村へ流れてゆく宿無し 戦争で多くの人が宿無しになった
\\	~が大好きです(だいすき) ~何かとても好きです
\\	ベジタリアン
\\	修士論文(しゅうしろんぶん)
\\	目次(もくじ)
\\	~から・に 借りる(かります)
\\	上司(じょうし) あなたは上司とうまくやっていけていますか
\\	この電車の行き方はどうですか
\\	半球(はんきゅう) 北半球(きたはんきゅう) 南半球(みなみはんきゅう)
\\	恨み(うらみ) 恨みを抱く(いだく) 君に恨みはないよ
\\	入社する(にゅうしゃ) 入社したばかりなのに毎日とても忙しいです
\\	優等生(ゆうとうせい) 模範生(もはんせい)
\\	特待生(とくたいせい)
\\	優等生に選ばれる
\\	チョッキ 上着(うわぎ)
\\	防弾チョッキ(ぼうだん)
\\	退ける(どけます) その間は親友をも退けた 彼を委員長の座から退けようとした
\\	コルク栓(せん) 瓶の栓をする/抜く(ぬきます)
\\	腹話術氏 (ふくわじゅつし)
\\	クリーニング屋(や) クリーニングに出す
\\	希望(きぼう) 望み(のぞみ)
\\	私の希望は医者になることです 彼は日本留学を希望している
\\	匹(ひき)
\\	仮想する(かそう) 火葬場(かそうば)
\\	(お)葬式(そうしき) 彼の葬式に出るつもりです
\\	洞窟(どうくつ) その洞窟には怪物(かいぶつ)が住んでると信じられている
\\	小包(こづつみ)
\\	大胆(な)(だいたん) そんな事を言うとは君は大胆だ
\\	インク 墨(すみ) 青インク(あお) 黒インク(くろ)
\\	腕白(な)(わんぱく) いたずら(な) 腕白(わんぱく) 悪者(わるもの) 
\\	うちのわんぱくもの
\\	屑(くず) ごみ 
\\	鉄くず 野菜のくず 社会のくず
\\	ハムスター ハムスターを飼う(かいます)
\\	保護者(ほごしゃ) 父兄(ふけい)
\\	ジャーナリスト (新聞)記者(きしゃ)
\\	偉大(な)(いだい) 偉い(えらい) 偉大な計画 彼は多くの点で偉大です
\\	きつい 
\\	窮屈(な)(きゅうくつ) 靴がきつい ベルトをきつくする
\\	スイマー 泳者(えいしゃ) スイマーは海で泳いでいる
\\	党(とう) 政党(せいとう) 党を組織する(そうしき) 党に加入する
\\	歴史(れきし)
\\	誤訳(ごやく) 誤訳の箇所(かしょ) 誤訳がない
\\	逃す(のがします) チャンスを逃すな!
\\	面倒(な)(めんどう) 面倒くさい(めんどうくさい) 面倒な頼み(たのみ)
\\	胡桃割り(くるみわり) 『くるみ割り人形』
\\	連立(れんりつ) 二党が連立して政権(せいけん)を取った 連立政権
\\	和食(わしょく)   洋食(ようしょく)
\\	黄色ブドウ球菌 (おうしょくブドウきゅうきん) ブドウ球菌 球菌 (きゅうきん)
\\	早口言葉(はやくちことば)
\\	小麦粉(こむぎこ) 小麦(こむぎ)
\\	重量級(じゅうりょうきゅう) 重量級の選手
\\	それで だから きのうは風をひきました。それで学校をやすんだんです
\\	~に~を 貸す(かす)
\\	後者(こうしゃ) 前者(ぜんしゃ)
\\	本田 
\\	降りる(おります) 本田駅から12番のバスに乗って、図書館前で降りてください。三つ目です
\\	住民(じゅうみん) 住人(じゅうにん)
\\	その地方の住民はほとんどが移民だ
\\	不動産(ふどうさん) 彼は不動産を子供達に譲った(ゆずる) 不動産屋(ふどうさんや)
\\	水泳(すいえい)
\\	知ったかぶり(をする人) 彼は知ったかぶりをして「もちろんさ」と言った
\\	気分(きぶん) 機嫌(きげん)
\\	~に 重視する(じゅうし) ~に 重点を置く(じゅうてん)
\\	この教科書は文法を重視するあまり
\\	お風呂に入る(はいります)
\\	食後(しょくご) 食後にサバンナ行ったとき 食後は何にしましょうか
\\	差別(さべつ)
\\	目が上がっている 目が下がっている
\\	クロスワード 彼はそのクラスワードパズルを簡単に解いた
\\	よだれ 
\\	唾(つば)
\\	唾液(だえき) よだれを垂らす(たらします) よだれが多い子
\\	矢(や) 矢を放つ/飛ばす 矢は的に当たった[を外れた]
\\	図表(ずひょう) グラフ
\\	消滅(しょうめつ) その種は自然消滅した
\\	反日の(はんにち) 高まる反日感情
\\	暴行(ぼうこう) 膀胱(ぼうこう)
\\	性的暴行(せいてきぼうこう) 強姦(ごうかん)
\\	(電気) 
\\	半導体(はんどうたい) 導体(どうたい)
\\	海水浴(かいすいよく) 海で泳ぐこと 海水浴中に彼は友達に会った 海水浴に行く
\\	象牙海岸(ぞうげかいがん)
\\	村人(むらびと) 村人はあの世があると思っていた
\\	蜘蛛(くも)
\\	雲(くも) 
\\	蜘蛛はいつも自分たちの巣を三次元に織る(おります)
\\	旅券審査(りょけんしんさ) パスポート審査 今、旅券なしで欧州の国境を越えられる
\\	ナツメヤシの実(み) デーツ
\\	友達になる(なります) 日本人と友達になった
\\	サラダ
\\	作り替える(つくりかえます) 元々は外国の料理を日本人が作り替えてきたものです
\\	毒きのこ(どく) これは毒きのこだから、食べてはいけない
\\	元(もと) 
\\	元首相 (もとしゅしょう) 元の同僚
\\	髪型(かみがた) それはとてもよく似合う髪形だ
\\	尖る(とがります)
\\	怒る(おこります) 彼はちょっとしたことにでもすぐとがる
\\	耽る(ふけります) 没頭する(ぼっとう) 彼は快楽にふけっている 彼女はアメリカ文学の勉強にふけっている
\\	巡礼(じゅんれい) 巡礼者(じゅんれいしゃ) 巡礼する
\\	脱ぐ(ぬぎます) 靴を脱ぐ
\\	事務所(じむしょ) オフィス
\\	勤勉(な)(きんべん) 一般に日本人は勤勉だ 繁栄は勤勉にかかっている
\\	命名する(めいめい) 名づける(なづけます) 彼は自分の父親の名前にちなんで息子にジョンと命名した/名づけた
\\	ローン もはやローンは必要ではない
\\	記入する(きにゅう) 書き込む(かきこむ) このカードに記入していただけませんか
\\	見える(みえます) 窓から川が見える
\\	寄生虫(きせいちゅう) 
\\	彼はまるで寄生虫だ
\\	和式(わしき) 日本風   和式のトイレ
\\	銀色の(ぎんいろ)
\\	泊まる(とまります) 宿泊する(しゅくはく) 旅館に泊まります
\\	馬に乗る(のります) 乗馬する(じょうば)
\\	膝(ひざ) 常用外
\\	優勝(ゆうしょう)
\\	勝利(しょうり)
\\	2チームが優勝を争っている
\\	能率(のうりつ) 
\\	能率的・能率のよい/悪い 
\\	全部まとめて 全部まとめて送ったほうがよい
\\	飾り(かざり)
\\	極度の(きょくど) 極度に興奮(こうふん)している 彼の怒りは極度に達した
\\	繭(まゆ) 眉(まゆ)
\\	蚕は繭を作る
\\	幼虫(ようちゅう) せみの幼虫
\\	鮮やか(な)(あざやか)
\\	活発(かっぱつ)
\\	生き生き(いきいき)
\\	その婦人の美しい姿が私に鮮やかな印象を与えた 鮮やかな木の緑
\\	意義(いぎ) 意味(いみ)
\\	私はしばしば人生の意義について深く考えることがある
\\	寿命(じゅみょう) 彼の政治寿命はもうない
\\	途中で(とちゅう) 私は家に帰る途中で嵐にあった
\\	~による 今日は飲みにいこうか? それは、裕子によるね
\\	福岡(ふくおか) 福岡に近い
\\	万里の長城 (ばんりのちょうじょう)
\\	足が痛くなった
\\	視力(しりょく)
\\	視覚(しかく)
\\	視力が鈍くなった 年をとって視力が衰えてきた
\\	滑る(すべります) 丘を滑り降りる 大学を滑った
\\	伸す(のします) ~に アイロンを掛ける
\\	ケイ素(けいそ) シリコンバレー
\\	地球儀(ちきゅうぎ)
\\	政界(せいかい) 政界の事情に通じている 政界を去る
\\	楽しみ(たのしみ) 快楽(かいらく) 喜び(よろこび) お金のために、本を書く人もいれば、楽しみで書く人もいる
\\	困る(こまります) 困っている子供
\\	馬肉(ばにく)
\\	オールドミス
\\	インフルェンザ
\\	抱き締める(だきしめます)
\\	抱く(い)だきます) 私は赤ん坊を抱き締めた
\\	吸引する(きゅういん) 
\\	吸い込む(すいこみます)
\\	居眠りする(いねむり) 昨夜私はテレビの前で居眠りをしてしまった
\\	点数(てんすう) 成績(せいせき)
\\	彼女は英語の点数がよい
\\	咽頭(いんとう)
\\	咽喉(いんこう)
\\	喉(のど) 
\\	(お)見舞い(みまい) 怪我人を見舞いに行った 病人に見舞いの言葉を述べる
\\	贅沢(な)(ぜいたく) タクシーを乗るのは私には贅沢だ
\\	好評(こうひょう) 彼の小説は好評だった 彼の講演は学生に好評であった
\\	脱税(だつぜい) 脱税する
\\	楽勝(らくしょう) 楽勝だよ 彼はその競争に楽勝した
\\	新年の誓い(ちかい) 新年の誓いを立てた
\\	捕まえる(つかまえます)
\\	本来(ほんらい) これは本来無害(むがい)です この詩は本来フランス語で書かれていた
\\	破壊する(はかい)
\\	壊す(こわします)
\\	彼の夢は完全に破壊された 町を破壊する
\\	見掛け(みかけ) 彼女は見掛けほど実際に若くない
\\	銀行でお金を出した
\\	サークル サークルに入る
\\	花火(はなび) 花火を上げる
\\	チアガール
\\	銀行口座(ぎんこうこうざ) ~を閉じる(とじます)開く(ひらく)
\\	試験に受かる(うかります) 試験に合格する(ごうかく) 試験を受ける
\\	湯飲み(ゆのみ)
\\	ティーカップ 
\\	捕食者(ほしょくしゃ) 捕食動物(ほしょくどうぶつ)
\\	車に乗る(のります) 車に乗車する(じょうしゃ) 下車(げしゃ)
\\	争い好き(な) (あらそいずき) けんか早い (けんかばやい)
\\	感心する(かんしん) 関心(かんしん)
\\	我々は彼女の勇気に感心する
\\	和室(わしつ) 座敷(ざしき)
\\	厳か(な)(おごそか) 葬式は厳かに行われた
\\	陳列する(ちんれつ) 展覧会(てんらんかい) その陶磁器は特別な棚に陳列されている
\\	おせわになりました。
\\	保険(ほけん) 保険証(ほけんしょう)
\\	~く/で 見える このスニーカーは新しく見える
\\	~に 座る(すわります)
\\	どきっとする 驚く(おどろきます) どきっとするような美人 どっきとさせる/驚かす
\\	早口(はやくち) 早口で話す 彼女はあまり早口なのでよく聞き取れない 早口言葉
\\	パイナップル
\\	~と申します(もうす) 池田和子と申します
\\	ドアのロック
\\	奴隷(どれい) 奴隷を解放する/自由にする
\\	グレー(の) 鼠色(の)(ねずみいろ) 灰色(の)(はいいろ) 鼠(ねずみ)
\\	灰(はい)
\\	時限爆弾 (じげんばくだん)
\\	~というほどでもないけど 悩みというほどでもないけど、ちょっと困っていることがあります
\\	嬉しい(うれしい) 嬉しいことに娘が試験に合格した
\\	~を フォーマットする
\\	昔は昔, 今は今だ
\\	散発する(さんぱつ) 髪を刈る(かります) 散髪してもらう 人の散髪をしてやる
\\	閉鎖的(な)(へいさてき) これまで日本は外国に対して閉鎖的な政策を取ってきた 彼の性格は閉鎖的である
\\	たまに 私はたまに、タクシーで大学へ行くことがあります
\\	合言葉(あいことば) パスワード 
\\	合い言葉を言わないと通さない
\\	前者(ぜんしゃ) 後者(こうしゃ) これら二つの意見のうち前者より後者のほうがいい
\\	だから ですから 暇です。だから散歩します
\\	逸話(いつわ) 彼女の逸話は誰にとってもうれしい
\\	エンジン 新しいエンジンの調子はどうですか
\\	延びる(のびます) 申し込みの期間が延びた 山並みは南北に延びていた
\\	弾丸列車(だんがんれっしゃ) 
\\	特に(とくに) とりわけ 特にケンタウロスはよかったですね
\\	またぐ(またぎます) 超える(こえます) 垣根をまたぐ 敷居をまたぐ
\\	乳製品(にゅうせいひん)
\\	寒い(さむい)
\\	~を 問わず(とわず) 年齢を問わず人々はこの歌が好きだ
\\	気配(けはい) 様子(ようす) その事について話し合った気配がなかった 物価が上がりそうな気配が濃い
\\	爆弾(ばくだん)   私は爆弾が爆発(ばくはつ)するのを聞いた
\\	伯母/叔母(さん)(おば(さん)) 叔母(父の妹)/伯母(父の姉)
\\	サウジアラビア
\\	山羊(やぎ)
\\	競技(きょうぎ)
\\	競争(きょうそう)
\\	水泳競技
\\	拍手する(はくしゅ) 手を叩く(たたく)
\\	アナウンサー
\\	何してんだい?
\\	摩天楼(まてんろう)
\\	高層ビル(こうそう) スカイスクレーパー 高層(こうそう) 
\\	厚かましい(あつかましい) 生意気(な)(なまいき)
\\	君の厚かましいのには呆れたよ(あきれる) 彼は厚かましくも僕の助言を無視した
\\	蛾(が) それは蝶ですか、それとも蛾ですか
\\	脈(みゃく) 医者は私の脈を取った
\\	水道屋(すいどうや)
\\	作家(さっか) 著者(ちょしゃ) 小説家(しょうせつか) この本の著者はだれですか
\\	車掌(さん)(しゃしょう) 電車に乗っているとき、車掌が切符を調べに来た
\\	税制(ぜいせい) 税金制度(ぜいきせいど)
\\	暴走族(ぼうそうぞく) 暴走する(ぼうそう)
\\	露天風呂(ろてんぶろ)
\\	ローン 住宅ローン(じゅうたく)
\\	~なら みんなが行くなら私も行く
\\	故障(こしょう) 湖沼(こしょう)
\\	車は故障した
\\	職業的(な)(しょくぎょうてき) プロ
\\	職業軍人(しょくぎょうぐんじん)
\\	一切れ(ひときれ) チーズの一切れ
\\	いろいろお願いした
\\	教授(きょうじゅ)
\\	古本屋 (ふるほんや)
\\	検査する(けんさ) 知能検査 この製品は検査に合格した
\\	年号(ねんごう) 年号を改める(あらためます)
\\	自転車(じてんしゃ) ちゃりこ ちゃり 
\\	自転車の空気入れ ここ数年自転車に乗っていない.
\\	感覚(かんかく) 指に感覚が無かった
\\	虫垂(ちゅうすい) 盲腸(もうちょう)
\\	虫垂炎(ちゅうすいえん) 盲腸を取ってもらう
\\	場面(ばめん)
\\	掲げる(かかげる) 旗を掲げる
\\	上がっている(あがっている) 目があがっている
\\	酢(す) 酢は酸っぱい味がする
\\	出血する(しゅっけつ) 血が出る 彼はひどく出血している
\\	そよ風(そよかぜ) そよ風が吹いている
\\	一種(いっしゅ) 鯨は哺乳動物の一種です
\\	通訳者(つうやくしゃ)
\\	滴る(したたります) 彼の体からは汗がしたたり落ちていた
\\	安全(な)(あんぜん)
\\	(俗) 
\\	童貞(どうてい) 童貞を守る[失う] 童貞になる
\\	ピンク(の) 桃色(の)(ももいろ)
\\	動作(どうさ) 身振り(みぶり)
\\	彼は動作が鈍い
\\	信じられないことだった
\\	いただきます
\\	固有名詞(こゆうめいし)
\\	付ける(つけます) 醤油を付ける
\\	樹立する(じゅりつ)
\\	確立する(かくりつ)
\\	作る(つくります) 政府の樹立 名声を確立する
\\	~度に(たびに) 彼に会う度に
\\	避ける(さけます) 私たちは危険を避けようとした
\\	船旅(ふなたび) 船旅はとても楽しい
\\	喘息(ぜんそく) 喘息患者
\\	昆虫(こんちゅう) 虫(むし)
\\	昆虫類(こんちゅうるい)
\\	医療(いりょう) 医療を受けている 彼は肺炎で医療を受けた
\\	必死に(ひっし) 必死(ひっし)
\\	必死の努力をする 彼らは必死に戦った
\\	舐める(なめます) キャンデーを舐める 切手の裏をなめる
\\	難破(なんぱ) 船は岩にぶつかって難破した
\\	記念日(きねんび)
\\	槍(やり) 槍術(そうじゅつ)
\\	疑問符(ぎもんふ)
\\	棚卸(たなおろし) 棚卸をする
\\	素子宮(しきゅう)(医)
\\	論じる(ろんじります) 討論する(とうろん) 政治を論じる こんな本, 論じる価値もない(かち)
\\	忘れっぽい(わすれっぽい) 彼は非常に忘れっぽい男だ
\\	陽気(ようき) 
\\	楽しい(たのしい) 明るい(あかるい)
\\	子供たちは陽気に遊んでいた 陽気な音楽
\\	ブルドーザー
\\	安楽(な)(あんらく)
\\	気楽(きらく)(な)
\\	安楽な暮らしをする
\\	とかげ
\\	盛る(もります) 盛り上げる(もりあげます) 盆にオレンジを盛る 毒を盛る
\\	青空(あおぞら)
\\	チタン
\\	誕生日です(たんじょうび)
\\	文学者(ぶんがくしゃ) 文学者はボーベル賞をもらった
\\	暫く(しばらく) 暫くすると 我々は暫くの間話し合った
\\	測定(そくてい) 速度を測定する方法はいくつがある
\\	朗らか(な)(ほがらか) 明るい(あかるい) 陽気(な)(ようき) 朗らかな人 朗らかな気分になる 朗らかな声で
\\	組織(そしき) 
\\	機構(きこう)
\\	ありには非常に組織だった社会がある
\\	スープ皿(ざら)
\\	努力する(どりょく) 彼は問題を解くために努力した
\\	タイプライター タイプライターを打つ
\\	水中に(すいちゅう) 水中が泳げますか 少年は水中から助け上げられた
\\	注射器(ちゅうしゃき)
\\	合意(ごうい) 停戦に合意する
\\	比べる(くらべます) 比較する(ひかく) 日本は中国に/と比べて島国です。
\\	百科事典(ひゃっかじてん)
\\	境界(きょうかい)
\\	境(さかい)
\\	国境(こっきょう)
\\	協会(きょうかい) 
\\	教会(きょうかい)
\\	町の境界 境界を画する(かくする)
\\	地方(ちほう) 地帯(ちたい)
\\	地域(ちいき)
\\	地方へ行くと、空気がきれいです ルール地方
\\	動物学(どうぶつがく) 動物
\\	反乱(はんらん)
\\	暴動(ぼうどう)
\\	その軍隊(ぐんたい)は簡単に反乱を鎮圧した(ちんあつ)
\\	消防署(しょうぼうしょ)
\\	雨の日(あめのひ) 雨降り(の日) (あめふり)
\\	おまちどうさま。
\\	応じる(おうじります) 
\\	彼は彼らの招待に応じなかった
\\	呼吸器系(こきゅうきけい) 呼吸器(こきゅうき) 呼吸する(こきゅう)
\\	息をする(いき) 
\\	本州(ほんしゅう) 本州の方が沖縄より大きいです。
\\	気持ち(きもち) 気持ちがいい
\\	(ほうきで)掃く(はきます) 履く(はきます) 
\\	吐く(はきます) 
\\	床を掃く
\\	ローマ法王(ほうおう)
\\	担当する(たんとう) 君はこの部屋の赤ちゃんを担当してもらいます
\\	どのくらい 
\\	どのぐらい!!! 「日本に来てどのくらいになりますか」 「かれこれ10年です」
\\	撫でる(なでます) 犬を撫でる
\\	ベーコン ベーコンかソーセージはいかがか
\\	何でも(なんでも)
\\	息子(むすこ)
\\	峡谷(きょうこく) キャニオン グランド・キャニオン
\\	ヘリコプター
\\	頬(ほほ) 常用外
\\	鉛筆(えんぴつ)
\\	伝言板(でんごんばん) 伝言(でんごん)
\\	腕立て伏せ (うでたてふせ) 毎朝腕立て伏せを 
\\	回する
\\	花屋(はなや) あの花屋の前に止めてください
\\	冬(ふゆ)
\\	スープ スープを作る
\\	俗語(ぞくご)
\\	口語(こうご)
\\	俗語を使う 口語と俗語の違いは何ですか?
\\	冒険(ぼうけん) 冒険家(ぼうけんか)
\\	~を 広げる(ひろげます) 鳥が翼(つばさ)を広げた 包みを広げる 広める(ひろめ)
\\	誰かを からかう 誰かに 馬鹿にする 
\\	猫をからかう
\\	句(く) 区(く)
\\	語句(ごく)
\\	表現(ひょうげん)
\\	言い回し(いいまわし)
\\	この句はどう意味ですか 句動詞
\\	状態(じょうたい) 情勢(じょうせい) 
\\	状況(じょうきょう) あなたの模型はいい状態で受け取りました けがの状態
\\	五十音(ごじゅうおん) 五十音図(ごじゅうおんず)
\\	何ヶ所(なんかしょ) 何ヶ所が見たいですか
\\	日程(にってい) 日の予定 スケジュール きつい日程 今日は何か日程があるか
\\	ワイパー
\\	化学の勉強
\\	娘(むすめ) 娘さん(むすめさん)
\\	得意(とくい) 山田さん先日の試験の成績がよかったので、得意になっている 私は物理学(ぶつり)より生物学のほうが得意です
\\	愛想(あいそう) 愛想のよい 愛想よく話しかけてきた
\\	ああ   ああいう人は見たことがない
\\	六法(ろっぽう)
\\	分析(ぶんせき) 我々は事故の原意を詳しく分析しなければならない
\\	事故(じこ) 自己(じこ)
\\	その事故で交通は止まった
\\	批判する(ひはん) 一言批判されただけで彼女はびくびくする
\\	予知能力者(よちのうりょくしゃ)
\\	冬眠(とうみん) 冬眠中の熊
\\	廃車にする(はい-しゃ) 歯医者(は-い-しゃ)
\\	私はぶつかって廃車にしちゃったんですよ
\\	(胸が)むかつく 腹が立つ(はらがたつ) 彼のひとりよがりを見たらむかついてしまった 彼の傲慢(ごうまん)な態度を見ると胸がむかつく
\\	水差し(みずさし)
\\	安楽死(あんらくし) 安楽(な)(あんらく)
\\	緑茶(りょくちゃ)
\\	間違って(まちがって)
\\	ユーフォー ユー・エフ・オー 未確認飛行物体 (みかくにんひこうぶったい) 
\\	直前に(ちょくぜん) 沸騰直前の湯(ふっとうちょくぜんのゆ) 試験の直前に 直後に(ちょくご)
\\	ピエロ 道化師(どうけし)
\\	ピエロはわざと転んだ(ころんだ)
\\	入学する(にゅうがく) 彼はこの春東京大学に入学した
\\	孤立語(こりつご)
\\	殺人犯(さつじんはん)/殺人者 殺害者(さつがいしゃ)
\\	思い出(おもいで) 子供のころの思い出 にふける
\\	真っ二つ(まっぷたつ)に割れる(われます)
\\	半分に分ける
\\	すいかを真っ二つに割った 医学界は真っ二つに割れた
\\	登る(のぼります) 登山する(とざん) 山に登る
\\	折れる(おれます) 手の骨が折れている
\\	探検版(たんけんばん)
\\	別品(べっぴん)
\\	美人(びじん)
\\	プロ・スポーツ 江戸時代に相撲はプロ・スポーツになった
\\	立ち寄る(たちよります) 
\\	昨日兄が私の家に立ち寄った
\\	外出(がいしゅつ) 出掛ける 外出嫌いの人 彼は外出中だ 彼の外出先を知っていますか (がいしゅつさき)
\\	可愛い(かわいい)
\\	野球場(やきゅうじょう)
\\	性格(せいかく) 
\\	正確(な)(せいかく) 
\\	人格(じんかく) 
\\	人柄(ひとがら) 
\\	ジョンは性格が陽気だ
\\	勝者(しょうしゃ) 
\\	勝利者(しょうりしゃ)
\\	喃語(なんご) 
\\	知能(ちのう) この子の知能は平均以上だ
\\	気球(ききゅう) 風船(ふうせん) 気球に乗る 気球が空に浮かんでいた
\\	ゼラチン
\\	強調する(きょうちょう) 18世紀の流行は胸の部分を強調していた
\\	料理が上手 一郎さんの母は美人だし、料理も上手です
\\	避雷針(ひらいしん) 羅針盤(らしんばん)
\\	手話(しゅわ) 手話法(しゅわほう) 手話 (法) で話す 手話通訳者
\\	ぺちゃんこ(の) 平ら(な)(たいら) ぺちゃんこの鼻 自転車のタイヤがぺちゃんこだ. 空気を入れなきゃ 地震で家がぺちゃんこになった
\\	近道(ちかみち) 学校への近道だ
\\	内側(うちがわ)
\\	室内(しつない)
\\	窓の内側に男の頭を見えた 門には内側から錠がおりていた 内側のポケット
\\	身振り(みぶり) ジェスチャー ジェスチャーは怒った(おこった)身振りをした
\\	恩恵(おんけい) 恩(おん)
\\	河馬(かば) 河馬の二頭
\\	劇場(げきじょう)
\\	マイクロホン マイク 盗聴器(とうちょうき)
\\	まあまあ(の) まあまあの成績(せいせき)
\\	暦(こよみ) カレンダー
\\	背びれ(せびれ) 尾びれ(おびれ)
\\	井の中の蛙(大界を知らず) いのなかのかわず(たいかいをしらず)
\\	トイレットペーパー ちり紙(ちりがみ) 
\\	(俗) 
\\	どんだけ...(俗) なんて
\\	道化師(どうけし)
\\	財産(ざいさん) 私有財産 父は株で財産を失った
\\	いつも通り(とおり) 私はいつも通り手を洗った
\\	寸法(すんぽう) スーツを作る時は正確に寸法を測る(はかる)必要がある
\\	自国(じこく) 時刻(じこく)
\\	母国(ぼこく)
\\	当局は自国の通貨を安定した
\\	刑務所(けいむしょ)
\\	探検(たんけん) 探検家(たんけんか) 彼はその無人島の探検を希望している
\\	毛虫(けむし)
\\	芋虫(いもむし)
\\	陰核(いんかく) クリトリス
\\	キャベツ
\\	~と 絶交する(ぜっこう) 
\\	絶好(の)(ぜっこう) 彼とはとっくに絶交した 君とはもう絶交だ
\\	唾(つば) 唾液(だえき) よだれ 唾を吐く(はく)
\\	切開 (せっかい) 切開する 切開手術
\\	羨ましい(うらやましい)
\\	嫉妬深い(しっとぶかい)
\\	私はあの人の成功がうらやましいです
\\	大豆(だいず)
\\	~てしまう 
\\	~きる 本を読んでしまった/読みきった
\\	~だらけ ~まみれ 泥だらけの靴 部屋はごみだらけだ
\\	履修する(りしゅう) 去年は何科目を履修しましたか
\\	ゴリラ
\\	降参(こうさん) 降伏(こうふく)
\\	ぜったい降参しない
\\	こけし(人形)
\\	長年の(ながねん) 長い間(ながいあいだ)
\\	長年の習慣(しゅうかん)
\\	(お)尻(しり)
\\	生け花(いけばな) 華道(かどう) 花を生ける(いける)
\\	毛を抜く(けをぬきます) 脱毛する(だつもう)
\\	狼煙(のろし) のろしを上げる
\\	開花(かいか) 咲く(さきます) 
\\	3日すれば開花するでしょう
\\	恐怖(きょうふ) 彼女は恐怖で失神した
\\	相違(そうい) 違い(ちがい) 二つの単語の間に意味の相違がない
\\	狒々(ひひ)
\\	狒々おやじ
\\	通信(つうしん)
\\	程度(ていど) 程度の低い話題 それはある程度本当だ
\\	惑星(わくせい) 
\\	ダークホース 政界の惑星
\\	海軍(かいぐん) 父は海軍にいました
\\	昼ご飯(ひるごはん) 
\\	昼食(ちゅうしょく) 
\\	お昼(おひる)
\\	役立つ(やくだちます) スポーツは筋肉の発達に役立つ
\\	頻繁(な)(ひんぱん) そこで台風が秋に頻繁に起こっている 頻繁に よく しばしば 度々(たびたび)
\\	引っ張る(ひっぱります) 車は故障したので、綱をつけて引っ張った このひもを強く引っ張って下さい
\\	仕事をやる
\\	燃える(もえます) 多くの家は燃えていた
\\	国際(こくさい)
\\	共存(きょうぞん・きょうそん) 平和共存 共存共栄 (きょうそんきょうえい)
\\	こいつ こいつは便利だ
\\	是非(ぜひ) 
\\	~たい ぜひあなたと踊りたいです
\\	はしご 私ははしごを使って屋根に登った
\\	出勤する(しゅっきん) 彼は雨の日以外は毎日徒歩で出勤する 1時間遅れて出勤した
\\	ほとんど 
\\	大抵(たいてい) 
\\	大半は/が(たいはん)
\\	エジプト
\\	一回(いっかい) 一度(いちど)
\\	教科書(きょうかしょ)
\\	答える(こたえます) 質問に答える
\\	人差し指(ひとさしゆび)
\\	認める(みとめます) 
\\	暗やみに二つの人影が認められた
\\	仏教(ぶっきょう)
\\	寝具(しんぐ) 寝具をたたむ
\\	安売り(やすうり) 私はこのセーターを安売りで買いました
\\	~に 関心がある(かんしん) 感心する(かんしん)
\\	~に 興味がある(きょうみ)
\\	ジャックは絵に関心がある
\\	ふるい 古い(ふるい)
\\	ふるいにかける
\\	年月(としつき) 期間(きかん) 10年というは、振り返ってみると(ふりかえる)短い年月 20年とは長い年月だ
\\	結核(けっかく)
\\	割る(わります) 砕く(くだきます)
\\	窓ガラスを割ってしまった
\\	駒(こま)
\\	言いたいことは 私が言いたいは、日本人はポーランド語が下手です
\\	またいらっしゃってください
\\	びっくりする 驚く(おどろきます) びくびくする 
\\	私のテストは100点でした。びっくりしました
\\	児童(じどう) 自動(じどう)
\\	小学校の児童
\\	蝗(いなご) 飛蝗(ばった)
\\	嘘吐き(うそつき) 
\\	うそつけ 
\\	恥じる(はじます) 私は彼女に顔を見せるのが恥じた
\\	傷跡(きずあと) 傷跡ははっきり残っている
\\	カフス 
\\	カフスボタン 
\\	豚肉(ぶたにく)
\\	農家(のうか) 
\\	百姓(ひゃくしょう) 
\\	私は農家の出だ 農家は秋に忙しい 彼の両親はお百姓でした
\\	振り返る(ふりかえます) 
\\	回想する(かいそう)
\\	思い出す(おもいだす) 過去の幸福な日々を振り返って涙を流した
\\	色彩豊か(しきさいゆたか) 雄孔雀の羽が色彩豊かだ
\\	あっち行け!
\\	削岩機(さくがんき)
\\	上着(うわぎ)
\\	~を 変更する(へんこう) ~を 変える(かえます)
\\	暗証番号を変更する
\\	駅員(えきいん)
\\	チーズ
\\	潜る(もぐります) 
\\	カエルは冬の間土の中に潜って過ごします
\\	聞く/訊く(きく) 質問する(しつもん) 問う(といます) 
\\	腕前(うでまえ)
\\	能力(のうりょく) 
\\	技術(ぎじゅつ) 
\\	彼の運転の腕前は大したものだ 彼は十分に腕前を発揮していない
\\	奨学金(しょうがくきん) 彼は奨学金をもらって勉強した
\\	それでも 彼女は優しいですけど、それでも彼女が好きになれない
\\	言い付ける(いいつけます) 
\\	母は私に部屋を掃除を言いつけた
\\	鮪(まぐろ) 
\\	枕(まくら) 
\\	土竜(もぐら)
\\	無数(むすう) 宇宙には無数の星がある
\\	~を 慰める(なぐさめます) 悲しい時に人を慰める 慰め合う(なぐさめあいます)
\\	中華民国(ちゅうかみんこく) 台湾(たいわん)
\\	粋(な)(いき) スマートな 
\\	いきな人 黒い服でかえっていきに見えた
\\	手作り(てづくり) 
\\	手製の(てせい)
\\	これらの手作りの品は品質が違ってる
\\	絶好(の)(ぜっこう) 
\\	~と 絶交する(ぜっこう) 理想的(りそうてき) 完璧(かんぺき) 今が絶好の時だ 絶好のコンディションで試験に臨んだ 今日は絶好のマラソン日和だ
\\	貧しい(まずしい) 貧乏(な)(びんぼう) 貧しい家
\\	もっと早く
\\	言い付ける(いいつけます) 
\\	パパに言いつけるよ
\\	斧(おの) 木こりは木を斧で切り倒す
\\	嗄れ声(しゃがれごえ) しゃがれる(しゃがれます) 彼の声はしゃがれていた
\\	塩分(えんぶん) 塩分の高い食事をとると、高血圧の原因になるかもしれない 分を含んだ水
\\	それとも 刺身を食べますか。 それともすき焼きにしますか 寿司?それとも、すき焼き?
\\	襲撃する(しゅうげき) 敵を襲撃する 都市を襲撃する
\\	パパ
\\	頭脳(ずのう)
\\	知能(ちのう)
\\	人工頭脳 彼は頭脳的プレーを見せた
\\	味わう(あじわいます) 美しい景色を味わう
\\	直腸(ちょくちょう)
\\	リアウインドー フロントガラス
\\	げっぷする
\\	名字(みょうじ)
\\	次々(つぎつぎ)
\\	キャッシュカード
\\	広告板(こうこくばん) 掲示板(けいじばん)
\\	広告する 掲示
\\	えび
\\	やれやれ(と思う) やれやれ, やっと着いた やれやれ, これで助かった
\\	カプチーノ
\\	アルバム
\\	雑費(ざっぴ)
\\	じゃが芋(じゃがいも) 
\\	ジャガイモ
\\	つながり 心のつながり 血のつながり
\\	前髪(まえがみ) 前髪が短い
\\	生計(せいけい) 生計を立てる 生計の道を失った 教師として生計を立てる 
\\	~で生計を立てる) 生計が苦しい/豊かである
\\	積む(つみます) 棚の上に本を積んだ
\\	里心(さとごころ)
\\	ホームシック 懐かしむ(なつかしむ)
\\	里心がつく この子犬は里心がついて泣いているんだ
\\	~でも ~でもない 彼の考え方は具体的でも中傷的(ちゅうしょうてき)でもない
\\	通勤する(つうきん) 仕事に通う(かよいます) 通勤手当(つうきんてあて)
\\	謎(なぞ) その謎を解いてみた
\\	順番(じゅんばん) 私たち順番にその本を読んだ
\\	期間(きかん)
\\	ファイト ファイトだ
\\	苦しい(くるしい) つらい 苦しい仕事 良薬(りょうやく)は口に苦し(にがし)
\\	漕ぐ(こぎます) 自転車を漕ぐ
\\	歌う(うたう) 歌(うた)
\\	みんなで みんなで写真を撮った
\\	ゴールキーパー ゴールを決める
\\	吸血鬼(きゅうけつき)
\\	しゃぶる(しゃぶります) 
\\	赤ん坊が指をしゃぶっている あめをしゃぶる
\\	形容詞(けいようし) イ形容詞 ナ形容詞
\\	~式(しき) ~風(ふう)
\\	壊す(こわします) おもちゃを壊す
\\	妹(いもうと)
\\	黄色い(きいろい) 黄色(きいろ)
\\	入れ墨(いれずみ) 彼は左腕にいかりの入れ墨をしている
\\	修道士(しゅうどうし) モンク 文句(もんく)
\\	尻尾(しっぽ)
\\	前期(ぜんき)
\\	ポケット辞典(じてん) 携帯用の辞書 (けいたいようのじしょ)
\\	クラス 教室(きょうしつ)
\\	折り目(おりめ) 折り目がつく/折り目をつける
\\	結晶(けっしょう) その子は彼らの愛の結晶だった
\\	四分の三(よんぶんのさん)
\\	いびき いびきをかく
\\	乾杯!(かんぱい)
\\	成人(せいじん)  聖人(せいじん)
\\	大人(おとな) 成人した息子
\\	近づく(ちかづきます) 接近する(せっきん) あの犬には近づくな!
\\	代名詞(だいめいし) 
\\	人称代名詞(にんしょうだいめいし)
\\	稲妻(いなずま) 雷(かみなり) 
\\	稲妻が光った(ひかる)
\\	冷蔵庫(れいぞうこ)
\\	織機(しょっき) 食器(しょっき)
\\	織機で織る(おります)
\\	ださい ださい服装 ださい言葉 ださいやつ
\\	引く(ひきます) 10から2を引くと、8残る
\\	並ぶ(ならびます)
\\	無駄(な)(むだ) 無駄なコピーが多いと言った
\\	~そびれる(そびれます) 終電に乗りそびれた この本を読みそびれた
\\	ハーフ ラリーはアメリカ人と日本人のハーフだ
\\	取り付かれる(とりつかれます) 彼女は悪霊に取り付かれている
\\	美食法(びしょくほう) 美食(びしょく)
\\	爆竹(ばくちく) 爆竹を鳴らす
\\	陰嚢(いんのう)
\\	だけ
\\	スキャンダル スキャンダルが発覚した
\\	親指 (おやゆび)
\\	ぼろ ぼろを着た人
\\	揃って(そろって) 一家そろって出かける
\\	縁起(えんぎ) 縁起がいい
\\	売店(ばいてん) キオスク
\\	ぐっすり寝る(ねます)
\\	混乱(こんらん)
\\	無秩序(むちつじょ)
\\	そのお知らせで多くの混乱が起こった
\\	クーラー エアコン
\\	提出する(ていしゅつ) 
\\	提出者(ていしゅつしゃ) 新しい法案(ほうあん)を議会に提出する
\\	大阪 今大阪の旅行をしている
\\	辛子(からし) 辛い(からい)
\\	粘液質(ねんえきしつ) 粘液質の人
\\	ポプラ ポプラ並木を散歩する ポプラ並木の道
\\	入学試験 (にゅうがくしけん) 入試 (にゅうし)
\\	新た(な)(あらた) 新しい(あたらしい) 新たに 新たな事実が明らかになった
\\	公共の(こうきょう) 公立の(こうりつ) 
\\	公共の場 公共の乗り物
\\	還暦(かんれき) 彼は今年還暦を迎えた
\\	同語反復(どうごはんぷく) 彼の書くものには同語反復が多い
\\	明かす(あかします) 
\\	暴く(あばきます) 
\\	解明する(かいめい) 彼は私の秘密を暴いた
\\	陰毛(いんもう) 息子に陰毛が生えてきた
\\	片付く(かたづきます) 仕事の半分は片付いた 部屋がすっかり片付いていた
\\	漁船(ぎょせん) 水平線に漁船がいくつ見えます
\\	入浴する(にゅうよく)
\\	浴びる(あびます) お風呂に入る 入浴中である 病人を入浴させる
\\	犯罪(はんざい) 犯罪学(はんざいがく)
\\	出産(しゅっさん) 姉は六月に出産の予定です(よてい)
\\	知り合い(しりあい) 知人(ちじん) 彼に知人は多いが、友人は少ない
\\	重力(じゅうりょく) 引力(いんりょく) 月の重力は地球の6分の1だ
\\	遺跡(いせき) 古代文明の遺跡
\\	ニュース (お)知らせ(しらせ)
\\	ありふれた 平凡(へいぼん)
\\	ありふれた間違い ありふれたしゃれ
\\	道順(みちじゅん) 行き方(いきかた) どの道順がいちばん早いですか 駅までの道順を教えてください
\\	ゲロを吐く(はく)(俗) へどを吐く 嘔吐(おうと)
\\	日帰りする(ひがえり) 日帰りで神戸へ行った 日帰り旅行 彼は日帰り旅行にいった
\\	栄光(えいこう) 神に栄光あれ!
\\	田舎っぺ(いなかっぺ)
\\	~を 諦める(あきらめます) ~を 辞める(やめます)
\\	回復の望みはないとあきらめた イギリス行きをあきらめた
\\	迷信深い(めいしんぶかい) 迷信的(な)(めいしんてき)
\\	コマーシャル
\\	無罪(むざい) 無罪の 
\\	無実の(むじつ) われわれは彼が無罪だと考えた
\\	料理する(りょうり) 食べ物・食事を作る 炊事する(すいじ)
\\	あのお祖父さんが中国人だった「人」は一郎さんです。
\\	試験に落ちる(おちます) 試験に滑る(すべます)
\\	一致する(いっち)   
\\	同意する(どうい) 賛成する(さんせい)  
\\	私の意見はあなたのと一致しています
\\	静かに!(しずか) 煩い!(うるさい) 静か(な) (しずか)
\\	調査(ちょうさ) 
\\	検査(けんさ)
\\	調査(ちょうさ) 国勢調査(こくせいちょうさ)
\\	女垂らし(おんなたらし) プレイボーイ 彼は大変な女たらしだった
\\	あべこべ あなたはシャツをあべこべに着いている
\\	チェッカー チェッカーをする
\\	やや すこし やや驚いた様子だった ややあって彼女が現れた(あらわれる)
\\	ナックルダスター
\\	薬缶(やかん) 夜間(やかん) 
\\	やかんで湯を沸かす
\\	職場電話番号
\\	晴天(せいてん) いい天気
\\	日本学科(にほんがっか)
\\	ゼムクリップ 文房具(ぶんぼうぐ)
\\	背骨(せぼね) 私の背骨の痛みはひどくなっています
\\	移住(いじゅう) 彼の家族は40年前に母国からブラジルへ移住した
\\	織る(おります) この布はウールで織ってある 織姫(おりひめ)
\\	ハミングする/ハミングで歌う 鼻歌を歌う(はなうた) 彼女は古い曲を低い声でハミングした
\\	祖父母(そふぼ) 祖父母と一緒に住んでいる
\\	~く/に 感じる(かんじます) 話はつまらないと、時間が長く感じる
\\	~てはいけない 食べてはいけません
\\	(~を)輸入(ゆにゅう)
\\	正月なので、休む
\\	おかげさまで
\\	~を 巡らす(めぐらします) 囲む(かこみます)
\\	塀を巡らしてある 堀を巡らした城
\\	ビールでもいい
\\	出す(だします) 手紙を出す。
\\	発作(ほっさ) 喘息(ぜんそく)の発作があった
\\	かゆい むずがゆい 背中がかゆい ああ, かゆい このシャツを着るとむずがゆくなる
\\	ソース入れ(いれ)
\\	乾燥(かんそう) 乾燥する(かんそう) 乾く(かわきます) 乾燥した手
\\	率直(な)(そっちょく) アメリカは思っていることを率直に言う
\\	アリクイ
\\	口がきけない 口がきけなくなる
\\	受刑者(じゅけいしゃ) 囚人(しゅうじん)
\\	喧嘩(けんか)する 争う(あらそいます) 恋人とけんかしたので、会わないことにした
\\	定食(ていしょく)
\\	横断する(おうだん) 彼らは大西洋を横断した
\\	コンクリート
\\	植える(うえます) 春は木を植える時期です
\\	昏睡(こんすい) 昏睡状態です   昏睡状態を陥る(おちいります)
\\	注文する(ちゅうもん) 頼む(たのみます)
\\	紳士(しんし) 紳士振る(しんしぶります)
\\	種類(しゅるい) この山には色々な種類の鳥がいる
\\	引用する(いんよう) ことわざを引用する
\\	岩石の多い (がんせきのおおい)
\\	滑りやすい(すべりやすい)
\\	特待(とくたい) 特待される 特待生(とくたいせい)
\\	ペンキ 壁をペンキで塗る
\\	ペンキ塗り立て(ぬりたて) 塗りたての(ぬりたて)
\\	最終的(さいしゅうてき) この決定は最終的だ
\\	竜の落とし子(タツノオトシゴ)
\\	シェパード
\\	本当に(ほんとう) 本当にすみません
\\	タジキスタン
\\	甘い(あまい)
\\	優先(ゆうせん)する ~を 優先させる この仕事が他のどれよりも優先する 家庭より仕事を優先させる
\\	同僚(どうりょう) 会社の仲間(なかま) 私は同僚と一緒に海外旅行をしたい
\\	頭痛(ずつう)
\\	損害(そんがい)
\\	被害(ひがい) 
\\	損害を与える 洪水による損害
\\	下書き(したがき)
\\	過密(かみつ) こんでこんで 込んでいる すし詰め 満員(まんいん) 過密都市 過密スケジュール
\\	サンドイッチ
\\	伝説(でんせつ) 
\\	昔話(むかしばなし) 
\\	民間伝説(みんかんでんせつ) このロマンチックな話は中国の伝説です
\\	酸素(さんそ)
\\	時間がない
\\	脚本(きゃくほん) 脚本を書く ~を脚本にする 脚本家
\\	同生共死(どうせいきょうし)
\\	ブレーンストーミング ブレーンストーミングする
\\	~合う 話し合います あげ合います
\\	町立(ちょうりつ) 町立の図書館
\\	驚く(おどろきます) びくびくする 
\\	友達は驚いた
\\	~を 振る(ふります) 
\\	犬が尾を振った 人に手を振る よく振ってからお飲みください
\\	目上(めうえ)
\\	クレヨン
\\	プレゼント 贈り物(おくりもの)
\\	~に 敏感(びんかん) 彼は寒さに大変敏感です
\\	簡単(な)(かんたん) 易しい (やさしい) 優しい(やさしい)
\\	サドル 鞍(くら)
\\	連邦(れんぽう) ドイツ連邦共和国 英連邦(えいれんぽう)
\\	エリート 上流階級 (じょうりゅうかいきゅう)
\\	食べながら(たべながら)
\\	以来(いらい) 依頼(いらい)
\\	先月以来 日本に来て以来、ずっと東京に住んでいます
\\	ペンネーム 芸名(げいめい)
\\	喪失する(そうしつ) 
\\	損する(そん) 
\\	損害する(そんがい) 
\\	失う(うしないます) 彼は記憶喪失で悩んでいる 記憶を喪失する
\\	晴れる(はれます) 晴れです 明日は晴れるでしょう
\\	文化の壁 文化の壁を越える
\\	覗く(のぞきます) 除く(のぞきます) 
\\	部屋をのぞく
\\	大和言葉(やまとことば)
\\	焦げる(こげます) 焦がす(こがします) 私は何が焦げるにおいがする
\\	耳たぶ(みみたぶ)
\\	乗せる(のせます) 置く(おきます) ~に
\\	もらう (いただく)
\\	あいにく 残念ながら あいにく私はお金を持っていない あいにくの雨で野球の試合は流れた
\\	はちみつ はちみつ酒(しゅ)
\\	海外(かいがい)(の) 外国(がいこく)(の)
\\	腕(うで)
\\	かばんの中に何が入っている?(はいっている)
\\	固まる(かたまります) 我々の計画はまだ固まってない
\\	野原(のはら) 
\\	草原(くさはら)
\\	畑(はたけ)
\\	聞き違える(ききちがえます) 私はお言葉を聞き違えてはいませんでしょうか
\\	看板(かんばん) 広告板(こうこくばん)
\\	店の前に看板を出す 看板には「太平ホテル」と書いてあった
\\	ある日
\\	椎骨(ついこつ)
\\	頭(とう)
\\	天の川 (あまのがわ)
\\	割る(わります) メロンを6人で6つに割る
\\	趣味(しゅみ)
\\	縁(ふち) テーブルの縁にグラスを置かないでください
\\	描写する(びょうしゃ) その場の様子(ようす)を描写する
\\	誰かに対ふくしゅうする 復習(ふくしゅう)する 
\\	太る(ふとります) 肥える(こえます)
\\	超える(こえます)
\\	ちり紙(ちりがみ) トイレット・ペーパー
\\	落ち着く(おちつきます) あなたは落ち着いて、歴史の試験の勉強をしなくちゃ
\\	国際調査(こくさいちょうさ)
\\	保証する(ほしょう) そのショーの成功は保証する
\\	詐欺師(さぎし)
\\	晩年(ばんねん) 晩年になってやっと成功した
\\	大部分(だいぶぶん) 大半(たいはん)
\\	過半数(かはんすう)
\\	参加者の大部分 大部分の国々
\\	億(おく) 奥(おく)
\\	欠伸をする(あくび)
\\	題する(だい) 友情という題で 「民主主義」と題する論文
\\	子音(しいん) 美土里さんは時々子音が落ちる
\\	政治(せいじ) 私達はしばしば日本の政治について話し合った
\\	番人(ばんにん) 万人(ばんにん) 
\\	(書)
\\	高血圧(こうけつあつ) 血圧が高い
\\	マンホール マンホールの蓋(ふた)
\\	料理法(りょうりほう) レシピ
\\	闘争する(とうそう)
\\	ストライキ 彼らは賃上げ闘争をやっている
\\	エギゾチック(な)
\\	昇進(しょうしん) かれらは先輩に昇進おめでとうと言った
\\	明らか(あきらか)(な) 君が間違っているのは明らかだ
\\	情勢(じょうせい) 
\\	状況(じょうきょう) 
\\	状態(じょうたい) 
\\	中東情勢についていかが考える
\\	~を 巻く(まきます) 時計のねじを巻く
\\	~が 要る(いります)
\\	砂漠(さばく) 砂漠は雨が少ない
\\	毛布(もうふ) 毛布を2枚掛けて寝た
\\	イガ(衣蛾) 蛾(が)
\\	心(こころ)
\\	手綱を握る(たづなをにぎります) 彼は奥さんにしっかりと手綱を握られている 彼はちょっとでも手綱を緩める(ゆるめます)と怠け始める(なまけはじめる)
\\	使う(つかう) 使用(しよう)する 利用(りよう)する
\\	先(さき) 鉛筆の先は丸くなった
\\	惑星(わくせい) 
\\	騎士(きし) ナイト
\\	多額(たがく) 金額(きんがく) 彼女は息子に多額のお金を残して亡くなった(なくなった) 多額の寄付
\\	進む(すすみます) 演説が進むにつれてますます退屈になった イギリスでは教育が進んでいる ドアのほうへ進む
\\	水疱瘡(みずぼうそう) その子供は水疱瘡症状がある
\\	語尾 
\\	けど はい、好きですけど。
\\	完済する(かんさい) 関西(かんさい)
\\	借金を完済する
\\	掴む(つかみます) この鎖(くさり)を掴みなさい!
\\	ヴィスワ川(がわ) オーデル川(がわ) ヴァルタ川(がわ)
\\	~を 目撃する(もくげき) 私は殺人を目撃した
\\	まんずりする
\\	これがいい 私がやります
\\	偶像(ぐうぞう) 像(ぞう)
\\	ローマ法王の偶像 偶像を崇拝する
\\	財布(さいふ)
\\	~において/~における (書) アメリカにおける学生生活 過去においてはこのようなことはあまり問題になたなかった
\\	宿題(しゅくだい) 宿題をやる
\\	出稼ぎする(でかせぎ) 出稼ぎ労働者 お父さんが出稼ぎから帰ってきた
\\	袖(そで)
\\	調理器具(ちょうりきぐ)
\\	命日(めいにち)
\\	~を 治す(なおす) 病気を治す
\\	件(けん) 事件(じけん)
\\	この件について、コメントが欲しいですか …の件については
\\	ヘアスタイル 髪型(かみがた) どんなヘアスタイルにしますか
\\	欧州連合(おうしゅうれんごう)
\\	中入り(なかいり)
\\	スリラー
\\	野うさぎ(のうさぎ) 私は野うさぎを捕まえることができなかった
\\	胡坐をかく(あぐら) 床の上にあぐらをかく
\\	少々(しょうしょう)
\\	政策(せいさく) 農業政策を立てる 党の政策に従う
\\	製造業(せいぞうぎょう)
\\	アフリカ
\\	仲良くする(なかよく) 友達になる 仲良くする 子供のころから仲良くしている
\\	閉じる(とじます) 閉める(しめます) 口座を閉じたいのですが 口を開けて目を閉じなさい
\\	今では
\\	栄養(えいよう) 牛乳とチーズとでは、どちらが栄養がある
\\	~に 違いない/違いありません あれはトンプソンさんに違いない
\\	解釈する(かいしゃく) この文章をどう解釈しましたか
\\	表示する(ひょうじ) 表す(あらわします)
\\	示す(しめます)
\\	非常口は緑の文字で表示してある
\\	議会(ぎかい) 国会(こっかい)
\\	3時間も
\\	次のように
\\	一石二鳥(いっせきにちょう) 
\\	どころか 病気どころか ジムは日本語の会話ができないどころか、簡単なあいさつもできない
\\	30年以上 小林さんは今から30年以上前に生まれました
\\	怒りっぽい(おこりっぽい)
\\	カタール
\\	ダイエットをする ケンさんは今ダイエットをしていると言っていました
\\	(どうも) ありがとうございます・ました。
\\	電子メール(アドレス) (でんし)
\\	短所(たんしょ) 弱点(じゃくてん)
\\	難点(なんてん)
\\	欠点(けってん)
\\	気が弱いのが彼の短所だ 彼女は短所が多い 長所(ちょうしょ)
\\	~て すみません 私は間違ってすみませんと誤った
\\	間欠泉(かんけつせん)
\\	アラビア文字(もじ)
\\	電気椅子(でんきいす) 電気椅子で処刑される
\\	勇ましい(いさましい) 勇気(ゆうき) 勇敢(ゆうかん) 勇ましい行為 勇ましい騎士
\\	~に ~時間を 費やす (ついやします) ~時間を無駄に(むだ) 過ごす(すごす)
\\	金銭を浪費する 
\\	つまらないことに時間を費やす 毎日30分を英語のヒアリングの練習に費やしている
\\	愛猫家(あいびょうか) 愛猫(あいびょう)
\\	裸(の)(はだか) ヌード(の) 裸足(はだし)
\\	変化する(へんか) 変わる(かわります) ~と 化する(かする)
\\	彼の意見は変化しやすい
\\	~を 折り畳む(おりたたみます) ~折る(おります)
\\	新聞を折り畳む 折り畳み椅子(おりたたみいす)
\\	誰かに 何かを謝る (あやまります) 誤る(あやまります) 
\\	畜生(ちくしょう)! 畜生、パンクだ。
\\	隣人(りんじん)
\\	隣の人(となりのひと) 隣人は外科の医者です(げかのいしゃ)
\\	予防(よぼう) 防ぐ(ふせぎます) 予防する 病気を予防することは出来ますか
\\	アメリカ映画(えいが)
\\	砂丘(さきゅう)
\\	タクシーを拾う(ひろいます)
\\	予定表(よていひょう) スケジュール ニューヨーク出張の予定表と資料ができた
\\	(に)等しい(ひとしい) 1ドルは100セントに等しいです
\\	無駄遣いする (むだづかい) (お金を)浪費する(ろうひ) 3万円も無駄遣いをしてしまった 子供に無駄遣いさせてはいけない
\\	計量する(けいりょう) 計る(はかります) 計量したら 
\\	キロあった 計量しえない
\\	体力(たいりょく) 体力がつく  体力が落ちる[なくなる]
\\	製造する(せいぞう) この社会はテレビを製造しています
\\	小鳥(ことり)
\\	揺り椅子(ゆりいす) ロッキングチェア
\\	辞書(じしょ) 辞典 (じてん) 字引(じびき)
\\	レモネード
\\	誘惑(ゆうわく) どんな誘惑にも負けるな 誘惑に勝つ 悪い友達が彼を誘惑してたばこを吸わせた
\\	東京を案内(あんない)
\\	標準(ひょうじゅん) 基準(きじゅん)
\\	標準の その赤ちゃんの発育はその年齢では標準だ あなたと私では考え方の標準が違う 標準に達する
\\	療法(りょうほう)
\\	治療(ちりょう)
\\	両方(りょうほう)
\\	電気療法 医者は彼女のがんに新療法を試みている(こころみます)
\\	命令する(めいれい) 父の命令でおじの家へ行った
\\	心中(しんちゅう) 心中では彼らを軽蔑(けいべつ)していた
\\	理学(りがく) 人文科学(じんぶんかがく) 理学士(りがくし)
\\	霜焼け(しもやけ)
\\	凍傷(とうしょう)
\\	指に霜焼けが出来ている 霜焼けのつま先
\\	座席(ざせき) 席(せき) 客席(きゃくせき)
\\	この劇場には300人分の座席がある 座席を予約する
\\	祖父(そふ) お祖父さん(おじいさん)
\\	~を 脅かす(おびやかします) 脅える(おびえます) 脅かして追い払う  殺すぞと脅かす
\\	会計(かいけい) 彼はあの店で会計を担当している 会計をお願いします 会計年度
\\	おかげで 君が間抜けだった、おかげで僕達はその試合に負けたんだ
\\	模範(もはん) 彼は正直(しょうじき)の模範だ 模範的(もはんてき)   先生は弟子に模範を示さなくてはならない 模型(もけい)
\\	四字熟語 
\\	馬耳東風(ばじとうふう) いくら言っても馬耳東風だった. 馬耳東風に聞き流す
\\	改める(あらためます) 今年から交通規則(こうつうきそく)が改められた。
\\	ロリコン 
\\	ロリータコンプレックス) あいつはロリコンだ
\\	コーヒーを入れる
\\	開ける(あけます) 開く(あきます)
\\	水道管(すいどうかん)
\\	氷(こおり)
\\	蛇口をひねる 水道の栓(せん)を捻る(ひねります)
\\	東口(ひがしぐち)
\\	いつまでも 両親にはいつまでも長生きしてほしい
\\	四字熟語 
\\	同床異夢(どうしょういむ) その夫婦は同床異夢の生活をしていた
\\	語る(かたります)
\\	行う(おこないます) 施す(ほどこします)
\\	実施する(じっし)
\\	応用する(おうよう)
\\	無声映画(むせいえいが)
\\	星座(せいざ) 正座(せいざ)
\\	おうし座(ざ)
\\	不思議(な)(ふしぎ)
\\	住居(じゅうきょ) 住宅(じゅうたく) 住居を持つ 住居を定める
\\	無能の(むのう) 警察の無能を非難する
\\	すててこ
\\	宜しい(よろしい) 朝ご飯に何が宜しいですか
\\	シャンピニオン マッシュルーム 
\\	口輪(くちわ)
\\	理解(りかい) 私は彼のジョークが理解できなかった
\\	いっぱい バスは人でいっぱいです
\\	さそり
\\	人間(にんげん)
\\	大量虐殺(たいりょうぎゃくさつ)  (たい 
\\	だい!!!)
\\	(~を)逃す(のがします) ~そびれる 
\\	大事な点を聞き逃した 私は映画スターになる機会を逃した(のがす)
\\	物価(ぶっか) 東京は人口も多いし、物価も高いので、住みたくないです
\\	永遠(えいえん) 永遠の命
\\	歯がガチガチ鳴る(なります) 彼の歯は寒さでガチガチ鳴った
\\	稼ぐ(かせぐ)
\\	祝日(しゅくじつ) 祭日(さいじつ) 私たちは国民の祝日には旗(はた)をたてる
\\	謙虚(な)(けんきょ) 彼は社会的地位が上がるにつれますます謙虚になった
\\	体罰(たいばつ) 体罰を加える
\\	かすか(な) 遠い灯台のかうかな光を見た かすかな音 そのことはかすかに覚えている
\\	鼻の穴(はなのあな)
\\	残り物(のこりもの) サンドイッチの残り物 晩御飯の残り物
\\	分泌する(ぶんぴつ) 分泌を出す 分泌物(ぶんぴつぶつ) この腺は唾液を分泌する
\\	さいころ
\\	会合(かいごう) 集まり(あつまり) 大阪での会合で 私たちは月に1度会合する
\\	紐(ひも) 
\\	縄(なわ)
\\	綱(つな)
\\	ロープ 
\\	ガレージ 
\\	車庫(しゃこ) 
\\	西欧(せいおう) 東欧(とうおう)
\\	大人(おとな) 成人(せいじん) 聖人(せいじん) 
\\	白鳥(はくちょう)
\\	ご免なさい(ごめんなさい)
\\	赤道(せきどう)
\\	~直す(なおす) 掛け直す 書き直す 考え直す やり直す
\\	腫瘍(しゅよう)  主要(な)(しゅよう)
\\	癌(がん)
\\	腫瘍学(しゅようがく) 腫瘍ができる
\\	~に 生息(せいそく) 生息地 生息に適した(てきした)
\\	おい おい、遠すぎる
\\	家庭教師(の先生)(かていきょうし) 英語の家庭教師をする
\\	風刺する(ふうし) 風刺的(ふうしてき) この本は書かれた当時を痛烈に風刺している
\\	射る(いります) 鳥を射る 矢を射る/飛ばす/放つ 的を射る
\\	書留(かきとめ)
\\	井戸(いど) この地方では井戸の水を使っています
\\	硬貨(こうか) 効果(こうか)
\\	高価(こうか)
\\	コイン 千円札を100円硬貨に替える
\\	訳す(やくす)
\\	格好いい” 
\\	~というのは ~ということです 
\\	格好いい”というのは「とてもハンサムだ」ということです
\\	消火器(しょうかき)
\\	馬の耳に念仏(ねんぶつ)/風(かぜ) 念仏(ねんぶつ) その講演会に彼女を連れて行っても馬の耳に念仏だろう
\\	庭園(ていえん)
\\	庭(にわ)
\\	血腫(けっしゅ) 
\\	鮮魚店(せんぎょてん)
\\	証人(しょうにん) 証人(しょうにん)
\\	あなたはその証人の陳述を信じますか 目撃者(もくげきしゃ)
\\	繊維(せんい) 化学繊維 繊維産業
\\	丸々(な)(まるまる) 丸々太った赤ちゃん
\\	連休(れんきゅう) 連休は何をしますか
\\	授業料(じゅぎょうりょう) 入学金(にゅうがくきん)
\\	年齢(ねんれい) おじいさんの年齢は僕の年齢の5倍だ
\\	結局(けっきょく) うわさは結局うそだと分かった
\\	昔(むかし)
\\	(表現)	経験が浅い (けいけんがあさい)
\\	コメディー
\\	結成(けっせい) 政党を結成する 形成する(けいせい)
\\	警報機(けいほうき) 盗難警報機(とうなんけいほうき) 警報機が鳴っている
\\	都市(とし)
\\	都会(とかい)
\\	独自(どくじ) 独自の泳ぎ方 私独自の見解(けんかい)です
\\	むずむずする 鼻がむずむずする 彼は家に帰りたくてむずむずしていた
\\	降る(ふります) 雪が降る
\\	忘却(ぼうきゃく)(書) 忘却する/忘れる その小説は長い間忘却されていた
\\	唇(くちびる)
\\	要素(ようそ) 宗教の神秘的要素 …の要素となる
\\	警護(けいご) 敬語の使い方が間違っている. 
\\	敬語を使って話す
\\	暴走する(ぼうそう) 暴走族(ぼうそうぞく)
\\	心配がない
\\	異性愛(いせいあい) 異性愛者
\\	乗馬する(じょうば) 馬に乗る(のります) 私は乗馬を体験した
\\	時速(じそく) 時速4 キロ
\\	きっと テストはきっと難しいだろう
\\	カラスの足跡(あしあと) あの人カラスの足跡があるからそんなに若くないよ
\\	確か(たしか) その病気の確かな治療法 (ちりょうほう)
\\	だけ ばかり 魚だけを食べた 魚を食べただけだ 雪子さんとはデートしただけだ
\\	偶然(ぐうぜん) 偶然に 彼の名前を偶然知っていません それは偶然に起こった
\\	折り畳み傘(おりたたみがさ) 折り畳む(おりたたみます) 傘の骨 (かさのほね)
\\	グラム
\\	にしん
\\	無人島(むじんとう)
\\	アルゼンチン
\\	松(まつ)
\\	天王星(てんのうせい)
\\	夫婦(ふうふ) 夫婦はけんかしたがすぐに仲直りした
\\	初級(しょきゅう)
\\	香港(ホンコン)
\\	人数(にんずう) 生徒の人数は何人ですか 大[小]人数のグループ
\\	楽しい(たのしい) 陽気(ようき)
\\	明るい(あかるい)
\\	鉄分(てつぶん) 鉄分に乏しい(とぼしい)食事 この水は鉄分が多い
\\	地獄(じごく) この世の地獄(このよ) 地獄のような 地獄に落ちる
\\	~を 覚える(おぼえます)
\\	遊園地(ゆうえんち)
\\	就職(しゅうしょく) 彼はその就職の面接を受けたが就職できなかった
\\	跡(あと) 車が通った跡(とおる) 歯の跡
\\	子牛(こうし) けんさん子牛のような大きい犬を買った
\\	寝台(しんだい)
\\	彼は何を言ったか分からない
\\	飛蝗(ばった)
\\	多く点で
\\	角度(かくど) 90度の角度は直角と呼ばれている
\\	~に 釘付けにする(くぎづけ) ふたを箱に釘付けにする 彼はテレビに釘付けだった
\\	火災(かさい) 火事(かじ) 火災を起こす 火災警報(かさいけいほう)
\\	ひそか(な)
\\	内緒(ないしょ)
\\	秘密(ひみつ)
\\	神秘(しんぴ)
\\	葬式はひそかに行われた 彼女に対するひそかな恋心 (こいごころ)
\\	とにかく とにかく彼に聞こう! とにかく途中で計画の変更(へんこう)はできない
\\	側(そば)
\\	ランプのそば 横(よこ)
\\	テーブルのよこ
\\	性的嗜好(せいてきしこう) 性的 嗜好(しこう)
\\	リボン 頭にリボンをつける
\\	指す(さします)
\\	本格的(ほんかくてき) 朝鮮料理がお好きならあのレストランを試してごらんなさい,本格的なものを出しますよ 本格的に雨が降り出した
\\	人情味(にんじょうみ) 人情にんじょう)
\\	とても人情味のある人だ
\\	御馳走様でした(ごちそうさま) 御馳走(ごちそう)
\\	名高い(なだかい) 有名(な)(ゆうめい) よく知られる 政界に名高い人物 世界に名高い音楽学校の一つ, ジュリアード学校
\\	バイブレーター
\\	乗車する(じょうしゃ) 車に乗る(のります) 電車に乗車する 下車(げしゃ)
\\	杉(すぎ)
\\	統計(とうけい) 
\\	東経(とうけい) 専門家がその統計を詳細に分析(ぶんせき)した
\\	着信(ちゃくしん) 着信局(ちゃくしんきょく)
\\	全部(ぜんぶ)
\\	健康(けんこう) 健康者(けんこうしゃ)
\\	下落(げらく) 物価が下落した
\\	あの世(よ)
\\	死後の命(しごのいのち)
\\	パーム パームをかけている
\\	禁煙(きんえん)
\\	お姉さん (おねえさん) 姉(あね)
\\	イヤリング 耳飾(みみかざり) イヤリングの一組(いちくみ) ピアス
\\	~にとって その数学の試験は私にとってはやさしかった
\\	亀(かめ) 常用外
\\	強火で(つよび) 弱火で(よわび) 強火で(つよび)煮る
\\	背広(せびろ) スーツ
\\	お願い(おねがい)
\\	北京語(ぺきんご) 普通話(ふつうわ)
\\	きのこ狩り(きのこがり) 来週、ミレックさんが私を茸狩りに連れていってくれます
\\	経過(けいか) 経つ(たちます) あれから2年経過した
\\	一応(いちおう) 
\\	奇跡(きせき) 人間の存在自体が奇跡です
\\	リュックサック ランドセル
\\	貴重(な)(きちょう) 価値がある(かち)
\\	貴重な経験 貴重なお時間 貴重品(きちょうひん)
\\	ポーランド分割(ぶんかつ) 分割(ぶんかつ)
\\	メス
\\	~が 落ちる(おちます) あ!お金が落ちています!
\\	遥かに(はるかに) 鉄は金よりはるかに役に立つ 彼は私よりも遥かぺらぺらな英語が話せる
\\	共和制(きょうわせい)
\\	岩山(いわやま)
\\	理想化する(りそうか)
\\	ダウン症候群 (しょうこうぐん)
\\	発見する(はっけん) 探検家は洞くつの中で骸骨を発見した
\\	しみじみ(と) 当時のことをしみじみと語った 彼のことをしみじみと思い出した
\\	声帯(せいたい) 人間の声は声帯の振動(しんどう)による
\\	強盗する(ごうとう) 強盗に入る(はいります) あの銀行は、一週間前に強盗に入られた
\\	~を 無視する(むし) 虫(むし)
\\	交通信号を無視して
\\	巻(まき) この全集は最終巻がかけている
\\	コンパス 私はコンパスで丸を描いた
\\	獣医(じゅうい) 獣医学(じゅういがく) 動物病院(どうぶつびょういん)
\\	近代(きんだい) 現代(げんだい) 最近小説の中でこれが最高作品です
\\	獲得する(かくとく) その建築家は世界的名声を獲得した
\\	やぶ 
\\	かん木(かんぼく)
\\	改築(かいちく) その家は改築された 風呂場(ぶろば)の改築を計画した
\\	(星座)	羊座(ひつじざ)
\\	刺身(さしみ) まぐろの刺身
\\	乗り換える(のりかえます) バスからトラムに乗り換えます
\\	改札口 (かいさつぐち) 改札口を通って行く(とおっていく) 改札(かいさつ) 
\\	腸(ちょう) 蝶(ちょう)
\\	小腸(しょうちょう) 大腸(だいちょう)
\\	(いいえ)どういたしまして。
\\	列(れつ) 子供の列 前の列の人
\\	ふた 洗濯機のふたを閉める
\\	高所恐怖症 (こうしょきょうふしょう) 私は高所恐怖症です
\\	顎(あご) 
\\	常用外 チン
\\	終身刑(しゅうしんけい) 終身刑受刑者 (しゅうしんけいじゅけいしゃ)
\\	出身(しゅっしん) 私は北海道出身ですよね あなたはどちらご出身ですか
\\	真似る(まねます) オウムは人の声を真似ることが出来る
\\	紛争(ふんそう) その紛争を解決するのは不可能だ
\\	おなら(する) パパがおならをした いもを食べるおならが出る
\\	損なう(そこないます) 飲み過ぎは健康を損なう 彼女の感情を損なわないように
\\	ディフェンス
\\	立候補する (りっこうほ) 対立候補者(たいりっこうほしゃ) 国会議員に立候補する 総選挙に立候補する
\\	少しも(すこしも)
\\	ない) 全然(ぜんぜん)
\\	ない) まったく
\\	ない) この部屋は少しも静かじゃない
\\	倒れる(たおれる) 転ぶ(ころびます)
\\	転ける(こけます)
\\	(関西弁) 石につまずいて倒れる  清朝は1912年に倒れた
\\	緩い(ゆるい) だぶだぶ(な) 
\\	緩い規則 運動靴が緩い
\\	~初める ~出す 
\\	車が動き出した しばらくして、彼らはまた歩き始めた
\\	盲腸(もうちょう)
\\	虫垂(ちゅうすい)
\\	虫垂炎(ちゅうすいえん) 盲腸炎(もうちょうえん)
\\	パイプ パイプを吹く(ふきます) パイプの愛用者(あいようしゃ)
\\	~を 擦り落とす(こすりおとします) さびをこすり落とす
\\	トンネル
\\	落し物(おとしもの) 忘れ物(わすれもの)
\\	場違い(ばちがい) あなたが言ったことは、かなり場違いだった
\\	整理する(せいり) 片付ける(かたづけます)
\\	彼は部屋をきちんと整理した 行政整理 書類を整理する
\\	反省(はんせい) 彼女は自分したことを反省した
\\	鞍(くら) 鞍を置いた馬 馬に鞍を置かずに乗る
\\	俳句(はいく) 俳人(はいじん)
\\	燕尾服(えんびふく) テールコート
\\	胸元(むなもと) 胸(むね)
\\	ブラウスの胸元 銃を胸元に突きつける(つきつける)
\\	狒々(ひひ)
\\	狒々おやじ
\\	学習誌(がくしゅうし)
\\	気圧(きあつ) 気圧計(きあつけい) 高気圧(こうきあつ) 低気圧(ていきあつ)
\\	住人(じゅうにん)
\\	不作法(ぶさほう)
\\	武器 (ぶき)
\\	価値権(かちけん) 価値観は人によって異なる
\\	長短(ちょうたん) 
\\	一長一短(いっちょういったん) 都会生活の長短
\\	夜が明けた(よがあけた)
\\	委員(いいん)
\\	花粉症(かふんしょう) 花粉(かふん) かわいそうにその子は花粉症に悩んでいる
\\	はやる(はやります) 広がる(ひろがります)
\\	悪い風邪がはやっています
\\	たこ・カイトを上げる (あげます)
\\	いいえ、まだです。
\\	ハサミムシ (挟み虫) 挟む(はさみます)
\\	懐かしむ(なつかしむ) 日本の懐かしさ 懐かしい
\\	そうですね
\\	12番のバス
\\	目標(もくひょう) 
\\	目的(もくてき)
\\	目標を立てる 私の家は、大きな木を目標にして来ればすぐ分かる
\\	押す(おす) ボタンを押す
\\	土地(とち)
\\	没する(ぼつ) 1995年没 編集者はその記事を没にした
\\	有する(ゆう) 持つ(もちます) 有る(あります) 天然資源を有する国
\\	劣等感(れっとうかん) 引け目(ひけめ) ~だれかに 劣る(おとります) 彼の友人に対して劣等感を持っている
\\	疲れる(つかれます) 疲れた
\\	雪片(せっぺん)
\\	自習(じしゅう) 彼は英語を自習した
\\	手製(の)(てせい) 
\\	手作り(の)(てづくり) 
\\	夕食は彼女のお手製だった
\\	隠者(いんじゃ)(書)
\\	悪夢(あくむ) 悪夢のような体験
\\	電話番号(でんわばんごう) 電話番号を教えて下さい
\\	略称(りゃくしょう) 略語(りゃくご) 略(りゃく) 
\\	はアメリカ合衆国の略称です
\\	~に 決める(きめます)
\\	根強い(ねづよい) 根強い反対 根強い努力
\\	活火山(かっかざん)
\\	寛容(かんよう)
\\	トレランス ~に 寛容する 彼らのあのふるまいは寛容できない
\\	構造(こうぞう) 文構造(ぶんこうぞう) 私はアイヌの家族の構造を研究した
\\	くすぐる (くすぐります) わきの下をくすぐる くすぐったい 足の裏がくすぐったい
\\	鶉(うずら)
\\	本気(ほんき)の・で 真面目(まじめ)な・で 本気に(ほんき)・まじめにする 冗談で言った,本気にするな
\\	相当する(そうとう) 
\\	ぴったり 
\\	適当(てきとう)
\\	適切(てきせつ)
\\	彼に相当した仕事だ
\\	ゼミナール ゼミ ゼミに参加する
\\	~を ほしがっている 
\\	(し)たがっている 赤ちゃんがミルクをほしがっている 弟はオーストラリアへ行きたがっています
\\	コンマ その2語の間にコンマを置く
\\	登場人物(とうじょうじんぶつ) 登場する(とうじょう)
\\	武装した/している(ぶそう) 武装した警官 その兵士たちは武装している
\\	雪達磨(ゆきだるま) 雪だるま 雪だるまを作る
\\	外国旅行(がいこくりょこう)
\\	特徴(とくちょう) 日本の夏の特徴は多湿(たしつ)なことだ
\\	祖母(そぼ) お祖母さん(おばあさん)
\\	牧場(ぼくじょう) 牛が牧場で草を食べている 牧場を経営する
\\	目を細める(ほそめます)
\\	神道(しんとう)
\\	並べる(ならべます) 並ぶ(ならびます) いすを横に2列に並べる
\\	洗剤(せんざい) 食器用洗剤 食器用洗剤で手に湿疹がある
\\	音を立てる 空きの容器は一番音を立てる
\\	マフラー 襟巻き(えりまき) スカーフ 
\\	マフラーをする
\\	正座する(せいざ)
\\	まくら 
\\	クッション 
\\	布団(ふとん) 
\\	熟字訓(じゅくじくん)
\\	時差ぼけ(じさぼけ) 時差ぼけする 時差ぼけがまだ治らない
\\	野球(やきゅう)
\\	ごめんください
\\	風刺画(ふうしが) 風刺画家(ふうしがか) 風刺(ふうし)
\\	個人旅行 (こじんりょこう)
\\	靴の紐(くつのひも) 靴の紐を結びなさい 靴のひもが解けた
\\	森(もり) 林(はやし) 森と森の間に川がある
\\	水族館(すいぞくかん)
\\	化合物(かごうぶつ) 工場では複雑な化学化合物を製造している
\\	アイリング 耳飾 (みみかざり)
\\	蜻蛉(とんぼ)
\\	両親(りょうしん) 良心(りょうしん)
\\	父母(ふぼ) 親(おや)
\\	ペット
\\	~きる(きります) ~てしまう 小説を読みきった
\\	てんかん てんかん発作
\\	再三(再四) (さいさん(さいし)) 彼は私に、再三電話をかけてきた
\\	鵞鳥(がちょう) 鵞鳥はすいちょうです
\\	悲しい(かなしい)
\\	大体(だいたい) 約(やく) 三十七万平方キロメートル
\\	ハンマー 金槌(かなづち)
\\	毒液(どくえき)
\\	ヒマワリ
\\	天ぷら
\\	消しゴム(けしごむ)
\\	糞(くそ) 常用外
\\	その上(そのうえ)
\\	さらに 
\\	滅多に(めったに)
\\	ない) 私の会社がめったに休まない
\\	ベルト 運転手はベルトを締めています
\\	足し算(たしざん) 引き算 掛け算 割り算
\\	信用(しんよう) 信用して! 信用ができる友達 しっかり友達
\\	(お)礼(れい) 例(れい) 
\\	感謝(かんしゃ) お礼の手紙 礼の気持ちを表わす 彼は私の親切に礼を言った
\\	ニコチン ニコチンガム
\\	かつら かつらを着ける(つける)
\\	塾(じゅく) 彼は塾で勉強している
\\	負んぶする(おんぶ) 子供をおんぶする
\\	盗む(ぬすみます) 渡辺さんは泥棒にかばんを盗まれた
\\	~の向かい側(むかいがわ) ~の向こう側(むこうがわ) 彼は言えはちょうど道路の向かい側だ
\\	残酷(な)(ざんこく) 残忍(な)(ざんにん)
\\	ペットを虐待(ぎゃくたい)すると彼は残酷だ 残酷な場面
\\	都心(としん) 兎唇(としん)
\\	東京の都心 都心に買い物
\\	~を 虐待する(ぎゃくたい) 妻を虐待する  動物を虐待する  心理的虐待
\\	催涙ガス(さいるい) 催涙(さいるい) 
\\	鼻をかむ
\\	暴露する(ばくろ) 発覚する(はっかく)
\\	その政治スキャンダルは二人のジャーナリストによって暴露にされた
\\	配る(くばります) そのパンフレットを皆さんに配ってください
\\	鬼ごっこ(おにごっこ) 私たちは昔鬼ごっこのよんな遊びとよくした
\\	飲食店(いんしょくてん) レストラン 彼は飲食店を経営している
\\	断頭台(だんとうだい) 灯台(とうだい)
\\	断頭台の露(つゆ)と消える 断頭台に上らせる(あがらせる)
\\	植物学者 植物学
\\	干草(ほしくさ)
\\	カプセル
\\	旅行会社の人 (りょこうがいしゃ)
\\	現在(げんざい) 今(いま)
\\	現在人々はかつてより以上に教育を受けている 現在の状況 現在は
\\	何でもない それは私になんでもないことです
\\	~となったら お客が来るとなったら
\\	辞任する(じにん) 辞める(やめます) 彼はついに大学の学長を辞任した
\\	コーヒーはいかがですか。
\\	食料品(しょくりょうひん) 食品(しょくひん) いつも日曜日に一週間分の食料を買っておきます
\\	~が 痛む(いたみます) ~が 痛い(いたい)です
\\	ラッシュ(の時)に 東京はラッシュ中だ
\\	中年(ちゅうねん) 中世(ちゅうせい)
\\	中年者(ちゅうねんもの)
\\	~を 習う(ならう) 
\\	クーラー 
\\	エアコン 
\\	冷暖房(れいだんぼう) 
\\	冷房(れいぼう) 
\\	運河(うんが) その運河は大西洋と太平洋をつないでいる
\\	入れ替わる(いれかわります) 空気を入れ替える 私は時計の電池を入れ替えてもらった
\\	カリスマ カリスマチック カリスマ性(の)
\\	炊事する(すいじ) 料理する(りょうり) 食べ物・食事を作る 私は炊事が全然できない
\\	雑巾(ぞうきん) モップ 床にぞうきんをかける
\\	試合(しあい) マーチ
\\	出来事(できごと) 事件(じけん)
\\	その出来事は2年前に起こった その日は大したできごともなかった
\\	パンツ
\\	大抵(な)(たいてい)
\\	講堂(こうどう) 行動(こうどう)
\\	スロバキア  スロベニア
\\	足が強い/弱い
\\	章魚(たこ) 
\\	たこ、カイト 
\\	たこ, 魚の目
\\	大統領(だいとうりょう)
\\	自動詞(じどうし) 他動詞(たどうし) 人が並んだ
\\	力学(りきがく)
\\	芸名(げいめい) ペンネーム …という芸名の女優
\\	とっさに 突然に(とつぜんに) とっさにそのことが頭に浮かんだ
\\	~に びくびくする 脅える(おびえます) 爬虫類にびくびくしている
\\	三食(さんしょく) 
\\	日に三食食べる
\\	原型(げんけい) 唐の長安が平城京の原型である
\\	減少する(げんしょう) 現象(げんしょう)
\\	減る(へります) 輸入が昨年の3分の2に減少した
\\	言い回し(いいまわし) 語句(ごく) 
\\	表現(ひょうげん)
\\	どうしてもいやだ まさか 
\\	決して+ない(けっして) あの人と結婚するのはどうしてもいやです そんなことをするのはどうしてもいやだ
\\	大男(おおおとこ)
\\	巨人(きょじん) 
\\	彼はかなりの大男だね
\\	帰り道(かえりみち) 帰り道が分からなくなった 学校からの帰り道で
\\	決心(けっしん) 決定(けってい) 決める(きめます) 彼女はついに外国に行く決心をした
\\	耳(みみ) 耳鼻科(じびか)
\\	相変わらず(あいかわらず) 彼は相変わらずタバコを吸っている
\\	悪天候(あくてんこう)
\\	悪い天気
\\	建築(けんちく)
\\	インボイス
\\	遺伝学(いでんがく)
\\	食(しょく) 月食(げっしょく)
\\	盆栽(ぼんさい)
\\	着替える(きがえます) スーツに着替えなさい
\\	円周(えんしゅう) 円周500メートルある 円周率(えんしゅうりつ)
\\	南米(なんべい)
\\	寝室(しんしつ)
\\	ホットドッグ
\\	料理教室 (りょうりきょうしつ) こちらで料理教室が開けた
\\	芽(め) 芽を出す スイセンはいつ芽が出ますか
\\	渡米する(とべい) アメリカ/米国に行く 彼は渡米してから10年になる
\\	素人(しろうと) 素人臭い (しろうとくさい) ゴルフに関してはずぶの素人だ
\\	本当は
\\	ブルジョア階級
\\	白血球(はっけっきゅう) 赤血球(せっけっきゅう)
\\	敵意(てきい) 敵意がある その制度に対する個人的な敵意はない
\\	南北(なんぼく) 川は南北に流れている 
\\	南北戦争
\\	~に 住む/住んでいる(すむ)
\\	おかえりなさい。
\\	大小(だいしょう) 大きさ(おおきさ) サイズ 道路上の大小の石に気をつけなさい
\\	学ぶ(まなぶ) 勉強する 習う(ならいます) 外国の文化について学ぶのはたのしい
\\	雑音(ざつおん) 騒音(そうおん) 往来の雑音がやかましくて話もできない ラジオに雑音が多い 余計な雑音を入れないでください
\\	革靴(かわぐつ)
\\	結論(けつろん) ~という結論に達する(たっする) 結論を引き出す
\\	本物(ほんもの) このダイヤは本物ですか 本物の真珠
\\	帝王切開(ていおうせっかい) カエサル切開(せっかい)
\\	飽きる(あきます) 毎日、魚ばかり食べていたので、もう飽きました。
\\	旅館(りょかん)
\\	公園(こうえん) 広い公園
\\	霜(しも) 霜が降りる(おります)
\\	扱う(あつかいます) 
\\	操作する(そうさ)
\\	この機械は扱いにくい 扱いにくい人
\\	シナモンスティック シナモンパウダー
\\	ナンバープレート
\\	避妊(ひにん) 避妊法(ひにんほう)
\\	申請する(しんせい) 申し込む(もうしこみます) パスポートを申請する 今日ビザの申請をするところだ 申請を出す
\\	集団(しゅうだん) グループ 集まり(あつまり)
\\	群れ(むれ)
\\	学生の集団 子供たちが集団で遊んでいた
\\	氷山(ひょうざん) 北風は明らかに氷山から吹き出す
\\	ナツメグ 肉荳蔲(ニクズク)
\\	片付ける(かたづけます) 掃除する(そうじ) 綺麗にする(きれい) 整理する(せいり)
\\	机の上を片付ける
\\	陳情(ちんじょう) 陳情書(ちんじょうしょ) 陳情を受け入れる[はねつける]
\\	脂質(ししつ) 
\\	脂肪(しぼう)
\\	チンチラ
\\	~が 溶ける(とけます) 砂糖がコーヒーに溶けていった
\\	癖(くせ) 習慣(しゅうかん)
\\	傾向(けいこう)
\\	誇張癖(こちょうくせ) 子供は親の癖をまねる
\\	デンタフロス
\\	ために 大学に入るために、一生懸命勉強した
\\	フェレット 
\\	弁護士(べんごし) 弁護士の黒川さん
\\	通常(つうじょう) 女性は通常男性よりも長生きする。
\\	幽霊(ゆうれい) 「あの城には幽霊がいる」と彼は震えながら言った
\\	電話帳(でんわちょう)
\\	広場(ひろば) 天安門広場(てんあんもん)
\\	何日も(なんにちも)
\\	大聖堂(だいせいどう)
\\	促進する(そくしん) 広める(ひろめます)
\\	人工肥料は植物の生長を促進する
\\	蟻(あり) 常用外
\\	焼餅焼き(やきもちやき) 嫉妬深い人
\\	借りる(かります) 会社の近くのアパートを借りるつもりだ
\\	恥ずかしい(はずかしい) こんなに易しい質問に答えられなっては、恥ずかしいです 恥ずかしくて、顔が赤くなった
\\	綿菓子(わたがし) 綿飴(わたあめ)   (東日本)
\\	いせえび
\\	郭公(かっこう) 格好(かっこう) 
\\	当たり前(あたりまえ) 借りた物を返すのは当たり前です
\\	札束(さつたば)
\\	どんどん(と) どんどん歩く
\\	青白い(あおじろい) 真っ青(な)(まっさお)
\\	顔が青白い
\\	ファッション 流行(りゅうこう)
\\	反発する(はんぱつ) 若い者は両親の考えに対して反発するものだ そのような言葉には反発を感じる
\\	建設する(けんせつ)
\\	建築する(けんちく) 建てる(たてます)
\\	建設的(けんせつてき) 道路を建設する 新しいビルが今,建設中だ
\\	音声学(おんせいがく)
\\	腐食(ふしょく) 屋根が腐食し始めた
\\	空き(あき)
\\	間隔(かんかく)
\\	余裕(よゆう) 空きを作る/場所をあける もう一人分の空きがありますか
\\	会釈(えしゃく) 
\\	お辞儀(おじぎ) 
\\	教務課(きょうむか)
\\	原料(げんりょう) その州は全国に様々な原料を供給している
\\	評価(ひょうか) その上司は自分の秘書を高く評価している
\\	応接間(おうせつま) 応接間に入ってはいけない
\\	ニコニコする 
\\	微笑む(ほほえみます) にっこりする 
\\	子供を見てにこにこした いつもにこにこしている
\\	小鹿(こじか) 小鹿が可愛らしかった
\\	南アフリカ(みなみ)
\\	四字熟語	牛飲馬食する (ぎゅういんばしょく)
\\	友人(ゆうじん) 僕の友人の坂本さんです
\\	折角(せっかく) 
\\	母がせっかく作ったご馳走がテーブルの上で冷たくなっている
\\	作曲する(さっきょく) この音楽はバッハによって作曲された
\\	日記(にっき)
\\	公の(おおやけ) 公立(こうりつ)
\\	公共(こうきょう)
\\	公/公共の場所 公の発表
\\	無神論(むしんろん) 無心論者(むしんろんしゃ) 彼は自分が無心論者だと告白した
\\	哀れみ(あわれみ) 哀れみ深い 哀れみと愛情は紙一重
\\	経度(けいど) 東経(とうけい)   統計(とうけい)
\\	西経(せいけい)
\\	ハイキングをする
\\	~に 参加する(さんか) 運動会に参加する
\\	領事館(りょうじかん) 領事(りょうじ)
\\	読書(どくしょ) 本を読むこと
\\	廊下(ろうか)
\\	固体(こたい) 個体(こたい)
\\	この物質の普通の状態は個体である
\\	冷え(ひえ) 寒さ(さむさ)
\\	冷えから起こる病気 朝方の冷え(あさがた)
\\	山脈(さんみゃく) ヒマラヤ山脈
\\	任せる(まかせます) 預ける(あずける)
\\	部下に仕事を任せる 後は私たちに任せて, 少し休んでください
\\	あいまい(な) あいまいなこと/あいまいさを避ける
\\	疑問(ぎもん)する 私は彼が成功するかどうか疑問がある
\\	特待生(とくたいせい) 特待生になる 特待(とくたい) 奨学金(しょうがくきん)
\\	顔立ち(かおだち) あなたはお母さんに頭立ちがよく似ているね
\\	海(うみ)
\\	親善(しんぜん) 友情(ゆうじょう)
\\	日米親善を促進する(そくしん) 国際親善
\\	感情(かんじょう) 勘定(かんじょう)
\\	感じ(かんじ)
\\	気持ち(きもち)
\\	感情を顔に表す(あらわす)
\\	繁栄(はんえい) 反映(はんえい)
\\	繁栄する
\\	流産する(りゅうざん) 去年流産しました
\\	国道(こくどう) 国道18号線
\\	長短(ちょうたん)
\\	長さ(ながさ) 背丈(せたけ)
\\	2本の ひもの長短を測る
\\	英字(えいじ) 英字新聞
\\	~に 集中する(しゅうちゅう) 私は読書に集中している 集中力(しゅうちゅうりょく)
\\	お子さん(おこさん)
\\	切り裂きジャック(きりさき) 切り裂く(きりさきます)
\\	悪霊(あくれい) 彼女は悪霊に取り付かれている 悪霊を追い払う
\\	(お)祭り(まつり)
\\	可哀想(な)(かわいそう) 可哀想な子供
\\	割引(わりびき) 学生割引 学生割引はありますか
\\	ゲイ・パレード
\\	ケチャップ
\\	全集(ぜんしゅう) 私はシェイクスピア全集を持っています
\\	(電気)洗濯機(でんきせんたくき)
\\	犠牲者(ぎせいしゃ) みんなが犠牲者達に深く同情した
\\	正味(しょうみ) 正味価格(しょうみかかく) 正味50グラム
\\	ボランティア
\\	羅針盤(らしんばん)
\\	卵の殻(から)
\\	勇気(ゆうき) (有機(ゆうき) 
\\	勇敢(ゆうかん) 勇ましい(いさましい) 勇気を出す 彼に逆らう勇気はなかった いよいよという時に勇気がなくなった
\\	スタジアム スタジアムには大勢がいた
\\	運動靴(うんどうぐつ)
\\	思わず(おもわず) 裁判官は思わず笑った
\\	規則(きそく)
\\	規制(きせい)
\\	ルール 
\\	規則は守らなければなりません
\\	指(ゆび)
\\	向ける(むけます) 銃を男に向けた 窓の方に顔を向ける
\\	肉たたき(にくたたき)
\\	尾骨(びこつ) 鼻骨(びこつ)
\\	主題(しゅだい) 映画の主題歌 …を主題とした詩
\\	くもの巣(す)
\\	ベイリーブ 月桂樹の葉 (げっけいじゅのは)
\\	~てはいけない ~てはならない ~てはだめ 
\\	ここに入ってはいけない ここに入ってはならない ここに入ってはだめ
\\	プロポーズ 何が彼にプロポーズするように誘惑(ゆうわく)したのか プロポーズを受け入れる/断る
\\	プレッシャー 圧力(あつりょく)
\\	プレッシャーをかける プレッシャーを受ける  高い目標を掲げて(かかげる)自分にプレッシャーをかけることも必要だ
\\	逆様に(さかさま) 逆様する ほうきをさかさまに立てる
\\	攻める(せめます) 攻撃する(こうげき)
\\	襲撃(しゅうげき) 城を攻める 後ろから攻める
\\	摩る(さすります) 胸を摩ってあげましょうか
\\	詰まる(つまります) 鼻が詰まっている 排水管が詰まっている
\\	何か温かい 私に何か温かい飲み物をください
\\	歯列矯正器 (しれつきょうせいき) 歯列矯正器をつける
\\	しゃっくり しゃっくりをする
\\	~と 知り合う(しりあいます) 彼女とパーティーで知り合った
\\	炭素(たんそ) 石炭は大部分が炭素からなっている
\\	客室(きゃくしつ)
\\	未成年(の)(みせいねん) 18歳未満(みまん) 
\\	彼女はまだ未成年だ
\\	神社(じんじゃ) 神社に参ります
\\	包帯(ほうたい) 医者は少年の怪我をした足に包帯をした
\\	脚(あし) 机の脚が折れてしみった(おれる)
\\	では、また。
\\	来日する(らいにち) ボブがいつ来日するが知りません
\\	メコン川(がわ) ガンジス川(がわ) インダス川(がわ)
\\	起床する(きしょう) 起きる(おきます) 覚める(さめます) 起床時間 (軍隊の)起床ラッパ
\\	鬣(たてがみ) たてがみのある
\\	両側(りょうがわ) 道の両側に店を並んでいます
\\	目隠し(めかくし) 人に目隠しをする
\\	跡継ぎ(あとつぎ) 相続人(そうぞくにん)
\\	叔父(おじ)の家には跡継ぎがいない
\\	頭上注意(ずじょうちゅうい)
\\	休日(きゅうじつ)
\\	休暇(きゅうか)
\\	バカンス (夏)休み 休日の計画
\\	円満(えんまん) 円満な人物 彼らは40年間夫婦として円満に暮らした
\\	データ このデータはあなたの役に立つでしょう
\\	妥協(だきょう) 妥協の余地がない これらの両国は平和のために互いに妥協した
\\	~よう と思う インターネットを始めようと思っている
\\	冷やす(ひやします) 冷える(ひえます) ビールを冷やす
\\	のどを鳴らす(ならす) 猫がのどを鳴らす
\\	エアコンをつける(つけます)
\\	帰国する(きこく) 父は帰国したばかりだ
\\	寄付(きふ) 彼はその病院に多額のお金を寄付した
\\	立ち入り禁止! (たちいりきんし)
\\	~する 
\\	それはどのぐらいしましたか
\\	~を 煮る(にます) 煮える(にえます) じゃがいもを煮る
\\	逆に(ぎゃくに) 薬を飲んだら、逆に熱が出た 逆にする アルファベットの順を逆にしなさい
\\	前半(ぜんはん) 後半(こうはん) 前半戦(ぜんはんせん)
\\	コンクール 
\\	コンテスト 
\\	競技(きょうぎ) 
\\	試合(しあい) 
\\	彼は写真のコンクールに参加した
\\	つまり 外国人の一団が江戸、つまり東京に到着した
\\	何て 何てすばらしい天気だろう
\\	輸出(ゆしゅつ) 輸入(ゆにゅう) 政府は米の輸出を禁止した
\\	鋭い(するどい) 鈍い(にぶい) このナイフは刃が鋭く、よく切れる
\\	大学院生(だいがくいんせい) 大学院
\\	早起き(はやおき) 朝型の人間 (あさがた)
\\	はっきり はっきり見えない
\\	お参りする(おまいり) 神社にお参りに行く
\\	~ください ~ちょうだい 
\\	スプーンをちょうだい
\\	数学(すうがく) 算数(さんすう)
\\	わくわくする アイスホッケーはわくわくするスポーツです
\\	昼寝(ひるね)
\\	手を振る(ふります)
\\	引く(ひきます)
\\	来なかった(こなかった) 昨日、学生はポズナニへ来なかったと思います
\\	拷問する(ごうもん) 拷問をかける 囚人は拷問で死んだ 拷問具(ごうもんぐ)
\\	表現の自由
\\	壊れる(こわれます) 故障する(こしょう) 自動販売機は壊れています
\\	思い切り(おもい(っ)きり) 全力(ぜんりょく) ベッドで体を思いきり伸ばす.
\\	セーター
\\	山道(やまみち)
\\	ハンサム(な)
\\	貝(かい)
\\	文部省(もんぶしょう)
\\	文部科学省 (もんぶかがくしょう)
\\	小指(こゆび)
\\	そこらじゅう そこらじゅうで流行する そこらじゅうに散らばった書類
\\	三つ編み(みつあみ) 三つ編みにした髪 髪を三つ編みにする
\\	板(いた) 板を張る(はります) 私はその板をインチで計った
\\	退院する(たいいん) 入院する(にゅういん)
\\	草原(くさはら) 草原で羊の世話をする人を何と言いますか。
\\	腹痛(ふくつう) 腹痛がする
\\	長い間(ながいあいだ) 長い間彼女を待った
\\	必ず(かならず)
\\	だろう) 必ず変化か起こるだろう 明日は必ず雨が降るだろう
\\	地区(ちく) ハリウッドはロサンゼルス市にある地区です 阪神地区(はんしんちく)
\\	挽き肉(ひきにく) ピーマンの挽き肉詰め
\\	怒る(おこります) 起こる(おこります)
\\	大丈夫(な)(だいじょうぶ)
\\	新入生(しんにゅうせい) 一年生(いちねんせい) 私に話しかけた少年は新入生
\\	唐辛子(とうがらし)
\\	札(ふだ) 包みに札を付けた
\\	古代(こだい) あの古代の廃墟かつては神社だった
\\	おかえり。
\\	あくどい 王女があくどい化粧をしてた
\\	不可能(ふかのう)
\\	無理(むり)
\\	不可能にする 不可能になる
\\	大学院(だいがくいん) 大学院生
\\	本音(ほんね) 本音を吐く 彼の本音がわからない 社長は言うことと本音が違う
\\	正常位(せいじょうい) 通常位(つうじょうい)
\\	互いに(たがい) 人は互いの家を詳しく調べるのが大好きだ
\\	バス停(ばすてい) バス乗り場(のりば) 停留所(ていりゅうじょ)
\\	じじい
\\	スプーン
\\	シンガポール シンガポールについてどう思う
\\	農民(のうみん) 農民の暮らし
\\	後書き(あとがき) 手紙に後書きを付け加える
\\	有名人(ゆうめいじん) 有名人に会ったことがありますか
\\	~に 出会う(であいます) 私は若者に出会った
\\	てくてく歩く
\\	カナリア
\\	大韓民国(だいかんみんこく) 韓国(かんこく)
\\	石鹸(せっけん) 石けん シャボン 石鹸皿(せっけんざら)
\\	相談する(そうだん) 誰と相談すればそれが分かるでしょうか
\\	魔法瓶(まほうびん)
\\	内臓(ないぞう) 内臓を摘出する(てきしゅつ)
\\	果肉(かにく)
\\	日傘(ひがさ) パラソル 傘立て(かさたて)
\\	フェラチオする しゃぶる(しゃぶります)
\\	長男(ちょうなん) 長女(ちょうじょ)
\\	社宅(しゃたく)
\\	江戸時代(えどじだい) 
\\	寂しい(さびしい) 寂しい(さみしい) 
\\	恋しい(こいしい) 
\\	試験を受ける(うける) 試験に受かる
\\	とりわけ とくに 彼女はとりわけ宝石を磨くのが好きだった
\\	~員(いん) 郵便局員(ゆうびんきょくいん)
\\	上下(じょうげ) 船が川を上下している 彼は旗を上下に動かした
\\	~するのに このナイフはチーズを切るのに便利です
\\	~に 水をやる(やります)
\\	誰かを騙す(だまします) 誰かをごまかす 詐欺を動く(うごきます)
\\	所有する(しょゆう) 有る(あります) 彼は広大な土地を所有している
\\	パートタイム 妻はパートタイムで働いている
\\	弁護する(べんご) 
\\	被告を弁護することにした 教授は自分の説を弁護した
\\	男役(おとこやく) 女役(おんなやく) 男役を演じる(えんじます)
\\	言語獲得(げんごかくとく) 第一言語獲得
\\	ピストル
\\	いただきます
\\	追求(ついきゅう) 彼は富(とみ)の追求だけに興味を持った
\\	逮捕する(たいほ) 私は彼の逃亡助けたので逮捕された
\\	-ている 本田さんは結婚している
\\	入れ歯(いれば)
\\	故郷(ふるさと) 故郷を出る
\\	運搬する(うんぱん) 運ぶ(はこぶ) 運搬(うんぱん) 運輸(うんゆ)
\\	履歴書(りれきしょ)
\\	先生(せんせい)
\\	教師(きょうし)
\\	罠 (わな) 鼠取り(ねずみとり)
\\	大きなねずみが罠に掛かった 娘は男の罠に掛かってしまった
\\	銀食器(ぎんしょっき)
\\	~を 主張する(しゅちょう) 彼は階級差別の廃止を主張した 計画の中止を主張する
\\	暇です
\\	苛め(いじめ) だれかを苛める(いじめます)
\\	馬小屋(うまごや) 馬を馬小屋にいれる
\\	施設(しせつ)
\\	設備(せつび)
\\	暖房施設
\\	のり 彼はメモをドアにのりではります
\\	口頭(こうとう) 口頭試験
\\	単語(たんご) 言葉(ことば)
\\	ショッピングカート
\\	麻痺(まひ) 吹雪のために交通は麻痺状態となった 彼女は両足が麻痺している
\\	厳しい(きびしい) 厳重(な)(げんじゅう) 厳格(げんかく) 
\\	直ぐ(に)(すぐ)
\\	溺れる(おぼれます) 溺れ死ぬ(おぼれしにます) ~を 溺れさせる
\\	押さえる/抑える(おさえます) はしごを押さえていてくれ 動けないようにしっかりと彼の手足を押さえた ボブは怒りを押さえる事が出来なかった
\\	何でも 
\\	そうだ/ようだ 何でもこの辺は物価が非常に高いそうですよ
\\	四川(しせん) 
\\	視線(しせん)
\\	四川省(しせんしょう)
\\	為さる(なさいます) どうぞ、ご心配なさらないでください 夏休みはどう為さいますか
\\	通貨(つうか) 通過する(つうか) 
\\	真鍮(しんちゅう) 心中(しんちゅう)
\\	金は色が真鍮と似ている 真鍮は銅と亜鉛の合金である
\\	悪循環(あくじゅんかん 高賃金と高物価の悪循環
\\	すみませんが、 すみませんが、あけてくだいませんか。
\\	ものの のに パソコンを買ったものの使い方が分からない
\\	賭け金(かけきん) 掛け金(かけきん) 
\\	暗い(くらい) そとはまだ暗いです。
\\	ハイエナ
\\	娯楽(ごらく) 遊び(あそび)
\\	楽しみ(たのしみ)
\\	娯楽番組 釣りは娯楽ではなく真剣にやっている
\\	抱える(かかえます) 
\\	物を脇(わき)の下に抱える
\\	信仰(しんこう) キリスト教を信仰していますか
\\	最善(さいぜん) 一番いい 最高(さいこう) 最もいい(もっとも) これがその問題を解決する最善の方法だ 最善を尽くす
\\	洒落る(しゃれます)
\\	冗談を言う ジョークを言う
\\	庭(にわ) 有名なお寺や庭がたくさんあちます
\\	常用外	トアル 手拭い(てぬぐい) 拭う(ぬぐう)
\\	気象学(きしょうがく) 気象(きしょう) 気象観測(きしょうかんそく)
\\	だれを車で送ってあげる
\\	トイレの水を流す
\\	ろうそくの心(しん)
\\	~を 終える(おえます) ~が 終わる(おわります)
\\	歌が上手です
\\	灯油(とうゆ) 灯油ランプ
\\	好き嫌い(すききらい) 食物に好き嫌いがある 好き嫌いを言うな
\\	キリン
\\	絶好の(ぜっこう)
\\	理想的(りそうてき)
\\	完璧な(かんぺき)
\\	今日は絶好のマラソン日和だ
\\	サタニズム
\\	削る(けずります) 鉛筆を削る
\\	洪水(こうずい)
\\	大水(おおみず) 洪水が村に大きな被害を与えた 質問の洪水
\\	爆発(ばくはつ)
\\	破裂(はれつ)
\\	水道管が破裂した
\\	キーワード
\\	掛ける(かけます) 3に5を掛けると15になる
\\	歩く(あるきます)
\\	昭和五十二年 
\\	万里の長城 (ばんりのちょうじょう) これは便利の長城で撮った写真です
\\	電力(でんりょく) 電力会社 電力危機
\\	いっぱい   お腹がいっぱいです
\\	音素(おんそ) 音素論(おんそろん)
\\	お願いします
\\	(豚肉)	(卵の)白身(しろみ) 黄身(きみ) 白身(しろみ)
\\	羨ましい(うらやましい)
\\	嫉妬深い(しっとぶかい)
\\	焼餅を焼く(やきもちをやく)
\\	彼の成功がうらやましかった
\\	ハンドバッグ かばん
\\	違反(いはん) 法律に違反する 条約(じょうやく)に違反する行為
\\	フライパン
\\	死者(ししゃ) 支社(ししゃ)
\\	交通事故の結果、死者が三人出た
\\	通す(とおす) 針に糸を通す 部屋に風を通す
\\	閉まっている(しまっている) レストランが閉まっています
\\	第二次世界大戦 (だいにじせかいたいせん) 多くの兵士が第二次世界大戦で亡くなった
\\	貪欲(どんよく) 貪欲(な) 名声に対して貪欲である
\\	扇風機(せんぷうき) 暖かかったので扇風機をつけた
\\	~つつある(書) ~ている 台風は九州に接近しつつある
\\	蒸し暑い(むしあつい)
\\	~を 大切にする 彼女は彼の昔の恋文を大切していた
\\	下痢をする(げり)
\\	美容院(びよういん)
\\	キックスクーター
\\	和英辞典(わえい) 英和辞典(えいわ)
\\	扁豆(ひらまめ) レンズ豆(まめ)
\\	手渡す(てわたす) 彼女は一枚の葉書を手渡した
\\	組(くみ) その牧師は今週三組の結婚式を行った
\\	~途端に(とたん) 部屋に入った途端に電話が鳴った 座った(ら)途端
\\	~通りに(とおりに) 説明書のとおりに、組み立ててください
\\	準備運動 ウォーミングアップ
\\	過剰(の)(かじょう) ~過ぎる(すぎます) 自信過剰な人 彼女は親切過剰である
\\	未熟(な)(みじゅく) 彼の運転は未熟だ 彼のフランス語の知識は未熟だ
\\	極道(な)(ごくどう) 極道者(ごくどうしゃ) 極道の限りを尽くす
\\	点く(つきます) 火が点いた
\\	~に基づいて/基づく (もとづいて) (書) このビルはある有名な建築家の設計に基づいて建てられた
\\	不安(な)(ふあん) ~について 不安になる 私は時々未来について不安になる
\\	洋服(ようふく)
\\	無理やり(むりやり) 強制的に(きょうせいてき)
\\	無理やり彼女を働かす
\\	我利我利(な)(がりがり) 利己的(な)(りこてき)
\\	我がまま(な)(わがまま)
\\	リビング 居間(いま)
\\	半日(はんにち) 反日(はんにち)
\\	昨日は半日寝て暮らした 半日で
\\	大家族(だいかぞく) 大家族の生活から核家族へと変わってきた
\\	材料(ざいりょう)
\\	社長(しゃちょう)
\\	いけない 何ていけない子だろう この封筒は小さすぎていけない
\\	練習が完成を生む
\\	継続(けいぞく) 持続(じぞく)
\\	存続(そんぞく)
\\	続く(つづきます) 科学者はその研究を継続することを強く要求した
\\	付け加える(つけくわえます) 一言付け加えたい もう何も付け加えることはない
\\	愛用者(あいようしゃ) パイプを愛用者(あいようしゃ) めがねよりコンタクトレンズの愛用者が増えている
\\	~に 負ける(まけます)
\\	名場面(めいばめん)
\\	アクセル アクセルを踏む(ふむ) アクセルから足を放す(はなす)
\\	言い出す(いいだします) だれがそんなデマを言い出したのか レストランに行こうと彼が言い出した だれが言い出すともなくわれわれは空き瓶を集め始めた
\\	温室(おんしつ) 温室効果(おんしつこうか) 温暖化現象(おんだんかげんしょう)
\\	紹介(しょうかい) 自己紹介する(じこしょうかい)
\\	親類(しんるい)
\\	親戚(しんせき)
\\	彼は親類の人と意見が合わない 彼は私の遠い親類です
\\	ホメオパシー
\\	額縁(がくぶち) これはプラスチック製の額縁です
\\	せいぜい 彼女はせいぜい18歳だ
\\	盗難(とうなん) 東南(とうなん)
\\	強盗(ごうとう)
\\	泥棒(どろぼう) 盗難の被害者 車の盗難
\\	~に あこがれる (あこがれます) あの人々は都会の生活にあこがれています
\\	手土産(てみやげ) これはほんの手土産です 手土産を持って訪ねる
\\	草食(な)(そうしょく) 草食動物 草食する
\\	壁(かべ)
\\	塀(へい)
\\	幅(はば) この道の幅はどれくらいですか
\\	砂場(すなば) その子供は砂場にトンネルを掘った
\\	梅(うめ) プラム
\\	(どうぞ)よろしく。
\\	鵜(う)
\\	電話にでる 電話をとる
\\	謎々(なぞなぞ) 謎(なぞ) 謎々遊びする
\\	歌舞伎(かぶき)
\\	鐘(かね) ベル 
\\	これがビッグベンと呼ばれる鐘
\\	母国(ぼこく)
\\	中々(なかなか)
\\	ない 日本では中々馬を見ることが出来ない
\\	黒後家蜘蛛(くろごけぐも)
\\	細胞(さいぼう) 単細胞生物(たんさいぼうせいぶつ)
\\	習字(しゅうじ)
\\	書道(しょどう)
\\	習字を習う 習字が上手[下手]である
\\	学位(がくい) 私は生物学で学位を取った
\\	米価(べいか)
\\	レコード
\\	信心深い(しんじんぶかい) 私は信心深い男で、死後の生命の存在を信じています
\\	擦る(こすります) 摩る(さすります)
\\	~を 擦り落とす(こすりおとします) 信じられなくて思わず目をこすったね
\\	育てる(そだてる) 
\\	野菜を育てる
\\	仕事をちょっと手伝って
\\	多数決(たすうけつ) 多数決は民主主義の基本的なルールです
\\	来春に(らいしゅん) 彼らは来春にそのプロジェクトを実施する(じっし)
\\	茶筅(ちゃせん)
\\	後背位(こうはいい) バック
\\	衛生(えいせい) 衛星(えいせい)
\\	衛生的(えいせいてき)
\\	ばね
\\	鼻が高い 誇る(ほこります) 自慢する(じまん) そう言われると私も鼻が高い
\\	閉会する(へいかい) 誰もそれ以上いわなかったので、閉会した
\\	同感する(どうかん)
\\	同意する(どいする)
\\	一致する(いっち)
\\	賛成する(さんせい)
\\	税金に関するあなたの意見に同感する
\\	公開する(こうかい) 後悔する(こいかい)
\\	航海する(こうかい)
\\	陶磁器収集品は一般に公開されている
\\	短編小説 (たんぺんしょうせつ) 短編映画
\\	発祥地(はっしょうち) 中東は文明の発祥地である
\\	予算(よさん) 日本政府は今年の予算を決めた
\\	丸い(まるい) 壁に丸くて、青い月の写真がはってあります
\\	上がる(あがります) 値段が上がります
\\	急(きゅう)
\\	留年する(りゅうねん) 来年は留年だな
\\	宇宙旅行(うちゅうりょこう)
\\	施設(しせつ) 
\\	慈善施設(じぜんしせつ) 公共施設
\\	~を分ける(わける) ~を分割する(ぶんかつ)
\\	忙しい(いそがしい) 多忙(な)(たぼう)
\\	数々の(かずかず) 色々(いろいろ) 様々(さまざま)
\\	ピーマン 赤ピーマン(あか) 緑ピーマン(みどり) 黄色ピーマン(きいろ) 唐辛子(とうがらし)
\\	終電(しゅうでん) 終電に乗りそびれた
\\	割り勘(わりかん) 食事代を割り勘にした 割り勘勝ち(がち)/割り勘負け(まけ)
\\	才能(さいのう) 
\\	能力(のうりょく) 
\\	才能ある人 才能を伸ばす 彼女は音楽に才能がある
\\	(語)	広東語(かんとんご)
\\	刺さる(ささります) 骨がのどに刺さりました
\\	あげる(あげます) くれる(くれます)
\\	石炭(せきたん) 炭(すみ)
\\	時間を戻す(もどします) 時間を止める(とめます)
\\	愛撫する(あいぶ) 母の愛撫を今でも覚えている
\\	国際連合(こくさいれんごう) 国連(こくれん) 国連加盟国 (こくれんかめいこく)
\\	民族主義(みんぞくしゅぎ) 彼らの民族主義がその戦争の原因の一つだった
\\	なまず
\\	高価(な)(こうか) 効果(こうか)
\\	硬貨(こうか)
\\	(値段が)高い 高価で買う 高価な機械
\\	ドアをとんとん叩く ドアをどんどんたたく
\\	蘭(らん)
\\	ごめんなさい。
\\	塩(しお) 
\\	食塩(しょくえん)
\\	塩入れ(しおいれ)
\\	冷たい(つめたい)
\\	遠距離恋愛 (えんきょりれんあい) 遠恋(えんれん)(略)
\\	煮える(にえます) 煮る(にます) このにんじんはすぐ煮える
\\	良心(りょうしん) 両親(りょうしん)
\\	あなたは自分の良心に従って行動すべきだ
\\	役人(やくにん) 公務員(こうむいん)
\\	彼はこの町の住人によって役人に選ばれた
\\	武道(ぶどう) 葡萄(ぶどう)
\\	教育する(きょういく) 彼は公立学校で教育を受けた
\\	そっくり よく似ている(にる) 瓜二つ(うりふたつ) 父親そっくりだ モンローのそっくりさん
\\	~端(たん) 消化管の最前端(さいぜんたん)は口です 島の最南端にある
\\	ブス 
\\	あざ 両足にあざができた
\\	~版(はん) ペーパーバック版
\\	食べるばかり デザートは食べるばかりになっているいます
\\	新聞(しんぶん)
\\	~を 盗む(ぬすみます)
\\	姉妹都市(しまいとし)
\\	刈る(かります)
\\	私は髪を短く刈ってもらった
\\	シャベル 喋る(しゃべります) 
\\	彼らは歩道の雪をシャベルで片付けていた
\\	ジョーク 冗談(じょうだん) しゃれ 
\\	口蓋(こうがい) 郊外(こうがい)
\\	軟口蓋(なんこうがい) 硬口蓋 (こうこうがい) 口蓋垂(こうがいすい) のどちんこ 
\\	運動場(うんどうじょう) ケイトは今運動場で走っている
\\	粗末(な)(そまつ) 粗末な家に住んでいる 粗末な紙
\\	中途で(ちゅうと) 家に帰る途中で 家に帰る途中
\\	増大する(ぞうだい) 大きくなる 増える(ふえます) 貿易が国富(こくふ)を増大させた
\\	吸い物(すいもの) スープ 
\\	吸い物を吸う スープを飲む
\\	スピーカー おかしいな、スピーカーがガーガー言ってるよ
\\	鎮圧(ちんあつ)する 警察がデモ隊(でもたい)を鎮圧した
\\	過去形(かこけい)
\\	物騒(な)(ぶっそう) 物騒な世の中(よのなか)だ 夜の一人歩きは物騒だ
\\	軍隊(ぐんたい) 軍隊に入る 入隊する(にゅうたい)
\\	つながる(つながります) 北海道と本州はトンネルでつながっている
\\	肥大する(ひだい) 扁桃腺肥大
\\	拒食症(きょしょくしょう)
\\	薬屋(くすりや) 薬局(やっきょく)
\\	狼(おおかみ) 狼が五匹いる
\\	溝(みぞ) 溝を掘る(ほります)
\\	塩辛い(しおからい)
\\	宝箱(たからばこ) 宝(たから)
\\	爬虫類の動物 爬虫類(はちゅうるい) 爬虫類が大嫌いです
\\	体の調子 (からだのちょうし)
\\	賃貸(ちんたい) 賃貸アパートを探している
\\	誤る(あやまります) 謝る(あやまります)
\\	間違う(まちがいます)  錯誤に陥る(さくごをおちいります)
\\	さじ 彼女はほんの些事にも心を乱す
\\	雲(くも) 蜘蛛(くも)
\\	旧市街(きゅうしがい)
\\	愛情(あいじょう) 
\\	愛(あい) 
\\	親子, 
\\	恋愛(れんあい) 
\\	恋(こい) 
\\	子供に対する親の愛情は何よりも強いです 二人の愛情はますます深くなっていました
\\	ジム 体育館(たいいくかん)
\\	竜巻(たつまき) 竜巻で村全体が破壊された(はかい)
\\	ねじ回し(ねじまわし)
\\	ブラウス
\\	腐敗する(ふはい) 腐る(くさります) 冷凍庫は食品の腐敗を守る わが国の政治家の道徳(どうとく)は腐敗した 腐敗を早める
\\	世界の終わり
\\	強制収容所(きょうせいしゅうようじょ) 収容所(しゅうようじょ)
\\	公立の(こうりつ) 公共の(こうきょう)
\\	公の(おおやけ) 
\\	公立図書館 公立学校
\\	主婦(しゅふ) 専業主婦(せんぎょうしゅふ) (職に就かないで、家事に専念する主婦)
\\	~ものだと思う 国際的なお付き合いは盛んになったものだと思う
\\	動詞(どうし)
\\	建国(けんこく) アメリカ合衆国建国の父
\\	ひじ
\\	(し)に 行く 神戸(こうべ)へ船を見に行く
\\	予定(よてい) 
\\	計画(けいかく) 
\\	今日は予定がない
\\	おしっこ おしっこする
\\	湿っぽい(しめっぽい)
\\	暗い(くらい)
\\	湿っぽい話 湿っぽくなる
\\	マーガリン
\\	大通り(おおどおり) 大通りを通る(とおります)
\\	その時 その時、あの人に信用できないと分りました
\\	文字通りに(もじどおり) 彼女が言うことを何から何まで文字通りに取らないほうがよい
\\	流し(ながし) お皿を流しに置いてもらえますか
\\	カシミヤ/カシミア カシミヤのコート
\\	人物(じんぶつ) 歴史上の人物 どんな人物ですか
\\	煎餅(せんべい)
\\	蹄(ひづめ) 蹄鉄(ていてつ)
\\	おどおど(する)(の) 彼女の前に出る時おどおどする
\\	新作(しんさく) 新作を発表する
\\	何日(なんにち) 夏休みは何日ぐらい取れますか
\\	詳細(しょうさい) 詳細(な) 詳しい(くわしい) それを詳細に説明しなさい
\\	(俗)	マジで?
\\	山登り(やまのぼり) 登山(とざん)
\\	飼い主(かいぬし) 私は飼い主を食べてしまった犬の話を読んだのを覚えている
\\	水平(な/の)(すいへい) 垂直(な/の)(すいちょく)
\\	(電気) 
\\	導体(どうたい) 胴体(どうたい)
\\	半導体(はんどうたい) 導体にする
\\	服(ふく) 衣服(いふく) そのお金がすべて衣服に使われた
\\	危ない(あぶない) 危うい(あやうい)
\\	危険(きけん)
\\	芸術(げいじゅつ) 美術(びじゅつ) 芸術的 芸術家(げいじゅつか) 芸術院
\\	すし詰め(すしづめ) 満員(まんいん) 込んでいる(こんでいる) すし詰めの電車で行った
\\	せっせと 一生懸命(いっしょうけんめい) アリのようにせっせと働く 彼はせっせと試験準備をしている
\\	裁判官(さいばんかん) 裁判(さいばん) 裁判所(さいばんしょ)
\\	コーナー スパイス・コーナー
\\	役に立つ(やくにたつ) この機械はとても役に立つ
\\	見本市(みほんいち) 国際見本市(こくさいみほんいち) その見本市には注目に値(あたい)するものは何もなかった
\\	その間に(そのあいだに)
\\	漢学(かんがく) サイノロジー
\\	せいぜい 
\\	せいぜい20ドルしか払わない
\\	しゃべる (しゃべります) シャベル 
\\	少女(しょうじょ) お嬢さん(おじょうさん)
\\	エッグスタンド
\\	修理工(しゅうりこう) 自動車修理工
\\	傲慢(な)(ごうまん) あの婦人はひどく傲慢です
\\	マレーシア
\\	南瓜(かぼちゃ)
\\	直後に(ちょくご) 夏休みの直後に試験だ 直前に(ちょくぜん)
\\	絆(きずな) その兄弟の間には強い絆がある 家族の絆
\\	一方(いっぽう)
\\	演劇(えんげき) 演劇を見に行く
\\	著作権(ちょさくけん) その本の著作権を所有している
\\	マーチ 試合(しあい)
\\	犬猿の中である (けんえんのなか) 彼女と彼が犬猿の中です
\\	しばしば よく 度々(たびたび)
\\	モルディブ
\\	隙間風/透き間風 (すきまかぜ) 隙間風が入る(はいります) き間風の来る所に座る そのことがあってから, 二人の間にすき間風が吹き始めた
\\	式服(しきふく) フォームルウエア 平服(へいふく) カジュアルウエア
\\	儒教(じゅきょう)
\\	~を 凍らせる(こおらせます) ~が 凍る(こおります) 水を凍らせた
\\	肉屋(にくや)
\\	包装紙(ほうそうし) 風呂敷(ふろしき) 包装する 包む(つつみます)
\\	開始する(かいし) 始める/始まる 両国は平和交渉を開始した 試合開始は午後2時
\\	翼(つばさ) 羽(はね)
\\	翼を広げる 想像の翼を広げる
\\	頭がいい 知能的(ちのうてき)/知能が高い
\\	広い庭 広い庭があるうちが欲しいです 
\\	広い庭 広い庭があるうちが欲しいです
\\	背中(せなか)
\\	夜遅く(よるおそく) リンダは夜遅く帰ってきた
\\	接吻する(せっぷん)
\\	キスする 彼は私におやすみ(お別れ)の接吻をした
\\	馬車(ばしゃ)
\\	長引く(ながびきます) 質問が次々と続いてインタビューが長引いた
\\	蜂蜜(はちみつ) 蜂蜜酒(はちみつしゅ)
\\	設備(せつび) 施設(しせつ)
\\	この病院は設備がよい 暖房設備(だんぼうせつび)
\\	学歴(がくれき) 彼女はいつも自分の学歴を自慢している(じまん)
\\	提言する(ていげん) 専門家による提言 都市計画に関する提言
\\	二十日大根(はつかだいこん)
\\	曇りです 曇っている(くもる)
\\	変な味(へんなあじ) この魚は少し変な味がする
\\	ずいぶん(な)
\\	大変(たいへん) かなり 
\\	ずいぶん背が高くなった
\\	食欲(しょくよく)
\\	クリスマスカード
\\	無害(な)(むがい) この島ではたいていの蛇は無害である
\\	(では)おげんきで。
\\	砕く(くだく) 砕ける(くだけます) 砕けた皿 落とした鏡が粉々に砕けた
\\	毛織物(けおりもの) スコットランドは毛織物で有名です 私達は冬に毛織物を着る
\\	馬鹿は死ななきゃ治らない (ばかわしななきゃなおらない)
\\	よく知られている
\\	知的障害者(ちてきしょうがいしゃ)
\\	演説(えんぜつ) スピーチ 彼の演説に数々のすばらしい言い回しが含まれて医ら
\\	校長 (こうちょう)
\\	レーズン 干しぶどう(ほしぶどう)
\\	カバー 
\\	カバー
\\	確か(たしか) 確か本と傘が入っています
\\	下着(したぎ)
\\	研ぐ(とぎます) 磨く(みがきます) 削る(けずります) 
\\	ナイフを研ぐ 包丁を砥石で研ぐ
\\	水道(すいどう) 水道の水 料金を払わなかったので水道を止められた 水道が出ています
\\	思い付く (おもいつきます) 私は彼が新しいよい考えを思い付くと思う
\\	雀蜂(スズメバチ) すずめばちはこっぴどく刺す(さす)ことが出来る
\\	お手伝い(さん)
\\	ナトー 北大西洋条約機構 (きたたいせいようじょうやくきこう)
\\	北米(ほくべい)
\\	花輪(はなわ) 花輪を作る
\\	握手する(あくしゅ) 固い(かたい)握手 握手を交わす(かわします)
\\	伝染病(でんせんびょう) 伝染病が広がる
\\	亡くす(なくす) 落とす(おとします) 
\\	カメラを亡くした
\\	間違える(まちがえます)
\\	間違う(まちがいます)
\\	誤る(あやまります)
\\	電話番号を間違えてしまった
\\	チベット チベットに自由を!
\\	街角(まちかど) 街角の交番 街角で彼と出会った
\\	偉そう(な)(えらそう) 偉そうな名前 偉そうに言う 偉そうにふるまう
\\	首(くび)
\\	憎しみ(にくしみ) 憎悪(ぞうお) 嫌い(きらい) 彼は父親に憎しみを抱いている
\\	公衆電話(こうしゅうでんわ) 公衆
\\	消える(きえます) 火が消えた
\\	スリッパ
\\	正確(な)(せいかく) 性格(せいかく)
\\	正確さ 正確に知らない
\\	絵筆(えふで)
\\	背丈(せたけ)
\\	身長(しんちょう)
\\	長さ 
\\	彼はお父さんと同じぐらいの背丈だ コートの背丈
\\	元(もと) 
\\	はさみを使ったら、元の所を戻しておいてください けんかの元
\\	予測する(よそく)
\\	何が起こるかだれも予測できない 実験は2か月かかると予測している
\\	戦争(せんそう)する
\\	失礼(しつれい)
\\	~を 患う(わずらいます) 病気である 病む(やみます) 最近、肺結核(はいけっかく)を患う人はほとんどいない
\\	幸せ(な)(しあわせ) 幸福(な)(こうふく) 幸運(な)(こううん) 幸い(な)(さいわい)
\\	千尋(の)(せんじん) 千尋の谷(たに)
\\	身障者(しんしょうしゃ) 視聴者(しちょうしゃ) 
\\	初心者(しょしんしゃ)
\\	火花(ひばな) 火花が出る
\\	消耗する(しょうもう) 電力を消耗する
\\	醜い(みにくい) この町には非常に醜いところがいくつがある
\\	おや! おや、今朝は早いですね おや,まあ
\\	性差別(せいさべつ) 男女差別(だんじょさべつ)
\\	カンニングする
\\	風が強い 強風(きょうふう)
\\	絞首台(こうしゅだい) 断頭台(だんとうだい) 犯罪人は絞首台に送られた
\\	墓地(ぼち)
\\	真っ白(な)(まっしろ) 真っ白な歯 見渡す限り、野原は真っ白だった
\\	上から三段目(さんだんめ)
\\	床屋(とこや) 美容師(びようし)
\\	~という 高橋という人電話があった
\\	影響力のある 有力(な)(ゆうりょく)
\\	食器棚(しょっきだな) 食器(しょっき)
\\	猫の額(ねこのひたい) 猫の額のような部屋
\\	マラソン マラソンの選手 医者は私にマラソンに参加することを許可しなかった
\\	坂(さか) 我々は坂をスキーで滑り降りた(すべりおりた) 道は急に下り坂(さかりざか)・上り坂(くだりざか)になっている
\\	抱く(だきます) 姉妹は抱き合った 赤ん坊は母親に抱かれて眠っている
\\	折角(せっかく) 
\\	せっかくのご招待ですから喜んで伺います せっかくですが,今回はお伴できません
\\	陰暦(いんれき)
\\	地道に(じみち) 地道な努力をする 地道に働く
\\	一昨昨日(さきおととい)
\\	酔っ払い運転 (よっぱらいうんてん)
\\	(キリストの)福音(ふくいん) この知らせは福音であった 彼らは福音を全世界に広めた
\\	目出し帽(めだしぼう) 
\\	ボーナス ボーナス全部を買い物と旅行に使ってしまった
\\	ペンギン
\\	一等賞(いっとうしょう) 彼は100m競争で一等賞を取った
\\	ごま
\\	枝(えだ) 大枝 (おおえだ)  小枝 (こえだ)
\\	ヨーロッパ旅行(りょこう)
\\	ペニス 陰茎 (いんけい) ちんちん 
\\	パズル そのパズルを解くのは難しい
\\	史跡(しせき) この辺りは鎌倉時代の史跡が多い
\\	ウラン ウラン(の)分裂 ウラン濃縮(のうしゅく)
\\	~に 重点を置く(じゅうてん) ~を 重視する(じゅうし)
\\	我が校は科学に重点を置いている
\\	大気(たいき) 大気圏(たいきけん)
\\	雰囲気(ふんいき)
\\	爆撃(ばくげき)する 爆弾を落とす(ばくだん) 市の中心部を爆撃した 爆撃機
\\	元素(げんそ) 原子(げんし) 鉄や酸素は元素です
\\	だからと言って  (わけではない/とは限らない(かぎらない)) 彼は若い。経験も浅い。だからと言って教えられないわけでわない・とは限らない。
\\	帽子(ぼうし) 帽子を取る
\\	日射病(にっしゃびょう) 日射病にかかる 日射病から回復する
\\	並外れる(なみはずれます) 群を抜く(ぐんをぬく) 彼女は並外れた記憶力を持ってる
\\	海んとこ 
\\	海のところ
\\	へえ!
\\	感動的(な)(かんどうてき) あの映画はとても感動的だった
\\	単に(たんに) 彼は単に冗談としてをそれを言った
\\	再生する(さいせい) トカゲの尾が再生した 音声の再生
\\	控える(ひかえます) 
\\	遠慮する(えんりょ) 医者は患者にワインを控えるように命令した
\\	キャンプファイヤ キャンプファイヤをする
\\	色々(な)(いろいろ)
\\	生意気(な)(なまいき)
\\	厚かましい(あつかましい)
\\	まったく生意気だ 彼の生意気な態度は我慢できない
\\	手をつないで歩く
\\	バーレーン
\\	~を 調印(ちょういん) 
\\	封印する(ふういん)
\\	署名する(しょめい) 
\\	サインする 
\\	条約に調印する 協定に調印する
\\	スイッチ テレビのスイッチを入れる ラジオのスイッチを切る
\\	警察(けいさつ)
\\	じゃあ
\\	上司(じょうし) 
\\	目上の人(めうえ)
\\	国歌(こっか) 国家(こっか)
\\	国花(こっか)
\\	半旗(はんき) 半旗を掲げる(かかげる)
\\	玉ねぎ(たまねぎ)
\\	ひざの屈伸 (ひざのくっしん) (運動) ひざの屈伸運動を2, 3回する
\\	悲観(ひかん) 悲観的(ひかんてき) 彼は悲観的な人生観を持っている 彼は前途を悲観していた
\\	穏やか(な)(おだやか) この道は右へ穏やかにカーブしている
\\	侍(さむらい)
\\	畑(はたけ)
\\	町中(まちなか) いとこは私を町中案内してくれた
\\	蓮(はす) 蓮の花
\\	月影(げつえい)
\\	軋る(きしります) 新しい靴のきしる音 ドアがきしりながら開いた
\\	奴(やつ) 何てひどい奴だ
\\	切り倒す(きりたおす) 木こりは木を斧で切り倒す
\\	一応(いちおう) 
\\	一応の合意
\\	伝える(つたえます) 奥さんにどうぞよろしくお伝えください 「あしたは授業がない」と彼女に伝えてください
\\	~に 対して(たい) ~に 対し (書) 鈴木さんは皆に対しても丁寧だ 女性の地位は低いのに対して、ヨーロッパでは比較的高い
\\	貧血(ひんけつ) 医者は貧血の検査をするためサンプルを分析した
\\	糖質(とうしつ)
\\	積荷(つみに) 積む(つみます) 積荷する
\\	太股(ふともも) 股(もも)
\\	眼鏡をかける あの眼鏡をかけている人は山田さんです
\\	年下(としした)
\\	クロアチア
\\	往来(おうらい) 
\\	交通(こうつう) 
\\	ここは車の往来が激しい(はげしい)
\\	空港(くうこう) 飛行場(ひこうじょう)
\\	明け方(あけがた) 日の出(ひので) 赤ちゃんは明け方近くに生まれた
\\	多細胞生物(たさいぼうせいぶつ) 単細胞生物(たんさいぼうせいぶつ)
\\	あり塚(ありづか) 
\\	ありの巣(す) 
\\	お前だ!
\\	すずめ すずめが飛び回ります (とびまわる)
\\	慣れる(なれます) ~に
\\	きつつき
\\	だぶだぶ(な) 緩い(ゆるい) 
\\	だぶだぶなズボン
\\	合唱団(がっしょうだん) クワイア 合唱団で歌う
\\	先ず(まず)
\\	ウェスタン 西部劇(せいぶげき)
\\	二つ三つ(ふたつ みっつ) 一人二人 三四日(さんよっか)
\\	偏る(かたよりいます) 何かに偏る あの人の知識は偏りすぎている
\\	鈍感(な)(どんかん) エリートたちは批判に対して鈍感じゃない
\\	悲鳴(ひめい) うれしい悲鳴 悲鳴をあげる わめく 叫ぶ(さけびます) 大きな声で言う
\\	苔(こけ) 常用外
\\	練習長 (れんしゅうちょう)
\\	漢方(かんぽう) 漢方薬(かんぽうやく)
\\	平野(へいや) 平野がすくないです
\\	もっと後に(のち) 後で(あとで) もっと後にそのことを説明する
\\	制服(せいふく) 学校の制服
\\	防水(ぼうすい)(の) 防水時計(ぼうすいどけい)
\\	使い方を教えて下さい
\\	歩いて行く(あるいていく)
\\	ウインドーショッピングする
\\	食用の(しょくよう)/食用になる 食用にならない このきのこは食用になりますか
\\	更年期(こうねんき) 更年期障害(こうねんきしょうがい) 更年期を迎える
\\	近所(きんじょ) 隣(となり)
\\	懸賞 (けんしょう)
\\	検証(けんしょう) 
\\	褒美(ほうび)
\\	1文学賞(ぶんがくしょう)
\\	生える(はえます) きのこが生える
\\	木の幹(みき)
\\	お上がりください(おあがり)
\\	ユリウス暦(れき)
\\	メーカー サイズはメーカーによって違う
\\	トルコ
\\	シャワーを浴びる(あびます) 水を浴びる
\\	~が 賭ける(かけます) 掛ける(かけます)
\\	欠ける(かけます)
\\	競馬に金をかける あの男がやったにきまっている
\\	かけてもいい どちらが勝つかかけようじゃないか
\\	知っている(しる)
\\	学友(がくゆう)
\\	イケメン 男前(おとこまえ) 
\\	ハンサムな男
\\	~が 嫌いです(きらい) 私が/の嫌いな「物」は納豆です
\\	麺棒(めんぼう)
\\	変装する(へんそう) その泥棒は叔母さんに変装してた
\\	多数(たすう) 多数の友達
\\	ビニール袋(ぶくろ)
\\	~を 侮辱する(ぶじょく) あなたが彼を侮辱したので、彼はあなたに腹をたてています
\\	画鋲(がびょう) 押しピン(おし)(中部地方)
\\	「。。。」と書いてある
\\	でたらめさ でたらめを言う でたらめな話
\\	援助(えんじょ)
\\	支援(しえん)
\\	あなたが援助してくれば、私たちの計画は成功する
\\	胸(むね) 乳癌(にゅうがん)
\\	声援(せいえん)を送る(おくります) その野球の試合の観客(かんきゃく)は自分達のチームに声援を送った
\\	げじげじ眉(まゆ)(俗) 眉が太い げじげじ眉をしている
\\	力士(りきし)
\\	では では,また来週 ではどうしたらよいと思いますか
\\	子馬(こうま)
\\	兎(うさぎ) 常用外
\\	時(とき)
\\	時間(じかん)
\\	時刻(じこく)
\\	彼は時刻をたずねた
\\	全国(ぜんこく) 全国テスト
\\	夢(ゆめ) 夢を見る
\\	定刻に(ていこく) 定刻に到着する 定刻に 
\\	分遅れる
\\	階(かい)
\\	る-
\\	う-
\\	一段動詞(いちだんどうし) 五段動詞(ごだんどうし)
\\	新型(しんがた) 新型の自動車 最新型の帽子
\\	下宿(げしゅく) 私は学校のそばに下宿しています 私は山田さんのところに下宿している
\\	おめでたい
\\	楽園(らくえん) ハワイは地上の楽園と呼ばれている
\\	パレット 
\\	調色板(ちょうしょくばん)
\\	家鴨(あひる) 鴨(かも) アヒルの群れ(むれ)
\\	ブラインド 彼女はブラインドを降ろした(おろす)
\\	読み書き(よみかき) すべての少年少女は読み書きを教えられる
\\	包装する(ほうそう) 放送(ほうそう)
\\	荷造りする(にづくり)
\\	包む(つつみます) これらの贈り物と一緒に放送してください 包装紙(ほうそうし)
\\	消火栓(しょうかせん)
\\	爪先(つまさき) つま先で歩く
\\	ルビー 紅玉(こうぎょく)
\\	エレベーター エレベーターに乗る
\\	板紙(いたがみ)
\\	議論する(ぎろん) その問題は 議論する価値がない
\\	(お)およそ 
\\	一般に(いっぱん)
\\	およそ両親が子供を愛情する およそ親というものは子の幸せを願うものである
\\	頭(あたま) 頭(かしら)
\\	留守番電話 (るすばんでんわ) 彼の留守番電話に用件を吹き込んだ
\\	手荷物(てにもつ) 彼はトラックに手荷物を積んだ(つむ)
\\	豊満(な)(ほうまん) 豊満な胸
\\	人称代名詞 (にんしょうだいめいし)
\\	落語(らくご) 落語をやる
\\	紫(の)(むらさき)
\\	生活費(せいかつひ)
\\	物価(ぶっか)
\\	東京は生活費が高い 生活費をかせぐために働く
\\	蒸気機関車 (じょうききかんしゃ) 汽車(きしゃ) 
\\	機関車(きかんしゃ)
\\	周辺(しゅうへん) その州は周辺に様々な原料を供給している 町の周辺に
\\	疑う(うたがいます) 疑問する(ぎもん)
\\	危ぶむ(あやぶみます)
\\	彼らは彼の話を疑った 彼の成功を疑う
\\	相当(そうとう) 
\\	相当の金額だ それは相当難しそうだ 彼は相当な暮らしをしている
\\	重要(な)(じゅうよう) 
\\	大切(な)(たいせつ) 
\\	大事(な)(だいじ) 
\\	重大(な)(じゅうだい)
\\	この辞書の全ての単語が重要です
\\	真っ赤(な)(まっか) 彼は顔を真っ赤にして怒った
\\	餌(え/えさ) 
\\	生餌(いきえ) 釣り針[わな]に餌をつける
\\	自治省(じちしょう) 自治(じち)
\\	全部で500円です
\\	点滴(てんてき)
\\	点滴をする 点滴を受ける
\\	科学(かがく) 化学(かがく)
\\	何しろ(なにしろ) とにかく 何しろ話してごらん 何しろやってみないことにはわからない
\\	大名(だいみょう) 大名旅行をする 
\\	通る(とおります) 駅の前を通る 大通りを通る 道を通った人
\\	成人する(せいじん) 成年に達する 
\\	大人になる ヘンリーは今年の三月に成人する
\\	~を 忘れる(わすれます)
\\	一つは...、もう一つは... 一つは白で、もう一つは黒だ
\\	騒ぐ(さわぎます) あの酒を飲んで騒いでいる人はだれですか
\\	遠慮する(えんりょ) タバコを遠慮する
\\	人情がある(にんじょう) 人情味がある(にんじょうみ)
\\	彼は人情がない 彼がしたことは人情にもとる(もとる-
\\	人情の厚い/薄い人だ(あつい/うすい)
\\	噂(うわさ)
\\	デマ 
\\	伝聞(でんぶん)
\\	鶏冠(とさか)
\\	小脳(しょうのう)
\\	成功(せいこう) 成功する
\\	大人しい(おとなしい) 素直(な)(すなお)
\\	猫は彼のひざの上に大人しく座っていた
\\	親日の(しんにち) 親日家(しんにちか) 親日派(しんにちは) 反日の(はんにち)
\\	柳 (やなぎ) 常用漢字
\\	掛け替えがない(かけがえ) 彼女は、僕にとって掛け替えがない人だ 掛け替え(かけがえ)
\\	結ぶ(むすびます) 手を結ぶ
\\	絵画(かいが)
\\	絵(え)
\\	絵画を鑑賞する(かんしょう) 絵画的な美しさ
\\	浮浪者(ふろうしゃ) 
\\	路上生活社(ろじょうせいかつしゃ) 
\\	市は浮浪者に毛布を供給した(きょうきゅう)
\\	スタートする その走者はうまいスタートをした 父の援助で商売をスタートした
\\	声(こえ)
\\	無知(むち) 鞭(むち) 
\\	政治に関しては全く無知だ 学べば学ぶほど、ますます自分の無知であることが分かる
\\	キロ(グラム)
\\	形式(の)(けいしき) 正式(せいしき)
\\	堅苦しいこと(かたくるしい)
\\	一般に若者は形式を嫌う
\\	ハンドウイルカ 半道海豚(はんどういるか)
\\	クール(な) クールな性格 クールな態度
\\	なまり 鉛(なまり)
\\	彼の日本語は外人なまりがある[ない] 彼はやや田舎なまりでしゃべる 彼hのなまりから考えれば、九州の出身に違いない
\\	面談(めんだん) 
\\	面接(めんせつ) 
\\	インタビュー 
\\	この件について面談したい 親子面談〔学校の〕
\\	切腹(せっぷく) 腹切り(はらきり)
\\	ロープウエー
\\	逃げる(にげます) 去る(さります) 逃亡する 
\\	彼らは四方八方に逃げた
\\	幼い(おさない) この物語の女主人公(じょしゅじんこう)は幼い少女です
\\	泥酔(でいすい) 泥酔している 彼は完全に泥酔していた
\\	~を 参すにする(さんこう) 彼女の本を参考にした 参考のために教えてほしいのですが,このコンピュータはいくらでしたか
\\	竹の子(たけのこ)
\\	思い出す(おもいだします) 私は家族を思い出した。
\\	ちゃんと あなたはそこへちゃんとした衣服で行ったほうがいい 家賃をちゃんと支払う
\\	晩御飯(ばんごはん) 夕食(ゆうしょく)
\\	観覧する(かんらん) 神社の宝物が今観覧に供されている (きょうする) 観覧車
\\	回す(まわす) 渡す(わたす) これを次の人に回してください
\\	隠喩(いんゆ) 「怒ると彼はとらになる」というのは隠喩の例です
\\	しらみ しらみがわく しらみをとる しらみの卵
\\	白いの(しろい) 白いのは、かわいい
\\	嗜好品(しこうひん)
\\	恋しい(こいしい) 寂しい(さびしい) 寂しい(さみしい) 
\\	私の恋しい人 故郷が恋しい
\\	直線(ちょくせん) この二つの直線は直角に交わっている
\\	クリスタル・ガラス
\\	真珠(しんじゅ) それは豚に真珠だ
\\	連続(れんぞく) 連続する出来事 連続8時間の労働 3日間連続して雨が降った
\\	天体(てんたい) 水星は小天体(しょうてんたい)
\\	積もり(つもり) 学校へ行くつもりです
\\	炭水化物(たんすいかぶつ) 炭水化物の少ない食事療法
\\	(薬)	粉薬(こなぐすり)
\\	空気パージ(くうき)
\\	ラップトップ
\\	よく しばしば
\\	銅(どう) 
\\	赤金・銅(あかがね) 銅像(どうぞう)
\\	宛名(あてな) 住所(じゅうしょ)
\\	宛名が違っている 宛名不明
\\	一般の(いっぱん) 一般の本屋 一般の人 一般人(いっぱんじん)
\\	孫息子(まごむすこ) 孫娘(まごむすめ)
\\	詐欺(さぎ) 詐欺を動く(うごきます) 騙す(だまします) 詐欺で金を巻き上げる(まきあげる)
\\	すり すりのカルテット
\\	羊歯(しだ)
\\	そもそも 元々(もともと) 初めから(はじめ) その答えがそもそも間違っていたのだ
\\	高山(こうざん) 高い山
\\	含む(ふくみます) オレンジにビタミン
\\	が多く含まれている
\\	勉学(べんがく) 彼は勉学にいそしんでいる
\\	完全主義者(かんぜんしゅぎしゃ) 凝り性の人(こりしょう)
\\	隣(となり) 隣の国 隣国
\\	ゴルフ
\\	ルール 
\\	規則(きそく)
\\	規制(きせい)
\\	それはルール違反だ
\\	張る(はります) ネットを張る  テントを張る
\\	惚ける(とぼけます) 彼はよくとぼけて人を笑わせた とぼけちゃ困るよ, そりゃ僕の金だ
\\	浴びる (あびます)
\\	よかったてすね
\\	あちこち あちらこちら あちこち電話する  あちこちの学校
\\	タイプを打つ(うつ) ミシンで書く
\\	(就職の)願書(がんしょ) インターネットで願書を取り寄せた(とりよせる)
\\	音符(おんぷ) 彼はピアノを音符を見ずに弾いた
\\	フラミンゴ
\\	軽い(かるい) 軽い昼食
\\	二年前に(にねんまえに)
\\	豊か(な)(ゆたか) 豊富(な)(ほうふ)
\\	この地方は鉄と石炭が豊かである 豊かな家に生まれた
\\	会社を辞める(やめます)
\\	凝り性の(こりしょう) 完全主義者(かんぜんしゅぎしゃ) 凝り性の人(こりしょう)
\\	港(みなと) シドニーは天然の美しい港がある
\\	控訴(こうそ) 地方裁判所の判決に対して控訴する
\\	中央(ちゅうおう) アルパス山脈はヨーロッパの中央にある
\\	注意深い(ちゅういぶかい) 科学は注意深い観察(かんさつ)に基づいている(もとづく)
\\	悪魔(あくま)
\\	火力(かりょく) このストーブは火力が強い
\\	大敵(たいてき) 湿気はピアノの大敵だ
\\	新婚夫婦(しんこんふうふ) その新婚夫婦は手をつないで歩いた
\\	学期末試験 (がっきまつしけん)
\\	年を取った人 老人(ろうじん)
\\	遅れる(おくれます) 遅くなる(おそく) 遅刻する(ちこく) 遅くなってすみません
\\	それは意味をなさない
\\	切断(せつだん) 足を膝(ひざ)から切断する  強風のため電線が数箇所で切断された
\\	頭上(ずじょう) 頭上には月と星が輝いていた(かがやく)
\\	パーティーで着る
\\	恵比寿 (えびす)
\\	岩(いわ) 岩石(がんせき)
\\	トラクター
\\	対策(たいさく) 対策を講じる(こうじます)/とります 交通事故の防止対策を講じなければならない
\\	ごみを出す どこにごみを出したらいいですか
\\	休養する(きゅうよう) 休憩する(きゅうけい)
\\	昨日は一日ゆっくり休養した
\\	零す(こぼします) ジュースをこぼす 塩をこぼす 涙をこぼす 一人もこぼさず卒業させる
\\	暇(な)(ひま)
\\	私書箱(ししょばこ)
\\	外国人恐怖症 (がいこくじんきょうふしょう)
\\	鈍い(にぶい) 鋭い(するどい) 鈍いナイフ 視力が鈍くなった
\\	放つ(はなちます) 放す(はなします) うさぎを野に放つ 大砲を放つ
\\	百足(むかで) げじげじ
\\	郵便配達(ゆうびんはいたつ) 配達(はいたつ)
\\	瞑想(めいそう) 彼は昨晩瞑想した時のイメージを描いた
\\	~を 要求する(ようきゅう) ~を 要する(ようする) とりわけ、科学用語には正確な定義が要求される
\\	膳(ぜん) 一膳(いちぜん) 箸(はし)
\\	入隊する(にゅうたい) 軍隊に入る 
\\	入隊者
\\	偉い(えらい) 偉大(な)(いだい) 偉い/偉大な人
\\	白熊(しろくま) 北極熊(ホッキョクグマ)
\\	紙表紙の(かみびょうし) ペーパーバック(の) 紙表紙の本
\\	~かもしれない 
\\	~でしょう 来週のテストは、楽かもしれない
\\	ホラー 恐怖映画(きょうふえいが)
\\	~起こす(おこします) 
\\	子供を起こす 子供が起きた
\\	孤独(こどく) 寂しさ(さびしさ) 孤独(な)(こどく) 寂しい(さびしい) 寂しい(さみしい)
\\	孤独な人 ときには孤独もまた楽しいものだ 孤独な日々を送る
\\	ノイローゼ 
\\	神経症(しんけいしょう)
\\	赤十字社(せきじゅうじしゃ) 赤十字社看護師
\\	麻薬(まやく) 麻薬犬(まやくけん)
\\	絵の具(えのぐ) 赤い絵の具に青い絵の具を混ぜなさい
\\	イチジク
\\	スクール 学校(がっこう) クッキングスクール 料理教室(りょうりきょうしつ)
\\	税務署(ぜいむしょ) 税務署員
\\	匿名(とくめい) 無名(むめい) 
\\	その人は匿名希望だ 無名の手紙
\\	銀河(系)(ぎんがけい) 天の川(あまのがわ) 宇宙にたくさんの銀河がある
\\	制度(せいど) システム 新制度は成功だと分かった
\\	オートバイ 私のオートバイは途中で故障した
\\	(お)およそ 
\\	約(やく)
\\	大ざっぱ(な)(おおざっぱ) 
\\	およそ50 おおよその日程を決める いくつぐらい必要なんだ. おおよその数字だけでも教えてくれ.
\\	鍋物(なべもの) 犬なべもの
\\	改革(かいかく) 政府は教育を改革している 宗教改革(しゅうきょうかいかく)
\\	美容師(びようし)
\\	自慰(じい)をする/にふける オナニーする
\\	スカンク
\\	ポン引き(ぽんびき)
\\	大文字(おおも(ん)じ)
\\	地質学(ちしつがく) 地質学者
\\	遠足(えんそく) 動物園に遠足に行く
\\	頭文字(かしらもじ) 木の幹に自分の頭文字を彫った
\\	民族の舞踊 (みんぞくのぶよう) 舞踊(ぶよう)
\\	新車(しんしゃ) 彼は私の新車を誉める(ほめる)
\\	振る(ふります)  
\\	次の漢字にかなを振りなさい
\\	おしゃべり お茶でも飲みながらおしゃべりしていただけませんか
\\	であろう (だろう、まい) その交渉(こうしょう)はさわめて困難(こんなん)であろう
\\	性器(せいき) 世紀(せいき)
\\	熱湯(ねっとう) (お)湯 
\\	私はカップに熱湯を注いだ
\\	ペンキ 壁をペンキで塗る
\\	義務(ぎむ) あなたを助けるのが私の義務だ 義務を怠る(おこたります)
\\	質問(しつもん)
\\	乗せる(のせます) この箱の上には、何も載せないで下さい この小説は何という雑誌に載せるのですか
\\	ホック 鉤(かぎ) つりばり
\\	居間(いま) 
\\	茶の間(ちゃのま) 
\\	選挙権(せんきょけん)
\\	パラシュート パラシュートが開いた
\\	アイスクリーム
\\	片道(かたみち) 片道切符 図書館まで片道は歩いて帰りはバスに乗った
\\	大雨(おおあめ) 豪雨(ごうう) 大雨が降った
\\	孤児(こじ) 故事(こじ)
\\	その孤児はお金持ちに育てられた
\\	副大統領(ふくだいとうりょう)
\\	長靴(ながぐつ)
\\	黒子(ほくろ)
\\	運搬車(うんぱんしゃ) トラック
\\	足取り(あしどり) 宇宙飛行士はしっかりした足取りで地球に降りたった
\\	極東(きょくとう)
\\	怪我人(けがにん) 負傷者(ふしょうしゃ) 負傷者が友達に支えられながら病院に入っていきました
\\	完ぺき(かんぺき)(な) ダンサーの一つ一つの動きは完璧だった 完璧さ(かんぺきさ)
\\	マンゴー
\\	増やす(ふやす) 知識を増やすためにたくさん本を読まなきゃ
\\	髪を結う(ゆいます) 彼女はいつも有名な美容師に髪を結ってもらう
\\	寄る(よります) 訪ねる(たずねる)
\\	訪れる(おとずれる)
\\	訪問する(ほうもん) 私の帰宅の途中で彼の家に寄るつもりだ
\\	(お)弁当(べんとう)
\\	違法(な)(いほう) 違法性(いほうせい)
\\	(星座)	牡牛座(おうしざ)
\\	ニンニクひとかけら 欠片(かけら)
\\	~の内臓を抜く (ないぞうをぬく) 魚の内臓を抜く
\\	一寸(ちょっと)
\\	署名する(しょめい) サインする ~に 名前を署名する
\\	~が 欠ける(かけます) 彼の食事にはカルシウムが欠けている
\\	木陰(こかげ) 木の影/陰(きのかげ) 木陰で 木陰に入る
\\	物置(ものおき) 物置に鍵をかけるのを忘れちゃった
\\	宝箱(たからばこ) 宝(たから)
\\	飛行(ひこう) 危険な飛行だった
\\	英雄(えいゆう) ヒーロー 国民的英雄 英雄の/英雄的
\\	持久(じきゅう) 忍耐(にんたい)
\\	持久力(じきゅうりょく)
\\	完全(な)(かんぜん) 私は彼を完全に信用しています
\\	上級(じょうきゅう)
\\	日雇い(ひやとい) 日雇いで働く
\\	他動詞(たどうし) 自動詞(じどうし) 人を並べる
\\	鯱(しゃち)
\\	水蒸気(すいじょうき)
\\	蒸気(じょうき)
\\	雲は蒸気のかたまりです
\\	花壇(かだん)
\\	四つんばい(よ(っ)つんがい)
\\	~を 鼻で嗅ぐ(かぎいます)
\\	修理(しゅうり) 直す(なおします) 修繕(しゅうぜん) この機械を修理できる/が直せる
\\	少数民族(しょうすうみんぞく)
\\	ハンガー 衣紋掛け (古) (えもんかけ)
\\	抱える(かかえます) 
\\	問題を抱える あの会社は借金を抱えている
\\	~に そっくり ~に よく似ている(にる) 瓜二つ(うりふたつ) 彼は父親にそっくりだ
\\	欧米(おうべい)
\\	西洋(せいよう)
\\	欧米風の考え方
\\	手綱(たづな) 馬の手綱をとる 手綱を引く 手綱を緩める(ゆるめます) 子供たちの手綱を締める
\\	栄える(さかえます) 15世紀のイタリアでは美術が栄えた
\\	コーンフレークス
\\	余韻(よいん) 戦争の余韻 パイプオルガンはやんだが,しばらく余韻が続いていた
\\	煩い!(うるさい) 静かに(しずかに) 黙れ(だまれ) 煩い (うるさい)
\\	化粧品(けしょうひん) あの店でもう化粧品は売っていない
\\	国務省 〔米国の〕(こくむしょう) 
\\	国務長官 〔ちょうかん〕
\\	ベトナム 越南(えつなん)
\\	ベトナム戦争
\\	~など 例えば(たとえば)
\\	宗教改革(しゅうきょうかいかく)
\\	降伏(こうふく) 幸福(こうふく)
\\	降参(こうさん)
\\	自首(じしゅ)
\\	彼は自発的に降伏した …という条件(じょうけん)で降伏する
\\	水着(みずぎ)
\\	霙(みぞれ) みぞれが降っている
\\	スツール スツールに腰掛ける
\\	円(えん) 八十円の切手
\\	買い物 (かいもの)する
\\	食料(しょくりょう) 食べ物(たべもの) 3日分の食料
\\	日当たりが良い(ひあたり) 日当たり 家の居間は日当たりが良い
\\	制定する(せいてい) 憲法を制定する あの国は国旗を制定したところだ
\\	強制的(きょうせいてき) 彼女は強制的に私を一緒に連れ出した
\\	新茶(しんちゃ)
\\	働く(はたらきます) 勤める(つとめます)
\\	彼は銀行で働きます
\\	深い(ふかい) 
\\	不快(な)(ふかい)
\\	大げさ(おおげさ) 誇張(こちょう) 大げさに言う 誇張して話す 大げさな身振りで
\\	~に 間に合う (まにあう) 電車に間に合った
\\	芥子の実(けしのみ)
\\	肝臓(かんぞう)
\\	再生可能エネルギー源 (さいせいかのうエネルギーみなもと)
\\	英文学(えいぶんがく) 彼は英文学の大家(たいか)
\\	排出(はいしゅつ) 二酸化炭素の排出(はいしゅつ)
\\	依存(いぞん) その鳥の経済は漁業に依存している 人に依存して暮らす それはすべて天候に依存している/にかかる
\\	伝統(でんとう)
\\	宝石(ほうせき) 宝石を身につける
\\	組み合わせる (くみあわせます) 両手を組み合わせる
\\	大騒ぎ(おおさわぎ) 首相が暗殺され,国中が大騒ぎとなった
\\	打ち合わせ(うちあわせ) 計画(けいかく)
\\	打ち合わせ通りに事が運んだ 打ち合わせをする 打ち合わせる(うちあわせます)
\\	鎧(よろい) 鎧をつけている人は、転ぶと大きな音がする
\\	夜行バス(やこう) 夜行性の動物 夜行する(やこう)
\\	車道の向こう側 (しゃどうのむこうがわ)
\\	教える (おしえます)
\\	瓜二つ(うりふたつ) 
\\	その二人の兄弟は瓜二つである
\\	どう思って?
\\	バクテリア 細菌(さいきん) 
\\	最近(さいきん)
\\	ドレス
\\	を 引退する(いんたい) 引退している 政界から引退する
\\	バスを降りる(おります) ここで降りて下さい
\\	開会する(かいかい) ここにオリンピック大会の開会を宣言します(せんげん)
\\	~を 勘当する(かんどう) 感動する(かんどう)
\\	息子を勘当する
\\	自動販売機 (じどうはんばいき)
\\	部族(ぶぞく) その部族は祖先(そせん)を崇拝している
\\	宿題を出す
\\	そんなの(は) そんなのは、あるの? そんなのは、あるかよ!
\\	ダンスパー
\\	~に 気をつける ~に 注意する(ちゅうい) スープに塩をたくさん入れると、塩からくなるから、気をつけてください
\\	鼠取り(ねずみとり) 罠(わな)
\\	ピーピー 小鳥がぴーぴー鳴いている そんなにぴーぴー言うな
\\	拭く(ふきます) 拭う(ぬぐいます) 顔をタオルでふいた
\\	おもちゃ屋   おもちゃ がんぐ
\\	真夏(まなつ)
\\	献血する(けんけつ) 私たちはその子供を救うために献血した 献血車(けんけつしゃ)
\\	衛星(えいせい) 衛生(えいせい)
\\	放送衛星 人工衛星
\\	白血病(はっけつびょう)
\\	減る(へります) 減少する(げんしょう) 体重が減っている
\\	現れる(あらわれます) 舞台(ぶたい)の右の方から、一人の漁師(りょうし)が現れた
\\	縁起が悪い(えんぎ) 2は英語では縁起の悪い数字です
\\	東洋人(とうようじん)
\\	~に 値する(あたい) この本は一読に値する(いちどく) この問題はこれ以上の論議に値しない
\\	滑る(すべります) 
\\	丘を滑り降りる 戸がよく滑る
\\	御曹司(おんぞうし) 相続人(そうぞくにん)
\\	領地(りょうち) 領土(りょうど) ~領(りょう) 
\\	韓国領(かんこくりょう) 陸地(りくち)
\\	夢中に/でする(むちゅう) 彼女は夢中で本を読んだ
\\	波(なみ) 波蘭(ポーランド)
\\	急死する(きゅうし) 心臓発作で急死する
\\	フライ返し(ふらいがえし)
\\	芸能人(げいのうじん) 芸能人ぶる
\\	マッコウクジラ マッコウクジラは1000メートルの深さまで潜ることができる
\\	ろば
\\	のんびり(とした) 心配のない 楽天的(らくてんて) 
\\	穏やか(おだやか) のんびりとした人 のんびりと暮らす
\\	モルモット
\\	冗談(じょうだん)を言う 冗談だろう!
\\	味(あじ) 変な味がする
\\	週末(しゅうまつ)
\\	によると によって 友達の話によると、朋子はやっと彼を見つけたらしい
\\	巻尺(まきじゃく)
\\	類義語(るいぎご) 類語(るいご) ~の類語である 類語をあげる
\\	前に(まえに) もっと早く 前に本で読んだ
\\	初めて(はじめて)
\\	出身地(しゅっしんち) どの学生も名前と出身地を聞かれた
\\	送る(おくります) 出す(だします) 送信する(そうしん) 
\\	恐らく(おそらく) おそらく来ないでしょう
\\	中級(ちゅうきゅう)
\\	~ば よかった あの本を読めばよかった
\\	缶(かん) 缶は空っぽです 缶詰の(かんづめ) 缶詰の食べ物
\\	革ひも(かわひも) 綱(つな) 革ひもでつないで犬を散歩させる
\\	生姜(しょうが)
\\	(諺)
\\	昼には目あり、夜には耳あり 
\\	ボランティア活動(かつどう) 彼はボランティア活動に専念した
\\	お正月(おしょうがつ)
\\	そこで さてそこで問題です
\\	だけど しかし 
\\	このコートは素敵だけど高価すぎる
\\	ニホンザル
\\	軽蔑する(けいべつ) 我々は彼を臆病(おくびょう)だといって軽蔑していた
\\	頭文字(かしらもじ) 木の幹に自分の頭文字を彫った 頭文字のついているハンカチ
\\	非常(ひじょう) 非常の場合には、ここから出てください 試験は非常に難しかった
\\	仲良し(なかよし) 親友(しんゆう)
\\	大の仲良し
\\	お凸(おでこ)(俗) 額(ひたい) おでこをさする 彼女はおでこが広い[狭い]
\\	魚の目(うおのめ) たこ 蛸(たこ)
\\	凧(たこ)
\\	修士(しゅうし) 修士の学位(がくい) 彼は法学修士の学位を得た(えた)
\\	~に 気が付く(きがつく) 私は時計が壊れているのに気が付いた
\\	鉄(てつ) 鉄分(てつぶん)
\\	~だけ 一週間だけ休む 一週間しか休まない
\\	抗鬱剤 (こううつやく)
\\	失敗(しっぱい) すべての努力は失敗に終わった
\\	お腹が空いた時 店が休みのとき 暇な時
\\	大根(だいこん)
\\	腎臓(じんぞう) 腎臓が悪いです
\\	引き分け(ひきわけ) その試合は6対6引き分けに終わった
\\	臨時(な)(りんじ) 臨時教員
\\	追う(おいます) 最新の流行を追うのは大変お金がかかる 彼は警察に追われている
\\	時刻表(じこくひょう) ダイヤ 電車時刻表
\\	ヒンドゥー教 ヒンズー教 
\\	ブーツ
\\	~を 要する(ようする) ~を 要求する(ようきゅう)
\\	間違いを直す(なおす) 先生、日本語で手紙を書いたんですが、間違いを直していただけませんか
\\	ドリル ドリルで穴を開ける
\\	ゆっくり ゆっくり休んでください
\\	陳述(ちんじゅつ) 陳述書(ちんじゅつしょ) 弁護士は詳しい陳述をした
\\	~を 汚す(よごします) 汚れる(よごれます
\\	有毒(な)(ゆうどく)
\\	火口(かこう) 河口(かこう)
\\	勤める(つとめます) 働く(はたらきます)
\\	彼女は病院に看護師として勤めている
\\	共同する(きょうどう)
\\	協力する(きょうりょく)
\\	共同で店を経営する
\\	歌詞(かし) 菓子(かし)
\\	車の免許(めんきょ)
\\	ほっそりした スマートな 腰のほっそりとした娘
\\	証券(しょうけん) 株(かぶ) 
\\	証券会社(しょうけんがいしゃ)
\\	オタク お宅(おたく)
\\	鼻面(はなづら)
\\	めっきり めっきり涼しくなった
\\	読書家(どくしょか)
\\	空ける(あけます) 開ける(あける)
\\	箱を空ける 席を開ける 部屋を開ける
\\	問題(もんだい)
\\	すっかり このあたりはすっかり変わってしまった
\\	財布(さいふ) どこかで財布を落としてしまった
\\	垣(かき) 塀(へい) 垣根(かきね)
\\	電気屋(でんきや)
\\	霞(かすみ) 霧(きり) 霞む(かすみます)  年をとって目が霞んだ
\\	梅(うめ), プラム 梅の花は3月に咲く(さく)
\\	反対する(はんたい)   反対に、... 私はこれに反対する
\\	一年生(いちねんせい) 新入生(しんにゅうせい) 一年生の中ではだれが一番髪が長い
\\	冷暖房(れいだんぼう) 
\\	冷房(れいぼう) 
\\	クーラー 
\\	エアコン 
\\	一環(いっかん) 
\\	それは一連の事件の一環に過ぎない
\\	末っ子(すえっこ)
\\	万引する(まんびき) 万引きするところを見つかる
\\	電話室(でんわしつ) 電話ボックス
\\	堅苦しい(かたくるしい) ずいぶん堅苦しい人だ 堅苦しいことはやめよう
\\	長崎(ながさき) 長崎が一番好きです。
\\	中華料理(ちゅうかりょうり)
\\	こうのとり
\\	手品(てじな) 手品を使う 彼はトランプの手品が上手だ
\\	芸者(げいしゃ)
\\	超える(こえます) 超過する(ちょうか)
\\	私は文化の壁を越える方法を知っています
\\	水タバコ(みずたばこ) 水パイプ(みずパイプ) 水タバコを吸う
\\	ミュージシャン 彼女意見では、今まで見た中で彼が最高のミュージシャンです
\\	軍靴(ぐんか) 
\\	木苺(きいちご)
\\	未亡人(みぼうじん)
\\	生徒(せいと) 学徒(がくと)
\\	肩甲骨(けんこうこつ)
\\	いちゃいちゃする 私は昨日彼らが喫茶店でいちゃいちゃしているのを見た
\\	ゆっくり ゆっくり話してください
\\	独り善がり(な/の) (ひとりよがり) 生意気(な)(なまいき)
\\	彼はいつも独り善がりだ
\\	比較(ひかく)
\\	踊る(おどります) ダンスする
\\	しんどい 
\\	面倒 めんどうくさい  この資料を全部読むのはしんどい しんどい仕事
\\	路肩(ろかた)
\\	海綿(かいめん) スポンジ
\\	飛ぶ(とびます)
\\	処方せん(しょほうせん)   処方箋を書く 薬局にこの処方箋を持っていってください
\\	行事(ぎょうじ) 日本の行事たいてい中国から来ました
\\	言語学(げんごがく)
\\	外部(の)(がいぶ)
\\	外側(そとがわ)
\\	屋外(おくがい)
\\	外部から 建物の外部 外部からの刺激
\\	滑稽(な)(こっけい) 滑稽なことを言う
\\	引っくり返す(ひっくりかえす) クレープをひっくり返す アルバムを引っ繰り返してこの写真を見つけた
\\	民間伝説 (みんかんでんせつ) 「民間伝説」といのはどういうことですか
\\	さあ さあ、行こう! さあ、知りません
\\	テキスト 
\\	教科書(きょうかしょ) テキストの10ページを開いて下さい
\\	流入(りゅうにゅう) 移民の流入 流入する
\\	小物(こもの) 
\\	小物入れ
\\	太平洋(たいへいよう)
\\	饂飩(うどん)
\\	蕎麦(そば) 
\\	快適(な)(かいてき) 居心地がいい 
\\	快適なベッド
\\	キルギス
\\	闇市(やみいち)
\\	地震(じしん) 自身(じしん)
\\	自信(じしん)
\\	人前に出る(ひとまえにでる) 俳優は人前に出る
\\	大切(な)(たいせつ)
\\	重要(な)(じゅうよう)
\\	大事(な)(だいじ)
\\	重大(な)(じゅうだい)
\\	空っぽ(な/の)(からっぽ) 頭のからっぽな人 からっぽの箱
\\	赤っ恥(あかっぱじ)
\\	恥(はじ)
\\	赤っ恥をかく 赤恥をかかせる
\\	たくましい 村の女たちはたくましい体つきをしていた
\\	ランプ ランプが付けてあります
\\	音(おと) このピアノはいい音がする
\\	直(な)(ちょくせつ) 私は直接にその情報を手に入れた
\\	囚人(しゅうじん) 囚人服(しゅうじんふく)
\\	コンピューター パソコン パソコンで書く
\\	加える(くわえます) 足す(たします) 追加する(ついか)
\\	3に4を加えると7だ 危害を加える
\\	ステンレス製のスチール ステンレス製の鉄鋼(てっこう)
\\	そっと 彼は私のポケットにそっとお金を入れた
\\	沖(おき) 海岸から約1マイル沖に漁船が見えた
\\	無くなる(なくなります) その中のいくつかはもう無くなりました
\\	支持する(しじ) 支える(ささえます)
\\	できるだけ君を支持するよ
\\	きつい 厳しい(きびしい) つらい きつい日程 きつい仕事 きつい規則
\\	(俗) 
\\	団子っ鼻(だんごっぱな) 団子鼻(だんごばな)
\\	監督する(かんとく) 監督(する人) そのチームの監督は突然辞めた 生徒を監督する
\\	親せき(しんせき) 親類(しんるい) 母方(ははかた)の親類 父方(ちちかた)の親類
\\	ステッキ 祖父はステッキがないと歩ける
\\	一回(いっかい) 一度(いちど)
\\	~につき 一時間につき100円 この仕事は一時間につき六ドルもらえる
\\	酸素ボンベ   酸素ボンベなしで潜る
\\	と共に(とともに) 同時に(どうじに) 小林は学生を指導すると共に、研究にも力を入れている
\\	平らげる(たいらげます)
\\	食べきる 大きなステーキをまたたく間に平らげた
\\	ジェット機(き)
\\	岸(きし)  
\\	騎士(きし) 海辺(うみべ)
\\	海岸(かいがん)
\\	川の向こう岸に 壊れたボートが岸に打ち上げられた
\\	クレープ パンケーキ
\\	日にち(ひにち) 日付(ひづけ)
\\	年月日(ねんがっぴ)
\\	出発の日にちを決めよう
\\	にわか雨(にわかあめ) 学校に行く途中、私はにわか雨に会った
\\	支配する(しはい) 
\\	統治する(とうち)
\\	だれかの支配の下にある これらの地方は皆ローマ帝国の支配の下にあった
\\	見ぬが花 
\\	エチケット それはエチケットの問題だ
\\	祖先(そせん) 先祖(せんぞ) 子孫(しそん)
\\	~に 詳しい(くわしい) ~に 精通する(せいつう) ~を 通じる(つうじます) 彼は日本の歴史に詳しいです
\\	霊安室(れいあんしつ)
\\	間柄(あいだがら)
\\	関係(かんけい) 
\\	お二人の間柄は? いとこの間柄です
\\	水一杯(みずいっぱい)
\\	集める(あつめます) この本には多くの英語のことわざが集められています
\\	日の出(ひので) 明け方(あけがた)
\\	粘土(ねんど) 年度(ねんど)
\\	太陽の熱は粘土を固くする(かたく)
\\	祭る(まつります) 崇拝する(すうはい) 
\\	祖先(の霊(れい))を祭る
\\	診る(みます) 風邪が中々治らないので、医者に診てもらいました
\\	あることができない 
\\	ありえない それは、ありえない話だよ
\\	~したくて うずうずする ~したくて むずむずする 
\\	子供は外へ遊びに行きたくてうずうずしていた フランス語を習いたくてうずうずしている
\\	塗る(ぬります)
\\	トーストにジャムを塗ります
\\	切る(きります)
\\	蛍光灯(けいこうとう)
\\	二つで 二つで60円です
\\	大学を出る 卒業する(そつぎょう)
\\	名残(なごり) この町には戦前の名残は何もない
\\	別れ(の挨拶) (わかれ(のあいさつ) 別れが本当につらかった
\\	盛る(もります) 
\\	毒を盛る 皿に盛る
\\	男前(おとこまえ) ハンサムな男の人 彼はなかなかの男前だ
\\	替える(かえます) お金を替える 古い機械を新しいのに替える必要がある
\\	葉書(はがき)
\\	~に 由来する(ゆらい) この単語はラテン語に由来している
\\	バックル
\\	~しておきます 次の会議までに、何をしておいたらいいですか
\\	着陸する(ちゃくりく) 国際空港に着陸する 離陸する(りりく)
\\	つばめ つばめが二羽頭上に飛んでいます
\\	熱がある(ねつ)
\\	注ぐ(そそぎます) 火に油を注ぐだけだ
\\	普通列車(ふつうれっしゃ)
\\	~おく(おきます) 将来のために、今、どんなことをしておく?
\\	崖(がけ) 崖からバスは落ちていった
\\	軍用空港(ぐんようくうこう)
\\	未熟者(みじゅくもの) 彼らは私たちを未熟者として軽蔑する(けいべつ)
\\	すずめの涙(なみだ)
\\	蝶ネクタイ(ちょうねくたい)
\\	滅亡する(めつぼう) 没落する(ぼつらく)
\\	ローマ帝国の滅亡/没落  滅亡の道をたどる
\\	乳(ちち) 牛の乳を搾る(しぼります)
\\	拒否(きょひ) その計画に同意することを拒否した
\\	団体旅行 (だんたいりょこう)
\\	戦後(せんご) 戦後の日本憲法
\\	盲人(もうじん)
\\	のこぎり 彼が木をのこぎりでひいているのを見た のこぎりざめ
\\	決まり(きまり) じゃ、決まりだ。 ナイフとフォークの使い方には決まりがある
\\	刺(とげ) この指に刺が刺さって取れません
\\	ホーム 新宿に行く電車のホームはどちらですか
\\	粘る(ねばります) 粘々する(ねばねば) ガムが靴底に粘りついている
\\	教育がある(きょういく)
\\	植民地(しょくみんち) ケニアはかつてイギリスの植民地だった
\\	観覧車(かんらんしゃ) 観覧(かんらん) 観覧車に乗る
\\	メートル
\\	納屋(なや) 干草を納屋に置いた
\\	投票する(とうひょう) 投票者 投票箱(とうひょうばこ) 投票日(とうひょうび) 投票所(とうひょうしょ)
\\	愛の結晶(けっしょう) その子は彼らの愛の結晶だった
\\	腰(こし)
\\	愛人(あいじん)
\\	恋人(こいびと)
\\	彼はあなたの愛人ですか
\\	胡桃(くるみ) くるみを割る(わります)
\\	かご 買物かご 鳥かご(とりかご)
\\	し難い(しかたい) し難い(しにくい) 訳し難い(やくしかたい)
\\	略称(りゃくしょう) 略(りゃく) 省略(しょうりゃく) 
\\	はアメリカ合衆国(がっしゅうこく)の略称です
\\	触角(しょっかく) 触覚(しょっかく)
\\	雌鳥(めんどり)(鳥の雌) 雄鶏(おんどり)(鳥の雄) 鶏(にわとり)
\\	転ぶ(ころびます) 転ける(こけます)(関西弁) 倒れる(たおれる)
\\	展覧会(てんらんかい) ピカソの展覧会を見に今度の金曜儀に行こうと思っています
\\	ソーセージ
\\	雇う(やといます) さっそく彼を雇う あの会社に運転手として雇われている
\\	母音(ぼいん) フランス語に日本語よりずっと多くの母音がある
\\	航海する(こうかい) 
\\	後悔する(こうかい) 
\\	公開する(こうかい)
\\	巡航する(じゅんこう) 
\\	太平洋を航海する 去年彼は航海で三ヶ月を過ごした 太平洋を航海する
\\	組み立てる(くみたてる) 自動車を組み立てる 説明書のとおりに、組み立ててください
\\	宝(たから) 宝くじ(たからくじ)
\\	風(かぜ) 強風(きょうふう)
\\	(名詞) 
\\	(形容詞) 
\\	容疑者(ようぎしゃ) 怪しい(あやしい) 彼女はその事件の容疑者を逮捕(たいほ)した
\\	分離(ぶんり) マヨネーズの油は分離することがある 政治と教育の分離 政教分離(せいきょうぶんり)
\\	説明書(せつめいしょ) 説明書のとおりに、組み立ててください
\\	牧場(ぼくじょう)
\\	農場(のうじょう)
\\	孫息子(まごむすこ)
\\	~てから、 手を洗ってから、お菓子を食べましょう
\\	熟語(じゅくご) できるだけたくさん熟語を暗記しなさい 英語の熟語を覚える
\\	雛菊(ひなぎく)
\\	(俗) 
\\	すっぴん 素(す-
\\	ぴん(俗 
\\	すっぴんである
\\	修道院(しゅうどういん) 修道院の黒ミサは楽しい
\\	メモ ヤンさんがここへ来たら、このメモを渡してください
\\	国立(大学)(こくりつ)
\\	ヘルメット 兜(かぶと) 
\\	犯人(はんにん)
\\	栄える(さかえます) 15世紀のイタリアでは美術が栄えた
\\	和暦(われき)
\\	渋滞する(じゅうたい) 高速道路が渋滞をしている
\\	旧姓(きゅうせい) 急性(きゅうせい)
\\	メアリー・スミス,旧姓ベーカー
\\	魂(たましい)
\\	霊魂(れいこん) 私は霊魂の不滅(ふめつ)を信じる
\\	ロックンロール
\\	摘む(つまみます) 新茶を摘む時期になった
\\	追放(ついほう) 追い払う(おいはらいます) 学校から追放される
\\	旅立つ(たびだつ)/旅に立つ 出発する (しゅっぱつ) 彼はタイコクに旅立った
\\	過疎(かそ) 過疎の村 過密の(かみつ)
\\	協力(きょうりょく) 妻の協力がなければ成功しなかった 協力して
\\	楽(な)(らく)
\\	容易(な)(ようい) 簡単(な)(かんたん) 易しい(やさしい)
\\	日没(にちぼつ)
\\	死亡記事(しぼうきじ)
\\	約束(やくそく) 
\\	誓(ちかい) 
\\	誓いを守る[破る]
\\	つりばり
\\	一文無しです(いちもんなし) 今日は一文無しだよ
\\	平ら(な)(たいら) ぺちゃんこ 平らにする
\\	~を ~に 溶かす(とかします) 解かす(とかします)
\\	彼女は水に砂糖を溶かした
\\	蟷螂(かまきり)
\\	叩く(たたきます) 打つ(うちます) 犬を叩くなんて彼は虐待だ(ぎゃくたい) 尻(しり)を叩く
\\	幸運(こううん) 幸運を祈る(いのります) 幸運児(こううんじ)
\\	スポーツ どんなスポーツが好き?
\\	政治(せいじ) 最近の若い人は政治について話さない
\\	ヨット
\\	場合に(ばあい) 雨が降った場合にはピクニックに行かない 場合によっては、明日来られないかもしれない
\\	資料(しりょう) 新しい資料を送ってください
\\	引きつり(ひきつり)
\\	エメラルド 
\\	翠玉(すいぎょく)
\\	字幕(じまく) 日本の字幕
\\	そらで そらで言う
\\	最中(さいちゅう) 今が寒い最中です 暑い最中でも彼女は厚い靴下をはいている
\\	偉人(いじん) 英雄(えいゆう) 
\\	ヒーロー 偉人伝(いじんでん)
\\	尺度(しゃくど) 基準(きじゅん)
\\	財産は人の幸せの尺度にはならない 君とは人生を見る尺度が違う
\\	響く(ひびきます) 滝の音が響いている トンネルの中では声が響きます
\\	休火山(きゅうかざん)
\\	回る(まわります) 地球は太陽の周りを回っている
\\	提案(ていあん) 彼はその提案に賛成していた
\\	やる(やります) 花に水をやります
\\	航空機(こうくうき)
\\	飛行機(ひこうき) 
\\	招く(まねきます) 招待する(しょうたい) 福の神を家の中へ招く
\\	田植え(たうえ) 田植え歌(たうえうた) この地方では6月中頃(なかごろ)に田植えをする
\\	気管支炎(きかんしえん)
\\	友達を呼ぶ(よびます) 広いアパートに引越しして、友達が呼べるようになった
\\	人権(じんけん) 人権の問題
\\	風景(ふうけい) 景色(けしき)
\\	この地方は美しい風景で知られている 風景画(が)
\\	かしこまりました。
\\	刃(は) 刃の鋭いナイフを手に入れた
\\	右翼(うよく) 左翼(さよく) 右翼的 右翼政党(うよくせいとう)
\\	グレープフルーツ
\\	混合する(こんごう) 混ぜる(まぜます) これらの物質の混合は危険だ
\\	偽善者(ぎぜんしゃ) 偽善(ぎぜん) 偽善的(な)(ぎぜんてき) 偽善を行う
\\	ぐっと 彼はドアをぐっと引いた ぐっと飲む
\\	秘書(ひしょ)
\\	拳(こぶし) 拳骨(げんこつ) 拳骨を食らう(くらいます)
\\	加速(かそく) 計画の実行を加速する 列車は少しずつ加速していって時速 
\\	キロに達した
\\	賢い(かしこい) 賢明(けんめい)(な)
\\	イギリス 英国(えいこく) イングランド 
\\	ウェールズ 
\\	スコットランド 
\\	北アイルランド(きた) 
\\	仏(ほとけ)
\\	幻覚(げんかく) 厳格(げんかく)
\\	幻覚を起こす 幻覚が見えると言っている
\\	最寄(もより)(の) 最寄の地下鉄駅はどこ?
\\	可能動詞(かのうどうし)
\\	お誕生日おめでとうございます
\\	主(な)(おも) 主要(な)(しゅよう)
\\	主な島
\\	幸運児(こううんじ)
\\	でっち上げる 作り上げる(つくりあげます) 
\\	彼はこの話をでっちあげた
\\	協会(きょうかい) 教会(きょうかい)
\\	境界(きょうかい) 
\\	いかが
\\	我慢する(がまん) これ以上その騒音を我慢することが出来ない
\\	彫る(ほります) 掘る(ほります)
\\	彫刻する(ちょうこく) 木の幹に自分の頭文字を彫った 大理石で像を彫った
\\	こんな時間に
\\	ご馳走する(ごちそう) ご馳走様でした(ごちそうまでした) テーブルの上には、ご馳走がたくさん並んでいた 夕食をごちそうになりました
\\	"逸品(いっぴん) 
\\	逸品 
\\	(お)土産(みやげ)
\\	さび さびる(さびます) さびのついた そのナイフはずいぶんさびている
\\	あらゆる すべて あらゆる手段を尽くす
\\	ゴムの ゴムの手袋
\\	鑑賞する(かんしょう) 音楽鑑賞 趣味は音楽鑑賞です
\\	書斎(しょさい) その詩人は自分の書斎で自殺した
\\	ご飯を炊く(たきます) 炊飯器(すいはんき)
\\	~なくてもいいです レポートは出さなくてもいいです
\\	会社を休む(やすみます)
\\	両方(とも)(りょうほう) どちらも
\\	両方の手に花を持っていた 両方とも好きでない 山田さんは英語もフランス語も両方上手です
\\	最初(さいしょ) 最初の日
\\	網棚(あみだな)
\\	天才児(てんさいじ)
\\	塩素(えんそ)
\\	静けさ(しずけさ) 沈黙(ちんもく)
\\	救急車(きゅうきゅうしゃ) 救急車を呼んでください
\\	指切り(ゆびきり) 約束する(やくそく)
\\	私たちはそれを絶対に他言しないと指切りした
\\	旅をする(たび) 旅行をする(りょこう)
\\	盛ん(さかん) この国では、サッカーが盛んだ
\\	同い年(おないどし) 私と同い年の女性
\\	デマ 
\\	噂(うわさ)
\\	伝聞(でんぶん)
\\	悪質なデマ 根拠(こんきょ)のないデマ
\\	本は中にちょっと書いている
\\	表面(ひょうめん) 月の表面を歩いているところを想像できる?
\\	~と 化する(かする) ~に 変わる(かわります) その建物は廃墟と化した
\\	専門家(せんもんか)
\\	レッサーパンダ パンダ ジャイアントパンダ
\\	~を ありがとう 贈り物をありがとう
\\	統治者(とうちしゃ) 元首(げんしゅ)
\\	ミシンで縫う(ぬう)
\\	マイナス29点5
\\	ミニスカート
\\	罰金(ばっきん) つばを吐いた罰金は5ポンドです
\\	塩基(えんき) 延期(えんき)
\\	塩基的 塩基度(えんきど)
\\	需要(じゅよう) ビールの需要が多い[少ない] 需要を満たす(みたす)
\\	所(ところ) 場所(ばしょ)
\\	税金(ぜいきん) 税金をかける
\\	丸顔(まるがお) 丸顔の少女
\\	尋ねる(たずねます) 訪ねる(たずねます)
\\	交番で道を尋ねた
\\	首輪(くびわ) 犬はやすやすと首輪を外した(はずす)
\\	鶏肉(とりにく)
\\	ピスタチオ
\\	下品(げひん) 下品な言葉 下品な冗談
\\	ジャングル 密林(みつりん) 熱帯雨林(ねったいうりん) 
\\	笑顔(えがお)
\\	領収書(りょうしゅうしょ) 私たちはサインある領収書がいる
\\	病気(びょうき) 病(やまい)
\\	電話をください
\\	突っ走る(つっぱしります) ジープは砂漠を突っ走って行った
\\	言いにくい
\\	眠い(ねむい)
\\	人種差別(じんしゅさべつ) レイシズム 人種差別主義者 レイシスト
\\	勘定する(かんじょう) 感情(かんじょう)
\\	計算する(けいさん) 数える(かぞえます) 勘定が上手だ 勘定を払う
\\	(東京弁と関西弁)	ちょうあつい (東京弁) めっちゃあつい (関西弁)
\\	人造の(じんぞう) 腎臓(じんぞう)
\\	人工の(じんこう) これらの真珠は人造じゃなくて本物です
\\	~緒に(いっしょに) ~揃って(そろって)
\\	友達 (ともだち) 友人(ゆうじん) 仲良し(なかよし)
\\	親友(しんゆう)
\\	説得(せっとく) 説得上手 説得力 私はとうとう彼女を説得してキャンプに行った
\\	~を申し込む(もうしこむ)
\\	不満(ふまん) 満足(まんぞく) 何がそんなに不満ですか
\\	青葉(あおば) 青い草 青いリンゴ
\\	聴衆(ちょうしゅう) 音楽会の聴衆
\\	注目を引く(ちゅうもく) 丁寧なことが注目を引く
\\	開封する(かいふう) 封筒を開ける(ふうとう) 手紙を開封で出す
\\	葉っぱ(はっぱ) 葉(は) 歯(は)
\\	こっぴどく 誰かをこっぴどくしかられる
\\	出掛ける(でかけます) 出る(でます)
\\	島国(しまぐに) 島々(しまじま)
\\	答案(とうあん) 答案を返す 答案を提出せよ
\\	塔(とう) タワー 
\\	半分(はんぶん)
\\	餌(えさ) 
\\	餌(え/えさ) 鳩に餌をやる
\\	放出(ほうしゅつ) 植物は栄養を作ってるとき酸素を放出する
\\	大幅に(おおはば) 急に(きゅうに)
\\	月給の大幅引き上げ 料金の大幅な引き上げ
\\	ぱいおつ 
\\	おっぱい
\\	恋文(こいぶみ)
\\	ロブレター 彼女は彼の昔の恋文を大切していた
\\	胸骨(きょうこつ)
\\	生活(せいかつ) 命(いのち) 一生(いっしょう)
\\	罰点(ばってん) 間違いをばっ点で消した 線の上にばっ点を書いた
\\	じゅうたん カーペット
\\	海の上(うみのうえ) 空港は海の上にある
\\	スイセン スイセンはいつ芽が出ますか
\\	出世する(しゅっせ) 両親は息子の出世を見て喜んだ 彼は局長にまで出世した 出世頭(しゅっせがしら)
\\	スパイス 調味料(ちょうみりょう)
\\	精神医学(せいしんいがく) 心理学(しんりがく)
\\	かん木 (かんぼく) やぶ
\\	ブラジャー
\\	空想(くうそう) 空想をふける(ふけります) 私は空想にふけって多くの時間を無駄(むだ)に過ごす
\\	雑草(ざっそう) 雑草を取る 花の間に多くの雑草が生えていた
\\	対話(たいわ) 対話する
\\	四字熟語 
\\	中途半端(な) (ちゅうとはんぱ) 中途半端な仕事 中途半端な態度(たいど)をとる
\\	円滑(えんかつ) 円滑に運ぶ(はこびます) 円滑にする
\\	勇者(ゆうしゃ) 英雄(えいゆう) 真の勇者
\\	祭日(さいじつ) 国民の祝日(こくみんのしゅくじつ) 祝日(しゅくじつ)
\\	模型(もけい) 私は模型の飛行機を作った 模型飛行機
\\	髪が黒い人
\\	写生する(しゃせい) この富士山の写生はよくできている
\\	スカート ドレス
\\	基礎(きそ) この家の基礎がしっかりしている どんな勉強でも基礎が大切だ
\\	血脈(けつみゃく) 
\\	血管(けっかん)
\\	地位(ちい) 身分(みぶん)
\\	水力(すいりょく) 水力発電所(はつでんしょ) 水力で発電する
\\	眼科医(がんかい)
\\	二日酔いになる(ふつかよい)
\\	戦場(せんじょう)
\\	毛皮(けがわ)
\\	~が 欲しい(ほしい) ~したい (です)
\\	卵型の(たまごがた) 卵型の(たまごがたのかお) 面長の(おもなが)
\\	幼児語(ようじご)
\\	しびれる(しびれます) 寒さで手足がしびれた
\\	お祖母ちゃん(おばあちゃん) お祖母さん(おばあさん)
\\	権力(けんりょく) ヒトラー1933年に権力を取った
\\	人種(じんしゅ) 人種差別をする人
\\	涼しい(すずしい)
\\	頭蓋骨(ずがいこつ)
\\	油(あぶら)
\\	圧力(あつりょく) 橋が圧力で落ちた 世論の圧力で
\\	湿疹(しっしん) 湿疹ができる 肛門の周りに湿疹がある
\\	ずっと
\\	どっち 海と山と、どっちが好き?
\\	弱い(よわい)
\\	一週間に一回/一度
\\	診察(しんさつ) 診察時間 昨日キング先生の診察を受けました
\\	語尾 
\\	~な 食事はおいしくなかったな これでいいと思うな
\\	格好いい (かっこういい)
\\	不注意(ふちゅうい) 不注意で本にコーヒーをこぼしてしまった
\\	点在する(てんざい) 谷には農家が点在していた
\\	ハープ
\\	中古車(ちゅうこしゃ) この中古車は売り物です
\\	近眼的(な)(きんがんてき) それは近視眼的な政策だ
\\	上の兄   下の兄
\\	蘭国(らんこく) オーランド
\\	擬態語	ごろごろ
\\	歩道(ほどう) 私は凍った歩道で滑って転んだ (こおったほどうですべってころんだ)
\\	強い(つよい) 強烈(な)(きょうれつ)
\\	乳を絞る(ちちをしぼります) ああ、牛の乳はそんな風に絞るんですか
\\	光熱費(こうねつひ)
\\	台風(たいふう)
\\	~ばいいです このスイッチを押せばいいです
\\	現金(げんきん) 現金で払う
\\	(天然) 資源(しげん) 資源が乏しい(とぼしい) 資源が豊かである
\\	軒(けん) 2軒の家 彼は私の家から5軒目に住んでいる
\\	偶数(ぐうすう) 奇数(きすう) 
\\	5という数は奇数と言い、2,4,6は偶数と言います。
\\	機械(きかい) 工場にはたくさんの機械が必要だ。
\\	両性愛(りょうせいあい) 両性愛者
\\	以下(いか) ドイツの人口は米国の半分以下です
\\	膣(ちつ)
\\	外国製の(がいこくせい) 外国製の香水
\\	そのまま 本はそのままにして置きなさい!
\\	子孫(しそん) 子孫に伝える
\\	寝間着/寝巻き(ねまき) パジャマ
\\	話を聞いてる?
\\	迂回路(うかいろ) バイパス
\\	来い(こい) 来い来い!
\\	顕微鏡(けんびきょう) だれが顕微鏡を発明したかご存知ですか
\\	休校(きゅうこう) 急行(きゅうこう)
\\	休校になる 休校は大雪のためです
\\	朝ごはん(あさごはん) 朝食(ちょうしょく)
\\	登る(のぼります) 山を登る 登山する(とざん) 山登りをする
\\	信者(しんじゃ) キリスト教の信者だったことはない
\\	定住(ていじゅう) その部族の民はその川沿い(かわぞい)に定住した 定住者
\\	かつて
\\	通す(とおす) 続ける(つづける) 彼女は一生独身で通した 救急車を通す
\\	女性の胸の谷間 (たにま)
\\	成るべく~(なるべく) 成るべく速く来てください 成るべくゆっくり話してください
\\	ワールドカップ 
\\	杯(ダブルはい)
\\	体力(たいりょく)
\\	教材(きょうざい) 英字新聞を教材として使う
\\	出陣する(しゅつじん) いよいよ出陣だ
\\	ストロー ストローで飲む 
\\	藁(わら)
\\	本日(ほんじつ) 
\\	今日(きょう)
\\	本店(ほんてん) 支店(してん)
\\	芝刈り機(しばかりき) 芝を刈る
\\	一等(いっとう) コンテストで一等になった/を取った
\\	点字(てんじ) 展示(てんじ) 
\\	点字を読む 点字の新聞
\\	真菌症(しんきんしょう)
\\	染み(しみ) 染みを付ける
\\	~弱(じゃく) ~強(きょう) ~ちょっと 1メートル弱 
\\	000名弱の学生
\\	木綿(もめん) 
\\	コットン
\\	住宅(じゅうたく) 住居(じゅうきょ) 
\\	その建物は住宅には向かない(むかない)
\\	一年生(いちねんせい) 新入生(しんにゅうせい) 一年生の中ではだれが髪が一番長いですか
\\	~で 満足する(まんぞく) こんな安い給料では満足できない
\\	詰め合わせ(つめあわせ)
\\	混合(こんごう)
\\	クッキーの詰め合わせ
\\	忍耐する(にんたい) 持久(じきゅう)
\\	忍耐強い人だ 外国語の習得には忍耐が必要だ
\\	絞首台(こうしゅだい)
\\	ごぞんじですか。
\\	メンバー 会員(かいいん) 一員(いちいん)
\\	粉々(こなごな) 私のコップが粉々になっている
\\	大洋(たいよう) 太平洋は世界で一番大き大洋です
\\	地図帳(ちずちょう) アトラス
\\	反応(はんのう) 化学反応 ~に 反応する
\\	語尾(ごび)
\\	三日月(みかづき)
\\	死刑(しけい) 死刑に処する(しょします) 死刑に処せられる (しょせられます)
\\	子守(こもり) サリーは今夜スミスさんのところで子守をしている
\\	敵(てき)
\\	すこししか 日本語がすごししか話せません
\\	中肉中背の人  (ちゅうにくちゅうぜい) 中肉中背の紳士(しんし)
\\	座敷(ざしき) 和室(わしつ) 
\\	座敷 
\\	気力(きりょく) 年をとり、身体も不自由であったが、彼にはその仕事をする気力かあった
\\	赤らむ(あからみます) 赤面する(せきめん) 恥ずかしくて、彼女のほおが赤らんだ
\\	煙突(えんとつ) 暖炉(だんろ) この煙突はレンガで出来ている
\\	曲げる(まげます) 怪我(けが)をして、足を曲げることができなくなりました
\\	通路(つうろ) これが海への通路
\\	解剖(かいぼう) 私たちは内臓を調べるためにカエルを解剖した 解剖学(かいぼうがく) 解剖室(かいぼうしつ)
\\	~を とがめる(とがめます) いつも人をとがめてばかりいる
\\	犀(さい) ~才、~歳 
\\	訴訟(そしょう) 私はその医者を訴訟を相手取って訴訟を起こした 訴訟を取り下げた
\\	~を 完成する(かんせい) ~を 終える(おえます)
\\	煙突掃除夫 (えんとつそうじふ)
\\	白髪(しらが)
\\	西欧(せいおう) 西欧諸国(せいおうしょこく)
\\	熱烈に(ねつれつ) 熱心に(ねっしん)
\\	情熱的に(じょうねつてき) チームを熱烈に声援した
\\	陸地(りくち) 大陸(たいりく) 地球の表面の約三分の一が陸地です
\\	と共に(とともに) その発見は建設的であるとともに革命的だった
\\	何かを どこかに 持って行く
\\	~を 銃殺する(じゅうさつ)
\\	地下牢(ちかろう)
\\	過ごす(すごす) 休暇を過ごす
\\	野良犬(のらいぬ)
\\	外側(そとがわ) 外側に大きいポケットが付いていました
\\	決して
\\	ない (けっして) その問題の解決は決して簡単ではない
\\	バーベキューする
\\	中立(ちゅうりつ) 中立の 中立国(ちゅうりつこく) 中立を宣言する
\\	民族言語(みんぞくげんご)
\\	人力車(じんりきしゃ) 人力車を引く
\\	会議(かいぎ) 懐疑(かいぎ) 
\\	気持ちが悪い 気持ちが悪いので、家に帰ってもいいでしょうか
\\	機会(きかい) チャンス 私は映画スターになる機会を逃した(のがす)
\\	猿も木から落ちる 猿(さる)
\\	~を 望む(のぞみます) ~を 期待する(きたい)
\\	~を 希望する(きぼう)
\\	そうでないことを望む 幸福を望む 彼は水夫になることを望んだ(のぞんだ)
\\	操り人形 (あやつりにんぎょう) 人形 (にんぎょう)
\\	脂肪(しぼう)
\\	資質(ししつ)
\\	死亡(しぼう)
\\	脂肪の多い[少ない]肉
\\	~に 腹を立てる(はらをたてる) ~に 怒る(おこります)
\\	雪崩(なだれ) 大なだれで家々はつぶされた
\\	軟体動物 (なんたいどうぶつ)
\\	ブレスレット ブレスレットを着ける(つけます)
\\	世界一(せかいいち)
\\	観光ルート(かんこう) 観光ルートをたどる
\\	人気がある(にんき) 彼は生徒に人気がある
\\	速度(そくど) 突然彼は車の速度を上げた
\\	度々(たびたび) しばしば よく 頻繁に(ひんぱん) 彼は度々学校に遅刻する
\\	倹約(な)(けんやく)
\\	私が言うように書いてください
\\	排泄(はいせつ) 排泄物(はいせつぶつ) 排泄する
\\	~に 励む(はげみます) ~に 専念する(せんねん) 仕事に励む
\\	鏡(かがみ) メアリーは鏡で自分を見た
\\	曲がる(まがります) 右へ曲がってください
\\	天使(てんし) アンゼル
\\	ぽ	半濁点(はんだくてん)
\\	試着する(しちゃく) 着てみる 上衣(うわぎ)を試着する
\\	如雨露(じょ
\\	うろ) 語源:(ポルトガル語) 
\\	う-
\\	る-
\\	五段動詞(ごだんどうし) 一段動詞(いちだんどうし)
\\	箸置き(はしおき)
\\	目薬(めぐすり) 目薬を差す(さす)
\\	困難(な)(こんなん) 難しい(むずかしい) ~が財政(ざいせい)困難です
\\	引き上げ(ひきあげ) 月給の引き上げ 税金の引き上げ
\\	ビザ ビザを取る
\\	お見合い結婚(おみあいけっこん)
\\	職人(しょくにん)
\\	~を 身につける (みにつけます)
\\	ハンセン病(びょう)
\\	脳(のう)
\\	薄暗い(うすぐらい) 薄暗い部屋
\\	晴れる(はれます) 晴れている
\\	熱傷(ねっしょう)
\\	火傷(やけど)
\\	失策する(しっさく) 間違い 誤り(あやまり) 錯誤(さくご)(書) つまらぬ失策で絶好のチャンスを逃した ちょっとした失策
\\	うふ(ふ)と笑い くすくす笑い 
\\	うふっと笑ってごまかす
\\	入れ替わり(いれかわり) 佐藤選手が池田選手と入れ替わりになった
\\	3時間しか
\\	お玉杓子(おたまじゃくし)
\\	罪(つみ) 罪人(つみびと)
\\	硫黄(いおう) 硫黄はマッチを作るのに使われる
\\	着メロ(ちゃく)
\\	宣言する(せんげん) 神父は彼らを夫婦であると宣言した
\\	~を 終わりにする(おわり) 終わり(おわり)
\\	備える(そなえます) この事務所には机といすが備えてある 地震に備える
\\	こんばんは。
\\	~を 救う(すくいます)
\\	~を 助ける(たすける) 彼は私を危険から救った
\\	時差(じさ) ニューヨークとサンフランシスコでは4時間の時差がある
\\	近く(ちかく) 辺(へん) 近所(きんじょ) 近くに
\\	爆発(ばくはつ)
\\	破裂(はれつ)
\\	工場で爆発がおこった
\\	天の助け (てんのたすけ)
\\	小説家の一生 一生
\\	かなり このパイはかなりおいしい かなりの収入
\\	ドナウ川(がわ) ライン川(がわ)
\\	ワッフル
\\	(諺)
\\	雨が降って地固まる (あめがふってじかたまる)
\\	深夜(しんや) 夜遅く(よるおそく) 深夜まで勉強する このバーは深夜営業(えいぎょう)だ
\\	~と言えば 京都へ行く。京都と言えば、私は京都先週行った
\\	酢キャベツ(すきゃべつ) 酢(す)
\\	塀(へい) 塀に登るな!
\\	肺(はい) 灰(はい)
\\	酸素は私達が呼吸すると肺の中に取り込まれる
\\	競争する(きょうそう) 競技(きょうぎ)
\\	はヨーロッパと競争する
\\	直角(ちょっかく) 直角に この二つの直線は直角に交わっている
\\	フィリピン
\\	寝袋(ねぶくろ)
\\	話し手(はなして)
\\	検閲官(けんえつかん) 検閲(けんえつ) 煙突(えんとつ)
\\	利用者(りようしゃ)
\\	ユーザー 
\\	使用者(しようしゃ)
\\	小説(エスエフしょうせつ)
\\	株式会社(かぶしきがいしゃ) 株式 (かぶしき)
\\	泣く(なきます)
\\	ごまかす 騙す(だます) 彼は年齢をごまかした 水を飲んで空腹をごまかす
\\	残業(ざんぎょう) 残業をする
\\	しまった!
\\	旗(はた) 国旗(こっき)
\\	風の神様
\\	クローバー 四つ葉のクローバー (よつば)
\\	人魚(にんぎょ) 人形(にんぎょう)
\\	(お願いが) 叶う(かないます) もし病気の人が千羽の鶴を折ると、その人のお願いが叶うの。
\\	鳴く(なく) 鳥が鳴きます
\\	子午線(しごせん) 緯線(いせん) 日本の子午線は東経(とうけい)135度です
\\	ファミコン ファミコンで遊ぶ
\\	もしか(したら/すると/して) 多分(たぶん)
\\	恐らく(おそらく)
\\	かもしれない ~でしょう もしかすると彼が正しいのかもしれない もしかして彼に会えたら,私が会いたがっていると伝えてくれ
\\	頭脳労働(ずのうろうどう) 肉体労働
\\	仕事場(しごとば) 職場(しょくば)
\\	感電死する(かんでんし)
\\	アブラムシ
\\	怠惰(たいだ)(書) 怠け(なまけ) 怠惰な人 怠け者(なまけもの) 怠惰な生活を送る
\\	紅茶(こうちゃ) 緑茶(りょくちゃ)
\\	~が 残る(のこります) ~を 残す(のこします)
\\	枕(まくら)
\\	管(くだ/かん) 水道管 管から水が吹き出した
\\	仏像(ぶつぞう) 仏(ほとけ)
\\	注射する(ちゅうしゃ) 駐車する(ちゅうしゃ)
\\	予防注射(よぼうちゅうしゃ) ワクチン 
\\	ハエ
\\	湯たんぽ (ゆたんぽ) 湯たんぽを入れて寝ます
\\	寂しい(さみしい) 君がいなくて寂しいよ
\\	華道(かどう) 生け花(いけばな)
\\	ーば  -ほど 大学はよければよいほど入るのが難しい
\\	特技(とくぎ) 才能(さいのう) 
\\	人の名前を覚えることが彼女の特技だ
\\	~でも ~たって そんなことは子供でも、分かります 先生でも違う 私は難しい仕事でもする
\\	酸素(さんそ) 酸素と水素から水ができる
\\	好奇心(こうきしん) 好奇心に満ちている
\\	駐車場(ちゅうしゃじょう)
\\	その代わり(かわり)
\\	経つ(たつ) 三ヶ月が経ちました
\\	武力(ぶりょく) 武力で
\\	誰でも(だれでも)
\\	国旗(こっき)
\\	中世(ちゅうせい) 暗黒時代(あんこくじだい)
\\	疑う(うたがいます)
\\	危ぶむ(あやぶみます)
\\	疑問する(ぎもん)
\\	彼の成功を危ぶむ
\\	正午(しょうご)
\\	春(はる)
\\	夜食(やしょく) 夜食にインスタントラーメンを食べた
\\	妊娠する(にんしん)
\\	哲学(てつがく) 哲学的 哲学者
\\	一面(いちめん) 
\\	その事件にはこっけいな一面もあった
\\	太鼓(たいこ)
\\	貴重品(きちょうひん) 貴重品を金庫にしまっておく
\\	代表する(だいひょう) ジョンは水泳の試合でクラスを代表した
\\	~を 驚かす(おどろかします) 友達を驚かしました
\\	赤血球(せっけっきゅう) 白血球(はっけっきゅう)
\\	~を 憎む(にくみます) ~を 憎悪する(ぞうお) ~に 憎しみを抱く(いだく) ~が 嫌いです 彼は父親に憎しみを抱いている
\\	ハンドル
\\	どちらも/ 同じぐらい 海と山どちらも好きです
\\	雰囲気(ふんいき) あなたの家はとても居心地のよい雰囲気ですね
\\	南(みなみ)
\\	ギプス 折れた足をギプスで固定する
\\	~風(ふう) 田舎風(いなかふう) ポーランド風
\\	安物(やすもの) 安物の酒
\\	ロッカー
\\	法律学(ほうりつがく)
\\	着物(きもの)
\\	民法(みんぽう)
\\	焚く(たきます) ストーブを焚いて、部屋を暖めた
\\	バール 
\\	調味料(ちょうみりょう) スパイス
\\	~振り(ぶり) 30振り 10年振りの最悪の嵐だ
\\	特筆すべき(とくひつ) 特筆する これは特筆すべき事件だ 新聞はその犯罪を特筆した
\\	~てごらんなさい 
\\	~てみなさい ちょっと食べてごらんなさい 来てごらんなさい
\\	難点(なんてん)
\\	短所(たんしょ)
\\	欠点(けってん)
\\	弱点(じゃくてん)
\\	この家の難点は台所が狭いことだ
\\	警報する(けいほう)
\\	刑法(けいほう)
\\	警告(けいこく)
\\	注意(ちゅうい)
\\	警報を出す 警報機(けいほうき)
\\	取っ手(とって) 取っ手をつかんで持つ
\\	不眠(ふみん) 不眠症(ふみんしょう)
\\	土竜(もぐら)
\\	最悪(さいあく) これは今までに見た中で最悪の映画です
\\	狂気(きょうき) 天才と狂気は紙一重(かみひとえ)
\\	ダウンロードする
\\	チップ ウエーターにチップをやる
\\	厚生省(こうせいしょう) 厚生大臣(こうせいだいじん)
\\	人生観(じんせいかん)
\\	あらら
\\	争いの種 (あらそいのたね)
\\	植木鉢(うえきばち) 彼女は植木鉢に新しい土を入れた
\\	~が 豊富です(ほうふ) ~に 富む(とみます) ~が 豊か(ゆたか) この国は石油に豊富です
\\	悲劇(ひげき) 
\\	人生の悲劇 悲劇的(ひげきてき)
\\	魔女(まじょ)  魔法使い(まほうつかい)
\\	パスポート 旅券(りょけん)
\\	髪を解かす(かみをとかす)
\\	ガーガー言う がーがー鳴く(なきます)
\\	競売(きょうばい) 競売人(きょうばいにん) 競売所(きょうばいしょ) 競売でその絵画を手に入れた
\\	トリュフ
\\	ほうれん草(そう)
\\	ミネラルウオーター
\\	精液(せいえき) 精液銀行
\\	知能指数(ちのうしすう) アイキュー 指数(しすう) 
\\	挟む(はさみます) パンにハムを挟む 鉛筆を耳に挟んだ 小指をドアに挟んだ
\\	デモ隊(でもたい) デモ
\\	削除(さくじょ) 消す(けします)
\\	このファイルは削除してください
\\	復習しておく (ふくしゅう) 復讐(ふくしゅう) 
\\	今日テストがありましたが、全然復習せずにテストを受けました
\\	海月(くらげ)
\\	一円玉(いちえんだま)
\\	一円硬貨(いちえんこうか) 硬貨(こうか)
\\	略語(りゃくご) 略称(りょくしょう) 
\\	は四月の略語です
\\	直感(ちょっかん) 彼は直感で彼女の嘘をついていると分かった
\\	運転手(うんてんしゅ) 
\\	運転者(うんてんしゃ) 
\\	ドライバー
\\	隅(すみ) 墨(すみ)
\\	炭(すみ)
\\	印象(いんしょう) この本の印象をどうぞ 誤った印象 この著者は, 日本についての誤った印象を与える 鮮やかな印象
\\	記憶(きおく) その問題では記憶が全然ない
\\	情報(じょうほう) 最新の情報
\\	重婚(じゅうこん) 重婚者(じゅうこんしゃ)
\\	スコール
\\	免疫(めんえき) 予防注射(よぼう)をしたのでインフルエンザには免疫になってる
\\	待つ(まちます) 最初の星を待ちます。
\\	顔色(かおいろ) 顔色が悪いですよ。どうしたんですか
\\	台所(だいどころ) 台所の仕事をします
\\	サイズ 大きさ(おおきさ)
\\	購買(こうばい) 販売(はんばい) 購買者 販売(はんばい)
\\	賠償(ばいしょう) そのけがに対しいくら賠償を要求しているのか 賠償金(ばいしょうきん)
\\	外す(はずします) 外れる(はずれます) ボタンを外す 箱のふたを外した
\\	就活(しゅうかつ) 就職活動(しゅうしょくかつどう)
\\	板チョコ(一枚)(いた) 板チョコをかじる
\\	副作用(ふくさよう) 副作用が強い薬
\\	30歳過ぎです
\\	洒落(しゃれ)
\\	冗談(じょうだん) ジョーク おしゃれ(な) 
\\	苦境(くきょう) 苦境にある[立つ]  苦境を乗り越える
\\	卒業論文 (そつぎょうろんぶん) 修士論文 (しゅうしろんぶん) 卒業論文のテーマは「日本の寺」にした
\\	気候(きこう) この国に、気候が穏やかです
\\	誤解(ごかい) 彼らは私の丁寧さと友情を誤解した
\\	録音する(ろくおん) 音楽をテープに録音した
\\	望ましい(のぞましい) だれも怪我をしないことが望ましい
\\	秘書室(ひしょしつ) 事務室(じむしつ)
\\	ドキュメンタリー 記録映画(きろくえいが)
\\	出来るだけ
\\	前代未聞(ぜんだいみもん) そりゃ前代未聞のスキャンダルだ
\\	夏(なつ) 夏季(かき)
\\	右上(みぎうえ) 左上(ひだりうえ) 右下(みぎした) 左下(ひだりした)
\\	注ぎ口(そそぎぐち)
\\	オリーブ
\\	トランプをする
\\	お茶でもいかがですか
\\	パチパチ
\\	与える(あたえます) 加える(くわえます)
\\	悪い病気 重病です(じゅうびょう) 悪い病気じゃありません
\\	述べる(のべます) 伝える(つたえます)
\\	英語で自分の考えを述べることは難しいことだと思います
\\	歯茎(はぐき)
\\	歯肉(しにく) 死肉(しにく)
\\	(人工)甘味料(かんみりょう) 
\\	わいろ 
\\	誇り(ほこり) あなたのお父さんを誇りに思いますか
\\	継母(ままはは)
\\	つなぐ 結ぶ(むすびます)
\\	犬をつないでおいてください この電話を社長室につないでください
\\	空(そら)
\\	天(てん)
\\	空中(くうちゅう)
\\	部屋を綺麗にする 綺麗(な)(きれい)
\\	真新しい(まあたらしい)
\\	騒音(そうおん) 騒ぐ(さわぎます)
\\	心理学(しんりがく) 心理学者 精神医学(せいしんいがく)
\\	押入れ(おしいれ)
\\	よろよろ歩く 酔っぱらいはよろよろと彼のところへやってきた
\\	行為(こい)
\\	行動(こうどう)
\\	あなたはその馬鹿な行為をやめらなければならない
\\	魔女(まじょ)
\\	後輩(こうはい) 彼は私より3年後輩です 先輩(せんぱい)
\\	本文(ほんぶん) 内容(ないよう)
\\	書類の本文
\\	サボテン
\\	歯肉(しにく) 歯茎(はぐき)
\\	死肉(しにく)
\\	募集する(ぼしゅう) クラブの新会員を募集する
\\	補習(ほしゅう) 英語の補習授業を行う/受ける
\\	ボーイフレンド 彼(かれ)
\\	~に/と比べて(くらべて)
\\	凍る(こおります) 凍った(こおった)
\\	名付け親(なづけおや)
\\	国民総生産 (こくみんそうせいさん) 
\\	人当たり国民総生産
\\	流れる(ながれます) 涙がほおを流れた
\\	田舎(いなか)
\\	村(むら) 
\\	夏の間私達は田舎で生活する
\\	今夜(こんや)
\\	アドバイスする 助言する(じょげん) 彼女がいいアドバイスをしてくれました 私のアドバイスを聞き入れなさい(ききいれなさい)
\\	落雷(らくらい)
\\	雷(かみなり)
\\	野原の大木に落雷があった 落雷する 木に落雷した
\\	新鮮(な)(しんせん) 新鮮な空気
\\	パンチ 
\\	量(りょう) 砂糖の量はこのぐらいでいいですか
\\	境内(けいだい) 境内に/で 神社の境内に/で
\\	扇(おうぎ) 扇子(せんす) 扇の二本 扇子をたたむ 扇子を開く
\\	人名用漢字 (じんめいようかんじ)
\\	うんち
\\	うんこ
\\	糞(くそ)
\\	種(たね) 種を蒔く(まきます)
\\	汚染(おせん) 環境汚染(かんきょうおせん)
\\	食塩(しょくえん) 
\\	塩(しお)
\\	炭鉱所(たんこうじょ)
\\	意外(な)(いがい) 案外(あんがい) 
\\	意外なこと 実に意外なことには, 
\\	バルコニー
\\	走る(はしります)
\\	光陰矢のごとし (こういんやのごとし)
\\	耕す(たがやします) 耕せる土地 土地を耕す
\\	天文学(てんもんがく) 
\\	天文学は宇宙の科学です 占星術(せんせいじゅつ)
\\	ユースホステル
\\	だに だにに刺される
\\	法案(ほうあん) 法案を提出する(ていしゅつ)
\\	統制する(とうせい) 規制する(きせい) 
\\	米価は政府によって統制されている 思想(しそう)の統制
\\	~のに 毎日運動したのに、全然痩せなかった
\\	汚水(おすい)
\\	目方(めかた) 体重(たいじゅう) 
\\	重さ(おもさ) 目方が増える 目方で売る
\\	あまり好きじゃない
\\	お待たせしました 
\\	お待たせ(おまたせ) 
\\	正直(な)(しょうじき) 彼は質問に正直に答えた
\\	先日(せんじつ)
\\	(お)もち
\\	お腹(おなか) 腹(はら)
\\	腹部(ふくぶ) 
\\	発行する(はっこう) 出版する(しゅっぱん) 発行者(はっこうしゃ) 今日発行の朝日新聞 この週刊は毎週一回発行される
\\	暴力団(ほうりょくだん) ギャング  暴力団員 ギャング 
\\	玩具(がんぐ) 
\\	冊(さつ)
\\	岬(みさき) ロカ岬(ろかみさき)
\\	仕事中毒(しごとちゅうどく) ワーカホリック
\\	飲むヨーグルト
\\	自爆(じばく) 自爆テロ(じばく)
\\	日曜ドライバー
\\	各国(かっこく) 各国にはそれぞれの習慣がある
\\	窒素(ちっそ) 私たちは吸っている空気は酸素と窒素からできる
\\	音量(おんりょう) 音量を下げてください
\\	黒い(くろい) 真っ黒(な)(まっくろ)
\\	低温(ていおん) 低温で
\\	高気圧(こうきあつ) 低気圧(ていきあつ)
\\	入学金(にゅうがくきん) 授業料(じゅぎょうりょう)
\\	メロン 瓜(うり)
\\	海水(かいすい) 海水から塩を取りなさい 淡水(たんすい)
\\	流星(りゅうせい)
\\	どうして なぜ
\\	~が 済む(すみます)
\\	~が 止む(やみます)
\\	~が 終わる(おわります) 試験が済んだ
\\	団結する(だんけつ) 2つのグループが団結して1つの政党になった 彼らは共同の敵に対して団結した
\\	北朝鮮(きたちょうせん)   朝鮮(ちょうせん)
\\	(雨)どい (あま)どい
\\	レモン レモンは酸っぱい味がする
\\	自動車”? 
\\	自動車” 
\\	「自動車」というのは何のことですか 「自動車」とういうのは車のことです
\\	燃え上がる(もえあがります) 遠くに火が燃え上がるのが見えた
\\	前立腺(ぜんりつせん) 前立腺は肥大(ひだい)になった 前立腺癌(ぜんりつせんがん)
\\	キャバレー
\\	半島(はんとう)
\\	気に入ってもらえばうれしい
\\	へその緒(お)
\\	固まり/塊(かたまり) 肉の大きな塊 土の塊 雲は蒸気のかたまりです
\\	旋回する(せんかい) 先回(せんかい) 
\\	コンドルは我々の頭上を旋回した
\\	手首(てくび)
\\	血液型(けつえきがた) あなとの血液型は何型ですか。
\\	口頭の(こうとう) 口頭のコミュニケーション 口頭で, または文書で 口頭で伝える 口頭試験
\\	五線譜(ごせんふ)
\\	遠い(とおい)
\\	たか たかは空を旋回した(せんかい)
\\	見開く(みひらきます) 目を見開く 彼の目は恐怖に見開かれていた
\\	~過ぎる(すぎます) 食べ過ぎた
\\	タンデム車 2人乗り自転車
\\	車椅子(くるまいす) 電動(でんどう)の車椅子を用いる(もちいます)
\\	年を取る(としをとる) 老ける(ふけます)
\\	神経系(しんけいけい)
\\	どっと 突然(とつぜん) その話を聞いた時、彼らはどっと笑い出した
\\	海軍(かいぐん) 彼は海軍にいる
\\	葉巻(はまき) 葉巻はいかが
\\	~に 出席(しゅっせき) ~に 出る(でます) 君自身が会議に出席すべきです
\\	弱点(じゃくてん) 難点(なんてん)
\\	短所(たんしょ)
\\	欠点(けってん)
\\	数学は彼女の弱点です
\\	竜/龍(りゅう) 竜/龍(たつ) ドラゴン 龍座(たつざ)
\\	部分(ぶぶん) 部品(ぶひん) 
\\	体の部分
\\	消化管(しょうかかん) 消化管の最前端(さいぜんたん)は口です
\\	甥(おい)
\\	甥御さん(おいごさん)
\\	番組(ばんぐみ) その番組は面白そうだった
\\	飛行機に乗る
\\	本島(ほんとう) 本当(ほんとう) 
\\	サーベル
\\	燃え殻(もえがら)
\\	さっき 先程(さきほど) ~したところ/ばかりです そのことは先程聞きました 先程のニュースによれば
\\	医院(いいん) 委員(いいん)
\\	カニバリズム 人食い(ひとくい)
\\	トースター トースト
\\	第一人称 (だいいちにんしょう) 第二人称 第三人称
\\	~に力を入れる (ちからをいれる) 小林は学生を指導すると共に、研究にも力を入れている
\\	伸ばす(のばす) 手足をいっぱいに伸ばす
\\	霰(あられ) 雹(ひょう) あられ/ひょうが降っている
\\	分類する(ぶんるい) これらの本を主題別に分類してください
\\	すき焼き(すきやき) 焼く(やきます)
\\	禁固(きんこ) 金庫(きんこ)
\\	彼は禁錮5年の判決を受けた
\\	タブロイド 大衆紙(たいしゅうし)
\\	加入する(かにゅう) クラブに加入する クラブに入る(はいります)
\\	海面(かいめん) 海綿(かいめん) 
\\	海面上[下]に
\\	野心(やしん) 野心的 金持ちになりたいという野心などありません 長年の野心が実現された
\\	(俗) 
\\	ちび 「ちび」というあだ名の少年  背が低いからって、チビって呼ぶなら、お前とは絶好するぜ
\\	椿(つばき) 椿油(つばきあぶら)
\\	催眠(さいみん) 誰かを催眠にかける
\\	規制(きせい) 規則(きそく)
\\	ルール 
\\	交通規制(こうつうきせい) 交通規制している
\\	バリケード
\\	近視(きんし) 禁止(きんし)
\\	近眼(きんがん) 彼は近眼だ
\\	苦い(にがい) 苦いコーヒー
\\	輪廻(りんね)
\\	ずぶ濡れ(の)(ずぶぬれ) にわか雨にあってずぶぬれになった 君のコートはずぶぬれだ ずぶぬれの服を早く脱ぎなさい
\\	嫁ぐ(とつぎます)
\\	結婚する(けっこん) 彼女は田中家に嫁いだ 彼女の嫁ぐ日が近づいた
\\	自殺する(じさつ)
\\	味方うする(みかた) 見方(みかた) 
\\	ケンさんは父母がけんかする時いつも母の味方をしている
\\	愛国的(な)(あいこくてき)
\\	~てばかりいる 彼は泣いてばかりいる
\\	(お)詫び(わび) 謝り(あやまり) わびを言う わびを入れる
\\	敷居(しきい) 敷居をまたぐ 二度とうちの敷居をまたぐな
\\	弾く(ひく) ピアノを弾く
\\	私物(しぶつ) 持ち物(もちもの) この辞書は彼の私物だ
\\	指す(さします) 刺す(さします)
\\	時計の針は6時を指していました 人差し指
\\	示す(しめします)
\\	見せる(みせます)
\\	りっぱな統率力を示した 身分証明書を示した
\\	追い出す(おいだす) 豆で鬼を追い出した
\\	牛肉(ぎゅうにく)
\\	一般人(いっぱんじん) 一般人はそれは間違ってと感じている
\\	並ぶ(ならびます) 新しい切手を買う人が並んでいた 並べる(ならべます)
\\	取り除く(とりのぞきます) 靴から泥を取り除いてください
\\	自伝(じでん) 伝記(でんき) 自伝小説(じでんしょうせつ)
\\	工学(こうがく) 化学工学
\\	~と 同じくらい この車はそれと同じくらい高い
\\	ずつ 私は漢字を毎日五つずつ覚える ゴルフが少しずつ上手になってきた 私は子供たちに本を二冊ずつやった
\\	腐る(くさります) 腐敗する(ふはい) 塩は魚が腐るのを防ぐ
\\	ごっこ 忍者ごっこをしよう カウボーイごっこをしよう
\\	布(ぬの) 生地(きじ)
\\	布切れ(ぬのぎれ) 布地(ぬのじ) 布製(ぬのせい)
\\	採用する(さいよう) 新教授法を採用する その説は広く採用された
\\	真実(しんじつ) 事実(じじつ)
\\	彼が言ったことは真実ではなかった 真実の 真実の愛
\\	勝負(しょうぶ) 水泳で兄といい勝負だ 意外に簡単に勝負がついた
\\	合わせる(あわせます) 
\\	目覚まし時計を朝6時に合わせる
\\	合わせる(あわせます) 
\\	手を合わせる 声を合わせる 心を合わせる
\\	もったいない 一度にたくさん買わないほうがいいですよ。もし腐ったら、もったいないでしょう まあもったいない この靴はまだはける. もったいなくて捨てられない
\\	不倫(ふりん) 浮気(うわき)
\\	隣の人が若い女の人と不倫しているらしいよ
\\	推量する(すいりょう) あなたは推量が当たった 推量の助動詞
\\	助動詞(じょどうし)
\\	撮影(さつえい) 撮影する 写真を撮る 撮影者(さつえいしゃ) 写真家(しゃしんか)
\\	貸間(かしま) 下宿(げしゅく)
\\	この家に貸間はありますか
\\	陰性(な)(いんせい) 検査の結果は陰性だった 娠反応テストの結果は陰性であった
\\	体験(たいけん) 経験(けいけん) 私が体験した恐怖は言葉では表現できません
\\	絵画(かいが) 競売でその絵画を手に入れた
\\	カウボーイ カウボーイごっこをしよう
\\	自画像(じがぞう) 画像(がぞう) 自画像を描く時、誰でも座ってじっと自分を見つめます
\\	お~ する 私は先生のスーツケースをお持ちした そのことは私がお話しましょう
\\	組(くみ) 彼は新しい手袋を1組買った
\\	うちに 間(あいだ) 静かなうちに 休みうちに 走っているうちにお腹が痛くなった
\\	(文)
\\	~より(ほかに)仕方がない バスで行くより(ほかに)仕方がない これより仕方がない
\\	~んじゃないでしょうか もう少し考えたほうがいいんじゃないでしょうか
\\	として 田中さんはセールスマンとして採用された このはやは物置として使っている
\\	~としては ~にしては ジョンソンは日本語の一年生としては日本語が上手だ
\\	~にしては ~としては これは本にしては軽すぎるね
\\	~と 類似する(るいじ) この車はそれと形が類似している
\\	当たる(あたります) 外れる(はずれる) 今日の天気予報は当たりました ピストルの弾丸がちょうど真ん中に当たった
\\	外れる(はずれる) 当たる(あたります) 今日の天気予報は外れた
\\	(文) 
\\	~はず です 大野先生はサンドラを知っているははずだ そのアパートはきれいなはずだ マーサもきますか。はい、そうはずだ
\\	かい 日本語は面白いかい 上野さんはアメリカへ行くかい
\\	~ことになる ことにする 日本では車は道の左側を走ることになっている
\\	じっと見つめる(みつめます) じろじろ見る 君はじっと何を見るんだい?
\\	献血(けんけつ) 献血者(けんけつしゃ)
\\	鼓膜(こまく) 科目(かもく) 
\\	彼は鼓膜が破れた(やぶれた)
\\	鐘撞堂(かねつきどう)
\\	瘡蓋(かさぶた)
\\	こめかみ
\\	写真の現像(げんぞう)
\\	拘束衣(こうそくい) 拘束服(こうそくふく) 拘束(こうそく)
\\	拘束する(こうそく) 
\\	高速(こうそく) 
\\	現体制では言論の自由は拘束されている
\\	間接(的) (かんせつてき) 
\\	関節(かんせつ)
\\	(解剖学)) 間接喫煙 直接(ちょくせつ)
\\	間接喫煙 (かんせつきつえん)
\\	区別する(くべつ) あいつは善悪(ぜんあく)の区別出来ない 私にはトムとトムの弟の区別ができない
\\	近世(きんせい) 近代(きんだい) 最近(さいきん) この頃(このごろ) この間(このあいだ)
\\	組む(くみます) 勝也がそのトーナメントを組むでしょう アダムとパウルは組んでテニスをした
\\	振り向く(ふりむきます) 振り返る(ふりかえります) 彼は僕の方を振り向いた 後ろを振り返る
\\	直近(ちょっきん) 直ぐ側(すぐそば) 貯金(ちょきん)
\\	都道府県 (とどうふけん) 都道府県名 (とどうふけんめい) 
\\	1/都(と) 
\\	1/道(どう) 
\\	2/府(ふ) 
\\	43/県(けん)
\\	以降(いこう) ~後で 10時以降に朝御飯は出されないだろう
\\	時間割り(じかんわり)
\\	割(かつ) 国土の8割近くが山だ
\\	二つの部屋とダイニングキッチンがあるうちのこと	
\\	(にディーケー)
\\	再入国許可 (さいにゅうこくきょか)
\\	私のうちへ遊びに来てください
\\	ねぎ こねぎ
\\	こねぎ
\\	(俗)	きもい
\\	声に出す
\\	意識化の(いしきか)
\\	皮肉(ひにく) 彼の皮肉はあなたに向けてではない
\\	お節介(おせっかい) 余計なお節介だ (余計(よけい) 
\\	屏風(びょうぶ)
\\	日中韓 (にっちゅうかん)
\\	「どうも」 ~ようだ どうも山田さんは今日休みのようです
\\	居心地がいい(いごこち) この喫茶店は居心地がよい
\\	くまんばち 成虫(せいちゅう),
\\	誠忠(せいちゅう) 
\\	ビーツ 赤蕪(あかかぶ)
\\	優勝(ゆうしょう) 
\\	勝利(しょうり)
\\	勝つ(かつ) 誰がそのトーナメントで優勝すると思いますか
\\	~を(見)捨てる 私を捨ててもっと若い子の方へ行ってしまった
\\	蚊帳(かや)
\\	膝頭(ひざがしら)
\\	ポストイット 
\\	付箋紙(ふせんし) 付箋(ふせん) 
\\	何で気が変わったの?
\\	磁場(じば) だから地球の周りには磁場がある
\\	深刻(な)(しんこく) あまり深刻に考えるな、ゲームなんだから
\\	超過する(ちょうか) 超える(こえます) その車は制限速度(せいげんそくど)を超過している
\\	制限速度 (せいげんそくど) 制限(せいげん)
\\	磁石(じしゃく) もちろん磁石は周りに磁場をもっている
\\	急増する(きゅうぞう) 急に増える(ふえます) 旅行会社の利益が急増した
\\	余計(よけい) 余計な努力 余計なことはするな!
\\	看病(かんびょう) メアリーと妹は交代で(こうたい)母の看病した
\\	パウダーシュガー
\\	趣味(しゅみ), 
\\	悪趣味(あくしゅみ) あなたは服の趣味がいい
\\	カヤック
\\	マシンガン 彼女はマシンガンのようにはなす
\\	マシンガンのように話す 早口する(はやくち)
\\	クーデター 
\\	そのクーデターの間接的な情報しかない
\\	日差し(ひざし) 強い日差しで彼の背中はひどく焼けた
\\	早退する(そうたい) 今日学校を早退してもよろしいでしょうか
\\	鍵穴(かぎあな) 鍵穴から中をのぞく
\\	全力(全力)(を尽くす) 私はその仕事を全力を尽くす(つくす)
\\	金髪(きんぱつ) 彼の髪は金髪で、彼は若く見える
\\	結び目(むすびめ) 結び目を解く(ほどく)
\\	制限速度(せいげん) 限る(かぎります) 制限速度(せいげんそくど) 時間制限(じかんせいげん) あなたの意見を「はい」か「いいえ」だけに制限しないでください
\\	隠れん坊 (かくれんぼう)
\\	漢字と仮名の辞典	漢和辞典 (かんわじてん)
\\	指間接(ゆびかんせつ)
\\	ラード
\\	ラム肉 子羊肉(こひつじにく)
\\	レーザー
\\	イースト パン粉と水とイーストから作られる
\\	ソフト(ウエア)
\\	ソフトドリンク
\\	りんごのように赤い 山田さんは頬(ほお)がまるでりんごのように赤いです
\\	苦労 (くろう) 心配(しんぱい) 
\\	手数(てすう) 
\\	ご苦労さま! 親には大変苦労をかけた どれほど苦労してきたか誰にもわからない
\\	君はまだ苦労が足りないね
\\	~中(じゅう) 学校中(がっこうじゅう) 日本中 私は家中を探しちゃった
\\	家中(うちじゅう) うち中を探す 彼はうち中の人気者だ
\\	人気者(にんきもの) 彼はうち中の人気者だ
\\	足音(あしおと) 誰かの足音が聞こえてきました
\\	足が遅い/速い
\\	~は~なりの~ ~はそれなりの~ ジャズ音楽はジャズ音楽なりの魅力がある ジャズ音楽はそれなりの魅力がある
\\	良好(な)(りょうこう) 手術後の経過(けいか)は良好だ
\\	維持(いじ) こんな大きな家を維持するのはお金がかかる
\\	はと時計
\\	私の時計は3分進んでいる[遅れている]
\\	右回りの[に](みぎまわり) 時計回り (とけいまわり) 取っ手を右いっぱいに回す 時計回りをする
\\	左回りの[に] 右回りの[に](みぎまわり) 時計回り(とけいまわり)
\\	これと似ている(にている) 山田さんはこれと似ている時計がある
\\	フランス語とスペイン語は文法がそっくりだ フランス語はスペイン語と文法がそっくりだ
\\	明瞭(な)(めいりょう) はっきり その意味は極めて明瞭だ 明瞭な発音 もっと明瞭に言ってくれませんか
\\	新世界秩序 (しんせかいちつじょ)
\\	人参 
\\	人参的药用价值很高 
\\	责任不在我 
\end{CJK}
\end{document}