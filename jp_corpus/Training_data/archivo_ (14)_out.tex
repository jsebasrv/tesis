\documentclass[8pt]{extreport} 
\usepackage{hyperref}
\usepackage{CJKutf8}
\begin{document}
\begin{CJK}{UTF8}{min}
\\	[冠詞]	その、随一の	
\\	その猫は可愛らしい。	
\\	[前置詞]	…の	
\\	しなくてはならない宿題がたくさんある。	
\\	[接続詞]	…と…、足す、そして	
\\	私は地理と歴史が好きだ。	
\\	[前置詞]	…へ、…に対して	
\\	彼はあなたに結婚を申し込んだのですか。	
\\	[前置詞]	…の中に、…において	
\\	君は政治に関心をもっているかい。	
\\	[代名詞]	私は	
\\	我思う、故に我あり。	
\\	[代名詞]	それ、あれ	
\\	そのホテルはあなたの期待通りでしたか。	
\\	動詞]	
\\	の一人称・三人称・単数・過去、だった	
\\	彼の会社は簡単に見つかった。	
\\	[限定詞]	彼の、彼のもの	
\\	ついに彼は自分の誤りに気付いた。	
\\	[代名詞]	彼は	
\\	「黙って」と彼はささやいた。	
\\	[代名詞]	それは	
\\	おじいちゃんに買ってもらったんだー!	
\\	[前置詞]	…と一緒に	
\\	ナンシーが僕とデートするなんて有り得っこないんだ。高嶺の花だよ。	
\\	動詞]	
\\	の三人称・単数・現在、です	
\\	その猫は可愛らしい。	
\\	[前置詞] [接続詞]	…の間、…のために	
\\	経費として二千二百ドルを計上しております。	
\\	[副詞] [接続詞]	同じくらい、同様に	
\\	畏まりました。	
\\	[代名詞]	あなた(たち)は、あなた(たち)を、あなた(たち)に	
\\	「気分はどうですか。」と彼は尋ねた。	
\\	[副詞]	…ではない	
\\	なぜいけないか。	
\\	動詞]	ある、存在する	
\\	ばかげたことを言うな。	
\\	[限定詞] [代名詞]	彼女の、彼女に、彼女を	
\\	彼女は料理が大好きです。	
\\	[前置詞]	…の上に、…について	
\\	右わき腹を下にして横になってください。	
\\	[前置詞]	…で、…に	
\\	結局、私たちはそのボロボロなレストランにご飯を食べてしまいました。	
\\	[前置詞]	…のそばに、までに、によって	
\\	15日までに準備出来ていなければならない。	
\\	[代名詞]	どちら、(関係詞)	
\\	ロンドン行きの列車のプラットホームはどちらですか。	
\\	[動詞] [不規則動詞]	持っている	
\\	頭痛がする。	
\\	[接続詞]	または、さもないと	
\\	そのネズミは生きているのか死んでいるのか?	
\\	[前置詞]	…から	
\\	太陽に手をかざした。	
\\	[代名詞]	これ、この	
\\	このテーブルは予約済みです。	
\\	[代名詞]	彼を、彼に	
\\	私は彼のあごに1発食らわしてやった。	
\\	[副詞] [前置詞] [接続詞]	しかし、を除いて、だけ	
\\	しかしなぜ。	
\\	[形容詞] [代名詞]	全ての	
\\	私たちみんな恒久的な世界平和を願っています。	
\\	[代名詞]	彼女は、彼女が	
\\	彼女はパンプスをはいていた。	
\\	[代名詞]	彼らが、人々	
\\	彼らは断った。	
\\	動詞]	
\\	の二人称単数・三人称複数・過去、だった	
\\	だれも満足しなかった。	
\\	[限定詞]	私の	
\\	私の母は行儀作法にやかましい。	
\\	動詞]	である	
\\	まだ怒ってる?	
\\	[代名詞]	私に、私を	
\\	トムは僕に質問し続けた。	
\\	[名詞] [数詞]	一つ、1	
\\	一つ選んで。	
\\	[限定詞]	彼らの	
\\	彼らは婚約を破棄した。	
\\	[副詞] [接続詞] [代名詞]	そのように、それほど、なので	
\\	だから何?	
\\	[冠詞]	一つの、ある、
\\	と同じだが、母音の音で終わる言葉の前の冠詞)	
\\	愛は作りごとにすぎない。	
\\	[代名詞]	彼らを、あれら	
\\	彼は彼らを騒がしいとたしなめた。	
\\	[代名詞]	私たちは、人は	
\\	さぁ始めましょう。	
\\	[代名詞]	だれが、 (関係詞)	
\\	ジュリエットはバルコニーから外のロミオに語りかけました。	
\\	[助動詞]	…だったろう	
\\	誰も気にしませんよ。	
\\	動詞]	
\\	の過去分詞形、ずっと…だった	
\\	最近、出生率は低下し続けている。	
\\	[助動詞]	…だろう、…するつもりである	
\\	人生に事故はつきもの。	
\\	[形容詞] [副詞]	一つも…ない、いいえ	
\\	これは何のことかチンプンカンプンだ。	
\\	[代名詞]	いつ、(関係詞)	
\\	泳いでいたら、足がつった。	
\\	[副詞]	そこで、…がある	
\\	そこにいて。	
\\	[接続詞]	もし、…かどうか	
\\	日本語版があったらいいな。	
\\	[名詞] [形容詞] [副詞] [代名詞]	もっと多く、それ以上の	
\\	もっとゆっくり歩きなさい。	
\\	[名詞] [形容詞] [副詞] [前置詞]	外へ	
\\	危ない!	
\\	[名詞] [形容詞] [副詞] [前置詞]	上へ	
\\	急いで!	
\\	[前置詞]	…の中へ	
\\	オフィスへどうぞお入りください。	
\\	[動詞] [不規則動詞]	する	
\\	ちなみに、それはこの話の後日談にする予定です。	
\\	[名詞] [形容詞] [副詞]	いくらかの、どんな…でも	
\\	質問はありますか?	
\\	[限定詞]	あなた[がた]の	
\\	大きなお世話だ。	
\\	[代名詞]	何	
\\	トムは何考えたのかな。	
\\	[名詞]	男、人類、人	
\\	よう!元気?	
\\	複数形チェック
\\	[助動詞]	することができた	
\\	説明してください。	
\\	[形容詞] [副詞] [代名詞]	別の、もう一方の	
\\	他人は地獄だ。	
\\	複数形チェック
\\	[前置詞] [接続詞]	よりも	
\\	遅くなっても何もないよりはまし。	
\\	[限定詞]	私たちの	
\\	わがチームが負けた。	
\\	[形容詞] [副詞] [代名詞]	いくらかの	
\\	何人か遅刻した。	
\\	複数形チェック
\\	[形容詞] [副詞]	非常に	
\\	本当にありがとう。	
\\	[名詞] [動詞]	時間、時代、時間を計る	
\\	時早く過ぎる。	
\\	複数形チェック
\\	[前置詞]	…の上に	
\\	名案を思いついたぞ。	
\\	[形容詞] [副詞] [前置詞]	…について、およそ	
\\	回れ右!	
\\	[助動詞]	かもしれない、してもよい	
\\	雨になるかもしれない。	
\\	[限定詞] [代名詞]	それの	
\\	物には限界というものがある。	
\\	複数形チェック
\\	[形容詞] [副詞] [接続詞]	唯一の、たった	
\\	ほんの15分ですね。	
\\	[名詞] [形容詞] [副詞] [接続詞]	今、ただちに	
\\	その番組は今放送中です。	
\\	複数形チェック
\\	[名詞] [動詞] [形容詞] [副詞] [接続詞]	好む、似たような、っぽい	
\\	私はロウソクの灯りって好きだな。	
\\	複数形チェック
\\	[形容詞]	小さい、ほとんどない	
\\	私はほとんどお金を持っていない。	
\\	[形容詞] [副詞]	その時、それから	
\\	その時に会いましょう。	
\\	[助動詞]	…することができる、…してもよい	
\\	君は泳げますか。	
\\	[助動詞]	すべきである、…のはずである	
\\	眠った方がいいよ。	
\\	[代名詞]	私たちを	
\\	仲間になって。	
\\	複数形チェック
\\	[形容詞] [副詞] [代名詞]	こんな、そんな、など	
\\	何て綺麗な花なんでしょう!	
\\	複数形チェック
\\	[冠詞]	一つの、ある	
\\	1ヶ月以内に帰ってきなさい。	
\\	[形容詞]	大きい、偉大な	
\\	すごいぞ!	
\\	[前置詞] [接続詞]	の前に、以前に	
\\	入る前にノックして。	
\\	[助動詞]	ねばならない、にちがいない	
\\	いいなー。	
\\	[名詞] [数詞]	
\\	ビールを2つください。	
\\	複数形チェック
\\	[代名詞]	これら、これらの 
\\	の複数形)	
\\	これらはあなたの物ですか。	
\\	複数形チェック
\\	[動詞] [不規則動詞]	見る、理解する	
\\	前記参照。	
\\	[動詞] [不規則動詞]	知る、見て分かる	
\\	誰にも分からないよ。	
\\	[副詞] [前置詞]	の上に、を越えて	
\\	私の家に来て下さい。	
\\	[名詞] [形容詞] [副詞]	多くの	
\\	本当にありがとう。	
\\	複数形チェック
\\	[形容詞] [副詞] [前置詞]	下へ	
\\	床に伏せろ!	
\\	[形容詞] [副詞] [前置詞]	の後に	
\\	お先にどうぞ。	
\\	[名詞] [形容詞] [副詞]	第1の、最初に、初めて	
\\	ご婦人からどうぞ。	
\\	複数形チェック
\\	[名詞] [形容詞]	よい、十分な、善	
\\	さようなら!	
\\	複数形チェック
\\	[動詞] [形容詞]	自分自身の、私自身の、所有する	
\\	大きなお世話だ。	
\\	[副詞]	決して…ない	
\\	気にするなよ。	
\\	[名詞] [副詞]	最も多い、たいていの 
\\	と
\\	の最上級)	
\\	たいていの蜘蛛はクモの巣を張る。	
\\	複数形チェック
\\	[形容詞]	古い、年とった	
\\	彼は年を取っている。	
\\	[助動詞]	…すべきである	
\\	さぁ始めましょう。	
\\	[名詞]	昼間、1日	
\\	こんにちは。	
\\	複数形チェック
\\	[代名詞]	どこに	
\\	どこ?	
\\	複数形チェック
\\	[代名詞]	それら、あれらの 
\\	の複数形)	
\\	それはひまわりです。	
\\	複数形チェック
\\	[動詞] [不規則動詞]	来る、行く	
\\	来い!	
\\	[代名詞]	彼自身、彼自身を	
\\	トムは怪我をした。	
\\	複数形チェック
\\	[名詞]	道、やり方	
\\	勤勉は成功への道だ。	
\\	複数形チェック
\\	[動詞]	仕事、働く	
\\	これは効くわ。	
\\	[名詞]	生命、人生	
\\	人生は美しい。	
\\	複数形チェック
\\	[副詞] [前置詞]	…なしに、なしで	
\\	彼は無断でやめた。	
\\	[動詞] [不規則動詞]	行く	
\\	行け。	
\\	[動詞] [不規則動詞]	作る	
\\	君の好きなものを選びなさい。	
\\	[名詞] [副詞]	良く、十分に、井戸	
\\	お見事!	
\\	複数形チェック
\\	[副詞] [前置詞]	…の中を、…じゅうずっと	
\\	もういい。	
\\	動詞] [名詞]	生き物、存在、実在 
\\	の現在分詞)	
\\	僕は今我慢しているんだよ。	
\\	複数形チェック
\\	[形容詞]	長い	
\\	どれくらい?	
\\	[動詞] [不規則動詞]	言う	
\\	うちの子供、「きんぴらごぼう」が言えなくて、いつも「きんぴらぼごう」になっちゃうの。	
\\	[名詞] [助動詞]	かもしれない、権力	
\\	力は正義なり。	
\\	複数形チェック
\\	[名詞] [副詞]	どんなふうに、どれほど	
\\	どう?	
\\	複数形チェック
\\	動詞]	
\\	動詞の一人称単数、直説法、現在形、です	
\\	私は好奇心が強いです。	
\\	[副詞]	…もまた、あまりにも	
\\	遅すぎる。	
\\	[名詞] [動詞] [形容詞] [副詞]	平らな、偶数の、…でさえ、平らになる・する	
\\	こんばんは。	
\\	複数形チェック
\\	[副詞]	再び、また	
\\	もう一度?	
\\	[名詞] [形容詞] [代名詞]	多くの	
\\	本当にどうも。	
\\	複数形チェック
\\	[名詞] [形容詞] [副詞]	背、後方へ	
\\	後ろを見ろ。	
\\	複数形チェック
\\	[副詞]	ここで、ここに	
\\	ちょっと来て。	
\\	[動詞] [不規則動詞]	考える	
\\	たぶん、この女性は自分のことを、仲のよい男友達の一人としかみていないのだと思います。	
\\	[形容詞]	すべての、毎…、	
\\	毎日運動するようにする。	
\\	[名詞]	人々	
\\	何人ですか。	
\\	複数形チェック
\\	[形容詞] [代名詞]	同一の、同じ	
\\	私もそうです。	
\\	複数形チェック
\\	[名詞] [動詞] [形容詞] [副詞]	最後の(何か)、保つ	
\\	ラスト・オーダーは締め切らせていただきました。	
\\	複数形チェック
\\	[名詞]	考えること、思想、考え、思考	
\\	あなたは理解しているのだと思っていました。	
\\	複数形チェック
\\	[形容詞] [副詞]	離れて、遠い、(ある場所に)いない	
\\	向こうへ行け!	
\\	[形容詞] [副詞] [前置詞]	…の下に	
\\	「道路工事中」	
\\	[動詞] [不規則動詞]	取る	
\\	彼はコインを1枚ポケットから取り出した。	
\\	[名詞] [動詞] [形容詞]	手、手渡す、もともと大きくて手にはまるようにされたもの	
\\	触るな!	
\\	複数形チェック
\\	[形容詞] [副詞]	静止した、まだ、それでも	
\\	じっとしてて。	
\\	[名詞] [動詞]	場所、位置、置く	
\\	住めば京。	
\\	複数形チェック
\\	[名詞] [接続詞]	…する間に、一方では、(し)ながら	
\\	私たちはしばらく休んだ。	
\\	複数形チェック
\\	[形容詞] [副詞]	正しい、ちょうど、したところだ	
\\	とにかく落ち着いて。	
\\	[副詞] [接続詞]	もまた、さらに	
\\	私も行った。	
\\	[形容詞]	若い	
\\	彼は若く見えた。	
\\	[副詞] [接続詞]	もう、まだ〜でない	
\\	まだです。	
\\	[副詞] [接続詞]	…にもかかわらず、しかし	
\\	彼はお金持ちだが、親切ではない。	
\\	[前置詞]	に対して	
\\	私達は戦争に反対だ。	
\\	[動詞] [不規則動詞]	手に入れる	
\\	もっといい成績を得るように努力しなさい。	
\\	[副詞]	かつて	
\\	彼はめったに働かない。	
\\	[動詞] [不規則動詞]	与える	
\\	沈黙は同意を表す。	
\\	[名詞]	神	
\\	神は存在する。	
\\	複数形チェック
\\	[形容詞] [副詞] [前置詞]	から離れたところで、切れて	
\\	触るな!	
\\	[名詞] [動詞]	顔、表面、直面する、向く	
\\	回れ右!	
\\	複数形チェック
\\	[名詞] [副詞]	何も…ない	
\\	何事も起こらなかった。	
\\	複数形チェック
\\	[名詞] [動詞] [形容詞]	右(の)、正しい、立て直す	
\\	その交差点を右に曲がってください。	
\\	複数形チェック
\\	[形容詞] [副詞] [接続詞]	1度、かつて、(し)たら	
\\	もう一回。	
\\	[形容詞] [代名詞]	もう一つの	
\\	私にはもう一人姉がいる。	
\\	複数形チェック
\\	[名詞] [動詞] [接頭辞]	部分、役目、分ける、別れる、(形容詞・名詞)の部分のある	
\\	この工場では自動車の部品を製作している。	
\\	複数形チェック
\\	[名詞] [動詞]	家、宿す
\\	は 
\\	の音がする)	
\\	君の家は修理が必要だ。	
\\	複数形チェック
\\	[名詞]	世界	
\\	世界、こんにちは!	
\\	複数形チェック
\\	[名詞] [動詞] [形容詞]	頭、長、主席の、向かう	
\\	頭痛がする。	
\\	複数形チェック
\\	[名詞] [数詞]	
\\	もうすぐ3時だ。	
\\	複数形チェック
\\	[形容詞]	新しい	
\\	何かあった?	
\\	[名詞] [動詞]	愛(する)、大好きである	
\\	愛してる。	
\\	複数形チェック
\\	[副詞]	いつも	
\\	君はいつも不平を言っている。	
\\	[動詞] [不規則動詞]	置く	
\\	そこに置いて。	
\\	[名詞] [形容詞]	夜、夜の	
\\	おやすみなさい。	
\\	複数形チェック
\\	[形容詞] [副詞] [代名詞]	各々の、それぞれ	
\\	私たちはお互いに愛し合っています。	
\\	複数形チェック
\\	[名詞]	王	
\\	王様は裸だ!	
\\	複数形チェック
\\	[副詞] [前置詞]	…の間に	
\\	両者は何という違いだろう。	
\\	[動詞] [不規則動詞]	話す、見分ける	
\\	教えて。	
\\	[名詞] [動詞]	心、気にする	
\\	気にするなよ。	
\\	複数形チェック
\\	[名詞]	心臓、心、中心	
\\	こころを開いて。	
\\	複数形チェック
\\	[名詞] [形容詞] [代名詞]	少しの(量)、ほとんどない	
\\	そう思っている人が少ない。	
\\	複数形チェック
\\	[前置詞] [接続詞]	…だから	
\\	テレビを見ているからです。	
\\	[名詞]	もの、こと、ものごと	
\\	かわいそうに!	
\\	複数形チェック
\\	[代名詞]	だれを	
\\	才子多病、佳人薄命。	
\\	複数形チェック
\\	[形容詞] [副詞]	遠くに	
\\	そこって遠いですか?	
\\	[名詞] [形容詞]	全体(の)	
\\	私は全身が痛い。	
\\	複数形チェック
\\	[動詞] [不規則動詞]	置く、配置する、一組	
\\	よし始め。	
\\	[形容詞] [接続詞] [代名詞]	両方の	
\\	2人共生きている。	
\\	複数形チェック
\\	[動詞] [不規則動詞]	見つける、分かる	
\\	トムを探してきて。	
\\	[動詞]	見える、見る	
\\	見ててよ。	
\\	[名詞]	外観、みため	
\\	彼は青い顔をしている。	
\\	複数形チェック
\\	[名詞] [動詞] [形容詞]	名前、主な、名付ける、名付けられる	
\\	私の名前はジャックです。	
\\	複数形チェック
\\	[動詞] [不規則動詞]	させる、しよう	
\\	今日は従業員たちを早く家に帰らせよう。	
\\	[名詞]	国	
\\	誰がこの国を支配していたか。	
\\	複数形チェック
\\	[形容詞] [副詞]	いっそうよい、ましな・に	
\\	本当さ。信じた方がいいぜ。	
\\	[名詞]	家	
\\	家に来て。	
\\	複数形チェック
\\	[名詞] [動詞] [形容詞]	面、側面、(他人に対して誰かの)肩を持つ、〇〇側、ある側にある	
\\	右わき腹を下にして横になってください。	
\\	複数形チェック
\\	[名詞]	何か、知らないものや特定されていないものなど	
\\	何とかしろ!	
\\	複数形チェック
\\	[名詞]	瞬間、(特定の)時	
\\	ちょっとまて、君。	
\\	複数形チェック
\\	[名詞] [動詞]	父、父として育てる	
\\	父が家に帰ってきた。	
\\	複数形チェック
\\	[前置詞]	
\\	の間で	
\\	君たち三人でそのケーキを分けなさい。	
\\	[名詞]	進路、コース、講座、教科	
\\	もちろんだよ!	
\\	複数形チェック
\\	[名詞]	女、女性[全体]	
\\	あの女性はだれですか。	
\\	複数形チェック
\\	[形容詞] [副詞]	十分な、足りる、十分に	
\\	もう十分よ!	
\\	[名詞] [動詞]	母、母として育てる	
\\	私の母は音楽が大好きです。	
\\	複数形チェック
\\	[副詞]	間もなく、速やかに	
\\	すぐ来て。	
\\	[形容詞]	いっぱいの	
\\	もう満腹です。	
\\	[名詞] [動詞]	端、終り、終わる	
\\	結局、私たちはそのボロボロなレストランにご飯を食べてしまいました。	
\\	複数形チェック
\\	[名詞]	部屋、場所	
\\	部屋を片付けなさい。	
\\	複数形チェック
\\	[副詞]	ほとんど	
\\	彼女は危うく溺死するところだった。	
\\	[形容詞]	小さい、少ない	
\\	この本は小さい。	
\\	[短縮形]	
\\	の結合した形、できない	
\\	あなたは、上れませんよ。	
\\	[名詞] [動詞]	水、(植物とかに)水をやる	
\\	水をください。	
\\	複数形チェック
\\	[名詞] [動詞]	ほしい、欠乏	
\\	ぼくは涼宮ハルヒの憂鬱を読みたくない。	
\\	複数形チェック
\\	[副詞]	しかしながら	
\\	私は専門家ではないけど。	
\\	[動詞] [不規則動詞]	明かり、火をつける	
\\	水は光を反射する。	
\\	[副詞]	なかなか、かなり	
\\	それはまったく無意味だ。	
\\	[名詞] [動詞]	言葉、言葉にする	
\\	彼はどうしても言葉が出なかった。	
\\	複数形チェック
\\	[代名詞]	だれの	
\\	これ誰の?	
\\	複数形チェック
\\	[名詞]	扉	
\\	戸を閉めろ。	
\\	複数形チェック
\\	[形容詞]	いちばんよい、最高	
\\	よろしくお願いします。	
\\	[動詞]	使う、役立つこと	
\\	これ使って。	
\\	[名詞] [形容詞]	朝、朝の	
\\	おはようございます。	
\\	複数形チェック
\\	[代名詞]	私自身、私自身を	
\\	私は自分を嫌う。	
\\	複数形チェック
\\	[前置詞] [接続詞]	…までずっと	
\\	来週までごきげんよう。	
\\	[副詞] [前置詞] [接続詞]	ので、以来	
\\	いつから?	
\\	[名詞] [動詞]	力、電力、電気を供給する、上手で早くて問題なさそうに済ませる	
\\	彼は権力欲の強い人間だ。	
\\	複数形チェック
\\	[代名詞]	彼ら自身	
\\	彼らはライフルで武装していた。	
\\	複数形チェック
\\	[形容詞]	使い古しの、使われた、中古の、慣れて	
\\	彼はその辞書を使った。	
\\	[副詞]	〜というよりむしろ、いくぶん、それどころか	
\\	それはいささか唐突だ。	
\\	[名詞] [動詞] [形容詞]	ある、現在(の)、プレゼント、プレゼンテーションをする	
\\	彼は私のプレゼントを受け取ってくれた。	
\\	複数形チェック
\\	[名詞] [動詞]	声、発言権、述べる	
\\	もっと大きな声で話しなさい。	
\\	複数形チェック
\\	[名詞] [形容詞]	白	
\\	看護婦さんは白衣を着ている。	
\\	複数形チェック
\\	[形容詞]	もっと少ない、より少ない	
\\	急がば回れ。	
\\	[名詞]	金	
\\	金がものを言う。	
\\	複数形チェック
\\	[形容詞] [副詞] [前置詞]	次の、いちばん近い	
\\	次の方どうぞ。	
\\	[形容詞]	貧相な	
\\	かわいそうに!	
\\	[名詞]	死	
\\	九死に一生を得ました。	
\\	複数形チェック
\\	[名詞] [動詞]	形、形づくる、申込)用紙、書式	
\\	この用紙に必須事項を記入してください。	
\\	複数形チェック
\\	[前置詞]	…以内に、〜の範囲内で	
\\	1ヶ月以内に帰ってきなさい。	
\\	[副詞]	いっしょに	
\\	彼らは一緒に休暇を過ごした。	
\\	[前置詞] [接続詞]	…までずっと 
\\	の俗語)	
\\	いつまでもぶつくさ言っているのは君の勝手だけどね、それじゃあ事態は何一つ変わらないよ。	
\\	[形容詞]	大きい	
\\	フライポテトのラージサイズをください。	
\\	[名詞] [動詞]	物質、事柄、問題、構う	
\\	世に真の大事なし。	
\\	複数形チェック
\\	[名詞] [形容詞]	親切な、種類	
\\	譲り合いましょう。	
\\	複数形チェック
\\	[副詞]	しばしば	
\\	ドラえモンがよくにっこり笑います。	
\\	[形容詞]	確信している、確かな	
\\	確かだよ。	
\\	[代名詞]	彼女自身	
\\	彼女は自分を嫌う。	
\\	複数形チェック
\\	[名詞]	年、1年間、歳	
\\	明けましておめでとう。	
\\	複数形チェック
\\	[名詞] [動詞]	友だち 、 味方、友達になる	
\\	彼らは友達なんですか。	
\\	複数形チェック
\\	[名詞] [形容詞] [副詞]	半分	
\\	半分当たってるわ。	
\\	複数形チェック
\\	[名詞] [動詞]	命令(する)、順序、秣序、注文(する)、並ぶ	
\\	私は新しい家具を注文した。	
\\	複数形チェック
\\	[名詞] [動詞] [形容詞]	ぐるっと回って、丸い、ラウンド、四捨五入する	
\\	ボールは丸い。	
\\	複数形チェック
\\	[名詞] [形容詞] [副詞]	ほんとうの、本物の、正しく	
\\	本当?	
\\	複数形チェック
\\	[名詞] [副詞] [代名詞]	何か、なんでも、ある程度	
\\	何かあった?	
\\	複数形チェック
\\	[動詞] [不規則動詞]	保つ、守る	
\\	静かにしていなさい。	
\\	[名詞]	妻	
\\	彼の奥さんはフランス人です。	
\\	複数形チェック
\\	[名詞]	方法、手段	
\\	どう手を尽くしてもダメだった。	
\\	複数形チェック
\\	[動詞]	信じる	
\\	俺は信じて!	
\\	[形容詞] [副詞] [前置詞]	近くに	
\\	クリスマスが近づいた。	
\\	[名詞] [形容詞]	公共(の)、一般人の	
\\	世論に聞け。	
\\	複数形チェック
\\	[名詞] [動詞]	状態、様子、国家、述べる	
\\	アメリカでは州によって法律が違う。	
\\	複数形チェック
\\	[名詞]	息子	
\\	父も父なら、息子も息子。	
\\	複数形チェック
\\	[名詞] [数詞]	百	
\\	彼は100ドル貯金した。	
\\	複数形チェック
\\	[副詞]	このように、従って、こうして	
\\	沖縄県民斯く戦えり。	
\\	[名詞] [動詞]	望む、希望	
\\	私はそう望みます。	
\\	複数形チェック
\\	[形容詞] [副詞]	ひとりで	
\\	ひとりにさせて!	
\\	[形容詞] [副詞] [前置詞]	の上方へ	
\\	前記参照。	
\\	[名詞] [動詞]	事例、訴訟、(秘密にある場所に入るために)探る	
\\	一件落着!	
\\	複数形チェック
\\	[名詞] [形容詞] [副詞]	親愛なる、貴方・貴男(愛称)、〇〇へ(手紙など)	
\\	親愛なる兄弟、姉妹の皆さん。	
\\	複数形チェック
\\	[名詞]	人	
\\	次の方どうぞ。	
\\	複数形チェック
\\	[形容詞]	高い	
\\	コレステロール値が高いのです。	
\\	[動詞] [不規則動詞]	読む	
\\	今は本読んでる方が楽しいな。基本的にずっと本読みの部類ではあったけど、読まない時期はあんまり読まないんだよね。今はまたたくさん読む波が来ている感じかな。	
\\	[名詞]	都市、市	
\\	なんて美しい町なんだろう!	
\\	複数形チェック
\\	[副詞]	もう、すでに	
\\	さっさと死ね!	
\\	[形容詞]	一般に受け入れられた	
\\	私は歓迎をうけた。	
\\	[名詞]	事実	
\\	その事実は否定できない。	
\\	複数形チェック
\\	[名詞]	女の子	
\\	あの少女は誰ですか。	
\\	複数形チェック
\\	[形容詞]	名高い、知られた、既知の	
\\	彼は皆に知られている。	
\\	[動詞] [不規則動詞]	聞こえる	
\\	音楽が聞こえる。	
\\	[名詞] [前置詞]	…倍する、時代	
\\	彼は数回来ました。	
\\	複数形チェック
\\	[形容詞] [副詞]	最も小さい、少なくとも、せめて	
\\	ノックぐらいしろよ.	
\\	[副詞]	おそらく	
\\	ひょっとすると彼は来るだろう。	
\\	[形容詞] [副詞]	確信して、確実な	
\\	本当?	
\\	[副詞] [感動詞]	実に、なるほど、本当に	
\\	彼は実に頭がいい。	
\\	[名詞] [形容詞]	英語、イギリス人	
\\	君は英語を勉強する。	
\\	複数形チェック
\\	[動詞]	開く、開いている	
\\	開けてくれ。	
\\	[名詞]	身体	
\\	絶対にダメだ。	
\\	複数形チェック
\\	[代名詞]	そのもの、それ自身(を)	
\\	歴史は繰り返す。	
\\	複数形チェック
\\	[副詞] [前置詞]	…をたどって、沿って	
\\	さあ、来たまえ。	
\\	[名詞] [動詞]	陸、土地、到着する	
\\	この土地は誰が所有しているのですか。	
\\	複数形チェック
\\	[名詞] [動詞]	帰る、返す、帰り	
\\	大使は戻りました。	
\\	複数形チェック
\\	[動詞] [不規則動詞]	去る	
\\	左に曲がって。	
\\	[名詞]	空気、空中	
\\	蒲団を干して下さい。	
\\	複数形チェック
\\	[名詞]	自然、本質	
\\	習慣は第二の天性である。	
\\	複数形チェック
\\	[副詞] [前置詞] [接続詞]	どちらかの、両方の1つ	
\\	どちらの日でも結構です。	
\\	[名詞]	法、法律	
\\	彼は法を犯した。	
\\	複数形チェック
\\	[動詞]	手伝う、助ける、助け	
\\	助けて!	
\\	[動詞] [不規則動詞]	置く、横たえる	
\\	彼は仰向けに寝ていた。	
\\	[名詞] [動詞]	点、要点、指し示す	
\\	コンパスは北を示す。	
\\	複数形チェック
\\	[名詞]	子供	
\\	私達は子供を養子にした。	
\\	複数形チェック
\\	[名詞] [動詞]	手紙、文字、文字をつける	
\\	あなたの手紙は私のと行き違いになった。	
\\	複数形チェック
\\	[名詞] [数詞]	
\\	4時より前にはビール禁止!	
\\	複数形チェック
\\	[名詞] [動詞]	望む、願い、願う	
\\	願い事をして。	
\\	複数形チェック
\\	[名詞] [動詞]	火、火事、首にする	
\\	火事だ!	
\\	複数形チェック
\\	[動詞] [不規則動詞]	話す	
\\	彼は話しかけられないとしゃべらなかった。	
\\	[名詞] [動詞]	数、数字をつける、数え込む	
\\	数字を一つ選んで。	
\\	複数形チェック
\\	[副詞]	それゆえ、ゆえに、その結果	
\\	我思う、故に我あり。	
\\	[名詞]	1時間、時刻	
\\	彼は1時間ねむった。	
\\	複数形チェック
\\	[動詞] [形容詞]	自由な、無料の、(生き物を)放す	
\\	私は暇だ。	
\\	[名詞] [動詞]	戦争、争い、(国が)戦う	
\\	戦争が好きな人は、いない	
\\	複数形チェック
\\	[前置詞]	…の間ずっと	
\\	留守中いい子でいるんですよ。	
\\	[形容詞]	いくつかの、3つ以上の	
\\	彼は数回来ました。	
\\	[名詞]	商売、事業	
\\	商売は商売。	
\\	複数形チェック
\\	[接続詞]	かどうか	
\\	その事は改善されないだろう。	
\\	[名詞]	方法、行儀、作法、マナー	
\\	勉強は効率的にしなくてはいけない。	
\\	複数形チェック
\\	[動詞] [形容詞] [副詞] [代名詞]	第2の、秒、2番目に、口を添える	
\\	ちょっと待って。	
\\	複数形チェック
\\	[名詞] [動詞]	理由、根拠、理性、(誰かと)ことを分ける	
\\	これは理屈で割り切れない。	
\\	複数形チェック
\\	[名詞] [動詞]	呼ぶ、電話をかける、通話	
\\	ラスト・オーダーは締め切らせていただきました。	
\\	複数形チェック
\\	[名詞] [形容詞]	一般の、将軍、全般の	
\\	市長は大衆を前に演説をした。	
\\	複数形チェック
\\	[副詞] [接続詞]	なぜ	
\\	なんで?	
\\	[形容詞] [副詞] [前置詞]	の後ろに、遅れて	
\\	後ろを見ろ。	
\\	[形容詞]	死んだ	
\\	彼女は死んでいます。	
\\	[名詞]	地球、地、土	
\\	地球は丸い。	
\\	複数形チェック
\\	[名詞]	少年	
\\	やれやれ・・・。	
\\	複数形チェック
\\	[形容詞]	失った、負けた、道に迷った	
\\	どっか行け。	
\\	[副詞]	前へ、先、以後	
\\	歓声が上がった。	
\\	[名詞] [数詞]	
\\	1万円でたりる?	
\\	複数形チェック
\\	[短縮形]	
\\	の短縮形、…する	
\\	僕のおごりですよ。	
\\	[名詞] [形容詞]	家族、家族の	
\\	家族は大切だ。	
\\	複数形チェック
\\	[名詞]	魂	
\\	ご冥福をお祈りします。	
\\	複数形チェック
\\	[動詞] [不規則動詞]	触れる、感じる	
\\	寒気がする。	
\\	[名詞] [形容詞]	来たるべき、次の、到来	
\\	今、行くわ。	
\\	複数形チェック
\\	[名詞]	イングランド、英国	
\\	彼はイギリス出身である。	
\\	複数形チェック
\\	[名詞] [動詞]	質問、問い、質問する	
\\	よい質問だ。	
\\	複数形チェック
\\	[名詞] [動詞]	心配、細心の注意、世話、(何かについて)気を使う、大切にする	
\\	用心しなさい。	
\\	複数形チェック
\\	[名詞]	真実性、真理、真実、事実	
\\	真実が勝つ。	
\\	複数形チェック
\\	[名詞]	地面	
\\	地面が動いた。	
\\	複数形チェック
\\	[副詞]	実際に、本当に	
\\	本当?	
\\	[名詞] [動詞]	休む、休める、休息、その他、休養、残ってる分	
\\	休息しろ。	
\\	複数形チェック
\\	[動詞] [不規則動詞]	意図する、意味する、平均値	
\\	本気で言ってるんです。	
\\	[形容詞]	異なった、個々の、違う	
\\	トムは以前とは違って見える。	
\\	[名詞]	作り、作成、
\\	の動名詞 作る	
\\	私の言っていることがおわかりですか。	
\\	複数形チェック
\\	[形容詞]	可能な、ありうる、可能性のある	
\\	可能性あるかな?	
\\	[前置詞]	…へ向かって	
\\	全員がトムに向かって駆けつけた。	
\\	[名詞] [形容詞]	人間(の)、人(の)	
\\	彼女は、人間です。	
\\	複数形チェック
\\	[形容詞]	短い	
\\	人生は短い。	
\\	[名詞]	町	
\\	トムは街を離れた。	
\\	複数形チェック
\\	[名詞] [形容詞]	次の、次に述べるもの、続く、次の、随伴	
\\	私は指示に従っているだけです。	
\\	複数形チェック
\\	[名詞]	必要、する・の必要がある、要る	
\\	君が必要だ。	
\\	複数形チェック
\\	[名詞] [動詞]	原因、理由、引き起こす	
\\	何が原因ですか。	
\\	複数形チェック
\\	[名詞]	夕方	
\\	こんばんは。	
\\	複数形チェック
\\	[名詞] [数詞]	
\\	ハイタッチ!	
\\	複数形チェック
\\	[形容詞]	力の強い、じょうぶな	
\\	彼は強い。	
\\	[形容詞]	できる、有能な	
\\	あなたは泳ぐ事が出来ますか。	
\\	[名詞] [形容詞]	フランス語、フランス人	
\\	彼はフランス語がわかる。	
\\	複数形チェック
\\	[動詞]	生きる、生きている	
\\	彼らはそこにすんでいる。	
\\	[名詞]	淑女、婦人	
\\	彼女はなかなかあか抜けている。	
\\	複数形チェック
\\	[名詞] [動詞]	話題、教科、主題、科目、(まずいことに)会わせる	
\\	彼女は話題を変えた。	
\\	複数形チェック
\\	[名詞] [動詞]	答え、答える	
\\	ともかく	
\\	複数形チェック
\\	[名詞]	海	
\\	その海は青い。	
\\	複数形チェック
\\	[名詞] [動詞]	恐れ、不安、恐れる	
\\	彼らはあなたを怖がっていた。	
\\	複数形チェック
\\	[動詞] [不規則動詞]	理解する	
\\	了解しました。	
\\	[形容詞]	堅い、難しい、熱心に	
\\	しっかり勉強しなさい。	
\\	[名詞] [動詞]	疑い、疑う	
\\	私はそれを疑った。	
\\	複数形チェック
\\	[副詞] [前置詞]	…の周りに	
\\	振り向いて。	
\\	[動詞]	質問する、頼む	
\\	アレックスに聞いてみなさい。	
\\	[名詞]	武器、腕の複数形	
\\	彼は腕の力が強い。	
\\	複数形チェック
\\	[動詞]	回す、回る	
\\	振り向いて。	
\\	[名詞] [動詞]	感覚、察する、センス	
\\	これは何のことかチンプンカンプンだ。	
\\	複数形チェック
\\	[形容詞]	黒い	
\\	カラスは黒い。	
\\	[動詞] [不規則動詞]	持ってくる	
\\	お茶を持ってきて。	
\\	[形容詞]	美しい	
\\	なんてきれいなこと。	
\\	[動詞]	閉じる、終える	
\\	おしい。	
\\	[形容詞]	暗い	
\\	暗かった。	
\\	[動詞] [不規則動詞]	持つ、保つ	
\\	動かないで!	
\\	[名詞]	性格、登場人物、人柄、キャラクター	
\\	彼は変わった人だ。	
\\	複数形チェック
\\	[名詞] [動詞]	種類、並べ替える	
\\	トムはいい人だ。	
\\	複数形チェック
\\	[名詞] [動詞]	見受ける、視力、見ること、景色、景観	
\\	陸が見えてきた。	
\\	複数形チェック
\\	[名詞] [数詞]	
\\	10日間たった。	
\\	複数形チェック
\\	[動詞] [不規則動詞]	見せる、見せ物	
\\	姿をあらわせ!	
\\	[名詞] [動詞]	社交的な集まり、党、パーティー(する)	
\\	さあ、みんなで盛り上がろう!	
\\	複数形チェック
\\	[名詞] [動詞] [副詞]	すばらしい、立派な、細かい、細い、結構、罰金(を取る)	
\\	万事具合がいい。	
\\	複数形チェック
\\	[名詞] [動詞]	準備ができた、準備する、用意する	
\\	15日までに準備出来ていなければならない。	
\\	複数形チェック
\\	[名詞]	話、物語、階	
\\	話がそれた。	
\\	複数形チェック
\\	[形容詞]	共通の、一般の	
\\	彼は非常識だ。	
\\	[名詞] [動詞]	本、帳簿、予約する	
\\	彼は作家です。	
\\	複数形チェック
\\	[名詞] [形容詞]	電子(の)	
\\	複数形チェック
\\	[名詞] [動詞]	話す、話し	
\\	金がものを言う。	
\\	複数形チェック
\\	[名詞] [動詞]	計算、口座、説明(する)	
\\	あなたの預金はありません。	
\\	複数形チェック
\\	[名詞] [動詞]	跡、印、マークする	
\\	マークが私の本を持っている。	
\\	複数形チェック
\\	[名詞] [動詞]	興味、利益、利子、(誰かなどの)興味になる	
\\	面白い。	
\\	複数形チェック
\\	[形容詞]	書かれた、書いてある、書いた	
\\	あなたの苗字はどのように綴るのですか?	
\\	[短縮形]	
\\	の短縮形、できない	
\\	有り得ない!	
\\	[名詞]	ベット	
\\	ベッドメイキングをしなさい。	
\\	複数形チェック
\\	[形容詞]	必要な、必然的な	
\\	それは必要だった	
\\	[名詞] [動詞]	年齢、古くなる、年を取る、保つ	
\\	「何歳?」「16歳です」	
\\	複数形チェック
\\	[副詞]	そのほかに、さもないと	
\\	他に何かありますか。	
\\	[名詞] [動詞]	力、強いる、勢い、無理やり~させる	
\\	トムは作り笑いをした。	
\\	複数形チェック
\\	[名詞]	考え、意見、アイデア	
\\	私には考えがある。	
\\	複数形チェック
\\	[副詞] [前置詞]	…を横断して	
\\	飛び越えろ!	
\\	[名詞]	兄弟、兄、弟	
\\	彼は私の兄です。	
\\	複数形チェック
\\	[名詞] [副詞]	早い、初期の、早めに	
\\	早かった。	
\\	複数形チェック
\\	[副詞]	ときどき	
\\	ルーシーはときどきメイを訪ねます。	
\\	[名詞] [動詞]	線、列、線を書きつける	
\\	それは私の台詞だよ!	
\\	複数形チェック
\\	[名詞]	言うこと、ことわざ、発言	
\\	言ってみただけだよ!	
\\	複数形チェック
\\	[名詞]	テーブル、表	
\\	このテーブルは予約済みです。	
\\	複数形チェック
\\	[名詞]	川	
\\	その川は幅が広い。	
\\	複数形チェック
\\	[名詞]	目	
\\	彼女はアイシャドーをつけている。	
\\	複数形チェック
\\	[名詞]	太陽、日光	
\\	太陽が沈みかかっていた。	
\\	複数形チェック
\\	[名詞]	情報、知識	
\\	具体的に話して欲しい。	
\\	複数形チェック
\\	[形容詞] [副詞]	もっと遅い、後ほど 
\\	の比較級)	
\\	また後で。	
\\	[名詞]	すべてのもの	
\\	万物は流転する。	
\\	複数形チェック
\\	[副詞]	突然に	
\\	物価が突然下がった。	
\\	[名詞] [形容詞]	過ぎ去った	
\\	苦しいときの神頼み。	
\\	複数形チェック
\\	[形容詞]	奇妙な、見たことのない	
\\	変だなあ。	
\\	[形容詞]	深い	
\\	どのくらい深い?	
\\	[動詞]	変える、変わる、変化	
\\	服を着替えなさい。	
\\	[名詞]	感覚、感情 
\\	の動名詞、感じる)	
\\	気分悪いの?	
\\	複数形チェック
\\	[動詞]	行為、行動する、演ずる	
\\	もっと早く行動しなさい。	
\\	[動詞] [不規則動詞]	会う	
\\	目と目が合った。	
\\	[名詞] [形容詞] [副詞]	もっと遠い、さらに、さらに遠い・く、進める	
\\	もうちょっと先までいってください。	
\\	複数形チェック
\\	[名詞]	目的、意図、用途	
\\	何のために?	
\\	複数形チェック
\\	[形容詞]	幸福な	
\\	ハローウィンおめでとう。	
\\	[動詞]	…のように見える	
\\	彼は幸福であるようだ。	
\\	[名詞]	血、血統	
\\	献血して下さい。	
\\	複数形チェック
\\	[名詞] [形容詞]	バラの花、バラ色の	
\\	物価が上がった。	
\\	複数形チェック
\\	[名詞]	南	
\\	私達は南米を旅行した。	
\\	複数形チェック
\\	[形容詞]	…を越えて、向こうへ	
\\	それはとても私には出来ない。	
\\	[形容詞]	寒い、冷たい	
\\	寒い。	
\\	[形容詞]	どちらも…ない	
\\	私は酒もたばこもやらない。	
\\	[名詞] [動詞] [形容詞]	前方へ、将来へ、進んで〜する、
\\	メールなどを)回す	
\\	彼は前のめりになった。	
\\	複数形チェック
\\	[名詞] [動詞]	視界、視野、見方、観る	
\\	なんて美しい眺めでしょう。	
\\	複数形チェック
\\	[短縮形]	
\\	の短縮形、…した	
\\	決めました。	
\\	[名詞] [動詞]	場所、地位、姿勢、位置付ける	
\\	彼には役職がついている。	
\\	複数形チェック
\\	[名詞]	音	
\\	あの音は何ですか。	
\\	複数形チェック
\\	[代名詞]	どれも…ない	
\\	だれも満足しなかった。	
\\	[動詞] [形容詞]	透き通った、はっきりした、透明な、片付ける、取り除く	
\\	何言っているか分かる?	
\\	[名詞]	道路、道	
\\	「道路工事中」	
\\	複数形チェック
\\	[形容詞]	遅れた、遅い	
\\	遅すぎる。	
\\	[動詞] [不規則動詞]	立つ、立っている	
\\	立ちなさい。	
\\	[動詞]	思う、推測する	
\\	そう思いますよ。	
\\	[名詞]	娘	
\\	こちらは娘です。	
\\	複数形チェック
\\	[形容詞]	本物の、実質の	
\\	冷静に考えて見ろよ!	
\\	[副詞]	ほとんど、もう少しで	
\\	彼はあやうくおぼれるところだった。	
\\	[代名詞]	私のもの、私の	
\\	私のをどうぞ。	
\\	[名詞]	知識、認識	
\\	全ての知識がよい物とは言えない。	
\\	複数形チェック
\\	[前置詞]	…のほうへ	
\\	私の方を向いてください。	
\\	[形容詞]	悪い、適当でない	
\\	残念・・・。	
\\	[動詞] [不規則動詞]	切る	
\\	やめなさい。	
\\	[名詞] [動詞]	写す、複写(する)、コピー(する)	
\\	コピーとっとかなきゃ。	
\\	複数形チェック
\\	[名詞]	夫	
\\	彼女は夫と離婚した。	
\\	複数形チェック
\\	[名詞] [数詞]	
\\	6時まで待ちなさい。	
\\	複数形チェック
\\	[名詞]	フランス	
\\	フランスが大のお気に入りです。	
\\	複数形チェック
\\	[形容詞]	生きている	
\\	生きるのは危険です。	
\\	[名詞]	平和、平穏	
\\	大統領は平和を望んでいる。	
\\	複数形チェック
\\	[短縮形]	
\\	の短縮形、…なかった	
\\	彼女は返答しなかった。	
\\	[形容詞]	低い	
\\	バッテリー残量が少なくなっています。	
\\	[名詞]	北	
\\	コンパスは北を示す。	
\\	複数形チェック
\\	[動詞]	思い出す、覚えている	
\\	思い出した。	
\\	[名詞] [動詞]	影響、結果、影響する・される	
\\	関数が副作用を持たない。	
\\	複数形チェック
\\	[形容詞]	自然のままの、当然の	
\\	それって、全くもって当然だよ。	
\\	[形容詞] [副詞]	かわいらしい、かなり	
\\	彼女らはかわいい。	
\\	[動詞] [不規則動詞]	落下する、倒れる	
\\	トムは落ちた。	
\\	[形容詞]	公平な、品評会	
\\	フェアに行こうぜ。	
\\	[副詞] [前置詞]	の下方へ	
\\	私の部屋はこの真下です。	
\\	[前置詞]	…を除いて、除く、〜以外は	
\\	来たのはメアリーだけだった。	
\\	[名詞] [形容詞]	アメリカの、アメリカ人	
\\	彼はアメリカ人ですか。	
\\	複数形チェック
\\	[名詞]	毛、髪の毛	
\\	マリアさんの髪は長いです。	
\\	複数形チェック
\\	[名詞]	ロンドン	
\\	もやがロンドンの町をすっぽりと包んだ。	
\\	複数形チェック
\\	[動詞]	通り過ぎる、合格する	
\\	年月が経った。	
\\	[名詞] [動詞]	著作権、著作権をつける	
\\	制度的に大きな枷となるのは、著作権の問題である。	
\\	複数形チェック
\\	[名詞] [形容詞]	軍隊、陸軍の	
\\	彼は陸軍に入った。	
\\	複数形チェック
\\	[動詞] [不規則動詞]	走る	
\\	走れ。	
\\	[名詞]	馬	
\\	乗馬は楽しい。	
\\	複数形チェック
\\	[名詞] [形容詞]	未来、未来の	
\\	未来は誰にも分からない。	
\\	複数形チェック
\\	[名詞] [動詞]	喜び、喜ばせる、(気持ちをよくする)	
\\	喜んで!	
\\	複数形チェック
\\	[名詞]	歴史	
\\	私は歴史が好きだ。	
\\	複数形チェック
\\	[名詞]	西	
\\	列車は西へ走っていた。	
\\	複数形チェック
\\	[動詞] [不規則動詞]	支払う、支払い	
\\	トムは払った。	
\\	[名詞] [形容詞]	赤い	
\\	サクランボは赤い。	
\\	複数形チェック
\\	[動詞]	覚え書き、書き留める	
\\	承知致しました。有難う御座います。	
\\	[接続詞]	ではあるが、〜だけれども	
\\	ジャックは疲れているけれども、元気そうな様子をしていた。	
\\	[名詞] [形容詞]	金	
\\	金は銀より重い。	
\\	複数形チェック
\\	[名詞] [動詞]	欲望、欲求、願望、切望する	
\\	大統領は平和を望んでいる。	
\\	複数形チェック
\\	[名詞] [動詞]	芝居、遊ぶ、競技をする、演ずる	
\\	私はテニスをしました。	
\\	複数形チェック
\\	[名詞] [動詞]	主人、名人、マスター(する)	
\\	はい、旦那様。	
\\	複数形チェック
\\	[名詞]	事務所	
\\	オフィスへどうぞお入りください。	
\\	複数形チェック
\\	[名詞] [形容詞]	前面、正面の	
\\	前歯が抜けました。	
\\	複数形チェック
\\	[形容詞]	大きい、重要な	
\\	どのくらい広い?	
\\	[副詞]	確かに、必ず	
\\	彼は実に如才ない男だ。	
\\	[名詞]	風	
\\	風が吹く。	
\\	複数形チェック
\\	[動詞]	受け取る	
\\	私は歓迎をうけた。	
\\	[名詞]	注意、思いやり	
\\	御知らせします。	
\\	複数形チェック
\\	[名詞]	政府、政治	
\\	彼女は地方公務員だ。	
\\	複数形チェック
\\	[前置詞]	…に、の方へ、まで	
\\	他人からしてもらいたいように他人にしなさい。	
\\	[名詞]	教会	
\\	あれは教会です。	
\\	複数形チェック
\\	[名詞]	強さ、力	
\\	彼の体力は弱まりつつある。	
\\	複数形チェック
\\	[名詞]	長さ	
\\	ついに彼は泣きはじめた。	
\\	複数形チェック
\\	[名詞]	会社、仲間	
\\	あの会社は倒産した。	
\\	複数形チェック
\\	[名詞]	紙、新聞	
\\	紙は燃えやすい。	
\\	複数形チェック
\\	[副詞]	多分、おそらく	
\\	トムは多分泣いた。	
\\	[形容詞]	うれしい	
\\	お気に召してよかったです。	
\\	[形容詞]	重要な、有力な	
\\	大事なものか?	
\\	[副詞]	特に	
\\	今日はとくに暑い。	
\\	[形容詞]	より巨大な、郊外を含んだ都市	
\\	牛蒡の天ぷらを食べました。	
\\	[代名詞]	あなた自身	
\\	行儀よくしなさい。	
\\	複数形チェック
\\	[名詞]	仲間、男、やつ	
\\	彼は愉快なやつだ。	
\\	複数形チェック
\\	[名詞]	クマ	
\\	猟師は熊を撃った。	
\\	複数形チェック
\\	[名詞]	意見、主観	
\\	世論に聞け。	
\\	複数形チェック
\\	[名詞]	窓、窓ガラス	
\\	窓を開けなさい。	
\\	複数形チェック
\\	[名詞]	信頼、信仰	
\\	信仰は山をも動かす。	
\\	複数形チェック
\\	[副詞]	前に、以前に	
\\	彼は一年前に死んだ。	
\\	[名詞]	同意、協定、合意	
\\	彼女はこっくりとうなずきました。	
\\	複数形チェック
\\	[動詞]	費用、非難、請求する	
\\	担当は私です。	
\\	[名詞]	美しさ、美人、美	
\\	彼女は美しい。	
\\	複数形チェック
\\	[名詞] [動詞]	腕、鎧う、鎧わせる	
\\	腕が痛む。	
\\	複数形チェック
\\	[名詞] [形容詞]	後者(の)、、後の	
\\	その芝居の後半は少し退屈だった。	
\\	複数形チェック
\\	[名詞]	義務、任務、税	
\\	義務を果たしなさい。	
\\	複数形チェック
\\	[動詞] [不規則動詞]	送る、行かせる	
\\	私をそこにやって下さい。	
\\	[名詞] [動詞]	距離、遠い地点、距離を置く	
\\	絵は少し離れたところに置くとより良く見える。	
\\	複数形チェック
\\	[名詞] [動詞]	静けさ、静寂、沈黙、騙させる	
\\	静かに!	
\\	複数形チェック
\\	[名詞]	足	
\\	足が痛い。	
\\	複数形チェック
\\	[名詞] [形容詞]	野生(の)	
\\	キツネは野生動物です。	
\\	複数形チェック
\\	[名詞] [動詞]	目的語、物、対象、目的、物体、物申す	
\\	彼の意見に反対する人もいる。	
\\	複数形チェック
\\	[動詞]	死ぬ、枯れる	
\\	死ね!	
\\	[名詞] [動詞]	救う、節約する、セーブ(する)	
\\	彼は船乗りを助けた。	
\\	複数形チェック
\\	[名詞]	紳士	
\\	あの紳士は誰ですか。	
\\	複数形チェック
\\	[形容詞]	緑色	
\\	アリスにはグリーンが似合う。	
\\	[名詞] [動詞]	トラブル、悩ます	
\\	彼は困っている。	
\\	複数形チェック
\\	[名詞] [動詞]	ほほえむ、笑顔	
\\	はい、チーズ。	
\\	複数形チェック
\\	[名詞] [形容詞]	正しくない、悪い、罪	
\\	この計算機は調子が良くない。	
\\	複数形チェック
\\	[形容詞]	様々な	
\\	彼等はいろいろな問題について話した。	
\\	[動詞] [不規則動詞]	睡眠、眠る	
\\	僕は眠りたい!生きるよりかは眠りたい!	
\\	[形容詞]	特有の、特別な	
\\	特に痛みはありません。	
\\	[副詞]	ほとんど…ない	
\\	彼はほとんど働かない。	
\\	[名詞] [形容詞]	…の長、最高位の、最も重要な	
\\	読書が彼のおもな楽しみだ。	
\\	複数形チェック
\\	[名詞]	チャンス、偶然	
\\	まず見込みはない。	
\\	複数形チェック
\\	[名詞]	初め	
\\	これより次第に高度を下げてまいります。	
\\	複数形チェック
\\	[名詞]	行動、行為	
\\	彼の行動は我々をぎょっとさせた。	
\\	複数形チェック
\\	[動詞] [不規則動詞]	取引する、処理する、扱う	
\\	先週私はその問題の対応に多くの時間を費やした。	
\\	[名詞] [動詞]	訪問する、…を見物に行く、訪問	
\\	京都を訪問するべきだよ。	
\\	複数形チェック
\\	[名詞] [形容詞]	思考力のある、思考	
\\	トムは考えている。	
\\	複数形チェック
\\	[動詞]	(後に)ついて行く、従う、沿う	
\\	私についてきて。	
\\	[名詞] [動詞]	努力する、試す、味見する	
\\	頑張ってみる。	
\\	複数形チェック
\\	[形容詞]	重い、激しい	
\\	トムは重い。	
\\	[名詞] [形容詞]	甘い、飴	
\\	甘すぎる	
\\	複数形チェック
\\	[名詞] [形容詞]	質素な、平原、平野	
\\	嫌なことをあからさまに言う。	
\\	複数形チェック
\\	[副詞]	ただちに、直ちに、早速	
\\	彼はすぐに出発した。	
\\	[名詞]	口	
\\	口を開けてください!	
\\	複数形チェック
\\	[形容詞]	金持ちの、富んだ	
\\	彼は金持ちになった。	
\\	[短縮形]	
\\	の短縮形、…しない	
\\	そんなことは起きないでしょう。	
\\	[形容詞]	恐れて	
\\	怖いですか。	
\\	[名詞]	パリ	
\\	パリはどこにありますか。	
\\	複数形チェック
\\	[名詞] [形容詞]	一つ、一人用の、独身	
\\	独身ですか。	
\\	複数形チェック
\\	[名詞]	喜び	
\\	誰もが喜びの歓声を上げました。	
\\	複数形チェック
\\	[名詞]	敵	
\\	おまえは俺の敵だ。	
\\	複数形チェック
\\	[形容詞]	折れた、壊れた、破られた、
\\	の過去分詞)	
\\	それは壊れた。	
\\	[接続詞]	もし…でなければ	
\\	彼は話しかけられない限り、決して自分からは口をきかない。	
\\	[名詞] [動詞]	船、送る	
\\	船は錨を下ろした。	
\\	複数形チェック
\\	[名詞] [動詞]	状態、条件、(あるために)用意する・訓練させる、体調	
\\	トムの容体が悪化した。	
\\	複数形チェック
\\	[動詞]	運ぶ	
\\	これ運んで。	
\\	[前置詞]	…を含んで、入れて、含めて	
\\	トムを含め、みんなが笑った。	
\\	[名詞] [動詞]	影響(する)、影響力	
\\	私、人に影響されやすいタイプなんです。	
\\	複数形チェック
\\	[動詞] [不規則動詞]	書く	
\\	何か書きなさい。	
\\	[動詞]	姿を現す、見える、見えてくる、現れる	
\\	彼は若く見えた。	
\\	[名詞] [副詞]	外側	
\\	戸外で運動しなさい。	
\\	複数形チェック
\\	[名詞]	秘密	
\\	それは秘密です。	
\\	複数形チェック
\\	[動詞] [副詞]	喜ばせる、どうぞ、~ください	
\\	クラスを休まないでください。	
\\	[名詞]	現れること、外観、外見、出現	
\\	私はちょっと顔を出した。	
\\	複数形チェック
\\	[名詞] [形容詞]	悪、邪悪な	
\\	こいつは悪いウサギだった。	
\\	複数形チェック
\\	[名詞] [動詞]	行進する、行進、行進曲、マーチ	
\\	彼は3月に学校を終えた。	
\\	複数形チェック
\\	[代名詞]	するものは何でも、何でも	
\\	まぁ、いいや。	
\\	複数形チェック
\\	[副詞]	遅く、ゆっくり	
\\	ゆっくり運転しなさい。	
\\	[名詞] [動詞]	とどまる、滞在(する)	
\\	そこにいて。	
\\	複数形チェック
\\	[副詞]	その代りとして、その代わりに	
\\	テレビを見てないで、外で遊びなさい。	
\\	[名詞] [形容詞]	青い、青色、悲しい	
\\	ブライアンは憂鬱そうに見える。	
\\	複数形チェック
\\	[形容詞]	不可能な	
\\	馬鹿な!	
\\	[名詞]	期間、ピリオド、生理	
\\	五時間目ですか。	
\\	複数形チェック
\\	[名詞]	妹	
\\	アンには一人も姉妹がいない。	
\\	複数形チェック
\\	[名詞] [動詞]	戦闘、戦う、戦い	
\\	私たちは戦いに勝った。	
\\	複数形チェック
\\	[名詞]	学校、派	
\\	彼は学校が嫌いだった。	
\\	複数形チェック
\\	[名詞]	機会、行事、際	
\\	私たちは時々彼らに会う。	
\\	複数形チェック
\\	[形容詞]	結婚した、結婚している	
\\	私は結婚しています。	
\\	[名詞] [形容詞]	かつての、前者(の)、旧…	
\\	京都は日本の古都だ。	
\\	複数形チェック
\\	[名詞]	食物	
\\	食べ物!	
\\	複数形チェック
\\	[名詞]	若さ、青年、若い人	
\\	俺の青春を返せ!	
\\	複数形チェック
\\	[形容詞]	学問のある、博学な	
\\	トムはフランス語を習った。	
\\	[副詞]	単に、ただ〜だけ	
\\	ただ互いに異なっているというだけだ。	
\\	[名詞] [動詞]	到着する、届く、リーチ(腕とかの長さ)	
\\	彼は目的を達成した。	
\\	複数形チェック
\\	[名詞]	体系、制度、システム	
\\	うちにはちゃんとした暖房施設があります。	
\\	複数形チェック
\\	[名詞] [数詞]	
\\	トムは腕立てを20回した。	
\\	複数形チェック
\\	[名詞]	夕食	
\\	夕飯の支度が出来ました。	
\\	複数形チェック
\\	[形容詞]	静かな	
\\	静かに!	
\\	[副詞]	容易に、楽に	
\\	紙は燃えやすい。	
\\	[名詞] [動詞]	(犬とかを)歩く、歩くこと、散歩	
\\	彼らは歩いた。	
\\	複数形チェック
\\	[名詞]	言語	
\\	トムは手話を学んだ。	
\\	複数形チェック
\\	[名詞]	表現、表情	
\\	何てあいまいな表現だ。	
\\	複数形チェック
\\	[名詞]	週	
\\	来週までごきげんよう。	
\\	複数形チェック
\\	[名詞]	会館、大広間、廊下、ホール	
\\	ホールは満員だった。	
\\	複数形チェック
\\	[名詞]	危険	
\\	苦しいときの神頼み。	
\\	複数形チェック
\\	[名詞]	資産、所有物、土地	
\\	奴隷は所有物として考えられていた。	
\\	複数形チェック
\\	[名詞] [動詞]	驚くべきもの、不思議に思う	
\\	誰かしら。	
\\	複数形チェック
\\	[形容詞]	いつもの	
\\	相変わらず忙しいの?	
\\	[名詞] [動詞]	図、数字、なんとなくわかる、思い描く	
\\	彼はその数字を合計した。	
\\	複数形チェック
\\	[形容詞]	生まれながらの、生まれた、
\\	の過去分詞)	
\\	トムはどこで生まれましたか。	
\\	[名詞]	中庭、コート、法廷、裁判所	
\\	彼女はその事件を裁判ざたにした。	
\\	複数形チェック
\\	[副詞]	一般的に、一般に、広く	
\\	一般的に言えば日本人は内気です。	
\\	[形容詞]	古代の、昔の	
\\	これが古代の法です。	
\\	[名詞] [動詞]	尊敬(する)	
\\	私はブラウンさんを尊敬しています。	
\\	複数形チェック
\\	[名詞] [形容詞]	第3番目(の)	
\\	彼は三等賞を得た。	
\\	複数形チェック
\\	[形容詞]	価値(がある)	
\\	精液は瓶詰めにする価値はあるよ。	
\\	[形容詞]	簡単な、単純な	
\\	それはとても単純です。	
\\	[名詞]	木	
\\	木が倒れた。	
\\	複数形チェック
\\	[動詞]	残る、… のまま	
\\	どうぞ、そのままお座りになっていてください。	
\\	[名詞]	社会	
\\	社会が変化している。	
\\	複数形チェック
\\	[動詞] [不規則動詞]	戦う、戦い	
\\	トムは喧嘩した。	
\\	[名詞]	壁	
\\	この壁は触ると冷たい感じがする。	
\\	複数形チェック
\\	[名詞] [動詞]	結果、ある結果になる	
\\	良い結果が予想される。	
\\	複数形チェック
\\	[名詞]	天国、楽園	
\\	私は最高に幸せだ。	
\\	複数形チェック
\\	[名詞] [動詞]	命令(する)、指揮する	
\\	彼は英語をうまく使いこなす。	
\\	複数形チェック
\\	[名詞]	音、口調、音色	
\\	彼は聴衆に静かな口調で演説した。	
\\	複数形チェック
\\	[動詞]	みなす、関連する、考慮(する)、考える	
\\	よろしくお願いします。	
\\	[形容詞]	ほんの、単なる	
\\	私はそれは単なる偶然だと思う。	
\\	[名詞]	月	
\\	先月引っ越しました。	
\\	複数形チェック
\\	[副詞] [前置詞]	のそばに	
\\	トムはメアリーの隣に座った。	
\\	[形容詞]	音がしない、沈黙の	
\\	彼は黙っていた。	
\\	[名詞] [動詞] [形容詞]	完全な、完璧な、完璧にする	
\\	完璧だ!	
\\	複数形チェック
\\	[名詞] [動詞]	経験、経験する	
\\	彼は経験不足だね。	
\\	複数形チェック
\\	[名詞]	街路	
\\	彼は道路を渡った。	
\\	複数形チェック
\\	[名詞]	書いたこと、執筆	
\\	書き始めなさい。	
\\	複数形チェック
\\	[副詞]	全く、すっかり、完全に	
\\	すべて私の責任です。	
\\	[形容詞]	新鮮な	
\\	もっと新鮮な野菜を食べなさい。	
\\	[名詞]	公爵	
\\	複数形チェック
\\	[形容詞]	おおいのある、…でおおわれた、だらけ 
\\	の過去分詞)	
\\	彼は汗びっしょりだった。	
\\	[動詞] [形容詞]	…行きの、はねる	
\\	かれらは彼の両足を縛りあわせた。	
\\	[名詞]	東	
\\	風は東から吹いている。	
\\	複数形チェック
\\	[名詞]	木材、森	
\\	木が燃える。	
\\	複数形チェック
\\	[名詞] [動詞]	石、(何かに)石を投げる	
\\	彼女はきれいな石を拾い集めた。	
\\	複数形チェック
\\	[副詞]	速く	
\\	早く!	
\\	[名詞] [動詞]	通知、掲示、気がつく	
\\	トムは気が付いた。	
\\	複数形チェック
\\	[形容詞]	輝いている、生き生きした、明るい、うららかな、眩しい	
\\	月は明るい。	
\\	[名詞]	イエス・キリスト、救世主	
\\	クリスト復活!	
\\	複数形チェック
\\	[名詞]	小型の船	
\\	彼は船で旅行した。	
\\	複数形チェック
\\	[名詞] [形容詞]	高貴(の)、貴族(の)	
\\	その老人は堂々とした顔つきをしていた。	
\\	複数形チェック
\\	[副詞]	幾分か、多少、いくらか	
\\	ああ、なんだかイライラする!	
\\	[形容詞]	突然の	
\\	彼は急に痛みを覚えた。	
\\	[名詞] [動詞]	価値する、真価	
\\	それは大変価値がある。	
\\	複数形チェック
\\	[名詞]	方向、指示、方角	
\\	あなたはどの方向へ行こうとしているのですか?	
\\	複数形チェック
\\	[名詞]	いす	
\\	俺のお気に入り椅子だぞ。	
\\	複数形チェック
\\	[形容詞]	当然の、支払期限の、締め切りの	
\\	私は今夜話すことになっている。	
\\	[名詞] [動詞]	支える、支持(する)、応援(する)	
\\	その提案を承認します。	
\\	複数形チェック
\\	[名詞] [動詞]	日付、(恋人と)付き合う、デート(する)	
\\	日にちを選んで。	
\\	複数形チェック
\\	[名詞] [形容詞]	キリスト教徒、キリスト教の	
\\	トムは敬虔なクリスチャンを自認している。	
\\	複数形チェック
\\	[名詞]	村	
\\	この村にはあなたの力が必要なんです。	
\\	複数形チェック
\\	[名詞]	読書、読み物、記事	
\\	読むのを止めなさい。	
\\	複数形チェック
\\	[動詞]	意見が一致する、同意する、賛成する	
\\	賛成です。	
\\	[形容詞]	よく考えた、熟慮した	
\\	奴隷は所有物として考えられていた。	
\\	[名詞]	野原、戦場、分野	
\\	私は野原中を走り回った。	
\\	複数形チェック
\\	[副詞]	かろうじて、ほとんど〜ない	
\\	彼はめったにテレビを見ない。	
\\	[動詞]	待つ、待ち受ける	
\\	待合室で少々お待ちください。	
\\	[名詞]	許可	
\\	あなたは私の許可は必要ありません。	
\\	複数形チェック
\\	[名詞]	成功	
\\	用意にぬかりがなければ成功は確実だ。	
\\	複数形チェック
\\	[名詞]	部分、断片、…個	
\\	こんなのちょろいちょろい。	
\\	複数形チェック
\\	[形容詞]	英国の	
\\	イギリス政府はその計画に反対です。	
\\	[名詞]	話すこと、談話	
\\	私ですが。	
\\	複数形チェック
\\	[名詞]	会話	
\\	教授は英会話を教えている。	
\\	複数形チェック
\\	[形容詞]	適切な、正式な	
\\	その職に就く為にはそれ相応の資格が必要です。	
\\	[名詞]	丘	
\\	私は丘を見た。	
\\	複数形チェック
\\	[名詞]	音楽	
\\	音楽が聞こえる。	
\\	複数形チェック
\\	[名詞]	機会	
\\	トムは昇進の機会を逃した。	
\\	複数形チェック
\\	[短縮形]	
\\	の短縮形	
\\	それは気のめいるような話だ。	
\\	[名詞] [形容詞]	ドイツ人、ドイツ語	
\\	私はドイツ人です。	
\\	複数形チェック
\\	[名詞]	午後	
\\	こんにちは。	
\\	複数形チェック
\\	[動詞]	泣く	
\\	彼女は泣いた。	
\\	[名詞] [動詞] [不規則動詞]	かかる、代価、経費、費用	
\\	「それはいくらかかりましたか」「千円かかりました」	
\\	複数形チェック
\\	[形容詞]	考慮に値する、かなりの、相当量の	
\\	かなりの数の学生が大学に行きたがっている。	
\\	[名詞] [数詞]	
\\	もう7時よ。	
\\	複数形チェック
\\	[形容詞]	私的な、内密に	
\\	話があるんだけど。	
\\	[動詞] [不規則動詞]	座る、止まる、位置する	
\\	座りなさい。	
\\	[名詞]	報道、事件	
\\	ニュースはすぐに広まった。	
\\	複数形チェック
\\	[名詞]	いちばん上の部分、最高位	
\\	てっぺんまで上がった。	
\\	複数形チェック
\\	[名詞]	場面、局面	
\\	声を立てて騒ぐな。	
\\	複数形チェック
\\	[名詞]	結婚(生活)、婚礼	
\\	あわてた結婚はあまり、うまくいったためしがない。	
\\	複数形チェック
\\	[名詞] [動詞]	一歩、階段、踏む	
\\	ちょっとどいて。	
\\	複数形チェック
\\	[名詞]	庭、公園	
\\	なんと美しい庭だこと。	
\\	複数形チェック
\\	[名詞] [動詞]	競走(する)、人種	
\\	トムはレースに勝った。	
\\	複数形チェック
\\	[動詞] [不規則動詞]	始める	
\\	最初から始めよう。	
\\	[前置詞]	につき、〜あたり	
\\	一箱いくらですか?	
\\	[名詞] [形容詞]	個人(の)、個々の、個人的な	
\\	一人あたり7千ドル払った。	
\\	複数形チェック
\\	[名詞]	座ること、	
\\	彼はそこに座っていた。	
\\	複数形チェック
\\	[動詞] [不規則動詞]	習う、覚える	
\\	謙虚さを身につけて。	
\\	[形容詞]	政治の、国家の	
\\	政治的意識を持つようになるには時間がかかる。	
\\	[形容詞]	難しい、気難しい	
\\	これは難しいです。	
\\	[名詞]	小片、少し	
\\	彼女は彼に噛みついた。	
\\	複数形チェック
\\	[名詞]	演説、スピーチ	
\\	フランソアは演説をしました。	
\\	複数形チェック
\\	[名詞] [動詞] [不規則動詞]	横たわる、位置する、嘘(をつく)	
\\	彼は嘘をつくことが大嫌いだった。	
\\	複数形チェック
\\	[動詞] [不規則動詞]	投げる、配役、投げ込む	
\\	豚の前に梨を放り出さないで。	
\\	[動詞] [不規則動詞]	食べる	
\\	食べなさい!	
\\	[名詞]	権威、当局	
\\	彼は職権を乱用する。	
\\	複数形チェック
\\	[副詞]	など	
\\	アルミニウム、マグネシウム、シリコンなどの粉末が爆発する。	
\\	[名詞]	床、階	
\\	何階なの?	
\\	複数形チェック
\\	[形容詞]	病気の、病気である	
\\	彼は病気らしい。	
\\	[形容詞]	独創的な、原物、初期の	
\\	斬新な意見だね。	
\\	[名詞]	幸福	
\\	誰でも幸福を望んでいる。	
\\	複数形チェック
\\	[名詞] [形容詞]	夏、夏の	
\\	夏はどうだった?	
\\	複数形チェック
\\	[動詞]	供給する、規定する、提供する	
\\	羊は羊毛を与えてくれる。	
\\	[名詞] [動詞]	勉強(する)、調査	
\\	勉強しなさい。	
\\	複数形チェック
\\	[名詞]	宗教、信仰	
\\	彼は人々を宗教により迫害しました。	
\\	複数形チェック
\\	[名詞] [動詞]	絵、写真、(イメージを)想像する	
\\	絵を見なさい。	
\\	複数形チェック
\\	[形容詞]	個人的な、個人の、私的な	
\\	私はパソコンが欲しい。	
\\	[名詞]	アメリカ	
\\	ルーシーはアメリカ出身だ。	
\\	複数形チェック
\\	[名詞] [動詞]	じっと見つめる、…に注意する、時計	
\\	危ない!	
\\	複数形チェック
\\	[形容詞]	喜んだ、嬉しい	
\\	トムは喜んだ。	
\\	[形容詞]	公然と認められた、公言された	
\\	彼は有罪と宣告された。	
\\	[形容詞]	熱い、辛い	
\\	暑いです。	
\\	[名詞]	努力、骨折り	
\\	トマス君、努力してよ!	
\\	複数形チェック
\\	[形容詞]	準備のできた	
\\	用意がほとんど出来ていなかった。	
\\	[名詞] [動詞]	逃げる、脱出(する)	
\\	洞窟を脱出!	
\\	複数形チェック
\\	[名詞] [動詞]	企てる、(試)してみる、企て	
\\	彼は逃亡を図った。	
\\	複数形チェック
\\	[形容詞]	想像上の、推定上の、仮定の、はず	
\\	明日雪が降るはずです。	
\\	[動詞] [形容詞]	速い、断食する	
\\	何て速いんだろう。	
\\	[名詞] [動詞]	著者、作者、作家、著す	
\\	彼は作家です。	
\\	複数形チェック
\\	[形容詞]	茶色の	
\\	私はブラウンさんを尊敬しています。	
\\	[形容詞]	断固として、決意した、決心して	
\\	彼女はしたたかな女だ。	
\\	[名詞] [動詞]	苦痛、痛み、苦しめる	
\\	労なくして益なし。	
\\	複数形チェック
\\	[動詞] [不規則動詞]	跳ぶ	
\\	もう春がきている。	
\\	[副詞] [前置詞]	…の下に、地下に	
\\	あるノルドはその木の下。	
\\	[名詞]	1世紀	
\\	彼は19世紀に生まれた。	
\\	複数形チェック
\\	[形容詞]	意図した、意図された	
\\	彼女は買い物に行くつもりだった。	
\\	[形容詞]	柔らかい	
\\	本当に柔らかかった!	
\\	[形容詞]	まっすぐな	
\\	まっすぐ行け。	
\\	[形容詞]	…しそうな、ありそうな、見込みのある	
\\	彼はやって来そうだ。	
\\	[名詞] [動詞]	角、追い詰める	
\\	彼はその角を曲がった。	
\\	複数形チェック
\\	[名詞]	正義、裁判官、公正	
\\	裁判はお金がかかる。	
\\	複数形チェック
\\	[形容詞]	簡単に、単に、質素な	
\\	見ているだけです。	
\\	[動詞]	産出する、生産する	
\\	カナダは良質の小麦を生産する。	
\\	[名詞] [動詞]	信頼(する)、信用(する)	
\\	任せて!	
\\	複数形チェック
\\	[名詞]	ローマ	
\\	ローマは至る所に遺跡がある。	
\\	複数形チェック
\\	[名詞] [動詞]	笑い、笑う、ハハハ	
\\	皆が笑った。	
\\	複数形チェック
\\	[動詞] [不規則動詞]	忘れる	
\\	起立!	
\\	[名詞]	ヨーロッパ	
\\	イタリアはヨーロッパにある。	
\\	複数形チェック
\\	[名詞]	節、通過すること、一節、引用された部分	
\\	左側に秘密の通路がある。	
\\	複数形チェック
\\	[名詞] [数詞]	
\\	彼は8歳だ。	
\\	複数形チェック
\\	[代名詞]	私たち自身、われわれ自身	
\\	我々は心ゆくまで楽しんだ。	
\\	複数形チェック
\\	[名詞] [動詞]	服を着る、衣服、着せる	
\\	着替えて。	
\\	複数形チェック
\\	[名詞]	通過(しつつある)	
\\	運転免許試験に合格して彼はみんなをびっくりさせた。	
\\	複数形チェック
\\	[形容詞]	恐ろしい、ひどい	
\\	それはひどくつらい。	
\\	[名詞]	中間の、媒体、手段	
\\	ミディアムレアにして、ポテトをひとつつけてください。	
\\	複数形チェック
\\	[名詞]	ため、目的、利益	
\\	彼女は議論のために抗議する。	
\\	複数形チェック
\\	[名詞]	呼吸	
\\	私は口臭がひどいのです。	
\\	複数形チェック
\\	[形容詞]	賢い、博識な	
\\	彼女はとても賢い人だ。	
\\	[名詞]	所有、所有物	
\\	今日、その努力が報われ彼らが決定的な証拠を掴んだ事は賞賛に値します。	
\\	複数形チェック
\\	[形容詞]	人に喜びを与える、心地よい	
\\	彼はとても楽しそうだ。	
\\	[副詞]	完全に、完璧に、	
\\	彼はロシア語を完璧に話す。	
\\	[名詞]	記憶、思い出	
\\	彼の記憶力にはほとほと感心する。	
\\	複数形チェック
\\	[副詞]	普通は	
\\	たいてい歩いて行きます。	
\\	[名詞]	墓穴、墓、墓場	
\\	彼は死にかけている。	
\\	複数形チェック
\\	[形容詞]	固定された、固定した	
\\	彼は網を張った。	
\\	[形容詞]	現代の、現代的な	
\\	トムには近代絵画の鑑賞能力がある。	
\\	[名詞] [動詞]	しみ、場所、目にする	
\\	満足です。	
\\	複数形チェック
\\	[動詞] [不規則動詞]	上がる	
\\	起立!	
\\	[動詞] [不規則動詞]	壊す、破る	
\\	文無しなんだ。	
\\	[名詞] [数詞]	
\\	たった50人しかこなかった。	
\\	複数形チェック
\\	[名詞]	島	
\\	日本は島国だ。	
\\	複数形チェック
\\	[名詞]	集会、会合	
\\	会議は終了しました。	
\\	複数形チェック
\\	[名詞] [動詞]	野営地、野営する、キャンプ(する)	
\\	私たちはサマーキャンプで出会った。	
\\	複数形チェック
\\	[名詞]	国民、国家	
\\	全国民が平和を望んでいる。	
\\	複数形チェック
\\	[名詞]	存在、人生	
\\	俺にはなんの意味もない無価値な存在。	
\\	複数形チェック
\\	[名詞] [動詞]	返事・返答(する)	
\\	お返事ください。	
\\	複数形チェック
\\	[短縮形]	
\\	の短縮形	
\\	私は同意するだろう。	
\\	[名詞]	空、天国	
\\	空がだんだん明るくなってきた。	
\\	複数形チェック
\\	[名詞] [動詞]	さわる、感動させる、接触、タッチ	
\\	手を触れるな。	
\\	複数形チェック
\\	[形容詞]	等しい	
\\	人間は全て平等である。	
\\	[名詞]	運、運勢、富、、多額の金	
\\	好運が彼に微笑みかけた。	
\\	複数形チェック
\\	[名詞]	岸、海岸	
\\	彼は海岸沿いに歩いた。	
\\	複数形チェック
\\	[名詞]	領地、領土、分野	
\\	1の定義域が
\\	のとき、
\\	の値域は
\\	となる。	
\\	複数形チェック
\\	[名詞]	状況、立場、境遇	
\\	情勢は絶望だ。	
\\	複数形チェック
\\	[名詞] [動詞]	約束、約束する	
\\	約束するよ。	
\\	複数形チェック
\\	[名詞]	程度、学位、度、資格	
\\	彼女は生物学の学位を持っている。	
\\	複数形チェック
\\	[名詞] [形容詞]	まん中(の)、中間(の)	
\\	その紙をまん中で折りなさい。	
\\	複数形チェック
\\	[名詞] [形容詞]	冬、冬の	
\\	私は冬が好きです。	
\\	複数形チェック
\\	[名詞] [動詞]	計画、予定(する)	
\\	予定があるんだ。	
\\	複数形チェック
\\	[動詞]	許可する、許す、認める	
\\	お酒を飲む人はお断り。	
\\	[形容詞]	青白い、薄い、青ざめる、⻘ざめた	
\\	君は顔色が悪いね?	
\\	[名詞] [動詞]	行為、導く、品行、指揮する	
\\	彼の振る舞いは見事だった。	
\\	複数形チェック
\\	[名詞]	走ること	
\\	走り始めなさい。	
\\	複数形チェック
\\	[形容詞]	宗教の、信仰の厚い、宗教的な	
\\	インカ人は信心深い民族だった。	
\\	[名詞] [動詞]	驚かす、驚き	
\\	これは驚いた。	
\\	複数形チェック
\\	[名詞] [形容詞]	分、瞬間、細かい
\\	少々お待ち下さい。	
\\	複数形チェック
\\	[動詞] [不規則動詞]	導く、率いる	
\\	オナニーは狂気につながる。	
\\	[名詞] [動詞]	動き、動く、感動させる	
\\	そーっと動かしてね。	
\\	複数形チェック
\\	[名詞] [動詞]	止まる、止める、やめる、止まり	
\\	やめろ!	
\\	複数形チェック
\\	[形容詞]	
\\	の比較級 、 高等な	
\\	生活費が高くなってきている。	
\\	[形容詞]	もっと悪い、いっそう悪い	
\\	視力がだんだん落ちています。	
\\	[形容詞]	話される、話す、言う(形容詞として)	
\\	トムは滑舌がいい。	
\\	[名詞] [動詞] [形容詞]	ガラス、コップ、グラスでできてる、ガラスなどで殴る	
\\	彼は眼鏡をかけていた。	
\\	複数形チェック
\\	[名詞] [動詞]	板、委員会、取締役会、搭乗する	
\\	乗客は皆乗りましたか。	
\\	複数形チェック
\\	[名詞]	例、具体的な例、実例	
\\	日本には美しい都市が多い。例えば京都、奈良だ。	
\\	複数形チェック
\\	[名詞] [形容詞]	確かな、金庫	
\\	トムは無事だろうか。	
\\	複数形チェック
\\	[名詞]	失うこと、損害、減少、喪失	
\\	あいつの損を埋めてやらなくちゃならないんだ。	
\\	複数形チェック
\\	[名詞]	医者、博士	
\\	私は医者です。	
\\	複数形チェック
\\	[名詞] [動詞]	申し出る、申し出、提供する	
\\	彼らは援助を申し出た。	
\\	複数形チェック
\\	[名詞]	種類、クラス、階級	
\\	明日授業があります。	
\\	複数形チェック
\\	[動詞]	完成する、全部の、完全な、完成した、終える、仕上げる	
\\	私たちは完全な文を求めています。	
\\	[名詞] [動詞]	接近、立ち入り、アクセス(する)	
\\	彼らは図書館を利用できる。	
\\	複数形チェック
\\	[動詞] [形容詞]	下げる、低くする、もっと低い	
\\	小さい声で話してください。	
\\	[短縮形]	
\\	の短縮形	
\\	あり得ねぇー。	
\\	[形容詞]	繰り返して言われる、繰り返される、反復された	
\\	「いいえ」とイギリス人は繰り返しました。	
\\	[名詞]	暗さ、やみ	
\\	深い闇だった。	
\\	複数形チェック
\\	[名詞] [形容詞]	軍隊(の)	
\\	ミャンマーは軍事独裁政権に支配されている。	
\\	複数形チェック
\\	[形容詞]	暖かい	
\\	あったかくしてなさい。	
\\	[動詞] [不規則動詞]	飲む、飲み物	
\\	トムはお酒を飲む。	
\\	[名詞]	激しい感情、情熱、熱中	
\\	情熱は苦悩を生む。	
\\	複数形チェック
\\	[形容詞]	物質の、物理的な、肉体の	
\\	彼は体の調子がよい。	
\\	[名詞]	例、見本、実例	
\\	別の例を示しなさい。	
\\	複数形チェック
\\	[名詞] [動詞]	出発する、始まる、始め(る)	
\\	クソっ。かかれ!	
\\	複数形チェック
\\	[名詞]	保管、保存	
\\	トムはうまく怒りを抑えることができない。	
\\	複数形チェック
\\	[副詞]	まっすぐに、直接に	
\\	彼はあなたに直接連絡すると思います。	
\\	[形容詞]	まじめな、重大な、本気である	
\\	冗談はやめて。	
\\	[名詞]	帽子	
\\	帽子を脱ぎなさい。	
\\	複数形チェック
\\	[名詞]	犬	
\\	その犬が跳んだ。	
\\	複数形チェック
\\	[名詞]	銀	
\\	金は銀より重い。	
\\	複数形チェック
\\	[形容詞]	主な	
\\	大麦は我国の主産物である。	
\\	[名詞]	誇り、自尊心	
\\	驕傲は人間の弱さである。	
\\	複数形チェック
\\	[名詞] [動詞]	群衆、人込み、囲む、群れる、押し込む	
\\	その電車は混んでいた。	
\\	複数形チェック
\\	[名詞] [動詞]	列車、訓練する	
\\	頭上で電車がガタガタ音を立てていた。	
\\	複数形チェック
\\	[形容詞]	すばらしい	
\\	すごいぞ!	
\\	[名詞] [形容詞]	道徳的な、倫理、道徳の、倫理的な	
\\	それは道徳上の問題だ。	
\\	複数形チェック
\\	[名詞] [形容詞]	瞬間、即時の	
\\	彼女を見たとたんに、彼は真っ青になった。	
\\	複数形チェック
\\	[名詞]	小道、通り道、細道、道	
\\	私たちにまっすぐな道を教えてください。	
\\	複数形チェック
\\	[名詞] [形容詞]	ギリシア人、ギリシア語、ガリシアの	
\\	この語はギリシャ語から出ている。	
\\	複数形チェック
\\	[名詞]	意味、目的	
\\	ほかの意見もある。	
\\	複数形チェック
\\	[名詞] [動詞] [形容詞]	合う、合わせる、嵌合、ふさわしい、嵌る	
\\	このコートは君にぴったりだ。	
\\	複数形チェック
\\	[形容詞]	整然とした、規則正しい	
\\	私は新しい家具を注文した。	
\\	[名詞]	くじ、用地	
\\	ありがとうございます!	
\\	複数形チェック
\\	[短縮形]	
\\	の短縮形	
\\	彼は強い。	
\\	[動詞]	入る、参加する	
\\	入れ!	
\\	[名詞] [動詞]	ルール、規則、支配する	
\\	ルールがないと駄目だ!	
\\	複数形チェック
\\	[名詞]	剣、刀	
\\	クリスはソードを振るった!	
\\	複数形チェック
\\	[名詞] [動詞]	攻撃する、攻撃	
\\	イスラエルはイランを攻撃するか。	
\\	複数形チェック
\\	[名詞] [動詞]	座席、着席させる	
\\	空席はありますか。	
\\	複数形チェック
\\	[名詞]	遊び、競技	
\\	借りパク禁止	
\\	複数形チェック
\\	[名詞]	健康	
\\	健康第一。	
\\	複数形チェック
\\	[名詞]	区切り、節、段落	
\\	10項を参照して下さい。	
\\	複数形チェック
\\	[名詞]	声明、明細書、陳述、述べること	
\\	彼の発言は混乱を引き起こした。	
\\	複数形チェック
\\	[形容詞]	社会の、社会的な	
\\	猫は社会的な動物です。	
\\	[名詞] [動詞]	払い戻す、払い戻し	
\\	返金して欲しい。	
\\	複数形チェック
\\	[形容詞] [感動詞]	ごめんなさい、すみません、ソリー、可愛そうな	
\\	ちょっと失礼。	
\\	[名詞]	勇気	
\\	君の勇気には感心する。	
\\	複数形チェック
\\	[名詞]	優雅さ、思いやり、優雅、上品	
\\	グレイスは怒った顔つきをした。	
\\	複数形チェック
\\	[名詞] [形容詞]	公務、公式の、職員、職務の	
\\	彼は公用で大阪に行った。	
\\	複数形チェック
\\	[動詞] [不規則動詞]	夢、夢を見る	
\\	みんな夢を見る。	
\\	[形容詞]	価値のある、ふさわしくて、~に値する	
\\	弘法も筆の誤り。	
\\	[名詞]	岩	
\\	私はロックが好きです。	
\\	複数形チェック
\\	[接続詞]	もし…ならば、…という条件で	
\\	私達は彼らに食べ物と衣服を与えた。	
\\	[形容詞]	特別な	
\\	カレーはまあまあでした。	
\\	[名詞] [動詞]	頼む、要請(する)	
\\	彼は助けを求めた。	
\\	複数形チェック
\\	[形容詞]	強力な、力強い	
\\	[名詞] [動詞]	ちらっと見る、一見	
\\	私はパンフレットにざっと目をとおした。	
\\	複数形チェック
\\	[名詞]	運動、動き	
\\	君の息子さん学生運動に加わってたそうだね。	
\\	複数形チェック
\\	[名詞]	料金、手数料	
\\	入場料は一人いくらですか。	
\\	複数形チェック
\\	[名詞] [動詞]	分け前、分担、株、割り当て、共にする、共通する	
\\	女たちは何でも共有している。	
\\	複数形チェック
\\	[動詞]	予期する、期待する	
\\	妊娠しています。	
\\	[短縮形]	
\\	の短縮形、できなかった	
\\	眠ることができなかった。	
\\	[動詞] [不規則動詞]	広げる、広まる	
\\	ニュースはすぐに広まった。	
\\	[形容詞]	向こう側の、正反対の、反対の	
\\	反対	
\\	[名詞]	光栄、栄光	
\\	いつまで過去の栄光にしがみついてんだよ。	
\\	複数形チェック
\\	[名詞] [数詞]	
\\	12歳です。	
\\	複数形チェック
\\	[名詞]	空間、間隔、場所	
\\	「スペース・エレベーター」は何ですか。	
\\	複数形チェック
\\	[形容詞]	関与している、 約束済みの、婚約している	
\\	ただいまお話中です。	
\\	[名詞]	ワイン	
\\	彼女はぶどう酒が好きだ。	
\\	複数形チェック
\\	[形容詞]	いつもの、普通の、ただの	
\\	彼はまったく平凡な男だ。	
\\	[名詞] [動詞]	味(がする)、好み、味わう、味見する	
\\	かび臭い味がする。	
\\	複数形チェック
\\	[名詞] [動詞]	鉄、アイロン(をかける)	
\\	私はハンカチにアイロンをかけた。	
\\	複数形チェック
\\	[短縮形]	
\\	の短縮形、…ではない	
\\	それらは彼らのではありません。	
\\	[動詞]	分配する、配布する	
\\	来てくれた人にこのパンフレットを配るのを手伝って下さい。	
\\	[名詞] [動詞]	取引、交換する、貿易	
\\	日本との貿易は容易ではない。	
\\	複数形チェック
\\	[動詞]	よく考える、みなす、考慮に入れる	
\\	まかせてちょうだい。	
\\	[副詞]	非常に、大いに	
\\	あなたのご親切をとてもありがたいと思います。	
\\	[形容詞]	受け入れられた	
\\	我々は彼の申し出を受けた。	
\\	[形容詞]	強いられた、強制的な	
\\	トムは作り笑いをした。	
\\	[名詞] [動詞]	有利な立場・条件(にする)、利点	
\\	彼女は私より有利だ。	
\\	複数形チェック
\\	[形容詞]	明白な、断固とした	
\\	あなたは決まりましたか。	
\\	[名詞]	公務員、将校、警官	
\\	彼は陸軍士官だ。	
\\	複数形チェック
\\	[名詞] [動詞]	比率、相場、割合、利率、評価する	
\\	一泊いくらですか。	
\\	複数形チェック
\\	[名詞]	衣服	
\\	服を着替えなさい。	
\\	複数形チェック
\\	[名詞] [動詞]	身ぶり、標識、署名する、手話をする	
\\	ここでサインして。	
\\	複数形チェック
\\	[名詞] [形容詞]	土着(の)、出生地、生まれつきの	
\\	スペイン語が彼女の母語です。	
\\	複数形チェック
\\	[名詞] [動詞]	裁判官、審査員、判断する	
\\	審査員は誰ですか?	
\\	複数形チェック
\\	[名詞]	違い、差異	
\\	五十歩百歩。	
\\	複数形チェック
\\	[形容詞]	気になる、切望して、心配な、不安な	
\\	私は成功したいんです。	
\\	[動詞]	結婚する	
\\	私と結婚して。	
\\	[名詞]	長、船長、主将	
\\	彼は主将に選ばれた。	
\\	複数形チェック
\\	[形容詞]	仕上がった、洗練された	
\\	トムは終わった。	
\\	[名詞]	範囲、程度	
\\	私はある程度は君に賛成だ。	
\\	複数形チェック
\\	[形容詞]	好奇心の強い、奇妙な	
\\	私は好奇心が強いです。	
\\	[形容詞]	外国の	
\\	外国人って面白いなあ。	
\\	[副詞] [前置詞]	に加えて、その上	
\\	ジャクージの傍には蝋燭が煌々と輝いていた。	
\\	[名詞]	方法	
\\	私たちは別の方法を採用した。	
\\	複数形チェック
\\	[形容詞]	すぐれた、優秀な、すばらしい	
\\	トムは素晴らしい。	
\\	[名詞]	信頼、自信	
\\	彼は私の信頼を食いものにした。	
\\	複数形チェック
\\	[形容詞]	印のついた、著しい、際立った	
\\	両者の間には格段の違いがある。	
\\	[名詞]	イエス	
\\	「来なさい」イエスは言われた。	
\\	複数形チェック
\\	[副詞]	正確に、その通りだ	
\\	その通り!	
\\	[名詞]	重要性、大切さ	
\\	音楽の重要性は過小評価されている。	
\\	複数形チェック
\\	[副詞]	最後に、ついに、最終的に	
\\	ついに!	
\\	[名詞]	勘定書、紙弊、法案、請求書、勘定	
\\	ねえ、ビル。	
\\	複数形チェック
\\	[形容詞]	広大な、莫大な	
\\	大多数の子供はアイスが大好きだ。	
\\	[動詞] [不規則動詞]	証明する	
\\	それ科学的に証明されてんの?	
\\	[名詞]	空想、思いつき	
\\	こんなところで会うなんて思いもしなかった。	
\\	複数形チェック
\\	[形容詞]	速い、瞬間の	
\\	何て速いんだろう。	
\\	[副詞] [感動詞]	はい、うん	
\\	はい。	
\\	[動詞]	防ぐ、阻止する	
\\	急用で彼は来る事が出来なかった。	
\\	[名詞]	首	
\\	今日首がいたいです。	
\\	複数形チェック
\\	[名詞]	自由、気まま、解放	
\\	自由の女神はアメリカの象徴である。	
\\	複数形チェック
\\	[形容詞]	おもしろい、興味深い、興味を起こさせる	
\\	面白い。	
\\	[形容詞]	法律の、合法の・な	
\\	これは合法ですか?	
\\	[形容詞]	乾いた	
\\	ポールの髪には潤いがない。	
\\	[形容詞]	混じりけのない、汚れていない、純粋な	
\\	トムは毎日果汁100
\\	のオレンジジュースを飲んでいる。	
\\	[前置詞]	について、...に関して	
\\	泳者の呼吸機能の特性について。	
\\	[動詞] [不規則動詞]	失う、負ける	
\\	どっか行け。	
\\	[形容詞]	取りつかれた	
\\	彼は大きな家と車を2台持っていた。	
\\	[名詞]	証拠	
\\	証拠がありません。	
\\	複数形チェック
\\	[形容詞]	遠い、離れた	
\\	遠くの物がかすんで見えます。	
\\	[名詞] [動詞]	前進、進歩(する)	
\\	仕事はすらすらと運んだ。	
\\	複数形チェック
\\	[形容詞]	似ている、類似した	
\\	俺たちは同類だよ。	
\\	[形容詞]	狭い	
\\	トムは肩幅が狭い。	
\\	[副詞]	全体、全部で、一緒に、	
\\	それは全く間違いです。	
\\	[名詞]	建物	
\\	あの建物を見て。	
\\	複数形チェック
\\	[名詞]	ページ	
\\	1ページ足りない。	
\\	複数形チェック
\\	[副詞]	特に、著しく	
\\	彼は彼女にとくにやさしい。	
\\	[形容詞]	知ったかぶりの、もの知りの、聡明な	
\\	彼女は秘密を知っていることを認めている。	
\\	[名詞]	しっかり握ること、 ホールディング、つかむこと、握り方	
\\	彼女は傘を持っていた。	
\\	複数形チェック
\\	[名詞]	山、山脈	
\\	カルロスは山を登った。	
\\	複数形チェック
\\	[名詞] [動詞]	くまなく捜す、探し、調べる、検索(する)	
\\	さあね。	
\\	複数形チェック
\\	[形容詞]	悲しい	
\\	悲しい。	
\\	[名詞]	罪、罪の行為	
\\	私は罪を告白した。	
\\	複数形チェック
\\	[形容詞]	誇りを持った、誇りに思う	
\\	自分に誇りを持ちなさい。	
\\	[副詞]	はっきりと、明瞭に	
\\	はっきり言いなさい。	
\\	[名詞]	価格	
\\	物価が上がった。	
\\	複数形チェック
\\	[名詞] [数詞]	
\\	私は今30歳です。	
\\	複数形チェック
\\	[形容詞]	病気の、病気である	
\\	悪酔いしました。	
\\	[形容詞]	正直な	
\\	彼は正直そうだ。	
\\	[動詞] [不規則動詞]	閉める	
\\	黙れ!	
\\	[名詞]	銀行、堤防	
\\	君はそれを当てにできる。	
\\	複数形チェック
\\	[名詞]	運命	
\\	そうなる運命だったんだよ。	
\\	複数形チェック
\\	[動詞]	受け取る、認める	
\\	同意すると思うよ。	
\\	[名詞]	森林	
\\	その時森は静かだった。	
\\	複数形チェック
\\	[名詞] [動詞]	測る、計る、対策、測定する	
\\	彼はすべての棚を測った。	
\\	複数形チェック
\\	[名詞]	(生物や生物の)種	
\\	1つの種が絶滅するたびに、我々は自然界の一部を永久に失うことになる。	
\\	複数形チェック
\\	[動詞] [不規則動詞]	探す、捜し求める	
\\	彼は職を求めている。	
\\	[副詞]	そうでなければ、じゃないと、別の方法というと	
\\	私は忙しい、そうでなければ君の招待を受けるだろう。	
\\	[名詞] [動詞]	小川、流れ、流れる、流す	
\\	流れはさほど速くない。	
\\	複数形チェック
\\	[副詞]	注意深く、慎重に	
\\	耳を澄ませてごらん。	
\\	[名詞]	耳	
\\	耳鳴りがするんですよ。	
\\	複数形チェック
\\	[名詞]	パン	
\\	誰がパンを食べたの?	
\\	複数形チェック
\\	[名詞]	底、下部	
\\	ページの下を呼んでごらん。	
\\	複数形チェック
\\	[形容詞]	付け加えた、追加の	
\\	非会員は50ドル増しとなります。	
\\	[名詞] [動詞]	助ける、援助(する)	
\\	私達はお金を出してあげて彼を援助した。	
\\	複数形チェック
\\	[動詞] [不規則動詞]	選択する	
\\	一つ選んで。	
\\	[形容詞]	一致した、同意した	
\\	トムは賛成した。	
\\	[名詞]	動物	
\\	ロジャーはパーティーが大好きだ。	
\\	複数形チェック
\\	[短縮形]	
\\	の短縮形	
\\	急ぐ必要はありません。	
\\	[副詞]	完全に、充分に	
\\	ホテルはその夜、満員だったので、遅い客は何人か断らざるをえなかった。	
\\	[名詞]	大喜び、楽しみ	
\\	彼女は料理が大好きです。	
\\	複数形チェック
\\	[名詞]	総計、量、金額	
\\	利益は300万ドルになるでしょう。	
\\	複数形チェック
\\	[動詞]	得る、取得する、、達成する	
\\	ペリーは彼から貴重な情報を入手した。	
\\	[名詞] [動詞]	税、税金、課税する	
\\	税金をお支払いください。	
\\	複数形チェック
\\	[名詞] [動詞] [形容詞]	十字架、十字形、横切る、怒ってる、交差する	
\\	彼は十字を切った。	
\\	複数形チェック
\\	[形容詞]	厚い、濃い	
\\	彼は厚い眼鏡をかけている。	
\\	[名詞]	能力、才能	
\\	彼の能力には全幅の信頼を寄せている。	
\\	複数形チェック
\\	[名詞]	手配、準備	
\\	この取り決めは一時的なものでしかない。	
\\	複数形チェック
\\	[副詞]	海外へ、海外で	
\\	彼は外国へ行った。	
\\	[名詞]	到着	
\\	彼が来るのを心待ちにしている。	
\\	複数形チェック
\\	[形容詞]	欠席の	
\\	トムは休んでいる。	
\\	[副詞]	わきへ、かたわらに	
\\	ちょっとどいて。	
\\	[名詞]	アクセント、訛り	
\\	トムには強いドイツ訛りがある。	
\\	複数形チェック
\\	[形容詞]	眠って	
\\	彼は眠っている。	
\\	[動詞]	援助する	
\\	手伝いませんよ。	
\\	[名詞]	事故、偶然	
\\	事故が起こったところである。	
\\	複数形チェック
\\	[動詞]	出席する	
\\	出席します。	
\\	[名詞]	観客、聴衆	
\\	観客は退屈しているように見えた。	
\\	複数形チェック
\\	[形容詞]	活発な	
\\	彼女は行動的です。	
\\	[名詞]	自動車	
\\	この工場では自動車の部品を製作している。	
\\	複数形チェック
\\	[名詞]	活動	
\\	その街は活気にあふれていた。	
\\	複数形チェック
\\	[形容詞]	目が覚めて	
\\	眠っちゃダメ。	
\\	[名詞]	俳優	
\\	彼は俳優です。	
\\	複数形チェック
\\	[形容詞]	気が付いている	
\\	彼女は親の視線にきづいていた。	
\\	[名詞]	女優	
\\	彼は女優と結婚した。	
\\	複数形チェック
\\	[形容詞]	ひどい、恐ろしい	
\\	ああ恐い!	
\\	[形容詞]	実際上の,事実上の	
\\	これは実際の出来事です。	
\\	[副詞]	後方へ	
\\	彼は仰向けに倒れた。	
\\	[副詞]	実際は、現に	
\\	むしろトムは嬉しそうだ。	
\\	[名詞]	ベーコン	
\\	ベーコンの匂いがしたと思いました。	
\\	複数形チェック
\\	[動詞]	加える、足す	
\\	注釈を加えないでください。	
\\	[副詞]	下手に、悪く	
\\	彼は行儀が悪かった。	
\\	[名詞]	住所、演説	
\\	君の住所をここにお書きなさい。	
\\	複数形チェック
\\	[名詞]	バランス	
\\	トムはバランスを取り戻した。	
\\	複数形チェック
\\	[動詞]	賞賛する、敬服する	
\\	彼女は彼にあこがれた。	
\\	[名詞]	風船	
\\	赤い風船、欲しい。	
\\	複数形チェック
\\	[名詞]	大人、成人	
\\	大人2枚ください。	
\\	複数形チェック
\\	[名詞]	楽団、バンド	
\\	このバンドの演奏を聞いているの。	
\\	複数形チェック
\\	[名詞]	冒険	
\\	彼は冒険が好きです。	
\\	複数形チェック
\\	[名詞]	棒	
\\	私たちはカラオケ・バーで焼酎を飲んだ。	
\\	複数形チェック
\\	[名詞]	助言、忠告	
\\	あなたのアドバイスが必要なんです。	
\\	複数形チェック
\\	[名詞]	基盤、基礎	
\\	我が社の本拠地は東京である。	
\\	複数形チェック
\\	[動詞]	助言する、忠告する	
\\	彼は用心するように忠告した。	
\\	[形容詞]	基礎の、根本的な	
\\	日本人は米と魚を主食にしています。	
\\	[副詞]	その後	
\\	彼の名前を後になってやっと思い出した。	
\\	[名詞]	代理店、特約店	
\\	トムは広告代理店を立ち上げた。	
\\	複数形チェック
\\	[名詞]	湾、入り江	
\\	その空港は大坂湾にあります。	
\\	複数形チェック
\\	[名詞]	代理人、代理業者	
\\	旅行代理店に問い合わせてみよう。	
\\	複数形チェック
\\	[名詞]	浜	
\\	海岸に行ったよ。	
\\	複数形チェック
\\	[名詞]	豆、実	
\\	奈津さんが安藤さんと結婚したら、「あんどうなつ」になっちゃうんだよね。	
\\	複数形チェック
\\	[感動詞]	ああという発声	
\\	素晴らしい。	
\\	[副詞]	前方へ、先に立って	
\\	どうぞお話し下さい。	
\\	[名詞]	あごひげ	
\\	トムは髭をより合わせた。	
\\	複数形チェック
\\	[名詞]	目的	
\\	私は作家を目指している。	
\\	複数形チェック
\\	[動詞] [不規則動詞]	たたく	
\\	どっか行け。	
\\	[名詞]	航空会社	
\\	コールセルフライはパリに本拠を置く航空会社です。	
\\	複数形チェック
\\	[副詞]	美しく	
\\	彼女は素晴らしい字を書く。	
\\	[形容詞]	生きている	
\\	2人共生きている。	
\\	[名詞]	ミツバチ	
\\	一匹の蜜蜂がブンブンと飛んでいる。	
\\	複数形チェック
\\	[名詞]	アーモンド	
\\	複数形チェック
\\	[名詞]	ビール	
\\	ビールをください。	
\\	複数形チェック
\\	[名詞]	アルファベット	
\\	アルファベットを逆から言ってください。	
\\	複数形チェック
\\	[動詞]	振舞う、行儀よくする	
\\	行儀よくしなさい。	
\\	[動詞]	属する	
\\	アイスランドはデンマークに属していた。	
\\	[名詞]	天使	
\\	君は天使のような人だ!	
\\	複数形チェック
\\	[名詞]	ベルト	
\\	シートベルトを着用して下さい。	
\\	複数形チェック
\\	[名詞]	怒り	
\\	トムは怒りを爆発させた。	
\\	複数形チェック
\\	[動詞] [不規則動詞]	曲げる	
\\	彼女は腰をかがめた。	
\\	[動詞]	発表する	
\\	彼は彼女との婚約を公表した。	
\\	[名詞]	アリ	
\\	シスコさんはアリのようにせっせと働きますね。	
\\	複数形チェック
\\	[名詞]	誕生、出産	
\\	カトリック教徒は産児制限に反対している。	
\\	複数形チェック
\\	[副詞]	とにかく	
\\	ともかく	
\\	[副詞]	どこでも	
\\	私はどこででも眠ることができる。	
\\	[動詞]	非難する	
\\	いまいましい雨だ!	
\\	[副詞]	別々に	
\\	トムは精神的に乱れた。	
\\	[名詞]	毛布	
\\	毛布をください。	
\\	複数形チェック
\\	[形容詞]	目が見えない	
\\	彼女は盲目になった。	
\\	[動詞]	閉鎖する	
\\	彼らは私たちと同じブロックに住んでいます。	
\\	[動詞]	適用する、申し込む	
\\	ビザを申請した。	
\\	[動詞]	近づく	
\\	クリスマスが近づいた。	
\\	[名詞]	エプロン	
\\	お母さんなら台所よ。前掛けかけて。	
\\	複数形チェック
\\	[動詞]	沸騰させる	
\\	卵を一つ茹でてよ。	
\\	[名詞]	面積、地域	
\\	あたりは静かだ。	
\\	複数形チェック
\\	[名詞]	骨	
\\	君は非常にぐうたらだ。	
\\	複数形チェック
\\	[名詞]	本屋	
\\	複数形チェック
\\	[動詞]	きちんと並べる	
\\	母は生け花を教えている。	
\\	[名詞]	本屋	
\\	彼は本屋に立ち寄った。	
\\	複数形チェック
\\	[形容詞]	退屈な	
\\	退屈だなあ。	
\\	[名詞]	小切手	
\\	勘定お願いします。	
\\	複数形チェック
\\	[動詞]	借りる	
\\	貴方は、本を借ります。	
\\	[名詞]	ほお	
\\	彼は彼女の頬にキスをした。	
\\	複数形チェック
\\	[名詞]	上司	
\\	彼は私の上司です。	
\\	複数形チェック
\\	[動詞]	励ます、元気付ける	
\\	乾杯!	
\\	[名詞]	サクランボ	
\\	彼はその桜の木を切り倒した。	
\\	複数形チェック
\\	[動詞]	お辞儀をする	
\\	少年は私におじぎをした。	
\\	[名詞]	胸、胸部	
\\	彼は胸毛が多い。	
\\	複数形チェック
\\	[名詞]	椀、どんぶり	
\\	ご飯は大盛りでお願いします。	
\\	複数形チェック
\\	[名詞]	選択	
\\	君の好きなものを選びなさい。	
\\	複数形チェック
\\	[名詞]	男友達	
\\	彼氏はいますか?	
\\	複数形チェック
\\	[名詞]	葉巻き	
\\	葉巻を吸いますか。	
\\	複数形チェック
\\	[名詞]	脳	
\\	彼は頭がいい。	
\\	複数形チェック
\\	[名詞]	タバコ	
\\	彼はタバコを一本くれとせがんだ。	
\\	複数形チェック
\\	[名詞]	ブレーキ	
\\	ブレーキがきかなかった。	
\\	複数形チェック
\\	[名詞]	映画館	
\\	彼は映画の切符をなくしてしまった。	
\\	複数形チェック
\\	[名詞]	支店、小枝	
\\	代数は数学の1部門です。	
\\	複数形チェック
\\	[名詞]	市⺠、国⺠	
\\	日本人ですか。	
\\	複数形チェック
\\	[形容詞]	勇敢な	
\\	彼女は勇ましかった。	
\\	[動詞]	要求する	
\\	ジョンはその絵の所有権を主張した。	
\\	[名詞]	掃除する	
\\	彼は自分の部屋の掃除をしていた。	
\\	複数形チェック
\\	[動詞]	呼吸する	
\\	普通に呼吸して。	
\\	[名詞]	煉瓦	
\\	彼に忠告したところで、馬の耳に念仏だよ。	
\\	複数形チェック
\\	[名詞]	事務員、フロント係	
\\	彼は銀行員です。	
\\	複数形チェック
\\	[名詞]	診療所	
\\	彼女は高須クリニックに行きました。	
\\	複数形チェック
\\	[形容詞]	あざやかな	
\\	トムはとてつもなく頭がいい。	
\\	[名詞]	布、布切れ	
\\	この布は滑らかな手触りです。	
\\	複数形チェック
\\	[形容詞]	広い	
\\	ジムは肩幅が広い。	
\\	[名詞]	石炭	
\\	石炭は化石燃料だ。	
\\	複数形チェック
\\	[動詞] [不規則動詞]	放送する	
\\	その番組は今放送中です。	
\\	[名詞]	海岸、沿岸	
\\	海岸線に沿って航海せよ。	
\\	複数形チェック
\\	[名詞]	ココア	
\\	カカオはとても苦いことがある。	
\\	複数形チェック
\\	[名詞]	ブラシ、毛筆	
\\	歯磨きをしなさい。	
\\	複数形チェック
\\	[名詞]	硬貨	
\\	彼はコインを1枚ポケットから取り出した。	
\\	複数形チェック
\\	[動詞] [不規則動詞]	燃やす、焼ける	
\\	火は燃える。	
\\	[動詞]	集める	
\\	ジャックは切手を集めている。	
\\	[動詞] [不規則動詞]	破裂する	
\\	彼は突然わっと泣き出した。	
\\	[名詞]	収集	
\\	彼は相当数の蔵書を集めた。	
\\	複数形チェック
\\	[名詞]	茂み	
\\	まわりくどいことをいうな。	
\\	複数形チェック
\\	[形容詞]	色彩に富んだ	
\\	女性はカラフルな傘を好む。	
\\	[名詞]	実業家	
\\	父はビジネスマンです。	
\\	複数形チェック
\\	[名詞]	快適さ	
\\	私は何不自由なく暮らしている。	
\\	複数形チェック
\\	[名詞]	蝶	
\\	あ!蝶々がいる!	
\\	複数形チェック
\\	[形容詞]	気楽な、快適な	
\\	皆さんどうぞお楽に。	
\\	[名詞]	ボタン	
\\	押しボタンを押して下さい。	
\\	複数形チェック
\\	[形容詞]	喜劇の、漫画の	
\\	この漫画は知っていますか?	
\\	[名詞]	タクシー	
\\	タクシーをつかまえないと。	
\\	複数形チェック
\\	[名詞]	電線、ケーブル	
\\	海底ケーブルが両国の間に引かれた。	
\\	複数形チェック
\\	[名詞]	仲間、話し相手	
\\	仲間がすべて寝ていた。	
\\	複数形チェック
\\	[名詞]	鳥かご、おり	
\\	このかごは、針金でできている。	
\\	複数形チェック
\\	[動詞]	比較する	
\\	ロンドンはパリに比べると大きい。	
\\	[名詞]	大学の構内	
\\	彼は学外に住んでいる。	
\\	複数形チェック
\\	[動詞]	不平を言う	
\\	君はいつも不平を言っている。	
\\	[名詞]	ろうそく	
\\	その蝋燭は白くありません。	
\\	複数形チェック
\\	[名詞]	首都、資本金、資金	
\\	私は死刑を廃止します。	
\\	複数形チェック
\\	[副詞]	完全に、すっかり	
\\	すっかり忘れたよ。	
\\	[名詞]	コンピュータ	
\\	パソコンを持ってきなさい。	
\\	複数形チェック
\\	[名詞]	コンサート	
\\	彼はこのコンサートが好きです。	
\\	複数形チェック
\\	[名詞]	大工	
\\	私のおじいちゃんは大工です。	
\\	複数形チェック
\\	[名詞]	現金	
\\	現金ですかクレジットカードですか。	
\\	複数形チェック
\\	[動詞]	含む、入っている	
\\	この箱にはりんごが入っている。	
\\	[名詞]	内容、中身	
\\	満足そうですね。	
\\	複数形チェック
\\	[名詞]	城	
\\	あの城は美しい。	
\\	複数形チェック
\\	[動詞]	続ける、持続する	
\\	話し続けて。	
\\	[名詞]	管理、支配	
\\	誰も私達をコントロールできない。	
\\	複数形チェック
\\	[名詞]	天井	
\\	天井に手が届きますか。	
\\	複数形チェック
\\	[名詞]	1セント銅貨	
\\	彼は私から有り金全部奪った。	
\\	複数形チェック
\\	[名詞]	料理	
\\	料理することは面白い。	
\\	複数形チェック
\\	[名詞]	センチメートル	
\\	1メートルは100センチである。	
\\	複数形チェック
\\	[形容詞]	中心の	
\\	この建物にはセントラル・ヒーターがありますか?	
\\	[名詞]	トウモロコシ	
\\	アメリカンドッグが好きです。	
\\	複数形チェック
\\	[形容詞]	正しい	
\\	正解!	
\\	[名詞]	鎖、連鎖	
\\	彼は立て続けにタバコを吸います。	
\\	複数形チェック
\\	[名詞]	綿	
\\	その工場は綿製品を生産する。	
\\	複数形チェック
\\	[名詞]	チャンピオン	
\\	彼はテニスの優勝者です。	
\\	複数形チェック
\\	[名詞]	夫婦、カップル	
\\	彼らは夫婦だ。	
\\	複数形チェック
\\	[名詞]	章	
\\	金曜日の予習に第4章を読んでおくこと。	
\\	複数形チェック
\\	[名詞]	封筒、包み	
\\	封筒をください。	
\\	複数形チェック
\\	[名詞]	釣り	
\\	彼は釣りを愛好する。	
\\	複数形チェック
\\	[名詞]	消しゴム	
\\	複数形チェック
\\	[動詞]	
\\	固定する	
\\	なおして下さい。	
\\	[名詞]	間違い	
\\	本文がありません!	
\\	複数形チェック
\\	[名詞]	旗	
\\	旗を変えてください。	
\\	複数形チェック
\\	[名詞]	フライト、便	
\\	良いフライトを!	
\\	複数形チェック
\\	[動詞]	浮く、浮かぶ	
\\	木が浮く。	
\\	[名詞]	出来事	
\\	これは重要な行事です。	
\\	複数形チェック
\\	[名詞]	洪水	
\\	洪水の被害はごくわずかだった。	
\\	複数形チェック
\\	[形容詞]	毎日の	
\\	小さな子供のいる家庭は、毎日がどたばたして大変だと思う。	
\\	[動詞]	流れ出る	
\\	万物は流転する。	
\\	[形容詞]	正確な、ちょうどの	
\\	私は彼の述べたことをそっくりそのまま返した。	
\\	[名詞]	試験	
\\	ジョンは試験に合格した。	
\\	複数形チェック
\\	[形容詞]	愚かな	
\\	彼はばかだ。	
\\	[動詞]	検査する、調査する	
\\	お医者さんはトムを診断した。	
\\	[名詞]	外国人	
\\	裕子は外国人と話したことがない。	
\\	複数形チェック
\\	[副詞]	永遠に	
\\	永遠に生き続ける者なし。	
\\	[形容詞]	興奮させる	
\\	スリルがあるなあ。	
\\	[動詞] [不規則動詞]	許す	
\\	許して。	
\\	[名詞]	運動、練習	
\\	戸外で運動しなさい。	
\\	複数形チェック
\\	[動詞]	存在する、生存する	
\\	幽霊は存在する。	
\\	[名詞]	形式的なもの	
\\	堅い書き物は大嫌いだ。	
\\	複数形チェック
\\	[名詞]	出口	
\\	出口はどこですか?	
\\	複数形チェック
\\	[名詞]	40番目	
\\	複数形チェック
\\	[形容詞]	高価な、ぜいたくな	
\\	裁判はお金がかかる。	
\\	[形容詞]	幸運な	
\\	彼の身に引き比べて私は幸福だ。	
\\	[動詞]	説明する	
\\	説明してください。	
\\	[名詞]	14番目、14日	
\\	14世紀、プラハは奇跡的にペストから免れた。	
\\	複数形チェック
\\	[動詞]	表現する	
\\	彼は疑問の声を上げた。	
\\	[名詞]	4番目、4日	
\\	アメリカ独立記念日おめでとうございます!	
\\	複数形チェック
\\	[動詞]	失敗する	
\\	トムは失敗した。	
\\	[名詞]	自由、解放	
\\	自由になりたい。	
\\	複数形チェック
\\	[動詞] [不規則動詞]	凍結する	
\\	動くな!	
\\	[形容詞]	偽りの、間違った	
\\	違うんだ。	
\\	[名詞]	友情	
\\	我々は親睦を深めた。	
\\	複数形チェック
\\	[形容詞]	見慣れている、親しい	
\\	ちょっとよく知らないんだけど。	
\\	[名詞]	燃料	
\\	ガソリンは燃料として使われる。	
\\	複数形チェック
\\	[名詞]	扇、扇風機	
\\	私はあなたのファンです。	
\\	複数形チェック
\\	[名詞]	家具	
\\	家具はほこりをかぶっていた。	
\\	複数形チェック
\\	[名詞]	運賃	
\\	バス賃はいくらですか。	
\\	複数形チェック
\\	[名詞]	服装の流行	
\\	流行を追うな。	
\\	複数形チェック
\\	[動詞]	得る、手に入れる	
\\	労なくして益なし。	
\\	[名詞]	欠点、誤り	
\\	彼は自分の落ち度を認めた。	
\\	複数形チェック
\\	[名詞]	車庫、ガレージ	
\\	車はガレージの中です。	
\\	複数形チェック
\\	[名詞]	好意、親切	
\\	頼みを聞いてくれませんか。	
\\	複数形チェック
\\	[名詞]	門、出入口	
\\	門にかぎを掛けなさい。	
\\	複数形チェック
\\	[形容詞]	お気に入りの	
\\	冬は私の一番好きな季節です。	
\\	[動詞]	集める	
\\	彼らは木の実を拾い集めている。	
\\	[名詞]	ファックス	
\\	ファックス送らなきゃ。	
\\	複数形チェック
\\	[動詞] [不規則動詞]	食事を与える、餌とする	
\\	ライオンは肉を餌とする。	
\\	[形容詞]	優しい、温和な	
\\	彼女はやさしい。	
\\	[名詞]	幽霊	
\\	私は本当に幽霊を見た。	
\\	複数形チェック
\\	[名詞]	女性、雌	
\\	メアリーは女性の名前です。	
\\	複数形チェック
\\	[名詞]	巨人、大男	
\\	ナンシーはそれ以前にはパンダを見たことがなかった。	
\\	複数形チェック
\\	[名詞]	囲い、柵	
\\	彼は塀を乗り越えた。	
\\	複数形チェック
\\	[名詞]	贈り物	
\\	彼は彼女から贈り物を受け取った。	
\\	複数形チェック
\\	[名詞]	お祭り	
\\	彼らはお祭りを中止した。	
\\	複数形チェック
\\	[名詞]	ガールフレンド	
\\	私は君のガールフレンドに会った。	
\\	複数形チェック
\\	[名詞]	第15番、15日	
\\	15日までに準備出来ていなければならない。	
\\	複数形チェック
\\	[名詞]	ゴリラ	
\\	あのね、ゴリラ豪雨じゃなくて、ゲリラ豪雨だよ。ちゃんとわかってる?	
\\	複数形チェック
\\	[名詞]	5番目、5日	
\\	五時間目ですか。	
\\	複数形チェック
\\	[名詞]	学年、等級	
\\	彼は一年飛び級をした。	
\\	複数形チェック
\\	[名詞]	50番目	
\\	50歳の誕生日を迎える少し前に、彼女は孫が生まれたという知らせを聞いた。	
\\	複数形チェック
\\	[名詞]	文法	
\\	文法はとてもややこしい。	
\\	複数形チェック
\\	[形容詞]	壮大な、豪華な	
\\	一等賞は王女からのキスです。	
\\	[動詞]	満たす	
\\	満タンにしてください。	
\\	[名詞]	孫娘	
\\	あなたの孫はどこ?	
\\	複数形チェック
\\	[形容詞]	最終の、最後の	
\\	審判の判定は絶対だ。	
\\	[名詞]	おばあちゃん	
\\	祖母もすっかり歳をとった。	
\\	複数形チェック
\\	[名詞]	おじいちゃん	
\\	おじいちゃんに買ってもらったんだー!	
\\	複数形チェック
\\	[形容詞]	硬い、堅固な	
\\	それは当社の関知するところではない。	
\\	[名詞]	孫息子	
\\	あなたの孫はどこ?	
\\	複数形チェック
\\	[名詞]	漁師	
\\	複数形チェック
\\	[名詞]	ブドウ	
\\	それワインじゃなくてぶどうジュースだよ。	
\\	複数形チェック
\\	[名詞]	いとこ	
\\	こちらは私のいとこです。	
\\	複数形チェック
\\	[名詞]	カウボーイ	
\\	くれぐれも浄水器の悪質訪問販売業者や無料点検商法にはご注意ください。	
\\	複数形チェック
\\	[名詞]	衝突、墜落	
\\	トムは飛行機事故で亡くなった。	
\\	複数形チェック
\\	[動詞]	見えなくなる、姿を消す	
\\	トムは行方不明になった。	
\\	[形容詞]	気が狂った	
\\	気は確かか。	
\\	[動詞]	割引する	
\\	もう少し安くなりませんか。	
\\	[名詞]	クリーム	
\\	アイスクリームが大好きです。	
\\	複数形チェック
\\	[動詞]	話し合う、議論する	
\\	彼らは毎日その問題について議論する。	
\\	[動詞]	創造する	
\\	情熱は苦悩を生む。	
\\	[名詞]	議論、話し合い	
\\	彼らは討論を始めた。	
\\	複数形チェック
\\	[名詞]	創造、作品	
\\	複数形チェック
\\	[名詞]	病気	
\\	医師たちは病気と闘っている。	
\\	複数形チェック
\\	[名詞]	乗組員、チーム	
\\	そのタンカーには乗組員が少ない。	
\\	複数形チェック
\\	[名詞]	ディスク	
\\	椎間板ヘルニアは背骨の間にある椎間板という軟骨が飛び出すものです。	
\\	複数形チェック
\\	[動詞]	嫌う	
\\	彼女は彼を嫌っていた。	
\\	[形容詞]	混雑した	
\\	その電車は混んでいた。	
\\	[名詞]	冠、王冠	
\\	マユコは花の冠をつけていた。	
\\	複数形チェック
\\	[名詞]	地区、地域	
\\	台風が関東地方に上陸した。	
\\	複数形チェック
\\	[名詞]	文化	
\\	文化は国々によって異なる。	
\\	複数形チェック
\\	[動詞]	分ける、分割する	
\\	君たち三人でそのケーキを分けなさい。	
\\	[名詞]	カーテン	
\\	幕があがった。	
\\	複数形チェック
\\	[形容詞]	2倍の	
\\	個室ですか2人部屋ですか。	
\\	[名詞]	曲線、屈曲	
\\	城からそのふもとをゆるやかに湾曲して流れる川全体が見えました。	
\\	複数形チェック
\\	[名詞]	慣習、伝統	
\\	この習慣は日本に特有のものである。	
\\	複数形チェック
\\	[名詞]	階下へ	
\\	アンは階下に降りてきた。	
\\	複数形チェック
\\	[名詞]	顧客	
\\	あいつは苦手だ。	
\\	複数形チェック
\\	[名詞]	繁華街	
\\	トムは買い物に街の方へ出た。	
\\	複数形チェック
\\	[名詞]	周期、循環	
\\	自転車で通勤しています。	
\\	複数形チェック
\\	[名詞]	ダース、12個	
\\	彼女は2ダースの卵を買った。	
\\	複数形チェック
\\	[形容詞]	毎日の、日課の	
\\	連日交通事故が起きる。	
\\	[名詞]	演劇、ドラマ	
\\	私は歌舞伎を研究しています。	
\\	複数形チェック
\\	[名詞]	損害、被害	
\\	もう手遅れだ。	
\\	複数形チェック
\\	[動詞]	ドリルで穴を開ける	
\\	これはドリルではない。	
\\	[名詞]	踊る人	
\\	彼女は踊るのが上手です。	
\\	複数形チェック
\\	[名詞]	飲酒	
\\	父は酒を飲むのをやめた。	
\\	複数形チェック
\\	[名詞]	踊り	
\\	踊り続けて。	
\\	複数形チェック
\\	[動詞]	溺れる、溺れ死ぬ	
\\	彼女は危うく溺死するところだった。	
\\	[名詞]	薬物、薬	
\\	その中毒者は麻薬の過剰摂取で死んだ。	
\\	複数形チェック
\\	[名詞]	挑戦	
\\	君はよくそんなことが言えるね。	
\\	複数形チェック
\\	[形容詞]	酔って	
\\	酔った。	
\\	[形容詞]	耳が不自由な	
\\	ヘレン・ケラーは耳が聞こえない上に目も見えなかった。	
\\	[名詞]	ほこり、ちり	
\\	雨で埃がおさまった。	
\\	複数形チェック
\\	[名詞]	決断	
\\	決めました。	
\\	複数形チェック
\\	[動詞]	稼ぐ	
\\	彼はおおいに稼ぐ。	
\\	[副詞]	深く、非常に	
\\	彼は深呼吸をした。	
\\	[名詞]	イヤリング	
\\	イヤリングを片方失くした。	
\\	複数形チェック
\\	[動詞]	守る	
\\	ジョナサン・スウィフトなら私を弁護するでしょう。	
\\	[名詞]	防衛、防御	
\\	発熱は体の防御反応の一つだ。	
\\	複数形チェック
\\	[名詞]	端、へり	
\\	彼の方が君より一枚上手だ。	
\\	複数形チェック
\\	[動詞]	教育する、指導する	
\\	彼は彼女のおじいさんに教育された。	
\\	[動詞]	配達する	
\\	これを配達してもらえますか。	
\\	[名詞]	教育	
\\	教育は家庭に始まる。	
\\	複数形チェック
\\	[動詞]	要求する	
\\	誰でも幸福を望んでいる。	
\\	[名詞]	⻭医者	
\\	彼は歯医者に行った。	
\\	複数形チェック
\\	[動詞]	拒否する	
\\	トムは全てを否定した。	
\\	[名詞]	18番目	
\\	18世紀の流行は胸の部分を強調していた。	
\\	複数形チェック
\\	[動詞]	あてにする、頼る	
\\	私達はあなたを頼りにしています。	
\\	[名詞]	8番目	
\\	私は中学二年生です。	
\\	複数形チェック
\\	[名詞]	深さ	
\\	私達は、その川の深さをはかった。	
\\	複数形チェック
\\	[名詞]	80番目	
\\	複数形チェック
\\	[動詞]	描写する	
\\	彼は何が起こったのか正確に記述した。	
\\	[名詞]	肘(ひじ)	
\\	ひじを痛めました。	
\\	複数形チェック
\\	[名詞]	砂漠	
\\	友人を見捨てるべきではない。	
\\	複数形チェック
\\	[形容詞]	年⻑者の	
\\	彼女は兄にあこがれを抱いている。	
\\	[名詞]	デザイン、設計	
\\	配管工は配管系の設計・組立を行う。	
\\	複数形チェック
\\	[形容詞]	電気の	
\\	トームは電気自動車を運転する。	
\\	[名詞]	デザイナー	
\\	彼女はデザイナーになりたがっている。	
\\	複数形チェック
\\	[名詞]	電気、電力	
\\	路面電車は電気で動く。	
\\	複数形チェック
\\	[動詞]	破壊する	
\\	お前はいつも何もかも壊すのよ。	
\\	[名詞]	エレベーター	
\\	エレベーターはありますか?	
\\	複数形チェック
\\	[名詞]	詳細	
\\	彼は詳しく説明した。	
\\	複数形チェック
\\	[名詞]	11番目、11日	
\\	私は高校2年生だ。	
\\	複数形チェック
\\	[動詞]	発展させる	
\\	フランス語はラテン語を母体として発達した。	
\\	[形容詞]	空の	
\\	空っぽだ。	
\\	[名詞]	文字盤	
\\	火事の際は119番に電話してください。	
\\	複数形チェック
\\	[名詞]	端、終わり	
\\	始めよければ終わりよし。	
\\	複数形チェック
\\	[名詞]	ダイヤモンド	
\\	ダイヤモンドほど硬いものはない。	
\\	複数形チェック
\\	[名詞]	活力、エネルギー	
\\	我々はエネルギーの需要を減らさなくてはならない。	
\\	複数形チェック
\\	[名詞]	難しさ、困難	
\\	食べ物が噛みにくいのです。	
\\	複数形チェック
\\	[名詞]	エンジン	
\\	このエンジンは調子よく動く。	
\\	複数形チェック
\\	[動詞] [不規則動詞]	掘る	
\\	そのまま掘り続けて。	
\\	[名詞]	技師、工学者	
\\	ボブは技師になりました。	
\\	複数形チェック
\\	[動詞]	指揮する	
\\	私は単刀直入なんです。	
\\	[名詞]	入り口	
\\	入り口はどこですか?	
\\	複数形チェック
\\	[名詞]	グレープフルーツ	
\\	毎食グレープフルーツを食べれば体重が減ると思っている人もいる。	
\\	複数形チェック
\\	[動詞]	想像する	
\\	僕の気のせいかな。	
\\	[形容詞]	感謝している	
\\	私はとても彼らに感謝している。	
\\	[形容詞]	即座の、即時の	
\\	薬はすぐ効いた。	
\\	[名詞]	偉大さ	
\\	この国の湾に爆弾が落下し、独裁が世界を支配しようとしたとき、時の国民が立ち上がり、偉業を達成し、そして民主主義を救うのをクーパーさんは見ていました。
\\	。私たちにはできるのです。	
\\	複数形チェック
\\	[名詞]	挨拶	
\\	挨拶より円札。	
\\	複数形チェック
\\	[動詞]	守る、護衛する	
\\	不意打ちを食らった。	
\\	[名詞]	印象	
\\	アメリカはいかがですか。	
\\	複数形チェック
\\	[動詞]	推測する	
\\	当ててみて。	
\\	[動詞]	改善する	
\\	君の英語は上達したね	
\\	[名詞]	体育館	
\\	トムは体育の先生だ。	
\\	複数形チェック
\\	[名詞]	インチ	
\\	彼は本当の紳士だ。	
\\	複数形チェック
\\	[名詞]	癖、習慣	
\\	習慣は第二の天性である。	
\\	複数形チェック
\\	[動詞]	含む、包含する	
\\	トムを含め、みんなが笑った。	
\\	[名詞]	ヘアブラシ、毛製ブラシ	
\\	複数形チェック
\\	[名詞]	産業	
\\	京都は観光産業に依存している。	
\\	複数形チェック
\\	[名詞]	散髪	
\\	トムは髪を切らなきゃいけない。	
\\	複数形チェック
\\	[動詞]	知らせる	
\\	何かあったらそのつど知らせてください。	
\\	[名詞]	ハム	
\\	彼はハムエッグが好きです。	
\\	複数形チェック
\\	[名詞]	インク、墨	
\\	彼女はインクで書いた。	
\\	複数形チェック
\\	[名詞]	ハンバーガー	
\\	ハンバーグ作りすぎちゃった。	
\\	複数形チェック
\\	[名詞]	昆虫、虫	
\\	英語でこの昆虫を何といいますか。	
\\	複数形チェック
\\	[名詞]	ハンマー	
\\	ハンマー借りてもよろしいでしょうか	
\\	複数形チェック
\\	[名詞]	ハンカチ	
\\	私はハンカチにアイロンをかけた。	
\\	複数形チェック
\\	[形容詞]	聡明な、理性的な	
\\	類人猿は知能が高い。	
\\	[動詞]	手を触れる、操作する	
\\	僕に任せてください。	
\\	[動詞]	意図する	
\\	彼らは攻撃するつもりだ。	
\\	[形容詞]	ハンサムな	
\\	トムはハンサムだ。	
\\	[形容詞]	国際的な	
\\	国際女性の日おめでとう!	
\\	[動詞] [不規則動詞]	吊るす	
\\	頑張れ!	
\\	[動詞]	紹介する	
\\	自己紹介をしてもいいですか。	
\\	[副詞]	幸福に	
\\	その物語はハッピーエンドだ。	
\\	[名詞]	招待	
\\	私たちは彼の招待に応じた。	
\\	複数形チェック
\\	[名詞] [動詞]	港、匿う	
\\	この国の湾に爆弾が落下し、独裁が世界を支配しようとしたとき、時の国民が立ち上がり、偉業を達成し、そして民主主義を救うのをクーパーさんは見ていました。
\\	。私たちにはできるのです。	
\\	複数形チェック
\\	[名詞]	損害、有害	
\\	君がひどい目にあうことはないよ。	
\\	複数形チェック
\\	[名詞]	上着、ジャケット	
\\	ジャケットが欲しかったんだ。	
\\	複数形チェック
\\	[動詞]	嫌う	
\\	化学が嫌いだ。	
\\	[名詞]	ジャズ	
\\	僕はジャズが好きだ。	
\\	複数形チェック
\\	[名詞]	頭痛	
\\	頭痛がする。	
\\	複数形チェック
\\	[名詞]	ジーンズ	
\\	彼女はいつもはジーンズをはいている。	
\\	複数形チェック
\\	[名詞]	ジェット機	
\\	ジェット機は東京に着陸した。	
\\	複数形チェック
\\	[名詞]	暑さ、熱さ	
\\	温めて。	
\\	複数形チェック
\\	[名詞]	旅	
\\	人生は旅のようなものだ。	
\\	複数形チェック
\\	[名詞]	高さ、身⻑	
\\	背の順に並んでください。	
\\	複数形チェック
\\	[名詞]	ジャングル	
\\	複数形チェック
\\	[名詞]	ヘリコプター	
\\	ヘリコプターが上空を旋回した。	
\\	複数形チェック
\\	[名詞]	キロ	
\\	パイナップルは1キロあたりいくらですか?	
\\	複数形チェック
\\	[名詞]	地獄	
\\	地獄へようこそ。	
\\	複数形チェック
\\	[名詞]	キログラム	
\\	私の体重は58キログラムです。	
\\	複数形チェック
\\	[形容詞]	役に立つ、助けになる	
\\	大変助かりました。	
\\	[名詞]	キロメートル	
\\	「ひかり」は時速200キロで走る。	
\\	複数形チェック
\\	[名詞]	英雄	
\\	トムは英雄だ。	
\\	複数形チェック
\\	[動詞]	殴る	
\\	やめなさい。	
\\	[感動詞]	へえ、おや	
\\	じゃ。	
\\	[名詞]	労働、労力	
\\	彼女たちの口から職場の労働問題が出てくることはほとんどなかった。	
\\	複数形チェック
\\	[動詞] [不規則動詞]	隠す、隠れる	
\\	何か隠してるの?	
\\	[名詞]	不足、欠乏	
\\	彼は酸素不足で死んだ。	
\\	複数形チェック
\\	[副詞]	非常に、高度に	
\\	あり得ねぇー。	
\\	[名詞]	ランプ	
\\	ねじを緩めて、ランプ・カバーを取り外します。	
\\	複数形チェック
\\	[名詞]	ハイキング	
\\	もし雨ならば、私たちはハイキングにいかないつもりです。	
\\	複数形チェック
\\	[名詞]	法律家、弁護士	
\\	あなたの弁護士は誰ですか。	
\\	複数形チェック
\\	[動詞]	雇う	
\\	彼女が彼を雇っていました。	
\\	[形容詞]	神聖な	
\\	神聖ローマ帝国は1806年に終わりを告げた。	
\\	[形容詞]	怠惰な	
\\	彼は尻が重い。	
\\	[名詞]	ホームルーム	
\\	彼らの担任の先生は誰ですか。	
\\	複数形チェック
\\	[名詞]	鉤、ホック	
\\	彼は上着を鉤にかけた。	
\\	複数形チェック
\\	[名詞]	指導者、統率者	
\\	トムは彼らのリーダーだ。	
\\	複数形チェック
\\	[名詞]	主人、主催者	
\\	たくさんの友人が見送りにやってきた。	
\\	複数形チェック
\\	[名詞]	革、革製品	
\\	それは皮でできています。	
\\	複数形チェック
\\	[名詞]	レモン	
\\	レモンは酸っぱい。	
\\	複数形チェック
\\	[動詞]	狩りをする	
\\	魔女狩りが始まってしまった。	
\\	[動詞] [不規則動詞]	貸す	
\\	金を貸す。	
\\	[名詞]	病気、疾患	
\\	彼は私の病気を治した。	
\\	複数形チェック
\\	[名詞]	像、イメージ	
\\	神は自らにかたどって人を創造された。	
\\	複数形チェック
\\	[名詞]	想像(力)	
\\	気のせいだよ。	
\\	複数形チェック
\\	[動詞]	持ち上げる	
\\	霧がはれた。	
\\	[名詞]	怪物	
\\	以前恐ろしい怪物が住んでいた。	
\\	複数形チェック
\\	[名詞]	制限、限度	
\\	物には限界というものがある。	
\\	複数形チェック
\\	[形容詞]	毎月の	
\\	私たちのクラブは次の水曜日に月例会を開きます。	
\\	[名詞]	積荷	
\\	乗せられるのは100キロがマックスです。	
\\	複数形チェック
\\	[名詞]	機嫌、(場所を漂う)空気	
\\	彼女の気分は次第にいら立ちはじめた。	
\\	複数形チェック
\\	[形容詞]	地方の	
\\	いくつかこの土地のワインもございます。	
\\	[名詞]	蚊	
\\	蚊が刺した。	
\\	複数形チェック
\\	[動詞]	鍵をする	
\\	ドアをロックして。	
\\	[名詞]	発動機、モーター	
\\	トムさん,モーターを回してください。	
\\	複数形チェック
\\	[名詞]	ロッカー	
\\	更衣室は男女に別れていましたが、お風呂場に入って行ったら 男女区別なく、混浴でした。	
\\	複数形チェック
\\	[形容詞]	寂しい、孤独な	
\\	淋しいです。	
\\	[名詞]	泥、ぬかるみ	
\\	彼は泥だらけだった。	
\\	複数形チェック
\\	[名詞]	殺人、殺人事件	
\\	彼は殺人を目撃した。	
\\	複数形チェック
\\	[形容詞]	大声で	
\\	大きすぎるわ。	
\\	[名詞]	ミュージカル	
\\	楽器をなにか演奏しますか。	
\\	複数形チェック
\\	[形容詞]	美しい、すばらしい	
\\	かわいい!	
\\	[名詞]	つき、運	
\\	元気でね。	
\\	複数形チェック
\\	[名詞]	謎、神秘	
\\	自然は神秘に満ちている。	
\\	複数形チェック
\\	[形容詞]	怒った、気が狂った	
\\	狂ったのかい?	
\\	[名詞]	手品、魔法	
\\	魔法だ。	
\\	複数形チェック
\\	[形容詞]	国⺠の	
\\	彼は国民的ヒーローになった。	
\\	[名詞]	製造業者	
\\	私は将来ビスクドール作家になりたい。	
\\	複数形チェック
\\	[副詞]	自然に、生まれつき	
\\	当然だよ。	
\\	[名詞]	男性	
\\	トムは男性の名前です。	
\\	複数形チェック
\\	[名詞]	ママ	
\\	ママ、トムがぼくのおもちゃ返してくれない!	
\\	複数形チェック
\\	[名詞]	海軍	
\\	父は海軍にいた。	
\\	複数形チェック
\\	[動詞]	管理する	
\\	何とかします。	
\\	[名詞]	管理職、管理者	
\\	妻はやりくりが上手だ。	
\\	複数形チェック
\\	[形容詞]	さっぱりした	
\\	ちょっと、これマジかわいい!	
\\	[名詞]	人類	
\\	笑いは人間の特質である。	
\\	複数形チェック
\\	[名詞]	ネックレス	
\\	すてきなネックレスしてるね。	
\\	複数形チェック
\\	[名詞]	針、注射	
\\	彼女は針仕事が上手だ。	
\\	複数形チェック
\\	[名詞]	甥	
\\	ジョンは私の甥です。	
\\	複数形チェック
\\	[名詞]	集まり、多数・量	
\\	雲は蒸気の固まりである。	
\\	複数形チェック
\\	[形容詞]	心配な、緊張した	
\\	トムは緊張した。	
\\	[名詞]	試合	
\\	彼はマッチをすった。	
\\	複数形チェック
\\	[名詞]	巣	
\\	子供はいつか巣立って行くんだよね。	
\\	複数形チェック
\\	[名詞]	仲間、相棒	
\\	複数形チェック
\\	[名詞]	網	
\\	彼は網を張った。	
\\	複数形チェック
\\	[名詞]	材料、資材	
\\	この生地は引っ張るとすぐ伸びる。	
\\	複数形チェック
\\	[名詞]	姪	
\\	メアリーは私の姪です。	
\\	複数形チェック
\\	[名詞]	数学	
\\	トムは数学に興味がある。	
\\	複数形チェック
\\	[名詞]	19番目、19日	
\\	19世紀は白人文明の時代だった。	
\\	複数形チェック
\\	[副詞]	もしかすると、多分	
\\	やっぱり、やめとく。	
\\	[名詞]	90番目	
\\	複数形チェック
\\	[名詞]	9番目、9日	
\\	スピーチコンテストは11月9日に行われた。	
\\	複数形チェック
\\	[名詞]	メダル	
\\	旅行で記念メダルを買いました。	
\\	複数形チェック
\\	[動詞]	うなずく	
\\	トムは頷いた。	
\\	[形容詞]	医学の	
\\	彼は治療を断られた。	
\\	[形容詞]	騒々しい、うるさい	
\\	トムはとても騒がしかった。	
\\	[名詞]	医学、薬	
\\	この薬は苦い。	
\\	複数形チェック
\\	[接続詞]	
\\	もまた...ない	
\\	私は酒もたばこもやらない。	
\\	[名詞]	メロン	
\\	このかぼちゃ、メロンの匂いがする。	
\\	複数形チェック
\\	[形容詞]	普通の、標準の	
\\	そんなものだよ。	
\\	[動詞]	溶ける、溶解する	
\\	氷が溶けている。	
\\	[動詞]	言及する	
\\	彼はそれについて話した。	
\\	[副詞]	どこにも	
\\	彼はどこからともなく現れた。	
\\	[名詞]	献立	
\\	英語のメニューはありますか?	
\\	複数形チェック
\\	[名詞]	木の実	
\\	そのボルトに嵌合するナットが必要です。	
\\	複数形チェック
\\	[形容詞]	陽気な	
\\	メリークリスマス!	
\\	[名詞]	メッセージ	
\\	伝言をお願いできますか。	
\\	複数形チェック
\\	[名詞]	金属、金属製品	
\\	彼らはメタルではない。	
\\	複数形チェック
\\	[名詞]	大洋、海	
\\	海は静かだった。	
\\	複数形チェック
\\	[形容詞]	軽度の、穏やかな	
\\	日本の気候は穏やかである。	
\\	[名詞]	ミリメートル	
\\	アメダスによればこの3日間で村の降水量は1200ミリを超えた。	
\\	複数形チェック
\\	[代名詞]	自分自身で	
\\	己を知ることは容易ではない。	
\\	複数形チェック
\\	[名詞]	100万	
\\	これは100万円の価値がある。	
\\	複数形チェック
\\	[名詞]	タマネギ	
\\	わぁ、たまねぎ味のベーグルだ。	
\\	複数形チェック
\\	[動詞]	混ぜる	
\\	彼女は彼を他の誰かとまちがえた。	
\\	[名詞]	オペラ	
\\	私はオペラが大好きだ。	
\\	複数形チェック
\\	[名詞]	モデル、型	
\\	模型飛行機を作ることが彼の唯一の楽しみです。	
\\	複数形チェック
\\	[助動詞]	〜すべきである	
\\	我々は勝つに決まっている。	
\\	[名詞]	皿、一皿	
\\	お皿が汚れています。	
\\	複数形チェック
\\	[名詞]	(駅の)ホーム	
\\	ロンドン行きの列車のプラットホームはどちらですか。	
\\	複数形チェック
\\	[名詞]	オーブン、かまど	
\\	あのパンはオーブンにある。	
\\	複数形チェック
\\	[名詞]	選手、競技者	
\\	3プレーヤーがほしい!	
\\	複数形チェック
\\	[動詞]	借りている	
\\	我々には何の恩義がありますか?	
\\	[名詞]	所有者	
\\	あなたが持ち主です。	
\\	複数形チェック
\\	[動詞]	荷造りする	
\\	荷物を準備してください。	
\\	[名詞]	たくさんのもの	
\\	これで十分なはずです。	
\\	複数形チェック
\\	[名詞]	小包	
\\	包みを開けて下さい。	
\\	複数形チェック
\\	[名詞]	詩、韻文	
\\	この詩を書いたのは誰ですか。	
\\	複数形チェック
\\	[名詞]	詩人	
\\	アレンは詩人です。	
\\	複数形チェック
\\	[形容詞]	痛い、苦しい	
\\	とても痛かった。	
\\	[名詞]	毒	
\\	あれは毒だ。	
\\	複数形チェック
\\	[名詞]	画家	
\\	彼は画家だ。	
\\	複数形チェック
\\	[名詞]	棒、さお	
\\	技師が電柱を上った。	
\\	複数形チェック
\\	[名詞]	絵、絵画	
\\	誰がこの絵を描いたのですか。	
\\	複数形チェック
\\	[名詞]	政策、方針	
\\	正直は必ずしも最上の策とは限らない。	
\\	複数形チェック
\\	[名詞]	パジャマ	
\\	複数形チェック
\\	[動詞]	みがく	
\\	この靴を磨く必要がある。	
\\	[名詞]	宮殿	
\\	宮殿は物々しい警戒ぶりだった。	
\\	複数形チェック
\\	[形容詞]	礼儀正しい	
\\	行儀良くしてください。	
\\	[名詞]	平鍋	
\\	私はフライパンで野菜を炒めた。	
\\	複数形チェック
\\	[名詞]	池	
\\	池が凍ってるよ	
\\	複数形チェック
\\	[名詞]	パンダ	
\\	きみはパンダ。	
\\	複数形チェック
\\	[名詞]	港	
\\	彼は港町に住んでいる。	
\\	複数形チェック
\\	[名詞]	パパ	
\\	パパはラテン語の素養もあったのです。	
\\	複数形チェック
\\	[名詞]	パレード	
\\	子どもたちがサーカスの行列を追いかけた。	
\\	複数形チェック
\\	[形容詞]	肯定的な、賛成の	
\\	これらの形容詞は全部が積極的です。	
\\	[動詞]	許す	
\\	もう一度繰り返していただけませんか。	
\\	[名詞]	可能性	
\\	それは可能性の一つです。	
\\	複数形チェック
\\	[副詞]	部分的に	
\\	ハゲタカが突っつく死んだ鹿とか、他の動物の食べ残しとか、 そういう腐った肉を死肉と書きました。	
\\	[動詞]	そそぐ	
\\	ざあざあ雨が降った。	
\\	[名詞]	粉、粉末	
\\	このケーキを作るためには膨らし粉と無塩バターが必要だ。	
\\	複数形チェック
\\	[名詞]	乗客	
\\	このバスは30人乗りです。	
\\	複数形チェック
\\	[形容詞]	強い、強力な	
\\	彼は強い。	
\\	[名詞]	パスポート	
\\	パスポートを盗まれました。	
\\	複数形チェック
\\	[形容詞]	実用的な	
\\	パソコンを持つことは実用的です。	
\\	[動詞]	ほめる	
\\	彼女はほめる言葉に困る程だ。	
\\	[名詞]	患者	
\\	辛抱しなさい。	
\\	複数形チェック
\\	[動詞]	祈る、懇願する	
\\	どうか許して下さい。	
\\	[名詞]	模様、様式	
\\	パターンがあるはずだ。	
\\	複数形チェック
\\	[名詞]	貴重な、高価な	
\\	この地方は貴重な鉱物を産出します。	
\\	複数形チェック
\\	[名詞]	休止	
\\	話がちょっと途切れた。	
\\	複数形チェック
\\	[動詞]	好む	
\\	コーヒーの方が好きです。	
\\	[名詞]	エンドウ、エンドウ豆	
\\	グリーンピースは鉄分が豊富で爪の色がよくなる成分が含まれている。	
\\	複数形チェック
\\	[動詞]	用意する、準備する	
\\	用意がほとんど出来ていなかった。	
\\	[名詞]	社⻑、大統領	
\\	大統領は奴隷制度を廃止した。	
\\	複数形チェック
\\	[形容詞]	平和な	
\\	私は、彼らが平和的和解に達する事を望む。	
\\	[動詞]	押す	
\\	我々はしゃにむに突き進んだ。	
\\	[名詞]	モモ	
\\	この桃は芸術品だよ。	
\\	複数形チェック
\\	[名詞]	山頂、頂点	
\\	頂上が雪で覆われたその山がみえますか。	
\\	複数形チェック
\\	[名詞]	ピーナッツ	
\\	何人かの子供達はピーナッツバターサンドイッチを持ってきた。何人かはハムで、他はチーズだった。	
\\	複数形チェック
\\	[名詞]	聖職者	
\\	トムは司祭だ。	
\\	複数形チェック
\\	[名詞]	真珠	
\\	妹の誕生日に真珠のネックレスを贈った。	
\\	複数形チェック
\\	[名詞]	王子	
\\	私が王子様だったらいいのに。	
\\	複数形チェック
\\	[名詞]	コショウ	
\\	コショウで彼の口がひりひりした。	
\\	複数形チェック
\\	[名詞]	王女	
\\	姫君は皇帝に寛恕を請いました。	
\\	複数形チェック
\\	[動詞]	印刷する	
\\	印刷術はいつ発明されましたか。	
\\	[動詞]	行う、演じる	
\\	トニーは演じている。	
\\	[名詞]	刑務所、拘置所	
\\	彼は脱獄した。	
\\	複数形チェック
\\	[名詞]	囚人、捕虜	
\\	彼らは捕虜になった。	
\\	複数形チェック
\\	[動詞]	許可する	
\\	滞在を許可して下さい。	
\\	[名詞]	賞、賞品	
\\	彼はその賞に値する。	
\\	複数形チェック
\\	[名詞]	カメラマン	
\\	私はカメラマンです。	
\\	複数形チェック
\\	[名詞]	パイ	
\\	トムはパイを少しもらった。	
\\	複数形チェック
\\	[名詞]	問題	
\\	問題ないよ。	
\\	複数形チェック
\\	[名詞]	枕、まくら状の物	
\\	枕を下さい。	
\\	複数形チェック
\\	[名詞]	過程	
\\	過程が大切です。	
\\	複数形チェック
\\	[動詞]	ピンで留める	
\\	暗証番号を忘れちゃった!	
\\	[名詞]	パイナップル	
\\	パイナップルシャーベットを食べます。	
\\	複数形チェック
\\	[名詞]	製品	
\\	大麦は我国の主産物である。	
\\	複数形チェック
\\	[名詞]	ピザ	
\\	ピザをお願いします。	
\\	複数形チェック
\\	[形容詞]	職業の、専門の	
\\	トムはプロのダンサーだ。	
\\	[名詞]	教授	
\\	私は教授です。	
\\	複数形チェック
\\	[名詞]	惑星	
\\	私はこの惑星に住んでいる。	
\\	複数形チェック
\\	[名詞]	利益	
\\	名を捨てて実を取る。	
\\	複数形チェック
\\	[名詞]	プラスチック	
\\	クレジットカードでも良いですか?	
\\	複数形チェック
\\	[名詞]	プログラム	
\\	この番組は再放送だ。	
\\	複数形チェック
\\	[名詞]	証拠、証拠品	
\\	証拠がありません。	
\\	複数形チェック
\\	[形容詞]	責任感のある、責任がある	
\\	少し責任感を感じている。	
\\	[動詞]	提案する、申し出る	
\\	彼はあなたに結婚を申し込んだのですか。	
\\	[動詞]	保護する、守る	
\\	その刑事は目撃者を守ってくれた。	
\\	[名詞]	リボン、飾りひも	
\\	複数形チェック
\\	[名詞]	保護、防護	
\\	彼女を保護する義務がある。	
\\	複数形チェック
\\	[名詞]	任務、役割	
\\	誰がお姫様の役を演じるの。	
\\	複数形チェック
\\	[動詞]	転がる	
\\	サイを振れ。	
\\	[名詞]	根、根源	
\\	牛蒡の天ぷらを食べました。	
\\	複数形チェック
\\	[形容詞]	荒い、粗野な	
\\	この紙はざらざらしている。	
\\	[動詞]	出版する	
\\	その本は1689年に出版された。	
\\	[名詞]	道順、ルート	
\\	路線の変更はできますか。	
\\	複数形チェック
\\	[動詞]	罰する	
\\	私はトムにお仕置きをした。	
\\	[名詞]	並び、列	
\\	一列に並んで下さい。	
\\	複数形チェック
\\	[名詞]	生徒、児童	
\\	例えば、たばこをぷかぷか吸う先生が、たばこを吸う生徒に注意してもその注意は矛盾しているだろう。	
\\	複数形チェック
\\	[形容詞]	王室の、王の	
\\	王宮は丘の上に建てられた。	
\\	[名詞]	定規、支配者	
\\	シュメール語で
\\	とは支配者(王)のことだ	
\\	複数形チェック
\\	[名詞]	紫色、深紅色	
\\	紫芋を食べたいよ。	
\\	複数形チェック
\\	[名詞]	噂	
\\	その噂、誰から聞いたの?	
\\	複数形チェック
\\	[名詞]	走者	
\\	彼女は走者です。	
\\	複数形チェック
\\	[名詞]	財布	
\\	私は財布を置き忘れた。	
\\	複数形チェック
\\	[動詞]	急いで行動する	
\\	急かさないで。	
\\	[動詞]	悩ます、困らせる	
\\	それは私にややこしいことだった。	
\\	[名詞]	安全	
\\	安全なんて幻想だ。	
\\	複数形チェック
\\	[副詞]	静かに	
\\	そーっと動かしてね。	
\\	[名詞]	帆、帆船	
\\	海岸線に沿って航海せよ。	
\\	複数形チェック
\\	[名詞]	航海	
\\	ボクに任せたんだ。君たちは大船に乗ったつもりでいるといい。	
\\	複数形チェック
\\	[名詞]	クイズ	
\\	複数形チェック
\\	[名詞]	船員	
\\	彼は船乗りになった。	
\\	複数形チェック
\\	[名詞]	ラケット	
\\	トムは自分のラケットを置きました。	
\\	複数形チェック
\\	[名詞]	給料	
\\	彼は高い給料をもらっている。	
\\	複数形チェック
\\	[名詞]	鉄道	
\\	その町には列車で行ける。	
\\	複数形チェック
\\	[名詞]	ソース	
\\	このソース試してみて。	
\\	複数形チェック
\\	[名詞]	鉄道、鉄道会社	
\\	駅はどこですか。	
\\	複数形チェック
\\	[名詞]	鉄道	
\\	駅はどこですか。	
\\	複数形チェック
\\	[名詞]	男子生徒	
\\	男子学生たちはひっきりなしにふざけあっていた。	
\\	複数形チェック
\\	[名詞]	虹	
\\	雨が止んだとたんにきれいな虹が出た。	
\\	複数形チェック
\\	[名詞]	女子生徒	
\\	彼女は学校時代のお母さんの写真を私にみせてくれた。	
\\	複数形チェック
\\	[動詞]	持ち上げる	
\\	もっと大きな声で話しなさい。	
\\	[形容詞]	科学の、科学的な	
\\	彼は科学の仕事に熱中している。	
\\	[名詞]	階級、地位	
\\	類人猿は知的には犬より上位である。	
\\	複数形チェック
\\	[名詞]	科学者、自然科学者	
\\	彼は科学者です。	
\\	複数形チェック
\\	[形容詞]	速い、急な	
\\	我々は急速な人口増加を防がなければならない。	
\\	[名詞]	得点、点数	
\\	彼はオペラを書いた。	
\\	複数形チェック
\\	[名詞]	(ドブ)ネズミ	
\\	そのネズミは生きているのか死んでいるのか?	
\\	複数形チェック
\\	[動詞]	叫ぶ	
\\	ケンタッキに「ファック」って悲鳴をあげる。	
\\	[名詞]	秘書、⻑官	
\\	秘書は誰ですか?	
\\	複数形チェック
\\	[名詞]	売り手、販売人	
\\	米屋や、八百屋に借金が出来て行った。	
\\	複数形チェック
\\	[形容詞]	生の	
\\	この食べ物は生だ。	
\\	[名詞]	読者	
\\	メアリーは読書家だ。	
\\	複数形チェック
\\	[名詞]	文	
\\	この文は削除してください。	
\\	複数形チェック
\\	[動詞]	気づく	
\\	ついに彼は自分の誤りに気付いた。	
\\	[形容詞]	別々の、異なる	
\\	サミとライラは別々の部屋で寝た。	
\\	[形容詞]	合理的な	
\\	これには納得がいきます。	
\\	[名詞]	使用人、召使	
\\	用心深い鳥は木を選ぶ。思慮深い従者は主を選ぶ。	
\\	複数形チェック
\\	[副詞]	最近は、ついこの間	
\\	この前、行きました。	
\\	[動詞]	仕える、奉仕する	
\\	それは当然の報いだ。	
\\	[動詞]	記録する	
\\	彼は記録を破った。	
\\	[名詞] [動詞]	奉仕、サービス(する)、修理する	
\\	サービスが悪かったのでクレームをつけた。	
\\	複数形チェック
\\	[動詞]	回復する、取り戻す	
\\	トムは回復した。	
\\	[名詞]	17番目、17日	
\\	この年の正月は十四日から十七日まで四日間の雪を見た。	
\\	複数形チェック
\\	[名詞]	7番目、7日	
\\	私は最高に幸せだ。	
\\	複数形チェック
\\	[形容詞]	規則正しい	
\\	トムは献血を定期的にしている。	
\\	[名詞]	70番目	
\\	父は今年喜寿を迎えたが、わたしなんかよりよほどエネルギッシュだ。	
\\	複数形チェック
\\	[名詞]	性、性別	
\\	女性だってエッチしたい。	
\\	複数形チェック
\\	[動詞]	取り除く	
\\	帽子を脱ぎなさい。	
\\	[名詞]	陰、日陰	
\\	太陽に手をかざした。	
\\	複数形チェック
\\	[動詞]	修理する	
\\	「道路工事中」	
\\	[名詞]	影	
\\	彼女はアイシャドーをつけている。	
\\	複数形チェック
\\	[動詞]	繰り返す	
\\	歴史は繰り返す。	
\\	[名詞]	シャンプー	
\\	私は朝シャンをしないからよ。	
\\	複数形チェック
\\	[動詞]	報告する	
\\	彼は報告書を改ざんした。	
\\	[名詞]	シーツ	
\\	私に紙を一枚下さい。	
\\	複数形チェック
\\	[名詞]	報道記者	
\\	新聞記者の方ですね。	
\\	複数形チェック
\\	[名詞]	衝撃、ショック	
\\	彼女はショック死した。	
\\	複数形チェック
\\	[動詞] [不規則動詞]	放つ、射る	
\\	撃て!	
\\	[動詞]	残しておく	
\\	このテーブルは予約済みです。	
\\	[名詞]	発射、発砲	
\\	その人たちが私を撃ったんです。	
\\	複数形チェック
\\	[名詞]	棒、つえ	
\\	舌を出してください。	
\\	複数形チェック
\\	[動詞]	叫ぶ、大声を出す	
\\	トムは大声で叫んだ。	
\\	[名詞]	胃、腹部、腹	
\\	胃が痛みます。	
\\	複数形チェック
\\	[名詞]	エビ、小エビ	
\\	複数形チェック
\\	[名詞]	嵐、豪雨	
\\	今にも嵐になりそうだ。	
\\	複数形チェック
\\	[名詞]	見知らぬ人	
\\	私は当地は不案内です。	
\\	複数形チェック
\\	[名詞]	信号	
\\	ちょっと屈めば、信号旗がチラチラと見え隠れします。	
\\	複数形チェック
\\	[名詞]	⻨わら、わらでできた	
\\	もう我慢の限界だ。	
\\	複数形チェック
\\	[名詞]	イチゴ	
\\	トムはパンに苺ジャムを塗った。	
\\	複数形チェック
\\	[名詞]	絹、絹布、絹織物	
\\	絹は手触りが柔らかくすべすべしている。	
\\	複数形チェック
\\	[形容詞]	愚かな、馬鹿げた	
\\	馬鹿みたい!	
\\	[動詞]	引き伸ばす	
\\	この生地は引っ張るとすぐ伸びる。	
\\	[動詞]	打つ	
\\	彼はマッチをすった。	
\\	[動詞] [不規則動詞]	沈む、沈没する	
\\	八方塞がりだ。	
\\	[動詞]	もがく、努力する	
\\	彼らは独裁者と戦った。	
\\	[名詞]	お客さん、閣下	
\\	お客様、もう少しお静かに。	
\\	複数形チェック
\\	[名詞]	もの、材料	
\\	あれは毒だ。	
\\	複数形チェック
\\	[名詞]	16番目、16日	
\\	選挙は5月16日に行われた。	
\\	複数形チェック
\\	[形容詞]	ばかな	
\\	バカみたい。	
\\	[名詞]	6番目、6日	
\\	彼は今回の事故で6回目だ。	
\\	複数形チェック
\\	[名詞]	流儀、様式	
\\	それぞれのスタイルがあるんだろうね。	
\\	複数形チェック
\\	[名詞]	60番目	
\\	複数形チェック
\\	[名詞]	スケート	
\\	彼女はスケートが出来る。	
\\	複数形チェック
\\	[動詞]	成功する	
\\	成功だ!	
\\	[名詞]	スケート	
\\	彼はスケートをしている。	
\\	複数形チェック
\\	[名詞]	スカート、すそ	
\\	どのスカートがお好みですか。	
\\	複数形チェック
\\	[形容詞]	成功した	
\\	実験は成功だった。	
\\	[動詞] [不規則動詞]	滑る、滑走する	
\\	アリスは長い滑り台を滑り降りた。	
\\	[動詞]	滑る、ずり落ちる	
\\	さっさと靴をはきなさい。	
\\	[動詞]	苦しむ	
\\	情熱は苦悩を生む。	
\\	[形容詞]	滑らかな	
\\	この布は滑らかな手触りです。	
\\	[動詞]	提案する、勧める	
\\	何がおいしいですか。	
\\	[名詞]	スーツケース	
\\	トムはスーツケースを開けた。	
\\	複数形チェック
\\	[名詞]	日曜日	
\\	今日は日曜日だ。	
\\	複数形チェック
\\	[名詞]	靴下	
\\	靴下に穴が開いているよ。	
\\	複数形チェック
\\	[名詞]	日光	
\\	日の光がだんだん衰えてきた。	
\\	複数形チェック
\\	[名詞]	ソファ	
\\	このソファーはいくらですか。	
\\	複数形チェック
\\	[名詞]	夕暮れ、夕日	
\\	なんてきれいな夕焼けだろう。	
\\	複数形チェック
\\	[名詞]	兵士	
\\	その兵士は自分の名前を言った。	
\\	複数形チェック
\\	[名詞]	日光	
\\	この部屋は日があたる。	
\\	複数形チェック
\\	[形容詞]	固体の	
\\	水の固まったものを氷と呼ぶ。	
\\	[形容詞]	極上の、すばらしい	
\\	超腹減った。	
\\	[動詞]	解決する、解く	
\\	誰も問題を解いたことがない。	
\\	[名詞]	夕食	
\\	私は夕食を作った。	
\\	複数形チェック
\\	[副詞]	どこかに	
\\	旅券をどこかに置き忘れました。	
\\	[副詞]	確かに、本当に	
\\	確かに彼は利口な少年だ。	
\\	[名詞]	スパゲティ	
\\	トムはスパゲッティが大好きです。	
\\	複数形チェック
\\	[形容詞]	驚いた	
\\	驚いた。	
\\	[動詞]	惜しんで使わない	
\\	1ドル貸してくれないか。	
\\	[動詞]	飲み込む	
\\	蛇が蛙を飲み込んだ。	
\\	[名詞]	白鳥	
\\	複数形チェック
\\	[名詞] [動詞]	汗、発汗、汗を掻く	
\\	彼は汗びっしょりだった。	
\\	複数形チェック
\\	[動詞] [不規則動詞]	綴る	
\\	「マサチューセッツ」ってどう綴るんですか?	
\\	[名詞]	セーター	
\\	このセーターを着てごらんなさい。	
\\	複数形チェック
\\	[名詞]	つづること、綴字法	
\\	つづりのテストが毎週月曜日にあります。	
\\	複数形チェック
\\	[動詞] [不規則動詞]	左右に揺れる	
\\	クリスはソードを振るった!	
\\	[名詞]	クモ	
\\	私は飢えた蜘蛛です。	
\\	複数形チェック
\\	[名詞]	象徴、シンボル	
\\	ハトは平和の表象である。	
\\	複数形チェック
\\	[名詞]	精神、霊魂、気分	
\\	彼はがっかりした。	
\\	複数形チェック
\\	[名詞]	尻尾	
\\	この種の猫には尾が無い。	
\\	複数形チェック
\\	[名詞]	涙	
\\	私たちはこの家を壊さないといけない。	
\\	複数形チェック
\\	[名詞]	舞台、ステージ	
\\	彼は舞台に立った。	
\\	複数形チェック
\\	[名詞]	ティーンエージャー	
\\	彼女は十代で結婚した。	
\\	複数形チェック
\\	[名詞]	階段	
\\	階段がある。	
\\	複数形チェック
\\	[形容詞]	10代の	
\\	[名詞]	基準、標準	
\\	それは定石通りというだけで、手を抜いたわけではないのです。	
\\	複数形チェック
\\	[名詞]	10代の少年	
\\	彼女は10代のころ拒食症になった。	
\\	複数形チェック
\\	[名詞]	寺	
\\	昔はここに寺がありました。	
\\	複数形チェック
\\	[名詞]	ステーキ	
\\	ステーキ食べたい。	
\\	複数形チェック
\\	[名詞]	10日、10番目	
\\	私は高校一年生です。	
\\	複数形チェック
\\	[動詞] [不規則動詞]	盗む	
\\	何を盗まれましたか。	
\\	[名詞]	期間、学期	
\\	「放射能」というのは化学の用語だ。	
\\	複数形チェック
\\	[名詞]	蒸気、スチーム	
\\	それだとストレスを発散できますね。	
\\	複数形チェック
\\	[名詞]	鋼鉄、鋼鉄産業	
\\	彼は鋼のような意志を持っている。	
\\	複数形チェック
\\	[名詞]	文言、本文	
\\	本文がありません!	
\\	複数形チェック
\\	[名詞]	ステレオ	
\\	複数形チェック
\\	[名詞]	劇場、映画館	
\\	「ロミオとジュリエット」が劇場で上演されている。	
\\	複数形チェック
\\	[名詞]	谷、谷間	
\\	川は谷をながれている。	
\\	複数形チェック
\\	[名詞]	泥棒	
\\	彼はその泥棒を追跡した。	
\\	複数形チェック
\\	[形容詞]	高価な、貴重な	
\\	彼女は多くの高価な本を持っている。	
\\	[形容詞]	のどが渇いた	
\\	喉が渇いた。	
\\	[名詞]	13番目、13日	
\\	テニスは13世紀にフランスで始まった。	
\\	複数形チェック
\\	[形容詞]	暴力的な	
\\	嵐はもっと激しくなってきた。	
\\	[名詞]	30番目、30日	
\\	先週誕生日を迎え、30歳になりました。	
\\	複数形チェック
\\	[名詞]	スミレ	
\\	彼女の菫色の瞳が忘れられない。	
\\	複数形チェック
\\	[名詞]	語彙	
\\	語彙力を増強したい。	
\\	複数形チェック
\\	[名詞]	喉	
\\	のどがヒリヒリします。	
\\	複数形チェック
\\	[名詞]	バレーボール	
\\	バレーボールをしよう。	
\\	複数形チェック
\\	[名詞]	親指	
\\	彼女は彼を牛耳っている。	
\\	複数形チェック
\\	[動詞]	投票する	
\\	我々は投票で決めた。	
\\	[形容詞]	きつい、堅い	
\\	ぎゅっと抱きしめて。	
\\	[名詞]	ウェイター	
\\	ジョンは、大きな声でウエイターを呼んだ。	
\\	複数形チェック
\\	[形容詞]	小さい	
\\	小さい!	
\\	[名詞]	ウェイトレス	
\\	ウエートレスはテーブルの上に白いテーブルクロスをかけた。	
\\	複数形チェック
\\	[名詞]	先端	
\\	いいアドバイスを ありがとう。	
\\	複数形チェック
\\	[名詞]	歩く人	
\\	彼は歩くのが速い。	
\\	複数形チェック
\\	[動詞]	疲れさせる	
\\	疲れているんだ。	
\\	[動詞]	警告する、注意する	
\\	トムに警告します。	
\\	[名詞]	表題、タイトル	
\\	英語の題名は
\\	と言います。	
\\	複数形チェック
\\	[名詞]	警告、注意	
\\	十八禁なので要注意!	
\\	複数形チェック
\\	[名詞]	タバコ	
\\	トムは噛みタバコを使用する。	
\\	複数形チェック
\\	[動詞]	浪費する	
\\	急がば回れ。	
\\	[名詞]	トン	
\\	しなくてはならない宿題がたくさんある。	
\\	複数形チェック
\\	[名詞]	富、豊富	
\\	彼は富を蓄えた。	
\\	複数形チェック
\\	[名詞]	工具、道具	
\\	私の工具箱を持ってきて。	
\\	複数形チェック
\\	[名詞]	天気	
\\	何と酷い天気だ。	
\\	複数形チェック
\\	[名詞]	⻭ブラシ	
\\	それはたぶんあなたの歯ブラシです。	
\\	複数形チェック
\\	[名詞]	結婚式	
\\	結婚記念日おめでとう。	
\\	複数形チェック
\\	[名詞]	練り⻭みがき	
\\	歯磨きはどこ?	
\\	複数形チェック
\\	[形容詞]	週刊の	
\\	彼は座って週刊誌を読んでいた。	
\\	[名詞]	話題、論題	
\\	トピずれです。すみません。	
\\	複数形チェック
\\	[動詞]	重さを量る	
\\	彼女は体重が120ポンドある。	
\\	[形容詞]	合計の	
\\	部屋は散らかり放題だった。	
\\	[名詞]	重量、体重	
\\	彼の意見には重みがある。	
\\	複数形チェック
\\	[名詞]	周遊、見学	
\\	来週土曜日にバスツアーをしない?	
\\	複数形チェック
\\	[形容詞]	⻄の、⻄洋の	
\\	西部戦線異状なし	
\\	[名詞]	観光客	
\\	私も観光客です。	
\\	複数形チェック
\\	[名詞]	車輪	
\\	車輪が回りだした。	
\\	複数形チェック
\\	[名詞]	跡、小道	
\\	時間がたつのを忘れてしまった。	
\\	複数形チェック
\\	[名詞]	ウィスキー	
\\	ウィスキーを飲みますか。	
\\	複数形チェック
\\	[名詞]	交通	
\\	連日交通事故が起きる。	
\\	複数形チェック
\\	[名詞]	トレーニング、研修	
\\	練習がどんなに厳しくても、彼女は決して泣かなかった。	
\\	複数形チェック
\\	[形容詞]	快くする	
\\	お手伝いしますよ。	
\\	[名詞]	策略、ぺてん	
\\	これはトリックじゃない。	
\\	複数形チェック
\\	[動詞] [不規則動詞]	勝つ	
\\	彼らは勝った。	
\\	[名詞]	トラック	
\\	トムはトラックの運転手をしている。	
\\	複数形チェック
\\	[名詞]	翼、羽	
\\	その鳥の翼は折れていた。	
\\	複数形チェック
\\	[名詞]	トランペット	
\\	トランペットを吹いてみるか。	
\\	複数形チェック
\\	[動詞]	拭く	
\\	涙を拭いて。	
\\	[名詞]	針金、電線	
\\	このかごは、針金でできている。	
\\	複数形チェック
\\	[名詞]	管	
\\	彼は歯磨きをチューブから絞り出した。	
\\	複数形チェック
\\	[名詞]	チューリップ	
\\	チューリップは今が満開です。	
\\	複数形チェック
\\	[名詞]	オオカミ	
\\	狼を飼い慣らすことはできません。	
\\	複数形チェック
\\	[名詞]	おなか	
\\	お腹痛い。	
\\	複数形チェック
\\	[名詞]	七面鳥、七面鳥の肉	
\\	この七面鳥おいしいね。	
\\	複数形チェック
\\	[形容詞]	木造の	
\\	私は木のくしを持っています。	
\\	[名詞]	12番目、12日	
\\	複数形チェック
\\	[名詞]	羊毛	
\\	羊毛は染め上がりがきれいです。	
\\	複数形チェック
\\	[名詞]	20番目、20日	
\\	複数形チェック
\\	[名詞]	労働者	
\\	トムはよく働く。	
\\	複数形チェック
\\	[副詞]	2倍、2度	
\\	二度確認しました。	
\\	[形容詞]	心配している、困った	
\\	私達は心配だった。	
\\	[名詞]	双子	
\\	彼女は双子です。	
\\	複数形チェック
\\	[形容詞]	できない	
\\	彼は泳げないらしい。	
\\	[名詞]	理解	
\\	あなたの中国語のレベルには心底感嘆してしまうわ。	
\\	複数形チェック
\\	[名詞]	作家、作者	
\\	彼は作家です。	
\\	複数形チェック
\\	[名詞]	下着	
\\	トムは下着姿だった。	
\\	複数形チェック
\\	[形容詞]	不幸な	
\\	彼女は嬉しくないようだ。	
\\	[副詞]	はい	
\\	そうか?	
\\	[形容詞]	一定の、一様な	
\\	私は彼女に駅で会ったが、制服を着ていたので彼女だとわからなかった。	
\\	[名詞]	大学	
\\	私の大学には寮があります。	
\\	複数形チェック
\\	[代名詞]	あなたたち自身	
\\	君らで見なさい。	
\\	複数形チェック
\\	[形容詞]	上の	
\\	彼は上流階級の人だ。	
\\	[副詞]	二階へ	
\\	私は二階に駆け上がった。	
\\	[名詞]	ゼロ	
\\	一の前にはゼロです。	
\\	複数形チェック
\\	[名詞]	不在、不参加	
\\	留守中いい子でいるんですよ。	
\\	複数形チェック
\\	[名詞]	地図帳、図解	
\\	私はこの地図に2、000円払った。	
\\	複数形チェック
\\	[形容詞]	絶対の、完全な	
\\	彼は絶対的な権力を持っている。	
\\	[名詞]	原子	
\\	水分子は、2個の水素原子と1個の酸素原子からなる。	
\\	複数形チェック
\\	[形容詞]	大学の、人文学科の	
\\	彼女はいつも自分の学歴を自慢している。	
\\	[形容詞]	原子の	
\\	ここに原子力発電所があります。	
\\	[動詞]	取り付ける	
\\	かばん全部に荷札をつけなさい。	
\\	[動詞]	同行する	
\\	私はあなたと同行しましょう。	
\\	[形容詞]	〜によれば、従った	
\\	動物は本能に従って行動する。	
\\	[名詞]	姿勢、態度	
\\	彼に対する態度が変わった。	
\\	複数形チェック
\\	[動詞]	達成する、成し遂げる	
\\	彼は目的を達成した。	
\\	[動詞]	魅惑する	
\\	がは灯に集まる。	
\\	[名詞]	付加、足し算	
\\	塩を加えたら味が大いに良くなった。	
\\	複数形チェック
\\	[形容詞]	魅力的な	
\\	彼女は魅力的だ。	
\\	[名詞]	音声	
\\	フランス語のオーディオガイドはありますか?	
\\	複数形チェック
\\	[動詞]	認める	
\\	それは認める。	
\\	[動詞]	採択する、身につける	
\\	私達は子供を養子にした。	
\\	[名詞]	進歩、進出	
\\	よろしくお願いいたします。	
\\	複数形チェック
\\	[形容詞]	自動の	
\\	これは自動ドアです。	
\\	[形容詞]	使用できる	
\\	空席はありますか。	
\\	[名詞]	相談相手	
\\	社長は会議のために顧問たちを集めた。	
\\	複数形チェック
\\	[名詞]	大通り	
\\	彼らは戦争を回避しようとあらゆる手段を探し求めた。	
\\	複数形チェック
\\	[名詞]	出来事,事件	
\\	浮気されたことある?	
\\	複数形チェック
\\	[名詞]	平均	
\\	彼は打率が高い。	
\\	複数形チェック
\\	[動詞]	影響する	
\\	聴衆は深く感動した。	
\\	[動詞]	避ける	
\\	彼女は私を避けている。	
\\	[動詞]	余裕がある	
\\	もうこれ以上ぐずぐずしていられない。	
\\	[名詞]	子守	
\\	新しいベビーシッターを探さなきゃね。	
\\	複数形チェック
\\	[名詞]	背景、経歴	
\\	一番下の背景レイヤーを消去します。	
\\	複数形チェック
\\	[名詞]	裏庭	
\\	トムの書斎は裏庭に面している。	
\\	複数形チェック
\\	[名詞]	航空機	
\\	この航空会社は貨物輸送だけを取り扱っています。	
\\	複数形チェック
\\	[名詞]	荷物	
\\	荷物を預かって欲しいのですが。	
\\	複数形チェック
\\	[名詞]	航空郵便	
\\	彼女は手紙を航空便で送った。	
\\	複数形チェック
\\	[名詞]	パン屋	
\\	街角のパン屋にアップルタルトはありますか?	
\\	複数形チェック
\\	[名詞]	注意、警報	
\\	火災報知器が鳴った。	
\\	複数形チェック
\\	[形容詞]	パン焼き用の	
\\	今、パイを焼いてるとこ。	
\\	[名詞]	アルコール	
\\	お酒は飲みますか。	
\\	複数形チェック
\\	[名詞]	バルコニー	
\\	ジュリエットはバルコニーから外のロミオに語りかけました。	
\\	複数形チェック
\\	[形容詞]	似ている、同様な	
\\	その双子の兄弟はそっくりだ。	
\\	[名詞]	銀行家	
\\	銀行員の給与カットは一時的なもので、恒久的なものではなかった。	
\\	複数形チェック
\\	[副詞]	大声で	
\\	独り言を言うなよ。	
\\	[名詞]	理髪師	
\\	床屋に行きなさい。	
\\	複数形チェック
\\	[形容詞]	裸の	
\\	多くの木は冬になると葉がなくなる。	
\\	[名詞]	契約、取引	
\\	俺たちは取引をした。	
\\	複数形チェック
\\	[名詞]	救急車	
\\	救急車を呼んでくれ。	
\\	複数形チェック
\\	[動詞]	ほえる	
\\	犬が吠える。	
\\	[名詞]	分析	
\\	彼はその問題を徹底的に分析した。	
\\	複数形チェック
\\	[名詞]	地下室	
\\	地下室のどこかに置いたか知っています。	
\\	複数形チェック
\\	[名詞]	角度	
\\	直角は90度である。	
\\	複数形チェック
\\	[名詞]	根拠、理由	
\\	私たちは出来高制で働いている。	
\\	複数形チェック
\\	[名詞]	足首	
\\	彼は足首をねんざした。	
\\	複数形チェック
\\	[動詞]	入浴させる	
\\	その部屋には日光がさんさんと注いでいた。	
\\	[名詞]	発表	
\\	首相は明日声明を出すことになっている。	
\\	複数形チェック
\\	[名詞]	浴槽	
\\	トムは湯船に浸かりました。	
\\	複数形チェック
\\	[名詞]	アナウンサー	
\\	トムはテレビのアナウンサーをしている。	
\\	複数形チェック
\\	[名詞]	電池	
\\	バッテリー残量が少なくなっています。	
\\	複数形チェック
\\	[副詞]	とにかく	
\\	ともかく会いましょう。	
\\	[名詞]	獣	
\\	人にとって人以上に恐怖となる獣は世界に存在しない。	
\\	複数形チェック
\\	[副詞]	もはや〜ない	
\\	もう電話するな。	
\\	[名詞]	寝る時間	
\\	寝る時間よ。	
\\	複数形チェック
\\	[動詞]	懇願する、要請する	
\\	被告は上告を認められた。	
\\	[名詞]	牛肉の切り身、ビフテキ	
\\	複数形チェック
\\	[名詞]	応用	
\\	彼の申し込みがやっと受理された。	
\\	複数形チェック
\\	[動詞]	請い求める	
\\	失礼ですが、もう一度おっしゃって下さい。	
\\	[動詞]	任命する	
\\	彼は議長に任命された。	
\\	[名詞]	初心者	
\\	彼女は初心者だ。	
\\	複数形チェック
\\	[名詞]	予約	
\\	予約はしてありますか。	
\\	複数形チェック
\\	[動詞]	認める	
\\	トムは承認した。	
\\	[名詞]	信じること、信心	
\\	知識は力なりというのは私の信念だ。	
\\	複数形チェック
\\	[名詞]	アーチ	
\\	複数形チェック
\\	[名詞]	信じる人	
\\	あなたは、信じられますか。	
\\	複数形チェック
\\	[動詞]	言い争う、口論する	
\\	トムはメアリーと喧嘩している。	
\\	[名詞]	口論、論争	
\\	彼の論旨は筋が通っている。	
\\	複数形チェック
\\	[名詞]	利点、手当て	
\\	彼が自分の利益のために彼らを援助したことは確かだ。	
\\	複数形チェック
\\	[動詞]	逮捕する	
\\	トムが逮捕された。	
\\	[名詞]	矢印、矢	
\\	私は矢を的に当てた。	
\\	複数形チェック
\\	[動詞] [不規則動詞]	賭ける	
\\	彼は彼女と賭けをした。	
\\	[名詞]	記事、条項	
\\	この品物は安い。	
\\	複数形チェック
\\	[名詞]	聖書	
\\	誰が聖書を書いたのか。	
\\	複数形チェック
\\	[形容詞]	恥じている	
\\	とても恥ずかしいです。	
\\	[動詞] [不規則動詞]	縛る	
\\	かれらは彼の両足を縛りあわせた。	
\\	[名詞]	アスパラガス	
\\	複数形チェック
\\	[名詞]	ビスケット	
\\	「ママ、ビスケット食べてもいいでしょ」「だめです。間食はいけません」	
\\	複数形チェック
\\	[名詞]	手伝い、援助	
\\	彼らは援助を申し出た。	
\\	複数形チェック
\\	[形容詞]	苦い、きびしい	
\\	この薬は苦い。	
\\	[名詞]	助手、アシスタント	
\\	彼は助手を非常に信頼している。	
\\	複数形チェック
\\	[形容詞]	白紙の、空白の	
\\	トムは白紙の答案を出した。	
\\	[動詞]	祝福する	
\\	神の恵みがありますように。	
\\	[名詞]	難問、挑戦	
\\	この問題はとても難しい。	
\\	複数形チェック
\\	[名詞]	ブラウス	
\\	そのブラウスを見せて下さい。	
\\	複数形チェック
\\	[名詞]	経路、ルート	
\\	チャンネル変えてもいい?	
\\	複数形チェック
\\	[動詞]	自慢する	
\\	彼女は料理がうまいと自慢している。	
\\	[名詞]	礼拝堂、教会堂	
\\	システィナ礼拝堂は、1473年にバティカン宮殿内に建立された壮大な礼拝堂です。	
\\	複数形チェック
\\	[形容詞]	勇敢な、大胆な	
\\	運命の女神は勇者に味方する。	
\\	[名詞]	表、グラフ	
\\	複数形チェック
\\	[名詞]	爆弾	
\\	この爆弾はたくさんの人を殺すことができる。	
\\	複数形チェック
\\	[動詞]	追跡する	
\\	トムはメアリーを追いかけた。	
\\	[名詞]	本箱	
\\	ジョンは本棚を作った。	
\\	複数形チェック
\\	[形容詞]	陽気な、快活な	
\\	彼はいつも明るい。	
\\	[名詞]	本屋	
\\	複数形チェック
\\	[形容詞]	化学の	
\\	条約は化学兵器の使用を禁止している。	
\\	[名詞]	書棚、蔵書	
\\	彼は彼女に本棚を作ってやった。	
\\	複数形チェック
\\	[名詞]	子供時代	
\\	幼馴染ってこわ・・・。	
\\	複数形チェック
\\	[名詞]	境界、国境	
\\	フランスはイタリアと国境を接する。	
\\	複数形チェック
\\	[名詞]	煙突	
\\	煙突から煙がもくもくと出ていた。	
\\	複数形チェック
\\	[動詞]	退屈させる	
\\	退屈だなあ。	
\\	[名詞]	チンパンジー	
\\	複数形チェック
\\	[形容詞]	退屈した	
\\	退屈だなあ。	
\\	[名詞]	あご	
\\	私は彼のあごに1発食らわしてやった。	
\\	複数形チェック
\\	[名詞]	ボウリング	
\\	ボーリングは好き?	
\\	複数形チェック
\\	[名詞]	サーカス	
\\	子どもたちがサーカスの行列を追いかけた。	
\\	複数形チェック
\\	[名詞]	ボクサー	
\\	複数形チェック
\\	[名詞]	古典、古典的作家	
\\	人気作家の翻案によって、古典に新たな命が吹き込まれた。	
\\	複数形チェック
\\	[名詞]	ボクシング	
\\	彼は全国高等学校総合体育大会ボクシング競技大会に参加しました。	
\\	複数形チェック
\\	[名詞]	清掃員	
\\	見てごらんなさい。私の家の方がお宅よりも奇麗ですよ。	
\\	複数形チェック
\\	[名詞]	ブランド、商標	
\\	「君の車はどこの製品ですか」「フォード社のです」	
\\	複数形チェック
\\	[形容詞]	清潔に,きれいに	
\\	ゆで卵の殻をむくのって、どうやったらきれいにできるの?何かコツとかあるの?	
\\	[名詞]	胸、胸部	
\\	母乳で育てています。	
\\	複数形チェック
\\	[名詞]	顧客、クライアント	
\\	彼は丁寧に求人の依頼を断った。	
\\	複数形チェック
\\	[名詞]	品種、種類	
\\	檻の中で飼われると子どもを産まない動物もいる。	
\\	複数形チェック
\\	[名詞]	崖	
\\	「ペット・セミタリー」は真に迫ったサスペンス映画だった。	
\\	複数形チェック
\\	[副詞]	明るく	
\\	太陽はきらきらと輝いていた。	
\\	[名詞]	気候	
\\	気候は変化している。	
\\	複数形チェック
\\	[名詞]	ブロッコリー	
\\	ブロッコリーは最も体にいい野菜の一つだ。	
\\	複数形チェック
\\	[名詞]	登ること、登山	
\\	そのまま登り続けて。	
\\	複数形チェック
\\	[名詞]	眉	
\\	彼は額の汗を拭った。	
\\	複数形チェック
\\	[形容詞]	閉鎖的な、非公開の	
\\	一件落着!	
\\	[名詞]	泡	
\\	私の夢を壊さないで。	
\\	複数形チェック
\\	[副詞]	接近して,ぴったりと	
\\	よく見て。	
\\	[名詞]	バケツ、手桶	
\\	りんごのバケツを持って来て。	
\\	複数形チェック
\\	[動詞] [不規則動詞]	衣服を着せる	
\\	服を着替えなさい。	
\\	[名詞]	雄牛	
\\	牡牛を捕まえるには角をつかめ。	
\\	複数形チェック
\\	[名詞]	衣類	
\\	私はこの服を自分で作りました。	
\\	複数形チェック
\\	[名詞]	銃弾、弾丸	
\\	彼は弾丸で負傷した。	
\\	複数形チェック
\\	[名詞]	カクテル	
\\	トムはメアリーにカクテルを手渡した。	
\\	複数形チェック
\\	[名詞]	房、束	
\\	本当にありがとう。	
\\	複数形チェック
\\	[名詞]	コーヒーポット	
\\	複数形チェック
\\	[名詞]	ハンバーガー	
\\	そのバーガーが不味かったんです!	
\\	複数形チェック
\\	[名詞]	コーラ	
\\	コーラを飲んだら舌がぴりぴりした。	
\\	複数形チェック
\\	[動詞]	埋める	
\\	彼女は一人息子を埋葬した。	
\\	[名詞]	襟	
\\	君のカラーにはしみがついている。	
\\	複数形チェック
\\	[副詞]	忙しそうに	
\\	其の日は一日淋しいペンの音を聞いて暮した。其の間には折々千代々々と云う声も聞えた。文鳥も淋しいから鳴くのではなかろうかと考えた。然し縁側へ出て見ると、二本の留り木の間を、彼方へ飛んだり、此方へ飛んだり、絶間なく行きつ戻りつしている。少しも不平らしい様子はなかった。	
\\	[名詞]	収集家	
\\	彼はちょっとした切手収集家だ。	
\\	複数形チェック
\\	[名詞]	実業家	
\\	複数形チェック
\\	[動詞]	櫛ですく	
\\	私は木のくしを持っています。	
\\	[名詞]	キャベツ	
\\	塩漬けキャベツを細かく切る。	
\\	複数形チェック
\\	[名詞]	組合せ、結合	
\\	いくつかのミスが重なってその事故が起こった。	
\\	複数形チェック
\\	[名詞]	食堂	
\\	カフェテリアは料理を自分で取るセルフサービスのレストランです。	
\\	複数形チェック
\\	[副詞]	快適に	
\\	しっかりと大学の課程を終えました。	
\\	[形容詞]	静かな、冷静な	
\\	落ち着けよ。	
\\	[名詞]	論評、コメント	
\\	コメントありがとうございます。	
\\	複数形チェック
\\	[副詞]	静かに	
\\	ドアまでゆっくり歩きなさい。	
\\	[形容詞]	商業の	
\\	私はあのCMが大好きです。	
\\	[動詞]	取り消す	
\\	彼らはお祭りを中止した。	
\\	[名詞]	委員会	
\\	委員会はその議案を可決した。	
\\	複数形チェック
\\	[名詞]	カヌー	
\\	複数形チェック
\\	[動詞]	理解しあう	
\\	私たちは言語という手段を使って意志の疎通をします。	
\\	[名詞]	炭素	
\\	地球上の生物はみな炭素をもっている。	
\\	複数形チェック
\\	[名詞]	コミュニケーション	
\\	コミュニケーションなしに進展はあり得ない。	
\\	複数形チェック
\\	[名詞]	職業、キャリア	
\\	トムは彼の経歴に注目された。	
\\	複数形チェック
\\	[名詞]	団体、地域社会	
\\	社会は個人から成り立っている。	
\\	複数形チェック
\\	[形容詞]	不注意な、軽率な	
\\	不注意だったわ。	
\\	[形容詞]	小さくて経済的な	
\\	小型車を借りたいのですが。	
\\	[名詞]	じゅうたん、一面の広がり	
\\	このカーペットは足ざわりが良い。	
\\	複数形チェック
\\	[名詞]	競争、競技会	
\\	それは競争の働きではありませんか?	
\\	複数形チェック
\\	[名詞]	車両、輸送	
\\	夫婦は、その日の夕方、馬車に乗りました。	
\\	複数形チェック
\\	[名詞]	文句、苦情	
\\	何か文句ある?	
\\	複数形チェック
\\	[名詞]	人参	
\\	「アメとムチ」で人をやる気にさせることができるのか?	
\\	複数形チェック
\\	[形容詞]	複雑な,入り組んだ	
\\	彼は優越感を持っている。	
\\	[名詞]	荷馬車、カート	
\\	そのショッピングカートを得ることができますか?	
\\	複数形チェック
\\	[名詞]	組成、構成	
\\	10個の文からなる作文を書きなさい。	
\\	複数形チェック
\\	[名詞]	カセット	
\\	私はあの店でビデオデッキを安く買った。	
\\	複数形チェック
\\	[動詞]	関係している	
\\	気にかけていただいてどうも。	
\\	[形容詞]	形式ばらない	
\\	タメ口でいいよ。	
\\	[動詞]	祝う	
\\	私たちはいっしょに祝うべきだ。	
\\	[名詞]	祝賀	
\\	複数形チェック
\\	[動詞]	確認する	
\\	その知らせで私の疑いはいっそう強くなった。	
\\	[名詞]	細胞	
\\	彼は独房に入れられた。	
\\	複数形チェック
\\	[動詞]	つなぐ	
\\	すべて関係がある。	
\\	[名詞]	式典	
\\	学校の終業式が昨日あった。	
\\	複数形チェック
\\	[名詞]	接続	
\\	私はその件とは無関係だ。	
\\	複数形チェック
\\	[名詞]	会⻑、議⻑	
\\	私達はジャックを議長に選んだ。	
\\	複数形チェック
\\	[形容詞]	意識のある	
\\	ドイツ人は環境に対する意識が非常に高い。	
\\	[動詞]	失望させる	
\\	君には幻滅した。	
\\	[名詞]	失望、期待はずれ	
\\	誰の目にも容易に彼の落胆ぶりが見て取れた。	
\\	複数形チェック
\\	[名詞]	接触、連絡	
\\	トムと連絡を取りなさい。	
\\	複数形チェック
\\	[名詞]	発見	
\\	これは驚くべき発見だ。	
\\	複数形チェック
\\	[名詞]	大陸	
\\	英国はイギリス海峡によって欧州大陸と隔てられている。	
\\	複数形チェック
\\	[動詞]	解散させる	
\\	私はその抗議を避けた。	
\\	[名詞]	契約	
\\	トムは契約書をフランス語に訳した。	
\\	複数形チェック
\\	[動詞]	展示する、陳列する	
\\	彼女は才能を発揮した。	
\\	[形容詞]	便利な	
\\	そりゃ便利だね。	
\\	[名詞]	配布、流通	
\\	『極道の妻たち』は、1986年、東映配給網により公開されたヤクザ映画。	
\\	複数形チェック
\\	[名詞]	衣装	
\\	彼はハロウィーンで海賊の衣装を着た。	
\\	複数形チェック
\\	[名詞]	書類	
\\	トムはその文書をフランス語に訳した。	
\\	複数形チェック
\\	[名詞]	せき	
\\	せきが出ます。	
\\	複数形チェック
\\	[名詞]	田舎、田園	
\\	田舎には娯楽がない。	
\\	複数形チェック
\\	[形容詞]	家庭の	
\\	主婦は家庭の仕事がたくさんある。	
\\	[名詞]	覆うこと	
\\	「隆ちゃん見えな~い~」「目隠ししてるんだから当然だ」	
\\	複数形チェック
\\	[形容詞]	劇的な、めざましい	
\\	音楽でも演説でも、間合いはすばらしい効果をあげるものとしてしばしば利用される。	
\\	[名詞]	ひび、割れ目	
\\	コップにひびが入ってるから中身は漏れてる。	
\\	複数形チェック
\\	[名詞]	引き出し、たんす	
\\	ひきだしを閉じて。	
\\	複数形チェック
\\	[名詞]	クレヨン	
\\	複数形チェック
\\	[名詞]	製図、図	
\\	春が近づいている。	
\\	複数形チェック
\\	[形容詞]	創造的な	
\\	トムはとても創造性のある人だ。	
\\	[名詞]	薬局	
\\	彼の母親は15年間薬局を経営している。	
\\	複数形チェック
\\	[名詞]	生き物、動物	
\\	ネズミは臆病な生き物だ。	
\\	複数形チェック
\\	[名詞]	ドラム、太鼓	
\\	ジョンは派手にドラムを打ち鳴らしていた。	
\\	複数形チェック
\\	[名詞]	功績、信用	
\\	現金ですかクレジットカードですか。	
\\	複数形チェック
\\	[名詞]	カモ、アヒル	
\\	そんなことを考えながら『我輩はカモである』を再見しました。	
\\	複数形チェック
\\	[名詞]	犯罪	
\\	彼女は犯罪を犯した。	
\\	複数形チェック
\\	[形容詞]	犯人の、犯罪の	
\\	犯罪者は緊張している。	
\\	[形容詞]	鈍い、退屈な	
\\	ナイフが鋭くない。	
\\	[名詞]	作物、収穫物	
\\	米作は今年豊作だ。	
\\	複数形チェック
\\	[形容詞]	熱心な、真剣な	
\\	ケンは熱心な生徒です。	
\\	[形容詞]	残酷な	
\\	彼はひどい人です。	
\\	[名詞]	ワシ	
\\	鷲が一羽空高く飛んでいた。	
\\	複数形チェック
\\	[動詞]	押しつぶす	
\\	彼はその箱を壊した。	
\\	[名詞]	地震	
\\	昨日地震があった。	
\\	複数形チェック
\\	[名詞]	食器棚、戶棚	
\\	複数形チェック
\\	[動詞]	楽にする、安心させる	
\\	安心しなさい。	
\\	[名詞]	治療	
\\	恋の病に薬なし。	
\\	複数形チェック
\\	[形容詞]	東の、東洋の	
\\	ハイジャック犯たちは中東諸国の出身だった。	
\\	[動詞]	カールする	
\\	猫が丸くなって寝ていた。	
\\	[形容詞]	経済の	
\\	経済の発展はゆっくりと進んだ。	
\\	[名詞]	カレー	
\\	トムは辛いカレーが好きだ。	
\\	複数形チェック
\\	[名詞]	経済、景気	
\\	景気が良い。	
\\	複数形チェック
\\	[名詞]	クッション、緩和策	
\\	複数形チェック
\\	[形容詞]	教育的な	
\\	トムは教育心理学を専攻している。	
\\	[名詞]	カッター	
\\	複数形チェック
\\	[形容詞]	効果的な	
\\	スグリは長寿に有効な果物だ。	
\\	[動詞]	選挙する	
\\	マイクは議長に選出された。	
\\	[名詞]	データ、情報	
\\	このデータは不正確である。	
\\	複数形チェック
\\	[名詞]	選挙	
\\	ポーランドでは今日が投票日です。	
\\	複数形チェック
\\	[名詞]	夜明け	
\\	夜が明けようとしている。	
\\	複数形チェック
\\	[名詞]	要素	
\\	教育は重要な要素である。	
\\	複数形チェック
\\	[名詞]	昼間	
\\	複数形チェック
\\	[副詞]	他のところで	
\\	パリなどでは、夏になると、細君や子供を避暑地にやつて置いて、夫は、土曜日の晩から日曜へかけてそこへ出掛けて行く風習があります。	
\\	[名詞]	甲板、デッキ	
\\	トランプを樫のテーブルに置いて。	
\\	複数形チェック
\\	[名詞]	感動、感情、情動	
\\	彼女は感極まって泣いた。	
\\	複数形チェック
\\	[動詞]	減少させる	
\\	日の光がだんだん衰えてきた。	
\\	[動詞]	雇用する	
\\	ジョンは200人の労働者を雇っている。	
\\	[名詞]	鹿	
\\	射手は鹿を射た。	
\\	複数形チェック
\\	[名詞]	従業員	
\\	彼はその従業員を解雇した。	
\\	複数形チェック
\\	[動詞]	打ち負かす	
\\	彼は負けを認めた。	
\\	[名詞]	雇用主	
\\	急に社員にやめられた。	
\\	複数形チェック
\\	[動詞]	遅らせる	
\\	123便は遅れていますか。	
\\	[名詞]	雇用、仕事	
\\	彼は職を求めている。	
\\	複数形チェック
\\	[動詞]	勇気づける	
\\	教授は私の研究を励ましてくれた。	
\\	[形容詞]	喜んだ	
\\	彼はあなたに会えてとても喜んでいました。	
\\	[名詞]	工学	
\\	その大学生は工学を学んでいます。	
\\	複数形チェック
\\	[名詞]	⺠主主義、⺠主制	
\\	民主主義は古代ギリシャに始まった。	
\\	複数形チェック
\\	[動詞]	もてなす、楽しませる	
\\	[形容詞]	⺠主主義の、⺠主的な	
\\	彼は民主党の支持者だ。	
\\	[名詞]	余興	
\\	田舎には娯楽がない。	
\\	複数形チェック
\\	[名詞]	描写、説明	
\\	その混乱実に名状すべからず。	
\\	複数形チェック
\\	[形容詞]	全体の、完全な	
\\	ほとんど全部分かった。	
\\	[動詞]	値する	
\\	君の身から出た錆だ。	
\\	[名詞]	環境、周囲の状況	
\\	環境を汚染する工場がある。	
\\	複数形チェック
\\	[動詞]	決意する	
\\	彼女はしたたかな女だ。	
\\	[副詞]	平等に、等しく	
\\	我が家の大きな土地を子供たちのために公平に区分する。	
\\	[名詞]	成⻑、発展	
\\	経済の発展はゆっくりと進んだ。	
\\	複数形チェック
\\	[名詞]	機材、機器	
\\	チームのメンバーは道具とユニフォームを提供されます。	
\\	複数形チェック
\\	[名詞]	装置、機器	
\\	こういう機器では買いたくなるような値段です。	
\\	複数形チェック
\\	[名詞]	エスカレーター	
\\	エスカレーターをご利用の際は、手すりにつかまり、黄色い線の内側にお乗りください。	
\\	複数形チェック
\\	[名詞]	悪魔	
\\	彼女は小悪魔なんだ。	
\\	複数形チェック
\\	[名詞]	随筆、エッセー	
\\	「友情」という題で作文を書きなさい。	
\\	複数形チェック
\\	[名詞]	日常の食物	
\\	誰がダイエット中なの?	
\\	複数形チェック
\\	[形容詞]	欠くことのできない、必須の	
\\	それは不可欠です。	
\\	[副詞]	異なって	
\\	夢は逆夢。	
\\	[動詞]	創業する、設立する	
\\	その学校は1650年に設立された。	
\\	[名詞]	取締役	
\\	その監督は批評を気にする。	
\\	複数形チェック
\\	[名詞]	土、泥	
\\	彼は君に塵を食べさせようとしている。	
\\	複数形チェック
\\	[名詞]	例外	
\\	例外は原則のある証拠。	
\\	複数形チェック
\\	[動詞]	怖がらせる	
\\	トムは怖がっていた。	
\\	[動詞]	交換する	
\\	彼は円をドルと換えた。	
\\	[形容詞]	おびえた	
\\	トムは怖がっていた。	
\\	[動詞]	興奮させる	
\\	彼女は興奮しているように見えた。	
\\	[動詞]	油でいためる	
\\	フライポテトのラージサイズをください。	
\\	[形容詞]	興奮した	
\\	彼女は興奮しているように見えた。	
\\	[名詞]	興奮	
\\	ルーシーは興奮して我を忘れた。	
\\	複数形チェック
\\	[名詞]	役割、機能	
\\	関数が副作用を持たない。	
\\	複数形チェック
\\	[名詞]	経費、出費	
\\	このために費用が増える。	
\\	複数形チェック
\\	[名詞]	資金、基金	
\\	彼は資金集めが上手い。	
\\	複数形チェック
\\	[名詞]	実験	
\\	実験は成功だった。	
\\	複数形チェック
\\	[名詞]	葬式	
\\	私は彼の葬儀に参加した。	
\\	複数形チェック
\\	[名詞]	達人、専門家	
\\	私は専門家ではないけど。	
\\	複数形チェック
\\	[名詞]	柔らかい毛の毛皮	
\\	羊の毛はもこもこしている。	
\\	複数形チェック
\\	[名詞]	説明	
\\	説明してくれてありがとう。	
\\	複数形チェック
\\	[名詞]	一団	
\\	3人組が白昼その銀行を襲った。	
\\	複数形チェック
\\	[動詞]	探検する	
\\	1497年に、ジョンカボットはカナダを探検しました。	
\\	[名詞]	隙間、隔たり	
\\	両者の溝が狭まった。	
\\	複数形チェック
\\	[名詞]	探検家	
\\	探検家が山中の道なき道を行った。	
\\	複数形チェック
\\	[名詞]	ニンニク	
\\	その食べ物は少しニンニクの味がした。	
\\	複数形チェック
\\	[動詞]	輸出する	
\\	武器の輸出は禁止されていた。	
\\	[名詞]	ガソリン	
\\	ガソリンは燃料として使われる。	
\\	複数形チェック
\\	[名詞]	世代	
\\	彼らは年上の世代と何の共通点もない。	
\\	複数形チェック
\\	[形容詞]	追加の、余分の	
\\	予備の枕をください。	
\\	[副詞]	優しく	
\\	みんなには優しく話してください。	
\\	[副詞]	極めて、極度に	
\\	今日は非常に暑い。	
\\	[名詞]	地理学、地理	
\\	私は地理と歴史が好きだ。	
\\	複数形チェック
\\	[名詞]	眉	
\\	彼はまゆを上げて難色を示した。	
\\	複数形チェック
\\	[名詞]	失敗	
\\	彼の計画は失敗に終わった。	
\\	複数形チェック
\\	[形容詞]	世界的な	
\\	活動の幅を世界に広げる。	
\\	[副詞]	かなり、いくぶん	
\\	彼は私に対して公正にふるまった。	
\\	[名詞]	のり	
\\	私のスパゲティブリッジに触らないでください! まだ接着剤が固まっている途中です。	
\\	複数形チェック
\\	[名詞]	妖精	
\\	彼女は妖精の役を演じた。	
\\	複数形チェック
\\	[名詞]	ゴールキーパー	
\\	複数形チェック
\\	[名詞]	ヤギ	
\\	馬、ライオン、犬、羊、これらは動物です。	
\\	複数形チェック
\\	[名詞]	物品、商品	
\\	その工場は綿製品を生産する。	
\\	複数形チェック
\\	[形容詞]	もっと遠い、遠いほうの	
\\	彼は非常に疲れていたのでそれ以上遠くにいけなかった。	
\\	[名詞]	ガチョウ、ガン	
\\	ガチョウは水鳥である。	
\\	複数形チェック
\\	[形容詞]	流行の、しゃれた	
\\	彼女はオシャレだ。	
\\	[名詞]	噂話	
\\	噂話のお相手をしている暇はない。	
\\	複数形チェック
\\	[名詞]	羽毛、羽	
\\	類をもって集まる。	
\\	複数形チェック
\\	[名詞]	特徴、特性	
\\	笑いは人間の特質である。	
\\	複数形チェック
\\	[名詞]	知事、総裁	
\\	複数形チェック
\\	[名詞]	熱	
\\	熱があるかい。	
\\	複数形チェック
\\	[副詞]	徐々に	
\\	だんだんエスカレートしてきた。	
\\	[名詞]	ファイル、綴じ込み	
\\	このファイルは削除して下さい。	
\\	複数形チェック
\\	[動詞]	卒業する	
\\	トムは卒業した。	
\\	[名詞]	財政、財政状態	
\\	戦争の資金を調達するために公債が発行された。	
\\	複数形チェック
\\	[名詞]	グラム	
\\	お米2合炊いといて。	
\\	複数形チェック
\\	[形容詞]	金融の	
\\	財政状態は週ごとに悪化している。	
\\	[名詞]	孫	
\\	あなたの孫はどこ?	
\\	複数形チェック
\\	[名詞]	消防士	
\\	トムは消防士だ。	
\\	複数形チェック
\\	[動詞]	承諾する、認める	
\\	グラントはすごくモテモテだった。	
\\	[名詞]	暖炉	
\\	彼は暖炉用に丸太を鋸で切った。	
\\	複数形チェック
\\	[副詞]	第一に	
\\	[動詞]	挨拶する	
\\	彼女たちは私たちを温かく迎えてくれた。	
\\	[名詞]	握ること	
\\	落ち着け!	
\\	複数形チェック
\\	[名詞]	閃光	
\\	ボートはたちまち沈んだ。	
\\	複数形チェック
\\	[形容詞]	茂って	
\\	大きくなったね。	
\\	[名詞]	群れ	
\\	類をもって集まる。	
\\	複数形チェック
\\	[名詞]	成⻑	
\\	我々は急速な人口増加を防がなければならない。	
\\	複数形チェック
\\	[名詞]	小⻨粉、粉	
\\	小麦粉を棚に入れて。	
\\	複数形チェック
\\	[形容詞]	有罪の	
\\	トムが犯人なんだろうか?	
\\	[名詞]	フルート	
\\	彼はフルートを吹くことが出来る。	
\\	複数形チェック
\\	[名詞]	男、やつ	
\\	あいつは、まったく頭にくるやつだ。	
\\	複数形チェック
\\	[名詞]	霧	
\\	霧がはれた。	
\\	複数形チェック
\\	[名詞]	ハンドバッグ	
\\	そのハンドバッグ、私の。	
\\	複数形チェック
\\	[形容詞]	⺠間の	
\\	「どこの迷信よ、それ」「庶民の知恵を言ってくれ」	
\\	[形容詞]	便利な	
\\	そりゃ便利だね。	
\\	[形容詞]	好んで	
\\	君は水泳が好きですか。	
\\	[名詞]	出来事	
\\	何かおかしなことが起こってる。	
\\	複数形チェック
\\	[名詞]	調和、一致	
\\	姉妹は睦まじく暮らした。	
\\	複数形チェック
\\	[副詞]	幸いなことに	
\\	幸い乗客に怪我はありませんでした。	
\\	[名詞]	ハープ	
\\	これは竪琴だ。	
\\	複数形チェック
\\	[名詞]	タカ	
\\	複数形チェック
\\	[動詞]	創設する、起こす	
\\	見つけた!	
\\	[名詞]	聴力	
\\	空耳ですよ。	
\\	複数形チェック
\\	[名詞]	基礎、基盤	
\\	この家は土台がしっかりしている。	
\\	複数形チェック
\\	[名詞]	かかと	
\\	アキレスはかかとを除いては不死身だった。	
\\	複数形チェック
\\	[名詞]	噴水、噴水池	
\\	万年筆を持っていますか。	
\\	複数形チェック
\\	[名詞]	ヘルメット	
\\	トムはヘルメットをかぶってバイクにまたがった。	
\\	複数形チェック
\\	[名詞]	骨組み、額縁	
\\	彼らはテープで枠を作りました	
\\	複数形チェック
\\	[形容詞]	無力な	
\\	[形容詞]	率直な	
\\	フランクはチーズを少し食べる。	
\\	[名詞]	めんどり	
\\	あの狐がめん鳥を殺したに違いない。	
\\	複数形チェック
\\	[名詞]	冷凍庫	
\\	複数形チェック
\\	[名詞]	幹線道路	
\\	街道沿いに家が並んでいた。	
\\	複数形チェック
\\	[名詞]	ほのめかし、ヒント	
\\	ヒントの 
\\	の数と語の中の字数とは関係がありません。	
\\	複数形チェック
\\	[名詞]	紹介、導入	
\\	まずは簡単に自己紹介させていただきます。	
\\	複数形チェック
\\	[名詞]	腰	
\\	複数形チェック
\\	[動詞]	発明する、創り出す	
\\	誰がカラオケを発明したのですか。	
\\	[形容詞]	自家製の	
\\	自家製です。	
\\	[動詞]	発行する	
\\	私達はその問題についてじっくり考えなくてはならない。	
\\	[名詞]	郷里	
\\	故郷を思い出さずにはいられない。	
\\	複数形チェック
\\	[名詞]	品目	
\\	どんな洋服が好みですか。	
\\	複数形チェック
\\	[名詞]	敬意、名誉	
\\	人にもよるけど、女性に対して「サー」(男性への敬称)を使うことは、とても失礼なことです。	
\\	複数形チェック
\\	[名詞]	刑務所、拘置所	
\\	私は服役中だ。	
\\	複数形チェック
\\	[形容詞]	希望に満ちた	
\\	しかし、アメリカよ、我々がそこに到達するため私は今夜ほど希望に満ちたことはない。	
\\	[名詞]	瓶	
\\	クッキーは瓶の中に入っている。	
\\	複数形チェック
\\	[形容詞]	絶望的な	
\\	情勢は絶望だ。	
\\	[名詞]	あご、口腔	
\\	あごが痛い。	
\\	複数形チェック
\\	[名詞]	水平線、地平線	
\\	太陽が地平線の下に沈んだ。	
\\	複数形チェック
\\	[形容詞]	嫉妬深い	
\\	羨ましいなあ。	
\\	[名詞]	角、警笛	
\\	トムはクラクションを鳴らし続けた。	
\\	複数形チェック
\\	[名詞]	ジープ	
\\	複数形チェック
\\	[形容詞]	恐ろしい	
\\	何と酷い天気だ。	
\\	[名詞]	ゼリージャム	
\\	彼の好きなのはゼリーです。	
\\	複数形チェック
\\	[名詞]	恐怖、ぞっとするもの	
\\	ホラー映画を見ることに楽しみを見いだす人もいる。	
\\	複数形チェック
\\	[名詞]	宝石	
\\	宝石は思いがけない場所で見つかった。	
\\	複数形チェック
\\	[名詞]	主婦	
\\	私は主婦です。	
\\	複数形チェック
\\	[名詞]	関節、つなぎ目	
\\	どこの関節がはずれたの?	
\\	複数形チェック
\\	[名詞]	家事	
\\	彼女は家事で忙しかった。	
\\	複数形チェック
\\	[名詞]	雑誌、日記	
\\	学生が英文論文誌に論文を投稿して、「条件付採録」になっています。	
\\	複数形チェック
\\	[名詞]	住宅、住居	
\\	政府は住宅問題を解決するために何の手段も取らなかった。	
\\	複数形チェック
\\	[名詞]	報道記者	
\\	私は記者です。	
\\	複数形チェック
\\	[形容詞]	巨大な	
\\	ネットは広大だわ。	
\\	[名詞]	ユーモア	
\\	『うぐいす侍』人間愛とユーモア溢れる侍物語。	
\\	複数形チェック
\\	[名詞]	カンガルー	
\\	君は今までにカンガルーを見たことがありますか。	
\\	複数形チェック
\\	[形容詞]	100番目の	
\\	百歳の誕生日の数日前になくなった。	
\\	[形容詞]	鋭利な、敏感な	
\\	トムはサーフィンに熱中している。	
\\	[名詞]	空腹、渇望	
\\	お腹がすくと胃が痛みます。	
\\	複数形チェック
\\	[名詞]	ケチャップ	
\\	ケチャップ を渡して。	
\\	複数形チェック
\\	[名詞]	狩猟家、猟師	
\\	猟師は熊を撃った。	
\\	複数形チェック
\\	[名詞]	やかん	
\\	薬缶のお湯が沸いている。	
\\	複数形チェック
\\	[名詞]	理想	
\\	トムは私の理想の男性だ。	
\\	複数形チェック
\\	[名詞]	鍵盤	
\\	俺、指が太いから、小さなキーボードじゃだめなんだよ。	
\\	複数形チェック
\\	[動詞]	見分ける	
\\	彼女は彼を殺人犯だと認めた。	
\\	[名詞]	殺人	
\\	時間をつぶしているだけです。	
\\	複数形チェック
\\	[動詞]	無視する	
\\	トムのことなど気にするな。	
\\	[形容詞]	親切な	
\\	本当にありがとう。	
\\	[名詞]	衝撃、影響	
\\	嵐は作物に多大な影響を与えた。	
\\	複数形チェック
\\	[名詞]	親切	
\\	彼女の親切が身にしみた。	
\\	複数形チェック
\\	[動詞]	輸入する	
\\	私たちはブラジルからコーヒーを輸入している。	
\\	[名詞]	王国	
\\	我が軍はその王国を不意打ちにした。	
\\	複数形チェック
\\	[動詞]	印象を与える	
\\	私は彼の音楽に感動した。	
\\	[名詞]	凧	
\\	たこは上へ上っていった。	
\\	複数形チェック
\\	[名詞]	改善、向上	
\\	クリスは自分の変わりように、ベスが喜ぶだろうと自信がありました。	
\\	複数形チェック
\\	[名詞]	子猫	
\\	子猫はぐっすりと眠った。	
\\	複数形チェック
\\	[名詞]	子猫ちゃん	
\\	キティちゃんの下敷きが欲しい。	
\\	複数形チェック
\\	[名詞]	収入	
\\	今の収入に満足している。	
\\	複数形チェック
\\	[名詞]	キーウィ・フルーツ	
\\	有機キウイです。どうぞご自由にお採り下さい。	
\\	複数形チェック
\\	[動詞]	増加する	
\\	あらゆる物品の価格が上昇した。	
\\	[動詞] [不規則動詞]	編む	
\\	トムは編み物をする。	
\\	[形容詞]	独立の	
\\	彼は経済的に自立した。	
\\	[名詞]	コアラ	
\\	向こうにいるあのコアラを見てごらん。	
\\	複数形チェック
\\	[名詞]	索引	
\\	適合率とは取り出した記事のうち、どの程度の記事が検索条件に合っているかを示す指標です。	
\\	複数形チェック
\\	[動詞]	ラベルをつける	
\\	私はラベルを読みます。	
\\	[名詞]	はしご	
\\	このはしごをしっかり押さえてくれ。	
\\	複数形チェック
\\	[形容詞]	屋内の	
\\	[名詞]	子羊	
\\	毒を食らわば皿まで。	
\\	複数形チェック
\\	[副詞]	屋内に	
\\	雨がひどく降ってきた、それで私たちは屋内で遊んだ。	
\\	[名詞]	着陸	
\\	離陸するほうが着陸するよりたやすい。	
\\	複数形チェック
\\	[形容詞]	工業の、産業の	
\\	工場の排気ガスが大気を汚染している。	
\\	[名詞]	車線、レーン	
\\	一番右側の車線は現在工事中である。	
\\	複数形チェック
\\	[名詞]	膝	
\\	万事トントン拍子にいった。	
\\	複数形チェック
\\	[形容詞]	普段の、気軽な	
\\	[副詞]	主として	
\\	青銅器は主として銅と錫の合金で出来ている。	
\\	[名詞]	怪我、負傷	
\\	怪我を防ぐために、運動前にはストレッチをしています。	
\\	複数形チェック
\\	[副詞]	最近	
\\	最近何か映画見た?	
\\	[形容詞]	内側の	
\\	彼女は自分自身の力に頼らなければならなくなった。	
\\	[形容詞]	最新の	
\\	これが最新の流行です。	
\\	[名詞]	主張する	
\\	例外は作るべきではないと思います。	
\\	複数形チェック
\\	[名詞]	芝生	
\\	彼女は芝生を横切っていそいだ。	
\\	複数形チェック
\\	[形容詞]	先導する、一流の	
\\	彼は大物政治家になるという野心を忘れたことはなかった。	
\\	[名詞]	施設、学会	
\\	複数形チェック
\\	[名詞]	リーグ、同盟	
\\	ナンシーが僕とデートするなんて有り得っこないんだ。高嶺の花だよ。	
\\	複数形チェック
\\	[名詞]	指示、教育	
\\	あなたは熱交換器の取扱説明書をこちらでダウンロードできます。	
\\	複数形チェック
\\	[動詞] [不規則動詞]	かがむ	
\\	寄りかかっていいよ。	
\\	[名詞]	器具	
\\	何か楽器を演奏するの?	
\\	複数形チェック
\\	[名詞]	保険、保証	
\\	私は生命保険に入っています。	
\\	複数形チェック
\\	[名詞]	学ぶこと	
\\	私は音楽を習います。	
\\	複数形チェック
\\	[名詞]	知能、聡明、知性	
\\	彼女はフランスの諜報機関のために働いている。	
\\	複数形チェック
\\	[名詞]	講義	
\\	講義は時間通り始まった。	
\\	複数形チェック
\\	[名詞]	意図、意向	
\\	そんなつもりではなかった。	
\\	複数形チェック
\\	[動詞]	邪魔する、中断する	
\\	ちょっとお邪魔してよろしいですか。	
\\	[名詞]	娯楽、レジャー	
\\	余暇には鹿狩りをしています。	
\\	複数形チェック
\\	[名詞]	面接、インタビュー	
\\	面接はどうでしたか。	
\\	複数形チェック
\\	[名詞]	レモネード	
\\	このレモネードが甘すぎる。	
\\	複数形チェック
\\	[名詞]	レンズ	
\\	コンタクトを入れるというのはいかがでしょう?	
\\	複数形チェック
\\	[名詞]	大臣	
\\	ボブは牧師になりました。	
\\	複数形チェック
\\	[名詞]	レタス	
\\	彼はレタスを洗います。	
\\	複数形チェック
\\	[名詞]	省、内閣	
\\	私の父は外務省に勤めている。	
\\	複数形チェック
\\	[形容詞]	下級の、2流の	
\\	彼女はまだ未成年だ。	
\\	[名詞]	免許、承諾	
\\	私の免許は没収になった。	
\\	複数形チェック
\\	[形容詞]	行方不明で	
\\	何がなくなっているんですか。	
\\	[名詞]	ふた	
\\	彼女は箱のふたを開けた。	
\\	複数形チェック
\\	[形容詞]	間違って	
\\	私が間違ってました。	
\\	[名詞]	生活様式	
\\	彼らの生活様式は私たちの生活様式とは違う。	
\\	複数形チェック
\\	[名詞]	ミキサー	
\\	複数形チェック
\\	[副詞]	軽く、静かに	
\\	軽率な行動はするな。	
\\	[名詞]	月光	
\\	月が湖に影を落としていた。	
\\	複数形チェック
\\	[名詞]	稲妻	
\\	稲妻は危険なこともある。	
\\	複数形チェック
\\	[名詞]	ユリ、ユリの花	
\\	本当かよ?ポールがリリの唇に口付けした?	
\\	複数形チェック
\\	[副詞]	そのうえ	
\\	さらに、横から見ると、前歯は出っ歯の状態にあります。	
\\	[名詞]	関連、つなぐもの	
\\	この文をスペイン語の文にリンクしてください。	
\\	複数形チェック
\\	[副詞]	ほとんど、主に	
\\	成功はたいてい努力次第だ。	
\\	[名詞]	聴取者	
\\	聞き上手になりたい。	
\\	複数形チェック
\\	[名詞]	動き、動作	
\\	第13週:絶対運動と相対運動について学習する。	
\\	複数形チェック
\\	[名詞]	リットル	
\\	複数形チェック
\\	[名詞]	モーターボート	
\\	複数形チェック
\\	[名詞]	文学	
\\	彼女はフランス文学を専攻している。	
\\	複数形チェック
\\	[名詞]	オートバイ	
\\	私はどうしてもバイクがほしい。	
\\	複数形チェック
\\	[動詞]	登る	
\\	冷戦の緊張が高まった。	
\\	[名詞]	パンの一塊	
\\	ミートローフが食べたいです。	
\\	複数形チェック
\\	[名詞]	口ひげ	
\\	トムは口ひげを生やした。	
\\	複数形チェック
\\	[名詞]	融資、ローン	
\\	ペンを貸してもらえますか。	
\\	複数形チェック
\\	[形容詞]	感動的な	
\\	さあ行った、行った。	
\\	[名詞]	位置、場所	
\\	場所を教えてください。	
\\	複数形チェック
\\	[名詞]	筋肉	
\\	この薬は筋肉痛を和らげる。	
\\	複数形チェック
\\	[形容詞]	ゆるんだ	
\\	このボタンは緩い。	
\\	[名詞]	キノコ	
\\	「椎茸」はキノコの一種だ。	
\\	複数形チェック
\\	[名詞]	主人、君主、統治者	
\\	天国にいらつしやる基督様。どうか私を御守り下さいまし。	
\\	複数形チェック
\\	[名詞]	からし	
\\	複数形チェック
\\	[形容詞]	神秘的な、不可解な	
\\	私はいつも不思議な性格の方が好きだった。	
\\	[形容詞]	裸の、むきだしの	
\\	赤ちゃんは裸だった。	
\\	[名詞]	恋人、恋人どうし、愛人	
\\	彼はスポーツの愛好者だ。	
\\	複数形チェック
\\	[名詞]	テーブルナプキン	
\\	ナプキンを貰ってもよろしいでしょうか?	
\\	複数形チェック
\\	[形容詞]	愛情のある	
\\	子供は、人に愛されることが必要なのです。	
\\	[名詞]	必要性	
\\	必要の前に法律はない。	
\\	複数形チェック
\\	[形容詞]	不必要な	
\\	言うまでもないが、彼は二度と来なかった。	
\\	[形容詞]	忠実な、誠実で	
\\	彼はボスに忠実だ。	
\\	[形容詞]	否定的な、陰性の	
\\	ネガティブな性格直したいな。	
\\	[名詞]	忠誠	
\\	彼女の忠誠が評価します。	
\\	複数形チェック
\\	[動詞]	無視する	
\\	彼は職務怠慢だった。	
\\	[名詞]	手荷物	
\\	どれがあなたの荷物ですか?	
\\	複数形チェック
\\	[名詞]	近所	
\\	近所に激安スーパーがあればなあ。	
\\	複数形チェック
\\	[名詞]	昼食時間	
\\	僕たちは昼飯にピザとチキンをたらふく食べた。	
\\	複数形チェック
\\	[名詞]	神経	
\\	何て図々しい奴だ!	
\\	複数形チェック
\\	[名詞]	肺	
\\	彼は肺がんで死んだ。	
\\	複数形チェック
\\	[名詞]	網状組織、ネットワーク	
\\	彼は社外で人脈を築いている。	
\\	複数形チェック
\\	[名詞]	奥様,お嬢様	
\\	複数形チェック
\\	[名詞]	あだ名、愛称	
\\	私のあだ名は「イッチ」です。	
\\	複数形チェック
\\	[名詞]	郵便配達人	
\\	郵便屋はもうきましたか。	
\\	複数形チェック
\\	[名詞]	ばかげた事	
\\	ばかなことを言うな!	
\\	複数形チェック
\\	[副詞]	主に	
\\	日本人は米と魚を主食にしています。	
\\	[形容詞]	北の、北向きの	
\\	2月7日は日本では北方領土の日です。	
\\	[動詞]	維持する	
\\	火災報知器はまったくメンテナンスされなかった。	
\\	[名詞]	小説	
\\	この小説は退屈だ。	
\\	複数形チェック
\\	[名詞]	専攻	
\\	専攻は何ですか。	
\\	複数形チェック
\\	[名詞]	観察	
\\	注意深く観察すれば違いがわかるでしょう。	
\\	複数形チェック
\\	[名詞]	大多数	
\\	彼は選挙で大勝した。	
\\	複数形チェック
\\	[動詞]	観察する、監視する	
\\	太陽の黒点を観察しよう。	
\\	[名詞]	経営、経営の方法	
\\	大学では経営学を専攻しました。	
\\	複数形チェック
\\	[形容詞]	明らかな	
\\	見るからに明らかだよ。	
\\	[動詞]	占める、占有する	
\\	兵士たちが建物を占拠した。	
\\	[名詞]	仮面、マスク	
\\	トムは身元を隠すためにマスクをしていた。	
\\	複数形チェック
\\	[動詞]	起こる	
\\	いつ起こったの?	
\\	[形容詞]	奇妙な	
\\	妙だな。	
\\	[名詞]	合間	
\\	複数形チェック
\\	[動詞]	感情をそこなう	
\\	トムはメアリーを傷つけた。	
\\	[名詞]	マスコミ	
\\	マスコミが報道しなかったのはなぜ?	
\\	複数形チェック
\\	[名詞]	違反	
\\	悪気はなかったのです。	
\\	複数形チェック
\\	[名詞]	旋律	
\\	このメロディーは多くの日本人に知られている。	
\\	複数形チェック
\\	[名詞]	オリーブ	
\\	複数形チェック
\\	[名詞]	会員	
\\	複数形チェック
\\	[名詞]	オムレツ	
\\	オムライスを食べた。	
\\	複数形チェック
\\	[形容詞]	精神の、心の	
\\	お前精神年齢低すぎだろ。	
\\	[前置詞]	の上へ	
\\	彼女は列車の方へ人を押しのけていった。	
\\	[名詞]	商人	
\\	複数形チェック
\\	[名詞]	開くこと、開設	
\\	花が咲き始めている。	
\\	複数形チェック
\\	[動詞]	機能する	
\\	彼女を手術室に運んでくれ。	
\\	[名詞]	⻑所、功績	
\\	複数形チェック
\\	[名詞]	運転、事業	
\\	手術が必要ですか。	
\\	複数形チェック
\\	[名詞]	マイクロホン	
\\	先程、どういうわけかマイクが入りませんでした。	
\\	複数形チェック
\\	[名詞]	交換手、操作者	
\\	あなただってなかなか隅に置けないって噂があるけど。	
\\	複数形チェック
\\	[名詞]	鉱物	
\\	最近アフガニスタンでは、豊富な鉱物資源が発見された。	
\\	複数形チェック
\\	[動詞]	反対する	
\\	彼に逆らうな。	
\\	[名詞]	選択、選択肢	
\\	僕は3つ目のオプションの方が好きだ。	
\\	複数形チェック
\\	[前置詞]	加えて	
\\	1足す2は3である。	
\\	[名詞]	オーケストラ	
\\	交響吹奏楽団でトランペットを吹いています。	
\\	複数形チェック
\\	[名詞]	詩	
\\	ワインとは、ボトルに詰められた詩である。	
\\	複数形チェック
\\	[形容詞]	尖った、厳しい	
\\	彼女は彼を指さした。	
\\	[名詞]	器官、臓器	
\\	脳は極めて複雑な臓器だ。	
\\	複数形チェック
\\	[名詞]	組織	
\\	あの組織は芯まで、腐っている。	
\\	複数形チェック
\\	[名詞]	政治家	
\\	トムは政治家です。	
\\	複数形チェック
\\	[名詞]	政治、政界	
\\	私は政治が大嫌いだ。	
\\	複数形チェック
\\	[動詞]	ポンと鳴る	
\\	彼女はロシアポップが好きだ。	
\\	[形容詞]	戶外の	
\\	野外コンサートは嵐のために中止になった。	
\\	[名詞]	人口	
\\	フランスの人口は何人ですか?	
\\	複数形チェック
\\	[名詞]	オーバーコート	
\\	彼はオーバーを脱いだ。	
\\	複数形チェック
\\	[名詞]	玄関	
\\	彼らは雨が止むまでポーチで待った。	
\\	複数形チェック
\\	[形容詞]	一泊の	
\\	一朝一夕にはできない。	
\\	[名詞]	姿勢、見せ掛け	
\\	あいつがガッツポーズするなんて、よっぽど嬉しかったんだろうな。	
\\	複数形チェック
\\	[名詞]	速度、ペース	
\\	自分のペースでやりなさい。	
\\	複数形チェック
\\	[動詞]	所有する	
\\	幽霊は人を憑くことが出来る。	
\\	[形容詞]	有給の	
\\	トムは払った。	
\\	[副詞]	もしかしたら	
\\	アリスはひょっとすると来るかもしれない。	
\\	[名詞]	葉書	
\\	葉書をおくれ。	
\\	複数形チェック
\\	[名詞]	手のひら	
\\	私はひざを曲げずに両方の手のひらを床に付けることができる。	
\\	複数形チェック
\\	[名詞]	郵便配達員	
\\	彼女は郵便局員になった。	
\\	複数形チェック
\\	[名詞]	ホットケーキ	
\\	私たちは朝食にパンケーキを作った。	
\\	複数形チェック
\\	[副詞]	実際に、ほとんど	
\\	家族みたいなものだしね。	
\\	[名詞]	楽園、エデンの園	
\\	フィンランドを楽園と呼ぼうとまでは思わない。	
\\	複数形チェック
\\	[名詞]	祈り	
\\	一部の人はひどい目に遭ったとき、他人に祈りを求めるのに自分は祈ることを忘れてしまう。	
\\	複数形チェック
\\	[名詞]	包み、小包、小荷物	
\\	このこづつみは宛名がちがう。	
\\	複数形チェック
\\	[名詞]	存在すること、出席、現存	
\\	彼は私がいることに気づいた。	
\\	複数形チェック
\\	[名詞]	駐車	
\\	駐車禁止。	
\\	複数形チェック
\\	[名詞]	圧力	
\\	高血圧です。	
\\	複数形チェック
\\	[動詞]	ふりをする	
\\	解っていないふりをするのは止めなさい。	
\\	[名詞]	共同経営者	
\\	パートナーはいますか?	
\\	複数形チェック
\\	[形容詞]	前の	
\\	トムには先約があった。	
\\	[形容詞]	主要な、主な	
\\	彼は校長になり立てだ。	
\\	[動詞]	軽くたたく	
\\	[名詞]	法則、原理	
\\	これは世界の常識であり、資源管理の大原則だ。	
\\	複数形チェック
\\	[名詞]	つぎあて	
\\	かわいい花のワッペンをつけた女の子を見た。	
\\	複数形チェック
\\	[名詞]	印刷機	
\\	コピー用紙が要ります。	
\\	複数形チェック
\\	[名詞]	忍耐	
\\	彼は大変忍耐強かった。	
\\	複数形チェック
\\	[形容詞]	起こりそうな	
\\	それが起こりそうだ。	
\\	[名詞]	巡回	
\\	10人の警官がその地域のパトロールについた。	
\\	複数形チェック
\\	[名詞]	生産者	
\\	この映像はコミコン2016に参加したファンのために 当映画の製作者により作られたものです	
\\	複数形チェック
\\	[名詞]	支払い	
\\	彼は借金の支払いを要求した。	
\\	複数形チェック
\\	[名詞]	生産、生産量	
\\	鶏卵生産におけるサルモネラ汚染を防止することが必要です。	
\\	複数形チェック
\\	[名詞]	⻄洋ナシ	
\\	梨を食べています。	
\\	複数形チェック
\\	[名詞]	職業	
\\	君の仕事は何ですか。	
\\	複数形チェック
\\	[名詞]	ペニー	
\\	複数形チェック
\\	[名詞]	プロジェクト	
\\	それは気が遠くなるような計画だ。	
\\	複数形チェック
\\	[名詞]	ペニー	
\\	あのペニーはいくらですか?	
\\	複数形チェック
\\	[形容詞]	有望な、期待できる	
\\	彼は前途有望な青年だ。	
\\	[副詞]	適切に、正しく	
\\	運動をする前ちゃんとストレッチをしたほうがいい。	
\\	[名詞]	割合、率	
\\	複数形チェック
\\	[名詞]	提案	
\\	彼女は彼の結婚の申し込みを断った。	
\\	複数形チェック
\\	[名詞]	仕事ぶり、業績	
\\	成功を祈るわ。	
\\	複数形チェック
\\	[動詞]	主張する、抗議する	
\\	私はその抗議を避けた。	
\\	[形容詞]	永久的な	
\\	私たちみんな恒久的な世界平和を願っています。	
\\	[名詞]	パブ	
\\	居酒屋まで駆けっこしましょう!	
\\	複数形チェック
\\	[名詞]	プリン	
\\	タピオカ・プディングが大好きなんです。	
\\	複数形チェック
\\	[名詞]	個性、性格	
\\	彼は個性が強い。	
\\	複数形チェック
\\	[動詞]	水をくみ出す	
\\	彼女はパンプスをはいていた。	
\\	[名詞]	写真術、写真撮影	
\\	写真撮影禁止。	
\\	複数形チェック
\\	[名詞]	カボチャ	
\\	かぼちゃきんとんを食べます。	
\\	複数形チェック
\\	[名詞]	慣用句、言い回し	
\\	いかにも敬虔なるクリスチャンが送る、礼節重き言葉です。	
\\	複数形チェック
\\	[名詞]	刑罰、罰	
\\	彼はその罰を受けるに値する。	
\\	複数形チェック
\\	[名詞]	品質	
\\	もっと品質の良いものはありますか。	
\\	複数形チェック
\\	[名詞]	物理学	
\\	彼は物理学に明るい。	
\\	複数形チェック
\\	[名詞]	口げんか、口論	
\\	つまらぬ事で喧嘩をするな。	
\\	複数形チェック
\\	[名詞]	ピアニスト	
\\	彼はピアニストになった。	
\\	複数形チェック
\\	[名詞]	レインコート	
\\	コロンボ刑事は、いつもすり切れたレインコートを着ている。	
\\	複数形チェック
\\	[名詞]	鳩	
\\	彼女は内股だ。	
\\	複数形チェック
\\	[名詞]	範囲	
\\	トムは多趣味だ。	
\\	複数形チェック
\\	[名詞]	(積まれた)山	
\\	校舎の跡地は瓦礫の山となっている。	
\\	複数形チェック
\\	[副詞]	速く	
\\	日本経済は急速に成長した。	
\\	[名詞]	錠剤	
\\	トムはビタミンとミネラルの錠剤を毎日飲んでいる。	
\\	複数形チェック
\\	[形容詞]	珍しい	
\\	ずいぶんと珍しいですね。	
\\	[名詞]	ピストル	
\\	彼はピストルを取ろうと手を伸ばした。	
\\	複数形チェック
\\	[副詞]	めったに〜しない	
\\	トムは滅多に笑わない。	
\\	[名詞]	投手	
\\	彼らはどんな体勢で捕っても地肩が強いからいい返球ができる。	
\\	複数形チェック
\\	[名詞]	光線、視線	
\\	レントゲン写真は二本の骨折した指を映し出していた。	
\\	複数形チェック
\\	[名詞]	哀れみ、残念な事	
\\	全く気の毒だ。	
\\	複数形チェック
\\	[名詞]	反応	
\\	このニュースに対する彼の反応はどうでしたか。	
\\	複数形チェック
\\	[名詞]	計画、立案	
\\	彼は企画課で働いている。	
\\	複数形チェック
\\	[名詞]	運動場、遊び場	
\\	トムとメアリーとジョンは、運動場で鬼ごっこをして遊んでいた。	
\\	複数形チェック
\\	[名詞]	現実、事実	
\\	ちゃんと現実を直視して生きるべきだ。	
\\	複数形チェック
\\	[名詞]	プラム	
\\	梅の花は今週が見所です。	
\\	複数形チェック
\\	[名詞]	領収書	
\\	これがレシートです。	
\\	複数形チェック
\\	[形容詞]	最近の、新しい	
\\	これは、最近の写真?	
\\	[動詞]	破滅させる	
\\	トムに週末を台無しにされた	
\\	[動詞]	認識する	
\\	彼女は彼が誰だかすぐに分かった。	
\\	[名詞]	ひと袋	
\\	もう寝ます。	
\\	複数形チェック
\\	[動詞]	推薦する	
\\	マウイ島がお勧めです。	
\\	[副詞]	悲しんで	
\\	悲しいことに多くの日本人が亡くなりました。	
\\	[名詞]	レコーダー	
\\	このビデオは正しく機能しない。	
\\	複数形チェック
\\	[副詞]	安全に	
\\	気をつけて運転してください。	
\\	[動詞]	減少する、引き下げる	
\\	我々はエネルギーの需要を減らさなくてはならない。	
\\	[動詞]	言及する	
\\	10頁を参照してください。	
\\	[名詞]	販売、売上(高)	
\\	今月は売上が落ちた。	
\\	複数形チェック
\\	[動詞]	反射する、映す	
\\	水は光を反射する。	
\\	[名詞]	セールスマン	
\\	私は営業マンです。	
\\	複数形チェック
\\	[動詞]	改正する	
\\	私はドイツ語の正書法改革に賛成です。	
\\	[名詞]	見本、試供品	
\\	は血液のサンプルから取れる。	
\\	複数形チェック
\\	[名詞]	冷蔵庫	
\\	冷蔵庫に入れておけば肉は痛まない。	
\\	複数形チェック
\\	[名詞]	サンダル	
\\	「言い忘れてたけどサンダル要るから持ってきてね」「え、それもっと早く言ってよ」	
\\	複数形チェック
\\	[動詞]	断る	
\\	彼らは断った。	
\\	[動詞]	満足させる	
\\	だれも満足しなかった。	
\\	[名詞]	後悔、悔恨	
\\	我々は彼の死を残念に思う。	
\\	複数形チェック
\\	[名詞]	受け皿	
\\	カップを受け皿の上に置いてください。	
\\	複数形チェック
\\	[名詞]	規則	
\\	複数形チェック
\\	[名詞]	ソーセージ	
\\	ベーコンかソーセージはいかがか。	
\\	複数形チェック
\\	[動詞]	拒否する	
\\	彼女は彼を拒んだ。	
\\	[名詞]	節約	
\\	車を買うために貯金をしている。	
\\	複数形チェック
\\	[動詞]	関連させる	
\\	キュウリはスイカの親戚だ。	
\\	[名詞]	のこぎり	
\\	私は彼に会った。	
\\	複数形チェック
\\	[名詞]	関係、関連	
\\	彼は彼女の遠い親戚だ。	
\\	複数形チェック
\\	[名詞]	目盛り	
\\	日本では看護婦さんは社会的地位が高いのですか。	
\\	複数形チェック
\\	[名詞]	関係、結びつき	
\\	需要と供給の間には密接な関係がある。	
\\	複数形チェック
\\	[名詞]	スカーフ	
\\	父方の祖父の米寿のお祝いにマフラーを贈った。	
\\	複数形チェック
\\	[形容詞]	相対的な	
\\	関係代名詞は「接続詞
\\	代名詞」のはたらきをしています。	
\\	[名詞]	予定表、計画表	
\\	列車は定刻に着いた。	
\\	複数形チェック
\\	[動詞]	楽にする、緩める	
\\	落ち着いて。	
\\	[名詞]	はさみ	
\\	じゃんけんぽん。	
\\	複数形チェック
\\	[動詞]	解放する、自由にする	
\\	彼を釈放して。	
\\	[動詞]	さえぎる	
\\	ついたてで部屋が2つに分かれている。	
\\	[名詞]	取り除くこと、安心	
\\	やれやれ	
\\	複数形チェック
\\	[名詞]	ねじ	
\\	クソくらえ!	
\\	複数形チェック
\\	[名詞]	海産食品	
\\	トムはめったに魚介類を食べない。	
\\	複数形チェック
\\	[動詞]	頼る、信頼する	
\\	彼に頼っておけばいい。	
\\	[名詞]	海岸	
\\	あの子は海辺の貝殻売りだ。	
\\	複数形チェック
\\	[動詞]	思い出させる	
\\	君をみていると母さんを思い出す。	
\\	[名詞]	海岸	
\\	旅行者たちは海岸のホテルにとまった。	
\\	複数形チェック
\\	[動詞]	賃借する	
\\	彼らはアパートを借りた。	
\\	[副詞]	第二に、次は	
\\	第二に、当時のヨーロッパの識字率が低かったことだ。	
\\	[動詞]	取り替える	
\\	取り替えてください。	
\\	[名詞]	部分、切片	
\\	課長さんが計画に変更を加えました。	
\\	複数形チェック
\\	[動詞]	代理をする、代表する	
\\	地図の青い線は川を表す。	
\\	[名詞]	安全、安全確保	
\\	誰か警備員を呼べる?	
\\	複数形チェック
\\	[動詞]	要求する、必要とする	
\\	お力添えをお願いいたします。	
\\	[名詞]	種	
\\	複数形チェック
\\	[名詞]	調査、研究	
\\	私はエイズの研究に従事している。	
\\	複数形チェック
\\	[動詞]	答える	
\\	トムは間髪を入れず、返答した。	
\\	[副詞]	めったに〜しない	
\\	ナンシーはなかなか笑顔を見せない。	
\\	[名詞]	応答、反応	
\\	それは、良い反応です。	
\\	複数形チェック
\\	[動詞]	選ぶ	
\\	クリスの行動を選択して下さい。	
\\	[名詞]	責任、責務	
\\	連帯責任は無責任。	
\\	複数形チェック
\\	[名詞]	選択	
\\	かつての鑑定人の選任は非常に不明朗でした。	
\\	複数形チェック
\\	[動詞]	引退する	
\\	トムは引退した。	
\\	[名詞]	自己、自分自身	
\\	トムは自己中心的だ。	
\\	複数形チェック
\\	[動詞]	再検討する、復習する	
\\	見直しをしようじゃないか。	
\\	[形容詞]	利己的な	
\\	私って自己中だな。	
\\	[名詞]	革命	
\\	革命記念日万歳!	
\\	複数形チェック
\\	[形容詞]	〜より年上	
\\	老人を尊敬せねばならない。	
\\	[名詞]	報酬	
\\	はい、ご褒美。	
\\	複数形チェック
\\	[形容詞]	知覚できる	
\\	彼女達は真面目な子たちです。	
\\	[名詞]	リズム、音律	
\\	あの歌のゆっくりしたリズムが好きです。	
\\	複数形チェック
\\	[形容詞]	敏感な	
\\	その監督は批評を気にする。	
\\	[動詞] [不規則動詞]	取り除く	
\\	そんなバカみたいな考えは捨てなさいよ。	
\\	[名詞]	一続き	
\\	彼らは一連の音楽会を開催した。	
\\	複数形チェック
\\	[名詞]	危険、リスク	
\\	リスクが大きすぎる。	
\\	複数形チェック
\\	[副詞]	真剣に、ひどく	
\\	本気?	
\\	[名詞]	競争相手	
\\	子供相手にムキになるなよ。	
\\	複数形チェック
\\	[名詞]	月、日が沈むこと	
\\	太陽が沈みかかっていた。	
\\	複数形チェック
\\	[動詞]	ローストする	
\\	ローストチキンが好きです。	
\\	[動詞]	解決する	
\\	落ち着いて!	
\\	[動詞]	奪う	
\\	君の怪しい行動を見て謎のままなんて僕の安眠を奪うつもりかい?	
\\	[形容詞]	性の、性的な	
\\	私は、あの婦人と性的な関係を持ちませんでした。	
\\	[名詞]	強盗	
\\	父は強盗と取っ組み合いをした。	
\\	複数形チェック
\\	[名詞]	恥ずかしい思い	
\\	恥を知れ!	
\\	複数形チェック
\\	[名詞]	バスローブ	
\\	複数形チェック
\\	[副詞]	鋭く、急に	
\\	どうして出生率がそんなに急激に減ったのだろうか。	
\\	[名詞]	棒、釣りざお	
\\	鞭を惜しむと子供は駄目になる。	
\\	複数形チェック
\\	[動詞] [不規則動詞]	そる	
\\	顔をそってください。	
\\	[形容詞]	ロマンチックな	
\\	とってもロマンチックだな~!	
\\	[名詞]	棚	
\\	小麦粉を棚に入れて。	
\\	複数形チェック
\\	[副詞]	おおよそ	
\\	子供達は本を乱暴に扱った。	
\\	[名詞]	貝、外皮、殻	
\\	彼は打ち解けた。	
\\	複数形チェック
\\	[動詞]	こする	
\\	肩揉んで。	
\\	[名詞]	店主、商人	
\\	複数形チェック
\\	[名詞]	ゴム	
\\	トムは指紋を残さないようにゴム手袋をつけた。	
\\	複数形チェック
\\	[形容詞]	内気な、用心深い	
\\	私は引っ込み思案です。	
\\	[形容詞]	失礼な、無礼な	
\\	私は無礼だった。	
\\	[名詞]	病気	
\\	彼女は病気で休んでいる。	
\\	複数形チェック
\\	[動詞]	ため息をつく	
\\	トムはため息をついた。	
\\	[名詞]	一撃	
\\	私は脳卒中を起こした。	
\\	複数形チェック
\\	[名詞]	署名	
\\	あなたのサインをいただけませんか。	
\\	複数形チェック
\\	[名詞]	構造	
\\	浅田彰の『構造と力―記号論を超えて』を読んだことがありますか?	
\\	複数形チェック
\\	[副詞]	静かに	
\\	それは彼女でした。彼女はそつと私に忍び寄らうとしてゐるのです。	
\\	[名詞]	撮影室、アトリエ	
\\	スタジオはとても狭くて、隠れる場所はない。	
\\	複数形チェック
\\	[動詞]	吸う	
\\	彼は鉛筆をなめるくせがある。	
\\	[形容詞]	誠実な	
\\	私の心からのお詫びをどうかお聞き下さい。	
\\	[名詞]	苦しみ、苦痛	
\\	情熱は苦悩を生む。	
\\	複数形チェック
\\	[副詞]	心から	
\\	本日のご搭乗、まことにありがとうございました。	
\\	[形容詞]	十分な	
\\	それだけで十分足りるだろう。	
\\	[名詞]	歌	
\\	歌い始めなさい。	
\\	複数形チェック
\\	[名詞]	提案、示唆	
\\	あなたの提案は、ごもっともですよ。	
\\	複数形チェック
\\	[形容詞]	ふさわしい、適切な	
\\	彼の演説はその場にふさわしいものだった。	
\\	[名詞]	スケッチ、略図	
\\	彼は林檎を写生しています。	
\\	複数形チェック
\\	[名詞]	合計、総額・数量	
\\	1000ドルというのは、多額の金だ。	
\\	複数形チェック
\\	[名詞]	スケッチブック	
\\	複数形チェック
\\	[形容詞]	日当たりのよい	
\\	天気は上々だ。	
\\	[名詞]	技能、技術	
\\	私は彼の運転の技術は素晴らしいと思う。	
\\	複数形チェック
\\	[名詞]	日の出	
\\	オンドリは日の出に鳴く。	
\\	複数形チェック
\\	[名詞]	奴隷	
\\	彼女は流行にとらわれている。	
\\	複数形チェック
\\	[動詞]	供給する	
\\	牛は我々に牛乳を供給する。	
\\	[名詞]	薄片、一切れ	
\\	チーズケーキを1個おねがいします。	
\\	複数形チェック
\\	[名詞]	支持者、応援する人	
\\	複数形チェック
\\	[形容詞]	わずかな	
\\	微熱があります。	
\\	[副詞]	わずかに	
\\	ちょっと小腹がすいたな。	
\\	[名詞]	表面	
\\	月の表面はでこぼこだ。	
\\	複数形チェック
\\	[名詞]	斜面、坂	
\\	ケンは坂を駆け上った。	
\\	複数形チェック
\\	[形容詞]	驚くべき	
\\	これは驚いた。	
\\	[形容詞]	頭の良い、賢明な	
\\	目がヒリヒリします。	
\\	[動詞]	囲っている	
\\	お前は包囲されている。	
\\	[副詞]	柔らかに、静かに	
\\	小さい声で話してください。	
\\	[動詞]	生き残る	
\\	あなたは生き残ります。	
\\	[名詞]	土壌	
\\	かれは土に水をやった。	
\\	複数形チェック
\\	[動詞]	疑う	
\\	警察は容疑者を逮捕した。	
\\	[副詞]	どうにか	
\\	何かあったかい。	
\\	[動詞] [不規則動詞]	宣誓する	
\\	それは私じゃない。誓うよ。	
\\	[副詞]	いつか	
\\	手紙書いてね。	
\\	[名詞]	スウェットシャツ	
\\	複数形チェック
\\	[動詞] [不規則動詞]	掃除する	
\\	床を掃除してください。	
\\	[形容詞]	痛い	
\\	足が痛い。	
\\	[動詞] [不規則動詞]	膨張する	
\\	手足がむくんでいます。	
\\	[名詞]	水泳選手	
\\	彼は泳ぎが上手です。	
\\	複数形チェック
\\	[形容詞]	酸っぱい	
\\	レモンは酸っぱい。	
\\	[名詞]	同情	
\\	私たちは彼女に同情した。	
\\	複数形チェック
\\	[形容詞]	南の、南部の	
\\	ヨーロッパとアフリカの西海岸をとおりすぎ、アフリカ南部の先端までとびます。	
\\	[名詞]	テーブル掛け	
\\	ごめんなさい、テーブルクロスにコーヒーをこぼしちゃったわ。	
\\	複数形チェック
\\	[名詞]	専門家、専門医	
\\	私は専門家ではないけど。	
\\	複数形チェック
\\	[名詞]	仕立屋	
\\	複数形チェック
\\	[副詞]	特別に	
\\	これは学生相手の店です。	
\\	[名詞]	物語	
\\	素敵に面白い冒険物語。	
\\	複数形チェック
\\	[形容詞]	明白な、明確な	
\\	具体的に話して欲しい。	
\\	[名詞]	才能	
\\	彼の才能にはおそれいったよ。	
\\	複数形チェック
\\	[形容詞]	速い	
\\	[名詞]	タンク、水槽	
\\	ガスタンクが突然爆発した。	
\\	複数形チェック
\\	[名詞]	香辛料、薬味	
\\	ひもじい時にまずいものなし。	
\\	複数形チェック
\\	[動詞]	軽くたたく	
\\	元栓、しまってる。	
\\	[動詞] [不規則動詞]	つむぐ	
\\	蚕が糸を吐いている。	
\\	[名詞]	標的、目標	
\\	彼女は的にねらいを定めた。	
\\	複数形チェック
\\	[動詞] [不規則動詞]	割る、裂く	
\\	細かいことにこだわるな。	
\\	[名詞]	指導	
\\	彼は教職に就いた。	
\\	複数形チェック
\\	[動詞] [不規則動詞]	だめにする	
\\	牛乳はすぐ悪くなるの?	
\\	[名詞]	ティーカップ	
\\	複数形チェック
\\	[名詞]	スポーツマン	
\\	トムは真正のスポーツマンだ。	
\\	複数形チェック
\\	[名詞]	チームメイト	
\\	複数形チェック
\\	[名詞]	スパイ	
\\	そのスパイは書類を燃やした。	
\\	複数形チェック
\\	[名詞]	ティーポット	
\\	複数形チェック
\\	[名詞]	競技場、野球場	
\\	熱狂的な観客が競技場になだれ込んだ。	
\\	複数形チェック
\\	[形容詞]	技術の	
\\	これらの専門用語はギリシャ語に由来している。	
\\	[名詞]	職員、社員	
\\	職員意外立ち入るべからず。	
\\	複数形チェック
\\	[名詞]	技術、技法	
\\	あのピアニストは技巧でよく知られている。	
\\	複数形チェック
\\	[名詞]	技術	
\\	我々は科学技術の時代に住んでいる。	
\\	複数形チェック
\\	[名詞]	像	
\\	胴像は頭を持っていない。	
\\	複数形チェック
\\	[名詞]	電報	
\\	私はおじが到着したという電報をうけとった。	
\\	複数形チェック
\\	[形容詞]	固定した	
\\	このはしごをしっかり押さえてくれ。	
\\	[名詞]	望遠鏡	
\\	望遠鏡のピントをあわせた。	
\\	複数形チェック
\\	[名詞]	客室乗務員	
\\	ジェーンは若い頃スチュワーデスだった。	
\\	複数形チェック
\\	[名詞]	気性、平常心	
\\	彼はすぐにかっとなる。	
\\	複数形チェック
\\	[形容詞]	硬い、硬直した	
\\	肩こりがある。	
\\	[名詞]	温度	
\\	平熱です。	
\\	複数形チェック
\\	[名詞]	在庫品、仕入れ品	
\\	在庫がありません。	
\\	複数形チェック
\\	[形容詞]	臨時の、一時的な	
\\	この取り決めは一時的なものでしかない。	
\\	[名詞]	ストッキング	
\\	やだストッキング伝線しちゃってる。	
\\	複数形チェック
\\	[動詞]	傾向がある	
\\	トムには誇張癖がある。	
\\	[形容詞]	盗まれた	
\\	何を盗まれましたか。	
\\	[副詞]	ひどく、たいへん	
\\	すごく腹がへっている。	
\\	[名詞]	料理用レンジ、ストーブ	
\\	ストーブちゃんと消した?	
\\	複数形チェック
\\	[名詞]	理論	
\\	その実験で彼の理論は確かめられた。	
\\	複数形チェック
\\	[名詞]	ストレス、重圧	
\\	カラオケはストレス解消によい。	
\\	複数形チェック
\\	[形容詞]	厳しい	
\\	私の母は行儀作法にやかましい。	
\\	[名詞]	糸	
\\	蚕が糸を吐いている。	
\\	複数形チェック
\\	[名詞]	ひも	
\\	ギターの弦が切れた。	
\\	複数形チェック
\\	[名詞]	脅威、脅し	
\\	嵐のきざしがある。	
\\	複数形チェック
\\	[動詞]	脅す、脅かす	
\\	彼女は彼を脅した。	
\\	[副詞]	始めから終わりまで	
\\	彼は国中をくまなく旅行した。	
\\	[名詞]	花瓶、瓶、つぼ	
\\	彼は花瓶を落とした。	
\\	複数形チェック
\\	[名詞]	雷	
\\	雷鳴がとどろいた。	
\\	複数形チェック
\\	[名詞]	乗り物、伝達手段	
\\	この少年はラジコンの天才よ!	
\\	複数形チェック
\\	[副詞]	堅く、しっかりと	
\\	彼女は私の手にしっかりとしがみ付いた。	
\\	[動詞]	与える	
\\	なぜ飛行機にはパラシュートではなくライフジャケットがあるのか。	
\\	[名詞]	時刻表	
\\	時刻表をいただけますか。	
\\	複数形チェック
\\	[名詞]	被害者、犠牲者	
\\	トムは凶悪犯罪の被害者になった。	
\\	複数形チェック
\\	[名詞]	スズ	
\\	銀とブリキの区別がつきますか。	
\\	複数形チェック
\\	[名詞]	勝利、優勝	
\\	これはあなた方の勝利だ。	
\\	複数形チェック
\\	[名詞]	トースト	
\\	朝食はコーヒーとパンに決めている。	
\\	複数形チェック
\\	[名詞]	酢、食用酢	
\\	複数形チェック
\\	[名詞]	暴行、暴力	
\\	私たちは暴力を憎む。	
\\	複数形チェック
\\	[動詞]	軽く投げる	
\\	トムはメーリにボールを投げました。	
\\	[名詞]	バイオリニスト	
\\	彼はバイオリンがうまい。	
\\	複数形チェック
\\	[副詞]	まったく、完全に	
\\	こんなことは許されない。	
\\	[名詞]	査証、ビザ	
\\	ビザを申請した。	
\\	複数形チェック
\\	[形容詞]	頑丈な、タフな	
\\	トムはタフなように見える。	
\\	[名詞]	視力、視界	
\\	最近視力が落ちてきた。	
\\	複数形チェック
\\	[名詞]	伝統、しきたり	
\\	複数形チェック
\\	[形容詞]	視覚の	
\\	副作用としては、視力障害があります。	
\\	[形容詞]	伝統的な	
\\	トムは男尊女卑思想の持ち主だ。	
\\	[名詞]	ビタミン	
\\	オレンジにはビタミンCが多く含まれている。	
\\	複数形チェック
\\	[動詞]	移転する	
\\	乗り換えは必要ですか。	
\\	[名詞]	量、売買高	
\\	音を大きくして。	
\\	複数形チェック
\\	[動詞]	翻訳する	
\\	翻訳してるとこ。	
\\	[名詞]	賃金	
\\	ここでは最低賃金は存在しない。	
\\	複数形チェック
\\	[動詞]	運ぶ	
\\	彼は地元の病院に運ばれました。	
\\	[名詞]	ワゴン	
\\	農夫は荷馬車の上へ干し草を放り上げた。	
\\	複数形チェック
\\	[名詞]	わな、計略	
\\	罠だよ!	
\\	複数形チェック
\\	[名詞]	ウエスト	
\\	彼女の腰は細いです。	
\\	複数形チェック
\\	[名詞]	旅行家,	
\\	僕はそんなに旅行好きではない。	
\\	複数形チェック
\\	[名詞]	待つこと	
\\	トムは待っていた。	
\\	複数形チェック
\\	[名詞]	盆、盛り皿	
\\	カップケーキ型に、ペーパーカップを並べます。	
\\	複数形チェック
\\	[名詞]	散歩、歩道	
\\	歩き続けて。	
\\	複数形チェック
\\	[名詞]	宝物	
\\	大切にします。	
\\	複数形チェック
\\	[名詞]	財布	
\\	私の財布が無くなった。	
\\	複数形チェック
\\	[動詞]	待遇する、取り扱う	
\\	君におごってやるよ。	
\\	[名詞]	スイカ	
\\	わたしはスイカが好きです。	
\\	複数形チェック
\\	[名詞]	傾向、動向	
\\	タトエバって何? 最近流行ってるの?	
\\	複数形チェック
\\	[名詞]	ワックス、蝋	
\\	床にワックスをかけなさい。	
\\	複数形チェック
\\	[名詞]	裁判	
\\	彼女は夫をころした容疑で裁判にかけられた。	
\\	複数形チェック
\\	[名詞]	弱さ、弱点	
\\	驕傲は人間の弱さである。	
\\	複数形チェック
\\	[名詞]	三角形	
\\	「三角形」を定義するのは難しいです。	
\\	複数形チェック
\\	[形容詞]	裕福な	
\\	彼女は金持ちです。	
\\	[名詞]	ズボン	
\\	それは僕のズボンです。	
\\	複数形チェック
\\	[名詞]	武器、攻撃の手段	
\\	銃をしまえ。	
\\	複数形チェック
\\	[副詞]	正しく	
\\	ほんとに申し訳ないです。	
\\	[名詞]	平日	
\\	彼女の平日はいつも忙しい。	
\\	複数形チェック
\\	[名詞]	幹	
\\	トムはトランクの中を見た。	
\\	複数形チェック
\\	[動詞] [不規則動詞]	泣く	
\\	トムは涙を流した。	
\\	[名詞]	曲、正しい音程	
\\	彼女は調子外れに歌う。	
\\	複数形チェック
\\	[名詞]	クジラ	
\\	あなたは今まで鯨を見たことがありますか。	
\\	複数形チェック
\\	[動詞]	ねじる	
\\	彼は私に圧力をかけた。	
\\	[名詞]	タイプライター	
\\	私のタイプライターは故障している。	
\\	複数形チェック
\\	[名詞]	小⻨	
\\	カナダは良質の小麦を生産する。	
\\	複数形チェック
\\	[形容詞]	典型的な、独特な	
\\	典型的です。	
\\	[接続詞]	いつでも	
\\	いつでも来て下さい。	
\\	[名詞]	タイピスト	
\\	彼女はタイピストです。	
\\	複数形チェック
\\	[接続詞]	する所はどこでも	
\\	どこへでも好きな所へ座りなさい。	
\\	[形容詞]	醜い	
\\	状況は最悪だ。	
\\	[代名詞]	どれでも	
\\	どちらでもほしいものを取りなさい。	
\\	複数形チェック
\\	[形容詞]	地下の	
\\	イギリスでは地下鉄のことを 
\\	ではなく 
\\	という。	
\\	[名詞]	警笛、笛、ホイッスル	
\\	誰かが口笛を吹くのが聞こえた。	
\\	複数形チェック
\\	[副詞]	下に	
\\	私はマットの下から鍵を見つけた。	
\\	[代名詞]	〜する人は誰でも	
\\	来たい人は誰でも招待してよろしい。	
\\	複数形チェック
\\	[形容詞]	予期せぬ	
\\	想定外を想定せよ。	
\\	[副詞]	広範囲に	
\\	ゲーテの詩「ミニヨン」は日本では森鷗外の名訳で広く愛誦されてきた。	
\\	[形容詞]	不運な	
\\	日本の論者にはこういう謙遜な態度が少ないのが残念だと思います。	
\\	[副詞]	乱暴に	
\\	彼女は胸がドキドキしていた。	
\\	[名詞]	結合、組合	
\\	我々の統合は完璧にできうる。	
\\	複数形チェック
\\	[名詞]	勝者	
\\	ワールドカップの初代王者はウルグアイだ。	
\\	複数形チェック
\\	[形容詞]	ユニークな	
\\	この花は独特の香りがする。	
\\	[形容詞]	勝者である	
\\	結局、私は勝ちました。	
\\	[名詞]	単位、ユニット	
\\	50ポイントのテレホンカード	
\\	複数形チェック
\\	[名詞]	証人、目撃者	
\\	その刑事は目撃者を守ってくれた。	
\\	複数形チェック
\\	[名詞]	宇宙	
\\	宇宙は無限だ。	
\\	複数形チェック
\\	[名詞]	虫	
\\	朝起きは三文の徳。	
\\	複数形チェック
\\	[形容詞]	未知の、不明の	
\\	事故の原因は不明だ。	
\\	[名詞]	傷	
\\	傷口からうみが出た。	
\\	複数形チェック
\\	[形容詞]	不幸な	
\\	ついてないや。	
\\	[動詞]	包む	
\\	それを包んで下さい。	
\\	[形容詞]	珍しい	
\\	トムは何か、ただならぬものを目撃した。	
\\	[名詞]	手首、そで口	
\\	その男は少女の手首をつかんだ。	
\\	複数形チェック
\\	[形容詞]	気が動転して	
\\	まだ怒ってる?	
\\	[形容詞]	無駄な	
\\	これは役に立たない。	
\\	[名詞]	ヨット	
\\	なんてエレガントなヨットなんでしょう!	
\\	複数形チェック
\\	[名詞]	利用者	
\\	「信頼できるユーザー
\\	になるにはどうすれば良いですか?	
\\	複数形チェック
\\	[動詞]	叫ぶ	
\\	彼らは叫んだ。	
\\	[形容詞]	漠然とした	
\\	彼はあいまいな返事をした。	
\\	[名詞]	ヨーグルト	
\\	プレーンヨーグルトと豆乳を加えます。	
\\	複数形チェック
\\	[名詞]	多様(性)	
\\	話題は多様多種だった。	
\\	複数形チェック
\\	[名詞]	地帯、区域	
\\	ここは水泳禁止となっています。	
\\	複数形チェック
\\	[副詞]	乗り物に乗って	
\\	皆さんお乗りください。	
\\	[名詞]	不安、心配	
\\	彼女は不安で気も狂わんばかりだった。	
\\	複数形チェック
\\	[副詞]	絶対に、完全に	
\\	絶対!	
\\	[動詞]	謝る	
\\	謝ったんだから、さっきのはチャラにしてよ。	
\\	[動詞]	吸収する	
\\	スポンジは液体を吸い取る。	
\\	[名詞]	謝罪	
\\	彼はわびを入れた。	
\\	複数形チェック
\\	[形容詞]	抽象的な	
\\	その理論は私にはちんぷんかんぷんだ。	
\\	[形容詞]	明らかな	
\\	明らかに彼は私が言ったことを理解していなかった。	
\\	[名詞]	乱用、虐待	
\\	トムは薬物乱用の経験がある。	
\\	複数形チェック
\\	[副詞]	どうやら〜らしい	
\\	どうやらトムはメアリーのことが嫌いらしい。	
\\	[形容詞]	容認できる	
\\	場合によっては腕力に訴えてもよい。	
\\	[名詞]	食欲	
\\	食欲がほとんどありません。	
\\	複数形チェック
\\	[名詞]	受け入れること	
\\	学生が英文論文誌に論文を投稿して、「条件付採録」になっています。	
\\	複数形チェック
\\	[動詞]	ありがたく思う	
\\	私は感謝しています。	
\\	[名詞]	アクセサリー	
\\	露店などの安いアクセサリーなどが販売しているお店でよく瑪瑙の商品を見るのですが、あれって本物ですか?	
\\	複数形チェック
\\	[形容詞]	適切な	
\\	しかるべき時期にお支払いいたしました。	
\\	[動詞]	成し遂げる	
\\	もっとゆっくり言ってください。	
\\	[名詞]	承認	
\\	私は彼らの賛成を求めた。	
\\	複数形チェック
\\	[動詞]	一致する	
\\	動物は本能に従って行動する。	
\\	[名詞]	アンズ	
\\	ルバーブタルトの方がアプリコットタルトより好きだ。	
\\	複数形チェック
\\	[形容詞]	正確な	
\\	その報告は残念ながら事実だ。	
\\	[形容詞]	
\\	しがちな	
\\	彼は遅刻しがちだ。	
\\	[動詞]	責める、非難する	
\\	トムは濡れ衣を着せられた。	
\\	[名詞]	水族館	
\\	その水族館では、小さな魚たちが泳いでいる。	
\\	複数形チェック
\\	[動詞]	痛む	
\\	頭痛がする。	
\\	[形容詞]	人工的な	
\\	人工的な光は電力という手段によって作られた。	
\\	[名詞]	達成	
\\	あなたの最大の業績は何だと思いますか。	
\\	複数形チェック
\\	[形容詞]	芸術的な	
\\	庭園の芸術的な美しさが素晴らしいです。	
\\	[形容詞]	酸味のある	
\\	レモンはクエン酸を含んでいる。	
\\	[名詞]	灰、燃え殻	
\\	車の窓から煙草の灰を落とす行為はいけないでしょう。	
\\	複数形チェック
\\	[名詞]	知人、知識	
\\	彼は私の妻の知り合いだ。	
\\	複数形チェック
\\	[名詞]	灰皿	
\\	灰皿を下さい。	
\\	複数形チェック
\\	[動詞]	習得する	
\\	彼はロシア語を素早く習得した。	
\\	[名詞]	求めること	
\\	トムは僕に質問し続けた。	
\\	複数形チェック
\\	[名詞]	エーカー	
\\	私は退職後のために8エーカーの農場を買った。	
\\	複数形チェック
\\	[名詞]	側面、局面	
\\	想像力は私たちの生活のどの側面にも影響を与える。	
\\	複数形チェック
\\	[名詞]	実演、演出、演技	
\\	彼女は本能のままに振る舞っているんですよ。	
\\	複数形チェック
\\	[名詞]	尻、肛門	
\\	キムちゃんはすごく閉まる太ももあるんで、石炭の塊を入れば、2週後で、ダイヤモンドを出してるよ!	
\\	複数形チェック
\\	[副詞]	活発に	
\\	1859年以降、プロテスタント宣教師がアメリカから派遣され、またカトリックとロシア正教も布教活動を行った。	
\\	[動詞]	断言する、主張する	
\\	彼らはそれが本当だと言い張った。	
\\	[動詞]	適応する	
\\	彼は環境に順応した。	
\\	[動詞]	連想する	
\\	君はああいった男たちと交際しないほうがいい。	
\\	[動詞]	調節する、調整する	
\\	その子は帽子をきちんとかぶり直した。	
\\	[名詞]	共同、提携	
\\	家は町内会に所属している。	
\\	複数形チェック
\\	[名詞]	調整	
\\	調節レバーを上に上げると座面の高さを調節できます。	
\\	複数形チェック
\\	[動詞]	仮定する	
\\	無料だと思った。	
\\	[名詞]	行政、政府	
\\	複数形チェック
\\	[動詞]	保証する	
\\	全力を尽くしますからご安心下さい。	
\\	[名詞]	感嘆	
\\	複数形チェック
\\	[名詞]	運動選手	
\\	彼は立派な運動選手です。	
\\	複数形チェック
\\	[名詞]	入学・入会金	
\\	学生に限り入場可。	
\\	複数形チェック
\\	[形容詞]	運動の	
\\	競技は予定どおりに進行した。	
\\	[形容詞]	発達した	
\\	彼は大佐に昇進した。	
\\	[名詞]	雰囲気	
\\	月には大気圏がない。	
\\	複数形チェック
\\	[名詞]	広告	
\\	非常に沢山の人々がその広告にだまされた。	
\\	複数形チェック
\\	[名詞]	魅力	
\\	複数形チェック
\\	[名詞]	愛情	
\\	子供達は愛情に飢えていた。	
\\	複数形チェック
\\	[副詞]	自動的に	
\\	ドアが自動的に閉まる。	
\\	[形容詞]	老齢の	
\\	彼女は17歳です。	
\\	[動詞]	待つ	
\\	明日は待ちに待った給料日だ。	
\\	[形容詞]	農業の	
\\	殺虫剤とは、植物に害を与える害虫を防除する農薬のこと。	
\\	[動詞]	賞などを与える	
\\	「秘めた恋」という題の作文で、メアリーは賞をもらった。	
\\	[名詞]	農業	
\\	彼は農業の研究をしている。	
\\	複数形チェック
\\	[副詞]	とても	
\\	くたくたに疲れた。	
\\	[形容詞]	外国の	
\\	「居住外国人」とは、どういう意味ですか。	
\\	[名詞]	おの	
\\	私は今まで斧を使った事がなかった。	
\\	複数形チェック
\\	[動詞]	変える	
\\	私たちは進路を変えてきた。	
\\	[名詞]	竹	
\\	パンダは竹林にすんでいます。	
\\	複数形チェック
\\	[形容詞]	二者択一の、代わりの	
\\	私たちは別の方法を採用した。	
\\	[動詞]	禁止する	
\\	国立公園内での狩猟は禁止されている。	
\\	[名詞]	アマチュア	
\\	僕も素人です。	
\\	複数形チェック
\\	[名詞]	銀行業	
\\	銀行は何時から何時まで開いていますか。	
\\	複数形チェック
\\	[動詞]	驚かす	
\\	すごいですね。	
\\	[副詞]	かろうじて	
\\	トムはほとんど何も言わなかった。	
\\	[形容詞]	驚くほどの	
\\	すごいですね。	
\\	[名詞]	樽	
\\	腐ったリンゴが1つあるとたる全体がだめになる。	
\\	複数形チェック
\\	[名詞]	野心、大望	
\\	彼は大望を抱いていた。	
\\	複数形チェック
\\	[副詞]	基本的に	
\\	基本的にあなたの意見に賛成です。	
\\	[形容詞]	野心的な	
\\	トムには野望がある。	
\\	[名詞]	洗面器	
\\	洗面器の水がかちかちに凍った。	
\\	複数形チェック
\\	[動詞]	面白がらせる	
\\	楽しんでるよ。	
\\	[名詞]	光線	
\\	複数形チェック
\\	[名詞]	楽しみ	
\\	読書が彼のおもな楽しみだ。	
\\	複数形チェック
\\	[名詞]	ビーバー	
\\	複数形チェック
\\	[形容詞]	面白い	
\\	彼女は子供たちに面白い話をしてあげた。	
\\	[名詞]	ベッドの傍ら	
\\	複数形チェック
\\	[動詞]	分析する、解析する	
\\	[名詞]	乞食、貧乏人	
\\	私は乞食も同然だ。	
\\	複数形チェック
\\	[名詞]	先祖	
\\	私たちは祖先をうやまうべきだ。	
\\	複数形チェック
\\	[名詞]	振る舞い、行動、習性	
\\	人間の行為が放射性物質よりも環境に危険だということは明らかです。	
\\	複数形チェック
\\	[名詞]	錨	
\\	船は錨を下ろした。	
\\	複数形チェック
\\	[名詞]	最愛の人	
\\	トムは最愛の息子を亡くした。	
\\	複数形チェック
\\	[名詞]	ベリー	
\\	夏には、それは鳥の卵、スズメバチの幼虫、ベリーおよび蜂蜜を食べる。	
\\	複数形チェック
\\	[名詞]	記念日	
\\	結婚記念日おめでとう。	
\\	複数形チェック
\\	[動詞]	裏切る	
\\	トムは私を裏切った。	
\\	[形容詞]	一年の	
\\	私の年収は500万円を超している。	
\\	[名詞]	10億	
\\	10億人の人たちが英語を話しています。	
\\	複数形チェック
\\	[形容詞]	骨董の	
\\	古いつぼが掘り出された。	
\\	[動詞] [不規則動詞]	噛む	
\\	彼女は彼に噛みついた。	
\\	[副詞]	ひどく	
\\	彼女はひどく泣いた。	
\\	[名詞]	慈善行為	
\\	自愛は我が家からはじまる。	
\\	複数形チェック
\\	[名詞]	刃	
\\	複数形チェック
\\	[名詞]	魅力、魔力	
\\	サミはレーラの魅力の虜になりました。	
\\	複数形チェック
\\	[動詞] [不規則動詞]	出血する	
\\	彼は鼻血を出した。	
\\	[動詞]	だます	
\\	彼は浮気していた。	
\\	[動詞]	混ぜ合わせる	
\\	空が海と一つに溶け込むように見えた。	
\\	[名詞]	チアリーダー	
\\	複数形チェック
\\	[名詞]	祝福、恵み	
\\	彼女は彼を幸せにしました。	
\\	複数形チェック
\\	[名詞]	シェフ、コック、料理人	
\\	その料理人は毎日違う料理を拵える。	
\\	複数形チェック
\\	[形容詞]	血なまぐさい	
\\	壁には血がついている。	
\\	[名詞]	化学者	
\\	薬剤師になる目的で薬学を志望する人は6年制学科を選んでください。	
\\	複数形チェック
\\	[名詞]	花、花盛り	
\\	100本の花が咲きますように!	
\\	複数形チェック
\\	[名詞]	化学	
\\	化学が嫌いだ。	
\\	複数形チェック
\\	[名詞]	花	
\\	リンゴの木が美しい花を咲かせている。	
\\	複数形チェック
\\	[名詞]	チェス	
\\	ジョンはチェスが好きだ。	
\\	複数形チェック
\\	[動詞]	顔を赤らめる	
\\	トムは赤くなった。	
\\	[動詞]	かむ	
\\	トムは噛みタバコを使用する。	
\\	[名詞]	ボディーガード	
\\	トムには個人的なボディガードがいる。	
\\	複数形チェック
\\	[副詞]	主に	
\\	[名詞]	ボイラー	
\\	複数形チェック
\\	[形容詞]	子供じみた	
\\	ピーターは子供じみた娘たちにほとほとうんざりして、真に大人の女性に会いたいと思った。	
\\	[動詞]	ボルトで止める	
\\	そのボルトに嵌合するナットが必要です。	
\\	[名詞]	冷え、悪寒	
\\	落ち着いて。	
\\	複数形チェック
\\	[名詞]	結束、束縛	
\\	戦争の資金を調達するために公債が発行された。	
\\	複数形チェック
\\	[動詞]	切り倒す	
\\	玉ねぎを切った。	
\\	[名詞]	ボーナス、予期しない物	
\\	通常のフライトマイルと同数のボーナスマイルをプレゼント!	
\\	複数形チェック
\\	[名詞]	小冊子	
\\	複数形チェック
\\	[動詞]	悩ます	
\\	どうぞお構いなく。	
\\	[名詞]	回路、回線	
\\	日本には1都1道2府43県があります。	
\\	複数形チェック
\\	[名詞]	境界線	
\\	この川が二つの県の境界になっている。	
\\	複数形チェック
\\	[名詞]	状況、事情	
\\	彼は環境に順応した。	
\\	複数形チェック
\\	[名詞]	ブランデー	
\\	泡立てたクリームに少量のブランデーを加えます。	
\\	複数形チェック
\\	[形容詞]	市⺠の、内政の	
\\	彼らはアメリカ南北戦争に負けました。	
\\	[名詞]	真ちゅう	
\\	金は色が真鍮と似ている。	
\\	複数形チェック
\\	[名詞]	拍手	
\\	複数形チェック
\\	[名詞]	崩壊	
\\	複数形チェック
\\	[形容詞]	古典の	
\\	ベティはクラシックが好きだ。	
\\	[名詞]	そよ風	
\\	そよ風が野原に吹いた。	
\\	複数形チェック
\\	[動詞]	クリップで留める	
\\	トムは爪を切っている。	
\\	[名詞]	花嫁	
\\	彼女は花嫁のような衣装を着ている。	
\\	複数形チェック
\\	[名詞]	収納室、戶棚	
\\	一番近い旅行代理店ってどこにあります?	
\\	複数形チェック
\\	[形容詞]	短い、簡潔な	
\\	要するに彼が間違っていたのです。	
\\	[名詞]	指導員、コーチ	
\\	トムはコーチです。	
\\	複数形チェック
\\	[名詞]	ブリーフケース	
\\	トムは自分の書類かばんの中を見てみた。	
\\	複数形チェック
\\	[名詞]	おんどり	
\\	コック・レーンの幽霊は英国の十八世紀の幽霊話のひとつです。	
\\	複数形チェック
\\	[副詞]	手短に、少しの間	
\\	簡単に自己紹介をお願いします。	
\\	[名詞]	暗号、コード	
\\	服装に関する特別な規則はない。	
\\	複数形チェック
\\	[名詞]	芽、つぼみ	
\\	バラはつぼみが出ている。	
\\	複数形チェック
\\	[動詞]	崩壊する	
\\	橋が崩れ落ちた。	
\\	[名詞]	予算	
\\	予算案は上院を通過した。	
\\	複数形チェック
\\	[名詞]	植⺠地、集団	
\\	ジンバブエはかつてイギリスの植民地だった。	
\\	複数形チェック
\\	[名詞]	ブルドーザー	
\\	道を平らにするのにブルドーザーが投入された。	
\\	複数形チェック
\\	[名詞]	支柱、縦列	
\\	このコラムは面白かった。	
\\	複数形チェック
\\	[名詞]	荷物、重荷	
\\	その責任は彼にとって重荷だ。	
\\	複数形チェック
\\	[動詞]	結合する	
\\	理論と実際を結び付けるべきだ。	
\\	[名詞]	肉屋	
\\	複数形チェック
\\	[名詞]	喜劇	
\\	コメディーはドラマよりも現実生活に近い。	
\\	複数形チェック
\\	[名詞]	バイヤー、買い手	
\\	複数形チェック
\\	[名詞]	小屋、客室	
\\	直紀は貧しくして小屋に住んでいた。	
\\	複数形チェック
\\	[名詞]	商業、通商	
\\	我が国は近隣諸国との貿易を促進せねばならない。	
\\	複数形チェック
\\	[名詞]	内閣	
\\	ファイルキャビネットの引き出しが開いている。	
\\	複数形チェック
\\	[名詞]	手数料	
\\	手数料はいくらですか。	
\\	複数形チェック
\\	[動詞]	計算する	
\\	彼は光の速度を計算した。	
\\	[動詞]	犯す	
\\	彼は自殺をした。	
\\	[名詞]	ラクダ	
\\	彼女はらくだに乗った。	
\\	複数形チェック
\\	[名詞]	比較、類似	
\\	東京と比べたら、ロンドンは小さい。	
\\	複数形チェック
\\	[名詞]	運動、遊説	
\\	選挙公約を果たしてくれない政治家がいる。	
\\	複数形チェック
\\	[動詞]	競う	
\\	10チームが賞を得ようと争った。	
\\	[名詞]	運河	
\\	船はパナマ運河を通り抜けた。	
\\	複数形チェック
\\	[名詞]	競合相手	
\\	女の子が父親の愛情を独占したいと思い、母親を競争者とみなしがちであった。	
\\	複数形チェック
\\	[名詞]	癌	
\\	あなたは癌です。	
\\	複数形チェック
\\	[形容詞]	複雑な	
\\	文法はとてもややこしい。	
\\	[形容詞]	有能な	
\\	彼女は英語を教える事ができる。	
\\	[名詞]	構成要素	
\\	その液から溶剤を揮発させる。残るのは、香り成分と植物ワックスの塊。	
\\	複数形チェック
\\	[名詞]	能力、力量	
\\	ホールは満員だった。	
\\	複数形チェック
\\	[名詞]	作曲家	
\\	「クラシックの作曲家で誰が一番好き?」「ドビュッシー」	
\\	複数形チェック
\\	[動詞]	捕らえる	
\\	私は捕虜になった。	
\\	[動詞]	混ぜ合わせる	
\\	一酸化炭素とは炭素化合物の不完全燃焼で発生する有害物質です。	
\\	[名詞]	カーディガン	
\\	複数形チェック
\\	[動詞]	集中する	
\\	トムは彼の仕事に専念した。	
\\	[名詞]	カタログ	
\\	カタログを送ってください。	
\\	複数形チェック
\\	[名詞]	集中	
\\	すぐに集中力を無くしてしまった。	
\\	複数形チェック
\\	[名詞]	種類、部門	
\\	ハリケーンの種類は風速によって決まります。	
\\	複数形チェック
\\	[形容詞]	関係している	
\\	君は政治に関心をもっているかい。	
\\	[名詞]	畜牛、ウシ	
\\	複数形チェック
\\	[名詞]	用心、警戒	
\\	彼は用心するように忠告した。	
\\	複数形チェック
\\	[動詞]	結論づける	
\\	彼らは彼が嘘をついていたという結論に達した。	
\\	[名詞]	洞窟、洞穴	
\\	洞窟を脱出!	
\\	複数形チェック
\\	[名詞]	結論、結末	
\\	その結論はしっかりした根拠に基づいている。	
\\	複数形チェック
\\	[名詞]	セロリ	
\\	複数形チェック
\\	[名詞]	会議、協議会	
\\	会議は5時に終わった。	
\\	複数形チェック
\\	[名詞]	セメント	
\\	複数形チェック
\\	[動詞]	告白する	
\\	私は罪を告白した。	
\\	[形容詞]	特徴的な	
\\	いかにも敬虔なるクリスチャンが送る、礼節重き言葉です。	
\\	[名詞]	告白	
\\	複数形チェック
\\	[形容詞]	確信して	
\\	彼女が自身がない。	
\\	[形容詞]	文化の	
\\	「文化相対主義」とはどういう意味ですか。	
\\	[名詞]	争い、衝突	
\\	その紛争を解決するのは不可能だ。	
\\	複数形チェック
\\	[名詞]	好奇心	
\\	好奇心がお旺盛なのは大変に結構なことである。	
\\	複数形チェック
\\	[名詞]	議会	
\\	連邦予算はかろうじて議会の承認を得た。	
\\	複数形チェック
\\	[動詞]	征服する	
\\	ノルマン人は1066年にイングランドを征服した。	
\\	[形容詞]	現在の、流行している	
\\	私たちは流れに逆らってボートを漕いでいました。	
\\	[名詞]	良心	
\\	良心が彼を苦しめた。	
\\	複数形チェック
\\	[形容詞]	かわいい	
\\	なんて可愛いんでしょう。	
\\	[名詞]	意識	
\\	トムは気を失った。	
\\	複数形チェック
\\	[名詞]	切り傷、切ること	
\\	カッターマシンがサーモンを切っていた。	
\\	複数形チェック
\\	[名詞]	結果、影響	
\\	後は野となれ、山となれ。	
\\	複数形チェック
\\	[名詞]	ダム	
\\	このダムは私たちに水と電力を供給している。	
\\	複数形チェック
\\	[名詞]	熟慮、考慮	
\\	この計画については現在協議中です。	
\\	複数形チェック
\\	[名詞]	いとしい人	
\\	複数形チェック
\\	[動詞]	成り立っている	
\\	アスピリンはどんな物質から成っていますか。	
\\	[動詞]	突進する	
\\	ケンは坂を駆け上った。	
\\	[形容詞]	一定の	
\\	日本の戦中は、飢えとの戦いに明け暮れていたとも言えます。	
\\	[形容詞]	致命的な、致死の	
\\	若い女性は心が希望に満ち溢れて病魔と闘ってきた。	
\\	[副詞]	絶えず	
\\	彼の妻は彼に一日中がみがみ言っている。	
\\	[名詞]	販売人	
\\	複数形チェック
\\	[名詞]	憲法、規約	
\\	憲法を侵してはならない。	
\\	複数形チェック
\\	[動詞]	討論する	
\\	我々はその問題について討議した。	
\\	[動詞]	建設する	
\\	私は木の建物を造る。	
\\	[名詞]	債務	
\\	彼が借金の保証人になってくれた。	
\\	複数形チェック
\\	[名詞]	建設	
\\	工事は10月に始まった。	
\\	複数形チェック
\\	[名詞]	10年間	
\\	過去十年間に多くの変化を経験してきた。	
\\	複数形チェック
\\	[名詞]	コンテナ	
\\	禁じられた蝿の容器を触っちゃいけません。	
\\	複数形チェック
\\	[名詞]	腐朽、腐食	
\\	そのりんごは腐り始めている。	
\\	複数形チェック
\\	[名詞]	競争、競技	
\\	メアリーはミスタトエバのコンテストに勝ちました。	
\\	複数形チェック
\\	[動詞]	だます	
\\	私を騙さないで下さい。	
\\	[名詞]	文脈、前後関係	
\\	それは文脈による。	
\\	複数形チェック
\\	[動詞]	断言する、宣言する	
\\	彼は有罪と宣告された。	
\\	[形容詞]	反対の	
\\	予想に反して楽に勝てましたね。	
\\	[名詞]	減退、減少	
\\	私を助けようという彼の申し出を断るつもりです。	
\\	複数形チェック
\\	[名詞]	相違、対照	
\\	両者は何という違いだろう。	
\\	複数形チェック
\\	[動詞]	飾る	
\\	彼女は壁を絵で飾った。	
\\	[名詞]	貢献、寄付(金)	
\\	複数形チェック
\\	[名詞]	行為	
\\	隣人を助ける事は善行です。	
\\	複数形チェック
\\	[名詞]	便利さ	
\\	ちょっとコンビニ行ってくるね。	
\\	複数形チェック
\\	[動詞]	定義する	
\\	あなたは「幸福」をどのように定義しますか。	
\\	[名詞]	大会、集会	
\\	当たり前のこと聞くなよ。	
\\	複数形チェック
\\	[形容詞]	確定した、明確な	
\\	明確な返事が欲しい。	
\\	[形容詞]	慣習的な、型にはまった	
\\	[副詞]	明確に、きっぱりと	
\\	もちろんだよ!	
\\	[動詞]	納得させる、確信させる	
\\	私はトムを説得できない。	
\\	[名詞]	定義	
\\	ジャズの定義って何ですか?	
\\	複数形チェック
\\	[形容詞]	確信させる	
\\	まだ納得していないようだね。	
\\	[形容詞]	優美な、繊細な	
\\	彼は繊細だ。	
\\	[名詞]	協力	
\\	ご協力に感謝します。	
\\	複数形チェック
\\	[名詞]	配送、配達	
\\	配達は有料ですか。	
\\	複数形チェック
\\	[名詞]	警官	
\\	私は警官だ。	
\\	複数形チェック
\\	[動詞]	証明する、実演する	
\\	トムはリンゴの芯の取り方をやってみせてくれた。	
\\	[名詞]	紐、コード	
\\	彼は機械にコードをつないだ。	
\\	複数形チェック
\\	[動詞]	出発する	
\\	あなたの飛行機は何時に出発しますか。	
\\	[形容詞]	法人の、企業の	
\\	同社の本社はロサンゼルスにあります。	
\\	[名詞]	出発	
\\	彼は出発の用意をした。	
\\	複数形チェック
\\	[名詞]	企業	
\\	政治家として企業のお金を受け取るのは、汚職でなくて何だろう。	
\\	複数形チェック
\\	[形容詞]	頼って、扶養されて	
\\	あまり他人を頼ってはいけません。	
\\	[副詞]	正しく	
\\	ユーザー名とパスワードが正しく入力されていることを確認してください。	
\\	[動詞]	預金する	
\\	私はこちらの銀行に1000ドルの預金がある。	
\\	[名詞]	宇宙	
\\	複数形チェック
\\	[動詞]	下る、傾斜する	
\\	ほとんどのアメリカ人は元をたどれば移民である。	
\\	[名詞]	小さな家、小別荘	
\\	その古い田舎の家にはある魅力がある。	
\\	複数形チェック
\\	[形容詞]	望ましい、魅力的な	
\\	[名詞]	会議、協議	
\\	評議会で正規ギルドに認めてもらえばいいのに。	
\\	複数形チェック
\\	[名詞]	絶望	
\\	絶望した彼は自殺を図った。	
\\	複数形チェック
\\	[名詞]	カウンター	
\\	切符売り場はどこですか?	
\\	複数形チェック
\\	[前置詞]	
\\	にもかかわらず	
\\	それでも私は成功を収めた。	
\\	[名詞]	同国人、同郷の人	
\\	複数形チェック
\\	[名詞]	デザート	
\\	デザートは何ぁ~に?	
\\	複数形チェック
\\	[名詞]	郡	
\\	複数形チェック
\\	[名詞]	行き先、目的地	
\\	旅人はついにその目的地にたどり着いた。	
\\	複数形チェック
\\	[名詞]	臆病	
\\	俺は気が小さい。	
\\	複数形チェック
\\	[名詞]	破壊	
\\	この儘では誰もが、永遠の滅びを免れないことになります。	
\\	複数形チェック
\\	[名詞]	技能、技術	
\\	これから何か手職を身に付けたいと思う方は、是非お読みになることをおすすめします。	
\\	複数形チェック
\\	[形容詞]	詳しい、詳細な	
\\	もっと詳しい情報が欲しい。	
\\	[動詞]	腹ばって進む	
\\	その赤ん坊ははいはいしている。	
\\	[名詞]	刑事、探偵	
\\	彼は刑事だ。	
\\	複数形チェック
\\	[動詞] [不規則動詞]	はって進む	
\\	気色悪い奴だ。	
\\	[名詞]	決断力、決意	
\\	民主党が大勝利を納めたが、我々の前進を疎外してきた分断を癒す決断と謙虚さをもって我々は対処しなければならない。	
\\	複数形チェック
\\	[名詞]	危機、重大局面	
\\	近い将来エネルギー危機が起こるだろう。	
\\	複数形チェック
\\	[形容詞]	重大な、決定的な	
\\	教育は重要な要素である。	
\\	[形容詞]	開発途上にある	
\\	政治的意識を持つようになるには時間がかかる。	
\\	[名詞]	批判、批評	
\\	その監督は批評を気にする。	
\\	複数形チェック
\\	[動詞]	異なる、意見を異にする	
\\	文化は国々によって異なる。	
\\	[動詞]	批判する、批評する	
\\	私を批判するのは止めて!	
\\	[動詞]	消化する	
\\	チーズは簡単に消化しない。	
\\	[名詞]	横断、踏み切り	
\\	少年が通りを横切っているのが見えた。	
\\	複数形チェック
\\	[形容詞]	デジタルの	
\\	彼はデジタルを買った。	
\\	[名詞]	クロスワードパズル	
\\	トムはクロスワードパズルをしている。	
\\	複数形チェック
\\	[名詞]	不利、不便	
\\	複数形チェック
\\	[名詞]	カラス	
\\	屋根の上にいる鳥はからすです。	
\\	複数形チェック
\\	[動詞]	争う、意見が合わない	
\\	私はそうは思いません。	
\\	[名詞]	残酷な行為	
\\	哀れみのない正義は冷酷である。しかし、正義のない哀れみは解体の母である。	
\\	複数形チェック
\\	[名詞]	災害、災難	
\\	彼はかろうじて災難を免れた。	
\\	複数形チェック
\\	[名詞]	水晶	
\\	複数形チェック
\\	[名詞]	規律、鍛錬	
\\	言語学は言語を記述しようとする学問である。	
\\	複数形チェック
\\	[動詞]	思いとどまらせる	
\\	私は気落ちしていた。	
\\	[形容詞]	熱心な	
\\	彼はテニスに熱心だ。	
\\	[形容詞]	明確な、独特な	
\\	二重焦点レンズは「バイフォーカル」とも呼ばれる。	
\\	[名詞]	入ること、入れること	
\\	日曜は入場無料です。	
\\	複数形チェック
\\	[動詞]	見分ける、区別する	
\\	銀とブリキの区別がつきますか。	
\\	[動詞]	ねたむ	
\\	私は彼がうらやましい。	
\\	[名詞]	エピソード	
\\	でもほんとに、エピソード21には涙が出るほど笑ったよ。	
\\	複数形チェック
\\	[動詞]	潜水する	
\\	彼はダイビングの妙手だ。	
\\	[名詞]	時代	
\\	革命は新たな時代をもたらした。	
\\	複数形チェック
\\	[名詞]	潜水夫	
\\	この潜水時計は少し値段が高すぎます。	
\\	複数形チェック
\\	[名詞]	本質	
\\	教育の本質って何だと思う?	
\\	複数形チェック
\\	[名詞]	部門、分割	
\\	絶望的な戦いの中、貴君の師団が善戦したことは高く評価されている。	
\\	複数形チェック
\\	[副詞]	本質的に、本来	
\\	基本的にサイトや雑誌等で販売しているペニス増大サプリメントは効き目がありません。	
\\	[名詞]	呼び鈴	
\\	玄関の呼び鈴が鳴りました。	
\\	複数形チェック
\\	[名詞]	設立	
\\	『カムイの剣』は、1868年の徳川将軍時代の崩壊と、明治天皇下での日本の復興という変革期を舞台にした、一種の侍/忍者物語だ。	
\\	複数形チェック
\\	[名詞]	戶口	
\\	複数形チェック
\\	[名詞]	私有地、土地	
\\	彼は田舎に地所を持っている。	
\\	複数形チェック
\\	[名詞]	点	
\\	11時カッキリに駅に来てください。	
\\	複数形チェック
\\	[動詞]	評価する、見積もる	
\\	損害は千ドルと見積もりしています。	
\\	[名詞]	下書き、下図	
\\	兄ちゃん、こっち生中二つー。	
\\	複数形チェック
\\	[名詞]	祭日の前夜	
\\	12月24日はクリスマスイブです。	
\\	複数形チェック
\\	[動詞]	引きずる	
\\	ポールはよく重いツールをずるずる引きずる。	
\\	[副詞]	いつかは、結局は	
\\	やがて彼はその事件のことを忘れた。	
\\	[名詞]	⻯、ドラゴン	
\\	私はドラゴンフルーツが好きだ。	
\\	複数形チェック
\\	[形容詞]	明らかな	
\\	君が嘘をついたということは明白だ。	
\\	[名詞]	包帯、手当て用品	
\\	服を着ましたか。	
\\	複数形チェック
\\	[名詞]	過剰、超過	
\\	運動も過度になると有害無益である。	
\\	複数形チェック
\\	[形容詞]	干からびた	
\\	レーズンはぶどうを乾燥させたものだ。	
\\	[動詞]	展示する	
\\	[動詞]	漂流する	
\\	多くの人が目的なく人生を漂う。	
\\	[名詞]	展示、展覧会	
\\	展覧会に出ている3つの裸像がどこですか?	
\\	複数形チェック
\\	[名詞]	酒飲み	
\\	トムは大酒飲みだ。	
\\	複数形チェック
\\	[形容詞]	推進する	
\\	私達は交替で運転した。	
\\	[形容詞]	現存する	
\\	現行の法律は人種の多様性を考慮していない。	
\\	[形容詞]	驚いてものが言えないほど	
\\	彼は黙ったままだった。	
\\	[動詞]	拡大する、広げる	
\\	宇宙は膨張しています。	
\\	[形容詞]	ほこりっぽい	
\\	家具はほこりをかぶっていた。	
\\	[名詞]	期待、予想	
\\	そのホテルはあなたの期待通りでしたか。	
\\	複数形チェック
\\	[形容詞]	生き生きした	
\\	登場人物の躍動的な関わり合いこそがこの小説をかくも偉大なものとしている。	
\\	[動詞]	爆発する、破裂する	
\\	彼女はおかしくて吹き出してしまった。	
\\	[名詞]	こだま、反響	
\\	複数形チェック
\\	[名詞]	爆発	
\\	彼はその爆発にぎょっとした。	
\\	複数形チェック
\\	[名詞]	版	
\\	日本語版があったらいいな。	
\\	複数形チェック
\\	[動詞]	暴露する	
\\	彼は危険に晒されていた。	
\\	[名詞]	編集者	
\\	編集者は原稿にさっと目を通した。	
\\	複数形チェック
\\	[動詞]	伸ばす、延⻑する	
\\	彼は右手を延ばした。	
\\	[名詞]	能率、効率	
\\	仕事場の管理者は、正確さ、効率性、献身を期待する。	
\\	複数形チェック
\\	[名詞]	延⻑	
\\	目下増築中である。	
\\	複数形チェック
\\	[形容詞]	効率的な、有能な	
\\	私はできる人間だ。	
\\	[形容詞]	かなり年配の	
\\	老人を尊敬せねばならない。	
\\	[形容詞]	外部の、外の	
\\	外用のみに使用し、内服しないで下さい。	
\\	[形容詞]	電気に関する	
\\	トムは電気技師です。	
\\	[形容詞]	並はずれた、非凡な	
\\	インターネットによると、非常に興味深いと思います。	
\\	[名詞]	電気工、電気技師	
\\	私は電気工です。	
\\	複数形チェック
\\	[形容詞]	極度の、非常な	
\\	君の提案は少し過激だ。	
\\	[名詞]	要因、因子	
\\	たくさんの要因が存在する。	
\\	複数形チェック
\\	[動詞]	出てくる、現れる	
\\	月が雲の後ろから顔を出した。	
\\	[動詞]	あせる、消えていく	
\\	誰かケガした?	
\\	[名詞]	緊急、緊急事態	
\\	緊急事態なのです。	
\\	複数形チェック
\\	[形容詞]	めまいがして	
\\	彼女は血を見て卒倒しそうになった。	
\\	[形容詞]	感情的な、感情の	
\\	心情的には賛成、けれど現実的に反対します。	
\\	[形容詞]	忠実な、誠実な	
\\	私は妻一筋です。	
\\	[名詞]	皇帝、天皇	
\\	天皇陛下万歳	
\\	複数形チェック
\\	[形容詞]	落ちた、落下した	
\\	雪がたくさん降った。	
\\	[名詞]	重要さ、強調	
\\	彼は教育の重要性を強調した。	
\\	複数形チェック
\\	[名詞]	名声、有名なこと	
\\	彼は名声を強く望んでいる。	
\\	複数形チェック
\\	[動詞]	強調する	
\\	彼は平和の大切さを強調した。	
\\	[名詞]	農業	
\\	僕は決して農業が嫌いな訳じゃない。	
\\	複数形チェック
\\	[名詞]	帝国	
\\	合衆国はかつて大英帝国の一部だった。	
\\	複数形チェック
\\	[形容詞]	最も遠い	
\\	[動詞]	可能にする	
\\	この奨学金のおかげで彼は留学できた。	
\\	[動詞]	締める	
\\	門にかぎを掛けなさい。	
\\	[動詞]	囲む、同封する	
\\	[形容詞]	致命的な、死に至る	
\\	致命的な誤りは不注意から起こる。	
\\	[動詞]	出会う	
\\	彼は道で友人と出くわした。	
\\	[形容詞]	終わりのない、無限の	
\\	宇宙は無限だ。	
\\	[形容詞]	好意的な、好ましい	
\\	その本についての彼の論評は好意的だった。	
\\	[動詞]	耐える	
\\	私はその痛みをほとんどがまんできなかった。	
\\	[形容詞]	恐れて	
\\	[動詞]	従事する、携わる	
\\	ただいまお話中です。	
\\	[名詞]	小説、作り事	
\\	彼はSFが好きだ。	
\\	複数形チェック
\\	[名詞]	ろ過器	
\\	当店の水は浄水器を使用しています。	
\\	複数形チェック
\\	[名詞]	約束	
\\	彼らは婚約を破棄した。	
\\	複数形チェック
\\	[名詞]	発見物	
\\	彼の会社は簡単に見つかった。	
\\	複数形チェック
\\	[形容詞]	楽しい	
\\	人生は楽しい。	
\\	[名詞]	最後の仕上げ	
\\	今最後の仕上げをするところです。	
\\	複数形チェック
\\	[名詞]	楽しみ	
\\	音楽と美術は人生を楽しむのに大いに役立つ。	
\\	複数形チェック
\\	[名詞]	握りこぶし	
\\	北斗の拳が大好きだよ!	
\\	複数形チェック
\\	[形容詞]	巨大な、莫大な	
\\	巨大だ。	
\\	[形容詞]	適当で、適切で	
\\	これを試着したいのですが。試着室はどこですか。	
\\	[名詞]	事業、企業	
\\	あなたの作業に対して外から見れる能力、そして、人間らしく道徳を中心にした態度に関して、尊敬をしています。	
\\	複数形チェック
\\	[動詞]	炎を出す、燃え上がる	
\\	家は炎上していた。	
\\	[名詞]	熱心、熱意	
\\	熱意は移る。	
\\	複数形チェック
\\	[名詞]	味	
\\	インドは地域色豊かな食文化があります。	
\\	複数形チェック
\\	[動詞] [不規則動詞]	逃げる、避ける	
\\	大統領が首都から逃げ去りました。	
\\	[形容詞]	有害な	
\\	煙草は健康に有害である。	
\\	[名詞]	(人間・動物の)肉	
\\	ライオンは肉を餌とする。	
\\	複数形チェック
\\	[形容詞]	厳しい、耳・目ざわりな	
\\	世間は厳しい。	
\\	[形容詞]	空を飛ぶ	
\\	鳥は飛ぶ。	
\\	[名詞]	収穫	
\\	家中が力を合わせて麦の刈入れをしました。	
\\	複数形チェック
\\	[動詞]	焦点を合わせる	
\\	その花にカメラの焦点を合わせてください。	
\\	[名詞]	急ぐこと、迅速	
\\	急がば回れ。	
\\	複数形チェック
\\	[動詞]	折りたたむ	
\\	地図を折りたたんでください。	
\\	[名詞]	憎悪、憎しみ	
\\	アンドレアスはアンゲラに対して憎しみを感じる。	
\\	複数形チェック
\\	[名詞]	予報、予想	
\\	天気予報は必ずしも当てになるとは限らない。	
\\	複数形チェック
\\	[動詞]	治す、治る	
\\	陰陽師身の上知らず。	
\\	[名詞]	編隊、編成	
\\	複数形チェック
\\	[名詞]	治癒	
\\	心の傷を癒すには時間がかかる。	
\\	複数形チェック
\\	[副詞]	以前は	
\\	この町は昔きれいでした。	
\\	[名詞]	加熱器	
\\	トムは暖房をつけた。	
\\	複数形チェック
\\	[名詞]	砦、駐屯地	
\\	彼らはとりでをインディアンの手に渡した。	
\\	複数形チェック
\\	[名詞]	暖房	
\\	うちにはちゃんとした暖房施設があります。	
\\	複数形チェック
\\	[名詞]	創設者	
\\	複数形チェック
\\	[副詞]	重く	
\\	彼は大きなあくびをした。	
\\	[副詞]	率直に言えば	
\\	彼はあっさり罪を白状した。	
\\	[名詞]	一盛り、一人前	
\\	彼女は彼を手伝っています。	
\\	複数形チェック
\\	[副詞]	自由に	
\\	自由にお取りください。	
\\	[形容詞]	一人前	
\\	ボクが憧れたのは翔太部長の『力』じゃない。体を張ってでも信念を貫こうとする雄々しい勇気だったはず。	
\\	[形容詞]	頻繁な	
\\	彼は商用でしばしば日本にやってくる。	
\\	[動詞]	躊躇する	
\\	トムは躊躇した。	
\\	[副詞]	頻繁に	
\\	日本は頻繁に地震に襲われる。	
\\	[形容詞]	隠された	
\\	彼は私がかくしたものを見つけることが出来なかった。	
\\	[形容詞]	油で揚げた	
\\	私はフライパンで野菜を炒めた。	
\\	[名詞]	隠れること、隠すこと	
\\	何か隠してるの?	
\\	複数形チェック
\\	[名詞]	国境	
\\	複数形チェック
\\	[名詞]	丘の斜面	
\\	退職後の家を建てようと、南フランスの丘のに小さな土地を買った。	
\\	複数形チェック
\\	[名詞]	霜、霜が降りること	
\\	霜で花が傷んだ。	
\\	複数形チェック
\\	[形容詞]	歴史上の、歴史的な	
\\	歴史小説はとても人気の高いジャンルでした。	
\\	[形容詞]	基本的な、根本的な	
\\	論じるまでもなく基本的人権は尊重されなければならない。	
\\	[名詞]	ホッケー	
\\	ボクシングやホッケーのようなスポーツは好きではない。	
\\	複数形チェック
\\	[名詞]	ギャラリー	
\\	美術館の東の陳列棟は清掃のため閉鎖されていた。	
\\	複数形チェック
\\	[動詞]	賭博をする	
\\	ギャンブルはやめなさい。	
\\	[名詞]	くぼみ、凹地、穴	
\\	彼のいない人生はむなしい。	
\\	複数形チェック
\\	[名詞]	ギャングの一員	
\\	彼はあるギャングに脅されている。	
\\	複数形チェック
\\	[形容詞]	家のない、ホームレスの	
\\	トムはホームレスの支援に多くの時間を費やすのが苦にならない。	
\\	[名詞]	庭師	
\\	植木屋さんに木を何本か植えてもらった。	
\\	複数形チェック
\\	[副詞]	正直に	
\\	私は力持ちに見えるかもしれないが、実はそうでもない。	
\\	[名詞]	同性愛者	
\\	彼はゲイです。	
\\	複数形チェック
\\	[名詞]	正直	
\\	正直は美徳のひとつです。	
\\	複数形チェック
\\	[形容詞]	寛大な、惜しみない	
\\	彼は金離れがよい。	
\\	[名詞]	新婚旅行	
\\	ウエスト夫妻は新婚旅行中です。	
\\	複数形チェック
\\	[名詞]	天才	
\\	彼は天才だ。	
\\	複数形チェック
\\	[名詞]	ボンネット、フード	
\\	どうやってボンネットを開けるのですか。	
\\	複数形チェック
\\	[名詞]	身振り	
\\	親切だけいただくよ ありがとう。	
\\	複数形チェック
\\	[動詞]	ぴょんと跳ぶ	
\\	乗れよ。	
\\	[名詞]	ショウガ	
\\	生姜は体を温める食材なので、冷え性の人にはお勧めです。	
\\	複数形チェック
\\	[名詞]	馬の背	
\\	乗馬がしたい。	
\\	複数形チェック
\\	[名詞]	地球儀、地球	
\\	彼女は地球のあちこちを旅行した。	
\\	複数形チェック
\\	[名詞]	主人役の女性	
\\	複数形チェック
\\	[名詞]	世帯、家庭	
\\	彼女は家事で忙しかった。	
\\	複数形チェック
\\	[動詞]	燃えるような色に輝く	
\\	[名詞]	人間性	
\\	人類は今朝亡くなった。	
\\	複数形チェック
\\	[名詞]	女神、絶世の美女	
\\	マネがこの絵を描くまで、女性の裸像は女神に限られていました。	
\\	複数形チェック
\\	[形容詞]	謙虚に、控えめな	
\\	トムはとても腰が低い。	
\\	[名詞]	善良、親切さ	
\\	わぁあ~~っ!?隠してっ!前、隠してよっ!	
\\	複数形チェック
\\	[名詞]	狩猟、狩り	
\\	ほんとに求職は嫌いだ。	
\\	複数形チェック
\\	[動詞]	統治する、治める	
\\	インドは長年にわたって英国に支配されていた。	
\\	[名詞]	小屋	
\\	あの小屋にはトカゲや虫がうようよしている。	
\\	複数形チェック
\\	[形容詞]	徐々の、ゆるやかな	
\\	[名詞]	独自性、アイデンティティ	
\\	トムは身元を隠すためにマスクをしていた。	
\\	複数形チェック
\\	[名詞]	穀類、(穀物の)粒	
\\	ライ麦は貧困の穀物と称されていた。	
\\	複数形チェック
\\	[動詞]	怠けて過ごす	
\\	その機械はアイドリング中です。	
\\	[名詞]	祖父、祖父母	
\\	トムは祖父と祖母と一緒にボストンに住んでいる。	
\\	複数形チェック
\\	[名詞]	無知	
\\	無知は幸福。	
\\	複数形チェック
\\	[動詞]	握る	
\\	その溺れていた人はロープをつかんだ。	
\\	[形容詞]	無知な	
\\	私は彼の計画を知らなかった。	
\\	[形容詞]	貪欲な、渇望して	
\\	さて人間とは欲深いものです。	
\\	[形容詞]	違法の	
\\	ファランフさんは駐車違反で罰金を払った。	
\\	[名詞]	深い悲しみ	
\\	彼は悲嘆にくれた。	
\\	複数形チェック
\\	[名詞]	実例	
\\	複数形チェック
\\	[名詞]	食料雑貨商人	
\\	複数形チェック
\\	[動詞]	まねる、見習う	
\\	ただ彼のすることをまねてごらん。	
\\	[形容詞]	全体の	
\\	キモいわー。	
\\	[形容詞]	巨大な	
\\	[動詞]	保証する	
\\	彼が借金の保証人になってくれた。	
\\	[形容詞]	皇帝の、皇室の	
\\	ゴジラは絶対に皇居・スポンサー会社の看板を壊さない。	
\\	[名詞]	指導、手引き	
\\	結果次第では、両社に対して2月以来の行政指導が行われる。	
\\	複数形チェック
\\	[動詞]	課する、強要する	
\\	王は人民を抑圧した。	
\\	[名詞]	案内書	
\\	このガイドブック、あんたの旅行の役に立つかもよ。	
\\	複数形チェック
\\	[形容詞]	見事な	
\\	彼はなんと印象的な人なのでしょう。	
\\	[名詞]	犯罪、罪	
\\	彼は自分の罪を白状した。	
\\	複数形チェック
\\	[名詞]	衝動	
\\	ついうっかり言わないでおけばいいことを言ってしまった。	
\\	複数形チェック
\\	[名詞]	湾	
\\	1990年代は湾岸戦争で始まった。	
\\	複数形チェック
\\	[名詞]	出来事、小事件	
\\	私がその事件について説明しよう。	
\\	複数形チェック
\\	[副詞]	途中の	
\\	途中であきらめるな。	
\\	[副詞]	ますます、段々と	
\\	現代の金融は一段と複雑かつ高度になってきている。	
\\	[名詞]	参考書	
\\	複数形チェック
\\	[名詞]	独立	
\\	今日は独立記念日です。	
\\	複数形チェック
\\	[名詞]	一握り	
\\	ほんの少数の人しかその事実を知らない。	
\\	複数形チェック
\\	[動詞]	示す	
\\	雷は、嵐が近づいている徴候だ。	
\\	[名詞]	つるすこと、絞首刑	
\\	負け犬とつるむな。	
\\	複数形チェック
\\	[形容詞]	避けられない	
\\	必至です。	
\\	[名詞]	幼児	
\\	十分な栄養摂取は幼児の発育にとって極めて重要です。	
\\	複数形チェック
\\	[名詞]	層	
\\	一番下の背景レイヤーを消去します。	
\\	複数形チェック
\\	[動詞]	相続する	
\\	ジョンは莫大な財産を相続した。	
\\	[動詞] [不規則動詞]	飛躍する	
\\	うるう年は366日です。	
\\	[形容詞]	初期の、頭の	
\\	その病気の初期症状は高熱とのどの痛みです。	
\\	[名詞]	学習者	
\\	彼女は教え方を学習の遅い生徒に合わせた。	
\\	複数形チェック
\\	[動詞]	傷つける	
\\	けが人はでなかった。	
\\	[名詞]	法律	
\\	現行の法律は人種の多様性を考慮していない。	
\\	複数形チェック
\\	[形容詞]	怪我をした	
\\	けが人はでなかった。	
\\	[名詞]	嘘つき	
\\	嘘つき!	
\\	複数形チェック
\\	[名詞]	宿屋、旅館	
\\	宿では昼食は出ません。	
\\	複数形チェック
\\	[形容詞]	気前の良い、自由な	
\\	安倍内閣は、自由民主党、公明党を与党とする連立内閣である。	
\\	[形容詞]	無実の、無罪の	
\\	私は無実だ。	
\\	[形容詞]	生物のいない	
\\	[動詞]	激励する、発奮させる	
\\	彼の話は少年たちみんなを元気づけた。	
\\	[形容詞]	一生の、終生の	
\\	[名詞]	生涯	
\\	人生は長いようで短い。	
\\	複数形チェック
\\	[名詞]	本能、直観	
\\	直感です。	
\\	複数形チェック
\\	[名詞]	手足	
\\	トムはメアリーの手足を粘着テープで縛った。	
\\	複数形チェック
\\	[動詞]	設ける、制定する	
\\	[形容詞]	限られた、有限の	
\\	我々の会社は有限会社だ。	
\\	[動詞]	教える、指示する	
\\	ちゃんと指示したじゃん。なんで言われた通りにできないの?	
\\	[名詞]	亜麻布	
\\	複数形チェック
\\	[名詞]	侮辱	
\\	それ褒めてんの、馬鹿にしてんの?	
\\	複数形チェック
\\	[名詞]	口紅	
\\	ほお紅や口紅を差すだけで表情や笑顔がどんどん変わってくるんです。	
\\	複数形チェック
\\	[形容詞]	知的な、知力の	
\\	彼は知識人ぶるのが好きだ。	
\\	[形容詞]	液体の、流体の	
\\	水は液体である。	
\\	[形容詞]	激しい、強烈な	
\\	強い日差しで彼の背中はひどく焼けた。	
\\	[形容詞]	文学の	
\\	彼女には文学の才能がある。	
\\	[名詞]	意図	
\\	トムはそれをやることを決意した。	
\\	複数形チェック
\\	[形容詞]	活発な、元気のよい	
\\	私の班はいつもにぎやかだ。	
\\	[動詞]	干渉する	
\\	僕のことに干渉しないでくれ。	
\\	[名詞]	ロビー	
\\	トムさんはロビーに会いに来てとメアリさんに頼みました。	
\\	複数形チェック
\\	[名詞]	内部	
\\	複数形チェック
\\	[動詞]	位置する	
\\	博多駅はどこにありますか?	
\\	[形容詞]	内部の	
\\	内部のタイマー用ボタン電池が切れた。	
\\	[動詞]	宿泊する、泊まる	
\\	彼は私を一晩泊めてくれた。	
\\	[名詞]	通訳	
\\	ホワイト博士が我々の通訳をして下さった。	
\\	複数形チェック
\\	[名詞]	丸太	
\\	みんな丸太は持ったな	
\\	複数形チェック
\\	[名詞]	間隔	
\\	複数形チェック
\\	[名詞]	論理	
\\	私たちは彼女の論理が理解できなかった。	
\\	複数形チェック
\\	[動詞]	侵略する、侵攻する	
\\	ヒトラーは1939年にポーランドを侵略した。	
\\	[形容詞]	論理的な	
\\	彼の論旨は筋が通っている。	
\\	[名詞]	発明	
\\	愛は作りごとにすぎない。	
\\	複数形チェック
\\	[副詞]	運よく	
\\	運良く、今週は暖かいです。	
\\	[動詞]	投資する	
\\	彼は自分のお金を株に投資した。	
\\	[名詞]	贅沢	
\\	贅沢な暮らしをしたい。	
\\	複数形チェック
\\	[動詞]	調査する	
\\	社会の悪弊を糾すべきだ。	
\\	[名詞]	嘘をつくこと	
\\	彼は嘘をつくことが大嫌いだった。	
\\	複数形チェック
\\	[名詞]	調査、捜査	
\\	その調査は進行中です。	
\\	複数形チェック
\\	[名詞]	機械	
\\	機械は多くの人手を省く。	
\\	複数形チェック
\\	[名詞]	投資	
\\	投資銀行家たちは暴落で途方に暮れています。	
\\	複数形チェック
\\	[名詞]	狂気	
\\	複数形チェック
\\	[形容詞]	目に見えない	
\\	バクテリアは肉眼では見えない。	
\\	[形容詞]	魔法のような、不思議な	
\\	毎年開催中に妙な魔力を感じるんだ。	
\\	[動詞]	伴う、巻き込む	
\\	君は政治に関心をもっているかい。	
\\	[名詞]	手品師	
\\	私は魔法使いではありません。	
\\	複数形チェック
\\	[形容詞]	関わっている	
\\	君は政治に関心をもっているかい。	
\\	[名詞]	磁石、磁鉄	
\\	これは
\\	字形磁石です。	
\\	複数形チェック
\\	[名詞]	嫉妬	
\\	彼女は嫉妬の炎を燃やした。	
\\	複数形チェック
\\	[名詞]	お手伝い	
\\	彼はメイドを使っている。	
\\	複数形チェック
\\	[名詞]	宝石	
\\	何者かがメアリーのフラットに押し入り、宝石を盗んだ。	
\\	複数形チェック
\\	[名詞]	持続、維持	
\\	あなたは車の定期点検をすべきです。	
\\	複数形チェック
\\	[名詞]	ジャーナリズム	
\\	複数形チェック
\\	[形容詞]	喜ばしい	
\\	[形容詞]	手の、手動の.	
\\	あなたは熱交換器の取扱説明書をこちらでダウンロードできます。	
\\	[名詞]	判断、見解	
\\	君の判断をいつも信じていました。	
\\	複数形チェック
\\	[動詞]	製造する	
\\	その工場は玩具を製造している。	
\\	[名詞]	作業用上着	
\\	トムはいつも車のトランクにブースターケーブルを積んでいる。	
\\	複数形チェック
\\	[名詞]	大理石	
\\	複数形チェック
\\	[動詞]	正当化する	
\\	[形容詞]	海の	
\\	[名詞]	番人、看守	
\\	サッカーは、ゴールキーパーがゴールを守るために体のすべての部位を使える以外は、他のプレイヤーは腕でボールに触れてはならないというスポーツです。	
\\	複数形チェック
\\	[動詞]	誘拐する	
\\	しかし、その貸金庫に預けているものを狙った男達に、ハナを誘拐されてしまう。その誘拐犯達とは、実は刑事だった。	
\\	[形容詞]	すばらしい、見事な	
\\	彼女には素晴らしいユーモアのセンスがある。	
\\	[名詞]	幼稚園	
\\	ほらほら、早く食べて幼稚園行かなきゃ。	
\\	複数形チェック
\\	[名詞]	帆柱、マスト	
\\	複数形チェック
\\	[名詞]	跪くこと	
\\	複数形チェック
\\	[形容詞]	成熟した、分別のある	
\\	熟成したチーズの方が好きだ。	
\\	[名詞]	騎士	
\\	クリスはヴァンパイア・ナイトを倒した!	
\\	複数形チェック
\\	[名詞]	最大値	
\\	次の関数の最大値と最小値を求めよ。	
\\	複数形チェック
\\	[名詞]	結び目	
\\	いつ結婚するつもりなの。	
\\	複数形チェック
\\	[名詞]	市⻑	
\\	彼は市長に選出された。	
\\	複数形チェック
\\	[名詞]	草地	
\\	複数形チェック
\\	[形容詞]	レースの	
\\	私は寝室にレースのカーテンを買った。	
\\	[副詞]	その間に	
\\	一方、全国紙の一社だけが夕刊に日米交渉は実質合意に至ったという見出しを掲げて、政府関係者を含め多くの人を驚かせた。	
\\	[名詞]	風景	
\\	日本の農村風景は大きく変わったといわれています。	
\\	複数形チェック
\\	[名詞]	測量、寸法	
\\	測定には寸分の狂いもない精度が要求される。	
\\	複数形チェック
\\	[形容詞]	機械の	
\\	トムさんは機械エンジニアを目指して勉強しています。	
\\	[名詞]	笑い	
\\	笑いは移る。	
\\	複数形チェック
\\	[名詞]	機械装置、仕組み	
\\	発熱は体の防御反応の一つだ。	
\\	複数形チェック
\\	[動詞]	進水する、発射する	
\\	彼らはロケットを発射した。	
\\	[名詞]	洗濯、洗濯物	
\\	ランドリーはどこですか。	
\\	複数形チェック
\\	[動詞]	直す	
\\	貴方の靴は修繕が必要だ。	
\\	[名詞]	慈悲、寛大さ	
\\	船は翻弄されていた。	
\\	複数形チェック
\\	[形容詞]	公然と	
\\	彼は過ちをあからさまに白状した。	
\\	[名詞]	反対、対立	
\\	当局は地域住民の頑強な抵抗にあった。	
\\	複数形チェック
\\	[副詞]	陽気に	
\\	姉は楽しそうに口笛を吹いていた。	
\\	[動詞]	組織する	
\\	いつも仕事場をきちんと整理しておきなさい。	
\\	[名詞]	乱れた状態	
\\	ふざけるな	
\\	複数形チェック
\\	[名詞]	起源	
\\	生命の起源については数多くの学説がある。	
\\	複数形チェック
\\	[名詞]	使者	
\\	彼女は使いの者です。	
\\	複数形チェック
\\	[副詞]	もとは、元来	
\\	彼らはもともと農業をやっていました。	
\\	[名詞]	成果、結果	
\\	これは私たちの研究の成果です。	
\\	複数形チェック
\\	[名詞]	製作所、製粉場	
\\	マメちゃんがコーヒー豆を挽く為にコーヒーミルを使います。	
\\	複数形チェック
\\	[副詞]	野外で	
\\	戸外で運動しなさい。	
\\	[名詞]	大富豪	
\\	私が百万長者であればよいのにな。	
\\	複数形チェック
\\	[形容詞]	外側の、外部の	
\\	宇宙でどうやって食べ物をみつけるの?	
\\	[名詞]	最小値	
\\	ここでは最低賃金は存在しない。	
\\	複数形チェック
\\	[名詞]	概要、あらまし	
\\	まず新しいサイトの概説をしようと思う。	
\\	複数形チェック
\\	[名詞]	少数派、少数	
\\	議長は少数意見を考慮に入れるべきだ。	
\\	複数形チェック
\\	[名詞]	生産高、出力	
\\	の名前は
\\	(出力する)
\\	(「文字列
\\	の頭文字)に由来します。	
\\	複数形チェック
\\	[名詞]	奇跡	
\\	それはまさに奇跡だった。	
\\	複数形チェック
\\	[形容詞]	目立つ、顕著な	
\\	トムはずばぬけたアーティストだ。	
\\	[形容詞]	みじめな	
\\	彼女のおかげで彼はみじめな暮らしをした。	
\\	[副詞]	外側に	
\\	外を眺める者は夢を見るが、内を眺める者は目を覚ます。	
\\	[名詞]	悲惨さ	
\\	複数形チェック
\\	[形容詞]	全体の	
\\	[名詞]	霧	
\\	川にはもやが立ち込めていた。	
\\	複数形チェック
\\	[動詞] [不規則動詞]	克服する	
\\	彼女の心は幸せに満ちあふれていた。	
\\	[名詞]	女主人	
\\	夫との間に波風が立つのが恐くて、愛人の存在は知らない振りをしています。	
\\	複数形チェック
\\	[動詞]	見渡せる	
\\	このようなミスは見逃しやすい。	
\\	[動詞] [不規則動詞]	誤解する	
\\	トムは誤解した。	
\\	[副詞]	海外へ	
\\	海外に行きたい。	
\\	[形容詞]	混ざった	
\\	彼女は彼を他の誰かとまちがえた。	
\\	[名詞]	フクロウ	
\\	僕は夜型なんだ。	
\\	複数形チェック
\\	[名詞]	方法、様式	
\\	複数形チェック
\\	[名詞]	牛、雄牛	
\\	やつらは雄牛を丸焼きにするくらいの火をおこしていた。	
\\	複数形チェック
\\	[形容詞]	謙虚な	
\\	トムはとても腰が低い。	
\\	[名詞]	酸素	
\\	彼は酸素不足で死んだ。	
\\	複数形チェック
\\	[動詞]	絶えず監視する	
\\	私はモニターの文字を読む事が困難です。	
\\	[名詞]	小さな束、小包	
\\	複数形チェック
\\	[名詞]	記念碑	
\\	複数形チェック
\\	[名詞]	板、枠	
\\	残念ながらお客様のご参加を記録することができませんでした。 原因としては、既に以前にご参加頂いているか、契約条件に反している行為があった可能性があります。 パネルプラットフォームへ転送いたします。	
\\	複数形チェック
\\	[形容詞]	泥でよごす	
\\	トムの長靴は泥だらけだった。	
\\	[名詞]	パニック	
\\	みんなパ二くり出した。	
\\	複数形チェック
\\	[動詞]	増やす、掛ける	
\\	3に5を掛けると15になる。	
\\	[形容詞]	お互いの	
\\	トムとマイクはお互いの友人を通じて知り合った。	
\\	[形容詞]	平行の	
\\	平行線は交差しません。	
\\	[副詞]	すなわち	
\\	[名詞]	議会、国会	
\\	この国から遠く離れたところで今夜を見つめているみなさん。外国の議会や宮殿で見ているみなさん、忘れ去られた世界の片隅でひとつのラジオの周りに身を寄せ合っているみなさん、私たちの物語はそれぞれ異なります。けれども私たちはみな、ひとつの運命を共有しているのです。アメリカのリーダーシップはもうすぐ、新たな夜明けを迎えます。	
\\	複数形チェック
\\	[名詞]	うたた寝	
\\	少し眠らないと。	
\\	複数形チェック
\\	[名詞]	オウム	
\\	そのインコが死んでいる。	
\\	複数形チェック
\\	[形容詞]	汚い、不快な	
\\	あいつは嫌な奴だ。	
\\	[形容詞]	部分的な	
\\	彼女は甘いものには目がない。	
\\	[名詞]	国籍	
\\	私はフランス国籍を持っているがベトナム出身です。	
\\	複数形チェック
\\	[形容詞]	すぐ近くの	
\\	トムはこの近くに住んでる。	
\\	[形容詞]	受身の、消極的な	
\\	山田くんはどちらかと言えば草食系だ。	
\\	[副詞]	必然的に、必ず	
\\	天気予報は必ずしも当てになるとは限らない。	
\\	[名詞]	舗装道路	
\\	歩道で自転車に乗ってはいけません。	
\\	複数形チェック
\\	[名詞]	交渉	
\\	両国は平和交渉を開始した。	
\\	複数形チェック
\\	[名詞]	農⺠、小作農	
\\	私は貧しい農民に過ぎない。	
\\	複数形チェック
\\	[副詞]	それにもかかわらず	
\\	私は疲れていたが、それにもかかわらず眠れなかった。	
\\	[形容詞]	特有の、奇妙な	
\\	この習慣は日本に特有のものである。	
\\	[副詞]	最近	
\\	ようやく稼いだ金を無駄遣いするなよ!	
\\	[名詞]	罰	
\\	彼は死刑の廃止を主張した。	
\\	複数形チェック
\\	[副詞]	正常に、いつもは	
\\	普通に呼吸して。	
\\	[動詞]	知覚する、気がつく	
\\	政府は、政治問題に関するネット上の会話を監視し、誤った情報と判断した場合にそれを訂正する方法を模索している。	
\\	[名詞]	北東部	
\\	複数形チェック
\\	[副詞]	自分としては	
\\	私が直接あなたを訪問します。	
\\	[名詞]	北⻄部	
\\	複数形チェック
\\	[動詞]	説得する	
\\	私たちは彼を説得するのに失敗した。	
\\	[名詞]	概念、意見	
\\	世の中には高級な人間と低級な人間が存在している、そんな考えが彼女のエッセイからはうかがえる。そのような考えは低級な人間の物だ。	
\\	複数形チェック
\\	[名詞]	哲学	
\\	哲学は難しいと見なされることが多い。	
\\	複数形チェック
\\	[副詞]	このごろは	
\\	近頃牛肉は高い。	
\\	[名詞]	医師、内科医	
\\	陰陽師身の上知らず。	
\\	複数形チェック
\\	[名詞]	核兵器	
\\	原子力は安全だ。	
\\	複数形チェック
\\	[名詞]	松	
\\	それは松の木です。	
\\	複数形チェック
\\	[名詞]	看護	
\\	ケアプランの作成等の居宅介護支援事業については、介護保険により負担されます。	
\\	複数形チェック
\\	[名詞]	開拓者、先駆者	
\\	彼はいわゆる先駆者だ。	
\\	複数形チェック
\\	[名詞]	オーク	
\\	楢です。	
\\	複数形チェック
\\	[名詞]	ピット、くぼみ	
\\	彼は奈落の底に落ちた。	
\\	複数形チェック
\\	[名詞]	オアシス	
\\	複数形チェック
\\	[動詞]	投げる	
\\	そのボール投げて。	
\\	[形容詞]	従順な	
\\	巷では「ガミガミ言わない子育て」が推奨されているが、それを実践するのはなかなか難しい。	
\\	[名詞]	設計者、計画者	
\\	複数形チェック
\\	[動詞]	従う	
\\	君自身に従え。	
\\	[名詞]	たくらみ、陰謀	
\\	筋書きは別として、その本は私をひきつけた。	
\\	複数形チェック
\\	[形容詞]	有毒な	
\\	ヘビの中には毒を持つものもある。	
\\	[副詞]	明らかに	
\\	当たり前だろ。	
\\	[名詞]	公害、汚染	
\\	汚染は抑えられます。	
\\	複数形チェック
\\	[形容詞]	時折の、時々の	
\\	スーパーヒーローにも、時には休息が必要だ。	
\\	[副詞]	貧しく、乏しく	
\\	冷夏のおかげで、ビールの売れ行きはよくなかった。	
\\	[副詞]	時々	
\\	彼らは時々私を訪ねて来た。	
\\	[名詞]	人気	
\\	彼女はその町での信望を得た。	
\\	複数形チェック
\\	[名詞]	職業	
\\	君の仕事は何ですか。	
\\	複数形チェック
\\	[名詞]	荷物運搬人、ポーター	
\\	私はボーイに、私のスーツケースを部屋まで運ばせた。	
\\	複数形チェック
\\	[形容詞]	油っこい	
\\	[名詞]	割り当て、取り分	
\\	トムはパイを少しもらった。	
\\	複数形チェック
\\	[名詞]	肖像画	
\\	壁には肖像画が飾ってあった。	
\\	複数形チェック
\\	[動詞]	思い出す	
\\	思いあたらないのですが。	
\\	[名詞]	受信機	
\\	複数形チェック
\\	[形容詞]	郵便の	
\\	[名詞]	受付	
\\	客間には誰もいない。	
\\	複数形チェック
\\	[名詞]	ポスター	
\\	このようなポスターを貼るには許可が必要ですし、だいいち街の美観を損ないます!	
\\	複数形チェック
\\	[名詞]	認識	
\\	私だとわかって彼女はにっこりとした。	
\\	複数形チェック
\\	[動詞]	延期する	
\\	ゲームは無期延期になった。	
\\	[名詞]	推薦、おすすめ	
\\	課長からもらった推薦状が物を言った。	
\\	複数形チェック
\\	[名詞]	可能性	
\\	もし臓器の販売を合法化したら、潜在的な健康問題が生じるかも知れません。	
\\	複数形チェック
\\	[名詞]	録音	
\\	録音が終わったらそれを再生して聞かせてください。	
\\	複数形チェック
\\	[名詞]	貧困	
\\	貧困が犯罪の源だという人もいる。	
\\	複数形チェック
\\	[名詞]	回復	
\\	早く回復しますように。	
\\	複数形チェック
\\	[動詞]	説く、説教する	
\\	彼の説教にはうんざりだ。	
\\	[名詞]	縮小、割引	
\\	彼は減税を唱えた。	
\\	複数形チェック
\\	[形容詞]	正確な	
\\	彼の答えは正確ではなかった。	
\\	[名詞]	参考、言及	
\\	匿名でも、詳細な情報を書いて頂いても、どちらでも結構です。	
\\	複数形チェック
\\	[名詞]	準備	
\\	準備は兄が全てしてくれた。	
\\	複数形チェック
\\	[前置詞]	に関して	
\\	私はあなたの提案に関していくつかの問題点を指摘したい。	
\\	[動詞]	保存する	
\\	冷蔵庫は食べ物を保存するのに役立つ。	
\\	[名詞]	部分、地方	
\\	この地方は貴重な鉱物を産出します。	
\\	複数形チェック
\\	[形容詞]	緊急の	
\\	組合は10パーセントの賃上げを迫っている。	
\\	[副詞]	定期的に	
\\	きちんと授業に出席すること。	
\\	[副詞]	以前は	
\\	しかしそこには山ほど未踏の問題があった。	
\\	[動詞]	喜ぶ	
\\	喜ぶ者と共によろこび、泣く者と共になけ。	
\\	[形容詞]	初頭の、原始的な	
\\	私の主たる関心事はあなたの身の安全なのです。	
\\	[形容詞]	関連がある	
\\	キュウリはスイカの親戚だ。	
\\	[形容詞]	始の、初期の	
\\	ひな祭りがいつ頃からある祭りなのか、はっきりとはしていないが、平安時代にはすでに流しびなの原形があったようだ。	
\\	[副詞]	相対的に	
\\	彼女は比較的早口だ。	
\\	[名詞]	優先	
\\	それが最優先事項です。	
\\	複数形チェック
\\	[形容詞]	くつろいだ	
\\	彼らはリラックスした。	
\\	[名詞]	プライバシー	
\\	プライバシーなんて存在しない。	
\\	複数形チェック
\\	[動詞]	言う	
\\	あなたのおっしゃった事に興味を持ちました。	
\\	[名詞]	特権、特典	
\\	我々はこの湾内で漁獲する特権を与えられた。	
\\	複数形チェック
\\	[形容詞]	注目すべき、驚くべき	
\\	近年における医学の進歩はめざましい。	
\\	[名詞]	処置、手順	
\\	手順を説明していただけますか。	
\\	複数形チェック
\\	[形容詞]	遠く離れた	
\\	リモコンを取ってくれますか?	
\\	[動詞]	進む	
\\	進め。	
\\	[動詞] [不規則動詞]	返金する	
\\	彼はお金を返したと思う。	
\\	[形容詞]	有益な、収益の高い	
\\	自国通貨建てで海外に投資することがいつも利益を生むとは限らない。	
\\	[動詞]	延⻑する	
\\	スグリは長寿に有効な果物だ。	
\\	[名詞]	代わりのもの	
\\	代わりと言っては何ですが、・・・私が、いささか、ご指南いたしましょう。	
\\	複数形チェック
\\	[動詞]	促進する、寄与する	
\\	彼女は昇格した。	
\\	[形容詞]	代表する、代理の	
\\	彼はわが社の代表として会議に参加しました。	
\\	[名詞]	昇進	
\\	彼は昇進の資格がある。	
\\	複数形チェック
\\	[名詞]	共和国	
\\	彼はすかさず共和国への次の攻撃を開始した。	
\\	複数形チェック
\\	[動詞]	発音する	
\\	それはどうやって発音するのですか?	
\\	[名詞]	評判、名声	
\\	お噂はかねがね伺っております。	
\\	複数形チェック
\\	[名詞]	発音	
\\	どうか発音で誤りがあったら直してください。	
\\	複数形チェック
\\	[名詞]	必要条件	
\\	友情も幸せに必要だってことでしょうか。	
\\	複数形チェック
\\	[名詞]	割合、釣合い	
\\	ロバートは利益のほんの一部しかもらわなかった。	
\\	複数形チェック
\\	[動詞]	救出する	
\\	私はあなたを助けた。	
\\	[名詞]	散文	
\\	複数形チェック
\\	[動詞]	似ている	
\\	彼は祖父似だ。	
\\	[名詞]	見通し、見込み	
\\	彼が回復する見込みはあるのだろうか。	
\\	複数形チェック
\\	[名詞]	住まい	
\\	在留届に変更があったり、未だに変更届を提出されていない方は、至急提出頂きますようお願いします。	
\\	複数形チェック
\\	[名詞]	繁栄	
\\	国家の繁栄は市民の手にかかっている。	
\\	複数形チェック
\\	[形容詞]	住み込みの	
\\	「居住外国人」とは、どういう意味ですか。	
\\	[副詞]	誇らしげに	
\\	[動詞]	抵抗する	
\\	誰も抵抗できない。	
\\	[名詞]	行政区画、範囲	
\\	それは私の領分ですよ。	
\\	複数形チェック
\\	[名詞]	抵抗力	
\\	チャールズはいつも一番容易な方法をとる。	
\\	複数形チェック
\\	[名詞]	出版、発行	
\\	本は現在出版の準備が出来ている。	
\\	複数形チェック
\\	[名詞]	決意、解決	
\\	事態は収拾に向かいつつある。	
\\	複数形チェック
\\	[副詞]	公然と	
\\	彼は、人前で自分はリーダーと言い切った。	
\\	[動詞]	決心する、決定する	
\\	一件落着!	
\\	[名詞]	子犬	
\\	犬が一匹欲しい。	
\\	複数形チェック
\\	[名詞]	行楽地	
\\	絶対に暴力に訴えてはいけません。	
\\	複数形チェック
\\	[動詞]	購入する	
\\	必要な物品を急いで購入しなさい。	
\\	[名詞]	資源、資金	
\\	これは世界の常識であり、資源管理の大原則だ。	
\\	複数形チェック
\\	[副詞]	まったく、単に	
\\	私がそれに気づいたのはほんの偶然に過ぎません。	
\\	[動詞]	修復する、回復する	
\\	1週間田舎にいてすっかり元気になったような気がする。	
\\	[動詞]	追う、追求する	
\\	人々はより多くの自由と平等を求める。	
\\	[動詞]	抑制する	
\\	彼は怒りを抑えようとした。	
\\	[名詞]	追撃、追求	
\\	猟犬たちはそのキツネを追いかけている。	
\\	複数形チェック
\\	[名詞]	抑制、自制	
\\	制度的に大きな枷となるのは、著作権の問題である。	
\\	複数形チェック
\\	[名詞]	量	
\\	一般的に、消費者は質よりも量を選ぶ。	
\\	複数形チェック
\\	[動詞]	保持する	
\\	[動詞] [不規則動詞]	やめる	
\\	私、辞めます。	
\\	[形容詞]	退職した	
\\	トムは引退した。	
\\	[形容詞]	急進的な、過激な	
\\	最も急進的な革命家が革命の翌日には保守になる。	
\\	[動詞]	明らかにする	
\\	メアリーは露出度の高い服をよく着ている。	
\\	[名詞]	激怒	
\\	彼はかんかんに怒っている。	
\\	複数形チェック
\\	[名詞]	復讐、復讐心	
\\	トムは報復を望んでいる。	
\\	複数形チェック
\\	[名詞]	評価	
\\	あなた同様私もヤフオクの評価が信じられません。	
\\	複数形チェック
\\	[名詞]	逆	
\\	アルファベットを逆から言ってください。	
\\	複数形チェック
\\	[動詞]	がらがらと鳴る	
\\	頭上で電車がガタガタ音を立てていた。	
\\	[動詞] [不規則動詞]	書き直す	
\\	この文章は書き換えなければならない。	
\\	[名詞]	かみそり	
\\	この電気かみそりは買う前に試してみるべきだった。	
\\	複数形チェック
\\	[名詞]	ライフル銃	
\\	トムは自分のライフルを撃った。	
\\	複数形チェック
\\	[副詞]	すぐに、快く	
\\	彼は進んで私の提案に賛成した。	
\\	[形容詞]	熟した	
\\	りんごが熟している。	
\\	[形容詞]	現実的な	
\\	私は現実的な人間だ。	
\\	[名詞]	決定、裁定	
\\	男尊女卑のどこがいけないんですか?	
\\	複数形チェック
\\	[名詞]	最後部、後方	
\\	彼の後ろにいる。	
\\	複数形チェック
\\	[形容詞]	田舎の	
\\	ニックは地方出身者ならだれであろうと見下している。	
\\	[動詞]	反対する	
\\	彼は反逆者だ。	
\\	[名詞]	いけにえ、犠牲	
\\	彼らはやぎを神への捧げ物として殺した。	
\\	複数形チェック
\\	[名詞]	悲しみ	
\\	彼は微笑みで悲しみを隠した。	
\\	複数形チェック
\\	[名詞]	軽食	
\\	別荘にもどって、三時のおやつ。	
\\	複数形チェック
\\	[名詞]	聖人、聖...	
\\	彼は聖者のようにふるまった。	
\\	複数形チェック
\\	[動詞]	パチンと鳴らす	
\\	彼はすぐキレる。	
\\	[名詞]	サケ、鮭	
\\	私の好きな魚は鮭です。	
\\	複数形チェック
\\	[名詞]	ソフトウェア	
\\	「ソフトウェア開発」は「モノ作り」ではない。	
\\	複数形チェック
\\	[形容詞]	塩気のある	
\\	この料理は塩がききすぎている。	
\\	[名詞]	答え、解決策	
\\	それは、解決ではありません。	
\\	複数形チェック
\\	[名詞]	満足、喜び	
\\	複数形チェック
\\	[副詞]	より早く	
\\	早ければ早いほどいい。	
\\	[形容詞]	満足した	
\\	だれも満足しなかった。	
\\	[名詞]	悲しみ、悲痛	
\\	彼の心は苦しみに満ちていた。	
\\	複数形チェック
\\	[名詞]	スキャンダル	
\\	彼はそのスキャンダルにかかわりがあることを認めている。	
\\	複数形チェック
\\	[名詞]	源、原因	
\\	楽あれば苦あり。	
\\	複数形チェック
\\	[名詞]	南東、南東部	
\\	東南アジアのインフラ整備で日本が中心的役割を担うため、官民一体となった努力が求められている。	
\\	複数形チェック
\\	[動詞]	怖がらせる	
\\	私、怖い!	
\\	[名詞]	南⻄、南⻄部	
\\	複数形チェック
\\	[形容詞]	おびえた	
\\	私、怖い!	
\\	[名詞]	すき	
\\	トムは何でもずばずば言う人だ。	
\\	複数形チェック
\\	[動詞]	まき散らす	
\\	彼らは四方八方に散らばった。	
\\	[名詞]	火花	
\\	複数形チェック
\\	[名詞]	景色	
\\	彼らはその美しい景色にうっとりした。	
\\	複数形チェック
\\	[名詞]	香り	
\\	猟犬たちはキツネの匂いの跡をたどった。	
\\	複数形チェック
\\	[名詞]	光景、美観、壮観	
\\	奇妙な光景を目にする。	
\\	複数形チェック
\\	[名詞]	基本構想、計画	
\\	誰も世界を完成させる案を持っている。	
\\	複数形チェック
\\	[動詞] [不規則動詞]	こぼす、もらす	
\\	トムは牛乳をこぼした。	
\\	[名詞]	学者	
\\	彼はいわゆる学者である。	
\\	複数形チェック
\\	[形容詞]	精神の	
\\	これに対して霊的なことがらに関する熱心な会話は、霊的な進歩に大いなる助けとなります。	
\\	[名詞]	学問、奨学金	
\\	この奨学金のおかげで彼は留学できた。	
\\	複数形チェック
\\	[名詞] [動詞]	悪意、意地悪、当てつける	
\\	雨にもかかわらず、私は外出した。	
\\	複数形チェック
\\	[動詞]	叱る	
\\	彼は彼女をしかった。	
\\	[形容詞]	素晴らしい	
\\	素晴らしきかな、この世界。	
\\	[名詞]	偵察、スカウト	
\\	複数形チェック
\\	[名詞]	代弁者、スポークスマン	
\\	スポークスマンはその報告は真実であると確認した。	
\\	複数形チェック
\\	[動詞]	ひっかく	
\\	彼女は頭を掻いた。	
\\	[名詞]	スポンジ	
\\	スポンジは液体を吸い取る。	
\\	複数形チェック
\\	[名詞]	封印	
\\	ここで印鑑を押して下さい。	
\\	複数形チェック
\\	[名詞]	スポンサー	
\\	この番組はご覧のスポンサーの提供で送りしました。	
\\	複数形チェック
\\	[形容詞]	綿密な、徹底的な	
\\	その少年は無くした鍵を探していた。	
\\	[名詞]	さじ一杯	
\\	複数形チェック
\\	[形容詞]	第二次的なもの	
\\	この問題は副次的な重要性を持つに過ぎない。	
\\	[動詞]	スプレーを吹きかける	
\\	彼女は花に殺虫剤を振りまいた。	
\\	[副詞]	ひそかに	
\\	彼は僕に彼女の写真をこっそり見せてくれた。	
\\	[動詞]	圧搾する	
\\	トムはメアリーの手を握りしめた。	
\\	[名詞]	部門、分野	
\\	複数形チェック
\\	[形容詞]	安定した	
\\	このところ物価が安定している。	
\\	[動詞]	安全を確保する	
\\	離着陸の際にはシートベルトをしめて下さい。	
\\	[名詞]	利害関係	
\\	大きな賭けだ。	
\\	複数形チェック
\\	[動詞]	つかむ	
\\	彼は恐怖におびえた。	
\\	[名詞]	小屋、露店	
\\	届け出なしに路上に出店してはならない。	
\\	複数形チェック
\\	[名詞]	大評判	
\\	そのニュースは大評判を巻き起こした。	
\\	複数形チェック
\\	[動詞]	じっと見つめる	
\\	彼は私をにらんだ。	
\\	[名詞]	開会、開廷	
\\	彼等は貿易問題について臨時の会合を開いた。	
\\	複数形チェック
\\	[動詞]	飢える	
\\	おなか空いた!	
\\	[名詞]	解決、和解	
\\	両国は危機解決に向けて交渉をするでしょう。	
\\	複数形チェック
\\	[名詞]	地位、身分	
\\	プロジェクトの経過は下記の通りです。	
\\	複数形チェック
\\	[形容詞]	激しい、深刻な	
\\	彼は子供達に厳しかった。	
\\	[名詞]	蒸気船	
\\	運よく汽船が通りかかって彼らは救助された。	
\\	複数形チェック
\\	[動詞] [不規則動詞]	縫う	
\\	彼女は裁縫が上手です。	
\\	[形容詞]	険しい	
\\	我々の登山は厳しいだろう。	
\\	[形容詞]	浅い	
\\	この川は浅いですか?	
\\	[名詞]	茎(くき)	
\\	彼はワイングラスを落として、その足を折ってしまった。	
\\	複数形チェック
\\	[名詞]	避難所、隠れ場	
\\	彼らは雨宿りをする場所を捜した。	
\\	複数形チェック
\\	[名詞]	シチュー	
\\	トムは夕食にシチューを作った。	
\\	複数形チェック
\\	[名詞]	羊飼い	
\\	羊飼いはいつも羊に「羊の利害と自分の利害は同じだ」と説得しようとしている。	
\\	複数形チェック
\\	[形容詞]	ねばねばした	
\\	体中が汗でベタベタだ。早く風呂に入ってさっぱりしたいよ。	
\\	[動詞]	保護する、かばう	
\\	[動詞] [不規則動詞]	刺す	
\\	良心が彼を苦しめた。	
\\	[名詞]	変化、変更	
\\	トムは夜勤で働いている。	
\\	複数形チェック
\\	[動詞]	かき回す	
\\	スープをかき混ぜなさい。	
\\	[形容詞]	衝撃を受けた	
\\	トムは驚いた。	
\\	[形容詞]	暴風の、激しい	
\\	彼が死んだ夜は嵐だった。	
\\	[名詞]	不足	
\\	この地域では水が欠乏している。	
\\	複数形チェック
\\	[名詞]	緊張	
\\	綱は張りすぎて切れた。	
\\	複数形チェック
\\	[副詞]	間もなく	
\\	すぐに戻ります。	
\\	[副詞]	妙に、変に	
\\	妙な話だが、私は何の痛みも感じなかった。	
\\	[名詞]	歩道	
\\	植木鉢が歩道に落ちてガチャンと割れた。	
\\	複数形チェック
\\	[副詞]	厳格に	
\\	これは厳密にいえば間違えである。	
\\	[形容詞]	重要な、意義深い	
\\	恋人は外国語学校に勤めていて、その勤務がとても好きようだ。	
\\	[形容詞]	目立つ	
\\	[名詞]	細⻑いきれ	
\\	複数形チェック
\\	[名詞]	現場、場所	
\\	このサイトをブックマークしておいて。	
\\	複数形チェック
\\	[副詞]	強く	
\\	風がまだ強く吹きます。	
\\	[名詞]	スキーヤー	
\\	両親はスキーヤー用のペンションを経営しています。	
\\	複数形チェック
\\	[名詞]	スタイリスト	
\\	複数形チェック
\\	[形容詞]	熟練した	
\\	[動詞]	提出する	
\\	彼は辞表を提出することに決めた。	
\\	[動詞]	跳ねる	
\\	やめておけ。	
\\	[名詞]	物質	
\\	一酸化炭素とは炭素化合物の不完全燃焼で発生する有害物質です。	
\\	複数形チェック
\\	[名詞]	袖	
\\	私は彼のそでをつかんだ。	
\\	複数形チェック
\\	[名詞]	郊外、近郊	
\\	彼らはロンドン郊外に居を構えることに決めた。	
\\	複数形チェック
\\	[形容詞]	ほっそりした	
\\	あの子供は痩せている。	
\\	[名詞]	頂、首脳会談	
\\	大輔は頂上に登った。	
\\	複数形チェック
\\	[名詞]	スリッパ	
\\	スリッパ、よこせえ! コラ! よこせと云うに!	
\\	複数形チェック
\\	[名詞]	ヒマワリ	
\\	それはひまわりです。	
\\	複数形チェック
\\	[動詞]	粉砕する	
\\	瓶は粉々に割れた。	
\\	[名詞]	降参、降伏	
\\	降参します。	
\\	複数形チェック
\\	[形容詞]	周囲の	
\\	[形容詞]	不親切な	
\\	友達に意地悪をしてはいけません。	
\\	[動詞]	調査する	
\\	料金には調査をするのに必要な専門的な仕事に対する支払いも含まれています。	
\\	[形容詞]	ありそうもない	
\\	それはありそうにないな。	
\\	[形容詞]	無益な、不必要な	
\\	その必要はありません。	
\\	[形容詞]	同情的な	
\\	トムって思いやりがあると思う。	
\\	[形容詞]	まっすぐ立った	
\\	ない袖は振れぬ。	
\\	[名詞]	大さじ一杯	
\\	複数形チェック
\\	[形容詞]	上向きの	
\\	たこは上へ上っていった。	
\\	[形容詞]	飼い慣らされた	
\\	狼を飼い慣らすことはできません。	
\\	[動詞]	促す、勧める	
\\	トムは悲鳴をぐっと堪えた。	
\\	[名詞]	仕事	
\\	彼はその仕事を承知した。	
\\	複数形チェック
\\	[形容詞]	緊急の	
\\	至急お願いします。	
\\	[形容詞]	科学技術の	
\\	[形容詞]	うぬぼれの強い、虚栄心の強い、むだな	
\\	我々の努力はすべてむだだった。	
\\	[形容詞]	多くを語る、意味あり気な	
\\	まったくその通りです。	
\\	[名詞]	ワゴン車、バン	
\\	彼はゴッホの模写をした。	
\\	複数形チェック
\\	[名詞]	誘惑	
\\	彼は誘惑に負けてしまった。	
\\	複数形チェック
\\	[名詞]	バニラ	
\\	複数形チェック
\\	[名詞]	傾向	
\\	トムは早合点する癖がある。	
\\	複数形チェック
\\	[動詞]	消える	
\\	彼らは姿を消した。	
\\	[形容詞]	やわらかい	
\\	この牛肉は柔らかい。	
\\	[名詞]	変化	
\\	複数形チェック
\\	[名詞]	時制	
\\	この文の時制は現在です。	
\\	複数形チェック
\\	[動詞]	変化する	
\\	トムは多趣味だ。	
\\	[名詞]	緊張	
\\	冷戦の緊張が高まった。	
\\	複数形チェック
\\	[名詞]	冒険的事業、投機的企業	
\\	宇宙空間に冒険してどんな地球の人々は十分好奇心がありますか?	
\\	複数形チェック
\\	[名詞]	領土、地域	
\\	彼らは領土の5分の1を支配していた。	
\\	複数形チェック
\\	[名詞]	韻文	
\\	彼はその一節を英語に翻訳した。	
\\	複数形チェック
\\	[名詞]	恐怖、恐ろしさ	
\\	彼女は恐怖で悲鳴をあげた。	
\\	複数形チェック
\\	[名詞]	版、バージョン	
\\	最新のバージョンを持っていますか。	
\\	複数形チェック
\\	[副詞]	厚く	
\\	[名詞]	船	
\\	彼の脳内で血管が破裂した。	
\\	複数形チェック
\\	[名詞]	喉の渇き	
\\	のどがかわいて死にそうです。	
\\	複数形チェック
\\	[前置詞]	〜を通って	
\\	彼はパリ経由でロンドンへ行った。	
\\	[形容詞]	徹底的な、完全な	
\\	彼はその問題を徹底的に分析した。	
\\	[名詞]	悪徳	
\\	誰もが知っているが、彼は彼女を好きで、彼女も彼が好きだ。	
\\	複数形チェック
\\	[副詞]	完全に	
\\	私はそれを徹底的に調べた。	
\\	[名詞]	村人	
\\	そのボランティアグループは村人に水を提供した。	
\\	複数形チェック
\\	[形容詞]	千番目の	
\\	[名詞]	美徳、徳	
\\	正直は美徳のひとつです。	
\\	複数形チェック
\\	[動詞]	わくわくさせる、感動させる	
\\	トムは感激するだろう。	
\\	[形容詞]	見える	
\\	モロ見えです。	
\\	[名詞]	潮、潮流	
\\	とはどういう意味ですか。	
\\	複数形チェック
\\	[名詞]	ボランティア	
\\	ボランティア活動をしたことがありますか。	
\\	複数形チェック
\\	[形容詞]	きちんとした	
\\	部屋の中をきちんと整理しなさい。	
\\	[名詞]	航海、船旅	
\\	タイタニック号は処女航海で沈没した。	
\\	複数形チェック
\\	[形容詞]	時宜を得た	
\\	[動詞]	歩き回る、ぶらつく	
\\	話がそれた。	
\\	[名詞]	しるし	
\\	感謝のしるしにこれをさし上げたいと存じます。	
\\	複数形チェック
\\	[副詞]	熱心に、熱烈に	
\\	生暖かい目で見てやって下さい。	
\\	[名詞]	墓	
\\	愛を取り去れ、そうしたら、我らの地球は墓となる。	
\\	複数形チェック
\\	[名詞]	暖かさ	
\\	彼の腕の温かさを思い出す。	
\\	複数形チェック
\\	[名詞]	跡、痕跡	
\\	彼は跡形もなく姿を消してしまったんだ。	
\\	複数形チェック
\\	[動詞]	弱る、衰弱する	
\\	嗅覚が鈍っています。	
\\	[名詞]	牽引車、トラクター	
\\	僕の夢はランボルギーニトラクターを買うことなんです。	
\\	複数形チェック
\\	[動詞] [不規則動詞]	織る、編む	
\\	たいていの蜘蛛はクモの巣を張る。	
\\	[名詞]	跡、痕跡	
\\	複数形チェック
\\	[名詞]	くもの巣、編物	
\\	お前の好きなインターネット閲覧ソフトはなんだ。	
\\	複数形チェック
\\	[名詞]	訓練する人	
\\	複数形チェック
\\	[名詞]	雑草	
\\	雑草とはなんだろう。その美質をいまだ見出されていない植物である。	
\\	複数形チェック
\\	[名詞]	翻訳	
\\	この訳合ってますか?	
\\	複数形チェック
\\	[名詞]	福祉、生活保護	
\\	車も何もなく、生活保護で生きてます。	
\\	複数形チェック
\\	[名詞]	輸送、交通	
\\	この町は交通の便がすごくいい。	
\\	複数形チェック
\\	[動詞]	むち打つ	
\\	囚人は鞭撻して答えた。	
\\	[名詞]	待遇、扱い	
\\	彼は治療を断られた。	
\\	複数形チェック
\\	[動詞]	ひそひそと話す	
\\	「黙って」と彼はささやいた。	
\\	[名詞]	条約	
\\	その条約は調印された。	
\\	複数形チェック
\\	[形容詞]	邪悪な、不道徳な、不正な	
\\	昔、イングランドにたいへん意地の悪い王様が住んでいた。	
\\	[名詞]	部族	
\\	彼は35年間自分の部族の酋長だった。	
\\	複数形チェック
\\	[名詞]	幅	
\\	マイクロファイバークロスは、髪の毛の100分の1という細さの繊維で編みあげられたクロスです。	
\\	複数形チェック
\\	[動詞]	3倍にする	
\\	トムは三塁打を打った。	
\\	[名詞]	原野、荒野	
\\	複数形チェック
\\	[名詞]	勝利、大成功	
\\	真実だけが勝利する。	
\\	複数形チェック
\\	[名詞]	野生生物	
\\	市内の川に禁猟区があって、カモ類を見るならそこがいいです。	
\\	複数形チェック
\\	[形容詞]	熱帯の	
\\	トムは熱帯魚を飼っている。	
\\	[副詞]	喜んで	
\\	進んでそれをやりました。	
\\	[名詞]	回転、曲がり角	
\\	彼女は赤くなる。	
\\	複数形チェック
\\	[形容詞]	風の強い	
\\	風が強いな。	
\\	[名詞]	家庭教師	
\\	最近塾の講師を始めた。主に数学と英語の個別指導をしている。	
\\	複数形チェック
\\	[動詞]	ウインクをする	
\\	トムはウィンクした。	
\\	[形容詞]	不確実な	
\\	彼は自分の将来について自信がない。	
\\	[名詞]	知恵、賢明さ	
\\	彼は知恵のある人だ。	
\\	複数形チェック
\\	[形容詞]	はっきりしない	
\\	方向性がはっきりしない。	
\\	[形容詞]	擦り切れた、使い古しの	
\\	疲れたなあ。	
\\	[形容詞]	心地の悪い	
\\	この椅子は座り心地が悪い。	
\\	[形容詞]	心配な、やっかいな	
\\	何が君を苦しめているのだ?	
\\	[動詞]	強調する	
\\	[副詞]	疑いようもなく	
\\	英語は明らかに、最も簡単であると同時に最も効率の良い国際コミュニケーションの手段です。	
\\	[動詞]	あくびをする	
\\	トムはあくびした。	
\\	[副詞]	残念ながら	
\\	残念ながらそれは本当なのです。	
\\	[形容詞]	年一回の、毎年の	
\\	[動詞]	一体となる	
\\	万国の労働者よ。団結せよ!	
\\	[動詞]	産する	
\\	ついに彼は屈服した。	
\\	[形容詞]	結合した、団結した、一致協力した	
\\	は[国連]を表します。	
\\	[名詞]	子供、若者	
\\	まだ幼い時に歌を始めました。	
\\	複数形チェック
\\	[形容詞]	一般的な、普遍的な	
\\	音楽は世界の共通言語だ。	
\\	[形容詞]	若々しい	
\\	[動詞]	捨て去る	
\\	彼は社会主義から転向した。	
\\	[名詞]	集会、会合	
\\	25歳以上のすべてのチベット人は亡命チベット代表者議会の議員選挙に立候補する権利を有する。	
\\	複数形チェック
\\	[形容詞]	見捨てられた	
\\	彼は社会主義から転向した。	
\\	[動詞]	割り当てる	
\\	私たちは大きな教室を割り当てられた。	
\\	[動詞]	廃止する	
\\	大統領は奴隷制度を廃止した。	
\\	[名詞]	課題	
\\	その課題の提出期限は今日から2週間以内です。	
\\	複数形チェック
\\	[名詞]	道理に反した	
\\	ばかげたことを言うな。	
\\	複数形チェック
\\	[名詞]	想定、憶測	
\\	複数形チェック
\\	[名詞]	豊富	
\\	そこは鉄鉱に富む。	
\\	複数形チェック
\\	[名詞]	保証	
\\	この通話は、サービス向上およびトレーニングのためにモニターまたは録音されることがあります。ご用件は何でしょう?	
\\	複数形チェック
\\	[形容詞]	豊富な	
\\	彼女は豊かな髪をしている。	
\\	[動詞]	驚かす	
\\	驚きの余り話す力がなくなってしまった。	
\\	[名詞]	学会、王立美術院	
\\	黒魔法の話を聞きたいです。学院ではさわりしか聞いたことがないんですが興味があります。	
\\	複数形チェック
\\	[形容詞]	驚くべき	
\\	[形容詞]	偶然の	
\\	[動詞]	達成する	
\\	彼は目的を達成した。	
\\	[名詞]	宿泊設備	
\\	複数形チェック
\\	[名詞]	接客係、付人	
\\	「従者」の意味だったのである。	
\\	複数形チェック
\\	[形容詞]	成就した	
\\	彼は任務を成し遂げた。	
\\	[名詞]	自動車	
\\	ホワイトバランスをオートにした場合と白いコピー用紙でプリセットした場合の比較です。	
\\	複数形チェック
\\	[名詞]	完成、成就	
\\	リンドバーグの大西洋横断無着陸単独飛行はめざましい偉業であった。	
\\	複数形チェック
\\	[名詞]	自覚すること	
\\	政治的意識を持つようになるには時間がかかる。	
\\	複数形チェック
\\	[副詞]	それに応じて	
\\	ワインにはいろいろなタイプがあり、それによって育成の仕方もさまざまです。	
\\	[名詞]	畏怖	
\\	深い畏敬の念を抱きながら、私は燭台を前あった場所に戻した。	
\\	複数形チェック
\\	[副詞]	正確に	
\\	一遍で正確に答えてください。	
\\	[形容詞]	ぎこちない	
\\	彼女は不器用である。	
\\	[形容詞]	告発された	
\\	トムは濡れ衣を着せられた。	
\\	[名詞]	バレエ	
\\	バレエ続けてればよかった。	
\\	複数形チェック
\\	[動詞]	慣らす	
\\	彼女は座ることに慣れている。	
\\	[名詞]	ボールペン	
\\	ボールペンで書いてください。	
\\	複数形チェック
\\	[形容詞]	慣れた	
\\	彼女は座ることに慣れている。	
\\	[名詞]	強打、ごう音	
\\	ダイナマイトがドカンと爆発した。	
\\	複数形チェック
\\	[動詞]	知らせる、熟知させる	
\\	私は彼と知り合いです。	
\\	[名詞]	納屋	
\\	嵐を避けて近くの納屋に逃げ込んだ。	
\\	複数形チェック
\\	[名詞]	買収、入手(したもの)	
\\	彼は会社にとって重要な人材だ。	
\\	複数形チェック
\\	[名詞]	妨げるもの	
\\	駅員に連絡したが、男はその場を立ち去り、改札口を出て行った。	
\\	複数形チェック
\\	[形容詞]	鋭い、急性の	
\\	彼は急性リンパ性白血病で死亡しました。	
\\	[形容詞]	打ち負かされた	
\\	トムは打ち負かされた。	
\\	[名詞]	冒険家	
\\	君みたいによく冒険してたけど、ある時矢で膝を怪我したんだ。	
\\	複数形チェック
\\	[名詞]	鼓動	
\\	まわりくどいことをいうな。	
\\	複数形チェック
\\	[形容詞]	冒険好きな	
\\	[名詞]	カブトムシ	
\\	「こちらがダイオウグソクムシのパフェになります」「オエッ。こんなの頼んだ覚えないよ」	
\\	複数形チェック
\\	[動詞]	広告を出す	
\\	トムは広告代理店を立ち上げた。	
\\	[名詞]	援助、支持、利益	
\\	彼は自分の利益のために行動している。	
\\	複数形チェック
\\	[名詞]	広告	
\\	トムは広告代理店を立ち上げた。	
\\	複数形チェック
\\	[名詞]	腹部	
\\	複数形チェック
\\	[形容詞]	攻撃的な、強引な	
\\	彼女は気が強い。	
\\	[名詞]	所有物、所属	
\\	日歯連闇献金事件とは日本歯科医師連盟が自由民主党所属の国会議員に闇献金した事件。	
\\	複数形チェック
\\	[名詞]	苦悩	
\\	複数形チェック
\\	[形容詞]	気持ちのよい	
\\	ヘレンはパーティーでお世辞を振りまいた。	
\\	[形容詞]	生物学的な	
\\	[名詞]	通路	
\\	通路側には座れる。	
\\	複数形チェック
\\	[名詞]	生物学者	
\\	彼は生物学者だ。	
\\	複数形チェック
\\	[名詞]	警戒警報	
\\	津波警報は解除されました。	
\\	複数形チェック
\\	[名詞]	生物学	
\\	生物学は好きになれません。	
\\	複数形チェック
\\	[名詞]	提携、同盟	
\\	戦争の直前、日本がフランスと同盟を結びました。	
\\	複数形チェック
\\	[名詞]	火事、炎	
\\	複数形チェック
\\	[名詞]	アリゲーター	
\\	ワニが犬を食べた。	
\\	複数形チェック
\\	[名詞]	ブロンドの人	
\\	彼は金髪だ。	
\\	複数形チェック
\\	[名詞]	手当、小遣い	
\\	トムは親の仕送りで生活している。	
\\	複数形チェック
\\	[名詞]	ブルーベリー	
\\	ブルーベリーをとった後、タルトを作る。	
\\	複数形チェック
\\	[名詞]	大使、使節	
\\	大使は戻りました。	
\\	複数形チェック
\\	[名詞]	退屈	
\\	自閉症の子供は飽きることを知らない。	
\\	複数形チェック
\\	[名詞]	アニメーション	
\\	ふたりはアニメーションの専門学校で知り合った。	
\\	複数形チェック
\\	[名詞]	借り手	
\\	金利は、債務者が負担する事業リスクに応じて定まるものと存じます。	
\\	複数形チェック
\\	[動詞]	いらいらさせる	
\\	彼はうっとうしいやつだ。	
\\	[名詞]	少年時代	
\\	複数形チェック
\\	[名詞]	アンテナ	
\\	その男はアンテナを取り付けようとした。	
\\	複数形チェック
\\	[名詞]	ブラジャー	
\\	複数形チェック
\\	[動詞]	予想する、予期する	
\\	彼が来るのを心待ちにしている。	
\\	[名詞]	ブレスレット	
\\	複数形チェック
\\	[名詞]	器具、装置	
\\	複数形チェック
\\	[副詞]	勇敢に	
\\	彼女はその痛みを立派に耐えた。	
\\	[名詞]	感謝	
\\	感謝のしるしにこれをさし上げたいと存じます。	
\\	複数形チェック
\\	[名詞]	幅	
\\	誰しもこの深淵の深さや広さを言い当てることはできないであろう。	
\\	複数形チェック
\\	[形容詞]	おおよそ、約	
\\	この市の人口は何人ぐらいですか?	
\\	[名詞]	破壊者	
\\	ブレーカーが落ちて真っ暗闇になった時、赤ちゃんが怖くて泣きだしました。	
\\	複数形チェック
\\	[名詞]	アーケード	
\\	複数形チェック
\\	[名詞]	呼吸	
\\	呼吸は?	
\\	複数形チェック
\\	[形容詞]	アーチ形の	
\\	[名詞]	縁	
\\	複数形チェック
\\	[名詞]	建築家	
\\	逆境にもかかわらず、その建築家は世界的名声を獲得した。	
\\	複数形チェック
\\	[名詞]	放送	
\\	複数形チェック
\\	[名詞]	建築、建築学	
\\	64ビットのソフトは32ビット環境では動かないだろう。	
\\	複数形チェック
\\	[名詞]	闘技場、舞台	
\\	複数形チェック
\\	[名詞]	銅、⻘銅	
\\	青銅器は主として銅と錫の合金で出来ている。	
\\	複数形チェック
\\	[名詞]	算数	
\\	算数では正確さが重要だ。	
\\	複数形チェック
\\	[形容詞]	残虐な	
\\	[形容詞]	武装した	
\\	彼らはライフルで武装していた。	
\\	[名詞]	欠陥、誤り	
\\	どの虫があなたを傷つけるんですか。	
\\	複数形チェック
\\	[名詞]	よろい	
\\	複数形チェック
\\	[名詞]	電球	
\\	トマス・エジソンが電球を発明した。	
\\	複数形チェック
\\	[動詞]	登る	
\\	[名詞]	大部分	
\\	トムは給料の大半を日本の美少女フィギュアにつぎ込んでいる。	
\\	複数形チェック
\\	[名詞]	アスピリン	
\\	鎮痛剤を飲みなさい。	
\\	複数形チェック
\\	[名詞]	こぶ	
\\	宇宙飛行士は重力を感じないので、どんな方向にも寝ることができる。しかし、体が浮き上がって何かに衝突してしまわないように自分の身を固定しなくてはいけない。	
\\	複数形チェック
\\	[動詞]	組み立てる	
\\	配管工は配管系の設計・組立を行う。	
\\	[名詞]	束	
\\	彼は古い手紙を束にした。	
\\	複数形チェック
\\	[名詞]	局、事務局	
\\	彼女は政府機関に勤めているのでしょう?	
\\	複数形チェック
\\	[名詞]	共産主義者	
\\	喜んで中国共産党に入ります。	
\\	複数形チェック
\\	[名詞]	強盗	
\\	その泥棒は夜更かしすることに慣れている。	
\\	複数形チェック
\\	[形容詞]	競争的な	
\\	トムとメアリーは二人ともとても競争的だ。	
\\	[形容詞]	燃えるように	
\\	何かが焦げてる匂いがする。	
\\	[動詞]	構成する	
\\	彼はすばらしい詩を作る。	
\\	[名詞]	計算	
\\	複数形チェック
\\	[名詞]	理解、理解力	
\\	彼の小説は私には理解できない。	
\\	複数形チェック
\\	[名詞]	子牛	
\\	トムさんは仔牛に焼き印を押しました。	
\\	複数形チェック
\\	[名詞]	妥協	
\\	私は、彼らが平和的和解に達する事を望む。	
\\	複数形チェック
\\	[名詞]	カロリー	
\\	高カロリーの食べ物は私の健康に悪い。	
\\	複数形チェック
\\	[動詞]	隠す	
\\	トムは身元を隠すためにマスクをしていた。	
\\	[名詞]	候補者	
\\	果たして、ゴア氏は大統領候補として立つのか?	
\\	複数形チェック
\\	[動詞]	考える	
\\	彼らは敵をびっくりさせる計画を思いついた。	
\\	[名詞]	粗布、画布	
\\	彼は絵筆をカンバスにさっと走らせた。	
\\	複数形チェック
\\	[形容詞]	集中した	
\\	トムは彼の仕事に専念した。	
\\	[名詞]	岬	
\\	木曜島はオーストラリア最北端ヨーク岬とニューギニア島の間のトレス海峡に位置する。	
\\	複数形チェック
\\	[名詞]	概念	
\\	数学教師が偏微分の概念について説明した。	
\\	複数形チェック
\\	[名詞]	資本主義	
\\	資本主義とは人間による人間の搾取である。共産主義はその真逆だ。	
\\	複数形チェック
\\	[形容詞]	具体的な	
\\	コンクリート桟橋なのですが、途中何カ所か崩壊しています。	
\\	[名詞]	資本家	
\\	複数形チェック
\\	[動詞]	非難する	
\\	彼は死刑を宣告された。	
\\	[副詞]	不注意に	
\\	ぼくはうっかり電柱にぶつかった。	
\\	[動詞]	凝縮する	
\\	[形容詞]	世話をする、福祉の	
\\	君には心配してくれる友達がたくさんいる。	
\\	[名詞]	車掌、指揮者	
\\	バレエ音楽の指揮者として彼にかなうものはいない。	
\\	複数形チェック
\\	[動詞]	彫刻する、(肉を)切り分ける	
\\	彼は木から仏像を彫った。	
\\	[動詞]	制限する.	
\\	彼は床に伏せっている。	
\\	[名詞]	捕手、キャッチャー	
\\	トムはキャッチャーです。	
\\	複数形チェック
\\	[動詞]	直面する、対決する	
\\	[名詞]	聖堂	
\\	複数形チェック
\\	[動詞]	混乱させる	
\\	彼は混乱しているように見える。	
\\	[形容詞]	キリスト教徒の	
\\	カトリック教会は離婚に反対している。	
\\	[形容詞]	混乱した	
\\	彼は混乱しているように見える。	
\\	[形容詞]	慎重な	
\\	[形容詞]	紛らわしい	
\\	この文法の説明、ややこしくてよくわからないよ。	
\\	[名詞]	中断	
\\	喧嘩はやめてください。	
\\	複数形チェック
\\	[名詞]	混乱	
\\	その混乱実に名状すべからず。	
\\	複数形チェック
\\	[名詞]	穀物、穀草	
\\	複数形チェック
\\	[名詞]	祝賀	
\\	おめでとうございます。	
\\	複数形チェック
\\	[形容詞]	困難な	
\\	私の出典に異議があるのですか?	
\\	[名詞]	征服者	
\\	複数形チェック
\\	[名詞]	部屋	
\\	憲法の規定に従い衆院の議決が参院に優越する。	
\\	複数形チェック
\\	[名詞]	征服	
\\	複数形チェック
\\	[名詞]	選手権大会	
\\	私は彼と優勝を争った。	
\\	複数形チェック
\\	[名詞]	一致、同意	
\\	複数形チェック
\\	[名詞]	大混乱	
\\	小さな子供のいる家庭は、毎日がどたばたして大変だと思う。	
\\	複数形チェック
\\	[名詞]	同意	
\\	沈黙は同意を表す。	
\\	複数形チェック
\\	[形容詞]	魅惑的な	
\\	彼女は魅力的な女性だ。	
\\	[副詞]	結果として	
\\	[名詞]	憲章、国連憲章	
\\	複数形チェック
\\	[名詞]	保護、維持	
\\	複数形チェック
\\	[名詞]	おしゃべり	
\\	彼は20歳になったのに、まだ女の子とおしゃべりすることに臆病だ。	
\\	複数形チェック
\\	[形容詞]	保守的な	
\\	トムは保守的だった。	
\\	[名詞]	かけら	
\\	ボブは親にそっくりね。	
\\	複数形チェック
\\	[前置詞]	考慮すると	
\\	この問題は考慮する価値がある。	
\\	[名詞]	コーラス	
\\	彼らは声をそろえて言った。	
\\	複数形チェック
\\	[形容詞]	一貫した	
\\	一貫した行動指針というものが無い人もいます。彼らはただ、その時々で自分に都合の良い指針があれば、何でも受け入れるのです。	
\\	[名詞]	キリスト教	
\\	キリスト教とイスラム教は異なった2つの宗教だ。	
\\	複数形チェック
\\	[動詞]	構成する	
\\	北朝鮮が6か国協議の合意に基づき核開発計画を申告した26日、米国が「テロ支援国」の指定解除手続きに入ったことで、拉致被害者の家族らには「拉致問題が置き去りにされるのでは」という不安が広がった。	
\\	[名詞]	リンゴジュース	
\\	林檎酒が大好きです。	
\\	複数形チェック
\\	[動詞]	相談する	
\\	トムさんは技術顧問に相談したほうがいい。	
\\	[形容詞]	円の	
\\	循環論法すれば賞賛を浴るのは哲学だけです。	
\\	[名詞]	コンサルタント	
\\	僕はコンサルタントとして働いている。	
\\	複数形チェック
\\	[名詞]	⺠間人	
\\	私は一般人です。	
\\	複数形チェック
\\	[動詞]	消費する	
\\	その火事で家は全焼した。	
\\	[名詞]	文明	
\\	愛は文明の奇跡である。	
\\	複数形チェック
\\	[名詞]	消費者	
\\	米消費者団体が
\\	を「バッドウェア」と認定。	
\\	複数形チェック
\\	[動詞]	文明化する	
\\	税金は文明社会の代償である。	
\\	[名詞]	消費	
\\	私は肉を食べるのを控えた。	
\\	複数形チェック
\\	[名詞]	粘土	
\\	複数形チェック
\\	[名詞]	同時代のもの	
\\	漱石は鴎外と同時代の人であった。	
\\	複数形チェック
\\	[名詞]	カチッという音	
\\	またアフィ狙いの記事か。	
\\	複数形チェック
\\	[名詞]	軽蔑、侮辱罪	
\\	かつて私のラテン語教師は、メガネの縁からいかめしく私を見下していた。でも今は、それはただ、彼女が読書用のメガネをかけていたので、それをしょっちゅう外す事をやっかいに思ったのだと分かる。つまり、私たち生徒に向けてまるで蔑んでいるように見えていたものは、どうやら実際は優しさだったのである。	
\\	複数形チェック
\\	[名詞]	絶頂	
\\	複数形チェック
\\	[形容詞]	連続的な	
\\	あらゆる点で微分可能な関数は連続関数です。	
\\	[名詞]	クローバー	
\\	四つ葉のクローバーを見つけたことがありますか。	
\\	複数形チェック
\\	[動詞]	寄付する、貢献する	
\\	彼は大金を寄付した。	
\\	[名詞]	手がかり	
\\	杳として行方が知れない。	
\\	複数形チェック
\\	[名詞]	改宗者	
\\	変換にはかなりの時間がかかる。	
\\	複数形チェック
\\	[名詞]	ココナツ	
\\	ココナッツミルクが飲みたい。	
\\	複数形チェック
\\	[動詞]	伝える、伝達する	
\\	基本的に翻訳は訓練してないから出来ないとは言いつつ、正しい翻訳よりも例文として自然な表現を目指すという目標が掲げられると、これはなかなか病みつきになる体験かもしれない。	
\\	[名詞]	喫茶店	
\\	複数形チェック
\\	[名詞]	確信、信念	
\\	ベティは窃盗の前科がある。	
\\	複数形チェック
\\	[名詞]	同僚	
\\	彼は私の仕事仲間である。	
\\	複数形チェック
\\	[名詞]	調理器具	
\\	スロークッカーを買いました。	
\\	複数形チェック
\\	[形容詞]	植⺠地の	
\\	[動詞]	対処する	
\\	彼女はストレスに対処できない。	
\\	[名詞]	入植者	
\\	複数形チェック
\\	[名詞]	銅	
\\	真鍮は銅と亜鉛の合金である。	
\\	複数形チェック
\\	[名詞]	戦闘、格闘	
\\	彼は前線へやられた。	
\\	複数形チェック
\\	[名詞]	芯、中心	
\\	彼は骨の髄まで腐りきっている。	
\\	複数形チェック
\\	[名詞]	喜劇役者	
\\	彼は三枚目だ。	
\\	複数形チェック
\\	[名詞]	廊下	
\\	廊下は大変混雑していたので歩けなかった。	
\\	複数形チェック
\\	[名詞]	指揮者	
\\	あの兵士は先生になりました。	
\\	複数形チェック
\\	[名詞]	割引券	
\\	回数券を下さい。	
\\	複数形チェック
\\	[名詞]	献身、約束	
\\	彼は国民と約束したことを実行することを強調したが、彼は多くの約束を守っていない。	
\\	複数形チェック
\\	[形容詞]	勇敢な	
\\	[副詞]	一般に	
\\	抗生物質は一般的に酵素の阻害剤である。	
\\	[名詞]	揺りかご	
\\	赤ん坊がゆりかごの中で眠っている。	
\\	複数形チェック
\\	[形容詞]	クリーム状の	
\\	[名詞]	恐⻯	
\\	彼は恐竜の骨を盗んだことで非難された。	
\\	複数形チェック
\\	[名詞]	創造性	
\\	紙をくしゃくしゃに丸めたり、びりびりに破ったりする遊びは、子供の創造力を高めます。	
\\	複数形チェック
\\	[名詞]	下落	
\\	ディップが辛くて、舌がヒリヒリした。	
\\	複数形チェック
\\	[名詞]	コオロギ	
\\	トムは有能なクリケット選手だ。	
\\	複数形チェック
\\	[形容詞]	外交の	
\\	それ以来カナダとイランの国交は断絶している。	
\\	[名詞]	評論家	
\\	あの若い評論家はあちこちで引っ張りだこである。	
\\	複数形チェック
\\	[形容詞]	がっかりした	
\\	君には幻滅した。	
\\	[名詞]	ワニ	
\\	トムさんはワニを食べました。	
\\	複数形チェック
\\	[名詞]	排出	
\\	複数形チェック
\\	[名詞]	交差点、岐路	
\\	その交差点を右に曲がってください。	
\\	複数形チェック
\\	[名詞]	ディスコ	
\\	複数形チェック
\\	[名詞]	横断歩道	
\\	道路を渡る前には、信号が青の横断歩道であっても全方位を確認しないといけない。	
\\	複数形チェック
\\	[名詞]	不名誉	
\\	お前のふるまいはわが家の名を汚した。	
\\	複数形チェック
\\	[形容詞]	非常に重要な	
\\	[名詞]	変装	
\\	ボブは変装していたけれども、一目で彼と分かった。	
\\	複数形チェック
\\	[名詞]	立方体	
\\	私はホットコーヒーに氷を入れるのが好きです。大抵熱すぎるので。	
\\	複数形チェック
\\	[名詞]	むかつくような嫌悪	
\\	複数形チェック
\\	[名詞]	キュウリ	
\\	なんて長いキュウリだ。	
\\	複数形チェック
\\	[形容詞]	むかむかする	
\\	気色悪いー!	
\\	[動詞]	育む、助⻑する	
\\	かれらは土地をたがやした。	
\\	[名詞]	汚れ水	
\\	複数形チェック
\\	[名詞]	狡猾さ	
\\	トムは賢いというよりずる賢いイメージだな。	
\\	複数形チェック
\\	[名詞]	乱雑、混乱	
\\	部屋は散らかり放題だった。	
\\	複数形チェック
\\	[名詞]	通貨	
\\	自国通貨建てで海外に投資することがいつも利益を生むとは限らない。	
\\	複数形チェック
\\	[動詞]	処分する	
\\	彼はそのごみを処分した。	
\\	[形容詞]	現在のところ	
\\	現在-10
\\	です。	
\\	[名詞]	論争、口論	
\\	論争にやっとけりが付いた。	
\\	複数形チェック
\\	[名詞]	乳製品	
\\	複数形チェック
\\	[動詞]	溶かす	
\\	空気中の酸素は水に溶解する。	
\\	[名詞]	ヒナギク	
\\	彼女にヒナギクの花を摘んでやった。	
\\	複数形チェック
\\	[名詞]	違い、区別	
\\	すべて人は、人種、皮膚の色、性、言語、宗教、政治上その他の意見、国民的若しくは社会的出身、財産、門地その他の地位又はこれに類するいかなる事由による差別をも受けることなく、この宣言に掲げるすべての権利と自由とを享有することができる。	
\\	複数形チェック
\\	[動詞]	酷評する	
\\	こんちくしょう。	
\\	[副詞]	はっきりと	
\\	内容や考え方の多くは言語に依存しないものではありますが、単語の分かち書きをしない点や統語構造等の違いから、日本語を対象とする場合、いくつか気をつけなければいけない点があります。	
\\	[名詞]	湿気	
\\	洗濯物はまだ湿っている。	
\\	複数形チェック
\\	[形容詞]	卓越した、著名な	
\\	人は笑うことが出来ることで、他のすべての被造物と区別される。	
\\	[名詞]	大胆不敵	
\\	蟹座の女の子って『だいたん』なんだって。	
\\	複数形チェック
\\	[名詞]	苦痛、悩み	
\\	一部の人はひどい目に遭ったとき、他人に祈りを求めるのに自分は祈ることを忘れてしまう。	
\\	複数形チェック
\\	[動詞]	暗くする	
\\	[動詞]	邪魔する	
\\	私たちは彼の邪魔をした。	
\\	[副詞]	陰気に	
\\	[名詞]	離婚	
\\	離婚するんじゃないかしら。	
\\	複数形チェック
\\	[名詞]	夜明け	
\\	僕は日の光で本を読むことが好き。	
\\	複数形チェック
\\	[名詞]	波止場	
\\	巨大なタンカーがついに今し方出港した。	
\\	複数形チェック
\\	[名詞]	締切、期日	
\\	〆切に間に合いますか。	
\\	複数形チェック
\\	[名詞]	イルカ	
\\	トムはイルカの絵を描いた。	
\\	複数形チェック
\\	[名詞]	商取り引き	
\\	トムは麻薬を売買している。	
\\	複数形チェック
\\	[形容詞]	支配的な	
\\	爵位があるから、貴族だというわけにはいかないんだぜ。爵位が無くても、天爵というものを持っている立派な貴族のひともあるし、おれたちのように爵位だけは持っていても、貴族どころか、賤民にちかいのもいる。	
\\	[副詞]	深く	
\\	[動詞]	支配する	
\\	多くの人は英語が世界を支配する現代の言語情勢を不公平さを容易に理解できる。とはいえ、たとえば異なる言語の知識人や指導者間での英語を通した比較的良好かつ直接的な対話というような全体的見地からの有用性も、また自身の英語知識による個人的な利点もまた認められるのである。エスペラントのような中立言語では、多くを知らず、情報を探すこともできない。	
\\	[名詞]	ロバ	
\\	馬はロバと違います。	
\\	複数形チェック
\\	[名詞]	宣言	
\\	関税申告書が必要になります。	
\\	複数形チェック
\\	[名詞]	戶口の上がり段	
\\	複数形チェック
\\	[名詞]	装飾	
\\	彼は古い木製のオールを壁にかけて飾っている。	
\\	複数形チェック
\\	[形容詞]	疑っている	
\\	私は彼の成功を危ぶむ。	
\\	[名詞]	欠陥、不備	
\\	完璧であるには、欠点だけが欠けていた。	
\\	複数形チェック
\\	[形容詞]	疑いのない	
\\	[形容詞]	楽しい	
\\	君に会うのはいつも楽しい。	
\\	[名詞]	ハト	
\\	ハトは平和の表象である。	
\\	複数形チェック
\\	[名詞]	⺠主党員	
\\	民主主義者というのは、多数派の意思に従う自由な市民だ。	
\\	複数形チェック
\\	[形容詞]	下方への	
\\	[名詞]	デモ	
\\	何十人という若い人がデモに参加した。	
\\	複数形チェック
\\	[動詞]	排出させる	
\\	下水の通りがよくなった。	
\\	[形容詞]	密集した	
\\	生い茂った生垣が庭を囲んでいた。	
\\	[名詞]	憂鬱、うつ病	
\\	トムは過労で鬱になった。	
\\	複数形チェック
\\	[動詞]	ひどく怖がる	
\\	羹に懲りて鱠を吹く。	
\\	[動詞]	奪う	
\\	彼は市民権を奪われた。	
\\	[名詞]	たんす	
\\	鏡は鏡台の上にあります。	
\\	複数形チェック
\\	[動詞]	得る、派生する	
\\	これらの専門用語はギリシャ語に由来している。	
\\	[名詞]	洋裁師	
\\	複数形チェック
\\	[形容詞]	必死の	
\\	トムは自分の評判を守るのに必死だった。	
\\	[名詞]	乾燥器	
\\	複数形チェック
\\	[副詞]	死に物狂いで	
\\	[名詞]	しずく	
\\	私は病院で点滴を受けた。	
\\	複数形チェック
\\	[動詞]	軽蔑する、嫌がる	
\\	彼女は彼を軽蔑した。	
\\	[動詞] [不規則動詞]	居住する	
\\	彼は田舎に住んでいる。	
\\	[名詞]	運命	
\\	運命って信じる?	
\\	複数形チェック
\\	[動詞]	染める	
\\	羊毛は染め上がりがきれいです。	
\\	[形容詞]	破壊的な	
\\	[動詞]	感知する、探知する	
\\	重力波は探知するのが非常に難しい。	
\\	[副詞]	熱心に	
\\	雪だるまが作りたくて、子供たちは初雪を楽しみにしていた。	
\\	[名詞]	熱望、熱意	
\\	トムがあまりにも一生懸命話すもんだから、途中から何だか可愛く思えて来ちゃった。	
\\	複数形チェック
\\	[名詞]	宅地開発業者、開発者	
\\	彼はオクテです。	
\\	複数形チェック
\\	[形容詞]	まじめ	
\\	本格的に雨が降り出した。	
\\	[動詞]	捧げる	
\\	彼女は彼に自分をささげた。	
\\	[名詞]	稼ぐ	
\\	複数形チェック
\\	[名詞]	献身	
\\	複数形チェック
\\	[形容詞]	世俗的な、この世の	
\\	わたしが地上のことを語っているのに、あなたがたが信じないならば、天上のことを語った場合、どうしてそれを信じるだろうか	
\\	[名詞]	威厳、気品	
\\	君は課長としての貫禄がないね。	
\\	複数形チェック
\\	[名詞]	復活祭	
\\	イースターおめでとう。	
\\	複数形チェック
\\	[形容詞]	勤勉な	
\\	彼は勤勉です。	
\\	[形容詞]	経済的な	
\\	ナンシーはなかなか笑顔を見せない。	
\\	[形容詞]	ほの暗い	
\\	私は祖母のことをかすかに覚えている。	
\\	[名詞]	経済学者	
\\	多くの経済学者はその事実に気づいていない。	
\\	複数形チェック
\\	[動詞]	食事をする	
\\	食堂車はどこにありますか。	
\\	[動詞]	編集する	
\\	ブラックリストに載っていれば編集を拒否するてなこと考えてました。	
\\	[形容詞]	教養のある	
\\	彼は彼女のおじいさんに教育された。	
\\	[名詞]	夢想、空想	
\\	トムは空想の世界に生きている。	
\\	複数形チェック
\\	[名詞]	教育者	
\\	複数形チェック
\\	[名詞]	別れ	
\\	さようなら!	
\\	複数形チェック
\\	[副詞]	効果的に	
\\	することを考え、経済的かつ効率的に研究を遂行する必要があります。	
\\	[名詞]	農家	
\\	複数形チェック
\\	[形容詞]	選択の	
\\	[形容詞]	魅了する	
\\	すごく面白そうですね。	
\\	[名詞]	電子工学	
\\	複数形チェック
\\	[名詞]	留め具、ファスナー	
\\	複数形チェック
\\	[形容詞]	上品な	
\\	なんてエレガントなヨットなんでしょう!	
\\	[名詞]	ごちそう	
\\	複数形チェック
\\	[名詞]	初等の	
\\	私は名古屋の小学校に通った。	
\\	複数形チェック
\\	[形容詞]	連合の	
\\	連邦予算はかろうじて議会の承認を得た。	
\\	[名詞]	選ばれた者	
\\	彼はもう大統領ではないものの、政界エリートの中でいまだに相当な影響力を保持している。	
\\	複数形チェック
\\	[名詞]	フェリー	
\\	夜通しの豪雨は、被災したフェリーからの生存者の救出を妨げた。	
\\	複数形チェック
\\	[動詞]	気まずい思いをさせる	
\\	恥ずかしいなぁ!	
\\	[形容詞]	肥沃な	
\\	[形容詞]	当惑させる	
\\	恥ずかしいなぁ!	
\\	[動詞]	取ってくる	
\\	彼は井戸へ行って水をくんできた。	
\\	[名詞]	大使館	
\\	日本大使館はどこにありますか。	
\\	複数形チェック
\\	[名詞]	繊維	
\\	光ファイバーケーブルは膨大な量の情報を搬送できる。	
\\	複数形チェック
\\	[副詞]	感情的に	
\\	[形容詞]	激しい	
\\	彼らの犬はとても獰猛なので、誰も近づけなかった。	
\\	[動詞]	拡大する、広くする	
\\	[名詞]	戦士	
\\	図体ばかり大きいから喧嘩強そうだが、実際喧嘩弱いだろ。	
\\	複数形チェック
\\	[動詞]	確実にする	
\\	用意にぬかりがなければ成功は確実だ。	
\\	[副詞]	細かく	
\\	塩漬けキャベツを細かく切る。	
\\	[名詞]	芸能人	
\\	複数形チェック
\\	[名詞]	懐中電灯	
\\	それはトムの懐中電灯です。	
\\	複数形チェック
\\	[動詞]	資格を与える	
\\	日本では助産師国家試験の受験資格は女性のみとなっている。	
\\	[動詞]	お世辞を言う	
\\	そう言っていただけると嬉しいです。	
\\	[形容詞]	環境の	
\\	[形容詞]	曲げやすい	
\\	より柔軟であれ。	
\\	[名詞]	平等	
\\	人々はより多くの自由と平等を求める。	
\\	複数形チェック
\\	[形容詞]	栄える、全盛の	
\\	[動詞]	消す、消去する	
\\	検閲とは:真実を消し去ることである。	
\\	[名詞]	インフルエンザ	
\\	インフルエンザにかかりました。	
\\	複数形チェック
\\	[動詞]	立てる	
\\	この建物は三百年前に建てられました。	
\\	[形容詞]	流暢な	
\\	君は流暢な英語を話す。	
\\	[名詞]	使い	
\\	彼女は用事で外出中です。	
\\	複数形チェック
\\	[名詞]	泡	
\\	それ歯磨き粉じゃなくて洗顔フォームだよ!	
\\	複数形チェック
\\	[名詞]	尊敬	
\\	複数形チェック
\\	[名詞]	紙ばさみ	
\\	どのフォルダにファイルを保存しましたか。	
\\	複数形チェック
\\	[形容詞]	永遠の	
\\	永遠の愛なんてあるわけないじゃん。	
\\	[名詞]	信奉者	
\\	キリスト教信者は、一人の神が存在し、イエスキリストは神の子であると信じている。	
\\	複数形チェック
\\	[形容詞]	⺠族の	
\\	少数民族は偏見、貧困、抑圧と戦っている。	
\\	[名詞]	足跡	
\\	テーブルに猫の足跡が付いている。	
\\	複数形チェック
\\	[動詞]	評価する	
\\	会社の資産を評価するのはスコットさんの仕事です。	
\\	[名詞]	歩幅、足跡	
\\	トムは家業を継いだ。	
\\	複数形チェック
\\	[名詞]	評価	
\\	この財務調査は会社の資産を評価することも含みます。	
\\	複数形チェック
\\	[動詞] [不規則動詞]	禁じる	
\\	彼女は今外出禁止です。	
\\	[副詞]	明らかに	
\\	彼女は明らかに病気だ。	
\\	[名詞]	進化	
\\	トムは進化論を信じていない。	
\\	複数形チェック
\\	[動詞]	誇張する	
\\	案ずるより産むが易し。	
\\	[名詞]	ひたい	
\\	おでこから血出てるよ。	
\\	複数形チェック
\\	[動詞]	超える	
\\	その車は制限速度を超過している。	
\\	[形容詞]	忘れっぽい	
\\	トムは忘れっぽい。	
\\	[形容詞]	過度の	
\\	過度な飲酒は差し控えるようにしてください。	
\\	[名詞]	書式、構成	
\\	毎度のことですが、源泉徴収税納付後、納税証明をPDFで結構ですから、送って いただけるよう、お願いします。	
\\	複数形チェック
\\	[動詞]	叫ぶ	
\\	「これが探していたものだ」と彼は叫んだ。	
\\	[名詞]	高速道路	
\\	高速道路で悲惨な事故があった。	
\\	複数形チェック
\\	[動詞]	除く	
\\	[名詞]	振動数、周波数	
\\	四種類の金属を使うことで特定の共振を抑えます。	
\\	複数形チェック
\\	[名詞]	重役	
\\	複数形チェック
\\	[名詞]	1年生、新入生	
\\	新人です。	
\\	複数形チェック
\\	[名詞]	排気ガス	
\\	複数形チェック
\\	[形容詞]	ぎょっとさせる	
\\	[形容詞]	疲れ果てた	
\\	疲れたなあ。	
\\	[動詞]	しかめる	
\\	トムは眉をひそめた。	
\\	[名詞]	拡張、発展	
\\	複数形チェック
\\	[名詞]	遠征、探検	
\\	複数形チェック
\\	[形容詞]	有益な	
\\	[形容詞]	経験した	
\\	彼は経験豊かな教師だ。	
\\	[名詞]	いらいらすること	
\\	複数形チェック
\\	[形容詞]	試験的な	
\\	実験の結果は今ひとつぱっとしなかった。	
\\	[動詞]	満たす	
\\	彼の長年の望みが実現した。	
\\	[名詞]	探査、探索	
\\	複数形チェック
\\	[動詞]	供給する	
\\	彼女は部屋に美しい家具を備えた。	
\\	[名詞]	さらすこと	
\\	複数形チェック
\\	[動詞]	全速力で走る	
\\	[形容詞]	大規模な、広範囲の	
\\	[名詞]	ごみ、生ごみ	
\\	これはゴミです。	
\\	複数形チェック
\\	[名詞]	外部、外観	
\\	中身で人を判断すると言うけれど、結局僕たちは表面に表れた部分で判断するしかないじゃないですか。	
\\	複数形チェック
\\	[名詞]	衣服、衣類	
\\	身頃の肩を縫い合わせる。	
\\	複数形チェック
\\	[形容詞]	顔の	
\\	面疔ができた。小鼻の奥に1個、痛い塊がある。	
\\	[名詞]	凝視	
\\	私はこの景色をずっと眺めていたい。	
\\	複数形チェック
\\	[名詞]	施設	
\\	メグは語学の才能がある。	
\\	複数形チェック
\\	[名詞]	⻭車	
\\	複数形チェック
\\	[名詞]	公正	
\\	複数形チェック
\\	[副詞]	寛大に	
\\	[副詞]	正確に	
\\	[形容詞]	心からの、真の	
\\	あなたは本当のダイヤモンドを見たことがないのです。	
\\	[名詞]	飢饉	
\\	食糧不足のため、家畜が餓死した。	
\\	複数形チェック
\\	[名詞]	キリン	
\\	麒麟はどこにいますか。	
\\	複数形チェック
\\	[形容詞]	すばらしい	
\\	すごいぞ!	
\\	[名詞]	きらめき	
\\	複数形チェック
\\	[形容詞]	悲観的な	
\\	前途は暗澹としていた。	
\\	[名詞]	騎手、馬術家	
\\	複数形チェック
\\	[形容詞]	愉快な、すばらしい	
\\	[名詞]	主婦	
\\	複数形チェック
\\	[名詞]	ゴルフをする人	
\\	彼はゴルファーだ。	
\\	複数形チェック
\\	[名詞]	家事	
\\	ホテルのメイドとしてパートで仕事していましたが、その仕事はあまり好きではありませんでした。	
\\	複数形チェック
\\	[名詞]	ガウン	
\\	君の卒業式での姿を見るのが楽しみだ。	
\\	複数形チェック
\\	[名詞]	遠ぼえの声	
\\	彼の声は悪霊の突然の咆哮にかき消される。	
\\	複数形チェック
\\	[動詞]	つかむ	
\\	捕まえろ。	
\\	[名詞]	抱きしめること	
\\	抱きしめて。	
\\	複数形チェック
\\	[形容詞]	優美な	
\\	[形容詞]	こっけいな	
\\	そのパーティーで彼の振る舞いはあまりに滑稽だったので、私は笑わずにはいられなかった。	
\\	[名詞]	卒業	
\\	トムは昨日卒業論文を提出した。	
\\	複数形チェック
\\	[名詞]	暴風	
\\	ハリケーンの種類は風速によって決まります。	
\\	複数形チェック
\\	[名詞]	グラフ、図表	
\\	複数形チェック
\\	[形容詞]	氷で冷やした	
\\	アイスコーヒーは好きじゃない。	
\\	[形容詞]	図表による	
\\	[形容詞]	氷の、氷のような	
\\	[名詞]	温室	
\\	彼らは温室でいちごを栽培している。	
\\	複数形チェック
\\	[副詞]	理想的に	
\\	[動詞]	悲しむ	
\\	[名詞]	ばか、白痴	
\\	バカ!	
\\	複数形チェック
\\	[形容詞]	ぞっとする	
\\	このような仕事で怖い顔をしたら、お客さんはいらっしゃらないでしょう。	
\\	[名詞]	幻想	
\\	万事は夢。	
\\	複数形チェック
\\	[動詞]	にこっと笑う	
\\	我々を見てにたりと笑った。	
\\	[動詞]	説明する、示す	
\\	新しい語彙を説明するために誰でもデータシステムに追加をする事ができます。	
\\	[動詞] [不規則動詞]	細かく砕く	
\\	地面が動いた。	
\\	[形容詞]	想像上の	
\\	龍は空想の動物だ。	
\\	[動詞]	うめく	
\\	トムは大きなうめき声を発した。	
\\	[形容詞]	想像力に富んだ	
\\	彼は非常に想像力にとんだ作家です。	
\\	[名詞]	食料雑貨類	
\\	この食料品店は自然食品のみを売っている。	
\\	複数形チェック
\\	[名詞]	物まね	
\\	彼はその模造品を本物と比較した。	
\\	複数形チェック
\\	[名詞]	指針、指導基準	
\\	複数形チェック
\\	[副詞]	非常に	
\\	旅は私たちの心を大いに豊かにするだろう。	
\\	[名詞]	ギター奏者	
\\	あなたの大好きなヘヴィメタルのギタリストは誰ですか?	
\\	複数形チェック
\\	[名詞]	移⺠	
\\	ほとんどのアメリカ人は元をたどれば移民である。	
\\	複数形チェック
\\	[名詞]	⻭ぐき	
\\	ガム持ってない?	
\\	複数形チェック
\\	[形容詞]	我慢できない	
\\	私は気が短いんだ。	
\\	[名詞]	美容師、美容院	
\\	彼女は美容室に行った。	
\\	複数形チェック
\\	[形容詞]	不完全な	
\\	[名詞]	停止	
\\	そのエレベーターは急停止した。	
\\	複数形チェック
\\	[名詞]	道具	
\\	複数形チェック
\\	[名詞]	取り扱い	
\\	手数料はかかりますか。	
\\	複数形チェック
\\	[動詞]	ほのめかす	
\\	何か意味があるのですか。	
\\	[名詞]	洋服掛け	
\\	「ペット・セミタリー」は真に迫ったサスペンス映画だった。	
\\	複数形チェック
\\	[名詞]	傾斜	
\\	複数形チェック
\\	[名詞]	硬度	
\\	複数形チェック
\\	[形容詞]	傾向を示して	
\\	彼はカッとなる傾向がある。	
\\	[名詞]	金属製品	
\\	彼は金物を商っている。	
\\	複数形チェック
\\	[形容詞]	信じられない	
\\	そいつは凄い!	
\\	[形容詞]	無害の	
\\	細菌はただの小さい細胞です、無害です。	
\\	[副詞]	独立して	
\\	トムとビルは別々に考えて同じ結論に達した。	
\\	[名詞]	ハーモニカ	
\\	複数形チェック
\\	[形容詞]	間接の、二次的な	
\\	[動詞]	急がせる	
\\	[形容詞]	勤勉な、熱心な	
\\	[副詞]	急いで	
\\	彼は急いで荷物をまとめた。	
\\	[名詞]	感染、感染症	
\\	トムさんは目の感染がかかっています。	
\\	複数形チェック
\\	[形容詞]	性急な、軽率な	
\\	あわてた結婚はあまり、うまくいったためしがない。	
\\	[形容詞]	下位の、下級の	
\\	そのため、派遣労働者は劣悪な労働条件の下で働いている。	
\\	[名詞]	干し草	
\\	トムは花粉症がある。	
\\	複数形チェック
\\	[名詞]	膨脹、インフレ	
\\	零細企業はインフレで苦境に陥っています。	
\\	複数形チェック
\\	[名詞]	表題、見出し	
\\	あなたは何処へ向かっているのですか。	
\\	複数形チェック
\\	[形容詞]	影響力がある	
\\	彼は押しが強い。	
\\	[名詞]	見出し	
\\	一方、全国紙の一社だけが夕刊に日米交渉は実質合意に至ったという見出しを掲げて、政府関係者を含め多くの人を驚かせた。	
\\	複数形チェック
\\	[動詞]	住む	
\\	その島々にはどんな動物が住んでいるのですか。	
\\	[名詞]	ヘッドホン	
\\	ヘッドホンが壊れたので夜音楽を聴けなくなった。	
\\	複数形チェック
\\	[名詞]	住⺠	
\\	複数形チェック
\\	[名詞]	本社、本部	
\\	同社の本社はロサンゼルスにあります。	
\\	複数形チェック
\\	[副詞]	初めに	
\\	我々のコンピューター・システムには最初若干のトラブルがあったが、今は全部解決されている。	
\\	[名詞]	積み重ね、かたまり、山	
\\	そんな計画は捨ててしまいなさい。	
\\	複数形チェック
\\	[名詞]	入力	
\\	複数形チェック
\\	[形容詞]	素晴らしい	
\\	宇宙には無数の天体がある。	
\\	[動詞]	質問する	
\\	「気分はどうですか。」と彼は尋ねた。	
\\	[名詞]	生け垣、垣根	
\\	彼は垣根を飛び越えた。	
\\	複数形チェック
\\	[名詞]	内部者	
\\	政治家が、インサイダー取引を包み隠そうとした。	
\\	複数形チェック
\\	[名詞]	後継者、相続人	
\\	と 
\\	は同音異義語である。	
\\	複数形チェック
\\	[動詞]	検査する	
\\	車を買う前によく調べてみたほうがいい。	
\\	[名詞]	(動物の)群れ	
\\	習慣は無思慮な人々を支配する。	
\\	複数形チェック
\\	[名詞]	検査官	
\\	警部、もっと警官を投入できるようになるまでは、事態を表に出さないようにしたまえ。	
\\	複数形チェック
\\	[名詞]	ヒロイン	
\\	複数形チェック
\\	[名詞]	ひらめき	
\\	私はひらめきを失った。	
\\	複数形チェック
\\	[動詞]	目立たせる、強調する	
\\	この先ネタバレ
\\	大丈夫な方はドラッグ(反転)して読んでね
\\	[動詞]	取り付ける	
\\	リナックスをインストールしろ!	
\\	[動詞]	ハイキングする	
\\	ハイキングは中止にしたほうがいいわね。	
\\	[副詞]	ただちに	
\\	トムは間髪を入れず、返答した。	
\\	[名詞]	丘の頂上	
\\	複数形チェック
\\	[名詞]	指導者	
\\	トムはスキーのインストラクターだ。	
\\	複数形チェック
\\	[名詞]	歴史家	
\\	鈴木先生は歴史の専門家です。	
\\	複数形チェック
\\	[動詞]	保険をかける	
\\	保険をかけて下さい。	
\\	[形容詞]	歴史的な	
\\	京都には多くの古い有名な建物がある。	
\\	[動詞]	通訳する	
\\	彼の言葉をどう解釈していいのか分からない。	
\\	[動詞]	ヒッチハイクする	
\\	[名詞]	面接者	
\\	複数形チェック
\\	[形容詞]	ホームシックの	
\\	彼女は故郷が恋しくてたまらなかった。	
\\	[名詞]	侵略、侵害	
\\	複数形チェック
\\	[形容詞]	高貴な	
\\	われわれの社会には、高潔な人もいれば、詐欺師もいる。	
\\	[名詞]	発明家	
\\	あの人は発明王です。	
\\	複数形チェック
\\	[形容詞]	水平線の	
\\	[名詞]	投資家	
\\	外国人投資家は資金をアメリカから引き揚げた。	
\\	複数形チェック
\\	[副詞]	内部へ	
\\	外を眺める者は夢を見るが、内を眺める者は目を覚ます。	
\\	[副詞]	精神的に	
\\	彼は心ここにあらずという感じだった。	
\\	[名詞]	皮肉	
\\	複数形チェック
\\	[名詞]	電子レンジ	
\\	これ何分チンしたらいい?	
\\	複数形チェック
\\	[形容詞]	孤立した	
\\	私は孤独を感じた。	
\\	[形容詞]	中の	
\\	彼女は40代半ばで再婚した。	
\\	[名詞]	象牙	
\\	複数形チェック
\\	[名詞]	正午	
\\	牛に乾草をやって、昼の仕事はおわりです。	
\\	複数形チェック
\\	[副詞]	油断無く	
\\	[名詞]	牛乳配達人	
\\	トムの知らないうちに、メーリは牛乳屋さんと浮気した。	
\\	複数形チェック
\\	[名詞]	ジョギング	
\\	私は毎朝走ることにしている。	
\\	複数形チェック
\\	[名詞]	第100万	
\\	複数形チェック
\\	[名詞]	キーワード	
\\	新しい文を追加するまえに、その文によって当サイトに新しい語彙がもたらされるかどうか、キーワード検索してご確認ください。	
\\	複数形チェック
\\	[名詞]	マイナス	
\\	92引く9は83です。	
\\	複数形チェック
\\	[名詞]	殺人者	
\\	トムはプロの殺し屋だ。	
\\	複数形チェック
\\	[名詞]	いたずら	
\\	また何かやらかしたわね!	
\\	複数形チェック
\\	[名詞]	編み物	
\\	彼女は編み物に夢中になっている。	
\\	複数形チェック
\\	[名詞]	逆境、災難	
\\	彼はなんて不運なのだろう。	
\\	複数形チェック
\\	[名詞]	実験室	
\\	このラボラトリーは最新のコンピューターを整備している。	
\\	複数形チェック
\\	[名詞]	ミサイル	
\\	ミサイル攻撃で多くの人命が失われた。	
\\	複数形チェック
\\	[名詞]	労働者	
\\	複数形チェック
\\	[名詞]	任務、使命	
\\	彼は任務を成し遂げた。	
\\	複数形チェック
\\	[形容詞]	足が不自由な	
\\	のろまにはなりたくない。かっこ良くなりたい!!	
\\	[名詞]	ミトン	
\\	複数形チェック
\\	[名詞]	手提げランプ	
\\	トムはランタンを掲げた。	
\\	複数形チェック
\\	[名詞]	混合物、混合	
\\	空気はさまざまな気体の混合物である。	
\\	複数形チェック
\\	[名詞]	ヒバリ	
\\	複数形チェック
\\	[動詞]	うめく、うなる	
\\	彼女は時間がないといつも不平たらたらだ。	
\\	[副詞]	最後に	
\\	[形容詞]	節度のある	
\\	適度な運動は血液の循環を活発にする。	
\\	[名詞]	指導力	
\\	彼はその政党の主導権をにぎった。	
\\	複数形チェック
\\	[動詞]	変更する、修正する	
\\	名詞を修飾するもの(形容詞、または形容詞相当語句)。	
\\	[副詞]	法律的に	
\\	彼女は夫と正式に離婚した。	
\\	[形容詞]	湿った	
\\	空気は穏やかで、土には湿り気がある。	
\\	[名詞]	伝説	
\\	伝説によれば、その森には幽霊が出たので、誰も足を踏み入れようとしなかったそうだ。	
\\	複数形チェック
\\	[名詞]	湿気	
\\	お米は湿気を嫌います。保管にはご注意ください。	
\\	複数形チェック
\\	[名詞]	なめること	
\\	脱水症状には、塩をひと舐め。	
\\	複数形チェック
\\	[名詞]	君主、主権者	
\\	複数形チェック
\\	[名詞]	ライター	
\\	トムが私のライターを持っていた。	
\\	複数形チェック
\\	[副詞]	道徳的に見て	
\\	[名詞]	点火、照明	
\\	複数形チェック
\\	[名詞]	コケ	
\\	「転がる石は苔をもつけず」はことわざである。	
\\	複数形チェック
\\	[副詞]	同じく	
\\	こちらこそ、私もお会いできてよかったです。	
\\	[名詞]	動機	
\\	彼女は何か利己的な動機で行動している。	
\\	複数形チェック
\\	[名詞]	制限、限界	
\\	さらに、個人の属する国又は地域が独立国であると、信託統治地域であると、非自治地域であると、又は他のなんらかの主権制限の下にあるとを問わず、その国又は地域の政治上、管轄上又は国際上の地位に基づくいかなる差別もしてはならない。	
\\	複数形チェック
\\	[名詞]	オートバイ	
\\	バイクはいいなあ、渋滞の中でもすいすいと通り抜けられて。	
\\	複数形チェック
\\	[名詞]	酒	
\\	この店では酒類は売っていない。	
\\	複数形チェック
\\	[名詞]	自動車運転者	
\\	複数形チェック
\\	[名詞]	表に載せること、表の作成	
\\	複数形チェック
\\	[動詞]	嘆き悲しむ	
\\	喪服をクリーニングに出せますか?	
\\	[副詞]	文字どおりに	
\\	あんまり真に受けるなよ。	
\\	[形容詞]	多数の	
\\	[名詞]	肝臓	
\\	アルコールは肝臓に被害をもたらす。	
\\	複数形チェック
\\	[名詞]	ミイラ	
\\	ママー、今あのおにいちゃんすごいころんだよー。	
\\	複数形チェック
\\	[名詞]	大えび	
\\	僕は伊勢えびにアレルギーがあります。	
\\	複数形チェック
\\	[名詞]	ささやき、かすかな人声	
\\	複数形チェック
\\	[名詞]	孤独	
\\	人生には老年にならぬと解らない淋しい気持があるものだ。	
\\	複数形チェック
\\	[名詞]	神話	
\\	彼の話は作りごとだったとわかった。	
\\	複数形チェック
\\	[副詞]	大まかに、漫然と	
\\	[動詞]	交渉する	
\\	両国は条約を取り決めた。	
\\	[動詞]	ぶらぶらする	
\\	休日は映画を見に行ったり、カフェでのんびり過ごしたり、料理を作ったりしています。	
\\	[形容詞]	中立の	
\\	スイスは中立国である。	
\\	[名詞]	塊	
\\	面疔ができた。小鼻の奥に1個、痛い塊がある。	
\\	複数形チェック
\\	[副詞]	うまく	
\\	羊毛は染め上がりがきれいです。	
\\	[形容詞]	壮麗な	
\\	立派ですね。	
\\	[名詞]	ニッケル	
\\	複数形チェック
\\	[名詞]	陛下	
\\	複数形チェック
\\	[名詞]	寝巻き	
\\	複数形チェック
\\	[名詞]	化粧	
\\	彼女は化粧が濃い。	
\\	複数形チェック
\\	[名詞]	寝巻き	
\\	複数形チェック
\\	[名詞]	マンゴー	
\\	マンゴーが食べたい。	
\\	複数形チェック
\\	[名詞]	悪夢	
\\	悪夢に魘された。	
\\	複数形チェック
\\	[名詞]	製造業者	
\\	ロッキード事件とは、アメリカの航空機製造大手のロッキード社による、主に同社の旅客機の受注をめぐって1976年2 月に明るみに出た世界的な大規模汚職事件。	
\\	複数形チェック
\\	[形容詞]	途中で止まらない	
\\	この飛行機は東京までの直行便です。	
\\	[名詞]	マラソン	
\\	トムはボストンマラソンに参加した。	
\\	複数形チェック
\\	[形容詞]	有名な	
\\	承知致しました。有難う御座います。	
\\	[名詞]	幅、余地	
\\	ウチだって閑古鳥が鳴くようなカツカツの状態だから、バイトを雇う余裕なんてない。	
\\	複数形チェック
\\	[形容詞]	目立つ	
\\	両者の間には著しい違いがある。	
\\	[名詞]	目印	
\\	複数形チェック
\\	[名詞]	名詞	
\\	って名詞句でいいんですよね?	
\\	複数形チェック
\\	[名詞]	マーケティング	
\\	彼は営業部の部長です。	
\\	複数形チェック
\\	[名詞]	小説家	
\\	彼は教師でもあり小説家でもある。	
\\	複数形チェック
\\	[名詞]	しるし	
\\	隣の家との境を示す柵がある。	
\\	複数形チェック
\\	[名詞]	裸体の人	
\\	子供たちは裸で泳いでいた。	
\\	複数形チェック
\\	[形容詞]	巨大な	
\\	本日、マグニチュード8
\\	8の巨大地震が日本列島を襲いました。	
\\	[形容詞]	数多くの	
\\	恋人たちはたくさんの手紙を交わした。	
\\	[名詞]	畳	
\\	私はマットの下から鍵を見つけた。	
\\	複数形チェック
\\	[名詞]	保育園、託児所	
\\	向かいの保育園から、保育士だろう男の声で「ガオー、食べちゃうぞ!」というのが聞こえて来る。	
\\	複数形チェック
\\	[形容詞]	数学の	
\\	彼は数学の天才だ。	
\\	[名詞]	ナイロン	
\\	ナイロンストッキングはすぐ伝線する。	
\\	複数形チェック
\\	[名詞]	マットレス	
\\	複数形チェック
\\	[名詞]	オール	
\\	彼は古い木製のオールを壁にかけて飾っている。	
\\	複数形チェック
\\	[名詞]	修理工	
\\	複数形チェック
\\	[名詞]	服従	
\\	娘のつとめは従うことにある。	
\\	複数形チェック
\\	[形容詞]	中世の、中世風の	
\\	私の専門は中世ヨーロッパ史です。	
\\	[名詞]	目標	
\\	客観的な歴史は存在するだろうか。	
\\	複数形チェック
\\	[名詞]	メモ	
\\	メモ用紙か何か貸してくれ。	
\\	複数形チェック
\\	[名詞]	義務	
\\	もし生命の権利というものがあるなら、死についての権利もあることになる。そうでなければ生命の権利は権利ではなく義務になってしまうだろう。	
\\	複数形チェック
\\	[名詞]	記念物	
\\	革命記念日万歳!	
\\	複数形チェック
\\	[動詞]	強いる	
\\	われわれは計画をあきらめなければならなかった。	
\\	[名詞]	観察者	
\\	歴史は量子物理学のようなもので、観測者が観測した事象に影響してしまう。ケネディ暗殺は粒子だろうか、それとも波動だろうか?	
\\	複数形チェック
\\	[動詞]	滅びる	
\\	たくさんの魚が死んだ。	
\\	[名詞]	攻撃	
\\	ふたをあけた瞬間、ツーンと悪臭が鼻をつく。	
\\	複数形チェック
\\	[形容詞]	悲観的な	
\\	トムって悲観的だと思う。	
\\	[名詞]	供え物	
\\	彼らは高品質の商品を提供する事によって顧客を引き付けている。	
\\	複数形チェック
\\	[名詞]	ガソリン	
\\	複数形チェック
\\	[副詞]	公に、公式	
\\	先月正式に妻と離婚しました。	
\\	[名詞]	段階、局面	
\\	ロジバンでは開発当初より一貫して文化的中立を保つ事が志されています。	
\\	複数形チェック
\\	[動詞]	省略する	
\\	[名詞]	現象	
\\	ヒートアイランド現象の実体解析と対策のあり方について。	
\\	複数形チェック
\\	[形容詞]	前方への	
\\	[名詞]	哲学者	
\\	私は哲学者になるつもりでいた。	
\\	複数形チェック
\\	[名詞]	敵対者	
\\	彼は敵を料理した。	
\\	複数形チェック
\\	[副詞]	肉体的に	
\\	男性は女性よりも筋力がある。	
\\	[形容詞]	楽観的な	
\\	トムって楽観的だと思う。	
\\	[動詞]	突き刺す、貫く	
\\	くぎが壁を突きぬけた。	
\\	[形容詞]	口頭の	
\\	彼は口頭試験を受けた。	
\\	[名詞]	まくらカバー	
\\	複数形チェック
\\	[形容詞]	有機体の	
\\	この食料品店は自然食品のみを売っている。	
\\	[名詞]	窮地	
\\	小麦粉約4カップと塩ひとつまみを混ぜる。	
\\	複数形チェック
\\	[名詞]	生物、有機体	
\\	複数形チェック
\\	[副詞]	はっきりと	
\\	[名詞]	オルガン奏者	
\\	複数形チェック
\\	[名詞]	遊び仲間	
\\	複数形チェック
\\	[形容詞]	組織化された	
\\	いつも仕事場をきちんと整理しておきなさい。	
\\	[動詞] [不規則動詞]	嘆願する	
\\	彼女は彼に、行かないで欲しいと懇願した。	
\\	[形容詞]	東洋の	
\\	東洋の陶器に興味があります。	
\\	[形容詞]	楽しい	
\\	[動詞]	源を発する	
\\	フロイトは精神分析を考案した。	
\\	[名詞]	すき (犂)	
\\	このとき除雪車は確かに私たちの英雄でした。	
\\	複数形チェック
\\	[名詞]	装飾(品)	
\\	複数形チェック
\\	[動詞]	飛込む	
\\	私たちはそのまま洞窟の口にボートで突っ込んで行きました。	
\\	[名詞]	オンス	
\\	自分がかつてテロ集団の一員であったことをメアリーに打ち明けるのに、トムはありったけの勇気を振り絞らなくてはならなかった。	
\\	複数形チェック
\\	[名詞]	複数	
\\	の複数形は
\\	です。	
\\	複数形チェック
\\	[名詞]	出口	
\\	すごいわね。できてまだ間もない会社なのに店舗も多いし、流行ってるし。	
\\	複数形チェック
\\	[名詞]	文庫本	
\\	複数形チェック
\\	[名詞]	見解、見通し	
\\	職場で
\\	のスケジュールを使用していますが、PDAを購入しシンクロさせたいと考えています。	
\\	複数形チェック
\\	[名詞]	(地図・黑板などをさす)さし	
\\	を学ぼうとするたびに、ポインタに手こずる。	
\\	複数形チェック
\\	[動詞]	氾濫する	
\\	街は若いカップルで溢れていた。	
\\	[名詞]	世論調査、投票	
\\	複数形チェック
\\	[形容詞]	頭上の	
\\	頭上で電車がガタガタ音を立てていた。	
\\	[名詞]	小さい馬	
\\	私の世界では誰もがポニーで、虹を食べて蝶のうんこをしている。	
\\	複数形チェック
\\	[動詞] [不規則動詞]	寝過ごす	
\\	寝過ごしちゃった。	
\\	[名詞]	ローマ教皇	
\\	われらは教皇を得た。	
\\	複数形チェック
\\	[動詞] [不規則動詞]	追い越す	
\\	彼は車を加速し、私に追いついた。	
\\	[名詞]	ポータブル	
\\	トムは
\\	は欲しくない。彼が欲しいのは
\\	をサポートした携帯デバイスだ。	
\\	複数形チェック
\\	[形容詞]	圧倒的な	
\\	初めから果敢に前へ出て圧倒し、体勢を崩した相手を押し出した。	
\\	[動詞]	描く	
\\	[名詞]	カキ	
\\	私は牡蠣を食べません。	
\\	複数形チェック
\\	[形容詞]	無力な	
\\	[名詞]	包装	
\\	彼女は喜びの叫びを上げた。包装紙の中の人形に飛びつきたい気持ちだった。	
\\	複数形チェック
\\	[副詞]	正確に	
\\	そこが正しく重要なところなんだ。	
\\	[名詞]	詰め物	
\\	ちょうど荷造りが終わったところです。	
\\	複数形チェック
\\	[動詞]	予測する	
\\	激しい雷雨が予想されます。	
\\	[名詞]	詰め物(パッド)	
\\	マウスパッドが要ります。	
\\	複数形チェック
\\	[名詞]	県	
\\	私の両親は愛知県で生まれた。	
\\	複数形チェック
\\	[副詞]	苦しんで	
\\	彼女は痛いほどやせこけていた。	
\\	[名詞]	好み	
\\	音楽の好みは人によって好きずきです。	
\\	複数形チェック
\\	[名詞]	友達	
\\	複数形チェック
\\	[名詞]	偏見	
\\	少数民族は偏見、貧困、抑圧と戦っている。	
\\	複数形チェック
\\	[名詞]	パンフレット	
\\	詳しくはパンフレットをご覧ください。	
\\	複数形チェック
\\	[名詞]	処方箋	
\\	処方箋を見せてください。	
\\	複数形チェック
\\	[名詞]	パンジー	
\\	彼女はチューリップ、パンジーやひな菊のような花を育てている。	
\\	複数形チェック
\\	[名詞]	表現、発表	
\\	発表は30分以内に制限して下さい。	
\\	複数形チェック
\\	[名詞]	ペーパーバック	
\\	文庫本にカバーをかけた方がいいです。	
\\	複数形チェック
\\	[副詞]	まもなく、やがて	
\\	横には喧嘩状態の妻が眠っている。	
\\	[名詞]	日傘	
\\	複数形チェック
\\	[形容詞]	大統領の	
\\	果たして、ゴア氏は大統領候補として立つのか?	
\\	[動詞]	参加する、加わる	
\\	だれでも参加できる。	
\\	[動詞]	広がる	
\\	真実が勝つ。	
\\	[名詞]	粒子	
\\	専門は素粒子物理学です。	
\\	複数形チェック
\\	[名詞]	防止	
\\	9月10日は世界自殺予防デーです。	
\\	複数形チェック
\\	[名詞]	提携	
\\	複数形チェック
\\	[副詞]	主に	
\\	原書では主に英語を対象とした自然言語処理を取り扱っています。	
\\	[形容詞]	情熱的な	
\\	トムは熱烈なキスでメアリーを出迎えた。	
\\	[名詞]	全盛期	
\\	2539は素数です。	
\\	複数形チェック
\\	[名詞]	合いことば	
\\	合言葉?	
\\	複数形チェック
\\	[名詞]	印刷、印刷物	
\\	印刷術はいつ発明されましたか。	
\\	複数形チェック
\\	[名詞]	パスタ	
\\	パスタを食べますか。	
\\	複数形チェック
\\	[副詞]	ひそかに	
\\	[名詞]	のり状のもの	
\\	私はのりで二枚の紙を貼り合わせた。	
\\	複数形チェック
\\	[名詞]	行列	
\\	複数形チェック
\\	[名詞]	娯楽、趣味	
\\	あなたのお好きな娯楽は何ですか。	
\\	複数形チェック
\\	[動詞]	宣言する	
\\	その法律は即日施行された。	
\\	[名詞]	牧草地	
\\	牧場には羊がたくさんいる。	
\\	複数形チェック
\\	[形容詞]	生産的な	
\\	そんなことをしたら逆効果になる。	
\\	[副詞]	忍耐強く	
\\	[名詞]	生産性	
\\	一人当たりの生産性はどれくらいですか?	
\\	複数形チェック
\\	[名詞]	後援者	
\\	あの人はこの店の長年の馴染み客です。	
\\	複数形チェック
\\	[名詞]	人物紹介、横顔	
\\	プロフィール写真を変えたほうが良いと思うよ。	
\\	複数形チェック
\\	[動詞]	皮をむく	
\\	ジャガイモの皮をこそげ取ってください。	
\\	[形容詞]	進歩的な	
\\	ここで使われている文法形式は現在進行形である。	
\\	[名詞]	同僚	
\\	複数形チェック
\\	[動詞]	禁じる	
\\	ポイ捨ては禁止です。	
\\	[名詞]	知覚、理解	
\\	複数形チェック
\\	[形容詞]	著名な、顕著な	
\\	あいつ、おでこが広いだけだって言ってるけどさ、要は生え際が後退してるってことだろ。	
\\	[名詞]	役者、歌手	
\\	私たちはその演奏者に拍手を送った。	
\\	複数形チェック
\\	[形容詞]	迅速な、機敏な	
\\	彼は僕に即答をしつこく求めた。	
\\	[名詞]	香水	
\\	カーズ2で
\\	の歌が使われているらしい。	
\\	複数形チェック
\\	[副詞]	即座に	
\\	俺たちは小さなプレハブ小屋の中に荷物を置き、さっそく準備をする。	
\\	[動詞]	繁栄する	
\\	長寿と繁栄を。	
\\	[副詞]	著しく	
\\	父は病後、めっきり老け込んでしまった。	
\\	[形容詞]	繁栄している	
\\	観光場所以外はあまり栄えていないと思います	
\\	[名詞]	治療	
\\	かぜの治療には蜂蜜入りのホットレモンがよく効く。	
\\	複数形チェック
\\	[形容詞]	保護の	
\\	[名詞]	除去、取り外し	
\\	複数形チェック
\\	[形容詞]	
\\	の過去分詞	
\\	それ科学的に証明されてんの?	
\\	[名詞]	ルネサンス	
\\	ルネッサンス時代、ベニスの芸術は全盛期を迎えていました。	
\\	複数形チェック
\\	[名詞]	格言	
\\	好きな諺は何ですか?	
\\	複数形チェック
\\	[動詞]	更新する	
\\	この本を読んで、友情の大切さを改めて実感しました。	
\\	[名詞]	供給者、扶養者	
\\	複数形チェック
\\	[名詞]	賃貸料	
\\	バットルフィルドというゲームがレンタルサーバー買うことができるぞ。	
\\	複数形チェック
\\	[形容詞]	地方の	
\\	[動詞]	作り直す	
\\	音楽を頭の中で再生してみなさい。	
\\	[名詞]	供給、支給	
\\	複数形チェック
\\	[名詞]	共和党員	
\\	共和党の旗を初めてホワイトハウスにもっていたのはこの州出身の男性で、その政党は自己依存と個人の自由と国家の統一の価値観にもとづいていたことを思い出そう。	
\\	複数形チェック
\\	[形容詞]	慎重な、分別のある	
\\	用心深い鳥は木を選ぶ。思慮深い従者は主を選ぶ。	
\\	[名詞]	研究者	
\\	トムは日本文学の研究者だ。	
\\	複数形チェック
\\	[形容詞]	精神の	
\\	[名詞]	予約	
\\	予約してある。	
\\	複数形チェック
\\	[名詞]	心理学者	
\\	トムは医者じゃなくて心理学者だ。	
\\	複数形チェック
\\	[形容詞]	控えめな、内気な	
\\	このテーブルは予約済みです。	
\\	[名詞]	心理学	
\\	この本は心理学を取り扱っている。	
\\	複数形チェック
\\	[動詞]	辞職する	
\\	辞めます。	
\\	[名詞]	評判、知名度	
\\	それは売名行為だ。	
\\	複数形チェック
\\	[形容詞]	尊敬すべき	
\\	[名詞]	出版社、発行者	
\\	ジョンはその本を合衆国の出版社に注文した。	
\\	複数形チェック
\\	[動詞]	制限する	
\\	言論の自由が制限されている国もある。	
\\	[名詞]	出版業	
\\	複数形チェック
\\	[名詞]	制限	
\\	輸入規制は最近だいぶ緩められた。	
\\	複数形チェック
\\	[名詞]	プルオーバー	
\\	私は自分の黄色いプルオーバーが大好きだ。	
\\	複数形チェック
\\	[動詞]	再び始める	
\\	私は彼の経歴を疑わしいと思っている。	
\\	[動詞]	殴る(パンチする)	
\\	私は彼のあごに1発食らわしてやった。	
\\	[名詞]	退職	
\\	退職後、彼はガーデニングを始めた。	
\\	複数形チェック
\\	[形容詞]	時間を守る	
\\	遅れるなよ。	
\\	[名詞]	退却	
\\	ジャクソンは身をひくつもりはなかった。	
\\	複数形チェック
\\	[名詞]	ピラミッド	
\\	複数形チェック
\\	[名詞]	再生、回復	
\\	複数形チェック
\\	[動詞]	資格を得る	
\\	彼は昇進の資格がある。	
\\	[動詞]	復活させる	
\\	日本を再現してみた。	
\\	[動詞]	引用する	
\\	彼はよくミルトンの言葉を引用する。	
\\	[形容詞]	革命の	
\\	最も急進的な革命家が革命の翌日には保守になる。	
\\	[形容詞]	人種の、人種間の	
\\	[名詞]	肋骨	
\\	複数形チェック
\\	[名詞]	競馬、競走	
\\	トムはレースについてほとんど知らない。	
\\	複数形チェック
\\	[名詞]	富	
\\	富はあまたの苦痛を覆う。	
\\	複数形チェック
\\	[名詞]	ぼろ服	
\\	雑巾とタオルを一緒くたにしてはいけない。	
\\	複数形チェック
\\	[名詞]	騎手	
\\	棹立ちした栗毛の牝馬は嘶き、騎手を落とそうとした。	
\\	複数形チェック
\\	[名詞]	雨滴	
\\	複数形チェック
\\	[形容詞]	ばかげた	
\\	馬鹿な!	
\\	[名詞]	干しブドウ	
\\	レーズンはぶどうを乾燥させたものだ。	
\\	複数形チェック
\\	[名詞]	乗ること	
\\	乗馬は楽しい。	
\\	複数形チェック
\\	[形容詞]	でたらめの	
\\	被験者は無作為に抽出された。	
\\	[副詞]	当然に	
\\	[名詞]	ラズベリー	
\\	私はキイチゴが好きです。	
\\	複数形チェック
\\	[動詞]	すすぐ	
\\	口をゆすいでください。	
\\	[形容詞]	理性のある	
\\	この考え方は合理的ではない。	
\\	[形容詞]	昇る、上がる	
\\	物価が上昇している。	
\\	[動詞]	反応する	
\\	素早く対処する必要があった。	
\\	[形容詞]	危険な	
\\	リスクは大き過ぎる。	
\\	[副詞]	手頃な、適度な	
\\	そこそこ腕が立つようになるにはどれくらいの時間がかかるものでしょうか?	
\\	[名詞]	川辺	
\\	複数形チェック
\\	[名詞]	休み時間	
\\	昼食のため12時から1時まで1時間の休みがある。	
\\	複数形チェック
\\	[動詞]	どなる、大声で言う	
\\	雷鳴がとどろいた。	
\\	[名詞]	不景気	
\\	来年は不況が避けられませんよ。	
\\	複数形チェック
\\	[名詞]	強盗、盗難	
\\	警察はその強盗に関連のある容疑者を逮捕した。	
\\	複数形チェック
\\	[名詞]	レシピ、調理法	
\\	トムは母親のオートミールクッキーのレシピをメアリーに教えた。	
\\	複数形チェック
\\	[形容詞]	岩だらけの	
\\	トムは波乱のスタートを切った。	
\\	[名詞]	新人	
\\	マスコミがリクルート事件を調査した結果数多くの汚職が明るみに出た。	
\\	複数形チェック
\\	[名詞]	恋愛	
\\	私は恋愛小説が好きです。	
\\	複数形チェック
\\	[動詞]	再生利用する	
\\	再利用するの?	
\\	[形容詞]	バラ色の	
\\	[動詞]	磨く、洗練する	
\\	トムは教養がある。	
\\	[名詞]	日課	
\\	英語の日課はだいぶ削減した。	
\\	複数形チェック
\\	[名詞]	反射	
\\	僕は湖に映った自分の姿を見てるんです。	
\\	複数形チェック
\\	[副詞]	無作法に	
\\	あんまり無礼に扱われたので、カチンときた。	
\\	[動詞]	再び元気づける	
\\	シャワーを浴びたら、気分がスッキリした。	
\\	[名詞]	ラグビー	
\\	彼はラグビーが得意だ。	
\\	複数形チェック
\\	[名詞]	避難	
\\	嵐を避けて近くの納屋に逃げ込んだ。	
\\	複数形チェック
\\	[形容詞]	神聖な	
\\	牛はヒンズー教徒にとって神聖な動物です。	
\\	[名詞]	難⺠、避難⺠	
\\	彼は難民救済基金に1万ドルを寄付した。	
\\	複数形チェック
\\	[名詞]	鞍、鞍下肉	
\\	複数形チェック
\\	[名詞]	拒否	
\\	あっさり断られると思いきや、彼女は承諾してくれました。	
\\	複数形チェック
\\	[名詞]	店員	
\\	彼は店員に金を渡した。	
\\	複数形チェック
\\	[副詞]	それでも、とにかく	
\\	我々はしゃにむに突き進んだ。	
\\	[名詞]	販売員	
\\	複数形チェック
\\	[形容詞]	地方の	
\\	インドは地域色豊かな食文化があります。	
\\	[名詞]	販売員	
\\	複数形チェック
\\	[動詞]	登録する	
\\	被害届を出すためにきたのか?	
\\	[形容詞]	砂の	
\\	レディ・ガガはハリケーン・サンディの被災者救済のため、赤十字に百万ドルを寄付した。	
\\	[動詞]	調節する、規制する	
\\	[名詞]	衛星	
\\	月は地球のただ一つの衛星である。	
\\	複数形チェック
\\	[名詞]	くつろぐこと	
\\	愚かさは知性の休息である。	
\\	複数形チェック
\\	[形容詞]	満足のいく	
\\	あなたの話は全然満足のいかない物だ。	
\\	[動詞]	和らげる	
\\	安心しました。	
\\	[形容詞]	満足な	
\\	[形容詞]	ほっとした	
\\	安心しました。	
\\	[名詞]	傷あと	
\\	腕の傷は跡を残した。	
\\	複数形チェック
\\	[形容詞]	嫌がる	
\\	トムは渋っている。	
\\	[形容詞]	不足して	
\\	良質の果物が冬には不足し、値段も高くなる。	
\\	[形容詞]	恐ろしい	
\\	何が怖いの?	
\\	[副詞]	明確に、はっきりと	
\\	う~ん、わからんか。まあこれはあまり簡単な問題ばかりだったら失礼だから、君用に出しただけだから。余裕があったら覚えといて。	
\\	[名詞]	教科書	
\\	複数形チェック
\\	[名詞]	観客	
\\	熱狂的な観客が競技場になだれ込んだ。	
\\	複数形チェック
\\	[名詞]	校舎	
\\	複数形チェック
\\	[名詞]	球形、球体、領域	
\\	神は無限の球である。その中心は到るところにあり、その円周はどこにもない。	
\\	複数形チェック
\\	[名詞]	男子教員	
\\	経験は最良の教師である。ただし授業料が高い。	
\\	複数形チェック
\\	[形容詞]	正々堂々した	
\\	[名詞]	教室	
\\	複数形チェック
\\	[動詞]	まき散らす	
\\	雨がぱらぱら降り出しました。	
\\	[名詞]	学校教師	
\\	この辺で、むかし松本訓導という優しい先生が、教え子を救おうとして、かえって自分が溺死なされた。	
\\	複数形チェック
\\	[名詞]	リス	
\\	リスです。	
\\	複数形チェック
\\	[名詞] [動詞] [形容詞]	がらくた(にする)、スクラップ(する)、がらくたである	
\\	そんな計画は捨ててしまいなさい。	
\\	複数形チェック
\\	[名詞]	汚れ	
\\	君のカラーにはしみがついている。	
\\	複数形チェック
\\	[名詞]	彫刻	
\\	複数形チェック
\\	[名詞]	身分、立場	
\\	彼は立っていた。	
\\	複数形チェック
\\	[副詞]	一見したところ	
\\	[動詞]	びっくりさせる	
\\	驚くじゃありませんか!	
\\	[名詞]	上院	
\\	複数形チェック
\\	[名詞]	政治家	
\\	リンカーンは偉大な政治家であった。	
\\	複数形チェック
\\	[名詞]	感情、情緒	
\\	メアリーは気持ちのやさしい女の子だ。	
\\	複数形チェック
\\	[副詞]	着実に	
\\	景気は着実に悪くなった。	
\\	[副詞]	別々に	
\\	彼らは支払いを別々にした。	
\\	[動詞]	舵をとる	
\\	[名詞]	別離	
\\	会うは別れの始め。	
\\	複数形チェック
\\	[形容詞]	厳しい	
\\	彼女は怖い顔をして彼を黙らせた。	
\\	[名詞]	真剣	
\\	複数形チェック
\\	[動詞]	刺激する、促す	
\\	私はその講義には本当に刺激を受けた。	
\\	[副詞]	厳しく	
\\	彼は厳しく市長を講評した。	
\\	[名詞]	ひと針、縫い目	
\\	転ばぬ先の杖。	
\\	複数形チェック
\\	[形容詞]	挑発的な	
\\	彼はとてもセクシーだ。	
\\	[名詞]	腹痛	
\\	おなかが痛い。	
\\	複数形チェック
\\	[動詞] [不規則動詞]	発する、放つ	
\\	彼女は涙を流した。	
\\	[名詞]	革ひも、ひも	
\\	これはストラップです。	
\\	複数形チェック
\\	[名詞]	シリング	
\\	挨拶より円札。	
\\	複数形チェック
\\	[名詞]	戦略、計略	
\\	複数形チェック
\\	[形容詞]	輝く、光沢のある	
\\	メアリーの髪は輝くようなブロンドの髪だ。	
\\	[名詞]	路面電車	
\\	市電は混んでいた。	
\\	複数形チェック
\\	[名詞]	発送、船積み	
\\	複数形チェック
\\	[動詞]	強化する	
\\	基礎を固めることは大切です。	
\\	[形容詞]	ぞっとするような	
\\	[名詞]	大股に歩くこと	
\\	複数形チェック
\\	[名詞]	靴屋	
\\	複数形チェック
\\	[名詞]	縞	
\\	複数形チェック
\\	[動詞]	短くする	
\\	短くしてもらわなければなりません。	
\\	[動詞] [不規則動詞]	努力する	
\\	商品の品質には万全を期しておりますが、万一商品が破損・汚損していた場合、またはご注文と異なる場合は、お届け後7日以内にご連絡ください。	
\\	[名詞]	スコップ、シャベル	
\\	シャベル持ってきて。	
\\	複数形チェック
\\	[形容詞]	頑固な	
\\	彼女は強情です。	
\\	[動詞] [不規則動詞]	縮む	
\\	彼は危険をものともしない。	
\\	[動詞]	よろめく	
\\	どうして僕はこんな不器用な生き方しかできないんだろう。	
\\	[名詞]	観光	
\\	町中見物でもするか?	
\\	複数形チェック
\\	[形容詞]	流行の	
\\	これはあまりおしゃれじゃない。	
\\	[名詞]	意義	
\\	今日は、現代社会におけるスポーツの重要性についてお話しします。	
\\	複数形チェック
\\	[形容詞]	実体のある	
\\	この機会のおかげで私たちは大いに手間暇が省ける。	
\\	[名詞]	似ている点	
\\	複数形チェック
\\	[動詞]	代わりに用いる	
\\	彼は父親の代理をした。	
\\	[副詞]	同様に	
\\	人が肉体を持ち、地上に生まれたからといって、地上のことを全て理解するわけではないのと同様に、死んだからといって霊的世界を全て理解しているわけでもないのです。	
\\	[動詞]	訴訟を起こす	
\\	彼女は彼を訴えた。	
\\	[名詞]	単数	
\\	この国から遠く離れたところで今夜を見つめているみなさん。外国の議会や宮殿で見ているみなさん、忘れ去られた世界の片隅でひとつのラジオの周りに身を寄せ合っているみなさん、私たちの物語はそれぞれ異なります。けれども私たちはみな、ひとつの運命を共有しているのです。アメリカのリーダーシップはもうすぐ、新たな夜明けを迎えます。	
\\	複数形チェック
\\	[名詞]	苦しむ人	
\\	複数形チェック
\\	[形容詞]	置かれた	
\\	日本はアジアにあります。	
\\	[名詞]	自殺	
\\	彼は自殺をした。	
\\	複数形チェック
\\	[形容詞]	熟練した	
\\	トムは時間の使い方がうまい。	
\\	[名詞]	夏時間	
\\	複数形チェック
\\	[動詞]	平手で打つ	
\\	彼は彼女を殴った。	
\\	[動詞]	召喚する	
\\	勇気を出してメアリーを映画に誘ったのに、すっぽかされてしまった。	
\\	[形容詞]	細⻑い	
\\	トムは着痩せするタイプだ。	
\\	[名詞]	サングラス	
\\	私はサングラスを持っている。	
\\	複数形チェック
\\	[形容詞]	滑りやすい	
\\	床滑るから気をつけてね。	
\\	[形容詞]	上位の、上級の	
\\	上司に許可を取ってきます。	
\\	[動詞]	ひったくる	
\\	[名詞]	くしゃみ	
\\	よくくしゃみが出ます。	
\\	複数形チェック
\\	[形容詞]	最高権威の	
\\	最高裁は原判決をくつがえしました。	
\\	[動詞]	舞い上がる	
\\	旅行会社の収益が急増した。	
\\	[名詞]	サーフィン	
\\	サーフィンは好きですか。	
\\	複数形チェック
\\	[動詞]	すすり泣く	
\\	その迷子の女の子は泣きじゃくりながら名前を言った。	
\\	[名詞]	外科医	
\\	彼は優秀な脳外科医だ。	
\\	複数形チェック
\\	[動詞]	軟化する	
\\	私に対する彼の怒りは和らいでいない。	
\\	[名詞]	外科、手術	
\\	手術を受けなくてはならない。	
\\	複数形チェック
\\	[形容詞]	唯一の	
\\	それは唯一の関心事です。	
\\	[副詞]	驚くほど	
\\	トムはいつも驚くほど仕事が早い。	
\\	[副詞]	単独で	
\\	トムはコーランを原文で読んで理解したいと思い、それだけのためにアラビア語の勉強を始めた。	
\\	[名詞]	生き残り	
\\	危険を避けて安全に生きるのが生存本能でしょうか。	
\\	複数形チェック
\\	[副詞]	固く	
\\	[名詞]	遺族	
\\	夜通しの豪雨は、被災したフェリーからの生存者の救出を妨げた。	
\\	複数形チェック
\\	[名詞]	孤独、寂しさ	
\\	独りは良いものだ。だが「独りは良い」と言うには相手が必要だ。	
\\	複数形チェック
\\	[動詞]	一時中断する	
\\	学校長はトムを2日間の停学にした。	
\\	[名詞]	単独	
\\	トムは一人旅に出かけた。	
\\	複数形チェック
\\	[名詞]	疑い、容疑	
\\	彼は明らかに潔白である。	
\\	複数形チェック
\\	[副詞]	何時か	
\\	いつか負かしてやる!	
\\	[形容詞]	疑っている	
\\	あいつが臭い。	
\\	[動詞] [不規則動詞]	種をまく	
\\	瓜のつるに茄子はならぬ。	
\\	[動詞]	持続させる	
\\	栄養のある朝食をとった。	
\\	[名詞]	きらめき、火花	
\\	複数形チェック
\\	[形容詞]	速い、迅速な	
\\	ジョナサン・スウィフトなら私を弁護するでしょう。	
\\	[名詞]	スズメ	
\\	其の日は一日淋しいペンの音を聞いて暮した。其の間には折々千代々々と云う声も聞えた。文鳥も淋しいから鳴くのではなかろうかと考えた。然し縁側へ出て見ると、二本の留り木の間を、彼方へ飛んだり、此方へ飛んだり、絶間なく行きつ戻りつしている。少しも不平らしい様子はなかった。	
\\	複数形チェック
\\	[名詞]	交響曲	
\\	ロンドンの交響楽団のコンサートスケジュールを知りませんか。	
\\	複数形チェック
\\	[動詞]	専門とする	
\\	[形容詞]	才能がある	
\\	彼女は歌を歌うのが上手です。	
\\	[動詞]	いじめる	
\\	トムはメアリーをからかった。	
\\	[動詞]	暴露する	
\\	[副詞]	厳密に言えば	
\\	[動詞] [不規則動詞]	経験する	
\\	彼は危険な手術を受けた。	
\\	[動詞]	誘惑する	
\\	心が動くね。	
\\	[名詞]	大学の学部生	
\\	複数形チェック
\\	[副詞]	やさしく、親切に	
\\	[動詞] [不規則動詞]	引き受ける	
\\	[形容詞]	素晴らしい	
\\	すごいぞ!	
\\	[名詞]	失業	
\\	高い失業率の責任は誰にあるのか。	
\\	複数形チェック
\\	[名詞]	テロリスト	
\\	米国の核廃棄物、テロ攻撃の標的になりうる。	
\\	複数形チェック
\\	[副詞]	思いがけなく	
\\	おじが神戸よりひょっこりやってきた。	
\\	[名詞]	検査	
\\	複数形チェック
\\	[形容詞]	不公平な、不当な	
\\	それは公平ではない。	
\\	[名詞]	感謝祭	
\\	11月23日は勤労感謝の日で、勤労の大切さを伝えるために制定された祝日です。	
\\	複数形チェック
\\	[形容詞]	なじみのない	
\\	その部族は祖先を崇拝し、私たちにはなじみのない独自の言語を話す。	
\\	[名詞]	主題、テーマ	
\\	彼の最新の小説は何ですか。	
\\	複数形チェック
\\	[形容詞]	重要でない	
\\	俺はなんの意味もない無価値な存在。	
\\	[名詞]	温度計、体温計	
\\	温度は七十度です。	
\\	複数形チェック
\\	[名詞]	統一、団結	
\\	まとまりは我々の力だ!	
\\	複数形チェック
\\	[名詞]	思想家	
\\	生前に無名であった多くの偉大な思想家は、死後に名声を得た。	
\\	複数形チェック
\\	[形容詞]	不公平な	
\\	[副詞]	薄く	
\\	[形容詞]	同じでない	
\\	定住してる生活するほとんどのイタチ科と違って、クズリは、絶えず餌食を捜して、1500-2000平方キロまでを占める自分の土地の上で歩き回る。	
\\	[名詞]	とげ、棘毛	
\\	喉に刺さった棘の如く。	
\\	複数形チェック
\\	[形容詞]	不愉快な	
\\	この果物はいやなにおいがする。	
\\	[形容詞]	考え込んだ	
\\	お、ありがと。気が利くね。	
\\	[副詞]	一風変わって	
\\	いつになくよく食べるね。	
\\	[名詞]	ワクワクさせるもの	
\\	複数形チェック
\\	[形容詞]	好まなくて	
\\	彼は行きたくなかった。	
\\	[形容詞]	ぞくぞくする	
\\	何あんた、またパン?相変わらず変わり映えのしない食生活送ってんのね。	
\\	[形容詞]	都会の	
\\	自転車は都会の環境を破壊することなく資源利用を持続することのできる道具である。	
\\	[名詞]	王位	
\\	誰が王位を継ぐのですか。	
\\	複数形チェック
\\	[形容詞]	まったくの	
\\	彼はまったくの見知らぬ人だった。	
\\	[動詞]	強化する	
\\	守りを固めろ!来るぞ!	
\\	[副詞]	全く	
\\	堅い書き物は大嫌いだ。	
\\	[名詞]	タイツ	
\\	複数形チェック
\\	[名詞]	真空	
\\	ネコは掃除機が大嫌い。	
\\	複数形チェック
\\	[形容詞]	時を超越した	
\\	この映画はまさしく不朽の名作である。	
\\	[名詞]	おおい、口実	
\\	複数形チェック
\\	[名詞]	(生物の)組織	
\\	トムにティッシュをあげて。	
\\	複数形チェック
\\	[名詞]	ベランダ	
\\	其の日は一日淋しいペンの音を聞いて暮した。其の間には折々千代々々と云う声も聞えた。文鳥も淋しいから鳴くのではなかろうかと考えた。然し縁側へ出て見ると、二本の留り木の間を、彼方へ飛んだり、此方へ飛んだり、絶間なく行きつ戻りつしている。少しも不平らしい様子はなかった。	
\\	複数形チェック
\\	[動詞]	耐える	
\\	やんちゃな子供が堪えられない。	
\\	[名詞]	動詞	
\\	完全自動詞は補語も目的語もとらない。	
\\	複数形チェック
\\	[名詞]	⻭痛	
\\	歯が痛いです。	
\\	複数形チェック
\\	[形容詞]	勝利を得た	
\\	[名詞]	トッピング	
\\	複数形チェック
\\	[名詞]	観点、見解	
\\	下手に独自の方向性を出すと、攻撃されてしまうから、安全パイの報道しかしない。	
\\	複数形チェック
\\	[名詞]	懐中電灯	
\\	複数形チェック
\\	[名詞]	精力、気力	
\\	複数形チェック
\\	[形容詞]	感動的な	
\\	腕を離せ。人に触れられるのにはがまんがならんのだ。	
\\	[名詞]	ブドウの木	
\\	瓜のつるに茄子はならぬ。	
\\	複数形チェック
\\	[名詞]	トーナメント	
\\	20チームがトーナメントに出場した。	
\\	複数形チェック
\\	[動詞]	違反する	
\\	憲法を侵してはならない。	
\\	[名詞]	トレーダー	
\\	あやうく某業者の甘言に騙され、大損するところでした。	
\\	複数形チェック
\\	[副詞]	激しく、猛烈に	
\\	[名詞]	取引	
\\	日本との貿易は容易ではない。	
\\	複数形チェック
\\	[形容詞]	きわめて重大な	
\\	十分な栄養摂取は幼児の発育にとって極めて重要です。	
\\	[名詞]	悲劇、惨事	
\\	核の悲劇が起きる危険は非常に高そうです。	
\\	複数形チェック
\\	[形容詞]	生き生きとした	
\\	[形容詞]	悲劇的な	
\\	悲惨!	
\\	[名詞]	火山	
\\	火山を見たいよ。	
\\	複数形チェック
\\	[動詞]	変える	
\\	「そっかぁ、アルクさんはワーウルフなんですね」「俺は混血だから変身したりはしねーよ」	
\\	[名詞]	誓い	
\\	私は夫であるドン・ホセに誓った婚姻の誓約を破ったのです。	
\\	複数形チェック
\\	[名詞]	移行、移り変わり	
\\	『カムイの剣』は、1868年の徳川将軍時代の崩壊と、明治天皇下での日本の復興という変革期を舞台にした、一種の侍/忍者物語だ。	
\\	複数形チェック
\\	[動詞]	目を覚まさせる	
\\	汽車の汽笛で夜明けに目を覚まされた。	
\\	[動詞] [不規則動詞]	踏む	
\\	それで雪を踏み固めてブロックを作り、それを積むような方法が適しているんですね。	
\\	[名詞]	病棟	
\\	本当に暇なら本を読め。区立図書館で借りて。	
\\	複数形チェック
\\	[名詞]	流行の先端を行く人	
\\	複数形チェック
\\	[名詞]	温めること	
\\	ヒーターが部屋を暖めている。	
\\	複数形チェック
\\	[動詞]	切る	
\\	彼は結婚式のために、あごひげの手入れをした。	
\\	[名詞]	戦士	
\\	すべての兵士を一騎当千のつわものに育てるカギは訓練を強化することである。	
\\	複数形チェック
\\	[名詞]	トロンボーン	
\\	複数形チェック
\\	[名詞]	戦時	
\\	複数形チェック
\\	[名詞]	軍隊	
\\	その軍隊は十分な武器を持っていた。	
\\	複数形チェック
\\	[副詞]	完全に	
\\	[名詞]	たらい	
\\	男の子はたらいの中でジャブジャブ水をはねかしていた。	
\\	複数形チェック
\\	[形容詞]	広範囲に渡った	
\\	「ドイツワインは甘口が多く料理に合わない」というイメージが広まっている。	
\\	[名詞]	⻲	
\\	亀には歯が無いです。	
\\	複数形チェック
\\	[名詞]	未亡人	
\\	夫に死なれた女性は未亡人と呼ばれる。	
\\	複数形チェック
\\	[名詞]	夕暮れ	
\\	複数形チェック
\\	[名詞]	ワイングラス	
\\	彼はワイングラスを落として、その足を折ってしまった。	
\\	複数形チェック
\\	[動詞]	きらきら光る	
\\	夜空に星がきらきら輝いていた。	
\\	[名詞]	無線	
\\	複数形チェック
\\	[名詞]	台風	
\\	台風は去った。	
\\	複数形チェック
\\	[名詞]	機知	
\\	その本を書いた人は、機知とユーモアを兼ね備えた人だね。	
\\	複数形チェック
\\	[副詞]	典型的に	
\\	[名詞]	魔女	
\\	私は魔女ではありません。	
\\	複数形チェック
\\	[形容詞]	最後の	
\\	教育の究極の目標とは何だろうか。	
\\	[動詞] [不規則動詞]	引き出す	
\\	議案を撤した。	
\\	[形容詞]	容認できない	
\\	こんなことは許されない。	
\\	[名詞]	森林地	
\\	そうだ、どうせなら散歩がてらに、林道に行ってプチ森林浴でも・・・。	
\\	複数形チェック
\\	[形容詞]	気づかない	
\\	彼は事態を知らなかった。	
\\	[名詞]	作業場	
\\	複数形チェック
\\	[形容詞]	信じられない	
\\	信じられない!	
\\	[形容詞]	世界的な	
\\	逆境にもかかわらず、その建築家は世界的名声を獲得した。	
\\	[名詞]	不確かさ	
\\	昨年は経済的に不確実な時代でした。	
\\	複数形チェック
\\	[名詞]	シマウマ	
\\	複数形チェック
\\	[名詞]	珍しい	
\\	90歳以上生きることは決してまれではない。	
\\	複数形チェック
\\	[名詞]	ファスナー	
\\	彼は鞄のファスナーを開けた。	
\\	複数形チェック
\\	[名詞]	妊娠中絶	
\\	あなたは妊娠中絶に賛成ですか、反対ですか。	
\\	複数形チェック
\\	[名詞]	陸上競技	
\\	複数形チェック
\\	[形容詞]	突然の	
\\	そのエレベーターは急停止した。	
\\	[名詞]	出席	
\\	次回につづく、ということで、どうかお付き合いのほど、平にお願いいたします。	
\\	複数形チェック
\\	[形容詞]	接近できる	
\\	その町には列車で行ける。	
\\	[形容詞]	注意深い	
\\	彼女はおばあちゃんをたいそう大事にする。	
\\	[副詞]	偶然に	
\\	僕は自分の傘をなくしてしまった。	
\\	[動詞]	〈結果を〉
\\	に〕帰する	
\\	この絵はモネの作とされている。	
\\	[動詞]	収容する	
\\	このホテルは500人の客を収容できる。	
\\	[動詞]	権限を与える	
\\	関係者以外立ち入り禁止。	
\\	[名詞]	会計士	
\\	アラン君は運良く税理士試験に合格しました。	
\\	複数形チェック
\\	[名詞]	赤ん坊	
\\	複数形チェック
\\	[名詞]	会計、決算	
\\	蓼食う虫も好き好き。	
\\	複数形チェック
\\	[名詞]	背骨、背	
\\	複数形チェック
\\	[動詞]	蓄積する	
\\	彼は蔵書をふやした。	
\\	[名詞]	細菌	
\\	細菌はただの小さい細胞です、無害です。	
\\	複数形チェック
\\	[名詞]	正確さ	
\\	算数では正確さが重要だ。	
\\	複数形チェック
\\	[名詞]	バドミントン	
\\	バドミントン部はまともな部員が極端に少ないため、ほとんど廃部ないし休部状態だった。	
\\	複数形チェック
\\	[名詞]	告発	
\\	彼を告訴する理由は何ですか。	
\\	複数形チェック
\\	[形容詞]	禿げた	
\\	トムには十円ハゲがある。	
\\	[名詞]	名手	
\\	ボク、クローバーのエー、持ってるんだ。	
\\	複数形チェック
\\	[動詞]	追放する	
\\	彼は大逆罪で島流しにされた。	
\\	[動詞]	認める	
\\	私は自分の誤りを認める。	
\\	[形容詞]	破産した	
\\	あの会社は倒産した。	
\\	[名詞]	活動家	
\\	複数形チェック
\\	[名詞]	垂れ幕,横断幕	
\\	多くの人がこの自由の旗印の下に集った。	
\\	複数形チェック
\\	[形容詞]	適切な	
\\	3人にはこの食事で十分だ。	
\\	[名詞]	宴会、祝宴	
\\	複数形チェック
\\	[名詞]	形容詞	
\\	名詞を修飾するもの(形容詞、または形容詞相当語句)。	
\\	複数形チェック
\\	[名詞]	バーベキュー	
\\	海岸でバーベキューをしよう。	
\\	複数形チェック
\\	[形容詞]	管理の	
\\	管理用画面に管理者しかアクセスできないようにしました。	
\\	[名詞]	交渉、取引	
\\	複数形チェック
\\	[名詞]	管理者	
\\	管理人によってアカウントの認証が行われたあと、あなたに確認のメールが送られます。	
\\	複数形チェック
\\	[名詞]	バザー	
\\	毎週日曜日午後は、店へ行きます。	
\\	複数形チェック
\\	[形容詞]	賞賛に値する	
\\	彼の振る舞いは見事だった。	
\\	[名詞]	くちばし	
\\	複数形チェック
\\	[名詞]	ファン、愛慕者	
\\	トムは自分に隠れファンがいることを知らなかった。	
\\	複数形チェック
\\	[名詞]	打ちたたく器具	
\\	複数形チェック
\\	[名詞]	副詞	
\\	しかしながら、前置詞
\\	関係代名詞
\\	の部分が関係副詞
\\	となっています。	
\\	複数形チェック
\\	[副詞]	事前に	
\\	事前に弁護士と相談した方がいいよ。	
\\	[名詞]	広告主	
\\	複数形チェック
\\	[形容詞]	有益な、ためになる	
\\	にんにくを一日ひとかけ食べるのは健康にプラスかな?	
\\	[名詞]	アフターケア	
\\	複数形チェック
\\	[名詞]	裏切り	
\\	彼女の微笑みの裏には裏切りがある。	
\\	複数形チェック
\\	[感動詞]	ははあ、へえ!	
\\	ああ、君がやったんだね。	
\\	[形容詞]	二か国語併用の	
\\	一般に利用できる英日2言語コーパスを作成した。	
\\	[名詞]	定期旅客機	
\\	旅客機のパイロットは乗客の安全に対し責任がある。	
\\	複数形チェック
\\	[名詞]	伝記	
\\	良い伝記はおもしろくて、ためにもなる。	
\\	複数形チェック
\\	[感動詞]	ああ、あーあ	
\\	ああ、彼女は若くして逝ってしまった。	
\\	[形容詞]	鋭い、痛烈な	
\\	トムは爪を噛む癖がある。	
\\	[形容詞]	アルコールの	
\\	彼女はアルコール中毒だ。	
\\	[名詞]	恨み、敵意	
\\	複数形チェック
\\	[名詞]	路地、横町	
\\	その後、この横町で、ふたたび鬼婆のすがたを認めたという者はなかった。	
\\	複数形チェック
\\	[動詞]	爆破する	
\\	今何時だと思ってるんだ?こんな時間に音楽をガンガンかけたら近所迷惑なのがわからないのか。	
\\	[動詞]	同盟する	
\\	ドイツはかつてイタリアの同盟国であった。	
\\	[動詞]	漂白する	
\\	トムは茶髪より黒髪の方が似合うと思う。	
\\	[形容詞]	全能の	
\\	[形容詞]	祝福を受けた	
\\	司祭は幸せな二人の結婚を祝福した。	
\\	[副詞]	そばに	
\\	街道沿いに家が並んでいた。	
\\	[名詞]	汚点	
\\	私はインクでこの用紙にしみをつけてしまった。	
\\	複数形チェック
\\	[形容詞]	アルファベットの	
\\	先生は生徒の名前をアルファベット順に呼んだ。	
\\	[名詞]	設計図	
\\	複数形チェック
\\	[名詞]	アルミニウム	
\\	アルミニウム、マグネシウム、シリコンなどの粉末が爆発する。	
\\	複数形チェック
\\	[形容詞]	肉体の	
\\	傷害罪ってどの程度の怪我を負わせると傷害罪に問われますか?	
\\	[形容詞]	十分な	
\\	列車に十分間に合うだけの時間がある。	
\\	[形容詞]	沸き立つ	
\\	薬缶のお湯が沸いている。	
\\	[名詞]	分析者	
\\	複数形チェック
\\	[名詞]	ずうずうしさ	
\\	複数形チェック
\\	[形容詞]	うるさい	
\\	彼はうっとうしいやつだ。	
\\	[名詞]	予約	
\\	複数形チェック
\\	[名詞]	予想、期待	
\\	複数形チェック
\\	[名詞]	にわか景気,ブーム	
\\	日本経済は当時、前例のない好況にあった。	
\\	複数形チェック
\\	[名詞]	類人猿	
\\	類人猿は知能が高い。	
\\	複数形チェック
\\	[動詞]	押し上げる	
\\	このビタミンは病気に対する抵抗力を強めます。	
\\	[動詞]	拍手を送る	
\\	トムは拍手した。	
\\	[名詞]	売店	
\\	一番近くの電話ボックスはどこですか。	
\\	複数形チェック
\\	[名詞]	拍手かっさい、賞賛	
\\	拍手が湧き上がった	
\\	複数形チェック
\\	[動詞]	ボールがはずむ	
\\	トムは元気を取り戻した。	
\\	[形容詞]	応用された	
\\	ビザを申請した。	
\\	[名詞]	ボイコット	
\\	複数形チェック
\\	[形容詞]	建築上の	
\\	[名詞]	勇敢さ	
\\	我々は彼女の勇気に感心する。	
\\	複数形チェック
\\	[名詞]	肘掛け椅子	
\\	俺はアームチェアーに戻って本を開いた。	
\\	複数形チェック
\\	[形容詞]	〔ニュースなどが〕最新の	
\\	夜が明けようとしている。	
\\	[名詞]	上昇、昇進	
\\	複数形チェック
\\	[名詞]	賄賂	
\\	彼女は彼が賄賂を受け取るだろうと思った。	
\\	複数形チェック
\\	[動詞]	評価する	
\\	あなたはどのようにして学生を評価しますか。	
\\	[動詞]	明るくする	
\\	空がだんだん明るくなってきた。	
\\	[名詞]	評価、査定	
\\	複数形チェック
\\	[名詞]	輝き	
\\	部屋の明かりのスイッチをカチッとつけた。	
\\	複数形チェック
\\	[名詞]	資産	
\\	会社の資産を評価するのはスコットさんの仕事です。	
\\	複数形チェック
\\	[名詞]	すばらしさ	
\\	複数形チェック
\\	[名詞]	驚き	
\\	彼女は驚いて、彼をじっと見つめた。	
\\	複数形チェック
\\	[名詞]	小川	
\\	小川では子供たちが遊んでいるのが見えました。	
\\	複数形チェック
\\	[名詞]	宇宙飛行士	
\\	僕は宇宙飛行士になりたい。	
\\	複数形チェック
\\	[名詞]	ほうき	
\\	複数形チェック
\\	[名詞]	天文学者	
\\	天文学者が局部超銀河団の存在を推定しました。	
\\	複数形チェック
\\	[名詞]	仏教	
\\	彼は仏教からキリスト教へ改宗した。	
\\	複数形チェック
\\	[名詞]	天文学	
\\	僕は天文学が大好きだよ。	
\\	複数形チェック
\\	[名詞]	仏教徒	
\\	彼は木から仏像を彫った。	
\\	複数形チェック
\\	[名詞]	建築者	
\\	複数形チェック
\\	[名詞]	彗星(すいせい)	
\\	複数形チェック
\\	[名詞]	埋葬	
\\	なぜ、ムスリムの人たちは、土葬にこだわるのか。	
\\	複数形チェック
\\	[形容詞]	こっけいな	
\\	忘年会の余興で大喜利をやることになった。	
\\	[形容詞]	実務的な	
\\	[動詞]	開始する	
\\	[動詞]	ハチがブンブン飛ぶ	
\\	一匹の蜜蜂がブンブンと飛んでいる。	
\\	[名詞]	日用品、商品	
\\	古代において塩は希少で貴重な商品であった。	
\\	複数形チェック
\\	[名詞]	計算機	
\\	この計算機は調子が良くない。	
\\	複数形チェック
\\	[形容詞]	匹敵する	
\\	[名詞]	訪問者	
\\	複数形チェック
\\	[形容詞]	比較の	
\\	[名詞]	冷静	
\\	複数形チェック
\\	[副詞]	比較的に	
\\	モールの中に、比較的大きなシネコンがある。	
\\	[名詞]	キャンプ場	
\\	複数形チェック
\\	[名詞]	コンパス	
\\	彼らは方位磁石のようなものをもっているのですか。	
\\	複数形チェック
\\	[名詞]	大砲	
\\	複数形チェック
\\	[動詞]	強制する	
\\	ひどい雨のために私たちは出発を延ばさねばならなかった。	
\\	[名詞]	峡谷	
\\	グランド・キャニオンはアメリカ有数の人気スポットだ。	
\\	複数形チェック
\\	[名詞]	賠償	
\\	複数形チェック
\\	[名詞]	能力	
\\	複数形チェック
\\	[形容詞]	有能な	
\\	彼はやり手だと思う。	
\\	[名詞]	愛撫	
\\	複数形チェック
\\	[名詞]	積荷	
\\	複数形チェック
\\	[名詞]	完成、完了	
\\	この建物は完成間近だ。	
\\	複数形チェック
\\	[名詞]	運搬人	
\\	複数形チェック
\\	[名詞]	複雑さ、複雑なもの	
\\	複数形チェック
\\	[名詞]	漫画、アニメ	
\\	この漫画は知っていますか?	
\\	複数形チェック
\\	[動詞]	複雑にする	
\\	文法はとてもややこしい。	
\\	[名詞]	レジ係	
\\	複数形チェック
\\	[名詞]	複雑、面倒な事態	
\\	複数形チェック
\\	[副詞]	偶然に	
\\	彼は愛想のいい奴だが気軽に踏み込ませない一線を持っていた。	
\\	[名詞]	ほめ言葉、お世辞	
\\	それは誉め言葉として受け取っておこう。	
\\	複数形チェック
\\	[名詞]	チェロ	
\\	オルガは8歳のときからチェロを弾いている。	
\\	複数形チェック
\\	[形容詞]	落ち着いた	
\\	1チームは11人の選手で構成されている。	
\\	[名詞]	墓地	
\\	ケネディはアーリントン墓地に埋葬された。	
\\	複数形チェック
\\	[動詞]	理解する	
\\	しかし貴方は分からない。老いを感じることがどういうことなのか理解できる者は誰もいない。	
\\	[名詞]	確実性、確信	
\\	複数形チェック
\\	[形容詞]	強制的な	
\\	英語は必修科目です。	
\\	[名詞]	証明書	
\\	もし予防接種証明書があったらお持ちください。	
\\	複数形チェック
\\	[名詞]	概念、考え	
\\	トムは、恋愛をするということがどんな感じなのか全くわかってない。	
\\	複数形チェック
\\	[名詞]	議⻑	
\\	マイクは議長に選出された。	
\\	複数形チェック
\\	[形容詞]	条件付きの	
\\	学生が英文論文誌に論文を投稿して、「条件付採録」になっています。	
\\	[名詞]	挑戦者	
\\	1986年1月28日に、チャレンジャーが爆発しました。	
\\	複数形チェック
\\	[名詞]	コンドーム	
\\	コンドーム持ってる?	
\\	複数形チェック
\\	[形容詞]	変わりやすい	
\\	今日は一日中、雨が降っては止み、降っては止みする、はっきりしない天気だ。	
\\	[動詞]	お祝いを述べる	
\\	私は彼に、男子出産のお祝いを述べた。	
\\	[動詞]	特徴づける	
\\	[形容詞]	かなり	
\\	以前と比べると状況は随分良くなった。	
\\	[副詞]	安く	
\\	これだけで厄介払いできたら安いもので、みな大喜びした。	
\\	[形容詞]	建設的な	
\\	人の揚げ足を取るばかりじゃなく、もっと建設的な意見を言ってくれないかな。	
\\	[名詞]	安っぽさ	
\\	複数形チェック
\\	[名詞]	相談	
\\	複数形チェック
\\	[名詞]	検診	
\\	この記事が何処まで本当か調べてくれ。	
\\	複数形チェック
\\	[名詞]	満足(感)、充実感	
\\	満足の色が彼の顔に浮かんだ。	
\\	複数形チェック
\\	[形容詞]	冷え冷えする	
\\	今年、夏は寒いです。	
\\	[形容詞]	大陸の	
\\	[動詞]	窒息させる	
\\	彼女は彼の息の根を止めた。	
\\	[形容詞]	絶え間ない	
\\	[名詞]	循環	
\\	適度な運動は血液の循環を活発にする。	
\\	複数形チェック
\\	[副詞]	継続的に、絶えず	
\\	人間は、自分でそれを意識しているといないとにかかわらず、幸福を追求し続けています。	
\\	[動詞]	引用する	
\\	[副詞]	絶え間なく	
\\	三日間えんえんと雨が降っていた。	
\\	[動詞]	衝突する	
\\	反対派と警察の間で武力衝突が生じた。	
\\	[形容詞]	論争上の	
\\	[名詞]	分類	
\\	サポートベクターマシンとは、教師あり学習モデルであり、判別分析や回帰分析に用いられる。	
\\	複数形チェック
\\	[名詞]	論争	
\\	新しい学校の建設地については論争があった。	
\\	複数形チェック
\\	[形容詞]	機密扱いの	
\\	星雲はその形や見え方によって科学者が分類しています。	
\\	[形容詞]	会話の	
\\	[動詞]	分類する	
\\	木工芸では、木を、硬木、軟木と、唐木に分類します。	
\\	[動詞]	有罪と宣告する	
\\	トムさんは断罪されて、死刑判決を受けました。	
\\	[名詞]	条項、節	
\\	時の副詞節(ここでは
\\	は、未来のことがらを現在時制で表します。	
\\	複数形チェック
\\	[形容詞]	説得力のある	
\\	トムさんに正しいことをするように説得するのが大変でした。	
\\	[名詞]	清算	
\\	誤解を解いてくださってありがとう。	
\\	複数形チェック
\\	[名詞]	涼しさ	
\\	複数形チェック
\\	[名詞]	透明	
\\	複数形チェック
\\	[動詞]	協力する	
\\	協力できたら嬉しいです。	
\\	[形容詞]	気候の	
\\	[形容詞]	協力的な	
\\	私は協調性がある。	
\\	[名詞]	マント、おおい隠すもの	
\\	複数形チェック
\\	[名詞]	座標	
\\	複数形チェック
\\	[名詞]	終了、締め切り	
\\	学校の終業式が昨日あった。	
\\	複数形チェック
\\	[名詞]	コルク	
\\	複数形チェック
\\	[名詞]	群れ、集団	
\\	銀河系は,お互いに重力によってお互いに引きつけ合っている星や塵そしてガスの塊です。	
\\	複数形チェック
\\	[名詞]	訂正、校正	
\\	彼はいくつか訂正をした。	
\\	複数形チェック
\\	[形容詞]	粗い	
\\	[動詞]	一致する	
\\	日本の「将棋」は、チェスに相当する。	
\\	[名詞]	上塗り	
\\	複数形チェック
\\	[名詞]	特派員	
\\	その特派員はモスクワから記事を送った。	
\\	複数形チェック
\\	[形容詞]	集合的な	
\\	連帯責任は無責任。	
\\	[動詞]	堕落させる	
\\	ファイルが破損しています。	
\\	[形容詞]	着色した	
\\	色鉛筆が要ります。	
\\	[名詞]	汚職、腐敗	
\\	講演者は政界の腐敗をほのめかした。	
\\	複数形チェック
\\	[名詞]	着色	
\\	複数形チェック
\\	[形容詞]	化粧用の	
\\	[形容詞]	結合した	
\\	ノルディックコンバインドは、スキージャンプとクロスカントリースキーの2つの競技を組み合わせて行われる。	
\\	[形容詞]	費用のかかる	
\\	どうしてそんなに高い辞書を買いましたか。	
\\	[名詞]	復帰	
\\	複数形チェック
\\	[名詞]	ソファー	
\\	トムさんはソファに眠りに落ちました。	
\\	複数形チェック
\\	[名詞]	顧問官,評議員	
\\	複数形チェック
\\	[形容詞]	気がめいるような	
\\	それは気のめいるような話だ。	
\\	[名詞]	相談、協議	
\\	賢明で良心的な人に助言を求めなさい。	
\\	複数形チェック
\\	[名詞]	代理人	
\\	被告人が審理中に保安係りのピストルを掴み判事を撃った。	
\\	複数形チェック
\\	[名詞]	カウンセリング	
\\	「今日お父さんに会いに行く。進路相談だ」「いいですよ、そんなの」	
\\	複数形チェック
\\	[名詞]	子孫	
\\	アメリカの太平洋北西海岸沿いに住む原住民は、おそらくアジアから移住した種族の子孫なのである。	
\\	複数形チェック
\\	[名詞]	カウンセラー	
\\	複数形チェック
\\	[名詞]	降下	
\\	これより次第に高度を下げてまいります。	
\\	複数形チェック
\\	[名詞]	対照物、相手	
\\	複数形チェック
\\	[形容詞]	人の住まない	
\\	島にはまったく人気がなかった。	
\\	[形容詞]	数え切れない	
\\	空には数え切れないほどの星が輝いていた。	
\\	[動詞]	考案する	
\\	[名詞]	礼儀、親切さ	
\\	複数形チェック
\\	[形容詞]	献身的な	
\\	彼女は彼に自分をささげた。	
\\	[名詞]	報道	
\\	複数形チェック
\\	[名詞]	露	
\\	今朝は露が降りた。	
\\	複数形チェック
\\	[形容詞]	臆病な、卑劣な	
\\	[名詞]	図表	
\\	実施設計とは基本設計で決まった計画を、工事ができる図面に書く作業です。	
\\	複数形チェック
\\	[形容詞]	居心地のよい	
\\	彼は住み心地がよい小さな家に住んでいます。	
\\	[名詞]	対話	
\\	複数形チェック
\\	[名詞]	クラッカー	
\\	お母さん、このおせんべい、しけってておいしくないよ。	
\\	複数形チェック
\\	[名詞]	消化	
\\	複数形チェック
\\	[名詞]	職人、熟練工	
\\	腕のいい工芸師は自分の作品に誇りを持っている。	
\\	複数形チェック
\\	[名詞]	
\\	セント貨	
\\	10セント貸してください。	
\\	複数形チェック
\\	[名詞]	ツル	
\\	クレーンを操縦するのには、勿論、免許が必要です。	
\\	複数形チェック
\\	[名詞]	寸法	
\\	複数形チェック
\\	[名詞]	創造者	
\\	神は創造者です。天も地も人も、神以外は全て被造物です。	
\\	複数形チェック
\\	[動詞]	減少する	
\\	この国の人口は徐々に減少している。	
\\	[形容詞]	食物がパリパリする	
\\	砂肝のコリコリした食感が大好き。	
\\	[名詞]	外交官	
\\	かけ引きがお上手ですね。	
\\	複数形チェック
\\	[名詞]	心が曲がった人、いかさま師	
\\	トムさんは泥棒ではありません。	
\\	複数形チェック
\\	[名詞]	不一致	
\\	労働側と経営側の意見不一致はストライキに行きつくかもしれない。	
\\	複数形チェック
\\	[動詞]	かがむ,しゃがむ	
\\	クラウチさんでいらっしゃいますか。	
\\	[形容詞]	がっかりさせる	
\\	彼の新しい映画にはがっかりだ。	
\\	[名詞]	クルーズ、巡航	
\\	ハワイで通年航行中できます。	
\\	複数形チェック
\\	[名詞]	不賛成	
\\	彼はまゆを上げて難色を示した。	
\\	複数形チェック
\\	[名詞]	カッコウ	
\\	カッコウを見た事がありますか。	
\\	複数形チェック
\\	[動詞]	不賛成である	
\\	彼女は私がウィーンへ行くことに難色を示した。	
\\	[名詞]	合図、手がかり	
\\	男性たちは合図に合わせて微笑んでいる。	
\\	複数形チェック
\\	[動詞]	公開する	
\\	その情報はまだ公開できない。	
\\	[名詞]	カップ1杯	
\\	複数形チェック
\\	[名詞]	差別、識別	
\\	性別による差別は禁じられています。	
\\	複数形チェック
\\	[副詞]	奇妙に	
\\	町の人達は物珍しそうに私を見た。	
\\	[形容詞]	不正直な	
\\	皆が不正直とは限らない。	
\\	[名詞]	履修課程	
\\	今年の3月頃に「小学校の新しい教育課程では、円周率を
\\	として教えるのだそうだ」という話が広まった。	
\\	複数形チェック
\\	[副詞]	不正直に	
\\	[動詞]	呪う	
\\	呪いだよ。	
\\	[動詞]	不快にする	
\\	トムは不機嫌そうに額に皺を寄せた。	
\\	[名詞]	サイクリング	
\\	私たちみんな自転車に乗るのが好きです。	
\\	複数形チェック
\\	[動詞]	不満をいだかせる	
\\	私は不満です。	
\\	[名詞]	自転車に乗る人	
\\	複数形チェック
\\	[形容詞]	特徴的な	
\\	[名詞]	ダーツ	
\\	複数形チェック
\\	[名詞]	不信	
\\	社会保険庁や厚生労働省への不信感は募る一方である。	
\\	複数形チェック
\\	[名詞]	データベース	
\\	新しい語彙を説明するために誰でもデータシステムに追加をする事ができます。	
\\	複数形チェック
\\	[名詞]	妨害	
\\	カモメはうるさい鳴き声やフン害、ゴミ袋を裂いたり、 食べ物を狙って攻撃するなどの迷惑行為を引き起こします。	
\\	複数形チェック
\\	[名詞]	夜明け	
\\	複数形チェック
\\	[形容詞]	不安な	
\\	私たちは彼の邪魔をした。	
\\	[名詞]	偽り	
\\	複数形チェック
\\	[形容詞]	不穏にする	
\\	[形容詞]	上品な	
\\	彼は非常に親切だ。	
\\	[名詞]	溝、排水溝	
\\	彼は背水の陣で闘った。	
\\	複数形チェック
\\	[名詞]	欺くこと	
\\	複数形チェック
\\	[形容詞]	多様な	
\\	人生を通じて、私は世界中を旅して回り、多種多様な国々で働くという、素晴らしい喜びを得てきました。	
\\	[動詞]	捧げる	
\\	彼は医療に一生を捧げた。	
\\	[名詞]	多様性	
\\	この森は、多様性に富んでいます。	
\\	複数形チェック
\\	[形容詞]	献身的な	
\\	彼は医療に一生を捧げた。	
\\	[形容詞]	神の	
\\	間違いをするのが人間で、許すのは神だ。	
\\	[動詞]	深くする	
\\	我々は親睦を深めた。	
\\	[形容詞]	めまいがする	
\\	めまいがします。	
\\	[名詞]	被告人	
\\	被告は死刑判決を受けた。	
\\	複数形チェック
\\	[名詞]	ドキュメンタリー	
\\	ドキュメンタリー番組を見ました。	
\\	複数形チェック
\\	[名詞]	防御者	
\\	複数形チェック
\\	[名詞]	寄贈、寄付	
\\	はい、お賽銭・・・10円ずつね。はい、美羽ちゃん。	
\\	複数形チェック
\\	[形容詞]	防御的な	
\\	[形容詞]	非常に恐ろしい	
\\	何か恐ろしいことがデイヴィッドソンさんの身に起こったんですよ。	
\\	[名詞]	赤字	
\\	彼がその穴埋めをした。	
\\	複数形チェック
\\	[名詞]	私道	
\\	複数形チェック
\\	[動詞]	反抗する	
\\	当たり前のこと聞くなよ。	
\\	[名詞]	中途退学者	
\\	複数形チェック
\\	[形容詞]	慎重に	
\\	[形容詞]	酒に酔った	
\\	トムは飲酒運転で逮捕された。	
\\	[副詞]	わざと	
\\	[形容詞]	繊細に	
\\	[名詞]	鈍感	
\\	複数形チェック
\\	[名詞]	欠点	
\\	複数形チェック
\\	[動詞]	投げ捨てる	
\\	彼女は彼を振った。	
\\	[名詞]	デモ参加者	
\\	警官隊はデモ隊と乱闘になった。	
\\	複数形チェック
\\	[名詞]	小人	
\\	複数形チェック
\\	[名詞]	否定	
\\	複数形チェック
\\	[名詞]	住居	
\\	複数形チェック
\\	[形容詞]	⻭の	
\\	[副詞]	本気で	
\\	[名詞]	依存	
\\	複数形チェック
\\	[副詞]	経済的に	
\\	私は経済的には両親からひとり立ちしている。	
\\	[動詞]	落ち込ませる	
\\	彼は落ち込んでいる。	
\\	[名詞]	経済学	
\\	彼は大学で経済を学んだ。	
\\	複数形チェック
\\	[形容詞]	消沈した	
\\	彼は落ち込んでいる。	
\\	[名詞]	ナス	
\\	私はナスが好きです。	
\\	複数形チェック
\\	[名詞]	自我	
\\	この問題は、アイドルのエゴとヲタのエゴの衝突だ。	
\\	複数形チェック
\\	[名詞]	花火	
\\	私たちは花火大会に行きました。	
\\	複数形チェック
\\	[動詞]	念入りに作る	
\\	説明してください。	
\\	[名詞]	健康	
\\	日本では2000年頃からの健康ブームの一端で、「ダンベル体操」という家庭向けのフィットネスも登場した。	
\\	複数形チェック
\\	[名詞]	電子	
\\	原子は陽子・中性子・電子で構成されている。	
\\	複数形チェック
\\	[動詞]	平らにする	
\\	[動詞]	除去する	
\\	日本が不公平な関税を撤廃しなければ、米国は制裁措置をとるだろう。	
\\	[名詞]	艦隊	
\\	捕らえることのできないもの、身軽で素早いもののたとえ、だそうです。	
\\	複数形チェック
\\	[形容詞]	当惑した	
\\	私は困惑した。	
\\	[名詞]	適応性	
\\	複数形チェック
\\	[名詞]	困惑	
\\	彼女はうふふと照れながらコンドームを差し出した。	
\\	複数形チェック
\\	[動詞] [不規則動詞]	投げ飛ばす	
\\	敵は新兵力を投入した。	
\\	[名詞]	激励	
\\	複数形チェック
\\	[形容詞]	流動性の	
\\	赤ちゃんは子宮内の羊水で発育します。	
\\	[名詞]	試み、努力	
\\	複数形チェック
\\	[動詞]	はためく	
\\	葉っぱがはらはらと散っている。	
\\	[名詞]	忍耐	
\\	複数形チェック
\\	[形容詞]	折りたたみの	
\\	この折りたたみの傘を持って行きなさい。役に立つかもしれませんから。	
\\	[形容詞]	精力的な	
\\	なぜいつもそんなに元気ですか?	
\\	[名詞]	愚行	
\\	君の愚かさには笑わずにはいられない。	
\\	複数形チェック
\\	[副詞]	ばく大に	
\\	[名詞]	許すこと	
\\	姫君は皇帝に寛恕を請いました。	
\\	複数形チェック
\\	[形容詞]	面白い	
\\	[名詞]	形式的行為	
\\	複数形チェック
\\	[形容詞]	ねたむ	
\\	いい上司で羨ましいですね。	
\\	[名詞]	公式、式	
\\	水の化学式は
\\	である。	
\\	複数形チェック
\\	[名詞]	方程式	
\\	微分方程式を解くというのは、それを満足するような関数を見つけ出すということだ。	
\\	複数形チェック
\\	[名詞]	二週間	
\\	彼女は二週間だけ彼を知っています。	
\\	複数形チェック
\\	[動詞]	装備する	
\\	各ロボットには通話機が取り付けられています。	
\\	[名詞]	化石	
\\	石炭は化石燃料だ。	
\\	複数形チェック
\\	[形容詞]	同等の	
\\	名詞を修飾するもの(形容詞、または形容詞相当語句)。	
\\	[動詞]	養育する	
\\	彼は戦災孤児をひきとって養女として育てています。	
\\	[副詞]	永遠に	
\\	[名詞]	反則	
\\	彼がした事が違反でなくて何だろう。	
\\	複数形チェック
\\	[形容詞]	倫理の	
\\	[名詞]	鳥類	
\\	複数形チェック
\\	[名詞]	倫理学	
\\	倫理委員会では, クローン実験の一時停止が決定した。	
\\	複数形チェック
\\	[名詞]	一部、断片	
\\	時とは、永遠の一部分である。	
\\	複数形チェック
\\	[形容詞]	永久に続く	
\\	[名詞]	破片、断片	
\\	鏡の破片が床に散乱していた。	
\\	複数形チェック
\\	[動詞]	進化する	
\\	鳥は恐竜から進化しました。	
\\	[名詞]	枠組み	
\\	複数形チェック
\\	[名詞]	誇張	
\\	彼を天才と呼んでも誇張ではない。	
\\	複数形チェック
\\	[副詞]	新しく	
\\	僕は淹れたてのコーヒーの香りが好きなんだ。	
\\	[名詞]	検査官	
\\	複数形チェック
\\	[動詞]	挫折させる	
\\	カミーユはアランのうわの空の態度に不満だ。	
\\	[副詞]	非常に	
\\	アンはチョコレートに目がない。	
\\	[動詞]	性交する	
\\	くそっ!	
\\	[名詞]	優れていること	
\\	複数形チェック
\\	[形容詞]	ひどい	
\\	くそやりまん。	
\\	[形容詞]	並外れた	
\\	[名詞]	実現	
\\	複数形チェック
\\	[形容詞]	排他的な	
\\	[名詞]	資金	
\\	複数形チェック
\\	[副詞]	独占的に	
\\	『メトロイドプライム3 コラプション』は、任天堂発売の
\\	専用ゲームソフト。	
\\	[副詞]	その上に	
\\	尚、会社説明会後でも、会社見学は随時受け付けております。	
\\	[名詞]	遠足	
\\	もし万一明日雨が降れば、遠足は中止されるだろう。	
\\	複数形チェック
\\	[名詞]	空騒ぎ、大騒ぎ	
\\	何を小さなことに大騒ぎしているのか。	
\\	複数形チェック
\\	[動詞]	実行する	
\\	トムは処刑された。	
\\	[形容詞]	綿毛状の	
\\	[名詞]	消耗	
\\	佐藤さんは疲れ切って倒れた。	
\\	複数形チェック
\\	[名詞]	ガロン	
\\	複数形チェック
\\	[動詞]	搾取する	
\\	[動詞]	はっと息をのむ	
\\	[形容詞]	爆発の	
\\	[名詞]	出入り口、通路	
\\	複数形チェック
\\	[名詞]	輸出業者	
\\	複数形チェック
\\	[名詞]	集まり	
\\	彼らは木の実を拾い集めている。	
\\	複数形チェック
\\	[動詞]	抜く	
\\	は血液のサンプルから取れる。	
\\	[名詞]	宝石	
\\	ついに彼はその宝石を手に入れた。	
\\	複数形チェック
\\	[名詞]	寓話、作り話	
\\	次の一節はある有名な寓話から引用したものです。	
\\	複数形チェック
\\	[名詞]	遺伝子	
\\	一部の疾患の原因は欠陥遺伝子です。	
\\	複数形チェック
\\	[名詞]	ファクシミリ,ファックス	
\\	複数形チェック
\\	[動詞]	発生させる	
\\	観光が多くの新しい仕事を生み出した。	
\\	[名詞]	学部	
\\	あの人何学部だっけ?	
\\	複数形チェック
\\	[名詞]	寛大	
\\	複数形チェック
\\	[形容詞]	偽の	
\\	それは偽物です。	
\\	[形容詞]	遺伝子の	
\\	私たちの細胞の染色体は遺伝物質のすべてを含んでいる。	
\\	[名詞]	よく知っていること	
\\	複数形チェック
\\	[名詞]	幾何学	
\\	複数形チェック
\\	[形容詞]	遠方の	
\\	遠くの物が望遠鏡で見える。	
\\	[動詞]	すべるように動く	
\\	次第にふけてゆくおぼろ夜に、沈黙の人二人を載せた高瀬舟は、黒い水の面をすべって行った。	
\\	[名詞]	農地	
\\	複数形チェック
\\	[名詞]	グライダー	
\\	複数形チェック
\\	[動詞]	魅了する	
\\	すごく面白そうですね。	
\\	[名詞]	ちらっと見ること	
\\	彼をちらっと見た。	
\\	複数形チェック
\\	[名詞]	疲労(感)	
\\	疲労が彼の健康をむしばんでいる。	
\\	複数形チェック
\\	[動詞]	きらきら輝く	
\\	水は日の光を受けて輝く。	
\\	[形容詞]	恐れない	
\\	[形容詞]	華麗な	
\\	この町はなんて美しいんでしょう。	
\\	[名詞]	連盟	
\\	複数形チェック
\\	[形容詞]	文法の	
\\	その文には文法的な誤りはない。	
\\	[形容詞]	弱い	
\\	幸福はか弱い花である。	
\\	[名詞]	感謝	
\\	卒業前に、大変お世話になった先生のところへ、お礼のご挨拶に伺った。	
\\	複数形チェック
\\	[名詞]	男女同権主義者	
\\	複数形チェック
\\	[名詞]	重力	
\\	潮の満ち干は月の引力によって起こる。	
\\	複数形チェック
\\	[副詞]	財政的に	
\\	彼は経済的に自立した。	
\\	[動詞]	草を食う	
\\	牛が野原で草をはんでいた。	
\\	[動詞]	耐火性にする	
\\	この家は火事でも燃えない。	
\\	[名詞]	焼き肉(魚)	
\\	鯖は煮ても焼いても美味しい。	
\\	複数形チェック
\\	[名詞]	グループにすること	
\\	複数形チェック
\\	[副詞]	必然的に	
\\	鬼を恐れている者には鬼が必ず現れる。	
\\	[名詞]	木立ち	
\\	複数形チェック
\\	[形容詞]	安い	
\\	このハンバーガーは日本のハンバーガーとしては安いです。	
\\	[名詞]	栽培者	
\\	複数形チェック
\\	[動詞]	感染させる	
\\	トムのパソコンがウイルスに感染した。	
\\	[形容詞]	慎重な	
\\	宮殿は物々しい警戒ぶりだった。	
\\	[形容詞]	伝染性の	
\\	笑いは移る。	
\\	[名詞]	銃で武装した男	
\\	複数形チェック
\\	[形容詞]	無限の	
\\	だが宇宙は無限だ。	
\\	[形容詞]	常習の	
\\	[動詞]	(苦痛を)加える	
\\	トムがわざとけがをしたと本当に思っているのですか。	
\\	[名詞]	ひょう	
\\	雹が降るのを見たことがありますか。	
\\	複数形チェック
\\	[形容詞]	知識のある	
\\	何かあったらそのつど知らせてください。	
\\	[名詞]	玄関、廊下	
\\	廊下は滑りやすいので、足元に気を付けなさい。	
\\	複数形チェック
\\	[名詞]	原料、材料	
\\	トマトは生でも、煮ても、焼いてもおいしい食材です。	
\\	複数形チェック
\\	[名詞]	不利な条件	
\\	複数形チェック
\\	[名詞]	相続	
\\	いったいどれくらいの相続税を払うことになるんだろう。	
\\	複数形チェック
\\	[形容詞]	障害のある	
\\	[動詞]	始める	
\\	[動詞]	硬化させる	
\\	私のスパゲティブリッジに触らないでください! まだ接着剤が固まっている途中です。	
\\	[名詞]	率先、独創力	
\\	社長が社員の主体的な取り組みを促した。	
\\	複数形チェック
\\	[名詞]	苦難	
\\	この100年、ユダヤ人ほど苦難をなめてきた民族はいないだろう。	
\\	複数形チェック
\\	[名詞]	革新、刷新	
\\	力、英知、忍耐と革新。あなたはこれら全てを持っているのだろうか?その王冠を手にしようとする汝はその重みに耐えることとなる。	
\\	複数形チェック
\\	[形容詞]	苦難に耐えられる	
\\	[名詞]	問い合わせ	
\\	さらに質問がございましたら、どうぞご遠慮なくフリーダイヤル0120
\\	0000までお電話下さい。	
\\	複数形チェック
\\	[形容詞]	憎むべき	
\\	[動詞]	挿入する	
\\	彼は鍵を鍵穴に差し込んだ。	
\\	[動詞]	出没する	
\\	トムの家には幽霊が出る。	
\\	[名詞]	洞察、洞察力	
\\	複数形チェック
\\	[名詞]	ヘッドライト	
\\	ヘッドライトが故障したんです。	
\\	複数形チェック
\\	[名詞]	主張	
\\	彼女の主張は正しかった。彼女が肩身の狭い思いをする必要などなかったのだ。	
\\	複数形チェック
\\	[名詞]	校⻑	
\\	彼は校長になり立てだ。	
\\	複数形チェック
\\	[形容詞]	執ような	
\\	[副詞]	それゆえに	
\\	[名詞]	検査	
\\	検査の結果が出るまで待合室でお待ちください。	
\\	複数形チェック
\\	[名詞]	薬草	
\\	複数形チェック
\\	[形容詞]	奮い立たせる	
\\	[名詞]	遺産、伝統	
\\	彼の見地によれば、英語による支配は人類の遺産を脅かす。「単一の言語」は、カネと消費主義に取り憑かれた「単一の考え方」に導くという、より悪しき危険をもたらす。	
\\	複数形チェック
\\	[形容詞]	制度の、慣習の	
\\	[名詞]	ためらい	
\\	真実を語ることに私は何のためらいもない。	
\\	複数形チェック
\\	[形容詞]	ためになる	
\\	良い伝記はおもしろくて、ためにもなる。	
\\	[名詞]	高地地方	
\\	複数形チェック
\\	[形容詞]	役立つ	
\\	僕は器楽曲が好きだ。	
\\	[形容詞]	丘の多い	
\\	オーストリアの山がちな地形を見て、ボガート氏の『サウンド・オブ・ミュージック』を思い出しました。	
\\	[形容詞]	不十分な	
\\	この地域では水インフラの整備が十分でない。	
\\	[名詞]	支える物	
\\	複数形チェック
\\	[名詞]	知性	
\\	健康と知性は生きることの恵み。	
\\	複数形チェック
\\	[名詞]	神聖なこと、聖下	
\\	複数形チェック
\\	[名詞]	親切なもてなし	
\\	お邪魔しました。	
\\	複数形チェック
\\	[名詞]	強度	
\\	母親の怒りの厳しさにびっくりした。	
\\	複数形チェック
\\	[形容詞]	敵対する	
\\	我々
\\	としては、全力を持って敵対的買収に対抗するものと、明言しておこう。	
\\	[形容詞]	中間の	
\\	本人は全然下手クソと言ってるが、彼なら十分中級者で通るだろう。	
\\	[名詞]	敵意、反抗	
\\	複数形チェック
\\	[名詞]	解釈、通訳	
\\	私はあなたの解釈に大いに賛成だ。	
\\	複数形チェック
\\	[動詞]	ブンブンいう	
\\	彼がシャワーで鼻歌を歌っているのが聞こえた。	
\\	[名詞]	中断	
\\	お邪魔しました。	
\\	複数形チェック
\\	[名詞]	賛美歌	
\\	複数形チェック
\\	[名詞]	介入	
\\	複数形チェック
\\	[動詞]	理想化する	
\\	[形容詞]	親密な	
\\	芳子と時雄との関係は単に師弟の間柄としては余りに親密であった。	
\\	[名詞]	身分証明書	
\\	複数形チェック
\\	[名詞]	かかわり合い	
\\	僕は彼女とかかわったために身も心もずたぼろにされてしまった。	
\\	複数形チェック
\\	[名詞]	観念	
\\	複数形チェック
\\	[形容詞]	不規則な	
\\	不規則動詞の勉強は好きではない。	
\\	[名詞]	慣用句	
\\	複数形チェック
\\	[形容詞]	無責任な	
\\	[名詞]	怠惰	
\\	彼が失敗したのは怠けたせいだ。	
\\	複数形チェック
\\	[動詞]	いらいらさせる	
\\	彼は腹をたてた。	
\\	[名詞]	未成熟な者	
\\	複数形チェック
\\	[名詞]	不滅の、不滅の	
\\	この数年間に彼は不朽の名詩を書いた。	
\\	複数形チェック
\\	[名詞]	いらだち	
\\	彼女の気分は次第にいら立ちはじめた。	
\\	複数形チェック
\\	[名詞]	短気	
\\	複数形チェック
\\	[名詞]	イスラム教	
\\	1977年には、ファディルさんがイスラム教へ改宗しました。	
\\	複数形チェック
\\	[形容詞]	失礼な	
\\	私は無礼だった。	
\\	[名詞]	島	
\\	複数形チェック
\\	[副詞]	重大に	
\\	可愛い。そして何より品がいい。	
\\	[動詞]	孤立させる	
\\	私は孤独を感じた。	
\\	[名詞]	不可能	
\\	複数形チェック
\\	[名詞]	孤立、分離	
\\	私は孤独感を感じている。	
\\	複数形チェック
\\	[動詞]	投獄する	
\\	圧政的な政府はしばしば政敵を投獄する。	
\\	[動詞]	急にぐいと引くこと	
\\	私のボスは最低なやつだ。	
\\	[形容詞]	不適切な、不十分な	
\\	[形容詞]	陽気な	
\\	[形容詞]	無能な	
\\	トムは自分を愛せなくなっている。	
\\	[形容詞]	水分の多い	
\\	[名詞]	傾向	
\\	複数形チェック
\\	[名詞]	がらくた	
\\	それはくずだ。捨てなさい。	
\\	複数形チェック
\\	[形容詞]	不正確な	
\\	これは間違っている。	
\\	[名詞]	陪審員	
\\	複数形チェック
\\	[名詞]	表示、しるし	
\\	複数形チェック
\\	[名詞]	正当化	
\\	複数形チェック
\\	[名詞]	無関心	
\\	彼女は知らん顔をした。	
\\	複数形チェック
\\	[名詞]	道具一式	
\\	うちの体操服って普通の短パンだったのに今日から急にスパッツになっちゃったのね。	
\\	複数形チェック
\\	[形容詞]	無関心な	
\\	彼らは政治に無関心だ。	
\\	[形容詞]	欠けている	
\\	メアリーは思いやりに欠けている。	
\\	[副詞]	個々に	
\\	最近塾の講師を始めた。主に数学と英語の個別指導をしている。	
\\	[名詞]	少年	
\\	若者はかがみこんで、トムの耳元で大声を上げた。	
\\	複数形チェック
\\	[形容詞]	⻑続きする	
\\	アルプスの景観は私の印象に残った。	
\\	[形容詞]	通貨の	
\\	[動詞]	漏れる	
\\	秘密が外部に漏れた。	
\\	[名詞]	独占	
\\	これは当店の専売です。	
\\	複数形チェック
\\	[接続詞]	
\\	しないように	
\\	ニーチェは「化け物に成らないように,化け物と喧嘩するな」と言いました。	
\\	[名詞]	道徳	
\\	あらゆる道徳の基礎は同情である。	
\\	複数形チェック
\\	[名詞]	てこ、レバー	
\\	調節レバーを上に上げると座面の高さを調節できます。	
\\	複数形チェック
\\	[形容詞]	死ぬべき運命の	
\\	人間は死ぬべき運命にある。	
\\	[名詞]	司書	
\\	複数形チェック
\\	[動詞]	やる気にさせる	
\\	私は大やる気が出る。	
\\	[名詞]	好み	
\\	彼女はこってりした食べ物が好きだ。	
\\	複数形チェック
\\	[名詞]	自動車	
\\	複数形チェック
\\	[名詞]	定期船	
\\	ニュースによれば、もう一機のジェット旅客機が、中近東でハイジャックされたそうだ。	
\\	複数形チェック
\\	[名詞]	高速道路	
\\	複数形チェック
\\	[名詞]	裏地	
\\	あんた、なぜに並んでいるのだい。	
\\	複数形チェック
\\	[名詞]	一口	
\\	複数形チェック
\\	[名詞]	宿、宿泊	
\\	彼は私を一晩泊めてくれた。	
\\	複数形チェック
\\	[名詞]	マルチメディア	
\\	複数形チェック
\\	[形容詞]	孤立した	
\\	「そういえば谷本君は?」「あいかわらず一匹オオカミ気取っているよ」	
\\	[名詞]	多数	
\\	富はあまたの苦痛を覆う。	
\\	複数形チェック
\\	[動詞]	緩める、解く	
\\	ねじを緩めて、ランプ・カバーを取り外します。	
\\	[名詞]	殺人者	
\\	君は殺人犯だ。	
\\	複数形チェック
\\	[名詞]	敗者	
\\	彼は諦めが早い。	
\\	複数形チェック
\\	[名詞]	イスラム教徒	
\\	私はイスラム教徒です。	
\\	複数形チェック
\\	[名詞]	拡声器	
\\	複数形チェック
\\	[動詞]	つぶやく	
\\	彼女はひとりでブツブツ言っている。	
\\	[名詞]	材木、木材	
\\	複数形チェック
\\	[名詞]	羊肉	
\\	複数形チェック
\\	[形容詞]	磁石の	
\\	口座番号などを記憶したプラスチック製の磁気カードはキャッシュカードと言いますか?	
\\	[名詞]	話法	
\\	複数形チェック
\\	[形容詞]	処女の,未婚の	
\\	タイタニック号は処女航海で沈没した。	
\\	[形容詞]	話術の	
\\	[名詞]	本土	
\\	複数形チェック
\\	[名詞]	国家主義者	
\\	複数形チェック
\\	[名詞]	哺乳類	
\\	蝙蝠は鳥ではなく哺乳類だ。	
\\	複数形チェック
\\	[形容詞]	品行が悪い	
\\	彼は悪童である。	
\\	[名詞]	⻘壮年	
\\	彼は今男盛りだ。	
\\	複数形チェック
\\	[形容詞]	海軍の	
\\	[形容詞]	男らしい	
\\	トムはあまり男らしくない。	
\\	[名詞]	航海	
\\	昔の探検家たちは航海するのに星を利用した。	
\\	複数形チェック
\\	[名詞]	原稿	
\\	彼は私の原稿にざっと目を通してくれた。	
\\	複数形チェック
\\	[副詞]	きちんと	
\\	彼女はさっぱりした服装をしている。	
\\	[名詞]	市場	
\\	複数形チェック
\\	[形容詞]	困窮している	
\\	[名詞]	驚異	
\\	複数形チェック
\\	[形容詞]	隣接した、近隣の	
\\	我が国は近隣諸国との貿易を促進せねばならない。	
\\	[名詞]	マッサージ	
\\	そのメッセージは心地良かった。	
\\	複数形チェック
\\	[名詞]	新入り	
\\	やっぱり新人さんは初々しくていいね。	
\\	複数形チェック
\\	[名詞]	傑作	
\\	この映画はまさしく不朽の名作である。	
\\	複数形チェック
\\	[名詞]	ニュースキャスター	
\\	複数形チェック
\\	[名詞]	数学者	
\\	若いけれども彼はこれまでに例のないほどの偉大な数学者である。	
\\	複数形チェック
\\	[名詞]	会報、社報	
\\	複数形チェック
\\	[形容詞]	熟している	
\\	[名詞]	ナイトクラブ	
\\	旅行者は、そのナイトクラブでむしり取られた。	
\\	複数形チェック
\\	[名詞]	溶けている	
\\	氷が溶けている。	
\\	複数形チェック
\\	[名詞]	気高さ	
\\	爵位があるから、貴族だというわけにはいかないんだぜ。爵位が無くても、天爵というものを持っている立派な貴族のひともあるし、おれたちのように爵位だけは持っていても、貴族どころか、賤民にちかいのもいる。	
\\	複数形チェック
\\	[動詞]	記憶する	
\\	これを暗記しなさい。	
\\	[名詞]	貴族	
\\	複数形チェック
\\	[形容詞]	金属の	
\\	地球は内部の高温の液体金属の中心核とともに高速回転するので、地球全体を覆うような磁場を発生させる。	
\\	[形容詞]	音のしない	
\\	[名詞]	隠喩	
\\	「怒ると彼はトラになる」というのは隠喩の例である。	
\\	複数形チェック
\\	[副詞]	騒がしく	
\\	[名詞]	顕微鏡	
\\	トムは顕微鏡が欲しい。	
\\	複数形チェック
\\	[名詞]	指名、推薦	
\\	四十七人が指名を望んでいた。	
\\	複数形チェック
\\	[名詞]	中央	
\\	そうしたなかで急速な進歩を遂げてきたのが、
\\	である。	
\\	複数形チェック
\\	[副詞]	それでもなお	
\\	[副詞]	穏やかに	
\\	私は少しがっかりした。	
\\	[名詞]	めん類	
\\	私はラーメンを食べます。	
\\	複数形チェック
\\	[形容詞]	乳のような	
\\	私たちは銀河に住んでいますよ。	
\\	[名詞]	厄介な人・物事	
\\	周りに迷惑かけてる自覚はある?	
\\	複数形チェック
\\	[動詞]	入り混じる	
\\	[名詞]	修道女	
\\	マザー・テレサはインドのカルカッタに住んで働いていたカトリックの修道女であった。	
\\	複数形チェック
\\	[名詞]	ミニスカート	
\\	ミニスカートがまた流行している。	
\\	複数形チェック
\\	[名詞]	栄養	
\\	十分な栄養摂取は幼児の発育にとって極めて重要です。	
\\	複数形チェック
\\	[名詞]	小型模型	
\\	複数形チェック
\\	[名詞]	ムギ	
\\	複数形チェック
\\	[名詞]	鉱山業	
\\	鉱業はチリの主要な収入源の一つである。	
\\	複数形チェック
\\	[名詞]	反対、異議	
\\	異議あり!	
\\	複数形チェック
\\	[動詞] [不規則動詞]	誤解を招く	
\\	今回の試験は、難しかった上に質問も紛らわしかった。	
\\	[形容詞]	はっきりしない	
\\	私にはその意味がいまだに曖昧だ。	
\\	[名詞]	
\\	さん	
\\	どうしてミス・イタリアがあるのにミスター・イタリアはないのか?	
\\	複数形チェック
\\	[名詞]	障害、障害物	
\\	先人たちは数多の障害を乗り越えてきました。	
\\	複数形チェック
\\	[名詞]	誤解	
\\	誤解です。	
\\	複数形チェック
\\	[名詞]	タコ	
\\	食べられることなくそのたこは、海に帰ったのであった。	
\\	複数形チェック
\\	[名詞]	あざけること	
\\	50分の模擬試験に挑戦して、試験であなたの実力がどれくらいかがわかります。	
\\	複数形チェック
\\	[名詞]	悪臭	
\\	複数形チェック
\\	[副詞]	適度に	
\\	[名詞]	犯罪者、違反者	
\\	複数形チェック
\\	[名詞]	適度	
\\	ほどほどに飲めば、アルコールは害にはならない。	
\\	複数形チェック
\\	[形容詞]	海外の	
\\	[副詞]	適度に	
\\	[名詞]	子、子孫	
\\	もし核戦争になったら、われわれの子孫はどうなるのでしょう。	
\\	複数形チェック
\\	[名詞]	謙遜	
\\	褒められても謙虚が大事。	
\\	複数形チェック
\\	[名詞]	開く人、開始者	
\\	トムはレターオープナーをデスクの引き出しにしまった。	
\\	複数形チェック
\\	[名詞]	型	
\\	複数形チェック
\\	[形容詞]	選択できる	
\\	[名詞]	分子	
\\	水分子は、2個の水素原子と1個の酸素原子からなる。	
\\	複数形チェック
\\	[名詞]	軌道	
\\	複数形チェック
\\	[形容詞]	つかの間の	
\\	話がちょっと途切れた。	
\\	[名詞]	果樹園、果樹	
\\	彼らは私の果実園からりんごを盗んだ。	
\\	複数形チェック
\\	[動詞]	方向づける	
\\	彼は自己中心な人だ。	
\\	[名詞]	大農園	
\\	複数形チェック
\\	[名詞]	独創力	
\\	オリジナリティーとは、既にわかっている事を、普通の方法ではない方法で組み合わせることである。	
\\	複数形チェック
\\	[動詞]	誓約する	
\\	[形容詞]	飾りの	
\\	[形容詞]	豊富な	
\\	[名詞]	孤児	
\\	彼はその孤児を養子にした。	
\\	複数形チェック
\\	[動詞]	栓をする	
\\	お客様のパソコンのプラグが電源に差し込まれているか確認してくださいませんか?	
\\	[形容詞]	正統派の	
\\	1859年以降、プロテスタント宣教師がアメリカから派遣され、またカトリックとロシア正教も布教活動を行った。	
\\	[形容詞]	詩的な	
\\	トムは詩的でロマンティストだ。	
\\	[名詞]	勃発、突発	
\\	複数形チェック
\\	[副詞]	人気のある	
\\	このような民主主義の世代にあっては、皆自分がどう思うかではなく、一般に人気があるものを強く求めるのです。洗練されたものではなく、値段の高いものを。美しいものではなく、おしゃれなものを。	
\\	[名詞]	部外者	
\\	複数形チェック
\\	[副詞]	積極的に	
\\	[名詞]	郊外、周辺	
\\	トムはボストン郊外の小さな町に住んでいる。	
\\	複数形チェック
\\	[名詞]	郵便料金	
\\	複数形チェック
\\	[名詞]	卵形、卵形のもの	
\\	アブミ骨は内耳の前庭窓に接している。	
\\	複数形チェック
\\	[名詞]	用心,警戒	
\\	壊滅的な被害を防ぐために世界的な警戒を強化していく重要性を確認した。	
\\	複数形チェック
\\	[動詞] [不規則動詞]	転覆する	
\\	地球側の圧政に苦しむコロニーの人々の中で、テロによる現状打破を目論む勢力が誕生。	
\\	[名詞]	予想	
\\	トムの予言はすべて当たった。	
\\	複数形チェック
\\	[名詞]	時間外労働	
\\	今日は残業をしないつもりです。	
\\	複数形チェック
\\	[名詞]	妊娠	
\\	複数形チェック
\\	[形容詞]	重量超過の	
\\	彼は肥満体だ。	
\\	[形容詞]	妊娠している	
\\	妊娠しています。	
\\	[動詞]	圧倒する、打ちのめす	
\\	私たちは部屋に入るなり、腐っている肉の匂いが圧倒した。	
\\	[形容詞]	予備的な	
\\	[名詞]	働きすぎ、過労	
\\	彼は働きすぎで死んだ。	
\\	複数形チェック
\\	[名詞]	前提	
\\	複数形チェック
\\	[形容詞]	支払うべき	
\\	嵐のため彼らは遅れて着いた。	
\\	[形容詞]	就学前の	
\\	就学前の子供は入場無料です。	
\\	[名詞]	所有、所有者	
\\	複数形チェック
\\	[名詞]	保存、保護、維持	
\\	複数形チェック
\\	[動詞]	あえぐ	
\\	それは僕のズボンです。	
\\	[名詞]	見せかけ	
\\	複数形チェック
\\	[名詞]	逆説	
\\	「急げば急ぐほど遅くなる」は逆説である。	
\\	複数形チェック
\\	[名詞]	餌食	
\\	その鹿はライオンの餌食になった。	
\\	複数形チェック
\\	[形容詞]	議会の	
\\	[形容詞]	非常に貴重な	
\\	真の友情は千金にも代え難い。	
\\	[名詞]	店	
\\	複数形チェック
\\	[動詞]	刺す	
\\	大丈夫、この注射は痛くないからね。少しチクッとするだけだよ。	
\\	[副詞]	部分的に	
\\	[名詞]	確率	
\\	きっと私たちの方が彼らよりも前に到着するだろう。	
\\	複数形チェック
\\	[名詞]	参加者	
\\	参加者は彼を不注意だと責めた。	
\\	複数形チェック
\\	[名詞]	調査	
\\	複数形チェック
\\	[名詞]	参加	
\\	残念ながらお客様のご参加を記録することができませんでした。 原因としては、既に以前にご参加頂いているか、契約条件に反している行為があった可能性があります。 パネルプラットフォームへ転送いたします。	
\\	複数形チェック
\\	[名詞]	成り行き	
\\	複数形チェック
\\	[名詞]	焼き菓子	
\\	序に、日本でシユウ・クリイムと呼んでゐる菓子は、英国へ行つても仏蘭西へ行つてもその名前では通用しない。英吉利でシユウ・クリイムを持つて来いと云つたら、靴墨を持つて来たといふおとしばなしもできてゐるくらゐだ。	
\\	複数形チェック
\\	[形容詞]	熟練した	
\\	彼女はフランス語が堪能だ。	
\\	[名詞]	特許	
\\	東京特許許可局はどこですか?	
\\	複数形チェック
\\	[形容詞]	心からの	
\\	患者は深い昏睡状態であります。	
\\	[名詞]	足	
\\	ある朝洞穴から子どもの狐が 出ようとしましたが、「あっ。」と叫んで眼をおさえながら母さん狐のところへころげてきました。「母ちゃん、眼になにか刺さった、ぬいてちょうだい早く早く。」といいました。	
\\	複数形チェック
\\	[名詞]	顕著	
\\	複数形チェック
\\	[副詞]	穏やかに	
\\	ここでなら平和に暮らせるだろう。	
\\	[名詞]	発起人	
\\	複数形チェック
\\	[名詞]	クジャク	
\\	孔雀だよ。いま鳴いたのは孔雀だよ。	
\\	複数形チェック
\\	[名詞]	提案	
\\	彼はその提案に賛成していた。	
\\	複数形チェック
\\	[動詞]	くちばしでつつく	
\\	ハゲタカが突っつく死んだ鹿とか、他の動物の食べ残しとか、 そういう腐った肉を死肉と書きました。	
\\	[名詞]	保護者、後援者	
\\	複数形チェック
\\	[名詞]	特色,特性	
\\	複数形チェック
\\	[名詞]	プロテスタント	
\\	1859年以降、プロテスタント宣教師がアメリカから派遣され、またカトリックとロシア正教も布教活動を行った。	
\\	複数形チェック
\\	[副詞]	奇妙に,異様に	
\\	[名詞]	ペダル,踏み板	
\\	彼はブレーキを踏んだ。	
\\	複数形チェック
\\	[接続詞]	
\\	との条件で	
\\	さらに、即戦力となる政策関連の科学的成果や手段の提供が必要であると強調した。	
\\	[動詞]	のぞき見する	
\\	彼女が『今、覗いてたでしょう。この出歯亀野郎。』と怒りました。	
\\	[動詞]	刺激する	
\\	トムは口論をふっかけるのを楽しんでいるようだ。	
\\	[動詞]	突き刺さる	
\\	日本からの輸入製品がアメリカの消費者市場に浸透するには長い時間がかかった。	
\\	[名詞]	プッと吹くこと	
\\	複数形チェック
\\	[名詞]	貫通,洞察	
\\	複数形チェック
\\	[名詞]	傀儡(かいらい)	
\\	このパペットは私の唯一の友達です。	
\\	複数形チェック
\\	[形容詞]	金のない	
\\	彼は事業に失敗して一文なしになった。	
\\	[名詞]	純粋、清潔	
\\	複数形チェック
\\	[名詞]	年金	
\\	その老人は年金で暮らしている。	
\\	複数形チェック
\\	[名詞]	資格	
\\	その職に就く為にはそれ相応の資格が必要です。	
\\	複数形チェック
\\	[動詞]	鳥が木にとまる	
\\	其の日は一日淋しいペンの音を聞いて暮した。其の間には折々千代々々と云う声も聞えた。文鳥も淋しいから鳴くのではなかろうかと考えた。然し縁側へ出て見ると、二本の留り木の間を、彼方へ飛んだり、此方へ飛んだり、絶間なく行きつ戻りつしている。少しも不平らしい様子はなかった。	
\\	[名詞]	探求	
\\	複数形チェック
\\	[名詞]	完全	
\\	完璧というのはちょっと退屈なものなのである。	
\\	複数形チェック
\\	[名詞]	列	
\\	すみません、これ何の列ですか?	
\\	複数形チェック
\\	[名詞]	(差し迫った)危険	
\\	我々が今夜、祝ったとしても、我々は明日にある課題-二つの戦争、危機的状況にある地球、100年で最悪の金融危機-が人生の中でもっとも大きなものだと私たちは知っている。	
\\	複数形チェック
\\	[名詞]	静けさ	
\\	複数形チェック
\\	[動詞]	続く、やり通す	
\\	頑張れさえすれば体力や知力がなくても何でもできるという一種の精神主義もある。	
\\	[名詞]	棚、ラック	
\\	彼女はバッグを棚に置いたまま電車を降りてしまいました。	
\\	複数形チェック
\\	[形容詞]	固執する、頑固な	
\\	しつこいせきが出て、なかなか治りません。	
\\	[名詞]	レーダー	
\\	ジャスティン班長、亜空間レーダーに反応出ました!	
\\	複数形チェック
\\	[名詞]	人材、人員	
\\	関係者以外立ち入り禁止。	
\\	複数形チェック
\\	[名詞]	ダイコン	
\\	大根と人参はイチョウ切り、他の野菜は一口大に切る。	
\\	複数形チェック
\\	[名詞]	展望、観点	
\\	私たちは正しい見方でその問題を見なくてはならない。	
\\	複数形チェック
\\	[形容詞]	ぼろぼろの	
\\	[形容詞]	写真の	
\\	被写体に気づかれないようにそぉっとカメラを取り出し、ファインダーを覗く。	
\\	[名詞]	降雨	
\\	今年の夏は少雨による水不足が懸念される一方で、局地的なゲリラ豪雨は昨年の3.5倍になるとの予想も出ている。	
\\	複数形チェック
\\	[名詞]	物理学者	
\\	彼は物理学者だ。	
\\	複数形チェック
\\	[動詞]	結集する	
\\	[名詞]	柱	
\\	複数形チェック
\\	[名詞]	雄羊	
\\	メモリが足りないのかなぁ。	
\\	複数形チェック
\\	[名詞]	パイント	
\\	このジョッキは1パイント入る。	
\\	複数形チェック
\\	[名詞]	強姦	
\\	複数形チェック
\\	[名詞]	伝染病	
\\	私がインドから戻ってからというもの、どういうわけか人々が私をまるで疫病であるかのように避けている。	
\\	複数形チェック
\\	[名詞]	比率	
\\	親に対する子どもの比率が大きくなればなるほど子どもを育てるのは難しくなる。	
\\	複数形チェック
\\	[名詞]	プラネタリウム	
\\	複数形チェック
\\	[名詞]	領域	
\\	複数形チェック
\\	[動詞]	実る、収穫する	
\\	[名詞]	儀式、儀式的行事	
\\	正月の伝統的な行事が嫌いな人もある。	
\\	複数形チェック
\\	[名詞]	推理、論法	
\\	複数形チェック
\\	[動詞]	歩き回る	
\\	野生の動物たちが草原を歩き回っていた。	
\\	[動詞]	安心させる	
\\	医師の言葉はその患者を安心させた。	
\\	[名詞]	ごみ	
\\	馬鹿なこと言うなよ。	
\\	複数形チェック
\\	[名詞]	反乱	
\\	複数形チェック
\\	[名詞]	錆び	
\\	そのはしごは埃と錆で覆われていた。	
\\	複数形チェック
\\	[動詞] [不規則動詞]	再建する	
\\	わが家は全面的に改築された。	
\\	[形容詞]	給料を受け取る	
\\	[名詞]	受付係	
\\	複数形チェック
\\	[名詞]	大広間、しゃれた店	
\\	麻里子はエステティック・サロンに行きました。	
\\	複数形チェック
\\	[形容詞]	無謀な	
\\	トムは運転が荒い。	
\\	[動詞]	敬礼する	
\\	[名詞]	レクリエーション	
\\	複数形チェック
\\	[名詞]	悪魔	
\\	サタンのこうした行いを神がなぜ許しているのか、その理由は聖書のヨブ記で説明されている。	
\\	複数形チェック
\\	[名詞]	リール	
\\	私は昨夜ニユース映画で英国の女兵隊といふものをみたが、なかには男か女かわからないやうなのもゐたけれども、大部分は、なかなか女らしいところがあつて、しかもさうグロテスクな感じはしなかつた。	
\\	複数形チェック
\\	[名詞]	未開人	
\\	何か重い兇器でやられたらしく、頭蓋骨は粉砕された。	
\\	複数形チェック
\\	[形容詞]	反映する	
\\	[名詞]	筋書き、シナリオ	
\\	最悪のシナリオは何ですか?	
\\	複数形チェック
\\	[形容詞]	心身をさわやかにする	
\\	運動した後でシャワーを浴びると気分がさわやかになる。	
\\	[名詞]	学校教育	
\\	学校教育の義務的側面は子どもの学習意欲を改善させる様々な研究の多くの取り組みの中ではめったに分析されない。	
\\	複数形チェック
\\	[名詞]	飲食物	
\\	トムさんは飲み物を買ってきました。	
\\	複数形チェック
\\	[名詞]	スクーター	
\\	複数形チェック
\\	[動詞]	回復する	
\\	トムはバランスを取り戻した。	
\\	[名詞]	範囲、領域	
\\	活動の幅を世界に広げる。	
\\	複数形チェック
\\	[名詞]	登録	
\\	ユーチューブの登録が本名と住所が要ります。	
\\	複数形チェック
\\	[名詞]	嘲笑、軽蔑される人	
\\	トムは小川さんから多くの人が成金を軽蔑すると聞いた。	
\\	複数形チェック
\\	[名詞]	規則正しさ	
\\	複数形チェック
\\	[動詞]	先を争う	
\\	お化けなんてこわくないと言っていたタカシだが、誰もいないはずの2階で物音がした途端、すたこらさっさと逃げ出した。	
\\	[名詞]	リハーサル	
\\	10分休憩してからまたリハーサルを続けました。	
\\	複数形チェック
\\	[動詞]	こすりとる	
\\	ジャガイモの皮をこそげ取ってください。	
\\	[動詞]	君臨する	
\\	フィリップ2世による統治は40年間続いた。	
\\	[名詞]	文字、手書き	
\\	アラビア文字は読めません。	
\\	複数形チェック
\\	[名詞]	拒絶	
\\	複数形チェック
\\	[名詞]	港	
\\	複数形チェック
\\	[形容詞]	リラックスさせる	
\\	ちょっとくつろぎたい気分だ。	
\\	[名詞]	探照灯	
\\	複数形チェック
\\	[形容詞]	関連した	
\\	これはどういう関係があるのですか?	
\\	[名詞]	海藻	
\\	複数形チェック
\\	[形容詞]	信頼できる	
\\	彼女は信頼できる人だ。	
\\	[形容詞]	中古の	
\\	ぼくはこの本を古本屋で見つけた。	
\\	[名詞]	残り、余り	
\\	を3で割った余りを求めよ。	
\\	複数形チェック
\\	[名詞]	分派、宗派、派閥	
\\	宗派によっては、お坊さんは坊主にしなくてもいいんだってね。知らなかったよ。	
\\	複数形チェック
\\	[名詞]	思い出	
\\	複数形チェック
\\	[名詞]	セミナー、ゼミ	
\\	大学における演習とは、少人数の学生が教員指導の下に研究発表、討論を行うゼミナールのことを言います。	
\\	複数形チェック
\\	[動詞]	変える、する	
\\	彼女は素っ裸で部屋へ入り込んで彼を絶句させた。	
\\	[名詞]	上院議員	
\\	つい先程、マケイン上院議員から非常に丁寧な電話を頂きました。	
\\	複数形チェック
\\	[動詞]	再開する	
\\	[名詞]	発送人	
\\	複数形チェック
\\	[副詞]	繰り返して	
\\	[名詞]	感性、感受性	
\\	わたしは良心を持っていない。わたしの持っているのは神経ばかりである。	
\\	複数形チェック
\\	[名詞]	繰り返すこと	
\\	複数形チェック
\\	[名詞]	順序、連続	
\\	収束する数列は有界である。	
\\	複数形チェック
\\	[名詞]	再演奏	
\\	リプレーする度に彼女の反応にいちいち萌えてしまいます。	
\\	複数形チェック
\\	[名詞]	裁縫、縫い物	
\\	彼女はドレスを縫っている。	
\\	複数形チェック
\\	[動詞]	休息する	
\\	[形容詞]	日陰になった	
\\	叔父の兄の倅が変なビジネスに手を染めて、家名を落とすことになった。	
\\	[名詞]	代表、代理	
\\	複数形チェック
\\	[形容詞]	恥ずべき	
\\	逃げるは恥だが, 役に立つ。	
\\	[名詞]	生殖	
\\	同種個体への攻撃性は繁殖に関係するいくつかの目的にかなう。	
\\	複数形チェック
\\	[名詞]	サメ	
\\	兄が高利貸しに借金したんです。	
\\	複数形チェック
\\	[名詞]	類似、類似点	
\\	複数形チェック
\\	[動詞]	鋭くする	
\\	鉛筆を削りなさい。	
\\	[動詞]	腹を立てる	
\\	彼の失礼な態度には腹が立つ。	
\\	[名詞]	鋭さ	
\\	複数形チェック
\\	[名詞]	憤慨、恨み	
\\	複数形チェック
\\	[動詞]	粉砕する	
\\	彼女は金づちで鏡を粉々にしました。	
\\	[動詞]	居住する	
\\	[形容詞]	全くの、完全な	
\\	お客さま、そちらのドレスはすとんとしたシルエットがとても上品で素敵なんです。	
\\	[形容詞]	居住の	
\\	昨日、閑静な住宅街にある猫のひたいほどの売り地を見つかって、すぐ買うことに決めた。	
\\	[名詞]	神道	
\\	神道でいう神は無数にあり、初めは自然物や自然現象をも神としていた。	
\\	複数形チェック
\\	[形容詞]	敬意を表する	
\\	目上の人のすることについて話す時、尊敬語を使います。	
\\	[名詞]	くそ,大便	
\\	くそっ。	
\\	複数形チェック
\\	[形容詞]	落ち着かない、不安な	
\\	彼女は何もすることがなくて、落ち着かなかった。	
\\	[動詞]	寒さで震える	
\\	トムは寒さで震えていた。	
\\	[名詞]	修復、回復	
\\	複数形チェック
\\	[名詞]	聖地、殿堂	
\\	初詣、ご利益があるのはどこの神社?	
\\	複数形チェック
\\	[動詞]	再構築する	
\\	[名詞]	雨戶	
\\	カメラのシャッターはカシャッという音がするまで押さえてください。	
\\	複数形チェック
\\	[名詞]	小売り、小売店	
\\	新型は小売価格3万円で販売される。	
\\	複数形チェック
\\	[副詞]	横に	
\\	人込みを体を横にして抜けた。	
\\	[名詞]	収益、収入	
\\	映画館は、インターネット上に出回る違法コピーのせいで、収益をますます失いつつある。	
\\	複数形チェック
\\	[動詞]	示す、表す	
\\	[動詞]	訂正する	
\\	毎日、規則正しく復習してください。	
\\	[名詞]	質素	
\\	私は彼女のドレスのシンプルなところが好きだ。	
\\	複数形チェック
\\	[名詞]	反乱、反抗	
\\	1918年に全国で米騒動が起こりました。	
\\	複数形チェック
\\	[副詞]	同時に	
\\	彼らは一斉に立ち上がった。	
\\	[動詞]	回転する	
\\	惑星は太陽の周囲を回転する。	
\\	[名詞]	誠実	
\\	トムはこの上なく実直な男だ。	
\\	複数形チェック
\\	[形容詞]	リズミカルな	
\\	[名詞]	超高層ビル	
\\	ブルジュ・ハリーファは現在、世界で最も高い超高層ビルです。	
\\	複数形チェック
\\	[名詞]	なぞなぞ	
\\	なぞなぞか何かかい?	
\\	複数形チェック
\\	[動詞]	バタンと閉める	
\\	トムは電話をガチャンと切った。	
\\	[名詞]	山の尾根	
\\	複数形チェック
\\	[名詞]	奴隷制度	
\\	大統領は奴隷制度を廃止した。	
\\	複数形チェック
\\	[形容詞]	硬直した、固い	
\\	[形容詞]	眠れない	
\\	トムはメアリーのことを考えて眠れない夜を過ごした。	
\\	[動詞]	切り裂く	
\\	私は封筒を破って開けた。	
\\	[名詞]	スローガン	
\\	複数形チェック
\\	[形容詞]	煙る、煙のような	
\\	タバコの煙が充満したレストランで子供が食事することは安全でしょうか?	
\\	[名詞]	連続	
\\	このファンタジー小説は派手な魔法のオンパレードで逆に単調だ。	
\\	複数形チェック
\\	[名詞]	吹雪	
\\	吹雪はまる一週間荒れ狂った。	
\\	複数形チェック
\\	[形容詞]	連続的な	
\\	5日間雨が続いた。	
\\	[形容詞]	雪の降る	
\\	雪道でアクシデントがなかったら、無事に帰って来れると思います。	
\\	[名詞]	後継者	
\\	複数形チェック
\\	[動詞]	浸す	
\\	トムはずぶ濡れになった。	
\\	[名詞]	十分なこと	
\\	複数形チェック
\\	[形容詞]	石けんのような	
\\	[名詞]	概要、一覧	
\\	要約を火曜日までに提出しなさい。	
\\	複数形チェック
\\	[動詞]	酔いをさます	
\\	私は酔っていない。	
\\	[名詞]	監督;管理	
\\	複数形チェック
\\	[名詞]	社会主義者	
\\	はソビエト社会主義共和国連邦を表しています。	
\\	複数形チェック
\\	[動詞]	補う	
\\	あのマヨネーズの中、添加物ばっかり!	
\\	[名詞]	社会学	
\\	複数形チェック
\\	[動詞]	鎮圧する、抑圧する	
\\	私はあくびをかみ殺した。	
\\	[形容詞]	太陽の	
\\	その家は太陽エネルギーで暖房されている。	
\\	[動詞]	波乗り遊びをする	
\\	サーフィンは好きですか。	
\\	[副詞]	厳粛に	
\\	[名詞]	姓	
\\	私は周です。	
\\	複数形チェック
\\	[形容詞]	ひとりの	
\\	彼女は孤独な生活を送った。	
\\	[名詞]	はらはらすること	
\\	複数形チェック
\\	[形容詞]	洗練された	
\\	現代の金融は一段と複雑かつ高度になってきている。	
\\	[名詞]	沼地	
\\	複数形チェック
\\	[名詞]	宇宙船	
\\	複数形チェック
\\	[動詞]	揺らす	
\\	強風が吹けば高層ビルは揺れるだろう。	
\\	[名詞]	宇宙船	
\\	複数形チェック
\\	[名詞]	愛人	
\\	正直に言いますよ、ねえあなた、私は全く興味がないのです。	
\\	複数形チェック
\\	[名詞]	期間、全⻑	
\\	トムは集中力がない。	
\\	複数形チェック
\\	[副詞]	甘く	
\\	[動詞]	明記する、指定する	
\\	彼はいつ帰るかはっきり言わなかった。	
\\	[名詞]	優しさ	
\\	複数形チェック
\\	[名詞]	見世物	
\\	山頂からの眺めは壮観だった。	
\\	複数形チェック
\\	[名詞]	水着	
\\	複数形チェック
\\	[名詞]	推測、投機、考察	
\\	複数形チェック
\\	[名詞]	音節	
\\	詩歌とは、詩や短歌のことです。	
\\	複数形チェック
\\	[形容詞]	辛い	
\\	韓国料理は一般的に辛い。	
\\	[形容詞]	象徴的な	
\\	カトリックを信じていない人々でさえ象徴的指導者として法王を敬っている。	
\\	[名詞]	ホウレンソウ	
\\	ほうれん草をのこさずにたべなさい。	
\\	複数形チェック
\\	[動詞]	同情する	
\\	ケイトは彼に同情した。	
\\	[動詞] [不規則動詞]	唾をはく	
\\	吐き出しなさい。	
\\	[名詞]	症状	
\\	よくこういう症状が起きますか。	
\\	複数形チェック
\\	[形容詞]	スポーツマンらしい	
\\	[形容詞]	組織的な	
\\	[名詞]	春	
\\	春はあけぼの、夏はよる。	
\\	複数形チェック
\\	[名詞]	錠剤	
\\	私はせっかく社まで登ったのだから、絵馬を納めました。	
\\	複数形チェック
\\	[名詞]	安定、安定性	
\\	複数形チェック
\\	[動詞]	取り組む	
\\	世界が一丸となって気候変動の問題に取り組む必要がある。	
\\	[名詞]	山積みのもの	
\\	流しの横にお皿を積んでおいてくだされば、後で私が洗います。	
\\	複数形チェック
\\	[名詞]	札	
\\	鬼ごっこしよう。	
\\	複数形チェック
\\	[動詞]	ふらつく	
\\	[形容詞]	美味しい	
\\	わあ、おいしそう。	
\\	[名詞]	階段	
\\	わたしの家には階段がふたつある。	
\\	複数形チェック
\\	[名詞]	課税	
\\	国民は重税に苦しんだ。	
\\	複数形チェック
\\	[形容詞]	新鮮でない	
\\	そのパンは新鮮さを失くし始めていた。	
\\	[名詞]	納税者	
\\	それは納税者のお金の無駄遣いだけですよ。	
\\	複数形チェック
\\	[名詞]	茎	
\\	あ、茶柱が立ってる。何かいいこと起きるかも。	
\\	複数形チェック
\\	[名詞]	チームワーク	
\\	複数形チェック
\\	[名詞]	観点	
\\	複数形チェック
\\	[名詞]	茶さじ一杯	
\\	複数形チェック
\\	[形容詞]	驚くような	
\\	[名詞]	茶さじ一杯の量	
\\	複数形チェック
\\	[名詞]	飢餓	
\\	かわいそうにその猫は餓死寸前だった。	
\\	複数形チェック
\\	[名詞]	技術者、技巧家	
\\	彼は技術者になるべく生まれてきた。	
\\	複数形チェック
\\	[名詞]	統計	
\\	統計によれば世界の人口は増加している。	
\\	複数形チェック
\\	[名詞]	電報	
\\	電信を発明したのは誰だか知っていますか。	
\\	複数形チェック
\\	[名詞]	汽船	
\\	複数形チェック
\\	[名詞]	テンポ	
\\	リズムは軽快でテンポは速く、その曲は若者に好まれている。	
\\	複数形チェック
\\	[名詞]	静けさ、平静	
\\	複数形チェック
\\	[名詞]	賃借人、住人	
\\	店子は毎月末に大家に家賃を払わなければならない。	
\\	複数形チェック
\\	[名詞]	刺激	
\\	複数形チェック
\\	[名詞]	優しさ	
\\	複数形チェック
\\	[形容詞]	石のような	
\\	[形容詞]	末端の	
\\	南ターミナルはどこですか。	
\\	[名詞]	いす、足台	
\\	複数形チェック
\\	[名詞]	テラス	
\\	複数形チェック
\\	[動詞]	かがむ	
\\	おばあちゃんは身をかがめて糸の付いた針を拾った。	
\\	[動詞]	ひどく怖がらせる	
\\	彼女は病院を恐れている。	
\\	[名詞]	保管	
\\	トムは箱を二つ収納室の中へ運んだ。	
\\	複数形チェック
\\	[形容詞]	感謝している	
\\	私は感謝しています。	
\\	[名詞]	物語作家	
\\	複数形チェック
\\	[副詞]	有り難いことに	
\\	今日とても大きな地震がありましたが、おかげさまで家族全員無事でした。	
\\	[形容詞]	物語を話す	
\\	[形容詞]	演劇的な	
\\	[動詞]	まっすぐにする	
\\	「お前が折った木の種類わかるか?」「いや知らない」刑事に問われ俺は答えた。刑事は開いたドアの方を見ながら「おい、こいつの折った木の種類わかるか?」と声を上げた。「えっと、槙らしいです」と若い女性の声。刑事は居住まいを正すと「お前が折った木の種類わかるか?」ともう一度聞いた。	
\\	[名詞]	盗み	
\\	彼には窃盗の罪がある。	
\\	複数形チェック
\\	[名詞]	海峡	
\\	木曜島はオーストラリア最北端ヨーク岬とニューギニア島の間のトレス海峡に位置する。	
\\	複数形チェック
\\	[名詞]	療法	
\\	近年作業療法の需要は急速に高まった。	
\\	複数形チェック
\\	[動詞]	迷う	
\\	野犬がうちの庭を踏み荒した。	
\\	[名詞]	厚さ	
\\	複数形チェック
\\	[名詞]	争い、競争	
\\	結婚生活は夫婦仲が良ければ天国であり、夫婦仲が悪ければこの世の地獄である。	
\\	複数形チェック
\\	[動詞] [不規則動詞]	押す	
\\	たくさんの人々が他人を押し分けて後部出口に向かった。	
\\	[形容詞]	構造の	
\\	水の構造式は 
\\	である。	
\\	[動詞]	カチッと音を立てる	
\\	時計は刻んでいる	
\\	[名詞]	切り株	
\\	複数形チェック
\\	[形容詞]	臆病な	
\\	彼は極めて臆病だ。	
\\	[名詞]	愚かさ	
\\	愚かさは知性の休息である。	
\\	複数形チェック
\\	[形容詞]	疲れる	
\\	今日は疲れる一日だった。	
\\	[形容詞]	繊細な、微妙な	
\\	その2つの絵には微妙な違いがある。	
\\	[形容詞]	時事問題の	
\\	[名詞]	苦痛	
\\	複数形チェック
\\	[動詞]	散らかす	
\\	[名詞]	カメ	
\\	複数形チェック
\\	[形容詞]	愚かな	
\\	彼女は彼を無視したが、それは結果的には分別のない行為だった。	
\\	[名詞]	拷問、苦悩	
\\	彼は警察で拷問を受けた。	
\\	複数形チェック
\\	[形容詞]	価値のない	
\\	[名詞]	観光業	
\\	観光が多くの新しい仕事を生み出した。	
\\	複数形チェック
\\	[名詞]	有用性	
\\	複数形チェック
\\	[名詞]	業務,取引	
\\	複数形チェック
\\	[名詞]	有益	
\\	複数形チェック
\\	[名詞]	変化	
\\	複数形チェック
\\	[動詞]	利用する、活用する	
\\	例外もあるが、その会社はたいていは資産をとてもうまく利用している。	
\\	[名詞]	翻訳家	
\\	私は翻訳家です。	
\\	複数形チェック
\\	[形容詞]	最高の	
\\	これは最重要案件です。	
\\	[名詞]	伝達	
\\	あの、わたしオートマしか運転できないんですけど。	
\\	複数形チェック
\\	[形容詞]	空の	
\\	椅子が二つ空いていた。	
\\	[動詞]	送信する	
\\	人生とは致命的な性感染症だ。	
\\	[形容詞]	合法的な、有効な	
\\	有効なメールアドレスを入力してください。	
\\	[名詞]	透明	
\\	複数形チェック
\\	[形容詞]	無価値な	
\\	[名詞]	ごみ	
\\	それはゴミです。	
\\	複数形チェック
\\	[名詞]	虚栄心、空虚	
\\	色即是空、空即是色。	
\\	複数形チェック
\\	[名詞]	財務	
\\	複数形チェック
\\	[形容詞]	さまざまの	
\\	トムは多趣味だ。	
\\	[動詞]	震える	
\\	彼女は恐ろしくて身震いした。	
\\	[名詞]	ビロード	
\\	この布はビロードのような手ざわりがする。	
\\	複数形チェック
\\	[形容詞]	とても大きい	
\\	東南アジアのインフラ整備で日本が中心的役割を担うため、官民一体となった努力が求められている。	
\\	[形容詞]	凶悪な	
\\	[形容詞]	三角形の	
\\	[名詞]	視聴者	
\\	複数形チェック
\\	[形容詞]	取るに足らない	
\\	被弾しました。ダメージは軽微です。	
\\	[形容詞]	精力的な	
\\	私は太り気味かもしれないが、すこぶる健康だよ。	
\\	[名詞]	トロフィー、優勝旗	
\\	トロフィーをもらったのですか。	
\\	複数形チェック
\\	[名詞]	別荘	
\\	別荘にもどって、三時のおやつ。	
\\	複数形チェック
\\	[形容詞]	厄介な、煩わしい	
\\	あんたは世話の焼けるひとだね。	
\\	[名詞]	違反	
\\	女性が下半身が見える状態で公道を歩いていると公然猥褻罪となり、上半身裸で歩いていると軽犯罪法第1条第20号の罪となります。	
\\	複数形チェック
\\	[形容詞]	信じている	
\\	[名詞]	処女	
\\	トムはまだ童貞だ。	
\\	複数形チェック
\\	[形容詞]	信頼できる	
\\	彼の情報の信頼性は疑わない。なぜならそれは信用できる筋からの情報なのだから。	
\\	[形容詞]	実質上の	
\\	[形容詞]	誠実な	
\\	彼の言うことは本当ではないと思う。	
\\	[名詞]	ウイルス	
\\	トムのパソコンがウイルスに感染した。	
\\	複数形チェック
\\	[動詞]	押し込む	
\\	彼はそのハンカチをポケットに押し込んだ。	
\\	[名詞]	活力	
\\	複数形チェック
\\	[名詞]	マグロ	
\\	サメの皮はマグロの皮よりはるかにざらざらしている。	
\\	複数形チェック
\\	[形容詞]	口頭の	
\\	[副詞]	最終的に	
\\	ついに反逆者は捕らえられ、拘置所に入れられた。	
\\	[名詞]	電圧	
\\	複数形チェック
\\	[名詞]	判定者、審判	
\\	試合後のインタビューで、監督は審判への不満を口にした。	
\\	複数形チェック
\\	[形容詞]	自発的な	
\\	あの会社は今、希望退職者に退職金を多く都合することで人員削減を図っている。	
\\	[形容詞]	武器を持たない	
\\	[名詞]	投票者、有権者	
\\	複数形チェック
\\	[形容詞]	避けがたい	
\\	[名詞]	母音	
\\	フランス語には日本語よりずっと多くの母音がある。	
\\	複数形チェック
\\	[形容詞]	耐えられない	
\\	忍耐が耐えられないこともある。	
\\	[名詞]	クルミ	
\\	複数形チェック
\\	[形容詞]	意識を失った	
\\	意識がありません。	
\\	[名詞]	令状	
\\	捜索令状は持っているのですか。	
\\	複数形チェック
\\	[動詞] [不規則動詞]	取り消す	
\\	靴ひもがとけた。	
\\	[形容詞]	無駄な	
\\	[形容詞]	疑う余地のない	
\\	[名詞]	番犬	
\\	複数形チェック
\\	[動詞]	衣類をとる	
\\	私は服を脱いでいる。	
\\	[形容詞]	用心深い	
\\	[形容詞]	不安な、心配な	
\\	彼は不安になった。	
\\	[名詞]	警備員	
\\	複数形チェック
\\	[形容詞]	失業の	
\\	田辺さんは無職となりました。	
\\	[名詞]	水彩画	
\\	複数形チェック
\\	[動詞]	広げる	
\\	その計画の詳細がだんだん明らかになってきた。	
\\	[名詞]	滝	
\\	トムは滝の近くの小さな小屋にたった一人で住んでいる。	
\\	複数形チェック
\\	[名詞]	不幸	
\\	長い不幸な年月ののち、彼女は彼と離婚した。	
\\	複数形チェック
\\	[形容詞]	防水の	
\\	私のカメラは防水だ。	
\\	[副詞]	普遍的に	
\\	トムは自他ともに認める雨男だ。	
\\	[形容詞]	弱い	
\\	[形容詞]	無限の	
\\	[接続詞]	
\\	であるのに	
\\	都会では気付かれず通り過ぎることもあるだろうが、村の中ではそうはいかない。	
\\	[動詞]	(錠・鍵を)開ける	
\\	トムは正門の鍵を開けた。	
\\	[動詞]	渦巻く	
\\	葉が空中でぐるぐる回っていた。	
\\	[形容詞]	独身の	
\\	彼は結婚してないです。	
\\	[名詞]	純粋、純白	
\\	複数形チェック
\\	[形容詞]	間違えようのない	
\\	[動詞]	広くする、広げる	
\\	[形容詞]	不自然な	
\\	彼は不慮の死を遂げた。	
\\	[名詞]	撤退	
\\	複数形チェック
\\	[形容詞]	未払いの	
\\	未払い給料の受取りを諦めました。	
\\	[形容詞]	羊毛の	
\\	[形容詞]	筋の通らない	
\\	無理が通れば、道理が引っ込む。	
\\	[名詞]	工員、肉体労働者	
\\	下手な職人はいつも道具に難くせをつける。	
\\	複数形チェック
\\	[形容詞]	不満足な	
\\	[形容詞]	この世の	
\\	[形容詞]	初見の	
\\	[動詞]	崇拝する	
\\	その部族は祖先を崇拝し、私たちにはなじみのない独自の言語を話す。	
\\	[形容詞]	利己的でない	
\\	[形容詞]	価値のない	
\\	この機械は価値がない。	
\\	[形容詞]	不安定な	
\\	酒を飲みすぎて足もとがふらついた。	
\\	[形容詞]	価値がある	
\\	利益になるような本を読みなさい。	
\\	[形容詞]	失敗した、不運の	
\\	[名詞]	しわ	
\\	私はズボンのしわをアイロンで伸ばした。	
\\	複数形チェック
\\	[形容詞]	不適当な	
\\	[副詞]	不法に	
\\	メアリーは、10年近くの間、不当に投獄されていました。	
\\	[名詞]	放棄	
\\	北朝鮮が6か国協議の合意に基づき核開発計画を申告した26日、米国が「テロ支援国」の指定解除手続きに入ったことで、拉致被害者の家族らには「拉致問題が置き去りにされるのでは」という不安が広がった。	
\\	複数形チェック
\\	[形容詞]	おおよその	
\\	[名詞]	略語	
\\	複数形チェック
\\	[名詞]	考古学	
\\	複数形チェック
\\	[形容詞]	異常な	
\\	こんな暖かい陽気は2月にしては異常だ。	
\\	[形容詞]	北極の	
\\	彼女はついに北極に到達した。	
\\	[名詞]	廃止	
\\	ヨーロッパでの奴隷制度の廃止は、やがてアメリカにも波及しました。	
\\	複数形チェック
\\	[形容詞]	装甲した	
\\	[形容詞]	口汚い	
\\	[動詞]	目を覚まさせる	
\\	気になる男性の性欲を刺激するにはどうすればよいのでしょうか?	
\\	[名詞]	伴うもの	
\\	複数形チェック
\\	[形容詞]	傲慢な	
\\	偉そうなこと言ってごめん。	
\\	[名詞]	一致	
\\	金利は、債務者が負担する事業リスクに応じて定まるものと存じます。	
\\	複数形チェック
\\	[副詞]	浜に岸に	
\\	[名詞]	適合順応	
\\	この映画は小説をもとにしている。	
\\	複数形チェック
\\	[名詞]	主張	
\\	複数形チェック
\\	[動詞]	運営する	
\\	[形容詞]	大気の	
\\	空気の汚染が激しい地域でオリンピックをやることは選手にとって大きな危険性をはらんでいる。	
\\	[名詞]	海軍大将、提督	
\\	複数形チェック
\\	[名詞]	付着物	
\\	月曜日までに、ファックスまたはメール添付などで、お振込み証明をお送りいただけますか。	
\\	複数形チェック
\\	[名詞]	採用	
\\	複数形チェック
\\	[名詞]	弁護士	
\\	事前に弁護士と相談した方がいいよ。	
\\	複数形チェック
\\	[動詞]	大好きである	
\\	私たちはピクニックが大好きです。	
\\	[名詞]	競売	
\\	あなた同様私もヤフオクの評価が信じられません。	
\\	複数形チェック
\\	[動詞]	装飾する	
\\	[名詞]	オーディション	
\\	複数形チェック
\\	[名詞]	昇進	
\\	複数形チェック
\\	[名詞]	講堂	
\\	複数形チェック
\\	[形容詞]	有利な	
\\	[名詞]	自叙伝	
\\	複数形チェック
\\	[形容詞]	助言を与える	
\\	[動詞]	利用する	
\\	君はこの機会を利用する方がよい。	
\\	[形容詞]	愛情のこもった	
\\	[名詞]	回避	
\\	複数形チェック
\\	[動詞]	断言する	
\\	面白いですかと聞いたら、彼は優しい笑みを浮かべて、面白いことなどありません、と断言した。	
\\	[名詞]	軸	
\\	放物線 
\\	を
\\	軸方向に3,
\\	軸方向に
\\	2だけ平行移動して得られる放物線の方程式を求めよ。	
\\	複数形チェック
\\	[名詞]	議事日程	
\\	複数形チェック
\\	[名詞]	独身男性	
\\	独身ですか。	
\\	複数形チェック
\\	[名詞]	侵略	
\\	恐怖は犬を向こう気にさせるのである。	
\\	複数形チェック
\\	[名詞]	支持	
\\	彼にはある政治家の息がかかっている。	
\\	複数形チェック
\\	[名詞]	補佐官	
\\	複数形チェック
\\	[名詞]	えさ	
\\	複数形チェック
\\	[形容詞]	目的のない	
\\	[名詞]	投票用紙、投票	
\\	今夜私の頭にあったうちの一つは、アトランタで投票した一人の女性のことです。	
\\	複数形チェック
\\	[名詞]	航空会社	
\\	複数形チェック
\\	[名詞]	包帯目隠しの布	
\\	彼女は傷口に包帯をした。	
\\	複数形チェック
\\	[形容詞]	大変な	
\\	[名詞]	倒産破綻	
\\	父の会社は破産寸前である。	
\\	複数形チェック
\\	[名詞]	アルコール中毒	
\\	複数形チェック
\\	[名詞]	野蛮人、未開人	
\\	複数形チェック
\\	[名詞]	代数学	
\\	代数は数学の1部門です。	
\\	複数形チェック
\\	[形容詞]	裸足の	
\\	熱い石炭の上を裸足で歩く男を見たことがある。	
\\	[動詞]	主張する	
\\	[名詞]	大⻨	
\\	大麦は我国の主産物である。	
\\	複数形チェック
\\	[形容詞]	同盟している	
\\	[名詞]	兵舎	
\\	複数形チェック
\\	[名詞]	祭壇	
\\	勉強する子ども、仕事につく大人、病を克服して健康を取り戻した人、その一人一人が、祭壇にささげられたろうそくのように、信じる人すべての希望を明るくします。	
\\	複数形チェック
\\	[動詞]	バリケードを築く	
\\	労働者たちは人垣を作った。	
\\	[名詞]	変更	
\\	複数形チェック
\\	[名詞]	寝具	
\\	複数形チェック
\\	[形容詞]	交互の	
\\	[動詞] [不規則動詞]	眺める	
\\	奇妙な光景を目にする。	
\\	[名詞]	高度、海抜	
\\	複数形チェック
\\	[名詞]	へそ	
\\	複数形チェック
\\	[形容詞]	驚嘆した	
\\	私はその結果に仰天した。	
\\	[動詞]	授ける	
\\	皇帝の位は神から授かったものだと信じられていた。	
\\	[名詞]	驚き	
\\	彼らは彼女の水着を驚きの目でじろじろと見た。	
\\	複数形チェック
\\	[名詞]	飲み物	
\\	お茶でも飲みながら待とう。	
\\	複数形チェック
\\	[名詞]	アーメン	
\\	天におられるわたしたちの父よ、 み名が聖とされますように。 み国が来ますように。 みこころが天に行われるとおり 地にも行われますように。 わたしたちの日ごとの糧を 今日もお与えください。 わたしたちの罪をおゆるしください。 わたしたちも人をゆるします。 わたしたちを誘惑におちいらせず、 悪からお救いください。 国と力と栄光は、永遠にあなたのものです。 [アーメン]	
\\	複数形チェック
\\	[動詞]	改正する	
\\	[名詞]	偏見	
\\	複数形チェック
\\	[名詞]	修正	
\\	複数形チェック
\\	[名詞]	入札、付値	
\\	複数形チェック
\\	[名詞]	祖先	
\\	複数形チェック
\\	[名詞]	隔月刊行物	
\\	複数形チェック
\\	[名詞]	不快感	
\\	複数形チェック
\\	[名詞]	縛ること束縛	
\\	複数形チェック
\\	[形容詞]	匿名の	
\\	[名詞]	出生地	
\\	バレンシアはスペイン第3の都市で、パエリア発祥の地としても知られている。	
\\	複数形チェック
\\	[形容詞]	南極の	
\\	[名詞]	まばたき	
\\	彼があっと言う間に飛び出しました。	
\\	複数形チェック
\\	[名詞]	人類学	
\\	複数形チェック
\\	[名詞]	至福天国	
\\	無知は幸福。	
\\	複数形チェック
\\	[名詞]	付録	
\\	複数形チェック
\\	[形容詞]	鈍い	
\\	トムが辛いのは分かるけど、本当に傷ついてるのはメアリーの方なんじゃないかな。	
\\	[名詞]	前菜	
\\	複数形チェック
\\	[形容詞]	自慢して	
\\	君の自慢話はもう聞き飽きた。	
\\	[形容詞]	食欲をそそる	
\\	[名詞]	胸	
\\	18世紀の流行は胸の部分を強調していた。	
\\	複数形チェック
\\	[名詞]	電化製品	
\\	ディスプレイはモニタともいい、コンピュータなどの機器から出力される静止画または動画の映像信号を表示する機器である。	
\\	複数形チェック
\\	[形容詞]	植物の	
\\	トムとメアリーは植物園で会った。	
\\	[形容詞]	適用できる	
\\	その規則は我々外国人にも当てはまりますか。	
\\	[名詞]	大枝	
\\	実るほど頭のさがる稲穂かな。	
\\	複数形チェック
\\	[名詞]	応募者	
\\	応募者にふさわしい職務遂行能力があるか?	
\\	複数形チェック
\\	[名詞]	花束	
\\	複数形チェック
\\	[名詞]	懸念	
\\	複数形チェック
\\	[名詞]	腸	
\\	お通じがよくなりますよ。	
\\	複数形チェック
\\	[名詞]	見習い、初心者	
\\	複数形チェック
\\	[名詞]	支え、同類	
\\	複数形チェック
\\	[形容詞]	息を切らした	
\\	[動詞]	明らかにする	
\\	本稿の結論をどこまで一般化できるかは、今後の研究で明らかにしていきたい。	
\\	[名詞]	花婿	
\\	複数形チェック
\\	[動詞]	カタカタ鳴る	
\\	[形容詞]	活発な元気のよい	
\\	同社の売上げは輸出の需要が強いおかげで伸びたが、競争が激しく利益はそれほど伸びなかった。	
\\	[名詞]	聖職者	
\\	複数形チェック
\\	[動詞]	広げる、広がる	
\\	今、政府はなし崩し的に憲法九条を拡大解釈している。	
\\	[名詞]	聖職者	
\\	複数形チェック
\\	[副詞]	大雑把に	
\\	我々を見てにたりと笑った。	
\\	[動詞] [不規則動詞]	しがみつく	
\\	その猫は彼女の着物にまつわりついた。	
\\	[名詞]	パンフレット	
\\	私はパンフレットにざっと目をとおした。	
\\	複数形チェック
\\	[形容詞]	臨床の	
\\	2000年に日本の医療制度が世界一だが、2003年に研修医制度が始まって目に見えて悪くなった。	
\\	[名詞]	あざ、打ち身	
\\	ラグビーの試合の後、全身打ち身だらけだった。	
\\	複数形チェック
\\	[名詞]	携帯品預かり所	
\\	私は鞄と傘を一時預かり室に預けた。	
\\	複数形チェック
\\	[名詞]	昼食兼用朝食	
\\	複数形チェック
\\	[名詞]	時計仕掛け	
\\	複数形チェック
\\	[名詞]	獣	
\\	「なんで私のこと嫌いなの?」「意地悪だから」	
\\	複数形チェック
\\	[名詞]	道化者	
\\	この田舎者めが。	
\\	複数形チェック
\\	[名詞]	ビュッフェ	
\\	ビュッフェの内容を見せていただけますか。	
\\	複数形チェック
\\	[形容詞]	不器用な	
\\	トムは滑舌が悪い。	
\\	[名詞]	公報	
\\	トムたちは毎月学校の印刷機を使って生徒会報を発行している。	
\\	複数形チェック
\\	[動詞]	つかむ	
\\	クラッチペダルが壊れてしまった。	
\\	[名詞]	官僚	
\\	複数形チェック
\\	[形容詞]	沿岸の	
\\	ブリティッシュコロンビアの沿岸生態系の劣化が急速に進行している。	
\\	[形容詞]	燃えた	
\\	指先を火傷しました。	
\\	[名詞]	棺桶	
\\	複数形チェック
\\	[名詞]	バイパス	
\\	複数形チェック
\\	[名詞]	巻いたものとぐろ	
\\	電流が流れる電線にコイルを近づけると、コイルにも電気が流れます。	
\\	複数形チェック
\\	[名詞]	災難、不幸	
\\	複数形チェック
\\	[動詞]	同時に起こる	
\\	[名詞]	カルシウム	
\\	複数形チェック
\\	[名詞]	衝突	
\\	イージス衝突 
\\	なぜ避けられなかったか。	
\\	複数形チェック
\\	[名詞]	運動家	
\\	複数形チェック
\\	[名詞]	大佐連隊⻑	
\\	彼は大佐に昇進した。	
\\	複数形チェック
\\	[名詞]	カナリア	
\\	世界一大きな望遠鏡はカナリア諸島にある。	
\\	複数形チェック
\\	[名詞]	コラムニスト	
\\	複数形チェック
\\	[名詞]	杖むち	
\\	杖がないと困る。	
\\	複数形チェック
\\	[形容詞]	気分がやすらぐ	
\\	[名詞]	カプセル	
\\	このビルは男女宿泊可のカプセルホテルです。	
\\	複数形チェック
\\	[動詞]	ほめる、賞賛する	
\\	[名詞]	見出し、表題	
\\	複数形チェック
\\	[名詞]	監督官	
\\	それは私じゃないよ、理事長!	
\\	複数形チェック
\\	[名詞]	捕虜	
\\	この不確実な時代を生き抜くには、従来の価値観に囚われない柔軟な発想が必要だ。	
\\	複数形チェック
\\	[名詞]	ありふれた事	
\\	さまざまな才能に恵まれて、なんでもそつなくこなす姉にとって、平々凡々たる俺の存在はひどく許せないものらしい。	
\\	複数形チェック
\\	[名詞]	隊商幌馬車	
\\	犬は吠えても隊商は進む。	
\\	複数形チェック
\\	[名詞]	連邦、州	
\\	複数形チェック
\\	[名詞]	ボール紙	
\\	これらのダンボールは壊れ易い。	
\\	複数形チェック
\\	[形容詞]	話し好きな通信の	
\\	[名詞]	カートリッジ	
\\	プリンター内の空のインクカートリッジを交換して下さい。	
\\	複数形チェック
\\	[名詞]	共産主義	
\\	私の生きているうちに共産主義が効力を持つことは無いだろう。	
\\	複数形チェック
\\	[名詞]	彫刻	
\\	この包丁・・・有名な刀匠が作ったものなんです。	
\\	複数形チェック
\\	[名詞]	通勤者	
\\	どの電車も通勤者でぎゅうぎゅう詰めだった。	
\\	複数形チェック
\\	[名詞]	大災害、大惨事	
\\	不幸にも、その大惨事で生き残った乗客はほとんどいなかった。	
\\	複数形チェック
\\	[名詞]	区画、仕切った部分	
\\	このバッグは上の棚に入れた方がいいですか。	
\\	複数形チェック
\\	[動詞]	料理を提供する	
\\	これは学生相手の店です。	
\\	[動詞]	補償をする	
\\	病院で不愉快な思いをしている分を取り戻そうと思って、トムは自分の適量より少し多めにお酒を飲んだ。	
\\	[名詞]	地下室	
\\	複数形チェック
\\	[名詞]	能力、力量	
\\	複数形チェック
\\	[名詞]	国勢調査	
\\	複数形チェック
\\	[名詞]	補足して完全にするもの	
\\	完全自動詞は補語も目的語もとらない。	
\\	複数形チェック
\\	[形容詞]	中心にある	
\\	トムは自己中心的だ。	
\\	[形容詞]	幅広い、包括的な	
\\	すべてを網羅した教科書など存在しない。	
\\	[形容詞]	儀式上の	
\\	[動詞]	含む、構成する	
\\	[動詞]	証明する認定する	
\\	[名詞]	仲間	
\\	複数形チェック
\\	[形容詞]	障害のある	
\\	ベティは私にテニスの試合を挑んだ。	
\\	[動詞]	認める	
\\	[名詞]	首相	
\\	ゲアハルト・シュレーダーは第二次世界大戦を経験していない初のドイツ首相です。	
\\	複数形チェック
\\	[形容詞]	概念上の	
\\	[形容詞]	混沌とした	
\\	[名詞]	譲歩	
\\	複数形チェック
\\	[動詞]	あかぎれになる	
\\	毎年冬になると唇がカサカサに乾燥してしまいます。	
\\	[形容詞]	簡潔な	
\\	「ゲティスバーグ演説」は簡潔なスピーチです。	
\\	[名詞]	戦車	
\\	複数形チェック
\\	[名詞]	円錐	
\\	もう一つソフトクリームを食べようかな。	
\\	複数形チェック
\\	[名詞]	おしゃべり	
\\	政府は、政治問題に関するネット上の会話を監視し、誤った情報と判断した場合にそれを訂正する方法を模索している。	
\\	複数形チェック
\\	[動詞]	相談する	
\\	何か悩みがあるのなら、彼を相談するといいですよ。	
\\	[動詞]	大切にする	
\\	誰もが、心に故郷を抱いています。	
\\	[動詞]	信頼する	
\\	私は彼に心配事を打ち明けた。	
\\	[名詞]	栗	
\\	アナの髪は栗色だがマグダレナは金髪です。	
\\	複数形チェック
\\	[形容詞]	秘密の、機密の	
\\	敵対企業を貶めたり、秘密情報を入手したりと、大企業に欠かせない存在であるため、ハイリスクながらハイリターンが望める仕事だ。	
\\	[名詞]	ひよこ	
\\	彼女はひよこを買った。	
\\	複数形チェック
\\	[名詞]	確認	
\\	管理人によってアカウントの認証が行われたあと、あなたに確認のメールが送られます。	
\\	複数形チェック
\\	[動詞]	時刻を知らせる	
\\	「私もそう思うわ」と彼女は相づちを打った。	
\\	[動詞]	従う一致する	
\\	君は規則に従うべきだ。	
\\	[名詞]	聖歌隊合唱団	
\\	複数形チェック
\\	[名詞]	対決	
\\	複数形チェック
\\	[名詞]	雑用	
\\	最近何もかもめんどくさいって思ってしまう。	
\\	複数形チェック
\\	[形容詞]	結果として生じる	
\\	[動詞]	くすくす笑う	
\\	その手紙を読みながら、ブルースはひとりでくすくす笑っていた。	
\\	[動詞]	慰める	
\\	トムはメアリーを慰めている。	
\\	[名詞]	教会墓地	
\\	複数形チェック
\\	[名詞]	子音子音字	
\\	拗音(ゃ ゅ ょ)と促音(っ)の出し方も、加えていただけると参考になるかもしれません。	
\\	複数形チェック
\\	[動詞]	循環する巡回する	
\\	血液は体内を循環する。	
\\	[名詞]	陰謀	
\\	ハッキリ言って、この「陰謀説」はナンセンスだと思う。	
\\	複数形チェック
\\	[名詞]	市⺠権	
\\	彼は米国の市民権を得た。	
\\	複数形チェック
\\	[形容詞]	憲法の	
\\	憲法裁判所は、今日の朝の内に判決を下すだろう。	
\\	[形容詞]	市⺠の	
\\	[動詞]	熟考する	
\\	彼女は旅行を計画している。	
\\	[動詞]	競う	
\\	[動詞]	むさぼり食べる	
\\	彼はトラに食われてしまった。	
\\	[名詞]	続き延⻑	
\\	続きがあるのだ。	
\\	複数形チェック
\\	[名詞]	診断	
\\	飼鳥の医学―病気の診断とその治療。	
\\	複数形チェック
\\	[名詞]	連続継続	
\\	複数形チェック
\\	[名詞]	方言	
\\	うちなーぐちの勉強はまだやってるの?	
\\	複数形チェック
\\	[名詞]	短縮	
\\	陣痛の間隔はどれくらいですか。	
\\	複数形チェック
\\	[名詞]	直径	
\\	円周から直径を求めることは可能ですか?	
\\	複数形チェック
\\	[名詞]	請負業者	
\\	複数形チェック
\\	[動詞]	口述書き取りする	
\\	トムは秘書に口述筆記で手紙を書かせた。	
\\	[動詞]	相反する、反論する	
\\	お言葉を返すようで、申し訳ありませんが。	
\\	[名詞]	口授書き取り	
\\	今日は漢字の書き取りがある。	
\\	複数形チェック
\\	[名詞]	矛盾	
\\	恋とは巨大な矛盾であります。それなくしては生きられず、しかもそれによって傷つく。	
\\	複数形チェック
\\	[動詞]	区別する	
\\	鳩は訓練さえすれば、ピカソとモネの絵画の区別ができる。	
\\	[動詞]	考案する	
\\	[名詞]	勤勉	
\\	勤勉は成功への道だ。	
\\	複数形チェック
\\	[名詞]	監査役、経理部⻑	
\\	コントローラーを接続してください。	
\\	複数形チェック
\\	[名詞]	食事	
\\	食堂車はどこにありますか。	
\\	複数形チェック
\\	[名詞]	転換	
\\	複数形チェック
\\	[名詞]	免状卒業証書	
\\	複数形チェック
\\	[形容詞]	誠心誠意の	
\\	彼らは心からの歓迎を受けた。	
\\	[名詞]	外交	
\\	複数形チェック
\\	[名詞]	軍団	
\\	複数形チェック
\\	[名詞]	指令	
\\	複数形チェック
\\	[名詞]	死体	
\\	はアメリカのデスメタルバンドです。	
\\	複数形チェック
\\	[形容詞]	不愉快な	
\\	トムは気難しい顔で、人々があわただしくドアから出て行くのを見ていた。	
\\	[名詞]	通信	
\\	複数形チェック
\\	[名詞]	消失失踪	
\\	その医者の行方不明事件を調べなければならない。	
\\	複数形チェック
\\	[形容詞]	礼儀正しい	
\\	礼状を書くとは彼も礼儀正しかった。	
\\	[名詞]	武装解除	
\\	複数形チェック
\\	[名詞]	カニ	
\\	鱈場蟹は大型甲殻類である。	
\\	複数形チェック
\\	[形容詞]	悲惨なひどい	
\\	[形容詞]	割れ目のある	
\\	重みで氷が砕けた。	
\\	[名詞]	不信	
\\	複数形チェック
\\	[名詞]	債権者、貸主	
\\	複数形チェック
\\	[動詞]	捨てる	
\\	ふと見ると道端に自転車が捨てられていた。	
\\	[名詞]	小川	
\\	水から上がり、入江のほとりの砂の上を、よちよち歩き出しました。	
\\	複数形チェック
\\	[名詞]	弟子使徒	
\\	その時、二人の間に師弟を越えた友情のようなものが芽生えた事は言うまでもない。	
\\	複数形チェック
\\	[名詞]	手足の不自由な人	
\\	結婚したくないからおばあちゃんは僕の事をカタワだとか言ってました。	
\\	複数形チェック
\\	[名詞]	不快	
\\	かげでは、二人して僕のことを迂濶な奴、頓馬な奴、助平な奴などあざ笑っているのかも知れないと、僕は非常に不愉快を感じた。	
\\	複数形チェック
\\	[名詞]	基準	
\\	ひとつの重要な採用基準に「リーダーシップ」があることは間違いないです。	
\\	複数形チェック
\\	[名詞]	落胆	
\\	複数形チェック
\\	[名詞]	原油原料	
\\	原油価格の値下がりが続いている。	
\\	複数形チェック
\\	[形容詞]	いや気がさした	
\\	君にはうんざりしたよ。	
\\	[名詞]	パンくず	
\\	複数形チェック
\\	[名詞]	不名誉	
\\	複数形チェック
\\	[名詞]	ぱんの耳	
\\	複数形チェック
\\	[形容詞]	憂うつな	
\\	[名詞]	松葉づえ	
\\	松葉杖ってけっこうムズいな・・・。「すみません、ストレッチャー通ります!」	
\\	複数形チェック
\\	[名詞]	狼狽落胆	
\\	複数形チェック
\\	[形容詞]	教養のある、耕作された	
\\	かれらは土地をたがやした。	
\\	[名詞]	解雇	
\\	複数形チェック
\\	[名詞]	耕作	
\\	無農薬栽培とは、農薬を使わずに穀物、野菜、果物などの植物を栽培する方法です。	
\\	複数形チェック
\\	[名詞]	急送急派	
\\	複数形チェック
\\	[形容詞]	巻き毛の	
\\	彼は巻き毛だ。	
\\	[動詞]	取ってかわる	
\\	[名詞]	円柱	
\\	錠そのものは普通のシリンダー錠ですから、ドアを閉めたらオートロックするということはありません。	
\\	複数形チェック
\\	[名詞]	処分、処理	
\\	私は自由に使えるお金がたくさんある。	
\\	複数形チェック
\\	[名詞]	まぶしさ	
\\	複数形チェック
\\	[名詞]	性質傾向	
\\	彼女は性格が明るい。	
\\	複数形チェック
\\	[動詞]	耳を聞こえなくする	
\\	[名詞]	不満	
\\	複数形チェック
\\	[名詞]	学部⻑	
\\	複数形チェック
\\	[形容詞]	不満そうな	
\\	私は不満です。	
\\	[名詞]	デビュー	
\\	ならデビュー前に2回ライブ行ったけどな。	
\\	複数形チェック
\\	[動詞]	歪める	
\\	一部の新聞はそのニュースを歪めて報道した。	
\\	[形容詞]	人をだます	
\\	[形容詞]	ゆがんだ	
\\	一部の新聞はそのニュースを歪めて報道した。	
\\	[副詞]	はっきりと	
\\	[動詞]	気を散らす	
\\	カミーユはアランのうわの空の態度に不満だ。	
\\	[形容詞]	断固たる	
\\	[名詞]	配給者	
\\	複数形チェック
\\	[動詞]	と考える	
\\	彼は申し出を受諾することが賢明だと考えた。	
\\	[動詞]	そらす	
\\	[名詞]	欠乏	
\\	複数形チェック
\\	[名詞]	配当	
\\	複数形チェック
\\	[名詞]	代表、代理人	
\\	複数形チェック
\\	[名詞]	主義学説	
\\	進化論は研究で証明されているというのに、いまだに科学というより神話に基づいたような使い古された説明にしがみついている人たちがいる。	
\\	複数形チェック
\\	[名詞]	代表団	
\\	両国の代表団はジュネーブで会見した。	
\\	複数形チェック
\\	[形容詞]	教義上の	
\\	[名詞]	ほら穴、巣穴	
\\	世界は奇人の巣だ。	
\\	複数形チェック
\\	[名詞]	密集、密度	
\\	世界で一番人口が密集している国はどこですか?	
\\	複数形チェック
\\	[名詞]	ドーム	
\\	複数形チェック
\\	[名詞]	依存	
\\	複数形チェック
\\	[動詞]	寄贈・寄付する	
\\	献血して下さい。	
\\	[形容詞]	先祖伝来の	
\\	[名詞]	運命破滅死	
\\	複数形チェック
\\	[動詞]	荒廃させる	
\\	[名詞]	寮、寄宿舎	
\\	私の大学には寮があります。	
\\	複数形チェック
\\	[形容詞]	運命づけられた	
\\	彼は二度と妻に会うことはなかった。	
\\	[名詞]	服用量	
\\	1日に3度1錠ずつ服用してください。	
\\	複数形チェック
\\	[名詞]	駆逐艦	
\\	複数形チェック
\\	[形容詞]	思い切った、強烈な	
\\	日本は工業化の結果急激な変貌を遂げた。	
\\	[動詞]	引き離す	
\\	[形容詞]	夢見心地にする	
\\	[名詞]	洗剤	
\\	誤って洗剤を飲んでしまいました。	
\\	複数形チェック
\\	[形容詞]	眠い	
\\	彼女は夕食後眠くなった。	
\\	[形容詞]	発達の	
\\	[名詞]	継続時間、継続期間	
\\	複数形チェック
\\	[名詞]	夕暮れ	
\\	ロジャーは朝から晩まで働いている。	
\\	複数形チェック
\\	[名詞]	布,織物	
\\	複数形チェック
\\	[名詞]	はたき雑巾	
\\	複数形チェック
\\	[形容詞]	事実の	
\\	[形容詞]	義務に忠実な	
\\	[名詞]	華氏	
\\	水は華氏32度で凍る。	
\\	複数形チェック
\\	[名詞]	東方	
\\	飛行機は東へ旋回した。	
\\	複数形チェック
\\	[名詞]	虚偽	
\\	動物は虚偽と真実を区別できない。	
\\	複数形チェック
\\	[形容詞]	穏やかなのんきな	
\\	私ってのんきなの。	
\\	[名詞]	魅力	
\\	複数形チェック
\\	[形容詞]	風変わりな	
\\	母は変わり者です。	
\\	[名詞]	断食	
\\	断食中です。	
\\	複数形チェック
\\	[名詞]	生態学、生態系	
\\	複数形チェック
\\	[形容詞]	欠点のない	
\\	非の打ち所がない計画は、窮屈だなあ。	
\\	[形容詞]	食用の	
\\	これって食べられる?	
\\	[名詞]	反応	
\\	複数形チェック
\\	[名詞]	社説、論説	
\\	複数形チェック
\\	[名詞]	飼育者	
\\	複数形チェック
\\	[形容詞]	伸縮性のある	
\\	「どーした、もじもじして」「あーいや、何かパンツのゴム切れちゃったみたいで」	
\\	[形容詞]	女の	
\\	ジャスティンは女っぽい。	
\\	[動詞]	上げる、持ち上げる	
\\	そのうち、小さな村落の真上を高架線が行ったり来たりするようになりました。	
\\	[形容詞]	熱のある	
\\	とても熱があります。	
\\	[形容詞]	雄弁な	
\\	[形容詞]	激しい	
\\	[動詞]	船に積み込む	
\\	彼らは希望に胸をふくらませてその新しい事業に乗り出した。	
\\	[名詞]	イチジク	
\\	ぼくは履歴書なんか気にしないよ。	
\\	複数形チェック
\\	[動詞]	抱擁する	
\\	夢を抱きしめて。	
\\	[名詞]	指紋	
\\	トムは念入りにナイフから指紋を拭き取った。	
\\	複数形チェック
\\	[名詞]	出現	
\\	複数形チェック
\\	[形容詞]	会計の	
\\	[形容詞]	著名な	
\\	[名詞]	一片薄片	
\\	ヘロヘロになる前に、夏バテ対策をしっかり行い元気に夏を乗り切りましょう。	
\\	複数形チェック
\\	[名詞]	排気、排出(物)	
\\	陽電子放射断層撮影法では、放射性トレーサーを利用して体内の代謝活動を可視化します。	
\\	複数形チェック
\\	[動詞]	はためく	
\\	[動詞]	心を奪う	
\\	[名詞]	お世辞	
\\	複数形チェック
\\	[名詞]	囲い	
\\	複数形チェック
\\	[動詞]	ぽいと投げる	
\\	部屋の明かりのスイッチをカチッとつけた。	
\\	[名詞]	アンコール	
\\	複数形チェック
\\	[動詞]	顔を赤くする	
\\	顔赤いよ。	
\\	[動詞]	施行する、強制する	
\\	[名詞]	敵	
\\	複数形チェック
\\	[名詞]	執行	
\\	複数形チェック
\\	[形容詞]	霧の立ちこめた	
\\	霧が出ていますか?	
\\	[形容詞]	興味をそそる	
\\	[名詞]	⺠間伝承	
\\	複数形チェック
\\	[動詞]	増す、高める	
\\	[名詞]	歩道	
\\	複数形チェック
\\	[動詞]	豊かにする	
\\	旅は私たちの心を大いに豊かにするだろう。	
\\	[形容詞]	不気味なこわい	
\\	ベッドでの煙草を禁ずる厳しい規則がある。	
\\	[名詞]	愛好家	
\\	彼は音楽狂だ。	
\\	複数形チェック
\\	[形容詞]	力強い	
\\	[動詞]	包む	
\\	もやがロンドンの町をすっぽりと包んだ。	
\\	[名詞]	祖先	
\\	複数形チェック
\\	[名詞]	時代	
\\	複数形チェック
\\	[動詞] [不規則動詞]	先を見通す	
\\	その結果がどうなるか、誰も予見できない。	
\\	[名詞]	赤道	
\\	赤道における経度1度当たりの長さ。	
\\	複数形チェック
\\	[形容詞]	恐るべき	
\\	[形容詞]	性欲を刺激する	
\\	隠してあったエロ漫画を彼氏に見つけられてしまった。	
\\	[動詞]	公式化する	
\\	[動詞]	誤る	
\\	ブブー!残念でした。はずれです!	
\\	[動詞] [不規則動詞]	見捨てる	
\\	私を見捨てないで!	
\\	[名詞]	護衛	
\\	複数形チェック
\\	[形容詞]	やがて来る	
\\	[副詞]	平等に	
\\	そのお金は彼ら二人で平等に分けられるでしょう。	
\\	[名詞]	公開討論会	
\\	このフォーラムは素晴らしいです。	
\\	複数形チェック
\\	[形容詞]	結果として起こる	
\\	[名詞]	詐欺	
\\	最近、タチの悪い詐欺事件が多いわね。	
\\	複数形チェック
\\	[動詞]	優れる、秀でている	
\\	[名詞]	積荷、貨物	
\\	この航空会社は貨物輸送だけを取り扱っています。	
\\	複数形チェック
\\	[名詞]	叫び感嘆のことば	
\\	文を書くときには、ふつう大文字で始め、ピリオド
\\	、または感嘆符
\\	、疑問符
\\	、で終わる。	
\\	複数形チェック
\\	[名詞]	摩擦	
\\	両国の間では貿易摩擦がいつ生じてもおかしくない。	
\\	複数形チェック
\\	[名詞]	除外	
\\	複数形チェック
\\	[名詞]	恐怖	
\\	人々の心は恐怖でいっぱいだった。	
\\	複数形チェック
\\	[名詞]	執行、実行	
\\	複数形チェック
\\	[形容詞]	恐ろしい	
\\	[動詞]	用いる、発揮する	
\\	もっといい成績を得るように努力しなさい。	
\\	[名詞]	へり周辺	
\\	前髪が、おでこにかからないように髪を上げることをおすすめします。	
\\	複数形チェック
\\	[形容詞]	骨の折れる	
\\	[形容詞]	機能的な	
\\	[名詞]	亡命者	
\\	2006年9月の軍事クーデターで失脚、事実上亡命中だったタイのタクシン元首相が28日、約1年半ぶりに帰国した。	
\\	複数形チェック
\\	[形容詞]	激怒した	
\\	謝罪はあったにせよ、私はまだ彼に腹を立てている。	
\\	[形容詞]	存在する	
\\	[名詞]	炉	
\\	複数形チェック
\\	[動詞]	費やす	
\\	[形容詞]	家具付きの	
\\	彼女は部屋に美しい家具を備えた。	
\\	[名詞]	支出、経費	
\\	複数形チェック
\\	[名詞]	激しい怒り	
\\	複数形チェック
\\	[名詞]	専門知識	
\\	複数形チェック
\\	[名詞]	園芸	
\\	退職後、彼はガーデニングを始めた。	
\\	複数形チェック
\\	[形容詞]	表現豊かな	
\\	彼女の歌声はとても表現力豊かだ。	
\\	[動詞]	うがいする	
\\	[形容詞]	非常に美しい	
\\	[名詞]	ジャンル	
\\	歴史小説はとても人気の高いジャンルでした。	
\\	複数形チェック
\\	[名詞]	まつ毛	
\\	複数形チェック
\\	[形容詞]	地理的な	
\\	[名詞]	まぶた	
\\	左のまぶたがぴくぴくする。	
\\	複数形チェック
\\	[名詞]	地質学、地質	
\\	トムは地質学を専攻してた。	
\\	複数形チェック
\\	[名詞]	視力	
\\	彼は視力がいい。	
\\	複数形チェック
\\	[形容詞]	巨大な膨大な	
\\	あのフットボール選手はガタイがいい。	
\\	[動詞]	くすくす笑う	
\\	メアリーはクックッと笑った。	
\\	[名詞]	困難	
\\	中学生が英語を学ぶ際の最難関の一つが関係代名詞です。	
\\	複数形チェック
\\	[名詞]	当然のこと	
\\	私はタバコを吸うのをやめた。	
\\	複数形チェック
\\	[動詞]	投げ付ける	
\\	[名詞]	氷河	
\\	複数形チェック
\\	[動詞]	静かにさせる	
\\	静かに!	
\\	[名詞]	ぎらぎらする光	
\\	複数形チェック
\\	[形容詞]	がんじょうな	
\\	[名詞]	金魚	
\\	金魚には思い出がある。	
\\	複数形チェック
\\	[名詞]	水素	
\\	水分子は、2個の水素原子と1個の酸素原子からなる。	
\\	複数形チェック
\\	[名詞]	福音	
\\	複数形チェック
\\	[名詞]	仮説	
\\	これらのデータはその仮説を支持している。	
\\	複数形チェック
\\	[形容詞]	丁重な	
\\	慈悲深く慈悲あまねき神の御名において	
\\	[形容詞]	理想主義的な	
\\	信じられないような文化的退廃、驚くほどの無知、理想ばかりのガラクタ、真実を隠蔽する政策がアメリカの保守層のあいだでもっとも根強い支持を得ているのだ。	
\\	[動詞]	喜ばせる	
\\	[形容詞]	全く同じの	
\\	私にはそっくりの双子がいます。	
\\	[名詞]	潤滑剤	
\\	複数形チェック
\\	[形容詞]	身元を確認しうる	
\\	[名詞]	強欲	
\\	たくさんのおもちゃを持っていたが、強欲から彼はもっと欲しがった。	
\\	複数形チェック
\\	[名詞]	偶像	
\\	この問題は、アイドルのエゴとヲタのエゴの衝突だ。	
\\	複数形チェック
\\	[名詞]	花婿	
\\	次の曲を、新郎と新婦に捧げたいと思います。	
\\	複数形チェック
\\	[動詞]	照らす	
\\	天才とは、彼らの世紀を照らして光輝くべく運命づけられた流星である。	
\\	[動詞]	うなる	
\\	お腹がグーグー鳴っている。	
\\	[名詞]	照明照度	
\\	複数形チェック
\\	[動詞]	不平を言う	
\\	愚痴はやめろよ。	
\\	[動詞]	懇願する	
\\	[名詞]	保護者管理者	
\\	クリストファーコロンブスはある時、ケルベロスという3つの頭を持った冥界の番犬と、自分の帽子しか使わないで戦った。	
\\	複数形チェック
\\	[形容詞]	威圧するような	
\\	すき焼きをごちそうになって、一晩泊めてもらったうえに、朝めしまで食べてきたなんて、ちょっと厚かましすぎるんじゃないか。	
\\	[名詞]	カモメ	
\\	複数形チェック
\\	[名詞]	投獄	
\\	複数形チェック
\\	[名詞]	射撃砲撃	
\\	パリで起きた同時多発テロや米カリフォルニア州サンバーナディノでの銃乱射事件を背景に、米国に在住するイスラム教徒(ムスリム)に対する風当たりが近年になく強まっている。	
\\	複数形チェック
\\	[形容詞]	不適当な	
\\	適切にアースされていないマイクに触れると、命に関わる場合があります。何人かの歌手は実際そうやって亡くなっているのです。	
\\	[形容詞]	毛深い	
\\	彼は胸毛が多い。	
\\	[名詞]	無力	
\\	迷信というのは、偶然はただの偶然なのだと、人間が理解できないことから生じる。	
\\	複数形チェック
\\	[動詞]	半分にする	
\\	何事も中途半端にするな。	
\\	[形容詞]	不活発な	
\\	[名詞]	手芸手芸品	
\\	複数形チェック
\\	[名詞]	報奨	
\\	複数形チェック
\\	[名詞]	プリントちらし	
\\	複数形チェック
\\	[形容詞]	絶え間のない	
\\	[名詞]	握手	
\\	複数形チェック
\\	[名詞]	発生	
\\	複数形チェック
\\	[名詞]	手書き筆跡	
\\	彼は字が下手だ。	
\\	複数形チェック
\\	[副詞]	偶然に	
\\	ちなみに、告げることがあるんだけどさ。	
\\	[名詞]	野ウサギ	
\\	クズリは多くのウサギやクロライチョウ、エゾライチョウ、小さなげっ歯動物を捕獲する。	
\\	複数形チェック
\\	[名詞]	含有	
\\	複数形チェック
\\	[形容詞]	調和のとれた	
\\	[形容詞]	包括的な	
\\	[動詞]	ふ化する	
\\	めんどりは卵がかえるまで抱く。	
\\	[名詞]	不便	
\\	ご面倒をお掛けして申し訳ございません。	
\\	複数形チェック
\\	[動詞]	引っ張る、運ぶ	
\\	[形容詞]	不便な	
\\	私は入り口のない都市に住むと困ると思います。	
\\	[名詞]	危険	
\\	複数形チェック
\\	[動詞]	合併する	
\\	トムの会社はケイマン諸島で設立されました。	
\\	[名詞]	炉辺家庭	
\\	複数形チェック
\\	[名詞]	指標	
\\	複数形チェック
\\	[副詞]	心から	
\\	[形容詞]	憤慨した	
\\	彼女は息子がひどい扱いを受けたと憤慨していた。	
\\	[形容詞]	心からの	
\\	[名詞]	憤り、憤慨	
\\	複数形チェック
\\	[動詞]	心に留める	
\\	世論に聞け。	
\\	[名詞]	屈辱、冷遇	
\\	複数形チェック
\\	[動詞]	高くする	
\\	[形容詞]	絶対必要な	
\\	水は生命に必要不可欠です。	
\\	[名詞]	今後,将来	
\\	これに関して詳細は後で。	
\\	複数形チェック
\\	[動詞]	引き起こす	
\\	[動詞]	乗っ取る	
\\	[動詞]	甘やかす	
\\	彼は妻を失ってから酒にふけり始めた。	
\\	[形容詞]	後ろの,後部の	
\\	[形容詞]	経験不足の	
\\	君はまだ未熟だ。	
\\	[動詞]	妨げる	
\\	私の邪魔をするものは何もない。	
\\	[形容詞]	発明の才のある	
\\	利口なその少年は自分の発明品で賞を得た。	
\\	[名詞]	妨害	
\\	やる気ないなら帰れ。邪魔になるだけだから。	
\\	複数形チェック
\\	[形容詞]	生来の、特有の	
\\	[動詞]	シューという音を立てる	
\\	[名詞]	注射	
\\	大丈夫、この注射は痛くないからね。少しチクッとするだけだよ。	
\\	複数形チェック
\\	[名詞]	ミツバチの巣	
\\	ミツバチは、においのサンプルを巣に持ち帰ることによって食糧のありかを伝える。	
\\	複数形チェック
\\	[名詞]	不正	
\\	不正に対して反対の声を上げろ。	
\\	複数形チェック
\\	[名詞]	同窓会	
\\	複数形チェック
\\	[名詞]	内陸	
\\	三人の帰還軍人が瀬戸内海沿岸のある小さな町のはずれに一軒の家を借りて共同生活をしている。	
\\	複数形チェック
\\	[名詞]	ひづめ	
\\	かなり疲れたよ、4時間も歩いたからさ。	
\\	複数形チェック
\\	[名詞]	無罪	
\\	潔白であることは美しいことだ。	
\\	複数形チェック
\\	[副詞]	うまくいけば	
\\	[形容詞]	たくさんの	
\\	宇宙には無数の星がある。	
\\	[名詞]	星占い	
\\	複数形チェック
\\	[形容詞]	正気でない	
\\	お前ら狂ってる。	
\\	[名詞]	蹄鉄	
\\	これは
\\	字形磁石です。	
\\	複数形チェック
\\	[形容詞]	わずかばかりの	
\\	俺はなんの意味もない無価値な存在。	
\\	[名詞]	ユースホステル	
\\	この辺りにユースホステルはありますか?	
\\	複数形チェック
\\	[名詞]	設置	
\\	複数形チェック
\\	[名詞]	猟犬	
\\	猟犬たちはそのキツネを追いかけている。	
\\	複数形チェック
\\	[形容詞]	本能的な	
\\	[形容詞]	時間ごとの	
\\	[形容詞]	侮辱的な	
\\	「不具者」という言葉は侮辱かもしれませんね。	
\\	[名詞]	世帯主	
\\	複数形チェック
\\	[形容詞]	無傷の	
\\	[形容詞]	湿気の多い	
\\	湿度がすごく高いですね。	
\\	[名詞]	摂取(量)	
\\	日本では、諸外国と比較して食生活におけるトランス脂肪酸の平均摂取量は少なく、相対的に健康への影響は少ないと主張する説もある。	
\\	複数形チェック
\\	[名詞]	謙虚	
\\	謙虚さを身につけて。	
\\	複数形チェック
\\	[形容詞]	理解できる	
\\	[動詞]	強める増強する	
\\	[形容詞]	寂しい,心細い	
\\	彼女は寂しそうに見える。	
\\	[形容詞]	集中的な	
\\	[形容詞]	⻑い間の	
\\	[形容詞]	故意の	
\\	彼の誤りは意図的なものだった。	
\\	[動詞]	輪でくくる	
\\	今朝、ジムが旗に乳を付けていた。	
\\	[名詞]	相互作用	
\\	登場人物の躍動的な関わり合いこそがこの小説をかくも偉大なものとしている。	
\\	複数形チェック
\\	[名詞]	トラック	
\\	複数形チェック
\\	[名詞]	妨害	
\\	複数形チェック
\\	[名詞]	化粧水	
\\	複数形チェック
\\	[名詞]	交差点	
\\	その交差点を右に曲がってください。	
\\	複数形チェック
\\	[名詞]	宝くじ	
\\	結婚は運次第。	
\\	複数形チェック
\\	[動詞]	侵入する立ち入る	
\\	[名詞]	昼食会	
\\	複数形チェック
\\	[形容詞]	無効な、妥当でない	
\\	小学校低学年の頃、僕は父に俳句を教えられ、俳人に憧れた。七夕の願い事も「はい人になれますように。」と書いた。あれから30年、願いはかなった。今や僕は紛れもない廃人だ。	
\\	[形容詞]	ぜいたくな,豪華な	
\\	私は豪華なホテルでは落ち着かない。	
\\	[形容詞]	いつも	
\\	[副詞]	猛烈に	
\\	[形容詞]	皮肉な	
\\	[動詞]	拡大する	
\\	彼女は自分の苦しみを誇張して話した。	
\\	[名詞]	かゆみ	
\\	目がかゆいです。	
\\	複数形チェック
\\	[名詞]	大きいこと	
\\	彼は犯した罪の大きさを悟った。	
\\	複数形チェック
\\	[名詞]	宝石商	
\\	複数形チェック
\\	[名詞]	主流、本流	
\\	複数形チェック
\\	[名詞]	騎手	
\\	複数形チェック
\\	[名詞]	悪意	
\\	ベンは悪意から私の噂をばらまいた。	
\\	複数形チェック
\\	[名詞]	水差し	
\\	複数形チェック
\\	[形容詞]	明白な	
\\	この文を三回読むと、眠っている間に幽霊がやってくるんだ。	
\\	[名詞]	接合点	
\\	複数形チェック
\\	[名詞]	大邸宅	
\\	彼女はちょっとした邸宅に住んでいる。	
\\	複数形チェック
\\	[名詞]	粒、中心部	
\\	複数形チェック
\\	[名詞]	(地球の)マントル	
\\	複数形チェック
\\	[名詞]	血縁,親族	
\\	複数形チェック
\\	[名詞]	カエデ	
\\	トムはパンケーキにかかってるメープルシロップが好きだ。	
\\	複数形チェック
\\	[名詞]	取っ手つまみ	
\\	ノブを回して、ドアをあけなさい。	
\\	複数形チェック
\\	[名詞]	沼地	
\\	その沼にはサギがたくさん住んでいる。	
\\	複数形チェック
\\	[形容詞]	物知りの	
\\	トムは現代のポピュラー音楽について知識が豊富だ。	
\\	[名詞]	元帥将官	
\\	複数形チェック
\\	[名詞]	指の関節	
\\	複数形チェック
\\	[形容詞]	男らしい,力強い	
\\	[形容詞]	嘆き悲しむ	
\\	[名詞]	成熟	
\\	複数形チェック
\\	[名詞]	地主	
\\	大家に見つからぬようにしなくてはならぬようになってきた。	
\\	複数形チェック
\\	[名詞]	調停者	
\\	複数形チェック
\\	[名詞]	経過	
\\	複数形チェック
\\	[動詞]	瞑想する	
\\	メアリさんは凝らししながら幽体離脱を経験しました。	
\\	[名詞]	むち打ち	
\\	複数形チェック
\\	[名詞]	瞑想	
\\	父は目を閉じてじっと考え込んで座っていた。	
\\	複数形チェック
\\	[名詞]	緯度	
\\	タスマニアは南緯40度のところにあります。	
\\	複数形チェック
\\	[形容詞]	憂うつな,陰気な	
\\	ぼくは涼宮ハルヒの憂鬱を読みたくない。	
\\	[名詞]	洗面所	
\\	複数形チェック
\\	[名詞]	メモ	
\\	複数形チェック
\\	[形容詞]	合法の	
\\	[名詞]	威嚇、脅威	
\\	複数形チェック
\\	[名詞]	ちらし広告	
\\	複数形チェック
\\	[名詞]	考え方精神状態	
\\	複数形チェック
\\	[名詞]	賃貸、賃貸契約	
\\	複数形チェック
\\	[名詞]	商品	
\\	複数形チェック
\\	[名詞]	講演者	
\\	複数形チェック
\\	[形容詞]	慈悲深い	
\\	慈悲深く慈悲あまねき神の御名において	
\\	[名詞]	残り物	
\\	食べ残しはお持ち帰りください。	
\\	複数形チェック
\\	[名詞]	水銀	
\\	水星は太陽に一番近い惑星です。	
\\	複数形チェック
\\	[形容詞]	立法上の	
\\	[形容詞]	取り散らかした	
\\	今日部屋の掃除したばかりなのに、なんでこんなに散らかってるの?	
\\	[形容詞]	合法の	
\\	どちらの王子が正統な王位継承者か。	
\\	[名詞]	主要都市	
\\	日本には1都1道2府43県があります。	
\\	複数形チェック
\\	[動詞]	⻑くする延ばす	
\\	[形容詞]	首都の、大都市の	
\\	この4人の若者が都内でルームシェアしている。	
\\	[名詞]	(動物の)ヒョウ	
\\	トムは豹のようにすばしっこい。	
\\	複数形チェック
\\	[名詞]	総マイル数	
\\	複数形チェック
\\	[動詞]	減らす	
\\	痛みが少し減りました。	
\\	[名詞]	坑夫	
\\	複数形チェック
\\	[名詞]	負債、債務	
\\	複数形チェック
\\	[形容詞]	奇跡的な	
\\	[形容詞]	責任がある	
\\	[動詞]	置き忘れる	
\\	私は昨晩キーを置き忘れたらしい。	
\\	[名詞]	大尉中尉	
\\	複数形チェック
\\	[名詞]	修正	
\\	複数形チェック
\\	[名詞]	救助艇	
\\	複数形チェック
\\	[名詞]	君主制、君主国	
\\	英国では多くの人が王室への公的支出の廃止を望んでいる。	
\\	複数形チェック
\\	[動詞]	明るくする	
\\	気楽にいけよ。	
\\	[名詞]	修道士	
\\	複数形チェック
\\	[名詞]	灯台	
\\	複数形チェック
\\	[形容詞]	単調な	
\\	彼女は生活が単調だという。	
\\	[名詞]	足の不自由	
\\	複数形チェック
\\	[形容詞]	まったくひどい	
\\	[形容詞]	直線の線状の	
\\	[名詞]	死ぬ運命	
\\	複数形チェック
\\	[動詞]	いつまでも残る	
\\	[名詞]	抵当,担保	
\\	どうやって住宅ローンを組んだり、医療費を払ったり、子供の大学教育のための貯金をすればよいのだろうかろと子供が寝たあとにも目を覚ましたまま横になり、考える父母がいる。	
\\	複数形チェック
\\	[形容詞]	言葉の、言語の	
\\	[名詞]	静止した	
\\	複数形チェック
\\	[名詞]	暮らし、生計(の手段)	
\\	複数形チェック
\\	[名詞]	動機づけ	
\\	彼はやる気がない。	
\\	複数形チェック
\\	[名詞]	トカゲ	
\\	彼女はももにトカゲの入れ墨がついています。	
\\	複数形チェック
\\	[名詞]	引っ越し業者	
\\	複数形チェック
\\	[名詞]	機関車	
\\	汽車の汽笛で夜明けに目を覚まされた。	
\\	複数形チェック
\\	[形容詞]	地方自治体の	
\\	自治体はゴミ処理問題に頭を抱えている。	
\\	[形容詞]	高慢な	
\\	[動詞]	熟考する	
\\	「小説? あたし小説は嫌ひですの」  おゝ、ミュウズよ、彼女の冒涜を赦せ。	
\\	[形容詞]	説得力のある	
\\	トムって説得力があると思う。	
\\	[形容詞]	無言の,沈黙した	
\\	聾唖者は手話で話す。	
\\	[名詞]	悲観主義	
\\	彼の悲観論は周囲の人々滅入らせた。	
\\	複数形チェック
\\	[名詞]	愛国主義、国家主義	
\\	ナショナリズムをファシズムと同一視してはいけない。	
\\	複数形チェック
\\	[名詞]	悲観主義者	
\\	複数形チェック
\\	[動詞]	国有化する	
\\	[名詞]	嘆願	
\\	私は市長に請願を出した。	
\\	複数形チェック
\\	[動詞]	操縦する	
\\	[名詞]	薬剤師	
\\	複数形チェック
\\	[名詞]	航行者	
\\	複数形チェック
\\	[形容詞]	絵のように美しい	
\\	ナポリは絵のように美しい都市です。	
\\	[名詞]	過失	
\\	複数形チェック
\\	[形容詞]	刺し通すような	
\\	トムの声はよく通る。	
\\	[動詞]	推薦する	
\\	[形容詞]	信仰深い	
\\	いかにも敬虔なるクリスチャンが送る、礼節重き言葉です。	
\\	[形容詞]	顕著な、有名な	
\\	[名詞]	パイプライン	
\\	複数形チェック
\\	[接続詞]	それにもかかわらず	
\\	[形容詞]	みじめな	
\\	[動詞]	栄養を与える	
\\	水分と栄養しっかり摂って、早く治してね。	
\\	[形容詞]	惑星の	
\\	[名詞]	誓い誓言	
\\	騎士が王への忠誠を誓った。	
\\	複数形チェック
\\	[動詞]	しっくいを塗る	
\\	ギプスのせいで左手の自由がきかない。	
\\	[形容詞]	頑固な、強情な	
\\	彼女は強情です。	
\\	[形容詞]	遊び好きな	
\\	彼女は笑い、私をいたずらっぽく小突いた。	
\\	[名詞]	発生	
\\	複数形チェック
\\	[動詞]	引きぬく	
\\	彼女にヒナギクの花を摘んでやった。	
\\	[副詞]	奇妙に	
\\	彼の振る舞いは奇妙である。	
\\	[名詞]	配管工	
\\	複数形チェック
\\	[名詞]	省略	
\\	省略は立派な表現技法の一つであり、多くの文法書でも紹介されています。	
\\	複数形チェック
\\	[形容詞]	肉付きの良い	
\\	私は太り気味かもしれないが、すこぶる健康だよ。	
\\	[形容詞]	経営上の	
\\	彼女を手術室に運んでくれ。	
\\	[名詞]	中毒	
\\	食中毒に罹ったことはありますか。	
\\	複数形チェック
\\	[動詞]	圧迫する	
\\	王は人民を抑圧した。	
\\	[形容詞]	極の、極地の	
\\	[名詞]	圧迫	
\\	少数民族は偏見、貧困、抑圧と戦っている。	
\\	複数形チェック
\\	[名詞]	ありあわせの料理	
\\	複数形チェック
\\	[名詞]	楽天主義	
\\	複数形チェック
\\	[名詞]	陶器	
\\	東洋の陶器に興味があります。	
\\	複数形チェック
\\	[名詞]	楽天主義者	
\\	ヘレンは生来楽天家だ。	
\\	複数形チェック
\\	[名詞]	小袋	
\\	複数形チェック
\\	[形容詞]	整然とした	
\\	[動詞]	先に起こる	
\\	お噂はかねがね伺っております。	
\\	[副詞]	普通は	
\\	[名詞]	正確さ	
\\	測定には寸分の狂いもない精度が要求される。	
\\	複数形チェック
\\	[名詞]	服装ひとそろい	
\\	彼女は立派な服装をしていた。	
\\	複数形チェック
\\	[名詞]	前任者	
\\	複数形チェック
\\	[名詞]	出費,支出	
\\	彼は外向的だ。	
\\	複数形チェック
\\	[名詞]	序文前置き	
\\	複数形チェック
\\	[名詞]	遠足	
\\	社員旅行の予定、あまりぱっとしないよ。	
\\	複数形チェック
\\	[形容詞]	望ましい	
\\	就職の面接には派手なネクタイよりも地味なネクタイの方が好ましい。	
\\	[名詞]	暴行不法行為	
\\	複数形チェック
\\	[形容詞]	第一位の	
\\	今日の新聞によると、首相は渡米を断念したそうだ。	
\\	[動詞]	転覆させる	
\\	最高裁は原判決をくつがえしました。	
\\	[動詞]	(薬を)処方する	
\\	医師は彼女に薬を処方した。	
\\	[名詞]	概略	
\\	経営者には企業全体を俯瞰する目が欠かせない。	
\\	複数形チェック
\\	[動詞]	主宰する、管理する	
\\	[名詞]	バケツ	
\\	複数形チェック
\\	[名詞]	名声、威信	
\\	複数形チェック
\\	[名詞]	広大な眺め	
\\	複数形チェック
\\	[形容詞]	たぶん、おそらく	
\\	向かいの保育園から、保育士だろう男の声で「ガオー、食べちゃうぞ!」というのが聞こえて来る。	
\\	[形容詞]	ローマ教皇の	
\\	[動詞]	推定する	
\\	彼はお金を返したと思う。	
\\	[動詞]	麻痺させる	
\\	僕は僕の妻を半身不随の動物としか思えないのだ。	
\\	[形容詞]	流行の	
\\	[名詞]	パセリ	
\\	複数形チェック
\\	[形容詞]	普及している	
\\	この冬風邪が大流行である。	
\\	[動詞] [不規則動詞]	参加する	
\\	[形容詞]	前の,先の	
\\	届け出なしに路上に出店してはならない。	
\\	[名詞]	牧師	
\\	ボブは牧師になりました。	
\\	複数形チェック
\\	[動詞]	公言する	
\\	[形容詞]	哀れな	
\\	[名詞]	投影	
\\	まず発声練習から始めましょう。	
\\	複数形チェック
\\	[名詞]	小道	
\\	複数形チェック
\\	[名詞]	代名詞	
\\	関係代名詞は「接続詞
\\	代名詞」のはたらきをしています。	
\\	複数形チェック
\\	[名詞]	愛国者	
\\	率直に言えば、彼は愛国者というよりはむしろ偽善者だ。	
\\	複数形チェック
\\	[動詞]	立てかける	
\\	[形容詞]	愛国的な	
\\	建国記念の日(2月11日):建国をしのび、国を愛する心を養う。	
\\	[名詞]	宣伝	
\\	複数形チェック
\\	[動詞]	舗装する	
\\	この砂利道は来年舗装されるだろう。	
\\	[名詞]	預言者	
\\	預言者は何世紀もの間、この世の終わりを予言し続けてきた。	
\\	複数形チェック
\\	[形容詞]	支払い可能な	
\\	小切手はABC社あてに支払われるように振り出してください。	
\\	[形容詞]	見込みのある	
\\	[名詞]	支払人払渡人	
\\	複数形チェック
\\	[動詞]	切り払う	
\\	この木は剪定できません。余りに高すぎます。	
\\	[名詞]	小石	
\\	投げた小石は水を切って飛んだ。	
\\	複数形チェック
\\	[名詞]	脈拍	
\\	あなたの脈は普通です。	
\\	複数形チェック
\\	[名詞]	歩行者	
\\	複数形チェック
\\	[名詞]	時間厳守	
\\	ヨーロッパでは、時間厳守は当然のこととみなされる。	
\\	複数形チェック
\\	[名詞]	掛けくぎ栓	
\\	ドルペッグの意味は、強い国につけということです。	
\\	複数形チェック
\\	[形容詞]	古風で趣きのある	
\\	[名詞]	半島	
\\	複数形チェック
\\	[形容詞]	資格のある	
\\	彼は昇進の資格がある。	
\\	[形容詞]	永続する	
\\	ここからはあの山の万年雪が見えます。	
\\	[名詞]	クォート	
\\	複数形チェック
\\	[動詞]	当惑させる	
\\	彼は戸惑っていた。	
\\	[形容詞]	奇妙な、変な	
\\	彼は奇妙な考えを持っている。	
\\	[名詞]	説得説得力	
\\	複数形チェック
\\	[動詞]	速める急がせる	
\\	[名詞]	引用引用文	
\\	次の一節はある有名な寓話から引用したものです。	
\\	複数形チェック
\\	[形容詞]	腐った	
\\	この食べ物は腐った匂いがする。	
\\	[名詞]	急襲	
\\	複数形チェック
\\	[形容詞]	回り道の	
\\	回りくどい言い方はしないでくれ。	
\\	[動詞]	くま手でかきならす	
\\	彼は儲けている。	
\\	[動詞]	目覚めさせる	
\\	その音で彼女は眠りから覚めた。	
\\	[名詞]	牧場	
\\	複数形チェック
\\	[名詞]	著作権料印税	
\\	複数形チェック
\\	[名詞]	発疹	
\\	無茶なことするなよ。	
\\	複数形チェック
\\	[名詞]	敷物,じゅうたん	
\\	彼は同僚に足をすくわれた。	
\\	複数形チェック
\\	[名詞]	合理性	
\\	複数形チェック
\\	[形容詞]	でこぼこした	
\\	[動詞]	合理化する	
\\	[動詞]	サラサラ鳴る	
\\	[名詞]	実現、認識	
\\	複数形チェック
\\	[名詞]	保護保護手段	
\\	複数形チェック
\\	[動詞]	再び現れる	
\\	[名詞]	酒場	
\\	複数形チェック
\\	[形容詞]	反抗的な謀反の	
\\	息子はこの頃反抗期みたいで、やたらと私たちに反発するの。	
\\	[形容詞]	正気の	
\\	トムは正気なんだろうか。	
\\	[動詞]	暗唱する	
\\	アルファベットを逆から言ってください。	
\\	[形容詞]	衛生的な	
\\	アフリカ大陸のどこかへ行くとしたらどこが一番衛生的で安全ですか?	
\\	[動詞]	数え上げる、思う	
\\	私たちはもういちど彼に会おうと思います。	
\\	[名詞]	風刺	
\\	複数形チェック
\\	[動詞]	思い出す	
\\	[形容詞]	恥ずべき	
\\	[名詞]	記憶	
\\	複数形チェック
\\	[形容詞]	乏しい	
\\	[動詞]	和解する	
\\	ねえ、仲直りしよう。	
\\	[名詞]	緋色緋色の服	
\\	複数形チェック
\\	[名詞]	和解	
\\	宗教間の和解は、世界平和の礎である。	
\\	複数形チェック
\\	[名詞]	ひしゃく	
\\	すいません、レギュラーサイズのダブルコーンで、チョコミントとオレンジシャーベットをお願いします。	
\\	複数形チェック
\\	[名詞]	葦	
\\	複数形チェック
\\	[形容詞]	軽蔑するさげすむ	
\\	[名詞]	審判員	
\\	複数形チェック
\\	[名詞]	縫い目継ぎ目	
\\	複数形チェック
\\	[名詞]	精練洗練上品	
\\	複数形チェック
\\	[名詞]	航海者水夫	
\\	複数形チェック
\\	[名詞]	繰り返し語句	
\\	喫煙をお控え下さい。	
\\	複数形チェック
\\	[形容詞]	季節の	
\\	遅延便については、オリジナルの出発日に基づくシーズナリティを適用するため、マイル差額の払い戻しはいたしません。	
\\	[名詞]	政権	
\\	複数形チェック
\\	[名詞]	調味料	
\\	気晴らしを求める王女の意欲はとどまるところを知りませんでした。そのため、私たちはそうした気晴らしに、それをいっそう刺激的なものにしうる新たな調味料を加えることばかりを考えていました。	
\\	複数形チェック
\\	[形容詞]	残念な	
\\	ほんとに申し訳ないです。	
\\	[名詞]	探求者	
\\	複数形チェック
\\	[名詞]	リハビリ	
\\	コストダウンもできない生ぬるいリストラが
\\	再建を危うくする。	
\\	複数形チェック
\\	[形容詞]	選択できる	
\\	[名詞]	手綱	
\\	複数形チェック
\\	[名詞]	準決勝試合	
\\	複数形チェック
\\	[動詞]	補強する	
\\	[形容詞]	人騒がせな	
\\	[名詞]	リレー	
\\	リレー用のバトンが1本足りないんです。	
\\	複数形チェック
\\	[形容詞]	非常識な無意味な	
\\	[名詞]	気が進まないこと	
\\	複数形チェック
\\	[形容詞]	感傷的な	
\\	[名詞]	再開発、再生	
\\	複数形チェック
\\	[名詞]	軍曹	
\\	複数形チェック
\\	[動詞]	再編成する	
\\	[名詞]	説教	
\\	複数形チェック
\\	[動詞]	後悔する	
\\	[名詞]	蛇	
\\	信じる人々には次のような印が伴います。すなわち、私の名によって悪霊を追い出し、新しい言葉を語り、蛇をもつかみ、たとい毒を飲んでも決して害を受けず、また、病人に手を置けば病人は癒やされます。	
\\	複数形チェック
\\	[動詞]	再版する	
\\	[名詞]	配置、組立て	
\\	複数形チェック
\\	[動詞]	非難する	
\\	[名詞]	性欲、性行為	
\\	複数形チェック
\\	[名詞]	爬虫類	
\\	爬虫類が大嫌い。	
\\	複数形チェック
\\	[形容詞]	見すぼらしい	
\\	結局、私たちはそのボロボロなレストランにご飯を食べてしまいました。	
\\	[形容詞]	耐性のある	
\\	除草剤耐性品種でなぜ収量が増えるのか?	
\\	[形容詞]	暗い	
\\	[形容詞]	それぞれの	
\\	タトエバ:私たちは私たちのそれぞれの言語に有識者であってみる場所…ええと、やってみる。	
\\	[名詞]	柄矢柄	
\\	複数形チェック
\\	[形容詞]	それぞれ	
\\	[名詞]	ショール、肩掛け	
\\	複数形チェック
\\	[形容詞]	すぐに反応する	
\\	最も力の強い種や、最も知能の高い種が生き残るのではなく、変化に最も良く適応できる種が生き残るのです。	
\\	[名詞]	シェリー酒	
\\	ガブリエルは熱いスープとシェリー酒を少し飲んだだけだった。	
\\	複数形チェック
\\	[形容詞]	制限する	
\\	[形容詞]	光る	
\\	太陽が輝いている。	
\\	[形容詞]	意味深い	
\\	メアリーは露出度の高い服をよく着ている。	
\\	[動詞]	金切り声を出す	
\\	そんなに耳元でギャーギャーわめくなよ。	
\\	[名詞]	暴露発覚	
\\	複数形チェック
\\	[名詞]	低木	
\\	「どうしたの?」「あのね、植木の枝の剪定をして欲しかったのよ」	
\\	複数形チェック
\\	[名詞]	崇拝	
\\	ベンヤミンは優れた芸術作品を前にして人が経験するであろう畏怖や崇敬の感覚を指して「アウラ」という語を用いた。	
\\	複数形チェック
\\	[動詞]	肩をすくめる	
\\	トムは肩をすくめた。	
\\	[名詞]	言葉遣い	
\\	複数形チェック
\\	[形容詞]	絹のような	
\\	[動詞]	あざ笑う	
\\	最近のぼくは、「生き恥」「赤恥」をかくことを恐れなくなりました。	
\\	[動詞]	簡素化する	
\\	簡単なフランス語に書き直された名作とか読んでる。	
\\	[名詞]	縁	
\\	かつて私のラテン語教師は、メガネの縁からいかめしく私を見下していた。でも今は、それはただ、彼女が読書用のメガネをかけていたので、それをしょっちゅう外す事をやっかいに思ったのだと分かる。つまり、私たち生徒に向けてまるで蔑んでいるように見えていたものは、どうやら実際は優しさだったのである。	
\\	複数形チェック
\\	[形容詞]	同時の	
\\	私が同時通訳を務めた。	
\\	[名詞]	暴動、騒動	
\\	庭には様々な花が咲き乱れていた。	
\\	複数形チェック
\\	[動詞]	ちびちび飲む	
\\	トムは水の入ったグラスを持ち上げ、一口飲んだ。	
\\	[動詞]	成熟する	
\\	[動詞]	位置を与える	
\\	日本はアジアにあります。	
\\	[名詞]	ローラー	
\\	ローラースケートが大好きです。	
\\	複数形チェック
\\	[動詞]	ざっと読む	
\\	彼は報告書にざっと目を通した。	
\\	[名詞]	おんどり	
\\	オンドリは日の出に鳴く。	
\\	複数形チェック
\\	[形容詞]	やせこけた	
\\	彼女は痛いほどやせこけていた。	
\\	[動詞]	腐る	
\\	ここのリンゴは腐り初めていた。	
\\	[名詞]	地平線	
\\	複数形チェック
\\	[名詞]	回転	
\\	会社を経営する上で、社員の配置転換を行う必要があります。	
\\	複数形チェック
\\	[名詞]	俗語	
\\	外国語のスラングを知るのは楽しい。	
\\	複数形チェック
\\	[名詞]	傾斜	
\\	つり目のおじさんはいつ来るの?	
\\	複数形チェック
\\	[動詞]	征服する	
\\	しめやかな女はだいたい一番きれいだと思う。にぎやかな人はひどいな。	
\\	[動詞]	殺害する	
\\	[形容詞]	主観の	
\\	[名詞]	寝台車	
\\	年を取ると目ざとくなる。	
\\	複数形チェック
\\	[形容詞]	崇高な	
\\	[名詞]	細⻑い穴・溝	
\\	ベガスに着いて最初に耳に入ってくるのは、チン、チン、チンというスロットマシンの音だ。	
\\	複数形チェック
\\	[名詞]	潜水艦	
\\	潜水艦は沈んで二度と浮上しなかった。	
\\	複数形チェック
\\	[名詞]	ナメクジ	
\\	ナメクジに塩をかけたら死んじゃうの?	
\\	複数形チェック
\\	[名詞]	服従、従順	
\\	「今日すごくいいサイト見つけたんだ。タトエバ・プロジェクトっていうんだけど」「タトエバならもう10年前から知ってるし投稿してるよ」	
\\	複数形チェック
\\	[動詞]	活動を休止する	
\\	[名詞]	定期購読	
\\	を定期購読しています。	
\\	複数形チェック
\\	[形容詞]	ずるい	
\\	彼はとてもずる賢いので私は嫌いだ。	
\\	[形容詞]	その後の	
\\	[名詞]	スモッグ	
\\	複数形チェック
\\	[副詞]	その後	
\\	[名詞]	カタツムリ、巻貝	
\\	ボブはカタツムリを食べて、その後吐いた。	
\\	複数形チェック
\\	[形容詞]	補助の	
\\	[名詞]	罠	
\\	複数形チェック
\\	[名詞]	補助金、交付金	
\\	英国では多くの人が王室への公的支出の廃止を望んでいる。	
\\	複数形チェック
\\	[動詞]	いびきをかく	
\\	トムはいびきをかいている。	
\\	[動詞]	引き算をする	
\\	その数字から3引いて。	
\\	[名詞]	社会主義	
\\	彼は社会主義から転向した。	
\\	複数形チェック
\\	[形容詞]	不機嫌な	
\\	[名詞]	社会学者	
\\	複数形チェック
\\	[動詞]	要約する	
\\	それでは要点をノートにまとめてください。	
\\	[名詞]	独奏者	
\\	複数形チェック
\\	[名詞]	太陽光線	
\\	目蓋には淡い日の光、ゆるやかな色合いが部屋の中を染める。	
\\	複数形チェック
\\	[動詞]	和らげる	
\\	彼らは沈没する船を見捨てた。	
\\	[名詞]	2年生	
\\	複数形チェック
\\	[形容詞]	素晴らしい	
\\	その博物館にはケルト時代の文化遺産のすばらしいコレクションがある。	
\\	[形容詞]	悲しんでいる	
\\	つらいとき、哀しいときは人のために何かしてみましょう。	
\\	[名詞]	超自然のもの	
\\	トムは超常現象を信じている。	
\\	複数形チェック
\\	[名詞]	お土産	
\\	これ、日本のお土産です。気に入っていただけるといいのですが。	
\\	複数形チェック
\\	[名詞]	超大国	
\\	複数形チェック
\\	[形容詞]	宝石をちりばめた	
\\	[名詞]	迷信	
\\	「どこの迷信よ、それ」「庶民の知恵を言ってくれ」	
\\	複数形チェック
\\	[形容詞]	火花を発する	
\\	主人は晩酌が楽しみなので、発泡酒を6缶以上は空けていますし、日本酒の燗が5本も6本も空いていて、たまに休肝日ということで飲まない日もあるのですが2日以上続いたことはありません。	
\\	[動詞]	監督する	
\\	サポートベクターマシンとは、教師あり学習モデルであり、判別分析や回帰分析に用いられる。	
\\	[名詞]	槍やり	
\\	ブリトニー・スピアーズはどこですか。	
\\	複数形チェック
\\	[名詞]	供給者	
\\	複数形チェック
\\	[名詞]	見本、実例	
\\	これは大変珍しい例だ。	
\\	複数形チェック
\\	[形容詞]	支持となる	
\\	[名詞]	⻑くぎ	
\\	シューズ・・スパイクのあるものを禁止しているコースが多いので気をつけましょう。	
\\	複数形チェック
\\	[名詞]	抑圧、鎮圧	
\\	複数形チェック
\\	[動詞]	水を撒き散らす	
\\	トムが浅瀬をぽちゃぽちゃと歩いて行った。	
\\	[動詞]	越える	
\\	山頂からの眺めは、期待をはるかに上回るものだった。	
\\	[名詞]	光彩立派さ	
\\	複数形チェック
\\	[名詞]	余り、過剰	
\\	複数形チェック
\\	[名詞]	スポーク	
\\	トムは話した。	
\\	複数形チェック
\\	[名詞]	停学支払い停止	
\\	東北・秋田新幹線が運転を見合わせるなど交通も混乱した。	
\\	複数形チェック
\\	[名詞]	後援	
\\	この番組はご覧のスポンサーの提供でお送りします。	
\\	複数形チェック
\\	[名詞]	大群	
\\	庭にあった鉢植えをどけたら、小さな虫がうじゃうじゃといて、思わず悲鳴をあげてしまった。	
\\	複数形チェック
\\	[形容詞]	自発的な	
\\	彼らから自発的な申し出がたくさんあった。	
\\	[動詞]	甘くする	
\\	[形容詞]	しみのついた	
\\	最近息子がこそこそしてると思ったら、どうやらお化粧して女装した写真をブログにアップしているらしい。	
\\	[名詞]	恋人かわいいもの	
\\	複数形チェック
\\	[形容詞]	バネ入りの	
\\	[名詞]	隆起	
\\	彼の膝は腫れてぞっとするような黒紫色になってしまった。	
\\	複数形チェック
\\	[名詞]	拍車	
\\	複数形チェック
\\	[動詞]	表す象徴化する	
\\	[動詞]	押しつぶす	
\\	このかぼちゃ、メロンの匂いがする。	
\\	[名詞]	酒宴討論会	
\\	複数形チェック
\\	[動詞]	突き刺す	
\\	やつは僕を裏切ったんだ!	
\\	[名詞]	持ち帰り用の食物	
\\	複数形チェック
\\	[名詞]	足の位置	
\\	論争の多い話題:積極的差別是正措置とバス通学。	
\\	複数形チェック
\\	[形容詞]	話し好きな	
\\	日本人よ、もっとしゃべれ!	
\\	[名詞]	主要素主題	
\\	若い男女が人里離れた洋館で恐怖の一夜を過ごすという、ホラーの定番スタイルだ。	
\\	複数形チェック
\\	[動詞]	日焼けする	
\\	メアリーは日焼けサロンに行った。	
\\	[名詞]	ホチキス	
\\	ホッチキスを貸してくれませんか。	
\\	複数形チェック
\\	[名詞]	タンカー	
\\	そのタンカーには乗組員が少ない。	
\\	複数形チェック
\\	[形容詞]	静止の	
\\	[名詞]	関税	
\\	日本が不公平な関税を撤廃しなければ、米国は制裁措置をとるだろう。	
\\	複数形チェック
\\	[名詞]	文房具	
\\	複数形チェック
\\	[形容詞]	上品な趣きのある	
\\	その食べ物はおいしいでしょう。	
\\	[形容詞]	統計の	
\\	[形容詞]	涙もろい	
\\	[形容詞]	確固とした	
\\	[形容詞]	退屈な	
\\	この手の冗談にはもういい加減、うんざりだ。	
\\	[名詞]	客室係	
\\	「どうぞ」と乗務員が言います。	
\\	複数形チェック
\\	[名詞]	電気通信	
\\	複数形チェック
\\	[動詞]	堅くする	
\\	[名詞]	窓口係	
\\	複数形チェック
\\	[名詞]	栓	
\\	複数形チェック
\\	[形容詞]	温暖な	
\\	[名詞]	倉庫	
\\	複数形チェック
\\	[名詞]	大騒ぎ、騒動	
\\	複数形チェック
\\	[名詞]	店主	
\\	複数形チェック
\\	[副詞]	一時的に、仮に	
\\	このサービスは一時的にご利用いただけません。	
\\	[形容詞]	肥満した	
\\	[形容詞]	領土の	
\\	[形容詞]	まっすぐな	
\\	[動詞]	証言する	
\\	彼女はその男を見たと証言した。	
\\	[名詞]	織り糸	
\\	複数形チェック
\\	[名詞]	織物	
\\	あの店には織物類が置いてありますか。	
\\	複数形チェック
\\	[名詞]	ストライキ参加者	
\\	複数形チェック
\\	[名詞]	織り方,織地	
\\	砂肝のコリコリした食感が大好き。	
\\	複数形チェック
\\	[形容詞]	文体論の	
\\	[動詞]	雪が溶ける	
\\	[形容詞]	理論的な	
\\	アインシュタインにとって理論物理学は遊びのようなものだったが、そんな彼も税金の計算となるとからきし駄目だった。	
\\	[形容詞]	未熟な	
\\	その頃は自転車の流行り出した始めで、下手な素人がそこでも此処でも人を轢いたり、塀を突き破ったりした。	
\\	[副詞]	その後は	
\\	彼はその後、累進して社長にまで進んだ。	
\\	[形容詞]	確信のない	
\\	私たちは、トムがどんな人かも、私たちと同じ家で暮らしたいのかどうかも、よく分からなかった。	
\\	[名詞]	論文	
\\	論文を書き終えましたか。	
\\	複数形チェック
\\	[動詞]	最新のものにする	
\\	アップデートしなきゃよかった。	
\\	[動詞]	厚くする濃くする	
\\	[形容詞]	上り坂の	
\\	この道は上り坂になっている。	
\\	[名詞]	雑木林	
\\	真直ぐにその家から竹林まで伸びた道路は石畳で整備されておる。	
\\	複数形チェック
\\	[動詞] [不規則動詞]	支持する	
\\	この国に言論の自由ってものはないみたいだね。	
\\	[名詞]	大腿	
\\	彼女はももにトカゲの入れ墨がついています。	
\\	複数形チェック
\\	[名詞]	利用、使用量	
\\	英国人なので、彼はとりわけイギリス語法とアメリカ語法の違いに敏感である。	
\\	複数形チェック
\\	[形容詞]	思慮を欠く	
\\	彼は何も考えずに行動するような、そんな浅はかな人ではありません。	
\\	[名詞]	家庭用品、器具	
\\	複数形チェック
\\	[名詞]	敷居入口発端	
\\	トムはメアリーを抱いて敷居をまたいだ。	
\\	複数形チェック
\\	[名詞]	発言	
\\	複数形チェック
\\	[動詞] [不規則動詞]	成功する、繁栄する	
\\	覚えておこう、もしこの金融危機が我々に何かを教えたのならば、それはメインストリートが苦しんでると同時にウォール街に繁栄はないということを。	
\\	[名詞]	空室、空き	
\\	申し訳ありません、満室です。	
\\	複数形チェック
\\	[名詞]	激しい雷雨	
\\	花火大会は雷雨のため開始30分で中止になった。	
\\	複数形チェック
\\	[副詞]	ぼんやりと	
\\	[形容詞]	潮の	
\\	[名詞]	弁	
\\	元栓、しまってる。	
\\	複数形チェック
\\	[名詞]	タイルかわら	
\\	屋根から落ちてきたタイルは粉々に砕けた。	
\\	複数形チェック
\\	[名詞]	蒸気	
\\	雲は蒸気の固まりである。	
\\	複数形チェック
\\	[名詞]	木材	
\\	カナダは木材に富む。	
\\	複数形チェック
\\	[形容詞]	変わりやすい	
\\	[名詞]	タイマー	
\\	内部のタイマー用ボタン電池が切れた。	
\\	複数形チェック
\\	[名詞]	静脈	
\\	ユーモアの無い一日は、極めて寂しい一日である。	
\\	複数形チェック
\\	[名詞]	タイミング	
\\	いいタイミングだ。	
\\	複数形チェック
\\	[形容詞]	言葉の	
\\	[形容詞]	疲れを知らない	
\\	[形容詞]	垂直の	
\\	[形容詞]	退屈な	
\\	[動詞]	震える	
\\	[動詞]	骨折って働く	
\\	労苦と心配で彼の健康はそこなわれることになった。	
\\	[名詞]	勝利者勝者	
\\	複数形チェック
\\	[形容詞]	我慢できる	
\\	[副詞]	実質的には	
\\	ほとんどのアメリカ人がその決定に賛成した。	
\\	[名詞]	寛容、度量	
\\	言語を学ぶには曖昧さに対する寛容性が必要である。	
\\	複数形チェック
\\	[形容詞]	有徳の貞淑な	
\\	[形容詞]	寛大な	
\\	[名詞]	天職、使命	
\\	複数形チェック
\\	[名詞]	通行料	
\\	ミサイル攻撃で多くの人命が失われた。	
\\	複数形チェック
\\	[形容詞]	下品な低俗な	
\\	[名詞]	商人	
\\	複数形チェック
\\	[名詞]	壁紙	
\\	複数形チェック
\\	[名詞]	軍事訓練をうける人	
\\	複数形チェック
\\	[名詞]	放浪者	
\\	複数形チェック
\\	[名詞]	裏切者	
\\	彼は裏切り者になった。	
\\	複数形チェック
\\	[名詞]	瀬戶物陶器	
\\	青銅器は主として銅と錫の合金で出来ている。	
\\	複数形チェック
\\	[名詞]	路面電車	
\\	複数形チェック
\\	[名詞]	交戦状態戦争	
\\	複数形チェック
\\	[動詞]	どしんどしんと歩く	
\\	野犬がうちの庭を踏み荒した。	
\\	[形容詞]	敵対する	
\\	中国でも戦国時代の墳墓からガラスが出土している。	
\\	[動詞]	踏み付ける	
\\	[名詞]	洗面用タオル	
\\	複数形チェック
\\	[形容詞]	落ち着いた、静かな	
\\	[名詞]	散水	
\\	涙が出てきますよ。	
\\	複数形チェック
\\	[形容詞]	旅行する	
\\	私は旅行が好きです。	
\\	[動詞]	揺れる	
\\	[名詞]	財務部⻑	
\\	複数形チェック
\\	[形容詞]	疲れた	
\\	[名詞]	賛辞ささげ物	
\\	複数形チェック
\\	[形容詞]	変な、奇妙な	
\\	変なのー。	
\\	[形容詞]	手の込んだ	
\\	松葉杖ってけっこうムズいな・・・。「すみません、ストレッチャー通ります!」	
\\	[形容詞]	⻄に向かう	
\\	[名詞]	つまらないもの	
\\	完璧というのはちょっと退屈なものなのである。	
\\	複数形チェック
\\	[名詞]	車いす	
\\	車椅子の女性、スーパーに立てこもる。	
\\	複数形チェック
\\	[名詞]	管財人	
\\	理事会は海外の持ち株を放棄することを議決しました。	
\\	複数形チェック
\\	[名詞]	卸し売り	
\\	今朝、山羽さんが卸で胡桃を30キロ買いました。	
\\	複数形チェック
\\	[動詞]	転ぶ	
\\	ひょんなことからほんとうのことがわかったんだ、ほとんど手を引きかけたときに。	
\\	[形容詞]	健康そうな、健全な	
\\	すべて国民は、健康で文化的な最低限度の生活を営む権利を有する。	
\\	[動詞]	ひっくり返る	
\\	彼は寝返りをうった。	
\\	[名詞]	柳	
\\	複数形チェック
\\	[名詞]	小枝	
\\	彼は小枝を束にした。	
\\	複数形チェック
\\	[名詞]	冬季	
\\	複数形チェック
\\	[名詞]	圧制、横暴	
\\	地球側の圧政に苦しむコロニーの人々の中で、テロによる現状打破を目論む勢力が誕生。	
\\	複数形チェック
\\	[動詞]	枯れる、萎える	
\\	運よ!月のようにひたひたと変わって、大きくなって、そしてまた枯れる。	
\\	[名詞]	専制君主	
\\	民衆は残酷な暴君によって虐げられていた。	
\\	複数形チェック
\\	[動詞] [不規則動詞]	耐える	
\\	[形容詞]	満場一致の	
\\	[形容詞]	機知に富む	
\\	[形容詞]	不変の	
\\	当期純利益は50億円の見込みで、前期とほとんど変わりません。	
\\	[名詞]	花輪	
\\	複数形チェック
\\	[形容詞]	理解できる	
\\	良い医者は患者に説明する時、人体模型を使うなどして、わかりやすく説明してくれる。	
\\	[名詞]	難破	
\\	僕は彼女とかかわったために身も心もずたぼろにされてしまった。	
\\	複数形チェック
\\	[名詞]	約束	
\\	複数形チェック
\\	[名詞]	レスリング	
\\	複数形チェック
\\	[形容詞]	好ましくない	
\\	[形容詞]	惨めな	
\\	[形容詞]	不健康な	
\\	一日にオレンジを二十個以上食べるのは不健康だと思う。	
\\	[動詞] [不規則動詞]	絞る	
\\	トムを絞め殺したいという衝動を抑えるのは難しかった。	
\\	[名詞]	均一性	
\\	複数形チェック
\\	[名詞]	糸	
\\	複数形チェック
\\	[形容詞]	非公式な	
\\	[名詞]	熱意	
\\	私たちは勿論サポートしますが最後に決めるのは貴方の熱意だということを覚えていてください。	
\\	複数形チェック
\\	[形容詞]	人気のない	
\\	彼はどういうわけか人気がない。	
\\	[形容詞]	熱心な	
\\	[形容詞]	予知できない	
\\	[動詞]	ファスナーを締める	
\\	彼はかばんのジッパーを開けた。	
\\	[名詞]	異常	
\\	複数形チェック
\\	[形容詞]	感謝している	
\\	[形容詞]	原住⺠の	
\\	[形容詞]	任意の	
\\	[動詞]	たくさんある	
\\	ローマは至る所に遺跡がある。	
\\	[名詞]	弧、弓形	
\\	複数形チェック
\\	[形容詞]	心を奪うような	
\\	素敵に面白い冒険物語。	
\\	[形容詞]	古風の	
\\	[名詞]	吸収、併合、夢中	
\\	複数形チェック
\\	[動詞] [不規則動詞]	生じる、起こる	
\\	致命的な誤りは不注意から起こる。	
\\	[動詞]	加速する	
\\	彼はスピードを上げた。	
\\	[名詞]	貴族階級	
\\	複数形チェック
\\	[名詞]	蓄積	
\\	複数形チェック
\\	[名詞]	貴族	
\\	複数形チェック
\\	[名詞]	承認	
\\	複数形チェック
\\	[形容詞]	貴族の	
\\	[副詞]	鋭く	
\\	算数教育が非常に大事だと痛切に感じていた。	
\\	[名詞]	配列	
\\	複数形チェック
\\	[形容詞]	順応性のある	
\\	[名詞]	ごう慢さ	
\\	彼の傲慢が凋落に繋がった。	
\\	複数形チェック
\\	[名詞]	アダプター	
\\	複数形チェック
\\	[名詞]	職人、熟練工	
\\	複数形チェック
\\	[名詞]	常用者	
\\	彼はヘロインの常習者だ。	
\\	複数形チェック
\\	[動詞]	確かめる	
\\	[名詞]	熱狂的傾倒	
\\	複数形チェック
\\	[動詞]	帰する	
\\	[動詞]	付着する	
\\	彼は原案に固執した。	
\\	[名詞]	大志、志望	
\\	彼は非常に悲観的で希望を持っていない。	
\\	複数形チェック
\\	[形容詞]	調節できる	
\\	[動詞]	熱望する	
\\	彼女は女優になることを志した。	
\\	[名詞]	思春期	
\\	自分の思春期を振り返ると、いつも憂うつな気分になる。	
\\	複数形チェック
\\	[動詞]	暗殺する	
\\	リンカーンは1865年に暗殺された。	
\\	[形容詞]	⻘年期の	
\\	「思春期の少年にとって、鼻をほじくることは普通の行動である」という医学的発見を突き止めた。	
\\	[名詞]	暴行	
\\	彼らは山頂にアタックしました。	
\\	複数形チェック
\\	[形容詞]	不利な	
\\	[形容詞]	選別した	
\\	[形容詞]	賢明で	
\\	列車で行くのが賢明だよ。	
\\	[形容詞]	生意気な	
\\	[名詞]	主唱者、支持者	
\\	彼は規則の改正を唱えている。	
\\	複数形チェック
\\	[名詞]	達成	
\\	複数形チェック
\\	[形容詞]	空中の	
\\	[名詞]	屋根裏	
\\	トムは屋根裏部屋にいる。	
\\	複数形チェック
\\	[名詞]	航空宇宙	
\\	複数形チェック
\\	[名詞]	監査	
\\	この財務調査は会社の資産を評価することも含みます。	
\\	複数形チェック
\\	[形容詞]	美学の	
\\	麻里子はエステティック・サロンに行きました。	
\\	[形容詞]	本物の	
\\	フランスには本物の中華料理店はありますか?	
\\	[副詞]	親しみをこめて	
\\	[形容詞]	権威のある	
\\	[形容詞]	肯定的な	
\\	[名詞]	自治	
\\	複数形チェック
\\	[動詞]	苦しめる	
\\	[名詞]	航空	
\\	なぜ飛行機にはパラシュートではなくライフジャケットがあるのか。	
\\	複数形チェック
\\	[名詞]	老化、時間効果、熟成	
\\	老化現象の一種です。	
\\	複数形チェック
\\	[名詞]	アボカド	
\\	複数形チェック
\\	[名詞]	動揺	
\\	複数形チェック
\\	[動詞]	片言を言う	
\\	[動詞]	苦しみ悶える	
\\	[名詞]	バッジ	
\\	娘のバッグにワッペンを縫いつけてあげた。	
\\	複数形チェック
\\	[名詞]	申し立て、陳述	
\\	複数形チェック
\\	[名詞]	保釈	
\\	複数形チェック
\\	[形容詞]	申し立てられた	
\\	[形容詞]	釣合いのとれた	
\\	彼は丸太の上に上手く乗っていた。	
\\	[副詞]	申し立てによると	
\\	[名詞]	バラード	
\\	複数形チェック
\\	[動詞]	割り当てる	
\\	経費として二千二百ドルを計上しております。	
\\	[名詞]	男爵、豪商	
\\	複数形チェック
\\	[名詞]	アルファ	
\\	アルファ、ベータ、ガンマはギリシャ文字です。	
\\	複数形チェック
\\	[形容詞]	不毛の	
\\	[副詞]	どちらか一方で	
\\	[名詞]	低音	
\\	彼女は、あのベースが好き。	
\\	複数形チェック
\\	[名詞]	曖昧さ	
\\	複数形チェック
\\	[名詞]	ひと焼き分	
\\	複数形チェック
\\	[形容詞]	紛らわしい、曖昧な	
\\	何てあいまいな表現だ。	
\\	[名詞]	バッター	
\\	彼は目下首位打者だ。	
\\	複数形チェック
\\	[前置詞]	
\\	のまん中に	
\\	[名詞]	運搬人、持参人	
\\	本状の持参人は友人の鈴木君です。	
\\	複数形チェック
\\	[形容詞]	面白がっている	
\\	楽しんでるよ。	
\\	[名詞]	態度	
\\	忍耐が耐えられないこともある。	
\\	複数形チェック
\\	[形容詞]	アナログの	
\\	[動詞]	当惑させる	
\\	彼は戸惑っていた。	
\\	[名詞]	類比、たとえ	
\\	複数形チェック
\\	[名詞]	当惑	
\\	複数形チェック
\\	[形容詞]	分析的な	
\\	[形容詞]	偏見のある	
\\	私は先入観に縛られている。	
\\	[形容詞]	分析的な	
\\	[名詞]	伝記作家	
\\	複数形チェック
\\	[名詞]	解剖学	
\\	複数形チェック
\\	[形容詞]	伝記の	
\\	[形容詞]	先祖伝来の	
\\	[名詞]	出生率	
\\	最近、出生率は低下し続けている。	
\\	複数形チェック
\\	[名詞]	苦痛	
\\	複数形チェック
\\	[名詞]	司教、主教	
\\	同性愛者の司祭が主教に昇格させられたと聞いたが、結局捏造したものであった。	
\\	複数形チェック
\\	[形容詞]	困って、悩んで	
\\	彼女は私にとても腹を立てている。	
\\	[動詞]	恐喝する	
\\	彼は私を脅迫した。	
\\	[形容詞]	反対する	
\\	僕はバイク盗難防止装置を買いました。	
\\	[名詞]	鍛冶屋	
\\	鍛冶屋になるのは,鉄を鍛えながらだ。	
\\	複数形チェック
\\	[名詞]	抗体	
\\	複数形チェック
\\	[形容詞]	寒い	
\\	[名詞]	衣装	
\\	複数形チェック
\\	[動詞]	ぼんやりさせる	
\\	遠くの物がかすんで見えます。	
\\	[名詞]	国境線	
\\	そろそろこんな犯罪まがいの仕事から足をあらったらどうなんだ。	
\\	複数形チェック
\\	[名詞]	操縦室	
\\	複数形チェック
\\	[名詞]	植物学	
\\	複数形チェック
\\	[名詞]	男女共学	
\\	複数形チェック
\\	[動詞]	踏ん張る	
\\	娘は歯を矯正中です。	
\\	[形容詞]	明瞭な、筋の通った	
\\	[名詞]	違反、不履行	
\\	某連盟元会長が背任の容疑で逮捕されました。	
\\	複数形チェック
\\	[名詞]	同時発生	
\\	私はそれは単なる偶然だと思う。	
\\	複数形チェック
\\	[名詞]	躍進	
\\	ほとんどの科学的進歩は当然のことについての発見に過ぎない。	
\\	複数形チェック
\\	[動詞]	衝突する	
\\	彼の乗った自動車が電車に衝突した。	
\\	[名詞]	分解、分裂	
\\	複数形チェック
\\	[動詞]	植⺠地化する	
\\	あなたはいつか人類が月に植民すると思いますか。	
\\	[名詞]	醸造	
\\	複数形チェック
\\	[名詞]	コンマ	
\\	カンマを入れないようにしたいんだけど、6はカンマがないと不自然な気がします。	
\\	複数形チェック
\\	[名詞]	収賄	
\\	彼は贈収賄を軽蔑した。	
\\	複数形チェック
\\	[名詞]	論評、批評	
\\	複数形チェック
\\	[名詞]	瀬戶際	
\\	1939年には、1914年と同様、世界は戦争の危機に瀕していた。	
\\	複数形チェック
\\	[名詞]	解説者	
\\	複数形チェック
\\	[名詞]	株式仲買人	
\\	利益追求に必死のブローカーは、必ずしもルールブックにのっとっているとは限らないのです。	
\\	複数形チェック
\\	[名詞]	信仰などを共にすること	
\\	複数形チェック
\\	[動詞]	卵を抱く	
\\	[動詞]	通勤する	
\\	加奈子ちゃんは千葉から東京に通います。	
\\	[名詞]	ブラウニー	
\\	複数形チェック
\\	[名詞]	思いやり	
\\	あらゆる道徳の基礎は同情である。	
\\	複数形チェック
\\	[動詞]	バックルで締める	
\\	もっとしっかり仕事をしてよ。今回は物見遊山の旅じゃないんだから。	
\\	[形容詞]	互換性のある	
\\	[名詞]	親友、仲間	
\\	トムは私の友人です。	
\\	複数形チェック
\\	[形容詞]	強制的な	
\\	[名詞]	野牛	
\\	コロンブスがアメリカ大陸を発見したとき、バイソン(アメリカ野牛)は、広大な地域に棲んでいた。	
\\	複数形チェック
\\	[動詞]	編集する	
\\	辞書を編さんするには膨大な時間を要する。	
\\	[形容詞]	分厚い、太った	
\\	[動詞]	従う、適合する	
\\	[動詞]	いじめる	
\\	弱い者いじめはやめろ。	
\\	[形容詞]	合成の	
\\	その、ベッドの側に、合成アルミニュームのロボットが人体と
\\	肌と、同じように巧妙に塗料を施されたゴムを密着して、裸体のまま突立っていた。	
\\	[名詞]	浮浪者、ぐうたら者	
\\	複数形チェック
\\	[動詞]	圧縮する	
\\	2時間毎に局部を湿布しなさい。	
\\	[名詞]	バンパー	
\\	「これを見て」「え?」「バンパーの右側に衝突痕があるわ」	
\\	複数形チェック
\\	[動詞]	計算する	
\\	[名詞]	バンガロー	
\\	複数形チェック
\\	[動詞]	コンピュータで処理する	
\\	[名詞]	強盗	
\\	複数形チェック
\\	[動詞]	だます	
\\	この映像はコミコン2016に参加したファンのために 当映画の製作者により作られたものです	
\\	[名詞]	ボタン穴	
\\	彼はボタンの穴に花を挿した。	
\\	複数形チェック
\\	[名詞]	うぬぼれ	
\\	複数形チェック
\\	[名詞]	強い衝動、欲求	
\\	どちら様でしょうか。	
\\	複数形チェック
\\	[形容詞]	想像しうる	
\\	[名詞]	取り消し	
\\	無断キャンセルにはほとほと困っています。	
\\	複数形チェック
\\	[形容詞]	決定的な	
\\	今日、その努力が報われ彼らが決定的な証拠を掴んだ事は賞賛に値します。	
\\	[名詞]	監禁	
\\	彼は捕虜になった。	
\\	複数形チェック
\\	[名詞]	分譲マンション	
\\	彼がマンションを買ったのはごぞんじですか。	
\\	複数形チェック
\\	[名詞]	カーニバル	
\\	リオのカーニバルは二月に開催される。	
\\	複数形チェック
\\	[名詞]	監禁	
\\	複数形チェック
\\	[名詞]	カジノ	
\\	複数形チェック
\\	[形容詞]	確認された	
\\	その知らせで私の疑いはいっそう強くなった。	
\\	[名詞]	負傷者、死傷者	
\\	複数形チェック
\\	[形容詞]	矛盾する	
\\	[形容詞]	絶対的な、断言的な	
\\	[名詞]	集合	
\\	複数形チェック
\\	[名詞]	名声、有名人	
\\	彼はちょっとした有名人です。	
\\	複数形チェック
\\	[形容詞]	議会の	
\\	[名詞]	摂氏	
\\	水は摂氏0度で凍る。	
\\	複数形チェック
\\	[形容詞]	良心的な、注意深くて	
\\	[名詞]	炭、木炭	
\\	私は、新居へ移ると共に、私の部屋へ引っ込んで、自分で炊事できるだけの道具ーー土釜、土鍋、七輪の類をととのえた。	
\\	複数形チェック
\\	[動詞]	保存する	
\\	節水にご協力ください。	
\\	[形容詞]	慈悲深い	
\\	[形容詞]	思いやりのある	
\\	蓮くんは察しは良い様だけど、もう少し心の機微というものを理解出来る様に努力するべきじゃないかしら。	
\\	[名詞]	一覧表	
\\	複数形チェック
\\	[名詞]	一貫性	
\\	この志望理由書は、論旨の展開に一貫性が無く、散漫な印象です。	
\\	複数形チェック
\\	[名詞]	育児、児童保護	
\\	複数形チェック
\\	[動詞]	統合する	
\\	[名詞]	チューチューという鳴き声	
\\	こんな鳴き声の鳥、オーストラリアにもいる?	
\\	複数形チェック
\\	[名詞]	合併	
\\	複数形チェック
\\	[名詞]	楽器の弦	
\\	複数形チェック
\\	[形容詞]	目立つ	
\\	スーツを着て彼は目立っていた。	
\\	[名詞]	明瞭、清澄さ	
\\	複数形チェック
\\	[動詞]	共謀する	
\\	[動詞]	握りしめる	
\\	少年は小犬を胸に抱き締めた。	
\\	[名詞]	強制、制約	
\\	複数形チェック
\\	[名詞]	爪	
\\	その猫は爪を隠している。	
\\	複数形チェック
\\	[形容詞]	伝染性の	
\\	風邪は伝染する。	
\\	[動詞]	清潔にする	
\\	それ歯磨き粉じゃなくて洗顔フォームだよ!	
\\	[動詞]	汚染する	
\\	東電は高濃度の放射性物質を含む汚染水の流出を防ぐことができない。	
\\	[名詞]	洗濯人、磨き粉	
\\	複数形チェック
\\	[形容詞]	満足した	
\\	満足そうですね。	
\\	[名詞]	清掃	
\\	部屋を片付けなさい。	
\\	複数形チェック
\\	[名詞]	寄付、貢献者	
\\	複数形チェック
\\	[動詞]	食いしばる	
\\	[名詞]	料理法	
\\	複数形チェック
\\	[形容詞]	聖職者の	
\\	[名詞]	対等関係	
\\	複数形チェック
\\	[副詞]	巧みに	
\\	その、ベッドの側に、合成アルミニュームのロボットが人体と
\\	肌と、同じように巧妙に塗料を施されたゴムを密着して、裸体のまま突立っていた。	
\\	[名詞]	相互関係	
\\	複数形チェック
\\	[名詞]	クローン、複製品	
\\	複数形チェック
\\	[形容詞]	対応する	
\\	あなたと文通することを楽しみにしています。	
\\	[名詞]	終結、閉鎖	
\\	複数形チェック
\\	[形容詞]	全世界的な	
\\	[形容詞]	雲のない	
\\	[名詞]	表情、顔つき	
\\	我々は絶対にテロを許さない。	
\\	複数形チェック
\\	[名詞]	連合	
\\	同盟軍はバグダッドの検問所で彼女の車を襲撃した。	
\\	複数形チェック
\\	[名詞]	大当たり、大成功	
\\	クーデター計画はぎりぎりのところで阻止されました。	
\\	複数形チェック
\\	[名詞]	中庭	
\\	複数形チェック
\\	[名詞]	障害	
\\	わあ、ジャクソン!学習障害があるよね!	
\\	複数形チェック
\\	[動詞]	詰め込む	
\\	塾の先生と付き合っている。	
\\	[形容詞]	障害を持った	
\\	身体障害者を助けるのが本当に楽しいのです。	
\\	[名詞]	痙攣	
\\	泳いでいたら、足がつった。	
\\	複数形チェック
\\	[動詞]	武装解除をする	
\\	[名詞]	噴火口	
\\	複数形チェック
\\	[動詞]	識別する	
\\	しかしながら、未知の外国語を聞くとき、我々は異なる単語を聞き分けることができず、発言をほとんど休止のない音の連続であると認識します。	
\\	[動詞]	発狂させる	
\\	[形容詞]	訓練上の、懲戒の	
\\	[形容詞]	信用できる	
\\	[形容詞]	訓練された	
\\	彼は躾がなってない。	
\\	[名詞]	信条	
\\	そして今夜、彼女がアメリカで100年にわたって見てきた全てのものについて考えています。悲嘆や希望、もがきや前進、出来やしないと言われた回数、そしてあのアメリカ的信条を掲げて前進した人々、そう我々はできる、という。	
\\	複数形チェック
\\	[名詞]	不満	
\\	不平不満を治す薬はない。	
\\	複数形チェック
\\	[名詞]	批評、評論	
\\	複数形チェック
\\	[動詞]	中止する	
\\	この電車は車両の不具合のため当駅で打ち切りとさせていただきます。	
\\	[形容詞]	曲がった	
\\	私の鼻は、とても大きく、しかも曲がっている。	
\\	[名詞]	不信	
\\	複数形チェック
\\	[副詞]	残酷に	
\\	[名詞]	思慮深さ、慎重さ	
\\	複数形チェック
\\	[名詞]	クルーザー	
\\	複数形チェック
\\	[動詞]	差別する、識別する	
\\	[名詞]	改革運動	
\\	複数形チェック
\\	[動詞]	軽蔑する	
\\	彼は贈収賄を軽蔑した。	
\\	[形容詞]	圧倒的な	
\\	[動詞]	幻滅を感じさせる	
\\	君には幻滅した。	
\\	[形容詞]	立体的な	
\\	一立方メートルは千リットルにあたる。	
\\	[形容詞]	公平無私な	
\\	[形容詞]	のろわれた	
\\	[名詞]	不服従、違反	
\\	複数形チェック
\\	[形容詞]	曲がった	
\\	[動詞]	従わない	
\\	[名詞]	カスタード	
\\	おやつにプリンを食べた。	
\\	複数形チェック
\\	[動詞]	分配する	
\\	機械は多くの人手を省く。	
\\	[名詞]	保護	
\\	複数形チェック
\\	[名詞]	転置、変位	
\\	複数形チェック
\\	[形容詞]	習慣的な	
\\	[名詞]	不機嫌	
\\	複数形チェック
\\	[名詞]	短剣	
\\	複数形チェック
\\	[形容詞]	処分できる	
\\	和食のレストランへ行くたびに、割り箸を家へもって帰ります。	
\\	[名詞]	タンポポ	
\\	妊娠中、たんぽぽコーヒを飲んだ方がいいですか?	
\\	複数形チェック
\\	[名詞]	無関心	
\\	複数形チェック
\\	[名詞]	臨終	
\\	複数形チェック
\\	[動詞]	混乱させる	
\\	サイトの空気を悪くする書き込みを控えてください。	
\\	[名詞]	礼儀正しさ	
\\	複数形チェック
\\	[名詞]	中断、妨害	
\\	複数形チェック
\\	[名詞]	小数	
\\	複数形チェック
\\	[名詞]	解散、解消	
\\	哀れみのない正義は冷酷である。しかし、正義のない哀れみは解体の母である。	
\\	複数形チェック
\\	[形容詞]	装飾の	
\\	[名詞]	注意散漫	
\\	気晴らしと言えば碁を打つことぐらいだ。	
\\	複数形チェック
\\	[名詞]	熱心、献身	
\\	仕事場の管理者は、正確さ、効率性、献身を期待する。	
\\	複数形チェック
\\	[名詞]	わきへそらすこと	
\\	複数形チェック
\\	[形容詞]	不完全な	
\\	描画が不良です。	
\\	[形容詞]	分割された	
\\	ファランプさんはケーキを二つに分けた。	
\\	[名詞]	下落、堕落	
\\	人類は生態系の退化を防ぐことができるだろうか?	
\\	複数形チェック
\\	[名詞]	神性	
\\	複数形チェック
\\	[動詞]	削除する	
\\	このファイルは削除して下さい。	
\\	[名詞]	書類	
\\	のセマンティクスと表記法の詳細については、UMLドキュメントを参照してください。	
\\	複数形チェック
\\	[名詞]	優雅、上品さ	
\\	メアリーは思いやりに欠けている。	
\\	複数形チェック
\\	[動詞]	身をかわす	
\\	クリスは素早く身をかわした!	
\\	[名詞]	三角州	
\\	複数形チェック
\\	[名詞]	優越、支配	
\\	複数形チェック
\\	[名詞]	実演、試用	
\\	デモの様子は参加者によって
\\	を通じて実況された。	
\\	複数形チェック
\\	[名詞]	統治、支配	
\\	彼女のフェチが緊縛と支配でしょう。	
\\	複数形チェック
\\	[名詞]	悪魔	
\\	魔王はついに気を失ってどうと倒れた。	
\\	複数形チェック
\\	[動詞]	着用する	
\\	サム、止めて!	
\\	[名詞]	防臭剤	
\\	デオドラント使ったほうがいいよ。	
\\	複数形チェック
\\	[名詞]	提供者、寄贈者	
\\	あなたに献血は無理。	
\\	複数形チェック
\\	[動詞]	描く、描写する	
\\	[名詞]	こね粉、パン生地	
\\	やつは金がうなってる。	
\\	複数形チェック
\\	[名詞]	剥奪、喪失	
\\	睡眠不足で、心筋梗塞リスクが上昇。	
\\	複数形チェック
\\	[名詞]	下り坂、滑降	
\\	複数形チェック
\\	[形容詞]	描写して	
\\	[名詞]	劇作家	
\\	彼は劇作家である。	
\\	複数形チェック
\\	[形容詞]	当然の	
\\	トムのものは全て彼が手に入れて当然のものだった。	
\\	[動詞]	劇的に表現する	
\\	[動詞]	指名する、指定する	
\\	トムは僕らの指名運転手です。	
\\	[名詞]	欠点、障害	
\\	優秀な人にはひとつ大きな欠点がある。負け方を知らないことだ。	
\\	複数形チェック
\\	[名詞]	設計	
\\	妹の趣味はウェブ用のアイコンのデザインで、暇を見つけて作っては自分のサイトで無料で配布している。	
\\	複数形チェック
\\	[名詞]	デスクトップ	
\\	デスクトップがグチャグチャになってきた。	
\\	複数形チェック
\\	[副詞]	正しく、適当に	
\\	[名詞]	自暴自棄	
\\	後数秒で私は絶望の分岐点を越えるのだ。	
\\	複数形チェック
\\	[形容詞]	偽の	
\\	[形容詞]	分離した	
\\	[名詞]	投げ捨て	
\\	ここにゴミ捨てるべからず。	
\\	複数形チェック
\\	[名詞]	分離、脱離	
\\	複数形チェック
\\	[形容詞]	永続性のある	
\\	これは耐久性に欠けているようです。	
\\	[動詞]	荒廃させる	
\\	壊滅的な被害を防ぐために世界的な警戒を強化していく重要性を確認した。	
\\	[名詞]	居住者	
\\	複数形チェック
\\	[名詞]	さいころ	
\\	サイを振れ。	
\\	複数形チェック
\\	[名詞]	力学、動力学	
\\	ボストンダイナミクスは走ることさえ出来るロボットを造りましたよ。	
\\	複数形チェック
\\	[名詞]	独裁者	
\\	彼らは独裁者と戦った。	
\\	複数形チェック
\\	[名詞]	爆弾	
\\	その岩はダイナマイトで動かした。	
\\	複数形チェック
\\	[動詞]	威厳を付ける	
\\	[名詞]	風変わり、突飛さ	
\\	君は彼の奇行を気にしないほうがよい。	
\\	複数形チェック
\\	[名詞]	板ばさみ、ジレンマ	
\\	複数形チェック
\\	[形容詞]	生態学的な	
\\	[名詞]	人名録	
\\	私は電話帳で彼の電話番号を調べた。	
\\	複数形チェック
\\	[名詞]	生態学者	
\\	複数形チェック
\\	[名詞]	有頂天	
\\	エクスタシーはアッパー系のドラッグだ。	
\\	複数形チェック
\\	[名詞]	目撃者	
\\	複数形チェック
\\	[名詞]	ウナギ	
\\	私のホバークラフトは鰻でいっぱいです。	
\\	複数形チェック
\\	[動詞]	促進する	
\\	[形容詞]	選挙の	
\\	[名詞]	派閥	
\\	複数形チェック
\\	[名詞]	優雅	
\\	彼女は飾り気のない格好をしてはいる。しかし育ちから来るエレガンスは隠せないものだ。	
\\	複数形チェック
\\	[名詞]	一時的な流行	
\\	複数形チェック
\\	[名詞]	除去	
\\	「なんで私?」「消去法で仕方がなかったから。」	
\\	複数形チェック
\\	[形容詞]	徐々に色あせる	
\\	誰かケガした?	
\\	[名詞]	雄弁	
\\	複数形チェック
\\	[副詞]	かすかに	
\\	彼女の頬はウッスラ赤くなっていた。	
\\	[動詞]	刺繍する	
\\	彼女は白いハンカチに自分のイニシャルを刺繍した。	
\\	[動詞]	口ごもる	
\\	[動詞]	移住する	
\\	[形容詞]	空想上の	
\\	[名詞]	移⺠	
\\	複数形チェック
\\	[名詞]	農家の庭	
\\	複数形チェック
\\	[動詞]	放射する	
\\	アイスランドには大量の煙を吐き出している火山があります。	
\\	[名詞]	ファシズム	
\\	ナショナリズムをファシズムと同一視してはいけない。	
\\	複数形チェック
\\	[形容詞]	強固な	
\\	[名詞]	死、死者	
\\	複数形チェック
\\	[副詞]	強調して	
\\	[形容詞]	好意をもたれている	
\\	[形容詞]	心強い	
\\	[名詞]	友情	
\\	複数形チェック
\\	[名詞]	百科事典	
\\	この百科事典は検索に便利である。	
\\	複数形チェック
\\	[名詞]	男女同権主義	
\\	複数形チェック
\\	[動詞]	危険にさらす	
\\	[名詞]	繁殖(受精)力	
\\	複数形チェック
\\	[動詞]	承認する、裏書きをする	
\\	俺が保証するよ。	
\\	[動詞]	受精させる	
\\	[動詞]	授ける	
\\	彼は多くの才能を与えられている。	
\\	[名詞]	肥料	
\\	複数形チェック
\\	[名詞]	寄付	
\\	複数形チェック
\\	[形容詞]	祭りの	
\\	[形容詞]	永続する	
\\	アメリカの光がまだ輝きとして燃えているどうかを疑う人へ、今夜、我々の国の真の強さは軍隊の力や富の規模からではなく、民主主義、自由、機会、不屈の希望といった我々の理想への揺るぎない力から来るともう一度証明した。	
\\	[名詞]	祭りの行事	
\\	複数形チェック
\\	[名詞]	悟り、啓発	
\\	複数形チェック
\\	[名詞]	ヒレ肉、切り身	
\\	複数形チェック
\\	[動詞]	入隊する	
\\	[名詞]	詰め物	
\\	トムはエントリーシートを書いている。	
\\	複数形チェック
\\	[動詞]	登録する、入学させる	
\\	[名詞]	映画製作者	
\\	複数形チェック
\\	[動詞]	もつれさせる	
\\	[形容詞]	汚い	
\\	[形容詞]	企業心に富んだ	
\\	[名詞]	指の爪	
\\	グリーンピースは鉄分が豊富で爪の色がよくなる成分が含まれている。	
\\	複数形チェック
\\	[名詞]	実在、存在	
\\	複数形チェック
\\	[名詞]	企業家	
\\	シルヴィオ・ベルルスコーニはイタリアの政治家、現在のイタリア首相、また成功した実業家。	
\\	複数形チェック
\\	[形容詞]	限りのある	
\\	[形容詞]	伝染性の	
\\	[名詞]	漁業	
\\	複数形チェック
\\	[名詞]	均衡	
\\	複数形チェック
\\	[名詞]	金属の薄片	
\\	複数形チェック
\\	[名詞]	公平、公正	
\\	複数形チェック
\\	[名詞]	舞台	
\\	複数形チェック
\\	[名詞]	噴火	
\\	ヴェスヴィオ火山の噴火は、ポンペイの街を壊滅させた。	
\\	複数形チェック
\\	[名詞]	最前部、まっ先	
\\	複数形チェック
\\	[動詞]	段階的に拡大する	
\\	だんだんエスカレートしてきた。	
\\	[名詞]	林業	
\\	複数形チェック
\\	[形容詞]	おおよその、概算の	
\\	英語には50万語以上の単語があると見積もられた。	
\\	[名詞]	公式化	
\\	複数形チェック
\\	[名詞]	見積もり	
\\	所有する喜びは、私の思うところでは、手に入れる為の苦労に対して割に合わない。	
\\	複数形チェック
\\	[形容詞]	壊れやすい	
\\	これらのダンボールは壊れ易い。	
\\	[名詞]	永遠	
\\	永遠はあるよ。ここにあるよ・・・。	
\\	複数形チェック
\\	[名詞]	香り	
\\	この花は強い香りを放つ。	
\\	複数形チェック
\\	[名詞]	礼儀作法	
\\	複数形チェック
\\	[形容詞]	もろい	
\\	[動詞]	高める	
\\	[形容詞]	死に物狂いの	
\\	目が回るほど忙しいよ。	
\\	[副詞]	例外的に	
\\	僕は殊の外不運だった。	
\\	[名詞]	自由契約者	
\\	複数形チェック
\\	[形容詞]	徹底的な	
\\	[名詞]	好意、友情	
\\	複数形チェック
\\	[形容詞]	異国(風)の	
\\	木工芸では、木を、硬木、軟木と、唐木に分類します。	
\\	[形容詞]	イライラする	
\\	[名詞]	広がり	
\\	複数形チェック
\\	[名詞]	満ちること	
\\	複数形チェック
\\	[動詞]	追い出す	
\\	彼は放校処分にされた。	
\\	[名詞]	銀河	
\\	アンドロメダ銀河は自宅です。	
\\	複数形チェック
\\	[動詞]	期限が満了する	
\\	彼女は賞味期限切れの牛乳を飲んだ。	
\\	[名詞]	強風	
\\	暴風や高波には十分警戒して下さい。	
\\	複数形チェック
\\	[形容詞]	説明的な	
\\	投稿一覧の下の凡例を見て原因が分かりました。	
\\	[名詞]	計測器	
\\	複数形チェック
\\	[形容詞]	明白な	
\\	[名詞]	性別	
\\	トムは男尊女卑思想の持ち主だ。	
\\	複数形チェック
\\	[名詞]	開発、搾取	
\\	資本主義とは人間による人間の搾取である。共産主義はその真逆だ。	
\\	複数形チェック
\\	[名詞]	一般化	
\\	複数形チェック
\\	[名詞]	高速道路	
\\	高速道路で渋滞に巻き込まれてしまいました。	
\\	複数形チェック
\\	[動詞]	一般化する	
\\	本稿の結論をどこまで一般化できるかは、今後の研究で明らかにしていきたい。	
\\	[形容詞]	絶滅した	
\\	鯨は保護されなければ、絶滅するだろう。	
\\	[副詞]	本物の	
\\	[動詞]	(火を)消す	
\\	彼は火事を消し止めた。	
\\	[名詞]	つや、上薬	
\\	複数形チェック
\\	[形容詞]	贅沢な、過度の	
\\	贅沢な趣味のため、彼女は収入の範囲内でやっていくことができませんでした。	
\\	[名詞]	憂うつ	
\\	複数形チェック
\\	[名詞]	眼球	
\\	複数形チェック
\\	[名詞]	名付け親	
\\	複数形チェック
\\	[名詞]	親切	
\\	複数形チェック
\\	[形容詞]	写真入りの	
\\	[名詞]	食通、グルメ	
\\	複数形チェック
\\	[名詞]	心像、比喩的表現	
\\	複数形チェック
\\	[形容詞]	想像できる	
\\	[形容詞]	政府の	
\\	[名詞]	アンバランス	
\\	女性ホルモン分泌の乱れが不妊症の大きな原因です。	
\\	複数形チェック
\\	[名詞]	学年	
\\	私は中学二年生です。	
\\	複数形チェック
\\	[動詞]	浸す	
\\	好きなことに没頭する力を、僕は急速に失いつつある。	
\\	[名詞]	基礎	
\\	複数形チェック
\\	[動詞]	移住する	
\\	[名詞]	ゲリラ兵	
\\	あのね、ゴリラ豪雨じゃなくて、ゲリラ豪雨だよ。ちゃんとわかってる?	
\\	複数形チェック
\\	[名詞]	入国管理	
\\	複数形チェック
\\	[動詞]	ぐっと飲む	
\\	グレタ君がビールを一口で飲み干した。	
\\	[形容詞]	不道徳な	
\\	小説家が嘘を言っても非道徳的と批判されることはありません。それどころか、その嘘が大きければ大きいほど、うまい嘘であればいっそう、一般市民や批評家からの称賛が大きくなります。	
\\	[名詞]	指導者	
\\	複数形チェック
\\	[名詞]	不死	
\\	複数形チェック
\\	[動詞]	噴出する	
\\	[名詞]	非難	
\\	複数形チェック
\\	[名詞]	消化管	
\\	数年にわたって大量に飲んだのでジョンのおなかはビール腹になってしまった。	
\\	複数形チェック
\\	[形容詞]	非人間的な	
\\	[名詞]	体育館	
\\	雨のため、ピクニックは体育館で行われた。	
\\	複数形チェック
\\	[名詞]	含み	
\\	複数形チェック
\\	[名詞]	生息地、居住環境	
\\	彼らの生息地は、森林破壊に脅かされています。	
\\	複数形チェック
\\	[形容詞]	暗黙の	
\\	[副詞]	習慣的に	
\\	彼女は遅刻魔だ。	
\\	[形容詞]	暗に含まれた	
\\	彼女は我々と行きたいのだとほのめかした。	
\\	[名詞]	二日酔い	
\\	二日酔いだ。	
\\	複数形チェック
\\	[形容詞]	実用的でない	
\\	[名詞]	悩ますこと	
\\	同性をセクハラで訴えることできないのかしら。	
\\	複数形チェック
\\	[動詞]	判を押す	
\\	[形容詞]	勤勉な	
\\	彼女はとても勉強熱心です。	
\\	[形容詞]	ありそうもない	
\\	[動詞]	調和させる	
\\	[形容詞]	衝動的な	
\\	トムは直情的だ。	
\\	[名詞]	ハシバミの実	
\\	複数形チェック
\\	[形容詞]	近づきにくい	
\\	[名詞]	有力者	
\\	複数形チェック
\\	[形容詞]	不正確な	
\\	[名詞]	半球	
\\	赤道は地球を2つの半球に分ける。	
\\	複数形チェック
\\	[名詞]	不十分	
\\	複数形チェック
\\	[形容詞]	遺伝的な	
\\	彼女の顔、あんなに腫れていたのにおたふくかぜじゃないんだって。なんか原因不明の伝性病らしいよ。	
\\	[名詞]	収入	
\\	複数形チェック
\\	[名詞]	ヘロイン	
\\	彼はヘロインの常習者だ。	
\\	複数形チェック
\\	[形容詞]	不完全な	
\\	一酸化炭素とは炭素化合物の不完全燃焼で発生する有害物質です。	
\\	[形容詞]	躊躇した	
\\	[形容詞]	不可解な	
\\	[名詞]	階層、序列	
\\	複数形チェック
\\	[形容詞]	一致しない	
\\	[名詞]	ちょうつがい	
\\	ドアの蝶番がギーギーうるさい。	
\\	複数形チェック
\\	[形容詞]	株式会社の	
\\	トムの会社はケイマン諸島で設立されました。	
\\	[名詞]	障害	
\\	複数形チェック
\\	[名詞]	個性	
\\	複数形チェック
\\	[副詞]	これまでのところは	
\\	カール・マルクス曰く「あらゆる社会の歴史は階級闘争の歴史である」だそうです。	
\\	[名詞]	道楽、甘やかし	
\\	複数形チェック
\\	[名詞]	祖国	
\\	彼らは故国を捨てた。	
\\	複数形チェック
\\	[形容詞]	寛大な、甘い	
\\	[名詞]	同性愛の人	
\\	野生のアヒルの中に同性愛の屍姦愛好者いるらしい。	
\\	複数形チェック
\\	[動詞]	産業化する	
\\	[形容詞]	名誉上の	
\\	[形容詞]	効果のない	
\\	運の良いことに、その治療法に害はなく、効果が無いだけだった。	
\\	[名詞]	輪	
\\	複数形チェック
\\	[形容詞]	無能な	
\\	[形容詞]	怖がっている	
\\	[名詞]	不平等	
\\	複数形チェック
\\	[動詞]	ぞっとさせる	
\\	[名詞]	幼年時代	
\\	複数形チェック
\\	[名詞]	ホース	
\\	複数形チェック
\\	[名詞]	劣っていること	
\\	劣等感―それは正確にはどういうことか。	
\\	複数形チェック
\\	[動詞]	入院させる	
\\	トムはまだ入院している。	
\\	[副詞]	無限に	
\\	[名詞]	人質	
\\	ロシア:人質事件の犠牲となった方々に哀悼の意を表す。	
\\	複数形チェック
\\	[形容詞]	曲がらない	
\\	トムって融通が利かないと思う。	
\\	[名詞]	人間主義	
\\	複数形チェック
\\	[形容詞]	情報を与える	
\\	[名詞]	人道主義者	
\\	複数形チェック
\\	[動詞]	抑制する	
\\	[名詞]	湿度	
\\	湿度がすごく高いですね。	
\\	複数形チェック
\\	[名詞]	開始	
\\	複数形チェック
\\	[動詞]	恥をかかせる	
\\	私は多くの人のいる前で恥をかかされた。	
\\	[動詞]	注射する	
\\	[名詞]	屈辱	
\\	どうしてそんな屈辱が、我慢ができるのですか。	
\\	複数形チェック
\\	[名詞]	先天的な	
\\	複数形チェック
\\	[名詞]	万歳!、フレー!	
\\	複数形チェック
\\	[名詞]	旅館の主人	
\\	複数形チェック
\\	[形容詞]	雑種の	
\\	[名詞]	碑文	
\\	良く晴れたある春の日に、ジャンが裏庭の砂場を掘っていると、小さな箱を見つけた。箱の中にはぴかぴかの飛び出しナイフが入っており、不思議な刻印が入れてあった。	
\\	複数形チェック
\\	[名詞]	偽善、偽善行為	
\\	複数形チェック
\\	[形容詞]	分離できない	
\\	[形容詞]	ヒステリックな	
\\	すぐにヒステリーになります。	
\\	[形容詞]	不可欠な	
\\	[名詞]	理想主義	
\\	頑張れさえすれば体力や知力がなくても何でもできるという一種の精神主義もある。	
\\	複数形チェック
\\	[動詞]	統合する	
\\	その学校は、中・高一貫校だということを頭の片隅にでも入れておいて下さい。	
\\	[形容詞]	イデオロギーの	
\\	[形容詞]	統合した	
\\	その学校は、中・高一貫校だということを頭の片隅にでも入れておいて下さい。	
\\	[名詞]	点火	
\\	複数形チェック
\\	[名詞]	統合、集積化、積分	
\\	外務省では、今後の日本における外国人の受入れ政策や社会統合のあり方についての議論が続いていた。	
\\	複数形チェック
\\	[形容詞]	読み書きのできない	
\\	[名詞]	誠実、正直	
\\	われわれの社会には、高潔な人もいれば、詐欺師もいる。	
\\	複数形チェック
\\	[動詞]	相互に影響する	
\\	トムは、他の人と普通に付き合うことができないようです。	
\\	[名詞]	旅客送迎用空港バス	
\\	車?ああ・・・あのリムジンでしたら、私がチャーターした物ですわ。	
\\	複数形チェック
\\	[形容詞]	相互に作用する	
\\	[名詞]	読み書きの能力	
\\	第二に、当時のヨーロッパの識字率が低かったことだ。	
\\	複数形チェック
\\	[動詞]	途中で捕らえる	
\\	[形容詞]	文字どおりの	
\\	それは直訳じゃなくてただの誤訳だよ。	
\\	[動詞]	交換する	
\\	[形容詞]	読み書きのできる	
\\	[名詞]	交際	
\\	複数形チェック
\\	[名詞]	ごみ	
\\	複数形チェック
\\	[名詞]	境界面	
\\	複数形チェック
\\	[名詞]	場所、地方	
\\	複数形チェック
\\	[動詞]	干渉する	
\\	上昇しすぎているとみられるときは、主要国の中央銀行が協力して介入に当たります。	
\\	[副詞]	特定の場所で、地元で	
\\	今年の夏は少雨による水不足が懸念される一方で、局地的なゲリラ豪雨は昨年の3.5倍になるとの予想も出ている。	
\\	[名詞]	親交	
\\	複数形チェック
\\	[名詞]	下宿人	
\\	複数形チェック
\\	[形容詞]	耐えられない	
\\	[名詞]	あこがれ	
\\	あの人達は都会の生活にあこがれている。	
\\	複数形チェック
\\	[形容詞]	複雑な	
\\	[名詞]	見張り	
\\	おそらく、見張りに重大な欠陥があったんではないかと考えています。	
\\	複数形チェック
\\	[形容詞]	入門的な	
\\	『私は人種差別主義者ではありませんが、』と前置きするような文に限って、往々にして人種差別的であったりするものだ。	
\\	[形容詞]	輝く	
\\	[名詞]	侵入者	
\\	トムは空き巣を捕まえた。	
\\	複数形チェック
\\	[形容詞]	月の	
\\	昨日の月食見た?	
\\	[名詞]	侵入	
\\	お邪魔してしまってすみません。	
\\	複数形チェック
\\	[名詞]	狂人	
\\	彼は狂人のように振る舞った。	
\\	複数形チェック
\\	[名詞]	直観	
\\	複数形チェック
\\	[名詞]	魅惑、魅力	
\\	女の髪の毛には大象もつながる。	
\\	複数形チェック
\\	[名詞]	侵入者	
\\	インベーダーと和議の交渉をしないことに決めた。	
\\	複数形チェック
\\	[動詞]	待ち伏せる	
\\	殺人事件の背後に潜むミステリーを解き明かせ。	
\\	[形容詞]	計り知れないほど貴重な	
\\	[名詞]	欲望	
\\	恋愛はただ性欲の詩的表現をうけたものである。	
\\	複数形チェック
\\	[形容詞]	独創的な	
\\	[形容詞]	オペラ風の	
\\	[名詞]	捜査員、研究者	
\\	複数形チェック
\\	[名詞]	行政⻑官	
\\	複数形チェック
\\	[名詞]	請求書	
\\	複数形チェック
\\	[形容詞]	壮大な	
\\	[形容詞]	故意でない	
\\	[名詞]	マンモス	
\\	複数形チェック
\\	[形容詞]	皮肉な	
\\	皮肉だな。	
\\	[名詞]	表明	
\\	複数形チェック
\\	[名詞]	不合理なもの	
\\	複数形チェック
\\	[名詞]	荘園、領地	
\\	複数形チェック
\\	[形容詞]	関連のない	
\\	なんてどうでもいい質問!	
\\	[名詞]	人的資源	
\\	遅れているソフトウェアプロジェクトに人員を追加するとさらに遅れる。	
\\	複数形チェック
\\	[名詞]	ツタ	
\\	医者は自分でミスを隠蔽できるが、建築家は依頼主にツタを植えるよう、言うことしかできない。	
\\	複数形チェック
\\	[動詞]	損なう	
\\	火星は二つ衛星がある。	
\\	[名詞]	クラゲ	
\\	複数形チェック
\\	[形容詞]	周辺的な	
\\	[動詞]	リンリンと鳴る	
\\	[形容詞]	市場性のある	
\\	[形容詞]	失業中の	
\\	2015年9月の日本の失業率は3.4パーセントだった。	
\\	[名詞]	マーマレード	
\\	複数形チェック
\\	[形容詞]	うれしい	
\\	[名詞]	熟達	
\\	複数形チェック
\\	[形容詞]	司法の	
\\	[名詞]	物質主義	
\\	複数形チェック
\\	[名詞]	司法権	
\\	亡命チベット代表者議会は、亡命チベット人社会で最高の権限を持つ立法機関で、1960年に設立された。	
\\	複数形チェック
\\	[動詞]	具体化する	
\\	[形容詞]	少年少女の	
\\	[名詞]	迷路	
\\	市の北のはずれは路地の迷路である。	
\\	複数形チェック
\\	[副詞]	熱心に	
\\	長吉は人間というものは年を取ると、若い時分に経験した若いものしか知らない煩悶不安をばけろりと忘れてしまって、次の時代に生れて来る若いものの身の上を極めて無頓着に訓戒批評する事のできる便利な性質を持っているものだ、年を取ったものと若いものの間には到底一致されない懸隔のある事をつくづく感じた。	
\\	[形容詞]	意味のある、重要な	
\\	[名詞]	血族関係	
\\	複数形チェック
\\	[形容詞]	無意味な、無益な	
\\	それはまったく無意味だ。	
\\	[名詞]	女主人、女家主	
\\	時々大家の婆さんが見にきて 「泣かない子ね」 と云ったが、私の子守は天才的に上手であった。	
\\	複数形チェック
\\	[動詞]	干渉する	
\\	彼に干渉するな。	
\\	[名詞]	画期的な事件	
\\	複数形チェック
\\	[動詞]	調停する	
\\	[名詞]	土地所有者	
\\	複数形チェック
\\	[名詞]	薬物治療	
\\	その患者は新しい薬を飲んでも気分が良くならなかった。	
\\	複数形チェック
\\	[形容詞]	ラップトップの	
\\	このノートパソコンは薄くて軽いです。	
\\	[名詞]	メロドラマ	
\\	複数形チェック
\\	[名詞]	訴訟	
\\	複数形チェック
\\	[形容詞]	記憶すべき	
\\	その後まもなく私はクセノフォンの「ソクラテスの思い出」を手に入れたが、この中にはそれと同じ論法の実例がたくさんある。	
\\	[名詞]	配置	
\\	自分のウェブサイトの構成を変えた。	
\\	複数形チェック
\\	[形容詞]	無慈悲な	
\\	冷徹であれ!	
\\	[形容詞]	葉の多い	
\\	ビタミンEを豊富に含む食物には、濃い緑色をした菜っ葉類・豆類・ナッツ類・全粒穀類がある。	
\\	[動詞]	合併する	
\\	10年前に協和銀行と埼玉銀行は合併してあさひ銀行になった。	
\\	[名詞]	遺産	
\\	複数形チェック
\\	[名詞]	合併	
\\	合併その他の交渉のために同社は25人の担当者を加えた。	
\\	複数形チェック
\\	[形容詞]	伝説的な	
\\	ビルはこの会社での語り草になっている。	
\\	[名詞]	網の目	
\\	警棒で空をなぐりつけ、よろめいて体勢をたてなおしたガードマンは、自分に向かって突進してくる金網づくりのクズカゴを見た。	
\\	複数形チェック
\\	[名詞]	貸し手、金融業者	
\\	複数形チェック
\\	[形容詞]	メートル法の	
\\	[形容詞]	より重要でない	
\\	[形容詞]	中途に	
\\	[動詞]	解放する	
\\	[動詞]	移住する	
\\	なぜ鳥は移り住むのでしょうか。	
\\	[名詞]	解放	
\\	複数形チェック
\\	[名詞]	移住	
\\	17世紀に国際経済論の発展が労働力の大移動を導いたとハンセンが述べているが、これは誤りである。	
\\	複数形チェック
\\	[形容詞]	認可された	
\\	[名詞]	千年間	
\\	複数形チェック
\\	[名詞]	ライト級の選手	
\\	複数形チェック
\\	[動詞]	最小にする	
\\	[名詞]	見込み	
\\	大統領はほぼ間違いなく2期目を務めるだろうね。	
\\	複数形チェック
\\	[名詞]	ハッカ	
\\	今日夕方ニコラと一緒にレストランで無料のミントティーを飲んだ。	
\\	複数形チェック
\\	[名詞]	類似	
\\	複数形チェック
\\	[形容詞]	いたずら好きな	
\\	[形容詞]	人を誤らせる	
\\	今回の試験は、難しかった上に質問も紛らわしかった。	
\\	[名詞]	ダチョウ	
\\	ダチョウは羽があるが飛べない。	
\\	複数形チェック
\\	[名詞]	宣教師	
\\	1859年以降、プロテスタント宣教師がアメリカから派遣され、またカトリックとロシア正教も布教活動を行った。	
\\	複数形チェック
\\	[動詞]	不法とする	
\\	[形容詞]	霧の深い	
\\	ミスティーさんは虫について違う感情があると思う。	
\\	[動詞] [不規則動詞]	度を超す	
\\	無理するなって。	
\\	[名詞]	暴徒	
\\	複数形チェック
\\	[動詞]	重ねる	
\\	[名詞]	機動性のある	
\\	携帯を充電する必要がある。	
\\	複数形チェック
\\	[動詞]	負かす、圧倒する	
\\	[名詞]	移動性	
\\	複数形チェック
\\	[名詞]	オゾン	
\\	複数形チェック
\\	[名詞]	あざけり	
\\	複数形チェック
\\	[動詞]	こぐ	
\\	漕ぎ続けて。	
\\	[名詞]	交換可能な部分	
\\	このモジュールを使用すると、エンターキーでイベントを生成することができます。	
\\	複数形チェック
\\	[名詞]	同等、同価	
\\	複数形チェック
\\	[形容詞]	気分屋の	
\\	トムって気分屋だよね。	
\\	[名詞]	寓話	
\\	複数形チェック
\\	[動詞]	モップでふく	
\\	よし!掃討完了!	
\\	[名詞]	パラシュート	
\\	なぜ飛行機にはパラシュートではなくライフジャケットがあるのか。	
\\	複数形チェック
\\	[名詞]	ガ	
\\	それは蝶ですか、それとも蛾ですか。	
\\	複数形チェック
\\	[形容詞]	矛盾した	
\\	[名詞]	金言、格言	
\\	「自由・平等・博愛」はハイチ共和国のスローガンでもある。	
\\	複数形チェック
\\	[動詞]	言い換える	
\\	[名詞]	小山	
\\	直径23
\\	(下段)及び18
\\	(上段)、高さ5mの二段式の円墳である。	
\\	複数形チェック
\\	[形容詞]	親の	
\\	[名詞]	喪中	
\\	喪服をクリーニングに出せますか?	
\\	複数形チェック
\\	[名詞]	別れ	
\\	逢うは別れの始め。	
\\	複数形チェック
\\	[形容詞]	動かせる	
\\	[名詞]	通行人	
\\	複数形チェック
\\	[動詞] [不規則動詞]	刈り取る	
\\	母は私に芝を刈るように言った。	
\\	[名詞]	牧歌的な	
\\	複数形チェック
\\	[名詞]	掛け算	
\\	複数形チェック
\\	[名詞]	展示館	
\\	雪に包まれた金閣の美しさは、比べるものがなかった。	
\\	複数形チェック
\\	[形容詞]	ものを言わないで	
\\	お母さんはいらっしゃいますか?	
\\	[形容詞]	未決定で	
\\	[動詞]	もぐもぐ言う	
\\	「ふむ」 武田先生は口ひげを捻りながらうなった。	
\\	[形容詞]	危険な	
\\	[副詞]	相互に	
\\	[名詞]	定期刊行物、雑誌	
\\	複数形チェック
\\	[形容詞]	神秘的な	
\\	[名詞]	永久不変	
\\	複数形チェック
\\	[名詞]	語り手	
\\	複数形チェック
\\	[動詞]	迫害する	
\\	彼は人々を宗教により迫害しました。	
\\	[副詞]	かろうじて	
\\	われわれはかろうじて事故を免れた。	
\\	[名詞]	迫害	
\\	複数形チェック
\\	[形容詞]	⺠族主義の	
\\	[名詞]	忍耐	
\\	全ての場合を通じて、恋愛は忍耐である。	
\\	複数形チェック
\\	[形容詞]	全国的な	
\\	[名詞]	粘り強さ	
\\	複数形チェック
\\	[名詞]	ネクタイ	
\\	彼女は私の好みにぴったり合ったネクタイをくれた。	
\\	複数形チェック
\\	[名詞]	害虫	
\\	くそー、渚のやつ、本当に先生にチクりやがった。	
\\	複数形チェック
\\	[形容詞]	怠慢な	
\\	彼は職務怠慢だった。	
\\	[名詞]	薬局、薬学	
\\	この近くに薬局はありますか?	
\\	複数形チェック
\\	[形容詞]	無視してよい	
\\	洪水の被害はごくわずかだった。	
\\	[形容詞]	哲学の	
\\	[名詞]	日暮れ	
\\	この空模様から察すると、日暮れ前にひょっとしたら一雨降るかも知れません。	
\\	複数形チェック
\\	[名詞]	スリ	
\\	彼はすりの襟首を掴んだ。	
\\	複数形チェック
\\	[形容詞]	夜ごとの	
\\	[名詞]	埠頭、防波堤	
\\	彼女は彼を埠頭から突き落とした。	
\\	複数形チェック
\\	[名詞]	北方	
\\	複数形チェック
\\	[名詞]	巡礼者	
\\	遍路さんたちと一緒に去る事に決めた。	
\\	複数形チェック
\\	[副詞]	著しく、とりわけ	
\\	[名詞]	海賊	
\\	彼はハロウィーンで海賊の衣装を着た。	
\\	複数形チェック
\\	[動詞]	知らせる	
\\	娘が職場で突然倒れたって連絡があった時は驚いたよ。頭も打ってたから心配だったけど、病院の検査では何ともないって聞いてほっとしたよ。	
\\	[名詞]	配置	
\\	複数形チェック
\\	[名詞]	栄養	
\\	複数形チェック
\\	[名詞]	原告	
\\	複数形チェック
\\	[名詞]	斬新さ	
\\	複数形チェック
\\	[名詞]	厚板	
\\	複数形チェック
\\	[名詞]	初心者	
\\	愛することにかけては、女性こそ専門家で、男性は永遠に素人である。	
\\	複数形チェック
\\	[名詞]	農園主	
\\	複数形チェック
\\	[名詞]	数字	
\\	ローマ数字わかる?	
\\	複数形チェック
\\	[名詞]	白金	
\\	複数形チェック
\\	[動詞]	大事に育てる	
\\	君のトカゲは、たとえフラッフィーと名付け育てたところでその愛に応えてくれはしない。	
\\	[名詞]	嘆願	
\\	ヘタクソな人は英訳しないでください。英訳依頼者からのお願いです。	
\\	複数形チェック
\\	[形容詞]	わいせつな	
\\	それとも、あまりに猥褻な写真があるから出版できないということか?	
\\	[副詞]	愛想よく	
\\	嬉しい驚きでした。	
\\	[名詞]	不明箇所	
\\	彼のいっしょうは社会の下積みで終わった。	
\\	複数形チェック
\\	[動詞]	密猟する	
\\	彼はライバル会社に引き抜かれました。	
\\	[名詞]	遵守	
\\	複数形チェック
\\	[動詞]	汚染する	
\\	車はかなりの汚染を引き起こします。	
\\	[動詞]	取りつく	
\\	私の妻は掃除のことばかり考えている。	
\\	[動詞]	生息する	
\\	[形容詞]	職業の	
\\	近年作業療法の需要は急速に高まった。	
\\	[名詞]	ポルノ	
\\	複数形チェック
\\	[動詞] [不規則動詞]	相殺する	
\\	[名詞]	金融資産の一覧	
\\	複数形チェック
\\	[形容詞]	進行中の	
\\	[形容詞]	大学院の	
\\	[形容詞]	オンラインの	
\\	ネットゲームはしたことがない。	
\\	[名詞]	姿勢	
\\	彼は知識人ぶるのが好きだ。	
\\	複数形チェック
\\	[形容詞]	運転可能な	
\\	は、お客様の多様な環境やニーズにも柔軟に対応し、さまざまな運用方法をご提供します。	
\\	[形容詞]	戦後の	
\\	小沢は外地から復員して、今夜やっと故郷の大阪へ帰って来たばかしだが、終戦後の都会や近郊の辻強盗の噂は、汽車の中できいて知っていた。	
\\	[形容詞]	圧迫的な	
\\	[名詞]	潜在能力	
\\	複数形チェック
\\	[形容詞]	組織に関する	
\\	[副詞]	潜在的に	
\\	[名詞]	方向性	
\\	宇宙飛行士は重力を感じないので、どんな方向にも寝ることができる。しかし、体が浮き上がって何かに衝突してしまわないように自分の身を固定しなくてはいけない。	
\\	複数形チェック
\\	[名詞]	開業医	
\\	彼は7年間、合気道をやっていた。	
\\	複数形チェック
\\	[形容詞]	実用的な	
\\	心情的には賛成、けれど現実的に反対します。	
\\	[名詞]	現実主義	
\\	複数形チェック
\\	[名詞]	大草原	
\\	複数形チェック
\\	[名詞]	現実主義者	
\\	複数形チェック
\\	[名詞]	説教者	
\\	ボブは牧師になりました。	
\\	複数形チェック
\\	[動詞]	退く	
\\	あいつ、おでこが広いだけだって言ってるけどさ、要は生え際が後退してるってことだろ。	
\\	[形容詞]	予想どおりの	
\\	[名詞]	受取人	
\\	複数形チェック
\\	[形容詞]	先史の	
\\	[名詞]	独唱会	
\\	複数形チェック
\\	[名詞]	保険の掛け金	
\\	複数形チェック
\\	[形容詞]	認識できる	
\\	[名詞]	司会者	
\\	複数形チェック
\\	[動詞]	再検討する	
\\	考え直してください。	
\\	[名詞]	大統領の地位	
\\	大統領には彼が適任だ。	
\\	複数形チェック
\\	[動詞]	復元する	
\\	彼は3
\\	プリンターによって顔と生活を取り戻した。	
\\	[形容詞]	名声のある	
\\	[名詞]	再建	
\\	複数形チェック
\\	[名詞]	見せかけ	
\\	複数形チェック
\\	[動詞]	詳しく話す	
\\	満天の星空の下、トムとメアリーは時間を忘れて夢を語り合った。	
\\	[形容詞]	うぬぼれた、偽りの	
\\	彼女の態度は気取りやわざとらしさがない。	
\\	[動詞]	再創造する	
\\	[形容詞]	予防の	
\\	[名詞]	新会員募集	
\\	複数形チェック
\\	[形容詞]	ちくちくする	
\\	[形容詞]	赤みがかった	
\\	[動詞]	⺠営化する	
\\	[動詞]	再び定義する	
\\	[名詞]	賛成投票	
\\	プロが撮るからいい写真になるんです。	
\\	複数形チェック
\\	[動詞]	再度設計する	
\\	[形容詞]	問題のある	
\\	[形容詞]	減じた	
\\	私は肉を食べるのを控えた。	
\\	[名詞]	処理装置	
\\	彼らはワープロにいつも手を焼いている。	
\\	複数形チェック
\\	[形容詞]	余剰の	
\\	トムはリストラされた。	
\\	[名詞]	熟達	
\\	昨年は日本語能力試験N3を受験したんだ。	
\\	複数形チェック
\\	[名詞]	お代わりすること	
\\	複数形チェック
\\	[副詞]	深く	
\\	[名詞]	改革	
\\	複数形チェック
\\	[名詞]	進行	
\\	デモの様子は参加者によって
\\	を通じて実況された。	
\\	複数形チェック
\\	[名詞]	宗教改革者	
\\	複数形チェック
\\	[名詞]	プロローグ	
\\	複数形チェック
\\	[名詞]	補強	
\\	複数形チェック
\\	[動詞]	推進する	
\\	[名詞]	適切さ	
\\	複数形チェック
\\	[名詞]	予言能力、予言	
\\	複数形チェック
\\	[名詞]	信頼	
\\	共和党の旗を初めてホワイトハウスにもっていたのはこの州出身の男性で、その政党は自己依存と個人の自由と国家の統一の価値観にもとづいていたことを思い出そう。	
\\	複数形チェック
\\	[形容詞]	比例する	
\\	マフィンを受ける数は、
\\	でしゃべる回数に反比例して変化する。	
\\	[名詞]	遺物、遺跡	
\\	ローマは至る所に遺跡がある。	
\\	複数形チェック
\\	[動詞]	起訴する	
\\	公務員が秘密を漏らして、秘密漏洩罪として罰せられた。	
\\	[名詞]	風味	
\\	複数形チェック
\\	[名詞]	起訴	
\\	複数形チェック
\\	[形容詞]	思い出させて	
\\	[名詞]	検察官	
\\	その検事は裁判の証拠を取り下げた。	
\\	複数形チェック
\\	[形容詞]	離れて	
\\	シャツを脱いだ。	
\\	[動詞]	売春する	
\\	ある少女の職業は娼婦だったが、彼女は自分のおばあちゃんにこのことを知られたくなかった。	
\\	[動詞]	はじく	
\\	[名詞]	たん白質	
\\	トムは、低脂肪・高タンパク質ダイエットをしています。	
\\	複数形チェック
\\	[副詞]	伝える所によれば	
\\	[副詞]	わざと	
\\	[形容詞]	憤慨している	
\\	[動詞]	身震いする	
\\	福島第一原子力発電所では、地震直後にすべての原子炉が停止した。	
\\	[名詞]	辞職	
\\	彼は辞表を提出した。	
\\	複数形チェック
\\	[副詞]	季刊で	
\\	四半期1
\\	の成長は年率4
\\	の成長率を意味する。	
\\	[形容詞]	断固とした	
\\	[名詞]	質問	
\\	複数形チェック
\\	[形容詞]	制限された	
\\	言論の自由が制限されている国もある。	
\\	[名詞]	アンケート	
\\	アンケートにお答えいただいた方の中から、毎月抽選で30名様に1万円分の商品券をプレゼントいたします。	
\\	複数形チェック
\\	[名詞]	小売業者	
\\	複数形チェック
\\	[名詞]	キルト	
\\	冬の時に掛け布団二枚使って寝るものです。	
\\	複数形チェック
\\	[動詞]	言い返す	
\\	[動詞]	震える	
\\	彼女の声は怒りに震えていた。	
\\	[名詞]	訂正、改訂	
\\	彼は規則の改正を唱えている。	
\\	複数形チェック
\\	[名詞]	割り当て	
\\	どうやってこのノルマを達成すればいいんだ?	
\\	複数形チェック
\\	[動詞]	さざなみを立たせる	
\\	[名詞]	放射能	
\\	放射能漏れの実態や安全確認することは難しいのです。	
\\	複数形チェック
\\	[名詞]	儀式	
\\	複数形チェック
\\	[名詞]	放熱器	
\\	複数形チェック
\\	[名詞]	競争	
\\	複数形チェック
\\	[名詞]	ラジウム	
\\	複数形チェック
\\	[動詞]	回転させる	
\\	[名詞]	いかだ、ゴムボート	
\\	複数形チェック
\\	[名詞]	口紅	
\\	ほお紅や口紅を差すだけで表情や笑顔がどんどん変わってくるんです。	
\\	複数形チェック
\\	[形容詞]	高くなった	
\\	彼の発言は混乱を引き起こした。	
\\	[動詞]	ゴロゴロ鳴る	
\\	さっきからずっとお腹鳴ってるけど大丈夫?	
\\	[名詞]	森林警備隊	
\\	複数形チェック
\\	[名詞]	滑走路	
\\	複数形チェック
\\	[名詞]	コツンとたたくこと	
\\	ラップは好き?	
\\	複数形チェック
\\	[形容詞]	さびた	
\\	その部屋で古びた錆だらけ鍵を見つけました。	
\\	[名詞]	速さ	
\\	複数形チェック
\\	[名詞]	ライ⻨	
\\	ライ麦は貧困の穀物と称されていた。	
\\	複数形チェック
\\	[名詞]	歓喜	
\\	複数形チェック
\\	[名詞]	正気	
\\	複数形チェック
\\	[名詞]	珍しさ	
\\	複数形チェック
\\	[形容詞]	皮肉な	
\\	[名詞]	激賞	
\\	複数形チェック
\\	[動詞]	ざっと見る	
\\	彼は私の原稿にざっと目を通してくれた。	
\\	[名詞]	原子炉	
\\	原子力発電所の1号機で、原子炉内の燃料の溶融が進んでいる可能性が高い、と発表された。	
\\	複数形チェック
\\	[形容詞]	乏しい	
\\	五十歩百歩。	
\\	[名詞]	用意	
\\	複数形チェック
\\	[名詞]	不足	
\\	複数形チェック
\\	[形容詞]	風景の	
\\	スイスは風光明媚なことで有名である。	
\\	[名詞]	主権者、君主	
\\	複数形チェック
\\	[名詞]	資格審査	
\\	昨日、新作映画の試写会に行ってきた。	
\\	複数形チェック
\\	[名詞]	温泉、温泉地	
\\	複数形チェック
\\	[名詞]	ねじまわし	
\\	俺のマイナスドライバーどこにやった!!	
\\	複数形チェック
\\	[形容詞]	広々とした、雄大な	
\\	広いな!	
\\	[名詞]	巻き物、古文書	
\\	複数形チェック
\\	[形容詞]	空間的な	
\\	[動詞]	ごしごしこすって洗う	
\\	トムは足をゴシゴシ洗った。	
\\	[名詞]	専門	
\\	それは彼の十八番だ。	
\\	複数形チェック
\\	[名詞]	彫刻家	
\\	複数形チェック
\\	[動詞]	推測する	
\\	[名詞]	秘密	
\\	複数形チェック
\\	[名詞]	背骨	
\\	恐ろしい光景に彼は戦慄した。	
\\	複数形チェック
\\	[名詞]	部分、区分	
\\	この部品はフォード社から君に届いたものだ。	
\\	複数形チェック
\\	[名詞]	スポットライト	
\\	複数形チェック
\\	[名詞]	感受性	
\\	複数形チェック
\\	[名詞]	配偶者	
\\	配偶者を殺すのは、婚姻関係を終わらせる一つの方法です。ただし、良しとされることではありません。	
\\	複数形チェック
\\	[名詞]	感知装置、センサー	
\\	複数形チェック
\\	[動詞]	手足を伸ばして横になる	
\\	彼はソファーに寝そべっている。	
\\	[形容詞]	感覚の、敏感な	
\\	[動詞]	安定させる	
\\	[形容詞]	のどかな、穏やかな	
\\	[名詞]	スタミナ	
\\	サムにはマラソンを完走するだけの力がない。	
\\	複数形チェック
\\	[形容詞]	連続的な	
\\	来月号から新連載小説が始まります。	
\\	[動詞]	どもる	
\\	彼は口ごもりながら二言三言しゃべった。	
\\	[名詞]	給仕する人	
\\	サーバーがダウンしていました。	
\\	複数形チェック
\\	[動詞]	標準化する	
\\	[形容詞]	解決した、人の住んでいる	
\\	論争にやっとけりが付いた。	
\\	[名詞]	星明かり	
\\	複数形チェック
\\	[名詞]	開拓移⺠	
\\	複数形チェック
\\	[名詞]	先発投手	
\\	複数形チェック
\\	[形容詞]	厚かましい	
\\	彼女は無邪気なのか羞恥心がないのか。	
\\	[形容詞]	静止した	
\\	[名詞]	株主	
\\	複数形チェック
\\	[名詞]	身⻑	
\\	からだは、その娘とは違って、丈が低く、横にでぶでぶ太って、豚の体に人の首がついているようだ。	
\\	複数形チェック
\\	[名詞]	貝	
\\	複数形チェック
\\	[名詞]	法令	
\\	この犯罪については時効が成立している。	
\\	複数形チェック
\\	[動詞]	棚に載せる	
\\	彼はすべての棚を測った。	
\\	[形容詞]	無菌の	
\\	[名詞]	発送、発送品	
\\	複数形チェック
\\	[名詞]	ステッカー	
\\	複数形チェック
\\	[名詞]	射撃	
\\	見てごらん、流れ星だ!	
\\	複数形チェック
\\	[名詞]	刺激	
\\	複数形チェック
\\	[名詞]	海岸線	
\\	複数形チェック
\\	[動詞]	絞め殺す	
\\	私の仕事は鼠を絞め殺す事です。	
\\	[動詞]	ぐいと押す	
\\	彼は手紙をポケットにねじ込んだ。	
\\	[形容詞]	戦略の	
\\	[形容詞]	鋭い	
\\	[名詞]	一続きの幸運	
\\	複数形チェック
\\	[動詞]	鋭く鳴る	
\\	突然鋭い叫び声が聞こえた。	
\\	[形容詞]	ストレスの多い	
\\	都会は刺激的な場所だが、ストレスも多い。	
\\	[動詞]	震える	
\\	恐ろしい光景に彼は戦慄した。	
\\	[形容詞]	縞のある	
\\	あのグリーンの縞のシャツ、今洗濯してるん?	
\\	[動詞]	病気になる	
\\	彼は納豆の匂いを嗅ぐと気分が悪くなる。	
\\	[名詞]	散歩する	
\\	そうだ、どうせなら散歩がてらに、林道に行ってプチ森林浴でも・・・。	
\\	複数形チェック
\\	[名詞]	食器戶棚	
\\	複数形チェック
\\	[形容詞]	潜在意識の	
\\	[名詞]	サイレン	
\\	サイレンが急変を知らせました。	
\\	複数形チェック
\\	[動詞]	沈める	
\\	くじらは長い間水に潜っていられる。	
\\	[名詞]	骸骨	
\\	彼は骸骨のようにやせている。	
\\	複数形チェック
\\	[形容詞]	従属する	
\\	[名詞]	横すべり	
\\	複数形チェック
\\	[動詞]	静まる	
\\	[動詞]	切り払う	
\\	[副詞]	実質的に、十分に	
\\	[名詞]	屠殺、虐殺	
\\	複数形チェック
\\	[名詞]	代理、代用	
\\	複数形チェック
\\	[名詞]	スラム街	
\\	メキシコにはたくさんのスラムがある	
\\	複数形チェック
\\	[形容詞]	郊外の	
\\	[動詞]	はまり込む	
\\	トムは最近スランプに陥っている。	
\\	[動詞]	満足させる	
\\	言うまでもなく、エドナはなにもできない。	
\\	[名詞]	平手打ち	
\\	複数形チェック
\\	[副詞]	十分に	
\\	田中さんは、十分なトレーニングをしました。	
\\	[動詞]	密輸入する	
\\	[形容詞]	表面上の	
\\	うわべだけの友達なんか要らない。	
\\	[名詞]	密輸者	
\\	複数形チェック
\\	[名詞]	優勢	
\\	彼は優越感を持っている。	
\\	複数形チェック
\\	[名詞]	スナップ写真	
\\	「そろそろ、人物写真でもやってみたらどうだ?」「え?スナップ・・・ですか?それは・・・」	
\\	複数形チェック
\\	[形容詞]	迷信深い	
\\	[動詞] [不規則動詞]	こっそり逃れる	
\\	彼はこっそりと歩いて家の裏手のドアへと回った。	
\\	[名詞]	上司、監督	
\\	複数形チェック
\\	[動詞]	冷笑する	
\\	彼はばかなことをして良く笑われた。	
\\	[副詞]	おそらく	
\\	おそらくトムは良いコックだ。	
\\	[動詞]	くんくんかぐ	
\\	犬は見知らぬ人の匂いをかいだ。	
\\	[副詞]	疑わしげに	
\\	彼女は疑わしそうに私を見つめた。	
\\	[動詞]	鼻息荒く言う	
\\	[動詞]	渦を巻く	
\\	トムとメリーがデートしていたという噂が数週間渦巻いていた。	
\\	[形容詞]	社交的な	
\\	彼はとても社交的だ。	
\\	[名詞]	象徴主義	
\\	複数形チェック
\\	[動詞]	社交的に交際する	
\\	[名詞]	銀行団	
\\	複数形チェック
\\	[名詞]	ソケット、穴	
\\	眼窩とは、目をいれている骨のいれもののことです。	
\\	複数形チェック
\\	[動詞]	びょうで留める	
\\	要点を絞って、決定を下しましょう。	
\\	[形容詞]	厳粛な	
\\	お坊さんが突然大声で笑い出し、厳粛な雰囲気を台無しにした。	
\\	[名詞]	買収、乗っ取り	
\\	我々
\\	としては、全力を持って敵対的買収に対抗するものと、明言しておこう。	
\\	複数形チェック
\\	[名詞]	洗練された	
\\	複数形チェック
\\	[名詞]	タール	
\\	コールタールを蒸留して得られる成分名は下表の通りである。	
\\	複数形チェック
\\	[名詞]	南部	
\\	複数形チェック
\\	[名詞]	酒場	
\\	居酒屋で変なおっさんにからまれた。	
\\	複数形チェック
\\	[名詞]	技術	
\\	複数形チェック
\\	[形容詞]	忘れられない	
\\	彼女の菫色の瞳が忘れられない。	
\\	[名詞]	気質	
\\	複数形チェック
\\	[形容詞]	友情のない	
\\	彼はお金持ちだが、親切ではない。	
\\	[形容詞]	心をそそる	
\\	心が動くね。	
\\	[動詞]	統合する	
\\	貴君に、この統合軍の指揮をとってもらいたい。	
\\	[形容詞]	恐ろしい	
\\	食べ物の恨みは恐ろしいんだからね!	
\\	[形容詞]	実在しない	
\\	それは非現実的だ。	
\\	[名詞]	テロ行為	
\\	我々は絶対にテロを許さない。	
\\	複数形チェック
\\	[形容詞]	関連のない	
\\	仕事が終わった後に、仕事以外の勉強をするためには、強い意志も必要ですね。	
\\	[名詞]	遺言	
\\	サタンのこうした行いを神がなぜ許しているのか、その理由は聖書のヨブ記で説明されている。	
\\	複数形チェック
\\	[形容詞]	頼りにならない	
\\	トムは全く頼りにならない。	
\\	[名詞]	証言	
\\	被害者の供述だけを聞いていては、事件の真相がわかりません。	
\\	複数形チェック
\\	[形容詞]	安全でない	
\\	この建物は安全じゃない。	
\\	[名詞]	理論家	
\\	複数形チェック
\\	[形容詞]	定まらない	
\\	[名詞]	療法士	
\\	複数形チェック
\\	[形容詞]	不安定な	
\\	[名詞]	倹約	
\\	複数形チェック
\\	[形容詞]	疑わない	
\\	[動詞]	くすぐる	
\\	トムはメアリーをくすぐった。	
\\	[動詞]	ほどく	
\\	靴紐が解けていますよ。	
\\	[動詞]	傾ける	
\\	僕は首をひねる。このポーンではなくて、こっちのナイトを・・・。	
\\	[形容詞]	誠実でない	
\\	もし私が浮気したらどうする?	
\\	[名詞]	強壮剤	
\\	ジン・トニックをください。	
\\	複数形チェック
\\	[形容詞]	不要な	
\\	[名詞]	⻯巻	
\\	竜巻で村全体が破壊された。	
\\	複数形チェック
\\	[動詞]	格上げする	
\\	防衛庁が防衛省に格上げされたのはなぜですか?	
\\	[名詞]	牽引する	
\\	この車をけん引することはできない。	
\\	複数形チェック
\\	[動詞]	持ち上げる	
\\	[形容詞]	有毒な	
\\	[動詞]	根絶する	
\\	[名詞] [動詞]	商標(にする・をつける)	
\\	複数形チェック
\\	[名詞]	緊急性	
\\	複数形チェック
\\	[名詞]	追跡者、トレーラー	
\\	複数形チェック
\\	[名詞]	菜食主義者	
\\	僕はベジタリアンだ。	
\\	複数形チェック
\\	[名詞]	特徴	
\\	のび太君は、人の幸せを願い、 人の不幸を悲しむことができる青年だ。 それは、人間にとって最も大切なこと。 きっと君を幸せにしてくれるよ。	
\\	複数形チェック
\\	[名詞]	速度	
\\	複数形チェック
\\	[名詞]	謄本、成績証明書	
\\	複数形チェック
\\	[名詞]	穴、通風孔	
\\	複数形チェック
\\	[名詞]	透明性	
\\	複数形チェック
\\	[前置詞]	
\\	対	
\\	ホワイトバランスをオートにした場合と白いコピー用紙でプリセットした場合の比較です。	
\\	[動詞]	移植する	
\\	外科医に説得されて、彼は臓器移植手術を受けることにした。	
\\	[名詞]	退役軍人	
\\	私の叔父はベトナム戦争の退役軍人だ。	
\\	複数形チェック
\\	[動詞]	横切る	
\\	[名詞]	振動	
\\	複数形チェック
\\	[形容詞]	裏切りの、不誠実な	
\\	[動詞]	犠牲にする	
\\	[名詞]	深い溝	
\\	複数形チェック
\\	[形容詞]	火山の	
\\	[形容詞]	部族の	
\\	[動詞]	吐く	
\\	胸がムカムカします。	
\\	[名詞]	引き金	
\\	トムには引き金を引く勇気がなかった。	
\\	複数形チェック
\\	[名詞]	クーポン券	
\\	アンケートにお答えいただいた方の中から、毎月抽選で30名様に1万円分の商品券をプレゼントいたします。	
\\	複数形チェック
\\	[名詞]	
\\	重唱	
\\	複数形チェック
\\	[形容詞]	傷つきやすい	
\\	[形容詞]	勝利を得た	
\\	あいつがガッツポーズするなんて、よっぽど嬉しかったんだろうな。	
\\	[動詞]	振る	
\\	[動詞]	小走りする	
\\	[名詞]	ワルツ	
\\	ワルツとサンバは踊りです。	
\\	複数形チェック
\\	[名詞]	マス	
\\	これでこの一画は少なくとも20匹ニジマスがいるわけだ。	
\\	複数形チェック
\\	[形容詞]	放浪する	
\\	[動詞]	ぐいと引く	
\\	「委員長、後は何を運ぶんだっけ?」「得点板と大玉もう一個、綱引きの綱」	
\\	[名詞]	衣裳だんす	
\\	ジャネット・ジャクソンは衣装の不具合で、見えてはいけないところが見えてしまった。	
\\	複数形チェック
\\	[名詞]	大混乱	
\\	複数形チェック
\\	[名詞]	倉庫	
\\	彼は倉庫の警備人として仕事をしている。	
\\	複数形チェック
\\	[名詞]	(植)カブ	
\\	複数形チェック
\\	[動詞]	ゆがめる	
\\	[形容詞]	生まれていない	
\\	[名詞]	軍艦	
\\	その大砲8門を備えた軍艦の船級は「クイーンエリザベス」です。	
\\	複数形チェック
\\	[形容詞]	途切れない	
\\	[名詞]	洗濯機	
\\	複数形チェック
\\	[形容詞]	無条件の	
\\	[名詞]	スズメバチ、ワスプ	
\\	夏には、それは鳥の卵、スズメバチの幼虫、ベリーおよび蜂蜜を食べる。	
\\	複数形チェック
\\	[形容詞]	制御できない	
\\	[名詞]	河岸、海岸通り	
\\	複数形チェック
\\	[動詞]	過小評価する	
\\	私を甘く見るな。	
\\	[名詞]	航路、運河	
\\	複数形チェック
\\	[動詞] [不規則動詞]	横たわる	
\\	[名詞]	織り手	
\\	複数形チェック
\\	[形容詞]	潜在的な	
\\	[名詞]	改札口	
\\	複数形チェック
\\	[形容詞]	水中の	
\\	潜水して泳げるかい。	
\\	[名詞]	風車	
\\	オランダでは、どこへ行っても風車が見られる。	
\\	複数形チェック
\\	[形容詞]	邪魔されない、平穏な	
\\	[動詞] [不規則動詞]	保留する、抑える	
\\	その検事は裁判の証拠を取り下げた。	
\\	[形容詞]	終わりのない	
\\	[名詞]	仕事中毒の人	
\\	君は仕事の鬼だ。	
\\	複数形チェック
\\	[形容詞]	等しくない	
\\	日本の結婚制度は男に不利な男女不平等条約 だと思います。	
\\	[名詞]	職場	
\\	いつも仕事場をきちんと整理しておきなさい。	
\\	複数形チェック
\\	[形容詞]	非倫理的な	
\\	[動詞]	悪化する	
\\	トムの容体が悪化した。	
\\	[形容詞]	一様でない	
\\	[名詞]	包み紙	
\\	複数形チェック
\\	[形容詞]	好ましくない	
\\	[名詞]	不正行為	
\\	君の不正行為は我が校の不名誉だ。	
\\	複数形チェック
\\	[形容詞]	完結していない	
\\	それは未完に終わった小説です。	
\\	[動詞]	切望する	
\\	日本シンクロ界の悲願である金には、あと一歩で届かなかった。	
\\	[形容詞]	不向きで	
\\	彼女はその仕事に不向きだ。	
\\	[動詞]	映像を拡大する	
\\	同社の売上げは輸出の需要が強いおかげで伸びたが、競争が激しく利益はそれほど伸びなかった。	
\\	[動詞]	短縮する	
\\	は
\\	と略記される。	
\\	[副詞]	いつでも	
\\	いつでも来て。	
\\	[動詞]	中止になる	
\\	[名詞]	無感動無関心冷淡	
\\	それは、無関心層の神話を拒否した若者-家と家族から離れ、賃金が安く、睡眠時間も短い仕事をする-からの強さを育てた。	
\\	複数形チェック
\\	[名詞]	抽象概念	
\\	複数形チェック
\\	[副詞]	個々に	
\\	[名詞]	ばかげた行為	
\\	複数形チェック
\\	[動詞]	理解する	
\\	[名詞]	加速度	
\\	複数形チェック
\\	[名詞]	徒弟期間	
\\	内弟子とは師と寝食をともにし、あらゆる武術のノウハウと秘伝を伝える制度だ。	
\\	複数形チェック
\\	[名詞]	音響学	
\\	音を吸収する素材のことを吸音材といいます。	
\\	複数形チェック
\\	[名詞]	才能、適性	
\\	複数形チェック
\\	[動詞]	活動的にする	
\\	管理人によってアカウントの認証が行われたあと、あなたに確認のメールが送られます。	
\\	[形容詞]	水生の	
\\	[名詞]	現実、実在	
\\	私は力持ちに見えるかもしれないが、実はそうではない。	
\\	複数形チェック
\\	[名詞]	大司教大主教	
\\	複数形チェック
\\	[形容詞]	常習的な	
\\	彼女はアルコール中毒だ。	
\\	[名詞]	腕一杯	
\\	複数形チェック
\\	[形容詞]	習慣性の	
\\	シンナーなどの常習性のある物をしてると言う事は、更正する可能性が低いと思います。	
\\	[形容詞]	巧妙な	
\\	[名詞]	適切さ	
\\	複数形チェック
\\	[名詞]	暗殺者	
\\	複数形チェック
\\	[名詞]	執着、忠実	
\\	複数形チェック
\\	[名詞]	暗殺	
\\	年配の人はまだケネディー暗殺事件を覚えている。	
\\	複数形チェック
\\	[形容詞]	隣接する	
\\	うちと彼の家は隣り合っている。	
\\	[形容詞]	断言的な	
\\	彼女は気が強い。	
\\	[名詞]	入場許可	
\\	複数形チェック
\\	[動詞]	同化する	
\\	家族はまもなく新しい環境に溶け込んだ。	
\\	[副詞]	明らかに	
\\	彼は明らかに有能な指導者だ。	
\\	[副詞]	道に迷って	
\\	[形容詞]	愛らしい、魅力的な	
\\	その猫は可愛らしい。	
\\	[形容詞]	天文学的な	
\\	[名詞]	装飾品	
\\	複数形チェック
\\	[動詞]	証明する	
\\	[名詞]	逆境	
\\	逆境にもかかわらず、その建築家は世界的名声を獲得した。	
\\	複数形チェック
\\	[名詞]	会計検査官、聴講生	
\\	複数形チェック
\\	[形容詞]	エアロビクスの	
\\	[名詞]	権威主義者	
\\	複数形チェック
\\	[副詞]	遠くに	
\\	遠くから見れば、大抵のものは綺麗に見える。	
\\	[動詞]	自動化する	
\\	[形容詞]	影響を受けた	
\\	聴衆は深く感動した。	
\\	[形容詞]	自動車の	
\\	中国輸入自動車市場に、緩やかな成長の兆しが現れている。	
\\	[名詞]	肯定宣誓	
\\	このような形を二重否定といいますが、結果として肯定を表すこととなります。	
\\	複数形チェック
\\	[名詞]	雪崩	
\\	複数形チェック
\\	[動詞]	添付する	
\\	[動詞]	そらす	
\\	[形容詞]	豊富な	
\\	トムは裕福な男だ。	
\\	[形容詞]	畏敬の念に満ちた	
\\	すごいぞ!	
\\	[動詞]	悪化させる	
\\	[名詞]	アナグマ	
\\	アナグマは深い穴を掘る。	
\\	複数形チェック
\\	[名詞]	侵略者	
\\	複数形チェック
\\	[名詞]	バンジョー	
\\	複数形チェック
\\	[動詞]	動揺させる	
\\	[名詞]	野蛮	
\\	己の慣習でないものを、人は野蛮と呼ぶ。	
\\	複数形チェック
\\	[形容詞]	苦しい	
\\	[名詞]	気圧計	
\\	血圧は健康のバロメーターとして重要である。	
\\	複数形チェック
\\	[名詞]	病気	
\\	複数形チェック
\\	[動詞]	物々交換する	
\\	[形容詞]	類似して	
\\	自由は山巓の空気に似ている。どちらも弱い者には堪えることは出来ない。	
\\	[名詞]	戦場	
\\	バットルフィルドというゲームがレンタルサーバー買うことができるぞ。	
\\	複数形チェック
\\	[動詞]	疎外する	
\\	[名詞]	ビートビートの根	
\\	複数形チェック
\\	[名詞]	疎外感	
\\	複数形チェック
\\	[形容詞]	行動の	
\\	[形容詞]	アレルギーの	
\\	僕はグルテンアレルギーだ。	
\\	[名詞]	ボーイ	
\\	複数形チェック
\\	[名詞]	アレルギー	
\\	トムは花粉症がある。	
\\	複数形チェック
\\	[名詞]	ベータ	
\\	アルファ、ベータ、ガンマはギリシャ文字です。	
\\	複数形チェック
\\	[名詞]	割り当て	
\\	複数形チェック
\\	[名詞]	ビリヤード	
\\	複数形チェック
\\	[動詞]	割り当てる	
\\	[名詞]	容器、貯蔵所	
\\	アンインストール方法:フォルダごとごみ箱にぽいっ。	
\\	複数形チェック
\\	[形容詞]	離れて	
\\	彼はぽつんと一人立っていた。	
\\	[名詞]	バインダー	
\\	複数形チェック
\\	[形容詞]	高山の	
\\	[名詞]	生物工学	
\\	「バイオテクノロジー」という言葉が普通に使われるようになったのは、いつ頃ですか。	
\\	複数形チェック
\\	[副詞]	交替で	
\\	[名詞]	あばずれ女	
\\	この野郎!	
\\	複数形チェック
\\	[名詞]	快適さ	
\\	複数形チェック
\\	[形容詞]	奇怪な	
\\	これは非常に奇怪な動物です。	
\\	[形容詞]	親しみやすい	
\\	彼女はだれにでも愛想がいい。	
\\	[動詞]	黑くする	
\\	何をどうやったら、鍋がこんなに真っ黒焦げになるわけ?	
\\	[動詞]	拡大する	
\\	[名詞]	大吹雪	
\\	吹雪でえらい目に遭った。	
\\	複数形チェック
\\	[形容詞]	似ている	
\\	君の立場は僕の立場に似ている。	
\\	[形容詞]	花が咲いている	
\\	「花が咲いているか」と聞いた。	
\\	[名詞]	無政府主義者	
\\	カーチャは無政府主義者であったことを否認し、政治体制に変革をもたらすことを望んだだけで、破壊しようとしたわけではない、と主張し続けた。	
\\	複数形チェック
\\	[名詞]	下宿人	
\\	複数形チェック
\\	[名詞]	政治的混乱、無秩序	
\\	複数形チェック
\\	[名詞]	搭乗	
\\	メアリーは搭乗券を失くした。	
\\	複数形チェック
\\	[名詞]	逸話、秘話	
\\	複数形チェック
\\	[動詞]	動きが取れない	
\\	外出禁止とか、その他のありふれた罰を与えても、彼の息子には全く通じなかった。	
\\	[形容詞]	天使のような	
\\	[名詞]	爆撃手	
\\	自爆テロ犯はベナジル・ブット元首相に向けて発砲したが、弾はそれた。	
\\	複数形チェック
\\	[動詞]	生命を与える	
\\	[名詞]	苦役、強制労役	
\\	彼女のフェチが緊縛と支配でしょう。	
\\	複数形チェック
\\	[形容詞]	生きているような	
\\	[名詞]	たき火	
\\	複数形チェック
\\	[名詞]	古いこと	
\\	複数形チェック
\\	[名詞]	恩恵、恵み	
\\	複数形チェック
\\	[名詞]	植物学者	
\\	複数形チェック
\\	[名詞]	コカコーラ	
\\	コークをください。	
\\	複数形チェック
\\	[形容詞]	底なしの	
\\	彼は奈落の底に落ちた。	
\\	[動詞]	協力する	
\\	[名詞]	大きな丸い石	
\\	里さんは岩の上に立っていました。	
\\	複数形チェック
\\	[名詞]	協力	
\\	複数形チェック
\\	[名詞]	大通り	
\\	複数形チェック
\\	[名詞]	コロン	
\\	複数形チェック
\\	[名詞]	ボウリングをする人	
\\	複数形チェック
\\	[動詞]	記念となる	
\\	建国記念の日(2月11日):建国をしのび、国を愛する心を養う。	
\\	[名詞]	ぬか	
\\	複数形チェック
\\	[名詞]	営利主義	
\\	私はハリウッド映画の商業主義に懸念を抱かずにはいられない。	
\\	複数形チェック
\\	[名詞]	飼育者	
\\	複数形チェック
\\	[動詞]	親しく交わる	
\\	[名詞]	繁殖教養	
\\	複数形チェック
\\	[名詞]	伝達者	
\\	複数形チェック
\\	[名詞]	簡潔な説明	
\\	複数形チェック
\\	[形容詞]	情け深い	
\\	[名詞]	放送者	
\\	複数形チェック
\\	[形容詞]	自己満足の	
\\	[動詞]	拾い読みする	
\\	[形容詞]	わかりやすい	
\\	[名詞]	ドル	
\\	1ドル貸してくれないか。	
\\	複数形チェック
\\	[名詞]	衝動	
\\	複数形チェック
\\	[形容詞]	煩わしい	
\\	[形容詞]	強迫観念にとらわれた	
\\	[動詞] [不規則動詞]	破産する	
\\	[形容詞]	コンピュータの	
\\	[名詞]	執事	
\\	使用人がしらは声を上げて「スミスご夫妻がおいでになりました」と告げた。	
\\	複数形チェック
\\	[形容詞]	うぬぼれの強い	
\\	トムはちょっと自信過剰だ。	
\\	[名詞]	映画・テレビの撮影技師	
\\	戦場に赴くカメラマンが不発弾の危険性を知らないのは不思議だ、新聞社は教育を怠ってる。	
\\	複数形チェック
\\	[名詞]	非難	
\\	複数形チェック
\\	[動詞]	偽装を施す	
\\	[名詞]	空気調節装置	
\\	エアコンがないんですか。	
\\	複数形チェック
\\	[名詞]	戒律、規範	
\\	複数形チェック
\\	[名詞]	調整、条件付け	
\\	私は夏はエアコンなしで済ますことはできない。	
\\	複数形チェック
\\	[動詞]	魅了する	
\\	僕は彼女の美貌のとりこになった。	
\\	[名詞]	服従、従順	
\\	複数形チェック
\\	[形容詞]	心配のない	
\\	のんきな人が変わりやすいものだ。	
\\	[動詞]	混乱させる	
\\	[名詞]	管理人	
\\	向かいの保育園から、保育士だろう男の声で「ガオー、食べちゃうぞ!」というのが聞こえて来る。	
\\	複数形チェック
\\	[形容詞]	気性の合った	
\\	[名詞]	喜びの歌	
\\	キャロルはスペイン語を勉強している。	
\\	複数形チェック
\\	[名詞]	混雑	
\\	複数形チェック
\\	[名詞]	鯉	
\\	鯉がおたまじゃくしを食べるって知ってた?	
\\	複数形チェック
\\	[名詞]	推測判読	
\\	複数形チェック
\\	[名詞]	鋳物	
\\	イエスは、ガリラヤ湖のほとりを歩いておられたとき、シモンとシモンの兄弟アンデレが湖で網を打っているのをご覧になった。	
\\	複数形チェック
\\	[形容詞]	連結した連続した	
\\	すべて関係がある。	
\\	[名詞]	イモムシ、キャタピラー	
\\	いつの日かこの芋虫は美しい蝶へと変化することでしょう。	
\\	複数形チェック
\\	[名詞]	慰め	
\\	誰も死なかったのは不幸中の幸いでした。	
\\	複数形チェック
\\	[名詞]	カリフラワー	
\\	ブロッコリーは好きですが、カリフラワーは大嫌いです。	
\\	複数形チェック
\\	[名詞]	治安官巡査	
\\	複数形チェック
\\	[形容詞]	絶え間のない	
\\	[名詞]	選挙⺠	
\\	複数形チェック
\\	[名詞]	ヒマラヤスギ	
\\	複数形チェック
\\	[形容詞]	憲法制定の	
\\	[形容詞]	天体の	
\\	僕は神世界の神になる。	
\\	[名詞]	汚染物質	
\\	複数形チェック
\\	[名詞]	検閲官	
\\	最初に
\\	を検閲する国はどこかな。	
\\	複数形チェック
\\	[名詞]	汚染	
\\	ホウレンソウと春菊から放射性物質が検出されました。	
\\	複数形チェック
\\	[名詞]	検閲検閲制度	
\\	検閲とは:真実を消し去ることである。	
\\	複数形チェック
\\	[形容詞]	人をばかにした	
\\	[動詞]	中心に集まる	
\\	[名詞]	論点	
\\	複数形チェック
\\	[名詞]	シャンパン	
\\	シャンペンを下さい。	
\\	複数形チェック
\\	[名詞]	論争者競争相手	
\\	複数形チェック
\\	[名詞]	歌聖歌	
\\	複数形チェック
\\	[形容詞]	等高を示す	
\\	[名詞]	運転手	
\\	トムはメアリーのお抱え運転手として何年もずっと働いている。	
\\	複数形チェック
\\	[形容詞]	矛盾した	
\\	例えば、たばこをぷかぷか吸う先生が、たばこを吸う生徒に注意してもその注意は矛盾しているだろう。	
\\	[形容詞]	生意気な	
\\	[形容詞]	逆の	
\\	聾者は手話で会話ができる。	
\\	[形容詞]	上機嫌の陽気な	
\\	[副詞]	逆に	
\\	特定の企業を賞賛する書き込みや、逆にライバル企業を貶める書き込みが多い2
\\	[名詞]	出産	
\\	多くの女性がより高い教養とキャリアを追求し、それ故に結婚と出産を先延ばしにしている。	
\\	複数形チェック
\\	[形容詞]	子供のない	
\\	[形容詞]	宇宙の	
\\	[名詞]	コレステロール	
\\	コレステロール値が高いのです。	
\\	複数形チェック
\\	[名詞]	臆病	
\\	複数形チェック
\\	[形容詞]	慢性の、常習の	
\\	慢性の皮膚炎があります。	
\\	[動詞]	切望する	
\\	急激にかりかりした鳥皮が食べたくなった。	
\\	[名詞]	年代記、記録	
\\	複数形チェック
\\	[名詞]	信用信頼性	
\\	トムは信用を失った。	
\\	複数形チェック
\\	[名詞]	かなりの量	
\\	昨夜、私はもどしてしまった。	
\\	複数形チェック
\\	[名詞]	三日月	
\\	5月8日は世界赤十字デーです。	
\\	複数形チェック
\\	[形容詞]	状況の	
\\	状況証拠としちゃあ、十分だね。	
\\	[名詞]	鳥のとさか	
\\	ランカスター家の紋章が赤バラだったか、白バラだったか忘れてしまった。	
\\	複数形チェック
\\	[名詞]	清潔	
\\	妻は潔癖症だ。	
\\	複数形チェック
\\	[形容詞]	あかね色の	
\\	[名詞]	清算、在庫一掃	
\\	複数形チェック
\\	[動詞]	崩壊する、砕ける	
\\	最近、睡眠不足だったから、お肌がボロボロ。	
\\	[動詞]	詰まらせる	
\\	木靴を履くのが好きだ。	
\\	[動詞]	バリバリかむ	
\\	私たちには砂利を踏みしめる足音が聞こえた。	
\\	[名詞]	海岸線	
\\	複数形チェック
\\	[動詞]	寄り添う	
\\	[名詞]	コカイン	
\\	複数形チェック
\\	[名詞]	料理	
\\	中華料理が好きですか。	
\\	複数形チェック
\\	[名詞]	首尾一貫性	
\\	複数形チェック
\\	[形容詞]	新興宗教の	
\\	[名詞]	耕作者	
\\	この耕運機が百万円もしたと聞いたときはびっくりしたよ。	
\\	複数形チェック
\\	[名詞]	歪み	
\\	複数形チェック
\\	[名詞]	縁石	
\\	痛みを抑える薬を何かください。	
\\	複数形チェック
\\	[形容詞]	乱れた取り乱した	
\\	カミーユはアランのうわの空の態度に不満だ。	
\\	[名詞]	門限	
\\	門限がありますか。	
\\	複数形チェック
\\	[動詞]	分岐する	
\\	[形容詞]	あざ笑って	
\\	[名詞]	教義	
\\	複数形チェック
\\	[動詞]	道楽半分で手を出す	
\\	[名詞]	衰退	
\\	彼の傲慢が凋落に繋がった。	
\\	複数形チェック
\\	[名詞]	ラッパズイセン	
\\	複数形チェック
\\	[形容詞]	下流に	
\\	[形容詞]	いまいましい	
\\	[動詞]	居眠りする	
\\	彼は居眠りをした。	
\\	[動詞]	つくろう	
\\	しまった!	
\\	[名詞]	空想家	
\\	複数形チェック
\\	[動詞]	ぼう然とさせる	
\\	[形容詞]	物寂しい	
\\	そして、つい先日迄、暗い感じだったのに、どこにもかしこにも花、花、花。	
\\	[形容詞]	論争の余地がある	
\\	[動詞]	うなだれる	
\\	[形容詞]	人をだますような	
\\	[名詞]	干ばつ	
\\	日照り続きが収穫にたいへんな損害を与えた。	
\\	複数形チェック
\\	[名詞]	法令	
\\	複数形チェック
\\	[動詞]	吹き替える	
\\	[動詞]	差し引く	
\\	[形容詞]	半信半疑の、疑わしい	
\\	トムは怪訝そうな顔をしている。	
\\	[名詞]	控除	
\\	複数形チェック
\\	[名詞]	公爵夫人	
\\	複数形チェック
\\	[形容詞]	無防備の	
\\	彼女がとった行動はあまりにも軽率で無防備だった。	
\\	[形容詞]	複製の	
\\	[動詞]	ゆがめる	
\\	[名詞]	王朝	
\\	チューダー朝は1603年に崩壊した。	
\\	複数形チェック
\\	[動詞]	品位を落とす	
\\	何人も、拷問又は残虐な、非人道的な若しくは屈辱的な取扱若しくは刑罰を受けることはない。	
\\	[名詞]	ミミズ	
\\	複数形チェック
\\	[動詞]	表示する	
\\	何も元を持たない集合は空集合と呼ばれ、
\\	と表される。	
\\	[名詞]	引き潮	
\\	複数形チェック
\\	[動詞]	非難する	
\\	[名詞]	食	
\\	昨日の月食見た?	
\\	複数形チェック
\\	[名詞]	堆積物	
\\	複数形チェック
\\	[形容詞]	有頂天の	
\\	[名詞]	貯蔵所、倉庫	
\\	複数形チェック
\\	[形容詞]	努力を要しない	
\\	[名詞]	指示	
\\	北朝鮮が6か国協議の合意に基づき核開発計画を申告した26日、米国が「テロ支援国」の指定解除手続きに入ったことで、拉致被害者の家族らには「拉致問題が置き去りにされるのでは」という不安が広がった。	
\\	複数形チェック
\\	[名詞]	利己主義者	
\\	複数形チェック
\\	[形容詞]	望んで	
\\	[形容詞]	利己主義の	
\\	[動詞]	留置する	
\\	トムは警察に逮捕されました。	
\\	[名詞]	手の込んでいること	
\\	複数形チェック
\\	[名詞]	発見	
\\	複数形チェック
\\	[名詞]	弾力	
\\	複数形チェック
\\	[動詞]	阻止する	
\\	[動詞]	電化する	
\\	[動詞]	悪化させる	
\\	描画が故障して。	
\\	[形容詞]	自然の力による	
\\	[動詞]	通貨安になる	
\\	[動詞]	具体化する	
\\	[形容詞]	欠けている	
\\	その部屋には家具が無かった。	
\\	[名詞]	移⺠	
\\	複数形チェック
\\	[名詞]	熱愛者	
\\	複数形チェック
\\	[動詞]	能力を与える	
\\	[名詞]	糖尿病	
\\	11月14日は世界糖尿病デーです。	
\\	複数形チェック
\\	[名詞]	皇后女帝	
\\	皇后はオーストラリアを訪問中である。	
\\	複数形チェック
\\	[名詞]	独裁政治	
\\	ミャンマーは軍事独裁政権に支配されている。	
\\	複数形チェック
\\	[動詞]	取り巻く	
\\	[形容詞]	格差の	
\\	微分方程式を解くというのは、それを満足するような関数を見つけ出すということだ。	
\\	[名詞]	承認、裏書	
\\	我々はそれほどお金も多くなく、支持もないところから始まった。	
\\	複数形チェック
\\	[動詞]	拡散させる	
\\	[名詞]	拡大	
\\	基本的にサイトや雑誌等で販売しているペニス増大サプリメントは効き目がありません。	
\\	複数形チェック
\\	[名詞]	アラビア数字	
\\	複数形チェック
\\	[動詞]	教化する啓蒙する	
\\	私たちに教えてくれませんか。	
\\	[動詞]	水で薄める	
\\	[名詞]	憎しみ、悪意	
\\	複数形チェック
\\	[動詞]	鳴り響く	
\\	食堂車はどこにありますか。	
\\	[動詞]	激怒させる	
\\	[名詞]	二酸化物	
\\	地球温暖化は二酸化炭素の排出と直接関係があると言われている。	
\\	複数形チェック
\\	[名詞]	強化	
\\	複数形チェック
\\	[名詞]	情報公開	
\\	複数形チェック
\\	[名詞]	登録	
\\	複数形チェック
\\	[動詞]	接続を断つ	
\\	[動詞]	そそのかしてさせる	
\\	[名詞]	不和、不一致	
\\	夫との間に波風が立つのが恐くて、愛人の存在は知らない振りをしています。	
\\	複数形チェック
\\	[動詞]	委託する	
\\	彼女は別れた夫に赤ちゃんを任せた。	
\\	[名詞]	講話、論説	
\\	複数形チェック
\\	[名詞]	環境保護論者	
\\	複数形チェック
\\	[名詞]	発見者	
\\	複数形チェック
\\	[名詞]	大作	
\\	複数形チェック
\\	[形容詞]	分別のある	
\\	[形容詞]	誤った	
\\	温暖化を否定する人と論争すれば、根本的な問題は、その人が非論理的に主張をすることです。	
\\	[動詞]	美を損なう	
\\	[動詞]	噴火する	
\\	その火山は突然噴火し、多くの人が亡くなった。	
\\	[形容詞]	従順でない	
\\	[名詞]	倫理	
\\	倫理委員会では, クローン実験の一時停止が決定した。	
\\	複数形チェック
\\	[形容詞]	秩序を乱す	
\\	その子最近授業中遊び倒しているので、正直今後が心配。	
\\	[動詞]	避難する	
\\	警報が鳴り響き、全員が立ち退かなければならなかった。	
\\	[名詞]	不賛成、異議	
\\	複数形チェック
\\	[名詞]	避難	
\\	複数形チェック
\\	[名詞]	嫌うこと	
\\	複数形チェック
\\	[動詞]	逃れる回避する	
\\	彼は脱税で非難された。	
\\	[動詞]	蒸留する	
\\	[動詞]	蒸発する	
\\	露は太陽が昇ると蒸発した。	
\\	[形容詞]	出来事の多い	
\\	[動詞]	きらきら輝く	
\\	[名詞]	常緑樹	
\\	複数形チェック
\\	[名詞]	教母女保護者	
\\	複数形チェック
\\	[名詞]	喚起する	
\\	複数形チェック
\\	[名詞]	峡谷	
\\	複数形チェック
\\	[形容詞]	進化の	
\\	[名詞]	祖父	
\\	その間ずっと、じいさんの言ったことが、メロンの周りを飛び回る蠅の羽音のように、頭の中でぶんぶんと鳴っていました。	
\\	複数形チェック
\\	[形容詞]	非常な、すばらしい	
\\	その車は制限速度を超過している。	
\\	[名詞]	壮麗壮観	
\\	複数形チェック
\\	[名詞]	抜粋	
\\	複数形チェック
\\	[名詞]	バッタ	
\\	複数形チェック
\\	[動詞]	免除する	
\\	[名詞]	満足すること	
\\	複数形チェック
\\	[名詞]	免除	
\\	「エグゼンプション」の議論を機に働き方を見直そう。	
\\	複数形チェック
\\	[名詞]	砂利	
\\	私たちには砂利を踏みしめる足音が聞こえた。	
\\	複数形チェック
\\	[名詞]	努力	
\\	複数形チェック
\\	[名詞]	墓石	
\\	複数形チェック
\\	[形容詞]	⻑期にわたる	
\\	彼は右手を延ばした。	
\\	[名詞]	肉汁	
\\	複数形チェック
\\	[名詞]	絶滅	
\\	これらの動物は、絶滅の危機に瀕している。	
\\	複数形チェック
\\	[形容詞]	異様な	
\\	私は昨夜ニユース映画で英国の女兵隊といふものをみたが、なかには男か女かわからないやうなのもゐたけれども、大部分は、なかなか女らしいところがあつて、しかもさうグロテスクな感じはしなかつた。	
\\	[名詞]	浪費	
\\	複数形チェック
\\	[名詞]	火薬	
\\	複数形チェック
\\	[形容詞]	信じられないほど	
\\	信じられないくらいうまいね。	
\\	[名詞]	突風	
\\	複数形チェック
\\	[形容詞]	不誠実な	
\\	[名詞]	樋排水溝	
\\	5月18日、1歳になる彼らの子供がプラスティック袋に包まれて排水溝に棄てられているのが見つかり、若い日本人カップルが逮捕された。	
\\	複数形チェック
\\	[名詞]	祖国	
\\	私は祖国を愛している。	
\\	複数形チェック
\\	[名詞]	驚き・悲しみ・喜びを表す	
\\	わははは!そりゃ傑作だ!	
\\	複数形チェック
\\	[名詞]	偉業	
\\	複数形チェック
\\	[形容詞]	手作りの	
\\	この手作りのイタリア製チタン自転車は、恐ろしく軽い。	
\\	[形容詞]	封建制の	
\\	前世紀の中頃まで日本が封建国家であったことは、よくしられている。	
\\	[動詞]	悩ます	
\\	我が軍は絶えずゲリラの攻撃を受けた。	
\\	[形容詞]	作り事の	
\\	[動詞]	馬具でつなぐ	
\\	利用すべきエネルギー、つくり出すべき雇用、建てるべき学校、対話すべき脅威、修復すべき同盟がある。	
\\	[名詞]	指先	
\\	指先を火傷しました。	
\\	複数形チェック
\\	[名詞]	避難所	
\\	まだ決めていません。	
\\	複数形チェック
\\	[形容詞]	怪しい	
\\	俺は締め出されてるんだぞ?後ろ暗いことがあるに決まってる。	
\\	[形容詞]	危険な	
\\	[名詞]	定着物備品	
\\	複数形チェック
\\	[名詞]	霞	
\\	もやがロンドンの町をすっぽりと包んだ。	
\\	複数形チェック
\\	[名詞]	勘才能	
\\	子供にお金を残そうなんて考えずにパーッと派手に遊んで使い切って下さいね。	
\\	複数形チェック
\\	[形容詞]	かすんだ	
\\	[名詞]	閃光	
\\	複数形チェック
\\	[名詞]	頭飾り	
\\	複数形チェック
\\	[名詞]	不備、欠陥	
\\	不足しているものを数えるな、恵まれたものを数えてみよう。	
\\	複数形チェック
\\	[名詞]	ヒース	
\\	複数形チェック
\\	[名詞]	ノミ	
\\	明日フリーマーケットに行こうよ!	
\\	複数形チェック
\\	[形容詞]	熱狂的な	
\\	[形容詞]	流動的な	
\\	ボールが川を流れている。	
\\	[名詞]	相続人	
\\	複数形チェック
\\	[名詞]	脚注	
\\	5ページ脚注を参照してください。	
\\	複数形チェック
\\	[名詞]	先駆者先触れ	
\\	複数形チェック
\\	[形容詞]	力ずくの	
\\	[名詞]	遺伝	
\\	我々は環境と遺伝の両方の影響を受けている。	
\\	複数形チェック
\\	[形容詞]	前方の	
\\	[名詞]	英雄的精神	
\\	複数形チェック
\\	[動詞]	あらかじめ武装する	
\\	転ばぬ先の杖。	
\\	[名詞]	ニシン	
\\	複数形チェック
\\	[名詞]	人差し指	
\\	彼女はその箱を人差し指でとんとんとたたいた。	
\\	複数形チェック
\\	[名詞]	高位殿下	
\\	複数形チェック
\\	[名詞]	前景	
\\	複数形チェック
\\	[名詞]	ヒッピー	
\\	複数形チェック
\\	[形容詞]	予見できる	
\\	[名詞]	くわ	
\\	複数形チェック
\\	[動詞] [不規則動詞]	予言する	
\\	[名詞]	セイヨウヒイラギ	
\\	複数形チェック
\\	[形容詞]	香りのよい	
\\	この花は美しいだけでなく、よい香りがします。	
\\	[名詞]	大虐殺	
\\	複数形チェック
\\	[名詞]	変人ファン	
\\	あの蜘蛛が大嫌い。いつもそこにいて私が掃除するときいらいらさせるの。	
\\	複数形チェック
\\	[名詞]	同性愛	
\\	複数形チェック
\\	[形容詞]	非常に寒い	
\\	すごく寒いです。	
\\	[名詞]	自動車の警笛	
\\	トムはクラクションを鳴らし続けた。	
\\	複数形チェック
\\	[動詞]	さっぱりした気持ちにさせる	
\\	[動詞]	叫ぶ、やじる	
\\	[形容詞]	取るに足りない	
\\	[名詞]	ホルモン	
\\	女性ホルモン分泌の乱れが不妊症の大きな原因です。	
\\	複数形チェック
\\	[形容詞]	正面の	
\\	8歳未満の子供は前頭葉が発達しきっておらず、そのため現実と空想の区別が付かなくなる場合がある。一部の子どもは、例えば洋服ダンスの中やベッドの下に化け物がいると信じることがあるかもしれない。彼らの一部はまた、夢と現実の区別ができないこともある。	
\\	[形容詞]	恐ろしい	
\\	[形容詞]	倹約な	
\\	[名詞]	ホスピス	
\\	複数形チェック
\\	[名詞]	融合	
\\	複数形チェック
\\	[形容詞]	もてなしのよい	
\\	[形容詞]	無駄な、効果のない	
\\	無駄です。	
\\	[名詞]	空に舞うこと	
\\	人間の体温は37度付近で上下している。	
\\	複数形チェック
\\	[副詞]	陽気に	
\\	[形容詞]	思いやりのある	
\\	知識そのものは目的ではなく、充実した人間らしい生活を送るという目的のための手段である。	
\\	[名詞]	発電機発生器	
\\	複数形チェック
\\	[名詞]	人文学者	
\\	複数形チェック
\\	[形容詞]	一般的な	
\\	[名詞]	衛生	
\\	複数形チェック
\\	[名詞]	紳士階級	
\\	複数形チェック
\\	[名詞]	偽善者	
\\	率直に言えば、彼は愛国者というよりはむしろ偽善者だ。	
\\	複数形チェック
\\	[形容詞]	地質学の	
\\	[名詞]	偶像、アイコン	
\\	の中止ボタンがしいたけに見えて困る。	
\\	複数形チェック
\\	[名詞]	病原菌	
\\	精密機械などの除菌には電解した水が使われています。	
\\	複数形チェック
\\	[名詞]	理想主義者	
\\	私は理想主義者だ。自分がどこに向かっているのかは分からないが、どこかに向かっているのは確かだ。	
\\	複数形チェック
\\	[形容詞]	才能のある	
\\	有難くも、10000
\\	越えのお祝いイラストを頂いてしまいました!本当にかたじけないです。	
\\	[副詞]	無益に、無為に	
\\	彼女はぼんやりと窓の外を眺めていた。	
\\	[形容詞]	違法の	
\\	まず彼は未婚の母から生まれた私生児である。	
\\	[形容詞]	耐えられない	
\\	[名詞]	無学	
\\	複数形チェック
\\	[名詞]	複雑	
\\	複数形チェック
\\	[形容詞]	不合理な	
\\	全く筋が通っていない。私は酔っぱらって外で叫んでいたわけじゃない。	
\\	[動詞]	陰謀を企てる	
\\	それは面白い。	
\\	[形容詞]	測ることのできない	
\\	[形容詞]	本来備わっている	
\\	[形容詞]	免疫の	
\\	このビタミンは病気に対する抵抗力を強めます。	
\\	[形容詞]	直覚的な	
\\	[名詞]	免疫	
\\	複数形チェック
\\	[名詞]	在庫品	
\\	複数形チェック
\\	[動詞]	損なう、弱める	
\\	[名詞]	不適切、的外れ	
\\	複数形チェック
\\	[動詞]	弾劾する	
\\	[形容詞]	抵抗できない	
\\	あなたは抵抗できない。	
\\	[名詞]	不完全欠点	
\\	複数形チェック
\\	[動詞]	水を引く	
\\	[名詞]	帝国主義	
\\	複数形チェック
\\	[名詞]	灌漑	
\\	複数形チェック
\\	[形容詞]	無気力な	
\\	[名詞]	イタリック体	
\\	複数形チェック
\\	[形容詞]	軽率な	
\\	[名詞]	起重機、ジャッキ、押上げ万力	
\\	ジャックは切手を集めている。	
\\	複数形チェック
\\	[形容詞]	ふさわしくない	
\\	[名詞]	大当たり	
\\	くじは私に当たった。	
\\	複数形チェック
\\	[形容詞]	生まれつきの	
\\	[名詞]	清掃作業員	
\\	複数形チェック
\\	[形容詞]	偶然の、偶発の	
\\	[名詞]	冗談	
\\	冗談に言ったことが本当になる場合がたくさんある。	
\\	複数形チェック
\\	[形容詞]	一貫しない	
\\	[名詞]	糸のこ	
\\	比較的難しいジグソーだと感じました。	
\\	複数形チェック
\\	[形容詞]	両立しない	
\\	[形容詞]	関節のある	
\\	[形容詞]	想像も及ばない	
\\	彼がそんなことをするなんて私には考えられない。	
\\	[動詞]	書き留めておく	
\\	[形容詞]	決定的でない	
\\	[名詞]	司法裁判官	
\\	複数形チェック
\\	[名詞]	不一致	
\\	複数形チェック
\\	[動詞]	空中に投げあげて受ける	
\\	ジャグリングのやり方がわかりますか。	
\\	[動詞]	(損害、傷などを)被る	
\\	[形容詞]	正当と認められる	
\\	[形容詞]	下品な	
\\	[名詞]	腎臓	
\\	トムは腎臓と肝臓が悪い。	
\\	複数形チェック
\\	[副詞]	無期限に	
\\	宇宙は無限に膨張し続けるのか?	
\\	[動詞]	火をつける	
\\	大学時代の友達がもう使っていないというので
\\	をくれた。
\\	という画面の大きいモデルだ。操作性は良くないが、画面に関してはなるほどいいもんだ。今まではそう気にならなかったが、もはや普通の液晶画面で物を読むのが辛い体になってしまった。	
\\	[名詞]	個人主義利己主義	
\\	複数形チェック
\\	[名詞]	売店	
\\	複数形チェック
\\	[名詞]	非能率	
\\	複数形チェック
\\	[形容詞]	打撃が猛烈な	
\\	[形容詞]	逃げられない	
\\	[形容詞]	骨の折れる	
\\	[形容詞]	説明のつかない	
\\	バーミューダトライアングルでの失踪は不可解です。	
\\	[名詞]	街灯柱	
\\	複数形チェック
\\	[動詞]	推測する	
\\	天文学者が局部超銀河団の存在を推定しました。	
\\	[名詞]	レーザー	
\\	私のレーザープリンターは白黒印刷しかできない。	
\\	複数形チェック
\\	[名詞]	推論	
\\	複数形チェック
\\	[名詞]	掛け金	
\\	複数形チェック
\\	[名詞]	燃焼炎症	
\\	子宮内膜炎は、細菌が子宮内に入り、内膜に炎症を起す病気である。	
\\	複数形チェック
\\	[形容詞]	気前の良い	
\\	[名詞]	基盤となる設備	
\\	このブログは情報インフラを中心としたトピックを取り上げています。	
\\	複数形チェック
\\	[名詞]	素人	
\\	複数形チェック
\\	[形容詞]	まれな	
\\	[動詞]	解雇する	
\\	謝ったんだから、さっきのはチャラにしてよ。	
\\	[動詞]	激怒させる	
\\	[名詞]	棚、突起物	
\\	複数形チェック
\\	[動詞]	吹き込む	
\\	[名詞]	立法府	
\\	亡命チベット代表者議会は、亡命チベット人社会で最高の権限を持つ立法機関で、1960年に設立された。	
\\	複数形チェック
\\	[名詞]	発明の才	
\\	複数形チェック
\\	[名詞]	正当性	
\\	彼にはまったく正当性がない。	
\\	複数形チェック
\\	[形容詞]	非人間的な	
\\	何人も、拷問又は残虐な、非人道的な若しくは屈辱的な取扱若しくは刑罰を受けることはない。	
\\	[形容詞]	⻑い	
\\	長い和平交渉を経て、現在の平和がある。	
\\	[名詞]	イニング	
\\	複数形チェック
\\	[名詞]	同性愛の女性	
\\	私は、レズです。	
\\	複数形チェック
\\	[形容詞]	不安定な	
\\	彼女が自身がない。	
\\	[形容詞]	致命的な	
\\	適切にアースされていないマイクに触れると、命に関わる場合があります。何人かの歌手は実際そうやって亡くなっているのです。	
\\	[名詞]	不安	
\\	複数形チェック
\\	[名詞]	自由主義	
\\	複数形チェック
\\	[形容詞]	感受性のない	
\\	トムって無神経だと思う。	
\\	[形容詞]	生きているような	
\\	そのロボットはあまりにも真に迫りすぎて気持ち悪かった。	
\\	[形容詞]	誠意のない	
\\	[形容詞]	陽気な	
\\	[形容詞]	即時の	
\\	[名詞]	石灰	
\\	その女は出てきたジンライムを一口で飲み干した。	
\\	複数形チェック
\\	[名詞]	合間	
\\	複数形チェック
\\	[名詞]	言語学者	
\\	複数形チェック
\\	[名詞]	休止	
\\	複数形チェック
\\	[名詞]	言語学	
\\	私は言語学専攻です。	
\\	複数形チェック
\\	[名詞]	実習生、インターン	
\\	複数形チェック
\\	[名詞]	雌ライオン	
\\	複数形チェック
\\	[動詞]	尋問する	
\\	[形容詞]	荷を積んだ	
\\	老人はラバに砂のいっぱい入った袋をのせた。	
\\	[名詞]	インタビューされる人	
\\	複数形チェック
\\	[名詞]	荷積み	
\\	彼らは石油を船に積み込んでいる。	
\\	複数形チェック
\\	[名詞]	腸	
\\	複数形チェック
\\	[動詞]	限定する	
\\	[動詞]	怖がらせる	
\\	あの人は、見た目はこわそうだけど、フレンドリーな人ですよ。心配しないで。	
\\	[名詞]	寿命	
\\	複数形チェック
\\	[名詞]	不寛容	
\\	複数形チェック
\\	[名詞]	支配権	
\\	複数形チェック
\\	[形容詞]	愛らしい	
\\	彼は愛すべき男だ。	
\\	[名詞]	士気	
\\	でも、それで士気が下がっては本末転倒ではないかと。	
\\	複数形チェック
\\	[名詞]	忠臣	
\\	官軍に入城を許すとはどういう了見だ。	
\\	複数形チェック
\\	[名詞]	モザイクモザイク画	
\\	局部には必ずモザイクをかけて下さい。	
\\	複数形チェック
\\	[形容詞]	儲かる	
\\	[名詞]	母性	
\\	複数形チェック
\\	[形容詞]	なまぬるい	
\\	ぬるいカルピス甘いだけだが、氷入れれば酸味出る。	
\\	[名詞]	主題動機	
\\	複数形チェック
\\	[形容詞]	感情豊かな、興奮的な	
\\	恋愛はただ性欲の詩的表現をうけたものである。	
\\	[名詞]	登山	
\\	トムは登山に興味がある。	
\\	複数形チェック
\\	[動詞]	激怒させる	
\\	[形容詞]	悲しみに沈んだ	
\\	[形容詞]	悪意のある	
\\	悪口雑言をいわれても、いわれた人の価値が下がるわけではない。	
\\	[名詞]	マフィン	
\\	あなたは朝食にマフィンを食べている。	
\\	複数形チェック
\\	[動詞]	巧みに操縦する	
\\	[形容詞]	多国籍の	
\\	[名詞]	具体的な公約	
\\	複数形チェック
\\	[動詞]	むしゃむしゃ食べる	
\\	ビールの摘まみがほしい。	
\\	[動詞]	操作する	
\\	トムは他人を操る癖のある者が気に食わない。	
\\	[形容詞]	筋肉の	
\\	[名詞]	操作	
\\	複数形チェック
\\	[名詞]	無数	
\\	複数形チェック
\\	[名詞]	雌馬雌ロバ	
\\	棹立ちした栗毛の牝馬は嘶き、騎手を落とそうとした。	
\\	複数形チェック
\\	[動詞]	惑わせる	
\\	[名詞]	マリファナ	
\\	複数形チェック
\\	[名詞]	神話	
\\	世界神話の中でも、大洪水の話はとても有名だ。	
\\	複数形チェック
\\	[形容詞]	結婚の	
\\	[形容詞]	経験が少ない	
\\	マリーは素朴な学生だ。	
\\	[形容詞]	戦争の	
\\	トムは武道家です。	
\\	[形容詞]	交渉できる	
\\	[名詞]	大虐殺	
\\	複数形チェック
\\	[名詞]	交渉者	
\\	スタッフに交渉事がうまいのがいます。	
\\	複数形チェック
\\	[形容詞]	物質主義の	
\\	[名詞]	中立	
\\	複数形チェック
\\	[形容詞]	母の	
\\	[動詞]	中和する	
\\	[名詞]	数列、数学の行列	
\\	複数形チェック
\\	[名詞]	切り目、刻み目	
\\	ニックを軽蔑しているんだろう?	
\\	複数形チェック
\\	[名詞]	格言	
\\	複数形チェック
\\	[動詞]	つねる	
\\	[動詞]	最大にする	
\\	[名詞]	標準、規範	
\\	複数形チェック
\\	[名詞]	マヨネーズ	
\\	あのマヨネーズの中、添加物ばっかり!	
\\	複数形チェック
\\	[形容詞]	北東の	
\\	[形容詞]	乏しい	
\\	[形容詞]	北⻄の	
\\	「これはアラビア語ですか?」「いいえ、それはウイグル語といって、中国の北西部で話されている言語です。」	
\\	[名詞]	力学、機械学	
\\	複数形チェック
\\	[形容詞]	目だった顕著な	
\\	[名詞]	中央値	
\\	複数形チェック
\\	[形容詞]	名目の	
\\	[形容詞]	平凡な	
\\	人気者になるためには、凡庸でなくてはならない。	
\\	[形容詞]	悪名高い	
\\	トムは先延ばし癖で有名だ。	
\\	[形容詞]	おとなしい	
\\	基本はアスカがシンジをいじめる。で、たまにキレたシンジがアスカを襲う。その時のアスカはやけにしおらしい。	
\\	[名詞]	核	
\\	複数形チェック
\\	[名詞]	商品化計画	
\\	複数形チェック
\\	[名詞]	裸	
\\	複数形チェック
\\	[名詞]	
\\	の複数形	
\\	複数形チェック
\\	[名詞]	栄養素	
\\	グリーンピースは鉄分が豊富で爪の色がよくなる成分が含まれている。	
\\	複数形チェック
\\	[名詞]	地下鉄	
\\	複数形チェック
\\	[形容詞]	栄養価の高い	
\\	栄養のある朝食をとった。	
\\	[名詞]	週の中ほど	
\\	複数形チェック
\\	[名詞]	客観性	
\\	複数形チェック
\\	[名詞]	画期的な出来事	
\\	複数形チェック
\\	[形容詞]	義務的な	
\\	学校へ行くことが義務でなければ、行くのを止めるのだが。	
\\	[形容詞]	戦闘的な	
\\	[名詞]	忘却	
\\	複数形チェック
\\	[名詞]	模倣者	
\\	複数形チェック
\\	[名詞]	天文台	
\\	複数形チェック
\\	[形容詞]	愚かな頭を使わない	
\\	[名詞]	執着	
\\	複数形チェック
\\	[名詞]	不作法	
\\	複数形チェック
\\	[形容詞]	時代遅れの	
\\	共産主義は失敗し、プロレタリア独裁は時代遅れとなったので、資本主義そのものは、未だ人類を脅かすものであるにもかかわらず、今迄のところ一番良い体制であることがわかる。	
\\	[動詞]	間違って教える	
\\	[動詞]	ふさぐ	
\\	下水管がつまっている。	
\\	[名詞]	守銭奴、けち	
\\	複数形チェック
\\	[名詞]	障害物	
\\	複数形チェック
\\	[名詞]	適応できない人	
\\	複数形チェック
\\	[形容詞]	⻄洋の	
\\	[形容詞]	見当違いの	
\\	[名詞]	占有者	
\\	複数形チェック
\\	[動詞]	用しない	
\\	[名詞]	風変わりなこと	
\\	複数形チェック
\\	[名詞]	誤用	
\\	複数形チェック
\\	[形容詞]	実施されている	
\\	[動詞]	動員する	
\\	[動詞]	選ぶ	
\\	[名詞]	変復調装置、モデム	
\\	複数形チェック
\\	[形容詞]	光学上の	
\\	二重焦点レンズは「バイフォーカル」とも呼ばれる。	
\\	[動詞]	近代化する	
\\	[形容詞]	オーケストラ用の	
\\	[名詞]	勢い	
\\	複数形チェック
\\	[名詞]	試練	
\\	ボブは自分の力でこの厳しい試練を乗り切らねばいけない。	
\\	複数形チェック
\\	[名詞]	独白	
\\	複数形チェック
\\	[名詞]	鉱石	
\\	そこは鉄鉱に富む。	
\\	複数形チェック
\\	[動詞]	独占する	
\\	女の子が父親の愛情を独占したいと思い、母親を競争者とみなしがちであった。	
\\	[名詞]	創立委員	
\\	複数形チェック
\\	[名詞]	単調退屈	
\\	複数形チェック
\\	[動詞]	数でまさる	
\\	[動詞]	停泊させる	
\\	[形容詞]	無法な	
\\	女だてらに自転車に乗るなんてけしからん。女は男の真似はよした方がいい。	
\\	[形容詞]	期日を過ぎた	
\\	列車は30分遅れている。	
\\	[形容詞]	洗練された	
\\	彼女はなかなかあか抜けている。	
\\	[動詞]	過大評価をする	
\\	すみません、自分の実力を過信してました。	
\\	[名詞]	汚染物質	
\\	複数形チェック
\\	[動詞] [不規則動詞]	監督する	
\\	[動詞]	熟考する	
\\	[名詞]	見過ごし監視	
\\	複数形チェック
\\	[形容詞]	人口の多い	
\\	インドは人口が多い。	
\\	[形容詞]	公然の	
\\	[名詞]	ポリッジ米がゆ	
\\	複数形チェック
\\	[形容詞]	すし詰めの	
\\	市電は混んでいた。	
\\	[名詞]	後世子孫	
\\	複数形チェック
\\	[名詞]	条約契約	
\\	複数形チェック
\\	[名詞]	有力な	
\\	初詣、ご利益があるのはどこの神社?	
\\	複数形チェック
\\	[名詞]	異教徒	
\\	自然崇拝者は生命を、キリスト教徒は死を崇めてきました。	
\\	複数形チェック
\\	[名詞]	家禽	
\\	養鶏業は遅かれ早かれ破産するだろうという旨の手紙を彼は私によこした。	
\\	複数形チェック
\\	[形容詞]	骨身を惜しまない	
\\	[名詞]	いたずら	
\\	鈴木さんはトムのいたずらにご立腹だ。	
\\	複数形チェック
\\	[名詞]	窓ガラス	
\\	雨は窓ガラスを叩く。	
\\	複数形チェック
\\	[形容詞]	不安定な	
\\	[名詞]	激痛、苦しみ	
\\	お腹がすくと胃が痛みます。	
\\	複数形チェック
\\	[名詞]	判例、先例	
\\	複数形チェック
\\	[名詞]	寄生虫	
\\	複数形チェック
\\	[動詞]	断定する	
\\	動詞は述語動詞のことです。述語動詞は、主語や表す時によって形を変えます。	
\\	[名詞]	教区	
\\	複数形チェック
\\	[形容詞]	優勢な	
\\	[名詞]	分詞	
\\	進行形の文では、動詞の
\\	形、つまり現在分詞となります。	
\\	複数形チェック
\\	[形容詞]	早産の	
\\	トムは未熟児で生まれた。	
\\	[名詞]	仕切り	
\\	複数形チェック
\\	[名詞]	偏見没頭	
\\	複数形チェック
\\	[形容詞]	通行できる	
\\	及第点よ。これぐらい大したことないわ。	
\\	[形容詞]	予備の	
\\	[名詞]	愛国心	
\\	アメリカほど愛国心を煽る国はあるでしょうか。	
\\	複数形チェック
\\	[名詞]	言い訳	
\\	あいつに因縁を付けられた。	
\\	複数形チェック
\\	[名詞]	泥炭	
\\	複数形チェック
\\	[名詞]	予告、下見	
\\	昨日の夜、特別試写会があった。	
\\	複数形チェック
\\	[名詞]	彫像などの台座	
\\	複数形チェック
\\	[形容詞]	戦前の	
\\	[動詞]	そっとのぞく	
\\	雲の切れ間から太陽が見えます。	
\\	[副詞]	第一に	
\\	[名詞]	ペンギン	
\\	この赤ちゃんペンギンとっても可愛い!	
\\	複数形チェック
\\	[形容詞]	光栄で	
\\	[名詞]	年金受給者	
\\	複数形チェック
\\	[名詞]	プログラマー	
\\	彼女はコンピュータープログラマーですか。	
\\	複数形チェック
\\	[形容詞]	周期的な定期的な	
\\	[名詞]	禁止	
\\	複数形チェック
\\	[動詞]	永存させる	
\\	[形容詞]	傾向がある	
\\	彼は興奮しやすい。	
\\	[動詞]	辛抱する	
\\	何か偉大なことを達成するためにはたゆまず努力しなければならない。	
\\	[名詞]	推進物推進器	
\\	ジェット機はプロペラ機よりもずっと速い。	
\\	複数形チェック
\\	[動詞]	全体に普及する	
\\	[名詞]	条約議定書	
\\	プロトコルだけに固執すればいい、分かった?	
\\	複数形チェック
\\	[形容詞]	ひねくれた	
\\	変なこと言うな!	
\\	[名詞]	見本	
\\	ページが読み込まれたら更新情報をページ内に流し込む 
\\	複数形チェック
\\	[動詞]	悩ます	
\\	彼は彼女に車を運転させるつもりはなかったのだが、彼女があまりにせがむものだから、彼の方がとうとう折れた。	
\\	[名詞]	摂理	
\\	複数形チェック
\\	[名詞]	農薬、殺虫剤	
\\	彼女は花に殺虫剤を振りまいた。	
\\	複数形チェック
\\	[動詞]	詮索する	
\\	トムは15分もかけて引き出しをかなてこでこじ開けようとしたが、結局開けられなかった。	
\\	[名詞]	石油	
\\	石油は古代よりずっと重要なものであった。	
\\	複数形チェック
\\	[名詞]	精神科医	
\\	トムは精神科医をしている。	
\\	複数形チェック
\\	[形容詞]	些細な	
\\	こんな些細なことでガタガタ文句言うな。	
\\	[名詞]	精神医学	
\\	複数形チェック
\\	[形容詞]	驚異的な	
\\	彼は驚異的な力で自動車を持ち上げた。	
\\	[形容詞]	霊魂の	
\\	[形容詞]	音声の	
\\	[動詞]	公表する宣伝する	
\\	[動詞]	コピーする	
\\	[名詞]	水たまり	
\\	黒い小猫がぴょんぴょんと水溜をさけて、隣の生垣の下をくぐった。	
\\	複数形チェック
\\	[名詞]	漬物、ピクルス	
\\	複数形チェック
\\	[名詞]	果肉	
\\	複数形チェック
\\	[形容詞]	間に合わせの	
\\	[動詞]	句読点を付ける、中断する	
\\	[名詞]	信心、敬愛	
\\	複数形チェック
\\	[名詞]	句読点、句読法	
\\	複数形チェック
\\	[名詞]	頂上	
\\	複数形チェック
\\	[動詞]	一掃する	
\\	[形容詞]	正確な	
\\	[動詞]	浄化する	
\\	[名詞]	プラカード	
\\	複数形チェック
\\	[形容詞]	当惑させる	
\\	それは私にややこしいことだった。	
\\	[形容詞]	穏やかで落ち着いた	
\\	[名詞]	水晶	
\\	複数形チェック
\\	[名詞]	飾り額記念額	
\\	複数形チェック
\\	[名詞]	人種差別	
\\	複数形チェック
\\	[名詞]	高原	
\\	離婚率はやがて頭打ちになるでしょう。	
\\	複数形チェック
\\	[名詞]	人種差別主義者	
\\	あんな人種差別の発言をするなんて、あの政治家も取り返しのつかないことを口にしたものだ。	
\\	複数形チェック
\\	[形容詞]	精神的な	
\\	彼氏じゃないわ、利益をともなう純愛よ!	
\\	[形容詞]	明るい光を放つ	
\\	[名詞]	脚本家	
\\	複数形チェック
\\	[形容詞]	放射性の	
\\	放射性廃棄物の問題は未解決のままである。	
\\	[名詞]	羽	
\\	複数形チェック
\\	[名詞]	半径	
\\	複数形チェック
\\	[動詞]	まっすぐに落ちる	
\\	[名詞]	傾斜路	
\\	25号線北方面の入り口付近にいます。	
\\	複数形チェック
\\	[動詞]	浴びせる	
\\	[名詞]	ならず者	
\\	複数形チェック
\\	[形容詞]	要領を得ない	
\\	愛のない人生など全く無意味だ。	
\\	[名詞]	略奪荒廃	
\\	しばしば、それは狩人の小屋を壊し、罠から獲物を盗む。	
\\	複数形チェック
\\	[動詞]	突く	
\\	先生は何かと言うと人のことに口出しする。	
\\	[形容詞]	読みやすい	
\\	もっと読みやすく書いたほうがいいよ。	
\\	[動詞]	再び配列する	
\\	もしアルファベットを並べ直してもいいなら、
\\	と
\\	を隣同士に置こうかなあ。	
\\	[名詞]	反転逆転	
\\	複数形チェック
\\	[動詞]	非難する	
\\	あなたがこれらの事をしたのを、わたしが黙っていたので、あなたはわたしを全く自分とひとしい者と思った。しかしわたしはあなたを責め、あなたの目の前にその罪をならべる。「詩篇50:21」	
\\	[名詞]	回転式ピストル	
\\	複数形チェック
\\	[動詞]	再び捕らえる	
\\	5人の囚人は捕まったが、残り3人は今も逃走中だ。	
\\	[動詞]	装備する	
\\	[形容詞]	受け取るべき	
\\	[形容詞]	道義的に正しい	
\\	[名詞]	感覚器官受容体	
\\	研究者は頭脳の新しい受容器の複合体を識別する。	
\\	複数形チェック
\\	[形容詞]	厳格な	
\\	既に述べたが如く、徹底的な調査の結果によると、我が社の経営には問題ないことは明らかになりました。	
\\	[形容詞]	相互の	
\\	[名詞]	スケートリンク	
\\	複数形チェック
\\	[形容詞]	レクリエーションの	
\\	[名詞]	道端	
\\	ふと見ると道端に自転車が捨てられていた。	
\\	複数形チェック
\\	[名詞]	⻑方形	
\\	正方形は長方形のうちの一つですし、長方形は平行四辺形の一つです。	
\\	複数形チェック
\\	[名詞]	ヨーロッパコマドリ	
\\	複数形チェック
\\	[形容詞]	⻑方形の	
\\	「家」という単語は、草木に囲まれ、幸せな家族が住んでいる屋根や煙突がある長方形の建物、という心象を呼び起こす。	
\\	[形容詞]	たくましい	
\\	雨ニモマケズ/風ニモマケズ/雪ニモ夏ノ暑サニモマケヌ/丈夫ナカラダヲモチ/慾ハナク/決シテ瞋ラズ/イツモシズカニワラッテイル	
\\	[動詞]	修正する	
\\	[名詞]	揺りいす、ロック歌手	
\\	複数形チェック
\\	[動詞]	繰り返される	
\\	このようなことが二度と起こらないようにしなければならない。	
\\	[名詞]	ごろつき、悪党	
\\	複数形チェック
\\	[名詞]	再発	
\\	複数形チェック
\\	[形容詞]	逃亡した	
\\	[動詞]	赤くなる	
\\	[形容詞]	いなかの	
\\	商店街に入ると、陽菜はまるでおのぼりさんのようにキョロキョロ辺りを見回した。	
\\	[動詞]	弁済する	
\\	[形容詞]	無慈悲な	
\\	[名詞]	償還	
\\	複数形チェック
\\	[名詞]	賢人、哲人	
\\	彼は賢者だ。	
\\	複数形チェック
\\	[動詞]	再発見する	
\\	[名詞]	救済	
\\	複数形チェック
\\	[名詞]	余分なこと	
\\	あの会社は今、希望退職者に退職金を多く都合することで人員削減を図っている。	
\\	複数形チェック
\\	[名詞]	試食	
\\	このソフトウエアはギブスサンプリングのアルゴリズムによりマルコフ連鎖モンテカルロ法の計算を行います。	
\\	複数形チェック
\\	[名詞]	礁	
\\	複数形チェック
\\	[名詞]	制裁	
\\	この建物に入るには当局の許可を得る必要がある。	
\\	複数形チェック
\\	[動詞]	再び入る	
\\	人のオフィスに入るときは、まずノックしなくては。	
\\	[名詞]	神聖な場所	
\\	けがを治療してもらったトラは、村人たちの手で無事、保護区に帰された。	
\\	複数形チェック
\\	[動詞]	再建する	
\\	『カムイの剣』は、1868年の徳川将軍時代の崩壊と、明治天皇下での日本の復興という変革期を舞台にした、一種の侍/忍者物語だ。	
\\	[名詞]	樹液	
\\	複数形チェック
\\	[名詞]	皮肉	
\\	彼女は皮肉を解さない。	
\\	複数形チェック
\\	[名詞]	連隊	
\\	複数形チェック
\\	[名詞]	イワシ	
\\	複数形チェック
\\	[動詞]	下稽古をする	
\\	プライヤー氏はパリでの公演のリハーサル中にくも膜下出血で倒れ、市内の病院に運ばれましたが午後4時23分息を引き取りました。	
\\	[形容詞]	学問的な	
\\	[名詞]	トナカイ	
\\	複数形チェック
\\	[名詞]	聖書	
\\	コーランはイスラム教徒の聖典です。	
\\	複数形チェック
\\	[副詞]	いやいやながら	
\\	[名詞]	詳細な調査	
\\	複数形チェック
\\	[名詞]	思い出させるもの	
\\	図書館から督促状が届いた。	
\\	複数形チェック
\\	[形容詞]	世俗の	
\\	[名詞]	回想	
\\	複数形チェック
\\	[動詞]	誘惑する	
\\	あの男は手が早いから気をつけた方がいいよ。	
\\	[名詞]	残物	
\\	複数形チェック
\\	[形容詞]	うわべの	
\\	[形容詞]	再生可能な	
\\	[動詞]	分離する	
\\	更衣室は男女に別れていましたが、お風呂場に入って行ったら 男女区別なく、混浴でした。	
\\	[動詞]	放棄する	
\\	[名詞]	隔離、人種隔離	
\\	複数形チェック
\\	[名詞]	名声	
\\	複数形チェック
\\	[名詞]	捕らえること、発作	
\\	複数形チェック
\\	[形容詞]	有名な	
\\	[名詞]	半導体	
\\	複数形チェック
\\	[名詞]	後悔悔い改め	
\\	複数形チェック
\\	[形容詞]	感覚の	
\\	[名詞]	写し複製	
\\	複数形チェック
\\	[形容詞]	分離できる	
\\	[動詞]	抑える抑圧する	
\\	[名詞]	続編	
\\	ちなみに、それはこの話の後日談にする予定です。	
\\	複数形チェック
\\	[名詞]	弾圧	
\\	この段階を踏まえ、抑圧という概念をさらに詳細に検証していくことになります。	
\\	複数形チェック
\\	[名詞]	後退、挫折	
\\	複数形チェック
\\	[形容詞]	抑圧的な	
\\	[名詞]	厳しさ	
\\	複数形チェック
\\	[形容詞]	生殖の、再生の	
\\	陰茎は雄の生殖器のひとつ。	
\\	[名詞]	下水道	
\\	複数形チェック
\\	[名詞]	評判好評	
\\	複数形チェック
\\	[形容詞]	形の	
\\	この果物はオレンジのような形をして、パイナップルのような味がします。	
\\	[名詞]	貯水池	
\\	複数形チェック
\\	[名詞]	シャーベット	
\\	メロンのシャーベットと生チョコレートのケーキを食べましょうか。	
\\	複数形チェック
\\	[名詞]	残留物	
\\	複数形チェック
\\	[名詞]	保安官	
\\	保安官は容疑者を相手が死んでしまうのではないかと思われるまで何回も殴りつけた。	
\\	複数形チェック
\\	[形容詞]	機知に富んだ	
\\	[名詞]	造船	
\\	複数形チェック
\\	[副詞]	敬意を表して	
\\	[名詞]	靴紐	
\\	あ、ちょっとタンマ。クツのヒモがほどけちゃった。	
\\	複数形チェック
\\	[名詞]	回答者	
\\	複数形チェック
\\	[動詞]	万引きする	
\\	メアリーは万引きをして捕まった。	
\\	[名詞]	保持	
\\	複数形チェック
\\	[名詞]	買い物客	
\\	複数形チェック
\\	[動詞]	回想する	
\\	[名詞]	ショールーム	
\\	我が社のショールームは若い女性に大人気だった。	
\\	複数形チェック
\\	[動詞]	取り戻す	
\\	パスワードを復旧する事はできませんでした。再試行してください。	
\\	[形容詞]	目立つ派手な	
\\	[名詞]	回顧	
\\	複数形チェック
\\	[動詞]	避ける	
\\	[形容詞]	回顧の	
\\	[名詞]	兄弟姉妹	
\\	複数形チェック
\\	[名詞]	同窓会、親睦会	
\\	30年ぶりにクラス会があった。	
\\	複数形チェック
\\	[形容詞]	吐き気を催すような	
\\	[名詞]	包囲攻撃包囲	
\\	複数形チェック
\\	[名詞]	でんぷん	
\\	洗濯のりでののりづけの仕方を教えてください。	
\\	複数形チェック
\\	[動詞]	ふるいにかける	
\\	[名詞]	蒸気船	
\\	複数形チェック
\\	[名詞]	しるし、標識	
\\	署名する前に契約書をしっかり調べたほうがいいよ。	
\\	複数形チェック
\\	[名詞]	教会などの尖塔	
\\	複数形チェック
\\	[名詞]	道しるべ	
\\	複数形チェック
\\	[名詞]	継父	
\\	複数形チェック
\\	[名詞]	輪郭	
\\	お客さま、そちらのドレスはすとんとしたシルエットがとても上品で素敵なんです。	
\\	複数形チェック
\\	[名詞]	継母	
\\	メアリーはトムの義理の母だ。	
\\	複数形チェック
\\	[名詞]	珪素	
\\	アルミニウム、マグネシウム、シリコンなどの粉末が爆発する。	
\\	複数形チェック
\\	[名詞]	固定観念、決まり文句	
\\	複数形チェック
\\	[形容詞]	銀の銀白色の	
\\	[動詞]	窒息させる	
\\	非の打ち所がない計画は、窮屈だなあ。	
\\	[名詞]	模擬実験	
\\	複数形チェック
\\	[名詞]	興奮剤	
\\	エクスタシーはアッパー系のドラッグだ。	
\\	複数形チェック
\\	[形容詞]	不吉な、邪悪な	
\\	[動詞] [不規則動詞]	悪臭を放つ	
\\	超臭いよ。	
\\	[名詞]	罪人	
\\	複数形チェック
\\	[形容詞]	感動させる	
\\	暗闇の中で何かがもぞもぞと動いていた。	
\\	[形容詞]	疑い深い	
\\	私は、5つ以上の言語を流暢に話せるという主張には、疑いを持ちます。	
\\	[動詞]	唖然とする	
\\	美しい!	
\\	[名詞]	懐疑的な考え	
\\	複数形チェック
\\	[名詞]	頑丈な	
\\	この縄は丈夫だ。	
\\	複数形チェック
\\	[名詞]	四角く幅の広い厚板	
\\	複数形チェック
\\	[名詞]	小委員会	
\\	複数形チェック
\\	[形容詞]	ゆるんだ	
\\	いいかげんにして。	
\\	[形容詞]	控えめな	
\\	しめやかな女はだいたい一番きれいだと思う。にぎやかな人はひどいな。	
\\	[名詞]	そり	
\\	冬に家のそばの小さな丘を滑り降りるのが好きです。	
\\	複数形チェック
\\	[名詞]	下位グループ	
\\	複数形チェック
\\	[形容詞]	滑る	
\\	壁に耳有り障子に目有り。	
\\	[形容詞]	従順な	
\\	[名詞]	つり縄三角布	
\\	複数形チェック
\\	[動詞]	定期購読する	
\\	うちは新聞を購読しています。	
\\	[動詞]	融解して精錬する	
\\	[名詞]	予約者購読者	
\\	複数形チェック
\\	[名詞]	お高くとまった人	
\\	複数形チェック
\\	[名詞]	微妙	
\\	複数形チェック
\\	[名詞]	雪玉	
\\	複数形チェック
\\	[形容詞]	暗示に富む	
\\	[動詞]	冷たく扱う	
\\	[形容詞]	自殺的な自滅的な	
\\	[形容詞]	居心地のよい	
\\	くしゃっとした感じで、ふんわりとした髪型にしたいんです。	
\\	[名詞]	ひと揃い	
\\	彼女はホテルのスイートルームを独り占めにした。	
\\	複数形チェック
\\	[形容詞]	びしょぬれで	
\\	トムは湯船に浸かりました。	
\\	[形容詞]	水没した	
\\	[形容詞]	社会経済的な	
\\	[名詞]	監督者、支配人	
\\	複数形チェック
\\	[形容詞]	社会学的な	
\\	[形容詞]	超音速の	
\\	[動詞]	強く求める	
\\	[名詞]	スーパースター	
\\	複数形チェック
\\	[名詞]	懇願者勧誘人	
\\	複数形チェック
\\	[形容詞]	補足の	
\\	補足情報には、副詞や副詞のはたらきをするものが入ります。	
\\	[名詞]	ソナタ	
\\	ソナタを一曲弾いてあげよう。	
\\	複数形チェック
\\	[名詞]	主権優位	
\\	複数形チェック
\\	[形容詞]	南東の、南東への	
\\	クロアチアはヨーロッパの東南部に位置する国である。	
\\	[名詞]	急上昇	
\\	最大10メートルの津波が押し寄せるとみられます。	
\\	複数形チェック
\\	[形容詞]	南⻄部の	
\\	[形容詞]	外科の	
\\	[名詞]	主権、統治権	
\\	憲法記念日は、憲法の基本的精神である、国民主権、基本的人権の尊重、平和主義を再確認するための日です。	
\\	複数形チェック
\\	[動詞]	推測する	
\\	[名詞]	大豆	
\\	複数形チェック
\\	[形容詞]	持続可能な	
\\	彼は持続可能な開発の専門家です。	
\\	[動詞]	お尻をたたく	
\\	子どもの頃、悪さをするといつも尻を叩かれた。	
\\	[動詞]	交換する	
\\	流れを渡っている間に馬を取り替えるな。	
\\	[副詞]	専門の	
\\	[形容詞]	ふくれた誇張した	
\\	手足がむくんでいます。	
\\	[名詞]	仕様、明細事項	
\\	複数形チェック
\\	[動詞]	急降下する	
\\	[名詞]	小さい点	
\\	複数形チェック
\\	[名詞]	授業予定表	
\\	ほとんどの学生はシラバスをあまりよく読まない。	
\\	複数形チェック
\\	[名詞]	スペクトル、範囲	
\\	物体の電磁波的分光を観測することで、科学者は物体が地球に近づいているか、遠ざかっているかを判別することができる。	
\\	複数形チェック
\\	[名詞]	症候群	
\\	前腕と手くびが痛いです。手根管症候群が原因かもしれない。	
\\	複数形チェック
\\	[形容詞]	推論的な、投機的な	
\\	[名詞]	同意語	
\\	「類義語」という意味の言葉は他にありますか?	
\\	複数形チェック
\\	[形容詞]	口のきけない無言の	
\\	トムは驚きのあまり何も言えなかった。	
\\	[動詞]	総合する、合成する	
\\	[名詞]	紡ぎ手	
\\	複数形チェック
\\	[形容詞]	合成の	
\\	[名詞]	紡績	
\\	蚕が糸を吐いている。	
\\	複数形チェック
\\	[名詞]	タブロイド新聞	
\\	複数形チェック
\\	[形容詞]	らせん形の	
\\	[名詞]	タブー、禁制	
\\	近親相姦はほとんど全ての文化において禁忌とされている。	
\\	複数形チェック
\\	[名詞]	とげ破片	
\\	トムはメアリーの指からとげを抜こうとした。	
\\	複数形チェック
\\	[名詞]	機転	
\\	複数形チェック
\\	[名詞]	代弁者、スポークスマン	
\\	複数形チェック
\\	[名詞]	方策、戦術	
\\	複数形チェック
\\	[名詞]	自発性	
\\	複数形チェック
\\	[名詞]	汚れ	
\\	複数形チェック
\\	[名詞]	集計表	
\\	項目が増えたり減ったりする表示なら、
\\	でなくて表計算ソフトを活用すべきだと思います。	
\\	複数形チェック
\\	[名詞]	離陸	
\\	コートを脱ぎなさい。	
\\	複数形チェック
\\	[名詞]	部隊チーム	
\\	ジャスティン班長、亜空間レーダーに反応出ました!	
\\	複数形チェック
\\	[形容詞]	触れて感知できる	
\\	[副詞]	真正面に	
\\	[名詞]	先細り逓減	
\\	複数形チェック
\\	[動詞]	チューチュー鳴く	
\\	自分たち以外には誰もいないはずの公園で、突然ブランコのギイギイとゆれる音が聞こえた。	
\\	[名詞]	タルト	
\\	ブルーベリーをとった後、タルトを作る。	
\\	複数形チェック
\\	[名詞]	階段	
\\	複数形チェック
\\	[形容詞]	味のない無味乾燥な	
\\	今日のブロッコリー、ゆだりすぎでおいしくないよ。	
\\	[名詞]	入れ墨	
\\	刺青があるの?	
\\	複数形チェック
\\	[動詞]	荷を解く	
\\	ちょうど着いたばかり。まだスーツケースから荷物を出してもいない。	
\\	[名詞]	節制	
\\	複数形チェック
\\	[形容詞]	先例のない	
\\	東京市場は空前の損失を記録しました。	
\\	[形容詞]	現世の	
\\	[形容詞]	準備のできていない	
\\	[形容詞]	仮の	
\\	[名詞]	不安	
\\	動乱は3日間続いた。	
\\	複数形チェック
\\	[動詞]	終わらせる	
\\	[動詞]	不安にする	
\\	[副詞]	それによって	
\\	[形容詞]	売れ残りの	
\\	[形容詞]	熱の、温度の	
\\	[動詞]	明らかにする	
\\	[名詞]	温度自動調節器	
\\	複数形チェック
\\	[形容詞]	歓迎されない	
\\	[名詞]	革ひも	
\\	複数形チェック
\\	[名詞]	高地	
\\	複数形チェック
\\	[名詞]	なんじは、そなたは	
\\	汝殺すなかれ。	
\\	複数形チェック
\\	[名詞]	ウラン	
\\	複数形チェック
\\	[動詞]	群がる、殺到する	
\\	[名詞]	理想郷	
\\	複数形チェック
\\	[名詞]	ツグミ	
\\	複数形チェック
\\	[形容詞]	理想郷の	
\\	[動詞]	ゴツンと打つ	
\\	[名詞]	ワクチン	
\\	ジョナス・ソークは1952年にポリオワクチンを開発した。	
\\	複数形チェック
\\	[形容詞]	雷を発する	
\\	[動詞]	有効にする	
\\	[名詞]	つまようじ	
\\	つまようじをください。	
\\	複数形チェック
\\	[名詞]	妥当性	
\\	複数形チェック
\\	[名詞]	豪雨	
\\	複数形チェック
\\	[名詞]	評価、評価額	
\\	複数形チェック
\\	[名詞]	都会人	
\\	町の人達は物珍しそうに私を見た。	
\\	複数形チェック
\\	[副詞]	様々な	
\\	[動詞]	書き写す	
\\	先ずは憧れの作家の文章の呼吸をつかむためにひたすら筆写、丸写しをする。	
\\	[名詞]	植生、草木	
\\	複数形チェック
\\	[名詞]	注入輸血	
\\	輸血が必要です。	
\\	複数形チェック
\\	[名詞]	売主	
\\	複数形チェック
\\	[名詞]	交通、通過	
\\	今、飛行機の乗り換えだ。今まで飛行機8時間乗って、乗り換えてこれから又2時間だ。	
\\	複数形チェック
\\	[名詞]	(陪審員の)評決	
\\	複数形チェック
\\	[形容詞]	過渡的な	
\\	[名詞]	端境界	
\\	これらの動物は、絶滅の危機に瀕している。	
\\	複数形チェック
\\	[名詞]	送信機送話器	
\\	複数形チェック
\\	[動詞]	実証する	
\\	お客様のパソコンのプラグが電源に差し込まれているか確認してくださいませんか?	
\\	[形容詞]	
\\	倍の	
\\	[名詞]	悪者悪役	
\\	複数形チェック
\\	[名詞]	裁判所	
\\	複数形チェック
\\	[動詞]	視覚化する	
\\	陽電子放射断層撮影法では、放射性トレーサーを利用して体内の代謝活動を可視化します。	
\\	[動詞]	したたる	
\\	涙が彼女のほおをつたって流れた。	
\\	[形容詞]	職業上の	
\\	ふたりはアニメーションの専門学校で知り合った。	
\\	[名詞]	一兆	
\\	光年は光が一年間に進む距離です。それは9兆5千億kmです。	
\\	複数形チェック
\\	[名詞]	流行	
\\	複数形チェック
\\	[名詞]	路面電車	
\\	複数形チェック
\\	[名詞]	空虚感	
\\	複数形チェック
\\	[名詞]	切り札	
\\	君の奥の手を使うときが来た。	
\\	複数形チェック
\\	[動詞]	泣き叫ぶ	
\\	[名詞]	学費、授業料	
\\	学校は授業料の値上げを発表した。	
\\	複数形チェック
\\	[名詞]	チョッキ	
\\	複数形チェック
\\	[名詞]	ふくらみ腫瘍	
\\	複数形チェック
\\	[形容詞]	欠乏している	
\\	欲しくないという事は持っていると同じことだ。	
\\	[名詞]	騒動	
\\	複数形チェック
\\	[名詞]	保証	
\\	複数形チェック
\\	[形容詞]	荒れ狂う、乱れている	
\\	我らの頭上に荒れ狂う砂と剣の煌きは、まるで夜空に降り注ぐ流星の如し。	
\\	[形容詞]	防水の耐水の	
\\	[名詞]	芝地	
\\	複数形チェック
\\	[形容詞]	水を含んだ	
\\	[形容詞]	家庭教師の	
\\	このチュートリアルではサンプルマンガを使用して、ダウンロードを含めてマンガを見る手順に慣れていただきます。	
\\	[名詞]	ワット	
\\	複数形チェック
\\	[動詞]	ぐいと引く	
\\	左のまぶたがぴくぴくする。	
\\	[形容詞]	どんなものであれ	
\\	トムはおしゃれに全く関心がない。	
\\	[形容詞]	専制君主的な	
\\	圧政的な政府はしばしば政敵を投獄する。	
\\	[副詞]	それによって	
\\	[形容詞]	利用できない	
\\	[名詞]	男やもめ	
\\	複数形チェック
\\	[動詞]	ボタンを外す	
\\	トムはシャツのボタンを外した。	
\\	[名詞]	快くすること	
\\	複数形チェック
\\	[形容詞]	心配しない	
\\	最近は他人を傷つけることを比較的気にしなくなったので、そういう点ではメール打つのも気楽になったよ。	
\\	[名詞]	配線	
\\	複数形チェック
\\	[形容詞]	慣例に従わない	
\\	[名詞]	悲哀災難	
\\	複数形チェック
\\	[形容詞]	否定しがたい	
\\	[名詞]	木工品	
\\	複数形チェック
\\	[動詞]	いつのまにか害する	
\\	疲労が彼の健康をむしばんでいる。	
\\	[名詞]	言葉づかい	
\\	複数形チェック
\\	[形容詞]	もとどおりにする	
\\	靴ひもがとけた。	
\\	[形容詞]	実行できる、有効な	
\\	[形容詞]	身元不明の未確認の	
\\	[名詞]	汝ら	
\\	ニーチェは「化け物に成らないように,化け物と喧嘩するな」と言いました。	
\\	複数形チェック
\\	[名詞]	統合、統一	
\\	複数形チェック
\\	[名詞]	あこがれ、熱望	
\\	複数形チェック
\\	[形容詞]	意図的でない	
\\	[名詞]	酵母	
\\	複数形チェック
\\	[形容詞]	連続した、一体となった	
\\	[形容詞]	曲がりやすい	
\\	[動詞]	解き放す	
\\	[名詞]	くびき	
\\	複数形チェック
\\	[形容詞]	愛されていない	
\\	[名詞]	卵の⻩身	
\\	両手でもきれいに卵が割れないんだよ。	
\\	複数形チェック
\\	[形容詞]	人目につかない	
\\	老齢は気づかぬうちに我々に忍び寄る。	
\\	[名詞]	ジグザグ	
\\	複数形チェック
\\	[名詞]	腹部、腹	
\\	複数形チェック
\\	[名詞]	反意語	
\\	複数形チェック
\\	[動詞] [不規則動詞]	我慢する	
\\	あなたは法に従わなくてはならない。	
\\	[形容詞]	謝罪の、すまなそうな	
\\	[形容詞]	未成熟の、不成功の	
\\	[形容詞]	とてもひどい	
\\	彼女は料理が恐ろしく下手だ。	
\\	[動詞]	慎む、控える	
\\	彼は禁煙した。	
\\	[名詞]	充当	
\\	複数形チェック
\\	[名詞]	加速装置	
\\	しばらく車を運転していないと、アクセルとブレーキがごっちゃになることない?	
\\	複数形チェック
\\	[名詞]	考古学者	
\\	複数形チェック
\\	[名詞]	大かっさい、歓呼	
\\	複数形チェック
\\	[名詞]	弓の射手	
\\	射手は鹿を射た。	
\\	複数形チェック
\\	[形容詞]	親切な	
\\	[名詞]	記録保管所	
\\	火災で全てのコンピューターディスクが駄目になってしまった時、会社はもうお手上げの状況だった。	
\\	複数形チェック
\\	[名詞]	伴奏者	
\\	複数形チェック
\\	[形容詞]	熱烈な、熱心な	
\\	[名詞]	説明義務	
\\	複数形チェック
\\	[形容詞]	困難な、骨の折れる	
\\	[形容詞]	釈明の義務のある	
\\	その損失について私は彼に責任を負わなくてはならない。	
\\	[名詞]	軍備、装備、武装	
\\	複数形チェック
\\	[名詞]	添加物	
\\	複数形チェック
\\	[名詞]	香り	
\\	僕は淹れたてのコーヒーの香りが好きなんだ。	
\\	複数形チェック
\\	[名詞]	達人	
\\	彼はうそつきの名人だ。	
\\	複数形チェック
\\	[名詞]	兵器	
\\	複数形チェック
\\	[名詞]	支持者	
\\	複数形チェック
\\	[名詞]	動脈	
\\	複数形チェック
\\	[名詞]	接着剤	
\\	複数形チェック
\\	[名詞]	はっきり発音する	
\\	複数形チェック
\\	[名詞]	敬愛	
\\	複数形チェック
\\	[名詞]	大砲	
\\	最近僕の義弟Y砲兵少佐が、三年間の巴里駐在を終へて帰つて来た。数々の土産物を取巻いて、われわれはいろいろな土産話を聴いた。	
\\	複数形チェック
\\	[名詞]	出現	
\\	複数形チェック
\\	[名詞]	アスファルト	
\\	複数形チェック
\\	[名詞]	敵、対戦相手	
\\	昨日の敵は今日の友。	
\\	複数形チェック
\\	[名詞]	保険金受取人	
\\	全力を尽くしますからご安心下さい。	
\\	複数形チェック
\\	[名詞]	支持、弁護	
\\	複数形チェック
\\	[名詞]	ぜんそく	
\\	喘息の発作が起きました。	
\\	複数形チェック
\\	[名詞]	美学	
\\	複数形チェック
\\	[動詞]	びっくり仰天させる	
\\	彼の英語力には本当に驚かされる。	
\\	[名詞]	供述書	
\\	複数形チェック
\\	[名詞]	残虐行為	
\\	複数形チェック
\\	[名詞]	関係者、系列会社	
\\	なぜ、A社の傘下に入ることを選んだのか。	
\\	複数形チェック
\\	[形容詞]	未遂の	
\\	彼は逃亡を図った。	
\\	[名詞]	提携	
\\	複数形チェック
\\	[形容詞]	起因する	
\\	[名詞]	親近感、密接な関係	
\\	複数形チェック
\\	[形容詞]	聞き取れる	
\\	[名詞]	豊富、富裕、流入	
\\	複数形チェック
\\	[名詞]	真正	
\\	複数形チェック
\\	[副詞]	再び	
\\	[名詞]	サイン	
\\	あなたのサインをいただけませんか。	
\\	複数形チェック
\\	[名詞]	結果	
\\	複数形チェック
\\	[形容詞]	自治の、自主的な	
\\	これは自律的な機械です。	
\\	[名詞]	集合	
\\	複数形チェック
\\	[名詞]	嫌悪感	
\\	彼女は彼を嫌っていた。	
\\	複数形チェック
\\	[名詞]	飛行士	
\\	複数形チェック
\\	[名詞]	目覚め、気づくこと	
\\	複数形チェック
\\	[名詞]	飛行船	
\\	複数形チェック
\\	[名詞]	肯定の答、賛成投票者	
\\	複数形チェック
\\	[名詞]	エール	
\\	複数形チェック
\\	[名詞]	逆火	
\\	複数形チェック
\\	[名詞]	アリバイ	
\\	彼にはちゃんとしたアリバイがあった。	
\\	複数形チェック
\\	[名詞]	裏打ち、支援	
\\	複数形チェック
\\	[動詞] [不規則動詞]	降り立つ	
\\	[形容詞]	バクテリアの	
\\	[名詞]	忠誠、献身	
\\	騎士が王への忠誠を誓った。	
\\	複数形チェック
\\	[名詞]	困惑	
\\	複数形チェック
\\	[名詞]	割当額	
\\	複数形チェック
\\	[名詞]	洗礼	
\\	複数形チェック
\\	[名詞]	合金	
\\	真鍮は銅と亜鉛の合金である。	
\\	複数形チェック
\\	[動詞]	洗礼を行う	
\\	[動詞]	ほのめかす	
\\	[名詞]	平底荷船	
\\	いくら我が家がテレビで紹介されたからと言って、勝手にずかずかと庭に入り込んで写真を取るのはやめてほしい。	
\\	複数形チェック
\\	[名詞]	魅力	
\\	複数形チェック
\\	[名詞]	強打	
\\	複数形チェック
\\	[副詞]	上空に	
\\	空高く一群の鳥が飛んでいるのを見た。	
\\	[名詞]	バジル	
\\	複数形チェック
\\	[名詞]	アルト歌手	
\\	わたしはアルトの方が良かったんだけど、メゾソプラノに入れられた。	
\\	複数形チェック
\\	[名詞]	いやなやつ	
\\	学校に行けよ!怠け者。	
\\	複数形チェック
\\	[動詞]	集める	
\\	[名詞]	打者	
\\	複数形チェック
\\	[名詞]	琥珀、こはく色	
\\	時を超えて古代の昆虫たちが琥珀の中で生き生きと踊る。	
\\	複数形チェック
\\	[名詞]	ビーズ	
\\	複数形チェック
\\	[名詞]	待ち伏せ	
\\	彼を待ち伏せる一計を案じた。	
\\	複数形チェック
\\	[形容詞]	いやな	
\\	[名詞]	アンモニア	
\\	複数形チェック
\\	[動詞]	合図する	
\\	[名詞]	弾薬	
\\	1920年代初期のメキシコの無法者たちは盗んだ小火器や爆薬で十分に武装していた。	
\\	複数形チェック
\\	[名詞]	ベージュ色	
\\	複数形チェック
\\	[名詞]	恩赦	
\\	国際アムネスティは、政治囚への支援として一般市民による抗議活動を組織することがしばしばある。	
\\	複数形チェック
\\	[名詞]	ほえ声、どなり声	
\\	複数形チェック
\\	[名詞]	停泊地	
\\	複数形チェック
\\	[名詞]	受益者	
\\	複数形チェック
\\	[名詞]	対立、敵意	
\\	複数形チェック
\\	[形容詞]	慈悲深い、優しい	
\\	[名詞]	対立者	
\\	複数形チェック
\\	[形容詞]	温和な、優しい	
\\	[名詞]	人類学者	
\\	複数形チェック
\\	[動詞] [不規則動詞]	包囲する、ふさぐ	
\\	[名詞]	抗生物質	
\\	複数形チェック
\\	[動詞]	包囲する、押し寄せる	
\\	ロベルさんはいつも人をいろいろな依頼で攻め立てた。	
\\	[名詞]	参考文献	
\\	複数形チェック
\\	[形容詞]	原因の、因果関係の	
\\	[名詞]	広告板	
\\	複数形チェック
\\	[名詞]	穴	
\\	アイス食べた時、歯がしみたんだ。虫歯があるかもしれないな。	
\\	複数形チェック
\\	[名詞]	ビンゴ	
\\	複数形チェック
\\	[名詞]	携帯電話	
\\	携帯電話がほしいのですが、支払うのに十分なお金がありません。	
\\	複数形チェック
\\	[名詞]	生化学	
\\	複数形チェック
\\	[名詞]	非難	
\\	複数形チェック
\\	[名詞]	生得権	
\\	避妊はちゃんとコンドームつけてたのよね?	
\\	複数形チェック
\\	[名詞]	中央集権	
\\	複数形チェック
\\	[名詞]	停電	
\\	停電があったのでレコーダーの日時を設定しなおしたんだけど、1日間違って予約をミスってしまった。	
\\	複数形チェック
\\	[形容詞]	陶器の	
\\	[名詞]	膀胱	
\\	複数形チェック
\\	[名詞]	反転音	
\\	複数形チェック
\\	[形容詞]	温和な、刺激のない	
\\	引っ越してきたばかりだから、まだ部屋が殺風景なんだよ。	
\\	[形容詞]	儀式を重んじる	
\\	[名詞]	おおい隠すもの	
\\	複数形チェック
\\	[名詞]	洗練、上品	
\\	複数形チェック
\\	[名詞]	圏、議員連合	
\\	複数形チェック
\\	[名詞]	出産	
\\	複数形チェック
\\	[名詞]	男	
\\	あんな奴とはもう二度と口をききたくない。	
\\	複数形チェック
\\	[名詞]	彫刻刀	
\\	複数形チェック
\\	[名詞]	大失敗	
\\	なんて不覚!	
\\	複数形チェック
\\	[動詞]	命名する	
\\	[名詞]	⻑枕	
\\	複数形チェック
\\	[名詞]	軽くつつくこと	
\\	アンインストール方法:フォルダごとごみ箱にぽいっ。	
\\	複数形チェック
\\	[形容詞]	骨ばった	
\\	骨がたくさんある魚は嫌いだ。	
\\	[名詞]	攪乳器	
\\	複数形チェック
\\	[名詞]	後援者、熱狂的支持者	
\\	複数形チェック
\\	[名詞]	ハマグリ、二枚貝	
\\	複数形チェック
\\	[名詞]	借用	
\\	トムはいつもお金を借りている。	
\\	複数形チェック
\\	[名詞]	よじ登ること	
\\	複数形チェック
\\	[名詞]	ブルジョア	
\\	複数形チェック
\\	[名詞]	氏族、一族	
\\	江戸時代に、各藩の奨励策によって、全国各地に地場産業が興った。	
\\	複数形チェック
\\	[名詞]	勝負、競争	
\\	複数形チェック
\\	[名詞]	クラリネット	
\\	トムとメアリーは両方ともクラリネットを吹かない。	
\\	複数形チェック
\\	[名詞]	ブティック	
\\	複数形チェック
\\	[名詞]	陳腐な表現	
\\	複数形チェック
\\	[形容詞]	少年のような	
\\	[形容詞]	時計回りの	
\\	[名詞]	自慢話	
\\	複数形チェック
\\	[名詞]	物干し綱	
\\	複数形チェック
\\	[名詞]	編んだ髪	
\\	お母さん、髪の毛を編み込みにして欲しいんだけど。	
\\	複数形チェック
\\	[名詞]	洗濯ばさみ	
\\	複数形チェック
\\	[名詞]	醸造業者	
\\	彼の仕事はビール職人だ。	
\\	複数形チェック
\\	[動詞]	おだてる	
\\	[名詞]	ビール醸造所	
\\	うちが援助してる酒造さんがね、新しい甘酒を造ったから試飲してみてくださいって持ってきてくれたんだけど・・・。	
\\	複数形チェック
\\	[名詞]	クモの巣	
\\	複数形チェック
\\	[名詞]	結婚式	
\\	複数形チェック
\\	[名詞]	タラ	
\\	複数形チェック
\\	[名詞]	団	
\\	複数形チェック
\\	[動詞]	共存する	
\\	[副詞]	きびきびと	
\\	おもちゃコンサルタントの方々が甲斐甲斐しく働く姿がとても印象的でした。	
\\	[名詞]	共存	
\\	複数形チェック
\\	[名詞]	剛毛	
\\	複数形チェック
\\	[名詞]	認知	
\\	複数形チェック
\\	[名詞]	スープ	
\\	料理人が多すぎるとスープができそこなう。	
\\	複数形チェック
\\	[名詞]	結合、団結	
\\	複数形チェック
\\	[名詞]	残虐行為	
\\	複数形チェック
\\	[形容詞]	口語的な	
\\	酒豪とは、アルコール飲料を大量に飲むことが出来、さらにこれで酔態をさらさない者の俗称である。	
\\	[名詞]	もみ革	
\\	複数形チェック
\\	[形容詞]	巨大な	
\\	[名詞]	強化、蓄積	
\\	語彙力を増強したい。	
\\	複数形チェック
\\	[名詞]	昏睡	
\\	患者は深い昏睡状態であります。	
\\	複数形チェック
\\	[名詞]	でっぱり	
\\	複数形チェック
\\	[名詞]	卒業式	
\\	北朝鮮が6か国協議の合意に基づき核開発計画を申告した26日、米国が「テロ支援国」の指定解除手続きに入ったことで、拉致被害者の家族らには「拉致問題が置き去りにされるのでは」という不安が広がった。	
\\	複数形チェック
\\	[形容詞]	でこぼこの	
\\	[名詞]	庶⺠	
\\	複数形チェック
\\	[名詞]	官僚	
\\	複数形チェック
\\	[形容詞]	共同社会の、自治体の	
\\	寮には共用のキッチンがある。	
\\	[形容詞]	官僚的な	
\\	あんなに頑張ってきたのに官僚という仕事は報われない。	
\\	[形容詞]	補足的な	
\\	テレビと新聞は互いの機能を補完しあっている。	
\\	[名詞]	燃料枯渇	
\\	複数形チェック
\\	[名詞]	顔色	
\\	日に当たって彼の顔色はよくなった。	
\\	複数形チェック
\\	[名詞]	大騒ぎ、ざわめき	
\\	複数形チェック
\\	[名詞]	遵守	
\\	複数形チェック
\\	[名詞]	残り	
\\	ぶらぶらしないで、何とかしなさい!	
\\	複数形チェック
\\	[形容詞]	優待の、無料の	
\\	[名詞]	キャディー	
\\	複数形チェック
\\	[名詞]	落ち着き	
\\	複数形チェック
\\	[形容詞]	抜け目のない	
\\	[名詞]	圧縮	
\\	複数形チェック
\\	[名詞]	キャンプする人	
\\	複数形チェック
\\	[名詞]	隠すこと	
\\	複数形チェック
\\	[名詞]	スナップ写真	
\\	どう思うって・・・それじゃあ忌憚のない意見を言わせてもらうけどね。	
\\	複数形チェック
\\	[名詞]	協和音	
\\	デモインの裏庭、コンコードの居間、チャールストンの玄関からそれは始まった。	
\\	複数形チェック
\\	[名詞]	炭水化物	
\\	複数形チェック
\\	[動詞]	同意する、是認する	
\\	[名詞]	枢機卿	
\\	複数形チェック
\\	[名詞]	秘密性	
\\	複数形チェック
\\	[名詞]	風刺画、風刺文	
\\	あの人によって、大統領が戯画化され、政権の権威と綱紀が乱されてはなるまい。	
\\	複数形チェック
\\	[名詞]	配置	
\\	このゲームは基本的にキーボードとゲームパッドで動くようになっていますがコンフィグだけはマウスしか操作できません。	
\\	複数形チェック
\\	[名詞]	小滝	
\\	複数形チェック
\\	[名詞]	連結、合同	
\\	関係代名詞は「接続詞
\\	代名詞」のはたらきをしています。	
\\	複数形チェック
\\	[名詞]	焼きなべ料理	
\\	複数形チェック
\\	[名詞]	連結器	
\\	複数形チェック
\\	[形容詞]	大惨事の	
\\	[形容詞]	連続的な	
\\	来週の月曜と火曜は連休だ。	
\\	[動詞]	分類する	
\\	[動詞]	委託する、発送する	
\\	[名詞]	共同企業体	
\\	複数形チェック
\\	[名詞]	死亡、消滅	
\\	複数形チェック
\\	[動詞]	〜を強いる、強要する	
\\	[動詞]	取り壊す	
\\	[名詞]	解釈する	
\\	複数形チェック
\\	[名詞]	デニム	
\\	このGジャンはもう15年着ている。	
\\	複数形チェック
\\	[名詞]	領事	
\\	複数形チェック
\\	[動詞]	遺憾に思う	
\\	[名詞]	領事館	
\\	複数形チェック
\\	[動詞]	配置する、配備する	
\\	[名詞]	熟考	
\\	複数形チェック
\\	[名詞]	派生商品	
\\	複数形チェック
\\	[名詞]	競争相手	
\\	複数形チェック
\\	[形容詞]	値する	
\\	[名詞]	持続、継続時間	
\\	複数形チェック
\\	[名詞]	探知器	
\\	複数形チェック
\\	[動詞]	招集する	
\\	[名詞]	勾留	
\\	私は収容所から脱走した。	
\\	複数形チェック
\\	[動詞]	集まる	
\\	は
\\	に収束する。	
\\	[名詞]	戦争抑止力	
\\	複数形チェック
\\	[形容詞]	改造[改装]できる	
\\	コンバーチブルでアメリカを横断したい。	
\\	[形容詞]	破壊的な、ひどい	
\\	壊滅的な被害を防ぐために世界的な警戒を強化していく重要性を確認した。	
\\	[名詞]	運搬	
\\	複数形チェック
\\	[形容詞]	信心深い	
\\	トムは敬虔なクリスチャンを自認している。	
\\	[名詞]	護衛隊	
\\	複数形チェック
\\	[名詞]	器用さ、機敏さ	
\\	複数形チェック
\\	[名詞]	さんご	
\\	複数形チェック
\\	[動詞]	診断する	
\\	[名詞]	戴冠(式)、即位(式)	
\\	複数形チェック
\\	[名詞]	下痢	
\\	複数形チェック
\\	[名詞]	秒読み	
\\	真剣にやってよね。学園祭は待ってくれないんだよ、もう大詰めなんだから。	
\\	複数形チェック
\\	[名詞]	ディーゼル	
\\	列車は石油で走らせるのですか。	
\\	複数形チェック
\\	[動詞]	和らげる	
\\	[名詞]	食糧	
\\	極端な食事制限をする拒食症の女子高生が増えている。	
\\	複数形チェック
\\	[形容詞]	左回り、反時計方向	
\\	[名詞]	消化剤	
\\	複数形チェック
\\	[名詞]	宅配便業者、宅配便	
\\	複数形チェック
\\	[形容詞]	小さい、小型の	
\\	[名詞]	法廷	
\\	複数形チェック
\\	[動詞]	無能にする	
\\	身体障害者を助けるのが本当に楽しいのです。	
\\	[名詞]	契約(条項)	
\\	複数形チェック
\\	[形容詞]	恵まれない	
\\	[動詞]	むやみに欲しがる	
\\	日本の官僚達は国家が破綻するまで天下りを続けて税金をむさぼる国賊に過ぎない!	
\\	[形容詞]	認められる、識別できる	
\\	[形容詞]	ずる賢い	
\\	ドイツ人はとてもずる賢い。	
\\	[名詞]	食い違い、不一致	
\\	複数形チェック
\\	[名詞]	変人、クランク	
\\	なんてつむじ曲がりなやつだ。	
\\	複数形チェック
\\	[形容詞]	恥ずべき	
\\	[形容詞]	賞賛に値する	
\\	[名詞]	不正直	
\\	複数形チェック
\\	[形容詞]	むずむずする、ぞっとする	
\\	そのロボットはあまりにも真に迫りすぎて気持ち悪かった。	
\\	[動詞]	崩壊させる	
\\	[名詞]	ベビーベッド	
\\	複数形チェック
\\	[動詞]	分解する	
\\	[形容詞]	十文字に置いた、交差した	
\\	彼は十字を切った。	
\\	[形容詞]	秩序を乱す	
\\	[動詞]	十字架にかける	
\\	駆け落ちをした二人は磔になってしまったんです。	
\\	[動詞]	追い散らす、払い去る	
\\	[動詞]	最高潮に達する	
\\	[動詞]	分散する	
\\	[名詞]	犯罪者	
\\	複数形チェック
\\	[名詞]	動揺	
\\	複数形チェック
\\	[形容詞]	累積する	
\\	[形容詞]	意見を異にす、反体制の	
\\	[名詞]	管理人	
\\	複数形チェック
\\	[動詞]	多様化する	
\\	[名詞]	食卓用刃物	
\\	複数形チェック
\\	[形容詞]	部分的な	
\\	[名詞]	うまいもの	
\\	複数形チェック
\\	[名詞]	支配権、支配	
\\	複数形チェック
\\	[動詞]	ぶら下がる、ぶら下げる	
\\	[形容詞]	休眠中の、休止中の	
\\	[名詞]	破片、がれき	
\\	複数形チェック
\\	[名詞]	滴下、したたり	
\\	「ハナ出てる」「ご、ごめん」	
\\	複数形チェック
\\	[名詞]	債務者	
\\	複数形チェック
\\	[名詞]	一滴、しずく	
\\	生徒の数がだんだん減ってきた。	
\\	複数形チェック
\\	[名詞]	死亡	
\\	複数形チェック
\\	[名詞]	ドラム奏者	
\\	複数形チェック
\\	[名詞]	故人	
\\	彼は昨日この世から去った。	
\\	複数形チェック
\\	[形容詞]	二重の	
\\	彼は二重人格者だ。	
\\	[動詞]	推論する	
\\	[名詞]	アヒルの子	
\\	複数形チェック
\\	[名詞]	債務不履行	
\\	水道料金を滞納すると給水を停止されるということをお聞きしましたが本当ですか?	
\\	複数形チェック
\\	[名詞]	デュエット	
\\	その姉妹はヴァイオリンとピアノの二重奏を演奏した。	
\\	複数形チェック
\\	[動詞]	延期する、延ばす	
\\	[名詞]	2人組	
\\	複数形チェック
\\	[名詞]	抵抗、挑戦	
\\	その恥知らずな抵抗活動のために、彼は危うく命を失いかけた。	
\\	複数形チェック
\\	[形容詞]	陰うつな	
\\	[形容詞]	反抗的な	
\\	彼の反抗的な態度に上司は腹を立てた。	
\\	[動詞]	次第に減少する	
\\	[形容詞]	不足した、欠陥のある	
\\	[動詞]	経済的に使用する	
\\	[形容詞]	決定的な、最終的な	
\\	[形容詞]	気味の悪い	
\\	[動詞]	そらせる	
\\	[形容詞]	自己中心的な	
\\	[名詞]	非行少年	
\\	妹の彼氏はハリー・ポッターがちょっとグレたような感じだ。	
\\	複数形チェック
\\	[形容詞]	自己中心癖の	
\\	[動詞]	欺く、惑わす	
\\	[動詞]	追い出す	
\\	[名詞]	有権者	
\\	複数形チェック
\\	[形容詞]	欠陥のある	
\\	描画が不良です。	
\\	[形容詞]	気高い	
\\	そのうち、小さな村落の真上を高架線が行ったり来たりするようになりました。	
\\	[形容詞]	実行可能な	
\\	[動詞]	誘い出す、導き出す	
\\	市民の皆様の御意見をちょうだいする予定です。	
\\	[動詞]	のふりをする	
\\	何カマトトぶってんだアホ!	
\\	[形容詞]	適格で	
\\	大統領には彼が適任だ。	
\\	[形容詞]	猛烈な	
\\	[動詞]	逃れる	
\\	[動詞]	捜し出す、探索する	
\\	[形容詞]	捕らえにくい	
\\	[名詞]	熱情、白熱	
\\	複数形チェック
\\	[名詞]	通商禁止	
\\	複数形チェック
\\	[名詞]	封建制度	
\\	複数形チェック
\\	[動詞]	はめ込む	
\\	[名詞]	忠誠	
\\	複数形チェック
\\	[名詞]	胎児、胚	
\\	複数形チェック
\\	[形容詞]	比喩的な	
\\	[名詞]	高名	
\\	彼女は黒幕としてかかわっていた。	
\\	複数形チェック
\\	[名詞]	洗練	
\\	複数形チェック
\\	[形容詞]	実験に基づいた	
\\	[名詞]	発火	
\\	過去に陸上自衛隊演習場で行われた射撃訓練で、砲弾が場外に着弾する事故が起きている。	
\\	複数形チェック
\\	[動詞]	真似る	
\\	[名詞]	修理、固定	
\\	お母さんは今、夕食を作っている。	
\\	複数形チェック
\\	[動詞]	制定する	
\\	[名詞]	わき腹	
\\	複数形チェック
\\	[動詞]	取り巻く、包囲する	
\\	[名詞]	炎の輝き	
\\	複数形チェック
\\	[形容詞]	絶滅の危機に瀕した	
\\	[形容詞]	お世辞の	
\\	[動詞]	彫る	
\\	凝った彫刻のしてある鞘に収まった長剣。	
\\	[形容詞]	欠点のない	
\\	君の英語は完璧だ。	
\\	[動詞]	飲み込む	
\\	[名詞]	映画	
\\	複数形チェック
\\	[形容詞]	不思議な、不可解な	
\\	[動詞]	明滅する、ちらちらす	
\\	蛍光灯がチカチカしてるな。新しいのと交換しないとダメかな。	
\\	[動詞]	続いて起こる	
\\	そのニュースで非常に混乱した。	
\\	[名詞]	火打ち石	
\\	複数形チェック
\\	[動詞]	必然的に伴う	
\\	[動詞]	ふざける、いちゃつ	
\\	[名詞]	権利、資格	
\\	複数形チェック
\\	[名詞]	床材	
\\	複数形チェック
\\	[動詞]	見る、予見する	
\\	[名詞]	ばたばたする音	
\\	複数形チェック
\\	[動詞]	心に描く	
\\	[名詞]	フロッピーディスク	
\\	「フロッピーディスクって知ってる?」「知ってますよ」「使ったことある?」「使ったことはないですね」	
\\	複数形チェック
\\	[名詞]	使節、使者	
\\	複数形チェック
\\	[名詞]	開花	
\\	池のほとりに花ショウブが綺麗に咲いていました。	
\\	複数形チェック
\\	[名詞]	酵素	
\\	複数形チェック
\\	[形容詞]	焦点の	
\\	[動詞]	等しくする	
\\	[名詞]	足がかり	
\\	複数形チェック
\\	[名詞]	根絶する	
\\	複数形チェック
\\	[形容詞]	第一の、一流の	
\\	彼は心臓外科の第一人者だ。	
\\	[名詞]	直立、起立	
\\	朝立ちやしょんべんまでの命かな。	
\\	複数形チェック
\\	[名詞]	先見の明	
\\	ジェシーはチャールズの先見の明をほめたたえた。	
\\	複数形チェック
\\	[動詞]	腐食する、浸食する	
\\	[動詞]	形作る、偽造する	
\\	偽造カードと暗証番号が揃えば、口座にある限りの現金が引き出されてしまう。	
\\	[名詞]	浸食	
\\	湿地帯は生態系において水の浄化、洪水と侵食の防止、野生生物への食料とすみかの提供という重要な役割を演じる。	
\\	複数形チェック
\\	[動詞]	強化する、活力を与える	
\\	[名詞]	不規則な、風変わりな	
\\	複数形チェック
\\	[名詞]	要塞	
\\	要塞は敵の手にあった。	
\\	複数形チェック
\\	[名詞]	有頂点	
\\	複数形チェック
\\	[名詞]	骨折	
\\	複数形チェック
\\	[動詞]	憤慨させる	
\\	[形容詞]	断片的な	
\\	[動詞]	例示する	
\\	[名詞]	選挙権、販売権	
\\	複数形チェック
\\	[形容詞]	存在の、実存主義の	
\\	[名詞]	兄弟愛、友愛	
\\	彼は友愛会の会員だ。	
\\	複数形チェック
\\	[名詞]	大量移住	
\\	複数形チェック
\\	[形容詞]	はらんで、伴って	
\\	[名詞]	期待、予期	
\\	最も寿命が長い国の主要な死亡原因に自殺がある。	
\\	複数形チェック
\\	[動詞]	ぼろぼろにする	
\\	コロンボ刑事は、いつもすり切れたレインコートを着ている。	
\\	[名詞]	志望者	
\\	複数形チェック
\\	[名詞]	乱心、熱狂、狂乱	
\\	複数形チェック
\\	[名詞]	展覧会	
\\	複数形チェック
\\	[形容詞]	淡水の	
\\	[形容詞]	はっきりと、明白に	
\\	[動詞]	いらいらさせる、悩ます	
\\	焦っちゃだめだよ。最初から完璧にできる人なんていないんだから。	
\\	[名詞]	除名、排除	
\\	複数形チェック
\\	[名詞]	余分な飾り、虚飾	
\\	複数形チェック
\\	[名詞]	抽出、摘出	
\\	複数形チェック
\\	[形容詞]	霜の降りる、凍る寒さの	
\\	[動詞]	組み立てて製造する	
\\	[形容詞]	実を結ばない	
\\	水掛け論はやめよう。	
\\	[名詞]	かけ、外見	
\\	複数形チェック
\\	[名詞]	逃亡者	
\\	その逃亡犯はここから50マイル離れた小さな町に現れたが、それからまた姿を消してしまった。	
\\	複数形チェック
\\	[名詞]	宝石の小面	
\\	天才の一面は明らかに醜聞を起し得る才能である。	
\\	複数形チェック
\\	[動詞]	手探りする	
\\	[形容詞]	容易な	
\\	[名詞]	煙	
\\	複数形チェック
\\	[名詞]	へり取り、縁飾り	
\\	中学生が英語を学ぶ際の最難関の一つが関係代名詞です。	
\\	複数形チェック
\\	[名詞]	資金調達	
\\	複数形チェック
\\	[名詞]	狂信者	
\\	複数形チェック
\\	[名詞]	真菌類、菌類	
\\	複数形チェック
\\	[形容詞]	運命を決する	
\\	[名詞]	家具の取付、供給	
\\	湿地帯は生態系において水の浄化、洪水と侵食の防止、野生生物への食料とすみかの提供という重要な役割を演じる。	
\\	複数形チェック
\\	[名詞]	でぶちゃん	
\\	複数形チェック
\\	[動詞]	溶解(融解)させる	
\\	くそっ、ヒューズがとんだよ!	
\\	[名詞]	便利な小道具	
\\	これは主婦の手間を省く便利な器具です。	
\\	複数形チェック
\\	[名詞]	豚	
\\	思いっきりやりましょう。	
\\	複数形チェック
\\	[名詞]	口止め	
\\	複数形チェック
\\	[形容詞]	家庭的な、魅力の無い	
\\	[形容詞]	勇敢な	
\\	[名詞]	農場	
\\	複数形チェック
\\	[動詞]	ぽかんと大口をあける	
\\	[名詞]	群れ、寄せ集め	
\\	複数形チェック
\\	[形容詞]	胃の	
\\	彼は胃ガンで死んだ。	
\\	[名詞]	色合い、色彩	
\\	複数形チェック
\\	[名詞]	まあ	
\\	それはぜいたくな悩みだよ。	
\\	複数形チェック
\\	[名詞]	ごまかし、誇大広告	
\\	複数形チェック
\\	[名詞]	概略、概論	
\\	複数形チェック
\\	[名詞]	病的興奮、熱狂	
\\	複数形チェック
\\	[形容詞]	親切な	
\\	[名詞]	氷山	
\\	タイタニックは氷山に衝突した。	
\\	複数形チェック
\\	[形容詞]	幾何学的な	
\\	[形容詞]	差し迫った	
\\	今にも嵐になりそうだ。	
\\	[形容詞]	幽霊のような	
\\	[形容詞]	動かせない	
\\	[名詞]	請負などの仕事	
\\	複数形チェック
\\	[名詞]	障害	
\\	複数形チェック
\\	[名詞]	〔酒の〕ジン	
\\	ジン・トニックをください。	
\\	複数形チェック
\\	[形容詞]	差し迫った	
\\	[形容詞]	用心深い	
\\	[名詞]	命令法	
\\	複数形チェック
\\	[形容詞]	魅力に満ちた	
\\	彼女はグラマーだね。	
\\	[名詞]	帝国主義者	
\\	複数形チェック
\\	[名詞]	魅力、性的魅力	
\\	複数形チェック
\\	[名詞]	刺激	
\\	複数形チェック
\\	[名詞]	腺	
\\	複数形チェック
\\	[動詞]	含蓄する、巻き込む	
\\	[名詞]	光沢	
\\	複数形チェック
\\	[動詞]	即興で演奏する	
\\	忘年会の余興で大喜利をやることになった。	
\\	[名詞]	用語集	
\\	複数形チェック
\\	[副詞]	不注意にも	
\\	[形容詞]	光沢のある	
\\	[名詞]	就任演説	
\\	複数形チェック
\\	[名詞]	花崗岩	
\\	複数形チェック
\\	[動詞]	就任させる	
\\	[名詞]	墓地	
\\	フレデイはこの一ヶ月というものずっと夜勤だったため、正常な時間に働いている友人達に会う機会がまるでなかった。	
\\	複数形チェック
\\	[名詞]	就任	
\\	複数形チェック
\\	[名詞]	引力、重力	
\\	複数形チェック
\\	[名詞]	香	
\\	複数形チェック
\\	[形容詞]	油でよごれた	
\\	[形容詞]	比較できない、無比の	
\\	[名詞]	手榴弾、催涙弾	
\\	僕には手榴弾がある。	
\\	複数形チェック
\\	[形容詞]	無能な	
\\	世の中は藪医者だらけ。	
\\	[名詞]	快速の猟犬	
\\	複数形チェック
\\	[形容詞]	目立たない、地味な	
\\	[名詞]	苦情,不満	
\\	其の日は一日淋しいペンの音を聞いて暮した。其の間には折々千代々々と云う声も聞えた。文鳥も淋しいから鳴くのではなかろうかと考えた。然し縁側へ出て見ると、二本の留り木の間を、彼方へ飛んだり、此方へ飛んだり、絶間なく行きつ戻りつしている。少しも不平らしい様子はなかった。	
\\	複数形チェック
\\	[名詞]	現職者	
\\	複数形チェック
\\	[名詞]	しかめっ面	
\\	複数形チェック
\\	[形容詞]	不明確な	
\\	[名詞]	溝	
\\	ついに我がデジタルグルーヴクラブの定款が完成しました。	
\\	複数形チェック
\\	[動詞]	起訴する	
\\	[形容詞]	根拠のない	
\\	[名詞]	起訴	
\\	複数形チェック
\\	[名詞]	同業組合	
\\	評議会で正規ギルドに認めてもらえばいいのに。	
\\	複数形チェック
\\	[形容詞]	その土地固有の	
\\	[形容詞]	罪のない	
\\	[名詞]	消化不良	
\\	複数形チェック
\\	[名詞]	体育、体操	
\\	彼は体操が得意だ。	
\\	複数形チェック
\\	[形容詞]	はっきりしない	
\\	[動詞]	たたき切る	
\\	妻の乾咳が気になります。	
\\	[名詞]	誘因	
\\	複数形チェック
\\	[形容詞]	元気のいい	
\\	[名詞]	慣性、惰性、不活発	
\\	複数形チェック
\\	[動詞]	阻止する、妨げる	
\\	音楽を聞きながら書き物をする人もいるが、音楽を聞きながらだと能率が下がるという人もいる。	
\\	[形容詞]	悪名高い	
\\	[名詞]	手仕事、仕業	
\\	複数形チェック
\\	[名詞]	歩兵	
\\	複数形チェック
\\	[形容詞]	高慢な	
\\	[名詞]	無限	
\\	複数形チェック
\\	[形容詞]	幽霊の出る	
\\	トムの家には幽霊が出る。	
\\	[形容詞]	怒りをかきたてる	
\\	[名詞]	大破壊	
\\	彼の発言は混乱を引き起こした。	
\\	複数形チェック
\\	[動詞]	ふくらませる	
\\	[名詞]	校⻑	
\\	複数形チェック
\\	[名詞]	インフルエンザ	
\\	私の父は流感にかかっている。	
\\	複数形チェック
\\	[名詞]	心痛	
\\	そして今夜、彼女がアメリカで100年にわたって見てきた全てのものについて考えています。悲嘆や希望、もがきや前進、出来やしないと言われた回数、そしてあのアメリカ的信条を掲げて前進した人々、そう我々はできる、という。	
\\	複数形チェック
\\	[名詞]	流入	
\\	複数形チェック
\\	[名詞]	断腸の思い	
\\	別に失恋が好きなわけじゃないけど、失恋ソングはすごく好きなんだよね。	
\\	複数形チェック
\\	[名詞]	感謝しないこと	
\\	複数形チェック
\\	[形容詞]	無情な	
\\	病人にそんなことを言うなんて、彼には思いやりがなかった。	
\\	[名詞]	抑制、抑圧	
\\	複数形チェック
\\	[動詞] [不規則動詞]	持ち上げる	
\\	盛大な溜息をつきながら、テーブルの上にどさどさとパンを置き、椅子にどっかりと腰掛ける。	
\\	[形容詞]	有害な	
\\	[名詞]	ヘクタール	
\\	複数形チェック
\\	[名詞]	収容者、同室者	
\\	複数形チェック
\\	[名詞]	操舵装置	
\\	複数形チェック
\\	[名詞]	最も内側の部分	
\\	心の中まで理解して欲しいとは思っていませんが、どうしたら旦那の関心を得る事ができるんでしょうか。	
\\	複数形チェック
\\	[形容詞]	薬草の、草の	
\\	[名詞]	不安定、不安定状態	
\\	複数形チェック
\\	[形容詞]	ぞっとするような	
\\	トムがとんでもない二日酔いをしてしまいました。	
\\	[名詞]	分割払いの一回分	
\\	複数形チェック
\\	[形容詞]	階層制の	
\\	[動詞]	絶縁する	
\\	軟木の絶縁された部屋およびよいヒーターはサウナのための絶対必要である。	
\\	[動詞]	貯蔵する	
\\	私の友人は私の財産だ。ですので友人を蓄えたがる私の貪欲さを見逃してください。	
\\	[名詞]	保険会社・業者	
\\	複数形チェック
\\	[形容詞]	互いに頼り合う	
\\	[名詞]	委任、権限	
\\	複数形チェック
\\	[名詞]	媒介者、媒介、手段	
\\	複数形チェック
\\	[形容詞]	義務的な、強制の	
\\	学校教育の義務的側面は子どもの学習意欲を改善させる様々な研究の多くの取り組みの中ではめったに分析されない。	
\\	[名詞]	尋問	
\\	複数形チェック
\\	[名詞]	異常な熱中	
\\	複数形チェック
\\	[形容詞]	興味をそそる	
\\	それは面白い。	
\\	[名詞]	狂人、熱中者	
\\	複数形チェック
\\	[動詞]	無効にする	
\\	[名詞]	マントルピース、炉棚	
\\	複数形チェック
\\	[形容詞]	不変の	
\\	[名詞]	肥料	
\\	複数形チェック
\\	[動詞]	逆にする、反対にする	
\\	この先ネタバレ
\\	大丈夫な方はドラッグ(反転)して読んでね
\\	[名詞]	マーガリン	
\\	彼女はバターの代わりにマーガリンを使った。	
\\	複数形チェック
\\	[形容詞]	調査の	
\\	[形容詞]	海の	
\\	ボーイング社は海上自衛隊用の飛行艇を開発しました。	
\\	[形容詞]	無敵の	
\\	[名詞]	殉教者	
\\	複数形チェック
\\	[動詞]	訴える	
\\	[形容詞]	調和する	
\\	彼女はグリーンのコートを着、それに合ったミニスカートをはいていた。	
\\	[名詞]	イオン	
\\	接尾辞について、1:
\\	「行動、状態、過程、結果」などの意を表す名詞を作る。	
\\	複数形チェック
\\	[名詞]	母であること、母性	
\\	複数形チェック
\\	[名詞]	専門語	
\\	複数形チェック
\\	[名詞]	調停	
\\	複数形チェック
\\	[動詞]	危険にさらす	
\\	[形容詞]	メロドラマ風な	
\\	[動詞]	揺すぶる	
\\	車はがたがた揺れた。	
\\	[名詞]	良き指導者	
\\	複数形チェック
\\	[名詞]	巨大なもの	
\\	私はジャンボ機が離陸するのを見た。	
\\	複数形チェック
\\	[名詞]	(女の)人魚、マーメイド	
\\	複数形チェック
\\	[動詞]	遅れる	
\\	時差ボケで辛い。	
\\	[形容詞]	秩序立った	
\\	トムの仕事の仕方はとても几帳面だ。	
\\	[形容詞]	悲しむべき、みじめな	
\\	[名詞]	方法、やり方	
\\	データギャップを打開するように簡単な方法論を開発した。	
\\	複数形チェック
\\	[名詞]	槍、槍騎兵	
\\	複数形チェック
\\	[形容詞]	極めて注意深い	
\\	[名詞]	幼虫(単数)	
\\	複数形チェック
\\	[形容詞]	非常に小さな	
\\	[形容詞]	隠れている、見えない	
\\	[名詞]	中部	
\\	複数形チェック
\\	[動詞]	せっけんの泡を塗る	
\\	ひげをそる前に石けんをつける。	
\\	[名詞]	夏至	
\\	複数形チェック
\\	[形容詞]	おかしい	
\\	[名詞]	助産婦	
\\	病室には白い服をまとった看護婦と産婆とが出産の準備を致して居りました。	
\\	複数形チェック
\\	[名詞]	月桂冠、栄誉	
\\	複数形チェック
\\	[名詞]	移住者、渡り鳥	
\\	複数形チェック
\\	[名詞]	立法	
\\	複数形チェック
\\	[名詞]	⺠兵	
\\	複数形チェック
\\	[形容詞]	だらしのない	
\\	[名詞]	ひき臼	
\\	複数形チェック
\\	[名詞]	傾き	
\\	彼は壁にもたれていた。	
\\	複数形チェック
\\	[名詞]	こま切れ肉	
\\	彼はずけずけものをいうからね。	
\\	複数形チェック
\\	[名詞]	合法	
\\	複数形チェック
\\	[形容詞]	最少量の	
\\	水、ガス、電気が遠くから運ばれて我々の要求を満たすためになんなく供給されているように、いつか画像と音楽はちょっとした合図みたいなシンプルな手の仕草によって提供されることにもなります。	
\\	[名詞]	立法者、立法府の議員	
\\	複数形チェック
\\	[副詞]	細かく	
\\	[名詞]	てこの作用、影響力	
\\	複数形チェック
\\	[名詞]	見当違いの考え	
\\	複数形チェック
\\	[動詞]	課税する	
\\	日本では、ほとんどの物やサービスに5%の消費税がかけられる。	
\\	[名詞]	不正行為	
\\	彼は自分の違法行為を恥じていない。	
\\	複数形チェック
\\	[名詞]	連絡	
\\	複数形チェック
\\	[名詞]	災難	
\\	複数形チェック
\\	[動詞]	自由化する	
\\	[名詞]	現代主義、モダニズム	
\\	「これ、トムが訳したんだよね?」「うん、そうだけど」「モデム主義って何?」「分かんないけど原文に 
\\	って書いてあったから」「ほんとに?」「ほんとだって。ほら」「あ、なんだ、
\\	じゃん」	
\\	複数形チェック
\\	[名詞]	薄紫色	
\\	複数形チェック
\\	[名詞]	現代主義者	
\\	複数形チェック
\\	[名詞]	構成、番組予定表	
\\	背の順に並んでください。	
\\	複数形チェック
\\	[動詞]	湿らせる	
\\	[名詞]	流動資金	
\\	複数形チェック
\\	[形容詞]	分子の	
\\	[名詞]	家畜	
\\	複数形チェック
\\	[形容詞]	重大な	
\\	[名詞]	バッタ	
\\	複数形チェック
\\	[名詞]	修道院	
\\	複数形チェック
\\	[名詞]	屋根裏	
\\	複数形チェック
\\	[名詞]	一本調子、単調さ	
\\	複数形チェック
\\	[名詞]	ロゴ、モットー	
\\	私はタトエバのロゴが好きだ。	
\\	複数形チェック
\\	[形容詞]	記念碑の	
\\	[形容詞]	積年の	
\\	[名詞]	支払猶予期間	
\\	複数形チェック
\\	[名詞]	はた織り機	
\\	複数形チェック
\\	[名詞]	モルタル、しっくい	
\\	複数形チェック
\\	[名詞]	戦利品、盗品	
\\	複数形チェック
\\	[名詞]	イスラム教寺院	
\\	トムは何考えたのかな。	
\\	複数形チェック
\\	[形容詞]	ひどい、劣った	
\\	こいつの性根が悪いから非モテなんだろ。	
\\	[名詞]	モーテル 《旅行者の宿泊所	
\\	複数形チェック
\\	[名詞]	小康	
\\	雨が小やみになった。	
\\	複数形チェック
\\	[形容詞]	母の	
\\	[名詞]	壮大さ	
\\	複数形チェック
\\	[名詞]	登山家、山岳⺠	
\\	複数形チェック
\\	[名詞]	大型コンピューター	
\\	複数形チェック
\\	[名詞]	混乱	
\\	複数形チェック
\\	[形容詞]	悪性の、有害な	
\\	[名詞]	マグ	
\\	私のマグカップ取って。	
\\	複数形チェック
\\	[名詞]	⻨芽酒	
\\	ビールは麦芽から醸造される。	
\\	複数形チェック
\\	[形容詞]	多文化的な	
\\	[形容詞]	扱いやすい	
\\	[形容詞]	ありふれた	
\\	[形容詞]	経営上の	
\\	[名詞]	召集	
\\	思い切って聞き出せない、弱腰な僕。	
\\	複数形チェック
\\	[名詞]	突然変異、変化	
\\	遺伝子の突然変異がなければ、進化は起こりえない。	
\\	複数形チェック
\\	[名詞]	水増し	
\\	複数形チェック
\\	[形容詞]	秘法の、秘伝の	
\\	[名詞]	野外劇、見せびらかし	
\\	複数形チェック
\\	[形容詞]	架空の、神話の	
\\	[名詞]	判例、典型	
\\	複数形チェック
\\	[名詞]	麻酔剤、麻薬	
\\	複数形チェック
\\	[名詞]	麻痺	
\\	複数形チェック
\\	[名詞]	自然主義者、博物学者	
\\	複数形チェック
\\	[名詞]	媒介変数、限界	
\\	複数形チェック
\\	[動詞]	必要とする	
\\	[形容詞]	最高の、主要な	
\\	[形容詞]	不注意な	
\\	[動詞]	煎る、乾ききらせる	
\\	[名詞]	ネオン 《元素記号 
\\	複数形チェック
\\	[名詞]	括弧	
\\	複数形チェック
\\	[形容詞]	神経質な	
\\	[名詞]	滑稽なものまね	
\\	ハリソンがこの作品をパロディとして意図していなかったという証拠は、エヴァンズ夫人へ当てた手紙に見ることができる。	
\\	複数形チェック
\\	[名詞]	中性子	
\\	複数形チェック
\\	[名詞]	特殊性、私事	
\\	複数形チェック
\\	[名詞]	新聞販売業者	
\\	複数形チェック
\\	[名詞]	熱心な支持者	
\\	複数形チェック
\\	[名詞]	ニッチ市場、隙間産業	
\\	複数形チェック
\\	[名詞]	パステル	
\\	複数形チェック
\\	[形容詞]	遊牧の	
\\	[形容詞]	父系の	
\\	父方の祖母は茶道を嗜んでいる。	
\\	[名詞]	指名された人	
\\	複数形チェック
\\	[名詞]	質、抵当物、人質	
\\	僕は首をひねる。このポーンではなくて、こっちのナイトを・・・。	
\\	複数形チェック
\\	[名詞]	正午	
\\	真昼の太陽がテニスコートを目も眩むほどに照らしあげた。	
\\	複数形チェック
\\	[名詞]	報酬、賄路	
\\	複数形チェック
\\	[名詞]	郷愁、懐かしさ	
\\	複数形チェック
\\	[名詞]	平和維持	
\\	複数形チェック
\\	[名詞]	刻み目	
\\	複数形チェック
\\	[名詞]	平和な時	
\\	複数形チェック
\\	[名詞]	微妙な差異	
\\	複数形チェック
\\	[形容詞]	刑罰の	
\\	[動詞]	麻痺させる	
\\	トムは元気。	
\\	[名詞]	ペンダント	
\\	君にこのペンダントをあげるよ。	
\\	複数形チェック
\\	[名詞]	妖精	
\\	せみの幼虫が、くもの巣につかまって羽化をした。	
\\	複数形チェック
\\	[名詞]	振り子	
\\	複数形チェック
\\	[名詞]	オートミール	
\\	トムは母親のオートミールクッキーのレシピをメアリーに教えた。	
\\	複数形チェック
\\	[名詞]	ペニス	
\\	陰茎は雄の生殖器のひとつ。	
\\	複数形チェック
\\	[形容詞]	不快な	
\\	[名詞]	元気	
\\	複数形チェック
\\	[動詞]	消し去る、痕跡をなくす	
\\	[形容詞]	芸当のできる	
\\	トニーは演じている。	
\\	[形容詞]	忘れっぽい、ぼんやりした	
\\	女というものは、あなたがうまくやった100ものことには気付かず、ただ失敗したことを暴き立てる、そういうことが出来る生き物だ。	
\\	[名詞]	周辺機器	
\\	2010年のデンマークの今年の言葉は、「火山灰の雲」、「デンマークの周辺」、「ブブゼラ」、そして、「ウィキリークス」です。	
\\	複数形チェック
\\	[形容詞]	用心深い、機敏な	
\\	[名詞]	当惑、困った事、難局	
\\	複数形チェック
\\	[名詞]	神秘、オカルトの	
\\	ネットで知り合ったこの女性、プロフィールでは分からなかったけど、前世とか引き寄せといったオカルト的なことがらが興味の中心みたいだ。	
\\	複数形チェック
\\	[名詞]	発汗	
\\	複数形チェック
\\	[名詞]	オクターブ	
\\	俺キーチェンジしたら音取れないんだよね。原曲のキーで行ってたまにオクターブ下げて歌うっていう方がいい。	
\\	複数形チェック
\\	[動詞]	関係する	
\\	[名詞]	開始、着手	
\\	複数形チェック
\\	[形容詞]	適切な、直接関係する	
\\	[形容詞]	時宜を得た、好都合な	
\\	あなたはちょうどよい折に来た。	
\\	[形容詞]	広がる	
\\	[動詞]	定める	
\\	[動詞]	判断などを誤らせる	
\\	エロチックな描写が倒錯的過ぎてこの人の本は俺には合わない。	
\\	[形容詞]	飾り立てた	
\\	[名詞]	ペチコート	
\\	複数形チェック
\\	[名詞]	正統派	
\\	複数形チェック
\\	[名詞]	フェニックス	
\\	複数形チェック
\\	[動詞]	解雇する、解任する	
\\	[形容詞]	にせの	
\\	[名詞]	爆発、ほとばしり	
\\	複数形チェック
\\	[名詞]	おしっこ	
\\	君は僕をいらいらさせる。	
\\	複数形チェック
\\	[名詞]	追い出された人	
\\	複数形チェック
\\	[形容詞]	無慈悲な	
\\	[名詞]	叫び、抗議	
\\	複数形チェック
\\	[形容詞]	もっともらしい	
\\	[形容詞]	明白な	
\\	あっさり断られると思いきや、彼女は承諾してくれました。	
\\	[名詞]	おもちゃ	
\\	複数形チェック
\\	[名詞]	最初	
\\	もとより私は最悪の事態を覚悟していた。	
\\	複数形チェック
\\	[形容詞]	愉快な	
\\	[動詞]	まさる	
\\	[名詞]	ひどい状態、苦境	
\\	複数形チェック
\\	[動詞]	混雑させる	
\\	彼は過密スケジュールを都合して、私の舞台を見に来てくれた。	
\\	[動詞]	略奪する	
\\	[形容詞]	一面に茂った	
\\	庭には雑草が生い茂っていた。	
\\	[名詞]	つり合い、落ち着き	
\\	複数形チェック
\\	[名詞]	精密検査	
\\	複数形チェック
\\	[名詞]	ポーカー	
\\	トムさんはポーカーが全然好きじゃないんです。	
\\	複数形チェック
\\	[動詞] [不規則動詞]	小耳にはさむ	
\\	彼は偶然その会話を耳にした。	
\\	[形容詞]	賢明な	
\\	[動詞]	荷を積みすぎる	
\\	[名詞]	工芸学校、工芸	
\\	複数形チェック
\\	[動詞] [不規則動詞]	支払いすぎる	
\\	[名詞]	磁器、磁器製品	
\\	祖父が大事にしていた伊万里焼の皿は、ずっしりと重かった。	
\\	複数形チェック
\\	[名詞]	人口過剰	
\\	複数形チェック
\\	[名詞]	無効にすること	
\\	憲法の規定に従い衆院の議決が参院に優越する。	
\\	複数形チェック
\\	[名詞]	陶工	
\\	ハリポタは前に全部読んだよ。	
\\	複数形チェック
\\	[名詞]	超過	
\\	せっかくの広い庭なのに、草ぼうぼうじゃ台無しだな。	
\\	複数形チェック
\\	[形容詞]	粉の	
\\	[動詞]	大げさに言う	
\\	[形容詞]	実行できる、実用向きの	
\\	[名詞]	平和主義者	
\\	複数形チェック
\\	[形容詞]	称賛に値する	
\\	[名詞]	クルマエビ	
\\	複数形チェック
\\	[名詞]	依頼、頼みとするもの	
\\	複数形チェック
\\	[動詞]	排除する、防止する	
\\	[形容詞]	再発する	
\\	[名詞]	奪い取る人、捕食者	
\\	攻撃性のもっとも顕著なタイプは捕食者と被食者間によく見られる。	
\\	複数形チェック
\\	[動詞]	再選する	
\\	彼は市長に再選された。	
\\	[形容詞]	予報の	
\\	[名詞]	国⺠投票	
\\	政府が住民に問う一般投票を実施しました。	
\\	複数形チェック
\\	[形容詞]	圧倒的に、主として	
\\	[動詞]	改装する	
\\	[名詞]	初演	
\\	複数形チェック
\\	[動詞]	反論する	
\\	進化という科学的真理が反対しようもないほど確証されるもので、実質的に論駁できない。	
\\	[形容詞]	夢中になった	
\\	頭がいっぱいです。	
\\	[名詞]	取り締まる人	
\\	複数形チェック
\\	[動詞]	夢中にさせる	
\\	頭がいっぱいです。	
\\	[形容詞]	規定の	
\\	[形容詞]	推定できる	
\\	[動詞]	回復させる	
\\	[名詞]	推定	
\\	複数形チェック
\\	[形容詞]	親類の、文法関係を示す	
\\	[形容詞]	主要な	
\\	[名詞]	相対性	
\\	アルベルト・アインシュタインの相対性理論は、世界で最も有名な法則である。	
\\	複数形チェック
\\	[動詞]	獲得する	
\\	その後まもなく私はクセノフォンの「ソクラテスの思い出」を手に入れたが、この中にはそれと同じ論法の実例がたくさんある。	
\\	[動詞]	追放する、追いやる	
\\	[動詞]	刺激する	
\\	[動詞]	心が解ける、弱まる	
\\	[名詞]	発起人、投光器	
\\	複数形チェック
\\	[形容詞]	しつこい、情け容赦のない	
\\	[名詞]	拡散	
\\	複数形チェック
\\	[形容詞]	頼って	
\\	[名詞]	散歩	
\\	複数形チェック
\\	[名詞]	送金、送金額	
\\	複数形チェック
\\	[名詞]	支持者	
\\	複数形チェック
\\	[動詞]	廃止する、撤回する	
\\	[形容詞]	つり合いのとれた	
\\	[名詞]	レパートリー、演奏曲目	
\\	複数形チェック
\\	[名詞]	礼儀正しさ、適正	
\\	複数形チェック
\\	[形容詞]	繰返しの	
\\	彼の仕事は反復作業だ。	
\\	[名詞]	保護貿易主義	
\\	複数形チェック
\\	[名詞]	収納場所、倉庫	
\\	複数形チェック
\\	[形容詞]	諺の、諺にある	
\\	[動詞] [不規則動詞]	再び売る	
\\	[形容詞]	暫定的な	
\\	[動詞] [不規則動詞]	置き直す、はめ直す	
\\	停電があったのでレコーダーの日時を設定しなおしたんだけど、1日間違って予約をミスってしまった。	
\\	[形容詞]	挑発的な、刺激的な	
\\	彼女の悩ましげな姿を見て、彼はすっかり夢中になってしまいました。	
\\	[名詞]	樹脂、合成樹脂	
\\	複数形チェック
\\	[名詞]	うろつき	
\\	複数形チェック
\\	[名詞]	共鳴、共振	
\\	四種類の金属を使うことで特定の共振を抑えます。	
\\	複数形チェック
\\	[名詞]	近接	
\\	複数形チェック
\\	[名詞]	再起動	
\\	セーブされたゲームは破損しているようで、利用できません。ゲームを終了した後、セーブデータを消去してからやり直してください。	
\\	複数形チェック
\\	[名詞]	代理	
\\	複数形チェック
\\	[名詞]	再開、回収、回復	
\\	複数形チェック
\\	[形容詞]	精神医学の	
\\	彼女は精神保健福祉士を目指していて、大学では特に精神障害者のインテグレーションについて勉強している。	
\\	[動詞]	遅らせる	
\\	いい年してラノベ読んでる奴って普通に知恵遅れの障害者だろうな。	
\\	[名詞]	説教壇、牧師	
\\	複数形チェック
\\	[形容詞]	知能の遅れた	
\\	[名詞]	不良	
\\	複数形チェック
\\	[名詞]	再考	
\\	複数形チェック
\\	[名詞]	意図	
\\	複数形チェック
\\	[形容詞]	内気な、まもなく引退する	
\\	[名詞]	ピグミー 《アフリカの原住⺠》	
\\	複数形チェック
\\	[名詞]	回復、挽回、検索	
\\	複数形チェック
\\	[形容詞]	けんか好きな、怒りっぽい	
\\	[動詞]	再結合する、再会する	
\\	[名詞]	採石場、源泉	
\\	複数形チェック
\\	[動詞]	崇敬する、あがめる	
\\	[形容詞]	疑わしい	
\\	私は彼の経歴を疑わしいと思っている。	
\\	[名詞]	裏表着用できる衣服	
\\	複数形チェック
\\	[名詞]	かみたばこ	
\\	複数形チェック
\\	[動詞]	逆戻りする	
\\	[名詞]	競馬用走路	
\\	複数形チェック
\\	[名詞]	批評家	
\\	複数形チェック
\\	[動詞]	放射する	
\\	[動詞]	革命を起こす	
\\	[名詞]	散歩	
\\	複数形チェック
\\	[名詞]	韻	
\\	複数形チェック
\\	[動詞]	批准する	
\\	[名詞]	不和	
\\	複数形チェック
\\	[名詞]	割り当て、配給品	
\\	複数形チェック
\\	[名詞]	円形交差点	
\\	複数形チェック
\\	[名詞]	ワタリガラス	
\\	複数形チェック
\\	[名詞]	王制主義者	
\\	複数形チェック
\\	[形容詞]	反動的な、反作用的な	
\\	[名詞]	砕石、瓦礫	
\\	校舎の跡地は瓦礫の山となっている。	
\\	複数形チェック
\\	[名詞]	再保証、安心	
\\	複数形チェック
\\	[名詞]	ルビー、真紅の	
\\	複数形チェック
\\	[名詞]	一部払い戻し、割戻し	
\\	複数形チェック
\\	[名詞]	梯子(はしご)の横木	
\\	僕はすでにベルは鳴らした。	
\\	複数形チェック
\\	[名詞]	再生、復活	
\\	複数形チェック
\\	[名詞]	妨害行為、破壊行為	
\\	複数形チェック
\\	[動詞]	はね返る	
\\	[名詞]	サファリ	
\\	複数形チェック
\\	[形容詞]	受容する、理解力がある	
\\	[名詞]	武勇談、大河小説	
\\	家名を嵯峨と改姓したのは明治3年でした。	
\\	複数形チェック
\\	[動詞]	再充電する	
\\	電気自動車は家での再充電が可能です。	
\\	[名詞]	砂岩	
\\	複数形チェック
\\	[名詞]	朗読	
\\	複数形チェック
\\	[動詞]	ずぶぬれにする、飽和させる	
\\	ゲーム機の国内市場は飽和状態で、大きなヒットは期待できない。	
\\	[動詞]	取り戻す、回収する	
\\	その問いかけに答えるチャンスを今、私たちは手にしました。今この時こそが、私たちの瞬間です。今この時にこそ、私たちは人々がまた仕事につけるようにしなくてはなりません。子供たちのために、チャンスの扉を開かなくてはなりません。繁栄を取り戻し、平和を推進しなくてはなりません。今この時にこそ、アメリカの夢を取り戻し、基本的な真理を再確認しなくてはなりません。大勢の中にあって、私たちはひとつなのだと。息をし続ける限り、私たちは希望をもち続けるのだと。そして疑り深く悲観し否定する声に対しては、そんなことできないという人たちに対しては、ひとつ国民の魂を端的に象徴するあの不朽の信条でもって、必ずやこう答えましょう。	
\\	[名詞]	風味、趣き	
\\	たまにはビール以外のお酒をちびちび舐めるのも良いでしょう。	
\\	複数形チェック
\\	[名詞]	身代わり、贖罪のやぎ	
\\	複数形チェック
\\	[名詞]	すす	
\\	複数形チェック
\\	[動詞]	焦がす	
\\	金曜日からすごいあったかくなるらしいよ。	
\\	[副詞]	激しく、非常に	
\\	アメリカに行くだけで英語が話せるようになると思ったら大間違いだよ。	
\\	[動詞]	走り書きをする	
\\	[形容詞]	控えがちな	
\\	[形容詞]	金切り声をあげる	
\\	彼女の母親は叫び出した。	
\\	[名詞]	けいれん	
\\	複数形チェック
\\	[名詞]	落書き	
\\	複数形チェック
\\	[名詞]	卵、菌糸	
\\	惑星が命を生み出すために何が必要だろうか。	
\\	複数形チェック
\\	[動詞]	綿密に調べる	
\\	[名詞]	投機家	
\\	複数形チェック
\\	[名詞]	小走り	
\\	複数形チェック
\\	[名詞]	精子	
\\	変態だから,100個目の例文を精子で汚すことに歓びを覚えてしまう。	
\\	複数形チェック
\\	[形容詞]	縫い目のない	
\\	[動詞]	吹き出す、噴出する	
\\	[動詞]	焼く、焦がす	
\\	[名詞]	芽、若枝	
\\	苗は赤ちゃんを扱うようにやさしく扱ってください。	
\\	複数形チェック
\\	[形容詞]	秘書の	
\\	[名詞]	噴出	
\\	複数形チェック
\\	[形容詞]	秘密主義の	
\\	[形容詞]	全くの,正真正銘の	
\\	この台風の中全裸で外に出てみたい。	
\\	[名詞]	セダン	
\\	複数形チェック
\\	[形容詞]	法定の、法に触れる	
\\	[名詞]	選別機	
\\	複数形チェック
\\	[名詞]	英国通貨	
\\	複数形チェック
\\	[名詞]	外形、外観	
\\	複数形チェック
\\	[形容詞]	けちくさい	
\\	彼はとてもけちだ。	
\\	[名詞]	学期	
\\	二学期は終わった。	
\\	複数形チェック
\\	[動詞]	規定する、明記する	
\\	[名詞]	半円	
\\	複数形チェック
\\	[動詞]	燃料をくべる	
\\	[名詞]	年⻑	
\\	複数形チェック
\\	[名詞]	途中下車、立ち寄り先	
\\	複数形チェック
\\	[形容詞]	官能的な	
\\	[名詞]	停止、障害	
\\	複数形チェック
\\	[名詞]	ゴマ、ゴマの実	
\\	拓け胡麻!	
\\	複数形チェック
\\	[形容詞]	精力的な、奮闘的な	
\\	彼はただ、きついこと、難しいこと、骨の折れることを全て、自然に避けるのです。	
\\	[動詞]	断絶させる、隔てる	
\\	彼女はきっぱり断った。	
\\	[名詞]	支柱	
\\	複数形チェック
\\	[形容詞]	毛むくじゃらの	
\\	[名詞]	控え、半券	
\\	複数形チェック
\\	[名詞]	見せかけ、偽物	
\\	複数形チェック
\\	[動詞]	ちりばめる	
\\	[形容詞]	羊のような	
\\	[形容詞]	気絶させる、すばらしい	
\\	美しい!	
\\	[名詞]	ゆらめく光	
\\	複数形チェック
\\	[名詞]	サブカルチャー	
\\	複数形チェック
\\	[名詞]	難破、難破船	
\\	複数形チェック
\\	[動詞]	細分する、細分される	
\\	[名詞]	欠点、不足	
\\	複数形チェック
\\	[名詞]	分譲地	
\\	複数形チェック
\\	[名詞]	速記	
\\	彼女はその演説の速記をとった。	
\\	複数形チェック
\\	[動詞]	助成金を支給する	
\\	[名詞]	陳列ケース	
\\	複数形チェック
\\	[形容詞]	独立の、自立の	
\\	[名詞]	暴露、対決	
\\	複数形チェック
\\	[名詞]	請願者	
\\	複数形チェック
\\	[名詞]	断片、僅少	
\\	あなたの羽根を千切り棄ててしまいましょう。	
\\	複数形チェック
\\	[名詞]	硫⻩ 《非金属元素》	
\\	複数形チェック
\\	[名詞]	死体を包む布、覆い	
\\	複数形チェック
\\	[名詞]	日焼け、日焼け色	
\\	日焼けして痛い。	
\\	複数形チェック
\\	[名詞]	トランプを切ること	
\\	トランプをよく切ってください。	
\\	複数形チェック
\\	[形容詞]	余分の、あり余る	
\\	[名詞]	折返し運転	
\\	6時のリムジンなんです。	
\\	複数形チェック
\\	[名詞]	仮定	
\\	複数形チェック
\\	[名詞]	とろ火、とろ火の状態	
\\	鯖は煮ても焼いても美味しい。	
\\	複数形チェック
\\	[動詞]	打ち勝つ、乗り越える	
\\	[形容詞]	単純化しすぎた	
\\	[名詞]	監視、監督	
\\	君が挙動不審でいるから、観察していたのさ。	
\\	複数形チェック
\\	[動詞]	まねる	
\\	提案された手法を、三通りの仮想ケーススタディに適用します。	
\\	[名詞]	測量者、測量技師	
\\	複数形チェック
\\	[名詞]	懐疑論者	
\\	複数形チェック
\\	[名詞]	サスペンダー	
\\	複数形チェック
\\	[名詞]	頭蓋骨、頭	
\\	調べを進めるうちに、頭蓋骨が、何か重い一撃を受けて打ち砕かれているのが明らかになった。	
\\	複数形チェック
\\	[名詞]	生命を維持するもの	
\\	複数形チェック
\\	[名詞]	石板、スレート	
\\	複数形チェック
\\	[名詞]	それること	
\\	複数形チェック
\\	[形容詞]	つるつる滑る	
\\	一日頭を洗わないと髪の毛が油でベットリしてぺしゃんこになってしまいます。	
\\	[形容詞]	左右対称の	
\\	地球は赤道に関して対称ではない。	
\\	[形容詞]	ずさんな	
\\	トムは仕事が雑だ。	
\\	[名詞]	対称	
\\	均整とは倦怠のことだ。	
\\	複数形チェック
\\	[形容詞]	停滞した、不振な	
\\	スミスさんは自分のマイペースな馬を田舎道で走らせた。	
\\	[名詞]	総合	
\\	複数形チェック
\\	[動詞]	塗りつける、よごす	
\\	[名詞]	シロップ	
\\	トムはパンケーキにかかってるメープルシロップが好きだ。	
\\	複数形チェック
\\	[名詞]	上っ張り	
\\	複数形チェック
\\	[名詞]	つけ札、ラベル	
\\	この費用は誰が出すの。	
\\	複数形チェック
\\	[動詞]	窒息死させる	
\\	彼は煙に巻かれて窒息した。	
\\	[形容詞]	機転のきく、適切な	
\\	[名詞]	雪片	
\\	複数形チェック
\\	[名詞]	戦略、戦術	
\\	複数形チェック
\\	[動詞]	鼻で吸う	
\\	[形容詞]	戦術上の	
\\	[名詞]	社交的にすること	
\\	あらゆる社交はおのずから虚偽を必要とするものである。	
\\	複数形チェック
\\	[動詞]	もつれる、からみ合う	
\\	[名詞]	厳粛	
\\	複数形チェック
\\	[名詞]	ゴルフのティー	
\\	複数形チェック
\\	[名詞]	団結、連帯	
\\	複数形チェック
\\	[動詞]	テレビ放送する	
\\	時々、政治家の一人がテレビの討論会に出て傍聴者の意見を押さえつけようとする場面をみる。	
\\	[名詞]	在職期間	
\\	複数形チェック
\\	[形容詞]	手をつけてない	
\\	[名詞]	満了、終了	
\\	複数形チェック
\\	[名詞]	大変動、動乱	
\\	複数形チェック
\\	[名詞]	終点	
\\	複数形チェック
\\	[名詞]	反乱、上り坂	
\\	複数形チェック
\\	[名詞]	上面、上昇傾向	
\\	何もかも逆さまだ。	
\\	複数形チェック
\\	[名詞]	神学、神論	
\\	ヘーゲルの弁証法の秘密は究極的に次のポイントにだけ存在する;神学を経て哲学を否定するために哲学を経て、神学を否定します。	
\\	複数形チェック
\\	[名詞]	尿	
\\	尿瓶でおしっこするのを手伝ってあげることを専門用語でなんと言いますか?	
\\	複数形チェック
\\	[副詞]	理論上は、名目上は	
\\	理論的には、私は数学をしている。	
\\	[形容詞]	使用できる、便利な	
\\	[形容詞]	治療の	
\\	[名詞]	案内係	
\\	複数形チェック
\\	[動詞]	脈打つ、鼓動する	
\\	ここがずきずき痛みます。	
\\	[名詞]	実利主義者	
\\	複数形チェック
\\	[名詞]	臆病な様	
\\	複数形チェック
\\	[名詞]	異形	
\\	複数形チェック
\\	[名詞]	ひりひりする痛み	
\\	コーラを飲んだら舌がぴりぴりした。	
\\	複数形チェック
\\	[名詞]	ニス、ごまかし	
\\	複数形チェック
\\	[名詞]	鋳掛け屋	
\\	複数形チェック
\\	[名詞]	地下室	
\\	複数形チェック
\\	[名詞]	色、ほのかな色	
\\	複数形チェック
\\	[名詞]	子牛の肉	
\\	仔牛肉を食べますか。	
\\	複数形チェック
\\	[名詞]	よちよち歩く人	
\\	どうすれば小さな子供に野菜を食べさせることができるでしょうか。	
\\	複数形チェック
\\	[形容詞]	尊敬すべき、立派な	
\\	[形容詞]	⻭のない	
\\	[動詞]	換気する	
\\	[動詞]	倒す、ぐらつかせる	
\\	本の山は今にも崩れそうに見える。	
\\	[名詞]	開催地	
\\	複数形チェック
\\	[名詞]	広大な土地、広がり	
\\	複数形チェック
\\	[形容詞]	多才な、融通の利く	
\\	[名詞]	失神	
\\	私は夢中で水の中に飛び込んだ。	
\\	複数形チェック
\\	[名詞]	拒否権	
\\	複数形チェック
\\	[名詞]	静けさ	
\\	複数形チェック
\\	[動詞] [不規則動詞]	いらだたせる	
\\	[動詞]	処理する	
\\	[形容詞]	存続可能な	
\\	[動詞]	超越する、超える	
\\	スポーツに国境はない。	
\\	[名詞]	教区牧師	
\\	複数形チェック
\\	[形容詞]	大陸横断の	
\\	[形容詞]	下劣な	
\\	[名詞]	短期滞在客	
\\	母親の機嫌が悪いのは一過性だ。	
\\	複数形チェック
\\	[名詞]	ぶどう酒	
\\	旧車ですか。	
\\	複数形チェック
\\	[名詞]	トランジスター	
\\	複数形チェック
\\	[名詞]	視界	
\\	複数形チェック
\\	[名詞]	精神的外傷	
\\	複数形チェック
\\	[名詞]	空想家	
\\	複数形チェック
\\	[名詞]	裏切り、背信(行為)	
\\	複数形チェック
\\	[名詞]	展望、回想、追憶	
\\	のセキュリティ機能を迂回する方法を研究者が公表します。	
\\	複数形チェック
\\	[名詞]	論文	
\\	複数形チェック
\\	[形容詞]	揮発性の	
\\	[名詞]	行程、(⻑くて辛い)旅	
\\	複数形チェック
\\	[名詞]	一⻫射撃	
\\	複数形チェック
\\	[名詞]	侵入、侵害訴訟	
\\	彼は愛想のいい奴だが気軽に踏み込ませない一線を持っていた。	
\\	複数形チェック
\\	[動詞]	歩いて渡る	
\\	[名詞]	飾り、添え物	
\\	複数形チェック
\\	[名詞]	賭け、賭け金	
\\	複数形チェック
\\	[名詞]	乱気流	
\\	ただ今気流の関係で揺れております。	
\\	複数形チェック
\\	[形容詞]	戦争が起こりそうな	
\\	[名詞]	潰瘍、病幣	
\\	口内炎ができているのです。	
\\	複数形チェック
\\	[形容詞]	用心深い、慎重な	
\\	[名詞]	紫外線	
\\	複数形チェック
\\	[名詞]	くさび	
\\	複数形チェック
\\	[形容詞]	説明のできない	
\\	[形容詞]	時刻が非常に早い	
\\	[形容詞]	あいまいでない	
\\	[形容詞]	重要な	
\\	[形容詞]	権限のない	
\\	[名詞]	溶接	
\\	複数形チェック
\\	[形容詞]	負けない、無敵の	
\\	[名詞]	波止場	
\\	複数形チェック
\\	[形容詞]	妥協しない、断固とした	
\\	[名詞]	気まぐれ	
\\	いったん職業を決めたらむやみに変えてはいけない。	
\\	複数形チェック
\\	[形容詞]	抑制されていない	
\\	[名詞]	かつら	
\\	トムは昔、かつらをつけていた。	
\\	複数形チェック
\\	[形容詞]	秘密の、内密の	
\\	[名詞]	ひるみ	
\\	複数形チェック
\\	[名詞]	下部を切り取ること	
\\	複数形チェック
\\	[形容詞]	もの思いに沈んだ	
\\	[名詞]	下側	
\\	複数形チェック
\\	[動詞]	言い寄る、くどく	
\\	[形容詞]	進行中の	
\\	パーティーを始めようぜ。	
\\	[形容詞]	樹木の茂った	
\\	[動詞] [不規則動詞]	署名する、支払いを保証する	
\\	[名詞]	難破貨物、残骸	
\\	複数形チェック
\\	[形容詞]	予期しない	
\\	[名詞]	ぐいとねじること、レンチ	
\\	複数形チェック
\\	[名詞]	労働組合主義者	
\\	複数形チェック
\\	[動詞]	取り組む、戦う	
\\	腕相撲しようか。	
\\	[形容詞]	正統でない、異端の	
\\	[名詞]	レスラー	
\\	そのレスラーは200キロ以上体重がある。	
\\	複数形チェック
\\	[形容詞]	前代未聞の	
\\	[名詞]	恥知らずな人	
\\	複数形チェック
\\	[形容詞]	利益にならない、無駄な	
\\	[形容詞]	しかめた	
\\	[形容詞]	無言の	
\\	[名詞]	ヨガ	
\\	複数形チェック
\\	[形容詞]	想像もできない	
\\	要するにタイマン勝負なんだけど、その試合内容は常人では考えられないほどの高レベル。	
\\	[形容詞]	動物学の	
\\	[名詞]	修道院、大寺院	
\\	複数形チェック
\\	[形容詞]	不安な	
\\	[名詞]	常軌を逸すること	
\\	複数形チェック
\\	[名詞]	高木	
\\	複数形チェック
\\	[動詞]	けしかける、教唆する	
\\	[形容詞]	難解な	
\\	[形容詞]	変わることのない	
\\	[名詞]	群島、多島海	
\\	日本列島は酷暑に苛まれている。	
\\	複数形チェック
\\	[形容詞]	明るく輝いて	
\\	[名詞]	歌、旋律	
\\	複数形チェック
\\	[名詞]	研磨剤	
\\	複数形チェック
\\	[形容詞]	乾燥した	
\\	[名詞]	節制、禁欲	
\\	複数形チェック
\\	[名詞]	ノアの箱舟	
\\	私の進化した時のアークがあなたたちの能力を底上げするわ。	
\\	複数形チェック
\\	[名詞]	溝、深淵	
\\	複数形チェック
\\	[名詞]	放火	
\\	複数形チェック
\\	[名詞]	共犯者	
\\	複数形チェック
\\	[名詞]	関節炎	
\\	複数形チェック
\\	[動詞]	生じる	
\\	[名詞]	優位	
\\	複数形チェック
\\	[動詞]	無罪にする	
\\	彼は無罪になった。	
\\	[名詞]	ポプラ	
\\	複数形チェック
\\	[名詞]	無罪放免	
\\	うつ病だから,何を言っても無罪放免さ。	
\\	複数形チェック
\\	[名詞]	分類	
\\	複数形チェック
\\	[名詞]	頭字語	
\\	複数形チェック
\\	[名詞]	小惑星	
\\	複数形チェック
\\	[名詞]	明敏さ	
\\	複数形チェック
\\	[形容詞]	抜け目のない、ずるい	
\\	あなたはとても鋭い。	
\\	[名詞]	非常に固いもの	
\\	複数形チェック
\\	[名詞]	保護施設	
\\	社会は患者によって運営されている精神病院です。	
\\	複数形チェック
\\	[形容詞]	隣接する	
\\	[名詞]	自然減少	
\\	複数形チェック
\\	[動詞]	訓戒する、勧告する	
\\	彼は彼らを騒がしいとたしなめた。	
\\	[形容詞]	大胆な、無謀な	
\\	[形容詞]	漂って	
\\	目が覚めてみると我々は大海原を漂流していた。	
\\	[名詞]	厚かましさ	
\\	複数形チェック
\\	[名詞]	不貞	
\\	多くの宗教において、姦淫は罪悪とされる。	
\\	複数形チェック
\\	[名詞]	雰囲気、霊気	
\\	ベンヤミンは優れた芸術作品を前にして人が経験するであろう畏怖や崇敬の感覚を指して「アウラ」という語を用いた。	
\\	複数形チェック
\\	[形容詞]	愛想のよい	
\\	[形容詞]	厳格な	
\\	[形容詞]	系列の	
\\	[名詞]	厳粛	
\\	複数形チェック
\\	[形容詞]	苦悩に満ちた	
\\	[形容詞]	独裁的な	
\\	[形容詞]	病気の	
\\	[動詞]	かたきを討つ	
\\	誰がカールの仇討ちをする?	
\\	[形容詞]	風媒の	
\\	[形容詞]	熱心な、熱烈な	
\\	[名詞]	別名	
\\	複数形チェック
\\	[名詞]	背景	
\\	複数形チェック
\\	[動詞]	揃える	
\\	[名詞]	裏書人、後援者	
\\	複数形チェック
\\	[動詞]	緩和する、軽減する	
\\	[名詞]	反発、反動	
\\	複数形チェック
\\	[形容詞]	魅惑的な	
\\	[形容詞]	だぶだぶの	
\\	病気をしてから夫は20
\\	も体重が落ちて、ズボンなどはもうぶかぶかです。	
\\	[名詞]	言及	
\\	複数形チェック
\\	[名詞]	妨害、障害	
\\	複数形チェック
\\	[名詞]	利他主義	
\\	複数形チェック
\\	[名詞]	災い	
\\	複数形チェック
\\	[名詞]	同窓生	
\\	複数形チェック
\\	[名詞]	気さくな会話、ひやかし	
\\	複数形チェック
\\	[形容詞]	反対感情両立的な	
\\	[形容詞]	とげのある	
\\	[副詞]	十分に	
\\	[名詞]	吟遊詩人	
\\	複数形チェック
\\	[名詞]	貧血症	
\\	複数形チェック
\\	[名詞]	バロック様式	
\\	複数形チェック
\\	[名詞]	麻酔	
\\	複数形チェック
\\	[名詞]	連発、続出	
\\	複数形チェック
\\	[名詞]	麻酔剤	
\\	亜酸化窒素は麻酔としてもはや使用されていない。	
\\	複数形チェック
\\	[動詞]	日光浴をする	
\\	少年は春の日差しを浴びて仰向けに寝ていた。	
\\	[副詞]	改めて、新たに、新規に	
\\	新たに始める。	
\\	[名詞]	稜堡、要塞、とりで	
\\	複数形チェック
\\	[名詞]	敵意、憎悪	
\\	複数形チェック
\\	[名詞]	大隊	
\\	複数形チェック
\\	[名詞]	別館、離れ	
\\	複数形チェック
\\	[名詞]	信号塔、灯台	
\\	アメリカの光がまだ輝きとして燃えているどうかを疑う人へ、今夜、我々の国の真の強さは軍隊の力や富の規模からではなく、民主主義、自由、機会、不屈の希望といった我々の理想への揺るぎない力から来るともう一度証明した。	
\\	複数形チェック
\\	[名詞]	年金	
\\	複数形チェック
\\	[名詞]	基盤	
\\	複数形チェック
\\	[動詞]	無効にする	
\\	[動詞]	だます	
\\	[名詞]	例外	
\\	複数形チェック
\\	[名詞]	交戦国、戦闘員	
\\	複数形チェック
\\	[名詞]	ポーカーの掛け金	
\\	複数形チェック
\\	[動詞]	嘆く	
\\	[名詞]	賛歌	
\\	複数形チェック
\\	[名詞]	基準、水準	
\\	複数形チェック
\\	[名詞]	解毒剤	
\\	複数形チェック
\\	[名詞]	後援者、恩人	
\\	複数形チェック
\\	[名詞]	憎悪、毛嫌い	
\\	複数形チェック
\\	[名詞]	入札者	
\\	複数形チェック
\\	[形容詞]	古臭くなった、古風な	
\\	[名詞]	偏狭	
\\	複数形チェック
\\	[名詞]	啓示	
\\	複数形チェック
\\	[名詞]	生体検査	
\\	複数形チェック
\\	[名詞]	使徒、主唱者	
\\	複数形チェック
\\	[名詞]	かばの木	
\\	複数形チェック
\\	[動詞]	なだめる、鎮める	
\\	[形容詞]	騒々しい、見え透いた	
\\	[名詞]	だめにするもの	
\\	複数形チェック
\\	[名詞]	中央に置く飾り物	
\\	複数形チェック
\\	[名詞]	水ぶくれ	
\\	手のまめがつぶれて痛いよ。	
\\	複数形チェック
\\	[名詞]	施設付き司祭	
\\	複数形チェック
\\	[名詞]	はったり	
\\	複数形チェック
\\	[形容詞]	黑焦げの	
\\	この焦げ目が美味しいんだよね。	
\\	[名詞]	出し抜けに言うこと	
\\	『何だこりゃ?』って思わず言いたくなるような変な車でした。	
\\	複数形チェック
\\	[名詞]	深い割れ目、隙間	
\\	複数形チェック
\\	[名詞]	重役用会議室	
\\	複数形チェック
\\	[動詞]	懲らしめる、鍛える	
\\	[形容詞]	いんちきの	
\\	[動詞] [不規則動詞]	たしなめる、叱る	
\\	[形容詞]	騒がしい、バカ騒ぎの	
\\	[名詞]	振付師	
\\	複数形チェック
\\	[名詞]	射石砲	
\\	複数形チェック
\\	[名詞]	染色体	
\\	複数形チェック
\\	[名詞]	砲撃	
\\	複数形チェック
\\	[動詞]	回避する	
\\	法は脱法されるために作られる。	
\\	[名詞]	ベビーブーム世代	
\\	複数形チェック
\\	[名詞]	引用	
\\	複数形チェック
\\	[名詞]	戦利品	
\\	複数形チェック
\\	[名詞]	礼儀正しさ	
\\	複数形チェック
\\	[名詞]	アルコール飲料	
\\	複数形チェック
\\	[名詞]	騒々しい音	
\\	このような民主主義の世代にあっては、皆自分がどう思うかではなく、一般に人気があるものを強く求めるのです。洗練されたものではなく、値段の高いものを。美しいものではなく、おしゃれなものを。	
\\	複数形チェック
\\	[名詞]	へたな仕事	
\\	複数形チェック
\\	[名詞]	かすがい	
\\	賢明な経営者はコストをぎりぎりまでさげる方法を知っていますよ。	
\\	複数形チェック
\\	[名詞]	報奨金	
\\	トムは常々、賞金稼ぎになりたいと夢見ていた。	
\\	複数形チェック
\\	[形容詞]	秘密の、内密の	
\\	[名詞]	木陰	
\\	複数形チェック
\\	[名詞]	聖職者	
\\	複数形チェック
\\	[動詞]	振り回す	
\\	彼女はわめき、警官たちに向かってナイフを振り回して脅すようなしぐさをしました。	
\\	[名詞]	顧客	
\\	複数形チェック
\\	[名詞]	吹出物	
\\	複数形チェック
\\	[名詞]	激しい抱擁	
\\	複数形チェック
\\	[名詞]	虚勢	
\\	本当の自信と空威張りは別問題だ。	
\\	複数形チェック
\\	[名詞]	固まり	
\\	複数形チェック
\\	[名詞]	喧嘩	
\\	複数形チェック
\\	[名詞]	影響力	
\\	複数形チェック
\\	[動詞]	決然と対決する	
\\	その恥知らずな抵抗活動のために、彼は危うく命を失いかけた。	
\\	[名詞]	混乱、散乱した物	
\\	複数形チェック
\\	[形容詞]	息をのむような	
\\	[動詞]	融合する、合体する	
\\	[名詞]	同胞	
\\	複数形チェック
\\	[名詞]	丸石	
\\	複数形チェック
\\	[名詞]	もろい	
\\	複数形チェック
\\	[名詞]	財源	
\\	複数形チェック
\\	[名詞]	緩衝材	
\\	複数形チェック
\\	[名詞]	集団	
\\	複数形チェック
\\	[形容詞]	雄牛のような	
\\	[名詞]	担保物件	
\\	複数形チェック
\\	[名詞]	ロールパン	
\\	毎日コンビニで肉まんを食べています。	
\\	複数形チェック
\\	[名詞]	子馬	
\\	複数形チェック
\\	[形容詞]	浮力のある	
\\	[形容詞]	昏睡状態の	
\\	[名詞]	新芽	
\\	複数形チェック
\\	[名詞]	結合	
\\	複数形チェック
\\	[形容詞]	たくましい	
\\	[名詞]	将来有望な人	
\\	複数形チェック
\\	[動詞]	磨く	
\\	[名詞]	戒律	
\\	複数形チェック
\\	[名詞]	隠れ場、避難所	
\\	複数形チェック
\\	[名詞]	奇襲部隊	
\\	複数形チェック
\\	[名詞]	支え、控え壁	
\\	複数形チェック
\\	[名詞]	公式発表	
\\	複数形チェック
\\	[名詞]	買い占め	
\\	オクは面倒なので即決で一回買っただけ。	
\\	複数形チェック
\\	[名詞]	編集したもの	
\\	複数形チェック
\\	[名詞]	隠し場所	
\\	複数形チェック
\\	[名詞]	自己満足	
\\	複数形チェック
\\	[名詞]	士官候補生、見習い	
\\	複数形チェック
\\	[名詞]	共犯	
\\	複数形チェック
\\	[名詞]	微積分学	
\\	微積分を学ぶには、練習問題をたくさんこなす必要があります。	
\\	複数形チェック
\\	[名詞]	会計検査官	
\\	複数形チェック
\\	[名詞]	力量	
\\	複数形チェック
\\	[形容詞]	協定の	
\\	[名詞]	犬、イヌ科の動物	
\\	複数形チェック
\\	[名詞]	協奏曲	
\\	複数形チェック
\\	[形容詞]	抜け目のない、用心深い	
\\	[動詞]	企てる	
\\	労働力の40パーセントはホワイトカラー労働者であり、その大部分は、退屈でばかばかしいデッチアゲ仕事をしている。	
\\	[動詞]	資本に組み入れる	
\\	[動詞]	大目に見る、許す	
\\	[動詞]	降伏する	
\\	[名詞]	同類、同盟国	
\\	複数形チェック
\\	[名詞]	死体、人の死体	
\\	複数形チェック
\\	[動詞]	没収・押収する	
\\	私の免許は没収になった。	
\\	[名詞]	強心剤	
\\	心停止の患者を応急処置で生かした。	
\\	複数形チェック
\\	[名詞]	大火災	
\\	小規模の森林火災はみるみる広がり大火災となった。	
\\	複数形チェック
\\	[名詞]	心臓血管	
\\	複数形チェック
\\	[名詞]	複合企業	
\\	複数形チェック
\\	[名詞]	企業連合	
\\	複数形チェック
\\	[動詞]	魔法で出す	
\\	魔女集会で悪魔を召喚する。	
\\	[名詞]	社会的地位	
\\	複数形チェック
\\	[名詞]	鑑定家	
\\	複数形チェック
\\	[名詞]	触媒	
\\	複数形チェック
\\	[名詞]	自然保護論者	
\\	複数形チェック
\\	[名詞]	射出器	
\\	テヲ	複数形チェック
\\	[動詞]	完成する	
\\	[名詞]	党員集会	
\\	複数形チェック
\\	[名詞]	抑制	
\\	複数形チェック
\\	[名詞]	100年祭	
\\	複数形チェック
\\	[名詞]	黙想にふける人	
\\	複数形チェック
\\	[形容詞]	議論好きな、争いを好む	
\\	[名詞]	離反、脱会、背信	
\\	複数形チェック
\\	[名詞]	偶発性、偶発事件	
\\	複数形チェック
\\	[名詞]	服従、敬意	
\\	複数形チェック
\\	[形容詞]	偶発的な、偶然の	
\\	[動詞]	しぼむ	
\\	[名詞]	連続体	
\\	複数形チェック
\\	[形容詞]	器用な	
\\	[名詞]	避妊	
\\	複数形チェック
\\	[名詞]	故人	
\\	複数形チェック
\\	[名詞]	避妊具	
\\	日本ではリング状のものが早くから普及したため「避妊リング」と呼ばれることも多い。	
\\	複数形チェック
\\	[名詞]	悪化する、堕落する	
\\	複数形チェック
\\	[名詞]	後悔	
\\	複数形チェック
\\	[名詞]	神	
\\	複数形チェック
\\	[名詞]	難問	
\\	複数形チェック
\\	[名詞]	大洪水	
\\	複数形チェック
\\	[名詞]	修道会、修道院	
\\	複数形チェック
\\	[動詞]	深く探求する	
\\	[名詞]	集中性、収束度	
\\	複数形チェック
\\	[名詞]	ふるまい	
\\	複数形チェック
\\	[名詞]	樽類製造者、桶屋、酒屋	
\\	彼女は声を聞いてもらうためにこの選挙で長い列をつくった他の何百万人と同様なのですが、一つ違います、アン・ニクソン・クーパーさんは106歳なのです。	
\\	複数形チェック
\\	[名詞]	痴呆	
\\	複数形チェック
\\	[名詞]	基礎、不可欠のもの	
\\	複数形チェック
\\	[形容詞]	人口統計学の	
\\	[名詞]	冠動脈	
\\	複数形チェック
\\	[名詞]	破壊	
\\	複数形チェック
\\	[名詞]	伍⻑	
\\	私は体罰には断固反対です。	
\\	複数形チェック
\\	[名詞]	異議	
\\	複数形チェック
\\	[名詞]	全集	
\\	一般に利用できる英日2言語コーパスを作成した。	
\\	複数形チェック
\\	[名詞]	額面金額、単位名	
\\	複数形チェック
\\	[動詞]	確証する、正式に確認する	
\\	[名詞]	へこみ	
\\	複数形チェック
\\	[形容詞]	腐食性の	
\\	[動詞]	使い果たす	
\\	[名詞]	皮質、皮層	
\\	複数形チェック
\\	[動詞]	国外に退去させる	
\\	[形容詞]	偽造の、にせの	
\\	[名詞]	国外追放	
\\	複数形チェック
\\	[名詞]	裁判所	
\\	複数形チェック
\\	[動詞]	退位させる、免職する	
\\	2006年9月の軍事クーデターで失脚、事実上亡命中だったタイのタクシン元首相が28日、約1年半ぶりに帰国した。	
\\	[名詞]	求婚	
\\	複数形チェック
\\	[動詞]	馬鹿にする	
\\	事実は明白だ。それを混乱は憤慨するだろう、無知は嘲笑するだろう、悪意は曲げようとするだろう、しかしそれが事実なのだ。	
\\	[形容詞]	ひそかな、隠れた	
\\	[名詞]	あざけり、物笑いの種	
\\	複数形チェック
\\	[形容詞]	遠慮がちな、内気な	
\\	[形容詞]	困窮した	
\\	[名詞]	厳重な取締まり	
\\	複数形チェック
\\	[名詞]	悪化	
\\	複数形チェック
\\	[名詞]	パチパチ鳴る音、活気	
\\	複数形チェック
\\	[名詞]	平価切り下げ	
\\	複数形チェック
\\	[名詞]	くだらんこと	
\\	複数形チェック
\\	[形容詞]	ずるい、よこしまな	
\\	[名詞]	木箱	
\\	複数形チェック
\\	[動詞]	ゆだねる、負わせる	
\\	[名詞]	旧友	
\\	複数形チェック
\\	[形容詞]	糖尿病の	
\\	[動詞]	しわくちゃにする	
\\	どうしよう!お父さんの会社の書類をゴミと間違えてくしゃくしゃにしちゃった!	
\\	[名詞]	おむつ	
\\	複数形チェック
\\	[名詞]	野獣の子	
\\	ある朝洞穴から子どもの狐が 出ようとしましたが、「あっ。」と叫んで眼をおさえながら母さん狐のところへころげてきました。「母ちゃん、眼になにか刺さった、ぬいてちょうだい早く早く。」といいました。	
\\	複数形チェック
\\	[名詞]	格言	
\\	複数形チェック
\\	[名詞]	そで口	
\\	和服を着て椅子に腰をかけると、何となく心細い。裾から風がはひるやうな気がする。	
\\	複数形チェック
\\	[形容詞]	悲惨な、緊急の	
\\	[形容詞]	台所の、料理の	
\\	[動詞]	解散する	
\\	[動詞]	摘む、摘み集める	
\\	[形容詞]	自由裁量の	
\\	[名詞]	最高点、頂点	
\\	複数形チェック
\\	[形容詞]	不満な、不機嫌な	
\\	[形容詞]	煩わしい	
\\	[名詞]	崩壊	
\\	複数形チェック
\\	[名詞]	館⻑、管理者	
\\	複数形チェック
\\	[動詞]	けなす、見くびる	
\\	ネット上では悪質な、個人を貶めようとする行動が多く見られる気がします。	
\\	[形容詞]	ぶっきらぼうな	
\\	[形容詞]	本質的に異なる、異種の	
\\	[動詞]	削減する、短縮する	
\\	[名詞]	相違、不つり合い	
\\	複数形チェック
\\	[名詞]	削減	
\\	複数形チェック
\\	[形容詞]	不釣り合いな、不相応な	
\\	[名詞]	皮肉屋	
\\	シニックと言うのは何だ?あらゆる物の価格を知って、いかなる物の価値を知らぬ人間である。	
\\	複数形チェック
\\	[形容詞]	取り乱した、錯乱状態の	
\\	[名詞]	冷笑、皮肉なことば	
\\	その問いかけに答えるチャンスを今、私たちは手にしました。今この時こそが、私たちの瞬間です。今この時にこそ、私たちは人々がまた仕事につけるようにしなくてはなりません。子供たちのために、チャンスの扉を開かなくてはなりません。繁栄を取り戻し、平和を推進しなくてはなりません。今この時にこそ、アメリカの夢を取り戻し、基本的な真理を再確認しなくてはなりません。大勢の中にあって、私たちはひとつなのだと。息をし続ける限り、私たちは希望をもち続けるのだと。そして疑り深く悲観し否定する声に対しては、そんなことできないという人たちに対しては、ひとつ国民の魂を端的に象徴するあの不朽の信条でもって、必ずやこう答えましょう。	
\\	複数形チェック
\\	[動詞]	漏らす、公表する	
\\	[名詞]	皇帝	
\\	複数形チェック
\\	[名詞]	博士号	
\\	複数形チェック
\\	[形容詞]	手ごわい、困難な	
\\	[名詞]	(ウサギ・羊などの)雌	
\\	複数形チェック
\\	[名詞]	欠乏、不足	
\\	複数形チェック
\\	[形容詞]	執拗な	
\\	[名詞]	総崩れ、大敗	
\\	複数形チェック
\\	[名詞]	お人形	
\\	複数形チェック
\\	[動詞]	品質を低下させる	
\\	[名詞]	麻薬	
\\	複数形チェック
\\	[動詞]	衰弱させる	
\\	[形容詞]	気難しい	
\\	[動詞]	解読する	
\\	[動詞]	重要視しない	
\\	[名詞]	装飾、舞台装置	
\\	複数形チェック
\\	[形容詞]	まっすぐな、率直な	
\\	[名詞]	中傷	
\\	複数形チェック
\\	[名詞]	くすんだ⻩褐色	
\\	複数形チェック
\\	[動詞]	衣類で飾る	
\\	[名詞]	逃亡犯罪人引き渡し	
\\	複数形チェック
\\	[動詞]	ものうげに言う	
\\	[名詞]	過激主義者	
\\	トムは過激な思想の持ち主だ。	
\\	複数形チェック
\\	[動詞]	掘り起こす	
\\	[形容詞]	元気あふれる	
\\	[動詞]	ずぶぬれにする	
\\	彼は雨でずぶぬれになった。	
\\	[動詞]	歓喜する	
\\	[動詞]	ブンブンうなる	
\\	ドローンを買いたいんだけど親が買わせてくれない。	
\\	[名詞]	鷹	
\\	複数形チェック
\\	[名詞]	ダクト、輸送管	
\\	涙腺が詰まっていると言われました。	
\\	複数形チェック
\\	[名詞]	茶番	
\\	だが、レーガン政権と最高裁判所に潰される前でさえ、OSHAは茶番であった。	
\\	複数形チェック
\\	[名詞]	決闘	
\\	「親父これって・・・真剣?」「当たり前だろう、決闘なのだからな」	
\\	複数形チェック
\\	[名詞]	尋(水深の測定単位)	
\\	複数形チェック
\\	[形容詞]	折衷的な、取捨選択の	
\\	[名詞]	(1歳未満の)子鹿	
\\	複数形チェック
\\	[名詞]	建物、大建築物	
\\	複数形チェック
\\	[動詞]	発酵させる	
\\	彼は納豆の匂いを嗅ぐと気分が悪くなる。	
\\	[名詞]	効力、効き目	
\\	複数形チェック
\\	[名詞]	残虐行為	
\\	複数形チェック
\\	[名詞]	小妖精	
\\	複数形チェック
\\	[形容詞]	熱い、熱烈な	
\\	[名詞]	有資格	
\\	複数形チェック
\\	[名詞]	胎児	
\\	複数形チェック
\\	[動詞]	発出する	
\\	[名詞]	不和	
\\	複数形チェック
\\	[名詞]	具体化、肉体化	
\\	複数形チェック
\\	[名詞]	大失敗	
\\	複数形チェック
\\	[動詞]	巻き込む	
\\	[名詞]	命令	
\\	複数形チェック
\\	[名詞]	共感、感情移入	
\\	複数形チェック
\\	[形容詞]	変わりやすい、気まぐれな	
\\	[名詞]	立法	
\\	複数形チェック
\\	[動詞]	いじくる、もてあそぶ	
\\	マウスとキーボードを行ったりきたりするのが煩雑なので、キーボードのみで操作できるようショートカットキーはないのでしょうか?	
\\	[名詞]	飛び領土、飛び地	
\\	複数形チェック
\\	[名詞]	魚のひれ	
\\	複数形チェック
\\	[動詞]	いとしく思わせる	
\\	[名詞]	資本家、金融業者	
\\	複数形チェック
\\	[形容詞]	一地方特有の	
\\	[名詞]	技巧、腕のさえ、策略	
\\	複数形チェック
\\	[名詞]	謎、不可解なもの	
\\	複数形チェック
\\	[名詞]	火器	
\\	複数形チェック
\\	[名詞]	合奏団	
\\	わたしはギターアンサンブルでバスギターを弾いている。	
\\	複数形チェック
\\	[形容詞]	極悪の、目にあまる	
\\	[名詞]	側近、同僚	
\\	複数形チェック
\\	[名詞]	旗艦、最高のもの	
\\	敵の旗艦を撃破しました!	
\\	複数形チェック
\\	[動詞]	懇願する	
\\	[名詞]	燃えるような、派手な	
\\	複数形チェック
\\	[動詞]	ざんごうで囲む	
\\	[動詞]	見せびらかす	
\\	[名詞]	地震の震央	
\\	複数形チェック
\\	[名詞]	⻘二才	
\\	そこに羽の生えたばかりの小鳥がありました。	
\\	複数形チェック
\\	[名詞]	墓碑銘	
\\	複数形チェック
\\	[形容詞]	はかない	
\\	お客たちは、チラッとその人の方をみるが、またすぐに話を始める。	
\\	[動詞]	避ける、控える	
\\	[名詞]	柔軟性	
\\	複数形チェック
\\	[名詞]	スパイ行為、偵察	
\\	複数形チェック
\\	[形容詞]	薄っぺらな	
\\	[動詞]	採用する、支持する	
\\	[名詞]	いちゃつくこと	
\\	複数形チェック
\\	[動詞]	刻む	
\\	競走馬を描いたエッチングで彼に並ぶものはいない。	
\\	[動詞]	もがきながら進む	
\\	[形容詞]	極めて優美な	
\\	[名詞]	まぐれ当たり	
\\	複数形チェック
\\	[名詞]	気風、エトス	
\\	複数形チェック
\\	[形容詞]	蛍光性の	
\\	蛍光灯がチカチカしてるな。新しいのと交換しないとダメかな。	
\\	[名詞]	婉曲的表現	
\\	複数形チェック
\\	[名詞]	動揺、混乱	
\\	複数形チェック
\\	[名詞]	幸福(感)、高揚感	
\\	複数形チェック
\\	[名詞]	流動、絶え間ない変化	
\\	複数形チェック
\\	[名詞]	安楽死	
\\	複数形チェック
\\	[名詞]	飼料	
\\	複数形チェック
\\	[形容詞]	福音派の	
\\	[動詞]	かわいがる	
\\	[名詞]	福音伝道者	
\\	複数形チェック
\\	[名詞]	フィート数、場面	
\\	複数形チェック
\\	[名詞]	回避、言い抜け、脱出	
\\	複数形チェック
\\	[名詞]	足の運び、足元、足場	
\\	複数形チェック
\\	[形容詞]	回避的な	
\\	[名詞]	侵略	
\\	複数形チェック
\\	[形容詞]	呼び起こす	
\\	[名詞]	力、勢い	
\\	お客様を脅迫して契約をサインさせるとは、銀行員として許すべからざる行為である。	
\\	複数形チェック
\\	[動詞]	悪化させる、つのらせる	
\\	[名詞]	職⻑	
\\	母さんは、女だてらに男だらけの中、建設会社で現場監督をしている。	
\\	複数形チェック
\\	[形容詞]	苛酷な、骨の折れる	
\\	[形容詞]	法廷の	
\\	[名詞]	消費税	
\\	複数形チェック
\\	[動詞]	機先を制する	
\\	[形容詞]	模範的な、称賛すべき	
\\	[動詞] [不規則動詞]	〜なしですませる	
\\	[動詞]	吐き出す	
\\	[形容詞]	絶望した、絶望的な	
\\	[動詞]	うきうきさせる	
\\	[形容詞]	言いたい事を言う、率直な	
\\	[動詞]	促す、奨励する	
\\	[名詞]	分裂、破砕	
\\	複数形チェック
\\	[名詞]	海外駐在者、国外在住者	
\\	複数形チェック
\\	[形容詞]	兄弟の、友愛の	
\\	[動詞]	称賛する、褒め称える	
\\	[形容詞]	不正の	
\\	くれぐれも浄水器の悪質訪問販売業者や無料点検商法にはご注意ください。	
\\	[形容詞]	極寒の、冷淡な	
\\	[名詞]	優良の証明、太鼓判	
\\	複数形チェック
\\	[名詞]	浮かれ騒ぎ	
\\	子供のすきな小さい神さまがありました。いつもは森の中で、歌をうたったり笛を吹いたりして、小鳥やけものと遊んでいましたが、ときどき人のすんでいる村へ出てきて、すきな子供たちと遊ぶのでした。	
\\	複数形チェック
\\	[名詞]	幻覚、妄想、錯覚	
\\	複数形チェック
\\	[名詞]	悪臭	
\\	複数形チェック
\\	[名詞]	後光	
\\	複数形チェック
\\	[名詞]	漏斗(じょうご)	
\\	複数形チェック
\\	[名詞]	手錠	
\\	複数形チェック
\\	[名詞]	熱狂的興奮	
\\	複数形チェック
\\	[形容詞]	不運な、あわれな	
\\	[名詞]	細⻑いくぼみ、わだち	
\\	複数形チェック
\\	[形容詞]	痛ましい	
\\	[名詞]	祝祭、競技会	
\\	複数形チェック
\\	[動詞]	執拗に攻撃する	
\\	ハリーはまだ40歳だ。	
\\	[動詞]	悩ます、感情を傷付ける	
\\	[名詞]	混乱、苦闘	
\\	栗の皮をむく作業に一苦労した。	
\\	複数形チェック
\\	[動詞]	刺激する、亜鉛めっきする	
\\	[形容詞]	向こう見ずな、軽率な	
\\	[名詞]	花輪、栄冠	
\\	複数形チェック
\\	[形容詞]	うきうきした	
\\	[名詞]	穀倉、たくわえ	
\\	複数形チェック
\\	[動詞]	励ます	
\\	[名詞]	駐屯地	
\\	複数形チェック
\\	[名詞]	ヘザー、ヒース	
\\	複数形チェック
\\	[名詞]	⻑手袋	
\\	複数形チェック
\\	[形容詞]	屈強な	
\\	[名詞]	家系	
\\	複数形チェック
\\	[名詞]	主導権、覇権	
\\	複数形チェック
\\	[名詞]	起源	
\\	複数形チェック
\\	[名詞]	肝炎	
\\	複数形チェック
\\	[名詞]	遺伝学	
\\	複数形チェック
\\	[名詞]	異端、異説	
\\	複数形チェック
\\	[名詞]	生殖器	
\\	複数形チェック
\\	[名詞]	異性愛の人	
\\	複数形チェック
\\	[形容詞]	上品ぶった	
\\	[名詞]	全盛期	
\\	複数形チェック
\\	[形容詞]	ぞっとするような	
\\	彼の膝は腫れてぞっとするような黒紫色になってしまった。	
\\	[名詞]	割れ目、中断	
\\	複数形チェック
\\	[名詞]	スラム街	
\\	複数形チェック
\\	[形容詞]	愉快な	
\\	[形容詞]	めまいがする、	
\\	[名詞]	後知恵、後になっての判断	
\\	複数形チェック
\\	[動詞] [不規則動詞]	金めっきする	
\\	[名詞]	人をかつぐこと	
\\	私の祖父は月面着陸のことが信じられない。	
\\	複数形チェック
\\	[動詞]	収集する	
\\	[名詞]	敬意	
\\	複数形チェック
\\	[名詞]	歓喜	
\\	複数形チェック
\\	[名詞]	殺人	
\\	トムさんが退職するまで、殺人担当刑事でした。	
\\	複数形チェック
\\	[形容詞]	口達者な	
\\	[名詞]	砥石	
\\	複数形チェック
\\	[名詞]	ちらちらする光	
\\	ここもダメか……そんな顔になりながら、少年がそれでも一縷の望みで俺達の顔を見上げている。	
\\	複数形チェック
\\	[名詞]	大群	
\\	複数形チェック
\\	[動詞]	ほくそえむ	
\\	[形容詞]	恐ろしい	
\\	[形容詞]	熱のこもった	
\\	[名詞]	中心	
\\	こういう風に合体すれば、14ポートのUSBハブになります。	
\\	複数形チェック
\\	[形容詞]	ふさぎ込んだ、不機嫌な	
\\	[名詞]	ずうたいの大きな人	
\\	複数形チェック
\\	[名詞]	供給過剰、過多	
\\	安い輸入品で綿製品が供給過剰になった。	
\\	複数形チェック
\\	[名詞]	船体	
\\	複数形チェック
\\	[動詞]	かじる	
\\	[形容詞]	屈辱的な	
\\	[名詞]	駆り立てるもの	
\\	複数形チェック
\\	[名詞]	こぶ	
\\	ラクダは背中のこぶに大量の脂肪をためることができる。	
\\	複数形チェック
\\	[名詞]	金細工師	
\\	複数形チェック
\\	[名詞]	予感	
\\	雨が降るような気がする。	
\\	複数形チェック
\\	[名詞]	菓子、特別いいもの	
\\	複数形チェック
\\	[動詞]	音を立てていく	
\\	[名詞]	丸のみ	
\\	お前の目を抉り出すぞ!	
\\	複数形チェック
\\	[名詞]	騒ぎ	
\\	複数形チェック
\\	[名詞]	落書き	
\\	建物は落書きだらけで見た目が悪い。	
\\	複数形チェック
\\	[名詞]	誇張	
\\	複数形チェック
\\	[名詞]	不正利得	
\\	複数形チェック
\\	[名詞]	高血圧	
\\	複数形チェック
\\	[形容詞]	大げさな	
\\	[形容詞]	特異質の	
\\	[名詞]	つかむこと	
\\	エスペラント運動は常に永遠の初心者の問題を抱えています。	
\\	複数形チェック
\\	[形容詞]	牧歌的な	
\\	[形容詞]	社交的な	
\\	[動詞]	点火する、高度に熱する	
\\	[名詞]	格子	
\\	複数形チェック
\\	[名詞]	不法な	
\\	複数形チェック
\\	[名詞]	根性、勇気	
\\	複数形チェック
\\	[動詞]	伝授する、知らせる	
\\	[名詞]	痴漢	
\\	複数形チェック
\\	[名詞]	袋小路、行き詰まり、難局	
\\	複数形チェック
\\	[名詞]	ぶつぶつ言うこと	
\\	複数形チェック
\\	[形容詞]	欠点のない、完全な	
\\	[名詞]	恨み	
\\	リサはスタンに恨みがある。	
\\	複数形チェック
\\	[動詞]	邪魔する、妨げる	
\\	[形容詞]	いやいやの	
\\	[形容詞]	容赦のない、執念深い	
\\	[形容詞]	厳しい	
\\	[動詞]	移植する	
\\	2回目の人工内耳移植への低所得者医療扶助制度の適用不許可を取り下げる。	
\\	[形容詞]	恐ろしい	
\\	[名詞]	課税	
\\	複数形チェック
\\	[名詞]	ブーブー鳴く声	
\\	複数形チェック
\\	[形容詞]	貧困に陥った	
\\	[名詞]	身なり、仮装、見せかけ	
\\	複数形チェック
\\	[名詞]	即興曲	
\\	複数形チェック
\\	[名詞]	刑罰を免れること	
\\	複数形チェック
\\	[名詞]	受賞者	
\\	複数形チェック
\\	[名詞]	権化、化身、具体化	
\\	複数形チェック
\\	[名詞]	ろ過	
\\	複数形チェック
\\	[名詞]	放火魔、扇動者	
\\	複数形チェック
\\	[名詞]	多数、多勢	
\\	複数形チェック
\\	[名詞]	近親相姦	
\\	近親相姦はほとんど全ての文化において禁忌とされている。	
\\	複数形チェック
\\	[形容詞]	のんびりした	
\\	やっぱり朝はこのくらいゆっくり出発したいものです。	
\\	[動詞]	扇動する	
\\	[名詞]	白血病	
\\	複数形チェック
\\	[名詞]	無能力	
\\	複数形チェック
\\	[形容詞]	みだらな、下劣な	
\\	[動詞]	罪を負わせる	
\\	[名詞]	辞書、語彙	
\\	複数形チェック
\\	[形容詞]	消えない	
\\	[名詞]	名誉毀損	
\\	※基本的な禁止事項(誹謗・中傷の禁止等)は「はじめにお読み下さい」に記載してあります。必ずお読みください。	
\\	複数形チェック
\\	[形容詞]	不適当な	
\\	[動詞]	たとえる	
\\	[形容詞]	容赦のない、無情な	
\\	[名詞]	血統	
\\	複数形チェック
\\	[名詞]	地獄	
\\	複数形チェック
\\	[名詞]	下着	
\\	複数形チェック
\\	[動詞]	はびこる	
\\	その家にガーターヘビが出ることが分かった。あいつらはどこにでもいるぞ!	
\\	[名詞]	訴訟	
\\	複数形チェック
\\	[動詞]	燃え上がらせる	
\\	[動詞]	ひどく嫌う	
\\	[名詞]	情報提供者	
\\	複数形チェック
\\	[動詞]	弧を描くよう投げる	
\\	[名詞]	赤外線	
\\	私達の目は赤外線を見ることはできません。	
\\	複数形チェック
\\	[名詞]	材木切り出し	
\\	複数形チェック
\\	[名詞]	侵害	
\\	複数形チェック
\\	[名詞]	抜穴	
\\	複数形チェック
\\	[名詞]	流入	
\\	複数形チェック
\\	[名詞]	⺠間伝承	
\\	複数形チェック
\\	[動詞]	吸い込む	
\\	そんなガツガツ食べないで、もうちょっとゆっくり食べようよ。	
\\	[形容詞]	ばかげた	
\\	[名詞]	命令、禁止命令	
\\	複数形チェック
\\	[名詞]	突然の揺れ	
\\	複数形チェック
\\	[形容詞]	無害の、退屈な	
\\	[形容詞]	ぞっとする	
\\	[名詞]	予防接種	
\\	複数形チェック
\\	[名詞]	酒	
\\	複数形チェック
\\	[形容詞]	飽くことのない	
\\	[名詞]	光沢	
\\	複数形チェック
\\	[形容詞]	狡猾な、陰険な	
\\	[名詞]	薄いピンク色	
\\	複数形チェック
\\	[動詞]	ほのめかす	
\\	何が言いたいわけ?	
\\	[名詞]	間に合わせの物	
\\	複数形チェック
\\	[名詞]	破産者	
\\	複数形チェック
\\	[名詞]	不定愁訴、不安、不調	
\\	複数形チェック
\\	[名詞]	教え込む	
\\	複数形チェック
\\	[動詞]	悪口を言う	
\\	[名詞]	反乱分子	
\\	複数形チェック
\\	[名詞]	型にはまった手法	
\\	左藤くんの高い声とクネクネした感じがオカマっぽいですね。	
\\	複数形チェック
\\	[名詞]	反乱	
\\	複数形チェック
\\	[名詞]	入口のひさし、大テント	
\\	複数形チェック
\\	[前置詞]	相互に	
\\	[名詞]	骨髄	
\\	複数形チェック
\\	[形容詞]	腸の	
\\	[名詞]	石工、れんが職人	
\\	多くの人たちがメーソンに同意した。	
\\	複数形チェック
\\	[形容詞]	手に負えない、強情な	
\\	[形容詞]	熟達した	
\\	[形容詞]	静脈内の	
\\	私は病院で点滴を受けた。	
\\	[名詞]	一匹狼、異端者	
\\	複数形チェック
\\	[形容詞]	大胆不敵な	
\\	[名詞]	曲がりくねり	
\\	複数形チェック
\\	[形容詞]	内観的な	
\\	[名詞]	はしか	
\\	先日三日ばしかにかかりました。	
\\	複数形チェック
\\	[動詞]	氾濫する	
\\	[名詞]	溶けること	
\\	複数形チェック
\\	[形容詞]	怒った	
\\	[名詞]	更年期	
\\	複数形チェック
\\	[動詞]	ぐいと突く	
\\	[名詞]	新陳代謝	
\\	「いくら食べても太らない体質だから大丈夫だ」「今この瞬間世界中の人を敵に回したよ」	
\\	複数形チェック
\\	[形容詞]	ギザギザした	
\\	[名詞]	形而上派詩人	
\\	複数形チェック
\\	[名詞]	カケス類【鳥】	
\\	複数形チェック
\\	[名詞]	いん石	
\\	隕石が燃えて、流星となる。	
\\	複数形チェック
\\	[名詞]	危険	
\\	複数形チェック
\\	[名詞]	微生物	
\\	複数形チェック
\\	[名詞]	放棄	
\\	複数形チェック
\\	[名詞]	仲買人	
\\	複数形チェック
\\	[形容詞]	喜ぶ	
\\	[名詞]	環境	
\\	複数形チェック
\\	[名詞]	ごたまぜ	
\\	複数形チェック
\\	[名詞]	蜃気楼	
\\	蜃気楼は幻影だと言われている。	
\\	複数形チェック
\\	[名詞]	軍事政権	
\\	複数形チェック
\\	[名詞]	流産	
\\	流産はとても辛い経験です。	
\\	複数形チェック
\\	[名詞]	技巧、こつ、わざ	
\\	ついに剣道の呼吸をつかんだ。	
\\	複数形チェック
\\	[名詞]	軽犯罪	
\\	複数形チェック
\\	[名詞]	迷路	
\\	複数形チェック
\\	[名詞]	疑念	
\\	複数形チェック
\\	[名詞]	ラマ僧	
\\	複数形チェック
\\	[動詞]	軽減させる、和らげる	
\\	[動詞]	弱る、しおれる	
\\	[名詞]	ほくろ	
\\	針ほどのことを棒ほどにいうのはどうかと思います。	
\\	複数形チェック
\\	[動詞]	ほめたたえる	
\\	[形容詞]	病的な	
\\	[動詞]	洗濯する	
\\	[形容詞]	瀕死の、消滅しかかった	
\\	[動詞]	悔しがらせる	
\\	[名詞]	甘やかす	
\\	複数形チェック
\\	[名詞]	まだら	
\\	複数形チェック
\\	[名詞]	ヒョウ、クロヒョウ	
\\	複数形チェック
\\	[名詞]	据え付け	
\\	複数形チェック
\\	[名詞]	被害妄想	
\\	複数形チェック
\\	[名詞]	ラバ	
\\	老人はラバに砂のいっぱい入った袋をのせた。	
\\	複数形チェック
\\	[名詞]	等価、同等	
\\	複数形チェック
\\	[動詞]	じっくり考える	
\\	[形容詞]	視野の狭い	
\\	[名詞]	壁画	
\\	複数形チェック
\\	[名詞]	病理学	
\\	複数形チェック
\\	[形容詞]	暗い	
\\	[名詞]	家⻑	
\\	複数形チェック
\\	[名詞]	突然変異体	
\\	複数形チェック
\\	[名詞]	貴族	
\\	複数形チェック
\\	[動詞]	がみがみ小言を言う	
\\	彼の妻は彼に一日中がみがみ言っている。	
\\	[名詞]	ご愛顧	
\\	日頃のご愛顧にお礼申し上げます。	
\\	複数形チェック
\\	[動詞]	述べる	
\\	[動詞]	ひいきにする	
\\	あいつの慇懃無礼なものの言い方、思い出すだけで不愉快になるよ。	
\\	[名詞]	鼻音	
\\	複数形チェック
\\	[名詞]	給与、従業員名簿	
\\	彼はまだその会社に雇われている。	
\\	複数形チェック
\\	[名詞]	吐き気	
\\	複数形チェック
\\	[動詞]	売り歩く	
\\	物売るっていうレベルじゃねぇぞ。	
\\	[動詞]	寄り添う	
\\	子犬たちは母親にぴったり寄り添っていた。	
\\	[名詞]	系図、家系、系統	
\\	複数形チェック
\\	[形容詞]	神経の	
\\	[名詞]	強い好み、傾向	
\\	複数形チェック
\\	[名詞]	神経単位	
\\	複数形チェック
\\	[名詞]	多年生植物、続くもの	
\\	複数形チェック
\\	[動詞]	少しずつかじる	
\\	つまみ食いと言うか、完食していたように見えたが。	
\\	[名詞]	周囲	
\\	複数形チェック
\\	[形容詞]	軽快な	
\\	公園で見かけたリスを写真に収めようとしたが、すばしっこくて一枚もいいのが撮れなかった。	
\\	[名詞]	偽証	
\\	男性よりも女性のほうが偽証罪を犯す。	
\\	複数形チェック
\\	[名詞]	乳首	
\\	複数形チェック
\\	[名詞]	手当、特権	
\\	有料会員専用のコミュに入会出来る特典があります。	
\\	複数形チェック
\\	[名詞]	窒素	
\\	複数形チェック
\\	[形容詞]	意気揚々とした	
\\	[形容詞]	名ばかりの	
\\	[形容詞]	破壊的な、致命的な	
\\	[名詞]	軽い突き	
\\	複数形チェック
\\	[動詞]	犯す	
\\	[名詞]	ゼロ	
\\	複数形チェック
\\	[名詞]	ペルソナ	
\\	複数形チェック
\\	[名詞]	結婚式	
\\	複数形チェック
\\	[名詞]	花びら	
\\	複数形チェック
\\	[形容詞]	気の狂った	
\\	[名詞]	座席	
\\	複数形チェック
\\	[形容詞]	肥満の	
\\	[名詞]	幽霊、幻、幻影	
\\	複数形チェック
\\	[形容詞]	不快な	
\\	[名詞]	薬	
\\	複数形チェック
\\	[名詞]	わいせつ	
\\	複数形チェック
\\	[名詞]	慈善活動	
\\	複数形チェック
\\	[名詞]	妄想を持った人	
\\	この問題は、アイドルのエゴとヲタのエゴの衝突だ。	
\\	複数形チェック
\\	[名詞]	杭、ピケ	
\\	複数形チェック
\\	[名詞]	前兆	
\\	私はいつも思っていた、心筋梗塞を患うことは死期を知らせる前兆だと。	
\\	複数形チェック
\\	[名詞]	巡礼の旅	
\\	複数形チェック
\\	[形容詞]	不吉な、不気味な	
\\	空が怪しい。雨が降るかな。	
\\	[名詞]	ポン引き	
\\	複数形チェック
\\	[名詞]	猛攻撃	
\\	複数形チェック
\\	[名詞]	落とし穴	
\\	複数形チェック
\\	[名詞]	泥	
\\	複数形チェック
\\	[名詞]	軸頭、中心点、要点	
\\	複数形チェック
\\	[名詞]	不透明体	
\\	複数形チェック
\\	[形容詞]	重要な	
\\	[名詞]	アヘン	
\\	複数形チェック
\\	[動詞]	静める、なだめる	
\\	[名詞]	目、光学の	
\\	光ファイバーケーブルは膨大な量の情報を搬送できる。	
\\	複数形チェック
\\	[名詞]	偽薬	
\\	複数形チェック
\\	[名詞]	著作、作品	
\\	複数形チェック
\\	[名詞]	盗作	
\\	複数形チェック
\\	[名詞]	神のお告げ	
\\	オラクルマスターって言うからスピリチュアル関係かと思ったらコンピューター関係じゃん。つまらん。	
\\	複数形チェック
\\	[名詞]	小隊	
\\	複数形チェック
\\	[名詞]	法令	
\\	複数形チェック
\\	[名詞]	広場	
\\	その広場は人で埋め尽くされている。	
\\	複数形チェック
\\	[名詞]	乱飲乱舞の酒宴	
\\	複数形チェック
\\	[名詞]	弁論、訴答	
\\	複数形チェック
\\	[形容詞]	率直な	
\\	[名詞]	ぜいたくなもの	
\\	複数形チェック
\\	[動詞]	追い抜く	
\\	[名詞]	合板、ベニヤ板	
\\	複数形チェック
\\	[名詞]	熱烈な歓迎	
\\	複数形チェック
\\	[名詞]	肺炎	
\\	肺炎にかかると呼吸困難になる。	
\\	複数形チェック
\\	[形容詞]	誇張した	
\\	[名詞]	さや	
\\	複数形チェック
\\	[名詞]	過剰投与	
\\	その中毒者は麻薬の過剰摂取で死んだ。	
\\	複数形チェック
\\	[形容詞]	心を打つ、胸を刺すような	
\\	[副詞]	過度に、非常に	
\\	あいつの慇懃無礼なものの言い方、思い出すだけで不愉快になるよ。	
\\	[形容詞]	落ち着いた、自制心のある	
\\	[形容詞]	決定的な、最も重要な	
\\	[名詞]	花粉	
\\	トムは花粉症がある。	
\\	複数形チェック
\\	[動詞]	影を薄くさせる	
\\	[名詞]	毛穴	
\\	複数形チェック
\\	[名詞]	交渉開始、提案	
\\	複数形チェック
\\	[名詞]	門	
\\	複数形チェック
\\	[形容詞]	口に合う、趣味にかなう	
\\	[形容詞]	豪華な、スマートな	
\\	[形容詞]	容易に分かる、明白な	
\\	[名詞]	管区、選挙区	
\\	複数形チェック
\\	[形容詞]	ませた、早咲きの	
\\	彼女はちょっとおませさんだね。	
\\	[名詞]	反動主義者	
\\	複数形チェック
\\	[名詞]	先駆者、前兆	
\\	複数形チェック
\\	[名詞]	そっけない拒絶	
\\	複数形チェック
\\	[形容詞]	略奪する	
\\	[形容詞]	反抗的な、扱いにくい	
\\	[名詞]	苦境	
\\	複数形チェック
\\	[名詞]	調査、偵察	
\\	複数形チェック
\\	[名詞]	前ぶれ、前兆	
\\	複数形チェック
\\	[動詞]	取り戻す、弁償する	
\\	[形容詞]	非常識な、不合理な	
\\	[名詞]	正直	
\\	複数形チェック
\\	[名詞]	霊⻑類	
\\	複数形チェック
\\	[名詞]	反射的な動作	
\\	複数形チェック
\\	[形容詞]	手つかずの	
\\	[形容詞]	帝王の、王にふさわしい	
\\	[名詞]	執行猶予	
\\	私の試用期間は終わりに近づいてきている。	
\\	複数形チェック
\\	[名詞]	摂生	
\\	複数形チェック
\\	[名詞]	獲得	
\\	複数形チェック
\\	[名詞]	登記所	
\\	複数形チェック
\\	[形容詞]	けた外れの、驚異的な	
\\	[動詞]	復帰させる	
\\	[名詞]	神童	
\\	かれは本当に神童である。	
\\	複数形チェック
\\	[名詞]	逆戻り	
\\	複数形チェック
\\	[動詞]	神聖を汚す	
\\	[動詞]	(権利などを)放棄する	
\\	[動詞]	申し出る	
\\	[名詞]	後悔	
\\	彼には良心の呵責ってものがない。	
\\	複数形チェック
\\	[動詞]	激増する、繁殖する	
\\	[名詞]	会う約束	
\\	どこで待ち合わせ?	
\\	複数形チェック
\\	[形容詞]	多作の、多産な	
\\	[名詞]	演奏、表現、翻訳	
\\	複数形チェック
\\	[形容詞]	顕著な	
\\	それはどうやって発音するのですか?	
\\	[名詞]	裏切り者、背教者	
\\	複数形チェック
\\	[名詞]	所有者	
\\	今回はプロプライエタリソフトウェアの危険性について話すべくに参りました。	
\\	複数形チェック
\\	[動詞]	元気を回復させる	
\\	[名詞]	所有者	
\\	複数形チェック
\\	[動詞]	補給する	
\\	[名詞]	売春、変節	
\\	援助交際をほのめかすネット掲示板に書き込んだ女子高生が検挙されました。	
\\	複数形チェック
\\	[動詞]	複写する	
\\	[名詞]	主人公、主唱者	
\\	本当の主人公は誰にも撃ち殺されなかった。	
\\	複数形チェック
\\	[名詞]	懲戒、非難、叱責	
\\	複数形チェック
\\	[名詞]	勇気	
\\	複数形チェック
\\	[名詞]	必需品、必要条件	
\\	複数形チェック
\\	[形容詞]	偽りの	
\\	[形容詞]	弾力のある	
\\	[名詞]	ペンネーム、偽名	
\\	複数形チェック
\\	[形容詞]	反響する、共鳴する	
\\	[名詞]	精神、心	
\\	複数形チェック
\\	[名詞]	りっぱな態度	
\\	複数形チェック
\\	[名詞]	だじゃれ	
\\	これは駄洒落です。	
\\	複数形チェック
\\	[形容詞]	呼吸の	
\\	泳者の呼吸機能の特性について。	
\\	[動詞]	パンクする	
\\	一本だけタイヤの空気圧が極端に減ってるんだ。釘でも刺さってるのかな。	
\\	[名詞]	一休み	
\\	複数形チェック
\\	[名詞]	専門家、評論家	
\\	複数形チェック
\\	[動詞]	復活させる	
\\	[形容詞]	懲罰的な、刑罰の	
\\	[名詞]	復活	
\\	複数形チェック
\\	[形容詞]	推定上の	
\\	[動詞]	仕返しをする	
\\	[名詞]	パット	
\\	複数形チェック
\\	[名詞]	報復	
\\	複数形チェック
\\	[名詞]	4倍	
\\	複数形チェック
\\	[名詞]	網膜	
\\	複数形チェック
\\	[名詞]	苦境、窮地、泥沼	
\\	複数形チェック
\\	[名詞]	報い、応報、天罰	
\\	複数形チェック
\\	[名詞]	量、量子	
\\	日本語は量子的な言語です。物理学にとっての量子物理学であるように、ちょうど言語にとってのそれであるのです。	
\\	複数形チェック
\\	[動詞]	改良する	
\\	[動詞]	鎮める	
\\	[動詞]	鳴り響く	
\\	[動詞]	抑える、鎮める	
\\	[動詞]	悪口を言う	
\\	[形容詞]	典型的な、真髄の	
\\	[動詞]	生気を回復させる	
\\	[名詞]	警句、辛辣なことば、皮肉	
\\	複数形チェック
\\	[名詞]	取消し	
\\	複数形チェック
\\	[名詞]	奇癖	
\\	複数形チェック
\\	[名詞]	強い嫌悪、反感	
\\	複数形チェック
\\	[形容詞]	過激な、激しい、狂暴な	
\\	あなたは病犬に噛まれたら、破傷風の予防注射が必要となってしまいます。	
\\	[形容詞]	広まっている	
\\	[名詞]	乗っ取り屋	
\\	複数形チェック
\\	[名詞]	厳格	
\\	複数形チェック
\\	[形容詞]	蔓延した	
\\	[名詞]	リベット	
\\	複数形チェック
\\	[名詞]	城壁、防御	
\\	複数形チェック
\\	[名詞]	ゴキブリ	
\\	複数形チェック
\\	[名詞]	深い恨み	
\\	複数形チェック
\\	[名詞]	齧⻭動物、ネズミ	
\\	複数形チェック
\\	[名詞]	身代金	
\\	乗っ取り犯人は200万ドルの身の代金を要求した。	
\\	複数形チェック
\\	[名詞]	跳ね回ること	
\\	複数形チェック
\\	[名詞]	大言壮語、怒号	
\\	複数形チェック
\\	[名詞]	名簿	
\\	複数形チェック
\\	[名詞]	関係、信頼	
\\	複数形チェック
\\	[名詞]	総くずれ	
\\	複数形チェック
\\	[名詞]	友好回復	
\\	複数形チェック
\\	[名詞]	歩きまわること	
\\	複数形チェック
\\	[名詞]	理論的解釈、根本的理由	
\\	複数形チェック
\\	[名詞]	放浪者	
\\	複数形チェック
\\	[形容詞]	耳障りな、騒々しい	
\\	[形容詞]	騒々しい	
\\	弟の乱暴はこのごろ目に余る。	
\\	[動詞]	後悔する	
\\	[動詞]	妨げる	
\\	[形容詞]	悲しそうな	
\\	[動詞]	(⻭をむきだして)うなる	
\\	[名詞]	破裂	
\\	複数形チェック
\\	[名詞]	ナトリウム《記号
\\	複数形チェック
\\	[動詞]	下落する	
\\	[名詞]	慰め、癒し	
\\	複数形チェック
\\	[名詞]	海難救助	
\\	複数形チェック
\\	[形容詞]	陰気な、憂うつな	
\\	[名詞]	高潔、神聖、尊厳	
\\	複数形チェック
\\	[形容詞]	不潔な、みすぼらしい	
\\	[形容詞]	快活な、楽天的な	
\\	[形容詞]	まばらな	
\\	[形容詞]	冷笑的な、あざけりの	
\\	[名詞]	言葉などのほとばしり	
\\	複数形チェック
\\	[名詞]	救済者	
\\	イエス・キリストは私の救い主です。	
\\	複数形チェック
\\	[名詞]	槍の穂先、先鋒	
\\	複数形チェック
\\	[名詞]	頭皮	
\\	複数形チェック
\\	[名詞]	幽霊、こわいもの	
\\	一匹の妖怪がヨーロッパを徘徊している。共産主義という妖怪が。	
\\	複数形チェック
\\	[形容詞]	痛烈な	
\\	[形容詞]	脊柱の	
\\	[名詞]	精神分裂症	
\\	複数形チェック
\\	[名詞]	継いだもの、接ぎ木	
\\	複数形チェック
\\	[名詞]	あざけり	
\\	複数形チェック
\\	[名詞]	スパイ、幽霊	
\\	複数形チェック
\\	[名詞]	磨くこと	
\\	複数形チェック
\\	[形容詞]	散発性の、時々起こる	
\\	[名詞]	災難	
\\	複数形チェック
\\	[名詞]	ばか騷ぎ、酒宴	
\\	複数形チェック
\\	[名詞]	金切り声	
\\	複数形チェック
\\	[名詞]	短距離競走	
\\	複数形チェック
\\	[形容詞]	几帳面な、綿密な	
\\	[名詞]	にべもない拒絶	
\\	その申し出はお受けしかねる。	
\\	複数形チェック
\\	[動詞]	あわてて走る	
\\	ほふく前進はもっと素早くゴキブリのように這え!	
\\	[動詞]	早口でしゃべる	
\\	[動詞]	脱退する	
\\	[名詞]	口論	
\\	複数形チェック
\\	[名詞]	党派心の強い人	
\\	複数形チェック
\\	[名詞]	騎兵大隊、戦隊	
\\	複数形チェック
\\	[名詞]	鎮静剤	
\\	複数形チェック
\\	[動詞]	浪費する	
\\	ようやく稼いだ金を無駄遣いするなよ!	
\\	[形容詞]	魅惑的な	
\\	[名詞]	悲鳴	
\\	黒塗りのリムジンがキーというタイヤの音と共に角を曲がった。	
\\	複数形チェック
\\	[形容詞]	種の多い	
\\	[名詞]	横目、斜視	
\\	複数形チェック
\\	[動詞]	しみ出る	
\\	[名詞]	もがき、もじもじすること	
\\	複数形チェック
\\	[形容詞]	地震の	
\\	[形容詞]	よどんだ、臭い	
\\	[形容詞]	独創性に富んだ、重大な	
\\	[名詞]	停滞、不況	
\\	複数形チェック
\\	[名詞]	神学校	
\\	複数形チェック
\\	[名詞]	行き詰まり	
\\	複数形チェック
\\	[名詞]	下水	
\\	下水管がつまっている。	
\\	複数形チェック
\\	[名詞]	忠実な党員	
\\	複数形チェック
\\	[名詞]	性差別論者	
\\	そうじゃない。性差別やら女性蔑視やら、そういうものとは関係なく、男と女は違うんだ。	
\\	複数形チェック
\\	[形容詞]	威厳のある	
\\	政府役人の豪邸が略奪された。	
\\	[名詞]	足かせ、束縛	
\\	あなたの足に銀の足かせをはめましょう。	
\\	複数形チェック
\\	[形容詞]	頑丈な	
\\	[名詞]	ずれ、せん断	
\\	彼らの仕事は羊の毛を刈ることだ。	
\\	複数形チェック
\\	[動詞] [不規則動詞]	突き破る	
\\	[名詞]	輝き	
\\	複数形チェック
\\	[名詞]	秘かなやり方、内密	
\\	複数形チェック
\\	[名詞]	屋根板	
\\	複数形チェック
\\	[名詞]	ビール用のジョッキ	
\\	複数形チェック
\\	[名詞]	まがい物	
\\	複数形チェック
\\	[形容詞]	星の、恒星の	
\\	[名詞]	かま	
\\	複数形チェック
\\	[名詞]	不名誉、汚名	
\\	複数形チェック
\\	[名詞]	副業	
\\	複数形チェック
\\	[名詞]	一定期間の仕事	
\\	複数形チェック
\\	[名詞]	連峰	
\\	複数形チェック
\\	[名詞]	蓄積	
\\	複数形チェック
\\	[名詞]	雄親	
\\	複数形チェック
\\	[動詞] [不規則動詞]	まき散らす	
\\	[形容詞]	かなり大きな	
\\	[形容詞]	苦しんでいる、襲われた	
\\	夜通しの豪雨は、被災したフェリーからの生存者の救出を妨げた。	
\\	[名詞]	皮を剥ぐ人	
\\	複数形チェック
\\	[形容詞]	耳ざわりな、あくどい	
\\	[名詞]	小競り合い	
\\	複数形チェック
\\	[形容詞]	厳しい、厳重な	
\\	[形容詞]	安っぽい	
\\	[名詞]	要塞	
\\	あの村が敵の最後の牙城です。	
\\	複数形チェック
\\	[動詞]	なめらかにする	
\\	[名詞]	離れわざ、スタント	
\\	それは売名行為だ。	
\\	複数形チェック
\\	[名詞]	小魚の片身、薄片	
\\	複数形チェック
\\	[形容詞]	破壊する	
\\	[名詞]	前かがみ	
\\	うつむかないで。	
\\	複数形チェック
\\	[動詞]	打倒する、堕落させる	
\\	[名詞]	どろどろした沈澱物	
\\	複数形チェック
\\	[動詞]	屈服する	
\\	サミはレーラの魅力の虜になりました。	
\\	[名詞]	中傷	
\\	複数形チェック
\\	[形容詞]	豪華な	
\\	[名詞]	にやにや笑い、作り笑い	
\\	複数形チェック
\\	[動詞]	取って代わる	
\\	[動詞] [不規則動詞]	襲う	
\\	諸神は悪を雷で打つという。	
\\	[形容詞]	しなやかな	
\\	[形容詞]	気取った	
\\	[名詞]	割り増し料金	
\\	複数形チェック
\\	[名詞]	代理人	
\\	複数形チェック
\\	[形容詞]	至る所に存在する	
\\	[形容詞]	影響されやすい	
\\	彼女は風邪を引きやすかった。	
\\	[名詞]	最後通告	
\\	複数形チェック
\\	[名詞]	いばり散らすこと	
\\	複数形チェック
\\	[形容詞]	異様な、不可解な	
\\	[名詞]	掃除、一掃すること	
\\	複数形チェック
\\	[動詞]	掘り出す	
\\	[名詞]	失神	
\\	複数形チェック
\\	[形容詞]	一方的な	
\\	[形容詞]	暗黙の	
\\	二人の間には暗黙の了解があったに違いない。	
\\	[動詞]	ほぐれる、解明される	
\\	[名詞]	計算、勘定	
\\	複数形チェック
\\	[形容詞]	手に負えない、荒れ狂う	
\\	[動詞]	勝手に変更する	
\\	書類は勝手にいじられた形跡があった。	
\\	[名詞]	楽天的な	
\\	ノリノリの音楽がかかると、つい踊っちゃうんだよね。	
\\	複数形チェック
\\	[形容詞]	もつれた	
\\	[名詞]	養育、しつけ	
\\	彼らは子供の育てかたと躾けかたではお互いに意見が合わなかった。	
\\	複数形チェック
\\	[形容詞]	じれったい	
\\	[名詞]	大騒ぎ、騒音	
\\	私は誕生日の日に友達に飲まされて大騒ぎをしてしまいました。	
\\	複数形チェック
\\	[名詞]	癇癪	
\\	子供がおもちゃを買って欲しいと駄々をこねていた。	
\\	複数形チェック
\\	[名詞]	結果	
\\	複数形チェック
\\	[名詞]	錆び	
\\	複数形チェック
\\	[名詞]	成り上がり者	
\\	複数形チェック
\\	[名詞]	あざけり	
\\	複数形チェック
\\	[名詞]	故意の破壊、蛮行	
\\	複数形チェック
\\	[形容詞]	はりつめた	
\\	[名詞]	先駆者、指導的地位	
\\	複数形チェック
\\	[名詞]	けばけばしい装飾	
\\	複数形チェック
\\	[動詞]	方向を変える	
\\	[動詞]	満ちている	
\\	[名詞]	復讐、報復	
\\	複数形チェック
\\	[形容詞]	頑強な	
\\	[名詞]	確認、立証、検証、証明	
\\	複数形チェック
\\	[名詞]	主義、教義	
\\	複数形チェック
\\	[形容詞]	本当の、真実の	
\\	[名詞]	方針、大意	
\\	複数形チェック
\\	[名詞]	土地ことば	
\\	複数形チェック
\\	[形容詞]	薄弱な、微細な	
\\	[形容詞]	鼓動して	
\\	散歩をしながら、トムは鮮やかな赤とオレンジ色の紅葉を堪能しました。	
\\	[形容詞]	なまぬるい	
\\	[動詞]	競う	
\\	[名詞]	地域、地形、地勢	
\\	暖かい軽い空気は山越えすると上空に上昇し、地上には下りてこないことなど、風は地形によって変わります。	
\\	複数形チェック
\\	[名詞]	見張り	
\\	複数形チェック
\\	[形容詞]	簡潔な	
\\	[形容詞]	絶えず警戒している	
\\	警戒するに越したことはない。	
\\	[名詞]	証拠、証明書、推薦状	
\\	複数形チェック
\\	[動詞]	嫌疑を晴らす	
\\	[形容詞]	神学の	
\\	[名詞]	弁明、正当性の立証	
\\	複数形チェック
\\	[名詞]	絞り弁、気管	
\\	複数形チェック
\\	[動詞]	放棄する、差し控える	
\\	[名詞]	殺し屋、悪党	
\\	チンピラがコンビニの店員をぶちのめした。	
\\	複数形チェック
\\	[名詞]	放棄	
\\	複数形チェック
\\	[動詞]	妨害する	
\\	[名詞]	減少、衰微	
\\	複数形チェック
\\	[名詞]	横並び、層	
\\	複数形チェック
\\	[名詞]	番人	
\\	複数形チェック
\\	[名詞]	色合い、〜じみたところ	
\\	東の空が赤みを帯びている。	
\\	複数形チェック
\\	[名詞]	ウサギの巣穴	
\\	複数形チェック
\\	[名詞]	(重い・学術的な)本	
\\	複数形チェック
\\	[名詞]	分水嶺	
\\	複数形チェック
\\	[名詞]	魚雷、発破	
\\	複数形チェック
\\	[形容詞]	わがままな	
\\	[名詞]	胴	
\\	人形のおなかの横にあるねじをまくと、腕を回してでんぐり返しをしながら前に進みます。	
\\	複数形チェック
\\	[名詞]	殴打	
\\	複数形チェック
\\	[形容詞]	怒りっぽい、敏感な	
\\	[名詞]	ひと吹き、かすかな匂い	
\\	複数形チェック
\\	[名詞]	客引き、予想屋	
\\	複数形チェック
\\	[形容詞]	気まぐれな	
\\	[形容詞]	高くそびえる	
\\	[名詞]	泣き言	
\\	複数形チェック
\\	[名詞]	郡区	
\\	複数形チェック
\\	[名詞]	旋風	
\\	複数形チェック
\\	[名詞]	軌道	
\\	「ああ。きれいなお星様」  呟いた時、ふと星が流れて、青い光がすっと斜に、あえかな尾を引いて、消えた。	
\\	複数形チェック
\\	[名詞]	泡立て器	
\\	複数形チェック
\\	[形容詞]	半透明の	
\\	[名詞]	売春婦、娼婦	
\\	くそやりまん。	
\\	複数形チェック
\\	[名詞]	反逆罪、不信、裏切り	
\\	彼は大逆罪で島流しにされた。	
\\	複数形チェック
\\	[動詞]	用いる、行使する	
\\	民衆の熱狂的な彼への支持は、首相の座にとどまりながらも大統領の権限の発揮を可能にしそうだ。	
\\	[名詞]	震動、震え	
\\	私は小さな揺れを感じたが、地震ではなく、ただトラックが通っただけだった。	
\\	複数形チェック
\\	[名詞]	たなぼた	
\\	複数形チェック
\\	[名詞]	人⺠の保護者	
\\	複数形チェック
\\	[形容詞]	内にこもった	
\\	2回目の人工内耳移植への低所得者医療扶助制度の適用不許可を取り下げる。	
\\	[名詞]	騎兵	
\\	複数形チェック
\\	[名詞]	魔法使い	
\\	エドガーさんは魔法使いだった事もあるんですか。	
\\	複数形チェック
\\	[名詞]	(劇などの)一座	
\\	複数形チェック
\\	[名詞]	ぐらつき、動揺	
\\	複数形チェック
\\	[名詞]	停戦	
\\	複数形チェック
\\	[名詞]	子宮	
\\	赤ちゃんは子宮内の羊水で発育します。	
\\	複数形チェック
\\	[名詞]	てくてく歩き	
\\	複数形チェック
\\	[名詞]	口論	
\\	複数形チェック
\\	[名詞]	結核、肺結核	
\\	何年か前には、結核にかかっていると知らされることは死の宣告を聞くのに等しかった。	
\\	複数形チェック
\\	[名詞]	激怒	
\\	複数形チェック
\\	[形容詞]	騒々しい、騒動を起こす	
\\	[名詞]	令状	
\\	複数形チェック
\\	[名詞]	ろくろ師、旋盤工	
\\	複数形チェック
\\	[名詞]	狂信者	
\\	複数形チェック
\\	[名詞]	出席者、観客	
\\	複数形チェック
\\	[名詞]	亜	
\\	真鍮は銅と亜鉛の合金である。	
\\	複数形チェック
\\	[動詞]	和らぐ、治まる	
\\	風が和らいだ。	
\\	[名詞]	前例、前項、先祖	
\\	先行詞が
\\	の場合は
\\	を用いるのが普通です。	
\\	複数形チェック
\\	[動詞]	放棄する	
\\	[名詞]	控えの間、待合室	
\\	複数形チェック
\\	[形容詞]	腹部の	
\\	[名詞]	おどけた仕草	
\\	複数形チェック
\\	[動詞]	ひどく嫌う	
\\	私たちは暴力を憎む。	
\\	[名詞]	抗原	
\\	複数形チェック
\\	[形容詞]	大嫌いな、憎むべき	
\\	[名詞]	防腐剤、消毒剤	
\\	複数形チェック
\\	[形容詞]	惨めな	
\\	[名詞]	対照、正反対、対照法	
\\	複数形チェック
\\	[形容詞]	ひどい	
\\	ヒマラヤ山中で雪男の足跡が発見されたらしい。	
\\	[名詞]	金敷	
\\	複数形チェック
\\	[形容詞]	並んで	
\\	医者というものは、医学の最新の成果に遅れないように、ついていかなければならない。	
\\	[名詞]	穴	
\\	複数形チェック
\\	[動詞]	同意する	
\\	[名詞]	頂点	
\\	複数形チェック
\\	[名詞]	継承	
\\	エリザベス女王の即位50年の記念に贈られたものみたいです。	
\\	複数形チェック
\\	[名詞]	幽霊	
\\	複数形チェック
\\	[名詞]	拍手かっさい	
\\	複数形チェック
\\	[動詞]	見積もる、評価する	
\\	[名詞]	賞賛	
\\	複数形チェック
\\	[名詞]	耕作に適した土地	
\\	複数形チェック
\\	[名詞]	どんぐり	
\\	複数形チェック
\\	[名詞]	仲裁	
\\	仲裁では普通6ヶ月に裁決を下す。	
\\	複数形チェック
\\	[名詞]	黙って従うこと	
\\	複数形チェック
\\	[名詞]	休戦	
\\	複数形チェック
\\	[名詞]	エーカー数、地所	
\\	複数形チェック
\\	[名詞]	腋の下	
\\	複数形チェック
\\	[形容詞]	苦い	
\\	[名詞]	覚醒	
\\	複数形チェック
\\	[名詞]	格言	
\\	複数形チェック
\\	[名詞]	砒素	
\\	複数形チェック
\\	[動詞]	休会にする、延期する	
\\	会議は次の週まで延期された。	
\\	[名詞]	結合、関節	
\\	複数形チェック
\\	[名詞]	付属物	
\\	複数形チェック
\\	[名詞]	策略	
\\	複数形チェック
\\	[名詞]	忠告、警告	
\\	複数形チェック
\\	[名詞]	襲撃する、悩ませる	
\\	複数形チェック
\\	[名詞]	保護	
\\	イージス衝突 
\\	なぜ避けられなかったか。	
\\	複数形チェック
\\	[名詞]	検査、分析	
\\	複数形チェック
\\	[名詞]	ふりをすること、気取り	
\\	複数形チェック
\\	[名詞]	査定者、補佐役	
\\	複数形チェック
\\	[名詞]	侮辱	
\\	複数形チェック
\\	[名詞]	同化	
\\	複数形チェック
\\	[形容詞]	故郷を遠く離れた	
\\	[副詞]	船尾に	
\\	[形容詞]	徒歩で	
\\	[副詞]	ばらばらに	
\\	[形容詞]	後の	
\\	[名詞]	無神論者	
\\	まともな無神論者は、神の不存在を証明しようとはしない。	
\\	複数形チェック
\\	[名詞]	凝集、集合	
\\	複数形チェック
\\	[動詞]	償いをする	
\\	彼は死んで罪を清算した。	
\\	[形容詞]	機敏な	
\\	[名詞]	服装	
\\	トムは女性のどんなファッションが好き?	
\\	複数形チェック
\\	[名詞]	土地均分論者	
\\	複数形チェック
\\	[動詞]	増大させる、増やす	
\\	[名詞]	苦しみ	
\\	複数形チェック
\\	[形容詞]	吉兆の	
\\	[形容詞]	空気を通さない	
\\	[名詞]	自閉症	
\\	複数形チェック
\\	[名詞]	市会議員	
\\	複数形チェック
\\	[名詞]	補助者	
\\	この文は現在完了形です。
\\	は動詞ではなく、助動詞。	
\\	複数形チェック
\\	[名詞]	藻、藻類	
\\	複数形チェック
\\	[名詞]	強欲	
\\	私の友人は私の財産だ。ですので友人を蓄えたがる私の貪欲さを見逃してください。	
\\	複数形チェック
\\	[名詞]	調整	
\\	複数形チェック
\\	[形容詞]	嫌って	
\\	[動詞]	静める	
\\	[名詞]	鳥類飼育場	
\\	複数形チェック
\\	[名詞]	寓意物語、寓意、象徴	
\\	複数形チェック
\\	[名詞]	梱	
\\	複数形チェック
\\	[名詞]	口論	
\\	複数形チェック
\\	[名詞]	小石、砂利	
\\	複数形チェック
\\	[動詞]	合併する、融合する	
\\	[名詞]	香油	
\\	複数形チェック
\\	[名詞]	アンブル	
\\	複数形チェック
\\	[名詞]	たわ言	
\\	複数形チェック
\\	[名詞]	従順な	
\\	複数形チェック
\\	[名詞]	盗賊	
\\	複数形チェック
\\	[名詞]	償い	
\\	複数形チェック
\\	[名詞]	手押し車	
\\	複数形チェック
\\	[名詞]	親善	
\\	複数形チェック
\\	[形容詞]	内気な	
\\	[名詞]	両生動物	
\\	複数形チェック
\\	[名詞]	床張り板、目板	
\\	複数形チェック
\\	[名詞]	円形競技場	
\\	複数形チェック
\\	[動詞] [不規則動詞]	こしらえる、生じさせる	
\\	金が金を生む。	
\\	[名詞]	補助者、付属物	
\\	複数形チェック
\\	[動詞]	偽って伝える、裏切る	
\\	スウィート博士の性格はその名とは裏腹であった。	
\\	[名詞]	魚釣りをする人	
\\	複数形チェック
\\	[名詞]	慈悲心	
\\	複数形チェック
\\	[名詞]	併合、付加物	
\\	複数形チェック
\\	[動詞]	遺言で譲る	
\\	[動詞]	全滅させる	
\\	[名詞]	遺族、後に残された者	
\\	ペットロスとは、ペットを失った飼い主に起きるショックや怒りなどの反応をいう。	
\\	複数形チェック
\\	[名詞]	食欲不振	
\\	彼女は10代のころ拒食症になった。	
\\	複数形チェック
\\	[名詞]	寝台	
\\	複数形チェック
\\	[動詞]	敵に回す	
\\	[動詞] [不規則動詞]	懇願する	
\\	[名詞]	二頭筋	
\\	男性はフィットネスセンターへ行くと二頭筋をよく鍛えるみたいだ。	
\\	複数形チェック
\\	[名詞]	塩素	
\\	水道水には塩素、鉛などの有害物質が含まれている可能性がある。	
\\	複数形チェック
\\	[名詞]	二者会談	
\\	複数形チェック
\\	[名詞]	年代記、年表	
\\	いくつかの情報科学年表から重要項目を抜きだし、各項目について簡単なコメントを付けました。	
\\	複数形チェック
\\	[名詞]	胆汁	
\\	複数形チェック
\\	[名詞]	仲良し	
\\	複数形チェック
\\	[名詞]	大波	
\\	複数形チェック
\\	[名詞]	円周	
\\	円周から直径を求めることは可能ですか?	
\\	複数形チェック
\\	[名詞]	2進数	
\\	2015という数は二進法で表すと11111011111となる。	
\\	複数形チェック
\\	[動詞]	制限する、抑える	
\\	[名詞]	双眼鏡、双眼顕微鏡	
\\	複数形チェック
\\	[形容詞]	用意周到な	
\\	[名詞]	バイソン	
\\	コロンブスがアメリカ大陸を発見したとき、バイソン(アメリカ野牛)は、広大な地域に棲んでいた。	
\\	複数形チェック
\\	[名詞]	水槽、貯水池	
\\	複数形チェック
\\	[動詞]	⻘ざめる	
\\	[名詞]	城、要塞	
\\	複数形チェック
\\	[名詞]	冒涜	
\\	「小説? あたし小説は嫌ひですの」  おゝ、ミュウズよ、彼女の冒涜を赦せ。	
\\	複数形チェック
\\	[名詞]	原告	
\\	複数形チェック
\\	[名詞]	きず、汚点	
\\	複数形チェック
\\	[名詞]	透視者	
\\	複数形チェック
\\	[形容詞]	楽しげな、軽率な	
\\	[名詞]	閉所恐怖症	
\\	複数形チェック
\\	[動詞]	前兆となる	
\\	彼の怠惰は将来に対する悪い前兆だった。	
\\	[名詞]	裂開、分裂、卵割	
\\	複数形チェック
\\	[名詞]	自治都市	
\\	複数形チェック
\\	[動詞]	切り裂く、割る	
\\	[形容詞]	豊富な	
\\	[名詞]	群れ	
\\	複数形チェック
\\	[名詞]	ウシ属の動物	
\\	複数形チェック
\\	[動詞]	甘やかす	
\\	[名詞]	稼ぎ手	
\\	複数形チェック
\\	[動詞]	強制する	
\\	私は強要には屈しません。	
\\	[名詞]	短さ、簡潔さ	
\\	複数形チェック
\\	[名詞]	強制	
\\	脅しによるリーダーシップは、いまぼくたちが目にするような結果を生み出しはしない。	
\\	複数形チェック
\\	[名詞]	拘束	
\\	複数形チェック
\\	[名詞]	認知	
\\	複数形チェック
\\	[名詞]	臭化物、陳腐な考え	
\\	複数形チェック
\\	[名詞]	大腸炎	
\\	複数形チェック
\\	[名詞]	気管支炎	
\\	複数形チェック
\\	[名詞]	共謀、談合	
\\	この際、犯罪と癒着を一掃すべきです。	
\\	複数形チェック
\\	[名詞]	主力	
\\	複数形チェック
\\	[名詞]	燃焼	
\\	一酸化炭素とは炭素化合物の不完全燃焼で発生する有害物質です。	
\\	複数形チェック
\\	[形容詞]	ぶっきらぼうな、がさつな	
\\	[名詞]	原告	
\\	複数形チェック
\\	[名詞]	乳母車	
\\	複数形チェック
\\	[形容詞]	従順な、素直な	
\\	[名詞]	棚式寝台	
\\	複数形チェック
\\	[名詞]	堆肥	
\\	複数形チェック
\\	[名詞]	臀部	
\\	複数形チェック
\\	[名詞]	凹	
\\	複数形チェック
\\	[名詞]	サボテン	
\\	サボテンのトゲって何のためにあるか知ってる?	
\\	複数形チェック
\\	[動詞]	なだめる	
\\	[名詞]	リズム	
\\	複数形チェック
\\	[形容詞]	和解の、懐柔的な	
\\	[名詞]	皮膚のたこ	
\\	複数形チェック
\\	[名詞]	付随物	
\\	複数形チェック
\\	[名詞]	天蓋	
\\	複数形チェック
\\	[名詞]	一致	
\\	複数形チェック
\\	[名詞]	うわべだけの言葉	
\\	複数形チェック
\\	[動詞]	恩着せがましく...する	
\\	なんか彼女の言い方って上から目線なんだよね。時々カチンとくる。	
\\	[名詞]	食堂	
\\	複数形チェック
\\	[名詞]	悔やみ、哀悼	
\\	複数形チェック
\\	[名詞]	帆布	
\\	複数形チェック
\\	[名詞]	連合	
\\	複数形チェック
\\	[名詞]	気まぐれ、気まま	
\\	複数形チェック
\\	[名詞]	紙吹雪	
\\	複数形チェック
\\	[形容詞]	気まぐれな、変わりやすい	
\\	[名詞]	合流、合流点	
\\	複数形チェック
\\	[名詞]	発癌物質	
\\	複数形チェック
\\	[動詞]	集まる	
\\	[名詞]	回転木馬、回転ラック	
\\	複数形チェック
\\	[名詞]	含意	
\\	複数形チェック
\\	[名詞]	小箱	
\\	複数形チェック
\\	[動詞]	神聖にする、神に捧げる	
\\	[名詞]	大変動	
\\	複数形チェック
\\	[名詞]	音楽学校	
\\	複数形チェック
\\	[名詞]	白内障、混濁部	
\\	半年前に右目の白内障の手術をしました。	
\\	複数形チェック
\\	[名詞]	積荷、出荷品	
\\	複数形チェック
\\	[名詞]	火線、腐食薬	
\\	複数形チェック
\\	[名詞]	星座	
\\	複数形チェック
\\	[名詞]	いんぎんな紳士	
\\	複数形チェック
\\	[名詞]	非常な驚き、狼狽	
\\	複数形チェック
\\	[名詞]	騎兵、騎兵隊	
\\	複数形チェック
\\	[名詞]	便秘	
\\	複数形チェック
\\	[名詞]	洞窟	
\\	複数形チェック
\\	[動詞]	引き締める	
\\	[名詞]	休止	
\\	複数形チェック
\\	[名詞]	完成	
\\	複数形チェック
\\	[名詞]	もみがら	
\\	ほんとお前人間のクズだな。	
\\	複数形チェック
\\	[名詞]	伝染病	
\\	複数形チェック
\\	[名詞]	偽医者	
\\	複数形チェック
\\	[動詞]	違反する、反する	
\\	[形容詞]	純潔な	
\\	[名詞]	回復	
\\	父は病後、めっきり老け込んでしまった。	
\\	複数形チェック
\\	[名詞]	厳しく非難する	
\\	複数形チェック
\\	[名詞]	痙攣、発作	
\\	複数形チェック
\\	[名詞]	熱狂的愛国主義	
\\	複数形チェック
\\	[名詞]	かご	
\\	複数形チェック
\\	[名詞]	騎士道	
\\	複数形チェック
\\	[形容詞]	豊富な、おびただしい	
\\	[名詞]	非常線	
\\	複数形チェック
\\	[形容詞]	軽蔑的な	
\\	彼女は同僚たちの悪口を言った。	
\\	[名詞]	必然的な帰結	
\\	複数形チェック
\\	[形容詞]	失望した	
\\	[名詞]	検死官	
\\	複数形チェック
\\	[形容詞]	専制的な	
\\	[名詞]	確証	
\\	複数形チェック
\\	[名詞]	困窮	
\\	複数形チェック
\\	[動詞]	腐食する、むしばむ	
\\	[名詞]	制止、阻止、引き止め	
\\	複数形チェック
\\	[名詞]	簡易寝台、吊り床	
\\	複数形チェック
\\	[動詞]	憎む	
\\	彼らはお互いに心の底から憎んでいる。	
\\	[名詞]	入り江	
\\	複数形チェック
\\	[名詞]	迂回、回り道	
\\	複数形チェック
\\	[動詞]	すくむ	
\\	[動詞]	損なう	
\\	[名詞]	キーキー鳴る音	
\\	複数形チェック
\\	[名詞]	損害	
\\	複数形チェック
\\	[名詞]	ひだ、しわ	
\\	複数形チェック
\\	[形容詞]	有害な	
\\	[名詞]	割れ目	
\\	複数形チェック
\\	[動詞]	それる、はずれる	
\\	本題からそれないようにしましょう。	
\\	[動詞]	ぺこぺこする、へつらう	
\\	もうちょっと落ち着いたらどうなの? まったく危なっかしいんだから。	
\\	[名詞]	権限委譲	
\\	複数形チェック
\\	[名詞]	かぎ針編み	
\\	複数形チェック
\\	[形容詞]	悪魔の	
\\	[名詞]	キリストのはりつけ	
\\	複数形チェック
\\	[名詞]	対角線	
\\	「ああ。きれいなお星様」  呟いた時、ふと星が流れて、青い光がすっと斜に、あえかな尾を引いて、消えた。	
\\	複数形チェック
\\	[名詞]	最重要点、難問	
\\	複数形チェック
\\	[名詞]	弁証法	
\\	ヘーゲルの弁証法の秘密は究極的に次のポイントにだけ存在する;神学を経て哲学を否定するために哲学を経て、神学を否定します。	
\\	複数形チェック
\\	[名詞]	あなぐら、地下室	
\\	複数形チェック
\\	[名詞]	意見の相違	
\\	複数形チェック
\\	[形容詞]	不可解な	
\\	[名詞]	言葉遣い	
\\	複数形チェック
\\	[名詞]	骨董品	
\\	複数形チェック
\\	[形容詞]	気後れしている	
\\	[形容詞]	大急ぎで行う、いい加減な	
\\	[名詞]	堤防	
\\	複数形チェック
\\	[動詞]	軽くたたく	
\\	[形容詞]	荒れはてた、傾いた	
\\	[形容詞]	ばかな、気のふれた	
\\	[動詞]	拡張する	
\\	[動詞]	時間を浪費する	
\\	[名詞]	救命ボート	
\\	複数形チェック
\\	[名詞]	塗料	
\\	複数形チェック
\\	[形容詞]	薄汚ない	
\\	[動詞]	ひるませる	
\\	[形容詞]	識別力のある	
\\	[形容詞]	耳を聞こえなくするような	
\\	[名詞]	困惑	
\\	複数形チェック
\\	[名詞]	借方、借方記入	
\\	複数形チェック
\\	[形容詞]	分離した、個別的な	
\\	[名詞]	堕落、頽廃期	
\\	複数形チェック
\\	[形容詞]	だらしのない	
\\	[動詞]	分散化する	
\\	[動詞]	取り出す、追い払う	
\\	[動詞]	多くを破壊する	
\\	[名詞]	薬局、医務室	
\\	複数形チェック
\\	[名詞]	礼儀正しいこと、上品さ	
\\	全生徒がプロムの夜礼儀正しく振る舞っているのを見て驚いた。	
\\	複数形チェック
\\	[動詞]	解剖する	
\\	[名詞]	おとり	
\\	複数形チェック
\\	[動詞]	(情報を)広める	
\\	[形容詞]	老朽化した	
\\	[名詞]	学位論文	
\\	卒論のテーマ決まった?	
\\	複数形チェック
\\	[名詞]	森林伐採	
\\	彼らの生息地は、森林破壊に脅かされています。	
\\	複数形チェック
\\	[動詞]	散らす	
\\	幼少の時、父が死んで、その弟が、時代の衰勢と、自分の怠惰とから、すっかり、身代をつぶしてしまったらしく、後に、筆墨行商人になって、私の家へ、よく来たが、くると、母に叱られて、よわっていた。	
\\	[形容詞]	落胆した	
\\	[名詞]	不一致、不調和	
\\	複数形チェック
\\	[名詞]	描写	
\\	複数形チェック
\\	[形容詞]	末端の	
\\	[名詞]	非行	
\\	複数形チェック
\\	[形容詞]	異なる	
\\	[形容詞]	精神錯乱した	
\\	うちの弟は、よく寝ぼけて変なことを言い出す。	
\\	[形容詞]	従順な	
\\	[名詞]	精神錯乱	
\\	複数形チェック
\\	[名詞]	停滞状態	
\\	複数形チェック
\\	[名詞]	救出、救助	
\\	複数形チェック
\\	[名詞]	住居、住所	
\\	複数形チェック
\\	[名詞]	境界	
\\	複数形チェック
\\	[動詞]	水をぶっかける	
\\	[名詞]	人口統計学	
\\	複数形チェック
\\	[形容詞]	意気消沈した	
\\	彼女は伏し目がちに部屋から出ていった。	
\\	[動詞]	士気をくじく	
\\	[動詞]	滴る、ドリブルする	
\\	[形容詞]	とりすました	
\\	[名詞]	霧雨	
\\	複数形チェック
\\	[名詞]	非難	
\\	複数形チェック
\\	[名詞]	単調でつらい仕事	
\\	複数形チェック
\\	[名詞]	枯渇	
\\	複数形チェック
\\	[名詞]	公爵領	
\\	複数形チェック
\\	[形容詞]	嘆かわしい、気の毒な	
\\	[名詞]	砂丘	
\\	複数形チェック
\\	[形容詞]	堕落した	
\\	[名詞]	糞	
\\	複数形チェック
\\	[名詞]	遺棄物、遺棄船	
\\	複数形チェック
\\	[形容詞]	陶製の	
\\	私は、新居へ移ると共に、私の部屋へ引っ込んで、自分で炊事できるだけの道具ーー土釜、土鍋、七輪の類をととのえた。	
\\	[形容詞]	嘲笑的な	
\\	[動詞]	盗み聞きする	
\\	トムは盗み聞きしています。	
\\	[名詞]	皮膚科医	
\\	複数形チェック
\\	[形容詞]	教会の、聖職者の	
\\	[名詞]	反主流、渦	
\\	複数形チェック
\\	[名詞]	誤信、誤った考え	
\\	複数形チェック
\\	[動詞]	教化する	
\\	[形容詞]	(口)うるさい	
\\	[動詞]	削除する	
\\	[名詞]	動物区系	
\\	複数形チェック
\\	[形容詞]	興奮した	
\\	[名詞]	実行可能なこと	
\\	複数形チェック
\\	[名詞]	平等主義者	
\\	複数形チェック
\\	[名詞]	シダ、シダの茂み	
\\	複数形チェック
\\	[名詞]	意気揚々	
\\	複数形チェック
\\	[名詞]	残忍な人	
\\	複数形チェック
\\	[名詞]	楕円	
\\	複数形チェック
\\	[形容詞]	子の	
\\	[動詞]	明らかにする	
\\	[名詞]	汚物、不潔	
\\	複数形チェック
\\	[動詞]	解放する	
\\	[名詞]	華美な装飾	
\\	複数形チェック
\\	[名詞]	堤防	
\\	複数形チェック
\\	[名詞]	モミ	
\\	複数形チェック
\\	[動詞]	装飾する、飾り立てる	
\\	[名詞]	分裂、分体、核分裂	
\\	複数形チェック
\\	[動詞]	横領する、着服する	
\\	[名詞]	裂け目、割れ目	
\\	複数形チェック
\\	[名詞]	刺繍	
\\	複数形チェック
\\	[形容詞]	断続的な	
\\	[形容詞]	胎児の、胚の	
\\	[形容詞]	たるんだ	
\\	[名詞]	使者、特使	
\\	複数形チェック
\\	[名詞]	⻨打ち用からざお	
\\	複数形チェック
\\	[動詞]	侵食する	
\\	[名詞]	羊毛	
\\	メリーさんは、雪のように真っ白な毛をした子羊を飼っていました。	
\\	複数形チェック
\\	[動詞]	引き起こす、発生させる	
\\	白髪が知恵を生み出すわけではない。	
\\	[名詞]	しりごみ	
\\	複数形チェック
\\	[動詞]	命じる,禁止する	
\\	[形容詞]	軽薄な	
\\	[動詞]	祭る	
\\	[名詞]	引っ越し	
\\	複数形チェック
\\	[動詞]	明確に述べる	
\\	[名詞]	植物相	
\\	複数形チェック
\\	[名詞]	近郊	
\\	複数形チェック
\\	[動詞]	変動する、上下する	
\\	メアリーは過剰な自信と劣等感との間で揺らいでいる。	
\\	[名詞]	きわめて短命なもの	
\\	複数形チェック
\\	[名詞]	綿毛	
\\	複数形チェック
\\	[名詞]	てんかん	
\\	複数形チェック
\\	[形容詞]	ふわふわした	
\\	羊の毛はもこもこしている。	
\\	[名詞]	典型、権化	
\\	複数形チェック
\\	[名詞]	フッ化物	
\\	複数形チェック
\\	[動詞]	典型的に示す	
\\	[名詞]	葉	
\\	複数形チェック
\\	[名詞]	馬術家、曲馬師	
\\	馬術競技は男女混合で行われる唯一のオリンピック競技です。	
\\	複数形チェック
\\	[形容詞]	むこうみずな	
\\	[名詞]	はめを外した行為	
\\	複数形チェック
\\	[形容詞]	絶対確実な	
\\	[名詞]	食道	
\\	複数形チェック
\\	[名詞]	飼料	
\\	複数形チェック
\\	[名詞]	秘教に通じた人、奥義	
\\	複数形チェック
\\	[動詞]	除外する	
\\	[動詞]	立ち退かせる	
\\	[名詞]	罰として失うもの、罰金	
\\	複数形チェック
\\	[名詞]	追い立て	
\\	立ち退きの予告を受けた。	
\\	複数形チェック
\\	[名詞]	偽造	
\\	複数形チェック
\\	[名詞]	雌羊	
\\	複数形チェック
\\	[名詞]	得意なもの	
\\	複数形チェック
\\	[名詞]	憤激	
\\	複数形チェック
\\	[名詞]	堅忍、不屈の精神	
\\	複数形チェック
\\	[動詞]	発掘する	
\\	中国でも戦国時代の墳墓からガラスが出土している。	
\\	[名詞]	鋳造所	
\\	複数形チェック
\\	[名詞]	国庫	
\\	複数形チェック
\\	[名詞]	玄関の広間	
\\	複数形チェック
\\	[名詞]	手本	
\\	複数形チェック
\\	[名詞]	托鉢修道士	
\\	複数形チェック
\\	[名詞]	激励	
\\	「ウィキペディアを引用するな」という主張が昨日だか一昨日だかの朝日新聞に載っていた。	
\\	複数形チェック
\\	[名詞]	フロックコート	
\\	複数形チェック
\\	[形容詞]	法外な	
\\	[名詞]	泡、内容のないもの	
\\	複数形チェック
\\	[動詞]	払い清める	
\\	[名詞]	倹約	
\\	複数形チェック
\\	[形容詞]	好都合で	
\\	[名詞]	でたらめ	
\\	やだストッキング伝線しちゃってる。	
\\	複数形チェック
\\	[動詞]	促進する	
\\	緊急配送には、10ドル追加料金がかかります。	
\\	[名詞]	職員	
\\	複数形チェック
\\	[名詞]	説明者、解釈者	
\\	複数形チェック
\\	[名詞]	休暇	
\\	複数形チェック
\\	[動詞]	明細に説く、説明する	
\\	[形容詞]	コソコソする	
\\	[形容詞]	即興の	
\\	[名詞]	胴体	
\\	複数形チェック
\\	[動詞]	引き渡す	
\\	[形容詞]	小うるさい	
\\	私は几帳面なんだ。	
\\	[動詞]	解放する	
\\	[名詞]	むだ	
\\	複数形チェック
\\	[名詞]	外向的な人	
\\	複数形チェック
\\	[名詞]	ガレー船	
\\	複数形チェック
\\	[動詞]	しみ出す、発散する	
\\	[形容詞]	けばけばしい	
\\	[名詞]	製作、偽造	
\\	同性愛者の司祭が主教に昇格させられたと聞いたが、結局捏造したものであった。	
\\	複数形チェック
\\	[名詞]	付け合わせ	
\\	複数形チェック
\\	[形容詞]	ふざけた	
\\	[形容詞]	げっそりとやせた	
\\	[名詞]	異邦人	
\\	複数形チェック
\\	[形容詞]	無感動の	
\\	[名詞]	属	
\\	複数形チェック
\\	[動詞]	促す、強いて...させる	
\\	[動詞]	発芽する	
\\	[動詞]	まねをする	
\\	[名詞]	えら	
\\	複数形チェック
\\	[形容詞]	無礼な	
\\	[名詞]	金箔、金色	
\\	複数形チェック
\\	[形容詞]	影響されない、動じない	
\\	雨露しのげて、寒暑防げる家というものは、人間にとって大変ありがたいものです。	
\\	[動詞]	締める	
\\	[形容詞]	衝動的な	
\\	[形容詞]	氷のように	
\\	[動詞]	侵害する、衝突する	
\\	[名詞]	きらめき	
\\	複数形チェック
\\	[形容詞]	厚かましい	
\\	[名詞]	ブドウ糖、グルコース	
\\	複数形チェック
\\	[形容詞]	不注意の	
\\	[名詞]	大食家、凝り屋	
\\	彼は大食らいです。	
\\	複数形チェック
\\	[形容詞]	白熱光を発する	
\\	[名詞]	伝説上の⻤	
\\	複数形チェック
\\	[名詞]	開始、発端	
\\	複数形チェック
\\	[名詞]	勾配、斜面	
\\	複数形チェック
\\	[名詞]	焼却炉	
\\	複数形チェック
\\	[名詞]	石墨	
\\	複数形チェック
\\	[名詞]	不一致	
\\	複数形チェック
\\	[名詞]	火床	
\\	複数形チェック
\\	[形容詞]	不調和の、釣り合わない	
\\	[形容詞]	心からの、無料の	
\\	[形容詞]	容易に信じない	
\\	[名詞]	チップ	
\\	複数形チェック
\\	[名詞]	侵入、侵略	
\\	複数形チェック
\\	[名詞]	鼠径(そけい)部	
\\	複数形チェック
\\	[名詞]	保護、免責、賠償金	
\\	複数形チェック
\\	[形容詞]	汚れた	
\\	[形容詞]	怠惰な	
\\	[形容詞]	だまされやすい	
\\	トムは人を信用しすぎる。	
\\	[名詞]	誘導	
\\	ガス炊飯器が壊れたので、
\\	炊飯器に変えました。	
\\	複数形チェック
\\	[名詞]	小峡谷	
\\	複数形チェック
\\	[名詞]	不適格者	
\\	複数形チェック
\\	[名詞]	ゴボゴボ流れる音	
\\	複数形チェック
\\	[形容詞]	不活性な	
\\	[名詞]	嗜好、心からの楽しみ	
\\	複数形チェック
\\	[名詞]	絶対確実な人	
\\	複数形チェック
\\	[形容詞]	ありふれた	
\\	[名詞]	夢中	
\\	複数形チェック
\\	[名詞]	野生のタカ	
\\	複数形チェック
\\	[形容詞]	弱い	
\\	[名詞]	飛行機の格納庫	
\\	複数形チェック
\\	[名詞]	弱いこと	
\\	複数形チェック
\\	[名詞]	偶然	
\\	複数形チェック
\\	[動詞]	侵害する	
\\	ホームページをつくるのもいいけど、著作権の侵害になるようなことはしてはいません。	
\\	[動詞]	聴く	
\\	[名詞]	不正	
\\	複数形チェック
\\	[名詞]	まぐわ	
\\	複数形チェック
\\	[形容詞]	はめ込まれた、象眼の	
\\	[名詞]	手斧	
\\	一時、私たちは敵だったが、和解したからには、もう親しい間柄である。	
\\	複数形チェック
\\	[名詞]	入り江	
\\	複数形チェック
\\	[名詞]	噂	
\\	伝聞であることを示す表現はいろいろある。	
\\	複数形チェック
\\	[名詞]	当てこすり、注釈句	
\\	複数形チェック
\\	[名詞]	ハリネズミ	
\\	複数形チェック
\\	[形容詞]	詮索好きな	
\\	[名詞]	快楽主義	
\\	複数形チェック
\\	[動詞]	記す、彫る、記名する	
\\	[名詞]	へり	
\\	複数形チェック
\\	[形容詞]	不可解な	
\\	[名詞]	大出血	
\\	私は14歳のときに、父をくも膜下出血で亡くした。	
\\	複数形チェック
\\	[名詞]	殺虫剤	
\\	複数形チェック
\\	[名詞]	異教徒	
\\	複数形チェック
\\	[形容詞]	味気ない	
\\	[名詞]	世捨人	
\\	世捨て人となり仙人のように山中をさまよい自給自足で誰とも接触せず野垂れ死にをしない方法とは絶対ある。	
\\	複数形チェック
\\	[名詞]	横柄な人	
\\	複数形チェック
\\	[名詞]	しゃっくり	
\\	よくしゃっくりが出る。	
\\	複数形チェック
\\	[名詞]	不眠症	
\\	睡眠の問題は不眠症と言います。	
\\	複数形チェック
\\	[形容詞]	しわがれ声の	
\\	しゃべりすぎて声がかれてしまった。	
\\	[動詞]	けしかける、開始する	
\\	[名詞]	引き上げ	
\\	複数形チェック
\\	[形容詞]	島国根性の	
\\	[形容詞]	同種の、同質の	
\\	[名詞]	断熱材	
\\	博文は天井にグラスファイバーの断熱材を入れた。	
\\	複数形チェック
\\	[形容詞]	単調な	
\\	[形容詞]	乗り越えられない	
\\	[名詞]	殻、トウモロコシの皮	
\\	複数形チェック
\\	[形容詞]	触れることのできない	
\\	[名詞]	催眠、催眠状態	
\\	複数形チェック
\\	[形容詞]	学際的な	
\\	[形容詞]	下劣な、下品な	
\\	[名詞]	連結	
\\	複数形チェック
\\	[形容詞]	著名な	
\\	[名詞]	間奏曲	
\\	複数形チェック
\\	[動詞]	浸透させる	
\\	[形容詞]	果てしない、⻑たらしい	
\\	[形容詞]	完璧な、無傷の	
\\	部屋はきちんと整っている。	
\\	[名詞]	(酒に)酔うこと	
\\	酒豪とは、アルコール飲料を大量に飲むことが出来、さらにこれで酔態をさらさない者の俗称である。	
\\	複数形チェック
\\	[形容詞]	関連のない	
\\	[名詞]	自己反省	
\\	複数形チェック
\\	[形容詞]	太古からの	
\\	[名詞]	虹彩	
\\	複数形チェック
\\	[動詞]	免疫性を与える	
\\	[形容詞]	うんざりする	
\\	[名詞]	買い戻しのできない	
\\	『トムとメアリー』は本当に救いようのない駄作だった。	
\\	複数形チェック
\\	[名詞]	オオヤマネコ	
\\	他の捕食動物(オオカミ、ヤマネコ)の獲物を力ずくで奪うことができる。	
\\	複数形チェック
\\	[形容詞]	取り消し不能の	
\\	[名詞]	サバ	
\\	鯖は煮ても焼いても美味しい。	
\\	複数形チェック
\\	[形容詞]	巡回する	
\\	[名詞]	ウジ虫	
\\	複数形チェック
\\	[名詞]	旅行計画	
\\	複数形チェック
\\	[名詞]	有力者	
\\	複数形チェック
\\	[名詞]	投げ槍、槍投げ	
\\	複数形チェック
\\	[名詞]	病気	
\\	複数形チェック
\\	[名詞]	あざけり	
\\	複数形チェック
\\	[形容詞]	悪意のある	
\\	[名詞]	押し合い	
\\	複数形チェック
\\	[名詞]	たてがみ	
\\	複数形チェック
\\	[形容詞]	陽気な	
\\	[名詞]	巧みに操作する人	
\\	複数形チェック
\\	[名詞]	記念祭	
\\	複数形チェック
\\	[名詞]	結婚、夫婦関係	
\\	複数形チェック
\\	[形容詞]	賢明な	
\\	[名詞]	既婚婦人	
\\	複数形チェック
\\	[名詞]	突起	
\\	複数形チェック
\\	[名詞]	細胞膜	
\\	子宮内膜炎は、細菌が子宮内に入り、内膜に炎症を起す病気である。	
\\	複数形チェック
\\	[名詞]	並列、並置	
\\	複数形チェック
\\	[名詞]	思い出の品	
\\	複数形チェック
\\	[名詞]	キール	
\\	複数形チェック
\\	[名詞]	召し使い	
\\	複数形チェック
\\	[名詞]	灯油	
\\	雪の多いところでは、灯油は必需品です。	
\\	複数形チェック
\\	[名詞]	傭兵	
\\	複数形チェック
\\	[名詞]	主音、要旨、基調	
\\	複数形チェック
\\	[名詞]	陽気なお祭り騒ぎ	
\\	複数形チェック
\\	[名詞]	口きき料	
\\	複数形チェック
\\	[名詞]	歓楽	
\\	複数形チェック
\\	[名詞]	窯	
\\	複数形チェック
\\	[形容詞]	種々雑多な、その他の	
\\	どんなに焦っても庶務的な仕事がどんどん溜まってきて追いつかない。	
\\	[名詞]	親族	
\\	複数形チェック
\\	[名詞]	少量、少々	
\\	複数形チェック
\\	[形容詞]	運動の、動的な	
\\	[名詞]	堀	
\\	複数形チェック
\\	[名詞]	凶兆	
\\	はインターネットにとって死神なんだって。	
\\	複数形チェック
\\	[動詞]	調節する	
\\	[名詞]	潟	
\\	複数形チェック
\\	[名詞]	造形	
\\	複数形チェック
\\	[名詞]	ねぐら	
\\	クズリは、岩の割れ目や、引き抜かれた根の下、および他の辺鄙な場所に巣を作る。	
\\	複数形チェック
\\	[動詞]	淫らなことをする	
\\	[名詞]	平信徒	
\\	複数形チェック
\\	[名詞]	修道士	
\\	複数形チェック
\\	[形容詞]	活気がない、気だるい	
\\	[動詞]	議題にのせる	
\\	[名詞]	側面	
\\	複数形チェック
\\	[形容詞]	不機嫌な、気難しい	
\\	[名詞]	旋盤	
\\	複数形チェック
\\	[名詞]	遺体安置室	
\\	複数形チェック
\\	[名詞]	格子、格子戶	
\\	複数形チェック
\\	[名詞]	ごみ、汚い物	
\\	複数形チェック
\\	[形容詞]	称賛に値する	
\\	[動詞]	音を消す	
\\	[名詞]	溶岩	
\\	複数形チェック
\\	[名詞]	おたふくかぜ	
\\	おたふくかぜは伝染病である。	
\\	複数形チェック
\\	[形容詞]	鉛色の	
\\	腰の痛みから足が鉛のように重く引きずるように歩いていました。	
\\	[名詞]	自治体、市当局	
\\	複数形チェック
\\	[名詞]	元帳	
\\	複数形チェック
\\	[動詞]	手足を切断する	
\\	[名詞]	ヒル	
\\	複数形チェック
\\	[名詞]	反乱	
\\	複数形チェック
\\	[形容詞]	読みやすい	
\\	[名詞]	口輪	
\\	なぜ君の犬は口輪を付けていないの?	
\\	複数形チェック
\\	[形容詞]	寛大な	
\\	[名詞]	スイセン	
\\	ナルキッソスは水面に映った自分の姿に恋をした。	
\\	複数形チェック
\\	[名詞]	病変、損傷、損害	
\\	複数形チェック
\\	[名詞]	無価値	
\\	我らの頭上に荒れ狂う砂と剣の煌きは、まるで夜空に降り注ぐ流星の如し。	
\\	複数形チェック
\\	[形容詞]	無気力な	
\\	力が出ません。	
\\	[名詞]	身廊	
\\	複数形チェック
\\	[名詞]	靭帯、きずな	
\\	複数形チェック
\\	[動詞]	否定する	
\\	ヘーゲルの弁証法の秘密は究極的に次のポイントにだけ存在する;神学を経て哲学を否定するために哲学を経て、神学を否定します。	
\\	[動詞]	清算する、解散する	
\\	[名詞]	イラクサ	
\\	複数形チェック
\\	[名詞]	舌足らずの発音	
\\	複数形チェック
\\	[名詞]	ノイローゼ	
\\	生化学的には、熱烈な恋愛と強迫神経症の疾患とは区別できないようだ。	
\\	複数形チェック
\\	[形容詞]	無関心な、ものうげな	
\\	だるいし頭がずきずきする。	
\\	[名詞]	無	
\\	この宇宙には知的生命体が存在するのはまず間違いないと思うが、その生命体が地球に来る可能性は殆ど無し。	
\\	複数形チェック
\\	[形容詞]	しなやかな	
\\	[名詞]	硝酸塩	
\\	複数形チェック
\\	[名詞]	訴訟当事者	
\\	複数形チェック
\\	[形容詞]	夜行性の	
\\	[名詞]	(葉などの)裂片	
\\	8歳未満の子供は前頭葉が発達しきっておらず、そのため現実と空想の区別が付かなくなる場合がある。一部の子どもは、例えば洋服ダンスの中やベッドの下に化け物がいると信じることがあるかもしれない。彼らの一部はまた、夢と現実の区別ができないこともある。	
\\	複数形チェック
\\	[名詞]	節、交点、節点	
\\	複数形チェック
\\	[名詞]	軌跡	
\\	複数形チェック
\\	[名詞]	遊牧⺠	
\\	複数形チェック
\\	[動詞]	ぶらつく、ひまどる	
\\	[形容詞]	無関心な	
\\	[名詞]	経度	
\\	赤道における経度1度当たりの長さ。	
\\	複数形チェック
\\	[名詞]	特徴のないもの	
\\	複数形チェック
\\	[動詞]	油を差す、滑らかにする	
\\	[名詞]	隅	
\\	複数形チェック
\\	[形容詞]	明快な、分かりやすい	
\\	[名詞]	鼻の穴	
\\	面疔ができた。小鼻の奥に1個、痛い塊がある。	
\\	複数形チェック
\\	[名詞]	突き、突進	
\\	複数形チェック
\\	[形容詞]	有害な	
\\	[形容詞]	性欲の強い、好色な	
\\	[動詞]	無効にする、取り消す	
\\	[名詞]	死亡記事	
\\	複数形チェック
\\	[形容詞]	こしょうの味がする	
\\	[名詞]	斜線	
\\	複数形チェック
\\	[形容詞]	お座なりの	
\\	[名詞]	産科医	
\\	複数形チェック
\\	[名詞]	周囲、末梢、周辺	
\\	複数形チェック
\\	[形容詞]	押しつけがましい	
\\	[名詞]	潜望鏡	
\\	サブマリンのペリスコープが水中からにょっきり突き出ていた。	
\\	複数形チェック
\\	[名詞]	頌歌	
\\	複数形チェック
\\	[動詞]	染み込む	
\\	[名詞]	副産物	
\\	複数形チェック
\\	[名詞]	垂線	
\\	複数形チェック
\\	[名詞]	軟膏	
\\	複数形チェック
\\	[名詞]	柿	
\\	複数形チェック
\\	[形容詞]	全能の	
\\	万能の神は、自分が持ち上げることのできない重い石を創ることができるのか?	
\\	[形容詞]	生意気な	
\\	[形容詞]	全知の	
\\	[動詞]	熟読する	
\\	[形容詞]	雑食性の	
\\	クズリは雑食だ。	
\\	[形容詞]	怒りっぽい	
\\	[名詞]	同一性	
\\	複数形チェック
\\	[名詞]	キジ	
\\	あれはキジです。	
\\	複数形チェック
\\	[形容詞]	煩わしい、面倒な	
\\	[名詞]	リン	
\\	複数形チェック
\\	[名詞]	見物人	
\\	複数形チェック
\\	[名詞]	生理学	
\\	食べてすぐ寝るのはいけないとよく言われるが、食べた後眠くなる生理と合わないじゃないかと思う。	
\\	複数形チェック
\\	[名詞]	義務、負担、重荷	
\\	複数形チェック
\\	[名詞]	少し	
\\	複数形チェック
\\	[形容詞]	自説を固執する	
\\	[名詞]	色素	
\\	トマトに含まれる色素リコピンには、美肌効果やダイエット効果があると言われている。	
\\	複数形チェック
\\	[形容詞]	最良の	
\\	[名詞]	槍(やり)、ほこ	
\\	複数形チェック
\\	[名詞]	最適条件	
\\	複数形チェック
\\	[動詞]	盗む	
\\	[名詞]	演説者、雄弁家	
\\	複数形チェック
\\	[名詞]	略奪品	
\\	複数形チェック
\\	[名詞]	ラン、ランの花	
\\	複数形チェック
\\	[名詞]	立腹	
\\	複数形チェック
\\	[動詞]	変動する	
\\	[名詞]	悲しげな(声)	
\\	複数形チェック
\\	[形容詞]	表向きの	
\\	[名詞]	国⺠投票	
\\	複数形チェック
\\	[形容詞]	これみよがしの	
\\	[形容詞]	全員出席の	
\\	[名詞]	追放	
\\	紀元前508年にアテネの指導者となったクレイステネスは、僭主の出現を防止するために陶片追放の制度を始めた。	
\\	複数形チェック
\\	[名詞]	過多、過剰	
\\	複数形チェック
\\	[名詞]	カワウソ	
\\	複数形チェック
\\	[名詞]	とぼとぼとした歩み	
\\	複数形チェック
\\	[名詞]	支出、経費	
\\	複数形チェック
\\	[名詞]	羽	
\\	複数形チェック
\\	[動詞]	出し抜く	
\\	[名詞]	おもり	
\\	複数形チェック
\\	[名詞]	卵巣	
\\	複数形チェック
\\	[動詞]	分裂させる、両極に分ける	
\\	[形容詞]	横柄な	
\\	[名詞]	論争、反論	
\\	複数形チェック
\\	[動詞]	過大評価する	
\\	[名詞]	数か国語に通じた人	
\\	複数形チェック
\\	[名詞]	ニュアンス	
\\	複数形チェック
\\	[名詞]	華やかさ	
\\	複数形チェック
\\	[名詞]	酸化物	
\\	亜酸化窒素は麻酔としてもはや使用されていない。	
\\	複数形チェック
\\	[形容詞]	大げさな	
\\	たいしたことではないのに必要以上に物々しい言い方です。	
\\	[動詞]	静める	
\\	[形容詞]	重々しい	
\\	[名詞]	南京錠	
\\	複数形チェック
\\	[名詞]	ポピー	
\\	複数形チェック
\\	[名詞]	棺、とばり	
\\	複数形チェック
\\	[形容詞]	透過性の	
\\	[名詞]	万能薬	
\\	複数形チェック
\\	[形容詞]	死後の	
\\	[名詞]	平なべの柄	
\\	複数形チェック
\\	[名詞]	仮説	
\\	複数形チェック
\\	[名詞]	聖職者	
\\	複数形チェック
\\	[名詞]	カリウム	
\\	複数形チェック
\\	[名詞]	銀行通帳	
\\	複数形チェック
\\	[名詞]	急襲	
\\	複数形チェック
\\	[形容詞]	病理学の	
\\	[名詞]	不機嫌	
\\	複数形チェック
\\	[名詞]	パタパタという音	
\\	複数形チェック
\\	[名詞]	先行	
\\	複数形チェック
\\	[名詞]	響き	
\\	複数形チェック
\\	[名詞]	教訓	
\\	複数形チェック
\\	[名詞]	教育学	
\\	複数形チェック
\\	[名詞]	絶壁	
\\	複数形チェック
\\	[形容詞]	学者ぶった	
\\	[名詞]	落下、降水	
\\	アメダスによればこの3日間で村の降水量は1200ミリを超えた。	
\\	複数形チェック
\\	[名詞]	行商人	
\\	幼少の時、父が死んで、その弟が、時代の衰勢と、自分の怠惰とから、すっかり、身代をつぶしてしまったらしく、後に、筆墨行商人になって、私の家へ、よく来たが、くると、母に叱られて、よわっていた。	
\\	複数形チェック
\\	[形容詞]	険しい	
\\	[形容詞]	気難しい	
\\	[名詞]	傾向	
\\	複数形チェック
\\	[名詞]	丸薬	
\\	複数形チェック
\\	[名詞]	予告、徴候、前兆、予感	
\\	嫌な予感が的中した。	
\\	複数形チェック
\\	[動詞]	激しく降る	
\\	[名詞]	優勢	
\\	複数形チェック
\\	[名詞]	骨盤	
\\	複数形チェック
\\	[名詞]	前提条件	
\\	複数形チェック
\\	[名詞]	ざんげ	
\\	彼は罪滅ぼしに難行苦行をしている。	
\\	複数形チェック
\\	[名詞]	特権、大権	
\\	複数形チェック
\\	[形容詞]	考え込んでいる	
\\	[名詞]	予感	
\\	複数形チェック
\\	[形容詞]	堅苦しい、取り澄ました	
\\	[名詞]	防虫剤	
\\	複数形チェック
\\	[名詞]	基本原理	
\\	複数形チェック
\\	[形容詞]	満たされた、十分備えた	
\\	[形容詞]	内々関与して	
\\	[名詞]	一時的救済	
\\	複数形チェック
\\	[動詞]	ぐずぐずする	
\\	ぐずぐずしないで。	
\\	[動詞]	退ける、拒否する	
\\	[名詞]	浪費家	
\\	複数形チェック
\\	[名詞]	撃退、拒絶	
\\	複数形チェック
\\	[名詞]	大量	
\\	結婚式ではあまりにたくさんのごちそうが出されたので、新郎新婦はもっと多くの人を呼ぶべきだったかなと思い始めた。	
\\	複数形チェック
\\	[動詞]	無効にする	
\\	[名詞]	予測	
\\	複数形チェック
\\	[名詞]	残余	
\\	複数形チェック
\\	[名詞]	投射物	
\\	複数形チェック
\\	[形容詞]	華麗な	
\\	[形容詞]	乱雑な	
\\	見境なくセックスしまくる人はエイズなどの性病が怖くないのですか?	
\\	[名詞]	返還、被害弁償	
\\	複数形チェック
\\	[名詞]	岬	
\\	複数形チェック
\\	[動詞]	生き返らせる、復興する	
\\	[動詞]	繁殖する	
\\	[動詞]	撤回する	
\\	[名詞]	傾向	
\\	複数形チェック
\\	[名詞]	お祭り騒ぎ	
\\	複数形チェック
\\	[形容詞]	平凡な	
\\	[名詞]	空想	
\\	複数形チェック
\\	[動詞]	禁止する	
\\	[名詞]	木の皮	
\\	複数形チェック
\\	[名詞]	案内書	
\\	複数形チェック
\\	[名詞]	止まり木	
\\	複数形チェック
\\	[動詞]	うつぶせにする	
\\	[名詞]	演壇、指揮台	
\\	複数形チェック
\\	[動詞]	突き出す、突き出る	
\\	さらに、横から見ると、前歯は出っ歯の状態にあります。	
\\	[名詞]	回転部	
\\	複数形チェック
\\	[名詞]	条件	
\\	複数形チェック
\\	[名詞]	検挙、総括	
\\	複数形チェック
\\	[名詞]	賛美歌	
\\	複数形チェック
\\	[名詞]	舵	
\\	ジブの裏帆とラダーを使って、バックしながら船首を進みたい方向に向けました。	
\\	複数形チェック
\\	[名詞]	思春期	
\\	女子は10~11歳前後、男子は11~12歳前後で思春期を迎える。	
\\	複数形チェック
\\	[形容詞]	赤くする	
\\	[形容詞]	刺激臭、辛辣な	
\\	[形容詞]	基本の	
\\	[名詞]	のどを鳴らす音	
\\	複数形チェック
\\	[動詞]	波立たせる	
\\	[名詞]	ウズラ	
\\	複数形チェック
\\	[動詞]	反芻する、思いめぐらす	
\\	彼は自分の不運についてじっくり考えた。	
\\	[名詞]	検疫	
\\	複数形チェック
\\	[名詞]	寄せ集め、がらくた	
\\	複数形チェック
\\	[名詞]	震音	
\\	複数形チェック
\\	[名詞]	しり肉	
\\	複数形チェック
\\	[名詞]	埠頭	
\\	複数形チェック
\\	[名詞]	決まりきったやり方	
\\	高校時代からつきあっている彼氏はいるものの、つきあいが惰性になっていると感じ、不満をつのらせている。	
\\	複数形チェック
\\	[動詞]	渇きを癒す	
\\	[名詞]	安息日	
\\	複数形チェック
\\	[名詞]	定足数	
\\	複数形チェック
\\	[名詞]	有給休暇	
\\	複数形チェック
\\	[名詞]	狂犬病	
\\	複数形チェック
\\	[形容詞]	神聖を汚す	
\\	[名詞]	分岐、支脈、結果	
\\	複数形チェック
\\	[形容詞]	利口な、機敏な	
\\	[動詞]	ものを奪う	
\\	[名詞]	突出部	
\\	複数形チェック
\\	[形容詞]	没頭した、有頂天の	
\\	[名詞]	唾液	
\\	複数形チェック
\\	[名詞]	しゃがれ声、耳障りな音	
\\	複数形チェック
\\	[形容詞]	有益な、健康回復に良い	
\\	[名詞]	批准	
\\	複数形チェック
\\	[名詞]	通学用かばん	
\\	複数形チェック
\\	[動詞]	完全に破壊する、消す	
\\	[名詞]	飽和	
\\	複数形チェック
\\	[名詞]	反証	
\\	複数形チェック
\\	[形容詞]	図式の、図式的な	
\\	[名詞]	世捨人	
\\	複数形チェック
\\	[名詞]	分離	
\\	複数形チェック
\\	[名詞]	はね返り	
\\	複数形チェック
\\	[名詞]	しかめっ面	
\\	複数形チェック
\\	[名詞]	補償	
\\	複数形チェック
\\	[名詞]	ひっかくこと、走り書き	
\\	複数形チェック
\\	[名詞]	非難し返すこと	
\\	複数形チェック
\\	[形容詞]	みすぼらしい	
\\	俺は青二才の頃は蛮カラぶって真夏に二週間ほど、お風呂に入らなかったこともあった。	
\\	[動詞]	回復する	
\\	[名詞]	良心のとがめ、ためらい	
\\	複数形チェック
\\	[名詞]	救済、矯正、補償	
\\	複数形チェック
\\	[名詞]	乱闘、小競り合い	
\\	複数形チェック
\\	[動詞]	払い戻す、返済する	
\\	[名詞]	人間のくず	
\\	煮え立たせる寸前で弱火にして、もう一度灰汁を取り除きます。	
\\	複数形チェック
\\	[動詞]	繰り返す	
\\	[名詞]	離脱	
\\	複数形チェック
\\	[名詞]	口答え	
\\	複数形チェック
\\	[動詞]	引きこもる	
\\	[動詞]	若返る	
\\	[形容詞]	人里離れた	
\\	クズリは、岩の割れ目や、引き抜かれた根の下、および他の辺鄙な場所に巣を作る。	
\\	[動詞]	思い出を語る	
\\	[動詞]	こっそり隠す、分泌する	
\\	女性が自慰をすれば女性ホルモンが分泌され、女性らしさが出てきます。	
\\	[名詞]	鎮静、免除	
\\	複数形チェック
\\	[動詞]	落ち着かせる	
\\	[名詞]	償い、賠償	
\\	複数形チェック
\\	[形容詞]	すわりがちの人、坐業者	
\\	[名詞]	送還者	
\\	複数形チェック
\\	[名詞]	沈殿物・堆積物	
\\	超高圧水の噴射により、ほとんどの詰まりが排除される。	
\\	複数形チェック
\\	[動詞]	煮えくり返る	
\\	[動詞]	噴出する	
\\	[形容詞]	語義に関する	
\\	[名詞]	雄ジカ	
\\	複数形チェック
\\	[形容詞]	老いぼれた	
\\	[動詞]	停滞する	
\\	流れる水は腐らず。	
\\	[名詞]	歩哨、張り番	
\\	複数形チェック
\\	[名詞]	種馬、種付け用の雄	
\\	複数形チェック
\\	[動詞]	隠退させる	
\\	[名詞]	代わり、助け、利益	
\\	複数形チェック
\\	[名詞]	血清	
\\	複数形チェック
\\	[形容詞]	ひそかな	
\\	[名詞]	軍人	
\\	複数形チェック
\\	[名詞]	悪臭	
\\	部屋がタバコ臭くて発狂しそう。	
\\	複数形チェック
\\	[形容詞]	奴隷の	
\\	[名詞]	速記者	
\\	複数形チェック
\\	[名詞]	下水	
\\	複数形チェック
\\	[動詞]	殺菌する	
\\	[名詞]	鞘、さや状のもの	
\\	やがて彼は白刃を鞘に収めた。	
\\	複数形チェック
\\	[形容詞]	がっしりした	
\\	[形容詞]	ずるい	
\\	[動詞]	しまい込む	
\\	脱いだ服はちゃんと片付けてよね。	
\\	[名詞]	脛、すねの骨	
\\	。新大阪の新幹線出口に行くわ。	
\\	複数形チェック
\\	[名詞]	層	
\\	複数形チェック
\\	[名詞]	浅瀬	
\\	複数形チェック
\\	[名詞]	非難、酷評、弾劾	
\\	複数形チェック
\\	[名詞]	ふるい、茶こし	
\\	複数形チェック
\\	[名詞]	無精ひげ	
\\	複数形チェック
\\	[名詞]	筋肉	
\\	複数形チェック
\\	[形容詞]	風通しの悪い	
\\	この部屋は息がつまりそうだ。	
\\	[名詞]	船⻑、主将	
\\	複数形チェック
\\	[形容詞]	驚くべき、並外れた	
\\	それは気が遠くなるような計画だ。	
\\	[動詞]	緩める	
\\	[名詞]	無感覚、麻痺	
\\	患者は深い昏睡状態であります。	
\\	複数形チェック
\\	[名詞]	中傷	
\\	※基本的な禁止事項(誹謗・中傷の禁止等)は「はじめにお読み下さい」に記載してあります。必ずお読みください。	
\\	複数形チェック
\\	[動詞]	征服する	
\\	[名詞]	どたばた喜劇	
\\	複数形チェック
\\	[名詞]	生活、生計	
\\	複数形チェック
\\	[名詞]	そり	
\\	複数形チェック
\\	[動詞]	立証する、実証する	
\\	[名詞]	すべること	
\\	複数形チェック
\\	[形容詞]	地下の	
\\	[名詞]	泥地	
\\	複数形チェック
\\	[名詞]	転覆、打倒、壊滅	
\\	複数形チェック
\\	[形容詞]	だらしない	
\\	[形容詞]	簡潔な	
\\	[名詞]	水門	
\\	複数形チェック
\\	[名詞]	多肉植物	
\\	複数形チェック
\\	[名詞]	くすぶる火	
\\	複数形チェック
\\	[動詞]	窒息する	
\\	彼は煙に巻かれて窒息した。	
\\	[名詞]	汚れ	
\\	複数形チェック
\\	[名詞]	参政権、選挙	
\\	複数形チェック
\\	[名詞]	チョキンと切ること	
\\	複数形チェック
\\	[名詞]	不機嫌	
\\	複数形チェック
\\	[名詞]	居眠り	
\\	複数形チェック
\\	[名詞]	トルコ皇帝(スルタン)	
\\	かつてサルタンが今日のトルコを統治していた。	
\\	複数形チェック
\\	[名詞]	芝、芝生	
\\	複数形チェック
\\	[形容詞]	蒸し暑い	
\\	どこかへ逃避行したい。蒸し暑い夏はそんな気分になりがち。	
\\	[動詞]	びっしょり濡らす	
\\	[形容詞]	種々さまざまの	
\\	[形容詞]	ずぶぬれの	
\\	[動詞]	重ね合わせる	
\\	[名詞]	滞在	
\\	複数形チェック
\\	[動詞]	取って代わる	
\\	[名詞]	至、至点、最高点	
\\	複数形チェック
\\	[形容詞]	不機嫌な	
\\	[名詞]	溶媒	
\\	その液から溶剤を揮発させる。残るのは、香り成分と植物ワックスの塊。	
\\	複数形チェック
\\	[名詞]	影響されやすいこと	
\\	複数形チェック
\\	[名詞]	魔法使い	
\\	複数形チェック
\\	[動詞]	だまして金を取る	
\\	[名詞]	出撃	
\\	複数形チェック
\\	[名詞]	豚	
\\	豚の前に梨を放り出さないで。	
\\	複数形チェック
\\	[動詞]	スパーリングをする	
\\	この儘では誰もが、永遠の滅びを免れないことになります。	
\\	[名詞]	教会会議	
\\	複数形チェック
\\	[形容詞]	見掛け倒しの	
\\	[名詞]	概要	
\\	複数形チェック
\\	[名詞]	とがり屋根、尖塔	
\\	遠くに教会の尖塔が見えた。	
\\	複数形チェック
\\	[名詞]	構文	
\\	複数形チェック
\\	[名詞]	小枝	
\\	複数形チェック
\\	[名詞]	絵画、活人画	
\\	複数形チェック
\\	[形容詞]	陽気な、活発な	
\\	[名詞]	おたまじゃくし	
\\	複数形チェック
\\	[名詞]	妖精	
\\	子供のすきな小さい神さまがありました。いつもは森の中で、歌をうたったり笛を吹いたりして、小鳥やけものと遊んでいましたが、ときどき人のすんでいる村へ出てきて、すきな子供たちと遊ぶのでした。	
\\	複数形チェック
\\	[名詞]	お守り	
\\	複数形チェック
\\	[動詞]	身なりを整える	
\\	[名詞]	爪、かぎづめ状のもの	
\\	複数形チェック
\\	[形容詞]	偽造の、偽りの	
\\	[動詞]	じらす、気を持たせる	
\\	[形容詞]	不潔な、惨めな	
\\	難民がごみごみした裏町に住んでいました。	
\\	[名詞]	遅参	
\\	複数形チェック
\\	[名詞]	不潔さ、みすぼらしさ	
\\	複数形チェック
\\	[動詞]	滞在する	
\\	[名詞]	しゃがむこと	
\\	体を捻ってから下までしゃがみます。	
\\	複数形チェック
\\	[形容詞]	大あらしの	
\\	[名詞]	無断居住者	
\\	複数形チェック
\\	[名詞]	借家、借地、小作地	
\\	複数形チェック
\\	[形容詞]	気難しい	
\\	[名詞]	借家、アパート	
\\	複数形チェック
\\	[名詞]	イカ	
\\	イカには10本の足がある。	
\\	複数形チェック
\\	[名詞]	触手、触糸、触毛	
\\	複数形チェック
\\	[名詞]	専門用語	
\\	尿瓶でおしっこするのを手伝ってあげることを専門用語でなんと言いますか?	
\\	複数形チェック
\\	[名詞]	勇気	
\\	複数形チェック
\\	[形容詞]	陸上の	
\\	[名詞]	風見、翼、視準板	
\\	複数形チェック
\\	[名詞]	破傷風	
\\	あなたは病犬に噛まれたら、破傷風の予防注射が必要となってしまいます。	
\\	複数形チェック
\\	[動詞]	破る	
\\	[名詞]	つなぎ繩、範囲、限界	
\\	複数形チェック
\\	[名詞]	利益、優勢、有利な立場	
\\	複数形チェック
\\	[名詞]	わらぶき屋根	
\\	複数形チェック
\\	[形容詞]	変化に富んだ	
\\	[副詞]	その中に	
\\	そこに問題がある。	
\\	[名詞]	大おけ	
\\	複数形チェック
\\	[名詞]	類語辞典	
\\	複数形チェック
\\	[名詞]	ベクトル	
\\	サポートベクターマシンとは、教師あり学習モデルであり、判別分析や回帰分析に用いられる。	
\\	複数形チェック
\\	[名詞]	指貫、はめ輪	
\\	複数形チェック
\\	[形容詞]	激烈な、熱烈な	
\\	[名詞]	サラブレッド	
\\	複数形チェック
\\	[名詞]	毒液	
\\	複数形チェック
\\	[名詞]	大通り	
\\	複数形チェック
\\	[形容詞]	有毒な	
\\	この蛇は毒蛇ではありません。	
\\	[動詞]	打ちのめす	
\\	[名詞]	換気	
\\	複数形チェック
\\	[形容詞]	すり切れた	
\\	彼はくたくたの背広を着ていた。	
\\	[形容詞]	既得の	
\\	[形容詞]	節約的な	
\\	浪費する人を守るために倹約する人に税金をかけるのは、賢明なことではないし、結局はためにならないのである。	
\\	[名詞]	痕跡、名残	
\\	複数形チェック
\\	[名詞]	ドサッという音	
\\	買い物袋をどすんとテーブルの上に置いた。	
\\	複数形チェック
\\	[名詞]	獣医	
\\	複数形チェック
\\	[名詞]	10分の1	
\\	複数形チェック
\\	[名詞]	付近	
\\	本日、空港付近の天候は晴れ、気温は摂氏20度となっております。	
\\	複数形チェック
\\	[名詞]	ヒキガエル、いやなやつ	
\\	複数形チェック
\\	[形容詞]	執念深い、報復の	
\\	[名詞]	物を挟む道具	
\\	複数形チェック
\\	[形容詞]	男らしい、力強い	
\\	[名詞]	地形学、地形	
\\	複数形チェック
\\	[名詞]	顔、顔つき、容貌	
\\	複数形チェック
\\	[動詞]	よろよろ歩く、よろめく	
\\	老人がたどたどしい足取りで歩いてきた。	
\\	[形容詞]	活発な	
\\	[動詞]	落ち着かせる、静める	
\\	[形容詞]	騒々しい、言い張った	
\\	[形容詞]	はかない	
\\	[名詞]	意志	
\\	複数形チェック
\\	[形容詞]	震える、おののく	
\\	[形容詞]	舌のよくまわる	
\\	[名詞]	やりなどの三叉道具	
\\	複数形チェック
\\	[形容詞]	官能的な	
\\	[形容詞]	使い古された	
\\	「双子のトラの子が離れ離れになって・・・」というストーリーでちょっとありふれたような感じでした。	
\\	[動詞]	保証する	
\\	[名詞]	熱帯地方	
\\	複数形チェック
\\	[名詞]	下品、野卑、俗悪	
\\	複数形チェック
\\	[名詞]	樋、溝	
\\	複数形チェック
\\	[名詞]	ハゲタカ	
\\	複数形チェック
\\	[名詞]	怠け者	
\\	複数形チェック
\\	[名詞]	浮動	
\\	複数形チェック
\\	[名詞]	有料高速道路	
\\	複数形チェック
\\	[名詞]	棒、指揮棒	
\\	複数形チェック
\\	[名詞]	さえずり	
\\	ツイッターやってる人いる?	
\\	複数形チェック
\\	[形容詞]	理不尽な、むちゃくちゃな	
\\	私よりもっとエッチな人もいて安心しました。	
\\	[名詞]	腸チフス	
\\	複数形チェック
\\	[名詞]	さえずり、震え声	
\\	複数形チェック
\\	[形容詞]	粗野な	
\\	・・・言いたいことは色々あるのだが、ここで口を挟むのは野暮というもの。	
\\	[名詞]	いぼ、こぶ	
\\	複数形チェック
\\	[動詞]	補強する、支持する	
\\	[名詞]	波⻑	
\\	紫の光は、赤の光よりも短い波長です。	
\\	複数形チェック
\\	[形容詞]	過度の、不適当な	
\\	[動詞]	乳離れさせる	
\\	[形容詞]	疑う余地のない	
\\	[名詞]	雑然とした状態	
\\	複数形チェック
\\	[動詞]	広げる	
\\	[動詞]	甘言で誘う	
\\	[形容詞]	見苦しい	
\\	[名詞]	ゼイゼイいう音	
\\	複数形チェック
\\	[形容詞]	手入れのされていない	
\\	ずぼらな彼女は、整理整頓が大の苦手だ。	
\\	[名詞]	研磨、刺激	
\\	複数形チェック
\\	[形容詞]	慎み深い	
\\	[名詞]	しくしく泣く声	
\\	複数形チェック
\\	[形容詞]	無傷の	
\\	[名詞]	渦	
\\	複数形チェック
\\	[形容詞]	無法な、不徳な	
\\	[名詞]	ほおひげ	
\\	複数形チェック
\\	[形容詞]	擁護できない	
\\	[名詞]	しっくい	
\\	複数形チェック
\\	[名詞]	室内装飾品	
\\	複数形チェック
\\	[名詞]	枝編み細工	
\\	複数形チェック
\\	[形容詞]	高所得者向けの、高級な	
\\	[名詞]	習慣	
\\	複数形チェック
\\	[形容詞]	洗練された、丁重な	
\\	[名詞]	ねじり	
\\	複数形チェック
\\	[名詞]	つぼ、骨つぼ	
\\	祖母の遺灰は寺院に置かれた骨壷に入っている。	
\\	複数形チェック
\\	[名詞]	身をくねらせること	
\\	複数形チェック
\\	[動詞]	不法に手に入れる	
\\	[動詞]	もだえ苦しむ	
\\	[動詞]	立ち退く	
\\	[名詞]	外国人恐怖症	
\\	複数形チェック
\\	[名詞]	放浪者	
\\	複数形チェック
\\	[名詞]	頂点、絶頂	
\\	複数形チェック
\\	[名詞]	谷、谷間	
\\	複数形チェック
\\	[名詞]	熱意	
\\	複数形チェック
\\	[形容詞]	勇気ある、勇敢な	
\\	[名詞]	十二宮	
\\	複数形チェック
\end{CJK}
\end{document}