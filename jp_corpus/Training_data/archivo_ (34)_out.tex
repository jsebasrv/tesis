\documentclass[8pt]{extreport} 
\usepackage{hyperref}
\usepackage{CJKutf8}
\begin{document}
\begin{CJK}{UTF8}{min}
\\	中止?私は何も聞いてないぞ。 
\\	はい、川本さんが来られません。	
\\	はい、川本さんが来られません。 
\\	来られない?何で。	
\\	来られない?何で。 
\\	そこまで、ちょっとわかりませんが。	
\\	そこまで、ちょっとわかりませんが。 
\\	わかった。川本さんに連絡して日程を変更して。	
\\	わかった。川本さんに連絡して日程を変更して。 
\\	はい、わかりました。	
\\	はい、わかりました。 
\\	ぼちぼち
\\	ばっちり
\\	ラブラブ
\\	イケメン
\\	モテモテ
\\	久しぶり!元気?	
\\	久しぶり!元気? 
\\	ぼちぼち。かおるは?	
\\	ぼちぼち。かおるは? 
\\	ばっちり。新しい彼氏ができた。もーラブラブだよ。優しくて。	
\\	ばっちり。新しい彼氏ができた。もーラブラブだよ。優しくて。 
\\	よかったね。前の彼はいつもイライラしてたからね。	
\\	よかったね。前の彼はいつもイライラしてたからね。 
\\	でもね、彼はイケメンだからモテモテで、ちょっと心配。	
\\	でもね、彼はイケメンだからモテモテで、ちょっと心配。 
\\	大丈夫だよ!	
\\	大丈夫だよ! 
\\	っていうか
\\	ひどい
\\	口にする
\\	マジで
\\	かわいそう
\\	どうだった?	
\\	どうだった? 
\\	ばっちりだったよ。	
\\	ばっちりだったよ。 
\\	っていうか。もっとうるさくなったけど
\\	何て言ったの?	
\\	っていうか。もっとうるさくなったけど
\\	何て言ったの? 
\\	あまり言いたくない。いや、言えない。ひどい事を言ったから。	
\\	あまり言いたくない。いや、言えない。ひどい事を言ったから。 
\\	ドアにノックの音
\\	違法
\\	捕まる
\\	命
\\	飲酒運転
\\	捕まえる
\\	久しぶりのドライブは楽しいね。	
\\	久しぶりのドライブは楽しいね。 
\\	(よしの電話が鳴る)	
\\	もしもし。おー。元気?	
\\	もしもし。おー。元気? 
\\	運転しながら、携帯で話しているの?違法だよ。つかまるよ。	
\\	運転しながら、携帯で話しているの?違法だよ。つかまるよ。 
\\	大丈夫だよ。つかまらないよ。	
\\	大丈夫だよ。つかまらないよ。 
\\	それよりも、命の問題でしょう。	
\\	それよりも、命の問題でしょう。 
\\	昇進
\\	熱心
\\	一生懸命
\\	働く
\\	決まる
\\	降格
\\	決まり
\\	今度の昇進についてですが、誰がいいですかね。	
\\	今度の昇進についてですが、誰がいいですかね。 
\\	山田さんはどうですか。	
\\	山田さんはどうですか。 
\\	そうですね。彼がいいです。	
\\	そうですね。彼がいいです。 
\\	セールス
\\	覗き
\\	文句
\\	覗き窓
\\	ビデオ付インターホン
\\	口説く
\\	口説き文句
\\	はい、ちょっと待ってください。誰かな?	
\\	はい、ちょっと待ってください。誰かな? 
\\	セールスの人だろう。	
\\	セールスの人だろう。 
\\	セールスの人?ちょっと覗き窓から見えるかな。	
\\	セールスの人?ちょっと覗き窓から見えるかな。 
\\	隣の人?
\\	出なくていいよ。きっと文句を言いに来たんだ。	
\\	隣の人?
\\	出なくていいよ。きっと文句を言いに来たんだ。 
\\	そうかな?何かくつを持ってるよ。あれ君のじゃないの?	
\\	そうかな?何かくつを持ってるよ。あれ君のじゃないの? 
\\	進路
\\	決定
\\	留学
\\	決定する
\\	来年
\\	卒業
\\	君は来年卒業だね。	
\\	君は来年卒業だね。 
\\	はい、そうです。	
\\	はい、そうです。 
\\	進路はどうする?	
\\	進路はどうする? 
\\	まだ決定ではないのですが、アメリカへ行こうと思っています。	
\\	まだ決定ではないのですが、アメリカへ行こうと思っています。 
\\	アメリカか。私もアメリカに留学したことがある。	
\\	アメリカか。私もアメリカに留学したことがある。 
\\	ニューヨークへ行こうと思っています。	
\\	ニューヨークへ行こうと思っています。 
\\	ニューヨークもいいね。頑張って。	
\\	ニューヨークもいいね。頑張って。 
\\	夫妻
\\	スポーツセンター
\\	サスペンス
\\	小説
\\	手料理
\\	一冊
\\	野村夫妻、週末に何をしますか。	
\\	野村夫妻、週末に何をしますか。 
\\	私は読むのが好きです。週末に本を一冊読みます。サスペンス小説が好きです。	
\\	私は読むのが好きです。週末に本を一冊読みます。サスペンス小説が好きです。 
\\	私は運動するのが好きです。土曜日と日曜日にスポーツセンターへ行きます。	
\\	私は運動するのが好きです。土曜日と日曜日にスポーツセンターへ行きます。 
\\	私は料理するのも好きです。日曜日の夜に料理を作ります。	
\\	私は料理するのも好きです。日曜日の夜に料理を作ります。 
\\	彼の手料理がとても好きです。	
\\	彼の手料理がとても好きです。 
\\	裏表
\\	謝る
\\	謝り
\\	逆に
\\	逆
\\	はい、どうぞ。お入りください。	
\\	はい、どうぞ。お入りください。 
\\	お邪魔します。	
\\	お邪魔します。 
\\	すみませんでした。友達の態度はとても悪かったです。申し訳ございません。	
\\	すみませんでした。友達の態度はとても悪かったです。申し訳ございません。 
\\	ええ、どういうことですか。	
\\	ええ、どういうことですか。 
\\	私の友達はひどかったでしょう。彼には裏表があります。	
\\	私の友達はひどかったでしょう。彼には裏表があります。 
\\	ええ、すみません。何を言っているのか、よくわかりません。	
\\	ええ、すみません。何を言っているのか、よくわかりません。 
\\	ええ
\\	ええ
\\	その人はあなたの友達でしょう。	
\\	その人はあなたの友達でしょう。 
\\	はい、そうです。	
\\	はい、そうです。 
\\	彼はとても優しかったですよ。逆に私が悪かったです。謝りに来ました。	
\\	彼はとても優しかったですよ。逆に私が悪かったです。謝りに来ました。 
\\	たけが
\\	たけが
\\	番号カード
\\	契約
\\	サインする
\\	サイン
\\	料金
\\	基本
\\	基本料金
\\	不満
\\	解約
\\	満足
\\	いらっしゃいませ。番号カードをお取りください。	
\\	いらっしゃいませ。番号カードをお取りください。 
\\	63番の方。	
\\	63番の方。 
\\	解約をお願いします。	
\\	解約をお願いします。 
\\	解約ですか。ご不満の点がありましたか?	
\\	解約ですか。ご不満の点がありましたか? 
\\	電話はあまり使わないし。基本料金は高いし。	
\\	電話はあまり使わないし。基本料金は高いし。 
\\	はい、かしこまりました。少々お待ち下さい
\\	はい、かしこまりました。少々お待ち下さい
\\	どうもありがとうございました。	
\\	どうもありがとうございました。 
\\	お片付け
\\	片付ける
\\	懐かしい
\\	全部
\\	紫
\\	青
\\	水色
\\	緑
\\	黄緑
\\	黄色
\\	オレンジ
\\	桃色
\\	赤
\\	茶色
\\	黒
\\	灰色
\\	白
\\	色鉛筆
\\	お昼寝
\\	皆さん、おかたづけの時間ですよ。	
\\	皆さん、おかたづけの時間ですよ。 
\\	おかたづけ嫌い!	
\\	おかたづけ嫌い! 
\\	色鉛筆の箱はどこ?	
\\	色鉛筆の箱はどこ? 
\\	ここだよ。色鉛筆は?	
\\	ここだよ。色鉛筆は? 
\\	ここにあるよ。「白」「灰色」「黒」「茶色」「赤」「桃色」「オレンジ」があるよ。	
\\	ここにあるよ。「白」「灰色」「黒」「茶色」「赤」「桃色」「オレンジ」があるよ。 
\\	ここにもあるよ。「黄色」「黄緑」「緑」「水色」「青」「紫」がある。	
\\	ここにもあるよ。「黄色」「黄緑」「緑」「水色」「青」「紫」がある。 
\\	これで全部。	
\\	これで全部。 
\\	ファン
\\	変わっている
\\	先ほど
\\	現れる
\\	埋め合わせ
\\	バンド
\\	ライブ
\\	乱暴
\\	喜んで
\\	はい、忘れ物ですが。	
\\	はい、忘れ物ですが。 
\\	すみません。どうもありがとうございます。	
\\	すみません。どうもありがとうございます。 
\\	先ほどはすみませんでした。変わっているファンが多くて。知らない人が現れて、つまり、ごめんなさい。	
\\	先ほどはすみませんでした。変わっているファンが多くて。知らない人が現れて、つまり、ごめんなさい。 
\\	いいえ、こちらこそ。乱暴にドアを叩いて申し訳ございません。	
\\	いいえ、こちらこそ。乱暴にドアを叩いて申し訳ございません。 
\\	今夜埋め合わせをします。あたしが聴いていた音楽が好きって言ってましたよね。	
\\	今夜埋め合わせをします。あたしが聴いていた音楽が好きって言ってましたよね。 
\\	はい、そうです。	
\\	はい、そうです。 
\\	は!	
\\	は! 
\\	今夜あのバンドのライブに行こうと思っていますが、一緒に行きませんか。	
\\	今夜あのバンドのライブに行こうと思っていますが、一緒に行きませんか。 
\\	うそ!	
\\	うそ! 
\\	喜んで!	
\\	喜んで! 
\\	じゃ、今夜8時に渋谷で。では、お邪魔しました。	
\\	じゃ、今夜8時に渋谷で。では、お邪魔しました。 
\\	だます
\\	芸能人
\\	信じる
\\	無理
\\	貸す
\\	渋谷
\\	送る
\\	信じられない!よくもだましたな。	
\\	信じられない!よくもだましたな。 
\\	ごめん。なんで芸能人って言わなかったの!俺はびっくりしたよ。	
\\	ごめん。なんで芸能人って言わなかったの!俺はびっくりしたよ。 
\\	俺だって!俺もびっくりした。	
\\	俺だって!俺もびっくりした。 
\\	ところで、今夜車を貸してもらえる?	
\\	ところで、今夜車を貸してもらえる? 
\\	それは無理だ。貸してあげられないけど、渋谷まで送ってあげる。	
\\	それは無理だ。貸してあげられないけど、渋谷まで送ってあげる。 
\\	悪いね。ありがとう。	
\\	悪いね。ありがとう。 
\\	美味い
\\	太る
\\	痩せる
\\	生きる
\\	自由
\\	自由に
\\	結婚
\\	だめよ。知らない人と話してはいけません。	
\\	だめよ。知らない人と話してはいけません。 
\\	はい、はい。あっ、うまそう!食べたい。	
\\	はい、はい。あっ、うまそう!食べたい。 
\\	だめよ。太るわ。今食べてはいけません。	
\\	だめよ。太るわ。今食べてはいけません。 
\\	なんで!僕は子供じゃない!自由に生きたい。	
\\	なんで!僕は子供じゃない!自由に生きたい。 
\\	だめよ。自由に生きてはいけません。あなたは私と結婚してるんだから。	
\\	だめよ。自由に生きてはいけません。あなたは私と結婚してるんだから。 
\\	案内
\\	一零四
\\	赤坂
\\	ジャパニーズポッド101
\\	お問い合わせ
\\	番号
\\	はい、104「いちれいよん」の堺です。	
\\	はい、104「いちれいよん」の堺です。 
\\	赤坂のジャパニーズポッド101をお願いします。	
\\	赤坂のジャパニーズポッド101をお願いします。 
\\	はい、ご案内します。ありがとうございました。	
\\	はい、ご案内します。ありがとうございました。 
\\	お問い合わせの番号は03の2342の8923です。ありがとうございました。	
\\	お問い合わせの番号は03の2342の8923です。ありがとうございました。 
\\	電話番号
\\	女性
\\	駄目
\\	待つ
\\	絶対
\\	どれぐらい
\\	二週間
\\	預かる
\\	その電話番号どうしたの?	
\\	その電話番号どうしたの? 
\\	昨日会った女性に電話する。	
\\	昨日会った女性に電話する。 
\\	駄目だよ。早すぎるよ。待ったほうがいいよ。	
\\	駄目だよ。早すぎるよ。待ったほうがいいよ。 
\\	早すぎる?	
\\	早すぎる? 
\\	うん、絶対待った方がいいよ。	
\\	うん、絶対待った方がいいよ。 
\\	どれぐらい?	
\\	どれぐらい? 
\\	二週間でいいよ。あっ、友達だから、その番号を預かるよ。じゃあね。	
\\	二週間でいいよ。あっ、友達だから、その番号を預かるよ。じゃあね。 
\\	二人の会話
\\	聞こえる
\\	アドバイス
\\	自分の心
\\	さっき
\\	ほら
\\	受ける
\\	気を付ける
\\	すみません。二人の会話が聞こえましたが、あの人は友達ですか。	
\\	すみません。二人の会話が聞こえましたが、あの人は友達ですか。 
\\	はい、そうです。何でですか。	
\\	はい、そうです。何でですか。 
\\	彼のアドバイスはよくないと思います。	
\\	彼のアドバイスはよくないと思います。 
\\	よくないですか。	
\\	よくないですか。 
\\	そうですよ。彼のアドバイスを聞かないほうがいいです。自分の心に聞いたほうがいいですよ。さっきの人、彼女がいないでしょう?	
\\	そうですよ。彼のアドバイスを聞かないほうがいいです。自分の心に聞いたほうがいいですよ。さっきの人、彼女がいないでしょう? 
\\	まあ
\\	まあ
\\	ほら、アドバイスを受ける時は気をつけたほうがいいですよ。彼女に電話して。	
\\	ほら、アドバイスを受ける時は気をつけたほうがいいですよ。彼女に電話して。 
\\	わかりました。	
\\	わかりました。 
\\	立ち上がる
\\	ブラウザ
\\	じゃらん
\\	ホームページ
\\	オッケー
\\	新潟
\\	朝飯付き
\\	禁止
\\	違う
\\	立ち上がったよ。	
\\	立ち上がったよ。 
\\	じゃ、ブラウザを開いて、じゃらんのホームページに行って。	
\\	じゃ、ブラウザを開いて、じゃらんのホームページに行って。 
\\	はい、オッケー。	
\\	はい、オッケー。 
\\	じゃ、新潟のホテルを見て、すぐ行く
\\	どう?	
\\	じゃ、新潟のホテルを見て、すぐ行く
\\	どう? 
\\	いいところがあった。	
\\	いいところがあった。 
\\	いいね。ここがよさそうだ。安くて、朝飯付き。決まり。	
\\	いいね。ここがよさそうだ。安くて、朝飯付き。決まり。 
\\	よかった。	
\\	よかった。 
\\	ああ、だめだ。犬は禁止だ。	
\\	ああ、だめだ。犬は禁止だ。 
\\	そうね。じゃ、違うところを探そう。	
\\	そうね。じゃ、違うところを探そう。 
\\	正直
\\	花瓶
\\	壊れる
\\	壊す
\\	嘘
\\	犯人
\\	許す
\\	つぶる
\\	只今!	
\\	只今! 
\\	お帰り。	
\\	お帰り。 
\\	あれ!ママの大好きな花瓶が壊れている!うそ!けんじくん。	
\\	あれ!ママの大好きな花瓶が壊れている!うそ!けんじくん。 
\\	はい。	
\\	はい。 
\\	誰が私の花瓶を壊したの?	
\\	誰が私の花瓶を壊したの? 
\\	僕じゃない。家の猫が壊した。	
\\	僕じゃない。家の猫が壊した。 
\\	けんじくん。	
\\	けんじくん。 
\\	はい、ごめんなさい。犯人は僕です。許してください。	
\\	はい、ごめんなさい。犯人は僕です。許してください。 
\\	わかった。わかった。今回の嘘は目をつぶってあげるけど、これからは正直に言うのよ。	
\\	わかった。わかった。今回の嘘は目をつぶってあげるけど、これからは正直に言うのよ。 
\\	わかった。	
\\	わかった。 
\\	東京ドーム
\\	イベント
\\	虫
\\	集める
\\	集まる
\\	昆虫採集
\\	昆虫
\\	採集
\\	今日は何でこんなに電車がこんでいるの?	
\\	今日は何でこんなに電車がこんでいるの? 
\\	ああそうだ。今日東京ドームでイベントがある。	
\\	ああそうだ。今日東京ドームでイベントがある。 
\\	何のイベント?	
\\	何のイベント? 
\\	ああ、なんだっけ?あの、虫を集めること。ああ、思い出せない!	
\\	ああ、なんだっけ?あの、虫を集めること。ああ、思い出せない! 
\\	虫を集めること?	
\\	虫を集めること? 
\\	そう。虫を集める人が東京ドームに集まるイベント。	
\\	そう。虫を集める人が東京ドームに集まるイベント。 
\\	虫を集める人が東京ドームに集まる?	
\\	虫を集める人が東京ドームに集まる? 
\\	何て言うんだっけ?わかるでしょ。ほら、あれ、あれ。	
\\	何て言うんだっけ?わかるでしょ。ほら、あれ、あれ。 
\\	昆虫採集でしょう。	
\\	昆虫採集でしょう。 
\\	あ、そうだ。すいません。ありがとうございます。	
\\	あ、そうだ。すいません。ありがとうございます。 
\\	注文
\\	激辛
\\	かなり
\\	弱い
\\	普通
\\	お客さん
\\	すみません。注文お願いします。	
\\	すみません。注文お願いします。 
\\	はい、かしこまりました。	
\\	はい、かしこまりました。 
\\	マーポー豆腐お願いします。	
\\	マーポー豆腐お願いします。 
\\	辛さはどうしますか。	
\\	辛さはどうしますか。 
\\	そうですね。激辛お願いします。	
\\	そうですね。激辛お願いします。 
\\	激辛ですか?かなり辛いですよ。辛いものは大丈夫ですか。	
\\	激辛ですか?かなり辛いですよ。辛いものは大丈夫ですか。 
\\	辛いものにはちょっと弱いですが。	
\\	辛いものにはちょっと弱いですが。 
\\	普通の辛さがいいと思います。激辛はお客さんには辛すぎるかもしれません。	
\\	普通の辛さがいいと思います。激辛はお客さんには辛すぎるかもしれません。 
\\	はい、普通でお願いします。	
\\	はい、普通でお願いします。 
\\	警備
\\	鳴らす
\\	確認
\\	商店
\\	初日
\\	泥棒
\\	落ち着く
\\	ブザー
\\	防犯
\\	鳴る
\\	防犯ブザーがなっています!どうしょう!皆さん!ブザーがなっています!	
\\	防犯ブザーがなっています!どうしょう!皆さん!ブザーがなっています! 
\\	落ち着いてよ。	
\\	落ち着いてよ。 
\\	泥棒かもしれない。	
\\	泥棒かもしれない。 
\\	君はいつからこの仕事に入った?	
\\	君はいつからこの仕事に入った? 
\\	今日は初日ですが。	
\\	今日は初日ですが。 
\\	なるほど。普段は泥棒じゃない。よく風とか動物とかブザーを鳴らすんだ。今回もそれが原因だろう。ちょっと見せて。	
\\	なるほど。普段は泥棒じゃない。よく風とか動物とかブザーを鳴らすんだ。今回もそれが原因だろう。ちょっと見せて。 
\\	どうなっている?	
\\	どうなっている? 
\\	やっぱり。また小林商店だ。	
\\	やっぱり。また小林商店だ。 
\\	また!まったく。	
\\	また!まったく。 
\\	小林商店って。	
\\	小林商店って。 
\\	小林商店に犬がいる。あの犬はよく防犯ブザーを鳴らす。今回もあの犬だろう。じゃ、一緒に確認しに行こう。	
\\	小林商店に犬がいる。あの犬はよく防犯ブザーを鳴らす。今回もあの犬だろう。じゃ、一緒に確認しに行こう。 
\\	はい、わかりました。	
\\	はい、わかりました。 
\\	行きます
\\	中央
\\	迷惑
\\	暗証番号
\\	私が先に行きます。失礼します。	
\\	私が先に行きます。失礼します。 
\\	あら、やっぱり来てくれましたね。	
\\	あら、やっぱり来てくれましたね。 
\\	誰が来たの?	
\\	誰が来たの? 
\\	警備さんよ。格好いい警備さんが来てくれましたよ。	
\\	警備さんよ。格好いい警備さんが来てくれましたよ。 
\\	中央警備の高橋と申します。お忙しいところ、もうしわけございません。	
\\	中央警備の高橋と申します。お忙しいところ、もうしわけございません。 
\\	いいえ、いいえ。来てくれてよかったです。安心できます。こんなにいい警備会社がついていて、よかったです。	
\\	いいえ、いいえ。来てくれてよかったです。安心できます。こんなにいい警備会社がついていて、よかったです。 
\\	ありがとうございます。	
\\	ありがとうございます。 
\\	ああ、素敵ですね。初めまして。小林えりこと申します。宜しくお願いします。	
\\	ああ、素敵ですね。初めまして。小林えりこと申します。宜しくお願いします。 
\\	ありがとうございます。どうも。高橋と申します。	
\\	ありがとうございます。どうも。高橋と申します。 
\\	高橋さん、ご迷惑をかけてしまいましてもうしわけございません。私が防犯ブザーを鳴らしてしまいました。暗証番号を忘れてしまったので、消せないんです。	
\\	高橋さん、ご迷惑をかけてしまいましてもうしわけございません。私が防犯ブザーを鳴らしてしまいました。暗証番号を忘れてしまったので、消せないんです。 
\\	大丈夫です。すぐ消します!	
\\	大丈夫です。すぐ消します! 
\\	ありがとうございます!	
\\	ありがとうございます! 
\\	では、失礼します。	
\\	では、失礼します。 
\\	さようなら。	
\\	さようなら。 
\\	一緒に
\\	晩ご飯
\\	声
\\	低い
\\	公園
\\	散歩
\\	明子さん、やあ、こんにちは!元気ですか。ああ、よかった!あの、今週の金曜日なんですが、もしよかったら一緒に晩ご飯を食べませんか。	
\\	明子さん、やあ、こんにちは!元気ですか。ああ、よかった!あの、今週の金曜日なんですが、もしよかったら一緒に晩ご飯を食べませんか。 
\\	ははははは!お兄さんは馬鹿ね。	
\\	ははははは!お兄さんは馬鹿ね。 
\\	こら!いつから聞いてた!	
\\	こら!いつから聞いてた! 
\\	もしよかったら、もしよかったら、明子さん、あたしの馬鹿なお兄さんと付き合ってあげて。ははははは!	
\\	もしよかったら、もしよかったら、明子さん、あたしの馬鹿なお兄さんと付き合ってあげて。ははははは! 
\\	そんなに
\\	だまされる
\\	亡くなる
\\	奥さん
\\	優しい
\\	すべて
\\	はじめて
\\	鳴らす
\\	やっぱり
\\	早い
\\	警察
\\	はやかったな〜。なんでそんなに早いんだ?	
\\	はやかったな〜。なんでそんなに早いんだ? 
\\	問題なかったからですよ。	
\\	問題なかったからですよ。 
\\	よかった!やっぱりあの犬だよね。	
\\	よかった!やっぱりあの犬だよね。 
\\	いいえ。今回は小林さんが鳴らしてしまいました。	
\\	いいえ。今回は小林さんが鳴らしてしまいました。 
\\	ああ、そう?それははじめてだね。すべてチェックした?	
\\	ああ、そう?それははじめてだね。すべてチェックした? 
\\	もちろん!とても楽しかったです。	
\\	もちろん!とても楽しかったです。 
\\	楽しい?	
\\	楽しい? 
\\	うん、小林さんはとても優しい人でした。	
\\	うん、小林さんはとても優しい人でした。 
\\	そう?小林さんは去年奥さんがなくなってから、ちょっと冷たくなってきたけど。	
\\	そう?小林さんは去年奥さんがなくなってから、ちょっと冷たくなってきたけど。 
\\	ちょっと待って!何だって?彼の奥さんは去年なくなった?	
\\	ちょっと待って!何だって?彼の奥さんは去年なくなった? 
\\	うん、何で知ってる?	
\\	うん、何で知ってる? 
\\	おはよう。今日の調子はどうですか。	
\\	おはよう。今日の調子はどうですか。 
\\	今日は最悪です。	
\\	今日は最悪です。 
\\	どうしたんですか。	
\\	どうしたんですか。 
\\	今朝5時にカラスに起こされてから眠れないし、	
\\	今朝5時にカラスに起こされてから眠れないし、 
\\	まあでも、今朝の会議の発表が終わったら、もう心配ないね。もう少しだよ。	
\\	まあでも、今朝の会議の発表が終わったら、もう心配ないね。もう少しだよ。 
\\	轢く
\\	骨折
\\	次々
\\	全員
\\	暴走族
\\	また怪我したの?足はどうしたの?	
\\	また怪我したの?足はどうしたの? 
\\	今朝、自転車に乗っていて、車に轢かれて骨折したのよ。	
\\	今朝、自転車に乗っていて、車に轢かれて骨折したのよ。 
\\	一台だけ?僕は昔暴走族に次々に轢かれたことがあるよ。	
\\	一台だけ?僕は昔暴走族に次々に轢かれたことがあるよ。 
\\	え~、一台だけじゃないの?	
\\	え~、一台だけじゃないの? 
\\	いいえ、違います。全員でございます。	
\\	いいえ、違います。全員でございます。 
\\	じゃ、あなたの勝ちですね。	
\\	じゃ、あなたの勝ちですね。 
\\	ありがとうございます。	
\\	ありがとうございます。 
\\	でも、ある日ね。。。	
\\	でも、ある日ね。。。 
\\	貸す
\\	卒業
\\	高校
\\	文法
\\	違う
\\	同じ
\\	玉蜀黍
\\	カラス
\\	目的
\\	怖い
\\	かかし
\\	立てる
\\	全部
\\	ね、宏、パパに手を貸して。	
\\	ね、宏、パパに手を貸して。 
\\	パパは何してる?	
\\	パパは何してる? 
\\	かかしを立ててる。	
\\	かかしを立ててる。 
\\	やだ!そのかかしが怖い〜。	
\\	やだ!そのかかしが怖い〜。 
\\	それが目的だよ。怖くないとカラスにとうもろこしを食べられるよ。	
\\	それが目的だよ。怖くないとカラスにとうもろこしを食べられるよ。 
\\	カラスがとうもろこしを食べる?	
\\	カラスがとうもろこしを食べる? 
\\	そうだよ。全部!だから、このかかしがある。	
\\	そうだよ。全部!だから、このかかしがある。 
\\	ね、パパ、カラスにとうもろこしを食べられるとカラスがとうもろこしを食べる。。。同じことを言ってるね。	
\\	ね、パパ、カラスにとうもろこしを食べられるとカラスがとうもろこしを食べる。。。同じことを言ってるね。 
\\	そうだ。	
\\	そうだ。 
\\	何が違う?	
\\	何が違う? 
\\	文法。	
\\	文法。 
\\	文法って何?	
\\	文法って何? 
\\	文法ってパパが高校卒業できないように、作られたものさ。	
\\	文法ってパパが高校卒業できないように、作られたものさ。 
\\	大尉
\\	直接
\\	戻る
\\	両方
\\	あるいは
\\	見返す
\\	了解
\\	見張る
\\	敵
\\	命令
\\	説明する
\\	大尉、命令を待っています。	
\\	大尉、命令を待っています。 
\\	敵をそのまま見張って下さい。	
\\	敵をそのまま見張って下さい。 
\\	はい、了解!	
\\	はい、了解! 
\\	ね、大尉、敵に見られてますけど。どうします?	
\\	ね、大尉、敵に見られてますけど。どうします? 
\\	じゃあ、見返せば?	
\\	じゃあ、見返せば? 
\\	あれ、大尉は見ないんですか。	
\\	あれ、大尉は見ないんですか。 
\\	あ。どうしよう?あ、今マージャンで忙しいから。後で。	
\\	あ。どうしよう?あ、今マージャンで忙しいから。後で。 
\\	大尉!大変です!今敵に見られています!命令を!	
\\	大尉!大変です!今敵に見られています!命令を! 
\\	ちょっと待って。今見てる方あるいは見られてる方?	
\\	ちょっと待って。今見てる方あるいは見られてる方? 
\\	両方です!	
\\	両方です! 
\\	両方?どういうこと?	
\\	両方?どういうこと? 
\\	敵を見てるけど、敵に見られてます!	
\\	敵を見てるけど、敵に見られてます! 
\\	ええと。。。まだわからない。戻って直接説明しろ!	
\\	ええと。。。まだわからない。戻って直接説明しろ! 
\\	只今
\\	部長
\\	意味
\\	昇進
\\	営業部
\\	裏切り者
\\	おはよう!	
\\	おはよう! 
\\	部長!大変です!	
\\	部長!大変です! 
\\	高田はまだ?	
\\	高田はまだ? 
\\	只今電話がありまして、高田さんは会社にもう来ません。	
\\	只今電話がありまして、高田さんは会社にもう来ません。 
\\	うちのエースが会社来ないって?今日って言う意味?	
\\	うちのエースが会社来ないって?今日って言う意味? 
\\	いいえ、ずっとです。	
\\	いいえ、ずっとです。 
\\	どういうこと?	
\\	どういうこと? 
\\	スカウトされました。	
\\	スカウトされました。 
\\	誰に!	
\\	誰に! 
\\	あの会社、名前も言いたくない、あのライバル会社です。	
\\	あの会社、名前も言いたくない、あのライバル会社です。 
\\	あの裏切り者!もういい。高田さんはいらない!	
\\	あの裏切り者!もういい。高田さんはいらない! 
\\	彼女もスカウトされました。	
\\	彼女もスカウトされました。 
\\	まさか!	
\\	まさか! 
\\	そうです。同じ会社です。	
\\	そうです。同じ会社です。 
\\	って言うか、営業部、誰もいないけど。	
\\	って言うか、営業部、誰もいないけど。 
\\	全員です!営業部まるごとスカウトされました!	
\\	全員です!営業部まるごとスカウトされました! 
\\	電気
\\	最後
\\	消す
\\	鍵
\\	間違い
\\	おはようございます。	
\\	おはようございます。 
\\	上原さん、大変です。社長が切れてます。	
\\	上原さん、大変です。社長が切れてます。 
\\	ええどうしたんですか。	
\\	ええどうしたんですか。 
\\	昨日最後の人が 電気と電気湯沸かしポットを消すのを忘れていました。	
\\	昨日最後の人が 電気と電気湯沸かしポットを消すのを忘れていました。 
\\	でも、よくあることじゃないですか。	
\\	でも、よくあることじゃないですか。 
\\	それはそうですけど。鍵をかけるのも忘れてました!かなりやばい間違いですよ。	
\\	それはそうですけど。鍵をかけるのも忘れてました!かなりやばい間違いですよ。 
\\	まずい。実は、最後の人は私でした。	
\\	まずい。実は、最後の人は私でした。 
\\	ふり
\\	態度
\\	尋問
\\	留守
\\	勝手に
\\	ばかばかしい
\\	法律
\\	許可
\\	警告
\\	お母さん!お父さん!話があります!	
\\	お母さん!お父さん!話があります! 
\\	亮くん、どうしたの?	
\\	亮くん、どうしたの? 
\\	お母さん、知らないふりしないで。	
\\	お母さん、知らないふりしないで。 
\\	亮くん、ママに対してその態度はやめなさい。	
\\	亮くん、ママに対してその態度はやめなさい。 
\\	お父さんも。	
\\	お父さんも。 
\\	ああ、また始まった。	
\\	ああ、また始まった。 
\\	じゃあ、尋問を始めましょう。昨夜、私が留守にしていたときに私のパソコンを勝手に使ってしまったのは誰ですか。	
\\	じゃあ、尋問を始めましょう。昨夜、私が留守にしていたときに私のパソコンを勝手に使ってしまったのは誰ですか。 
\\	亮くん、このばかばかしい質問をやめなさい。	
\\	亮くん、このばかばかしい質問をやめなさい。 
\\	答えが出ました。	
\\	答えが出ました。 
\\	亮くん!私達がそのパソコンを買ってあげたのよ。	
\\	亮くん!私達がそのパソコンを買ってあげたのよ。 
\\	やっぱり、お母さん。あげたとすれば、法律の上で私の許可が必要ですよ。	
\\	やっぱり、お母さん。あげたとすれば、法律の上で私の許可が必要ですよ。 
\\	はい、はい。それで、どうするの?	
\\	はい、はい。それで、どうするの? 
\\	今回は警告だけです。以上です。失礼します。	
\\	今回は警告だけです。以上です。失礼します。 
\\	あなたをロースクールに行かせたのは誰?	
\\	あなたをロースクールに行かせたのは誰? 
\\	天才
\\	ユーチューブ
\\	番組
\\	心の準備
\\	その通り
\\	はい、もしもし。	
\\	はい、もしもし。 
\\	俺は天才だ!	
\\	俺は天才だ! 
\\	何?	
\\	何? 
\\	最近ユーチューブ、すごい人気あるじゃん?	
\\	最近ユーチューブ、すごい人気あるじゃん? 
\\	あ、うん。	
\\	あ、うん。 
\\	俺たちで番組を作ろうぜ!	
\\	俺たちで番組を作ろうぜ! 
\\	どういう(番組)?	
\\	どういう(番組)? 
\\	心の準備はできた?アーユーレディー?	
\\	心の準備はできた?アーユーレディー? 
\\	あ、うん、できた。	
\\	あ、うん、できた。 
\\	じゃ、行くぜ。	
\\	じゃ、行くぜ。 
\\	甘いのか。辛いのか。どっちか。	
\\	"甘いのか。辛いのか。どっちか。 
\\	何?!た、食べ物?	
\\	何?!た、食べ物? 
\\	その通り!	
\\	その通り! 
\\	いやあ。わかるけど。	
\\	いやあ。わかるけど。 
\\	もちろん、今からに決まってんじゃん。	
\\	もちろん、今からに決まってんじゃん。 
\\	控え室
\\	叱る
\\	文句
\\	戦術
\\	崩す
\\	落ち着く
\\	得策
\\	攻撃
\\	蹴る
\\	さっき
\\	状況
\\	(サッカーの試合 ハーフタイム控え室)山田!さっきのプレー、なんで右サイドに蹴って、右に動かなかったのよ!	
\\	(サッカーの試合 ハーフタイム控え室)山田!さっきのプレー、なんで右サイドに蹴って、右に動かなかったのよ! 
\\	えーっ!あそこは左に球を出して、左サイドから攻撃するのが得策でしょう!	
\\	えーっ!あそこは左に球を出して、左サイドから攻撃するのが得策でしょう! 
\\	おいおい。二人とも大声だして、どうしたんだ。もっと落ち着いて、話し合いなさい。	
\\	おいおい。二人とも大声だして、どうしたんだ。もっと落ち着いて、話し合いなさい。 
\\	左から崩すっていう監督の戦術通りにしたのに、島田が文句言うんです!監督、島田を叱ってください!	
\\	左から崩すっていう監督の戦術通りにしたのに、島田が文句言うんです!監督、島田を叱ってください! 
\\	山田の奴が状況も考えずに、ずっと左にしかパスをださないのが問題なんです。あいつのせいで何度チャンスを潰したことか!	
\\	山田の奴が状況も考えずに、ずっと左にしかパスをださないのが問題なんです。あいつのせいで何度チャンスを潰したことか! 
\\	今度は
\\	どういうこと
\\	どうしたの
\\	力を合わせる
\\	きっと
\\	勝利
\\	美酒
\\	酔う
\\	(サッカーの試合中 後半)ちょっと山田!なんで左の私にボール出して、走らないのよ!	
\\	(サッカーの試合中 後半)ちょっと山田!なんで左の私にボール出して、走らないのよ! 
\\	どういうこと!	
\\	(コーチ二人を呼ぶ)どうしたのよ。二人とも落ち着きなさい。二人が力を合わせればきっと勝てるわ!勝って勝利の美酒に酔おうじゃないの!	
\\	(コーチ二人を呼ぶ)どうしたのよ。二人とも落ち着きなさい。二人が力を合わせればきっと勝てるわ!勝って勝利の美酒に酔おうじゃないの! 
\\	遊園地
\\	空飛ぶじゅうたん
\\	お化け屋敷
\\	怖い
\\	苦手
\\	さあ、遊園地だぞ。たくさん遊ぶぞー。	
\\	さあ、遊園地だぞ。たくさん遊ぶぞー。 
\\	お父さん、あれに乗ってみたい。	
\\	お父さん、あれに乗ってみたい。 
\\	空飛ぶじゅうたんか・・・どれどれ。あー、10歳以上からじゃないと、乗れないんだよ。	
\\	空飛ぶじゅうたんか・・・どれどれ。あー、10歳以上からじゃないと、乗れないんだよ。 
\\	えー。じゃあ、あそこに行ってみたい。	
\\	えー。じゃあ、あそこに行ってみたい。 
\\	げっ。あれは、お化け屋敷じゃないか。怖いぞー。	
\\	げっ。あれは、お化け屋敷じゃないか。怖いぞー。 
\\	大丈夫だよ。いつもゲゲゲの鬼太郎みてるもん。どんなお化けが出てくるか見てみたい。	
\\	"大丈夫だよ。いつもゲゲゲの鬼太郎みてるもん。どんなお化けが出てくるか見てみたい。 
\\	困ったなあ。お父さん怖いの本当に苦手なんだよ。	
\\	困ったなあ。お父さん怖いの本当に苦手なんだよ。 
\\	えーっ、お化けが恐いの?!お父さん、ダサすぎー!	
\\	えーっ、お化けが恐いの?!お父さん、ダサすぎー! 
\\	病人
\\	浮気
\\	すきに
\\	鉄則
\\	どうしたの。そんな病人みたいな顔してー。	
\\	どうしたの。そんな病人みたいな顔してー。 
\\	昨日、太郎に浮気がばれて、大変だったのよ。	
\\	昨日、太郎に浮気がばれて、大変だったのよ。 
\\	えーっ!ホストとの浮気ばれたの。なんで、ばれたの。	
\\	えーっ!ホストとの浮気ばれたの。なんで、ばれたの。 
\\	それが、携帯を見たみたいなのよ。私がお風呂に入っているすきに。	
\\	それが、携帯を見たみたいなのよ。私がお風呂に入っているすきに。 
\\	メール消してなかったの。バカだなあ。メール消すのは鉄則でしょう。	
\\	メール消してなかったの。バカだなあ。メール消すのは鉄則でしょう。 
\\	形相
\\	ほんの
\\	一体
\\	二股
\\	修羅場
\\	お風呂から出て来たらさ、太郎が、鬼みたいな形相で立っててさ・・・もう修羅場だったよ。	
\\	お風呂から出て来たらさ、太郎が、鬼みたいな形相で立っててさ・・・もう修羅場だったよ。 
\\	それで、どうするの。やっぱり、ホストとは別れるんでしょ。	
\\	それで、どうするの。やっぱり、ホストとは別れるんでしょ。 
\\	う〜ん。ほんの遊びのつもりが、私、結構まじになっちゃったみたいなのよね。あー、私は一体どうしたらいいんだろう。	
\\	う〜ん。ほんの遊びのつもりが、私、結構まじになっちゃったみたいなのよね。あー、私は一体どうしたらいいんだろう。 
\\	そんなの簡単じゃん。二股だよ、二股!	
\\	そんなの簡単じゃん。二股だよ、二股! 
\\	避難訓練
\\	まさか
\\	正解
\\	すばやく
\\	静かに
\\	あわてない
\\	意味
\\	大切
\\	緊急
\\	緊急時
\\	今日は、避難訓練がある。	
\\	今日は、避難訓練がある。 
\\	えーっ。やだー。	
\\	えーっ。やだー。 
\\	緊急の場合に、守らないといけない、大切な言葉は何だったか、わかる人は手を上げて?	
\\	緊急の場合に、守らないといけない、大切な言葉は何だったか、わかる人は手を上げて? 
\\	はい!はい!はい!はい!	
\\	はい!はい!はい!はい! 
\\	じゃあ、山田。	
\\	じゃあ、山田。 
\\	えーっと、「オアシス」です。	
\\	"えーっと、「オアシス」です。 
\\	「オアシス」ってどういう意味だ。	
\\	"「オアシス」ってどういう意味だ。 
\\	「おさない」「あわてない」「静かに」「すばやく」の頭の文字をとったものです。	
\\	"「おさない」「あわてない」「静かに」「すばやく」の頭の文字をとったものです。 
\\	おおー、正解。まさか、山田が正解するとは・・・。	
\\	おおー、正解。まさか、山田が正解するとは・・・。 
\\	廊下
\\	大声
\\	悪戯
\\	意地悪
\\	ちょっかい
\\	先生、山田くんは「オアシス」なんて、できないと思います。	
\\	先生、山田くんは「オアシス」なんて、できないと思います。 
\\	なんで、そう思うんだ。	
\\	なんで、そう思うんだ。 
\\	悪戯ばかりするし。女の子に意地悪ばかりするんです。	
\\	どうなんだ、山田。そうなのか。	
\\	どうなんだ、山田。そうなのか。 
\\	えーっ!そんなことしないよ。	
\\	えーっ!そんなことしないよ。 
\\	嘘だあ。いつも、ちょっかい出してるくせに。べー。	
\\	嘘だあ。いつも、ちょっかい出してるくせに。べー。 
\\	うるさいな。のぞみには何もしてないだろっ!	
\\	うるさいな。のぞみには何もしてないだろっ! 
\\	山田:・・・。	
\\	なっなんなんだ、この二人は!?	
\\	なっなんなんだ、この二人は!? 
\\	会議
\\	資料
\\	着信音
\\	孫
\\	推薦
\\	すまないが、会議の資料を会議室に置いておいてくれ。	
\\	すまないが、会議の資料を会議室に置いておいてくれ。 
\\	(10分後)	
\\	会議室に人数分、置いておきました。あのぉ・・・	
\\	会議室に人数分、置いておきました。あのぉ・・・ 
\\	なんだ。	
\\	なんだ。 
\\	部長はモーニング娘のファンなのですか。机の携帯の着信音が「ラブ・マシーン」だったもので。	
\\	"部長はモーニング娘のファンなのですか。机の携帯の着信音が「ラブ・マシーン」だったもので。 
\\	(小声で)そうなんだよー。孫の影響でね。最初は馬鹿にしてたんだが、今はファンなんだ。でもコンサートのチケットが、どうしてもとれなくてね。	
\\	(小声で)そうなんだよー。孫の影響でね。最初は馬鹿にしてたんだが、今はファンなんだ。でもコンサートのチケットが、どうしてもとれなくてね。 
\\	モー娘のチケットなら、とれるかもしれませんよ。友達がレコード会社で働いているので。	
\\	モー娘のチケットなら、とれるかもしれませんよ。友達がレコード会社で働いているので。 
\\	本当か?!桐原君、場合によっては、来月の人事会議で、君のことを係長に推薦してもいいんだが・・・。	
\\	本当か?!桐原君、場合によっては、来月の人事会議で、君のことを係長に推薦してもいいんだが・・・。 
\\	予約を頼んでおきます!	
\\	予約を頼んでおきます! 
\\	明日
\\	メッセージ
\\	留守電
\\	到着
\\	前
\\	分
\\	時
\\	午後
\\	彼女
\\	着く
\\	早く
\\	山田さんは明日の何時に着きますか。	
\\	山田さんは明日の何時に着きますか。 
\\	彼女は午後6時15分前に到着します。	
\\	彼女は午後6時15分前に到着します。 
\\	ああ、留守電にメッセージがあります。	
\\	ああ、留守電にメッセージがあります。 
\\	もしもし、明日なんですが。	
\\	もしもし、明日なんですが。 
\\	申す
\\	致す
\\	謙譲語
\\	初めまして。	
\\	初めまして。 
\\	渡辺といいます。	
\\	渡辺といいます。 
\\	よろしくお願いします。	
\\	よろしくお願いします。 
\\	初めまして。	
\\	初めまして。 
\\	山口と申します。	
\\	山口と申します。 
\\	どうぞよろしくお願いいたします。	
\\	どうぞよろしくお願いいたします。 
\\	どうぞこちらへ。	
\\	どうぞこちらへ。 
\\	今日
\\	元気
\\	まあまあ
\\	普通
\\	こちら
\\	中学校
\\	友達
\\	夏子
\\	高見沢
\\	夏子、今日元気?	
\\	夏子、今日元気? 
\\	まあまあ。ヨシは?	
\\	まあまあ。ヨシは? 
\\	ふつう。こちらはたけです。中学校の友達。	
\\	ふつう。こちらはたけです。中学校の友達。 
\\	あ、初めまして。夏子です。よろしくお願いします。	
\\	あ、初めまして。夏子です。よろしくお願いします。 
\\	初めまして。高見沢です。よろしくお願いします。	
\\	初めまして。高見沢です。よろしくお願いします。 
\\	次
\\	円
\\	合計
\\	本
\\	ビール
\\	箱
\\	カレール一
\\	じゃが芋
\\	にんじん
\\	個
\\	たまねぎ
\\	お客さん
\\	おつり
\\	はい、次のお客さんどうぞ。	
\\	はい、次のお客さんどうぞ。 
\\	どうもありがとうございました。	
\\	どうもありがとうございました。 
\\	皆
\\	試合
\\	大事
\\	真剣
\\	楽しい
\\	必勝
\\	気持ちよい
\\	皆さん、今日の試合はとても大事です。真剣にやりましょう!	
\\	皆さん、今日の試合はとても大事です。真剣にやりましょう! 
\\	やあ、皆さん、今日はいい天気ですね。今日は楽しくやりましょう!	
\\	やあ、皆さん、今日はいい天気ですね。今日は楽しくやりましょう! 
\\	皆さん、必勝!必勝!必勝!	
\\	皆さん、必勝!必勝!必勝! 
\\	皆さん、気持ちよくやりましょう。	
\\	皆さん、気持ちよくやりましょう。 
\\	それでは、今日のジュニアサッカーの試合を始めます。	
\\	それでは、今日のジュニアサッカーの試合を始めます。 
\\	年末
\\	仕事
\\	残る
\\	戻る
\\	鳥取
\\	家族
\\	過ごす
\\	今年
\\	冬休み
\\	お土産
\\	もう年末ですね。	
\\	もう年末ですね。 
\\	そうですね。	
\\	そうですね。 
\\	再来週は冬休みですね。石川さんは今年の冬休みをどこで過ごしますか。	
\\	再来週は冬休みですね。石川さんは今年の冬休みをどこで過ごしますか。 
\\	そうですね。今年は家族が鳥取へ戻りますが、私は東京に残ります。仕事がありますから。長谷川さんはどうですか。	
\\	そうですね。今年は家族が鳥取へ戻りますが、私は東京に残ります。仕事がありますから。長谷川さんはどうですか。 
\\	私たちは今年もハワイへ行きます。	
\\	私たちは今年もハワイへ行きます。 
\\	いいですね。お土産、よろしくお願いします。	
\\	いいですね。お土産、よろしくお願いします。 
\\	首相
\\	ベッド
\\	パジャマ
\\	ロシア
\\	飲む
\\	お茶
\\	銀座
\\	イギリス
\\	食べる
\\	朝ごはん
\\	大統領
\\	ドイツ
\\	予定
\\	眠い
\\	子守歌
\\	川本首相おはようございます!	
\\	川本首相おはようございます! 
\\	おはよう。ああ、眠い~。	
\\	おはよう。ああ、眠い~。 
\\	今日の予定は?	
\\	今日の予定は? 
\\	今日のご予定は:	
\\	今日のご予定は: 
\\	9時15分にドイツの首相と朝御飯を食べます。	
\\	9時15分にドイツの首相と朝御飯を食べます。 
\\	それから、12時半にイギリスの首相と銀座でお茶を飲みます。	
\\	それから、12時半にイギリスの首相と銀座でお茶を飲みます。 
\\	そして、6時36分にロシアの大統領と晩御飯を食べます。	
\\	そして、6時36分にロシアの大統領と晩御飯を食べます。 
\\	夜は、10時2分前にパジャマを着て、ベッドに入ります。	
\\	夜は、10時2分前にパジャマを着て、ベッドに入ります。 
\\	その後、私が子守歌を歌います。	
\\	その後、私が子守歌を歌います。 
\\	今日も一日大変だ。	
\\	今日も一日大変だ。 
\\	いらっしゃいませ
\\	かなり
\\	楽天
\\	店
\\	買う
\\	結構
\\	弦楽器
\\	バイオリン
\\	千
\\	万
\\	ギター
\\	楽器
\\	お得
\\	いらっしゃいませ!	
\\	いらっしゃいませ! 
\\	楽器は高いですね!このギターは3万5千円ですよ!	
\\	楽器は高いですね!このギターは3万5千円ですよ! 
\\	3万5千円は悪くないですよ。そちらを見て。	
\\	3万5千円は悪くないですよ。そちらを見て。 
\\	6万3千円、7万4千円、8万9千円、10万円!	
\\	6万3千円、7万4千円、8万9千円、10万円! 
\\	ギターとかバイオリンとか、弦楽器は結構高いですよ。	
\\	ギターとかバイオリンとか、弦楽器は結構高いですよ。 
\\	高いですね。今日はどの楽器を買いますか。	
\\	高いですね。今日はどの楽器を買いますか。 
\\	店では買いません。店でいろいろ見ますが、楽天で買います。かなりお得です。	
\\	店では買いません。店でいろいろ見ますが、楽天で買います。かなりお得です。 
\\	大晦日
\\	いくら
\\	切符
\\	詰まらない
\\	過ごす
\\	一緒に
\\	思い出す
\\	ブラジル
\\	ケイワン
\\	ああ〜。楽しかった!
\\	1はなかなか面白いね。最高の大晦日!	
\\	ああ〜。楽しかった!
\\	1はなかなか面白いね。最高の大晦日! 
\\	はあ〜。全然楽しくなかった。切符はいくらだった?	
\\	はあ〜。全然楽しくなかった。切符はいくらだった? 
\\	俺の切符は5万2千円とお前の切符は5千円。。。	
\\	俺の切符は5万2千円とお前の切符は5千円。。。 
\\	切符も高かったし。今日は超つまらなかった!	
\\	切符も高かったし。今日は超つまらなかった! 
\\	でも、二人は一緒に大晦日を過ごした。	
\\	でも、二人は一緒に大晦日を過ごした。 
\\	一緒に???	
\\	一緒に??? 
\\	まあ、同じ建物だけど。。。	
\\	まあ、同じ建物だけど。。。 
\\	ああ〜。思い出した。面白いところがあった。	
\\	ああ〜。思い出した。面白いところがあった。 
\\	ああああああ	
\\	ああああああ 
\\	結構面白かった。	
\\	結構面白かった。 
\\	おはようございます
\\	コーヒー
\\	まずい
\\	あんパン
\\	木村屋
\\	白あんぱん
\\	秘書
\\	社長
\\	おはようございます!	
\\	おはようございます! 
\\	おはよう。	
\\	おはよう。 
\\	はい、どうぞ。コーヒーとあんパンです。	
\\	はい、どうぞ。コーヒーとあんパンです。 
\\	どうも。「噛む音から、吐き出す音」不味い!何これ?高橋さん!これは何!	
\\	どうも。「噛む音から、吐き出す音」不味い!何これ?高橋さん!これは何! 
\\	それはあんパンです。	
\\	それはあんパンです。 
\\	これは木村屋のあんパンじゃない!それに、これはしろあんパンだよ!しろあんパンはすきじゃない!	
\\	これは木村屋のあんパンじゃない!それに、これはしろあんパンだよ!しろあんパンはすきじゃない! 
\\	すみませんでした。	
\\	すみませんでした。 
\\	いらっしゃい
\\	新鮮な
\\	蟹
\\	エビ
\\	ホタテ
\\	美味しい
\\	季節
\\	まけてください
\\	いらっしゃい!いらっしゃいませ!新鮮な蟹、エビ、ホタテ、どうぞ、どうぞ!	
\\	いらっしゃい!いらっしゃいませ!新鮮な蟹、エビ、ホタテ、どうぞ、どうぞ! 
\\	すみません。	
\\	すみません。 
\\	はい、どうぞ!	
\\	はい、どうぞ! 
\\	この蟹はいくらですか。	
\\	この蟹はいくらですか。 
\\	その蟹は1万5千円ですよ。	
\\	その蟹は1万5千円ですよ。 
\\	1万5千円?高いな〜	
\\	1万5千円?高いな〜 
\\	この蟹は美味しいよ。蟹の季節ですよ。	
\\	この蟹は美味しいよ。蟹の季節ですよ。 
\\	ええ、まけてください.	
\\	ええ、まけてください. 
\\	じゃ、1万3千円。	
\\	じゃ、1万3千円。 
\\	いや、高いな~。500円で	
\\	いや、高いな~。500円で 
\\	500円!そりゃ海老一匹だよ。	
\\	500円!そりゃ海老一匹だよ。 
\\	叔父さん、頼むよ。まけてよ。	
\\	叔父さん、頼むよ。まけてよ。 
\\	じゃ、しょうがない。一万でいいよ。	
\\	じゃ、しょうがない。一万でいいよ。 
\\	やった!今夜は蟹鍋だ~。	
\\	やった!今夜は蟹鍋だ~。 
\\	やった!新しい鍋という家が見つかった蟹!皆、さようなら。	
\\	やった!新しい鍋という家が見つかった蟹!皆、さようなら。 
\\	名古屋
\\	すばらしい
\\	100万ドルの夜景
\\	ドル
\\	100万
\\	ホテル
\\	結婚式
\\	友達
\\	旅行
\\	うらやましい
\\	恵子さん、こんにちは。	
\\	恵子さん、こんにちは。 
\\	あ〜たけしさん、こんにちは。今からどこへ行きますか。	
\\	あ〜たけしさん、こんにちは。今からどこへ行きますか。 
\\	名古屋へ行きます。	
\\	名古屋へ行きます。 
\\	えっ、名古屋?旅行ですか?	
\\	えっ、名古屋?旅行ですか? 
\\	いいえ、旅行じゃありません。名古屋で友達の結婚式があります。	
\\	いいえ、旅行じゃありません。名古屋で友達の結婚式があります。 
\\	わ〜、いいですね。	
\\	わ〜、いいですね。 
\\	六甲山ホテルは100万ドルです。	
\\	六甲山ホテルは100万ドルです。 
\\	えっ、何?100万ドル?	
\\	えっ、何?100万ドル? 
\\	そう。ホテルで見る夜景は100万ドルです。100万ドルの夜景です。	
\\	そう。ホテルで見る夜景は100万ドルです。100万ドルの夜景です。 
\\	すばらしい!うらやましいですね。	
\\	すばらしい!うらやましいですね。 
\\	バレンタイン・デー
\\	かわいそう
\\	逆
\\	西洋
\\	聖バレンタイン
\\	自己新記録
\\	以上
\\	チョコレート
\\	部長
\\	東洋
\\	失礼します。	
\\	失礼します。 
\\	どうぞ!旨い!本当に旨い!これも。それも。バレンタイン・デー大好き!	
\\	どうぞ!旨い!本当に旨い!これも。それも。バレンタイン・デー大好き! 
\\	部長、チョコレートをいっぱいもらいましたね!	
\\	部長、チョコレートをいっぱいもらいましたね! 
\\	そうだね。田中さんからトリュフチョコレートをもらって、鈴木さんからこのケーキをもらって、	
\\	そうだね。田中さんからトリュフチョコレートをもらって、鈴木さんからこのケーキをもらって、 
\\	人面犬チョコ!あ...でもよく見るとキモカワ?	
\\	人面犬チョコ!あ...でもよく見るとキモカワ? 
\\	25個以上もらった!自己新記録だ!今日は幸せ!世界中の男が幸せだ。聖バレンタインにお礼をしたい!	
\\	25個以上もらった!自己新記録だ!今日は幸せ!世界中の男が幸せだ。聖バレンタインにお礼をしたい! 
\\	でも、西洋でバレンタイン・デーは逆ですよ。	
\\	でも、西洋でバレンタイン・デーは逆ですよ。 
\\	逆?	
\\	逆? 
\\	逆ですよ。	
\\	逆ですよ。 
\\	というと?	
\\	というと? 
\\	男が女からもらうのではなく、女が男からもらいます。	
\\	男が女からもらうのではなく、女が男からもらいます。 
\\	かわいそうな西洋の男達!ずっと日本にいたい。	
\\	かわいそうな西洋の男達!ずっと日本にいたい。 
\\	誰か
\\	受信料
\\	徴収
\\	伺う
\\	隠れる
\\	引っ越す
\\	絶対
\\	支払う
\\	(ベルが鳴る)誰かな?ちょっと見てくる。	
\\	(ベルが鳴る)誰かな?ちょっと見てくる。 
\\	すいませーん。
\\	でーす。受信料の徴収に伺いました。	
\\	すいませーん。
\\	でーす。受信料の徴収に伺いました。 
\\	どうしたの?	
\\	どうしたの? 
\\	静かにして!	
\\	静かにして! 
\\	誰が来た?	
\\	誰が来た? 
\\	の人だ。	
\\	の人だ。 
\\	また??	
\\	また?? 
\\	うん、受信料を取りにきた。	
\\	うん、受信料を取りにきた。 
\\	いつまで隠れるのよ?	
\\	いつまで隠れるのよ? 
\\	引っ越すまで、絶対支払わない!!	
\\	引っ越すまで、絶対支払わない!! 
\\	先週
\\	木曜日
\\	寒い
\\	捕る
\\	暖冬
\\	全部
\\	凍る
\\	逃げる
\\	デイブさん、いつ日本へ帰って来ましたか。	
\\	デイブさん、いつ日本へ帰って来ましたか。 
\\	先週の木曜日に帰って来ました。	
\\	先週の木曜日に帰って来ました。 
\\	カナダはどうでしたか。	
\\	カナダはどうでしたか。 
\\	寒かったです。	
\\	寒かったです。 
\\	カナダのシロクマは元気でしたか。	
\\	カナダのシロクマは元気でしたか。 
\\	いいえ、元気じゃありませんでした。	
\\	いいえ、元気じゃありませんでした。 
\\	どうして?	
\\	どうして? 
\\	おいしいアザラシが捕れません。	
\\	おいしいアザラシが捕れません。 
\\	どうして?	
\\	どうして? 
\\	カナダも暖冬です。海が全部凍りません。だから、アザラシが逃げます。	
\\	カナダも暖冬です。海が全部凍りません。だから、アザラシが逃げます。 
\\	まあ、かわいそうですね。シロクマは。でも、アザラシにはよかったかも!	
\\	まあ、かわいそうですね。シロクマは。でも、アザラシにはよかったかも! 
\\	失礼
\\	部長
\\	最悪
\\	自己新記録
\\	すっからかん
\\	失礼します。	
\\	失礼します。 
\\	何だ?	
\\	何だ? 
\\	ええ〜部長、どうしたんですか。大丈夫ですか。	
\\	ええ〜部長、どうしたんですか。大丈夫ですか。 
\\	ああ〜、今日は最悪だ。年に一番嫌いな日だ。ホワイト・デーだよ。	
\\	ああ〜、今日は最悪だ。年に一番嫌いな日だ。ホワイト・デーだよ。 
\\	ああ、そういうことですか。	
\\	ああ、そういうことですか。 
\\	25人にチョコをあげましたよ。自己新記録だ。今日はスッカラカンのからっけつだよ!誰がこの日を作ったんだ。ああ〜。	
\\	25人にチョコをあげましたよ。自己新記録だ。今日はスッカラカンのからっけつだよ!誰がこの日を作ったんだ。ああ〜。 
\\	でも、先月いっぱいもらいましたよね。	
\\	でも、先月いっぱいもらいましたよね。 
\\	でも、先月のことは先月だ。西洋はどうなってる?	
\\	でも、先月のことは先月だ。西洋はどうなってる? 
\\	西洋にホワイト・デーはありません。	
\\	西洋にホワイト・デーはありません。 
\\	いいな〜。やっぱり海外がいいな〜。	
\\	いいな〜。やっぱり海外がいいな〜。 
\\	だめだこりゃ。	
\\	だめだこりゃ。 
\\	元気
\\	箱
\\	大きい
\\	小さい
\\	声
\\	残念
\\	こんにちは。	
\\	こんにちは。 
\\	ワァーおじいちゃんだ!いらっしゃい。	
\\	ワァーおじいちゃんだ!いらっしゃい。 
\\	オー花子、元気だったかい。	
\\	オー花子、元気だったかい。 
\\	うん。	
\\	うん。 
\\	花子はいい子だね。はい、おみやげだよ。これは大きい箱、これは小さい箱。花子はどっちが好き?	
\\	花子はいい子だね。はい、おみやげだよ。これは大きい箱、これは小さい箱。花子はどっちが好き? 
\\	もちろん大きい箱!	
\\	もちろん大きい箱! 
\\	じゃあ、どうぞ。	
\\	じゃあ、どうぞ。 
\\	ワーイ、くまのぬいぐるみだ!!ありがとう、おじいちゃん。ところで小さい箱は何?	
\\	ワーイ、くまのぬいぐるみだ!!ありがとう、おじいちゃん。ところで小さい箱は何? 
\\	「ゲームボーイ」だよ。	
\\	"「ゲームボーイ」だよ。 
\\	えーっ。(小さい声で)それがいい…	
\\	えーっ。(小さい声で)それがいい… 
\\	残念でした。	
\\	残念でした。 
\\	映画
\\	大好き
\\	姫
\\	野球
\\	嫌い(な)
\\	映画、好きですか。	
\\	映画、好きですか。 
\\	はい、大好きです。先週も映画を見ました。	
\\	はい、大好きです。先週も映画を見ました。 
\\	そうですか。よかった!ここに映画のチケットがあります。どうぞ。	
\\	そうですか。よかった!ここに映画のチケットがあります。どうぞ。 
\\	やったー!「もののけ姫」だ!ありがとう!!	
\\	やったー!「もののけ姫」だ!ありがとう!! 
\\	私は映画が嫌いです。	
\\	私は映画が嫌いです。 
\\	金子:えー!私は野球が好きな人が嫌いです。	
\\	引っ越し
\\	便利
\\	不便
\\	家賃
\\	静か
\\	田中さん、久しぶり!	
\\	田中さん、久しぶり! 
\\	あ〜、久しぶり。元気?	
\\	あ〜、久しぶり。元気? 
\\	うん、元気よ。実は私、先週引っ越したの。	
\\	うん、元気よ。実は私、先週引っ越したの。 
\\	そうですか。新しい家はどう?	
\\	そうですか。新しい家はどう? 
\\	駅から近くて便利よ。そしてとても静かなの。それに、家賃も安いのよ。	
\\	駅から近くて便利よ。そしてとても静かなの。それに、家賃も安いのよ。 
\\	それはすばらしい!私の家は駅から遠くて、不便です。そして、うるさいです。それに、家賃も高いです。私も引っ越したい!	
\\	それはすばらしい!私の家は駅から遠くて、不便です。そして、うるさいです。それに、家賃も高いです。私も引っ越したい! 
\\	でも.......私の新しい家は、茨城県よ!	
\\	でも.......私の新しい家は、茨城県よ! 
\\	元気
\\	昨日
\\	風邪
\\	大変
\\	大丈夫
\\	薬屋
\\	帰る
\\	寝る
\\	(坂本さんと山下さんが道で偶然出会う 。)(弱々しく)あら、山下さんこんにちは。	
\\	(坂本さんと山下さんが道で偶然出会う 。)(弱々しく)あら、山下さんこんにちは。 
\\	あー!坂本さんこんにちは。元気ですか。	
\\	あー!坂本さんこんにちは。元気ですか。 
\\	それが....クシュンクシュン(くしゃみ)、コンコンコン(咳)昨日から風邪をひいていて....コン!コン!.....元気じゃありません。	
\\	それが....クシュンクシュン(くしゃみ)、コンコンコン(咳)昨日から風邪をひいていて....コン!コン!.....元気じゃありません。 
\\	まあ、それは大変ですね。大丈夫ですか。	
\\	まあ、それは大変ですね。大丈夫ですか。 
\\	あんまり大丈夫じゃないです。これから薬屋さんへよって、家に帰ります。お粥を食べて、そして寝ます。おやすみなさい。	
\\	あんまり大丈夫じゃないです。これから薬屋さんへよって、家に帰ります。お粥を食べて、そして寝ます。おやすみなさい。 
\\	おやすみなさい?まだ午後なんだけど。本当に大丈夫かな?	
\\	おやすみなさい?まだ午後なんだけど。本当に大丈夫かな? 
\\	一緒に
\\	食事
\\	お気に入り
\\	郵便局
\\	隣
\\	おいしい
\\	スパゲティ屋
\\	残念
\\	この辺
\\	あっ、もう12時半ですね。田淵さん一緒に食事しませんか。	
\\	あっ、もう12時半ですね。田淵さん一緒に食事しませんか。 
\\	それはいいですね。この辺に岡田さんのお気に入りのレストランはありますか。	
\\	それはいいですね。この辺に岡田さんのお気に入りのレストランはありますか。 
\\	え〜っと......じゃあ、あの郵便局の隣においしいスパゲティ屋さんがありますが.....どうですか。	
\\	え〜っと......じゃあ、あの郵便局の隣においしいスパゲティ屋さんがありますが.....どうですか。 
\\	そのスパゲティ屋さんにはどんなメニューがありますか。	
\\	そのスパゲティ屋さんにはどんなメニューがありますか。 
\\	ナポリタン、ミートソース、ボンゴレ......何でもありますよ。	
\\	ナポリタン、ミートソース、ボンゴレ......何でもありますよ。 
\\	それはすばらしい。ピザはありますか。	
\\	それはすばらしい。ピザはありますか。 
\\	あー、残念ですが、ピザはありません。スパゲティだけです。あっ、サラダもあります。	
\\	あー、残念ですが、ピザはありません。スパゲティだけです。あっ、サラダもあります。 
\\	そうですか。じゃあ行きましょう。	
\\	そうですか。じゃあ行きましょう。 
\\	いらっしゃいませ。	
\\	いらっしゃいませ。 
\\	二人ですが、席ありますか。	
\\	二人ですが、席ありますか。 
\\	すみません。今いっぱいで、席がありません。	
\\	すみません。今いっぱいで、席がありません。 
\\	なぁ〜んだ!残念!	
\\	なぁ〜んだ!残念! 
\\	お腹
\\	いっぱい
\\	残る
\\	残す
\\	帰る
\\	主人
\\	やっぱり、ここのスパゲティはおいしかったね。でももうお腹いっぱい!	
\\	やっぱり、ここのスパゲティはおいしかったね。でももうお腹いっぱい! 
\\	私もおなかいっぱい!ミートソーススパゲティたくさん残ったね〜。もったいないね。岡田さんの家に犬はいますか。	
\\	私もおなかいっぱい!ミートソーススパゲティたくさん残ったね〜。もったいないね。岡田さんの家に犬はいますか。 
\\	いいえ、犬はいません。でも猫はいますよ。	
\\	いいえ、犬はいません。でも猫はいますよ。 
\\	じゃあ、猫にこのスパゲティをもって帰りませんか。	
\\	じゃあ、猫にこのスパゲティをもって帰りませんか。 
\\	わぁ〜ありがとう。もちろん持って帰ります。今日、主人はいません。だから、このスパゲティは私の夕食です。ヤッター!	
\\	わぁ〜ありがとう。もちろん持って帰ります。今日、主人はいません。だから、このスパゲティは私の夕食です。ヤッター! 
\\	携帯電話
\\	小学生
\\	必要
\\	それなら
\\	メール
\\	ゲーム
\\	ねぇ、お母さん、携帯電話買ってよ~。	
\\	ねぇ、お母さん、携帯電話買ってよ~。 
\\	ダメ。小学生に携帯電話は必要ないでしょ。	
\\	ダメ。小学生に携帯電話は必要ないでしょ。 
\\	みんな持ってるよ。いくつになったら買ってくれるの?	
\\	みんな持ってるよ。いくつになったら買ってくれるの? 
\\	携帯電話で何をするの?	
\\	携帯電話で何をするの? 
\\	ゲームとか、メールするんだよ。	
\\	ゲームとか、メールするんだよ。 
\\	電話をするだけなら、買ってもいいわよ。	
\\	電話をするだけなら、買ってもいいわよ。 
\\	えー!!それなら、携帯電話いらないよ!	
\\	えー!!それなら、携帯電話いらないよ! 
\\	鳥のから揚げ
\\	じゃんけん
\\	ばっかり
\\	鳥のから揚げ、買ってきたわよ。	
\\	鳥のから揚げ、買ってきたわよ。 
\\	わ~い、やったー!	
\\	わ~い、やったー! 
\\	ふたりで分けて、食べてね。	
\\	ふたりで分けて、食べてね。 
\\	何個あるのかな?	
\\	何個あるのかな? 
\\	6個あるよ。	
\\	6個あるよ。 
\\	じゃあ、じゃんけんで勝った人が好きなのをとろう。	
\\	じゃあ、じゃんけんで勝った人が好きなのをとろう。 
\\	じゃんけんぽん!	
\\	じゃんけんぽん! 
\\	わーい、勝った!じゃあ、これと、これと、これね。	
\\	わーい、勝った!じゃあ、これと、これと、これね。 
\\	ちぇっ、大きいのばっかりとったな!	
\\	ちぇっ、大きいのばっかりとったな! 
\\	着る
\\	鏡
\\	素晴らしい
\\	お似合い
\\	お客様
\\	すいません。このジャケット着ていいですか。	
\\	すいません。このジャケット着ていいですか。 
\\	どうぞ。こちらに鏡がありますよ。	
\\	どうぞ。こちらに鏡がありますよ。 
\\	とてもお似合いですよ。ダークグレーはお客様にピッタリです。	
\\	とてもお似合いですよ。ダークグレーはお客様にピッタリです。 
\\	そうですかね。	
\\	そうですかね。 
\\	あわせて、あちらの黒いズボンもはいてみてはどうですか。	
\\	あわせて、あちらの黒いズボンもはいてみてはどうですか。 
\\	はあ。	
\\	はあ。 
\\	素晴らしい。この色はお客様にピッタリでございますよ。	
\\	素晴らしい。この色はお客様にピッタリでございますよ。 
\\	はあ・・・。	
\\	はあ・・・。 
\\	すてき(な)
\\	帽子
\\	足
\\	靴下
\\	思う
\\	こちらの靴下を、はいてみてはいかがでしょうか。ズボンと、ばっちり合うと思いますよ。	
\\	こちらの靴下を、はいてみてはいかがでしょうか。ズボンと、ばっちり合うと思いますよ。 
\\	(木ノ下、靴下をはく)	
\\	素晴らしい。なんとお客様の足にフィットすることでしょう。	
\\	素晴らしい。なんとお客様の足にフィットすることでしょう。 
\\	はあ…。	
\\	はあ…。 
\\	(木ノ下、帽子をかぶる)	
\\	着せ替え人形
\\	点
\\	全部
\\	俳優
\\	お買い上げ
\\	(木ノ下、サングラスをかける)	
\\	お客様、素晴らしいです。とってもダンディーでございます。いかがでしょうか。	
\\	お客様、素晴らしいです。とってもダンディーでございます。いかがでしょうか。 
\\	私は、あなたの着せ替え人形じゃありません!ただジャケットを着てみたかっただけなんです。	
\\	私は、あなたの着せ替え人形じゃありません!ただジャケットを着てみたかっただけなんです。 
\\	本当ですか。	
\\	本当ですか。 
\\	私は、嘘など申しません。	
\\	私は、嘘など申しません。 
\\	本当に。	
\\	本当に。 
\\	本当ですよ。サングラスをかけますと、どこぞの俳優さんかと思っちゃいます。	
\\	本当ですよ。サングラスをかけますと、どこぞの俳優さんかと思っちゃいます。 
\\	まじですか。じゃあ、5点全部買います。	
\\	まじですか。じゃあ、5点全部買います。 
\\	お買い上げありがとうございます!	
\\	お買い上げありがとうございます! 
\\	信号
\\	拍手
\\	進む
\\	止まる
\\	渡る
\\	それぞれ
\\	信号には、色が3つあるけど、何色かわかる人は手をあげてー。	
\\	信号には、色が3つあるけど、何色かわかる人は手をあげてー。 
\\	はいはいはいはいはいっ。	
\\	はいはいはいはいはいっ。 
\\	じゃあ、マサル。	
\\	じゃあ、マサル。 
\\	はい。青と黄と赤です。	
\\	はい。青と黄と赤です。 
\\	先生:じゃあ、マサル。青と黄色と赤、それぞれどんな意味かわかるかい。	
\\	はい、先生。青色は進んでよい。黄色は止まれるなら止まれ。赤色は止まれ、です。	
\\	はい、先生。青色は進んでよい。黄色は止まれるなら止まれ。赤色は止まれ、です。 
\\	マサル、よくできました。よく知っていたなあ。	
\\	マサル、よくできました。よく知っていたなあ。 
\\	先生、先生。マサルくんは、わかっていないと思います。	
\\	先生、先生。マサルくんは、わかっていないと思います。 
\\	なんでそう思うんだ。	
\\	なんでそう思うんだ。 
\\	だって、マサルくんは「赤信号、皆で渡ればこわくない」って、いつも赤信号なのに渡ろうとするんです。	
\\	だって、マサルくんは「赤信号、皆で渡ればこわくない」って、いつも赤信号なのに渡ろうとするんです。 
\\	なっ、なんだとー。	
\\	なっ、なんだとー。 
\\	郵便局
\\	どれくらい
\\	かかる
\\	料金
\\	差
\\	船便
\\	航空便
\\	小包
\\	局員
\\	着く
\\	すみません。これをイギリスに送りたいのですが。	
\\	すみません。これをイギリスに送りたいのですが。 
\\	小包ですか。	
\\	小包ですか。 
\\	はい。	
\\	はい。 
\\	航空便と船便とありますが、どちらになさいますか。	
\\	航空便と船便とありますが、どちらになさいますか。 
\\	着くのに、どれくらい差がありますか。	
\\	着くのに、どれくらい差がありますか。 
\\	船便ですと、航空便より料金は安いですが、一ヶ月くらいかかります。航空便ですと、一週間以内には着きます。	
\\	船便ですと、航空便より料金は安いですが、一ヶ月くらいかかります。航空便ですと、一週間以内には着きます。 
\\	どうしようかな・・・。	
\\	どうしようかな・・・。 
\\	違う
\\	ちょうど
\\	半値
\\	間に合う
\\	用紙
\\	記入
\\	宛先
\\	お預かりする
\\	(局員、小包の重さをはかる)	
\\	小包はちょうど5
\\	ですので、航空便ですと10,150円、船便ですと4,000円になります。	
\\	小包はちょうど5
\\	ですので、航空便ですと10,150円、船便ですと4,000円になります。 
\\	正倫:じゃあ、航空便でお願いします。	
\\	では、こちらの用紙に記入して下さい。	
\\	では、こちらの用紙に記入して下さい。 
\\	正倫:お願いします。	
\\	お預かりしました。	
\\	お預かりしました。 
\\	一杯
\\	他
\\	袋
\\	合計
\\	追加
\\	ビールを一杯ください。	
\\	ビールを一杯ください。 
\\	キリンとハイネケンとサッポロがありますが。	
\\	キリンとハイネケンとサッポロがありますが。 
\\	じゃあ、ハイネケンで。	
\\	じゃあ、ハイネケンで。 
\\	かしこまりました。他には何か。	
\\	かしこまりました。他には何か。 
\\	えーっと、ミックスナッツを一つと、ポテトチップスを二袋ください。	
\\	えーっと、ミックスナッツを一つと、ポテトチップスを二袋ください。 
\\	かしこまりました。合計で1200円です。	
\\	かしこまりました。合計で1200円です。 
\\	あっ、後ハートチョコを三個追加で。	
\\	あっ、後ハートチョコを三個追加で。 
\\	誰
\\	本当
\\	来週
\\	楽しみ
\\	改札口
\\	山田さん、もしよかったら、クラシックのコンサートに行きませんか。	
\\	山田さん、もしよかったら、クラシックのコンサートに行きませんか。 
\\	誰のコンサートですか。	
\\	誰のコンサートですか。 
\\	小沢征爾のコンサートです。チケットを二枚もらったので、山田さんと行きたいなあと思って。	
\\	小沢征爾のコンサートです。チケットを二枚もらったので、山田さんと行きたいなあと思って。 
\\	本当ですか。小沢征爾は大好きなんです。	
\\	本当ですか。小沢征爾は大好きなんです。 
\\	じゃあ決まり。では、来週の日曜日の6時に上野駅の改札口の前で。	
\\	じゃあ決まり。では、来週の日曜日の6時に上野駅の改札口の前で。 
\\	わかりました。楽しみにしてます。	
\\	わかりました。楽しみにしてます。 
\\	申し訳ありません
\\	前売り券
\\	売り切れ
\\	大人気
\\	困る
\\	小沢征爾のコンサートのチケットを二枚お願いします。	
\\	小沢征爾のコンサートのチケットを二枚お願いします。 
\\	申し訳ありません。前売り券はもう売り切れてしまいました。	
\\	申し訳ありません。前売り券はもう売り切れてしまいました。 
\\	ええーっ!	
\\	ええーっ! 
\\	すいません。大人気なので、もう手に入らないと思いますよ。	
\\	すいません。大人気なので、もう手に入らないと思いますよ。 
\\	困ったなあ・・・山田さんに、チケット二枚もらったって言っちゃったよ。あーどうしよう!	
\\	困ったなあ・・・山田さんに、チケット二枚もらったって言っちゃったよ。あーどうしよう! 
\\	太め
\\	好み
\\	ガッチリ
\\	少しくらい
\\	痩せ
\\	ねえ。美穂って太めの男の人が好みなのよね。	
\\	ねえ。美穂って太めの男の人が好みなのよね。 
\\	太めっていうか、ガッチリした人がいいわ。	
\\	太めっていうか、ガッチリした人がいいわ。 
\\	ガッチリしてるって・・・私はマッチョな人は嫌だわ。	
\\	ガッチリしてるって・・・私はマッチョな人は嫌だわ。 
\\	少しくらいマッチョの方が、男らしくていいじゃない。	
\\	少しくらいマッチョの方が、男らしくていいじゃない。 
\\	えーっ。私は細くて痩せてる人の方がいいわ。	
\\	えーっ。私は細くて痩せてる人の方がいいわ。 
\\	それは、あなたがデブだからよ。デブは痩せが好きなの!	
\\	それは、あなたがデブだからよ。デブは痩せが好きなの! 
\\	ひっひどい・・・。気にしてるのに。	
\\	ひっひどい・・・。気にしてるのに。 
\\	ボール
\\	侮辱
\\	貴様
\\	一塁
\\	フォアボール
\\	へぼ
\\	程
\\	我慢
\\	直球
\\	三振
\\	気を取り直す
\\	ストライク
\\	退場
\\	ボール!	
\\	ボール! 
\\	(あれ。さっきはあのコースはストライクだったんだけどな。まあいいや、まだ2ストライク1ボール。気を取り直してと。)	
\\	(あれ。さっきはあのコースはストライクだったんだけどな。まあいいや、まだ2ストライク1ボール。気を取り直してと。) 
\\	ボール!	
\\	ボール! 
\\	(えっ。ストライクだろう!。どこ見てんだ、あのアンパイア。これで三振だ。)	
\\	(えっ。ストライクだろう!。どこ見てんだ、あのアンパイア。これで三振だ。) 
\\	ボール。	
\\	ボール。 
\\	(おいおい。あのボールがボールだって。直球だったのに。我慢にも程があるぞ、あの、へぼアンパイアー!)	
\\	(おいおい。あのボールがボールだって。直球だったのに。我慢にも程があるぞ、あの、へぼアンパイアー!) 
\\	ボール、フォアボール、ランナー一塁!	
\\	ボール、フォアボール、ランナー一塁! 
\\	(大きい声で)おい、ふざけんなよアンパイア!今のがボールだって。さっきからどこに目つけてんだよ!何球ストライク・ボール投げてるとおもってんだ。	
\\	(大きい声で)おい、ふざけんなよアンパイア!今のがボールだって。さっきからどこに目つけてんだよ!何球ストライク・ボール投げてるとおもってんだ。 
\\	貴様、私を侮辱する気か。ピッチャー退場!	
\\	貴様、私を侮辱する気か。ピッチャー退場! 
\\	砂漠
\\	難しい
\\	暑い
\\	面白い
\\	乾燥する
\\	先生、砂漠ってどんな所ですか。	
\\	先生、砂漠ってどんな所ですか。 
\\	砂漠かあ。どうしてだ?	
\\	砂漠かあ。どうしてだ? 
\\	面白そうだから、砂漠に行ってみたいんです。	
\\	面白そうだから、砂漠に行ってみたいんです。 
\\	大変な所だぞ。暑いし、乾燥するし、雨はふらないし、夜は寒いし。	
\\	大変な所だぞ。暑いし、乾燥するし、雨はふらないし、夜は寒いし。 
\\	のどがカラカラになっても水が飲めないんですか。	
\\	のどがカラカラになっても水が飲めないんですか。 
\\	難しいかもしれないな。	
\\	難しいかもしれないな。 
\\	そんなー。先生、私やっぱり行きたくないです。	
\\	そんなー。先生、私やっぱり行きたくないです。 
\\	三塁
\\	判定
\\	微妙
\\	解説者
\\	猛抗議
\\	形相
\\	激突
\\	クロスプレー
\\	中継プレー
\\	本塁
\\	ベースコーチ
\\	荒れる
\\	(腕を振り回す)。まわれ、まわれ、まわれ!	
\\	(腕を振り回す)。まわれ、まわれ、まわれ! 
\\	おおっと、二塁ランナー、三塁をまわったー。本塁間に合うかあ。	
\\	おおっと、二塁ランナー、三塁をまわったー。本塁間に合うかあ。 
\\	キャッチャーがボールを落としたー。	
\\	キャッチャーがボールを落としたー。 
\\	おおっと、アンパイアはアウトのコールだあ。三塁ベースコーチがすごい形相でアンパイアに猛抗議だあ。見た感じですと、セーフに見えたんですがどうですかね。	
\\	おおっと、アンパイアはアウトのコールだあ。三塁ベースコーチがすごい形相でアンパイアに猛抗議だあ。見た感じですと、セーフに見えたんですがどうですかね。 
\\	そうですね。タイミング的にはアウトでしたが、キャッチャーがボールを落としてますからね。ただ、ボールを落としたのがランナーにタッチをした後と判断したようですね。微妙な判定です。抗議も当然ですね。	
\\	そうですね。タイミング的にはアウトでしたが、キャッチャーがボールを落としてますからね。ただ、ボールを落としたのがランナーにタッチをした後と判断したようですね。微妙な判定です。抗議も当然ですね。 
\\	忘れ物
\\	届く
\\	財布
\\	受付
\\	入口
\\	すいません。忘れ物は届いていませんか。	
\\	すいません。忘れ物は届いていませんか。 
\\	何を忘れたんですか。	
\\	何を忘れたんですか。 
\\	オープン・エリアに財布を忘れてしまったんです。	
\\	オープン・エリアに財布を忘れてしまったんです。 
\\	何時くらいですか。	
\\	何時くらいですか。 
\\	二時過ぎだと思います。	
\\	二時過ぎだと思います。 
\\	届いているとしたら、受付にあると思うので、入口の受付で聞いてもらえますか。	
\\	届いているとしたら、受付にあると思うので、入口の受付で聞いてもらえますか。 
\\	はい。わかりました。	
\\	はい。わかりました。 
\\	お腹
\\	学食
\\	大学生
\\	鳥の唐揚げ
\\	豚の生姜焼き
\\	もう1時だよ。お腹空いたね。	
\\	もう1時だよ。お腹空いたね。 
\\	今日は、
\\	ランチが鳥の唐揚げで、
\\	ランチが豚の生姜焼きだって。	
\\	今日は、
\\	ランチが鳥の唐揚げで、
\\	ランチが豚の生姜焼きだって。 
\\	どっちにしようかな。	
\\	どっちにしようかな。 
\\	ランチは390円です。	
\\	ランチは390円です。 
\\	じゃ、
\\	ランチをください。	
\\	じゃ、
\\	ランチをください。 
\\	すいません。僕も、それをください。	
\\	すいません。僕も、それをください。 
\\	メイドバー
\\	終わらせる
\\	小柴、仕事終わったらメイドバーに行かない?	
\\	小柴、仕事終わったらメイドバーに行かない? 
\\	お、いいじゃん!行くべ!	
\\	お、いいじゃん!行くべ! 
\\	じゃあ、何時に仕事終わらせる?	
\\	じゃあ、何時に仕事終わらせる? 
\\	んじゃ、8時までに終わらせるべ。	
\\	んじゃ、8時までに終わらせるべ。 
\\	。んじゃそういうことで。	
\\	。んじゃそういうことで。 
\\	お帰りなさい
\\	主人
\\	様
\\	久しぶり
\\	最近
\\	忙しい
\\	萌え
\\	お帰りなさいませご主人様!	
\\	お帰りなさいませご主人様! 
\\	おー、メイドバー来るの久しぶりじゃね?	
\\	おー、メイドバー来るの久しぶりじゃね? 
\\	最近忙しかったからなー。	
\\	最近忙しかったからなー。 
\\	小柴、おまえ何飲む?おれ、ビール飲むけど。	
\\	小柴、おまえ何飲む?おれ、ビール飲むけど。 
\\	ゴクゴク	
\\	萌え~~
\\	萌え~~
\\	おい
\\	俺
\\	借りる
\\	かわいい
\\	ちょっと
\\	触る
\\	おい、山本!このキティーちゃん、かわいいだろ!	
\\	おい、山本!このキティーちゃん、かわいいだろ! 
\\	そうですね!かわいいです!ちょっと借りても良いですか?	
\\	そうですね!かわいいです!ちょっと借りても良いですか? 
\\	何を言ってんだ!だめだ、だめだ!!	
\\	何を言ってんだ!だめだ、だめだ!! 
\\	す、すいません!	
\\	す、すいません! 
\\	俺の、キティーちゃんに触るんじゃねー!!	
\\	俺の、キティーちゃんに触るんじゃねー!! 
\\	うらやましい
\\	欲しい
\\	もちろん
\\	お前
\\	修理
\\	部品
\\	取り寄せる
\\	かかる
\\	できあがる
\\	値段
\\	結構
\\	すいませーん。この腕時計を修理してもらいたいのですが。	
\\	すいませーん。この腕時計を修理してもらいたいのですが。 
\\	はいはい、どれですか。	
\\	はいはい、どれですか。 
\\	これです。	
\\	これです。 
\\	あー、これは部品を取り寄せないとダメですね。	
\\	あー、これは部品を取り寄せないとダメですね。 
\\	修理にどれぐらいかかりますか?	
\\	修理にどれぐらいかかりますか? 
\\	部品を取り寄せるのに四日、修理に三日かなー。来週の水曜日にはできあがると思いますよ。	
\\	部品を取り寄せるのに四日、修理に三日かなー。来週の水曜日にはできあがると思いますよ。 
\\	お値段はいくらですか。	
\\	お値段はいくらですか。 
\\	000円になります。	
\\	000円になります。 
\\	新しいの買うので結構です。。。。	
\\	新しいの買うので結構です。。。。 
\\	週末
\\	暑い
\\	水着
\\	売り場
\\	似合う
\\	暑いから、週末に海に行こうか。	
\\	暑いから、週末に海に行こうか。 
\\	(デパートの水着売り場で)	
\\	うーん、どっちの水着がいいかなあ。	
\\	うーん、どっちの水着がいいかなあ。 
\\	俺はこっちのビキニのほうがワンピースの水着より似合うと思うな。	
\\	俺はこっちのビキニのほうがワンピースの水着より似合うと思うな。 
\\	じゃあ、ビキニの方にする。10000円するけど、買ってね。	
\\	じゃあ、ビキニの方にする。10000円するけど、買ってね。 
\\	えーっ!俺が買うのー。	
\\	えーっ!俺が買うのー。 
\\	ユニフォーム
\\	だるい
\\	セール
\\	暑い
\\	ご主人様
\\	お帰りなさいませご主人様!!	
\\	お帰りなさいませご主人様!! 
\\	あー、だりぃ。このユニフォーム暑いし。	
\\	あー、だりぃ。このユニフォーム暑いし。 
\\	お帰りなさいませご主人様だって。だるいなぁ。仕事終わってセールに行きたいんだけどなぁ。。	
\\	"お帰りなさいませご主人様だって。だるいなぁ。仕事終わってセールに行きたいんだけどなぁ。。 
\\	今日も来てくれたの?!	
\\	今日も来てくれたの?! 
\\	みき、とってもうれしい!!	
\\	みき、とってもうれしい!! 
\\	萌え
\\	サラリーマン
\\	かっこ良い
\\	ルックス
\\	悪い
\\	古い
\\	はい、お待たせしましたご主人様。ビールでございます!!	
\\	はい、お待たせしましたご主人様。ビールでございます!! 
\\	萌えー	
\\	萌えー 
\\	あのサラリーマンの二人、「萌えー」って言ってるし。。。	
\\	"あのサラリーマンの二人、「萌えー」って言ってるし。。。 
\\	「萌えー」って。。。。古いわ。。。	
\\	「萌えー」って。。。。古いわ。。。 
\\	あの二人、かっこ良いんだけどなぁ。	
\\	あの二人、かっこ良いんだけどなぁ。 
\\	ルックスは悪くないわねぇ。	
\\	ルックスは悪くないわねぇ。 
\\	から揚げ
\\	お代わり
\\	失礼
\\	手作り
\\	来週
\\	お待たせしました、から揚げとカレーライスでございます!	
\\	お待たせしました、から揚げとカレーライスでございます! 
\\	(メイドたちが久保と小柴食べさせる)	
\\	それじゃ、失礼します。	
\\	それじゃ、失礼します。 
\\	ご主人様、あーん!	
\\	ご主人様、あーん! 
\\	あーん!	
\\	あーん! 
\\	おいしい?ご主人様?	
\\	おいしい?ご主人様? 
\\	あー、みみちゃん達が作った手作りカレーとから揚げ、最高じゃね?!	
\\	あー、みみちゃん達が作った手作りカレーとから揚げ、最高じゃね?! 
\\	流暢
\\	漢字
\\	覚える
\\	韓国
\\	日本語がとても上手ですね。日本に来られてどれくらいなんですか。	
\\	日本語がとても上手ですね。日本に来られてどれくらいなんですか。 
\\	今、ちょうど三ヶ月です。	
\\	今、ちょうど三ヶ月です。 
\\	えっ。たったの三ヶ月ですか。なんでそんなに日本語が流暢なんですか。	
\\	えっ。たったの三ヶ月ですか。なんでそんなに日本語が流暢なんですか。 
\\	日本語は11歳から勉強し始めました。でも漢字はなかなか覚えられません。	
\\	日本語は11歳から勉強し始めました。でも漢字はなかなか覚えられません。 
\\	韓国では漢字は使わないのですか。	
\\	韓国では漢字は使わないのですか。 
\\	昔は使っていたのですが、今はほとんど使われていないんです。	
\\	昔は使っていたのですが、今はほとんど使われていないんです。 
\\	そうなんですか。	
\\	そうなんですか。 
\\	分かる
\\	電源
\\	何度
\\	宿題
\\	テレビゲーム
\\	いいところ
\\	またテレビゲームしてる。もう宿題はしたの?	
\\	またテレビゲームしてる。もう宿題はしたの? 
\\	えー、これ終わったらするよ。	
\\	えー、これ終わったらするよ。 
\\	まだ終わってないの?!宿題してからテレビゲームしなさいって、何度言わせるの?!	
\\	まだ終わってないの?!宿題してからテレビゲームしなさいって、何度言わせるの?! 
\\	電源を切る	
\\	あっ!なにするんだよ、今いいところだったのに!	
\\	あっ!なにするんだよ、今いいところだったのに! 
\\	宿題が終わらないうちは、ゲームしちゃだめっていつも言ってるでしょ、まだ分からないの??!!	
\\	宿題が終わらないうちは、ゲームしちゃだめっていつも言ってるでしょ、まだ分からないの??!! 
\\	彼女
\\	彼氏
\\	許す
\\	身分
\\	呼ぶ
\\	お父さん、真由美さんと結婚させてください!!!	
\\	お父さん、真由美さんと結婚させてください!!! 
\\	だめだ、だめだ!!	
\\	だめだ、だめだ!! 
\\	お父さん、おねがい私たちの結婚を許して。	
\\	お父さん、おねがい私たちの結婚を許して。 
\\	学生の身分で何を言っているんだ。	
\\	学生の身分で何を言っているんだ。 
\\	お父さん、おねがいします!!	
\\	お父さん、おねがいします!! 
\\	お前に、「お父さん」なんて呼ばせんぞ!	
\\	"お前に、「お父さん」なんて呼ばせんぞ! 
\\	実は
\\	黙る
\\	なんか
\\	やる
\\	一緒に
\\	家
\\	俺
\\	お前
\\	娘
\\	実はお父さん、話さなければいけない事があるの。	
\\	実はお父さん、話さなければいけない事があるの。 
\\	おい、ちょ、待てって。。。それは俺から言うから。	
\\	おい、ちょ、待てって。。。それは俺から言うから。 
\\	いいえ、私に言わせて。お父さん、いままで黙ってたけど。	
\\	いいえ、私に言わせて。お父さん、いままで黙ってたけど。 
\\	お前なんかに、娘はやらんぞ!!	
\\	お前なんかに、娘はやらんぞ!! 
\\	もう、一緒にいえを買ったの。だから、結婚させて下さい!	
\\	もう、一緒にいえを買ったの。だから、結婚させて下さい! 
\\	マジで!?ほんとに?。。。。。うっそー!そりゃーないよー。	
\\	マジで!?ほんとに?。。。。。うっそー!そりゃーないよー。 
\\	しょうがない
\\	専業主夫
\\	代わりに
\\	駄目
\\	一生懸命
\\	やっぱり
\\	お願いします!!!!	
\\	お願いします!!!! 
\\	しょうがない。。。わかった。だが、その前に質問させてくれ。本当に娘を幸せにできるか?	
\\	しょうがない。。。わかった。だが、その前に質問させてくれ。本当に娘を幸せにできるか? 
\\	はい、もちろんできます!	
\\	はい、もちろんできます! 
\\	それで、仕事は?	
\\	それで、仕事は? 
\\	やりません!専業主夫になります!代わりに娘さんを一生懸命働かせます。	
\\	やりません!専業主夫になります!代わりに娘さんを一生懸命働かせます。 
\\	やっぱり、結婚は駄目だぁ!!	
\\	やっぱり、結婚は駄目だぁ!! 
\\	雑草
\\	芝刈り
\\	ことわざ
\\	いっそ
\\	素敵
\\	青々
\\	生える
\\	張り替える
\\	困る
\\	次から次へと
\\	また庭の雑草がボーボーになってきて困ったわ。取っても取っても、次から次へと生えてくるし。	
\\	また庭の雑草がボーボーになってきて困ったわ。取っても取っても、次から次へと生えてくるし。 
\\	いっそのこと、芝を張り替えようか。	
\\	いっそのこと、芝を張り替えようか。 
\\	張り替えても、また雑草が生えるだけよ。誰が雑草取りするのよ。	
\\	張り替えても、また雑草が生えるだけよ。誰が雑草取りするのよ。 
\\	俺は仕事で忙しいし。。。	
\\	俺は仕事で忙しいし。。。 
\\	お隣なんて、いつもご主人が芝刈りしてるでしょ。だから、青々として素敵な芝生じゃない。	
\\	お隣なんて、いつもご主人が芝刈りしてるでしょ。だから、青々として素敵な芝生じゃない。 
\\	だから、「隣の芝生は青い」っていうじゃないか!?	
\\	"だから、「隣の芝生は青い」っていうじゃないか!? 
\\	それって、ことわざでしょ。本当に隣の芝生は青いのよ。	
\\	それって、ことわざでしょ。本当に隣の芝生は青いのよ。 
\\	恋愛
\\	授業
\\	旅行
\\	あり余る
\\	男子校
\\	図書館
\\	片っ端から
\\	毎日エネルギーがあり余っているんですが、どうしたらいいですか。	
\\	毎日エネルギーがあり余っているんですが、どうしたらいいですか。 
\\	それなら、たくさん旅行に行ったほうがいいよ。	
\\	それなら、たくさん旅行に行ったほうがいいよ。 
\\	旅行かあ・・・でも授業があります。	
\\	旅行かあ・・・でも授業があります。 
\\	では、たくさん恋愛をした方がいいよ。	
\\	では、たくさん恋愛をした方がいいよ。 
\\	恋愛かあ・・・ここ男子校ですよ。	
\\	恋愛かあ・・・ここ男子校ですよ。 
\\	じゃあ、たくさん本を読んだ方がいいよ。	
\\	じゃあ、たくさん本を読んだ方がいいよ。 
\\	本かあ・・・じゃあ早速、夏休みに図書館の本を片っ端から読んでみます。	
\\	本かあ・・・じゃあ早速、夏休みに図書館の本を片っ端から読んでみます。 
\\	疲れる
\\	旅館
\\	料理
\\	温泉
\\	最高
\\	一泊二日
\\	時期
\\	もうそろそろ
\\	スノーボード疲れたねぇ。	
\\	スノーボード疲れたねぇ。 
\\	そうねぇ。でも、この旅館、料理はおいしいし、温泉はあるし、サービスも良いし、最高じゃない。	
\\	そうねぇ。でも、この旅館、料理はおいしいし、温泉はあるし、サービスも良いし、最高じゃない。 
\\	一泊二日の旅行じゃ、もったいないね。二泊三日の旅行にすればよかったよ。	
\\	一泊二日の旅行じゃ、もったいないね。二泊三日の旅行にすればよかったよ。 
\\	でも、この時期に安く旅行できるなんてラッキーよ。	
\\	でも、この時期に安く旅行できるなんてラッキーよ。 
\\	あ、そうだ。もうそろそろ僕からのクリスマスプレゼントが。。。	
\\	あ、そうだ。もうそろそろ僕からのクリスマスプレゼントが。。。 
\\	え、何をくれるの?	
\\	え、何をくれるの? 
\\	わぁーきれい!!	
\\	わぁーきれい!! 
\\	メリークリスマス!!!	
\\	メリークリスマス!!! 
\\	アダプター
\\	ギガ
\\	ハードディスク ドライブ
\\	ご覧
\\	商品
\\	ポイント
\\	付く
\\	申す
\\	いらっしゃいませー	
\\	いらっしゃいませー 
\\	あれっ?!あなた、この前のデパートの店員さんじゃない?!	
\\	あれっ?!あなた、この前のデパートの店員さんじゃない?! 
\\	はぁ?いえいえ、デパートで働いたことなどありません。	
\\	はぁ?いえいえ、デパートで働いたことなどありません。 
\\	ほんとうー。。。?	
\\	ほんとうー。。。? 
\\	私はウソなど申しません!	
\\	私はウソなど申しません! 
\\	はぁ。。。。あ、ところで今日は
\\	アダプターが欲しいんだけど。	
\\	はぁ。。。。あ、ところで今日は
\\	アダプターが欲しいんだけど。 
\\	お客さま、ご覧ください!!!500ギガのハードディスクドライブ。きっとお客様の新しい
\\	アダプターに合うと思いますよ。	
\\	お客さま、ご覧ください!!!500ギガのハードディスクドライブ。きっとお客様の新しい
\\	アダプターに合うと思いますよ。 
\\	いやいや、僕はこの
\\	アダプターだけ欲しいんだけど。。。。	
\\	いやいや、僕はこの
\\	アダプターだけ欲しいんだけど。。。。 
\\	ですがお客様、こちらの商品を買えば200ポイント付いてきますよ!!	
\\	ですがお客様、こちらの商品を買えば200ポイント付いてきますよ!! 
\\	メモリー
\\	ウィンドウズ
\\	アップグレード
\\	レオパードOS
\\	気に入る
\\	サービスする
\\	マウスパッド
\\	サウンドカード
\\	オフ
\\	パーセント
\\	ものすごく
\\	なさる
\\	お客様、ご覧下さい、このすばらしいメモリーを!	
\\	お客様、ご覧下さい、このすばらしいメモリーを! 
\\	こちらの商品でお客様のコンピュータはものすごーく速くなります!	
\\	こちらの商品でお客様のコンピュータはものすごーく速くなります! 
\\	はぁ。。。	
\\	はぁ。。。 
\\	あっ、こちらのサウンドカードはいかがですか?お客様の
\\	カードスロットに合うと思いませんか?	
\\	あっ、こちらのサウンドカードはいかがですか?お客様の
\\	カードスロットに合うと思いませんか? 
\\	はぁ。。。。	
\\	はぁ。。。。 
\\	あらららららららら。。。。こちらの商品、絶対お客様は気に入ると思いますよ。	
\\	あらららららららら。。。。こちらの商品、絶対お客様は気に入ると思いますよ。 
\\	えっ!私はウィンドウズを使っているんで結構です!	
\\	えっ!私はウィンドウズを使っているんで結構です! 
\\	荷物
\\	レジ
\\	ウェブカメラ
\\	ポート
\\	ハブ
\\	ポイントカード
\\	足す
\\	買い物上手
\\	お買い上げ
\\	お客様こちらでございます。わたくしが、お荷物をレジまで運びますので。。。	
\\	お客様こちらでございます。わたくしが、お荷物をレジまで運びますので。。。 
\\	はぁ。。。。	
\\	はぁ。。。。 
\\	ああああっ!!!!ウェブカメラでございます!!!!これなら、8ポートUSBハブに絶対に合いますよ!	
\\	ああああっ!!!!ウェブカメラでございます!!!!これなら、8ポートUSBハブに絶対に合いますよ! 
\\	レジにて	
\\	お客様、ポイントカードはお持ちですか?	
\\	お客様、ポイントカードはお持ちですか? 
\\	はい、今日はポイントカードを使いたいんですけど。	
\\	はい、今日はポイントカードを使いたいんですけど。 
\\	はうあっ!!!!	
\\	はうあっ!!!! 
\\	お客様、驚きました!	
\\	お客様、驚きました! 
\\	え、本当に???(うれしそうに)	
\\	え、本当に???(うれしそうに) 
\\	まずい
\\	遅刻
\\	開演
\\	もうすぐ
\\	着く
\\	怖い
\\	文句
\\	待つ
\\	(携帯)	
\\	今、どこ?開演まであと、3分しかないよ。	
\\	今、どこ?開演まであと、3分しかないよ。 
\\	ごめん、ごめん、もうすぐ着く。今着く。	
\\	ごめん、ごめん、もうすぐ着く。今着く。 
\\	遅い!いつも、タローは自分のことしか考えていない!(プチッ)	
\\	遅い!いつも、タローは自分のことしか考えていない!(プチッ) 
\\	おー、怖い怖い。まったく。クミコは文句しか言わないね。	
\\	おー、怖い怖い。まったく。クミコは文句しか言わないね。 
\\	お客様、開演まで、あと1分しかありませんよ。	
\\	お客様、開演まで、あと1分しかありませんよ。 
\\	あ、はい。でももう少し、ここで待ちます。	
\\	あ、はい。でももう少し、ここで待ちます。 
\\	嘘
\\	警備員
\\	カイロ
\\	最近
\\	喧嘩
\\	誕生日
\\	きっと
\\	もう7時半か。「今着く」なんて、タローはいつも嘘ばっかり。はぁ・・・。あーあー。	
\\	"もう7時半か。「今着く」なんて、タローはいつも嘘ばっかり。はぁ・・・。あーあー。 
\\	お客さん、大丈夫ですか。寒いですよね。カイロ使いますか?どうぞ。	
\\	お客さん、大丈夫ですか。寒いですよね。カイロ使いますか?どうぞ。 
\\	ありがとうございます。私の彼、毎日仕事ばかりしていて、私はいつもひとり。	
\\	ありがとうございます。私の彼、毎日仕事ばかりしていて、私はいつもひとり。 
\\	多分
\\	昔
\\	妻
\\	子供達
\\	世話
\\	寂しい
\\	大切
\\	死ぬ
\\	家族
\\	もういいです。多分来ませんよ。帰ります。カイロありがとうございました。	
\\	もういいです。多分来ませんよ。帰ります。カイロありがとうございました。 
\\	あと10分だけ待ちませんか。きっと、来ますよ。	
\\	あと10分だけ待ちませんか。きっと、来ますよ。 
\\	じゃ、あと10分だけ。	
\\	じゃ、あと10分だけ。 
\\	私も、昔、毎日仕事のことだけ考えていました。仕事が楽しかったんです。妻は家で子供達の世話だけしていました。	
\\	私も、昔、毎日仕事のことだけ考えていました。仕事が楽しかったんです。妻は家で子供達の世話だけしていました。 
\\	お父さん!	
\\	お父さん! 
\\	タロウ!	
\\	タロウ! 
\\	すき焼き
\\	俺
\\	どうして
\\	豆腐
\\	歯
\\	やわらかい
\\	アレルギー
\\	さっき
\\	タロウ、さっきから、お肉ばかり食べているー。ほら、野菜も食べて。	
\\	タロウ、さっきから、お肉ばかり食べているー。ほら、野菜も食べて。 
\\	ははは、タロウは野菜が嫌いで、肉しか食べないからな。	
\\	ははは、タロウは野菜が嫌いで、肉しか食べないからな。 
\\	俺、野菜アレルギーがあるんだ。	
\\	俺、野菜アレルギーがあるんだ。 
\\	もー、嘘ばっかり。それから、どうして、お父さんは豆腐だけ食べているの。	
\\	もー、嘘ばっかり。それから、どうして、お父さんは豆腐だけ食べているの。 
\\	あ…。	
\\	あ…。 
\\	父さんは歯が悪いんだ。だからやわらかいものしか食べないんだ。	
\\	父さんは歯が悪いんだ。だからやわらかいものしか食べないんだ。 
\\	これはいいお肉だからやわらかいの。お肉も食べて!	
\\	これはいいお肉だからやわらかいの。お肉も食べて! 
\\	さっきから、クミは、人の世話ばかりしているな。	
\\	さっきから、クミは、人の世話ばかりしているな。 
\\	お母さんみたいだ...。なぁ、タロウ。	
\\	お母さんみたいだ...。なぁ、タロウ。 
\\	危ない
\\	かかる
\\	くらい
\\	遠い
\\	決める
\\	おばあちゃん
\\	おじいちゃん
\\	後
\\	卒業
\\	先週
\\	レポート
\\	最後
\\	大丈夫
\\	最後のレポート、書き終わったか?	
\\	最後のレポート、書き終わったか? 
\\	うん。先週、書き終えたよ。	
\\	うん。先週、書き終えたよ。 
\\	卒業の後、どうする?働き始める?	
\\	卒業の後、どうする?働き始める? 
\\	日本に行く。	
\\	日本に行く。 
\\	日本?	
\\	日本? 
\\	日本は面白い国だよ。日本には侍と芸者がいるんだ。	
\\	日本は面白い国だよ。日本には侍と芸者がいるんだ。 
\\	へぇ…。決めた!俺も日本へ行く。	
\\	へぇ…。決めた!俺も日本へ行く。 
\\	え?本当?でも、日本人は魚が好きだから、ちょっと危ないよ。	
\\	え?本当?でも、日本人は魚が好きだから、ちょっと危ないよ。 
\\	忙しい
\\	向かう
\\	利根川
\\	オープンする
\\	パチンコ屋
\\	怖い
\\	針
\\	入院する
\\	おじさん
\\	ナマズ
\\	隣の
\\	ミミズ
\\	食事する
\\	急ぐ
\\	ねぇ、ウナ助君、今、忙しい?	
\\	ねぇ、ウナ助君、今、忙しい? 
\\	いや、食事しているところだけど、何?	
\\	いや、食事しているところだけど、何? 
\\	あ、そのミミズ だめだよ。	
\\	あ、そのミミズ だめだよ。 
\\	え?おお。本当だ。針がある。日本人は怖いな。	
\\	え?おお。本当だ。針がある。日本人は怖いな。 
\\	ねぇ、隣の川に新しいパチンコ屋がオープンしたんだ。行かない?	
\\	ねぇ、隣の川に新しいパチンコ屋がオープンしたんだ。行かない? 
\\	パチンコ?行く。行く。	
\\	パチンコ?行く。行く。 
\\	あれ?ウナ吉君?	
\\	あれ?ウナ吉君? 
\\	あ、ウナ美ちゃん!ウナ美ちゃんも利根川に来たの?	
\\	あ、ウナ美ちゃん!ウナ美ちゃんも利根川に来たの? 
\\	かわいい
\\	負ける
\\	勝つ
\\	ついている
\\	リーチ
\\	だめ
\\	産卵する
\\	多分
\\	結婚する
\\	年上
\\	紹介する
\\	今度
\\	おごります
\\	おいおい、かわいいうなぎだなぁ。今度紹介して。	
\\	おいおい、かわいいうなぎだなぁ。今度紹介して。 
\\	(パチンコ屋)	
\\	ウナ助君、どう?出てる?	
\\	ウナ助君、どう?出てる? 
\\	・・・一時間後・・・	
\\	はっはっはー!ついているね。10万も勝ったよ!。	
\\	はっはっはー!ついているね。10万も勝ったよ!。 
\\	いいなぁ。僕は3万も負けてしまったよ。ついてない・・・。	
\\	いいなぁ。僕は3万も負けてしまったよ。ついてない・・・。 
\\	よし!今日は、俺が美味しいミミズのさしみをおごるよ。	
\\	よし!今日は、俺が美味しいミミズのさしみをおごるよ。 
\\	飲み物
\\	モモンガ
\\	以上
\\	かしこまりました。
\\	ドレッシング
\\	ガーデンサラダ
\\	注文
\\	グレープ フルーツ ジュース
\\	お決まりですか
\\	少々
\\	お飲み物はお決まりですか。	
\\	お飲み物はお決まりですか。 
\\	グレープフルーツジュース、お願いします。	
\\	グレープフルーツジュース、お願いします。 
\\	かしこまりました。	
\\	かしこまりました。 
\\	グレープフルーツジュースです。どうぞ。	
\\	グレープフルーツジュースです。どうぞ。 
\\	注文いいですか。	
\\	注文いいですか。 
\\	はい、どうぞ。	
\\	はい、どうぞ。 
\\	えっと、ガーデンサラダを一つと、、、	
\\	えっと、ガーデンサラダを一つと、、、 
\\	はい。	
\\	はい。 
\\	ドレッシングなしで、お願いします。	
\\	ドレッシングなしで、お願いします。 
\\	ガーデンサラダを、ドレッシングなしで。…かしこまりました。	
\\	ガーデンサラダを、ドレッシングなしで。…かしこまりました。 
\\	あとは…えっと…あっ!そうそうそう、モモンガのチーズメンチカツを一つ。	
\\	あとは…えっと…あっ!そうそうそう、モモンガのチーズメンチカツを一つ。 
\\	モモンガチーズメンチですね…以上で?	
\\	モモンガチーズメンチですね…以上で? 
\\	はい。以上です。	
\\	はい。以上です。 
\\	かしこまりました。少々お待ちください。	
\\	かしこまりました。少々お待ちください。 
\\	ごゆっくりどうぞ
\\	吐く
\\	かかる
\\	げっ
\\	おえっ
\\	半生
\\	ヌケ作
\\	お待たせいたしました!ガーデンサラダとモモンガのチーズメンチカツでございます!	
\\	お待たせいたしました!ガーデンサラダとモモンガのチーズメンチカツでございます! 
\\	どうも。	
\\	どうも。 
\\	ごゆっくりどうぞ。	
\\	ごゆっくりどうぞ。 
\\	まずっ!なんだこれ?食べられねぇよ。	
\\	まずっ!なんだこれ?食べられねぇよ。 
\\	サラダは大丈夫かな。げっ!ドレッシングかかってる。	
\\	サラダは大丈夫かな。げっ!ドレッシングかかってる。 
\\	はい!	
\\	はい! 
\\	あの、このモモンガチーズメンチ、半生ですよ!	
\\	あの、このモモンガチーズメンチ、半生ですよ! 
\\	申し訳ございません!グレープフルーツジュースはいかかでしょうか。	
\\	申し訳ございません!グレープフルーツジュースはいかかでしょうか。 
\\	死ぬ
\\	うまい
\\	やばい
\\	初めて
\\	料理
\\	おすすめ
\\	焼肉屋
\\	他の
\\	食事
\\	大変、お待たせいたしました。	
\\	大変、お待たせいたしました。 
\\	死ぬほどうまい!このサラダやばい!	
\\	死ぬほどうまい!このサラダやばい! 
\\	ありがとうございます。このモモンガはいかがでしょうか。	
\\	ありがとうございます。このモモンガはいかがでしょうか。 
\\	えぇ?ふ~ん。	
\\	えぇ?ふ~ん。 
\\	いかがでしょうか。	
\\	いかがでしょうか。 
\\	うまい!!こんなうまい料理は初めてだ!	
\\	うまい!!こんなうまい料理は初めてだ! 
\\	王子
\\	昔
\\	ファンクラブ
\\	時々
\\	コンサート
\\	熱烈な
\\	王女
\\	噂
\\	一番君、「おにぎり王子」知っていますか。	
\\	一番君、「おにぎり王子」知っていますか。 
\\	はい、僕は昔から、大ファンなんですよ。	
\\	はい、僕は昔から、大ファンなんですよ。 
\\	へー、熱烈なファンですね。	
\\	へー、熱烈なファンですね。 
\\	えっ、あの噂の!?	
\\	えっ、あの噂の!? 
\\	登場
\\	大ファン
\\	今度
\\	サイン
\\	入れる
\\	タコウインナー
\\	中身
\\	お弁当
\\	追いかける
\\	必ず
\\	おにぎり王子とおにぎり王女!追いかけよう!!	
\\	おにぎり王子とおにぎり王女!追いかけよう!! 
\\	ねえダーリン、今度、お弁当作るから、ピクニックに行きましょう。	
\\	ねえダーリン、今度、お弁当作るから、ピクニックに行きましょう。 
\\	本当に!!じゃあ、あとタコウインナーも作って。	
\\	本当に!!じゃあ、あとタコウインナーも作って。 
\\	ええっ、タコウインナーは、作ったことがないんだけど。	
\\	ええっ、タコウインナーは、作ったことがないんだけど。 
\\	だって、僕のママは、必ずタコウインナーをお弁当に入れるよ。	
\\	だって、僕のママは、必ずタコウインナーをお弁当に入れるよ。 
\\	事務所
\\	デート
\\	じゃま
\\	態度
\\	柔道
\\	投げる
\\	ごめんねぇ。事務所がうるさくて、僕はサインすることができないんだ。	
\\	ごめんねぇ。事務所がうるさくて、僕はサインすることができないんだ。 
\\	(アオ怒る)	
\\	なによ!その態度!!	
\\	なによ!その態度!! 
\\	私は、柔道三段だから、あなた達を投げることができるのよ!!	
\\	私は、柔道三段だから、あなた達を投げることができるのよ!! 
\\	ひい~、分かりました~。サインします。	
\\	ひい~、分かりました~。サインします。 
\\	あ、ありがとうございます。	
\\	あ、ありがとうございます。 
\\	じゃ、じゃあ、さようなら。	
\\	じゃ、じゃあ、さようなら。 
\\	20世紀
\\	小学校
\\	卒業
\\	埋める
\\	賛成
\\	集合
\\	掘る
\\	目印
\\	銀杏
\\	今日で20世紀最後だね。	
\\	今日で20世紀最後だね。 
\\	うん。そして、来年は私たちももう小学校卒業ね。	
\\	うん。そして、来年は私たちももう小学校卒業ね。 
\\	なあなあ、おれたち3人でタイムカプセルを埋めようぜ?	
\\	なあなあ、おれたち3人でタイムカプセルを埋めようぜ? 
\\	賛成!!	
\\	賛成!! 
\\	(ゾウさん公園)	
\\	じゃあ、みんな準備は良いか。	
\\	じゃあ、みんな準備は良いか。 
\\	みんなの持ってきた物を、ここに埋めておこう。	
\\	みんなの持ってきた物を、ここに埋めておこう。 
\\	同窓会
\\	以来
\\	振り
\\	ブタ
\\	違う
\\	まき!はじめ!久しぶり!	
\\	まき!はじめ!久しぶり! 
\\	久しぶり!小学校以来だな。	
\\	久しぶり!小学校以来だな。 
\\	そうよね。だから、28年ぶりじゃない。	
\\	そうよね。だから、28年ぶりじゃない。 
\\	なあ、みんな、昔タイムカプセル埋めたよな?掘ってみようぜ!	
\\	なあ、みんな、昔タイムカプセル埋めたよな?掘ってみようぜ! 
\\	あぁ!そうそう、「ブタさん公園」に埋めたよな?	
\\	あぁ!そうそう、「ブタさん公園」に埋めたよな? 
\\	マクドナルド
\\	ちゃんと
\\	腹ごしらえする
\\	売り切れる
\\	ポテト
\\	あれ?ぞうさん公園がない。マクドナルドになってしまったよ。	
\\	あれ?ぞうさん公園がない。マクドナルドになってしまったよ。 
\\	本当だ。とにかくマクドナルドに入ってみよう。	
\\	本当だ。とにかくマクドナルドに入ってみよう。 
\\	いらっしゃいませ。	
\\	いらっしゃいませ。 
\\	あのー、ハンバーガーひとつ下さい。	
\\	あのー、ハンバーガーひとつ下さい。 
\\	はじめ!ちゃんと公園について聞いて。	
\\	はじめ!ちゃんと公園について聞いて。 
\\	まずは腹ごしらえしておこうよ。	
\\	まずは腹ごしらえしておこうよ。 
\\	じゃあ、僕はチーズバーガーで。	
\\	じゃあ、僕はチーズバーガーで。 
\\	申し訳ございません。チーズバーガーは、売り切れてしまいました。	
\\	申し訳ございません。チーズバーガーは、売り切れてしまいました。 
\\	じゃあ、ポテトの
\\	で。	
\\	じゃあ、ポテトの
\\	で。 
\\	いっぱい
\\	店員
\\	工事
\\	裏
\\	残る
\\	お腹
\\	なる
\\	はじめ、まだ食べているのー?	
\\	はじめ、まだ食べているのー? 
\\	(モグモグ)ごちそうさまでした。よし、お腹も一杯になったし、店員さんにゾウさん公園についてきいてみよう。	
\\	(モグモグ)ごちそうさまでした。よし、お腹も一杯になったし、店員さんにゾウさん公園についてきいてみよう。 
\\	はい。	
\\	はい。 
\\	ここのマクドナルドって、昔は公園でしたよね。	
\\	ここのマクドナルドって、昔は公園でしたよね。 
\\	はい、3年前に工事して、公園はなくなってしまいました。	
\\	はい、3年前に工事して、公園はなくなってしまいました。 
\\	そうですか。。。	
\\	そうですか。。。 
\\	じゃあ、イチョウの木もないですよね。	
\\	じゃあ、イチョウの木もないですよね。 
\\	あっ、それだったら、裏に残っていますよ。	
\\	あっ、それだったら、裏に残っていますよ。 
\\	缶
\\	開ける
\\	覚える
\\	下手
\\	字
\\	きけん(危険)
\\	あった!!あのイチョウの木よ!!	
\\	あった!!あのイチョウの木よ!! 
\\	じゃあ、俺たちのタイムカプセルは、この木の下に埋めてあるんだね。	
\\	じゃあ、俺たちのタイムカプセルは、この木の下に埋めてあるんだね。 
\\	(掘る)	
\\	あった!!あったぞ!!	
\\	あった!!あったぞ!! 
\\	じゃあ、缶を開けてみましょう。	
\\	じゃあ、缶を開けてみましょう。 
\\	何を入れたんだ?全然覚えてないよ。	
\\	何を入れたんだ?全然覚えてないよ。 
\\	あ、缶に何か書いてある...。	
\\	あ、缶に何か書いてある...。 
\\	下手な字ね。	
\\	下手な字ね。 
\\	(ドッカーン)	
\\	会計
\\	何人前
\\	ごちそうさま。
\\	1人前
\\	焼肉
\\	気持ち悪い
\\	払う
\\	先輩
\\	半分
\\	マジ
\\	終電
\\	お会計、3万5600円でございます。	
\\	お会計、3万5600円でございます。 
\\	2人で、3万?5600円?うわっマジ?たけーな。	
\\	2人で、3万?5600円?うわっマジ?たけーな。 
\\	先輩、本当にいいんですか?僕、半分払いますよ。	
\\	先輩、本当にいいんですか?僕、半分払いますよ。 
\\	いいよ。	
\\	いいよ。 
\\	そうですか。いつもすみません。ごちそうさまです。	
\\	そうですか。いつもすみません。ごちそうさまです。 
\\	お前、食べ過ぎたんだよ。焼肉15人前も注文して…。どんだけだよ。	
\\	お前、食べ過ぎたんだよ。焼肉15人前も注文して…。どんだけだよ。 
\\	あ、先輩、しゅ、終電が…。	
\\	あ、先輩、しゅ、終電が…。 
\\	しまった! ゆっくり歩き過ぎた!	
\\	しまった! ゆっくり歩き過ぎた! 
\\	タクシー代
\\	貸す
\\	かかります
\\	現金
\\	漫画喫茶
\\	泊まる
\\	寂しい
\\	あのぉ、先輩。大変言いにくいのですが、タクシー代を貸してください。	
\\	あのぉ、先輩。大変言いにくいのですが、タクシー代を貸してください。 
\\	え?俺も、もう現金ねーよ。	
\\	え?俺も、もう現金ねーよ。 
\\	じゃ、僕のお気に入りの漫画喫茶に泊まりましょう。	
\\	じゃ、僕のお気に入りの漫画喫茶に泊まりましょう。 
\\	「泊まりましょう?」俺はタクシーで帰るよ。	
\\	「泊まりましょう?」俺はタクシーで帰るよ。 
\\	え~。先輩。お願いしますぅ。	
\\	え~。先輩。お願いしますぅ。 
\\	嫌だよ。	
\\	嫌だよ。 
\\	え~。一人では入りにくいし、寂しいし…。(泣)	
\\	え~。一人では入りにくいし、寂しいし…。(泣) 
\\	付き合う
\\	漫画
\\	両方とも
\\	黙る
\\	部長
\\	名作
\\	査定
\\	勤務
\\	お勧め
\\	会社
\\	アニメ
\\	先輩は本当に付き合いやすい人ですね。	
\\	先輩は本当に付き合いやすい人ですね。 
\\	うるさいな。	
\\	うるさいな。 
\\	先輩、何を読んでるんですか。	
\\	先輩、何を読んでるんですか。 
\\	ウォールストリートジャーナル?ダメダメ。僕のお勧めは「ナナ」と「猫村さん」です。	
\\	ウォールストリートジャーナル?ダメダメ。僕のお勧めは「ナナ」と「猫村さん」です。 
\\	顔色
\\	ライバル
\\	しめしめ
\\	くだらない
\\	騙す
\\	推薦する
\\	駐在員
\\	ふざける
\\	名作
\\	全巻
\\	徹夜
\\	夕べ
\\	減る
\\	失礼します。	
\\	失礼します。 
\\	おお、星野君。顔色が悪いけれど、大丈夫か?	
\\	おお、星野君。顔色が悪いけれど、大丈夫か? 
\\	いえいえ、実は夕べ、漫画を読んで徹夜してしまいまして。	
\\	いえいえ、実は夕べ、漫画を読んで徹夜してしまいまして。 
\\	何?漫画?	
\\	何?漫画? 
\\	ええ。ナナを全巻と猫村さんを全巻読みました。	
\\	ええ。ナナを全巻と猫村さんを全巻読みました。 
\\	君、ふざけているのか?君をニューヨークの駐在員に推薦するつもりだったが、	
\\	君、ふざけているのか?君をニューヨークの駐在員に推薦するつもりだったが、 
\\	皆さん
\\	今年
\\	会う
\\	結婚する
\\	成田市
\\	千葉県
\\	足立区
\\	東京都
\\	彼女
\\	名前
\\	プロボクサー
\\	出身
\\	緊張する
\\	皆さん、こんにちは。はじめまして。大空晴夜(おおぞらはるや)と申します。	
\\	皆さん、こんにちは。はじめまして。大空晴夜(おおぞらはるや)と申します。 
\\	こんにちは。天道美雨(てんどう みう)と申します。	
\\	こんにちは。天道美雨(てんどう みう)と申します。 
\\	犬
\\	久しぶり。
\\	こちら
\\	彼氏
\\	母
\\	あら
\\	つまらない
\\	ママー。ただいま。	
\\	ママー。ただいま。 
\\	おかえりなさい。美雨ちゃん。あら?	
\\	おかえりなさい。美雨ちゃん。あら? 
\\	ママ、こちら、大空晴夜さん。私の彼氏。	
\\	ママ、こちら、大空晴夜さん。私の彼氏。 
\\	はじめまして。美雨の母です。よろしく。	
\\	はじめまして。美雨の母です。よろしく。 
\\	あ、どうもはじめまして。大空晴夜と申します。	
\\	あ、どうもはじめまして。大空晴夜と申します。 
\\	さあ、お上がりください。どうぞ、どうぞ。	
\\	さあ、お上がりください。どうぞ、どうぞ。 
\\	おじゃまします。	
\\	おじゃまします。 
\\	あら、わざわざすみません。	
\\	あら、わざわざすみません。 
\\	今年
\\	いくつ
\\	才
\\	兄弟
\\	一人
\\	出身
\\	足立区
\\	来る
\\	パパ
\\	晴夜さんは、おいくつですか。	
\\	晴夜さんは、おいくつですか。 
\\	あ、今年、三十一です。	
\\	あ、今年、三十一です。 
\\	あら、三十一才?...ご兄弟がいますか。	
\\	あら、三十一才?...ご兄弟がいますか。 
\\	いえ、母一人。子一人です。	
\\	いえ、母一人。子一人です。 
\\	そうですか。...ご出身は?	
\\	そうですか。...ご出身は? 
\\	足立区です。	
\\	足立区です。 
\\	あ、パパだ。晴夜、パパ、来るよ。	
\\	あ、パパだ。晴夜、パパ、来るよ。 
\\	(ゴクリ)	
\\	(ゴクリ) 
\\	お嬢さん
\\	一緒に
\\	君
\\	それとも
\\	金
\\	欲しい
\\	結婚する
\\	ずっと
\\	ぼく
\\	俺
\\	パパ
\\	(天道家) は、はじめまして。大空晴夜と申します。	
\\	(天道家) は、はじめまして。大空晴夜と申します。 
\\	は?	
\\	は? 
\\	あ、あのぉ。お、お、お父さん!お嬢さんを 下さい!	
\\	あ、あのぉ。お、お、お父さん!お嬢さんを 下さい! 
\\	は?	
\\	は? 
\\	お、お、俺は、美雨と ずっと一緒に いたいです。	
\\	お、お、俺は、美雨と ずっと一緒に いたいです。 
\\	美雨と 結婚したいです。	
\\	美雨と 結婚したいです。 
\\	俺...。美雨...。(フン)	
\\	俺...。美雨...。(フン) 
\\	あ...すみません。僕は 美雨さんと 結婚したいです。	
\\	あ...すみません。僕は 美雨さんと 結婚したいです。 
\\	お、お嬢さんを 下さい。	
\\	お、お嬢さんを 下さい。 
\\	君は...何が 欲しい?金か?家か?あ?	
\\	君は...何が 欲しい?金か?家か?あ? 
\\	それとも うちの会社が 欲しいのか?	
\\	それとも うちの会社が 欲しいのか? 
\\	パパ!	
\\	パパ! 
\\	お子さん
\\	二番目
\\	変
\\	早く
\\	考える
\\	心配
\\	本当に
\\	独身
\\	息子
\\	娘
\\	孫
\\	実は
\\	天道社長、元気がありませんね。どうしましたか。	
\\	天道社長、元気がありませんね。どうしましたか。 
\\	いえ。実は、昨日 娘の彼氏と 会いました。	
\\	いえ。実は、昨日 娘の彼氏と 会いました。 
\\	おお、いいですね。	
\\	おお、いいですね。 
\\	でもねぇ・・・。変な男ですよ。好きじゃないですね。本当に心配です。	
\\	でもねぇ・・・。変な男ですよ。好きじゃないですね。本当に心配です。 
\\	ええ。息子が 二人、娘が 一人 います。	
\\	ええ。息子が 二人、娘が 一人 います。 
\\	そうですか。(ため息) 早く、孫の顔が 見たいですが、あの 二人の結婚は 考えたくないですね。	
\\	そうですか。(ため息) 早く、孫の顔が 見たいですが、あの 二人の結婚は 考えたくないですね。 
\\	じゃあ、うちの二番目の息子は どうですか。独身ですよ。	
\\	じゃあ、うちの二番目の息子は どうですか。独身ですよ。 
\\	山川先生の息子さんですか?	
\\	山川先生の息子さんですか? 
\\	有名(な)
\\	呼ぶ
\\	下手
\\	上手
\\	プレーする
\\	教える
\\	苦手
\\	パター
\\	週末
\\	毎週
\\	歩く
\\	ゴルフコース
\\	ゴルフ
\\	プロゴルファー
\\	サラリーマン
\\	先生の 息子さん?あの、有名な プロゴルファーの山川ひょうさん?	
\\	先生の 息子さん?あの、有名な プロゴルファーの山川ひょうさん? 
\\	いや、あれは一番目の息子です。次男はサラリーマンです。	
\\	いや、あれは一番目の息子です。次男はサラリーマンです。 
\\	ええ。ゴルフコースを 歩くのが 大好きです。	
\\	ええ。ゴルフコースを 歩くのが 大好きです。 
\\	毎週末、ゴルフを しますが、ゴルフは 難しいですよ。	
\\	毎週末、ゴルフを しますが、ゴルフは 難しいですよ。 
\\	元気(な)
\\	ママ
\\	毎日
\\	忙しい
\\	代わる
\\	今週
\\	もしもし、ママ?	
\\	もしもし、ママ? 
\\	あらー、美雨ちゃん。元気?	
\\	あらー、美雨ちゃん。元気? 
\\	うん。まぁ、元気。	
\\	うん。まぁ、元気。 
\\	ママは 元気そうね。パパは 元気?	
\\	ママは 元気そうね。パパは 元気? 
\\	お父さんは、忙しそうだけれど、元気よ。	
\\	お父さんは、忙しそうだけれど、元気よ。 
\\	お父さんと 代わりましょうか?	
\\	お父さんと 代わりましょうか? 
\\	え?パパいるの?	
\\	え?パパいるの? 
\\	ええ。ここにいるわよ。代わりましょうか。	
\\	ええ。ここにいるわよ。代わりましょうか。 
\\	あ、ううん。いい。	
\\	あ、ううん。いい。 
\\	飲み物
\\	ピラニア
\\	刺身
\\	手伝う
\\	水
\\	ミルク
\\	砂糖
\\	何でも
\\	紅茶
\\	えさ
\\	飲み物は何にしますか?	
\\	飲み物は何にしますか? 
\\	あ、、、な、何でもいいです。	
\\	あ、、、な、何でもいいです。 
\\	じゃ、紅茶を飲みませんか。	
\\	じゃ、紅茶を飲みませんか。 
\\	お砂糖とミルクは?	
\\	お砂糖とミルクは? 
\\	あ、、、、どちらでもいいです。	
\\	あ、、、、どちらでもいいです。 
\\	(プッ)晴夜(はるや)さん...。飲みたいの?飲みたくないの?	
\\	(プッ)晴夜(はるや)さん...。飲みたいの?飲みたくないの? 
\\	あ、す、す、すみません。じゃ、水を下さい。水が飲みたいです。	
\\	あ、す、す、すみません。じゃ、水を下さい。水が飲みたいです。 
\\	あ、手伝いましょうか。・・・あ、お刺身ですか。	
\\	あ、手伝いましょうか。・・・あ、お刺身ですか。 
\\	そう、ピラニアのえさ。	
\\	そう、ピラニアのえさ。 
\\	掃除
\\	入れる
\\	バケツ
\\	スイッチ
\\	つける
\\	ホース
\\	開ける
\\	ふた
\\	気をつける
\\	水槽
\\	ネット
\\	手袋
\\	捕まえる
\\	ピラニア?	
\\	ピラニア? 
\\	じゃあ、晴夜さん。掃除をお願いします。	
\\	じゃあ、晴夜さん。掃除をお願いします。 
\\	はい、手袋とネット。	
\\	はい、手袋とネット。 
\\	え?	
\\	え? 
\\	まず、その手袋をつけてください。	
\\	まず、その手袋をつけてください。 
\\	一緒に、ピラニアの水槽の掃除をしましょう。楽しいですよ。でも、気をつけてください。	
\\	一緒に、ピラニアの水槽の掃除をしましょう。楽しいですよ。でも、気をつけてください。 
\\	ええ?	
\\	ええ? 
\\	これ、家のピラニアちゃん。	
\\	これ、家のピラニアちゃん。 
\\	はい、水槽のふたを開けてください。	
\\	はい、水槽のふたを開けてください。 
\\	次に、ホースを入れてください。	
\\	次に、ホースを入れてください。 
\\	スイッチを入れてください。	
\\	スイッチを入れてください。 
\\	それから、このバケツにピラニアを入れてください。気をつけて。	
\\	それから、このバケツにピラニアを入れてください。気をつけて。 
\\	バケツに?ピラニアを?(ゴクッ)	
\\	バケツに?ピラニアを?(ゴクッ) 
\\	さあ、早く、ピラニアを捕まえて。	
\\	さあ、早く、ピラニアを捕まえて。 
\\	冷たい
\\	貸す
\\	トイレ
\\	二十分
\\	ゴルフ場
\\	飲む
\\	休む
\\	終わる
\\	(ふう)終わりました。	
\\	(ふう)終わりました。 
\\	ありがとう、晴夜さん。休んでください。	
\\	ありがとう、晴夜さん。休んでください。 
\\	はい、冷たいお茶。どうぞ。飲んでくださいね。	
\\	はい、冷たいお茶。どうぞ。飲んでくださいね。 
\\	あ、ありがとうございます。(ゴクゴクゴクゴク)	
\\	あ、ありがとうございます。(ゴクゴクゴクゴク) 
\\	はい。天道でございます。	
\\	はい。天道でございます。 
\\	あら、パパ。	
\\	あら、パパ。 
\\	コンビニ
\\	妻
\\	とめる
\\	上がる
\\	時間
\\	メートル
\\	信号
\\	まっすぐ
\\	交差点
\\	次
\\	曲がる
\\	角
\\	紹介する
\\	次の交差点を 左に 曲がってください。	
\\	次の交差点を 左に 曲がってください。 
\\	この、交差点を 左ですね。	
\\	この、交差点を 左ですね。 
\\	はい。少し、まっすぐ 行ってください。	
\\	はい。少し、まっすぐ 行ってください。 
\\	ニ番目の信号を 右に 曲がってください。	
\\	ニ番目の信号を 右に 曲がってください。 
\\	この信号を 右・・・。	
\\	この信号を 右・・・。 
\\	で、コンビニの角を 左に 曲がってください。	
\\	で、コンビニの角を 左に 曲がってください。 
\\	三百メートルくらい まっすぐ 行ってください。	
\\	三百メートルくらい まっすぐ 行ってください。 
\\	あ、ここです。ここで とめてください。	
\\	あ、ここです。ここで とめてください。 
\\	あ、雪男(ゆきお)君、時間ありますか。	
\\	あ、雪男(ゆきお)君、時間ありますか。 
\\	許す
\\	開ける
\\	出る
\\	ベランダ
\\	着る
\\	上着
\\	持つ
\\	くつ
\\	待つ
\\	困る
\\	僕達
\\	説明
\\	ただいまぁー。	
\\	ただいまぁー。 
\\	おと...、おと...、おと...、おとおとおと、お父~さん。僕達の結婚を許してください。	
\\	おと...、おと...、おと...、おとおとおと、お父~さん。僕達の結婚を許してください。 
\\	何だ、君は?	
\\	何だ、君は? 
\\	天道社長~、天道社長~。	
\\	天道社長~、天道社長~。 
\\	雪男君、ちょ、ちょっと待ってくださ~い。	
\\	雪男君、ちょ、ちょっと待ってくださ~い。 
\\	君ィ、困るよ。	
\\	君ィ、困るよ。 
\\	はい、これ、君。この靴を持って。ちょっとこっちに来て。早く、こっち来て。	
\\	はい、これ、君。この靴を持って。ちょっとこっちに来て。早く、こっち来て。 
\\	病院
\\	待つ
\\	ちょっと
\\	テレビ
\\	リビング
\\	いる
\\	こちらこそ
\\	主人
\\	君
\\	妻
\\	いつも
\\	お客様
\\	お帰りなさい。あら、お客様?	
\\	お帰りなさい。あら、お客様? 
\\	こちらは、山川先生の 息子さん。山川雪男(やまかわゆきお)君だ。	
\\	こちらは、山川先生の 息子さん。山川雪男(やまかわゆきお)君だ。 
\\	雪男君、これは、私の妻のきりです。	
\\	雪男君、これは、私の妻のきりです。 
\\	ああ、成田病院の山川先生の息子さん?天道きり と申します。	
\\	ああ、成田病院の山川先生の息子さん?天道きり と申します。 
\\	いつも、主人が お世話になっています。	
\\	いつも、主人が お世話になっています。 
\\	はじめまして。山川雪男と 申します。	
\\	はじめまして。山川雪男と 申します。 
\\	今夜
\\	社会学
\\	社会
\\	専攻
\\	勤める
\\	コンサルティング
\\	通う
\\	大学院
\\	住む
\\	帰る
\\	男
\\	お客さん
\\	会社
\\	そうそう、あなた、今夜、晴夜さん...来ているのよ。	
\\	そうそう、あなた、今夜、晴夜さん...来ているのよ。 
\\	あ、お客さんですか。	
\\	あ、お客さんですか。 
\\	え?え?あ、いや。あの男はもう帰りました。	
\\	え?え?あ、いや。あの男はもう帰りました。 
\\	あ、パパ、お帰りなさい。	
\\	あ、パパ、お帰りなさい。 
\\	雪男君、うちの娘の 美雨です。東京に 住んでいます。東京で 大学院に 通っています。	
\\	雪男君、うちの娘の 美雨です。東京に 住んでいます。東京で 大学院に 通っています。 
\\	あ、どうも。はじめまして。山川雪男と 申します。	
\\	あ、どうも。はじめまして。山川雪男と 申します。 
\\	コンサルティング会社に 勤めています。美雨さんの 専攻は 何ですか。	
\\	コンサルティング会社に 勤めています。美雨さんの 専攻は 何ですか。 
\\	飲み物
\\	ベランダ
\\	灰皿
\\	タバコ
\\	真面目
\\	ビール
\\	帰る
\\	車
\\	もらう
\\	すう
\\	雪男さん、お飲み物は何にしますか。	
\\	雪男さん、お飲み物は何にしますか。 
\\	あ、じゃぁ、冷たいビールを もらってもいいですか。	
\\	あ、じゃぁ、冷たいビールを もらってもいいですか。 
\\	え、でも、車で帰りますよね。 じゃ、ビールを 飲んでは いけませんよね。	
\\	え、でも、車で帰りますよね。 じゃ、ビールを 飲んでは いけませんよね。 
\\	あ、ははは。そう。そうですよね。いけませんね。ははは。美雨ちゃんは 真面目だな。じゃ、タバコを すってもいいですか。	
\\	あ、ははは。そう。そうですよね。いけませんね。ははは。美雨ちゃんは 真面目だな。じゃ、タバコを すってもいいですか。 
\\	あ、どうぞ。どうぞ。おーい。灰皿ある?	
\\	あ、どうぞ。どうぞ。おーい。灰皿ある? 
\\	すみませんが、タバコは ちょっと...。	
\\	すみませんが、タバコは ちょっと...。 
\\	そこのベランダで すってください。	
\\	そこのベランダで すってください。 
\\	だめ(な)
\\	人
\\	リビング
\\	やっぱり
\\	渋滞
\\	道
\\	大会
\\	公園
\\	近く
\\	花火
\\	音
\\	失礼します。
\\	もしかして
\\	じゃ、うちへ 帰って タバコを すいますね。失礼します。	
\\	じゃ、うちへ 帰って タバコを すいますね。失礼します。 
\\	(パンパン!ドドーン!)	
\\	(パンパン!ドドーン!) 
\\	何の音ですか?	
\\	何の音ですか? 
\\	ああ、花火の音ですよ。	
\\	ああ、花火の音ですよ。 
\\	今夜、近くの公園で 花火大会が あって、道は 大渋滞ですよ。	
\\	今夜、近くの公園で 花火大会が あって、道は 大渋滞ですよ。 
\\	さぁ、ベランダに行って、花火を 見ましょう。ほら、美雨も一緒に来て。	
\\	さぁ、ベランダに行って、花火を 見ましょう。ほら、美雨も一緒に来て。 
\\	(トントントン)	
\\	(トントントン) 
\\	あ...ベランダは だめだ。やっぱり、リビングで見ましょう。	
\\	あ...ベランダは だめだ。やっぱり、リビングで見ましょう。 
\\	パパ?ベランダに 何が あるの?	
\\	パパ?ベランダに 何が あるの? 
\\	な、何もいない。	
\\	な、何もいない。 
\\	台所
\\	呼ぶ
\\	救急車
\\	警察
\\	探す
\\	連れてくる
\\	母さん
\\	持ってくる
\\	おつまみ
\\	強盗
\\	晴夜!あれ?いない...。	
\\	晴夜!あれ?いない...。 
\\	じゃ、ベランダで花火を見ましょう。	
\\	じゃ、ベランダで花火を見ましょう。 
\\	美雨、台所から、飲み物とおつまみを持って来て。	
\\	美雨、台所から、飲み物とおつまみを持って来て。 
\\	はい。	
\\	はい。 
\\	それから、灰皿を持って来て。	
\\	それから、灰皿を持って来て。 
\\	あ、あと、母さんも連れて来て。	
\\	あ、あと、母さんも連れて来て。 
\\	救急車
\\	喧嘩
\\	西口
\\	事件
\\	事故
\\	場所
\\	教える
\\	あなた
\\	落ち着く
\\	よく
\\	追いかける
\\	犯人
\\	倒れる
\\	キッチン
\\	火事
\\	ナイフ
\\	ええっと、救急車119	
\\	ええっと、救急車119 
\\	はい。119番です。火事ですか。救急ですか。	
\\	はい。119番です。火事ですか。救急ですか。 
\\	キッチンに行って、ママが倒れていて…。	
\\	キッチンに行って、ママが倒れていて…。 
\\	晴夜は犯人を追いかけて、 よく分かりません。	
\\	晴夜は犯人を追いかけて、 よく分かりません。 
\\	落ち着いてください。あなたのお名前を 教えてください。	
\\	落ち着いてください。あなたのお名前を 教えてください。 
\\	天道 美雨(てんどう みう)です。	
\\	天道 美雨(てんどう みう)です。 
\\	場所は どこですか。	
\\	場所は どこですか。 
\\	住所が 分かりますか。	
\\	住所が 分かりますか。 
\\	成田市さくら4-5-6	
\\	成田市さくら4-5-6 
\\	(ボーナス)	
\\	(ボーナス) 
\\	降りる
\\	メモ
\\	一回
\\	乗る
\\	京成線
\\	日暮里
\\	山手線
\\	京成成田駅
\\	乗り換える
\\	まず
\\	えっと、うーん…。	
\\	えっと、うーん…。 
\\	どうしましたか。大丈夫ですか。どこへ行きますか。	
\\	どうしましたか。大丈夫ですか。どこへ行きますか。 
\\	あ、あのぉ…京成成田駅へ。	
\\	あ、あのぉ…京成成田駅へ。 
\\	ああ、京成成田駅ですね。山手線で 日暮里に 行って、日暮里で 京成線に 乗ってください。	
\\	ああ、京成成田駅ですね。山手線で 日暮里に 行って、日暮里で 京成線に 乗ってください。 
\\	はぁ…。すみませんが、もう一回、ゆっくり言ってください。メモをしますから。	
\\	はぁ…。すみませんが、もう一回、ゆっくり言ってください。メモをしますから。 
\\	まず、山手線に乗って、日暮里に行ってください。	
\\	まず、山手線に乗って、日暮里に行ってください。 
\\	はい。はい。山手線で、日暮里まで行く …。	
\\	はい。はい。山手線で、日暮里まで行く …。 
\\	それから、日暮里で山手線を降りて、京成線に乗り換えてください。	
\\	それから、日暮里で山手線を降りて、京成線に乗り換えてください。 
\\	日暮里で京成線に乗り換える…。ありがとうございました。	
\\	日暮里で京成線に乗り換える…。ありがとうございました。 
\\	駅
\\	京成スカイライナー
\\	普通
\\	どちら
\\	安い
\\	ずっと
\\	速い
\\	駅員
\\	老人
\\	あのぉ、すみません。これは 京成線ですか。	
\\	あのぉ、すみません。これは 京成線ですか。 
\\	はい、京成スカイライナーです。	
\\	はい、京成スカイライナーです。 
\\	京成スカイライナー?これは成田へ行きますか。	
\\	京成スカイライナー?これは成田へ行きますか。 
\\	ええ、行きますよ。	
\\	ええ、行きますよ。 
\\	普通の京成線と 京成スカイライナーと どちらのほうが 安いですか。	
\\	普通の京成線と 京成スカイライナーと どちらのほうが 安いですか。 
\\	京成線のほうが ずっと 安いですよ。京成線は 750円、京成スカイライナーは 1670円です。	
\\	京成線のほうが ずっと 安いですよ。京成線は 750円、京成スカイライナーは 1670円です。 
\\	どちらのほうが速いですか。	
\\	どちらのほうが速いですか。 
\\	もちろん、京成線より京成スカイライナーのほうが速いですよ。	
\\	もちろん、京成線より京成スカイライナーのほうが速いですよ。 
\\	乗りますか。乗りませんか。	
\\	乗りますか。乗りませんか。 
\\	あ、乗ります!乗ります!	
\\	あ、乗ります!乗ります! 
\\	気分
\\	お陰様で
\\	でも
\\	心配
\\	旦那
\\	ペット
\\	退院
\\	情けない
\\	こんにちは、天道さん。今日は、昨日より 元気そうですね。気分はどうですか。	
\\	こんにちは、天道さん。今日は、昨日より 元気そうですね。気分はどうですか。 
\\	あ、山川先生、お陰様で。昨日より、いいです。	
\\	あ、山川先生、お陰様で。昨日より、いいです。 
\\	でも、ピラニアと犬が心配です。	
\\	でも、ピラニアと犬が心配です。 
\\	ははは。旦那さんよりペットのほうが心配ですか。	
\\	ははは。旦那さんよりペットのほうが心配ですか。 
\\	ふふふ。ペットは雷三さんよりかわいいですから。	
\\	ふふふ。ペットは雷三さんよりかわいいですから。 
\\	ま、そうですね。(ははは)天道さん、来週、退院してもいいですよ。	
\\	ま、そうですね。(ははは)天道さん、来週、退院してもいいですよ。 
\\	本当ですか。ありがとうございます。あ、息子さんはどうですか。	
\\	本当ですか。ありがとうございます。あ、息子さんはどうですか。 
\\	病室
\\	大切
\\	幸せ
\\	関係
\\	大事
\\	愛
\\	強い
\\	ボーイフレンド
\\	おじいちゃん
\\	体調
\\	奥
\\	上がる
\\	二階
\\	エレベーター
\\	号室
\\	わし
\\	あのぉ、すみません。天道きりの病室はどこですか。	
\\	あのぉ、すみません。天道きりの病室はどこですか。 
\\	天道...きりさん?ああ、天道さんの病室は205号室です。	
\\	天道...きりさん?ああ、天道さんの病室は205号室です。 
\\	そのエレベーターで二階に上がってください。205号室は一番奥です。	
\\	そのエレベーターで二階に上がってください。205号室は一番奥です。 
\\	2階の一番奥、一番奥…。ここか…。	
\\	2階の一番奥、一番奥…。ここか…。 
\\	(コンコン)きり?どうだ?体調は?	
\\	(コンコン)きり?どうだ?体調は? 
\\	あ、お父さん。	
\\	あ、お父さん。 
\\	おじいちゃん。	
\\	おじいちゃん。 
\\	おお。美雨。いいボーイフレンドを持っているね。強い男はいい。わしはスポーツの中でボクシングが一番好きだ。はっはっは。	
\\	おお。美雨。いいボーイフレンドを持っているね。強い男はいい。わしはスポーツの中でボクシングが一番好きだ。はっはっは。 
\\	でも、パパは。	
\\	でも、パパは。 
\\	愛が一番大事だよ、美雨。パパは関係ない。お金も関係ない。美雨の幸せが一番大切だ。ははは。	
\\	愛が一番大事だよ、美雨。パパは関係ない。お金も関係ない。美雨の幸せが一番大切だ。ははは。 
\\	すう
\\	倒れる
\\	血
\\	言う
\\	困る
\\	本当に
\\	ふざける
\\	耳
\\	出す
\\	かっこわるい
\\	お酒
\\	クローゼット
\\	うるさい
\\	打つ
\\	山川さん!ここは、病院ですよ。タバコをすわないでください。	
\\	山川さん!ここは、病院ですよ。タバコをすわないでください。 
\\	うるさいなぁ。あ、クローゼットを開けないでくださいよ。	
\\	うるさいなぁ。あ、クローゼットを開けないでくださいよ。 
\\	…山川さん…。これは何ですか。お酒ですね?お酒を飲まないでください。	
\\	…山川さん…。これは何ですか。お酒ですね?お酒を飲まないでください。 
\\	大きい声を出さないでくださいよ。耳が痛いです。お~、イタイイタイ。	
\\	大きい声を出さないでくださいよ。耳が痛いです。お~、イタイイタイ。 
\\	ふざけないでください。	
\\	ふざけないでください。 
\\	山川先生の息子さん、本当に困ります。	
\\	山川先生の息子さん、本当に困ります。 
\\	ああ...言わないでくださいね。	
\\	ああ...言わないでくださいね。 
\\	山川先生の息子さんは血を見て、倒れて、頭を打って、天道きりさんと一緒に救急車でここに来ました。	
\\	山川先生の息子さんは血を見て、倒れて、頭を打って、天道きりさんと一緒に救急車でここに来ました。 
\\	わざわざ
\\	正直
\\	感謝
\\	反対
\\	氷
\\	ウイスキー
\\	うまい
\\	駅前
\\	夕飯
\\	帰る
\\	上がる
\\	別
\\	あ、お父さん。わざわざありがとうございます。	
\\	あ、お父さん。わざわざありがとうございます。 
\\	さぁ、どうぞ。上がってください。あれ、美雨は。	
\\	さぁ、どうぞ。上がってください。あれ、美雨は。 
\\	ああ、東京に帰った。	
\\	ああ、東京に帰った。 
\\	あ・・・そうですか。	
\\	あ・・・そうですか。 
\\	あ・・・ お父さん、夕飯は?	
\\	あ・・・ お父さん、夕飯は? 
\\	美雨と成田駅前で食べた。	
\\	美雨と成田駅前で食べた。 
\\	あ・・・そうですか。	
\\	あ・・・そうですか。 
\\	あ・・・そうだ。お父さん、うまいウイスキーがありますが、飲みますか。	
\\	あ・・・そうだ。お父さん、うまいウイスキーがありますが、飲みますか。 
\\	ああ。飲む。	
\\	ああ。飲む。 
\\	砂糖
\\	さっき
\\	祝い
\\	すし屋
\\	上着
\\	もたもた
\\	許す
\\	ご飯
\\	アルバム
\\	二階
\\	入れる
\\	みんなで
\\	はい、コーヒー。どうぞ。	
\\	はい、コーヒー。どうぞ。 
\\	晴夜さんはいつもコーヒーにお砂糖とミルクを入れて飲みますか。	
\\	晴夜さんはいつもコーヒーにお砂糖とミルクを入れて飲みますか。 
\\	いえ。入れないで飲みます。いただきます。アチッ。	
\\	いえ。入れないで飲みます。いただきます。アチッ。 
\\	…あ、あのぉ…お父さんは…。	
\\	…あ、あのぉ…お父さんは…。 
\\	二階にいます。さっきから、何も言わないで、美雨のアルバムを見ているわ。	
\\	二階にいます。さっきから、何も言わないで、美雨のアルバムを見ているわ。 
\\	あ、晴夜さんは、ご飯を食べてきましたか。	
\\	あ、晴夜さんは、ご飯を食べてきましたか。 
\\	あ、いえ。食べないで来ました。(グウー)	
\\	あ、いえ。食べないで来ました。(グウー) 
\\	(ガラガラ)	
\\	(ガラガラ) 
\\	観光
\\	ごろ
\\	一時間
\\	着く
\\	鎌倉
\\	明月院
\\	円覚寺
\\	北鎌倉
\\	まず
\\	言う
\\	予定
\\	ガイド
\\	近く
\\	皆様、おはよう ございます。	
\\	皆様、おはよう ございます。 
\\	私は 観光ガイドの 下山新(しもやま・しん)と申します。どうぞ よろしく おねがい いたします。	
\\	私は 観光ガイドの 下山新(しもやま・しん)と申します。どうぞ よろしく おねがい いたします。 
\\	では、今日の予定を 言いますね。	
\\	では、今日の予定を 言いますね。 
\\	まず、このバスで 北鎌倉に 行きます。	
\\	まず、このバスで 北鎌倉に 行きます。 
\\	北鎌倉で 円覚寺(えんがくじ)と 明月院(めいげついん)を 観光します。	
\\	北鎌倉で 円覚寺(えんがくじ)と 明月院(めいげついん)を 観光します。 
\\	それから、電車で 鎌倉駅に 行きます。	
\\	それから、電車で 鎌倉駅に 行きます。 
\\	ねぇ、ガイドさん、何時に 北鎌倉に 着く?	
\\	ねぇ、ガイドさん、何時に 北鎌倉に 着く? 
\\	大仏
\\	六時間
\\	ホテル
\\	ですから
\\	長谷寺
\\	位
\\	歩き
\\	たくさん
\\	残念
\\	自由時間
\\	大体
\\	ガイドさん、今日、大仏さんは 見ない?	
\\	ガイドさん、今日、大仏さんは 見ない? 
\\	はい。残念ですが、鎌倉大仏へは 行きません。でも...	
\\	はい。残念ですが、鎌倉大仏へは 行きません。でも... 
\\	ええ?大仏さんに行かない。	
\\	ええ?大仏さんに行かない。 
\\	はい。でも、明日、皆さんは 自由時間が たくさん ありますので…	
\\	はい。でも、明日、皆さんは 自由時間が たくさん ありますので… 
\\	あ、そう。自由時間ある?	
\\	あ、そう。自由時間ある? 
\\	はい。朝九時から 午後三時十五分まで 自由時間です。	
\\	はい。朝九時から 午後三時十五分まで 自由時間です。 
\\	ですから...六時間位あります。	
\\	ですから...六時間位あります。 
\\	皆さんのホテルから 大仏まで 歩きで大体15分ですので…	
\\	皆さんのホテルから 大仏まで 歩きで大体15分ですので… 
\\	じゃあ、長谷寺は?今日、長谷寺に行く?	
\\	じゃあ、長谷寺は?今日、長谷寺に行く? 
\\	出発
\\	ロビー
\\	待つ
\\	いってらっしゃい。
\\	デジカメ
\\	前
\\	教える
\\	地図
\\	気をつける
\\	呼ぶ
\\	集まる
\\	使う
\\	皆さん、おはようございます。今から、自由時間です。	
\\	皆さん、おはようございます。今から、自由時間です。 
\\	三時半に ホテルを 出発しますから、3時15分に ロビーに 集まってください。	
\\	三時半に ホテルを 出発しますから、3時15分に ロビーに 集まってください。 
\\	では、気をつけて。いってらっしゃい!	
\\	では、気をつけて。いってらっしゃい! 
\\	あのぉ、ガイドさん。地図の見方を 教えてください。	
\\	あのぉ、ガイドさん。地図の見方を 教えてください。 
\\	あ、はいはい。ホテルは これですね。それで...	
\\	あ、はいはい。ホテルは これですね。それで... 
\\	ちょっと、タクシー、呼んで。	
\\	ちょっと、タクシー、呼んで。 
\\	あ、タクシーは 駅の前に います。	
\\	あ、タクシーは 駅の前に います。 
\\	リヤカー
\\	人力車
\\	速い
\\	気持ち
\\	乗る
\\	危ない
\\	安全(な)
\\	それに
\\	おばあちゃん
\\	おばあちゃん、どう?のる?10分、二千円!	
\\	おばあちゃん、どう?のる?10分、二千円! 
\\	何ですか。リヤカーですか。	
\\	何ですか。リヤカーですか。 
\\	知らないの?リヤカーじゃないよ。人力車だよ。速くて気持ちいいよ。	
\\	知らないの?リヤカーじゃないよ。人力車だよ。速くて気持ちいいよ。 
\\	面白そう...。おじいさん、乗りませんか。	
\\	面白そう...。おじいさん、乗りませんか。 
\\	危なくないですか。	
\\	危なくないですか。 
\\	全然、危なくない。安全だよ。	
\\	全然、危なくない。安全だよ。 
\\	それに、10分 二千円は 高くない。	
\\	それに、10分 二千円は 高くない。 
\\	そうですよ。高くありませんよ。ねぇ、おじいさん、だめですか。	
\\	そうですよ。高くありませんよ。ねぇ、おじいさん、だめですか。 
\\	もう
\\	孫
\\	育つ
\\	生まれる
\\	ブラジル
\\	日系
\\	初めて
\\	歩く
\\	様
\\	疲れる
\\	よく
\\	起きる
\\	着く
\\	実は
\\	もう、3時10分...。	
\\	もう、3時10分...。 
\\	おじいさん、おじいさん。ホテルに 着きましたよ。起きてください。	
\\	おじいさん、おじいさん。ホテルに 着きましたよ。起きてください。 
\\	ん?もう着いた?そうか...。よく 寝た。	
\\	ん?もう着いた?そうか...。よく 寝た。 
\\	疲れましたね。	
\\	疲れましたね。 
\\	ああ、疲れた、疲れた。ああ、ガイドさん。	
\\	ああ、疲れた、疲れた。ああ、ガイドさん。 
\\	あ、中川様。お帰りなさい。	
\\	あ、中川様。お帰りなさい。 
\\	ふう…。今日は よく 歩きましたよ。	
\\	ふう…。今日は よく 歩きましたよ。 
\\	そうですか。鎌倉は 初めてですか。	
\\	そうですか。鎌倉は 初めてですか。 
\\	はい。私達は、日系ブラジル人です。	
\\	はい。私達は、日系ブラジル人です。 
\\	出発
\\	質問
\\	たくさん
\\	買い物
\\	土産
\\	忘れ物
\\	乗る
\\	バス
\\	時間
\\	申し訳ありません。
\\	皆さん、出発時間です。バスに乗ってください。	
\\	皆さん、出発時間です。バスに乗ってください。 
\\	忘れ物をしないでくださいね。	
\\	忘れ物をしないでくださいね。 
\\	よいしょ。よいしょ。	
\\	よいしょ。よいしょ。 
\\	あ、上田さん、お帰りなさい。	
\\	あ、上田さん、お帰りなさい。 
\\	たくさん お土産を 買いましたね。	
\\	たくさん お土産を 買いましたね。 
\\	うん...。ずっと、買い物。	
\\	うん...。ずっと、買い物。 
\\	昨日、たくさん 質問したけど、観光しなかった。時間がなかった。	
\\	昨日、たくさん 質問したけど、観光しなかった。時間がなかった。 
\\	ええ?時間が ありませんでしたか?申し訳ありません。鎌倉大仏は?	
\\	ええ?時間が ありませんでしたか?申し訳ありません。鎌倉大仏は? 
\\	見なかった…。	
\\	見なかった…。 
\\	長谷寺は?	
\\	長谷寺は? 
\\	お帰り
\\	連れて行く
\\	クラブ
\\	六本木
\\	話す
\\	多い
\\	年寄り
\\	ツアー
\\	気持ち
\\	寒い
\\	楽しい
\\	なかなか
\\	旅行
\\	ただいま
\\	若い
\\	(ピンポーン)	
\\	(ピンポーン) 
\\	おじいちゃん、おばあちゃん、お帰り。	
\\	おじいちゃん、おばあちゃん、お帰り。 
\\	ただいま、マルシアちゃん。	
\\	ただいま、マルシアちゃん。 
\\	鎌倉旅行は どうだった?	
\\	鎌倉旅行は どうだった? 
\\	なかなか 楽しかったですよ。ねぇ、おじいさん。	
\\	なかなか 楽しかったですよ。ねぇ、おじいさん。 
\\	ああ。人力車は 楽しかったね。また 乗りたいね。	
\\	ああ。人力車は 楽しかったね。また 乗りたいね。 
\\	大丈夫だった?寒くなかった?	
\\	大丈夫だった?寒くなかった? 
\\	寒かったけど、なかなか 気持ちよかった。	
\\	寒かったけど、なかなか 気持ちよかった。 
\\	でも、ツアーのお客さんは 年寄りが 多かったですね。	
\\	でも、ツアーのお客さんは 年寄りが 多かったですね。 
\\	そうそう。もっと若い人と 話したかった。	
\\	そうそう。もっと若い人と 話したかった。 
\\	ただいま。
\\	誕生会
\\	みんなで
\\	デート
\\	だから
\\	知る
\\	誕生日
\\	普通
\\	大変
\\	寂しい
\\	毎日
\\	先輩
\\	お帰りなさい。
\\	上司
\\	ただいま帰りました。	
\\	ただいま帰りました。 
\\	あ、下山先輩!お帰りなさーい。	
\\	あ、下山先輩!お帰りなさーい。 
\\	先輩がいなかったから、絵理花、毎日 寂しかったですぅ。	
\\	先輩がいなかったから、絵理花、毎日 寂しかったですぅ。 
\\	鎌倉ツアーは大変でしたか。	
\\	鎌倉ツアーは大変でしたか。 
\\	大変じゃなかったよ。まぁ、普通だった。	
\\	大変じゃなかったよ。まぁ、普通だった。 
\\	良かったですね。	
\\	良かったですね。 
\\	ねぇ、先輩、私の誕生日を知っています?あ・し・た・ですよ。	
\\	ねぇ、先輩、私の誕生日を知っています?あ・し・た・ですよ。 
\\	だから...明日デートしましょう。	
\\	だから...明日デートしましょう。 
\\	結婚式
\\	似合う
\\	着物
\\	若い
\\	上品(な)
\\	エレガント(な)
\\	大変
\\	試着
\\	パリ
\\	ちょうど
\\	ございます
\\	もちろん
\\	ドレス
\\	用
\\	旦那
\\	いらっしゃいませ。	
\\	いらっしゃいませ。 
\\	結婚式用のドレスはあります?	
\\	結婚式用のドレスはあります? 
\\	結婚式用のドレスですかぁ?	
\\	結婚式用のドレスですかぁ? 
\\	はい、もちろん ございます。こちらへどうぞ。	
\\	はい、もちろん ございます。こちらへどうぞ。 
\\	このドレスはいかがですか。	
\\	このドレスはいかがですか。 
\\	昨日、ちょうどパリから入ってきました。	
\\	昨日、ちょうどパリから入ってきました。 
\\	あら、すてき。	
\\	あら、すてき。 
\\	試着してもいいですか。	
\\	試着してもいいですか。 
\\	あなた
\\	サイン
\\	悲しい
\\	どうして
\\	出る
\\	人気
\\	はく
\\	軽い
\\	買う
\\	くつ
\\	書く
\\	ねぇ、あなた。あなたのくつも 買いましょうか。	
\\	ねぇ、あなた。あなたのくつも 買いましょうか。 
\\	結婚式用のくつ。	
\\	結婚式用のくつ。 
\\	では、このくつは いかがですか。	
\\	では、このくつは いかがですか。 
\\	このくつは 軽くて、はきやすいですよ。	
\\	このくつは 軽くて、はきやすいですよ。 
\\	一番人気があります。	
\\	一番人気があります。 
\\	ふん。...結婚式には 出ない。	
\\	ふん。...結婚式には 出ない。 
\\	あららららら。どうしてですか。それは 悲しい。悲しすぎます。	
\\	あららららら。どうしてですか。それは 悲しい。悲しすぎます。 
\\	君は、うるさすぎるよ。	
\\	君は、うるさすぎるよ。 
\\	とうとう
\\	頼りない
\\	弱い
\\	最初
\\	思う
\\	みがく
\\	出席
\\	晴れる
\\	降る
\\	雨
\\	心配
\\	とうとう、明日は 結婚式ね。	
\\	とうとう、明日は 結婚式ね。 
\\	うん。ねぇ、ママ。明日 雨が降ると思う?	
\\	うん。ねぇ、ママ。明日 雨が降ると思う? 
\\	う~ん。ママは晴れると思う。	
\\	う~ん。ママは晴れると思う。 
\\	大丈夫。大丈夫。ママは晴れ女。	
\\	大丈夫。大丈夫。ママは晴れ女。 
\\	パパも晴れ男よ。	
\\	パパも晴れ男よ。 
\\	でも、私は雨女よ。	
\\	でも、私は雨女よ。 
\\	晴夜も雨男。	
\\	晴夜も雨男。 
\\	素敵(な)
\\	悪い
\\	性格
\\	頼りない
\\	タイプ
\\	問題外
\\	金持ち
\\	背が高い
\\	頭がいい
\\	やさしい
\\	完璧(な)
\\	ハンサム(な)
\\	どうして
\\	ひどい
\\	はぁ...下山先輩...、素敵。	
\\	はぁ...下山先輩...、素敵。 
\\	絵理花さんは、どうして、下山さんが好きですか。	
\\	絵理花さんは、どうして、下山さんが好きですか。 
\\	どうして?	
\\	どうして? 
\\	うーん。ハンサムだし、やさしいし、頭がいいし、背が高いし、かんぺきだから。	
\\	うーん。ハンサムだし、やさしいし、頭がいいし、背が高いし、かんぺきだから。 
\\	ぼ、僕も、ハンサムでかんぺきな男ですよ。	
\\	ぼ、僕も、ハンサムでかんぺきな男ですよ。 
\\	それに、家は金持ちです。	
\\	それに、家は金持ちです。 
\\	あ、あのぉ...。絵理花さん。ぼ、僕はだめですか?	
\\	あ、あのぉ...。絵理花さん。ぼ、僕はだめですか? 
\\	え?...問題外。	
\\	え?...問題外。 
\\	すぎる
\\	ランチ
\\	一緒に
\\	僕
\\	聞く
\\	知る
\\	彼女
\\	性格
\\	泣く
\\	分かりやすい
\\	ひどい・・・ひどすぎる。ひどすぎます。絵理花さん。	
\\	ひどい・・・ひどすぎる。ひどすぎます。絵理花さん。 
\\	あ、ごめんなさい。あ、泣かないで。	
\\	あ、ごめんなさい。あ、泣かないで。 
\\	でも、下山先輩は本当に素敵だと思う。	
\\	でも、下山先輩は本当に素敵だと思う。 
\\	絵理花、下山先輩、だーい好き。話しやすいし、性格いいし。	
\\	絵理花、下山先輩、だーい好き。話しやすいし、性格いいし。 
\\	ねぇ、下山先輩は彼女がいると思う?	
\\	ねぇ、下山先輩は彼女がいると思う? 
\\	座る
\\	一人で
\\	六本木
\\	主人
\\	歌舞伎
\\	そっくり(な)
\\	人違い
\\	違う
\\	食事
\\	あのぉ、どうぞ、座ってください。	
\\	あのぉ、どうぞ、座ってください。 
\\	え?ああ、どうもありがとうございます。よいしょ。	
\\	え?ああ、どうもありがとうございます。よいしょ。 
\\	あれ?観光ガイドの下山さん?	
\\	あれ?観光ガイドの下山さん? 
\\	え?違います。	
\\	え?違います。 
\\	あ、人違い?すみません。	
\\	あ、人違い?すみません。 
\\	でも、本当にそっくり。あはは、ごめんなさい。	
\\	でも、本当にそっくり。あはは、ごめんなさい。 
\\	いえいえ。おばさんは、浅草観光ですか。	
\\	いえいえ。おばさんは、浅草観光ですか。 
\\	着く
\\	チケット
\\	調べる
\\	地下鉄
\\	東銀座
\\	まだ
\\	多分
\\	決める
\\	もう
\\	連れて行く
\\	浅草~。浅草~。	
\\	浅草~。浅草~。 
\\	あ、もう、浅草に着きましたよ。どこで歌舞伎をみますか。	
\\	あ、もう、浅草に着きましたよ。どこで歌舞伎をみますか。 
\\	ええっと、まだ、決めていません。・・・	
\\	ええっと、まだ、決めていません。・・・ 
\\	多分・・・歌舞伎座に行きます。	
\\	多分・・・歌舞伎座に行きます。 
\\	歌舞伎座?歌舞伎座は浅草にありませんよ。東銀座ですよ。	
\\	歌舞伎座?歌舞伎座は浅草にありませんよ。東銀座ですよ。 
\\	ここから、地下鉄で十分位です。	
\\	ここから、地下鉄で十分位です。 
\\	歌舞伎は何時からですか。	
\\	歌舞伎は何時からですか。 
\\	さあ・・・。まだ、調べていません。	
\\	さあ・・・。まだ、調べていません。 
\\	係員
\\	売り切れる
\\	着く
\\	まだ
\\	もう
\\	早く
\\	すぐ
\\	始まる
\\	お店
\\	チケット、お願いします。	
\\	チケット、お願いします。 
\\	ねぇ、もう、始まった?	
\\	ねぇ、もう、始まった? 
\\	まだ、始まっていませんよ。	
\\	まだ、始まっていませんよ。 
\\	でも、もうすぐ始まります。	
\\	でも、もうすぐ始まります。 
\\	まだ、始まっていない?よかった。	
\\	まだ、始まっていない?よかった。 
\\	チケットはまだある?	
\\	チケットはまだある? 
\\	ええ、まだあります。	
\\	ええ、まだあります。 
\\	ありがとう。	
\\	ありがとう。 
\\	記念
\\	記念撮影
\\	クーポン
\\	キャンセル
\\	問題
\\	変える
\\	撮影
\\	頼む
\\	誕生日
\\	使う
\\	予約
\\	写真
\\	はい、マジックバー101でございます。	
\\	はい、マジックバー101でございます。 
\\	来週木曜日の予約をすることはできますか。	
\\	来週木曜日の予約をすることはできますか。 
\\	七時からニ名です。	
\\	七時からニ名です。 
\\	はい、大丈夫です。	
\\	はい、大丈夫です。 
\\	じゃ、予約をお願いします。	
\\	じゃ、予約をお願いします。 
\\	あとクーポンを使うことができますか。	
\\	あとクーポンを使うことができますか。 
\\	はい、大丈夫です。	
\\	はい、大丈夫です。 
\\	それから、誕生日マジックを頼むことと記念撮影をすることはできますか。	
\\	それから、誕生日マジックを頼むことと記念撮影をすることはできますか。 
\\	多分
\\	おめでとう。
\\	誕生日
\\	今週
\\	食事
\\	一緒
\\	楽しむ
\\	飲み放題
\\	すごい
\\	取る
\\	迎えに来る
\\	あ、絵理花さーん。木曜の夜、会えますよね。	
\\	あ、絵理花さーん。木曜の夜、会えますよね。 
\\	一緒に食事することができますよね。	
\\	一緒に食事することができますよね。 
\\	木曜の夜?ああ、忘れていた。	
\\	木曜の夜?ああ、忘れていた。 
\\	うん。多分大丈夫。会える・・・と思う。	
\\	うん。多分大丈夫。会える・・・と思う。 
\\	六本木のマジックバーを予約することができました。	
\\	六本木のマジックバーを予約することができました。 
\\	へー。あそこ、取れたの?すごい。	
\\	へー。あそこ、取れたの?すごい。 
\\	今週の木曜日は、僕の誕生日ですからね。	
\\	今週の木曜日は、僕の誕生日ですからね。 
\\	お疲れ様
\\	ロビー
\\	後
\\	十五分後
\\	料理
\\	もの
\\	韓国
\\	食べ放題
\\	辛い
\\	ごちそうさま。
\\	珍しい
\\	昼飯
\\	ちょっと
\\	おごる
\\	はい、下山です。	
\\	はい、下山です。 
\\	前田です。お疲れ様。	
\\	前田です。お疲れ様。 
\\	下山君、今からちょっと出られる?	
\\	下山君、今からちょっと出られる? 
\\	ええ…まぁ。でも、どうしてですか。	
\\	ええ…まぁ。でも、どうしてですか。 
\\	昼飯食べに行かない?おごるよ。	
\\	昼飯食べに行かない?おごるよ。 
\\	お!めずらしい!ごちそう様です。	
\\	お!めずらしい!ごちそう様です。 
\\	辛いものは食べられる?	
\\	辛いものは食べられる? 
\\	ええ、食べれます。	
\\	ええ、食べれます。 
\\	じゃ、食べ放題の韓国料理屋に行こう。	
\\	じゃ、食べ放題の韓国料理屋に行こう。 
\\	生まれる
\\	子供
\\	ロシア
\\	ポルトガル
\\	シンガポール
\\	国
\\	仕事
\\	死ぬ
\\	病気
\\	兄
\\	双子
\\	兄弟
\\	全然
\\	ぺらぺら
\\	外国語
\\	色々
\\	チリ
\\	ふー、食べた食べた。もう、食べられない。	
\\	ふー、食べた食べた。もう、食べられない。 
\\	ですよね。僕ももう、食べれません。	
\\	ですよね。僕ももう、食べれません。 
\\	下山君、出身は?東京?	
\\	下山君、出身は?東京? 
\\	ええ。東京の足立区で生まれましたが、父の仕事で色々な国に住みました。	
\\	ええ。東京の足立区で生まれましたが、父の仕事で色々な国に住みました。 
\\	シンガポール、南アフリカ、ロシア、ポルトガル、チリ・・・	
\\	シンガポール、南アフリカ、ロシア、ポルトガル、チリ・・・ 
\\	へぇー。すごいなぁ。じゃ、外国語がペラペラ話せる?	
\\	へぇー。すごいなぁ。じゃ、外国語がペラペラ話せる? 
\\	子供の時、ペラペラでした。	
\\	子供の時、ペラペラでした。 
\\	でも、今は全然話せません。(笑)	
\\	でも、今は全然話せません。(笑) 
\\	言う
\\	彼氏
\\	付き合う
\\	クビにする
\\	アルバイト
\\	似ている
\\	同じ
\\	顔
\\	覚える
\\	双子
\\	そういえば
\\	双子のお兄さん?	
\\	双子のお兄さん? 
\\	ええ。もう、顔を覚えていませんが・・・。	
\\	ええ。もう、顔を覚えていませんが・・・。 
\\	顔は・・・同じじゃない?	
\\	顔は・・・同じじゃない? 
\\	いや、それが、母は僕たちは似ていなかったと言っていました。	
\\	いや、それが、母は僕たちは似ていなかったと言っていました。 
\\	へぇー。あ、そういえば、三門、「下山は、夜六本木でアルバイトをしています。クビにしてください。」って言っていたぞ。	
\\	へぇー。あ、そういえば、三門、「下山は、夜六本木でアルバイトをしています。クビにしてください。」って言っていたぞ。 
\\	三門ねぇ・・・	
\\	三門ねぇ・・・ 
\\	そういえば、三門と左さんは付き合ってますよね。	
\\	そういえば、三門と左さんは付き合ってますよね。 
\\	最近
\\	なる
\\	もともと
\\	髪の毛
\\	エクステ
\\	暖かい
\\	成功
\\	作戦
\\	降る
\\	鳴る
\\	雷
\\	強く
\\	お疲れ様でした。	
\\	お疲れ様でした。 
\\	どうしよう・・・。	
\\	どうしよう・・・。 
\\	雨が強く降っているし・・・、	
\\	雨が強く降っているし・・・、 
\\	雷も鳴っているし・・・	
\\	雷も鳴っているし・・・  
\\	おお、左さん、どうした?	
\\	おお、左さん、どうした? 
\\	傘を持っていない?じゃあ、駅まで一緒に行こう。	
\\	傘を持っていない?じゃあ、駅まで一緒に行こう。 
\\	いいですか。ありがとうございます。	
\\	いいですか。ありがとうございます。 
\\	(心の中の声; 作戦成功!)	
\\	(心の中の声; 作戦成功!) 
\\	最近あたたかくなりましたね。	
\\	最近あたたかくなりましたね。 
\\	この前
\\	違う
\\	遅い
\\	お待たせ
\\	男性
\\	年上
\\	若い
\\	やっと
\\	二十歳
\\	大人
\\	この前、はたちになりました。	
\\	この前、はたちになりました。 
\\	やっと大人になりました。	
\\	やっと大人になりました。 
\\	はたち?それは若いなぁ。	
\\	はたち?それは若いなぁ。 
\\	俺は今年、三十二になるよ。	
\\	俺は今年、三十二になるよ。 
\\	もう、おじさんですね。	
\\	もう、おじさんですね。 
\\	そうか。はたちの女の子には32才はおじさんかぁ・・・。	
\\	そうか。はたちの女の子には32才はおじさんかぁ・・・。 
\\	私は年上の男性の方が好きですよ。	
\\	私は年上の男性の方が好きですよ。 
\\	エリカはいつも、年上の人を好きになります。	
\\	エリカはいつも、年上の人を好きになります。 
\\	あのね、先輩、エリカね・・・。	
\\	あのね、先輩、エリカね・・・。 
\\	絵理花さーん、お待たせ~。	
\\	絵理花さーん、お待たせ~。 
\\	遅くなってごめんね。	
\\	遅くなってごめんね。 
\\	チッ。(舌打ち)	
\\	チッ。(舌打ち) 
\\	彼氏、待ってるよ。	
\\	彼氏、待ってるよ。 
\\	じゃあ、また明日。お疲れ。	
\\	じゃあ、また明日。お疲れ。 
\\	乾杯!
\\	黒い
\\	どれ
\\	怒る
\\	いけない
\\	禁止
\\	服
\\	トランプ
\\	隠す
\\	タイミング
\\	別に
\\	機嫌
\\	うち
\\	乾杯!	
\\	乾杯! 
\\	どうしました?絵理花さん、機嫌が悪いですね。	
\\	どうしました?絵理花さん、機嫌が悪いですね。 
\\	別に。	
\\	別に。 
\\	あ、タイミングが悪かったですか。怒っていますか。	
\\	あ、タイミングが悪かったですか。怒っていますか。 
\\	別に。	
\\	別に。 
\\	僕、面白いことを知っています。	
\\	僕、面白いことを知っています。 
\\	下山がかくしている事です。	
\\	下山がかくしている事です。 
\\	あそこでマジックをしている人を見てください。	
\\	あそこでマジックをしている人を見てください。 
\\	は?どれ?	
\\	は?どれ? 
\\	あの、トランプを持っている人です。	
\\	あの、トランプを持っている人です。 
\\	お先に失礼します。
\\	迷惑
\\	店長
\\	泣く
\\	お元気で。
\\	さっき
\\	もっと
\\	手紙
\\	渡す
\\	金持ち
\\	掛ける
\\	じゃ、お先に失礼します。	
\\	じゃ、お先に失礼します。 
\\	おい、晴夜、さっき、金持ちそうな、おじいさんとおばあさんが来たぞ。	
\\	おい、晴夜、さっき、金持ちそうな、おじいさんとおばあさんが来たぞ。 
\\	「これ晴夜さんに渡してください」って言ってた。	
\\	「これ晴夜さんに渡してください」って言ってた。  
\\	コーヒーだ…。手紙もある。	
\\	コーヒーだ…。手紙もある。 
\\	晴夜さん、本当にどうもありがとう。	
\\	晴夜さん、本当にどうもありがとう。 
\\	そして、迷惑をかけて、ごめんなさい。	
\\	そして、迷惑をかけて、ごめんなさい。 
\\	晴夜さんと会えて、よかったです。	
\\	晴夜さんと会えて、よかったです。 
\\	動く
\\	押す
\\	新年
\\	明けましておめでとうございます。
\\	こちらこそ
\\	本年
\\	浅草
\\	つける
\\	全然
\\	たくさん
\\	レポート
\\	新年になったのね。ジョシュさん、今年もよろしく。	
\\	新年になったのね。ジョシュさん、今年もよろしく。 
\\	こちらこそ、よろしくお願いします。	
\\	こちらこそ、よろしくお願いします。 
\\	ねぇ、テレビつけよう。	
\\	ねぇ、テレビつけよう。 
\\	新年 明けましておめでとうございます。	
\\	新年 明けましておめでとうございます。 
\\	本年も どうぞ よろしく お願い いたします。	
\\	本年も どうぞ よろしく お願い いたします。 
\\	今、私は浅草におります。	
\\	今、私は浅草におります。 
\\	人が たくさんいて、全然 動くことが できません。	
\\	人が たくさんいて、全然 動くことが できません。 
\\	え?あ、押さないでください。キャー!	
\\	え?あ、押さないでください。キャー! 
\\	あ、カメラさん、大丈夫ですか?	
\\	あ、カメラさん、大丈夫ですか? 
\\	失礼いたしました。浅草から レポートいたしました。	
\\	失礼いたしました。浅草から レポートいたしました。 
\\	座る
\\	働く
\\	ホームステイ
\\	住む
\\	群馬
\\	先生
\\	アシスタント
\\	英語
\\	日本語
\\	かける
\\	引越し
\\	どうぞお入りください。	
\\	どうぞお入りください。 
\\	失礼いたします。ジョシュアと申します。	
\\	失礼いたします。ジョシュアと申します。 
\\	どうぞよろしくお願いいたします。	
\\	どうぞよろしくお願いいたします。 
\\	おかけください。…あ、どうぞ、お座りください。	
\\	おかけください。…あ、どうぞ、お座りください。 
\\	ああ、すみません。ありがとうございます。	
\\	"ああ、すみません。ありがとうございます。 
\\	ジョシュアさんは、
\\	…んー。この
\\	は日本語で何ですか?	
\\	ジョシュアさんは、
\\	…んー。この
\\	は日本語で何ですか? 
\\	英語の先生のアシスタントです。	
\\	英語の先生のアシスタントです。 
\\	今、群馬に住んでいるんですか。	
\\	今、群馬に住んでいるんですか。 
\\	はい。ホームステイをしています。	
\\	はい。ホームステイをしています。 
\\	でも、4月から東京で働きたいと思っています。	
\\	でも、4月から東京で働きたいと思っています。 
\\	鬼は外、福は内。
\\	早く
\\	まずい
\\	いわし
\\	終わる
\\	もう
\\	数える
\\	豆
\\	個
\\	宿題
\\	おにはー そと、ふくはー うち。おにはー そと、ふくはー うち。	
\\	おにはー そと、ふくはー うち。おにはー そと、ふくはー うち。 
\\	ジョシュは25才だから、25個、豆を食べるんだよ。	
\\	ジョシュは25才だから、25個、豆を食べるんだよ。 
\\	ええ?25個も食べるの?...数えるの?	
\\	ええ?25個も食べるの?...数えるの? 
\\	うん。	
\\	うん。 
\\	あら、もう豆まき、終わってしまったの?	
\\	あら、もう豆まき、終わってしまったの? 
\\	はい。豆も食べてしまいました。	
\\	はい。豆も食べてしまいました。 
\\	じゃ、これも食べてしまってね。	
\\	じゃ、これも食べてしまってね。 
\\	なんですか?これ?	
\\	なんですか?これ? 
\\	いわし。	
\\	いわし。 
\\	あ、まずいよ、それ。食べないほうがいいよ。	
\\	あ、まずいよ、それ。食べないほうがいいよ。 
\\	大地、早く宿題をしてしまいなさい。	
\\	大地、早く宿題をしてしまいなさい。 
\\	はーい。	
\\	はーい。 
\\	出す
\\	持って来る
\\	疲れる
\\	冗談
\\	つまらない
\\	犬
\\	めずらしい
\\	忘れる
\\	しまった!
\\	成績
\\	はい。みなさん、宿題を出してください。	
\\	はい。みなさん、宿題を出してください。 
\\	宿題?しまった!	
\\	宿題?しまった! 
\\	すみません。先生、宿題を忘れてしまいました。	
\\	すみません。先生、宿題を忘れてしまいました。 
\\	忘れてしまった?めずらしいですね。どうしたんですか。	
\\	忘れてしまった?めずらしいですね。どうしたんですか。 
\\	犬が食べてしまったんです。	
\\	犬が食べてしまったんです。 
\\	…私、つまらない冗談、大嫌いなの。	
\\	…私、つまらない冗談、大嫌いなの。 
\\	すみません。疲れて寝てしまいました。	
\\	すみません。疲れて寝てしまいました。 
\\	じゃ、明日持ってきてくださいね。	
\\	じゃ、明日持ってきてくださいね。 
\\	成績、悪くなってしまいますよ。	
\\	成績、悪くなってしまいますよ。 
\\	はーい。	
\\	はーい。 
\\	かぎ
\\	消す
\\	つける
\\	火
\\	ろうそく
\\	ジュース
\\	入れる
\\	冷蔵庫
\\	並べる
\\	作る
\\	料理
\\	しめる
\\	電気
\\	お父さん、かぎ、しめた?	
\\	お父さん、かぎ、しめた? 
\\	ああ、さっきしめておいたよ。	
\\	ああ、さっきしめておいたよ。 
\\	料理は?	
\\	料理は? 
\\	作って、テーブルの上に並べておいたよ。	
\\	作って、テーブルの上に並べておいたよ。 
\\	じゃ、飲み物は?	
\\	じゃ、飲み物は? 
\\	さっき、買った。冷蔵庫に入れておいたよ。	
\\	さっき、買った。冷蔵庫に入れておいたよ。 
\\	ビールとジュースでいいんだろう?	
\\	ビールとジュースでいいんだろう? 
\\	そう。あ、大地、ろうそくは?	
\\	そう。あ、大地、ろうそくは? 
\\	買っておいた。火はどうする?つけておく?	
\\	買っておいた。火はどうする?つけておく? 
\\	まだ。	
\\	まだ。 
\\	シー!来た!	
\\	シー!来た! 
\\	電気を消して!	
\\	電気を消して! 
\\	ただいま・・・。	
\\	ただいま・・・。 
\\	あ、誰もいないんだ・・・。	
\\	あ、誰もいないんだ・・・。 
\\	ジョシュ、誕生日おめでとう!	
\\	ジョシュ、誕生日おめでとう! 
\\	間違い
\\	直す
\\	食事
\\	似顔絵
\\	ネクタイ
\\	おかしい
\\	最初
\\	驚く
\\	サプライズ
\\	くもり
\\	起きる
\\	借りる
\\	親切
\\	そろそろ
\\	色々
\\	運ぶ
\\	荷物
\\	運転する
\\	貸す
\\	となり
\\	楽しい
\\	この車、借りたの?	
\\	この車、借りたの? 
\\	ああ。となりの山田君が貸してくれたんだ。	
\\	ああ。となりの山田君が貸してくれたんだ。 
\\	山田君が東京まで運転してくれるんだ。	
\\	山田君が東京まで運転してくれるんだ。 
\\	へー。親切な人だね。	
\\	へー。親切な人だね。 
\\	じゃ、荷物運びましょうか。	
\\	じゃ、荷物運びましょうか。 
\\	荷物これだけ?	
\\	荷物これだけ? 
\\	そうです。手伝ってくれて、本当にありがとうございました。	
\\	そうです。手伝ってくれて、本当にありがとうございました。 
\\	じゃ、そろそろ行きましょうか。ジョシュさん。	
\\	じゃ、そろそろ行きましょうか。ジョシュさん。 
\\	みなさん、色々とお世話になりました。	
\\	みなさん、色々とお世話になりました。 
\\	こちらこそ、ウチにホームステイをしてくれてありがとう。	
\\	こちらこそ、ウチにホームステイをしてくれてありがとう。 
\\	楽しかったわ。お元気で。	
\\	楽しかったわ。お元気で。 
\\	またね、ジョシュ!	
\\	またね、ジョシュ! 
\\	シェフ
\\	紹介する
\\	材料
\\	きっと
\\	手作り
\\	持って行く
\\	教える
\\	お弁当
\\	お花見
\\	スタジオ
\\	コマーシャル
\\	こんにちは。ワタルのワクワククッキングの時間です。	
\\	"こんにちは。ワタルのワクワククッキングの時間です。 
\\	今日は、レストラン 
\\	の料 理鉄(リョウ リテツ)シェフにスタジオに来てもらいました。	
\\	今日は、レストラン 
\\	の料 理鉄(リョウ リテツ)シェフにスタジオに来てもらいました。 
\\	こんにちは。	
\\	こんにちは。 
\\	料 理鉄シェフに「お花見弁当」の作り方を教えてもらいます。	
\\	"料 理鉄シェフに「お花見弁当」の作り方を教えてもらいます。 
\\	今年は、お花見にお弁当を持っていきましょう。	
\\	今年は、お花見にお弁当を持っていきましょう。 
\\	手作りのお弁当はきっとおいしいですよ。	
\\	手作りのお弁当はきっとおいしいですよ。 
\\	では、料 理鉄シェフに材料を紹介してもらいましょう。	
\\	では、料 理鉄シェフに材料を紹介してもらいましょう。 
\\	おっとその前にコマーシャル!	
\\	おっとその前にコマーシャル! 
\\	よく
\\	初めて
\\	営業
\\	呼ぶ
\\	自己紹介
\\	書く
\\	うち
\\	署名
\\	押す
\\	印鑑
\\	読む
\\	頑張る
\\	これをよく読んで、印鑑を押してもらえますか。	
\\	これをよく読んで、印鑑を押してもらえますか。 
\\	あ、印鑑がないんですが・・・。	
\\	あ、印鑑がないんですが・・・。 
\\	じゃ、署名してくれますか。	
\\	じゃ、署名してくれますか。 
\\	はい。	
\\	はい。 
\\	みなさん、ちょっと聞いてくれますか。	
\\	みなさん、ちょっと聞いてくれますか。 
\\	こちらは、ジョシュアさんです。今日、うちの会社に入りました。	
\\	こちらは、ジョシュアさんです。今日、うちの会社に入りました。 
\\	じゃ、ジョシュアさん、自己紹介をしてもらえますか。	
\\	じゃ、ジョシュアさん、自己紹介をしてもらえますか。 
\\	え?ちょっと・・・あの・・・待ってもらえませんか。	
\\	え?ちょっと・・・あの・・・待ってもらえませんか。 
\\	ちょっと、書きます。	
\\	ちょっと、書きます。 
\\	はい。大丈夫です。	
\\	はい。大丈夫です。 
\\	ジョシュア・ブラウンと申します。ジョシュと呼んでください。	
\\	ジョシュア・ブラウンと申します。ジョシュと呼んでください。 
\\	営業の仕事は初めてですが、頑張りたいと思います。	
\\	営業の仕事は初めてですが、頑張りたいと思います。 
\\	よろしくお願いいたします。	
\\	よろしくお願いいたします。 
\\	送る
\\	連絡
\\	持って行く
\\	やっぱり
\\	迷惑メール
\\	本社
\\	翻訳
\\	急ぐ
\\	届く
\\	半分
\\	森野さん、昨日のメール読んだ?送っといたんだけど。	
\\	森野さん、昨日のメール読んだ?送っといたんだけど。 
\\	え?メール?いえ、届いてないですよ。	
\\	え?メール?いえ、届いてないですよ。 
\\	急いで翻訳してって書いといたんだよ。まだ、読んでない?	
\\	急いで翻訳してって書いといたんだよ。まだ、読んでない? 
\\	じゃ、もう一度 メールチェックしてみますね。	
\\	じゃ、もう一度 メールチェックしてみますね。 
\\	やっぱり、届いてないですね・・・。	
\\	やっぱり、届いてないですね・・・。 
\\	あ、迷惑メールにありました。アハ。	
\\	あ、迷惑メールにありました。アハ。 
\\	じゃ、急いで翻訳して。それ、3時に本社にもってくんだ。	
\\	じゃ、急いで翻訳して。それ、3時に本社にもってくんだ。 
\\	3時?30分しかないですよ。ちょっと難しいですね。	
\\	3時?30分しかないですよ。ちょっと難しいですね。 
\\	急いで!本社に連絡しとくから。	
\\	急いで!本社に連絡しとくから。 
\\	あ、ジョシュさん、悪いけど、半分手伝ってあげて。	
\\	あ、ジョシュさん、悪いけど、半分手伝ってあげて。 
\\	はい。わかりました。	
\\	はい。わかりました。 
\\	割る
\\	連れて行く
\\	遅刻する
\\	保育園
\\	お腹がすく
\\	いる
\\	ご飯
\\	早く
\\	気をつける
\\	コップ
\\	嫌い
\\	あ、割っちゃった。	
\\	あ、割っちゃった。 
\\	パパ、コップ割っちゃったの?気をつけなくちゃ。	
\\	パパ、コップ割っちゃったの?気をつけなくちゃ。 
\\	はいはい。風歌、早くご飯食べちゃいなさい。	
\\	はいはい。風歌、早くご飯食べちゃいなさい。 
\\	いらない。	
\\	いらない。 
\\	おなかすいちゃうよ。食べなくちゃだめ。	
\\	おなかすいちゃうよ。食べなくちゃだめ。 
\\	やだ、やだ、やだ。	
\\	やだ、やだ、やだ。 
\\	わかった、わかった。じゃ、保育園に行こう。	
\\	わかった、わかった。じゃ、保育園に行こう。 
\\	やだ!	
\\	やだ! 
\\	風歌。パパ、会社に遅刻しちゃうよ。	
\\	風歌。パパ、会社に遅刻しちゃうよ。 
\\	ママ、お仕事、行っちゃった?	
\\	ママ、お仕事、行っちゃった? 
\\	うん。だから、パパ、嵐と風歌を保育園に連れて行かなくちゃいけないんだ。	
\\	うん。だから、パパ、嵐と風歌を保育園に連れて行かなくちゃいけないんだ。 
\\	やだ。パパ嫌い。	
\\	やだ。パパ嫌い。 
\\	風歌!	
\\	風歌! 
\\	免許
\\	ただ
\\	触る
\\	交通違反
\\	止まる
\\	見せる
\\	留守電
\\	メッセージ
\\	携帯電話
\\	なるほど
\\	前の車、止まりなさい。	
\\	前の車、止まりなさい。 
\\	あ、しまった!	
\\	あ、しまった! 
\\	はい、免許、見せてね。...	
\\	はい、免許、見せてね。... 
\\	森野モエさん、電話をしながら、車を運転していましたね。	
\\	森野モエさん、電話をしながら、車を運転していましたね。 
\\	いいえ。話してませんでしたよ。ただ、留守電を聞いてたんです。	
\\	いいえ。話してませんでしたよ。ただ、留守電を聞いてたんです。 
\\	なるほど。メッセージをききながら、運転していた・・・。	
\\	なるほど。メッセージをききながら、運転していた・・・。 
\\	あ、知っています?運転しながら携帯電話をさわってはいけないんですよ。	
\\	あ、知っています?運転しながら携帯電話をさわってはいけないんですよ。 
\\	交通違反なんです。	
\\	交通違反なんです。 
\\	ええぇ。モエ、知りませんでした。	
\\	ええぇ。モエ、知りませんでした。 
\\	もう
\\	昼休み
\\	課長
\\	給料
\\	上げる
\\	コールバック
\\	銀行
\\	外出
\\	件
\\	メール
\\	ただいま帰りました。
\\	ただいま帰りました。
\\	森野さん、外出している間に101銀行のファブリツィオさんから電話があったよ。	
\\	森野さん、外出している間に101銀行のファブリツィオさんから電話があったよ。 
\\	「コールバックお願いします」って。	
\\	「コールバックお願いします」って。 
\\	はい、101銀行ですね。ありがとうございました。	
\\	はい、101銀行ですね。ありがとうございました。 
\\	いない間に、
\\	トラベルの左さんから電話がありました。	
\\	いない間に、
\\	トラベルの左さんから電話がありました。 
\\	また電話するそうです。	
\\	また電話するそうです。 
\\	はい、わかりました。ふー。	
\\	はい、わかりました。ふー。 
\\	えぇ?お昼休みの間に、メールが50件?もういや!	
\\	えぇ?お昼休みの間に、メールが50件?もういや! 
\\	課長、給料、上げてください!	
\\	課長、給料、上げてください! 
\\	事故
\\	看板
\\	勉強
\\	外国
\\	始める
\\	やめる
\\	とる
\\	意味
\\	立入禁止
\\	危ない
\\	漢字
\\	怒鳴る
\\	そこの人!危ないよ!そこは立入禁止だよ!	
\\	そこの人!危ないよ!そこは立入禁止だよ! 
\\	え?	
\\	え? 
\\	「立入禁止」って言ったの。意味わからない?	
\\	"「立入禁止」って言ったの。意味わからない? 
\\	大きい看板あったでしょ。	
\\	大きい看板あったでしょ。 
\\	あ…お客さん、外国の人?	
\\	あ…お客さん、外国の人? 
\\	そうです。	
\\	そうです。 
\\	日本に来たばかり?	
\\	日本に来たばかり? 
\\	来たばかりではないです。	
\\	来たばかりではないです。 
\\	でも、漢字の勉強を始めたばかりなんです。	
\\	でも、漢字の勉強を始めたばかりなんです。 
\\	あー、だから、読めなかったんだ。	
\\	あー、だから、読めなかったんだ。 
\\	怒鳴って、ごめんね。でも、何してたの?	
\\	怒鳴って、ごめんね。でも、何してたの? 
\\	写真をとっていました。	
\\	写真をとっていました。 
\\	そこで、事故があったばかりなんだ。	
\\	そこで、事故があったばかりなんだ。 
\\	だからやめたほうがいいよ。	
\\	だからやめたほうがいいよ。 
\\	遅刻
\\	壊れる
\\	風邪
\\	電車
\\	少し
\\	動く
\\	悪い
\\	体調
\\	休む
\\	台風
\\	お疲れ様です。森野です。体調が悪いため、会社を休みます。すみません。	
\\	お疲れ様です。森野です。体調が悪いため、会社を休みます。すみません。 
\\	おはようございます。ジョシュです。台風で電車が動いていないため、少し遅刻します。	
\\	おはようございます。ジョシュです。台風で電車が動いていないため、少し遅刻します。 
\\	はい。イマイチ電気です。	
\\	はい。イマイチ電気です。 
\\	トラベルの左と申しますが、森野さんをお願いします。	
\\	トラベルの左と申しますが、森野さんをお願いします。 
\\	申し訳ございません。森野は風邪のため休んでいます。	
\\	申し訳ございません。森野は風邪のため休んでいます。 
\\	そうですか。じゃ、ジョシュさんは?	
\\	そうですか。じゃ、ジョシュさんは? 
\\	ジョシュは台風のため、まだ来ていません。	
\\	ジョシュは台風のため、まだ来ていません。 
\\	うーん。じゃ、課長さんは?	
\\	うーん。じゃ、課長さんは? 
\\	台風で家が壊れたため、しばらく会社を休むと言っていました。	
\\	台風で家が壊れたため、しばらく会社を休むと言っていました。 
\\	電気代
\\	野菜
\\	困る
\\	育つ
\\	エアコン
\\	夏
\\	毎晩
\\	売れる
\\	意見
\\	売る
\\	涼しい
\\	植物
\\	勤める
\\	今年の夏は涼しいですね。	
\\	今年の夏は涼しいですね。 
\\	みなさんは、涼しい夏が好きですか?嫌いですか?	
\\	みなさんは、涼しい夏が好きですか?嫌いですか? 
\\	みなさんの意見を聞いてみました。	
\\	みなさんの意見を聞いてみました。 
\\	毎晩涼しいおかげで、よく寝られるの。	
\\	毎晩涼しいおかげで、よく寝られるの。 
\\	だから、体調がいいのよ。	
\\	だから、体調がいいのよ。 
\\	この天気のおかげで、電気代が安いよ。	
\\	この天気のおかげで、電気代が安いよ。 
\\	エアコン使わないから。	
\\	エアコン使わないから。 
\\	植物が育たないせいで、野菜、高いんです。	
\\	植物が育たないせいで、野菜、高いんです。 
\\	一杯
\\	メール
\\	予定
\\	給料日
\\	誘う
\\	残念
\\	海外
\\	働く
\\	今度
\\	返事
\\	あ、今日は、給料日ですね!やったー。	
\\	あ、今日は、給料日ですね!やったー。 
\\	ジョシュ、仕事の後、一杯どうですか。	
\\	ジョシュ、仕事の後、一杯どうですか。 
\\	今日はちょっと。	
\\	今日はちょっと。 
\\	だめなの?	
\\	だめなの? 
\\	すみません。明日の朝早く、海外のお客さんに電話しなくちゃいけないんです。	
\\	すみません。明日の朝早く、海外のお客さんに電話しなくちゃいけないんです。 
\\	明日土曜日なのに働くの?	
\\	明日土曜日なのに働くの? 
\\	ええ。残念ですが、また今度誘ってください。	
\\	ええ。残念ですが、また今度誘ってください。 
\\	あーあ、誕生日なのに、予定がない・・・。あーあ。	
\\	あーあ、誕生日なのに、予定がない・・・。あーあ。 
\\	え?彼氏と会わないんですか?	
\\	え?彼氏と会わないんですか? 
\\	メールしたのに、返事がないんだ。	
\\	メールしたのに、返事がないんだ。 
\\	ネット
\\	キロ
\\	くじ
\\	かかる
\\	参加
\\	迷う
\\	申し込む
\\	手数料
\\	当たる
\\	マラソン
\\	ええっと…東京マラソン、参加料…	
\\	ええっと…東京マラソン、参加料… 
\\	フルマラソンだったら、10000円。10キロだったら、5000円…。	
\\	フルマラソンだったら、10000円。10キロだったら、5000円…。 
\\	うーん。迷うなー。	
\\	うーん。迷うなー。 
\\	どうしたの?	
\\	どうしたの? 
\\	でも、もしフルマラソンを走ったら、次の日動けないよね…。	
\\	でも、もしフルマラソンを走ったら、次の日動けないよね…。 
\\	ああ、東京マラソン?近森、走るの?	
\\	ああ、東京マラソン?近森、走るの? 
\\	うん。くじに当たったら。	
\\	うん。くじに当たったら。 
\\	そっか。当たらなかったら、走れないよね。	
\\	そっか。当たらなかったら、走れないよね。 
\\	あ、ネットから申し込んだら、手数料がかからないんだ。	
\\	あ、ネットから申し込んだら、手数料がかからないんだ。 
\\	あ!「8月31日までに申し込んでください」だって。もう九月だよ。	
\\	"あ!「8月31日までに申し込んでください」だって。もう九月だよ。 
\\	うそー。	
\\	うそー。 
\\	病気
\\	水
\\	ポテト
\\	うらやましい
\\	自分
\\	やせる
\\	細い
\\	もっと
\\	ハンバーガー
\\	なんで
\\	太る
\\	野菜
\\	健康的
\\	いただきまーす。	
\\	いただきまーす。 
\\	いいなぁ。	
\\	いいなぁ。 
\\	なんで?	
\\	なんで? 
\\	食べても太らない。	
\\	食べても太らない。 
\\	ええ?そう?	
\\	ええ?そう? 
\\	ハンバーガーを食べても、ポテトを食べても、太らないでしょ。	
\\	ハンバーガーを食べても、ポテトを食べても、太らないでしょ。 
\\	そうかなぁ。	
\\	そうかなぁ。 
\\	自分は、野菜を食べても、水を飲んでも太るんだよ。	
\\	自分は、野菜を食べても、水を飲んでも太るんだよ。 
\\	運動しても、やせない。	
\\	運動しても、やせない。 
\\	あーあ。歩がうらやましい。	
\\	あーあ。歩がうらやましい。 
\\	でも、渡も細いじゃん。	
\\	でも、渡も細いじゃん。 
\\	細くても、もっと細くなりたいの。	
\\	細くても、もっと細くなりたいの。 
\\	もっと細かったら、病気だよ。健康的じゃないよ。	
\\	もっと細かったら、病気だよ。健康的じゃないよ。 
\\	健康的じゃなくてもいいの!	
\\	健康的じゃなくてもいいの! 
\\	・・・病気だね。	
\\	・・・病気だね。 
\\	掃除
\\	となり
\\	におい
\\	ガス
\\	変
\\	調子
\\	洗濯
\\	コンロ
\\	始める
\\	風
\\	危ない
\\	おはようございます、ジョシュさん。	
\\	おはようございます、ジョシュさん。 
\\	あ、おはようございます。・・・あの、ちょっといいですか。	
\\	あ、おはようございます。・・・あの、ちょっといいですか。 
\\	はい。	
\\	はい。 
\\	エアコンの調子が悪いんです。	
\\	エアコンの調子が悪いんです。 
\\	どんな風に悪いんですか?	
\\	どんな風に悪いんですか? 
\\	エアコンをつけると、変なにおいがするんです。	
\\	エアコンをつけると、変なにおいがするんです。 
\\	あら・・・。	
\\	あら・・・。 
\\	あと、ガスコンロの火をつけると、ガスのにおいがするんですよ。	
\\	あと、ガスコンロの火をつけると、ガスのにおいがするんですよ。 
\\	それは危ないですね。	
\\	それは危ないですね。 
\\	それから、夜になると となりの人、掃除と洗濯を始めるんです。	
\\	それから、夜になると となりの人、掃除と洗濯を始めるんです。 
\\	あー、大空さんね。話しておきます。	
\\	あー、大空さんね。話しておきます。 
\\	亡くなる
\\	待つ
\\	約束
\\	休暇届
\\	もうすぐ
\\	結構
\\	有名
\\	お休み
\\	銀行
\\	先月
\\	銀行のユウ レイと申します。	
\\	銀行のユウ レイと申します。 
\\	ジョシュさんと約束があるのですが。	
\\	ジョシュさんと約束があるのですが。 
\\	ジョシュはもうすぐ来るはずなんですが…。	
\\	ジョシュはもうすぐ来るはずなんですが…。 
\\	じゃ、待ちます。	
\\	じゃ、待ちます。 
\\	・・・来ないですね。あ、お昼でもどうですか?	
\\	・・・来ないですね。あ、お昼でもどうですか? 
\\	近くに有名なそば屋があるんですよ。	
\\	近くに有名なそば屋があるんですよ。 
\\	私は行ったことがありませんが、おいしいはずですよ。	
\\	私は行ったことがありませんが、おいしいはずですよ。 
\\	結構です。	
\\	結構です。 
\\	・・・そうですか。	
\\	・・・そうですか。 
\\	ジョシュ、来た?	
\\	ジョシュ、来た? 
\\	え?お休みのはずです。	
\\	え?お休みのはずです。 
\\	は?	
\\	は? 
\\	プロポーズ
\\	違う
\\	実は
\\	先月
\\	機嫌
\\	小説家
\\	連絡
\\	尊敬
\\	招待
\\	出版
\\	小説
\\	機嫌がいいですね。どうしたんですか。	
\\	機嫌がいいですね。どうしたんですか。 
\\	え?わかる?実は、彼にプロポーズされたの。	
\\	え?わかる?実は、彼にプロポーズされたの。 
\\	へー。彼氏から 連絡あったんですね。	
\\	へー。彼氏から 連絡あったんですね。 
\\	違う 違う。新しい彼。	
\\	違う 違う。新しい彼。 
\\	先月、課長に小説家を紹介されたの。	
\\	先月、課長に小説家を紹介されたの。 
\\	はい。	
\\	はい。 
\\	次の週、彼のうちのパーティーに招待されたの。	
\\	次の週、彼のうちのパーティーに招待されたの。 
\\	ええ。それで?	
\\	ええ。それで? 
\\	で、昨日プロポーズされたの。	
\\	で、昨日プロポーズされたの。 
\\	はい?	
\\	はい? 
\\	彼の新しい小説、来週出版されるから、買ってね。	
\\	彼の新しい小説、来週出版されるから、買ってね。 
\\	彼、みんなに尊敬されているんだ。すごいでしょ。	
\\	彼、みんなに尊敬されているんだ。すごいでしょ。 
\\	呼ぶ
\\	結婚披露宴
\\	司会
\\	自慢
\\	さっき
\\	子
\\	頼む
\\	一目ぼれ
\\	だます
\\	うらやましい
\\	さっき、そこで夏目先生と会ったんだけど・・・。	
\\	さっき、そこで夏目先生と会ったんだけど・・・。 
\\	結婚するって言ってた?	
\\	結婚するって言ってた? 
\\	どうして知ってるの?	
\\	どうして知ってるの? 
\\	先生に結婚披露宴の司会を頼まれたから。	
\\	先生に結婚披露宴の司会を頼まれたから。 
\\	へー。自分も、披露宴に呼ばれた。	
\\	へー。自分も、披露宴に呼ばれた。 
\\	「若くてかわいい子なんだ。一目ぼれだった」って、自慢されたわ。	
\\	"「若くてかわいい子なんだ。一目ぼれだった」って、自慢されたわ。 
\\	へー。うらやましい。	
\\	へー。うらやましい。 
\\	でも、その若い子にだまされていないかな?大丈夫かな?	
\\	"でも、その若い子にだまされていないかな?大丈夫かな? 
\\	大丈夫でしょ。でも、「君、早く結婚をしなさい」って言われたわ。	
\\	大丈夫でしょ。でも、「君、早く結婚をしなさい」って言われたわ。 
\\	大きなお世話だよね。	
\\	大きなお世話だよね。 
\\	場所
\\	切る
\\	モデル
\\	会費
\\	日にち
\\	決める
\\	お化け屋敷
\\	座る
\\	居酒屋
\\	忘年会
\\	ところで
\\	森野さん、髪切った?いいね。モデルみたい。	
\\	森野さん、髪切った?いいね。モデルみたい。 
\\	ええ。そうですか?	
\\	ええ。そうですか? 
\\	松田さんのネクタイも、コメディアンみたいで、素敵です。	
\\	松田さんのネクタイも、コメディアンみたいで、素敵です。 
\\	・・・ところで、忘年会の日にちと場所は決めたの?	
\\	・・・ところで、忘年会の日にちと場所は決めたの? 
\\	12月22日午後七時から。場所は居酒屋101です。	
\\	12月22日午後七時から。場所は居酒屋101です。 
\\	会費は?	
\\	会費は? 
\\	一人、5500円です。	
\\	"一人、5500円です。 
\\	いらっしゃいませ。こちらへどうぞ。	
\\	いらっしゃいませ。こちらへどうぞ。 
\\	森野さん、ここ、お化け屋敷みたいなお店ですね。	
\\	森野さん、ここ、お化け屋敷みたいなお店ですね。 
\\	そうね…。あ、ジョシュ、そこに座っちゃだめ。	
\\	そうね…。あ、ジョシュ、そこに座っちゃだめ。 
\\	え?…どうしてですか?	
\\	え?…どうしてですか? 
\\	そこは上座。私たちは、あっちに座るの。	
\\	そこは上座。私たちは、あっちに座るの。 
\\	残高
\\	入力
\\	引き出す
\\	暗証番号
\\	やる
\\	口座
\\	また
\\	カード
\\	金額
\\	入れる
\\	あの…すみません。	
\\	あの…すみません。 
\\	はい。	
\\	はい。 
\\	暗証番号を入れても、お金が引き出せないんですけど。	
\\	暗証番号を入れても、お金が引き出せないんですけど。 
\\	そうですか。	
\\	そうですか。 
\\	では、もう一度、カードを入れてもらえますか。	
\\	では、もう一度、カードを入れてもらえますか。 
\\	はい。	
\\	はい。 
\\	次に、暗証番号をご入力ください。	
\\	次に、暗証番号をご入力ください。 
\\	はい。	
\\	はい。 
\\	最後に、金額をご入力ください。	
\\	最後に、金額をご入力ください。 
\\	引き出したい金額ですね。	
\\	引き出したい金額ですね。 
\\	あ、まただ…。ね?何度やっても、ダメなんですよ。	
\\	あ、まただ…。ね?何度やっても、ダメなんですよ。 
\\	あの…大変失礼ですが、残高はございますか。	
\\	あの…大変失礼ですが、残高はございますか。 
\\	残高がないようですね。	
\\	残高がないようですね。 
\\	行きます
\\	みなさん
\\	かれし
\\	です
\\	こんにちは
\\	らいげつ
\\	しゅっしん
\\	よろしくおねがいします
\\	みなさん、こんにちは。	
\\	みなさん、こんにちは。 
\\	わたし は エミリー・マーティン です。	
\\	わたし は エミリー・マーティン です。 
\\	しゅっしんはオーストラリアのブリスベンです。	
\\	しゅっしんはオーストラリアのブリスベンです。 
\\	25さい です。	
\\	25さい です。 
\\	らいげつ、かれしと日ほんにいきます。	
\\	らいげつ、かれしと日ほんにいきます。 
\\	よろしく おねがいします。	
\\	よろしく おねがいします。 
\\	メートル
\\	です
\\	どうぞ
\\	そうですか。
\\	は
\\	きます
\\	あれ
\\	すみません
\\	どうぞ。	
\\	どうぞ。 
\\	ありがとうございます。…アメリカから?	
\\	ありがとうございます。…アメリカから? 
\\	いいえ。オーストラリアのブリスベンからきました。	
\\	いいえ。オーストラリアのブリスベンからきました。 
\\	そうですか。	
\\	そうですか。 
\\	すみません、あれは、とうきょうタワーですか。	
\\	すみません、あれは、とうきょうタワーですか。 
\\	あれ?あれは、とうきょうスカイツリーです。	
\\	あれ?あれは、とうきょうスカイツリーです。 
\\	634メートルです。	
\\	634メートルです。 
\\	ろっぴゃく、さんじゅう、よん…。	
\\	ろっぴゃく、さんじゅう、よん…。 
\\	これ
\\	~ともうします
\\	みょうじ
\\	はじめまして
\\	よろしくおねがいします
\\	こちら
\\	さん
\\	めいし
\\	さん、こちらはあかさかデザインの森さんです。	
\\	さん、こちらはあかさかデザインの森さんです。 
\\	はじめまして。
\\	ともうします。	
\\	はじめまして。
\\	ともうします。 
\\	森です。よろしくおねがいします。	
\\	森です。よろしくおねがいします。 
\\	もり・・・。「もり」はみょうじですか。	
\\	"もり・・・。「もり」はみょうじですか。 
\\	はい。これは、わたしのめいしです。どうぞ。	
\\	はい。これは、わたしのめいしです。どうぞ。 
\\	ありがとうございます。	
\\	ありがとうございます。 
\\	の
\\	こちら
\\	おくさん
\\	~とよんでください
\\	かのじょ
\\	こちらは
\\	さんのおくさんですか。	
\\	こちらは
\\	さんのおくさんですか。 
\\	おくさん?	
\\	おくさん? 
\\	あ・・・ワイフ?	
\\	あ・・・ワイフ? 
\\	いいえ。かのじょです。かのじょのエミリー・マーティンです。	
\\	いいえ。かのじょです。かのじょのエミリー・マーティンです。 
\\	エミリーとよんでください。よろしくおねがいします。	
\\	エミリーとよんでください。よろしくおねがいします。 
\\	森 はな子です。森とよんでください。よろしくおねがいします。	
\\	森 はな子です。森とよんでください。よろしくおねがいします。 
\\	おねがいします
\\	マンション
\\	じゅうしょ
\\	すみません。
\\	つきます
\\	りょうしゅうしょ
\\	イノベーティブマンションまで、おねがいします。	
\\	イノベーティブマンションまで、おねがいします。 
\\	イノベーティブマンション?	
\\	イノベーティブマンション? 
\\	じゅうしょはなんですか。	
\\	じゅうしょはなんですか。 
\\	これです。
\\	これです。
\\	あかさか一の二の三。。。	
\\	あかさか一の二の三。。。 
\\	はい、つきました。1380えんです。	
\\	はい、つきました。1380えんです。 
\\	せん、さんびゃく、はちじゅうえん・・・はい。	
\\	せん、さんびゃく、はちじゅうえん・・・はい。 
\\	すみません、りょうしゅうしょ、おねがいします。	
\\	すみません、りょうしゅうしょ、おねがいします。 
\\	かい
\\	ごうしつ
\\	せんしゅう
\\	どうぞよろしく
\\	となり
\\	なんかいですか?	
\\	なんかいですか? 
\\	四かい、おねがいします。	
\\	四かい、おねがいします。 
\\	はい?	
\\	はい? 
\\	四、おねがいします。	
\\	四、おねがいします。 
\\	四かい?401(よんまるいち)ごうしつ?	
\\	四かい?401(よんまるいち)ごうしつ? 
\\	はい。	
\\	はい。 
\\	わたしは402ごうしつのはやしです。	
\\	わたしは402ごうしつのはやしです。 
\\	となりですね。どうぞよろしく。	
\\	となりですね。どうぞよろしく。 
\\	エミリー・マーティンです。せんしゅう、日ほんにきました。	
\\	エミリー・マーティンです。せんしゅう、日ほんにきました。 
\\	よろしくおねがいします。	
\\	よろしくおねがいします。 
\\	の
\\	えいご
\\	おもしろい
\\	ともだち
\\	本
\\	どくしん
\\	さん?森です。	
\\	さん?森です。 
\\	あ…おはようございます。	
\\	あ…おはようございます。 
\\	それは、えいごの本ですか。	
\\	それは、えいごの本ですか。 
\\	え?
\\	はい。	
\\	え?
\\	はい。 
\\	もりさんのですか。
\\	もりさんのですか。 
\\	わたしのではありません。ともだちのです。	
\\	わたしのではありません。ともだちのです。 
\\	へー。
\\	へー。
\\	わたしはどくしんじゃないです。でも、これはおもしろいです。	
\\	わたしはどくしんじゃないです。でも、これはおもしろいです。 
\\	きれい
\\	さかな
\\	スプーン
\\	カレーライス
\\	ください
\\	おちゃ
\\	つめたい
\\	さかなていしょく。	
\\	さかなていしょく。 
\\	カレーライス と つめたいおちゃ・・・ください。	
\\	カレーライス と つめたいおちゃ・・・ください。 
\\	水とおちゃはセルフサービスです。あそこにあります。	
\\	水とおちゃはセルフサービスです。あそこにあります。 
\\	はい。	
\\	はい。 
\\	おまたせしました。カレーライスです。	
\\	おまたせしました。カレーライスです。 
\\	このスプーン…。	
\\	このスプーン…。 
\\	すみません。きれいなスプーンください。	
\\	すみません。きれいなスプーンください。 
\\	よ
\\	こわい
\\	こら!
\\	小さい
\\	おねえさん
\\	ごめんなさい。
\\	だいじょうぶ
\\	なまえ
\\	きゃ!	
\\	きゃ! 
\\	こら!ゴハン!	
\\	こら!ゴハン! 
\\	おねえさん、ごめんなさい。だいじょうぶ?	
\\	おねえさん、ごめんなさい。だいじょうぶ? 
\\	だいじょうぶです。	
\\	だいじょうぶです。 
\\	小さいですね。これ、チワワですか。	
\\	小さいですね。これ、チワワですか。 
\\	ううん。パピヨン。	
\\	ううん。パピヨン。 
\\	きゃ!	
\\	きゃ! 
\\	ゴハン!しーっ!…	
\\	ゴハン!しーっ!… 
\\	ゴハンはこのいぬのなまえですか。	
\\	ゴハンはこのいぬのなまえですか。 
\\	うん。ソン・ゴハン。	
\\	うん。ソン・ゴハン。 
\\	…おねえさん、いぬ、こわい?	
\\	…おねえさん、いぬ、こわい? 
\\	こ、こ、こわくないですよ。かわいいですよ。	
\\	こ、こ、こわくないですよ。かわいいですよ。 
\\	も
\\	本とう?
\\	しょうがない
\\	どうぶつ
\\	あります
\\	へん
\\	アレルギー
\\	本とう?おねえさん、いぬきらい?	
\\	本とう?おねえさん、いぬきらい? 
\\	きらいじゃないです。でも…。	
\\	きらいじゃないです。でも…。 
\\	でも?	
\\	でも? 
\\	アレルギーがあります。	
\\	アレルギーがあります。 
\\	どうぶつアレルギー?	
\\	どうぶつアレルギー? 
\\	犬アレルギーです。	
\\	犬アレルギーです。 
\\	犬アレルギー?	
\\	犬アレルギー? 
\\	へんですか。	
\\	へんですか。 
\\	へんじゃないよ。しょうがないよ。	
\\	へんじゃないよ。しょうがないよ。 
\\	ゴハンもアレルギーがあるよ。ね?ゴハン。	
\\	ゴハンもアレルギーがあるよ。ね?ゴハン。 
\\	だから
\\	むずかしい
\\	はる
\\	高い
\\	ながい
\\	も
\\	ことば
\\	ふーん
\\	すき(な)
\\	ところ
\\	犬のアレルギーですか。	
\\	犬のアレルギーですか。 
\\	うん。かふんしょう。	
\\	うん。かふんしょう。 
\\	わたしも、かふんしょうです。	
\\	わたしも、かふんしょうです。 
\\	だから、はるがすきじゃありません。	
\\	だから、はるがすきじゃありません。 
\\	ふーん。ぼくは、高いところがこわくてきらい。	
\\	ふーん。ぼくは、高いところがこわくてきらい。 
\\	きょうふ・しょう?	
\\	きょうふ・しょう? 
\\	うん。こうしょきょうふしょう。	
\\	うん。こうしょきょうふしょう。 
\\	こうしょ・きょう・ふしょう…。	
\\	こうしょ・きょう・ふしょう…。 
\\	ながくて、むずかしいことばですね。
\\	ながくて、むずかしいことばですね。
\\	カン
\\	どよう日
\\	びん
\\	かよう日
\\	きんよう日
\\	木よう日
\\	日よう日
\\	げつよう日
\\	なんよう日
\\	水よう日
\\	おはようございます。	
\\	おはようございます。 
\\	おはようございます。それはもえるごみですか。	
\\	おはようございます。それはもえるごみですか。 
\\	いいえ。カンとびんです。	
\\	いいえ。カンとびんです。 
\\	あ、それは木よう日ですよ。	
\\	あ、それは木よう日ですよ。 
\\	はい?なんよう日ですか。	
\\	はい?なんよう日ですか。 
\\	も・く・よ・う・び。	
\\	も・く・よ・う・び。 
\\	げつ・か・すい・もく…。	
\\	げつ・か・すい・もく…。 
\\	人
\\	ちょっと
\\	で
\\	たくさん
\\	きょう
\\	はじめて
\\	まつり
\\	人がたくさんいますね。	
\\	人がたくさんいますね。 
\\	きょうは、まつりですよ。	
\\	きょうは、まつりですよ。 
\\	まつり?	
\\	まつり? 
\\	まつりは、えいごで、
\\	です。	
\\	"まつりは、えいごで、
\\	です。 
\\	まつりは、はじめてですか。	
\\	まつりは、はじめてですか。 
\\	はい。	
\\	はい。 
\\	じゃ、ちょっといきませんか。	
\\	じゃ、ちょっといきませんか。 
\\	あれは、何ですか。	
\\	"あれは、何ですか。 
\\	みこしです。	
\\	"みこしです。 
\\	みこし?はじめてみました。	
\\	"みこし?はじめてみました。 
\\	みこしは、えいごで 何ですか。	
\\	みこしは、えいごで 何ですか。 
\\	どうですか
\\	おごります
\\	はじめて
\\	みせ
\\	高い
\\	何か
\\	たこ
\\	とうもろこし
\\	二つ
\\	おみせがたくさんありますね。	
\\	おみせがたくさんありますね。 
\\	やたいもはじめてですか。	
\\	やたいもはじめてですか。 
\\	はい。	
\\	はい。 
\\	じゃ、何か食べましょう。	
\\	じゃ、何か食べましょう。 
\\	いいですね。	
\\	いいですね。 
\\	たこやきはどうですか。	
\\	たこやきはどうですか。 
\\	たこ?たこはちょっと・・・。	
\\	たこ?たこはちょっと・・・。 
\\	とうもろこしはどうですか。	
\\	とうもろこしはどうですか。 
\\	いいですね。おごりますよ。	
\\	いいですね。おごりますよ。 
\\	いらっしゃいませ。	
\\	いらっしゃいませ。 
\\	二つ、おねがいします。	
\\	二つ、おねがいします。 
\\	1000えんです。	
\\	1000えんです。 
\\	せんえん?高い...。	
\\	せんえん?高い...。 
\\	円
\\	つかいます
\\	きゅうしゅう
\\	チケット
\\	レジぶくろ
\\	おめでとうございます。
\\	りょこう
\\	かいけい
\\	おかいけい、1000円です。	
\\	おかいけい、1000円です。 
\\	レジぶくろ、ごりようですか。	
\\	レジぶくろ、ごりようですか。 
\\	ごりよう?…すみません。何ですか。	
\\	ごりよう?…すみません。何ですか。 
\\	レジぶくろを つかいますか。	
\\	レジぶくろを つかいますか。 
\\	いいえ、つかいません。	
\\	いいえ、つかいません。 
\\	こちらは、ふくびきのチケットです。どうぞ。	
\\	こちらは、ふくびきのチケットです。どうぞ。 
\\	こちらにどうぞ。	
\\	こちらにどうぞ。 
\\	おめでとうございます!きゅうしゅうりょこう、 にめいさま!	
\\	おめでとうございます!きゅうしゅうりょこう、 にめいさま! 
\\	かえります
\\	ゆうめい
\\	いきます
\\	ふるさと
\\	どこ
\\	おすすめ
\\	おんせん
\\	こんにちは。りょこうですか。	
\\	こんにちは。りょこうですか。 
\\	はい。九しゅうにいきます。	
\\	はい。九しゅうにいきます。 
\\	九しゅうのどこですか。	
\\	九しゅうのどこですか。 
\\	かごしまです。	
\\	かごしまです。 
\\	わたしのふるさと!かごしま!いいところですよ。	
\\	わたしのふるさと!かごしま!いいところですよ。 
\\	おんせんにいきたいです。おすすめ、ありますか。	
\\	おんせんにいきたいです。おすすめ、ありますか。 
\\	いぶすきおんせんはゆうめいですよ。	
\\	いぶすきおんせんはゆうめいですよ。 
\\	いいですね。わたしもかごしまにかえりたい。	
\\	いいですね。わたしもかごしまにかえりたい。 
\\	また
\\	けいさつ
\\	おまたせしました
\\	おまちください
\\	もういちど
\\	ようけん
\\	まで
\\	はい、イノベイティブ・デザインです。	
\\	はい、イノベイティブ・デザインです。 
\\	かごしまけいさつですが、
\\	さんはいますか。	
\\	かごしまけいさつですが、
\\	さんはいますか。 
\\	ですか?しょうしょうおまちください。	
\\	ですか?しょうしょうおまちください。 
\\	おまたせしました。
\\	は木よう日までやすみです。	
\\	おまたせしました。
\\	は木よう日までやすみです。 
\\	すみませんが、もういちど、おなまえをおねがいします。	
\\	すみませんが、もういちど、おなまえをおねがいします。 
\\	かごしまけいさつの金木です。	
\\	かごしまけいさつの金木です。 
\\	けいさつ…。ごようけんは?	
\\	けいさつ…。ごようけんは? 
\\	だいじょうぶです。また、でんわします。	
\\	だいじょうぶです。また、でんわします。 
\\	はらいます
\\	ぶた
\\	たばこ
\\	カード
\\	二人
\\	さいふ
\\	きんえん
\\	二つ
\\	生ビール
\\	いらっしゃいませ。何めいさまですか。	
\\	いらっしゃいませ。何めいさまですか。 
\\	二人です。	
\\	二人です。 
\\	おたばこは。	
\\	おたばこは。 
\\	すいません。きんえんせき、おねがいします。	
\\	すいません。きんえんせき、おねがいします。 
\\	こちらにどうぞ。	
\\	こちらにどうぞ。 
\\	生ビールをふたつ。それから・・・ぶたのしゃぶしゃぶコースをふたつ、おねがいします。	
\\	生ビールをふたつ。それから・・・ぶたのしゃぶしゃぶコースをふたつ、おねがいします。 
\\	生ビール、ふたつと、ぶたのしゃぶしゃぶコース、ふたつですね。しょうしょうおまちください。	
\\	生ビール、ふたつと、ぶたのしゃぶしゃぶコース、ふたつですね。しょうしょうおまちください。 
\\	おかいけい、20000円です。	
\\	おかいけい、20000円です。 
\\	カードではらいます。あれ?さいふ・・・。さいふが・・・。	
\\	カードではらいます。あれ?さいふ・・・。さいふが・・・。 
\\	おとします
\\	たんじょう日
\\	せいねんがっぴ
\\	せいれき
\\	オペレーター
\\	いつ
\\	ピサカードです。	
\\	ピサカードです。 
\\	カードをおとしました。	
\\	カードをおとしました。 
\\	おなまえをおねがいします。	
\\	おなまえをおねがいします。 
\\	です。	
\\	です。 
\\	せいねんがっぴをおねがいします。	
\\	せいねんがっぴをおねがいします。 
\\	せいねん…?	
\\	せいねん…? 
\\	たんじょう日はいつですか。せいれきからおねがいします。	
\\	たんじょう日はいつですか。せいれきからおねがいします。 
\\	10月…。すみません、えいごのオペレーター、いますか。	
\\	10月…。すみません、えいごのオペレーター、いますか。 
\\	でんわ
\\	れんらく
\\	さいあく
\\	どうぞ
\\	おみやげ
\\	たいへん
\\	とても
\\	かごしまに行きました。おみやげです。どうぞ。
\\	かごしまに行きました。おみやげです。どうぞ。
\\	ありがとうございます。りょこうはたのしかったですか。	
\\	ありがとうございます。りょこうはたのしかったですか。 
\\	さいあくでした。さいふをおとしました。	
\\	さいあくでした。さいふをおとしました。 
\\	えー?!たいへんでしたね。	
\\	えー?!たいへんでしたね。 
\\	はい。とても高いさいふでした。
\\	はい。とても高いさいふでした。
\\	けいさつにれんらくしましたか。	
\\	けいさつにれんらくしましたか。 
\\	けいさつ?いいえ。	
\\	けいさつ?いいえ。 
\\	さん、かごしまけいさつから、でんわです。	
\\	さん、かごしまけいさつから、でんわです。 
\\	だから
\\	うれしい
\\	休み
\\	また
\\	わかりました
\\	ふくおか
\\	そうですか。しりませんでした。	
\\	そうですか。しりませんでした。 
\\	…はい、わかりました。…ありがとうございました。	
\\	…はい、わかりました。…ありがとうございました。 
\\	うれしそうですね。	
\\	うれしそうですね。 
\\	はい。おさいふがありました。	
\\	はい。おさいふがありました。 
\\	よかったですね。	
\\	よかったですね。 
\\	はい。でも、ふくおかにあります。	
\\	はい。でも、ふくおかにあります。 
\\	ふくおか…。	
\\	ふくおか…。 
\\	だから、またふくおかに行きます。	
\\	だから、またふくおかに行きます。 
\\	ぶちょう、お休みをください。	
\\	ぶちょう、お休みをください。 
\\	また?	
\\	また? 
\\	もう
\\	しずか
\\	どちらさまですか。
\\	ごうしつ
\\	すみませんが、
\\	たのしい
\\	パーティー
\\	はい。どちらさまですか。	
\\	はい。どちらさまですか。 
\\	402ごうしつの林です。こんばんは。	
\\	402ごうしつの林です。こんばんは。 
\\	こんばんは。おひさしぶりです。	
\\	こんばんは。おひさしぶりです。 
\\	たのしそうですね。パーティーですか。	
\\	たのしそうですね。パーティーですか。 
\\	はい。	
\\	はい。 
\\	すみませんが、しずかにおねがいします。もう、12じですから。	
\\	すみませんが、しずかにおねがいします。もう、12じですから。 
\\	すみません。	
\\	すみません。 
\\	しゅうり
\\	ひつよう
\\	止まります
\\	もしもし
\\	すぐ
\\	トイレ
\\	ながれます
\\	こわれます
\\	もしもし、401ごうしつのエミリー・マーティンですが、 トイレがこわれました。	
\\	もしもし、401ごうしつのエミリー・マーティンですが、 トイレがこわれました。 
\\	水がながれませんか?	
\\	水がながれませんか? 
\\	いいえ、水が止まりません。	
\\	いいえ、水が止まりません。 
\\	ええ!?すぐ、行きます!	
\\	ええ!?すぐ、行きます! 
\\	水が止まりました。でも、しゅうりがひつようですね。	
\\	水が止まりました。でも、しゅうりがひつようですね。 
\\	あした、しゅうりやにでんわしますね。	
\\	あした、しゅうりやにでんわしますね。 
\\	おねがいします。	
\\	おねがいします。 
\\	しつれいします。
\\	ばんごう
\\	でんわ
\\	よろしくおねがいします
\\	でんわばんごう
\\	では
\\	もしもし、イノベーティブリペアの木山です。	
\\	もしもし、イノベーティブリペアの木山です。 
\\	トイレのしゅうりのことで、でんわしました。	
\\	トイレのしゅうりのことで、でんわしました。 
\\	すみませんが、あとで、でんわをおねがいします。	
\\	すみませんが、あとで、でんわをおねがいします。 
\\	でんわばんごうは、89-1234-567です。	
\\	でんわばんごうは、89-1234-567です。 
\\	何時
\\	きょう
\\	に
\\	あした
\\	うーん
\\	だいじょうぶ
\\	いい
\\	イノベーティブリペア、木山です。	
\\	イノベーティブリペア、木山です。 
\\	ですが、トイレのしゅうりのことで、でんわしました。	
\\	ですが、トイレのしゅうりのことで、でんわしました。 
\\	ああ、
\\	さん、こんにちは。	
\\	ああ、
\\	さん、こんにちは。 
\\	きょう、トイレのしゅうりに行きますが、だいじょうぶですか。	
\\	きょう、トイレのしゅうりに行きますが、だいじょうぶですか。 
\\	うーん。あしたがいいです。	
\\	うーん。あしたがいいです。 
\\	何時がいいですか。	
\\	何時がいいですか。 
\\	三時がいいです。	
\\	三時がいいです。 
\\	じゃ、あした、三時に行きます。	
\\	じゃ、あした、三時に行きます。 
\\	化け物
\\	垢舐め
\\	恐ろしい
\\	唾
\\	重い病気
\\	かかる
\\	軽い病気
\\	ねえちゃん、日本の中で一番怖い化け物は何?	
\\	ねえちゃん、日本の中で一番怖い化け物は何? 
\\	簡単だよ。垢舐めだよ。	
\\	簡単だよ。垢舐めだよ。 
\\	垢舐めってどういう化け物?	
\\	垢舐めってどういう化け物? 
\\	恐ろしい化け物だよ。お風呂が汚くなると垢舐めが出てくる。あいつは汚いものを食べるんだ。	
\\	恐ろしい化け物だよ。お風呂が汚くなると垢舐めが出てくる。あいつは汚いものを食べるんだ。 
\\	いいんじゃない?掃除しなくていいから!	
\\	いいんじゃない?掃除しなくていいから! 
\\	そう思うでしょう。でも違うよ。あいつの唾には毒が入っていて、	
\\	そう思うでしょう。でも違うよ。あいつの唾には毒が入っていて、 
\\	怖い。	
\\	怖い。 
\\	そして、私たちのお風呂は汚いね。超汚い!	
\\	そして、私たちのお風呂は汚いね。超汚い! 
\\	僕がきれいにする!	
\\	僕がきれいにする! 
\\	(お姉さんに対して)ちゃんとお風呂きれいにした?	
\\	(お姉さんに対して)ちゃんとお風呂きれいにした? 
\\	今よしがやってる。	
\\	今よしがやってる。 
\\	よしが?おまえがやれっていっただろう。なんでよしがやってるんだ?	
\\	よしが?おまえがやれっていっただろう。なんでよしがやってるんだ? 
\\	彼がやりたいって。	
\\	彼がやりたいって。 
\\	よし、本当にやりたいのか?	
\\	よし、本当にやりたいのか? 
\\	怖いよ!早く掃除しないと。	
\\	怖いよ!早く掃除しないと。 
\\	わかった。やりなさい。彼はいい子だな。	
\\	わかった。やりなさい。彼はいい子だな。 
\\	冬
\\	夏のボーナス
\\	寿命
\\	増える
\\	減る
\\	万円
\\	平均
\\	ボーナス
\\	余裕
\\	四国
\\	貰う
\\	冬のボーナス
\\	来月は冬のボーナスを貰う。	
\\	来月は冬のボーナスを貰う。 
\\	こっちも!来月をとても楽しみにしてる。ところで、まだ四国旅行へ行くつもり?	
\\	こっちも!来月をとても楽しみにしてる。ところで、まだ四国旅行へ行くつもり? 
\\	うん、そう、そう!去年会社の冬のボーナスは平均67万5千円だった。	
\\	うん、そう、そう!去年会社の冬のボーナスは平均67万5千円だった。 
\\	今年はまた増えるから、お金は余裕がある。	
\\	今年はまた増えるから、お金は余裕がある。 
\\	薔薇の木
\\	象徴
\\	枯れる
\\	復活
\\	僕があげた薔薇の木はどこ?	
\\	僕があげた薔薇の木はどこ? 
\\	こっち。	
\\	こっち。 
\\	ひどい!!死にそう!これは僕らの愛の象徴だよ。	
\\	ひどい!!死にそう!これは僕らの愛の象徴だよ。 
\\	僕らの愛も死にそう?	
\\	僕らの愛も死にそう? 
\\	冬だから花も愛も枯れるのよ。	
\\	冬だから花も愛も枯れるのよ。 
\\	じゃ、春には復活するんだね。	
\\	じゃ、春には復活するんだね。 
\\	おかしい
\\	約束
\\	ちょうど
\\	点く
\\	折角
\\	誕生日
\\	はい、ここです。(ベルがなる音)	
\\	はい、ここです。(ベルがなる音) 
\\	おかしいな。何で出ないのかなあ?	
\\	おかしいな。何で出ないのかなあ? 
\\	何時の約束?	
\\	何時の約束? 
\\	8時。	
\\	8時。 
\\	今は?	
\\	今は? 
\\	ちょうど8時。おかしいなあ。	
\\	ちょうど8時。おかしいなあ。 
\\	電気もついていないし、彼はいなさそうだ。	
\\	電気もついていないし、彼はいなさそうだ。 
\\	彼の車もなさそう。彼はどうしたんだろう。	
\\	彼の車もなさそう。彼はどうしたんだろう。 
\\	せっかく君の誕生日なのにね。	
\\	せっかく君の誕生日なのにね。 
\\	結婚
\\	記念日
\\	新婚旅行
\\	出発日
\\	切符
\\	枚
\\	可能
\\	再来週
\\	いらっしゃいませ。	
\\	いらっしゃいませ。 
\\	来月私の結婚50年記念日で。妻と一緒に新婚旅行と同じところへもう一度行きたいんです。	
\\	来月私の結婚50年記念日で。妻と一緒に新婚旅行と同じところへもう一度行きたいんです。 
\\	おめでとうございます。どちらのほうへ?出発日はいつでしょうか。	
\\	おめでとうございます。どちらのほうへ?出発日はいつでしょうか。 
\\	グアムへ。出発日は来週の月曜日でお願いします。	
\\	グアムへ。出発日は来週の月曜日でお願いします。 
\\	少々お待ちください。申し訳ございません。その日は切符が一枚しかございません。	
\\	少々お待ちください。申し訳ございません。その日は切符が一枚しかございません。 
\\	火曜日は?	
\\	火曜日は? 
\\	申し訳ございません。その日は切符が一枚もございません。	
\\	申し訳ございません。その日は切符が一枚もございません。 
\\	水曜日は?	
\\	水曜日は? 
\\	信じられません。また、切符が一枚しかございません。	
\\	信じられません。また、切符が一枚しかございません。 
\\	行ける日はいつですか。	
\\	行ける日はいつですか。 
\\	少々お待ちください。再来週の月曜日はあいています。	
\\	少々お待ちください。再来週の月曜日はあいています。 
\\	お願いします。	
\\	お願いします。 
\\	神戸
\\	神戸ルミナリエ
\\	きらきら
\\	雰囲気
\\	何よりも
\\	おしゃれ
\\	まるで
\\	銀座
\\	とても不満
\\	先週末神戸へ行ってきました。	
\\	先週末神戸へ行ってきました。 
\\	神戸ですか。	
\\	神戸ですか。 
\\	そうですよ。神戸ルミナリエって聞いたことありますか。	
\\	そうですよ。神戸ルミナリエって聞いたことありますか。 
\\	ええ、見てきましたか。	
\\	ええ、見てきましたか。 
\\	見てきましたよ。	
\\	見てきましたよ。 
\\	いいな~。ええ、どうでしたか。	
\\	いいな~。ええ、どうでしたか。 
\\	最高でした。何よりも、夜の街の雰囲気は一番よかったです。道のライトがきらきら光っていて、おしゃれな人が多くて、	
\\	最高でした。何よりも、夜の街の雰囲気は一番よかったです。道のライトがきらきら光っていて、おしゃれな人が多くて、 
\\	あの、先週末彼女と銀座へ行ってきましたが、雰囲気はちょっとちがいますね。	
\\	あの、先週末彼女と銀座へ行ってきましたが、雰囲気はちょっとちがいますね。 
\\	そうですね。	
\\	そうですね。 
\\	泥棒
\\	お先にどうぞ
\\	レディーファースト
\\	誕生日
\\	びっくりパーティー
\\	秘密
\\	大成功
\\	あれ!誰もいなさそうだけど、ドアが開いている。	
\\	あれ!誰もいなさそうだけど、ドアが開いている。 
\\	開いている?	
\\	開いている? 
\\	おかしいなあ~。泥棒?	
\\	おかしいなあ~。泥棒? 
\\	わからない。調べてみよう。	
\\	わからない。調べてみよう。 
\\	危ないかもしれない。	
\\	危ないかもしれない。 
\\	そうだね。じゃ、お先にどうぞ。レディファースト。	
\\	そうだね。じゃ、お先にどうぞ。レディファースト。 
\\	「ドアが開く音」	
\\	ただ今!誰かいるの?警察を呼ぶわよ。	
\\	ただ今!誰かいるの?警察を呼ぶわよ。 
\\	誕生日おめでとうございます!!!	
\\	誕生日おめでとうございます!!! 
\\	ああ。。。	
\\	ああ。。。 
\\	誕生日おめでとう!	
\\	誕生日おめでとう! 
\\	誕生日おめでとう!	
\\	誕生日おめでとう! 
\\	お誕生日おめでとうございます!	
\\	お誕生日おめでとうございます! 
\\	信じられない!びっくりパーティー、私のために?	
\\	信じられない!びっくりパーティー、私のために? 
\\	一生懸命秘密にしていたんだよ。	
\\	一生懸命秘密にしていたんだよ。 
\\	大成功。泣きそうになった。本当にありがとう。皆さんありがとう。	
\\	大成功。泣きそうになった。本当にありがとう。皆さんありがとう。 
\\	今年、何歳?	
\\	今年、何歳? 
\\	うるさい!	
\\	うるさい! 
\\	ピカピカ
\\	床
\\	鏡
\\	掃除
\\	旦那
\\	専業主夫
\\	コーヒー
\\	ケーキ
\\	羨ましい
\\	お邪魔します!	
\\	お邪魔します! 
\\	どうぞ。	
\\	どうぞ。 
\\	わ~素敵な家。この家はピカピカね。この床は鏡みたい!どうやって掃除するの?	
\\	わ~素敵な家。この家はピカピカね。この床は鏡みたい!どうやって掃除するの? 
\\	えっ、そう?気づかなかったわ。	
\\	えっ、そう?気づかなかったわ。 
\\	気づかなかったってどういうこと?	
\\	気づかなかったってどういうこと? 
\\	私は掃除しないから。	
\\	私は掃除しないから。 
\\	掃除しない?じゃあ誰が掃除するの?	
\\	掃除しない?じゃあ誰が掃除するの? 
\\	お帰りなさい。	
\\	お帰りなさい。 
\\	只今。金田さん、こちらは私の主人です。	
\\	只今。金田さん、こちらは私の主人です。 
\\	初めまして。金田です。宜しくお願いします。	
\\	初めまして。金田です。宜しくお願いします。 
\\	こちらこそ。宜しくお願いします。	
\\	こちらこそ。宜しくお願いします。 
\\	コーヒー二つね。	
\\	コーヒー二つね。 
\\	はい、はい、わかりました。いまケーキを焼いているところだから後でその手作りケーキも持っていきます。でもとりあえず、コーヒを持っていきますね。	
\\	はい、はい、わかりました。いまケーキを焼いているところだから後でその手作りケーキも持っていきます。でもとりあえず、コーヒを持っていきますね。 
\\	ありがとう。	
\\	ありがとう。 
\\	すみません。	
\\	すみません。 
\\	どうして?いいの?今のは?	
\\	どうして?いいの?今のは? 
\\	いいのよ。旦那は専業主夫で、なんでも真剣にやるタイプだから。	
\\	いいのよ。旦那は専業主夫で、なんでも真剣にやるタイプだから。 
\\	いいな〜。羨ましいな〜。	
\\	いいな〜。羨ましいな〜。 
\\	年末宝くじ
\\	可能性
\\	やり直せる
\\	もしかしたら
\\	兎に角
\\	億
\\	趣味
\\	将来性
\\	貧乏
\\	ふる
\\	アイディア
\\	聞いた?聞いた?	
\\	聞いた?聞いた? 
\\	どうしたの?	
\\	どうしたの? 
\\	山田さんが年末宝くじ当たったらしい!	
\\	山田さんが年末宝くじ当たったらしい! 
\\	マジで!	
\\	マジで! 
\\	うん、3億円だよ!!!	
\\	うん、3億円だよ!!! 
\\	3億円?	
\\	3億円? 
\\	君たち二人は幸せだよね。ええ。。。どうしたの?すごい顔してるけど。	
\\	君たち二人は幸せだよね。ええ。。。どうしたの?すごい顔してるけど。 
\\	昨日彼女をふった。	
\\	昨日彼女をふった。 
\\	なんだって?	
\\	なんだって? 
\\	彼女は運が悪くて、貧乏で将来性がないと思って、もうお婆さんだし。服の趣味も悪いし。	
\\	彼女は運が悪くて、貧乏で将来性がないと思って、もうお婆さんだし。服の趣味も悪いし。 
\\	わかった。わかったよ。	
\\	わかった。わかったよ。 
\\	まあ、とにかく、昨日彼女と別れたんだ。しまったな。	
\\	まあ、とにかく、昨日彼女と別れたんだ。しまったな。 
\\	彼女は自分の誕生日と君の誕生日を使ったらしい。もしかしたら。。。やり直せる可能性があるかもしれない。電話してみたら。。。	
\\	彼女は自分の誕生日と君の誕生日を使ったらしい。もしかしたら。。。やり直せる可能性があるかもしれない。電話してみたら。。。 
\\	ああ、それはいいアイディアだ。さっそく電話してみよう。	
\\	ああ、それはいいアイディアだ。さっそく電話してみよう。 
\\	電話番号
\\	後悔
\\	ずっと
\\	っぱなし
\\	芸能人
\\	有名な人
\\	ちょっと
\\	たけさんですか。	
\\	たけさんですか。 
\\	はい、西川たけですが。のりこさんはいますか。	
\\	はい、西川たけですが。のりこさんはいますか。 
\\	またか。のりこさんはいません。	
\\	またか。のりこさんはいません。 
\\	この電話番号は060
\\	3412ですよね。	
\\	この電話番号は060
\\	3412ですよね。 
\\	はい、そうです。	
\\	はい、そうです。 
\\	や、僕、のりこさんの前の彼氏なんですが。	
\\	や、僕、のりこさんの前の彼氏なんですが。 
\\	でも、私はのりこさんを知りません。昨日この電話を買ったばっかりです。	
\\	でも、私はのりこさんを知りません。昨日この電話を買ったばっかりです。 
\\	ああ、そうなんですか。	
\\	ああ、そうなんですか。 
\\	とても後悔しています。今日ずっと鳴っています。本当に鳴りっぱなし!のりこさんってどういう人ですか。芸能人ですか。有名な人でしょうか。	
\\	とても後悔しています。今日ずっと鳴っています。本当に鳴りっぱなし!のりこさんってどういう人ですか。芸能人ですか。有名な人でしょうか。 
\\	やあ、ちょっとわからないです。いや、では、どうもありがとうございます。	
\\	やあ、ちょっとわからないです。いや、では、どうもありがとうございます。 
\\	いいえ、どういたしまして。	
\\	いいえ、どういたしまして。 
\\	葉書
\\	本格的
\\	出張
\\	相変わらず
\\	埼玉
\\	返事
\\	ああ、佐々木さんから葉書が来た。	
\\	ああ、佐々木さんから葉書が来た。 
\\	いよいよ本格的な寒さになってまいりました。お元気でお過ごしでしょうか。	
\\	佐々木	
\\	懐かしい
\\	お手紙
\\	実は
\\	ご無沙汰
\\	詳しい
\\	返事が来た。	
\\	返事が来た。 
\\	佐々木さん、ご無沙汰していました。お手紙ありがとうございました。	
\\	佐々木さん、ご無沙汰していました。お手紙ありがとうございました。 
\\	お知らせ
\\	ご婚約
\\	親切
\\	立派
\\	奥様
\\	やっと返事が来た。	
\\	やっと返事が来た。 
\\	春とは名ばかりでまだ厳しい寒さが続いています。しかし、石原さんの良いお知らせを読んだら、心があたたかくなりました。	
\\	季節
\\	確認
\\	出張
\\	複雑
\\	初めて
\\	前回
\\	場所
\\	参る
\\	心配
\\	葉書
\\	青葉
\\	若葉
\\	すがすがしい
\\	方
\\	無事に
\\	訪ねる
\\	宣言
\\	受ける
\\	皮肉
\\	少々
\\	調べる
\\	かしこまりました
\\	申し訳ございません
\\	空いている
\\	7時半発
\\	会計
\\	(電話をかける音)	
\\	はい、
\\	の中谷でございます。	
\\	はい、
\\	の中谷でございます。 
\\	もしもし、すぐハワイへいきたいんですが。	
\\	もしもし、すぐハワイへいきたいんですが。 
\\	はい、かしこまりました。いつがよろしいでしょうか。	
\\	はい、かしこまりました。いつがよろしいでしょうか。 
\\	今日はどうですか。	
\\	今日はどうですか。 
\\	はい、かしこまりました。お調べいたしますので、少々お待ちください。。。	
\\	はい、かしこまりました。お調べいたしますので、少々お待ちください。。。 
\\	明日は?	
\\	明日は? 
\\	明日はございませんが、明後日一番早い便に席が空いています。	
\\	明日はございませんが、明後日一番早い便に席が空いています。 
\\	お願いします。	
\\	お願いします。 
\\	ありがとうございました。帰りの日はいつでしょうか。	
\\	ありがとうございました。帰りの日はいつでしょうか。 
\\	いつでもいいです。	
\\	いつでもいいです。 
\\	来週の火曜日に席がございます。	
\\	来週の火曜日に席がございます。 
\\	はい、お願いします。	
\\	はい、お願いします。 
\\	では、明後日全日空、7時半発、ホノルル行きをおとりしておきます、	
\\	では、明後日全日空、7時半発、ホノルル行きをおとりしておきます、 
\\	結婚式
\\	馬鹿
\\	控え室
\\	婚約者
\\	試す
\\	着替える
\\	佐々木さんの結婚式はどこですか。	
\\	佐々木さんの結婚式はどこですか。 
\\	チャペルのほうですが、今佐々木さんは控え室のほうにいます。	
\\	チャペルのほうですが、今佐々木さんは控え室のほうにいます。 
\\	(ノックの音)	
\\	佐々木さん!	
\\	佐々木さん! 
\\	どうぞ。お入りください。	
\\	どうぞ。お入りください。 
\\	佐々木さん、あなたは結婚してはいけません!	
\\	佐々木さん、あなたは結婚してはいけません! 
\\	なんでですか。	
\\	なんでですか。 
\\	僕は思い当たるすべてのチャペルを探し続けて、今やっとあなたを見つけたんです。大変でした。	
\\	僕は思い当たるすべてのチャペルを探し続けて、今やっとあなたを見つけたんです。大変でした。 
\\	だから、ちゃんと理由を言ってください。	
\\	だから、ちゃんと理由を言ってください。 
\\	理由は、わたくしがあなたのことを愛しています。ずっと愛してきました。	
\\	理由は、わたくしがあなたのことを愛しています。ずっと愛してきました。 
\\	もう一度愛していますと言ってください。	
\\	もう一度愛していますと言ってください。 
\\	愛しています。	
\\	愛しています。 
\\	もっと大きな声で。	
\\	もっと大きな声で。 
\\	大きな声で愛しています!	
\\	大きな声で愛しています! 
\\	うるせえな~。	
\\	うるせえな~。 
\\	すみません。	
\\	すみません。 
\\	もう一度、私は馬鹿だって言ってください。	
\\	もう一度、私は馬鹿だって言ってください。 
\\	私は馬鹿だ。	
\\	私は馬鹿だ。 
\\	はい、石原さんのプロポーズを受けます。	
\\	はい、石原さんのプロポーズを受けます。 
\\	本当?	
\\	本当? 
\\	はい。	
\\	はい。 
\\	婚約者はどこ?	
\\	婚約者はどこ? 
\\	ここにいます。	
\\	ここにいます。 
\\	でも、二人しかいない。	
\\	でも、二人しかいない。 
\\	はい、私の婚約者はあなたです。	
\\	はい、私の婚約者はあなたです。 
\\	どういうこと?	
\\	どういうこと? 
\\	私もあなたを試しました。	
\\	私もあなたを試しました。 
\\	えええ!!	
\\	えええ!! 
\\	すみません、石原さんのタキシードを持ってきてください。	
\\	すみません、石原さんのタキシードを持ってきてください。 
\\	提出日
\\	せまってきている
\\	完全に
\\	育児
\\	消息不明
\\	支局
\\	つないで
\\	対策を練る
\\	(オフィスにて)社長、例の仕事ですが、提出日が来月とせまってきているので、そろそろ対策を練りましょう。	
\\	(オフィスにて)社長、例の仕事ですが、提出日が来月とせまってきているので、そろそろ対策を練りましょう。 
\\	うーむ。どうしたものか…。	
\\	うーむ。どうしたものか…。 
\\	大事な取引先かと思いますので、エースの木ノ下にやらせましょう。	
\\	大事な取引先かと思いますので、エースの木ノ下にやらせましょう。 
\\	うーむ。木ノ下か。よし!木ノ下を呼んでくれ。	
\\	うーむ。木ノ下か。よし!木ノ下を呼んでくれ。 
\\	なにー!中国だあ!?この忙しい時期に奴は何をやってるんだ。上海支局の李につないでくれ。木ノ下を探すんだ。	
\\	なにー!中国だあ!?この忙しい時期に奴は何をやってるんだ。上海支局の李につないでくれ。木ノ下を探すんだ。 
\\	かどうか
\\	放浪癖
\\	すら
\\	気まぐれ
\\	なんとしても
\\	大の〜好き
\\	行方
\\	社長、木ノ下がまだ中国にいるかどうかはわかりません。	
\\	社長、木ノ下がまだ中国にいるかどうかはわかりません。 
\\	さっき中国にいると言ったじゃないか。	
\\	さっき中国にいると言ったじゃないか。 
\\	実は、木ノ下には放浪癖があるんです。アジアの国にいるかどうかすら、我々にははっきりとわからないのです。彼が行くと言った場所に本当に行くかどうかは誰にもわかりません。気まぐれな男なんです。	
\\	実は、木ノ下には放浪癖があるんです。アジアの国にいるかどうかすら、我々にははっきりとわからないのです。彼が行くと言った場所に本当に行くかどうかは誰にもわかりません。気まぐれな男なんです。 
\\	確か・・・木ノ下は大のチベット仏教好きだったな。よし!チベット支局のテンバにつないでくれ。木ノ下の行方を追うんだ。	
\\	健やか
\\	証
\\	誓約
\\	会衆
\\	病める
\\	誓う
\\	生涯
\\	尽くす
\\	いたわる
\\	牧師
\\	石原さんは、佐々木さんを妻として、健やかなる時も病める時も、いたわり、尽くし、生涯を共にすることを誓いますか?	
\\	石原さんは、佐々木さんを妻として、健やかなる時も病める時も、いたわり、尽くし、生涯を共にすることを誓いますか? 
\\	誓います。	
\\	誓います。 
\\	佐々木さんは、石原さんを夫とし、健やかなる時も病める時も、いたわり、尽くし、生涯を共にすることを誓いますか?	
\\	佐々木さんは、石原さんを夫とし、健やかなる時も病める時も、いたわり、尽くし、生涯を共にすることを誓いますか? 
\\	誓います。	
\\	誓います。 
\\	(誓いの口づけをする)	
\\	小判
\\	高価
\\	価値
\\	ありがたい
\\	ことわざ
\\	ほとんど
\\	真珠
\\	お母さん、今日、学校で、ことわざを習ったよ。	
\\	お母さん、今日、学校で、ことわざを習ったよ。 
\\	どんなことわざ?	
\\	どんなことわざ? 
\\	え~と、「ネコに小判」と「ブタに真珠」だよ。	
\\	え~と、「ネコに小判」と「ブタに真珠」だよ。 
\\	へぇ~、どんな意味なの?	
\\	へぇ~、どんな意味なの? 
\\	どっちもほとんど同じ意味なんだって。ネコに小判をあげても、ブタに真珠をあげても、ありがたいと思わないでしょ。だから、どんなに価値があるものでも、その価値がわからない人には何の価値もないっていう意味なんだって。	
\\	どっちもほとんど同じ意味なんだって。ネコに小判をあげても、ブタに真珠をあげても、ありがたいと思わないでしょ。だから、どんなに価値があるものでも、その価値がわからない人には何の価値もないっていう意味なんだって。 
\\	そうね。今のタクミに、2万円もする高価なバットを買ってあげても、ブタに真珠っていうことね。	
\\	そうね。今のタクミに、2万円もする高価なバットを買ってあげても、ブタに真珠っていうことね。 
\\	えっ!そんなことないよ!	
\\	えっ!そんなことないよ! 
\\	盛り上がる
\\	はりきる
\\	美男子
\\	一体
\\	分厚い本
\\	オタク系
\\	筆頭
\\	謎
\\	昨日の同窓会は、結構盛り上がったな。ちょっと、はりきりすぎたよ。	
\\	昨日の同窓会は、結構盛り上がったな。ちょっと、はりきりすぎたよ。 
\\	いやあ〜、昨日は飲みすぎて、まだ気分が悪いよ。二日酔いってやつ。	
\\	いやあ〜、昨日は飲みすぎて、まだ気分が悪いよ。二日酔いってやつ。 
\\	本当に久しぶりだったよな。	
\\	本当に久しぶりだったよな。 
\\	15年振りくらいじゃないかしら。	
\\	15年振りくらいじゃないかしら。 
\\	みんな変わったよなあ。クラスのマドンナだったナオミちゃんなんて、最初見た時、ナオミちゃんかどうか、わからなかったもん。昔はスリムだったのに、太りすぎ。女って変わるよなあ…。	
\\	みんな変わったよなあ。クラスのマドンナだったナオミちゃんなんて、最初見た時、ナオミちゃんかどうか、わからなかったもん。昔はスリムだったのに、太りすぎ。女って変わるよなあ…。 
\\	でも、一番の驚きは木ノ下だったでしょ。最初入って来た時、この美男子は一体誰だ?って・・・まさか木ノ下とは思いもしなかった。	
\\	"でも、一番の驚きは木ノ下だったでしょ。最初入って来た時、この美男子は一体誰だ?って・・・まさか木ノ下とは思いもしなかった。 
\\	高校の時は、いつも髪ボサボサで、服はよれよれで、(分厚い本しか読んでなくて)オタク系の筆頭だったのにな。いったい、奴には何があったんだろう。	
\\	高校の時は、いつも髪ボサボサで、服はよれよれで、(分厚い本しか読んでなくて)オタク系の筆頭だったのにな。いったい、奴には何があったんだろう。 
\\	謎だね・・・。	
\\	謎だね・・・。 
\\	度肝
\\	世界一周
\\	不思議
\\	徹底的
\\	遺跡
\\	木ノ下に電話してみようぜ。	
\\	木ノ下に電話してみようぜ。 
\\	(木ノ下に電話)	
\\	もしもし、木ノ下ですが。やあイワオくん。どうしたの?	
\\	もしもし、木ノ下ですが。やあイワオくん。どうしたの? 
\\	おう木ノ下。いやあ、昨日は懐かしかったな。それにしても、木ノ下の変わりようには度肝を抜かれたよ(面食らったよ)。一体どうしてたんだよ。その辺、話してくれよ。	
\\	おう木ノ下。いやあ、昨日は懐かしかったな。それにしても、木ノ下の変わりようには度肝を抜かれたよ(面食らったよ)。一体どうしてたんだよ。その辺、話してくれよ。 
\\	ははは。確かに、高校の時の自分とは全然違ってうつるみたいだね。実はあれからすぐ、世界一周の旅に出たんだ。途中、南米のペルーにある、世界遺産のマチュピチュ遺跡で、不思議な女性に出会ったんだ。なんと、あのロックフェラー財団の幹部だった人で、その人から、ありとあらゆることを徹底的に教え込まれたわけなんだ。	
\\	ははは。確かに、高校の時の自分とは全然違ってうつるみたいだね。実はあれからすぐ、世界一周の旅に出たんだ。途中、南米のペルーにある、世界遺産のマチュピチュ遺跡で、不思議な女性に出会ったんだ。なんと、あのロックフェラー財団の幹部だった人で、その人から、ありとあらゆることを徹底的に教え込まれたわけなんだ。 
\\	えっ、何があったんだよ!	
\\	えっ、何があったんだよ! 
\\	電話では長くなりそうだから、続きはまた今度会った時に話すよ。	
\\	電話では長くなりそうだから、続きはまた今度会った時に話すよ。 
\\	おいおい。気になって眠れないよ。じゃあ、時間がある時に連絡しろよ!	
\\	おいおい。気になって眠れないよ。じゃあ、時間がある時に連絡しろよ! 
\\	載せる
\\	正直な
\\	抜く
\\	大好物
\\	野郎
\\	隠れる
\\	無駄
\\	縛りつける
\\	尋問
\\	ここにあったはずの天ぷら、誰が食べたの〜?夕飯のうどんに載せるはずだったのに〜。	
\\	ここにあったはずの天ぷら、誰が食べたの〜?夕飯のうどんに載せるはずだったのに〜。 
\\	僕じゃないよ。今帰ってきたばっかりだもん。	
\\	僕じゃないよ。今帰ってきたばっかりだもん。 
\\	僕でもないよ。ずっとあっちでプレステやってたもん。	
\\	僕でもないよ。ずっとあっちでプレステやってたもん。 
\\	かぼちゃのてんぷらでしょう。僕、かぼちゃ嫌いだもん。食べるはずないよ〜。	
\\	かぼちゃのてんぷらでしょう。僕、かぼちゃ嫌いだもん。食べるはずないよ〜。 
\\	正直に言わないと、あんた達、夕飯抜きよ!	
\\	(一斉に)僕たちじゃないよー!	
\\	(一斉に)僕たちじゃないよー! 
\\	(キレるお母さん)	
\\	あの野郎〜!!お父さんはどこなの?隠れても無駄よ。家のどこかにいるはずよ。連れてきなさい!みんなで、縛りつけて尋問するわよ!	
\\	あの野郎〜!!お父さんはどこなの?隠れても無駄よ。家のどこかにいるはずよ。連れてきなさい!みんなで、縛りつけて尋問するわよ! 
\\	異常気象
\\	熱い
\\	暑い
\\	雨雲
\\	晴れ
\\	陰る
\\	梅雨
\\	雨空
\\	蒸し暑い
\\	むしむし
\\	今日って少し寒くない?	
\\	今日って少し寒くない? 
\\	もう5月なのに寒いよね。今年って異常気象って話だよ。	
\\	もう5月なのに寒いよね。今年って異常気象って話だよ。 
\\	さっきまで、雨空だったのに、雨が止んで、陽が出て来て、むしむしするよー。	
\\	確かに、蒸し暑くなってきたね。もう梅雨入りかなあ。今年は桜も早かったしね。	
\\	確かに、蒸し暑くなってきたね。もう梅雨入りかなあ。今年は桜も早かったしね。 
\\	あっ、また陽が陰ってきた。なんだか今日は、イギリスの天気みたいだなあ。	
\\	あっ、また陽が陰ってきた。なんだか今日は、イギリスの天気みたいだなあ。 
\\	えっ、どういう意味?	
\\	えっ、どういう意味? 
\\	晴れても、曇りになったり、雨が降ったり。寒くなったり、暑くなったり。	
\\	それだけ、変わりやすいってことなんだね。	
\\	それだけ、変わりやすいってことなんだね。 
\\	そうなんだよ。	
\\	そうなんだよ。 
\\	なるほどー。	
\\	なるほどー。 
\\	レジ袋
\\	割引
\\	削減
\\	影響
\\	無駄
\\	たまる
\\	スーパーに買い物に行ってきてくれる?このマイバッグを持って行ってよね。レジ袋はもらわないで、このカードにスタンプを押してもらってきてよ。	
\\	スーパーに買い物に行ってきてくれる?このマイバッグを持って行ってよね。レジ袋はもらわないで、このカードにスタンプを押してもらってきてよ。 
\\	はいはい。ところで、このスタンプを集めると何かいいことあるの?	
\\	はいはい。ところで、このスタンプを集めると何かいいことあるの? 
\\	20個たまったら、100円割引してくれるのよ。	
\\	20個たまったら、100円割引してくれるのよ。 
\\	ということは、1枚5円っていうことか。でも、レジ袋を削減したところで、それほど環境にいい影響があるとは思えないなぁ~。	
\\	ということは、1枚5円っていうことか。でも、レジ袋を削減したところで、それほど環境にいい影響があるとは思えないなぁ~。 
\\	そんなことないわよ。そういう小さいことをしたところで無駄だと思うことが良くないのよ!	
\\	そんなことないわよ。そういう小さいことをしたところで無駄だと思うことが良くないのよ! 
\\	デリバリー
\\	頼む
\\	登録
\\	初めて
\\	店員
\\	住所
\\	ねえ。お腹すいたから、ピザでも食べない。	
\\	ねえ。お腹すいたから、ピザでも食べない。 
\\	うん、いいねえ。デリバリーで、ピザ頼もうか。	
\\	うん、いいねえ。デリバリーで、ピザ頼もうか。 
\\	お電話ありがとうございます。ダイナマイト・ピザです。お電話番号からお願いします。	
\\	お電話ありがとうございます。ダイナマイト・ピザです。お電話番号からお願いします。 
\\	あのー、初めてなんですが。	
\\	あのー、初めてなんですが。 
\\	それでは、最初に電話番号と住所とお名前を登録させてください。	
\\	それでは、最初に電話番号と住所とお名前を登録させてください。 
\\	はい。電話番号は03-3456-7890で、住所は港区赤坂3-4-1です。名前はユウです。	
\\	はい。電話番号は03-3456-7890で、住所は港区赤坂3-4-1です。名前はユウです。 
\\	ありがとうございます。登録させていただきました。	
\\	ありがとうございます。登録させていただきました。 
\\	注文
\\	伺う
\\	生地
\\	種類
\\	お届け
\\	売り物
\\	半額
\\	すぎる
\\	では、ご注文をお伺いします。ピザ生地とサイズをお選び下さい。	
\\	では、ご注文をお伺いします。ピザ生地とサイズをお選び下さい。 
\\	ピザ生地とサイズは、どんな種類があるんですか。	
\\	ピザ生地とサイズは、どんな種類があるんですか。 
\\	生地は、パン生地とクリスピー生地がございます。クリスピー生地は、うすい生地です。サイズは、
\\	と
\\	の2種類です。
\\	は約25
\\	で、
\\	は約31
\\	です。	
\\	生地は、パン生地とクリスピー生地がございます。クリスピー生地は、うすい生地です。サイズは、
\\	と
\\	の2種類です。
\\	は約25
\\	で、
\\	は約31
\\	です。 
\\	生地はパン生地で、サイズは
\\	でお願いします。	
\\	生地はパン生地で、サイズは
\\	でお願いします。 
\\	かしこまりました。それでは、ピザの種類をお選びください。	
\\	かしこまりました。それでは、ピザの種類をお選びください。 
\\	えーっと、マリゲリータと、ペスカトーレの、ハーフ&ハーフでお願いします。	
\\	えーっと、マリゲリータと、ペスカトーレの、ハーフ&ハーフでお願いします。 
\\	かしこまりました。ご一緒にサイドメニューはいかがですか。	
\\	かしこまりました。ご一緒にサイドメニューはいかがですか。 
\\	コーラを三つお願いします。	
\\	コーラを三つお願いします。 
\\	ありがとうございます。合計で4,000円です。45分ほどでお届けに上がります。スピードが売り物なので、45分を過ぎましたら半額にサービスさせていただきます。	
\\	ありがとうございます。合計で4,000円です。45分ほどでお届けに上がります。スピードが売り物なので、45分を過ぎましたら半額にサービスさせていただきます。 
\\	ほんとうですか!	
\\	ほんとうですか! 
\\	ご注文ありがとうございました。	
\\	ご注文ありがとうございました。 
\\	見逃す
\\	お腹
\\	我慢
\\	新聞
\\	編み物
\\	ご飯を食べながらテレビを見るのはやめなさい。	
\\	ご飯を食べながらテレビを見るのはやめなさい。 
\\	えぇっ、サザエさんを見逃すなんていやだよぉ。	
\\	えぇっ、サザエさんを見逃すなんていやだよぉ。 
\\	じゃあ、サザエさんを見た後で、ご飯にしなさい。	
\\	じゃあ、サザエさんを見た後で、ご飯にしなさい。 
\\	えぇっ、お腹もペコペコだよ。昼から何も食べてないから、後30分も我慢できないよ。	
\\	えぇっ、お腹もペコペコだよ。昼から何も食べてないから、後30分も我慢できないよ。 
\\	ダメよ。ダメ。どっちかにしなさい!	
\\	ダメよ。ダメ。どっちかにしなさい! 
\\	えぇっ、お父さんはいつも新聞読みながら、ご飯たべてるよ。お母さんだって、いつも編み物しながら、テレビみてるじゃないかあ。	
\\	えぇっ、お父さんはいつも新聞読みながら、ご飯たべてるよ。お母さんだって、いつも編み物しながら、テレビみてるじゃないかあ。 
\\	お母さんの言うことをきけないなら、今夜のカレーライスはぬきよ!	
\\	お母さんの言うことをきけないなら、今夜のカレーライスはぬきよ! 
\\	いつも通り
\\	きっちり
\\	捨てる
\\	残す
\\	耐える
\\	女々しい
\\	軟弱
\\	フラれる
\\	自分自身
\\	毛先
\\	見返す
\\	今回もいつも通りでよろしいですか。	
\\	今回もいつも通りでよろしいですか。 
\\	この写真みたいにしてください。	
\\	この写真みたいにしてください。 
\\	いいんです。	
\\	いいんです。 
\\	あのー。いつも毛先を少し切るだけじゃないですか。	
\\	あのー。いつも毛先を少し切るだけじゃないですか。 
\\	僕は自分自身を変えたいんです。	
\\	僕は自分自身を変えたいんです。 
\\	失礼ですけど、何かあったんですか。	
\\	失礼ですけど、何かあったんですか。 
\\	(小声で)恋人にフラレたとか・・・。	
\\	(小声で)恋人にフラレたとか・・・。 
\\	彼女は僕のことを軟弱だ、なんて言ったんです。「あなたの女々しいところがもう耐えられない」なんて捨てゼリフを残して、僕を捨てたんだ。だから、だから僕は・・・(涙がこみ上げてくる)。	
\\	"彼女は僕のことを軟弱だ、なんて言ったんです。「あなたの女々しいところがもう耐えられない」なんて捨てゼリフを残して、僕を捨てたんだ。だから、だから僕は・・・(涙がこみ上げてくる)。 
\\	わかりました。きっちり、男らしいモヒカン・ヘアーにさせていただきます。キーワードは「パンク」でいきます!彼女を見返してやりましょう。	
\\	わかりました。きっちり、男らしいモヒカン・ヘアーにさせていただきます。キーワードは「パンク」でいきます!彼女を見返してやりましょう。 
\\	お願いします!	
\\	お願いします! 
\\	派出所
\\	左手
\\	おいおい
\\	迷う
\\	そうすると
\\	番目
\\	曲がる
\\	真っすぐ
\\	行き方
\\	聞く
\\	派出所にて	
\\	派出所にて 
\\	すみません。このユースホステルへの行き方を教えてもらいたいのですが。	
\\	すみません。このユースホステルへの行き方を教えてもらいたいのですが。 
\\	(地図を見せる)	
\\	(地図を見せる) 
\\	う〜ん、どれどれ。う〜ん・・・う〜ん。	
\\	う〜ん、どれどれ。う〜ん・・・う〜ん。 
\\	(10分経過)	
\\	(10分経過) 
\\	えーっと、ここを真っすぐ行ってですねー、四番目の信号を右に曲がってください。四番目ですよ。	
\\	えーっと、ここを真っすぐ行ってですねー、四番目の信号を右に曲がってください。四番目ですよ。 
\\	はい。	
\\	はい。 
\\	そうすると、左手に見えますので。	
\\	そうすると、左手に見えますので。 
\\	なんだ。一時間ず〜っと迷ってたのに、そんな簡単に行けるんですか。	
\\	なんだ。一時間ず〜っと迷ってたのに、そんな簡単に行けるんですか。 
\\	運動会
\\	頑張る
\\	優勝
\\	一生懸命
\\	徒競走
\\	負ける
\\	喜び
\\	半分
\\	はちまき
\\	明日は運動会だ。みんな頑張るように。	
\\	明日は運動会だ。みんな頑張るように。 
\\	先生、うちのクラスは何組ですか。	
\\	先生、うちのクラスは何組ですか。 
\\	うちのクラスは赤組だ。みんな、白のハチマキを巻いたらダメだぞー。	
\\	うちのクラスは赤組だ。みんな、白のハチマキを巻いたらダメだぞー。 
\\	アリサ、運動会はどうだった。	
\\	アリサ、運動会はどうだった。 
\\	赤組は優勝したけど、私は徒競走で2位だったの。	
\\	赤組は優勝したけど、私は徒競走で2位だったの。 
\\	優勝かあ。すごいじゃん。	
\\	優勝かあ。すごいじゃん。 
\\	でも、一生懸命走ったのに徒競走で負けちゃったから、喜び半分って感じだわ。	
\\	でも、一生懸命走ったのに徒競走で負けちゃったから、喜び半分って感じだわ。 
\\	彼氏
\\	優しい
\\	感じ
\\	包容力
\\	柔和
\\	笑顔
\\	素敵
\\	ねえねえ、マミの彼氏どうだった。会ったんでしょう。	
\\	ねえねえ、マミの彼氏どうだった。会ったんでしょう。 
\\	そうねえ。優しそうな感じだったかな。	
\\	そうねえ。優しそうな感じだったかな。 
\\	優しそうな感じって、どういうこと。	
\\	優しそうな感じって、どういうこと。 
\\	包容力がありそうな感じ。	
\\	包容力がありそうな感じ。 
\\	包容力があるって・・・どんな感じなの。	
\\	包容力があるって・・・どんな感じなの。 
\\	すぐ怒ったりしなそうな感じ。	
\\	すぐ怒ったりしなそうな感じ。 
\\	へー。柔和で、笑顔が素敵そうなイメージだわ。	
\\	へー。柔和で、笑顔が素敵そうなイメージだわ。 
\\	コントローラ
\\	使い方
\\	簡単
\\	パンチ
\\	キック
\\	技
\\	必殺技
\\	ボスキャラ
\\	強い
\\	ねえ。コントローラーの使い方を教えてよ。	
\\	ねえ。コントローラーの使い方を教えてよ。 
\\	。簡単だよ。まず、
\\	ボタンを押してみて。	
\\	。簡単だよ。まず、
\\	ボタンを押してみて。 
\\	うん。パンチだね。	
\\	うん。パンチだね。 
\\	次は
\\	ボタン。	
\\	次は
\\	ボタン。 
\\	今度はキックだね。	
\\	今度はキックだね。 
\\	次は
\\	ボタン。	
\\	次は
\\	ボタン。 
\\	おっ、ジャンプだ。	
\\	おっ、ジャンプだ。 
\\	最後に、
\\	を一緒に押してみて。	
\\	最後に、
\\	を一緒に押してみて。 
\\	おっ、なんかすごい技がでたっ!	
\\	おっ、なんかすごい技がでたっ! 
\\	必殺技だからね。この必殺技は1ステージで2回しか使えないから、強いボスキャラの時に使った方がいいよ。	
\\	必殺技だからね。この必殺技は1ステージで2回しか使えないから、強いボスキャラの時に使った方がいいよ。 
\\	わかった!	
\\	わかった! 
\\	貸し出し
\\	提示
\\	在架
\\	図書館
\\	黄色
\\	必要事項
\\	記入
\\	(学生証を提示する)。	
\\	現在、7冊お借りになってますね。学部生の方は、7冊までしか貸し出しできないんです。	
\\	現在、7冊お借りになってますね。学部生の方は、7冊までしか貸し出しできないんです。 
\\	そうですか・・・。	
\\	そうですか・・・。 
\\	スタッフ:湘南の図書館にあるようですが、お取り寄せしますか。	
\\	はい。お願いします。	
\\	はい。お願いします。 
\\	問題なければ、明日の夕方くらいには届くと思います。	
\\	わかりました。ありがとうございます。	
\\	わかりました。ありがとうございます。 
\\	誰
\\	声変わり
\\	普段
\\	忘れっぽい
\\	女々しい
\\	昨日、久しぶりに実家に電話したんだけど、母さん、俺のこと最初誰だかわからなくてさあ。	
\\	昨日、久しぶりに実家に電話したんだけど、母さん、俺のこと最初誰だかわからなくてさあ。 
\\	あんた、声変わりでもしたんじゃないの。	
\\	あんた、声変わりでもしたんじゃないの。 
\\	まさか。	
\\	まさか。 
\\	ユウジ今、風邪ひいてるから、風邪声だっただけよ。	
\\	ユウジ今、風邪ひいてるから、風邪声だっただけよ。 
\\	そうかなあ。年をとると、忘れっぽくなるって言うから、俺の声も忘れちゃったのかと思った。	
\\	そうかなあ。年をとると、忘れっぽくなるって言うから、俺の声も忘れちゃったのかと思った。 
\\	何言ってんの。あんたの声は普段から子供っぽいっていうか、子供みたいに高いじゃない。でも今はガラガラ声だから誰でも間違うわよ。	
\\	何言ってんの。あんたの声は普段から子供っぽいっていうか、子供みたいに高いじゃない。でも今はガラガラ声だから誰でも間違うわよ。 
\\	(情けない声で)お母ちゃん、俺のこと忘れたりしないよね。	
\\	(情けない声で)お母ちゃん、俺のこと忘れたりしないよね。 
\\	あたり前じゃない。あんたはいつも女々しいわね。このマザコンが!!	
\\	あたり前じゃない。あんたはいつも女々しいわね。このマザコンが!! 
\\	感情表現
\\	喜怒哀楽
\\	新人
\\	家庭訪問
\\	婉曲
\\	こういう言い方はよくないですが、あの子は子供のくせに妙にさめている感じがするんです。	
\\	こういう言い方はよくないですが、あの子は子供のくせに妙にさめている感じがするんです。 
\\	どういう風に。	
\\	どういう風に。 
\\	休憩時間でも、お友達の輪に入らないですし感情表現というか、喜怒哀楽を全然おもてに出さないんです。	
\\	休憩時間でも、お友達の輪に入らないですし感情表現というか、喜怒哀楽を全然おもてに出さないんです。 
\\	でも、いろいろな子供がいるでしょう。あなたも新人の先生じゃあるまいし、いままでの経験があるでしょう。	
\\	でも、いろいろな子供がいるでしょう。あなたも新人の先生じゃあるまいし、いままでの経験があるでしょう。 
\\	授業中も答えがわかっているくせに、当てても何も言いません。そのくせに、テストの点数はいつもクラスで一番です。心を開かせようと、色々努力しているんですが・・・一向に。	
\\	授業中も答えがわかっているくせに、当てても何も言いません。そのくせに、テストの点数はいつもクラスで一番です。心を開かせようと、色々努力しているんですが・・・一向に。 
\\	そうですか。来月から家庭訪問が始まるので、その辺りを親御さんに婉曲に聞いてみてください。それからまた話しましょう。	
\\	そうですか。来月から家庭訪問が始まるので、その辺りを親御さんに婉曲に聞いてみてください。それからまた話しましょう。 
\\	わかりました。	
\\	わかりました。 
\\	突然
\\	腹が減る
\\	就職
\\	数回
\\	結婚
\\	昨日の深夜、あなたが寝た後で、突然帰ってきてまだずーっと上で寝てるのよ。	
\\	昨日の深夜、あなたが寝た後で、突然帰ってきてまだずーっと上で寝てるのよ。 
\\	よっぽど疲れているんでしょう。飛行機ではあまり眠れないって言っていたからな。	
\\	よっぽど疲れているんでしょう。飛行機ではあまり眠れないって言っていたからな。 
\\	それにしても、もう夜の8時よ。	
\\	それにしても、もう夜の8時よ。 
\\	腹が減れば、下に降りてくるだろう。そんなに心配しなくても大丈夫だって。	
\\	腹が減れば、下に降りてくるだろう。そんなに心配しなくても大丈夫だって。 
\\	あの子がロンドンで就職してからもう2年になるわ。その間に連絡してきたのは数回だけ。	
\\	あの子がロンドンで就職してからもう2年になるわ。その間に連絡してきたのは数回だけ。 
\\	よっぽど忙しいんだろう。便りのないのは元気な印っていうし。後で、下に降りてきたら、仕事の事とか聞いてみたらいいだろう。それより、アイツ結婚はどうなんだ。	
\\	"よっぽど忙しいんだろう。便りのないのは元気な印っていうし。後で、下に降りてきたら、仕事の事とか聞いてみたらいいだろう。それより、アイツ結婚はどうなんだ。 
\\	知らないわよ。	
\\	知らないわよ。 
\\	販売促進
\\	立候補
\\	幹事
\\	手配
\\	歓迎会
\\	改めて
\\	業務
\\	連絡事項
\\	課長
\\	朝会
\\	お楽しみ
\\	おはようございます。これから販売促進第1課の朝会を始めます。では、まず、課長から連絡事項をお願いします。	
\\	おはようございます。これから販売促進第1課の朝会を始めます。では、まず、課長から連絡事項をお願いします。 
\\	おはようございます。実は人事異動で、明日、業務課からうちの課に異動してくる人がいます。明日の朝、改めて紹介しますが、今日のうちに、新しい人の机とパソコンを用意しておいてください。	
\\	おはようございます。実は人事異動で、明日、業務課からうちの課に異動してくる人がいます。明日の朝、改めて紹介しますが、今日のうちに、新しい人の机とパソコンを用意しておいてください。 
\\	どんな人が来るんですか?	
\\	どんな人が来るんですか? 
\\	それは、明日のお楽しみということで。それから、歓迎会の手配もお願いしますよ。	
\\	それは、明日のお楽しみということで。それから、歓迎会の手配もお願いしますよ。 
\\	それでは、机やパソコンを用意する人と、歓迎会の幹事を決めたいと思います。どなたか立候補してくれる人はいますか?	
\\	それでは、机やパソコンを用意する人と、歓迎会の幹事を決めたいと思います。どなたか立候補してくれる人はいますか? 
\\	(立候補してもいいけど、正直言うと、どんな人が来るかに寄るんだけどな。。。)>課長、やはり、どんな人が来るか教えて頂けないでしょうか。その方が、歓迎会の店を選ぶのに都合がよいかと思います。	
\\	(立候補してもいいけど、正直言うと、どんな人が来るかに寄るんだけどな。。。)>課長、やはり、どんな人が来るか教えて頂けないでしょうか。その方が、歓迎会の店を選ぶのに都合がよいかと思います。 
\\	どうせ、君たちは、若い女の子がいいと思っているんだろう。やっぱり、明日のお楽しみにしよう。	
\\	どうせ、君たちは、若い女の子がいいと思っているんだろう。やっぱり、明日のお楽しみにしよう。 
\\	久しく
\\	面
\\	何事
\\	悲哀
\\	既婚者
\\	周り
\\	勇気
\\	踏み切る
\\	兆候
\\	独身
\\	目に付く
\\	そろそろ結婚式のシーズンね。	
\\	そろそろ結婚式のシーズンね。 
\\	そうだな。もう久しく結婚式なんて行ってないな。	
\\	そうだな。もう久しく結婚式なんて行ってないな。 
\\	あなたのお友達に独身はいっぱいいるのにねぇ。全然その兆候すらないわね。	
\\	あなたのお友達に独身はいっぱいいるのにねぇ。全然その兆候すらないわね。 
\\	独身時代が長すぎて、今さら、結婚に踏み切れないんじゃないかな。	
\\	独身時代が長すぎて、今さら、結婚に踏み切れないんじゃないかな。 
\\	結婚に踏み切るって、そんなに勇気がいることなの?	
\\	結婚に踏み切るって、そんなに勇気がいることなの? 
\\	うーん、周りの既婚者たちの悲哀を見過ぎたんじゃないか。何事も知りすぎるとよくない面が目に付くからね。	
\\	うーん、周りの既婚者たちの悲哀を見過ぎたんじゃないか。何事も知りすぎるとよくない面が目に付くからね。 
\\	それって、どういうこと?周りの既婚者って、あなたのことじゃない!!	
\\	"それって、どういうこと?周りの既婚者って、あなたのことじゃない!! 
\\	今時
\\	直筆
\\	ロマンチスト
\\	見かけ
\\	付き合う
\\	伝わる
\\	便利
\\	確か
\\	古典的
\\	面倒くさい
\\	もう、聞いて聞いてよ。2週間も前にコウジに手紙を書いたのに、なかなか返事が来ないのよ。	
\\	もう、聞いて聞いてよ。2週間も前にコウジに手紙を書いたのに、なかなか返事が来ないのよ。 
\\	今時、手紙を書くなんて古典的ね。今はメールよ。メール。	
\\	今時、手紙を書くなんて古典的ね。今はメールよ。メール。 
\\	メールは確かに早いし便利だけど・・・なんて言うか、気持ちとかがなかなか伝わらないじゃない。	
\\	メールは確かに早いし便利だけど・・・なんて言うか、気持ちとかがなかなか伝わらないじゃない。 
\\	あんた達、もうつきあってかなり長いじゃないの。いまさら、相手の気持ちがどうのこうのなんて・・・あんた、見かけによらずロマンチストねえ。	
\\	あんた達、もうつきあってかなり長いじゃないの。いまさら、相手の気持ちがどうのこうのなんて・・・あんた、見かけによらずロマンチストねえ。 
\\	直筆を読むと、メールより気持ちが伝わってくるから嬉しいじゃない。	
\\	直筆を読むと、メールより気持ちが伝わってくるから嬉しいじゃない。 
\\	確かにそうだけど。やっぱり面倒くさいわ。メールでいいわよ。	
\\	確かにそうだけど。やっぱり面倒くさいわ。メールでいいわよ。 
\\	うるさいわねー。私はコウジの直筆じゃないと嫌なの!	
\\	うるさいわねー。私はコウジの直筆じゃないと嫌なの! 
\\	弱々しい
\\	飲み過ぎ
\\	珍しい
\\	腹が減る
\\	朝ご飯
\\	おはよう。	
\\	おはよう。 
\\	(弱々しく)おはよう。昨日、飲み過ぎたせいか、頭がガンガンする。	
\\	(弱々しく)おはよう。昨日、飲み過ぎたせいか、頭がガンガンする。 
\\	あんたが二日酔いなんて、珍しいわね。	
\\	あんたが二日酔いなんて、珍しいわね。 
\\	腹が減ってる時に、食べずにガンガン飲んじゃったせいかな。	
\\	腹が減ってる時に、食べずにガンガン飲んじゃったせいかな。 
\\	朝ご飯はどうする?	
\\	朝ご飯はどうする? 
\\	飲み過ぎたせいか、胃がむかむかする。コーヒー入れてくれない。	
\\	飲み過ぎたせいか、胃がむかむかする。コーヒー入れてくれない。 
\\	りょうかい。	
\\	りょうかい。 
\\	遅れる
\\	社会人
\\	失格
\\	心を入れ替える
\\	広告代理店
\\	遅い。遅い。もーう。今何時だと思ってるのよ。	
\\	遅い。遅い。もーう。今何時だと思ってるのよ。 
\\	ごめん、ごめん。	
\\	ごめん、ごめん。 
\\	何でトモはいつも約束の時間に遅れるのよ。時間にルーズなのは社会人として失格よ。	
\\	何でトモはいつも約束の時間に遅れるのよ。時間にルーズなのは社会人として失格よ。 
\\	だから、ごめんて。最近は心を入れ替えて、ちゃんと時間通りに来てるじゃん。今日は本当にたまたまだよ。	
\\	だから、ごめんて。最近は心を入れ替えて、ちゃんと時間通りに来てるじゃん。今日は本当にたまたまだよ。 
\\	なんで遅れたのよっ。	
\\	なんで遅れたのよっ。 
\\	原宿駅出たところで、コージとばったり会ったもんだからさ・・・すげー久しぶりだったから、ちょっと話こんじゃって。	
\\	原宿駅出たところで、コージとばったり会ったもんだからさ・・・すげー久しぶりだったから、ちょっと話こんじゃって。 
\\	えーっ、コージって大学の時のあのコージ君!?	
\\	えーっ、コージって大学の時のあのコージ君!? 
\\	そうだよ。	
\\	そうだよ。 
\\	広告代理店だって。	
\\	広告代理店だって。 
\\	やっぱなー。コージ君は誰かさんと違って、すごいオシャレだったもんね。	
\\	やっぱなー。コージ君は誰かさんと違って、すごいオシャレだったもんね。 
\\	それって、俺はどんくさかったってことかよー。	
\\	それって、俺はどんくさかったってことかよー。 
\\	補習授業
\\	現役
\\	部活動
\\	現代文
\\	古文
\\	期末テスト
\\	結構
\\	周り
\\	苦手
\\	都合上
\\	先生、古文の問題でどうしてもわからないところがあるので、夏休みに補習授業をやってもらえませんか。	
\\	先生、古文の問題でどうしてもわからないところがあるので、夏休みに補習授業をやってもらえませんか。 
\\	先生、夏休みは忙しいんだけどなあ。	
\\	先生、夏休みは忙しいんだけどなあ。 
\\	古文を苦手にしている人、周りでも結構多いんです。先生の時間があるときでよいのでお願いします。	
\\	古文を苦手にしている人、周りでも結構多いんです。先生の時間があるときでよいのでお願いします。 
\\	古文だけでいいの。期末テストの結果、現代文もあまりよくなかったんじゃないの。	
\\	古文だけでいいの。期末テストの結果、現代文もあまりよくなかったんじゃないの。 
\\	はい。できれば現代文もお願いしたいです。	
\\	はい。できれば現代文もお願いしたいです。 
\\	君は確か、この間まで部活動をやっていたよね。ひょっとして現役で合格するつもり。	
\\	君は確か、この間まで部活動をやっていたよね。ひょっとして現役で合格するつもり。 
\\	はい!家の都合上、どうしても現役で合格しなきゃいけないんです。	
\\	はい!家の都合上、どうしても現役で合格しなきゃいけないんです。 
\\	じゃあ、頑張らないとね。	
\\	じゃあ、頑張らないとね。 
\\	貼る
\\	作文
\\	比べる
\\	きちんと
\\	つまらない
\\	切り上げる
\\	文章
\\	そろそろ
\\	あら、学くんのお母さん、お久しぶりです。	
\\	あら、学くんのお母さん、お久しぶりです。 
\\	あっ、徹くんのお母さん、お久しぶりです。教室の後ろに貼ってあった作文、読みました?	
\\	あっ、徹くんのお母さん、お久しぶりです。教室の後ろに貼ってあった作文、読みました? 
\\	読んだわよ。もう、うちの子、作文が下手でいやになっちゃうわ。	
\\	読んだわよ。もう、うちの子、作文が下手でいやになっちゃうわ。 
\\	あら、そんなことないわよ、うちの子に比べたら、全然上手よ!まったく、うちの学ときたら、4年にもなって、1年生みたいな作文しか書けないんだから!!	
\\	あら、そんなことないわよ、うちの子に比べたら、全然上手よ!まったく、うちの学ときたら、4年にもなって、1年生みたいな作文しか書けないんだから!! 
\\	でも、学くんの作文、面白いこと書いてあるわよ。	
\\	でも、学くんの作文、面白いこと書いてあるわよ。 
\\	面白くてもだめよ。きちんとした文章を書かないと。	
\\	面白くてもだめよ。きちんとした文章を書かないと。 
\\	そうかしら、きちんとしてても、内容が面白くないとつまらないじゃない。	
\\	そうかしら、きちんとしてても、内容が面白くないとつまらないじゃない。 
\\	だめだめ、作文なんだから、きちんとした文章じゃないと。。	
\\	だめだめ、作文なんだから、きちんとした文章じゃないと。。 
\\	岩盤浴
\\	ダイエット
\\	効果
\\	新陳代謝
\\	溶岩浴
\\	どろどろ
\\	溶ける
\\	火傷
\\	かなり
\\	早紀ちゃん、岩盤浴って行ったことある?	
\\	早紀ちゃん、岩盤浴って行ったことある? 
\\	あるよ。すっごい汗をかいて、気持ちいいよね。	
\\	あるよ。すっごい汗をかいて、気持ちいいよね。 
\\	岩盤浴って、ダイエット効果があるだけじゃなくて、新陳代謝も良くなるから、健康にいいらしいよ。	
\\	岩盤浴って、ダイエット効果があるだけじゃなくて、新陳代謝も良くなるから、健康にいいらしいよ。 
\\	確かに、岩盤浴に行った後はちょっと動いただけでも汗が出るよね。	
\\	確かに、岩盤浴に行った後はちょっと動いただけでも汗が出るよね。 
\\	そういえば、岩盤浴より効果が高い溶岩浴っていうのもあるよね。	
\\	そういえば、岩盤浴より効果が高い溶岩浴っていうのもあるよね。 
\\	あるある。でも溶岩浴って言葉、ちょっと恐い感じがあるね。溶岩って、真っ赤でどろどろしたイメージがあるから、火傷しちゃいそうな気がするわ。	
\\	あるある。でも溶岩浴って言葉、ちょっと恐い感じがあるね。溶岩って、真っ赤でどろどろしたイメージがあるから、火傷しちゃいそうな気がするわ。 
\\	早紀ちゃん、どろどろの溶けた溶岩を想像する人は、かなりめずらしいよ。	
\\	早紀ちゃん、どろどろの溶けた溶岩を想像する人は、かなりめずらしいよ。 
\\	料理
\\	本場
\\	地方
\\	半煮え
\\	最高
\\	マチコの彼ってすごく料理が上手いんだって。スペイン料理とかフランス料理とか作ってくれるんだって。いいなー。	
\\	マチコの彼ってすごく料理が上手いんだって。スペイン料理とかフランス料理とか作ってくれるんだって。いいなー。 
\\	スペイン料理くらい俺でも作れるよ。	
\\	スペイン料理くらい俺でも作れるよ。 
\\	えっ、ほんとに。じゃあ作ってよ。	
\\	えっ、ほんとに。じゃあ作ってよ。 
\\	出来た。これが本場のパエリア。バレンシア地方の代表料理だよ。	
\\	出来た。これが本場のパエリア。バレンシア地方の代表料理だよ。 
\\	おいしい〜!	
\\	おいしい〜! 
\\	日本で食べるパエリアはどちらかというとチャーハンみたいなんだけど、本場スペインのは、リゾットに近くて米がパラパラじゃなくて半煮えなんだよ。	
\\	日本で食べるパエリアはどちらかというとチャーハンみたいなんだけど、本場スペインのは、リゾットに近くて米がパラパラじゃなくて半煮えなんだよ。 
\\	じゃあ、今度食べ比べてみようよ。	
\\	じゃあ、今度食べ比べてみようよ。 
\\	えーっ。お金出してパエリア食べるくらいなら、その金で別の料理を食べた方がいいって。こんな最高のパエリア作れるのは、東京では俺くらいしかいないもんね。	
\\	えーっ。お金出してパエリア食べるくらいなら、その金で別の料理を食べた方がいいって。こんな最高のパエリア作れるのは、東京では俺くらいしかいないもんね。 
\\	対
\\	廻し蹴り
\\	一戦
\\	突き
\\	油断
\\	盛り上がる
\\	解説者
\\	一体
\\	皆さんこんにちは。本日のメインイベント、ピーターアーツ選手対アーネストホースト選手の一戦です。谷川さん、今日の戦いどう思いますか?	
\\	皆さんこんにちは。本日のメインイベント、ピーターアーツ選手対アーネストホースト選手の一戦です。谷川さん、今日の戦いどう思いますか? 
\\	そうですねぇ、アーネストホースト選手の廻し蹴りは正確、そして速いですからね。また、ピーターアーツ選手の突きは世界一!ですから、アーネストホースト選手はピーターアーツ選手の突きを打たれると、難しくなります。	
\\	そうですねぇ、アーネストホースト選手の廻し蹴りは正確、そして速いですからね。また、ピーターアーツ選手の突きは世界一!ですから、アーネストホースト選手はピーターアーツ選手の突きを打たれると、難しくなります。 
\\	ですが、ピーターアーツ選手は今年のMVPに選ばれた選手です。油断できませんねぇ。	
\\	ですが、ピーターアーツ選手は今年のMVPに選ばれた選手です。油断できませんねぇ。 
\\	お、そろそろ試合が始まるようですよ!	
\\	お、そろそろ試合が始まるようですよ! 
\\	会場も盛り上がってまいりました!さて、一体どっちが勝つのでしょうか!!!!!	
\\	会場も盛り上がってまいりました!さて、一体どっちが勝つのでしょうか!!!!! 
\\	選手
\\	どうやら
\\	勝利
\\	連打
\\	やられる
\\	負ける
\\	感心する
\\	もらう
\\	蹴り
\\	鳴らす
\\	ゴング
\\	解説者
\\	今度
\\	さぁ、ゴングが鳴らされました。おおぉっといきなり、ピーターアーツ選手後ろに回られて蹴りをもらった!!!	
\\	さぁ、ゴングが鳴らされました。おおぉっといきなり、ピーターアーツ選手後ろに回られて蹴りをもらった!!! 
\\	アーネストホースト選手すばらしい動きですねぇ!!あのフットーワークには感心させられます。	
\\	アーネストホースト選手すばらしい動きですねぇ!!あのフットーワークには感心させられます。 
\\	だが、ピーターアーツ選手負けていない!!あんなすばらしい蹴りをやられましたが、今度は突きの連打だぁ!!だが、アーネストホースト選手、全ての突きを受け流す!!どうやら、ピーターアーツ選手の技は研究されたようだぁ!!	
\\	だが、ピーターアーツ選手負けていない!!あんなすばらしい蹴りをやられましたが、今度は突きの連打だぁ!!だが、アーネストホースト選手、全ての突きを受け流す!!どうやら、ピーターアーツ選手の技は研究されたようだぁ!! 
\\	髪形
\\	そんなに
\\	もみ上げ
\\	長め
\\	カラーリング
\\	明るさ
\\	後ろ
\\	いらっしゃいませー。今日はどんな髪型になさいますか?	
\\	いらっしゃいませー。今日はどんな髪型になさいますか? 
\\	いやー、もう冬だからね。そんなに短くしなくても良いんだけど。だけど、もみ上げは切らないでよ!!	
\\	いやー、もう冬だからね。そんなに短くしなくても良いんだけど。だけど、もみ上げは切らないでよ!! 
\\	分かりました。それではいつもより少し長めにしておきましょう。カラーリングはどうしますか?	
\\	分かりました。それではいつもより少し長めにしておきましょう。カラーリングはどうしますか? 
\\	んー、どうしようかなぁ。	
\\	んー、どうしようかなぁ。 
\\	この写真の感じの明るさなんてどうでしょう?	
\\	この写真の感じの明るさなんてどうでしょう? 
\\	あー良いね、んじゃこれで。	
\\	あー良いね、んじゃこれで。 
\\	シャンプーの方はどうしますか?	
\\	シャンプーの方はどうしますか? 
\\	はい、お願いしまーす。	
\\	はい、お願いしまーす。 
\\	お客様、こんな感じでいかがですか?	
\\	お客様、こんな感じでいかがですか? 
\\	ん~。。。もうちょっと後ろの方を短くしてもらえないかな。	
\\	ん~。。。もうちょっと後ろの方を短くしてもらえないかな。 
\\	はい、分かりました。。。。こんな感じですかね?	
\\	はい、分かりました。。。。こんな感じですかね? 
\\	うんうん、そんな感じで。	
\\	うんうん、そんな感じで。 
\\	はい、それではお疲れ様でした。料金は8,000円になります。	
\\	はい、それではお疲れ様でした。料金は8,000円になります。 
\\	奇抜
\\	挑戦
\\	感じ
\\	振り返る
\\	任せる
\\	頼む
\\	不意をつく
\\	攻撃
\\	気絶
\\	いらっしゃいませー、こんにちは。今日はどんな髪型になさいますか?	
\\	いらっしゃいませー、こんにちは。今日はどんな髪型になさいますか? 
\\	んん、そうね。もうすぐ夏だし、今日は奇抜な髪型に挑戦しちゃおうかなっ!!	
\\	んん、そうね。もうすぐ夏だし、今日は奇抜な髪型に挑戦しちゃおうかなっ!! 
\\	どんな感じで奇抜に?	
\\	どんな感じで奇抜に? 
\\	なんていうかな。歩いてると、人が振り返ってしまいそうなやつ!んん。。そうだな。。もう、店員さんにお任せしちゃうよ!	
\\	なんていうかな。歩いてると、人が振り返ってしまいそうなやつ!んん。。そうだな。。もう、店員さんにお任せしちゃうよ! 
\\	ほほー、人が振り返ってしまいそうなやつ。。。わたしに任せちゃっていいんですか?	
\\	ほほー、人が振り返ってしまいそうなやつ。。。わたしに任せちゃっていいんですか? 
\\	頼んだ!	
\\	頼んだ! 
\\	本当に良いんですね?分かりました。それでは。。。。ウりゃ!!!(後ろから不意を付いて攻撃)	
\\	本当に良いんですね?分かりました。それでは。。。。ウりゃ!!!(後ろから不意を付いて攻撃) 
\\	あっ。。(気絶)	
\\	あっ。。(気絶) 
\\	お客さん?。。。お客さま??。。。おきゃくさまーーー!!!起きてください、もう終わりましたよ。	
\\	お客さん?。。。お客さま??。。。おきゃくさまーーー!!!起きてください、もう終わりましたよ。 
\\	はっ!ここはどこ?	
\\	はっ!ここはどこ? 
\\	お客様、出来ましたよ!奇抜な髪型!	
\\	お客様、出来ましたよ!奇抜な髪型! 
\\	はっ・・・・えっ・・・・あぁ・・・・・・・・なんじゃこりゃ~!!	
\\	はっ・・・・えっ・・・・あぁ・・・・・・・・なんじゃこりゃ~!! 
\\	現場
\\	かっこいい (=格好いい)
\\	気を失う
\\	なるほど
\\	押えつける
\\	転ぶ
\\	捕まえる
\\	取り押える
\\	見事
\\	強盗
\\	ドジな
\\	こちら、現場の山中です。昨日、こちらのコンビニに強盗が入りました。ですが、強盗は一人の青年によって見事取り押えられました。	
\\	こちら、現場の山中です。昨日、こちらのコンビニに強盗が入りました。ですが、強盗は一人の青年によって見事取り押えられました。 
\\	すいません、ちょっとよろしいですか?	
\\	すいません、ちょっとよろしいですか? 
\\	あ、はい。	
\\	あ、はい。 
\\	アナタが、昨日ここで強盗を捕まえたという方ですか?!昨日の事を詳しく教えてください!	
\\	アナタが、昨日ここで強盗を捕まえたという方ですか?!昨日の事を詳しく教えてください! 
\\	いやー、あれは学校から帰っているところでした。このコンビニでお菓子を買って、出るところに強盗が入ってきて。	
\\	いやー、あれは学校から帰っているところでした。このコンビニでお菓子を買って、出るところに強盗が入ってきて。 
\\	なるほど!素晴らしいですねぇ!かっこいいですねぇ!!!で、強盗は逃げようとしたんですか?	
\\	なるほど!素晴らしいですねぇ!かっこいいですねぇ!!!で、強盗は逃げようとしたんですか? 
\\	吸う
\\	体力
\\	運動
\\	身が入る
\\	〜代が浮く
\\	禁煙
\\	ギャンブル
\\	似た者同士
\\	スー。。。ハー。。。	
\\	スー。。。ハー。。。 
\\	またタバコ吸ってんの?タバコばかり吸い過ぎると病気になるよ!	
\\	またタバコ吸ってんの?タバコばかり吸い過ぎると病気になるよ! 
\\	いやー、これだけは止められないよ。	
\\	いやー、これだけは止められないよ。 
\\	タバコばっかり吸って、運動もしていないんだから、体力も落ちてくるでしょ?	
\\	タバコばっかり吸って、運動もしていないんだから、体力も落ちてくるでしょ? 
\\	んん、でも、タバコ無いと仕事に身が入らないんだよね。あれ、お前この前までタバコ吸ってたよな。	
\\	んん、でも、タバコ無いと仕事に身が入らないんだよね。あれ、お前この前までタバコ吸ってたよな。 
\\	うん、だけど、この前止めたばかりなんだ。体の調子も良いし、ご飯もおいしいんだよねー。それに、タバコ代も浮くしさ。	
\\	うん、だけど、この前止めたばかりなんだ。体の調子も良いし、ご飯もおいしいんだよねー。それに、タバコ代も浮くしさ。 
\\	そうか、じゃあ、俺も頑張って禁煙してみるかなぁ。	
\\	そうか、じゃあ、俺も頑張って禁煙してみるかなぁ。 
\\	だけどな、タバコ代が浮いたお金でギャンブルを始めてしまったんだ。。	
\\	だけどな、タバコ代が浮いたお金でギャンブルを始めてしまったんだ。。 
\\	えー!ギャンブルは止めた方がいいよ。実は俺、この前ギャンブルを止めたばかりなんだ。	
\\	えー!ギャンブルは止めた方がいいよ。実は俺、この前ギャンブルを止めたばかりなんだ。 
\\	いやーでも、面白くてさぁ!止められないよ。	
\\	いやーでも、面白くてさぁ!止められないよ。 
\\	なぁ、、、俺らって似た者同士だな。	
\\	なぁ、、、俺らって似た者同士だな。 
\\	材料
\\	まぶす
\\	フライパン
\\	かぶる
\\	中火
\\	揚げる
\\	手に入る
\\	小麦粉
\\	から揚げ粉
\\	はたく
\\	火が通る
\\	一かけ
\\	まんべんなく
\\	おく
\\	もむ
\\	片栗粉
\\	大さじ
\\	杯
\\	適宜
\\	鶏モモ肉
\\	にんにく
\\	しょうが
\\	すりおろす
\\	ボール
\\	たれ
\\	しみこむ
\\	一口大
\\	(2人分)	
\\	(2人分) 
\\	一かけ	
\\	一かけ 
\\	一かけ	
\\	一かけ 
\\	大さじ1杯	
\\	大さじ1杯 
\\	大さじ1杯	
\\	大さじ1杯 
\\	大さじ2杯	
\\	大さじ2杯 
\\	適宜	
\\	適宜 
\\	肉じゃが
\\	くし型
\\	乱切り
\\	炒める
\\	水を切る
\\	あく
\\	お玉
\\	味見
\\	調味料
\\	顆粒
\\	だしの素
\\	煮崩れ
\\	面取り
\\	切り口
\\	そぎとる
\\	直径
\\	一回り小さい
\\	お袋の味
\\	水にさらす
\\	むく
\\	牛肉
\\	薄切り
\\	じゃがいも
\\	たまねぎ
\\	にんじん
\\	かつお節
\\	みりん
\\	砂糖
\\	油
\\	一握り
\\	程度
\\	だし
\\	沸騰
\\	煮立つ
\\	すくい取る
\\	下準備
\\	皮
\\	親しむ
\\	ジェーポッド・ワンオーワンのクッキング・ワンオーワンへようこそ!	
\\	ジェーポッド・ワンオーワンのクッキング・ワンオーワンへようこそ! 
\\	材料(2人分)	
\\	材料(2人分) 
\\	・牛肉(薄切り)- 
\\	・牛肉(薄切り)- 
\\	作り方	
\\	作り方 
\\	まず、だしをとります。水2カップをなべに入れて沸騰させます。そこに、かつお節を一握り程度入れて、1分程度煮立たせてから、かつお節をすくい取ります。	
\\	まず、だしをとります。水2カップをなべに入れて沸騰させます。そこに、かつお節を一握り程度入れて、1分程度煮立たせてから、かつお節をすくい取ります。 
\\	一口メモ	
\\	一口メモ 
\\	・だし汁は、顆粒のだしの素を使ってもいいです。ステップ4で、だし汁の代わりに水を入れて、だしの素を加えれば
\\	です。	
\\	・だし汁は、顆粒のだしの素を使ってもいいです。ステップ4で、だし汁の代わりに水を入れて、だしの素を加えれば
\\	です。 
\\	肉じゃがは、お袋の味として多くの日本人に親しまれています!是非、挑戦してみてください。	
\\	"肉じゃがは、お袋の味として多くの日本人に親しまれています!是非、挑戦してみてください。 
\\	親御さん
\\	手本
\\	保護者
\\	普段
\\	無理
\\	感想
\\	家庭環境
\\	起こす
\\	子どもたちに本を読ませるためには、まず、親御さんが読書することが大切です。	
\\	子どもたちに本を読ませるためには、まず、親御さんが読書することが大切です。 
\\	(子どものお手本になれってことね。)	
\\	(子どものお手本になれってことね。) 
\\	普段、好きなだけゲームで遊ばせておいて、急に「本を読みなさい」と言っても無理です。	
\\	普段、好きなだけゲームで遊ばせておいて、急に「本を読みなさい」と言っても無理です。 
\\	(うちは、親が二人ともゲーム好きなんて、最悪だわ。)	
\\	(うちは、親が二人ともゲーム好きなんて、最悪だわ。) 
\\	それから、親が読んだ本の感想を子どもに聞かせることも大切です。そういう家庭環境が、子どもに読書する気を起こさせることになるんです。	
\\	それから、親が読んだ本の感想を子どもに聞かせることも大切です。そういう家庭環境が、子どもに読書する気を起こさせることになるんです。 
\\	習慣
\\	無理やり
\\	自主的に
\\	がっかりする
\\	実感する
\\	姿
\\	それでは、各ご家庭で子どもに読書習慣をつけさせるために、どのようなことをしているか、紹介していただけませんか?	
\\	それでは、各ご家庭で子どもに読書習慣をつけさせるために、どのようなことをしているか、紹介していただけませんか? 
\\	子どもが、無理やり読書させられたと感じないように、子どもと一緒に本を選ぶようにしています。	
\\	子どもが、無理やり読書させられたと感じないように、子どもと一緒に本を選ぶようにしています。 
\\	なるほど。自主的に読書することが大切ですね。	
\\	なるほど。自主的に読書することが大切ですね。 
\\	でも、子どもが選ぶ本にがっかりさせられることもあります。	
\\	でも、子どもが選ぶ本にがっかりさせられることもあります。 
\\	子どもが選んだ本をそのまま読ませる方がいいですよ。	
\\	子どもが選んだ本をそのまま読ませる方がいいですよ。 
\\	そうなんです。嬉しそうに読む姿に、やっぱり読みたい本を読むのが大切だと実感させられました。	
\\	そうなんです。嬉しそうに読む姿に、やっぱり読みたい本を読むのが大切だと実感させられました。 
\\	寿司屋
\\	大将
\\	たこわさ
\\	試食
\\	そっか
\\	すまん
\\	代わりに
\\	真っ黄色い
\\	からし
\\	寿司屋でバイト 
\\	おはようございます、大将!	
\\	寿司屋でバイト 
\\	おはようございます、大将! 
\\	おう、わさびちゃん。今日もよろしく頼むよ!	
\\	おう、わさびちゃん。今日もよろしく頼むよ! 
\\	私、わさび大嫌いなんですよ。。	
\\	私、わさび大嫌いなんですよ。。 
\\	時間が経つ。。	
\\	なんじゃこりゃーー!!!!	
\\	なんじゃこりゃーー!!!! 
\\	大将、どうしました?!	
\\	大将、どうしました?! 
\\	わさびが全部無くなってる。。。代わりに全部、真っ黄色いからしになってるぞ。。。	
\\	わさびが全部無くなってる。。。代わりに全部、真っ黄色いからしになってるぞ。。。 
\\	店
\\	捨てる
\\	全部
\\	泥棒
\\	わさび畑
\\	ちょっと
\\	開ける
\\	~の仕業
\\	変わる
\\	ソバ屋
\\	隣
\\	消える
\\	~中
\\	売る
\\	校長
\\	ただいま。。。	
\\	ただいま。。。 
\\	どうでした、大将?!お店にわさびは売っていましたか?!	
\\	どうでした、大将?!お店にわさびは売っていましたか?! 
\\	どうやら伊豆中のわさびが消えち(てし)まったらしい。隣のソバ屋のわさびも全部からしに変わったそうだ。。	
\\	どうやら伊豆中のわさびが消えち(てし)まったらしい。隣のソバ屋のわさびも全部からしに変わったそうだ。。 
\\	大将、ちょっと出かけてきます。。	
\\	大将、ちょっと出かけてきます。。 
\\	わさび畑に来る	
\\	きっと、ここにもわさび泥棒が来ているはず。。。	
\\	きっと、ここにもわさび泥棒が来ているはず。。。 
\\	まったく、なんでこんなにワサビばっかり。。。全部捨ててやる!!	
\\	まったく、なんでこんなにワサビばっかり。。。全部捨ててやる!! 
\\	あ、校長!	
\\	あ、校長! 
\\	所
\\	続ける
\\	無理
\\	シュウマイ
\\	おでん
\\	断然
\\	ツンと来る
\\	大嫌い(な)
\\	思う
\\	取る
\\	雑草
\\	ここら辺
\\	無駄
\\	え、わさび君?!こんな所で何しているんだね?!	
\\	え、わさび君?!こんな所で何しているんだね?! 
\\	いえー、ちょっと。。。校長こそ何しているんですか?	
\\	いえー、ちょっと。。。校長こそ何しているんですか? 
\\	いやね、ちょっとここら辺の雑草でも取ろうと思って。。	
\\	いやね、ちょっとここら辺の雑草でも取ろうと思って。。 
\\	校長、、それワサビですけど。。	
\\	校長、、それワサビですけど。。 
\\	え!?あ、、、、わさび君、僕はね、わさびが大嫌いなんだよ!!緑だし、ツンと来るし!からしの方が断然おいしいじゃないか!	
\\	え!?あ、、、、わさび君、僕はね、わさびが大嫌いなんだよ!!緑だし、ツンと来るし!からしの方が断然おいしいじゃないか! 
\\	校長、それは無理があります。。	
\\	校長、それは無理があります。。 
\\	ええええーーー!?まったく、分からん子だ。。。君にこんな話を続けていくのは無駄だ。	
\\	ええええーーー!?まったく、分からん子だ。。。君にこんな話を続けていくのは無駄だ。 
\\	死ぬ
\\	噂
\\	なんだって
\\	呼ぶ
\\	わく
\\	特別な
\\	空手
\\	やられる
\\	戦闘シーン
\\	バリバリ食べる
\\	引っこ抜く
\\	こうなったら
\\	しょうがない
\\	相手
\\	わさび君、君には死んでもらうよ!でやぁー!!!	
\\	わさび君、君には死んでもらうよ!でやぁー!!! 
\\	校長やられる。	
\\	いたたた。。。。なんでそんなに君は強いんだ・・・空手か何かでもやっているのかね?	
\\	いたたた。。。。なんでそんなに君は強いんだ・・・空手か何かでもやっているのかね? 
\\	いいえ、校長。ワサビです。わさびを食べると、特別なパワーがわいてくるのです。	
\\	いいえ、校長。ワサビです。わさびを食べると、特別なパワーがわいてくるのです。 
\\	な、なんだって!!君が噂の「わさびくノ一」。。。。相手が悪かった。。。グフっ。。。	
\\	な、なんだって!!君が噂の「わさびくノ一」。。。。相手が悪かった。。。グフっ。。。 
\\	失礼
\\	大学
\\	専攻
\\	忙しい
\\	漫画
\\	字幕
\\	ナンパ
\\	(アメリカのカフェで。)すみません、あの、失礼ですけど、日本の方ですか。	
\\	(アメリカのカフェで。)すみません、あの、失礼ですけど、日本の方ですか。 
\\	あ、はい、そうですけど。	
\\	あ、はい、そうですけど。 
\\	私、今大学で日本語を専攻しているんですけど。	
\\	私、今大学で日本語を専攻しているんですけど。 
\\	え!?あの、私、最近忙しいので・・・。	
\\	え!?あの、私、最近忙しいので・・・。 
\\	お願いします。日本語ができるようになりたいんです!	
\\	お願いします。日本語ができるようになりたいんです! 
\\	日本の漫画が読めるようになりたいし、	
\\	日本の漫画が読めるようになりたいし、 
\\	日本の歌が聞き取れるようになりたいし、	
\\	日本の歌が聞き取れるようになりたいし、 
\\	日本の映画を字幕なしで見られるようになりたいし、	
\\	日本の映画を字幕なしで見られるようになりたいし、 
\\	そうですか・・・。でも日本語もう十分お上手ですよね。	
\\	そうですか・・・。でも日本語もう十分お上手ですよね。 
\\	ところで
\\	覚える
\\	暇
\\	気に入った
\\	セリフ
\\	意外
\\	熱心
\\	ところでトムさんはどうしてそんなに日本語が上手なんですか。	
\\	ところでトムさんはどうしてそんなに日本語が上手なんですか。 
\\	毎日、一日一個は新しい単語を覚えるようにしています。	
\\	毎日、一日一個は新しい単語を覚えるようにしています。 
\\	それから、暇なときは日本のアニメや映画を見るようにしています。	
\\	それから、暇なときは日本のアニメや映画を見るようにしています。 
\\	へー、意外に勉強熱心なんですね。	
\\	へー、意外に勉強熱心なんですね。 
\\	連絡
\\	登録
\\	つながる
\\	わざと
\\	偽物
\\	ところで、今後連絡できるように、携帯の番号教えてもらえませんか。	
\\	ところで、今後連絡できるように、携帯の番号教えてもらえませんか。 
\\	えっと、010-1972-1982です。	
\\	えっと、010-1972-1982です。 
\\	あ、はいどうも。登録しました。じゃあ早速今かけてみますね。	
\\	あ、はいどうも。登録しました。じゃあ早速今かけてみますね。 
\\	え・・・。	
\\	え・・・。 
\\	あれ、おかしいなぁ。つながりませんけど。	
\\	あれ、おかしいなぁ。つながりませんけど。 
\\	紹介する
\\	暇な
\\	ちょうどいい
\\	是非
\\	直接
\\	電話をかける
\\	絶対に
\\	じゃあ誰かお友達紹介してもらえませんか。	
\\	じゃあ誰かお友達紹介してもらえませんか。 
\\	あ、いいですよ。みさっていう暇な子がいるんで、ちょうどいいと思います。	
\\	あ、いいですよ。みさっていう暇な子がいるんで、ちょうどいいと思います。 
\\	はい、是非お願いします!どんな子ですか。かわいいですか。	
\\	はい、是非お願いします!どんな子ですか。かわいいですか。 
\\	(プープー)あれ、電話に出ませんね。じゃあ後でトムさんに直接電話をかけるように言っておきます。	
\\	(プープー)あれ、電話に出ませんね。じゃあ後でトムさんに直接電話をかけるように言っておきます。 
\\	マスク
\\	うそ
\\	指切りげんまん
\\	内緒
\\	声
\\	本物
\\	俳優
\\	実は
\\	ばれる
\\	針
\\	ところでじゅんこさんはどうしてアメリカに来たんですか。	
\\	ところでじゅんこさんはどうしてアメリカに来たんですか。 
\\	トムクルーズに会うためです。なんちゃって。英語を勉強するためです。	
\\	トムクルーズに会うためです。なんちゃって。英語を勉強するためです。 
\\	ぼく、実は有名な俳優なんです。ばれないようにいつもこうしてるんです。	
\\	ぼく、実は有名な俳優なんです。ばれないようにいつもこうしてるんです。 
\\	え?一回マスク取ってみてください!	
\\	え?一回マスク取ってみてください! 
\\	引っ越す
\\	家賃
\\	線
\\	乗る
\\	乗り換える
\\	ややこしい
\\	頼む
\\	この前、川越に引っ越したんだ。	
\\	この前、川越に引っ越したんだ。 
\\	え、川越?遠くない?	
\\	え、川越?遠くない? 
\\	うん、川越まで行くと家賃安いし広いところを借りられるからね。	
\\	うん、川越まで行くと家賃安いし広いところを借りられるからね。 
\\	いいの?川越なんて行った事ないんだけど、どうやって行くの?何線だっけ?	
\\	いいの?川越なんて行った事ないんだけど、どうやって行くの?何線だっけ? 
\\	東武線を使うと早いよ。新宿で池袋行きの山手線に乗る。	
\\	東武線を使うと早いよ。新宿で池袋行きの山手線に乗る。 
\\	各駅停車
\\	急行
\\	焼く
\\	超
\\	うまい
\\	乗客
\\	すみません、この電車は小川町に行きますか?	
\\	すみません、この電車は小川町に行きますか? 
\\	はい、ですが各駅停車ですよ。小川町に行くなら、急行電車の方が速いですよ。	
\\	はい、ですが各駅停車ですよ。小川町に行くなら、急行電車の方が速いですよ。 
\\	急行は川越に止まりますか?	
\\	急行は川越に止まりますか? 
\\	そう思いますけど。あ、急行はあれですよ!	
\\	そう思いますけど。あ、急行はあれですよ! 
\\	どうもありがとうございます!	
\\	どうもありがとうございます! 
\\	どういたしまして。	
\\	どういたしまして。 
\\	あ、もしもし?	
\\	あ、もしもし? 
\\	おお、ヤスミ。いまどこ?	
\\	おお、ヤスミ。いまどこ? 
\\	いま、池袋駅。東武東上線に乗るところ。	
\\	いま、池袋駅。東武東上線に乗るところ。 
\\	早く来なよ!彼女がピザを焼いたところなんだ。超うまそうだよ。	
\\	早く来なよ!彼女がピザを焼いたところなんだ。超うまそうだよ。 
\\	そば屋
\\	直進
\\	コインランドリー
\\	商店街
\\	携帯
\\	オブジェ
\\	真っ直ぐ
\\	めんどくさい
\\	通行人
\\	「駅を出たら右に曲がって、そば屋が見えるまでまっすぐ進む。」ああ、ここか!	
\\	"「駅を出たら右に曲がって、そば屋が見えるまでまっすぐ進む。」ああ、ここか! 
\\	はい?	
\\	"はい? 
\\	「大きな携帯のオブジェ」を探しているんですが、どっちの道に行ったらいいですかね?	
\\	「大きな携帯のオブジェ」を探しているんですが、どっちの道に行ったらいいですかね? 
\\	「大きな携帯のオブジェ」?この道を行ってください。あっちの道は駅の方に行ってしまいますよ。	
\\	「大きな携帯のオブジェ」?この道を行ってください。あっちの道は駅の方に行ってしまいますよ。 
\\	あ、そうですか、ありがとうございます。	
\\	あ、そうですか、ありがとうございます。 
\\	いえいえ。	
\\	いえいえ。 
\\	あの道をまっすぐ行くと駅に着くんだよね。。。じゃあなんで一はこんなめんどくさい道を俺に教えたんだ?	
\\	あの道をまっすぐ行くと駅に着くんだよね。。。じゃあなんで一はこんなめんどくさい道を俺に教えたんだ? 
\\	冒険
\\	やっと
\\	ややこしい
\\	ピザ
\\	マジ
\\	ありえねえ
\\	近道
\\	軽く
\\	どうとかこうとか
\\	はぁ、、はぁ、、やっとついた。。。ここに着くのに、軽く冒険をしてしまった。	
\\	はぁ、、はぁ、、やっとついた。。。ここに着くのに、軽く冒険をしてしまった。 
\\	おお、ヤスミ!やっと来たか。なんでそんなに時間がかかったんだよ?	
\\	おお、ヤスミ!やっと来たか。なんでそんなに時間がかかったんだよ? 
\\	お前の教えてくれた道、、かなりややこしかったぞ。。まぁ着けたけどさ。	
\\	お前の教えてくれた道、、かなりややこしかったぞ。。まぁ着けたけどさ。 
\\	そうそう俺の彼女が作ったピザだぞ。	
\\	そうそう俺の彼女が作ったピザだぞ。 
\\	マジ!俺ピザ大好きなんだよね。	
\\	マジ!俺ピザ大好きなんだよね。 
\\	もう少し早く着いてれば、食べられたのにな。みんな食べちゃったよ。わりぃね	
\\	もう少し早く着いてれば、食べられたのにな。みんな食べちゃったよ。わりぃね 
\\	ナンパ
\\	喫茶店
\\	大ファン
\\	作品
\\	今回
\\	もしかして
\\	いまどき
\\	気取る
\\	観に行く
\\	映画
\\	彼女
\\	次
\\	駄目
\\	話
\\	すすすすすすすすいません!!!あ、あの、お一人ですか?!あ、ああ、あなたをナナナナナナナナンパしてもイイですか?	
\\	すすすすすすすすいません!!!あ、あの、お一人ですか?!あ、ああ、あなたをナナナナナナナナンパしてもイイですか? 
\\	うわ、キモッ!!	
\\	うわ、キモッ!! 
\\	はぁ。。。。あんなナンパじゃ、ダメよ。。。よし次はあの人に!!	
\\	はぁ。。。。あんなナンパじゃ、ダメよ。。。よし次はあの人に!! 
\\	おい、そこのイケメン、一人かい?もしよかったら、私と映画でも観に行かないかい?	
\\	おい、そこのイケメン、一人かい?もしよかったら、私と映画でも観に行かないかい? 
\\	はぁ?おめ、何気取ってんだよ?そんなナンパするやつ見たことねぇよ。	
\\	はぁ?おめ、何気取ってんだよ?そんなナンパするやつ見たことねぇよ。 
\\	ダメだなぁ。。。あれ?もしかして、それって村上春樹の新作ですよね?	
\\	ダメだなぁ。。。あれ?もしかして、それって村上春樹の新作ですよね? 
\\	あ、はい。今回の作品はとても面白いですよ!	
\\	あ、はい。今回の作品はとても面白いですよ! 
\\	へぇ、私も買わなくちゃ!	
\\	へぇ、私も買わなくちゃ! 
\\	村上春樹の作品、読むんですか?	
\\	村上春樹の作品、読むんですか? 
\\	はい、大ファンですよ!	
\\	はい、大ファンですよ! 
\\	俺もです!!	
\\	俺もです!! 
\\	もしよかったら、喫茶店でコーヒーを飲みながら村上春樹の作品の話でもしませんか?	
\\	もしよかったら、喫茶店でコーヒーを飲みながら村上春樹の作品の話でもしませんか? 
\\	村上春樹
\\	六本木
\\	微妙
\\	歴史
\\	壬生義士伝
\\	週末
\\	釣り
\\	趣味
\\	他に
\\	日本酒
\\	へぇ!!そんなに村上春樹が好きなんだ。すごいねぇ。	
\\	へぇ!!そんなに村上春樹が好きなんだ。すごいねぇ。 
\\	ゆうこちゃんは他に趣味はあるの?	
\\	ゆうこちゃんは他に趣味はあるの? 
\\	そうねぇ、ダーツをしたり、お酒を飲んだり、あ、あと映画を見るのも好きかな。	
\\	そうねぇ、ダーツをしたり、お酒を飲んだり、あ、あと映画を見るのも好きかな。 
\\	ダーツはしたこと無いけど、俺も酒を飲んだり、映画を見るのが好きだよ。	
\\	ダーツはしたこと無いけど、俺も酒を飲んだり、映画を見るのが好きだよ。 
\\	どうだった、この映画?『壬生義士伝』(みぶぎしでん)。面白かった?	
\\	どうだった、この映画?『壬生義士伝』(みぶぎしでん)。面白かった? 
\\	う、、、うん、、、まぁね。でも、歴史って難しいよね。	
\\	う、、、うん、、、まぁね。でも、歴史って難しいよね。 
\\	微妙だったよね。。。。	
\\	微妙だったよね。。。。 
\\	いや、そんなこと無いよ!面白かった!!	
\\	いや、そんなこと無いよ!面白かった!! 
\\	デート
\\	美人
\\	眺め
\\	最上階
\\	六本木ヒルズ
\\	最高
\\	うまい
\\	お酒
\\	料理
\\	雰囲気
\\	お勧め
\\	幸せ
\\	お互い
\\	まぁ、映画はつまらなかったけど、いいデートだったのよぉ	
\\	まぁ、映画はつまらなかったけど、いいデートだったのよぉ 
\\	へぇ、、、、良いわね、、、、幸せそうね。。。	
\\	へぇ、、、、良いわね、、、、幸せそうね。。。 
\\	その後に行った、たかおさんお勧めのお店も雰囲気が良くてさ、	
\\	その後に行った、たかおさんお勧めのお店も雰囲気が良くてさ、 
\\	はいはい。。。それは良かったわね。。。	
\\	はいはい。。。それは良かったわね。。。 
\\	で、あなたの方はどうだったのよ?あなたもデートだったんでしょ?	
\\	で、あなたの方はどうだったのよ?あなたもデートだったんでしょ? 
\\	うん、、、はは、すごく楽しかったよ!六本木ヒルズの最上階のレストランで二人で食事してさ。	
\\	うん、、、はは、すごく楽しかったよ!六本木ヒルズの最上階のレストランで二人で食事してさ。 
\\	口座
\\	公共料金
\\	身分証明書
\\	健康保険証
\\	運転免許証
\\	引き出し
\\	預け入れ
\\	普通預金
\\	お嬢さん
\\	印鑑
\\	外国人登録証
\\	領収証
\\	ねぇ、君。かわいいね。銀行の仕事は楽しいかい?僕、この銀行に口座を開きたいんだけど。	
\\	ねぇ、君。かわいいね。銀行の仕事は楽しいかい?僕、この銀行に口座を開きたいんだけど。 
\\	本日、外国人登録証とご印鑑をお持ちですか。	
\\	本日、外国人登録証とご印鑑をお持ちですか。 
\\	お嬢さん、確かに、僕は、ブロンドでハンサムだけど、外国人じゃないんだよ。	
\\	お嬢さん、確かに、僕は、ブロンドでハンサムだけど、外国人じゃないんだよ。 
\\	そうですか。では、こちらのカードをどうぞ。番号が呼ばれるまでお待ちください。	
\\	そうですか。では、こちらのカードをどうぞ。番号が呼ばれるまでお待ちください。 
\\	91番?はいはーい。91番!僕です。	
\\	91番?はいはーい。91番!僕です。 
\\	普通預金の口座でよろしいですか。	
\\	普通預金の口座でよろしいですか。 
\\	普通…預金?	
\\	普通…預金? 
\\	普通預金は、いつでも、預け入れや引き出しができる口座です。	
\\	普通預金は、いつでも、預け入れや引き出しができる口座です。 
\\	あ、そうそう、それ。普通預金の口座をお願いします。	
\\	あ、そうそう、それ。普通預金の口座をお願いします。 
\\	本日、運転免許証や健康保険証などの身分証明書と公共料金の領収証、それからご印鑑をお持ちですか。	
\\	本日、運転免許証や健康保険証などの身分証明書と公共料金の領収証、それからご印鑑をお持ちですか。 
\\	結構
\\	馬鹿
\\	名刺
\\	粗品
\\	自宅
\\	キャッシュカード
\\	通帳
\\	入金
\\	申し込み用紙
\\	記入
\\	それでは、こちらの申し込み用紙にご記入ください。	
\\	それでは、こちらの申し込み用紙にご記入ください。 
\\	はい。(カリカリカリカリ)これで大丈夫ですか。	
\\	はい。(カリカリカリカリ)これで大丈夫ですか。 
\\	ありがとうございます。本日、おいくらご入金なさいますか。	
\\	ありがとうございます。本日、おいくらご入金なさいますか。 
\\	え?お金が必要なの?	
\\	え?お金が必要なの? 
\\	いくらでも結構です。1円でも結構です。	
\\	いくらでも結構です。1円でも結構です。 
\\	ふーん。じゃ、2円入金します。	
\\	ふーん。じゃ、2円入金します。 
\\	かしこまりました。用意が出来ましたら、お名前をお呼びいたしますので、少々お待ちください。	
\\	かしこまりました。用意が出来ましたら、お名前をお呼びいたしますので、少々お待ちください。 
\\	虎畑様。虎畑大河様。	
\\	虎畑様。虎畑大河様。 
\\	はーい。大河と呼んでください。	
\\	はーい。大河と呼んでください。 
\\	(コホン)こちらが、虎畑大河様の通帳でございます。キャッシュカードは1週間から10日後に、ご自宅にお送りいたします。	
\\	(コホン)こちらが、虎畑大河様の通帳でございます。キャッシュカードは1週間から10日後に、ご自宅にお送りいたします。 
\\	どうも。あ、そうだ。これ、僕の名刺です。よかったら、メールか電話ください。じゃまた。(チュッ)バァ~イ。	
\\	どうも。あ、そうだ。これ、僕の名刺です。よかったら、メールか電話ください。じゃまた。(チュッ)バァ~イ。 
\\	連れて行く
\\	弁護士
\\	全額
\\	給料日
\\	残高
\\	お金をおろす
\\	スッカラカン
\\	感心
\\	積み立てる
\\	積み立て用
\\	振り込む
\\	ねぇねぇ。君達。今、暇?今からうちの店に来ない?	
\\	ねぇねぇ。君達。今、暇?今からうちの店に来ない? 
\\	忙しいの。ごめんねぇ。	
\\	忙しいの。ごめんねぇ。 
\\	(ちっ。:舌打ち)だめだなぁ。今月、金ないから、お客連れて行きたいんだけどなぁ。	
\\	(ちっ。:舌打ち)だめだなぁ。今月、金ないから、お客連れて行きたいんだけどなぁ。 
\\	最近金使いすぎているから、積立用の口座をもう一つ作ったんだ。	
\\	最近金使いすぎているから、積立用の口座をもう一つ作ったんだ。 
\\	へぇー。感心感心。	
\\	へぇー。感心感心。 
\\	俺なんか月末にはいつもスッカラカンだよ。	
\\	俺なんか月末にはいつもスッカラカンだよ。 
\\	(ピピピ)	
\\	え?全額おろすの?マジで?	
\\	え?全額おろすの?マジで? 
\\	今夜、ハツコさんと食事に行くんだ。	
\\	今夜、ハツコさんと食事に行くんだ。 
\\	ハツコさんって、あの、弁護士の?	
\\	ハツコさんって、あの、弁護士の? 
\\	一緒に食事をすると、大金を振り込んでくれるし。	
\\	一緒に食事をすると、大金を振り込んでくれるし。 
\\	え?どうして。	
\\	え?どうして。 
\\	ちんちくりん
\\	両替
\\	用件
\\	半額
\\	記入
\\	硬貨
\\	手数料
\\	小銭
\\	運命
\\	偶然
\\	以上
\\	いらっしゃいま…あ。(ゲッ また来た! ちんちくりんホストが!)	
\\	いらっしゃいま…あ。(ゲッ また来た! ちんちくりんホストが!) 
\\	あ、偶然ですね。また君かぁ。僕は運命を感じるなぁ。	
\\	あ、偶然ですね。また君かぁ。僕は運命を感じるなぁ。 
\\	(コホン)それで、今日は、どのようなご用件ですか。	
\\	(コホン)それで、今日は、どのようなご用件ですか。 
\\	小銭をお札に両替したいんですけど、手数料はかかりますか。	
\\	小銭をお札に両替したいんですけど、手数料はかかりますか。 
\\	硬貨50枚までは無料でございますが、51枚からは手数料が315円かかります。	
\\	硬貨50枚までは無料でございますが、51枚からは手数料が315円かかります。 
\\	たくさんございますよぉ。(カウンターに硬貨を置く)	
\\	たくさんございますよぉ。(カウンターに硬貨を置く) 
\\	では、お待ちの間に、この用紙にご記入をお願いいたします。(硬貨の入った重い袋を持ち上げる)	
\\	では、お待ちの間に、この用紙にご記入をお願いいたします。(硬貨の入った重い袋を持ち上げる) 
\\	虎畑様。虎畑大河さま。	
\\	虎畑様。虎畑大河さま。 
\\	だから。大河って呼んでくださいよ。	
\\	だから。大河って呼んでくださいよ。 
\\	お待たせいたしました。全部で6万5332円でございました。	
\\	お待たせいたしました。全部で6万5332円でございました。 
\\	半額を普通預金の口座に入れたいんですけど。	
\\	半額を普通預金の口座に入れたいんですけど。 
\\	早速
\\	次回
\\	弁護士
\\	成長する
\\	小遣い
\\	生活費
\\	貯蓄用
\\	貯金箱
\\	急用
\\	今回
\\	よう、大河!銀行で何してたの?早速、金をおろしていたのか?	
\\	よう、大河!銀行で何してたの?早速、金をおろしていたのか? 
\\	ああ、聖夜か。いや、金を引き出していたんじゃないよ。	
\\	ああ、聖夜か。いや、金を引き出していたんじゃないよ。 
\\	え?リッチな弁護士、ハツコ先生から、振込みあったんでしょ?	
\\	え?リッチな弁護士、ハツコ先生から、振込みあったんでしょ? 
\\	いや…。それが、急用ができたからって、キャンセルされちゃって。	
\\	いや…。それが、急用ができたからって、キャンセルされちゃって。 
\\	じゃ、残高0円か。	
\\	じゃ、残高0円か。 
\\	俺、貯金はないけど、いつも、一円玉と五円玉と十円玉は貯金箱に入れるんだ。	
\\	俺、貯金はないけど、いつも、一円玉と五円玉と十円玉は貯金箱に入れるんだ。 
\\	で。(あいづち)	
\\	で。(あいづち) 
\\	で、たくさん貯まったから、今日銀行に持ってきたんだ。6万5332円あった。	
\\	で、たくさん貯まったから、今日銀行に持ってきたんだ。6万5332円あった。 
\\	へぇー(あいづち)すげー。で、貯蓄用の口座は作った?	
\\	へぇー(あいづち)すげー。で、貯蓄用の口座は作った? 
\\	それは、また次回。	
\\	それは、また次回。 
\\	今回は6万5332円の半額を普通預金の口座に入れて、半額の3万2666円だけおろしてきた。	
\\	今回は6万5332円の半額を普通預金の口座に入れて、半額の3万2666円だけおろしてきた。 
\\	迷子
\\	お知らせ
\\	付いて行く
\\	お越しください
\\	至急
\\	保護者
\\	控え室
\\	かける
\\	かぶる
\\	履く
\\	保護する
\\	詳しい
\\	お勧め
\\	なるべく
\\	日帰り
\\	賛成
\\	ロマンスカー
\\	駅弁
\\	誘う
\\	なんだ
\\	ジョン、夏休みは、どうする?	
\\	ジョン、夏休みは、どうする? 
\\	温泉に行こうよ。	
\\	温泉に行こうよ。 
\\	温泉、いいねぇ。どこの温泉に行こうか?	
\\	温泉、いいねぇ。どこの温泉に行こうか? 
\\	うーん。温泉に詳しくないからな・・・。加奈子のお勧めはどこ?	
\\	うーん。温泉に詳しくないからな・・・。加奈子のお勧めはどこ? 
\\	そうねぇ。なるべく近いところがいいね。箱根はどう?	
\\	そうねぇ。なるべく近いところがいいね。箱根はどう? 
\\	箱根は、日帰りできるし、富士山も見られるし…	
\\	箱根は、日帰りできるし、富士山も見られるし… 
\\	それに、ロマンスカーで行けるから、駅弁も食べられるし、楽しいよ!	
\\	それに、ロマンスカーで行けるから、駅弁も食べられるし、楽しいよ! 
\\	わぉ!駅弁?食べよう、食べよう!	
\\	わぉ!駅弁?食べよう、食べよう! 
\\	じゃ、箱根に決まり!トオルも温泉に行きたがっていたから、誘おうよ。	
\\	じゃ、箱根に決まり!トオルも温泉に行きたがっていたから、誘おうよ。 
\\	困る
\\	助かる
\\	勘違い
\\	どういう意味?
\\	欲しい
\\	ため
\\	あのさぁ、夏休みに、箱根の温泉に行こうと思っているんだけど、一緒に行かない?	
\\	あのさぁ、夏休みに、箱根の温泉に行こうと思っているんだけど、一緒に行かない? 
\\	箱根かぁ!いいね。行く、行く!	
\\	箱根かぁ!いいね。行く、行く! 
\\	よかったぁ。困ってたんだ。助かったわ。	
\\	よかったぁ。困ってたんだ。助かったわ。 
\\	えっ、どういう意味?	
\\	えっ、どういう意味? 
\\	ジョンも一緒に行くんだけど、彼に温泉の入り方を教えて欲しいんだ。	
\\	ジョンも一緒に行くんだけど、彼に温泉の入り方を教えて欲しいんだ。 
\\	ふーん、そのために、僕を誘ったんだ。	
\\	ふーん、そのために、僕を誘ったんだ。  
\\	そのためだけじゃないよ。勘違いしないでよ。	
\\	そのためだけじゃないよ。勘違いしないでよ。 
\\	ふん。まぁ、いいよ。	
\\	ふん。まぁ、いいよ。 
\\	おいしいものをたくさん食べて、温泉に入って、楽しもうね!	
\\	おいしいものをたくさん食べて、温泉に入って、楽しもうね! 
\\	浴槽
\\	引き戸
\\	かけ湯する
\\	慣らす
\\	残念
\\	湯船
\\	温度
\\	ジョン、この小さいタオル、持った?	
\\	ジョン、この小さいタオル、持った? 
\\	うん。でも、このタオルは、何のために使うの?	
\\	うん。でも、このタオルは、何のために使うの? 
\\	(温泉の浴室への引き戸を開ける音)	
\\	わぁ、広いなぁ!泳げるね。	
\\	わぁ、広いなぁ!泳げるね。 
\\	皆、泳ぎたがるんだよ。でも、泳いじゃだめだぞ。	
\\	皆、泳ぎたがるんだよ。でも、泳いじゃだめだぞ。 
\\	へぇー。残念。	
\\	へぇー。残念。 
\\	体、洗った?そしたら、浴槽に入る前に、かけ湯をするんだ。	
\\	体、洗った?そしたら、浴槽に入る前に、かけ湯をするんだ。 
\\	かけ湯?	
\\	かけ湯? 
\\	(お湯を流す音)	
\\	よし、湯船に入ろう!	
\\	よし、湯船に入ろう! 
\\	たたむ
\\	舌打ち
\\	浴びる
\\	上がる
\\	そろそろ
\\	倒れる
\\	長湯
\\	鼻歌
\\	なるほど
\\	載せる
\\	効果
\\	どうした?ジョンも入れよ。	
\\	どうした?ジョンも入れよ。 
\\	うん。(ジャッパーン)	
\\	うん。(ジャッパーン) 
\\	あ、だめだめ。ジョン。タオルは、浴槽の中に入れちゃだめだよ。	
\\	あ、だめだめ。ジョン。タオルは、浴槽の中に入れちゃだめだよ。 
\\	え?どうするの?	
\\	え?どうするの? 
\\	僕は、小さくたたんで、頭の上に載せるよ。	
\\	僕は、小さくたたんで、頭の上に載せるよ。 
\\	(鼻歌)	
\\	君達、あんまり長湯するなよ。倒れるぞ。	
\\	君達、あんまり長湯するなよ。倒れるぞ。 
\\	あ、はい。	
\\	あ、はい。 
\\	そ、そろそろ、上がろうか。	
\\	そ、そろそろ、上がろうか。 
\\	そっ…そうだね。シャ、シャワー浴びよう。じゃ、お先に・・・。	
\\	そっ…そうだね。シャ、シャワー浴びよう。じゃ、お先に・・・。 
\\	シャワー?(舌打;チッチッチ)シャワーは浴びちゃダメダメ。	
\\	シャワー?(舌打;チッチッチ)シャワーは浴びちゃダメダメ。 
\\	国連
\\	事務局
\\	地球
\\	爪
\\	宇宙人
\\	星
\\	貿易
\\	おじいさん。見てください。今日は星がきれいですよ。	
\\	おじいさん。見てください。今日は星がきれいですよ。 
\\	(ドドーン)	
\\	(ウィーン)	
\\	スミマセン。ココハ ドコ デスカ。	
\\	スミマセン。ココハ ドコ デスカ。 
\\	あわわわわわわわ。と、とととととうきょう。ジャ、ジャパン。	
\\	あわわわわわわわ。と、とととととうきょう。ジャ、ジャパン。 
\\	オオ。ニホン デスカ。マチガエマシタ。	
\\	オオ。ニホン デスカ。マチガエマシタ。 
\\	こここ、国連の事務局ぅ??多分、ニューヨークだと思います。	
\\	こここ、国連の事務局ぅ??多分、ニューヨークだと思います。 
\\	ドッチデスカ?	
\\	ドッチデスカ? 
\\	どっちって…よくわからないですけど、もっと、西だと思います。	
\\	どっちって…よくわからないですけど、もっと、西だと思います。 
\\	ソウデスカ。アリガトウ ゴザイマシタ。	
\\	ソウデスカ。アリガトウ ゴザイマシタ。 
\\	ワタクシハ、ピーター ト モウシマス。トオイホシ カラ キマシタ。コンゴトモ ヨロシク オネガイシマス。	
\\	ワタクシハ、ピーター ト モウシマス。トオイホシ カラ キマシタ。コンゴトモ ヨロシク オネガイシマス。 
\\	地球と貿易・・・。	
\\	地球と貿易・・・。 
\\	ワタシタチノ ホシ デハ、ニンゲンノ ツメ ハ トテモ タカイデス。	
\\	ワタシタチノ ホシ デハ、ニンゲンノ ツメ ハ トテモ タカイデス。 
\\	知り合い
\\	女っけ
\\	流行る
\\	アプリ
\\	使いこなす
\\	まし
\\	メル友
\\	俺、最近、メル友ができたんだ。	
\\	俺、最近、メル友ができたんだ。 
\\	ほんと?でも、それって、ちょっと変じゃないか?	
\\	ほんと?でも、それって、ちょっと変じゃないか? 
\\	直接会って友達になるのが普通だろう?	
\\	直接会って友達になるのが普通だろう? 
\\	まぁ、そうだけど、俺、女の子の知り合いもいないし、俺の友達も女っけないし。だから、ミクシーに入ってみたんだ。	
\\	まぁ、そうだけど、俺、女の子の知り合いもいないし、俺の友達も女っけないし。だから、ミクシーに入ってみたんだ。 
\\	で、誰かいい子見つかった?	
\\	で、誰かいい子見つかった? 
\\	ううん。だめ。それで、海外で流行っているフェイスブックを試してみたんだ。	
\\	ううん。だめ。それで、海外で流行っているフェイスブックを試してみたんだ。 
\\	へぇ、で、どうだった?	
\\	へぇ、で、どうだった? 
\\	アプリのほとんどが英語だからさ。最初は、使いこなすのが大変だったよ。	
\\	アプリのほとんどが英語だからさ。最初は、使いこなすのが大変だったよ。 
\\	ふーん。でもさ、ネットで彼女を見つけるのって、どうなの?大丈夫なの?	
\\	ふーん。でもさ、ネットで彼女を見つけるのって、どうなの?大丈夫なの? 
\\	さぁ。。。	
\\	さぁ。。。 
\\	でも、確かに、一人よりはましか….	
\\	でも、確かに、一人よりはましか…. 
\\	そうだよ。これ見てよ。毎日この子からメールが来るんだぜ。	
\\	そうだよ。これ見てよ。毎日この子からメールが来るんだぜ。 
\\	彼女からのメールを待つのも、返事をするのも、すっごく楽しいんだ。	
\\	彼女からのメールを待つのも、返事をするのも、すっごく楽しいんだ。 
\\	もちろん
\\	お陰様で
\\	嫌う
\\	やり取り
\\	ただの
\\	今日も、彼女からメール来た?	
\\	今日も、彼女からメール来た? 
\\	もちろん。毎日来るよ。	
\\	もちろん。毎日来るよ。 
\\	へぇ、すごいな。このごろ、お前、前より元気そうだな。	
\\	へぇ、すごいな。このごろ、お前、前より元気そうだな。 
\\	あの子のお陰だよ。	
\\	あの子のお陰だよ。 
\\	で、いつ結婚するの?	
\\	で、いつ結婚するの? 
\\	ええぇ??まだ会ったことないのに結婚するなんて、考えたこともないよ。	
\\	ええぇ??まだ会ったことないのに結婚するなんて、考えたこともないよ。 
\\	そうなのか?じゃ、いつ会うんだよ。	
\\	そうなのか?じゃ、いつ会うんだよ。 
\\	会いたいような、会いたくないような。。。	
\\	会いたいような、会いたくないような。。。 
\\	はぁ?毎日、メールやり取りしているのに、会ったこともないなんて、おかしいよ。	
\\	はぁ?毎日、メールやり取りしているのに、会ったこともないなんて、おかしいよ。 
\\	後悔
\\	自己紹介
\\	携帯
\\	お互い
\\	だいたい
\\	メールを打つ
\\	俺、昨日、愛ちゃんに会ったよ。	
\\	俺、昨日、愛ちゃんに会ったよ。 
\\	へぇ、彼女の名前、愛っていうんだ!	
\\	へぇ、彼女の名前、愛っていうんだ! 
\\	とうとう彼女に会ったか!で、いつ結婚?	
\\	とうとう彼女に会ったか!で、いつ結婚? 
\\	やめてくれよ!だいたい、直接会ったのを今は後悔してるんだ。	
\\	やめてくれよ!だいたい、直接会ったのを今は後悔してるんだ。 
\\	何で、後悔してるんだよ。	
\\	何で、後悔してるんだよ。 
\\	確かに顔は可愛かったよ。けど、最初に、	
\\	"確かに顔は可愛かったよ。けど、最初に、 
\\	「こんにちは、愛です」って、自己紹介だけして、すぐに、携帯メールを打ち出したんだぜ。	
\\	「こんにちは、愛です」って、自己紹介だけして、すぐに、携帯メールを打ち出したんだぜ。 
\\	友達にメールしてたのか?	
\\	友達にメールしてたのか? 
\\	いや、俺にメールしてたんだよ。一緒にいるのに、メールのやり取りだけ。。。	
\\	いや、俺にメールしてたんだよ。一緒にいるのに、メールのやり取りだけ。。。 
\\	一緒に食事をしているのに、俺の顔も見ないんだよ。	
\\	一緒に食事をしているのに、俺の顔も見ないんだよ。 
\\	そんなデート、聞いたこともないよ。じゃ、食事の注文はどうしたんだよ。	
\\	そんなデート、聞いたこともないよ。じゃ、食事の注文はどうしたんだよ。 
\\	食事のオーダーもメールで来た。	
\\	食事のオーダーもメールで来た。 
\\	はぁ?食事の後は?食事の後はどうしたんだ?また、会うんだよな?	
\\	はぁ?食事の後は?食事の後はどうしたんだ?また、会うんだよな? 
\\	レストランを出た後に、メールが来て	
\\	"レストランを出た後に、メールが来て 
\\	残業
\\	物足りない
\\	珍しい
\\	変わった
\\	フェレット
\\	プレーリードッグ
\\	飼う
\\	最近、仕事おもしろくないし、これからはあんまり残業しないことにしたから時間もあるし、ペットでも飼いたいんだよね。	
\\	最近、仕事おもしろくないし、これからはあんまり残業しないことにしたから時間もあるし、ペットでも飼いたいんだよね。 
\\	へぇ、ペットって、たとえば、犬とか、猫とか?	
\\	へぇ、ペットって、たとえば、犬とか、猫とか? 
\\	いろいろ考えたんだけど、犬とか猫じゃ、すぐに物足りなくなるんじゃないかなって思ってるんだ。	
\\	いろいろ考えたんだけど、犬とか猫じゃ、すぐに物足りなくなるんじゃないかなって思ってるんだ。 
\\	じゃ、何がいいのよ?	
\\	じゃ、何がいいのよ? 
\\	うーん、まだわからないんだけど。。。珍しいペットを売ってるショップにでも探しに行こうかな。	
\\	うーん、まだわからないんだけど。。。珍しいペットを売ってるショップにでも探しに行こうかな。 
\\	そうしなよ。日本でも、変わったペットは珍しくなくなったからね。	
\\	そうしなよ。日本でも、変わったペットは珍しくなくなったからね。 
\\	そうだね。フェレットとか、小さいサルをペットにしている人もいるし、ペンギンを飼ってる家の話も聞いたことあるよ。	
\\	そうだね。フェレットとか、小さいサルをペットにしている人もいるし、ペンギンを飼ってる家の話も聞いたことあるよ。 
\\	あぁ、そういえば、プレーリードッグを飼ってる友達がいたわ。	
\\	あぁ、そういえば、プレーリードッグを飼ってる友達がいたわ。 
\\	俺も、そういう特別なペットがいいな。	
\\	俺も、そういう特別なペットがいいな。 
\\	目が高い
\\	蛇
\\	タコ
\\	イカ
\\	子連れ
\\	お世辞
\\	アシカ
\\	アザラシ
\\	獰猛
\\	マムシ
\\	個性的な
\\	毒
\\	すいませ~ん。変わったペットを探してるんですけど。	
\\	すいませ~ん。変わったペットを探してるんですけど。 
\\	変わったペット?お客さん、お目が高い!うちは、日本一個性的なペットショップです。	
\\	変わったペット?お客さん、お目が高い!うちは、日本一個性的なペットショップです。 
\\	ショーウィンドーの中の蛇は何ていう蛇ですか?	
\\	ショーウィンドーの中の蛇は何ていう蛇ですか? 
\\	あれは、マムシです。毒を持っていますし、獰猛です。お薦めとは言えませんが、スリルは楽しめます。お子さんは、いらっしゃいますか?	
\\	あれは、マムシです。毒を持っていますし、獰猛です。お薦めとは言えませんが、スリルは楽しめます。お子さんは、いらっしゃいますか? 
\\	いませんが、マムシは結構です。じゃ、むこうにいるのは何ていう鳥?	
\\	いませんが、マムシは結構です。じゃ、むこうにいるのは何ていう鳥? 
\\	あれは、野生の七面鳥です。	
\\	あれは、野生の七面鳥です。 
\\	へぇ~。野生の七面鳥か!あっ、あれはアザラシ?	
\\	へぇ~。野生の七面鳥か!あっ、あれはアザラシ? 
\\	いえ、あれはアシカです。お世辞にもかわいいとは言えませんが、子連れなんですよ。すごいでしょ。	
\\	いえ、あれはアシカです。お世辞にもかわいいとは言えませんが、子連れなんですよ。すごいでしょ。 
\\	なるほど。。 あっ、イカだ!かわいいですね!	
\\	なるほど。。 あっ、イカだ!かわいいですね! 
\\	お客様、あれは、タコです。 まぁ、イカと似てないとは言えませんがね。。。	
\\	お客様、あれは、タコです。 まぁ、イカと似てないとは言えませんがね。。。 
\\	タコ、かぁ。。。	
\\	タコ、かぁ。。。 
\\	専門店
\\	たこ焼き
\\	ゆでる
\\	長湯
\\	バスタブ
\\	目移りする
\\	河童
\\	熊
\\	トカゲ
\\	サメ
\\	七面鳥
\\	野生
\\	ありとあらゆる
\\	召し上がる
\\	ねぇ、変なペット、飼うことにしたの?この前、変わったペット専門店に行くって言ってたじゃない。	
\\	ねぇ、変なペット、飼うことにしたの?この前、変わったペット専門店に行くって言ってたじゃない。 
\\	おぅ!行ったよ。すっごい店だったよ。ありとあらゆる動物がいたぜ。	
\\	おぅ!行ったよ。すっごい店だったよ。ありとあらゆる動物がいたぜ。 
\\	えっ本当?たとえば、どんな動物がいたのよ。	
\\	えっ本当?たとえば、どんな動物がいたのよ。 
\\	野生の七面鳥、アシカ、サメ、それから、タコ、サル、トカゲ、熊、あと、河童もいたよ。	
\\	野生の七面鳥、アシカ、サメ、それから、タコ、サル、トカゲ、熊、あと、河童もいたよ。 
\\	目移りして、なかなか決められなかったけど、結局、タコを飼うことにしたよ。	
\\	目移りして、なかなか決められなかったけど、結局、タコを飼うことにしたよ。 
\\	えっ、タコ?どこで飼ってるの?!	
\\	えっ、タコ?どこで飼ってるの?! 
\\	バスタブで飼ってたんだけどね。。。	
\\	バスタブで飼ってたんだけどね。。。 
\\	あー、無理とは言わないけど。。。でも、お風呂は、どうしてるの?!	
\\	あー、無理とは言わないけど。。。でも、お風呂は、どうしてるの?! 
\\	いやぁ、一緒にお風呂に入ろう…と思ってさ、でも、長湯し過ぎちゃったんだ。	
\\	いやぁ、一緒にお風呂に入ろう…と思ってさ、でも、長湯し過ぎちゃったんだ。 
\\	気づいたときには、タコ君。。。動かなくなっていたんだ。	
\\	気づいたときには、タコ君。。。動かなくなっていたんだ。 
\\	浴衣
\\	ひとカラ
\\	ギャル
\\	交代する
\\	へたくそ(な)
\\	済む
\\	ねぇ、来週って江戸城公園の花火大会じゃない?	
\\	ねぇ、来週って江戸城公園の花火大会じゃない? 
\\	アヤ、花火大好きなの。みんなで、浴衣を着て行かない?	
\\	アヤ、花火大好きなの。みんなで、浴衣を着て行かない? 
\\	いいねぇ。で、何曜日?	
\\	いいねぇ。で、何曜日? 
\\	土曜日。7時から。場所は男子がとってね。	
\\	土曜日。7時から。場所は男子がとってね。 
\\	アヤ、お弁当つくっていくから。	
\\	アヤ、お弁当つくっていくから。 
\\	土曜日かぁ・・・。土曜日はちょっとなぁ・・・。	
\\	土曜日かぁ・・・。土曜日はちょっとなぁ・・・。 
\\	マサさん、もしかして、渋谷のギャルとデートですかぁ?	
\\	マサさん、もしかして、渋谷のギャルとデートですかぁ? 
\\	違うよ。ひとカラしに行くんだよ。	
\\	違うよ。ひとカラしに行くんだよ。 
\\	あたしもひとカラ大好きですよ!他の人と交代しないですむし、ほんと楽しいですよねぇ。	
\\	あたしもひとカラ大好きですよ!他の人と交代しないですむし、ほんと楽しいですよねぇ。 
\\	オレも大好きですよ。	
\\	オレも大好きですよ。 
\\	駅前
\\	常連
\\	貯まる
\\	決まり
\\	間違っている
\\	オレも、先週末、駅前のビックエコーで、5時間ヒトカラしてましたよ。	
\\	オレも、先週末、駅前のビックエコーで、5時間ヒトカラしてましたよ。 
\\	まじで?俺も、昨日あそこで、5時間ヒトカラしてた。	
\\	まじで?俺も、昨日あそこで、5時間ヒトカラしてた。 
\\	え、私もあそこの常連ですよ。	
\\	え、私もあそこの常連ですよ。 
\\	週に3回くらい行ってたら、ポイントカードが3枚くらい貯まっちゃいましたよ。	
\\	週に3回くらい行ってたら、ポイントカードが3枚くらい貯まっちゃいましたよ。 
\\	みんなしてるんだったら、アヤも一回やってみたいなぁ!	
\\	みんなしてるんだったら、アヤも一回やってみたいなぁ! 
\\	いいじゃん、がんばって。	
\\	いいじゃん、がんばって。 
\\	えー。一人じゃ、いやだよ。そうだ、いい考えがある。。。 みんなで、浴衣着てひとカラしよう!	
\\	えー。一人じゃ、いやだよ。そうだ、いい考えがある。。。 みんなで、浴衣着てひとカラしよう! 
\\	みんなで?それはひとカラじゃないですよ。	
\\	みんなで?それはひとカラじゃないですよ。 
\\	カラオケ屋さんに入ってから、別々の部屋で歌ったら、ひとカラじゃないの?	
\\	カラオケ屋さんに入ってから、別々の部屋で歌ったら、ひとカラじゃないの? 
\\	まぁ、それも、ひとカラですね。	
\\	まぁ、それも、ひとカラですね。 
\\	じゃ、決まり!来週の土曜日は、みんなでひとカラしよっ!	
\\	じゃ、決まり!来週の土曜日は、みんなでひとカラしよっ! 
\\	実は
\\	音痴
\\	後悔
\\	恥ずかしい
\\	いらっしゃいませ。何名様ですか。	
\\	いらっしゃいませ。何名様ですか。 
\\	4名です。あ、でも、4部屋お願いします。	
\\	4名です。あ、でも、4部屋お願いします。 
\\	ねぇ。本当にカラオケするの?アヤ帰ってもいい?実は音痴なの。	
\\	ねぇ。本当にカラオケするの?アヤ帰ってもいい?実は音痴なの。 
\\	帰ってもかまわないけど、絶対、後悔するよ。	
\\	帰ってもかまわないけど、絶対、後悔するよ。 
\\	大丈夫、大丈夫。自分しかいないんだから、音痴でも恥ずかしくないですよ。	
\\	大丈夫、大丈夫。自分しかいないんだから、音痴でも恥ずかしくないですよ。 
\\	もしもし。	
\\	もしもし。 
\\	アヤさん、ひとカラはどうですか。楽しいですか。	
\\	アヤさん、ひとカラはどうですか。楽しいですか。 
\\	うん、すごく楽しい!なんで今までヒトカラしなかったんだろう。	
\\	うん、すごく楽しい!なんで今までヒトカラしなかったんだろう。 
\\	もしもし?秀美ちゃん?匠とアヤに連絡できないんだ。なんか聞いてない?	
\\	もしもし?秀美ちゃん?匠とアヤに連絡できないんだ。なんか聞いてない? 
\\	聞いてないですぅ。	
\\	聞いてないですぅ。 
\\	そうかぁ。オレ、そろそろ帰ってもかまわないかな?	
\\	そうかぁ。オレ、そろそろ帰ってもかまわないかな? 
\\	あ、じゃ、私も帰ります!	
\\	あ、じゃ、私も帰ります! 
\\	隣
\\	相手
\\	人生
\\	独身
\\	年頃
\\	家出
\\	くわえる
\\	ねずみ
\\	ずっと
\\	屋根
\\	うるさい
\\	隣のミケちゃんどうしたの。昨日から屋根の上にずっと座ったままよ。	
\\	隣のミケちゃんどうしたの。昨日から屋根の上にずっと座ったままよ。 
\\	ミケちゃんの子供が一人、家を出たまま帰ってこないんだって。	
\\	ミケちゃんの子供が一人、家を出たまま帰ってこないんだって。 
\\	あらあら、いつまでも子供のままだと思っていたけど、ボーイフレンドとデート?	
\\	あらあら、いつまでも子供のままだと思っていたけど、ボーイフレンドとデート? 
\\	ねずみをくわえたまま歩いているのをご主人さんが見て、すごく怒ったんだって。	
\\	ねずみをくわえたまま歩いているのをご主人さんが見て、すごく怒ったんだって。 
\\	ああ、それで、家出。	
\\	ああ、それで、家出。 
\\	難しい年頃だからね。ご主人さんに腹を立てて、家を出て、そのままだって。	
\\	難しい年頃だからね。ご主人さんに腹を立てて、家を出て、そのままだって。 
\\	親は大変よね。私、独身のままで本当によかったわ。	
\\	親は大変よね。私、独身のままで本当によかったわ。 
\\	ま、姉さんの場合、相手がいないからね。	
\\	ま、姉さんの場合、相手がいないからね。 
\\	でも、いつまでも若いままじゃないんだからさ、すこし、考えたら、自分の人生。	
\\	でも、いつまでも若いままじゃないんだからさ、すこし、考えたら、自分の人生。 
\\	連絡
\\	送別会
\\	同僚
\\	クラブ
\\	気づく
\\	携帯
\\	怒る
\\	別れる
\\	信用する
\\	(カチャッ)	
\\	(カチャッ) 
\\	(ミシッミシッ)	
\\	(ミシッミシッ) 
\\	(電気がつく)	
\\	(電気がつく) 
\\	ちょっと。	
\\	ちょっと。 
\\	は、はい。	
\\	は、はい。 
\\	連絡もせずに、今までどこに行ってたの。	
\\	連絡もせずに、今までどこに行ってたの。 
\\	会社の送別会があったんだ。同僚とカラオケに行って、そのあと、クラブに行って。	
\\	会社の送別会があったんだ。同僚とカラオケに行って、そのあと、クラブに行って。 
\\	今、何時か知ってる?	
\\	今、何時か知ってる? 
\\	さぁ…。4時くらい・・・ですか・・・ね?	
\\	さぁ…。4時くらい・・・ですか・・・ね? 
\\	ほう。電話にもメールにも気づかずに、遊んでいたんだ?	
\\	ほう。電話にもメールにも気づかずに、遊んでいたんだ? 
\\	まさか
\\	殴る
\\	蚊
\\	刺す
\\	かんかん
\\	追い出す
\\	怒る
\\	よう、おはよう。けんちゃん。昨夜、彼女に怒られたか?	
\\	よう、おはよう。けんちゃん。昨夜、彼女に怒られたか? 
\\	ああ、最悪だよ。悪かったなぁ、山田。クミ、昨日の夜遅く電話しただろう?ごめんな。	
\\	ああ、最悪だよ。悪かったなぁ、山田。クミ、昨日の夜遅く電話しただろう?ごめんな。 
\\	いいんだよ・・・。え、け、けんちゃん、どうした?その顔?まさか、彼女に殴られたのか?	
\\	いいんだよ・・・。え、け、けんちゃん、どうした?その顔?まさか、彼女に殴られたのか? 
\\	まさか。蚊に刺されたんだよ。	
\\	まさか。蚊に刺されたんだよ。 
\\	蚊?	
\\	蚊? 
\\	クミ、昨夜、カンカンでさ。俺、部屋から追い出されて、ベランダで寝たんだ。家、ワンルームだからさ。	
\\	クミ、昨夜、カンカンでさ。俺、部屋から追い出されて、ベランダで寝たんだ。家、ワンルームだからさ。 
\\	ひぇ~。	
\\	ひぇ~。 
\\	で、朝起きたら、こんな顔になっていた。で、何時くらいに電話があった?	
\\	で、朝起きたら、こんな顔になっていた。で、何時くらいに電話があった? 
\\	ああ。2時くらいだったかな。(あくび)	
\\	ああ。2時くらいだったかな。(あくび) 
\\	ええ?2時?あいつ、そんな時間に電話したのか?	
\\	ええ?2時?あいつ、そんな時間に電話したのか? 
\\	ああ、(あくび)5分おきに電話がきたよ。「連絡つかないんですぅ・・・。」って。	
\\	"ああ、(あくび)5分おきに電話がきたよ。「連絡つかないんですぅ・・・。」って。 
\\	格好
\\	お出掛け
\\	付き合う
\\	取引先
\\	元彼
\\	振る
\\	モテモテ
\\	ねぇ、梨花、今日、何かあるの?すごくかわいい格好してる。	
\\	ねぇ、梨花、今日、何かあるの?すごくかわいい格好してる。 
\\	・・・実は、今日、ケンジ君とお出掛けなの。	
\\	・・・実は、今日、ケンジ君とお出掛けなの。 
\\	梨花、とうとう、ケンジ君をデートに誘ったの?	
\\	梨花、とうとう、ケンジ君をデートに誘ったの? 
\\	ううん。私が、ケンジ君に誘われたんだ。 「明日、ちょっと、買い物付き合って」って。	
\\	"ううん。私が、ケンジ君に誘われたんだ。 「明日、ちょっと、買い物付き合って」って。 
\\	へぇ。。でも、どうするの? この間、部長が梨花に取引先の人を紹介するって言ってたじゃない。	
\\	へぇ。。でも、どうするの? この間、部長が梨花に取引先の人を紹介するって言ってたじゃない。 
\\	う...ん。そうなんだよね。先週、部長に取引先の人を紹介されたんだ。それでその後、課長に別の取引先の人を紹介されたんだ。	
\\	う...ん。そうなんだよね。先週、部長に取引先の人を紹介されたんだ。それでその後、課長に別の取引先の人を紹介されたんだ。 
\\	どっちの人とも何度か食事に行ってるんだけど、今は様子を見ているかんじかな。	
\\	どっちの人とも何度か食事に行ってるんだけど、今は様子を見ているかんじかな。 
\\	へぇー。もてるねぇ。	
\\	へぇー。もてるねぇ。 
\\	告白する
\\	社内恋愛
\\	噂する
\\	嫉妬深い
\\	旦那
\\	あっさり
\\	で、結局、ケンジ君と買い物行ったら、「彼女の指輪を一緒に選んで。」って言われたの。超ショックよぉ。	
\\	"で、結局、ケンジ君と買い物行ったら、「彼女の指輪を一緒に選んで。」って言われたの。超ショックよぉ。 
\\	へぇー。で、ケンジ君には告白したの?	
\\	へぇー。で、ケンジ君には告白したの? 
\\	一応、したけど、あっさり振られたわ。「結婚したい子がいるから」って言われた。	
\\	"一応、したけど、あっさり振られたわ。「結婚したい子がいるから」って言われた。 
\\	あらあら・・・。	
\\	あらあら・・・。 
\\	ま、しょうがないわね。次よ。次。もっと、良い男を見つけるんだ。でも、社内恋愛はやめるわ。皆に、噂されるからね。	
\\	ま、しょうがないわね。次よ。次。もっと、良い男を見つけるんだ。でも、社内恋愛はやめるわ。皆に、噂されるからね。 
\\	確かにね。	
\\	確かにね。 
\\	(ため息)あかりちゃんは、いいわね。素敵な旦那様に愛されていて。	
\\	(ため息)あかりちゃんは、いいわね。素敵な旦那様に愛されていて。 
\\	素敵かどうかは、分からないけど、まぁ、愛されてるかもね。でも、嫉妬深くて大変よ。	
\\	"素敵かどうかは、分からないけど、まぁ、愛されてるかもね。でも、嫉妬深くて大変よ。 
\\	この前、旦那に携帯のメールを見られて、男の子のアドレスを全部消されたわ!	
\\	この前、旦那に携帯のメールを見られて、男の子のアドレスを全部消されたわ! 
\\	七面鳥
\\	農場
\\	ローストターキー
\\	オーブン
\\	羽
\\	絞める
\\	出発
\\	楽しみ
\\	連れて行く
\\	期待
\\	感謝祭
\\	サンクスギビング
\\	農夫
\\	(農場)	
\\	(農場) 
\\	あ、おはようございます、七面鳥さん。・・・もうすぐサンクスギビングですね。	
\\	あ、おはようございます、七面鳥さん。・・・もうすぐサンクスギビングですね。 
\\	そうなんですよ!僕、楽しみで楽しみで。サンクスギビングのイベントは初めてなんです。	
\\	そうなんですよ!僕、楽しみで楽しみで。サンクスギビングのイベントは初めてなんです。 
\\	・・・そう・・・ですか。	
\\	"・・・そう・・・ですか。 
\\	「感謝祭は、君がヒーローだ!」とご主人さんに言われているんですよ。すごくないですか。僕、みんなに期待されているんです。	
\\	「感謝祭は、君がヒーローだ!」とご主人さんに言われているんですよ。すごくないですか。僕、みんなに期待されているんです。 
\\	・・・そう・・・ですか。さみしくなりますね。どこに連れて行かれるんですか。	
\\	・・・そう・・・ですか。さみしくなりますね。どこに連れて行かれるんですか。 
\\	ホワイトハウスと言われました。そこで、兄弟達を紹介されるので、とても楽しみなんです。	
\\	"ホワイトハウスと言われました。そこで、兄弟達を紹介されるので、とても楽しみなんです。 
\\	あ、ご主人さんが、呼んでいる。そろそろ出発の時間です。じゃ、また。	
\\	あ、ご主人さんが、呼んでいる。そろそろ出発の時間です。じゃ、また。 
\\	・・・「じゃ、また。」って・・・。かわいそうね。知らないのかしら?	
\\	・・・「じゃ、また。」って・・・。かわいそうね。知らないのかしら? 
\\	絞められて、羽をとられて、オーブンに入れられて、焼かれて…。	
\\	絞められて、羽をとられて、オーブンに入れられて、焼かれて…。 
\\	やめて!やめて!もう十分。	
\\	やめて!やめて!もう十分。 
\\	だます
\\	にらまれる
\\	本気にする
\\	水に流す
\\	お歳暮
\\	好み
\\	星野先輩!お久しぶりです!実は、相談があるんですが。	
\\	星野先輩!お久しぶりです!実は、相談があるんですが。 
\\	あっ、五味。お前には、もう、だまされないぞ。	
\\	あっ、五味。お前には、もう、だまされないぞ。 
\\	えっ!何のことですか?	
\\	えっ!何のことですか? 
\\	俺の部長は漫画が大嫌いなのに、漫画好きだって、俺をだましただろ。俺は、あれから部長ににらまれて大変だったんだぞ!	
\\	俺の部長は漫画が大嫌いなのに、漫画好きだって、俺をだましただろ。俺は、あれから部長ににらまれて大変だったんだぞ! 
\\	すいません。本気にすると思わなかったんです。本当にすいません。	
\\	すいません。本気にすると思わなかったんです。本当にすいません。 
\\	まぁ、いいだろ。水に流してやるよ。	
\\	まぁ、いいだろ。水に流してやるよ。 
\\	ありがとうございます!ところで、先輩、そろそろお歳暮の季節ですね。やっぱり、上司にお歳暮を贈った方がいいでしょうか。	
\\	ありがとうございます!ところで、先輩、そろそろお歳暮の季節ですね。やっぱり、上司にお歳暮を贈った方がいいでしょうか。 
\\	そりゃそうだろ。	
\\	そりゃそうだろ。 
\\	何が喜ばれると思いますか?	
\\	何が喜ばれると思いますか? 
\\	う~ん、それは、その人の性格とか好みによって違うよ。五味の部長は、坂口部長だよな。今度、会う機会があるから、ちょっと聞いてみようか。	
\\	う~ん、それは、その人の性格とか好みによって違うよ。五味の部長は、坂口部長だよな。今度、会う機会があるから、ちょっと聞いてみようか。 
\\	助かります!よろしくお願いします!	
\\	助かります!よろしくお願いします! 
\\	仕方(が)ない
\\	媚びる
\\	第一
\\	参考になる
\\	態度
\\	禁止
\\	あ、坂口部長、ちょっと、よろしいですか。	
\\	あ、坂口部長、ちょっと、よろしいですか。 
\\	なんだ、星野君。	
\\	なんだ、星野君。 
\\	実は、お客さんへのお歳暮を何にしたらいいか、ご意見を聞きたいと思いまして。。.	
\\	実は、お客さんへのお歳暮を何にしたらいいか、ご意見を聞きたいと思いまして。。. 
\\	そうだなぁ。子供がいる家庭なら、子供に喜ばれるお菓子がいいんじゃないか。	
\\	そうだなぁ。子供がいる家庭なら、子供に喜ばれるお菓子がいいんじゃないか。 
\\	それから、スーパーで売られていない高級な食べ物もいいな。	
\\	それから、スーパーで売られていない高級な食べ物もいいな。 
\\	なるほど~。	
\\	なるほど~。 
\\	まぁ、取引先は仕方がないが、お歳暮は、贈るのも、贈られるのも、嫌いだね。	
\\	まぁ、取引先は仕方がないが、お歳暮は、贈るのも、贈られるのも、嫌いだね。 
\\	最近、部下からもお歳暮が贈られてくるが、はっきり言って、迷惑なんだ。	
\\	最近、部下からもお歳暮が贈られてくるが、はっきり言って、迷惑なんだ。 
\\	あ、そうなんですか?	
\\	あ、そうなんですか? 
\\	包み
\\	けしからん
\\	超
\\	高級
\\	無視する
\\	詰め合わせ
\\	いい気味
\\	あなた~。五味さんからお歳暮が贈られてきましたよ。	
\\	あなた~。五味さんからお歳暮が贈られてきましたよ。 
\\	え?お歳暮?あいつめ。(ガサガサ;包みを開ける)	
\\	え?お歳暮?あいつめ。(ガサガサ;包みを開ける) 
\\	こんなに、高いものを・・・。。五味は、私の性格をわかっていないな。	
\\	こんなに、高いものを・・・。。五味は、私の性格をわかっていないな。 
\\	迷惑だと思われることがわからなかったのか。まったくけしからん。	
\\	迷惑だと思われることがわからなかったのか。まったくけしからん。 
\\	星野先輩!この間はありがとうございました。。。	
\\	星野先輩!この間はありがとうございました。。。 
\\	先輩のアドバイスどおり、超高級フルーツ詰め合わせを贈りました。	
\\	先輩のアドバイスどおり、超高級フルーツ詰め合わせを贈りました。 
\\	そうか。	
\\	そうか。 
\\	でも、最近、部長に無視されるんですよね。。。それに、お礼も言われないんですよ。。。	
\\	でも、最近、部長に無視されるんですよね。。。それに、お礼も言われないんですよ。。。 
\\	さえない
\\	まし
\\	それはない。
\\	ドタキャン
\\	予定をあけておく
\\	(電話をする)	
\\	(電話をする) 
\\	茂?クリスマスディナーの予約とっておいたよ。	
\\	茂?クリスマスディナーの予約とっておいたよ。 
\\	おー、サンキュー!	
\\	おー、サンキュー! 
\\	男二人で、クリスマスイブを過ごすなんて、さえないけど、しょうがないな。	
\\	男二人で、クリスマスイブを過ごすなんて、さえないけど、しょうがないな。 
\\	一人よりは、ましだろ。	
\\	一人よりは、ましだろ。 
\\	じゃ、大江戸線の六本木駅改札に6時半にしよう。	
\\	じゃ、大江戸線の六本木駅改札に6時半にしよう。 
\\	オッケー。	
\\	オッケー。 
\\	茂、遅いなぁ~。	
\\	茂、遅いなぁ~。 
\\	(電話がなる)	
\\	(電話がなる) 
\\	あっ、茂。 えっ!!!!ドタキャン?!	
\\	あっ、茂。 えっ!!!!ドタキャン?! 
\\	親友
\\	連続
\\	届ける
\\	恥ずかしい
\\	人生
\\	裏
\\	体育館
\\	落ちる
\\	手紙
\\	下駄箱
\\	初
\\	さ、帰るか。今年も、誰もチョコレートをくれなかった...。	
\\	さ、帰るか。今年も、誰もチョコレートをくれなかった...。 
\\	(はぁ・・・)俺は、16年連続バレンタインデーのチョコレートをもらう事ができなかったのか。	
\\	(はぁ・・・)俺は、16年連続バレンタインデーのチョコレートをもらう事ができなかったのか。 
\\	あれっ、手紙だ!読んでみよう。	
\\	"あれっ、手紙だ!読んでみよう。 
\\	圭介君へ、今日の5時に体育館裏に来てください。夏子より。	
\\	圭介君へ、今日の5時に体育館裏に来てください。夏子より。 
\\	おー!!!!俺に、夏子ちゃんがチョコレートをくれる!やったー!(走る!)	
\\	おー!!!!俺に、夏子ちゃんがチョコレートをくれる!やったー!(走る!) 
\\	(どきどき)こんにちはー。	
\\	(どきどき)こんにちはー。 
\\	あのー、圭介君。これ・・・	
\\	あのー、圭介君。これ・・・ 
\\	(心の声で)やったぞ!ついに、人生初のチョコレートをもらうぞ!	
\\	(心の声で)やったぞ!ついに、人生初のチョコレートをもらうぞ! 
\\	このチョコレートを浩二君にあげて!	
\\	このチョコレートを浩二君にあげて! 
\\	私…恥ずかしくてあげることが出来ないの。	
\\	私…恥ずかしくてあげることが出来ないの。 
\\	くらい
\\	出る
\\	ニキビ
\\	できる
\\	用
\\	上がる
\\	浩二、ちょっと出てくれるー?	
\\	浩二、ちょっと出てくれるー? 
\\	はーい。(ドタバタ)	
\\	はーい。(ドタバタ) 
\\	おっす!!どうした?ま、上がれよ。	
\\	おっす!!どうした?ま、上がれよ。 
\\	お邪魔しまーす。	
\\	お邪魔しまーす。 
\\	うわっ!!チョコレートがたくさんある!!今年もたくさんもらったな。	
\\	うわっ!!チョコレートがたくさんある!!今年もたくさんもらったな。 
\\	ああ、20個くらいもらったよ。全部食べられないから、ちょっと食べてくれる?	
\\	ああ、20個くらいもらったよ。全部食べられないから、ちょっと食べてくれる? 
\\	あまりチョコを食べないことにしてるんだ。ニキビができちゃうから。で、用は何?	
\\	あまりチョコを食べないことにしてるんだ。ニキビができちゃうから。で、用は何? 
\\	これ、夏子ちゃんからのバレンタインチョコ。	
\\	これ、夏子ちゃんからのバレンタインチョコ。 
\\	うらやましい
\\	性格
\\	合う
\\	おとなしい
\\	わざわざ
\\	返す
\\	連れて行く
\\	本命
\\	去年も夏子ちゃんからもらったのか。本当にうらやましいな。	
\\	去年も夏子ちゃんからもらったのか。本当にうらやましいな。 
\\	実は、彼女、ちょっと性格が合わないんだよね。すごいおとなしいから、話が続かないんだ。	
\\	実は、彼女、ちょっと性格が合わないんだよね。すごいおとなしいから、話が続かないんだ。 
\\	圭介は、今年誰かからチョコもらった?	
\\	圭介は、今年誰かからチョコもらった? 
\\	ゼロだよ・・・。	
\\	ゼロだよ・・・。 
\\	じゃあ、夏子ちゃんのチョコレートも、もらってくれるか。	
\\	じゃあ、夏子ちゃんのチョコレートも、もらってくれるか。 
\\	え?いいの?いただきまーす。	
\\	え?いいの?いただきまーす。 
\\	(もぐもぐ)うまい!!こんなおいしいチョコは初めてだ。	
\\	(もぐもぐ)うまい!!こんなおいしいチョコは初めてだ。 
\\	…あ、でも、食べてあげないの?わざわざ作ってくれたんだよ。	
\\	…あ、でも、食べてあげないの?わざわざ作ってくれたんだよ。 
\\	俺は、本命の冬花ちゃんのチョコだけでいいんだ。	
\\	俺は、本命の冬花ちゃんのチョコだけでいいんだ。 
\\	お返し
\\	渡す
\\	もしかして
\\	分ける
\\	ひどい
\\	義理チョコ
\\	届ける
\\	おはよう!夏子ちゃん。そのバックかわいいね。	
\\	おはよう!夏子ちゃん。そのバックかわいいね。 
\\	あ、ありがとう。お父さんに買ってもらったの。ホワイトデーのお返し。	
\\	あ、ありがとう。お父さんに買ってもらったの。ホワイトデーのお返し。 
\\	夏子ちゃんどうしたの?元気ない?	
\\	夏子ちゃんどうしたの?元気ない? 
\\	ねぇ、圭介くん。浩二君にチョコレート渡してくれた?	
\\	ねぇ、圭介くん。浩二君にチョコレート渡してくれた? 
\\	え?ああ。もちろん、届けてあげたよ。	
\\	え?ああ。もちろん、届けてあげたよ。 
\\	そうだよね。届けてくれたよね。	
\\	そうだよね。届けてくれたよね。 
\\	あ、浩二君、私のチョコレート食べてくれたかなぁ。	
\\	あ、浩二君、私のチョコレート食べてくれたかなぁ。 
\\	あ、もしかして、チョコレート美味しくなかったのかなぁ。	
\\	あ、もしかして、チョコレート美味しくなかったのかなぁ。 
\\	いや、あれは、すごくおいしかったよ。	
\\	いや、あれは、すごくおいしかったよ。 
\\	え?なに?!私のチョコ食べたの?	
\\	え?なに?!私のチョコ食べたの? 
\\	いやいやいやいや。食べてないよ。	
\\	いやいやいやいや。食べてないよ。 
\\	浩二が食べきれないから、ちょっとだけ分けてもらったんだよ。	
\\	浩二が食べきれないから、ちょっとだけ分けてもらったんだよ。 
\\	右側
\\	客室乗務員
\\	通路
\\	座席
\\	電源
\\	携帯電話
\\	予定
\\	便
\\	飛行機
\\	緊張する
\\	離陸する
\\	すみません。僕の座席はどこですか。	
\\	すみません。僕の座席はどこですか。 
\\	こちらの通路をまっすぐ行って、右側です。	
\\	こちらの通路をまっすぐ行って、右側です。 
\\	ありがとうございます。	
\\	ありがとうございます。 
\\	よいしょ。緊張するなあ。日本に行くの初めてだから。	
\\	よいしょ。緊張するなあ。日本に行くの初めてだから。 
\\	でも、大丈夫。あっちでは、和樹が待っているから。	
\\	でも、大丈夫。あっちでは、和樹が待っているから。 
\\	この飛行機はジャパニーズポッド、101便です。私(わたくし)は、客室乗務員の田中です。	
\\	この飛行機はジャパニーズポッド、101便です。私(わたくし)は、客室乗務員の田中です。 
\\	揺れる
\\	隣
\\	頼む
\\	免許
\\	反対側
\\	機内食
\\	窓側
\\	久しぶり、和樹!	
\\	久しぶり、和樹! 
\\	久しぶり、ジャック!フライトはどうだった?	
\\	久しぶり、ジャック!フライトはどうだった? 
\\	良かったよ。全然揺れなかったし、機内食、カレーで美味しかった。	
\\	良かったよ。全然揺れなかったし、機内食、カレーで美味しかった。 
\\	それより、トイレ行っても良い?一度もトイレ行かなかったから。	
\\	それより、トイレ行っても良い?一度もトイレ行かなかったから。 
\\	何で行かなかったんだよ。	
\\	何で行かなかったんだよ。 
\\	窓側に座ってたんだけど、隣の日本人がすごい美人でさ。	
\\	窓側に座ってたんだけど、隣の日本人がすごい美人でさ。 
\\	だから、緊張して、頼めなかったんだよ。	
\\	だから、緊張して、頼めなかったんだよ。 
\\	じゃ、この荷物、見ててくれない?	
\\	じゃ、この荷物、見ててくれない? 
\\	分かったよ。ここで待ってるね。	
\\	分かったよ。ここで待ってるね。 
\\	ただいま。	
\\	ただいま。 
\\	じゃあ、行こうか。あ、俺、先月免許取ったんだ。	
\\	じゃあ、行こうか。あ、俺、先月免許取ったんだ。 
\\	じゃあ、そのジャックのスーツケース、トランクに入れてくれない?	
\\	じゃあ、そのジャックのスーツケース、トランクに入れてくれない? 
\\	間違える
\\	にんじん
\\	じゃがいも
\\	隠し味
\\	探す
\\	お使い
\\	玉ねぎ
\\	豚肉
\\	ただいまー。	
\\	ただいまー。 
\\	お邪魔しまーす。	
\\	お邪魔しまーす。 
\\	あら、いらっしゃい。チャック君ね。	
\\	あら、いらっしゃい。チャック君ね。 
\\	お母さん。チャックじゃないよ。ジャック。間違えないでよ。	
\\	お母さん。チャックじゃないよ。ジャック。間違えないでよ。 
\\	ごめんなさいね。ジャック君。	
\\	ごめんなさいね。ジャック君。 
\\	着いたばかりで悪いんだけど、二人で、お使いに行ってもらえない?	
\\	着いたばかりで悪いんだけど、二人で、お使いに行ってもらえない? 
\\	はーい。	
\\	はーい。 
\\	ジャック、メモになんて書いてある?読んでくれない?	
\\	ジャック、メモになんて書いてある?読んでくれない? 
\\	えっと、玉ねぎ、にんじん、じゃがいも、豚肉、チョコレート。	
\\	えっと、玉ねぎ、にんじん、じゃがいも、豚肉、チョコレート。 
\\	あっ、今夜はカレーだね!	
\\	あっ、今夜はカレーだね! 
\\	えっ!またカレー?!	
\\	えっ!またカレー?! 
\\	でも、なんで、カレーにチョコレートなんて入れるの?	
\\	でも、なんで、カレーにチョコレートなんて入れるの? 
\\	隠し味だよ。隠し味。	
\\	隠し味だよ。隠し味。 
\\	(・・・でも、かくしあじ?かくしあじってなんだ?)	
\\	(・・・でも、かくしあじ?かくしあじってなんだ?) 
\\	じゃあ、おれはチョコレートを探すから、ジャックは豚肉探してもらえない?	
\\	じゃあ、おれはチョコレートを探すから、ジャックは豚肉探してもらえない? 
\\	連れて行く
\\	捕まえる
\\	写る
\\	下がる
\\	下げる
\\	凝る
\\	空気
\\	ジャック、今日はどこ行きたい?	
\\	ジャック、今日はどこ行きたい? 
\\	うーん、自然が多いところに連れて行ってくれない?	
\\	うーん、自然が多いところに連れて行ってくれない? 
\\	分かった。じゃあ、高尾山に行こう!	
\\	分かった。じゃあ、高尾山に行こう! 
\\	いやー、高尾山は、空気がきれいだね。	
\\	いやー、高尾山は、空気がきれいだね。 
\\	あっ、あれリスだ!!和樹、捕まえようぜ!	
\\	あっ、あれリスだ!!和樹、捕まえようぜ! 
\\	おっけー!	
\\	おっけー! 
\\	やったー!捕まえた!(リスの鳴き声)	
\\	やったー!捕まえた!(リスの鳴き声) 
\\	じゃあ、リスと一緒に記念写真を撮ろう。	
\\	じゃあ、リスと一緒に記念写真を撮ろう。 
\\	すみませーん。写真を撮っていただけませんか。	
\\	すみませーん。写真を撮っていただけませんか。 
\\	写らないので、もうちょっと、下がっていただけませんか。	
\\	写らないので、もうちょっと、下がっていただけませんか。 
\\	あ・・・もうちょっと右ですね。	
\\	あ・・・もうちょっと右ですね。 
\\	かける
\\	事件
\\	解決する
\\	用件
\\	海外旅行
\\	盗む
\\	払う
\\	探偵
\\	購入する
\\	錦鯉
\\	事務所
\\	ごめんくださーい。 探偵事務所Xはこちらですか。	
\\	ごめんくださーい。 探偵事務所Xはこちらですか。 
\\	そうです。さぁ、どうぞ、こちらへおかけください。	
\\	そうです。さぁ、どうぞ、こちらへおかけください。 
\\	私(わたくし)、探偵
\\	と申します。どんな事件でも解決します。	
\\	私(わたくし)、探偵
\\	と申します。どんな事件でも解決します。 
\\	今日は、どんなご用件ですか。	
\\	今日は、どんなご用件ですか。 
\\	徳川キミと申します。	
\\	徳川キミと申します。 
\\	実は、家(うち)の池にいた錦鯉がいなくなってしまったんです。見つけてもらえますか。	
\\	実は、家(うち)の池にいた錦鯉がいなくなってしまったんです。見つけてもらえますか。 
\\	そんなの朝飯前ですよ。いついなくなったんですか。	
\\	そんなの朝飯前ですよ。いついなくなったんですか。 
\\	昨日、1週間の海外旅行から帰ってきたんです。。	
\\	昨日、1週間の海外旅行から帰ってきたんです。。 
\\	庭
\\	謎
\\	普段
\\	聞き込み
\\	開始する
\\	不審
\\	些細
\\	ここが、キミさんの家の庭ですね。	
\\	ここが、キミさんの家の庭ですね。 
\\	そして、一週間前までは、この池に錦鯉が3匹いた。	
\\	そして、一週間前までは、この池に錦鯉が3匹いた。 
\\	はい、そうだったと思います。	
\\	はい、そうだったと思います。 
\\	では、近所の人に聞き込みを開始しましょう。	
\\	では、近所の人に聞き込みを開始しましょう。 
\\	はい。	
\\	はい。 
\\	この1週間で、誰か、不審な人は見ませんでしたか。	
\\	この1週間で、誰か、不審な人は見ませんでしたか。 
\\	変な人はいなかったと思います。	
\\	変な人はいなかったと思います。 
\\	不審者はいない・・・。	
\\	不審者はいない・・・。 
\\	では、何か、音は聞こえませんでしたか。バシャバシャという水の音とか。	
\\	では、何か、音は聞こえませんでしたか。バシャバシャという水の音とか。 
\\	いえ、別に大きい音は聞かなかったと思いますよ。	
\\	いえ、別に大きい音は聞かなかったと思いますよ。 
\\	物音
\\	隣人
\\	弟子
\\	真犯人
\\	極端に
\\	習性
\\	つかむ
\\	犯人
\\	推理力
\\	放送
\\	うーん。不審な人はいなかった。物音もしなかった。	
\\	うーん。不審な人はいなかった。物音もしなかった。 
\\	わからない・・・。犯人は誰だろう。	
\\	わからない・・・。犯人は誰だろう。 
\\	ん・・!!ひらめきました!	
\\	ん・・!!ひらめきました! 
\\	私、犯人が分かりました。	
\\	私、犯人が分かりました。 
\\	えっっ!!	
\\	えっっ!! 
\\	犯人は、人間ではありません。	
\\	犯人は、人間ではありません。 
\\	動物か!もしかして猫ですか。	
\\	動物か!もしかして猫ですか。 
\\	いいえ、猫の習性を考えてください。	
\\	いいえ、猫の習性を考えてください。 
\\	猫は、水にぬれるのを極端に嫌うんです。おそらく、水には近づかないでしょう。	
\\	猫は、水にぬれるのを極端に嫌うんです。おそらく、水には近づかないでしょう。 
\\	なるほど。ということは、犬が真犯人ですね!	
\\	なるほど。ということは、犬が真犯人ですね! 
\\	微妙
\\	くわえる
\\	ばれる
\\	必要
\\	結末
\\	逮捕する
\\	間抜け
\\	この推理ドラマ微妙な話だなぁ。	
\\	この推理ドラマ微妙な話だなぁ。 
\\	しかも、結末は来週?	
\\	しかも、結末は来週? 
\\	私は、鳥が犯人だと思うな。	
\\	私は、鳥が犯人だと思うな。 
\\	そうかな。鳥が錦鯉をくわえられるはずがないよ。	
\\	そうかな。鳥が錦鯉をくわえられるはずがないよ。 
\\	やっぱり、人間が犯人のはずだよ。	
\\	やっぱり、人間が犯人のはずだよ。 
\\	確かに、錦鯉って本当に高いし、犯人は、盗んだ錦鯉をネットオークションで売るかもしれないよね。	
\\	確かに、錦鯉って本当に高いし、犯人は、盗んだ錦鯉をネットオークションで売るかもしれないよね。 
\\	そうそう。犯人は、錦鯉をどこかで売るはずだよ。	
\\	そうそう。犯人は、錦鯉をどこかで売るはずだよ。 
\\	社員
\\	清掃
\\	教育する
\\	社長
\\	部長
\\	新人
\\	サボる
\\	やり直す
\\	部下
\\	担当
\\	汚い
\\	気持ち
\\	掃除
\\	部署
\\	宮島物産では社員が毎日トイレ掃除をしているんですか。それは、どうしてでしょうか。	
\\	宮島物産では社員が毎日トイレ掃除をしているんですか。それは、どうしてでしょうか。 
\\	社員を教育するためです。	
\\	社員を教育するためです。 
\\	私は、社員に、他の人の気持ちを考えるように言っています。	
\\	私は、社員に、他の人の気持ちを考えるように言っています。 
\\	なるほど。ユニークなアイディアですね。	
\\	なるほど。ユニークなアイディアですね。 
\\	さて、今日もトイレをチェックするか。	
\\	さて、今日もトイレをチェックするか。 
\\	ん!なんだ、この汚いトイレは!	
\\	ん!なんだ、この汚いトイレは! 
\\	ここの清掃の担当は・・・・(メモをチェックする音)田中の部署か。	
\\	ここの清掃の担当は・・・・(メモをチェックする音)田中の部署か。 
\\	田中君、2階のトイレ、汚かったぞ。	
\\	田中君、2階のトイレ、汚かったぞ。 
\\	申し訳ありません。部下に、やり直すように伝えます。	
\\	申し訳ありません。部下に、やり直すように伝えます。 
\\	吉川君、また、トイレ掃除をサボったでしょ。	
\\	吉川君、また、トイレ掃除をサボったでしょ。 
\\	え?・・・はい。	
\\	え?・・・はい。 
\\	この前も、サボらないように言ったでしょう。	
\\	この前も、サボらないように言ったでしょう。 
\\	給料カットするわよ(笑)。	
\\	給料カットするわよ(笑)。 
\\	申し訳ありませんでした。これからは、気をつけます。	
\\	申し訳ありませんでした。これからは、気をつけます。 
\\	(心の声で)「新人の上田に、掃除するように頼んだんだけど、あいつ、サボったな。」	
\\	(心の声で)「新人の上田に、掃除するように頼んだんだけど、あいつ、サボったな。」 
\\	仕事
\\	片付ける
\\	一緒に
\\	引越し
\\	新居
\\	参加する
\\	開く
\\	持っていく
\\	ペーターさんの引越しパーティー行く?	
\\	ペーターさんの引越しパーティー行く? 
\\	はい。ペーターさんの新しい新居見てみたいですね。	
\\	はい。ペーターさんの新しい新居見てみたいですね。 
\\	私も参加します。誰が来るんですか。	
\\	私も参加します。誰が来るんですか。 
\\	誰が来るか分からないんだよね。	
\\	誰が来るか分からないんだよね。 
\\	いつ開くかもまだ決めてないみたい。	
\\	いつ開くかもまだ決めてないみたい。 
\\	じゃ、僕、ペーターさんにちょっと聞いてみます。	
\\	じゃ、僕、ペーターさんにちょっと聞いてみます。 
\\	ペーターさん、いつ引越しパーティーをするか決めましたか。	
\\	ペーターさん、いつ引越しパーティーをするか決めましたか。 
\\	来週の金曜の夕方にするつもり。みんな、来れる?	
\\	来週の金曜の夕方にするつもり。みんな、来れる? 
\\	行けますよ。	
\\	行けますよ。 
\\	私は、仕事を片付けてから、行きます。	
\\	私は、仕事を片付けてから、行きます。 
\\	私も仕事が何時に終わるかわからないけど、たぶん行きます。	
\\	私も仕事が何時に終わるかわからないけど、たぶん行きます。 
\\	直子さん、明日のパーティーに何を持っていくか決めました?	
\\	直子さん、明日のパーティーに何を持っていくか決めました? 
\\	あ・・・まだ。そうだ、みんなで引越しのプレゼント一緒に買わない?	
\\	あ・・・まだ。そうだ、みんなで引越しのプレゼント一緒に買わない? 
\\	いいですね。プレゼント何にするか考えましょう。	
\\	いいですね。プレゼント何にするか考えましょう。 
\\	入り口
\\	表札
\\	外食する
\\	食器
\\	決定
\\	賛成
\\	通う
\\	微妙
\\	気に入る
\\	引越し祝い
\\	引越し祝いに、食器はどうですか。	
\\	引越し祝いに、食器はどうですか。 
\\	うーん。ペーターさん、毎日外食していますよね。	
\\	うーん。ペーターさん、毎日外食していますよね。 
\\	だから食器を使うかどうか、わかりませんよね。	
\\	だから食器を使うかどうか、わかりませんよね。 
\\	じゃあ、表札はどう?	
\\	じゃあ、表札はどう? 
\\	え?表札(笑)?表札って...家の入り口にある、表札ですか?	
\\	え?表札(笑)?表札って...家の入り口にある、表札ですか? 
\\	ペーターさんが気に入るかどうか微妙ですね。	
\\	ペーターさんが気に入るかどうか微妙ですね。 
\\	あと、英語で書くか、日本語で書くかわからないですね。	
\\	あと、英語で書くか、日本語で書くかわからないですね。 
\\	そっか・・・。表札はだめか。	
\\	そっか・・・。表札はだめか。 
\\	ねぇ、ぺーターが何が好きか分かる?	
\\	ねぇ、ぺーターが何が好きか分かる? 
\\	仕事?	
\\	仕事? 
\\	会社?	
\\	会社? 
\\	奥さん?	
\\	奥さん? 
\\	プレゼントできないじゃない!	
\\	プレゼントできないじゃない! 
\\	あ、ペーターさん、今一カレーのカレーライス大好きですよ。	
\\	あ、ペーターさん、今一カレーのカレーライス大好きですよ。 
\\	週一で通っているみたいですよ。	
\\	週一で通っているみたいですよ。 
\\	今一カレーには、ギフトセットがあるか分かる?	
\\	今一カレーには、ギフトセットがあるか分かる? 
\\	はい、ギフト用レトルトカレーがあります。	
\\	はい、ギフト用レトルトカレーがあります。 
\\	これで、ぺーターさんはオフィスで大好きなカレーが食べられますね。	
\\	これで、ぺーターさんはオフィスで大好きなカレーが食べられますね。 
\\	賛成です!	
\\	賛成です! 
\\	決定!	
\\	決定! 
\\	大勢
\\	大豪邸
\\	用意する
\\	乾杯!
\\	ビール
\\	間に合う
\\	でかい
\\	任せる
\\	妻
\\	家賃
\\	家具
\\	チューハイ
\\	おじゃまします。	
\\	おじゃまします。 
\\	気に入るかどうか分かりませんが、引越し祝いです。どうぞ。	
\\	気に入るかどうか分かりませんが、引越し祝いです。どうぞ。 
\\	ありがとう。	
\\	ありがとう。 
\\	大勢の人がいますね。そして、大豪邸ですね。	
\\	大勢の人がいますね。そして、大豪邸ですね。 
\\	いやいや。家具がほとんどまだないからね。	
\\	いやいや。家具がほとんどまだないからね。 
\\	家賃いくらですか。	
\\	家賃いくらですか。 
\\	いくらか知らない。妻に任せているから。	
\\	いくらか知らない。妻に任せているから。 
\\	そんなに高くないと思うよ。	
\\	そんなに高くないと思うよ。 
\\	このテレビすごいでかいですね!	
\\	このテレビすごいでかいですね! 
\\	まだ、どこに置くか決めていないんだよね。	
\\	まだ、どこに置くか決めていないんだよね。 
\\	あれ?直子は?	
\\	あれ?直子は? 
\\	直子さんは間に合うかどうか分からないって言っていましたよ。	
\\	直子さんは間に合うかどうか分からないって言っていましたよ。 
\\	そうか、じゃあ先に乾杯しようか。	
\\	そうか、じゃあ先に乾杯しようか。 
\\	あ、みんなが何を飲むか分からなかったから、ビール、チューハイ、ワインを用意したんだ。	
\\	あ、みんなが何を飲むか分からなかったから、ビール、チューハイ、ワインを用意したんだ。 
\\	じゃあ、みなさん、グラスを持ってください。	
\\	じゃあ、みなさん、グラスを持ってください。 
\\	今日は来てくれてありがとうございます。では、乾杯!	
\\	今日は来てくれてありがとうございます。では、乾杯! 
\\	乾杯!	
\\	乾杯! 
\\	教科書
\\	開く (ひらく)
\\	傘
\\	覚える
\\	声
\\	仕方
\\	天気予報
\\	写す
\\	通訳
\\	じゃあ、授業を、始めましょう。	
\\	じゃあ、授業を、始めましょう。 
\\	教科書の138ページを開いて下さい。	
\\	教科書の138ページを開いて下さい。 
\\	アレキサンドラさん、まず例文を読んでもらえますか。	
\\	アレキサンドラさん、まず例文を読んでもらえますか。 
\\	はい。「天気・・・」	
\\	"はい。「天気・・・」 
\\	「天気予報」	
\\	"「天気予報」 
\\	天気予報によると、明日は雨だそうです。だから、・・・	
\\	"天気予報によると、明日は雨だそうです。だから、・・・ 
\\	傘を	
\\	"傘を 
\\	だから、傘を持って、出かけて下さい。	
\\	"だから、傘を持って、出かけて下さい。 
\\	はい、ありがとうございます。漢字が少し難しいですね。	
\\	はい、ありがとうございます。漢字が少し難しいですね。 
\\	アレキサンドラさんは、どうやって漢字を勉強していますか。	
\\	アレキサンドラさんは、どうやって漢字を勉強していますか。 
\\	えっと、私は…漢字を見ます。何回も見ます。	
\\	えっと、私は…漢字を見ます。何回も見ます。 
\\	なるほど。アレキサンドラさんは、何度も見て、覚えるそうです。	
\\	なるほど。アレキサンドラさんは、何度も見て、覚えるそうです。 
\\	他の人はどうですか。	
\\	他の人はどうですか。 
\\	僕は、10回ずつ漢字をノートに写して、覚えます。	
\\	僕は、10回ずつ漢字をノートに写して、覚えます。 
\\	あと、中国人の友達は、何度も声に出して読むそうです。	
\\	あと、中国人の友達は、何度も声に出して読むそうです。 
\\	色々な勉強の仕方がありますね。	
\\	色々な勉強の仕方がありますね。 
\\	先生は、韓国語と日本語の通訳をしている友達がいます。	
\\	先生は、韓国語と日本語の通訳をしている友達がいます。 
\\	その人は日本語の本をたくさん読んで、漢字を覚えたそうですよ。	
\\	その人は日本語の本をたくさん読んで、漢字を覚えたそうですよ。 
\\	すし屋
\\	納豆
\\	におい
\\	常連
\\	味
\\	値段
\\	うまい
\\	試す
\\	できる
\\	はまる
\\	高橋先生!お疲れ様です。	
\\	高橋先生!お疲れ様です。 
\\	ああ、金城先生。お疲れ様です。	
\\	ああ、金城先生。お疲れ様です。 
\\	あー、お腹減った。	
\\	あー、お腹減った。 
\\	先生は、お昼休み、外に食べに行きますか?	
\\	先生は、お昼休み、外に食べに行きますか? 
\\	あ、決めてなかったけど。。。	
\\	あ、決めてなかったけど。。。 
\\	そういえば、生徒が、近くにすし屋ができたって言ってたんですけど、試してみませんか。	
\\	そういえば、生徒が、近くにすし屋ができたって言ってたんですけど、試してみませんか。 
\\	いいですね!行きましょう。	
\\	いいですね!行きましょう。 
\\	うーん。やっぱり、寿司はうまい。しかも、この値段は安い。	
\\	うーん。やっぱり、寿司はうまい。しかも、この値段は安い。 
\\	味も値段も良いので、はまりますね。	
\\	味も値段も良いので、はまりますね。 
\\	ところで、最近の生徒は、日本食が好きだと思いませんか。	
\\	ところで、最近の生徒は、日本食が好きだと思いませんか。 
\\	そうですね。生徒のアレキサンドラさんは、ラーメン屋の常連だって言っていましたよ。	
\\	そうですね。生徒のアレキサンドラさんは、ラーメン屋の常連だって言っていましたよ。 
\\	あと、焼き鳥も生徒達に人気があるそうです。	
\\	あと、焼き鳥も生徒達に人気があるそうです。 
\\	そういえば、この前、生徒に嫌いな日本食があるか聞いたんですよ。	
\\	そういえば、この前、生徒に嫌いな日本食があるか聞いたんですよ。 
\\	みんな納豆が嫌いだそうです。	
\\	みんな納豆が嫌いだそうです。 
\\	シェリフさんが「納豆は足のにおいです」って言っていましたよ。(笑)	
\\	シェリフさんが「納豆は足のにおいです」って言っていましたよ。(笑) 
\\	実は、私も納豆が食べられないんですよ。	
\\	実は、私も納豆が食べられないんですよ。 
\\	勤める
\\	節約
\\	お小遣い
\\	巷
\\	実は
\\	もてる
\\	環境
\\	生まれる
\\	早速
\\	流行る
\\	方
\\	続く
\\	会社に勤めている方は、お昼に何を食べていますか。	
\\	会社に勤めている方は、お昼に何を食べていますか。 
\\	最近、巷ではお弁当を持ってくる「弁当男子」が流行っているようです。	
\\	"最近、巷ではお弁当を持ってくる「弁当男子」が流行っているようです。 
\\	三浦アナウンサーが、レポートにいっています。三浦サーン。	
\\	三浦アナウンサーが、レポートにいっています。三浦サーン。 
\\	はいはーい。今、私は赤坂のイノベイティブランゲージのオフィスに来ています。	
\\	はいはーい。今、私は赤坂のイノベイティブランゲージのオフィスに来ています。 
\\	あ、早速、美味しそうなお弁当を食べている男性達を発見しました。	
\\	あ、早速、美味しそうなお弁当を食べている男性達を発見しました。 
\\	すみませーん。お弁当、美味しそうですね。そのお弁当は、自分で作ったんですか?	
\\	すみませーん。お弁当、美味しそうですね。そのお弁当は、自分で作ったんですか? 
\\	はい。・・・子供が生まれて、来月から、・・・お小遣いが下がりそうなので・・・。	
\\	はい。・・・子供が生まれて、来月から、・・・お小遣いが下がりそうなので・・・。 
\\	確かに、節約できそうですね。お隣の方は?	
\\	確かに、節約できそうですね。お隣の方は? 
\\	環境を考えて、お弁当を作っています。	
\\	環境を考えて、お弁当を作っています。 
\\	なるほどー。あと、料理ができる男性は、もてそうですよね。	
\\	なるほどー。あと、料理ができる男性は、もてそうですよね。 
\\	はい、実はそれが1番大きな理由なんです・・・。	
\\	はい、実はそれが1番大きな理由なんです・・・。 
\\	弁当男子には、さまざまなメリットがありそうですね。	
\\	"弁当男子には、さまざまなメリットがありそうですね。 
\\	この流行はこれからも続きそうです。	
\\	この流行はこれからも続きそうです。 
\\	変わる
\\	エコ
\\	半数
\\	水筒
\\	特集する
\\	言葉
\\	不景気
\\	結果
\\	調査
\\	一石二鳥
\\	以前、この番組で「弁当男子」について特集しました。	
\\	"以前、この番組で「弁当男子」について特集しました。 
\\	最近は、「水筒男子」が流行っているそうです。	
\\	"最近は、「水筒男子」が流行っているそうです。 
\\	ゲストのお二人は知っていますか。	
\\	ゲストのお二人は知っていますか。 
\\	会社に自分の飲み物を水筒に入れて、持ってくる人ですよね。	
\\	会社に自分の飲み物を水筒に入れて、持ってくる人ですよね。 
\\	聞いた事ないですな。日本の男性も変わりましたな。	
\\	聞いた事ないですな。日本の男性も変わりましたな。 
\\	アンケートの調査結果を見てみましょう。	
\\	アンケートの調査結果を見てみましょう。 
\\	なんと、20代男性の会社員の約半数が、マイドリンクを持って来ているそうです。	
\\	なんと、20代男性の会社員の約半数が、マイドリンクを持って来ているそうです。 
\\	へー。半分も。	
\\	へー。半分も。 
\\	理由は、「節約のため」が一番多かったそうです。	
\\	"理由は、「節約のため」が一番多かったそうです。 
\\	毎日ペットボトルの飲み物を買わないでいいですから、一ヶ月3000円くらい節約できそうですね。	
\\	毎日ペットボトルの飲み物を買わないでいいですから、一ヶ月3000円くらい節約できそうですね。 
\\	不景気ですからねー。	
\\	不景気ですからねー。 
\\	私は、水筒を持っていくのはできないな。恥ずかしい。	
\\	私は、水筒を持っていくのはできないな。恥ずかしい。 
\\	また、環境のために持ってきている人もいるそうです。	
\\	また、環境のために持ってきている人もいるそうです。 
\\	節約とエコで、一石二鳥ですね。	
\\	節約とエコで、一石二鳥ですね。 
\\	そういえば、「弁当女子」「水筒女子」とは言わないのは、何でなんでしょうね。	
\\	そういえば、「弁当女子」「水筒女子」とは言わないのは、何でなんでしょうね。 
\\	吠える
\\	めちゃくちゃ
\\	寒がり
\\	身につける
\\	抜け毛
\\	防止
\\	芸
\\	一輪車
\\	回す
\\	こら、ポチ!吠えちゃだめ。すみません。	
\\	こら、ポチ!吠えちゃだめ。すみません。 
\\	(あ、この人、めちゃくちゃ綺麗な人だな。)	
\\	(あ、この人、めちゃくちゃ綺麗な人だな。) 
\\	あ、大丈夫ですよ。かわいい洋服を着ていますね。トイプードルですか。	
\\	あ、大丈夫ですよ。かわいい洋服を着ていますね。トイプードルですか。 
\\	そうです。そちらこそ、スカートを履かせて、可愛いですね。名前は何と言うんですか。	
\\	そうです。そちらこそ、スカートを履かせて、可愛いですね。名前は何と言うんですか。 
\\	マリアです。	
\\	マリアです。 
\\	母が「チワワは寒がりだから」って言って、洋服を身につけさせているんです。	
\\	母が「チワワは寒がりだから」って言って、洋服を身につけさせているんです。 
\\	ああ、抜け毛の防止にもなるそうですね。	
\\	ああ、抜け毛の防止にもなるそうですね。 
\\	あ、そうなんですか。	
\\	あ、そうなんですか。 
\\	すごいんですよ。母は、マリアにクラシックを聞かせたり、音楽にあわせて、歌わせたりしているんですよ。	
\\	すごいんですよ。母は、マリアにクラシックを聞かせたり、音楽にあわせて、歌わせたりしているんですよ。 
\\	へー。すごいですね。僕は今、ポチに芸を覚えさせているんですよ。	
\\	へー。すごいですね。僕は今、ポチに芸を覚えさせているんですよ。 
\\	一輪車に乗らせて、皿を回させたいんです。	
\\	一輪車に乗らせて、皿を回させたいんです。 
\\	早速
\\	入浴する
\\	駄洒落
\\	気付く
\\	ペンネーム
\\	今晩
\\	機会
\\	意識する
\\	無理矢理
\\	悩む
\\	おっちょこちょい
\\	苦笑いする
\\	みなさん、こんばんは。DJマキコです。	
\\	みなさん、こんばんは。DJマキコです。 
\\	今晩も、エイサクさんと一緒に、お送りします「ニッポン101」の時間です。	
\\	今晩も、エイサクさんと一緒に、お送りします「ニッポン101」の時間です。 
\\	では、早速ですが、今日の質問を紹介します。	
\\	では、早速ですが、今日の質問を紹介します。 
\\	高校2年生、ペンネーム「おっちょこちょい」さんから来ています。	
\\	高校2年生、ペンネーム「おっちょこちょい」さんから来ています。 
\\	こんばんは、おっちょこちょいくん。	
\\	こんばんは、おっちょこちょいくん。 
\\	実は、今クラスに好きな子がいます。	
\\	"実は、今クラスに好きな子がいます。 
\\	どうしたら、彼女に僕の気持ちを気付かせることができるか、それから、どうしたら、彼女を笑わせることができるのか悩んでいます。アドバイスお願いします!	
\\	"どうしたら、彼女に僕の気持ちを気付かせることができるか、それから、どうしたら、彼女を笑わせることができるのか悩んでいます。アドバイスお願いします! 
\\	最初の答えは、簡単だ。「すきだ」って言えばいい。笑わせるのは難しい。。。	
\\	最初の答えは、簡単だ。「すきだ」って言えばいい。笑わせるのは難しい。。。 
\\	そうだ!駄洒落はどう?例えば「ニューヨークで入浴する」。	
\\	"そうだ!駄洒落はどう?例えば「ニューヨークで入浴する」。 
\\	・・・きっと、彼女を苦笑いさせてしまいますね。	
\\	・・・きっと、彼女を苦笑いさせてしまいますね。 
\\	やっぱり、無理やり笑わせようとするのは、よくないか。	
\\	やっぱり、無理やり笑わせようとするのは、よくないか。 
\\	あまり意識しすぎないで、色々な機会に話しかけてみることはどうですか。	
\\	あまり意識しすぎないで、色々な機会に話しかけてみることはどうですか。 
\\	彼女のことを知らないと、喜ばせることはできませんからね。	
\\	彼女のことを知らないと、喜ばせることはできませんからね。 
\\	教授
\\	意見
\\	反対
\\	苦情
\\	話し合う
\\	新聞記事
\\	実践する
\\	納得する
\\	成功する
\\	環境
\\	保護者
\\	栄養
\\	賛成
\\	学生
\\	事例
\\	では、授業を始めます。	
\\	では、授業を始めます。 
\\	小学校の一部が「子どもが作る弁当の日」を実践しています。	
\\	"小学校の一部が「子どもが作る弁当の日」を実践しています。 
\\	今日は、この弁当の日の新聞記事を読んで、3人のグループで話し合ってください。	
\\	今日は、この弁当の日の新聞記事を読んで、3人のグループで話し合ってください。 
\\	「弁当の日」についてどう思う?	
\\	「弁当の日」についてどう思う? 
\\	私は、賛成。子どもに自分でお弁当を年に5回作らせるんでしょ。	
\\	私は、賛成。子どもに自分でお弁当を年に5回作らせるんでしょ。 
\\	そうしたら、子どもに栄養バランスの大事さを考えさせることができると思う。	
\\	そうしたら、子どもに栄養バランスの大事さを考えさせることができると思う。 
\\	そうだね。でも、この記事によると、保護者から、苦情や不満が出ていて、反対意見もあるみたいだね。	
\\	そうだね。でも、この記事によると、保護者から、苦情や不満が出ていて、反対意見もあるみたいだね。 
\\	留学生
\\	家庭料理
\\	体験する
\\	荷物
\\	銭湯
\\	楽しむ
\\	お世話になる
\\	泊まる
\\	普通
\\	なるべく
\\	「なるべく、留学生には、日本語を話させてあげてください。留学生には日本の普通の生活を体験させてあげてください。」って先生は言ってたけど、	
\\	"「なるべく、留学生には、日本語を話させてあげてください。留学生には日本の普通の生活を体験させてあげてください。」って先生は言ってたけど、 
\\	本当に特別なことをしなくても大丈夫かしら。	
\\	本当に特別なことをしなくても大丈夫かしら。 
\\	去年、太一はジョー君の家に泊まらせてもらったり、おいしいタイの家庭料理を食べさせてもらって、本当にお世話になったみたいだから、俺達もジョー君を楽しませてあげたいよな。	
\\	去年、太一はジョー君の家に泊まらせてもらったり、おいしいタイの家庭料理を食べさせてもらって、本当にお世話になったみたいだから、俺達もジョー君を楽しませてあげたいよな。 
\\	そうね。	
\\	そうね。 
\\	あ、明日は、ジョー君に銭湯を体験させてあげようかな。	
\\	あ、明日は、ジョー君に銭湯を体験させてあげようかな。 
\\	でも、タイには銭湯がないから、嫌がるかもよ。	
\\	でも、タイには銭湯がないから、嫌がるかもよ。 
\\	そうかー。とにかく、リラックスさせてあげないとね。	
\\	そうかー。とにかく、リラックスさせてあげないとね。 
\\	遅いわね。何かあったのかなぁ・・・。	
\\	遅いわね。何かあったのかなぁ・・・。 
\\	あ、来た来た。いらっしゃーい。ジョー君!	
\\	あ、来た来た。いらっしゃーい。ジョー君! 
\\	あ、はじめまして。ジョーと申します。遅くなってすみません・・・	
\\	あ、はじめまして。ジョーと申します。遅くなってすみません・・・ 
\\	あ、あのー。すみませんが、電話使わせてくれますか?	
\\	あ、あのー。すみませんが、電話使わせてくれますか? 
\\	タクシーの中に、荷物全部忘れてきたんです!!	
\\	タクシーの中に、荷物全部忘れてきたんです!! 
\\	えええ?	
\\	えええ? 
\\	相談
\\	芸術的
\\	苦労する
\\	バカ
\\	女友達
\\	型
\\	血液型
\\	サークル
\\	確かに
\\	自由
\\	個性的
\\	意見
\\	掲示板
\\	話題
\\	検索する
\\	束縛する
\\	愛、どうしたの?相談があるって。	
\\	愛、どうしたの?相談があるって。 
\\	実は、サークルに好きな人ができて。	
\\	実は、サークルに好きな人ができて。 
\\	え?本当?どんな感じの人?	
\\	え?本当?どんな感じの人? 
\\	それが、
\\	型の人なの・・・。	
\\	それが、
\\	型の人なの・・・。 
\\	ほら、
\\	型の人って変わった人が多いでしょ。それに、彼、女友達が多いみたいだし・・・	
\\	ほら、
\\	型の人って変わった人が多いでしょ。それに、彼、女友達が多いみたいだし・・・ 
\\	どうしたら、私のこと好きになってもらえるかな・・・。	
\\	どうしたら、私のこと好きになってもらえるかな・・・。 
\\	ふーん。彼、
\\	型なんだ。	
\\	ふーん。彼、
\\	型なんだ。 
\\	ちなみに、わたしも
\\	型なんだよ。知ってた?	
\\	"ちなみに、わたしも
\\	型なんだよ。知ってた? 
\\	ええ?じゃあ、さくら、同じ血液型だったら、B型の人の気持ちも分かるでしょ?	
\\	ええ?じゃあ、さくら、同じ血液型だったら、B型の人の気持ちも分かるでしょ? 
\\	バカなこと聞かないでよ。血液型で全部わかったら、苦労しないよ。(笑)	
\\	バカなこと聞かないでよ。血液型で全部わかったら、苦労しないよ。(笑) 
\\	でも、ネットで検索したら面白いんじゃない。	
\\	でも、ネットで検索したら面白いんじゃない。 
\\	授業
\\	変わっている
\\	別に
\\	間違える
\\	信じる
\\	気分屋
\\	説明書
\\	卒業する
\\	占い
\\	ねえねえ、愛。最近彼とどうなった?	
\\	ねえねえ、愛。最近彼とどうなった? 
\\	彼の気持ちがわからない。彼、気分屋だし・・・。	
\\	彼の気持ちがわからない。彼、気分屋だし・・・。 
\\	はい、これプレゼント。	
\\	はい、これプレゼント。 
\\	何?この本。
\\	型自分の説明書』。おおー。	
\\	"何?この本。
\\	型自分の説明書』。おおー。 
\\	これ読むと、彼の気持ちがわかるかもよ。	
\\	これ読むと、彼の気持ちがわかるかもよ。 
\\	あ、愛ちゃん。	
\\	あ、愛ちゃん。 
\\	あ、つ、つ、翼君!ひさしぶり。ははは。	
\\	あ、つ、つ、翼君!ひさしぶり。ははは。 
\\	どうしたの?い、今から授業?	
\\	どうしたの?い、今から授業? 
\\	そうそう。たまには、授業に出ないと、卒業できないからね。	
\\	そうそう。たまには、授業に出ないと、卒業できないからね。 
\\	あ、その本、今すごい人気だよねー。あれ?愛ちゃんはB型?	
\\	あ、その本、今すごい人気だよねー。あれ?愛ちゃんはB型? 
\\	う、ううん。わたしは
\\	型。	
\\	う、ううん。わたしは
\\	型。 
\\	ちょっと見せて。	
\\	ちょっと見せて。 
\\	あ、うん。もちろん!!	
\\	あ、うん。もちろん!! 
\\	(小さい声で)ねぇねぇ、愛、この人?	
\\	(小さい声で)ねぇねぇ、愛、この人? 
\\	(小さい声で)そう!	
\\	(小さい声で)そう! 
\\	へー。
\\	型は変わっているって言われるとうれしい。・・・へー。	
\\	へー。
\\	型は変わっているって言われるとうれしい。・・・へー。 
\\	や、やっぱりさ、翼君も、個性的だね・・・って言われるとうれしいの?	
\\	や、やっぱりさ、翼君も、個性的だね・・・って言われるとうれしいの? 
\\	え?俺?別に。ていうか、俺、
\\	型だし。	
\\	え?俺?別に。ていうか、俺、
\\	型だし。 
\\	ええ?
\\	型じゃないの?	
\\	ええ?
\\	型じゃないの? 
\\	ああ、よく間違えられるけど、
\\	型。	
\\	ああ、よく間違えられるけど、
\\	型。 
\\	しかも、俺、占い絶対信じないし。	
\\	しかも、俺、占い絶対信じないし。 
\\	発売
\\	観客
\\	近所
\\	送料
\\	消費税込み
\\	(辞書を)引く
\\	画面
\\	役に立つ
\\	場面
\\	日常生活
\\	学習
\\	電子辞書
\\	注文
\\	無料
\\	値段
\\	光る
\\	機能
\\	旅行
\\	驚く
\\	はい、今日ご紹介するのは、こちら!シャポーから発売されている電子辞書です!	
\\	はい、今日ご紹介するのは、こちら!シャポーから発売されている電子辞書です! 
\\	電子辞書は「学習」「旅行」「日常生活」「ビジネス」の場面でとても役に立ちます。	
\\	電子辞書は「学習」「旅行」「日常生活」「ビジネス」の場面でとても役に立ちます。 
\\	ご家族、お友達、近所の人、まだ持っていない人が周りにいたら、是非プレゼントしてあげてください。	
\\	ご家族、お友達、近所の人、まだ持っていない人が周りにいたら、是非プレゼントしてあげてください。 
\\	では、機能の説明をします。まず、大きな画面。	
\\	では、機能の説明をします。まず、大きな画面。 
\\	そして、このボタンを押すと、バックライトが光ります。	
\\	そして、このボタンを押すと、バックライトが光ります。 
\\	おー!!	
\\	おー!! 
\\	漢字の読み方がわからなかったら、ペンで、ここに、漢字を書きます。	
\\	漢字の読み方がわからなかったら、ペンで、ここに、漢字を書きます。 
\\	漢字を書くと、辞書が引けます。	
\\	漢字を書くと、辞書が引けます。 
\\	はい、この素晴らしい内容で、お値段はなんと、消費税込み24800円。	
\\	はい、この素晴らしい内容で、お値段はなんと、消費税込み24800円。 
\\	おー!!	
\\	おー!! 
\\	まだ驚いたら、だめですよ。送料無料。	
\\	まだ驚いたら、だめですよ。送料無料。 
\\	そして、今日中に電話で注文すると、5000円引き。	
\\	そして、今日中に電話で注文すると、5000円引き。 
\\	ええー!!	
\\	ええー!! 
\\	ということで、お値段は、19800円。19800円です。	
\\	ということで、お値段は、19800円。19800円です。 
\\	さらにさらに、2つ買うと、電子辞書がもう一つもらえます。どうですか!	
\\	さらにさらに、2つ買うと、電子辞書がもう一つもらえます。どうですか! 
\\	手相占い
\\	間違いなく
\\	悩み
\\	理想
\\	独身
\\	一生
\\	年収
\\	成功する
\\	気になる
\\	結婚
\\	出会う
\\	お嬢さん、悩みがあるんじゃないですか。	
\\	お嬢さん、悩みがあるんじゃないですか。 
\\	手相占いをしませんか。	
\\	手相占いをしませんか。 
\\	ええ?悩みがあるって、どうしてわかったの?	
\\	ええ?悩みがあるって、どうしてわかったの? 
\\	顔を見ればわかりますよ。そこに、すわりなさい。	
\\	顔を見ればわかりますよ。そこに、すわりなさい。 
\\	悩みは結婚とお仕事ですか?	
\\	悩みは結婚とお仕事ですか? 
\\	ええ?なんで、結婚と仕事で悩んでいるって、わかったんですか?	
\\	ええ?なんで、結婚と仕事で悩んでいるって、わかったんですか? 
\\	女性だと、一番気になることですよね。じゃあ、手を見せてください。	
\\	女性だと、一番気になることですよね。じゃあ、手を見せてください。 
\\	(ふむふむ)あなた、仕事で成功できたら、結婚はしないで良いと思っていますよね。	
\\	(ふむふむ)あなた、仕事で成功できたら、結婚はしないで良いと思っていますよね。 
\\	え?いや、そんな・・・。	
\\	え?いや、そんな・・・。 
\\	自分より、年収が高くて、頭がよくて、仕事ができる男性がいれば、結婚したいですよ。	
\\	自分より、年収が高くて、頭がよくて、仕事ができる男性がいれば、結婚したいですよ。 
\\	そんなことを言っていると、一生独身ですよ。	
\\	そんなことを言っていると、一生独身ですよ。 
\\	あ、いや・・・待ってください・・・あなた、結婚できますよ。	
\\	あ、いや・・・待ってください・・・あなた、結婚できますよ。 
\\	ええ?本当ですか?	
\\	ええ?本当ですか? 
\\	間違いなく理想の相手と出会います。	
\\	間違いなく理想の相手と出会います。 
\\	よかったー。いつですか。	
\\	よかったー。いつですか。 
\\	45年後ですね。	
\\	45年後ですね。 
\\	飲酒運転
\\	はく
\\	強化する
\\	交通事故
\\	取り締まり
\\	息
\\	検問
\\	鈍い
\\	判断
\\	警察
\\	捕まる
\\	罰金
\\	一生を棒にふる
\\	ご注文を伺います。	
\\	ご注文を伺います。 
\\	とりあえずビールで。	
\\	とりあえずビールで。 
\\	おれも。	
\\	おれも。 
\\	私は、キウイサワー。	
\\	私は、キウイサワー。 
\\	かしこまりました。	
\\	かしこまりました。 
\\	あれ、康平。お前、今日車で来たんじゃなかったっけ?	
\\	あれ、康平。お前、今日車で来たんじゃなかったっけ? 
\\	ああ、そうだよ。	
\\	ああ、そうだよ。 
\\	飲酒運転はだめだぞ。	
\\	飲酒運転はだめだぞ。 
\\	ちょっと飲むだけなら、大丈夫だよ。	
\\	ちょっと飲むだけなら、大丈夫だよ。 
\\	最近、取り締まり厳しいし、罰金高いんだから。	
\\	最近、取り締まり厳しいし、罰金高いんだから。 
\\	おれが通る道なら、警察いないはずだから、捕まらないよ。	
\\	おれが通る道なら、警察いないはずだから、捕まらないよ。 
\\	捕まらないなら、良いってことじゃないんだよ。	
\\	"捕まらないなら、良いってことじゃないんだよ。 
\\	お酒が入っていると、判断も鈍くなるんだぞ。	
\\	お酒が入っていると、判断も鈍くなるんだぞ。 
\\	もし、交通事故を起こしたら、どうするんだよ。一生を棒に振るぞ。	
\\	もし、交通事故を起こしたら、どうするんだよ。一生を棒に振るぞ。 
\\	分かったよ。	
\\	分かったよ。 
\\	すみませーん、ウーロン茶お願いします。	
\\	すみませーん、ウーロン茶お願いします。 
\\	じゃあ、お疲れー。またねー。	
\\	じゃあ、お疲れー。またねー。 
\\	あれ、検問か。	
\\	あれ、検問か。 
\\	はい、止まってください。	
\\	はい、止まってください。 
\\	『飲んだら乗るな。乗るなら飲むな。』。飲酒運転の取締りを強化しています。	
\\	"『飲んだら乗るな。乗るなら飲むな。』。飲酒運転の取締りを強化しています。 
\\	はい、息をはいてください。	
\\	はい、息をはいてください。 
\\	はい、大丈夫ですねー。どうぞ。	
\\	はい、大丈夫ですねー。どうぞ。 
\\	ああ、良かったー。	
\\	ああ、良かったー。 
\\	『飲んだら乗るな。乗るなら飲むな。』ね。	
\\	"『飲んだら乗るな。乗るなら飲むな。』ね。 
\\	航空券
\\	代金
\\	世界遺産
\\	巡る
\\	お手ごろ
\\	激安
\\	申し込む
\\	お得
\\	値段
\\	空く
\\	予約
\\	既に
\\	出発日
\\	参加する
\\	価格
\\	燃油サーチャージ
\\	期間限定
\\	空港施設使用料
\\	いっらっしゃいませ。	
\\	いっらっしゃいませ。 
\\	すみません、イタリアローマ旅行のプランを探しているんですけど・・・	
\\	すみません、イタリアローマ旅行のプランを探しているんですけど・・・ 
\\	こちらは、どうでしょう。	
\\	こちらは、どうでしょう。 
\\	期間限定の激安プランです。	
\\	期間限定の激安プランです。 
\\	4月と5月限定かー。それに、航空券とホテルだけのセットなんですね。	
\\	4月と5月限定かー。それに、航空券とホテルだけのセットなんですね。 
\\	本当にお手ごろな値段ですね。	
\\	本当にお手ごろな値段ですね。 
\\	世界遺産を巡るオプショナルツアーをつけても、いいと思いますよ。	
\\	世界遺産を巡るオプショナルツアーをつけても、いいと思いますよ。 
\\	あー、オプションはいらないです。	
\\	あー、オプションはいらないです。 
\\	そうですか。ま、オプショナルツアーに参加しなくても、十分楽しめますからね。	
\\	そうですか。ま、オプショナルツアーに参加しなくても、十分楽しめますからね。 
\\	出発日は決まってますか?	
\\	出発日は決まってますか? 
\\	ええっと、5月2日がいいんですが。	
\\	ええっと、5月2日がいいんですが。 
\\	申し訳ありません、その日は既に予約がいっぱいでして・・・	
\\	申し訳ありません、その日は既に予約がいっぱいでして・・・ 
\\	じゃあ、4月30日の出発は空いていますか。	
\\	じゃあ、4月30日の出発は空いていますか。 
\\	まだ、空きがあります。	
\\	まだ、空きがあります。 
\\	代金は一緒ですか。	
\\	代金は一緒ですか。 
\\	はい、いつ出発しても、同じ値段です。	
\\	はい、いつ出発しても、同じ値段です。 
\\	その金額に燃油サーチャージは含まれていますか?	
\\	その金額に燃油サーチャージは含まれていますか? 
\\	はい。燃油サーチャージと空港施設使用料を含めて、この価格です。	
\\	はい。燃油サーチャージと空港施設使用料を含めて、この価格です。 
\\	本当にお得なんですね。	
\\	本当にお得なんですね。 
\\	じゃ、このプランで申し込みます。	
\\	じゃ、このプランで申し込みます。 
\\	ありがとうございます。	
\\	ありがとうございます。 
\\	量
\\	それだけ
\\	半熟卵
\\	冷める
\\	ご当地
\\	辛い
\\	本場
\\	満席
\\	平日
\\	載る
\\	王様
\\	「安いのにおいしい、
\\	級グルメ」を今日も、皆さんに紹介していきたいと思います。	
\\	「安いのにおいしい、
\\	級グルメ」を今日も、皆さんに紹介していきたいと思います。 
\\	今日紹介する、
\\	級グルメは「焼きそば」。そう!秋田県、横手の ご当地グルメです。	
\\	今日紹介する、
\\	級グルメは「焼きそば」。そう!秋田県、横手の ご当地グルメです。 
\\	では、レポートを見てみましょう。	
\\	では、レポートを見てみましょう。 
\\	わたしは今、「じゅうじゅう、ごくり、焼きそばの街」横手 にきていまーす。	
\\	わたしは今、「じゅうじゅう、ごくり、焼きそばの街」横手 にきていまーす。 
\\	では、早速お店に入ってみます。	
\\	では、早速お店に入ってみます。 
\\	うわー。平日なのに、満席ですね。	
\\	うわー。平日なのに、満席ですね。 
\\	あのぉ、焼きそばください。	
\\	あのぉ、焼きそばください。 
\\	あいよ。	
\\	あいよ。 
\\	はい、これが、本場の横手やきそば。うまいよ。	
\\	はい、これが、本場の横手やきそば。うまいよ。 
\\	うーん。ソースがいいにおい!焼きそばの上に、目玉焼きが載っているんですね。	
\\	うーん。ソースがいいにおい!焼きそばの上に、目玉焼きが載っているんですね。 
\\	そう。それが、横手焼きそばの特徴。	
\\	そう。それが、横手焼きそばの特徴。 
\\	冷める前に食べてよ。	
\\	冷める前に食べてよ。 
\\	はい。いただきまーす。うーん。辛いのに、優しい味。	
\\	はい。いただきまーす。うーん。辛いのに、優しい味。 
\\	あ、これは、半熟卵のおかげですね。	
\\	あ、これは、半熟卵のおかげですね。 
\\	そう。それだけの量があるのに、たった500円だよ。	
\\	そう。それだけの量があるのに、たった500円だよ。 
\\	本当ですね。	
\\	本当ですね。 
\\	こんなにおいしくて、すごいボリュームがあるのに、この値段!	
\\	こんなにおいしくて、すごいボリュームがあるのに、この値段! 
\\	まさに、
\\	級グルメの王様ですね!	
\\	まさに、
\\	級グルメの王様ですね! 
\\	緊張する
\\	気楽
\\	連休
\\	不眠症
\\	くま
\\	やる気
\\	だるい
\\	一気に
\\	環境
\\	あせる
\\	おっす!ん、目の下にくまがあるじゃん?どうしたんだよ?	
\\	おっす!ん、目の下にくまがあるじゃん?どうしたんだよ? 
\\	え?ほんと?フアー。(あくび)	
\\	え?ほんと?フアー。(あくび) 
\\	実は、最近眠れないんだ。	
\\	実は、最近眠れないんだ。 
\\	全然、眠れないのか?	
\\	全然、眠れないのか? 
\\	ああ、不眠症ってやつかな。フアー。	
\\	ああ、不眠症ってやつかな。フアー。 
\\	ゴールデンウィークが終わってから、全然やる気も出ないんだよね。	
\\	ゴールデンウィークが終わってから、全然やる気も出ないんだよね。 
\\	あ、おれも!朝全然起きられないし、一日中だるいんだよな。	
\\	あ、おれも!朝全然起きられないし、一日中だるいんだよな。 
\\	あ、馬場先生だ。	
\\	あ、馬場先生だ。 
\\	こんにちは。	
\\	こんにちは。 
\\	こんにちは。どうした?二人とも、眠そうな顔して。	
\\	こんにちは。どうした?二人とも、眠そうな顔して。 
\\	今、「最近なんだかだるくて・・・」という話をしていたところだったんですよ。	
\\	今、「最近なんだかだるくて・・・」という話をしていたところだったんですよ。 
\\	あー、それは五月病じゃないですか。	
\\	あー、それは五月病じゃないですか。 
\\	五月病?	
\\	五月病? 
\\	そう。大学一年の四月は、新しいことばかりで、毎日緊張するでしょう。	
\\	そう。大学一年の四月は、新しいことばかりで、毎日緊張するでしょう。 
\\	で、その疲れが、5月の連休の後に、一気に出てきたんだよ。	
\\	で、その疲れが、5月の連休の後に、一気に出てきたんだよ。 
\\	どうしたら・・・いいんですか。	
\\	どうしたら・・・いいんですか。 
\\	新しい環境に慣れるのは、難しいんだから、あせる必要はないよ。	
\\	新しい環境に慣れるのは、難しいんだから、あせる必要はないよ。 
\\	ま、気楽にね。	
\\	ま、気楽にね。 
\\	はい!ありがとうございます!	
\\	はい!ありがとうございます! 
\\	あ、そうそう。僕、五月病についての本を書いているから、本屋で買ってね。	
\\	あ、そうそう。僕、五月病についての本を書いているから、本屋で買ってね。 
\\	合コン
\\	想像する
\\	弾む
\\	共通
\\	話題になる
\\	講師
\\	料理教室
\\	だらけ
\\	続く
\\	関係
\\	魅力
\\	初対面
\\	参加者
\\	私は、今、
\\	料理教室にきています。	
\\	私は、今、
\\	料理教室にきています。 
\\	今日は、最近話題になっている、「料理合コン」についてレポートします。	
\\	今日は、最近話題になっている、「料理合コン」についてレポートします。 
\\	では、
\\	料理教室の講師森光先生にお話しを聞いてみましょう。	
\\	では、
\\	料理教室の講師森光先生にお話しを聞いてみましょう。 
\\	先生よろしくお願いします。	
\\	先生よろしくお願いします。 
\\	よろしくお願いします。	
\\	よろしくお願いします。 
\\	先生、「料理合コン」というのは・・・	
\\	"先生、「料理合コン」というのは・・・ 
\\	料理教室での合コン・・・	
\\	料理教室での合コン・・・ 
\\	つまり、初対面の男女のグループが料理を一緒にして、知り合うことができるパーティーです。	
\\	つまり、初対面の男女のグループが料理を一緒にして、知り合うことができるパーティーです。 
\\	料理合コンの魅力は、何でしょうか。	
\\	料理合コンの魅力は、何でしょうか。 
\\	まず、料理の勉強になります。	
\\	まず、料理の勉強になります。 
\\	そして、料理合コンで知り合ったカップルは長く続くと言われています。	
\\	そして、料理合コンで知り合ったカップルは長く続くと言われています。 
\\	そうなんですかー。どうして関係が続きやすいんですか。	
\\	そうなんですかー。どうして関係が続きやすいんですか。 
\\	一緒に料理を作るので、相手の性格がわかりやすいんです。	
\\	一緒に料理を作るので、相手の性格がわかりやすいんです。 
\\	共通の話題があって、話も弾みますし。	
\\	共通の話題があって、話も弾みますし。 
\\	なるほど。	
\\	なるほど。 
\\	参加者の中には、「結婚後の生活が想像しやすい」って言う人もいるんですよ。	
\\	参加者の中には、「結婚後の生活が想像しやすい」って言う人もいるんですよ。 
\\	料理が習えて、最高のパートナーも見つけられるんですね。	
\\	料理が習えて、最高のパートナーも見つけられるんですね。 
\\	まさに、良いことだらけ!いいなー。私も参加したーい。	
\\	まさに、良いことだらけ!いいなー。私も参加したーい。 
\\	者
\\	景品
\\	購読
\\	宅配便
\\	学費
\\	引越し
\\	差し上げる
\\	商品券
\\	契約
\\	挟む
\\	どかす
\\	宅配便でーす。	
\\	宅配便でーす。 
\\	はーい。	
\\	はーい。 
\\	すみませーん、わたくし、赤坂新聞の者なんですが・・・	
\\	すみませーん、わたくし、赤坂新聞の者なんですが・・・ 
\\	え・・(心の声-宅配便じゃないじゃん!どうしよう。うち新聞いらないし。)	
\\	え・・(心の声-宅配便じゃないじゃん!どうしよう。うち新聞いらないし。) 
\\	新聞を購読しませんか。	
\\	新聞を購読しませんか。 
\\	あー、申し訳ないんですけど、もう間に合っていますので、大丈夫です。	
\\	あー、申し訳ないんですけど、もう間に合っていますので、大丈夫です。 
\\	(心の声-あー!あー!この人、ドアに足を挟んでる!)	
\\	(心の声-あー!あー!この人、ドアに足を挟んでる!) 
\\	いま、契約すると、景品が付きますよ。	
\\	いま、契約すると、景品が付きますよ。 
\\	洗剤5箱、商品券3000円分、そして、ディズニーランドのパスポートを差し上げます。	
\\	洗剤5箱、商品券3000円分、そして、ディズニーランドのパスポートを差し上げます。 
\\	私、来年の4月には引っ越しするんですけど・・・	
\\	私、来年の4月には引っ越しするんですけど・・・ 
\\	では、引っ越すまで、新聞を購読しませんか?	
\\	では、引っ越すまで、新聞を購読しませんか? 
\\	すみません、結構です!!!	
\\	すみません、結構です!!! 
\\	ドアを閉めますので、足どかしてください!!!!	
\\	ドアを閉めますので、足どかしてください!!!! 
\\	あれ?もしかして・・・赤坂小学校の出身?	
\\	あれ?もしかして・・・赤坂小学校の出身? 
\\	はい?あ、健ちゃん?!?	
\\	はい?あ、健ちゃん?!? 
\\	久しぶり!	
\\	久しぶり! 
\\	健ちゃん。。。新聞のセールスしているんだ。	
\\	健ちゃん。。。新聞のセールスしているんだ。 
\\	あ・・・うん。学費のためにね。	
\\	あ・・・うん。学費のためにね。 
\\	わかった!じゃ、契約する。	
\\	わかった!じゃ、契約する。 
\\	知識
\\	社交性
\\	損をする
\\	述べる
\\	結論
\\	そもそも
\\	理系
\\	文系
\\	しっかりと
\\	とにかく
\\	だます
\\	くだらない
\\	作家
\\	みなさんこんばんは。DJマキコです。	
\\	みなさんこんばんは。DJマキコです。 
\\	今日のゲストは ベストセラー作家の、夏目五郎さんです。よろしくお願いします。	
\\	今日のゲストは ベストセラー作家の、夏目五郎さんです。よろしくお願いします。 
\\	よろしく。	
\\	よろしく。 
\\	では、最初の質問を読みますね。	
\\	では、最初の質問を読みますね。 
\\	マキコさん、夏目先生、こんにちは。	
\\	マキコさん、夏目先生、こんにちは。 
\\	はい。こんにちは。	
\\	はい。こんにちは。 
\\	私は、高校一年生です。	
\\	"私は、高校一年生です。 
\\	「数学」や「物理」などの理系の科目は嫌いではないのですが、「国語」が一番好きです。	
\\	「数学」や「物理」などの理系の科目は嫌いではないのですが、「国語」が一番好きです。 
\\	将来は、小説家になりたいと思っているので、理系の科目は勉強しなくても良いですか。	
\\	"将来は、小説家になりたいと思っているので、理系の科目は勉強しなくても良いですか。 
\\	結論を述べます。「絶対に勉強しなさい!」	
\\	"結論を述べます。「絶対に勉強しなさい!」 
\\	そもそも、文系と理系に分ける考え方がおかしいんだ。	
\\	そもそも、文系と理系に分ける考え方がおかしいんだ。 
\\	どうして、おかしいんですか。	
\\	どうして、おかしいんですか。 
\\	人間、知識を持っていて、損をすることはないんだよ。	
\\	人間、知識を持っていて、損をすることはないんだよ。 
\\	例えば、わたしは小説家だけど、
\\	小説を書くときには理系の知識が必要になる。	
\\	例えば、わたしは小説家だけど、
\\	小説を書くときには理系の知識が必要になる。 
\\	なるほどー。	
\\	なるほどー。 
\\	理系の人は社交性がないというイメージもあるみたいですけど。	
\\	理系の人は社交性がないというイメージもあるみたいですけど。 
\\	くだらない。どうして、そんなステレオタイプにだまされるんだ。	
\\	くだらない。どうして、そんなステレオタイプにだまされるんだ。 
\\	とにかく、しっかりと今は、何でも勉強するように!	
\\	とにかく、しっかりと今は、何でも勉強するように! 
\\	特集する
\\	足りる
\\	日本各地
\\	王国
\\	駅前
\\	胃袋
\\	鉄
\\	無理(な)
\\	種類
\\	別腹
\\	ねえ、ユウキ、この前テレビで「デザート王国」を特集しているのを見たんだけど、行かない?	
\\	"ねえ、ユウキ、この前テレビで「デザート王国」を特集しているのを見たんだけど、行かない? 
\\	デザート王国?何それ?	
\\	デザート王国?何それ? 
\\	「フードテーマパーク」って言うのかな。聞いたことない?	
\\	「フードテーマパーク」って言うのかな。聞いたことない? 
\\	日本各地のおいしいデザートを、食べることができるんだよ。	
\\	日本各地のおいしいデザートを、食べることができるんだよ。 
\\	へー。ホントに、沙織は食べるのが好きだよな。ま、いいよ。週末、行ってみようか。	
\\	へー。ホントに、沙織は食べるのが好きだよな。ま、いいよ。週末、行ってみようか。 
\\	えっと、このチケットで、デザートを20種類まで食べることができるのね。	
\\	えっと、このチケットで、デザートを20種類まで食べることができるのね。 
\\	一人で20種類も食べるのは無理じゃない?	
\\	一人で20種類も食べるのは無理じゃない? 
\\	10種類用のチケットを買ったほうが良くない?	
\\	10種類用のチケットを買ったほうが良くない? 
\\	大丈夫大丈夫。私は鉄の胃袋の持ち主だから。	
\\	大丈夫大丈夫。私は鉄の胃袋の持ち主だから。 
\\	あ、プリンだ!勇気、食べよう!	
\\	あ、プリンだ!勇気、食べよう! 
\\	あーよく食べた。楽しかったねー。	
\\	あーよく食べた。楽しかったねー。 
\\	でも、20個じゃ、全然足りなかったね。	
\\	でも、20個じゃ、全然足りなかったね。 
\\	(ウップ)え?あ。。。そう?俺、腹が(ゴロゴロゴロ)・・・。うっ。	
\\	(ウップ)え?あ。。。そう?俺、腹が(ゴロゴロゴロ)・・・。うっ。 
\\	ね、勇気。駅前においしいクレープ屋さんがあるの。クレープ食ベて帰ろう!	
\\	ね、勇気。駅前においしいクレープ屋さんがあるの。クレープ食ベて帰ろう! 
\\	腹いっぱいだよ。これ以上食べるのは無理。(ウップ)	
\\	腹いっぱいだよ。これ以上食べるのは無理。(ウップ) 
\\	デザートは別腹!	
\\	デザートは別腹! 
\\	しょうがない
\\	借りる
\\	貸し出し中
\\	自宅
\\	作品
\\	月額
\\	登録する
\\	(はぁ・・・)ただいま。	
\\	(はぁ・・・)ただいま。 
\\	お帰り。どうしたの?元気ないけど。	
\\	お帰り。どうしたの?元気ないけど。 
\\	が貸し出し中だったのよ。韓国ドラマ。	
\\	が貸し出し中だったのよ。韓国ドラマ。 
\\	次のストーリーが気になって気になって、しょうがないのよ。	
\\	次のストーリーが気になって気になって、しょうがないのよ。 
\\	あの店は、借りられているDVDばかりで、本当に困るわ。	
\\	あの店は、借りられているDVDばかりで、本当に困るわ。 
\\	だったら、ネットでDVDを借りれば良いじゃない?	
\\	だったら、ネットでDVDを借りれば良いじゃない? 
\\	お店には置いてない作品もあるって、友達が言ってたよ。	
\\	お店には置いてない作品もあるって、友達が言ってたよ。 
\\	インターネットで借りられるの?	
\\	インターネットで借りられるの? 
\\	そうそう。自分が観たいDVDを予約すると、自宅に届くんだって。	
\\	そうそう。自分が観たいDVDを予約すると、自宅に届くんだって。 
\\	本当に!返す時は?	
\\	本当に!返す時は? 
\\	見終わったDVDをポストに入れるだけ。	
\\	見終わったDVDをポストに入れるだけ。 
\\	へー便利ね!でも高いんでしょ?	
\\	へー便利ね!でも高いんでしょ? 
\\	そんなに高くないと思うよ。ちょっと、調べるね。	
\\	そんなに高くないと思うよ。ちょっと、調べるね。 
\\	あ、月額2000円で、八枚まで借りることができるって。	
\\	あ、月額2000円で、八枚まで借りることができるって。 
\\	お母さんがいつも借りているお店より安いじゃない。	
\\	お母さんがいつも借りているお店より安いじゃない。 
\\	最初の一か月は、ただで、借りることができるって書いてあるよ。	
\\	最初の一か月は、ただで、借りることができるって書いてあるよ。 
\\	すごい。今すぐ、登録して。	
\\	すごい。今すぐ、登録して。 
\\	いいよ。	
\\	いいよ。 
\\	じゃ、私のアメリカのドラマも一緒に借りてね。	
\\	じゃ、私のアメリカのドラマも一緒に借りてね。 
\\	ダメよ。自分で借りなさい。	
\\	ダメよ。自分で借りなさい。 
\\	出張
\\	噂
\\	意見
\\	基本料金
\\	仕組み
\\	回転寿司
\\	回る
\\	加算する
\\	機械
\\	こんにちは、みなさん。	
\\	こんにちは、みなさん。 
\\	会社で回転寿司を食べられるという噂を聞いたのですが、本当なのでしょうか。	
\\	会社で回転寿司を食べられるという噂を聞いたのですが、本当なのでしょうか。 
\\	今日は、ここの会社に回転寿司が来ていると聞いたのですが。	
\\	今日は、ここの会社に回転寿司が来ていると聞いたのですが。 
\\	(コンコン)失礼します。	
\\	(コンコン)失礼します。 
\\	こんにちは。	
\\	こんにちは。 
\\	すごい!!オフィスの中で、お寿司が回っています!	
\\	すごい!!オフィスの中で、お寿司が回っています! 
\\	らっしゃい!ようこそ、出張回転寿司へ!	
\\	らっしゃい!ようこそ、出張回転寿司へ! 
\\	出張回転寿司?この機械は、持ってきたんですか。	
\\	出張回転寿司?この機械は、持ってきたんですか。 
\\	そうだよ。基本料金は2万。	
\\	そうだよ。基本料金は2万。 
\\	それに食べた分の料金を加算するシステムだよ。	
\\	それに食べた分の料金を加算するシステムだよ。 
\\	そして、なんと、お寿司1皿は、普通の回転寿司とは、変わらない値段!	
\\	そして、なんと、お寿司1皿は、普通の回転寿司とは、変わらない値段! 
\\	なるほどー。	
\\	なるほどー。 
\\	機械をセットアップする代金と、板前さんに来てもらう代金が2万円。	
\\	機械をセットアップする代金と、板前さんに来てもらう代金が2万円。 
\\	あとは、お寿司を食べた料金を払えば良いんですね。	
\\	あとは、お寿司を食べた料金を払えば良いんですね。 
\\	それは、お手ごろですね!	
\\	それは、お手ごろですね! 
\\	だろ?	
\\	だろ? 
\\	どうして、出張回転寿司を始めたんですか?	
\\	どうして、出張回転寿司を始めたんですか? 
\\	自宅や会社で回転寿司を楽しみたいという意見があってね。	
\\	自宅や会社で回転寿司を楽しみたいという意見があってね。 
\\	すばらしいですね!	
\\	すばらしいですね! 
\\	会社やイベントだけでなく、人がたくさん集まるホームパーティーなどでも、お勧めですね!	
\\	会社やイベントだけでなく、人がたくさん集まるホームパーティーなどでも、お勧めですね! 
\\	やせる
\\	実は
\\	細い
\\	果物
\\	肌
\\	ぺこぺこ
\\	夜食
\\	久しぶりー。あれー?真由ちゃん、やせたー?	
\\	久しぶりー。あれー?真由ちゃん、やせたー? 
\\	あ、分かる?少し細くなったでしょ。	
\\	あ、分かる?少し細くなったでしょ。 
\\	実は、ダイエットしているんだー。	
\\	実は、ダイエットしているんだー。 
\\	そうなんだー!何してるの?	
\\	そうなんだー!何してるの? 
\\	フルーツダイエット。すごく簡単だよ。	
\\	フルーツダイエット。すごく簡単だよ。 
\\	毎朝、果物をたべるようにするだけ。	
\\	毎朝、果物をたべるようにするだけ。 
\\	へー!いつ頃からはじめたの?	
\\	へー!いつ頃からはじめたの? 
\\	半年前。	
\\	半年前。 
\\	肌もきれいになったんじゃない?	
\\	肌もきれいになったんじゃない? 
\\	いいなぁ・・・。私も頑張ろうっと。	
\\	いいなぁ・・・。私も頑張ろうっと。 
\\	今、何かしているの?	
\\	今、何かしているの? 
\\	うん。実はウォーキングダイエットを始めたの。	
\\	うん。実はウォーキングダイエットを始めたの。 
\\	へー。健康によさそう!	
\\	へー。健康によさそう! 
\\	毎朝、公園を歩くようにしているの。	
\\	毎朝、公園を歩くようにしているの。 
\\	それから、夕飯も早い時間に食べるようにしているんだ。	
\\	それから、夕飯も早い時間に食べるようにしているんだ。 
\\	へー。でも、毎朝歩くのは、大変でしょ。えらいねぇ。	
\\	へー。でも、毎朝歩くのは、大変でしょ。えらいねぇ。 
\\	あたし歩くの好きだから、大丈夫。でも・・・	
\\	あたし歩くの好きだから、大丈夫。でも・・・ 
\\	でも?	
\\	でも? 
\\	ウォーキング始めてから太ったんだよね。	
\\	ウォーキング始めてから太ったんだよね。 
\\	運動すると、お腹がすごい空くから、ご飯をたくさん食べるようになって・・・。	
\\	運動すると、お腹がすごい空くから、ご飯をたくさん食べるようになって・・・。 
\\	ああ・・・。	
\\	ああ・・・。 
\\	それに、夕飯の時間が早いから、夜10時にはお腹ぺこぺこなの。	
\\	それに、夕飯の時間が早いから、夜10時にはお腹ぺこぺこなの。 
\\	だから、夜食を食べるようになって・・・。三キロくらい太った。	
\\	だから、夜食を食べるようになって・・・。三キロくらい太った。 
\\	電子レンジ
\\	揚がる
\\	栄養素
\\	ポタポタ
\\	減らす
\\	揚げる
\\	壊れる
\\	壊す
\\	落とす
\\	落ちる
\\	油
\\	機能
\\	余分
\\	はい、今日ご紹介する新製品は、こちら!	
\\	はい、今日ご紹介する新製品は、こちら! 
\\	シャーポの電子レンジ「ヘルカロ」です!	
\\	シャーポの電子レンジ「ヘルカロ」です! 
\\	大変人気の商品ですね。	
\\	大変人気の商品ですね。 
\\	そうです!塩と油の量を減らすことができるので、とても健康に良いんです。	
\\	そうです!塩と油の量を減らすことができるので、とても健康に良いんです。 
\\	なるほど。	
\\	なるほど。 
\\	たとえば、から揚げを作りましょう。	
\\	たとえば、から揚げを作りましょう。 
\\	はい。	
\\	はい。 
\\	このヘルカロは余分な油を落とします。	
\\	このヘルカロは余分な油を落とします。 
\\	見えますか?油が落ちていますよね。	
\\	見えますか?油が落ちていますよね。 
\\	うわー、本当ですね。ぽたぽた落ちてる・・・。	
\\	うわー、本当ですね。ぽたぽた落ちてる・・・。 
\\	このように、ヘルシーにから揚げを揚げることもできるんです。	
\\	このように、ヘルシーにから揚げを揚げることもできるんです。 
\\	はい、どうぞ。食べてください。	
\\	はい、どうぞ。食べてください。 
\\	(もぐもぐ)おいしい!ヘルシーなのに、おいしく揚がるんですね。	
\\	(もぐもぐ)おいしい!ヘルシーなのに、おいしく揚がるんですね。 
\\	そうなんですよー。しかも、この「ヘルカロ」はビタミンなども壊しません。	
\\	そうなんですよー。しかも、この「ヘルカロ」はビタミンなども壊しません。 
\\	栄養素がこわれないのは素晴らしいですね。	
\\	栄養素がこわれないのは素晴らしいですね。 
\\	さらに!温めた料理のカロリーは、自動で表示されます。	
\\	さらに!温めた料理のカロリーは、自動で表示されます。 
\\	それは、便利な機能ですね!さすが、「ヘルカロ」!	
\\	それは、便利な機能ですね!さすが、「ヘルカロ」! 
\\	で、気になるお値段は?	
\\	で、気になるお値段は? 
\\	今回も驚きの価格!!	
\\	今回も驚きの価格!! 
\\	なんと、39800円!どうでしょう!	
\\	なんと、39800円!どうでしょう! 
\\	慣れる
\\	化粧品
\\	財布
\\	節約
\\	流れ
\\	社員食堂
\\	こっそり
\\	営業する
\\	美容
\\	美香に会うのは、卒業以来だよな。どう?仕事は慣れた?	
\\	美香に会うのは、卒業以来だよな。どう?仕事は慣れた? 
\\	うん、まあまあかなー。最近やっと慣れてきた感じ。	
\\	うん、まあまあかなー。最近やっと慣れてきた感じ。 
\\	そうかー、おれもようやく仕事の流れがわかってきたよ。	
\\	そうかー、おれもようやく仕事の流れがわかってきたよ。 
\\	美香は、化粧品会社で働いているんだよね?	
\\	美香は、化粧品会社で働いているんだよね? 
\\	そう。うちの会社、女性が多いから、社員食堂のメニューの横にはカロリーが書いてあるんだ。	
\\	そう。うちの会社、女性が多いから、社員食堂のメニューの横にはカロリーが書いてあるんだ。 
\\	美容と健康のために、野菜をたくさん使ったメニューもたくさんあって、結構おいしいんだ。	
\\	美容と健康のために、野菜をたくさん使ったメニューもたくさんあって、結構おいしいんだ。 
\\	女の人が多いとそうなんだね。	
\\	女の人が多いとそうなんだね。 
\\	あと、健康をチェックするためのコーナーもあるんだよー。	
\\	あと、健康をチェックするためのコーナーもあるんだよー。 
\\	克夫の会社も社員食堂あるでしょ?	
\\	克夫の会社も社員食堂あるでしょ? 
\\	あるある。うちの社員食堂、無料なんだ。	
\\	あるある。うちの社員食堂、無料なんだ。 
\\	えーー!!タダなの?!	
\\	えーー!!タダなの?! 
\\	そう。すごいだろ?社員のために、って社長が決めたんだって。	
\\	そう。すごいだろ?社員のために、って社長が決めたんだって。 
\\	良いなー。お財布に優しいね。	
\\	良いなー。お財布に優しいね。 
\\	うん。でも、節約のためだけじゃなくて、働こうっていう気持ちが強くなるよ。	
\\	うん。でも、節約のためだけじゃなくて、働こうっていう気持ちが強くなるよ。 
\\	しかも、会社に長く残っている人のために、夜も営業してるんだ。	
\\	しかも、会社に長く残っている人のために、夜も営業してるんだ。 
\\	うらやましい!今度、こっそり食べに行きたい!	
\\	うらやましい!今度、こっそり食べに行きたい! 
\\	仕切り
\\	こってり
\end{CJK}
\end{document}