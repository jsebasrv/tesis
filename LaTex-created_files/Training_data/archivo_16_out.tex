\documentclass[8pt]{extreport} 
\usepackage{hyperref}
\usepackage{CJKutf8}
\begin{document}
\begin{CJK}{UTF8}{min}
\\	彼女は僕の肩に頭をもたせかけた。 ・
\\	その規制案は議会に提出された。 ・
\\	私の家には水道が引いてある。 ・
\\	水が引いているので底の様子が見えます。◆【参考】
\\	ロープのよりの反対方向に二回り、回してください。 ・
\\	産卵可能になった初年鶏はすべて、最初は小型の卵を産む。	変化形 
\\	【1自動】卵を産む、産卵{さんらん}する〈俗〉横たわる、横になる◆【文法】この意味で使われることがよくあるが、文法的には間違っているので使用しない方が無難。この意味では
\\	が正しい。◆【参考】
\\	金を賭ける、賭けをする一生懸命{いっしょう けんめい}[夢中{むちゅう}になって]取り組む《海事》〔ある位置{いち}や方向{ほうこう}に〕船を向ける、〔船員{せんいん}が〕所定{しょてい}の位置{いち}に着く【1他動】〔水平{すいへい}な位置{いち}に〕~を置く[おろす・据える]〔楽な位置{いち}に〕~を横たえる[寝かせる] ・
\\	彼女は僕の肩に頭をもたせかけた。〔地面{じめん}に〕~を埋める[〔人を〕埋葬{まいそう}する]〔平らな面に〕~を敷く[広げる・押し付ける]〔食卓{しょくたく}などに〕~を並べる[用意{ようい}する]〔立っている物を〕倒す、平らにする、なでつける《動物》〔卵を〕産む〔金を〕賭ける〔義務{ぎむ}や罰などを〕科する、負わせる、押し付ける〔悪いことなどを〕(人)に帰する、(人)のせいにする〔重要性{じゅうよう せい}などを〕~に置く、~にあるとする〔計画{けいかく}などを〕練る、準備{じゅんび}する〔基礎{きそ}などを〕築く、構築{こうちく}する〔要求などを〕提出する ・
\\	その規制案は議会に提出された。〔設備を〕取り付ける、敷設する ・
\\	私の家には水道が引いてある。〔ロープなどを〕より合わせる、よって作る《軍事》〔大砲{たいほう}の〕照準{しょうじゅん}を合わせる〈性俗〉〔~と〕セックスする◆【同】
\\	【1名】〔物などが〕置かれているさま、形状{けいじょう} ・
\\	水が引いているので底の様子が見えます。◆【参考】
\\	〔ロープやケーブルの〕より(の方向) ・
\\	ロープのよりの反対方向に二回り、回してください。産卵可能{さんらん かのう}な状態{じょうたい} ・
\\	産卵可能になった初年鶏はすべて、最初は小型の卵を産む。〈性俗〉性行為{せいこうい}[セックス](の相手{あいて})、床上手な女【2形】俗人{ぞくじん}の、在家{ざいけ}の、平信徒{ひら しんと}の素人{しろうと}の、一般{いっぱん}の、非専門家{ひ せんもんか}の【3動】
\\	の過去形{かこ けい}レベル2、発音
\\	、カナレイ、変化《動》
\\	《謝る》約束します。二度としません。 ・
\\	~することもある、~ということはままある。 ・
\\	仕方ないよ。 ・
\\	まさか!/あり得ない!/そんなばかな。◆【場面】困った事態の発生について。文字通りには「起きるはずない」だが、通例、それが起きてしまっている状態。 ・
\\	何か変わったことない? ・
\\	どういうことになりますか? ・
\\	何か起こるのだろうか?	変化形 
\\	【自動】〔偶然に〕起こる、発生{はっせい}する、降り懸かる ・
\\	《謝る》約束します。二度としません。 ・
\\	~することもある、~ということはままある。 ・
\\	仕方ないよ。 ・
\\	まさか!/あり得ない!/そんなばかな。◆【場面】困った事態の発生について。文字通りには「起きるはずない」だが、通例、それが起きてしまっている状態。 ・
\\	何か変わったことない? ・
\\	どういうことになりますか? ・
\\	何か起こるのだろうか?たまたま~する、偶然{ぐうぜん}~するレベル1、発音
\\	、カナハップン、ハプン、変化《動》
\\	、分節
\\	《コ》行を削除中にエラーが発生しました。 ・
\\	事故が起きたのは午前_時頃だった。 ・
\\	これはよくあること。 ・
\\	《マニュアル》エラーの発生により印刷が停止した場合、赤のインジケーターランプが点灯します。 ・
\\	世界各地で異常気象が頻発する。 ・
\\	いい考えを思い付いた。 ・
\\	あなたに言われて初めて気が付きました。	変化形 
\\	【自動】〔偶然に〕起こる、発生{はっせい}する、生じる、現れる、存在{そんざい}する ・
\\	《コ》行を削除中にエラーが発生しました。 ・
\\	事故が起きたのは午前_時頃だった。 ・
\\	これはよくあること。 ・
\\	《マニュアル》エラーの発生により印刷が停止した場合、赤のインジケーターランプが点灯します。 ・
\\	世界各地で異常気象が頻発する。頭[心]に浮かぶ、思い付く、気付く ・
\\	いい考えを思い付いた。 ・
\\	あなたに言われて初めて気が付きました。《生物》生息{せいそく}するレベル3、発音
\\	、カナアカー、オカー、変化《動》
\\	、分節
\\	《コ》指定した名前のディレクトリはすでに存在します。 ・
\\	彼は神が存在すると信じています。 ・
\\	それが存在することすら知りませんでした。	変化形 
\\	【自動】存在{そんざい}する、生きている、生存{せいぞん}する、存続{そんぞく}する ・
\\	《コ》指定した名前のディレクトリはすでに存在します。 ・
\\	彼は神が存在すると信じています。 ・
\\	それが存在することすら知りませんでした。生きている、生存{せいぞん}する、存続{そんぞく}するレベル2、発音
\\	、カナイグジスト、エグズィスト、変化《動》
\\	、分節
\\	その患者は依然として危篤[危険な]状態です。 ・
\\	その音楽教授は退職するまで、彼の下にいる教職員を担当していました。 ・
\\	彼はクラブ活動に参加するために放課後、学校に残りました。 ・
\\	科学者たちはわにの遺骨を発見しました。	変化形 
\\	【自動】〔状態{じょうたい}が〕依然{いぜん}として[引き続き・相変わらず]~のままである ・
\\	その患者は依然として危篤[危険な]状態です。 ・
\\	その音楽教授は退職するまで、彼の下にいる教職員を担当していました。〔ある場所に〕とどまる、残る、滞在{たいざい}する ・
\\	彼はクラブ活動に参加するために放課後、学校に残りました。生き残る、残存{ざんそん}する【名】
\\	残りもの、残額{ざんがく}
\\	遺作{いさく}、遺跡{いせき}、遺構{いこう}、跡、形跡{けいせき}
\\	遺体{いたい}、遺骸{いがい}、遺骨{いこつ}、亡骸{ぼうがい} ・
\\	科学者たちはわにの遺骨を発見しました。レベル2、発音
\\	、カナリメイン、変化《動》
\\	、分節
\\	動くな!◆【同】
\\	台風は北西に移動しています。 ・
\\	弊社は本日をもって下記に移転致しましたので、ご通知申し上げます。 ・
\\	その研究所は大阪から九州に移設されました。 ・
\\	感動して涙が出ました。 ・
\\	議会の休会を提議します。	変化形 
\\	【自動】〔物理的な位置が〕移動{いどう}する、動く ・
\\	動くな!◆【同】
\\	台風は北西に移動しています。引っ越す、移転{いてん}[転居{てんきょ}]する ・
\\	弊社は本日をもって下記に移転致しましたので、ご通知申し上げます。転職{てんしょく}する、異動{いどう}する、昇進{しょうしん}する〔人が〕体を動かす[揺らす]〔人が〕行動{こうどう}する、行動{こうどう}に移す、取り掛かる〈話〉立ち去る、出発{しゅっぱつ}する、出掛ける{でかける}〔状況{じょうきょう}や筋が〕変化{へんか}する、展開{てんかい}する、進展{しんてん}する〔意見{いけん}や見方{みかた}が〕変わる、うつろう〔商品{しょうひん}が〕よく売れる、さばける〔正式{せいしき}に〕提案{ていあん}する、動議{どうぎ}を出す〔チェスなどで〕駒を動かす〔腸が〕通じがある、排せつする【他動】〔~の物理的位置{ぶつり てき いち}を〕動かす、移動{いどう}する〔人・住所{じゅうしょ}・職場{しょくば}などを〕引っ越させる、転居{てんきょ}させる、移転{いてん}する ・
\\	その研究所は大阪から九州に移設されました。〔状況{じょうきょう}や筋などを〕展開{てんかい}[進展{しんてん}]させる〔人の意見{いけん}や見方{みかた}を〕変えさせる、覆す〔人に~するよう〕促す、行動{こうどう}させる〔機械{きかい}などを〕動作{どうさ}[機能{きのう}]させる〔人の〕心を動かす、感動{かんどう}させる ・
\\	感動して涙が出ました。〔~を議会{ぎかい}や法廷{ほうてい}で〕提案{ていあん}する、動議{どうぎ}として提出{ていしゅつ}する ・
\\	議会の休会を提議します。〔商品{しょうひん}を〕うまく売る、さばく〔腸を〕通じを良くさせる、排せつさせる【名】動くこと、運動{うんどう}、移動{いどう}〔ある様態{ようたい}の〕動き(方)〔一連{いちれん}の〕手だて、戦略{せんりゃく}引っ越し、転居{てんきょ}〔チェスなどの〕手番{てつがい}、指し手レベル1、発音
\\	、カナムーブ、変化《動》
\\	その支社は今月、東京から大阪に移りました[移転しました]。 ・
\\	外国からここ日本の私の口座に送金してください。 ・
\\	ジョン・ドウさんにつないでいただけますか?◆電話 ・
\\	その情報はサーバーからプリンターへ転送されます。	変化形 
\\	【自動】〔人や物の場所{ばしょ}が〕移動{いどう}[転移{てんい}]する〔組織内{そしきない}で〕転勤{てんきん}[転任{てんにん}]する〔情報{じょうほう}などが〕伝わる、伝達{でんたつ}する〔大学{だいがく}などで〕転校{てんこう}[転部{てんぶ}・編入{へんにゅう}]する〔公共交通機関{こうきょう こうつう きかん}で〕乗り換える【他動】〔人や物の場所を〕移す、移動{いどう}させる ・
\\	その支社は今月、東京から大阪に移りました[移転しました]。 ・
\\	外国からここ日本の私の口座に送金してください。〔社内{しゃない}で人を〕転勤{てんきん}[転任{てんにん}]させる〔情報{じょうほう}などを〕伝える、伝達{でんたつ}する ・
\\	ジョン・ドウさんにつないでいただけますか?◆電話 ・
\\	その情報はサーバーからプリンターへ転送されます。〔交通機関{こうつう きかん}を〕乗り換える〔大学{だいがく}や学部{がくぶ}などを〕転校{てんこう}[転部{てんぶ}]する《法律》〔所有権{しょゆうけん}を〕譲渡{じょうと}する〔デザインなどを〕転写{てんしゃ}する【名】〔人や物の〕移動{いどう}、異動{いどう}、転勤{てんきん}◆【同】
\\	〔大学{だいがく}などの〕転校生{てんこうせい}、転部生{てんぶ せい}〔交通機関{こうつう きかん}の〕乗り換え切符{きっぷ}[駅]《法律》〔財産{ざいさん}や権利{けんり}の〕譲渡{じょうと}◆【同】
\\	〔熱や圧力{あつりょく}による画像{がぞう}の〕転写{てんしゃ}〔株式{かぶしき}や債券{さいけん}の〕名義変更{めいぎ へんこう}レベル3、発音《名》
\\	《動》
\\	、カナトランスファー、変化《動》
\\	、分節
\\	政府による贈収賄捜査の結果、彼は解任された。	変化形 
\\	【自動】移動{いどう}する〔塗料などが〕取れる、はがれる転居{てんきょ}[移転{いてん}]する【他動】〔物を〕取り除く、取り去る、取り外す、除去{じょきょ}する、持ち去る〔武器{ぶき}や兵器{へいき}を〕廃絶{はいぜつ}する〔衣服を〕脱ぐ〔人を〕解任{かいにん}する、去らせる、排除{はいじょ}する、追い出す ・
\\	政府による贈収賄捜査の結果、彼は解任された。〔制裁{せいさい}を〕解除{かいじょ}する~を移動{いどう}させる、移す、移転{いてん}させる殺す、削除{さくじょ}する〔化粧{けしょう}や汚れなどを〕落とす【名】移動{いどう}、移転{いてん}、転居{てんきょ}距離{きょり}、隔たり、差、親等{しんとう}の差、世代差{せだい さ}、世代{せだい}のずれレベル2、発音
\\	、カナリムーブ、変化《動》
\\	、分節
\\	事態は急を要します。 ・
\\	キーボードの
\\	キーを押してください。 ・
\\	《マニュアル》このランプが点滅した場合は、停止ボタンを押してから作業を続行してください。 ・
\\	彼らに無理強いをすべきではない。	変化形 
\\	【1自動】〔物理的{ぶつり てき}に〕押す、圧力{あつりょく}を掛ける〔精神的{せいしん てき}に〕圧迫{あっぱく}する、プレッシャーを掛ける〔焦って〕突き進む、前進{ぜんしん}する〔衣類{いるい}などに〕アイロンを掛ける〈文〉〔群衆{ぐんしゅう}が〕群がる、ひしめく〈文〉急ぐ必要がある、緊急{きんきゅう}を要する ・
\\	事態は急を要します。〔バスケットボールで〕プレスを掛ける〔重量挙げで〕プレスで挙げる【1他動】~を押す[押し付ける] ・
\\	キーボードの
\\	キーを押してください。 ・
\\	《マニュアル》このランプが点滅した場合は、停止ボタンを押してから作業を続行してください。〔果物{くだもの}やジュースなどを〕搾る、搾り出す〔押して〕~を圧縮{あっしゅく}する[固める]〔衣類{いるい}などに〕アイロンを掛ける〔腕や手で〕~を抱き[握り]締める〔人に~するよう〕無理強いする、強要{きょうよう}する ・
\\	彼らに無理強いをすべきではない。〔人に〕~を懇願{こんがん}する[せがむ]〔重要{じゅうよう}な点を〕強調{きょうちょう}する、力説{りきせつ}する〔レコードなどを〕プレスして作る〔花などを〕押し花にする〔バスケットボールで相手{あいて}に〕プレスを掛ける【1名】押すこと、圧迫{あっぱく}〔果物{くだもの}などの〕絞り器、圧搾機{あっさくき}〔びっしりと集まった〕群衆{ぐんしゅう}、人混{ひとご}み〈文〉〔人混{ひとご}みの〕押し合いへし合い〔洋服{ようふく}の〕プレス機、〔機械{きかい}の〕プレス機、〔印刷{いんさつ}の〕プレス機〈米〉〔大型{おおがた}の〕リネン[タオル]棚出版業{しゅっぱんぎょう}、出版物{しゅっぱんぶつ}、新聞{しんぶん}、新聞界{しんぶんかい}、雑誌{ざっし}、報道機関{ほうどう きかん}、報道陣{ほうどうじん}、マスコミ、記者{きしゃ}〔新聞{しんぶん}や雑誌{ざっし}の〕論評{ろんぴょう}、記事{きじ}〔洋服{ようふく}への〕アイロンがけ〔アイロンでついた〕折り目〔重量挙げの〕プレス◆肩の位置まで持ち上げてから足を動かさずに頭上に上げること。〔バスケットボールの〕(オールコート)プレス◆相手選手に常にプレッシャーをかける攻撃的なディフェンス。【2他動】(人)を徴兵{ちょうへい}[徴用{ちょうよう}]する【2名】徴用{ちょうよう}、徴兵{ちょうへい}レベル2、発音
\\	、カナプレス、変化《動》
\\	彼は最近の訓練[トレーニング]で膝腱を痛めた。 ・
\\	私は二晩続けて徹夜しました。 ・
\\	ビールの注ぎ方を学んだ。	変化形 
\\	【自動】引く、引っ張る漕ぐ、骨折って進む【他動】~を引く、引っ張る、引き寄せる取る、抜く、むしる〔相手{あいて}を〕引き付ける、勝ち取る〔筋肉{きんにく}などを〕無理{むり}に引き伸ばして痛める ・
\\	彼は最近の訓練[トレーニング]で膝腱を痛めた。《野球・ゴルフ》〔打者{だしゃ}がボールを〕引っ張る〈話〉〔計画などを〕行う ・
\\	私は二晩続けて徹夜しました。〈話〉〔悪事を〕しでかす〔武器などを〕引き抜いて用意{ようい}する〔ビールなどをたるから〕つぐ、注ぐ ・
\\	ビールの注ぎ方を学んだ。【名】引くこと強み、影響{えいきょう}〔酒・水の〕1杯、一口{ひとくち}〔たばこの〕一服{いっぷく}コネ
\\	引き手《ゴルフ》プル・ショットレベル1、発音
\\	、カナプル、変化《動》
\\	温かい海水が北から押し寄せる。 ・
\\	私はもうすぐ30です。 ・
\\	私はもうすぐ70で、体が相当に悪い。 ・
\\	彼女はもうすぐ50だが、いまだにきれいな脚をしている。 ・
\\	私がお奨めしていたのは
\\	を
\\	とは分けておくことでした。 ・
\\	そのイギリスの企業が最近発表した日本企業との合弁事業は、日本市場開拓の動きの一環である。 ・
\\	連合軍は予想より何日も前に首都への軍事行動を開始しました。	変化形 
\\	【自動】押す、押し寄せる、押し進む、前進{ぜんしん}する、突き出る ・
\\	温かい海水が北から押し寄せる。〔出産の時に〕力む、気張る{きばる}、息{いき}む【他動】~を押す、つきのける、突く推し進める、後押{あとお}しする人に無理強いする、人に強要{きょうよう}する、強いて~させる、~を押圧する〔商品などを〕無理{むり}に勧める、押し付ける〔年齢{ねんれい}・数などに〕近づく ・
\\	私はもうすぐ30です。 ・
\\	私はもうすぐ70で、体が相当に悪い。 ・
\\	彼女はもうすぐ50だが、いまだにきれいな脚をしている。《野球》(打席{だせき}の)反対側{はんたいがわ}に打つ、流し打ちをする《ゴルフ》プッシュする〈俗〉麻薬{まやく}を売る《コ》〔ファイルを〕転送{てんそう}する◆【反】
\\	◆【同】
\\	《コ》プッシュ(型)配信{(がた)はいしん}する《コ》〔データなどを〕プッシュする◆1次元的なデータ構造の先頭または末尾にデータを追加すること。【名】押すこと、押し、ひと押し突進{とっしん}、奮闘{ふんとう}奨励{しょうれい}、お勧め ・
\\	私がお奨めしていたのは
\\	を
\\	とは分けておくことでした。〔市場開拓・販売・政治・軍事などにおける〕動き、活動、行動 ・
\\	そのイギリスの企業が最近発表した日本企業との合弁事業は、日本市場開拓の動きの一環である。 ・
\\	連合軍は予想より何日も前に首都への軍事行動を開始しました。圧力{あつりょく}、影響力{えいきょうりょく}危機{きき}、窮地{きゅうち}《ゴルフ》プッシュ〈俗〉悪党{あくとう}《囲碁》押しレベル1、発音
\\	、カナプッシュ、変化《動》
\\	トヨタ、アメリカ工場の建設地にケンタッキーを選ぶ。 ・
\\	お好きなものをどうぞ[選んでください]。 ・
\\	発掘は鋤とつるはしで始まる。	変化形 
\\	【1自動】つつく、ほじる、つくついばむ、〔食欲{しょくよく}がなさそうに〕少しずつ口にする選ぶ、選定{せんてい}する、選別{せんべつ}する、精選{せいせん}する摘み取る[取れる]盗む、盗み取る、かすめる【1他動】摘み取る、採集{さいしゅう}する、採取{さいしゅ}する
\\	に
\\	を〕摘んでやる◆【用法】
\\	または
\\	念入りに選ぶ、精選{せいせん}する ・
\\	トヨタ、アメリカ工場の建設地にケンタッキーを選ぶ。〔細長い物で〕~を突く[つつく・穴を掘る]〔鼻や耳の穴を〕ほじる〔肉などを〕しゃぶる
\\	から
\\	を〕抜き取る◆【用法】
\\	〔毛を〕むしり取る、〔とげを〕抜く〔鳥が〕~をついばむ〔食欲{しょくよく}なさそうに〕食べる〔欠点{けってん}・あらなどを〕捜す〔他人の所有物を〕盗む、盗み取る、かすめる〔錠を〕こじ開ける◆【参考】
\\	〔弦楽器{げんがっき}を〕弾く〔けんかを〕仕掛{しか}ける~のきっかけを得る《野球》難しいゴロをさばく【1名】選択{せんたく}、選ぶ権利{けんり} ・
\\	お好きなものをどうぞ[選んでください]。〔選ばれた〕最高{さいこう}のもの[人]〔とがったもので〕突くこと〔手で摘んだ〕収穫物{しゅうかく ぶつ}〔バスケットボールの〕ピック◆ボールを持った味方が楽にシュートできるように、相手ディフェンスの前でその動きを妨害する動き。【2名】つるはし◆地面を掘る道具。 ・
\\	発掘は鋤とつるはしで始まる。つまようじ〈主に米〉=
\\	レベル1、発音
\\	、カナピック、変化《動》
\\	子どもは親を選べない。 ・
\\	《名言》運命が君の親族を選ぶが、君の友人を選ぶのは君です。 ・
\\	それぞれの質問に対して最も適切な答えを選びましょう。◆ヒアリングテストの説明	変化形 
\\	【他動】~を選ぶ、~を選択{せんたく}する、~を人選{じんせん}する、~をえり抜く ・
\\	子どもは親を選べない。 ・
\\	《名言》運命が君の親族を選ぶが、君の友人を選ぶのは君です。 ・
\\	それぞれの質問に対して最も適切な答えを選びましょう。◆ヒアリングテストの説明~を望む[欲する・欲しいと思う]レベル1、発音
\\	、カナチューズ、変化《動》
\\	支払方法をお選びください。	変化形 
\\	【自動】選び出す、選定{せんてい}する【他動】~を選択{せんたく}[選定{せんてい}・選抜{せんばつ}]する
\\	に比べて、より広範囲のもの中から慎重に選ぶという意味を持つ、やや硬い表現。 ・
\\	支払方法をお選びください。〔スイッチなどを〕~に切り替える《コ》
\\	の画面{がめん}でオブジェクトを〕選択{せんたく}[セレクト]する◆文字列や図形などを選択して反転表示させること。【形】厳選{げんせん}した、えり抜きの上等{じょうとう}の、優良{ゆうりょう}な〔会員資格{かいいん しかく}が〕限定{げんてい}された〔牛枝肉{ぎゅう えだにく}の等級{とうきゅう}が〕セレクトの◆米農務省の定めた規格で、一般に流通している三つの中では下位の等級。プライム
\\	、チョイス
\\	に比べて脂肪の少ない赤身肉。レベル3、発音
\\	、カナスィレクト、セレクト、変化《動》
\\	、分節
\\	少しお金を貸していただけないでしょうか? ・
\\	何か書くものを貸してくれませんか?	変化形 
\\	【自動】〔人に物・金を〕貸す【他動】〔人に物・金を〕貸す、〔人に物を〕貸し出す ・
\\	少しお金を貸していただけないでしょうか? ・
\\	何か書くものを貸してくれませんか?〔人に力を〕貸す〔~を…に〕添える〔物が〕~に役立つ{やくだつ}レベル2、発音
\\	、カナレンド、変化《動》
\\	ペンを貸してもらえますか?	変化形 
\\	【自動】借り入れをする、貸し付けを受ける《ゴルフ》〔パットを打つときに〕ボロウする◆傾斜を考慮に入れて右または左に打つこと。《ゴルフ》〔ボールが〕ボロウする◆グリーンの傾斜で右や左に曲がること。【他動】〔他の人の所有物{しょゆうぶつ}を〕借りる、借用{しゃくよう}する◆通例、許可を得て一時的に使用することを指す。 ・
\\	ペンを貸してもらえますか?〔有利子{ゆう りし}の貸付金{かしつけきん}を〕借りる、借用{しゃくよう}する〔作品{さくひん}や思想{しそう}などの〕まねをする、取り入れる《言語学》〔他の言語{げんご}の語を〕借用{しゃくよう}する〔引き算で上の位から〕借りる【名】《ゴルフ》〔ボールの〕ボロウ◆グリーンの傾斜による右や左への曲がり具合。レベル2、発音
\\	、カナバロウ、ボーロウ、変化《動》
\\	、分節
\\	変化形 
\\	【名】〈俗〉親◆【参考】
\\	お支払いはどうされますか? ・
\\	私のおごりです。 ・
\\	賭け事は割に合わない。 ・
\\	誘拐?それは割に合わない仕事です。 ・
\\	正直者がばかをみる。 ・
\\	それは割に合わない[お金をかける価値はない]と思いますよ。◆【用法】古いものなど、「修理をするより買い換えた方がいいですよ」というようなときに使う。 ・
\\	うそをつくことは良い結果をもたらしません。 ・
\\	後が怖い。えらいことになりそうだ。 ・
\\	この恨みは返すからな。 ・
\\	私は100ドル払いました。 ・
\\	私は彼に1ドル払いました。 ・
\\	彼は、彼女に口止め料を[黙っていてもらうためにお金を]払いました。	変化形 
\\	【1自動】支払う、代金{だいきん}を払う ・
\\	お支払いはどうされますか? ・
\\	私のおごりです。〔仕事{しごと}などが〕もうかる、利益{りえき}になる〔苦労・手間・行為などが〕得になる、報われる、割に合う、良い結果をもたらす、引き合う ・
\\	賭け事は割に合わない。 ・
\\	誘拐?それは割に合わない仕事です。 ・
\\	正直者がばかをみる。 ・
\\	それは割に合わない[お金をかける価値はない]と思いますよ。◆【用法】古いものなど、「修理をするより買い換えた方がいいですよ」というようなときに使う。 ・
\\	うそをつくことは良い結果をもたらしません。償いをする、苦しむ ・
\\	後が怖い。えらいことになりそうだ。 ・
\\	この恨みは返すからな。【1他動】~を支払う ・
\\	私は100ドル払いました。(人)に~を払う、(人)にお金を払う ・
\\	私は彼に1ドル払いました。 ・
\\	彼は、彼女に口止め料を[黙っていてもらうためにお金を]払いました。【1名】支払い〔労働などに対する〕賃金{ちんぎん}、給料{きゅうりょう}、報酬{ほうしゅう}有料{ゆうりょう}◆入場料などは
\\	。駐車場・公衆トイレでは
\\	。通信・道路などでは
\\	。【1形】有料{ゆうりょう}の【2他動】《航海》〔船底{ふなぞこ}に〕タールを塗る◆この意味では過去・過去分詞に
\\	が使われることもあるレベル1、発音
\\	、カナペイ、変化《動》
\\	フィールドはカンマで区切れている。 ・
\\	私たちは今では別々の部屋で寝ている。	変化形 
\\	【自動】〔物の距離{きょり}が〕バラバラになる、離れる、はがれる〔群衆{ぐんしゅう}などが〕分かれる、散らばる〔大きな組織{そしき}から〕分離{ぶんり}する、独立{どくりつ}する《法律》〔夫婦{ふうふ}などが〕離婚{りこん}する、別居{べっきょ}する〔物が小さな単位{たんい}に〕分離{ぶんり}する、分解{ぶんかい}する【他動】〔人や物の距離{きょり}をお互いに〕離す、分ける、バラバラにする〔二人{ふたり}や二つの物の間を〕隔てる、仕切る{しきる}、区切る{くぎる} ・
\\	フィールドはカンマで区切れている。〔人や物をほかと〕異なるものにする、区別{くべつ}する〔人や物を〕分類{ぶんるい}する、グループ分けする〔複数{ふくすう}のものの〕違いを際立{きわだ}たせる、差異{さい}を明らかにする〔物を小さな単位{たんい}に〕分ける、分解{ぶんかい}する〔大きな組織{そしき}を小さく〕分ける、独立{どくりつ}させる〔裁判所{さいばんしょ}が人を〕離婚{りこん}させる、別居{べっきょ}させる〔仕事{しごと}や軍務{ぐんむ}などから人を〕解雇{かいこ}する、免除{めんじょ}する【名】〔雑誌{ざっし}などの〕抜き刷り◆【同】
\\	〔洋服{ようふく}の〕セパレーツ◆スカート、ズボン、上着など別々のものを買ってさまざまな組み合わせを楽しむことができるように作られた洋服◆通例
\\	◆【略】
\\	【形】〔位置{いち}や場所{ばしょ}が〕接していない、分かれた、離れた ・
\\	私たちは今では別々の部屋で寝ている。〔もともと一緒{いっしょ}のものが〕分けられた、別々{べつべつ}になった〔問題{もんだい}などがほかと〕関連{かんれん}がない、結び付かない、独特{どくとく}の〔考えなどが他の人と〕共有{きょうゆう}していない、別個{べっこ}の、単独{たんどく}の〔もとの宗派{しゅうは}から〕分派{ぶんぱ}した、独立{どくりつ}した◆通例、
\\	レベル2、発音《名・形》
\\	《動》
\\	、カナセパレイト、変化《動》
\\	、分節
\\	身の回りの品をまとめ、部屋から歩いて出ようとした。 ・
\\	空港にあなたを迎えに行くように手配しましょう。 ・
\\	彼は1965年にオスロでノーベル賞を受けた。 ・
\\	混乱した考えを何とかまとめて彼女をじっと見返しました。	変化形 
\\	【1自動】集まる、まとまる受け取る寄付{きふ}を募る、募金{ぼきん}する【1他動】〔あちこちの場所から〕~を集める[まとめる]◆【同】
\\	身の回りの品をまとめ、部屋から歩いて出ようとした。〔同じ種類{しゅるい}のものを〕収集{しゅうしゅう}する◆趣味や学問などのために関心のあるものを組織的に集めること。〔人を〕迎えに行く ・
\\	空港にあなたを迎えに行くように手配しましょう。〔賞金{しょうきん}などを〕受け取る ・
\\	彼は1965年にオスロでノーベル賞を受けた。〔税金{ぜいきん}などを〕徴収{ちょうしゅう}する〔寄付{きふ}などを〕募集{ぼしゅう}する〔考えなどを〕まとめる ・
\\	混乱した考えを何とかまとめて彼女をじっと見返しました。〔気持ちなどを〕引き締める【1形】〈米〉〔電話{でんわ}が〕料金受信人払い{りょうきん じゅしんにん ばらい}の◆【同】
\\	◆【略】
\\	【1副】〈米〉〔電話{でんわ}が〕料金受信人払い{りょうきん じゅしんにん ばらい}で◆【略】
\\	【2名】《カトリック》集祷文{しゅうとう ぶん}◆ミサのときに使徒書簡の朗読の前に唱えられる短い祈り。レベル2、発音
\\	、カナコレクト、変化《動》
\\	、分節
\\	暗雲が垂れ込めてきた。 ・
\\	できるうちにバラを摘みなさい。/若いうちに青春を楽しみなさい。 ・
\\	コーチは最後の言葉をかけるために、選手を自分の周りに集めた。	変化形 
\\	【自動】〔人や物が〕寄せ集まる、集合{しゅうごう}する〔物が徐々{じょじょ}に〕積み重なる、蓄積{ちくせき}する ・
\\	暗雲が垂れ込めてきた。〔程度{ていど}が〕増加{ぞうか}する、増進{ぞうしん}するまゆにしわを寄せる、まゆをひそめる《病理》腫れ物[膿瘍{のうよう}]ができる〔野イチゴなどを〕探し回る、あさる【他動】〔穀物や自然の産物を〕収穫する、摘み取る ・
\\	できるうちにバラを摘みなさい。/若いうちに青春を楽しみなさい。〔物を〕寄せ集める、収集{しゅうしゅう}する〔徐々{じょじょ}に〕~を積み重ねる[蓄積{ちくせき}する]〔人を〕集合{しゅうごう}させる、招集{しょうしゅう}する ・
\\	コーチは最後の言葉をかけるために、選手を自分の周りに集めた。〔支持者{しじしゃ}や聴衆{ちょうしゅう}を〕周りに集める、寄せ集める〔速度{そくど}などを〕上げる、増加{ぞうか}させる〔勇気{ゆうき}や力を〕奮い立たせる、奮い起こす
\\	以下だと〕推測{すいそく}[推量{すいりょう}]する~を引き寄せる、~を手元{てもと}に寄せる〔子どもなどを〕持ち上げる、抱き上げる〔まゆを〕ひそめる、〔まゆに〕しわを寄せる〔衣類{いるい}に〕ギャザー[襞]を付ける《印刷》〔折丁{おりちょう}を〕ページ順に並べる〔吹きガラスのギャザを〕パイプの先に付ける【名】集めた[収穫{しゅうかく}した]物〔衣類{いるい}の一つの〕ギャザー、襞{ひだ}◆【参考】
\\	〔吹きガラスの〕ギャザ◆パイプの先端に付けた、溶けたガラス玉。レベル2、発音
\\	、カナギャザー、変化《動》
\\	、分節
\\	「よろしければご一緒にどうぞ」「ありがとう。喜んで」 ・
\\	〈話〉同じ境遇ですね!/こちらも同様[同じ状況]だ!◆【同】
\\	数人の人々の協力を得て、彼はプロジェクトを開始しました。 ・
\\	そのパーティーには50人を超えるゲストが参加しました。 ・
\\	このプロジェクトに力を結集しよう。 ・
\\	空港(内の)道は幹線道路につながっています。
\\	変化形 
\\	"【自動】加わる、加入{かにゅう}する、参加{さんか}する、交わる、一緒{いっしょ}になる【他動】~に加わる、~に加入{かにゅう}する、~に参加{さんか}する、~と交わる、~と一緒{いっしょ}になる ・
\\	「よろしければご一緒にどうぞ」「ありがとう。喜んで」 ・
\\	〈話〉同じ境遇ですね!/こちらも同様[同じ状況]だ!◆【同】
\\	数人の人々の協力を得て、彼はプロジェクトを開始しました。 ・
\\	そのパーティーには50人を超えるゲストが参加しました。〔二つ以上の物をじかに〕つなぎ合わせる、結び付ける、結合{けつごう}する、連結{れんけつ}する、合わせる、つなぐ、接合{せつごう}する、接続{せつぞく}する ・
\\	このプロジェクトに力を結集しよう。 ・
\\	空港(内の)道は幹線道路につながっています。【名】加入{かにゅう}、結び、結合{けつごう}、接合{せつごう}(個所{かしょ})、継ぎ目レベル1、発音
\\	、カナジョイン、変化《動》
\\	私はそのテニスクラブに所属しています。
\\	とはならない点に注意。 ・
\\	あなたにはどこにも居場所がありません。	変化形 
\\	【自動】〔~に〕属する、〔~に〕所属{しょぞく}する、〔~の〕一員{いちいん}である、〔~の〕所有物{しょゆうぶつ}である ・
\\	私はそのテニスクラブに所属しています。
\\	とはならない点に注意。〔物が〕あるべき所にある、〔人が〕いるべき所にいる、〔~に〕合っている ・
\\	あなたにはどこにも居場所がありません。レベル2、発音
\\	、カナビローング、ビロング、変化《動》
\\	、分節
\\	消費税が料金に含まれています。/消費税込みの料金になっています。 ・
\\	私たちは、そのプログラムにもう1曲加えました。 ・
\\	「あなたの人生に私は含まれていますか?」「あなたなしで人生を過ごすなんて、とうてい考えられません。あなたがいるからこそ私の人生なんです」◆恋人同士の会話
\\	変化形 
\\	"【他動】〔全体の一部として〕~を含める[含有{がんゆう}する・包含{ほうがん}する]◆【略】
\\	消費税が料金に含まれています。/消費税込みの料金になっています。 ・
\\	私たちは、そのプログラムにもう1曲加えました。 ・
\\	「あなたの人生に私は含まれていますか?」「あなたなしで人生を過ごすなんて、とうてい考えられません。あなたがいるからこそ私の人生なんです」◆恋人同士の会話同封{どうふう}する、収録{しゅうろく}する、盛り込む〔主語〕としては[には・の中には]~がある[挙げられる・含まれる・顔を連ねる]《コ》インクルードする◆(プログラム)をソースプログラムに組み込むこと、
\\	言語などで用いる命令の一つレベル2、発音
\\	、カナインクルード、変化《動》
\\	、分節
\\	この手紙には何も同封できない。 ・
\\	そのフロッピー・ディスクにはとても重要なデータが入っている。 ・
\\	その混合物は_%の水を含んでいる。 ・
\\	この箱の中にはガラスが入っている。	変化形 
\\	【他動】〔容器などが~を全体的に〕含む、包含{ほうがん}する ・
\\	この手紙には何も同封できない。 ・
\\	そのフロッピー・ディスクにはとても重要なデータが入っている。 ・
\\	その混合物は_%の水を含んでいる。 ・
\\	この箱の中にはガラスが入っている。〔場所が人々を〕収容{しゅうよう}できる〔敵などを〕封じ込める、阻止{そし}する、食い止める〔感情などを〕抑える、抑制{よくせい}するレベル2、発音
\\	、カナコンテイン、変化《動》
\\	、分節
\\	変化形 
\\	【自動】〔~から〕成る、成り立つ〔本質的{ほんしつ てき}なものが~に〕存在{そんざい}する、ある〔~を〕基礎とする、〔~によって〕定義される〔~と〕一致{いっち}するレベル4、発音
\\	、カナコンシスト、コンスィスト、変化《動》
\\	、分節
\\	変化形 
\\	【自動】欠いている、ほとんどない不足{ふそく}している【他動】~を欠いている、~がほとんどない~が不足{ふそく}している【名】〔~の〕不足{ふそく}、欠乏{けつぼう}◆【用法】
\\	~不足{ふそく}[欠乏{けつぼう}]しているものレベル2、発音
\\	、カナラック、変化《動》
\\	もう一つ付け加えます。	変化形 
\\	【自動】《数学》加算{かさん}する、足し算をする〔ある特質{とくしつ}を〕増大{ぞうだい}する、強める【他動】《数学》〔数字{すうじ}を〕足す、合計{ごうけい}する、足し算をする〔物を別の物に〕加える、付け足す、結び付ける〔ある特質{とくしつ}を〕加味{かみ}する、もたらす〔言葉{ことば}を〕言い足す、書き加える ・
\\	もう一つ付け加えます。レベル2、発音
\\	、カナアッド、変化《動》
\\	日本と中国の貿易は増加している。 ・
\\	20名の職員を40名に増員しました。 ・
\\	喫煙は、肺がんにな危険性を高める。	変化形 
\\	【自動】〔数量などが〕増える、増加{ぞうか}[増大{ぞうだい}・増殖{ぞうしょく}]する ・
\\	日本と中国の貿易は増加している。〔程度・可能性などが〕高まる、向上{こうじょう}[上昇{じょうしょう}・増大{ぞうだい}]する〔範囲などが〕拡大{かくだい}する、大きくなる【他動】〔数量などを〕増やす、増加{ぞうか}[増大{ぞうだい}・増殖{ぞうしょく}]させる ・
\\	20名の職員を40名に増員しました。〔程度・可能性などを〕高める、向上{こうじょう}[上昇{じょうしょう}・増大{ぞうだい}]させる ・
\\	喫煙は、肺がんにな危険性を高める。〔範囲などを〕拡大{かくだい}する、大きくする【名】〔数量などの〕増加{ぞうか}、増大{ぞうだい}、増殖{ぞうしょく}〔程度・可能性などの〕向上{こうじょう}、上昇{じょうしょう}〔範囲などの〕拡大{かくだい}伸び率レベル3、発音《名》
\\	《動》
\\	、カナインクリース、変化《動》
\\	、分節
\\	変化形 
\\	【自動】〔数量などが〕減る、減少{げんしょう}[低下{ていか}]する〔程度・可能性などが〕低まる、低下{ていか}する〔範囲などが〕縮小{しゅくしょう}【他動】〔数量などを〕減らす、減少{げんしょう}[低下{ていか}]させる〔程度・可能性などを〕低める、低下{ていか}させる〔範囲などを〕縮小{しゅくしょう}する【名】〔数量などの〕減少{げんしょう}、低下{ていか}〔程度・可能性などの〕低下{ていか}〔範囲などの〕縮小{しゅくしょう}レベル3、発音《名》
\\	《動》
\\	、カナディークリース、ディクリース、変化《動》
\\	、分節
\\	この製品の販売コストを下げることは可能でしょうか。	変化形 
\\	【自動】〔大きさ・量・程度{ていど}などが〕減る、縮小{しゅくしょう}する〔スープなどが〕煮詰まる《化学》還元{かんげん}する《生物》減数分裂{げんすう ぶんれつ}する【他動】〔大きさ・量・程度{ていど}などを〕少なくする、減らす、縮小{しゅくしょう}する〔人やものの状態{じょうたい}を〕悪化{あっか}させる、弱める~を征服{せいふく}する、~を支配{しはい}する、~を従わせる~を破壊{はかい}する、~を取り壊す~を粉末{ふんまつ}[粉状{ふんじょう}]にする〔ペンキなどを溶剤{ようざい}で〕薄める、希釈{きしゃく}する〔スープなどを〕煮詰める{につめる}、煮て濃くする〔販売価格{はんばい かかく}を〕下げる、値引きする ・
\\	この製品の販売コストを下げることは可能でしょうか。〔罰として人を〕降格{こうかく}する〔要約{ようやく}して〕~を単純化{たんじゅんか}する[まとめる]〔説明{せつめい}のために〕~を分類{ぶんるい}[分析{ぶんせき}]する《数学》通分{つうぶん}する、約分{やくぶん}する、縮尺{しゅくしゃく}する《化学》〔電子{でんし}を付加{ふか}して原子{げんし}を〕還元{かんげん}する《化学》〔酸素{さんそ}を奪って化合物{かごうぶつ}を〕還元{かんげん}する《化学》〔水素{すいそ}を付加{ふか}して化合物{かごうぶつ}を〕還元{かんげん}する〔冶金{やきん}で〕~を精錬{せいれん}する〔写真{しゃしん}のネガを〕減力{げんりょく}する◆明るい部分の濃度を下げること。《生物》〔細胞{さいぼう}を〕減数分裂{げんすう ぶんれつ}させる《医》〔けがをした部位{ぶい}を〕整復{せいふく}する、回復{かいふく}させるレベル3、発音
\\	、カナリジュース、レジュース、レデュース、リデュース、変化《動》
\\	、分節
\\	変化形 
\\	【他動】〔形あるものを〕損なう、駄目{だめ}にする、完全{かんぜん}に使えなくする〔建物{たてもの}や都市{とし}などを〕破壊{はかい}する、粉砕{ふんさい}する、荒廃{こうはい}させる〔可能性{かのう せい}や計画{けいかく}などを〕打ち砕く、廃止{はいし}する、終わらせる〔動物{どうぶつ}などを〕殺す◆【用法】通例、受け身で用いられる。〔戦いで敵を〕粉砕{ふんさい}する、打ち破る〔証言{しょうげん}などの信用性{しんよう せい}を〕失わせる、台無{だいな}しにするレベル2、発音
\\	、カナディストゥロイ、ディストロイ、変化《動》
\\	、分節
\\	彼は、
\\	からプレッシャーをかけられて、自分の友人を密告しました。 ・
\\	契約が取り消されたという通知がありました。 ・
\\	この件に関しては決定次第すぐにお知らせ致します。
\\	は 
\\	""よりも丁寧。
\\	""は偉そうに聞こえる。 ・
\\	彼女の文体[ライティングスタイル]はいつも深い情熱で満たされています。
\\	変化形 
\\	"【自動】〔人に〕情報{じょうほう}[知識{ちしき}]を与える、情報{じょうほう}を提供{ていきょう}する密告{みっこく}する、告発{こくはつ}する、告訴{こくそ}する◆【用法】
\\	彼は、
\\	からプレッシャーをかけられて、自分の友人を密告しました。【他動】〔人に〕情報{じょうほう}を与える、知らせる、通知{つうち}[報告{ほうこく}・連絡{れんらく}]する、告げる ・
\\	契約が取り消されたという通知がありました。 ・
\\	この件に関しては決定次第すぐにお知らせ致します。
\\	は 
\\	""よりも丁寧。
\\	""は偉そうに聞こえる。〔ある感情{かんじょう}・特質{とくしつ}・特徴{とくちょう}で〕~を満たす ・
\\	彼女の文体[ライティングスタイル]はいつも深い情熱で満たされています。〔作品{さくひん}などを〕特徴付けるレベル2、発音
\\	、カナインフォーム、変化《動》
\\	、分節
\\	ご紹介したい(人)がいます。 ・
\\	競合他社を当社顧客に紹介してはならない。 ・
\\	そのシステム導入の動きがある。 ・
\\	本プログラムはこのコンセプトを最初に取り入れた製品である。 ・
\\	共通の友人が私たちを引き合わせてくれました。/私たちは共通の友人を通じて出会いました。	変化形 
\\	【他動】〔~を…に〕紹介{しょうかい}する、引き合わせる ・
\\	ご紹介したい(人)がいます。 ・
\\	競合他社を当社顧客に紹介してはならない。〔~を…に〕案内{あんない}する〔~を…に〕導入{どうにゅう}する〔アイデアなどを〕取り入れる ・
\\	そのシステム導入の動きがある。 ・
\\	本プログラムはこのコンセプトを最初に取り入れた製品である。〔新製品などを〕発表{はっぴょう}する〔ウイルスを〕取り込む〔複数{ふくすう}の人を〕互いに知り合いにする、出会わせる ・
\\	共通の友人が私たちを引き合わせてくれました。/私たちは共通の友人を通じて出会いました。レベル2、発音
\\	、カナイントゥロデュース、イントロジュース、変化《動》
\\	、分節
\\	この国では技術が急速に発展しています。 ・
\\	私たちはその製品を販売する新しい計画を立てました。 ・
\\	彼は通信システムを改良するための新しい装置を開発しました。 ・
\\	デジタルカメラが発明される前は、フィルムを写真屋で現像してもらわなければなりませんでした。	変化形 
\\	【自動】〔生物{せいぶつ}や能力{のうりょく}が〕成長{せいちょう}する、発育{はついく}する〔組織{そしき}や物事{ものごと}などが〕発達{はったつ}する、発展{はってん}する ・
\\	この国では技術が急速に発展しています。〔問題{もんだい}などが〕生じる、高じる〔人が性的{せいてき}に〕成熟{せいじゅく}する、大人{おとな}になる〔事実{じじつ}などが〕明らかになる、表面化{ひょうめんか}する《生物》〔昆虫{こんちゅう}などが〕変態{へんたい}する《生物》〔生物{せいぶつ}が〕進化{しんか}する【他動】〔生物{せいぶつ}や能力{のうりょく}などを〕成長{せいちょう}させる、発育{はついく}させる〔組織{そしき}や物事{ものごと}などを〕発達{はったつ}させる、発展{はってん}させる〔新しいものを〕開発{かいはつ}する、開拓{かいたく}する〔計画・考えなどを〕進展{しんてん}[展開{てんかい}]させる ・
\\	私たちはその製品を販売する新しい計画を立てました。 ・
\\	彼は通信システムを改良するための新しい装置を開発しました。〔傾向{けいこう}・資質{ししつ}などを〕発現{はつげん}させる〔癖が〕つく◆【参考】
\\	〔病気を〕患う、発症する〔フィルムを〕現像{げんぞう}する ・
\\	デジタルカメラが発明される前は、フィルムを写真屋で現像してもらわなければなりませんでした。レベル2、発音
\\	、カナディベロップ、デベロップ、変化《動》
\\	、分節
\\	バンク選択回路はバンク選択信号を作り出す。	変化形 
\\	【自動】生み出す、創作{そうさく}する〔土地{とち}が〕産出{さんしゅつ}する、〔農作物{のうさくぶつ}が〕実る製品{せいひん}を生産{せいさん}する、経済的価値{けいざい てき かち}を生む【他動】〔新しいものを〕作り出す、生み出す◆【派生語】
\\	(生産)、
\\	(生産性)、
\\	(生産的な)など ・
\\	バンク選択回路はバンク選択信号を作り出す。〔鉱物{こうぶつ}や農作物{のうさくぶつ}を〕産出{さんしゅつ}する、生産{せいさん}する〔製品{せいひん}を〕作る、製造{せいぞう}する〔精神活動{せいしん かつどう}で作品{さくひん}を〕作る、創作{そうさく}する〔結果{けっか}などを〕引き起こす、生じさせる〔物を〕取り出す、提示{ていじ}する〔映画{えいが}や演劇{えんげき}などを〕製作{せいさく}する、プロデュースする【名】〔野菜{やさい}や果物{くだもの}の〕農産物{のうさんぶつ}レベル2、発音《名》
\\	《動》
\\	、カナプロジュース、プロデュース、変化《動》
\\	、分節
\\	一つのピクセルは4種類のプロセスカラーから作られる。	変化形 
\\	【自動】創造{そうぞう}する、作り出す、建設的{けんせつてき}なことを行う〈英俗〉騒ぎ立てる【他動】〔神が〕~を創造{そうぞう}する、〔自然{しぜん}の力で〕作り出す~を作る、(全く新しいもの・製品{せいひん}・商品{しょうひん}を)開発{かいはつ}する[新開発{しん かいはつ}する・生み出す・創り出す・創出{そうしゅつ}する・世に送り出す] ・
\\	一つのピクセルは4種類のプロセスカラーから作られる。〔物・事が〕~を引き起こす〔俳優{はいゆう}が〕~を初演{しょえん}する〔人を職・地位{ちい}に〕就ける〔人に地位{ちい}などを〕与える、付与{ふよ}する、授ける〔部署{ぶしょ}などを〕新たに設置{せっち}するレベル2、発音
\\	、カナクリエイト、変化《動》
\\	、分節
\\	お支払いが遅れる場合は金利を上乗せした額をお支払いいただきます。	変化形 
\\	【他動】~に興味{きょうみ}を持たせる、~に興味{きょうみ}を起こさせる、~の興味{きょうみ}を引く~に関係{かんけい}させる、~に関与{かんよ}させる【名】興味{きょうみ}、関心{かんしん}、趣味{しゅみ}利子{りし}、利息{りそく}、金利{きんり} ・
\\	お支払いが遅れる場合は金利を上乗せした額をお支払いいただきます。利益{りえき}、利害{りがい}(関係{かんけい})◆通例、
\\	〔利害関係が共通する〕同業者{どうぎょうしゃ}所有権{しょゆうけん}、株式{かぶしき}影響力{えいきょうりょく}、勢力{せいりょく}レベル1、発音
\\	、カナインタレスト、変化《動》
\\	、分節
\\	どうすれば君の気に入るんだい?	変化形 
\\	【自動】要求{ようきゅう}を満たす、希望{きぼう}をかなえる【他動】(人)を満足{まんぞく}させる、(人)の希望{きぼう}をかなえる ・
\\	どうすれば君の気に入るんだい?〔要求{ようきゅう}や希望{きぼう}を〕満たす、かなえる〔人に真実{しんじつ}などを〕納得{なっとく}させる〔疑いや疑問{ぎもん}を〕晴らす、一掃{いっそう}する〔借金{しゃっきん}や義務{ぎむ}を〕支払う、果たす〔損害{そんがい}や損失{そんしつ}を〕償う、埋め合わせる〔規則{きそく}や基準{きじゅん}を〕満たす、満足{まんぞく}させる《数学》〔方程式{ほうていしき}を〕満たす、成立{せいりつ}させるレベル3、発音
\\	、カナサティスファイ、サテスファイ、変化《動》
\\	、分節
\\	あなたにそんなことができたなんて、今考えても驚きです。 ・
\\	彼は、それを聞いても驚きませんでした。 ・
\\	「驚くようなもの[こと・プレゼント]が欲しいでしょ?」「もちろん!」 ・
\\	驚いた。 ・
\\	君が結婚すると聞いて、いい意味で驚いたよ。 ・
\\	これは驚きだな。 ・
\\	あらびっくりした。◆【場面】思いがけない人から電話がかかって来たとき。 ・
\\	驚いたなあ! ・
\\	びっくりさせたいから彼女には秘密にしておいてくださいね。
\\	変化形 
\\	"【他動】~を驚かす、~をあっと言わせる ・
\\	あなたにそんなことができたなんて、今考えても驚きです。 ・
\\	彼は、それを聞いても驚きませんでした。【名】驚き、予期{よき}しない驚き、驚くべきこと[もの]、意外{いがい}なこと、奇襲{きしゅう} ・
\\	「驚くようなもの[こと・プレゼント]が欲しいでしょ?」「もちろん!」 ・
\\	驚いた。 ・
\\	君が結婚すると聞いて、いい意味で驚いたよ。 ・
\\	これは驚きだな。 ・
\\	あらびっくりした。◆【場面】思いがけない人から電話がかかって来たとき。 ・
\\	驚いたなあ!〔うれしい驚きなどの効果{こうか}を狙って〕相手{あいて}に内緒{ないしょ}で企画{きかく}するイベント、サプライズ・パーティー
\\	びっくりさせたいから彼女には秘密にしておいてくださいね。【形】突然{とつぜん}の、出し抜けの、抜き打ちのレベル1、発音
\\	、カナサープライズ、サプライズ、変化《動》
\\	、分節
\\	価格の違いには驚かされました。 ・
\\	それには我ながらあきれた。 ・
\\	自分の頑固さに我ながらあきれた。	変化形 
\\	【自動】びっくりさせる【他動】~を仰天{ぎょうてん}させる、感心{かんしん}させる、びっくりさせる、驚嘆{きょうたん}させる ・
\\	価格の違いには驚かされました。~をあきれさせる、当惑{とうわく}させる ・
\\	それには我ながらあきれた。 ・
\\	自分の頑固さに我ながらあきれた。【名】= 
\\	レベル4、発音
\\	、カナアメイズ、変化《動》
\\	、分節
\\	変化形 
\\	【1自動】好印象{こう いんしょう}を与える◆【用法】受け身で用いられることが多い。【1他動】(人)に強い影響{えいきょう}を与える、(人)を感動{かんどう}させる(人)に好[良い]印象{いんしょう}を与える(人)にはっきりと理解{りかい}[記憶{きおく}]させる〔図案{ずあん}や型を〕押し付ける、刻印{こくいん}する《電気》〔回路{かいろ}などに〕電圧{でんあつ}を加える【1名】〔人に〕強い印象{いんしょう}を与えること、〔物を〕押し付けること〈文〉〔押してできた〕型、跡〈文〉押印{おういん}、捺印{なついん}【2他動】〔財産{ざいさん}などを公用{こうよう}のために〕徴発{ちょうはつ}する〈古〉〔人を恣意的{しい てき}に〕軍隊{ぐんたい}に徴用{ちょうよう}する、徴兵{ちょうへい}する【2名】〔人や財産{ざいさん}の〕徴発{ちょうはつ}、徴用{ちょうよう}◆【同】
\\	レベル3、発音《名》
\\	《動》
\\	、カナインプレス、変化《動》
\\	、分節
\\	競争によって、その企業は製品の品質改良に関する考え方が変わった。 ・
\\	悪天候がその野球選手のプレーに影響を与えました。 ・
\\	彼の言葉は私の心に深く影響を与えました。	変化形 
\\	【1他動】~に作用{さよう}する、~に影響{えいきょう}を及ぼす、~に響く ・
\\	競争によって、その企業は製品の品質改良に関する考え方が変わった。 ・
\\	悪天候がその野球選手のプレーに影響を与えました。(人)の心に影響を与える、(人)の心を動かす、(人)を動揺{どうよう}[感動{かんどう}]させる、(人)の感情{かんじょう}をかき立てる◆【類】
\\	彼の言葉は私の心に深く影響を与えました。〔病気{びょうき}が人を〕侵す、襲う【1名】《心理学》〔表情や身振りに表れる〕感情{かんじょう}、情緒{じょうちょ}《精神医学》情動{じょうどう}【2他動】〔変わった衣服を〕好んで身に着ける、〔変わったものを〕好んで用いる[選ぶ]、~に愛着{あいちゃく}を感じる~するふりをする、~のふりをする◆【類】
\\	〔動植物{どうしょくぶつ}がある場所{ばしょ}〕に住む、〔同じ場所{ばしょ}〕にいつもいる〔物がある形を〕とるレベル3、発音《動》
\\	、カナアフェクト、変化《動》
\\	、分節
\\	私は歩けるが、すぐに疲れてしまう。 ・
\\	長旅で疲れたよ。	変化形 
\\	【1名】〈米〉〔自動車{じどうしゃ}の車輪{しゃりん}の〕タイヤ◆車輪の外縁のゴム製の輪と圧縮空気の入ったチューブから構成される。〈米〉〔乳母車{うばぐるま}などの車輪{しゃりん}の〕タイヤ◆車輪の外縁のゴム製のベルト。〈米〉〔馬車{ばしゃ}などの車輪{しゃりん}の〕輪金{わがね}◆車輪の外縁の金属製の輪。【2自動】疲れる ・
\\	私は歩けるが、すぐに疲れてしまう。飽きる【2他動】~を疲れさせる ・
\\	長旅で疲れたよ。~を飽きさせる【3他動】〈文〉~を着飾る{きかざる}【3名】〈古〉衣装{いしょう}、正装{せいそう}〈古〉〔女性{じょせい}の〕頭飾りレベル2、発音
\\	、カナタイアー、変化《動》
\\	自分の悩み事で人をうんざりさせてはいけないよ。 ・
\\	あ、ごめん。こんな話、退屈よね。 ・
\\	彼は私が今まで会った人の中で一番退屈な人です。 ・
\\	その番組[ショー]はものすごく退屈でした。	変化形 
\\	【1他動】~をうんざりさせる、退屈{たいくつ}させる ・
\\	自分の悩み事で人をうんざりさせてはいけないよ。 ・
\\	あ、ごめん。こんな話、退屈よね。【1名】うんざりさせる人[物]、退屈{たいくつ}な人[こと] ・
\\	彼は私が今まで会った人の中で一番退屈な人です。 ・
\\	その番組[ショー]はものすごく退屈でした。【2自動】穴があく押し分けて進む、ジリジリ進む【2他動】~を突き通す、掘る、くりぬく、えぐる〔人混{ひとご}みの中に穴を掘るかのごとく〕~を押しのけて進む【2名】穴{あな}〔穴などの〕内径〔銃・管などの〕口径{こうけい}穴あけ器、穿孔機{せんこうき}、掘削機{くっさくき}ボア◆管楽器の内径およびその形状〈豪〉掘抜き井戸{ほりぬき いど}、〔家畜用{かちく よう}の〕水飲み池【3名】潮津波、ボア【4他動】
\\	の過去形{かこ けい}レベル3、発音
\\	、カナボーア、ボア、変化《動》
\\	思い過ごしですよ。 ・
\\	あなたに何か起こったのではないかと心配しました。 ・
\\	ひどくお疲れのご様子で心配です。◆この
\\	は驚きなどの感情を表す ・
\\	彼女は、緊張すると唇をかむ。	変化形 
\\	【自動】心配{しんぱい}する、気をもむ、気になる、気掛かりである、気に病む、くよくよする ・
\\	思い過ごしですよ。苦労{くろう}して進む〔口でくわえて何度{なんど}も〕引く[ゆする・ちぎる]◆【語源】もともと「窒息させる」という意味で、その後オオカミなどがのどにかみついて獲物を殺す様を表すようになり、徐々に「相手を攻撃する」「悩ます」という意味をも持つようになった。【他動】~を心配{しんぱい}させる、悩ます、イライラさせる、~の気をもませる、~に心配{しんぱい}[苦労{くろう}]を掛ける ・
\\	あなたに何か起こったのではないかと心配しました。 ・
\\	ひどくお疲れのご様子で心配です。◆この
\\	は驚きなどの感情を表すうるさく攻めたてる、しつように攻撃{こうげき}する〔不安{ふあん}になって・緊張{きんちょう}して唇を〕かむ ・
\\	彼女は、緊張すると唇をかむ。〔犬などが口でくわえて何度{なんど}も〕引く、ゆする、ちぎる、振り回す、もてあそぶ【名】不安{ふあん}、心配{しんぱい}、気掛かり、気苦労{きぐろう}、懸念{けねん}、憂慮{ゆうりょ}悩み[心配{しんぱい}]の種動物{どうぶつ}[猟犬{りょうけん}]が獲物{えもの}をくわえて振りまわすことレベル1、発音
\\	、カナウォーリー、ワーリー、変化《動》
\\	、分節
\\	構わないでください。 ・
\\	なぜ私がわざわざそんなことをしなければいけないの。/私がわざわざそんな面倒なこと、するわけないでしょう。 ・
\\	ボブ、なんでしけた面してんだ?ちょっと失敗したからっていちいち悩んでたらきりがないんだよ。 ・
\\	それで私が困ることはない。 ・
\\	少々気掛かりなことがあります。 ・
\\	何で困っているのですか?◆【直訳】何があなたを困らせているのですか? ・
\\	そんなこと気にしちゃ駄目ですよ。 ・
\\	ちょっといいかい。 ・
\\	ご面倒を掛けて申し訳ないことでした。 ・
\\	警察のご厄介になるようなことは何もしていない。 ・
\\	もう迷惑は掛けません。 ・
\\	面倒くさい。	変化形 
\\	【自動】思い悩む、気にする、苦にする、心配{しんぱい}する、てこずる、たじたじする ・
\\	構わないでください。わざわざ
\\	する ・
\\	なぜ私がわざわざそんなことをしなければいけないの。/私がわざわざそんな面倒なこと、するわけないでしょう。【他動】悩ます、悩ませる、困らせる、困惑{こんわく}させる、狼狽{ろうばい}させる、~に嫌な思いをさせる、(人)にとって〔主語〕が気になる〔悪い意味{いみ}で〕、(人)を落ち着かない気分{きぶん}にする ・
\\	ボブ、なんでしけた面してんだ?ちょっと失敗したからっていちいち悩んでたらきりがないんだよ。 ・
\\	それで私が困ることはない。 ・
\\	少々気掛かりなことがあります。 ・
\\	何で困っているのですか?◆【直訳】何があなたを困らせているのですか? ・
\\	そんなこと気にしちゃ駄目ですよ。~に迷惑{めいわく}[面倒{めんどう}・手数{てすう}]を掛ける、~の邪魔{じゃま}をする、~の手を焼かせる、~にちょっかいを出す ・
\\	ちょっといいかい。 ・
\\	ご面倒を掛けて申し訳ないことでした。 ・
\\	警察のご厄介になるようなことは何もしていない。 ・
\\	もう迷惑は掛けません。【名】困ったこと、面倒{めんどう}、騒ぎ、悩みの種、厄介(者){やっかい(しゃ)} ・
\\	面倒くさい。
\\	手数がかかるレベル4、発音
\\	、カナボザー、バザー、変化《動》
\\	、分節
\\	がっかりさせないでよ。 ・
\\	申し訳ありませんが、期待に沿えそうにありません。	変化形 
\\	【自動】〔期待通りでなくて〕失望{しつぼう}[がっかり]させる【他動】(人)を失望{しつぼう}させる、(人)の期待{きたい}を裏切{うらぎ}る ・
\\	がっかりさせないでよ。 ・
\\	申し訳ありませんが、期待に沿えそうにありません。レベル3、発音
\\	、カナディスアポイント、デサポイント、デスアポイント、変化《動》
\\	、分節
\\	自分のことが大嫌いだった。 ・
\\	ボブほど嫌われている人物はいない。 ・
\\	待つのは嫌です。 ・
\\	太ってるのは嫌です。 ・
\\	学校へ行くのは嫌です。 ・
\\	あなたとは言い争いたくない。 ・
\\	税金を払うのは嫌です。 ・
\\	人前で話をするのは嫌です。 ・
\\	こんな知らせを持ってきたくなかったのですが。 ・
\\	こんなことは言いたくないのですが。 ・
\\	憎悪は愛に勝る。 ・
\\	一番嫌いなものはベトベトするキーボードだ。	変化形 
\\	【自動】憎む、ひどく嫌う[嫌がる] ・
\\	自分のことが大嫌いだった。【他動】~を憎む、〔怒りや憎しみを覚えるほど〕~をひどく嫌う ・
\\	ボブほど嫌われている人物はいない。〔~することを〕嫌がる、〔~するのは〕いやだ、~したがらない ・
\\	待つのは嫌です。 ・
\\	太ってるのは嫌です。 ・
\\	学校へ行くのは嫌です。 ・
\\	あなたとは言い争いたくない。 ・
\\	税金を払うのは嫌です。 ・
\\	人前で話をするのは嫌です。〔~するのを〕残念{ざんねん}に思う ・
\\	こんな知らせを持ってきたくなかったのですが。 ・
\\	こんなことは言いたくないのですが。【名】憎悪{ぞうお}、憎しみ、嫌悪{けんお} ・
\\	憎悪は愛に勝る。憎悪{ぞうお}[嫌悪{けんお}]の対象{たいしょう}、嫌いな[嫌な]もの ・
\\	一番嫌いなものはベトベトするキーボードだ。レベル2、発音
\\	、カナヘイト、変化《動》
\\	弾丸はもう少しで私の頭に当たるところだった。 ・
\\	彼のことは一生忘れないだろう。 ・
\\	「(あなたがいないと)寂しくなります」「私もです」◆長期間会えなくなるときの別れの挨拶{あいさつ} ・
\\	日本の物で何が一番恋しいですか。◆国外で暮らす日本の人などに。 ・
\\	あなたがそばにいないので寂しい。 ・
\\	家族と夕食を共にできないのが寂しい。 ・
\\	あなたがいなくて寂しく思っています。/会いたかったよ。◆【場面】別離の状態、またはその後の再会。 ・
\\	この特売(の機会)を逃すな! ・
\\	今年はまだ一度も授業に欠席していない。 ・
\\	見逃すことはないですよ。/すぐに分かりますよ。◆道案内 ・
\\	的に近くても当たらぬは遠く外れたと同じ。/いかに成功に近くても失敗は失敗。/少しでも失敗は失敗。/失敗したことには変わりない。/五十歩百歩。◆【直訳】1インチの失敗も1マイルの失敗と同じ。◆ことわざ
\\	変化形 
\\	"【1自動】的を外す、当て[打ち]損なう失敗{しっぱい}する、不首尾に終わる〔エンジンが〕着火{ちゃっか}[点火{てんか}]しない【1他動】~に当たらない、~に届かない、~と擦れ違う ・
\\	弾丸はもう少しで私の頭に当たるところだった。~に失敗{しっぱい}する、~が不首尾{ふしゅび}に終わる~がなくて不自由{ふじゆう}する、~がない[いない]のを寂しく思う、~が恋しい ・
\\	彼のことは一生忘れないだろう。 ・
\\	「(あなたがいないと)寂しくなります」「私もです」◆長期間会えなくなるときの別れの挨拶{あいさつ} ・
\\	日本の物で何が一番恋しいですか。◆国外で暮らす日本の人などに。 ・
\\	あなたがそばにいないので寂しい。 ・
\\	家族と夕食を共にできないのが寂しい。 ・
\\	あなたがいなくて寂しく思っています。/会いたかったよ。◆【場面】別離の状態、またはその後の再会。〔必要{ひつよう}なときに~が〕ないことに気付く、いないことが分かる〔機会{きかい}などを〕逃す、見逃す{みのがす} ・
\\	この特売(の機会)を逃すな!〔会議{かいぎ}などに〕欠席{けっせき}する、〔仕事{しごと}などを〕休む、〔乗り物などに〕乗り損なう ・
\\	今年はまだ一度も授業に欠席していない。〔不注意{ふちゅうい}などで〕~を見落{みお}とす、~を聞き漏{も}らす、~が頭に入らない ・
\\	見逃すことはないですよ。/すぐに分かりますよ。◆道案内~から逃れる、~を免れる~を除外{じょがい}する、~をもらす【1名】打ち[当て]損ない、擦れ違い ・
\\	的に近くても当たらぬは遠く外れたと同じ。/いかに成功に近くても失敗は失敗。/少しでも失敗は失敗。/失敗したことには変わりない。/五十歩百歩。◆【直訳】1インチの失敗も1マイルの失敗と同じ。◆ことわざ失敗{しっぱい}、不成功{ふせいこう}〔エンジンの〕ミスファイア(リング)、不点火{ふ てんか}【2名】
\\	~嬢、~さん◆独身女性の名字の前に付ける敬称
\\	《呼び掛け》お嬢さん〔若い〕独身{どくしん}の女性{じょせい}
\\	ミス~◆コンテストなどの、若い女性の勝者に付ける敬称。
\\	~の女性{じょせい}◆性格などを表す語の前に付けて、ある若い女性を指す言葉を作るもの。◆【参考】
\\	〔洋服{ようふく}のサイズの〕ミス◆平均的な女性の、洋服のサイズ。レベル1、発音
\\	、変化《複》
\\	、《動》
\\	誰にも過ちはある。/猿も木から落ちる。	変化形 
\\	【自動】誤る、間違える{まちがえる}【他動】〔言葉{ことば}や行動{こうどう}などの真意{しんい}〕を誤解{ごかい}する〔他の人やものと〕~を思い違いする〔うっかりと時間{じかん}や場所{ばしょ}などを〕間違える【名】ミス、誤り、過ち ・
\\	誰にも過ちはある。/猿も木から落ちる。思い違い、誤解{ごかい}レベル1、発音
\\	、カナミステイク、変化《動》
\\	、分節
\\	その老コメディアンのユーモアは必ず受けた。 ・
\\	最近私の視力は衰えてきた。 ・
\\	トムは数学のテストで落第点を取った。 ・
\\	彼は入学[入社]試験に落ちた。	変化形 
\\	【自動】失敗{しっぱい}する、しくじる ・
\\	その老コメディアンのユーモアは必ず受けた。〔企業などが〕破産{はさん}[倒産{とうさん}・破綻{はたん}]する〔テストに〕落第{らくだい}する、落ちる、落第点{らくだい てん}を取る機能{きのう}しなくなる、役に立たなくなる ・
\\	最近私の視力は衰えてきた。欠乏{けつぼう}する、不足{ふそく}する【他動】〔人の期待{きたい}などを〕裏切{うらぎ}る、(人)を失望{しつぼう}させる、~の役に立たない~を不合格{ふごうかく}にする、~で落第点{らくだい てん}を取る ・
\\	トムは数学のテストで落第点を取った。 ・
\\	彼は入学[入社]試験に落ちた。~を見捨{みす}てる、~をあきらめる〔期待{きたい}された義務{ぎむ}などを〕怠る、履行{りこう}しないレベル2、発音
\\	、カナフェイル、変化《動》
\\	彼は娘に前進するように励ました。 ・
\\	彼女は、私が勉強して医者になりたいと言ったら励ましてくれた。 ・
\\	私はあなたにこの仕事に応募するようお勧めします。 ・
\\	彼はオーディションに合格できなかったが、私は彼にあきらめないように励ました。 ・
\\	私たちの提案を受け入れてくださるようお願いします。	変化形 
\\	【他動】勇気{ゆうき}づける、励ます、~に自信{じしん}[希望{きぼう}]を与える ・
\\	彼は娘に前進するように励ました。 ・
\\	彼女は、私が勉強して医者になりたいと言ったら励ましてくれた。 ・
\\	私はあなたにこの仕事に応募するようお勧めします。 ・
\\	彼はオーディションに合格できなかったが、私は彼にあきらめないように励ました。〔~するように〕勧める、奨励{しょうれい}する、働きかける、けしかける、仕向ける、促す、振興{しんこう}する ・
\\	私たちの提案を受け入れてくださるようお願いします。〔発育{はついく}・発達{はったつ}などを〕促す、促進{そくしん}する、助長{じょちょう}する、後押{あとお}しするレベル3、発音
\\	、カナインカリッジ、インカレッジ、エンカレッジ、変化《動》
\\	、分節
\\	変化形 
\\	【自動】支援{しえん}する、(~に)力を貸す、(~を)手伝う{てつだう}、(~の)助けとなる、役に立つ〔会・式などに〕列席{れっせき}する《野球》補殺{ほさつ}する【他動】〔補助的{ほじょ てき}に人を〕手伝う{てつだう}、手助けする、力を貸す【名】援助{えんじょ}、助力{じょりょく}《野球》補殺{ほさつ}◆【略】
\\	《サッカー》アシスト◆ゴールを決めるために味方が送るパスレベル2、発音
\\	、カナアシスト、変化《動》
\\	、分節
\\	《名言》避けられない運命には従え。	変化形 
\\	【自動】〔人と共通の目的を達成するために〕協力{きょうりょく}[協働{きょうどう}]する ・
\\	《名言》避けられない運命には従え。レベル6、発音
\\	、カナコーオペレイト、コウアペレイト、コウオペレイト、変化《動》
\\	、分節
\\	この素晴{すば}らしい椅子{いす}[チェア]が私の背中を快適に支えます。 ・
\\	そのファンたちは、そのチームを応援し続けています。	変化形 
\\	【他動】〔不安定な物などを〕支える、支持{しじ}する ・
\\	この素晴{すば}らしい椅子{いす}[チェア]が私の背中を快適に支えます。〔人を〕支える、支援{しえん}[援助{えんじょ}・サポート]する、後押{あとお}しする、下支えする~を応援{おうえん}する ・
\\	そのファンたちは、そのチームを応援し続けています。~を養う、扶養{ふよう}する《法律》〔陳述{ちんじゅつ}などを〕立証{りっしょう}する、裏付ける{うらづける}〔態度{たいど}を〕貫く、変えない〔部材{ぶざい}を〕かつぐように持つ〔部材{ぶざい}を担体{たんたい}に〕支持{しじ}する、担持{たんじ}する【名】〔倒れないようにする〕支持物{しじ ぶつ}〔物を固定{こてい}する〕補助物{ほじょ ぶつ}〔人や組織{そしき}などへの〕支援{しえん}、援助{えんじょ}、サポート〔家庭{かてい}の〕生活費{せいかつひ}、必要経費{ひつよう けいひ}〔人や組織{そしき}の〕後援者{こうえんしゃ}、支持者{しじしゃ}、援助者{えんじょしゃ}〔公演{こうえん}の〕助演者{じょえん しゃ}、脇役{わきやく}レベル2、発音
\\	、カナサポート、変化《動》
\\	、分節
\\	変化形 
\\	【自動】助力{じょりょく}する、助けとなる【他動】~を援助{えんじょ}[補助{ほじょ}・支援{しえん}]する【名】援助{えんじょ}、救済{きゅうさい}、補助{ほじょ}、助成{じょせい}、補佐{ほさ}助力者{じょりょく しゃ}、助手{じょしゅ}補助{ほじょ}器具{きぐ}[教材{きょうざい}]= 
\\	補助{ほじょ}[助成{じょせい}]金《軍事》副官{ふくかん}◆【同】
\\	〔臣下{しんか}の封建領主{ほうけん りょうしゅ}への〕上納金{じょうのう きん}〔イギリス国王{こくおう}への〕援助金{えんじょ きん}◆議会が与える、国王の特別出費に対する補助金で、ノルマン征服から18世紀まで行われた。レベル3、発音
\\	、カナエイド、変化《動》
\\	ごめんなさい。起こしちゃいました?	変化形 
\\	【1自動】目が覚める、眼を覚ます起きている、寝ないでいる〔危険{きけん}などに〕気付く、〔状況{じょうきょう}などを〕自覚{じかく}する〔アイルランドで〕通夜{つや}をする【1他動】(人)の目を覚まさせる、(人)を起こす ・
\\	ごめんなさい。起こしちゃいました?〔活動{かつどう}していなかったものを〕目覚めさせる、活動{かつどう}させる〔注意{ちゅうい}などを人に〕喚起{かんき}する、気付{きづ}かせる〔アイルランドで死者{ししゃ}の〕通夜{つや}をする【1名】〔見張りのための〕徹夜{てつや}、寝ずの番〔アイルランドの〕通夜{つや}、死者{ししゃ}のための集まり◆埋葬前の夜に集まり、死者の思い出を語りながら楽しく飲み明かす伝統的な儀式。死者の家で行われていたが、葬儀場で行われることも多くなった。
\\	〈英〉〔教会{きょうかい}の〕守護聖人{しゅご せいじん}の祭り◆教区の守護聖人を記念して年に1度行われる伝統的な行事。【2名】水面{すいめん}にできる波紋{はもん}、航跡{こうせき}、通った跡、伴流{はんりゅう}レベル1、発音
\\	、カナウェイク、変化《動》
\\	または
\\	または
\\	とても楽しかった。	変化形 
\\	【自動】笑い声を上げる、〔声を立てて〕笑う楽しく感じる、おかしいと思うばかばかしい[くだらない]と思う、笑い物にするしてやったりと思う、〔優越感{ゆうえつかん}に浸って〕喜ぶ〔笑い声に似た〕音を立てる【他動】~を一笑{いっしょう}に付す~と笑いながら言う【名】笑うこと、笑い声〈話〉笑わせる[愉快{ゆかい}な・気が晴れる]こと[もの]、陽気{ようき}な人 ・
\\	とても楽しかった。〈話〉ばかばかしいこと、笑い物{わらいもの}レベル1、発音
\\	、カナラーフ、変化《動》
\\	すごい匂{にお}いがする。 ・
\\	何かいい匂{にお}いがする。 ・
\\	私は、何かが焦げている匂いを感じます。/何か焦げ臭いよ。
\\	でもほぼ同じ。 ・
\\	私は、ここに問題が潜んでいるように感じます。	変化形 
\\	【自動】匂{にお}いがする ・
\\	すごい匂{にお}いがする。 ・
\\	何かいい匂{にお}いがする。嫌な[くさい]匂{にお}いがする感づく【他動】~の匂{にお}いを嗅ぐ ・
\\	私は、何かが焦げている匂いを感じます。/何か焦げ臭いよ。
\\	でもほぼ同じ。 ・
\\	私は、ここに問題が潜んでいるように感じます。【名】匂{にお}い嗅覚{きゅうかく}レベル1、発音
\\	、カナスメル、変化《動》
\\	または
\\	おまえが生まれたのは、つい昨日のことのように思える。 ・
\\	そうみたい。	変化形 
\\	【自動】~のように見える、~するように思われる、~と思われる、~のように思える、~のようだ、~らしい ・
\\	おまえが生まれたのは、つい昨日のことのように思える。 ・
\\	そうみたい。レベル1、発音
\\	、カナシーム、スィーム、変化《動》
\\	私はXジャーナルに掲載された広告に興味を持ちました。 ・
\\	その言葉は1970年代末に現れた。 ・
\\	実行中のプログラムやタスクの一覧を含む画面が表示されます。	変化形 
\\	【自動】〔~のように〕見える、〔~と〕思われる、〔どうやら〕~らしい現れる、出現{しゅつげん}する、登場{とうじょう}する ・
\\	私はXジャーナルに掲載された広告に興味を持ちました。 ・
\\	その言葉は1970年代末に現れた。 ・
\\	実行中のプログラムやタスクの一覧を含む画面が表示されます。出頭{しゅっとう}するレベル2、発音
\\	、カナアピア、変化《動》
\\	、分節
\\	今日あるところの
\\	社を貴殿ご自身がお作りになったということを、私はよく存じております。	変化形 
\\	【他動】〔自分{じぶん}で見て・人から聞いて〕~だと分かる、~に見覚えがある、(人)の顔を覚えている、~を見分ける{みわける}、~の見分けがつく〔業績{ぎょうせき}などを〕評価{ひょうか}する、認める ・
\\	今日あるところの
\\	社を貴殿ご自身がお作りになったということを、私はよく存じております。〔議長{ぎちょう}が〕(人)の発言{はつげん}を許可{きょか}する〔国際法{こくさいほう}で国家{こっか}を〕承認{しょうにん}する(人)を表彰{ひょうしょう}する、(人)に褒美{ほうび}を与える〔~が本当{ほんとう}であると〕認める、受け入れる《生物》〔相補的構造{そうほ てき こうぞう}の分子{ぶんし}を〕結合{けつごう}させる《コ》~を認識{にんしき}する、~を認証{にんしょう}するレベル3、発音
\\	、カナレコグナイズ、変化《動》
\\	、分節
\\	変化形 
\\	【自動】注視{ちゅうし}[凝視{ぎょうし}]する注意{ちゅうい}を払う、留意{りゅうい}する【他動】~を注視{ちゅうし}[凝視{ぎょうし}]する、~をじっと見る〔~をある性質{せいしつ}や役割{やくわり}だと〕見なす、思う、考える◆【用法】性質や役割などに当たるものを
\\	以下で表す。〔~をある感情{かんじょう}を持って〕見る、眺める、感じる◆【用法】感情などに当たるものを
\\	以下で表す。~を(高く)評価{ひょうか}する、~を尊敬{そんけい}する~を考慮{こうりょ}する、~に関心{かんしん}を示す【名】〈文〉注視{ちゅうし}、凝視{ぎょうし}〔ある特定{とくてい}の〕観点{かんてん}、見地{けんち}◆【用法】
\\	で使われる。注意{ちゅうい}、関心{かんしん}、心に留めること尊敬{そんけい}、敬意{けいい}、好感{こうかん}
\\	〔挨拶{あいさつ}の言葉{ことば}の〕敬意{けいい}の念◆【用法】
\\	の形で「~によろしくお伝えください」という意味で使われる。レベル2、発音
\\	、カナリガード、変化《動》
\\	、分節
\\	彼がそれに気付くまではすべてが順調にいっていた。 ・
\\	それは知らなかった。 ・
\\	あれ、あなた、そんな所に立ってたの。気が付かなかった。	変化形 
\\	【他動】~に気が付く、悟る、自覚{じかく}する、実感{じっかん}する、~だとつくづく思う、~を完全{かんぜん}[明確{めいかく}]に理解{りかい}する、体得{たいとく}する ・
\\	彼がそれに気付くまではすべてが順調にいっていた。 ・
\\	それは知らなかった。 ・
\\	あれ、あなた、そんな所に立ってたの。気が付かなかった。~を実現{じつげん}する〔資産を〕換金{かんきん}[現金化{げんきんか}]するレベル2、発音
\\	、カナリアライズ、変化《動》
\\	、分節
\\	変化形 
\\	【自動】気が付く、分かる【他動】〔目で見て〕~がいる[ある]ことが分かる[に気が付く]〔~だという事実{じじつ}に〕気が付く~に言及{げんきゅう}する、~について簡単{かんたん}に話す、~にふれる(人)に配慮{はいりょ}する、(人)を丁重{ていちょう}に扱う~を通知{つうち}[告知{こくち}・公示{こうじ}]する【名】〔公共{こうきょう}の場の〕掲示{けいじ}、案内{あんない}、看板{かんばん}通知(書){つうち(しょ)}、公示(書){こうじ(しょ)}◆書かれている案内や通知で、新聞や雑誌の記事の場合もある。〔事前{じぜん}の〕予告{よこく}、警告{けいこく}〔発効{はっこう}までの〕通知{つうち}[予告{よこく}]期間{きかん}〔契約{けいやく}の〕解除{かいじょ}[解約{かいやく}]通告{つうこく}[予告{よこく}](期間{きかん})解雇{かいこ}[雇用契約解除{こよう けいやく かいじょ}]通告{つうこく}[予告{よこく}](期間{きかん})注意(力){ちゅうい(りょく)}、注目{ちゅうもく}、観察{かんさつ}配慮{はいりょ}、関心{かんしん}〔本や映画{えいが}などの〕批評{ひひょう}(記事{きじ})レベル2、発音
\\	、カナノーティス、ノウティス、変化《動》
\\	、分節
\\	このことを話したら彼、どんな顔するかな。 ・
\\	自分の反応の仕方を変えることはできる。◆【用法】「相手や状況を変えることは難しいが、自分は変えられる」という意味で用いられる。	変化形 
\\	【自動】〔刺激{しげき}や言動{げんどう}などに〕反応{はんのう}する、反応{はんのう}を示す◆【用法】
\\	このことを話したら彼、どんな顔するかな。 ・
\\	自分の反応の仕方を変えることはできる。◆【用法】「相手や状況を変えることは難しいが、自分は変えられる」という意味で用いられる。〔従来{じゅうらい}の物事{ものごと}に〕反対{はんたい}する、反抗{はんこう}する◆【用法】
\\	〔力や作用{さよう}に〕相互{そうご}[反対{はんたい}の]作用{さよう}を及ぼす◆【用法】
\\	《化学》〔物質{ぶっしつ}が〕化学反応{かがく はんのう}を起こす【他動】《化学》〔物質{ぶっしつ}を〕反応{はんのう}させるレベル5、発音
\\	、カナリアクト、変化《動》
\\	、分節
\\	なかなか返事が来ない。	変化形 
\\	【自動】返答{へんとう}する、答える、返事{へんじ}をする、返信{へんしん}を出す、応じる、応酬{おうしゅう}する、反響{はんきょう}する【他動】~と返事{へんじ}をする、答える【名】返事{へんじ}、回答{かいとう} ・
\\	なかなか返事が来ない。《コ》応答{おうとう}レベル2、発音
\\	、カナリプライ、変化《動》
\\	、分節
\\	変化形 
\\	【自動】〔質問{しつもん}などに〕答える、返事{へんじ}をする《生物》〔刺激{しげき}などに〕反応{はんのう}する《医》〔薬が〕効果{こうか}がある、効き目を現す《医》〔刺激{しげき}に生体{せいたい}が〕反応{はんのう}する〈米〉賠償{ばいしょう}する【他動】~と答える[応答{おうとう}する]【名】《建築》〔アーチを支える〕リスポンド、片蓋柱{かたふたばしら}◆【参考】
\\	《キリスト教》唱和{しょうわ}◆【同】
\\	レベル3、発音
\\	、カナリスパンド、リスポンド、変化《動》
\\	、分節
\\	その水はもはや飲料用として安全とは見なされて[考えられて]いません。 ・
\\	あなたは私の気持ちなんてちっとも考えてくれなかった。 ・
\\	あなたは私の気持ちを考えてくれましたか? ・
\\	あなたは私の気持ちを考えてくれたことあるの? ・
\\	いつも私の気持ちを考えてくださってありがとうございます。 ・
\\	私の提案をご検討ください。◆ビジネスシーンでよく使われる表現。 ・
\\	来週食事をごいっしょにいかがでしょうか。ご検討いただければ幸いです。 ・
\\	今月末日までに到着した申込書のみを検討(の)対象とさせていただきます。	変化形 
\\	【自動】よく考える、熟考{じゅっこう}する【他動】~と考える[見なす・認める] ・
\\	その水はもはや飲料用として安全とは見なされて[考えられて]いません。 ・
\\	あなたは私の気持ちなんてちっとも考えてくれなかった。 ・
\\	あなたは私の気持ちを考えてくれましたか? ・
\\	あなたは私の気持ちを考えてくれたことあるの? ・
\\	いつも私の気持ちを考えてくださってありがとうございます。~を考慮{こうりょ}[考察{こうさつ}・検討{けんとう}]する ・
\\	私の提案をご検討ください。◆ビジネスシーンでよく使われる表現。 ・
\\	来週食事をごいっしょにいかがでしょうか。ご検討いただければ幸いです。 ・
\\	今月末日までに到着した申込書のみを検討(の)対象とさせていただきます。じっと見る、凝視{ぎょうし}するレベル3、発音
\\	、カナコンシダー、コンスィダ、変化《動》
\\	、分節
\\	大体100万というとこじゃないかな。 ・
\\	でも私はそれは不可能ではないと思います。 ・
\\	どんなことでも可能でしょう。 ・
\\	私もそうだろうと思います。/そう思いますよ。 ・
\\	私に翻訳してもらいたいんでしょう。 ・
\\	賛成するしかないだろう。 ・
\\	ここで何が起こったとお考えですか。 ・
\\	それは何でできていると思いますか。 ・
\\	たぶんあなたが正しいのでしょう。 ・
\\	このばかげたことが本当だと思うのか。 ・
\\	例えば~があるとする。	変化形 
\\	【他動】〔~を真実だと〕思う、仮定{かてい}[想定{そうてい}・推定{すいてい}]する◆【語源】ラテン語「下に置く」 ・
\\	大体100万というとこじゃないかな。 ・
\\	でも私はそれは不可能ではないと思います。 ・
\\	どんなことでも可能でしょう。 ・
\\	私もそうだろうと思います。/そう思いますよ。 ・
\\	私に翻訳してもらいたいんでしょう。 ・
\\	賛成するしかないだろう。 ・
\\	ここで何が起こったとお考えですか。 ・
\\	それは何でできていると思いますか。 ・
\\	たぶんあなたが正しいのでしょう。 ・
\\	このばかげたことが本当だと思うのか。~と想像{そうぞう}する、推量{すいりょう}するもし~だったらどうだろう、~してはどうだろう、~しよう ・
\\	例えば~があるとする。前提{ぜんてい}とするレベル3、発音
\\	、カナサポーズ、サポウズ、変化《動》
\\	、分節
\\	私にもよく分かりません。 ・
\\	想像ですが。 ・
\\	ねえ聞いて。/あのさあ。◆話を切り出すときに使う ・
\\	昨日私が誰に出くわしたか聞いたら、きっとびっくりするよ。 ・
\\	そのことは全く予想していませんでした。 ・
\\	《相手に誘われて》じゃあそうさせてもらおうかな。 ・
\\	とても無理のようです。 ・
\\	そのようだね。 ・
\\	皆目見当がつかないよ。	変化形 
\\	【自動】~と推測{すいそく}する ・
\\	私にもよく分かりません。 ・
\\	想像ですが。【他動】~が…であると推測{すいそく}する、想像{そうぞう}する、~を推量{すいりょう}する ・
\\	ねえ聞いて。/あのさあ。◆話を切り出すときに使う ・
\\	昨日私が誰に出くわしたか聞いたら、きっとびっくりするよ。うまく言い当てる、言い当てる、解きあてる ・
\\	そのことは全く予想していませんでした。〈米俗〉~だと思う ・
\\	《相手に誘われて》じゃあそうさせてもらおうかな。 ・
\\	とても無理のようです。 ・
\\	そのようだね。【名】推測{すいそく}、推量{すいりょう} ・
\\	皆目見当がつかないよ。推測{すいそく}[推量{すいりょう}]すること《囲碁》見当{けんとう}レベル2、発音
\\	、カナゲス、変化《動》
\\	~とはとても考えられません。 ・
\\	その気持ち分かるなあ。 ・
\\	65歳になった時自分は何をしていると思いますか? ・
\\	国籍だけで差別されるのがどんな気持ちか、分かりますか。 ・
\\	気のせいかな。誰かが私の部屋にいるような気がしたんだけど。	変化形 
\\	【自動】想像{そうぞう}する、思う、推測{すいそく}する【他動】想像{そうぞう}する、思う、心に描く、推測{すいそく}する、仮定{かてい}する ・
\\	~とはとても考えられません。 ・
\\	その気持ち分かるなあ。 ・
\\	65歳になった時自分は何をしていると思いますか? ・
\\	国籍だけで差別されるのがどんな気持ちか、分かりますか。気のせい[思い違い]で~と考える ・
\\	気のせいかな。誰かが私の部屋にいるような気がしたんだけど。レベル2、発音
\\	、カナイマジン、変化《動》
\\	、分節
\\	あなたは私にビジネスパートナーとして何を求めていらっしゃるのでしょうか。 ・
\\	「ねえ、ケリー!ボーナスがたくさん出たんだって!」「そう。そんなに期待してなかったので、すごく驚いたよ!」 ・
\\	私は来週末までにレポートを完成させるつもりです。 ・
\\	まさか私がそんな話を信じるなんて、本気で思ってませんよね。 ・
\\	こうなることは予期していました。 ・
\\	今日の午後に仕事上の来客があります。 ・
\\	「驚いた?! 近くまで来たので、挨拶{あいさつ}しようと思って」「あら!きょうは誰も来ないと思ってたので、まだパジャマのままですよ」 ・
\\	「パーティーには何時頃来れますか?」「ちょっと分からないですね!今夜は残業になるかもしれないので、期待しないで待っててくださいね」
\\	""は、直訳すると、「あなたが実際に私の顔を見たときに、私が来たんだと思ってください」となる。 ・
\\	期待しないで待ってるよ。◆まれに使われる表現 ・
\\	御社のご要求は弊社の予想以上のものでした。 ・
\\	あなたのうそにはもううんざりしていますので、もうこれ以上私には何も期待しないでください。 ・
\\	あと2~3週間で私たちの最初の子どもが生まれます。
\\	変化形 
\\	"【自動】〔物事{ものごと}が当然起きると〕予期{よき}する、期待{きたい}する〈話〉妊娠{にんしん}している、出産{しゅっさん}を間近{まぢか}にしている【他動】〔~が当然起きるだろうと〕予期{よき}する、期待{きたい}する ・
\\	あなたは私にビジネスパートナーとして何を求めていらっしゃるのでしょうか。 ・
\\	「ねえ、ケリー!ボーナスがたくさん出たんだって!」「そう。そんなに期待してなかったので、すごく驚いたよ!」 ・
\\	私は来週末までにレポートを完成させるつもりです。 ・
\\	まさか私がそんな話を信じるなんて、本気で思ってませんよね。 ・
\\	こうなることは予期していました。〔当然起きるまたは来るであろう〕~を待つ[楽しみにする] ・
\\	今日の午後に仕事上の来客があります。 ・
\\	「驚いた?! 近くまで来たので、挨拶{あいさつ}しようと思って」「あら!きょうは誰も来ないと思ってたので、まだパジャマのままですよ」 ・
\\	「パーティーには何時頃来れますか?」「ちょっと分からないですね!今夜は残業になるかもしれないので、期待しないで待っててくださいね」
\\	""は、直訳すると、「あなたが実際に私の顔を見たときに、私が来たんだと思ってください」となる。 ・
\\	期待しないで待ってるよ。◆まれに使われる表現〔~を当然{とうぜん}のこととして〕要求{ようきゅう}する、求める ・
\\	御社のご要求は弊社の予想以上のものでした。 ・
\\	あなたのうそにはもううんざりしていますので、もうこれ以上私には何も期待しないでください。〈話〉〔子どもを〕産む予定{よてい}である◆【用法】通例進行形で用いられる。 ・
\\	あと2~3週間で私たちの最初の子どもが生まれます。レベル2、発音
\\	、カナイクスペクト、エクスペクト、変化《動》
\\	、分節
\\	君を信頼してる。 ・
\\	信じてくれ。僕は生まれ変わるよ。◆何度もまずいことをしたうえで相手に謝るときの典型的表現 ・
\\	そんな言葉を信じては駄目。/そんな話は少しも信じてはいけません。 ・
\\	君が言っていることを信じるよ。/《相槌》そうでしょうね。 ・
\\	これは信じてもらえないかな。◆これから話すことについて、相手の興味を引くために用いる表現。 ・
\\	お目にかかるのは初めてですよね。/あなたとは会ったことがありませんね。	変化形 
\\	【自動】信じる ・
\\	君を信頼してる。【他動】信じる、真に受ける、確信{かくしん}する、信頼{しんらい}する ・
\\	信じてくれ。僕は生まれ変わるよ。◆何度もまずいことをしたうえで相手に謝るときの典型的表現 ・
\\	そんな言葉を信じては駄目。/そんな話は少しも信じてはいけません。 ・
\\	君が言っていることを信じるよ。/《相槌》そうでしょうね。 ・
\\	これは信じてもらえないかな。◆これから話すことについて、相手の興味を引くために用いる表現。思う、考える ・
\\	お目にかかるのは初めてですよね。/あなたとは会ったことがありませんね。レベル1、発音
\\	、カナビリーブ、変化《動》
\\	、分節
\\	ぼくがすべて面倒みるから任せてくれ。 ・
\\	私を信用しなさい。 ・
\\	あなたが正しいことをすると信じている。 ・
\\	私を信頼できなくて、誰を信頼するというの? ・
\\	私の判断に任せてくれてありがとう。 ・
\\	男などに信頼も誠も正直もあったものじゃない。◆シェークスピアの「ロミオとジュリエット」より。 ・
\\	信用は、銀行にとって最も重要です。 ・
\\	彼らの関係は信頼とお互いへの理解力で成り立っています。 ・
\\	みんな私を信頼してくれています。	変化形 
\\	【自動】信用{しんよう}する、信頼{しんらい}する当てにする、頼りにする ・
\\	ぼくがすべて面倒みるから任せてくれ。【他動】信用{しんよう}[信頼{しんらい}]する ・
\\	私を信用しなさい。 ・
\\	あなたが正しいことをすると信じている。 ・
\\	私を信頼できなくて、誰を信頼するというの?~に任せる ・
\\	私の判断に任せてくれてありがとう。~を委託{いたく}する、~に掛け売りする【名】信頼{しんらい}、信用{しんよう} ・
\\	男などに信頼も誠も正直もあったものじゃない。◆シェークスピアの「ロミオとジュリエット」より。 ・
\\	信用は、銀行にとって最も重要です。 ・
\\	彼らの関係は信頼とお互いへの理解力で成り立っています。 ・
\\	みんな私を信頼してくれています。頼りとする人[物]委託{いたく}、委託物{いたく ぶつ}、法信託{ほう しんたく}、預かり物《商業》掛け(売り)企業合同{きぎょう ごうどう}、トラストレベル2、発音
\\	、カナトゥラスト、トラスト、変化《動》
\\	失礼ですが、不思議でならないのです。あなたは誰でどこから来たのですか。 ・
\\	あなたはもしかしてボブ・グラハムさんではありませんか? ・
\\	《誘う》明日、あなたがお暇かどうかと思ったのですが。 ・
\\	《誘う》明日、食事でもどうですか? ・
\\	彼が私の英語を理解してくれるだろうか。 ・
\\	彼は50歳過ぎてるのかなぁ。 ・
\\	もう2~3日休みが取れたらいいのだが。 ・
\\	《誘う》次の日曜日、ぼくと付き合ってもらえないかと思って。 ・
\\	彼がその事故で生き残ったことは驚きだ。 ・
\\	当然です。 ・
\\	驚きの念を禁じえない。	変化形 
\\	【自動】驚嘆{きょうたん}する、感嘆{かんたん}する知りたいと思う、疑問{ぎもん}に思う ・
\\	失礼ですが、不思議でならないのです。あなたは誰でどこから来たのですか。 ・
\\	あなたはもしかしてボブ・グラハムさんではありませんか?【他動】~かどうかと思う、~を知りたいと思う ・
\\	《誘う》明日、あなたがお暇かどうかと思ったのですが。 ・
\\	《誘う》明日、食事でもどうですか? ・
\\	彼が私の英語を理解してくれるだろうか。 ・
\\	彼は50歳過ぎてるのかなぁ。 ・
\\	もう2~3日休みが取れたらいいのだが。 ・
\\	《誘う》次の日曜日、ぼくと付き合ってもらえないかと思って。~に感嘆{かんたん}[驚嘆{きょうたん}]する ・
\\	彼がその事故で生き残ったことは驚きだ。【名】感嘆{かんたん}すべき[驚くべき]こと[もの]◆美しいものや新奇なものに用いることが多い。 ・
\\	当然です。 ・
\\	驚きの念を禁じえない。感嘆{かんたん}[驚嘆{きょうたん}]の念不可思議{ふかしぎ}な[超自然的{ちょうしぜん てき}な]こと、奇跡{きせき}戸惑い{とまどい}、疑い【形】驚嘆{きょうたん}[称賛{しょうさん}]すべき〔これまでになく〕素晴{すば}らしい、極めて優れたレベル2、発音
\\	、カナワンダー、変化《動》
\\	、分節
\\	それはそうでしょう。/それもそうだ。/それに疑いはありません。 ・
\\	そのことについては分かりません[知りません]。/私はそうは思いません。/さあどうでしょうね。/(本当だか)怪しいもんだ。/そんなことないですよ。 ・
\\	~ということはまず間違いないでしょう。	変化形 
\\	【自動】〔~に〕疑惑{ぎわく}を抱く、〔~を〕不確{ふたし}かだと思う【他動】~を疑う◆否定的に ・
\\	それはそうでしょう。/それもそうだ。/それに疑いはありません。 ・
\\	そのことについては分かりません[知りません]。/私はそうは思いません。/さあどうでしょうね。/(本当だか)怪しいもんだ。/そんなことないですよ。~を疑問{ぎもん}に思う、~ではなさそうだと思う、~とは思えない
\\	節・
\\	-節を導く【名】〔断定的な〕疑い、疑念{ぎねん}、疑惑{ぎわく} ・
\\	~ということはまず間違いないでしょう。懸念{けねん}、心配{しんぱい}、不安{ふあん}◆【類】
\\	◆古典的レベル2、発音
\\	、カナダウト、変化《動》
\\	完全な証拠もなしに彼を疑ってはいけない。 ・
\\	彼の意図は信用できないが、そのやり口の方がもっと信用できない。 ・
\\	彼女は彼が他の(人)と付き合っているのではないかと疑い始めた。 ・
\\	そうかもしれませんね。 ・
\\	私が疑われてるんですか? ・
\\	この殺人の有力な容疑者の一人が死亡しました。	変化形 
\\	【他動】〔通例証拠なしに〕(人)に嫌疑を掛ける、(人)に罪があると思う ・
\\	完全な証拠もなしに彼を疑ってはいけない。~を疑わしく思う、~を本当{ほんとう}ではないと思う、~を信用{しんよう}しない ・
\\	彼の意図は信用できないが、そのやり口の方がもっと信用できない。~ではないかと疑う[うすうす感じる]、~だろうと思う ・
\\	彼女は彼が他の(人)と付き合っているのではないかと疑い始めた。 ・
\\	そうかもしれませんね。【名】容疑者{ようぎしゃ}、被疑者{ひぎしゃ}、嫌疑者{けんぎ しゃ} ・
\\	私が疑われてるんですか? ・
\\	この殺人の有力な容疑者の一人が死亡しました。【形】疑わしいレベル3、発音《名・形》
\\	《動》
\\	、カナサスペクト、変化《動》
\\	、分節
\\	それは私たちが決めることだ。 ・
\\	〔他人の意見に影響されずに〕自分で決めなさい。 ・
\\	私はどれを買うべきでしょうか? 迷っちゃうな。 ・
\\	日本語(の授業)を取るのはきっと楽しいことに違いないと思った。	変化形 
\\	【自動】決定{けってい}する、決心{けっしん}する、決意{けつい}する ・
\\	それは私たちが決めることだ。 ・
\\	〔他人の意見に影響されずに〕自分で決めなさい。 ・
\\	私はどれを買うべきでしょうか? 迷っちゃうな。判決{はんけつ}[評決{ひょうけつ}]を下すきっと~だと思いこむ【他動】~を解決{かいけつ}する~しようと決心{けっしん}する、~することに決める~だと確信{かくしん}する、きっと~だと思う ・
\\	日本語(の授業)を取るのはきっと楽しいことに違いないと思った。レベル1、発音
\\	、カナディサイド、デサイド、変化《動》
\\	、分節
\\	彼は、これからはもっといい仕事をしようと決心しました。 ・
\\	捜査官たちは、いまだにその事故の原因を究明していない。	変化形 
\\	【自動】決心{けっしん}する《法律》〔権利{けんり}などが〕終了{しゅうりょう}[消滅{しょうめつ}]する【他動】〔人が~することを〕決心{けっしん}[決定{けってい}]する ・
\\	彼は、これからはもっといい仕事をしようと決心しました。〔物事などが人に〕決心{けっしん}させる〔出来事・人などが〕~を決定{けってい}[確定{かくてい}・断定{だんてい}]する〔事実などを〕見つけ出す、〔原因などを〕究明{きゅうめい}する ・
\\	捜査官たちは、いまだにその事故の原因を究明していない。〔位置・形・程度・数量などを〕測定{そくてい}する〔紛争などを〕終結{しゅうけつ}させる《法律》〔権利{けんり}などを〕終了{しゅうりょう}[消滅{しょうめつ}]させるレベル3、発音
\\	、カナデターミン、変化《動》
\\	、分節
\\	冗談なんかじゃない。/冗談で言っているのではない。	変化形 
\\	【他動】~するつもりである、~を意図{いと}する ・
\\	冗談なんかじゃない。/冗談で言っているのではない。~を意味{いみ}する、~を表すレベル2、発音
\\	、カナインテンド、変化《動》
\\	、分節
\\	それは時と場合による。/それは場合によりけり[状況次第・ケースバイケース]だ。/事と次第によってはね。/一概には言えない。	変化形 
\\	【自動】〔~によって〕決まる、左右{さゆう}される ・
\\	それは時と場合による。/それは場合によりけり[状況次第・ケースバイケース]だ。/事と次第によってはね。/一概には言えない。〔他人の助けなどを〕当てにする〔人・物を〕信頼{しんらい}する、頼みにするレベル2、発音
\\	、カナデペンド、ディペンド、変化《動》
\\	、分節
\\	信頼できる情報はかなり限られている。	変化形 
\\	【自動】~を頼る[当てにする]◆【用法】
\\	~を信頼{しんらい}[信用{しんよう}]する◆【用法】
\\	◆【類】
\\	信頼できる情報はかなり限られている。レベル3、発音
\\	、カナリライ、変化《動》
\\	、分節
\\	その仕事の内容を教えてください。 ・
\\	この仕事の内容を詳しく知りたいのですが。 ・
\\	それはたくさんの事務処理と時間がかかるだろう。 ・
\\	誰か付き合っている人はいるんですか? ・
\\	私はその新しいプロジェクトとは関係ありません。 ・
\\	あなたには関係ないことですから、私たちの会話に入らないでください。 ・
\\	彼はその銀行強盗計画に加担しています。	変化形 
\\	【他動】〔不可欠{ふかけつ}なものとして〕~を含む[取り込む] ・
\\	その仕事の内容を教えてください。 ・
\\	この仕事の内容を詳しく知りたいのですが。~を伴う、~を引き起こす ・
\\	それはたくさんの事務処理と時間がかかるだろう。(人)と関わる、(人)に関与{かんよ}する ・
\\	誰か付き合っている人はいるんですか? ・
\\	私はその新しいプロジェクトとは関係ありません。 ・
\\	あなたには関係ないことですから、私たちの会話に入らないでください。(人)を参加{さんか}させる、(人)を巻き込む ・
\\	彼はその銀行強盗計画に加担しています。~に影響{えいきょう}を与える、~に作用{さよう}する(人)の注意{ちゅうい}を奪う、(人)を没頭{ぼっとう}させる~を複雑{ふくざつ}にする、~を困難{こんなん}にさせる◆【用法】通例、受け身で用いられる。〈文〉~を包み込む、~を囲む◆【用法】通例、受け身で用いられる。〈古〉~を巻く、~を巻き上げるレベル4、発音
\\	、カナインバルブ、インボルブ、変化《動》
\\	、分節
\\	名声、富、外見といったことは私にとってはどうでもいいことです。 ・
\\	それは私の知ったことではありません。/それは私には関係のないことです。 ・
\\	私が首を突っ込む問題じゃない。/そんなこと私の知ったことではありません。 ・
\\	お気遣い[ご心配・ご配慮]いただきありがとうございます。	変化形 
\\	【他動】〔~にとって〕興味{きょうみ}がある、重要{じゅうよう}である ・
\\	名声、富、外見といったことは私にとってはどうでもいいことです。(人)を心配{しんぱい}させる、(人)に気をもませる(人)に関係{かんけい}する、(人)を関与{かんよ}させる〔~にとって〕テーマ[主題{しゅだい}]となる【名】関心事{かんしんじ}、懸案事項{けんあん じこう} ・
\\	それは私の知ったことではありません。/それは私には関係のないことです。 ・
\\	私が首を突っ込む問題じゃない。/そんなこと私の知ったことではありません。気遣い{きづかい}、懸念{けねん}、心配{しんぱい}、不安{ふあん} ・
\\	お気遣い[ご心配・ご配慮]いただきありがとうございます。関連{かんれん}、利害関係{りがい かんけい}事業{じぎょう}、会社{かいしゃ}、企業{きぎょう}〈話〉〔つまらない〕仕掛け、物、人◆古めかしい言い方。レベル3、発音
\\	、カナコンサーン、変化《動》
\\	、分節
\\	その二つのサイン[署名]を比べてみたらぴったり一致しました。	変化形 
\\	【自動】〔~に〕匹敵{ひってき}する、〔~と〕同等{どうとう}であると見なす【他動】~を(…と)比べる、比較{ひかく}する、対比{たいひ}する、対照{たいしょう}する ・
\\	その二つのサイン[署名]を比べてみたらぴったり一致しました。~を(…に)なぞらえる、例えるレベル2、発音
\\	、カナコンペア、変化《動》
\\	、分節
\\	彼女はその男の名前と住所を警官に話しました。	変化形 
\\	【自動】関係{かんけい}[つながり]がある、関連{かんれん}する〔人が〕うまく折り合う、仲良くする〈話〉〔人が〕好意的{こうい てき}に受け取る[理解{りかい}する]【他動】〔人や物同士{ぶつ どうし}を〕結び付ける、関連付ける〔話や出来事{できごと}を〕物語る{ものがたる}、説明{せつめい}する ・
\\	彼女はその男の名前と住所を警官に話しました。レベル3、発音
\\	、カナリレイト、変化《動》
\\	、分節
\\	動物たちにとって、獣医は注射や薬を連想させるものである。 ・
\\	お友達やお仲間をわれわれにご紹介していただくと、ご紹介者とお友達に特別プレゼントを差し上げます。 ・
\\	審判はいつでも同僚の意見を求めることができる。 ・
\\	共同経営者の同意がなければどんな事業も実行してはならない。 ・
\\	これは
\\	と
\\	が同伴であるときに、かつそのときにのみ、成立します。/これは、
\\	と
\\	が同伴であることと同値である。	変化形 
\\	【自動】付き合う、仲間{なかま}に加わる、交際{こうさい}する共同{きょうどう}で事業{じぎょう}を行う、提携{ていけい}する【他動】~と(~を)結び付ける、結合{けつごう}させる、結び付けて考える、関連付ける、関連{かんれん}させる◆【用法】
\\	~~で(~を)連想{れんそう}する、思い出す◆【用法】
\\	動物たちにとって、獣医は注射や薬を連想させるものである。~を仲間{なかま}に加える、提携{ていけい}させる【名】〔気心{きごころ}の知れた〕友人{ゆうじん}、〔多くの時を過ごす〕仲間{なかま}、〔仕事{しごと}の〕同僚{どうりょう} ・
\\	お友達やお仲間をわれわれにご紹介していただくと、ご紹介者とお友達に特別プレゼントを差し上げます。 ・
\\	審判はいつでも同僚の意見を求めることができる。共同経営者{きょうどう けいえいしゃ}、共同出資者{きょうどう しゅっし しゃ} ・
\\	共同経営者の同意がなければどんな事業も実行してはならない。〔学会{がっかい}や協会{きょうかい}などの〕準会員{じゅん かいいん}準学士{じゅんがくし}◆2年制大学課程修了者に与えられる称号。《数学》同伴(数){どうはん(すう)}、同伴{どうはん}(な)元 ・
\\	これは
\\	と
\\	が同伴であるときに、かつそのときにのみ、成立します。/これは、
\\	と
\\	が同伴であることと同値である。【形】準~、副~、次席{じせき}のレベル4、発音《名・形》
\\	《動》
\\	、カナアソーシエイト、アソウシエイト、変化《動》
\\	、分節
\\	彼の死に対する説明は全く違ったものでした。	変化形 
\\	【自動】〔形や性質{せいしつ}が〕異なる、似ていない〔意見{いけん}が〕一致{いっち}しない、合わない ・
\\	彼の死に対する説明は全く違ったものでした。レベル4、発音
\\	、カナディファー、デファー、変化《動》
\\	、分節
\\	《コ》アクセスが拒否されました。	変化形 
\\	【他動】〔~が事実{じじつ}でないとして〕否定{ひてい}する〔価値{かち}などを〕認めない、信じない〔人を〕知らないと言う、〔人との〕関係{かんけい}を否定{ひてい}する〔必要{ひつよう}なものなどを〕与えない、自由{じゆう}にさせない〔要求{ようきゅう}などを〕拒否{きょひ}する、拒絶{きょぜつ}する ・
\\	《コ》アクセスが拒否されました。〔自分自身{じぶん じしん}を〕律する、自制{じせい}する◆【用法】再帰代名詞を目的語に取る。レベル2、発音
\\	、カナディナイ、デナイ、変化《動》
\\	、分節
\\	二人は翌日に駅で落ち合うことを申し合わせた。 ・
\\	全くその通り。/確かにそうだ。/全く同感です。 ・
\\	二つの腕時計は、同じ時刻を指している。 ・
\\	商売敵は気が合わぬもの。 ・
\\	その気候は私の体に合っています。 ・
\\	脂っこい食物は私に合いません。	変化形 
\\	【自動】〔提案・意見などに〕同意{どうい}[合意{ごうい}・賛成{さんせい}・賛同{さんどう}・承諾{しょうだく}・承知{しょうち}]する、同感{どうかん}である ・
\\	二人は翌日に駅で落ち合うことを申し合わせた。 ・
\\	全くその通り。/確かにそうだ。/全く同感です。一致{いっち}する ・
\\	二つの腕時計は、同じ時刻を指している。 ・
\\	商売敵は気が合わぬもの。性に合う、仲良くやっていく〔飲食物・気候などが人の〕体に合う ・
\\	その気候は私の体に合っています。 ・
\\	脂っこい食物は私に合いません。【他動】〔~ということを〕認めるレベル2、発音
\\	、カナアグリ、変化《動》
\\	、分節
\\	私の電子メールを受信しましたか?	変化形 
\\	【自動】受け取る、受け手となる歓迎{かんげい}する、喜んで迎える《スポーツ》レシーブする《電気》受信{じゅしん}する《キリスト教》聖体拝領{せいたい はいりょう}を受ける【他動】〔与えられた物を〕受け取る、受理{じゅり}する〔知らせや情報{じょうほう}を〕聞く、知る〔手紙{てがみ}や電話{でんわ}を〕もらう、受ける ・
\\	私の電子メールを受信しましたか?〔人からある取り扱いを〕受ける、経験{けいけん}する〔称号{しょうごう}などを〕受ける、授与{じゅよ}される〔損害{そんがい}や罰を〕受ける、被る〔力や衝撃{しょうげき}を〕受け止める、かけられる〔物の重さを〕受ける、支える〔訪問客{ほうもん きゃく}を〕出迎える{でむかえる}、迎え入れる〔訪問者{ほうもんしゃ}と〕接見{せっけん}する、面会{めんかい}する〔会などに人を〕受け入れる、入会{にゅうかい}させる《スポーツ》〔ボールを〕レシーブする〔液体{えきたい}などを〕中に入れる[ためる]《電気》〔電気信号{でんき しんごう}を〕受信{じゅしん}する〔印象{いんしょう}などを〕受ける、与えられる〔誓いなどを正式{せいしき}に〕受け入れる、聞き届けるレベル1、発音
\\	、カナリスィーブ、レシーブ、変化《動》
\\	、分節
\\	《礼を述べる》感謝の印にこれをお受け取りください。 ・
\\	そこが私が受け入れられないところなんです。 ・
\\	しばらく前に
\\	大学に出願してさあ、すっかり忘れてたんだよね。でもその後どうなったと思う?!今日手紙来てて、合格だって!	変化形 
\\	【他動】〔贈り物などを〕快く受け取る ・
\\	《礼を述べる》感謝の印にこれをお受け取りください。〔申し出などを〕引き受ける、受諾{じゅだく}する〔手形・小切手などによる支払いを〕引き受ける〔提出物などを〕受理する〔招待などに〕応じる〔挑戦{ちょうせん}などを〕受けて立つ〔申し込み・謝罪{しゃざい}・弁解{べんかい}などを〕受け入れる承認{しょうにん}する、承服{しょうふく}する、認める、容認{ようにん}する、受け入れる、納得{なっとく}する ・
\\	そこが私が受け入れられないところなんです。~を入学{にゅうがく}させる、入隊{にゅうたい}させる ・
\\	しばらく前に
\\	大学に出願してさあ、すっかり忘れてたんだよね。でもその後どうなったと思う?!今日手紙来てて、合格だって!あえて一身{いっしん}に~を引き受ける、甘受{かんじゅ}する、順応{じゅんのう}する〔機器{きき}に〕対応{たいおう}するレベル2、発音
\\	、カナアクセプト、変化《動》
\\	、分節
\\	そのことにはあまりお答えしたくありません。◆【直訳】そのことには答えないことを選びます。 ・
\\	当社と致しましては本船渡し(値段)を希望致します。	変化形 
\\	【他動】~を(選り)好む、むしろ~の方を好む、~の方を選ぶ ・
\\	そのことにはあまりお答えしたくありません。◆【直訳】そのことには答えないことを選びます。 ・
\\	当社と致しましては本船渡し(値段)を希望致します。〔訴訟・請求・要求などを〕提出する、申請する、申し入れるレベル2、発音
\\	、カナプリファー、変化《動》
\\	、分節
\\	当社は、従業員に午後10時以降の残業を認めていません。 ・
\\	私の両親は非常に厳しいのでデートをさせてくれません。 ・
\\	それらのバッグを運ぶのをお手伝いさせてください。 ・
\\	ほとんどのお店がペット禁止です。	変化形 
\\	【自動】考慮{こうりょ}に入れる、念頭に置いておく◆【用法】
\\	の形をとる余地{よち}がある、許す◆【用法】通例否定文で用い、
\\	の形をとる【他動】〔人が~するのを〕許す、許可{きょか}する ・
\\	当社は、従業員に午後10時以降の残業を認めていません。 ・
\\	私の両親は非常に厳しいのでデートをさせてくれません。 ・
\\	それらのバッグを運ぶのをお手伝いさせてください。〔人や動物{どうぶつ}などがある場所{ばしょ}に〕入る[いる]ことを許可{きょか}する ・
\\	ほとんどのお店がペット禁止です。〔人に〕~を与える[持たせる]◆その人にとって好ましいものを持つことを許す。〔金・時間{じかん}などを〕割り当てる、配分{はいぶん}する〔前もって余裕{よゆう}を〕見越す{みこす}、持たせる〔金額を〕値引き[割引]をする〈米方言〉〔~が本当{ほんとう}だと〕認める、思う〈米方言〉〔~すべきだと〕言い張る、決めつけるレベル2、発音
\\	、カナアラウ、アロー、アロウ、変化《動》
\\	、分節
\\	《就業規則》署名者として権限を付与された従業員のみが
\\	の代表者として通信文、確認書、支払関連書類に署名することができる。	変化形 
\\	【他動】許可{きょか}する、許す、認める、容認{ようにん}する、同意{どうい}する、可能{かのう}にする ・
\\	《就業規則》署名者として権限を付与された従業員のみが
\\	の代表者として通信文、確認書、支払関連書類に署名することができる。【名】許可(証){きょか(しょう)}レベル2、発音《名》
\\	《動》
\\	、カナパーミット、パミット、変化《動》
\\	、分節
\\	ボブはローラといつも一緒にいる。 ・
\\	窓にレースのカーテンがかかっている。	変化形 
\\	【自動】掛かる、垂れ下がるさまよう、ほっつき回る◆【用法】
\\	〔友人{ゆうじん}と一緒{いっしょ}に〕リラックスする、のんびりする、たむろする◆【用法】
\\	ボブはローラといつも一緒にいる。絞首刑{こうしゅけい}になる《野球》(ボールが)すっぽぬける《コ》ハングアップする【他動】〔~から〕~をつるす、〔~に〕~を掛ける◆過去形=
\\	、過去分詞形=
\\	窓にレースのカーテンがかかっている。~を取り付ける~を絞首刑{こうしゅけい}にする、~の首をつる◆過去形=
\\	、過去分詞形=
\\	《コ》~をハング(アップ)させる、〔主語{しゅご}〕が原因{げんいん}で~が応答{おうとう}しない状態{じょうたい}になる【名】こつ、扱い方〔友人との〕たまり場レベル2、発音
\\	、カナハング、変化《動》
\\	「去年、
\\	村に住んでたんだ。」「へぇ、ど田舎だね」 ・
\\	その猫は木から下りられなくなってしまった。◆上ったはいいものの戻れない。 ・
\\	弱ったなあ。 ・
\\	彼女には我慢できなかった。
\\	変化形 
\\	"【1他動】〔庭の草花{くさばな}に〕つっかい棒をする〈古〉《印刷》〔活字{かつじ}を〕組み版ステッキに組む【1名】〔木から切り取った〕細枝{しもと}、小枝{こえだ/しょうし}〔燃料{ねんりょう}の〕薪、そだ〔競技{きょうぎ}で使う〕木製{もくせい}の棒◆ホッケーのスティック、ビリヤードのキュー、陸上のハードルなどが含まれる。〔細長い木製{ほそながい もくせい}の〕棒、さお◆権威の象徴としての杖や指揮棒などが含まれる。〔人を打つための〕こん棒、警棒{けいぼう}、むち〈英〉〔歩くための〕杖、ステッキ◆【同】
\\	〔木以外の細い〕棒状{ぼうじょう}の物、スティック◆食べ物のニンジン・スティックや棒状のダイナマイトなどが含まれる。〈俗〉マリファナたばこ《海事》マスト〔航空機{こうくうき}の〕操縦{そうじゅう}かん〔自動車{じどうしゃ}の〕手動変速機{しゅどう へんそくき}◆【同】
\\	《印刷》組み版ステッキ◆【同】
\\	《軍事》〔投下爆弾{とうか ばくだん}の〕一連{いちれん}◆等間隔で次々に落として、一列に着弾させるもの。〈俗〉《軍事》〔パラシュート部隊{ぶたい}の〕一斉降下{いっせい こうか}〈話〉〔1点の〕家具{かぐ}〈英話〉〔痛烈{つうれつ}な〕非難{ひなん}、批判{ひはん}〈話〉退屈{たいくつ}な[堅苦{かたぐる}しい]やつ〈豪俗〉サーフボード
\\	〈話〉ど田舎◆【類】
\\	「去年、
\\	村に住んでたんだ。」「へぇ、ど田舎だね」ソーズ、ソウズ、索子{ソーズ}◆麻雀牌の種類◆【同】
\\	【2自動】突き刺す[通す]〔接着剤{せっちゃくざい}で〕くっつく、貼る突き出す、突き出る〔止まったまま〕動かない、進まない〔心に〕とどまる、こびり付く《イ》〔特定{とくてい}のサイトを〕繰り返し見る【2他動】〔とがった物で〕~を突き刺す[刺し通す]〔とがった物を〕突き刺す、刺し通す〔とがった物に〕~を(突き)刺す〔動物{どうぶつ}を〕刺し殺す〔接着剤{せっちゃくざい}で〕~をくっつける[貼る]〔ピンなどで〕~を留める[付ける]~を突き出させる、~を飛び出させる~を動けなくさせる、~を立ち往生させる ・
\\	その猫は木から下りられなくなってしまった。◆上ったはいいものの戻れない。〈話〉〔ある場所{ばしょ}に物を〕置く〈話〉(人)を困らせる、(人)を当惑{とうわく}させる◆【用法】通例、受け身で用いられる。 ・
\\	弱ったなあ。〈話〉(人)をだます、(人)を利用{りよう}する◆【用法】通例、受け身で用いられる。〈英話〉(人)を我慢{がまん}する、(人)を受け入れる◆【用法】
\\	と共に否定・疑問文で用いる。 ・
\\	彼女には我慢できなかった。【2名】〔接着剤{せっちゃくざい}などの〕接着力{せっちゃく りょく}レベル2、発音
\\	、カナスティック、ステッキ、変化《動》
\\	変化形 
\\	【自動】〔ゲーム・試合{しあい}で〕同点{どうてん}になる、引き分ける結べる、縛れる【他動】~を結ぶ、結合{けつごう}する、縛る、くくる~と同記録{どう きろく}[同点{どうてん}]になる【名】結ぶもの、ひも、靴ひも、ネクタイ〔ひもなどの〕結び目〔~との〕つながり、絆{きずな}、関係{かんけい}、提携{ていけい}、連携{れんけい}同点{どうてん}、同順位{どう じゅんい}引き分け試合{しあい}《囲碁》持碁{じご}〈米〉枕木{まくらぎ}◆【同】
\\	《音楽》タイ◆二つの同じピッチの音符に付けられる弧線で、切れ目なしに演奏することを表す指示。◆【同】
\\	レベル1、発音
\\	、カナタイ、変化《動》
\\	私たちは霧が晴れるのを待ちました。 ・
\\	そのへんまで乗せてあげようか?	変化形 
\\	【自動】〔物が〕(持ち)上がる〔気分が〕高揚{こうよう}する〔霧が〕晴れる、消える ・
\\	私たちは霧が晴れるのを待ちました。〔雨が〕一時的{いちじてき}にやむ【他動】〔物を〕持ち上げる〔倒れた人・物などを〕引き起こす〔声などを〕張り上げる〔地位や名声などを〕高める、向上{こうじょう}させる〔気分などを〕高揚{こうよう}させる〔人や物を〕空輸{くうゆ}する〔禁止令などを〕解除{かいじょ}する、取り除く、撤廃{てっぱい}する〈話〉〔店などから物を〕盗む、万引{まんびき}する〈話〉〔他人の文章・アイデアなどを〕盗む、盗用{とうよう}する〈俗〉逮捕{たいほ}する〔借金{しゃっきん}などを〕払い終わる、片を付ける〔顔に〕美容整形{びよう せいけい}をする、しわ取り手術をする〔ゴルフでボールを〕高く打ち上げる〔ゴルフでボールを〕拾い上げる◆別の打ちやすい場所へ移すために。【名】持ち上げること、上昇{じょうしょう}させること持ち上げる[上昇{じょうしょう}させる]力持ち上げた[持ち上げ可能な]重量{じゅうりょう}持ち上げた距離{きょり}、上昇{じょうしょう}の度合い{どあい}〔土地{とち}の〕隆起{りゅうき}、上昇{じょうしょう}〔精神{せいしん}や気分{きぶん}の〕高まり、高揚{こうよう}〈話〉〔車にただで〕乗せること ・
\\	そのへんまで乗せてあげようか?手助け{てだすけ}、手伝い{てつだい}〔ダンスやスケートの〕リフト◆パートナーを持ち上げること。〈英〉エレベーター◆【同】〈米〉
\\	〔重量物{じゅうりょうぶつ}を持ち上げる〕リフト、昇降機{しょうこうき}〔航空機{こうくうき}の〕揚力{ようりょく}〔靴の〕リフト、かかと皮◆【同】
\\	〔体の〕隆起部{りゅうき ぶ}レベル2、発音
\\	、カナリフト、変化《動》
\\	私はビートルズを聞いて育ちました。/私はビートルズ世代です。 ・
\\	女手一つで5人の子どもを育てた。 ・
\\	彼の逮捕は法律違反ではないが、人権に関する厄介な問題を引き起こす。 ・2 
\\	2の10乗は1024です。	変化形 
\\	【自動】〔トランプで〕レイズする、賭け金を上げる〔物や体が〕上がる、持ち上げられる◆【用法】非標準的な用法であり、通常は
\\	を用いる。【他動】〔人や物を〕(持ち)上げる、掲げる〔横になっている人や物を〕立たせる、起こす〔顔や目などを〕上げる、上を向かせる〔建物{たてもの}などを〕建てる、建築{けんちく}する〔親が子どもを〕育てる、養う ・
\\	私はビートルズを聞いて育ちました。/私はビートルズ世代です。 ・
\\	女手一つで5人の子どもを育てた。〔家畜{かちく}や野菜{やさい}などを〕飼育{しいく}[栽培{さいばい}]する〔まめやみみず腫れなどを〕作る、こしらえる〔洋服{ようふく}のフリルなどを〕浮かび上がらせる〔パン生地{きじ}などを発酵{はっこう}させて〕膨らませる〔数や量を〕大きくする、増やす〔強さや程度{ていど}を〕強くする、高める〔声を〕(張り)上げる、荒げる〔ゲームで賭け金などを〕上げる、レイズする〔人を〕昇進{しょうしん}させる、地位{ちい}を高くする〔人や気持{きも}ちを〕元気{げんき}づける、高揚{こうよう}させる〔人を〕眠りから覚ます、よみがえらせる〔問題{もんだい}などを〕引き起こす、惹起{じゃっき}する、もたらす ・
\\	彼の逮捕は法律違反ではないが、人権に関する厄介な問題を引き起こす。〔問題{もんだい}などを〕取り上げる、話題{わだい}にする〔資金{しきん}などを〕募る、集める〔小切手{こぎって}などを不正{ふせい}に〕書き換える、額面{がくめん}を増やす〔無線{むせん}で〕~と交信{こうしん}する、~を呼び出す〔霊媒{れいばい}などによって〕(人)と交信{こうしん}する、(人)を呼び出す《海事》〔近づいて遠くのものを〕水平線上{すいへいせん じょう}に見る◆【同】
\\	《軍事》〔兵を引き揚げて包囲{ほうい}を〕解く、終わらせる〔禁止措置{きんし そち}や制約{せいやく}などを〕解く、取り消す〔咳をして痰を〕吐き出す〔笑顔{えがお}などを〕引き出す《音楽》〔母音{ぼいん}を〕上昇調{じょうしょうちょう}にする《数学》〔数を〕累乗{るいじょう}する、べき乗する ・2 
\\	2の10乗は1024です。【名】〔人や物を〕(持ち)上げること〔賭け金を〕上げる[高くする]こと〈主に米〉昇給{しょうきゅう}、賃上げ{ちんあげ}〔気持{きも}ちなどの〕高まり、高揚{こうよう}〔道路{どうろ}の〕上り坂《言語学》上昇調{じょうしょうちょう}レベル2、発音
\\	、カナレイズ、変化《動》
\\	その約束はまだ生きている。 ・
\\	少々[少し・ちょっと]お待ちいただけますか?◆【場面】電話を取り次ぐとき ・
\\	すみません。キャッチホンが入ったので少しお待ちいただけますか?◆電話 ・
\\	彼は別の電話に出ています。お待ちいただけますか?◆電話 ・
\\	「金返してくれよ。100ドル貸してからもう2週間近くたつぞ」「今、金ないんだよ。ちょっと待ってくれよ」 ・
\\	彼は私の喉{のど}にナイフを突きつけた。 ・
\\	私は彼に何の偏見も持っていない。 ・
\\	を押しながら
\\	を押すと、ファイルをセーブできます。 ・
\\	彼に抱き締められてちょっと焦ったんじゃないの? ・
\\	その米国出願は放棄されたものと見なされるだろう。 ・
\\	〔重い物を支えたりするのが〕そろそろ限界です。/〔おしっこが〕漏れそうだ。 ・
\\	このボトルには1リットルの水が入る。 ・
\\	この水筒は容量が1ガロンである。 ・
\\	チーズバーガーをケチャップ抜きで下さい。
\\	変化形 
\\	"【1自動】持ちこたえる、耐える、~の状態{じょうたい}のままである〔天候{てんこう}などが〕持続{じぞく}する〔~に〕有効{ゆうこう}である、適用{てきよう}できる ・
\\	その約束はまだ生きている。電話{でんわ}を切らずに待つ ・
\\	少々[少し・ちょっと]お待ちいただけますか?◆【場面】電話を取り次ぐとき ・
\\	すみません。キャッチホンが入ったので少しお待ちいただけますか?◆電話 ・
\\	彼は別の電話に出ています。お待ちいただけますか?◆電話〔ボタンなどが〕しっかり留まっている〈米黒人俗〉お金を持っている◆【文法】常に現在進行形で用いられ、
\\	'とつづられることが多い。 ・
\\	「金返してくれよ。100ドル貸してからもう2週間近くたつぞ」「今、金ないんだよ。ちょっと待ってくれよ」【1他動】~を手に持つ、握る ・
\\	彼は私の喉{のど}にナイフを突きつけた。 ・
\\	私は彼に何の偏見も持っていない。 ・
\\	を押しながら
\\	を押すと、ファイルをセーブできます。支える、縛り付ける抱き締める ・
\\	彼に抱き締められてちょっと焦ったんじゃないの?〔パーティーなどを〕開く[開催{かいさい}する]、〔式を〕挙げる〔ある場所に一定期間〕とどめておく、~の状態にしておく〔考えなどを〕心に抱く、~であると考える ・
\\	その米国出願は放棄されたものと見なされるだろう。〔人・動物{どうぶつ}を〕抑える、制する、拘束{こうそく}する維持{いじ}する、保持{ほじ}する、持続{じぞく}する ・
\\	〔重い物を支えたりするのが〕そろそろ限界です。/〔おしっこが〕漏れそうだ。保留{ほりゅう}する〔容器{ようき}が液体{えきたい}などを〕~だけ入れることができる ・
\\	このボトルには1リットルの水が入る。 ・
\\	この水筒は容量が1ガロンである。〈俗〉〔料理{りょうり}から嫌いな材料{ざいりょう}を〕抜く ・
\\	チーズバーガーをケチャップ抜きで下さい。【1名】つかむ[握る]ことつかむ[持つ]ところ、取っ手〔レスリングなどの技の〕ホールド、押さえ込み〔岩登りなどの〕ホールド、手[足]掛かりせき止めておくもの、水門{すいもん}、容器{ようき}〔人の心を〕つかんで[把握{はあく}して]いること、掌握(力){しょうあく(りょく)}〔強い〕支配(力){しはい(りょく)}、影響(力){えいきょう(りょく)}〔完全{かんぜん}な〕理解(力){りかい(りょく)}、把握(力){はあく(りょく)}〔一時的{いちじ てき}な〕停止{ていし}、遅れ{おくれ}、保留{ほりゅう}《音楽》フェルマータ(記号{きごう})◆【同】
\\	《野球》ホールド◆【略】
\\	◆リリーフピッチャーがセーブのできるところで登板し、一人以上のバッター(1/3回以上)を倒し、リードを守って退場すること。◆【語源】
\\	の
\\	(ジョン・デワン)と
\\	(マイク・オドネル)の造語(1986年)刑務所{けいむしょ}、拘置{こうち}(所)、拘留{こうりゅう}〈古〉とりで、要塞{ようさい}◆【同】
\\	【2名】《海事》船倉{せんそう}◆甲板の下の貨物室。〔航空機{こうくうき}の〕貨物室{かもつ しつ}レベル1、発音
\\	、カナホールド、ホウルド、変化《動》
\\	雷は高い物に落ちやすい。 ・
\\	天災は忘れた頃にやって来る。◆ことわざ ・
\\	ついに取引[(売買)契約]が成立しました。 ・
\\	時計の鐘が6時を告げました。 ・
\\	《野球》ノーストライク・スリーボールです。	変化形 
\\	【自動】〔地震{じしん}・雷などが〕襲う ・
\\	雷は高い物に落ちやすい。 ・
\\	天災は忘れた頃にやって来る。◆ことわざストライキをする決める、突然思い付く〔拳{こぶし}や道具{どうぐ}で〕打つ、なぐる、たたく〔軍隊{ぐんたい}が〕攻撃{こうげき}を開始{かいし}する〔動物{どうぶつ}が攻撃{こうげき}しようと不意{ふい}に〕飛びかかる〔寒さなどが〕しみ渡る、突き刺す〔新たな方向{ほうこう}に〕出発{しゅっぱつ}する、進む、動きだす【他動】〔拳{こぶし}や道具を使って〕~を突く[打つ]~を突き当てる[ぶつける]~を削除{さくじょ}する[切り取る]〔スプーンから粉末状{ふんまつ じょう}のものを〕擦り切る(人)の心を打つ、(人)を感動{かんどう}させる、(人)に印象{いんしょう}を与える〔恐怖心{きょうふしん}を〕引き起こす〔病気{びょうき}が人を〕襲う、悩ます突然~の状態にする〔姿勢{しせい}や態度{たいど}を〕とる〔契約などを〕取り決める、締結{ていけつ}する ・
\\	ついに取引[(売買)契約]が成立しました。〔時計{とけい}が正時{しょうじ}などを〕音で知らせる ・
\\	時計の鐘が6時を告げました。〔硬貨{こうか}やメダルなどを〕鋳造{ちゅうぞう}する【名】ストライキ、同盟罷業{どうめい ひぎょう}打つこと、攻撃{こうげき}、(集中的{しゅうちゅうてき}な)空爆{くうばく}あたり、魚信{ぎょしん}、食い《野球》ストライク ・
\\	《野球》ノーストライク・スリーボールです。《地層》走向{そうこう}《ボウリング》ストライク◆フレームの1投目で全部のピンを倒すこと。◆【参考】
\\	レベル2、発音
\\	、カナストゥライク、ストライク、変化《動》
\\	または
\\	私は、涙が出ました。	変化形 
\\	【1自動】涙が出る ・
\\	私は、涙が出ました。【1名】涙、涙液{るいえき}◆通例複数形で
\\	〔涙状の〕滴{しずく}、水滴{すいてき}
\\	悲しみ、悲嘆{ひたん}レベル2、発音
\\	、カナティア、変化《複》
\\	、《動》
\\	【2自動】裂ける、破れる急ぐ、慌てる【2他動】~を引き裂く[ちぎる]〔裂け目などを〕開ける、作る〔皮膚{ひふ}などに〕切り傷[裂傷{れっしょう}]を作る〔無理{むり}に〕~を引きはがす[離す]〔組織{そしき}などを〕分裂{ぶんれつ}[分離{ぶんり}]させる(人)を苦しめる[悲しませる]【2名】引き裂く[ちぎる]こと裂け目、割れ目急ぐ[慌てる]こと発音
\\	、カナテア、変化《動》
\\	地震で家はかなり揺れた。 ・
\\	嫌な予感がして仕方ない。/不安を拭い切れない。 ・
\\	彼を揺すって真っすぐに立たせた。 ・
\\	大体時速60マイルでハンドルが震動するのが分かる。 ・
\\	大きな地震が起きたのはその旅行中だった。 ・
\\	年輪の間にある小さな割れ目が見えますか。 ・
\\	彼のシェイクとビブラートとトレモロの使い方はエルビス・プレスリーを思い起こさせる。 ・
\\	大急ぎでミルクセーキをがぶ飲みした。 ・
\\	ぬれた木のこけら屋根はとても滑りやすい。 ・
\\	ストレスで震えが起きるときがある。 ・
\\	その折衝の場で公平な扱いを受けないのではないかと思っている。	変化形 
\\	【自動】震える、振動{しんどう}する、揺れる ・
\\	地震で家はかなり揺れた。【他動】~を振る、振動{しんどう}させる、揺する、揺るがす~を振り落とす~を動揺{どうよう}させる、〔信念{しんねん}を〕ぐらつかせる、〔勇気{ゆうき}を〕くじく、~をうろたえさせる ・
\\	嫌な予感がして仕方ない。/不安を拭い切れない。~を震撼{しんかん}させる〔人から〕うまく逃れる、〔尾行{びこう}を〕まく、〔追跡{ついせき}などを〕振り切る【名】振ること、揺すること ・
\\	彼を揺すって真っすぐに立たせた。振動{しんどう}、震動{しんどう}、震え ・
\\	大体時速60マイルでハンドルが震動するのが分かる。〈話〉地震{じしん} ・
\\	大きな地震が起きたのはその旅行中だった。〔岩や木にできた〕裂け目、割れ目 ・
\\	年輪の間にある小さな割れ目が見えますか。〈話〉瞬間{しゅんかん}◆【参考】
\\	〈英〉《音楽》シェイク、顫音{せんおん}◆【同】
\\	彼のシェイクとビブラートとトレモロの使い方はエルビス・プレスリーを思い起こさせる。ミルクセーキ ・
\\	大急ぎでミルクセーキをがぶ飲みした。《建築》〔粗末{そまつ}な木の〕こけら、シェイク ・
\\	ぬれた木のこけら屋根はとても滑りやすい。
\\	〔恐怖{きょうふ}や病気{びょうき}などにより止まらない〕震え、悪寒{おかん} ・
\\	ストレスで震えが起きるときがある。〈俗〉〔妥当{だとう}な〕取り扱い、〔成功{せいこう}の〕機会{きかい} ・
\\	その折衝の場で公平な扱いを受けないのではないかと思っている。〈米俗〉マリフアナ◆つぼみを落とした葉または葉の一部を指す。レベル1、発音
\\	、カナシェイク、変化《動》
\\	参ったなあ、
\\	を入れ直す必要があるよ。 ・
\\	その猫はネズミをかみ殺しました。 ・
\\	「一口もらっていい?」「どうぞ」 ・
\\	一口かじってごらん。 ・
\\	すてきな朝食だったのに、彼は一口も食べなかった。
\\	変化形 
\\	"【自動】かみつく話に乗る〈俗〉悪いニュースだ、ひどくこたえる、手痛い◆困惑・苛立{いらだ}ちなど。◆【参考】
\\	参ったなあ、
\\	を入れ直す必要があるよ。〈米俗〉最低である◆【同】
\\	【他動】かむ、かみ切る、かみつく、刺す、刺激{しげき}する ・
\\	その猫はネズミをかみ殺しました。〈米俗〉〔他人{たにん}のアイデアなどを〕パクる、盗む、盗用{とうよう}する【名】かむこと、かみつくこと、咬合{こうごう}かまれた跡、刺傷{ししょう}、咬傷{こうしょう}ひとかじり、〔かみ切った〕一片 ・
\\	「一口もらっていい?」「どうぞ」 ・
\\	一口かじってごらん。 ・
\\	すてきな朝食だったのに、彼は一口も食べなかった。軽い食事{しょくじ}切れ味、鋭さ、辛辣{しんらつ}さ〈話〉分け前、分担額、支払額〈話〉〔全額から〕差し引かれる分〔魚の〕食いレベル4、発音
\\	、カナバイト、変化《動》
\\	私はコンピューターと相性が悪い。/私はコンピューターが苦手です。	変化形 
\\	【自動】混ざる交際{こうさい}する、付き合う〔衣服のサイズ・相性などが〕合う、適合する ・
\\	私はコンピューターと相性が悪い。/私はコンピューターが苦手です。【他動】混ぜる、混ぜ合わせる、調和させるごちゃ混ぜにする交際{こうさい}させる【名】〔物同士{ぶつ どうし}を〕混ぜること、混合{こんごう}〔異なる要素{ようそ}を〕混ぜ合わせたもの、融合体{ゆうごう たい}〔粉末{ふんまつ}の〕素、ミックス◆市販されている、水などを加えるだけで出来上がる粉末食品。〔レコーディングの〕ミキシング◆別々に録音されたトラックを組み合わせて音楽を作ること。〔アレンジされた曲の〕ミックス◆リミックス
\\	することによって新しく作り出された音楽で、
\\	や
\\	のように複合語で用いられる。《建築》〔モルタルやコンクリートの〕配合比{はいごう ひ}レベル2、発音
\\	、カナミックス、変化《動》
\\	やかんの湯が沸騰しています。 ・
\\	私は義憤を感じた。 ・
\\	彼の言葉は私を激怒させた。 ・
\\	煎じ詰めれば、こういうことです。 ・
\\	急いで台所に駆け込みカラカラになった鍋の火を止めた。 ・
\\	とんでもなく暑いよ! ・
\\	彼女は私に卵をゆでてくれた。 ・
\\	一番むかつくのは彼女の態度です。	変化形 
\\	【1自動】沸く、沸騰{ふっとう}する、沸点{ふってん}に達する、〔ぐつぐつと〕煮える ・
\\	やかんの湯が沸騰しています。 ・
\\	私は義憤を感じた。 ・
\\	彼の言葉は私を激怒させた。結局{けっきょく}(~ということに)なる、煮詰まる、要約{ようやく}する◆【用法】
\\	煎じ詰めれば、こういうことです。沸騰{ふっとう}してなくなる◆【用法】
\\	急いで台所に駆け込みカラカラになった鍋の火を止めた。激怒{げきど}する、カンカンに怒る、思わずカッとなる◆【用法】
\\	ゆだる、暑くて死にそうである ・
\\	とんでもなく暑いよ!【1他動】〔容器{ようき}や液体{えきたい}を〕沸かす、沸騰{ふっとう}させる、煮沸{しゃふつ}する、煮沸消毒{しゃふつ しょうどく}する〔ぐつぐつと〕煮る、ゆでる ・
\\	彼女は私に卵をゆでてくれた。〔飯を〕炊く激怒{げきど}させる、カンカンに怒らせる、激高{げっこう}させる ・
\\	一番むかつくのは彼女の態度です。【1名】沸騰(状態)、沸点◆常に単数形で用いる【2名】炎症性{えんしょうせい}の腫れ物、おできレベル2、発音
\\	、カナボイル、変化《動》
\\	木は燃えやすい。 ・
\\	私の肌は日焼けしやすいんです。 ・
\\	トーストが焦げちゃった。 ・
\\	パイが焦げているよ。 ・
\\	恥ずかしくて顔から火が出る思いだった。 ・
\\	君のうわさをしていたんだよ、くしゃみが出なかった?◆アメリカではうわさをすると、本人の耳が熱くなる、といわれる。 ・
\\	データ
\\	や
\\	を焼くことができます。 ・
\\	ディクスに音楽を焼くことができます。	変化形 
\\	【1自動】燃える、燃焼{ねんしょう}する ・
\\	木は燃えやすい。熱くなる、〔強いお酒を飲んで舌・口などが〕ヒリヒリする、〔皮膚{ひふ}が焼けて〕ヒリヒリする、赤くなる、かっかする、ピリピリする ・
\\	私の肌は日焼けしやすいんです。〔料理{りょうり}・物が〕焦げる ・
\\	トーストが焦げちゃった。 ・
\\	パイが焦げているよ。〔身体の一部が〕火照る{ほてる} ・
\\	恥ずかしくて顔から火が出る思いだった。 ・
\\	君のうわさをしていたんだよ、くしゃみが出なかった?◆アメリカではうわさをすると、本人の耳が熱くなる、といわれる。〔考えが〕焼き付く【1他動】~を焼く、焼きこがす、燃やす、焼却{しょうきゃく}する〔身体の一部を〕やけどさせる〈俗〉殺す、射殺{しゃさつ}する(人)の気持ち{きもち}を傷つける、(人)を利用{りよう}して見捨{みす}てる
\\	などのディスクを〕焼く ・
\\	データ
\\	や
\\	を焼くことができます。
\\	などのディスクにデータを〕焼く ・
\\	ディクスに音楽を焼くことができます。【1名】日焼け{ひやけ}、やけど、熱傷{ねっしょう}〈米俗〉詐欺行為{さぎ こうい}〈米俗〉侮辱{ぶじょく}焼け(焦がし)《コ》
\\	を焼くこと【2名】〈スコット・北イング〉中小河川{ちゅうしょう かせん}レベル2、発音
\\	、カナバーン、変化《動》
\\	または
\\	第一撃が勝負を半ば左右する。/先んずれば人を制す。 ・
\\	大統領補佐官の一人が汚職事件で辞任に追い込まれた。これは支持率低迷と戦っている大統領にとっては大きなダメージである。 ・
\\	売り上げがここ数年間激減しているそのアパレル会社にとっては、人気デザイナーが同社を離れたことは大きな打撃である。 ・
\\	彼の母親の死は、大きな心の打撃となった。 ・
\\	何だよ、くそ。わざわざここまで来たのに無駄骨かよ。 ・
\\	偏東風が海の表層の温かい水を吹き寄せる。 ・
\\	風で扉が閉まった。 ・
\\	どういう風の吹き回しでここに来たのですか? ・
\\	映画のオーディションでどじっちゃった。 ・
\\	失敗したなあ! ・
\\	私が率いていた小隊は喜んで一休みするために立ち止まった。	変化形 
\\	【1名】殴打{おうだ}、一撃{いちげき}、強打{きょうだ}、殴り合い ・
\\	第一撃が勝負を半ば左右する。/先んずれば人を制す。〔精神的な〕打撃{だげき}、ダメージ、ショック ・
\\	大統領補佐官の一人が汚職事件で辞任に追い込まれた。これは支持率低迷と戦っている大統領にとっては大きなダメージである。 ・
\\	売り上げがここ数年間激減しているそのアパレル会社にとっては、人気デザイナーが同社を離れたことは大きな打撃である。 ・
\\	彼の母親の死は、大きな心の打撃となった。〔不意の〕攻撃、襲撃【2自動】〔風が〕吹く〔物が風に〕吹かれる〔笛などが〕鳴る〔ヒューズが〕飛ぶ〔タイヤが〕パンクする爆発{ばくはつ}する〔クジラが〕潮を吹く〈俗〉激怒{げきど}する、カッとなる〈俗〉急いで立ち去る息を吐き出す、鼻をかむ〈俗〉〔事態などが〕非常に嫌だ◆【同】
\\	何だよ、くそ。わざわざここまで来たのに無駄骨かよ。【2他動】~を吹く ・
\\	偏東風が海の表層の温かい水を吹き寄せる。 ・
\\	風で扉が閉まった。 ・
\\	どういう風の吹き回しでここに来たのですか?~を吹き飛ばす、~を爆破{ばくは}する~を発表{はっぴょう}する〔せりふなどを〕とちる、忘れる、しくじる、失敗する ・
\\	映画のオーディションでどじっちゃった。 ・
\\	失敗したなあ!〔今までに得たもの・達成したことなど〕ふいにする、台無{だいな}しにする吹奏{すいそう}する〈俗〉〔~に金を〕浪費する〈俗〉自慢{じまん}する〈俗〉〔マリファナ・コカインなどを〕吸引{きゅういん}、吸入{きゅうにゅう}する【2名】〔空気・風などの〕一吹き鼻をかむこと〈米話〉自慢{じまん}〈俗〉コカイン〈話〉一休み{ひとやすみ} ・
\\	私が率いていた小隊は喜んで一休みするために立ち止まった。〔クジラなどの〕潮吹き{しおふき}【3自動】花が咲く【3他動】〔花を〕咲かせるレベル1、発音
\\	、カナブロー、ブロウ、変化《動》
\\	「東芝のラップトップ
\\	-3100を今日注文したよ」「キャンセルしたほうがいいよ。新バージョンの
\\	-3100が来月でるんだから」 ・
\\	隣の家のペットのイグアナが、私の家の前庭にいたの。だから、郵便受けまで行くのをやめたの。
\\	変化形 
\\	"【自動】輝く、光る◆過去形=
\\	、過去分詞形=
\\	異彩{いさい}を放つ【他動】~を磨く、光らせる◆過去形=
\\	、過去分詞形=
\\	〈俗〉~をキャンセルする、(~すること)をとりやめる◆【類】
\\	「東芝のラップトップ
\\	-3100を今日注文したよ」「キャンセルしたほうがいいよ。新バージョンの
\\	-3100が来月でるんだから」 ・
\\	隣の家のペットのイグアナが、私の家の前庭にいたの。だから、郵便受けまで行くのをやめたの。【名】〔光源{こうげん}からの〕光、明るさ、輝き〔反射光{はんしゃこう}の〕輝き、光沢{こうたく}〈俗・軽蔑的〉黒人{こくじん}〔天気{てんき}の〕晴れ
\\	〈話〉いたずら、悪ふざけ〈俗〉ウイスキー◆【同】
\\	〈米俗〉アクセサリー、装飾品{そうしょくひん}レベル1、発音
\\	、カナシャイン、変化《動》
\\	私は、しばらくじっくり考える時間が必要です。 ・
\\	彼らの業績は、わが校の名誉となるものです。 ・
\\	その事件は、大統領のイメージを悪くするものでした。	変化形 
\\	【自動】真剣{しんけん}に[じっくり]考える、内省{ないせい}する、自分{じぶん}と向かい合う◆【参考】
\\	私は、しばらくじっくり考える時間が必要です。評価{ひょうか}[評判{ひょうばん}]をもたらす◆主に
\\	良しあしを表す副詞 
\\	評価を受ける対象」の形で。◆【参考】
\\	彼らの業績は、わが校の名誉となるものです。 ・
\\	その事件は、大統領のイメージを悪くするものでした。【他動】~を反射{はんしゃ}する、反響{はんきょう}する~の像を映す~を映す、示す、反映{はんえい}する~を熟考{じゅっこう}する、思案{しあん}するレベル3、発音
\\	、カナリフレクタ、レフレクト、リフレクト、変化《動》
\\	、分節
\\	バルブは閉じたままだ。 ・
\\	被害者の数は増え続けている。	変化形 
\\	【自動】続く、継続{けいぞく}[持続{じぞく}・存続{そんぞく}]する進み[歩み]続ける再開{さいかい}する〔途切れることなく〕続いている依然{いぜん}~のままである ・
\\	バルブは閉じたままだ。〔役職{やくしょく}に〕とどまる【他動】~を続ける、継続{けいぞく}[持続{じぞく}・存続{そんぞく}]する ・
\\	被害者の数は増え続けている。~を(途切れることなく)延ばす~を再開{さいかい}する、引き続き述べる~を留任{りゅうにん}させる、とどまらせるレベル2、発音
\\	、カナコンティニュー、コンテニュー、変化《動》
\\	、分節
\\	配膳は入来の順。 ・
\\	最近彼女に会ったのはいつ? ・
\\	その戦争は9年3カ月続いた。 ・
\\	どうせ[どのみち]永遠に続くものなどありはしない[この世にはない]。 ・
\\	《マニュアル》1個のインクカートリッジで約20時間の印刷が行えます。	変化形 
\\	【1名】〔順序{じゅんじょ}が〕最後{さいご}のもの[こと・人]〔時間{じかん}や時期{じき}の〕末、終わり、終末{しゅうまつ}〔時間的{じかん てき}に〕すぐ前の出来事{できごと}[言及{げんきゅう}]〔二度{にど}とない〕最後{さいご}の姿[言葉{ことば}・出現{しゅつげん}]〔適切{てきせつ}さが〕最もふさわしく[ありそうも]ないもの[こと・人]【1形】〔順序{じゅんじょ}が〕最後{さいご}の、一番後の最後{さいご}に残った、とっておきの〔時間的{じかん てき}に〕すぐ前の、最近{さいきん}の〔時間的{じかん てき}に〕一番遅い、ぎりぎりの〔生命{せいめい}の〕最期{さいご}の、末期{まっき}の最新{さいしん}の、一番新{いちばん あたら}しい最大限{さいだいげん}の、最高度{さいこう ど}の最も起こりそうも[ありそうに]ない最も望ましく[ふさわしく]ない〔重要性{じゅうよう せい}などが〕最低{さいてい}の、一番下{いちばん した}の【1副】〔順序が〕最後{さいご}に、一番終わりに ・
\\	配膳は入来の順。この前に、最近 ・
\\	最近彼女に会ったのはいつ?終わりに当たり、最後{さいご}になるが【2自動】続く、存続{そんぞく}する、持続{じぞく}する ・
\\	その戦争は9年3カ月続いた。 ・
\\	どうせ[どのみち]永遠に続くものなどありはしない[この世にはない]。耐える、持ちこたえる必要{ひつよう}を満たす、足りる、間に合う、十分{じゅうぶん}である ・
\\	《マニュアル》1個のインクカートリッジで約20時間の印刷が行えます。【2他動】~を存続させる、~を持ちこたえる【2名】長続きすること、持続力{じぞく りょく}、耐久力{たいきゅうりょく}、根気{こんき}【3他動】〔靴型{くつがた}を使って靴の〕型を取る【3名】靴型{くつがた}レベル1、発音
\\	、カナラスト、ラースト、変化《動》
\\	どうにか生きてますよ。 ・
\\	祖母は祖父より5年長生きした。 ・
\\	彼は一人の息子を残して亡くなりました。	変化形 
\\	【自動】〔危険{きけん}などを〕生き残る[延びる]〔困難{こんなん}な状況{じょうきょう}で〕何とかやっていく ・
\\	どうにか生きてますよ。存続{そんぞく}する、存在{そんざい}し続ける【他動】〔人より〕長生きする ・
\\	祖母は祖父より5年長生きした。 ・
\\	彼は一人の息子を残して亡くなりました。〔危機などを〕切り抜けて生き残る、乗り切るレベル3、発音
\\	、カナサーバイブ、サバイブ、変化《動》
\\	、分節
\\	彼は命の恩人です。 ・
\\	あなたは私の命を救ってくれたんだから、それはただにしておくよ。 ・
\\	宿題は、週末にやることにしてるんだ。 ・
\\	できればそれを残すつもりでした。 ・
\\	《結婚しない理由を聞かれて》これといういい人が現れるまで自分を取っておいてあるのです。 ・
\\	真打ちは最後にあります。 ・
\\	ディスク領域とメモリを確保したいなら、よく使うプログラムだけインストールすればいい。 ・
\\	この貯金箱には小遣いからためた15ドルが入っているんだ。 ・
\\	今日の1針、明日の10針。/早いうちに手を打てば手遅れになってから大騒ぎしなくて済む。/転ばぬ先の杖。◆ことわざ◆小さなほころびも大きくなってから縫うのは大変であるという例えから ・
\\	このクーポン券を使えば3ドルの節約になります。 ・
\\	お得ですよ。/お買い得です。 ・
\\	このサービスをご利用になればさらにお得になります。 ・
\\	今すぐお電話いただければ5割引です。◆通販 ・
\\	ありがとう。とても助かりました。◆助けてくれた人に対して。 ・
\\	おまえの説教など聞きたくない。 ・
\\	同情はいらない。 ・
\\	慰めはいらない。	変化形 
\\	【1自動】貯金{ちょきん}する倹約{けんやく}する〔食べ物が〕持つ《野球》セーブする《サッカー》ゴールを守る【1他動】救う、助ける ・
\\	彼は命の恩人です。 ・
\\	あなたは私の命を救ってくれたんだから、それはただにしておくよ。名誉{めいよ}を保つ確保{かくほ}しておく、取っておく、残しておく ・
\\	宿題は、週末にやることにしてるんだ。 ・
\\	できればそれを残すつもりでした。 ・
\\	《結婚しない理由を聞かれて》これといういい人が現れるまで自分を取っておいてあるのです。 ・
\\	真打ちは最後にあります。 ・
\\	ディスク領域とメモリを確保したいなら、よく使うプログラムだけインストールすればいい。お金を蓄える、貯金{ちょきん}する、貯蓄{ちょちく}する ・
\\	この貯金箱には小遣いからためた15ドルが入っているんだ。〔コンピュータープログラム・データファイル・文書{ぶんしょ}をフロッピーディスクやハードディスクに〕~を記録{きろく}する、セーブする、保存{ほぞん}する~を節約{せつやく}する、割り引く、出費{しゅっぴ}を減じる ・
\\	今日の1針、明日の10針。/早いうちに手を打てば手遅れになってから大騒ぎしなくて済む。/転ばぬ先の杖。◆ことわざ◆小さなほころびも大きくなってから縫うのは大変であるという例えから ・
\\	このクーポン券を使えば3ドルの節約になります。 ・
\\	お得ですよ。/お買い得です。 ・
\\	このサービスをご利用になればさらにお得になります。 ・
\\	今すぐお電話いただければ5割引です。◆通販〔苦労{くろう}・手間{てま}などを〕省く ・
\\	ありがとう。とても助かりました。◆助けてくれた人に対して。〔気遣いなどを〕省く ・
\\	おまえの説教など聞きたくない。 ・
\\	同情はいらない。 ・
\\	慰めはいらない。《野球》セーブを記録{きろく}する【1名】保存{ほぞん}、退避{たいひ}《野球》セーブ(数)◆【略】
\\	《サッカー》ゴールキーパーがゴールを防ぐ[守る]こと【2前】~を除いて、~のほかは、~は別として、~以外{いがい}の、ただし~は除外{じょがい}してレベル1、発音
\\	、カナセイブ、変化《動》
\\	《コ》文書は保護されています。	変化形 
\\	【自動】保護{ほご}する【他動】〔被害{ひがい}を受けないよう〕~を保護{ほご}する[守る・かばう] ・
\\	《コ》文書は保護されています。〔関税{かんぜい}などで国内産業{こくない さんぎょう}を〕保護{ほご}する〈英〉〔手形{てがた}などの〕支払金{しはらいきん}を準備{じゅんび}する◆基金を取り崩すことで資金を用意する。レベル2、発音
\\	、カナプロテクト、変化《動》
\\	、分節
\\	カクテル・パーティーは午後7時から、またディナーは8時からです。 ・
\\	御社と御社のお客様のために、いつでも喜んでお役に立ちたいと存じます。 ・
\\	皆さまにでき得る限りのサービスをさせていただきます。	変化形 
\\	【自動】仕える、勤務{きんむ}する、勤める食事{しょくじ}[飲み物]を出す、給仕{きゅうじ}をする〔~の〕役目{やくめ}[目的{もくてき}]を果たす、役に立つ刑期{けいき}を務める、服役{ふくえき}する_人分である【他動】~を供給{きょうきゅう}する、(食事{しょくじ}・飲み物を)出す ・
\\	カクテル・パーティーは午後7時から、またディナーは8時からです。(人)に仕える、(人)のために働く[奉公{ほうこう}する・尽くす] ・
\\	御社と御社のお客様のために、いつでも喜んでお役に立ちたいと存じます。 ・
\\	皆さまにでき得る限りのサービスをさせていただきます。~に役立つ{やくだつ}、(目的{もくてき}を)果たす(人)を扱う[遇する・待遇{たいぐう}する]、(人)に報いる〔法的な文書〕を送達{そうたつ}する【名】《球技》サーブレベル2、発音
\\	、カナサーブ、変化《動》
\\	われわれは生命・自由・幸福の追求という米国の理想を守るために変化を求める。	変化形 
\\	【他動】~を保つ、保存{ほぞん}する、貯蔵{ちょぞう}する、保護{ほご}する、失わないようにする ・
\\	われわれは生命・自由・幸福の追求という米国の理想を守るために変化を求める。~を維持{いじ}[持続{じぞく}]する、守る、順守{じゅんしゅ}する【名】保存{ほぞん}するもの禁猟地{きんりょうち}、自然保護地域{しぜん ほご ちいき}、生簀{いけす}領分{りょうぶん}、領域{りょういき}貯蔵食料{ちょぞう しょくりょう}、ジャム、〔瓶詰{びん づ}めの〕果物{くだもの}の砂糖煮{さとう に}レベル4、発音
\\	、カナプリザーブ、変化《動》
\\	、分節
\\	変化形 
\\	【他動】~を用意{ようい}しておく、予約{よやく}する、取っておく、蓄えておく、予定{よてい}しておく~を蓄えておく〔権利を〕有する、保有{ほゆう}する、留保{りゅうほ}する【名】蓄え、予備品{よびひん}蓄積{ちくせき}、積立(金){つみたて(きん)}、準備金{じゅんびきん}、引当金{ひきあてきん}〈文〉遠慮{えんりょ}、控えめな態度{たいど}予備軍{よびぐん}、増援部隊{ぞうえん ぶたい}保存{ほぞん}、保留{ほりゅう}特別保留地{とくべつ ほりゅう ち}、特別保有地{とくべつ ほゆう ち}、指定地{してい ち}、保護地{ほご ち}制限{せいげん}、条件{じょうけん}、限定{げんてい}〔石油{せきゆ}の〕埋蔵量{まいぞうりょう}《スポーツ》控え選手、補欠{ほけつ}【形】予備{よび}の、とっておきのレベル2、発音
\\	、カナリザーブ、変化《動》
\\	、分節
\\	それらの大規模開発は空き地を急速に減少させている。	変化形 
\\	【自動】破壊{はかい}される、荒廃{こうはい}する、消耗{しょうもう}する消費{しょうひ}する、購入{こうにゅう}する【他動】〔商品{しょうひん}やサービスを〕消費{しょうひ}する、購入{こうにゅう}する〔二度{にど}と使えないよう〕~を使い切る[使い果たす]◆【同】
\\	それらの大規模開発は空き地を急速に減少させている。〔火災{かさい}や病気{びょうき}などで〕~を破壊{はかい}する[消耗{しょうもう}させる]〔大量{たいりょう}の食物{しょくもつ}などを〕摂取{せっしゅ}する、吸収{きゅうしゅう}する〔人の気持{きも}ちや注意{ちゅうい}を〕奪う、夢中{むちゅう}にさせる◆【用法】受け身で用いられることが多い。レベル5、発音
\\	、カナコンシューム、コンスューム、変化《動》
\\	、分節
\\	恋愛は若者に無駄遣いされる。 ・
\\	私の努力は無駄ではなかった。 ・
\\	急げ! 彼をやっつけろ! ・
\\	もったいない!/何という無駄だ! ・
\\	もったいないよ。	変化形 
\\	【自動】浪費[無駄使い]される、無駄になる〔体力・権力などが〕衰弱する、痩{や}せ衰える〔物が〕徐々に消耗する[減る]【他動】~を無駄{むだ}にする、空費{くうひ}する、浪費{ろうひ}する、無駄遣いする、消耗{しょうもう}する、すり減らす、(スペースを)無駄{むだ}に消費{しょうひ}する ・
\\	恋愛は若者に無駄遣いされる。 ・
\\	私の努力は無駄ではなかった。~を(みすみす)逃す~をやっつける、殺す、完全{かんぜん}にぶちのめす ・
\\	急げ! 彼をやっつけろ!~を壊す、破壊{はかい}する◆【同】
\\	【名】浪費{ろうひ}、無駄{むだ}、消耗{しょうもう}、衰弱{すいじゃく} ・
\\	もったいない!/何という無駄だ! ・
\\	もったいないよ。廃棄物{はいきぶつ}、廃石{はいせき}、くず《医》排せつ物{はいせつぶつ}荒れ地〔体内{たいない}でできる〕老廃物{ろうはいぶつ}【形】廃棄された、不用な荒れ果てた、荒れた、荒廃したレベル2、発音
\\	、カナウェイスト、ウエイスト、変化《動》
\\	《マニュアル》ほこりや汚れがあると、インクシステムを損傷する可能性があります。 ・
\\	どの程度の損害でしたか? ・
\\	いくら?/どのくらいかかる?◆費用を尋ねる	変化形 
\\	【他動】~にダメージ[損害{そんがい}・損傷{そんしょう}]を与える、~を損なう、損傷{そんしょう}する、(名誉{めいよ}・体面{たいめん}・評判{ひょうばん}を)傷つける ・
\\	《マニュアル》ほこりや汚れがあると、インクシステムを損傷する可能性があります。【名】〔人や物の価値を減らす〕損害{そんがい}、被害{ひがい}、ダメージ、損傷{そんしょう} ・
\\	どの程度の損害でしたか?
\\	損害賠償(金){そんがい ばいしょう(きん)}〈米俗〉コスト、費用{ひよう}、経費{けいひ} ・
\\	いくら?/どのくらいかかる?◆費用を尋ねるレベル2、発音
\\	、カナダミッジ、ダメージ、ダメッジ、変化《動》
\\	、分節
\\	それは、助けになるどころか返って裏目に出ることもある。 ・
\\	悪意はありませんでした。	変化形 
\\	【他動】〔人や物事を〕害する、損害{そんがい}を与える、危害{きがい}を加える ・
\\	それは、助けになるどころか返って裏目に出ることもある。【名】〔精神{せいしん}・肉体的{にくたい てき}〕痛手{いたで}、損傷{そんしょう}〔精神{せいしん}・肉体的{にくたい てき}〕悪意{あくい}、仕打{しうち}ち ・
\\	悪意はありませんでした。〔物質的{ぶっしつ てき}な〕損害{そんがい}、損失{そんしつ}レベル2、発音
\\	、カナハーム、変化《動》
\\	右目が痛い。 ・
\\	まだ痛みますか? ・
\\	冷たいものを食べたり飲んだりすると、歯が痛みます[しみます]。 ・
\\	真実ほどこたえるものはない。 ・
\\	それはつらいね[こたえるね]。 ・
\\	その支払いは本当に応えるよ。 ・
\\	彼女が家にいるか電話して確かめてみたら? ・
\\	~してもいい。 ・
\\	金も大事です。/金は必要なものだ。/金はあったほうがいい。 ・
\\	減るもんじゃないんだから「ありがとう」くらい言ったら? ・
\\	喫煙は胎児を傷つける。 ・
\\	知らぬが仏。◆ことわざ ・
\\	景気の悪化により大きな痛手を受けている。 ・
\\	評判が悪くなると売り上げが落ち込む。
\\	変化形 
\\	"【自動】〔身体の一部が〕痛む ・
\\	右目が痛い。 ・
\\	まだ痛みますか? ・
\\	冷たいものを食べたり飲んだりすると、歯が痛みます[しみます]。〔精神的打撃によって心が〕痛む、苦しい、こたえる ・
\\	真実ほどこたえるものはない。 ・
\\	それはつらいね[こたえるね]。 ・
\\	その支払いは本当に応えるよ。害を与える、困ったことになる ・
\\	彼女が家にいるか電話して確かめてみたら? ・
\\	~してもいい。 ・
\\	金も大事です。/金は必要なものだ。/金はあったほうがいい。 ・
\\	減るもんじゃないんだから「ありがとう」くらい言ったら?【他動】~を傷つける ・
\\	喫煙は胎児を傷つける。 ・
\\	知らぬが仏。◆ことわざ~の感情{かんじょう}を損なう、~の感情{かんじょう}を害する~に苦痛{くつう}を与える~に損害{そんがい}を与える ・
\\	景気の悪化により大きな痛手を受けている。 ・
\\	評判が悪くなると売り上げが落ち込む。~を困らせる【名】傷、けが、苦痛{くつう}〔心の〕痛み、苦痛{くつう}悪意{あくい}、不正{ふせい}【形】けがをしたレベル1、発音
\\	、カナハート、変化《動》
\\	変化形 
\\	【他動】〔人や動物{どうぶつ}や体の一部{いちぶ}を〕傷つける、痛める、けがをさせる〔行為{こうい}や言葉{ことば}で人の心を〕苦しめる、悲しませる、傷つける〔人の評判{ひょうばん}などを〕おとしめる、傷つけるレベル4、発音
\\	、カナインジャー、インジュア、変化《動》
\\	、分節
\\	ガンを持った犯人は3人を殺し、さらに3人を負傷させた。 ・
\\	さらに何百人かの人が負傷しました。	変化形 
\\	【他動】〔銃器・刃物などで故意に〕~を傷つける ・
\\	ガンを持った犯人は3人を殺し、さらに3人を負傷させた。 ・
\\	さらに何百人かの人が負傷しました。【名】〔銃器・刃物などによる〕外傷{がいしょう}、創傷{そうしょう}心の傷レベル3、発音
\\	、カナウーンド、ワウンド、変化《動》
\\	恋をすることは悲しみを知ることである。 ・
\\	たくさん苦労するか若くして死ぬかだ。 ・
\\	学校へ戻るとすぐ彼は体調を崩してしまう。 ・
\\	苦しみは苦いが、その果実は甘い。 ・
\\	なんで女だけがこんなに耐えなきゃなんないの? ・
\\	そのサーファーは鮫に咬まれた。	変化形 
\\	【自動】〔肉体的{にくたい てき}・精神的{せいしん てき}に〕苦しむ、苦痛{くつう}を感じる、不快{ふかい}な経験{けいけん}をする、病気{びょうき}をする ・
\\	恋をすることは悲しみを知ることである。 ・
\\	たくさん苦労するか若くして死ぬかだ。 ・
\\	学校へ戻るとすぐ彼は体調を崩してしまう。 ・
\\	苦しみは苦いが、その果実は甘い。嫌なことに耐える、不快{ふかい}なことを我慢{がまん}する ・
\\	なんで女だけがこんなに耐えなきゃなんないの?弱点{じゃくてん}がある、劣っている【他動】〔痛みなどに〕苦しむ、〔嫌なことを〕経験{けいけん}する、〔損害{そんがい}などを〕被る ・
\\	そのサーファーは鮫に咬まれた。〔嫌なこと・不快{ふかい}なことなどを〕我慢{がまん}する、〔嫌なこと・不快{ふかい}なことなどに〕耐える〈古〉~を許すレベル2、発音
\\	、カナサファー、変化《動》
\\	、分節
\\	後で文句言うなよ。 ・
\\	君の愚痴はもう聞きたくないよ。 ・
\\	何も文句を言うことはない。 ・
\\	文句を言っているわけではないんです。/構わないけどね。 ・
\\	文句言っても始まらない。 ・
\\	泣き言を言っても始まりません[仕方ありません・むなしいだけです]。 ・
\\	妻に頭が痛いと言った。	変化形 
\\	【自動】不満{ふまん}[苦情{くじょう}・不平{ふへい}]を言う[漏らす]、文句{もんく}を言う、愚痴{ぐち}を言う[こぼす] ・
\\	後で文句言うなよ。 ・
\\	君の愚痴はもう聞きたくないよ。 ・
\\	何も文句を言うことはない。 ・
\\	文句を言っているわけではないんです。/構わないけどね。 ・
\\	文句言っても始まらない。 ・
\\	泣き言を言っても始まりません[仕方ありません・むなしいだけです]。〔病状{びょうじょう}や痛みを〕訴える、言う ・
\\	妻に頭が痛いと言った。〔苦情{くじょう}でも言うかのように〕うめく、きしみ音を出す〔正式{せいしき}な機関{きかん}に〕苦情{くじょう}を申し立てる、訴え出る、クレームを付ける〔コンピューターやソフトウェアが〕エラーを出す、言うことを聞かないレベル2、発音
\\	、カナコンプレイン、変化《動》
\\	、分節
\\	下手な職人は能力不足を道具のせいにする。◆ことわざ ・
\\	彼が怒るのも無理はない。 ・
\\	あなたがそう考えるのも無理はありません。 ・
\\	けんか両成敗。 ・
\\	悪い[いけない]のは君です。 ・
\\	世の中とはこんなもんだ。仕方がない。◆【直訳】誰のせい[責任]でもない。 ・
\\	自分のミスなのに、彼は私に罪をなすり付けるのだ。	変化形 
\\	【他動】非難{ひなん}する、とがめる、責める、~の責任{せきにん}にする、~のせいにする、責任{せきにん}を~に負わせる◆【反】
\\	下手な職人は能力不足を道具のせいにする。◆ことわざ ・
\\	彼が怒るのも無理はない。 ・
\\	あなたがそう考えるのも無理はありません。 ・
\\	けんか両成敗。 ・
\\	悪い[いけない]のは君です。 ・
\\	世の中とはこんなもんだ。仕方がない。◆【直訳】誰のせい[責任]でもない。 ・
\\	自分のミスなのに、彼は私に罪をなすり付けるのだ。〈米俗〉~をのろう【名】非難{ひなん}、責任{せきにん}、責めレベル2、発音
\\	、カナブレイム、変化《動》
\\	謝るよ。 ・
\\	おわびの申し上げようもございません。 ・
\\	何とも申し訳なく存じます[申し訳ないことです]。 ・
\\	昨日のこと、そんなに謝らなくていいよ。 ・
\\	トムに謝ったほうがいいよ。 ・
\\	あなたが謝ることなんて何もありません。謝るのは私の方です。	変化形 
\\	【自動】わびる、謝る、謝罪{しゃざい}する、陳謝{ちんしゃ}する◆【語源】
\\	(謝罪)
\\	(~する) ・
\\	謝るよ。 ・
\\	おわびの申し上げようもございません。 ・
\\	何とも申し訳なく存じます[申し訳ないことです]。 ・
\\	昨日のこと、そんなに謝らなくていいよ。 ・
\\	トムに謝ったほうがいいよ。 ・
\\	あなたが謝ることなんて何もありません。謝るのは私の方です。レベル4、発音
\\	、カナアパロジャイズ、アポロジャイズ、変化《動》
\\	、分節
\\	変化形 
\\	【他動】~を輸入{ゆにゅう}する、運び込む、持ち込む◆【反】
\\	《コ》~をインポートする、取り込む◆【反】
\\	〈古〉~を示す、~を意味{いみ}する、~をほのめかす【名】輸入{ゆにゅう}(すること)◆【反】
\\	輸入品{ゆにゅうひん}◆通例
\\	◆【略】
\\	◆【反】
\\	重要性{じゅうようせい}、意味{いみ}、趣旨{しゅし}《コ》インポート、取り込み◆あるソフトで作成したファイルを別のソフトで取り込むこと。◆【反】
\\	レベル3、発音《名》
\\	《動》
\\	、カナインポート、変化《動》
\\	、分節
\\	変化形 
\\	【自動】輸出{ゆしゅつ}する【他動】~を輸出{ゆしゅつ}する、運び去る◆【反】
\\	《コ》~をエクスポートする◆【反】
\\	【名】輸出{ゆしゅつ}、国外{こくがい}への持ち出し、国外{こくがい}への輸送{ゆそう}◆【反】
\\	輸出品{ゆしゅつひん}◆通例複数形◆【反】
\\	《コ》エクスポート、他ソフトへのデータ転送{てんそう}(機能{きのう})◆あるソフトで作成したデータを、別のソフトでも読める形式にして書き出す[出力する・保存する]こと。◆【反】
\\	レベル3、発音《名》
\\	《動》
\\	、カナエクスポート、変化《動》
\\	、分節
\\	機械の輸送および移動は必ず第5節の記載通りに行ってください。	変化形 
\\	【他動】~を運ぶ、輸送{ゆそう}する ・
\\	機械の輸送および移動は必ず第5節の記載通りに行ってください。~を夢中{むちゅう}にさせる、有頂天{うちょうてん}にする【名】〈英〉運送{うんそう}、運搬{うんぱん}、運輸{うんゆ}、輸送{ゆそう}、移送{いそう}◆【同】〈米〉
\\	〈英〉輸送{ゆそう}機関{きかん}[手段{しゅだん}]◆【同】〈米〉
\\	《軍事》人員{じんいん}[兵員{へいいん}・物資{ぶっし}]輸送{ゆそう}機[船]歓喜{かんき}、有頂天{うちょうてん}流罪{るざい}になった[島流しされた]罪人{ざいにん}レベル3、発音《名》
\\	《動》
\\	、カナトゥランスポート、変化《動》
\\	、分節
\\	これはその惨事を扱った最初の映画である。 ・
\\	倫理基準に違反するものには厳しく対応する。 ・
\\	彼の腹を一発殴ってやった。 ・
\\	いいとも。/商談成立だな。/それで結構です。 ・
\\	よし取引成立です。 ・
\\	それで手を打とう。/じゃあそうしよう。 ・
\\	この条件でいいですか?◆【同】
\\	この取引は、
\\	社の労働者3万人の職を救うことになる。 ・
\\	松材の椅子に座った。
\\	は粗末な椅子を連想させる。	変化形 
\\	【1自動】扱う、関係{かんけい}する ・
\\	これはその惨事を扱った最初の映画である。〔ある様態{ようたい}で〕対処{たいしょ}する、対応{たいおう}する ・
\\	倫理基準に違反するものには厳しく対応する。〔~に対して〕行動{こうどう}する取引{とりひき}する、商う《トランプ》札を配る【1他動】~を分け与える、配る、分配{ぶんぱい}する~を売る、~を販売{はんばい}する〔報償{ほうしょう}や罰などを(人)に〕与える、加える ・
\\	彼の腹を一発殴ってやった。《トランプ》〔札を〕配る【1名】取引{とりひき}、取り決め{とりきめ}、契約{けいやく}、協定{きょうてい} ・
\\	いいとも。/商談成立だな。/それで結構です。 ・
\\	よし取引成立です。 ・
\\	それで手を打とう。/じゃあそうしよう。 ・
\\	この条件でいいですか?◆【同】
\\	この取引は、
\\	社の労働者3万人の職を救うことになる。〈話〉掘り出し物、バーゲン品◆通例複数形の
\\	〔経済{けいざい}または社会{しゃかい}の〕政策{せいさく}◆通例
\\	裏[不正{ふせい}]取引{とりひき}、秘密協定{ひみつ きょうてい}〈話〉もてなし、待遇{たいぐう}《トランプ》ディール、〔札を〕配ること、〔配られた〕手札{てふだ}、〔札を配る〕順番{じゅんばん}〔不特定{ふとくてい}の〕量、程度{ていど}◆【用法】
\\	や
\\	などの語が前に来ることが多い。【2名】松[モミ]材 ・
\\	松材の椅子に座った。
\\	は粗末な椅子を連想させる。レベル2、発音
\\	、カナデール、ディール、変化《動》
\\	昨日1アメリカ・ドルは4ディナールに両替できた。 ・
\\	それを正しいものと交換してください。 ・
\\	昨日の朝方に彼といくつか
\\	メールのやりとりをした。 ・
\\	危険と自由とを引き替えに安全を犠牲にした。	変化形 
\\	【自動】交換{こうかん}する、〔引き替えに〕差し出す両替{りょうがえ}できる ・
\\	昨日1アメリカ・ドルは4ディナールに両替できた。【他動】~を交換{こうかん}する、~を両替{りょうがえ}する ・
\\	それを正しいものと交換してください。~をやりとりする ・
\\	昨日の朝方に彼といくつか
\\	メールのやりとりをした。〔あるものと引き替えに〕~をあきらめる、~を犠牲{ぎせい}にする ・
\\	危険と自由とを引き替えに安全を犠牲にした。【名】換えること、交換{こうかん}、〔情報{じょうほう}などの〕やりとり交換{こうかん}されるもの[人]、交換品{こうかん ひん}交換{こうかん}が行われる場所{ばしょ}、〔証券{しょうけん}や商品{しょうひん}の〕取引所{とりひきじょ}電話交換(機){でんわこうかん(き)}、電話局{でんわきょく}為替制度{かわせ せいど}、為替手数料{かわせ てすうりょう}為替手形{かわせ てがた}◆【同】
\\	為替相場{かわせ そうば}◆【同】
\\	為替差額{かわせ さがく}レベル3、発音
\\	、カナイクスチェインジ、エクスチェインジ、エクスチェンジ、変化《動》
\\	、分節
\\	ぎりぎりのところだけど何とかできそうです。 ・
\\	そうでなければ、とてもやってこられなかったと思います。 ・
\\	妻の希望通りのものを何とか購入することができました。 ・
\\	あいにくどうしても都合がつきそうにありません。 ・
\\	私も風邪をひいてしまいました。	変化形 
\\	【自動】何とか成し遂げる、何とかやっていく ・
\\	ぎりぎりのところだけど何とかできそうです。 ・
\\	そうでなければ、とてもやってこられなかったと思います。経営{けいえい}[運営{うんえい}]する、世話{せわ}をする【他動】〔道具{どうぐ}や武器{ぶき}などを巧みに〕使う、扱う、操る〔困難{こんなん}な中で何とか〕~を成し遂げる[やりくりする] ・
\\	妻の希望通りのものを何とか購入することができました。 ・
\\	あいにくどうしても都合がつきそうにありません。〈反語的〉〔うかつにも嫌なことを〕してしまう ・
\\	私も風邪をひいてしまいました。〔扱いにくい人や動物{どうぶつ}などを〕手なずける、しつける、操る〔会社{かいしゃ}などを〕経営{けいえい}[管理{かんり}・運営{うんえい}]するレベル2、発音
\\	、カナマニッジ、マネージ、マネッジ、変化《動》
\\	、分節
\\	私どもと致しましては、
\\	社をご利用になることを大いにお勧め致します。 ・
\\	使用する処理方法による。	変化形 
\\	【他動】〔労賃を払って人を〕雇う、雇用{こよう}する~に従事{じゅうじ}する〔手段{しゅだん}を〕用いる、採用{さいよう}する、使用{しよう}する ・
\\	私どもと致しましては、
\\	社をご利用になることを大いにお勧め致します。 ・
\\	使用する処理方法による。【名】雇用{こよう}レベル3、発音
\\	、カナインプロイ、エンプロイ、変化《動》
\\	、分節
\\	それはあなたの努力のたまものです。	変化形 
\\	【他動】〔労働の対価として〕金を得る〔暮らしを〕立てる〔当然の報いとして名声・信用などを〕得る、獲得{かくとく}する ・
\\	それはあなたの努力のたまものです。〔商売などが利益を〕生む〔投資・銀行預金などが利子を〕生むレベル2、発音
\\	、カナアーン、変化《動》
\\	私にはそんな高い家賃は払えない。 ・
\\	この種の車が買えるならガソリンの費用など何でもないはずです。 ・
\\	私立学校の費用を賄う余裕はない。 ・
\\	自分のジェット機を買う余裕はない。 ・
\\	〔一人だけの収入でも母親は〕子どもと一緒に家庭で生活できます。 ・
\\	職場へ車で行くのにさらに1時間もかけることはできない。 ・
\\	このイベントによってすべての参加者が人間関係を構築しお互いの考えを知る機会が得られます。	変化形 
\\	【他動】~を買うことができる、~に対する金銭的{きんせん てき}な余裕{よゆう}がある ・
\\	私にはそんな高い家賃は払えない。 ・
\\	この種の車が買えるならガソリンの費用など何でもないはずです。 ・
\\	私立学校の費用を賄う余裕はない。〔自分{じぶん}が不利益{ふりえき}を被らないで〕~をすることができる、~をする余裕{よゆう}がある ・
\\	自分のジェット機を買う余裕はない。 ・
\\	〔一人だけの収入でも母親は〕子どもと一緒に家庭で生活できます。〔不利{ふり}な結果{けっか}をもたらさないで〕~を手放すことができる、~をあきらめられる ・
\\	職場へ車で行くのにさらに1時間もかけることはできない。~を提供{ていきょう}する、~を利用可能{りよう かのう}にする ・
\\	このイベントによってすべての参加者が人間関係を構築しお互いの考えを知る機会が得られます。レベル3、発音
\\	、カナアフォード、変化《動》
\\	、分節
\\	もうこれ以上自分を偽ることはできない。 ・
\\	彼は強がっている。 ・
\\	私たちは初対面、ということにしておきましょう。 ・
\\	今のは聞かなかったことにしておきましょう。◆【場面】何らかの意味で不都合な相手の発言をなかったことにする。例えば失言を見逃したり、自分に都合悪いことを黙殺。	変化形 
\\	【自動】装う、取り繕う ・
\\	もうこれ以上自分を偽ることはできない。【他動】~のふりをする、~だと偽る、見せ掛ける、あえて~しようとする ・
\\	彼は強がっている。 ・
\\	私たちは初対面、ということにしておきましょう。 ・
\\	今のは聞かなかったことにしておきましょう。◆【場面】何らかの意味で不都合な相手の発言をなかったことにする。例えば失言を見逃したり、自分に都合悪いことを黙殺。【形】〈話〉うその、偽の、架空{かくう}の◆【用法】子どもが使うことが多い。レベル3、発音
\\	、カナプリテンド、変化《動》
\\	、分節
\\	変化形 
\\	【他動】〔からかって人の〕物まねをする~を模範{もはん}とする、~に倣う~に似せる、~と同じに見えるようにする〔芸術{げいじゅつ}のスタイルを〕模倣{もほう}する、再現{さいげん}するレベル4、発音
\\	、カナイミテイト、変化《動》
\\	、分節
\\	私は背が低いという点で夫と似ている。	変化形 
\\	【他動】~のようである、~と[に]似ている、~と共通点{きょうつうてん}がある ・
\\	私は背が低いという点で夫と似ている。〈古〉たとえる、象徴{しょうちょう}するレベル4、発音
\\	、カナリゼンブル、変化《動》
\\	、分節
\\	カシの大樹は小さなどんぐりから育つ。◆ことわざ ・
\\	金のなる木はない。/お金を楽に得る方法はない。◆ことわざ ・
\\	ボブ君、あなた本当に大きくなりましたね。見違えちゃったよ。 ・
\\	抗議は大きくなっている。 ・
\\	売り上げは伸びている。	変化形 
\\	【自動】〔人・動物・植物などが〕成長{せいちょう}する、育つ、大きくなる ・
\\	カシの大樹は小さなどんぐりから育つ。◆ことわざ ・
\\	金のなる木はない。/お金を楽に得る方法はない。◆ことわざ ・
\\	ボブ君、あなた本当に大きくなりましたね。見違えちゃったよ。増える、増大{ぞうだい}する◆【同】
\\	抗議は大きくなっている。 ・
\\	売り上げは伸びている。〔髪・爪などが〕長くなる、伸びる〔徐々に変化して~の〕状態{じょうたい}になる〔企業の規模などが〕拡大{かくだい}する【他動】~を育てる[生育{せいいく}する]〔作物を〕栽培{さいばい}する〔ひげなどを〕生やす〔細菌{さいきん}を〕培養{ばいよう}するレベル1、発音
\\	、カナグロー、グロウ、変化《動》
\\	3年前にケンブリッジ大学を卒業しました。 ・
\\	社会学部を卒業しました。 ・
\\	その大学を1970年に卒業しました。◆【参考】
\\	{自動-1} ・
\\	2006年の夏に
\\	を卒業しました。 ・
\\	彼女はX大学の卒業生です。	変化形 
\\	【自動】〔大学{だいがく}を〕卒業{そつぎょう}する、〔大学{だいがく}から〕学位{がくい}を得る◆【用法】他動詞用法{他動-1}に比較して、大学よりもむしろ自分の力で卒業したという意味合いが強い。他動詞用法より新しい表現でこちらの方が一般的。 ・
\\	3年前にケンブリッジ大学を卒業しました。 ・
\\	社会学部を卒業しました。少しずつ変化{へんか}する、徐々{じょじょ}に変わっていく〔能力{のうりょく}などが〕次の段階{だんかい}に進む、レベルが上がる【他動】〔学生{がくせい}を〕卒業{そつぎょう}させる、〔学生{がくせい}に〕学位{がくい}を与える ・
\\	その大学を1970年に卒業しました。◆【参考】
\\	{自動-1}〈話〉〔大学{だいがく}を〕卒業{そつぎょう}する、〔大学{だいがく}から〕学位{がくい}を受ける◆【用法】この表現を正しいものとして認めない人も多いので注意が必要である。 ・
\\	2006年の夏に
\\	を卒業しました。~に等級{とうきゅう}[目盛{めも}り]を付ける【名】卒業生{そつぎょうせい}、学士{がくし} ・
\\	彼女はX大学の卒業生です。〈米〉大学院生{だいがく いんせい}【形】〈米〉大学院{だいがくいん}の、大学{だいがく}を卒業{そつぎょう}した◆【同】〈英〉
\\	レベル3、発音《名》
\\	《動》
\\	、カナグラジュエイト、グラジュエット、グラデュエート、変化《複》
\\	、《動》
\\	、分節
\\	結婚のことなんか全然考えたことがない。 ・
\\	早くお婿さんもらって子ども生みなさい。 ・
\\	僕と結婚してくれる?◆男性が女性の前で片膝をついて言うトラディショナルな表現 ・
\\	いいかげんにしなさい。うそまでついて彼と一緒になっておいて。 ・
\\	彼女が美しいから結婚したのではない。 ・
\\	俺{おれ}、本当に真剣なんだ。結婚のことをちゃんと考えてほしい。 ・
\\	あなたの婿選びに間違いはないんだよ。 ・
\\	王は娘を隣国の王子のところに嫁がせたいと思っていました。 ・
\\	牧師が彼らの結婚式を執り行いました。	変化形 
\\	【自動】嫁[婿]をもらう、嫁ぐ、結婚{けっこん}する、身を固める ・
\\	結婚のことなんか全然考えたことがない。 ・
\\	早くお婿さんもらって子ども生みなさい。〔2種類以上{しゅるい いじょう}の飲食物{いんしょくぶつ}の味が〕合う【他動】~と一緒{いっしょ}になる、~と結婚{けっこん}する、~と縁組{えんぐみ}する ・
\\	僕と結婚してくれる?◆男性が女性の前で片膝をついて言うトラディショナルな表現 ・
\\	いいかげんにしなさい。うそまでついて彼と一緒になっておいて。 ・
\\	彼女が美しいから結婚したのではない。 ・
\\	俺{おれ}、本当に真剣なんだ。結婚のことをちゃんと考えてほしい。 ・
\\	あなたの婿選びに間違いはないんだよ。〔親などが〕(人)を結婚させる◆しばしば
\\	を伴う。 ・
\\	王は娘を隣国の王子のところに嫁がせたいと思っていました。〔聖職者などが〕(人)の結婚式を執り行う ・
\\	牧師が彼らの結婚式を執り行いました。~を結合{けつごう}させるレベル1、発音
\\	、カナマリー、変化《動》
\\	、分節
\\	彼女は彼と離婚しました。 ・
\\	彼らは離婚しました。	変化形 
\\	【自動】離婚{りこん}する【他動】~を離婚{りこん}させる、~と離婚{りこん}する ・
\\	彼女は彼と離婚しました。〔結合{けつごう}している物を〕分離{ぶんり}させる【名】離婚{りこん} ・
\\	彼らは離婚しました。〔結合{けつごう}していた物の〕分離{ぶんり}レベル5、発音
\\	、カナディボース、変化《動》
\\	、分節
\\	風邪には大食、熱には小食が良い。◆ことわざ ・
\\	犬は3日の恩を3年忘れない。◆ことわざ ・
\\	小銭持ってないかな?パーキングメーターにコインを入れないといけないんだ。	変化形 
\\	【自動】〔動物{どうぶつ}が餌{えさ}を〕食べる送り込まれる《機械》(旋盤{せんばん}などの切削機械{せっさく きかい}で)送りを与える、給電{きゅうでん}する、給炭{きゅうたん}する、給油{きゅうゆ}する、動力{どうりょく}を供給{きょうきゅう}する【他動】~に食物{たべもの}を与える、〔動物{どうぶつ}を〕飼う、〔人を〕養う ・
\\	風邪には大食、熱には小食が良い。◆ことわざ ・
\\	犬は3日の恩を3年忘れない。◆ことわざ(人)にとって飯の種になる、(人)は〔主語〕で食っている〔コンピューターにデータを〕入力{にゅうりょく}する〔パーキングメーターに〕コインを入れる ・
\\	小銭持ってないかな?パーキングメーターにコインを入れないといけないんだ。搬送{はんそう}する、供給{きょうきゅう}する目を楽しませる、心の糧になる煽る、大きくさせる、増大{ぞうだい}させる〔ぼけ役に〕つっこみを入れる【名】〔動物{どうぶつ}の〕餌{えさ}、〔幼児{ようじ}の〕食事{しょくじ}〔動物{どうぶつ}に〕餌{えさ}を与えること、〔幼児{ようじ}に〕食べ物を与えること〈話・やや古〉たっぷりの食事{しょくじ}〔機械{きかい}などへの材料{ざいりょう}の〕供給(量){きょうきゅう(りょう)}〔機械{きかい}などの材料{ざいりょう}の〕供給{きょうきゅう}口[装置{そうち}]〔地域{ちいき}のラジオやテレビの〕放送{ほうそう}(番組{ばんぐみ})《イ》フィード◆【同】
\\	レベル2、発音
\\	、カナフィード、変化《動》
\\	《誘う》タ食にお誘いしたいのですが。 ・
\\	《礼を述べる》ご招待いただいてどうもありがとうございます。 ・
\\	彼は私を部屋に上がらせてくれた。 ・
\\	彼は勝手に私の部屋に上がり込んできた。 ・
\\	社をご用命くださるようお勧め致します。 ・
\\	誰か私を招待してくれませんか。 ・
\\	私は招かれなかった。	変化形 
\\	【他動】招待{しょうたい}する、招く、案内{あんない}する、招聘{しょうへい}する、誘う ・
\\	《誘う》タ食にお誘いしたいのですが。 ・
\\	《礼を述べる》ご招待いただいてどうもありがとうございます。 ・
\\	彼は私を部屋に上がらせてくれた。 ・
\\	彼は勝手に私の部屋に上がり込んできた。~を勧誘{かんゆう}する、勧める ・
\\	社をご用命くださるようお勧め致します。~を頼む、懇願{こんがん}する事態{じたい}を引き起こす、~の誘因{ゆういん}になる【名】招待{しょうたい} ・
\\	誰か私を招待してくれませんか。 ・
\\	私は招かれなかった。レベル1、発音《名》
\\	《動》
\\	、カナインバイト、変化《動》
\\	、分節
\\	あなた様にご出席いただけるなら、大変光栄です。 ・
\\	最高経営責任者あるいは取締役社長のみ参加が認められております。	変化形 
\\	【自動】出席{しゅっせき}する、参加{さんか}する、参列{さんれつ}する◆【用法】
\\	あなた様にご出席いただけるなら、大変光栄です。 ・
\\	最高経営責任者あるいは取締役社長のみ参加が認められております。〔職務として人の〕世話{せわ}をする◆【用法】
\\	付き添う◆【用法】
\\	精を出す、関心{かんしん}[注意{ちゅうい}]を向ける、注意{ちゅうい}して見る、注意{ちゅうい}する◆【用法】
\\	【他動】〔催し物・会合{かいごう}・儀式{ぎしき}など〕に出席{しゅっせき}[参加{さんか}・参会{さんかい}・参列{さんれつ}]する〔学校{がっこう}などに〕行く、通う〔物・事が〕~に伴う〔職務{しょくむ}として〕~の世話{せわ}をする、~を看護{かんご}する~に付き添う、随行{ずいこう}する~に関心{かんしん}[注意{ちゅうい}]を向ける、注意{ちゅうい}する、気を付ける、~を注意{ちゅうい}して見るレベル2、発音
\\	、カナアテンド、変化《動》
\\	、分節
\\	彼らは未経験労働者の相当部分を構成するだろう。 ・
\\	1日二度の食事と労働が牢獄内の犯罪者の生活のすべてだった。	変化形 
\\	【他動】~の構成要素{こうせい ようそ}となる、~を構成{こうせい}する、~の性質{せいしつ}である ・
\\	彼らは未経験労働者の相当部分を構成するだろう。 ・
\\	1日二度の食事と労働が牢獄内の犯罪者の生活のすべてだった。~を制定{せいてい}する〔人を〕~に選ぶ[任命{にんめい}する]レベル5、発音
\\	、カナコンスチチュート、コンスティチュート、コンスティテュート、変化《動》
\\	、分節
\\	変化形 
\\	【他動】~を構成{こうせい}する、作る、組み立てる、~から成る~を創作{そうさく}する、作曲{さっきょく}する、(詩などを)書く~を収拾{しゅうしゅう}する~を落ちつかせるレベル5、発音
\\	、カナコンポーズ、コンポウズ、変化《動》
\\	、分節
\\	メイド求む、高給優遇、住み込み賄い付き。◆募集広告で使われる。	変化形 
\\	【1他動】
\\	の過去・過去分詞形【1名】〔無料{むりょう}の〕住居{じゅうきょ}と食事{しょくじ} ・
\\	メイド求む、高給優遇、住み込み賄い付き。◆募集広告で使われる。【1形】見つけられた、発見された【2他動】~の基礎{きそ}を築く、~を設立{せつりつ}する、創立{そうりつ}する、建てる◆【類】
\\	【2形】必要なものが備え付けられた、備品[設備]の調った【3他動】〔金属・ガラスなどを〕溶かす鋳造{ちゅうぞう}するレベル3、発音
\\	、カナファウンド、変化《動》
\\	私たちは新会社を設立する予定です。 ・
\\	政府は新しい祝日を定めました。 ・
\\	小児に対する安全性は確立していない。 ・
\\	競合状況に直面する中で、貴社の製品をどのように位置付けていますか。 ・
\\	事情を飲み込むまでにしばらくかかった[時間を要した]。	変化形 
\\	【自動】〔植物{しょくぶつ}が〕新しい場所に根付く、定着{ていちゃく}する【他動】〔会社や学校などを〕設置{せっち}[設立{せつりつ}・開設{かいせつ}・創設{そうせつ}]する ・
\\	私たちは新会社を設立する予定です。〔制度{せいど}・法律{ほうりつ}などを〕制定{せいてい}する、成立{せいりつ}させる ・
\\	政府は新しい祝日を定めました。〔制度{せいど}・法律{ほうりつ}などが〕~を規定{きてい}する[定める]〔人を〕任命{にんめい}する、ある地位{ちい}に就ける、定住{ていじゅう}させる〔権威{けんい}・要求{ようきゅう}・習慣{しゅうかん}・先例{せんれい}・地位{ちい}・名声{めいせい}などを〕確立{かくりつ}する ・
\\	小児に対する安全性は確立していない。 ・
\\	競合状況に直面する中で、貴社の製品をどのように位置付けていますか。〔記録{きろく}を〕達成{たっせい}する、樹立{じゅりつ}する〔~として〕世間{せけん}に認めさせる、〔地歩{ちほ}を〕固める、~として身を立てる〔事実{じじつ}・理論{りろん}などを〕確証{かくしょう}する、立証{りっしょう}する、証明{しょうめい}する、はっきりさせる ・
\\	事情を飲み込むまでにしばらくかかった[時間を要した]。〔教会{きょうかい}を〕国教化{こっきょうか}するレベル3、発音
\\	、カナイスタブリッシュ、エスタブリッシュ、変化《動》
\\	、分節
\\	そのロボットを動かすには、バッテリーが必要です。 ・
\\	国によっては、男性に兵役が義務付けられています。	変化形 
\\	【他動】〔ある目的{もくてき}のために〕~を必要{ひつよう}とする ・
\\	そのロボットを動かすには、バッテリーが必要です。〔条件{じょうけん}として〕~を要求{ようきゅう}[必要{ひつよう}と]する〔法律{ほうりつ}などで〕~を義務付ける{ぎむづける}◆【用法】通例、受け身で用いる。 ・
\\	国によっては、男性に兵役が義務付けられています。(人)に~することを求める[強制{きょうせい}する]◆【用法】
\\	レベル3、発音
\\	、カナリクワイア、変化《動》
\\	、分節
\\	あなたが何でここにいるのか説明してもらおうか。 ・
\\	私は理由を問い詰めた。 ・
\\	仕事の要請が私の許容度を超えている。 ・
\\	市場が求めるものが変わっている。 ・
\\	その仕事の要求水準は高い。	変化形 
\\	【他動】〔~することを強く〕求める、要求{ようきゅう}する ・
\\	あなたが何でここにいるのか説明してもらおうか。〔正当{せいとう}な理由{りゆう}により〕~を要求{ようきゅう}[請求{せいきゅう}]する〔強くまたは即座{そくざ}に〕~の説明{せつめい}を求める、~について問いただす ・
\\	私は理由を問い詰めた。~を必要{ひつよう}とする、〔必要{ひつよう}なものを〕求める ・
\\	仕事の要請が私の許容度を超えている。《法律》〔法廷{ほうてい}に〕(人)を召喚{しょうかん}する《法律》〔法令{ほうれい}によって〕~を要求{ようきゅう}[請求{せいきゅう}]する【名】要求{ようきゅう}[要望{ようぼう}・請求{せいきゅう}]すること求めるもの、要求(物){ようきゅう(ぶつ)}、要望(品){ようぼう(ひん)} ・
\\	市場が求めるものが変わっている。 ・
\\	その仕事の要求水準は高い。〔時間{じかん}・資源{しげん}・行動{こうどう}などの緊急{きんきゅう}の〕必要(性){ひつよう(せい)}《経済》需要(量){じゅよう(りょう)}《コ》〔仮想記憶{かそう きおく}の〕デマンド◆記憶装置の内容を実際に参照するときになって、補助記憶から読み込むページングの方式。《法律》請求{せいきゅう}、督促{とくそく}レベル2、発音
\\	、カナデマンド、ディマンド、変化《動》
\\	、分節
\\	その他のすべての色(の製品)は10月末までにご提供できます。	変化形 
\\	【他動】供給{きょうきゅう}する、支給{しきゅう}する、提供{ていきょう}する ・
\\	その他のすべての色(の製品)は10月末までにご提供できます。~の代理{だいり}を務める、代役{だいやく}をする【名】供給(量)生活必需品、補給品◆通例
\\	レベル3、発音
\\	、カナサプライ、変化《動》
\\	、分節
\\	われわれは、部品を50以上の会社に供給しています。 ・
\\	その雑誌は国内ニュースよりも海外ニュースを多く載せている。 ・
\\	その方法によって柔軟性が一層高まる。	変化形 
\\	【自動】備える、準備{じゅんび}する◆【用法】
\\	養う、扶養{ふよう}する◆【用法】
\\	規定{きてい}する、条件{じょうけん}を付ける◆【用法】
\\	【他動】〔物を〕供給{きょうきゅう}[提供{ていきょう}]する ・
\\	われわれは、部品を50以上の会社に供給しています。〔機会などを〕もたらす、与える ・
\\	その雑誌は国内ニュースよりも海外ニュースを多く載せている。 ・
\\	その方法によって柔軟性が一層高まる。《法律》~と規定{きてい}する、~を条件{じょうけん}とする〈古〉~を準備{じゅんび}するレベル2、発音
\\	、カナプロバイド、変化《動》
\\	、分節
\\	彼は私にグラス1杯のワインを勧めてくれました。 ・
\\	この公園にはピクニック用の設備があります。 ・
\\	彼らは私に家に泊まるように言ってくれました。 ・
\\	彼は私に、(車で)家まで送ろうかと言ってくれました。 ・
\\	このパソコンは手頃{てごろ}な値段で売られています。 ・
\\	特別な祈りがその寺院でささげられました。 ・
\\	彼は嫌悪以外の気持ちを表すことができませんでした。 ・
\\	その家を6万ドルで入札して落札しました。 ・
\\	私の友人は、昨日、その仕事の申し出を受け入れました。	変化形 
\\	【他動】〔人が~を受け入れるよう〕申し出る、提示する ・
\\	彼は私にグラス1杯のワインを勧めてくれました。〔他の人のために〕~を提供する[用意する] ・
\\	この公園にはピクニック用の設備があります。〔他の人のために~しようと〕提案する、言う ・
\\	彼らは私に家に泊まるように言ってくれました。 ・
\\	彼は私に、(車で)家まで送ろうかと言ってくれました。~を売り出す、~を貸し出す ・
\\	このパソコンは手頃{てごろ}な値段で売られています。《宗教》〔神に祈りを〕ささげる ・
\\	特別な祈りがその寺院でささげられました。〔態度などを~に〕示す、表す ・
\\	彼は嫌悪以外の気持ちを表すことができませんでした。〔物に値段を〕付ける、〔ある金額で〕入札する ・
\\	その家を6万ドルで入札して落札しました。【名】申し出、申入れ{もうしいれ}、提案{ていあん}、提供{ていきょう} ・
\\	私の友人は、昨日、その仕事の申し出を受け入れました。指し値{さしね}、付け値{つけね}、売買申し込み割引価格[値段]、値引き〔商品の〕売り出し、市場{しじょう}(への)投入{とうにゅう}◆【参考】
\\	《法律》正式オファー◆受け入れれば拘束力のある契約に結び付く提案。試み、試しレベル2、発音
\\	、カナオーファ、オファー、変化《動》
\\	、分節
\\	空には花火が舞った。 ・
\\	時間が余ってしまったので、適当に質問などして持たせた。	変化形 
\\	【他動】〔容器{ようき}や場所{ばしょ}を〕いっぱいに満たす、あふれさせる◆【用法】主語が容器などに入る場合と、主語が
\\	以下の物や人を容器などに満たす場合がある。〔光や匂{にお}いなどが場所{ばしょ}に〕広がる、立ち込める、充満{じゅうまん}する ・
\\	空には花火が舞った。〔隙間{すきま}やひび割れを〕埋める、塞{ふさ}ぐ〔虫歯{むしば}に〕詰め物をする《建築》〔場所{ばしょ}に〕盛り土をする〔用紙{ようし}や空欄{くうらん}を〕記入{きにゅう}する、塗りつぶす〔食物{しょくもつ}が人を〕満腹{まんぷく}にさせる、おなかいっぱいにする〔仕事{しごと}や役職{やくしょく}に人を〕指名{しめい}する、任命{にんめい}する〔職務{しょくむ}に〕就く、〔職務{しょくむ}を〕こなす〔要求{ようきゅう}を〕満たす、〔目的{もくてき}に〕かなう、〔命令{めいれい}に〕従う〔処方箋の指示{しじ}に従って〕薬を調合{ちょうごう}する〔自由{じゆう}な時間{じかん}を〕埋める、つぶす ・
\\	時間が余ってしまったので、適当に質問などして持たせた。〔感情{かんじょう}や考えなどが人の〕心を満たす、頭をいっぱいになる〔料理{りょうり}に〕詰め物をする、具を挟む〔金属{きんぞく}の表面{ひょうめん}に〕めっきをする《海事》〔帆を〕膨らませる《海事》〔帆桁{ほげた}を〕調整{ちょうせい}する【名】
\\	必要{ひつよう}な分、欲しいだけ〔容器{ようき}〕1杯の量[分]《建築》盛り土(をすること)《音楽》フィル◆【同】
\\	レベル2、発音
\\	、カナフィル、変化《動》
\\	手術を遅らせたところで何も得るものはない。 ・
\\	蒔かぬ種は生えない。◆ことわざ	変化形 
\\	【1自動】〔数量{すうりょう}・価値{かち}・度合いなどが〕増加{ぞうか}する、上昇{じょうしょう}する〔健康{けんこう}などが〕改善{かいぜん}する、向上{こうじょう}する〔仕事{しごと}などで〕利益{りえき}を得る、得をする、もうける〔追っているものに〕近づく、追い付いていく〔追っているものを〕引き離す、差を広げる体重{たいじゅう}が増える、太る〔時計{とけい}が〕進む【1他動】〔努力して〕~を得る[獲得{かくとく}する] ・
\\	手術を遅らせたところで何も得るものはない。〔競争{きょうそう}によって~を〕勝ち取る、達成{たっせい}する〔報酬{ほうしゅう}や利益{りえき}などを〕稼ぐ、もうける〔~の数量{すうりょう}や度合い{どあい}を〕増加{ぞうか}させる、高くする〔数量分{すうりょう ぶん}だけ〕増加{ぞうか}する、上昇{じょうしょう}する〔人と〕関係{かんけい}を築く、〔人を〕味方{みかた}にする〔時計{とけい}が正しい時間{じかん}より時間分{じかん ぶん}だけ〕進む〈文〉〔目的地{もくてきち}に〕到達{とうたつ}する【1名】得たもの、利益{りえき} ・
\\	蒔かぬ種は生えない。◆ことわざ得ること、獲得{かくとく}前進{ぜんしん}、進歩{しんぽ}、発展{はってん}〔数量{すうりょう}や程度{ていど}の〕増加{ぞうか}、増大{ぞうだい}《電気》利得{りとく}、ゲイン【2他動】〔材木{ざいもく}に〕溝[ほぞ穴]を作る[掘る]〔材木{ざいもく}を〕溝[ほぞ穴]にはめる【2名】〔材木{ざいもく}の〕溝、ほぞ穴、切り込みレベル2、発音
\\	、カナゲイン、変化《動》
\\	戦争中は食料を確保するのが難しかった。	変化形 
\\	【自動】〔制度{せいど}などが〕定着{ていちゃく}[確立{かくりつ}]している、受け入れられている〈文〉結果{けっか}として起きる【他動】〔努力して〕手に入れる、得る、取得{しゅとく}する、獲得{かくとく}する、入手{にゅうしゅ}する、調達{ちょうたつ}する ・
\\	戦争中は食料を確保するのが難しかった。レベル4、発音
\\	、カナオブテイン、変化《動》
\\	、分節
\\	彼は努力して英語の知識を身に付けた。	変化形 
\\	【他動】〔時間をかけて努力して〕手に入れる、獲得{かくとく}する、取得{しゅとく}する、得る、入手{にゅうしゅ}する、確保{かくほ}する、買収{ばいしゅう}する〔努力{どりょく}して学力{がくりょく}・技術力{ぎじゅつ りょく}・知識{ちしき}・習慣{しゅうかん}・学問{がくもん}などを〕習得{しゅうとく}する、我がものにする、学ぶ、身に付ける ・
\\	彼は努力して英語の知識を身に付けた。〔評判{ひょうばん}などを〕もたらす〔悪評{あくひょう}を〕買う~を捕らえる、不法{ふほう}に入手{にゅうしゅ}する、捕捉{ほそく}する~を盗む◆【類】
\\	レベル4、発音
\\	、カナアクワイア、アクワイヤ、変化《動》
\\	、分節
\\	これまでの討議について議事録を用意致しました。 ・
\\	報告書は1991年8月1日現在で作成された。 ・
\\	《契約書》本契約は、英語だけで作成されるものとする。	変化形 
\\	【自動】〔ある目的{もくてき}に〕使えるようにする、整える、準備{じゅんび}する〔事前{じぜん}に〕用意{ようい}[心構え]をさせる【他動】〔物や場所{ばしょ}をある目的{もくてき}に〕使えるようにする、整える、準備{じゅんび}する〔事前{じぜん}に人に〕用意{ようい}[心構え]をさせる〔必要{ひつよう}なものを事前{じぜん}に〕作成{さくせい}する、立案{りつあん}する ・
\\	これまでの討議について議事録を用意致しました。 ・
\\	報告書は1991年8月1日現在で作成された。 ・
\\	《契約書》本契約は、英語だけで作成されるものとする。〔材料{ざいりょう}を組み合わせて食事{しょくじ}や薬などを〕調理{ちょうり}する、調合{ちょうごう}する〔探検{たんけん}などのために〕~に装備{そうび}を付ける、~を艤装{ぎそう}する《音楽》〔不協和音{ふきょうわ おん}を予備音{よび おん}で〕和らげるレベル2、発音
\\	、カナプリペア、変化《動》
\\	、分節
\\	それは日本の慢性的な黒字減らしにいくらか貢献しました。 ・
\\	電力の発見は文明の発展に大きく貢献しました。 ・
\\	積極的な行動が問題解決に大いに貢献しました。 ・
\\	石油の不足がひっ迫した電力事情の原因になっている。	変化形 
\\	【自動】貢献{こうけん}する、寄与{きよ}する ・
\\	それは日本の慢性的な黒字減らしにいくらか貢献しました。 ・
\\	電力の発見は文明の発展に大きく貢献しました。 ・
\\	積極的な行動が問題解決に大いに貢献しました。寄付{きふ}する、献金{けんきん}する〔出版{しゅっぱん}や放送{ほうそう}に〕寄稿{きこう}する、投稿{とうこう}する一因{いちいん}となる、原因{げんいん}の一つになる ・
\\	石油の不足がひっ迫した電力事情の原因になっている。【他動】~に貢献{こうけん}する、~に寄与{きよ}する〔資金{しきん}や援助{えんじょ}などを〕与える、提供{ていきょう}するレベル5、発音
\\	、カナコントゥリビュート、コントリビュート、変化《動》
\\	、分節
\\	いつどのようにして情報がインターネット上に広まるか分からない。 ・
\\	その国においてその病気がこれほど早くまん延した一つの理由はセックス産業にある。◆性的表現 ・
\\	わあ、すごいごちそうですね。	変化形 
\\	【自動】〔畳んだ物が〕広がる、開く〔たくさんの物が〕広がる、分散{ぶんさん}する〔離れた物が〕伸びる、広がる、離れる〔物が表面{ひょうめん}に〕広がる、伸びる〔勢力{せいりょく}などが〕広がる、拡張{かくちょう}する〔情報{じょうほう}やうわさなどが〕広まる、伝わる〔景色{けしき}などが〕広がる、横たわる【他動】〔畳んでいる物を一杯{いっぱい}に〕広げる、開く〔離れている物同士{ぶつ どうし}を〕広げる、遠くに離す〔手の指などを〕伸ばす、広げる〔表面{ひょうめん}に〕~を塗る[〔液体{えきたい}などを~に〕塗る]〔広い地域{ちいき}に勢力{せいりょく}などを〕広める、拡張{かくちょう}する〔ある場所{ばしょ}に多くの物を〕広げる、ばらまく、まき散らす〔見えるように全体{ぜんたい}を〕広げる、展開{てんかい}させる〔情報{じょうほう}やうわさなどを〕広める、流布{るふ}させる ・
\\	いつどのようにして情報がインターネット上に広まるか分からない。〔食卓{しょくたく}に〕料理{りょうり}を並べる、〔料理{りょうり}を〕食卓{しょくたく}に並べる〔金属{きんぞく}などを〕たたいて伸ばす【名】広がること、発散{はっさん}、展開{てんかい}、普及{ふきゅう}、まん延 ・
\\	その国においてその病気がこれほど早くまん延した一つの理由はセックス産業にある。◆性的表現土地{とち}の広がり、広大{こうだい}な土地{とち}〔地点{ちてん}や物事{ものごと}の間の〕隔たり、離れていること〔物事{ものごと}の〕多様性{たよう せい}、広がりがあること〈話〉テーブル一杯{いっぱい}のごちそう ・
\\	わあ、すごいごちそうですね。〔食べ物の〕スプレッド◆パンなどに塗って食べるバターやジャムなど。《銀行》スプレッド、調達運用金利差{ちょうたつ うんよう きんり さ}◆資金の調達金利と運用金利の差(銀行のもうけ)。《証券・為替》スプレッド◆買値と売値の差(証券会社、証券発行引受会社のもうけ)。《オプション》スプレッド◆【同】
\\	◆オプション取引で、同種類のオプションの価格差。ベッドカバー、テーブルカバー〔本・新聞などの〕見開き〔雑誌{ざっし}・新聞{しんぶん}などの〕広告特集{こうこく とくしゅう}、特集記事{とくしゅう きじ}〈米方言〉牧場{ぼくじょう}〈話〉太った尻[ヒップ]〈話〉〔飛行機{ひこうき}の〕翼幅{よくはば}【形】広がった、拡張{かくちょう}された〔宝石{ほうせき}が〕スプレッドの◆テーブルが大きく、クラウン
\\	が浅いものを指す。《言語学》〔調音時{ちょうおん じ}に〕平唇{へいしん}の◆【同】
\\	レベル2、発音
\\	、カナスプレッド、変化《動》
\\	私はあくびと伸びをしました。 ・
\\	セーターが伸びちゃうでしょ。◆服を引っ張る相手に ・
\\	それはちょっとこじつけではありませんか。 ・
\\	彼は今、3年の刑期を務めています。	変化形 
\\	【自動】伸びる、背伸びする、伸びをする、広がる ・
\\	私はあくびと伸びをしました。能力一杯{のうりょく いっぱい}に働く【他動】~を伸ばす、張る、広げる~を限界{げんかい}まで引き伸ばす、いっぱいに伸ばす、無理{むり}に使う、引き延ばす、長引{ながび}かせる、(神経{しんけい}を)張り詰める~を能力一杯{のうりょく いっぱい}に働かせる、ぶっ倒れそうになるくらいに働かせる~を拡大解釈{かくだい かいしゃく}する、水増{みずま}しする~を悪用{あくよう}する、乱用{らんよう}する〔服などを〕引っ張ってたるませる、伸びさせる ・
\\	セーターが伸びちゃうでしょ。◆服を引っ張る相手に【名】伸ばすこと、引き伸ばし、ストレッチ、伸び、背伸びすること広がり拡大解釈{かくだい かいしゃく} ・
\\	それはちょっとこじつけではありませんか。無理{むり}に使うこと、乱用{らんよう}一続きの時間{じかん}《野球》ストレッチ◆ある期間内の試合数= 
\\	伸縮性布地{しんしゅく せい ぬのじ}、伸張反射{しんちょう はんしゃ}、伸長変換{しんちょう へんかん}〈俗〉刑期{けいき} ・
\\	彼は今、3年の刑期を務めています。【形】伸縮性{しんしゅくせい}のあるレベル2、発音
\\	、カナストゥレッチ、ストレッチ、変化《動》
\\	中東和平交渉は1週間延長された。 ・
\\	喜んで訪問を延長したく存じます。 ・
\\	母親の具合が悪くなったので、ジェーンは滞在を2日延ばした。 ・
\\	まずは、お父上のご逝去、お悔やみ申し上げます。 ・
\\	ブラッドに対してご援助とご協力を頂ければありがたく存じます。 ・
\\	その馬は死力を尽くして頑張った。	変化形 
\\	【自動】〔線状・棒状の物が直線状に〕伸びる、広がる◆【語源】外へ
\\	伸ばす
\\	【他動】〔線状・棒状の物を〕伸ばす、引き伸ばす、広げる〔予定日{よていび}・期間{きかん}・期限{きげん}などを〕延ばす、延長{えんちょう}する ・
\\	中東和平交渉は1週間延長された。 ・
\\	喜んで訪問を延長したく存じます。 ・
\\	母親の具合が悪くなったので、ジェーンは滞在を2日延ばした。〔物の長さなどを〕延長{えんちょう}する〔寿命{じゅみょう}・能力{のうりょく}などを〕伸ばす〔区域{くいき}・勢力{せいりょく}・領土{りょうど}・言葉{ことば}の意味{いみ}などを〕広げる、伸ばす、拡張{かくちょう}する、拡大{かくだい}する、拡大適用{かくだい てきよう}する〔歓迎{かんげい}・同情{どうじょう}などを〕示す ・
\\	まずは、お父上のご逝去、お悔やみ申し上げます。〔言葉{ことば}・気持ちなどを〕伝える、述べる〔親切・援助などを〕与える、施す、差し伸べる〔援助金{えんじょ きん}などを〕与える、供与{きょうよ}する ・
\\	ブラッドに対してご援助とご協力を頂ければありがたく存じます。《コ》拡張{かくちょう}する〔集計表などの合計が合っていることを確かめるために〕縦横の計算をする
\\	精いっぱい努力{どりょく}する、力の限り頑張る ・
\\	その馬は死力を尽くして頑張った。レベル4、発音
\\	、カナイクステンド、エクステンド、変化《動》
\\	、分節
\\	彼は5人のスキンヘッドたちに取り囲まれたと言っている。	変化形 
\\	【他動】〔人や物を〕取り囲む、周りを囲む ・
\\	彼は5人のスキンヘッドたちに取り囲まれたと言っている。《軍事》〔敵を〕包囲{ほうい}する、包囲{ほうい}して孤立{こりつ}させる〔状況{じょうきょう}などが人を〕包む、周りに漂う【名】〈英〉〔場所{ばしょ}や物を〕取り囲む物、縁取り{ふちどり}〔場所{ばしょ}や物の〕周辺{しゅうへん}、周り〔周囲{しゅうい}の〕状況{じょうきょう}、環境{かんきょう}、雰囲気{ふんいき}〔狩猟法{しゅりょうほう}の〕巻き狩りレベル3、発音
\\	、カナサラウンド、変化《動》
\\	、分節
\\	変化形 
\\	【自動】前へ進む、前進{ぜんしん}する進歩{しんぽ}する、進展{しんてん}する〔価値{かち}などが〕上がる、向上{こうじょう}する〔地位{ちい}などが〕上がる、昇進{しょうしん}する〈米話〉〔選挙運動{せんきょ うんどう}などの〕先遣部隊員{せんけん ぶたい いん}として旅行{りょこう}する【他動】~を前へ進める、~を前進{ぜんしん}させる〔案などを〕提案{ていあん}する、提示{ていじ}する〔成長{せいちょう}や進歩{しんぽ}を〕促進{そくしん}する(人)を昇進{しょうしん}させる〔物事{ものごと}が起きるのを〕早める、前倒{まえだお}しする〔価格{かかく}や金額{きんがく}を〕上げる、高くする〔お金を〕前払いする~を掛け売りする、~を信用貸{しんようがし}しする〔行事{ぎょうじ}などを〕先送りする〈話〉~を事前{じぜん}に宣伝{せんでん}する〈米話〉〔旅行{りょこう}を〕事前宣伝{じぜん せんでん}のために行う〈古〉〔旗などを〕上げる、掲げる【名】〔空間的{くうかん てき}な〕前進{ぜんしん}〔目標{もくひょう}に対する〕前進{ぜんしん}、進歩{しんぽ}〔地位{ちい}の〕昇進{しょうしん}
\\	〔合意{ごうい}に向けた〕打診{だしん}、〔女性{じょせい}への〕口説き(文句{もんく})〔価格{かかく}の〕増額{ぞうがく}前払い(金)〔新聞{しんぶん}のイベント開催{かいさい}などの〕案内記事{あんない きじ}《軍事》先遣部隊{せんけん ぶたい}〈米話〉〔ビジネスや選挙運動{せんきょ うんどう}などの〕先遣部隊員{せんけん ぶたい いん}◆事前に派遣されて、イベントやキャンペーンの知名度を上げるために活動する。
\\	《証券》値上{ねあ}がり株【形】あらかじめの、事前{じぜん}の先発{せんぱつ}のレベル3、発音
\\	、カナアドゥバンス、アドバーンス、変化《動》
\\	、分節
\\	運動と休息、そしてよい食事が子どもたちの成長を促進する。 ・
\\	曙は横綱に昇進しました。	変化形 
\\	【他動】進める、進展{しんてん}させる、推進{すいしん}する、~の普及{ふきゅう}[発展{はってん}・成長{せいちょう}]を促進{そくしん}する ・
\\	運動と休息、そしてよい食事が子どもたちの成長を促進する。助成{じょせい}する、奨励{しょうれい}する、働きかける、振興{しんこう}する〔商品などを〕販売促進{はんばい そくしん}する、宣伝{せんでん}する〔社員などを〕昇進{しょうしん}させる、昇格{しょうかく}させる ・
\\	曙は横綱に昇進しました。《将棋》〔駒を〕成らせる◆【参考】
\\	レベル4、発音
\\	、カナプロモート、プロモウト、変化《動》
\\	、分節
\\	余分に持ち過ぎるということもある。時計を1個持っているものは何時かが分かっているが、2個持っている者には、はっきりしない。 ・
\\	私の勝ちだ。 ・
\\	自分は(他人とは違う)我が道を行く人間です。	変化形 
\\	【他動】所有{しょゆう}する ・
\\	余分に持ち過ぎるということもある。時計を1個持っているものは何時かが分かっているが、2個持っている者には、はっきりしない。~を認める~をうまくやってのける〈俗〉(人)に勝つ◆ゲームや議論の対戦で ・
\\	私の勝ちだ。〔ペットなどを〕飼っている◆【類】
\\	【形】自分(自身)の、独自{どくじ}の、自らの、自己の ・
\\	自分は(他人とは違う)我が道を行く人間です。特有{とくゆう}の、独特のレベル2、発音
\\	、カナオウン、変化《動》
\\	何があなたをそうさせたのですか? ・
\\	「信じられないよ。間違ったファイルを削除するなんて」「本当にごめんなさい!何が私をそうさせたのかは、私にも分からないよ!」
\\	変化形 
\\	"【他動】〔財産などとして物を〕所有{しょゆう}[保有{ほゆう}]する〔才能{さいのう}を〕有する〔悪霊{あくりょう}などが人に〕取り付く(人)に~させる ・
\\	何があなたをそうさせたのですか? ・
\\	「信じられないよ。間違ったファイルを削除するなんて」「本当にごめんなさい!何が私をそうさせたのかは、私にも分からないよ!」レベル3、発音
\\	、カナポゼス、変化《動》
\\	、分節
\\	我が社はこのビルの5階以上を占めています。 ・
\\	ビルの3階以下を店舗が占めている。 ・
\\	部屋は全部塞{ぜんぶ ふさ}がっているといわれた。 ・
\\	この席は空いていますか? ・
\\	私はクラスを指導する立場にある。 ・
\\	我が社はこの業界で重要な地位を占めている。	変化形 
\\	【他動】〔場所を家や事務所として〕居住{きょじゅう}[使用{しよう}・占有{せんゆう}]する ・
\\	我が社はこのビルの5階以上を占めています。 ・
\\	ビルの3階以下を店舗が占めている。〔場所や時間などを〕塞{ふさ}ぐ、とる、陣取る{じんどる}◆【用法】受け身で用いられることが多い。 ・
\\	部屋は全部塞{ぜんぶ ふさ}がっているといわれた。 ・
\\	この席は空いていますか?~の職にある、~の地位{ちい}を占める ・
\\	私はクラスを指導する立場にある。 ・
\\	我が社はこの業界で重要な地位を占めている。《軍事》〔場所{ばしょ}を〕占領{せんりょう}する、占拠{せんきょ}する〔人の心や注意{ちゅうい}を〕引く、満たす◆【用法】受け身で用いられることが多い。〔自分自身{じぶん じしん}を〕従事{じゅうじ}させる、忙しくさせる◆【用法】再帰代名詞を目的語に取る。レベル3、発音
\\	、カナアキュパイ、オキュパイ、変化《動》
\\	、分節
\\	現在のところ私のホテルの予約はまだ確定していません。 ・
\\	歯医者さんは、歯が痛いときに治してくれる人です。 ・
\\	申し訳なく存じます[申し訳ないことです]が、その修理にはしばらくかかります。 ・
\\	この問題を解決するには、どうしたらいいと思いますか。 ・
\\	肉ジャガ作ってあなたと一緒に食べようって思ってたんだけど。 ・
\\	《希望を聞かれて》あなたと同じものでいいですよ。 ・
\\	ジョージを2階へ案内してあげて。ジェスと私で食事の用意をするから。 ・
\\	その選手は、試合を買収したとして告発された。 ・
\\	雌猫の避妊手術は、どのくらい費用がかかりますか? ・
\\	お茶を入れますね。 ・
\\	その試合は八百長でした。	変化形 
\\	【自動】固まる、固定{こてい}する、固着{こちゃく}する〔人が〕定住{ていじゅう}する、腰を据える〔注意{ちゅうい}などを〕向ける、集中{しゅうちゅう}させる〈米方言〉〔~する〕準備{じゅんび}をする、用意{ようい}ができる◆【用法】進行形で用いられ、
\\	不定詞を伴う〈米俗〉麻薬{まやく}を打つ【他動】〔正しい位置{いち}にしっかりと〕~を固定{こてい}する[取り付ける]、〔記憶{きおく}に〕~をとどめる〔正確{せいかく}な時間{じかん}・価格{かかく}・場所{ばしょ}などを〕決める、決定{けってい}する、確定{かくてい}する ・
\\	現在のところ私のホテルの予約はまだ確定していません。元の[本来{ほんらい}の・正常{せいじょう}な]状態{じょうたい}に戻す、回復{かいふく}させる、元通りにする、癒す、治療{ちりょう}する、直す、修正{しゅうせい}する、修理{しゅうり}する、修繕{しゅうぜん}する、調整{ちょうせい}する、整える、調節{ちょうせつ}する◆【類】
\\	歯医者さんは、歯が痛いときに治してくれる人です。 ・
\\	申し訳なく存じます[申し訳ないことです]が、その修理にはしばらくかかります。〔問題{もんだい}などを〕解決{かいけつ}する ・
\\	この問題を解決するには、どうしたらいいと思いますか。〔身だしなみを〕整える、〔髪や化粧{けしょう}を〕直す〔必要なものを〕与える、用意{ようい}する、~に備える〔食事{しょくじ}や飲み物を〕用意{ようい}する、準備{じゅんび}する、作る ・
\\	肉ジャガ作ってあなたと一緒に食べようって思ってたんだけど。 ・
\\	《希望を聞かれて》あなたと同じものでいいですよ。 ・
\\	ジョージを2階へ案内してあげて。ジェスと私で食事の用意をするから。恩恵{おんけい}[免罪{めんざい}]を得る、買収{ばいしゅう}する、見逃{みのが}して[大目{おおめ}に見て]もらう ・
\\	その選手は、試合を買収したとして告発された。~をじっと見詰める、目を凝らす、〔思いや注意{ちゅうい}を~に〕じっと向ける、〔~が注意{ちゅうい}を〕引き付ける〔表情{ひょうじょう}を〕かたくさせる《化学》~を凝固{ぎょうこ}させる〔フィルムなどの〕定着処理{ていちゃく しょり}を行う、色留めを行う〔罪や責任{せきにん}などを〕負わせる〔犬や猫に〕去勢{きょせい}[避妊{ひにん}]手術{しゅじゅつ}を行う◆インフォーマルな表現。◆【参考】
\\	雌猫の避妊手術は、どのくらい費用がかかりますか?~に仕返しをする、灸を据える、~を懲らしめる、懲りさせる〔人を〕殺す、片付ける{かたづける}、仕留める{しとめる}〔選挙{せんきょ}や試合{しあい}で〕~を買収{ばいしゅう}する、~に不正工作{ふせい こうさく}を行う、八百長{やおちょう}をする〈米俗〉~に麻薬{まやく}を打つ〔飲食物{いんしょくぶつ}などを〕(人)のために用意{ようい}する ・
\\	お茶を入れますね。【名】〈話〉苦境{くきょう}、困った立場{たちば}調整{ちょうせい}、調節{ちょうせつ}、修理{しゅうり}〈話〉〔応急{おうきゅう}の〕処置{しょち}、解決策{かいけつ さく}〈話〉〔賄賂{わいろ}による〕買収{ばいしゅう}、不正{ふせい}〈話〉〔やめられない〕楽しみ、スリル◆【用法】おどけて使われることが多い。〈俗〉〔麻薬{まやく}の〕注射{ちゅうしゃ}〈話〉〔はっきりとした〕理解{りかい}、認識{にんしき}、決意{けつい}〔船や航空機{こうくうき}の〕航行位置{こうこう いち}(決定{けってい})特定{とくてい}の結果{けっか}になるように手を回してある競争{きょうそう}[競技{きょうぎ}・選挙{せんきょ}]、八百長{やおちょう}の事例{じれい} ・
\\	その試合は八百長でした。レベル2、発音
\\	、カナフィクス、フィックス、変化《動》
\\	屋根を修理すべき時は、太陽が照っている時です。◆問題が起こったら、手遅れになる前に解決すべきであるという意味。	変化形 
\\	【1自動】修理{しゅうり}[修繕{しゅうぜん}]する【1他動】~を直す、~を修理{しゅうり}[修繕{しゅうぜん}]する ・
\\	屋根を修理すべき時は、太陽が照っている時です。◆問題が起こったら、手遅れになる前に解決すべきであるという意味。~を矯正{きょうせい}する、~を治療{ちりょう}する~を生き返らせる、~を復興{ふっこう}させる〔損失{そんしつ}などを〕埋め合わせる、補償{ほしょう}する【1名】修理{しゅうり}、修繕{しゅうぜん}〔修理後{しゅうり ご}の〕状態{じょうたい}◆【参考】
\\	修理個所{しゅうり かしょ}、修理品{しゅうり ひん}【2自動】行く、赴く足しげく通う、しょっちゅう行く【2名】行くこと、〔行って〕滞在{たいざい}することよく行く場所{ばしょ}、しょっちゅう訪れるところレベル2、発音
\\	、カナリペアー、変化《動》
\\	、分節
\\	彼の状態が良くなることを心から願っています。 ・
\\	私の目標は、英語力を徐々に上げていくことです。	変化形 
\\	【自動】良くなる、好転{こうてん}する ・
\\	彼の状態が良くなることを心から願っています。〔能力などが〕伸びる、上達{じょうたつ}する【他動】〔性能{せいのう}・見栄え{みばえ}など〕さらに[一層{いっそう}]良くする、改良{かいりょう}する、改善{かいぜん}する、向上{こうじょう}させる、磨きをかける ・
\\	私の目標は、英語力を徐々に上げていくことです。〔価値{かち}を〕高める〔記録{きろく}などを〕伸ばす〔計画{けいかく}・設備{せつび}・組織{そしき}などを〕拡充{かくじゅう}する、強化{きょうか}するレベル2、発音
\\	、カナインプルーブ、変化《動》
\\	、分節
\\	このランプが点灯したら、インクカートリッジを交換しなければならないことを示しています。◆プリンタの説明書 ・
\\	私たちは古いコンピューターを最新型の新品に替えた。 ・
\\	ビジネスの世界では、キーボードがペンに代わりつつあります。 ・
\\	どんなロボットも人間に完璧に取って代わることはできません。 ・
\\	私の退職後、パーク氏が私の後任になる予定です。 ・
\\	おまえの代わりなんて、いくらでもいるんだぞ。◆【用法】「真面目{まじめ}にやらないなら、やめさせる」などの意味を込めた、非常に冷酷な表現。	変化形 
\\	【他動】~を取り[付け・差し・置き]替える、~を交換{こうかん}[置換{ちかん}]する ・
\\	このランプが点灯したら、インクカートリッジを交換しなければならないことを示しています。◆プリンタの説明書 ・
\\	私たちは古いコンピューターを最新型の新品に替えた。~を元(の位置{いち})に戻す~の後を継ぐ、~に取って代る、~の後任{こうにん}となる ・
\\	ビジネスの世界では、キーボードがペンに代わりつつあります。 ・
\\	どんなロボットも人間に完璧に取って代わることはできません。 ・
\\	私の退職後、パーク氏が私の後任になる予定です。 ・
\\	おまえの代わりなんて、いくらでもいるんだぞ。◆【用法】「真面目{まじめ}にやらないなら、やめさせる」などの意味を込めた、非常に冷酷な表現。レベル3、発音
\\	、カナリープレイス、リプレイス、変化《動》
\\	、分節
\\	《図書館》この本を返却したいんですけど。 ・
\\	返品理由を明記してください。 ・
\\	きょうの良き日が何度も巡って来ますように!◆誕生日の挨拶{あいさつ}。 ・
\\	開票結果が出ない模様です。	変化形 
\\	【自動】戻る、帰る、返還する〔電源{でんげん}が〕復帰{ふっき}する〔病気{びょうき}などが〕再発{さいはつ}する【他動】~を返す、戻す、返却{へんきゃく}する、返品{へんぴん}する、返送{へんそう}する、返上{へんじょう}する ・
\\	《図書館》この本を返却したいんですけど。~に答える、答申{とうしん}する公表{こうひょう}する選出{せんしゅつ}する反響{はんきょう}する【名】返すこと、返却{へんきゃく}、返還{へんかん}、返送{へんそう}◆物を元の場所に返すこと。 ・
\\	返品理由を明記してください。帰ること、帰宅{きたく}、帰国{きこく}、帰郷{ききょう}◆人が元の場所に戻ること。元に戻ること、回復{かいふく}、復帰{ふっき}◆状態や状況が元に戻ること。再び現れること、回帰{かいき}、再来{さいらい}、再発{さいはつ} ・
\\	きょうの良き日が何度も巡って来ますように!◆誕生日の挨拶{あいさつ}。返された[戻された・帰って来た]もの
\\	返品商品{へんぴん しょうひん}返礼{へんれい}、お返し、返済金{へんさい きん}返答{へんとう}、応答{おうとう}、回答{かいとう}〔商品{しょうひん}の売買{ばいばい}による〕利益{りえき}〔投資{とうし}などによる〕利益{りえき}、運用益{うんようえき}◆通例、
\\	◆【同】
\\	〔定期的{ていき てき}な〕財務{ざいむ}報告書{ほうこくしょ}[統計資料{とうけい しりょう}]選挙結果{せんきょ けっか}(報告書{ほうこくしょ})◆通例、
\\	開票結果が出ない模様です。〈英〉選挙{せんきょ}《トランプ》〔ブリッジの〕リターン◆パートナーがリードしたカードと同じスーツをリードすること、またはそのカード。《テニス》リターン、返球{へんきゅう}◆相手にボールを打ち返すこと、またはそのボール。《アメフト》リターン◆キックオフやパントしたボールを敵側が持って前進すること、または前進した距離。〔河川{かせん}や道路{どうろ}の〕曲がり角、湾曲部{わんきょく ぶ}〔タイプライターやキーボードの〕リターン・キー、キャリッジ・リターン〈英〉往復乗車券{おうふく じょうしゃけん}◆【同】
\\	《法律》執行報告(書){しっこう ほうこく(しょ)}◆裁判所の召喚状や令状などに執行官が簡単な報告を書いて返送すること、またはその報告書。〔所得税{しょとくぜい}の〕納税申告書{のうぜい しんこくしょ}◆【同】
\\	【形】戻りの、返りの、返信{へんしん}の返礼{へんれい}の、お返しの再度{さいど}の、もう一度行う復路{ふくろ}の、帰り道の再び戻すための、再循環{さいじゅんかん}の〈英〉〔切符{きっぷ}が〕往復{おうふく}の◆【同】
\\	〔道などの〕方向転換{ほうこう てんかん}の、曲がり角のレベル1、発音
\\	、カナリターン、変化《動》
\\	、分節
\\	もうこれ以上何も思い出したくない。 ・
\\	この情報が将来の計画に役に立つだろうって合意したことを思い出してくれるね。 ・
\\	母はヨーロッパ旅行したときのことを思い出すのが好きだ。 ・
\\	記憶にございません。◆法廷など公の場の質疑応答で ・
\\	その企業は死に至る危険のあるバクテリアに汚染された2万パウンドのひき肉を回収しました。	変化形 
\\	【他動】思い出す、思い起こす ・
\\	もうこれ以上何も思い出したくない。 ・
\\	この情報が将来の計画に役に立つだろうって合意したことを思い出してくれるね。 ・
\\	母はヨーロッパ旅行したときのことを思い出すのが好きだ。 ・
\\	記憶にございません。◆法廷など公の場の質疑応答で〔不良製品{ふりょう せいひん}を〕回収{かいしゅう}する ・
\\	その企業は死に至る危険のあるバクテリアに汚染された2万パウンドのひき肉を回収しました。呼び戻す、撤回{てっかい}する【名】思い出す[起こす]こと、想起{そうき}、回想力{かいそう りょく}呼び戻し、召還{しょうかん}、更迭{こうてつ}取り消し{とりけし}、撤回{てっかい}リコール、欠陥商品回収{けっかん しょうひん かいしゅう}、不良品回収{ふりょうひん かいしゅう}改修{かいしゅう}◆医療機器や自動車などを回収しないで不良部分だけ交換すること《情報検索》再現率{さいげん りつ}レベル4、発音
\\	、カナリコール、変化《動》
\\	、分節
\\	思い出させないでくれよ。◆くぎを刺す	変化形 
\\	【他動】(人)に思い出させる、(人)に気付{きづ}かせる ・
\\	思い出させないでくれよ。◆くぎを刺すレベル3、発音
\\	、カナリマインド、変化《動》
\\	、分節
\\	変化形 
\\	【自動】〔くまなく〕探す、捜索{そうさく}する【他動】〔場所{ばしょ}などを〕探す、捜索{そうさく}する◆いなくなった人や紛失物を見つけようとして行う。〔場所{ばしょ}などを〕探査{たんさ}する、調査{ちょうさ}する◆新しい物を発見するために行う。〔持ち物や体を〕検査{けんさ}する、捜索{そうさく}する◆隠している違法な物などを見つけようとして行う。〔人の気持{きも}ちなどを〕探る、うかがう〔新しいことを〕発見{はっけん}する、探し出す〔公文書{こうぶんしょ}を〕精査{せいさ}する《コ》〔ファイルなどを〕検索{けんさく}する【名】〔徹底{てってい}した〕捜索{そうさく}、探査{たんさ}、追求{ついきゅう}〔所持品{しょじひん}などの〕検査{けんさ}、調査{ちょうさ}《コ》〔ファイルなどの〕検索{けんさく}《法律》〔国際法{こくさいほう}に基づく船舶{せんぱく}の〕捜索{そうさく}レベル3、発音
\\	、カナサーチ、変化《動》
\\	変化形 
\\	【自動】探し求める、捜索{そうさく}する【他動】〔物や場所{ばしょ}などを〕見つけようとする、探し求める〔求めるものを〕得よう[達成{たっせい}しよう]と努力{どりょく}する〔助言{じょげん}や解決策{かいけつ さく}を〕求める、依頼{いらい}する〔場所{ばしょ}や物へ〕向かう、進むしようとする[試みる・努める]◆【用法】
\\	レベル3、発音
\\	、カナシーク、スィーク、変化《動》
\\	引き続きキング博士を司会者としてパネル・ディスカッションが開かれます。 ・
\\	あなたのプレゼンテーションの後で質問が出るかもしれません。 ・
\\	詳細は追って今後数週間以内にお知らせします。 ・
\\	彼女は誰かに後をつけられているような気がしました。 ・
\\	彼女は、ボーイフレンドがどこに行くにもついていきました。 ・
\\	私の子犬は、私がどこに行くにもついてきました。 ・
\\	柳に風。◆ことわざ ・
\\	損失を隠匿するようにとのボスの指示に従う以外に選択の余地がほとんどなかった。 ・
\\	その患者は健康的な生活スタイルをとっていますか? ・
\\	私はスポーツの最新情報を追いかけるのが好きです。 ・
\\	すみませんが、話についていけ(て)ない[おっしゃっていることがよく分からない]のですが。	変化形 
\\	【自動】次に[後に続いて]起きる[来る] ・
\\	引き続きキング博士を司会者としてパネル・ディスカッションが開かれます。 ・
\\	あなたのプレゼンテーションの後で質問が出るかもしれません。 ・
\\	詳細は追って今後数週間以内にお知らせします。話について行く、理解{りかい}する、分かる結果{けっか}として起きる[生じる]【他動】~に続く、~の次に来る~の後について行く、~を追いかける ・
\\	彼女は誰かに後をつけられているような気がしました。 ・
\\	彼女は、ボーイフレンドがどこに行くにもついていきました。 ・
\\	私の子犬は、私がどこに行くにもついてきました。〔容疑者{ようぎしゃ}などを〕監視{かんし}する、尾行{びこう}する〔道などを〕たどる、進む ・
\\	柳に風。◆ことわざ〔標識{ひょうしき}などの方向{ほうこう}へ〕行く、進む〔命令や助言などに〕従う、守る ・
\\	損失を隠匿するようにとのボスの指示に従う以外に選択の余地がほとんどなかった。〔習慣{しゅうかん}や模範{もはん}などに〕従う、倣う〔指導者{しどうしゃ}などを〕信奉{しんぽう}する、模範{もはん}とする〔行動様式{こうどう ようしき}などを〕守る、こだわる ・
\\	その患者は健康的な生活スタイルをとっていますか?〔物の動きなどを〕目で追う、観察{かんさつ}する〔事態の進展などを〕見守る、常に把握{はあく}する ・
\\	私はスポーツの最新情報を追いかけるのが好きです。〔人の話などを〕しっかり聞く、〔人の話などに〕注意{ちゅうい}を向ける ・
\\	すみませんが、話についていけ(て)ない[おっしゃっていることがよく分からない]のですが。〔仕事{しごと}や職業{しょくぎょう}に〕就く、従事{じゅうじ}する〔ツイッターなどで投稿を〕フォローする◆他の人のツイート(つぶやき)を読むために登録すること。【名】後を追うこと、続けて起きること〔ビリヤードの〕押し玉◆【同】
\\	〔新聞{しんぶん}の〕補足記事{ほそく きじ}、続報{ぞくほう}◆【同】
\\	レベル2、発音
\\	、カナファロウ、フォロー、フォロウ、変化《動》
\\	、分節
\\	喜んでお供させていただきます。 ・
\\	私にあなたのお宅までお供させてください。 ・
\\	この書類と共に120ドルの請求書をお送り致します。 ・
\\	料金を添えてお申し込みください。	変化形 
\\	【他動】~について行く、同行{どうこう}する、付き添う、同伴{どうはん}する、一緒{いっしょ}に行く[する]◆【類】
\\	喜んでお供させていただきます。 ・
\\	私にあなたのお宅までお供させてください。~に伴って起こる、付随{ふずい}して起こる、~と同時{どうじ}に生じる~に添える、添付{てんぷ}する、加える ・
\\	この書類と共に120ドルの請求書をお送り致します。 ・
\\	料金を添えてお申し込みください。《音楽》~に伴奏{ばんそう}するレベル3、発音
\\	、カナアカムパニ、アカンパニ、アコンパニ、変化《動》
\\	、分節
\\	変化形 
\\	《単》
\\	《複》
\\	【1自動】〔水面{すいめん}に頭から〕飛び込む、ダイブする〔水底{すいてい}に向かって真っすぐに〕潜る、潜水{せんすい}する〔鳥や航空機{こうくうき}が地面{じめん}や海面{かいめん}に〕急降下{きゅうこうか}する〔人が前後左右{ぜんご さゆう}に〕飛び込む、体を投げ出す〔ある方向{ほうこう}をめがけて〕飛び出す、突進{とっしん}する〔ポケットなどに〕手を突っ込む〔活動{かつどう}に〕のめり込む、熱中{ねっちゅう}する〔価格{かかく}や価値{かち}が〕急落{きゅうらく}する〔水中{すいちゅう}でスキューバを使って〕ダイビングする《スポーツ》〔競技{きょうぎ}で〕飛び込みをする《軍事》〔潜水艦{せんすいかん}が〕潜水{せんすい}する【1他動】〔ポケットなどに手を〕突っ込む〔航空機{こうくうき}などを〕急降下{きゅうこうか}させる【1名】〔水面{すいめん}への頭からの〕飛び込み〔水底{すいてい}への〕潜水{せんすい}〔鳥や航空機{こうくうき}の〕急降下{きゅうこうか}〔前後左右{ぜんご さゆう}への〕ジャンプ、体を投げ出すこと〔ある方向{ほうこう}への〕飛び出し、突進{とっしん}〔価格{かかく}や価値{かち}の〕急落{きゅうらく}〔スキューバを使った〕ダイビング《スポーツ》〔競技{きょうぎ}の〕飛び込み、ダイビング《軍事》〔潜水艦{せんすいかん}の〕潜水{せんすい}《アメフト》ダイブ◆ボールを持つ選手がパスをしないで、比較的短い距離を稼ぐために前進すること。〈話〉〔サッカーの〕ダイブ◆フリー・キックやペナルティ・キック権を得るために、故意にまたは大げさに倒れること。〈話〉いかがわしい[怪しげな]バー[クラブ]〈俗〉〔ボクシングの〕八百長{やおちょう}ノックアウト◆試合に負けるために、ノックアウトされたように見せること。◆【参考】
\\	【2名】
\\	の複数形{ふくすう けい}レベル4、発音
\\	、カナダイブ、変化《動》
\\	変化形 
\\	【自動】〔手足{てあし}を使ってはしごなどを〕(よじ)登る〔ゆっくりと楽に階段{かいだん}などを〕上る〔ある場所{ばしょ}から苦労{くろう}して〕抜け出る、はい出す〔坂道{さかみち}などが〕上り坂である〔スポーツとして〕登山{とざん}をする、ロック・クライミングをする《植物》〔つるなどで〕伸びる、上る〔量や価値{かち}が〕増大{ぞうだい}する、上昇{じょうしょう}する〔人が〕出世{しゅっせ}する、昇進{しょうしん}する【他動】〔手足{てあし}を使ってはしごなどを〕(よじ)登る〔ゆっくりと楽に階段{かいだん}などを〕上る《植物》〔つるなどで~を伝って〕伸びる、上る【名】〔一番上{いちばん うえ}に〕登ること、上がること登山{とざん}場所{ばしょ}[道]〔量や価値{かち}の〕増大{ぞうだい}、上昇{じょうしょう}レベル1、発音
\\	、カナクライム、変化《動》
\\	急ぐことはない。 ・
\\	何を慌てているんだ。	変化形 
\\	【自動】急いでする、急ぐ【他動】~を急がせる【名】急いで行う[進める]こと急ぐ必要、急ぎたいこと ・
\\	急ぐことはない。〔不必要{ふひつよう}に〕慌てること ・
\\	何を慌てているんだ。レベル1、発音
\\	、カナハーリ、ハリー、変化《動》
\\	、分節
\\	変化形 
\\	【1自動】追いかける急ぐ◆【同】
\\	【1他動】~を追う、追跡{ついせき}する、追いかける、(女の尻を)追い回す~を追い払う《野球俗》〔投手{とうしゅ}〕をノックアウトする、降板{こうばん}させる【1名】追跡{ついせき}、追撃{ついげき}追い求めるもの、追求物{ついきゅう ぶつ}狩り、〔狩りの〕獲物〔競馬{けいば}の〕障害競走{しょうがい きょうそう}◆【同】
\\	《音楽》〔ジャズの〕チェイス、掛け合い◆交互にソロで演奏すること。【2他動】~に溝[刻み]をつける〔ねじ山を〕切る〔金属を〕くぼませる、浮き出させる【2名】銃口{じゅうこう}の後ろの砲身部{ほうしん ぶ}〔部品{ぶひん}などに刻んだ〕溝【3名】《印刷》チェース◆活字を組むための四角い枠。レベル3、発音
\\	、カナチェイス、変化《動》
\\	変化形 
\\	【自動】追跡{ついせき}する、追いかける続ける、続行{ぞっこう}する【他動】~を追跡{ついせき}する、~を追いかける〔努力{どりょく}して~を〕追求{ついきゅう}する、達成{たっせい}しようとする〔コースなどを〕たどる、進む〔仕事{しごと}などを〕続行{ぞっこう}する、先に進める〔女性{じょせい}に〕求愛{きゅうあい}する、言い寄る~に付きまとう、~をしつこく追い回すレベル4、発音
\\	、カナパースー、パスー、変化《動》
\\	、分節
\\	この家系の子孫は現在までたどることができる。 ・
\\	犯人の手掛かりはまだつかめていない。	変化形 
\\	【1自動】〔道に沿って〕進む、たどる〔~の間を〕探す、突き止める〔源泉{げんせん}や歴史{れきし}が〕たどれる、遡{さかのぼ}れる【1他動】~の跡を追う[たどる・付ける]〔調査{ちょうさ}して〕~を探し出す[突き止める]〔~の発達{はったつ}や歴史{れきし}を〕明らかにする、解明{かいめい}する ・
\\	この家系の子孫は現在までたどることができる。〔透明{とうめい}な紙を当てて図などを〕トレースする、敷き写す〔線や図などを〕描く、〔物や場所{ばしょ}を〕スケッチする〔文字{もじ}を〕丁寧{ていねい}に書く《建築》~にはざま飾りを施す〔数値{すうち}をグラフなどに〕トレースする【1名】〔目に見える〕跡、足跡{あしあと}、形跡{けいせき}〔人がいたことや物があったことの〕手掛{てが}かり、証拠{しょうこ} ・
\\	犯人の手掛かりはまだつかめていない。〔辛うじて感知{かんち}できるわずかな〕しるし、分量{ぶんりょう}《化学》痕跡{こんせき}、微量{びりょう}◆一定基準以下または測定限界値以下の量。〔人や動物{どうぶつ}が踏みならした〕小道{こみち}、山道{やまみち}〔心電計{しんでんけい}や地震計{じしんけい}などの〕出力{しゅつりょく}[記録{きろく}](波形{はけい})〔製図{せいず}や美術{びじゅつ}の〕トレース図[画]《数学》跡、トレース◆正方行列の対角成分の和。《気象》微量{びりょう}◆降水はあるが測定装置では0と表示される量。《言語学》痕跡{こんせき}◆生成文法で、移動した要素の跡に残ると考えられるもの。記憶痕跡{きおく こんせき}◆【同】
\\	《電子工学》〔プリント配線基板上の〕線、配線{はいせん}◆【同】
\\	【2名】〔馬具{ばぐ}の〕引き革[綱]レベル4、発音
\\	、カナトゥレイス、トレイス、変化《動》
\\	スカウトされたの?すごい! ・
\\	もっと若い時から水泳を楽しんでいればよかった。	変化形 
\\	【他動】~を発見{はっけん}する、見いだす、スカウトする、発掘{はっくつ}する ・
\\	スカウトされたの?すごい!〔秘密{ひみつ}を〕かぎつける、〔秘密{ひみつ}などを〕つかむ~の良さが分かる[を理解{りかい}する]、~を楽しむ ・
\\	もっと若い時から水泳を楽しんでいればよかった。レベル1、発音
\\	、カナディスカバー、変化《動》
\\	、分節
\\	彼の考えは的中しました。 ・
\\	現実は違っていた。 ・
\\	あなたは医学のことが何も分かっていませんね。今のコメントが何よりの証拠です。 ・
\\	そのことは、チームワークの大切さを改めて示しました。 ・
\\	あなたは何を証明しようとしているのですか。/何をそんなにむきになってるの。 ・
\\	できるものなら私が間違っていると証明してください。◆「できないだろう」という含み。自説に自信がある場合など。	変化形 
\\	【自動】〔~であると〕分かる、判明{はんめい}する ・
\\	彼の考えは的中しました。 ・
\\	現実は違っていた。〔探針などで〕探る【他動】〔~ということを〕証明{しょうめい}する ・
\\	あなたは医学のことが何も分かっていませんね。今のコメントが何よりの証拠です。 ・
\\	そのことは、チームワークの大切さを改めて示しました。 ・
\\	あなたは何を証明しようとしているのですか。/何をそんなにむきになってるの。〈英〉〔遺言書を〕検認{けんにん}する~が…であることを示す[証明{しょうめい}する] ・
\\	できるものなら私が間違っていると証明してください。◆「できないだろう」という含み。自説に自信がある場合など。レベル2、発音
\\	、カナプルーブ、変化《動》
\\	または
\\	それでは何の解決にもなりません。	変化形 
\\	【自他動】〔謎・問題などを〕解く、解決{かいけつ}する ・
\\	それでは何の解決にもなりません。溶解{ようかい}する負債{ふさい}を支払うレベル2、発音
\\	、カナサルブ、ソルブ、変化《動》
\\	結論を出すには早過ぎると思う。 ・
\\	彼らには何らかの公にできない事情があるのだと結論せざるを得ない。 ・
\\	警察は死因を自殺と断定しました。 ・
\\	彼は話を諺で結んだ。 ・
\\	ではこれで職員会議を終了します。 ・
\\	彼は今日保険の契約を5件も成立させた。	変化形 
\\	【自動】結論{けつろん}を出す[下す] ・
\\	結論を出すには早過ぎると思う。 ・
\\	彼らには何らかの公にできない事情があるのだと結論せざるを得ない。〔話や会議{かいぎ}が〕終わる【他動】~と結論{けつろん}を出す[下す]、結論{けつろん}としては~である、~と判断{はんだん}を下す、断定{だんてい}する、推論{すいろん}する、~に結末{けつまつ}をつける、~を決定{けってい}する ・
\\	警察は死因を自殺と断定しました。〔話を〕結ぶ ・
\\	彼は話を諺で結んだ。~を完結{かんけつ}する、終える、無事{ぶじ}に済ませる、完了{かんりょう}する、終了{しゅうりょう}する、終結{しゅうけつ}させる ・
\\	ではこれで職員会議を終了します。〔条約{じょうやく}・契約{けいやく}などを〕結ぶ、締結{ていけつ}させる、成立{せいりつ}させる、妥結{だけつ}する ・
\\	彼は今日保険の契約を5件も成立させた。レベル4、発音
\\	、カナコンクルード、変化《動》
\\	、分節
\\	以下の通り注文致します。 ・
\\	ピザをもう一人前注文しよう。 ・
\\	彼らは武器の捜索の一層の強化を命じた。 ・
\\	要請ではなく命令です。 ・
\\	ご注文、ありがとうございます。発注情報を受信致しました。このあとすぐに注文確認のメールをお送りします。◆オンラインショッピング ・
\\	好きな食べ物は
\\	の順です。 ・
\\	身辺をきちんと整理しなさい。 ・
\\	静粛に! ・
\\	この法廷では静粛に願います。	変化形 
\\	【自動】注文{ちゅうもん}する ・
\\	以下の通り注文致します。指図{さしず}する、命令{めいれい}する順序付ける、整える、整理{せいり}する【他動】~を注文{ちゅうもん}する、発注{はっちゅう}する ・
\\	ピザをもう一人前注文しよう。~を命令{めいれい}する、命じる、指図{さしず}する ・
\\	彼らは武器の捜索の一層の強化を命じた。~を順序付ける、整える、整理{せいり}する【名】指令{しれい}、命令{めいれい}◆【語源】ラテン語「列、配列」 ・
\\	要請ではなく命令です。発注{はっちゅう}、注文{ちゅうもん}、注文書{ちゅうもんしょ}、注文品{ちゅうもんひん}、注文料理{ちゅうもん りょうり} ・
\\	ご注文、ありがとうございます。発注情報を受信致しました。このあとすぐに注文確認のメールをお送りします。◆オンラインショッピング順、順序{じゅんじょ}、順番{じゅんばん}、順位{じゅんい}、序列{じょれつ}、系列{けいれつ} ・
\\	好きな食べ物は
\\	の順です。《数学》位数{いすう}、次数{じすう}、階数{かいすう}整頓{せいとん}、整列{せいれつ} ・
\\	身辺をきちんと整理しなさい。健康{けんこう}な状態{じょうたい}、常態{じょうたい}自然{しぜん}の理法{りほう}、秩序{ちつじょ}、道理{どうり} ・
\\	静粛に! ・
\\	この法廷では静粛に願います。治安{ちあん}、体制{たいせい}慣例{かんれい}、規則{きそく}、習慣{しゅうかん}階級{かいきゅう}、地位{ちい}、等級{とうきゅう}、品級勲章{くんしょう}種類{しゅるい}〔生物分類{せいぶつ ぶんるい}の〕目◆【参考】
\\	教団{きょうだん}、修道会{しゅうどうかい}、聖職{せいしょく}聖職者{せいしょくしゃ}や天使{てんし}の階位{かいい}議事規則{ぎじ きそく}、規定{きてい}為替{かわせ}割引入場券{わりびき にゅうじょうけん}、許可書{きょかしょ}
\\	《英史》騎士団{きしだん}《建築》オーダー、円柱形式{えんちゅう けいしき}、建築様式{けんちく ようしき}◆柱によって区別される古典建築の様式、または一般に建築様式を指す。古典建築の円柱形式はルネサンス期に五つに分類され、古代ギリシャのドリス式
\\	、イオニア式
\\	、コリント式
\\	、古代ローマのコンポジット式
\\	、エトルリアのトスカナ式
\\	であるとされた。レベル1、発音
\\	、カナオーダー、変化《動》
\\	、分節
\\	変化形 
\\	【他動】~に従う、(人)の言うことに従う[を聞く]レベル4、発音
\\	、カナオウベイ、オベイ、変化《動》
\\	、分節
\\	国連は、平和維持部隊を派遣しました。 ・
\\	彼は警察を罷免された。	変化形 
\\	【他動】~を強要{きょうよう}する、~に(…すること)を余儀{よぎ}なくさせる、強いる、無理{むり}にさせる、強制{きょうせい}する~を押し進める、強引{ごういん}に通す《野球》(走者{そうしゃ})を封殺{ふうさつ}する【名】〔人や物が持つ〕力、エネルギー、強さ{つよさ}、体力{たいりょく}、腕力{わんりょく}〔軍事行動・治安維持などを目的とする〕部隊{ぶたい} ・
\\	国連は、平和維持部隊を派遣しました。風力{ふうりょく}◆【略】
\\	影響力{えいきょうりょく}、勢力{せいりょく}(法律{ほうりつ}などの)実施{じっし}、施行{しこう}、効力{こうりょく}《物理》力◆【略】
\\	〔警察官の集合体としての〕警察{けいさつ} ・
\\	彼は警察を罷免された。
\\	〔国の〕軍隊{ぐんたい}
\\	〔組織としての〕警察◆【同】
\\	レベル2、発音
\\	、カナフォース、変化《動》
\\	変化形 
\\	【他動】~に抵抗{ていこう}[反抗{はんこう}・反撃{はんげき}]する、~と抗争{こうそう}する~に侵されない、耐える、我慢{がまん}する【名】レジスト◆パターンを塗り分けするための保護膜レベル4、発音
\\	、カナリズィスト、レジスト、変化《動》
\\	、分節
\\	こんなことが起きているなんて私は絶対信じません。 ・
\\	私は負けませんよ。負けてたまるか。 ・
\\	彼の金銭援助の申し出を断った。 ・
\\	労働者は、規定の時間を超える残業を拒否する権利があります。 ・
\\	その店の主人は私にトイレを使わせてくれなかった。	変化形 
\\	【1自動】拒む、許さない〔馬が〕障害物{しょうがいぶつ}の前で立ち止まる【1他動】〔~することを〕嫌がる、したくないと言う◆断固とした気持ちを表す。心理的態度の問題なので客観的には拒否不可能な場合にも使う。 ・
\\	こんなことが起きているなんて私は絶対信じません。 ・
\\	私は負けませんよ。負けてたまるか。〔~を受け入れるのを〕拒む、拒絶{きょぜつ}する、断る ・
\\	彼の金銭援助の申し出を断った。 ・
\\	労働者は、規定の時間を超える残業を拒否する権利があります。〔人に~を与えるのを〕許さない ・
\\	その店の主人は私にトイレを使わせてくれなかった。〔馬が障害物{しょうがいぶつ}を〕越えようとしない【2名】くず、ごみ、塵芥{じんかい}、廃物{はいぶつ}、遺棄物{いき ぶつ}【2形】くずの、ごみの、無価値{むかち}のレベル3、発音《動》
\\	《名》
\\	、カナリフューズ、レフュース、変化《動》
\\	それらの犯罪は同一人物によって犯された。 ・
\\	私は調停役でいようとしてきたし、子どもたちにはいつも罪を犯すなと言い聞かせている。 ・
\\	ジェーンは週末はずっと勉強していた。 ・
\\	彼の家族は仕方なく彼を精神科病院に収容しました。	変化形 
\\	【自動】約束{やくそく}する、誓約{せいやく}する【他動】〔罪などを〕犯す、〔悪事{あくじ}を〕はたらく ・
\\	それらの犯罪は同一人物によって犯された。 ・
\\	私は調停役でいようとしてきたし、子どもたちにはいつも罪を犯すなと言い聞かせている。〔~を…に〕委ねる、委任{いにん}する~に責任{せきにん}を持つ、~と約束{やくそく}する、~を誓う、~を明言{めいげん}[表明{ひょうめい}]する~にすべてをささげる、~に全力{ぜんりょく}を傾ける、~に最大限{さいだいげん}の努力{どりょく}を投じる ・
\\	ジェーンは週末はずっと勉強していた。〔~を…に〕収容{しゅうよう}する ・
\\	彼の家族は仕方なく彼を精神科病院に収容しました。~を危うくする〔戦闘部隊{せんとう ぶたい}を〕送るレベル4、発音
\\	、カナコミット、変化《動》
\\	、分節
\\	どのくらい割引してもらえますか。	変化形 
\\	【他動】提案{ていあん}する、結婚{けっこん}を申し込む、~を提案{ていあん}する、計画{けいかく}する、提議{ていぎ}する、発議{はつぎ}する、申し込む、企てる、建議{けんぎ}する、もくろむ、画策{かくさく}する ・
\\	どのくらい割引してもらえますか。レベル2、発音
\\	、カナプロポーズ、プロポウズ、変化《動》
\\	、分節
\\	時間に追われていらっしゃるのでしたら、近道を行くことをお勧めします。 ・
\\	水曜日の1時にスターバックスで待ち合わせるということでいかがでしょうか。 ・
\\	お会いする場所は東京ホテルでいかがでしょうか。 ・
\\	もっと気を付けた方がいいですよ。 ・
\\	私たちはボスに何と言えばいいのか教えてくださいますか? ・
\\	私たちはどうすべきかなのか教えてくださいますか? ・
\\	ではどうしたらいいというんだ。◆けんかを売っているように聞こえるので使用に注意。 ・
\\	何をおっしゃりたいのか、教えていただけますか? ・
\\	彼のそぶりからすると興味がないようです。	変化形 
\\	【他動】~を[~しようと]提案{ていあん}する[提言{ていげん}する・勧める]、〔提案{ていあん}などを〕持ちかける ・
\\	時間に追われていらっしゃるのでしたら、近道を行くことをお勧めします。 ・
\\	水曜日の1時にスターバックスで待ち合わせるということでいかがでしょうか。 ・
\\	お会いする場所は東京ホテルでいかがでしょうか。 ・
\\	もっと気を付けた方がいいですよ。 ・
\\	私たちはボスに何と言えばいいのか教えてくださいますか? ・
\\	私たちはどうすべきかなのか教えてくださいますか? ・
\\	ではどうしたらいいというんだ。◆けんかを売っているように聞こえるので使用に注意。〔悪いことを〕入れ知恵する〔名前や表現などが〕~を意味{いみ}する、示唆{しさ}する、暗示{あんじ}する、それとなく示す[言う] ・
\\	何をおっしゃりたいのか、教えていただけますか? ・
\\	彼のそぶりからすると興味がないようです。レベル2、発音
\\	、カナサジェスト、変化《動》
\\	、分節
\\	私なら、それに反対する勧告を出します。/それはお勧めできません[やめておいた方がいいと思いますよ]。 ・
\\	《コ》テープ・ドライブおよびディスク・ドライブを回転させるのは好ましくない。 ・
\\	何がお勧めですか。 ・
\\	3年プランを選択なさり、4年目のサービスを無料でお受けになることをお勧めします。 ・
\\	その医師は私に体重を減らすことを勧めました。	変化形 
\\	【自動】勧告{かんこく}する[を出す] ・
\\	私なら、それに反対する勧告を出します。/それはお勧めできません[やめておいた方がいいと思いますよ]。【他動】~を推薦{すいせん}[推奨{すいしょう}]する、~を薦める◆【対】
\\	《コ》テープ・ドライブおよびディスク・ドライブを回転させるのは好ましくない。 ・
\\	何がお勧めですか。 ・
\\	3年プランを選択なさり、4年目のサービスを無料でお受けになることをお勧めします。〔人に~することを〕勧める、提言{ていげん}[勧告{かんこく}]する ・
\\	その医師は私に体重を減らすことを勧めました。〔人に〕~を気に入らせるレベル3、発音
\\	、カナレコメンド、変化《動》
\\	、分節
\\	彼は、その営業担当者が商品に関してうそをついたと主張しました。 ・
\\	彼は、その契約を終了できる権利を主張しました。 ・
\\	その地震はかなり多くの人々の命を奪いました。 ・
\\	私は、スクラッチくじで賞金500ドルを獲得しました。 ・
\\	子どもたちはさらに問題を起こしてもっと注意を引こうとします。	変化形 
\\	【他動】〔証拠なしに~が本当であると〕主張{しゅちょう}する、言い張る、断言する ・
\\	彼は、その営業担当者が商品に関してうそをついたと主張しました。〔当然のこととして〕~を要求{ようきゅう}[請求{せいきゅう}]する、~を求める ・
\\	彼は、その契約を終了できる権利を主張しました。〔~の所有権を〕主張{しゅちょう}する〔命を〕奪う ・
\\	その地震はかなり多くの人々の命を奪いました。〈主に英〉〔賞品などを〕獲得する ・
\\	私は、スクラッチくじで賞金500ドルを獲得しました。〈主に英〉〔人の注意を〕引く ・
\\	子どもたちはさらに問題を起こしてもっと注意を引こうとします。【名】〔当然の〕要求{ようきゅう}、主張{しゅちょう}、申し立て{もうしたて}〔~を〕要求{ようきゅう}[請求{せいきゅう}]する権利{けんり}、〔~の〕正当{せいとう}な資格{しかく}〔消費者から商品製造会社などに対する〕苦情{くじょう}特許請求{とっきょ せいきゅう}の範囲{はんい}、宣伝文句{せんでん もんく}《法律》債権{さいけん}レベル2、発音
\\	、カナクレイム、変化《動》
\\	ぜひと[どうしてもと・そこまで]おっしゃるなら。/お言葉に甘えて。◆【用法】スピーチを頼まれたときなど、断るに断れない場合に使う。もちろん本当はやりたいのにポーズで言うこともある。	変化形 
\\	【自動】〔反対{はんたい}にもかかわらず〕主張{しゅちょう}する、断言{だんげん}する〔しつこく〕要求{ようきゅう}する、求める ・
\\	ぜひと[どうしてもと・そこまで]おっしゃるなら。/お言葉に甘えて。◆【用法】スピーチを頼まれたときなど、断るに断れない場合に使う。もちろん本当はやりたいのにポーズで言うこともある。【他動】〔反対{はんたい}にもかかわらず~であると〕主張{しゅちょう}する、断言{だんげん}する◆【用法】
\\	節が続く。〔~であることをしつこく〕要求{ようきゅう}する、求める◆【用法】
\\	節が続く。レベル3、発音
\\	、カナインシスト、インスィスト、変化《動》
\\	、分節
\\	彼と言い争っても、ろくなことにならない。	変化形 
\\	【自動】〔感情的{かんじょう てき}にまたは執拗{しつよう}に〕異議{いぎ}を唱える、反対{はんたい}する〔議論{ぎろん}に肯定的{こうてい てき}な〕根拠{こんきょ}[論拠{ろんきょ}]を示す言い争う、論争{ろんそう}[口論{こうろん}]する ・
\\	彼と言い争っても、ろくなことにならない。【他動】
\\	節を〕主張{しゅちょう}する、正当{せいとう}だと論じる〔多くの観点{かんてん}から~について〕議論{ぎろん}する、討論{とうろん}する〔論拠{ろんきょ}を示して人を〕説得{せっとく}する、説きつける〔主語が〕~の証拠{しょうこ}となる、~を示しているレベル3、発音
\\	、カナアーギュー、変化《動》
\\	、分節
\\	最初にご用件から伺いましょう。 ・
\\	そういう話題は議論を避けた方がいいですよ。 ・
\\	冷静になって話そう。 ・
\\	この章では、次[以下]のテーマについて説明します。	変化形 
\\	【他動】〔他の人と~について〕話し合う、意見{いけん}を交わす、議論{ぎろん}する ・
\\	最初にご用件から伺いましょう。 ・
\\	そういう話題は議論を避けた方がいいですよ。〔独りで〕~を考察{こうさつ}[検討{けんとう}・説明{せつめい}]する ・
\\	冷静になって話そう。 ・
\\	この章では、次[以下]のテーマについて説明します。レベル2、発音
\\	、カナディスカス、デスカス、変化《動》
\\	、分節
\\	私は、夕飯に何を食べようか考え中です[迷っています]。 ・
\\	私は、それを買おうか迷っています。	変化形 
\\	【自動】公開討論{こうかい とうろん}する、ディベートする熟慮{じゅくりょ}する、熟考{じゅっこう}する ・
\\	私は、夕飯に何を食べようか考え中です[迷っています]。論争{ろんそう}する、議論{ぎろん}する【他動】〔論点{ろんてん}について公開{こうかい}で〕討論{とうろん}する、ディベートする〔いろいろな角度{かくど}から~かどうかを〕じっくり考える ・
\\	私は、それを買おうか迷っています。~について論争{ろんそう}[議論{ぎろん}]する【名】〔賛成{さんせい}・反対{はんたい}に分かれた公開{こうかい}の〕討論会{とうろんかい}、ディベート〔長い〕熟慮{じゅくりょ}、熟考{じゅっこう}〔長く続く〕議論{ぎろん}、論争{ろんそう}レベル4、発音
\\	、カナディベイト、デベイト、ディベート、変化《動》
\\	、分節
\\	テレコンの時間が8時から8時半に変わりましたのでご連絡します。	変化形 
\\	【自動】〔人に〕アドバイス[助言{じょげん}]を求める〔~について〕アドバイス[忠告{ちゅうこく}・助言{じょげん}・勧告{かんこく}]する【他動】〔人に〕アドバイス[忠告{ちゅうこく}・助言{じょげん}・勧告{かんこく}]する、勧める〔情報を〕通知{つうち}する、報告{ほうこく}する、教える ・
\\	テレコンの時間が8時から8時半に変わりましたのでご連絡します。レベル2、発音
\\	、カナアドゥバイズ、アドバイズ、変化《動》
\\	、分節
\\	《就業規則》(電話に応答するときは)必ず最初に社名を述べること。 ・
\\	そういうことはもっと早く言ってくださいよ。 ・
\\	その失敗については誰も何も言わなかった。	変化形 
\\	【他動】〔会話の中で簡潔に・簡単に・手短に〕~に言及{げんきゅう}する、~の名前{なまえ}を挙{あ}げる、~に触れる、~のことを話に出す、~を口にする ・
\\	《就業規則》(電話に応答するときは)必ず最初に社名を述べること。 ・
\\	そういうことはもっと早く言ってくださいよ。《軍事》〔報告書{ほうこくしょ}で人の名前{なまえ}を挙げて〕勇気{ゆうき}をたたえる、顕彰{けんしょう}する【名】言及{げんきゅう}、名前{なまえ}を挙げること ・
\\	その失敗については誰も何も言わなかった。《軍事》〔報告書{ほうこくしょ}での勇気{ゆうき}ある行動{こうどう}の〕顕彰{けんしょう}〔一般{いっぱん}に功績{こうせき}などの〕顕彰{けんしょう}、表彰{ひょうしょう}《イ》メンション◆ツイッターにおいて、
\\	ユーザー名を含むツイートのこと。◆【参考】
\\	レベル2、発音
\\	、カナメンション、変化《動》
\\	、分節
\\	他に意見はございませんか。 ・
\\	あなたと同意見です。	変化形 
\\	【自動】コメントする、批評{ひひょう}する、論評{ろんぴょう}する【他動】~と批評{ひひょう}する
\\	節を従える。【名】論評{ろんぴょう}、コメント、意見{いけん}、批評{ひひょう}、所感{しょかん} ・
\\	他に意見はございませんか。 ・
\\	あなたと同意見です。解説{かいせつ}、注釈{ちゅうしゃく}《言語》評言{ひょうげん}◆文中で新しい情報を伝える部分のこと。◆【同】
\\	《コ》コメント◆コンパイル時などには無視されるプログラムの注釈。レベル3、発音
\\	、カナカメント、コメント、変化《動》
\\	、分節
\\	説明してくれる? ・
\\	もう一度説明してもらえませんか? ・
\\	そういうことだったのか。	変化形 
\\	【自動】〔詳細{しょうさい}について〕説明{せつめい}する、解説{かいせつ}する ・
\\	説明してくれる?〔原因{げんいん}について〕明らかにする、説明{せつめい}する【他動】〔~を詳細{しょうさい}に〕説明{せつめい}する、解説{かいせつ}する ・
\\	もう一度説明してもらえませんか?〔~の意味{いみ}を〕明確{めいかく}にする、解釈{かいしゃく}する〔~の原因{げんいん}を〕明らかにする、説明{せつめい}する ・
\\	そういうことだったのか。〔自分{じぶん}の考えや行動{こうどう}を〕釈明{しゃくめい}する、弁明{べんめい}する◆【用法】再帰代名詞を目的語に取る。レベル2、発音
\\	、カナイクスプレイン、エクスプレイン、変化《動》
\\	、分節
\\	変化形 
\\	【他動】〔考えや気持{きも}ちを言葉{ことば}で〕述べる、言う〔気持{きも}ちなどを態度{たいど}で〕表す、示す〔自分{じぶん}の考えや気持{きも}ちを〕表す、伝える◆【用法】再帰代名詞を目的語に取る。〔作品{さくひん}などで意味{いみ}を〕伝達{でんたつ}する、表現{ひょうげん}する〔数値{すうち}などを〕記号{きごう}[式]で表す〔ジュースなどを〕搾り出す〔手紙{てがみ}や荷物{にもつ}などを〕速達{そくたつ}[至急便{しきゅう びん}]で送る《生物》〔遺伝子{いでんし}や遺伝的性質{いでん てき せいしつ}を〕発現{はつげん}させる【名】速達(便){そくたつ(びん)}、至急便{しきゅう びん}◆速達の制度やそれを利用した郵便や荷物。〔バスや電車{でんしゃ}の〕急行{きゅうこう}〈英〉特使{とくし}、特報{とくほう}【形】〔述べられたことが〕明確{めいかく}な、明白{めいはく}な〔そのためだけの〕特別{とくべつ}の、特定{とくてい}の〔荷物{にもつ}などが〕速達{そくたつ}の、至急便{しきゅう びん}の〔電車{でんしゃ}やバスなどが〕急行{きゅうこう}の、直行{ちょっこう}の〔道路{どうろ}やレーンが〕高速(用){こうそく(よう)}の【副】〔荷物や輸送が〕急行{きゅうこう}で、速達{そくたつ}で〈古〉=
\\	レベル2、発音
\\	、カナイクスプレス、エクスプレス、変化《動》
\\	、分節
\\	どう言って説明したらいいのかなあ。 ・
\\	それがどんなものか説明してくれませんか? ・
\\	私の感情[気持ち]を述べる[言い表す]のは簡単なことではありません。 ・
\\	あなたの手作り料理にどんなに印象付けられた[感動した]か、言葉で言い表すのは簡単ではありません。	変化形 
\\	【他動】〔言葉・文章で〕~を言い表す[説明{せつめい}する] ・
\\	どう言って説明したらいいのかなあ。 ・
\\	それがどんなものか説明してくれませんか? ・
\\	私の感情[気持ち]を述べる[言い表す]のは簡単なことではありません。 ・
\\	あなたの手作り料理にどんなに印象付けられた[感動した]か、言葉で言い表すのは簡単ではありません。〔絵などが〕~を描写{びょうしゃ}[表現{ひょうげん}]する〔人や物事{ものごと}を~だと〕特徴付ける、類型化{るいけい か}する◆【用法】「~だと」は
\\	句で表される。〈文〉〔図や輪郭{りんかく}を〕描く、引くレベル2、発音
\\	、カナディスクライブ、デスクライブ、変化《動》
\\	、分節
\\	・この絵画は自然の美しさを表しています。 ・
\\	これらの数字は、当社の過去10年間の年間売上高を示している。	変化形 
\\	【他動】〔ものが〕~を表す[示す・象徴{しょうちょう}する・意味{いみ}する] ・
\\	・この絵画は自然の美しさを表しています。 ・
\\	これらの数字は、当社の過去10年間の年間売上高を示している。~の公式{こうしき}な代表{だいひょう}[代理人{だいりにん}]を務める~選出{せんしゅつ}の議員{ぎいん}[代議士{だいぎし}]を務める~の代弁者{だいべんしゃ}[スポークスマン]として話す[行動{こうどう}する]〔絵などで〕~を描く[描写{びょうしゃ}する]〔言葉{ことば}で〕~を説明{せつめい}[表現{ひょうげん}]する〔はっきりと〕~を心に描く[浮かべる]〔抗議{こうぎ}や苦情{くじょう}を〕指摘{してき}する、突き付ける〔人やものが性質{せいしつ}〕の典型{てんけい}となる[を代表{だいひょう}する]~に相当{そうとう}する、~と同等{どうとう}である〔劇などを〕上演{じょうえん}する、制作{せいさく}する〔劇などの役を〕演じるレベル3、発音
\\	、カナリプリゼント、レプリゼント、変化《動》
\\	、分節
\\	画面には人体の熱分布が表示される。 ・
\\	《コ》サーバーのモニターにはエラー・メッセージが表示される。	変化形 
\\	【他動】~を表示{ひょうじ}する、掲示{けいじ}する、陳列{ちんれつ}する、展示{てんじ}する ・
\\	画面には人体の熱分布が表示される。 ・
\\	《コ》サーバーのモニターにはエラー・メッセージが表示される。~を誇示{こじ}する、(行動{こうどう}や振る舞いを)見せる~を広げる【名】表示{ひょうじ}、露呈{ろてい}ディスプレー、表示部{ひょうじ ぶ}陳列{ちんれつ}、展示{てんじ}、展示品{てんじ ひん}レベル3、発音
\\	、カナディスプレイ、デスプレイ、変化《動》
\\	、分節
\\	作品には手を触れないでください。 ・
\\	その展覧会には何枚の絵が展示されているのですか? ・
\\	別紙
\\	を見よ。	変化形 
\\	【自動】展示{てんじ}する、展覧会{てんらんかい}を開く【他動】〔作品などを公然{こうぜん}と〕展示{てんじ}する、発表{はっぴょう}する、見せる、示す、誇示{こじ}する◆【直訳】外へ
\\	差し出す
\\	〔感情{かんじょう}を〕表に出す、示す〔証拠{しょうこ}を〕提示{ていじ}する、提出{ていしゅつ}する【名】展示{てんじ}、展覧{てんらん}展覧会{てんらんかい}、展示会{てんじかい}、展示品{てんじ ひん} ・
\\	作品には手を触れないでください。 ・
\\	その展覧会には何枚の絵が展示されているのですか?証拠{しょうこ}、証拠{しょうこ}物件{ぶっけん}[書類{しょるい}]添付書類{てんぷ しょるい}、別紙{べっし} ・
\\	別紙
\\	を見よ。レベル4、発音《動》
\\	、カナイグジビット、エグズィビット、変化《動》
\\	、分節
\\	雑誌は毎週火曜日発刊される。	変化形 
\\	【他動】~を出版{しゅっぱん}する、刊行{かんこう}する、発行{はっこう}する ・
\\	雑誌は毎週火曜日発刊される。〔正式{せいしき}に〕発表{はっぴょう}[公表{こうひょう}・公開{こうかい}]するレベル2、発音
\\	、カナパブリッシ、パブリッシュ、変化《動》
\\	、分節
\\	変化形 
\\	【他動】~を尊敬{そんけい}する、~に敬意{けいい}を表する~に配慮{はいりょ}する、~を気遣う{きづかう}〔法令{ほうれい}などを〕順守{じゅんしゅ}する、違反{いはん}しない【名】〔~との〕関連{かんれん}、関係{かんけい}◆【用法】
\\	〔~への〕関心{かんしん}、注意{ちゅうい}◆【用法】
\\	〔~への〕尊敬{そんけい}、敬意{けいい}◆【用法】
\\	〔~で〕尊敬{そんけい}されていること[状態{じょうたい}]◆【用法】
\\	尊敬{そんけい}の表現{ひょうげん}[言葉{ことば}・態度{たいど}]◆【参考】
\\	〔特定{とくてい}の〕点、事項{じこう}レベル2、発音
\\	、カナリスペクト、変化《動》
\\	、分節
\\	【間投】〈話〉大変結構です◆賛成・承認・満足などを表す応答として用いられる。【形】極めて[この上なく]良い、優秀{ゆうしゅう}な、優れた、卓越{たくえつ}した、絶品{ぜっぴん}の◆最高の褒め言葉◆【類】 
\\	〔成績{せいせき}・評定{ひょうじょう}などが〕優で◆優秀を意味する最も一般的な評語。◆【参考】
\\	レベル2、発音
\\	、カナエクセレント、分節
\\	このプロジェクトは完了していますが、やるべきことはまだたくさん残っています。 ・
\\	そのプロジェクトチームは、新製品の開発を成功裏に完了しました。 ・
\\	彼は、今月末までにすべてを終わらせる予定です。 ・
\\	診察前にこれらの書類にご記入ください。 ・
\\	これら2枚のアルバムで、私の大好きな
\\	のすべてがやっとそろいました。 ・
\\	もうすぐ開かれる毎年恒例の夏祭りのための準備はすべて完了しています。 ・
\\	探し物って、探しているときには見つからなくて、忘れた頃に見つかる[出てくる]。 ・
\\	このランプの点滅が停止したら、印刷処理は完了です。◆プリンタの取扱説明書 ・
\\	そのソフトウェアのインストールが完了すると、コンピューターを再起動するように指示が出ます。 ・
\\	彼女みたいな人は他にはいません。◆ほめる ・
\\	彼女は、二人の子どもの全面的な親権を要求しています。	変化形 
\\	【他動】~を完了{かんりょう}する、仕上げる、終える、完結{かんけつ}する、完成{かんせい}する、達成{たっせい}する、修了{しゅうりょう}する ・
\\	このプロジェクトは完了していますが、やるべきことはまだたくさん残っています。 ・
\\	そのプロジェクトチームは、新製品の開発を成功裏に完了しました。 ・
\\	彼は、今月末までにすべてを終わらせる予定です。~のすべての項目{こうもく}に記入{きにゅう}する ・
\\	診察前にこれらの書類にご記入ください。~を全部そろえる、~の必要{ひつよう}をすべて満たす、数や量を満たす、契約{けいやく}を履行{りこう}する ・
\\	これら2枚のアルバムで、私の大好きな
\\	のすべてがやっとそろいました。《アメフト》フォワードパスに成功{せいこう}する《野球》完投する【形】〔必要なすべての構成要素を含んでおり〕全部そろった、完全{かんぜん}な、全部{ぜんぶ}の、~一式{いっしき}、~を完備{かんび}した◆【反】
\\	〔仕事などが〕完結{かんけつ}した、完成{かんせい}した ・
\\	もうすぐ開かれる毎年恒例の夏祭りのための準備はすべて完了しています。 ・
\\	探し物って、探しているときには見つからなくて、忘れた頃に見つかる[出てくる]。 ・
\\	このランプの点滅が停止したら、印刷処理は完了です。◆プリンタの取扱説明書 ・
\\	そのソフトウェアのインストールが完了すると、コンピューターを再起動するように指示が出ます。〈強意語〉完全{かんぜん}な、全くの ・
\\	彼女みたいな人は他にはいません。◆ほめる徹底的{てっていてき}な、全面的{ぜんめんてき}な ・
\\	彼女は、二人の子どもの全面的な親権を要求しています。《植物》雌雄同花{しゆう どうか}の《数学》(公理系{こうりけい}が)完全{かんぜん}な◆その公理系において記述可能ないかなる命題も、その公理系内で証明または反証されることレベル2、発音
\\	、カナコンプリート、変化《形》
\\	、《動》
\\	、分節
\\	変化形 
\\	《複》
\\	【名】理想{りそう}、究極{きゅうきょく}の目標{もくひょう}極致{きょくち}、最高{さいこう}の完成度{かんせいど}架空{かくう}[想像上{そうぞう じょう}]のもの〔理想的{りそう てき}な〕人、もの、見本{みほん}《数学》〔環論{かんろん}の〕イデアル、イデヤル【形】〔ある種類{しゅるい}の中で〕理想的{りそう てき}な、最良{さいりょう}の申し分のない、極めて満足{まんぞく}できる空想{くうそう}の、想像上{そうぞう じょう}の非現実的{ひ げんじつ てき}な、実用性{じつよう せい}に欠ける《哲学》〔プラトン哲学{てつがく}の〕イデアの《哲学》観念論{かんねんろん}のレベル3、発音
\\	、カナアイデアル、アイディーアル、変化《複》
\\	、分節
\\	来週は期末(試験)だ。 ・
\\	そっち今週、期末試験なんだ? うちは来週だよ。 ・
\\	授業は全部終わった。あとは期末試験だけだ。 ・
\\	つい先日、正式に離婚しました。	変化形 
\\	《複》
\\	【名】最後{さいご}のもの決勝戦{けっしょうせん}最終試験{さいしゅう しけん}、期末試験{きまつ しけん} ・
\\	来週は期末(試験)だ。 ・
\\	そっち今週、期末試験なんだ? うちは来週だよ。 ・
\\	授業は全部終わった。あとは期末試験だけだ。【形】最後{さいご}の、最終{さいしゅう}の、決定的{けっていてき}な、確定的{かくていてき}な ・
\\	つい先日、正式に離婚しました。レベル2、発音
\\	、カナファイナル、変化《複》
\\	、分節
\\	人から必要とされる人間になるべし。 ・
\\	いいえ、結構です。	イディオムやフレーズ 
\\	【名】必需品{ひつじゅひん}、必要不可欠{ひつよう ふかけつ}なもの【形】必要{ひつよう}な、必須{ひっす}の、欠くことのできない、なくてはならない ・
\\	人から必要とされる人間になるべし。 ・
\\	いいえ、結構です。当然{とうぜん}のこととして起きる、必然的{ひつぜん てき}な〔権威{けんい}や慣習{かんしゅう}により〕強制{きょうせい}される、義務的{ぎむ てき}な《論理学》〔命題{めいだい}が真であることが〕必然的{ひつぜん てき}な《論理学》〔条件{じょうけん}が〕必要{ひつよう}なレベル1、発音
\\	、カナネサセリ、ネセサリー、分節
\\	単刀直入に要点だけを教えてください。 ・
\\	私のような職業に携わる者にとってネクタイは必需品です。 ・
\\	出発する前に、必須の安全器具を持っていることを確認しなさい。	変化形 
\\	《複》
\\	【名】本質的要素{ほんしつてき ようそ}、要点{ようてん} ・
\\	単刀直入に要点だけを教えてください。(生活{せいかつ})必需品{ひつじゅひん}、必要{ひつよう}なもの、不可欠{ふかけつ}なもの【形】絶対必要{ぜったい ひつよう}な、絶対不可欠{ぜったい ふかけつ}な、欠くことのできない、必須{ひっす}の、最も重要{じゅうよう}な、肝心{かんじん}な ・
\\	私のような職業に携わる者にとってネクタイは必需品です。 ・
\\	出発する前に、必須の安全器具を持っていることを確認しなさい。本質{ほんしつ}の、本質的{ほんしつてき}な、根本的{こんぽんてき}な、基幹的{きかん てき}な《植物》エキスの、エキスを含むレベル3、発音
\\	、カナイセンシャル、エセンシャル、エッセンシャル、変化《複》
\\	、分節
\\	肺がんの分野における彼の研究成果は意義深かった。		【形】重要{じゅうよう}な、意義深い{いぎぶかい}、大きな影響{えいきょう}を与える ・
\\	肺がんの分野における彼の研究成果は意義深かった。意味{いみ}のある、意味{いみ}を表す意味ありげな、暗示的{あんじてき}な大幅{おおはば}な、かなりの、著しい《統計》有意{ゆうい}なレベル4、発音
\\	、カナシグニフィカント、スィグニフィカント、分節
\\	「数字164で6の相当する数は?」「60です」 ・
\\	時間をかけていくつかの店を見て回っただけの価値はあった。
\\	イディオムやフレーズ 
\\	【もっとイディオムを見る】
\\	"【名】価値{かち}《数学》相当する数[量]◆数字の各けたの持つ意味を表す時に用いられる。 ・
\\	「数字164で6の相当する数は?」「60です」財産{ざいさん}~の量【形】~の価値{かち}[値打ち・かい]がある、~に値する、~相当{そうとう}の、~相当量{そうとうりょう}の ・
\\	時間をかけていくつかの店を見て回っただけの価値はあった。~の財産{ざいさん}があるレベル2、発音
\\	、カナワース
\\	夫がこんなに弱って見えたことはありません。	変化形 
\\	【名】
\\	弱者{じゃくしゃ}【形】〔人の心身{しんしん}が〕弱った、元気{げんき}のない ・
\\	夫がこんなに弱って見えたことはありません。〔人の性格{せいかく}が〕意思{いし}が弱い、意気地{いくじ}がない〔人の能力{のうりょく}が〕低い、劣る〔臓器{ぞうき}の機能{きのう}が〕弱った、衰えた〔物が〕壊れやすい、もろい〔食べ物が〕水っぽい、味が薄い〔強さの度合いなどが〕弱い、かすかな〔議論{ぎろん}などの根拠{こんきょ}が〕薄弱{はくじゃく}な、説得力{せっとくりょく}がない〔政府{せいふ}などの統治能力{とうち のうりょく}が〕弱体{じゃくたい}な、無力{むりょく}な〔詩の強勢{きょうせい}が〕弱の、無強勢{む きょうせい}の《言語学》〔動詞{どうし}の変化{へんか}が〕弱の、規則的{きそく てき}な〔価格{かかく}や市場{しじょう}が〕弱含み{よわふくみ}の、下がっているレベル1、発音
\\	、カナウィーク、変化《形》
\\	申し立てによれば警察は彼を手荒に扱ったとされる。 ・
\\	紙ヤスリで表面を荒仕上げした。 ・
\\	強風で海が荒れた。 ・
\\	着陸は少々荒っぽかった。 ・
\\	大まかな意見を言ってください。	変化形 
\\	【他動】~を手荒く扱う ・
\\	申し立てによれば警察は彼を手荒に扱ったとされる。〔ホッケーなどで相手{あいて}に〕ラフプレーをする~に荒仕上げをする ・
\\	紙ヤスリで表面を荒仕上げした。~をザラザラにする、~を乱す[荒らす] ・
\\	強風で海が荒れた。【名】でこぼこの土地{とち}《ゴルフ》ラフスケッチ、下書き{したがき}〈英〉乱暴者{らんぼう もの}未完成品{みかんせい ひん}、雑に仕上げた物【形】〔土地{とち}などが〕でこぼこの、起伏{きふく}のある柔らかく[滑らかで]ない、ザラザラした毛深い{けぶかい}、毛むくじゃらの〔風が〕強い、〔天候{てんこう}が〕荒れた〔行く手が〕原野{げんや}のままの、人を阻む〔取り扱いなどが〕手荒{てあら}な、乱暴{らんぼう}な ・
\\	着陸は少々荒っぽかった。洗練{せんれん}されていない、不作法{ぶさほう}な、粗野{そや}な〔音が〕耳障{みみざわ}りな、〔匂{にお}いなどが〕きつい〔アイデア・意見・構想・推定などが〕おおよその、大まかな、概略{がいりゃく}の ・
\\	大まかな意見を言ってください。〔設備{せつび}などが〕粗末{そまつ}な、何とか使える程度{ていど}の〔仕上げなどが〕雑な、大ざっぱな、未完成{みかんせい}のつらい、苦しい、ひどい〔人が〕騒々{そうぞう}しい、乱暴{らんぼう}な〔地域{ちいき}が〕治安{ちあん}の悪い、不穏{ふおん}な【副】手荒くレベル2、発音
\\	、カナラフ、変化《形》
\\	、《動》
\\	変化形 
\\	《複》
\\	【名】苦しみ、災難{さいなん}、不運{ふうん}、不幸{ふこう}罪悪{ざいあく}、悪、害悪{がいあく}、~悪【形】病気{びょうき}で、不健全{ふけんぜん}な、不健康{ふけんこう}な、気分{きぶん}が悪い邪悪{じゃあく}な、不吉{ふきつ}な、不幸{ふこう}な〈俗〉とても[すごく]良い、素晴{すば}らしい、格好いい、いかした、粋な◆特にラップ音楽について。この意味では比較級・最上級として非標準の
\\	が用いられることがある。【副】悪く、まずく不適切{ふてきせつ}に不親切{ふしんせつ}にレベル1、発音
\\	、カナイル、変化《形》
\\	、《複》
\\	変化形 
\\	《複》
\\	【名】《動物》雄〔人間{にんげん}の〕男性{だんせい}、男子{だんし}《植物》雄株{おかぶ}、雄花{おばな}【形】《動物》雄の〔人間{にんげん}の〕男性{だんせい}の、男子{だんし}の《植物》〔花や植物{しょくぶつ}が〕雄の《機械》雄型{おがた}の男らしい、男性的{だんせい てき}なレベル2、発音
\\	、カナメイル、マレ、変化《複》
\\	変化形 
\\	《複》
\\	【名】〈軽蔑的〉女《動物》雌《植物》雌株{めかぶ}【形】女性{じょせい}の、女性{じょせい}らしい《生物》〔子どもが作れる性の〕雌の《植物》雌しべの、子房{しぼう}の《植物》〔花が〕雌(性)の◆雄しべを持たず、雌しべだけを持つもの。《機械》〔接合部{せつごうぶ}が〕雌(型)のレベル2、発音
\\	、カナフィーメイル、変化《複》
\\	、分節
\\	1ドル未満分(99セント以下)は四捨五入されます。 ・
\\	《数学》5814を千の位に四捨五入すれば何になりますか? ・
\\	ここの気候は温暖で冬のない年を送ることができる。◆明確な始めと終わりがあり、その期間内に出来事が決められた順序で継起するようなものを指す。	変化形 
\\	【自動】丸くなる【他動】~を丸くする、丸める〔数値を〕四捨五入{ししゃ ごにゅう}する、概数{がいすう}で表す◆【参考】
\\	1ドル未満分(99セント以下)は四捨五入されます。 ・
\\	《数学》5814を千の位に四捨五入すれば何になりますか?〔角などを〕曲がる、回る◆【参考】
\\	【名】円形{えんけい}、輪、丸い物〔円形物の〕周囲、周回、〔円形になった〕範囲〔繰り返し出現する〕期間 ・
\\	ここの気候は温暖で冬のない年を送ることができる。◆明確な始めと終わりがあり、その期間内に出来事が決められた順序で継起するようなものを指す。〔スポーツなどの〕ラウンド、一試合〔定められコースの〕巡回、巡視、歴訪◆【参考】
\\	往診{おうしん}、回診{かいしん}一斉射撃、ひとわたりの拍手、1回の給仕〔はしごなどの〕横木◆接合部が円形になっている物= 
\\	〔銃・武器{ぶき}などの〕一発分{いっぱつ ぶん}の弾【形】丸い、円形{えんけい}の一回りする大体{だいたい}の円熟{えんじゅく}した活発{かっぱつ}な多数{たすう}の【前】~を取り巻いて、回って、回転{かいてん}して【副】~の間中レベル1、発音
\\	、カナラウンド、変化《形》
\\	、《動》
\\	円の面積は、円周率×半径二乗です。 ・
\\	たばこ1本、もらえる? ・
\\	そんなに堅苦しくしないで。/そんなふうに古くさい考えで[社交性が乏しくて]は駄目ですよ。 ・
\\	つまらない人間になりたくなかったら、必ず来ること。◆この文章を紙などに書く場合、
\\	という単語の代わりに四角記号が描かれることがある。	変化形 
\\	【自動】《数学》直角{ちょっかく}である、直交{ちょっこう}する一致{いっち}する、同意{どうい}する【他動】~を正方形{せいほうけい}[長方形{ちょうほうけい}]にする[作る]〔紙などを〕正方形{せいほうけい}[長方形{ちょうほうけい}]に切り分ける《数学》〔数を〕2乗する ・
\\	円の面積は、円周率×半径二乗です。〔物や体の一部{いちぶ}を〕真っすぐ[水平{すいへい}]にする~を直角{ちょっかく}にする[置く]、~が直角{ちょっかく}かどうかを調べる〔借金{しゃっきん}などを〕清算{せいさん}する、けりをつける《スポーツ》〔試合{しあい}を〕同点{どうてん}にする〈話〉(人)を買収{ばいしゅう}する、(人)に賄賂{わいろ}を贈る〔人や考えなどに〕一致{いっち}する、同意{どうい}する〔自分自身{じぶん じしん}を〕(人)と折り合いを良くさせる【名】正方形{せいほうけい}、四角{しかく}四角い(形の)物〔チェスなどの〕ゲーム盤〔四角い〕広場{ひろば}、公園{こうえん}〔四角い〕街区{がいく}直角定規{ちょっかく じょうぎ}、差し金、
\\	定規{じょうぎ}《数学》平方{へいほう}、二乗{にじょう}◆1スクエアメーター 
\\	一辺が1メートルの正方形の面積◆【略】
\\	〈俗〉四角張った人、堅苦{かたぐる}しい人、真面目人間{まじめ にんげん}、古くさい人《軍事》〔兵舎内{へいしゃ ない}の〕練兵場{れんぺいじょう}、演習場{えんしゅう じょう}《軍事》〔隊形{たいけい}の〕方陣{ほうじん}〈話〉しっかりした[十分{じゅうぶん}な]食事{しょくじ}◆【同】
\\	〈英〉〔電話{でんわ}などの〕ナンバー記号{きごう}(のキー)◆【同】〈米〉
\\	〈英〉
\\	〈俗〉たばこ◆【語源】恐らく、たばこの箱の形が四角いことから。ただし、
\\	は「1箱」ではなく「1本」のたばこを指す。 ・
\\	たばこ1本、もらえる?【形】《数学》正方形{せいほうけい}の《数学》直角{ちょっかく}の、直交{ちょっこう}する《数学》2乗の〔面積{めんせき}が〕平方{へいほう}の〔長さが〕~四方{しほう}の四角い{しかくい}、正方形{せいほうけい}に近いブロック状の、がっしりとした真っすぐな、水平{すいへい}な、きちっとそろった正直{しょうじき}な、率直{そっちょく}な、はっきりした公平{こうへい}な、公正{こうせい}な、ごまかしのない〔金の〕貸し借りがない、精算{せいさん}が済んだ《スポーツ》互角{ごかく}の、同点{どうてん}の〈俗〉〔人などが〕新しい文化に疎い、頭が古い、堅物の、ださい、社交性に乏しい◆【語源】もともとはジャズの世界の俗語。1940年代頃から使われ始めた。一説に、
\\	(正方形)は「機械的な4拍子=ジャズとしての面白みがない(スイングしていない)状態」を表した。◆【対】
\\	そんなに堅苦しくしないで。/そんなふうに古くさい考えで[社交性が乏しくて]は駄目ですよ。 ・
\\	つまらない人間になりたくなかったら、必ず来ること。◆この文章を紙などに書く場合、
\\	という単語の代わりに四角記号が描かれることがある。【副】《数学》直角{ちょっかく}に四角形{しかくけい}に、正方形{せいほうけい}に〈話〉しっかりと、断固{だんこ}として〈話〉まともに、思い切り〈話〉正直{しょうじき}に、直截{ちょくせつ}にレベル1、発音
\\	、カナスクウェア、スクエア、変化《動》
\\	まず定義をはっきりさせましょう。 ・
\\	はっきりおっしゃってください。/私に隠し事はしないでよ。 ・
\\	率直に言わせてください。/はっきり言いましょう。 ・
\\	寄り道せずに真っすぐ帰ってきなさいよ。◆親が子どもに言うときなど ・
\\	3日続けて雨が降った。 ・
\\	私は紅茶をストレートで飲む[には何も入れない主義だ]。 ・
\\	完全にあなたが間違っている。 ・
\\	はっきり言ってください。	変化形 
\\	【名】〔直線{ちょくせん}などの〕真っすぐなもの[形]両切り[フィルターのない]たばこ〈英〉〔道路{どうろ}・レース場などの〕直線区間{ちょくせん くかん}◆【同】〈米〉
\\	《ポーカー》ストレート〈俗〉非同性愛者{ひ どうせいあい しゃ}〈俗〉麻薬{まやく}をやっていない人〈俗〉伝統{でんとう}に縛られる人【形】〔曲がらないで〕真っすぐな、一直線{いっちょくせん}の連続{れんぞく}した、間を置かない〔思考などが〕正しい、きちんとした、はっきりした ・
\\	まず定義をはっきりさせましょう。〔情報などが〕信頼{しんらい}できる包みかくしのない、率直{そっちょく}な、誠実{せいじつ}な ・
\\	はっきりおっしゃってください。/私に隠し事はしないでよ。 ・
\\	率直に言わせてください。/はっきり言いましょう。値引きをしない、貸借{たいしゃく}のない〈米〉〔酒が〕ストレートの、水やソーダで割ってない〔位置{いち}が〕水平{すいへい}な、垂直{すいちょく}な正直{しょうじき}な、正当{せいとう}な、公正{こうせい}な片付いた、整頓{せいとん}された、整然{せいぜん}とした《トランプ》〔ポーカーで〕ストレートの〔演劇{えんげき}などが〕ストレートの◆喜劇でなく、また音楽劇でもないもの。〈俗〉同性愛{どうせいあい}でない、異性愛{いせい あい}の〈俗〉酒[薬物{やくぶつ}]を飲んでいない〈俗〉〔外見{がいけん}や生き方が〕保守的{ほしゅ てき}な、昔ながらの【副】〔曲がらないで〕一直線{いっちょくせん}に、真っすぐに〔寄り道をしないで〕真っすぐに、急いで ・
\\	寄り道せずに真っすぐ帰ってきなさいよ。◆親が子どもに言うときなど〔位置{いち}が〕直立{ちょくりつ}して、水平{すいへい}に、垂直{すいちょく}に、真っすぐに連続{れんぞく}して、立て続けに、続けざまに、休まずに ・
\\	3日続けて雨が降った。薄めないで、ストレートで、氷を入れないで、水割りにしないで ・
\\	私は紅茶をストレートで飲む[には何も入れない主義だ]。〈俗〉全く、完全{かんぜん}に ・
\\	完全にあなたが間違っている。率直{そっちょく}に、端的{たんてき}に ・
\\	はっきり言ってください。整頓{せいとん}して、片付いて伝統{でんとう}[法律{ほうりつ}]に従って、真面目{まじめ}にレベル1、発音
\\	、カナストゥレイト、ストレイト、変化《形》
\\	変化形 
\\	《複》
\\	【名】《クリケット》〔投手の〕暴投(によって打者が得る1点)【形】~の幅がある、~幅の◆【用法】
\\	のように後置して用いる。〔物の〕幅が広い、幅広{はばびろ}の大きく開いた、いっぱいに開けた〔選択{せんたく}の〕範囲{はんい}が広い、種類{しゅるい}が多い〔活動{かつどう}などが〕多数{たすう}の人が関与{かんよ}する〔視野{しや}が〕広い、〔物の見方{みかた}が〕広範{こうはん}な〔狙ったところから〕外れた、離れた〔衣類{いるい}などが〕ゆったりした、だぼだぼの《野球》〔ボールが〕外角{がいかく}の◆【同】
\\	《言語学》弛緩音{しかんおん}の◆【同】
\\	【副】十分{じゅうぶん}に、すっかり広い範囲{はんい}に、至る所に〔狙ったところから〕外れて、それてレベル1、発音
\\	、カナワイド、変化《形》
\\	経常収支の赤字が急速に減少している。 ・
\\	日本政府は5倍あった税率格差を現行の3倍に縮小しました。 ・
\\	正味の金利利鞘は第2四半期にマネー・センター銀行で2.1%に縮小した。 ・
\\	今回の試験は出題範囲が狭い。	変化形 
\\	【自動】狭くなる、狭まる◆【反】
\\	低下{ていか}する、減少{げんしょう}する ・
\\	経常収支の赤字が急速に減少している。【他動】~を狭くする、狭める~を削減{さくげん}する、縮小{しゅくしょう}する、制限{せいげん}する ・
\\	日本政府は5倍あった税率格差を現行の3倍に縮小しました。 ・
\\	正味の金利利鞘は第2四半期にマネー・センター銀行で2.1%に縮小した。【名】狭い部分{ぶぶん}[場所{ばしょ}・道路{どうろ}]海峡{かいきょう}【形】〔幅・面積などが〕狭い〔範囲などが〕限られた ・
\\	今回の試験は出題範囲が狭い。心の狭い〔利鞘などが〕薄い、小さい、少ない〔勝利などが〕紙一重{かみひとえ}の、ぎりぎりの、すれすれの〔危険回避などが〕間一髪{かんいっぱつ}の〔調査などが〕綿密{めんみつ}なレベル1、発音
\\	、カナナロー、ナロウ、変化《形》
\\	、《動》
\\	、分節
\\	この肉は脂身が多い。 ・
\\	私、でぶになりたくない!でぶは格好悪いもん!	変化形 
\\	《複》
\\	【名】脂肪{しぼう}、脂、脂質{ししつ}、油脂{ゆし}、料理用{りょうり よう}の油 ・
\\	この肉は脂身が多い。ぜい肉
\\	太った人、太らせた家畜{かちく}
\\	〈話〉最上{さいじょう}のもの[部分{ぶぶん}]【形】〔侮蔑的〕でぶの ・
\\	私、でぶになりたくない!でぶは格好悪いもん!油っこい多額{たがく}のもうかるレベル1、発音
\\	、カナファット、変化《形》
\\	、《複》
\\	ナイフで切れるくらいに濃い霧が立ち込めていた。 ・
\\	ピリピリ[非常に緊迫]した雰囲気だった。/空気がピンと張り詰めていた。 ・
\\	彼はアイルランドなまりが強かったので、彼の言うことが理解できないことが多かった。	変化形 
\\	【名】最も活発{かっぱつ}な[激しい・密集{みっしゅう}した]部分{ぶぶん}最も激しい所、密集{みっしゅう}した部分{ぶぶん}、にぎやかな所、最中{さいちゅう/さなか}、たけなわ【形】〔幅が〕厚い、広い〔測った幅が〕~の厚さの、~幅の〔直径{ちょっけい}が〕大きい、太い、大口径{だい こうけい}の〔体や体格{たいかく}が〕がっしりした、ずんぐりした〔ある物を構成{こうせい}する部分{ぶぶん}が〕密集{みっしゅう}した、びっしりと配置{はいち}された〔人や物が〕いっぱいの、うようよしている〔気体{きたい}や液体{えきたい}が〕視界{しかい}が悪い、見通せない、どんよりした、濃い ・
\\	ナイフで切れるくらいに濃い霧が立ち込めていた。〔液体{えきたい}が〕粘度{ねんど}が高い、ドロッと[どろりと・どろどろ]した、粘度がある〔心配{しんぱい}や不安{ふあん}が〕高まった、激しい ・
\\	ピリピリ[非常に緊迫]した雰囲気だった。/空気がピンと張り詰めていた。〔言葉{ことば}が〕聞き取れない、はっきりしない、不明瞭{ふめいりょう}な〔なまりが〕強い、ひどい ・
\\	彼はアイルランドなまりが強かったので、彼の言うことが理解できないことが多かった。〔生地{きじ}が〕厚手{あつで}の、厚地{あつじ}の〈話・軽蔑的〉〔頭が〕鈍い、悪い〈話〉関係{かんけい}が濃い、親密{しんみつ}な〈話〉我慢{がまん}がならないほどの、度を超えた【副】厚く、太く、重く密集{みっしゅう}して、ぎっしりとレベル1、発音
\\	、カナシック、スィック、変化《形》
\\	苗が2
\\	ほどになったら10
\\	間隔に間引きます。 ・
\\	彼女の唇は薄い。 ・
\\	ボブ、あなた痩せて見えるけど、ちゃんと食事してる? ・
\\	痩せていることは大切なの? ・
\\	痩せなさい。 ・
\\	彼は髪が薄い。 ・
\\	きょうは懐{ふところ}が寒い。 ・
\\	山は空気が薄い。 ・
\\	政府が発表した口実は見え透いていて説得力のないものだった。 ・
\\	彼らの話はどちらも内容がなかった。	変化形 
\\	【自動】弱くなる【他動】~を薄くする、細くする~をまばらにする、間引く{まびく} ・
\\	苗が2
\\	ほどになったら10
\\	間隔に間引きます。【名】細い部分{ぶぶん}、薄い部分{ぶぶん}【形】〔板などの厚みが平均的厚みよりも〕薄い、厚くない、太くない、〔ロープやひもが〕細い ・
\\	彼女の唇は薄い。痩{や}せた、太っていない ・
\\	ボブ、あなた痩せて見えるけど、ちゃんと食事してる? ・
\\	痩せていることは大切なの? ・
\\	痩せなさい。〔必要{ひつよう}なものが〕十分{じゅうぶん}にない、乏しいまばらな、希薄{きはく}な〔髪が〕薄い ・
\\	彼は髪が薄い。懐が寒い、一文{いちもん}なしの、金欠{きんけつ}の ・
\\	きょうは懐{ふところ}が寒い。〔アルコール度が〕弱い、水っぽい〔液体{えきたい}・気体{きたい}などの濃さが〕薄い ・
\\	山は空気が薄い。〔動作{どうさ}などが〕弱々{よわよわ}しい〔声が〕かぼそい、力のない熱意{ねつい}に乏しい、気乗りしない〔言い訳が〕見え透いた ・
\\	政府が発表した口実は見え透いていて説得力のないものだった。〔話などが〕内容{ないよう}のない ・
\\	彼らの話はどちらも内容がなかった。〔定義{ていぎ}が〕不明確{ふめいかく}な〔論拠{ろんきょ}が〕弱いレベル1、発音
\\	、カナシン、スィン、変化《形》
\\	、《動》
\\	彼の会社での急速な出世は、スキャンダルが原因で終わった。	変化形 
\\	《複》
\\	【名】〔川の〕早瀬{はやせ}、急流{きゅうりゅう}◆通例、
\\	【形】迅速{じんそく}な、急速{きゅうそく}な、敏しょうな、素早い、早急{そうきゅう}の、とんとん拍子{ひょうし}の、短時間{たんじかん}の ・
\\	彼の会社での急速な出世は、スキャンダルが原因で終わった。〔電車が〕快速の坂道{さかみち}が急な、険しい、急激{きゅうげき}な《写真》高感度{こう かんど}の安易{あんい}なレベル2、発音
\\	、カナラピッド、変化《複》
\\	、分節
\\	迅速な決定をありがとうございます。 ・
\\	足の速い茶色のキツネがぐうたら犬を跳び越える。◆このセンテンスは、アルファベット26文字すべてが含まれているので、タイプライターの印字テストに使われた。 ・
\\	お母さん、早く来て。 ・
\\	急に寒くなりました。	変化形 
\\	【名】生身{なまみ}、生きている人々{ひとびと}感情{かんじょう}の中枢{ちゅうすう}、急所{きゅうしょ}【形】迅速{じんそく}な、速い、即座{そくざ}の、素早い ・
\\	迅速な決定をありがとうございます。 ・
\\	足の速い茶色のキツネがぐうたら犬を跳び越える。◆このセンテンスは、アルファベット26文字すべてが含まれているので、タイプライターの印字テストに使われた。頭の回転{かいてん}が速い、理解{りかい}が速い、利口{りこう}な〔感覚{かんかく}が〕鋭い、敏感{びんかん}な短気{たんき}な、せっかちな、怒りっぽい〔資産{しさん}などが〕すぐ現金化{げんきん か}できる、すぐ換金{かんきん}できる〈俗〉〔衣服{いふく}が〕窮屈{きゅうくつ}な、きつい〈古〉妊娠{にんしん}して【副】〈話〉速く、素早く、迅速{じんそく}に◆【同】
\\	お母さん、早く来て。 ・
\\	急に寒くなりました。〈俗〉〔衣服{いふく}が〕窮屈{きゅうくつ}で、きつくてレベル1、発音
\\	、カナクイック、クウィック、変化《形》
\\	景気が急減速している。 ・
\\	事業があまり振るわない。/客足はぼちぼちだ。 ・
\\	ゆっくりやって見せます。 ・
\\	ダウンロードが遅い。/なかなか落ちてこない。◆サーバからファイルをダウンロードするときなど。	変化形 
\\	【自動】遅くなる、減速{げんそく}する ・
\\	景気が急減速している。【他動】~を遅くする、~の速度{そくど}を落とす【形】〔動きが〕遅い、のろい《音楽》〔テンポが〕ゆったりした〔物事{ものごと}が〕時間{じかん}がかかる、多くの時間{じかん}が必要{ひつよう}な〔予定{よてい}よりも〕遅れている、進みが遅い〔時計{とけい}が〕遅れている〔取引{とりひき}や売り上げが〕沈滞気味{ちんたい ぎみ}の、不振{ふしん}な ・
\\	事業があまり振るわない。/客足はぼちぼちだ。〔反応{はんのう}などが〕乗り気でない、いやそうな〈話・軽蔑的〉〔人が〕鈍い、頭が悪い面白くない、つまらない〔オーブンなどが〕火力{かりょく}の弱い《ゴルフ》(グリーンが)遅い〔レンズの口径比{こうけいひ}が大きく〕暗い【副】遅く、ゆっくり ・
\\	ゆっくりやって見せます。 ・
\\	ダウンロードが遅い。/なかなか落ちてこない。◆サーバからファイルをダウンロードするときなど。レベル1、発音
\\	、カナスロー、スロウ、変化《形》
\\	、《動》
\\	車は横2列、縦3列に駐車しました。	変化形 
\\	《複》
\\	【名】深み、深遠{しんえん}、海淵{かいえん}【形】深い、深さ[奥行き]がある◆【略】
\\	~列の、~列に並んだ ・
\\	車は横2列、縦3列に駐車しました。深い洞察力{どうさつりょく}がある意味{いみ}の深い、深甚{しんじん}な、深遠{しんえん}な〔感情が〕根深い◆【同】
\\	難解{なんかい}な、困難{こんなん}な、難しい、手が付けられない、理解{りかい}に苦しむ、不可解{ふかかい}な強度{きょうど}の、極端{きょくたん}な、凄まじい、激しい、切実{せつじつ}な、重大{じゅうだい}な、深刻{しんこく}な〔色彩{しきさい}が〕濃い〔声や音が〕太い、低い大幅{おおはば}な、規模{きぼ}の大きい遠い昔の、遠く離れたずる賢い、腹黒い{はらぐろい}、抜け目のない【副】深く、内奥{ないおう}に、突き詰めて〔時間的{じかん てき}に〕後の方で、遅くまで《スポーツ》〔フィールドや陣地{じんち}の〕遠くに、深い位置{いち}にレベル1、発音
\\	、カナデープ、ディープ、変化《形》
\\	、《複》
\\	一つの扉が閉じるときもう片方の扉が開く。 ・
\\	この学校は今年で廃校になります。 ・
\\	私たちは以前は親しい間柄でした。 ・
\\	当たらずといえども遠からずだ。 ・
\\	そんなに近寄らないでください。	変化形 
\\	【1自動】〔戸や窓などが〕閉じる、閉まる ・
\\	一つの扉が閉じるときもう片方の扉が開く。〔店や会社{かいしゃ}などが〕営業{えいぎょう}を終了{しゅうりょう}する、閉店{へいてん}する、廃業{はいぎょう}する ・
\\	この学校は今年で廃校になります。〔話などが〕終わる、締めくくる合意{ごうい}に達する、折り合いを付ける、契約{けいやく}が成立{せいりつ}する〔競技{きょうぎ}などで相手{あいて}が〕接近{せっきん}する、攻め寄せる〔株式市場{かぶしき しじょう}が〕大引けになる【1他動】〔戸や窓などを〕閉じる、閉める◆開いている所を覆うために、物を動かすこと。◆【参考】
\\	〔本や目などを〕閉じる、畳む◆物を動かして、見えていた物を見えなくすること。〔店や会社{かいしゃ}などを〕閉める、畳む◆営業時間の終わりに閉店したり、完全に廃業したりすることも含む。◆【参考】
\\	〔道路{どうろ}などを〕閉鎖{へいさ}する、封鎖{ふうさ}する◆【同】
\\	〔話や会議{かいぎ}などを〕終わらせる、締めくくる〔取引{とりひき}などを〕成立{せいりつ}させる、完了{かんりょう}させる〔競技{きょうぎ}などで相手{あいて}に〕接近{せっきん}する〔二者{にしゃ}の距離{きょり}を〕縮める《コ》〔ファイルなどを〕閉じる《電気》〔回路{かいろ}を〕閉じる【1名】閉める[閉じる・終わる]こと
\\	の発音は
\\	〔行為{こうい}や話などの〕結末{けつまつ}、終了{しゅうりょう}
\\	の発音は
\\	《音楽》終止形{しゅうしけい}◆【同】
\\	の発音は
\\	〔大寺院{だい じいん}などの〕中庭{なかにわ}
\\	の発音は
\\	〈英〉路地{ろじ}、狭い通り
\\	の発音は
\\	【1形】〔しっかりと〕閉じ込めた、包囲{ほうい}したレベル1、発音
\\	、カナクローズ、クロウズ、変化《形》
\\	、《動》
\\	【2形】〔空間・時間的に〕近い、接近{せっきん}した、近接{きんせつ}した〔人や物の関係{かんけい}が〕近縁{きんえん}の、近親{きんしん}の〔人同士{じん どうし}が〕親しい、親密{しんみつ}な、密接{みっせつ}な ・
\\	私たちは以前は親しい間柄でした。
\\	以下{いか}のことが〕起きそうな、寸前{すんぜん}の◆【用法】
\\	〔調査{ちょうさ}などが〕徹底的{てってい てき}な、綿密{めんみつ}な〔元のものに〕そっくりな、忠実{ちゅうじつ}な〔頭髪{とうはつ}などが〕刈り込んだ、短く切った〔場所{ばしょ}が〕狭苦{せまくる}しい、混雑{こんざつ}した〔勝敗{しょうはい}などが〕僅差{きんさ}の、際どい〔問題に対する答えなどが〕正解{せいかい}に近い、惜しい、いい線行っている ・
\\	当たらずといえども遠からずだ。〔室内などの空気が〕こもっている、ムッとする、人いきれがする〔織物{おりもの}の〕目が詰んでいる
\\	以下の数量{すうりょう}に〕近い、ほとんど同じ〔秘密{ひみつ}などを〕隠した、人目{ひとめ}に出さない〔必要{ひつよう}なものが〕手に入らない、乏しい〈話〉〔人が〕けちな、しみったれた《言語学》〔母音{ぼいん}が〕狭の
\\	などのように、舌を口蓋に近づけて発声すること。【2副】〔時間{じかん}・空間的{くうかん てき}に〕近くに、そばに ・
\\	そんなに近寄らないでください。びっしりと、ぎっちりと発音
\\	、カナクロース、クロウス、変化《形》
\\	、《動》
\\	変化形 
\\	《複》
\\	【自動】〈話〉普通列車{ふつうれっしゃ}で行く[に乗る]【名】土地{とち}の人、地元{じもと}の人、その土地{とち}の人近所{きんじょ}の酒場{さかば}[バー]普通列車{ふつう れっしゃ}、各駅停車{かくえき ていしゃ}の列車{れっしゃ}地方記事{ちほう きじ}局所{きょくしょ}、支部{しぶ}、出張所{しゅっちょうじょ}〈話〉局所麻酔(剤){きょくしょ ますい(ざい)}◆【同】
\\	地元{じもと}チーム【形】地元{じもと}の、特定{とくてい}の場所{ばしょ}の、現地{げんち}の、その地域{ちいき}の、地場{じば}の(一)地方{ちほう}の局部{きょくぶ}の、局部的{きょくぶてき}な、部分的{ぶぶんてき}な、局所{きょくしょ}の、近くのその場所{ばしょ}でしか使えない[通用{つうよう}しない]各駅停車{かくえき ていしゃ}の、鈍行{どんこう}の〈英〉市内郵便{しない ゆうびん}の、同一区域内{どういつ くいき ない}の◆郵便物に表記される。ネットワークに接続{せつぞく}されていない、ローカル接続{せつぞく}の《数学》軌跡{きせき}の
\\	の形容詞形レベル2、発音
\\	、カナロウカル、ローカル、変化《複》
\\	、分節
\\	変化形 
\\	《複》
\\	【他動】~を中央{ちゅうおう}に置く《海事》〔帆を〕半分{はんぶん}に畳む【名】〔空間的{くうかん てき}な〕中心{ちゅうしん}、中央{ちゅうおう}〔時間的{じかん てき}な〕中間{ちゅうかん}、最中{さいちゅう/さなか}〔立場{たちば}や地位{ちい}の〕中間{ちゅうかん}〔物の〕内部{ないぶ}〈話〉〔人間{にんげん}の〕胴部{どうぶ}《野球》回の間◆例えば、
\\	は、7回表攻撃終了と7回裏の攻撃開始の間の時間のこと。《言語学》〔ギリシャ語などの〕中間態{ちゅうかんたい}◆再帰的な動作を表す態
\\	。《論理学》中名辞{ちゅうめいじ}◆【同】
\\	【形】〔空間的{くうかん てき}に〕真ん中の、中央{ちゅうおう}の、中心{ちゅうしん}の〔時間的{じかん てき}に〕(中)間の、最中{さいちゅう/さなか}の〔地位{ちい}や立場{たちば}が〕(中)間の〔程度{ていど}が〕中庸{ちゅうよう}の、中程度{ちゅう ていど}の、平均{へいきん}の《言語学》中間態{ちゅうかんたい}の
\\	《言語学》中期{ちゅうき}の
\\	《地学》中期{ちゅうき}のレベル1、発音
\\	、カナミドゥル、ミドル、変化《複》
\\	、分節
\\	太陽は太陽系の中心に位置する。	変化形 
\\	《複》
\\	【形】中心{ちゅうしん}の、中央{ちゅうおう}の ・
\\	太陽は太陽系の中心に位置する。主要{しゅよう}な、中心{ちゅうしん}となる、中心性{ちゅうしんせい}の、中枢性{ちゅうすうせい}の重要{じゅうよう}な支配的{しはいてき}なレベル2、発音
\\	、カナセントゥラル、セントラル、分節
\\	今夜の最低気温は20度くらいでしょう。 ・
\\	この値段で精いっぱいです。/これ以上まけられません[安くできません]。 ・
\\	あいつは動物以下です。	変化形 
\\	【1名】低い位置{いち}[程度{ていど}・レベル]
\\	最低気温{さいてい きおん} ・
\\	今夜の最低気温は20度くらいでしょう。《気象》低気圧{ていきあつ}(地域{ちいき})〔自動車{じどうしゃ}の〕ロー[低速{ていそく}]ギア【1形】〔人や物の高さが〕低い、高くない〔物の位置{いち}が〕低いところにある、下がった〔水面{すいめん}などが〕下がっている、浅い〔数や量が〕少ない、少数[量]の水平線{すいへいせん}[地平線{ちへいせん}]に近い緯度{いど}が低い、赤道{せきどう}に近い〔価格{かかく}や値段{ねだん}が〕安い、下がった ・
\\	この値段で精いっぱいです。/これ以上まけられません[安くできません]。〈話〉金がない、金欠状態{きんけつ じょうたい}の〔品質{ひんしつ}などが〕低い、悪い〔重要度{じゅうようど}や緊急性{きんきゅう せい}などが〕低い《言語学》〔母音{ぼいん}が〕低位{ていい}の〔声や音が〕低い、小さい、かすかな〔照明{しょうめい}などが〕薄暗い{うすぐらい}、かすかな〔衣類{いるい}の胸元{むなもと}が〕下がった、えぐれた《生物》〔構造{こうぞう}が〕単純{たんじゅん}な、下等{かとう}な〔人が〕粗野{そや}な、品のない、下劣{げれつ}な ・
\\	あいつは動物以下です。〔体が〕元気{げんき}がない、弱った〔精神的{せいしん てき}に〕落ち込んだ、意気消沈{いき しょうちん}した〔意見{いけん}などが〕人をけなす、軽蔑的{けいべつ てき}な〔自動車{じどうしゃ}のギアが〕ローの、低速{ていそく}の= 
\\	【1副】低い位置{いち}に、〔位置{いち}を〕低くして〔地位{ちい}や立場{たちば}などを〕低くして〔声や音の大きさが〕低く、静かに《音楽》〔音程{おんてい}が〕低く〔値段{ねだん}が〕安く、安価{あんか}に【2自動】〔ウシが〕モーと鳴く【2名】〔ウシの〕モーという鳴き声レベル1、発音
\\	、カナロー、ロウ、変化《形》
\\	、《動》
\\	明かりをつけ、かばんをベッドにひょいと投げた。 ・
\\	それをはっきりと見るために、後ろにいる男にどいて光を遮らないように言った。 ・
\\	火を貸してくれませんか。◆たばこを吸うときなど。 ・
\\	外はまだかなり明るい。 ・
\\	印字が薄過ぎてほとんど見えないことがある。 ・
\\	印字が薄過ぎて読み取れない。 ・
\\	このラップトップは非常に軽くて小さいので、どこへ行く時も必ず持って行きます。 ・
\\	人手が多ければ仕事は楽になる。	変化形 
\\	【1自動】火がつく、燃え始める輝く、光を発する【1他動】~に火を付ける、~を燃やす~を点灯{てんとう}する、〔照明{しょうめい}を〕つける~を明るくする、~を照らす〔光で〕~を導く[案内{あんない}する]~を生き生きさせる、~を明るくさせる【1名】《物理》〔約4000~7700オングストロームの〕可視光{かし こう}、〔あらゆる波長{はちょう}の〕電磁波{でんじは}〔知覚{ちかく}される〕光(の加減{かげん})、明るさ(具合{ぐあい})〔人工的{じんこう てき}な〕光源{こうげん}、ライト、明かり ・
\\	明かりをつけ、かばんをベッドにひょいと投げた。〔物や人を照らす〕光(の道筋{みちすじ}) ・
\\	それをはっきりと見るために、後ろにいる男にどいて光を遮らないように言った。交通信号{こうつう しんごう}、信号灯{しんごうとう}、警告灯{けいこく とう}、〔灯台{とうだい}の〕明かり日光{にっこう}、太陽光{たいようこう}夜明け{よあけ}、明け方〔物事の〕見方{みかた}、観点{かんてん}〔一般的{いっぱん てき}な〕知識{ちしき}、認識{にんしき}〔マッチなどの〕発火物{はっか ぶつ}、火 ・
\\	火を貸してくれませんか。◆たばこを吸うときなど。〔目の〕輝き、きらめき〔芸術作品{げいじゅつ さくひん}の〕光の表現方法{ひょうげん ほうほう}《建築》窓、明かり取り優れた人、指導者{しどうしゃ}【1形】明るい、輝いている ・
\\	外はまだかなり明るい。〔色が〕薄い、淡い ・
\\	印字が薄過ぎてほとんど見えないことがある。 ・
\\	印字が薄過ぎて読み取れない。〈米〉〔コーヒーに〕ミルクを入れた【2自動】〔車や馬から〕下りる、降りる〔鳥などが〕下りる、降り立つ出くわす、偶然出会う、突然降り懸かる【2形】〔重さが標準や見掛けよりも〕軽い、軽量の ・
\\	このラップトップは非常に軽くて小さいので、どこへ行く時も必ず持って行きます。〔力や衝撃{しょうげき}が〕軽い、柔らかい〔量が〕少ない、つつましい〔刑罰{けいばつ}などが〕軽い、厳しくない簡単{かんたん}な、易しい ・
\\	人手が多ければ仕事は楽になる。重要{じゅうよう}でない、つまらない、軽薄{けいはく}な心配{しんぱい}のない、気楽{きらく}な、陽気{ようき}な変わりやすい、気まぐれな軽快{けいかい}な、素早く動く有害物質{ゆうがい ぶっしつ}が少ない、低カロリーの、低アルコールのすぐに目が覚める、眠りが浅い〔頭などが〕ボーッとした、ぼんやりした《軍事》軽装備{けい そうび}の《化学》沸点{ふってん}が低い〔土壌{どじょう}などが〕もろい、多孔質{たこうしつ}の〔食べ物が〕消化{しょうか}の良い、さくさくした《言語》〔音節{おんせつ}が〕軽の◆短母音または短母音
\\	子音で終わる音節、またはストレスを置かない音節を指す。〈古〉〔性的{せいてき}に〕不道徳{ふどうとく}な、不品行{ふひんこう}な【2副】〔旅行{りょこう}をするときに〕荷物{にもつ}を少なくして手加減{てかげん}して、寛大{かんだい}にレベル1、発音
\\	、カナライト、変化《形》
\\	、《動》
\\	または
\\	夜明け前がいちばん暗い。/悪いことは続かない。/案ずるより生むが易し。◆良くない事態から良いことを見つけるときの決まり文句。 ・
\\	急ごう。もう日が暮れる。 ・
\\	印字が濃過ぎる。	変化形 
\\	【名】闇{やみ}、暗黒{あんこく}〔絵画{かいが}の〕暗部{あんぶ}、暗色{あんしょく}夜、日暮れ{ひぐれ}【形】〔光がなくて場所や時期が〕暗い、闇の、暗黒{あんこく}の ・
\\	夜明け前がいちばん暗い。/悪いことは続かない。/案ずるより生むが易し。◆良くない事態から良いことを見つけるときの決まり文句。 ・
\\	急ごう。もう日が暮れる。〔光を反射{はんしゃ}しないで物が〕暗い、黒い〔髪や目の色が〕黒い、黒みがかった〔皮膚{ひふ}や顔色{かおいろ}が〕黒ずんだ、浅黒い{あさぐろい}〔色が〕濃い、暗い ・
\\	印字が濃過ぎる。〔状況{じょうきょう}が〕陰鬱{いんうつ}な、悲観的{ひかん てき}な〔表情{ひょうじょう}などが〕怒った、敵意{てきい}に満ちた〔言動{げんどう}などが〕悪意{あくい}ある、邪悪{じゃあく}な〔意味{いみ}が〕分かりにくい、曖昧{あいまい}な〈話〉隠された、秘密{ひみつ}にされた、神秘的{しんぴ てき}な〈文〉〔国などが〕未開{みかい}の、文明{ぶんめい}が遅滞{ちたい}した〔劇場{げきじょう}が〕閉館{へいかん}している、公演{こうえん}していない〔声や音が〕甘美{かんび}な、豊かなレベル1、発音
\\	、カナダーク、変化《形》
\\	やっと霧が晴れた。 ・
\\	その煙はようやく晴れた。 ・
\\	私の潔白を証明する一番の方法は、殺人犯を警察に突き出すことだ。 ・
\\	1997年度予算が衆議院を通過しました。 ・
\\	1億ドル売り上げた。 ・
\\	《ゴルフ》グリーンは空いています。	変化形 
\\	【自動】きれいになる、片付く{かたづく}、消える〔雲・霧・煙などが〕消え去ってなくなる、晴れる ・
\\	やっと霧が晴れた。 ・
\\	その煙はようやく晴れた。〔二日酔いなどが〕治る、消えてなくなる【他動】〔障害物・不要物・邪魔なものを〕取り除く、除去{じょきょ}する、一掃する、片付ける{かたづける}、きれいにする《コ》〔データやファイルの内容{ないよう}を〕消去{しょうきょ}する、クリアする〔ゴホッと言って〕のどをすっきりさせる〔借金{しゃっきん}などを〕清算{せいさん}する、すべて支払う〔人の〕潔白{けっぱく}を証明{しょうめい}する、無実{むじつ}を証明{しょうめい}する、疑いを晴らす、(無罪{むざい})解放{かいほう}する、(無罪{むざい})放免{ほうめん}する ・
\\	私の潔白を証明する一番の方法は、殺人犯を警察に突き出すことだ。〔審査{しんさ}・審議{しんぎ}を無事{ぶじ}に〕通過{つうか}する、パスする ・
\\	1997年度予算が衆議院を通過しました。〔障害物{しょうがいぶつ}を〕飛び越える、(障害{しょうがい}・混乱{こんらん}を無事{ぶじ}に)通過{つうか}する〔居住者{きょじゅうしゃ}を〕立ち退かせる利益{りえき}を挙げる、売り上げ[もうけ]をもたらす、純益{じゅんえき}を挙げる ・
\\	1億ドル売り上げた。【形】〔物の見た目が〕澄んだ、きれいな、汚れていない〔空や空気{くうき}が〕晴れ[澄み]渡った◆雲、霧、ほこりなどがない。〔物が〕透明{とうめい}な、透き通った〔色が〕鮮やかな、純粋{じゅんすい}な〔物や物事{ものごと}が〕傷[欠点{けってん}]がない〔物や音が〕はっきり見える[聞こえる]〔音が〕澄んだ、清らかな明快{めいかい}な、曖昧{あいまい}さがない確実{かくじつ}な、疑う余地{よち}がないしっかりと[はっきりと]分かった[理解{りかい}された]◆【用法】叙述用法で
\\	の形で用いられる。〔人の気持{きも}ちが〕罪の意識{いしき}がない、責めを負っていない〔場所{ばしょ}が〕開けた、空いている、障害物{しょうがいぶつ}がない ・
\\	《ゴルフ》グリーンは空いています。〔人や心が〕明晰{めいせき}な、明敏{めいびん}な〔物の中が〕空の、中身{なかみ}がない[取り外された]〔利益{りえき}が〕正味{しょうみ}の、純の〔危険{きけん}などから〕遠ざかった、離れた借金{しゃっきん}[債務{さいむ}]がない〔馬術{ばじゅつ}の飛越競技{ひえつ きょうぎ}で〕減点{げんてん}がない【副】ずっと離れてレベル1、発音
\\	、カナクリアー、変化《形》
\\	、《動》
\\	彼は彼女の髪を優しくなでつけました。 ・
\\	私たちの間で事がうまく運ぶことを願っています。 ・
\\	《男女の会話》「どうして私の名前を知ってるの?」「この部屋で一番の美人は誰かって聞いて回ったのさ」「まあ(口が)うまいわね」
\\	変化形 
\\	"【自動】平らになる、滑らかになる【他動】~を平ら[滑らか]にする、~のざらつき[でこぼこ]を減らす[なくす]、~を整えるようになでる ・
\\	彼は彼女の髪を優しくなでつけました。〔問題{もんだい}・障害{しょうがい}などを〕取り除く、軽減{けいげん}する〔いさかいなどを〕鎮める【形】滑らかな、スベスベした、ツルツルした、ツルッとした、つるりとした平らな、平坦{へいたん}な、なだらかな順調{じゅんちょう}な、円滑{えんかつ}な、円滑{えんかつ}に動く、速やかな ・
\\	私たちの間で事がうまく運ぶことを願っています。流ちょうな、口先{くちさき}のうまい ・
\\	《男女の会話》「どうして私の名前を知ってるの?」「この部屋で一番の美人は誰かって聞いて回ったのさ」「まあ(口が)うまいわね」洗練{せんれん}された、上品{じょうひん}な◆言葉が人当たりの良い〔整数{せいすう}が〕小さな素因数しか持たないレベル2、発音
\\	、カナスムーズ、変化《形》
\\	、《動》
\\	彼女の服はますますきつくなった。◆「太ってきた」の遠回しな言い方。 ・
\\	私の予定が大変詰まっていますので、6月13日から15日の間のいつでも結構ですので、お会いできれば大変ありがたいと存じます。 ・
\\	あいにく来週は予定が詰まっておりまして、その会議を入れる都合がつきません。 ・
\\	お金に余裕がありません。/資金繰りがきついです。 ・
\\	資金繰りが厳しくなってきています。 ・
\\	警備は極めて厳重だった。 ・
\\	彼は私をギュッと抱き締めました。 ・
\\	私は目をギュッとつぶりました。	変化形 
\\	《複》
\\	【名】
\\	〈米俗〉苦境{くきょう}【形】〔ひもなどが〕ピンと張った、強く引っ張った〔物同士{ぶつ どうし}がしっかりと〕締まった、収まった、結んだ〔すきまがぴっちりと〕閉まった、水漏れしない〔衣服{いふく}・靴などが〕きつい、体にぴったりした ・
\\	彼女の服はますますきつくなった。◆「太ってきた」の遠回しな言い方。〔スケジュールが〕詰まった、余裕{よゆう}がない ・
\\	私の予定が大変詰まっていますので、6月13日から15日の間のいつでも結構ですので、お会いできれば大変ありがたいと存じます。 ・
\\	あいにく来週は予定が詰まっておりまして、その会議を入れる都合がつきません。〔予算などが〕逼迫{ひっぱく}した、厳しい ・
\\	お金に余裕がありません。/資金繰りがきついです。 ・
\\	資金繰りが厳しくなってきています。〔表情{ひょうじょう}が〕厳しい、こわばった〔警戒{けいかい}が〕厳重{げんじゅう}な ・
\\	警備は極めて厳重だった。〈話〉〔関係{かんけい}などが〕密接{みっせつ}な、親密{しんみつ}な〔布地{ぬのじ}などが〕目が詰まった〔場所{ばしょ}が〕窮屈{きゅうくつ}な、狭い〔状況{じょうきょう}が〕非情{ひじょう}に苦しい、どうしようもない〔競争{きょうそう}が〕接戦{せっせん}の、激しい〈俗〉〔酒に〕酔った〔演技{えんぎ}などが〕一糸乱れない〔表現{ひょうげん}が〕簡潔{かんけつ}な、明瞭{めいりょう}な〔経済活動{けいざい かつどう}が〕利益{りえき}が出ない〈話・軽蔑的〉〔人が〕けちな、締まり屋の《野球》〔ボールが〕内角{ないかく}の〈俗〉格好いい、素晴{すば}らしい〈英俗〉不公平{ふこうへい}な、意地悪{いじわる}な【副】〈話〉堅く、しっかりと◆【同】
\\	彼は私をギュッと抱き締めました。 ・
\\	私は目をギュッとつぶりました。〔締め方が〕きつく、きっちりと〔睡眠{すいみん}などが〕ぐっすりとレベル2、発音
\\	、カナタイト、変化《形》
\\	、《複》
\\	彼女の服や宝石の好みはお金のかかるものだ。 ・
\\	ワインは必ずしも高ければ良いというものではありません。		【形】高価{こうか}な、〔値段{ねだん}が〕高い、費用{ひよう}のかかる、高額{こうがく}の、金がかかる、高くつく、値の張る、高級{こうきゅう}な、もとがかかる、ぜいたくな◆値打ちとしてはそれ位かもしれないが、自分にとってはちょっと払えないという場合に用いる。 ・
\\	彼女の服や宝石の好みはお金のかかるものだ。 ・
\\	ワインは必ずしも高ければ良いというものではありません。レベル2、発音
\\	、カナイクスペンシブ、エクスペンスィブ、分節
\\	けちけちするなよ。	変化形 
\\	【形】安価{あんか}な、〔値段が〕安い、安物{やすもの}の、ちゃちな、ちんけな◆品質や見た目が悪いというニュアンスがある。◆【参考】
\\	けちな、せこい、しみったれた ・
\\	けちけちするなよ。なまはんかな高潔{こうけつ}でない、尊敬{そんけい}に値しない、卑劣{ひれつ}な、ずるい【副】廉価{れんか}でレベル1、発音
\\	、カナチープ、変化《形》
\\	彼は黒人奴隷だったが1813年に自由の身になった。 ・
\\	他人からの強制に対してはいやであると自由に言って構いません。 ・
\\	ここは自由の国だからね。 ・
\\	アメリカは1776年7月4日に独立国になった。 ・
\\	自由詩は19世紀後半にフランスで生まれた。 ・
\\	この本を読むと、面倒なことに巻き込まれない旅行をするための役立つ情報が得られます。 ・
\\	無脂肪乳に代えれば、簡単にもっと低脂肪の食事にすることができます。 ・
\\	その博物館は1週間のうち1日は入場無料になります。 ・
\\	そのホテルには空いている部屋は残っていません。 ・
\\	今晩、暇ある? ・
\\	手が空いたら手伝ってください。 ・
\\	結ばれていない方のロープの端を取ってオールをオール受けに結び付けなさい。 ・
\\	その構造物は水の自由な流れを妨げない。 ・
\\	脂肪腫の塊のせいでその犬の右前足は自由に動かない。 ・
\\	父は困っている人に本当に惜しみなく金を与えた。 ・
\\	彼の意訳は直訳によっては得られないもっと多くのことを教えてくれる。 ・
\\	結婚は完全な自由意思による行為により成立する。	変化形 
\\	【他動】~を自由{じゆう}にする、解放{かいほう}する〔重荷{おもに}や義務{ぎむ}などを〕軽減{けいげん}する、和らげる〔障害物{しょうがいぶつ}を〕取り除く【形】〔身体的{しんたい てき}に〕捕らわれていない、奴隷{どれい}でない、自由{じゆう}の身の ・
\\	彼は黒人奴隷だったが1813年に自由の身になった。〔他人{たにん}の意志{いし}による〕束縛{そくばく}を受けていない、〔他人{たにん}の意志{いし}から〕自由{じゆう}な ・
\\	他人からの強制に対してはいやであると自由に言って構いません。〔正当{せいとう}な権利{けんり}が〕制限{せいげん}[侵害{しんがい}]されていない、〔市民{しみん}として〕自由{じゆう}な ・
\\	ここは自由の国だからね。他国{たこく}の支配{しはい}を受けない、〔国が〕独立{どくりつ}した ・
\\	アメリカは1776年7月4日に独立国になった。〔伝統{でんとう}に〕とらわれない、〔伝統{でんとう}から〕自由{じゆう}な ・
\\	自由詩は19世紀後半にフランスで生まれた。〔望ましくないものの〕影響{えいきょう}を受けていない◆【用法】複合語で用いられることが多い。 ・
\\	この本を読むと、面倒なことに巻き込まれない旅行をするための役立つ情報が得られます。~が含まれていない、~を除いた◆【用法】通例複合語で用いられる。 ・
\\	無脂肪乳に代えれば、簡単にもっと低脂肪の食事にすることができます。無料{むりょう}の、ただの ・
\\	その博物館は1週間のうち1日は入場無料になります。空いている、使われていない ・
\\	そのホテルには空いている部屋は残っていません。暇な、用がない、忙しくない ・
\\	今晩、暇ある? ・
\\	手が空いたら手伝ってください。つながれて[結ばれて]いない、ほどけている ・
\\	結ばれていない方のロープの端を取ってオールをオール受けに結び付けなさい。〔流れが〕障害物{しょうがいぶつ}がない、自由{じゆう}な ・
\\	その構造物は水の自由な流れを妨げない。〔動きが〕妨げられない、自由{じゆう}な ・
\\	脂肪腫の塊のせいでその犬の右前足は自由に動かない。~を惜しみなく与える、~がすぐに出てくる ・
\\	父は困っている人に本当に惜しみなく金を与えた。意訳{いやく}の、〔訳が原文{げんぶん}に〕忠実{ちゅうじつ}でない ・
\\	彼の意訳は直訳によっては得られないもっと多くのことを教えてくれる。自発的{じはつ てき}な、自由意思{じゆう いし}の ・
\\	結婚は完全な自由意思による行為により成立する。《化学》結合{けつごう}していない、遊離{ゆうり}した《物理》〔電子{でんし}が〕自由{じゆう}な、〔エネルギー準位{じゅんい}が〕空位{くうい}の《海事》〔風が〕順風{じゅんぷう}の《言語学》〔形態素{けいたいそ}が〕自由{じゆう}な◆【対】
\\	◆【参考】
\\	《菌類》〔ひだなどが〕離生{りしょう}の[する]【副】自由{じゆう}に、束縛{そくばく}されずにただで、無料{むりょう}で、料金{りょうきん}を取らずにレベル1、発音
\\	、カナフリー、変化《形》
\\	、《動》
\\	私はいつものにします。◆レストランで	イディオムやフレーズ 
\\	【もっとイディオムを見る】
\\	【名】いつものやつ、お決まりのもの ・
\\	私はいつものにします。◆レストランで【形】通常{つうじょう}の、通例{つうれい}の、普通{ふつう}の、いつもの、よくあるレベル1、発音
\\	、カナユージュアル、分節
\\	私は家族を失ってやっと当たり前の幸せがどんなに大切か分かった。	イディオムやフレーズ 
\\	【名】
\\	普通{ふつう}の状態{じょうたい}[もの・こと]〈米〉オーディナリ型自転車◆非常に大きな前輪と小さな後輪から成る初期の自転車。《紋章》オーディナリ◆ベンド
\\	や十字などの単純な図を指す。《カトリック》〔裁判権{さいばんけん}を持つ〕司教{しきょう}
\\	とも表記《カトリック》〔ミサの〕通常文{つうじょう ぶん}◆暦によって変わらない部分。
\\	とも表記〈英古〉定食(屋){ていしょく(や)}【形】いつもの、平素の、普段の ・
\\	私は家族を失ってやっと当たり前の幸せがどんなに大切か分かった。よく目にする、どこにでもある、当たり前の目立{めだ}たない、つまらない、平凡{へいぼん}な《法律》〔判事{はんじ}が〕直轄{ちょっかつ}する《数学》〔微分方程式{びぶん ほうていしき}が〕常の◆未知関数がただ一つの変数を持つもの。レベル3、発音
\\	、カナオーディナリー、オーディネリ、オーデナリー、変化《複》
\\	、分節
\\	私はそのレストランの常連です。 ・
\\	彼は白い並びの良い歯をしている。 ・
\\	結婚証明書は正規の書類に手書きしなければならない。 ・
\\	清潔と運動と規則正しい習慣が健康を促進させるのに最も効果がある。 ・
\\	いつも最新のニュースをポストしておりますので、このページに定期的にアクセスしてください。 ・
\\	その女性は正常な生理以外の原因により出血しました。 ・
\\	新聞で毎日定職を探していた。 ・
\\	本職の音楽家はハイファイ・スピーカーにお金をかけている。 ・
\\	根っからの悪党を演じることから映画の仕事を始めた。 ・
\\	彼はいいやつで、有名人であろうと貧乏人であろうと人が好きなんだ。 ・
\\	整正花は放射相称をしている。 ・
\\	規則活用動詞と語幹が変化する動詞の違いを見分けられるようになりなさい。 ・
\\	正規軍で自爆者を使うところはない。	変化形 
\\	《複》
\\	【名】常連{じょうれん}(客)、お得意(様)、正会員{せいかいいん} ・
\\	私はそのレストランの常連です。標準{ひょうじゅん}サイズ、普通{ふつう}サイズ◆【略】
\\	常備軍{じょうびぐん}◆通例
\\	【形】習慣的{しゅうかん てき}な、いつもの、常連{じょうれん}の整然{せいぜん}とした、左右対称{さゆう たいしょう}の、均一{きんいつ}な ・
\\	彼は白い並びの良い歯をしている。手続きに従った、原則{げんそく}にのっとった、正規{せいき}の ・
\\	結婚証明書は正規の書類に手書きしなければならない。〔行動{こうどう}などが〕きちょうめんな、きちんとした、規則正{きそく ただ}しい ・
\\	清潔と運動と規則正しい習慣が健康を促進させるのに最も効果がある。〔出来事{できごと}が〕定期的{ていき てき}な、一定{いってい}の間隔{かんかく}ごとの ・
\\	いつも最新のニュースをポストしておりますので、このページに定期的にアクセスしてください。〔生理{せいり}やお通じなどが〕規則正{きそく ただ}しい、正常{せいじょう}な ・
\\	その女性は正常な生理以外の原因により出血しました。不変{ふへん}の、定まった ・
\\	新聞で毎日定職を探していた。〔職業{しょくぎょう}の〕資格{しかく}を持った、〔仕事{しごと}が〕本職{ほんしょく}の ・
\\	本職の音楽家はハイファイ・スピーカーにお金をかけている。〈話〉全くの、根っからの ・
\\	根っからの悪党を演じることから映画の仕事を始めた。〈話〉良い、好ましい ・
\\	彼はいいやつで、有名人であろうと貧乏人であろうと人が好きなんだ。《植物》整正{せいせい}の◆植物の部分が対称的な形や大きさをしていること。 ・
\\	整正花は放射相称をしている。《文》〔屈折{くっせつ}や派生{はせい}の変化{へんか}が〕規則的{きそく てき}な ・
\\	規則活用動詞と語幹が変化する動詞の違いを見分けられるようになりなさい。《数学》〔多角形{たかくけい}や多面体{ためんたい}が〕正の◆【参考】
\\	〔軍隊{ぐんたい}が〕正規{せいき}の ・
\\	正規軍で自爆者を使うところはない。レベル2、発音
\\	、カナレギュラー、変化《複》
\\	、分節
\\	ただの一人も英語を話せる人はいない。	変化形 
\\	【自動】《野球》安打{あんだ}[単打{たんだ}]を打つ、一塁打{いちるいだ}を打つ【他動】~を選び出す◆【用法】通例
\\	と共に用いる。《野球》〔走者{そうしゃ}を〕シングル・ヒットで返す[進める]《野球》〔打点{だてん}を〕シングル・ヒットで上げる【名】〔ホテルなどの〕一人部屋{ひとり べや}、シングル《野球》シングル・ヒット◆【同】
\\	《ゴルフ》〔1対1の〕マッチ・プレー◆【同】
\\	《音楽》シングルレコード
\\	・カセット]〈俗〉1ドル札〈和製英語〉《ゴルフ》シングル◆【標準英語】
\\	【形】ただ一つの、たった一つの、たった一人{ひとり}の、単独{たんどく}の ・
\\	ただの一人も英語を話せる人はいない。一つだけから成る、一つの要素{ようそ}で構成{こうせい}されるそれぞれの、一つ一つのひと塊の、分割{ぶんかつ}されていない均一{きんいつ}の、同一{どういつ}の〔競技{きょうぎ}が〕1対1の、二人{ふたり}で争われる〔ホテルの部屋{へや}などが〕一人用{ひとり よう}の、シングルの独身{どくしん}の、配偶者{はいぐうしゃ}がいない《植物》〔花が〕単弁{たんべん}の、一重{いちじゅう}のレベル2、発音
\\	、カナシングル、スィングル、変化《動》
\\	、分節
\\	【形】数個[人]の◆2~3個より多いが、
\\	ほどではないものを指す。それぞれの、各自{かくじ}の《法律》〔連帯債務者{れんたい さいむ しゃ}の〕個別{こべつ}のレベル1、発音
\\	、カナセブラル、セベラル、分節
\\	変化形 
\\	《複》
\\	【名】100万、百万{ひゃくまん}◆【略】
\\	100万ドル[ポンド]
\\	膨大{ぼうだい}な数、極めて多数{たすう}、無数{むすう}
\\	大衆{たいしゅう}、民衆{みんしゅう}【形】100万の、何百万{なんびゃくまん}という、無数{むすう}のレベル2、発音
\\	、カナミリオン、変化《複》
\\	、分節
\\	【代名】〔異なる〕いくつかのもの、何人{なんにん}かの人◆【用法】この用法を認めない人もいる。◆複数として扱われる。【形】〔種類{しゅるい}が〕さまざまな、いろいろな◆【用法】複数名詞を修飾する。〔同じ種類{しゅるい}の〕別々{べつべつ}の、個々{ここ}の◆【用法】複数名詞を修飾する。〔数が〕数個{すうこ}の、数人{すうにん}の、いくつかの◆【用法】複数名詞を修飾する。多様{たよう}な、多方面{たほうめん}の、多面的{ためん てき}な〔色が〕まだら(模様{もよう})のレベル3、発音
\\	、カナベアリアス、分節
\\	まる1週間、風呂に入れなかった。	変化形 
\\	《複》
\\	【名】〔欠けるもののない〕全部{ぜんぶ}、全体{ぜんたい}〔部分{ぶぶん}から成る〕統一{とういつ}体[組織{そしき}]【形】すべてを含んだ、欠けたものがない分けられていない、一体{いったい}となった ・
\\	まる1週間、風呂に入れなかった。〔時間{じかん}などが〕途切れていない、ひと続きの〔物が〕壊れていない、損傷{そんしょう}のない、無傷{むきず}の〔人が〕健康{けんこう}になった、回復{かいふく}した〔子どもが〕同父母{どう ふぼ}の◆【参考】
\\	《数学》整数{せいすう}の◆【参考】
\\	【副】〔いくつかの物を〕丸ごと、ひとまとめにして〈話〉完全{かんぜん}に、全くレベル2、発音
\\	、カナホール、ホウル、変化《複》
\\	サイコロの目は裏側の目と合わせて7にならなければならない。	変化形 
\\	【自動】合計{ごうけい}で[総計{そうけい}して]~になる〈米俗〉〔自動車などが〕全壊{ぜんかい}[全損{ぜんそん}]する【他動】~を総計{そうけい}する、合計{ごうけい}で~となる ・
\\	サイコロの目は裏側の目と合わせて7にならなければならない。~を完全{かんぜん}に壊す、めちゃくちゃにする、廃車{はいしゃ}にする【名】合計{ごうけい}、総計{そうけい}、総量{そうりょう}、総数{そうすう}、総額{そうがく}【形】全部{ぜんぶ}の、すべての、全体{ぜんたい}の、全面的{ぜんめんてき}な、総計{そうけい}の、総体{そうたい}の、総力{そうりょく}の完全{かんぜん}な、全くの【副】= 
\\	レベル2、発音
\\	、カナトータル、トウタル、変化《動》
\\	、分節
\\	変化形 
\\	《複》
\\	【名】〔社会的集団{しゃかい てき しゅうだん}と対比{たいひ}させた〕個人{こじん}、一人{ひとり}の人間{にんげん}〔それぞれ異なる個性{こせい}を持つ〕人間{にんげん}〔変わった個性{こせい}を持つ〕人、人間{にんげん}《哲学》〔分けられない〕個体{こたい}◆分割すればその性質を維持できないもの。《生物》〔集団{しゅうだん}や種と対比{たいひ}させた〕個体{こたい}《法律》〔法人{ほうじん}に対する〕自然人{しぜんじん}【形】〔社会的集団{しゃかい てき しゅうだん}と対比{たいひ}させた〕個人{こじん}の個人用{こじんよう}の、個人向け{こじん むけ}の〔別れている〕個別{こべつ}の、個々{ここ}の、単一{たんいつ}の〔それぞれが〕独自{どくじ}の、他とは異なっている独特{どくとく}な、個性的{こせい てき}な、目立{めだ}ったレベル3、発音
\\	、カナインディビジュアル、インデビジュアル、変化《複》
\\	、分節
\\	彼は強い独立性を発揮して党を離脱しました。	変化形 
\\	《複》
\\	【名】独立{どくりつ}[自立{じりつ}]している人[物]無所属{むしょぞく}の人、政党{せいとう}に属さない人無党派層{むとうは そう}◆複数形で【形】〔他の国や組織{そしき}から〕独立{どくりつ}した、支配{しはい}されない、自治権{じちけん}を持つ〔動作{どうさ}や機能{きのう}などが〕独立{どくりつ}した、影響{えいきょう}を受けない、自律的{じりつ てき}な〔経済的{けいざい てき}に〕自活{じかつ}できる、自立{じりつ}した、依存{いそん}していない〔人や精神{せいしん}が〕独立心{どくりつしん}ある、自主性{じしゅ せい}がある、人に頼らない ・
\\	彼は強い独立性を発揮して党を離脱しました。〔利害関係者{りがい かんけいしゃ}から〕独立{どくりつ}した、介入{かいにゅう}されない、干渉{かんしょう}を受けない〔収入{しゅうにゅう}などが〕働かなくても十分{じゅうぶん}な[やっていける]〔政党{せいとう}などから〕拘束{こうそく}を受けない、無所属{むしょぞく}の
\\	とも表記。《数学》〔連立方程式{れんりつ ほうていしき}が〕独立{どくりつ}の、独立{どくりつ}した《統計》〔変数{へんすう}や事象{じしょう}が〕独立{どくりつ}の、独立{どくりつ}した《論理学》〔命題{めいだい}や公理{こうり}が〕独立{どくりつ}の、独立{どくりつ}した
\\	〔17世紀{せいき}イギリスの国教会{こっきょうかい}からの〕独立派{どくりつは}のレベル3、発音
\\	、カナインディペンデント、インデペンデント、変化《複》
\\	、分節
\\	あの人は厚かまし過ぎる。 ・
\\	そんなになれなれしくしないで。	変化形 
\\	【名】初期{しょき}、〔一日{いちにち}の〕早い時間帯{じかんたい}= 
\\	【形】新鮮{しんせん}な、未使用{みしよう}の、到着{とうちゃく}したばかりの《料理》生の、乾燥{かんそう}させていない出来たての、焼きたての、作りたての、産みたての新たな、斬新{ざんしん}な〔記憶{きおく}に〕生々{なまなま}しい生意気{なまいき}な、厚かましい、なれなれしい ・
\\	あの人は厚かまし過ぎる。 ・
\\	そんなになれなれしくしないで。〈俗〉非常{ひじょう}に良い[素晴{すば}らしい・いけてる]〔人の外見などが〕清潔感{せいけつ かん}[清涼感{せいりょう かん}]のあるレベル1、発音
\\	、カナフレッシュ、フレッシ、変化《形》
\\	変化形 
\\	《複》
\\	【名】日刊新聞{にっかん しんぶん}、日刊紙{にっかんし}
\\	とのスペリングの違いに注意〈英話〉通いのお手伝いさん【形】毎日{まいにち}の、日常{にちじょう}の【副】日々{ひび}、常に、絶えずレベル2、発音
\\	、カナデイリー、デーリー、変化《複》
\\	、分節
\\	変化形 
\\	《複》
\\	【名】年鑑{ねんかん}、年次{ねんじ}、年報{ねんぽう}年刊{ねんかん}の刊行物{かんこうぶつ}《植物》一年生植物{いちねん せい しょくぶつ}【形】年に1度の、例年{れいねん}の、1年の、年々{ねんねん}の、毎年{まいとし/まいねん}の、年1回の、毎年恒例{まいとし こうれい}の《植物》一年生{いちねんせい}のレベル4、発音
\\	、カナアニュアル、変化《複》
\\	、分節
\\	部屋に誰もいないときは電気を消してください。ご協力に感謝します。 ・
\\	私たちは、あなたが会議に出席してくださることを望んでおります。 ・
\\	生徒は以下5件のトピックから好きなものを一つ選択して論文を書くように指示されました。 ・
\\	私の父を紹介させていただきます。 ・
\\	あなたへのプレゼントです。気に入っていただけるといいのですが。 ・
\\	彼女は誕生日プレゼントとして金の指輪をもらいました。	変化形 
\\	【1名】今、現在{げんざい}
\\	《法律》文書類{ぶんしょるい}、本証書{ほん しょうしょ}《文法》現在時制{げんざい じせい}◆【略】
\\	【1形】存在{そんざい}している、そこにある ・
\\	部屋に誰もいないときは電気を消してください。ご協力に感謝します。出席{しゅっせき}[出勤{しゅっきん}・参列{さんれつ}]して、居合わせて ・
\\	私たちは、あなたが会議に出席してくださることを望んでおります。現在{げんざい}の、今の、目下の、当面{とうめん}の《文法》現在時制{げんざい じせい}の存在感{そんざいかん}がある【2自動】《医》〔検査{けんさ}で患者{かんじゃ}に〕症状{しょうじょう}が見つかる【2他動】〔人に物を正式{せいしき}に〕贈る、贈呈{ぞうてい}[進呈{しんてい}]する〔物や賞などを〕贈り物にする、贈り物として進呈{しんてい}する〔おわびや褒め言葉を〕伝える、述べる〔見えるように〕~を示す[提示{ていじ}する]〔正式{せいしき}に〕~を渡す[提示{ていじ}する]◆相手が検査をしたり検討したりできるように、提案書や見本などを渡すこと。 ・
\\	生徒は以下5件のトピックから好きなものを一つ選択して論文を書くように指示されました。〔問題点{もんだいてん}などを人に〕示す、提起{ていき}する《法律》(人)を告訴{こくそ}する、~について法廷{ほうてい}で争う〈文〉〔目上{めうえ}の人に正式{せいしき}に人を〕紹介{しょうかい}する、謁見{えっけん}させる ・
\\	私の父を紹介させていただきます。〔テレビなどの番組{ばんぐみ}の〕司会{しかい}をする、ホストを務める〔映画{えいが}や演劇{えんげき}などを〕放送{ほうそう}する、上演{じょうえん}する〔若い女性{じょせい}を社交界{しゃこうかい}に〕送り出す、デビューさせる〔芸術作品{げいじゅつ さくひん}で人や物を~として〕表現{ひょうげん}する、描く〔自分{じぶん}をある場所{ばしょ}と時間{じかん}に〕出頭{しゅっとう}させる、姿を現させる◆【用法】再帰代名詞を目的語に取る。〔自らを〕発現{はつげん}させる、引き起こさせる◆【用法】再帰代名詞を目的語に取る。〔敬礼{けいれい}のために武器{ぶき}を〕ささげる【2名】プレゼント、贈り物 ・
\\	あなたへのプレゼントです。気に入っていただけるといいのですが。 ・
\\	彼女は誕生日プレゼントとして金の指輪をもらいました。レベル1、発音《名》
\\	《動》
\\	、カナプレズント、プレゼント、プリゼント、変化《動》
\\	この種の犬にはしっぽがありません。	変化形 
\\	【他動】~を欠席{けっせき}する【形】〔いるべき場所{ばしょ}に〕いない、不在{ふざい}の、留守{るす}の〔学校{がっこう}・会社{かいしゃ}・会合{かいごう}などを〕欠席{けっせき}して、欠勤{けっきん}して〔あるべきものが〕ない、欠けて ・
\\	この種の犬にはしっぽがありません。放心状態{ほうしん じょうたい}の、ぼんやりした【前】~がないので、~がないときは、~がなくて、~がない限りレベル2、発音《形》
\\	《動》
\\	、カナアブセント、変化《動》
\\	、分節
\\	6時5分です。 ・
\\	いつも寝る時間を2時間もオーバーしている。	変化形 
\\	《複》
\\	【名】過去形{かこ けい}、過去時制{かこ じせい}
\\	素性【形】〔時が〕過ぎた過去{かこ}の、これまでの《文法》過去形{かこ けい}の【前】~を過ぎて ・
\\	6時5分です。 ・
\\	いつも寝る時間を2時間もオーバーしている。~のそばを(通り)過ぎて、~を過ぎ去って〔能力・限度などを〕超えて【副】過ぎて、(通り)越えて、(通り)越してレベル1、発音
\\	、カナパースト、パスト、変化《複》
\\	変化形 
\\	《複》
\\	【名】未来{みらい}、将来(性){しょうらい(せい)}、今後{こんご}、先行き{さきゆき}、将来像{しょうらい ぞう}《文法》未来時制{みらい じせい}先物{さきもの}(取引{とりひき})◆複数形で【形】未来{みらい}[将来{しょうらい}・今後{こんご}]の[向けた]、次世代{じせだい}のレベル1、発音
\\	、カナフューチャー、変化《複》
\\	、分節
\\	現代家屋は、デザインはいいが造りがあまりしっかりしていない。		【名】現代人{げんだい じん}、現代{げんだい}に生きる人〔考え方などが〕現代的{げんだい てき}な人《印刷》〔書体{しょたい}の〕モダン【形】現代{げんだい}の、今の ・
\\	現代家屋は、デザインはいいが造りがあまりしっかりしていない。〔歴史上{れきしじょう}の〕近代{きんだい}の、近世{きんせい}の〔手法・考え方などが〕現代的{げんだい てき}な、最新{さいしん}の〔考えや技術{ぎじゅつ}が〕前衛的{ぜんえい てき}な、実験的{じっけん てき}な
\\	《言語学》〔言語{げんご}が〕近代{きんだい}のレベル2、発音
\\	、カナマダン、モダーン、分節
\\	【形】〔出来事{できごと}などが〕最近{さいきん}の、少し前の近頃{ちかごろ}の、新しい
\\	《地学》完新世{かんしんせい}◆【同】
\\	レベル3、発音
\\	、カナリースント、リーセント、分節
\\	変化形 
\\	《複》
\\	【名】古代人{こだい じん}【形】古代{こだい}の、古くからの、古い、古びた、古色蒼然{こしょく そうぜん}とした、古来{こらい}の、太古{たいこ}からの、古風{こふう}な、時代{じだい}がかった〈米話〉年老いた、もうろくしたレベル4、発音
\\	、カナエインシェント、エンシェント、エンシエント、変化《複》
\\	、分節
\\	変化形 
\\	《複》
\\	【名】基本{きほん}、基礎{きそ}
\\	基本的{きほん てき}なもの[要素{ようそ}・考え方]、主要{しゅよう}ポイント、基本原則{きほん げんそく}《化学》塩基性{えんきせい}【形】基礎{きそ}の、基本的{きほんてき}な簡単{かんたん}な、簡易{かんい}の◆【同】
\\	必要最小限{ひつよう さいしょうげん}の塩基(性){えんき(せい)}の◆【参考】塩基
\\	レベル2、発音
\\	、カナベイシック、ベイスィック、変化《複》
\\	、分節
\\	変化形 
\\	【名】〈古〉単純なもの、単体〈古・軽蔑的〉単純な人、ばか〈古〉薬草【形】易しい、難しくない簡素{かんそ}な、簡略{かんりゃく}した、シンプルな質素{しっそ}な、地味{じみ}な、豪華{ごうか}でない単一{たんいつ}の、一つだけで構成{こうせい}される単純{たんじゅん}な、複雑{ふくざつ}でない〔人が〕気取らない、控えめな、見えを張らない〈軽蔑的〉〔知的{ちてき}レベルが〕単純{たんじゅん}な、ばかな〈軽蔑的〉教養{きょうよう}がない、無知{むち}な〔人が〕素朴{そぼく}な、洗練{せんれん}されていない〔人が〕真面目{まじめ}な、誠実{せいじつ}な、うそをつかない普通{ふつう}の、いつもの、ありふれた基本{きほん}の、初歩{しょほ}のささいな、重要{じゅうよう}でない、つまらない《植物》単一{たんいつ}の◆【対】
\\	《化学》〔化合物{かごうぶつ}が〕単純{たんじゅん}な《言語学》〔文構造{ぶん こうぞう}が〕単純{たんじゅん}なレベル1、発音
\\	、カナスィンポゥ、シンプル、スィンプル、変化《形》
\\	、分節
\\	変化形 
\\	《複》
\\	【他動】~を複雑{ふくざつ}にする【名】コンプレックス、強迫観念{きょうはく かんねん}、固定概念{こてい がいねん}複合体{ふくごうたい}、合成物{ごうせいぶつ}、複合{ふくごう}[総合{そうごう}]施設{しせつ}、総合{そうごう}ビル、コンビナート《生化学》錯体{さくたい}、錯イオン【形】複雑{ふくざつ}な、複合{ふくごう}の、込み入った、手間{てま}のかかる《数学》複素数{ふくそすう}の、複素{ふくそ}~◆定義上は実数を含む。実際には、実数以外の複素数やそれに関連する概念を指すことが多い。◆【参考】
\\	レベル3、発音《動・形》
\\	《名》
\\	、カナカンプレックス、コンプレックス、変化《複》
\\	、分節
\\	どんなことでも簡単にできるようになるまでは難しい。/習うより慣れよ。 ・
\\	彼は、ときどき気難しく[扱いにくく]なることがあります。		【形】〔達成が〕困難{こんなん}な、厳しい ・
\\	どんなことでも簡単にできるようになるまでは難しい。/習うより慣れよ。〔耐えるのが〕苦しい、つらい〔人が〕気難{きむずか}しい、扱いが面倒{めんどう}な ・
\\	彼は、ときどき気難しく[扱いにくく]なることがあります。〔人が〕説得{せっとく}しにくい、頑固{がんこ}な〔理解{りかい}や解決{かいけつ}が〕難しい、難解{なんかい}なレベル1、発音
\\	、カナディフィカルト、デフィカルト、分節
\\	以下の文章に誤り[間違い]があれば、訂正しなさい。◆試験問題などで。 ・
\\	間違いを発見してくださった場合は、訂正をお願いします。 ・
\\	私たちは、自分に関する不正確な個人情報を修正する権利を持っています。 ・
\\	まだ翻訳が終わってないんでしたよね。◆【参考】
\\	変化形 
\\	【他動】〔間違い・誤りなどを〕訂正{ていせい}[修正{しゅうせい}]する、正す ・
\\	以下の文章に誤り[間違い]があれば、訂正しなさい。◆試験問題などで。 ・
\\	間違いを発見してくださった場合は、訂正をお願いします。 ・
\\	私たちは、自分に関する不正確な個人情報を修正する権利を持っています。〔悪い状態などを〕是正{ぜせい}[補正{ほせい}]する〔変形などを〕直す、矯正{きょうせい}する【形】正しい、正確{せいかく}な、誤りのない、間違っていない、合っている、的確{てきかく}な〈話〉そうですよね、合ってる?◆付加疑問的に ・
\\	まだ翻訳が終わってないんでしたよね。◆【参考】
\\	レベル2、発音
\\	、カナコレクト、変化《動》
\\	、分節
\\	その数字が正確[正しい]かどうかは分からない。		【形】〔情報・計算結果などが〕正確な ・
\\	その数字が正確[正しい]かどうかは分からない。〔機械などが〕精密な〔人の仕事の結果などが〕間違いのない〔予想・占いなどが〕あたった、的中{てきちゅう}したレベル4、発音
\\	、カナアキュレイト、アキュレット、分節
\\	変化形 
\\	【自動】等しくなる、同じようになる【他動】〔価値{かち}などが~に〕等しい、釣り合う〔~に等しいことを〕成し遂げる、作り上げる【名】〔他の人やものと〕同等{どうとう}の[匹敵{ひってき}する]人[もの]【形】〔大きさや価値{かち}などが〕等しい、同一{どういつ}の〔権利や機会などが〕平等{びょうどう}な、公平{こうへい}な〔対立{たいりつ}しているものの力などが〕互角{ごかく}の、均衡{きんこう}している〈文〉〔人が仕事{しごと}などに対して〕十分{じゅうぶん}な力がある、対応{たいおう}できる〈文〉〔取り扱いや影響{えいきょう}などが〕公平{こうへい}な、偏らない〈文〉〔心や気分{きぶん}が〕平静{へいせい}な、穏やかなレベル2、発音
\\	、カナイークワル、イコール、変化《動》
\\	、分節
\\	イディオムやフレーズ 
\\	【形】〔同一{どういつ}ではないが〕類似{るいじ}の、同様{どうよう}の《数学》〔図形{ずけい}が〕相似(形){そうじ(けい)}のレベル2、発音
\\	、カナシミラー、スィミラ、分節
\\	正反対同士は引き合う。◆ことわざ ・
\\	正反対[全く逆]のことが起こった。 ・
\\	私は逆の問題がある。	変化形 
\\	《複》
\\	【名】〔位置の〕反対{はんたい}〔方向・性質などが〕正反対{せいはんたい}の人[物・こと]、逆の物、反対{はんたい}の物 ・
\\	正反対同士は引き合う。◆ことわざ ・
\\	正反対[全く逆]のことが起こった。反意語{はんいご}〔演劇などの主役の〕相手役{あいてやく}【形】〔位置が〕反対側{はんたいがわ}の、向かい側の、向かい合った〔方向・性質などが〕正反対{せいはんたい}の、逆の、あべこべの、対照的{たいしょうてき}な ・
\\	私は逆の問題がある。【前】~の反対{はんたい}[向かい]側{がわ}に、~に向かい合って〔演劇などで主役の〕相手役{あいてやく}で[として]レベル2、発音
\\	、カナアポズィット、オポジット、変化《複》
\\	、分節
\\	実際はその反対です。	イディオムやフレーズ 
\\	【名】正反対{せいはんたい} ・
\\	実際はその反対です。逆張り{ぎゃくばり}【形】正反対{せいはんたい}の、相いれない、逆の、反して、反対{はんたい}の、不利{ふり}なレベル4、発音
\\	、カナカントゥレリ、コントラリー、変化《複》
\\	、分節
\\	歯を大事にしなければ、歯に裏切られてしまう。/歯の手入れを怠ると、入れ歯
\\	をしなければならない羽目になる。	変化形 
\\	【形】〔判断・判定・識別などが〕正しくない、誤った〔主張・申し立てなどが〕うその、虚偽{きょぎ}の〔人が〕不誠実{ふせいじつ}な、裏切{うらぎ}る〔だますための〕見せ掛けの、惑わせる本物{ほんもの}でない、偽の人工{じんこう}の、模造{もぞう}の ・
\\	歯を大事にしなければ、歯に裏切られてしまう。/歯の手入れを怠ると、入れ歯
\\	をしなければならない羽目になる。《植物》〔別の植物{しょくぶつ}に似た〕ニセの【副】〈文〉裏切{うらぎ}って◆【用法】
\\	として使われる。レベル2、発音
\\	、カナフォールス、変化《形》
\\	ばかなことを言ってごめんなさい。 ・
\\	《疑いを表して》全くばかげたことだ。 ・
\\	みっともないことはできない。 ・
\\	あの犬はめちゃくちゃかわいいです。	変化形 
\\	【名】〈話〉ばか、あほ、とんま、間抜け、能なし、愚か者{おろかもの}、鈍いやつ、うすのろ、ぐず、ぼんくら《呼び掛け》ばかだな、いいかい◆文尾で用いられるが、あまり上品な表現ではない。【形】頭が悪い、頭の鈍い、ばかな、愚かな、愚劣{ぐれつ}な ・
\\	ばかなことを言ってごめんなさい。 ・
\\	《疑いを表して》全くばかげたことだ。〈話〉〔行為などが〕ばかげた、みっともない、くだらない ・
\\	みっともないことはできない。〈話〉〔本・催し物などが〕ひどくつまらない〈話〉腹立たしい、いまいましい〈米俗〉いかす、格好いい
\\	、良い
\\	〔ショック・疲労{ひろう}・酩酊{めいてい}などで〕ボーッとした【副】〈米俗〉気もくるわんばかりに、ものすごく ・
\\	あの犬はめちゃくちゃかわいいです。レベル2、発音
\\	、カナスチューピッド、ステューピッド、変化《複》
\\	、《形》
\\	、分節
\\	しみますか。◆外傷に薬をつけてあげている人が尋ねる ・
\\	それを受け入れた方が賢明だ。 ・
\\	彼は、見掛けによらず切れ者です。 ・
\\	彼は美形で本当に頭がいいです。	変化形 
\\	【自動】〔身体の一部が〕うずく、ズキズキ[ヒリヒリ]痛む、〔薬などが〕しみる ・
\\	しみますか。◆外傷に薬をつけてあげている人が尋ねる精神的苦痛{せいしん てき くつう}を受ける、ひどく感情{かんじょう}を害する〔重い〕罰を受ける、罪を償う【名】苦痛{くつう}、痛み、うずき苦悩、傷心
\\	〈米俗〉知性{ちせい}、利口{りこう}さ◆【同】
\\	【形】利口{りこう}な、賢明{けんめい}な、賢い、頭が切れる、頭の回転{かいてん}が速い ・
\\	それを受け入れた方が賢明だ。 ・
\\	彼は、見掛けによらず切れ者です。 ・
\\	彼は美形で本当に頭がいいです。〔機器が〕高性能の、コンピューターを用いた、コンピューター化された、ハイテクの気が利く〈主に英〉〔身なり・外見などが〕格好いい、おしゃれな、洗練{せんれん}された、小粋{こいき}なハイカラの、最新流行の〔人や店などが〕上流社会の、粋な、高級な、あか抜けした〔人の動作や物事の動きが〕きびきびした、活発な、急激な〔人の言動が〕生意気な、こしゃくな抜け目のない〈和製英語〉スマートな〔痩{や}せている〕◆【標準英語】
\\	【副】厳しく、きびきびと、抜け目なくレベル3、発音
\\	、カナスマート、変化《形》
\\	、《動》
\\	賢い人もいれば、そうではない人もいる。 ・
\\	愚者が賢者から学ぶことより多くを賢者は愚者から学ぶ。	変化形 
\\	【形】賢い、博学{はくがく}の、博識{はくしき}な、賢そうな ・
\\	賢い人もいれば、そうではない人もいる。 ・
\\	愚者が賢者から学ぶことより多くを賢者は愚者から学ぶ。生意気{なまいき}な、偉そうな、うぬぼれたレベル1、発音
\\	、カナワイズ、変化《形》
\\	、《動》
\\	変化形 
\\	【形】〔物事を遂行するために必要な〕能力{のうりょく}[資格{しかく}]のあるレベル1、発音
\\	、カナエイブル、変化《形》
\\	、分節
\\	率直な意見を聞かせてください。 ・
\\	正直に言ってください。	イディオムやフレーズ 
\\	【名】
\\	正直者{しょうじきもの}【形】〔人が〕正直{しょうじき}な、誠実{せいじつ}な、実直{じっちょく}な〔言動{げんどう}などが〕ごまかしがない、率直{そっちょく}な ・
\\	率直な意見を聞かせてください。 ・
\\	正直に言ってください。〔物が〕本物{ほんもの}の、純正{じゅんせい}の〔物事{ものごと}が〕公平{こうへい}な、公正{こうせい}な〔悪意{あくい}がなくて〕納得{なっとく}できる、受け入れられる〔人などが〕素朴{そぼく}な、見えを張らない、飾りのないレベル1、発音
\\	、カナアネスト、オネスト、分節
\\	率直におっしゃっていただいたことを本当に感謝しています。 ・
\\	率直に言わせてもらいますよ。	変化形 
\\	【1他動】〔郵便物{ゆうびんぶつ}に〕無料配達印{むりょう はいたつ いん}を押す〔郵便物{ゆうびんぶつ}を〕無料{むりょう}で配達{はいたつ}する〔郵便物{ゆうびんぶつ}の切手{きって}に〕スタンプを押す〔人を〕自由{じゆう}に通す[行き来させる]【1名】〔郵便物{ゆうびんぶつ}の〕無料配達印{むりょう はいたつ いん}〔議員{ぎいん}などに与えられる〕無料{むりょう}で郵便物{ゆうびんぶつ}を送る権利{けんり}無料配送郵便物{むりょう はいそう ゆうびんぶつ}【1形】〔意見{いけん}などが〕率直{そっちょく}な、偽りのない〔人が〕正直{しょうじき}な、はっきり物を言う ・
\\	率直におっしゃっていただいたことを本当に感謝しています。 ・
\\	率直に言わせてもらいますよ。〔態度{たいど}などが〕露骨{ろこつ}な、ぶっきらぼうな〔出来事{できごと}などが〕明らかな、公然{こうぜん}の【2名】〈話〉フランクフルト・ソーセージ◆【同】
\\	レベル3、発音
\\	、カナフランク、変化《形》
\\	、《動》
\\	【形】明らかな、疑う余地{よち}のない見え透いた、露骨{ろこつ}なレベル3、発音
\\	、カナアブビアス、オブビアス、分節
\\	晴れのち時々曇り。 ・
\\	恋愛と戦争は手段を選ばない。/どんなことでも起こり得る。/勝てば官軍。 ・
\\	人生は公平であると思うな。 ・
\\	お互い公平に行こう。 ・
\\	ちゃんと警告しましたからね。 ・
\\	女はバラのようなものでひとたび美しく花開いたらそれは散る時である。◆シェークスピアの「十二夜」より。	変化形 
\\	《複》
\\	【1自動】天気{てんき}が良くなる、晴れる、好転{こうてん}する【1他動】〔変形{へんけい}したものを〕元の形に戻す、正しい形にする、真っすぐにする、整形{せいけい}する〔配置などが乱れたものを〕整列{せいれつ}させる、正しく配列{はいれつ}する、調整{ちょうせい}する〔文書{ぶんしょ}を〕清書{せいしょ}する〔粗い表面などを〕平らにする、滑らかにする、スムーズにする【1名】〈古〉美女{びじょ}、最愛{さいあい}の女性{じょせい}〈古〉美しさ、美【1形】快晴{かいせい}の、晴れの ・
\\	晴れのち時々曇り。公平{こうへい}な、公正{こうせい}な、公明正大{こうめい せいだい}な、偏見{へんけん}のない、偏りのない ・
\\	恋愛と戦争は手段を選ばない。/どんなことでも起こり得る。/勝てば官軍。 ・
\\	人生は公平であると思うな。 ・
\\	お互い公平に行こう。 ・
\\	ちゃんと警告しましたからね。正しい、規則{きそく}に従ったきれいな、美しい、魅力的{みりょくてき}な ・
\\	女はバラのようなものでひとたび美しく花開いたらそれは散る時である。◆シェークスピアの「十二夜」より。(肌の)色が白い、色白{いろじろ}の、色の淡い〔数量{すうりょう}・程度{ていど}が〕かなりの見込みが十分{じゅうぶん}にある、有望{ゆうぼう}な口先{くちさき}だけの、まことしやかな〈豪〉完全{かんぜん}な、疑いのない【1副】公正{こうせい}に、フェアに、公明正大{こうめい せいだい}に、正々堂々{せいせいどうどう}と、真っすぐ(に)、うまく、順調{じゅんちょう}に【2名】〔定期的{ていき てき}に開かれる〕市、市場{しじょう}〈米〉〔農産物{のうさんぶつ}や商品{しょうひん}の〕見本市{みほんいち}、品評会{ひんぴょうかい}◆【同】〈英〉
\\	展示会{てんじかい}、博覧会{はくらんかい}慈善{じぜん}バザー、チャリティー販売会{はんばい かい}〈英〉〔巡回{じゅんかい}〕カーニバル、移動{いどう}アミューズメント・パーク◆【同】
\\	レベル2、発音
\\	、カナフェア、変化《形》
\\	、《動》
\\	【形】〔人が〕くつろいだ、ゆったりした、苦痛{くつう}がない〔物が〕くつろがせる、心地良い〔~に対して〕心配{しんぱい}していない、安心{あんしん}している〔大きさや量が〕十分{じゅうぶん}な、安心{あんしん}できるだけの〔収入{しゅうにゅう}があって〕ある程度裕福{ていど ゆうふく}な〈英〉〔病状{びょうじょう}が〕安定{あんてい}した、穏やかな容易{ようい}にこなせる、簡単{かんたん}にできるレベル2、発音
\\	、カナカムファタブル、カンファタブル、分節
\\	変化形 
\\	《複》
\\	【名】〔レースの〕基礎{きそ}、下地{したじ}【形】好きな、好んで優しい、愛情{あいじょう}のこもったでき愛のレベル3、発音
\\	、カナファンド、フォンド、変化《形》
\\	当社は、このサイトの内容について一切責任を負いません。	変化形 
\\	【1名】入っているもの、内容(物){ないよう(ぶつ)}、中身{なかみ}、在中物{ざいちゅう ぶつ} ・
\\	当社は、このサイトの内容について一切責任を負いません。容量{ようりょう}、含有量{がんゆうりょう}、容積{ようせき}、体積{たいせき}《コ》コンテンツ◆ウェブサイトなどで利用可能な情報レベル2、発音
\\	、変化《複》
\\	【2他動】~を満足{まんぞく}させる【2名】満足{まんぞく}【2形】〔~に〕満足{まんぞく}している発音
\\	、変化《動》
\\	、分節
\\	撃つのをやめてください。私たちは味方です。 ・
\\	私の犬はあまり人に懐きません。	変化形 
\\	《複》
\\	【名】〈話〉友好的{ゆうこう てき}な人〈英〉親善試合{しんぜん じあい}〈米俗〉〔特定{とくてい}の候補者{こうほしゃ}の〕支持{しじ}[支援{しえん}]者〔軍事行動{ぐんじこうどう}などにおける〕味方{みかた}[友軍{ゆうぐん}]の人[部隊{ぶたい}・乗り物]、味方{みかた}[友軍{ゆうぐん}]機 ・
\\	撃つのをやめてください。私たちは味方です。【形】〔態度{たいど}や言葉{ことば}などが〕友達{ともだち}にふさわしい、信頼{しんらい}できる〔人が〕友達{ともだち}のような、優しい、親切{しんせつ}な〔場所{ばしょ}や雰囲気{ふんいき}などが〕心地良い、ホッとできる味方{みかた}の、友軍{ゆうぐん}の〔機械{きかい}・ソフトウェアなどが〕使いやすい、扱いやすい、分かりやすい〔動物{どうぶつ}などが〕人懐{ひとなつ}こい ・
\\	私の犬はあまり人に懐きません。【副】友達{ともだち}のようにレベル1、発音
\\	、カナフレンドゥリ、フレンドリー、変化《形》
\\	、《複》
\\	、分節
\\	聞き覚えのある[どこかで聞いたことのある]声だと思いました。	変化形 
\\	《複》
\\	【名】〈文〉親友{しんゆう}、親しい仲間{なかま}[友人{ゆうじん}]精通{せいつう}している[詳しい]人〔ある物事{ものごと}に〕、よく出入りする[訪れる]人〔ある場所{ばしょ}に〕〔魔女{まじょ}などの〕使い魔◆【同】
\\	《キリスト教》〔教皇{きょうこう}や主教{しゅきょう}の〕用人{ようにん}◆身の回りの世話をする役目の人。【形】よく知っている、精通{せいつう}している、造詣{ぞうもう}が深い聞き覚えのある ・
\\	聞き覚えのある[どこかで聞いたことのある]声だと思いました。打ち解けた、なじんだ、親しみがある、親しい〔道具などが〕使いやすい〈まれ〉家族(内)の、家庭(内)のレベル2、発音
\\	、カナファミリオー、ファミリアー、分節
\\	変化形 
\\	【形】寂しい、ひとりの、孤独{こどく}の、孤立{こりつ}した、心細い{こころぼそい}、連れのいない
\\	孤独{こどく}に苛{さいな}まれている人々{ひとびと}レベル2、発音
\\	、カナロンリー、ロウンリ、変化《形》
\\	、分節
\\	静かにしなさい。◆【用法】教師が生徒に言う。 ・
\\	静かにしてください。 ・
\\	彼は無口なたちだ。	変化形 
\\	【自動】静まる、穏やかになる、おさまる、静かになる【他動】~を静かにさせる、静める〔気分{きぶん}などを〕和らげる、なだめる【名】静けさ、静寂{せいじゃく}、しじま静穏{せいおん}、平穏{へいおん}、無事{ぶじ}に過ごすこと〔心の〕穏やかさ、平安{へいあん}【形】静かな、静粛{せいしゅく}な ・
\\	静かにしなさい。◆【用法】教師が生徒に言う。 ・
\\	静かにしてください。無口{むくち}な、寡黙{かもく}な ・
\\	彼は無口なたちだ。物静{ものしず}かな、朴訥{ぼくとつ}としたひっそりとした目立たない、地味{じみ}な内密{ないみつ}の、内に秘めた平穏{へいおん}な、穏やかな、平和{へいわ}なレベル1、発音
\\	、カナクワイアット、クワイアト、変化《形》
\\	、《動》
\\	、分節
\\	【名】無声映画{むせい えいが}【形】静かな、音のしない、しんとした、ひっそりしている無言{むごん}の、黙りこくった、寡黙{かもく}な、口をつぐんで、沈黙{ちんもく}した、物を言わないご無沙汰{ぶさた}する《医》無症状{むしょうじょう}の、無症候性{むしょうこうせい}の、無痛性{むつう せい}のレベル1、発音
\\	、カナサイレント、分節
\\	私には申し分なく聞こえますが。 ・
\\	ひどいもんだ。◆嫌なことを聞いたときに ・
\\	それは素晴{すば}らしい。/それは結構です。 ・
\\	それは面白{おもしろ}そうですね。 ・
\\	それはちょっとおかしいね。 ・
\\	あまり確信がなさそうだね。 ・
\\	彼、機嫌悪そうな声だね。 ・
\\	健全な身体に健全な精神が宿る。 ・
\\	スポーツは体も心も強くする。	変化形 
\\	【1自動】音を出す[たてる]、鳴る、鳴動{めいどう}する〔人から聞いた情報が〕~に聞こえる、~に思われる◆【用法】
\\	形容詞」、または
\\	名詞(句)」 ・
\\	私には申し分なく聞こえますが。 ・
\\	ひどいもんだ。◆嫌なことを聞いたときに ・
\\	それは素晴{すば}らしい。/それは結構です。 ・
\\	それは面白{おもしろ}そうですね。 ・
\\	それはちょっとおかしいね。 ・
\\	あまり確信がなさそうだね。 ・
\\	彼、機嫌悪そうな声だね。【1他動】〔音を〕発する、鳴らす【1名】〔聞こえる〕音、物音{ものおと}、響き、声《物理》音波{おんぱ}◆可聴範囲の物体の振動。〔聴覚器官{ちょうかく きかん}の〕音感覚{おんかんかく}〔機械{きかい}で再生{さいせい}される〕音声{おんせい}、サウンド◆再生音の品質や音量に関して用いられる。〔映画{えいが}などの録音{ろくおん}された〕音声{おんせい}、サウンド◆放送や音楽業界で録音された素材を指す。〔最近見聞きした〕様子{ようす}、調子{ちょうし}、印象{いんしょう}〔意味{いみ}のない〕ただの音、騒音{そうおん}、雑音{ざつおん}〔音や声の〕聞こえる範囲{はんい}[距離{きょり}]《言語学》〔発声器官{はっせい きかん}による〕言語音{げんご おん}《言語学》〔ある特徴{とくちょう}を持つ〕音声{おんせい}《音楽》〔固有{こゆう}の特徴{とくちょう}を持つ〕音楽{おんがく}(のスタイル)、サウンド【2形】〔心や体が〕健康{けんこう}な、健全{けんぜん}な◆けがや病気などがない状態。 ・
\\	健全な身体に健全な精神が宿る。 ・
\\	スポーツは体も心も強くする。〔物が〕傷んで[腐って]いない〔構造{こうぞう}が〕堅固{けんご}な、安定{あんてい}した〔考えなどが〕理にかなった、正当{せいとう}な〔考えなどが〕穏健{おんけん}な、穏当{おんとう}な《論理学》〔論証{ろんしょう}が〕妥当{だとう}な◆すべての前提が真であり、そこから結論が導かれるもの。〔経済{けいざい}や投資{とうし}が〕堅実{けんじつ}な、安全{あんぜん}な〔批判{ひはん}などが〕徹底{てってい}した、完膚{かんぷ}なきまでの〔道徳的{どうとく てき}に〕正しい、高潔{こうけつ}な〔眠りが〕深い、十分{じゅうぶん}な《法律》合法{ごうほう}な、有効{ゆうこう}な【2副】〔睡眠などが〕ぐっすりと、十分{じゅうぶん}に【3名】海峡{かいきょう}、入り江、瀬戸{せと}【4自動】《航海》測る、測定{そくてい}◆水深{すいしん}を◆【参考】
\\	それとなく探りを入れる、当たってみる、打診{だしん}する〔クジラなどが〕急潜行{きゅう せんこう}する、素早く潜る【4他動】《航海》(水深{すいしん})を測る、測定{そくてい}する◆【参考】
\\	~を観測{かんそく}する、探る、調査{ちょうさ}する◆【参考】
\\	~にそれとなく探りを入れる、当たってみる、打診{だしん}する【4名】《医》探り針、探り、診断{しんだん}、消息子{しょうそくし}レベル1、発音
\\	、カナサウンド、変化《形》
\\	、《動》
\\	変化形 
\\	【名】〈英話〉苦味{にがみ}の少ない黒ビール【形】〔人や言動{げんどう}などが〕穏やかな、優しい〔気候が〕温和{おんわ}な、温暖{おんだん}な〔厳しさや勢いが〕弱い、軽度{けいど}の〔食べ物や味が〕刺激{しげき}の少ない、マイルドな〔ビールが〕苦みの少ない、ホップを効かせていない〔薬剤{やくざい}の副作用{ふくさよう}が〕軽度{けいど}のレベル2、発音
\\	、カナマイルド、変化《形》
\\	彼女の優しい声が興奮した動物をなだめた。	変化形 
\\	【他動】〈文〉(人)をなだめる、(人)を落ち着かせる〈文〉〔動物{どうぶつ}を〕なつかせる、飼いならす【形】〔性格が〕優しい、寛大な〔物腰や調子が〕優しい、穏やかな ・
\\	彼女の優しい声が興奮した動物をなだめた。〔坂などが〕なだらかな、急でない育ちが良い、立派{りっぱ}な家柄{いえがら}のレベル2、発音
\\	、カナジェントゥル、ジェントル、変化《形》
\\	、分節
\\	私、犬が怖いんです。 ・
\\	私は彼のことが前から好きなのですが、告白する勇気がありません。 ・
\\	残念ですが私はもうおいとましなくては。 ・
\\	接続状態が悪いようです。◆電話 ・
\\	電話番号をお間違いのようですが。	イディオムやフレーズ 
\\	【形】恐れて、心配{しんぱい}して、怖がって、おじけづいて、気遣って、ためらいがちな ・
\\	私、犬が怖いんです。 ・
\\	私は彼のことが前から好きなのですが、告白する勇気がありません。残念{ざんねん}ながら(~しなければならないと思う) ・
\\	残念ですが私はもうおいとましなくては。~ではないかと思う、~のようである ・
\\	接続状態が悪いようです。◆電話 ・
\\	電話番号をお間違いのようですが。レベル1、発音
\\	、カナアフレイド、分節
\\	何かの冗談じゃないの?/それ、本気で言っているの?		【形】〔態度{たいど}などが〕真面目{まじめ}な、厳粛{げんしゅく}な〔人や言葉{ことば}などが〕本気{ほんき}の、冗談{じょうだん}ではない ・
\\	何かの冗談じゃないの?/それ、本気で言っているの?〔行為{こうい}が〕真剣{しんけん}に取り組む、どっぶりとはまる〔関与{かんよ}が〕本格的{ほんかく てき}な、プロ並みの〔芸術{げいじゅつ}の分野{ぶんや}が〕シリアスな、純粋{じゅんすい}の〈話〉〔程度{ていど}が〕相当{そうとう}な、かなりの〔状況{じょうきょう}や病気{びょうき}などが〕重大{じゅうだい}な、重篤{じゅうとく}な〔問題{もんだい}などが〕容易{ようい}でない、手に負えないレベル2、発音
\\	、カナシリアス、スィアリアス、分節
\\	我慢してください。	変化形 
\\	《複》
\\	【名】〔治療を受けている〕病人{びょうにん}、患者{かんじゃ}【形】〔人が苦痛{くつう}や困難{こんなん}に〕我慢{がまん}する、忍耐強い{にんたいづよい} ・
\\	我慢してください。〔言動{げんどう}などが〕我慢強{がまん づよ}さを示す、忍耐強い{にんたいづよい}〔人や言動{げんどう}が〕寛大{かんだい}な、物分{ものわ}かりが良い〔行為{こうい}や組織{そしき}などが〕持続的{じぞく てき}な、不変{ふへん}のレベル2、発音
\\	、カナペイシャント、ペイシェント、変化《複》
\\	、分節
\\	言うことには気を付けなさい。/口は禍の元。 ・
\\	俺{おれ}に何か言うときは口の聞き方に気を付けろ! ・
\\	気を付けろ。わなかもしれん。		【形】〔人が〕注意深い{ちゅういぶかい}、気を付ける、用心深い{ようじんぶかい}、油断{ゆだん}しない、用心{ようじん}する、注意{ちゅうい}する ・
\\	言うことには気を付けなさい。/口は禍の元。 ・
\\	俺{おれ}に何か言うときは口の聞き方に気を付けろ! ・
\\	気を付けろ。わなかもしれん。〔行為などが〕きちょうめんな、念入り{ねんいり}な、入念{にゅうねん}な、丁寧{ていねい}な〈英〉〔お金などを〕無駄{むだ}に使わない、節約{せつやく}する〔人を〕気に掛ける、案じる〈古〉心配{しんぱい}した、不安{ふあん}なレベル1、発音
\\	、カナケアフル、分節
\\	変化形 
\\	【形】丁寧{ていねい}な、礼儀正{れいぎ ただ}しい上品{じょうひん}な、洗練{せんれん}されたレベル2、発音
\\	、カナポライト、変化《形》
\\	、分節
\\	《コ》ダイアログ・ボックスがアクティブです。 ・
\\	表示される操作は画面により異なる。	イディオムやフレーズ 
\\	【名】《言語学》能動態{のうどうたい}【形】〔物理的{ぶつり てき}な動きが〕活発{かっぱつ}な、素早い、機敏{きびん}な〔人や活動{かつどう}が〕盛んな、忙しい、活発{かっぱつ}な〔人が〕積極的{せっきょく てき}な、自ら働き掛ける、能動的{のうどう てき}な〔活動{かつどう}が〕エネルギーが必要{ひつよう}な、大変{たいへん}な、激しい参加{さんか}[関与{かんよ}・活動{かつどう}]している、現役{げんえき}の機能{きのう}[動作{どうさ}・進行{しんこう}]している《言語学》〔構文{こうぶん}が〕能動態{のうどうたい}の《言語学》〔動詞{どうし}が〕動作{どうさ}を表す
\\	や
\\	などの動詞を指す。◆【対】
\\	〔火山{かざん}が〕活動中{かつどう ちゅう}の◆【参考】
\\	〔太陽{たいよう}が〕活動期{かつどうき}の〔ビジネスが〕利益{りえき}が挙がる〔債券{さいけん}などが〕利付きの〔投資{とうし}が〕アクティブな、能動的{のうどう てき}な〔電子機器{でんし きき}が〕アクティブな、電源{でんげん}を備えた《コ》〔ウィンドウが〕アクティブな◆その時点で入力などの操作の対象となっているもの。◆【対】
\\	《コ》ダイアログ・ボックスがアクティブです。 ・
\\	表示される操作は画面により異なる。レベル2、発音
\\	、カナアクチブ、アクティブ、分節
\\	【名】《言語学》受動態{じゅどうたい}【形】受動的{じゅどうてき}な、受け身の抵抗{ていこう}しない、いいなりになる、従順{じゅうじゅん}な参加{さんか}しない、活動的{かつどう てき}でない、不活発{ふかっぱつ}な〔太陽熱利用{たいようねつ りよう}が〕パッシブな、受動的{じゅどう てき}な◆太陽熱を加工や変換をしないでそのまま利用すること。《言語学》〔構文{こうぶん}が〕受動態{じゅどうたい}の、受け身の〔投資{とうし}が〕パッシブの、受動的{じゅどう てき}な《化学》不活性{ふかっせい}の、不動態{ふどうたい}の《電気》〔電子機器{でんし きき}が〕パッシブな、電源{でんげん}を備えていないレベル4、発音
\\	、カナパシブ、パッスィブ、分節
\\	何が起きても責任は持てません。	イディオムやフレーズ 
\\	【形】〔人が行為などの〕責任{せきにん}がある[を負うべき] ・
\\	何が起きても責任は持てません。〔地位・仕事などが〕責任{せきにん}の重い、重要{じゅうよう}な〔人が〕信用{しんよう}のおける、信頼{しんらい}できる〔不都合{ふつごう}の〕原因{げんいん}である、張本人{ちょうほんにん}である〔人が自分{じぶん}の行為{こうい}の〕責任{せきにん}を負える、責任能力{せきにん のうりょく}がある〔しっかりした〕判断{はんだん}[思考{しこう}]に基づいた〔借金{しゃっきん}の〕支払能力{しはらい のうりょく}がある〔義務{ぎむ}の〕実行能力{じっこう のうりょく}があるレベル2、発音
\\	、カナリスパンスィブル、リスポシブル、分節
\\	本当に?/確かですか? ・
\\	確かではありません。 ・
\\	この映画には楽観的な場面がほとんどない。 ・
\\	今のところ好意的な回答はもらっていない。 ・
\\	上院議員は、選挙民から肯定的な反応が寄せられたことを喜んだ。	イディオムやフレーズ 
\\	【名】〈話〉肯定{こうてい}[賛成{さんせい}・支持{しじ}]すること[もの]《数学》正数{せいすう}《文法》原級{げんきゅう}《写真》ポジ《電気》陽[正]電極{でんきょく}《音楽》小型{こがた}オルガン◆【同】
\\	【形】確かな、疑いようがない ・
\\	本当に?/確かですか? ・
\\	確かではありません。前向き{まえむき}な、楽天的{らくてん てき}な、建設的{けんせつ てき}な ・
\\	この映画には楽観的な場面がほとんどない。有益{ゆうえき}な、ためになる肯定的{こうてい てき}な、賛成{さんせい}の、好意的{こうい てき}な ・
\\	今のところ好意的な回答はもらっていない。 ・
\\	上院議員は、選挙民から肯定的な反応が寄せられたことを喜んだ。明確{めいかく}な、はっきりした、明白{めいはく}に述べられた〔理論{りろん}ではなく〕実際的{じっさい てき}な、実務上{じつむ じょう}の実在{じつざい}の、現実{げんじつ}の〈話〉全くの、絶対{ぜったい}の、本当{ほんとう}の、素晴{すば}らしい《医》〔病気{びょうき}に対して〕陽性{ようせい}の《医》〔血液{けつえき}が〕
\\	陽性{ようせい}の◆【同】
\\	《哲学》実証主義{じっしょう しゅぎ}の〔法律{ほうりつ}が〕実定の、人定の《法律》反論{はんろん}の余地{よち}がない《数学》〔数などが〕正の、プラスの《数学》正[プラス]記号{きごう}の《電気》〔電荷{でんか}が〕正の、陽の《電気》〔電位{でんい}が〕高い《生物》〔刺激{しげき}に対する屈性{くっせい}が〕正の〔写真{しゃしん}が〕ポジの、陽画{ようが}の《言語学》〔比較{ひかく}が〕原級{げんきゅう}の《機械》確動{かくどう}の◆中間の部品を少なくして動力を確実に伝えること。◆【参考】
\\	〔レンズが〕凸の〔占星術{せんせいじゅつ}の宮が〕陽性{ようせい}の◆火と空気に関連する宮(サイン)を指す。レベル2、発音
\\	、カナパズィティブ、ポジティブ、分節
\\	通例
\\	は否定節の後ろに来る。 ・
\\	その法案に反対票を投じた。 ・
\\	会議であったその男に良い印象は持たなかった。 ・
\\	電子の電荷は負です。	変化形 
\\	《複》
\\	【他動】〈文〉〔~の受け入れを〕拒否{きょひ}する、拒絶{きょぜつ}する《論理学》~の誤りを証明{しょうめい}する【名】〔写真{しゃしん}の〕ネガ、陰画{いんが}〔返答{へんとう}の〕否定{ひてい}、拒否{きょひ}◆【参考】
\\	《言語学》否定語{ひてい ご}《論理学》否定命題{ひてい めいだい}《電気》負の電極{でんきょく}、陰極版{いんきょく ばん}〈話〉欠点{けってん}、難点{なんてん}《数学》負数{ふすう}【形】〔表現{ひょうげん}などが〕否定{ひてい}の、否定{ひてい}の意味{いみ}を持つ ・
\\	通例
\\	は否定節の後ろに来る。拒否{きょひ}の、反対{はんたい}の ・
\\	その法案に反対票を投じた。悪い、嫌な、良くない ・
\\	会議であったその男に良い印象は持たなかった。後ろ向きの、否定的{ひてい てき}な、悲観的{ひかん てき}な《医》〔病原菌{びょうげんきん}などが〕陰性{いんせい}の《医》
\\	陰性{いんせい}の◆【同】
\\	《数学》〔数量{すうりょう}が〕負の、マイナスの《数学》引き算の、減算{げんざん}の《数学》〔速度{そくど}や角度{かくど}などが〕負の《物理》〔電荷{でんか}が〕負の、マイナスの ・
\\	電子の電荷は負です。《物理》〔電流{でんりゅう}の方向{ほうこう}が〕負の《化学》〔イオンが〕陰性{いんせい}の、負の、マイナスの《生物》〔向性{こうせい}や屈性{くっせい}が〕負の◆刺激とは逆の方向に移動または成長すること。《論理学》〔命題{めいだい}が〕否定{ひてい}のレベル3、発音
\\	、カナネガティブ、ネガテブ、変化《複》
\\	、分節
\\	【形】攻撃的{こうげきてき}な、挑戦的{ちょうせんてき}な、好戦的{こうせんてき}な、すぐに人につっかかるような、侵略的{しんりゃく てき}な、強引{ごういん}な◆否定的意味活動的{かつどうてき}な、積極果敢{せっきょく かかん}な、積極的{せっきょく てき}な、迫力{はくりょく}たっぷりの、攻めの◆肯定的意味〔病気{びょうき}が〕侵攻性{しんこうせい}の、悪性{あくせい}のレベル5、発音
\\	、カナアグレシブ、アグレッシブ、分節
\\	変だね。/変だなあ。 ・
\\	今朝の彼女はどこか変です。 ・
\\	それはおかしいです。◆苦情	変化形 
\\	【形】奇妙{きみょう}な、変わった、変な、見知らぬ{みしらぬ}、面妖{めんよう}な ・
\\	変だね。/変だなあ。 ・
\\	今朝の彼女はどこか変です。 ・
\\	それはおかしいです。◆苦情未知{みち}の不慣れ{ふなれ}な、未熟{みじゅく}なレベル1、発音
\\	、カナストゥレインジ、ストレンジ、変化《形》
\\	変化形 
\\	《複》
\\	【名】はんぱ物、奇妙{きみょう}なこと、風変わりなこと余り奇数{きすう}◆【同】
\\	【形】普通{ふつう}でない、奇妙{きみょう}な、変な、おかしな、異様{いよう}な奇数{きすう}の残りの、余りの、時々{ときどき}の、半端{はんぱ}な、片方{かたほう}の、妙な臨時{りんじ}のレベル3、発音
\\	、カナアッド、オッド、変化《形》
\\	、《複》
\\	ひどいことを言いますね。		【形】極度{きょくど}の、極端{きょくたん}なひどく嫌な、とても不快{ふかい}な ・
\\	ひどいことを言いますね。〔品質{ひんしつ}や内容{ないよう}が〕ひどく悪い、ぶざまな〔体調{たいちょう}が〕優れない、〔顔色{かおいろ}が〕悪い恐ろしい、怖いレベル2、発音
\\	、カナテリブル、分節
\\	「今日のサッカーの試合に来てたファンの人たち、かなり荒々しかったな」「本当だよ。あいつら、だいぶ羽目外してたな」 ・
\\	「あのね、知ってる?ステーキを刻んでチャーハンの中に入れたんだ」「へえ、面白{おもしろ}そうだね。どうだった?」
\\	変化形 
\\	《複》
\\	"【自動】〈米俗〉羽目{はめ}を外す、ばかな行動{こうどう}をとる◆【文法】常に現在進行形で用いられ、
\\	'とつづられることが多い。 ・
\\	「今日のサッカーの試合に来てたファンの人たち、かなり荒々しかったな」「本当だよ。あいつら、だいぶ羽目外してたな」【名】荒れ地、荒野、不毛地帯【形】野生{やせい}の、野生種{やせいしゅ}の、自然{しぜん}に生えている荒れ果てた、荒涼{こうりょう}とした気の荒い、荒っぽい、野蛮{やばん}な乱れた、狂気{きょうき}じみた、荒れ狂った、とっぴな、大それたひどく興奮{こうふん}して、夢中{むちゅう}になって、激しい面白{おもしろ}い
\\	に近いが、何か大胆なものに対して使われることが多い表現。 ・
\\	「あのね、知ってる?ステーキを刻んでチャーハンの中に入れたんだ」「へえ、面白{おもしろ}そうだね。どうだった?」【副】〔野原などが〕野生状態{やせい じょうたい}で〔植物の成長などが〕自然{しぜん}の状態{じょうたい}ででたらめに、乱暴{らんぼう}にレベル1、発音
\\	、カナワイルド、変化《形》
\\	、《複》
\\	彼は安全なドライバーです。 ・
\\	ご家族や友人の方々が全員ご無事であるよう願って[祈って]います。 ・
\\	この辺は近頃物騒でしてね。 ・
\\	用心するに越したことはない。 ・
\\	「20ポンド貸してくれない?」「いいよ」「助かった!」 ・
\\	彼はあなたが好きなのだと考えて、まず間違いありません。 ・
\\	無事に帰宅してください。/お気を付けてお帰りください。
\\	変化形 
\\	"【名】金庫{きんこ}食料品{しょくりょうひん}を貯蔵{ちょぞう}するための戸棚{とだな}〈俗〉コンドーム【形】安全{あんぜん}な、無事{ぶじ}な、安泰{あんたい}で、別状{べつじょう}がない、無難{ぶなん}な ・
\\	彼は安全なドライバーです。 ・
\\	ご家族や友人の方々が全員ご無事であるよう願って[祈って]います。 ・
\\	この辺は近頃物騒でしてね。 ・
\\	用心するに越したことはない。〈英俗〉素晴{すば}らしい、ありがたい ・
\\	「20ポンド貸してくれない?」「いいよ」「助かった!」まず間違いない、ほぼ確実{かくじつ}で ・
\\	彼はあなたが好きなのだと考えて、まず間違いありません。【副】安全{あんぜん}に、無事{ぶじ}に、別状{べつじょう}なく ・
\\	無事に帰宅してください。/お気を付けてお帰りください。レベル1、発音
\\	、カナセイフ、変化《複》
\\	、《形》
\\	できないことはない。 ・
\\	残念ながら[申し訳なく存じますが]それはちょっと無理なようです。 ・
\\	それは不可能[無理]です。 ・
\\	よろしいでしょうか?◆【場面】仕事を依頼するときなど。	変化形 
\\	《複》
\\	【名】
\\	可能{かのう}なもの[こと]、可能性{かのうせい}有力{ゆうりょく}な候補{こうほ}[容疑者{ようぎしゃ}]全力{ぜんりょく}、最善{さいぜん}【形】可能性{かのうせい}がある、起こり得る、あり得る、なし得る、実行{じっこう}できる、できるだけの ・
\\	できないことはない。 ・
\\	残念ながら[申し訳なく存じますが]それはちょっと無理なようです。 ・
\\	それは不可能[無理]です。 ・
\\	よろしいでしょうか?◆【場面】仕事を依頼するときなど。見込まれる、考えられる、予定{よてい}される、予定{よてい}の危険性{きけん せい}があるレベル2、発音
\\	、カナパッセボー、パスィブル、ポシブル、変化《複》
\\	、分節
\\	残念なことに彼には成功の見込みがなかったので、私たちは彼を解雇しました。 ・
\\	彼は、息子が自分で思っている以上に可能性を持っていることを知っていました。		【名】〔本来備わっている〕潜在力{せんざいりょく}、将来性{しょうらい せい} ・
\\	残念なことに彼には成功の見込みがなかったので、私たちは彼を解雇しました。 ・
\\	彼は、息子が自分で思っている以上に可能性を持っていることを知っていました。〔あることが起きる〕可能性{かのう せい}、見込み{みこみ}将来性{しょうらい せい}[潜在力{せんざいりょく}]がある人[もの]《言語学》〔動詞句{どうしく}の〕可能形{かのう けい}《電気》電位{でんい}◆【同】
\\	《物理》ポテンシャル《囲碁》あじ、勢力{せいりょく}【形】可能性{かのう せい}[見込み・将来性{しょうらい せい}]がある潜在的{せんざい てき}な、潜在力{せんざいりょく}がある《言語学》〔動詞句{どうしく}が〕可能{かのう}の、可能{かのう}を表す
\\	や
\\	などを含む場合を指す。《電気》電位{でんい}の《物理》位置{いち}のレベル4、発音
\\	、カナポテンシャル、変化《複》
\\	、分節
\\	今は無理だとしても、絶対に不可能だとはいえない。 ・
\\	診断が遅れた場合は死亡する可能性が高い。	変化形 
\\	【形】
\\	~しそうである、起こり[あり]得る◆【用法】「~する」は後に続く
\\	句や
\\	節で表す。 ・
\\	今は無理だとしても、絶対に不可能だとはいえない。 ・
\\	診断が遅れた場合は死亡する可能性が高い。もっともらしい、ありそうなふさわしい、適したうまくいきそうな、期待{きたい}できる〈方〉魅力的{みりょく てき}な、好ましい【副】たぶん、恐らく◆【用法】
\\	や
\\	と共に使われることが多い。単独で用いられる例も増えているが、正式な文書では避けるべきであると考える人が多い。レベル2、発音
\\	、カナライクリー、変化《形》
\\	、分節
\\	イディオムやフレーズ 
\\	【もっとイディオムを見る】
\\	【代名】〈文〉いくらか、いくつか【形】〔人が〕~を確信{かくしん}している、~に疑いを持たない◆【用法】叙述用法で、後ろに
\\	句や
\\	節などが来る。〔人が〕必ず[きっと]~する◆【用法】叙述用法で、後ろに
\\	が来る。〔出来事{できごと}が〕必ず起きる、避けられない〔物事{ものごと}が〕明白{めいはく}な、疑う余地{よち}がない〔方法{ほうほう}などが〕信頼{しんらい}できる、当てになる〔数量{すうりょう}や程度{ていど}などが〕わずかの、若干{じゃっかん}の◆【用法】限定用法〔特定{とくてい}できるが名前{なまえ}が分からない〕ある、例の◆【用法】限定用法〔名前{なまえ}は分かるが初めて言及{げんきゅう}する〕とかいう◆【用法】限定用法で、後ろに人・物・場所などの名前が来る。レベル1、発音
\\	、カナサーテン、サートゥン、分節
\\	【名】実費用{じつ ひよう}
\\	《金融》現物{げんぶつ}(商品{しょうひん})◆【対】
\\	現実{げんじつ}、実際{じっさい}【形】〔可能性{かのう せい}ではなく〕実在{じつざい}の、現実{げんじつ}の〔現時点{げんじてん}で〕存在{そんざい}している、行われている事実{じじつ}に基づいた、実際{じっさい}のレベル2、発音
\\	、カナアクチャル、アクチュアル、分節
\\	すぐにその国の言語を第一言語とする人と同じくらいに話せるでしょう。	変化形 
\\	《複》
\\	【名】先住民{せんじゅうみん}、土地{とち}の人、~出身{しゅっしん}の人その国の言語{げんご}を第一言語{だいいち げんご}とする人 ・
\\	すぐにその国の言語を第一言語とする人と同じくらいに話せるでしょう。〔ある地域{ちいき}に〕固有{こゆう}の動植物{どうしょくぶつ}【形】生まれた、出身{しゅっしん}の天然{てんねん}の、自然{しぜん}の、野生型{やせいがた}の、土地{とち}の、土着{どちゃく}の自分{じぶん}の国の、故郷{こきょう}の生まれつきの、生来{せいらい}の、天賦{てんぷ}の母語{ぼご}のレベル2、発音
\\	、カナネイティブ、変化《複》
\\	、分節
\\	【形】都会{とかい}の、都市{とし}に住む都会風{とかい ふう}の、都会派{とかい は}のレベル5、発音
\\	、カナアーバン、分節
\\	【形】田舎{いなか}の、地方{ちほう}の田舎{いなか}じみた、田舎風{いなか ふう}の田舎{いなか}に住む、農業{のうぎょう}を営む農業{のうぎょう}の、小作{こさく}のレベル4、発音
\\	、カナルーラル、ルアラル、分節
\\	リモコン貸して。 ・
\\	彼のよそよそしい態度から、彼女は、彼は何か悩んでいるのだろうかと不思議に思った。 ・
\\	その可能性はほとんどありません。 ・
\\	私はそれについてこれっぽっちも興味を持っていません。	変化形 
\\	【名】〔現場{げんば}からの〕中継放送{ちゅうけい ほうそう}◆【同】
\\	〈英〉
\\	〈話〉〔家庭電気機器の〕リモコン ・
\\	リモコン貸して。《コ》リモート[遠隔{えんかく}]コンピューター[デバイス]【形】〔距離的に〕遠い、遠く離れた、遠隔{えんかく}の、人里離れた、辺ぴな《コ》通信回線を介して利用可能な〔時間的に〕遠い、ずっと以前{いぜん}の〔人の気持ちが遠く離れて〕よそよそしい ・
\\	彼のよそよそしい態度から、彼女は、彼は何か悩んでいるのだろうかと不思議に思った。ごくわずかの、〔可能性{かのう せい}などが〕ありそうにない ・
\\	その可能性はほとんどありません。 ・
\\	私はそれについてこれっぽっちも興味を持っていません。レベル4、発音
\\	、カナリモート、リモウト、変化《複》
\\	、《形》
\\	、分節
\\	水くさいぞ。 ・
\\	北イタリアでの地震の恐れは現実的でないように見えた。		【形】遠い、離れた、距離{きょり}がある時間{じかん}が隔たった、昔の敬遠{けいえん}した、他人行儀{たにん ぎょうぎ}な、付き合いの悪い、よそよそしい、冷ややかな ・
\\	水くさいぞ。遠くを見るような、ぼんやりした、かすかな現実的{げんじつてき}でない、可能性{かのうせい}が低い ・
\\	北イタリアでの地震の恐れは現実的でないように見えた。《医》遠位{えんい}のレベル2、発音
\\	、カナディスタント、デスタント、分節
\\	【形】〔場所{ばしょ}が〕外国{がいこく}にある、在外{ざいがい}の〔ものや性質{せいしつ}などが〕海外生まれの、外国風{がいこく ふう}の〔自国{じこく}ではなく〕対外{たいがい}の、外交{がいこう}の、外国{がいこく}との〔体などに侵入{しんにゅう}したものが〕異物{いぶつ}の、異質{いしつ}な〔性質{せいしつ}などが〕合わない、なじまない無関係{むかんけい}な、無縁{むえん}な《法律》管轄区域外{かんかつ くいき がい}のレベル2、発音
\\	、カナフォーリン、分節
\\	変化形 
\\	《複》
\\	【名】同国人{どうこくじん}、(内)国民{こくみん}【形】国の、国家{こっか}の、国家的{こっかてき}な国民{こくみん}の、国民的{こくみんてき}な全国的{ぜんこくてき}な国立{こくりつ}の、国営{こくえい}の愛国的{あいこくてき}な、国家主義的{こっか しゅぎ てき}なレベル2、発音
\\	、カナナショナル、変化《複》
\\	、分節
\\	変化形 
\\	《複》
\\	【名】家事奉公人{かじ ほうこう にん}、〔家事をする〕メイド〔輸入品ではなく〕国産品{こくさんひん}家庭用{かてい よう}リンネル製品{せいひん}◆通例
\\	〔航空路線の〕国内線{こくないせん}家庭内{かてい ない}での口論{こうろん}[暴力{ぼうりょく}]【形】家庭(内){かてい(ない)}の〔人が〕家庭的な〔動物が〕人になれた自国{じこく}の、国内{こくない}の、国産{こくさん}の、国内向けの◆【反】
\\	レベル3、発音
\\	、カナドメスチック、ドメスティック、変化《複》
\\	、分節
\\	【形】政治{せいじ}の[に関する・に携わる]、政治上{せいじ じょう}の政党{せいとう}の[に関する]国家{こっか}の[に関する]、国政{こくせい}の[に関する]、行政{ぎょうせい}の[に関する]策略的{さくりゃく てき}な、抜け目のない、賢明{けんめい}な、思慮{しりょ}のある、分別{ふんべつ}のあるレベル3、発音
\\	、カナポリティカル、分節
\\	【形】金銭上{きんせん じょう}の、財務{ざいむ}の、財政(上){ざいせい(じょう)}の、金融(上){きんゆう(じょう)}の金融関係者{きんゆう かんけいしゃ}の、財界人{ざいかいじん}のレベル3、発音
\\	、カナファイナンシャル、フィナンシャル、分節
\\	【名】法律{ほうりつ}を守る人
\\	〈米〉法定投資{ほうてい とうし}◆貯蓄銀行や保険会社などが、合法的に信託資金を運用すること。【形】法律(上){ほうりつ(じょう)}の弁護士{べんごし}の、法律家{ほうりつか}の法定{ほうてい}の、法の下の合法{ごうほう}の、適法{てきほう}の〔衡平法{こうへいほう}ではなく〕普通法{ふつうほう}による〈話〉〔年齢{ねんれい}が〕法定{ほうてい}に達したレベル3、発音
\\	、カナリーガル、分節
\\	公式[正式]には何も述べられていない。	変化形 
\\	《複》
\\	【名】公務員{こうむいん}、役人{やくにん}〔組織・会社などの〕職員{しょくいん}〔政府機関などの〕当局者{とうきょくしゃ}【形】当局{とうきょく}の、公職{こうしょく}の、公務{こうむ}の〔当局{とうきょく}が〕公認{こうにん}した、正式{せいしき}[公式{こうしき}]なものとした ・
\\	公式[正式]には何も述べられていない。〔行事{ぎょうじ}などが〕公式{こうしき}な、正式{せいしき}な、形式張った〔薬が〕薬局方{やっきょくほう}に収載{しゅうさい}されている[準拠{じゅんきょ}した]レベル2、発音
\\	、カナオフィシャル、変化《複》
\\	、分節
\\	【形】形式張らない、堅苦{かたぐる}しくない非公式{ひ こうしき}の、私的{してき}な日常{にちじょう}の、普段{ふだん}の《言語学》〔表現{ひょうげん}が〕口語体{こうごたい}の、砕けたレベル3、発音
\\	、カナインフォーマル、分節
\\	参加者は本物の軍服を着た。 ・
\\	その軍事作戦は実行されなかった。		【名】軍隊{ぐんたい}、軍部{ぐんぶ}〔集合的{しゅうごう てき}に〕軍人{ぐんじん}、将校{しょうこう}【形】軍人{ぐんじん}の、兵士{へいし}の◆【略】
\\	参加者は本物の軍服を着た。軍の、軍隊{ぐんたい}の、軍部{ぐんぶ}の◆海空軍ではなく陸軍を指すことが多い。◆【略】
\\	戦争{せんそう}の、軍事的{ぐんじ てき}な◆【略】
\\	その軍事作戦は実行されなかった。レベル3、発音
\\	、カナミリタリー、ミリテリ、分節
\\	【形】《物理》原子核{げんしかく}の《物理》原子力{げんしりょく}の、原子炉{げんしろ}の《軍事》核兵器{かく へいき}の、核爆弾{かく ばくだん}の《生物》細胞核{さいぼうかく}のレベル4、発音
\\	、カナニュークリア、ニュークレア、ヌークリア、ヌークレア、変化《複》
\\	、分節
\\	変化形 
\\	《複》
\\	【名】化学{かがく}薬品{やくひん}[物質{ぶっしつ}・製品{せいひん}]〈俗〉麻薬{まやく}◆通例
\\	【形】化学{かがく}の、化学作用{かがく さよう}による、化学的{かがくてき}な〔化学作用のように〕異な、妙なレベル3、発音
\\	、カナケミカル、変化《複》
\\	、分節
\\	変化形 
\\	《複》
\\	【名】〔組織{そしき}の〕長、局長{きょくちょう}、長官{ちょうかん}、所長{しょちょう}◆ある組織で最高の権限を持つもの。〔人類学上{じんるいがく じょう}の〕族長{ぞくちょう}、首長{しゅちょう}〈話〉上司{じょうし}、上役{うわやく}、ボス〈話〉《軍事》=
\\	《海事》機関長{きかん ちょう}◆【同】
\\	〔紋章{もんしょう}の〕チーフ◆盾の上3分の1の部分。【形】〔地位{ちい}や権限{けんげん}が〕最高位{さいこうい}の最重要{さいじゅうよう}の、根本的{こんぽん てき}なレベル1、発音
\\	、カナチーフ、変化《複》
\\	変化形 
\\	《複》
\\	【1名】首都{しゅと}元金{がんきん/もときん}、資源{しげん}、資本(金){しほん(きん)}大文字{おおもじ}【1形】主要{しゅよう}な、重要{じゅうよう}な資本{しほん}の大文字{おおもじ}の死刑{しけい}に値する〈俗〉素晴{すば}らしい、優れている◆【同】
\\	【2名】《建築》柱頭{ちゅうとう}レベル2、発音
\\	、カナカピタル、キャピタル、変化《複》
\\	、分節
\\	専攻(科目)は何ですか? ・
\\	大きな手術をしたことはありますか? ・
\\	私はひどくがっかりしました。 ・
\\	これは、すごく面白いです。	変化形 
\\	【自動】専攻{せんこう}する【名】〔米陸軍{べい りくぐん}・空軍{くうぐん}・海兵隊{かいへいたい}の〕少佐{しょうさ}◆中佐
\\	より下で大尉
\\	より上の地位またはその地位の人。《法律》成人{せいじん}〔大学の〕専攻{せんこう}(科目{かもく}) ・
\\	専攻(科目)は何ですか?〔~の〕専攻学生{せんこう がくせい}《論理学》大前提{だいぜんてい}、大名辞{だいめいじ}《音楽》長音程{ちょうおんてい}、長音階{ちょうおんかい}、長調{ちょうちょう}
\\	〔スポーツの〕大[メジャー・]リーグ
\\	〔ある分野{ぶんや}の企業{きぎょう}の〕大手資本{おおて しほん}、メジャー【形】〔大きさや量がより〕大きい、多い〔重要性{じゅうよう せい}や地位{ちい}が〕重要{じゅうよう}な、主要{しゅよう}な、一流{いちりゅう}の〔範囲{はんい}や効果{こうか}がより〕広範{こうはん}な、大幅{おおはば}な、大規模{だいきぼ}な〔危険{きけん}や病気{びょうき}などが〕深刻{しんこく}な、命に関わる◆【反】
\\	大きな手術をしたことはありますか?《法律》成人{せいじん}の〔大学{だいがく}の〕(主)専攻{(しゅ)せんこう}の《音楽》長音程{ちょうおんてい}の、長音階{ちょうおんかい}の、長調{ちょうちょう}の【副】〈俗〉すごく、めちゃ
\\	の意味の口頭表現。 ・
\\	私はひどくがっかりしました。 ・
\\	これは、すごく面白いです。レベル3、発音
\\	、カナメイジャー、メジャ、変化《動》
\\	、分節
\\	副専攻(科目)は何ですか? ・
\\	そんなことは大した問題じゃない。	変化形 
\\	《複》
\\	【自動】副専攻科目{ふく せんこう かもく}に選ぶ◆【参考】
\\	【名】《法律》未成年(者){みせいねん(しゃ)}◆米国では21歳または18歳未満(州によって異なる)、英国では18歳未満二流{にりゅう}の物[人]〔大学の〕副専攻{ふく せんこう}(科目{かもく}) ・
\\	副専攻(科目)は何ですか?〔~の〕副専攻学生{ふく せんこう がくせい}《数学》小行列式【形】〔大きさや量などがより〕小さい、少ない〔重要性{じゅうよう せい}や地位{ちい}が〕二流{にりゅう}の、大したことのない ・
\\	そんなことは大した問題じゃない。〔範囲{はんい}や効果{こうか}がより〕小規模{しょうきぼ}な、小さな〔危険{きけん}や病気{びょうき}などが〕深刻{しんこく}でない、命に関わらない《法律》未成年{みせいねん}の〈英〉〔学校{がっこう}で同じ姓を持つ〕年下{としした}の、年少{ねんしょう}の《音楽》短音程{たんおんてい}の、短音階{たんおんかい}の、短調{たんちょう}の〔大学{だいがく}の〕副専攻{ふく せんこう}のレベル3、発音
\\	、カナマイナー、マイナス、変化《動》
\\	、分節
\\	それは私にぴったりだ。 ・
\\	服がきつくなった。/服が合わなくなった。 ・
\\	鍵が錠に合わない。	変化形 
\\	【1自動】〔形や大きさが〕合う、適合{てきごう}する〔物事{ものごと}が〕適している、適切{てきせつ}である〔物事{ものごと}が〕調和{ちょうわ}する、一致{いっち}する【1他動】〔形や大きさなどが~に〕合う、適合{てきごう}する、フィットする ・
\\	それは私にぴったりだ。 ・
\\	服がきつくなった。/服が合わなくなった。 ・
\\	鍵が錠に合わない。〔~の形や大きさを変えて〕合わせる、適合{てきごう}させる〔人の〕寸法{すんぽう}を測る、〔人に〕試着{しちゃく}させる〔物事{ものごと}が~に〕ふさわしい、適切{てきせつ}である〔物事{ものごと}が~に〕調和{ちょうわ}する、一致{いっち}する〔人に仕事{しごと}などの〕準備{じゅんび}[用意{ようい}]をさせる〔装備{そうび}を~に〕取り付ける、〔人に〕身支度{みじたく}させる〔~を場所{ばしょ}や時間{じかん}に〕納める、入れる【1名】〔物事{ものごと}の〕適合{てきごう}、一致{いっち}、調和{ちょうわ}〔部品などの〕かみ合い、はめ合い◆定められた位置にぴったりと納められているかどうかの度合い。〔部品{ぶひん}の〕適合度{てきごうど}◆隣り合う部品の表面同士の、密着の度合い。〔衣服{いふく}が体に〕合って[ぴったりして]いること《数学》適合{てきごう}【1形】〔物事{ものごと}が〕合う、適合{てきごう}した、適切{てきせつ}な〔人が〕値する、能力{のうりょく}がある、ふさわしい〔体が〕健康{けんこう}な、元気{げんき}な、壮健{そうけん}な〈話〉今にも~しそうな、~する準備{じゅんび}ができている〈主に英話〉〔人が〕とても魅力的{みりょくてき}な、ハンサムな【2名】《医》〔てんかんによる〕引き付け、発作{ほっさ}《医》〔咳などの〕発作{ほっさ}〔感情{かんじょう}の突然{とつぜん}の〕爆発{ばくはつ}、噴出{ふんしゅつ}、発作{ほっさ}〔突然{とつぜん}の〕激しい[力強い]行動{こうどう}[動き]レベル2、発音
\\	、カナフィット、変化《形》
\\	、《動》
\\	日本の若者の多くはきちんとした日本語が話せない。 ・
\\	時間は各観測者に固有である。◆相対性理論における固有時の考え方 ・
\\	当然と言えば当然だ。		【形】適した、適切{てきせつ}な正確{せいかく}な、きちんとした、厳密{げんみつ}な(意味{いみ}での)、正式{せいしき}の、正常{せいじょう}の ・
\\	日本の若者の多くはきちんとした日本語が話せない。妥当な、適当{てきとう}な、ふさわしい、相応の特有{とくゆう}の、固有{こゆう}の、独特{どくとく}の ・
\\	時間は各観測者に固有である。◆相対性理論における固有時の考え方全くの、当然{とうぜん}の、まっとうな ・
\\	当然と言えば当然だ。礼儀正しい、上品{じょうひん}な、堅苦{かたぐる}しい、気取った《数学》真のレベル2、発音
\\	、カナプラパ、プロパー、分節
\\	教室の中では適切な言葉を使うべきです。 ・
\\	この集まりでは、明確で適切な言葉遣いが望ましいです。 ・
\\	ほめられたときに返すべき適切な言葉は「ありがとう」です。 ・
\\	(人)からお世辞を言われたときの適切な返答方法の一つとしては、一言「ありがとう」と言うことです。
\\	変化形 
\\	"【他動】〔特定の目的などに〕~を充てる[充当{じゅうとう}する・割り当てる・使用{しよう}する]〔無断{むだん}で金などを〕私物化{しぶつ か}する、独り占めする、着服{ちゃくふく}する【形】〔特定の人・条件・場所などに〕適した、適切{てきせつ}な、適当{てきとう}な、妥当{だとう}な、ふさわしい ・
\\	教室の中では適切な言葉を使うべきです。 ・
\\	この集まりでは、明確で適切な言葉遣いが望ましいです。 ・
\\	ほめられたときに返すべき適切な言葉は「ありがとう」です。 ・
\\	(人)からお世辞を言われたときの適切な返答方法の一つとしては、一言「ありがとう」と言うことです。《言語学》〔発話{はつわ}が〕適切{てきせつ}な◆文法的に正しく
\\	ても、適切でない発話が存在し、母語の話者は文法性と適切性の両方に関する知識を持っていると考えられている。レベル4、発音《形》
\\	《動》
\\	、カナアプロウプリエート、アプロウプリエット、アプロプリエート、アプロプリエット、変化《形》
\\	、《動》
\\	、分節
\\	もう十分に飲んだでしょ。 ・
\\	もうたくさんだ。 ・
\\	おしゃべりはこのくらいにしておきましょう。 ・
\\	満腹はごちそうも同様。/物事はほどほどにせよ。◆ことわざ ・
\\	これで足りるはずだ。◆お金を差し出すときなど。 ・
\\	いつも時間が足りない。 ・
\\	感謝の気持ちは言葉では言い表せない。◆【直訳】言葉は十分ではない。 ・
\\	30分あれば十分ですか? ・
\\	こんなもんで十分です。 ・
\\	もう十分です。/もうたくさんだ。/もう結構。/やめなさい。 ・
\\	野球は飽きた。サッカーをしようよ。	イディオムやフレーズ 
\\	【もっとイディオムを見る】
\\	【代名】十分{じゅうぶん}な量、十分{じゅうぶん}な数、たくさん ・
\\	もう十分に飲んだでしょ。 ・
\\	もうたくさんだ。【形】十分{じゅうぶん}な、足りる ・
\\	おしゃべりはこのくらいにしておきましょう。 ・
\\	満腹はごちそうも同様。/物事はほどほどにせよ。◆ことわざ ・
\\	これで足りるはずだ。◆お金を差し出すときなど。 ・
\\	いつも時間が足りない。 ・
\\	感謝の気持ちは言葉では言い表せない。◆【直訳】言葉は十分ではない。 ・
\\	30分あれば十分ですか?もう十分{じゅうぶん}な、飽き飽きする量の ・
\\	こんなもんで十分です。 ・
\\	もう十分です。/もうたくさんだ。/もう結構。/やめなさい。 ・
\\	野球は飽きた。サッカーをしようよ。【副】十分{じゅうぶん}に全くまあまあ、まずまずレベル1、発音
\\	、カナイナフ、分節
\\	《コ》ファイルの最後まで検索しました。		【形】全体{ぜんたい}の、全部{ぜんぶ}の、全部そろっている ・
\\	《コ》ファイルの最後まで検索しました。全くの、完全{かんぜん}なレベル3、発音
\\	、カナインタイア、エンタイア、分節
\\	変化形 
\\	【形】〔大きさや量が〕巨大{きょだい}な、大量{たいりょう}の〔範囲{はんい}や程度{ていど}が〕大規模{だいきぼ}な、莫大{ばくだい}な〈話〉〔大きな〕成功{せいこう}を収めた、影響力{えいきょうりょく}を持つレベル3、発音
\\	、カナヒュージ、変化《形》
\\	変化形 
\\	【名】〈話〉ちっちゃな子ども【形】とても小さい、ちっぽけなレベル2、発音
\\	、カナタイニー、変化《形》
\\	、分節
\\	彼女の笑顔には時々うつろさが見える。	変化形 
\\	【自動】空になる【他動】〔容器などから中身を〕出す、〔容器などを〕空にする【名】
\\	〈話〉空車{くうしゃ}、空っぽの入れ物、空き箱、空き部屋【形】中身{なかみ}のない、空{から}の、空{あ}いている、空{から}っぽの、(何も入っていなくて)ガラガラの〔室内・店内・道路などが〕誰もいない、無人の、人けのない、人影{ひとかげ}のない~を欠いた〈軽蔑的〉口先{くちさき}[上っ面]だけの、空虚{くうきょ}な〔心の中が〕空{から}っぽの、むなしい、魂の抜けたような〔表情などが〕うつろな ・
\\	彼女の笑顔には時々うつろさが見える。レベル2、発音
\\	、カナエンプティー、変化《形》
\\	、《動》
\\	、分節
\\	ディスクが満杯になった。 ・
\\	私は純日本人です。 ・
\\	私はおなかがいっぱいで動けません。 ・
\\	《野球》満塁です。	変化形 
\\	【1他動】〔襞などを付けて衣類{いるい}を〕膨らませる【1名】全部{ぜんぶ}、完全{かんぜん}、十分{じゅうぶん}全盛{ぜんせい}、絶頂{ぜっちょう}【1形】〔許容量{きょようりょう}まで〕いっぱいの、満ちた ・
\\	ディスクが満杯になった。大量{たいりょう}の、数多くの、たくさんの〔地位{ちい}などの資格{しかく}を〕満たした、満足{まんぞく}した ・
\\	私は純日本人です。〔程度{ていど}が〕最大限{さいだいげん}の、限度いっぱいの欠けるもののない、不足{ふそく}していない、完全{かんぜん}な〔形状が〕ふっくらした、丸々{まるまる}とした、豊満{ほうまん}な〔体が〕太った、超えた、ぽっちゃりした〔衣類などが〕ゆったりした、ゆとりのある〔ワインなどが〕熟成{じゅくせい}した真っ盛りの、最盛期{さいせいき}の、最高潮{さいこうちょう}の〔声や音が〕朗々{ろうろう}とした、豊かなおなかがいっぱいの、たらふくの、満腹{まんぷく}の ・
\\	私はおなかがいっぱいで動けません。〔兄弟{きょうだい}などが〕両親{りょうしん}が同じの、同じ両親{りょうしん}から生まれた〔生活{せいかつ}などが〕充実{じゅうじつ}した、活動的{かつどう てき}な〔人が〕没頭{ぼっとう}した、心を奪われた《野球》〔塁が〕埋まった、ランナーがいる ・
\\	《野球》満塁です。【1副】〔程度{ていど}が〕完全{かんぜん}に、十分{じゅうぶん}に〔位置{いち}が〕正確{せいかく}に、真ん中に【2自動】〔毛織物{けおりもの}が〕縮絨{しゅくじゅう}[縮充{しゅくじゅう}]する【2他動】〔毛織物{けおりもの}を〕縮絨{しゅくじゅう}[縮充{しゅくじゅう}]させる、縮絨{しゅくじゅう}[縮充{しゅくじゅう}]加工{かこう}する◆アルカリ溶液などにつけたり、たたいたりして毛を絡ませてフェルト状にすること。レベル1、発音
\\	、カナフル、変化《形》
\\	価格は安定している。	変化形 
\\	【1名】〔法人格{ほうじんかく}を持たない〕商会{しょうかい}、(合資{ごうし})会社{かいしゃ}、事務所{じむしょ}【2自動】固まる、安定{あんてい}する【2他動】固める、安定{あんてい}させる【2形】〔物・土台などが〕堅い、堅固{けんご}な、頑丈{がんじょう}な身が締まった、引き締まった◆【反】
\\	〔動作・態度などが〕しっかりした、安定{あんてい}した〔信念・主義などが〕断固{だんこ}とした、確固{かっこ}たる〔物価・市場などが〕堅調{けんちょう}な、安定{あんてい}した ・
\\	価格は安定している。【2副】堅く、しっかりと、断固{だんこ}としてレベル2、発音
\\	、カナファーム、変化《形》
\\	、《動》
\\	北部の鳥は平均すると南部のものよりも大きい。	変化形 
\\	【自動】平均{へいきん}すると~になる ・
\\	北部の鳥は平均すると南部のものよりも大きい。〔株式{かぶしき}で〕難平{なんぴん}買い下がり[売り上がり]をする【他動】《数学》~の平均値{へいきんち}を求める~が平均値{へいきんち}となる、平均{へいきん}すると~になる~を平均化{へいきんか}する、~をならす【名】《数学》平均(値){へいきん(ち)}《数学》〔統計{とうけい}の〕算術平均{さんじゅつ へいきん}◆【同】
\\	〔人やものの〕並、標準{ひょうじゅん}〔スポーツの〕率、アベレージ◆勝率や打率などの成功した割合。〔証券取引所{しょうけん とりひきじょ}の〕平均株価{へいきん かぶか}《海事》海損{かいそん}◆【参考】
\\	《海事》小海損{しょう かいそん}◆【同】
\\	【形】《数学》平均{へいきん}の中間{ちゅうかん}の、中くらいの普通{ふつう}の、通常{つうじょう}の《海事》海損{かいそん}のレベル3、発音
\\	、カナアベレージ、アベレッジ、変化《動》
\\	、分節
\\	変化形 
\\	《複》
\\	【名】
\\	概要{がいよう}、通則{つうそく}、総則{そうそく}、一般{いっぱん}《軍事》〔米陸・空軍{くうぐん}・海兵隊{かいへいたい}の〕大将{たいしょう}◆中将
\\	より上の階級。◆【略】
\\	《軍事》〔米陸・空軍{くうぐん}・海兵隊{かいへいたい}の〕将官{しょうかん}◆【同】
\\	◆【略】
\\	【形】全般的{ぜんぱんてき}な、全員{ぜんいん}の、全体的{ぜんたいてき}な、全面的{ぜんめんてき}な一般{いっぱん}の、一般的{いっぱんてき}な、通例{つうれい}の〔映画などが〕一般向けの◆【略】
\\	概略{がいりゃく}の、大綱{たいこう}の、大まかなさまざまな、いろいろな、雑多{ざった}な専門家{せんもんか}ではない、専門知識{せんもん ちしき}のない、普通{ふつう}の〔職位{しょくい}などが〕高位{こうい}の、〔職位{しょくい}につける〕総~レベル2、発音
\\	、カナジェネラル、ゼネラル、変化《複》
\\	、分節
\\	変化形 
\\	《複》
\\	【名】〔地域共同体{ちいき きょうどうたい}が利用{りよう}する〕共有地{きょうゆうち}、公園{こうえん}、広場{ひろば}《法律》入会権{いりあいけん}= 
\\	一般人{いっぱんじん}、庶民{しょみん}
\\	庶民{しょみん}[平民{へいみん}]階級{かいきゅう}
\\	下院{かいん}(議員{ぎいん})
\\	イギリス下院{かいん}◆【同】
\\	〔大学{だいがく}の〕食堂{しょくどう}【形】誰でも知っている[やっている・使っている]、一般的{いっぱん てき}な、普及{ふきゅう}している公共{こうきょう}の、公衆{こうしゅう}の、公のよく起こる[目にする]、日常的{にちじょう てき}な、普通{ふつう}の〔動植物{どうしょくぶつ}などが〕どこにでもいる、ありふれた専門家{せんもんか}でない、特別{とくべつ}な知識{ちしき}を持たない、役職{やくしょく}に就いていない〔能力{のうりょく}などが〕並の、平凡{へいぼん}な、劣った〔態度{たいど}などが〕粗野{そや}な、不作法{ぶさほう}な、雑な《数学》共通{きょうつう}の、公の《文法》〔性が〕通性{つうせい}[両性{りょうせい}]の、〔名詞{めいし}が〕普通{ふつう}の《キリスト教》共同{きょうどう}の、コモン《コ》〔変数{へんすう}が〕共通{きょうつう}の、コモン~レベル2、発音
\\	、カナカモン、コモン、変化《形》
\\	、《複》
\\	、分節
\\	よくあることだ。	イディオムやフレーズ 
\\	【形】典型的{てんけいてき}な、代表的{だいひょうてき}な、標準的{ひょうじゅんてき}な、象徴的{しょうちょうてき}な、象徴{しょうちょう}する◆【反】
\\	よくあることだ。特有{とくゆう}な、独特{どくとく}のレベル3、発音
\\	、カナティピカル、分節
\\	今夜、何か特にすることある?◆【場面】食事などに誘う。 ・
\\	彼は、ただ者ではありません。	変化形 
\\	《複》
\\	【名】特別{とくべつ}な[臨時{りんじ}の]人[もの]〔スーパーなどの〕特売{とくばい}[特価{とっか}]品〔レストランの〕特別料理{とくべつ りょうり}、お勧めメニュー〔テレビの〕特別番組{とくべつ ばんぐみ}、スペシャル特別授業{とくべつ じゅぎょう}◆特にアメリカの小学校で、音楽や体育といった毎日は行われない授業を指す。【形】〔通常{つうじょう}と異なる〕特別{とくべつ}な、特殊{とくしゅ}な ・
\\	今夜、何か特にすることある?◆【場面】食事などに誘う。〔人やものに〕特有{とくゆう}の、独特{どくとく}な〔ある目的{もくてき}のための〕特設{とくせつ}の、臨時{りんじ}の〔ある人が自分{じぶん}にとって〕大切{たいせつ}な、大好き{だいすき}な〔通常{つうじょう}より〕上等{じょうとう}な、優れた ・
\\	彼は、ただ者ではありません。〔学校{がっこう}などが〕障害児{しょうがいじ}のためのレベル1、発音
\\	、カナスペシャル、変化《複》
\\	、分節
\\	詳細は申し上げられません。 ・
\\	弊社の計画の正確な詳細については申し上げられないのですが、社内の状況が変わるということだけは確かです。 ・
\\	私が過去に対処した状況の具体的な例がこれです。 ・
\\	私の具体的な目的は、生徒たちが語学力をアップできるように手助けすることです。 ・
\\	その件についてもう少し具体的に説明して[話して]いただけませんでしょうか。 ・
\\	この情報は、あなたのカスタマイズ[特注]コンピューターだけにあてはまるものです。 ・
\\	このサービスの特質について説明させていただきます。	変化形 
\\	《複》
\\	【名】仕様(書){しよう(しょ)}、明細(書){めいさい(しょ)}、詳述{しょうじゅつ}、細目{さいもく} ・
\\	詳細は申し上げられません。 ・
\\	弊社の計画の正確な詳細については申し上げられないのですが、社内の状況が変わるということだけは確かです。〔特定の病気の〕特効薬{とっこうやく}【形】明確{めいかく}な、はっきり限定{げんてい}された、具体的{ぐたいてき}な、詳しい、曖昧{あいまい}でない◆【反】
\\	私が過去に対処した状況の具体的な例がこれです。 ・
\\	私の具体的な目的は、生徒たちが語学力をアップできるように手助けすることです。 ・
\\	その件についてもう少し具体的に説明して[話して]いただけませんでしょうか。特定{とくてい}の(ことに関する) ・
\\	この情報は、あなたのカスタマイズ[特注]コンピューターだけにあてはまるものです。特有{とくゆう}の、固有{こゆう}の、個別{こべつ}の ・
\\	このサービスの特質について説明させていただきます。《生物》(生物{せいぶつ})種の《医》〔特定{とくてい}の病状{びょうじょう}に〕特効{とっこう}がある《医》〔病気{びょうき}が〕特異{とくい}の◆特定の伝染性病原菌によって引き起こされること。《物理》固有{こゆう}の〔税が〕従量制{じゅうりょう せい}の◆容積、重さ、数量に応じて課税を行う方式。レベル3、発音
\\	、カナスペシフィック、スペスィフィック、変化《複》
\\	、分節
\\	デザイン・色およびサイズの詳細はすべて同封の注文書に記載の通りです。 ・
\\	この外科手術の詳細を教えていただけますか?	変化形 
\\	《複》
\\	【名】〔個々{ここ}の〕項目{こうもく}、事項{じこう}詳細{しょうさい}、明細{めいさい}◆通例、
\\	デザイン・色およびサイズの詳細はすべて同封の注文書に記載の通りです。 ・
\\	この外科手術の詳細を教えていただけますか?《論理学》特称{とくしょう}(命題{めいだい})【形】特定{とくてい}の、個別{こべつ}の〔特定{とくてい}の人やものに〕固有{こゆう}の、特有{とくゆう}の特別{とくべつ}な、際だった、注目{ちゅうもく}すべき〔人が〕小うるさい、好みのうるさい、気難しい〔人が〕入念な、きちょうめんな〈文〉詳細{しょうさい}な、細かい《論理学》特称{とくしょう}の、特殊{とくしゅ}なレベル2、発音
\\	、カナパーティキュラー、パティキュラー、変化《複》
\\	、分節
\\	変化形 
\\	【1形】まれな、めったにない、珍しい◆他動詞 
\\	名詞のパターンは、「まれに~する」と訳したほうが良い場合がある。素晴{すば}らしい、最高{さいこう}の〔空気{くうき}が〕希薄{きはく}な【2形】《料理》〔肉などが〕レアの、生焼け{なまやけ}のレベル3、発音
\\	、カナレア、変化《形》
\\	今夜までには、すべてが元通りになるよ。		【名】標準{ひょうじゅん}、正常{せいじょう}標準的{ひょうじゅん てき}な人、心身共{しんしん とも}に健康{けんこう}な人《数学》法線{ほうせん}、垂線{すいせん}《電気》定位{ていい}《数学》法線{ほうせん}◆【同】
\\	【形】通常{つうじょう}の、標準{ひょうじゅん}の、平常{へいじょう}の ・
\\	今夜までには、すべてが元通りになるよ。《生物》〔形状{けいじょう}や機能{きのう}が〕正常{せいじょう}な、健康{けんこう}な《心理学》〔知能{ちのう}の発達{はったつ}が〕平均的{へいきん てき}な《化学》〔溶液{ようえき}が〕規定度{きていど}の◆1リットルの溶液当たり1グラムの物質が溶けているもの。《化学》〔脂肪族炭水化物{しぼう ぞく たんすいかぶつ}が〕枝分{えだわ}かれしていない《数学》垂直{すいちょく}な、垂線{すいせん}のレベル2、発音
\\	、カナノーマル、分節
\\	私も(ただの)人間だ。◆「だから間違いもする」という含みがある。	変化形 
\\	《複》
\\	【名】〈話〉〔動物と異なる〕人、人間{にんげん}【形】人[人間{にんげん}]の[に関する]〔動物や機械と異なり〕人間{にんげん}らしい[くさい]、思いやりのある〔神と異なり〕間違いを犯す、不完全{ふかんぜん}な ・
\\	私も(ただの)人間だ。◆「だから間違いもする」という含みがある。〔神や霊などが〕人の形を取った、人となった人から構成{こうせい}される、人間{にんげん}で作られた人間のようなレベル2、発音
\\	、カナヒューマン、変化《複》
\\	、分節
\\	【名】人類{じんるい}、人間{にんげん}◆【無性語】
\\	レベル2、発音
\\	、カナマンカインド、分節
\\	家柄よりも育てられ方が大切。/氏より育ち。	変化形 
\\	《複》
\\	【他動】〈米南部方〉〔赤ん坊を〕生む、出産{しゅっさん}する【名】〔人や動物{どうぶつ}の〕誕生{たんじょう}、出生{しゅっせい}、生まれること〔過程{かてい}としての〕出産{しゅっさん}、分娩{ぶんべん}出産{しゅっさん}[誕生{たんじょう}]場所{ばしょ}[地域{ちいき}・時期{じき}]家系{かけい}、家柄{いえがら}、出自{しゅつじ} ・
\\	家柄よりも育てられ方が大切。/氏より育ち。〔物事{ものごと}の〕始まり、誕生{たんじょう}、起源{きげん}レベル2、発音
\\	、カナバース、変化《複》
\\	死は万人に訪れる。 ・
\\	死を恐れていません。 ・
\\	死因は何ですか? ・
\\	彼には死刑では甘い。/何と極悪なやつだ。	変化形 
\\	《複》
\\	【間投】畜生{ちくしょう}、くそっ、しまった〈俗〉お見事、うまい、すごい【名】死、死亡{しぼう} ・
\\	死は万人に訪れる。 ・
\\	死を恐れていません。 ・
\\	死因は何ですか?死んでいること、死んだ状態{じょうたい}死に方、死にざま
\\	死の原因{げんいん}、死因{しいん}、致命傷{ちめいしょう}
\\	破滅{はめつ}、消滅{しょうめつ}
\\	死神{しにがみ}◆大鎌を持つ骸骨として象徴化される。死刑{しけい}、殺人{さつじん} ・
\\	彼には死刑では甘い。/何と極悪なやつだ。《法律》民事死{みんじ し}◆【同】
\\	《囲碁》死に、つぶれレベル1、発音
\\	、カナデス、変化《複》
\\	《名言》燃えることのない青春は経験の蓄積のない老年をもたらす。	変化形 
\\	《複》
\\	【名】青年時代{せいねん じだい}、青春時代{せいしゅん じだい} ・
\\	《名言》燃えることのない青春は経験の蓄積のない老年をもたらす。年が若いこと、若さ若い人、若者{わかもの}レベル2、発音
\\	、カナユース、変化《複》
\\	変化形 
\\	【他動】~を経験{けいけん}[体験{たいけん}]する〔経験{けいけん}によって~であると〕学ぶ、気が付く〔感覚{かんかく}や感情{かんじょう}を〕感じる、感じがする【名】〔知識{ちしき}や技能{ぎのう}を得るために〕体験{たいけん}[経験{けいけん}]すること◆不可算〔人が出会う個々{ここ}の〕経験{けいけん}、体験{たいけん}◆可算〔経験{けいけん}で獲得{かくとく}した〕知識{ちしき}、技能{ぎのう}〔個人{こじん}のこれまでの〕全経験{ぜん けいけん}〔知覚{ちかく}や理性{りせい}による〕認識{にんしき}、理解{りかい}《哲学》経験{けいけん}◆抽象的思考ではなく、感覚によって得られる知識。レベル2、発音
\\	、カナイクスピアリアンス、エクスピアリアンス、エクスペリエンス、変化《動》
\\	、分節
\\	父の記憶力は確かなようだ。 ・
\\	記憶力がだんだん不確かになってきた。 ・
\\	それについては記憶にございません。 ・
\\	年を取るにつれて子どもの頃の思い出は色あせてくる。 ・
\\	父はニューヨークに行った日の出来事を思い出して話してくれた。 ・
\\	この記念碑はその戦いで死んだアメリカ人兵士を追悼して建てられた。 ・
\\	私の義務は亡き王の名を賛美することである。 ・
\\	その戦争を知っている人はまだ現存している。	変化形 
\\	《複》
\\	【名】記憶力{きおくりょく}、記銘力{きめいりょく}、想起力{そうき りょく} ・
\\	父の記憶力は確かなようだ。 ・
\\	記憶力がだんだん不確かになってきた。〔ある人の思い出せるすべての〕記憶{きおく} ・
\\	それについては記憶にございません。〔個々{ここ}の出来事{できごと}の〕思い出 ・
\\	年を取るにつれて子どもの頃の思い出は色あせてくる。〔思い出された〕出来事{できごと}、事実{じじつ} ・
\\	父はニューヨークに行った日の出来事を思い出して話してくれた。〔故人{こじん}の〕記念{きねん}、追悼{ついとう} ・
\\	この記念碑はその戦いで死んだアメリカ人兵士を追悼して建てられた。〔死後{しご}に残る人の〕評判{ひょうばん}、名声{めいせい} ・
\\	私の義務は亡き王の名を賛美することである。〔思い出される〕期間{きかん} ・
\\	その戦争を知っている人はまだ現存している。《コ》メモリ、記憶装置{きおく そうち}《コ》記憶容量{きおく ようりょう}〔プラスチックなどが元の形へ戻ろうとする〕復元力{ふくげんりょく}、〔元の形状{けいじょう}や状態{じょうたい}の〕記憶能力{きおく のうりょく}《トランプ》神経衰弱{しんけい すいじゃく}◆【同】
\\	レベル1、発音
\\	、カナメモリー、変化《複》
\\	、分節
\\	君ほどの脳みそはないよ。 ・
\\	私はそのプロジェクトのブレーンだった。	変化形 
\\	【他動】~の頭を殴る[たたき割る]、~の頭を打ち砕いて殺す【名】脳、大脳{だいのう}頭脳{ずのう}、知力{ちりょく} ・
\\	君ほどの脳みそはないよ。〈話〉秀才{しゅうさい}、とても頭の良い人〔飛行機{ひこうき}や船などの〕コントロール・センター、中枢部{ちゅうすう ぶ}ブレーン、知恵袋{ちえぶくろ}◆通例
\\	◆ある集団の中で最も知力が優れ、計画を作ったり戦略を練ったりする役割の人。 ・
\\	私はそのプロジェクトのブレーンだった。レベル2、発音
\\	、カナブレイン、変化《動》
\\	どちらでもいいです。 ・
\\	《嫌みっぽく言って》いいかげんにしてくれない? ・
\\	その理由をお尋ねしてもよろしいでしょうか? ・
\\	一緒に行っても構わない? ・
\\	《食事の誘いを断った後に》お茶を一杯だけなら。 ・
\\	どういうことですか? ・
\\	ちょっと言わせてもらっていいかな?君は態度が悪い。◆態度の横柄な人に忠告する時 ・
\\	言葉遣いに気を付けなさい。 ・
\\	私のことなど気にしないでください。/ご自由に。/おかまいなく。 ・
\\	そんなこと全然気にするな。 ・
\\	何を考えて[悩んで]いるのですか?/何か心配事があるのですか? ・
\\	へそまがり! ・
\\	あなたは私の心が分かっていない。 ・
\\	ご安心ください。	変化形 
\\	【自動】嫌だと思う、嫌がる、気にする、気に障る ・
\\	どちらでもいいです。 ・
\\	《嫌みっぽく言って》いいかげんにしてくれない?言うことを聞く、従順{じゅうじゅん}である用心{ようじん}する、注意{ちゅうい}する【他動】~を嫌だと思う、~が気に障る ・
\\	その理由をお尋ねしてもよろしいでしょうか? ・
\\	一緒に行っても構わない? ・
\\	《食事の誘いを断った後に》お茶を一杯だけなら。 ・
\\	どういうことですか? ・
\\	ちょっと言わせてもらっていいかな?君は態度が悪い。◆態度の横柄な人に忠告する時~に気を付ける、~に注意{ちゅうい}を払う ・
\\	言葉遣いに気を付けなさい。 ・
\\	私のことなど気にしないでください。/ご自由に。/おかまいなく。~の面倒{めんどう}を見る、~の世話{せわ}をする~が思い浮かぶ、~を思い出す~に気付く、~が分かる(人)の言うことを聞く、(人)に従う【名】〔思考{しこう}・感情{かんじょう}・記憶{きおく}などをつかさどる〕心、精神{せいしん} ・
\\	そんなこと全然気にするな。 ・
\\	何を考えて[悩んで]いるのですか?/何か心配事があるのですか? ・
\\	へそまがり! ・
\\	あなたは私の心が分かっていない。 ・
\\	ご安心ください。〔感情{かんじょう}ではない〕知性{ちせい}、判断力{はんだんりょく}〔思考{しこう}の〕注意{ちゅうい}、集中{しゅうちゅう}、専念{せんねん}〔物事{ものごと}に対する〕考え方、感じ方、意見{いけん}〔~したいという〕意向{いこう}、望み、気持{きも}ち正気{しょうき}、正常{せいじょう}[健全{けんぜん}]な精神状態{せいしん じょうたい}[考え方]〔個人{こじん}の〕意識{いしき}、記憶{きおく}、思い出〔特定{とくてい}の人に固有{こゆう}の〕心理{しんり}、意識{いしき}、気質{きしつ}◆【参考】
\\	〔優れた知性{ちせい}を持つ〕人、人間{にんげん}レベル1、発音
\\	、カナマインド、変化《動》
\\	変化形 
\\	《複》
\\	【他動】〈俗〉(人)を感動{かんどう}させる、(人)に感情{かんじょう}を表出{ひょうしゅつ}させる【名】〔生理的{せいり てき}な変化{へんか}として現れる強い〕感情{かんじょう}〔感情{かんじょう}によって引き起こされる〕興奮{こうふん}〔知性{ちせい}と対照{たいしょう}される感情的{かんじょう てき}な〕感性{かんせい}
\\	喜怒哀楽{き ど あい らく}レベル3、発音
\\	、カナイモウション、エモーション、エモウション、変化《複》
\\	、分節
\\	水泳は私の情熱の対象です。/私は水泳に情熱を傾けています。	変化形 
\\	《複》
\\	【名】情熱{じょうねつ}、熱情{ねつじょう}、激情{げきじょう}、理性{りせい}のない感情{かんじょう}愛情{あいじょう}、愛着{あいちゃく}、熱中{ねっちゅう}
\\	キリスト受難{じゅなん}(曲)
\\	好きな道 ・
\\	水泳は私の情熱の対象です。/私は水泳に情熱を傾けています。レベル3、発音
\\	、カナパッション、変化《複》
\\	、分節
\\	隠された[秘めた]動機はない。 ・
\\	あなたは殺人の動機を隠そうとしていた。	変化形 
\\	《複》
\\	【名】〔ある行動{こうどう}を取る〕動機{どうき}、真意{しんい}、誘因{ゆういん} ・
\\	隠された[秘めた]動機はない。 ・
\\	あなたは殺人の動機を隠そうとしていた。〔建築{けんちく}や文学作品{ぶんがく さくひん}の〕主題{しゅだい}◆【同】
\\	【形】〔運動{うんどう}の〕原動力{げんどうりょく}となる〔行動{こうどう}の〕動機{どうき}となるレベル5、発音
\\	、カナモーティブ、モウティブ、変化《複》
\\	、分節
\\	彼の態度にも問題があった。	変化形 
\\	《複》
\\	【名】〔相手{あいて}と接しているときの〕態度{たいど}、立ち居振る舞い ・
\\	彼の態度にも問題があった。〔物事に対する〕考え方〈話〉偉そうな[押しが強い]態度{たいど}[言動{げんどう}]〔人の感情が表れた〕姿勢{しせい}アチチュード◆バレエのポーズ名。片足で立ち、もう片方の足を膝を曲げて上げる。〔航空機{こうくうき}の〕姿勢{しせい}◆水平面に対する航空機の軸の傾き。〔宇宙船{うちゅうせん}の〕姿勢{しせい}◆進行方向に対する機体の傾き。レベル3、発音
\\	、カナアチチュード、アティチュード、アティテュード、変化《複》
\\	、分節
\\	彼は面白いやつだよ。/彼は個性的です。	変化形 
\\	《複》
\\	【名】性格{せいかく}、個性{こせい}、特徴{とくちょう}人格{じんかく}、人柄{ひとがら}、品性{ひんせい}身分{みぶん}、地位{ちい}登場人物{とうじょう じんぶつ}、キャラクター、役柄{やくがら}、人物{じんぶつ}変わり者{かわりもの}、個性的{こせいてき}な人、面白い人◆悪い意味はない ・
\\	彼は面白いやつだよ。/彼は個性的です。文字{もじ}◆数字・符号も含む《遺伝》形質{けいしつ}〈英古〉〔前雇用主{ぜん こよう ぬし}が前使用人{ぜん しよう にん}に付与{ふよ}する〕推薦状{すいせん じょう}、人物証明書{じんぶつ しょうめいしょ}レベル2、発音
\\	、カナキャラクター、変化《複》
\\	、分節
\\	父があんなふうに泣くなんて想像できませんでした。 ・
\\	モナリザの最も有名な特徴はその笑顔です。	変化形 
\\	【自動】〔映画{えいが}などで〕主役{しゅやく}を演じる〔計画{けいかく}などで〕考慮{こうりょ}する、計算{けいさん}に入れる【他動】〔イベントや記事{きじ}で〕~を呼び物にする、~を特集{とくしゅう}する〔映画{えいが}で人を〕主役{しゅやく}にする、〔歌手{かしゅ}を〕客演{きゃくえん}させる~の特色{とくしょく}となる、~を特徴付ける〈話〉~を心に描く、~を想像{そうぞう}する ・
\\	父があんなふうに泣くなんて想像できませんでした。【名】特徴{とくちょう}(づけるもの)、特性{とくせい}、特長{とくちょう}、素性{すじょう} ・
\\	モナリザの最も有名な特徴はその笑顔です。顔立ち{かおだち}、顔の造作{ぞうさく}、容貌{ようぼう}、外観{がいかん}◆通例
\\	主要{しゅよう}なもの、メインのもの、呼び物、目玉{めだま}(商品{しょうひん})メイン記事{きじ}、特集記事{とくしゅう きじ}、呼び物記事◆【同】
\\	◆新聞、雑誌などの。主要作品{しゅよう さくひん}、メイン上映作品{じょうえい さくひん}地物{ちぶつ}◆地図学や地理学の世界では従来、地形上に乗っている自然/人工物を地物というが、地理情報学ではより広義に解釈して、地球空間に生起する現象の抽象概念(道路、家屋、河川、台風、雲など)を指す。〔装置などの〕機能{きのう}、機構{きこう}レベル3、発音
\\	、カナフィーチャー、変化《動》
\\	、分節
\\	変化形 
\\	《複》
\\	【名】〔目的達成{もくてき たっせい}のための〕尽力{じんりょく}、努力{どりょく}◆目的を達成するために使われる精神的・肉体的エネルギーの量。〔目的達成{もくてき たっせい}に向けて〕努力{どりょく}すること、骨を折ること◆目的の達成や困難の克服のために大変なエネルギーを使うこと。〔努力{どりょく}を必要{ひつよう}とする〕試み、取り組み〔努力{どりょく}の結果{けっか}の〕成果{せいか}、作品{さくひん}《物理》〔てこなどに〕加える力レベル2、発音
\\	、カナエファト、エフォート、変化《複》
\\	、分節
\\	変化形 
\\	《複》
\\	【名】〔経験{けいけん}や訓練{くんれん}によって得られる〕技能{ぎのう}、腕前{うでまえ}、スキル〔高度{こうど}な技能{ぎのう}が要求{ようきゅう}される〕熟練{じゅくれん}仕事{しごと}[作業{さぎょう}]レベル3、発音
\\	、カナスキル、変化《複》
\\	変化形 
\\	《複》
\\	【名】〔人や物事{ものごと}への〕信頼{しんらい}、信用{しんよう}自信{じしん}◆【同】
\\	秘密{ひみつ}、打ち明け話〔秘密{ひみつ}が話せる〕親密{しんみつ}な間柄{あいだがら}~の確実性{かくじつ せい}、~が確かであること◆【用法】
\\	ずうずうしさ、厚かましさレベル3、発音
\\	、カナカンフィデンス、コンフィデンス、変化《複》
\\	、分節
\\	《名言》勇気はとても大事であり、筋肉と同じで使うことによって鍛えられます。	イディオムやフレーズ 
\\	【名】勇気{ゆうき}、勇敢{ゆうかん}、度胸{どきょう} ・
\\	《名言》勇気はとても大事であり、筋肉と同じで使うことによって鍛えられます。レベル2、発音
\\	、カナカーリッジ、カリッジ、カレッジ、分節
\\	【名】名声{めいせい}、有名{ゆうめい}なこと、高名{こうめい}レベル4、発音
\\	、カナフェイム
\\	山田さんからの約束手形が不渡りになった。 ・
\\	特にお急ぎの注文は購入者様側の費用負担にて承ります。 ・
\\	彼女に会えて、とても光栄に思った。 ・
\\	陸軍将校に払われる敬意は将校という階級に対する尊敬の念を表しているのであり将校の地位を担う人に対するものではない。 ・
\\	シーザーは自分の名声と命を危険にさらしてもその川を渡った。 ・
\\	21発の礼砲は外国の高官に対する最高の儀礼である。 ・
\\	あなたに対する私の誠実さを疑おうというのか。 ・
\\	武士は封建君主に仕えることにより道義心を獲得しようとする。 ・
\\	純潔を守ろうと死に物狂いになり彼女は崖{がけ}から飛び降りた。 ・
\\	市長閣下に対してそれぞれの緊急の課題について至急声明を出すように配達証明郵便を送りましょう。 ・
\\	最優等や優等などの優等賞をもらって卒業するかどうかは成績評価点の全体の平均値にかかっている。 ・
\\	生命科学の優等生特別クラスを提供しています。	イディオムやフレーズ 
\\	【他動】~に尊敬{そんけい}の念を持つ、~を高く評価{ひょうか}する〔報償{ほうしょう}や称号{しょうごう}などで人に〕栄誉{えいよ}を授ける〔公の場で人を〕称賛{しょうさん}する、褒めたたえる〔人の出席{しゅっせき}などが~に〕栄誉{えいよ}となる、光栄{こうえい}と感じさせる〔手形{てがた}などを有効{ゆうこう}と見なして〕受け取る、引き受ける ・
\\	山田さんからの約束手形が不渡りになった。 ・
\\	特にお急ぎの注文は購入者様側の費用負担にて承ります。〔合意{ごうい}や約束{やくそく}を〕守る、履行{りこう}する〔ダンスで相手{あいて}に〕おじぎ[礼]をする【名】光栄{こうえい}、名誉{めいよ}、恩恵{おんけい} ・
\\	彼女に会えて、とても光栄に思った。〔功績{こうせき}などに対して払われる〕敬意{けいい} ・
\\	陸軍将校に払われる敬意は将校という階級に対する尊敬の念を表しているのであり将校の地位を担う人に対するものではない。名声{めいせい}、評判{ひょうばん} ・
\\	シーザーは自分の名声と命を危険にさらしてもその川を渡った。表彰{ひょうしょう}、勲章{くんしょう}、称号{しょうごう}、儀礼{ぎれい}、名誉{めいよ}の印 ・
\\	21発の礼砲は外国の高官に対する最高の儀礼である。高潔{こうけつ}、誠実{せいじつ}、正直{しょうじき}さ ・
\\	あなたに対する私の誠実さを疑おうというのか。〔中世{ちゅうせい}の男性{だんせい}の〕道義心{どうぎしん}、自尊心{じそんしん}、道徳規範{どうとく きはん} ・
\\	武士は封建君主に仕えることにより道義心を獲得しようとする。〔女性{じょせい}の〕純潔{じゅんけつ}、貞潔{ていけつ} ・
\\	純潔を守ろうと死に物狂いになり彼女は崖{がけ}から飛び降りた。
\\	閣下{かっか}、~殿◆市長や裁判官などの高位の人に対する尊称で、
\\	と共に用いられる。 ・
\\	市長閣下に対してそれぞれの緊急の課題について至急声明を出すように配達証明郵便を送りましょう。
\\	優等{ゆうとう}(賞)、優等生特別{ゆうとうせい とくべつ}クラス ・
\\	最優等や優等などの優等賞をもらって卒業するかどうかは成績評価点の全体の平均値にかかっている。 ・
\\	生命科学の優等生特別クラスを提供しています。〔ブリッジの〕オナー・カード◆各組のエース、キング、クイーン、ジャック、10のカード。《ゴルフ》オナー◆先にティー・ショットする権利【形】名誉{めいよ}の[に関する]レベル3、発音
\\	、カナアナ、変化《動》
\\	、分節
\\	別に恥ずかしがるほどのことじゃありません。 ・
\\	こんな天気のいい日に家の中にいるなんてもったいない。 ・
\\	行けなくて残念でした。 ・
\\	食べてしまうのがもったいないくらいだ。◆美しい料理・菓子など。	変化形 
\\	【他動】(人)に恥ずかしい思いをさせる、(人)を恥じ入らせる(人)が恥じて~させるようにする、(人)が~しないのは恥だと思わせる〔優れていることを見せつけて〕(人)の面目{めんもく}をつぶす、(人)の名誉{めいよ}を傷つける【名】恥かしさ、恥ずかしい思い ・
\\	別に恥ずかしがるほどのことじゃありません。恥、不名誉{ふめいよ}〈話〉残念{ざんねん}[遺憾{いかん}]な事、慙愧{ざんき}の念 ・
\\	こんな天気のいい日に家の中にいるなんてもったいない。 ・
\\	行けなくて残念でした。 ・
\\	食べてしまうのがもったいないくらいだ。◆美しい料理・菓子など。レベル3、発音
\\	、カナシェイム、変化《動》
\\	冗談言ってるの?/からかってるの?◆【同】
\\	楽しんできてね、そして気を付けてね。◆これから旅行に出掛ける人に言う。 ・
\\	昨夜ほど楽しいことはなかったよ。 ・
\\	大変なんだから。 ・
\\	今日は楽しかったなあ。 ・
\\	彼は一緒に働いていて[仕事のパートナーとして]、とても楽しい人です。 ・
\\	「私は
\\	か
\\	のどちらかを買うつもりです。どちらのゲームがより面白いですか」
\\	を入手してください。史上最高の面白いゲームです」 ・
\\	そんなことして何が面白いの。 ・
\\	私たちは楽しいパーティーを開きました。
\\	イディオムやフレーズ 
\\	【もっとイディオムを見る】
\\	"【自動】〈話〉ふざける、冗談{じょうだん}を言う【他動】〈話〉~をからかう ・
\\	冗談言ってるの?/からかってるの?◆【同】
\\	【名】楽しみ、面白さ ・
\\	楽しんできてね、そして気を付けてね。◆これから旅行に出掛ける人に言う。 ・
\\	昨夜ほど楽しいことはなかったよ。 ・
\\	大変なんだから。 ・
\\	今日は楽しかったなあ。面白い[楽しい]こと[もの] ・
\\	彼は一緒に働いていて[仕事のパートナーとして]、とても楽しい人です。〔人をだしにした〕からかい、冗談{じょうだん}〔陽気{ようき}な〕大騒ぎ{おおさわぎ}【形】〈話〉楽しい、ゆかいな◆【語法】一般に比較級は
\\	とする。
\\	が本来名詞であったためで、
\\	も
\\	名詞」を流用したもの。
\\	の形も徐々に認められる傾向にあるが、現状「誤用・ふざけてわざと言っている・俗語的表現」と見なす人が多い。最上級も同様で
\\	が穏当。(2010年現在) ・
\\	「私は
\\	か
\\	のどちらかを買うつもりです。どちらのゲームがより面白いですか」
\\	を入手してください。史上最高の面白いゲームです」 ・
\\	そんなことして何が面白いの。 ・
\\	私たちは楽しいパーティーを開きました。〔商品{しょうひん}が〕けばけばしい、イミテーションの◆戯れに身に着ける宝石やアクセサリーなど。レベル1、発音
\\	、カナファン、変化《動》
\\	、《形》
\\	冗談ですよ。 ・
\\	これは何かの冗談ですか? ・
\\	冗談のつもりですか? ・
\\	われわれが一致団結すれば、ちょろいものさ。 ・
\\	所詮{しょせん}、おまえらなんかわたしの敵じゃない。	変化形 
\\	【自動】冗談{じょうだん}を言う、からかう ・
\\	冗談ですよ。【他動】~をからかう【名】冗談{じょうだん}、ジョーク◆【略】
\\	これは何かの冗談ですか? ・
\\	冗談のつもりですか?悪ふざけ、いたずら取るに足りないこと、つまらないこと、たわいもないこと ・
\\	われわれが一致団結すれば、ちょろいものさ。 ・
\\	所詮{しょせん}、おまえらなんかわたしの敵じゃない。物笑いの種、お笑い草レベル1、発音
\\	、カナジョーク、ジョウク、変化《動》
\\	今度はその手は食わないよ。 ・
\\	敵を欺くにはまず味方から。 ・
\\	彼らは甘くはない。◆簡単にごまかせる相手ではない。 ・
\\	人は疑問を呈しなくなるまでは本当は愚か者にならない。 ・
\\	あきれたやつ。	変化形 
\\	【自動】ふざける、おどける、ばかなまねをする◆【用法】
\\	【他動】だます、欺く、ごまかす、惑わす、けむに巻く、かつぐ、~に一杯食わせる、からかう、ばかにする ・
\\	今度はその手は食わないよ。 ・
\\	敵を欺くにはまず味方から。 ・
\\	彼らは甘くはない。◆簡単にごまかせる相手ではない。だまして~を奪う【名】ばか者、愚か者、あほ、間抜け、能なし ・
\\	人は疑問を呈しなくなるまでは本当は愚か者にならない。 ・
\\	あきれたやつ。
\\	衆愚{しゅうぐ}レベル1、発音
\\	、カナフール、変化《動》
\\	これに勝る喜びはありません。 ・
\\	失礼ですがどちら様ですか。 ・
\\	お会いできてうれしいです。	変化形 
\\	《複》
\\	【名】楽しみ、喜び、満足{まんぞく} ・
\\	これに勝る喜びはありません。 ・
\\	失礼ですがどちら様ですか。楽しいこと、喜ばせるもの ・
\\	お会いできてうれしいです。〔性的{せいてき}な〕享楽{きょうらく}、喜び〔仕事{しごと}に対しての〕娯楽{ごらく}、気晴{きば}らし〈文〉(人)の望み[好み]レベル2、発音
\\	、カナプレジャー、変化《複》
\\	、分節
\\	おめでとうございます!	変化形 
\\	《複》
\\	【名】喜び、歓喜{かんき} ・
\\	おめでとうございます!喜び[満足{まんぞく}]の種[元]満足{まんぞく}、成功{せいこう}◆【用法】通例、疑問文や否定文で用いるレベル1、発音
\\	、カナジョイ、変化《複》
\\	ご一緒願えますか? ・
\\	犬はよく脚を引きずったり、片方の脚をかばったりするものです。 ・
\\	厚かましいお願いですが~していただけませんか? ・
\\	頼みたいことがあります。/お願いがあるのですが。	イディオムやフレーズ 
\\	【もっとイディオムを見る】
\\	【他動】~の方を好む、~をえり好みする~への支持{しじ}を表明{ひょうめい}する、~に賛意{さんい}を示す〔貴重なもので人に〕好意{こうい}を示す ・
\\	ご一緒願えますか?~に有利{ゆうり}に働く、~に都合{つごう}が良い(人)をえこひいき[特別扱い]する〈米方〉〔親などに〕似ている〔体の傷んでいる部分{ぶぶん}などを〕かばう ・
\\	犬はよく脚を引きずったり、片方の脚をかばったりするものです。【名】〔善意からの〕親切な行為、世話 ・
\\	厚かましいお願いですが~していただけませんか? ・
\\	頼みたいことがあります。/お願いがあるのですが。親切心{しんせつ しん}、好意的{こうい てき}な態度{たいど}〔人に対する不公平{ふこうへい}な〕えこひいき、情実{じょうじつ}好都合{こうつごう}、おかげ、助け味方{みかた}、ため、利益{りえき}〈米〉〔パーティーの客への〕景品{けいひん}、記念品{きねん ひん}
\\	〈古〉〔女性{じょせい}が〕性的関係{せいてき かんけい}を許すこと〈英古〉〔忠誠{ちゅうせい}や支持{しじ}を表す〕バッジ、リボンレベル2、発音
\\	、カナフェイバー、変化《動》
\\	、分節
\\	変化形 
\\	《複》
\\	【名】富、財産{ざいさん}、大金{たいきん}(幸)運運勢{うんせい}、運命{うんめい}レベル3、発音
\\	、カナフォーチューン、フォーチュン、変化《複》
\\	、分節
\\	それは私の人生最大のチャンスでした。 ・
\\	この機会をみすみす逃すのは惜しい。	変化形 
\\	《複》
\\	【名】有利{ゆうり}な条件{じょうけん}[状況{じょうきょう}]良い機会{きかい}、好機{こうき}、チャンス ・
\\	それは私の人生最大のチャンスでした。 ・
\\	この機会をみすみす逃すのは惜しい。レベル3、発音
\\	、カナアパテューニティー、オポチューニティー、オポチュニティー、変化《複》
\\	、分節
\\	時々彼女と会って話をするのが楽しみでした。	変化形 
\\	【他動】~の誘因となる、~を引き起こす、~をもたらす、(人)に~させる【名】出来事{できごと}、〔ある事がおきる〕時、場合{ばあい} ・
\\	時々彼女と会って話をするのが楽しみでした。大事{だいじ}な行事{ぎょうじ}〔ある事をする〕好機{こうき}、チャンス〔ある事を引き起こす直接{ちょくせつ}の〕原因{げんいん}、理由{りゆう}〔ある事をする〕必要性{ひつよう せい}、根拠{こんきょ}
\\	〈古〉日用必需品{にちよう ひつじゅひん}レベル2、発音
\\	、カナオケイジョン、変化《動》
\\	、分節
\\	あなたのフルネームと現在の職業を教えていただけますか? ・
\\	本製品を返送される理由を書いてください。 ・
\\	その会社は、先月、ひどい状態だった。 ・
\\	私たちが訪れたそのビルはひどい状態でした。 ・
\\	あなたは何州の出身ですか? ・
\\	「アメリカの最大の州は?」「アラスカです」 ・
\\	ムジルリツグミは、アイダホ州およびネバダ州の州鳥です。
\\	変化形 
\\	"【他動】〔正式に〕~を述べる[言葉{ことば}にする] ・
\\	あなたのフルネームと現在の職業を教えていただけますか? ・
\\	本製品を返送される理由を書いてください。《法律》〔法的{ほうてき}に有効{ゆうこう}であることを〕宣言{せんげん}する、言明{げんめい}する《音楽》〔モチーフやテーマを最初{さいしょ}に〕提示{ていじ}する【名】〔人や物が置かれた〕状態{じょうたい}、ありさま ・
\\	その会社は、先月、ひどい状態だった。 ・
\\	私たちが訪れたそのビルはひどい状態でした。〔動植物{どうしょくぶつ}の成長{せいちょう}の〕段階{だんかい}、過程{かてい}〔精神{せいしん}や感情{かんじょう}の〕状態{じょうたい}、様子{ようす}〈話〉〔人の〕興奮{こうふん}[緊張{きんちょう}]状態{じょうたい}《物理》〔物質{ぶっしつ}の〕相、状態{じょうたい}〔社会的{しゃかい てき}な〕地位{ちい}、階層{かいそう}〔儀式{ぎしき}などの〕威厳{いげん}、荘厳{そうごん}さ〔アメリカなどの〕州◆連邦を構成するが、ある程度の自治権がある政治単位。 ・
\\	あなたは何州の出身ですか? ・
\\	「アメリカの最大の州は?」「アラスカです」 ・
\\	ムジルリツグミは、アイダホ州およびネバダ州の州鳥です。〔主権{しゅけん}を持つ〕国、国家{こっか}〔国の〕政府{せいふ}、行政{ぎょうせい}
\\	アメリカ合衆国{がっしゅうこく}【形】儀式用{ぎしき よう}の、公式{こうしき}の政府{せいふ}の、国家{こっか}の州のレベル2、発音
\\	、カナステイト、変化《動》
\\	私は問題を抱えています。	変化形 
\\	《複》
\\	【名】〔ある場所{ばしょ}や社会{しゃかい}の〕状況{じょうきょう}、情勢{じょうせい}、事態{じたい}〔ある物が置かれている〕場所{ばしょ}、用地{ようち}〔人のある時点{じてん}の〕状態{じょうたい}、立場{たちば}難局{なんきょく}、難問{なんもん}、大詰め{おおづめ} ・
\\	私は問題を抱えています。〈英〉〔募集広告{ぼしゅう こうこく}などで使われる〕仕事{しごと}、勤め口{つとめぐち}〔映画{えいが}や小説{しょうせつ}などの〕場面{ばめん}、状況{じょうきょう}レベル3、発音
\\	、カナスィチュエイション、シチュエイション、変化《複》
\\	、分節
\\	パブロフは犬たちを条件付け、ベルを聞くと唾液を分泌するようにした。 ・
\\	条件が許せば、当方は後払いを希望します。 ・
\\	その薬が利用される病気は人間が持つあらゆる病気にわたる。	変化形 
\\	【他動】~の調子{ちょうし}を整える、~を適当{てきとう}な状態{じょうたい}にする、調整{ちょうせい}する、調節{ちょうせつ}する~を条件付ける、~を条件{じょうけん}として設ける ・
\\	パブロフは犬たちを条件付け、ベルを聞くと唾液を分泌するようにした。【名】状態{じょうたい}、状況{じょうきょう}、様子{ようす}、様相{ようそう}事情{じじょう}、条件{じょうけん} ・
\\	条件が許せば、当方は後払いを希望します。健康状態{けんこう じょうたい}、調子{ちょうし}身分{みぶん}、社会的地位{しゃかいてき ちい}《文法》条件文{じょうけん ぶん}[節]◆【略】
\\	前提{ぜんてい}病気{びょうき}、疾患{しっかん} ・
\\	その薬が利用される病気は人間が持つあらゆる病気にわたる。レベル2、発音
\\	、カナコンディション、コンデション、変化《動》
\\	、分節
\\	量より質が大切。 ・
\\	そんなことどうでもいいよ。 ・
\\	本当に重要なのは手持ちのわずかなものを利用して何をするかだ。 ・
\\	それが問題になるの?/どうでもいいんじゃない?/重要なことかい? ・
\\	いくら遅くてもかまいません。 ・
\\	どうでもいいよ。/気にしないでください。/構いません。◆【用法】謝られたときの返事としても使える。 ・
\\	それはわれわれにとっては死活問題だ。 ・
\\	人々がどのように暮らすかはその人々と神の間の問題であるべきだ。	変化形 
\\	【自動】重要{じゅうよう}[問題{もんだい}]である、大きな違いがある ・
\\	量より質が大切。 ・
\\	そんなことどうでもいいよ。 ・
\\	本当に重要なのは手持ちのわずかなものを利用して何をするかだ。 ・
\\	それが問題になるの?/どうでもいいんじゃない?/重要なことかい? ・
\\	いくら遅くてもかまいません。 ・
\\	どうでもいいよ。/気にしないでください。/構いません。◆【用法】謝られたときの返事としても使える。《医》膿が出る【名】〔考慮{こうりょ}すべき〕こと、事柄{ことがら}、問題{もんだい} ・
\\	それはわれわれにとっては死活問題だ。 ・
\\	人々がどのように暮らすかはその人々と神の間の問題であるべきだ。〔困っている問題{もんだい}の〕原因{げんいん}、理由{りゆう}、根拠{こんきょ}〔話や文書{ぶんしょ}の〕中身{なかみ}、内容{ないよう}
\\	〔その時の〕状況{じょうきょう}、事態{じたい}〔感覚器官{かんかく きかん}で認識{にんしき}できる特定{とくてい}の〕物、物質{ぶっしつ}《物理》〔空間{くうかん}を占める〕物質{ぶっしつ}◆エネルギーに変換できる存在。印刷物{いんさつぶつ}、郵便物{ゆうびんぶつ}《医》〔体の〕排せつ物、膿〔おおよその〕程度{ていど}、数量{すうりょう}《法律》〔解決{かいけつ}すべき〕事件{じけん}、争点{そうてん}《哲学》〔アリストテレス哲学{てつがく}の〕質料{しつりょう}◆それ自身は形のない可能態で、変化・発展してある形相
\\	を取って現実の物になる。レベル1、発音
\\	、カナマター、変化《動》
\\	、分節
\\	その数学の問題を解くのに10分かかった。 ・
\\	あなたは(痩{や}せてるから)カロリーの心配をする必要はないですね。 ・
\\	あなたの問題はどのように起こりましたか? ・
\\	これは人ごとはないのです。	変化形 
\\	《複》
\\	【名】〔答えるべき〕問題{もんだい}、課題{かだい} ・
\\	その数学の問題を解くのに10分かかった。〔困難・害・迷惑を及ぼす解決されるべき〕問題{もんだい}困難{こんなん}な状況{じょうきょう}[事態{じたい}]、扱いにくい人◆可算 ・
\\	あなたは(痩{や}せてるから)カロリーの心配をする必要はないですね。 ・
\\	あなたの問題はどのように起こりましたか? ・
\\	これは人ごとはないのです。【形】〔人(の性格)などが〕扱いにくい、問題のあるレベル2、発音
\\	、カナプラブレム、プロブレム、変化《複》
\\	、分節
\\	ご面倒[お手数]をお掛けしてすみません。 ・
\\	御社の最新カタログを1通お送りいただけませんでしょうか。 ・
\\	どうぞお構いなく。 ・
\\	母が死んだことで何年もの間悩まされた。 ・
\\	最近あなたは変ですね。何か悩みでもあるんですか? ・
\\	関節炎にひどく苦しめられた。 ・
\\	水面が荒れていると魚が良く釣れると信じられている。 ・
\\	ひどく困難な状況にいることが分かった。 ・
\\	あなたを悩ませ、苦しませたことをおわびします。 ・
\\	どうしたんですか? ・
\\	「トラブルに巻き込まれてしまいました」「(巻き込まれたのではなく)おまえ自身がトラブルの元凶なんだよ」
\\	を強く発音する。
\\	""の
\\	は「(状況としての)困難」の意。 ・
\\	お手数掛けてすみません。 ・
\\	どう致しまして。/そんなことお安い御用です。 ・
\\	いろいろとご面倒を掛けました。 ・
\\	ご苦労さまです。 ・
\\	彼の息子は学校で騒ぎを起こしてばかりいる。 ・
\\	エンジンの故障で遅れた。 ・
\\	心臓病を患ってだいぶたつ。
\\	変化形 
\\	"【自動】わざわざ~する、骨を折る【他動】(人)に面倒{めんどう}[迷惑{めいわく}・厄介{やっかい}]を掛ける、(人)に骨を折らせる、(人)を煩わせる ・
\\	ご面倒[お手数]をお掛けしてすみません。 ・
\\	御社の最新カタログを1通お送りいただけませんでしょうか。 ・
\\	どうぞお構いなく。〔精神的{せいしんてき}に〕(人)を困らせる[悩ます・苦しめる]、(人)に心配{しんぱい}を掛ける ・
\\	母が死んだことで何年もの間悩まされた。 ・
\\	最近あなたは変ですね。何か悩みでもあるんですか?〔肉体的{にくたいてき}に〕(人)を困らせる[悩ます・苦しめる・痛めつける] ・
\\	関節炎にひどく苦しめられた。〔水面{すいめん}を〕乱す、荒らす◆【用法】通例受け身で用いられる。 ・
\\	水面が荒れていると魚が良く釣れると信じられている。【名】〔状況{じょうきょう}としての〕困難{こんなん}、悩み、苦しみ ・
\\	ひどく困難な状況にいることが分かった。 ・
\\	あなたを悩ませ、苦しませたことをおわびします。悩みの種、苦しみの元凶{げんきょう}、困難{こんなん}の原因{げんいん} ・
\\	どうしたんですか? ・
\\	「トラブルに巻き込まれてしまいました」「(巻き込まれたのではなく)おまえ自身がトラブルの元凶なんだよ」
\\	を強く発音する。
\\	""の
\\	は「(状況としての)困難」の意。骨折り{ほねおり}、厄介{やっかい}、面倒{めんどう}、手数{てすう} ・
\\	お手数掛けてすみません。 ・
\\	どう致しまして。/そんなことお安い御用です。 ・
\\	いろいろとご面倒を掛けました。 ・
\\	ご苦労さまです。事故{じこ}、争議{そうぎ}、紛争{ふんそう}、混乱{こんらん}、騒乱{そうらん}、騒ぎ ・
\\	彼の息子は学校で騒ぎを起こしてばかりいる。不具合{ふぐあい}、不良{ふりょう}、故障{こしょう}、支障{ししょう}、異常{いじょう} ・
\\	エンジンの故障で遅れた。疾患{しっかん}、病気{びょうき}、苦痛{くつう} ・
\\	心臓病を患ってだいぶたつ。弱み、弱点{じゃくてん}、欠点{けってん}レベル1、発音
\\	、カナトゥラブル、トラブル、変化《動》
\\	、分節
\\	彼らは責任を認めようとしないことで責められた。 ・
\\	それは私のせい[責任]です。◆謝罪 ・
\\	君のせい[責任]じゃないよ。 ・
\\	悪いのはおまえだ。 ・
\\	何もかも彼のせいである。/すべては彼のせいである。 ・
\\	これは誰の責任というわけでもない。 ・
\\	欠点のない人はいない。	変化形 
\\	【自動】過失{かしつ}を犯す【他動】~のあらを探す、~を…で責める ・
\\	彼らは責任を認めようとしないことで責められた。【名】責任{せきにん} ・
\\	それは私のせい[責任]です。◆謝罪 ・
\\	君のせい[責任]じゃないよ。 ・
\\	悪いのはおまえだ。 ・
\\	何もかも彼のせいである。/すべては彼のせいである。 ・
\\	これは誰の責任というわけでもない。誤り、過ち、過失{かしつ}、落ち度欠点{けってん}、至らないところ ・
\\	欠点のない人はいない。不良{ふりょう}、障害{しょうがい}《地学》断層{だんそう}◆【同】
\\	《テニス》フォールトレベル2、発音
\\	、カナフォールト、変化《動》
\\	犯罪は報われない[ペイしない・割に合わない]。 ・
\\	これで犯罪は割に合わないって分かっただろ。	変化形 
\\	《複》
\\	【名】〈集合的に〉〔罰則のある〕犯罪{はんざい}、罪{つみ}◆不可算 ・
\\	犯罪は報われない[ペイしない・割に合わない]。 ・
\\	これで犯罪は割に合わないって分かっただろ。〔罰則{ばっそく}のある〕犯罪{はんざい}[違法{いほう}]行為{こうい}◆可算〔道徳的{どうとく てき}な〕悪事{あくじ}、悪行{あくぎょう}〈文〉愚かな[嘆かわしい]行いレベル3、発音
\\	、カナクライム、変化《動》
\\	事故は起こるもの。/人生に事故はつきもの。 ・
\\	あの事故は本当にただの事故だったのだろうか。◆【用法】誰かが意図的に起こしたのではと疑って。	変化形 
\\	《複》
\\	【名】〔意図{いと}しない〕不幸{ふこう}な出来事{できごと}、災難{さいなん}、事故{じこ} ・
\\	事故は起こるもの。/人生に事故はつきもの。 ・
\\	あの事故は本当にただの事故だったのだろうか。◆【用法】誰かが意図的に起こしたのではと疑って。〔予期{よき}しない〕偶然{ぐうぜん}、運〔予期{よき}しない〕偶然{ぐうぜん}の出来事{できごと}〈話〉〔子どもの〕おもらし◆【用法】婉曲的表現〈話〉避妊{ひにん}の失敗{しっぱい}(で生まれた子ども)《哲学》偶性{ぐうせい}、偶有性{ぐうゆう せい}◆物事の非本質的な特性。レベル2、発音
\\	、カナアクスィデント、変化《複》
\\	、分節
\\	彼は我が国にとって危険な存在です。	変化形 
\\	《複》
\\	【名】脅迫{きょうはく}、脅し脅かすもの[人]、危険{きけん}な存在{そんざい} ・
\\	彼は我が国にとって危険な存在です。危険{きけん}な兆候{ちょうこう}レベル3、発音
\\	、カナスレット、変化《複》
\\	人生に恐れるべきものは何もない。 ・
\\	私たちが恐れるべき唯一のものは恐れることそのものであります。◆米国大統領フランクリン・デラノ・ルーズベルトが1933年4月3日に行った就任演説(第1期)からの一節。 ・
\\	残念だが君が正しくて私が間違っているようだ。 ・
\\	長時間の飛行機の旅行ではぜんそくが大きな心配の種です。 ・
\\	1990年代にはその国でこれらの暴動が再燃する可能性が多分にあった。	変化形 
\\	【自動】恐れる心配{しんぱい}する【他動】~を恐れる、~を怖がる ・
\\	人生に恐れるべきものは何もない。 ・
\\	私たちが恐れるべき唯一のものは恐れることそのものであります。◆米国大統領フランクリン・デラノ・ルーズベルトが1933年4月3日に行った就任演説(第1期)からの一節。~を心配{しんぱい}する、~を懸念{けねん}する〔神を〕崇敬{すうけい}する〔残念{ざんねん}なことを〕言う、〔残念{ざんねん}なことが確実{かくじつ}だと〕思う ・
\\	残念だが君が正しくて私が間違っているようだ。【名】恐れ、恐怖{きょうふ}懸念{けねん}、心配{しんぱい}、不安{ふあん}〔神への強い〕敬い、崇敬{すうけい}恐ろしいもの、心配{しんぱい}の種 ・
\\	長時間の飛行機の旅行ではぜんそくが大きな心配の種です。〔望ましくないことが起きる〕見込み{みこみ}、可能性{かのう せい} ・
\\	1990年代にはその国でこれらの暴動が再燃する可能性が多分にあった。レベル2、発音
\\	、カナフェア、フィア、変化《動》
\\	変化形 
\\	《複》
\\	【他動】〔~を重度の〕病気にかからせる【名】〔伝染病などの重度の〕病気{びょうき}、疾病{しっぺい}、疾患{しっかん}〔社会{しゃかい}などの〕病弊{びょうへい}、退廃{たいはい}、不健全{ふけんぜん}な傾向{けいこう}レベル2、発音
\\	、カナデジーズ、デズィーズ、ディズィーズ、変化《複》
\\	、分節
\\	変化形 
\\	《複》
\\	【名】《医》がん、がん腫{しゅ}、悪性腫瘍{あくせい しゅよう}〈比喩〉社会悪{しゃかいあく}レベル4、発音
\\	、カナカンサー、キャンサー、変化《複》
\\	、分節
\\	胃のあたりが痛いんです。 ・
\\	胃が痛みます。 ・
\\	もう痛みはありません。 ・
\\	ほとんど痛みは(感じ)ない。 ・
\\	なんか最近学校がしんどくってさ。 ・
\\	いろいろやることがあって面倒です。 ・
\\	面倒くさい。 ・
\\	苦しみの種は尽きない。 ・
\\	いやだねー。/納得いかないな。	変化形 
\\	【自動】痛む【他動】~に苦痛{くつう}[痛み]を与える、~を苦しめる【名】〔急性の〕痛み、痛覚{つうかく}、疼痛{とうつう} ・
\\	胃のあたりが痛いんです。 ・
\\	胃が痛みます。 ・
\\	もう痛みはありません。 ・
\\	ほとんど痛みは(感じ)ない。苦痛{くつう}、骨折り{ほねおり}、苦労{くろう}、おっくうなこと、面倒{めんどう}(なこと)、悩みの種、苦悩{くのう} ・
\\	なんか最近学校がしんどくってさ。 ・
\\	いろいろやることがあって面倒です。 ・
\\	面倒くさい。 ・
\\	苦しみの種は尽きない。不快{ふかい}な人[こと・もの] ・
\\	いやだねー。/納得いかないな。レベル2、発音
\\	、カナペイン、変化《動》
\\	変化形 
\\	《複》
\\	【名】薬、医薬{いやく}、薬剤{やくざい}、薬物{やくぶつ}〔手術{しゅじゅつ}を行わない〕医学{いがく}、医術{いじゅつ}、内科学{ないかがく}薬物{やくぶつ}療法{りょうほう}[治療{ちりょう}]〔苦しいが避けて通れない〕懲らしめ、罰〔北米先住民族{ほくべい せんじゅう みんぞく}の〕まじない◆超自然的な力によって病気を予防したり直したりするためのもの。レベル2、発音
\\	、カナメディシン、メディスン、変化《複》
\\	、分節
\\	彼女の強みは作文能力にあります。	変化形 
\\	《複》
\\	【名】力、強さ{つよさ}、体力{たいりょく}知力{ちりょく}、精神力{せいしんりょく}強み、長所{ちょうしょ} ・
\\	彼女の強みは作文能力にあります。強度{きょうど}〔溶液の〕濃度{のうど}《軍事》兵力数{へいりょく すう}レベル2、発音
\\	、カナストゥレンクス、ストレンクス、ストレングス、変化《複》
\\	変化形 
\\	《複》
\\	【名】〔下から上までの〕高さ、身長{しんちょう}、肩高{かたこう}〔基準地点{きじゅん ちてん}からの〕高さ、高度{こうど}、海抜{かいばつ}〔他より目立{めだ}って〕高いこと、〔かなりの〕高さ〔最も高い〕頂点{ちょうてん}、山頂{さんちょう}、最高地点{さいこう ちてん}〔周りが見渡せる〕高台{たかだい}、高くなっている場所{ばしょ}◆通例、
\\	〔活動{かつどう}や人気{にんき}などの〕絶頂期{ぜっちょう き}、最高潮{さいこうちょう}極端{きょくたん}な例、極地{きょくち}、極みレベル2、発音
\\	、カナハイト、変化《複》
\\	メッセージはどのような長さでもよい。	変化形 
\\	《複》
\\	【名】長さ、全長{ぜんちょう}、丈 ・
\\	メッセージはどのような長さでもよい。時間の長さ、期間{きかん}〔ワインなどの〕あと味が長いこと〔思考・言動などの〕範囲、程度〔ボートレースや競馬{けいば}の〕艇身{ていしん}、馬身{ばしん}〈性俗〉ペニスレベル2、発音
\\	、カナレンクス、レングス、変化《複》
\\	変化形 
\\	【自動】水面{すいめん}に浮上{ふじょう}する、表面{ひょうめん}に出る明るみに出る、隠していることがばれる、表れる【他動】浮上{ふじょう}させる~の表面{ひょうめん}を仕上げる、舗装{ほそう}する【名】表面{ひょうめん}、面、水面{すいめん}外見{がいけん}〔空中・水中・地下に対して〕陸上輸送、船便【形】表面{ひょうめん}の、外見上{がいけんじょう}のレベル3、発音
\\	、カナサーフェス、サーフィス、変化《動》
\\	、分節
\\	箱の底が抜ける。 ・
\\	バケツの底が抜ける。	変化形 
\\	【自動】底に達する、根拠{こんきょ}とする、基づく【他動】〔いすなどに〕座部{ざぶ}をつける【名】〔一番低いまたは深い〕底(部)〔ある物の〕下部{かぶ}、底面{ていめん} ・
\\	箱の底が抜ける。 ・
\\	バケツの底が抜ける。〔支えとなる〕基底{きてい}、基部{きぶ}〔一番遠い場所{ばしょ}の〕端〔一覧表{いちらんひょう}などの〕最後{さいご}(尾)〔地位{ちい}などの〕最下位{さいかい}、最下層{さい かそう}〔問題{もんだい}などの〕根底{こんてい}、底流{ていりゅう}〔水底{すいてい}の地面{じめん}を指す〕水底{すいてい}、湖底{こてい}、海底{かいてい}〔川沿いの〕沖積層低地{ちゅうせきそう ていち}◆通例、
\\	◆【同】
\\	《海事》船底{ふなぞこ}◆喫水線より下の部分。〈文〉《海事》船〔パジャマなどの〕(半)ズボン◆通例、
\\	〈話〉尻〔椅子{いす}の〕座る部分{ぶぶん}〔野球{やきゅう}の回の〕裏◆【対】
\\	〈卑俗〉〔同性愛{どうせいあい}などの性的関係{せいてき かんけい}で〕受動的役割{じゅどう てき やくわり}を好む人、受け◆【対】
\\	【形】底のレベル2、発音
\\	、カナバトム、ボトム、変化《動》
\\	、分節
\\	ここでは目的語が文頭に来ています。 ・
\\	父は優しくて車を修理する金を先にくれた。 ・
\\	その建物の正面は白く塗られている。 ・
\\	最前列に誰が座るかで、すったもんだの大騒ぎがありました。 ・
\\	彼は公民権運動の最前線にいた。 ・
\\	寒冷前線が本州を移動しています。	変化形 
\\	【自動】〔~に〕面する〔~の〕代役となる〈米俗〉前払いする〈米俗〉格好{かっこう}を付ける、見えを張る【他動】~に面する、~に直面{ちょくめん}する~の代表(者){だいひょう(しゃ)}になる、~を率いる、~の(陣頭{じんとう}で)指揮{しき}を執る《言語学》~を最初{さいしょ}[文頭{ぶんとう}]に置く、~を倒置{とうち}する◆語順について。文法上の要求や強調・挿入などのために、要素を文や節の先頭に置く ・
\\	ここでは目的語が文頭に来ています。(人)に前渡しする、〔前金として〕~を渡す ・
\\	父は優しくて車を修理する金を先にくれた。【名】〔建物{たてもの}などの〕正面{しょうめん}、前面{ぜんめん} ・
\\	その建物の正面は白く塗られている。〔場所{ばしょ}や位置{いち}の〕前部{ぜんぶ}、前方{ぜんぽう} ・
\\	最前列に誰が座るかで、すったもんだの大騒ぎがありました。〔建物{たてもの}の〕正面玄関{しょうめん げんかん}、玄関口{げんかん ぐち}〔本や雑誌{ざっし}の〕扉◆通例、本文の前にあるノンブルの付かないタイトル・ページ。〔川や通りなどに〕面した地域{ちいき}◆【参考】
\\	〈英〉〔行楽地{こうらくち}の〕湖畔{こはん}[河畔{かはん}・海岸{かいがん}]遊歩道{ゆうほどう}〔活動分野{かつどう ぶんや}などの〕第一線{だいいっせん}、最前線{さいぜんせん} ・
\\	彼は公民権運動の最前線にいた。〔活動中{かつどう ちゅう}の〕分野{ぶんや}、領域{りょういき}〔洋服{ようふく}の〕胸部{きょうぶ}、前部{ぜんぶ}〔ドレス・シャツの取り外せる〕胸当て、フロント◆【同】
\\	《軍事》〔展開中{てんかい ちゅう}の軍の〕前線{ぜんせん}《軍事》〔戦闘中{せんとう ちゅう}の〕戦線{せんせん}◆【同】
\\	〔気象{きしょう}の〕前線{ぜんせん} ・
\\	寒冷前線が本州を移動しています。〔政治運動{せいじ うんどう}の〕戦線{せんせん}◆共通の目標のために広範囲な勢力が連合すること。見せ掛け、ごまかし、装い〔不正{ふせい}を隠すための〕隠れみの〔力のない〕表看板{おもてかんばん}、名目上{めいもく じょう}の指導者{しどうしゃ}〈英〉厚かましさ〈古〉顔、額【形】正面{しょうめん}の、最前部{さいぜんぶ}のレベル1、発音
\\	、カナフラント、フロント、変化《動》
\\	変化形 
\\	【1名】〔物の〕後部{こうぶ}、背面{はいめん}《軍事》後方部隊{こうほう ぶたい}、後衛{こうえい}〈話〉尻、けつ◆【同】
\\	【1形】後方{こうほう}の、後部{こうぶ}の、後ろの【2自動】〔馬などが〕後ろ脚で立つ【2他動】〔子どもを〕育てる、養育{よういく}する〔動物を〕飼育{しいく}する〈文〉〔記念碑・寺院などを〕建立{こんりゅう}する〔物を〕まっすぐに立てるレベル4、発音
\\	、カナリアー、リヤ、変化《動》
\\	君はどっちの味方? ・
\\	いかなる問題にも二つの側面がある。 ・
\\	彼女にはかわいいところがある。	変化形 
\\	【自動】〔~の〕側につく◆【参考】
\\	【他動】~に側面{そくめん}をつける~と並ぶ~の側につく、~に味方{みかた}する、~を支持{しじ}する【名】《数学》〔平面図形{へいめん ずけい}の〕辺《数学》〔立体図形{りったい ずけい}の〕面〔平面{へいめん}からできている箱などの〕側面{そくめん}◆上面と底面を除く面。〔平たい物の両側{りょうがわ}の〕面◆表面や裏面など。〔物の左右{さゆう}の〕横、脇{わき}◆上・底・奥・手前を除く部分。〔観察者{かんさつしゃ}からの距離{きょり}や方角{ほうがく}による〕側◆向こう側、右側など。〔中心{ちゅうしん}からの方角{ほうがく}による〕側◆北側、東側など。〔線路{せんろ}などの障害物{しょうがいぶつ}で隔てられた〕側〔人・物・場所{ばしょ}の〕そば、脇{わき}
\\	や
\\	など。〔人や動物{どうぶつ}の左右{さゆう}の〕体側{たいそく}〔対立{たいりつ}する個人{こじん}や集団{しゅうだん}の〕側、組 ・
\\	君はどっちの味方?〔論争{ろんそう}などの意見{いけん}や立場{たちば}の〕一方{いっぽう}〔出来事{できごと}や性格{せいかく}などの〕側面{そくめん}、様相{ようそう} ・
\\	いかなる問題にも二つの側面がある。 ・
\\	彼女にはかわいいところがある。〔父方{ちちかた}や母方{ははかた}などの〕血統{けっとう}、血筋{ちすじ}〔料理{りょうり}の〕付け合わせ◆【同】
\\	〈英〉〔ビリヤードの球の〕ひねり◆【同】〈米〉
\\	【形】側面{そくめん}の、脇{わき}にある横からの、斜めの副次的{ふくじ てき}な、付随{ふずい}するレベル1、発音
\\	、カナサイド、変化《動》
\\	変化形 
\\	《複》
\\	【名】様子{ようす}、外見{がいけん}、顔つき、姿{すがた}、表情{ひょうじょう}、容姿{ようし}、容貌{ようぼう}◆【同】
\\	〔心に映る〕姿、イメージ、印象{いんしょう}性状{せいじょう}、態度{たいど}、物腰{ものごし}外観{がいかん}、景観{けいかん}、光景{こうけい}、様子{ようす}、様相{ようそう}形勢{けいせい}、局面{きょくめん}、状況{じょうきょう}、側面{そくめん}、特徴{とくちょう}〔情勢{じょうせい}・問題{もんだい}などの〕捕らえ方、解釈{かいしゃく}、角度{かくど}、見地{けんち}、見方{みかた}〔家・部屋{へや}などの〕向き、方位{ほうい}
\\	、方向{ほうこう}〔ある方向{ほうこう}に面した〕面、側《航空》アスペクト《占星》アスペクト、視座{しざ}、星・天体{てんたい}の相、星位{せいい}《生態》〔植生{しょくせい}の〕季観{きかん}《特許》態様{たいよう}レベル4、発音
\\	、カナアスペクト、変化《複》
\\	、分節
\\	これほどの規模のものは日本でも他に例がない。	変化形 
\\	【1自動】うろこや薄片{はくへん}がはがれる外皮{がいひ}がつく【1他動】〔魚のうろこを〕おとす、〔層になっている薄片{はくへん}を〕はがす、〔歯石{しせき}を〕取る〔湯垢{ゆあか}や外皮{がいひ}を〕~につける〔平らな物を〕投げて遠くにとばす、水切りさせる〈豪〉~をだます、~にただ乗りする【1名】〔魚・爬虫類の〕うろこ〔チョウ・ガの〕鱗粉{りんぷん}〔皮膚{ひふ}の〕鱗屑{りんせつ}薄片{はくへん}《植物》芽鱗{がりん}、鱗片{りんぺん}◆【参考】
\\	《動物》カイガラムシ(病)◆【参考】
\\	〔金属{きんぞく}を熱したときにできる〕薄い酸化膜{さんかまく}、スケール〔ボイラーややかんの内側{うちがわ}にできる〕湯垢{ゆあか}歯石{しせき}◆【同】
\\	【2自動】はしごで登る〔段階{だんかい}やレベルが〕上がる【2他動】〔はしごなどを〕(よじ)登る、〔山などを〕登り切る〔一定{いってい}の基準{きじゅん}に従って〕~を増やす、~を減らす〔縮尺比{しゅくしゃく ひ}に従って〕~を縮小{しゅくしょう}する、~を拡大{かくだい}する〔材木{ざいもく}の量を〕計る、見積{みつ}もる【2名】目盛{めも}り、定規{じょうぎ}、物差{ものさ}し、尺度{しゃくど}、基準{きじゅん}〔地図{ちず}などの〕縮尺比{しゅくしゃく ひ}、縮尺{しゅくしゃく}目盛り[バー]階級{かいきゅう}、等級{とうきゅう}規模{きぼ}、スケール ・
\\	これほどの規模のものは日本でも他に例がない。《数学》記数法{きすうほう}〔契約上{けいやく じょう}の〕最低賃金{さいてい ちんぎん}《音楽》音階{おんかい}【3自動】〔はかりで量った結果{けっか}〕~の重さがある【3他動】~をはかりで量る【3名】はかり、てんびん◆通例
\\	はかりの皿、てんびん皿レベル3、発音
\\	、カナスケール、変化《動》
\\	研究は多くの対象に及んだ。	変化形 
\\	【自動】動く、変動{へんどう}する並ぶ、広がっている及ぶ ・
\\	研究は多くの対象に及んだ。分布{ぶんぷ}する、生息{せいそく}する歩き回る、さまよう【他動】~を並べる、整列{せいれつ}させる、整理{せいり}する、分類{ぶんるい}する~を求めて探し回る《コ》値域{ちいき}を定める【名】〔値が変動{へんどう}する〕幅、域、範囲{はんい}〔上限{じょうげん}と下限{かげん}が定められた〕区切り{くぎり}、範囲{はんい}〔同種{どうしゅ}で異なるものの〕集まり、並び〔効果{こうか}がおよぶ〕範囲{はんい}、距離{きょり}、限界{げんかい}〔知識{ちしき}や経験{けいけん}などの〕領域{りょういき}、範囲{はんい}《軍事》射距離{しゃきょり}〔銃などの〕標的{ひょうてき}までの距離{きょり}射撃{しゃげき}(練習{れんしゅう})場◆【同】
\\	〔航空機{こうくうき}などの〕航続距離{こうぞく きょり}◆途中で給油しないで運行できる最大の距離。〔家畜{かちく}の〕放牧地{ほうぼく ち}◆【同】
\\	《音楽》〔楽器{がっき}や人間{にんげん}が出せる〕音[声]域《音楽》〔ある楽節{がくせつ}などで使う〕音域{おんいき}《地学》山脈{さんみゃく}◆【同】
\\	《数学》〔関数{かんすう}の〕値域{ちいき}〔調理用{ちょうり よう}の〕レンジ、コンロ◆通例、ガスや電気で加熱する、同時に複数の調理ができるものを指す。◆【参考】
\\	〔広い範囲{はんい}の〕移動{いどう}〔動植物{どうしょくぶつ}の〕生息{せいそく}[分布{ぶんぷ}]域〔統計{とうけい}の分布{ぶんぷ}の〕範囲{はんい}〈米〉〔タウンシップの〕レンジ◆測量区画のタウンシップ
\\	を南北の方向に(子午線に沿って)とらえた区域。レベル3、発音
\\	、カナレインジ、レンジ、変化《動》
\\	《数学》上図の三角形の面積は?	変化形 
\\	《複》
\\	【名】〔地表上{ちひょう じょう}の〕地域{ちいき}、地帯{ちたい}、地方{ちほう}、地区{ちく}◆【同】
\\	〔特定{とくてい}の目的{もくてき}に使われる〕場所{ばしょ}、エリア◆通例、何も置いていない、空いている場所を指す。面積{めんせき} ・
\\	《数学》上図の三角形の面積は?〔知識{ちしき}や経験{けいけん}などの〕領域{りょういき}、分野{ぶんや}= 
\\	《コ》領域{りょういき}、エリア◆情報を格納するための場所。レベル2、発音
\\	、カナエーリア、エアリア、エリア、変化《複》
\\	、分節
\\	変化形 
\\	【他動】〔物を〕異なる形にする、成形{せいけい}する〔考えなどを〕形作る{かたちづくる}、方向付ける〔仕組みなどを〕工夫{くふう}する、構想{こうそう}する〔考えなどを〕具現化{ぐげん か}する、具体化{ぐたいか}する《心理学》〔報酬{ほうしゅう}と罰により行動{こうどう}を〕形成{けいせい}する【名】〔物の〕外形{がいけい}、輪郭{りんかく}〔ぼんやり見える〕姿、形、影〔考えや計画{けいかく}などの〕全体像{ぜんたい ぞう}、明確{めいかく}な形〔物の〕もともとの[あるべき]形〔人の〕健康{けんこう}、調子{ちょうし}〔物事{ものごと}の〕状態{じょうたい}、様子{ようす}《数学》〔幾何学的{きかがく てき}〕形状{けいじょう}〔物を形作る〕ひな形、原型{げんけい}幻影{げんえい}、幽霊{ゆうれい}の姿レベル1、発音
\\	、カナシェイプ、変化《動》
\\	だろうと思ったよ。 ・
\\	ああ、やっぱりそうか。/じゃないかと思った…。 ・
\\	どうもふに落ちないんですけど。 ・
\\	一応お知らせしておこうと思っただけです。 ・
\\	このブログで何をしたいのか自分でもよく分からないのですが、まあ試しにやってみようと思ったわけです。 ・
\\	どうせそんなことだろうと思ったよ。 ・
\\	を図
\\	に示す。 ・
\\	《マニュアル》図2(を)参照。 ・
\\	彼女はスタイルが良い。 ・
\\	その数字は8桁です。	変化形 
\\	【自動】計算{けいさん}する ・
\\	だろうと思ったよ。人目{ひとめ}に付く、目立つ{めだつ}、卓越{たくえつ}する関係{かんけい}する、関わる、巻き込まれる〈話〉当然{とうぜん}である、予想{よそう}[思った]通りである ・
\\	ああ、やっぱりそうか。/じゃないかと思った…。【他動】~を数字{すうじ}で表す、計算{けいさん}する~を図形{ずけい}に表す、図で表す~を象徴{しょうちょう}する~を心の中に描く、~と想像{そうぞう}する~と思う[考える・判断{はんだん}する・推測{すいそく}する・推定{すいてい}する] ・
\\	どうもふに落ちないんですけど。 ・
\\	一応お知らせしておこうと思っただけです。 ・
\\	このブログで何をしたいのか自分でもよく分からないのですが、まあ試しにやってみようと思ったわけです。 ・
\\	どうせそんなことだろうと思ったよ。【名】形、形状{けいじょう}、形態{けいたい}、外観{がいかん}図、図表{ずひょう}、挿絵{さしえ}、図形{ずけい}、図式{ずしき}◆【略】
\\	を図
\\	に示す。 ・
\\	《マニュアル》図2(を)参照。フィギュア◆フィギュア・スケートで氷上に描く図形。姿{すがた}、人影{ひとかげ}人物{じんぶつ}、大人物{だいじんぶつ}、大立者{おおだてもの}〔人の〕体つき、体形{たいけい}、スタイル ・
\\	彼女はスタイルが良い。〔女性のバスト・ウェスト・ヒップの〕スリーサイズ象徴{しょうちょう}肖像{しょうぞう}、人物像{じんぶつ ぞう}、彫像{ちょうぞう}数字{すうじ}、〔数字{すうじ}の〕けた ・
\\	その数字は8桁です。数量{すうりょう}、価格{かかく}、値段{ねだん}、総額{そうがく}、合計数{ごうけい すう}言葉{ことば}のあや◆【参考】
\\	統計{とうけい}、データ◆【類】
\\	〔レンズや鏡などの〕面形状精度{めん けいじょう せいど}《論理学》〔三段論法{さんだん ろんぽう}の〕格◆大前提、小前提、結論における、小名辞、大名辞、中名辞の組み合わせのパターンで、第1格~第4格までの4種類がある。レベル2、発音
\\	、カナフィガー、フィギャ、フィギュア、変化《動》
\\	、分節
\\	変化形 
\\	【他動】~を構造化{こうぞうか}する、構築{こうちく}する~をスケジュール化する【名】構造{こうぞう}、骨組み{ほねぐみ}、仕組み{しくみ}、作り体制{たいせい}、機構{きこう}、組織{そしき}、システム〔複数の部分から作られた〕建造物{けんぞうぶつ}、構造体{こうぞうたい}、構造物{こうぞうぶつ}、営造物{えいぞうぶつ}、構築物{こうちく ぶつ}《化学》化学構造{かがく こうぞう}適切{てきせつ}に行動{こうどう}できるようにするための環境{かんきょう}[スケジュール]《囲碁》構え《コ》構造体{こうぞうたい}レベル3、発音
\\	、カナストゥラクチャ、ストラクチャー、変化《動》
\\	、分節
\\	どういう割合で混ぜましょうか。	変化形 
\\	《複》
\\	【他動】〔~を…に〕釣り合わせる、比例{ひれい}させる【名】〔~に対する〕割合{わりあい}、比率{ひりつ}、比◆【類】
\\	どういう割合で混ぜましょうか。〔全体を構成する部分間の適切な〕釣り合い、バランス、均衡{きんこう}〔全体に占める割合を考慮したときの〕一部、分け前《数学》比例{ひれい}
\\	〔他と比較したときの〕大きさ、サイズ
\\	〔物事の〕重要度{じゅうようど}レベル4、発音
\\	、カナプロポーション、変化《動》
\\	、分節
\\	世の中にはいろいろな種類の人や物が必要です。 ・
\\	会話の学校にはいろいろな種類がある。	変化形 
\\	【他動】〔種類{しゅるい}や大きさなどで〕~を分類{ぶんるい}する[仕分けする]〔別の物から〕~を取り出す[選別{せんべつ}する]◆【用法】通例
\\	の形で用いられる。《コ》ソートする【名】種類{しゅるい}、性質{せいしつ} ・
\\	世の中にはいろいろな種類の人や物が必要です。 ・
\\	会話の学校にはいろいろな種類がある。分類{ぶんるい}、仕分け{しわけ}《コ》ソート◆データを
\\	順、五十音順、数値の昇順・降順に並べ替えること〈俗〉いい女、魅力的{みりょくてき}な女◆【同】
\\	レベル2、発音
\\	、カナソート、変化《動》
\\	水面には白鳥が優雅にたたずんでいる。 ・
\\	ゆっくりお休みなさい。 ・
\\	後はご想像にお任せします。 ・
\\	後はあなたの努力次第。 ・
\\	今日はもう帰っていいよ。	変化形 
\\	【1自動】休む、休息{きゅうそく}する、休憩{きゅうけい}する、休養{きゅうよう}する、眠る静止{せいし}する〔物がある場所{ばしょ}に〕ある、置かれている ・
\\	水面には白鳥が優雅にたたずんでいる。~次第{しだい}である【1他動】~を休ませる、休息{きゅうそく}させる、休憩{きゅうけい}させる、休養{きゅうよう}させる〔問題{もんだい}などを〕そのままの状態{じょうたい}にしておく~を置く、~に基づかせる【1名】休息{きゅうそく}、休養{きゅうよう}、睡眠{すいみん}、眠り、静養{せいよう}、保養{ほよう} ・
\\	ゆっくりお休みなさい。〔機器などの〕停止{ていし}《音楽》休止、休符〔物・足などを載せる〕台◆通例、複合として用いられる。
\\	など。〔旅行者などの〕宿泊所永眠〔ビリヤードの〕キュー台◆【同】
\\	【2自動】~のままである、依然{いぜん}として~である、相変わらず~である【2名】残り、残っているもの、残りの部分{ぶぶん}、残余{ざんよ}、後(のこと) ・
\\	後はご想像にお任せします。 ・
\\	後はあなたの努力次第。 ・
\\	今日はもう帰っていいよ。その他、残った人たち〔中断{ちゅうだん}した事柄{ことがら}の〕続きレベル2、発音
\\	、カナレスト、変化《動》
\\	変化形 
\\	【他動】〔部分{ぶぶん}を少しずつ〕つなぎ合わせる、まとめ上げる〔当て布などで〕~を修理{しゅうり}する[繕う]【名】〔取り出された〕1本、1枚、1個
\\	や
\\	のように全体から取り出された物だが、それ自身で完結しているもの。〔分けられた〕一切れ{ひときれ}、一かけら、一片{いっぺん}
\\	のように全体から分けられた一部であることが分かる物。〔そろっている物の〕一品{いっぴん}、一点{いってん}◆【参考】
\\	〔全体{ぜんたい}を緊密{きんみつ}に構成{こうせい}する〕要素{ようそ}、部品{ぶひん}、構成員{こうせいいん}〔美術{びじゅつ}・音楽{おんがく}・文学{ぶんがく}などの〕作品{さくひん}、曲〔抽象的{ちゅうしょう てき}な物事{ものごと}の〕一例{いちれい}、実例{じつれい}、見本{みほん}◆【参考】
\\	〔特定{とくてい}のテーマに関する〕意見{いけん}、主張{しゅちょう}〔新聞{しんぶん}や雑誌{ざっし}の〕記事{きじ}、論説{ろんせつ}〔特定{とくてい}の価値{かち}の〕硬貨{こうか}、コイン
\\	のように用いる。〔ボードゲームの〕駒〔チェスの〕ピース◆ポーン
\\	以外の駒の総称。〈俗〉拳銃{けんじゅう}、ピストル、はじき〈俗・軽蔑的〉〔セックスの対象{たいしょう}の〕女、すけ〈卑〉セックス、性交{せいこう}〈米方言〉〔ある程度{ていど}の〕距離{きょり}レベル1、発音
\\	、カナピース、変化《動》
\\	特売場に大勢の人が集まった。 ・
\\	押さないでください。◆混雑している場所などで。 ・
\\	せっつくな。/無理を言うな。 ・
\\	年内に彼は
\\	になっているでしょう。	変化形 
\\	【1自動】押し寄せる、群がる ・
\\	特売場に大勢の人が集まった。《野球》満塁{まんるい}にする【1他動】~に押し寄せる、~に殺到{さっとう}する ・
\\	押さないでください。◆混雑している場所などで。~に迫る、~に圧力{あつりょく}をかける ・
\\	せっつくな。/無理を言うな。~に手が届く ・
\\	年内に彼は
\\	になっているでしょう。【1名】群衆{ぐんしゅう}、集まった大勢{おおぜい}の人大衆{たいしゅう}、民衆{みんしゅう}観衆{かんしゅう}、聴衆{ちょうしゅう}、観客{かんきゃく}〔年齢{ねんれい}や仕事{しごと}などの共通点{きょうつうてん}がある〕集団{しゅうだん}、グループ、仲間{なかま}〔一緒{いっしょ}になっている〕多数{たすう}[たくさん]の物【2名】《音楽》クルース◆【同】
\\	〈英〉バイオリン◆【同】
\\	レベル2、発音
\\	、カナクラウド、変化《動》
\\	月の質量は地球の約81分の1です。	変化形 
\\	《複》
\\	【自動】ひと塊になる、集結{しゅうけつ}する、集合{しゅうごう}する【他動】~をひと塊にする、集結{しゅうけつ}させる、集合{しゅうごう}させる【名】〔物質{ぶっしつ}が一体化{いったいか}した〕塊、まとまり〔多くの部分{ぶぶん}の〕集団{しゅうだん}、集まり《物理》質量{しつりょう}◆【略】
\\	月の質量は地球の約81分の1です。〔不特定{ふとくてい}の〕多数{たすう}、多量{たりょう}、大量{たいりょう}〔全体{ぜんたい}の中の〕大部分{だいぶぶん}、主要部{しゅようぶ}〔絵画{かいが}の〕マッス◆画面の中の、光や色合いが均一に近い、大きな部分。
\\	大衆{たいしゅう}、庶民{しょみん}【形】大衆{たいしゅう}の、庶民{しょみん}の大量{たいりょう}の、多数{たすう}の、大規模{だいきぼ}な〔範囲{はんい}や効果{こうか}が〕全体{ぜんたい}の、広いレベル2、発音
\\	、カナマス、マッス、変化《動》
\\	ゆうべ彼女はたっぷり眠った。 ・
\\	私もずいぶん変わりました。 ・
\\	どうもありがとう。	変化形 
\\	《複》
\\	【自動】くじを引く【他動】分け与える、区分けする【名】たくさん、1組、多量{たりょう}、多額{たがく}、いろいろくじ(引き)分け前{わけまえ}抽選{ちゅうせん}運命{うんめい}、宿命{しゅくめい}一区画{いち くかく}、用地{ようち}ロット、製造{せいぞう}の一単位{いち たんい}、商品発注時{しょうひん はっちゅう じ}の数量単位{すうりょう たんい}人の群れ、組〈話〉連中{れんちゅう}、やつ【副】
\\	大いに、たっぷり、たくさん ・
\\	ゆうべ彼女はたっぷり眠った。 ・
\\	私もずいぶん変わりました。 ・
\\	どうもありがとう。レベル2、発音
\\	、カナロット、変化《動》
\\	イディオムやフレーズ 
\\	【名】多量、多数、豊富さ【形】たくさんの、たっぷりの、豊富{ほうふ}な、十分{じゅうぶん}な【副】たくさん、たっぷりと、豊富{ほうふ}に、十分{じゅうぶん}に非常{ひじょう}に、全くレベル2、発音
\\	、カナプレンティー、分節
\\	変化形 
\\	【自動】合計{ごうけい}[総計{そうけい}・総量{そうりょう}・総額{そうがく}]~に達する[のぼる]~の状態{じょうたい}に達する、結局{けっきょく}~になる、~に等しい【名】量、(金)額
\\	総計{そうけい}、総量{そうりょう}、総額{そうがく}〔全体的{ぜんたいてき}な〕価値{かち}レベル2、発音
\\	、カナアマウント、変化《動》
\\	、分節
\\	変化形 
\\	《複》
\\	【名】〔物の持つ性質{せいしつ}としての〕量◆【対】
\\	〔ある物の〕分量{ぶんりょう}、数量{すうりょう}◆【略】
\\	大量{たいりょう}、多量{たりょう}、多数{たすう}《数学》数、量《物理》量◆【同】
\\	《論理学》〔命題{めいだい}の〕量◆伝統論理学の用語で、ある命題が全称であるか特称であるかということ。《言語学》〔韻律{いんりつ}の〕音節{おんせつ}の長短{ちょうたん}、音量{おんりょう}レベル4、発音
\\	、カナクウァンティーティー、クォンティーティー、クワンティーティー、変化《複》
\\	、分節
\\	品質はあらかじめ計画されなければならない。 ・
\\	雪質が良い。 ・
\\	良質の魚は捕まえてから数時間のうちにはらわたを取って冷凍する。	変化形 
\\	《複》
\\	【名】〔量と区別{くべつ}される〕質、性質{せいしつ}、品質{ひんしつ}◆【対】
\\	品質はあらかじめ計画されなければならない。 ・
\\	雪質が良い。〔ものに固有{こゆう}の〕本質{ほんしつ}、属性{ぞくせい}〔人の〕資質{ししつ}、素養{そよう}、人柄{ひとがら}上質{じょうしつ}、良質{りょうしつ}、高級品質{こうきゅう ひんしつ}◆【用法】名詞の前で形容詞的に用いられることが多い。〈話〉上流階級{じょうりゅう かいきゅう}の[高貴{こうき}な]身分{みぶん}◆古めかしい表現。〈話〉上流階級{じょうりゅう かいきゅう}の人◆古めかしい表現。《言語学》〔母音{ぼいん}の〕質◆口腔の形などによって特徴付けられる母音の性質。《音楽》音色{おんしょく}、音質{おんしつ}◆倍音によって特徴付けられる音の性質。《論理学》〔命題{めいだい}の〕質◆伝統論理学の用語で、ある命題が肯定か否定であること。【形】質の良い[高い]、良質{りょうしつ}の、上質{じょうしつ}の、高品質{こうひんしつ}の、素晴{すば}らしい ・
\\	良質の魚は捕まえてから数時間のうちにはらわたを取って冷凍する。上流社会{じょうりゅう しゃかい}のレベル3、発音
\\	、カナクウァリティー、クォリティー、クワリティー、変化《複》
\\	、分節
\\	四つ割りにしたチコリーをグラタン皿の底が隠れるように敷き詰める。	変化形 
\\	【自動】〔与えられた宿舎{しゅくしゃ}に〕宿泊{しゅくはく}するくまなく捜索{そうさく}する《海事》〔風が〕斜め後ろから吹く【他動】~を4分する、4等分{とうぶん}する~を宿泊{しゅくはく}させる~を部署{ぶしょ}につかせる〔紋章{もんしょう}を〕4等分{とうぶん}する〔機械{きかい}の部品{ぶひん}を〕直角{ちょっかく}に取り付ける〔ある地域{ちいき}を〕くまなく捜索{そうさく}する◆陸上や海上をジグザグ状に移動してもれなく探す。〔人を〕四つ裂きにする【名】4分の1、4半分{はんぶん}、四つ切りの一つ ・
\\	四つ割りにしたチコリーをグラタン皿の底が隠れるように敷き詰める。〈米〉25セント硬貨{こうか}、クオーター15分◆正時の15分前および15分過ぎの時間を指すときなどに使われる。3カ月、四半期{しはんき}〔4学期制{がっき せい}の〕学期{がっき}◆大学の学期で通例10~12週間。《天文》〔月の公転周期{こうてん しゅうき}の〕4分の1《天文》〔月の相の〕弦
\\	(上弦)および
\\	(下弦)を指す。◆【同】
\\	〈米〉〔重量単位{じゅうりょう たんい}の〕クオーター◆ショート・ハンドレッドウェイト
\\	の4分の1で、25ポンド、約11.34キログラムに相当する。〈英〉〔重量単位{じゅうりょう たんい}の〕クオーター◆ロング・ハンドレッドウェイト
\\	の4分の1で、28ポンド、約12.71キログラムに相当する。〈英〉〔容積単位{ようせき たんい}の〕クオーター◆穀物の容量の単位で約8ブッシェル
\\	に相当する。〈話〉〔1ポンドの〕クオーター◆1ポンドの4分の1で、4オンス、約113.4グラムに相当する。〔1ヤードや1マイルの〕クオーター◆それぞれ9インチ、2ハロン
\\	のこと。〔アメフト・バスケットボールなどの〕クオーター◆4等分されたゲーム時間の一つ。〔方位{ほうい}や羅針盤{らしんばん}の〕四方位起点{しほうい きてん}◆東西南北の一つ。〔方位{ほうい}や羅針盤{らしんばん}の〕中間方位{ちゅうかん ほうい}◆北東、南東、北西、南西の一つ。〔地球上{ちきゅう じょう}のある〕地域{ちいき}、地区{ちく}《海事》コーター◆最後尾のマストから船尾方向を見たときの、左または右の部分のいずれか。〔紋章{もんしょう}の〕クオーター◆紋章の盾を均等に4分割したうちの一つ。〔特定{とくてい}の人が住む都市{とし}の〕地区{ちく}、街区{がいく}◆通例、
\\	と表記。〔特定{とくてい}しない〕方面筋{ほうめん きん}◆情報源などを明らかにしないときに使う。◆通例、
\\	と表記。〔靴の〕クオーター◆かかとからつま先までの甲革を指す。〔馬の〕蹄側{ていそく}◆ひづめの左右の側面のいずれか。
\\	〔軍隊{ぐんたい}の〕営舎{えいしゃ}◆軍人やその家族の住む兵舎。〔艦船内{かんせん ない}の乗組員{のりくみいん}の〕配備地点{はいび ちてん}◆通例、
\\	と表記。〔通例降伏した敵に示す〕慈悲心{じひしん}、寛大{かんだい}な処置{しょち}【形】4分の1の、4半分{はんぶん}の四つ割りにした~、~の四つ割りレベル1、発音
\\	、カナクウォータ、クォーター、変化《動》
\\	、分節
\\	私はドーナツを1ダース、卵を2ダース買いました。◆名詞の前の
\\	は形容詞に似ているが、それ自身に
\\	や
\\	が付く点が形容詞と異なる。 ・
\\	そのカキはとてもおいしく、私は2ダースを平らげました。 ・
\\	問題が何十個もあります。	変化形 
\\	《複》
\\	【名】〔同種{どうしゅ}の物の〕12個、ダース◆【略】
\\	◆数詞などの後ろでは、単数形と同形の複数形が使われる。
\\	の形は、漠然と「ダース単位の数(数十から約100)」を表す場合に使われる。 ・
\\	私はドーナツを1ダース、卵を2ダース買いました。◆名詞の前の
\\	は形容詞に似ているが、それ自身に
\\	や
\\	が付く点が形容詞と異なる。 ・
\\	そのカキはとてもおいしく、私は2ダースを平らげました。 ・
\\	問題が何十個もあります。【形】12の、1ダースの〈話〉かなりたくさんのレベル2、発音
\\	、カナダース、ダズン、変化《複》
\\	、分節
\\	変化形 
\\	《複》
\\	【他動】〔文書{ぶんしょ}などの〕出典{しゅってん}を明らかにする〔会社{かいしゃ}が部品{ぶひん}や材料{ざいりょう}を〕調達{ちょうたつ}する【名】もと、源、起源{きげん}◆物事が始まった場所や、その元となった人や物事。〔情報{じょうほう}の〕発信元{はっしん もと}、情報源{じょうほうげん}、消息筋{しょうそくすじ}◆【用法】通例
\\	《地学》〔川などの〕水源(地){すいげん(ち)}《電気》〔電界効果型{でんかい こうか がた}トランジスタの〕ソース《コ》ソース・コード◆【同】
\\	レベル4、発音
\\	、カナソース、変化《動》
\\	変化形 
\\	《複》
\\	【他動】〔組織{そしき}・部署{ぶしょ}などに〕調達{ちょうたつ}する、提供{ていきょう}する、準備{じゅんび}する◆【用法】
\\	目的語 
\\	資金・設備など【名】〔頼みとなる〕援助{えんじょ}、手助け{てだすけ}〔予備{よび}の〕資金{しきん}、要員{よういん}、供給源{きょうきゅうげん}〔問題解決{もんだい かいけつ}のための〕機転{きてん}、手腕{しゅわん}天然資源{てんねん しげん}◆【同】
\\	〔いざというときの人の〕底力{そこぢから}、資質{ししつ}
\\	〔国全体{くに ぜんたい}の〕資源{しげん}、資力{しりょく}、国力{こくりょく}
\\	〔会社{かいしゃ}の〕資産{しさん}、財産{ざいさん}レベル4、発音
\\	、カナ
\\	リーソース、
\\	リゾース、変化《複》
\\	、分節
\\	私は、食卓のテーブルクロスを作るための生地を買いました。 ・
\\	彼は指導者の器ではありません。	変化形 
\\	《複》
\\	【名】〔物を作っている〕原料{げんりょう}、材料{ざいりょう}〔織物の〕素材{そざい}、生地{きじ}、服地{ふくじ} ・
\\	私は、食卓のテーブルクロスを作るための生地を買いました。〔本や映画などの〕資料{しりょう}、題材{だいざい}、ネタ
\\	〔ある作業{さぎょう}の〕器具{きぐ}、用具{ようぐ}~に向いている人、~としての資質{ししつ}のある人 ・
\\	彼は指導者の器ではありません。【形】物質{ぶっしつ}でできた、物質的{ぶっしつてき}な形のある、有形{ゆうけい}の、実体{じったい}の肉体的{にくたいてき}な、身体的{しんたいてき}な物欲的{ぶつよく てき}な、世俗的{せぞくてき}な重要{じゅうよう}な、適切{てきせつ}なレベル2、発音
\\	、カナマティアリアル、マテリアル、変化《複》
\\	、分節
\\	先般の増税は、大企業よりも中小企業にはるかに大きな打撃を与えた。 ・
\\	それはあなたの問題でしょう。私は関係ありませんよ。 ・
\\	それは彼が決めることだ。あなたがとやかく言うことではありません。 ・
\\	私が何をしようと、あなたには関係ないでしょう。	変化形 
\\	《複》
\\	【名】〔従事{じゅうじ}している特定{とくてい}の〕業種{ぎょうしゅ}、職種{しょくしゅ}、業界{ぎょうかい}、ビジネス◆不可算〔営利事業{えいり じぎょう}を行う〕会社{かいしゃ}、企業{きぎょう}、事業所{じぎょうしょ}◆可算 ・
\\	先般の増税は、大企業よりも中小企業にはるかに大きな打撃を与えた。〔物やサービスを売買{ばいばい}する〕商取引{しょうとりひき}、ビジネス、商売{しょうばい}〔特定{とくてい}の時期{じき}の〕取引{とりひき}高[状況{じょうきょう}]取引相手{とりひき あいて}、顧客{こきゃく}商慣行{かんこう}[習慣{しゅうかん}]〔個人的{こじん てき}な〕事柄{ことがら} ・
\\	それはあなたの問題でしょう。私は関係ありませんよ。 ・
\\	それは彼が決めることだ。あなたがとやかく言うことではありません。 ・
\\	私が何をしようと、あなたには関係ないでしょう。〔困難{こんなん}で厄介{やっかい}な〕状況{じょうきょう}、問題{もんだい}〔真剣{しんけん}に取り組むべき〕課題{かだい}、仕事{しごと}〔役者{やくしゃ}のさりげない〕所作{しょさ}、しぐさ◆【同】
\\	〈話〉小言{こごと}、苦言{くげん}、目玉{めだま}〈話〉素晴{すば}らしいもの〈古〉忙しいこと、多忙{たぼう}【形】ビジネス[仕事{しごと}]の、商売上{しょうばいじょう}のレベル1、発音
\\	、カナビジネス、ビズィネス、ビズネス、変化《複》
\\	、分節
\\	変化形 
\\	《複》
\\	【名】節約{せつやく}、倹約{けんやく}経済{けいざい}、景気{けいき}、経済活動{けいざい かつどう}経済圏{けいざいけん}、経済組織{けいざい そしき}〔動き・言葉などの〕簡潔{かんけつ}さ、無駄{むだ}のなさ【副】割引値段{わりびき ねだん}でレベル3、発音
\\	、カナイカーノミ、イコノミー、エコノミー、変化《複》
\\	、分節
\\	【名】〔大規模{だいきぼ}な〕商業{しょうぎょう}、交易{こうえき}、貿易{ぼうえき}〔学問{がくもん}としての〕商科{しょうか}、商業学{しょうぎょうがく}〈古〉〔知的{ちてき}・社会的{しゃかい てき}〕交流{こうりゅう}、交際{こうさい}〈古〉性的交渉{せいてき こうしょう}レベル4、発音
\\	、カナコマース、分節
\\	変化形 
\\	《複》
\\	【名】〔大規模{だいきぼ}な〕工業{こうぎょう}、産業{さんぎょう}、製造業{せいぞうぎょう}〔工業{こうぎょう}のある分野{ぶんや}の〕~工業{こうぎょう}、~産業{さんぎょう}
\\	などのように分野を示す語と共に用いる。〔経済界{けいざいかい}の中の〕工業界{こうぎょう かい}、産業界{さんぎょうかい}〈文〉勤勉{きんべん}、精励{せいれい}レベル2、発音
\\	、カナインダストゥリ、インダストリー、変化《複》
\\	、分節
\\	変化形 
\\	《複》
\\	【名】〔物の大量製造{たいりょう せいぞう}・加工{かこう}・組立{くみたて}を行う〕工場{こうじょう}、製造{せいぞう}[製作{せいさく}]所〈話〉〔同じものが大量{たいりょう}に作られる比喩{ひゆ}としての〕工場{こうじょう}、製造所{せいぞう じょ}〈古〉〔代理人{だいりにん}が運営{うんえい}する〕海外交易所{かいがい こうえき じょ}レベル1、発音
\\	、カナファクトリー、変化《複》
\\	、分節
\\	この仕事はまるでシーシュポスの罰のようだ。◆【参考】
\\	陣痛が始まったんだと思います。	イディオムやフレーズ 
\\	【自動】〔汗水流して〕働く、労働{ろうどう}する努力{どりょく}する、苦労{くろう}する、骨折る{ほねおる}骨折って進む〔不利益{ふりえき}などにより〕苦労{くろう}する、苦しむ生みの苦しみを味わう【他動】~を徹底的{てってい てき}に処理{しょり}する、~を詳細{しょうさい}に論じる~を苦しめる、~を悩ませる【名】〔肉体的な・骨の折れる〕仕事{しごと}、労務{ろうむ}〔集合的{しゅうごう てき}に〕(単純{たんじゅん})労働者{ろうどうしゃ}〔通例骨が折れる特定{とくてい}の〕作業{さぎょう}、仕事{しごと}◆通例
\\	この仕事はまるでシーシュポスの罰のようだ。◆【参考】
\\	〔特定{とくてい}の仕事{しごと}の〕労働{ろうどう}者[力]労働組合{ろうどう くみあい}(運動{うんどう})
\\	〈英〉労働党{ろうどうとう}陣痛{じんつう}◆【同】
\\	陣痛が始まったんだと思います。【形】(肉体{にくたい})労働(者){ろうどう(しゃ)}の、労働組合{ろうどう くみあい}の
\\	〈英〉労働党{ろうどうとう}のレベル2、発音
\\	、カナレイバー、変化《動》
\\	、分節
\\	席を替わってもらえますか。 ・
\\	「ブルブル。今朝は寒いよ」「天気の取り換えっこしよう!」「そっち暑いの?」「カ氏80度ある、夜の10時半だっていうのに…。今日は90度まで上がった」 ・
\\	自分の仕事は決して忘れないものだ。/昔とった杵柄。
\\	変化形 
\\	"【自動】商売{しょうばい}をする、商業{しょうぎょう}に従事{じゅうじ}する人や物を交換{こうかん}する〔特定{とくてい}の店で〕買い物をする【他動】〔商品{しょうひん}などを〕売買{ばいばい}する、取引{とりひき}する〔人や物を〕交換{こうかん}[トレード]する ・
\\	席を替わってもらえますか。 ・
\\	「ブルブル。今朝は寒いよ」「天気の取り換えっこしよう!」「そっち暑いの?」「カ氏80度ある、夜の10時半だっていうのに…。今日は90度まで上がった」〔悪口{わるぐち}などを〕やりとりする、応酬{おうしゅう}する【名】〔特定{とくてい}の〕業界{ぎょうかい}、産業{さんぎょう}〔ある企業や業界の〕顧客{こきゃく}〔一般に〕職業{しょくぎょう}、家業{かぎょう} ・
\\	自分の仕事は決して忘れないものだ。/昔とった杵柄。〔手先{てさき}の熟練{じゅくれん}を要する〕職業{しょくぎょう}、仕事{しごと}〔特定{とくてい}の業界{ぎょうかい}の〕同業者{どうぎょうしゃ}〔業種{ぎょうしゅ}としての〕商業{しょうぎょう}、通商{つうしょう}、貿易{ぼうえき}〔商品{しょうひん}の〕売買{ばいばい}、取引{とりひき}、物々交換{ぶつぶつこうかん}〔人や物の〕交換{こうかん}、トレード
\\	貿易風{ぼうえきふう}◆通例、
\\	◆【同】
\\	レベル2、発音
\\	、カナトゥレイド、トレイド、変化《動》
\\	変化形 
\\	《単》
\\	【名】商品{しょうひん}、物品{ぶっぴん}、品物{しなもの}貨物{かもつ}財{ざい}、財貨{ざいか}、家財{かざい}、財産{ざいさん}、所有物{しょゆうぶつ}服地{ふくじ}、反物{たんもの}、織物{おりもの}〈俗〉犯罪{はんざい}の証拠{しょうこ}[情報{じょうほう}]◆【参考】
\\	おあつらえ向きの物[人]レベル3、発音
\\	、カナグッズ
\\	変化形 
\\	《複》
\\	【名】〔専門家{せんもんか}に助言{じょげん}を求める〕依頼人{いらいにん}◆【語源】ラテン語
\\	(忠告を聞く人)〔弁護士{べんごし}への〕訴訟依頼人{そしょう いらいにん}〔仕事上{しごと じょう}の〕客、顧客{こきゃく}、得意{とくい}[取引{とりひき}]先、お得意様[さん]建築主{けんちく しゅ}、施主{せしゅ}クライアント◆コンピューターのネットワーク上で、サーバにつながっているコンピューターのこと。レベル3、発音
\\	、カナクライアント、クライエント、変化《複》
\\	、分節
\\	変化形 
\\	《複》
\\	【名】〔共和国{きょうわこく}の〕大統領{だいとうりょう}、〔中国{ちゅうごく}の〕国家主席{こっか しゅせき}、〔台湾{たいわん}の〕総統{そうとう}◆通例
\\	◆【略】
\\	アメリカ合衆国大統領{がっしゅうこく だいとうりょう}◆【略】
\\	〔会や会議{かいぎ}などの〕会長{かいちょう}、議長{ぎちょう}、委員長{いいんちょう}◆【略】
\\	〔大学{だいがく}などの〕総長{そうちょう}、学長{がくちょう}、校長{こうちょう}、理事長{りじちょう}◆【略】
\\	〔政府機関{せいふ きかん}の〕総裁{そうさい}、長官{ちょうかん}◆【略】
\\	〔会社{かいしゃ}の〕社長{しゃちょう}、代表取締役{だいひょう とりしまりやく}、頭取{とうどり}◆【略】
\\	レベル2、発音
\\	、カナプレジデント、プレズィデント、変化《複》
\\	、分節
\\	人の善悪は、その交わる友達によって分かる。 ・
\\	悪い仲間といるよりは一人でいる方が良い。 ・
\\	君が一緒なら心強い。 ・
\\	客を呼んで夕食を一緒に食べた。 ・
\\	散らかっていてごめんね。来客があるとは思わなかったので…。◆【場面】予定外の訪問客などで部屋が片付いていない。	変化形 
\\	《複》
\\	【名】会社{かいしゃ}、企業{きぎょう}、商社{しょうしゃ}、法人{ほうじん}◆【略】
\\	仲間{なかま}、友達{ともだち}、友人{ゆうじん} ・
\\	人の善悪は、その交わる友達によって分かる。 ・
\\	悪い仲間といるよりは一人でいる方が良い。独りではなくて他の人といっしょにいること、同伴{どうはん} ・
\\	君が一緒なら心強い。親交{しんこう}、交わり、交友{こうゆう}、交際{こうさい}訪問者{ほうもんしゃ}、来客{らいきゃく} ・
\\	客を呼んで夕食を一緒に食べた。 ・
\\	散らかっていてごめんね。来客があるとは思わなかったので…。◆【場面】予定外の訪問客などで部屋が片付いていない。同席{どうせき}の人〔仕事・演技・演奏などをいっしょにする〕集団{しゅうだん}、一団{いちだん}〔2個小隊以上の〕中隊{ちゅうたい}〔比較的小規模な〕隊{たい}〔陸軍の歩兵〕中隊{ちゅうたい}
\\	〈米俗〉米国中央情報局{べいこく ちゅうおう じょうほうきょく}◆【同】
\\	レベル1、発音
\\	、カナカンパニー、変化《複》
\\	、分節
\\	変化形 
\\	《複》
\\	【名】《法律》法人{ほうじん}◆民間企業、政府の団体、教育団体など、人格を付与された団体を指す。ただし、イギリスでは単独法人
\\	をも含む概念であり、また民間企業は
\\	と呼ばれる点でアメリカとは意味が異なる。〈英〉〔独立{どくりつ}した〕自治体{じちたい}◆【同】
\\	〔一般{いっぱん}に一つの実体{じったい}と見なされる〕団体{だんたい}レベル4、発音
\\	、カナコーポレイション、変化《複》
\\	、分節
\\	変化形 
\\	《複》
\\	【名】専門的職業{せんもん てき しょくぎょう}、知的職業{ちてき しょくぎょう}団体{だんたい}、会◆専門的、学術的な。〔信念・考え・気持ちなどの〕表明{ひょうめい}、宣言{せんげん}、公言{こうげん}、告白{こくはく}◆【同】
\\	レベル3、発音
\\	、カナプロフェッション、変化《複》
\\	、分節
\\	変化形 
\\	《複》
\\	【他動】~に仕事{しごと}を課する、~に苛酷{かこく}な負担{ふたん}をかける【名】〔与えられた〕任務{にんむ}、課題{かだい}〔困難な・つらい〕仕事{しごと}、職務{しょくむ}〔重要{じゅうよう}な〕役割{やくわり}、目的{もくてき}《コ》タスク◆コンピューターが処理する仕事の単位◆【参考】
\\	レベル4、発音
\\	、カナタスク、変化《動》
\\	変化形 
\\	《複》
\\	【名】〔道徳的・法的な〕義務{ぎむ}、務め、責務{せきむ}職務{しょくむ}、任務{にんむ}◆通例
\\	税(金){ぜい(きん)}◆【同】
\\	〔機関{きかん}の〕負荷{ふか}〔幼児{ようじ}の〕お通じレベル2、発音
\\	、カナデューティー、変化《複》
\\	、分節
\\	このバッテリーパックの良いところは、15分で充電できることです。 ・
\\	彼らはこのサービスにいくら請求しているのですか? ・
\\	彼は、飲酒運転で告発されました。	変化形 
\\	【自動】突撃{とつげき}する、突進{とっしん}する、突然進む〔商品代金などの支払いを〕請求{せいきゅう}する〔裁判官{さいばんかん}が〕陪審員{ばいしんいん}に説示{せつじ}する〔犬が人から〕命令{めいれい}を受けて座る〔電池や機器がある時間で〕充電{じゅうでん}できる ・
\\	このバッテリーパックの良いところは、15分で充電できることです。【他動】〔支払い・代価{だいか}・代金{だいきん}を〕請求{せいきゅう}する ・
\\	彼らはこのサービスにいくら請求しているのですか?〔負担{ふたん}を〕かける〔商品を〕つけ[クレジットカード]で買う〔罪・責めなどを〕~に負わせる、~のせいにする~に…を委ねる、課する、委託{いたく}する、託す、預ける~を…で非難{ひなん}する、責める、とがめる、告諭{こくゆ}する、告訴{こくそ}する、起訴{きそ}する、~だと非難{ひなん}して言う、~に説諭{せつゆ}する ・
\\	彼は、飲酒運転で告発されました。~に…を詰める、(水など)に炭酸{たんさん}ガスを加える、銃に~を装填{そうてん}する、火薬{かやく}を詰める、~に…を満たす、入れる、積み込む~を充電{じゅうでん}する、~に帯電{たいでん}させる~を襲撃{しゅうげき}する、~に体当たりする、突撃{とつげき}する~するよう説示{せつじ}する、指令{しれい}する、命じる~に…がみなぎる、~を…で飽和{ほうわ}させる【名】〔役務に対する〕料金{りょうきん}、使用料{しよう りょう}、手数料{てすうりょう}借方記入{かりかた きにゅう}、つけ、負債{ふさい}、課税金{かぜいきん}、税金{ぜいきん}、課徴金{かちょうきん}、負担{ふたん}、掛かり義務{ぎむ}、責任{せきにん}、問責{もんせき}管理{かんり}、運営{うんえい}、保護{ほご}、監督{かんとく}委託{いたく}、預かり物、保管{ほかん}説示{せつじ}、指令{しれい}、命令{めいれい}、訓令{くんれい}告訴{こくそ}、告発{こくはつ}、罪、非難{ひなん}、嫌疑{けんぎ}、容疑{ようぎ}充電{じゅうでん}、帯電{たいでん}、電荷{でんか}◆【略】
\\	装填{そうてん}、装薬{そうやく}突進{とっしん}、攻撃{こうげき}、突撃{とつげき}〔紋章{もんしょう}の〕図案{ずあん}、模様{もよう}〔容器{ようき}の〕分量{ぶんりょう}、〔機械{きかい}の動作{どうさ}などの〕必要量{ひつようりょう}〔一度{いちど}に爆発{ばくはつ}する量の〕弾薬{だんやく}◆【同】
\\	レベル2、発音
\\	、カナチャージ、変化《動》
\\	この地域は近い将来交通渋滞が予想される。	変化形 
\\	【自動】突き出る、突出{とっしゅつ}する◆【反】
\\	提案{ていあん}する【他動】~を投影{とうえい}する、発射{はっしゃ}する、投げ出す〔印象{いんしょう}・イメージを〕与える、醸し出す、伝える~を計画{けいかく}する、企画{きかく}する、考案{こうあん}する~と見積{みつ}もる、推定{すいてい}する、予想{よそう}する ・
\\	この地域は近い将来交通渋滞が予想される。~を突き出す、突出{とっしゅつ}させる◆【反】
\\	~を提案{ていあん}[提示{ていじ}]する【名】計画{けいかく}、企画{きかく}事業{じぎょう}(計画{けいかく})、プロジェクト研究課題{けんきゅう かだい}= 
\\	〈米俗〉公営{こうえい}[公共{こうきょう}]住宅{じゅうたく}◆【同】
\\	レベル3、発音《名》
\\	《動》
\\	、カナプロジェクト、変化《動》
\\	、分節
\\	私は、このトレーニングコースがあなたのためになると確信しています。 ・
\\	ウェブデザイナー[ホームページ制作者]には、この研修会が役立ちます。	変化形 
\\	【自動】得をする、利益{りえき}を得る ・
\\	私は、このトレーニングコースがあなたのためになると確信しています。 ・
\\	ウェブデザイナー[ホームページ制作者]には、この研修会が役立ちます。【他動】~のため[プラス・利益{りえき}]になる【名】便益{べんえき}、恩恵{おんけい}、利益{りえき}[ため]になること援助{えんじょ}、手助け{てだすけ}〔金銭的{きんせん てき}〕利益{りえき}、利得{りとく}〔社会福祉の〕給付金{きゅうふきん}、福祉手当{ふくし てあて}、扶助金{ふじょきん}◆通例
\\	慈善{じぜん}[募金{ぼきん}]興行{こうぎょう}◆【同】
\\	【語源】ラテン語
\\	(良いこと)
\\	(行う)レベル3、発音
\\	、カナベニフィット、ベネフィット、変化《動》
\\	、分節
\\	10万円の利益。	変化形 
\\	【自動】益する、利益{りえき}を得る、もうける、恩恵{おんけい}を受ける、得をする、役に立つ、ためになる【名】利益{りえき}、利潤{りじゅん}、得 ・100,000 
\\	10万円の利益。レベル2、発音
\\	、カナプラフィット、プロフィット、変化《動》
\\	、分節
\\	アイスクリームを分けてくれないでありがとうね!◆【場面】分けてくれてもいいだろうという皮肉または催促。 ・
\\	「あっ、おなかが痛い」「アイスクリームを独り占めするからそういうことになるんですよ」 ・
\\	あなたと同意見です。
\\	変化形 
\\	"【1自動】共同使用{きょうどう しよう}する、共有{きょうゆう}する責任{せきにん}などを共に負う[担う](人)に使わせる、(人)と一緒{いっしょ}に食べる【1他動】〔物や場所{ばしょ}を〕共同使用{きょうどう しよう}する、共有{きょうゆう}する〔責任{せきにん}を〕共に負う[担う]〔物を〕(人)に使わせる、(人)と一緒{いっしょ}に食べる ・
\\	アイスクリームを分けてくれないでありがとうね!◆【場面】分けてくれてもいいだろうという皮肉または催促。 ・
\\	「あっ、おなかが痛い」「アイスクリームを独り占めするからそういうことになるんですよ」〔金などを〕(人)に分ける[分配{ぶんぱい}する]〔感情{かんじょう}や経験{けいけん}を〕共有{きょうゆう}する、共に味わう ・
\\	あなたと同意見です。〔感情{かんじょう}などを〕(人)に伝える【1名】分かち合い、共有{きょうゆう}分け前、割り当て
\\	分担金{ぶんたんきん}役割{やくわり}、負担{ふたん}市場占有率{しじょう せんゆう りつ}、分担率{ぶんたんりつ}〔売買の単位としての〕株{かぶ}、株式{かぶしき}◆【略】
\\	【2名】= 
\\	レベル2、発音
\\	、カナシェアー、変化《動》
\\	それには何かしら良い点があるに違いない。 ・
\\	現職は選挙時に挑戦者よりも有利な立場にある。	変化形 
\\	《複》
\\	【自動】利益{りえき}を得る【他動】~に利益{りえき}をもたらす、役立つ{やくだつ}、~のためになる~を促進{そくしん}する、助長{じょちょう}する、~に弾みをつける【名】好都合{こうつごう}[有益{ゆうえき}]な点 ・
\\	それには何かしら良い点があるに違いない。得{とく}、利益{りえき}強み、長所{ちょうしょ}、優位{ゆうい} ・
\\	現職は選挙時に挑戦者よりも有利な立場にある。《スポーツ》アドバンテージ◆【参考】
\\	◆【同】
\\	◆テニス、バレーボール。ジュース
\\	の最初の得点。レベル3、発音
\\	、カナアドバネィジ、アドゥバンテッジ、アドバンテイジ、アドバンテッジ、変化《複》
\\	、分節
\\	その改革は急速に進んできた。 ・
\\	10代になるとますます自意識が過剰になった。 ・
\\	私(の体調)は順調に回復しています。 ・
\\	発掘が進んでもう一つの遺跡が現れた。 ・
\\	私はその会社で徐々に高い地位に就くようになった。 ・
\\	一部のアナリストは進展が遅過ぎると感じているようだ。 ・
\\	その薬は新しく作られた骨の周りにある細胞の成長を阻害する。	変化形 
\\	【自動】〔物事が〕進む、前進する ・
\\	その改革は急速に進んできた。 ・
\\	10代になるとますます自意識が過剰になった。 ・
\\	私(の体調)は順調に回復しています。 ・
\\	発掘が進んでもう一つの遺跡が現れた。〔絶えず〕進歩する、向上する ・
\\	私はその会社で徐々に高い地位に就くようになった。【他動】~を進ませる、~を達成させる【名】前進、進展 ・
\\	一部のアナリストは進展が遅過ぎると感じているようだ。発達、進歩、向上成長、発育 ・
\\	その薬は新しく作られた骨の周りにある細胞の成長を阻害する。〔作業などの〕進捗{しんちょく}、進み具合
\\	〔王室の〕行幸レベル2、発音《名》
\\	《動》
\\	、カナプログレス、変化《動》
\\	、分節
\\	彼は立派に面接を受けたが、それでもその職を得られなかった。 ・
\\	《就職》本日は当社の面接にお越しいただきましてありがとうございます。	変化形 
\\	【自動】インタビューする、対談{たいだん}する、面接{めんせつ}する、面接{めんせつ}を受ける ・
\\	彼は立派に面接を受けたが、それでもその職を得られなかった。【他動】~と会見{かいけん}する、~にインタビューする【名】面接{めんせつ}、面会{めんかい}、会談{かいだん} ・
\\	《就職》本日は当社の面接にお越しいただきましてありがとうございます。インタビュー、記者会見{きしゃ かいけん}、取材訪問{しゅざい ほうもん}、尋問{じんもん}〈話〉インタビューを受ける人レベル3、発音
\\	、カナインタビュー、インタブュー、変化《動》
\\	、分節
\\	変化形 
\\	《複》
\\	【名】所得{しょとく}、収入{しゅうにゅう}、収益{しゅうえき}〈古〉入ること、流入{りゅうにゅう}レベル3、発音
\\	、カナインカム、変化《複》
\\	、分節
\\	変化形 
\\	《複》
\\	【名】〔サービスに対する〕料金{りょうきん}、手数料{てすうりょう}、謝礼{しゃれい}〔機関{きかん}に支払う〕会費{かいひ}、入場料{にゅうじょうりょう}、免許料{めんきょりょう}、入学金{にゅうがくきん}、授業料{じゅぎょうりょう}《法律》相続{そうぞく}(可能{かのう})不動産{ふどうさん}〈古〉チップ、心付け{こころづけ}〔封建制において領主から与えられる〕封土(権)◆【同】
\\	レベル3、発音
\\	、カナフィー、変化《複》
\\	【名】〔膨大{ぼうだい}な〕富、財産{ざいさん}裕福{ゆうふく}[富裕{ふゆう}]であること豊富{ほうふ}[大量{たいりょう}]にあること《経済》〔個人{こじん}や共同体{きょうどうたい}の〕富レベル2、発音
\\	、カナウェルス
\\	貧すれば鈍する。 ・
\\	その組織に関する情報が慢性的に不足している。 ・
\\	野菜中の栄養素の量は、育てられた土地の豊かさや痩{や}せ具合によって異なってくる。		【名】貧乏{びんぼう}、貧困{ひんこん}、貧窮{ひんきゅう} ・
\\	貧すれば鈍する。欠乏{けつぼう}、不足{ふそく}、欠如{けつじょ} ・
\\	その組織に関する情報が慢性的に不足している。〔土地{とち}などの〕不毛{ふもう}、痩{や}せ具合{ぐあい} ・
\\	野菜中の栄養素の量は、育てられた土地の豊かさや痩{や}せ具合によって異なってくる。レベル4、発音
\\	、カナパバティー、ポバティー、分節
\\	ひもじいときにまずいものなし。/空腹なときは何でもおいしい。	変化形 
\\	【自動】飢える、渇望{かつぼう}する【名】飢え、飢饉{ききん}、空腹{くうふく}(感) ・
\\	ひもじいときにまずいものなし。/空腹なときは何でもおいしい。渇望{かつぼう}、熱望{ねつぼう}レベル3、発音
\\	、カナハンガー、変化《動》
\\	、分節
\\	男は皆彼女の形の良い脚を見たいと切望した。	変化形 
\\	【他動】喉が渇く、切望{せつぼう}[渇望{かつぼう}]する ・
\\	男は皆彼女の形の良い脚を見たいと切望した。【名】渇き、のどの渇き、口渇{こうかつ}強い欲望{よくぼう}、渇望{かつぼう}、熱望{ねつぼう}、切望{せつぼう}レベル4、発音
\\	、カナサースト、変化《動》
\\	あの子はそんなに格好良くないけど、正直な点は評価されるべきよ。 ・
\\	彼女の気持ちを大事にしたいんだ。 ・
\\	ドルはここ1週間で3%その価値を下げた。 ・
\\	《数学》
\\	の値は何ですか? ・
\\	《コ》その値はゼロに設定される。	変化形 
\\	【他動】〔物の価値{かち}を〕評価{ひょうか}する、査定{さてい}する、見積{みつ}もる〔~の能力{のうりょく}や有用性{ゆうよう せい}などを〕評価{ひょうか}する、見極める ・
\\	あの子はそんなに格好良くないけど、正直な点は評価されるべきよ。~を高く評価{ひょうか}する、~を尊重{そんちょう}する ・
\\	彼女の気持ちを大事にしたいんだ。【名】〔物の適正{てきせい}と考えられる〕価値{かち}、値打{ねうち}ち〔金銭的{きんせん てき}な〕値段{ねだん}、価額{かがく}、価格{かかく}〔持ち主にとっての〕有用性{ゆうよう せい}、ありがたみ
\\	〔望ましい〕価値{かち}観[基準{きじゅん}]〔貨幣{かへい}や切手{きって}などの〕額面{がくめん}(価格{かかく}) ・
\\	ドルはここ1週間で3%その価値を下げた。〔言葉{ことば}の〕本当{ほんとう}[額面通り]の意味{いみ}《数学》(数)値 ・
\\	《数学》
\\	の値は何ですか? ・
\\	《コ》その値はゼロに設定される。《音楽》〔音符{おんぷ}や休止符{きゅうしふ}の〕長さ〔色の明るさの〕色値《言語学》音価{おんか}◆【同】
\\	レベル2、発音
\\	、カナバリュー、変化《動》
\\	、分節
\\	変化形 
\\	《複》
\\	【名】《地学》地域{ちいき}、地帯{ちたい}、地方{ちほう}◆地球上の地理的、政治的、文化的に他と区別された部分。〔行政上{ぎょうせい じょう}の〕区〔動植物{どうしょくぶつ}の〕分布範囲{ぶんぷ はんい}活動{かつどう}[関心{かんしん}]分野{ぶんや}[領域{りょういき}]〔体の器官{きかん}や組織{そしき}の〕部(位)、領域{りょういき}〔変動{へんどう}する数値{すうち}などの〕域、幅◆【同】
\\	〔無限{むげん}に広がる〕広大{こうだい}な領域{りょういき}レベル4、発音
\\	、カナリージョン、変化《複》
\\	、分節
\\	変化形 
\\	《複》
\\	【名】社会{しゃかい}、地域社会{ちいき しゃかい}協会{きょうかい}、クラブ、会、団体{だんたい}、学会{がっかい}、結社{けっしゃ}社交界{しゃこうかい}交際{こうさい}レベル2、発音
\\	、カナソサイアティー、ソサイエティー、ソサエティー、変化《複》
\\	、分節
\\	変化形 
\\	《複》
\\	【名】地域{ちいき}(社会{しゃかい})◆住んでいる場所としての町や村のこと。前後関係によって「町」と訳してよい場合がある。〔共通{きょうつう}の利害{りがい}を持つ〕社会{しゃかい}、団体{だんたい}、グループ◆【参考】
\\	群衆{ぐんしゅう}、群生{ぐんせい}同一性{どういつ せい}、類似性{るいじ せい}社会全体{しゃかい ぜんたい}、公衆{こうしゅう}〔生態学{せいたいがく}の〕群落{ぐんらく}レベル3、発音
\\	、カナコミューニティー、コミュニティー、変化《複》
\\	、分節
\\	変化形 
\\	《複》
\\	【名】〔出生{しゅっせい}または帰化{きか}により国籍{こくせき}を得た〕国民{こくみん}、市民{しみん}〔都市{とし}に住む〕市民{しみん}〔軍人{ぐんじん}や警官{けいかん}ではない〕一般人{いっぱんじん}、民間人{みんかんじん}◆【同】
\\	〔ある地域{ちいき}に代々住む〕住民{じゅうみん}レベル2、発音
\\	、カナシチズン、シティズン、スィティズン、変化《複》
\\	、分節
\\	変化形 
\\	《複》
\\	【名】〔人間社会{にんげん しゃかい}の〕文明{ぶんめい}◆知的、文化的、物質的に発達した社会の状態。〔特定{とくてい}の〕文明{ぶんめい}◆ある時代や地域に発達した特定の文明。
\\	などのように用いられる。〔特定{とくてい}の社会{しゃかい}の〕文明化{ぶんめい か}、開化{かいか}◆文明に至る過程を指す。〔知的文化的{ちてき ぶんか てき}〕洗練{せんれん}、品位{ひんい}〔豊かで便利{べんり}な〕現代{げんだい}社会{しゃかい}[生活{せいかつ}]レベル5、発音
\\	、カナシビライゼイション、シビリゼイション、スィビリゼイション、変化《複》
\\	、分節
\\	何の宗教をお持ちですか?	変化形 
\\	《複》
\\	【名】宗教{しゅうきょう}、宗派{しゅうは}、信条{しんじょう}、宗旨{しゅうし}、信仰{しんこう}、儀式{ぎしき} ・
\\	何の宗教をお持ちですか?レベル3、発音
\\	、カナリリジョン、レリジョン、変化《複》
\\	、分節
\\	変化形 
\\	《複》
\\	【名】〔宗教的{しゅうきょう てき}・伝統的{でんとう てき}な〕式{しき}、儀式{ぎしき}、祭式{さいしき}、式典{しきてん}〔伝統的{でんとう てき}な〕儀礼{ぎれい}、礼儀作法{れいぎ さほう}〔作法{さほう}に従って〕形式張ること、堅苦{かたぐる}しくすること〔形式{けいしき}だけの〕いんぎんさ、仰々しさレベル3、発音
\\	、カナセリモウニ、セレモウニ、セレモニー、変化《複》
\\	、分節
\\	言い伝えによるとその皇帝は自分の母親を殺したことになっている。 ・
\\	禅は日本における多くの仏教の一つの流派であると聞かされた。 ・
\\	キリスト教の戒めは、私たちが決して希望を捨ててはならないことを教えてくれる。	変化形 
\\	《複》
\\	【名】〔伝承{でんしょう}された〕伝統{でんとう}、しきたり、習慣{しゅうかん}、習わし〔口伝{くでん}による〕伝承{でんしょう}、言い伝え、伝説{でんせつ} ・
\\	言い伝えによるとその皇帝は自分の母親を殺したことになっている。〔伝承{でんしょう}された芸術{げいじゅつ}などの〕様式{ようしき}、流儀{りゅうぎ}、流派{りゅうは} ・
\\	禅は日本における多くの仏教の一つの流派であると聞かされた。《宗教》〔成文化{せいぶんか}されていない〕教え、教訓{きょうくん}、戒め ・
\\	キリスト教の戒めは、私たちが決して希望を捨ててはならないことを教えてくれる。《法律》〔所有権{しょゆうけん}の〕引き渡し、譲渡{じょうと}、移転{いてん}レベル3、発音
\\	、カナトゥラディション、トラディション、変化《複》
\\	、分節
\\	商品にかかる関税はお客様のご負担になります。 ・
\\	これらの商品には関税がかかりません。	変化形 
\\	《複》
\\	【名】〔地域{ちいき}や集団{しゅうだん}の〕風習{ふうしゅう}、慣習{かんしゅう}、慣行{かんこう}、習わし〔個人{こじん}の〕習慣{しゅうかん}、習癖{しゅうへき}《法律》慣習法{かんしゅうほう}〔商店{しょうてん}などの〕引き立て、愛顧{あいこ}◆【同】
\\	〔商店{しょうてん}などの〕常連客{じょうれん きゃく}、ひいき客
\\	税関{ぜいかん}(手続き)、〔輸入{ゆにゅう}〕関税{かんぜい}◆通例単数扱い ・
\\	商品にかかる関税はお客様のご負担になります。 ・
\\	これらの商品には関税がかかりません。
\\	〔封建領主{ほうけん りょうしゅ}への〕貢ぎ物、年貢{ねんぐ}、地代{ちだい}◆通例単数扱い【形】あつらえの、特別注文のレベル2、発音
\\	、カナカスタム、カストム、変化《複》
\\	、分節
\\	変化形 
\\	《複》
\\	【名】人口{じんこう}、住民数{じゅうみんすう}◆【略】
\\	〔動物の〕個体数{こたいすう}〔ある地域に住む〕全住民{ぜん じゅうみん}、人々{ひとびと}〔統計{とうけい}の〕母集団{ぼしゅうだん}◆【同】
\\	〔生物{せいぶつ}の〕個体群{こたいぐん}、集団{しゅうだん}《天文》種族{しゅぞく}◆銀河系内で恒星を、誕生後まもない若い種族
\\	と銀河形成時に誕生した古い種族
\\	とに分類する。種族
\\	の星には重元素が多く存在するが種族
\\	の星には少ない。種族
\\	と種族
\\	の中間の性質を持つ星を中間種族星というレベル3、発音
\\	、カナパピュレイション、ポピュレイション、変化《複》
\\	、分節
\\	変化形 
\\	《複》
\\	【名】〔特定{とくてい}の場所{ばしょ}に長期間住む〕居住者{きょじゅうしゃ}〈米〉〔専門分野{せんもん ぶんや}の〕研修医{けんしゅうい}◆インターンとして研修を受けた後で、2~7年にわたって専門分野の研修を受ける研修医。◆【同】
\\	〈英〉
\\	◆【参考】
\\	〔老人{ろうじん}ホームなどの〕入居者{にゅうきょしゃ}〔海外駐在{かいがい ちゅうざい}の〕外交官{がいこうかん}〔海外駐在{かいがい ちゅうざい}の〕諜報員{ちょうほう いん}、情報部員{じょうほう ぶいん}〈英〉〔旧植民地{きゅう しょくみんち}の〕総督代理{そうとく だいり}、理事官{りじかん}◆知事や総督の補佐をする役目だが、実質的な最高権限を握っていた。《動物》〔渡りや回遊{かいゆう}をしない〕定住動物{ていじゅう どうぶつ}【形】〔一定場所{いってい ばしょ}に長期間{ちょうきかん}〕居住{きょじゅう}している〔仕事{しごと}としての〕住み込みの〔性質{せいしつ}などが〕固有{こゆう}の、内在{ないざい}する《動物》定住{ていじゅう}する《コ》〔記憶装置内{きおく そうち ない}に〕常駐{じょうちゅう}するレベル4、発音
\\	、カナレジデント、レズィデント、変化《複》
\\	、分節
\\	私の近所にブラジル人移民がたくさんいる。	イディオムやフレーズ 
\\	【名】近所{きんじょ}、近隣{きんりん}、自宅周辺{じたく しゅうへん} ・
\\	私の近所にブラジル人移民がたくさんいる。近所{きんじょ}の人、近隣住民{きんりん じゅうみん}そば、近く、付近{ふきん}〔ある特定の特色を有する〕地区{ちく}、地域{ちいき}、区域{くいき}〈話〉おおよその量[大きさ・範囲{はんい}]近しさ、親しさ《数学》近傍{きんぼう}レベル3、発音
\\	、カナネイバーフッド、ネイバフッド
\\	このテクノロジーは
\\	の環境で何年も利用されています。	変化形 
\\	《複》
\\	【名】環境{かんきょう}、周囲{しゅうい}(の状況{じょうきょう})、情勢{じょうせい} ・
\\	このテクノロジーは
\\	の環境で何年も利用されています。
\\	生活環境{せいかつ かんきょう}レベル3、発音
\\	、カナインバイアロンメント、エンバイアロンメント、エンバイロンメント、変化《複》
\\	、分節
\\	【名】汚染{おせん}、汚すこと、公害{こうがい}、汚れ、汚濁{おだく}《医》遺精{いせい}レベル4、発音
\\	、カナポリューション、ポルーション、変化《複》
\\	、分節
\\	甲の薬は乙の毒。/人によって好みは違う。◆ことわざ	変化形 
\\	【他動】~に毒を盛る、毒する、食中毒{しょくちゅうどく}にかからせる、~を毒殺{どくさつ}する、汚染{おせん}する【名】毒、毒物{どくぶつ}、毒薬{どくやく} ・
\\	甲の薬は乙の毒。/人によって好みは違う。◆ことわざ有害{ゆうがい}なもの、害になるもの、弊害{へいがい}、害悪{がいあく}、害毒{がいどく}きつい酒、安い酒レベル2、発音
\\	、カナプワゾン、ポイズン、変化《動》
\\	、分節
\\	ほんの少し間違えば大惨事になっていただろう。 ・
\\	最悪だ! ・
\\	今日は最悪の一日だった。	変化形 
\\	《複》
\\	【名】〔突然起こる〕(大)災害{さいがい}、天災{てんさい}、災難{さいなん}、(大)惨事{さんじ}、厄災{やくさい} ・
\\	ほんの少し間違えば大惨事になっていただろう。思いがけない大きな不幸{ふこう}、大失敗{だい しっぱい}、最悪{さいあく}のこと[事態{じたい}]、厄介{やっかい}なこと、とても大変{たいへん}なこと ・
\\	最悪だ! ・
\\	今日は最悪の一日だった。レベル4、発音
\\	、カナディザスター、変化《複》
\\	、分節
\\	私の性分としてやすやすと降参できない。 ・
\\	私の気質としてうそはつけない。 ・
\\	やってもみずにあきらめるなんてできない。 ・
\\	私は性格的にうそがつけない。 ・
\\	私は黙っていられない性分です。	変化形 
\\	《複》
\\	【名】自然界{しぜんかい}、物質界{ぶっしつかい}〔人知{じんち}の及ばない〕自然{しぜん}の力[驚異{きょうい}]、造物主{ぞうぶつしゅ}◆【同】
\\	とも表記〔手付{てつ}かずの〕自然{しぜん}◆人が立ち入らない、動植物のすみかとなる場所。〔文明{ぶんめい}の影響{えいきょう}を受けていない〕原始{げんし}状態{じょうたい}[生活{せいかつ}]《キリスト教》〔神の恵みを受ける前の人の〕自然{しぜん}[生まれつき]の姿種類{しゅるい}、タイプ◆ものの固有の性質で分類したもの。〔人や物の〕本質{ほんしつ}、特質{とくしつ}、本性{ほんしょう}〔人の生来{せいらい}の〕気質{きしつ}、気性{きしょう}、性分{しょうぶん} ・
\\	私の性分としてやすやすと降参できない。 ・
\\	私の気質としてうそはつけない。 ・
\\	やってもみずにあきらめるなんてできない。 ・
\\	私は性格的にうそがつけない。 ・
\\	私は黙っていられない性分です。〔人や場所{ばしょ}などの〕本物{ほんもの}、実物{じつぶつ}〔普遍的{ふへん てき}な〕人間性{にんげん せい}、人間{にんげん}の本質{ほんしつ}〔人の体の自然{しぜん}な〕働き、機能{きのう}◆婉曲的に排尿などの欲求を指すことがある。レベル2、発音
\\	、カナネイチャー、変化《複》
\\	、分節
\\	私の目の前にたくさんの可能性が広がっている。	変化形 
\\	《複》
\\	【名】《天文》宇宙{うちゅう}、銀河{ぎんが}
\\	〔存在{そんざい}するすべてを含む〕宇宙{うちゅう}、万物{ばんぶつ}、森羅万象{しんら ばんしょう}
\\	〔人間{にんげん}を中心{ちゅうしん}とした〕全世界{ぜんせかい}、全人類{ぜんじんるい}〔活動{かつどう}の〕領域{りょういき}、分野{ぶんや}《論理学》=
\\	《統計》母集団{ぼしゅうだん}◆【同】
\\	多数{たすう}、多量{たりょう} ・
\\	私の目の前にたくさんの可能性が広がっている。レベル3、発音
\\	、カナユニバース、変化《複》
\\	、分節
\\	この欄には何も記入しないでください。	変化形 
\\	【自動】〈俗〉ぼうぜんとする、ぼんやりする◆【用法】通例
\\	で用いる。【他動】~に間隔{かんかく}を空ける、~の間にスペースを置く~を分ける、~を引き離す〈俗〉〔薬などで〕(人)をボーッとさせる【名】〔時間{じかん}の〕合間{あいま}、期間{きかん}〔特定{とくてい}の用途{ようと}の〕場所{ばしょ}、スペース、空き地{あきち}〔公共交通機関{こうきょう こうつう きかん}の〕予約席{よやく せき}、空き座席〔印刷面{いんさつ めん}の〕スペース、間隔{かんかく}、余白{よはく}、空欄{くうらん} ・
\\	この欄には何も記入しないでください。〈話〉〔人が〕内にこもる自由{じゆう}、そっとしておかれること《数学》空間{くうかん}◆ある公理群に従う要素または点の集合。《物理》〔古典力学{こてん りきがく}の〕空間{くうかん}◆時間や質量と同じく、独立して定義される基本的な量。《天文》宇宙{うちゅう}(空間{くうかん})◆大気圏外、または天体の間の空間。〔テレビや新聞{しんぶん}などの〕コマーシャル時間{じかん}、広告{こうこく}スペース《音楽》〔五線譜{ごせんふ}の〕線間{せん かん}《印刷》〔活字{かつじ}の〕込め物◆スペースをつくるためのもの。〔電信通信{でんしん つうしん}の〕スペース◆キーが押されていない間を指す。【形】〔料理{りょうり}などが〕大麻{たいま}を使ったレベル1、発音
\\	、カナスペイス、変化《動》
\\	変化形 
\\	《複》
\\	【名】《天文》惑星{わくせい}、遊星{ゆうせい}◆恒星の周りを回転する、自らは光を発しないで恒星の光を反射する天体。◆【語源】天球上の動きが恒星と比べて不規則だったことから、「さまよう者」を意味するギリシャ語の
\\	がラテン語、フランス語を経由して英語に伝わったもの。〔地動説{ちどうせつ}の〕惑星{わくせい}◆宇宙の中心である地球と、お互いに位置の変わらない恒星を除く、肉眼で見える天体で、太陽・月・水星・金星・火星・木星・土星の七つを指す。〔占星術{せんせいじゅつ}の〕惑星{わくせい}◆人生や性格に影響を与えると考える天体で、伝統的には水星・金星・火星・木星・土星だったが、後になって太陽と月を含めて、7天体とするようになった。
\\	地球{ちきゅう}レベル2、発音
\\	、カナプラネット、変化《複》
\\	、分節
\\	水は地球表面の70%以上を覆っています。 ・
\\	太陽から地球まで光が届くのに、どのくらい時間がかかりますか。 ・
\\	水星、金星、地球、火星は地球型惑星です。	変化形 
\\	【他動】~を接地{せっち}する【名】地球{ちきゅう}◆伝統的には定冠詞を付けて
\\	と呼ぶ。「惑星」という意味合いでは
\\	という表記も好まれる。この場合、完全に固有名詞扱いして冠詞を付けないこともある。特に他の惑星名と並べる場合、無冠詞の
\\	がよく使われる。 ・
\\	水は地球表面の70%以上を覆っています。 ・
\\	太陽から地球まで光が届くのに、どのくらい時間がかかりますか。 ・
\\	水星、金星、地球、火星は地球型惑星です。〔空・海と対比{たいひ}させた〕(大)地、陸(地)〔植物{しょくぶつ}が育つ〕土、土壌{どじょう}〈話〉〔地球上{ちきゅう じょう}の〕すべての人々{ひとびと}、全人類{ぜんじんるい}〈英〉接地{せっち}、アース(線)◆【同】〈米〉
\\	〔精神世界{せいしん せかい}に対する〕現世{げんせ}、この世、現実{げんじつ}の生活{せいかつ}〈英〉〔キツネやアナグマの〕巣穴{すあな}《哲学》〔四大元素{よんだい げんそ}の〕土レベル1、発音
\\	、カナアース、変化《動》
\\	昨日こっちで地震があった。	変化形 
\\	《複》
\\	【名】地震{じしん} ・
\\	昨日こっちで地震があった。レベル3、発音
\\	、カナアースクウェイク、アースクェイク、アースクエイク、変化《複》
\\	、分節
\\	変化形 
\\	《複》
\\	【名】大洋{たいよう}、海◆地球上の海全体を指す言葉。◆【略】
\\	~洋◆五大海
\\	の一つを指す。◆通例
\\	◆【略】
\\	広大{こうだい}な広がりレベル2、発音
\\	、カナオーシャン、オウシャン、変化《複》
\\	、分節
\\	変化形 
\\	《複》
\\	【1名】大陸{たいりく}〔イギリスから見た〕ヨーロッパ大陸{たいりく}◆通例
\\	【2形】自制(心){じせい(しん)}の〔欲望・性欲などを〕抑えるレベル3、発音
\\	、カナカンティネント、コンティネント、変化《複》
\\	、分節
\\	変化形 
\\	《複》
\\	【他動】~を孤立{こりつ}させる~を島にする【名】島◆【略】
\\	島に似たもの、孤立{こりつ}した丘台所{だいどころ}の中央{ちゅうおう}に置かれた調理台{ちょうりだい}[流し]〔ガソリン・スタンドのポンプがある〕台座{だいざ}◆コンクリート製で周囲よりやや高くなっている部分。レベル1、発音
\\	、カナアイランド、変化《複》
\\	、分節
\\	変化形 
\\	《複》
\\	【名】〔川が流れる〕谷(間)、渓谷{けいこく}〔川の〕流域{りゅういき}〔谷に似た〕低地{ていち}、くぼみ《建築》〔二つの屋根{やね}の〕谷部{たにぶ}
\\	〈米〉セントラル・バレー◆【同】
\\	レベル2、発音
\\	、カナバリ、バレー、バレイ、変化《複》
\\	、分節
\\	バナナは草になるのであって木ではありません。 ・
\\	学校の施設より良い教師の方が大事である。	変化形 
\\	【他動】~を植える~をしっかりと立てる〈俗〉(盗聴器{とうちょうき}・盗品{とうひん}・麻薬{まやく}などをこっそり)置く、仕掛{しか}ける〔ウイルスを〕植え付ける〔思想{しそう}などを〕植え付ける、〔考えなどを〕吹き込む【名】植物(体){しょくぶつ(たい)}〔木茎を持たない〕草、花 ・
\\	バナナは草になるのであって木ではありません。〔植える前の〕苗木{なえぎ}、切り枝工場{こうじょう}(設備{せつび})、プラント〔公共機関{こうきょう きかん}などの〕建物{たてもの}、施設{しせつ}、設備{せつび} ・
\\	学校の施設より良い教師の方が大事である。〈俗〉わな、おとり〈俗〉桜〔客のふりをする人〕、密偵{みってい}、スパイ、諜報員{ちょうほう いん}、回し者{まわしもの}〈俗〉策略{さくりゃく}、謀略{ぼうりゃく}、計略{けいりゃく}〈俗〉張り込みレベル1、発音
\\	、カナプラント、変化《動》
\\	変化形 
\\	【自動】流れる光が差し込む【他動】~を流す、噴き出す【名】小川{おがわ}、水路{すいろ}、細流{さいりゅう}〔絶え間ない流体{りゅうたい}の〕流れ〔人・物・出来事{できごと}などの〕連続{れんぞく}、流れ◆通例、一列または同じ方向に流れる場合を指す。〔考え方や世間{せけん}などの〕傾向{けいこう}、動向{どうこう}、方向{ほうこう}光線{こうせん}、一筋{ひとすじ}の光〈英〉能力別{のうりょく べつ}(編成{へんせい})クラス[グループ]◆【同】〈米〉
\\	レベル2、発音
\\	、カナストゥリーム、ストリーム、変化《動》
\\	最新のニュースや出来事を常に知って[把握して]おくのは良いことです。	変化形 
\\	《複》
\\	【名】〔空気{くうき}や水などの〕流れ〔流れている〕水、空気{くうき}《電気》電流(量){でんりゅう(りょう)}〔一般的{いっぱん てき}な〕傾向{けいこう}、動向{どうこう}【形】現在{げんざい}の、最新{さいしん}の、進行中{しんこうちゅう}の ・
\\	最新のニュースや出来事を常に知って[把握して]おくのは良いことです。受け入れられている、認められている人気{にんき}がある、流行{りゅうこう}している〔貨幣{かへい}などが〕流通{りゅうつう}している〈古〉流れている、動いているレベル4、発音
\\	、カナカーラント、カレント、変化《複》
\\	、分節
\\	昨日は嵐だった。 ・
\\	こんな所へ乗り込んできてどういうつもりなんだ?	変化形 
\\	【自動】あらしが吹く、荒れる ・
\\	昨日は嵐だった。怒鳴{どな}る、攻撃{こうげき}する ・
\\	こんな所へ乗り込んできてどういうつもりなんだ?〔激しく〕飛び出す、動く【他動】~を急襲{きゅうしゅう}する、~を猛攻撃{もうこうげき}する~を怒鳴りつける、~に激しくくってかかる、ガミガミ言う〔雨や雪などを〕降らせる~に突入{とつにゅう}する、~に乱暴{らんぼう}に飛び込む【名】嵐、荒天{こうてん}◆強い風と共に雨、雪、みぞれなどが降り、雷や稲妻も伴うもの。《気象》暴風{ぼうふう}◆【同】
\\	〔銃弾{じゅうだん}などの〕嵐、雨あられ〔感情{かんじょう}などの〕突発{とっぱつ}、激発{げきはつ}〔社会状況{しゃかい じょうきょう}などの〕大混乱{だい こんらん}、大騒ぎ{おおさわぎ}〔敵陣地{てき じんち}に対する〕襲撃{しゅうげき}、猛攻撃{もうこうげき}レベル2、発音
\\	、カナストーム、変化《動》
\\	変化形 
\\	【自動】〔外気{がいき}や風雨{ふうう}に〕変色{へんしょく}する、色あせる、風化{ふうか}する〔外気{がいき}や風雨{ふうう}に〕耐える、色あせない【他動】〔嵐・困難{こんなん}などを〕切り抜ける、乗り切る、乗り越える~を外気{がいき}[風雨{ふうう}]にさらす〔風雨{ふうう}にさらして〕~を変色{へんしょく}させる[風化{ふうか}させる・ぼろぼろにする]《建築》〔水はけのために屋根{やね}などに〕傾斜{けいしゃ}を付ける《海事》〔船が~の〕風上{かざかみ}を通る【名】〔ある時の〕天気{てんき}、天候{てんこう}、気候{きこう}◆【参考】
\\	悪天候{あくてんこう}、嵐、風雨{ふうう}
\\	〔人生{じんせい}の〕浮き沈み、移り変わり{うつりかわり}【形】《海事》風上{かざかみ}の[に向かった]気象予報{きしょう よほう}に用いる、気象観測用{きしょうかんそく よう}のレベル2、発音
\\	、カナウェザー、変化《動》
\\	、分節
\\	変化形 
\\	《複》
\\	【名】気候{きこう}◆ある地域の長期間の平均的な気象状態。〔ある気候{きこう}の〕地域{ちいき}、地帯{ちたい}〔ある社会{しゃかい}の〕風潮{ふうちょう}、風土{ふうど}レベル3、発音
\\	、カナクライミット、クライメイト、クライメット、変化《複》
\\	、分節
\\	変化形 
\\	【自動】次第{しだい}に変化{へんか}する【他動】〔~を光や熱から〕遮る、保護{ほご}する~に付きまとう、~に陰のように付き添う〔絵画{かいが}などに〕陰影{いんえい}を付ける、グラデーションを付ける〔場所{ばしょ}を〕薄暗くする〔意味{いみ}などを〕微妙{びみょう}に変化{へんか}させる〔価格{かかく}を〕わずかに下げる【名】〔日の光が遮られてできる〕(日)陰、木陰{こかげ}、物陰{ものかげ}◆物の形がはっきりしているものは
\\	。半暗がり、光がぼんやりしている場所{ばしょ}〔ランプなどの〕シェード、傘〔窓に取り付ける〕ロールスクリーン、巻き上げ式ブラインド〔微妙{びみょう}に異なる〕色合い{いろあい}、色調{しきちょう}◆元の色と少し明るさが異なる色。〔絵画{かいが}の〕陰影{いんえい}◆【同】
\\	わずかなもの、少量{しょうりょう}〔意見{いけん}などの〕微妙{びみょう}な食い違い亡霊{ぼうれい}、幽霊{ゆうれい}◆【同】
\\	〈俗〉サングラス
\\	〈文〉〔日暮れの〕夕闇{ゆうやみ}、暗がり
\\	〔昔を〕思い出させる物レベル2、発音
\\	、カナシェイド、変化《動》
\\	誰かが私たちの後をつけている。	変化形 
\\	【自動】〔色や明るさが〕緩やかに変化{へんか}する〔顔が〕曇る【他動】~を陰にする、影で覆う〔熱・光を〕遮断{しゃだん}する~を悲しませる、憂鬱{ゆううつ}[陰鬱{いんうつ}]な気分{きぶん}にさせる~の後をつける、~を尾行{びこう}する ・
\\	誰かが私たちの後をつけている。~を匂{にお}わせる[ほのめかす]、暗示{あんじ}[示唆{しさ}]する、漠然{ばくぜん}と示す[表す]、~の前兆{ぜんちょう}[前触れ]となる《コ》〔変数{へんすう}が別の変数{へんすう}などを〕シャドーイングする◆【参考】
\\	【名】〔光が当たってできる〕(物)影{(もの)かげ}◆【参考】
\\	〔光が届かない〕暗がり、暗部{あんぶ}気配{けはい}、気味{きみ}、かすかに感じさせるもの不吉{ふきつ}な前兆{ぜんちょう}[前触れ]〔付きまとう〕不安{ふあん}、恐れ〔目の下の〕くま◆疲れたときにできるもの。〔他人{たにん}によって〕目立たなく[影が薄く]なることいつも一緒{いっしょ}にいようとする人、分身{ぶんしん}のような人〔尾行{びこう}をする〕探偵{たんてい}、刑事{けいじ}、スパイ〔絵画{かいが}などの〕陰影{いんえい}◆【同】
\\	幽霊{ゆうれい}、亡霊{ぼうれい}◆【同】
\\	ほんのわずか、微量{びりょう}〔水面{すいめん}に映る〕影、像〔不完全{ふかんぜん}な〕写し、見せ掛け〔弱々{よわよわ}しい〕面影{おもかげ}、名残{なごり}〔一緒{いっしょ}にいて仕事{しごと}を教わる〕新人{しんじん}〔レントゲンに映る〕影〔ユング心理学{しんりがく}における〕影◆通常は抑えられている、人間の原始的な性的・攻撃的衝動の原型。保護{ほご}、擁護{ようご}【形】影の、影を使った、影の多い人目{ひとめ}の届かない所にいる、表に出てこない、隠れた存在{そんざい}の〈英〉〔野党{やとう}が作る内閣{ないかく}の〕影のレベル2、発音
\\	、カナシャドー、シャドウ、変化《動》
\\	、分節
\\	数学の学位があると何ができますか。	変化形 
\\	《複》
\\	【名】〔比較{ひかく}した〕程度{ていど}、度合い{どあい}、レベル〔進行{しんこう}の〕過程{かてい}、段階{だんかい}、ステップ〈文〉〔社会的{しゃかい てき}〕地位{ちい}、ランク〔大学{だいがく}の〕学位{がくい}◆【参考】
\\	数学の学位があると何ができますか。〔温度{おんど}の単位{たんい}の〕度◆セ氏
\\	とカ氏
\\	があり、セ氏は
\\	の記号で表される。〔角度{かくど}の単位{たんい}の〕度◆円周を360分した単位で、°の記号で表される。〔緯度{いど}や経度{けいど}の単位{たんい}の〕度◆緯度は赤道を0度、経度は旧グリニッジ天文台と極を結ぶ線を0度として、地球の1周を360度とした単位。°の記号で表される。〔家族関係{かぞく かんけい}の〕親等{しんとう}◆【参考】
\\	《音楽》〔音階{おんかい}の〕度◆1オクターブを8度とし、同じ音の隔たりを1度とする。《言語学》〔形容詞{けいようし}の〕級◆【参考】
\\	〔熱傷{ねっしょう}の〕度◆【参考】
\\	〔殺人{さつじん}の〕級◆【参考】
\\	《数学》〔単項式{たんこうしき}の最大{さいだい}の〕次数{じすう}《数学》〔多項式{たこうしき}の合計{ごうけい}の〕全次数{ぜん じすう}レベル2、発音
\\	、カナディグリー、デグリー、変化《複》
\\	、分節
\\	君は真相を知っている。	変化形 
\\	【自動】得点{とくてん}する成功する、うまくいく〔女を〕物にする〈俗〉麻薬{まやく}[非合法{ひ ごうほう}ドラッグ]を手に入れる【他動】~を得点{とくてん}する〔得点{とくてん}などを〕記録{きろく}する〔テストなどを〕採点{さいてん}する〔勝利{しょうり}・成功{せいこう}などを〕得る、納める〔曲を〕楽譜に書く〈米話〉~を非難{ひなん}する〈米話〉~をうまく手に入れる【名】得点{とくてん}、成績{せいせき}、スコア、点数{てんすう}楽譜{がくふ}〔ものや人の〕20の集まり◆複数形は
\\	または
\\	多数{たすう}
\\	〔事の〕真実{しんじつ}、真相{しんそう} ・
\\	君は真相を知っている。勘定覚え書、負債{ふさい}わだかまり、恨み理由{りゆう}、根拠{こんきょ}〈俗〉〔違法{いほう}な〕麻薬{まやく}の入手{にゅうしゅ}折り線、切り込み線、引っかき線〈話〉大成功{だいせいこう}、素晴{すば}らしい成果{せいか}◆「しめた」「やった」という意味で間投詞的にも使う。〈俗〉〔成功{せいこう}した〕強奪{ごうだつ}、略奪{りゃくだつ}レベル2、発音
\\	、カナスコアー、変化《動》
\\	〔香辛料の〕カルダモンは人の手で選別される。 ・
\\	彼は地元の小学校の2年生です。 ・
\\	私の成績は上がっています。前の学期には
\\	が1個、
\\	が2個ありました。 ・
\\	私の成績は下がっています。	変化形 
\\	【自動】等級別{とうきゅう べつ}になっている【他動】~を等級分けする、~を格付けする ・
\\	〔香辛料の〕カルダモンは人の手で選別される。~を採点{さいてん}する、(人)に成績{せいせき}をつける〔傾斜{けいしゃ}を〕水平{すいへい}[決められた角度{かくど}]に近づける〔色を〕ぼかす◆【同】
\\	〔純粋種{じゅんすい しゅ}を交配{こうはい}させて家畜{かちく}を〕品種改良{ひんしゅ かいりょう}する【名】等級{とうきゅう}、品質等級{ひんしつ とうきゅう}学年{がくねん} ・
\\	彼は地元の小学校の2年生です。成績{せいせき}の評点{ひょうてん}、評価{ひょうか} ・
\\	私の成績は上がっています。前の学期には
\\	が1個、
\\	が2個ありました。 ・
\\	私の成績は下がっています。勾配{こうばい}、傾斜{けいしゃ}度、結合度{けつごう ど}、品位{ひんい}《医》悪性度{あくせい ど}《医》異型度{いけい ど}レベル2、発音
\\	、カナグレイド、変化《動》
\\	要素の数は10の倍数で増加する。	変化形 
\\	《複》
\\	【他動】《数学》~を因数{いんすう}に分解{ぶんかい}する【名】〔結果{けっか}を生じさせるのに寄与{きよ}した〕要因{よういん}、因子{いんし}《数学》因数{いんすう}〔数値{すうち}の増減{ぞうげん}を示すときの〕倍数{ばいすう}、係数{けいすう} ・
\\	要素の数は10の倍数で増加する。《金融》=
\\	《生物》〔血液凝固{けつえき ぎょうこ}などの生理的過程{せいり てき かてい}の〕因子{いんし}〈古〉〔商売{しょうばい}の〕代理人{だいりにん}レベル4、発音
\\	、カナファクター、変化《動》
\\	、分節
\\	そうした事実は全くない。 ・
\\	事実は小説よりも奇なり。◆ことわざ	変化形 
\\	《複》
\\	【名】〔真実{しんじつ}の情報{じょうほう}としての〕事実{じじつ}、申し立て{もうしたて}◆本当のこととして語られること。 ・
\\	そうした事実は全くない。〔実際{じっさい}に起きたとされる〕事実{じじつ}、真相{しんそう} ・
\\	事実は小説よりも奇なり。◆ことわざ〔空想{くうそう}ではなく実在{じつざい}の〕現実{げんじつ}、実際{じっさい}《法律》〔価値判断{かち はんだん}を加えない〕犯行事実{はんこう じじつ}《法律》〔証拠{しょうこ}によって認定{にんてい}される〕事実{じじつ}レベル1、発音
\\	、カナファクト、変化《複》
\\	喫煙は肺がんを引き起こす可能性があります。 ・
\\	吹雪のせいで視界がまったく利かなくなりました。 ・
\\	あなたの今までのご支援には大変感謝しておりますが、これ以上ご面倒をお掛けしたくありません。 ・
\\	私のプログラミングミスによってご迷惑をお掛けしたことをおわび申し上げます。 ・
\\	今回の誤り[間違い]によってご迷惑をお掛けして、申し訳なく存じます。◆謝罪 ・
\\	彼女にはとても弱ってる。 ・
\\	ある生徒がいつもその先生をとても困らせています。 ・
\\	タマネギ、ニンニク、チョコレートは、犬を重病に至らしめます。 ・
\\	その医療研究者は、その病気の原因を発見しました。 ・
\\	私は、大義のために一生をかけて戦った女性を知っています。	変化形 
\\	【他動】~を引き起こす、~の原因となる ・
\\	喫煙は肺がんを引き起こす可能性があります。 ・
\\	吹雪のせいで視界がまったく利かなくなりました。〔迷惑などを〕(人)に掛ける◆【用法】
\\	あなたの今までのご支援には大変感謝しておりますが、これ以上ご面倒をお掛けしたくありません。 ・
\\	私のプログラミングミスによってご迷惑をお掛けしたことをおわび申し上げます。 ・
\\	今回の誤り[間違い]によってご迷惑をお掛けして、申し訳なく存じます。◆謝罪 ・
\\	彼女にはとても弱ってる。 ・
\\	ある生徒がいつもその先生をとても困らせています。~に…させる◆【用法】
\\	タマネギ、ニンニク、チョコレートは、犬を重病に至らしめます。【名】〔物事を引き起こす〕原因{げんいん}、要因{よういん} ・
\\	その医療研究者は、その病気の原因を発見しました。〔人の行為{こうい}や感情{かんじょう}の〕理由{りゆう}、動機{どうき}〔人が求める〕目的{もくてき}、目標{もくひょう}〔人を動かす〕理念{りねん}、信念{しんねん}、大義{たいぎ} ・
\\	私は、大義のために一生をかけて戦った女性を知っています。《法律》訴訟{そしょう}(理由{りゆう})〔議論{ぎろん}の〕議題{ぎだい}、テーマレベル2、発音
\\	、カナコーズ、変化《動》
\\	その結果は誰もが認めるところだ。 ・
\\	オリンピックは結果よりも参加することが重要である。 ・
\\	非暴力の政策が成果を生む。	変化形 
\\	【自動】〔結果的{けっかてき}に〕生じる、起こる終わる、帰着{きちゃく}する、~の結果{けっか}となる【名】結果{けっか}、結末{けつまつ}、成り行き ・
\\	その結果は誰もが認めるところだ。 ・
\\	オリンピックは結果よりも参加することが重要である。計算結果{けいさんけっか}、計算{けいさん}の答え効果{こうか}、成果{せいか}、成績{せいせき}、業績{ぎょうせき}、実績{じっせき} ・
\\	非暴力の政策が成果を生む。レベル2、発音
\\	、カナリザルト、変化《動》
\\	、分節
\\	その薬は私には効きませんでした。 ・
\\	不眠(症)は、カフェイン含有飲料の取り過ぎの結果です。 ・
\\	スト(ライキ)の影響がすでに出始めています。	変化形 
\\	【他動】~を発効{はっこう}させる〔目標などを〕達成{たっせい}する〔結果として〕~をもたらす〔変化などを〕もたらす、きたす、生じさせる【名】効果{こうか}、効き目、効力{こうりょく}、薬の効能{こうのう} ・
\\	その薬は私には効きませんでした。〔原因に対する〕結果{けっか}、影響{えいきょう} ・
\\	不眠(症)は、カフェイン含有飲料の取り過ぎの結果です。 ・
\\	スト(ライキ)の影響がすでに出始めています。〔法律{ほうりつ}・規則{きそく}などの〕発効{はっこう}〔~に与える〕印象{いんしょう}趣旨{しゅし}、要点{ようてん}
\\	私物{しぶつ}、個人資産{こじん しさん}、動産物件{どうさん ぶっけん}◆【参考】
\\	レベル2、発音
\\	、カナイフェクト、エフェクト、変化《動》
\\	、分節
\\	実力よりもコネが大切である。 ・
\\	彼は、子どもたちに悪影響を与える存在です。	変化形 
\\	【他動】(人)を感化{かんか}する、(人)を動かす〔~の性質{せいしつ}や状態{じょうたい}に〕影響{えいきょう}を与える【名】〔人や物事{ものごと}への〕影響{えいきょう}、感化{かんか}〔人の思考{しこう}や行動{こうどう}を左右{さゆう}する〕影響力{えいきょうりょく}、支配力{しはいりょく}〔富や地位{ちい}から生まれる〕威光{いこう}、コネ、七光り{ななひかり} ・
\\	実力よりもコネが大切である。〔影響力{えいきょうりょく}ある〕実力者{じつりょくしゃ}、勢力家{せいりょくか}〔占星術{せんせいじゅつ}の〕影響{えいきょう}◆星や天体から出ている霊液が人の性格や行動に影響を及ぼすとするもの。影響{えいきょう}を与える物[人・存在{そんざい}・事柄{ことがら}] ・
\\	彼は、子どもたちに悪影響を与える存在です。レベル3、発音
\\	、カナインフルーエンス、インフルエンス、変化《動》
\\	、分節
\\	直行便でなく乗り継ぎ便を使った。 ・
\\	彼は、父親のコネで仕事にありつきました。	変化形 
\\	《複》
\\	【名】つなぐこと、つながり、連結(部){れんけつ(ぶ)}、結合(部){けつごう(ぶ)}、接続(部){せつぞく(ぶ)}通信{つうしん}、連絡{れんらく}〔列車・飛行機などの〕乗り継ぎ、乗り換え、接続{せつぞく} ・
\\	直行便でなく乗り継ぎ便を使った。関係{かんけい}、関連(性){かんれん(せい)}顧客{こきゃく}、団体{だんたい}麻薬{まやく}の流通{りゅうつう}ルート《囲碁》継ぎ、つなぎコネ、人脈{じんみゃく}◆通例、複数形で使われる。 ・
\\	彼は、父親のコネで仕事にありつきました。レベル3、発音
\\	、カナコネクション、変化《複》
\\	、分節
\\	変化形 
\\	【自動】〔徹底的{てってい てき}に〕研究{けんきゅう}[調査{ちょうさ}]する【他動】〔熱心{ねっしん}に〕~を探求{たんきゅう}[追求{ついきゅう}]する〔学問{がくもん}として~を徹底的{てってい てき}に〕研究{けんきゅう}[調査{ちょうさ}]する【名】〔熱心{ねっしん}な〕探求{たんきゅう}、追求{ついきゅう}〔学問的{がくもん てき}な徹底{てってい}した〕研究{けんきゅう}、調査{ちょうさ}〔研究者{けんきゅうしゃ}の〕研究{けんきゅう}[調査{ちょうさ}]能力{のうりょく}[意欲{いよく}]レベル3、発音
\\	、カナリサーチ、変化《動》
\\	、分節
\\	変化形 
\\	《複》
\\	【名】分析{ぶんせき}、解析(学){かいせき(がく)}レベル3、発音
\\	、カナアナリシス、変化《複》
\\	、分節
\\	変化形 
\\	《複》
\\	【他動】テーマに合わせた構成{こうせい}を行う【名】〔話題{わだい}や議論{ぎろん}などの〕話題{わだい}、論題{ろんだい}、テーマ〔芸術作品{げいじゅつ さくひん}などの〕主題{しゅだい}、テーマ◆【同】
\\	《音楽》主旋律{しゅ せんりつ}、主題、テーマ◆【同】
\\	《歴史》(ビザンツ帝国{ていこく}の行政単位{ぎょうせい たんい})テマ、軍管区{ぐんかんく}◆小アジアの防衛として機能〔学校{がっこう}で生徒{せいと}に課す短い〕作文{さくぶん}《言語学》語幹{ごかん}◆【同】
\\	レベル5、発音
\\	、カナスィーム、変化《複》
\\	その本社[本店]は、立ち入り検査を受けなければなりませんでした。	変化形 
\\	【他動】〔~に考察{こうさつ}や研究{けんきゅう}などを〕する、行う ・
\\	その本社[本店]は、立ち入り検査を受けなければなりませんでした。〔人に不快{ふかい}な経験{けいけん}を〕(受け)させる〔人や国民{こくみん}を〕支配{しはい}する、服従{ふくじゅう}させる【名】〔議論{ぎろん}や研究{けんきゅう}などの〕主題{しゅだい}、題目{だいもく}、議題{ぎだい}、テーマ〔教育{きょういく}の〕科目{かもく}、教科{きょうか}◆通例、
\\	《言語学》主語{しゅご}、主部{しゅぶ}《音楽》〔フーガの〕主題{しゅだい}、テーマ〔絵画{かいが}や写真{しゃしん}などの〕題材{だいざい}、被写体{ひしゃたい}〔治療{ちりょう}や実験{じっけん}の〕対象者{たいしょうしゃ}、被験者{ひけんしゃ}〔犯罪{はんざい}の〕被疑者{ひぎしゃ}、容疑者{ようぎしゃ}〔からかいや不満{ふまん}などの〕種、的〔国王{こくおう}などの〕臣民{しんみん}、臣下{しんか}【形】~の支配{しはい}を受ける、~の権力下{けんりょく か}にある~しがちな、~の癖がある~にさらされる、~を被る~次第{しだい}である、~に依存{いそん}するレベル2、発音《名・形》
\\	《動》
\\	、カナサブジェクト、変化《動》
\\	、分節
\\	変化形 
\\	【自他動】反対{はんたい}する、異議{いぎ}を唱える【名】〔視覚{しかく}や触覚{しょっかく}で感知{かんち}できる〕物、物体{ぶったい}〔関心{かんしん}や意識{いしき}などの〕中心{ちゅうしん}、焦点{しょうてん}〔動作{どうさ}や行為{こうい}の〕目的{もくてき}、目標{もくひょう}《言語学》〔動詞{どうし}の〕目的語{もくてきご}《言語学》〔前置詞{ぜんちし}の〕目的語{もくてきご}《哲学》客体{きゃくたい}、客観{きゃっかん}◆【対】
\\	〔カメラなどの光学系{こうがくけい}の〕被写体{ひしゃたい}《コ》オブジェクト◆オブジェクト指向プログラミングにおける、変数、データ構造、手続きが一体になったもの。レベル2、発音《名》
\\	《動》
\\	、カナオブジェクト、変化《動》
\\	、分節
\\	ここに来た目的は何ですか? ・
\\	私のこの旅行の主な目的は、ここに来てあなたに会うことだったんです。	変化形 
\\	【自動】目的を持つ【他動】~を目的{もくてき}とする【名】〔望む〕目的{もくてき}、目標{もくひょう}、狙い ・
\\	ここに来た目的は何ですか? ・
\\	私のこの旅行の主な目的は、ここに来てあなたに会うことだったんです。〔存在{そんざい}などの〕理由{りゆう}、意義{いぎ}、意味{いみ}〔目標{もくひょう}に向かう〕決意{けつい}、決心{けっしん}レベル2、発音
\\	、カナパーパス、変化《動》
\\	、分節
\\	変化形 
\\	【名】手段{しゅだん}、方法{ほうほう}、手法{しゅほう}◆【用法】単数・複数扱い〔ある人の〕資力{しりょく}、財力{ざいりょく}大資産{だい しさん}、大きな富レベル3、発音
\\	、カナミーンズ
\\	彼は気が違っている割には筋道が通っている。/彼の振る舞いは見掛けほど無謀ではありません。	変化形 
\\	《複》
\\	【名】方法{ほうほう}、方式{ほうしき}、手法{しゅほう}、方途{ほうと}規則正{きそく ただ}しさ、順序正{じゅんじょ ただ}しさ、秩序{ちつじょ}、筋道{すじみち} ・
\\	彼は気が違っている割には筋道が通っている。/彼の振る舞いは見掛けほど無謀ではありません。《コ》メソッド◆オブジェクトへの操作命令。レベル2、発音
\\	、カナメソッド、メソド、変化《複》
\\	、分節
\\	このビルは屋根までの高さが300フィートある。 ・
\\	私は、新しいドレスを買う前に体のサイズを正確に測ってもらう必要があります。 ・
\\	グラムは重さの単位である。 ・
\\	大統領は言葉を慎重に選んだ。 ・
\\	人間は万物の尺度である。 ・
\\	は経済成長の一つの尺度である。 ・
\\	オリーブ油は香り付けのためだけに少し使う。 ・
\\	時間に限界はない。 ・
\\	私たちは、家屋への水害を減らすために、土嚢を一時的手段として使いました。 ・
\\	鳥インフルエンザのまん延を防ぐために、いろいろな手段が取られました。	変化形 
\\	【自動】〔大きさや重さが〕ある ・
\\	このビルは屋根までの高さが300フィートある。測る、測定{そくてい}する【他動】〔寸法・分量などを〕測る、測定{そくてい}する ・
\\	私は、新しいドレスを買う前に体のサイズを正確に測ってもらう必要があります。~を測って分ける、~を区画{くかく}する〔~の効果{こうか}や品質{ひんしつ}などを〕評価{ひょうか}する、査定{さてい}する〔~を…と〕比べる、比較{ひかく}する~の測定単位{そくてい たんい}となる ・
\\	グラムは重さの単位である。~を調整{ちょうせい}する、~を慎重{しんちょう}に用いる ・
\\	大統領は言葉を慎重に選んだ。【名】〔測定{そくてい}された〕寸法{すんぽう}、大きさ、量〔キログラム原器{げんき}などの〕基準原器{きじゅん げんき}〔メートル法などの〕度量衡体系{どりょうこう たいけい}〔メートルやグラムなどの〕測定単位{そくてい たんい}〔物差{ものさ}しや秤などの〕測定器{そくていき}、計量器{けいりょうき}、メジャー測定{そくてい}、計量{けいりょう}評価{ひょうか}[比較{ひかく}]基準{きじゅん}[尺度{しゃくど}] ・
\\	人間は万物の尺度である。 ・
\\	は経済成長の一つの尺度である。程度{ていど}、度合い{どあい} ・
\\	オリーブ油は香り付けのためだけに少し使う。〔量り分けた〕一定量{いっていりょう}限界{げんかい}、限度{げんど} ・
\\	時間に限界はない。手段{しゅだん}、方策{ほうさく} ・
\\	私たちは、家屋への水害を減らすために、土嚢を一時的手段として使いました。 ・
\\	鳥インフルエンザのまん延を防ぐために、いろいろな手段が取られました。《音楽》小節{しょうせつ}◆【同】
\\	《法律》〔成立前{せいりつ まえ}の〕法案{ほうあん}、〔成立{せいりつ}した〕法令{ほうれい}〔詩の〕韻律{いんりつ}、韻脚{いんきゃく}《数学》約数{やくすう}レベル2、発音
\\	、カナメジャー、変化《動》
\\	、分節
\\	理論と実際は全く別です。	変化形 
\\	《複》
\\	【名】学説{がくせつ}、説理論{りろん}、理屈{りくつ} ・
\\	理論と実際は全く別です。持論{じろん}、見解{けんかい}、意見{いけん}レベル3、発音
\\	、カナスィーオリ、セオリー、変化《複》
\\	、分節
\\	英語をうまく話せるようになるには練習が必要。 ・
\\	人体冷凍は60年代後半から実行されている。 ・
\\	人に説くことを自分でも実行しなさい。◆ことわざ ・
\\	習うより慣れろ。/練習[訓練・稽古{けいこ}・実践]することで完璧{かんぺき}に(できるように)なる。/練習[訓練・けいこ・実践]が完璧{かんぺき}[完全](な技[技術])を作り上げる。	変化形 
\\	【自動】〔上達のために〕練習{れんしゅう}[訓練{くんれん}]する ・
\\	英語をうまく話せるようになるには練習が必要。〔医師{いし}や弁護士{べんごし}などが〕開業{かいぎょう}[営業{えいぎょう}]する〔習慣的{しゅうかん てき}にまたは繰り返し〕行う、実施{じっし}する【他動】〔上達{じょうたつ}のために~を〕練習{れんしゅう}する、訓練{くんれん}する〔理論などを〕実行{じっこう}[実践{じっせん}]する ・
\\	人体冷凍は60年代後半から実行されている。 ・
\\	人に説くことを自分でも実行しなさい。◆ことわざ〔習慣的{しゅうかん てき}に~を〕行う、する〔医師{いし}などの専門職{せんもんしょく}に〕従事{じゅうじ}する、開業{かいぎょう}する〔宗教的{しゅうきょう てき}なしきたりなどを〕順守{じゅんしゅ}する、守る〔悪事{あくじ}などを〕犯す、しでかす【名】〔上達のための〕練習{れんしゅう}、訓練{くんれん}、演習{えんしゅう} ・
\\	習うより慣れろ。/練習[訓練・稽古{けいこ}・実践]することで完璧{かんぺき}に(できるように)なる。/練習[訓練・けいこ・実践]が完璧{かんぺき}[完全](な技[技術])を作り上げる。〔理論や計画などの〕実践{じっせん}、実行{じっこう}、履行{りこう}〔練習{れんしゅう}などによる〕上達{じょうたつ}、熟練{じゅくれん}〔日常的{にちじょう てき}な〕活動{かつどう}、行為{こうい}〔医師{いし}などの専門職{せんもんしょく}の〕仕事{しごと}、実務{じつむ}〔医師{いし}などの専門職{せんもんしょく}の〕営業{えいぎょう}、開業{かいぎょう}〔習慣的{しゅうかん てき}な〕慣行{かんこう}、慣例{かんれい}◆【用法】通例
\\	《法律》訴訟手続きレベル1、発音
\\	、カナプラクティス、変化《動》
\\	、分節
\\	新しい研究は「臨死体験」の影響について詳しく述べている。 ・
\\	このページでは機関車に関する情報の詳細がいくつかご覧になれます。 ・
\\	穏やかな村の生活を本にするために日常生活の一部始終を注視しました。 ・
\\	その提案全体の詳細を知る必要がある。	変化形 
\\	【他動】~を詳しく述べる、詳述{しょうじゅつ}する ・
\\	新しい研究は「臨死体験」の影響について詳しく述べている。~を列挙{れっきょ}する~に細部{さいぶ}の装飾{そうしょく}をする《軍事》〔軍隊{ぐんたい}をある任務{にんむ}に〕選抜{せんばつ}する、派遣{はけん}する【名】〔全体{ぜんたい}を構成{こうせい}する一部分{いちぶぶん}の要素{ようそ}の〕詳細{しょうさい}、細部{さいぶ}、細目{さいもく} ・
\\	このページでは機関車に関する情報の詳細がいくつかご覧になれます。〔全体{ぜんたい}を構成{こうせい}する個々{ここ}の要素{ようそ}の〕一つ一つ、あらゆる点 ・
\\	穏やかな村の生活を本にするために日常生活の一部始終を注視しました。〔全体{ぜんたい}を構成{こうせい}するすべての要素{ようそ}の集合{しゅうごう}としての〕詳細{しょうさい}、全体{ぜんたい} ・
\\	その提案全体の詳細を知る必要がある。〔全体{ぜんたい}と比較{ひかく}して〕ささいな点、枝葉末節{しよう まっせつ}〔建築{けんちく}や芸術作品{げいじゅつ さくひん}の〕細部{さいぶ}、ディテール《軍事》特殊任務部隊{とくしゅ にんむ ぶたい}、選抜部隊{せんばつ ぶたい}レベル2、発音
\\	、カナデーテイル、ディーテイル、デテイル、ディテール、変化《動》
\\	、分節
\\	変化形 
\\	【自動】〔科学{かがく}の〕実験{じっけん}を行う〔新しいことを〕試してみる◆【用法】新しい手段は
\\	、試す対象は
\\	、試す分野は
\\	で表す。【名】〔科学{かがく}の〕実験(法){じっけん(ほう)}新手法{しんしゅほう}、新企画{しん きかく}、新機軸{しんきじく}実験{じっけん}の結果{けっか}[成果{せいか}]レベル3、発音《名》
\\	《動》
\\	、カナイクスペリメント、エクスペリメント、変化《動》
\\	、分節
\\	変化形 
\\	《複》
\\	【名】〔自然現象{しぜん げんしょう}を研究{けんきゅう}する〕自然科学{しぜん かがく}〔特定{とくてい}の学問分野{がくもん ぶんや}の〕科学{かがく}◆【用法】通例、
\\	のように複合語で用いられる。〔自然現象{しぜん げんしょう}や特定{とくてい}の学問分野{がくもん ぶんや}に関する〕科学的知識{かがく てき ちしき}〔体系的{たいけい てき}になっている〕技術{ぎじゅつ}、方法{ほうほう}
\\	クリスチャン・サイエンス◆【同】
\\	レベル1、発音
\\	、カナサイエンス、変化《複》
\\	、分節
\\	変化形 
\\	《複》
\\	【名】科学{かがく}[工業{こうぎょう}]技術{ぎじゅつ}、テクノロジー◆科学や機器を利用して物の生産に結び付けることおよびその技術。特定の分野に関連付けて、
\\	と同じ意味で使われることが多い。〔科学技術的方法{かがく ぎじゅつ てき ほうほう}を使う〕テクノロジー◆知識やシステムなどの分野に科学技術的方法を使うこと、またはその技術。〔人類学{じんるいがく}の〕テクノロジー◆社会が持つ、物質文明に関する実用的な知識の総体。レベル3、発音
\\	、カナテクナロジ、テクノロジー、変化《複》
\\	、分節
\\	【名】電気{でんき}、電力{でんりょく}、電気学{でんきがく}レベル2、発音
\\	、カナイレクトリスィリー、イレクトゥリスィティー、イレクトリシティー、イレクトリセティー、エレクトリシティー、エレクトリセティー、分節
\\	この車にはあらゆる種類の装置が付いている。 ・
\\	その装置はオフラインで作動する。 ・
\\	そのデバイスは
\\	に接続される。	変化形 
\\	《複》
\\	【名】〔特定{とくてい}の比較的簡単{ひかく てき かんたん}な目的{もくてき}を果たす〕機器{きき}、装置{そうち}、道具{どうぐ}、発明品{はつめいひん}、考案品{こうあん ひん} ・
\\	この車にはあらゆる種類の装置が付いている。 ・
\\	その装置はオフラインで作動する。手段{しゅだん}、仕掛け{しかけ}、工夫{くふう}図案{ずあん}、意匠{いしょう}計画{けいかく}、策略{さくりゃく}
\\	意思{いし}、望み《コ》デバイス◆コンピューターの独立した機能を果たす装置で、周辺機器を指すことが多いが、電子部品を指すこともある。 ・
\\	そのデバイスは
\\	に接続される。《法律》考案{こうあん}◆実用新案法で保護される。レベル3、発音
\\	、カナディバイズ、デバイス、変化《複》
\\	、分節
\\	変化形 
\\	【他動】~を機械加工{きかい かこう}する【名】《機械》〔電気{でんき}などで動く〕機、機械{きかい}◆人手より効率よく仕事を行う、人力以外の動力で動く比較的複雑な装置を指す。《機械》〔人力{じんりょく}で動かす〕仕掛け、道具{どうぐ}◆人力を使って大きな力を出させる、てこや滑車などの比較的単純な装置を指す。〔エンジンで移動{いどう}する〕乗り物、自動車{じどうしゃ}、飛行機{ひこうき}、オートバイ《コ》コンピューター、マシン〔生物{せいぶつ}などの複雑{ふくざつ}な〕機構{きこう}、仕組み{しくみ}〔目的達成{もくてき たっせい}のために作られた複雑{ふくざつ}な〕機構{きこう}、機関{きかん}〈米〉〔政治{せいじ}の〕集票組織{しゅうひょう そしき}◆【同】
\\	〔機械{きかい}のように〕正確{せいかく}に[感情{かんじょう}を入れず]行動{こうどう}[仕事{しごと}を]する人〔劇場{げきじょう}の〕仕掛け、からくり◆古典的な演劇で使われる仕掛けを指す。〔文学作品{ぶんがく さくひん}の〕超自然的存在{ちょう しぜん てき そんざい}◆神や精霊など、筋を作者の都合の良いように解決してくれるもの。【形】機械{きかい}の、機械{きかい}に似たレベル1、発音
\\	、カナマシーン、マシン、変化《動》
\\	、分節
\\	変化形 
\\	《複》
\\	【名】〔ある働きをする〕機械{きかい}、装置{そうち}〔機械{きかい}の〕機構{きこう}、構造{こうぞう}、メカニズム〔機械{きかい}を連想{れんそう}させる〕機構{きこう}、働き、仕組み{しくみ}〔あるものを実現{じつげん}する〕方法{ほうほう}、過程{かてい}、作用{さよう}《心理学》機制{きせい}◆【参考】
\\	《化学》反応機構{はんのう きこう}◆【同】
\\	《哲学》機械論{きかいろん}◆あらゆる自然現象は物質とその作用によって説明できるとする考え方。《医》機序{きじょ}レベル4、発音
\\	、カナメカニズム、変化《複》
\\	、分節
\\	戦争では次に何が起こるか分からない。	変化形 
\\	【自動】戦う、戦争{せんそう}する【名】戦争{せんそう}、戦い、争い ・
\\	戦争では次に何が起こるか分からない。軍事{ぐんじ}撲滅キャンペーン◆社会の害悪を撲滅させるための運動レベル1、発音
\\	、カナウォー、変化《動》
\\	変化形 
\\	《複》
\\	【名】《軍事》〔空軍{くうぐん}や海軍{かいぐん}に対する〕陸軍{りくぐん}《軍事》〔国全体{くに ぜんたい}の〕陸軍{りくぐん}◆通例、
\\	《軍事》〔独立{どくりつ}して行動{こうどう}できる〕野戦軍{やせん ぐん}、軍団{ぐんだん}、旅団{りょだん}、師団{しだん}〔ある目的{もくてき}を持つ大きな〕集団{しゅうだん}、部隊{ぶたい}〔人や物の〕大勢{おおぜい}、多数{たすう}、大群{たいぐん}レベル2、発音
\\	、カナアーミー、アーミィ、変化《複》
\\	、分節
\\	変化形 
\\	【自動】軍人{ぐんじん}になる、兵役{へいえき}につく仕事をしているふりをする【名】《軍事》兵士{へいし}、軍人{ぐんじん}《軍事》下士官{かしかん}◆【同】
\\	〔大義{たいぎ}などのために闘う〕闘士{とうし}、戦士{せんし}〔犯罪組織{はんざい そしき}のボスの〕手下{てした}、兵隊{へいたい}《昆虫》兵隊{へいたい}アリ◆【同】
\\	レベル2、発音
\\	、カナソウルジャー、ソルジャー、変化《動》
\\	、分節
\\	変化形 
\\	《複》
\\	【他動】~を武装{ぶそう}させる【名】武器{ぶき}、兵器{へいき}、凶器{きょうき}、対抗手段{たいこう しゅだん}レベル3、発音
\\	、カナウェポン、ウエポン、変化《動》
\\	、分節
\\	勝利には100人の父親がいるが敗北は孤児である。◆カストロ政権の転覆を目的としたピッグズ湾
\\	侵攻作戦が失敗に終った際にケネディ大統領が引用した言葉。 ・
\\	アンの勝ちだ。	変化形 
\\	《複》
\\	【名】勝利{しょうり}、優勝{ゆうしょう}、克服{こくふく}、征服{せいふく} ・
\\	勝利には100人の父親がいるが敗北は孤児である。◆カストロ政権の転覆を目的としたピッグズ湾
\\	侵攻作戦が失敗に終った際にケネディ大統領が引用した言葉。 ・
\\	アンの勝ちだ。レベル3、発音
\\	、カナビクトリー、変化《複》
\\	、分節
\\	敵の敵は味方である。	変化形 
\\	《複》
\\	【名】敵、敵国{てきこく}、かたき ・
\\	敵の敵は味方である。敵国{てきこく}【形】敵の、ライバルのレベル2、発音
\\	、カナエナミー、エナミィ、エネミー、エネミィ、変化《複》
\\	、分節
\\	誰もが犯罪の被害者になり得る。 ・
\\	日本のメーカーは、彼らの製品の模造品を作る外国の偽造業者の餌食となっている。	変化形 
\\	《複》
\\	【名】〔人による暴力{ぼうりょく}の〕被害者{ひがいしゃ} ・
\\	誰もが犯罪の被害者になり得る。〔戦争{せんそう}や災害{さいがい}などの〕犠牲者{ぎせいしゃ}、被災者{ひさいしゃ}〔詐欺{さぎ}などの〕かも、餌食{えじき} ・
\\	日本のメーカーは、彼らの製品の模造品を作る外国の偽造業者の餌食となっている。〔宗教儀式{しゅうきょう ぎしき}の〕いけにえ、人身御供{ひとみ ごくう}レベル3、発音
\\	、カナビクチム、ビクティム、変化《複》
\\	、分節
\\	イディオムやフレーズ 
\\	【間投】静かに、黙ってさようなら、それじゃあ【名】〔戦争{せんそう}や紛争{ふんそう}のない〕平和{へいわ}、和平{わへい}〔紛争{ふんそう}などを終わらせる〕講話{こうわ}[和平{わへい}]条約{じょうやく}〔心や精神{せいしん}の〕安らぎ、平安{へいあん}〔社会{しゃかい}の〕治安{ちあん}、秩序{ちつじょ}〔人間関係{にんげん かんけい}の〕友好{ゆうこう}、良好{りょうこう}な関係{かんけい}〔動きや騒音{そうおん}のない〕静けさ、静謐{せいひつ}レベル2、発音
\\	、カナピース
\\	この緑ランプの点灯は、コンピューターに電源が投入されていることを示します。 ・
\\	アメリカは大国{たいこく}[強国{きょうこく}]だ。 ・
\\	5の7乗を求めなさい。	変化形 
\\	【他動】~に動力{どうりょく}・電力{でんりょく}を供給{きょうきゅう}する ・
\\	この緑ランプの点灯は、コンピューターに電源が投入されていることを示します。【名】〔人間・動物・機械などが生み出す〕力、能力{のうりょく}、勢力{せいりょく}(政治{せいじ})権力{けんりょく}、(法的{ほうてき})権限{けんげん}、権威{けんい}、支配権{しはいけん}、政権{せいけん}権力者{けんりょくしゃ}、有力者{ゆうりょくしゃ}、大国{たいこく} ・
\\	アメリカは大国{たいこく}[強国{きょうこく}]だ。〔法的権限{ほうてき けんげん}の〕委任状{いにんじょう}◆【同】
\\	出力{しゅつりょく}、電力{でんりょく}、電源{でんげん}、入力電圧{にゅうりょく でんあつ}◆【略】
\\	テレビのリモコン《数学》累乗{るいじょう}、指数{しすう}、乗 ・
\\	5の7乗を求めなさい。《統計》検出力{けんしゅつ りょく}仕事率{しごとりつ}知力{ちりょく}、体力{たいりょく}、精神力{せいしんりょく}、才能{さいのう}、才覚{さいかく}◆通例
\\	当局{とうきょく}
\\	能天使{のうてんし}◆第6位の天使レベル1、発音
\\	、カナパウア、パワー、変化《動》
\\	、分節
\\	変化形 
\\	《複》
\\	【名】〔自信{じしん}はあるが裏付けはない個人{こじん}の〕意見{いけん}、私見{しけん}〔裏付けのある専門家{せんもんか}の〕見解{けんかい}、鑑定{かんてい}〔人やものの〕評価{ひょうか}、判定{はんてい}〔世間{せけん}の優勢{ゆうせい}な〕考え方、見方{みかた}《法律》〔法廷{ほうてい}で述べられる〕判決理由{はんけつ りゆう}レベル2、発音
\\	、カナオピニオン、変化《複》
\\	、分節
\\	変化形 
\\	《複》
\\	【名】〔政治権力構造{せいじ けんりょく こうぞう}の〕革命{かくめい}〔考え方や慣例{かんれい}などの〕大改革{だい かいかく}、革命{かくめい}〔軸を中心{ちゅうしん}にした〕回転{かいてん}(運動{うんどう})〔季節{きせつ}などの〕一巡り[周り]《地学》変革{へんかく}◆地殻が大きく変動して、造山活動などが活発になる時期。《天文》公転{こうてん}◆【対】
\\	レベル3、発音
\\	、カナリボルーション、レボリューション、レボルーション、変化《複》
\\	、分節
\\	7分の1は6桁を一周期とする循環小数です。	変化形 
\\	《複》
\\	【名】ピリオド、終わり、終止符{しゅうしふ}、完結文{かんけつ ぶん}以上{いじょう}、〔議論はこれで〕もうおしまい、あとはつべこべ[ごちゃごちゃ]言うな◆発言や文章の最後で使う。〔一定の〕期間{きかん}、時期{じき}〔ある特定の〕年月、時代{じだい}〔授業などの〕時間{じかん}〔繰り返される現象の〕1周期{しゅうき}生理{せいり}、生理日{せいりび}、月経{げっけい}《数学》循環小数{じゅんかん しょうすう}、周期関数{しゅうき かんすう}の一周期{いっしゅうき}、数字{すうじ}のコンマで区切られたグループ ・
\\	7分の1は6桁を一周期とする循環小数です。= 
\\	【形】年代物{ねんだいもの}のレベル2、発音
\\	、カナピアリアッド、ピアリオッド、ピリオド、変化《複》
\\	、分節
\\	アメリカでは新学期は9月に始まります。 ・
\\	「5人中3人の割合で、兵士たちは
\\	で苦しんでいます」「おっしゃっているのは
\\	のことですよね」
\\	変化形 
\\	"【他動】~を称する[呼ぶ・名づける・命名{めいめい}する]、~という名前{なまえ}を付ける[授ける・与える・付与{ふよ}する]【名】〔限られた〕期間{きかん}、時間{じかん}〔仕えている〕任期{にんき}、刑期{けいき}、勤務期間{きんむ きかん}〔学校{がっこう}の〕学期{がっき}、〔裁判所{さいばんしょ}の〕開廷期{かいてい き}、〔議会{ぎかい}の〕会期{かいき} ・
\\	アメリカでは新学期は9月に始まります。〈古〉終わり、終末{しゅうまつ}〔正常{せいじょう}な出産{しゅっさん}の〕予定日{よていび}〔借金返済{しゃっきん へんさい}などの〕期限{きげん}、期日{きじつ}〔ある特別{とくべつ}な意味{いみ}を持つ〕用語{ようご}、言葉{ことば} ・
\\	「5人中3人の割合で、兵士たちは
\\	で苦しんでいます」「おっしゃっているのは
\\	のことですよね」
\\	〔ある人の〕言葉遣い{ことばづかい}、言い回し、表現{ひょうげん}
\\	〔契約{けいやく}で同意{どうい}された〕条項{じょうこう}、条件{じょうけん}
\\	〔人と人との〕仲、間柄{あいだがら}、関係{かんけい}《数学》〔数式{すうしき}や数列{すうれつ}の〕項《論理学》名辞{めいじ}境界{きょうかい}石[柱・標識{ひょうしき}]◆【同】
\\	レベル2、発音
\\	、カナターム、変化《動》
\\	私たちはいいムードで見詰め合っていました。彼は私にキスしようとしていると私は思いました。けれどそのとき突然ドアをノックする音がして、ムードが壊れてしまいました。	変化形 
\\	《複》
\\	【名】〔短い〕わずかの間、一瞬{いっしゅん}、瞬間{しゅんかん}〔ある特定{とくてい}の〕時点{じてん}、時期{じき}今、現在{げんざい}、この時点{じてん}〔重要{じゅうよう}な時の〕節目{ふしめ}、機会{きかい}〔短いが〕素晴{すば}らしい[興味深い]時[瞬間{しゅんかん}]◆通例、
\\	〈話〉〔顕著{けんちょ}な〕重要性{じゅうよう せい}、重大{じゅうだい}さ◆【参考】
\\	《物理》〔軸の周りの〕回転{かいてん}を起こす力◆【略】
\\	《物理》〔回転{かいてん}の〕モーメント◆力の大きさと回転の中心との距離の積で表される。◆【略】
\\	《統計》積率{せきりつ}、モーメント《哲学》〔弁証法哲学{べんしょうほう てつがく}の〕契機{けいき}〈俗〉〔恋愛{れんあい}などで〕いいムード(のひととき) ・
\\	私たちはいいムードで見詰め合っていました。彼は私にキスしようとしていると私は思いました。けれどそのとき突然ドアをノックする音がして、ムードが壊れてしまいました。【接続】
\\	~するとすぐにレベル2、発音
\\	、カナモーメント、モウメント、モメント、変化《複》
\\	、分節
\\	変化形 
\\	【自動】〔光学機器{こうがく きき}などで〕観測{かんそく}する〔銃で〕狙う【他動】~が目に入る、~がいることに気付く~を観測{かんそく}する、〔光学機器{こうがく きき}で〕~を調べる〔銃で〕~を狙う、~に照準{しょうじゅん}を定める〔銃の〕照準{しょうじゅん}を調整{ちょうせい}する【名】視力{しりょく}、視覚{しかく}、見る能力{のうりょく}見る[見える・目に留まる]こと視野{しや}、視界{しかい}、見える範囲{はんい}景色{けしき}、眺め、見え(てい)る物見所{みどころ}、名所{めいしょ}◆通例、
\\	〈話〉みっともないもの、だらしない格好{かっこう}〔銃や測量器{そくりょうき}の〕照準器{しょうじゅんき}〔銃や測量器{そくりょうき}の〕目標{もくひょう}、狙い近い将来{しょうらい}、間近{まぢか}意見{いけん}、見解{けんかい}〈米方言〉多数{たすう}、大量{たいりょう}レベル2、発音
\\	、カナサイト、変化《動》
\\	この電子メールをブラウザでご覧ください。 ・
\\	これを拒否とは受け取らないでください。 ・
\\	この部屋は夕日がよく見える。	変化形 
\\	【他動】~を見る、眺める、視察{しさつ}する ・
\\	この電子メールをブラウザでご覧ください。~を(…と)見なす、考える◆【用法】
\\	これを拒否とは受け取らないでください。~を思い巡らす【名】見ること、一見{いっけん}、一覧{いちらん}視界{しかい}、視野{しや}、視力{しりょく}光景{こうけい}、景色{けしき}、眺望{ちょうぼう} ・
\\	この部屋は夕日がよく見える。意見{いけん}、見識{けんしき}、考え、物の見方{みかた}、見解{けんかい}、見通し{みとおし}《コ》〔ある観点からのデータ・画像の〕表示{ひょうじ}〔学問{がくもん}などの〕概説{がいせつ}、概論{がいろん}目的{もくてき}、信念{しんねん}〔景色{けしき}や建物{たてもの}などの〕絵、図、写真{しゃしん}レベル2、発音
\\	、カナビュー、ブュー、変化《動》
\\	変化形 
\\	《複》
\\	【名】吸い込む[吐き出す]息、息吹{いぶき}一息{ひといき}、一呼吸{ひとこきゅう}呼吸{こきゅう}すること、息を吸う[吐く]こと、息遣い{いきづかい}〔自然{しぜん}な〕呼吸{こきゅう}の能力{のうりょく}〈文〉生気{せいき}、生命力{せいめいりょく}〔匂{にお}いなどがある〕漂ってくる空気{くうき}〔わずかな〕休止{きゅうし}、休息{きゅうそく}〔かすかな〕風のそよぎ、そよ風〔わずかな〕兆し、しるしささやき、静かな話し声《言語学》〔無声音{むせいおん}の〕呼気{こき}レベル2、発音
\\	、カナブレス、変化《複》
\\	心配するな。/心配することないよ。/気にすんなって。/気にしないで。/思い悩むな。/うろたえるな。 ・
\\	彼から真実を聞き出すことは簡単だった。 ・
\\	彼の顔から汗が噴き出した[したたり落ちた]。	変化形 
\\	【自動】汗をかく[出す]〔内部{ないぶ}から〕水分{すいぶん}がにじみ出る、〔物の表面{ひょうめん}が〕汗をかく〔水分{すいぶん}が〕蒸発{じょうはつ}する、〔作物{さくもつ}などが〕乾燥{かんそう}する、〔乾燥中{かんそう ちゅう}にタバコの葉が〕発酵{はっこう}する〈話〉あくせく[汗水垂{あせみず た}らして]働く〈話〉心配{しんぱい}する、やきもきする【他動】〔水分{すいぶん}を表面{ひょうめん}に〕浸出させる〔水蒸気{すいじょうき}を表面{ひょうめん}に〕凝結{ぎょうけつ}させる〈話〉〔汗をかくほど〕~を心配{しんぱい}する、気に病む ・
\\	心配するな。/心配することないよ。/気にすんなって。/気にしないで。/思い悩むな。/うろたえるな。(人)に汗をかかせる、(人)を発汗{はっかん}させる~を過重に働かせる〔汗で〕~をぬらす[湿らせる]〈俗〉(人)を尋問{じんもん}する、〔情報{じょうほう}などを無理{むり}に〕聞き出す ・
\\	彼から真実を聞き出すことは簡単だった。《冶金》~を溶接{ようせつ}する〔野菜{やさい}などを〕蒸す【名】汗◆【類】
\\	彼の顔から汗が噴き出した[したたり落ちた]。汗を流す[かいている]こと〔大気中{たいき ちゅう}から凝固{ぎょうこ}した〕水蒸気{すいじょうき}、水滴{すいてき}苦役{くえき}、つらい[骨の折れる・つまらない]仕事{しごと}〈話〉心配{しんぱい}している[苛立っている]状態{じょうたい}
\\	〈話〉=
\\	〔競馬{けいば}の出走前{しゅっそう まえ}の〕足慣らしレベル2、発音
\\	、カナスウェット、スエット、変化《動》
\\	変化形 
\\	【自動】騒音{そうおん}[雑音{ざつおん}]を立てる〈古〉ペラペラとしゃべる【他動】〔うわさなどを〕広める【名】〔不快{ふかい}な〕騒音{そうおん}、雑音{ざつおん}〔聞こえてくる〕音、声叫び声、怒号{どごう}、大騒ぎ{おおさわぎ}〈話〉不平{ふへい}、不満{ふまん}〈話〉うわさ話、陰口{かげぐち}、ひそひそ話《物理》〔電気信号{でんき しんごう}の〕ノイズ《コ》〔検索結果{けんさく けっか}などの〕ノイズ◆求めるものとは無関係な、不要な情報。レベル1、発音
\\	、カナノイズ、変化《動》
\\	彼女は低く甘い歌声で歌った。 ・
\\	あなたは定期的に音声訓練をしなければなりません。 ・
\\	湖の周囲の道を歩いていると聞こえるのは優しい風の音だけだった。 ・
\\	この会議では誰にでも発言権があります。 ・
\\	民の声は神の声なり。◆ことわざ	変化形 
\\	【他動】~を声に出す、表明{ひょうめい}する《イ》~を特権{とっけん}ユーザにする、
\\	モードで〕~に発言権{はつげんけん}を与える
\\	【名】〔人間の〕声《音楽》歌声 ・
\\	彼女は低く甘い歌声で歌った。発声[発話]能力 ・
\\	あなたは定期的に音声訓練をしなければなりません。〔人間の声に似た〕音 ・
\\	湖の周囲の道を歩いていると聞こえるのは優しい風の音だけだった。発言権 ・
\\	この会議では誰にでも発言権があります。〔表明された〕意見、考え ・
\\	民の声は神の声なり。◆ことわざ〔意見{いけん}などを〕表明{ひょうめい}する手段{しゅだん}、代弁者{だいべんしゃ}《音楽》〔楽曲のパートを担う〕歌手《音楽》〔楽曲の〕声部《言語学》〔声帯の〕振動《文法》態レベル1、発音
\\	、カナボイス、変化《動》
\\	変化形 
\\	《複》
\\	【名】運動{うんどう}[スポーツ]選手{せんしゅ}、スポーツマン、スポーツ愛好家{あいこう か}、競技者{きょうぎしゃ}〈英〉陸上競技選手{りくじょう きょうぎ せんしゅ}レベル4、発音
\\	、カナアスリート、変化《複》
\\	、分節
\\	変化形 
\\	《複》
\\	【名】〔集合的に〕聴衆{ちょうしゅう}、観客{かんきゃく}、観衆{かんしゅう}◆講演会、演劇、コンサートなどの実演を目の前にした、また映画などを見るために集まった観衆。〔ラジオやテレビなどの〕視聴者{しちょうしゃ}◆集合的に用いる。〔特定{とくてい}の作家{さっか}の〕読者{どくしゃ}◆集合的に用いる。〔集合的{しゅうごう てき}に〕信奉者{しんぽうしゃ}、支持者{しじしゃ}、ファン〔正式{せいしき}な〕引見{いんけん}、謁見{えっけん}、拝謁{はいえつ}意見{いけん}を述べる[聞いてもらう]機会{きかい}レベル2、発音
\\	、カナオージアンス、オーディアンス、オーディェンス、オーディエンス、変化《複》
\\	、分節
\\	《前言に対して》非難しているんじゃないよ。	変化形 
\\	《複》
\\	【名】〔人や物事{ものごと}への〕批判{ひはん}、批評{ひひょう}◆【用法】肯定的な批判も含まれるが、否定的な場合が多い。〔悪いことや欠点{けってん}と思われることへの〕非難{ひなん}、反論{はんろん} ・
\\	《前言に対して》非難しているんじゃないよ。〔芸術作品{げいじゅつ さくひん}の〕批評{ひひょう}、評論{ひょうろん}◆【同】
\\	〔聖書{せいしょ}の〕原典批判{げんてん ひはん}レベル4、発音
\\	、カナクリティシズム、クリティスィズム、クリテシズム、変化《複》
\\	、分節
\\	それは以前のような味がしない。 ・
\\	それらはどんな味がするのかしら。 ・
\\	食べたことがないのに、なぜ自分がそれを好きではないと分かるんだ? ・
\\	あなたの料理をいただくのはこれが初めてです。 ・
\\	こういう味は初めて[初体験]です。◆【場面】食べ物や飲み物について。普通は「すごくおいしい」という意味。 ・
\\	彼には貧乏の経験がない。 ・
\\	最近、人の好みは変わりつつある。	変化形 
\\	【自動】〔舌で〕味見{あじみ}をする、味加減{あじかげん}を調べる〔食べ物などが〕~の味がする ・
\\	それは以前のような味がしない。 ・
\\	それらはどんな味がするのかしら。〔少しだけ〕食べる、飲む、味わう〔初めてまたは少しだけ〕経験{けいけん}する、味わう【他動】〔舌で~の〕味が分かる、味を見分ける{みわける}〔食べ物などを〕味見{あじみ}する、味加減{あじかげん}を見る〔食べ物などを少しだけ〕食べる、飲む、味わう◆【用法】否定文で使われることが多い。 ・
\\	食べたことがないのに、なぜ自分がそれを好きではないと分かるんだ? ・
\\	あなたの料理をいただくのはこれが初めてです。 ・
\\	こういう味は初めて[初体験]です。◆【場面】食べ物や飲み物について。普通は「すごくおいしい」という意味。〔初めてまたは少しだけ物事{ものごと}を〕経験{けいけん}する、味わう〔食べ物が~の〕味がする【名】〔舌の味蕾{みらい}の〕味覚{みかく}、味感覚〔舌の味蕾{みらい}で感じる〕味、味感{みかん}◆甘味・苦味・酸味・辛味・渋味など。〔匂{にお}いや食感{しょっかん}を加えた食べ物の〕風味{ふうみ}、味わい〔食べ物などを〕試しに味わうこと、味見{あじみ}すること〔最初{さいしょ}のまたはわずかな〕経験{けいけん}、体験{たいけん} ・
\\	彼には貧乏の経験がない。〔個人{こじん}の〕好み、嗜好{しこう} ・
\\	最近、人の好みは変わりつつある。〔美に対する〕審美眼{しんびがん}、センス〔独特{どくとく}の〕風情{ふぜい}、味わい、趣〔わずかな量の〕一口{ひとくち}、一飲み〔経験{けいけん}がもたらす〕後味{あとあじ}、思い〔社会的規範{しゃかい てき きはん}に対する〕分別{ふんべつ}、わきまえレベル1、発音
\\	、カナテイスト、変化《動》
\\	資料かカタログがありましたら、ご提供ください。		【名】〔散文{さんぶん}や韻文{いんぶん}の優れた〕文学{ぶんがく}(作品{さくひん})〔特定{とくてい}の時代{じだい}や地域{ちいき}の〕文学{ぶんがく}〔特定{とくてい}の分野{ぶんや}の〕文献{ぶんけん}、論文{ろんぶん}〔文学作品{ぶんがく さくひん}の〕創作(術){そうさく(じゅつ)}作家{さっか}[著述{ちょじゅつ}]業〔印刷{いんさつ}された〕パンフレット、チラシ ・
\\	資料かカタログがありましたら、ご提供ください。レベル3、発音
\\	、カナリタラチャー、リテラチャー、リテラチュア、分節
\\	変化形 
\\	【他動】~を書く生み出す、加える【名】作者{さくしゃ}、著者{ちょしゃ}、執筆者{しっぴつ しゃ}、作家{さっか}◆【同】
\\	創造者{そうぞう しゃ}、起草者{きそうしゃ}、命名者{めいめい しゃ}〔悪事などの〕張本人{ちょうほんにん}〔ある作家{さっか}の〕作品{さくひん}、著書{ちょしょ}、著作物{ちょさくぶつ}レベル3、発音
\\	、カナオーサー、変化《動》
\\	、分節
\\	その男に弁護士として仕事をさせた。 ・
\\	彼は父親によって雑穀商に奉公に出された。 ・
\\	その店からは衣類を一枚たりとも買ったことがない。	変化形 
\\	《複》
\\	【他動】〔法律{ほうりつ}の業務{ぎょうむ}に〕(人)を契約{けいやく}により従事{じゅうじ}させる ・
\\	その男に弁護士として仕事をさせた。〈古〉(人)を年季奉公{ねんき ぼうこう}させる ・
\\	彼は父親によって雑穀商に奉公に出された。【名】〔雑誌{ざっし}や新聞{しんぶん}の〕記事{きじ}、論説{ろんせつ}、論文{ろんぶん}◆小説などのフィクションを含まない。〔協定書や憲法などの〕条項{じょうこう}品物{しなもの}、品目{ひんもく}、項目{こうもく}◆同類のものの一部を指す。 ・
\\	その店からは衣類を一枚たりとも買ったことがない。《文法》冠詞{かんし}レベル3、発音
\\	、カナアーティクル、アーテクル、変化《複》
\\	、分節
\\	変化形 
\\	《複》
\\	【名】〔演劇{えんげき}や歌劇{かげき}の〕場、シーン◆場所や時間などの設定が同一の間に行われる一連の演技を指す。いくつかの「場
\\	が集まって「幕
\\	を構成する。〔映画{えいが}や小説{しょうせつ}などの〕場面{ばめん}、シーン◆一般に一つの出来事が展開される間を指し、映画では一つのショット
\\	または相互に深く関連するショットの集まりで構成される。〔映画{えいが}や小説{しょうせつ}などの〕設定{せってい}、背景{はいけい}、舞台{ぶたい}◆物語のすべてまたは一部が展開される環境や背景。劇場などでは小道具を含めた舞台装置を指すこともある。〔ある出来事{できごと}が起きる〕現場{げんば}、場所{ばしょ}〔ある場所{ばしょ}や出来事{できごと}の〕光景{こうけい}、眺め◆絵や写真として表現されたものを指すことが多い。〔感情{かんじょう}をあらわにして演じる〕醜態{しゅうたい}、〔見ていて恥ずかしいほど〕大声{おおごえ}で騒ぐこと〔活動{かつどう}の〕領域{りょういき}、環境{かんきょう}、世界{せかい}〔語られる〕出来事{できごと}、エピソード〈話〉〔一般的{いっぱん てき}に〕状況{じょうきょう}、状態{じょうたい}〈俗〉好きな[得意{とくい}な]こと[もの]レベル1、発音
\\	、カナシーン、スィーン、変化《複》
\\	《マニュアル》用紙の方向は、ランドスケープ[横]でもポートレート[縦]でも可能である。	変化形 
\\	【自動】庭師{にわし}をする【他動】~を造園{ぞうえん}する〔造園整備して〕~を美化{びか}する【名】風景(画){ふうけい(が)}、景色{けしき}、景観{けいかん}、地勢{ちせい}、地形{ちけい}、造園{ぞうえん}展望{てんぼう}《印刷》横方向{よこ ほうこう}、横長書式{よこなが しょしき}◆【対】
\\	【形】《印刷》横長{よこなが}の◆【対】
\\	《マニュアル》用紙の方向は、ランドスケープ[横]でもポートレート[縦]でも可能である。【副】《印刷》横長{よこなが}にレベル4、発音
\\	、カナランドゥスケイプ、ランドスケープ、変化《動》
\\	、分節
\\	平日はお互いに会う時間がほとんど取れなかった。	イディオムやフレーズ 
\\	【名】〔義務から解放された〕余暇{よか}、暇、いとま、暇な[自由{じゆう}な]時間{じかん} ・
\\	平日はお互いに会う時間がほとんど取れなかった。〔暇な時間{じかん}を過ごす〕気晴{きば}らし、娯楽{ごらく}、レジャー【形】暇な、余暇{よか}の、用事{ようじ}のないレベル3、発音
\\	、カナ
\\	リージャー、
\\	レジャー、【発音の地域差】アメリカ西部ではリージャーと発音され、東部ではレジャーと発音されることがある。/分節
\\	変化形 
\\	【自動】〔遠方{えんぽう}の場所{ばしょ}に〕旅する、旅行{りょこう}する〔ある地域{ちいき}を〕周遊{しゅうゆう}する、巡回{じゅんかい}する〔光や音などがある速度{そくど}で〕伝わる、進む〈話〉す早く動く〔ある集団内{しゅうだん ない}で〕交際{こうさい}する、付き合いをする〔部品{ぶひん}などが〕動く、進む◆ガイドや溝などに沿って動くこと。〔知らせやうわさが〕伝わる、伝達{でんたつ}される〔輸送{ゆそう}で品質{ひんしつ}が〕落ちない、維持{いじ}される〔バスケットボールで〕トラベリングする◆【参考】
\\	【他動】〔ある地域{ちいき}を〕周遊{しゅうゆう}する、巡回{じゅんかい}する〔ある距離{きょり}を〕進む、渡る【名】〔遠方{えんぽう}への〕旅、旅行{りょこう}〔光や音の〕伝播{でんぱ}、移動{いどう}〔ピストンなどの〕運動距離{うんどう きょり}〔道路{どうろ}のある地点{ちてん}の〕交通量{こうつうりょう}
\\	〔長期{ちょうき}の〕旅行{りょこう}◆ある特定の人や集団による、複数の地域への長期旅行。
\\	旅行記{りょこうき}、紀行文{きこうぶん}【形】旅行者用{りょこうしゃ よう}の、旅行{りょこう}に適したレベル1、発音
\\	、カナトゥラブル、トラベル、変化《動》
\\	、分節
\\	私は石につまずいて転倒しました。 ・
\\	「大丈夫?」「うん、ちょっとつまずいただけ」 ・
\\	ブレーカーが落ちた[切れた・上がった]。 ・
\\	たわごとはやめろ。/ばかも休み休み言え。 ・
\\	良い旅を。 ・
\\	「見てよ! あの猫、鳥を追っかけてるよ!」「ほんと面白いコね!」 ・
\\	あなたたちは見ていてとても楽しいです。
\\	変化形 
\\	"【自動】軽快{けいかい}な足取りで歩む、軽快{けいかい}に動く踏み外す、つまずく、転ぶ、失敗{しっぱい}する、失策{しっさく}をする ・
\\	私は石につまずいて転倒しました。 ・
\\	「大丈夫?」「うん、ちょっとつまずいただけ」〔止め金などが〕外れる、〔回路{かいろ}が〕切れる ・
\\	ブレーカーが落ちた[切れた・上がった]。〔機器{きき}が〕作動{さどう}する〈俗〉麻薬{まやく}で幻覚症状{げんかく しょうじょう}を起こす、麻薬{まやく}で幻覚{げんかく}を感じる〈俗〉ばかげたことを言う◆「幻覚剤でトリップしているような発言」というほどの意味。 ・
\\	たわごとはやめろ。/ばかも休み休み言え。【他動】~をつまずかせる、転ばせる、妨害{ぼうがい}する、失敗{しっぱい}させる、(人)の足をすくう、揚げ足を取る〔止め金などを〕(引き)外す、〔回路{かいろ}を〕切る〔機器{きき}を〕始動{しどう}させる【名】〔休暇{きゅうか}や商用{しょうよう}の〕旅、旅行{りょこう}、遠出{とおで} ・
\\	良い旅を。〔ある目的{もくてき}で〕出掛けること、訪問{ほうもん}〔つまずいて〕転ぶこと、つまずき〔人を〕つまずかせること[技]軽い足取り[ステップ]間違い、しくじり、過ち《機械》歯止め{はどめ}〈俗〉〔麻薬{まやく}による〕幻覚体験{げんかく たいけん}、トリップ〈俗〉強烈{きょうれつ}[刺激的{しげき てき}]な体験{たいけん}〈俗〉〔一時的{いちじ てき}に熱中{ねっちゅう}する〕趣味{しゅみ}、関心事{かんしんじ}〈俗〉面白い[変わった]こと[もの・人] ・
\\	「見てよ! あの猫、鳥を追っかけてるよ!」「ほんと面白いコね!」 ・
\\	あなたたちは見ていてとても楽しいです。レベル1、発音
\\	、カナトゥリップ、トリップ、変化《動》
\\	【名】〈主に米〉〔旅行などの〕手荷物{てにもつ}、携行品{けいこうひん}〈話〉精神的{せいしんてき}な重荷{おもに}[負担{ふたん}]、情緒的{じょうちょ てき}な問題{もんだい}、心の悩み〈話〉〔前進・進展などを妨げる〕お荷物、障害{しょうがい}〈俗・侮蔑的〉売春婦{ばいしゅんふ}レベル3、発音
\\	、カナバゲージ、バゲッジ、分節
\\	変化形 
\\	【自動】売買{ばいばい}[取引{とりひき}・貿易{ぼうえき}]をする不正取引をする【他動】~を売買{ばいばい}する~を不正取引する【名】交通(量){こうつう(りょう)}、通行(量){つうこう(りょう)}人の出入り{でいり}、行き来、往来{おうらい}《コ》トラフィック、トラヒック◆通信回線の中を行き交うデータの“交通量”のこと。《コ》〔ウェブサイトなどの〕アクセス量、〔データの〕転送量{てんそう りょう}運輸{うんゆ}、運輸業{うんゆ ぎょう}、貨物輸送{かもつ ゆそう}商売{しょうばい}、交渉{こうしょう}、取引{とりひき}= 
\\	レベル2、発音
\\	、カナトゥラフィック、トラフィック、変化《動》
\\	、分節
\\	狩猟家はライオンを穴まで追跡しました。	変化形 
\\	【自動】跡を付ける[追う]道を進む[たどる]〔映画{えいが}で〕移動撮影{いどうさつえい}をする〔車の左右{さゆう}の車輪{しゃりん}が〕正しい間隔{かんかく}を保つ〈話〉〔困難{こんなん}な〕旅をする【他動】〔人や動物{どうぶつ}などの〕跡をたどる[追う] ・
\\	狩猟家はライオンを穴まで追跡しました。〔道を〕たどって[通って]いく〔進行{しんこう}や発展{はってん}を〕追っていく、見守{みまも}る〔映画{えいが}で動く対象{たいしょう}を〕撮影{さつえい}する〔レーダーで飛行機{ひこうき}などを〕監視{かんし}する、追跡{ついせき}する〔靴などで泥を持ち込んで〕跡を付ける〔路線{ろせん}に〕鉄道{てつどう}を敷く〈米〉〔生徒{せいと}を〕能力別編成{のうりょく べつ へんせい}クラスに入れる【名】〔人・動物{どうぶつ}・車などの通った〕跡、足跡{あしあと}、わだち〔踏み固められた〕小道{こみち}、通り道〔列車{れっしゃ}の〕線路{せんろ}、軌道{きどう}〔人や物が動く〕行路{こうろ}、進路{しんろ}〔思考{しこう}や行動{こうどう}の〕流れ、筋道{すじみち}、手順{てじゅん}〔思考{しこう}や出来事{できごと}の〕連続{れんぞく}、ひとつながり《スポーツ》〔競技場{きょうぎじょう}の〕トラック、走路{そうろ}〈米〉《スポーツ》トラック競技{きょうぎ}〈米〉《スポーツ》陸上競技{りくじょう きょうぎ}◆【同】〈米〉
\\	〈英〉
\\	《音楽》〔録音{ろくおん}の単位{たんい}の〕トラック◆歌や楽器のパートごとに、別々に録音されたもの。後で合成されて一つの音楽作品となる。《音楽》〔アルバムなどを構成{こうせい}する〕曲、トラック◆別々に録音された一つの曲。〔映画{えいが}の〕サウンド・トラック◆【同】
\\	〔映画{えいが}の〕トラッキング・ショット◆【同】
\\	《コ》トラック◆ディスク状の記録媒体の記録単位で、ハードディスクなどでは一つの同心円、
\\	などではらせん状の1回転分に当たる。〔戦車{せんしゃ}やブルドーザーの〕無限軌道{むげん きどう}〈米〉能力別編成{のうりょく べつ へんせい}クラス◆【同】〈英〉
\\	職業経歴{しょくぎょう けいれき}、キャリア・パス〔工場{こうじょう}の〕組立{くみたて}ラインレール◆カーテン・窓枠・照明器具などを取り付けて簡単に動かせるようにしたもの。《物理》〔霧箱{きりばこ}や泡箱{あわばこ}などの〕飛跡{ひせき}〔車の前輪{ぜんりん}や後輪{こうりん}の〕左右{さゆう}の間隔{かんかく}レベル2、発音
\\	、カナトゥラック、トラック、変化《動》
\\	変化形 
\\	《複》
\\	【自動】鉄道会社{てつどう がいしゃ}で働く【他動】〈米〉~を鉄道輸送{てつどう ゆそう}する〈米〉~に鉄道{てつどう}を敷設{ふせつ}する〈米話〉(確かな証拠{しょうこ}もないのに人を)有罪{ゆうざい}にする、無実{むじつ}の罪で投獄{とうごく}する~を強行採決{きょうこう さいけつ}する、~を無理{むり}に急がせる【名】〈米〉鉄道{てつどう}(路線{ろせん})◆【同】
\\	〈英〉
\\	◆【略】
\\	〈米〉鉄道輸送{てつどう ゆそう}システム◆【同】〈英〉
\\	◆【略】
\\	【形】〈米〉鉄道{てつどう}のレベル2、発音
\\	、カナレールロード、レイルロウド、変化《動》
\\	、分節
\\	そのバーの雰囲気は、暗くて物騒な感じだった。	変化形 
\\	《複》
\\	【名】〔地球{ちきゅう}の〕大気(圏){たいき(けん)}◆重力によってとどめられている空気の層。◆【略】
\\	〔特定{とくてい}の場所{ばしょ}の〕空気{くうき}、大気{たいき}《物理》気圧{きあつ}◆【同】
\\	〔特定の場所や状況に漂う〕空気{くうき}、雰囲気{ふんいき}、様子{ようす} ・
\\	そのバーの雰囲気は、暗くて物騒な感じだった。〔芸術作品{げいじゅつ さくひん}が与える〕印象{いんしょう}、趣〔ある場所{ばしょ}が醸し出す好ましい〕ムード、雰囲気{ふんいき}レベル4、発音
\\	、カナアトゥモスフィア、アトモスフィア、変化《複》
\\	、分節
\\	変化形 
\\	《複》
\\	【1名】不可解{ふかかい}なこと、未解明{み かいめい}なこと、謎の事柄{ことがら}不思議{ふしぎ}さ、神秘的雰囲気{しんぴ てき ふんいき}推理{すいり}小説{しょうせつ}[劇・映画{えいが}]、ミステリー《キリスト教》〔神の啓示{けいじ}により伝えられる人智{じんち}を越えた〕神秘{しんぴ}《キリスト教》秘跡{ひせき}◆【同】
\\	秘儀{ひぎ}、奥義{おくぎ/おうぎ}
\\	《キリスト教》聖体{せいたい}【2名】〈古〉〔商人{しょうにん}や職人{しょくにん}の〕組合{くみあい}、ギルド〈古〉商売{しょうばい}、職業{しょくぎょう}〈古〉奇跡劇{きせきげき}レベル2、発音
\\	、カナミステリー、変化《複》
\\	、分節
\\	秘密を守れる?/誰にも言わない? ・
\\	この秘密は決して誰にも漏らすな。 ・
\\	私は秘密を漏らしたりしません。	変化形 
\\	《複》
\\	【名】秘密{ひみつ}、機密{きみつ}、内緒事{ないしょ ごと} ・
\\	秘密を守れる?/誰にも言わない? ・
\\	この秘密は決して誰にも漏らすな。 ・
\\	私は秘密を漏らしたりしません。秘伝{ひでん}、秘訣{ひけつ}、こつ、極意{ごくい}解決{かいけつ}の鍵自然界{しぜんかい}の不思議{ふしぎ}、神秘{しんぴ}【形】秘密{ひみつ}の、内緒{ないしょ}の、ひそかな、機密{きみつ}の、公表{こうひょう}されていない隠れた、奥まった、人目{ひとめ}に付かない認められていないレベル1、発音
\\	、カナシークレット、スィークレット、変化《複》
\\	、分節
\\	奇跡は起こり得る。 ・
\\	奇跡でも起きない限り無理です。/それができたら奇跡です。/まず不可能です。 ・
\\	彼が助かったのは奇跡だった。/彼は奇跡的に一命を取り留めた。	変化形 
\\	《複》
\\	【名】奇跡{きせき}、驚くべきこと、奇跡的{きせき てき}な出来事{できごと}、不思議{ふしぎ}なこと、信じられないようなこと、偉業{いぎょう} ・
\\	奇跡は起こり得る。 ・
\\	奇跡でも起きない限り無理です。/それができたら奇跡です。/まず不可能です。 ・
\\	彼が助かったのは奇跡だった。/彼は奇跡的に一命を取り留めた。レベル4、発音
\\	、カナミラクル、変化《複》
\\	、分節
\\	私は元気がないときには、本の整理をする。	変化形 
\\	【自動】体系化{たいけい か}する、組織化{そしきか}する組合{くみあい}をつくる、組合{くみあい}に加入{かにゅう}する【他動】〔組織{そしき}を〕編成{へんせい}する、組織{そしき}する〔従業員{じゅうぎょういん}に〕組合{くみあい}を作らせる[に加入{かにゅう}させる]〔個々の要素を〕構造化{こうぞうか}する、整頓{せいとん}する ・
\\	私は元気がないときには、本の整理をする。〔考えなどを〕体系化{たいけい か}する、まとめる〔イベントなどを〕計画{けいかく}する、企画{きかく}する〔人に〕心の準備{じゅんび}をさせる、うまくできるように用意{ようい}させる◆【用法】目的語に再帰代名詞を取ったり、受け身で使われたりすることが多い。レベル4、発音
\\	、カナオーガナイズ、オルガナイズ、変化《動》
\\	、分節
\\	はっきり言おう。 ・
\\	率直に言わせていただきます。	変化形 
\\	【自動】案内{あんない}する、命令{めいれい}する〔映画{えいが}を〕監督{かんとく}する、〔オーケストラを〕指揮{しき}する【他動】~を管理{かんり}する、~を運営{うんえい}する〈話〉(人)に~するよう指示{しじ}[指図{さしず}]する~を…へ向かわせる[移動{いどう}させる](人)を道案内{みちあんない}する、(人)に~への道を教える〔手紙{てがみ}などに〕宛先を書く〔コメントなどを〕(人)に言う[向ける]《音楽》〔オーケストラなどを〕指揮{しき}する〔映画{えいが}や舞台{ぶたい}を〕監督{かんとく}する【形】〔進行方向などが〕真っすぐな〔接触などが〕直接{ちょくせつ}の率直{そっちょく}な、遠慮{えんりょ}のない、単刀直入{たんとう ちょくにゅう}の ・
\\	はっきり言おう。 ・
\\	率直に言わせていただきます。〔人と会うときに〕間に人を介さない、直々{じきじき}の〔引用{いんよう}などが〕文字通り{もんじどおり}の〔選挙{せんきょ}が〕直接制{ちょくせつ せい}の《数学》〔関係{かんけい}が〕正の◆【参考】
\\	《天文》〔運行{うんこう}が〕順行{じゅんこう}の◆惑星が太陽と同じく西から東へ動くこと。〔関係{かんけい}などが〕全くの◆【参考】
\\	【副】直接{ちょくせつ}に、真っすぐにレベル2、発音《動・形・副》
\\	《形》
\\	、カナダイレクト、ディレクト、変化《動》
\\	、分節
\\	ウェイターは、レストランの一番いいテーブルに私たちを案内してくれました。 ・
\\	英語を話せるガイドが私たちをその寺院の中に案内してくれました。 ・
\\	その会社は、25カ国で事業を行っています。 ・
\\	このような研究はまだ行われていない。 ・
\\	ヘルベルト・フォン・カラヤンはそのオーケストラを35年間指揮しました。 ・
\\	先生は、その生徒が教室で悪いことをしたので居残りをさせた。	変化形 
\\	【自動】〔道などが~に〕通じる〔楽団{がくだん}の〕指揮{しき}をする《物理》伝導{でんどう}する【他動】導く(人)を案内{あんない}する ・
\\	ウェイターは、レストランの一番いいテーブルに私たちを案内してくれました。 ・
\\	英語を話せるガイドが私たちをその寺院の中に案内してくれました。〔事業・取引・会議・調査・実験などを〕行う、実施{じっし}する ・
\\	その会社は、25カ国で事業を行っています。 ・
\\	このような研究はまだ行われていない。〔事業・会合などを〕運営{うんえい}する〔楽団{がくだん}を〕指揮{しき}する ・
\\	ヘルベルト・フォン・カラヤンはそのオーケストラを35年間指揮しました。振る舞う◆【用法】
\\	《物理》伝導{でんどう}する【名】行為{こうい}、品行{ひんこう}、振る舞い、日頃の行い ・
\\	先生は、その生徒が教室で悪いことをしたので居残りをさせた。経営{けいえい}、処置法{しょちほう}、処理{しょり}レベル3、発音《名》
\\	《動》
\\	、カナカンダクト、コンダクト、変化《動》
\\	、分節
\\	私のボスは、私にニューヨークでプレゼンテーション・セミナーをするように任命しました。 ・
\\	いずれの当事者も、相手方当事者の書面による事前の同意なしに本契約を第三者に譲渡する権利を有しないものとする。◆契約書	変化形 
\\	【他動】~を割り当てる、指定{してい}する、任命{にんめい}する、配属{はいぞく}する ・
\\	私のボスは、私にニューヨークでプレゼンテーション・セミナーをするように任命しました。~を当てがう、与える、譲渡{じょうと}する ・
\\	いずれの当事者も、相手方当事者の書面による事前の同意なしに本契約を第三者に譲渡する権利を有しないものとする。◆契約書~を(…の)せいにする、帰する【名】譲受人{ゆずりうけにん}レベル5、発音
\\	、カナアサイン、変化《動》
\\	、分節
\\	ホテルのオーナーは彼をマネージャーに任命しました。	変化形 
\\	【自動】指名{しめい}[任命{にんめい}]権を行使{こうし}する【他動】~を任命{にんめい}する、選任{せんにん}する、指名{しめい}する◆【類】
\\	ホテルのオーナーは彼をマネージャーに任命しました。〔日時{にちじ}・場所{ばしょ}などを〕決める、指定{してい}する、定める〔部屋{へや}などに〕必要{ひつよう}な備品{びひん}[設備{せつび}]を備え付けるレベル3、発音
\\	、カナアポイント、変化《動》
\\	、分節
\\	人は労働に対する報酬を得るのが当然である。◆新約聖書ルカ福音書10章7節からで、イエスが弟子を伝道に送り出すときに、その働きに対して施しが得られることを述べたもの。聖職者が報酬を得ることを正当化するために、またどんな労働に対しても報酬が与えられるべきだとして労働者側の権利を主張するために用いられる。	変化形 
\\	【自動】〈米〉雇われる、仕事{しごと}を得る【他動】〈米〉〔ある特別の仕事のために一時的に人を〕雇う◆「雇う」という意味では最も口語的な語。雇う契約をした時点にのみ焦点が合わせられる。〈米〉〔お金を払って一時的にものを〕借りる〈英〉賃借りする◆【同】
\\	【名】雇用{こよう}すること、雇われている状態{じょうたい}報酬{ほうしゅう}、賃金{ちんぎん}、賃料{ちんりょう} ・
\\	人は労働に対する報酬を得るのが当然である。◆新約聖書ルカ福音書10章7節からで、イエスが弟子を伝道に送り出すときに、その働きに対して施しが得られることを述べたもの。聖職者が報酬を得ることを正当化するために、またどんな労働に対しても報酬が与えられるべきだとして労働者側の権利を主張するために用いられる。〈話〉被雇用者{ひ こようしゃ}、採用者{さいようしゃ}レベル2、発音
\\	、カナハイア、変化《動》
\\	私たちは握手を交わし、それから彼は私に名刺を渡した。 ・
\\	話を進める前に、もう少しあなたのことについて教えてほしいのですが。
\\	""はビジネスシーンで使える丁寧な表現。
\\	ミーティングを続ける前に、自己紹介をお願いします。◆身分が上の人が下の人に言う ・
\\	細かいことことは気にしないで、この仕事を進めよう。 ・
\\	私はドアのところまで行って開けました。 ・
\\	当事務所は、訴訟継続中は閉鎖となります。 ・
\\	「富は労働から生まれる」ということわざは真理を突いている。 ・
\\	「収益はどこへ行くの?」「売り上げはすべてエイズの研究と慈善活動に使われます」
\\	変化形 
\\	"【自動】〔行為を〕始める、開始する ・
\\	私たちは握手を交わし、それから彼は私に名刺を渡した。〔中断の後で〕続行する、継続する ・
\\	話を進める前に、もう少しあなたのことについて教えてほしいのですが。
\\	""はビジネスシーンで使える丁寧な表現。
\\	ミーティングを続ける前に、自己紹介をお願いします。◆身分が上の人が下の人に言う〔着実に〕はかどる、進展する ・
\\	細かいことことは気にしないで、この仕事を進めよう。〔ある方向へ〕進む、前進する ・
\\	私はドアのところまで行って開けました。《法律》〔法的手続きが〕進む ・
\\	当事務所は、訴訟継続中は閉鎖となります。〔~から〕発生する、〔~に〕由来する ・
\\	「富は労働から生まれる」ということわざは真理を突いている。【名】
\\	収益{しゅうえき}、利益{りえき}、売上高{うりあげだか}、売却代金{ばいきゃく だいきん} ・
\\	「収益はどこへ行くの?」「売り上げはすべてエイズの研究と慈善活動に使われます」レベル4、発音《名》
\\	《動》
\\	、カナプロスィード、変化《動》
\\	、分節
\\	変化形 
\\	【自動】ゆっくり進む、先延ばしになる【他動】~を先延ばしにする、~の時期{じき}を後ろにずらす〔~の進み方などを〕遅らせる、〔人を〕遅刻{ちこく}させる【名】遅らせる[遅延{ちえん}させる]こと遅れていること、予定時間{よてい じかん}に起こらないこと遅延{ちえん}[遅滞{ちたい}]時間{じかん}〔音響効果{おんきょう こうか}の〕遅延{ちえん}、ディレイ◆【参考】
\\	レベル3、発音
\\	、カナディレイ、デレイ、変化《動》
\\	、分節
\\	《マニュアル》カバーを閉じて、このボタンを押すと、カバーがロックされ、処理中は開かないようになります。	変化形 
\\	【自動】約束{やくそく}する、保証{ほしょう}する従事{じゅうじ}する、関与{かんよ}する、参加{さんか}する〔歯車などが〕係合する、かみ合う、連動{れんどう}する《軍事》戦闘{せんとう}する、交戦{こうせん}する【他動】〔ある行為{こうい}に〕(人)を従事{じゅうじ}させる、(人)を引き込む、(人)を参加{さんか}させる◆【用法】受身で用いられることが多い。〔ある仕事{しごと}のために人を〕雇う、雇用{こよう}する〔利用{りよう}するために物や場所{ばしょ}を〕手配{てはい}する、予約{よやく}する《軍事》〔敵と〕交戦{こうせん}する、〔自軍{じぐん}を〕交戦{こうせん}させる〔人の注意{ちゅうい}・関心{かんしん}などを〕引く、〔人を〕魅了する〔歯車{はぐるま}・接合部{せつごうぶ}などを〕かみ合わせる、はめ込む ・
\\	《マニュアル》カバーを閉じて、このボタンを押すと、カバーがロックされ、処理中は開かないようになります。(人)と婚約{こんやく}する◆【用法】受身で用いられることが多い。〔時間{じかん}などを〕必要{ひつよう}とする、占める〈文〉~を保証{ほしょう}[約束{やくそく}]するレベル4、発音
\\	、カナインゲイジ、エンゲイジ、変化《動》
\\	、分節
\\	彼らは独自の研究を始める必要がある。 ・
\\	《契約書》売り手は、製品の修理、交換を速やかに行うことを約束する。	変化形 
\\	【他動】〔義務・責任・仕事・地位などを〕引き受ける、請け負う、負う企てる、始める、~に着手する、~に取り掛かる ・
\\	彼らは独自の研究を始める必要がある。保証{ほしょう}する、約束{やくそく}する、~と断言する ・
\\	《契約書》売り手は、製品の修理、交換を速やかに行うことを約束する。レベル5、発音
\\	、カナアンダテイク、変化《動》
\\	、分節
\\	お邪魔かな? ・
\\	お話の途中ですが、質問してもいいですか? ・
\\	ちょっと割り込んでもいいですか。/失礼、ちょっといいですか?/一瞬だけよろしいですか?/少しだけお時間いただけますか?	変化形 
\\	【自動】中断{ちゅうだん}する、遮断{しゃだん}する◆一時的に〔話の〕邪魔{じゃま}をする、〔話に〕割って入る ・
\\	お邪魔かな?【他動】中断{ちゅうだん}する、遮断{しゃだん}する◆一時的に割り込む、妨げる、妨害{ぼうがい}する、邪魔{じゃま}する、〔視界{しかい}などを〕遮る、水を差す(人)の言葉{ことば}を遮る、~の話の腰を折る、~に口を挟む、~に横やりを入れる ・
\\	お話の途中ですが、質問してもいいですか? ・
\\	ちょっと割り込んでもいいですか。/失礼、ちょっといいですか?/一瞬だけよろしいですか?/少しだけお時間いただけますか?【名】中断{ちゅうだん}、遮断{しゃだん}◆一時的な《コ》インタラプト、割り込み隔たりレベル3、発音《名》
\\	《動》
\\	、カナインタラプト、変化《動》
\\	、分節
\\	勝手に首を突っ込まないでくれ。	変化形 
\\	【自動】邪魔{じゃま}をする、妨げる、遅らせる〔他人{たにん}のことに〕干渉{かんしょう}する、余計{よけい}な首を突っ込む ・
\\	勝手に首を突っ込まないでくれ。《スポーツ》インターフェアをする、〔相手選手{あいて せんしゅ}のプレーを〕故意{こい}に妨害{ぼうがい}する《電気》〔信号{しんごう}に〕干渉{かんしょう}する〔馬が走行中{そうこう ちゅう}にひづめで〕別の脚を蹴{け}るレベル4、発音
\\	、カナインターフェア、インタフィア、変化《動》
\\	、分節
\\	旅行日程の都合で2月14日にロスへ行くことはできません。	変化形 
\\	【他動】〔~が起きるのを〕防ぐ、止める、阻む ・
\\	旅行日程の都合で2月14日にロスへ行くことはできません。〔人が~するのを〕止める、阻む、抑えるレベル2、発音
\\	、カナプリベント、変化《動》
\\	、分節
\\	君にもいずれ定年が来るんだよ。	変化形 
\\	【自動】引き下がる、立ち去る身を引く、退く、棄権{きけん}する、後退{こうたい}[撤退{てったい}・退場{たいじょう}・退去{たいきょ}・退却{たいきゃく}・隠退{いんたい}・退職{たいしょく}・退役{たいえき}・隠居{いんきょ}]する ・
\\	君にもいずれ定年が来るんだよ。床につく、寝る◆【同】
\\	《野球》〔打者{だしゃ}が〕アウトになる【他動】~を引退{いんたい}させる、退職{たいしょく}させる、辞めさせる、退役{たいえき}させる、退去{たいきょ}させる、退却{たいきゃく}させる~を引っ込める、隠す、引き戻す、回収{かいしゅう}する、償還{しょうかん}する〔打者{だしゃ}を〕アウトにする【名】リタイア、引き下がること、後退{こうたい}、撤退{てったい}、退場{たいじょう}、退去{たいきょ}、退却{たいきゃく}、隠退{いんたい}、退職{たいしょく}、退役{たいえき}、隠居{いんきょ}ルティレ◆バレエのステップレベル3、発音
\\	、カナリタイア、変化《動》
\\	、分節
\\	もう酒はやめた。◆飲み過ぎた後で	変化形 
\\	【自他動】〈話〉〔仕事・活動などを完了せずに〕やめる、中止する〈主に米〉〔悪習・迷惑な行為などを〕やめる ・
\\	もう酒はやめた。◆飲み過ぎた後で〈英〉〔借家を〕立ち退く〔場所を〕去る、立ち退く〔コンピュータープログラムなどの使用を〕終了するレベル4、発音
\\	、カナクイット、クウィット、変化《動》
\\	または
\\	これで授業は終わりです。/授業はここまで。 ・
\\	では、今日の授業はここまで。 ・
\\	企業は、
\\	感染者やエイズ患者を解雇してはいけない。 ・
\\	そんな考えは忘れてしまえ。/そんなアイデアを考えるのはやめた方がいい。 ・
\\	彼女の声明は英国国防相に「感情的でナンセンス」とはねつけられた。 ・
\\	無担保の負債は破産時に免責される。
\\	変化形 
\\	"【自動】解散{かいさん}する【他動】解散{かいさん}させる、解雇{かいこ}する、免職{めんしょく}する、解任{かいにん}する、追放{ついほう}する ・
\\	これで授業は終わりです。/授業はここまで。 ・
\\	では、今日の授業はここまで。 ・
\\	企業は、
\\	感染者やエイズ患者を解雇してはいけない。退ける、片付けてしまう、駄目にする ・
\\	そんな考えは忘れてしまえ。/そんなアイデアを考えるのはやめた方がいい。〔要求などを〕却下{きゃっか}する、棄却{ききゃく}する、はねつける ・
\\	彼女の声明は英国国防相に「感情的でナンセンス」とはねつけられた。免責{めんせき}する ・
\\	無担保の負債は破産時に免責される。《法律》免訴{めんそ}するレベル3、発音
\\	、カナディスミス、デスミス、変化《動》
\\	、分節
\\	私が出会った異文化の中には、何百万年にも遡{さかのぼ}る歴史を持つものもある。	変化形 
\\	【他動】〔偶然{ぐうぜん}・思いがけなく〕出合う{であう}、出くわす◆【直訳】~にぶつかって
\\	中へ入る
\\	◆【類】
\\	私が出会った異文化の中には、何百万年にも遡{さかのぼ}る歴史を持つものもある。〔問題{もんだい}・不運{ふうん}・危険{きけん}・反対{はんたい}などに〕遭う、出会{であ}う、遭遇{そうぐう}する〔困難{こんなん}・現実{げんじつ}などに〕直面{ちょくめん}する〔災難{さいなん}が~に〕降り懸かる〔敵に〕遭遇{そうぐう}する~と交戦{こうせん}する、対決{たいけつ}する、争う、戦う、衝突{しょうとつ}する【名】〔思いがけない〕出会い{であい}、巡り合い、遭遇{そうぐう}、接触{せっしょく}遭遇戦{そうぐうせん}、衝突{しょうとつ}、対決{たいけつ}レベル4、発音
\\	、カナインカウンター、エンカウンター、変化《動》
\\	、分節
\\	日本には伝統的に、デリケートな問題に正面からぶつかることを嫌う気風がある。 ・
\\	取調官は被告に証拠を突きつけた。	変化形 
\\	【他動】〔問題などに〕直面{ちょくめん}する◆【類】
\\	日本には伝統的に、デリケートな問題に正面からぶつかることを嫌う気風がある。〔~を…と〕対決させる、〔~に…を〕突きつける ・
\\	取調官は被告に証拠を突きつけた。~を突き合わせる、対比{たいひ}する、比較{ひかく}する困難{こんなん}などが(人)に持ち上がるレベル5、発音
\\	、カナコンフラント、コンフロント、変化《動》
\\	、分節
\\	私のおごりです。 ・
\\	「どうしてるの?」「まあまあだ」 ・
\\	うまくやってる?、調子はどう? ・
\\	《不満そうに》まるで子ども扱いね。 ・
\\	《誘う》今晩タ食をおごろうか? ・
\\	夕食おごるよ。 ・
\\	勘定は私が持ちます。/私がごちそうします。/おごるよ。 ・
\\	治療は受けましたか? ・
\\	早期に発見されたがんは最も治療しやすい[治りやすい]。 ・
\\	ここは私がおごるよ。/勘定は私が持ちます。◆【同】
\\	特別な楽しみが待ってますよ。
\\	変化形 
\\	"【自動】取り扱う、扱う、処理{しょり}するおごる、ごちそうする ・
\\	私のおごりです。【他動】扱う、取り扱う、処理{しょり}する、待遇{たいぐう}する、あしらう、論じる ・
\\	「どうしてるの?」「まあまあだ」 ・
\\	うまくやってる?、調子はどう? ・
\\	《不満そうに》まるで子ども扱いね。〔~を…と〕受け止める、とらえる〔食事を〕おごる、ごちそうする ・
\\	《誘う》今晩タ食をおごろうか? ・
\\	夕食おごるよ。 ・
\\	勘定は私が持ちます。/私がごちそうします。/おごるよ。〔病気・怪我などを〕治療{ちりょう}する、手当{てあて}する、治す ・
\\	治療は受けましたか? ・
\\	早期に発見されたがんは最も治療しやすい[治りやすい]。【名】もてなし、おごり、ごちそう、特別{とくべつ}な楽しみ ・
\\	ここは私がおごるよ。/勘定は私が持ちます。◆【同】
\\	特別な楽しみが待ってますよ。おやつ◆通例
\\	レベル3、発音
\\	、カナトゥリート、変化《動》
\\	変化形 
\\	【自動】《アメフト》タックルする【他動】〔問題などに〕タックルする、取り組む、立ち向かう、対応{たいおう}[対処{たいしょ}]する《アメフト》〔相手{あいて}の選手{せんしゅ}に〕タックルする〔馬などに〕馬具{ばぐ}をつける【名】〔釣りや登山{とざん}の〕用具{ようぐ}、道具{どうぐ}〔滑車{かっしゃ}とロープから成る〕滑車装置{かっしゃ そうち}《海事》テークル◆ブロック(滑車)とロープからなら索具装置。《アメフト》〔相手{あいて}を倒す〕タックル《アメフト》〔ポジションの〕タックル〈俗〉ペニスレベル6、発音
\\	、カナタックル、変化《動》
\\	、分節
\\	あなたはこの問題にどのように対処していますか? ・
\\	どうやって暮らしてるの?/どうしてるの?◆出会った時のあいさつとして用いられることもある。	変化形 
\\	【1自動】うまく処理{しょり}する、うまく対処{たいしょ}する、対抗{たいこう}する◆【同】
\\	あなたはこの問題にどのように対処していますか? ・
\\	どうやって暮らしてるの?/どうしてるの?◆出会った時のあいさつとして用いられることもある。【2他動】《教会》~にコープを着せる【2名】《教会》コープ◆司祭が儀式の時にアルバ
\\	やサープリス
\\	の上に着る長い袖なしの外套。外套{がいとう}、マント【3他動】〔塀に〕笠石{かさいし}を載せる〔二つの繰形{くりがた}を〕繰形継ぎにする【3名】〔塀の〕笠石{かさいし}◆【同】
\\	レベル5、発音
\\	、カナコープ、コウプ、変化《動》
\\	夏休みは研究にあてるつもりだ。	変化形 
\\	【他動】〔人の時間{じかん}や注意{ちゅうい}などを〕充てる、向ける ・
\\	夏休みは研究にあてるつもりだ。〔自分自身{じぶん じしん}を〕専念{せんねん}させる、打ち込ませる〔神聖{しんせい}なものに〕~をささげる[奉納{ほうのう}する]レベル5、発音
\\	、カナデボート、デボウト、ディボウト、変化《動》
\\	、分節
\\	変化形 
\\	【他動】~をささげる、献身{けんしん}する専念{せんねん}する、打ち込む◆【用法】
\\	(記念碑{きねんひ})の除幕式{じょまくしき}をする、(記念館{きねんかん}を)開所{かいしょ}する~を特定用途{とくてい ようと}のために取っておく~を奉る、奉納{ほうのう}するレベル6、発音
\\	、カナデジケート、デディケイト、変化《動》
\\	、分節
\\	私は二つの目的を成し遂げた。いい仕事といい友人を得ることだ。	変化形 
\\	【自動】一定{いってい}の標準{ひょうじゅん}[基準{きじゅん}]に達する、成功{せいこう}を収める【他動】〔目的{もくてき}のものを〕獲得{かくとく}する、得る〔勝利{しょうり}・名声{めいせい}を〕勝ち取る〔努力{どりょく}して目的{もくてき}・目標{もくひょう}などを〕成し遂げる、達成{たっせい}する、成就{じょうじゅ}する、やり遂げる、実現{じつげん}する ・
\\	私は二つの目的を成し遂げた。いい仕事といい友人を得ることだ。~に成功{せいこう}する〔功績{こうせき}を〕立てる〔幸福{こうふく}・平和{へいわ}などを〕もたらす〈俗〉〔女を〕物にする、落とすレベル3、発音
\\	、カナアチーブ、変化《動》
\\	、分節
\\	変化形 
\\	【他動】〔使命{しめい}・目標{もくひょう}・計画{けいかく}・仕事{しごと}・願望{がんぼう}などを〕成し遂げる、遂行{すいこう}する、果たす、完遂{かんすい}する、成就{じょうじゅ}する、達成{たっせい}する、遂げる、仕上げる、完成{かんせい}させる〔勝利{しょうり}・栄誉{えいよ}・名声{めいせい}などを〕勝ち取る、得る〔功績{こうせき}・手柄{てがら}を〕立てる〔完成{かんせい}・熟達{じゅくたつ}の域に〕達する~に熟達{じゅくたつ}している、堪能{たんのう}である◆【用法】
\\	〔年月を〕経る、〔長い時間{じかん}を〕過ごし終える〔年齢{ねんれい}に〕達する〔道程{どうてい}を〕走破{そうは}する、~の終着点{しゅうちゃく てん}に達するレベル4、発音
\\	、カナアカムプリッシュ、アカンプリッシュ、アコンプリッシ、アコンプリッシュ、変化《動》
\\	、分節
\\	彼は果たすべき使命があると信じている。 ・
\\	使命を全うするためにここに来たんだ。 ・
\\	私は契約上の自分の役割を果たした。 ・
\\	この委員会は本来の機能を果たしていない。 ・
\\	ひそかな夢を抱きながらそれをかなえないまま人生を送るなんてなんてとても悲しいことだ。 ・
\\	それが、あなたが完全に満足できる唯一の方法だと思う。	変化形 
\\	【他動】〔約束・契約・計画などを〕実行{じっこう}する、遂行{すいこう}する、履行{りこう}する、実現{じつげん}させる〔目的などを〕成就{じょうじゅ}する〔義務・命令・使命などを〕果たす、全うする ・
\\	彼は果たすべき使命があると信じている。 ・
\\	使命を全うするためにここに来たんだ。 ・
\\	私は契約上の自分の役割を果たした。 ・
\\	この委員会は本来の機能を果たしていない。〔希望・夢などを〕かなえる ・
\\	ひそかな夢を抱きながらそれをかなえないまま人生を送るなんてなんてとても悲しいことだ。〔条件・要求などを〕満たす、満足させる ・
\\	それが、あなたが完全に満足できる唯一の方法だと思う。〔契約などの期間を〕満了する、終えるレベル5、発音
\\	、カナフルフィル、変化《動》
\\	、分節
\\	指紋を見つけようとその車を細部まで調べた。 ・
\\	彼の研究論文を分析しなければならない。 ・
\\	このテストで読解力を検査します。 ・
\\	証人を尋問する。	変化形 
\\	【他動】〔詳細{しょうさい}に〕~を観察{かんさつ}する[調べる] ・
\\	指紋を見つけようとその車を細部まで調べた。~を分析{ぶんせき}[研究{けんきゅう}]する ・
\\	彼の研究論文を分析しなければならない。〔患者{かんじゃ}の健康状態{けんこう じょうたい}を〕検査{けんさ}[診察{しんさつ}]する〔知識{ちしき}や技術{ぎじゅつ}などを〕試験{しけん}[検査{けんさ}]する ・
\\	このテストで読解力を検査します。《法律》〔証人{しょうにん}を〕尋問{じんもん}する ・
\\	証人を尋問する。レベル2、発音
\\	、カナイグザミン、エグザミン、変化《動》
\\	、分節
\\	当初の世論調査では、国民の多くはボブ・スミス氏の当選を予想していなかった。	変化形 
\\	【自動】土地{とち}の測量{そくりょう}を行う【他動】〔景色{けしき}などを〕見渡す{みわたす}、見晴らす{みはらす}~を概観{がいかん}[概説{がいせつ}]する〔詳細{しょうさい}に〕~を調査{ちょうさ}[検査{けんさ}]する〔土地{とち}を〕測量{そくりょう}する〈英〉〔建物{たてもの}を〕測量{そくりょう}する、査定{さてい}する〔人に〕世論調査{せろん ちょうさ}を行う【名】概観{がいかん}、概説{がいせつ}〔状況{じょうきょう}などについての詳しい〕調査{ちょうさ}、検査{けんさ}〔土地{とち}の〕測量{そくりょう}土地測量{とち そくりょう}図[報告書{ほうこくしょ}]測量地{そくりょう ち}世論調査{せろん ちょうさ} ・
\\	当初の世論調査では、国民の多くはボブ・スミス氏の当選を予想していなかった。レベル4、発音《名》
\\	《動》
\\	、カナサーベイ、変化《動》
\\	、分節
\\	本製品に関しては、別にパンフレットをご用意しております。詳しくはそちらをご覧ください。 ・
\\	《マニュアル》第6章のエラーコード表を参照してください。 ・
\\	その場合は直ちに医師の診療を受けてください。◆薬の説明書 ・
\\	これらの薬の処方については医師に相談すべきです。	変化形 
\\	【自動】相談{そうだん}する、〔専門家{せんもんか}に〕助言{じょげん}を求める〔~の〕顧問{こもん}を務める、コンサルタントをする【他動】〔情報{じょうほう}やアドバイスを得るために本・辞書{じしょ}・参考書{さんこうしょ}・文献{ぶんけん}を〕調べる、閲覧{えつらん}する、参考{さんこう}にする ・
\\	本製品に関しては、別にパンフレットをご用意しております。詳しくはそちらをご覧ください。 ・
\\	《マニュアル》第6章のエラーコード表を参照してください。〔識者{しきしゃ}・専門家{せんもんか}などに〕尋ねる、意見{いけん}を聞く、相談{そうだん}する、助言{じょげん}を求める、指導{しどう}を求める ・
\\	その場合は直ちに医師の診療を受けてください。◆薬の説明書 ・
\\	これらの薬の処方については医師に相談すべきです。〔医者{いしゃ}〕にかかる、〔医者{いしゃ}の診察{しんさつ}を〕受けるレベル5、発音
\\	、カナコンサルト、変化《動》
\\	、分節
\\	この仕事を完了するには少なくとも3日かかると私たちは見積もっています。 ・
\\	アフリカへの緊急包括的[総合(的)]援助計画は50億ドルになると推定されます。 ・
\\	残念ながら御社の見積もりでは高過ぎます。 ・
\\	残念ながら、御社のお見積もりは私たちの予想よりも高いです。 ・
\\	昨日お送りした見積書に間違いがありましたことを深くおわび致します。ご依頼の商品につき修正した見積書を添付致しましたのでご覧ください。	変化形 
\\	【自動】評価{ひょうか}する、見積{みつ}もる、見積書{みつもりしょ}を作成{さくせい}する【他動】〔程度や数量などを大まかに〕見積{みつ}もる、推定{すいてい}する ・
\\	この仕事を完了するには少なくとも3日かかると私たちは見積もっています。 ・
\\	アフリカへの緊急包括的[総合(的)]援助計画は50億ドルになると推定されます。〔購入{こうにゅう}するものなどの〕見積書{みつもりしょ}を作成{さくせい}[提出{ていしゅつ}]させる〔人や物事{ものごと}を〕評価{ひょうか}する、判断{はんだん}する【名】〔価格{かかく}の〕見積{みつ}もり ・
\\	残念ながら御社の見積もりでは高過ぎます。 ・
\\	残念ながら、御社のお見積もりは私たちの予想よりも高いです。〔数や量などの〕推量{すいりょう}、推定(値){すいてい(ち)}、目算{もくさん}〔価格{かかく}の〕見積書{みつもりしょ}、概算書{がいさん しょ}◆曖昧{あいまい}な推測による見積もりは
\\	という。 ・
\\	昨日お送りした見積書に間違いがありましたことを深くおわび致します。ご依頼の商品につき修正した見積書を添付致しましたのでご覧ください。〔印象{いんしょう}などに基づく〕判断{はんだん}、評価{ひょうか}レベル4、発音《名》
\\	《動》
\\	、カナエスチメイト、エスチメット、エスティメイト、エスティメット、変化《動》
\\	、分節
\\	自分自身の基準で自分を評価しなさい。 ・
\\	党首はその選挙結果を自党の大勝利と見なした。 ・
\\	私の家は8万ドルと査定された。 ・
\\	変数が一つの数式の値を求めるのは簡単である。	変化形 
\\	【他動】〔~の重要性{じゅうよう せい}や品質{ひんしつ}などを〕評価{ひょうか}[審査{しんさ}・判断{はんだん}・診断{しんだん}]する ・
\\	自分自身の基準で自分を評価しなさい。 ・
\\	党首はその選挙結果を自党の大勝利と見なした。~を値付け[値踏み・査定{さてい}]する、~を見積{みつ}もる ・
\\	私の家は8万ドルと査定された。《数学》~の値を求める、~を評価{ひょうか}する ・
\\	変数が一つの数式の値を求めるのは簡単である。レベル5、発音
\\	、変化《動》
\\	、分節
\\	変化形 
\\	【自動】〔大学{だいがく}で〕コース[科目{かもく}]選択{せんたく}をする【他動】〔選挙{せんきょ}で人を〕選ぶ、選出{せんしゅつ}する〔大学{だいがく}でコースなどを〕選ぶ、選択{せんたく}する〔好んで~することを〕決める、選ぶ〔救いのために神が人を〕選ぶ【名】選ばれた人
\\	選ばれた人たち、エリート集団{しゅうだん}
\\	〔神が選んだ〕選民{せんみん}◆【同】
\\	【形】選ばれた、えり抜かれた〔就任{しゅうにん}していないが〕選挙{せんきょ}で選ばれた◆【用法】
\\	のように後置する。〔救いのために神に〕選ばれたレベル3、発音
\\	、カナイレクト、エレクト、変化《動》
\\	、分節
\\	議会は信仰の自由を保障した新しい法律を可決しました。 ・
\\	彼はハワイを世界で最も人気のある観光地であると認めた。 ・
\\	彼は投票で州の最優秀コーチに選ばれた。 ・
\\	音楽によって頭脳が一層明せきになる人もいれば鈍重になる人もいるという私の立場を明らかにした。 ・25 
\\	賛成25票に対し反対14票。	変化形 
\\	【自動】投票{とうひょう}する、投票{とうひょう}で決める【他動】~に投票{とうひょう}する〔投票{とうひょう}で意見{いけん}や選択{せんたく}を〕表明{ひょうめい}する〔投票{とうひょう}で〕~を決定{けってい}する、~を実現{じつげん}する ・
\\	議会は信仰の自由を保障した新しい法律を可決しました。~に従って投票{とうひょう}する◆【参考】
\\	~が…であると認める、投票{とうひょう}で(人)を~に選ぶ ・
\\	彼はハワイを世界で最も人気のある観光地であると認めた。 ・
\\	彼は投票で州の最優秀コーチに選ばれた。〈話〉〔意見{いけん}や選択{せんたく}を〕表明{ひょうめい}する、提案{ていあん}する ・
\\	音楽によって頭脳が一層明せきになる人もいれば鈍重になる人もいるという私の立場を明らかにした。【名】票、投票{とうひょう} ・25 
\\	賛成25票に対し反対14票。〔投票{とうひょう}による〕公式{こうしき}な賛否{さんぴ}の表明{ひょうめい}(方法{ほうほう})、表決{ひょうけつ}(方法{ほうほう})〔集合的{しゅうごう てき}に〕投票総数{とうひょう そうすう}〔共通{きょうつう}の特徴{とくちょう}を持つ〕投票集団{とうひょう しゅうだん}投票結果{とうひょう けっか}投票権{とうひょう けん}、参政権{さんせいけん}投票用紙{とうひょう ようし}レベル2、発音
\\	、カナボウト、変化《動》
\\	事実と意見の違いが分かりますか。 ・
\\	その車のナンバープレートの番号は分からなかった。 ・
\\	この特徴によりつるが他の森の植物と大きく異なっている。 ・
\\	彼はその勇猛さによって有名になった。	変化形 
\\	【自動】見分ける{みわける}、区別{くべつ}する、識別{しきべつ}する【他動】〔異なるものの〕違いが分かる、〔異なるものを〕区別{くべつ}する ・
\\	事実と意見の違いが分かりますか。~に気付く、~を認める ・
\\	その車のナンバープレートの番号は分からなかった。~を目立たせる、~を際立たせる{きわだたせる} ・
\\	この特徴によりつるが他の森の植物と大きく異なっている。〔業績{ぎょうせき}などにより再帰代名詞{さいき だいめいし}を〕有名{ゆうめい}[著名{ちょめい}]にする、優れていると認めさせる ・
\\	彼はその勇猛さによって有名になった。レベル4、発音
\\	、カナディスティングィッシュ、ディスティングウィッシュ、デスティングィッシ、デスティングウィッシ、デステングイッシ、デステングウィッシュ、変化《動》
\\	、分節
\\	捜査官たちは死体の身元確認を急いだ。 ・
\\	約150万の異なる生物が生物学者によって確認されている。 ・
\\	私は人生の目標を見つけて達成しました。 ・
\\	私は遺失物取扱所で私のコートを見つけようとした。	変化形 
\\	【自動】同じ心境{しんきょう}になる、意気投合{いき とうごう}する【他動】〔本物{ほんもの}・本人{ほんにん}であることを〕確認{かくにん}[識別{しきべつ}]する〔正体・身元などを〕確認{かくにん}[特定{とくてい}]する ・
\\	捜査官たちは死体の身元確認を急いだ。 ・
\\	約150万の異なる生物が生物学者によって確認されている。 ・
\\	私は人生の目標を見つけて達成しました。 ・
\\	私は遺失物取扱所で私のコートを見つけようとした。~を同一{どういつ}に扱う~を同一視{どういつし}する同定{どうてい}する◆動植物の種類を特定することや、物質の組成・機能を特定することなど。レベル3、発音
\\	、カナアイデンティファイ、アイデンテファイ、変化《動》
\\	、分節
\\	これらの言葉をもっと簡潔に定義していただけませんでしょうか? ・
\\	製品については、本契約書の付属書
\\	に記載し、定めるものとする。◆契約書	変化形 
\\	【他動】~を定義{ていぎ}する、~の意味{いみ}を明確{めいかく}[明らか]にする、~を特徴{とくちょう}づける ・
\\	これらの言葉をもっと簡潔に定義していただけませんでしょうか? ・
\\	製品については、本契約書の付属書
\\	に記載し、定めるものとする。◆契約書〔内容{ないよう}を〕明示{めいじ}する〔責任{せきにん}・立場{たちば}・役割{やくわり}などを〕明確{めいかく}にする〔範囲{はんい}を〕決める、規定{きてい}する~の輪郭{りんかく}を示すレベル4、発音
\\	、カナデファイン、ディファイン、変化《動》
\\	、分節
\\	子どもは新しい物事に大人よりも早く順応する傾向がある。	変化形 
\\	【自動】〔異なる状況などに合わせて〕順応{じゅんのう}[適応{てきおう}・変化{へんか}]する◆【用法】
\\	と違って、結果が出るまでにある程度の時間がかかる。◆【用法】
\\	名詞 ・
\\	子どもは新しい物事に大人よりも早く順応する傾向がある。【他動】〔異なる状況{じょうきょう}などに合わせて~を〕順応{じゅんのう}[適応{てきおう}・変化{へんか}]させる〔小説{しょうせつ}や戯曲{ぎきょく}などを映画化{えいがか}などのために〕脚色{きゃくしょく}[翻案{ほんあん}]するレベル4、発音
\\	、カナアダプト、変化《動》
\\	、分節
\\	ゆっくり始めて身体を慣らしたほうが良い。	変化形 
\\	【自動】〔~に〕順応{じゅんのう}する、適応{てきおう}する、慣れる ・
\\	ゆっくり始めて身体を慣らしたほうが良い。【他動】〔目的{もくてき}や状況{じょうきょう}に合わせて〕~を少し変える、~を調整{ちょうせい}[調節{ちょうせつ}・整備{せいび}]する、~を適合{てきごう}[適応{てきおう}・合致{がっち}]させる、~を整える、~を加減{かげん}する〔入れ歯などを〕ぴったりはめる〔意見{いけん}の食い違い・論争{ろんそう}・争い・対立{たいりつ}を〕調整{ちょうせい}する、解決{かいけつ}する、調停{ちょうてい}する◆【類】
\\	〔間違いなどを〕訂正{ていせい}する〔差異{さい}を〕調節{ちょうせつ}する《保》〔損害賠償請求{そんがい ばいしょう せいきゅう}に対する支払額{しはらいがく}を〕決める、清算{せいさん}するレベル4、発音
\\	、カナアジャスト、変化《動》
\\	、分節
\\	そのシステムは会社の要求に良く適合していた。 ・
\\	私に一番似合う色はどれだと思いますか。 ・
\\	それは一般向けの味です。 ・
\\	一つの料金プランで誰でも満足させることはできない。 ・
\\	どの色が一番私に似合うか教えてください。 ・
\\	そのドレス、よくお似合いですよ。 ・
\\	彼は短い髪の方が似合うと思う。 ・
\\	手段は目的に適合させなければならない。 ・
\\	母親は彼に戦闘服を着せた。 ・
\\	ある男性は女性上司に対して起こしたセクハラ訴訟で、100万ドルの損害賠償を勝ち取った。 ・
\\	プログラミングは私に向いていない。	変化形 
\\	【自動】適合{てきごう}する、満足{まんぞく}させる ・
\\	そのシステムは会社の要求に良く適合していた。似合う{にあう}、ふさわしい ・
\\	私に一番似合う色はどれだと思いますか。【他動】〔要求{ようきゅう}などに〕合う、適合{てきごう}する ・
\\	それは一般向けの味です。(人)を満足{まんぞく}させる、(人)に気に入る ・
\\	一つの料金プランで誰でも満足させることはできない。〔服装・髪型などが〕(人)に似合う{にあう}、(人)にふさわしい ・
\\	どの色が一番私に似合うか教えてください。 ・
\\	そのドレス、よくお似合いですよ。 ・
\\	彼は短い髪の方が似合うと思う。〔要求{ようきゅう}に合うように〕~を適合{てきごう}させる、~を作り変える ・
\\	手段は目的に適合させなければならない。(人)に服を着せる ・
\\	母親は彼に戦闘服を着せた。【名】スーツ、〔同じ素材の〕一組の衣服◆通例上着とズボンまたはスカートで構成される。《法律》控訴{こうそ}、訴訟{そしょう} ・
\\	ある男性は女性上司に対して起こしたセクハラ訴訟で、100万ドルの損害賠償を勝ち取った。〔特定{とくてい}の目的{もくてき}に合った〕服、スーツ◆【参考】
\\	一組{ひとくみ}、ひとそろい〈俗〉〔企業{きぎょう}の〕幹部{かんぶ}、重役{じゅうやく}《トランプ》スーツ◆クラブ、ダイヤ、ハート、スペードのいずれか。〔臣下{しんか}の封建領主{ほうけん りょうしゅ}への〕参上{さんじょう}〈古〉〔女性{じょせい}に対する〕求婚{きゅうこん}〔その人に〕適合するもの ・
\\	プログラミングは私に向いていない。レベル1、発音
\\	、カナスーツ、スート、変化《動》
\\	このクーポン[割引券]は、以前に購入された商品にはご利用いただけません。 ・
\\	このルールはあらゆる場合に適用できる。 ・
\\	この状況は私には当てはまらない。 ・
\\	毎朝毎晩お使いください。◆薬などの使用法の指示 ・
\\	1日2回(朝晩)適用してください。◆塗り薬などの説明書 ・
\\	複数回答可。◆アンケートなどで使われる表現。 ・
\\	私は彼にこの仕事に応募するように勧めました。 ・
\\	クレジットカードの申し込みは、どうすればいいですか? ・
\\	すべての注文には10ドルの手数料がかかります。 ・
\\	看護師は、彼女のけがをした腕にタオルを当てました。 ・
\\	上まつげと下まつげにマスカラを塗ります。 ・
\\	彼はバックブレーカーを食らわせようとしたが、うまくいかなかった。	変化形 
\\	【自動】当てはまる、妥当{だとう}する、適用{てきよう}される、適合{てきごう}する ・
\\	このクーポン[割引券]は、以前に購入された商品にはご利用いただけません。 ・
\\	このルールはあらゆる場合に適用できる。 ・
\\	この状況は私には当てはまらない。 ・
\\	毎朝毎晩お使いください。◆薬などの使用法の指示 ・
\\	1日2回(朝晩)適用してください。◆塗り薬などの説明書 ・
\\	複数回答可。◆アンケートなどで使われる表現。申し込む、申し込みを行う、応募{おうぼ}する、申請{しんせい}する、出願{しゅつがん}する、志願{しがん}する、問い合わせる、照会{しょうかい}する ・
\\	私は彼にこの仕事に応募するように勧めました。 ・
\\	クレジットカードの申し込みは、どうすればいいですか?専念{せんねん}する【他動】〔~を…に〕適用{てきよう}する、応用{おうよう}する、利用{りよう}する、生かす、当てはめる ・
\\	すべての注文には10ドルの手数料がかかります。〔物を~に〕当てる ・
\\	看護師は、彼女のけがをした腕にタオルを当てました。〔~を…に〕塗る、塗布{とふ}する、貼り付ける、付ける ・
\\	上まつげと下まつげにマスカラを塗ります。〔資金{しきん}などを〕~に振り向ける〔装置{そうち}などを〕作動{さどう}させる、使用{しよう}する、働かせる〔熱・電圧{でんあつ}などを〕加える~に専念{せんねん}する◆【用法】
\\	〔心・精神・労力などを~に〕傾ける、注ぐ〔プロレスの技などを〕掛ける、食らわせる、決める ・
\\	彼はバックブレーカーを食らわせようとしたが、うまくいかなかった。レベル2、発音
\\	、カナアプライ、変化《動》
\\	、分節
\\	彼女はいくつかの指針を採用しました。	変化形 
\\	【他動】〔複数{ふくすう}の選択肢{せんたくし}から〕~を選ぶ[採用{さいよう}する・導入{どうにゅう}]する ・
\\	彼女はいくつかの指針を採用しました。〔議案{ぎあん}・報告書{ほうこくしょ}などを〕採択{さいたく}する、可決{かけつ}する、承認{しょうにん}する~を養子{ようし}[養女{ようじょ}]にする〔外来語{がいらいご}として〕借用{しゃくよう}する〔教科書{きょうかしょ}などを〕選定{せんてい}する〔習慣{しゅうかん}・態度{たいど}などを〕身に付ける公認{こうにん}する、指名{しめい}する◆【類】
\\	レベル3、発音
\\	、カナアダプト、アドプト、変化《動》
\\	、分節
\\	あなたは私を誰か別の(人)と勘違いしています。 ・
\\	絵に描いたもちは食えない。 ・
\\	ちょっと混乱しちゃって。手伝ってくれる?	変化形 
\\	【他動】混同{こんどう}する ・
\\	あなたは私を誰か別の(人)と勘違いしています。 ・
\\	絵に描いたもちは食えない。〔人を〕混乱{こんらん}させる、困惑{こんわく}させる、戸惑わせる、まごつかせる、惑わせる、けむに巻く、おちょくる ・
\\	ちょっと混乱しちゃって。手伝ってくれる?レベル5、発音
\\	、カナコンフューズ、変化《動》
\\	、分節
\\	シンプルに考えなさい。	変化形 
\\	【他動】~を複雑{ふくざつ}にする、困難{こんなん}にする、込み入らせる、悪化{あっか}させる、面倒{めんどう}にする ・
\\	シンプルに考えなさい。【形】複雑{ふくざつ}な、込み入った、折り重ねられたレベル6、発音
\\	、カナカンプリケイト、コンプリケイト、変化《動》
\\	、分節
\\	彼があなたの気を悪くしなかったらいいのですが。 ・
\\	あなたを怒らせるつもりはありませんでした。 ・
\\	胃が弱いんです。 ・
\\	私は牛乳を飲むとおなかの調子が悪くなる。 ・
\\	それによって、私の予定が狂ってしまいます。 ・
\\	そんなに慌てないで。 ・
\\	何をくよくよしているんだ。 ・
\\	彼はおなかの調子が悪くて、こんなのを食べられる状態ではない。 ・
\\	胃の調子がおかしい。	変化形 
\\	【自動】ひっくり返る、転覆{てんぷく}する【他動】~をひっくり返す、転倒{てんとう}させる、転覆{てんぷく}させる〔精神的{せいしん てき}に人を〕動揺{どうよう}させる、苦しめる、困らせる ・
\\	彼があなたの気を悪くしなかったらいいのですが。 ・
\\	あなたを怒らせるつもりはありませんでした。〔消化器系{しょうかきけい}を壊して〕(人)を気持{きも}ち悪くさせる、(人)の体調{たいちょう}を悪くさせる ・
\\	胃が弱いんです。 ・
\\	私は牛乳を飲むとおなかの調子が悪くなる。〔機能{きのう}や秩序{ちつじょ}を〕乱す、壊す、駄目{だめ}にする ・
\\	それによって、私の予定が狂ってしまいます。〔スポーツなどで強い相手{あいて}を〕番狂わせで倒す[負かす]〔ボルトの端などを〕据え込み[アップセット]加工{かこう}する◆熱したボルトなどをたたいて短くし、径を太くすること。【名】ひっくり返ること、転倒{てんとう}、転覆{てんぷく}混乱{こんらん}、騒動{そうどう}、興奮状態{こうふん じょうたい}気の動転{どうてん}、動揺{どうよう}、イライラ〔スポーツや選挙{せんきょ}などの〕意外{いがい}な[予期{よき}しない]結果{けっか}、番狂わせ{ばんくるわせ}据え込み[アップセット]工具{こうぐ}据え込み[アップセット]材【形】ひっくり返って、転覆{てんぷく}して、調子{ちょうし}が悪い混乱{こんらん}して、〔計画{けいかく}などが〕めちゃくちゃの、番狂わせ{ばんくるわせ}の、番狂わせで負けた動揺{どうよう}して、気が動転{どうてん}して、取り乱して、狼狽{ろうばい}して、うろたえて、まごまごして、どぎまぎして、冷静{れいせい}を失って、腹を立てて、憤慨{ふんがい}して、おかんむりで、ブスッとして、くよくよして ・
\\	そんなに慌てないで。 ・
\\	何をくよくよしているんだ。〔胃が〕不調{ふちょう}の ・
\\	彼はおなかの調子が悪くて、こんなのを食べられる状態ではない。 ・
\\	胃の調子がおかしい。レベル3、発音《名》
\\	《形・動》
\\	、カナアップセット、変化《動》
\\	、分節
\\	こんなに恥ずかしい思いをしたのは生まれて初めてだ。/人生最大の恥辱です。	変化形 
\\	【他動】〔人を〕狼狽{ろうばい}させる、まごつかせる、当惑{とうわく}させる ・
\\	こんなに恥ずかしい思いをしたのは生まれて初めてだ。/人生最大の恥辱です。〔人を〕困窮{こんきゅう}させる、金に困らせる◆【用法】受身形で用いられることが多い。~を邪魔{じゃま}する、~を妨げる~を複雑{ふくざつ}にする、~をこじらせるレベル5、発音
\\	、カナインバラス、エンバラス、変化《動》
\\	、分節
\\	私は語学は好きだが科学には苦戦している。 ・
\\	私たちは非常に低い利幅を何とかしようと苦労しています。 ・
\\	消防士たちは鎮火のため苦闘している。/消火作業は難航している。 ・
\\	日本では一部の子どもたちにとって受験戦争は就学前にすでに始まっている。	変化形 
\\	【自動】〔体を使って〕奮闘{ふんとう}する、力を振り絞る〔課題{かだい}や困難{こんなん}などに〕取り組む、悪戦苦闘{あくせん くとう}する ・
\\	私は語学は好きだが科学には苦戦している。 ・
\\	私たちは非常に低い利幅を何とかしようと苦労しています。 ・
\\	消防士たちは鎮火のため苦闘している。/消火作業は難航している。〔逃れようと〕もがく、あがく〔人と〕取っ組み合う、格闘{かくとう}する〔人が〕苦労{くろう}して[骨を折って]進む【名】〔困難{こんなん}を克服{こくふく}する〕骨折り{ほねおり}、奮闘{ふんとう}〔長期にわたる〕争い{あらそい}、闘争{とうそう} ・
\\	日本では一部の子どもたちにとって受験戦争は就学前にすでに始まっている。〔奮闘{ふんとう}が必要{ひつよう}な〕大変{たいへん}な課題{かだい}[困難{こんなん}]レベル2、発音
\\	、カナストゥラッグル、ストラグル、変化《動》
\\	、分節
\\	いいかげんにしろ。/ばかげたことを聞くな。 ・
\\	《契約書》上記の技術情報の収集および欠陥の分析に要する費用は、メーカーの負担とする。 ・
\\	もし
\\	から始まる製品コードの付いた製品をお持ちでしたら、お買い上げいただいた販売店にお返しください。 ・
\\	両方の書類に税務当局の正式な証印がしてあります。 ・
\\	それは私に耐えられないものだった。	変化形 
\\	【1他動】〔相場{そうば}を〕下げる【1名】クマ ・
\\	いいかげんにしろ。/ばかげたことを聞くな。弱気{よわき}、弱気筋{よわきすじ}、弱気{よわき}の売り方、下げ相場◆【語源】
\\	という諺から。「クマを捕まえる前に毛皮を売る」という行為と「有価証券の空売り」(持っていない有価証券の売り付けを行うこと=
\\	が似ている。◆【反】
\\	〔クマのような〕乱暴者{らんぼう もの}、不作法者{ぶさほう もの}〈俗〉骨の折れる[非常{ひじょう}に困難{こんなん}な]仕事{しごと}[問題{もんだい}]、厄介{やっかい}なこと【1形】〔相場{そうば}が〕弱気{よわき}な【2自動】実をつける、子を産む向かう、進む〔~に〕もたれる位置{いち}する【2他動】〔重さに〕耐える、〔物を〕支える、〔重荷を〕担う〔重荷{おもに}を〕負う、〔義務{ぎむ}を〕負う、〔責任{せきにん}を〕持つ、〔費用{ひよう}を〕負担{ふたん}する ・
\\	《契約書》上記の技術情報の収集および欠陥の分析に要する費用は、メーカーの負担とする。〔重さに耐えて〕~を運ぶ[持って行く]~を身につける、持つ、有する ・
\\	もし
\\	から始まる製品コードの付いた製品をお持ちでしたら、お買い上げいただいた販売店にお返しください。 ・
\\	両方の書類に税務当局の正式な証印がしてあります。~を辛抱{しんぼう}する、我慢{がまん}する、~に耐える ・
\\	それは私に耐えられないものだった。〔恨み・悪意{あくい}・愛情{あいじょう}などを〕抱く〔子を〕産む〔利益{りえき}や利子{りし}を〕生む〔植物が花・果実を〕生じる〔努力{どりょく}を〕実らせるレベル2、発音
\\	、カナベア、【同音】
\\	、変化《動》
\\	または
\\	われわれは試練に遭いそれに耐えた。 ・
\\	我慢にもほどがあるんだよ。	変化形 
\\	【自動】耐える、持ちこたえる、持続{じぞく}する ・
\\	われわれは試練に遭いそれに耐えた。【他動】~に耐え抜く、持ちこたえる、耐える、~を我慢{がまん}する ・
\\	我慢にもほどがあるんだよ。~を許す、認めるレベル4、発音
\\	、カナインデュア、エンデュア、変化《動》
\\	、分節
\\	どうやって克服したかお聞きしたいものです。 ・
\\	私は理性の力が恐怖心を制するところを皆さんに示してほしいと思います。	変化形 
\\	【他動】〔困難・障害などを〕克服{こくふく}する、乗り越える、打開{だかい}する ・
\\	どうやって克服したかお聞きしたいものです。 ・
\\	私は理性の力が恐怖心を制するところを皆さんに示してほしいと思います。〔相手を〕圧倒{あっとう}する、打ち勝つ、制覇{せいは}するレベル4、発音
\\	、カナオーバーカム、オウバーカム、変化《動》
\\	、分節
\\	変化形 
\\	【自動】競争{きょうそう}する、立ち向かう匹敵{ひってき}する、太刀打{たちう}ちできるレベル4、発音
\\	、カナコンピート、変化《動》
\\	、分節
\\	公共の場での喫煙は固く禁じられている。	変化形 
\\	【他動】〔法令{ほうれい}などで人が~するのを〕禁止{きんし}する、差し止める◆制度的または法令上の禁止を意味することが多い。 ・
\\	公共の場での喫煙は固く禁じられている。〔人が~するのを〕妨げる、邪魔{じゃま}をするレベル5、発音
\\	、カナプロウヒビット、プロヒビット、変化《動》
\\	、分節
\\	時間が許さない。 ・
\\	私は酒を禁じられている。 ・
\\	喫煙は固く禁じられています。 ・
\\	《就業規則》収賄は厳禁とする。	変化形 
\\	【自動】禁じる ・
\\	時間が許さない。【他動】~を禁じる、禁止{きんし}する、妨げる、差し止める、許さない ・
\\	私は酒を禁じられている。 ・
\\	喫煙は固く禁じられています。 ・
\\	《就業規則》収賄は厳禁とする。レベル5、発音
\\	、カナファビッド、フォービッド、変化《動》
\\	、分節
\\	中国はダライラマの写真の展示を法律で禁じた。 ・
\\	その本は、暴力的な内容であるため、一部の図書館では禁書とされている。 ・
\\	その大学は構内のすべての食堂で喫煙を禁じた。 ・
\\	不審な人物が近所をうろついているといううわさを聞いて、心配した母親たちは、子どもたちが近くの公園で遊ぶのを禁止しました。 ・
\\	あなたはアクセス禁止です。 ・
\\	きっと誰かが
\\	をアクセス禁止にしたんだよ。 ・
\\	政府はエイズに感染している外国人の入国禁止措置を解除すると発表しました。	変化形 
\\	《複》
\\	【1他動】〔法律などにより〕~を禁じる、~を禁止する、~ができないようにする ・
\\	中国はダライラマの写真の展示を法律で禁じた。 ・
\\	その本は、暴力的な内容であるため、一部の図書館では禁書とされている。 ・
\\	その大学は構内のすべての食堂で喫煙を禁じた。 ・
\\	不審な人物が近所をうろついているといううわさを聞いて、心配した母親たちは、子どもたちが近くの公園で遊ぶのを禁止しました。《イ》(人)の接続{せつぞく}[入室{にゅうしつ}]を禁止{きんし}する、~からのアクセスを禁止{きんし}する◆サーバやチャット・ルームなどで権限者が特定の人[アドレスから]の今後の接続を禁じる ・
\\	あなたはアクセス禁止です。 ・
\\	きっと誰かが
\\	をアクセス禁止にしたんだよ。【1名】〔法律などによる〕禁(止)令、禁制 ・
\\	政府はエイズに感染している外国人の入国禁止措置を解除すると発表しました。〔教会{きょうかい}からの〕破門{はもん}、追放{ついほう}〈古〉〔大衆{たいしゅう}からの〕非難{ひなん}、酷評{こくひょう}〈古〉〔人に対する〕のろい【2名】バン◆ルーマニアの貨幣単位。レオ
\\	の100分の1◆【複】
\\	レベル4、発音
\\	、カナバヌ、バン、変化《動》
\\	このエリアは一般の人は[関係者以外]立ち入り禁止です。	変化形 
\\	【他動】〔限度内に〕~を制限{せいげん}[限定{げんてい}]する ・
\\	このエリアは一般の人は[関係者以外]立ち入り禁止です。《特許》〔クレームを〕減縮するレベル5、発音
\\	、カナリストゥリクト、リストリクト、変化《動》
\\	、分節
\\	あなたがなぜ彼の肩を持つのか理解できません。	変化形 
\\	【自動】防御{ぼうぎょ}する《法律》弁護{べんご}する【他動】〔敵の攻撃などから〕~を守る[防御{ぼうぎょ}する]~を擁護{ようご}する、(人)の肩を持つ、~を主張{しゅちょう}する ・
\\	あなたがなぜ彼の肩を持つのか理解できません。《法律》~を弁護{べんご}する、~に抗弁{こうべん}するレベル2、発音
\\	、カナデフェンド、ディフェンド、変化《動》
\\	、分節
\\	気分を害したよ。	変化形 
\\	【自動】(人)の気分{きぶん}を害する、(人)を怒らせる法に背く、規律{きりつ}を破る【他動】(人)の気分{きぶん}[感情{かんじょう}]を害する、(人)を怒らせる ・
\\	気分を害したよ。〔人の感覚器官{かんかく きかん}を〕不快{ふかい}にさせる、耳[目]障りだと感じさせる〔法律{ほうりつ}などを〕破る、背くレベル3、発音
\\	、カナオフェンド、変化《動》
\\	、分節
\\	本契約のいずれかの条項に違反した時点で、本ライセンスは直ちに自動的に終了します。 ・
\\	警察は家宅侵入罪と家庭内暴力法違反で男性を逮捕しました。	変化形 
\\	【他動】〔法律・契約などに〕違反{いはん}する、〔法を〕犯す、〔規則などを〕破る ・
\\	本契約のいずれかの条項に違反した時点で、本ライセンスは直ちに自動的に終了します。 ・
\\	警察は家宅侵入罪と家庭内暴力法違反で男性を逮捕しました。〔領海を〕侵犯{しんぱん}する妨害{ぼうがい}する、侵害{しんがい}する強姦{ごうかん}する邪魔{じゃま}する、乱すレベル5、発音
\\	、カナバイオレイト、変化《動》
\\	、分節
\\	「このドア開けて行った?僕が鍵かけなかったっけ?」「なんてこと!マーク!私たち強盗に入られたのよ!」 ・
\\	私の家は過去10年間に3度強盗に入られた。
\\	変化形 
\\	"【自動】〔強盗が〕強盗{ごうとう}を働く【他動】《法律》(人)に強盗{ごうとう}を働く、(人)から強奪{ごうだつ}する ・
\\	「このドア開けて行った?僕が鍵かけなかったっけ?」「なんてこと!マーク!私たち強盗に入られたのよ!」〔店や銀行などに〕強盗{ごうとう}に入る、〔人の所有物を〕略奪{りゃくだつ}する ・
\\	私の家は過去10年間に3度強盗に入られた。〔商品などを〕盗む、奪う〔不当に人から〕奪う、取り上げるレベル3、発音
\\	、カナラッブ、ロッブ、変化《動》
\\	彼は第2級謀殺の罪に問われていた。 ・
\\	悪事は必ず露見する。/悪事は必ずばれるもの。 ・
\\	警察は高利貸し、麻薬取引、雇われ殺人の容疑で12人を逮捕しました。	変化形 
\\	【自動】殺人{さつじん}を犯す【他動】〔故意に人を〕殺す、殺害{さつがい}する〈話〉~を台無{だいな}しにする【名】殺人{さつじん}、謀殺{ぼうさつ} ・
\\	彼は第2級謀殺の罪に問われていた。 ・
\\	悪事は必ず露見する。/悪事は必ずばれるもの。 ・
\\	警察は高利貸し、麻薬取引、雇われ殺人の容疑で12人を逮捕しました。〈俗〉大変難{たいへん むずか}しいこと、耐え難いこと〔カラスなどの〕群れレベル2、発音
\\	、カナマーダー、変化《動》
\\	、分節
\\	彼女は非難はされたが告発を受けたわけではない。 ・
\\	どうしていつも私ばかり責めるの。	変化形 
\\	【他動】〔法律違反{ほうりつ いはん}・義務{ぎむ}の不履行{ふりこう}・約束違反{やくそく いはん}などの理由{りゆう}で〕(人)に責任{せきにん}を問う、(人)を責める[非難{ひなん}する・糾弾{きゅうだん}する・訴える・告訴{こくそ}する・告発{こくはつ}する]、(人)に言い掛かり[いちゃもん]をつける、(人)にあれこれ文句{もんく}を言う◆【参考】
\\	彼女は非難はされたが告発を受けたわけではない。 ・
\\	どうしていつも私ばかり責めるの。レベル4、発音
\\	、カナアキューズ、変化《動》
\\	、分節
\\	重罪に当たる窃盗と横領の二つの訴因で逮捕された。 ・
\\	サッカーのワールドカップの試合前に起きた警官隊との衝突で何百人もが逮捕された。	変化形 
\\	【自動】心不全{しんふぜん}を起こす〔人が〕逮捕{たいほ}される【他動】〔警察{けいさつ}などが〕~を逮捕{たいほ}する[拘束{こうそく}する] ・
\\	重罪に当たる窃盗と横領の二つの訴因で逮捕された。 ・
\\	サッカーのワールドカップの試合前に起きた警官隊との衝突で何百人もが逮捕された。〈文〉〔注意{ちゅうい}などを一時的{いちじ てき}に〕引き付ける〈文〉〔進行{しんこう}などを〕阻む、止める〈文〉~を没収{ぼっしゅう}する、~を差し押さえる【名】〔違法行為{いほうこうい}による〕逮捕{たいほ}(すること)〔逮捕{たいほ}による〕拘留{こうりゅう}(されていること)〔動作{どうさ}や機能{きのう}の突然{とつぜん}の〕停止{ていし}〔可動部分{かどう ぶぶん}の〕停止装置{ていし そうち}レベル3、発音
\\	、カナアレスト、変化《動》
\\	、分節
\\	私はハエを小さな袋に生け捕りにして、しばらく眺めた。 ・
\\	彼らは敵から奪った高射砲を修理する。 ・
\\	イスラエルは1967年ヨルダン川西岸地区を占領しました。 ・
\\	クロアチア人たちはわずか5時間で三つの村を占拠しました。 ・
\\	事実、彼は市場のこの分野を独占するつもりだ。 ・
\\	彼らはそのドラマに夢中でした。 ・
\\	それはみんなの注意を引き付ける。 ・
\\	彼が作る何とも言えない音楽に引きつけられる気がする。 ・
\\	彼の演説が新聞・雑誌の話題をさらった。 ・
\\	けれども、レンズに捕らえられない漠としたものがある。 ・
\\	彼に起こったことはビデオに記録された。 ・
\\	彼の詩はしばしば厳しいその島の状態を表現している。	変化形 
\\	【他動】〔力や策略で〕~を捕らえる[捕獲{ほかく}する] ・
\\	私はハエを小さな袋に生け捕りにして、しばらく眺めた。 ・
\\	彼らは敵から奪った高射砲を修理する。〔力や策略{さくりゃく}で場所{ばしょ}を〕占拠{せんきょ}[占領{せんりょう}]する ・
\\	イスラエルは1967年ヨルダン川西岸地区を占領しました。 ・
\\	クロアチア人たちはわずか5時間で三つの村を占拠しました。〔試合{しあい}や競争{きょうそう}で〕~を勝ち取る[攻略{こうりゃく}する] ・
\\	事実、彼は市場のこの分野を独占するつもりだ。〔人の心などを〕魅了{みりょう}する、引き付ける ・
\\	彼らはそのドラマに夢中でした。 ・
\\	それはみんなの注意を引き付ける。 ・
\\	彼が作る何とも言えない音楽に引きつけられる気がする。 ・
\\	彼の演説が新聞・雑誌の話題をさらった。〔とらえにくいものを〕永続的な形に変える、形あるものに記録する◆絵画・彫刻・文章・映画などに記録すること。 ・
\\	けれども、レンズに捕らえられない漠としたものがある。 ・
\\	彼に起こったことはビデオに記録された。 ・
\\	彼の詩はしばしば厳しいその島の状態を表現している。《物理》〔原子核{げんしかく}などが電子{でんし}などを〕捕獲{ほかく}する《コ》〔データを〕ファイルに保存{ほぞん}する、キャプチャーする【名】〔力や策略{さくりゃく}による〕捕獲{ほかく}、捕虜{ほりょ}にすること〔力や策略{さくりゃく}によって〕捕獲{ほかく}した物、捕虜{ほりょ}《物理》〔電子{でんし}などの〕捕獲{ほかく}《コ》〔データの〕保存{ほぞん}、キャプチャー〔川の源流{げんりゅう}の〕付け替え◆工事や浸食により別の川につながること。〔チェスなどで相手{あいて}の〕駒を取ることレベル4、発音
\\	、カナカプチャー、キャプチャー、変化《動》
\\	、分節
\\	試合中のどんな暴力に対しても罰則が科されます。	変化形 
\\	【自動】罰する、懲らしめる【他動】(人)を罰する、(人)を懲らしめる〔違反{いはん}に〕罰則{ばっそく}を科す◆【用法】受け身で用いられることが多い。 ・
\\	試合中のどんな暴力に対しても罰則が科されます。~を手荒{てあら}に扱う、~を痛めつける(人)を不当{ふとう}[不公平{ふこうへい}]に取り扱うレベル2、発音
\\	、カナパニッシュ、変化《動》
\\	、分節
\\	今に後悔するよ。 ・
\\	私どもは彼女が去ってしまうことを残念に思います。 ・
\\	《謝る》自分のしたことを深く後悔しています。 ・
\\	残念ですが、彼の容態が改善されたと報告することはまだできません。 ・
\\	後悔してるの?/心残りはないの? ・
\\	今さら悔やんでももう遅い。 ・
\\	全然後悔していません。	変化形 
\\	【自動】後悔{こうかい}する【他動】~を残念{ざんねん}[遺憾{いかん}・気の毒]に思う、後悔{こうかい}する、悔しく思う、残念{ざんねん}ながら~する ・
\\	今に後悔するよ。 ・
\\	私どもは彼女が去ってしまうことを残念に思います。 ・
\\	《謝る》自分のしたことを深く後悔しています。 ・
\\	残念ですが、彼の容態が改善されたと報告することはまだできません。〔人の死などを〕惜しむ、悼む【名】〔悲しい出来事などに対する〕遺憾{いかん}(の意)、悲嘆{ひたん}、哀悼{あいとう}〔自分の行為・発言などに対する〕後悔{こうかい}、痛恨{つうこん}の念 ・
\\	後悔してるの?/心残りはないの? ・
\\	今さら悔やんでももう遅い。 ・
\\	全然後悔していません。〔招待に対する〕丁寧な断り(状)◆通例
\\	レベル3、発音
\\	、カナリグレット、変化《動》
\\	、分節
\\	収まるところに収まるものだ。 ・
\\	いかなるトラブルも、関与するバイヤーと売り手の間で直接に解決すべきである。 ・
\\	じゃあ決まった。 ・
\\	これで決まりだ。	変化形 
\\	【自動】〔人がある場所{ばしょ}で〕落ち着く、ゆったりする〔人がある場所{ばしょ}に〕定住{ていじゅう}する、住み着く〔浮かんでいる物が〕下に沈む、沈殿{ちんでん}する〔おりなどが沈んで液体{えきたい}が〕澄む、透明{とうめい}になる〔おりなどが沈んで〕積もる、固まる〔問題{もんだい}や紛争{ふんそう}などが〕解決{かいけつ}する、まとまる ・
\\	収まるところに収まるものだ。〔行動{こうどう}などについて〕決心{けっしん}する、確定{かくてい}する〔人が仕事{しごと}や新しい環境{かんきょう}に〕落ち着く、慣れる借金{しゃっきん}を払う[清算{せいさん}する]【他動】〔問題{もんだい}や紛争{ふんそう}を〕解決{かいけつ}する、終わらせる ・
\\	いかなるトラブルも、関与するバイヤーと売り手の間で直接に解決すべきである。〔条件{じょうけん}や行動{こうどう}を〕確定{かくてい}させる、決定{けってい}する ・
\\	じゃあ決まった。 ・
\\	これで決まりだ。〔負債{ふさい}などを〕清算{せいさん}する、決済{けっさい}する《法律》〔法的{ほうてき}な争いを合意{ごうい}により〕示談{じだん}にする〔人を〕住まわせる、定住{ていじゅう}させる〔場所{ばしょ}に〕定住{ていじゅう}する、入植{にゅうしょく}する〔人を場所{ばしょ}に〕落ち着かせる、ゆったりとさせる〔人を〕静かに[おとなしく]させる〔人を〕就職{しゅうしょく}[結婚{けっこん}]させる、身を固めさせる〔人に資産{しさん}を〕譲渡{じょうと}する、分与{ぶんよ}する〔バラバラの物を〕まとめる、固める〔浮いている物を下に〕沈ませる、落ち着かせる〔濁っている液体{えきたい}を〕澄ませる、濁りをなくす〔家畜{かちく}に〕種付けする【名】長椅子{ながいす}、ベンチ◆高い背もたれと肘掛がついていて多くの場合床から座部までが引き出しになっている。レベル3、発音
\\	、カナセトゥル、セトル、変化《動》
\\	、分節
\\	変化形 
\\	【他動】〔人の罪などを心底から〕許す、容赦{ようしゃ}する、勘弁{かんべん}する帳消{ちょうけ}しにする、免除{めんじょ}するレベル2、発音
\\	、カナファギブ、フォアギブ、変化《動》
\\	、分節
\\	状況がうそを正当化するかもしれない。/うそも方便。 ・
\\	お怒りはもっともでした。	変化形 
\\	【他動】〔不当だと批判された人・行為・考え方・意見などを〕正当{せいとう}だと説明{せつめい}する[証明{しょうめい}する・納得{なっとく}させる]、弁明{べんめい}する、正当化{せいとうか}する ・
\\	状況がうそを正当化するかもしれない。/うそも方便。 ・
\\	お怒りはもっともでした。《神学》〔神が罪人{ざいにん}〕を無罪{むざい}として許す《編集》〔行の左端または右端を〕そろえるレベル4、発音
\\	、カナジャスティファイ、ジャステファイ、変化《動》
\\	、分節
\\	《名言》人間は負けたら終わりなのではない、やめたら終わりなのだ。 ・
\\	参っているみたいだな。 ・
\\	人間{にんげん}は負けるようにはできていない。◆金言	変化形 
\\	【他動】〔敵を〕負かす、倒す、破る、打倒{だとう}する、打ち倒す ・
\\	《名言》人間は負けたら終わりなのではない、やめたら終わりなのだ。〔敵に〕(打ち)勝つ駄目{だめ}にする、無にする、挫折{ざせつ}させる、困らせる ・
\\	参っているみたいだな。《法律》無効化{むこう か}する【名】敗北{はいぼく}、負け ・
\\	人間{にんげん}は負けるようにはできていない。◆金言失敗{しっぱい}打倒{だとう}、打破{だは}レベル3、発音
\\	、カナデフィート、ディフィート、変化《動》
\\	、分節
\\	心拍が弱い。 ・
\\	私は胸が高鳴った[心臓がドキドキした]。◆興奮・病気など。 ・
\\	あいつなんかいつだってやっつけてやる。 ・
\\	彼らを打ち負かすことができないなら仲間になった方が安全です。/長い物には巻かれろ。 ・
\\	これ以上のものはないですよ。/それ以上は望めません。/あれにはかないません。 ・
\\	そのゲームはどのくらい(の時間)でクリアできますか。 ・
\\	あんなすごい男の子は見たことがない。◆【用法】通例
\\	の形で使われる。 ・
\\	彼の脈拍は1分当たり110でした。 ・
\\	くたびれちゃった。/疲れちゃった。 ・
\\	大丈夫? なんかやつれて見えるけど?	変化形 
\\	【自動】〔続けざまに〕打つ、たたく、殴る、殴打{おうだ}する〔心臓が〕鼓動{こどう}する ・
\\	心拍が弱い。 ・
\\	私は胸が高鳴った[心臓がドキドキした]。◆興奮・病気など。〔脈が〕打つ【他動】〔続けざまに〕~を打つ[たたく・殴る・殴打{おうだ}する]〔敵・競争相手{きょうそう あいて}などを〕打ち負かす、打ち砕く、やっつける、参らせる ・
\\	あいつなんかいつだってやっつけてやる。 ・
\\	彼らを打ち負かすことができないなら仲間になった方が安全です。/長い物には巻かれろ。 ・
\\	これ以上のものはないですよ。/それ以上は望めません。/あれにはかないません。~に先んじる、~を出し抜く〈米俗〉~をだます、詐欺{さぎ}にかける、だまし取る、詐取{さしゅ}する~を困惑{こんわく}させる、参らせる、~の理解{りかい}を拒む◆【類】
\\	~をかくはんする、スプーンや泡立て器などで強くかき混ぜる、泡立{あわだ}てる~を克服{こくふく}する〈米話〉〔料金{りょうきん}などを〕ごまかす、踏み倒す〔懲罰{ちょうばつ}・税金{ぜいきん}などを〕免れる、逃れる〔ゲームなどを〕クリアする、最後{さいご}まで終わらせる ・
\\	そのゲームはどのくらい(の時間)でクリアできますか。【名】〔等間隔{とう かんかく}で続けざまに〕打つ[たたく]こと、打って[たたいて]出る音〔警官{けいかん}などの〕巡回区域{じゅんかい くいき}、受け持ち区域、持ち場〈米方言〉優れている人[もの] ・
\\	あんなすごい男の子は見たことがない。◆【用法】通例
\\	の形で使われる。〈話〉=
\\	とも表記《医》拍動{はくどう}、脈拍{みゃくはく} ・
\\	彼の脈拍は1分当たり110でした。《物理》うなり《音楽》拍、拍子{ひょうし}《音楽》〔指揮者{しきしゃ}の〕振り《音楽》〔ロック音楽{おんがく}の〕強拍{きょうはく}、ビート〔韻文{いんぶん}の〕強韻{きょういん}〔釣りや猟で〕良く行く場所{ばしょ}、縄張り{なわばり}〔ジャーナリズムの〕スクープ短い間{ま}◆俳優などがせりふを言うときに、何らかの効果のために、特定の言葉と言葉の間で一拍置く(ごく短い時間沈黙する)こと。【形】〈話〉疲れ切って、へとへとで、くたくたで、くたびれて◆【類】
\\	くたびれちゃった。/疲れちゃった。 ・
\\	大丈夫? なんかやつれて見えるけど?ビート族のレベル2、発音
\\	、カナビート、変化《動》
\\	変化形 
\\	【他動】〔武力{ぶりょく}で領土{りょうど}などを〕獲得{かくとく}[制圧{せいあつ}・征服{せいふく}]する〔敵・対戦相手{たいせん あいて}に〕勝つ〔努力して名声などを〕勝ち取る、得る〔異性{いせい}の心を〕射止める{いとめる}〔悪習{あくしゅう}・因習{いんしゅう}などを〕打ち破る、打破{だは}する〔困難{こんなん}・障害{しょうがい}・病気{びょうき}などを〕克服{こくふく}する、乗り越える〔激情{げきじょう}などを〕抑えるレベル4、発音
\\	、カナカンカー、コンカー、変化《動》
\\	、分節
\\	労働に関しては多数の法律が設けられている。 ・
\\	《契約書》本契約はオーストラリア国法に準拠するものとする。 ・
\\	電極反応速度は、平たい電極表面の拡散によって影響される。	変化形 
\\	【自動】〔政治的{せいじ てき}に〕支配{しはい}する、統治{とうち}する〔決定的{けってい てき}な〕影響力{えいきょうりょく}を与える【他動】〔政治的{せいじ てき}に〕~を支配{しはい}[統治{とうち}]する~を管理{かんり}[運営{うんえい}・統制{とうせい}]する ・
\\	労働に関しては多数の法律が設けられている。 ・
\\	《契約書》本契約はオーストラリア国法に準拠するものとする。~を抑制{よくせい}する、~を抑える~に影響{えいきょう}を与える、~を左右{さゆう}する ・
\\	電極反応速度は、平たい電極表面の拡散によって影響される。《機械》〔エンジンの速度{そくど}を〕調節{ちょうせつ}する《法律》~の法律{ほうりつ}となる律則{りっそく}するレベル4、発音
\\	、カナガバーン、ガバン、変化《動》
\\	、分節
\\	アメリカのチームが優勢だった。 ・
\\	今週はハリケーンがニュースの中心だった。	変化形 
\\	【自動】支配{しはい}する、権勢{けんせい}を振るう卓越{たくえつ}する、優位{ゆうい}に立つ ・
\\	アメリカのチームが優勢だった。【他動】~を支配{しはい}する、~に権勢{けんせい}を振るう~で最も重要{じゅうよう}である、~より優位{ゆうい}に立つ ・
\\	今週はハリケーンがニュースの中心だった。~を特色付ける、~に影響{えいきょう}を及ぼす~を見下ろす{みおろす}、~にそびえ立つレベル5、発音
\\	、カナドミネイト、変化《動》
\\	、分節
\\	彼女は英語が上手です。 ・
\\	応募者はある程度日本語を使いこなすことができ、日常会話ではかなり流ちょうな日本語を話せなければなりません。	変化形 
\\	【自動】命令{めいれい}する、命令{めいれい}を下す指揮{しき}する、指揮権{しきけん}を持つ見渡す{みわたす}、見晴らす{みはらす}【他動】〔人に~しろと〕命令{めいれい}[要求{ようきゅう}]する〔軍隊などを〕指揮{しき}[統制{とうせい}]する〔場所{ばしょ}などを〕支配{しはい}する、統治{とうち}する〔言葉{ことば}や金などを〕自由{じゆう}に操る、意のままにする〔尊敬・関心・信頼・同情・高評価などを〕当然{とうぜん}のこととして受ける[集める]〔位置{いち}が場所{ばしょ}を〕見渡す{みわたす}、見晴らす{みはらす}【名】指揮{しき}すること、命令{めいれい}を出すこと〔権威{けんい}を伴う〕命令{めいれい}、指令{しれい}〔言語などの〕自由{じゆう}に使える能力、駆使{くし}[運用{うんよう}]能力{のうりょく}
\\	~(~を操るのがうまい/下手だ)のように、能力が高い場合にも低い場合にも使うことができる。 ・
\\	彼女は英語が上手です。 ・
\\	応募者はある程度日本語を使いこなすことができ、日常会話ではかなり流ちょうな日本語を話せなければなりません。《コ》コマンド◆利用者がコンピューターにある動作を行わせるために与える命令。《軍事》指揮権{しきけん}、指導権{しどう けん}《軍事》〔指揮官{しきかん}の〕統括部隊{とうかつ ぶたい}、管轄地域{かんかつ ちいき}《軍事》司令部{しれいぶ}眺望{ちょうぼう}、見晴{みは}らし【形】命令{めいれい}の[に関する・用の]指揮(者){しき(しゃ)}のレベル4、発音
\\	、カナコマンド、変化《動》
\\	、分節
\\	もう上司に企画書を提出しましたか。 ・
\\	これらの議論は間違っているし、誤解を招くものであると考えます。	変化形 
\\	【自動】服従{ふくじゅう}する、屈服{くっぷく}する、おとなしく従う【他動】~を提出{ていしゅつ}する、投稿{とうこう}する、提示{ていじ}する、提起{ていき}する ・
\\	もう上司に企画書を提出しましたか。~を服従{ふくじゅう}させる〔通例{つうれい}
\\	以下が事実{じじつ}または必要{ひつよう}であると〕言う、考える、提案{ていあん}する ・
\\	これらの議論は間違っているし、誤解を招くものであると考えます。レベル4、発音
\\	、カナサブミット、変化《動》
\\	、分節
\\	組合は会社側の提案を蹴{け}った。 ・
\\	彼女は見事に彼をふった。	変化形 
\\	【他動】〔要求{ようきゅう}・提案{ていあん}・訴えなどを〕拒絶{きょぜつ}する、拒否{きょひ}する、拒む、断る、受け入れない、認めない、退ける、はねつける ・
\\	組合は会社側の提案を蹴{け}った。〔法案{ほうあん}などを〕否決{ひけつ}する、棄却{ききゃく}する、却下{きゃっか}する〔受験者などを〕不合格{ふごうかく}にする〔不適切な人・ものを〕排斥{はいせき}する、不合格{ふごうかく}と判定する、はねる〔求愛者{きゅうあいしゃ}などを〕振る ・
\\	彼女は見事に彼をふった。〔胃が食物{たべもの}を〕受け付けない〔体が~に〕拒絶反応{きょぜつ はんのう}を示す【名】拒絶{きょぜつ}[拒否{きょひ}・却下{きゃっか}]された人[物]、不合格品{ふごうかく ひん}、不良品{ふりょうひん}、傷物{きずもの}、廃棄物{はいきぶつ}レベル3、発音《名》
\\	《動》
\\	、カナリジェクト、変化《動》
\\	、分節
\\	変化形 
\\	【他動】〔相手{あいて}と力ずくで〕争う、敵対{てきたい}する〔提案{ていあん}などに〕反対{はんたい}する、妨害{ぼうがい}する〔比較{ひかく}などのために〕~を対比{たいひ}[対立{たいりつ}]させる~を反対側{はんたいがわ}[向かい合わせ]に置く《動物》〔親指{おやゆび}を手のひらなどに〕置く《動物》〔手のひら同士{どうし}を〕合わせるレベル3、発音
\\	、カナオポーズ、オポウズ、変化《動》
\\	、分節
\\	私の娘は怖がりだ。 ・
\\	そんなふうに脅かすなよ。 ・
\\	そんなこけおどしは利かないぞ。 ・
\\	怖くて父親に嫌だとは言えなかった。 ・
\\	「蛇が怖いんです」「蛇の何がそんなに怖いんですか」
\\	変化形 
\\	"【自動】怖がる、恐ろしがる ・
\\	私の娘は怖がりだ。【他動】〔人を〕怖がらせる、びっくりさせる、驚かせる ・
\\	そんなふうに脅かすなよ。 ・
\\	そんなこけおどしは利かないぞ。 ・
\\	怖くて父親に嫌だとは言えなかった。 ・
\\	「蛇が怖いんです」「蛇の何がそんなに怖いんですか」【名】〔突然{とつぜん}の〕恐怖{きょうふ}、恐れ恐慌状態{きょうこう じょうたい}、極度{きょくど}の不安{ふあん}、パニック怖がらせるもの[状況{じょうきょう}]レベル4、発音
\\	、カナスケア、変化《動》
\\	怖がらないで。 ・
\\	怖いものは何もないんだよ。	変化形 
\\	【自動】怖がる、ギクリとする【他動】(人)を怖がらせる、(人)をギクリとさせる
\\	と共に最も一般的な「怖がらせる」動詞。
\\	より
\\	の方がやや口語的。 ・
\\	怖がらないで。 ・
\\	怖いものは何もないんだよ。〔怖がらせて〕~を追い出す[追い払う]レベル3、発音
\\	、カナフライトゥン、フライトン、変化《動》
\\	、分節
\\	犬が近所迷惑にならないようにしましょう。 ・
\\	隣のピアノの音に閉口している。 ・
\\	イライラしてきたよ。	変化形 
\\	【自動】イライラさせる、うるさがらせる、困らせる ・
\\	犬が近所迷惑にならないようにしましょう。【他動】〔嫌なことを繰り返して人を少し〕イライラさせる、いらつかせる、苛たせる、ムッとさせる、うるさがらせる◆【類】
\\	隣のピアノの音に閉口している。 ・
\\	イライラしてきたよ。〔繰り返し人を〕苦しめる、困らせる、悩ませるレベル5、発音
\\	、カナアノイ、変化《動》
\\	、分節
\\	こんなときに申し訳なく存じます[申し訳ないことです]。 ・
\\	おかまいなく、どうぞそのまま。 ・
\\	睡眠中は邪魔をしないでください。	変化形 
\\	【他動】〔忙しい人の〕邪魔{じゃま}をする、気をそらす ・
\\	こんなときに申し訳なく存じます[申し訳ないことです]。 ・
\\	おかまいなく、どうぞそのまま。(人)を不安{ふあん}に[動揺{どうよう}・困惑{こんわく}]させる〔仕事{しごと}などを〕阻害{そがい}する、妨げる〔平安{へいあん}や静けさを〕乱す、破る、かき乱す(人)を起こす、~を目覚めさせる ・
\\	睡眠中は邪魔をしないでください。レベル5、発音
\\	、カナディスターブ、デスターブ、変化《動》
\\	、分節
\\	私にやらせて[任せて]ください。 ・
\\	どうしよう? ・
\\	この頃酒が弱くなった。 ・
\\	私は辛い物でも平気です。 ・
\\	私はアルコールが駄目です。	変化形 
\\	【自動】〔機械{きかい}・乗り物が〕操縦{そうじゅう}[操作{そうさ}・運転{うんてん}]される[できる]【他動】〔手で〕~に触れる、~を持つ、~を動かす〔手で〕~を操作{そうさ}する、~を巧みに扱う、~を操縦{そうじゅう}する〔仕事{しごと}などを〕担当{たんとう}する、運営{うんえい}する ・
\\	私にやらせて[任せて]ください。〔問題{もんだい}などに〕対処{たいしょ}する、処理{しょり}する ・
\\	どうしよう?〔人を〕管理{かんり}する、処遇{しょぐう}する〔酒と〕うまく付き合う、〔酒に〕飲まれないようにする ・
\\	この頃酒が弱くなった。〈米〉〔商品{しょうひん}を〕取引{とりひき}する、売買{ばいばい}する~を問題{もんだい}なく摂取{せっしゅ}する◆食品などについて、体質的な(または好みの上での)問題の有無を表す。◆【用法】
\\	を伴うことが多い。 ・
\\	私は辛い物でも平気です。 ・
\\	私はアルコールが駄目です。【名】取っ手、ハンドル、手でつかむ部分{ぶぶん}、握り、柄〔目的達成{もくてき たっせい}の〕機会{きかい}、手段{しゅだん}〔ある賭の〕掛け金総額{きん そうがく}〈俗〉名前{なまえ}、あだ名、ニックネーム〈英〉〔生地{きじ}の〕手触り{てざわり}◆【同】
\\	〈和製英語〉〔車の〕ハンドル◆【標準英語】
\\	レベル2、発音
\\	、カナハンドゥル、ハンドル、変化《動》
\\	、分節
\\	その機械は最高温度200
\\	で動作させることができます。	変化形 
\\	【自動】〔機械などが〕動作{どうさ}[作動{さどう}・稼働{かどう}]する ・
\\	その機械は最高温度200
\\	で動作させることができます。〔会社{かいしゃ}や組織{そしき}が〕営業{えいぎょう}する、活動{かつどう}する〔思想{しそう}などが〕影響{えいきょう}を及ぼす〔薬などが〕効果{こうか}がある〔医者{いしゃ}が〕手術{しゅじゅつ}する〔軍隊{ぐんたい}などが〕作戦行動{さくせん こうどう}する〔証券{しょうけん}や商品市場{しょうひん しじょう}で〕取引{とりひき}する、売買{ばいばい}する〈話〉〔非合法{ひ ごうほう}な〕商売{しょうばい}をする、いかさまをする【他動】〔機械{きかい}などを〕操作{そうさ}する、運転{うんてん}する〔会社{かいしゃ}や組織{そしき}などを〕経営{けいえい}する、運営{うんえい}する〔結果{けっか}や変化{へんか}などを〕もたらす、起こす〔患部{かんぶ}などを〕手術{しゅじゅつ}するレベル3、発音
\\	、カナアパレイト、オペレイト、変化《動》
\\	、分節
\\	変化形 
\\	【他動】〔必要{ひつよう}なものを〕備え付ける、備える、用意{ようい}する、装備{そうび}する、配備{はいび}する、支度{したく}する〔知識{ちしき}・素養{そよう}などを人に〕授ける[身に付けさせる]〔必要{ひつよう}な機能{きのう}を組織{そしき}など〕に備えるレベル6、発音
\\	、カナイクウィップ、エキップ、エクィップ、変化《動》
\\	、分節
\\	参加したい人は誰でも参加できる。 ・
\\	その事故から生還された幸せを私も分かち合いたいのですが。	変化形 
\\	【自動】〔活動・出来事に〕参加{さんか}する、加わる、参画{さんかく}する ・
\\	参加したい人は誰でも参加できる。共有する、分かち合う ・
\\	その事故から生還された幸せを私も分かち合いたいのですが。〈古〉〔~の〕気味{きみ}がある【他動】〈古〉~を共にするレベル5、発音
\\	、カナパーティシペート、パーティスィペイト、パーテシペイト、変化《動》
\\	、分節
\\	変化形 
\\	【自動】〈話〉〔困難{こんなん}にもかかわらず〕やり通す、立場{たちば}を守り続ける【他動】〔動作{どうさ}を〕持続{じぞく}[継続{けいぞく}]する〔状態{じょうたい}を〕保持{ほじ}[維持{いじ}]する〔~を良好{りょうこう}な状態{じょうたい}に〕メンテナンスする、維持{いじ}する(人)を養う、(人)を扶養{ふよう}する~を生かしておく、~の生命{せいめい}を維持{いじ}する〔~が正しいことを〕断言{だんげん}する、主張{しゅちょう}する〔批判{ひはん}を受ける意見{いけん}や立場{たちば}を〕擁護{ようご}する、支持{しじ}するレベル3、発音
\\	、カナメインテイン、メインテン、変化《動》
\\	、分節
\\	彼には、そのように高度な組織を維持していくための技術力が備わっていない。 ・
\\	《裁判》異議認めます。 ・
\\	異議を認めます。 ・
\\	裁判官は彼女の異議を認めた。	変化形 
\\	【自動】持ちこたえる【他動】〔~の存在〕を持続{じぞく}[維持{いじ}]する ・
\\	彼には、そのように高度な組織を維持していくための技術力が備わっていない。〔構造物を下から〕支える、支持{しじ}する(人)を養う[扶養{ふよう}する](人)を元気{げんき}づける[精神的{せいしん てき}に支える]《法律》〔申し立てや異議〕を認める ・
\\	《裁判》異議認めます。 ・
\\	異議を認めます。 ・
\\	裁判官は彼女の異議を認めた。〔大変{たいへん}なこと〕に耐える[持ちこたえる]〔損害{そんがい}などを〕受ける、被る〔主張{しゅちょう}などを〕支持{しじ}する、裏付ける{うらづける}〔役割{やくわり}などを〕こなす、うまく務める【名】《音楽》サステイン、サスティーン◆楽音が減衰しながら保持されていること。レベル5、発音
\\	、カナサステイン、変化《動》
\\	、分節
\\	静かにして。気が散るでしょ。	変化形 
\\	【自動】〈文〉堕落{だらく}する、破滅{はめつ}する、没落{ぼつらく}する【他動】〔原型{げんけい}をとどめないほど〕~を破壊{はかい}[粉砕{ふんさい}]する(人)を破産{はさん}[没落{ぼつらく}]させる〔直せないほど〕~を台無{だいな}しにする[損なう] ・
\\	静かにして。気が散るでしょ。〈古〉〔女を誘惑{ゆうわく}して〕貞操{ていそう}を奪う【名】〔建物{たてもの}などの〕廃虚{はいきょ}、遺跡{いせき}◆通例、
\\	〔建物{たてもの}などの〕荒廃{こうはい}、破壊{はかい}、崩壊{ほうかい}〔精神的{せいしん てき}な〕荒廃{こうはい}、〔社会的{しゃかい てき}な〕堕落{だらく}、〔経済的{けいざい てき}な〕破産{はさん}荒廃{こうはい}[破壊{はかい}・破滅{はめつ}]の原因{げんいん}[元]破産者{はさんしゃ}、身を持ち崩した人、落ちぶれた人〈古〉貞操{ていそう}の喪失{そうしつ}レベル3、発音
\\	、カナルーイン、ルイン、変化《動》
\\	、分節
\\	この新しいソフトを使おうとしたらコンピューターがクラッシュしてしまった。 ・
\\	私たちは一夜をしのぐための[とりあえず一泊できる]場所を探していました。 ・
\\	今晩おまえの家に泊めてもらえないかなあ。 ・
\\	「買い物行く?」「行くわけないだろ。今から寝るんだよ」
\\	変化形 
\\	"【1自動】〔大きな音を立てて〕衝突{しょうとつ}する〔航空機{こうくうき}が〕墜落{ついらく}する〔物が〕つぶれる、砕ける〔事業{じぎょう}などが〕崩壊{ほうかい}する《コ》クラッシュする、故障{こしょう}する、機能{きのう}が停止{ていし}する ・
\\	この新しいソフトを使おうとしたらコンピューターがクラッシュしてしまった。〈俗〉〔他人{たにん}の家に〕泊まる、〔いつもとは違う場所{ばしょ}で〕寝る◆特に、予定外の場所に臨時で泊まること。◆【参考】
\\	私たちは一夜をしのぐための[とりあえず一泊できる]場所を探していました。 ・
\\	今晩おまえの家に泊めてもらえないかなあ。〈米俗〉寝る◆【同】
\\	「買い物行く?」「行くわけないだろ。今から寝るんだよ」〈俗〉酔いつぶれる、意識{いしき}を失う、麻薬{まやく}が切れる【1他動】〔大きな音を立てて〕~をつぶす[壊す]〔車などを〕~に衝突{しょうとつ}させる〔パーティーなどに強引{ごういん}に〕押し入る、押し掛ける〈俗〉〔人の断りもなしに〕家に住み着く【1名】衝突(音){しょうとつ(おん)}、破壊{はかい}、墜落{ついらく}、崩壊{ほうかい}《コ》クラッシュ、〔プログラムの〕強制終了{きょうせいしゅうりょう}、機能停止{きのう ていし}、システム破壊{はかい}、故障{こしょう}値崩れ{ねくずれ}、暴落{ぼうらく}、恐慌{きょうこう}、破綻{はたん}大失敗{だい しっぱい}〈俗〉《医》心(肺)停止{しん(ぱい)ていし}【1形】集中的{しゅうちゅうてき}な【2名】〔タオル・製本{せいほん}などに使う〕目の粗い綿布{めんぷ}[亜麻布{あまふ}]◆【語源】ロシア語
\\	(染めたリンネル)より。レベル2、発音
\\	、カナクラッシュ、変化《動》
\\	市場が崩壊する前に、無事に1200万ドル売った。 ・
\\	地震でビルが崩れた。 ・
\\	氷の薄い部分が割れた。 ・
\\	本の重みで床が抜けた。 ・
\\	アラバマ州で橋が落ち、47人が死亡。 ・
\\	集中豪雨で橋が流れた。 ・
\\	「彼女にすぐ朝食の用意をさせますから」「おなかが減ってもう倒れそうです」 ・
\\	疲れてぶっ倒れたら私のところへ来なさい。治してあげます。 ・
\\	父が過労で倒れた。 ・
\\	彼女は気を失って倒れた。 ・
\\	ツリーを折り畳むにはこのアイコンをクリックしてください。 ・
\\	共産主義の崩壊は、古い敵対関係と新たな危機を呼び起こした。/共産主義の崩壊に伴い、古い敵愾心と新たな危険が表面化してきました。 ・
\\	同国は経済破綻の瀬戸際にあった。
\\	変化形 
\\	"【自動】〔建造物・組織・体制などが〕つぶれる、崩れる、崩壊{ほうかい}する、倒壊{とうかい}する ・
\\	市場が崩壊する前に、無事に1200万ドル売った。 ・
\\	地震でビルが崩れた。 ・
\\	氷の薄い部分が割れた。 ・
\\	本の重みで床が抜けた。〔橋が〕落ちる、流れる ・
\\	アラバマ州で橋が落ち、47人が死亡。 ・
\\	集中豪雨で橋が流れた。崩れるように倒れる、卒倒{そっとう}する、へなへなになる、へたり込む、ヘタヘタと座り込む、衰弱{すいじゃく}する ・
\\	「彼女にすぐ朝食の用意をさせますから」「おなかが減ってもう倒れそうです」 ・
\\	疲れてぶっ倒れたら私のところへ来なさい。治してあげます。 ・
\\	父が過労で倒れた。 ・
\\	彼女は気を失って倒れた。〔心理的{しんりてき}に〕つぶれる、〔悲しみに〕崩れ落ちる、くじける、挫折{ざせつ}する〔交渉{こうしょう}などが〕決裂{けつれつ}する〔計画{けいかく}などが〕破綻{はたん}する〔慣習{かんしゅう}が〕廃れる【他動】つぶす、崩壊{ほうかい}させる《コ》折り畳む◆コンピューターの
\\	で、ツリー状になったフォルダを◆【対】
\\	ツリーを折り畳むにはこのアイコンをクリックしてください。【名】〔建造物・組織・体制などの〕崩壊{ほうかい}、倒壊{とうかい} ・
\\	共産主義の崩壊は、古い敵対関係と新たな危機を呼び起こした。/共産主義の崩壊に伴い、古い敵愾心と新たな危険が表面化してきました。〔経済{けいざい}などの〕破綻{はたん} ・
\\	同国は経済破綻の瀬戸際にあった。〔株価などの〕暴落{ぼうらく}〔需要などの〕急減{きゅうげん}〔計画などの〕失敗{しっぱい}〔希望などの〕挫折{ざせつ}〔体力などの〕衰弱{すいじゃく}《医》虚脱{きょだつ}レベル4、発音
\\	、カナコラップス、コラプス、変化《動》
\\	、分節
\\	料理人が多いとスープが駄目になる。/船頭多くして、船、山に上る。 ・
\\	彼は子どもの頃ずいぶん甘やかされた。 ・
\\	かわいい子には旅をさせよ。/むちを惜しめば子どもを駄目にする。◆ことわざ ・
\\	多くの兵士が戦利品を記念として祖国に持ち帰る。 ・
\\	利権を大事にする政治家は公共の利益をないがしろにする。 ・
\\	採鉱で出た廃物を近くの川に捨てた。	変化形 
\\	【自動】台無{だいな}しになる、役に立たなくなる、駄目{だめ}になる〔食べ物が〕腐る【他動】〔物・事を〕台無{だいな}しにする、悪くする、役に立たなくする ・
\\	料理人が多いとスープが駄目になる。/船頭多くして、船、山に上る。〔価値{かち}を〕損なう〔食べ物などを〕腐らせる〔子どもなどを〕甘やかす、〔甘やかして〕駄目{だめ}にする ・
\\	彼は子どもの頃ずいぶん甘やかされた。 ・
\\	かわいい子には旅をさせよ。/むちを惜しめば子どもを駄目にする。◆ことわざ【名】
\\	戦利品{せんりひん}、略奪品{りゃくだつ ひん} ・
\\	多くの兵士が戦利品を記念として祖国に持ち帰る。
\\	〔選挙{せんきょ}の勝利{しょうり}で獲得{かくとく}した〕利益{りえき}、特権{とっけん} ・
\\	利権を大事にする政治家は公共の利益をないがしろにする。〔略奪{りゃくだつ}の〕餌食{えじき}、獲物{えもの}〔発掘{はっくつ}などの際に出る〕廃物{はいぶつ} ・
\\	採鉱で出た廃物を近くの川に捨てた。〈古〉略奪{りゃくだつ}レベル3、発音
\\	、カナスポイル、変化《動》
\\	または
\\	その年下の男性は、同僚の優れた知識と経験に屈しました。 ・
\\	価格に関して少し譲歩してもかまいません。 ・
\\	もう一歩も後へ引けない。	変化形 
\\	【自動】〔農産物{のうさんぶつ}や鉱物{こうぶつ}が〕産出{さんしゅつ}する〔努力{どりょく}や投資{とうし}によって〕収益{しゅうえき}が出る、利益{りえき}が挙がる〔戦いなどで相手{あいて}に〕降伏{こうふく}する、屈服{くっぷく}する ・
\\	その年下の男性は、同僚の優れた知識と経験に屈しました。〔自然{しぜん}の力で〕押し戻される、へこむ〔議論{ぎろん}・説得{せっとく}・懇願{こんがん}などに〕折れる、応じる、譲歩{じょうほ}する ・
\\	価格に関して少し譲歩してもかまいません。〔他のものに〕取って代わられる〔他の車を通すために〕停止{ていし}する、速度{そくど}を落とす【他動】〔農産物{のうさんぶつ}を〕産出{さんしゅつ}[生産{せいさん}]する〔努力{どりょく}の結果{けっか}を〕生じさせる、生み出す〔報酬{ほうしゅう}や利益{りえき}を〕挙げる、得る〔屈服{くっぷく}によって〕~を明け渡す[放棄{ほうき}する] ・
\\	もう一歩も後へ引けない。〔敬意{けいい}などから〕~を譲る[差し出す]【名】〔作物{さくもつ}などの〕収穫高{しゅうかく だか}、生産量{せいさんりょう}、産出量{さんしゅつりょう}《金融》〔投資{とうし}の〕収益{しゅうえき}、利回り{りまわり}、配当{はいとう}《化学》収率{しゅうりつ}◆化学反応による生成物の量で、理論的に得られる量の百分率で表される。◆【同】
\\	《物理》核威力{かく いりょく}◆核爆発によって放出されるエネルギー量を
\\	火薬の量で表したもの。◆【同】
\\	レベル4、発音
\\	、カナイールド、変化《動》
\\	キャンペーンで桁外れの2万ドルが集められた。 ・
\\	ギャンブルは巨額の利益を生み、暴力団はそこから利益を得ている。 ・
\\	抗体は抗原を攻撃するために生み出される。	変化形 
\\	【他動】~を生む、~を起こす ・
\\	キャンペーンで桁外れの2万ドルが集められた。 ・
\\	ギャンブルは巨額の利益を生み、暴力団はそこから利益を得ている。〔物理化学的過程{ぶつり かがく てき かてい}で〕~を発生{はっせい}させる、~を作り出す ・
\\	抗体は抗原を攻撃するために生み出される。〔子孫{しそん}を〕もうける、産む◆【同】
\\	〔数学{すうがく}や言語学{げんごがく}で〕~を生成{せいせい}する◆厳密に定められた一連の規則を適用することによってある出力を得ること。レベル6、発音
\\	、カナジェネレイト、変化《動》
\\	、分節
\\	独特の音が出る。 ・
\\	その元素は生命の出現を可能にする。 ・
\\	最終的には誰が
\\	国の新しい指導者になるのだろうか。 ・
\\	両国間で考え方の差が明らかになってきた。 ・
\\	新計画の詳細が明らかになりつつある。	変化形 
\\	【自動】表面{ひょうめん}に出てくる、現れる、出現{しゅつげん}する、浮かび出る、浮かび上がる、浮上{ふじょう}する ・
\\	独特の音が出る。 ・
\\	その元素は生命の出現を可能にする。 ・
\\	最終的には誰が
\\	国の新しい指導者になるのだろうか。〔問題{もんだい}が〕持ち上がる、浮かび上がる分かってくる、明らかになる、はっきりする ・
\\	両国間で考え方の差が明らかになってきた。 ・
\\	新計画の詳細が明らかになりつつある。レベル4、発音
\\	、カナイマージ、エマージ、変化《動》
\\	、分節
\\	二つの宗教グループの間で内戦が起こった。 ・
\\	この問題が起きて残念です[申し訳なく存じます]。	変化形 
\\	【自動】起こる、生じる、現れる、生まれる、発生{はっせい}する ・
\\	二つの宗教グループの間で内戦が起こった。 ・
\\	この問題が起きて残念です[申し訳なく存じます]。起きる、起床{きしょう}する、行動{こうどう}を開始{かいし}する〔風が〕たつ〔死から〕よみがえる《法律》〔推定が〕働くレベル8、発音
\\	、カナアライズ、変化《動》
\\	、分節
\\	変化形 
\\	【他動】~を発明{はつめい}する、考案{こうあん}する、創作{そうさく}する~をでっち上げるレベル3、発音
\\	、カナインベント、変化《動》
\\	、分節
\\	変化形 
\\	【他動】〔土地{とち}や田畑{たはた}を〕耕す、耕作{こうさく}する〔土壌{どじょう}を耕運機{こううんき}で〕緩める、ほぐす〔植物{しょくぶつ}や農作物{のうさくぶつ}を〕育てる、栽培{さいばい}する〔微生物{びせいぶつ}などを〕培養{ばいよう}する、増殖{ぞうしょく}させる〔学習{がくしゅう}で能力{のうりょく}などを〕育てる、養成{ようせい}する〔教育{きょういく}で人を〕啓発{けいはつ}する、洗練{せんれん}させる〔自分{じぶん}の利益{りえき}のために人と〕知り合い[近づき]になるレベル5、発音
\\	、カナカルチベイト、カルティベイト、変化《動》
\\	、分節
\\	変化形 
\\	【他動】〔部品{ぶひん}を使って~を〕組み立てる、建設{けんせつ}する、建造{けんぞう}する〔議論{ぎろん}や文などを〕組み立てる、構成{こうせい}する《数学》〔コンパスなどを使って図形{ずけい}を〕作図{さくず}する、正確{せいかく}に描く【名】建築物{けんちくぶつ}、建造物{けんぞうぶつ}構図{こうず}、構成概念{こうせい がいねん}レベル4、発音《名》
\\	《動》
\\	、カナコンストゥラクト、コンストラクト、変化《動》
\\	、分節
\\	変化形 
\\	【他動】〔機械で大量に〕作る、製作{せいさく}する、製造{せいぞう}する〔原材料を〕加工{かこう}する〔話を〕でっち上げる【名】〔機械による量産方式の〕製造(物){せいぞう(ぶつ)}、製造業{せいぞうぎょう}製品{せいひん}◆通例、
\\	レベル4、発音
\\	、カナマニファクチャー、マニュファクチャー、変化《動》
\\	、分節
\\	ハリケーンに備えてたっぷりと食料を蓄えた。 ・
\\	ディスプレーはいつもきれいにして最新の商品を置くように気を付けた。 ・
\\	新しい納屋を建てて10頭の牛を入れた。 ・
\\	われわれはその川にマスを放流しました。 ・
\\	冷蔵庫には脂肪分の多いソフト・ドリンクやスナックではなく低脂肪牛乳と野菜を入れなさい。 ・
\\	これらの食品を家庭で蓄えておけば、毎日おいしい食事が作れます。 ・
\\	日本海におけるタラ資源の回復のための緊急手段を提案しました。 ・
\\	自分勝手な行動が彼の評判を劇的に落とした。 ・
\\	ユト・アステカ語はアメリカ先住民族の主要な語族の一つだった。 ・
\\	接ぎ木によって新品種を作ることはできない。接ぎ木も接ぎ穂もともに自分の特質を維持するからである。 ・
\\	その種子は染料の材料として用いられる。	変化形 
\\	【自動】物を蓄える ・
\\	ハリケーンに備えてたっぷりと食料を蓄えた。《植物》芽を吹く【他動】〔店に商品{しょうひん}を〕仕入れる{しいれる}、置く ・
\\	ディスプレーはいつもきれいにして最新の商品を置くように気を付けた。〔農場{のうじょう}などに家畜{かちく}を〕入れる、放つ ・
\\	新しい納屋を建てて10頭の牛を入れた。〔川などに魚を〕放つ、放流{ほうりゅう}する ・
\\	われわれはその川にマスを放流しました。〔物で〕~を満たす、〔物を〕~に入れる ・
\\	冷蔵庫には脂肪分の多いソフト・ドリンクやスナックではなく低脂肪牛乳と野菜を入れなさい。〈古〉(人)に足かせをつけて罰する【名】在庫{ざいこ}、在庫品{ざいこひん}、ストック〔将来{しょうらい}に備えた〕蓄え、備蓄{びちく}、貯蔵{ちょぞう}、ストック ・
\\	これらの食品を家庭で蓄えておけば、毎日おいしい食事が作れます。〔鉱物{こうぶつ}や生物{せいぶつ}などの〕資源{しげん} ・
\\	日本海におけるタラ資源の回復のための緊急手段を提案しました。〈英〉資本金{しほんきん}、払込資本金{はらいこみ しほんきん}◆【同】
\\	〔一社の全体の〕株{かぶ}、株式{かぶしき}〔人の〕評判{ひょうばん}、人気{にんき} ・
\\	自分勝手な行動が彼の評判を劇的に落とした。家畜{かちく}◆【同】
\\	〔共通{きょうつう}の〕祖先{そせん}、〔同一祖先{どういつ そせん}から生まれた〕家系{かけい}、〔動植物{どうしょくぶつ}の〕種族{しゅぞく}《言語学》語族{ごぞく} ・
\\	ユト・アステカ語はアメリカ先住民族の主要な語族の一つだった。《植物》幹《料理》ブイヨン◆スープやソースの素《植物》〔接ぎ木の〕台木{だいぎ} ・
\\	接ぎ木によって新品種を作ることはできない。接ぎ木も接ぎ穂もともに自分の特質を維持するからである。《植物》切り株、〔木材{もくざい}を切り出す〕丸太{まるた}〔家畜{かちく}の小さな〕囲い、おり〔銃の〕台尻{だいじり}、銃床{じゅうしょう}〔野戦砲{やせん ほう}の〕棒型{ぼうがた}の架尾{かび}◆架尾
\\	の形式の一つで、棒状に後部に突き出た構造をしており、弾丸発射時には二つの車輪と共に野戦砲を支える働きをする。◆【参考】
\\	〔工具{こうぐ}や釣り竿などの〕握り、柄〔鐘をつるす〕木部構造{もくぶ こうぞう}《海事》〔いかりの〕ストック◆いかりの先端にある十字型構造の横の部分。原料{げんりょう} ・
\\	その種子は染料の材料として用いられる。《冶金》金属材料{きんぞく ざいりょう}◆金属成型加工時に用いられる金属の材料全般を指す言葉で、その形態はどのようなものでも構わない。《植物》ストック、アラセイトウ◆ユーラシアおよび地中海沿岸原産のアブラナ科アラセイトウ属またはマティオラ属
\\	の一年草または多年草の総称だが、特にアラセイトウ(学名
\\	を指すことが多い。強い香りのする鮮やかな花を多数つけ、観賞用として栽培される。《植物》=
\\	〔乗馬{じょうば}の〕ストック◆18世紀に男性が乗馬時につけた幅広の襟飾りで、その後、男女の乗馬の正装の一部となった。◆【同】
\\	《トランプ》ストック、山札{やまふだ}、場札{ばふだ}◆ゲーム開始時に、配られた後に残って裏向きに積み重ねられたカードの山。《劇》レパートリー、上演題目{じょうえん だいもく}◆【同】
\\	《劇》=
\\	《地学》岩盤{がんばん}◆【同】
\\	鉄道車両{てつどう しゃりょう}◆【同】
\\	〔刑罰道具{けいばつ どうぐ}の〕ストック、足かせ◆19世紀半ばころまで使われた、公務妨害などの軽微な罪を犯した人間を罰するための道具。罪人は両足を足かせに通して決められた時間さらし者になったが、
\\	と違って手は自由に動かすことができた。1405年にはすべての町や村に一つはストックを置くことが決められた。
\\	《海事》造船台{ぞうせんだい}◆船を建造するときに船を支える枠組み。【形】手持ち{てもち}の陳腐{ちんぷ}な、平凡{へいぼん}なレベル3、発音
\\	、カナスタック、ストック、変化《動》
\\	変化形 
\\	【自動】混合{こんごう}する、結び付く合併{がっぺい}する、連合{れんごう}する《化学》化合物{かごうぶつ}になる〔コンバインで〕刈り取る、収穫{しゅうかく}する【他動】~を混ぜ合わせる、~を結び付ける《化学》〔複数{ふくすう}の物質{ぶっしつ}を〕化学結合{かがく けつごう}させる〔複数{ふくすう}の事柄{ことがら}を〕同時{どうじ}に行う、兼ね備える〔穀物{こくもつ}を〕コンバインで刈り取る[収穫{しゅうかく}する]【名】〔政治{せいじ}や経済団体{けいざい だんたい}の〕連合{れんごう}、合同{ごうどう}〔企業{きぎょう}の違法{いほう}な〕連合{れんごう}、カルテル〔農業機械{のうぎょうきかい}の〕コンバイン◆【同】
\\	レベル4、発音《名》
\\	《動》
\\	、カナカンバイン、コンバイン、変化《動》
\\	、分節
\\	親には大きな責任が伴う。 ・
\\	電源スイッチはサイドパネルに取り付けられている。	変化形 
\\	【自動】〔小さな物が大きな物に〕付着{ふちゃく}する、付随{ふずい}する、帰属{きぞく}する ・
\\	親には大きな責任が伴う。【他動】〔小さな物を大きな物に〕取り付ける、貼り付ける、添える、添付{てんぷ}する、加える ・
\\	電源スイッチはサイドパネルに取り付けられている。~に参加{さんか}させる、所属{しょぞく}させる、配属{はいぞく}する、配役{はいやく}する〔~に重きを〕置く《法律》〔財産などを〕差し押さえる◆【同】
\\	レベル3、発音
\\	、カナアタッシェ、アタッチ、変化《動》
\\	、分節
\\	変化形 
\\	【他動】~を分離{ぶんり}する、~を区分けする《医》〔感染者{かんせん しゃ}を〕隔離{かくり}する〔場所{ばしょ}を〕孤立{こりつ}させる、隔絶{かくぜつ}させる〔現象{げんしょう}から因子{いんし}を〕取り出す、探し出す《生物》〔ウイルスなどを〕分離{ぶんり}する《化学》〔混合物質{こんごう ぶっしつ}から要素{ようそ}を〕単離{たんり}する《電気》〔電気{でんき}・電子機器{でんし きき}を〕絶縁{ぜつえん}する【名】孤立{こりつ}[隔絶{かくぜつ}]している人[集団{しゅうだん}]《生物》分離株{ぶんり かぶ}◆研究室などで培養される微生物などを指す。《言語学》孤立言語{こりつ げんご}◆【同】
\\	レベル6、発音
\\	、カナアイソレイト、アイソレット、変化《動》
\\	、分節
\\	変化形 
\\	【他動】~を排除{はいじょ}する、締め出す~を除く、除外{じょがい}する、抜かす~の余地{よち}を与えない、~を全く許さない、考慮{こうりょ}しない、取り上げない◆【語源】外へ
\\	締め出す
\\	レベル5、発音
\\	、カナイクスクルード、エクスクルード、変化《動》
\\	、分節
\\	副流煙は主要な発がん物質として分類されるべきです。	変化形 
\\	【他動】~を分類{ぶんるい}する ・
\\	副流煙は主要な発がん物質として分類されるべきです。〔情報{じょうほう}・文書{ぶんしょ}・問題{もんだい}を〕機密扱いにする、秘密{ひみつ}にするレベル6、発音
\\	、カナクラシファイ、クラスィファイ、変化《動》
\\	、分節
\\	団結すれば栄え、分裂すれば倒れる。 ・
\\	《数式》30
\\	《数式》30
\\	《数式》16
\\	変化形 
\\	【自動】~に分かれる【他動】~を分ける、分割{ぶんかつ}する~と…を分離{ぶんり}する~を(…の間で)分け合う、分担{ぶんたん}する、~を…に分け与える~を分裂{ぶんれつ}させる、意見{いけん}を異にさせる ・
\\	団結すれば栄え、分裂すれば倒れる。~を…に分類{ぶんるい}する《数学》割る ・
\\	《数式》30
\\	《数式》30
\\	《数式》16
\\	2=8【名】分割{ぶんかつ}、分裂{ぶんれつ}見解{けんかい}の相違{そうい}《地学》分水嶺{ぶんすいれい}レベル2、発音
\\	、カナディバイドゥ、変化《動》
\\	、分節
\\	分配をして3
\\	-10を簡約化しなさい。 ・
\\	種の多様性は陸と海に均等に分布されているわけではない。 ・
\\	これらの仕事は異なるカテゴリーに区分される。 ・
\\	その商品を店に卸すために5台のトラックが使われた。 ・
\\	「ある
\\	は
\\	である」という命題においては、どの名辞も周延されていない。 ・
\\	かっこを展開するために
\\	をかっこ内のすべての項に分配しなければならない。	変化形 
\\	【自動】《数学》分配{ぶんぱい}する ・
\\	分配をして3
\\	-10を簡約化しなさい。【他動】~を分配{ぶんぱい}する、~を分け与える〔特定{とくてい}の地域{ちいき}や領域{りょういき}に〕~をばらまく、~を広げる、~を分布{ぶんぷ}させる ・
\\	種の多様性は陸と海に均等に分布されているわけではない。〔異なる種類{しゅるい}に〕~を区分{くぶん}する、~を割り振る ・
\\	これらの仕事は異なるカテゴリーに区分される。〔卸売業者{おろし うり ぎょうしゃ}が小売業者{こうり ぎょうしゃ}に商品{しょうひん}を〕卸す、流通{りゅうつう}させる ・
\\	その商品を店に卸すために5台のトラックが使われた。《論理学》〔名辞{めいじ}を〕周延{しゅうえん}する ・
\\	「ある
\\	は
\\	である」という命題においては、どの名辞も周延されていない。《数学》~を分配{ぶんぱい}する ・
\\	かっこを展開するために
\\	をかっこ内のすべての項に分配しなければならない。レベル4、発音
\\	、カナディストゥリビュート、ディストリビュート、変化《動》
\\	、分節
\\	あなた様からの季節[時候の挨拶{あいさつ}]のカードを受領[ちょうだい]致しました。	変化形 
\\	【他動】〔荷物などを〕運ぶ、搬送{はんそう}[運搬{うんぱん}]する◆【類】
\\	〔意思{いし}・情報{じょうほう}を〕伝達{でんたつ}する、伝える、告げる◆【類】
\\	あなた様からの季節[時候の挨拶{あいさつ}]のカードを受領[ちょうだい]致しました。〔財産{ざいさん}・権利{けんり}を〕譲渡{じょうと}する◆【類】
\\	レベル5、発音
\\	、カナコンベイ、変化《動》
\\	、分節
\\	彼は真の実行力のある指導者です。	変化形 
\\	【自動】出産{しゅっさん}する、子どもを産む配達{はいたつ}する約束通{やくそく どお}りに行う、職務{しょくむ}を遂行{すいこう}する[果たす]、実行力{じっこうりょく}がある ・
\\	彼は真の実行力のある指導者です。見解{けんかい}を述べる〔人が〕自由{じゆう}になる、〔物が人を〕開放的{かいほうてき}な気分{きぶん}にさせる【他動】〔商品{しょうひん}や手紙{てがみ}を〕配達{はいたつ}する、届ける、納品{のうひん}する、配信{はいしん}する、供給{きょうきゅう}する、送る、引き渡す、交付{こうふ}する〔考えなどを〕口に出して言う、述べる、発表{はっぴょう}する、言い渡す〔伝言{でんごん}などを〕伝える〔演説{えんぜつ}を〕する〔~を…から〕解放{かいほう}する、救い出す〔~を…に〕譲渡{じょうと}する、明け渡す分娩{ぶんべん}させる、産む《野球》〔ボールを〕投げる、打つ実現{じつげん}させる、業務{ぎょうむ}を遂行{すいこう}する〔政治家{せいじか}が公約{こうやく}を〕果たす、守るレベル2、発音
\\	、カナディリバ、デリバー、変化《動》
\\	、分節
\\	変化形 
\\	【自動】伝わる電波{でんぱ}で信号{しんごう}を送る、送信{そうしん}する【他動】送る、送信{そうしん}する、伝送{でんそう}する、発信{はっしん}する、放送{ほうそう}する、送金{そうきん}する〔情報などを〕伝える、伝達{でんたつ}する〔病気{びょうき}などを〕うつす、感染{かんせん}させる、伝染{でんせん}させるレベル6、発音
\\	、カナトランスミット、トランズミット、変化《動》
\\	、分節
\\	彼は毎週の公開練習で柔道の技量を披露しました。 ・
\\	サイコロを振って1の目が出る確率を数学的に実証しなさい。	変化形 
\\	【自動】デモ[実演{じつえん}・実例説明{じつれい せつめい}・示威運動{じい うんどう}]をする【他動】〔指導・説明・宣伝などのため〕実際にやってみせる、実演{じつえん}する、実地説明{じっち せつめい}をする、行動{こうどう}で示す ・
\\	彼は毎週の公開練習で柔道の技量を披露しました。〔実例{じつれい}によって〕はっきり示す、明らかにする、明示{めいじ}する論証{ろんしょう}する、実証{じっしょう}する、立証{りっしょう}する、証明{しょうめい}する ・
\\	サイコロを振って1の目が出る確率を数学的に実証しなさい。レベル4、発音
\\	、カナデモンストゥレイト、デモンストレイト、変化《動》
\\	、分節
\\	変化形 
\\	【自動】〔商品{しょうひん}などの〕宣伝{せんでん}[広告{こうこく}]をする〔求人{きゅうじん}などの〕広告{こうこく}[公募{こうぼ}]をする【他動】〔商品{しょうひん}などの〕宣伝{せんでん}[広告{こうこく}]をする〔求人{きゅうじん}などの〕広告{こうこく}を出す、公募{こうぼ}をする〔多くの人に〕~を公表{こうひょう}[通知{つうち}]するレベル5、発音
\\	、カナアドゥバタイズ、アドバタイズ、変化《動》
\\	、分節
\\	そのテレビ局は来週の木曜日にそのドキュメンタリーを放映する。 ・
\\	大方のニュース番組は6時と11時です。	変化形 
\\	【自動】〔テレビ・ラジオで〕放送{ほうそう}する広める、言い触らす、散布{さんぷ}する【他動】〔信号{しんごう}を〕送信{そうしん}する◆特定の受信機だけにではなく〔番組{ばんぐみ}を〕放送{ほうそう}する ・
\\	そのテレビ局は来週の木曜日にそのドキュメンタリーを放映する。〔うわさなどを〕広める、言い触らす【名】放送{ほうそう}、番組{ばんぐみ} ・
\\	大方のニュース番組は6時と11時です。散布{さんぷ}【形】放送{ほうそう}された、放送{ほうそう}の〔うわさが〕広められた【副】広くレベル2、発音
\\	、カナブロードゥキャスト、ブロードカスト、ブロードキャスト、変化《動》
\\	または
\\	、分節
\\	ボブが自白しても私は彼があの女性を殺したなんて信じない。 ・
\\	彼は有罪の評決が下った後でさえ、犯行を自白しなかった。 ・
\\	彼は州警察官に自分の犯した罪を自供しました。	変化形 
\\	【自動】〔自己{じこ}に不利益{ふりえき}な事実{じじつ}を〕認める ・
\\	ボブが自白しても私は彼があの女性を殺したなんて信じない。《宗》〔聖職者{せいしょくしゃ}などに〕罪を告白{こくはく}する、ゆるしの秘跡{ひせき}を求める、告解{こっかい}する、ざんげする【他動】〔過ち・罪などを〕告白{こくはく}[白状{はくじょう}・自供{じきょう}]する ・
\\	彼は有罪の評決が下った後でさえ、犯行を自白しなかった。 ・
\\	彼は州警察官に自分の犯した罪を自供しました。〔秘密などを〕打ち明ける◆【類】
\\	〔~を事実だと〕認める、自認{じにん}する◆【類】
\\	〔罪を神に〕ざんげするレベル4、発音
\\	、カナコンフェス、変化《動》
\\	、分節
\\	これから何が起こるかなんて誰にも分かるものか。 ・
\\	その報告書は、ポーランドの経済成長率が2%になると予測している。	変化形 
\\	【他動】~を予測{よそく}[予知{よち}・予報{よほう}・予言{よげん}・予見{よけん}]する ・
\\	これから何が起こるかなんて誰にも分かるものか。 ・
\\	その報告書は、ポーランドの経済成長率が2%になると予測している。レベル5、発音
\\	、カナプリディクト、プレディクト、変化《動》
\\	、分節
\\	《名言》予想したことはまず起こらない。起こるのは、大抵、まず予想していなかったことだ。 ・
\\	一体何人の経済アナリストが最近の日本の不景気を予測していただろうか。	変化形 
\\	【他動】~を予測{よそく}する、~を見込む{みこむ} ・
\\	《名言》予想したことはまず起こらない。起こるのは、大抵、まず予想していなかったことだ。 ・
\\	一体何人の経済アナリストが最近の日本の不景気を予測していただろうか。~を期待{きたい}して[楽しみに]待つ~を未然{みぜん}に防ぐ、~の機先{きせん}を制する〔負債{ふさい}を〕期限前{きげんまえ}に返済{へんさい}する〔幸福{こうふく}・破滅{はめつ}などを〕早める〔技術{ぎじゅつ}や思想{しそう}などで〕~に先んじる、先駆けて~に着手{ちゃくしゅ}する〔入手前{にゅうしゅ まえ}に資金{しきん}などを〕使う、利用{りよう}するレベル5、発音
\\	、カナアンティシペイト、アンテシペイト、変化《動》
\\	、分節
\\	そうだと推測します[思います]。 ・
\\	あなたは、このメールを部下たちに見せますよね。 ・
\\	このサイトを利用することによって発生する損害に関して、当社は一切の責任を負いません。 ・
\\	私は、いつも食事を終えたらすぐにいつもの場所に戻り、心地良い椅子に座って、パソコンの作業を続けます。 ・
\\	彼は最高執行責任者として新しく作られた役職に就任しました。	変化形 
\\	【他動】~と仮定{かてい}[想定{そうてい}・臆測{おくそく}・推測{すいそく}]する、思い込む、見なす、~を前提{ぜんてい}とする、当然{とうぜん}と思う ・
\\	そうだと推測します[思います]。 ・
\\	あなたは、このメールを部下たちに見せますよね。〔義務などを〕引き受ける〔役割{やくわり}などを〕担う〔責任などを〕負う ・
\\	このサイトを利用することによって発生する損害に関して、当社は一切の責任を負いません。〔債務を〕肩代わりする、引き継ぐ〔ある態度{たいど}を〕取る~を身につける、~を装う、~のふりをする〔姿勢{しせい}・体勢{たいせい}・ポーズなどを〕とる ・
\\	私は、いつも食事を終えたらすぐにいつもの場所に戻り、心地良い椅子に座って、パソコンの作業を続けます。〔任務{にんむ}・役職{やくしょく}・地位{ちい}などに〕就任{しゅうにん}する、就く ・
\\	彼は最高執行責任者として新しく作られた役職に就任しました。レベル4、発音
\\	、カナアシューム、アスーム、アスューム、変化《動》
\\	、分節
\\	彼はうなずいて肯定を示しました。 ・
\\	彼はうなずいて同意を示しました。	変化形 
\\	【自動】首を縦にふる、うなずく、会釈{えしゃく}する、相づちを打つ、うなずいて承諾{しょうだく}[命令{めいれい}]する眠くてうとうと[こっくり]する、舟を漕ぐ〈俗〉(麻薬{まやく}で意識{いしき}が)もうろうとしている【他動】うなずいて~を示す ・
\\	彼はうなずいて肯定を示しました。 ・
\\	彼はうなずいて同意を示しました。【名】うなずき、〔同意{どうい}の印として〕首を縦に振ること同意{どうい}、賛同{さんどう}レベル2、発音
\\	、カナノッド、変化《動》
\\	変化形 
\\	【間投】ハァ、やれやれ、タメイキ◆ため息の音【自動】ため息をつく〔ため息のような〕音を立てる[出す]切望{せつぼう}する、待ち望む【名】ため息ため息のような音レベル3、発音
\\	、カナサイ、変化《動》
\\	変化形 
\\	【自動】ささやく【他動】~をささやく、耳打ちする【名】ささやき声レベル4、発音
\\	、カナウィスパー、ホウィスパ、変化《動》
\\	、分節
\\	叫んだら殺すぞ! ・
\\	救急車のサイレンが鳴り響く。	変化形 
\\	【自動】鋭い叫び声を上げる、金切り声を出す、悲鳴{ひめい}を上げる、キャッ[キャー]と叫ぶ ・
\\	叫んだら殺すぞ!鳴る ・
\\	救急車のサイレンが鳴り響く。派手{はで}に宣伝{せんでん}して回る【他動】~を絶叫{ぜっきょう}する【名】悲鳴{ひめい}、叫び、甲高い笑い声とても愉快{ゆかい}な[面白い]人[こと・もの・ジョーク]レベル2、発音
\\	、カナスクリーム、変化《動》
\\	けちをつけるつもりはない。 ・
\\	けんかするには二人が必要です。/二人いなければけんかはできません。/けんか両成敗。	変化形 
\\	【自動】けんかする、口論{こうろん}する、争う苦情{くじょう}を言う、異議{いぎ}を唱える ・
\\	けちをつけるつもりはない。【名】〔長期間にわたる〕けんか、口論{こうろん}、言い争い、押し問答、もめ事、紛争{ふんそう}、もんちゃく、すったもんだ、いざこざ、仲たがい、反目{はんもく}、不和{ふわ} ・
\\	けんかするには二人が必要です。/二人いなければけんかはできません。/けんか両成敗。けんかの原因{げんいん}[言い分]レベル3、発音
\\	、カナクウォーレル、クオレル、変化《動》
\\	、分節
\\	私があなたに警告しなかったとは言わせないわ。	変化形 
\\	【他動】警告{けいこく}する、注意{ちゅうい}するように言う、注意{ちゅうい}する、通告{つうこく}する、予告{よこく}する、用心{ようじん}させる、警鐘{けいしょう}を鳴らす ・
\\	私があなたに警告しなかったとは言わせないわ。~に通知{つうち}するレベル2、発音
\\	、カナウォーン、変化《動》
\\	変化形 
\\	【他動】~をびっくりさせる、心配{しんぱい}させる、怖がらせる~に警報{けいほう}を鳴らす[発する・伝える]、危険{きけん}を知らせる、警戒{けいかい}させる~に警報器{けいほうき}を取り付ける【名】警報{けいほう}、警告{けいこく}、危険{きけん}を知らせる声[音]アラーム、警報器{けいほうき}、目覚まし時計{とけい}◆【略】
\\	〔危険を認識したことによって起こる〕恐怖{きょうふ}、驚き、警戒{けいかい}、懸念{けねん}、不安{ふあん}〔軍隊{ぐんたい}の〕招集命令{しょうしゅう めいれい}◆【同】
\\	レベル3、発音
\\	、カナアラーム、変化《動》
\\	、分節
\\	変化形 
\\	【自他動】叱る、説教{せっきょう}する、小言{こごと}を言う◆【類】
\\	【名】ガミガミ女レベル4、発音
\\	、カナスコールド、スコウルド、変化《動》
\\	彼女は私を説得して日曜日の手伝いに駆り出そうとしている。 ・
\\	そんなことで誰を説得できるとも思わない。 ・
\\	彼女を説得する自信がある。 ・
\\	説明させてくれませんか? ・
\\	彼は彼女に結婚を迫った。 ・
\\	彼には金がものをいう。	変化形 
\\	【他動】説得{せっとく}する、説得{せっとく}して…させる、説き伏せる、促す、促して…させる、納得{なっとく}させる ・
\\	彼女は私を説得して日曜日の手伝いに駆り出そうとしている。 ・
\\	そんなことで誰を説得できるとも思わない。 ・
\\	彼女を説得する自信がある。 ・
\\	説明させてくれませんか? ・
\\	彼は彼女に結婚を迫った。口説く{くどく}、口説き落とす{くどきおとす}、勧誘{かんゆう}する、丸め込む、信じ込ませる ・
\\	彼には金がものをいう。レベル4、発音
\\	、カナパースウェイド、パースエイド、パスウェイド、変化《動》
\\	、分節
\\	降伏するのが一番だと彼を説得してみるつもりだ。	変化形 
\\	【他動】確信{かくしん}させる、納得{なっとく}させる、説得{せっとく}する ・
\\	降伏するのが一番だと彼を説得してみるつもりだ。レベル4、発音
\\	、カナコンビンス、変化《動》
\\	、分節
\\	彼は引き返すことに決めた。 ・
\\	彼はアメリカへ留学しようと決心しました。 ・
\\	彼女は画家になろうと志を立ててフランスに渡った。 ・
\\	「猫が側にいると、いつも目がかゆくなるんです」「そうですか、それならこれを試してみてください。アレルギーの悩みが解決するはずですよ」 ・
\\	声明の中で彼らは麻薬の害を根絶するために効果のある国際的行動を続けるという決意を表明しました。 ・
\\	やっと会社を辞める決心が固まった。
\\	変化形 
\\	"【自動】決心{けっしん}する、決定{けってい}する、決議{けつぎ}する、解決{かいけつ}する ・
\\	彼は引き返すことに決めた。分解{ぶんかい}する、還元{かんげん}する、帰着{きちゃく}する《医》散る、消散{しょうさん}する《音楽》(不協和音が)解決{かいけつ}する、協和音{きょうわおん}になる《法律》無効{むこう}となる、消滅{しょうめつ}する、消失{しょうしつ}する【他動】〔~しようと〕決心{けっしん}[決意{けつい}・決定{けってい}]する ・
\\	彼はアメリカへ留学しようと決心しました。 ・
\\	彼女は画家になろうと志を立ててフランスに渡った。(人)に決心{けっしん}[決意{けつい}]させる~することを決議{けつぎ}[議決{ぎけつ}・投票{とうひょう}によって決定{けってい}]する〔~を構成要素{こうせい ようそ}に〕分解{ぶんかい}する、還元{かんげん}する〔~を別のものに〕変化{へんか}させる、転換{てんかん}させる〔問題{もんだい}や困難{こんなん}を〕解決{かいけつ}する、解消{かいしょう}する ・
\\	「猫が側にいると、いつも目がかゆくなるんです」「そうですか、それならこれを試してみてください。アレルギーの悩みが解決するはずですよ」〔疑いや心配{しんぱい}を〕晴らす、拭{ぬぐ}い去る〔不一致{ふいっち}や争いを〕決着{けっちゃく}させる、解決{かいけつ}する《病理》〔炎症{えんしょう}などを〕消散{しょうさん}させる《音楽》〔非和声音{ひわせいおん}を和声音{わせい おん}に〕解決{かいけつ}する《化学》〔ラセミ体を光学活性成分{こうがく かっせい せいぶん}に〕(光学{こうがく})分割{ぶんかつ}する《数学》〔ベクトルを成分{せいぶん}に〕分解{ぶんかい}する〔画像{がぞう}を〕解像{かいぞう}する【名】〔~しようとする〕決意{けつい}、決心{けっしん}、決断{けつだん} ・
\\	声明の中で彼らは麻薬の害を根絶するために効果のある国際的行動を続けるという決意を表明しました。 ・
\\	やっと会社を辞める決心が固まった。〔人の〕目的意識{もくてき いしき}、決断力{けつだんりょく}、意思{いし}〔審議機関{しんぎ きかん}が下す〕正式{せいしき}な決議{けつぎ}レベル4、発音
\\	、カナリザルブ、リゾルブ、変化《動》
\\	、分節
\\	「それでこの車、過去に問題は全然なかったって?」「その通りです。それは保証しますよ」 ・
\\	本書の指示通りに操作すれば、問題なく安全にこの機械を操作することができます。
\\	変化形 
\\	"【他動】〔人に〕~であると断言{だんげん}する、~であることを請け合う
\\	節の内容が確かであることを伝え、その内容に対する相手の疑念を取り除く。 ・
\\	「それでこの車、過去に問題は全然なかったって?」「その通りです。それは保証しますよ」〔人に〕~を保証{ほしょう}する、~を確信{かくしん}させる
\\	句の内容を相手に確信させる。~を確実{かくじつ}にする◆【同】
\\	本書の指示通りに操作すれば、問題なく安全にこの機械を操作することができます。〈英〉~に保険{ほけん}を掛ける◆確実に起きるもの、例えば「死亡」などに対して保険を掛けることを指す。レベル4、発音
\\	、カナアシュア、変化《動》
\\	、分節
\\	いますぐ彼に電話するよう、あなたに強く勧めます。 ・
\\	諮問委員会は政府に対し日本の経済構造を変えることを強く要請しました。 ・
\\	医者は私に禁煙を勧めた。 ・
\\	アメリカは日本に対し防衛計画を変更することを強く要求しました。 ・
\\	彼らは政府に安定した雇用を保証するよう要求しました。 ・
\\	国連事務総長はモザンビークでの国連平和維持活動に日本も参加するよう促しました。 ・
\\	大統領は何度もテロに対する忍耐を呼び掛けた。 ・
\\	彼女は契約書に署名することを迫られた。	変化形 
\\	【自動】刺激する、駆り立てる◆空腹・恐怖などが~するように【他動】〔人に~するよう強く〕促す、要請{ようせい}する、勧める ・
\\	いますぐ彼に電話するよう、あなたに強く勧めます。 ・
\\	諮問委員会は政府に対し日本の経済構造を変えることを強く要請しました。 ・
\\	医者は私に禁煙を勧めた。 ・
\\	アメリカは日本に対し防衛計画を変更することを強く要求しました。 ・
\\	彼らは政府に安定した雇用を保証するよう要求しました。 ・
\\	国連事務総長はモザンビークでの国連平和維持活動に日本も参加するよう促しました。〔~を真剣{しんけん}に繰り返し〕呼び掛ける、奨励{しょうれい}する ・
\\	大統領は何度もテロに対する忍耐を呼び掛けた。〔人に~するよう〕強いる、せき立てる ・
\\	彼女は契約書に署名することを迫られた。〈文〉〔人を〕興奮{こうふん}させる、激励{げきれい}する【名】〔~しようとする〕衝動{しょうどう}、本能{ほんのう}レベル4、発音
\\	、カナアージ、変化《動》
\\	もし症状が8日間以上続いてなくならない場合はこの製品の使用を止めてください。	変化形 
\\	【自動】しつこく主張{しゅちょう}する、言い張る〔困難{こんなん}であるにもかかわらず〕貫き通す、あくまでやり通す存続{そんぞく}する、持続{じぞく}する、続く、生き残る ・
\\	もし症状が8日間以上続いてなくならない場合はこの製品の使用を止めてください。〔疑問などが〕根強く残るレベル6、発音
\\	、カナパーシスト、パスィスト、変化《動》
\\	、分節
\\	変化形 
\\	【1自動】(差し)控える、自制{じせい}する、こらえる、やめる【2名】〔詩のスタンザの最後{さいご}の〕反復句{はんぷく く}《音楽》〔歌詞{かし}の〕反復句{はんぷく く}、リフレイン《音楽》旋律{せんりつ}、メロディー口癖{くちぐせ}の言葉{ことば}、繰り返される話レベル7、発音
\\	、カナリフレイン、変化《動》
\\	、分節
\\	長引く景気後退で歳入が減少している。 ・
\\	誠に残念ながらご辞退申し上げます。 ・
\\	なぜ株式市場があれほど大幅に下落したのか。 ・
\\	ローマ帝国は豊かで強力になり過ぎたとき、衰退し始めた。 ・
\\	ご親切にもご招待いただきながら、今回はご辞退申し上げなければなりませんが、事情をご理解ください。 ・
\\	若者の労働力が250万人ほど減った。	変化形 
\\	【自動】〔数量{すうりょう}が〕減る、減少{げんしょう}する ・
\\	長引く景気後退で歳入が減少している。〔申し出などに対して丁重に〕断る、辞退{じたい}する、謝絶{しゃぜつ}する ・
\\	誠に残念ながらご辞退申し上げます。〔道などが〕下る、下に傾く〔太陽{たいよう}などが〕沈む、欠ける〔力などが〕衰退{すいたい}する、低下{ていか}する ・
\\	なぜ株式市場があれほど大幅に下落したのか。 ・
\\	ローマ帝国は豊かで強力になり過ぎたとき、衰退し始めた。〔人が〕落ちぶれる、身を落とす《文法》語形{ごけい}[格]変化{へんか}する【他動】〔申し出・招待などを丁重に〕断る、辞退{じたい}する、謝絶{しゃぜつ}する、拒否{きょひ}する ・
\\	ご親切にもご招待いただきながら、今回はご辞退申し上げなければなりませんが、事情をご理解ください。~を下向きに曲げる[傾ける]《文法》〔語を〕語形{ごけい}[格]変化{へんか}させる【名】〔緩やかな〕下落{げらく}、減少{げんしょう}、減退{げんたい}、低落{ていらく}、落ち込み ・
\\	若者の労働力が250万人ほど減った。終末{しゅうまつ}、最終局面{さいしゅう きょくめん}、晩年{ばんねん}下り坂{くだりざか}、下りこう配
\\	《証券》値下{ねさ}がり株レベル4、発音
\\	、カナディクライン、デクライン、変化《動》
\\	、分節
\\	床をゴシゴシこすっていると、気分がスーッとする。	変化形 
\\	【自動】〔光・音などが〕消えていく、弱まる〔色などが〕薄くなる、薄れる、あせる色あせる〔力などが〕衰える、勢いがなくなる ・
\\	床をゴシゴシこすっていると、気分がスーッとする。〔植物が〕しおれる【他動】~の色をあせさせる〔力などを〕衰えさせる〔植物を〕しおれさせる【名】《ゴルフ》フェイド(ボール)《映画》フェードレベル4、発音
\\	、カナフェイド、変化《動》
\\	アリスティドはさらに11時間ためらった。 ・
\\	さらに情報が必要な場合は、遠慮なくご連絡ください。 ・
\\	二の足を踏む者は敗北する/ためらう者は敗れる。/ためらう者は機会を逃す。◆ことわざ ・
\\	私は全くためらわなかった。 ・
\\	良かれと思われることを行うのをためらうべからず。/善は急げ。	変化形 
\\	【自動】ためらう、ちゅうちょする、遠慮{えんりょ}する、気後れする、はばかる、迷う、もじもじする、ぐずぐずする、二の足を踏む、気が進まない、口ごもる、気兼ねする、(~することに)抵抗{ていこう}を感じる、(~することに)難色{なんしょく}を示す ・
\\	アリスティドはさらに11時間ためらった。 ・
\\	さらに情報が必要な場合は、遠慮なくご連絡ください。 ・
\\	二の足を踏む者は敗北する/ためらう者は敗れる。/ためらう者は機会を逃す。◆ことわざ ・
\\	私は全くためらわなかった。 ・
\\	良かれと思われることを行うのをためらうべからず。/善は急げ。レベル4、発音
\\	、カナヘジテイト、ヘズィテイト、変化《動》
\\	、分節
\\	異なる電荷は引き付け合う。 ・
\\	重力は物体を引き付ける。 ・
\\	どうしてこんなぼろぼろの車買ったの?これ走りさえもしないじゃないの!」「そんなこと言うなよ!ただ僕より古いものが欲しかったんだよ!」 ・
\\	どういうタイプの男性に引かれ[魅力を感じ]ますか?
\\	変化形 
\\	"【自動】引き付ける、引力がある ・
\\	異なる電荷は引き付け合う。魅力がある【他動】引く、引き込む、引き付ける ・
\\	重力は物体を引き付ける。魅惑{みわく}する、魅了{みりょう}する ・
\\	どうしてこんなぼろぼろの車買ったの?これ走りさえもしないじゃないの!」「そんなこと言うなよ!ただ僕より古いものが欲しかったんだよ!」 ・
\\	どういうタイプの男性に引かれ[魅力を感じ]ますか?呼び込む、招く、誘致{ゆうち}するレベル3、発音
\\	、カナアトゥラクト、アトラクト、変化《動》
\\	、分節
\\	変化形 
\\	【他動】~を魅惑{みわく}する、魅する、悩殺{のうさつ}する、うっとりさせる、(人)の心を捉えるレベル6、発音
\\	、カナファシネイト、ファッシネイト、変化《動》
\\	、分節
\\	その大統領は全国民から尊敬の念を受けた。 ・
\\	変化とは、引き起こされるものであって、課するものではない。	変化形 
\\	【自動】元気{げんき}[霊感{れいかん}・ひらめき]を与える〔肺に〕吸入{きゅうにゅう}する、吸い込む【他動】(人)に霊感{れいかん}[着想{ちゃくそう}・ひらめき]を与える(人)を元気{げんき}づける、(人)を明るくさせる~を動機付ける、~に刺激{しげき}を与える〔感情{かんじょう}などを〕引き出す、呼び起こす ・
\\	その大統領は全国民から尊敬の念を受けた。〔気体{きたい}を肺に〕吸入{きゅうにゅう}する~の原因{げんいん}となる、~を引き起こす ・
\\	変化とは、引き起こされるものであって、課するものではない。〈古〉~に命を吹き込むレベル4、発音
\\	、カナインスパイア、変化《動》
\\	、分節
\\	警察は、その近隣で複数の家族が失踪{しっそう}した事件について、捜査を開始したところだ。 ・
\\	その会社は来月、新商品を発売する予定です。 ・
\\	その企業は英国での多チャンネル衛星放送の開始を発表しました。	変化形 
\\	【1自動】〔急に〕始める、乗り出す【1他動】〔ロケットを〕打ち上げる、〔ミサイルを〕発射{はっしゃ}する、〔矢を〕放つ《野球》~を打つ〔船を〕進水{しんすい}させる、水に浮かべる始める、開始{かいし}する、着手{ちゃくしゅ}する、乗り出す、参入{さんにゅう}する◆目的語には、調査・販売・キャンペーンなど明確な目的・意図を持って行われる活動が使われる。 ・
\\	警察は、その近隣で複数の家族が失踪{しっそう}した事件について、捜査を開始したところだ。〔新商品を〕売り出す、発売する ・
\\	その会社は来月、新商品を発売する予定です。《コ》〔プログラムを〕立ち上げる、起動{きどう}する【1名】〔ロケットの〕打ち上げ、〔ミサイルの〕発射{はっしゃ}、〔船の〕進水{しんすい}〔活動などの〕開始{かいし}、着手{ちゃくしゅ}、立ち上げ ・
\\	その企業は英国での多チャンネル衛星放送の開始を発表しました。【2名】《海事》〔軍艦{ぐんかん}の〕ランチ◆艦載されているボートで最大のもの。◆【参考】
\\	《海事》ランチ◆大型でデッキのないモーターボートレベル4、発音
\\	、カナラーンチ、ランチ、ローンチ、変化《動》
\\	君は空々しいうそをつくつもりなのかい。 ・
\\	彼はあえてそうすべきか、悩んでいる。◆間接疑問の中でも用いられる ・
\\	どうしてそんな口の聞き方をするんだい。父親に対する尊敬の念は一体どこにあるんだね。
\\	の形をとり、怒りや苛立{いらだ}ちを表すことがある ・
\\	あえて君の両親にはまだ伝えていないよ。 ・
\\	家族の健康のためにどれくらいお金を使うかあえて考えようとはしません。 ・
\\	彼女はそこにいる彼にあえて話し掛けようとはしなかった。見ることすらほとんどなかったのだ。
\\	の過去形 ・
\\	引っぱたけるものなら引っぱたいてごらんなさいよ。
\\	節の中でも用いられる ・
\\	もし彼に勇気があれば、私を今この場で殺すでしょう。 ・
\\	「そうよ、そうよ」と彼女は答えた。
\\	と1語で書くこともある ・
\\	父親が懸念しているのは、彼女が音楽家になりたがっているかもしれないということではないでしょうか。 ・
\\	おやまあ、今日はずいぶんとかわいいですね。◆幾分古びて、滑稽{こっけい}な表現に響くこともある ・
\\	その政治家は大胆にも新しい憲法解釈を試みた。 ・
\\	彼は振り向くことなく走り去った。
\\	となる場合、
\\	の省略は不可。 ・
\\	私にとっては望むべくもない[期待するのもおこがましい]ことだった。 ・
\\	あなたは聖職に就く気はありますか。◆否定文・疑問文で
\\	の
\\	が消えることがある ・
\\	自分の家でパーティーをするつもりはないよ。 ・
\\	やめなさい!/やめてくれ!/とんでもない!/ばか言え!◆人の行動をやめさせるとき ・
\\	個人的なことに干渉するのはやめてくれないかな。◆禁止・命令を表す ・
\\	彼は私が木から飛び降りてみろと言うと、本当に飛び降りた。
\\	変化形 
\\	"【助動】あえて[思い切って・恐れずに・平気{へいき}で・大胆{だいたん}にも]~する、~する勇気{ゆうき}がある◆通例否定文・疑問文 ・
\\	君は空々しいうそをつくつもりなのかい。 ・
\\	彼はあえてそうすべきか、悩んでいる。◆間接疑問の中でも用いられる ・
\\	どうしてそんな口の聞き方をするんだい。父親に対する尊敬の念は一体どこにあるんだね。
\\	の形をとり、怒りや苛立{いらだ}ちを表すことがある ・
\\	あえて君の両親にはまだ伝えていないよ。 ・
\\	家族の健康のためにどれくらいお金を使うかあえて考えようとはしません。 ・
\\	彼女はそこにいる彼にあえて話し掛けようとはしなかった。見ることすらほとんどなかったのだ。
\\	の過去形 ・
\\	引っぱたけるものなら引っぱたいてごらんなさいよ。
\\	節の中でも用いられる ・
\\	もし彼に勇気があれば、私を今この場で殺すでしょう。多分{たぶん}~でしょう、きっとそうだ
\\	の形をとる ・
\\	「そうよ、そうよ」と彼女は答えた。
\\	と1語で書くこともある ・
\\	父親が懸念しているのは、彼女が音楽家になりたがっているかもしれないということではないでしょうか。言わせてもらえれば
\\	の形をとる ・
\\	おやまあ、今日はずいぶんとかわいいですね。◆幾分古びて、滑稽{こっけい}な表現に響くこともある【他動】あえて[思い切って・恐れずに・平気{へいき}で・大胆{だいたん}にも]~する、~する勇気{ゆうき}がある、~に敢然{かんぜん}と立ち向かう、~をやってみる ・
\\	その政治家は大胆にも新しい憲法解釈を試みた。 ・
\\	彼は振り向くことなく走り去った。
\\	となる場合、
\\	の省略は不可。 ・
\\	私にとっては望むべくもない[期待するのもおこがましい]ことだった。 ・
\\	あなたは聖職に就く気はありますか。◆否定文・疑問文で
\\	の
\\	が消えることがある ・
\\	自分の家でパーティーをするつもりはないよ。 ・
\\	やめなさい!/やめてくれ!/とんでもない!/ばか言え!◆人の行動をやめさせるとき ・
\\	個人的なことに干渉するのはやめてくれないかな。◆禁止・命令を表す〔人に〕~するよう挑む[けしかける]【名】挑戦{ちょうせん}、大胆{だいたん}◆【参考】
\\	彼は私が木から飛び降りてみろと言うと、本当に飛び降りた。〔薄気味悪い場所{うすきみわるい ばしょ}に行くなどの〕肝試し{きもだめし}レベル2、発音
\\	、カナデア、変化《動》
\\	《株》ダウ下がり、ナスダック上昇。
\\	と
\\	の語呂合わせになっている。 ・
\\	自分の馬にそのハードルを越えさせようとしてみた。	変化形 
\\	【自動】ピョンと飛ぶ、飛び跳ねる、跳ぶ、飛び越える
\\	では跳躍の「動作」に、
\\	では跳躍による「移動」に重点が置かれる ・
\\	《株》ダウ下がり、ナスダック上昇。
\\	と
\\	の語呂合わせになっている。〔突然または素早く〕跳ぶように動く急に[突然{とつぜん}](~の)状態{じょうたい}になる急上昇{きゅうじょうしょう}する、急増{きゅうぞう}する〔胸が〕高鳴る{たかなる}、踊る、ドキドキする【他動】~を飛び越える~に飛び越えさせる ・
\\	自分の馬にそのハードルを越えさせようとしてみた。【名】〔高い〕跳躍{ちょうやく}、ジャンプ跳躍{ちょうやく}[ジャンプ]の距離{きょり}飛び越える[飛び出す]場所{ばしょ}急激{きゅうげき}な増加{ぞうか}[上昇{じょうしょう}]急激{きゅうげき}な変化{へんか}[推移{すいい}]《音楽》跳躍進行{ちょうやく しんこう}◆次の音階が3度以上開いていること。レベル4、発音
\\	、カナリープ、変化《動》
\\	または
\\	変化形 
\\	【他動】~に傷をつける、~を引っかく、〔車体{しゃたい}などを〕(…に)こする~を書き殴る、~を走り書きする~を抹消{まっしょう}する◆書き間違えた単語などを、上に線を引いて消すこと《野球》~をラインアップから外す【名】かすり傷、引っかくこと、引っかき傷、擦り傷、擦過傷{さっかしょう}〔棒で地面を引っかいて描いた〕スタートライン〈俗〉お金、先立つ物〈米話〉《野球》ポテンヒット、内野安打{ないや あんだ}◆悪い当たりでヒットになることレベル4、発音
\\	、カナスクラッチ、変化《動》
\\	このソファは折り畳んでベッドになる。 ・
\\	彼は明らかに失敗している。 ・
\\	われわれは破産しないだろう。 ・
\\	私は服を全部たたんだ。 ・
\\	しわや折り目のある用紙を使用しないでください。◆プリンタの説明書	変化形 
\\	【1自動】折り重なる、折り畳める ・
\\	このソファは折り畳んでベッドになる。店を畳む、〔店・事業{じぎょう}などが〕つぶれる、破産{はさん}する、失敗{しっぱい}する、おしまいになる、〔雑誌{ざっし}などが〕廃刊{はいかん}になる ・
\\	彼は明らかに失敗している。 ・
\\	われわれは破産しないだろう。へこたれる負けを認める、降参{こうさん}する〔カードゲームなどで〕フォールドする、賭け[勝負{しょうぶ}]から降りる【1他動】~を収める、折り畳む、折り曲げる、折り重ねる、折り返す、折る ・
\\	私は服を全部たたんだ。~を組む、組み合わせる~を(…に)巻き付ける~を両腕{りょううで}に抱く、抱える、胸に抱き寄せる~を包む、くるむ、まとう、覆う【1名】折ること、畳むこと折った[畳んだ]個所{かしょ}[部分{ぶぶん}]折り目、ひだ、しわ ・
\\	しわや折り目のある用紙を使用しないでください。◆プリンタの説明書〔ヘビやロープなどの〕一巻き《地学》〔地層{ちそう}の〕褶曲{しゅうきょく}〈英〉〔丘陵地{きゅうりょう ち}の〕小さな谷《解剖》ひだ◆【同】
\\	〔タンパク質の〕折り畳み構造【2他動】〔家畜{かちく}を〕囲う【2名】〔家畜{かちく}などを入れる〕おり、囲い羊の群れ〔強い絆{きずな}で結ばれた〕集団{しゅうだん}、組織{そしき}◆共通の信念や目的を持つ集団や組織。《キリスト教》会衆{かいしゅう}、信徒{しんと}レベル4、発音
\\	、カナフォールド、フォウルド、変化《動》
\\	そのボートは海岸線沿いに進んだ。 ・
\\	その道路は川岸に沿って延びている。 ・
\\	パパを抱き締めてくれ。	変化形 
\\	【他動】~を抱き締める~に凝っている~に沿って進む ・
\\	そのボートは海岸線沿いに進んだ。 ・
\\	その道路は川岸に沿って延びている。【名】抱擁{ほうよう} ・
\\	パパを抱き締めてくれ。レベル5、発音
\\	、カナハッグ、変化《動》
\\	変化形 
\\	【自動】しっかりつかむ【他動】~を握る、~をしっかりつかむ~を引き付ける【名】しっかりつかむこと、握りかた、握力{あくりょく}つかみ部、握り◆【同】
\\	支配(力){しはい(りょく)}、制御(力){せいぎょ(りょく)}把握{はあく}、理解力{りかいりょく}〔小さな〕旅行{りょこう}かばん、スーツケース◆【同】
\\	レベル3、発音
\\	、カナグリップ、変化《動》
\\	折れるよりは曲がった方がいい。/壊れるよりはたわめ。◆ことわざ◆強風に逆らって折れるよりは、柳や竹のように曲がった方がいい。つまり、時には柔軟な姿勢が必要だということ。
\\	とも言う。いずれも
\\	の音が頭韻を踏んでいる。 ・
\\	折り曲げ厳禁。	変化形 
\\	【1自動】曲がる、カーブする、たわむ、屈服{くっぷく}する ・
\\	折れるよりは曲がった方がいい。/壊れるよりはたわめ。◆ことわざ◆強風に逆らって折れるよりは、柳や竹のように曲がった方がいい。つまり、時には柔軟な姿勢が必要だということ。
\\	とも言う。いずれも
\\	の音が頭韻を踏んでいる。【1他動】〔物・意志{いし}を〕曲げる、〔心を〕~に傾ける[向ける] ・
\\	折り曲げ厳禁。【1名】曲げる[曲がっている]こと、屈曲{くっきょく}曲げられたもの、〔道路{どうろ}の〕カーブ《海事》〔二つのロープの〕結び目
\\	《海事》舷側厚板{げんそく あついた}◆【同】
\\	〈話〉ケイソン病、減圧痛{げんあつ つう}、減圧病{げんあつびょう}、潜水病{せんすいびょう}、潜函病{せんかんびょう}◆【用法】単数または複数扱いベンド(奏法{そうほう})◆演奏中に舌や口の位置、息の強さを変えて音程を変えるブルースハープの演奏法【2名】〔紋章の〕盾形の右上から左下にかかる帯の模様◆【対】
\\	レベル2、発音
\\	、カナベンド、変化《動》
\\	変化形 
\\	【自動】〔ひもなどで〕縛る、結び付ける〔人が〕結び付く、団結{だんけつ}する〔要素{ようそ}が〕固まる、凝固{ぎょうこ}する〔衣服{いふく}が〕きつい、動きにくい《機械》〔部品{ぶひん}が〕固着{こちゃく}する《化学》〔物質{ぶっしつ}が〕化学結合{かがく けつごう}する〔法や約束{やくそく}などが〕強制{きょうせい}する、義務{ぎむ}を負わせる【他動】〔ひもなどで~を固く〕縛る、結び付ける〔ベルトなどで〕~を巻く[くるむ]〔けがなどに〕包帯{ほうたい}をする[巻く]〔人の手足{てあし}を〕縛り上げる〔生地{きじ}などに〕縁飾り{ふちかざり}をする〔本などを〕製本{せいほん}する、装丁{そうてい}する〔要素{ようそ}を〕くっつけて固める、凝固{ぎょうこ}させる〔人を〕結び付ける、団結{だんけつ}させる〔法や約束{やくそく}などで人に〕~する義務{ぎむ}を負わせる、強制{きょうせい}して~させる◆【用法】受け身で用いられることが多い。〔契約{けいやく}などを〕取り消せないようにする〔人を〕年季奉公{ねんき ぼうこう}で働かせる《化学》〔物質{ぶっしつ}を〕化学結合{かがく けつごう}させる《医》〔下痢{げり}の人や腸を〕便秘{べんぴ}にする【名】〔ひもなどで〕縛ること、結び付けること縛る[結び付ける]もの〈話〉困難{こんなん}な状況{じょうきょう}、苦境{くきょう}《音楽》結合線{けつごうせん}◆スラー
\\	やブレース
\\	など。レベル3、発音
\\	、カナバインド、変化《動》
\\	私の体重、いくらか分かる? ・
\\	このモーターは10
\\	である。 ・
\\	赤身の肉を食べることの結果をよく考えなくてはならない。それは心臓に関するさまざまな問題を引き起こす。	変化形 
\\	【自動】~の重さがある ・
\\	私の体重、いくらか分かる? ・
\\	このモーターは10
\\	である。重荷{おもに}になる重要{じゅうよう}である出航{しゅっこう}する【他動】~の重さ[体重{たいじゅう}・目方{めかた}]を量る、~をはかりにかける、~を秤量{ひょうりょう}する比較{ひかく}検討{けんとう}[考察{こうさつ}]する、評価{ひょうか}する、熟考{じゅっこう}する、よく考える、品定めする、較量{こうりょう}する ・
\\	赤身の肉を食べることの結果をよく考えなくてはならない。それは心臓に関するさまざまな問題を引き起こす。レベル2、発音
\\	、カナウェイ、ウエイ、変化《動》
\\	日没で試合が延期された。	変化形 
\\	【自動】ぶらさがる一時的{いちじてき}に止める、一時停止{いちじ ていし}する、一時中断{いちじ ちゅうだん}する、(途中{とちゅう})停止{ていし}する、延期{えんき}する、保留{ほりゅう}する【他動】つるす、下げる、浮遊{ふゆう}させる、浮かせる一時的{いちじてき}に止める、一時停止{いちじ ていし}する、一時中断{いちじ ちゅうだん}する、途中停止{とちゅう ていし}する、延期{えんき}する、保留{ほりゅう}する、〔出版物{しゅっぱんぶつ}を〕差し止める ・
\\	日没で試合が延期された。~を停職{ていしょく}(処分{しょぶん})にする、~に停学{ていがく}を命じる《生化学》~をけん濁する【名】《コ》サスペンド、一時停止{いちじ ていし}◆コンピューターの電源を切る直前の状態をメモリに記憶させて、省電力モードに移行してその記憶内容を保持しておき、一時的にコンピューターの使用を中断し、次に電源を入れた時に電源を切る直前の状態から作業を再開できる機能。レベル5、発音
\\	、カナサスペンド、変化《動》
\\	、分節
\\	頭が割れるように痛い。 ・
\\	さあ、早く行こう。 ・
\\	割り勘にしよう。
\\	取締役会において普通株式1株を2株に分割することが承認された。	変化形 
\\	【自動】〔物が〕割れて二つに分かれる、バラバラになる、裂ける〔人が〕分かれる、交際をやめる、離婚する〔頭が〕割れるように痛む ・
\\	頭が割れるように痛い。その場から去る[出る・帰る]、素早く去る
\\	さあ、早く行こう。【他動】〔目にそって〕~を裂く[割る]〔力ずくで〕~を引き裂く[バラバラに壊す]〔グループなどを〕分ける、分派する分け合う、分配{ぶんぱい}する ・
\\	割り勘にしよう。
\\	〔全体を部分や段階に〕分ける、分割する〔異なる政党の候補者に〕投票する〔株式を〕分割する〔ダブル・ヘッダーやシリーズ戦の〕半分に勝利する〈俗〉~を出発する【名】割ること、分割{ぶんかつ}、分裂{ぶんれつ} ・
\\	取締役会において普通株式1株を2株に分割することが承認された。〔友人・恋人などの〕けんか別れ、破局{はきょく}仲間割れ{なかまわれ}、対立{たいりつ}、不和{ふわ}、不協和{ふきょうわ}、衝突{しょうとつ}、あつれき、確執{かくしつ}、反目{はんもく}= 
\\	〈俗〉バナナスプリット◆【同】
\\	【形】割れた、分割{ぶんかつ}した、分裂{ぶんれつ}した、分けた《金融》株式分割{かぶしき ぶんかつ}したレベル3、発音
\\	、カナスプリット、変化《動》
\\	いかなる努力も惜しむべきではない。 ・
\\	ちょっと時間あるかな? ・
\\	ちょっといいですか? ・
\\	彼女の心を傷つけるな。 ・
\\	彼女を許してあげて。 ・
\\	テロリストは男を皆殺しにしたが、女や子どもは助命しました。 ・
\\	目が疲れる[見たくもない]からやめて。 ・
\\	小銭を分けてもらえませんか?、お金を恵んでください。◆物乞{もの ご}いなどが言う。	変化形 
\\	【他動】~を使わない、取っておく、控える、出し惜しむ ・
\\	いかなる努力も惜しむべきではない。〔時間{じかん}などを〕割く ・
\\	ちょっと時間あるかな? ・
\\	ちょっといいですか?割愛{かつあい}する、手放す{てばなす}(人)を見逃す[大目{おおめ}に見る]、~を勘弁{かんべん}する、容赦{ようしゃ}する、寛大{かんだい}に扱う、許す、容認{ようにん}する、助命{じょめい}する、救う、~に危害{きがい}を加えない ・
\\	彼女の心を傷つけるな。 ・
\\	彼女を許してあげて。 ・
\\	テロリストは男を皆殺しにしたが、女や子どもは助命しました。 ・
\\	目が疲れる[見たくもない]からやめて。〔人に物・お金などを〕分け与える、貸す ・
\\	小銭を分けてもらえませんか?、お金を恵んでください。◆物乞{もの ご}いなどが言う。~をなしで済ます【名】予備{よび}(の品)、スペア〔ボーリングの〕スペア◆スペアを取ること、またはそのスコア。【形】〔緊急用{きんきゅうよう}の〕予備{よび}の、スペアの余分{よぶん}な、余っている空いている、使われていない〔体が〕ぜい肉がない、締まった〔量や程度{ていど}が〕乏しい、不十分{ふじゅうぶん}なレベル2、発音
\\	、カナスペア、スペヤ、変化《動》
\\	変化形 
\\	【自動】〔磨かれて〕滑らかになる、つやが出る洗練{せんれん}される、完璧{かんぺき}になる【他動】〔磨いて〕~を滑らかにする[艶{つや}を出す]〔回転{かいてん}ドラムで米を〕精白{せいはく}する、精米{せいまい}する〔人や行動{こうどう}などを〕洗練{せんれん}させる、上品{じょうひん}にする〔文体{ぶんたい}や技量{ぎりょう}を〕完全{かんぜん}にする、完成{かんせい}させる【名】磨く[艶出しをする]こと〔表面{ひょうめん}や仕上{しあ}がりの〕滑らかさ、艶{つや}、光沢{こうたく}研磨{けんま}[艶出し]剤〔文体{ぶんたい}や動作{どうさ}などの〕洗練{せんれん}、上品{じょうひん}レベル2、発音
\\	、カナポリッシュ、変化《動》
\\	、分節
\\	新しいほうきはよくはける。/新任者は改革に熱心である。/新任者は現状を変えるのに熱心である。/新任者は仕事ぶりが良い。/新任者は旧弊一掃に熱心である。 ・
\\	美しい山並みが緩やかに海岸まで延びている。 ・
\\	私はポーチを掃きました。葉っぱ[落ち葉]だらけだったのです。 ・
\\	彼は選挙で圧勝しました。 ・
\\	そのチームは5戦のシリーズで全勝しました。 ・
\\	その映画は
\\	映画祭で各賞を独占しました。 ・
\\	彼はエルモンテ周辺地域の一掃(作戦)を手掛けた中隊の指揮官だった。 ・
\\	マンションの盗聴器の発見と撤去を依頼しました。 ・
\\	手をゆったりと回して彼女に入るよう合図しました。 ・
\\	長く続く砂浜の広がりと夕日の素晴{すば}らしい眺めを満喫しました。 ・
\\	アニメーターが絵筆を一振りさせるだけで、子どもたちはそれまで見たこともない世界に連れて行かれるのだ。 ・
\\	キャンプは曲がりくねった川の湾曲部にあった。	変化形 
\\	【自動】〔ほうきやブラシなどでごみやほこりなどを〕掃く、一掃{いっそう}する〔ほうきが〕掃ける ・
\\	新しいほうきはよくはける。/新任者は改革に熱心である。/新任者は現状を変えるのに熱心である。/新任者は仕事ぶりが良い。/新任者は旧弊一掃に熱心である。〔~のそばを〕サッと通る〔感情{かんじょう}などが〕わき上がる勢いよく動く[移動{いどう}]する、さっそうと歩く〔海岸線{かいがんせん}などが長い弧を描いて〕ゆったりと延びる ・
\\	美しい山並みが緩やかに海岸まで延びている。【他動】〔ほうきやブラシなどでごみやほこりなどを〕掃く、掃いて集める ・
\\	私はポーチを掃きました。葉っぱ[落ち葉]だらけだったのです。〔床・地面・表面などを〕掃除{そうじ}する〔不要な物〕一掃{いっそう}する〔ハリケーン・病気などが場所を〕素早く通過{つうか}する〔競争{きょうそう}などを〕圧倒的{あっとう てき}な強さで勝つ、〔一連{いちれん}の試合{しあい}などを〕全勝{ぜんしょう}する、〔賞などを〕独占{どくせん}する ・
\\	彼は選挙で圧勝しました。 ・
\\	そのチームは5戦のシリーズで全勝しました。 ・
\\	その映画は
\\	映画祭で各賞を独占しました。(人)を夢中{むちゅう}にさせる、(人)の心を奪う~を徹底的{てっていてき}に捜索{そうさく}する【名】〔ほうきやブラシなどによる〕掃除{そうじ}、清掃{せいそう}一掃{いっそう} ・
\\	彼はエルモンテ周辺地域の一掃(作戦)を手掛けた中隊の指揮官だった。〔盗聴器{とうちょうき}の〕スイープ◆仕掛けられた盗聴器を探して取り除くこと。 ・
\\	マンションの盗聴器の発見と撤去を依頼しました。〔滑らかな〕湾曲運動{わんきょく うんどう}、円を描く動き ・
\\	手をゆったりと回して彼女に入るよう合図しました。〔土地{とち}などの〕広がり ・
\\	長く続く砂浜の広がりと夕日の素晴{すば}らしい眺めを満喫しました。一振り{ひとふり}、サッと動くこと ・
\\	アニメーターが絵筆を一振りさせるだけで、子どもたちはそれまで見たこともない世界に連れて行かれるのだ。〔ゆったりした〕湾曲{わんきょく}、曲線{きょくせん} ・
\\	キャンプは曲がりくねった川の湾曲部にあった。煙突掃除人{えんとつ そうじ にん}◆【同】
\\	〔掃き集めた〕ごみ、ちり◆【同】
\\	圧倒的{あっとう てき}な勝利{しょうり}、完勝{かんしょう}◆【参考】
\\	《海事》長いオール、大櫂{おおかい}〔跳ねつるべ井戸{いど}の〕横木{よこぎ}
\\	〈話〉=
\\	〈米〉視聴率調査{しちょうりつ ちょうさ}(期間{きかん})、スイープス◆全国でテレビの視聴率を調査するもので、これにより広告料が決定される。ほとんどの州で2月、5月、7月、11月に行われる。《電気》掃引{そういん}、スイープ◆陰極線管内で電子が水平方向にスキャンしながら蛍光面に当たること。レベル3、発音
\\	、カナスイープ、スウィープ、変化《動》
\\	変化形 
\\	【自動】〔地面{じめん}などを〕掘る、掘り返す〔埋まっている物を〕掘り出す〔埋まっている物を〕掘って探す〔穴などを〕掘って進む〔固い物が体に〕突き当たる、刺さる〈話〉丹念{たんねん}に調査{ちょうさ}する、執拗{しつよう}に調べ上げる〈俗〉深く理解{りかい}する◆やや古めかしい表現。【他動】〔地面{じめん}などを〕掘る、掘り返す〔穴などを〕掘る、掘って作る〔埋まっている物を〕掘り出す、掘り当てる〈話〉〔事実{じじつ}や証拠{しょうこ}などを〕探り当てる、掘り出す〔手足{てあし}などを物の中に〕突っ込む、めり込ませる〔固い物を人に〕突き立てる、つつく〈俗〉〔話などを〕深く理解{りかい}する、共感{きょうかん}する◆やや古めかしい表現。〈俗〉~を好む、~を高く評価{ひょうか}する◆やや古めかしい表現。〈俗〉~を見る、~に気付く◆やや古めかしい表現。【名】皮肉{ひにく}、当て付け、当てこすり〈米俗〉がり勉家突っつき、こづき〔考古学{こうこがく}の〕発掘{はっくつ}、遺跡{いせき}、遺物{いぶつ}
\\	〈英〉〔家具付きの〕貸間◆【同】
\\	レベル2、発音
\\	、カナディグ、ディッグ、変化《動》
\\	死体がどこに埋められているか知りたい。 ・
\\	まだ北国は雪深いだろう。	変化形 
\\	【他動】〔土中{どちゅう}に〕~を埋める[埋蔵{まいぞう}する]〔遺体{いたい}を〕葬る、埋葬{まいそう}する ・
\\	死体がどこに埋められているか知りたい。〔人を死によって〕失う、なくす〔完全{かんぜん}に〕~を覆う、~に覆いかぶさる ・
\\	まだ北国は雪深いだろう。〔人が頭や顔を柔らかいものに〕うずめる、隠す〔見えないように物を〕隠す、間に入れる~を埋め込む、~を沈み込ませる〔不快{ふかい}なものを〕忘れ去る、葬り去る、思い出さないようにする〔自分自身{じぶん じしん}を〕集中{しゅうちゅう}[熱中{ねっちゅう}]させる《トランプ》〔一番上{いちばん うえ}のカードを〕中に入れるレベル3、発音
\\	、カナベリー、変化《動》
\\	、分節
\\	変化形 
\\	【1自動】〔垂直{すいちょく}な位置{いち}から〕傾く、かしぐ~にもたれる、~に寄り掛かる〔援助{えんじょ}などを〕頼る、当てにする〔あることを〕しがちである、する傾向{けいこう}がある〈話〉〔心理的圧力{しんり てき あつりょく}などを〕掛ける、向ける【1他動】~を寄り掛からせる、~をもたれさせる~を傾ける、~をかがめる【1名】〔垂直{すいちょく}な位置{いち}からの〕傾き【2名】赤身肉{あかみ にく}、脂肪分{しぼう ぶん}の少ない肉【2形】〔人や動物{どうぶつ}が〕ぜい肉の取れた、体が締まった、痩せた〔肉が〕脂肪分{しぼう ぶん}が少ない、赤身{あかみ}の〔収穫{しゅうかく}などが〕少ない、乏しい〔経営{けいえい}などが〕無駄{むだ}がない、効率的{こうりつ てき}な〔予算{よさん}などが〕余裕{よゆう}がない、限られている〔鉱物{こうぶつ}が〕含有率{がんゆうりつ}が低い、低品質{てい ひんしつ}の〔混合燃料{こんごう ねんりょう}が〕希薄{きはく}なレベル3、発音
\\	、カナリーン、変化《動》
\\	または
\\	水はセ氏0度で凍る。 ・
\\	動くな!さもないと撃つぞ。 ・
\\	コンピューターがフリーズしちゃったよ。 ・
\\	ときどきマウスのカーソルが動かなくなる。	変化形 
\\	【自動】凍る、凍結{とうけつ}[氷結{ひょうけつ}]する ・
\\	水はセ氏0度で凍る。凍り付く、冷凍{れいとう}[冷蔵{れいぞう}]可能{かのう}である凍るくらい寒い、底冷えする◆【用法】非人称代名詞
\\	を主語に用いる凝固{ぎょうこ}[凝結{ぎょうけつ}]する凍える、凍死{とうし}する(ほど寒い)〔恐怖{きょうふ}で体が〕凍り付く、ゾッとする凍ったように(ピタッと)動きを止める ・
\\	動くな!さもないと撃つぞ。 ・
\\	コンピューターがフリーズしちゃったよ。 ・
\\	ときどきマウスのカーソルが動かなくなる。〔さび付いて〕動きが悪くなる〔表情が〕こわばる〔態度が〕冷淡{れいたん}になる、硬化{こうか}する【他動】〔冷やして液体{えきたい}を〕凍らせる、氷にする〔湖などの表面{ひょうめん}を〕氷結{ひょうけつ}させる、氷を張らせる〔管などを〕凍結{とうけつ}させる、氷で詰まらせる〔地面{じめん}などを〕凍結{とうけつ}させる、凍らせて固くする〔物を〕凍り付かせる、凍らせて動かなくさせる〔作物{さくもつ}などを〕寒さで駄目{だめ}にする[枯らす]〔人を〕凍えさせる、ひどく寒い思いをさせる〔人を〕凍死{とうし}させる、凍え死なせる〔寒さや恐怖{きょうふ}で人を〕動けなくする、凍り付かせる〔人を〕冷たく扱う、心を冷えさせる〔開発{かいはつ}や生産{せいさん}などを〕中止{ちゅうし}させる、制限{せいげん}する〔価格{かかく}などを〕凍結{とうけつ}する、据え置く〔資産{しさん}などを〕凍結{とうけつ}する、引き出せないようにする〔動いている物を写真{しゃしん}で〕静止画{せいしが}に写す[する]《スポーツ》〔ボールを〕フリーズする◆バスケットボールなどで、敵にボールを渡さないで保持し続けること。【名】氷結{ひょうけつ}、凍結{とうけつ}フリーズ◆機器やコンピューター・プログラムの動作がピタッと止まってしまうこと据え置きレベル2、発音
\\	、カナフリーズ、変化《動》
\\	暑いです。私は暑くてたまりません。 ・
\\	彼はとてもかわいらしくて、私はとろけそう(な気分)です。	変化形 
\\	【自動】〔加熱によって〕溶ける、融解{ゆうかい}する〔溶けるように〕次第{しだい}になくなる、徐々{じょじょ}に消える〔感情{かんじょう}などが〕次第{しだい}に和らぐ[和む]〈話〉暑くてたまらない◆人が主語。通例進行形で。 ・
\\	暑いです。私は暑くてたまりません。〈話〉とろけそう(な気分{きぶん})だ、うっとりする◆通例、進行形で。 ・
\\	彼はとてもかわいらしくて、私はとろけそう(な気分)です。【他動】〔加熱して〕~を溶かす[融解{ゆうかい}させる]〔溶かすように〕徐々{じょじょ}に消失{しょうしつ}させる(人)を和ませる、(人)の心をとろけさせる【名】溶解(物){ようかい(ぶつ)}、融液{ゆうえき}レベル2、発音
\\	、カナメルト、変化《動》
\\	バブル経済がはじけた。 ・
\\	私はおなかがはち切れるくらい、たくさん食べました。	変化形 
\\	【自動】破裂{はれつ}する、爆発{ばくはつ}する、はじける ・
\\	バブル経済がはじけた。 ・
\\	私はおなかがはち切れるくらい、たくさん食べました。〔服などが〕きつ過ぎて破れる勢いよく出る急に~しだす、突発{とっぱつ}する、急に起こる、抑えきれない、充満{じゅうまん}する【他動】~を破裂{はれつ}させる、裂く【名】爆発{ばくはつ}、破裂{はれつ}、裂傷{れっしょう}突発{とっぱつ}、突進{とっしん}、一気{いっき}〔感情{かんじょう}の〕爆発{ばくはつ}、ほとばしり〔爆弾{ばくだん}・弾丸{だんがん}の〕爆発{ばくはつ}、集中射撃{しゅうちゅう しゃげき}レベル2、発音
\\	、カナバースト、変化《動》
\\	海を見ると船乗りの血が騒ぐ。	変化形 
\\	【1自動】〔スプーンなどで〕かき回せる[混ぜられる]〔位置{いち}が〕わずかに動く[ずれる・揺れる]〔休息後{きゅうそく ご}に〕目覚める、活動{かつどう}する〈文〉〔感情{かんじょう}が〕呼び覚まされる、わき上がる ・
\\	海を見ると船乗りの血が騒ぐ。興奮{こうふん}する、感激{かんげき}する〈話〉〔騒ぎなどが〕起きる、発生{はっせい}する【1他動】〔器具{きぐ}で液体{えきたい}を〕かき回す[混ぜる]◆冷ましたり、中の材料を溶かしたりするために、スプーンなどを円形に動かすこと。〔液体{えきたい}の中の材料{ざいりょう}などを〕かき混ぜる◆液体や混合物の中に材料などを入れて動かすこと。〔~の位置{いち}を〕わずかに動かす[ずらす・揺らす](人)に行動{こうどう}を起こさせる、(人)を扇動{せんどう}する〈文〉〔感情{かんじょう}や記憶{きおく}を〕呼び覚ます、目覚めさせる(人)を興奮{こうふん}[感激{かんげき}]させる【1名】〔液体{えきたい}を〕かき回す[混ぜる]こと〔わずかな〕動き、揺れ〔精神的{せいしん てき}な〕動揺{どうよう}、興奮{こうふん}〈豪話〉面倒{めんどう}、トラブル【2名】〈俗〉刑務所{けいむしょ}レベル4、発音
\\	、カナスター、変化《動》
\\	俺{おれ}のついだ酒が飲めないっていうのか。 ・
\\	どうぞご自由にもう一杯召し上がって。	変化形 
\\	【自動】雨が激しく降る【他動】~を注ぐ、液体{えきたい}をつぐ、光を放射{ほうしゃ}する、弾丸{だんがん}を浴びせる、殺到{さっとう}する ・
\\	俺{おれ}のついだ酒が飲めないっていうのか。 ・
\\	どうぞご自由にもう一杯召し上がって。レベル2、発音
\\	、カナポア、ポー、ポーア、変化《動》
\\	好きなことにはすぐ熱中する。	変化形 
\\	【他動】〔熱・光・気体{きたい}・液体{えきたい}・衝撃{しょうげき}・音などを徐々{じょじょ}に〕吸収{きゅうしゅう}する、吸い上げる〔光・音・熱・衝撃{しょうげき}などを吸収{きゅうしゅう}してその強さを〕和らげる、緩和{かんわ}する~を自分{じぶん}の中に取り込む、〔意味{いみ}を〕理解{りかい}する、〔学問{がくもん}・思想{しそう}・知識{ちしき}・教養{きょうよう}などを〕吸収{きゅうしゅう}する、自分{じぶん}のものとする、取り入れる、身に付ける、同化{どうか}する、受け入れる~を夢中{むちゅう}にさせる、〔人の注意{ちゅうい}・心・時間{じかん}などを〕すっかり奪う ・
\\	好きなことにはすぐ熱中する。〔時間{じかん}・体力{たいりょく}・収入{しゅうにゅう}・資源{しげん}などを〕使い尽くす、消耗{しょうもう}する〔移民{いみん}などを〕受け入れる、〔小国{しょうこく}・町村{ちょうそん}・会社{かいしゃ}・企業{きぎょう}・都市{とし}などを〕(~に)吸収合併{きゅうしゅう がっぺい}する、併合{へいごう}する〔企業側{きぎょう がわ}で増税{ぞうぜい}などを〕吸収{きゅうしゅう}する、〔費用{ひよう}を〕負担{ふたん}する〔パンチ・罰などを〕受ける、〔パンチを〕受けても倒れない、〔攻撃{こうげき}に〕耐える~を血中{けっちゅう}に吸収{きゅうしゅう}するレベル4、発音
\\	、カナアブゾーブ、アブソーブ、変化《動》
\\	、分節
\\	変化形 
\\	【自動】食べ物を消化{しょうか}する〔食べ物などが〕こなれる【他動】〔食べ物などを〕消化{しょうか}する〔知識{ちしき}などを〕消化{しょうか}する、会得{えとく}する〔意味{いみ}を〕飲み込む要約{ようやく}する、短くする我慢{がまん}する【名】〔論文{ろんぶん}や文学作品{ぶんがく さくひん}などの〕要約{ようやく}、梗概{こうがい}《法律》判例集{はんれいしゅう}、法律学説集{ほうりつがくせつ しゅう}〔定期刊行{ていき かんこう}される〕ダイジェスト誌、要約本{ようやく ほん}
\\	〔市民法大全{しみんほう たいぜん}の〕学説彙纂{がくせつ いさん}◆【同】
\\	レベル4、発音《名》
\\	《動》
\\	、カナダイジェスト、ディジェスト、変化《動》
\\	、分節
\\	変化形 
\\	【自動】円
\\	のようにグルグル回る、円運動{えんうんどう}をする、〔液体{えきたい}・気体{きたい}などが〕循環{じゅんかん}する、巡回{じゅんかい}する〔うわさ・デマなどが〕流れる、広まる〔雑誌{ざっし}・新聞{しんぶん}・広告{こうこく}などが〕配布{はいふ}される〔貨幣{かへい}が〕流通{りゅうつう}する【他動】~を循環{じゅんかん}させる~を人から人へと移動{いどう}させる[渡らせる・伝える]、〔うわさ・デマなどを〕流す、広める〔報告書などを〕回覧{かいらん}する〔広告{こうこく}・新聞{しんぶん}・雑誌{ざっし}などを〕配布{はいふ}する〔通貨{つうか}を〕流通{りゅうつう}させるレベル7、発音
\\	、カナサーキュレイト、変化《動》
\\	、分節
\\	変化形 
\\	【他動】〔程度{ていど}・限度{げんど}などを〕超える、上回る{うわまわる}、突破{とっぱ}する~をしのぐ、~に勝るレベル5、発音
\\	、カナエクスィード、変化《動》
\\	、分節
\\	仕事は割り当てられた時間いっぱいまで延びる。◆ことわざ ・
\\	コンプレッサーの配置を限定する必要がなく、自動車設計の自由度が大幅に増します。 ・
\\	市場占有率を拡大するため、その企業は競争相手を買収しようとした。	変化形 
\\	【自動】〔規模が〕大きくなる、拡大する ・
\\	仕事は割り当てられた時間いっぱいまで延びる。◆ことわざ〔ゴムのように伸縮性のあるものが引っ張られて〕広がる膨らむ、膨張{ぼうちょう}する【他動】〔規模を〕拡大{かくだい}する、拡張{かくちょう}する◆【語源】接頭辞
\\	-(外側へ)
\\	(広がる) ・
\\	コンプレッサーの配置を限定する必要がなく、自動車設計の自由度が大幅に増します。 ・
\\	市場占有率を拡大するため、その企業は競争相手を買収しようとした。広げる、展開{てんかい}する〔金額{きんがく}などを〕引き上げる膨張{ぼうちょう}させる〔話を〕膨らませる《コ》〔コンピューターの
\\	でツリー状になったフォルダを〕開く、展開する◆【対】
\\	レベル4、発音
\\	、カナイクスパンド、エクスパンド、変化《動》
\\	、分節
\\	夜空に花火が上がった。 ・
\\	胸が張り裂けそうだ。 ・
\\	社株は決算のニュース後急上昇しました。	変化形 
\\	【自動】爆発{ばくはつ}する、破裂{はれつ}する、炸裂{さくれつ}する ・
\\	夜空に花火が上がった。〔炭酸飲料{たんさん いんりょう}などが〕吹き出す感情{かんじょう}が激発{げきはつ}する、激怒{げきど}する、爆発的{ばくはつてき}に増える、一斉{いっせい}に怒鳴る[笑う] ・
\\	胸が張り裂けそうだ。〔株価などが〕急上昇する ・
\\	社株は決算のニュース後急上昇しました。【他動】爆発{ばくはつ}させる、論破{ろんぱ}するレベル4、発音
\\	、カナイクスプロウド、エクスプロード、エクスプロウド、変化《動》
\\	、分節
\\	変化形 
\\	【自動】〔人や動物{どうぶつ}が〕散り散りになる〔出来事{できごと}や雨などが〕時折{ときおり}起きる[降る]《物理》〔粒子{りゅうし}などが〕散乱{さんらん}する【他動】〔物を〕ばらまく、まき散らす〔場所{ばしょ}に〕まく、振りかける《物理》〔粒子{りゅうし}などを〕散乱{さんらん}させる《野球》~安打散発{あんだ さんぱつ}に抑える【名】まき散らすこと、散り散りになること〈文〉まき散らされた物レベル4、発音
\\	、カナスキャター、スキャッター、変化《動》
\\	、分節
\\	ディスクにほこりや汚れ、指紋などが付かないように注意してください。 ・
\\	その装置のパネルが壊れ、構造がむき出しになっています。	変化形 
\\	《複》
\\	【他動】〔保護されていない状態や危険なものなどに~を〕さらす、むき出しにする ・
\\	ディスクにほこりや汚れ、指紋などが付かないように注意してください。 ・
\\	その装置のパネルが壊れ、構造がむき出しになっています。〔フィルムを〕露光{ろこう}する、感光{かんこう}させる〔人を~に〕触れさせる、経験{けいけん}させる~を見えるようにする、~を掘り起こす、~を陳列{ちんれつ}する〔秘密や悪事などを〕暴{あば}く、ばらす、暴露{ばくろ}する、露呈{ろてい}する、顕在化{けんざいか}させる〔人の陰部{いんぶ}を〕露出{ろしゅつ}する、見せる〔子どもを〕遺棄{いき}する◆古代ギリシャ特にスパルタで行われていた、病弱な子どもを戸外に放置して死なせること。《カトリック》〔聖体{せいたい}を〕顕示{けんじ}する【名】暴露{ばくろ}[スキャンダル]記事{きじ}レベル4、発音
\\	、カナイクスポウズ、エクスポーズ、エクスポウズ、変化《複》
\\	、《動》
\\	、分節
\\	包み隠さず白状しろ! ・
\\	彼は証拠を隠していた。 ・
\\	彼女の居場所は秘密にされている。 ・
\\	わざわざ隠し立てをすると、疑惑を招く。 ・
\\	彼女は何かを隠しているかのような話し方をする。 ・
\\	真実は隠せない。 ・
\\	彼は不法行為を隠匿したために辞任を余儀なくされた。	変化形 
\\	【他動】〔物・情報・事実などを〕隠す、隠匿{いんとく}する、秘密{ひみつ}にする ・
\\	包み隠さず白状しろ! ・
\\	彼は証拠を隠していた。 ・
\\	彼女の居場所は秘密にされている。 ・
\\	わざわざ隠し立てをすると、疑惑を招く。 ・
\\	彼女は何かを隠しているかのような話し方をする。 ・
\\	真実は隠せない。 ・
\\	彼は不法行為を隠匿したために辞任を余儀なくされた。レベル5、発音
\\	、カナコンスィール、変化《動》
\\	、分節
\\	その国の政治家たちは、次に誰が脱税の捜査を受けるかと疑心暗鬼です。	変化形 
\\	【自動】〔詳細{しょうさい}に〕調査{ちょうさ}する、研究{けんきゅう}する〔捜査機関{そうさ きかん}が〕捜査{そうさ}する、取り調べる【他動】〔~を詳細{しょうさい}に〕調査{ちょうさ}する、研究{けんきゅう}する〔犯罪{はんざい}や人などを〕捜査{そうさ}する、取り調べる ・
\\	その国の政治家たちは、次に誰が脱税の捜査を受けるかと疑心暗鬼です。レベル4、発音
\\	、カナインベスティゲイト、変化《動》
\\	、分節
\\	多くのハイテク企業がその大学の施設を活用するためにここを本拠地に定めてきた。 ・
\\	戦場のあらゆる船が着色ピンで地図に示されている。 ・
\\	この機械の主電源は装置後部にある。	変化形 
\\	【自動】〔特定{とくてい}の場所{ばしょ}に〕定住{ていじゅう}する、落ち着く、本拠地{ほんきょち}を置く ・
\\	多くのハイテク企業がその大学の施設を活用するためにここを本拠地に定めてきた。【他動】〔~の場所{ばしょ}を〕決める、示す ・
\\	戦場のあらゆる船が着色ピンで地図に示されている。〔~の場所{ばしょ}を〕探す、見つける〔特定{とくてい}の場所{ばしょ}に〕~を置く、~を設置{せっち}する、~を位置付ける{いちづける} ・
\\	この機械の主電源は装置後部にある。《イ》検索するレベル4、発音
\\	、カナローケイト、ロウケイト、ロケイト、変化《動》
\\	、分節
\\	考えられるあらゆる可能性を調査しました。	変化形 
\\	【他動】~を探検{たんけん}[探索{たんさく}]する〔詳しく〕~を調査{ちょうさ}[研究{けんきゅう}]する ・
\\	考えられるあらゆる可能性を調査しました。《医》~を診察{しんさつ}するレベル3、発音
\\	、カナイクスプロー、エクスプロー、エクスプローア、変化《動》
\\	、分節
\\	うらやましいよ。	変化形 
\\	《複》
\\	【他動】〔人やその持っているものを〕うらやましがる、ねたむ、嫉妬{しっと}する ・
\\	うらやましいよ。【名】うらやましさ、嫉妬{しっと}、ねたみ、羨望{せんぼう}うらやましがられる人[もの]、羨望{せんぼう}の的レベル4、発音
\\	、カナエンビー、変化《動》
\\	、分節
\\	私は彼が嫌いだ。 ・
\\	彼のどこが気に入らないのですか?	変化形 
\\	【他動】~を嫌う、毛嫌いする ・
\\	私は彼が嫌いだ。 ・
\\	彼のどこが気に入らないのですか?【名】嫌悪{けんお}、反感{はんかん}、嫌い、嫌気{いやけ}、毛嫌い{けぎらい}、嫌いなこと[もの]レベル2、発音
\\	、カナデスライク、ディスライク、変化《動》
\\	、分節
\\	変化形 
\\	【他動】〔意図的ではなくて不注意で〕無視{むし}する、軽視{けいし}する、ほったらかす、顧みない、放置{ほうち}する、おざなりにする〔不注意{ふちゅうい}で〕~し忘れる〔仕事{しごと}などを〕怠る【名】〔人やものに対する配慮{はいりょ}などの〕無視{むし}、不履行{ふりこう}、おろそかにすること〔配慮{はいりょ}などの〕欠如{けつじょ}、おろそかになっている状態{じょうたい}ネグレクト◆保護者が子どもの養育などを放棄すること。レベル3、発音
\\	、カナネグレクト、ニグレクト、変化《動》
\\	、分節
\\	彼は困っている人を見捨てておけない性分です。	変化形 
\\	【他動】~を無視{むし}する、~に気付{きづ}かないふりをする ・
\\	彼は困っている人を見捨てておけない性分です。《法律》〔起訴状{きそじょう}を〕却下{きゃっか}するレベル3、発音
\\	、カナイグノー、変化《動》
\\	、分節
\\	変化形 
\\	【自動】ちらりと[チラッと・サッと・パッと]見る〔反射光{はんしゃこう}が一瞬{いっしゅん}〕きらめく、輝く〔浅い角度{かくど}で〕当たる、〔軽く当たって〕それていく〔簡単{かんたん}に〕言及{げんきゅう}する、〔話題{わだい}などに〕触れる【他動】~をひと目[ちらりと]見る〔浅い角度{かくど}で〕~に当たる、〔軽く〕~にぶつかる【名】ひと目、ちらりと[チラッと・サッと・パッと]見ること、一瞥{いちべつ}〔一瞬{いっしゅん}の〕きらめき、輝き〔浅い角度{かくど}で〕当たること、〔軽く当たって〕それていくこと〔簡単{かんたん}に〕言及{げんきゅう}すること、〔話題{わだい}などに〕触れることレベル3、発音
\\	、カナグランス、変化《動》
\\	変化形 
\\	【自動】〔人が目を見開いて〕じっと見る、凝視{ぎょうし}する〔目が驚きなどで〕見開いている目立{めだ}っている、あからさまである〔髪や毛が〕逆立{さかだ}っている【他動】~をじっと見る、~を凝視{ぎょうし}する〔答えや不快{ふかい}なことが〕(人)にとって明らかである、(人)に確実{かくじつ}に迫っている◆【参考】
\\	【名】じっと見ること、凝視{ぎょうし}〔驚いて目を見開いた〕凝視{ぎょうし}の表情{ひょうじょう}レベル4、発音
\\	、カナステア、変化《動》
\\	変化形 
\\	【自動】〔興味があって・驚いて~を〕じっと見る、凝視{ぎょうし}する、注視{ちゅうし}する【名】凝視{ぎょうし}、注視{ちゅうし}レベル5、発音
\\	、カナゲイズ、変化《動》
\\	価格は予告なく変更される場合があります。 ・
\\	アメリカ大使館事務所は休日は休業となります。 ・
\\	ご意見は承りました。◆【場面】内容の是非はともかく相手の言いたいことは分かったと告げる。 ・
\\	彼は私にメモを書きました。 ・
\\	彼は私にメモをよこした。	変化形 
\\	【他動】~を書き留める〈文〉~に気付く、~に気が付く~に注意{ちゅうい}[注目{ちゅうもく}・留意{りゅうい}]する、~を気に留める、~に気を配る ・
\\	価格は予告なく変更される場合があります。 ・
\\	アメリカ大使館事務所は休日は休業となります。 ・
\\	ご意見は承りました。◆【場面】内容の是非はともかく相手の言いたいことは分かったと告げる。~に言及{げんきゅう}する~を意味{いみ}する、指摘{してき}する【名】メモ、覚書{おぼえがき} ・
\\	彼は私にメモを書きました。 ・
\\	彼は私にメモをよこした。注釈{ちゅうしゃく}、注記{ちゅうき}、注意{ちゅうい}記録{きろく}〈英〉紙幣{しへい}◆【同】
\\	〈米〉
\\	記号{きごう}、符号{ふごう}重要(性){じゅうよう(せい)}、注目{ちゅうもく}有名{ゆうめい}、著名{ちょめい}兆候{ちょうこう}、雰囲気{ふんいき}、様子{ようす}声の調子{ちょうし}、響き、話し方{はなしかた}、態度{たいど}《音楽》音符{おんぷ}〔主に短めの〕手紙{てがみ}レベル1、発音
\\	、カナノート、ノウト、変化《動》
\\	変化形 
\\	【他動】〔目や耳などの感覚{かんかく}で〕~に気付く、~を見抜く{みぬく}~を理解{りかい}する、~を把握{はあく}する、~だと分かるレベル4、発音
\\	、カナパスィーブ、変化《動》
\\	、分節
\\	《マニュアル》操作手順の中で明示または暗示されている警告や注意事項に注目してください。 ・
\\	安全上の注意を必ずお守りください。 ・
\\	皆さまに安全で楽しい旅をしていただくため、基本的な注意事項をお守りくださいますようお願い致します。最初に機内の安全注意事項をよくお読みください。◆飛行機 ・
\\	どんな休暇があるのですか。 ・
\\	誤差が生じる場合がある。	変化形 
\\	【自動】観察{かんさつ}する、注意{ちゅうい}する意見{いけん}を述べる【他動】よく見る、観察{かんさつ}する、見学{けんがく}する、監視{かんし}する、観測{かんそく}する ・
\\	《マニュアル》操作手順の中で明示または暗示されている警告や注意事項に注目してください。〔規則{きそく}や法律{ほうりつ}を〕守る、順守{じゅんしゅ}する ・
\\	安全上の注意を必ずお守りください。 ・
\\	皆さまに安全で楽しい旅をしていただくため、基本的な注意事項をお守りくださいますようお願い致します。最初に機内の安全注意事項をよくお読みください。◆飛行機〔祝祭日{しゅくさいじつ}を〕祝う ・
\\	どんな休暇があるのですか。~に気付く ・
\\	誤差が生じる場合がある。レベル3、発音
\\	、カナオブザーブ、変化《動》
\\	、分節
\\	人は、何かを失うまで、その有り難さに気付かないものです。 ・
\\	失って初めて分かる大切さ。 ・
\\	お気遣いのお手紙、大変有り難く存じます。 ・
\\	そう言っていただいて感謝しております。	変化形 
\\	【自動】価値{かち}が上がる、値上{ねあ}がりする、相場{そうば}が上がる、騰貴{とうき}する【他動】~を正しく[正当{せいとう}に]評価{ひょうか}[理解{りかい}・認識{にんしき}]する、十分{じゅうぶん}[よく]理解{りかい}する、敏感{びんかん}に察する、~が分かる~の良さが分かる、真価{しんか}を認める、~を高く評価{ひょうか}する、称賛{しょうさん}する ・
\\	人は、何かを失うまで、その有り難さに気付かないものです。 ・
\\	失って初めて分かる大切さ。~をありがたく思う、感謝{かんしゃ}する
\\	のフォーマルな表現 ・
\\	お気遣いのお手紙、大変有り難く存じます。 ・
\\	そう言っていただいて感謝しております。~を面白く味わう、鑑賞{かんしょう}するレベル4、発音
\\	、カナアプリーシエイト、アプリシエイト、変化《動》
\\	、分節
\\	それが事実だと認めよう。	変化形 
\\	【自動】入ることを許す、余地{よち}がある◆【用法】
\\	~を認める◆【用法】
\\	~につながる[通じる]◆【用法】
\\	【他動】~を認める[承認{しょうにん}する] ・
\\	それが事実だと認めよう。~に入ることを許す、~への入場{にゅうじょう}を許可{きょか}する~に収容{しゅうよう}できるレベル3、発音
\\	、カナアドゥミット、アドミット、変化《動》
\\	、分節
\\	それは誰しも認めるところだ。 ・
\\	何があろうとそれは認められない。 ・
\\	あなたの伝送は受信され、確認されました。	変化形 
\\	【他動】〔~を事実{じじつ}だと〕認める、承認{しょうにん}する、同意{どうい}する、認識{にんしき}する、受け入れる、白状{はくじょう}する ・
\\	それは誰しも認めるところだ。 ・
\\	何があろうとそれは認められない。《法律》~を認知{にんち}する、実子{じっし}として承認{しょうにん}する〔手紙{てがみ}・荷物{にもつ}の到着{とうちゃく}を〕知らせる〔厚意・贈り物などに対して〕にお礼を言う、感謝{かんしゃ}する、感謝{かんしゃ}の念を示す〔通信{つうしん}や命令系統{めいれい けいとう}で〕了解{りょうかい}する、受信{じゅしん}を確認{かくにん}する、応答{おうとう}する◆【略】
\\	あなたの伝送は受信され、確認されました。【名】《コ》肯定応答{こうてい おうとう}レベル6、発音
\\	、カナアクナリッジ、アクノリッジ、変化《動》
\\	、分節
\\	変化形 
\\	【自動】良いと考える[思う]、気に入る、賛成{さんせい}する、満足{まんぞく}に思う承認{しょうにん}する、是認{ぜにん}する、認可{にんか}する【他動】~を良いと認める、~に同意{どうい}[賛成{さんせい}]する~を承認{しょうにん}[是認{ぜにん}]するレベル3、発音
\\	、カナアプルーブ、変化《動》
\\	、分節
\\	一つ確認したいことがあります。 ・
\\	私は彼が今日の午後アメリカに無事に到着したことを確認しました。 ・
\\	どちらの理論が正しいかはまだ確認できないままである。	変化形 
\\	【他動】〔真実であることを(証拠を示して)〕確かめる、確認{かくにん}する、裏付ける ・
\\	一つ確認したいことがあります。 ・
\\	私は彼が今日の午後アメリカに無事に到着したことを確認しました。 ・
\\	どちらの理論が正しいかはまだ確認できないままである。追認{ついにん}する、承認{しょうにん}する、正式{せいしき}に発表{はっぴょう}する〔決心{けっしん}・態度{たいど}・決意{けつい}・信念{しんねん}・意見{いけん}などを〕確かにする、強める、固める、支持{しじ}する《教会》堅信式{けんしん しき}を施す、授堅するレベル3、発音
\\	、カナコンファーム、変化《動》
\\	、分節
\\	その映画が終わってから友達にどんなに感動したか話しました。 ・
\\	7歳のときに自分と彼らの皮膚の色の違いに気付いた。 ・
\\	今のは一体どういう意味なんだ? ・
\\	特に注目するようなものは何もない部屋に入った。	変化形 
\\	【自動】感想{かんそう}[意見{いけん}・所見{しょけん}・コメント]を述べる【他動】〔簡単{かんたん}に意見{いけん}などを〕述べる、言う ・
\\	その映画が終わってから友達にどんなに感動したか話しました。~に気付く、~だと分かる ・
\\	7歳のときに自分と彼らの皮膚の色の違いに気付いた。【名】発言{はつげん}、意見{いけん}、見解{けんかい}、所感{しょかん}、所見{しょけん}、論評{ろんぴょう}、批評{ひひょう} ・
\\	今のは一体どういう意味なんだ?〈文〉気付くこと、注目{ちゅうもく} ・
\\	特に注目するようなものは何もない部屋に入った。《コ》=
\\	レベル4、発音
\\	、カナリマーク、変化《動》
\\	、分節
\\	詳しくは当社ホームページをご覧ください。 ・
\\	その患者は手術のため別の病院に差し向けられました。 ・
\\	《就業規則》マスコミとの情報のやりとり、またはマスコミからの問い合わせは、統括責任者に照会しなければならない。	変化形 
\\	【自動】〔~に〕言及{げんきゅう}する、言い及ぶ、触れる〔~を〕引用{いんよう}する〔~を〕参照{さんしょう}する ・
\\	詳しくは当社ホームページをご覧ください。〔~に〕問い合わせる、照会{しょうかい}する【他動】〔~を…に〕差し向ける、持ち込む ・
\\	その患者は手術のため別の病院に差し向けられました。〔~を…に〕参照{さんしょう}させる、照会{しょうかい}させる ・
\\	《就業規則》マスコミとの情報のやりとり、またはマスコミからの問い合わせは、統括責任者に照会しなければならない。〔問題の解決などを…に〕委託{いたく}する、付託{ふたく}する、委ねる、任せる〔~を…に〕属するものとする〔~の原因{げんいん}を…に〕帰する、〔~を…の〕せいにするレベル3、発音
\\	、カナリファー、変化《動》
\\	、分節
\\	このアイコンは音量を示している。 ・
\\	商品の発送方法についてご希望を示してください。◆注文書などで複数の発送方法から一つを選択するとき。	変化形 
\\	【自動】〈英〉〔車の〕方向指示器{ほうこう しじき}で合図{あいず}する【他動】〔物や方向{ほうこう}を〕指し示す、指さす〔ものの存在{そんざい}などを〕示す、意味{いみ}する〔測定{そくてい}した数値{すうち}などを〕示す、表示{ひょうじ}する ・
\\	このアイコンは音量を示している。〔人の希望{きぼう}や意図{いと}を〕述べる、表明{ひょうめい}する ・
\\	商品の発送方法についてご希望を示してください。◆注文書などで複数の発送方法から一つを選択するとき。〔対策{たいさく}や必要性{ひつよう せい}を〕暗示{あんじ}する、示唆{しさ}する〈英〉〔車の方向指示器{ほうこう しじき}を〕出す《医》〔病気{びょうき}の兆候{ちょうこう}を〕示すレベル4、発音
\\	、カナインジケイト、インディケイト、変化《動》
\\	、分節
\\	あまり確信のないことは絶対にほのめかすな。 ・
\\	何が言いたいのですか。 ・
\\	私がうぬぼれているということですか。 ・
\\	君と結婚したくないなんてほのめかしたことなど決してない。 ・
\\	いや、私たちはそう言ってはいないし、そうほのめかしてもいない。 ・
\\	あなたは何か重要な情報を持っていることを電話でほのめかしましたね。	変化形 
\\	【他動】〔はっきりとは言わず〕暗に伝える、暗示{あんじ}する、ほのめかす ・
\\	あまり確信のないことは絶対にほのめかすな。 ・
\\	何が言いたいのですか。 ・
\\	私がうぬぼれているということですか。 ・
\\	君と結婚したくないなんてほのめかしたことなど決してない。 ・
\\	いや、私たちはそう言ってはいないし、そうほのめかしてもいない。 ・
\\	あなたは何か重要な情報を持っていることを電話でほのめかしましたね。〔~という〕意味{いみ}を含む、ニュアンスになるレベル5、発音
\\	、カナインプライ、変化《動》
\\	、分節
\\	どうかわたしのベッドをお使いください。私は暖炉の前で横になります。	変化形 
\\	【自動】祈る、懇願{こんがん}する、嘆願{たんがん}する、~を祈り求める、懇願{こんがん}して~させる〈古〉どうぞ、どうか◆【同】
\\	どうかわたしのベッドをお使いください。私は暖炉の前で横になります。レベル2、発音
\\	、カナプレイ、変化《動》
\\	変化形 
\\	【他動】〔人を温かく〕出迎える{でむかえる}、歓迎{かんげい}する〔人に丁寧{ていねい}に〕挨拶{あいさつ}する、呼び掛ける〔物事{ものごと}をある様式{ようしき}で〕迎える、応じる〔人がにおいや音などに〕気付く、感知{かんち}するレベル3、発音
\\	、カナグリート、変化《動》
\\	変化形 
\\	【自動】発音{はつおん}する、言葉{ことば}を話す〔公式{こうしき}に〕公表{こうひょう}する、宣告{せんこく}する【他動】〔語や音声{おんせい}を聞こえるように〕発声{はっせい}する、声に出す〔言葉{ことば}をはっきりとまたはある方法{ほうほう}で〕話す、発音{はつおん}する《言語学》〔語を〕発音記号{はつおん きごう}で書き表す〔判断{はんだん}や見解{けんかい}を公式{こうしき}に〕公表{こうひょう}する、宣告{せんこく}するレベル4、発音
\\	、カナプロナウンス、変化《動》
\\	、分節
\\	《契約書》本契約の当事者はライセンサーとライセンシーという法的関係にあり、代理店や共同事業など、その他の形態にあると解釈しないものとする。 ・
\\	彼の表情は読み取るのが難しい。	変化形 
\\	【他動】通訳{つうやく}する解釈{かいしゃく}する、解明{かいめい}する、説明{せつめい}する ・
\\	《契約書》本契約の当事者はライセンサーとライセンシーという法的関係にあり、代理店や共同事業など、その他の形態にあると解釈しないものとする。〔文芸作品や人の表情を〕読み取る ・
\\	彼の表情は読み取るのが難しい。レベル5、発音
\\	、カナインタープリット、インタープレット、変化《動》
\\	、分節
\\	彼女が支払ったことを承諾と見て取った。	変化形 
\\	【自動】翻訳{ほんやく}[通訳{つうやく}]する翻訳{ほんやく}[通訳{つうやく}]できる[可能{かのう}である]翻訳{ほんやく}[通訳{つうやく}]者として働く変換{へんかん}[転換{てんかん}・転移{てんい}]できる[可能{かのう}である]【他動】〔別の言語{げんご}に〕~を翻訳{ほんやく}[通訳{つうやく}]する〔易しい言葉{ことば}で〕~を説明{せつめい}[理解{りかい}]する〔別の表現{ひょうげん}で〕~を言い換える[置き換える]〔言動{げんどう}・身ぶり・顔の表情{ひょうじょう}を〕解釈{かいしゃく}する、見て取る ・
\\	彼女が支払ったことを承諾と見て取った。〔形や機能{きのう}などを〕変換{へんかん}[転換{てんかん}]する〔人や物の場所{ばしょ}を〕移転{いてん}[移動{いどう}]する《物理》〔物体を回転させないで〕平行移動{へいこう いどう}させる《コ》〔プログラムを〕翻訳{ほんやく}する《キリスト教》〔別の管区{かんく}に司教{しきょう}を〕転任{てんにん}する《宗教》〔人を〕昇天{しょうてん}させる、天国{てんごく}に送る《生物》
\\	を〕翻訳{ほんやく}するレベル3、発音
\\	、カナトゥランスレイト、トランスレイト、トランズレイト、変化《動》
\\	、分節
\\	本当にそうなんだ。 ・
\\	その国は核兵器保有国であることを宣言しました。 ・
\\	オリッサ、ラジャスターン、ガジャラート州当局は、このストライキを違法だと宣言しました。	変化形 
\\	【自動】宣言{せんげん}[力説{りきせつ}・言明{げんめい}]する ・
\\	本当にそうなんだ。【他動】~を宣言{せんげん}する、宣する、布告{ふこく}する、断言{だんげん}[力説{りきせつ}・言明{げんめい}]する、うたう ・
\\	その国は核兵器保有国であることを宣言しました。 ・
\\	オリッサ、ラジャスターン、ガジャラート州当局は、このストライキを違法だと宣言しました。~を証拠{しょうこ}で明らかにする、打ち明ける、公表{こうひょう}する、表す、示す〔税関{ぜいかん}・税務署{ぜいむしょ}で〕~を申告{しんこく}するレベル4、発音
\\	、カナディクレア、デクレア、変化《動》
\\	、分節
\\	変化形 
\\	【自動】〔注意{ちゅうい}や努力{どりょく}を~に〕集中{しゅうちゅう}[専心{せんしん}]させる◆【用法】
\\	〔一カ所に〕集まる、集結{しゅうけつ}する〔物事{ものごと}の程度{ていど}が〕増す、強まる【他動】〔注意{ちゅうい}や努力{どりょく}を~に〕集中{しゅうちゅう}[専心{せんしん}]させる〔一カ所に〕~を集結{しゅうけつ}[集中{しゅうちゅう}]させる〔蒸発{じょうはつ}させて〕~を濃縮{のうしゅく}する〔料理{りょうり}を〕煮詰める{につめる}《鉱物》〔鉱石{こうせき}を〕選鉱{せんこう}する【名】濃縮{のうしゅく}物[食品{しょくひん}]◆【略】
\\	レベル4、発音
\\	、カナカンセントゥレイト、コンセントレイト、変化《動》
\\	、分節
\\	変化形 
\\	【自動】変わる、様変わりする、改まる【他動】〔部分的{ぶぶんてき}に〕変える、変更{へんこう}する、改める、改ざんする、改造{かいぞう}する、修正{しゅうせい}する、改正{かいせい}する、変造{へんぞう}する、様変わりさせる〔衣服{いふく}を〕仕立て直す〈米〉~を去勢{きょせい}する、~の卵巣{らんそう}を除去{じょきょ}する
\\	の婉曲語レベル4、発音
\\	、カナオールター、変化《動》
\\	、分節
\\	どのようにセ氏温度をカ氏に変換しますか? ・
\\	何回もチャンスはあったが得点には結び付かなかった。	変化形 
\\	【自動】〔性質{せいしつ}や形が別のものに〕変形{へんけい}[変質{へんしつ}]する〔機能{きのう}や用途{ようと}などが〕変わる、転換{てんかん}する〔人や信念{しんねん}などが〕転向{てんこう}[改心{かいしん}]する【他動】〔~の性質{せいしつ}や形を〕変形{へんけい}[変質{へんしつ}]させる〔~の機能{きのう}や用途{ようと}などを〕変更{へんこう}[改造{かいぞう}]する〔計量単位{けいりょう たんい}を別のものに〕変換{へんかん}[換算{かんさん}]する ・
\\	どのようにセ氏温度をカ氏に変換しますか?〔人の信念{しんねん}などを〕転向{てんこう}[改心{かいしん}]させる〔財産{ざいさん}などを等価{とうか}のものに〕換金{かんきん}[交換{こうかん}]する《論理学》〔定言文{ていげん ぶん}を〕換位{かんい}する◆主語と述語を入れ替えて新たな定言文を作ること。《法律》〔他人{たにん}の財産{ざいさん}を〕横領{おうりょう}する《法律》〔財産{ざいさん}の形態{けいたい}を〕転換{てんかん}する◆取引の過程で不動産を動産に変えることなどを〔球技でパスを受けて〕ゴール[追加得点{ついか とくてん}]する ・
\\	何回もチャンスはあったが得点には結び付かなかった。【名】改心者{かいしん しゃ}、転向者{てんこう しゃ}、改宗者{かいしゅう しゃ}《コ》〔文字などの〕変換レベル5、発音《名》
\\	《動》
\\	、カナコンバート、変化《動》
\\	、分節
\\	彼は改心しました。 ・
\\	10年間に及ぶ改革によって中国経済の規模は倍増しました。	変化形 
\\	【自動】〔人が〕改心{かいしん}する、生活{せいかつ}を改める【他動】〔組織{そしき}などを〕改革{かいかく}する、刷新{さっしん}する〔誤りなどを〕正す、修正{しゅうせい}する〔人を〕矯正{きょうせい}する、改心{かいしん}させる ・
\\	彼は改心しました。〔石油{せきゆ}などを〕改質{かいしつ}する【名】〔組織などの〕改革{かいかく}、刷新{さっしん} ・
\\	10年間に及ぶ改革によって中国経済の規模は倍増しました。〔制度{せいど}などの〕修正{しゅうせい}、訂正{ていせい}〔人の〕矯正{きょうせい}、改心{かいしん}〈和製英語〉衣服{いふく}のリフォーム◆【標準英語】
\\	〈和製英語〉建物{たてもの}のリフォーム◆【標準英語】
\\	レベル3、発音
\\	、カナリフォーム、変化《動》
\\	、分節
\\	この車は日本仕様に改造された。 ・
\\	《契約書》本契約書は、正式の権限を有する当事者の代表者が署名した文書で相互に合意した場合を除き、いかなる方法によっても変更不可能とする。	変化形 
\\	【他動】〔改良するために部分的に〕修正{しゅうせい}[変更{へんこう}]する ・
\\	この車は日本仕様に改造された。 ・
\\	《契約書》本契約書は、正式の権限を有する当事者の代表者が署名した文書で相互に合意した場合を除き、いかなる方法によっても変更不可能とする。《文法》修飾{しゅうしょく}するレベル5、発音
\\	、カナモディファイ、変化《動》
\\	、分節
\\	このデータを記憶するメモリがいっぱいになった。	変化形 
\\	【他動】~を使い尽くす、使い果たす、不毛{ふもう}にする ・
\\	このデータを記憶するメモリがいっぱいになった。~を疲れ果てさせる、ひどく疲れさせる~を排出{はいしゅつ}する、~の中身{なかみ}を空にする~の空きスペースを使い尽くす、~を満杯{まんぱい}[一杯{いっぱい}]にする【名】排気{はいき}、排(出)ガス、排出{はいしゅつ}排気管{はいきかん}、排気孔{はいき こう}、排気装置{はいき そうち}レベル5、発音
\\	、カナイグゾースト、エグゾースト、変化《動》
\\	、分節
\\	変化形 
\\	【1自動】引っ張る神経{しんけい}をピリピリさせる懸命{けんめい}に努力{どりょく}する、全力{ぜんりょく}を尽す、力む歪{ひず}む、曲がる漉される、裏ごしされる【1他動】〔ひも・ロープ・針金{はりがね}・網などを〕ピンと張る〔神経{しんけい}を〕ピンと張る、研ぎ澄ます、緊張{きんちょう}させる〔筋肉{きんにく}を〕ピンと張る、〔身体{からだ}の一部{いちぶ}などを〕酷使{こくし}する〔物を〕ひずませる、変形{へんけい}させる〔体の一部{いちぶ}を〕使い過ぎて痛める[損傷{そんしょう}する・悪くする]~の筋を違える、~を捻挫{ねんざ}する~に負担{ふたん}をかける、~を精神的{せいしんてき}に参らせる~を濾す、裏ごしする【1名】ピンと張ること、緊張{きんちょう}(感)、精神的緊張{せいしん てき きんちょう}、緊張{きんちょう}の色〔使い過ぎによって〕体の一部{いちぶ}を痛めること負担{ふたん}、重圧{じゅうあつ}、試練{しれん}身体{からだ}に負担{ふたん}のかかる作業{さぎょう}[運動{うんどう}]、過労{かろう}歪{ひず}み、変形{へんけい}、しわ寄せ口調{くちょう}、話し方{はなしかた}、話しぶり、調子{ちょうし}、書きぶり、文体{ぶんたい}、旋律{せんりつ}【2名】種族{しゅぞく}、血統{けっとう}、家系{かけい}《遺伝》菌株{きんかぶ}、型〔ウイルスの〕株気質{きしつ}、素質{そしつ}、傾向{けいこう}レベル4、発音
\\	、カナストゥレイン、ストレイン、変化《動》
\\	私がどれほどホッとしているかは言葉では言い表せません。 ・
\\	このお茶は下痢によい。	変化形 
\\	【他動】〔不安{ふあん}・心配{しんぱい}・困難{こんなん}・苦痛{くつう}などを〕取り除く、軽減{けいげん}する、和らげる〔不安・心配などを取り除いて〕(人)を安心{あんしん}[ホッと]させる ・
\\	私がどれほどホッとしているかは言葉では言い表せません。救い出す、救済{きゅうさい}する~(の治療{ちりょう}[緩和{かんわ}])に良い[利く] ・
\\	このお茶は下痢によい。〔任務などから〕(人)を解放{かいほう}する~と交代{こうたい}する、交代して~を休ませる~を交代{こうたい}させる、解任{かいにん}する~から盗む、ふんだくる◆【同】
\\	レベル5、発音
\\	、カナリリーブ、変化《動》
\\	、分節
\\	日本の経済は持ち直してきている。 ・
\\	経済は、望まれているようには回復していない。 ・
\\	タービンの排出ガスから水が回収される。	変化形 
\\	【自動】正常{せいじょう}な状態{じょうたい}に戻る、〔病気などから〕回復{かいふく}する ・
\\	日本の経済は持ち直してきている。 ・
\\	経済は、望まれているようには回復していない。〔任務などから〕戻る【他動】正常{せいじょう}な状態{じょうたい}に戻す、埋め合わせる、取り戻す、〔損失{そんしつ}を〕償う回収{かいしゅう}する ・
\\	タービンの排出ガスから水が回収される。レベル2、発音
\\	、カナリカバー、変化《動》
\\	、分節
\\	彼は、法と秩序の回復と、複数政党による総選挙の実施を約束しました。	変化形 
\\	【他動】~を元の場所{ばしょ}[持ち主]に返す[戻す]~を元の状態{じょうたい}に戻す、~を修復{しゅうふく}[復元{ふくげん}]する(人)を回復{かいふく}させる、(人)に元気{げんき}を取り戻させる(人)を復位{ふくい}[復職{ふくしょく}]させる〔なくなっていたものを〕復活{ふっかつ}させる、復興{ふっこう}する ・
\\	彼は、法と秩序の回復と、複数政党による総選挙の実施を約束しました。レベル4、発音
\\	、カナリストア、変化《動》
\\	、分節
\\	変化形 
\\	【他動】(人)を(ひどく)驚かす、びっくりさせるレベル5、発音
\\	、カナアスタニッシュ、アストニッシュ、変化《動》
\\	、分節
\\	もてなしを受けるために来たわけではありませんので、どうぞお気遣いなく。 ・
\\	突然のお越しでしたので、おもてなしの用意もしておりません。	変化形 
\\	【他動】〔人や聴衆などを〕楽しませる、慰める〔飲食を供して人を〕もてなす、歓待{かんたい}する、接待{せったい}する ・
\\	もてなしを受けるために来たわけではありませんので、どうぞお気遣いなく。 ・
\\	突然のお越しでしたので、おもてなしの用意もしておりません。〔提案{ていあん}・考えなどを〕受け入れる、考慮{こうりょ}する、検討{けんとう}する〔ある感情{かんじょう}・希望{きぼう}などを〕心に抱くレベル3、発音
\\	、カナエンターテイン、エンタテイン、変化《動》
\\	、分節
\\	面白くない。◆自分の気に入らないこと[もの]に対して。	変化形 
\\	【他動】~を面白{おもしろ}がらせる、楽しませる ・
\\	面白くない。◆自分の気に入らないこと[もの]に対して。レベル4、発音
\\	、カナアミューズ、変化《動》
\\	、分節
\\	変化形 
\\	【自動】大喜びする、楽しむ【他動】~を大喜びさせる、楽しませる【名】楽しみ、楽しいこと、歓喜{かんき}、(大)喜びレベル3、発音
\\	、カナデライト、ディライト、変化《動》
\\	、分節
\\	変化形 
\\	【他動】~をほめる、称賛{しょうさん}する、(神を)たたえる、(ほめて)おだてる[持ち上げる]【名】称賛{しょうさん}、褒めること、賛美{さんび}レベル2、発音
\\	、カナプレイズ、変化《動》
\\	私の分も彼にお祝いを言ってください。 ・
\\	彼は心から祝ってくれている。 ・
\\	〔結婚すると聞いて〕それは彼はめでたいね。 ・
\\	あなたにおめでとうを言いたい。 ・
\\	売上目標をついに達成した時、私たちはシャンパンでお祝いした。 ・
\\	全然おめでたくないよ。◆【用法】まずい事態なのに逆におめでとうと言われて。	変化形 
\\	【他動】〔人を〕祝う、〔人に〕祝辞{しゅくじ}[祝いの言葉{ことば}]を述べる ・
\\	私の分も彼にお祝いを言ってください。 ・
\\	彼は心から祝ってくれている。 ・
\\	〔結婚すると聞いて〕それは彼はめでたいね。 ・
\\	あなたにおめでとうを言いたい。 ・
\\	売上目標をついに達成した時、私たちはシャンパンでお祝いした。 ・
\\	全然おめでたくないよ。◆【用法】まずい事態なのに逆におめでとうと言われて。
\\	運良くできたことを喜ぶ、威張る、満足{まんぞく}している、一人{ひとり}で得意{とくい}がるレベル6、発音
\\	、カナコングラチュレイト、変化《動》
\\	、分節
\\	変化形 
\\	【自動】式典{しきてん}を挙行{きょこう}する祝杯{しゅくはい}を挙げる、浮かれ騒ぐ【他動】〔誕生日・特別な出来事などを〕祝う、祝賀する、記念{きねん}する〔祭典・結婚式・パーティーなどを〕挙行{きょこう}する、執り行う称賛{しょうさん}する、褒めたたえるレベル3、発音
\\	、カナセレブレイト、変化《動》
\\	、分節
\\	あいつに本当のことをいうなんて、おまえすごいなあ。	変化形 
\\	【他動】~を称賛{しょうさん}[敬服{けいふく}]する ・
\\	あいつに本当のことをいうなんて、おまえすごいなあ。〔皮肉{ひにく}を込めて〕~には恐れ入る、~とは感心{かんしん}するレベル2、発音
\\	、カナアドゥマイア、アドマイアー、アドマイヤー、変化《動》
\\	、分節
\\	《契約書》いかなる裁定も最終的かつ両当事者を拘束するものとする。	変化形 
\\	【他動】〔賞などを〕与える[授与{じゅよ}する]〔選挙{せんきょ}で〕(人)に勝利{しょうり}を与える[もたらす]〔裁判で賠償金の支払いを〕裁定{さいてい}する【名】賞、賞品{しょうひん}、賞金{しょうきん}裁定{さいてい}、仲裁判断{ちゅうさい はんだん} ・
\\	《契約書》いかなる裁定も最終的かつ両当事者を拘束するものとする。レベル4、発音
\\	、カナアウォード、アワード、変化《動》
\\	、分節
\\	努力は報われる。 ・
\\	苦労したかいがあった。 ・
\\	彼の逮捕につながる情報に対して5000ドルの賞金提供の申し出があった。 ・
\\	これは、うまみのある仕事です。	変化形 
\\	【他動】~に報いる、褒美{ほうび}[報酬{ほうしゅう}]を与える、(人)に見返り{みかえり}を与える ・
\\	努力は報われる。 ・
\\	苦労したかいがあった。【名】〔善い行いなどに対する〕褒美{ほうび}、褒賞{ほうしょう}〔情報提供{じょうほう ていきょう}などに対する〕報奨金、謝礼金{しゃれい きん} ・
\\	彼の逮捕につながる情報に対して5000ドルの賞金提供の申し出があった。〔行為{こうい}の結果{けっか}としての〕利益{りえき}、恩恵{おんけい} ・
\\	これは、うまみのある仕事です。《心理学》報酬{ほうしゅう}レベル3、発音
\\	、カナリウォード、リワード、変化《動》
\\	、分節
\\	この映画はもっと興行成績が良くてもいいはずだ。 ・
\\	ラウベ氏がその賞を受けるに値すると私は確信しています。 ・
\\	身に余る光栄です。 ・
\\	彼女には当然の報いだ!/自業自得です。◆【同】
\\	囚人[捕虜]でさえ、こんな扱いは受けない。 ・
\\	こんな目に遭うなんて、私が何をしたというのか。◆【場面】理不尽と思える災難に遭って。	変化形 
\\	【自動】〔報い・賞などを受けるに〕ふさわしい、値する ・
\\	この映画はもっと興行成績が良くてもいいはずだ。【他動】~を受けるに値する[足る]、~の価値{かち}がある、~にふさわしい ・
\\	ラウベ氏がその賞を受けるに値すると私は確信しています。 ・
\\	身に余る光栄です。 ・
\\	彼女には当然の報いだ!/自業自得です。◆【同】
\\	囚人[捕虜]でさえ、こんな扱いは受けない。 ・
\\	こんな目に遭うなんて、私が何をしたというのか。◆【場面】理不尽と思える災難に遭って。レベル3、発音
\\	、カナディザーブ、デザーブ、変化《動》
\\	、分節
\\	君は私の貴重な時間を無駄にしている。	イディオムやフレーズ 
\\	【形】高価{こうか}な、貴重{きちょう}な、重要{じゅうよう}な、大事{だいじ}な ・
\\	君は私の貴重な時間を無駄にしている。尊い、大切{たいせつ}な、かわいい凝り過ぎた、気取った【副】すごくレベル2、発音
\\	、カナプレシャス、分節
\\	ある種の番組は米国の音楽業界に根本的な変化をもたらした。	変化形 
\\	《複》
\\	【名】基本{きほん}、原理{げんり}◆通例
\\	《音楽》基音{きおん}《物理》基本波{きほんは}【形】基本的{きほん てき}な、基礎的{きそ てき}な根本的{こんぽんてき}な、抜本的{ばっぽんてき}な ・
\\	ある種の番組は米国の音楽業界に根本的な変化をもたらした。《音楽》〔和音{わおん}が〕基本形{きほん けい}の◆和音の根音が、最低音である構成を指す。《物理》〔周波数成分{しゅうはすう せいぶん}の〕基本波{きほんは}のレベル4、発音
\\	、カナファンダメンタル、変化《複》
\\	、分節
\\	彼は、米国がソマリアで極めて重要な役割を果たしていると確信している。		【形】命の、生命{せいめい}に関する生命維持{せいめい いじ}に必要{ひつよう}な、延命{えんめい}させる元気{げんき}な、生き生きした元気{げんき}づける、生き生きさせる〔存在{そんざい}に〕必須{ひっす}の、不可欠{ふかけつ}な ・
\\	彼は、米国がソマリアで極めて重要な役割を果たしていると確信している。致命的{ちめい てき}な、死を招く《生物》生体{せいたい}のレベル5、発音
\\	、カナバイタル、分節
\\	変化形 
\\	《複》
\\	【名】目上{めうえ}の人、上司{じょうし}◆【用法】通例
\\	◆【対】
\\	《印刷》上付き文字{もじ}◆【同】
\\	〔技能{ぎのう}などの点で〕上を行く人、上手、優れた人修道院長{しゅうどういん ちょう}◆普通
\\	【形】〔地位{ちい}や立場{たちば}が〕上位{じょうい}の、上官{じょうかん}の〔品質{ひんしつ}が〕上等{じょうとう}の、上質{じょうしつ}の〔知性{ちせい}や能力{のうりょく}などが〕優れた、秀でた〔数や量が〕勝る、多い〔位置{いち}が〕高い、上にある〔態度{たいど}などが〕見下{みくだ}したような、傲慢{ごうまん}な〔人が〕超然{ちょうぜん}としている、動じない、無関心{むかんしん}な《植物》〔子房{しぼう}が〕上位{じょうい}の◆花被
\\	より上に位置する。《印刷》上付の◆【同】
\\	《解剖》上の、上側{うわがわ}の◆体の部位が頭に近いこと。《天文》〔惑星{わくせい}が〕外側{そとがわ}の◆太陽から見て軌道が地球の外側にある。《論理学》〔名辞{めいじ}や命題{めいだい}が〕包括的{ほうかつ てき}なレベル5、発音
\\	、カナシューペリア、スピアリア、変化《複》
\\	、分節
\\	変化形 
\\	《複》
\\	【名】劣った人[物]、下の者下付き文字{もじ}【形】〔地位{ちい}や立場{たちば}が〕下位{かい}の〔品質{ひんしつ}が〕劣る、劣等{れっとう}の〔知性{ちせい}や能力{のうりょく}などが〕凡庸{ぼんよう}な、二流{にりゅう}の〔位置{いち}が〕低い、下にある《植物》〔子房{しぼう}が〕下位{かい}の◆花被
\\	より下に位置する。《印刷》下付{かふ}の◆【同】
\\	《解剖》下の、下側{したがわ}の◆体の部位が頭から遠いこと。《天文》〔惑星{わくせい}が〕内側{うちがわ}の◆軌道が太陽と地球の間にある。レベル5、発音
\\	、カナインフィアリア、インフェリア、変化《複》
\\	、分節
\\	変化形 
\\	《複》
\\	【名】年上{としうえ}の人、年長者{ねんちょう しゃ}、先輩{せんぱい}、先達{せんだつ}高齢者{こうれいしゃ}、老人{ろうじん}〈米〉〔大学{だいがく}・4年制高校{ねんせい こうこう}などの〕4年生{ねんせい}、〔3年制高校{ねんせい こうこう}などの〕3年生{ねんせい}【形】年上{としうえ}の〔役職などが〕上級{じょうきゅう}[上席{じょうせき}・高位{こうい}・上位{じょうい}]の~・シニア◆しばしば
\\	。親族で同姓同名の二人を区別するため(主に父と同じ名を息子に付けた場合)、年上の人の名の末尾に付加する語。◆【参考】
\\	◆【略】
\\	レベル3、発音
\\	、カナシーニア、シーニャ、シニア、スィーニァ、変化《複》
\\	、分節
\\	後輩が一人、交通事故で死んだ。	変化形 
\\	《複》
\\	【名】
\\	年下の人、年少者、後輩{こうはい} ・
\\	後輩が一人、交通事故で死んだ。〈米〉〔高校や大学の〕3年生{ねんせい}、〔3年制高校などの〕2年生【形】年下{としした}の、年少{ねんしょう}の、若手{わかて}の~・ジュニア◆しばしば
\\	。親族で同姓同名の二人を区別するため(主に父と同じ名を息子に付けた場合)、年下の人の名の末尾に付加する語。例えば
\\	は
\\	の子。しばしばミドルネームまで含めると完全に同名ではない場合にも用いる。まれに女性の名前でも用いる。◆【略】
\\	◆【参考】
\\	レベル1、発音
\\	、カナジューニァ、ジューニア、変化《複》
\\	、分節
\\	変化形 
\\	《複》
\\	【1名】
\\	年長者{ねんちょう しゃ}年配者{ねんぱい しゃ}、お年寄り〔部族{ぶぞく}などの〕長老{ちょうろう}【1形】年長{ねんちょう}の、年上{としうえ}の【2名】《植物》ニワトコ◆ニワトコ属の落葉低木の総称レベル2、発音
\\	、カナエルダー、変化《複》
\\	、分節
\\	《マニュアル》必要であれば、もう片方の手でパネルを押さえてください。	変化形 
\\	【間投】落ち着いて、しっかり《海事》ようそろ◆そのままの進路で進むよう、船長が操舵手に対して行う指示。【自他動】しっかり押さえる、安定{あんてい}させる、固定{こてい}させる ・
\\	《マニュアル》必要であれば、もう片方の手でパネルを押さえてください。【名】〈話〉ステディー、決まった恋人{こいびと}◆【語源】19世紀の終わり頃からアメリカで使われるようになった言葉。《機械》固定振れ止め◆【同】
\\	【形】〔場所{ばしょ}や位置{いち}が〕固定{こてい}した、動かない〔動作{どうさ}などが〕揺るぎない、確固{かっこ}とした〔流れや発達{はったつ}の程度{ていど}が〕一定{いってい}の、不変{ふへん}の、着実{ちゃくじつ}な〔感情{かんじょう}や性格{せいかく}などが〕安定{あんてい}した、落ち着いた〔人や態度{たいど}などが〕真面目{まじめ}な、固い〔理念{りねん}や目標{もくひょう}などが〕不動{ふどう}の、しっかりした〔習慣{しゅうかん}などが〕決まった、いつものレベル3、発音
\\	、カナステディ、変化《形》
\\	、《動》
\\	、分節
\\	彼は1990年に高砂部屋に入った。	変化形 
\\	【1形】しっかりした、安定{あんてい}した、断固{だんこ}とした、不変{ふへん}の【2自他動】馬小屋{うまごや}に入る、馬小屋{うまごや}に入れる【2名】家畜小屋{かちく ごや}、きゅう舎、馬小屋{うまごや}、馬屋{うまや}〔同業者{どうぎょうしゃ}の〕集まり◆通例同じ会社などに所属している人の集まり◆【参考】
\\	訓練所{くんれん じょ}◆同じ会社などに所属する人を集めて訓練する場所。相撲部屋{すもう べや} ・
\\	彼は1990年に高砂部屋に入った。レベル4、発音
\\	、カナステイブル、変化《動》
\\	、分節
\\	100人が入れる会議室を確保してください。 ・
\\	何よりも先に、まずは、仕事を確保する必要があります。 ・
\\	離陸前にシートベルトを正しく締めなければなりません。 ・
\\	私は、あなたと一緒にいると、いつも安心していられます。	変化形 
\\	【自動】安全{あんぜん}である【他動】~を確保{かくほ}する、~を確実に手に入れる ・
\\	100人が入れる会議室を確保してください。 ・
\\	何よりも先に、まずは、仕事を確保する必要があります。〔物を(~に)〕固定{こてい}する、締め付ける ・
\\	離陸前にシートベルトを正しく締めなければなりません。~を請け合う、保証{ほしょう}する、引き受ける~を保護{ほご}する、守る【形】不安{ふあん}のない、恐れる必要{ひつよう}のない、安心{あんしん}していられる ・
\\	私は、あなたと一緒にいると、いつも安心していられます。しっかり固定{こてい}した、動かないようにした〔崩れたり壊れたりしない〕頑丈{がんじょう}な、しっかりした〔情報{じょうほう}の漏洩{ろうえい}や盗聴{とうちょう}などに対して〕機密{きみつ}がしっかりした、安全{あんぜん}な〔損失{そんしつ}や失敗{しっぱい}のない〕安全{あんぜん}な、確実{かくじつ}な〔攻撃{こうげき}や盗難{とうなん}に対して〕堅牢{けんろう}な、安全{あんぜん}な〔成功{せいこう}や達成{たっせい}が〕確実{かくじつ}な、保証{ほしょう}されたレベル4、発音
\\	、カナスィキュア、セキュアー、変化《動》
\\	、分節
\\	死と税金は避けられない。 ・
\\	戦争は決して避けられないものではない。 ・
\\	その山火事は例年になく長い乾期のために起きるべきして起きたと考えられている。		【名】
\\	不可避{ふかひ}のもの、必ず起きること【形】〔出来事{できごと}などが〕防げない、避けられない、免れない ・
\\	死と税金は避けられない。 ・
\\	戦争は決して避けられないものではない。〔結末{けつまつ}などが〕当然{とうぜん}の、予測{よそく}できる、必然的{ひつぜん てき}な ・
\\	その山火事は例年になく長い乾期のために起きるべきして起きたと考えられている。
\\	〈話〉いつもの、お決まりの、お定まりのレベル4、発音
\\	、カナイネビタブル、インエビタブル、分節
\\	キリスト教において神は絶対の存在である。 ・
\\	彼の新しい仕事は、誠心誠意力を注ぐことが求められる。 ・
\\	そんなの全くナンセンスだ。	変化形 
\\	《複》
\\	【名】絶対的{ぜったいてき}なもの[規則{きそく}・事象{じしょう}・原理{げんり}]、絶対不変{ぜったい ふへん}の性質{せいしつ}[概念{がいねん}・基準{きじゅん}]絶対的存在{ぜったい てき そんざい}、宇宙{うちゅう}、神【形】絶対{ぜったい}の、絶対的{ぜったいてき}な◆【反】
\\	キリスト教において神は絶対の存在である。完全{かんぜん}な、完全無欠{かんぜん むけつ}の、全くの◆【類】
\\	彼の新しい仕事は、誠心誠意力を注ぐことが求められる。 ・
\\	そんなの全くナンセンスだ。疑問{ぎもん}の余地{よち}のない、明白{めいはく}な、確かな制約{せいやく}のない、無条件{むじょうけん}の、無制限{むせいげん}の専制的{せんせい てき}な、独裁的{どくさいてき}な、独断的{どくだんてき}な《文法》独立{どくりつ}の、遊離{ゆうり}したレベル3、発音
\\	、カナアブソリュート、アブソルート、変化《複》
\\	、分節
\\	【形】起こりそうな、ありそうな、確からしいもっともらしい、まことしやかなレベル3、発音
\\	、カナプラバブル、プロバブル、分節
\\	休暇の代わりになる別の方法としては、夜に熟睡することです。 ・
\\	もし、これらの案のどれもがあなたにとって魅力的でなければ、そちらから代案をお出しいただけますか? ・
\\	その新エネルギー源は、化石燃料に代わるものの一つとして期待されています。 ・
\\	私たちには他に選択肢がありません。 ・
\\	設計者はさまざまな選択肢の中から選択を行わなければならない。 ・
\\	彼らは降伏か死かの選択を迫られた。 ・
\\	他のご都合の良い日をご提案いただけますでしょうか。 ・
\\	私たちの新製品の販売を促進するために、もう一つ別の戦略計画を提案したいと思います。	変化形 
\\	《複》
\\	【名】別の可能性{かのう せい}、取って代わるもの、代替{だいたい}手段{しゅだん}[案]◆【同】
\\	◆「どれか選択できます」の意味 ・
\\	休暇の代わりになる別の方法としては、夜に熟睡することです。 ・
\\	もし、これらの案のどれもがあなたにとって魅力的でなければ、そちらから代案をお出しいただけますか? ・
\\	その新エネルギー源は、化石燃料に代わるものの一つとして期待されています。選択肢{せんたくし}、オプション◆【用法】選択肢の数は二つの場合だけを正しいとし、三つ以上の場合は誤りであるとする人もいる。 ・
\\	私たちには他に選択肢がありません。 ・
\\	設計者はさまざまな選択肢の中から選択を行わなければならない。〔二つのものの〕選択{せんたく}の可能性{かのう せい}、選択{せんたく}すべき状況{じょうきょう} ・
\\	彼らは降伏か死かの選択を迫られた。【形】〈英〉代わりの、別の、他に取り得る◆【同】〈米〉
\\	◆「どれか選択できます」の意味 ・
\\	他のご都合の良い日をご提案いただけますでしょうか。 ・
\\	私たちの新製品の販売を促進するために、もう一つ別の戦略計画を提案したいと思います。〔どちらか一方{いっぽう}を〕選択{せんたく}する、選ぶべき〔文化{ぶんか}や生活様式{せいかつ ようしき}が〕伝統{でんとう}[主流{しゅりゅう}]からはずれた、オルタナティブな《論理学》選言的{せんげん てき}な◆【同】
\\	レベル4、発音
\\	、カナオールタネイティブ、オールターナティブ、変化《複》
\\	、分節
\\	この日本語に相当する英語はない。 ・
\\	彼らの貢献度は同等です。 ・
\\	あなたは現在の給与の7%に相当する一時金を受け取ることになります。 ・
\\	質量とエネルギーは等価である。◆アルバート・アインシュタインが示した考え。	変化形 
\\	《複》
\\	【名】〔効果・価値などが〕同等{どうとう}[同意義{どういぎ}]のもの ・
\\	この日本語に相当する英語はない。《化学》当量{とうりょう}、等価重量{とうか じゅうりょう}◆【同】
\\	【形】〔効果{こうか}や価値{かち}などが〕同じ、同等{どうとう}の、同意義{どういぎ}の◆【略】
\\	彼らの貢献度は同等です。〔働きなどが〕相当{そうとう}する、等質{とうしつ}の ・
\\	あなたは現在の給与の7%に相当する一時金を受け取ることになります。《数学》〔図形{ずけい}が〕同面積{どう めんせき}の、等積{とうせき}の《数学》〔数式{すうしき}が〕等価{とうか}の ・
\\	質量とエネルギーは等価である。◆アルバート・アインシュタインが示した考え。《数学》〔集合{しゅうごう}が〕対等{たいとう}の《化学》〔結合{けつごう}が〕等価{とうか}のレベル6、発音
\\	、カナイクウィ
\\	ヴァレントゥ、変化《複》
\\	、分節
\\	運賃は釣り銭のいらぬようお願いします。/お釣りのないようにしてください。◆バスなどに掲示される表現。	変化形 
\\	【他動】~を要求{ようきゅう}する、~を強要{きょうよう}する、~を必要{ひつよう}とする〔力ずくで〕~を取り立てる、〔強制的{きょうせい てき}に〕~を取り出す【形】正確{せいかく}な、的確{てきかく}な ・
\\	運賃は釣り銭のいらぬようお願いします。/お釣りのないようにしてください。◆バスなどに掲示される表現。精密{せいみつ}な、きちょうめんな厳しいレベル2、発音
\\	、カナイグザクト、エグザクト、変化《動》
\\	、分節
\\	【形】正確{せいかく}な、ぴったりの、緻密{ちみつ}な、明確{めいかく}な、的確{てきかく}な、まさにその、ちょうどその〔サイズや量などが〕ぴったりとした堅苦{かたぐる}しい、厳格{げんかく}なレベル4、発音
\\	、カナプリサイス、分節
\\	サルが逃げた。◆おりなどから逃げてうろうろしている状態。	変化形 
\\	【自動】(解き)放つ、離れる【他動】緩{ゆる}める、解き放す、放つ〔ミサイルなどを〕撃つ、発射{はっしゃ}する、飛ばす【名】緩み、結末{けつまつ}、成り行き〔矢の〕発射{はっしゃ}ルース、裸石◆研磨済みで枠入りされていない宝石【形】ゆるい、ルーズな、緩んだ、ほぐれた、ゆったりとした、きつくない、ダブダブするしまりのない、だらしのない、ふしだらな、無規律{むきりつ}な、軽い、身持ちが悪い自由{じゆう}な、解き放された、リラックスした ・
\\	サルが逃げた。◆おりなどから逃げてうろうろしている状態。容器{ようき}に入っていない、遊離{ゆうり}した、束ねていない、粘着性{ねんちゃくせい}のない、字義通{じぎ どお}りでない〔歯が〕グラグラするレベル3、発音
\\	、カナルース、ルーズ、変化《形》
\\	、《動》
\\	変化形 
\\	《複》
\\	【名】臨時雇い(労働者{ろうどうしゃ})《軍事》待機兵{たいき へい}◆正規の部署に配属される前に、一時的に別の部隊で働く兵士。
\\	カジュアル衣料{いりょう}【形】〔出来事{できごと}が〕偶然{ぐうぜん}の、思いがけない〔行動{こうどう}が〕不用意{ふようい}の、思い付きの、無計画{むけいかく}な〔行動{こうどう}が〕うわべ[形]だけの、徹底{てってい}しない〔出来事{できごと}が〕時々{ときどき}の、たまに起きる〔仕事{しごと}が〕臨時雇い{りんじやとい}の〔人や物事{ものごと}に〕親しくない、詳しくない〔物事{ものごと}に〕寛容{かんよう}な、厳しくない〔人や行動{こうどう}が〕無関心{むかんしん}な、無頓着{むとんちゃく}な〔人や行動{こうどう}が〕冷静{れいせい}な、平然{へいぜん}とした〔雰囲気{ふんいき}などが〕形式張らない、ざっくばらんな〔衣類{いるい}が〕カジュアルな、普段着{ふだんぎ}のレベル3、発音
\\	、カナカジュアル、キャジュアリ、キャジュアル、分節
\\	君が何を考えていようと全く気にしちゃいない。	変化形 
\\	《複》
\\	【名】平原{へいげん}、平野{へいや}平原地帯{へいげん ちたい}、大平原地帯{だい へいげん ちたい}◆《米国》ミシシッピイ川周辺の大平原地帯【形】平易{へいい}な、明らかな、飾り気のない、率直{そっちょく}な、単純{たんじゅん}な、シンプルな、分かりやすい、はっきり聞こえる質素{しっそ}な、簡素{かんそ}な、飾りのない、無地{むじ}の、地味{じみ}な明白{めいはく}な味の付いていない、(味が)あっさりした【副】〈話〉〔強めの意味{いみ}で〕全く、全然{ぜんぜん} ・
\\	君が何を考えていようと全く気にしちゃいない。〈話〉明らかに、はっきりとレベル2、発音
\\	、カナプレイン、変化《複》
\\	《形》
\\	彼は国王の権勢といかにも幸福そうな生活への敬服の念をしばしば表明しました。		【形】見えている、容易{ようい}に[はっきり]見えるすぐに分かる、明らかな、明白{めいはく}な〔実際{じっさい}には違っても〕一見{いっけん}すると[あたかも]~のような、見掛{みか}けの ・
\\	彼は国王の権勢といかにも幸福そうな生活への敬服の念をしばしば表明しました。《物理》見掛{みか}けの《心理学》仮現{かげん}の《法律》〔王位{おうい}や爵位{しゃくい}の相続権{そうぞくけん}が〕法定推定{ほうてい すいてい}の◆【参考】
\\	レベル4、発音
\\	、カナアパラント、アパレント、アペアラント、アペアレント、変化《形》
\\	、分節
\\	その事故についての彼らの説明が明確だったので彼は満足しました。 ・
\\	確かな情報をすぐにご連絡します。		【形】はっきりと限定{げんてい}された、明確{めいかく}な、的確{てきかく}な、確かな、決め手となる、決定的{けっていてき}な、間違いなく、はっきりした、一定{いってい}の、明確{めいかく}な限界{げんかい}のある ・
\\	その事故についての彼らの説明が明確だったので彼は満足しました。 ・
\\	確かな情報をすぐにご連絡します。限定的{げんていてき}なレベル4、発音
\\	、カナデフィニット、ディフィニット、分節
\\	変化形 
\\	【自動】気が遠くなる、気絶{きぜつ}する、卒倒{そっとう}する、失神{しっしん}する【名】気絶{きぜつ}、失神{しっしん}【形】〔記憶などが〕おぼろげな、ぼんやりした、薄れた〔声・音などが〕かすかな〔望みなどが〕かすかな〔色などが〕淡い、ぼやけた〔光などが〕ほのかな〔体力・気力などが〕弱々{よわよわ}しい気の弱いレベル4、発音
\\	、カナフェイント、変化《形》
\\	、《動》
\\	変化形 
\\	【形】〔意味{いみ}や意図{いと}が〕曖昧{あいまい}な、不明瞭{ふめいりょう}な〔考えや表現{ひょうげん}が〕あやふやな、まとまらない〔記憶{きおく}などが〕ぼんやりした、曖昧{あいまい}な〔形や音が〕はっきりしない、ぼやけた〔情報{じょうほう}が〕確認{かくにん}されていない、不確{ふたし}かなレベル3、発音
\\	、カナベイグ、変化《形》
\\	これはちょっと唐突です。	イディオムやフレーズ 
\\	【形】突然{とつぜん}の、いきなりの、唐突{とうとつ}な、奇襲{きしゅう}の ・
\\	これはちょっと唐突です。レベル2、発音
\\	、カナサドゥン、サドン、分節
\\	患者の容体は一刻を争います。 ・
\\	事態が差し迫っていることに気付いた。		【形】即時{そくじ}の、即刻{そっこく}の〔現在{げんざい}の〕すぐ前[後]の差し迫った、緊急{きんきゅう}の ・
\\	患者の容体は一刻を争います。 ・
\\	事態が差し迫っていることに気付いた。〔空間的{くうかん てき}に〕接近{せっきん}した、すぐそばの、周辺{しゅうへん}の〔関係{かんけい}などが〕直近{ちょっきん}の、すぐ次の〔効果{こうか}などが〕直接{ちょくせつ}の、じかの《哲学》〔認識{にんしき}が経験{けいけん}から〕直接{ちょくせつ}のレベル2、発音
\\	、カナイミーディエット、イミーディアット、イミーディエント、イミディアット、分節
\\	今月5日付のレターをありがとうございました。 ・
\\	今月5日付の貴書を拝受致しました。	変化形 
\\	《複》
\\	【名】瞬間{しゅんかん}まさにその時[瞬間{しゅんかん}]インスタント食品{しょくひん}、インスタント飲料{いんりょう}【形】すぐの、即時{そくじ}の切迫{せっぱく}した、差し迫った〔飲食物{いんしょくぶつ}が〕インスタントの今月の ・
\\	今月5日付のレターをありがとうございました。 ・
\\	今月5日付の貴書を拝受致しました。レベル3、発音
\\	、カナインスタント、変化《複》
\\	、分節
\\	【名】《カトリック》〔ミサの〕昇階誦{しょうかいしょう}◆使徒書の朗読の後に唱えられる短い祈り。《カトリック》〔ミサの〕聖歌集{せいか しゅう}【形】漸進的{ぜんしん てき}な、段階的{だんかい てき}な、徐々{じょじょ}の、漸次{ぜんじ}の、逐次{ちくじ}の、緩やかなレベル4、発音
\\	、カナグラジュアル、グラデュアル、変化《複》
\\	、分節
\\	楽しいクリスマス休暇をお過ごしください。 ・
\\	良い旅を。◆これから飛行機に乗る人に対して言う。 ・
\\	彼らは、その都市をもっと住み心地の良い[住民にとってもっと快適な]場所にしようとしています。
\\	より
\\	の形の方が一般的。	変化形 
\\	【形】楽しい、愉快{ゆかい}な ・
\\	楽しいクリスマス休暇をお過ごしください。 ・
\\	良い旅を。◆これから飛行機に乗る人に対して言う。〔人の行いや様子{ようす}が〕感じの良い、親切{しんせつ}な心地良い、気持ちが良い ・
\\	彼らは、その都市をもっと住み心地の良い[住民にとってもっと快適な]場所にしようとしています。
\\	より
\\	の形の方が一般的。元気{げんき}な、生き生きしたレベル2、発音
\\	、カナプレザント、プレズント、変化《形》
\\	、分節
\\	彼は金離れのいい人です。 ・
\\	今日はお金があるから、奮発して200ドルあげよう。 ・
\\	私どもとの会合にお時間を割いてくださり、誠にありがとうございます。 ・
\\	《礼を述べる》そんなにしていただけるなんてありがとうございます。		【形】気前{きまえ}のよい、惜しみない、物惜しみしない ・
\\	彼は金離れのいい人です。 ・
\\	今日はお金があるから、奮発して200ドルあげよう。 ・
\\	私どもとの会合にお時間を割いてくださり、誠にありがとうございます。 ・
\\	《礼を述べる》そんなにしていただけるなんてありがとうございます。〔性格{せいかく}や行いが〕寛大{かんだい}な、寛容{かんよう}な、度量{どりょう}の広いたくさんの、豊富{ほうふ}な〔ワインが〕芳醇{ほうじゅん}な、芳香{ほうこう}のあるレベル4、発音
\\	、カナジェネラス、分節
\\	彼は金銭に関しては保守的だが、政治についてはリベラルだ。	変化形 
\\	《複》
\\	【名】進歩{しんぽ}[改革{かいかく}]主義者{しゅぎしゃ}、リベラル ・
\\	彼は金銭に関しては保守的だが、政治についてはリベラルだ。
\\	自由党員{じゆう とういん}◆カナダなどの
\\	のメンバー。【形】自由主義{じゆう しゅぎ}の、自由{じゆう}を認める、進歩的{しんぽてき}な、リベラルな〔人が〕気前{きまえ}の良い、寛大{かんだい}な、物惜しみしない◆【同】
\\	十二分{じゅうにぶん}な、たっぷりの、豊富{ほうふ}なレベル4、発音
\\	、カナリベラル、変化《複》
\\	、分節
\\	その物語の教訓は、悪いことはできないってことだ。 ・
\\	何もしないよりは不品行でいた方が良い。 ・
\\	そのホスピスのケア・プログラムは私にとって大きな精神的支えとなった。 ・
\\	寓話とはしばしば動物が登場する短い教訓物語である。 ・
\\	死後に罰を受けることを恐れて高潔な人生を送る必要はない。 ・
\\	天国に行けるということをほとんど疑いのないことと思っていた。	変化形 
\\	《複》
\\	【名】教訓{きょうくん} ・
\\	その物語の教訓は、悪いことはできないってことだ。金言{きんげん}、格言{かくげん}◆【同】
\\	品行{ひんこう}、身持ち{みもち}、道徳{どうとく}◆性的な行動規範を指すことが多い。 ・
\\	何もしないよりは不品行でいた方が良い。【形】道徳上{どうとくじょう}の、善悪{ぜんあく}の判断{はんだん}に関する〔個人{こじん}の〕良心{りょうしん}の、分別{ふんべつ}の〔物理的{ぶつり てき}ではなく〕精神的{せいしん てき}な、感情的{かんじょう てき}な ・
\\	そのホスピスのケア・プログラムは私にとって大きな精神的支えとなった。道徳的{どうとく てき}な、教訓的{きょうくん てき}な ・
\\	寓話とはしばしば動物が登場する短い教訓物語である。有徳{ゆうとく}の、高潔{こうけつ}な ・
\\	死後に罰を受けることを恐れて高潔な人生を送る必要はない。〔事実{じじつ}よりも〕見込み[信念{しんねん}]に基づいた ・
\\	天国に行けるということをほとんど疑いのないことと思っていた。レベル3、発音
\\	、カナモーラル、モラル、変化《複》
\\	、分節
\\	彼らは、無辜の人民[罪のない人々]を殺しました。
\\	の代わりに
\\	と言っても意味は同じ。	変化形 
\\	《複》
\\	【名】心の清い[無実{むじつ}の]人素朴{そぼく}な[経験{けいけん}が少ない]人間抜け{まぬけ}な[おめでたい]人幼い子ども罪のない人 ・
\\	彼らは、無辜の人民[罪のない人々]を殺しました。
\\	の代わりに
\\	と言っても意味は同じ。【形】《法律》無実{むじつ}の、潔白{けっぱく}な《法律》法定{ほうてい}の、法で認められた、遵法{じゅんぽう}の無害{むがい}の、危険{きけん}を及ぼさない率直{そっちょく}な、純真{じゅんしん}な、無邪気{むじゃき}な世間知らず{せけんしらず}な、うぶな、だまされやすい無知{むち}な、にぶい欠けている、不足{ふそく}しているレベル4、発音
\\	、カナイノセント、変化《複》
\\	、分節
\\	被告人を有罪とします。	変化形 
\\	【形】有罪{ゆうざい}の、犯罪的{はんざいてき}な、罪を犯した ・
\\	被告人を有罪とします。罪の意識{いしき}がある、後ろめたい、やましいレベル3、発音
\\	、カナギルティー、変化《形》
\\	、分節
\\	失礼でしょう?◆親が子供をたしなめるときなど。 ・
\\	「おまえのような貧乏人に何ができる」「こいつはご挨拶だな。/言ってくれるじゃないか。」
\\	変化形 
\\	"【形】〔人や行為{こうい}が〕不作法{ぶさほう}な、無礼{ぶれい}な ・
\\	失礼でしょう?◆親が子供をたしなめるときなど。 ・
\\	「おまえのような貧乏人に何ができる」「こいつはご挨拶だな。/言ってくれるじゃないか。」〔言葉{ことば}が〕下品{げひん}な、下卑{げび}た、侮辱的{ぶじょく てき}な〔人が〕野暮{やぼ}な、無学{むがく}な、教養{きょうよう}がない〔作られたものが〕粗雑{そざつ}な、未完成{みかんせい}の〔技術{ぎじゅつ}などが〕下手{へた}な、未熟{みじゅく}な〔人が〕不慣れ{ふなれ}な、訓練{くんれん}されていない〔取り扱いが〕雑な、乱暴{らんぼう}な〔材料{ざいりょう}が〕自然{しぜん}のままの、加工{かこう}されていない〔土地{とち}や経済{けいざい}などが〕未開{みかい}の、未発達{みはったつ}な〔考えなどが〕おおざっぱな、漠然{ばくぜん}とした〔あまりに〕突然{とつぜん}の、不意{ふい}の〈英〉頑健{がんけん}な、たくましいレベル3、発音
\\	、カナルード、変化《形》
\\	悪などというものは存在しない。	変化形 
\\	《複》
\\	【名】(邪)悪、不正{ふせい}、不道徳{ふどうとく} ・
\\	悪などというものは存在しない。悪事{あくじ}、不正{ふせい}な[不道徳{ふどうとく}な]こと悪の源、邪心{じゃしん}【形】道義{どうぎ}に反する、不道徳{ふどうとく}な、よこしまな不吉{ふきつ}な、不運{ふうん}の害を及ぼす、有害{ゆうがい}な悪魔{あくま}のような、邪悪{じゃあく}な悪意{あくい}ある、険悪{けんあく}な不快{ふかい}な、気持ちが悪いレベル3、発音
\\	、カナイーブル、変化《複》
\\	、《形》
\\	、分節
\\	日曜日だったので、私はのんびりとした朝を楽しみ[満喫し]ました。	変化形 
\\	【形】怠惰{たいだ}な、不精{ぶしょう}な、気を抜いてだらけさせる、のろのろした、眠気{ねむけ}を誘う《野球》(打球{だきゅう}に)力のない〔良い意味{いみ}で〕のんびり(と)した、リラックスした ・
\\	日曜日だったので、私はのんびりとした朝を楽しみ[満喫し]ました。レベル2、発音
\\	、カナレイジー、レイズィ、変化《形》
\\	、分節
\\	変化形 
\\	【形】〔人の扱いなどが〕非情{ひじょう}な、無情{むじょう}な、手厳{てきび}しい〔基準{きじゅん}などに対して〕厳密{げんみつ}な、厳守{げんしゅ}する〔態度{たいど}や声などが〕いかめしい、人を寄せ付けない〔実行{じっこう}や達成{たっせい}が〕厄介{やっかい}な、難儀{なんぎ}な〔形や様式{ようしき}が〕質素{しっそ}な、簡素{かんそ}な、飾り気がない〔天気{てんき}などが〕厳しい、極端{きょくたん}な〔苦痛{くつう}などが〕激しい、耐え難い〔病気{びょうき}などが〕深刻{しんこく}な、重大{じゅうだい}なレベル4、発音
\\	、カナシビア、スィビア、セビア、変化《形》
\\	、分節
\\	その先生は子どもに対してとても厳しい。 ・
\\	彼は、忠実なカソリック教徒である。 ・
\\	健康にはどうぞ注意してください。	変化形 
\\	【形】〔人が規則{きそく}などを守らせるのに〕厳しい、厳格{げんかく}な ・
\\	その先生は子どもに対してとても厳しい。〔規則{きそく}などが〕厳しい、厳重{げんじゅう}な〔人が原則{げんそく}などを守るのに〕忠実{ちゅうじつ}な、順守{じゅんしゅ}する ・
\\	彼は、忠実なカソリック教徒である。〔意味{いみ}などが〕正確{せいかく}な、厳密{げんみつ}な注意深い{ちゅういぶかい}、慎重{しんちょう}な ・
\\	健康にはどうぞ注意してください。完全{かんぜん}な、徹底的{てってい てき}な《植物》直立性{ちょくりつ せい}のレベル3、発音
\\	、カナストゥリクト、ストリクト、変化《形》
\\	変化形 
\\	【他動】~に勇敢{ゆうかん}に立ち向かう、(勇気{ゆうき}を出して)~に挑戦{ちょうせん}する、~を物ともしない【名】勇敢{ゆうかん}な人、勇士{ゆうし}◆【用法】
\\	は複数扱い。〔北米先住民族の〕戦士〈古〉ごろつき【形】勇敢{ゆうかん}な、勇ましい、肝が据わって派手{はで}な、目立つ優れた、立派なレベル2、発音
\\	、カナブレーブ、ブレイブ、変化《形》
\\	、《動》
\\	街に平穏が戻った。/街が落ち着きを取り戻しました。 ・
\\	瞑想{めいそう}すると大抵人はとても落ち着きます。 ・
\\	どうしてそんなに冷静で[落ち着いて]いられるの?	変化形 
\\	【自動】静まる、落ち着く【他動】静める、落ち着かせる、沈静化する、なだめる【名】静けさ、静止{せいし}冷静{れいせい}、落ち着き、平穏{へいおん} ・
\\	街に平穏が戻った。/街が落ち着きを取り戻しました。無風状態{むふう じょうたい}、凪{なぎ}《気象》平穏{へいおん}◆ビューフォート風力階級0の秒速0~0.3メートル未満の風。◆【参考】
\\	【形】〔動きがわずかしかないためほとんど物音がせず〕静かな〔海・気候などが〕穏やかな、静かな、無風{むふう}の ・
\\	瞑想{めいそう}すると大抵人はとても落ち着きます。〔人が動揺したりイライラしたりせずに〕冷静{れいせい}な、落ち着いた、穏和な、しとやかな、つつましい ・
\\	どうしてそんなに冷静で[落ち着いて]いられるの?〔気持ちが〕和む〔世の中が〕平穏{へいおん}なレベル3、発音
\\	、カナカーム、変化《形》
\\	、《動》
\\	恥ずかしがらずに彼女に話し掛けてみなさいよ。 ・
\\	恥ずかしがる必要ありません。	変化形 
\\	【1自動】〔驚きなどで〕ビクリとする、飛び上がる〔恐れなどから〕後ずさりする、尻込みする【1他動】~をよける、避ける【1名】後ずさり、飛びのき【1形】内気{うちき}な、はにかんだ、恥ずかしがりの◆気が弱くてやりたくてもできない、という否定的な意味。日本ではそれが美徳とされることもある◆【参考】
\\	恥ずかしがらずに彼女に話し掛けてみなさいよ。 ・
\\	恥ずかしがる必要ありません。臆病{おくびょう}な、気弱{きよわ}な、意気地{いくじ}がない用心深い{ようじんぶかい}、油断{ゆだん}しない、疑い深い{うたがいぶかい}遠慮{えんりょ}する、やりたがらない、乗り気でない不足{ふそく}した、欠けている《生物》〔動植物{どうしょくぶつ}が〕繁殖{はんしょく}しにくい【2自動】素早く投げる【2他動】~を素早く投げる【2名】〔素早く〕投げること〈話〉嘲{あざけ}り、冷やかし〈話〉試し、試み標的落とし、的当てレベル3、発音
\\	、カナシャイ、変化《形》
\\	、《動》
\\	イディオムやフレーズ 
\\	【形】惨めな、わずかな、悲惨{ひさん}な、ひどい状態{じょうたい}の、不幸{ふこう}な、哀れな、粗末{そまつ}な、みすぼらしい、恥ずべき、不愉快{ふゆかい}な、憂鬱{ゆううつ}な、見る影もない、救いようのない◆【語源】ラテン語
\\	(惨めな)〈豪〉けちなレベル4、発音
\\	、カナミゼラブル、分節
\\	低成長経済で労働者が不安になる。 ・
\\	来週のいつかに、ぜひお会いしたいと思っております。	イディオムやフレーズ 
\\	【形】〔これから起きることについて〕心配{しんぱい}して、気掛{きがか}かりで◆【用法】叙述用法で用いる。 ・
\\	低成長経済で労働者が不安になる。〔心配{しんぱい}な出来事{できごと}について〕切望{せつぼう}して、ぜひとも~したいと思って◆【用法】
\\	とほぼ同じ意味だが、
\\	が好ましい出来事を、
\\	が心配な出来事を待ち望む場合に使われる傾向が強い。叙述用法で用いる。 ・
\\	来週のいつかに、ぜひお会いしたいと思っております。〔出来事{できごと}などが〕心配{しんぱい}な、気に掛かる、気をもませる◆【用法】限定用法で用いる。レベル2、発音
\\	、カナアンクシャス、分節
\\	イディオムやフレーズ 
\\	【形】〔叙述用法{じょじゅつ ようほう}で〕しきりに~したがる、~を切望{せつぼう}した〔限定用法{げんてい ようほう}で〕熱心{ねっしん}な、待ちきれない様子{ようす}のレベル3、発音
\\	、カナイーガ、イガー、変化《形》
\\	、分節
\\	心は燃えても、肉体は弱い。◆聖書の言葉。	変化形 
\\	【形】
\\	〔求められれば〕~することをいとわない、~する気がある◆【用法】
\\	の形で用いられる。協力的{きょうりょく てき}な、熱心{ねっしん}な ・
\\	心は燃えても、肉体は弱い。◆聖書の言葉。〈限定用法で〉自分{じぶん}から進んで、快く[喜んで]やるレベル2、発音
\\	、カナウィリング、分節
\\	変化形 
\\	【動】
\\	の過去・過去分詞形【形】傾いた、斜めになった~したい気がする、~の傾向{けいこう}があるレベル5、発音
\\	、カナインクラインド、変化《動》
\\	、分節
\\	商売は不振です。/景気が悪い。	変化形 
\\	【自動】〔ナイフの切れなどが〕鈍くなる、鈍る、鈍化{どんか}する鈍感{どんかん}になる、〔痛みが〕和らぐ〔光・色などが〕くすむ、〔音が〕ぼんやりする【他動】〔ナイフの切れなどを〕鈍くする◆【対】
\\	〔光・色などを〕くすませる、〔音を〕ぼんやりさせる~を鈍感{どんかん}にする、〔痛みを〕和らげる【形】鈍い、動作{どうさ}がのろい、鈍感{どんかん}な、頭の回転{かいてん}が遅い〔天候が〕どんよりした〔色などが〕ぼんやりした、光沢{こうたく}のない、さえない〔刃が〕切れない〔本・話・仕事などが〕退屈{たいくつ}な、面白くない、つまらない味気{あじけ}ないありふれた苦痛{くつう}を和らげる〔商売などが〕不振{ふしん}な、暇な ・
\\	商売は不振です。/景気が悪い。レベル3、発音
\\	、カナダル、変化《形》
\\	、《動》
\\	あなた、そういうところは鋭いね。 ・
\\	鋭い! ・
\\	「今何時?」「9時ちょうど」 ・
\\	今日は2時ちょうどにベーカーさんと会うことになっている。
\\	を使う場合は、
\\	というように数字の前に付ける。
\\	変化形 
\\	《複》
\\	"【名】《音楽》嬰音{えいおん}[シャープ](記号{きごう})の付いた音
\\	〔ナイフ・はさみなど〕先のとがったもの、鋭利{えいり}なもの【形】鋭い、鋭敏{えいびん}な、(先が)尖っている、切れ味がいい急な〔輪郭などが〕はっきりした、くっきりした賢い、頭のいい、頭の切れる、抜け目のない、明敏{めいびん}な ・
\\	あなた、そういうところは鋭いね。 ・
\\	鋭い!とげとげしい、険のある、痛烈{つうれつ}な、辛辣{しんらつ}な、厳しい〔増加の程度などが〕激しい、著しいきちんとした、ファッショナブルな辛い、辛味{からみ}の強い《音楽》嬰音{えいおん}[シャープ](記号{きごう})の(ついた)【副】きっかり、ちょうど ・
\\	「今何時?」「9時ちょうど」 ・
\\	今日は2時ちょうどにベーカーさんと会うことになっている。
\\	を使う場合は、
\\	というように数字の前に付ける。レベル1、発音
\\	、カナシャープ、変化《形》
\\	、《複》
\\	カマトトぶるな。	イディオムやフレーズ 
\\	【形】無知{むち}な、意識{いしき}しない、無教養{むきょうよう}な ・
\\	カマトトぶるな。レベル4、発音
\\	、カナイグノラント、分節
\\	ばかにされないか、などと心配しないでください。	変化形 
\\	【形】〔人が〕愚かな、分別{ふんべつ}がない ・
\\	ばかにされないか、などと心配しないでください。〔話や考えなどが〕ばかげた、途方{とほう}もない〔話や考えなどが〕たわいない、くだらない〔人が〕ぼうぜんとした、ボーッとなったレベル2、発音
\\	、カナシリー、スィリ、変化《形》
\\	、分節
\\	変化形 
\\	【名】
\\	〈米話〉ハイビーム◆自動車などのヘッドライトの点灯状態で、真っすぐ前を強く照らすもの。◆【同】
\\	【形】輝く、輝いている、光っている、まぶしい◆強い光で明るい、晴れた、うららかな、晴朗{せいろう}な〔色が〕鮮やかな輝かしい、華やかな、きらびやかな〈話〉機転{きてん}の利く、利口{りこう}な、頭の良い[切れる]、頭脳明晰{ずのうめいせき}である快活{かいかつ}な、元気{げんき}のいいレベル2、発音
\\	、カナブライト、変化《形》
\\	【形】〔高度{こうど}な〕知的能力{ちてき のうりょく}を持った、知性{ちせい}がある理にかなった、合理的{ごうり てき}な、賢明{けんめい}な〈話〉知っている、知識{ちしき}がある、詳しい《コ》自動制御{じどう せいぎょ}ができる、インテリジェントな◆コンピューターによって、外界の変化に自動的に対応できるようになっていること。《コ》〔端末{たんまつ}が〕インテリジェントな◆データの入出力だけでなく保存や処理ができる、コンピューターとしての機能を持っていること。◆【対】
\\	レベル2、発音
\\	、カナインテリジェント、分節
\\	〔重要な問題について〕もう少し分別を持ちなさい。 ・
\\	分別を失うな。		【形】分別{ふんべつ}のある、思慮{しりょ}のある、常識{じょうしき}がある、賢明{けんめい}な、聡明{そうめい}な、気が利く ・
\\	〔重要な問題について〕もう少し分別を持ちなさい。 ・
\\	分別を失うな。目的{もくてき}にふさわしい[かなった]、理にかなっている、実用{じつよう}にかなった、実用本位{じつよう ほんい}の顕著{けんちょ}な気付いているレベル3、発音
\\	、カナセンシブル、センスィブル、分節
\\	その人物が真相を何もかも知っていると考えるのは筋が通っている。 ・
\\	チケットの値段が手頃だったため、4枚購入しました。		【形】〔人が〕理性的{りせい てき}な、分別{ふんべつ}がある ・
\\	その人物が真相を何もかも知っていると考えるのは筋が通っている。〔考え方などが〕合理的{ごうり てき}な、妥当{だとう}な〔考え方などが〕中庸{ちゅうよう}な、極端{きょくたん}でない〔値段{ねだん}などが〕それほど高くない、まあまあの◆値段が「安い」から「高い」順に形容詞を並べると、おおよそ
\\	(安い、粗末な)、
\\	(高くない)、
\\	(べらぼうな)となる。 ・
\\	チケットの値段が手頃だったため、4枚購入しました。〔品質{ひんしつ}などが〕まあまあ良い、悪くない〔分量{ぶんりょう}などが〕結構多い、まあまあのレベル2、発音
\\	、カナリーズナブル、リーゾナブル、分節
\\	金はどこでも有効なパスポートだ。 ・
\\	《コ》入力した名前は無効です。 ・
\\	チケットはこの5店に有効です。 ・90 
\\	選挙で投じられた票の90%は有効だった。		【形】根拠{こんきょ}がしっかりした、正当{せいとう}な、もっともな通用{つうよう}する、期限切れでない、有効{ゆうこう}な ・
\\	金はどこでも有効なパスポートだ。 ・
\\	《コ》入力した名前は無効です。 ・
\\	チケットはこの5店に有効です。《法律》合法的{ごうほう てき}な、有効{ゆうこう}な ・90 
\\	選挙で投じられた票の90%は有効だった。《法律》拘束力{こうそくりょく}がある、義務的{ぎむ てき}な意図{いと}した効果{こうか}が得られる、効果的{こうか てき}な《論理学》〔論理式{ろんりしき}が〕恒真{こうしん}の《論理学》〔演繹法{えんえきほう}における推論{すいろん}が〕妥当{だとう}な〈古〉健康的{けんこう てき}な、丈夫{じょうぶ}なレベル6、発音
\\	、カナバリッド、分節
\\	現在のところ、駅で利用できるタクシーがありません。 ・
\\	商品によってはお客様の国で発売されていないことがあります。 ・
\\	「もしもし。
\\	社のジェーン・ドーでございます。ロバート・スミス様をお願い致します」「少々お待ちくださいませ…。お待たせ致しました。スミスはただ今席を外しております」◆電話 ・
\\	山田さんはいらっしゃいますか?◆電話
\\	"【形】〔物が〕利用{りよう}[使用{しよう}]できる、入手{にゅうしゅ}できる、得られる ・
\\	現在のところ、駅で利用できるタクシーがありません。〔製品などが〕在庫{ざいこ}としてある、市販{しはん}されている、標準生産{ひょうじゅん せいさん}の ・
\\	商品によってはお客様の国で発売されていないことがあります。〔アパートなどが〕入居{にゅうきょ}できる、空いている〔人が〕手が空いている、応対できる、会うことができる、話をする時間がある〔会議{かいぎ}・電話{でんわ}などに人が〕出られる ・
\\	「もしもし。
\\	社のジェーン・ドーでございます。ロバート・スミス様をお願い致します」「少々お待ちくださいませ…。お待たせ致しました。スミスはただ今席を外しております」◆電話 ・
\\	山田さんはいらっしゃいますか?◆電話有効{ゆうこう}な、通用{つうよう}するレベル3、発音
\\	、カナアベイラブル、分節
\\	それは便利だね。 ・
\\	いつが一番都合がいいですか? ・
\\	月曜日はご都合はいかがですか。 ・
\\	あなたの都合のいいときにいつでもいらっしゃい。		【形】便利{べんり}な、使いやすい、使い勝手{かって}が良い、重宝{ちょうほう}な ・
\\	それは便利だね。都合{つごう}の良い、うってつけの、あつらえ向きの ・
\\	いつが一番都合がいいですか? ・
\\	月曜日はご都合はいかがですか。 ・
\\	あなたの都合のいいときにいつでもいらっしゃい。レベル3、発音
\\	、カナコンビーニァント、コンビニーエント、分節
\\	飛行機はあと1時間で到着の予定です。 ・
\\	は今月中に発売の予定です。 ・
\\	この書類は来週までだと思っていたら今週が締め切りだった。 ・
\\	当然受け取るべきボーナスを手にした。 ・
\\	私はいいものはいいと言っているだけです。	変化形 
\\	《複》
\\	【名】当然支払われるべきもの会費{かいひ}、手数料{てすうりょう}、税【形】~する予定{よてい}である、~することになっている、~になるはずである、もうすぐやって来る ・
\\	飛行機はあと1時間で到着の予定です。 ・
\\	は今月中に発売の予定です。期限{きげん}が来て、当然支払うべき、満期{まんき}の◆【用法】「(人)に対して当然支払うべき」の時は
\\	となることも、
\\	と
\\	が省略されることもある。◆【参考】
\\	この書類は来週までだと思っていたら今週が締め切りだった。 ・
\\	当然受け取るべきボーナスを手にした。〔子どもが〕生まれる予定{よてい}である正当{せいとう}な、当然{とうぜん}の、十分{じゅうぶん}な、〔尊敬{そんけい}が〕当然与えられるべきの ・
\\	私はいいものはいいと言っているだけです。【副】〔方角{ほうがく}・方向{ほうこう}が〕ちょうど、真~レベル3、発音
\\	、カナデュー、変化《複》
\\	姑が嫁をいじめるのは常識です。 ・
\\	時に愛は残酷なものです。	変化形 
\\	【形】残酷{ざんこく}な、むごい、残虐{ざんぎゃく}な、無慈悲{むじひ}な、非情{ひじょう}な ・
\\	姑が嫁をいじめるのは常識です。 ・
\\	時に愛は残酷なものです。厳しい、つらい、ひどい、苛酷{かこく}な、悲惨{ひさん}なレベル3、発音
\\	、カナクルーエル、クルアル、クルエル、変化《形》
\\	、分節
\\	【形】暴力的{ぼうりょくてき}な、乱暴{らんぼう}な、凶暴{きょうぼう}な、暴力{ぼうりょく}による、猛烈{もうれつ}な性格{せいかく}の、暴れる〔映画・小説・漫画・ゲームなどが〕暴力{ぼうりょく}シーン[描写{びょうしゃ}]の多い〔感情・言動などが〕激しい、強烈{きょうれつ}な〔解釈などが〕こじつけのレベル2、発音
\\	、カナバイオレントゥ、バイオレント、分節
\\	あんな嫌なやつには今まで出逢ったことがないよ。 ・
\\	~について何がそんなに恐ろしいのか。 ・
\\	嫌がらせにそんなことを言ってるんだろ。		【形】大変{たいへん}な、恐ろしい、ひどい、嫌な、すさまじい、不愉快{ふゆかい}な ・
\\	あんな嫌なやつには今まで出逢ったことがないよ。 ・
\\	~について何がそんなに恐ろしいのか。 ・
\\	嫌がらせにそんなことを言ってるんだろ。レベル2、発音
\\	、カナオーフル、オウフル、分節
\\	みっともないとこ見せちゃったわね。	変化形 
\\	【他動】〔容貌{ようぼう}を〕醜くする、見苦{みぐる}しくする【形】〔容貌{ようぼう}が〕醜い、不細工{ぶさいく}な、見苦{みぐる}しい、醜悪{しゅうあく}な ・
\\	みっともないとこ見せちゃったわね。不快{ふかい}な、嫌な〔道徳的{どうとくてき}に〕悪い、卑劣{ひれつ}な物騒{ぶっそう}な、厄介{やっかい}な、面倒{めんどう}な、〔天候{てんこう}が〕荒れ模様のレベル3、発音
\\	、カナアグラベイト、アグリー、変化《形》
\\	、分節
\\	フロリダの主な産物は何と何ですか。	変化形 
\\	《複》
\\	【名】長、支配者{しはいしゃ}、社長{しゃちょう}、主役{しゅやく}校長{こうちょう}、学長{がくちょう}〔活動などの〕主導者〔作業グループなどの〕主任主たる債務者{さいむしゃ}、第一債務者{だいいち さいむ しゃ}◆保証人に対して本人{ほんにん}◆代理人に対して元金{がんきん/もときん}、元本{がんぽん}《法律》正犯{せいはん}、主犯{しゅはん}【形】主な、主要{しゅよう}な、主たる、第一{だいいち}の、重要{じゅうよう}な ・
\\	フロリダの主な産物は何と何ですか。〔役職などが〕主席{しゅせき}[主任{しゅにん}・主幹{しゅかん}]の元金{がんきん/もときん}のレベル3、発音
\\	、カナプリンシパル、プリンスィパル、変化《複》
\\	、分節
\\	変化形 
\\	《複》
\\	【名】第一{だいいち}のこと、主要{しゅよう}な事物{じぶつ}、予備選挙{よび せんきょ}初列風切羽◆通例
\\	【形】第一{だいいち}の、第一義的{だいいち ぎ てき}な、第1位の、一次{いちじ}の主要{しゅよう}な、主~最初{さいしょ}の、初期{しょき}の、初級{しょきゅう}の、基礎的{きそてき}な、本来{ほんらい}の予備{よび}のレベル4、発音
\\	、カナプライマリー、プライメリ、変化《複》
\\	、分節
\\	変化形 
\\	【自動】〔将来の行動に向けて〕準備するたばこの下の葉を収穫する【他動】~を準備[用意]する〔火器に〕火薬をつめる、信管をつける〔気化器に〕ガソリンを入れる、〔ポンプに〕呼び水を入れる~に下塗りをする(人)にコーチする、(人)に事前に教える〈話〉(人)にたっぷり食わせる〔たばこの〕下の葉を収穫する【名】最盛期{さいせいき}、全盛(期){ぜんせい(き)}、盛り、青春{せいしゅん}《数学》素数◆【同】
\\	プライム[ダッシュ]記号
\\	(エー・ダッシュ)の
\\	記号のこと◆【同】
\\	〔フェンシングの〕第一{だいいち}の構え◆【参考】
\\	《カトリック》1時課{じか}◆【参考】
\\	【形】最も重要{じゅうよう}な、第1級の、主要{しゅよう}な、第一{だいいち}の、根本的{こんぽんてき}な一流{いちりゅう}の、最上等{さい じょうとう}の、最上{さいじょう}の、首位{しゅい}の、最高{さいこう}のレベル5、発音
\\	、カナプライム、変化《動》
\\	生物学者たちは原始的な生物に非常に興味を抱いている。		【名】原始人{げんしじん}、原始美術{げんし びじゅつ}【形】原始{げんし}の、太古{たいこ}の、初期{しょき}の、根源{こんげん}の、未発達{みはったつ}な、粗野{そや}な ・
\\	生物学者たちは原始的な生物に非常に興味を抱いている。旧式{きゅうしき}の、昔ながらのレベル4、発音
\\	、カナプリミティブ、変化《複》
\\	、分節
\\	【形】〔通常{つうじょう}と異なる〕異常{いじょう}な、途方{とほう}もない、桁外れ{けたはずれ}の〔非常{ひじょう}に〕目立{めだ}った、非凡{ひぼん}な、たぐいまれな〔出来事{できごと}などが〕特別{とくべつ}の、追加{ついか}の、臨時{りんじ}の〔人が〕特命{とくめい}の、特任{とくにん}の、員外{いんがい}のレベル4、発音
\\	、カナイクストロードネリ、エクストゥロードゥネリ、エクストラオーディナリー、エクストローデネリ、エクストロードネリ、分節
\\	米国は過去の圧倒的な力を失ったと彼は述べた。	変化形 
\\	【形】〔感情的{かんじょう てき}に人を〕圧倒{あっとう}する、打ちのめす〔力や数で〕圧倒{あっとう}[征服{せいふく}]する ・
\\	米国は過去の圧倒的な力を失ったと彼は述べた。〔大きさなどが〕圧倒的{あっとう てき}な、巨大{きょだい}なレベル5、発音
\\	、カナオーバーウェルミング、オーバーフェルミング、オウバーウェルミング、オウバーフェルミング、分節
\\	極端は極端を生む。 ・
\\	彼女は全く信じられないというような顔つきだった。 ・
\\	この仕事は非常な忍耐力が必要です。	変化形 
\\	《複》
\\	【名】〔物事の〕極度{きょくど}、極致{きょくち} ・
\\	極端は極端を生む。〔範囲{はんい}の一方{いっぽう}の〕極、端
\\	極端{きょくたん}な[思い切った]手段{しゅだん}《数学》〔比例{ひれい}の〕外項{がいこう}◆【同】
\\	《数学》〔級数{きゅうすう}の〕極致{きょくち}◆【同】
\\	【形】〔中心{ちゅうしん}から〕最遠{さいえん}の、最も端にある〔程度{ていど}が〕最も高い[強い]、極度{きょくど}の ・
\\	彼女は全く信じられないというような顔つきだった。 ・
\\	この仕事は非常な忍耐力が必要です。〔標準{ひょうじゅん}から見て〕行き過ぎた、極端{きょくたん}な〔手段{しゅだん}などが〕徹底的{てってい てき}な、急進的{きゅうしん てき}な〔スポーツが〕危険{きけん}をはらんだ、生命{せいめい}に関わるようなレベル4、発音
\\	、カナイクストリーム、エクストゥリーム、変化《複》
\\	、分節
\\	空気は音を伝える媒体である。	変化形 
\\	《複》
\\	【名】〔程度や位置などの〕中位{ちゅうぐらい}、中間{ちゅうかん}〔他のものを伝える〕媒体{ばいたい}、媒介物{ばいかいぶつ}、媒質{ばいしつ} ・
\\	空気は音を伝える媒体である。〔テレビなどの〕マス・メディア◆複数形は
\\	◆【用法】複数形の
\\	はしばしば単数形として扱われるが、これは間違いであると考える人の方が多い。〔言語{げんご}などの思考{しこう}や情報{じょうほう}の〕伝達手段{でんたつ しゅだん}〔死者{ししゃ}や霊と媒介{ばいかい}する〕霊媒{れいばい}、霊能者{れいのうしゃ}、ミディアム◆複数形は
\\	《コ》〔データを記録{きろく}する〕媒体{ばいたい}、メディア《生物》培地{ばいち}、培養基{ばいようき}《生物》〔生物{せいぶつ}の自然{しぜん}の〕生息{せいそく}場所{ばしょ}[環境{かんきょう}]◆【同】
\\	〔ペンキや顔料{がんりょう}の〕溶剤{ようざい}、展色剤{てんしょくざい}〔芸術家{げいじゅつか}の〕表現手段{ひょうげん しゅだん}、芸術領域{げいじゅつ りょういき}◆絵画や、彫刻などの芸術領域を指す。〔芸術家{げいじゅつか}の〕創作{そうさく}材料{ざいりょう}[素材{そざい}]◆絵画の絵の具や、彫刻の大理石などを指す。〔用紙{ようし}サイズの〕メディアム
\\	23インチ
\\	の大きさの用紙。【形】〔大きさ・程度{ていど}などが〕中間{ちゅうかん}の、中位{ちゅうぐらい}の〔肉の焼き具合が〕ミディアムのレベル4、発音
\\	、カナミーディァム、ミディアム、メディウム、変化《複》
\\	、分節
\\	変化形 
\\	《複》
\\	《複》
\\	【名】〔論争{ろんそう}などにおける〕中立(者){ちゅうりつ(しゃ)}〔戦争{せんそう}などにおける〕中立国(民){ちゅうりつ こく(みん)}〔車のギアの〕ニュートラル中間色{ちゅうかんしょく}、無彩色{むさいしょく}《電気》中性{ちゅうせい}[ニュートラル]端子{たんし}【形】〔争いで立場{たちば}が〕中立{ちゅうりつ}の、一方{いっぽう}に肩入れしない〔感情表現{かんじょう ひょうげん}や立場{たちば}などが〕どっちつかずの、はっきりしない〔色が〕無彩色{むさいしょく}の、中間色{ちゅうかんしょく}の《動物》去勢{きょせい}された◆【同】
\\	《電気》中性{ちゅうせい}の◆電荷を帯びていない。《化学》中性{ちゅうせい}の◆酸性でもアルカリ性でもない。《言語学》〔母音{ぼいん}が〕中性{ちゅうせい}の◆下の位置が中央で、弛緩させて調音する曖昧母音を指す。《機械》〔ギアなどが〕ニュートラルの◆動きが伝達されない状態を指す。レベル5、発音
\\	、カナニュートゥラル、ニュートラル、変化《複》
\\	、分節
\\	トムの体重はほんの50キロです。	変化形 
\\	《複》
\\	【1形】〔数量・金額などが〕ほんの、たったの
\\	数値 
\\	単位」のパターンでは、数値が1以外で単位が複数形になる場合でも冠詞の
\\	が付く。例えば、
\\	(たったの50ドル⇒「50ドル」が一つの特価)。類例として、
\\	(推定50人⇒「50人」が一つの推定値)や、
\\	(リーグ最多の50安打⇒「50安打」が一つの最多記録)がある。 ・
\\	トムの体重はほんの50キロです。単なる、全くの、全く~にすぎない【2名】〈古〉小さな湖、池、沼地{ぬまち}レベル4、発音
\\	、カナミア、メア、変化《形》
\\	大まかに言ってください。 ・
\\	手短にお願いします。	変化形 
\\	【他動】〔長い文章{ぶんしょう}などを〕短くまとめる、要約{ようやく}する~の要点{ようてん}を伝える[教える・かいつまんで話す]、~の概要{がいよう}を伝える[作る]、~を手短{てみじか}に話す[説明{せつめい}する・指示{しじ}する]、~に手短{てみじか}な[簡潔{かんけつ}な]指示{しじ}を与える~の摘要{てきよう}を作る【名】〔長い文書{ぶんしょ}の〕要約{ようやく}、概要{がいよう}状況報告{じょうきょうほうこく}(内容{ないよう})、ブリーフィング◆【同】
\\	短い[簡単{かんたん}な]説明{せつめい}[声明{せいめい}・報告{ほうこく}・記事{きじ}]《法律》弁論趣意書{べんろん しゅいしょ}、訴訟事件摘要書{そしょう じけん てきよう しょ}、準備書面{じゅんび しょめん}◆法廷のさまざまな段階で提出される、訴訟の論点を明らかにして自らの立場を有利に運ぶための書類。なお、法律事務所内で訴訟の準備のために作られる書類を指すこともある。内容により
\\	などと呼ばれることがある。◆【同】
\\	〔下着{したぎ}の〕ブリーフ《カトリック》教皇書簡{きょうこう しょかん}◆勅書
\\	ほど正式でないものを指す。〈英話〉法廷弁護士{ほうてい べんごし}【形】短時間{たんじかん}の、短い、短命{たんめい}な、しばらくの手短{てみじか}な、簡潔{かんけつ}な、簡素{かんそ}な、大まかな ・
\\	大まかに言ってください。 ・
\\	手短にお願いします。素っ気ないレベル4、発音
\\	、カナブリーフ、変化《形》
\\	、《動》
\\	変化形 
\\	【他動】~に頭文字{かしらもじ}を記す~に仮調印{かりちょういん}する【名】〔語頭{ごとう}の〕頭文字{かしらもじ}
\\	〔人の名前{なまえ}の〕イニシャル、頭文字{かしらもじ}〔章や段落{だんらく}の最初{さいしょ}の〕装飾頭文字{そうしょく かしらもじ}《生物》〔植物{しょくぶつ}の〕始原細胞{しげん さいぼう}◆【同】
\\	【形】最初{さいしょ}の、初めの、初の、初期{しょき}の、冒頭{ぼうとう}の頭文字{かしらもじ}のレベル4、発音
\\	、カナイニシャル、変化《動》
\\	、分節
\\	数名の6年生がその講義に出席しました。	イディオムやフレーズ 
\\	【名】
\\	前者{ぜんしゃ}作成者{さくせいしゃ}〈英〉~年生{ねんせい}(の生徒{せいと}) ・
\\	数名の6年生がその講義に出席しました。【形】前の、先のかつての、元のレベル2、発音
\\	、カナフォーマー、分節
\\	イディオムやフレーズ 
\\	【名】〔二者{にしゃ}のうちの〕後者{こうしゃ}【形】後の、終わりの最近{さいきん}の、近頃{ちかごろ}のレベル4、発音
\\	、カナラター、分節
\\	彼女の声はいつになく元気がなかった。 ・
\\	彼は100メートルを10秒ジャストで走った。
\\	00秒など。	変化形 
\\	《複》
\\	【1自動】平らになる《音楽》半音{はんおん}下がる失敗{しっぱい}に終わる〔会などが〕だらける【1他動】平らにする《音楽》〔音階を〕半音下げる◆【同】
\\	《音楽》~を半音下げて歌う[演奏する]◆【同】
\\	【1名】平面{へいめん}、平らな部分{ぶぶん}平地{へいち}、平原{へいげん}、湿地{しっち}、浅瀬{あさせ}◆通例
\\	《音楽》変音{へんおん}[フラット]記号{きごう}(のついた音)〈話〉ぺちゃんこの[空気{くうき}の抜けた]タイヤ〔舞台{ぶたい}の〕張物{はりもの}、フラット◆木の枠やベニヤ板に紙や布を貼って作る書割り。〔園芸用{えんげい よう}の種や苗を入れる〕薄い木箱{きばこ}= 
\\	フラット・シューズ◆【参考】
\\	大判{おおばん}の薄い郵便物{ゆうびんぶつ}《競馬》フラットレース◆【同】
\\	〔フラット・レースを走る〕競走馬{きょうそうば}〈英古〉間抜け{まぬけ}【1形】平たい、平らな、平坦{へいたん}な、タイヤの空気{くうき}が抜けた、扁平{へんぺい}な均一{きんいつ}の、変化{へんか}が少ない、横ばい状態{じょうたい}の薄い、薄っぺらな〔市況{しきょう}が〕不活発{ふかっぱつ}な元気{げんき}[活気{かっき}]のない ・
\\	彼女の声はいつになく元気がなかった。〈俗〉〔女性{じょせい}が〕ペチャパイの、胸がない、貧乳{ひんにゅう}の〔電池が〕切れた、〔電池{でんち}の〕電気{でんき}がなくなった〔ビールやソーダなどの飲料が〕気[炭酸{たんさん}]の抜けた【1副】〔数量・時間などが〕きっかり、ちょうど、ジャスト ・
\\	彼は100メートルを10秒ジャストで走った。
\\	00秒など。定額{ていがく}で、無利息勘定{むりそく かんじょう}で【2名】フラット、アパート、共同住宅{きょうどうじゅうたく}階レベル1、発音
\\	、カナフラット、変化《複》
\\	《形》
\\	奇数と奇数の和は偶数です。 ・
\\	これでおあいこだね。 ・
\\	壊れた時計も1日に2回正しい時刻を指す。/ばかも一芸。◆ことわざ ・
\\	子どもでもそれくらいのことはできる。 ・
\\	猿も木から落ちる。◆日本のことわざの直訳。 ・
\\	それは考えてみたこともありません。 ・
\\	まるで話が見えないのですが。 ・
\\	こんな所で何をやってるんですか。 ・
\\	あなた私に何て言ったか覚えてますか。 ・
\\	ちょっと、人の話聞いてるの。 ・
\\	私は動揺しました。というより腹が立った。	変化形 
\\	【1他動】~を平らにする【1名】偶数{ぐうすう}◆【同】
\\	奇数と奇数の和は偶数です。【1形】等しい、同等{どうとう}の、均等{きんとう}な、均一{きんいつ}な、一様{いちよう}な、規則正{きそく ただ}しい〔表面などが〕平らな、平坦{へいたん}な、同じ高さの公平{こうへい}なおあいこの、五分五分{ごぶ ごぶ}の、互角{ごかく}の、どっちもどっちの ・
\\	これでおあいこだね。偶数{ぐうすう}の〔色などが〕むらのない落ち着いた、冷静{れいせい}な、平静{へいせい}な、穏やかな【2副】~さえ(も)、~でさえ、~ですら、(たとえ)~でも ・
\\	壊れた時計も1日に2回正しい時刻を指す。/ばかも一芸。◆ことわざ ・
\\	子どもでもそれくらいのことはできる。 ・
\\	猿も木から落ちる。◆日本のことわざの直訳。〔比較級を強めて〕さらに、なおさら、なお一層{いっそう}〔強調して〕実に ・
\\	それは考えてみたこともありません。〔予想外・驚き・あきれ・苛立{いらだ}ちなどを表して〕まさに、ちょうど ・
\\	まるで話が見えないのですが。 ・
\\	こんな所で何をやってるんですか。 ・
\\	あなた私に何て言ったか覚えてますか。 ・
\\	ちょっと、人の話聞いてるの。より適切に言えば、もっと言えば ・
\\	私は動揺しました。というより腹が立った。【3名】〈古〉=
\\	◆【同】
\\	〈古〉=
\\	レベル1、発音
\\	、カナイーブン、変化《動》
\\	、分節
\\	変化形 
\\	《複》
\\	【名】〔物の〕平らで広い部分〈米俗・軽蔑的〉(若い)女、娘【形】〔幅が〕広い広々{ひろびろ}とした、広大{こうだい}な幅が~ある◆【用法】数字の後にくる。広範{こうはん}の、〔範囲{はんい}の〕広い〔計画などが〕大掛かりな大まかな、広義{こうぎ}の、一般的{いっぱんてき}な明白{めいはく}なレベル2、発音
\\	、カナブロード、変化《形》
\\	、《複》
\\	変化形 
\\	【形】広大{こうだい}な、非常{ひじょう}に広い、広漠{こうばく}とした、巨大{きょだい}な膨大{ぼうだい}な、莫大{ばくだい}な、巨額{きょがく}のレベル3、発音
\\	、カナバースト、バスト、変化《形》
\\	私は健康診断の結果を受け取った。 ・
\\	その物理的発見には矛盾がある。	変化形 
\\	《複》
\\	【名】身体検査{しんたい けんさ}、健康診断{けんこう しんだん}◆【同】
\\	私は健康診断の結果を受け取った。【形】物質{ぶっしつ}の、物質的{ぶっしつてき}な、自然{しぜん}の、天然{てんねん}の身体{しんたい/からだ}の、肉体{にくたい}の、身体的{しんたいてき}な相手{あいて}の体を求めたがる、好色{こうしょく}な物理学{ぶつりがく}の、物理学上{ぶつり がくじょう}の、物理的{ぶつりてき}な ・
\\	その物理的発見には矛盾がある。自然{しぜん}の法則{ほうそく}による、自然科学{しぜん かがく}の形而下{けいじか}の◆形而上と対比を成す、物質的な形を備えたもの。〈話〉荒っぽいレベル3、発音
\\	、カナフィジカル、フィズィカル、フィズィコー、変化《複》
\\	、分節
\\	イディオムやフレーズ 
\\	【1形】心の、精神(上){せいしん(じょう)}の知力{ちりょく}の、知的{ちてき}な〈英俗・軽蔑的〉気の狂った、精神病{せいしんびょう}の【2形】顎{あご}のレベル3、発音
\\	、カナメンタル、分節
\\	変化形 
\\	《複》
\\	【名】霊的{れいてき}な[宗教的{しゅうきょう てき}な]こと[もの]◆通例、
\\	《音楽》黒人霊歌{こくじん れいか}◆【同】
\\	精神界{せいしんかい}【形】〔物質的{ぶっしつ てき}と対比{たいひ}して〕精神的{せいしん てき}な、霊的{れいてき}な崇高{すうこう}な、気高い、魂が動かされる宗教的{しゅうきょう てき}な、教会{きょうかい}の、聖職{せいしょく}の気質{きしつ}[精神{せいしん}]が似通った{にかよった}超自然的{ちょうしぜん てき}な、心霊現象{しんれい げんしょう}のレベル4、発音
\\	、カナスピリチャリー、スピリチュアル、変化《複》
\\	、分節
\\	彼女はあのコンテストでかなり目立つ人物だった。		【形】〔物が〕目に見える、可視{かし}の〔景色{けしき}などが〕見える、目に入る〔世間{せけん}によく〕知られた、周知{しゅうち}された目立つ{めだつ}、華やかな ・
\\	彼女はあのコンテストでかなり目立つ人物だった。利用{りよう}できる、手元{てもと}にある一目瞭然{いちもく りょうぜん}の、明らかな思い当たる[浮かぶ]見やすい、一覧{いちらん}できるような《コ》可視{かし}の、見える◆ある場所において、別の場所で定義されている変数・関数などの識別子が使えること。レベル4、発音
\\	、カナビジブル、ビズィブル、分節
\\	【形】実質上{じっしつじょう}の、事実上{じじつじょう}の、実際上{じっさい じょう}の、実質的{じっしつてき}な◆実体・事実ではないが「本質」を示すもの。仮の、仮想{かそう}の、虚の、虚像{きょぞう}のネットワーク上のレベル6、発音
\\	、カナバーチュアル、分節
\\	変化形 
\\	《複》
\\	【形】言語(学){げんご(がく)}の、言語{げんご}[言葉{ことば}]の[に関する]、語学上{ごがく じょう}のレベル7、発音
\\	、分節
\\	彼は数々の小説を発表しました。 ・
\\	彼女は幾多の障害に打ち勝ち今日の地位を築いた。		【形】多数{たすう}の、非常{ひじょう}に[数え切れないほど]多くの ・
\\	彼は数々の小説を発表しました。 ・
\\	彼女は幾多の障害に打ち勝ち今日の地位を築いた。レベル5、発音
\\	、カナニューメラス、分節
\\	あまりに気色悪いので、それについてここに書くのも嫌なくらいだ。考えるだけでへどが出る。	変化形 
\\	【他動】~の総収益{そうしゅうえき}を挙げる【名】〔収入{しゅうにゅう}などの〕総額{そうがく}、総計{そうけい}◆【複】
\\	《単位》グロス◆12ダース=144個◆【複】
\\	【形】総計{そうけい}の、徹底的{てっていてき}な、全体{ぜんたい}の、総~、巨視的{きょしてき}、全くの〈話〉気持ち悪い、嫌な、ゾッとする、吐き気を催すような、いまいましい、ムカムカさせる、むかつく ・
\\	あまりに気色悪いので、それについてここに書くのも嫌なくらいだ。考えるだけでへどが出る。粗雑{そざつ}な、粗野{そや}な、不作法{ぶさほう}な、品のない、繊細{せんさい}さに欠ける、肉眼的{にくがん てき}なひどく太った、太り過ぎの繁茂{はんも}した、生い茂った、あふれるほどの言語道断{ごんご どうだん}の、目に余る◆【類】
\\	レベル4、発音
\\	、カナグロウス、グロス、変化《形》
\\	、《動》
\\	性格不一致は十分な離婚の根拠になる。 ・
\\	もう一人子どもができたら家計がもたない。	イディオムやフレーズ 
\\	【名】十分{じゅうぶん}な数[量]【形】〔必要とされている量があり〕十分{じゅうぶん}な、満足{まんぞく}な、足りる ・
\\	性格不一致は十分な離婚の根拠になる。 ・
\\	もう一人子どもができたら家計がもたない。レベル3、発音
\\	、カナサフィシェント、分節
\\	【形】〔道具{どうぐ}ややり方などが〕効率{こうりつ}の良い、高い効率{こうりつ}の〔仕事を効率よくこなして〕有能{ゆうのう}な、腕の立つレベル4、発音
\\	、カナイフィシェント、エフィシェント、分節
\\	彼は、チームリーダーとして有能であることをみんなに証明しました。	イディオムやフレーズ 
\\	【形】〔人・物が~する〕能力{のうりょく}がある〔人が経験を積んで〕有能{ゆうのう}な、手腕{しゅわん}のある ・
\\	彼は、チームリーダーとして有能であることをみんなに証明しました。レベル4、発音
\\	、カナケイパブル、分節
\\	【名】ブリリアントカットの宝石{ほうせき}、目覚ましい《印刷》ブリリアント活字{かつじ}【形】キラキラ光る[輝く]、さんさんと輝く、光り輝く、きらめく〈英話〉素晴{すば}らしい、見事{みごと}な、とてもいい、卓越{たくえつ}した、華々{はなばな}しい、目もあやな、すごい◆【略】
\\	才気あふれる、優れた、優秀{ゆうしゅう}な、才気あふれた、才能{さいのう}[才気{さいき}]のある、立派{りっぱ}な、(頭脳{ずのう})明晰{めいせき}な〔音が〕はっきりした、明瞭{めいりょう}な〔色が〕鮮明{せんめい}な、鮮やかな〔宝石{ほうせき}が〕ブリリアントカットの
\\	【形】光輝く、色鮮{いろあざ}やかな壮大{そうだい}な、華麗{かれい}な優れた、立派{りっぱ}な著名{ちょめい}な、名高い{なだかい}レベル4、発音
\\	、カナスプレンディッド、スプレンデッド、変化《形》
\\	、分節
\\	おまえの心は分厚いコンクリートで覆われてるから、自分の妻が死んでも悲しみもしないんだ。 ・
\\	コンクリートの壁だって、このシャツほど固くはないよ。 ・
\\	コンクリートの基礎にツーバイフォーの木材を取り付ける。 ・
\\	もっと具体的な言い方で、理由を説明していただけませんか。 ・
\\	はっきりした返事がもらえなかった。 ・
\\	もう少し具体例を示していただけませんか。 ・
\\	私の疑いは事実に基づいているわけではありません。 ・
\\	計画が具体化しました。	変化形 
\\	【自動】固まる、固くなる【他動】~をコンクリートで建てる[固める・舗装{ほそう}する]《物理》〔粒子{りゅうし}などを〕凝固{ぎょうこ}させる【名】《建築》コンクリート◆【参考】
\\	《物理》〔粒子{りゅうし}の〕凝集{ぎょうしゅう}[凝固{ぎょうこ}]体[もの]【形】コンクリートの、コンクリートで作られた、コンクリート製の ・
\\	おまえの心は分厚いコンクリートで覆われてるから、自分の妻が死んでも悲しみもしないんだ。 ・
\\	コンクリートの壁だって、このシャツほど固くはないよ。 ・
\\	コンクリートの基礎にツーバイフォーの木材を取り付ける。〔抽象的{ちゅうしょう てき}ではなく〕具体的{ぐたい てき}な、はっきりした、明確{めいかく}な◆【反】
\\	もっと具体的な言い方で、理由を説明していただけませんか。 ・
\\	はっきりした返事がもらえなかった。 ・
\\	もう少し具体例を示していただけませんか。〔想像{そうぞう}ではなく〕実在{じつざい}の、有形{ゆうけい}の、実体{じったい}がある ・
\\	私の疑いは事実に基づいているわけではありません。 ・
\\	計画が具体化しました。《物理》凝固{ぎょうこ}したレベル5、発音
\\	、カナカンクリート、コンクリート、変化《動》
\\	、分節
\\	彼女の財布から10ドルを失敬しました。 ・
\\	概念は感覚パターンとしての経験から抽象化される。 ・
\\	その工場は川からの取水の認可を受けている。 ・
\\	私は理論研究者であり応用科学者ではありません。 ・
\\	これは研究に多くを費やすべき難解な問題である。	変化形 
\\	【他動】〔何とか〕~を取り除く、~を取り去る~を失敬{しっけい}する、~をくすねる◆婉曲表現 ・
\\	彼女の財布から10ドルを失敬しました。~を概念化{がいねん か}する、~を抽象化{ちゅうしょうか}する ・
\\	概念は感覚パターンとしての経験から抽象化される。~を要約{ようやく}する、~を短くまとめる〔工場{こうじょう}などに川の水を〕取り込む ・
\\	その工場は川からの取水の認可を受けている。【名】抽象的{ちゅうしょうてき}なもの抜粋{ばっすい}、要約(書){ようやく(しょ)}、摘要{てきよう}抽象{ちゅうしょう}芸術{げいじゅつ}[絵画{かいが}]【形】抽象的{ちゅうしょう てき}な、概念上{がいねん じょう}の理論的{りろん てき}な、理論上{りろんじょう}の ・
\\	私は理論研究者であり応用科学者ではありません。難解{なんかい}な、深遠{しんえん}な ・
\\	これは研究に多くを費やすべき難解な問題である。〔音楽{おんがく}や美術{びじゅつ}が〕抽象{ちゅうしょう}の、非具象{ひ ぐしょう}の、概念{がいねん}の、コンセプチュアルなレベル4、発音《名・形》
\\	《動》
\\	、カナアブストゥラクト、アブストラクト、変化《動》
\\	、分節
\\	【形】人工{じんこう}の、人工的{じんこうてき}な、作り物の、人造{じんぞう}の、模造{もぞう}の◆【同】
\\	◆【反】
\\	人為{じんい}による、人為{じんい}の、人為的{じんいてき}な、人的{じんてき}な◆【同】
\\	作ったような、形だけの、わざとらしい、うわべだけの、見え透いた、見せ掛けの、不自然{ふしぜん}な、偽の、偽りの、偽物{にせもの}の、いかさまの理論上存在{りろん じょう そんざい}する《植物》栽培{さいばい}されたレベル4、発音
\\	、カナアーティフィシャル、アーテフィシャル、分節
\\	変化形 
\\	《複》
\\	【名】プラスチック、ビニール、樹脂{じゅし}クレジットカード、ポイントカード◆【同】
\\	〔プラスチック製の〕ラップ
\\	プラスチック工学{こうがく}◆【用法】単数扱い
\\	【形】プラスチックの、プラスチック製の、ビニールの、柔軟{じゅうなん}な、可塑性{かそせい}の、創造的{そうぞうてき}な、塑性体{そせいたい}のいかさまの、見せ掛けだけの誠実さに欠ける、白々しいレベル2、発音
\\	、カナプラスチック、プラスティック、変化《複》
\\	、分節
\\	変化形 
\\	《複》
\\	【名】携帯電話(機){けいたい でんわ(き)}◆【略】
\\	モビール◆【同】
\\	【形】〔人や物が〕動ける、動かしやすい、可動性{かどう せい}の《軍事》機動性{きどうせい}のある、機動力{きどうりょく}を備えた〔人が〕移り気な、気の変わりやすい〔表情{ひょうじょう}が〕よく動く、変わりやすい〔状況{じょうきょう}などが〕流動的{りゅうどう てき}な、不安定{ふあんてい}な〔社会集団間{しゃかい しゅうだん かん}で〕よく移動{いどう}する、流動型{りゅうどう がた}の〔人や社会{しゃかい}などが〕よく転居{てんきょ}する、移動型{いどう がた}の〔液体{えきたい}などが〕流動性{りゅうどう せい}がある[高い]携帯電話{けいたい でんわ}による、携帯電話{けいたい でんわ}を使ったレベル8、発音《名》
\\	《形》
\\	、カナモウビール、
\\	モウブル、
\\	モウバイル、変化《複》
\\	、分節
\\	呼吸は大抵無意識に行われる。	変化形 
\\	《複》
\\	【名】自動機械{じどう きかい}、オートマチック車、自動{じどう}拳銃{けんじゅう}[ピストル]【形】〔機械の動作などが〕自動{じどう}の、自動的{じどうてき}な◆【反】
\\	無意識{むいしき}の、自然{しぜん}に沸き起こる ・
\\	呼吸は大抵無意識に行われる。レベル3、発音
\\	、カナオートマチック、オートマティック、変化《複》
\\	、分節
\\	【形】〔場所{ばしょ}が〕内部{ないぶ}の、内側{うちがわ}の、中心部{ちゅうしんぶ}に近い〔意味{いみ}などが〕隠れた、表に出ない〔人の〕精神的{せいしん てき}な、内面的{ないめん てき}な〔権力{けんりょく}の〕中枢{ちゅうすう}の、側近{そっきん}のレベル3、発音
\\	、カナイナー、インナー、分節
\\	コカインはアッパー系の[覚醒作用のある]ドラッグです。	変化形 
\\	《複》
\\	【名】上位{じょうい}のもの[人]
\\	〈話〉上顎歯{じょうがく し}靴の甲〈話〉=
\\	〈俗〉覚醒{かくせい}[気分{きぶん}を高揚{こうよう}させる]作用{さよう}のある薬物{やくぶつ}[麻薬{まやく}]、アッパー系のドラッグ◆アンフェタミンなど。◆【同】
\\	◆【対】
\\	コカインはアッパー系の[覚醒作用のある]ドラッグです。〈俗〉ゾクゾク[ワクワク]すること[経験{けいけん}]【形】〔位置{いち}が〕上の、上部{じょうぶ}の、上方{じょうほう}の〔社会的立場{しゃかい てき たちば}が〕上の、上位{じょうい}の、上級{じょうきゅう}の《地学》高地{こうち}の、内陸{ないりく}の、北方{ほっぽう}の、上流{じょうりゅう}の
\\	《地学》〔地質{ちしつ}が〕後期{こうき}の、後代{こうだい}の《数学》〔ある集合{しゅうごう}の〕最大{さいだい}のレベル2、発音
\\	、カナアッパー、変化《複》
\\	、分節
\\	【形】〔物の〕内部{ないぶ}の、内側{うちがわ}の〔人の〕体内{たいない}の、器官内{きかんない}の《医》〔薬が〕内服{ないふく}の、経口{けいこう}の〔人の〕意識内{いしき ない}の、内面{ないめん}の〔物事{ものごと}に〕内在{ないざい}する、固有{こゆう}の〔出来事{できごと}などが〕国内{こくない}の、内政{ないせい}のレベル4、発音
\\	、カナインターナル、変化《複》
\\	、分節
\\	変化形 
\\	《複》
\\	【名】外部{がいぶ}、外観{がいかん}、外見{がいけん}、社外{しゃがい}、外面{がいめん}【形】外[外側{そとがわ}・外部{がいぶ}]の[にある]、外付けの、外用{がいよう}の、形式的{けいしきてき}な、外界{がいかい}の、社外{しゃがい}の、外国{がいこく}とのレベル4、発音
\\	、カナエクスターナル、変化《複》
\\	、分節
\\	【名】〈英話〉〔犯罪{はんざい}の〕前歴{ぜんれき}【形】〔時間的に〕前の、以前{いぜん}の〔順序が〕前の、先の〈話〉性急{せいきゅう}な、早計{そうけい}なレベル3、発音
\\	、カナプリービァス、プリービアス、プレビアス、分節
\\	【名】臨時雇いの人【形】一時的{いちじ てき}な、当座{とうざ}の、はかないレベル3、発音
\\	、カナテンポラリー、テンポレリ、変化《複》
\\	、分節
\\	常にリスクはある。 ・
\\	永遠に変わらないものは一つもない。 ・
\\	彼女に対してこれまで誠実だった。	変化形 
\\	《複》
\\	【名】不変{ふへん}のもの、普遍{ふへん} ・
\\	常にリスクはある。《数学》定数{ていすう}、常数{じょうすう}◆【略】
\\	【形】持続{じぞく}する、絶えず続く〔量・性質・温度などが〕不変{ふへん}の、一定{いってい}な ・
\\	永遠に変わらないものは一つもない。忠実{ちゅうじつ}な、誠実{せいじつ}な ・
\\	彼女に対してこれまで誠実だった。レベル4、発音
\\	、カナカンスタント、コンスタント、変化《複》
\\	、分節
\\	変化形 
\\	《複》
\\	【名】同時期{どう じき}の人、同年齢{どう ねんれい}の人、同年輩{どうねんぱい}【形】同時期{どうじき}に起こる[存在する]、同時代{どうじだい}に存在{そんざい}した◆【語源】
\\	-(一緒)
\\	(時間)今まさに起こっている、現代{げんだい}の、当代{とうだい}の〔人が〕(ほとんど)同じ年齢{ねんれい}のレベル5、発音
\\	、カナコンテンポラリー、変化《複》
\\	、分節
\\	その自動車は、それまでの自動車づくりとはほとんど無縁だった航空エンジニアたちによって設計された。		【形】〔社会的{しゃかい てき}〕慣習{かんしゅう}の、慣習{かんしゅう}として認められた伝統{でんとう}の、伝統{でんとう}として確立{かくりつ}された、従来続けられている、従来型{じゅうらい がた}の、標準{ひょうじゅん}となった ・
\\	その自動車は、それまでの自動車づくりとはほとんど無縁だった航空エンジニアたちによって設計された。オリジナリティ[独創性{どくそうせい}]のない、陳腐{ちんぷ}な、平凡{へいぼん}な、型にはまった《法律》契約上{けいやく じょう}の、協定{きょうてい}に従った〔芸術作品{げいじゅつ さくひん}が〕様式化{ようしきか}された、抽象的{ちゅうしょう てき}な核兵器{へいき}[エネルギー]を使わない、非核{ひかく}の、〔核を用いない〕従来型{じゅうらい がた}の集会{しゅうかい}の、会合{かいごう}のレベル4、発音
\\	、カナコンベンショナル、分節
\\	私はこの喫茶店によく行きます。 ・
\\	よく会議を開くのですか。	変化形 
\\	【他動】~をしばしば訪れる、~によく行く[出入りする・いる・入り浸る]、~に足しげく通う、~に常に集まる ・
\\	私はこの喫茶店によく行きます。常に~と交際{こうさい}する、~に親しむ《古》~に精通{せいつう}する、なじむ【形】たびたび起こる、しばしば起こる、多い、たびたびの、しばしばの、頻繁{ひんぱん}な、よくある、ありがちな、頻出{ひんしゅつ}する、再三{さいさん}にわたる ・
\\	よく会議を開くのですか。常に~する、常習的{じょうしゅうてき}ないくつものレベル4、発音《形》
\\	《動》
\\	、カナフリークウェント、フリークエント、フリクウェント、変化《動》
\\	、分節
\\	イディオムやフレーズ 
\\	【もっとイディオムを見る】
\\	【形】生存して、生きていて活動して、動作して〔場所が生き物で〕いっぱいの◆【用法】通例、
\\	の形で用いられる。生き生きとして、活動的でレベル2、発音
\\	、カナアライブ、分節
\\	私の両脚はしびれていました。	イディオムやフレーズ 
\\	【形】眠って(いる状態{じょうたい}の)、就寝中{しゅうしん ちゅう}の〔手や足が眠っているかのように〕無感覚{むかんかく}で、しびれて ・
\\	私の両脚はしびれていました。【副】眠って、休止{きゅうし}してレベル2、発音
\\	、カナアスリープ、分節
\\	恥ずかしい思いをしている。 ・
\\	肩身が狭いよ。 ・
\\	彼に合わす顔がない。	イディオムやフレーズ 
\\	【形】恥ずかしい◆自分の言動が常識や道徳に照らし合わせて ・
\\	恥ずかしい思いをしている。 ・
\\	肩身が狭いよ。~したことを[~であることを]恥ずかしく思う[恥じている]、面目{めんもく}ない、~することに気が引ける、恥ずかしくて~したくない[~することができない・~する気になれない]◆【用法】
\\	の形で ・
\\	彼に合わす顔がない。レベル3、発音
\\	、カナアシェイムド、分節
\\	イディオムやフレーズ 
\\	【形】
\\	~を知っている、~が分かっている◆直接見聞きしたり、人づてに聞いたりして。◆【用法】
\\	〔~の存在{そんざい}に〕気付いている、感じている◆本人が知覚して。◆【用法】
\\	〔現状{げんじょう}などについて〕よく知っている、詳しいレベル2、発音
\\	、カナアウェア、アウエア、分節
\\	彼にはほとんど意識がない。 ・
\\	彼女の意識はまだ完全には回復していない。 ・
\\	私は新種の音楽を作り出していることに気付いていなかった。 ・
\\	われわれはテレビ時代に生きていることを非常に意識している。 ・
\\	彼らはアメリカで起こっていることを十分に理解している。 ・
\\	なぜそれほどそれが気になるのだろう。 ・
\\	意識的差別はないと信じている。	イディオムやフレーズ 
\\	【名】
\\	意識{いしき}【形】〔身体的に〕意識{いしき}のある、知覚反応{ちかく はんのう}がある ・
\\	彼にはほとんど意識がない。 ・
\\	彼女の意識はまだ完全には回復していない。〔事実・状況などを〕意識{いしき}[自覚{じかく}]している、気にしている ・
\\	私は新種の音楽を作り出していることに気付いていなかった。 ・
\\	われわれはテレビ時代に生きていることを非常に意識している。 ・
\\	彼らはアメリカで起こっていることを十分に理解している。 ・
\\	なぜそれほどそれが気になるのだろう。意識的{いしきてき}な、わざとらしい ・
\\	意識的差別はないと信じている。自意識{じいしき}の強い、人目{ひとめ}を気にする〈通例複合語で〉~の意識{いしき}の強い、~を重視{じゅうし}するレベル3、発音
\\	、カナカンシャス、コンシャス、分節
\\	犬の鼻の方がこの機械より敏感です。 ・
\\	このボタンは軽く触れるだけで反応する。 ・
\\	あまり神経質にならないようにしましょう。 ・
\\	彼の髪のことは何も言うなよ。すごく気にしてるんだから。	イディオムやフレーズ 
\\	【名】傷つきやすい人間{にんげん}霊感{れいかん}のある人間{にんげん}【形】感覚{かんかく}がある、感知{かんち}できる《病理》〔身体{しんたい}が〕傷つきやすい、炎症{えんしょう}を起こしやすい〔外界{がいかい}の刺激{しげき}に対して身体{しんたい}が〕敏感{びんかん}な、良く反応{はんのう}する ・
\\	犬の鼻の方がこの機械より敏感です。〔芸術的{げいじゅつ てき}な表現{ひょうげん}などが〕繊細{せんさい}な、微妙{びみょう}な〔芸術作品{げいじゅつ さくひん}などに対して〕感性{かんせい}が豊かな〔外界{がいかい}に刺激{しげき}に〕反応{はんのう}する、影響{えいきょう}を受ける ・
\\	このボタンは軽く触れるだけで反応する。〔他の人に〕思いやりがある、親切{しんせつ}な〔他の人の言葉{ことば}などに〕傷つきやすい、神経質{しんけいしつ}な ・
\\	あまり神経質にならないようにしましょう。 ・
\\	彼の髪のことは何も言うなよ。すごく気にしてるんだから。〔問題{もんだい}などが〕公にしにくい、慎重{しんちょう}に扱うべき〔機械{きかい}などが〕微妙{びみょう}な差が検知{けんち}できる、精度{せいど}が高い《金融》〔市場{しじょう}などが〕不安定{ふあんてい}な、乱高下{らんこうげ}するレベル3、発音
\\	、カナセンシティブ、センシテブ、センスィティブ、分節
\\	緊張するなあ。 ・
\\	私はちょっとあがってます。/緊張しています。 ・
\\	がらにもなく固くなっている。		【形】神経(系){しんけい(けい)}の、神経{しんけい}に関する〔人が〕神経{しんけい}が高ぶった、緊張{きんちょう}した、やきもきする ・
\\	緊張するなあ。 ・
\\	私はちょっとあがってます。/緊張しています。 ・
\\	がらにもなく固くなっている。〔物事{ものごと}が〕ひやひや[イライラ・不安{ふあん}に]させる〔人や性格{せいかく}が〕神経質{しんけいしつ}な、臆病{おくびょう}な、気が小さい〔表現{ひょうげん}などが〕生き生きした、活気{かっき}あるレベル2、発音
\\	、カナナーバス、分節
\\	興味本位でそれをしてはいけません。	イディオムやフレーズ 
\\	【形】好奇心{こうきしん}をそそる、興味深い、好奇心{こうきしん}の強い、知りたがり屋で、物見高い{ものみだかい} ・
\\	興味本位でそれをしてはいけません。〔面白{おもしろ}そうなので〕気になるレベル4、発音
\\	、カナキュアリアス、キュリアス、分節
\\	《名言》天才とは苦しみを受け入れられる無限の抱擁能力。	変化形 
\\	《複》
\\	【名】〔世に知られた知的・芸術的〕天才{てんさい}◆【複】
\\	《名言》天才とは苦しみを受け入れられる無限の抱擁能力。
\\	が140以上{いじょう}の〕天才{てんさい}〔ある分野{ぶんや}の天賦{てんぷ}の〕才能{さいのう}、たぐいまれな能力{のうりょく}〔特定{とくてい}の技能{ぎのう}に秀でた〕才人{さいじん}、鬼才{きさい}〔時代{じだい}・人・場所{ばしょ}などの〕特質{とくしつ}、支配的{しはい てき}な精神{せいしん}〔ローマ神話{しんわ}の〕ゲニウス◆人や場所などの守護霊。◆【複】
\\	〔イスラム神話{しんわ}の〕精霊{せいれい}、悪霊{あくりょう}〔他の人に〕強い感化{かんか}を与える人レベル4、発音
\\	、カナジーニァス、ジーニアス、ジーニャス、分節
\\	変化形 
\\	《複》
\\	【他動】~に…を贈呈{ぞうてい}する【名】贈り物、プレゼント、与えること神からの贈り物、特別{とくべつ}の能力{のうりょく}、天賦{てんぷ}の才、天資{てんし}、天稟{てんぴん}《契約書》贈与書{ぞうよ しょ}◆贈与の意思表示を行う契約書のタイトルレベル2、発音
\\	、カナギフト、変化《動》
\\	変化形 
\\	《複》
\\	【名】知性{ちせい}、思考力{しこうりょく}、知力{ちりょく}レベル6、発音
\\	、カナインテレクト、変化《複》
\\	、分節
\\	ハードディスクのバックアップ時間は、容量が多いほど長くかかる。 ・
\\	《契約書》メーカーは、最低_年間、製造工程の運転能力を維持することに同意する。 ・
\\	当社は現在能力いっぱいとなっております。	変化形 
\\	《複》
\\	【名】〔建物{たてもの}や乗り物の〕収容可能{しゅうよう かのう}数{すう}[人数{にんずう}]、定員{ていいん}〔容器{ようき}などの〕限度容量{げんど ようりょう}、最大容積{さいだい ようせき} ・
\\	ハードディスクのバックアップ時間は、容量が多いほど長くかかる。〔生産{せいさん}できる〕能力{のうりょく}、〔最大{さいだい}の〕生産量{せいさんりょう} ・
\\	《契約書》メーカーは、最低_年間、製造工程の運転能力を維持することに同意する。 ・
\\	当社は現在能力いっぱいとなっております。〔学習{がくしゅう}や記憶{きおく}の〕知的能力{ちてき のうりょく}〔成長{せいちょう}や発展{はってん}の〕潜在的{せんざい てき}な可能性{かのう せい}〔適合{てきごう}できる〕能力{のうりょく}、性質{せいしつ}、資格{しかく}〔機能{きのう}を果たす〕地位{ちい}、立場{たちば}《法律》法的能力{ほうてき のうりょく}、法的資格{ほうてき しかく}、行為能力{こうい のうりょく}◆【参考】
\\	【形】〔建物・乗り物などが〕収容能力{しゅうよう のうりょく}いっぱいの〔容量・容積などが〕最大限{さいだいげん}の〔仕事などが〕処理能力{しょり のうりょく}いっぱいのレベル4、発音
\\	、カナカパシテー、カパスィティー、キャパシティー、変化《複》
\\	、分節
\\	変化形 
\\	《複》
\\	【名】理解{りかい}、知識{ちしき}、包括{ほうかつ}、了解{りょうかい}、理解力{りかいりょく}、読解力{どっかい りょく}包含{ほうがん}レベル5、発音
\\	、カナコンプリヘンション、変化《複》
\\	、分節
\\	変化形 
\\	《複》
\\	【名】意見{いけん}、考え、観念{かんねん}心象{しんしょう}、イメージ〔気まぐれな〕思い付き、出来心{できごころ}
\\	〈米〉小間物{こまもの}、裁縫道具{さいほう どうぐ}レベル4、発音
\\	、カナノーション、ノウション、変化《複》
\\	、分節
\\	変化形 
\\	《複》
\\	【名】〔絵画{かいが}などの〕遠近法{えんきんほう}、透視図法{とうし ずほう}〔遠くの物の〕遠近感{えんきん かん}、奥行き{おくゆき}景色{けしき}、眺め〔物事{ものごと}に対する〕見方{みかた}、態度{たいど}、視点{してん}〔状況{じょうきょう}や事実{じじつ}の〕全体像{ぜんたい ぞう}、大局観{たいきょくかん}〔物事{ものごと}を〕評価{ひょうか}する能力{のうりょく}、見通せる力【形】遠近法{えんきんほう}のレベル6、発音
\\	、カナパースペクチブ、パースペクティブ、変化《複》
\\	、分節
\\	変化形 
\\	【他動】(人)に先入観{せんにゅうかん}[偏見{へんけん}]を抱かせる~に損害{そんがい}をもたらす、毀損{きそん}する、(権利{けんり}を)害する【名】〔不確{ふたし}かな情報{じょうほう}による〕先入観{せんにゅうかん}◆相手に好意的でない意見を指すことが多い。〔人種{じんしゅ}や宗教{しゅうきょう}などに対する〕偏見{へんけん}、嫌悪感{けんおかん}◆合理的な判断に基づかないものを指す。《法律》不利益{ふりえき}、損害{そんがい}レベル5、発音
\\	、カナプレジュディス、変化《動》
\\	、分節
\\	変化形 
\\	《複》
\\	【名】〔あるものや出来事{できごと}に対する〕注目{ちゅうもく}、重要視{じゅうようし}、重点{じゅうてん}を置くこと〔重要性{じゅうよう せい}を示すための表現{ひょうげん}の〕力強{ちからづよ}さ、強調{きょうちょう}、力説{りきせつ}《言語学》〔音節{おんせつ}や語への〕強調{きょうちょう}、強意{きょうい}レベル4、発音
\\	、カナエンファシス、エンファスィス、変化《複》
\\	、分節
\\	彼女の健康の方が優先です。 ・
\\	彼らは早く来たから開拓者と名付けられたわけではなかった。 ・
\\	大統領は人権尊重の優先度を最も高くした。	変化形 
\\	《複》
\\	【名】〔重要度{じゅうようど}や緊急度{きんきゅう ど}により〕優先{ゆうせん}すること[されていること]◆不可算 ・
\\	彼女の健康の方が優先です。優先事項{ゆうせん じこう}、重要度{じゅうようど}の高いもの[人]◆加算〔時間的{じかん てき}に〕早いこと、先行{せんこう}◆不可算 ・
\\	彼らは早く来たから開拓者と名付けられたわけではなかった。優先{ゆうせん}[先行{せんこう}]権、優先{ゆうせん}度[順位{じゅんい}]◆不可算 ・
\\	大統領は人権尊重の優先度を最も高くした。レベル4、発音
\\	、カナプライオリティー、変化《複》
\\	、分節
\\	誰が勝つと予想していますか? ・
\\	これは彼の1年前の予想だった。 ・
\\	シンガポール・コンベンションセンターでは「2025年のアジア」の未来予測を公募しました。 ・
\\	天気が変われば予報も変わる。 ・
\\	昨日の天気予報では週末は雷雨になるとか言ってたけど、ならなかった。今度は明日は晴れって言ってるよ。	変化形 
\\	【自動】予見{よけん}する、予言{よげん}する前もって計画{けいかく}を立てる【他動】予想{よそう}する、予測{よそく}する、〔天気{てんき}を〕予報{よほう}する、~の前兆{ぜんちょう}を示す ・
\\	誰が勝つと予想していますか?予定{よてい}する、前もって計画{けいかく}する【名】予想{よそう}、予測{よそく}、予報{よほう}、見通し{みとおし} ・
\\	これは彼の1年前の予想だった。 ・
\\	シンガポール・コンベンションセンターでは「2025年のアジア」の未来予測を公募しました。天気予報{てんき よほう} ・
\\	天気が変われば予報も変わる。 ・
\\	昨日の天気予報では週末は雷雨になるとか言ってたけど、ならなかった。今度は明日は晴れって言ってるよ。先見{せんけん}(の明)レベル4、発音
\\	、カナフォーカースト、フォーキャスト、変化《動》
\\	または
\\	、分節
\\	戦争の前に、アメリカ政府はサダム・フセインに大統領を辞任することを要求し、さもなくば深刻な結果を招くと警告しました。 ・
\\	教育者として、われわれは子どもたちに判断のまずさがある否定的な結果を招くことを教える必要がある。 ・
\\	責任とは、自分の行動の結果を受け入れることだ。	変化形 
\\	《複》
\\	【名】〔行動{こうどう}や状況{じょうきょう}から必然的{ひつぜん てき}に導かれる〕結果{けっか}、結論{けつろん}、帰結{きけつ} ・
\\	戦争の前に、アメリカ政府はサダム・フセインに大統領を辞任することを要求し、さもなくば深刻な結果を招くと警告しました。 ・
\\	教育者として、われわれは子どもたちに判断のまずさがある否定的な結果を招くことを教える必要がある。 ・
\\	責任とは、自分の行動の結果を受け入れることだ。因果関係{いんが かんけい}、原因{げんいん}と結果{けっか}の関係{かんけい}〔正しい推論{すいろん}によって得られる〕論理的{ろんり てき}結論{けつろん}[帰結{きけつ}]〈文〉〔物事の〕重大{じゅうだい}さ、重大性{じゅうだい せい}、重要性{じゅうようせい}〈文〉〔地位{ちい}などの〕重要{じゅうよう}さ、偉さレベル4、発音
\\	、カナカンスィクウェンス、コンシクエンス、コンセクウェンス、コンセクエンス、変化《複》
\\	、分節
\\	人のことに口出しするな。 ・
\\	難しい状況だ。 ・
\\	彼女の夫は浮気をしている。
\\	変化形 
\\	《複》
\\	"【名】〔一般的{いっぱん てき}な〕事柄{ことがら}、出来事{できごと}〔社会的{しゃかい てき}に大きな〕事件{じけん}、出来事{できごと}、スキャンダルするべきこと、用事{ようじ}、用件{ようけん}
\\	〔専門的{せんもん てき}な〕業務{ぎょうむ}、実務{じつむ}、仕事{しごと}
\\	私事{しじ}、個人的{こじん てき}な事柄{ことがら} ・
\\	人のことに口出しするな。
\\	〔一般的{いっぱん てき}な〕状況{じょうきょう}、事態{じたい} ・
\\	難しい状況だ。〔個人{こじん}の〕関心事{かんしんじ}、心配事{しんぱいごと}〔特別{とくべつ}な〕物、品、装置{そうち}〔短い間の〕不倫{ふりん}、浮気{うわき}◆【類】
\\	彼女の夫は浮気をしている。レベル3、発音
\\	、カナアフェアー、変化《複》
\\	、分節
\\	大事件はいとも簡単に起こり、あっけなく解決するものだ。	変化形 
\\	《複》
\\	【名】〔1回だけの〕出来事{できごと}、〔偶発的{ぐうはつ てき}な〕事件{じけん} ・
\\	大事件はいとも簡単に起こり、あっけなく解決するものだ。付随物{ふずい ぶつ}、付帯条件{ふたい じょうけん}〔深刻{しんこく}な結果{けっか}を招きかねない国際的{こくさい てき}な〕事件{じけん}、衝突{しょうとつ}、紛争{ふんそう}インシデント◆航空・原子力などの分野で、安全上問題があった事例・異常事象のうち、事故
\\	より軽微なもの。区別は定義による。例えば民間航空(シカゴ条約)では、死者・重傷者や機体の損壊・喪失を伴えば事故、それ以外はインシデント。【形】起こりがちな、起こりやすい、ありがちな付帯的{ふたい てき}な、付随的{ふずい てき}な《物理》投射{とうしゃ}する、入射{にゅうしゃ}するレベル4、発音
\\	、カナインシデント、インスィデント、変化《複》
\\	、分節
\\	このことを私の尊敬のあかしと見なしていただければ幸いです。	変化形 
\\	【他動】~の証人{しょうにん}になる、~を証言{しょうげん}[証明{しょうめい}]する、~を証拠{しょうこ}だてる【名】〔真偽を明らかにする〕証拠{しょうこ}、証言{しょうげん} ・
\\	このことを私の尊敬のあかしと見なしていただければ幸いです。ネタ◆犯罪の証拠痕跡{こんせき}、形跡{けいせき}、兆候{ちょうこう}、印レベル3、発音
\\	、カナエビデンス、変化《複》
\\	、《動》
\\	、分節
\\	事情によっては話が違ってくる。 ・
\\	事情があって、私はそれをすることができませんでした。	変化形 
\\	《複》
\\	【名】周囲{しゅうい}の事情{じじょう}、環境{かんきょう}、状況{じょうきょう}◆通例
\\	事情によっては話が違ってくる。 ・
\\	事情があって、私はそれをすることができませんでした。境遇{きょうぐう}、身の上、素性{すじょう}◆通例
\\	〔情況を変化させる〕出来事{できごと}、事件{じけん}状況{じょうきょう}(を物語る)証拠{しょうこ}〔話の〕詳細物々{ものもの}しさレベル4、発音
\\	、カナサーカマスタンス、サーカムスタンス、サカマスタンス、変化《複》
\\	、分節
\\	そのスキャンダルによって会社の面目は丸つぶれになった。	変化形 
\\	《複》
\\	【名】〔一般{いっぱん}の人が抱く〕評判{ひょうばん}、風評{ふうひょう}〔一般{いっぱん}の人が抱く〕高い評価{ひょうか}[信用{しんよう}]、良い評判{ひょうばん} ・
\\	そのスキャンダルによって会社の面目は丸つぶれになった。〔人や物の知られている〕性質{せいしつ}、属性{ぞくせい}、能力{のうりょく}レベル4、発音
\\	、カナレピュテイション、変化《複》
\\	、分節
\\	ぜひお会いして、当社のプランについて再検討したいと存じます。 ・
\\	大事な試験の前夜、彼はすべてのノートを復習した。 ・
\\	これまでの進捗についてザッと説明していただけますか?	変化形 
\\	【自動】批評{ひひょう}を書く、批評{ひひょう}する、論評{ろんぴょう}する【他動】~を再調査{さい ちょうさ}[再検討{さいけんとう}・再考察{さい こうさつ}]する見直す◆【語源】
\\	(もう一度)
\\	(見る) ・
\\	ぜひお会いして、当社のプランについて再検討したいと存じます。~を復習{ふくしゅう}する、~をおさらいする ・
\\	大事な試験の前夜、彼はすべてのノートを復習した。~を回顧{かいこ}[回想{かいそう}]する、振り返る~を視察{しさつ}する、検閲{けんえつ}する、閲兵{えっぺい}する~を概説{がいせつ}する、概観{がいかん}する~を批評{ひひょう}する、論評{ろんぴょう}する《法律》〔上級審{じょうきゅうしん}で〕~を再審理{さい しんり}する【名】再調査{さい ちょうさ}、再検討{さいけんとう}、再考察{さい こうさつ}〔過去{かこ}の出来事{できごと}の〕報告{ほうこく}、説明{せつめい}、総括{そうかつ} ・
\\	これまでの進捗についてザッと説明していただけますか?〔評価{ひょうか}のための〕検査{けんさ}、点検{てんけん}、審査{しんさ}《法律》〔上級審{じょうきゅうしん}の〕再審理{さい しんり}〔作品{さくひん}や公演{こうえん}の〕批評{ひひょう}、書評{しょひょう}、レビュー《印刷》批評{ひひょう}[評論{ひょうろん}]誌〔テストの前の〕復習{ふくしゅう}、おさらい《軍事》閲兵(式){えっぺい(しき)}、観兵(式){かんぺい(しき)}〔ミュージカルの〕レビュー◆【同】
\\	レベル3、発音
\\	、カナリブュー、レビュー、変化《動》
\\	、分節
\\	【名】〔人に払われる〕敬意{けいい}、尊敬{そんけい}〔地位{ちい}などによって得られる〕名声{めいせい}、高い評判{ひょうばん}【形】高級{こうきゅう}な、立派{りっぱ}な、評判{ひょうばん}の高いレベル7、発音
\\	、カナプレステージ、プレスティージ、分節
\\	イディオムやフレーズ 
\\	【他動】~とうわさする、~をうわさする【名】うわさ、風説{ふうせつ}、風評{ふうひょう}、流言{りゅうげん}レベル2、発音
\\	、カナルーマー、変化《動》
\\	、分節
\\	彼らは新しい政治秩序のための諸原則について合意に達したことで評価された。	変化形 
\\	《複》
\\	【名】〔真理{しんり}の基礎{きそ}となる〕原理{げんり}、公理{こうり}〔行動{こうどう}の〕原則{げんそく}、指針{ししん}、主義{しゅぎ} ・
\\	彼らは新しい政治秩序のための諸原則について合意に達したことで評価された。〔道徳的{どうとく てき}〕道理{どうり}、道義{どうぎ}〔組織{そしき}や制度{せいど}の〕基本的性質{きほん てき せいしつ}、本質{ほんしつ}〔自然{しぜん}や機械{きかい}の〕動作{どうさ}[作用{さよう}]原理{げんり}〔物事{ものごと}の〕本源{ほんげん}、素因{そいん}《化学》素◆化学物質の性質を決定する成分。レベル3、発音
\\	、カナプリンシプル、プリンスィプル、変化《複》
\\	、分節
\\	変化形 
\\	【自動】一致{いっち}する、調和{ちょうわ}する【他動】~を(~と)一致{いっち}[合致{がっち}]させる〔許可{きょか}などを〕与える認める、許容{きょよう}する【名】調和{ちょうわ}、一致{いっち}、合意{ごうい}、合致{がっち}◆【類】
\\	〔他国{たこく}などとの〕協定{きょうてい}、協約{きょうやく}和音{わおん}レベル4、発音
\\	、カナアコード、変化《動》
\\	、分節
\\	細菌学の前提科目としての化学の重要性に対しては誰も異議を唱えなかった。 ・
\\	10チームがその賞金を争った。 ・
\\	戦時中政府の活動に抵抗した人が多くいた。 ・
\\	彼は、アラブ諸国とイスラエルの紛争解決のための新たなプランを示さなかった。	変化形 
\\	【自動】議論{ぎろん}する、口論{こうろん}する【他動】~を議論{ぎろん}する、~について論争{ろんそう}する~を疑う、~に異議{いぎ}を唱える◆あるものの真正さや正当性を疑うこと。 ・
\\	細菌学の前提科目としての化学の重要性に対しては誰も異議を唱えなかった。~を獲得{かくとく}しようとする、争って~を得ようとする ・
\\	10チームがその賞金を争った。~と対立{たいりつ}[敵対{てきたい}]する、~に抵抗{ていこう}する ・
\\	戦時中政府の活動に抵抗した人が多くいた。【名】議論{ぎろん}、論争{ろんそう}、口論{こうろん}不和{ふわ}、紛争{ふんそう} ・
\\	彼は、アラブ諸国とイスラエルの紛争解決のための新たなプランを示さなかった。労働争議{ろうどう そうぎ}レベル5、発音
\\	、カナディスピュート、変化《動》
\\	、分節
\\	その問題は論争になっている。◆通例、反対の立場にいる者同士が公開の場で行うものを指す。 ・
\\	論争するのは好きではない。	変化形 
\\	《複》
\\	【名】論争{ろんそう}、論議{ろんぎ}、議論{ぎろん}◆【用法】可算名詞 ・
\\	その問題は論争になっている。◆通例、反対の立場にいる者同士が公開の場で行うものを指す。論争{ろんそう}[論議{ろんぎ}]する[に加わる]こと◆【用法】不可算名詞 ・
\\	論争するのは好きではない。レベル6、発音
\\	、カナコントロバーシー、コントロバースィ、変化《複》
\\	、分節
\\	衝突が起きる可能性がある。 ・
\\	紛争と陰謀が国を苦しめた。	変化形 
\\	【自動】対立{たいりつ}する、矛盾{むじゅん}する〈古〉戦争{せんそう}になる【名】〔考え方などの〕不一致{ふいっち}、対立{たいりつ}、衝突{しょうとつ} ・
\\	衝突が起きる可能性がある。〔長期の〕争い{あらそい}、紛争{ふんそう}、闘争{とうそう}、戦争{せんそう}、戦闘{せんとう} ・
\\	紛争と陰謀が国を苦しめた。《心理学》葛藤{かっとう}〔文学作品{ぶんがく さくひん}の登場人物{とうじょう じんぶつ}の〕対立{たいりつ}、緊張{きんちょう}◆新たな展開を生み出す元となるもの。レベル4、発音《名》
\\	《動》
\\	、カナカンフリクト、コンフリクト、変化《動》
\\	、分節
\\	変化形 
\\	《複》
\\	【名】対戦{たいせん}[競争{きょうそう}]相手{あいて}、対抗者{たいこう しゃ}、敵{てき}、反対者{はんたいしゃ}、異議申立人{いぎ もうしたてにん}
\\	反対派{はんたい は}【形】敵対{てきたい}する、対立{たいりつ}するレベル5、発音
\\	、カナオポーネント、オポウネント、変化《複》
\\	、分節
\\	誰かが私のクレジットカード番号と暗証番号を洩らした。 ・
\\	と
\\	との間でライセンス契約についての妥協が成立しました。 ・
\\	妥協は最良の友である。 ・
\\	妥協の余地はない。 ・
\\	これによって必要な歩み寄りを達成するために強大国の妥協を図ることになる。	変化形 
\\	【自動】妥協{だきょう}する、譲歩{じょうほ}する、歩み寄る、和解{わかい}する◆【語源】
\\	(共に)+
\\	(約束{やくそく}する)【他動】危うくする、危険にさらす〔権限の無い人に秘密情報を〕洩らす ・
\\	誰かが私のクレジットカード番号と暗証番号を洩らした。〔セキュリティーシステムに〕不正{ふせい}アクセス[侵入{しんにゅう}]する〔信用・評判などを〕落とす〔名誉・経歴・体面などを〕傷つける~の体面{たいめん}に関わる〔契約上の権利を〕損なう【名】譲歩{じょうほ}、妥協(案){だきょう(あん)}、歩み寄り{あゆみより}、和解{わかい}、示談{じだん} ・
\\	と
\\	との間でライセンス契約についての妥協が成立しました。 ・
\\	妥協は最良の友である。 ・
\\	妥協の余地はない。 ・
\\	これによって必要な歩み寄りを達成するために強大国の妥協を図ることになる。危険にさらすこと〔名誉・経歴・体面などを〕傷つけること〔信用・評判などを〕落とすこと情報漏洩{じょうほう ろうえい}、セキュリティ侵害{しんがい}レベル5、発音
\\	、カナコンプロマイズ、変化《動》
\\	、分節
\\	変化形 
\\	《複》
\\	【名】発端{ほったん}、端緒{たんちょ}◆物事が存在を始めた時点やその原因。◆【用法】通例
\\	起源{きげん}、源泉{げんせん}◆物事が発展する元となったものや、それが起こった場所。◆【用法】通例
\\	〔人や家族{かぞく}の〕先祖{せんぞ}、血統{けっとう}、家柄{いえがら}◆【用法】通例
\\	《解剖》起始{きし}◆骨格筋が骨に付着している部分。《生化学》複製起点{ふくせい きてん}
\\	複製の際の合成開始部位のヌクレオチド配列《数学》〔座標{ざひょう}の〕原点{げんてん}レベル4、発音
\\	、カナオリジン、変化《複》
\\	、分節
\\	イエスズメは広く分布する種です。 ・
\\	この森には約20種の鳥が住んでいます。		【名】〔生物学{せいぶつがく}の〕種◆生物分類で属
\\	の下位分類にあたる。同じ種であれば相互に交配可能である。◆【参考】
\\	イエスズメは広く分布する種です。 ・
\\	この森には約20種の鳥が住んでいます。〔論理学{ろんりがく}の〕種◆共通の特徴によってまとめられ名前が付けられるグループ。類
\\	の下位概念で、例えば「動物」が類であれば「人間」が種となり、「人間」が類であれば「日本人」は種になる。〔一般{いっぱん}に共通{きょうつう}の特徴{とくちょう}を持つ〕種、種類{しゅるい}人類{じんるい}、人種{じんしゅ}《カトリック》〔聖変化{せいへんか}した聖体{せいたい}の〕形態{けいたい}◆キリストの体と血に聖変化された後も残る、パンとぶどう酒の性質や外観を指す。レベル4、発音
\\	、カナスピーシーズ、変化《複》
\\	、分節
\\	【名】進化(論){しんか(ろん)}漸進的変化{ぜんしん てき へんか}展開{てんかい}《機械》旋回{せんかい}〔熱などの〕発生{はっせい}レベル5、発音
\\	、カナエボリューション、エボルーション、分節
\\	変化形 
\\	《複》
\\	【名】革新{かくしん}、刷新{さっしん}新機軸{しんきじく}、新しい考え[工夫{くふう}]レベル6、発音
\\	、カナイノベイション、変化《複》
\\	、分節
\\	【名】消灯{しょうとう}、消火{しょうか}絶滅{ぜつめつ}、消滅{しょうめつ}、死滅{しめつ}レベル9、発音
\\	、カナイクステンクション、エクスティンクション、分節
\\	私の祖先はもともとアイルランド出身です。	変化形 
\\	《複》
\\	【名】先祖{せんぞ}、祖先{そせん}◆【対】
\\	◆【同】
\\	私の祖先はもともとアイルランド出身です。〔生物{せいぶつ}の〕原種{げんしゅ}、〔製品{せいひん}などの〕原型{げんけい}先駆者{せんくしゃ}、先人{せんじん}《法律》被相続人{ひ そうぞくにん}、直系尊属{ちょっけい そんぞく}レベル4、発音
\\	、カナアンセスター、変化《複》
\\	、分節
\\	変化形 
\\	《複》
\\	【名】幼児(期){ようじ(き)}、0歳児{さいじ}、まだ歩けない赤ん坊、赤子{あかご}、乳児{にゅうじ}《法律》未成年者{みせいねん しゃ}〔事業・組織などの〕未発達{みはったつ}(段階{だんかい})、初期{しょき}【形】幼児(用){ようじ(よう)}の、幼い、幼少{ようしょう}の《法律》未成年{みせいねん}の〔事業・組織などが〕未発達{みはったつ}(段階{だんかい})の、初期{しょき}のレベル4、発音
\\	、カナインファント、変化《複》
\\	、分節
\\	変化形 
\\	《複》
\\	【1名】墓、墓所{ぼしょ}、死に場所{ばしょ}【2名】重アクセント(記号{きごう})◆左から右に見て斜めに下がるアクセント記号
\\	。古典ギリシャ語では低いピッチ。一般には副次的ストレス・アクセント【2形】厳粛{げんしゅく}な、威厳{いげん}のある、重々しい、いかめしい、真面目{まじめ}な深刻{しんこく}な、憂慮{ゆうりょ}すべき、由々しき重大{じゅうだい}な、重要{じゅうよう}な、大切{たいせつ}な《色》くすんだ、沈んだ、地味{じみ}な重アクセント(記号{きごう})の【3自他動】〔記憶{きおく}に〕刻みつけるレベル3、発音
\\	、カナグレイブ、変化《複》
\\	、《形》
\\	日本では葬式にいくらかかりますか。 ・
\\	それは君の問題です。	変化形 
\\	《複》
\\	【名】葬儀{そうぎ}、告別式{こくべつしき}、葬列{そうれつ} ・
\\	日本では葬式にいくらかかりますか。関わり合いのあること ・
\\	それは君の問題です。レベル3、発音
\\	、カナフューネラル、変化《複》
\\	、分節
\\	変化形 
\\	《複》
\\	【他動】自殺{じさつ}する◆【用法】
\\	【名】自殺{じさつ}(すること)自殺{じさつ}者[する人]自殺行為{じさつ こうい}、自滅{じめつ}〈俗〉=
\\	レベル5、発音
\\	、カナシューイサイド、シュイサイド、スーアサイド、スーイサイド、変化《動》
\\	、分節
\\	それを考えると頭が痛い。 ・
\\	どこが痛いの?頭?耳?足? ・
\\	ここが痛い。	変化形 
\\	【自動】〔体の一部{いちぶ}が絶えず鈍く〕痛む、うずく ・
\\	それを考えると頭が痛い。 ・
\\	どこが痛いの?頭?耳?足?胸を痛める、心が痛む、心を痛める、つらい思いをする、気の毒に思う、同情{どうじょう}する〔~したくて〕たまらない[ムズムズする・ウズウズする]、〔~することを〕熱望{ねつぼう}する、切に願う、切望{せつぼう}する憧{あこが}れる、思いこがれる、ワクワクする【名】〔長く続く鈍い〕痛み、うずき◆【類】
\\	(鋭い痛み) ・
\\	ここが痛い。切望{せつぼう}、熱望{ねつぼう}レベル4、発音
\\	、カナエイク、変化《動》
\\	変化形 
\\	《複》
\\	【名】軍医{ぐんい}、船医{せんい}外科医{げかい}、執刀医{しっとうい}レベル5、発音
\\	、カナサージャン、変化《複》
\\	、分節
\\	変化形 
\\	【自動】〈話〉強引{ごういん}に押し進む、力ずくで割り込む【名】〔筋繊維{きんせんい}が構成{こうせい}する〕筋肉{きんにく}(の器官{きかん})◆骨格に付属する骨格筋
\\	すべて横紋筋)と、それ以外の内臓筋
\\	の2種類に分ける場合がある。◆【語源】ラテン語の
\\	からで、上腕部の筋肉の動きが、まるで皮膚の下に小さなネズミがいるように見えることから。筋(肉)(繊維{せんい})◆随意筋
\\	の横紋筋
\\	、不随意筋
\\	の平滑筋
\\	、および不随意筋の心筋
\\	の3種類がある。〈話〉筋力{きんりょく}、脚力{きゃくりょく}、腕力{わんりょく}〔政治{せいじ}や経済{けいざい}などの分野{ぶんや}の〕権力{けんりょく}、権威{けんい}、影響力{えいきょうりょく}〈俗〉用心棒{よう じんぼう}レベル3、発音
\\	、カナマスル、変化《動》
\\	、分節
\\	変化形 
\\	《複》
\\	【名】顕微鏡{けんびきょう}レベル6、発音
\\	、カナマイクロスコープ、マイクロスコウプ、ミクロスコープ、変化《複》
\\	、分節
\\	変化形 
\\	《複》
\\	【名】道具{どうぐ}、器具{きぐ}、機器{きき}、計器{けいき}楽器{がっき}手段{しゅだん}法律文書{ほうりつ ぶんしょ}〔法律文書としての〕証券、(約束)手形証書{しょうしょ}◆契約書など法律的効果を伴う文書レベル3、発音
\\	、カナインストゥルメント、インストルメント、変化《複》
\\	、分節
\\	変化形 
\\	《複》
\\	【名】《生物》遺伝子{いでんし}レベル6、発音
\\	、カナジーン、変化《複》
\\	5時すぎに携帯にかけて。	変化形 
\\	《複》
\\	【名】〔分割された〕小部屋{こべや}、個室{こしつ}《生物》細胞{さいぼう}、気泡{きほう}〔刑務所の〕監房{かんぼう}、独房{どくぼう}◆【同】
\\	《動物》ハチの巣の個室{こしつ}《宗教》修道士{しゅうどうし}が鍛錬{たんれん}する小部屋{こべや}《政治》小集団{しょう しゅうだん}、下部組織{かぶ そしき}《コ》〔表計算ソフトの〕セル、升目{ますめ}《電気》セル◆【同】
\\	◆電池のセル。セルを複数接続して、電池の電流・電圧を増やす。太陽電池◆【同】
\\	〈話〉携帯電話{けいたい でんわ}◆【同】
\\	5時すぎに携帯にかけて。レベル3、発音
\\	、カナセル、変化《複》
\\	変化形 
\\	《複》
\\	【名】《言語学》〔文法上{ぶんぽう じょう}の〕性、ジェンダー◆言語によっては、名詞や代名詞が
\\	などというジェンダーに分類される。ジェンダー、〔社会的{しゃかい てき}・文化的{ぶんか てき}〕性◆生物学上の性
\\	と対比されるが、
\\	という表現を避けるために、その遠回しな表現としても用いられる。レベル8、発音
\\	、カナジェンダー、分節
\\	戸外のアンテナは風雨にさらされるので定期的に点検する必要がある。	変化形 
\\	《複》
\\	【名】〔大きな集団{しゅうだん}を構成{こうせい}する〕成分{せいぶん}、要素{ようそ}《化学》〔化学反応{かがく はんのう}の最小構成要素{さいしょう こうせい ようそ}の〕元素{げんそ}〔原因{げんいん}となる〕要因{よういん}少し、わずか〈古〉四大元素{よんだい げんそ}の一つ◆宇宙の基本構成要素と考えられた地
\\	・風
\\	・火
\\	・水
\\	の四大元素
\\	のこと。〔動物{どうぶつ}の〕自然{しぜん}の生息地{せいそくち}[環境{かんきょう}]、〔人の〕得意分野{とくい ぶんや}《電気》〔電化製品{でんか せいひん}の〕発熱体{はつねつ たい}、ヒーター部《電気》〔回路{かいろ}の〕素子{そし}《数学》〔図形{ずけい}の〕構成要素{こうせい ようそ}◆幾何学図形を構成する点
\\	、線
\\	、面
\\	のこと。《数学》行列{ぎょうれつ}の元《数学》集合{しゅうごう}の要素{ようそ}光学素子{こうがく そし}◆光学系を構成するレンズ、ミラーなどを指す。《天文》〔天体{てんたい}の〕軌道要素{きどう ようそ}◆【同】
\\	〔天候{てんこう}を構成{こうせい}する〕自然{しぜん}の力◆風、雨、寒さ、日光など、自然の厳しい力を表すもの。◆通例
\\	戸外のアンテナは風雨にさらされるので定期的に点検する必要がある。
\\	〔学問{がくもん}などの〕原理{げんり}、基礎{きそ}
\\	〈キリスト教〉聖体{せいたい}の一つ◆パンとワインのこと。レベル3、発音
\\	、カナエリメント、エレメント、変化《複》
\\	、分節
\\	内容は形式よりもはるかに重要である。	変化形 
\\	《複》
\\	【名】物質{ぶっしつ}、個体{こたい}、材料{ざいりょう}実体{じったい}、実質{じっしつ}、本質{ほんしつ}内容{ないよう}、大要{たいよう}、要旨{ようし} ・
\\	内容は形式よりもはるかに重要である。財産{ざいさん}麻薬{まやく}、薬物{やくぶつ}
\\	大部分{だいぶぶん}レベル4、発音
\\	、カナサブスタンス、変化《複》
\\	、分節
\\	強風に煽られた火災は荒涼とした土地を少なくとも100エーカー焼き焦がした。	変化形 
\\	【自動】〔乗り物が〕燃料{ねんりょう}を積む[入れる]【他動】〔燃える物を~に〕くべる、入れる、注ぐ ・
\\	強風に煽られた火災は荒涼とした土地を少なくとも100エーカー焼き焦がした。〔感情{かんじょう}を〕あおる、刺激{しげき}する【名】〔薪やガスなどの〕燃料{ねんりょう}〔原子炉{げんしろ}の〕核燃料{かく ねんりょう}〔生体{せいたい}が代謝{たいしゃ}する〕食物{しょくもつ}、栄養{えいよう}〔感情{かんじょう}や行動{こうどう}の〕動機{どうき}、刺激{しげき}レベル2、発音
\\	、カナフューアル、フューエル、変化《動》
\\	、分節
\\	鼻がすっかり詰まっている。 ・
\\	おなかがいっぱいです。 ・
\\	買い物に行くよ。ちょっと買ってこないといけないものがあるから。 ・
\\	この話は本当に難しいね。 ・
\\	ここが君の腕の見せどころだよ。	変化形 
\\	【他動】~に詰め込む、詰める、詰め物をする ・
\\	鼻がすっかり詰まっている。~にガツガツ食べさせる、たらふく食べさせる ・
\\	おなかがいっぱいです。〔動物{どうぶつ}の(毛)皮に〕詰め物をする、〔動物{どうぶつ}を〕剥製{はくせい}にする〈卑〉〔女性{じょせい}と〕セックスする〔革を薬品{やくひん}で処理{しょり}して〕なめす〔投票箱{とうひょう ばこ}などに〕不正票{ふせい ひょう}を投じる【名】〈話〉物、代物{しろもの}、物事{ものごと}、事柄{ことがら} ・
\\	買い物に行くよ。ちょっと買ってこないといけないものがあるから。 ・
\\	この話は本当に難しいね。〈話〉価値{かち}のないもの、がらくた才能{さいのう}、素質{そしつ}、器量{きりょう} ・
\\	ここが君の腕の見せどころだよ。材料{ざいりょう}、要素{ようそ}《野球》球種{きゅうしゅ}、配球{はいきゅう}〔自分に関係した〕仕事内容〈英〉(毛)織物{おりもの}〈俗〉現金{げんきん}、現生{げんせい}〈俗〉麻薬{まやく}、ヘロインレベル2、発音
\\	、カナスタッフ、変化《動》
\\	まる1時間歩いた。	変化形 
\\	《複》
\\	【名】固体{こたい}、固形物{こけいぶつ}《数学》立体{りったい}〈俗〉好意{こうい}、親切{しんせつ}な行為{こうい}◆【同】
\\	◆【参考】
\\	【形】固体{こたい}の、固形{こけい}の、固形化{こけい か}した中身{なかみ}の詰まった、中空{ちゅうくう}でない〔金・銀などが〕中まで同じ材質の〔材質・色などが〕純粋の硬い、頑丈{がんじょう}な《数学》立体{りったい}の確かな、信頼{しんらい}できる、頼りになる真面目{まじめ}な、実直{じっちょく}な、堅実{けんじつ}な、良識{りょうしき}のある、良識派{りょうしき は}の色が同一の、無地の、無垢{むく}の、濃淡のない、ベタの隙間{すきま}のない、(線が)途切れていない、実線{じっせん}の〔時間が〕絶え間のない、まるまる ・
\\	まる1時間歩いた。〔意見{いけん}が〕一致{いっち}した、全会一致{ぜんかい いっち}の《言語学》〔語が〕一語{いちご}の、ハイフン[スペース]で区切られていない《言語学》〔複合語{ふくごうご}が〕1語につながった◆もともと分かれていた複合語が、ハイフンを使わずにつながったもの。◆【参考】
\\	〈俗〉素晴{すば}らしい、格好いい〈英俗〉頭の鈍い、ばかなレベル2、発音
\\	、カナサリッド、ソリッド、変化《複》
\\	、分節
\\	変化形 
\\	《複》
\\	【名】液体{えきたい}《音声》流音{りゅうおん}【形】液体{えきたい}の、液状{えきじょう}の、液化{えきか}した流ちょうな透明{とうめい}な《経済》流動性{りゅうどうせい}の、いつでも現金化{げんきん か}できる、すぐに換金{かんきん}できる、現金{げんきん}になっている不安定{ふあんてい}な《音声》流音{りゅうおん}のレベル4、発音
\\	、カナリキッド、リクウィッド、変化《複》
\\	、分節
\\	差し引きして10億ドルの財源不足である。	変化形 
\\	《複》
\\	【名】〔~の〕不足{ふそく}、欠乏{けつぼう}◆【用法】
\\	~不足{ふそく}高[金額{きんがく}] ・
\\	差し引きして10億ドルの財源不足である。レベル4、発音
\\	、カナショーテージ、ショーテッジ、変化《複》
\\	、分節
\\	変化形 
\\	《複》
\\	【名】押し下げること、下がること、下降{かこう}、沈下{ちんか}、陥没{かんぼつ}くぼみ、低地{ていち}恐慌{きょうこう}、不況{ふきょう}、不景気{ふけいき}〔気持ちの〕落ち込み、意気消沈{いき しょうちん}、絶望{ぜつぼう}《病理》鬱病{うつびょう}、抑鬱症{よくうつしょう}〔活動{かつどう}や機能{きのう}などの〕低下{ていか}、減退{げんたい}、減衰{げんすい}《気象》低気圧{ていきあつ}〔地磁気{ちじき}の〕伏角{ふっかく}◆観測者がいる平面と地磁気の向きがなす角度。北半球では
\\	極、南半球では
\\	極が水平面の下を指し、この角度が伏角になる。《天文》伏角{ふっかく}◆天球地平線の下に位置する天体の地平線からの角度。〔測量{そくりょう}の〕俯角{ふかく}◆観測者のいる平面と、水平線下にある被観測物との角度。◆【同】
\\	レベル5、発音
\\	、カナデプレッション、ディプレッション、変化《複》
\\	、分節
\\	米国麻薬取締局は彼の税金滞納に片をつけ、ボーナスを支給するという約束を破った。 ・
\\	彼は耳のところまで借金につかっている。/窮地に陥っている。/借金で首が回らない。	変化形 
\\	《複》
\\	【名】借金{しゃっきん}、借り、負債{ふさい} ・
\\	米国麻薬取締局は彼の税金滞納に片をつけ、ボーナスを支給するという約束を破った。借金{しゃっきん}[借り・恩義{おんぎ}]がある状態{じょうたい} ・
\\	彼は耳のところまで借金につかっている。/窮地に陥っている。/借金で首が回らない。恩義{おんぎ}、おかげレベル4、発音
\\	、カナデットゥ、変化《複》
\\	変化形 
\\	時給{じきゅう}
\\	変化形 
\\	【1他動】~を合計{ごうけい}する、要約{ようやく}する【1名】合計{ごうけい}、和{わ}、計、金額{きんがく}【2名】《貨幣》スム◆ウズベキスタン
\\	の貨幣単位。レベル3、発音
\\	、カナサム、変化《動》
\\	当
\\	トイ社の玩具をお買い求めいただきありがとうございます。 ・
\\	このクーポンは一点のお買い上げに限り使用できます。	変化形 
\\	【自動】購入{こうにゅう}する、仕入れる{しいれる}【他動】~を購入{こうにゅう}する、~を仕入れる{しいれる} ・
\\	当
\\	トイ社の玩具をお買い求めいただきありがとうございます。〔努力{どりょく}や犠牲{ぎせい}を払って〕~を獲得{かくとく}[取得{しゅとく}]する〔てこなどで〕~を動かす[持ち上げる]【名】購入{こうにゅう}、購買{こうばい}、仕入れ{しいれ}購入品{こうにゅう ひん}、仕入れ商品{しいれ しょうひん} ・
\\	このクーポンは一点のお買い上げに限り使用できます。〔努力{どりょく}や犠牲{ぎせい}を払った〕獲得{かくとく}、入手{にゅうしゅ}〔手などで〕つかむ[握る]こと、手[足]掛かり《機械》〔機械効率{きかい こうりつ}を生む〕滑車{かっしゃ}、てこ〔権力拡大{けんりょく かくだい}の〕手段{しゅだん}、足掛{あしが}かりレベル4、発音
\\	、カナパーチェス、変化《動》
\\	、分節
\\	【自動】商品{しょうひん}を売買{ばいばい}する【他動】〔商品{しょうひん}などを〕売買{ばいばい}する〔商品{しょうひん}などを〕宣伝{せんでん}する【名】〈集合的に〉商品{しょうひん}、製品{せいひん}、品物{しなもの}、在庫品{ざいこひん}レベル7、発音
\\	、カナマーチャンダイズ、変化《動》
\\	、分節
\\	イディオムやフレーズ 
\\	【名】〔集合的{しゅうごう てき}に〕家具{かぐ}、調度品{ちょうどひん}◆椅子やテーブルなどの、室内の移動できる備品。《印刷》ファーニチャー◆印刷面に余白を作るための込め物で、活字と活字の間に入れる、金属やプラスチック製の薄片。〈古〉〔船の〕艤装{ぎそう}、〔馬の〕馬具{ばぐ}レベル2、発音
\\	、カナファーニチャー、分節
\\	彼は首相になると、直ちに組閣に着手しました。	変化形 
\\	《複》
\\	【名】飾り戸棚、キャビネット閣僚{かくりょう}
\\	とも表記。 ・
\\	彼は首相になると、直ちに組閣に着手しました。《コ》キャビネット、データの保管庫{ほかんこ}〈古〉〔小さな〕私室{ししつ}【形】飾り戸棚の、キャビネットの内閣{ないかく}の
\\	とも表記。〔写真{しゃしん}が〕キャビネ判のレベル4、発音
\\	、カナキャビネット、変化《複》
\\	、分節
\\	通りは、きょうは、歩行者天国[車両の乗り入れ禁止区域]です。	変化形 
\\	《複》
\\	【名】〔地上{ちじょう}の輸送手段{ゆそう しゅだん}の〕乗り物、車、車両{しゃりょう}◆通例、車輪が付いており、道路やレール上を走行するものを指す。 ・
\\	通りは、きょうは、歩行者天国[車両の乗り入れ禁止区域]です。〔宇宙{うちゅう}の〕輸送{ゆそう}[運搬{うんぱん}]船通信{つうしん}[伝達{でんたつ}]手段{しゅだん}[媒体{ばいたい}]、目的達成手段{もくてき たっせい しゅだん}〔隠喩{いんゆ}の〕媒体{ばいたい}◆【対】
\\	〔画材{がざい}の〕展色剤{てんしょくざい}、媒剤{ばいざい}◆顔料と混ぜ合わせて絵の具を作り、乾燥させて画面に顔料を定着させる接着剤の働きをするもので、絵の具の種類により卵、乾性油、アクリル樹脂などがある。賦形剤{ふけいざい}◆薬の形を整えるために加えられるもの(薬効を持たない)。例えば、ビタミン剤は、主薬のビタミンの量が少ないので、増量剤として粉末が加えられ、整形される。レベル3、発音
\\	、カナビーイクル、ビークル、ビヒクル、変化《複》
\\	、分節
\\	裏には裏がある。	変化形 
\\	【自動】回転{かいてん}する◆車輪のように軸を中心としてくるりと向きを変える、振り向く〔考えや・態度などが〕変わる【他動】回転{かいてん}させる動かす、押す◆車輪の付いたものを【名】車輪{しゃりん}、〔回転する〕輪〔自動車{じどうしゃ}の〕ハンドル◆【同】
\\	〈話〉紡ぎ車、糸車{いとぐるま}〈話〉〔陶芸{とうげい}の〕ろくろ回転花火{かいてん はなび}〔いすなどの〕キャスター〈話〉自転車{じてんしゃ}〔中世{ちゅうせい}の〕拷問用車輪{ごうもん よう しゃりん}回転{かいてん}、輪転{りんてん}、旋回{せんかい}
\\	推進力{すいしんりょく}、動力{どうりょく}
\\	〈話〉自動車{じどうしゃ}、車〈俗〉大立て者{おおだてもの}、大物{おおもの}運命{うんめい}の紡ぎ車◆【同】
\\	変化{へんか}、変転{へんてん}、転変{てんぺん} ・
\\	裏には裏がある。レベル2、発音
\\	、カナホイール、ホウィール、変化《動》
\\	変化形 
\\	《複》
\\	【名】豪華{ごうか}さ、ぜいたくぜいたく品、高級品{こうきゅうひん}【形】豪華{ごうか}なぜいたくな、高級{こうきゅう}なレベル4、発音
\\	、カナラクシャリー、ラグジャリー、変化《複》
\\	、分節
\\	その村の繁栄は果樹栽培から生じている。 ・
\\	私はパレスチナ人が平和に暮らし繁栄すると信じている。		【名】繁盛{はんじょう}、幸運{こううん}、繁栄{はんえい}、隆盛{りゅうせい}、成功{せいこう}◆【動】
\\	◆【形】
\\	その村の繁栄は果樹栽培から生じている。 ・
\\	私はパレスチナ人が平和に暮らし繁栄すると信じている。レベル4、発音
\\	、カナプラスペリティー、プロスペリティー、分節
\\	イディオムやフレーズ 
\\	【名】〔心身{しんしん}および経済的{けいざい てき}〕幸福{こうふく}、安泰{あんたい}、繁栄{はんえい}福祉事業{ふくし じぎょう}◆【同】
\\	〔政府{せいふ}による〕生活保護{せいかつ ほご}【形】生活保護{せいかつ ほご}に関する、生活保護支給{せいかつ ほご しきゅう}のレベル4、発音
\\	、カナウェルフェア、分節
\\	位高ければ特権あり。◆ことわざ ・
\\	権力や特権が人々の声をかき消してはならない。 ・
\\	その国はまだ特権社会である。 ・
\\	彼はとても賢い子どもですので、その成長に関われたことを名誉に思います。	変化形 
\\	《複》
\\	【他動】~に特権{とっけん}を与える~を免責{めんせき}する【名】〔個人{こじん}や階層{かいそう}が享受{きょうじゅ}する〕特権{とっけん}、特典{とくてん}、恩恵{おんけい}〔身分{みぶん}や地位{ちい}などに伴って行使{こうし}できる〕(免責{めんせき})特権{とっけん}、特典{とくてん} ・
\\	位高ければ特権あり。◆ことわざ ・
\\	権力や特権が人々の声をかき消してはならない。〔社会{しゃかい}システムとしての〕特権{とっけん} ・
\\	その国はまだ特権社会である。〔個人{こじん}の〕名誉{めいよ}、栄誉{えいよ} ・
\\	彼はとても賢い子どもですので、その成長に関われたことを名誉に思います。《法律》秘匿{ひとく}[機密{きみつ}]特権{とっけん}◆弁護士と顧客、医師と患者などの間で交わされた情報は機密とすることができる権利。
\\	《法律》基本的人権{きほん てき じんけん}株式売買選択権{かぶしき ばいばい せんたくけん}◆【同】
\\	レベル4、発音
\\	、カナプリバレッジ、プリビレッジ、変化《動》
\\	、分節
\\	変化形 
\\	《複》
\\	【名】保険{ほけん}、保険{ほけん}契約{けいやく}[証書{しょうしょ}]保険金{ほけんきん}、保険料{ほけんりょう}保護手段{ほご しゅだん}【形】保険{ほけん}のレベル3、発音
\\	、カナインシュアランス、変化《複》
\\	、分節
\\	熱された金属は冷えるときに収縮する。 ・
\\	経済はインフレ調整後で年率0.6%収縮しました。	変化形 
\\	【自動】〔規模や程度が〕縮小{しゅくしょう}[収縮{しゅうしゅく}]する、縮まる ・
\\	熱された金属は冷えるときに収縮する。 ・
\\	経済はインフレ調整後で年率0.6%収縮しました。《医》〔筋肉が〕収縮{しゅうしゅく}する【他動】〔規模や程度を〕縮小{しゅくしょう}[収縮{しゅうしゅく}]させる契約{けいやく}する病気{びょうき}にかかる~の習慣{しゅうかん}がつく《言語学》〔語を〕縮約{しゅくやく}する【名】〔公式文書による〕契約{けいやく}、規約{きやく}、協定{きょうてい}、約定(書){やくじょう(しょ)}〈古〉婚姻契約書{こんいん けいやくしょ}〈話〉〔殺し屋の〕殺しの仕事{しごと}〔トランプのブリッジの〕コントラクト◆ビッドしたトリックの数、最高位のビッド、またはコントラクト・ブリッジそのものを指す。レベル3、発音《名》
\\	《動》
\\	、カナカントゥラクト、コントゥラクト、変化《動》
\\	、分節
\\	あなたは後悔しないこと請け合いですよ。 ・
\\	成功する保証はない。 ・
\\	これなら、きっとご満足いただけます。◆【場面】購入・使用などを勧めるとき。 ・
\\	弊社の製品はすべて保証付きです。 ・
\\	私たちは、あなたに対して100%の満足を保証します。/当社は、お客さまに100%満足していただけることをお約束致します。 ・
\\	私にできることなら何でもやる、ということは保証します。	変化形 
\\	【他動】〔~が確かであると〕保証{ほしょう}する、請け合う ・
\\	あなたは後悔しないこと請け合いですよ。 ・
\\	成功する保証はない。 ・
\\	これなら、きっとご満足いただけます。◆【場面】購入・使用などを勧めるとき。〔保証書{ほしょうしょ}で製品{せいひん}の品質{ひんしつ}を〕保証{ほしょう}する ・
\\	弊社の製品はすべて保証付きです。《法律》〔借金{しゃっきん}や義務{ぎむ}の〕保証人{ほしょうにん}となる(人)に対して~を保証{ほしょう}[約束{やくそく}]する ・
\\	私たちは、あなたに対して100%の満足を保証します。/当社は、お客さまに100%満足していただけることをお約束致します。 ・
\\	私にできることなら何でもやる、ということは保証します。【名】〔ある結果{けっか}の〕保証{ほしょう}、請け合い◆【同】
\\	〔製品{せいひん}の〕保証(書){ほしょう(しょ)}◆【同】
\\	《法律》保証契約{ほしょう けいやく}、請け合い◆【同】
\\	《法律》被保証人{ひ ほしょう にん}《法律》保証人{ほしょうにん}◆【同】
\\	レベル4、発音
\\	、カナギャランティー、変化《動》
\\	、分節
\\	人の好みはさまざまです。/十人十色。/人それぞれ意見は異なります。 ・
\\	給料は私の銀行口座に振り込まれました。	変化形 
\\	【自動】〔~の理由{りゆう}・原因{げんいん}を〕説明{せつめい}する ・
\\	人の好みはさまざまです。/十人十色。/人それぞれ意見は異なります。〔~の割合{わりあい}を〕占める責任{せきにん}を取る、釈明{しゃくめい}をする支出報告{ししゅつ ほうこく}をする【他動】~を…と見なす【名】(預金{よきん})口座{こうざ}◆【同】
\\	給料は私の銀行口座に振り込まれました。計算{けいさん}、計算書{けいさんしょ}、会計{かいけい}、勘定{かんじょう}、勘定書{かんじょうしょ}、明細書{めいさいしょ}掛け勘定{かんじょう}得意先{とくいさき}、顧客{こきゃく}、取引{とりひき}説明{せつめい}、説明書{せつめいしょ}、詳細{しょうさい}、報告{ほうこく}、記事{きじ}、答弁{とうべん}理由{りゆう}、根拠{こんきょ}評価{ひょうか}、価値{かち}、重要性{じゅうようせい}
\\	〈英〉《会計》財務諸表{ざいむ しょひょう}、計算書類{けいさん しょるい}、決算(書){けっさん(しょ)}レベル2、発音
\\	、カナアカウント、変化《動》
\\	、分節
\\	変化形 
\\	《複》
\\	【名】指図{さしず}、指示{しじ}、命令{めいれい}◆通例
\\	教育{きょういく}、教授{きょうじゅ}使用説明書{しよう せつめいしょ}、取扱説明書{とりあつかい せつめいしょ}レベル3、発音
\\	、カナインストゥラクション、インストラクション、変化《複》
\\	、分節
\\	統合参謀本部議長はソマリアから大急ぎで撤退すべきではないと警告している。 ・
\\	彼は私にこの部屋に入るなと警告しました。	変化形 
\\	【自動】警告{けいこく}する、戒める ・
\\	統合参謀本部議長はソマリアから大急ぎで撤退すべきではないと警告している。【他動】警告{けいこく}する、戒める ・
\\	彼は私にこの部屋に入るなと警告しました。【名】〔危険などに対する〕注意{ちゅうい}、用心{ようじん}、警戒{けいかい}、警告{けいこく}要注意人物{ようちゅうい じんぶつ}、何をしでかすか分からない人レベル4、発音
\\	、カナコーション、変化《動》
\\	、分節
\\	ブッシュ政権は、クリントン前大統領の政権とはかなり違う。	変化形 
\\	《複》
\\	【名】〔集団を運営管理する〕陣営{じんえい}◆【略】
\\	〔国を治める〕政権{せいけん}◆【略】
\\	ブッシュ政権は、クリントン前大統領の政権とはかなり違う。〔学校・病院・評議会などの団体の〕運営陣、経営陣{けいえいじん}◆【略】
\\	管理{かんり}、統治{とうち}◆【略】
\\	《イ》ネットワーク管理組織{かんり そしき}◆【略】
\\	〔薬の〕投与{とうよ}◆【略】
\\	レベル4、発音
\\	、カナアドゥミニストゥレイション、アドミニストレイション、変化《複》
\\	、分節
\\	頭があまり働かない。◆疲労や睡眠不足が原因で。 ・
\\	骨格と筋肉組織の構造と機能を学びます。 ・
\\	その晩は会合がありますので学校へは駐車しないでください。	変化形 
\\	【自動】働く、機能{きのう}する ・
\\	頭があまり働かない。◆疲労や睡眠不足が原因で。【名】機能{きのう}、作用{さよう}、働き、効用{こうよう} ・
\\	骨格と筋肉組織の構造と機能を学びます。《数学》関数{かんすう}《数学》写像{しゃぞう}◆【同】
\\	職務{しょくむ}、役割{やくわり}〔正式な〕会合{かいごう}、式典{しきてん}、催し物 ・
\\	その晩は会合がありますので学校へは駐車しないでください。《コ》関数{かんすう}《言語学》機能{きのう}レベル3、発音
\\	、カナファンクション、変化《動》
\\	、分節
\\	その学校には、素晴{すば}らしい設備が整っています。 ・
\\	市は、新しいレクリエーション施設を建設する計画です。 ・
\\	彼は最寄りの医療機関に運ばれました。	変化形 
\\	《複》
\\	【名】便宜{べんぎ}(をはかるもの)、融通{ゆうずう}施設{しせつ}、設備{せつび}、機関{きかん}◆この意味では、しばしば複数形の
\\	を用いられる。ただし、個別の1施設を指す場合は単数形が用いられる。 ・
\\	その学校には、素晴{すば}らしい設備が整っています。 ・
\\	市は、新しいレクリエーション施設を建設する計画です。 ・
\\	彼は最寄りの医療機関に運ばれました。便利{べんり}さ、たやすさ、容易{ようい}(さ)腕前{うでまえ}、器用{きよう}さ
\\	〈話〉〔ホテル・公共施設などの〕バス・トイレレベル5、発音
\\	、カナファシリティー、変化《複》
\\	、分節
\\	彼らは人権を守るための国連機構強化に合意したいと望んでいる。 ・
\\	彼はキャンパスの名物男です。ほとんど知らない人はいない。	変化形 
\\	《複》
\\	【名】〔特殊な目的を持つ〕(公共{こうきょう})機関{きかん}、(公共{こうきょう})施設{しせつ}、(公共{こうきょう})団体{だんたい}、協会{きょうかい}、機構{きこう}◆【略】
\\	彼らは人権を守るための国連機構強化に合意したいと望んでいる。制度{せいど}、慣例{かんれい}設立{せつりつ}、制定{せいてい}福祉{ふくし}[養護{ようご}]施設{しせつ}《金融》〔有力{ゆうりょく}な〕機関投資家{きかん とうしか}〈話〉〔ある地域内{ちいき ない}の〕よく知られた人[もの]、名物{めいぶつ} ・
\\	彼はキャンパスの名物男です。ほとんど知らない人はいない。レベル3、発音
\\	、カナインスチチューション、インスティチューション、インスティテューション、変化《複》
\\	、分節
\\	彼は旧ユーゴの和平実現を目指す国際会議に参画する予定です。	変化形 
\\	《複》
\\	【名】〔国・団体などの代表者たちが政治・経済・科学などの専門的な問題に関して行う正式な〕会議{かいぎ}、協議会{きょうぎかい} ・
\\	彼は旧ユーゴの和平実現を目指す国際会議に参画する予定です。〔少数の人たちが重要な問題に関して内々に行う〕相談{そうだん}、協議{きょうぎ}〈米・カナダ〉競技連盟{きょうぎ れんめい}レベル4、発音
\\	、カナカンファレンス、コンファレンス、変化《複》
\\	、分節
\\	変化形 
\\	《複》
\\	【名】地方議会{ちほう ぎかい}、〔集合的{しゅうごう てき}に〕地方議会議員{ちほう ぎかい ぎいん}〔委員{いいん}の集合{しゅうごう}としての〕評議会{ひょうぎかい}、協議会{きょうぎかい}、審議会{しんぎかい}、諮問委員会{しもん いいんかい}◆重要な決定事項や助言を行うために選出された委員の集まり。《キリスト教》〔教会{きょうかい}の〕評議員会{ひょうぎいん かい}〔評議会{ひょうぎかい}などにおける〕協議{きょうぎ}、審議{しんぎ}レベル4、発音
\\	、カナカウンシル、カウンスル、変化《複》
\\	、分節
\\	気象予報会社はミシシッピ川流域に洪水警報を出しました。 ・
\\	ホワイトハウスは連邦庁舎内での喫煙を禁止する大統領命令を出すだろう。 ・
\\	この件ではお金は論点ではありません。/これはお金の問題ではありません。 ・
\\	次号は12月発行予定です。 ・
\\	ご依頼頂きましたように、弊社月刊誌の最新号に加え、その前の2号も送らせていただきます。	変化形 
\\	【自動】〔物があることから〕生じる、出てくる〔結果{けっか}があることから〕起きる、由来{ゆらい}する〔利益{りえき}などが〕生まれる、増える〔債券{さいけん}などが〕流通{りゅうつう}する、発行{はっこう}される【他動】〔声明{せいめい}や警告{けいこく}などを〕公表{こうひょう}する、発布{はっぷ}する、発令{はつれい}する ・
\\	気象予報会社はミシシッピ川流域に洪水警報を出しました。 ・
\\	ホワイトハウスは連邦庁舎内での喫煙を禁止する大統領命令を出すだろう。〔公的機関{こうてき きかん}が衣服{いふく}や食料{しょくりょう}などを〕支給{しきゅう}する、配給{はいきゅう}する〔債券{さいけん}や手形{てがた}などを〕発行{はっこう}する、振り出す〔新聞{しんぶん}や書籍{しょせき}などを〕出版{しゅっぱん}する、刊行{かんこう}する〔液体{えきたい}や気体{きたい}を〕放つ、放射{ほうしゃ}する《野球》〔四球{しきゅう}〕を与える【名】〔議論{ぎろん}の〕論点{ろんてん}、争点{そうてん} ・
\\	この件ではお金は論点ではありません。/これはお金の問題ではありません。〔問題{もんだい}の〕核心{かくしん}、急所{きゅうしょ}〔困難{こんなん}や問題{もんだい}の〕決着{けっちゃく}、結果{けっか}〔世間{せけん}の〕関心事{かんしんじ}、注目{ちゅうもく}を集める事柄{ことがら}〔定期刊行物{ていき かんこうぶつ}の〕号◆ある日や時期に刊行される雑誌の一冊。 ・
\\	次号は12月発行予定です。 ・
\\	ご依頼頂きましたように、弊社月刊誌の最新号に加え、その前の2号も送らせていただきます。〔出版物{しゅっぱんぶつ}の〕版、刷◆同じ版を使って印刷される書籍。〔流通{りゅうつう}している〕株式{かぶしき}、株券{かぶけん}〔新規発行{しんき はっこう}される〕債券{さいけん}、切手{きって}、通貨{つうか}〔公的機関{こうてき きかん}による〕支給{しきゅう}、交付{こうふ}、割り当て{わりあて}〔公的機関{こうてき きかん}からの〕支給{しきゅう}[配給{はいきゅう}]品[量]〔液体{えきたい}などの〕流出{りゅうしゅつ}、放出{ほうしゅつ}〔液体{えきたい}などの〕出口{でぐち}、流出口{りゅうしゅつぐち}〔傷口{きずぐち}から出る〕膿、血液{けつえき}《病理》〔膿が出る〕傷口{きずぐち}、潰瘍{かいよう}〔土地{とち}や財産{ざいさん}からの〕利益{りえき}、収益{しゅうえき}《証券》〔株式{かぶしき}などの〕銘柄{めいがら}◆銘柄は有価証券の名称を意味する。〈古〉《法律》〔人の〕子孫{しそん}、子ども、跡継ぎ{あとつぎ}レベル3、発音
\\	、カナイシュー、イッシュー、変化《動》
\\	、分節
\\	変化形 
\\	《複》
\\	【名】場合{ばあい}、事実{じじつ}、〔事実{じじつ}を例証{れいしょう}するための〕例、事例{じれい}《コ》インスタンス◆オブジェクト指向言語で、しばしば「オブジェクト」とほぼ同義で用いられる。レベル4、発音
\\	、カナインスタンス、変化《複》
\\	、分節
\\	今は本当に予算を立て(て節約し)なきゃいけない。	変化形 
\\	【自動】予算{よさん}に計上{けいじょう}する、予算{よさん}を立てる ・
\\	今は本当に予算を立て(て節約し)なきゃいけない。【他動】~を予算{よさん}に計上{けいじょう}する〔時間{じかん}・資金{しきん}などを〕割り当てる【名】予算{よさん}、予算案{よさんあん}、予算額{よさん がく}、経費{けいひ}、運営費{うんえいひ}、家計{かけい}、生活費{せいかつひ}〔限りある〕物の集まり、量〔エネルギー・光・水などの〕量皮袋{かわぶくろ}、(皮の)財布{さいふ}保管量{ほかん りょう}、割当量{わりあてりょう}【形】予算{よさん}の[に関する]〔値段{ねだん}が〕予算{よさん}に合った、手頃{てごろ}なレベル4、発音
\\	、カナバジェット、バッジェット、変化《動》
\\	、分節
\\	ゴア(副大統領)と元大統領候補ロス・ペロー氏はテレビ討論会に出演しました。	変化形 
\\	《複》
\\	【名】候補者{こうほしゃ}、志願者{しがんしゃ}、志望者{しぼうしゃ} ・
\\	ゴア(副大統領)と元大統領候補ロス・ペロー氏はテレビ討論会に出演しました。~しそうな人◆【用法】
\\	〔調査・研究などの〕対象{たいしょう}〈主に英〉受験者{じゅけんしゃ}レベル5、発音
\\	、カナカンディデイト、カンディデット、キャンディデイト、キャンディデット、変化《複》
\\	、分節
\\	その試合に勝つために彼が考えた戦略は抜群だった。 ・
\\	消費者にわれわれの製品の優れている理由を示すのは有効な販売戦略です。	変化形 
\\	《複》
\\	【名】《軍事》戦略{せんりゃく}◆包括的で大規模な作戦遂行の計画をすること、またはその学問。◆【参考】
\\	〔目標達成{もくひょう たっせい}の周到{しゅうとう}な〕策略{さくりゃく}、方略{ほうりゃく} ・
\\	その試合に勝つために彼が考えた戦略は抜群だった。 ・
\\	消費者にわれわれの製品の優れている理由を示すのは有効な販売戦略です。《生物》戦略{せんりゃく}◆進化の過程で獲得した、子孫を残すための周到な性質や行動の組み合わせ。レベル5、発音
\\	、カナストラテジー、変化《複》
\\	、分節
\\	変化形 
\\	《複》
\\	【名】領土{りょうど}、領地{りょうち}、領域{りょういき}、領海{りょうかい}、領空{りょうくう}準州地域{ちいき}、地方{ちほう}販売区域{はんばい くいき}、受け持ち区域分野{ぶんや}縄張り{なわばり}、生息地{せいそくち}《野球》ポジション、守備{しゅび}位置{いち}[範囲{はんい}]《囲碁》地レベル4、発音
\\	、カナテリトーリ、テリトリー、変化《複》
\\	、分節
\\	彼の土地の境界はフェンスで区切られていた。 ・
\\	オリンピックの選手が人間の肉体的限界を定めている。	変化形 
\\	《複》
\\	【名】境界(線) ・
\\	彼の土地の境界はフェンスで区切られていた。限界、限度 ・
\\	オリンピックの選手が人間の肉体的限界を定めている。〔クリケットの〕バウンダリー◆フェンスなどで区切られたフィールドの境界線、およびボールが境界線に達したときまたは越えたときの得点を指す。レベル4、発音
\\	、カナバウンダリー、変化《複》
\\	、分節
\\	変化形 
\\	《複》
\\	【他動】~を地区{ちく}に分ける【名】〔行政{ぎょうせい}の〕区、地区{ちく}、区域{くいき}〔特徴{とくちょう}のある〕地域{ちいき}、地方{ちほう}◆【参考】
\\	コロンビア特別区{とくべつく}◆【同】
\\	レベル2、発音
\\	、カナディストゥリクト、ディストリクト、変化《動》
\\	、分節
\\	変化形 
\\	《複》
\\	【名】帝国{ていこく}、皇帝{こうてい}の統治{とうち}エンパイア
\\	と
\\	をかけ合わせたリンゴの品種神聖{しんせい}ローマ帝国{ていこく}◆通例、
\\	の形で用いられる。レベル4、発音
\\	、カナエンパイア、変化《複》
\\	、分節
\\	変化形 
\\	《複》
\\	【名】共和制{きょうわこくせい}、共和国{きょうわこく}◆特定の共和国を指す場合は
\\	となる。◆【略】
\\	〔共通の目的や利害を持つ〕団体{だんたい}、社会{しゃかい}◆メンバーは相互に対等の関係にある。〔連邦国家内{れんぽう こっか ない}の〕共和国{きょうわこく}レベル4、発音
\\	、カナリパブリック、変化《複》
\\	、分節
\\	変化形 
\\	《複》
\\	【名】〔明確{めいかく}に規定{きてい}された〕時代{じだい}〔ある特徴{とくちょう}を持つ〕年代{ねんだい}、時期{じき}〔地球史{ちきゅう し}の〕代◆古生代
\\	や中生代
\\	などで、いくつかの紀
\\	から構成される。〔ある時代{じだい}の最初{さいしょ}の〕出来事{できごと}、日付{ひづけ}レベル4、発音
\\	、カナイアラ、イラ、変化《複》
\\	、分節
\\	変化形 
\\	《複》
\\	【名】10年間{ねんかん}、10年《カトリック》ロザリオの一連{いちれん}◆一連は、大玉1個と小玉10個から構成され、ロザリオは通常15連からできている。一連で主の祈り
\\	を1回、天使祝詞
\\	を10回、栄唱
\\	を1回唱える。レベル4、発音
\\	、カナデケイド、変化《複》
\\	、分節
\\	事は急を要する。	イディオムやフレーズ 
\\	【自他動】〈古〉=
\\	【名】迅速{じんそく}、急ぐ必要、慌てること、急ぐこと、急ぎ、性急{せいきゅう}、軽率{けいそつ} ・
\\	事は急を要する。レベル4、発音
\\	、カナヘイスト、変化《動》
\\	変化形 
\\	《複》
\\	【名】〔空間的{くうかん てき}〕隔たり、間隔{かんかく}、距離{きょり}〔時間的{じかん てき}〕隔たり、間隔{かんかく}、合間{あいま}《数学》区間{くかん}《音楽》音程{おんてい}〈英〉休けい、幕あい◆【同】
\\	レベル4、発音
\\	、カナインターバル、インタバル、変化《複》
\\	、分節
\\	警察の定められた手順に従い犯行現場の見取り図が最初に作られる。 ・
\\	母がどんなに一生懸命働いたかといういつもの話に入ると気恥ずかしくなる。 ・
\\	個人情報の移動は所定の手続きに従って行われなければならない。 ・
\\	毎朝公園を一周する散歩を日課として楽しんでいる。 ・
\\	いつもの一日を始めたときに爆発が起きてビルが揺れた。	変化形 
\\	《複》
\\	【名】〔一連{いちれん}の〕決められた方法{ほうほう}[動作{どうさ}] ・
\\	警察の定められた手順に従い犯行現場の見取り図が最初に作られる。〔習慣的{しゅうかん てき}または機械的{きかい てき}に〕繰り返されるもの〔劇場{げきじょう}やクラブなどの〕型通りの演目{えんもく}[出し物・話・アクション]〈話〉〔通例心のこもっていない〕型にはまった行動{こうどう}[話] ・
\\	母がどんなに一生懸命働いたかといういつもの話に入ると気恥ずかしくなる。《コ》ルーティン◆ある限定された仕事を行うプログラム。【形】所定{しょてい}の、定められた通りの ・
\\	個人情報の移動は所定の手続きに従って行われなければならない。習慣{しゅうかん}[日課{にっか}]となっている、決まった、毎日のようにしている ・
\\	毎朝公園を一周する散歩を日課として楽しんでいる。ありふれた、いつもの ・
\\	いつもの一日を始めたときに爆発が起きてビルが揺れた。レベル5、発音
\\	、カナルーチーン、ルーチン、ルーティーン、変化《複》
\\	、分節
\\	習慣をやめるのは難しい。◆【同】
\\	彼は爪をかむ癖がある。 ・
\\	《名言》習慣(ハビット)に金(マニー)がかかり始めると、それは道楽(ハビー)と呼ばれる。	変化形 
\\	《複》
\\	【名】〔個人的{こじんてき}な〕習慣{しゅうかん}、癖 ・
\\	習慣をやめるのは難しい。◆【同】
\\	彼は爪をかむ癖がある。 ・
\\	《名言》習慣(ハビット)に金(マニー)がかかり始めると、それは道楽(ハビー)と呼ばれる。〈文〉〔性格{せいかく}などの〕傾向{けいこう}、気質{きしつ}〈俗〉〔薬物{やくぶつ}などへの〕依存{いそん}、中毒{ちゅうどく}〔動植物{どうしょくぶつ}の〕習性{しゅうせい}〔結晶{けっしょう}の〕癖、相修道衣{しゅうどう い}◆修道女や修道士の着る、長いゆったりしたガウン。レベル2、発音
\\	、カナハビット、変化《複》
\\	、分節
\\	変化形 
\\	《複》
\\	【名】《哲学》〔学問{がくもん}としての〕哲学{てつがく}《哲学》〔ある特定{とくてい}の〕哲学{てつがく}(体系{たいけい})《哲学》〔知識{ちしき}の体系{たいけい}の〕根本原理{こんぽん げんり}、哲理{てつり}〔人生{じんせい}の根底{こんてい}にある〕指針{ししん}、信条{しんじょう}、価値観{かちかん}〔哲学的{てつがく てき}な〕達観{たっかん}、悟りの境地{きょうち}〔考え方の〕視点{してん}、方向{ほうこう}〈やや古〉〔大学{だいがく}の〕人文科学{じんぶん かがく}◆医学、法学、神学を除くすべての学科を指す。レベル4、発音
\\	、カナフィラソフィ、フィロソフィ、変化《複》
\\	、分節
\\	変化形 
\\	《複》
\\	【名】心理学{しんりがく}、サイコロジー〔人や集団{しゅうだん}などの〕心理{しんり}(状態{じょうたい})〔人を操る〕心理作戦{しんり さくせん}レベル5、発音
\\	、カナサイコロジー、変化《複》
\\	、分節
\\	【名】〔学問分野{がくもん ぶんや}の〕生物学{せいぶつがく}〔動植物{どうしょくぶつ}の〕生態{せいたい}〔特定{とくてい}の地域{ちいき}と時代{じだい}の〕動物{どうぶつ}[植物{しょくぶつ}]相レベル5、発音
\\	、カナバイアロジィ、バイオロジー、分節
\\	変化形 
\\	《複》
\\	【名】地理学{ちりがく}地形{ちけい}レベル3、発音
\\	、カナジアグラフィ、ジオグラフィー、変化《複》
\\	、分節
\\	【名】天文学{てんもんがく}◆【略】
\\	レベル6、発音
\\	、カナアストゥラノミィ、アストゥロノミー、分節
\\	統計(学)は私の得意科目です。 ・
\\	この[これらの]統計は信頼できる。◆日本語で「一つの統計」と考える場合も、英語では複数。 ・
\\	ここ半年の統計によれば、わが社の売り上げの43%がたばこ製品による。	変化形 
\\	《単》
\\	【名】統計(学){とうけい(がく)}◆【略】
\\	◆通例、学問の「統計学」は単数扱い
\\	と同様)、収集された「数値・データ」は複数扱い。日本語の「統計」はどちらの意味にも使われる。 ・
\\	統計(学)は私の得意科目です。 ・
\\	この[これらの]統計は信頼できる。◆日本語で「一つの統計」と考える場合も、英語では複数。統計資料{とうけい しりょう}、統計{とうけい}データ、統計値{とうけいち}◆【略】
\\	ここ半年の統計によれば、わが社の売り上げの43%がたばこ製品による。レベル6、発音
\\	、カナスタティスティックス、スタティスティクス、スタチスティックス、分節
\\	変化形 
\\	《複》
\\	【名】〔学者{がくしゃ}の〕研究方法{けんきゅう ほうほう}、学風{がくふう}〔ある学問分野{がくもん ぶんや}の〕知識{ちしき}、学識{がくしき}〔大学{だいがく}の〕奨学金{しょうがくきん}、育英資金{いくえい しきん}レベル4、発音
\\	、カナスカラーシップ、スカラシップ、変化《複》
\\	、分節
\\	【名】〔文学{ぶんがく}の分野{ぶんや}としての韻文{いんぶん}による〕詩◆【同】
\\	〔ある作家{さっか}や時代{じだい}などの〕すべての詩◆【同】
\\	詩作{しさく}技法{ぎほう}[術]〔散文体{さんぶん たい}による〕詩的作品{してき さくひん}〔詩のような〕美しさ、優雅{ゆうが}さレベル3、発音
\\	、カナポウイトゥリ、ポウエトリ、ポエトリー、分節
\\	変化形 
\\	《複》
\\	【名】〔実際{じっさい}または空想{くうそう}の出来事{できごと}に関する〕話、物語{ものがたり}レベル3、発音
\\	、カナテイル、変化《複》
\\	彼はバスケットボール界の超大物です。	変化形 
\\	《複》
\\	【名】伝説{でんせつ}、伝説的人物{でんせつ てき じんぶつ}、言い伝え〔記念碑・メダルなどの〕銘〔写真・挿し絵などに添えられる〕説明文、キャプション〔地図{ちず}の〕凡例{はんれい}〔ある分野{ぶんや}などの〕巨匠{きょしょう}、偉大{いだい}な物[人] ・
\\	彼はバスケットボール界の超大物です。レベル5、発音
\\	、カナレジェンド、変化《複》
\\	、分節
\\	変化形 
\\	《複》
\\	【名】方言{ほうげん}レベル7、発音
\\	、カナダイアレクト、変化《複》
\\	、分節
\\	この世は仮の宿である。	変化形 
\\	《複》
\\	【名】〔建物{たてもの}の〕通路{つうろ}、廊下{ろうか}◆【同】
\\	〔通りにくい場所{ばしょ}の〕通路{つうろ}、通り道《解剖》管、道、孔〔自由{じゆう}な〕通行{つうこう}権[許可{きょか}]〔空路{くうろ}や海路{かいろ}の〕旅行{りょこう}、航海{こうかい}〔船旅{ふなたび}の〕乗船{じょうせん}券[権]〔異なる場所{ばしょ}への〕移動{いどう}、移住{いじゅう}〔異なる状況{じょうきょう}への〕推移{すいい}、移行{いこう} ・
\\	この世は仮の宿である。〔時の〕流れ、経過{けいか}〔文学作品{ぶんがく さくひん}などの〕一節{いっせつ}、一句{いっく}〔音楽{おんがく}の有名{ゆうめい}な〕節、楽句{がっく}《医》排便{はいべん}、排尿{はいにょう}《生化学》~代継代◆【用法】序数 
\\	〔議案{ぎあん}・法案{ほうあん}などの〕議院{ぎいん}[議会{ぎかい}・会議{かいぎ}]通過{つうか}、可決{かけつ}、成立{せいりつ}〔地名{ちめい}の一部{いちぶ}としての〕海峡{かいきょう}◆通例、
\\	と表記。レベル2、発音
\\	、カナパセジ、パセッジ、パッセージ、パッセッジ、変化《複》
\\	、分節
\\	どういう事情で?	変化形 
\\	《複》
\\	【名】文脈{ぶんみゃく}、〔文中{ぶんちゅう}の言葉{ことば}の〕前後関係{ぜんご かんけい}〔事件などの〕事情{じじょう}、背景{はいけい} ・
\\	どういう事情で?レベル4、発音
\\	、カナカンテクスト、コンテクスト、変化《複》
\\	、分節
\\	変化形 
\\	《複》
\\	【名】《建築》〔台座{だいざ}・柱身{ちゅうしん}・柱頭{ちゅうとう}から成る〕円柱{えんちゅう}〔建築{けんちく}の円柱{えんちゅう}に似た〕柱状{ちゅうじょう}の物《印刷》段、欄〔新聞{しんぶん}などの〕コラム、囲み記事{かこみきじ}〔部隊{ぶたい}や車などの〕隊列{たいれつ}、縦列{じゅうれつ}《植物》〔雄しべと雌しべが合体{がったい}した〕ずい柱《解剖》柱
\\	などの柱状の構造。〔行列{ぎょうれつ}などの〕列〔表などの〕縦の列◆【参考】
\\	レベル4、発音
\\	、カナカラム、コラム、変化《複》
\\	、分節
\\	列に並んでください。	変化形 
\\	【1他動】~を(列に)並べる、~を整列{せいれつ}させる【1名】列{れつ}、横列{おうれつ}、列になっている人 ・
\\	列に並んでください。《数学》行《コ》〔データベーステーブルの〕行家並み、街《編み物》編み段レベル2、発音
\\	、変化《複》
\\	、《動》
\\	【2自動】ボートを漕ぐ、船を漕ぐ【2他動】〔船を〕漕ぐ発音
\\	、変化《動》
\\	【3自動】口論{こうろん}する、騒ぐ【3他動】~を叱る【3名】口論{こうろん}、口げんか、騒ぎ、騒動{そうどう}発音
\\	、変化《複》
\\	、《動》
\\	【名】建築{けんちく}、建築術{けんちく じゅつ}、建築様式{けんちく ようしき}、構造{こうぞう}《コ》アーキテクチャ、〔ハードウェアまたはソフトウェアの〕基本設計概念{きほん せっけい がいねん}レベル5、発音
\\	、カナアーキテクチャー、分節
\\	変化形 
\\	【他動】~を彫刻{ちょうこく}する、~の像を彫る【名】彫刻{ちょうこく}、彫像{ちょうぞう}、彫塑{ちょうそ}、彫刻作品{ちょうこく さくひん}レベル5、発音
\\	、カナスカルプチャー、変化《動》
\\	、分節
\\	変化形 
\\	《複》
\\	【名】(彫)像、塑像{そぞう}◆【略】
\\	レベル3、発音
\\	、カナスタチュー、変化《複》
\\	、分節
\\	変化形 
\\	《複》
\\	【名】記念建造物{きねん けんぞうぶつ}、モニュメント、遺跡{いせき}、記念碑{きねんひ}、記念物{きねんぶつ}、不朽{ふきゅう}の業績{ぎょうせき}、墓石{はかいし}レベル4、発音
\\	、カナマニュメント、モニュメント、変化《複》
\\	、分節
\\	データのロード中です。お待ちください。 ・
\\	そのページは、すぐにロードされます[読み込みが完了します]。 ・
\\	私はそのリンクを開けません。 ・
\\	私の肩の荷が下りた。	変化形 
\\	【自動】〔船などが〕荷物{にもつ}を積み込む《コ》〔データなどが〕ロードされる、読み込まれる ・
\\	データのロード中です。お待ちください。 ・
\\	そのページは、すぐにロードされます[読み込みが完了します]。 ・
\\	私はそのリンクを開けません。【他動】〔乗客{じょうきゃく}や貨物{かもつ}などを乗り物に〕乗せる、積み込む〔荷物{にもつ}を人や動物{どうぶつ}に〕担がせる、積む〔銃弾{じゅうだん}を銃に〕装填{そうてん}する、込める〔必要{ひつよう}な物を機械{きかい}に〕投入{とうにゅう}する、装着{そうちゃく}する〔物で場所{ばしょ}を〕山積みにする、あふれさせる〔悩みなどで人に〕負担{ふたん}を掛ける、心配{しんぱい}させる《コ》〔主記憶{しゅ きおく}にデータなどを〕ロードする、読み込む《金融》〔保険料{ほけんりょう}に経費{けいひ}などを〕上乗せする《野球》〔塁を〕すべて埋める、満塁{まんるい}にする《電気》〔抵抗{ていこう}などを加えて出力{しゅつりょく}を〕増加{ぞうか}させる〈話〉〔酒などを〕薄める、混ぜ物をする〈話〉〔さいころの〕片側{へんそく}を重くする◆ルーレットで特定の目を出やすくする不正な操作。【名】〔構造物{こうぞうぶつ}が支える〕重さ、荷重{かじゅう}〔運ばれる〕積載貨物{せきさい かもつ}、積み荷〔一度{いちど}に運ぶ〕荷高{にだか}、積載{せきさい}[搭載{とうさい}]人数{にんずう}[貨物量{かもつ りょう}]◆【用法】通例
\\	のように複合語で用いられる。〔機械{きかい}や組織{そしき}などにかかる〕仕事量{しごとりょう}、負荷{ふか}〔精神的{せいしん てき}な〕負担{ふたん}、重荷{おもに} ・
\\	私の肩の荷が下りた。〔機械{きかい}の一度{いちど}の〕処理{しょり}[投入{とうにゅう}]量〔銃弾{じゅうだん}の1回の〕装薬{そうやく}《電気》負荷{ふか}◆【同】
\\	《電気》〔負荷{ふか}を受ける〕電気{でんき}機器{きき}[抵抗{ていこう}]《金融》〔投資信託{とうし しんたく}で最初{さいしょ}に引かれる〕販売手数料{はんばい てすうりょう}◆【同】
\\	《生物》遺伝荷重{いでん かじゅう}◆【同】
\\	《コ》ロード◆データを外部記憶装置から主記憶装置に読み込むこと。
\\	〈話〉たくさん、いっぱい〈俗〉良い眺め、目の保養{ほよう}〈俗〉〔1回分{かいぶん}の〕射精{しゃせい}(の量)【副】
\\	〈話〉非常{ひじょう}に、多量{たりょう}にレベル2、発音
\\	、カナロード、ロウド、変化《動》
\\	変化形 
\\	【他動】~に重い荷物{にもつ}を積む[運ばせる](人)に重い責任{せきにん}を負わせる、(人)に面倒{めんどう}なことを押し付ける【名】〔運ばれる重い〕荷物{にもつ}、積み荷〔精神的{せいしん てき}な〕負担{ふたん}、重荷{おもに}義務{ぎむ}、責任{せきにん}、重責{じゅうせき}〔船の〕積載量{せきさいりょう}、積載力{せきさい りょく}《鉱物》被覆岩{ひふく がん}、表土{ひょうど}《金属》高炉{こうろ}の装入原料{そうにゅう げんりょう}レベル4、発音
\\	、カナバーデン、バードン、変化《動》
\\	、分節
\\	その火災という悲劇で多くの人命が失われた。	変化形 
\\	《複》
\\	【名】悲劇{ひげき}、悲劇的事件{ひげき てき じけん}、惨事{さんじ} ・
\\	その火災という悲劇で多くの人命が失われた。レベル5、発音
\\	、カナトゥラジェディ、トラジェディ、変化《複》
\\	、分節
\\	危機が一時的に回避されている。 ・
\\	起こらなかったことが危機と同じく大事である。 ・
\\	欧州の最新の通貨危機は、ドイツ連邦銀行の決定によって引き起こされた。	変化形 
\\	《複》
\\	【名】危機{きき}、難局{なんきょく}◆何らかの手を打たなければ破局に至るような、極めて不安定な状態。 ・
\\	危機が一時的に回避されている。 ・
\\	起こらなかったことが危機と同じく大事である。 ・
\\	欧州の最新の通貨危機は、ドイツ連邦銀行の決定によって引き起こされた。〔重要{じゅうよう}な変化{へんか}が起きる〕転機{てんき}、岐路{きろ}、重大局面{じゅうだい きょくめん}《医》〔病気{びょうき}の〕峠、転換点{てんかんてん}◆悪化または快方に流れが変わる時点。〔物語中{ものがたり ちゅう}の〕危機的場面{きき てき ばめん}レベル4、発音
\\	、カナクライシス、クライスィーズ、変化《複》
\\	、分節
\\	成功する人というのは、ささいな障害には挫折しない。 ・
\\	発見を妨げる最大の障害は、無知ではなく、知っていると錯覚することである。	変化形 
\\	《複》
\\	【名】障害(物){しょうがい(ぶつ)}、妨害(物){ぼうがい(ぶつ)}、邪魔{じゃま} ・
\\	成功する人というのは、ささいな障害には挫折しない。 ・
\\	発見を妨げる最大の障害は、無知ではなく、知っていると錯覚することである。
\\	支障を来す、障害になる、障壁となるレベル6、発音
\\	、カナアブスタクル、オブスタクル、変化《複》
\\	、分節
\\	変化形 
\\	【自動】〔戦闘{せんとう}などで〕勝利{しょうり}する、征服{せいふく}する勝ち誇る、勝利{しょうり}に歓喜{かんき}する〔古代{こだい}ローマで将軍{しょうぐん}が〕凱旋{がいせん}する【名】勝利{しょうり}、征服{せいふく}勝利{しょうり}の喜び、勝ち誇り偉業{いぎょう}、大成功{だいせいこう}〔古代{こだい}ローマの〕凱旋式{がいせん しき}◆将軍の凱旋を祝い、カピトリウムの丘
\\	までパレードを行ったものを指す。◆【参考】
\\	〈古〉祝賀{しゅくが}パレードレベル4、発音
\\	、カナトゥライアンフ、トライアンフ、変化《動》
\\	、分節
\\	変化形 
\\	《複》
\\	【自動】誇りとする【名】栄光{えいこう}、名誉{めいよ}、誇り称賛{しょうさん}、賛美{さんび}壮観{そうかん}、荘厳{そうごん}後光{ごこう}レベル4、発音
\\	、カナグローリー、変化《動》
\\	、分節
\\	変化形 
\\	《複》
\\	【他動】〔悪い結果{けっか}を~に〕運命{うんめい}づける◆【用法】受け身で使われることが多い。【名】運命{うんめい}の力、物事{ものごと}の定め、神の摂理{せつり}〔避けられない〕運命{うんめい}、宿命{しゅくめい}〔最終的{さいしゅう てき}な〕結末{けつまつ}、結果{けっか}悲運{ひうん}、最期{さいご}、破滅{はめつ}
\\	《ギ・ロ神》運命{うんめい}の三女神{さん めがみ}の一人{ひとり}◆【参考】
\\	レベル4、発音
\\	、カナフェイト、変化《動》
\\	変化形 
\\	《複》
\\	【名】運命{うんめい}、さだめ、宿命{しゅくめい}レベル5、発音
\\	、カナデスティニ、デステニー、変化《複》
\\	、分節
\\	同情してくれなくてもいいよ。 ・
\\	お気の毒に。	変化形 
\\	《複》
\\	【名】共感{きょうかん}、同情{どうじょう} ・
\\	同情してくれなくてもいいよ。 ・
\\	お気の毒に。悔やみ、弔慰◆通例
\\	レベル3、発音
\\	、カナスィンパスィ、シンパシー、変化《複》
\\	、分節
\\	あなたがそれほどまでにひどい状況に耐えなければならないとは残念です。◆この
\\	は驚きなどの感情を表す	変化形 
\\	《複》
\\	【他動】~をかわいそうに思う、気の毒に思う、哀れむ、~に同情{どうじょう}する【名】哀れみ、かわいそうなこと、同情{どうじょう}、残念{ざんねん}なこと ・
\\	あなたがそれほどまでにひどい状況に耐えなければならないとは残念です。◆この
\\	は驚きなどの感情を表すレベル3、発音
\\	、カナピティー、変化《動》
\\	、分節
\\	悲しみは誰もが背負っている。	変化形 
\\	【名】〔人の〕悲しみ、悲哀{ひあい}、後悔{こうかい} ・
\\	悲しみは誰もが背負っている。不幸{ふこう}、苦しみ、悲しみを起こさせるもの悲しみの表出{ひょうしゅつ}[表現{ひょうげん}]レベル4、発音
\\	、カナサロウ、ソロウ、変化《動》
\\	、分節
\\	彼女の顔に悲しみの表情が現れている。	変化形 
\\	《複》
\\	【名】〔人の死などによる深い〕悲しみ、嘆き、苦悩{くのう} ・
\\	彼女の顔に悲しみの表情が現れている。心痛{しんつう}[悲嘆{ひたん}]の原因{げんいん}◆深い悲しみをもたらす状況や出来事を指す。苛立{いらだ}ち、欲求不満{よっきゅう ふまん}〈話〉厄介{やっかい}、面倒{めんどう}〈古〉不平{ふへい}、苦情{くじょう}レベル4、発音
\\	、カナグリーフ、変化《複》
\\	親しき仲にも礼儀あり。/親しい友の間にも礼儀は用いるべきだ。 ・
\\	気は心。 ・
\\	偉大な人物ほど礼儀をわきまえている。 ・
\\	同氏に対しご高配とご支援を賜れますなら、誠にありがたく存じます。 ・
\\	ロバートのこと、何とぞよろしくお願い致します。 ・
\\	ご丁寧にも貴殿より12月にご返事を頂いたことに対して、大変感謝致しております。	変化形 
\\	《複》
\\	【名】礼儀正{れいぎ ただ}しいこと、丁寧{ていねい}、礼儀{れいぎ}、礼儀正しさ、作法{さほう} ・
\\	親しき仲にも礼儀あり。/親しい友の間にも礼儀は用いるべきだ。 ・
\\	気は心。 ・
\\	偉大な人物ほど礼儀をわきまえている。丁重{ていちょう}な行為{こうい}、親切{しんせつ}、親切{しんせつ}な行い、厚意{こうい}、優遇{ゆうぐう}、優待(券){ゆうたい(けん)} ・
\\	同氏に対しご高配とご支援を賜れますなら、誠にありがたく存じます。 ・
\\	ロバートのこと、何とぞよろしくお願い致します。 ・
\\	ご丁寧にも貴殿より12月にご返事を頂いたことに対して、大変感謝致しております。レベル6、発音
\\	、カナカーティシー、カートスィ、変化《複》
\\	、分節
\\	戦争には憎しみ以外のすべての感情が欠けている。 ・
\\	感傷の入り込む隙間{すきま}なんかどこにもない。 ・
\\	そのバースデーカードには星座が描かれ、お祝いの言葉が印刷されていた。	変化形 
\\	《複》
\\	【名】気持ち{きもち}、感情{かんじょう}、心情{しんじょう}、感傷{かんしょう} ・
\\	戦争には憎しみ以外のすべての感情が欠けている。 ・
\\	感傷の入り込む隙間{すきま}なんかどこにもない。感想{かんそう}、意見{いけん}、所感{しょかん}、情趣{じょうしゅ}〔感情を表現する〕決まり文句、格言、〔お祝いなどの〕言葉 ・
\\	そのバースデーカードには星座が描かれ、お祝いの言葉が印刷されていた。レベル5、発音
\\	、カナセンチメント、センティメント、変化《複》
\\	、分節
\\	変化形 
\\	《複》
\\	【他動】~を熱望{ねつぼう}する【名】〔名声{めいせい}や権力{けんりょく}などへの〕野心{やしん}、熱望{ねつぼう}、大志{たいし}野心{やしん}[熱望{ねつぼう}]の的[対象{たいしょう}]〔活動{かつどう}や努力{どりょく}への〕エネルギー、熱意{ねつい}レベル4、発音
\\	、カナアンビション、変化《動》
\\	、分節
\\	変化形 
\\	《複》
\\	【名】熱中{ねっちゅう}、熱狂{ねっきょう}、熱烈{ねつれつ}な興味{きょうみ}[願望{がんぼう}]熱中{ねっちゅう}[熱狂{ねっきょう}]の対象{たいしょう}〈古〉宗教的{しゅうきょう てき}忘我{ぼうが}[狂信{きょうしん}]レベル4、発音
\\	、カナインシュージアズム、インスージアズム、エンシュージアズム、エンスージアズム、エンスューズィアズム、変化《複》
\\	、分節
\\	変化形 
\\	《複》
\\	【名】〔行動{こうどう}のきっかけの〕弾{はず}み、勢い〔突然起きる〕衝動{しょうどう}、強い欲求{よっきゅう}◆予測できない感情や行動を引き起こすもの。衝動的{しょうどう てき}行為{こうい}[行動{こうどう}]〔行動{こうどう}の〕誘発要因{ゆうはつ よういん}、傾向{けいこう}《電気》インパルス《物理》力積{りきせき}◆ある時間と、その間にかかる力の大きさの積。《生物》〔神経細胞{しんけい さいぼう}を伝わる〕インパルス◆【同】
\\	レベル4、発音
\\	、カナインパルス、変化《複》
\\	、分節
\\	変化形 
\\	《複》
\\	【名】〔行動{こうどう}などを起こさせる〕激励{げきれい}、励み〔興味{きょうみ}などを引き起こす〕刺激{しげき}、要因{よういん}〔生体{せいたい}を反応{はんのう}させる物理的{ぶつり てき}〕刺激(剤){しげき(ざい)}レベル6、発音
\\	、カナスティミュラス、変化《複》
\\	、分節
\\	彼にはテニスの素質がある。	変化形 
\\	《複》
\\	【名】本能{ほんのう}、直感{ちょっかん}生まれ持った才能{さいのう}、生来{せいらい}の素質{そしつ} ・
\\	彼にはテニスの素質がある。強い動機{どうき}、衝動{しょうどう}【形】〈文〉満ち満ちている、満ちあふれた〈古〉内なる衝動{しょうどう}に駆られたレベル4、発音《名》
\\	《形》
\\	、カナインスティンクト、変化《複》
\\	、分節
\\	食欲が回復しました。	変化形 
\\	《複》
\\	【名】食欲{しょくよく} ・
\\	食欲が回復しました。欲求{よっきゅう}レベル4、発音
\\	、カナアピタイト、アペタイト、変化《複》
\\	、分節
\\	【名】〔成長{せいちょう}のための〕栄養{えいよう}摂取{せっしゅ}[吸収{きゅうしゅう}]〔学問{がくもん}の〕栄養学{えいようがく}〔食物{しょくもつ}の〕栄養{えいよう}素[もの]レベル6、発音
\\	、カナニュートリション、ヌートリション、分節
\\	変化形 
\\	【自動】強く欲する、欲望{よくぼう}を抱く【他動】~が欲しいと強く思う、~であることを強く願う〈文〉~を要求{ようきゅう}[依頼{いらい}]する(人)に性的欲望{せいてき よくぼう}を抱く、(人)と性的関係{せいてき かんけい}を持ちたいと思う【名】熱望{ねつぼう}、切望{せつぼう}、欲望{よくぼう}〈文〉要求{ようきゅう}、依頼{いらい}〈文〉切望{せつぼう}するもの[人]〔人に対する〕性的欲望{せいてき よくぼう}レベル3、発音
\\	、カナディザイア、デザイア、変化《動》
\\	、分節
\\	徳はそれ自体が報いである。◆ことわざ	変化形 
\\	《複》
\\	【名】美徳{びとく}、徳、美点{びてん}、善、長所{ちょうしょ} ・
\\	徳はそれ自体が報いである。◆ことわざ効き目{ききめ}、効力{こうりょく}〔女性{じょせい}の〕純潔{じゅんけつ}、貞節{ていせつ}
\\	力天使{りきてんし}◆第5位の天使〈古〉剛勇{ごうゆう}、武勇{ぶゆう}レベル4、発音
\\	、カナバーチュー、変化《複》
\\	、分節
\\	正義は勝つ。 ・
\\	「この世に正義はありません」彼は悲しげに言いました。 ・
\\	私は、最終的には正義が勝利してほしいと思います。 ・
\\	不公平じゃないか。◆【直訳】公正さはどこにあるのか。⇒どこにもないではないか。
\\	変化形 
\\	《複》
\\	"【名】正義{せいぎ}、公正{こうせい} ・
\\	正義は勝つ。 ・
\\	「この世に正義はありません」彼は悲しげに言いました。 ・
\\	私は、最終的には正義が勝利してほしいと思います。 ・
\\	不公平じゃないか。◆【直訳】公正さはどこにあるのか。⇒どこにもないではないか。〔主張{しゅちょう}などの〕正当性{せいとう せい}、妥当性{だとう せい}賞罰{しょうばつ}、当然{とうぜん}の報い司法{しほう}、裁判{さいばん}〔上級審{じょうきゅうしん}の〕裁判官{さいばんかん}、判事{はんじ}レベル3、発音
\\	、カナジャスティス、ジャステス、変化《複》
\\	、分節
\\	昨日一晩考えてそれが分かってきた。 ・
\\	ようやく私はそれが分かってきた。 ・
\\	夜明け前が最も暗い。/つらいことの後には必ずいいことがあるものだ。 ・
\\	世界はまさに~の新たなる黄金時代の幕開けを迎えようとしている。	変化形 
\\	【自動】〔夜が〕明ける、〔周囲が〕明るくなる〔真相{しんそう}などが〕分かり始める、見え出す ・
\\	昨日一晩考えてそれが分かってきた。 ・
\\	ようやく私はそれが分かってきた。発達し始める【名】夜明け{よあけ}、暁{あかつき}、曙{あけぼの}、黎明{れいめい}◆【参考】
\\	(たそがれ) ・
\\	夜明け前が最も暗い。/つらいことの後には必ずいいことがあるものだ。幕開け{まくあけ}、(事の)始まり、兆し、発端{ほったん}、出現{しゅつげん} ・
\\	世界はまさに~の新たなる黄金時代の幕開けを迎えようとしている。レベル3、発音
\\	、カナドーン、変化《動》
\\	変化形 
\\	《複》
\\	【名】地平線{ちへいせん}、水平線{すいへいせん}◆【同】
\\	《天文》地理{ちり}地平{ちへい}[水平{すいへい}]◆【同】
\\	《天文》天文{てんもん}地平{ちへい}[水平{すいへい}]◆【同】
\\	〔人の知識{ちしき}や経験{けいけん}などの〕視野{しや}、領域{りょういき}、範囲{はんい}《地学》層位{そうい}◆明らかに他の層と区別される地層の並び。《地学》層準{そうじゅん}◆含まれる化石などによって年代が特定される地層。レベル3、発音
\\	、カナホライズン、ホライゾン、変化《複》
\\	、分節
\\	シグナルは人工衛星に届き、そこからケーブル会社が受け取る。	変化形 
\\	《複》
\\	【自他動】衛星中継{えいせい ちゅうけい}する【名】《天文》衛星{えいせい}人工衛星{じんこう えいせい}、通信衛星{つうしん えいせい}、サテライト◆【略】
\\	◆話し言葉では
\\	と略されることがある。 ・
\\	シグナルは人工衛星に届き、そこからケーブル会社が受け取る。〈話〉衛星放送{えいせい ほうそう}テレビ◆【同】
\\	衛星国{えいせいこく}◆政治・経済的に強国に依存している国。〔郊外{こうがい}の〕衛星都市{えいせい とし}サテライト教室{きょうしつ}[キャンパス・オフィス]〔権力者{けんりょくしゃ}の〕手下{てした}、子分{こぶん}ごますり、おべっか使い《生物》〔染色質{せんしょくしつ}の〕付随体{ふずいたい}【形】衛星{えいせい}の[を利用{りよう}した]レベル5、発音
\\	、カナサテライト、変化《動》
\\	、分節
\\	変化形 
\\	《複》
\\	【名】地球{ちきゅう}、球体{きゅうたい}、球、世界{せかい}、地球儀{ちきゅうぎ}、天球{てんきゅう}
\\	〈米俗〉乳房{にゅうぼう}レベル4、発音
\\	、カナグローブ、グロウブ、変化《複》
\\	変化形 
\\	《複》
\\	【他動】~を球状{きゅうじょう}[球形{きゅうけい}]にする〈文〉~を取り囲む、~を取り巻く【名】《数学》球、球体{きゅうたい}◆一点から一定の距離にある3次元の面、またはその面が作る立体図形。〔一般{いっぱん}に〕球形{きゅうけい}[球状{きゅうじょう}]の物〔知識{ちしき}や興味{きょうみ}の〕分野{ぶんや}、領域{りょういき}〔勢力{せいりょく}や影響{えいきょう}が及ぶ〕範囲{はんい}、領分{りょうぶん}〔社会{しゃかい}の中の〕集団{しゅうだん}、身分{みぶん}、階級{かいきゅう}〈文〉天体{てんたい}、星〈文〉空、天《天文》〔天動説{てんどう せつ}の〕天球{てんきゅう}レベル5、発音
\\	、カナスフィアー、スフェア、変化《複》
\\	変化形 
\\	《複》
\\	【1他動】~にポール[棒・さお]を付ける【1名】ポール、棒(状のもの)、柱、さお、マスト《長さ単位》ポール
\\	0292メートル。
\\	《面積単位》ポール
\\	293平方メートル。土地などの面積に用いる。〔スキーの〕ストック〔野球の〕バット〈性俗〉男性器{だんせいき}【2名】極レベル2、発音
\\	、カナポール、変化《複》
\\	変化形 
\\	《複》
\\	【名】温度計{おんどけい}、体温計{たいおんけい}、寒暖計{かんだんけい}、検温器{けんおんき}レベル5、発音
\\	、カナサーモメーター、サモミター、サモメター、サマミタ、変化《複》
\\	、分節
\\	変化形 
\\	《複》
\\	【名】温度{おんど}、気温{きおん}体温{たいおん}◆【略】
\\	レベル3、発音
\\	、カナテンペラチャー、テンペラチュア、変化《複》
\\	、分節
\\	変化形 
\\	【自動】氾濫{はんらん}する、殺到{さっとう}する【他動】~を氾濫{はんらん}させる、あふれさせる、水浸{みずびた}しにする、~が…にあふれる、~にみなぎる《イ》大量{たいりょう}の情報{じょうほう}[リクエスト]を送りつける【名】洪水{こうずい}、浸水{しんすい}、大水{おおみず}、冠水{かんすい}氾濫{はんらん}、横溢{おういつ}、おびただしい流入{りゅうにゅう}、殺到{さっとう}
\\	〔旧約聖書{きゅうやく せいしょ}の〕ノアの洪水{こうずい}レベル2、発音
\\	、カナフラッド、変化《動》
\\	変化形 
\\	《複》
\\	【名】昆虫{こんちゅう}、虫レベル2、発音
\\	、カナインセクト、変化《複》
\\	、分節
\\	砂地は水も無機栄養素もうまく保持することができない。 ・
\\	この本で使われている「白衣についた黒い印」は悪の温床を象徴している。◆文学的表現 ・
\\	その会社の評判はスキャンダルによって傷ついた。
\\	変化形 
\\	"【1名】地殻{ちかく}の表層土{ひょうそう ど}〔種類{しゅるい}を表す〕土、土壌{どじょう} ・
\\	砂地は水も無機栄養素もうまく保持することができない。国、国土{こくど}◆【参考】
\\	農業{のうぎょう}、農家{のうか}の生活{せいかつ}◆文学的表現〔成長{せいちょう}や発達{はったつ}を促す〕土壌{どじょう}、温床{おんしょう} ・
\\	この本で使われている「白衣についた黒い印」は悪の温床を象徴している。◆文学的表現【2他動】〔~の表面{ひょうめん}を〕汚す、汚くする〔名声{めいせい}や評判{ひょうばん}などを〕おとしめる、傷つける ・
\\	その会社の評判はスキャンダルによって傷ついた。〔便で〕~を汚すレベル3、発音
\\	、カナソイル、変化《動》
\\	変化形 
\\	《複》
\\	【他動】~を粒状{りゅうじょう}にする、粒にする【名】〔砂・塩などの〕粒子{りゅうし}(の大きさ)、一粒{ひとつぶ}〔集合的に〕穀物{こくもつ}、穀類{こくるい}少量{しょうりょう}布目{ぬのめ}木目{もくめ}、木理{もくり}《ゴルフ》芝目{しばめ}〔人の〕性質、気性グレイン、グレーン◆体毛のみを取り除いた状態の動物の皮〔皮革の〕しぼグレイン、宝石{ほうせき}グレイン、ゲレーン◆真珠・ダイヤモンドなどの宝石の重量単位。1グレイン=50ミリグラム=1/4カラット。1カラット=0.2グラム《重量・常衡単位》グレイン、常衡{じょうこう}グレイン◆1グレイン=64.79891ミリグラム、1(常衡)オンス=480(常衡)グレインレベル4、発音
\\	、カナグレイン、変化《動》
\\	変化形 
\\	【自動】〔家畜{かちく}などが〕草を食べる〔作物{さくもつ}などが〕収穫{しゅうかく}できる〔人が〕耕作{こうさく}する、栽培{さいばい}する〔岩盤{がんばん}などが〕露出{ろしゅつ}する〔問題{もんだい}などが〕露見{ろけん}する【他動】〔髪や芝などを〕刈る、切って短くする〔家畜{かちく}などが草の上部{じょうぶ}を〕食べる〔家畜{かちく}の耳の上を〕切る◆家畜を識別するために行う。〈英〉〔花や作物{さくもつ}を〕摘む、収穫{しゅうかく}する〔畑などを〕耕作{こうさく}する、作付けする〔作物{さくもつ}を〕栽培{さいばい}する、作付けする〔写真{しゃしん}の不要{ふよう}な部分{ぶぶん}を〕切り取る、トリミングする【名】作物{さくもつ}、農作物{のうさくぶつ}収穫高{しゅうかく だか}、産出高{さんしゅつだか}群れ、一団{いちだん}短髪{たんぱつ}、短く刈った頭、五分刈りの髪、ベリーショート〔髪型〕〔家畜{かちく}の〕耳印{みみじるし}短い乗馬むち、乗馬用短鞭{じょうば よう たんべん}、むちの柄《動物》素嚢{そのう}レベル3、発音
\\	、カナクラップ、クロップ、変化《動》
\\	変化形 
\\	【自動】収穫{しゅうかく}する【他動】〔作物などを〕収穫{しゅうかく}する、刈り取る、取り入れる〔動物や魚などを〕捕獲{ほかく}する《医》〔臓器などを〕摘出{てきしゅつ}する、〔細胞などを〕採取{さいしゅ}する【名】〔作物などの〕収穫(期){しゅうかく(き)}、収穫高{しゅうかく だか}〔動物や魚などの〕捕獲{ほかく}《生化学》集菌{しゅうきん}〈比喩〉〔行為の〕結果{けっか}、報い《医》〔臓器などの〕摘出{てきしゅつ}、〔細胞などの〕採取{さいしゅ}レベル4、発音
\\	、カナハーベスト、変化《動》
\\	、分節
\\	【名】農業{のうぎょう}、農学{のうがく}、農耕{のうこう}レベル4、発音
\\	、カナアグリカルチャー、分節
\\	宗教は皆それぞれに固有の儀式を持っている。	変化形 
\\	《複》
\\	【名】〔ある宗教{しゅうきょう}の〕儀式{ぎしき}、典礼{てんれい}◆特定の宗教の儀式を指すことが多い。 ・
\\	宗教は皆それぞれに固有の儀式を持っている。《宗教》典礼書{てんれい しょ}、ミサ式次第{しきしだい}〔決められた〕式次第{しきしだい}、式典{しきてん}の形式{けいしき}〈話〉〔いつもの〕決まったやり方◆儀式のように決まって行う日常の行為などを指す。《心理学》儀式{ぎしき}◆ある行為を病的に繰り返すこと。強迫性障害者に見られる。【形】儀式{ぎしき}の、祭礼{さいれい}の、儀礼{ぎれい}のレベル6、発音
\\	、カナリチュアル、変化《複》
\\	、分節
\\	この地域でのいくつかの激しい戦闘により、この任務の困難さはますます強まっている。 ・
\\	われわれの使命[目標]はこの町を住みよい町にすることだ。	変化形 
\\	《複》
\\	【他動】(人)を派遣{はけん}する~に伝道{でんどう}[布教{ふきょう}]を行う【名】〔海外{かいがい}での交渉{こうしょう}を担う〕派遣団{はけんだん}、交渉団{こうしょう だん}〔海外交渉団{かいがい こうしょう だん}の〕任務{にんむ}、使命{しめい} ・
\\	この地域でのいくつかの激しい戦闘により、この任務の困難さはますます強まっている。〔海外派遣{かいがい はけん}される〕政府高官{せいふ こうかん}、特命大使{とくめい たいし}〔人がそのすべてをささげる〕使命{しめい}、天職{てんしょく}、目的{もくてき}、目標{もくひょう} ・
\\	われわれの使命[目標]はこの町を住みよい町にすることだ。〔特定{とくてい}の目的{もくてき}を果たすための〕宇宙飛行{うちゅう ひこう}、ミッション《軍事》〔特定{とくてい}の目的{もくてき}を果たすための〕作戦{さくせん}、ミッション《キリスト教》〔海外{かいがい}などへの〕布教団{ふきょう だん}、伝道団{でんどうだん}◆布教、医療、社会福祉活動を行う。《キリスト教》布教{ふきょう}[伝道{でんどう}]活動{かつどう}《キリスト教》〔布教団{ふきょう だん}に割り当てられた〕布教{ふきょう}[伝道{でんどう}]地域{ちいき}《キリスト教》布教{ふきょう}[伝道{でんどう}]本部{ほんぶ}《キリスト教》〔地域{ちいき}の〕貧者救済本部{ひんじゃ きゅうさい ほんぶ}◆【同】
\\	《キリスト教》〔常駐{じょうちゅう}する聖職者{せいしょくしゃ}のいない〕小教会{しょう きょうかい}【形】伝道{でんどう}[布教{ふきょう}]団の[による]、伝道{でんどう}[布教{ふきょう}]を行うための◆【用法】限定的レベル5、発音
\\	、カナミッション、変化《複》
\\	、分節
\\	変化形 
\\	《複》
\\	【名】信頼{しんらい}、信用{しんよう}、信じること自信{じしん}、信念{しんねん}、確信{かくしん}信仰{しんこう}、信条{しんじょう}
\\	教義、~教レベル3、発音
\\	、カナフェイス、変化《複》
\\	彼女は、家庭に閉じ込められ自分を犠牲にするキャリアウーマンを象徴している。 ・
\\	死んだつもりになって頑張れば何だってできるさ。	変化形 
\\	【自動】犠牲{ぎせい}になる《野球》犠打{ぎだ}を打つ〔チェスや将棋などで戦略的に〕捨て駒にする【他動】〔神に〕~をささげる、~をいけにえにする~を犠牲{ぎせい}にする ・
\\	彼女は、家庭に閉じ込められ自分を犠牲にするキャリアウーマンを象徴している。 ・
\\	死んだつもりになって頑張れば何だってできるさ。〔貴重な〕~を断念する〔チェスや将棋などで〕~を捨て駒にする~を投げ売りにする【名】〔神へ〕ささげものをすること、〔神にささげられた〕いけにえ、人身御供{ひとみ ごくう}〔自分以外{じぶん いがい}のために貴重{きちょう}なものを〕犠牲{ぎせい}にすること、〔払われた〕犠牲{ぎせい}〔本来{ほんらい}より低い価値{かち}で〕放棄{ほうき}[断念{だんねん}]すること[したもの]《野球》犠打{ぎだ}◆【同】
\\	〔チェスや将棋{しょうぎ}などの〕捨て駒投げ売り(品)レベル4、発音
\\	、カナサクリファイス、変化《動》
\\	、分節
\\	変化形 
\\	【他動】~を大切{たいせつ}[大事{だいじ}]にする[保管{ほかん}する・保存{ほぞん}する]〔将来のために〕取っておく、ためておく心に銘記{めいき}する、心にしまっておく【名】宝物{たからもの}、財宝{ざいほう}◆金や宝石などを集めたもの。貴重品{きちょう ひん}、大切{たいせつ}な品いとおしい[大切{たいせつ}な]人レベル3、発音
\\	、カナトゥレジャ、トレジャー、変化《動》
\\	、分節
\\	【名】〔人が生まれながらにして持つ〕地位{ちい}、立場{たちば}、品性{ひんせい}〔相続人{そうぞくにん}が受け継ぐ〕相続財産{そうぞく ざいさん}、遺産{いさん}〔受け継がれる文化{ぶんか}・歴史的{れきし てき}〕遺産{いさん}、伝統{でんとう}レベル6、発音
\\	、カナヘリテージ、ヘリテッジ、変化《複》
\\	、分節
\end{CJK}
\end{document}