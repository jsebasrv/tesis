\documentclass[8pt]{extreport} 
\usepackage{hyperref}
\usepackage{CJKutf8}
\begin{document}
\begin{CJK}{UTF8}{min}
\\	三が日
\\	あけましておめでとうございます。
\\	麺
\\	項目
\\	注文する
\\	オーダー用紙
\\	集中する
\\	のれん
\\	両脇
\\	食券
\\	かため
\\	今日は、ラーメンに詳しい、岩上さんと一緒に、最近話題になっているユニークなラーメン屋さんにいってみようと思います。	
\\	今日は、ラーメンに詳しい、岩上さんと一緒に、最近話題になっているユニークなラーメン屋さんにいってみようと思います。 
\\	岩上さん、よろしくお願いします。	
\\	岩上さん、よろしくお願いします。 
\\	今回紹介したいラーメン店はこちらの「一番
\\	(いちばんぼし)です。	
\\	今回紹介したいラーメン店はこちらの「一番
\\	(いちばんぼし)です。 
\\	では、早速入ってみましょう。	
\\	では、早速入ってみましょう。 
\\	(ガラガラガラ)まず最初に、この機械にお金を入れて、食券を買います。	
\\	(ガラガラガラ)まず最初に、この機械にお金を入れて、食券を買います。 
\\	おや?カウンターのそれぞれ席の両脇に仕切りと、前にのれんがありますねー。	
\\	おや?カウンターのそれぞれ席の両脇に仕切りと、前にのれんがありますねー。 
\\	そうなんですよ。	
\\	そうなんですよ。 
\\	この仕切りは、お客さんにラーメンの味に集中してもらうために作ってあります。	
\\	この仕切りは、お客さんにラーメンの味に集中してもらうために作ってあります。 
\\	このカウンター席に座ると、両脇も前も見えませんからね。	
\\	このカウンター席に座ると、両脇も前も見えませんからね。 
\\	なるほどー。	
\\	なるほどー。 
\\	あと、このシステムがあるため、女性のお客さんも多いんです。	
\\	あと、このシステムがあるため、女性のお客さんも多いんです。 
\\	ああ、一人ではラーメン屋さんに入りにくい・・・と感じる女性は多いですからね。	
\\	ああ、一人ではラーメン屋さんに入りにくい・・・と感じる女性は多いですからね。 
\\	そのとおりです。	
\\	そのとおりです。 
\\	それから、このお店のラーメンのスープはトンコツ味なんですが・・・はい、オーダー用紙と鉛筆。	
\\	それから、このお店のラーメンのスープはトンコツ味なんですが・・・はい、オーダー用紙と鉛筆。 
\\	ん?このオーダー用紙に自分で書いて、注文するんですか?	
\\	ん?このオーダー用紙に自分で書いて、注文するんですか? 
\\	そうです。自分の食べたい味になるように、味を選ぶんです。	
\\	そうです。自分の食べたい味になるように、味を選ぶんです。 
\\	オーダー用紙に書いてある項目にマルをしてください。	
\\	オーダー用紙に書いてある項目にマルをしてください。 
\\	へー。じゃ、わたしは・・・・スープは...こってり。(マルをする音)麺はかため・・・	
\\	へー。じゃ、わたしは・・・・スープは...こってり。(マルをする音)麺はかため・・・ 
\\	伺う
\\	恨み
\\	怪しい
\\	恐ろしい
\\	過去
\\	推理小説家
\\	宝石店
\\	店主
\\	握る
\\	みなさん、こんばんは。
\\	マキコです。	
\\	みなさん、こんばんは。
\\	マキコです。 
\\	今日は、推理小説家、東五郎先生に先生の新作「ダイングメッセージ」についてお話を伺いたいと思います。	
\\	"今日は、推理小説家、東五郎先生に先生の新作「ダイングメッセージ」についてお話を伺いたいと思います。 
\\	先生よろしくお願いします。	
\\	先生よろしくお願いします。 
\\	はい。この小説は、「ダイングメッセージ」で犯人がわかったところから、物語が始まります。	
\\	"はい。この小説は、「ダイングメッセージ」で犯人がわかったところから、物語が始まります。 
\\	「ある宝石店の店主が殺された。当然、鍵は閉めてあった。	
\\	"「ある宝石店の店主が殺された。当然、鍵は閉めてあった。 
\\	店の床には、ダイアモンドが何個も落ちていた。	
\\	店の床には、ダイアモンドが何個も落ちていた。 
\\	店主はダイアモンドをひとつ握っていたが、お店の宝石や現金は一つも盗まれていなかった。	
\\	店主はダイアモンドをひとつ握っていたが、お店の宝石や現金は一つも盗まれていなかった。 
\\	警察は店主に恨みを持った者が犯人であると考え、野田節子、中村幸子、斉藤順子の3人が怪しいと考えた。	
\\	警察は店主に恨みを持った者が犯人であると考え、野田節子、中村幸子、斉藤順子の3人が怪しいと考えた。 
\\	一人の刑事が、容疑者の誕生日を調べた。	
\\	一人の刑事が、容疑者の誕生日を調べた。 
\\	野田は3月生まれ、中村は4月生まれで、斉藤は12月生まれである。」	
\\	"野田は3月生まれ、中村は4月生まれで、斉藤は12月生まれである。」 
\\	おおお、簡単に犯人が分かりましたね。私にも犯人がだれか分かります。	
\\	おおお、簡単に犯人が分かりましたね。私にも犯人がだれか分かります。 
\\	そして、そこから、犯人の悲しい人生、殺された店主の恐ろしい過去がわかっていく・・・という物語です。	
\\	そして、そこから、犯人の悲しい人生、殺された店主の恐ろしい過去がわかっていく・・・という物語です。 
\\	では、ここでコマーシャルが入ります。	
\\	では、ここでコマーシャルが入ります。 
\\	歴史
\\	ファイルする
\\	プラモデル
\\	年代順
\\	訪問する
\\	戦国武将
\\	眺める
\\	資料
\\	本棚
\\	城
\\	みなさんは、「歴女」という言葉を聞いた事はありますか?	
\\	"みなさんは、「歴女」という言葉を聞いた事はありますか? 
\\	「歴史が好きな女の人」を短くして、「歴女」ですが、今回は、そんな「歴女」さん、山田花さんのお家を訪問したいと思います。	
\\	"「歴史が好きな女の人」を短くして、「歴女」ですが、今回は、そんな「歴女」さん、山田花さんのお家を訪問したいと思います。 
\\	はーい。(ガチャ)	
\\	はーい。(ガチャ) 
\\	こんにちは、よろしくお願いします。	
\\	こんにちは、よろしくお願いします。 
\\	こちらこそ。	
\\	こちらこそ。 
\\	お邪魔します。	
\\	お邪魔します。 
\\	おっと、ものすごい数のプラモデルが飾ってありますね。	
\\	おっと、ものすごい数のプラモデルが飾ってありますね。 
\\	はい。これは、戦国武将のプラモデルなんです。	
\\	はい。これは、戦国武将のプラモデルなんです。 
\\	おおお、本棚は、さすがに歴史関係の本ばかりですね。	
\\	おおお、本棚は、さすがに歴史関係の本ばかりですね。 
\\	これは年代順に並べてあるんですね?	
\\	はい。これが、私の歴史ファイルです。	
\\	はい。これが、私の歴史ファイルです。 
\\	集めた資料をファイルしておくんです。	
\\	集めた資料をファイルしておくんです。 
\\	あと、休日に訪れたお城の写真も入れておきますね。	
\\	あと、休日に訪れたお城の写真も入れておきますね。 
\\	これを眺めるのが好きなんですよー。	
\\	これを眺めるのが好きなんですよー。 
\\	すごいですねー。ところで、どうして歴史が好きになったんですか?	
\\	すごいですねー。ところで、どうして歴史が好きになったんですか? 
\\	うーん。前に付き合っていた彼氏が、歴史のゲームをしているのを見て、面白そうだな・・・と思ったのが、きっかけです。	
\\	うーん。前に付き合っていた彼氏が、歴史のゲームをしているのを見て、面白そうだな・・・と思ったのが、きっかけです。 
\\	なるほど。	
\\	なるほど。 
\\	気付いたら、彼氏よりも歴史の方が好きになっていましたが・・・ははは・・・。	
\\	気付いたら、彼氏よりも歴史の方が好きになっていましたが・・・ははは・・・。 
\\	予報
\\	今週一杯
\\	変更する
\\	晴れ男
\\	のんびりする
\\	照る照る坊主
\\	鬼に金棒
\\	雨女
\\	今日は、午後から雨の予報ですので、傘を忘れないでください。	
\\	今日は、午後から雨の予報ですので、傘を忘れないでください。 
\\	また、今週一杯は、雨が続くようです。	
\\	また、今週一杯は、雨が続くようです。 
\\	えー、雨が降るのー?	
\\	えー、雨が降るのー? 
\\	なーんだ。公園でお弁当食べたかったのにー。	
\\	なーんだ。公園でお弁当食べたかったのにー。 
\\	そうね。じゃあ、プランを変更する?	
\\	そうね。じゃあ、プランを変更する? 
\\	いや、大丈夫。公園に行けば、晴れるから。	
\\	いや、大丈夫。公園に行けば、晴れるから。 
\\	本当!?	
\\	本当!? 
\\	そうだ。お父さんは、晴れ男なんだよ。	
\\	そうだ。お父さんは、晴れ男なんだよ。 
\\	晴れ男?	
\\	晴れ男? 
\\	お父さんが、大事なイベントで、外に出かけるときは、雨が降らないんだよ。	
\\	お父さんが、大事なイベントで、外に出かけるときは、雨が降らないんだよ。 
\\	そういう人を「晴れ男」「晴れ女」って言うんだ。	
\\	そういう人を「晴れ男」「晴れ女」って言うんだ。 
\\	すごーい!!	
\\	すごーい!! 
\\	そう言っても・・・外を見てよ。雨が降りそう。	
\\	そう言っても・・・外を見てよ。雨が降りそう。 
\\	そんなことを言っているから、雨が降っちゃうんだよ。	
\\	そんなことを言っているから、雨が降っちゃうんだよ。 
\\	予報も言っているし。今日は、家でのんびりしない?	
\\	予報も言っているし。今日は、家でのんびりしない? 
\\	えー、お外行きたい。	
\\	えー、お外行きたい。 
\\	昨日、照る照る坊主つくったの。だから、雨降らないよ。	
\\	昨日、照る照る坊主つくったの。だから、雨降らないよ。 
\\	晴れ男と照る照る坊主で、鬼に金棒だろう?	
\\	晴れ男と照る照る坊主で、鬼に金棒だろう? 
\\	うーん。	
\\	うーん。 
\\	どうして、そんなに嫌なんだよ?	
\\	どうして、そんなに嫌なんだよ? 
\\	実は、私、雨女なのよ。	
\\	実は、私、雨女なのよ。 
\\	そうか、雨女だったのか・・・。それは、困ったな。	
\\	そうか、雨女だったのか・・・。それは、困ったな。 
\\	もー、お外行こうよぉー。	
\\	もー、お外行こうよぉー。 
\\	左利き
\\	無理矢理
\\	脳科学者
\\	無理
\\	相談する
\\	利き手
\\	発達
\\	芸術家
\\	たまる
\\	みなさん、こんばんは。
\\	マキコです。	
\\	みなさん、こんばんは。
\\	マキコです。 
\\	今日のゲストは、脳科学者の酒井教授です。先生、よろしくお願いします。	
\\	今日のゲストは、脳科学者の酒井教授です。先生、よろしくお願いします。 
\\	よろしくお願いします。	
\\	よろしくお願いします。 
\\	では、リスナーからの質問です。	
\\	では、リスナーからの質問です。 
\\	「こんにちは。私には小学校一年生の息子がいます。	
\\	"「こんにちは。私には小学校一年生の息子がいます。 
\\	息子は、左手で字を書いたり、箸を持ったりします。」	
\\	"息子は、左手で字を書いたり、箸を持ったりします。」 
\\	おお、左利きの息子さんですね。いいじゃないですか。	
\\	おお、左利きの息子さんですね。いいじゃないですか。 
\\	「最近、息子が、学校で先生に「右手を使いなさい」と、言われたようです。	
\\	「最近、息子が、学校で先生に「右手を使いなさい」と、言われたようです。 
\\	息子は右手に変えたくないらしいのですが、どうしたらいいでしょうか。」	
\\	"息子は右手に変えたくないらしいのですが、どうしたらいいでしょうか。」 
\\	なるほど・・・。	
\\	なるほど・・・。 
\\	先生、これはどうなんでしょう?	
\\	先生、これはどうなんでしょう? 
\\	直す必要はありません。利き手を無理に変えるのは、脳の発達によくないです。	
\\	"直す必要はありません。利き手を無理に変えるのは、脳の発達によくないです。 
\\	脳の発達・・・。	
\\	脳の発達・・・。 
\\	あ、そういえば、芸術家は左利きが多いらしいですね。	
\\	あ、そういえば、芸術家は左利きが多いらしいですね。 
\\	そうですね。ダビンチやミケランジェロも左利きだったようです。	
\\	そうですね。ダビンチやミケランジェロも左利きだったようです。 
\\	また、利き手を変えるのは、大変なストレスがたまるようです。	
\\	え、そうなんですか。	
\\	え、そうなんですか。 
\\	ですので、息子さんが左利きのままで良いように、学校と相談するのが良いと思います。	
\\	ですので、息子さんが左利きのままで良いように、学校と相談するのが良いと思います。 
\\	そうですね。それが一番良さそうですね。	
\\	そうですね。それが一番良さそうですね。 
\\	先生、今日はありがとうございました。	
\\	先生、今日はありがとうございました。 
\\	いえいえ。	
\\	いえいえ。 
\\	それでは。	
\\	それでは。 
\\	面接
\\	店長
\\	履歴書
\\	時給
\\	働き始める
\\	工事中
\\	ギリギリ
\\	受ける
\\	うーん。もうちょっと・・・。	
\\	うーん。もうちょっと・・・。 
\\	ヤベ!今日はカフェのバイトの面接だった!	
\\	ヤベ!今日はカフェのバイトの面接だった! 
\\	何だよー。この道、工事中?ふざけんなよ・・・。	
\\	何だよー。この道、工事中?ふざけんなよ・・・。 
\\	あ!ここを左に曲がって行けばお店に行けるんだった!	
\\	あ!ここを左に曲がって行けばお店に行けるんだった! 
\\	面接は1時からだったよな。今50分だから、ギリギリ間に合うな。	
\\	面接は1時からだったよな。今50分だから、ギリギリ間に合うな。 
\\	あの、すみません。岡田と申します。面接を受けに来たのですが・・・。	
\\	あの、すみません。岡田と申します。面接を受けに来たのですが・・・。 
\\	あ、面接の方ですね。じゃ、こちらへどうぞ。	
\\	あ、面接の方ですね。じゃ、こちらへどうぞ。 
\\	はい。	
\\	はい。 
\\	どうぞ、座ってください。	
\\	どうぞ、座ってください。 
\\	私は、店長の片岡です。よろしく。	
\\	私は、店長の片岡です。よろしく。 
\\	よろしくお願いします。	
\\	よろしくお願いします。 
\\	ええっと・・・岡田健二君だったね。	
\\	ええっと・・・岡田健二君だったね。 
\\	あ、はい。そうです。	
\\	あ、はい。そうです。 
\\	この前送ってくれた、履歴書を見たんだけど、前にも、レストランで働いていた事があるんだったね。	
\\	この前送ってくれた、履歴書を見たんだけど、前にも、レストランで働いていた事があるんだったね。 
\\	はい。ウェイターをしてました。あと、飲み物やデザートを作ったりもしていました。	
\\	はい。ウェイターをしてました。あと、飲み物やデザートを作ったりもしていました。 
\\	ふーん。岡田君は大学生だったね。	
\\	ふーん。岡田君は大学生だったね。 
\\	そうです。	
\\	そうです。 
\\	夜遅くても大丈夫?	
\\	夜遅くても大丈夫? 
\\	夜は、12時まででしたよね。	
\\	夜は、12時まででしたよね。 
\\	店は、12時までだけど、スタッフは1時までだね。	
\\	店は、12時までだけど、スタッフは1時までだね。 
\\	何で
\\	冬休み
\\	ピアノ
\\	終わる
\\	近所
\\	弾く
\\	通う
\\	高校生
\\	高校
\\	起こす
\\	やばい
\\	(ピピピピー。)	
\\	(ピピピピー。) 
\\	うーん。え?8時?(バタバタバタバタ)	
\\	うーん。え?8時?(バタバタバタバタ) 
\\	もう!お母さん、何で起こしてくれなかったの?	
\\	もう!お母さん、何で起こしてくれなかったの? 
\\	何回も起こしたわよ。	
\\	何回も起こしたわよ。 
\\	行ってきます!	
\\	行ってきます! 
\\	私の名前は野沢菜味。田畑高校に通っている高校二年生。	
\\	私の名前は野沢菜味。田畑高校に通っている高校二年生。 
\\	おっす、ノザワナ。	
\\	おっす、ノザワナ。 
\\	やめてよ。「野沢」か、「菜味」って呼んでって何度も言っているでしょう。	
\\	"やめてよ。「野沢」か、「菜味」って呼んでって何度も言っているでしょう。 
\\	急げ、遅刻するぞ。	
\\	急げ、遅刻するぞ。 
\\	彼はうちの近所に住んでいる那須実。同じ田畑高校二年生。	
\\	彼はうちの近所に住んでいる那須実。同じ田畑高校二年生。 
\\	口も頭も悪いけど、実の弾くピアノはすごい。	
\\	口も頭も悪いけど、実の弾くピアノはすごい。 
\\	実のお母さんは、ピアノの先生で家にピアノが三台もある。	
\\	実のお母さんは、ピアノの先生で家にピアノが三台もある。 
\\	ねぇ、実、冬休みの前に出た宿題、終わった?	
\\	ねぇ、実、冬休みの前に出た宿題、終わった? 
\\	へ?	
\\	へ? 
\\	本を十冊読んで、レポートを百枚書く宿題。	
\\	本を十冊読んで、レポートを百枚書く宿題。 
\\	ハワイ
\\	用紙
\\	問題
\\	お年玉
\\	座る
\\	似合う
\\	開ける
\\	ブレスレット
\\	気に入る
\\	分かる
\\	全部
\\	あけましておめでとう。	
\\	あけましておめでとう。 
\\	今年もよろしくね、桜子。ハワイはどうだった。	
\\	今年もよろしくね、桜子。ハワイはどうだった。 
\\	大葉さん、ハワイ 行ったの?	
\\	大葉さん、ハワイ 行ったの? 
\\	うん。五泊七日で 行ってきた。おばあちゃんが 住んでいるんだ。	
\\	うん。五泊七日で 行ってきた。おばあちゃんが 住んでいるんだ。 
\\	はい、これお土産。気に入るかどうか わからないけど。	
\\	はい、これお土産。気に入るかどうか わからないけど。 
\\	ありがとう!開けてもいい?	
\\	ありがとう!開けてもいい? 
\\	うん。	
\\	うん。 
\\	ブレスレット!かわいい!二本も?いいの?	
\\	ブレスレット!かわいい!二本も?いいの? 
\\	似合うかどうか・・・。	
\\	似合うかどうか・・・。 
\\	うるさい。	
\\	うるさい。 
\\	大葉さん、俺のお土産は?	
\\	大葉さん、俺のお土産は? 
\\	あるよ。甘いものが好きかどうかわからなかったけど、チョコ買ったんだ。	
\\	あるよ。甘いものが好きかどうかわからなかったけど、チョコ買ったんだ。 
\\	サンキュー。	
\\	サンキュー。 
\\	はい、座って。あけましておめでとう。	
\\	はい、座って。あけましておめでとう。 
\\	先生からお年玉があります。	
\\	先生からお年玉があります。 
\\	みんなが休み中に勉強したかどうか知りたいので今からテストをします。	
\\	みんなが休み中に勉強したかどうか知りたいので今からテストをします。 
\\	問題用紙は全部で五枚。テスト時間は30分。	
\\	問題用紙は全部で五枚。テスト時間は30分。 
\\	遊ぶ
\\	多分
\\	謝る
\\	忘れる
\\	話
\\	匹
\\	増える
\\	家族
\\	遊びに行く
\\	なんだ
\\	ねぇねぇ、今から、家に 遊びに 来ない?家族が 一匹 増えたんだ。	
\\	ねぇねぇ、今から、家に 遊びに 来ない?家族が 一匹 増えたんだ。 
\\	家族が 一匹?	
\\	家族が 一匹? 
\\	そう。その家族は これ。ジャーン!この写真が なにか わかる? 
\\	ワンちゃんだ!ねぇ、実も 桜子の 家に 遊びに行こうよ。	
\\	ワンちゃんだ!ねぇ、実も 桜子の 家に 遊びに行こうよ。 
\\	う…ん。行きたいんだけど…。ちょっと…今 何時か わかる?	
\\	う…ん。行きたいんだけど…。ちょっと…今 何時か わかる? 
\\	4時半だよ。	
\\	4時半だよ。 
\\	やべ。ちょっと 先生に 話が あるんだ。あとで大葉さんの 家に 行くよ。	
\\	やべ。ちょっと 先生に 話が あるんだ。あとで大葉さんの 家に 行くよ。 
\\	家が どこか わかる?学校の 前の 白い家だよ。	
\\	家が どこか わかる?学校の 前の 白い家だよ。 
\\	ああ、あの家ね。わかった。	
\\	ああ、あの家ね。わかった。 
\\	待ってるね。	
\\	待ってるね。 
\\	…あ、栗林君。	
\\	…あ、栗林君。 
\\	野沢さん、実、どこに行ったか知ってる?	
\\	野沢さん、実、どこに行ったか知ってる? 
\\	先生と 話があるって 言ってたよ。どんな話か 知らないけど。	
\\	先生と 話があるって 言ってたよ。どんな話か 知らないけど。 
\\	ああ、宿題 忘れたから、謝りにいったんだよ。多分。	
\\	ああ、宿題 忘れたから、謝りにいったんだよ。多分。 
\\	なーんだ。	
\\	なーんだ。 
\\	捨てる
\\	体重
\\	許す
\\	えさ
\\	無責任
\\	連れて行く
\\	獣医
\\	拾う
\\	かわいそう
\\	遠慮
\\	いいなぁ。やっぱり、犬、欲しいなぁ。	
\\	いいなぁ。やっぱり、犬、欲しいなぁ。 
\\	体重は1キロ位?	
\\	体重は1キロ位? 
\\	うん。今、800グラム位かな。	
\\	うん。今、800グラム位かな。 
\\	何才?	
\\	何才? 
\\	何才かわからないの。捨て犬だったんだ。	
\\	何才かわからないの。捨て犬だったんだ。 
\\	寒がっていたし、かわいそうだから、拾って来たんだ。	
\\	寒がっていたし、かわいそうだから、拾って来たんだ。 
\\	この犬・・・足、どうしたの?痛そう。	
\\	この犬・・・足、どうしたの?痛そう。 
\\	痛がっているんだよね。明日、獣医さんに連れて行くんだ。	
\\	痛がっているんだよね。明日、獣医さんに連れて行くんだ。 
\\	犬を捨てる無責任な人がいるんだ…。許せない。	
\\	犬を捨てる無責任な人がいるんだ…。許せない。 
\\	あ、外に出たがってる。	
\\	あ、外に出たがってる。 
\\	違うよ。えさを欲しがっているんだよ。	
\\	違うよ。えさを欲しがっているんだよ。 
\\	みなさん、いらっしゃい。	
\\	みなさん、いらっしゃい。 
\\	お邪魔しています。	
\\	お邪魔しています。 
\\	はい、紅茶とケーキ。	
\\	はい、紅茶とケーキ。 
\\	どうぞ、お構いなく。(グー)あ・・・。	
\\	どうぞ、お構いなく。(グー)あ・・・。 
\\	遠慮しないでいいのよ。若いんだから。	
\\	遠慮しないでいいのよ。若いんだから。 
\\	すみません。じゃ、遠慮なくいただきます。	
\\	すみません。じゃ、遠慮なくいただきます。 
\\	八百屋
\\	消費税
\\	ちょうだい
\\	半額
\\	玉ねぎ
\\	なす
\\	キャベツ
\\	お勧め
\\	重い
\\	体
\\	親
\\	アルバイト
\\	毎度
\\	らっしゃい。あ、先生。	
\\	らっしゃい。あ、先生。 
\\	あ、野沢さん?八百屋さんで アルバイト?	
\\	あ、野沢さん?八百屋さんで アルバイト? 
\\	あ、うちの親、八百屋を してるんです。	
\\	あ、うちの親、八百屋を してるんです。 
\\	大変だね。	
\\	大変だね。 
\\	はい。重いもの 運ばなきゃいけないから、体 痛くなっちゃうし、お金も もらえないし、ほんと、やに なりますよ。	
\\	はい。重いもの 運ばなきゃいけないから、体 痛くなっちゃうし、お金も もらえないし、ほんと、やに なりますよ。 
\\	で、先生、何、買います?	
\\	で、先生、何、買います? 
\\	おすすめ、ある?	
\\	おすすめ、ある? 
\\	今日は…キャベツは 2割引、なすと玉ねぎは 半額っすよ。	
\\	今日は…キャベツは 2割引、なすと玉ねぎは 半額っすよ。 
\\	じゃ、なすと 玉ねぎと キャベツ、ちょうだい。	
\\	じゃ、なすと 玉ねぎと キャベツ、ちょうだい。 
\\	はい。	
\\	はい。 
\\	キャベツ、4分の1しか いらないんだけど…。	
\\	キャベツ、4分の1しか いらないんだけど…。 
\\	あ、いいっすよ。切ります。	
\\	あ、いいっすよ。切ります。 
\\	えっと…。120+100+25=245。で、消費税が5パーセントだから…。	
\\	えっと…。120+100+25=245。で、消費税が5パーセントだから…。 
\\	05は…257円です!	
\\	05は…257円です! 
\\	はい。	
\\	はい。 
\\	毎度あり。	
\\	毎度あり。 
\\	子犬
\\	飼う
\\	必要
\\	ペットショップ
\\	ずっと
\\	悲しい
\\	飼い主
\\	不自由
\\	生まれつき
\\	年齢
\\	探す
\\	道場
\\	空手
\\	他
\\	赤
\\	お見舞い
\\	授業
\\	噂
\\	赤い
\\	折る
\\	腰
\\	転ぶ
\\	練習
\\	先生、栗林君は休みですか?	
\\	先生、栗林君は休みですか? 
\\	ああ、道場に行くときに、転んで足を折ったそうだ。	
\\	ああ、道場に行くときに、転んで足を折ったそうだ。 
\\	ええ?転んで腰を打った?痛そう。	
\\	ええ?転んで腰を打った?痛そう。 
\\	なあ、聞いたか?うわさによると、栗林、道場に行ったときに、腰を折ったみたいだぞ。	
\\	なあ、聞いたか?うわさによると、栗林、道場に行ったときに、腰を折ったみたいだぞ。 
\\	ええ?大変!授業が終わってからみんなでお見舞いに行こうよ。	
\\	ええ?大変!授業が終わってからみんなでお見舞いに行こうよ。 
\\	お見舞いに行く前に、お花を買わない?	
\\	お見舞いに行く前に、お花を買わない? 
\\	じゃ、この赤い花がいいよ。栗林、赤が好きだし。	
\\	じゃ、この赤い花がいいよ。栗林、赤が好きだし。 
\\	お見舞いに行く時、赤い花を持っていっちゃいけないって聞いた事がある。	
\\	お見舞いに行く時、赤い花を持っていっちゃいけないって聞いた事がある。 
\\	へー。じゃ、他の色にしよう。	
\\	へー。じゃ、他の色にしよう。 
\\	こんにちは。これ、お見舞い。	
\\	こんにちは。これ、お見舞い。 
\\	お見舞い?	
\\	お見舞い? 
\\	空手の練習中に腰を折ったって…。	
\\	空手の練習中に腰を折ったって…。 
\\	僕は、「道場に行くときに、転んで足を打った」って言ったんだけど…。	
\\	僕は、「道場に行くときに、転んで足を打った」って言ったんだけど…。 
\\	ペパロニピザ
\\	親戚
\\	マカロニ
\\	店長
\\	混ぜる
\\	お湯
\\	小麦粉
\\	生地
\\	レシピ本
\\	注文
\\	そもそも
\\	20分位前に、ペパロニピザを3枚注文した野沢ですけど...まだですか。	
\\	20分位前に、ペパロニピザを3枚注文した野沢ですけど...まだですか。 
\\	申し訳ございません。今作っているところなので、もう少しお待ちください。	
\\	申し訳ございません。今作っているところなので、もう少しお待ちください。 
\\	店長、ピザできましたか?	
\\	店長、ピザできましたか? 
\\	今、レシピ本を読んだところなんだ。	
\\	今、レシピ本を読んだところなんだ。 
\\	今から生地を作るところだから、もう少し時間がかかるなぁ。	
\\	今から生地を作るところだから、もう少し時間がかかるなぁ。 
\\	さてと、小麦粉300
\\	とお湯200
\\	を混ぜて・・・。	
\\	さてと、小麦粉300
\\	とお湯200
\\	を混ぜて・・・。 
\\	ところで、店長、ペパロ二って何ですか。	
\\	ところで、店長、ペパロ二って何ですか。 
\\	マカロニの親戚だろ。	
\\	マカロニの親戚だろ。 
\\	さすが店長。でも、初めてピザの注文が来ましたね。	
\\	さすが店長。でも、初めてピザの注文が来ましたね。 
\\	そうだな、そもそも、うちはそば屋だからな。	
\\	そうだな、そもそも、うちはそば屋だからな。 
\\	あ、もしもし?あ、すみません。今、お店を出たところなんです。もう少しお待ちください。	
\\	あ、もしもし?あ、すみません。今、お店を出たところなんです。もう少しお待ちください。 
\\	店長、急いでください。	
\\	店長、急いでください。 
\\	胸
\\	辛い
\\	夢
\\	誘う
\\	図書館
\\	参る
\\	もてる
\\	入学
\\	どきどきする
\\	出会う
\\	笑う
\\	那須実さん。出会ったばかりで、まだ あなたのことを よく 知らないけど、超タイプです。	
\\	那須実さん。出会ったばかりで、まだ あなたのことを よく 知らないけど、超タイプです。 
\\	マジで?	
\\	マジで? 
\\	実君を 見るたび、胸が どきどきするの。大好き。私と 付き合って。	
\\	実君を 見るたび、胸が どきどきするの。大好き。私と 付き合って。 
\\	ええ?	
\\	ええ? 
\\	高校に 入学して以来、ずっと 那須君のことが 好きでした。私と 結婚してください。	
\\	高校に 入学して以来、ずっと 那須君のことが 好きでした。私と 結婚してください。 
\\	うそ?	
\\	うそ? 
\\	実のことが 大好き。	
\\	実のことが 大好き。 
\\	あなたのことを 愛しています	
\\	あなたのことを 愛しています 
\\	ムニャムニャ・・・・。もてる男は つらいなぁ。でへへへへ。参ったなぁー。	
\\	ムニャムニャ・・・・。もてる男は つらいなぁ。でへへへへ。参ったなぁー。 
\\	ねぇ、「図書館で みんなで 一緒に 勉強しよう」って誘った人は 誰だった?	
\\	ねぇ、「図書館で みんなで 一緒に 勉強しよう」って誘った人は 誰だった? 
\\	実だね。でも・・・来たとたん 寝ちゃったね。	
\\	実だね。でも・・・来たとたん 寝ちゃったね。 
\\	いい夢 見ているんでしょう。笑いながら 寝てる。	
\\	いい夢 見ているんでしょう。笑いながら 寝てる。 
\\	完成
\\	休憩
\\	意味
\\	世界
\\	平和
\\	普通
\\	眠る
\\	彫る
\\	裏
\\	すずめ
\\	猫
\\	集合
\\	ここは日光東照宮です。	
\\	ここは日光東照宮です。 
\\	これは1636年に徳川家光によって完成されました。	
\\	これは1636年に徳川家光によって完成されました。 
\\	では、歩きましょう。	
\\	では、歩きましょう。 
\\	これは眠り猫です。	
\\	これは眠り猫です。 
\\	「左甚五郎によって作られた」と言われています。	
\\	「左甚五郎によって作られた」と言われています。 
\\	見てください。ネコの裏には、スズメが彫られています。	
\\	見てください。ネコの裏には、スズメが彫られています。 
\\	普通、スズメはネコに食べられてしまいますよね。	
\\	普通、スズメはネコに食べられてしまいますよね。 
\\	でもすぐそこにスズメがいるのに、ネコはスズメを食べないで眠っていますね。	
\\	でもすぐそこにスズメがいるのに、ネコはスズメを食べないで眠っていますね。 
\\	これは「平和な世界が徳川幕府によって作られた」という意味だと言われています。	
\\	これは「平和な世界が徳川幕府によって作られた」という意味だと言われています。 
\\	では、今から一時間、お昼休憩をとりましょう。	
\\	では、今から一時間、お昼休憩をとりましょう。 
\\	集合時間は1時半。集合場所はバス。いいですか?	
\\	集合時間は1時半。集合場所はバス。いいですか? 
\\	特に、那須君、野沢さん、遅刻をしないでください。	
\\	特に、那須君、野沢さん、遅刻をしないでください。 
\\	はーい。	
\\	はーい。 
\\	正夢
\\	全く
\\	遠足
\\	口
\\	蹴る
\\	ビンテージ
\\	ジーンズ
\\	お気に入り
\\	かむ
\\	ほえる
\\	幸せ
\\	洗う
\\	この前、たくさんの女の子たちに好かれる夢を見たんだ。正夢かな。	
\\	この前、たくさんの女の子たちに好かれる夢を見たんだ。正夢かな。 
\\	実は幸せだな・・・。僕、今週、本当についてなかったよ。	
\\	実は幸せだな・・・。僕、今週、本当についてなかったよ。 
\\	そう?	
\\	そう? 
\\	月曜、犬にほえられて、転んで足を打ったし・・・。	
\\	月曜、犬にほえられて、転んで足を打ったし・・・。 
\\	犬にほえられたから転んだのか?でも、かまれなくてよかったじゃん。	
\\	犬にほえられたから転んだのか?でも、かまれなくてよかったじゃん。 
\\	まあね。で、火曜に、お気に入りのジーンズを 親に 洗われたんだ。	
\\	まあね。で、火曜に、お気に入りのジーンズを 親に 洗われたんだ。 
\\	え?ジーンズ 洗わないの?	
\\	え?ジーンズ 洗わないの? 
\\	洗わないよ。あれ、ビンテージジーンズで 高かったんだよ。	
\\	洗わないよ。あれ、ビンテージジーンズで 高かったんだよ。 
\\	・・・っていうか、親が 洗濯してくれるんだ。	
\\	・・・っていうか、親が 洗濯してくれるんだ。 
\\	いいなぁ。うちの親、洗濯してくれないよ。	
\\	いいなぁ。うちの親、洗濯してくれないよ。 
\\	ふーん。それから、水曜日は道場で先輩に顔をけられて、口の中を切っちゃったし。	
\\	ふーん。それから、水曜日は道場で先輩に顔をけられて、口の中を切っちゃったし。 
\\	うわ・・・。	
\\	うわ・・・。 
\\	昨日の遠足は、雨に降られたし、全く、ついてないよ。	
\\	昨日の遠足は、雨に降られたし、全く、ついてないよ。 
\\	庭
\\	国歌
\\	今後
\\	感想文
\\	生徒
\\	教室
\\	立つ
\\	私立
\\	歌う
\\	公立
\\	那須実の母です。先生、質問があるのですが。	
\\	那須実の母です。先生、質問があるのですが。 
\\	あ、はい。	
\\	あ、はい。 
\\	(モンスターペアレントか?)	
\\	"(モンスターペアレントか?) 
\\	この学校では生徒に掃除させているんですか。	
\\	この学校では生徒に掃除させているんですか。 
\\	ええ、そうですよ。教室やトイレ、庭を掃除させます。	
\\	ええ、そうですよ。教室やトイレ、庭を掃除させます。 
\\	生徒達にたくさん本を読ませるそうですね。	
\\	生徒達にたくさん本を読ませるそうですね。 
\\	はい。その後で、感想文も書かせます。	
\\	はい。その後で、感想文も書かせます。 
\\	生徒に君が代を歌わせると聞きましたが、それは本当ですか。	
\\	"生徒に君が代を歌わせると聞きましたが、それは本当ですか。 
\\	ええ。立たせて国歌を歌わせますよ。生徒に国歌を覚えさせるためです。…	
\\	ええ。立たせて国歌を歌わせますよ。生徒に国歌を覚えさせるためです。… 
\\	お母さん、うちの学校は私立ですからね。もし嫌だったら、公立に行かせてください。	
\\	お母さん、うちの学校は私立ですからね。もし嫌だったら、公立に行かせてください。 
\\	いえいえ。気に入りました。今後もうちの息子をよろしくお願いします。	
\\	いえいえ。気に入りました。今後もうちの息子をよろしくお願いします。 
\\	はあ…。	
\\	はあ…。 
\\	先生、これ、つまらないものですが、どうぞ。	
\\	先生、これ、つまらないものですが、どうぞ。 
\\	じゃ、失礼します。	
\\	じゃ、失礼します。 
\\	何なんだ?	
\\	何なんだ? 
\\	兄貴
\\	野球
\\	そる
\\	床屋
\\	驚く
\\	心配する
\\	始める
\\	丸刈り
\\	ファン
\\	眉毛
\\	どうして泣いてるの?桜子。	
\\	どうして泣いてるの?桜子。 
\\	だって…実君が…。	
\\	だって…実君が…。 
\\	こら、実。女の子を泣かせるな。何したのよ。	
\\	こら、実。女の子を泣かせるな。何したのよ。 
\\	泣かせたつもりはないんだけど。	
\\	泣かせたつもりはないんだけど。 
\\	違う、違う…。菜味、心配させてごめん。でも、実君が笑わせるから…。	
\\	違う、違う…。菜味、心配させてごめん。でも、実君が笑わせるから…。 
\\	笑わせたつもりもないんだけど。	
\\	笑わせたつもりもないんだけど。 
\\	なんだ、笑ってたの?驚かせないでよ。	
\\	なんだ、笑ってたの?驚かせないでよ。 
\\	…実、その頭どうしたの?	
\\	…実、その頭どうしたの? 
\\	昨日、兄貴の床屋に髪を切りに行ったんだ。	
\\	昨日、兄貴の床屋に髪を切りに行ったんだ。 
\\	ああ、お兄さん、床屋さんだよね。	
\\	ああ、お兄さん、床屋さんだよね。 
\\	最初は楽しく話していたんだけど、野球の話を始めて、怒らせちゃったんだ。	
\\	最初は楽しく話していたんだけど、野球の話を始めて、怒らせちゃったんだ。 
\\	ほら、俺はジャイアンツファンだけど、兄貴はタイガースファンだから。	
\\	ほら、俺はジャイアンツファンだけど、兄貴はタイガースファンだから。 
\\	今年、ジャイアンツ調子いいけど、タイガースは調子悪いからね。	
\\	今年、ジャイアンツ調子いいけど、タイガースは調子悪いからね。 
\\	ああ。で、丸刈りにされて、眉毛もそられた。	
\\	ああ。で、丸刈りにされて、眉毛もそられた。 
\\	喪服
\\	葬式
\\	履く
\\	ミニスカート
\\	ファッションショー
\\	読む
\\	講師
\\	マナー
\\	制服
\\	意見
\\	こんにちは。近森渡の「ちょこっとマナー」の時間です。	
\\	"こんにちは。近森渡の「ちょこっとマナー」の時間です。 
\\	マナー講師の持田かね先生に色々教えてもらいましょう。	
\\	マナー講師の持田かね先生に色々教えてもらいましょう。 
\\	よろしくお願いいたします。	
\\	よろしくお願いいたします。 
\\	まず、ある高校生からの質問を 読ませてください。	
\\	まず、ある高校生からの質問を 読ませてください。 
\\	「親せきのお葬式に行きます。	
\\	"「親せきのお葬式に行きます。 
\\	制服はつまらないので、黒いミニスカートを履こうと思っています。	
\\	制服はつまらないので、黒いミニスカートを履こうと思っています。 
\\	どう思いますか。意見を聞かせてください。」	
\\	"どう思いますか。意見を聞かせてください。」 
\\	じゃ、質問させてください。	
\\	じゃ、質問させてください。 
\\	どうしてミニスカートを履きたいんですか。	
\\	どうしてミニスカートを履きたいんですか。 
\\	ちょっと、言わせてください。	
\\	ちょっと、言わせてください。 
\\	お葬式はファッションショーじゃないんです。喪服を着なさい。	
\\	お葬式はファッションショーじゃないんです。喪服を着なさい。 
\\	あ、先生、この人は高校生みたいです。	
\\	あ、先生、この人は高校生みたいです。 
\\	じゃ、制服が一番いいと思いますよ。	
\\	じゃ、制服が一番いいと思いますよ。 
\\	死ぬ
\\	この度
\\	香典
\\	先日
\\	イグアナ
\\	親切
\\	教授
\\	留学生
\\	人間
\\	絶対に
\\	近森渡のちょこっとマナーの時間です。	
\\	"近森渡のちょこっとマナーの時間です。 
\\	今日も持田かね先生が皆さんの質問に答えてくれますよ。	
\\	今日も持田かね先生が皆さんの質問に答えてくれますよ。 
\\	では最初の質問です。	
\\	では最初の質問です。 
\\	「私は留学生です。私の教授はとても親切で、大学でアルバイトさせてくれたり、おいしい料理を食べさせてくれたりします。	
\\	"「私は留学生です。私の教授はとても親切で、大学でアルバイトさせてくれたり、おいしい料理を食べさせてくれたりします。 
\\	先日、先生が飼っていたイグアナが死んだので、お葬式に行くつもりです。	
\\	先日、先生が飼っていたイグアナが死んだので、お葬式に行くつもりです。 
\\	香典は必要ですか。」	
\\	"香典は必要ですか。」 
\\	うーん。人間の葬式ではないので香典は必要ないと思いますよ。	
\\	うーん。人間の葬式ではないので香典は必要ないと思いますよ。 
\\	教授には何と言ったらいいですか?	
\\	教授には何と言ったらいいですか? 
\\	「この度は、ご愁傷様でした」とか「お悔やみを申し上げます」ですか。	
\\	"「この度は、ご愁傷様でした」とか「お悔やみを申し上げます」ですか。 
\\	そうですね。「この度は、どうも…」だけでもいいです。	
\\	"そうですね。「この度は、どうも…」だけでもいいです。 
\\	「新しいペットを飼った方がいい」とか 絶対に 言ってはいけませんよ。	
\\	"「新しいペットを飼った方がいい」とか 絶対に 言ってはいけませんよ。 
\\	警察
\\	払う
\\	捕まる
\\	恥ずかしい
\\	新婚旅行
\\	披露宴
\\	お祝い
\\	出席
\\	北海道
\\	罰金
\\	桃屋君、結婚 おめでとう。	
\\	桃屋君、結婚 おめでとう。 
\\	ありがとう。	
\\	ありがとう。 
\\	乾杯!	
\\	乾杯! 
\\	結婚式に 出席できなくて、ごめん。	
\\	結婚式に 出席できなくて、ごめん。 
\\	仕事 休めなかったんだ。	
\\	仕事 休めなかったんだ。 
\\	これ、お祝い。気に入るかどうかわからないけど。	
\\	これ、お祝い。気に入るかどうかわからないけど。 
\\	ありがとう。	
\\	ありがとう。 
\\	で、どうだった、結婚式は?	
\\	で、どうだった、結婚式は? 
\\	うーん。人の前で キスをさせられて、恥ずかしかった。	
\\	うーん。人の前で キスをさせられて、恥ずかしかった。 
\\	ははは。披露宴は どうだった。	
\\	ははは。披露宴は どうだった。 
\\	ああ。たくさん お酒を 飲ませられたよ。	
\\	ああ。たくさん お酒を 飲ませられたよ。 
\\	次の日、一日中、気持ち悪くて大変だった。	
\\	次の日、一日中、気持ち悪くて大変だった。 
\\	へぇー。新婚旅行へは もう行ったの?	
\\	へぇー。新婚旅行へは もう行ったの? 
\\	ああ、北海道に 行ったんだ。楽しかったよ。	
\\	ああ、北海道に 行ったんだ。楽しかったよ。 
\\	でも…。警察に捕まっちゃったんだ。	
\\	でも…。警察に捕まっちゃったんだ。 
\\	なんで。スピード?	
\\	なんで。スピード? 
\\	ああ、罰金払わせられた。	
\\	ああ、罰金払わせられた。 
\\	いくら?	
\\	いくら? 
\\	8万。	
\\	8万。 
\\	うわー。8万円も払わせられたの?	
\\	うわー。8万円も払わせられたの? 
\\	それはご愁傷様。	
\\	それはご愁傷様。 
\\	盗む
\\	話しかける
\\	選ぶ
\\	一億
\\	ダイヤ
\\	指輪
\\	涙
\\	だます
\\	突然
\\	引越す
\\	煙
\\	逃げる
\\	女性
\\	ニュースです。午後2時半頃、
\\	ジュエリーから1億円のダイヤが盗まれました。	
\\	ニュースです。午後2時半頃、
\\	ジュエリーから1億円のダイヤが盗まれました。 
\\	2時頃、65歳くらいの男性と若い女性のお客様が入ってきました。	
\\	2時頃、65歳くらいの男性と若い女性のお客様が入ってきました。 
\\	指輪を見に来たと言っていました。	
\\	指輪を見に来たと言っていました。 
\\	突然、男性のカバンから白い煙が出てきたんです。	
\\	突然、男性のカバンから白い煙が出てきたんです。 
\\	涙がたくさん出てきて、何も見えなくなりました。 
\\	…え?その男性ですか?はい。もちろん、ダイヤを持って逃げていきました。	
\\	…え?その男性ですか?はい。もちろん、ダイヤを持って逃げていきました。 
\\	男の人が話しかけてきたんです。	
\\	男の人が話しかけてきたんです。 
\\	ジュエリーはどこかと聞かれました。	
\\	ジュエリーはどこかと聞かれました。 
\\	引越してきたばかりで、道がわからないって言っていました。	
\\	引越してきたばかりで、道がわからないって言っていました。 
\\	娘の指輪を見に行くんだけど、選んでくれませんかって言われたんです。	
\\	娘の指輪を見に行くんだけど、選んでくれませんかって言われたんです。 
\\	とてもやさしそうな人だったので、連れてきてあげたんです…。	
\\	とてもやさしそうな人だったので、連れてきてあげたんです…。 
\\	私、だまされたんですか。	
\\	私、だまされたんですか。 
\\	最高
\\	カエル
\\	帽子
\\	連休
\\	千葉県
\\	急に
\\	最近
\\	暗い
\\	降る
\\	おたまじゃくし
\\	こんにちは。天気予報のお時間です。	
\\	こんにちは。天気予報のお時間です。 
\\	最近、暖かくなってきましたね。	
\\	最近、暖かくなってきましたね。 
\\	今日から連休が始まりますが、今日の午後から暑くなってくるでしょう。	
\\	今日から連休が始まりますが、今日の午後から暑くなってくるでしょう。 
\\	外に行くとき、帽子を忘れないでくださいね。	
\\	外に行くとき、帽子を忘れないでくださいね。 
\\	うわー。ネズミーランド最高。楽しいね。	
\\	うわー。ネズミーランド最高。楽しいね。 
\\	お腹すいてきた。	
\\	お腹すいてきた。 
\\	じゃ、何か食べよう。	
\\	じゃ、何か食べよう。 
\\	うー。急に涼しくなってきた。	
\\	うー。急に涼しくなってきた。 
\\	空も暗くなってきた。雨が降りそう・・・。	
\\	空も暗くなってきた。雨が降りそう・・・。 
\\	あ、降ってきた。	
\\	あ、降ってきた。 
\\	痛い!	
\\	痛い! 
\\	いてっ!あれ?これ、おたまじゃくし?	
\\	いてっ!あれ?これ、おたまじゃくし? 
\\	ええ?おたまじゃくしが降ってきたの?	
\\	ええ?おたまじゃくしが降ってきたの? 
\\	今、入ったニュースです。午前11時半ごろ、おたまじゃくしが降りました。	
\\	今、入ったニュースです。午前11時半ごろ、おたまじゃくしが降りました。 
\\	次のニュースです。	
\\	次のニュースです。 
\\	4月1日に
\\	バーガーが発売した「わびさびバーガー」が人気のようです。	
\\	"4月1日に
\\	バーガーが発売した「わびさびバーガー」が人気のようです。 
\\	ハンバーガーから、肉、野菜、ピクルスなどの具をすべて抜いたパンだけの大変シンプルなハンバーガーです。	
\\	ハンバーガーから、肉、野菜、ピクルスなどの具をすべて抜いたパンだけの大変シンプルなハンバーガーです。 
\\	わびさびバーガーを発明したチン・ゲンサイ社長は	
\\	わびさびバーガーを発明したチン・ゲンサイ社長は 
\\	「今まで30年間ハンバーガーを作ってきた。	
\\	"「今まで30年間ハンバーガーを作ってきた。 
\\	肉が嫌いなので、ハンバーガーをおいしいと思ったことはない。	
\\	肉が嫌いなので、ハンバーガーをおいしいと思ったことはない。 
\\	もともと、エイプリルフールの冗談としてわびさびバーガーを発売した。	
\\	もともと、エイプリルフールの冗談としてわびさびバーガーを発売した。 
\\	どうして人気が出てきたかわからないが、これからも、面白いハンバーガーを作っていきたい。」	
\\	"どうして人気が出てきたかわからないが、これからも、面白いハンバーガーを作っていきたい。」 
\\	というコメントを発表しました。	
\\	"というコメントを発表しました。 
\\	テリヤキ
\\	経つ
\\	煮る
\\	沸騰
\\	鍋
\\	溶ける
\\	たれ
\\	酢
\\	しょうゆ
\\	ひっくり返す
\\	鶏のテリヤキを作りましょう。	
\\	鶏のテリヤキを作りましょう。 
\\	材料は、鶏肉1枚、しょうゆ50
\\	、酢50
\\	、砂糖30グラム。	
\\	材料は、鶏肉1枚、しょうゆ50
\\	、酢50
\\	、砂糖30グラム。 
\\	皆さん、材料はありますか。	
\\	皆さん、材料はありますか。 
\\	はーい。	
\\	はーい。 
\\	では、作り始めましょう。	
\\	では、作り始めましょう。 
\\	しょうゆと、酢、と 砂糖を混ぜてください。	
\\	しょうゆと、酢、と 砂糖を混ぜてください。 
\\	ほら、野沢さん、那須君、作り始めてください。	
\\	ほら、野沢さん、那須君、作り始めてください。 
\\	砂糖が溶けるまで、混ぜ続けてくださいね。	
\\	砂糖が溶けるまで、混ぜ続けてくださいね。 
\\	混ぜ終わりましたか。	
\\	混ぜ終わりましたか。 
\\	次に、鍋にたれと鶏肉をいれます。	
\\	次に、鍋にたれと鶏肉をいれます。 
\\	それから、火をつけてください。	
\\	それから、火をつけてください。 
\\	先生、沸騰しました。	
\\	先生、沸騰しました。 
\\	じゃ、火を弱くしてください。10分間煮てください。	
\\	じゃ、火を弱くしてください。10分間煮てください。 
\\	先生10分経ちました。	
\\	先生10分経ちました。 
\\	じゃ、ひっくり返して、また10分煮てください。	
\\	じゃ、ひっくり返して、また10分煮てください。 
\\	作り終わりましたか?	
\\	作り終わりましたか? 
\\	先生、食べ始めても良いですか。	
\\	先生、食べ始めても良いですか。 
\\	はい、いいですよ。	
\\	はい、いいですよ。 
\\	うーん。おいしかった。	
\\	うーん。おいしかった。 
\\	実、もう食べ終わったの?はやいなー。	
\\	実、もう食べ終わったの?はやいなー。 
\\	答える
\\	しかし
\\	考える
\\	少ない
\\	つまり
\\	約
\\	記事
\\	文章
\\	意識する
\\	現実
\\	ここに「常用漢字」についての記事があります。	
\\	ここに「常用漢字」についての記事があります。 
\\	野沢さん、読んでくれますか。	
\\	"野沢さん、読んでくれますか。 
\\	はい。「文章を書く時に、常用漢字を意識するかどうか、質問をした。	
\\	はい。「文章を書く時に、常用漢字を意識するかどうか、質問をした。 
\\	意識すると答えた人は38
\\	。意識しないと答えた人は約60
\\	であった。」	
\\	"意識すると答えた人は38
\\	。意識しないと答えた人は約60
\\	であった。」 
\\	ありがとう。つまり、「この漢字は常用漢字かな」と考えながら、漢字を使っている人は少ないんだね。	
\\	"ありがとう。つまり、「この漢字は常用漢字かな」と考えながら、漢字を使っている人は少ないんだね。 
\\	じゃ、次を大葉さん、お願いします。	
\\	"じゃ、次を大葉さん、お願いします。 
\\	はい。	
\\	はい。 
\\	「憂鬱(ゆう・うつ)の鬱、語彙(ご・い)の彙、処方箋(しょ・ほう・せん)の箋は、2010年に常用漢字になった。	
\\	「憂鬱(ゆう・うつ)の鬱、語彙(ご・い)の彙、処方箋(しょ・ほう・せん)の箋は、2010年に常用漢字になった。 
\\	しかし、3つの漢字が全て書けると答えた人は4
\\	、書けないと答えた人は58
\\	であった。	
\\	鬱 
\\	憂鬱 
\\	彙 
\\	語彙 
\\	箋 
\\	処方箋 
\\	"しかし、3つの漢字が全て書けると答えた人は4
\\	、書けないと答えた人は58
\\	であった。 
\\	鬱 
\\	憂鬱 
\\	彙 
\\	語彙 
\\	箋 
\\	処方箋 
\\	これが現実なのであろう。」	
\\	これが現実なのであろう。」 
\\	よかった。	
\\	"よかった。 
\\	書けない人は俺だけじゃないんだ!	
\\	書けない人は俺だけじゃないんだ! 
\\	番号札
\\	方
\\	タトゥー
\\	腕
\\	秋田県
\\	旅館
\\	温泉
\\	取る
\\	露天風呂
\\	いらっしゃいませ。	
\\	いらっしゃいませ。 
\\	こちらの番号札をお取りになって、お待ちください。	
\\	こちらの番号札をお取りになって、お待ちください。 
\\	はい。	
\\	はい。 
\\	三番の番号札をお持ちのお客さま、カウンターへどうぞ。	
\\	三番の番号札をお持ちのお客さま、カウンターへどうぞ。 
\\	あ、はい。	
\\	あ、はい。 
\\	今週末、妻と一緒に温泉に行こうと思っているんですが、まだ予約できますか。	
\\	今週末、妻と一緒に温泉に行こうと思っているんですが、まだ予約できますか。 
\\	どちらの温泉をお考えですか。	
\\	どちらの温泉をお考えですか。 
\\	名前はちょっと分からないんですが、この写真の温泉旅館です。	
\\	名前はちょっと分からないんですが、この写真の温泉旅館です。 
\\	あ、これは秋田県の旅館ですね。今は夏なので、雪はございませんが…。	
\\	あ、これは秋田県の旅館ですね。今は夏なので、雪はございませんが…。 
\\	大丈夫です。	
\\	大丈夫です。 
\\	二名様、一泊二日ですね。お煙草はお吸いになりますか。	
\\	二名様、一泊二日ですね。お煙草はお吸いになりますか。 
\\	吸いません。	
\\	吸いません。 
\\	客室係
\\	本日
\\	頃
\\	敷く
\\	布団
\\	非常口
\\	階
\\	案内
\\	利用
\\	館内
\\	桃屋様、客室係の梨本と申します。	
\\	桃屋様、客室係の梨本と申します。 
\\	お部屋までご案内いたします。	
\\	お部屋までご案内いたします。 
\\	あ、はい。お願いします。	
\\	あ、はい。お願いします。 
\\	お荷物お持ちいたします。	
\\	お荷物お持ちいたします。 
\\	あ、すみません。	
\\	あ、すみません。 
\\	では、館内の説明をいたします。	
\\	では、館内の説明をいたします。 
\\	こちらは非常口でございます。	
\\	こちらは非常口でございます。 
\\	お風呂は全て1階にございます。	
\\	お風呂は全て1階にございます。 
\\	ご利用時間は朝4時から、夜12時まででございます。	
\\	ご利用時間は朝4時から、夜12時まででございます。 
\\	桃屋様のお部屋はこちらでございます。	
\\	桃屋様のお部屋はこちらでございます。 
\\	うわー、いい部屋ですね。	
\\	うわー、いい部屋ですね。 
\\	浴衣とタオルはこちらに入っております。	
\\	浴衣とタオルはこちらに入っております。 
\\	ではお茶をお入れいたしますね。	
\\	ではお茶をお入れいたしますね。 
\\	本日、ご夕食は7時でいかがでしょうか。	
\\	本日、ご夕食は7時でいかがでしょうか。 
\\	はい。	
\\	はい。 
\\	では、7時ごろに夕食をお部屋にお運びします。	
\\	では、7時ごろに夕食をお部屋にお運びします。 
\\	今からお風呂に行くので、その間に布団を敷いてもらえますか。	
\\	今からお風呂に行くので、その間に布団を敷いてもらえますか。 
\\	かしこまりました。では、失礼いたします。	
\\	かしこまりました。では、失礼いたします。 
\\	海外
\\	手紙
\\	普通
\\	なるほど
\\	目上
\\	育つ
\\	苦手
\\	敬語
\\	この前
\\	丁寧
\\	近森渡のちょこっとマナーのお時間です。	
\\	近森渡のちょこっとマナーのお時間です。 
\\	持田かね先生、今日もよろしくお願いいたします。	
\\	持田かね先生、今日もよろしくお願いいたします。 
\\	はい。	
\\	はい。 
\\	最初の質問です。	
\\	最初の質問です。 
\\	「こんにちは。俺は海外で育ったので、敬語が苦手だ。	
\\	"「こんにちは。俺は海外で育ったので、敬語が苦手だ。 
\\	この前先輩に『手伝ってほしいですか』と言ったら、むっとされた。	
\\	"この前先輩に『手伝ってほしいですか』と言ったら、むっとされた。 
\\	何が悪かったんだ?」	
\\	"何が悪かったんだ?」 
\\	あぁ、かわいそう。	
\\	あぁ、かわいそう。 
\\	でも目上の人に「てほしいですか」は使わないほうがいいですね。	
\\	"でも目上の人に「てほしいですか」は使わないほうがいいですね。 
\\	「手伝いましょうか」とか「手伝わせてください」。	
\\	"「手伝いましょうか」とか「手伝わせてください」。 
\\	もっと簡単に「手伝います」と言うこともできますよ。	
\\	もっと簡単に「手伝います」と言うこともできますよ。 
\\	なるほど。	
\\	"なるほど。 
\\	あの・・・先生、普通手紙で「俺」は使わないですよね。	
\\	あの・・・先生、普通手紙で「俺」は使わないですよね。 
\\	そうですね。	
\\	"そうですね。 
\\	知らない人に手紙を書く時は、「です」「ます」を使って丁寧に書きましょうね。	
\\	"知らない人に手紙を書く時は、「です」「ます」を使って丁寧に書きましょうね。 
\\	暇
\\	久しぶり
\\	最近
\\	ちゃんと
\\	締め切り
\\	論文
\\	出す
\\	ただ
\\	久々
\\	来週
\\	今週
\\	授業
\\	鈴木さん、久しぶりですね。この授業は嫌いですか?	
\\	鈴木さん、久しぶりですね。この授業は嫌いですか? 
\\	いいえ、山田先生の授業は大好きです。ただ、最近忙しいんです。	
\\	いいえ、山田先生の授業は大好きです。ただ、最近忙しいんです。 
\\	論文は?	
\\	論文は? 
\\	そうですね。	
\\	そうですね。 
\\	締め切りは来週ですよ!	
\\	締め切りは来週ですよ! 
\\	はい。ちゃんと出します。	
\\	はい。ちゃんと出します。 
\\	頑張ってね。	
\\	頑張ってね。 
\\	頑張ります!	
\\	頑張ります! 
\\	まね
\\	スタバ
\\	楽勝
\\	まさか
\\	ずるすぎる
\\	ネット
\\	ばかり
\\	大人
\\	マジで
\\	旨い
\\	禁煙
\\	そうかも
\\	ぎりぎりセーフ
\\	スノボ
\\	気分
\\	かしら
\\	モカラテ
\\	うん
\\	ところで
\\	僕
\\	さゆり来た!さゆりちゃん、こっち。	
\\	さゆり来た!さゆりちゃん、こっち。 
\\	二人はいつもスタバね。何で?	
\\	二人はいつもスタバね。何で? 
\\	禁煙だし、モカラテがうまい!	
\\	禁煙だし、モカラテがうまい! 
\\	なるほど、ところで、久しぶり、ひでおとメル!	
\\	なるほど、ところで、久しぶり、ひでおとメル! 
\\	久々!	
\\	久々! 
\\	久々! 
\\	まさか、ひでおのまねしてるの?	
\\	まさか、ひでおのまねしてるの? 
\\	なかなかいい日本語の先生だし!	
\\	なかなかいい日本語の先生だし! 
\\	ありがとうございますー。なかなかいい学生だな。	
\\	ありがとうございますー。なかなかいい学生だな。 
\\	ああ、先生のおかげで、本当に感謝しています・・・。	
\\	ああ、先生のおかげで、本当に感謝しています・・・。 
\\	相変わらずね。二人はいつ大人になるのかしら?	
\\	相変わらずね。二人はいつ大人になるのかしら? 
\\	ところで、最近何している?	
\\	ところで、最近何している? 
\\	ずっと論文。今出したばかり。大変だったよ!	
\\	ずっと論文。今出したばかり。大変だったよ! 
\\	ちゃんと出した?	
\\	ちゃんと出した? 
\\	うん、ぎりぎりセーフだった!今はものすごくいい気分!	
\\	うん、ぎりぎりセーフだった!今はものすごくいい気分! 
\\	よかった!	
\\	よかった! 
\\	よかったなー。	
\\	よかったなー。 
\\	二人は?論文どうした?	
\\	二人は?論文どうした? 
\\	僕はちゃんとネットで買ったよ!	
\\	僕はちゃんとネットで買ったよ! 
\\	相変わらず、おもしろくない。メルと変わらない!	
\\	相変わらず、おもしろくない。メルと変わらない! 
\\	そうかも。	
\\	そうかも。 
\\	もう一回聞くけど、ふたりは論文をちゃんと出した?	
\\	もう一回聞くけど、ふたりは論文をちゃんと出した? 
\\	楽勝だった!1 週間前に出したよ。	
\\	楽勝だった!1 週間前に出したよ。 
\\	マジで?	
\\	マジで? 
\\	うん。	
\\	うん。 
\\	日本語は大変だったでしょう?	
\\	日本語は大変だったでしょう? 
\\	日本語??英語で書いたよ。	
\\	日本語??英語で書いたよ。 
\\	ずるすぎる!!フェアじゃない。	
\\	ずるすぎる!!フェアじゃない。 
\\	まあね。	
\\	まあね。 
\\	僕も今日出した。すごい、いい気分だ。今週スノボは?	
\\	僕も今日出した。すごい、いい気分だ。今週スノボは? 
\\	行く、行く!	
\\	行く、行く! 
\\	私も行く。	
\\	私も行く。 
\\	じゃ、決まりだ!	
\\	じゃ、決まりだ! 
\\	ところで、ようこは?	
\\	ところで、ようこは? 
\\	わからない。	
\\	わからない。 
\\	僕もわからない。	
\\	僕もわからない。 
\\	彼女はどこにいるのかしら?締め切りは3時、今は2時半。大丈夫かな?	
\\	彼女はどこにいるのかしら?締め切りは3時、今は2時半。大丈夫かな? 
\\	[電話をかける・・・]	
\\	出ない!大丈夫かな??	
\\	出ない!大丈夫かな?? 
\\	間に合う
\\	直す
\\	お待たせしました.
\\	件
\\	希望
\\	推薦状
\\	推薦
\\	たいしたもん
\\	テーマ
\\	やっぱり
\\	失礼
\\	学長
\\	もしもし陽子ちゃん?	
\\	もしもし陽子ちゃん? 
\\	うん、恵理ちゃん久しぶり。	
\\	うん、恵理ちゃん久しぶり。 
\\	大丈夫?最近何しているの?皆心配しているよ。	
\\	大丈夫?最近何しているの?皆心配しているよ。 
\\	ずっと論文だった。大変だったよ。ギリギリ間に合った。	
\\	ずっと論文だった。大変だったよ。ギリギリ間に合った。 
\\	今どこ?	
\\	今どこ? 
\\	今、大学で学長待っているところ。	
\\	今、大学で学長待っているところ。 
\\	えっ、どうしたの?	
\\	えっ、どうしたの? 
\\	分からないけど、論文の件で学長が私を呼んだ。	
\\	分からないけど、論文の件で学長が私を呼んだ。 
\\	えー!?	
\\	えー!? 
\\	お待たせしました。山口さん、どうぞ。	
\\	お待たせしました。山口さん、どうぞ。 
\\	もう、行くね。後で掛け直す。じゃあね	
\\	もう、行くね。後で掛け直す。じゃあね 
\\	失礼します。	
\\	失礼します。 
\\	どうぞ。座って。今日呼んだ理由は君の論文の事です。	
\\	どうぞ。座って。今日呼んだ理由は君の論文の事です。 
\\	すみません。やっぱりそのホリエモンのテーマが悪かったですか?	
\\	すみません。やっぱりそのホリエモンのテーマが悪かったですか? 
\\	いやいやいや、すばらしい論文だ。大したものだよ。	
\\	いやいやいや、すばらしい論文だ。大したものだよ。 
\\	本当ですか?	
\\	本当ですか? 
\\	本当だよ。それで
\\	に君を推薦したいんだが、どうかね?	
\\	本当だよ。それで
\\	に君を推薦したいんだが、どうかね? 
\\	ですか!?	
\\	ですか!? 
\\	いやいや、
\\	だよ。	
\\	いやいや、
\\	だよ。 
\\	ですか。ありがとうございます。第2希望でした。	
\\	ですか。ありがとうございます。第2希望でした。 
\\	お待たせ
\\	就職活動
\\	二人きり
\\	卒業旅行
\\	ピンチ・ヒッター
\\	楽しみ
\\	トリノオリンピック
\\	賛成
\\	何言ってるの?
\\	陽子、お待たせ!	
\\	陽子、お待たせ! 
\\	おはよう、メル!	
\\	おはよう、メル! 
\\	さゆりとひでおは?	
\\	さゆりとひでおは? 
\\	さゆりは論文を書き直してるところ。ひでおはまだ就活中。
\\	に決まったけど、今の状態では入社できないかも。二人は卒業旅行に行けるかどうかまだわからない。	
\\	さゆりは論文を書き直してるところ。ひでおはまだ就活中。
\\	に決まったけど、今の状態では入社できないかも。二人は卒業旅行に行けるかどうかまだわからない。 
\\	二人きりか!素敵だね!	
\\	二人きりか!素敵だね! 
\\	何言ってるの?ピンチ・ヒッターがいるよ!私が呼んだの。あっ!いらっしゃった!	
\\	何言ってるの?ピンチ・ヒッターがいるよ!私が呼んだの。あっ!いらっしゃった! 
\\	学長!!!	
\\	学長!!! 
\\	どうも、どうも。久しぶりだね、メルは元気?	
\\	どうも、どうも。久しぶりだね、メルは元気? 
\\	おかげさまで元気です。	
\\	おかげさまで元気です。 
\\	この旅行楽しみですね。どこがいい?	
\\	この旅行楽しみですね。どこがいい? 
\\	トリノがいい!まだオリンピックをやってます!	
\\	トリノがいい!まだオリンピックをやってます! 
\\	でも、切符を買えるかどうかわからないし。学長、ご希望はありますか?	
\\	・ご 
\\	でも、切符を買えるかどうかわからないし。学長、ご希望はありますか? 
\\	・ご 
\\	そうですね。私の卒業旅行は沖縄でした。もう一度、行きたいなあ。	
\\	そうですね。私の卒業旅行は沖縄でした。もう一度、行きたいなあ。 
\\	沖縄もいいですね。私は賛成です。	
\\	沖縄もいいですね。私は賛成です。 
\\	私も
\\	です!	
\\	私も
\\	です! 
\\	じゃ、決まり!	
\\	じゃ、決まり! 
\\	行きましょう!	
\\	行きましょう! 
\\	到着
\\	ゴーヤチャンプルー
\\	泡盛
\\	がりがり
\\	着替える
\\	シュノーケル
\\	懐かしい
\\	沖縄
\\	やっと
\\	タコライス
\\	到着!	
\\	やっと海に着いた!きれいね
\\	沖縄は最高!	
\\	やっと海に着いた!きれいね
\\	沖縄は最高! 
\\	そうだね。懐かしいな。30 年ぶりだ。	
\\	そうだね。懐かしいな。30 年ぶりだ。 
\\	気持ちいい!海が大好き!	
\\	気持ちいい!海が大好き! 
\\	シュノーケルをやってみたいな。	
\\	シュノーケルをやってみたいな。 
\\	いい考えですね!私もやってみたいです。	
\\	いい考えですね!私もやってみたいです。 
\\	今は4.30 で、まだやっているかどうか・・・	
\\	今は4.30 で、まだやっているかどうか・・・ 
\\	聞いてみよう!	
\\	聞いてみよう! 
\\	まだ大丈夫。早く着替えよう。	
\\	まだ大丈夫。早く着替えよう。 
\\	えー信じられない!学長すごい!格好いいー!!	
\\	えー信じられない!学長すごい!格好いいー!! 
\\	えぇぇぇ、すごい身体してます!	
\\	えぇぇぇ、すごい身体してます! 
\\	まあね。	
\\	まあね。 
\\	すごい!メルに全然勝っている!すごい学長!	
\\	すごい!メルに全然勝っている!すごい学長! 
\\	プルとタコライスを食べよう。それで肉をつけよう。	
\\	弁護士
\\	優秀
\\	スコール
\\	雷
\\	稲妻
\\	お土産
\\	ハブアイス
\\	ストロー
\\	久々。元気?	
\\	久々。元気? 
\\	元気じゃない。沖縄は大変だった!	
\\	元気じゃない。沖縄は大変だった! 
\\	どうしたの?	
\\	どうしたの? 
\\	シュノーケルするところでスコールが来て、雷はごろごろ、雨はざあざあ。命が危なかった。ぎりぎりで逃げた。	
\\	シュノーケルするところでスコールが来て、雷はごろごろ、雨はざあざあ。命が危なかった。ぎりぎりで逃げた。 
\\	大変だったね。	
\\	大変だったね。 
\\	大変だったけど、メルと学長の方が倍ぐらい大変だった。	
\\	大変だったけど、メルと学長の方が倍ぐらい大変だった。 
\\	大変?何やったの?	
\\	大変?何やったの? 
\\	それはちょっと言いにくい。本人から話を聞いて。	
\\	それはちょっと言いにくい。本人から話を聞いて。 
\\	じゃ、二人は今どこ?	
\\	じゃ、二人は今どこ? 
\\	それも言いにくいな。	
\\	それも言いにくいな。 
\\	教えて!	
\\	教えて! 
\\	無理。教えない。でも、優秀な弁護士が必要。ところで、お土産!	
\\	無理。教えない。でも、優秀な弁護士が必要。ところで、お土産! 
\\	ありがとう。何だろう?えぇぇぇ、これは…ハブアイス!	
\\	ありがとう。何だろう?えぇぇぇ、これは…ハブアイス! 
\\	そうそう!すごいでしょう!早く食べてみて。	
\\	そうそう!すごいでしょう!早く食べてみて。 
\\	溶けてる!どろどろ!	
\\	溶けてる!どろどろ! 
\\	飛行機の中暑かったから。まあ、でも味は変わってないよ。ストローで飲んでみて。	
\\	飛行機の中暑かったから。まあ、でも味は変わってないよ。ストローで飲んでみて。 
\\	うん。わかった。おいしい!ありがとう。	
\\	うん。わかった。おいしい!ありがとう。 
\\	でしょう?えっ、秀雄は?彼の分もあるの。	
\\	でしょう?えっ、秀雄は?彼の分もあるの。 
\\	彼の就活はヤバい。仕事が見つかるかどうか…	
\\	彼の就活はヤバい。仕事が見つかるかどうか… 
\\	えぇぇぇ…でも自分の方がもっと心配。仕事まだ決まってないんだ。ねぇ、頼みがあるんだけど。来週、大事な面接があるから手伝ってほしい。	
\\	えぇぇぇ…でも自分の方がもっと心配。仕事まだ決まってないんだ。ねぇ、頼みがあるんだけど。来週、大事な面接があるから手伝ってほしい。 
\\	いいよ。	
\\	いいよ。 
\\	ありがとう。助かった!	
\\	ありがとう。助かった! 
\\	お餅
\\	そっと
\\	とうとう
\\	せめて
\\	せっせと
\\	ニコニコ
\\	大晦日
\\	地蔵
\\	昔話
\\	村外れ
\\	激しい
\\	積もる
\\	後ろ姿
\\	とんでもない
\\	優秀な
\\	困る
\\	現在
\\	血祭り
\\	不法侵入
\\	餌
\\	鮫
\\	水族館
\\	べろべろ
\\	力になる
\\	時間を割く
\\	見積もり
\\	はい。	
\\	はい。 
\\	もしもし。陽子!どうなってる?	
\\	もしもし。陽子!どうなってる? 
\\	今、先生と会うところ。もうちょっと待ってて。	
\\	今、先生と会うところ。もうちょっと待ってて。 
\\	これが僕らの最後の電話、頼むよ。	
\\	これが僕らの最後の電話、頼むよ。 
\\	わかったけど、二人のせいで今日の面接行けないよ。全く、とんでもない二人だよ!	
\\	わかったけど、二人のせいで今日の面接行けないよ。全く、とんでもない二人だよ! 
\\	ごめん。よろしく!	
\\	ごめん。よろしく! 
\\	どうぞお入りください。	
\\	どうぞお入りください。 
\\	失礼します。	
\\	失礼します。 
\\	どうぞ。座ってください。	
\\	どうぞ。座ってください。 
\\	時間を割いていただいてありがとうございます。	
\\	時間を割いていただいてありがとうございます。 
\\	いえいえ、学長が昔の友人なので力になりますよ。	
\\	いえいえ、学長が昔の友人なので力になりますよ。 
\\	ありがとうございます。戸鍋先生がついていますのできっと問題ないと思います。	
\\	ありがとうございます。戸鍋先生がついていますのできっと問題ないと思います。 
\\	ところで、テレビで見ましたけど、もう一度あの夜のことを教えてください。	
\\	ところで、テレビで見ましたけど、もう一度あの夜のことを教えてください。 
\\	そうですね。あの夜、二人が泡盛を飲みすぎて、べろべろになるまで飲み続けました。話題が近くにあるチュラ海水族館になって、学長がサメと一緒に泳ぎたがっていました。サメがちゃんとえさを食べているから問題ないと思って、二人は水族館に不法侵入して、結局血祭りでした。現在二人は病院にいます。	
\\	そうですね。あの夜、二人が泡盛を飲みすぎて、べろべろになるまで飲み続けました。話題が近くにあるチュラ海水族館になって、学長がサメと一緒に泳ぎたがっていました。サメがちゃんとえさを食べているから問題ないと思って、二人は水族館に不法侵入して、結局血祭りでした。現在二人は病院にいます。 
\\	困った二人ですね。だけど、私は日本で一番優秀な弁護士だから大丈夫です。任せてください!絶対に助けます!では、見積もりですが・・・	
\\	困った二人ですね。だけど、私は日本で一番優秀な弁護士だから大丈夫です。任せてください!絶対に助けます!では、見積もりですが・・・ 
\\	えぇぇぇ!ゼロが多すぎて!どうしよう??	
\\	えぇぇぇ!ゼロが多すぎて!どうしよう?? 
\\	買い取り
\\	価値
\\	羨ましい
\\	子供の頃
\\	(~と)呼ぶ
\\	下の名前
\\	フェラーリ
\\	ランボルギーニ
\\	こんな所で
\\	ぼろい
\\	愛車
\\	処分
\\	いらっしゃいませ。「ピーター車買取」へようこそ。	
\\	いらっしゃいませ。「ピーター車買取」へようこそ。 
\\	すみません。車を売りたいのですが。	
\\	すみません。車を売りたいのですが。 
\\	はい。かしこまりました。ご愛車はどちらでしょうか。	
\\	はい。かしこまりました。ご愛車はどちらでしょうか。 
\\	あのぼろい赤のミニです。	
\\	あのぼろい赤のミニです。 
\\	ずいぶん古いものですね。少々お待ちください。こちらで座って待っていてください。	
\\	ずいぶん古いものですね。少々お待ちください。こちらで座って待っていてください。 
\\	はい。はぁ~。弁護士料払えないから、高く売りたいなあ。	
\\	はい。はぁ~。弁護士料払えないから、高く売りたいなあ。 
\\	陽子ちゃん!どうしたの、こんな所で。	
\\	陽子ちゃん!どうしたの、こんな所で。 
\\	戸鍋先生!こんなところで。	
\\	戸鍋先生!こんなところで。 
\\	下の名前は亮。亮って呼んでよ。	
\\	下の名前は亮。亮って呼んでよ。 
\\	はい。	
\\	はい。 
\\	今日、車を売りに来たんだよ。	
\\	今日、車を売りに来たんだよ。 
\\	お金が必要なんですか。	
\\	お金が必要なんですか。 
\\	今回の弁護料で新しい車を買うつもりなんだよ。子供の頃、ずっとイタリアのランボルギーニがほしくて、あるいは、フェラーリも悪くないなあ。まだ決めてないけどね。弁護士でよかったよ。	
\\	今回の弁護料で新しい車を買うつもりなんだよ。子供の頃、ずっとイタリアのランボルギーニがほしくて、あるいは、フェラーリも悪くないなあ。まだ決めてないけどね。弁護士でよかったよ。 
\\	いいですねえ。うらやましいです。	
\\	いいですねえ。うらやましいです。 
\\	陽子ちゃんは何しに来たの?	
\\	陽子ちゃんは何しに来たの? 
\\	あっ、えっと、車を探しに来たんです。	
\\	あっ、えっと、車を探しに来たんです。 
\\	お客様、お車の見積もりですが、お客様に二万円を支払っていただきます。車に価値はございません。処分するのには二万円必要です。	
\\	お客様、お車の見積もりですが、お客様に二万円を支払っていただきます。車に価値はございません。処分するのには二万円必要です。 
\\	えぇぇっ!それじゃ弁護士料払えないよ。	
\\	えぇぇっ!それじゃ弁護士料払えないよ。 
\\	ん~、じゃ今夜一緒に食事しながら、ゆっくり話しましょう。おいしい店いっぱい知ってるんだ。むふっふっふ・・・	
\\	ん~、じゃ今夜一緒に食事しながら、ゆっくり話しましょう。おいしい店いっぱい知ってるんだ。むふっふっふ・・・ 
\\	えっ~!戸鍋先生、目が変!どうしたんだろう!?	
\\	えっ~!戸鍋先生、目が変!どうしたんだろう!? 
\\	フレンチレストラン
\\	約束
\\	相談
\\	遠慮
\\	解決
\\	親友
\\	どうしても
\\	こういう時
\\	やばい
\\	迎えに行く
\\	連絡
\\	もしもし。亮ですけど。	
\\	もしもし。亮ですけど。 
\\	はい。	
\\	はい。 
\\	土曜日の6時に東京で一番おいしいフレンチレストランを予約しているので、5時半に僕の新しいフェラーリで迎えに行きます。	
\\	土曜日の6時に東京で一番おいしいフレンチレストランを予約しているので、5時半に僕の新しいフェラーリで迎えに行きます。 
\\	はい、わかりました。	
\\	はい、わかりました。 
\\	では、土曜日に。むふっふっふ・・・	
\\	では、土曜日に。むふっふっふ・・・ 
\\	どうしょう!?かなりやばい。どうしよう!本当に困った。あっ、分かった!こういう時には智子だ。高校の親友で、困ったとき遠慮なく相談していいって約束した。きっと智子が助けてくれる。しかし4年ぶりで、大丈夫かな。	
\\	どうしょう!?かなりやばい。どうしよう!本当に困った。あっ、分かった!こういう時には智子だ。高校の親友で、困ったとき遠慮なく相談していいって約束した。きっと智子が助けてくれる。しかし4年ぶりで、大丈夫かな。 
\\	智子ですけど。	
\\	智子ですけど。 
\\	もしもし、久しぶり・・・	
\\	もしもし、久しぶり・・・ 
\\	陽子ちゃん?	
\\	陽子ちゃん? 
\\	はい。	
\\	はい。 
\\	久しぶり!元気?	
\\	久しぶり!元気? 
\\	あんまり。最近ちょっと困ってる。ごめんね。ずっと連絡がなくて急に電話して。でも、今すごく困ってるんだ。ごめん、どうしても智子に力になってほしいことがある。今すぐ智子が必要なんだ。	
\\	あんまり。最近ちょっと困ってる。ごめんね。ずっと連絡がなくて急に電話して。でも、今すごく困ってるんだ。ごめん、どうしても智子に力になってほしいことがある。今すぐ智子が必要なんだ。 
\\	わかった。今すぐ行く!	
\\	わかった。今すぐ行く! 
\\	でも、何の問題とか、まだ何も説明してないよ。	
\\	でも、何の問題とか、まだ何も説明してないよ。 
\\	大丈夫。着いたら教えて。どんな問題でも解決できるから!	
\\	大丈夫。着いたら教えて。どんな問題でも解決できるから! 
\\	既に
\\	ロマネコンティ
\\	ソムリエさん
\\	プラス
\\	そういう訳で
\\	昔
\\	可能性がある
\\	失う
\\	件
\\	最年少合格者
\\	司法試験
\\	国際弁護士
\\	弁護士料
\\	よく来たね。どうぞ。	
\\	よく来たね。どうぞ。 
\\	実は、今日、私の友人も一緒なんです。	
\\	実は、今日、私の友人も一緒なんです。 
\\	彼女もかわいい?	
\\	彼女もかわいい? 
\\	とんでもない人ね。	
\\	とんでもない人ね。 
\\	いやぁ、かわいい、かわいい。はじめまして。私は・・・	
\\	いやぁ、かわいい、かわいい。はじめまして。私は・・・ 
\\	いや、もうわかってます。陽子から戸鍋先生の話はもうすでに聞いていますから。大したことありませんね。私は高橋智子です。国際弁護士として、現在、ニューヨークとチューリヒで仕事しています。司法試験の最年少合格者です。世界で一番優秀な弁護士でございます。さて、今夜のこのミーティングなんですが、もちろん仕事の話でしょうね。そうじゃないとあなた、免許を失う可能性がありますよ。	
\\	いや、もうわかってます。陽子から戸鍋先生の話はもうすでに聞いていますから。大したことありませんね。私は高橋智子です。国際弁護士として、現在、ニューヨークとチューリヒで仕事しています。司法試験の最年少合格者です。世界で一番優秀な弁護士でございます。さて、今夜のこのミーティングなんですが、もちろん仕事の話でしょうね。そうじゃないとあなた、免許を失う可能性がありますよ。 
\\	あっ、あっ、あっ、あ~。う~。	
\\	あっ、あっ、あっ、あ~。う~。 
\\	すか。	
\\	本当にごめんなさい。昔の僕は違った。全然違ったんです。ある女に昔、ひどく傷つけられまして。それ以来私はちょっと変わりました。許してください。私の本当の姿は違います。	
\\	本当にごめんなさい。昔の僕は違った。全然違ったんです。ある女に昔、ひどく傷つけられまして。それ以来私はちょっと変わりました。許してください。私の本当の姿は違います。 
\\	もちろん私達も知っています。戸鍋先生は本当にやさしい人です。そういう訳で、弁護士料をゼロにして・・・	
\\	もちろん私達も知っています。戸鍋先生は本当にやさしい人です。そういう訳で、弁護士料をゼロにして・・・ 
\\	プラス、今夜ごちそうしていただきます。	
\\	プラス、今夜ごちそうしていただきます。 
\\	そうですね。そうします。	
\\	そうですね。そうします。 
\\	では、ソムリエさん、ロマネコンティを3 本お願いします。	
\\	では、ソムリエさん、ロマネコンティを3 本お願いします。 
\\	は!?	
\\	は!? 
\\	先生、ありがとう!	
\\	先生、ありがとう! 
\\	は!?	
\\	は!? 
\\	うっかり
\\	図々しい
\\	全く
\\	何よりも
\\	あばよ
\\	馴染む
\\	あり得ない
\\	ろくでなし
\\	なかなか難しいケースですね。どうやったら二人を助けられるか・・・	
\\	なかなか難しいケースですね。どうやったら二人を助けられるか・・・ 
\\	そうですか?簡単ですよ。一体、何年弁護士やってるんですか。まったく。	
\\	そうですか?簡単ですよ。一体、何年弁護士やってるんですか。まったく。 
\\	そ、そうですか。さすが、最年少合格者!	
\\	そ、そうですか。さすが、最年少合格者! 
\\	あっ、そういえば、メルから手紙が来たんだった!うっかり忘れるところだった。今、読むね。	
\\	あっ、そういえば、メルから手紙が来たんだった!うっかり忘れるところだった。今、読むね。 
\\	陽子ちゃん、元気にしてますか。優秀な弁護士にはちゃんと会えましたか。学長の親友ならきっと親切な人だろうと思います。	
\\	陽子ちゃん、元気にしてますか。優秀な弁護士にはちゃんと会えましたか。学長の親友ならきっと親切な人だろうと思います。 
\\	その通りですよ!	
\\	その通りですよ! 
\\	何言ってるんですか、ずうずうしい。	
\\	何言ってるんですか、ずうずうしい。 
\\	まあ、待って。続きがあるから。えっと、	
\\	まあ、待って。続きがあるから。えっと、 
\\	こんなことを言うと陽子ちゃんがあきれるかもしれないけど、僕と学長はここでの生活を楽しんでます。	
\\	こんなことを言うと陽子ちゃんがあきれるかもしれないけど、僕と学長はここでの生活を楽しんでます。 
\\	何だって!?	
\\	何だって!? 
\\	どうして?	
\\	どうして? 
\\	ここは最高!食事がただで食べ放題。そして、何よりも、毎日ただで日本語のレッスンし放題!東京では誰も日本語で話しかけてくれなかったのに、ここではみんな全部日本語!	
\\	ここは最高!食事がただで食べ放題。そして、何よりも、毎日ただで日本語のレッスンし放題!東京では誰も日本語で話しかけてくれなかったのに、ここではみんな全部日本語! 
\\	そりゃ、そうだろう。	
\\	そりゃ、そうだろう。 
\\	陽子ちゃんには悪いけど、しばらくここでの生活を楽しみたくなってきました。	
\\	陽子ちゃんには悪いけど、しばらくここでの生活を楽しみたくなってきました。 
\\	あきれた!	
\\	あきれた! 
\\	学長もいっぱい友達ができて、毎日楽しそうです。僕らにもうしばらく時間をください。また手紙書きます。あばよ!メル。	
\\	学長もいっぱい友達ができて、毎日楽しそうです。僕らにもうしばらく時間をください。また手紙書きます。あばよ!メル。 
\\	いいなあ、メル君、すっかりなじんでて。うらやましいなあ。	
\\	いいなあ、メル君、すっかりなじんでて。うらやましいなあ。 
\\	こんなろくでなしとつきあってたの?こんな人たちのために私を呼んだなんてあり得ない!陽子にはがっかりよ。	
\\	こんなろくでなしとつきあってたの?こんな人たちのために私を呼んだなんてあり得ない!陽子にはがっかりよ。 
\\	成田空港
\\	態度が悪い
\\	態度がいい
\\	態度
\\	お客様
\\	関係ある
\\	関係ない
\\	関係
\\	おしゃれ(な)
\\	とにかく
\\	超える
\\	旅立つ
\\	もういい、陽子、私帰る!こんなことで時間を無駄にしたくない。	
\\	もういい、陽子、私帰る!こんなことで時間を無駄にしたくない。 
\\	ちょっと待って、智子、もう少しだけ話を聞いて!お願い!お願いだから!	
\\	ちょっと待って、智子、もう少しだけ話を聞いて!お願い!お願いだから! 
\\	タクシー!	
\\	こんばんは、どこまでですか。	
\\	こんばんは、どこまでですか。 
\\	成田空港へ。	
\\	成田空港へ。 
\\	空港ですと十万を超えますよ。	
\\	空港ですと十万を超えますよ。 
\\	いくらでもかまわない。とにかく空港へ行って。早くここを出たいの。早く自分のおしゃれな生活に戻りたい。やっぱり戻ってきたのは失敗だった。早くここを忘れたい。	
\\	いくらでもかまわない。とにかく空港へ行って。早くここを出たいの。早く自分のおしゃれな生活に戻りたい。やっぱり戻ってきたのは失敗だった。早くここを忘れたい。 
\\	どうしましたか。けんかでもしましたか。	
\\	どうしましたか。けんかでもしましたか。 
\\	けんかしてきたばかりよ 。でもあなたに関係ないでしょう。	
\\	けんかしてきたばかりよ 。でもあなたに関係ないでしょう。 
\\	いや、関係あります。私のお客様ですから。お客様の幸せが私の幸せです。お客様が悲しいと私も悲しいです。	
\\	いや、関係あります。私のお客様ですから。お客様の幸せが私の幸せです。お客様が悲しいと私も悲しいです。 
\\	すみません。やっぱり私の態度が悪かった。ごめんなさい。	
\\	すみません。やっぱり私の態度が悪かった。ごめんなさい。 
\\	いや、問題ございません。 でも、そんな気持ちで旅立たないでください。私でよければ、気分転換のお手伝いをしますよ。	
\\	いや、問題ございません。 でも、そんな気持ちで旅立たないでください。私でよければ、気分転換のお手伝いをしますよ。 
\\	そうね。そう言われたらお腹がすいてきたわ。	
\\	そうね。そう言われたらお腹がすいてきたわ。 
\\	もしよかったら、いいラーメン屋を知ってますよ。東京で一番汚いけど、一番安くて、一番おいしいラーメンです。	
\\	もしよかったら、いいラーメン屋を知ってますよ。東京で一番汚いけど、一番安くて、一番おいしいラーメンです。 
\\	じゃあ、そこへ行って。もしよかったら、運転手さんも一緒にどう?	
\\	じゃあ、そこへ行って。もしよかったら、運転手さんも一緒にどう? 
\\	喜んでお供します。	
\\	喜んでお供します。 
\\	ご無沙汰
\\	目印
\\	印
\\	お似合い
\\	意外(と)
\\	世間
\\	稚内
\\	(時間が)経つ
\\	出稼ぎ
\\	逆
\\	一生懸命
\\	地元
\\	丁
\\	気が合う
\\	美味しいという印
\\	ゴキブリ
\\	稼ぐ
\\	いらっしゃい・いらっしゃいませ	
\\	いらっしゃい・いらっしゃいませ 
\\	おかげさまで。ご無沙汰していました。お元気ですか。	
\\	おかげさまで。ご無沙汰していました。お元気ですか。 
\\	元気いっぱいですよ。お友達ですか。	
\\	元気いっぱいですよ。お友達ですか。 
\\	そうです。	
\\	そうです。 
\\	はじめまして、よろしくお願いします。	
\\	はじめまして、よろしくお願いします。 
\\	こちらこそ、さぁー、こちらへどうぞ。	
\\	こちらこそ、さぁー、こちらへどうぞ。 
\\	嘘じゃなかったんですね。本当に汚いな。あ!ゴキブリ!	
\\	嘘じゃなかったんですね。本当に汚いな。あ!ゴキブリ! 
\\	まあまあ、大丈夫ですよ。ゴキブリは美味しいという印ですよ。気にしないで。	
\\	まあまあ、大丈夫ですよ。ゴキブリは美味しいという印ですよ。気にしないで。 
\\	なんになさいますか。	
\\	なんになさいますか。 
\\	味噌ラーメンください・味噌ラーメンお願いします。	
\\	味噌ラーメンください・味噌ラーメンお願いします。 
\\	お二人、気が合いますね。お父さん、味噌二丁。	
\\	お二人、気が合いますね。お父さん、味噌二丁。 
\\	あいよう!	
\\	あいよう! 
\\	すごいおばあさんですね。	
\\	すごいおばあさんですね。 
\\	でしょう!ここに来ると地元を思い出します。	
\\	でしょう!ここに来ると地元を思い出します。 
\\	ご出身はどちらですか。	
\\	ご出身はどちらですか。 
\\	北海道の田舎から来ました。	
\\	北海道の田舎から来ました。 
\\	え?私も北海道です。でも田舎が大嫌いで、どうしても北海道を出たくて、一生懸命勉強して、東京の大学に入りました。	
\\	え?私も北海道です。でも田舎が大嫌いで、どうしても北海道を出たくて、一生懸命勉強して、東京の大学に入りました。 
\\	私は逆です。ここが嫌いだけど、出稼ぎのために出てきました。たまには帰ったりしますか。	
\\	私は逆です。ここが嫌いだけど、出稼ぎのために出てきました。たまには帰ったりしますか。 
\\	全然帰ってないです。もう六年経ちます。	
\\	全然帰ってないです。もう六年経ちます。 
\\	ところで、北海道のどこですか。	
\\	ところで、北海道のどこですか。 
\\	稚内です。	
\\	稚内です。 
\\	え?稚内?私も稚内です。	
\\	え?稚内?私も稚内です。 
\\	本当に稚内?じゃ、一緒ですね。	
\\	本当に稚内?じゃ、一緒ですね。 
\\	はい	
\\	はい 
\\	世間は狭いですね。二人は意外とお似合いですよ。お父さん、私たちも昔に戻りたいね。	
\\	世間は狭いですね。二人は意外とお似合いですよ。お父さん、私たちも昔に戻りたいね。 
\\	六本木ヒルズ
\\	招待券
\\	見送り
\\	無事
\\	行方不明
\\	落ち着く
\\	様子
\\	かしら
\\	宝物
\\	ロボット
\\	レジデンス
\\	商品券
\\	ドア開く音
\\	ただいま。	
\\	ただいま。 
\\	お帰り。	
\\	お帰り。 
\\	ねえ、ばあちゃん。子供がいなくて寂しいと思うことはないかい。ばあちゃんはずっとほしかったのに、わしがずっとロボットの仕事で忙しくて・・・もうこんな歳になっちゃったなあ。	
\\	ねえ、ばあちゃん。子供がいなくて寂しいと思うことはないかい。ばあちゃんはずっとほしかったのに、わしがずっとロボットの仕事で忙しくて・・・もうこんな歳になっちゃったなあ。 
\\	あなた、何言っているの。あなたとあなたの手で作ったこのロボットたちが私の宝物ですよ。私は今、十分幸せですよ。ほら見て。ここにいる500のロボットたちはみんな家族ですよ。	
\\	あなた、何言っているの。あなたとあなたの手で作ったこのロボットたちが私の宝物ですよ。私は今、十分幸せですよ。ほら見て。ここにいる500のロボットたちはみんな家族ですよ。 
\\	ありがとう。ばあさん。本当にありがとう。ところで、明日はお正月だよ。寿司でも買ってこようか。	
\\	ありがとう。ばあさん。本当にありがとう。ところで、明日はお正月だよ。寿司でも買ってこようか。 
\\	車の中
\\	久しぶりのドライブですね。	
\\	久しぶりのドライブですね。 
\\	そうだね。	
\\	そうだね。 
\\	このレクサスに乗るのは初めて。ルーフを開けて。。。いい気持ち!	
\\	このレクサスに乗るのは初めて。ルーフを開けて。。。いい気持ち! 
\\	ほんとに。	
\\	ほんとに。 
\\	あっ!あれ見て!こんなところで何やってるのかしら。	
\\	あっ!あれ見て!こんなところで何やってるのかしら。 
\\	アイボだ!危ない!	
\\	アイボだ!危ない! 
\\	轢かれた!かわいそうに。	
\\	轢かれた!かわいそうに。 
\\	助けないと...	
\\	助けないと... 
\\	おじいさん、アイボを取ってくる。
\\	大丈夫?	
\\	大丈夫? 
\\	かなりひどいよ。早くうちへ連れて帰らないと。	
\\	かなりひどいよ。早くうちへ連れて帰らないと。 
\\	早く帰りましょう。	
\\	早く帰りましょう。 
\\	もう3日だが、あまり様子が変わらないなあ。全然動かないし。もう駄目かな。	
\\	もう3日だが、あまり様子が変わらないなあ。全然動かないし。もう駄目かな。 
\\	そんなに簡単にあきらめないで。ほら。	
\\	そんなに簡単にあきらめないで。ほら。 
\\	動いた!よかった。本当によかった。じゃ、散歩に行こう。	
\\	動いた!よかった。本当によかった。じゃ、散歩に行こう。 
\\	あなた、まだ早いですよ。ちゃんと治るまで待ちましょう。	
\\	あなた、まだ早いですよ。ちゃんと治るまで待ちましょう。 
\\	そうだな。さすが、ばあさん。	
\\	そうだな。さすが、ばあさん。 
\\	ただいま。	
\\	ただいま。 
\\	2人ともお帰り。	
\\	2人ともお帰り。 
\\	ばあさん、ついに完全に治ったよ!	
\\	ばあさん、ついに完全に治ったよ! 
\\	あなたが直したんでしょう。	
\\	あなたが直したんでしょう。 
\\	まあね。	
\\	まあね。 
\\	せっかく家族になったのに、寂しいけど・・・元気になったから、そろそろ
\\	の店へ返さなきゃ。	
\\	せっかく家族になったのに、寂しいけど・・・元気になったから、そろそろ
\\	の店へ返さなきゃ。 
\\	まだ早すぎるよ。	
\\	まだ早すぎるよ。 
\\	じいさん。	
\\	じいさん。 
\\	はい、分かりました。アイボ、そろそろほんとのお家へ帰ろうか。	
\\	はい、分かりました。アイボ、そろそろほんとのお家へ帰ろうか。 
\\	おばあさんとおじいさんが
\\	に着く
\\	ここが噂の
\\	ビルですか。	
\\	ここが噂の
\\	ビルですか。 
\\	おばあさんとおじいさんが帰ろうとすると、おばあさんが何かに気づく。
\\	店のマネージャーが迷子だったアイボを見つけ、二人に気づく。
\\	ありがとうございました!1週間前に行方不明だったんです。無事に返してくださって、ほんとにありがとうございました。	
\\	ありがとうございました!1週間前に行方不明だったんです。無事に返してくださって、ほんとにありがとうございました。 
\\	いいえ、とんでもないです。	
\\	いいえ、とんでもないです。 
\\	少々お待ちください。はい、これはお礼です。少ないですが、イタリア料理の招待券と
\\	の商品券です。	
\\	少々お待ちください。はい、これはお礼です。少ないですが、イタリア料理の招待券と
\\	の商品券です。 
\\	お礼なんていらないんですよ。アイボのおかげで楽しい時間を過ごしました。	
\\	お礼なんていらないんですよ。アイボのおかげで楽しい時間を過ごしました。 
\\	おや、おかしいですね。アイボたちがみんなでこっちに来てます。こんなの見たことないです。よし!じゃ、みんなで、車までお見送りしましょう。	
\\	おや、おかしいですね。アイボたちがみんなでこっちに来てます。こんなの見たことないです。よし!じゃ、みんなで、車までお見送りしましょう。 
\\	会計
\\	勘定
\\	可能性
\\	御馳走
\\	いる・
\\	いっらしゃる
\\	負担
\\	転換
\\	恩返し
\\	そろそろ行きましょうか。	
\\	そろそろ行きましょうか。 
\\	そうですね。行きましょう。	
\\	そうですね。行きましょう。 
\\	おばあさん、お会計お願いします。	
\\	おばあさん、お会計お願いします。 
\\	ええと、味噌ラーメン二つだから、700円でございます。	
\\	ええ、あんなに美味しかったのに、そんなに安いんですか。信じられないです。ごちそうさまでした。	
\\	ええ、あんなに美味しかったのに、そんなに安いんですか。信じられないです。ごちそうさまでした。 
\\	気に入ってくれてよかった。	
\\	気に入ってくれてよかった。 
\\	(おばささん):おじいさん、いいから!	
\\	また、いっらしゃってね。	
\\	また、いっらしゃってね。 
\\	もちろん。また来ます。	
\\	もちろん。また来ます。 
\\	また彼女もいっしょにね。	
\\	また彼女もいっしょにね。 
\\	さ、どうでしょうね。	
\\	さ、どうでしょうね。 
\\	可能性はなくもないですね。ね、竜太さん。	
\\	可能性はなくもないですね。ね、竜太さん。 
\\	竜太君、頑張ってね。	
\\	竜太君、頑張ってね。 
\\	どうも。気分はどうですか。	
\\	どうも。気分はどうですか。 
\\	すっきりしました。本当にありがとうございます。	
\\	すっきりしました。本当にありがとうございます。 
\\	じゃ、空港でしたっけ?	
\\	じゃ、空港でしたっけ? 
\\	そうですね。	
\\	そうですね。 
\\	どこへ行くんでしたっけ?	
\\	どこへ行くんでしたっけ? 
\\	チューリッヒに帰るつもりでしたけど、今から海外へ行く気分ではなくなりました。あのおばあさんと会って、あのラーメンを食べて、久しぶりに北海道へ帰りたくなりました。	
\\	チューリッヒに帰るつもりでしたけど、今から海外へ行く気分ではなくなりました。あのおばあさんと会って、あのラーメンを食べて、久しぶりに北海道へ帰りたくなりました。 
\\	いいな〜。私も帰りたいな〜。でも出稼ぎ中だし。	
\\	いいな〜。私も帰りたいな〜。でも出稼ぎ中だし。 
\\	じゃ、この際一緒に行きませんか。	
\\	じゃ、この際一緒に行きませんか。 
\\	わ、私ですか。しかし、仕事がありますし。お金もなくて。	
\\	わ、私ですか。しかし、仕事がありますし。お金もなくて。 
\\	あなたのおかげで、私の気分が転換できて、何か恩返ししたいんです。せっかく同じ地元じゃないですか。	
\\	あ、ありがとうございます!行きましょう!	
\\	あ、ありがとうございます!行きましょう! 
\\	部長
\\	ビジネス
\\	まさか
\\	さっき
\\	頼む
\\	実は
\\	戻る
\\	辞める
\\	特別
\\	もらう); 
\\	いただく
\\	ただいま。帰りました。	
\\	ただいま。帰りました。 
\\	遅いぞ。何してた?	
\\	遅いぞ。何してた? 
\\	すみません部長、特別なお客様がいまして。	
\\	すみません部長、特別なお客様がいまして。 
\\	なんだそりゃ!特別な客?客に特別も何もないだろ。これはビジネスだ!	
\\	なんだそりゃ!特別な客?客に特別も何もないだろ。これはビジネスだ! 
\\	すみませんでした。	
\\	すみませんでした。 
\\	二度とないように気をつけろ。	
\\	二度とないように気をつけろ。 
\\	はい、わかりました。	
\\	はい、わかりました。 
\\	早く仕事戻れ。	
\\	早く仕事戻れ。 
\\	実は、ちょっと話があります。	
\\	実は、ちょっと話があります。 
\\	話って何だ?	
\\	話って何だ? 
\\	ここでお世話になり始めてもう3年です。	
\\	ここでお世話になり始めてもう3年です。 
\\	辞めるとか言うつもりか?無理だよ。君はうちの一番いいドライバーだから。さっきは悪かった。すまん。たのむ、辞めないでくれ!	
\\	辞めるとか言うつもりか?無理だよ。君はうちの一番いいドライバーだから。さっきは悪かった。すまん。たのむ、辞めないでくれ! 
\\	まさか。ただ、ここに入ってから、まだ一度も休んでいません。明日から1週間休みをいただきたいんですが。	
\\	まさか。ただ、ここに入ってから、まだ一度も休んでいません。明日から1週間休みをいただきたいんですが。 
\\	ああ、そういうことか。いいけど。でも今回だけだぞ。で、何をするつもりなんだ?	
\\	ああ、そういうことか。いいけど。でも今回だけだぞ。で、何をするつもりなんだ? 
\\	北海道へ帰って来ます。	
\\	北海道へ帰って来ます。 
\\	よし、分かった。はい、今からスタートな。	
\\	よし、分かった。はい、今からスタートな。 
\\	急ぐ
\\	焼きたて
\\	短気
\\	性格
\\	思いやりのある
\\	最終案内
\\	間に合う
\\	お待たせしました。	
\\	お待たせしました。 
\\	あー間に合いました。もう名前呼ばれたから、あまり時間がないですよ。出発まで10分ですから、急がないと。	
\\	あー間に合いました。もう名前呼ばれたから、あまり時間がないですよ。出発まで10分ですから、急がないと。 
\\	ごめんなさい。おみやげです。東京で一番うまいパン屋さんによって来ました。焼きたてです。	
\\	ごめんなさい。おみやげです。東京で一番うまいパン屋さんによって来ました。焼きたてです。 
\\	パン屋!パン屋に行ってて遅れたんですか!	
\\	パン屋!パン屋に行ってて遅れたんですか! 
\\	そうです。智子さんのご家族のお土産を買って来ました。	
\\	そうです。智子さんのご家族のお土産を買って来ました。 
\\	私の家族ですか。	
\\	私の家族ですか。 
\\	もちろん。智子さんが買う暇がないと思って、私は買いにいきました。	
\\	もちろん。智子さんが買う暇がないと思って、私は買いにいきました。 
\\	すみませんでした。短気な性格でごめんなさい。竜太さんみたいな思いやりのある人と会うのは久しぶりです。	
\\	すみませんでした。短気な性格でごめんなさい。竜太さんみたいな思いやりのある人と会うのは久しぶりです。 
\\	いいえ、こちらこそ。東京に来てから、さとこさんみたいなやさしい人は初めてです。	
\\	いいえ、こちらこそ。東京に来てから、さとこさんみたいなやさしい人は初めてです。 
\\	札幌行き321便でご出発の高橋さん、最終案内です。	
\\	札幌行き321便でご出発の高橋さん、最終案内です。 
\\	私達だ。	
\\	私達だ。 
\\	間に合うかな?	
\\	間に合うかな? 
\\	走ろう。	
\\	走ろう。 
\\	当機
\\	当社
\\	当人
\\	離陸
\\	着陸
\\	貿易
\\	機関
\\	世界貿易機関
\\	政府機関
\\	当店
\\	便
\\	急に
\\	冷静に
\\	記憶
\\	ゆがむ
\\	当時
\\	魔法
\\	何となく
\\	落ち着く
\\	メモリー
\\	札幌行き321便はまもなく札幌空港へ着陸致します。現在の札幌の天候は晴れ。気温は二十度となっております。またのご利用を心からお待ちしております。	
\\	札幌行き321便はまもなく札幌空港へ着陸致します。現在の札幌の天候は晴れ。気温は二十度となっております。またのご利用を心からお待ちしております。 
\\	理由がわからないけど、急にどきどきしてきました。	
\\	理由がわからないけど、急にどきどきしてきました。 
\\	大丈夫ですか。	
\\	大丈夫ですか。 
\\	おかしいです。すごくおかしいです。	
\\	おかしいです。すごくおかしいです。 
\\	何が?	
\\	何が? 
\\	こんな気分久しぶりです。私は普段世界貿易機関で冷静に話せるのに、なんで地元のことを考えると緊張するんでしょう。	
\\	こんな気分久しぶりです。私は普段世界貿易機関で冷静に話せるのに、なんで地元のことを考えると緊張するんでしょう。 
\\	まあ、人生とはそういうものです。時間が経つほど、記憶がゆがんでいく。	
\\	まあ、人生とはそういうものです。時間が経つほど、記憶がゆがんでいく。 
\\	どういうこと?	
\\	どういうこと? 
\\	心が記憶を少しずつ変えていく。例えば、昔の彼女がそうです。当時、彼女はすごくきれいでした。別れてから全然会っていません。頭の中で時間が止まっているから、彼女はずっと美しいんです。	
\\	心が記憶を少しずつ変えていく。例えば、昔の彼女がそうです。当時、彼女はすごくきれいでした。別れてから全然会っていません。頭の中で時間が止まっているから、彼女はずっと美しいんです。 
\\	婚約者
\\	相談
\\	ぴったり
\\	タイプ
\\	副社長
\\	意気投合
\\	表参道
\\	(電車の案内)間もなくドアが閉まります。ご注意ください。	
\\	(電車の案内)間もなくドアが閉まります。ご注意ください。 
\\	ああ、ぎりぎり間に合った。	
\\	ああ、ぎりぎり間に合った。 
\\	よかった。まさか!こんなところで!	
\\	よかった。まさか!こんなところで! 
\\	誰?	
\\	誰? 
\\	大学の同級生。陽子ちゃん。いつも話してたじゃない。	
\\	大学の同級生。陽子ちゃん。いつも話してたじゃない。 
\\	すごい。あの、陽子ちゃん?	
\\	すごい。あの、陽子ちゃん? 
\\	元気そうに見えないけど、大丈夫かな。陽子ちゃん。	
\\	元気そうに見えないけど、大丈夫かな。陽子ちゃん。 
\\	さゆりちゃん!こんなところで!	
\\	さゆりちゃん!こんなところで! 
\\	本当。陽子ちゃん、元気?	
\\	本当。陽子ちゃん、元気? 
\\	あんまり。	
\\	あんまり。 
\\	なんかあった?	
\\	なんかあった? 
\\	まあ、最近いろいろ。親友がいなくなったり、仕事がなくなったり。かなり大変だったんだ。	
\\	まあ、最近いろいろ。親友がいなくなったり、仕事がなくなったり。かなり大変だったんだ。 
\\	相談すればよかったのに。力になれたかも。	
\\	相談すればよかったのに。力になれたかも。 
\\	まあ、でも、私のことはどうでもいいよ。さゆりちゃんは?	
\\	まあ、でも、私のことはどうでもいいよ。さゆりちゃんは? 
\\	かなりいい感じ。私達、今、表参道へ向かってるとこ。今日は婚約者の彼と一緒に部屋を探しに行くの。	
\\	かなりいい感じ。私達、今、表参道へ向かってるとこ。今日は婚約者の彼と一緒に部屋を探しに行くの。 
\\	婚約者!	
\\	婚約者! 
\\	初めまして。谷ヒロと申します。どうぞよろしくお願いします。よくさゆりから、話を聞いています。	
\\	初めまして。谷ヒロと申します。どうぞよろしくお願いします。よくさゆりから、話を聞いています。 
\\	どうも。初めまして。素敵!ぴったり!しかもあたしのタイプ!うらやましい。どこで知り合ったの?	
\\	どうも。初めまして。素敵!ぴったり!しかもあたしのタイプ!うらやましい。どこで知り合ったの? 
\\	仕事で。彼は副社長だよ。	
\\	仕事で。彼は副社長だよ。 
\\	いいな〜。無事に卒業した?	
\\	いいな〜。無事に卒業した? 
\\	うん、危なかったけどね。私その日からずっとついてるみたい。実は、この仕事、決まってた人がいたんだけど、その人が来なかったから、私が入って、ヒロと出会って、あっというまに意気投合。	
\\	うん、危なかったけどね。私その日からずっとついてるみたい。実は、この仕事、決まってた人がいたんだけど、その人が来なかったから、私が入って、ヒロと出会って、あっというまに意気投合。 
\\	会社はほとんど男だったから、さゆりが入ってくれて、本当によかったです。 
\\	どこの会社?	
\\	どこの会社? 
\\	じゃなくて、
\\	っていう会社。	
\\	じゃなくて、
\\	っていう会社。 
\\	え、
\\	知ってる?ね、陽子ちゃん、どうかした?なんか言ってよ。	
\\	知ってる?ね、陽子ちゃん、どうかした?なんか言ってよ。 
\\	飼う
\\	どこからともなく
\\	別れ別れ
\\	困り者
\\	再会
\\	天の川
\\	水かさ
\\	待ちに待った
\\	一目見る
\\	あまりにも
\\	カササギ
\\	迎える
\\	祝福する
\\	年頃
\\	久しぶりに札幌の空気を吸った。本当にきれいですね。	
\\	久しぶりに札幌の空気を吸った。本当にきれいですね。 
\\	そうですね。やっぱり、北海道がいいです。	
\\	そうですね。やっぱり、北海道がいいです。 
\\	でも、今10時です。稚内までの電車がなくなった。どうしよう?	
\\	でも、今10時です。稚内までの電車がなくなった。どうしよう? 
\\	よかった。実は、行く前に、見せたいところある。僕を信用している?	
\\	よかった。実は、行く前に、見せたいところある。僕を信用している? 
\\	まあ。	
\\	まあ。 
\\	じゃ、行きましょう。	
\\	じゃ、行きましょう。 
\\	はい、到着。降りましょう。	
\\	はい、到着。降りましょう。 
\\	星がきれい!	
\\	星がきれい! 
\\	今日は何の日でしょう?	
\\	今日は何の日でしょう? 
\\	ああ!七夕だ!	
\\	ああ!七夕だ! 
\\	そうだよ。	
\\	そうだよ。 
\\	あ!天の川が見える!あの物語、なつかしいわ。覚えてる?	
\\	あ!天の川が見える!あの物語、なつかしいわ。覚えてる? 
\\	むかしむかし、天の川のそばには天の神さまが住んでいました。	
\\	むかしむかし、天の川のそばには天の神さまが住んでいました。 
\\	すごい!	
\\	すごい! 
\\	天の神さまには、一人の娘がいました。名前を	
\\	天の神さまには、一人の娘がいました。名前を 
\\	おり姫と言いました。	
\\	おり姫と言いました。 
\\	おっ!	
\\	おっ! 
\\	おり姫ははたをおって、神さまたちの着物をつくる仕事をしていました。おり姫がやがて年頃になり、天の神さまは娘に、おむこさんをむかえてやろうと思いました。	
\\	おり姫ははたをおって、神さまたちの着物をつくる仕事をしていました。おり姫がやがて年頃になり、天の神さまは娘に、おむこさんをむかえてやろうと思いました。 
\\	やるね。ひこぼしは、とても立派な若者でした。おり姫も、かがやくばかりに美しい娘です。	
\\	やるね。ひこぼしは、とても立派な若者でした。おり姫も、かがやくばかりに美しい娘です。 
\\	二人は結婚して、楽しい生活を送るようになりました。でも、なかが良すぎるのも困りもので、二人は仕事を忘れて、遊んでばかりいるようになったのです。	
\\	二人は結婚して、楽しい生活を送るようになりました。でも、なかが良すぎるのも困りもので、二人は仕事を忘れて、遊んでばかりいるようになったのです。 
\\	「おり姫がはたおりをしないので、みんなの着物が古くてボロボロです。はやく新しい着物をつくってください」	
\\	「おり姫がはたおりをしないので、みんなの着物が古くてボロボロです。はやく新しい着物をつくってください」 
\\	「ひこぼしが世話をしないので、ウシたちが病気になってしまいます」	
\\	「ひこぼしが世話をしないので、ウシたちが病気になってしまいます」 
\\	天の神さまに、みんなが文句を言いに来るようになりました。神さまは、すっかり怒ってしまい、	
\\	天の神さまに、みんなが文句を言いに来るようになりました。神さまは、すっかり怒ってしまい、 
\\	でも天の神さまは、おり姫があまりにも悲しそうにしているのを見て、こういいました。	
\\	でも天の神さまは、おり姫があまりにも悲しそうにしているのを見て、こういいました。 
\\	このまま
\\	構わない
\\	ゆっくり
\\	ぐずぐずする
\\	見抜く
\\	ベルト
\\	送る
\\	慣れる
\\	運転手
\\	本当に楽しかった。今夜どうする?かなり遅くなった。	
\\	本当に楽しかった。今夜どうする?かなり遅くなった。 
\\	そうですね。じゃ、寝ないでこのまま稚内へ帰る?	
\\	そうですね。じゃ、寝ないでこのまま稚内へ帰る? 
\\	でも、車だと竜太はずっと運転してリラックスできないから。電車の方が楽でなにも考えなくていい。	
\\	でも、車だと竜太はずっと運転してリラックスできないから。電車の方が楽でなにも考えなくていい。 
\\	僕は全然かまわない。運転することと考えることが好きだし。全然問題ない。慣れてるよ。毎日お客さんを遅くまで送っているから。本当に平気。	
\\	僕は全然かまわない。運転することと考えることが好きだし。全然問題ない。慣れてるよ。毎日お客さんを遅くまで送っているから。本当に平気。 
\\	あたしはお客さんじゃない。	
\\	あたしはお客さんじゃない。 
\\	もちろん。	
\\	もちろん。 
\\	ゆっくりしたいの。最近、最後にゆっくりしたのはいつ?	
\\	ゆっくりしたいの。最近、最後にゆっくりしたのはいつ? 
\\	まあ、	
\\	まあ、 
\\	決まり。ゆっくり行きましょう。	
\\	決まり。ゆっくり行きましょう。 
\\	さとこ、いくらぐずぐずしてもいつかは帰らないといけないよ。	
\\	さとこ、いくらぐずぐずしてもいつかは帰らないといけないよ。 
\\	ね、竜太、ちょっと気になってるんだけど、	
\\	ね、竜太、ちょっと気になってるんだけど、 
\\	何?	
\\	何? 
\\	私はたくさんの優秀な人と出会ったのに、竜太みたいな人ははじめて。タクシーの運転手さんらしくない。	
\\	私はたくさんの優秀な人と出会ったのに、竜太みたいな人ははじめて。タクシーの運転手さんらしくない。 
\\	今まで何人のタクシー運転手と話したことがある?	
\\	今まで何人のタクシー運転手と話したことがある? 
\\	ほら、また深い話をする。こんなに鋭くて変わっている人、なにか裏があるみたい。絶対見抜いてみせる。	
\\	ほら、また深い話をする。こんなに鋭くて変わっている人、なにか裏があるみたい。絶対見抜いてみせる。 
\\	見抜いてみてください。その間どうしよう?	
\\	見抜いてみてください。その間どうしよう? 
\\	じゃ、帰ろうか。	
\\	じゃ、帰ろうか。 
\\	了解!ベルトしてください。行きましょう。	
\\	了解!ベルトしてください。行きましょう。 
\\	浜辺
\\	玉手箱
\\	見あたる
\\	すら
\\	すれ違う
\\	老人
\\	切り
\\	本の
\\	もくもくと
\\	生き物
\\	年老いる
\\	亀
\\	叩く
\\	蹴る
\\	竜宮城
\\	立派(な)
\\	乙姫
\\	大喜び
\\	恋しい
\\	すっかり
\\	ごらん
\\	深呼吸
\\	なんて
\\	行ってくる
\\	お手洗い
\\	浅呼吸
\\	辛い
\\	事件
\\	ね、着いたよ。稚内!	
\\	ね、着いたよ。稚内! 
\\	今何時?	
\\	今何時? 
\\	7時半。朝ごはんを食べよう。	
\\	7時半。朝ごはんを食べよう。 
\\	ね、あの店まだあるね。懐かしいな~。そこで食べよう。	
\\	ね、あの店まだあるね。懐かしいな~。そこで食べよう。 
\\	あの店?	
\\	あの店? 
\\	うん。	
\\	うん。 
\\	(ドア開く音)	
\\	ちょっとお手洗いに行ってくる。	
\\	ちょっとお手洗いに行ってくる。 
\\	竜太!竜太なの?!	
\\	竜太!竜太なの?! 
\\	お久しぶりです。	
\\	お久しぶりです。 
\\	竜太がこの街に戻ってくると思わなかった。あの事件の後、急にいなくなって・・・どうしたの?どこにいってたの?何考えてるの?辛かったのは自分だけと思っているの?いなくなっちゃうなんて!	
\\	竜太がこの街に戻ってくると思わなかった。あの事件の後、急にいなくなって・・・どうしたの?どこにいってたの?何考えてるの?辛かったのは自分だけと思っているの?いなくなっちゃうなんて! 
\\	落ち着いて落ち着いて!分かった、分かったから。深呼吸してごらん。	
\\	落ち着いて落ち着いて!分かった、分かったから。深呼吸してごらん。 
\\	とにかく、あなたに会いたがっている人がいっぱいいるよ。行こう!	
\\	とにかく、あなたに会いたがっている人がいっぱいいるよ。行こう! 
\\	分かった。ちょっと待ってね。友達にメモを書く。彼女にこのメモを。よろしくお願いします。	
\\	分かった。ちょっと待ってね。友達にメモを書く。彼女にこのメモを。よろしくお願いします。 
\\	行こう。	
\\	行こう。 
\\	誤解
\\	過去
\\	突然
\\	向き合う
\\	正直
\\	手助けをする
\\	〜頃
\\	すみません。私の友達を知りませんか。	
\\	すみません。私の友達を知りませんか。 
\\	あ、竜太君?	
\\	あ、竜太君? 
\\	どうして知っているんですか。	
\\	どうして知っているんですか。 
\\	この町で竜太君を知らない人はいないよ。	
\\	この町で竜太君を知らない人はいないよ。 
\\	え。どういうこと?	
\\	え。どういうこと? 
\\	多分この手紙を見れば、分かるんじゃないかな。	
\\	多分この手紙を見れば、分かるんじゃないかな。 
\\	庭
\\	当然
\\	会わせる
\\	顔を合わせる
\\	ご存知
\\	御陰様で
\\	石庭
\\	誇り
\\	はい、ここでいいです。	
\\	はい、ここでいいです。 
\\	二千六百円でございます。どうも有り難うございました。	
\\	二千六百円でございます。どうも有り難うございました。 
\\	只今!お母さん、お父さん、皆出かけてるの?	
\\	こんにちは。	
\\	こんにちは。 
\\	ああ、どうも。こんにちは。	
\\	ああ、どうも。こんにちは。 
\\	隣の鈴木と申します。誰かをお探しですか。	
\\	隣の鈴木と申します。誰かをお探しですか。 
\\	はい、ここに住んでいる高橋ですが。	
\\	はい、ここに住んでいる高橋ですが。 
\\	ああ、智子ちゃん。お帰りなさい。	
\\	ああ、智子ちゃん。お帰りなさい。 
\\	えっ、何でわかるんですか。	
\\	えっ、何でわかるんですか。 
\\	当然ですよ。あなたがここにいた頃は、毎日顔合わせていましたよ。残念ですが、今は誰もこの家に住んでいませんよ。	
\\	当然ですよ。あなたがここにいた頃は、毎日顔合わせていましたよ。残念ですが、今は誰もこの家に住んでいませんよ。 
\\	どういうこと?何があったの?教えてください!	
\\	どういうこと?何があったの?教えてください! 
\\	長い話になります。お茶を飲みながら、ゆっくり話してあげます。私の家に行きましょう
\\	長い話になります。お茶を飲みながら、ゆっくり話してあげます。私の家に行きましょう
\\	只今。	
\\	只今。 
\\	ご無沙汰しています。	
\\	ご無沙汰しています。 
\\	あら、智子ちゃん?お久しぶりです。元気にしてた?	
\\	あら、智子ちゃん?お久しぶりです。元気にしてた? 
\\	はい御陰様で。	
\\	はい御陰様で。 
\\	大人の女性になったわね。国際弁護士の仕事はどう?	
\\	大人の女性になったわね。国際弁護士の仕事はどう? 
\\	ええ、何でそんなことをご存知なんですか。 
\\	智子ちゃんはこの町の誇りよ。智子ちゃんがやっていることは皆知ってる。	
\\	智子ちゃんはこの町の誇りよ。智子ちゃんがやっていることは皆知ってる。 
\\	本当ですか。	
\\	本当ですか。 
\\	本当よ。	
\\	本当よ。 
\\	有り難うございます。だけど、鈴木さん、お願いします。私の家族はどこへ?	
\\	有り難うございます。だけど、鈴木さん、お願いします。私の家族はどこへ? 
\\	私はお茶を持ってきます。	
\\	私はお茶を持ってきます。 
\\	智子ちゃん	
\\	智子ちゃん 
\\	ええ、何で泣いているんですか。	
\\	ええ、何で泣いているんですか。 
\\	悲しい知らせがあるのよ。連絡しようとしたんだけど。連絡が取れなくて。	
\\	悲しい知らせがあるのよ。連絡しようとしたんだけど。連絡が取れなくて。 
\\	家族は?	
\\	家族は? 
\\	食堂
\\	リハビリ
\\	責める
\\	怪我
\\	仕送り
\\	済む
\\	姉さん、久しぶり。	
\\	姉さん、久しぶり。 
\\	竜太!いつ帰ってきたの?	
\\	竜太!いつ帰ってきたの? 
\\	今朝。食堂に入ったら、さくらと会って、すぐここに来た。調子はどう?	
\\	今朝。食堂に入ったら、さくらと会って、すぐここに来た。調子はどう? 
\\	良くなってきた。 
\\	お金はちゃんと届いてる?	
\\	お金はちゃんと届いてる? 
\\	うん。ありがとう。	
\\	うん。ありがとう。 
\\	リハビリはうまくいっている?	
\\	リハビリはうまくいっている? 
\\	うん。がんばってる。竜太、なんであの事故の後、すぐ町を出たの?竜太が必要だったのに。	
\\	うん。がんばってる。竜太、なんであの事故の後、すぐ町を出たの?竜太が必要だったのに。 
\\	ごめん。本当にごめん。でも、お金が必要だと思ったから。沢山必要ってわかっていて。	
\\	ごめん。本当にごめん。でも、お金が必要だと思ったから。沢山必要ってわかっていて。 
\\	あの事故は竜太のせいじゃないよ。自分を責めるのをやめないと・・・	
\\	あの事故は竜太のせいじゃないよ。自分を責めるのをやめないと・・・ 
\\	でも、僕が運転していたら、きっと皆は大丈夫だった。お父さんとお母さんもあの家族も死んでなかったし、姉さんもけがをせずにすんだ。	
\\	でも、僕が運転していたら、きっと皆は大丈夫だった。お父さんとお母さんもあの家族も死んでなかったし、姉さんもけがをせずにすんだ。 
\\	いえ、もうやめましょう。済んだことを言っても仕方がないわ。あなたは一生懸命私のために仕送りしてくれているのに。ごめんね。	
\\	いえ、もうやめましょう。済んだことを言っても仕方がないわ。あなたは一生懸命私のために仕送りしてくれているのに。ごめんね。 
\\	いいよ、姉さんは早く良くなることだけを考えて。もうどこにも行かないから、安心して。	
\\	いいよ、姉さんは早く良くなることだけを考えて。もうどこにも行かないから、安心して。 
\\	暴走族
\\	豪邸
\\	露天風呂
\\	追い出す
\\	仲間
\\	覗く
\\	息の下
\\	ぶっ殺す
\\	解ける
\\	仕切っている
\\	害する
\\	腰掛ける
\\	乱暴
\\	気分を害する
\\	お墓
\\	天国
\\	大変
\\	慰める
\\	抱きかかえる
\\	事故
\\	一人前
\\	親孝行
\\	小屋
\\	和尚
\\	お知り合い
\\	すみません、子供のとき以来お墓参りしていないので、どうしたらいいのかわからなくて。教えていただけますか?	
\\	すみません、子供のとき以来お墓参りしていないので、どうしたらいいのかわからなくて。教えていただけますか? 
\\	あ、そうですか。まず、お墓をきれいにお掃除してあげてください。それからお花とお線香をあげてあげたらいいでしょう。	
\\	あ、そうですか。まず、お墓をきれいにお掃除してあげてください。それからお花とお線香をあげてあげたらいいでしょう。 
\\	はい、わかりました。	
\\	はい、わかりました。 
\\	バケツや何かは、あの小屋にある物を使ってください。	
\\	バケツや何かは、あの小屋にある物を使ってください。 
\\	どうも、ありがとうございます。	
\\	どうも、ありがとうございます。 
\\	お父さん、お母さん、どうして
\\	お父さん、お母さん、どうして
\\	私、何も親孝行出来なくてごめんね。やっと、一人前になって、これからだと思って帰ってきたのに
\\	お父さんとお母さんが事故にあった事も知らずに
\\	私、何も親孝行出来なくてごめんね。やっと、一人前になって、これからだと思って帰ってきたのに
\\	お父さんとお母さんが事故にあった事も知らずに
\\	あー、智子ちゃんか。覚えてないだろうな。あの頃はまだこんなに小さかったから。むかしよくお父さんとお母さんとお盆のお墓参りに来ていたんだが。こんなに大きくなって。	
\\	あー、智子ちゃんか。覚えてないだろうな。あの頃はまだこんなに小さかったから。むかしよくお父さんとお母さんとお盆のお墓参りに来ていたんだが。こんなに大きくなって。 
\\	あの頃の事は何となく覚えています。私がお墓に来るのが怖くて泣いているのを、父が抱きかかえてくれて、母が「智子が来たからおじいちゃんもおばあちゃんも喜んでいるわよ。」って慰めてくれて。	
\\	"あの頃の事は何となく覚えています。私がお墓に来るのが怖くて泣いているのを、父が抱きかかえてくれて、母が「智子が来たからおじいちゃんもおばあちゃんも喜んでいるわよ。」って慰めてくれて。 
\\	この度はいろいろ、大変だったね。でもお父さんとお母さんは毎年ここへ来ては智子ちゃんの事を自慢に話してくれていたよ。	
\\	この度はいろいろ、大変だったね。でもお父さんとお母さんは毎年ここへ来ては智子ちゃんの事を自慢に話してくれていたよ。 
\\	本当ですか?	
\\	本当ですか? 
\\	ああ、世界一の娘だって。本当にこんなに立派になって。天国のお父さんもお母さんも心配いらないな。	
\\	ああ、世界一の娘だって。本当にこんなに立派になって。天国のお父さんもお母さんも心配いらないな。 
\\	でも、私これから一人でどうしたらいいのか。	
\\	でも、私これから一人でどうしたらいいのか。 
\\	智子ちゃん、智子ちゃんのお父さんとお母さんは、天国へ行ってもいつも智子ちゃんと一緒だよ。辛いけど自信を持って生きていきなさい。	
\\	智子ちゃん、智子ちゃんのお父さんとお母さんは、天国へ行ってもいつも智子ちゃんと一緒だよ。辛いけど自信を持って生きていきなさい。 
\\	ありがとうございます。なんだか気分が楽になりました。私がんばります。天国にいるお父さんとお母さんのためにも。	
\\	ありがとうございます。なんだか気分が楽になりました。私がんばります。天国にいるお父さんとお母さんのためにも。 
\\	おや、あの方もお知り合いですか。	
\\	おや、あの方もお知り合いですか。 
\\	竜太さん!	
\\	竜太さん! 
\\	残酷な
\\	事実
\\	途中
\\	気づく
\\	黙る
\\	反応
\\	治る
\\	詫び
\\	何言えばいいか
\\	何言えばいいか
\\	なんて残酷な世界
\\	最初から知っていたの?	
\\	なんて残酷な世界
\\	最初から知っていたの? 
\\	いや、そんなことはない。最初は知らなかった。それは事実だよ。	
\\	いや、そんなことはない。最初は知らなかった。それは事実だよ。 
\\	でも、途中で気づいてたでしょう。でしょう!	
\\	でも、途中で気づいてたでしょう。でしょう! 
\\	それは
\\	それは
\\	何で言わなかったの。何で黙ってたの?何で!	
\\	何で言わなかったの。何で黙ってたの?何で! 
\\	落ち着いて、落ち着いて。	
\\	落ち着いて、落ち着いて。 
\\	こんな広い世界で何であなたと出会ってしまったの??何で黙っていたの。知っていたんでしょう?	
\\	こんな広い世界で何であなたと出会ってしまったの??何で黙っていたの。知っていたんでしょう? 
\\	本当にごめんなさい。時間が経てば経つほど僕の、僕が
\\	つまり、言いづらくなって、だって、智子はあんなに幸せで
\\	正直いうとわからない。	
\\	本当にごめんなさい。時間が経てば経つほど僕の、僕が
\\	つまり、言いづらくなって、だって、智子はあんなに幸せで
\\	正直いうとわからない。 
\\	もう聞きたくない。	
\\	もう聞きたくない。 
\\	智子、君は僕が何言っても、何て説明しても、絶対に聞かなかった。	
\\	智子、君は僕が何言っても、何て説明しても、絶対に聞かなかった。 
\\	そんなことはわからないでしょ。	
\\	そんなことはわからないでしょ。 
\\	こういう場合、僕が何言っても、智子は怒っていた。それは当然な反応だ。今は確かにつらいけど、知ったばかりだから。時間が経てば、傷は治るんだ。	
\\	こういう場合、僕が何言っても、智子は怒っていた。それは当然な反応だ。今は確かにつらいけど、知ったばかりだから。時間が経てば、傷は治るんだ。 
\\	僕は毎日この事故を防ぐことができたと思い返すんだ。その事実からは逃げられない。僕はお詫びとさようならを言いにきた。お元気で。	
\\	僕は毎日この事故を防ぐことができたと思い返すんだ。その事実からは逃げられない。僕はお詫びとさようならを言いにきた。お元気で。 
\\	くだらない
\\	如し
\\	矢
\\	光陰
\\	縁がない
\\	縁がある
\\	光陰矢の如し
\\	いなくなってしまう
\\	いつの間にか
\\	縁
\\	はい、陽子です。	
\\	はい、陽子です。 
\\	智子ですけど。	
\\	智子ですけど。 
\\	え、智子、大丈夫?どうして電話くれなかったの?	
\\	え、智子、大丈夫?どうして電話くれなかったの? 
\\	今、稚内にいるの。	
\\	今、稚内にいるの。 
\\	え、もどったの?どうして?仕事は?	
\\	え、もどったの?どうして?仕事は? 
\\	もう戻らない。	
\\	もう戻らない。 
\\	え、どうして?	
\\	え、どうして? 
\\	陽子、ごめんなさい。あんなくだらない事で怒ったりして。本当にごめんなさい。	
\\	陽子、ごめんなさい。あんなくだらない事で怒ったりして。本当にごめんなさい。 
\\	智子、大丈夫?	
\\	智子、大丈夫? 
\\	いつの間にか皆いなくなってしまう。光陰矢の如しね。	
\\	いつの間にか皆いなくなってしまう。光陰矢の如しね。 
\\	一体どうしたの?なんだか怖いわ。	
\\	一体どうしたの?なんだか怖いわ。 
\\	ごめん、いろいろあったから。でも、私は大丈夫よ
\\	ただ、陽子の声を聞きたかったの。いつ最後になるかわからないから。	
\\	ごめん、いろいろあったから。でも、私は大丈夫よ
\\	ただ、陽子の声を聞きたかったの。いつ最後になるかわからないから。 
\\	この2週間でいろんな事を学んで、陽子の事がどれだけ大切か分かった。そういう訳で謝りたかったの。許してくれるかな?	
\\	この2週間でいろんな事を学んで、陽子の事がどれだけ大切か分かった。そういう訳で謝りたかったの。許してくれるかな? 
\\	もちろん。私は何にも気にしてないよ。	
\\	もちろん。私は何にも気にしてないよ。 
\\	私はこれからしばらく稚内に残るけど、ちょっと紹介したい人がいるの。	
\\	私はこれからしばらく稚内に残るけど、ちょっと紹介したい人がいるの。 
\\	誰?	
\\	誰? 
\\	彼の名前は竜太。私には縁がなかったけど、あなたにならピッタリだと思うわ。	
\\	彼の名前は竜太。私には縁がなかったけど、あなたにならピッタリだと思うわ。 
\\	え、どういうこと? 
\\	もう行かなくちゃ。	
\\	もう行かなくちゃ。 
\\	模様
\\	初対面
\\	偶然
\\	固まる
\\	虹
\\	不思議
\\	お待たせしてすみません。	
\\	お待たせしてすみません。 
\\	いいえ、私も今来たところです。本当にごめんなさい、初対面なのに急に呼び出したりして。	
\\	いいえ、私も今来たところです。本当にごめんなさい、初対面なのに急に呼び出したりして。 
\\	いいんですよ。最近仕事をくびになって、暇してたところなんで。	
\\	いいんですよ。最近仕事をくびになって、暇してたところなんで。 
\\	え、そうなんですか。私も今仕事がなくて。何か偶然ですね。	
\\	え、そうなんですか。私も今仕事がなくて。何か偶然ですね。 
\\	そうですね。	
\\	そうですね。 
\\	最近何もかもがうまく行かなくて。	
\\	最近何もかもがうまく行かなくて。 
\\	そういう事もありますよ。でも、雨降って地固まるって言うじゃないですか。	
\\	"そういう事もありますよ。でも、雨降って地固まるって言うじゃないですか。 
\\	それもそうね。	
\\	それもそうね。 
\\	それに奇麗な虹だって雨の後にしか出ませんからね。	
\\	それに奇麗な虹だって雨の後にしか出ませんからね。 
\\	竜太さんてすごい、ロマンチックですね。	
\\	竜太さんてすごい、ロマンチックですね。 
\\	ははっ、はずかしいな。	
\\	ははっ、はずかしいな。 
\\	笑顔もすてき。ねえ、今日どうしましょうか?	
\\	笑顔もすてき。ねえ、今日どうしましょうか? 
\\	代々木公園でピクニックはどうですか!材料を買いに行って。	
\\	代々木公園でピクニックはどうですか!材料を買いに行って。 
\\	でも、天気予報によると今日は午後から雨だそうです。	
\\	でも、天気予報によると今日は午後から雨だそうです。 
\\	そうなんですか。	
\\	そうなんですか。 
\\	あの
\\	あの
\\	どうぞ。	
\\	どうぞ。 
\\	なんでだろう。初対面なのにこんなに
\\	何て言うか
\\	なんでだろう。初対面なのにこんなに
\\	何て言うか
\\	ええ
\\	不思議ですよね。そうだ、実はこの近くにすごくおいしいラーメン屋さんがあるんです。東京で一番汚いけど一番おいしいラーメン屋です。	
\\	ええ
\\	不思議ですよね。そうだ、実はこの近くにすごくおいしいラーメン屋さんがあるんです。東京で一番汚いけど一番おいしいラーメン屋です。 
\\	はい、ぜひ。	
\\	はい、ぜひ。 
\\	グラフィックデザイナー
\\	珍しい
\\	フィリピン人
\\	名刺
\\	企業
\\	大ファン
\\	挑戦
\\	一人暮らし
\\	さあ、もう行かないと電車に遅れちゃう。	
\\	さあ、もう行かないと電車に遅れちゃう。 
\\	東京で1人暮らしかー。私に出来るかな?でも、福島の田舎で終わる訳にはいかないもんね。やっと決心したんだもん。頑張らないと。私、ファイト!	
\\	東京で1人暮らしかー。私に出来るかな?でも、福島の田舎で終わる訳にはいかないもんね。やっと決心したんだもん。頑張らないと。私、ファイト! 
\\	えーと、六列目の右から2番目っと。お、ここか。失礼しますよ。	
\\	えーと、六列目の右から2番目っと。お、ここか。失礼しますよ。 
\\	はい、どうぞ。	
\\	はい、どうぞ。 
\\	いやー、やっぱり電車の旅はいいねー。おや、こんなに大きなカバンをもって、お嬢さんどちらまで?	
\\	いやー、やっぱり電車の旅はいいねー。おや、こんなに大きなカバンをもって、お嬢さんどちらまで? 
\\	東京です。仕事を探しに。	
\\	東京です。仕事を探しに。 
\\	どんなお仕事ですか。	
\\	どんなお仕事ですか。 
\\	グラフィックデザイナーです。	
\\	グラフィックデザイナーです。 
\\	それはいい。若いうちに何でも挑戦しないと。	
\\	それはいい。若いうちに何でも挑戦しないと。 
\\	私そんなに若くないですよ。	
\\	私そんなに若くないですよ。 
\\	はっはっは、私の年になるとみんな子供に見えてねー。	
\\	はっはっは、私の年になるとみんな子供に見えてねー。 
\\	私、村上アグネスと申します。	
\\	私、村上アグネスと申します。 
\\	お若いのにしっかりしてるね。遅れました。横田満夫です。	
\\	お若いのにしっかりしてるね。遅れました。横田満夫です。 
\\	初めまして。	
\\	初めまして。 
\\	初めまして。アグネスか珍しい名前だね。	
\\	初めまして。アグネスか珍しい名前だね。 
\\	ええ、母がフィリピン人で。	
\\	ええ、母がフィリピン人で。 
\\	どおりで。	
\\	どおりで。 
\\	でも、ただ父がアグネスチャンの大ファンっていうだけなんですけどね。本当のところ。	
\\	でも、ただ父がアグネスチャンの大ファンっていうだけなんですけどね。本当のところ。 
\\	それはいい、私も大ファンだ。まあ、これも何かの縁だ。東京で困った事があったらいつでも連絡しなさい。これは私の名刺です。	
\\	それはいい、私も大ファンだ。まあ、これも何かの縁だ。東京で困った事があったらいつでも連絡しなさい。これは私の名刺です。 
\\	どうもありがとうございます。	
\\	どうもありがとうございます。 
\\	さあ、これから新しい生活が始まる。ちょっと心配だけど頑張らないと。あっ、そういえばあの人どこか大きな企業の社長さんだって。幸先いいかも。	
\\	さあ、これから新しい生活が始まる。ちょっと心配だけど頑張らないと。あっ、そういえばあの人どこか大きな企業の社長さんだって。幸先いいかも。 
\\	鍵
\\	タバコ
\\	モーニングコール
\\	かしこまる
\\	お嬢さん、起きて!東京に着きましたよ。	
\\	お嬢さん、起きて!東京に着きましたよ。 
\\	あっ、どうも。	
\\	あっ、どうも。 
\\	まあ、これから色々あると思うが自分の夢に向けて頑張ってください。	
\\	まあ、これから色々あると思うが自分の夢に向けて頑張ってください。 
\\	はい、どうもありがとうございました。	
\\	はい、どうもありがとうございました。 
\\	それでは私はこれで。	
\\	それでは私はこれで。 
\\	はい、失礼します。	
\\	はい、失礼します。 
\\	そうそう、それから何かあったらいつでも連絡してきなさい。	
\\	そうそう、それから何かあったらいつでも連絡してきなさい。 
\\	はい、わかりました。	
\\	はい、わかりました。 
\\	それでは。	
\\	それでは。 
\\	さあ、まずはホテルに行かないと。	
\\	さあ、まずはホテルに行かないと。 
\\	こんにちは。	
\\	こんにちは。 
\\	こんにちは。予約した村上です。	
\\	こんにちは。予約した村上です。 
\\	はい、少々お待ち下さい。はい、村上アグネス様ですね。こちらがお部屋の鍵になります。御二階の右側です。	
\\	はい、少々お待ち下さい。はい、村上アグネス様ですね。こちらがお部屋の鍵になります。御二階の右側です。 
\\	はい、ありがとうございます。	
\\	はい、ありがとうございます。 
\\	あれ、この部屋禁煙じゃないんだ。私タバコ吸わないのに。フロントに言って変えてもらおうっと。	
\\	あれ、この部屋禁煙じゃないんだ。私タバコ吸わないのに。フロントに言って変えてもらおうっと。 
\\	はい、申し訳ございません。今すぐ変えさせて頂きます。	
\\	はい、申し訳ございません。今すぐ変えさせて頂きます。 
\\	はい、お願いします。	
\\	はい、お願いします。 
\\	こちらのお部屋でよろしいでしょうか?	
\\	こちらのお部屋でよろしいでしょうか? 
\\	はい、大丈夫です。ありがとうございます。それから、明日の七時にモーニングコールをお願いします。	
\\	はい、大丈夫です。ありがとうございます。それから、明日の七時にモーニングコールをお願いします。 
\\	はい、かしこまりました。	
\\	はい、かしこまりました。 
\\	不動産
\\	不動産屋
\\	畳
\\	ワンルーム
\\	予算
\\	帖
\\	徒歩
\\	日当たり
\\	出来れば
\\	いらっしゃい。	
\\	いらっしゃい。 
\\	こんにちは。部屋を探しにきたんですが。	
\\	こんにちは。部屋を探しにきたんですが。 
\\	いや、お客さんはついてる。何と言ってもここは世界で一番親切な不動産屋ですから。私に任せておけば必ずいい部屋見つかるよ。	
\\	いや、お客さんはついてる。何と言ってもここは世界で一番親切な不動産屋ですから。私に任せておけば必ずいい部屋見つかるよ。 
\\	よろしくお願いします。	
\\	よろしくお願いします。 
\\	どういう所をお探しですか?	
\\	どういう所をお探しですか? 
\\	1ルームでキッチンとお風呂とトイレが付いている所を探しているんですが。	
\\	1ルームでキッチンとお風呂とトイレが付いている所を探しているんですが。 
\\	そうですねー。あ、この近くに4畳半のアパートがあるけど。	
\\	そうですねー。あ、この近くに4畳半のアパートがあるけど。 
\\	もっと大きなお部屋はないですか。六畳ぐらいの。	
\\	もっと大きなお部屋はないですか。六畳ぐらいの。 
\\	えー、ご予算は?月どれくらい?	
\\	えー、ご予算は?月どれくらい? 
\\	まあ、5、6万円ぐらいで。出来れば駅から近い方がいいんですけど。 
\\	駅の近くは少し高くなるけど。ああ、6万5千円で駅から徒歩五分って所があるよ。	
\\	駅の近くは少し高くなるけど。ああ、6万5千円で駅から徒歩五分って所があるよ。 
\\	ああ、きれいな所ですね。他にはありますか?	
\\	ああ、きれいな所ですね。他にはありますか? 
\\	後は月5万円の所があるよ。新しくはないがアパートの二階で日当たりはいいね。駅からは少し歩くけど。	
\\	後は月5万円の所があるよ。新しくはないがアパートの二階で日当たりはいいね。駅からは少し歩くけど。 
\\	どのくらいですか?	
\\	どのくらいですか? 
\\	まあ、15分ぐらいかな。	
\\	まあ、15分ぐらいかな。 
\\	悪くはないですね。	
\\	悪くはないですね。 
\\	ところでお嬢さんはどちらの方?	
\\	ところでお嬢さんはどちらの方? 
\\	福島です。	
\\	福島です。 
\\	へー、いい所だね。	
\\	へー、いい所だね。 
\\	何もありませんが。自然はきれいです。	
\\	何もありませんが。自然はきれいです。 
\\	この二つ見てみますか?	
\\	この二つ見てみますか? 
\\	はい。ぜひ。	
\\	はい。ぜひ。 
\\	今日はこれから他の所を見せないといけないから、明日は空いてますか?	
\\	今日はこれから他の所を見せないといけないから、明日は空いてますか? 
\\	はい、じゃあ、明日お願いします。	
\\	はい、じゃあ、明日お願いします。 
\\	割く
\\	堪らない
\\	時間を割く
\\	ゴキブリ
\\	手入れ
\\	雲泥の差
\\	線路
\\	限り
\\	割に
\\	手続き
\\	こんにちは。	
\\	こんにちは。 
\\	お忙しい中時間を割いていただきましてありがとうございます。	
\\	お忙しい中時間を割いていただきましてありがとうございます。 
\\	いやいや、気にしないで。これが仕事だから。それに今日はそんなに忙しくないんだよ。	
\\	いやいや、気にしないで。これが仕事だから。それに今日はそんなに忙しくないんだよ。 
\\	そうですか。	
\\	そうですか。 
\\	じゃあ、一つ目の所に行ってみましょうか。	
\\	じゃあ、一つ目の所に行ってみましょうか。 
\\	はい、お願いします。	
\\	はい、お願いします。 
\\	ここが、6万5千円のアパートです。	
\\	ここが、6万5千円のアパートです。 
\\	え、写真と少し違いませんか?あっゴキブリ。ここって高い割には汚い限り。ちょっとここは
\\	え、写真と少し違いませんか?あっゴキブリ。ここって高い割には汚い限り。ちょっとここは
\\	まあ、駅から近いからとても便利ですけどね。	
\\	まあ、駅から近いからとても便利ですけどね。 
\\	近いのはいいですけど線路の隣だと少しうるさくないですか?	
\\	近いのはいいですけど線路の隣だと少しうるさくないですか? 
\\	まあ、すぐに慣れると思うけど。	
\\	まあ、すぐに慣れると思うけど。 
\\	ちょっと、写真で見たのと雲泥の差があるな。	
\\	ちょっと、写真で見たのと雲泥の差があるな。 
\\	まあ、こんな感じかな。つぎの所も行ってみますか?	
\\	まあ、こんな感じかな。つぎの所も行ってみますか? 
\\	そうですね。 
\\	さあ、こちらです。古い建物だけど、手入れされてるからきれいだよ。	
\\	さあ、こちらです。古い建物だけど、手入れされてるからきれいだよ。 
\\	へー、とても感じのいい所ですね。	
\\	へー、とても感じのいい所ですね。 
\\	日当たりもいいですよ。さあ、どうぞ。	
\\	日当たりもいいですよ。さあ、どうぞ。 
\\	うわー、新しい畳、このにおいがたまらない。	
\\	うわー、新しい畳、このにおいがたまらない。 
\\	この前変えてもらったばかりですからね。窓の外を見てごらんなさい。	
\\	この前変えてもらったばかりですからね。窓の外を見てごらんなさい。 
\\	はい、すごい、陽も入るしきれいですね。	
\\	はい、すごい、陽も入るしきれいですね。 
\\	まあ、駅からもそう遠くはないから。	
\\	まあ、駅からもそう遠くはないから。 
\\	月5万円ですよね。	
\\	月5万円ですよね。 
\\	そうです。	
\\	そうです。 
\\	私、ここに決めます。	
\\	私、ここに決めます。 
\\	ありがとうございます。ではお店に戻って手続きさせて頂きますね。	
\\	ありがとうございます。ではお店に戻って手続きさせて頂きますね。 
\\	待ちに待った
\\	買い出し
\\	散歩がてら
\\	揃える
\\	配達
\\	以降
\\	以前
\\	せっかく
\\	いらっしゃいませ。何かお探しですか?	
\\	いらっしゃいませ。何かお探しですか? 
\\	はい、実は一人暮らしを始めるので色々買い出しに。	
\\	はい、実は一人暮らしを始めるので色々買い出しに。 
\\	あ、そうですか。今丁度冬の大セールをやっていまして、ベッドや棚などお安くなっていますので、そちらもぜひご覧になっていってください。	
\\	あ、そうですか。今丁度冬の大セールをやっていまして、ベッドや棚などお安くなっていますので、そちらもぜひご覧になっていってください。 
\\	はい、ありがとうございます。	
\\	はい、ありがとうございます。 
\\	へー、ちょうど良かった。色々揃えられそう。うわー、このテーブルかわいい。何だか必要以上に買っちゃうかも。	
\\	へー、ちょうど良かった。色々揃えられそう。うわー、このテーブルかわいい。何だか必要以上に買っちゃうかも。   
\\	このテーブルと椅子はお買い得ですよ。	
\\	このテーブルと椅子はお買い得ですよ。 
\\	あのー、配達は出来ますか?	
\\	あのー、配達は出来ますか? 
\\	はい、やっております。	
\\	はい、やっております。 
\\	じゃあ、運ぶ心配はないわね。	
\\	じゃあ、運ぶ心配はないわね。 
\\	ただ、今日の分は明日以降になってしまいますが。	
\\	ただ、今日の分は明日以降になってしまいますが。 
\\	はい、大丈夫です。	
\\	はい、大丈夫です。 
\\	せっかくの一人暮らしだし色々揃えちゃおうかな。これに、あっ、あれも。	
\\	せっかくの一人暮らしだし色々揃えちゃおうかな。これに、あっ、あれも。 
\\	えー、合計が3万4千650円になります。	
\\	えー、合計が3万4千650円になります。 
\\	はい。	
\\	はい。 
\\	人里離れた
\\	お手玉
\\	石蹴り
\\	体操
\\	親愛
\\	仲良く
\\	全部
\\	組み立てる
\\	箱
\\	家具
\\	整理
\\	色々
\\	完成
\\	こんにちは、
\\	です。	
\\	こんにちは、
\\	です。 
\\	はーい、こちらにお願いします。	
\\	はーい、こちらにお願いします。 
\\	はい、こちらでよろしいですか?	
\\	はい、こちらでよろしいですか? 
\\	はい、お願いします。	
\\	はい、お願いします。 
\\	それではどうもありがとうございました。	
\\	それではどうもありがとうございました。 
\\	はい、どうも、って、あれ、でも全部箱に入ったままだ。えー、自分で組み立てないといけないの?どうしよう
\\	でも、やってみないと。よし、がんばるぞ。	
\\	はい、どうも、って、あれ、でも全部箱に入ったままだ。えー、自分で組み立てないといけないの?どうしよう
\\	でも、やってみないと。よし、がんばるぞ。 
\\	携帯が鳴る。	
\\	もしもし、え、もしかして満子?	
\\	もしもし、え、もしかして満子? 
\\	うん、アグネス?久しぶり。	
\\	うん、アグネス?久しぶり。 
\\	えー、何年ぶりかしら。元気にしてた?	
\\	えー、何年ぶりかしら。元気にしてた? 
\\	うん、相変わらずよ。アグネスはどうしてるの?	
\\	うん、相変わらずよ。アグネスはどうしてるの? 
\\	実はね今東京に出てきてるんだ。	
\\	実はね今東京に出てきてるんだ。 
\\	うそー、何でもっと早く言ってくれないのよ。	
\\	うそー、何でもっと早く言ってくれないのよ。 
\\	だってまだ出てきたばっかりよ。こっちで仕事を探そうと思って。	
\\	だってまだ出てきたばっかりよ。こっちで仕事を探そうと思って。 
\\	そうなんだ。で、今どこに住んでるの?	
\\	そうなんだ。で、今どこに住んでるの? 
\\	中野の近くなんだけど。	
\\	中野の近くなんだけど。 
\\	へー、今日は何してるの?	
\\	へー、今日は何してるの? 
\\	さっき
\\	から家具が届いたから、それを組み立てたり、色々整理しようと思って。	
\\	さっき
\\	から家具が届いたから、それを組み立てたり、色々整理しようと思って。 
\\	明日は暇?	
\\	明日は暇? 
\\	田中満子:じゃあ、また電話するから。	
\\	ふー、これで完成。やったー。できたわ。さあ、シャワーを浴びたら、窓際のこの新品のテーブルでさっそくビールを飲もうっと。明日は満子と会えるんだ。楽しみ!	
\\	ふー、これで完成。やったー。できたわ。さあ、シャワーを浴びたら、窓際のこの新品のテーブルでさっそくビールを飲もうっと。明日は満子と会えるんだ。楽しみ! 
\\	待たせる
\\	やせる
\\	お洒落
\\	雰囲気
\\	図々しい
\\	職場
\\	同僚
\\	合コン
\\	満子、ごめーん。待った?	
\\	満子、ごめーん。待った? 
\\	ううん、そんなに待たされてないよ。アグネス、久しぶり!ねえ、ちょっとやせた?	
\\	ううん、そんなに待たされてないよ。アグネス、久しぶり!ねえ、ちょっとやせた? 
\\	時々そう言われるけど、変わってないよ。満子はきれいになったね!	
\\	時々そう言われるけど、変わってないよ。満子はきれいになったね! 
\\	またあ。	
\\	またあ。 
\\	このお店、すごくお洒落だね。	
\\	このお店、すごくお洒落だね。 
\\	いい雰囲気でしょ?	
\\	いい雰囲気でしょ? 
\\	あ、いたいた。満子ー。	
\\	あ、いたいた。満子ー。 
\\	この席でいいの?	
\\	この席でいいの? 
\\	どうぞどうぞ。	
\\	どうぞどうぞ。 
\\	え?	
\\	え? 
\\	おお、この人が満子が言ってた人か。うわあ、本当にきれいな人だなあ。あ、すみません、隣に座ってもいいですか?	
\\	おお、この人が満子が言ってた人か。うわあ、本当にきれいな人だなあ。あ、すみません、隣に座ってもいいですか? 
\\	おい、お前ちょっと図々しいよ。	
\\	おい、お前ちょっと図々しいよ。 
\\	ちょっとちょっと、満子。どうなってるの?	
\\	ちょっとちょっと、満子。どうなってるの? 
\\	ああ、ゴメンね。アグネスを紹介して、って頼まれたの。こちらが松田浩君。私の職場の同僚。	
\\	ああ、ゴメンね。アグネスを紹介して、って頼まれたの。こちらが松田浩君。私の職場の同僚。 
\\	どうもー、こんにちはー。で、こっちが俺の中学校時代の友達で、橋本雄介っていいます。こう見えても医者なんですよ。	
\\	どうもー、こんにちはー。で、こっちが俺の中学校時代の友達で、橋本雄介っていいます。こう見えても医者なんですよ。 
\\	どうも、はじめまして。今日は浩に連れてこられちゃって…	
\\	どうも、はじめまして。今日は浩に連れてこられちゃって… 
\\	まあ、お医者様なの!かっこいい!	
\\	まあ、お医者様なの!かっこいい! 
\\	いやいや、たいしたことないですよ。	
\\	いやいや、たいしたことないですよ。 
\\	村上アグネスさんって言うんでしょ?満子から話を聞かされて、一度会いたいって思ってたんだ。	
\\	村上アグネスさんって言うんでしょ?満子から話を聞かされて、一度会いたいって思ってたんだ。 
\\	何これ、合コン?ひょっとして私、利用された?	
\\	何これ、合コン?ひょっとして私、利用された? 
\\	山奥
\\	競争
\\	人気者
\\	家来
\\	素晴らしい
\\	武士
\\	高校時代
\\	寂しい
\\	連絡
\\	規則正しい
\\	替わる
\\	はい、もしもし。	
\\	はい、もしもし。 
\\	アグネス?	
\\	アグネス? 
\\	お母さん!朝からどうしたの?	
\\	お母さん!朝からどうしたの? 
\\	まあいやだ、この子ったら。朝どころかもうお昼過ぎよ。今まで寝ていたの?	
\\	まあいやだ、この子ったら。朝どころかもうお昼過ぎよ。今まで寝ていたの? 
\\	ええ?あ、本当だ。うーん、昨日ちょっと遅くまで友達と飲んでたから。	
\\	ええ?あ、本当だ。うーん、昨日ちょっと遅くまで友達と飲んでたから。 
\\	あら、もうそっちでお友達ができたの?	
\\	あら、もうそっちでお友達ができたの? 
\\	ううん、高校時代の友達の田中満子。今東京にいて、連絡くれたんだ。	
\\	ううん、高校時代の友達の田中満子。今東京にいて、連絡くれたんだ。 
\\	まあ、そうなの。どうしてるかと思ったけど、元気みたいね。ちょっとお父さんと替わるわ。	
\\	まあ、そうなの。どうしてるかと思ったけど、元気みたいね。ちょっとお父さんと替わるわ。 
\\	もしもし、アグネスか?元気でやってるか?一人で寂しくないか?	
\\	もしもし、アグネスか?元気でやってるか?一人で寂しくないか? 
\\	やだ、お父さん。私、子供じゃないんだから。寂しいどころか、毎日忙しくて、お父さんとお母さんに電話するのも忘れちゃってたわ。	
\\	やだ、お父さん。私、子供じゃないんだから。寂しいどころか、毎日忙しくて、お父さんとお母さんに電話するのも忘れちゃってたわ。 
\\	そうか、それはよかった…。アグネスは寂しくないんだな…。	
\\	そうか、それはよかった…。アグネスは寂しくないんだな…。 
\\	ん、どうしたの、お父さん?	
\\	ん、どうしたの、お父さん? 
\\	いやいやなんでもない。また母さんに替わるよ。	
\\	いやいやなんでもない。また母さんに替わるよ。 
\\	アグネス、遊ぶのもいいけどちゃんと規則正しい生活をして、早く仕事を見つけなさい。	
\\	アグネス、遊ぶのもいいけどちゃんと規則正しい生活をして、早く仕事を見つけなさい。 
\\	はあい。ねえお母さん、お父さんはどうしたの?ちょっと元気がなかったけど。	
\\	はあい。ねえお母さん、お父さんはどうしたの?ちょっと元気がなかったけど。 
\\	元気がないどころか。お父さんったら毎日、子供の頃のアグネスの写真を見ながら泣いてるのよ。	
\\	元気がないどころか。お父さんったら毎日、子供の頃のアグネスの写真を見ながら泣いてるのよ。 
\\	何を言うんだ母さん!アグネス、お父さんは元気だぞ。泣いてなんかいないぞ!	
\\	何を言うんだ母さん!アグネス、お父さんは元気だぞ。泣いてなんかいないぞ! 
\\	お父さん、あんな強がり言ってるけど、アグネスがいなくて寂しいのよ。	
\\	お父さん、あんな強がり言ってるけど、アグネスがいなくて寂しいのよ。 
\\	あーあ、なんだか心配になってきちゃった。お父さん、大丈夫かしら。	
\\	あーあ、なんだか心配になってきちゃった。お父さん、大丈夫かしら。 
\\	神様
\\	我慢
\\	お手洗い
\\	一生懸命
\\	転がる
\\	方法
\\	履歴書
\\	職
\\	件
\\	承知
\\	助かる
\\	面接
\\	早速
\\	伺う
\\	是非
\\	はい、もしもし。	
\\	はい、もしもし。 
\\	村上アグネスさんですか?	
\\	村上アグネスさんですか? 
\\	はい、そうです。 
\\	デザインオフィスの杉田と申します。先日履歴書をお送りいただいた件でお電話しました。	
\\	デザインオフィスの杉田と申します。先日履歴書をお送りいただいた件でお電話しました。 
\\	あ、はい!お電話ありがとうございます。	
\\	あ、はい!お電話ありがとうございます。 
\\	グラフィックデザイナーの職をご希望ということで、間違いないですね。	
\\	グラフィックデザイナーの職をご希望ということで、間違いないですね。 
\\	はい。よろしくお願いします。	
\\	はい。よろしくお願いします。 
\\	早速ですが、近いうちにこちらに来ていただいて、直接お話をお聞きしたいのですが。	
\\	早速ですが、近いうちにこちらに来ていただいて、直接お話をお聞きしたいのですが。 
\\	わかりました!今日すぐにでもうかがいます。	
\\	わかりました!今日すぐにでもうかがいます。 
\\	いえ、今日はちょっと…。水曜日の夕方4時ではいかがでしょうか?	
\\	いえ、今日はちょっと…。水曜日の夕方4時ではいかがでしょうか? 
\\	あ、申し訳ありません、その日は予定が入っておりまして…。午後の早いうちなら空いているんですが…。	
\\	あ、申し訳ありません、その日は予定が入っておりまして…。午後の早いうちなら空いているんですが…。 
\\	それでは木曜日の午前中はいかがですか?	
\\	それでは木曜日の午前中はいかがですか? 
\\	はい、それならうかがえます。	
\\	はい、それならうかがえます。 
\\	では、木曜日の11時に。	
\\	では、木曜日の11時に。 
\\	承知いたしました。あ、すみません、会社は青山でしたよね?	
\\	承知いたしました。あ、すみません、会社は青山でしたよね? 
\\	そうです。表参道の駅から5分くらいです。地図をファックスしましょうか?	
\\	そうです。表参道の駅から5分くらいです。地図をファックスしましょうか? 
\\	ぜひお願いします。まだ東京に慣れていないので、地図があると助かります。	
\\	わかりました。履歴書にあるこちらの番号でいいんですね。では、後ほどファックスをお送りします.	
\\	わかりました。履歴書にあるこちらの番号でいいんですね。では、後ほどファックスをお送りします. 
\\	(電話を切る)	
\\	やった!早速面接だ!よーし、今日のうちにスーツにアイロンかけておこうっと。	
\\	やった!早速面接だ!よーし、今日のうちにスーツにアイロンかけておこうっと。 
\\	うっかり
\\	通りかかる
\\	思いっきり
\\	目掛ける
\\	霞む
\\	うっとり
\\	姿
\\	眺める
\\	ご機嫌
\\	天辺
\\	日が暮れる
\\	懲りる
\\	ううむ、富士はやはりきれいな山じゃのう。背たけも高くて、人々が朝に夕に手を合わせる気持ちもわかるわい。	
\\	ううむ、富士はやはりきれいな山じゃのう。背たけも高くて、人々が朝に夕に手を合わせる気持ちもわかるわい。  
\\	だめだ、だめだ! 富士がいるおかげで、わしの箱根山の美しさがかすんでしまう。人間どもは箱根に尻(しり)を向けて富士ばかり見ておる。なんとかしなくては・・・	
\\	だめだ、だめだ! 富士がいるおかげで、わしの箱根山の美しさがかすんでしまう。人間どもは箱根に尻(しり)を向けて富士ばかり見ておる。なんとかしなくては・・・ 
\\	しまった。夜が明けてしまっては、わしの力がなくなってしまう。	
\\	しまった。夜が明けてしまっては、わしの力がなくなってしまう。 
\\	市役所
\\	拝見
\\	年金手帳
\\	提示
\\	転出証明書
\\	番号札
\\	混雑
\\	記入
\\	書類
\\	窓口
\\	転入届
\\	役所
\\	印鑑証明
\\	えーと、市役所はどこかしら。あ、ここね。	
\\	えーと、市役所はどこかしら。あ、ここね。 
\\	すみません、転入届の受付はどちらですか?	
\\	すみません、転入届の受付はどちらですか? 
\\	はい、3番の窓口になります。まず、あちらにある書類に記入してから窓口でお申し込みください。	
\\	はい、3番の窓口になります。まず、あちらにある書類に記入してから窓口でお申し込みください。 
\\	はい、わかりました。	
\\	はい、わかりました。 
\\	ただいま受付が少々混雑しておりますので、あちらの番号札を取ってください。順番にお呼びします。	
\\	ただいま受付が少々混雑しておりますので、あちらの番号札を取ってください。順番にお呼びします。 
\\	わかりました。ありがとうございます。	
\\	わかりました。ありがとうございます。 
\\	番号札105番でお待ちの方、窓口までどうぞ。	
\\	番号札105番でお待ちの方、窓口までどうぞ。 
\\	あ、私だわ!	
\\	あ、私だわ! 
\\	えーと、転入届を出したいんですけど。	
\\	えーと、転入届を出したいんですけど。 
\\	はい、転出証明書をいただけますか?	
\\	はい、転出証明書をいただけますか? 
\\	こちらです。	
\\	こちらです。 
\\	本日、運転免許証やパスポートなどの身分証明書はお持ちですか?	
\\	本日、運転免許証やパスポートなどの身分証明書はお持ちですか? 
\\	え…私、免許は持っていないんですけど…。	
\\	申し訳ございませんが、転入届の受付にあたっては、身分証明書をご提示いただく必要がございまして。	
\\	申し訳ございませんが、転入届の受付にあたっては、身分証明書をご提示いただく必要がございまして。 
\\	あのう、前の市役所では年金手帳を持って行ったら大丈夫だったんですけど。だから今日は年金手帳を持ってきました。	
\\	あのう、前の市役所では年金手帳を持って行ったら大丈夫だったんですけど。だから今日は年金手帳を持ってきました。 
\\	そうですか。ちょっと拝見してよろしいですか?少々お待ちください。	
\\	そうですか。ちょっと拝見してよろしいですか?少々お待ちください。 
\\	お待たせしました。こちらで大丈夫です。お引越しにあたって、印鑑証明の手続きなどもなさいますか?	
\\	お待たせしました。こちらで大丈夫です。お引越しにあたって、印鑑証明の手続きなどもなさいますか? 
\\	いいえ、それは今回必要ありません。	
\\	いいえ、それは今回必要ありません。 
\\	では、これで手続きは終了です。	
\\	では、これで手続きは終了です。 
\\	ありがとうございました。	
\\	ありがとうございました。 
\\	海辺
\\	血の気が引く
\\	心当り
\\	危うい
\\	手ごたえ
\\	押し上げる
\\	強まる
\\	空模様
\\	機
\\	すっかり
\\	こりゃ、大雨になるぞ。	
\\	こりゃ、大雨になるぞ。 
\\	お父っつぁん、こりゃあ。	
\\	お父っつぁん、こりゃあ。 
\\	今日は、ひきあげよう。おばあさんに頼まれていた物を買ってから帰ろう。	
\\	今日は、ひきあげよう。おばあさんに頼まれていた物を買ってから帰ろう。 
\\	もし、あなたたちは、昨夜の大雨の時、どうしていましたか? 
\\	はい、わしたちは危ういところで島に逃れられました。	
\\	はい、わしたちは危ういところで島に逃れられました。 
\\	そうでしたか、それはよろしゅうございました。ところでここへ来る途中、千石船(せんごくぶね・江戸時代、米を千石ほど積める大形の和船)を見かけませんでしたか?	
\\	そうでしたか、それはよろしゅうございました。ところでここへ来る途中、千石船(せんごくぶね・江戸時代、米を千石ほど積める大形の和船)を見かけませんでしたか? 
\\	いいや、見なかったですな。ですが今日、わしらのアミに若い男の死骸(しがい)がかかって、島にうめてきました。	
\\	いいや、見なかったですな。ですが今日、わしらのアミに若い男の死骸(しがい)がかかって、島にうめてきました。  
\\	死骸ですと!	
\\	死骸ですと! 
\\	なにか、心当りでもありなさるのか?	
\\	なにか、心当りでもありなさるのか? 
\\	実は、息子が大阪に千石船で米を積んで出て行ったのですが、そこへあの大雨。心配しているところです。	
\\	実は、息子が大阪に千石船で米を積んで出て行ったのですが、そこへあの大雨。心配しているところです。 
\\	そうじゃったか。	
\\	そうじゃったか。 
\\	ごめんどうをおかけしますが、わたしをその島へ連れて行ってもらえますまいか?	
\\	ごめんどうをおかけしますが、わたしをその島へ連れて行ってもらえますまいか? 
\\	むっ、息子です。	
\\	むっ、息子です。 
\\	あなたたちには、すっかりお世話になりました。わたしの心からのお礼を港に用意しました。どうか受け取って下さい。	
\\	あなたたちには、すっかりお世話になりました。わたしの心からのお礼を港に用意しました。どうか受け取って下さい。 
\\	いや、お礼なんぞいりません。	
\\	いや、お礼なんぞいりません。 
\\	人事担当
\\	個性的
\\	後日
\\	歓迎
\\	熱心
\\	地味
\\	包装紙
\\	音楽関係
\\	個人的
\\	応募
\\	相性
\\	ごめんください。本日4時から面接をしていただく予定になっている、村上アグネスと申します。	
\\	ごめんください。本日4時から面接をしていただく予定になっている、村上アグネスと申します。 
\\	はい、うかがっています。少々お待ちください。	
\\	はい、うかがっています。少々お待ちください。 
\\	うわあ、いよいよ面接か。緊張するなあ。	
\\	うわあ、いよいよ面接か。緊張するなあ。 
\\	(ドアをノックする)	
\\	どうぞお入りください。	
\\	どうぞお入りください。 
\\	失礼します。	
\\	失礼します。 
\\	初めまして、人事担当の杉田と申します。	
\\	初めまして、人事担当の杉田と申します。 
\\	先日はお電話をありがとうございました。村上アグネスと申します。	
\\	先日はお電話をありがとうございました。村上アグネスと申します。 
\\	どうぞおかけください。	
\\	どうぞおかけください。 
\\	では、失礼します。	
\\	では、失礼します。 
\\	まず、うちの会社に応募されたきっかけは?	
\\	まず、うちの会社に応募されたきっかけは? 
\\	個人的な理由なのですが、学生の頃から音楽が好きなんです。それで、自分の好きなアーティストの
\\	ジャケットのデザインを、こちらのデザインオフィス
\\	さんが担当されているのを知って、興味を持ちました。	
\\	個人的な理由なのですが、学生の頃から音楽が好きなんです。それで、自分の好きなアーティストの
\\	ジャケットのデザインを、こちらのデザインオフィス
\\	さんが担当されているのを知って、興味を持ちました。 
\\	我が社の仕事をご存知でしたか。では、やはりそういった音楽関係の仕事をご希望ですか?	
\\	我が社の仕事をご存知でしたか。では、やはりそういった音楽関係の仕事をご希望ですか? 
\\	ええ、やってみたいという気持ちはあります。	
\\	ええ、やってみたいという気持ちはあります。 
\\	ただ、うちの仕事は音楽関係が特に多いというわけ ではないんです。たとえば包装紙のデザインなど、地味な仕事も多いですよ。	
\\	ただ、うちの仕事は音楽関係が特に多いというわけ ではないんです。たとえば包装紙のデザインなど、地味な仕事も多いですよ。 
\\	どんなに地味な仕事でも構いません。デザインという仕事を一生懸命勉強したいと思っています。	
\\	なかなか、熱心ですね。	
\\	なかなか、熱心ですね。 
\\	恐れ入ります。経験がないので、気持ちだけは前向きに、と思っています。	
\\	恐れ入ります。経験がないので、気持ちだけは前向きに、と思っています。 
\\	いいですね、仕事に前向きな人は歓迎ですよ。 それでは後日、改めてうちの社長とお会いになってください。昼食でもご一緒しながら、色々お話をうかがいたいので。	
\\	いいですね、仕事に前向きな人は歓迎ですよ。 それでは後日、改めてうちの社長とお会いになってください。昼食でもご一緒しながら、色々お話をうかがいたいので。 
\\	はい、ありがとうございます!	
\\	はい、ありがとうございます! 
\\	実は、うちの社長はかなり個性的な人物なので…社長との相性が結構大切なのですよ。	
\\	実は、うちの社長はかなり個性的な人物なので…社長との相性が結構大切なのですよ。 
\\	え…個性的?どんな方なんだろう…。	
\\	え…個性的?どんな方なんだろう…。 
\\	ご一緒する
\\	隠し味
\\	自信作
\\	肝心
\\	見所
\\	芸術家
\\	設備
\\	インスピレーション
\\	ちょうどよかった
\\	社長室
\\	採用
\\	失礼します。本日、社長に面接をしていただく予定の村上アグネスです。お昼をご一緒するとうかがっているんですが。	
\\	失礼します。本日、社長に面接をしていただく予定の村上アグネスです。お昼をご一緒するとうかがっているんですが。 
\\	(ドアをノックする)	
\\	失礼します。	
\\	失礼します。 
\\	あ、君!ちょうどよかった!そこの塩を取ってくれ!	
\\	あ、君!ちょうどよかった!そこの塩を取ってくれ! 
\\	は?こ、これですか?	
\\	は?こ、これですか? 
\\	そうそう!よし、これでばっちりだ!ああ、君が村上アグネス君? 
\\	はい…あの…。	
\\	はい…あの…。 
\\	社長の藤本だ。よろしく。まあ、座って座って。	
\\	社長の藤本だ。よろしく。まあ、座って座って。 
\\	このテーブルのところでいいんでしょうか?	
\\	このテーブルのところでいいんでしょうか? 
\\	そうそう、そこ。実は僕は料理が趣味なんだ!料理をしているとデザインのインスピレーションも湧くんだよ。今日は君に僕の手料理を食べてもらおうと思ってね。	
\\	そうそう、そこ。実は僕は料理が趣味なんだ!料理をしているとデザインのインスピレーションも湧くんだよ。今日は君に僕の手料理を食べてもらおうと思ってね。 
\\	だから、社長室にキッチンがあるんですか?	
\\	だから、社長室にキッチンがあるんですか? 
\\	そうそう。なかなかいい設備だろう?南フランスのレストランっぽいデザインにしてみたんだ。	
\\	そうそう。なかなかいい設備だろう?南フランスのレストランっぽいデザインにしてみたんだ。 
\\	え、ええ。このデザインなんか、素敵ですね。	
\\	え、ええ。このデザインなんか、素敵ですね。 
\\	おお、これに注目するとは、なかなか芸術家っぽいセンスがあるね!デザイナーとして見所があるぞ!	
\\	おお、これに注目するとは、なかなか芸術家っぽいセンスがあるね!デザイナーとして見所があるぞ! 
\\	本当ですか?	
\\	本当ですか? 
\\	うん、でも肝心なのは料理を味わうセンスだな。まあ、これを食べてみてくれ。最近の自信作なんだ。	
\\	うん、でも肝心なのは料理を味わうセンスだな。まあ、これを食べてみてくれ。最近の自信作なんだ。 
\\	はい、いただきます!	
\\	はい、いただきます! 
\\	どうだい?ソースの隠し味がわかるかな?	
\\	どうだい?ソースの隠し味がわかるかな? 
\\	とても美味しいです。えーと…この黄色っぽいの、なんでしょう…。ショウガかしら?	
\\	とても美味しいです。えーと…この黄色っぽいの、なんでしょう…。ショウガかしら? 
\\	正解だ!すばらしい!君、いいセンスをしているよ!採用だ!	
\\	正解だ!すばらしい!君、いいセンスをしているよ!採用だ! 
\\	突然の
\\	無理矢理
\\	折り入って
\\	本格的
\\	見習い
\\	ついてる
\\	お役に立つ
\\	はい、もしもし。	
\\	はい、もしもし。 
\\	こんにちは、村上アグネスさんですか?	
\\	こんにちは、村上アグネスさんですか? 
\\	そうですけど…。	
\\	そうですけど…。 
\\	突然の電話で、失礼します。僕、橋本雄介です。先日、田中満子さんと一緒にお会いした、松田の友人です。	
\\	突然の電話で、失礼します。僕、橋本雄介です。先日、田中満子さんと一緒にお会いした、松田の友人です。 
\\	え~と…ああ!あのお医者様の?	
\\	え~と…ああ!あのお医者様の? 
\\	そうです。松田から無理矢理電話番号を聞き出しちゃいました。	
\\	そうです。松田から無理矢理電話番号を聞き出しちゃいました。 
\\	そうなんですか。びっくりした~。	
\\	そうなんですか。びっくりした~。 
\\	驚かせてごめんなさい。実は、アグネスさんがデザイナーだって聞いて、折り入ってお願いがあるんですけど…。	
\\	驚かせてごめんなさい。実は、アグネスさんがデザイナーだって聞いて、折り入ってお願いがあるんですけど…。 
\\	いえ、私、本格的なデザイナーじゃなくて、まだ見習いですよ!	
\\	いえ、私、本格的なデザイナーじゃなくて、まだ見習いですよ! 
\\	いやいや、そういうセンスのある人の意見を聞きたいんですよ。	
\\	いやいや、そういうセンスのある人の意見を聞きたいんですよ。 
\\	私でお役に立てるかしら。どんなことですか?	
\\	私でお役に立てるかしら。どんなことですか? 
\\	う~ん、電話ではちょっと説明しづらいなあ。もし差し支えなければ、近いうちに一緒に食事でもいかがですか?そのときにゆっくりご説明します。急ぐ話じゃないんです。	
\\	う~ん、電話ではちょっと説明しづらいなあ。もし差し支えなければ、近いうちに一緒に食事でもいかがですか?そのときにゆっくりご説明します。急ぐ話じゃないんです。 
\\	ええ、私は構いませんよ。	
\\	ええ、私は構いませんよ。 
\\	そうですか、よかった。いつがいいですか?	
\\	そうですか、よかった。いつがいいですか? 
\\	来週から新しい仕事が始まるので、差し支えなければ今週中がいいんですが。	
\\	来週から新しい仕事が始まるので、差し支えなければ今週中がいいんですが。 
\\	わかりました。では、明後日の夜はいかがですか?	
\\	わかりました。では、明後日の夜はいかがですか? 
\\	ええ、大丈夫です。	
\\	ええ、大丈夫です。 
\\	(電話切れる)	
\\	やった!あのとき、満子の友達はちょっとしつこい感じだったけど、この人は感じのいい人だな。最近の私、ついてるみたい!	
\\	やった!あのとき、満子の友達はちょっとしつこい感じだったけど、この人は感じのいい人だな。最近の私、ついてるみたい! 
\\	人気商品
\\	カーディガン
\\	たまには
\\	手にとる
\\	ワンピース
\\	ウィンドー
\\	スリム
\\	色違い
\\	試着
\\	落ち着いた
\\	いらっしゃいませ。何かお探しですか?	
\\	いらっしゃいませ。何かお探しですか? 
\\	ウィンドウにあったワンピースが素敵だな、と思って。見せていただけますか?	
\\	ウィンドウにあったワンピースが素敵だな、と思って。見せていただけますか? 
\\	こちらですね。どうぞ、お手にとってご覧ください。これはこの春の人気商品なんですよ!春らしい感じで、お勧めです!	
\\	こちらですね。どうぞ、お手にとってご覧ください。これはこの春の人気商品なんですよ!春らしい感じで、お勧めです! 
\\	でも、このピンクは私には可愛らしすぎるかなあ。色違いはありますか?	
\\	でも、このピンクは私には可愛らしすぎるかなあ。色違いはありますか? 
\\	申し訳ございません、こちらの商品はこの色だけなんです。でも、きっとお似合いだと思いますよ。よろしかったら、ご試着なさいますか?	
\\	申し訳ございません、こちらの商品はこの色だけなんです。でも、きっとお似合いだと思いますよ。よろしかったら、ご試着なさいますか? 
\\	そうですね…。	
\\	そうですね…。 
\\	店員:(ノックの音)お客様、いかがですか?	
\\	うーん、あの…ちょっと小さいみたいで…もう一つ上のサイズはありますか?	
\\	うーん、あの…ちょっと小さいみたいで…もう一つ上のサイズはありますか? 
\\	店員:(ノックの音)お客様、いかがですか?	
\\	ええ、大丈夫みたい。	
\\	ええ、大丈夫みたい。 
\\	わあ、とてもお似合いですよ!お客様のイメージにぴったり!	
\\	わあ、とてもお似合いですよ!お客様のイメージにぴったり! 
\\	太って見えませんか?	
\\	太って見えませんか? 
\\	全然そんなことありませんよ!お客様は脚が長いし、とてもスリムに見えます。	
\\	全然そんなことありませんよ!お客様は脚が長いし、とてもスリムに見えます。 
\\	ええー、そうですか?でも、たまにはこういう色もいいかな。	
\\	ええー、そうですか?でも、たまにはこういう色もいいかな。 
\\	こちらの黒いカーディガンを合わせていただくと、落ち着いた感じになりますよ。	
\\	こちらの黒いカーディガンを合わせていただくと、落ち着いた感じになりますよ。 
\\	あ、本当。これなら大人っぽく見えますね。じゃあ、このワンピースとカーディガンをいただきます。	
\\	あ、本当。これなら大人っぽく見えますね。じゃあ、このワンピースとカーディガンをいただきます。 
\\	ありがとうございます。	
\\	ありがとうございます。 
\\	比較的
\\	餌
\\	マウス
\\	よれよれ
\\	ボサボサ
\\	実験
\\	白衣
\\	実態
\\	堅苦しい
\\	診療
\\	初出勤
\\	こんばんは。ごめんなさい、お待たせしちゃいました?	
\\	こんばんは。ごめんなさい、お待たせしちゃいました? 
\\	あ、こんばんは!こちらこそ、突然お誘いしちゃってすみません。	
\\	あ、こんばんは!こちらこそ、突然お誘いしちゃってすみません。 
\\	橋本さん、今日はお忙しかったんじゃないですか?	
\\	橋本さん、今日はお忙しかったんじゃないですか? 
\\	いえいえ。僕も今日は比較的暇だったんです。それより、橋本さん、なんて堅苦しいので、雄介でいいですよ。	
\\	"いえいえ。僕も今日は比較的暇だったんです。それより、橋本さん、なんて堅苦しいので、雄介でいいですよ。 
\\	じゃあ、雄介さん。お仕事先は病院なんですか?	
\\	じゃあ、雄介さん。お仕事先は病院なんですか? 
\\	ええ、そうです。この近くにある大学病院です。でも、僕は診療じゃなくて研究の方なんですよ。	
\\	ええ、そうです。この近くにある大学病院です。でも、僕は診療じゃなくて研究の方なんですよ。 
\\	わあ、研究なんて、かっこいい!	
\\	わあ、研究なんて、かっこいい! 
\\	いやー、実態はひどいんです。普段は頭もぼさぼさで、白衣もよれよれ。その格好で夜中に実験用のマウスに餌をやってるところは、ちょっと女性には見せられないね。	
\\	いやー、実態はひどいんです。普段は頭もぼさぼさで、白衣もよれよれ。その格好で夜中に実験用のマウスに餌をやってるところは、ちょっと女性には見せられないね。 
\\	あ、それは見てみたい。	
\\	あ、それは見てみたい。 
\\	やめてくれよー。それよりアグネスさんの方がかっこいいよ、デザイナーなんて。	
\\	やめてくれよー。それよりアグネスさんの方がかっこいいよ、デザイナーなんて。 
\\	全然。それにまだちゃんと仕事をしているわけじゃないの。	
\\	全然。それにまだちゃんと仕事をしているわけじゃないの。 
\\	そういえば、こっちで仕事を探してるんだっけ。	
\\	そういえば、こっちで仕事を探してるんだっけ。 
\\	そう!この間面接に行って、採用になったの。明日が初出勤なんだ。	
\\	そう!この間面接に行って、採用になったの。明日が初出勤なんだ。 
\\	えー、おめでとう!じゃあ、お祝いに美味しいワインを開けようよ。今日は僕がおごるから。	
\\	えー、おめでとう!じゃあ、お祝いに美味しいワインを開けようよ。今日は僕がおごるから。 
\\	え、いいの?あれ、そういえば何か相談があるって話じゃなかった?	
\\	え、いいの?あれ、そういえば何か相談があるって話じゃなかった? 
\\	いいのいいの、その話はまた今度でも。じゃあ、アグネスの就職にカンパーイ!	
\\	いいのいいの、その話はまた今度でも。じゃあ、アグネスの就職にカンパーイ! 
\\	ありがとう。カンパーイ!	
\\	ありがとう。カンパーイ! 
\\	無茶な
\\	人材
\\	景気
\\	就職活動
\\	仕様書
\\	プレゼン
\\	手一杯
\\	抱える
\\	おいしい仕事
\\	気合いを入れる
\\	おはようございます、社長。	
\\	おはようございます、社長。 
\\	昨日、ゴリから電話があったらしいな。今内容を確認したんだが、まったく、無茶な注文だよ。この忙しい時に!君が電話を受けたのか?	
\\	昨日、ゴリから電話があったらしいな。今内容を確認したんだが、まったく、無茶な注文だよ。この忙しい時に!君が電話を受けたのか? 
\\	すみません、お断りするべきでしたか?	
\\	すみません、お断りするべきでしたか? 
\\	いや、断るのは無理だろう。あいつは昔から強引なんだ。それにいつもの倍の金額を払うと言っている。おいしい仕事だよ。この仕事は受けるべきだな。	
\\	いや、断るのは無理だろう。あいつは昔から強引なんだ。それにいつもの倍の金額を払うと言っている。おいしい仕事だよ。この仕事は受けるべきだな。 
\\	そうなんですか。	
\\	そうなんですか。 
\\	よし、この件は君が担当しなさい。初仕事だ。	
\\	よし、この件は君が担当しなさい。初仕事だ。 
\\	ええっ?私がですか?	
\\	ええっ?私がですか? 
\\	他のスタッフは皆、急ぎの仕事を抱えていて手一杯なんだよ。後でゴリには電話をしておくから、今週の金曜日までにプレゼンの準備をしてくれ。できるか?	
\\	他のスタッフは皆、急ぎの仕事を抱えていて手一杯なんだよ。後でゴリには電話をしておくから、今週の金曜日までにプレゼンの準備をしてくれ。できるか? 
\\	は、はい!何とかなると思います。	
\\	は、はい!何とかなると思います。 
\\	よし、いいぞ!これが仕様書だ。詳しいことは、野村や他のスタッフに聞きなさい。	
\\	よし、いいぞ!これが仕様書だ。詳しいことは、野村や他のスタッフに聞きなさい。 
\\	はい。内容は、就職活動の学生向けのパンフレットですか…。	
\\	はい。内容は、就職活動の学生向けのパンフレットですか…。 
\\	あいつの会社は今景気がよくて、今年は優秀な人材をたくさん採用する気でいるんだ。気合いを入れろよ!この仕事が取れたら、特別ボーナスだ!	
\\	あいつの会社は今景気がよくて、今年は優秀な人材をたくさん採用する気でいるんだ。気合いを入れろよ!この仕事が取れたら、特別ボーナスだ! 
\\	が、頑張ります!	
\\	が、頑張ります! 
\\	自然破壊
\\	資源
\\	枯渇
\\	有害
\\	紫外線
\\	遮断
\\	皮膚がん
\\	白内障
\\	水かさ
\\	排出
\\	大気汚染物質
\\	土壌
\\	食糧危機
\\	砂漠化
\\	道徳
\\	地球
\\	自然環境
\\	取り巻く
\\	悪化
\\	破壊
\\	汚染
\\	オゾン層
\\	異常気象
\\	酸性雨
\\	森林破壊
\\	水質汚染
\\	今日の道徳の時間は、みんなでビデオを見ますよ。	
\\	今日の道徳の時間は、みんなでビデオを見ますよ。 
\\	先生、どんなビデオを見るんですか。	
\\	先生、どんなビデオを見るんですか。 
\\	今日は、地球の自然環境が、どんどん悪くなっていることについてのビデオを見ます。じゃあ、みんなテレビの前に集まって。	
\\	今日は、地球の自然環境が、どんどん悪くなっていることについてのビデオを見ます。じゃあ、みんなテレビの前に集まって。 
\\	年々、私たちを取り巻く自然環境はどんどん悪化しています。私たち人間は、自然を破壊し、汚染し続けてきました。地球温暖化やオゾン層の破壊、異常気象、酸性雨、森林破壊、砂漠化、食糧危機、資源の枯渇。順調に文明が発達してきたはずの地球に、今何が起こっているのでしょうか。そして、私たちは今何をすべきなのでしょうか。自然破壊が進む今、どのような影響が出ているのか見てみましょう。	
\\	年々、私たちを取り巻く自然環境はどんどん悪化しています。私たち人間は、自然を破壊し、汚染し続けてきました。地球温暖化やオゾン層の破壊、異常気象、酸性雨、森林破壊、砂漠化、食糧危機、資源の枯渇。順調に文明が発達してきたはずの地球に、今何が起こっているのでしょうか。そして、私たちは今何をすべきなのでしょうか。自然破壊が進む今、どのような影響が出ているのか見てみましょう。 
\\	みんな、自然破壊がどれだけこわいものかわかったでしょう。自然を守るために、みんなができることがあるって言っていたけど、それが何かわかる人。	
\\	みんな、自然破壊がどれだけこわいものかわかったでしょう。自然を守るために、みんなができることがあるって言っていたけど、それが何かわかる人。 
\\	ハイハイハイ!	
\\	ハイハイハイ! 
\\	じゃあ、サトシ。	
\\	じゃあ、サトシ。 
\\	はい。ゴミを出す時は、分別して出すことです。それと、ポイ捨ては絶対しないことです。	
\\	はい。ゴミを出す時は、分別して出すことです。それと、ポイ捨ては絶対しないことです。 
\\	その通り。みんなもこれからは自然のことを少し考えて行動すること。わかったわね。	
\\	その通り。みんなもこれからは自然のことを少し考えて行動すること。わかったわね。 
\\	(授業終わりのチャイムが鳴る)	
\\	早々
\\	任される
\\	期待される
\\	今さら
\\	終電
\\	はい、もしもし。	
\\	はい、もしもし。 
\\	アグネスさん?	
\\	アグネスさん? 
\\	あ、雄介さん?こんばんは。	
\\	あ、雄介さん?こんばんは。 
\\	遅くにごめんね。今、大丈夫かな?	
\\	遅くにごめんね。今、大丈夫かな? 
\\	えーと、今仕事中なの。	
\\	えーと、今仕事中なの。 
\\	えっ、こんな時間まで?もう12時過ぎてるよ!どうしたの?	
\\	えっ、こんな時間まで?もう12時過ぎてるよ!どうしたの? 
\\	うん、入社早々大きな仕事を任されちゃって、全然終わらないんだ。	
\\	うん、入社早々大きな仕事を任されちゃって、全然終わらないんだ。 
\\	そうか、大変だなあ。でも、すぐに大事な仕事を任されるなんて、期待されてるね。	
\\	そうか、大変だなあ。でも、すぐに大事な仕事を任されるなんて、期待されてるね。 
\\	うーん、どうかなあ…。あ、それより、何か用事だった?	
\\	うーん、どうかなあ…。あ、それより、何か用事だった? 
\\	いや、また週末にでも食事にどうかなあ、と思って。	
\\	いや、また週末にでも食事にどうかなあ、と思って。 
\\	そうねえ、この仕事のプレゼンが金曜日なの。だから、週末なら行かれると思うわ。	
\\	そうねえ、この仕事のプレゼンが金曜日なの。だから、週末なら行かれると思うわ。 
\\	そうか、よかった!じゃあ、また電話するよ。忙しいところごめんね。頑張って。	
\\	そうか、よかった!じゃあ、また電話するよ。忙しいところごめんね。頑張って。 
\\	(会社の電話が鳴る)	
\\	はい、デザインオフィスJでございます。	
\\	はい、デザインオフィスJでございます。 
\\	赤坂のゴリだけど。	
\\	赤坂のゴリだけど。 
\\	あ、担当の村上でございます!	
\\	おお、アンタか。遅くまで仕事させて、悪いねえ。ちょっと確認したいんだけどさ、俺、昨日の電話で表紙は青がいいって言ったっけ?	
\\	おお、アンタか。遅くまで仕事させて、悪いねえ。ちょっと確認したいんだけどさ、俺、昨日の電話で表紙は青がいいって言ったっけ? 
\\	ええ。	
\\	ええ。 
\\	やっぱりそうかー。あのさ、今から赤に変えられる?	
\\	やっぱりそうかー。あのさ、今から赤に変えられる? 
\\	えっ…と、承知しました。ご連絡ありがとうございました。	
\\	えっ…と、承知しました。ご連絡ありがとうございました。 
\\	(電話切れる)	
\\	うわぁ、今さら色を変えるなんて、本当に金曜日に間に合うのかしら。あ、いけない!もうこんな時間!終電に間に合わなーい!	
\\	うわぁ、今さら色を変えるなんて、本当に金曜日に間に合うのかしら。あ、いけない!もうこんな時間!終電に間に合わなーい! 
\\	下の子
\\	授業参観
\\	調子
\\	異文化
\\	たまには
\\	あのさあ、今日、下の子の授業参観に初めて行ったんだけど、すごく面白かったよ。	
\\	あのさあ、今日、下の子の授業参観に初めて行ったんだけど、すごく面白かったよ。 
\\	へー。どんなだったの。	
\\	へー。どんなだったの。 
\\	はじめにボールが3つ、ありました。あとで4つ、ふえました。ぜんぶでいくつになったかな?。。わかる人!	
\\	はじめにボールが3つ、ありました。あとで4つ、ふえました。ぜんぶでいくつになったかな?。。わかる人! 
\\	ハイハイハイ!	
\\	ハイハイハイ! 
\\	じゃあ、レオ君。わかるかな。	
\\	じゃあ、レオ君。わかるかな。 
\\	はい。7です。	
\\	はい。7です。 
\\	そう!ほんとう?ほかにだれか、ちがうと思う人は?	
\\	そう!ほんとう?ほかにだれか、ちがうと思う人は? 
\\	ハイハイハイ!	
\\	ハイハイハイ! 
\\	じゃあ、マサミちゃん。	
\\	じゃあ、マサミちゃん。 
\\	先生、7です。	
\\	先生、7です。 
\\	ああ、そう。7ですか。ほかには?ほかの答えは?	
\\	ああ、そう。7ですか。ほかには?ほかの答えは? 
\\	ハイハイハイ!	
\\	ハイハイハイ! 
\\	ユウキくん、わかるかな。	
\\	ユウキくん、わかるかな。 
\\	はい。7だと思います。	
\\	はい。7だと思います。 
\\	う〜ん。7かあ。ほかには?答えわかる人は?	
\\	う〜ん。7かあ。ほかには?答えわかる人は? 
\\	ってまあ、ずっとこんな調子で、さらに2回も、子どもたちのハイハイハイ!が続いてさあ。驚いちゃったよ。	
\\	ってまあ、ずっとこんな調子で、さらに2回も、子どもたちのハイハイハイ!が続いてさあ。驚いちゃったよ。 
\\	なるほどね。小学一年生ってそんなもんなのよね。	
\\	なるほどね。小学一年生ってそんなもんなのよね。 
\\	まさに、異文化コミュニケーション。子どもの世界ってすごいな。	
\\	まさに、異文化コミュニケーション。子どもの世界ってすごいな。 
\\	お兄ちゃんも仕事ばっかりじゃなくて、たまには子どもとゆっくり過ごしなさいって。	
\\	お兄ちゃんも仕事ばっかりじゃなくて、たまには子どもとゆっくり過ごしなさいって。 
\\	はーい。	
\\	はーい。 
\\	野球観戦
\\	勝手
\\	もっての他
\\	他球団
\\	応援団員
\\	手数
\\	仕方ない
\\	恐縮
\\	側
\\	候補日
\\	無料
\\	招待
\\	そうしてもらえると、 大変ありがたいです
\\	うちの子供から聞いたんですが、今度、ヤクルトが小学生を無料で野球観戦に招待してくれるんですって。一緒に行きませんか。	
\\	うちの子供から聞いたんですが、今度、ヤクルトが小学生を無料で野球観戦に招待してくれるんですって。一緒に行きませんか。 
\\	すごいですね。無料招待なんて。。。ぜひ、ご一緒させてください。	
\\	すごいですね。無料招待なんて。。。ぜひ、ご一緒させてください。 
\\	いくつか候補日があるんですが、どの試合にしますか?やっぱり、巨人戦かな。	
\\	いくつか候補日があるんですが、どの試合にしますか?やっぱり、巨人戦かな。 
\\	そうしましょう!私は大の巨人ファンなので、そうしてもらえると、大変ありがたいです!	
\\	そうしましょう!私は大の巨人ファンなので、そうしてもらえると、大変ありがたいです! 
\\	山田さんって、巨人ファンだったんですね。でも、ヤクルト側の席ですよ。それでもいいですか?	
\\	山田さんって、巨人ファンだったんですね。でも、ヤクルト側の席ですよ。それでもいいですか? 
\\	えー!それは、巨人ファンとしては耐えられないです。私だけ巨人側に座ってもいいですか?	
\\	えー!それは、巨人ファンとしては耐えられないです。私だけ巨人側に座ってもいいですか? 
\\	うーーん、多分それはできないと思いますよ。	
\\	うーーん、多分それはできないと思いますよ。 
\\	あのぉ、こんなことお願いするのは、大変恐縮なのですが、うちの子供だけ連れて行ってもらえませんか?	
\\	あのぉ、こんなことお願いするのは、大変恐縮なのですが、うちの子供だけ連れて行ってもらえませんか? 
\\	仕方ないですね。じゃぁ、私が連れて行きますよ。	
\\	仕方ないですね。じゃぁ、私が連れて行きますよ。 
\\	すみません。お手数をお掛けします。実は、私、巨人の応援団員なので、他球団の席に座るなんてことはもっての他なんです。勝手を言って申し訳ありませんが、どうぞよろしくお願いします。	
\\	すみません。お手数をお掛けします。実は、私、巨人の応援団員なので、他球団の席に座るなんてことはもっての他なんです。勝手を言って申し訳ありませんが、どうぞよろしくお願いします。 
\\	アンタ
\\	ロマン
\\	エロじじい
\\	やっぱり
\\	一応
\\	れっきとした
\\	乙女
\\	しびれる
\\	いつ読んでもドラゴン・ボールは最高だよな。	
\\	いつ読んでもドラゴン・ボールは最高だよな。 
\\	アンタ、授業中に漫画読んでたでしょ。	
\\	アンタ、授業中に漫画読んでたでしょ。 
\\	女には、ドラゴン・ボールのロマンはわからないだろうなあ。「カ・メ・ハ・メ・波ぁぁ~!」やっぱりコレだよな。	
\\	"女には、ドラゴン・ボールのロマンはわからないだろうなあ。「カ・メ・ハ・メ・波ぁぁ~!」やっぱりコレだよな。 
\\	何よ。悟空なんて子供じゃない。亀仙人はただのエロじじいだし。やっぱり、北斗の拳よ。	
\\	何よ。悟空なんて子供じゃない。亀仙人はただのエロじじいだし。やっぱり、北斗の拳よ。 
\\	北斗の拳~!?	
\\	北斗の拳~!? 
\\	「お前はもう、死んでいる。」ケンシロウ格好良すぎ!	
\\	"「お前はもう、死んでいる。」ケンシロウ格好良すぎ! 
\\	"お前 ""一応、女
\\	だろ。「動物のお医者さん」とか、普通はそういうんだろ。
\\	"お前 ""一応、女
\\	だろ。「動物のお医者さん」とか、普通はそういうんだろ。 
\\	何よ、その一応、女って。れっきとした乙女ですう。「あたたたたたたたたたたたたたたたたたたたた・・・」ケンシロウの北斗百裂拳ってほんとうに最高だわ。しびれるう~。	
\\	"何よ、その一応、女って。れっきとした乙女ですう。「あたたたたたたたたたたたたたたたたたたたた・・・」ケンシロウの北斗百裂拳ってほんとうに最高だわ。しびれるう~。 
\\	全巻
\\	マジ
\\	四次元
\\	わくわく
\\	ひみつ道具
\\	断然
\\	この間タツオの家に遊びに行ったら、タツオ、「ドラえもん」全巻持っててさあ。	
\\	"この間タツオの家に遊びに行ったら、タツオ、「ドラえもん」全巻持っててさあ。 
\\	マジで。すごいねー。	
\\	マジで。すごいねー。 
\\	お父さんに買ってもらったんだって。	
\\	お父さんに買ってもらったんだって。 
\\	あの、ドラえもんが四次元ポケットからひみつ道具を出す時の、あれ。「(ドラえもんになりきって)タケコプター」ってやつ。あれ聞くとすごくワクワクするよね。	
\\	"あの、ドラえもんが四次元ポケットからひみつ道具を出す時の、あれ。「(ドラえもんになりきって)タケコプター」ってやつ。あれ聞くとすごくワクワクするよね。 
\\	ひみつ道具でしょ。すっごいワクワクする。私はね、「(ドラえもんになりきって)どこでもドア」が一番わくわくするね。	
\\	"ひみつ道具でしょ。すっごいワクワクする。私はね、「(ドラえもんになりきって)どこでもドア」が一番わくわくするね。 
\\	恋人にするなら、シズカちゃんより断然ドラミちゃんだね。	
\\	恋人にするなら、シズカちゃんより断然ドラミちゃんだね。 
\\	なんで。	
\\	なんで。 
\\	だって、四次元ポケット持ってるじゃん。ドラミちゃん歌もうまいし。楽しいって。	
\\	だって、四次元ポケット持ってるじゃん。ドラミちゃん歌もうまいし。楽しいって。 
\\	上陸
\\	可能性
\\	最悪
\\	七月になって台風が多いよな	
\\	七月になって台風が多いよな 
\\	そうそう、七月から九月だけで30個ぐらいは日本に上陸するもんね。	
\\	そうそう、七月から九月だけで30個ぐらいは日本に上陸するもんね。 
\\	大雨だし、強風だし、、、けど台風が来れば学校が休みになる可能性も高いけどね。どこにも行けないけど。。	
\\	大雨だし、強風だし、、、けど台風が来れば学校が休みになる可能性も高いけどね。どこにも行けないけど。。 
\\	どうやら、ニ、三日したら、台風4号が来るかもしれないらしいよ。	
\\	どうやら、ニ、三日したら、台風4号が来るかもしれないらしいよ。 
\\	えーっ!今度の日曜日、釣りに行こうと思ってんのに。。。ダメじゃん。。。	
\\	えーっ!今度の日曜日、釣りに行こうと思ってんのに。。。ダメじゃん。。。 
\\	そりゃー、しょうがないわ。	
\\	そりゃー、しょうがないわ。 
\\	しかも台風が過ぎた後は台風一過ですごい晴れて、めっちゃ気温が上がって。良いことないわ。	
\\	しかも台風が過ぎた後は台風一過ですごい晴れて、めっちゃ気温が上がって。良いことないわ。 
\\	そうそう。最悪。	
\\	そうそう。最悪。 
\\	今朝
\\	残業
\\	かどうか
\\	やり直す
\\	修正
\\	要望
\\	内緒
\\	一段落
\\	落とす
\\	おはようございまーす。	
\\	おはようございまーす。 
\\	あ、野村さんおはようございます。	
\\	あ、野村さんおはようございます。 
\\	アグネスさん、今朝も早いね。昨日も遅くまで残業していたでしょ?	
\\	アグネスさん、今朝も早いね。昨日も遅くまで残業していたでしょ? 
\\	昨日、ゴリさんから電話があって。。。色を変えなきゃならなくなったんです。だから間に合うかどうか心配で。	
\\	昨日、ゴリさんから電話があって。。。色を変えなきゃならなくなったんです。だから間に合うかどうか心配で。 
\\	ええっ!今から?それはひどいねー。	
\\	ええっ!今から?それはひどいねー。 
\\	違う色でやり直してたら、全体を修正しないわけにはいかなくなってしまって。。。	
\\	違う色でやり直してたら、全体を修正しないわけにはいかなくなってしまって。。。 
\\	大丈夫?プレゼンは明日でしょう?	
\\	大丈夫?プレゼンは明日でしょう? 
\\	でも、クライアントの要望だから聞かざるをえないですよねぇ。	
\\	でも、クライアントの要望だから聞かざるをえないですよねぇ。 
\\	そうねえ。。。そういう困ったクライアント、多いのね。あ、今の、社長には内緒よ。	
\\	そうねえ。。。そういう困ったクライアント、多いのね。あ、今の、社長には内緒よ。 
\\	社長のお友達ですもんね。	
\\	社長のお友達ですもんね。 
\\	だから、言うことを聞かないわけにいかないんだけどね。	
\\	だから、言うことを聞かないわけにいかないんだけどね。 
\\	ああー、どうしよう。間に合うかしら。	
\\	ああー、どうしよう。間に合うかしら。 
\\	僕、今日の仕事が一段落したら手伝ってあげるよ。この仕事は落とすわけにはいかないでしょう。	
\\	僕、今日の仕事が一段落したら手伝ってあげるよ。この仕事は落とすわけにはいかないでしょう。 
\\	いいんですか?ごめんなさい。	
\\	いいんですか?ごめんなさい。 
\\	いいんだよ。入社したばかりでこんなきつい仕事、大変だろう。後で誰か他にも手伝えるかどうか聞いてみるよ。	
\\	いいんだよ。入社したばかりでこんなきつい仕事、大変だろう。後で誰か他にも手伝えるかどうか聞いてみるよ。 
\\	ありがとうございます!	
\\	ありがとうございます! 
\\	オタク
\\	ひるむ
\\	日本刀
\\	チンピラ
\\	筆頭
\\	モヒカン
\\	ぶつかる
\\	陰
\\	家路
\\	狩り
\\	覚悟
\\	いてっ!	
\\	いてっ! 
\\	(いきなり陰から出てきて、ぶつかってきたのはあなたじゃないですか。なんなんですか)	
\\	(いきなり陰から出てきて、ぶつかってきたのはあなたじゃないですか。なんなんですか) 
\\	肩がいてーなー。どうしてくれんだよー。	
\\	肩がいてーなー。どうしてくれんだよー。 
\\	3人組か。モヒカンを筆頭に全員チンピラ風。これって・・・世に言うオタク狩りってやつ!?)	
\\	3人組か。モヒカンを筆頭に全員チンピラ風。これって・・・世に言うオタク狩りってやつ!?) 
\\	許してほしければ、ドラゴンボールのマンガ全部と
\\	を渡しな。	
\\	許してほしければ、ドラゴンボールのマンガ全部と
\\	を渡しな。 
\\	(あっ、やっぱりオタク狩りってやつですか。。。)	
\\	(あっ、やっぱりオタク狩りってやつですか。。。) 
\\	(男、持っていた日本刀をだす。ひるむ若い男達。)	
\\	おっオタクが、そっ、そんな物だしていいのか。	
\\	おっオタクが、そっ、そんな物だしていいのか。 
\\	覚悟しろよ。	
\\	覚悟しろよ。 
\\	方面
\\	グルメ天国
\\	独特
\\	食べ尽くす
\\	記入する
\\	航空券
\\	予約
\\	町並み
\\	うだるような
\\	いらっしゃいませ。	
\\	いらっしゃいませ。 
\\	来月にアジア方面に旅行を考えているんですが、どこかオススメの場所はありませんか。	
\\	来月にアジア方面に旅行を考えているんですが、どこかオススメの場所はありませんか。 
\\	何日間くらいを考えていらっしゃいますか。	
\\	何日間くらいを考えていらっしゃいますか。 
\\	一週間くらいです。	
\\	一週間くらいです。 
\\	ちょっと調べてみますね。少々お待ちください。二名様でよろしいですか。	
\\	ちょっと調べてみますね。少々お待ちください。二名様でよろしいですか。 
\\	はい。お願いします。(三、四分後)	
\\	はい。お願いします。(三、四分後) 
\\	お待たせしました。今、一番のオススメは中国の上海ですね。7泊8日ホテル付きで1名様68,000円です。	
\\	お待たせしました。今、一番のオススメは中国の上海ですね。7泊8日ホテル付きで1名様68,000円です。 
\\	上海ですか。	
\\	上海ですか。 
\\	上海は今人気ですよ。ヨーロッパ文化と中国文化がミックスされた町並みは上海独特できれいですし、上海はアジアのグルメ天国と言われてますから、食べ物もすごくおいしいですよ。	
\\	上海は今人気ですよ。ヨーロッパ文化と中国文化がミックスされた町並みは上海独特できれいですし、上海はアジアのグルメ天国と言われてますから、食べ物もすごくおいしいですよ。 
\\	どうする?(同席のユミに聞く)	
\\	どうする?(同席のユミに聞く) 
\\	真夏の上海はものすごく暑いって聞くけど大丈夫かしら。	
\\	真夏の上海はものすごく暑いって聞くけど大丈夫かしら。 
\\	この時期、アジアはどこでもうだるような暑さだよ。	
\\	この時期、アジアはどこでもうだるような暑さだよ。 
\\	そうよね。中華料理食べ尽くしっていうのもいいわねえ。	
\\	そうよね。中華料理食べ尽くしっていうのもいいわねえ。 
\\	じゃあ、決まり!すいません、では上海でお願いします。	
\\	じゃあ、決まり!すいません、では上海でお願いします。 
\\	ありがとうございます。では早速予約を入れたいと思いますので、この用紙に必要事項をご記入ください。	
\\	ありがとうございます。では早速予約を入れたいと思いますので、この用紙に必要事項をご記入ください。 
\\	わかりました。	
\\	わかりました。 
\\	反省
\\	認識する
\\	無駄に
\\	へこむ
\\	廃棄
\\	廃棄処分
\\	ぎりぎり
\\	賞味期限
\\	浸透する
\\	配給
\\	福祉団体
\\	非営利
\\	引き取る
\\	無償
\\	過程
\\	製造販売
\\	企業
\\	社会派
\\	フードバンクっていう活動、知ってる?この間、たまたま雑誌で読んだんだけど、それを読んだら、今までの私のスーパーでの食品の買い方を反省させられちゃったわ。	
\\	フードバンクっていう活動、知ってる?この間、たまたま雑誌で読んだんだけど、それを読んだら、今までの私のスーパーでの食品の買い方を反省させられちゃったわ。 
\\	フードバンクって何のこと?	
\\	フードバンクって何のこと? 
\\	フードバンクって、企業が食品を製造販売する過程で、商品にならなくなってしまったものを、無償で引き取って、それを非営利の福祉団体などを通して、食べ物に困っている人々へ無償で配給しようとするシステムなんですって。	
\\	フードバンクって、企業が食品を製造販売する過程で、商品にならなくなってしまったものを、無償で引き取って、それを非営利の福祉団体などを通して、食べ物に困っている人々へ無償で配給しようとするシステムなんですって。 
\\	へぇー。なるほどね。	
\\	へぇー。なるほどね。 
\\	アメリカでは、ずいぶん浸透しつつある活動らしいよ。	
\\	アメリカでは、ずいぶん浸透しつつある活動らしいよ。 
\\	ふーん、だけど、なんで、それがスーパーでの食品の買い方と関係があるわけ?	
\\	ふーん、だけど、なんで、それがスーパーでの食品の買い方と関係があるわけ? 
\\	たとえば、牛乳を買うときは、賞味期限ができるだけ先の牛乳を選んで買ってたんだけど、みんながそういう買い方をするから、賞味期限ぎりぎりの牛乳ばかりが残っちゃって、それが結局廃棄処分になるわけよ。	
\\	たとえば、牛乳を買うときは、賞味期限ができるだけ先の牛乳を選んで買ってたんだけど、みんながそういう買い方をするから、賞味期限ぎりぎりの牛乳ばかりが残っちゃって、それが結局廃棄処分になるわけよ。 
\\	なるほどね。みんながなるべく賞味期限が近いものを買っていけば、それだけ廃棄される食べ物が少なくなるもんな。	
\\	なるほどね。みんながなるべく賞味期限が近いものを買っていけば、それだけ廃棄される食べ物が少なくなるもんな。 
\\	そういうことよ。それに、パッケージがちょっとでもへこんでたりする商品は買わなかったりするでしょ。だから、中身に何の問題もなくても、パッケージが汚れたり傷ついたりした商品は廃棄処分されているのが現状らしいわ。	
\\	そういうことよ。それに、パッケージがちょっとでもへこんでたりする商品は買わなかったりするでしょ。だから、中身に何の問題もなくても、パッケージが汚れたり傷ついたりした商品は廃棄処分されているのが現状らしいわ。 
\\	確かに、わざわざパッケージが汚れている商品は買わないよな。	
\\	確かに、わざわざパッケージが汚れている商品は買わないよな。 
\\	フードバンクの活動が浸透することは大切だけど、それと同時に、私たち消費者も食品が無駄に廃棄されていることを認識する必要があるわね。	
\\	フードバンクの活動が浸透することは大切だけど、それと同時に、私たち消費者も食品が無駄に廃棄されていることを認識する必要があるわね。 
\\	なんか、今日の君は、ずいぶん社会派だね。	
\\	なんか、今日の君は、ずいぶん社会派だね。 
\\	二度寝
\\	遠回り
\\	路線
\\	めどが立たない
\\	再開
\\	見合わせる
\\	全線
\\	発生
\\	人身事故
\\	長蛇の列
\\	うーん…あ、もう朝?昨日も終電まで残業してたから、眠い…あと少しだけ寝よう…。	
\\	うーん…あ、もう朝?昨日も終電まで残業してたから、眠い…あと少しだけ寝よう…。 
\\	(駅の音)	
\\	お客様にお知らせします。先ほど人身事故が発生いたしました関係で、ただいま全線で運転を見合わせております。	
\\	お客様にお知らせします。先ほど人身事故が発生いたしました関係で、ただいま全線で運転を見合わせております。 
\\	ええーっ!人身事故!いつ動くのかしら?	
\\	ええーっ!人身事故!いつ動くのかしら? 
\\	あ、駅員さん、すみません。運転再開まで、どれくらいかかりそうですか?	
\\	あ、駅員さん、すみません。運転再開まで、どれくらいかかりそうですか? 
\\	申し訳ございません、ただいま再開のめどが立っておりません。	
\\	申し訳ございません、ただいま再開のめどが立っておりません。 
\\	そんなあ!私、今日だけは遅刻できないのに…。	
\\	そんなあ!私、今日だけは遅刻できないのに…。 
\\	バスで別の路線の駅に向かっていただけないでしょうか。	
\\	バスで別の路線の駅に向かっていただけないでしょうか。 
\\	そんな遠回りしている場合じゃないのよー。仕方がないわ、タクシーで行こう。	
\\	そんな遠回りしている場合じゃないのよー。仕方がないわ、タクシーで行こう。 
\\	うわあ、タクシー乗り場も長蛇の列!これじゃ間に合わない。あ、そうだ、会社に電話しなきゃ。	
\\	うわあ、タクシー乗り場も長蛇の列!これじゃ間に合わない。あ、そうだ、会社に電話しなきゃ。 
\\	はい、デザインオフィスJでございます。 
\\	もしもし、村上アグネスです。実は、人身事故で電車が止まってしまって、会社に遅れそうなんです。	
\\	もしもし、村上アグネスです。実は、人身事故で電車が止まってしまって、会社に遅れそうなんです。 
\\	ええっ!だって今日、10時からプレゼンだろ?	
\\	ええっ!だって今日、10時からプレゼンだろ? 
\\	渋滞
\\	先頭
\\	昼過ぎ
\\	下道
\\	裏道
\\	彼氏
\\	彼女
\\	まじ
\\	この先渋滞中だってさー。	
\\	この先渋滞中だってさー。 
\\	えー、どれぐらい渋滞してるのー。	
\\	えー、どれぐらい渋滞してるのー。 
\\	ラジオでは談合坂を先頭に25キロの渋滞。。。たぶん、河口湖に着くのは昼過ぎになっちまうかも。	
\\	ラジオでは談合坂を先頭に25キロの渋滞。。。たぶん、河口湖に着くのは昼過ぎになっちまうかも。 
\\	まーじーでー。下道(したみち)で行ったらどんくらいかかる?	
\\	まーじーでー。下道(したみち)で行ったらどんくらいかかる? 
\\	んー。。これくらいの渋滞だったら、上で行っても、下道(したみち)で行くのと変わんねぇんじゃねーの。	
\\	んー。。これくらいの渋滞だったら、上で行っても、下道(したみち)で行くのと変わんねぇんじゃねーの。 
\\	裏道とか無いの?	
\\	裏道とか無いの? 
\\	わからん。。。	
\\	わからん。。。 
\\	だから休みの日に出かけるのは嫌なんだよねー。	
\\	だから休みの日に出かけるのは嫌なんだよねー。 
\\	逆立ち
\\	恥ずかしい
\\	人様
\\	手塩にかける
\\	跳ぶ
\\	一人前
\\	習得
\\	技
\\	でかした
\\	着地する
\\	完璧(な)
\\	よし、太郎、そこで壁をジャンプして逆立ちで着地するんだ!	
\\	よし、太郎、そこで壁をジャンプして逆立ちで着地するんだ! 
\\	キーキーキー	
\\	キーキーキー 
\\	いいぞ、太郎!でかした!!この技を習得するのに一年もかかったなぁ。	
\\	いいぞ、太郎!でかした!!この技を習得するのに一年もかかったなぁ。 
\\	だが、これでお前も一人前のサルの一員だな。	
\\	だが、これでお前も一人前のサルの一員だな。 
\\	逆立ちはできるし、こんなに高い壁だって跳べるんだ。	
\\	逆立ちはできるし、こんなに高い壁だって跳べるんだ。 
\\	沸く
\\	アイロンをかける
\\	営業
\\	家出
\\	調教
\\	とっ捕まえる
\\	エテ公
\\	もうそろそろ、メシの時間じゃねーの?おい、太郎、メシはまだか?!早くしろよーっ。	
\\	もうそろそろ、メシの時間じゃねーの?おい、太郎、メシはまだか?!早くしろよーっ。 
\\	キー!!	
\\	キー!! 
\\	おっ、今日はハンバーグかぁ。今日は魚の気分だったんだけど。。。まぁいいや…	
\\	おっ、今日はハンバーグかぁ。今日は魚の気分だったんだけど。。。まぁいいや… 
\\	おい、太郎。お前の毛がちょっと入ってんだけど。。。	
\\	おい、太郎。お前の毛がちょっと入ってんだけど。。。 
\\	キー?!	
\\	キー?! 
\\	あっ、風呂はもう沸いてんのか?メシ食ったら風呂に入りたいんだから、さっさと用意しとけよ。	
\\	キーキーキー!	
\\	キーキーキー! 
\\	あっ!ワイシャツにアイロンはかけといたのか?明日は大事な営業があるんだから、しっかりアイロンかけたシャツを着ていかないと。	
\\	あっ!ワイシャツにアイロンはかけといたのか?明日は大事な営業があるんだから、しっかりアイロンかけたシャツを着ていかないと。 
\\	キーキーキー……	
\\	キーキーキー…… 
\\	ええぇ???!!	
\\	ええぇ???!! 
\\	あっ、そうだそうだ、それから。。。	
\\	あっ、そうだそうだ、それから。。。 
\\	バタン(ドアが閉まる)	
\\	搭乗手続き
\\	締め切る
\\	拝見する
\\	オカン
\\	当機
\\	超過料金
\\	整う
\\	お預かりする
\\	成田発、バンクーバー行きのお客様の搭乗手続きをまもなく締め切らせていただきます!	
\\	成田発、バンクーバー行きのお客様の搭乗手続きをまもなく締め切らせていただきます! 
\\	待ったー!!!!まてまてまてまて!ここにもう一人、搭乗者が居ます!	
\\	待ったー!!!!まてまてまてまて!ここにもう一人、搭乗者が居ます! 
\\	お客様、お急ぎ下さい、あと三十分(30分)でバンクーバー行きが出発してしまいます。パスポートと搭乗券を拝見させていただきます。	
\\	お客様、お急ぎ下さい、あと三十分(30分)でバンクーバー行きが出発してしまいます。パスポートと搭乗券を拝見させていただきます。 
\\	あいよっ!	
\\	あいよっ! 
\\	はい、お預かりします。。。。お客様、このパスポート、どうやらお母様かどなたかのパスポートではないでしょうか?	
\\	はい、お預かりします。。。。お客様、このパスポート、どうやらお母様かどなたかのパスポートではないでしょうか? 
\\	はっ!やべー、しまった、オカンのパスポート持って来ちゃったよ!待って確かにこの中に入れたはずなんだ。。。ほらあった!!これでしょ、これ、はい!!	
\\	はっ!やべー、しまった、オカンのパスポート持って来ちゃったよ!待って確かにこの中に入れたはずなんだ。。。ほらあった!!これでしょ、これ、はい!! 
\\	はい、確かに。それではお荷物の方なんですが、、、三つ(3つ)でよろしいですか?	
\\	はい、確かに。それではお荷物の方なんですが、、、三つ(3つ)でよろしいですか? 
\\	えーっとはい。	
\\	えーっとはい。 
\\	恐れ入りますお客様。当機でのお一人様の最大積載重量は三十キロ
\\	となっております。	
\\	恐れ入りますお客様。当機でのお一人様の最大積載重量は三十キロ
\\	となっております。 
\\	しょうがない。。。はい大丈夫ですよ。	
\\	しょうがない。。。はい大丈夫ですよ。 
\\	はい、ではお客様、出発の準備はすべて整いました。	
\\	はい、ではお客様、出発の準備はすべて整いました。 
\\	窮屈
\\	たまらない
\\	税関
\\	抜ける
\\	しんどい
\\	干からびる
\\	飽きる
\\	いきなり
\\	ゲート45、ゲート45・・・どこだぁ?!。。。あ、あった!	
\\	ゲート45、ゲート45・・・どこだぁ?!。。。あ、あった! 
\\	飛行機出発	
\\	バンクーバー着	
\\	ああああぁぁぁあ(あくび)、やっとバンクーバーに着いた。エコノミークラスは窮屈でたまらんなぁ。まぁ、なんとか税関もすぐに抜けられたし、安心安心。	
\\	ああああぁぁぁあ(あくび)、やっとバンクーバーに着いた。エコノミークラスは窮屈でたまらんなぁ。まぁ、なんとか税関もすぐに抜けられたし、安心安心。 
\\	イヤーしんどかった。。。肩、腰が痛い。。トランクの中に10時間て、どんだけー!	
\\	イヤーしんどかった。。。肩、腰が痛い。。トランクの中に10時間て、どんだけー! 
\\	家来
\\	今宵
\\	幕府
\\	開国
\\	城主
\\	ワシ
\\	ふんどし
\\	貸し切る
\\	盛大に
\\	殿
\\	殿ぉ!お誕生日、おめでとうございます!今日は11月23日。マーキー様の50才の誕生日ですぞ!こうしては居られません。盛大に殿のお誕生日を祝いましょう。	
\\	そうよのー、なにかクレージーな事をせねばの。	
\\	そうよのー、なにかクレージーな事をせねばの。 
\\	何か、やりたいこと、欲しい物はありますか?	
\\	何か、やりたいこと、欲しい物はありますか? 
\\	それでは、姫路城を貸し切ってしまおう。	
\\	それでは、姫路城を貸し切ってしまおう。 
\\	。。。。。えっ?姫路城をですか?何を言ってるんですか、殿?	
\\	。。。。。えっ?姫路城をですか?何を言ってるんですか、殿? 
\\	いや、だから、姫路城でパーティーをせねば。ふんどしで。姫路城城主、酒井忠顕(さかいただてる)にはワシから申しておく。	
\\	いや、だから、姫路城でパーティーをせねば。ふんどしで。姫路城城主、酒井忠顕(さかいただてる)にはワシから申しておく。 
\\	姫路城で。ふんどしで。殿、殿は最近お忙しくお疲れの様子。どうやら熱でも出されているのでしょう。少しお休みになられた方が良いのではないかと。。。	
\\	姫路城で。ふんどしで。殿、殿は最近お忙しくお疲れの様子。どうやら熱でも出されているのでしょう。少しお休みになられた方が良いのではないかと。。。 
\\	(パーティー 
\\	姫路城)	
\\	どうだ、家来、楽しいだろ!!	
\\	どうだ、家来、楽しいだろ!! 
\\	はい、大変楽しゅうございます!!	
\\	はい、大変楽しゅうございます!! 
\\	あ、ペリー君!来てたのかね!どうだ、これが日本のパーティーだ。すごいだろ?!	
\\	開国シテクダサーイよ~!	
\\	開国シテクダサーイよ~! 
\\	ん?開国?うむうむ、わかったわかった。わしから幕府の方に開国するように申しておこう。今宵は最後まで楽しむぞよ!!!	
\\	ん?開国?うむうむ、わかったわかった。わしから幕府の方に開国するように申しておこう。今宵は最後まで楽しむぞよ!!! 
\\	診察
\\	混み合う
\\	胃
\\	断続的
\\	吐き気
\\	下痢
\\	症状
\\	食欲
\\	喉
\\	はい、こちら北山クリニックです。	
\\	はい、こちら北山クリニックです。 
\\	本日うかがいたいのですが、診察時間は何時までですか?	
\\	本日うかがいたいのですが、診察時間は何時までですか? 
\\	本日は土曜日ですので、診察の受付は午後1時までとなっております。土曜日は混み合いますので、お早めにおいでください。	
\\	本日は土曜日ですので、診察の受付は午後1時までとなっております。土曜日は混み合いますので、お早めにおいでください。 
\\	(待合室)	
\\	村上さん、村上アグネスさーん。	
\\	村上さん、村上アグネスさーん。 
\\	あ、はい。	
\\	あ、はい。 
\\	診察室へどうぞ。	
\\	診察室へどうぞ。 
\\	村上さん、本日はどうされました?	
\\	村上さん、本日はどうされました? 
\\	あのう、昨日の夜から胃が痛くて…。	
\\	あのう、昨日の夜から胃が痛くて…。 
\\	ふむ。どんな痛みですか?しくしく痛いとか、締め付けられるように痛いとか…。	
\\	ふむ。どんな痛みですか?しくしく痛いとか、締め付けられるように痛いとか…。 
\\	ええ、断続的にキリキリ痛みます。	
\\	ええ、断続的にキリキリ痛みます。 
\\	吐き気や下痢など、他に症状はありますか?	
\\	吐き気や下痢など、他に症状はありますか? 
\\	いえ、特にありません。	
\\	いえ、特にありません。 
\\	えーと、昨日何か変わったものを食べたりしましたか。	
\\	えーと、昨日何か変わったものを食べたりしましたか。 
\\	いいえ、昨日はほとんど何も食べていないんです。	
\\	いいえ、昨日はほとんど何も食べていないんです。 
\\	おや、痛くて食べられなかったんですか?	
\\	おや、痛くて食べられなかったんですか? 
\\	いえ、仕事でちょっとトラブルがあったもので、食欲がなくて。大事なプレゼンの日だったんですけど、遅刻しちゃったんです。職場の人に迷惑をかけた上に、その仕事は結局契約が取れなくて…ほとんど徹夜で頑張ったのに、無駄になっちゃったと思ったら、もう何も喉を通らなくて。	
\\	いえ、仕事でちょっとトラブルがあったもので、食欲がなくて。大事なプレゼンの日だったんですけど、遅刻しちゃったんです。職場の人に迷惑をかけた上に、その仕事は結局契約が取れなくて…ほとんど徹夜で頑張ったのに、無駄になっちゃったと思ったら、もう何も喉を通らなくて。 
\\	ううーむ。これは、ストレスから来る軽い胃炎かもしれませんなあ。	
\\	ううーむ。これは、ストレスから来る軽い胃炎かもしれませんなあ。 
\\	学会
\\	研究発表
\\	勤務
\\	新鮮な
\\	牡蠣
\\	容態
\\	粥
\\	はい、もしもし。	
\\	はい、もしもし。 
\\	もしもし?橋本だけど、アグネスさん、昨日電話くれた?	
\\	もしもし?橋本だけど、アグネスさん、昨日電話くれた? 
\\	うん、5回くらいかけた。	
\\	うん、5回くらいかけた。 
\\	ゴメンゴメン、この週末は学会で広島に来ているんだ。	
\\	ゴメンゴメン、この週末は学会で広島に来ているんだ。 
\\	ふーん。	
\\	ふーん。 
\\	昨日も一日研究発表があって…それさえなければすぐ連絡できたんだけど、何か大事な用事だった?	
\\	昨日も一日研究発表があって…それさえなければすぐ連絡できたんだけど、何か大事な用事だった? 
\\	もう用事は済んだからいいの。	
\\	もう用事は済んだからいいの。 
\\	えっ、何それ?	
\\	えっ、何それ? 
\\	雄介さん、病院勤務だって言ってたでしょ。だからお医者さんを紹介してもらおうかと思ったの。	
\\	雄介さん、病院勤務だって言ってたでしょ。だからお医者さんを紹介してもらおうかと思ったの。 
\\	医者?どうしたの?	
\\	医者?どうしたの? 
\\	胃が痛いの。ストレス性胃炎だって。薬を飲んだら、だいぶよくなったけど、今日は一日家にいるわ。	
\\	胃が痛いの。ストレス性胃炎だって。薬を飲んだら、だいぶよくなったけど、今日は一日家にいるわ。 
\\	胃炎?そうか、タイミング悪いなあ。	
\\	胃炎?そうか、タイミング悪いなあ。 
\\	何が?	
\\	何が? 
\\	いや、広島から新鮮な牡蠣を送ろうかと思ったんだけど、胃によくないよね。	
\\	いや、広島から新鮮な牡蠣を送ろうかと思ったんだけど、胃によくないよね。 
\\	牡蠣!?	
\\	牡蠣!? 
\\	あ、いや、アグネスさんの容態も心配だよ。大丈夫?	
\\	あ、いや、アグネスさんの容態も心配だよ。大丈夫? 
\\	(電話切れる)	
\\	あ、アグネスさん! 
\\	顔色
\\	気が重い
\\	魚介類
\\	炭火焼
\\	取引先
\\	信用
\\	直送
\\	おはようございます。	
\\	おはようございます。 
\\	おはようございまーす。あ、アグネスさん、社長がお昼前に社長室へ来るように、って。大丈夫?顔色悪いよ。	
\\	おはようございまーす。あ、アグネスさん、社長がお昼前に社長室へ来るように、って。大丈夫?顔色悪いよ。 
\\	えー、何だろう…まだ胃も痛いのに、気が重いなあ。	
\\	えー、何だろう…まだ胃も痛いのに、気が重いなあ。 
\\	(ノックの音)	
\\	失礼します。	
\\	失礼します。 
\\	村上君か。入りなさい。	
\\	村上君か。入りなさい。 
\\	社長、先週のプレゼンは本当に申し訳ありませんでした…って、何なさってるんですか?	
\\	社長、先週のプレゼンは本当に申し訳ありませんでした…って、何なさってるんですか? 
\\	何って君、見れば分かるだろう。新鮮な魚介類が手に入ったから、今日は炭火焼にしているんだ。まあ座りなさい。君ならこの味が分かるだろう。	
\\	何って君、見れば分かるだろう。新鮮な魚介類が手に入ったから、今日は炭火焼にしているんだ。まあ座りなさい。君ならこの味が分かるだろう。 
\\	社長、実は私、今日は胃が…	
\\	社長、実は私、今日は胃が… 
\\	いやー、先週はゴリの件でだいぶ無理をさせたねえ。あいつは昔から傍若無人なところがあってね。	
\\	いやー、先週はゴリの件でだいぶ無理をさせたねえ。あいつは昔から傍若無人なところがあってね。 
\\	はあ…あ、いえ。	
\\	はあ…あ、いえ。 
\\	まあでも、この世界も弱肉強食だからなあ。ビジネスには強引さも必要なんだ。しかし、それで取引先の信用を失ったら本末転倒だろう、そうは思わないか、君?	
\\	まあでも、この世界も弱肉強食だからなあ。ビジネスには強引さも必要なんだ。しかし、それで取引先の信用を失ったら本末転倒だろう、そうは思わないか、君? 
\\	あ、はい、そう思います。	
\\	あ、はい、そう思います。 
\\	うんうん、そうなんだ。今度あいつにはよく言っておかないと。あ、ちょうど牡蠣が焼けたぞ。ほら、食べなさい!広島から直送だ!	
\\	うんうん、そうなんだ。今度あいつにはよく言っておかないと。あ、ちょうど牡蠣が焼けたぞ。ほら、食べなさい!広島から直送だ! 
\\	う、また牡蠣…えーい、いつまでも意気消沈していられないわ!社長、いただきまーす!	
\\	う、また牡蠣…えーい、いつまでも意気消沈していられないわ!社長、いただきまーす! 
\\	ポスター
\\	試しに
\\	背景
\\	クレーム
\\	ボランティア活動
\\	ねえねえ、アグネスさん、ちょっと相談していい?	
\\	ねえねえ、アグネスさん、ちょっと相談していい? 
\\	ええ、何ですか?	
\\	ええ、何ですか? 
\\	あのね、このポスターなんだけど、クライアントに、全体的に暗いって言われちゃったんだけど。どう思う?	
\\	あのね、このポスターなんだけど、クライアントに、全体的に暗いって言われちゃったんだけど。どう思う? 
\\	うーん…この、木の色をもう少し明るい感じにするとか。	
\\	うーん…この、木の色をもう少し明るい感じにするとか。 
\\	僕もそう思って、試しにやってみたんだけど、そうするとすごく嘘っぽくなっちゃうの。ほら。	
\\	僕もそう思って、試しにやってみたんだけど、そうするとすごく嘘っぽくなっちゃうの。ほら。 
\\	あー、ホントだ。なんか、平面的ですねえ。	
\\	あー、ホントだ。なんか、平面的ですねえ。 
\\	でしょう?でも、基本的なデザインは変えられないんだよね。	
\\	でしょう?でも、基本的なデザインは変えられないんだよね。 
\\	この明るい色のまま、影を濃くしたら子供っぽい感じがなくなりませんか?	
\\	この明るい色のまま、影を濃くしたら子供っぽい感じがなくなりませんか? 
\\	あ、そうか。でもそうすると、この背景と合わない感じなんだよね。ここをもう少し黄色っぽい色にしてみようか。	
\\	あ、そうか。でもそうすると、この背景と合わない感じなんだよね。ここをもう少し黄色っぽい色にしてみようか。 
\\	あ、いい感じじゃないですか!	
\\	あ、いい感じじゃないですか! 
\\	じゃあ、これで出してみるか。あーあ、この会社っていつもこんな感じなんだよね。	
\\	じゃあ、これで出してみるか。あーあ、この会社っていつもこんな感じなんだよね。 
\\	クレームが具体的じゃないですよね。	
\\	クレームが具体的じゃないですよね。 
\\	そうなんだよ!この間も、ボランティア活動のポスターを、「もう少し情熱的な感じでお願いします」とか言われてさ。	
\\	"そうなんだよ!この間も、ボランティア活動のポスターを、「もう少し情熱的な感じでお願いします」とか言われてさ。 
\\	うわー、そういうの、一番困りますよねー。	
\\	うわー、そういうの、一番困りますよねー。 
\\	このところ
\\	留守がち
\\	従姉妹
\\	子育て
\\	いい年して
\\	嘆かわしい
\\	定年
\\	落ち着く
\\	はい、村上です。	
\\	はい、村上です。 
\\	アグネス?	
\\	アグネス? 
\\	あ、お母さん。久しぶりねー。	
\\	あ、お母さん。久しぶりねー。 
\\	まったくこの子は、ちっとも連絡してこないで。心配していたのよ。	
\\	まったくこの子は、ちっとも連絡してこないで。心配していたのよ。 
\\	ごめんごめん、このところちょっと仕事が忙しくて、留守がちだったから。	
\\	ごめんごめん、このところちょっと仕事が忙しくて、留守がちだったから。 
\\	そうだったのね。まだ忙しいの?	
\\	そうだったのね。まだ忙しいの? 
\\	もうそれほど忙しくないから大丈夫よ。	
\\	もうそれほど忙しくないから大丈夫よ。 
\\	ところでアグネス、あなたまだ知らないでしょ?あなたの従姉妹の由美ちゃん、今度子供が生まれるのよ。	
\\	ところでアグネス、あなたまだ知らないでしょ?あなたの従姉妹の由美ちゃん、今度子供が生まれるのよ。 
\\	ええー!由美ちゃんまだ若いのに、もうママになっちゃうの?	
\\	ええー!由美ちゃんまだ若いのに、もうママになっちゃうの? 
\\	何を言っているの。彼女だってもう24歳よ。	
\\	何を言っているの。彼女だってもう24歳よ。 
\\	十分若いわよ。確か結婚したときはまだ学生だったのよね。	
\\	十分若いわよ。確か結婚したときはまだ学生だったのよね。 
\\	そうよ。お姉さんの由紀ちゃんにはもう子供がいるし、由美ちゃんは子育てに関しては心配はいらないわね。それに比べてうちは…。	
\\	そうよ。お姉さんの由紀ちゃんにはもう子供がいるし、由美ちゃんは子育てに関しては心配はいらないわね。それに比べてうちは…。 
\\	まだ29歳よ!それに、いきなり結婚って何よー。まだ彼氏だっていないのに。	
\\	まだ29歳よ!それに、いきなり結婚って何よー。まだ彼氏だっていないのに。 
\\	あーあ。いい年してそれじゃあねえ…嘆かわしい。	
\\	あーあ。いい年してそれじゃあねえ…嘆かわしい。 
\\	ちょっとお母さん、気が早すぎ。	
\\	ちょっとお母さん、気が早すぎ。 
\\	あなた、もうこちらには帰らないでずっと東京にいるつもりなの?	
\\	あなた、もうこちらには帰らないでずっと東京にいるつもりなの? 
\\	そんなのまだ分からないわよ。	
\\	そんなのまだ分からないわよ。 
\\	お屠蘇
\\	おせち料理
\\	縁起もの
\\	マメに
\\	黒豆
\\	長生き
\\	目当て
\\	ぽち袋
\\	薄い
\\	さぁ、みんなそろったかな。今日から新しい1年が始まるぞ。お屠蘇でお祝いしよう。	
\\	さぁ、みんなそろったかな。今日から新しい1年が始まるぞ。お屠蘇でお祝いしよう。 
\\	さぁ、どうぞ。おじいちゃんから。	
\\	さぁ、どうぞ。おじいちゃんから。 
\\	新年あけましておめでとうございます!	
\\	新年あけましておめでとうございます! 
\\	新年おめでとうございます!	
\\	新年おめでとうございます! 
\\	新年おめでとうございます!	
\\	新年おめでとうございます! 
\\	お屠蘇、ぼくも飲んでみたいなぁー!	
\\	お屠蘇、ぼくも飲んでみたいなぁー! 
\\	おっ!大地も飲んでみるか。ちょっとだけだぞ。	
\\	おっ!大地も飲んでみるか。ちょっとだけだぞ。 
\\	ごく。。うぁ!ん!!ちょっと変な味だけど、甘いから、もっと飲みたい!!	
\\	ごく。。うぁ!ん!!ちょっと変な味だけど、甘いから、もっと飲みたい!! 
\\	だめだめ、甘くたってお酒なんだから!さぁさぁ、おせち料理食べましょう!	
\\	だめだめ、甘くたってお酒なんだから!さぁさぁ、おせち料理食べましょう! 
\\	まぁ、こんなにいっぱい作るの、大変だったでしょ。	
\\	まぁ、こんなにいっぱい作るの、大変だったでしょ。 
\\	おせち料理は縁起ものだからね。	
\\	おせち料理は縁起ものだからね。 
\\	じゃぁ、おじいちゃんとおばあちゃんは、この海老食べて長生きしてね。	
\\	じゃぁ、おじいちゃんとおばあちゃんは、この海老食べて長生きしてね。 
\\	やさしいねぇー、この子は。。。ううう。。。	
\\	やさしいねぇー、この子は。。。ううう。。。 
\\	だって、おばあちゃんとおじいちゃんから、ずーっとお年玉もらいたいんだもん!	
\\	だって、おばあちゃんとおじいちゃんから、ずーっとお年玉もらいたいんだもん! 
\\	なんだ、目当てはお年玉かい!しょうがないねぇ。はい、お年玉。はい、おにいちゃん。はい、大地。	
\\	なんだ、目当てはお年玉かい!しょうがないねぇ。はい、お年玉。はい、おにいちゃん。はい、大地。 
\\	ありがとう!! ねぇ、おにいちゃんのぽち袋触らせて!あれ、僕のより薄いよ!!わーい、僕の方がいっぱい入ってるんだ。わーい、わーい。	
\\	ありがとう!! ねぇ、おにいちゃんのぽち袋触らせて!あれ、僕のより薄いよ!!わーい、僕の方がいっぱい入ってるんだ。わーい、わーい。 
\\	警察
\\	暴走族
\\	追う
\\	振り切る
\\	初日の出
\\	間に合う
\\	吹っ飛ばす
\\	つかまる
\\	お年玉
\\	そこのバイク今すぐ止まりなさい!!!	
\\	そこのバイク今すぐ止まりなさい!!! 
\\	もっと早く走ってよ!	
\\	もっと早く走ってよ! 
\\	わかってるよ、けど、ただ、暴走族じゃないのになんで警察が追って来るんだよ~。	
\\	わかってるよ、けど、ただ、暴走族じゃないのになんで警察が追って来るんだよ~。 
\\	は、速いっ!!	
\\	は、速いっ!! 
\\	ふぅ、なんとか振り切った。。。。	
\\	ふぅ、なんとか振り切った。。。。 
\\	あ、日の出まで間に合わないじゃない。早くして!	
\\	あ、日の出まで間に合わないじゃない。早くして! 
\\	富士山に急ぐ	
\\	よし、何とか間に合った。	
\\	よし、何とか間に合った。 
\\	(バイクが動き出す。)
\\	あれ?俺のバイクが無いぞ。。。	
\\	あれ?俺のバイクが無いぞ。。。 
\\	、、、、あれ、、、、バイクの上、、、、	
\\	、、、、あれ、、、、バイクの上、、、、 
\\	ああああ!!!!サルが俺のバイクを運転してる。。。。	
\\	ああああ!!!!サルが俺のバイクを運転してる。。。。 
\\	あれ、もしかしてあれは、、、、太郎じゃないか?!アイツいつの間にバイクの運転ができるようになったんだ。。。。	
\\	あれ、もしかしてあれは、、、、太郎じゃないか?!アイツいつの間にバイクの運転ができるようになったんだ。。。。 
\\	御社
\\	国際学会
\\	恐れ入る
\\	担当
\\	総務課
\\	人事課
\\	差し上げる
\\	あいにく
\\	折り返し
\\	大変お待たせいたしました。矢神でございます。	
\\	大変お待たせいたしました。矢神でございます。 
\\	私、札幌堂の松平と申します。	
\\	私、札幌堂の松平と申します。 
\\	札幌堂の松平様ですね。担当の者とかわりますので少々お待ちください。	
\\	札幌堂の松平様ですね。担当の者とかわりますので少々お待ちください。 
\\	申し訳ございません。あいにく担当の者は外出しております。	
\\	申し訳ございません。あいにく担当の者は外出しております。 
\\	では、お手数ですが折り返しお電話をいただけますでしょうか。	
\\	では、お手数ですが折り返しお電話をいただけますでしょうか。 
\\	確かに申し伝えます。私、人事課の矢神が承りました。	
\\	確かに申し伝えます。私、人事課の矢神が承りました。 
\\	それでは、よろしくお願いいたします。	
\\	それでは、よろしくお願いいたします。 
\\	失礼いたします。	
\\	失礼いたします。 
\\	派遣
\\	恐縮
\\	名刺
\\	詳細
\\	研究者
\\	同時通訳
\\	出席者
\\	すみません。私、札幌堂の松平と申します。人事部の山本様と3時に打ち合わせの約束をしているのですが。	
\\	すみません。私、札幌堂の松平と申します。人事部の山本様と3時に打ち合わせの約束をしているのですが。 
\\	3階会議室	
\\	大変お待たせいたしました。私、人事部・派遣担当の山本次郎と申します。(名刺を差し出す)	
\\	大変お待たせいたしました。私、人事部・派遣担当の山本次郎と申します。(名刺を差し出す) 
\\	では早速、詳細について教えていただけますでしょうか。	
\\	では早速、詳細について教えていただけますでしょうか。 
\\	はい。来たる2008年2月26、27日に国際シンポジウムを開催いたします。	
\\	はい。来たる2008年2月26、27日に国際シンポジウムを開催いたします。 
\\	シンガポール、フランス、イギリス、から研究者を招き、日本人研究者を交えセッションを行う予定になっております。	
\\	シンガポール、フランス、イギリス、から研究者を招き、日本人研究者を交えセッションを行う予定になっております。 
\\	何カ国語の通訳者が必要なのでしょうか。	
\\	何カ国語の通訳者が必要なのでしょうか。 
\\	基本的には英語、フランス語になります。	
\\	基本的には英語、フランス語になります。 
\\	司会
\\	具体的
\\	聴衆
\\	機材
\\	手配
\\	拘束時間
\\	見積もり
\\	料金
\\	はい。フランスから1名、シンガポールから1名、イギリスから2名の予定です。それと、日本人研究者が3名と司会の方が1名です。	
\\	はい。フランスから1名、シンガポールから1名、イギリスから2名の予定です。それと、日本人研究者が3名と司会の方が1名です。 
\\	全部で8名ですね。会場全体ではどうでしょうか?	
\\	全部で8名ですね。会場全体ではどうでしょうか? 
\\	100名から200名程度だと思います。	
\\	100名から200名程度だと思います。 
\\	会場全体のおおよその参加者数は、より具体的に教えていただけますか。	
\\	会場全体のおおよその参加者数は、より具体的に教えていただけますか。 
\\	はい、わかりました。	
\\	はい、わかりました。 
\\	同時通訳機材の手配は人数によって変わってきますのでよろしくお願い致します。	
\\	同時通訳機材の手配は人数によって変わってきますのでよろしくお願い致します。 
\\	会議の長さや、おもな内容等もお教えください。通訳者も拘束時間によって料金が違いますので。	
\\	会議の長さや、おもな内容等もお教えください。通訳者も拘束時間によって料金が違いますので。 
\\	はい。わかりました。詳細はすぐに
\\	いたします。	
\\	はい。わかりました。詳細はすぐに
\\	いたします。 
\\	見積書
\\	予算
\\	諸々
\\	諸経費
\\	拝見
\\	再考
\\	設定
\\	実績
\\	松平さん、
\\	来てるよ。	
\\	松平さん、
\\	来てるよ。 
\\	ほーい。どれどれ。	
\\	ほーい。どれどれ。 
\\	うーん、まいったなあ。完全に予算オーバーだわ。	
\\	うーん、まいったなあ。完全に予算オーバーだわ。 
\\	同時通訳ってこんなに高いのね。	
\\	同時通訳ってこんなに高いのね。 
\\	諸経費がすごいなあ。	
\\	諸経費がすごいなあ。 
\\	もしもし。山本様でいらっしゃいますか。	
\\	もしもし。山本様でいらっしゃいますか。 
\\	はい。山本です。	
\\	はい。山本です。 
\\	札幌堂の松平でございます。	
\\	札幌堂の松平でございます。 
\\	さきほど、送っていただいた見積書の方を拝見させていただきました。	
\\	さきほど、送っていただいた見積書の方を拝見させていただきました。 
\\	はい。それで、いかがでしょうか。	
\\	はい。それで、いかがでしょうか。 
\\	はい。諸々の諸経費を含めますと、若干予算オーバー気味ですので、もう一度詳細の方を再考いたしまして、再提出させていただきたいのですが。	
\\	はい。諸々の諸経費を含めますと、若干予算オーバー気味ですので、もう一度詳細の方を再考いたしまして、再提出させていただきたいのですが。 
\\	わかりました。	
\\	わかりました。 
\\	あと、同時通訳の料金なのですが、4時間を超えますと料金の方は一日分となるわけですよね。	
\\	あと、同時通訳の料金なのですが、4時間を超えますと料金の方は一日分となるわけですよね。 
\\	はい。そうです。料金に関しましては、やや高く設定させていただいておりますが、我が社の通訳者は実績も実力も申し分ないと自信を持って言わせていただきます。	
\\	はい。そうです。料金に関しましては、やや高く設定させていただいておりますが、我が社の通訳者は実績も実力も申し分ないと自信を持って言わせていただきます。 
\\	はあ。わかりました。大事な国際会議ですので、どうぞよろしくお願いいたします。	
\\	はあ。わかりました。大事な国際会議ですので、どうぞよろしくお願いいたします。 
\\	(こりゃあ、先方と予算組み直しかもな・・・とほほ)	
\\	(こりゃあ、先方と予算組み直しかもな・・・とほほ) 
\\	健康
\\	禁煙
\\	挑戦
\\	失敗
\\	すう
\\	今日はどうしましたか?	
\\	今日はどうしましたか? 
\\	先生、私、健康のために禁煙したいんです。	
\\	先生、私、健康のために禁煙したいんです。 
\\	それなら、禁煙したらいいんじゃないかね。	
\\	それなら、禁煙したらいいんじゃないかね。 
\\	ええ、でも、14回も禁煙に挑戦したんですが全て失敗しました。	
\\	ええ、でも、14回も禁煙に挑戦したんですが全て失敗しました。 
\\	それじゃ禁煙のガムを。。	
\\	それじゃ禁煙のガムを。。 
\\	それはもうやりました!!禁煙ガム、禁煙フィルター、全てやりました。ですが駄目なんです。。。	
\\	それはもうやりました!!禁煙ガム、禁煙フィルター、全てやりました。ですが駄目なんです。。。 
\\	よし分かった。では、これを試してみよう。ベッドに横になってごらん。	
\\	よし分かった。では、これを試してみよう。ベッドに横になってごらん。 
\\	はい。。	
\\	はい。。 
\\	それではこのコインをじーっと見なさい。	
\\	それではこのコインをじーっと見なさい。 
\\	どん底
\\	躁鬱
\\	染める
\\	服用
\\	副作用
\\	動悸
\\	痺れる
\\	どうされましたか。	
\\	どうされましたか。 
\\	心が安定しないんです。そわそわしているって言うか。	
\\	心が安定しないんです。そわそわしているって言うか。 
\\	どんな風に。	
\\	どんな風に。 
\\	ノリノリでイケイケの時もあれば、どん底に落ちて何もできない時もあるんです。	
\\	ノリノリでイケイケの時もあれば、どん底に落ちて何もできない時もあるんです。 
\\	激しくやってくる感じ?	
\\	激しくやってくる感じ? 
\\	はい。	
\\	はい。 
\\	躁鬱の気があるなあ。	
\\	躁鬱の気があるなあ。 
\\	まわりにもそう言われます。	
\\	まわりにもそう言われます。 
\\	では、薬をだそう。まず一つ目は精神安定剤。そして二つ目はこれ。	
\\	では、薬をだそう。まず一つ目は精神安定剤。そして二つ目はこれ。 
\\	なんですか。	
\\	なんですか。 
\\	うむ。これは、「髪を染めるように性格を染める」と言われているくらい強い薬だ。	
\\	うむ。これは、「髪を染めるように性格を染める」と言われているくらい強い薬だ。 
\\	どうですか。	
\\	どうですか。 
\\	先生!あの強い薬また出してください!	
\\	先生!あの強い薬また出してください! 
\\	強い薬だから飲みすぎてはいけませんよ。副作用が出てくるかもしれないよ。	
\\	強い薬だから飲みすぎてはいけませんよ。副作用が出てくるかもしれないよ。 
\\	副作用!?	
\\	副作用!? 
\\	腕がしびれてくるとか、動悸が激しくなるとか、まあ人によっていろいろ。	
\\	腕がしびれてくるとか、動悸が激しくなるとか、まあ人によっていろいろ。 
\\	先生!腕、痺れてます。	
\\	先生!腕、痺れてます。 
\\	聴診器
\\	栄養失調
\\	精神的ショック
\\	吐く
\\	注射
\\	衰弱
\\	点滴
\\	恐怖症
\\	どうされましたか。	
\\	どうされましたか。 
\\	体調が悪いんです。食べても吐いてしまうし。	
\\	体調が悪いんです。食べても吐いてしまうし。 
\\	顔色が真っ青ですよ。では聴診器をあてるので、上を脱いでください。	
\\	顔色が真っ青ですよ。では聴診器をあてるので、上を脱いでください。 
\\	はい。	
\\	はい。 
\\	栄養失調ですね。どうされたんですか。	
\\	栄養失調ですね。どうされたんですか。 
\\	実は彼女にフラれてから、精神的ショックでほとんど何も食べられなくなってしまって。無理に食べても吐いてしまうんです。	
\\	実は彼女にフラれてから、精神的ショックでほとんど何も食べられなくなってしまって。無理に食べても吐いてしまうんです。 
\\	点滴を打ちましょう。まずは体に栄養を与えないといけません。	
\\	点滴を打ちましょう。まずは体に栄養を与えないといけません。 
\\	えーっ!腕に打つんですか。	
\\	えーっ!腕に打つんですか。 
\\	はい。体が衰弱していますから。おーい。点滴の用意をお願い。	
\\	はい。体が衰弱していますから。おーい。点滴の用意をお願い。 
\\	(患者暴れる)	
\\	はい、おとなしくして。君たち、ちょっと体を押さえて。	
\\	はい、おとなしくして。君たち、ちょっと体を押さえて。 
\\	(注射を打つ)	
\\	(患者気絶。患者は注射恐怖症だった。)	
\\	虫歯
\\	麻酔
\\	抜く
\\	経験
\\	奥歯
\\	痛む
\\	レントゲン
\\	歯科医師
\\	歯医者
\\	訴える
\\	今日は、どうしましたか。	
\\	今日は、どうしましたか。 
\\	一週間くらい前からずっと、右の上の歯が痛むんですが…。	
\\	一週間くらい前からずっと、右の上の歯が痛むんですが…。 
\\	はい?すみませんね。耳が遠いんですよ。	
\\	はい?すみませんね。耳が遠いんですよ。 
\\	右の 上の 歯が 痛いんです。	
\\	右の 上の 歯が 痛いんです。 
\\	ああ。右上の歯ね。見てみましょうか。はい、口を大きく開けて。はい、あーん。	
\\	ああ。右上の歯ね。見てみましょうか。はい、口を大きく開けて。はい、あーん。 
\\	あー。	
\\	あー。 
\\	痛いのはこの歯ですか?(コンコン)	
\\	痛いのはこの歯ですか?(コンコン) 
\\	違う・・・と思います。	
\\	違う・・・と思います。 
\\	じゃ、この歯かな? (コンコン)	
\\	じゃ、この歯かな? (コンコン) 
\\	あのぉ。先生。レントゲンを撮ったり...しないんですか。	
\\	あのぉ。先生。レントゲンを撮ったり...しないんですか。 
\\	レントゲン?私を誰だと思っているんですか。私は、80年歯科医師をしているんです。80年ですよ。	
\\	レントゲン?私を誰だと思っているんですか。私は、80年歯科医師をしているんです。80年ですよ。 
\\	先生、経験があろうがあるまいが、レントゲンは撮ってください。	
\\	先生、経験があろうがあるまいが、レントゲンは撮ってください。 
\\	やっぱり虫歯ですね。ほら、このレントゲンを見てください。かなり深いですよ。	
\\	やっぱり虫歯ですね。ほら、このレントゲンを見てください。かなり深いですよ。 
\\	あががががあ。せ、先生痛い!ま、麻酔をしてください。麻酔をお願いします!	
\\	(ウィィイイイーン キーィイイイイン ガガガガガ)	
\\	絶好の
\\	花見
\\	敷く
\\	貼る
\\	出版
\\	目印
\\	犬猿の仲
\\	日和
\\	見てのとおり
\\	(お花見対決 
\\	今日は込んでるなあ。すごい人だ。	
\\	(お花見対決 
\\	今日は込んでるなあ。すごい人だ。 
\\	絶好の花見日和ですからね。	
\\	絶好の花見日和ですからね。 
\\	場所などもうないのではないか。	
\\	場所などもうないのではないか。 
\\	大丈夫です。場所を取っておきましたから。	
\\	大丈夫です。場所を取っておきましたから。 
\\	どうやって。	
\\	どうやって。 
\\	昨日、夜中に来てシートを敷いておきました。「大原家」って紙を貼っておきましたから。	
\\	昨日、夜中に来てシートを敷いておきました。「大原家」って紙を貼っておきましたから。 
\\	たしかあの辺なんですけど・・・あれっ、誰かいますね。	
\\	たしかあの辺なんですけど・・・あれっ、誰かいますね。 
\\	ああっ!	
\\	ああっ! 
\\	まさか・・・。	
\\	まさか・・・。 
\\	田島社長じゃないか。	
\\	田島社長じゃないか。 
\\	ああ、田島社長ですねえ。(なんで緑山出版の田島社長がいるのよ。大原社長と犬猿の仲じゃん・・・)	
\\	ああ、田島社長ですねえ。(なんで緑山出版の田島社長がいるのよ。大原社長と犬猿の仲じゃん・・・) 
\\	これはこれは田島さん、こんなところで何を・・・。	
\\	これはこれは田島さん、こんなところで何を・・・。 
\\	大原さんじゃないですか。見てのとおりお花見ですよ。おたくは?	
\\	大原さんじゃないですか。見てのとおりお花見ですよ。おたくは? 
\\	われわれもお花見ですよ。	
\\	われわれもお花見ですよ。 
\\	あのー、この辺りにうちのシートが敷いてあったはずなんですけれど。	
\\	あのー、この辺りにうちのシートが敷いてあったはずなんですけれど。 
\\	シート?おい、山中、シートここに敷いてあったか?	
\\	シート?おい、山中、シートここに敷いてあったか? 
\\	いえ。朝6時に来たときは何もなかったです。おたく、場所を間違えてるんじゃないの。	
\\	いえ。朝6時に来たときは何もなかったです。おたく、場所を間違えてるんじゃないの。 
\\	たしかにここです。この大きい木が目印でしたから。あっ、あんなところに。	
\\	たしかにここです。この大きい木が目印でしたから。あっ、あんなところに。 
\\	関係ない
\\	受けて立つ
\\	怖じ気づく
\\	競う
\\	余興
\\	秘書
\\	真似
\\	汚い
\\	困る
\\	占領
\\	証拠
\\	連中
\\	(お花見対決 2、くしゃくしゃになったシートと紙がゴミ箱に捨ててあるのに気がつく。)ひどい。ちょっとひどいじゃないですか。	
\\	(お花見対決 2、くしゃくしゃになったシートと紙がゴミ箱に捨ててあるのに気がつく。)ひどい。ちょっとひどいじゃないですか。 
\\	我々は関係ないよ。我々がやったなんて証拠は?	
\\	我々は関係ないよ。我々がやったなんて証拠は? 
\\	ここを占領してるのが、明らかな証拠でしょ。	
\\	ここを占領してるのが、明らかな証拠でしょ。 
\\	朝来たときは何もなかったからね。	
\\	朝来たときは何もなかったからね。 
\\	ウソだ!あなたたちがやったにきまっている!	
\\	ウソだ!あなたたちがやったにきまっている! 
\\	君ィー、言いがかりは困るよ。	
\\	君ィー、言いがかりは困るよ。 
\\	緑山出版の連中はいつも汚い真似を!	
\\	緑山出版の連中はいつも汚い真似を! 
\\	なんだとー。	
\\	なんだとー。 
\\	まあまあまあまあ。お二人ともせっかくの休日なんですから。	
\\	まあまあまあまあ。お二人ともせっかくの休日なんですから。 
\\	お酒?	
\\	お酒? 
\\	お酒の飲み比べをするんですよ。	
\\	お酒の飲み比べをするんですよ。 
\\	飲み比べ?	
\\	飲み比べ? 
\\	青山出版さんとウチから代表者を一人出して、どちらが飲み続けられるか競うんですよ。	
\\	青山出版さんとウチから代表者を一人出して、どちらが飲み続けられるか競うんですよ。 
\\	なんですかそれ?!	
\\	なんですかそれ?! 
\\	青山出版さん、まさか怖じ気づいたんですか。	
\\	青山出版さん、まさか怖じ気づいたんですか。 
\\	何を~!緑山の連中には負けられん!!受けて立ってやる。	
\\	何を~!緑山の連中には負けられん!!受けて立ってやる。 
\\	相手
\\	丸つぶれ
\\	メンツ
\\	手前
\\	酔う
\\	程度
\\	大目玉
\\	意地
\\	恐れられる
\\	底なしの
\\	正義
\\	相手は底なしの山田と恐れられている男らしい。	
\\	相手は底なしの山田と恐れられている男らしい。 
\\	私にいかせてください。	
\\	私にいかせてください。 
\\	大丈夫なのか。	
\\	大丈夫なのか。 
\\	(次々とテキーラを飲む)	
\\	うー、ま、負けられないぞぉ・・・	
\\	うー、ま、負けられないぞぉ・・・ 
\\	ふふ。その程度か。勝てっこないさ。	
\\	ふふ。その程度か。勝てっこないさ。 
\\	まだまだまだだぁぁ。	
\\	まだまだまだだぁぁ。 
\\	山田さん、まさか酔ってるんじゃないでしょうね。	
\\	山田さん、まさか酔ってるんじゃないでしょうね。 
\\	だ、だいじょうぶですよ、ウィッ、ヒック。負けっこないですよ。	
\\	だ、だいじょうぶですよ、ウィッ、ヒック。負けっこないですよ。 
\\	青山出版に負けたら、社長から大目玉ですよ!	
\\	青山出版に負けたら、社長から大目玉ですよ! 
\\	十九杯目!二十・・・杯	
\\	十九杯目!二十・・・杯 
\\	ヒック・・・ヒック・・・か、勝った・・・ウィッ、ヒック。	
\\	ヒック・・・ヒック・・・か、勝った・・・ウィッ、ヒック。 
\\	宿泊
\\	一泊二日
\\	生憎
\\	団体
\\	満室
\\	旅館
\\	利用
\\	ありがとうございます。旅館わさびでございます。	
\\	ありがとうございます。旅館わさびでございます。 
\\	あ、あのぉ… 6月20日に宿泊をしたいのですが。	
\\	あ、あのぉ… 6月20日に宿泊をしたいのですが。 
\\	ありがとうございます。何名様でご宿泊でしょうか。	
\\	ありがとうございます。何名様でご宿泊でしょうか。 
\\	大人二人、子供二人の四名です。あ、全員同じ部屋でお願いします。	
\\	大人二人、子供二人の四名です。あ、全員同じ部屋でお願いします。 
\\	四名様御一室のご利用ですね。	
\\	四名様御一室のご利用ですね。 
\\	あ、はい。そうです。	
\\	あ、はい。そうです。 
\\	(パソコン操作する音)	
\\	お客様、申し訳ございません。生憎ですが、6月20日は団体のお客様のご予約が入っておりまして、全室満室でございます。	
\\	お客様、申し訳ございません。生憎ですが、6月20日は団体のお客様のご予約が入っておりまして、全室満室でございます。 
\\	あー。前の日も次の日もちょっとダメだなぁ…。	
\\	あー。前の日も次の日もちょっとダメだなぁ…。 
\\	せっかくお電話頂いたのに、大変申し訳ございません。	
\\	せっかくお電話頂いたのに、大変申し訳ございません。 
\\	(カチャ)	
\\	どうだった?	
\\	どうだった? 
\\	ダメだって。予約いっぱいだって。せっかく電話したのになぁ。	
\\	ダメだって。予約いっぱいだって。せっかく電話したのになぁ。 
\\	えー、せっかく良い旅館見つけたのに。	
\\	予約
\\	伺う
\\	連絡先
\\	代表者
\\	通常
\\	締め切る
\\	既に
\\	申し込む
\\	露天風呂
\\	空室
\\	若干
\\	係
\\	担当
\\	小学生
\\	くのいち旅館 予約担当係 服部でございます。	
\\	くのいち旅館 予約担当係 服部でございます。 
\\	あ、もしもし、6月20日から一泊二日、四名一室で宿泊を考えているのですが、まだ空いていますか。	
\\	あ、もしもし、6月20日から一泊二日、四名一室で宿泊を考えているのですが、まだ空いていますか。 
\\	(カチャカチャカチャ)はい。まだ若干 空室がございます。	
\\	(カチャカチャカチャ)はい。まだ若干 空室がございます。 
\\	ホームページを見てお電話しているのですが、露天風呂つき一泊二食宿泊プランというのはまだ申し込むことができますか。	
\\	ホームページを見てお電話しているのですが、露天風呂つき一泊二食宿泊プランというのはまだ申し込むことができますか。 
\\	あ、申し訳ございません。そちらのプランは大変人気がございまして、既に締め切らせていただいております。通常のお部屋でしたら、ご用意できるのですが…。	
\\	あ、申し訳ございません。そちらのプランは大変人気がございまして、既に締め切らせていただいております。通常のお部屋でしたら、ご用意できるのですが…。 
\\	そうですか。じゃ、普通の部屋の予約をお願いします。	
\\	そうですか。じゃ、普通の部屋の予約をお願いします。 
\\	ありがとうございます。では代表者様のお名前とご連絡先を伺ってもよろしいでしょうか。	
\\	ありがとうございます。では代表者様のお名前とご連絡先を伺ってもよろしいでしょうか。 
\\	伊賀 新太郎です。電話番号は03-6526-6969です。	
\\	伊賀 新太郎です。電話番号は03-6526-6969です。 
\\	四名様御一室のご利用ですね。	
\\	四名様御一室のご利用ですね。 
\\	はい。大人二人子供二人です。子供は二人とも小学生です。	
\\	はい。大人二人子供二人です。子供は二人とも小学生です。 
\\	送迎
\\	アレルギー
\\	送迎サービス
\\	部屋食
\\	一切
\\	従業員
\\	一同
\\	ナッツ
\\	あと、子供の一人がナッツアレルギーなので、ナッツ類は入れないようにお願いします。	
\\	あと、子供の一人がナッツアレルギーなので、ナッツ類は入れないようにお願いします。 
\\	はい。かしこまりました。	
\\	はい。かしこまりました。 
\\	電車で行くつもりなのですが、下田の駅まで迎えに来てもらうことは可能ですか。	
\\	電車で行くつもりなのですが、下田の駅まで迎えに来てもらうことは可能ですか。 
\\	はい、二時、三時、四時、五時に送迎サービスがございます。	
\\	はい、二時、三時、四時、五時に送迎サービスがございます。 
\\	つまり、それ以外の時間ならば、タクシーで行いくしかないんですね。	
\\	つまり、それ以外の時間ならば、タクシーで行いくしかないんですね。 
\\	左様でございます。下田の駅から二十五分くらいかかるとおもいます。	
\\	左様でございます。下田の駅から二十五分くらいかかるとおもいます。 
\\	ちなみに、チェックインは何時からでしたっけ。	
\\	ちなみに、チェックインは何時からでしたっけ。 
\\	チェックインは三時から、チェックアウトは十一時でございます。	
\\	チェックインは三時から、チェックアウトは十一時でございます。 
\\	わかりました。じゃぁ、二時の送迎サービスを利用します。	
\\	わかりました。じゃぁ、二時の送迎サービスを利用します。 
\\	かしこまりました。ではご予約内容を確認させていただきます。	
\\	かしこまりました。ではご予約内容を確認させていただきます。 
\\	伊賀 新太郎様、6月20日から一泊二日、四名様御一室のご利用。	
\\	伊賀 新太郎様、6月20日から一泊二日、四名様御一室のご利用。 
\\	大丈夫です。ちなみに、ペットは連れて行っては駄目なんですよね。	
\\	大丈夫です。ちなみに、ペットは連れて行っては駄目なんですよね。 
\\	そうですね。申し訳ないのですが、ペットは一切お断りしております。	
\\	そうですね。申し訳ないのですが、ペットは一切お断りしております。 
\\	わかりました。	
\\	わかりました。 
\\	忍者
\\	案内
\\	本日
\\	差し上げる
\\	散策する
\\	売店
\\	大浴場
\\	非常口
\\	申し付ける
\\	女将
\\	なんなり
\\	伊賀さま。お待ちしておりました。女将の百地でございます。本日は遠いところまで、足を運んでいただきまして、ありがとうございます。	
\\	伊賀さま。お待ちしておりました。女将の百地でございます。本日は遠いところまで、足を運んでいただきまして、ありがとうございます。 
\\	さぁ、どうぞ、お上がりください。お部屋の準備もできております。	
\\	さぁ、どうぞ、お上がりください。お部屋の準備もできております。 
\\	シノブ、伊賀様をお部屋までご案内して差し上げて。	
\\	シノブ、伊賀様をお部屋までご案内して差し上げて。 
\\	あ、どうも。	
\\	あ、どうも。 
\\	(歩きながら)非常口はあちらでございます。大浴場のご利用は朝六時から夜十二時までとなっております。喫茶コーナー、売店は夜の八時まででございます。	
\\	(歩きながら)非常口はあちらでございます。大浴場のご利用は朝六時から夜十二時までとなっております。喫茶コーナー、売店は夜の八時まででございます。 
\\	せっかくだから、散歩したいんだけど、この辺に、お勧めの散歩コースはある。	
\\	せっかくだから、散歩したいんだけど、この辺に、お勧めの散歩コースはある。 
\\	(カチャ、ガラガラ)	
\\	(コポコポコポコポ)	
\\	ご夕食は何時からになさいますか。	
\\	ご夕食は何時からになさいますか。 
\\	うーん。散歩して…、お風呂入って…、ゆっくりしてだから…、六時くらいかな。	
\\	うーん。散歩して…、お風呂入って…、ゆっくりしてだから…、六時くらいかな。 
\\	(キキー)	
\\	お客様?何かおっしゃいましたか?	
\\	お客様?何かおっしゃいましたか? 
\\	え..いえ、別に。キー…キーがないなぁ。。。なんてハハハ。	
\\	え..いえ、別に。キー…キーがないなぁ。。。なんてハハハ。 
\\	化粧
\\	サービスエリア
\\	車間距離
\\	追突事故
\\	後尾
\\	付近
\\	影響
\\	車線規制
\\	片側
\\	横転
\\	下り
\\	断続的に
\\	ジャンクション
\\	上り
\\	相次ぐ
\\	ピークを迎える
\\	ターンラッシュ
\\	渋滞
\\	発生する
\\	さっきから全然動かないね。	
\\	さっきから全然動かないね。 
\\	あぁ。ゴールデンウィーク最終日だからな。	
\\	あぁ。ゴールデンウィーク最終日だからな。 
\\	あぁあー。もっと早く宿を出ればよかったね。	
\\	あぁあー。もっと早く宿を出ればよかったね。 
\\	…誰だよ。二時間も化粧をしていたのは。	
\\	…誰だよ。二時間も化粧をしていたのは。 
\\	(ボリュームがだんだん上がる)	
\\	ゴールデンウィーク最終日の今日は、Uターンラッシュがピークを迎え、各地で渋滞が相次いでいます。	
\\	ゴールデンウィーク最終日の今日は、Uターンラッシュがピークを迎え、各地で渋滞が相次いでいます。 
\\	現在東京へ向かう中央自動車道上りは大月ジャンクションを先頭に断続的に44キロの渋滞となっております。	
\\	現在東京へ向かう中央自動車道上りは大月ジャンクションを先頭に断続的に44キロの渋滞となっております。 
\\	また、中央道下りは、正午過ぎに相模湖付近で発生したトラック横転事故のため、片側一車線の車線規制が行われております。現場付近を走行する場合は十分に注意してください。	
\\	また、中央道下りは、正午過ぎに相模湖付近で発生したトラック横転事故のため、片側一車線の車線規制が行われております。現場付近を走行する場合は十分に注意してください。 
\\	中央道上り渋滞44キロ? あり得ない…。	
\\	中央道上り渋滞44キロ? あり得ない…。 
\\	ねぇ、健ちゃん、あたし、トイレ行きたくなっちゃった。	
\\	ねぇ、健ちゃん、あたし、トイレ行きたくなっちゃった。 
\\	話題
\\	含まれる
\\	脳
\\	心臓
\\	過労
\\	自殺
\\	登録
\\	取引先
\\	うつ病
\\	強いる
\\	象徴する
\\	現状
\\	背景
\\	残業
\\	勤務
\\	呼吸器
\\	疾患
\\	発病
\\	至る
\\	発症
\\	みんなは、「過労死」という言葉を聞いた事がありますか?新聞やテレビのニュースなどでも話題になっていますね。今日は、この「過労死」についてみんなで考えようと思います。	
\\	みんなは、「過労死」という言葉を聞いた事がありますか?新聞やテレビのニュースなどでも話題になっていますね。今日は、この「過労死」についてみんなで考えようと思います。 
\\	うーん、まずさ、過労死の定義って、何?	
\\	うーん、まずさ、過労死の定義って、何? 
\\	教科書に書いてあるよ。	
\\	教科書に書いてあるよ。 
\\	「過労死とは、長時間の残業や休日なしの勤務を強いられた結果、過労やストレスが原因の一つとなって、脳や心臓、呼吸器などの疾患を発病し、死亡に至ることを意味します。	
\\	「過労死とは、長時間の残業や休日なしの勤務を強いられた結果、過労やストレスが原因の一つとなって、脳や心臓、呼吸器などの疾患を発病し、死亡に至ることを意味します。 
\\	なるほどね。そういえば、英語でも、そのまま 
\\	で通じるって聞いたことがあるよ。	
\\	なるほどね。そういえば、英語でも、そのまま 
\\	で通じるって聞いたことがあるよ。 
\\	ちょっと待って。。。(カチャカチャ)	
\\	ちょっと待って。。。(カチャカチャ) 
\\	労働時間
\\	残業代
\\	親父
\\	年間
\\	各国
\\	欧米
\\	国別
\\	一因
\\	生み出す
\\	申請する
\\	載る
\\	風呂敷残業
\\	報告する
\\	管理職
\\	ぼやく
\\	見て、この国別の労働時間のグラフ。	
\\	見て、この国別の労働時間のグラフ。 
\\	1988年頃は、欧米各国の年間労働時間が1600時間から1800時間なのに対して、日本は2000時間を越えている。	
\\	1988年頃は、欧米各国の年間労働時間が1600時間から1800時間なのに対して、日本は2000時間を越えている。 
\\	でも、これ、本当の数字なのかなぁ。	
\\	でも、これ、本当の数字なのかなぁ。 
\\	僕の親父は、管理職になったら、残業代が払われないってぼやいてたし、	
\\	"僕の親父は、管理職になったら、残業代が払われないってぼやいてたし、 
\\	労働災害
\\	保険金
\\	認める
\\	激務
\\	生活習慣
\\	責任
\\	認定
\\	推移
\\	そういえば、時々、ニュースで「過労死の労災認定」の話を耳にするよね。	
\\	"そういえば、時々、ニュースで「過労死の労災認定」の話を耳にするよね。 
\\	えっ、労災認定って何?	
\\	えっ、労災認定って何? 
\\	労働災害認定のこと。	
\\	仕事や通勤の途中で怪我をしたり、病気になったり、死亡してしまった場合は、労働災害として保険金が払われるんだ。	
\\	仕事や通勤の途中で怪我をしたり、病気になったり、死亡してしまった場合は、労働災害として保険金が払われるんだ。 
\\	なんで?	
\\	なんで? 
\\	まず、過労死は、激務の最中や直後ではなくて、一ヵ月後から数ヵ月後に起こる事が多い。 
\\	加えて、生活習慣が原因で脳や心臓の病気になることもあるから、なかなか激務が原因で死に至ったということを証明するのが難しいんだ。	
\\	加えて、生活習慣が原因で脳や心臓の病気になることもあるから、なかなか激務が原因で死に至ったということを証明するのが難しいんだ。 
\\	基準
\\	解決策
\\	根本的な
\\	緩和する
\\	考慮する
\\	判定する
\\	関連性
\\	平均
\\	改正
\\	裁判
\\	2001年と2002年の間で認定数が大きく伸びた理由がわかったよ。	
\\	"2001年と2002年の間で認定数が大きく伸びた理由がわかったよ。 
\\	2001年12月に、認定基準の改正があって、「過去6ヶ月間、月平均80時間の残業が続いていれば、過労と発病との関連性が強い」と判定することになったらしいんだ。	
\\	"2001年12月に、認定基準の改正があって、「過去6ヶ月間、月平均80時間の残業が続いていれば、過労と発病との関連性が強い」と判定することになったらしいんだ。 
\\	なるほどねぇ。認定されないより、された方がいいけど、過労死を減らすための根本的な解決策にはならないよね。	
\\	なるほどねぇ。認定されないより、された方がいいけど、過労死を減らすための根本的な解決策にはならないよね。 
\\	僕も、今は、大人になったら、絶対サービス残業なんかしないぞって思うよ。でも、実際に会社で働くようになったら、ノーとは言えないだろうな。	
\\	僕も、今は、大人になったら、絶対サービス残業なんかしないぞって思うよ。でも、実際に会社で働くようになったら、ノーとは言えないだろうな。 
\\	司会者
\\	ルール違反
\\	確かに
\\	やむを得ない
\\	選択
\\	違法
\\	法律違反
\\	繰り返し
\\	(テレビ)「朝まで話さナイト」のお時間です。	
\\	"(テレビ)「朝まで話さナイト」のお時間です。 
\\	こんばんは。薄井 ジョー(うすいじょう)です。	
\\	こんばんは。薄井 ジョー(うすいじょう)です。 
\\	薄井さん、薄井さん、そのお話は後ほどゆっくりお願いしますね~。	
\\	薄井さん、薄井さん、そのお話は後ほどゆっくりお願いしますね~。 
\\	よろしくお願いいたします。	
\\	よろしくお願いいたします。 
\\	「やむを得ない」って何ですか!「やむを得ない」って。	
\\	"「やむを得ない」って何ですか!「やむを得ない」って。 
\\	薄井さんが先ほどから、繰り返し使っておられる、「ママチャリ」というのは自転車の前の部分にカゴのついた買い物用自転車のことですね。	
\\	"薄井さんが先ほどから、繰り返し使っておられる、「ママチャリ」というのは自転車の前の部分にカゴのついた買い物用自転車のことですね。 
\\	そうですね。「ちゃりんこ」というのは自転車のことなので、ママの自転車という意味なんです。	
\\	"そうですね。「ちゃりんこ」というのは自転車のことなので、ママの自転車という意味なんです。 
\\	保育園
\\	嘆く
\\	滅びる
\\	少子化
\\	原則
\\	都内
\\	賛成
\\	規制
\\	覚悟
\\	罰金
\\	子育て
\\	環境
\\	え?駄目なんですか?知りませんでした。	
\\	え?駄目なんですか?知りませんでした。 
\\	へぇー。駄目なんだ。でも、よく見かけますよね。	
\\	へぇー。駄目なんだ。でも、よく見かけますよね。 
\\	はぁ?ガソリンの値上げよりも困りますよ!子育て中の母親達のことをまったく考えていないですよね。	
\\	はぁ?ガソリンの値上げよりも困りますよ!子育て中の母親達のことをまったく考えていないですよね。 
\\	いかがでしょうか。インタビューをした人の半数は自転車の3人乗りが違法だということを知らなかったそうです。	
\\	いかがでしょうか。インタビューをした人の半数は自転車の3人乗りが違法だということを知らなかったそうです。 
\\	都内は保育園への車での送迎が原則的に禁止のため、2人の子供を持つ親は、3人乗りせざるを得ないんです。	
\\	"都内は保育園への車での送迎が原則的に禁止のため、2人の子供を持つ親は、3人乗りせざるを得ないんです。 
\\	でも、ママチャリの3人乗りは本当に危険です。	
\\	でも、ママチャリの3人乗りは本当に危険です。 
\\	運動神経
\\	抜群
\\	おんぶ
\\	猛スピード
\\	迫る
\\	慣性の法則
\\	無事
\\	~様々
\\	もちろん、運動神経抜群の僕はひらりと身をかわして、そのママチャリをさけました。	
\\	もちろん、運動神経抜群の僕はひらりと身をかわして、そのママチャリをさけました。 
\\	自転車の前と後ろと?つまり3人乗りですね?	
\\	自転車の前と後ろと?つまり3人乗りですね? 
\\	そうです。そして反対方向からは、子供を自転車の前と後に乗せ、さらに赤ちゃんをおんぶした、ママさんが猛スピードで迫ってきていました。	
\\	そうです。そして反対方向からは、子供を自転車の前と後に乗せ、さらに赤ちゃんをおんぶした、ママさんが猛スピードで迫ってきていました。 
\\	ちょっと待ってください。自転車の前と後ろとおんぶ?つまり4人乗りですか?	
\\	ちょっと待ってください。自転車の前と後ろとおんぶ?つまり4人乗りですか? 
\\	はい。右からママチャリ、左からもママチャリ。後ろは壁です。そして、前からは…	
\\	はい。右からママチャリ、左からもママチャリ。後ろは壁です。そして、前からは… 
\\	前からは?	
\\	前からは? 
\\	前からは何が来たんですか?	
\\	前からは何が来たんですか? 
\\	トラックが猛スピードでバックして来ました。	
\\	トラックが猛スピードでバックして来ました。 
\\	ええ?トラックが?で?	
\\	ええ?トラックが?で? 
\\	幸い運転手が僕に気づいて、急ブレーキを踏んだので、助かったって訳です。	
\\	幸い運転手が僕に気づいて、急ブレーキを踏んだので、助かったって訳です。 
\\	お母さん達と子供達は?	
\\	お母さん達と子供達は? 
\\	全員無事。というのもそのトラックが運んでいたのはマットレスだったんです。	
\\	全員無事。というのもそのトラックが運んでいたのはマットレスだったんです。 
\\	駐車する
\\	家賃
\\	経済的
\\	警視庁
\\	認める
\\	複数
\\	先程
\\	外出
\\	幼児
\\	でもねぇ、薄井さん。じゃぁ、お母さん達はどうしたらいいんでしょうね。赤ちゃんを一人家において外出した方が安全ですか。	
\\	でもねぇ、薄井さん。じゃぁ、お母さん達はどうしたらいいんでしょうね。赤ちゃんを一人家において外出した方が安全ですか。 
\\	車で行けばいいでしょ。車がないなら、駅の近くに住めばいいんですよ。	
\\	車で行けばいいでしょ。車がないなら、駅の近くに住めばいいんですよ。 
\\	さっきも申し上げましたが、都内では車の送迎を禁止する保育園が多いんですよ。駐車スペースがありませんからね。	
\\	さっきも申し上げましたが、都内では車の送迎を禁止する保育園が多いんですよ。駐車スペースがありませんからね。 
\\	それから、先程「駅の近くに住むめばいい」っておっしゃいましたね。	
\\	それから、先程「駅の近くに住むめばいい」っておっしゃいましたね。 
\\	勿論、駅の近くに住めるに越したことはありませんが、駅から近いところは家賃が高いんです。	
\\	勿論、駅の近くに住めるに越したことはありませんが、駅から近いところは家賃が高いんです。 
\\	車を買う経済力も、駅の近くに住む経済力もないなら、子供を作らなければいいんですよ。	
\\	車を買う経済力も、駅の近くに住む経済力もないなら、子供を作らなければいいんですよ。 
\\	はぁ?あなた、なんて事言うんですか!!ひどいにも程があります!	
\\	はぁ?あなた、なんて事言うんですか!!ひどいにも程があります! 
\\	警視庁は「自転車の三人乗りは危険で認められない。しかし複数の幼児を持つ家庭の大変さは理解できる。	
\\	警視庁は「自転車の三人乗りは危険で認められない。しかし複数の幼児を持つ家庭の大変さは理解できる。 
\\	覆う
\\	高気圧
\\	見込み
\\	模様
\\	気象予報士
\\	おまけに
\\	こいつ
\\	不公平
\\	クミ、そんなに、テレビを近くで見たら目が悪くなるぞ。	
\\	クミ、そんなに、テレビを近くで見たら目が悪くなるぞ。 
\\	…うん…。	
\\	…うん…。 
\\	では、全国の明日のお天気です。	
\\	では、全国の明日のお天気です。 
\\	明日は、日本付近を覆う高気圧の影響で、晴れるところが多い見込みです。	
\\	明日は、日本付近を覆う高気圧の影響で、晴れるところが多い見込みです。 
\\	ただし、四国から関東の太平洋側では雨が降りやすくなるでしょう。	
\\	ただし、四国から関東の太平洋側では雨が降りやすくなるでしょう。 
\\	大阪や、名古屋は昼前から、東京は昼過ぎから雨となるので、朝、雨が降っていなくても、忘れずに傘をお持ちください。	
\\	大阪や、名古屋は昼前から、東京は昼過ぎから雨となるので、朝、雨が降っていなくても、忘れずに傘をお持ちください。 
\\	えー。明日は雨かぁ。 
\\	…ねぇ、健ちゃん、この人、カッコいいね。	
\\	…ねぇ、健ちゃん、この人、カッコいいね。 
\\	浮かれる
\\	文面
\\	スカイプ
\\	送信
\\	社交辞令
\\	優香ちゃんから、年賀状もらっちゃったよ。(鼻歌)	
\\	優香ちゃんから、年賀状もらっちゃったよ。(鼻歌) 
\\	何、浮かれてるんだか。	
\\	何、浮かれてるんだか。 
\\	じゃ、お前、あんなかわいい子から年賀状もらえるものなら、もらってみろよ!しかも、この文面、見てくれよ。	
\\	"じゃ、お前、あんなかわいい子から年賀状もらえるものなら、もらってみろよ!しかも、この文面、見てくれよ。 
\\	「今年は、ぜひ、上原さんと一緒にゴルフさせて頂きたいです!」だぜ~!	
\\	「今年は、ぜひ、上原さんと一緒にゴルフさせて頂きたいです!」だぜ~! 
\\	これで、誘わないのは、失礼だよなっ!	
\\	"これで、誘わないのは、失礼だよなっ! 
\\	はぁ、お前、熱でもあるんじゃないのか?!	
\\	はぁ、お前、熱でもあるんじゃないのか?! 
\\	本当に、あの優香ちゃんを誘えるもんなら、誘ってみろよ。	
\\	"本当に、あの優香ちゃんを誘えるもんなら、誘ってみろよ。 
\\	お~、見てろよ。さっそく、スカイプ、スカイプ。	
\\	お~、見てろよ。さっそく、スカイプ、スカイプ。 
\\	「優香ちゃん、年賀状ありがとう。ゴルフするなんて、知らなかったよ。さっそくだけど、今度の日曜日は、どう?」 送信!	
\\	"「優香ちゃん、年賀状ありがとう。ゴルフするなんて、知らなかったよ。さっそくだけど、今度の日曜日は、どう?」 送信! 
\\	(着信音)	
\\	(着信音) 
\\	口に合う
\\	口うるさい
\\	口出しする
\\	口がうまい
\\	口をきく
\\	口をはさむ
\\	後をつける
\\	口ごもる
\\	いい加減〜して
\\	口をつける
\\	逆効果
\\	なぁ、どうしたんだよ。いい加減、口をきいてくれよ。飲み物にも料理にも全然口をつけないし。	
\\	"なぁ、どうしたんだよ。いい加減、口をきいてくれよ。飲み物にも料理にも全然口をつけないし。 
\\	あ、「こんな安い料理は私の口に合いません。」なんていうのか?	
\\	あ、「こんな安い料理は私の口に合いません。」なんていうのか? 
\\	・・・先週の・・・先週の日曜日どこに行ってたのよ….	
\\	・・・先週の・・・先週の日曜日どこに行ってたのよ…. 
\\	えっ!!	
\\	えっ!! 
\\	行き先を聞いたときに、口ごもっていたから、気になって、あなたの後をつけたのよ。	
\\	行き先を聞いたときに、口ごもっていたから、気になって、あなたの後をつけたのよ。 
\\	お前! そんなことしたのか?! 信じられない!!!	
\\	お前! そんなことしたのか?! 信じられない!!! 
\\	そりゃいけないよ、奥さん。	
\\	そりゃいけないよ、奥さん。 
\\	そういうことは、目をつぶっていた方がいいと思いますよ。	
\\	そういうことは、目をつぶっていた方がいいと思いますよ。 
\\	は?あなた誰ですか。関係ない方が口をはさまないでください。	
\\	は?あなた誰ですか。関係ない方が口をはさまないでください。 
\\	信じられないのはこっちよ。もう、あなたとは口もききたくないわ。	
\\	信じられないのはこっちよ。もう、あなたとは口もききたくないわ。 
\\	誤解だよ。誤解。俺が愛しているのは、君だけなんだからさ。	
\\	誤解だよ。誤解。俺が愛しているのは、君だけなんだからさ。 
\\	よっ、旦那さん。口がうまいねぇ。	
\\	よっ、旦那さん。口がうまいねぇ。 
\\	だから!口出ししないでください!!!って言ったでしょ!	
\\	だから!口出ししないでください!!!って言ったでしょ! 
\\	ひざ掛け
\\	ズボン下
\\	体感温度
\\	推奨する
\\	設定する
\\	削減
\\	排出量
\\	二酸化炭素
\\	クールビズ
\\	ウォームビズ
\\	遅れをとる
\\	(オフィスにて)	
\\	(オフィスにて) 
\\	あっ、ひなこちゃん、かわいいひざ掛けしてるね!	
\\	あっ、ひなこちゃん、かわいいひざ掛けしてるね! 
\\	はい、ウォームビズですからね。	
\\	はい、ウォームビズですからね。 
\\	クールビズは知ってるけど。ウォームビズってなんだっけ?	
\\	クールビズは知ってるけど。ウォームビズってなんだっけ? 
\\	上原さん、遅れてますよ。	
\\	上原さん、遅れてますよ。 
\\	11月から3月は、ウォームビズ期間で、二酸化炭素の排出量削減のために、オフィスの暖房を23度から20度に下げて設定することが推奨されているんです。	
\\	11月から3月は、ウォームビズ期間で、二酸化炭素の排出量削減のために、オフィスの暖房を23度から20度に下げて設定することが推奨されているんです。 
\\	だから、最近、オフィスが寒いんだなぁ。	
\\	だから、最近、オフィスが寒いんだなぁ。 
\\	僕としたことが、知らなかったなぁ~~。	
\\	僕としたことが、知らなかったなぁ~~。 
\\	ひざ掛けをすると、体感温度が2.5度も上がるそうですよ。	
\\	ひざ掛けをすると、体感温度が2.5度も上がるそうですよ。 
\\	優先席
\\	譲る
\\	堂々と
\\	付近
\\	両手打ち
\\	~なる一方
\\	落ち
\\	逆ギレ
\\	目つき
\\	控える
\\	マナーモード
\\	勇気
\\	「優先席付近では、携帯電話の電源はお切りください。	
\\	「優先席付近では、携帯電話の電源はお切りください。 
\\	それ以外の場所では、マナーモードに設定の上、通話はお控えください。」	
\\	それ以外の場所では、マナーモードに設定の上、通話はお控えください。」 
\\	全く、この人、若いくせに堂々と優先席に座って、その上、携帯メールまでしてるなんて、許せないわ!	
\\	全く、この人、若いくせに堂々と優先席に座って、その上、携帯メールまでしてるなんて、許せないわ!  
\\	一言、言ってやろうかしら。	
\\	一言、言ってやろうかしら。 
\\	あっ、でも、目つきが悪いし、逆切れされるのが落ちかな。	
\\	あっ、でも、目つきが悪いし、逆切れされるのが落ちかな。 
\\	だめだめ、そんなことだから、日本人のマナーが悪くなる一方なのよ。	
\\	だめだめ、そんなことだから、日本人のマナーが悪くなる一方なのよ。 
\\	ここは、勇気を出さなくちゃ。	
\\	ここは、勇気を出さなくちゃ。 
\\	鼻歌
\\	鼻をあかす
\\	鼻にかける
\\	鼻が利く
\\	鼻につく
\\	せいぜい
\\	同期
\\	鼻が高い
\\	表彰する
\\	ベストセールスマン
\\	鼻をへし折る
\\	フフフン	
\\	フフフン 
\\	どうしたんだ?珍しいな。鼻歌なんか歌って。	
\\	どうしたんだ?珍しいな。鼻歌なんか歌って。 
\\	今度、俺、ベストセールスマンに表彰されるんだ。	
\\	今度、俺、ベストセールスマンに表彰されるんだ。 
\\	へぇえええ。す、すごいなぁ。俺も、同期として、鼻が高いよ。おめでとう。	
\\	へぇえええ。す、すごいなぁ。俺も、同期として、鼻が高いよ。おめでとう。 
\\	ま、上原も、せいぜい頑張れよ。お前、鼻が利くだろ。	
\\	ま、上原も、せいぜい頑張れよ。お前、鼻が利くだろ。 
\\	鼻が利く?	
\\	鼻が利く? 
\\	だって、買ってくれそうな客を見つけるの得意じゃないか。	
\\	だって、買ってくれそうな客を見つけるの得意じゃないか。 
\\	クンクンクンクン。買ってください。ワン!ワン!ははは!	
\\	クンクンクンクン。買ってください。ワン!ワン!ははは! 
\\	何だ、その言い方。鼻につくな。	
\\	何だ、その言い方。鼻につくな。 
\\	ベストセールスマンになったからって、そんなに鼻にかけるなよ。	
\\	ベストセールスマンになったからって、そんなに鼻にかけるなよ。 
\\	辞任する
\\	不祥事
\\	製菓
\\	建て直す
\\	処理する
\\	筋
\\	ミートミートの斉藤社長、辞任するらしいぞ!	
\\	"ミートミートの斉藤社長、辞任するらしいぞ! 
\\	え?今回の不祥事の責任を取って辞任するっていうこと?	
\\	え?今回の不祥事の責任を取って辞任するっていうこと? 
\\	そうらしいな。	
\\	そうらしいな。 
\\	斉藤社長は、責任を持って今後の対策を立てるべきよ。	
\\	斉藤社長は、責任を持って今後の対策を立てるべきよ。 
\\	社長にはそういう責任があると思うわ。それから辞めても遅くないんじゃない?	
\\	社長にはそういう責任があると思うわ。それから辞めても遅くないんじゃない? 
\\	確かに、そういう責任の取り方もあるな。	
\\	確かに、そういう責任の取り方もあるな。 
\\	この前の甘辛製菓の社長は、すぐに辞任せずに、頑張って会社を建て直したじゃない。	
\\	"この前の甘辛製菓の社長は、すぐに辞任せずに、頑張って会社を建て直したじゃない。 
\\	それが、社長としての責任ある態度だと思うわ。	
\\	それが、社長としての責任ある態度だと思うわ。 
\\	片思い
\\	不思議ちゃん
\\	はぐれる
\\	遭難する
\\	(電話が)通じる
\\	町はずれ
\\	幻
\\	狸
\\	まり子ちゃん、片思いしているって言っていたよね? 好きな人ってどんな人?	
\\	まり子ちゃん、片思いしているって言っていたよね? 好きな人ってどんな人? 
\\	えっ?う~ん、実は、名前、知らないんだぁ。	
\\	えっ?う~ん、実は、名前、知らないんだぁ。 
\\	1度会ったきりの人だから。	
\\	1度会ったきりの人だから。 
\\	(一度会ったきりか…。さすが不思議ちゃん。	
\\	(一度会ったきりか…。さすが不思議ちゃん。 
\\	ということは、まだ、俺にも可能性があるってことか?)	
\\	ということは、まだ、俺にも可能性があるってことか?) 
\\	それって、どういうこと?	
\\	それって、どういうこと? 
\\	絶対、笑わないでよ。。。	
\\	絶対、笑わないでよ。。。 
\\	誓うよ。	
\\	誓うよ。 
\\	実はね、高校生のとき、友達と山へハイキングに出かけたんだけど、友達とはぐれちゃって、遭難しそうになったことがあるの。	
\\	実はね、高校生のとき、友達と山へハイキングに出かけたんだけど、友達とはぐれちゃって、遭難しそうになったことがあるの。 
\\	山奥だから、当然電話もつうじないでしょ。	
\\	山奥だから、当然電話もつうじないでしょ。 
\\	そのうち、飲み物は飲みきってしまうし、食べ物も食べきってしまって、本当に困りきっていたときに、突然、彼が、どこからか現れたのよ。	
\\	そのうち、飲み物は飲みきってしまうし、食べ物も食べきってしまって、本当に困りきっていたときに、突然、彼が、どこからか現れたのよ。 
\\	それで?	
\\	それで? 
\\	人聞き
\\	お風呂上がり
\\	枝豆
\\	必ず
\\	頼みごと
\\	ばれる
\\	新作
\\	狸寝入り
\\	あぁ、いい湯だった。	
\\	あぁ、いい湯だった。 
\\	パパ、まずは、キーンと冷えたビールでもどうぞ。	
\\	パパ、まずは、キーンと冷えたビールでもどうぞ。 
\\	どうしたんだよ?いやにサービスがいいなぁ。	
\\	どうしたんだよ?いやにサービスがいいなぁ。 
\\	あら、人聞きの悪いことは言わないで!いつも、サービスいいでしょ。	
\\	あら、人聞きの悪いことは言わないで!いつも、サービスいいでしょ。 
\\	え~そうかぁ??まっ、そういうことにしておくよ。	
\\	え~そうかぁ??まっ、そういうことにしておくよ。 
\\	ゴクゴクゴク。あぁ~、風呂上りは冷えたビールに限る!!	
\\	ゴクゴクゴク。あぁ~、風呂上りは冷えたビールに限る!! 
\\	さっ、枝豆もどうぞ~。	
\\	さっ、枝豆もどうぞ~。 
\\	おっ!サンキュウ~。やっぱり、ビールのつまみは、枝豆に限るよな~。	
\\	おっ!サンキュウ~。やっぱり、ビールのつまみは、枝豆に限るよな~。 
\\	ところで、何だい?話は?	
\\	ところで、何だい?話は? 
\\	えっ! 何でわかったの?!	
\\	えっ! 何でわかったの?! 
\\	だって、君がサービス良くしてくれるときは、必ず何か頼みごとがあるじゃないか。	
\\	だって、君がサービス良くしてくれるときは、必ず何か頼みごとがあるじゃないか。 
\\	ばれちゃ、しょうがないわ。	
\\	ばれちゃ、しょうがないわ。 
\\	実はねぇ、今度のヴィトンの新作バック、とっても素敵なのよ~~。買って~。	
\\	実はねぇ、今度のヴィトンの新作バック、とっても素敵なのよ~~。買って~。 
\\	手巻き
\\	味覚
\\	嗅覚
\\	視覚障害
\\	目隠し
\\	におい
\\	舌触り
\\	当てる
\\	触覚
\\	寿司
\\	鋭い
\\	昨日、家で、手巻き寿司を食べたとき、ちょっと面白いゲームをやったよ。	
\\	昨日、家で、手巻き寿司を食べたとき、ちょっと面白いゲームをやったよ。 
\\	えっ!どんなゲームですか。	
\\	えっ!どんなゲームですか。 
\\	他の人が作った手巻き寿司を目をつぶったまま食べて、中身の具を当てるというゲームだ。	
\\	他の人が作った手巻き寿司を目をつぶったまま食べて、中身の具を当てるというゲームだ。 
\\	へぇ~、面白そうですね。	
\\	へぇ~、面白そうですね。 
\\	だろう?目をつぶっているから、舌触りとか、におい、食感が大切なんだ。	
\\	だろう?目をつぶっているから、舌触りとか、におい、食感が大切なんだ。 
\\	結構、むずかしそうですね。	
\\	結構、むずかしそうですね。 
\\	目をつぶって食べるという点では、ブラインドレストランに通じるものがありますね。	
\\	目をつぶって食べるという点では、ブラインドレストランに通じるものがありますね。 
\\	なんだ、そのブラインドレストランっていうのは?	
\\	なんだ、そのブラインドレストランっていうのは? 
\\	目隠しをして食事するレストランのことですよ。	
\\	目隠しをして食事するレストランのことですよ。 
\\	視覚障害を持つ人に対する理解を深めることを目的にヨーロッパで始められたものなんです。	
\\	視覚障害を持つ人に対する理解を深めることを目的にヨーロッパで始められたものなんです。 
\\	ほ~。上原、意外と物知りじゃないか。	
\\	ほ~。上原、意外と物知りじゃないか。 
\\	足が地に付かない
\\	イクラ
\\	足が早い
\\	足を伸ばす
\\	足が出る
\\	直撃
\\	有給休暇
\\	練り直す
\\	足が遠のく
\\	明日は、孝雄くんとデートだ~。	
\\	明日は、孝雄くんとデートだ~。 
\\	わくわくするな~!なんだか、足が地に付かない感じ!	
\\	わくわくするな~!なんだか、足が地に付かない感じ! 
\\	さぁて、お弁当は、何にしようかな~。孝雄君、イクラ好きだから、イクラのおにぎりにしようかな。	
\\	さぁて、お弁当は、何にしようかな~。孝雄君、イクラ好きだから、イクラのおにぎりにしようかな。 
\\	"あ、でも、いくらは、足が早いから、やめておいた方がいいわね。。。
\\	"あ、でも、いくらは、足が早いから、やめておいた方がいいわね。。。
\\	添付
\\	再生
\\	告白
\\	ダウンロード
\\	動画
\\	確か
\\	拡張子
\\	まめに
\\	再起動
\\	立ち上がる
\\	ファイル
\\	調子
\\	共有
\\	手数
\\	上原さん、おはようございます。	
\\	上原さん、おはようございます。 
\\	あのぉ、添付ファイルが開けられないんですが、見ていただけますか。	
\\	あのぉ、添付ファイルが開けられないんですが、見ていただけますか。 
\\	じゃぁ、課の共有フォルダに入れておいてよ。	
\\	じゃぁ、課の共有フォルダに入れておいてよ。 
\\	今、パソコンが立ち上がったと思ったら、フリーズしちゃって、再起動しているところなんだ。	
\\	今、パソコンが立ち上がったと思ったら、フリーズしちゃって、再起動しているところなんだ。 
\\	最近、上原さんのコンピュータ、調子悪いですね。 
\\	うん、まめに、デフラグしないといけないな。	
\\	うん、まめに、デフラグしないといけないな。 
\\	ところで、そのファイルの拡張子は何だった?	
\\	ところで、そのファイルの拡張子は何だった? 
\\	確か、
\\	だったと思います。	
\\	確か、
\\	だったと思います。 
\\	なんだ、それなら、動画ファイルだから、
\\	で開くはずだよ。	
\\	なんだ、それなら、動画ファイルだから、
\\	で開くはずだよ。 
\\	それ、私の
\\	に入ってないと思います。じゃぁ、ダウンロードしなくちゃいけないんですね。	
\\	それ、私の
\\	に入ってないと思います。じゃぁ、ダウンロードしなくちゃいけないんですね。 
\\	あ、ちょっと待って。やっと、パソコンが立ち上がった。どれどれ。。。	
\\	あ、ちょっと待って。やっと、パソコンが立ち上がった。どれどれ。。。 
\\	組む
\\	裏切る
\\	仲を取り持つ
\\	仕方(が)ない
\\	水臭い
\\	内緒
\\	断る
\\	付き合いが悪い
\\	道理で
\\	訳
\\	あれ?あそこにいるの、健と綾香じゃない?!	
\\	あれ?あそこにいるの、健と綾香じゃない?! 
\\	あ!腕なんか組んじゃって、あの二人、付き合ってるっての?	
\\	あ!腕なんか組んじゃって、あの二人、付き合ってるっての? 
\\	道理で、綾香、最近付き合いが悪くなったわけだ。	
\\	道理で、綾香、最近付き合いが悪くなったわけだ。 
\\	私たちが誘っても、理由をつけて、断っていたのには、こういう訳があったのね。	
\\	私たちが誘っても、理由をつけて、断っていたのには、こういう訳があったのね。 
\\	でも、内緒にするなんて、水臭いよねぇ。	
\\	でも、内緒にするなんて、水臭いよねぇ。 
\\	そうだね。でも、まぁ・・・仕方ないか・・・。	
\\	そうだね。でも、まぁ・・・仕方ないか・・・。 
\\	なんでよ?何か訳があるの?	
\\	なんでよ?何か訳があるの? 
\\	実は、さゆりも健のことがずーっと好きだったんだよね。	
\\	実は、さゆりも健のことがずーっと好きだったんだよね。 
\\	それで、さゆりは、綾香に、「健との仲を取り持って」って頼んでいたのよ。	
\\	それで、さゆりは、綾香に、「健との仲を取り持って」って頼んでいたのよ。 
\\	見かける
\\	板ばさみ
\\	友情
\\	愛情
\\	逆に
\\	告白する
\\	諦める
\\	思い詰める
\\	打ち明ける
\\	なかなか~ない
\\	微妙
\\	綾香、健と付き合ってるの?	
\\	綾香、健と付き合ってるの? 
\\	昨日、健と一緒にいるところ見かけたんだけど。	
\\	昨日、健と一緒にいるところ見かけたんだけど。 
\\	う、うん。実はね。。。言おう言おうと思ってたんだけど、なかなか言い出せなくて。。	
\\	う、うん。実はね。。。言おう言おうと思ってたんだけど、なかなか言い出せなくて。。 
\\	どういうことなの?さゆりを裏切ったってわけ?	
\\	どういうことなの?さゆりを裏切ったってわけ? 
\\	裏切るつもりはなかったの。	
\\	裏切るつもりはなかったの。 
\\	実は、私もずっと健のことが好きだったのよ。でも、さゆりから健のことを打ち明けられたとき、本当のことが言えなくて。。。	
\\	実は、私もずっと健のことが好きだったのよ。でも、さゆりから健のことを打ち明けられたとき、本当のことが言えなくて。。。 
\\	さゆりも、思いつめていたみたいだったから、健のことはあきらめようって思ったわけ。	
\\	さゆりも、思いつめていたみたいだったから、健のことはあきらめようって思ったわけ。 
\\	それで、何で付き合うことになったのよ?	
\\	それで、何で付き合うことになったのよ? 
\\	健に、さゆりの気持ちを伝えに行ったら、逆に、健から告白されちゃって。	
\\	健に、さゆりの気持ちを伝えに行ったら、逆に、健から告白されちゃって。 
\\	もう自分の気持ちにうそをつけなくて。。。	
\\	そう。。綾香も、愛情と友情の板ばさみって苦しんでたってわけね。	
\\	そう。。綾香も、愛情と友情の板ばさみって苦しんでたってわけね。 
\\	それなら、さゆりに正直に話した方がいいよ。	
\\	それなら、さゆりに正直に話した方がいいよ。 
\\	高齢
\\	家族旅行
\\	コネクションルーム
\\	プレイルーム
\\	旅行代理店
\\	新設する
\\	近場
\\	三世代
\\	一同
\\	親戚
\\	年配
\\	いらっしゃいませ。	
\\	いらっしゃいませ。  
\\	あのー、高齢の両親と幼児を連れて、1泊2日の旅行に行きたいんだけど。。。	
\\	あのー、高齢の両親と幼児を連れて、1泊2日の旅行に行きたいんだけど。。。 
\\	全員で何名様ですか。	
\\	全員で何名様ですか。 
\\	えっと、両親と、姉の家族4人と、僕の家族4人だから、全部で10人だね。	
\\	えっと、両親と、姉の家族4人と、僕の家族4人だから、全部で10人だね。 
\\	ご親戚ご一同で行かれるんですね。	
\\	ご親戚ご一同で行かれるんですね。 
\\	うん、そう。だから、三世代の家族旅行向きのプランがいいんだよね。	
\\	うん、そう。だから、三世代の家族旅行向きのプランがいいんだよね。 
\\	そうですね。近場で、箱根はいかがですか。	
\\	そうですね。近場で、箱根はいかがですか。 
\\	やっぱり、箱根かね。じゃ、ホテルは、どこかお勧めある?	
\\	やっぱり、箱根かね。じゃ、ホテルは、どこかお勧めある? 
\\	少々お待ちください。	
\\	少々お待ちください。 
\\	(キーボードをたたく音)ホテル芦ノ湖の湯は、いかがでしょうか。	
\\	(キーボードをたたく音)ホテル芦ノ湖の湯は、いかがでしょうか。 
\\	ここは、最近、お子様向けにプレイルームを新設したばかりなんです。	
\\	ここは、最近、お子様向けにプレイルームを新設したばかりなんです。 
\\	それに、ご年配のお客様向けのお食事も用意されています。	
\\	それに、ご年配のお客様向けのお食事も用意されています。 
\\	水漏れ
\\	拝見する
\\	現物
\\	作業員
\\	出払う
\\	漏れる
\\	根元
\\	蛇口
\\	修理
\\	見積もり
\\	もしもし、水漏れの修理をお願いしたいんですが。	
\\	もしもし、水漏れの修理をお願いしたいんですが。 
\\	はい、どういった症状ですか。	
\\	はい、どういった症状ですか。 
\\	台所の蛇口の根元のあたりから、水がピューピュー漏れるんです。	
\\	台所の蛇口の根元のあたりから、水がピューピュー漏れるんです。  
\\	最初のうちは、何とか我慢して使ってたんですけど、だんだんひどくなってきてしまって。。。	
\\	最初のうちは、何とか我慢して使ってたんですけど、だんだんひどくなってきてしまって。。。 
\\	今すぐに来てもらえませんか。	
\\	今すぐに来てもらえませんか。 
\\	今、作業員が出払っていまして。。。早くても、明日のお昼過ぎになってしまいます。	
\\	今、作業員が出払っていまして。。。早くても、明日のお昼過ぎになってしまいます。 
\\	明日の午後ですか。。。何とか、朝のうちに来てもらえませんか。	
\\	明日の午後ですか。。。何とか、朝のうちに来てもらえませんか。 
\\	うーん、わかりました。じゃ、何とか、調整してみます。	
\\	うーん、わかりました。じゃ、何とか、調整してみます。 
\\	お願いします。それで、明日すぐ直るんですよね。	
\\	お願いします。それで、明日すぐ直るんですよね。 
\\	はぁ、それは、状況によりますから。。。その場で直せない場合もあります。	
\\	はぁ、それは、状況によりますから。。。その場で直せない場合もあります。 
\\	じゃ、修理代は、大体いくらくらいかかりますか。	
\\	じゃ、修理代は、大体いくらくらいかかりますか。 
\\	ですから、現物を拝見しないと、お見積もりもできないのですが、、、	
\\	ですから、現物を拝見しないと、お見積もりもできないのですが、、、 
\\	特殊な
\\	しかも
\\	儲ける
\\	至る
\\	点検代
\\	出張
\\	ざっくり
\\	パッキン
\\	部品
\\	その手
\\	あ~、これは、特殊な水栓ですね。	
\\	あ~、これは、特殊な水栓ですね。 
\\	部品交換では直らないです。	
\\	部品交換では直らないです。 
\\	水栓を全部取り替えることになりますね。	
\\	え~、パッキンの交換だけで済むと思ってたんだけど?!	
\\	え~、パッキンの交換だけで済むと思ってたんだけど?! 
\\	いえ、現物を拝見してからでないと、わからないと申し上げたと思いますが。。。	
\\	いえ、現物を拝見してからでないと、わからないと申し上げたと思いますが。。。 
\\	それで、全部取り替えたら、いくら位かかるの?	
\\	それで、全部取り替えたら、いくら位かかるの? 
\\	そうですね。ざっくり見積もって、7-8万円くらいになると思います。	
\\	そうですね。ざっくり見積もって、7-8万円くらいになると思います。 
\\	そんなに高いの?!	
\\	そんなに高いの?! 
\\	主人と相談してからでないと、決められないわ。	
\\	主人と相談してからでないと、決められないわ。 
\\	後で、正式な見積書送ってくださいね。	
\\	後で、正式な見積書送ってくださいね。 
\\	はい、わかりました。	
\\	はい、わかりました。 
\\	じゃ、今日は、出張点検代として3000円頂きます。	
\\	じゃ、今日は、出張点検代として3000円頂きます。 
\\	は?!そんなの聞いてないわよ。	
\\	は?!そんなの聞いてないわよ。 
\\	修理に至らなかった場合は、点検代を頂くことになっていますので。	
\\	修理に至らなかった場合は、点検代を頂くことになっていますので。 
\\	まったく!!しっかりしてるわね!	
\\	まったく!!しっかりしてるわね! 
\\	実は、点検代だけで儲けてるんじゃないの?	
\\	実は、点検代だけで儲けてるんじゃないの? 
\\	ただいま。	
\\	ただいま。 
\\	この水漏れ、水栓を全部取り替えなきゃ直らないって。	
\\	この水漏れ、水栓を全部取り替えなきゃ直らないって。 
\\	しかも、7-8万もかかるって!	
\\	しかも、7-8万もかかるって! 
\\	あー、やっぱり。	
\\	あー、やっぱり。 
\\	その手の業者は、すぐに、取り替えないと直らないって、言うらしいぞ!	
\\	その手の業者は、すぐに、取り替えないと直らないって、言うらしいぞ! 
\\	え、うそー!	
\\	え、うそー! 
\\	不正融資
\\	腹黒い
\\	くくる
\\	戦
\\	養う
\\	闘う
\\	転勤
\\	海外赴任
\\	告発
\\	腹
\\	俺、腹をくくったから。	
\\	俺、腹をくくったから。 
\\	えっ、何よ、いきなり?	
\\	えっ、何よ、いきなり? 
\\	会社の不正融資を告発することにした。	
\\	会社の不正融資を告発することにした。 
\\	えぇ! 何?不正融資?告発?	
\\	えぇ! 何?不正融資?告発?  
\\	もうちょっと詳しく説明してよ。	
\\	もうちょっと詳しく説明してよ。 
\\	この前、経理の資料を見てたら、経理担当の役員が不正融資をしていることに気づいたんだよ。	
\\	この前、経理の資料を見てたら、経理担当の役員が不正融資をしていることに気づいたんだよ。 
\\	それで、部長に相談しようとしたら、「腹を割って話そう」って言われたんだ。	
\\	"それで、部長に相談しようとしたら、「腹を割って話そう」って言われたんだ。 
\\	部長も知ってたらしい。	
\\	部長も知ってたらしい。 
\\	だから、俺にも目をつぶっているように説得しようとしたんだよ。	
\\	だから、俺にも目をつぶっているように説得しようとしたんだよ。 
\\	そのうち、「このことは、僕の腹にしまっておくから。	
\\	"そのうち、「このことは、僕の腹にしまっておくから。 
\\	あ、確か、君は海外赴任を希望していたねぇ・・・」なんて言い出して・・・。	
\\	"あ、確か、君は海外赴任を希望していたねぇ・・・」なんて言い出して・・・。 
\\	きっと、僕を転勤させる腹なんだ。	
\\	きっと、僕を転勤させる腹なんだ。 
\\	腹黒いわね。	
\\	腹黒いわね。 
\\	しかも、こういうことは、初めてじゃないんだ。	
\\	しかも、こういうことは、初めてじゃないんだ。 
\\	だから、今回こそ、事実を明らかにしないと、俺の腹の虫がおさまらないんだよ。	
\\	だから、今回こそ、事実を明らかにしないと、俺の腹の虫がおさまらないんだよ。 
\\	わかった。私も腹を決めたわ。	
\\	わかった。私も腹を決めたわ。 
\\	あなたが、会社と闘う間、私があなたを養うからね!	
\\	あなたが、会社と闘う間、私があなたを養うからね! 
\\	よし、そうと決まったら、「腹が減っては戦が出来ぬ」って言うから、おいしいものでも食べに行こう!	
\\	"よし、そうと決まったら、「腹が減っては戦が出来ぬ」って言うから、おいしいものでも食べに行こう! 
\\	教授
\\	招く
\\	著書
\\	飢える
\\	衝撃的
\\	事態
\\	天候不順
\\	不作
\\	洋風化
\\	今日は、東都大学の草柳教授をお招きして、先生の最近の著書「近い将来、日本が飢える!」について、お話をお伺いしたいと思います。	
\\	"今日は、東都大学の草柳教授をお招きして、先生の最近の著書「近い将来、日本が飢える!」について、お話をお伺いしたいと思います。 
\\	草柳先生、よろしくお願いします。	
\\	草柳先生、よろしくお願いします。 
\\	はい、こちらこそよろしくお願いします。	
\\	はい、こちらこそよろしくお願いします。 
\\	早速ですが、このタイトル、かなり衝撃的ですね。	
\\	早速ですが、このタイトル、かなり衝撃的ですね。 
\\	この豊かな国が飢えるということが、実際に起こりうることなのでしょうか。	
\\	"この豊かな国が飢えるということが、実際に起こりうることなのでしょうか。 
\\	私は、起こりえると考えています。	
\\	私は、起こりえると考えています。 
\\	日本の食糧自給率は、1965年の73%から下がり続け、今は40%しかないんです。	
\\	日本の食糧自給率は、1965年の73%から下がり続け、今は40%しかないんです。 
\\	ですから、万が一、食料を輸入できなくなるような事態になったら、日本人は、今の4割の食料で生きていかなければならないのです。	
\\	ですから、万が一、食料を輸入できなくなるような事態になったら、日本人は、今の4割の食料で生きていかなければならないのです。 
\\	なるほど。。。そういう事態が起きないとも限らないですね。	
\\	なるほど。。。そういう事態が起きないとも限らないですね。 
\\	天候不順で、輸入先の国で作物が不作にならないとも限らないですし。。。 
\\	それに、いつまでも平和が続くとは限らないですからね。	
\\	それに、いつまでも平和が続くとは限らないですからね。 
\\	ところで、自給率の低下の主な原因は、何なんでしょう?	
\\	ところで、自給率の低下の主な原因は、何なんでしょう? 
\\	やはり日本人の食生活の洋風化です。	
\\	やはり日本人の食生活の洋風化です。 
\\	米の自給率は100%なのに比べ、パンの材料である小麦粉の自給率は、10%程度しかありません。	
\\	米の自給率は100%なのに比べ、パンの材料である小麦粉の自給率は、10%程度しかありません。 
\\	ですから、パン食が多くなればなるほど、自給率は下がることになります。	
\\	ですから、パン食が多くなればなるほど、自給率は下がることになります。 
\\	人手
\\	営業
\\	案外
\\	平気
\\	外回り
\\	腰抜け
\\	叱り飛ばす
\\	イケてる
\\	気さく
\\	愛沙、あなた、最近、仕事ばっかりしてない?	
\\	"愛沙、あなた、最近、仕事ばっかりしてない? 
\\	就職するとき、「仕事なんて、結婚するまでの腰掛けだ」って言っていたのに。	
\\	就職するとき、「仕事なんて、結婚するまでの腰掛けだ」って言っていたのに。 
\\	入社当初は、そのつもりだったんだけど、この会社で腰を据えてがんばることにしたんだぁ。	
\\	入社当初は、そのつもりだったんだけど、この会社で腰を据えてがんばることにしたんだぁ。 
\\	なんで、また?何かあったの?	
\\	なんで、また?何かあったの? 
\\	最初は、営業アシスタントをしていたんだけど、最近の人手不足で営業もさせられるようになったわけ。	
\\	"最初は、営業アシスタントをしていたんだけど、最近の人手不足で営業もさせられるようになったわけ。 
\\	で、私、結構、営業向きだってことに気づいたんだよ。	
\\	で、私、結構、営業向きだってことに気づいたんだよ。 
\\	それで、本腰を入れて働くことにしたってわけか。	
\\	それで、本腰を入れて働くことにしたってわけか。 
\\	営業先でちょっと怒られると、腰が引けちゃう人って、案外多いのよね。でも、私は、全然平気。	
\\	営業先でちょっと怒られると、腰が引けちゃう人って、案外多いのよね。でも、私は、全然平気。 
\\	それに、腰が軽いほうだから、外回りも苦にならないし。	
\\	それに、腰が軽いほうだから、外回りも苦にならないし。 
\\	かい
\\	招待状
\\	引き受ける
\\	おめでたい
\\	手短に
\\	漕ぎ着ける
\\	盛大な
\\	披露宴
\\	賜る
\\	ゴールイン
\\	愛沙さんから、結婚式の招待状が来てるよ。	
\\	愛沙さんから、結婚式の招待状が来てるよ。 
\\	来た来た!いよいよ、愛沙も社長とゴールインか!	
\\	来た来た!いよいよ、愛沙も社長とゴールインか! 
\\	愛沙もがんばったかいがあったよね。	
\\	愛沙もがんばったかいがあったよね。 
\\	ちょっと見せて。おっ、さすが社長だけあって、一流ホテルでやるんだな。。。	
\\	ちょっと見せて。おっ、さすが社長だけあって、一流ホテルでやるんだな。。。 
\\	あれ、もえに、挨拶を頼みたいっていう紙が入っているぞ。	
\\	あれ、もえに、挨拶を頼みたいっていう紙が入っているぞ。 
\\	え!ほんと?	
\\	え!ほんと?  
\\	「竹内様には、友人代表としてご挨拶を賜りたく、お願い申し上げます」??	
\\	"「竹内様には、友人代表としてご挨拶を賜りたく、お願い申し上げます」?? 
\\	うっそー!ちょっと愛沙に電話して聞いてみる。	
\\	うっそー!ちょっと愛沙に電話して聞いてみる。 
\\	もしもし、愛沙?あたしだけど、披露宴で挨拶なんて、聞いてないけど?!	
\\	もしもし、愛沙?あたしだけど、披露宴で挨拶なんて、聞いてないけど?! 
\\	あ、そうそう、是非お願いよ!	
\\	あ、そうそう、是非お願いよ! 
\\	招待客300人の盛大な披露宴だから、もえも挨拶のしがいがあるでしょ。	
\\	招待客300人の盛大な披露宴だから、もえも挨拶のしがいがあるでしょ。 
\\	それに、もえのアドバイスのかいもあって、結婚まで漕ぎ着けたんだから!よろしくね!	
\\	それに、もえのアドバイスのかいもあって、結婚まで漕ぎ着けたんだから!よろしくね! 
\\	そう?	
\\	そう? 
\\	短くていいの。ほかにいっぱ~い挨拶を頼まなくちゃいけない人がいるからさ。手短に頼むわ!	
\\	短くていいの。ほかにいっぱ~い挨拶を頼まなくちゃいけない人がいるからさ。手短に頼むわ! 
\\	うーん。まぁ、おめでたいことだから、断っちゃいけないよね。・・・わかった。引き受ける。	
\\	うーん。まぁ、おめでたいことだから、断っちゃいけないよね。・・・わかった。引き受ける。 
\\	賜る
\\	20年来
\\	新婦
\\	末永い
\\	魅かれる
\\	取り組む
\\	お目当て
\\	仕事人間
\\	あずかる(与る)
\\	就職
\\	続きまして、新婦のご友人を代表して、竹内様にご挨拶を賜りたいと存じます。	
\\	続きまして、新婦のご友人を代表して、竹内様にご挨拶を賜りたいと存じます。 
\\	竹内様、よろしくお願いいたします。	
\\	竹内様、よろしくお願いいたします。 
\\	ただ今、ご紹介にあずかりました竹内と申します。	
\\	ただ今、ご紹介にあずかりました竹内と申します。 
\\	新婦の友人を代表して、一言お祝いの挨拶をさせていただきます。	
\\	新婦の友人を代表して、一言お祝いの挨拶をさせていただきます。 
\\	新婦の愛沙さんとは、中学校から大学まで同じ学校に通った20年来の親友です。	
\\	新婦の愛沙さんとは、中学校から大学まで同じ学校に通った20年来の親友です。 
\\	就職してからも、よく会って、遊んでおりました。	
\\	就職してからも、よく会って、遊んでおりました。 
\\	でも。。。確か、就職して、3年目くらいでしょうか。急に、愛沙さんが仕事人間になってしまったのです。	
\\	でも。。。確か、就職して、3年目くらいでしょうか。急に、愛沙さんが仕事人間になってしまったのです。 
\\	何かあったのかと思って聞いてみますと、実は、仕事というより、社長の拓也さんがお目当てだったわけです。	
\\	何かあったのかと思って聞いてみますと、実は、仕事というより、社長の拓也さんがお目当てだったわけです。 
\\	ちょ、ちょっと、そんなこと言わないでよ!(心の中)	
\\	愛沙さんは、昔から何事にも一生懸命でした。ですから、熱心に仕事に取り組む姿に、拓也さんも魅かれたのだと思います。	
\\	愛沙さんは、昔から何事にも一生懸命でした。ですから、熱心に仕事に取り組む姿に、拓也さんも魅かれたのだと思います。 
\\	そうそう、私のいいところを言ってよね。あ~、いろんなことが思い出される。なつかしい~。。	
\\	そうそう、私のいいところを言ってよね。あ~、いろんなことが思い出される。なつかしい~。。 
\\	あれ、もえのスピーチ、よく聞かないうちに、もう終わっちゃうわ。(心の中)	
\\	あれ、もえのスピーチ、よく聞かないうちに、もう終わっちゃうわ。(心の中) 
\\	お二人の末永いお幸せをお祈りして、私のお祝いの言葉といたします。	
\\	お二人の末永いお幸せをお祈りして、私のお祝いの言葉といたします。 
\\	情けない
\\	首を切る
\\	人情
\\	義理
\\	薄情
\\	押し売り
\\	情
\\	田舎
\\	出稼ぎ
\\	つぶれる
\\	あれ、何、この水?	
\\	あれ、何、この水? 
\\	それね、昼間、セールスの人が来て、契約しちゃったの。	
\\	それね、昼間、セールスの人が来て、契約しちゃったの。 
\\	サーバー代はサービスだって言うし、ペットボトルの水を買うより安上がりだって言うからいいかなって思って。。。	
\\	サーバー代はサービスだって言うし、ペットボトルの水を買うより安上がりだって言うからいいかなって思って。。。 
\\	そんな、セールストークに引っかかって、情けないなぁ。	
\\	そんな、セールストークに引っかかって、情けないなぁ。 
\\	それに、その人、青森から出稼ぎに来ていて、今月あと10件、契約を取らないと、お正月に田舎にも帰れないって言うのよ。	
\\	それに、その人、青森から出稼ぎに来ていて、今月あと10件、契約を取らないと、お正月に田舎にも帰れないって言うのよ。 
\\	つい、情に流されちゃってね。	
\\	つい、情に流されちゃってね。 
\\	情につけこむのは、押し売りのよく使う手じゃないか。	
\\	情につけこむのは、押し売りのよく使う手じゃないか。 
\\	私、そういう話を聞いちゃうと、どうしても薄情なことできないのよ。	
\\	私、そういう話を聞いちゃうと、どうしても薄情なことできないのよ。 
\\	だって、私の父は、特に情に厚い人だったでしょ。	
\\	だって、私の父は、特に情に厚い人だったでしょ。 
\\	まぁ、確かに、君の、お父さんは、義理と人情を大切にする人だったな。	
\\	まぁ、確かに、君の、お父さんは、義理と人情を大切にする人だったな。 
\\	不景気になっても、従業員の首を切らずにがんばっていたよね。	
\\	不景気になっても、従業員の首を切らずにがんばっていたよね。 
\\	でも、結局、会社つぶれちゃったじゃないか。情にもろいのも問題だよ。	
\\	でも、結局、会社つぶれちゃったじゃないか。情にもろいのも問題だよ。 
\\	そうね。私も、父に似て、情にもろい面もあるわね。	
\\	そうね。私も、父に似て、情にもろい面もあるわね。 
\\	そういえば、あなたと付き合う前、雨の日も風の日も、毎日私の家に花を届けてくれたでしょ。	
\\	そういえば、あなたと付き合う前、雨の日も風の日も、毎日私の家に花を届けてくれたでしょ。 
\\	その姿をみて、つい、情にほだされて付き合うことにしたんだったわ。。。	
\\	その姿をみて、つい、情にほだされて付き合うことにしたんだったわ。。。 
\\	金回り
\\	ぼろ儲け
\\	儲かる
\\	業界
\\	情報収集
\\	ばかばかしい
\\	利子
\\	預ける
\\	低金利
\\	痛い目
\\	響、お前、つい半年前に新車を買ったと思ったら、もう、また車、買い替えたのか?!	
\\	響、お前、つい半年前に新車を買ったと思ったら、もう、また車、買い替えたのか?! 
\\	なんで、そんなに金回りがいいんだよ。	
\\	なんで、そんなに金回りがいいんだよ。 
\\	まぁな。この低金利だろ。銀行に預けても、利子がほとんどつかなくて、ばかばかしくてさ。	
\\	まぁな。この低金利だろ。銀行に預けても、利子がほとんどつかなくて、ばかばかしくてさ。 
\\	だから最近、株を始めたんだよ。	
\\	だから最近、株を始めたんだよ。 
\\	株?そんなの、どうせ損するのが落ちだろ。	
\\	株?そんなの、どうせ損するのが落ちだろ。 
\\	そうとも限らないぞ!ちゃんと企業研究や情報収集をすれば、なんとか儲かるもんだぜ。	
\\	そうとも限らないぞ!ちゃんと企業研究や情報収集をすれば、なんとか儲かるもんだぜ。 
\\	へぇ、そうなのか。じゃ、俺もちょっと挑戦してみようかな。	
\\	へぇ、そうなのか。じゃ、俺もちょっと挑戦してみようかな。 
\\	初心者向きの株で、何かお勧めはあるか?	
\\	初心者向きの株で、何かお勧めはあるか? 
\\	そうだなぁ。環境関連とかバイオテクノロジー関係とかの株がいいんじゃないか。	
\\	そうだなぁ。環境関連とかバイオテクノロジー関係とかの株がいいんじゃないか。 
\\	これから成長が見込めそうな業界だからな。	
\\	これから成長が見込めそうな業界だからな。 
\\	なるほど。よし、お前が、絶対損しないって保証してくれるなら、1度買ってみようかな。	
\\	なるほど。よし、お前が、絶対損しないって保証してくれるなら、1度買ってみようかな。 
\\	何言ってるんだよ。そんな保証、誰ができるか!	
\\	何言ってるんだよ。そんな保証、誰ができるか! 
\\	株は自分の責任で買うもんだよ。	
\\	株は自分の責任で買うもんだよ。 
\\	自己責任だなんて、そんな冷たいこと言うなよ。	
\\	自己責任だなんて、そんな冷たいこと言うなよ。 
\\	響は、ぼろ儲けしているんだろう?	
\\	響は、ぼろ儲けしているんだろう? 
\\	俺だって、いろいろ痛い目にあってるんだよ。	
\\	俺だって、いろいろ痛い目にあってるんだよ。 
\\	絶対損しない株なんてものがあるなら、俺の方が知りたいよ!	
\\	絶対損しない株なんてものがあるなら、俺の方が知りたいよ! 
\\	鍋奉行
\\	頃合
\\	鉄則
\\	本格的
\\	手軽
\\	素
\\	鶏ガラ
\\	煮出す
\\	腕によりをかける
\\	面倒くさい
\\	おじゃま。お、光一も、来てたのか。	
\\	おじゃま。お、光一も、来てたのか。 
\\	よぉ、新平。男三人で忘年会ってのもさえないけどな。	
\\	確かにな(笑)(クンクン)	
\\	確かにな(笑)(クンクン) 
\\	うまそうなにおいだな。今日は鍋か?	
\\	うまそうなにおいだな。今日は鍋か? 
\\	忘年会といえば、やっぱり鍋でしょ!俺、鍋にはちょっとうるさいよ。	
\\	忘年会といえば、やっぱり鍋でしょ!俺、鍋にはちょっとうるさいよ。 
\\	いわゆる鍋奉行ってやつか。それは、楽しみだな。	
\\	"いわゆる鍋奉行ってやつか。それは、楽しみだな。 
\\	じゃーん、今日のだしは、俺が腕によりを掛けて、鶏ガラを煮出してとったスープだぞ!	
\\	じゃーん、今日のだしは、俺が腕によりを掛けて、鶏ガラを煮出してとったスープだぞ! 
\\	鶏ガラスープの素は手軽だけど、本格的な味を出すには、やっぱり鶏ガラから煮出すに限るよ。	
\\	鶏ガラスープの素は手軽だけど、本格的な味を出すには、やっぱり鶏ガラから煮出すに限るよ。 
\\	すげぇいいにおい!さっそく、この肉とか野菜、入れていい?	
\\	すげぇいいにおい!さっそく、この肉とか野菜、入れていい? 
\\	ちょ、ちょっと待て。入れる順番があるんだ。	
\\	ちょ、ちょっと待て。入れる順番があるんだ。 
\\	まずは、火が通りにくい野菜と肉類から入れるのが鉄則なんだ。	
\\	まずは、火が通りにくい野菜と肉類から入れるのが鉄則なんだ。 
\\	頃合をみて、最後に白身魚とか、貝類を入れるんだ。火が通り過ぎて硬くなっちゃうからな。	
\\	頃合をみて、最後に白身魚とか、貝類を入れるんだ。火が通り過ぎて硬くなっちゃうからな。 
\\	ほぉぉ。なるほど。	
\\	ほぉぉ。なるほど。 
\\	さ、そろそろ、この牡蠣は食べごろだぞ。	
\\	さ、そろそろ、この牡蠣は食べごろだぞ。 
\\	牡蠣といえば、白ワインだよな。	
\\	牡蠣といえば、白ワインだよな。 
\\	あっ!しまった、俺としたことが、白ワインを買い忘れた!	
\\	あっ!しまった、俺としたことが、白ワインを買い忘れた! 
\\	まぁ、いいよいいよ。	
\\	まぁ、いいよいいよ。 
\\	それより、この牡蠣、最高だぜ!面倒くさいから、ここにある牡蠣、全部鍋に入れちゃおうぜ。	
\\	それより、この牡蠣、最高だぜ!面倒くさいから、ここにある牡蠣、全部鍋に入れちゃおうぜ。 
\\	だめだめ!次の牡蠣を入れるのは、今鍋の中にあるのを全部食べきってからだよ!	
\\	だめだめ!次の牡蠣を入れるのは、今鍋の中にあるのを全部食べきってからだよ! 
\\	保険
\\	テロ
\\	無駄
\\	加入
\\	親知らず
\\	一人旅
\\	盗難
\\	天災
\\	傷害保険
\\	代理店
\\	もしもし?	
\\	もしもし? 
\\	ユウちゃん?久しぶり。大学はもう夏休みに入ったんでしょ?いつ帰ってくるの?お盆には帰ってくるんでしょ。	
\\	ユウちゃん?久しぶり。大学はもう夏休みに入ったんでしょ?いつ帰ってくるの?お盆には帰ってくるんでしょ。 
\\	…それがさ、明日から海外に一人旅に行こうと思って…。	
\\	…それがさ、明日から海外に一人旅に行こうと思って…。 
\\	明日?まったく、親の心子知らずよね。ユウちゃんが帰ってくるのをお母さん楽しみに待っていたのに…。あ、保険には入っておいてよ。	
\\	明日?まったく、親の心子知らずよね。ユウちゃんが帰ってくるのをお母さん楽しみに待っていたのに…。あ、保険には入っておいてよ。 
\\	なんだよ、それ。俺が死ねばいいとか思っているわけ?	
\\	なんだよ、それ。俺が死ねばいいとか思っているわけ? 
\\	違うわよ。盗難とか、事故とか、テロとか、天災とかあるかもしれないでしょ。空港に保険の代理店があるから、そこで海外旅行傷害保険に入ってから行きなさい。	
\\	違うわよ。盗難とか、事故とか、テロとか、天災とかあるかもしれないでしょ。空港に保険の代理店があるから、そこで海外旅行傷害保険に入ってから行きなさい。 
\\	えー。金の無駄だよ。	
\\	えー。金の無駄だよ。 
\\	そんなこと言わずに加入しなさい。転ばぬ先の杖っていうでしょ。いいわね。	
\\	そんなこと言わずに加入しなさい。転ばぬ先の杖っていうでしょ。いいわね。 
\\	はいはい。	
\\	はいはい。 
\\	「はい」は、一回。それから、見知らぬ人に声をかけられてもついて行ったらダメだからね。	
\\	"「はい」は、一回。それから、見知らぬ人に声をかけられてもついて行ったらダメだからね。 
\\	はーい。	
\\	はーい。 
\\	それから…	
\\	それから… 
\\	母さん、もういいかな。親知らずが痛くて、これから歯医者に行くんだけど…。	
\\	母さん、もういいかな。親知らずが痛くて、これから歯医者に行くんだけど…。 
\\	そうなの。じゃ、早く行ってきなさい。じゃあ切るわよ、またね。	
\\	そうなの。じゃ、早く行ってきなさい。じゃあ切るわよ、またね。 
\\	ぱあ
\\	海底火山
\\	津波
\\	まさか
\\	災い
\\	腫れる
\\	付近
\\	噴火
\\	地震
\\	福
\\	ただいま。	
\\	ただいま。 
\\	あれ、どうしたんだ、優太?
\\	島に行ったんじゃないのか。	
\\	あれ、どうしたんだ、優太?
\\	島に行ったんじゃないのか。 
\\	そのつもりだったんだけど…。	
\\	そのつもりだったんだけど…。 
\\	旅行の前日に親知らずを抜いたらさ、血は止まらないわ、顔は腫れるわ、熱はでるわ…。それで、出発できなかったんだよ。	
\\	旅行の前日に親知らずを抜いたらさ、血は止まらないわ、顔は腫れるわ、熱はでるわ…。それで、出発できなかったんだよ。 
\\	別の日の飛行機に変えられただろうに。	
\\	別の日の飛行機に変えられただろうに。 
\\	安い航空券だからさ、他の便への変更はできなくてさ、航空券はぱあだよ。	
\\	安い航空券だからさ、他の便への変更はできなくてさ、航空券はぱあだよ。 
\\	ま、飛行機しか予約してなかったのは、不幸中の幸いだったけど。	
\\	ま、飛行機しか予約してなかったのは、不幸中の幸いだったけど。 
\\	若いんだから、またそのうち行けるさ。	
\\	若いんだから、またそのうち行けるさ。 
\\	お父さん、大変、
\\	島付近の海底火山が噴火して大地震だって。	
\\	お父さん、大変、
\\	島付近の海底火山が噴火して大地震だって。 
\\	津波も起きたみたいで、…あら?優太、どうしたの?
\\	島行かなかったの?	
\\	津波も起きたみたいで、…あら?優太、どうしたの?
\\	島行かなかったの? 
\\	親知らずを抜いたら熱がでちゃってさ、行くのやめたんだ。	
\\	親知らずを抜いたら熱がでちゃってさ、行くのやめたんだ。 
\\	よかったー。塞翁が馬ね。	
\\	よかったー。塞翁が馬ね。 
\\	ぬいぐるみ
\\	販売
\\	背後
\\	射殺
\\	全治
\\	離れる
\\	熊
\\	襲う
\\	重傷
\\	負う
\\	宅
\\	殺す
\\	お帰りなさい。東京出張はどうだった。	
\\	お帰りなさい。東京出張はどうだった。 
\\	出かけたっきり連絡ないから心配してたのよ。	
\\	出かけたっきり連絡ないから心配してたのよ。 
\\	ごめん、忙しくてさ。出張はこれっきりにしてほしいよ。	
\\	ごめん、忙しくてさ。出張はこれっきりにしてほしいよ。 
\\	お帰り、パパ。 
\\	はい、お土産。貞子の好きな熊のプー様のぬいぐるみだよ。	
\\	はい、お土産。貞子の好きな熊のプー様のぬいぐるみだよ。 
\\	わーい。ありがとう。	
\\	わーい。ありがとう。 
\\	それ、東京だけでしか買えないんだぞ。	
\\	"それ、東京だけでしか買えないんだぞ。 
\\	あら、ホント。東京でのみ販売って書いてある。何か食べる?	
\\	あら、ホント。東京でのみ販売って書いてある。何か食べる? 
\\	ビールだけもらうよ。	
\\	ビールだけもらうよ。 
\\	今、用意するね。	
\\	今、用意するね。 
\\	今日の午後 山形県港市の住宅街にて、女性が熊に襲われ重傷を負いました。	
\\	今日の午後 山形県港市の住宅街にて、女性が熊に襲われ重傷を負いました。 
\\	港市?すぐ近くじゃないか。	
\\	港市?すぐ近くじゃないか。 
\\	本日午後、一時半頃 山形県港市に住む持田金さん89歳が自宅の庭で掃除をしていたところ 背後から熊に襲われました。	
\\	本日午後、一時半頃 山形県港市に住む持田金さん89歳が自宅の庭で掃除をしていたところ 背後から熊に襲われました。 
\\	持田さんは顔の骨を折るなど、全治2ヶ月の怪我を負いました。	
\\	持田さんは顔の骨を折るなど、全治2ヶ月の怪我を負いました。 
\\	熊はその後、持田さん宅から1キロメートル離れた公園で遊んでいるところを射殺されました。	
\\	熊はその後、持田さん宅から1キロメートル離れた公園で遊んでいるところを射殺されました。 
\\	熊、殺されちゃったの?かわいそう。	
\\	熊、殺されちゃったの?かわいそう。 
\\	各地
\\	恐れる
\\	再生
\\	不足
\\	解決
\\	住宅地
\\	食料
\\	出没
\\	温暖化
\\	枯れる
\\	酸性雨
\\	被害
\\	人を襲ったら、殺されちゃうのに、どうして熊は人を襲っちゃったんだろう。	
\\	人を襲ったら、殺されちゃうのに、どうして熊は人を襲っちゃったんだろう。 
\\	熊がそんなこと分かるはずないでしょ。襲われた人こそかわいそうよ。	
\\	熊がそんなこと分かるはずないでしょ。襲われた人こそかわいそうよ。 
\\	熊は山に食料がないから人の住むところに食べ物を探しに来るんだろうね。	
\\	熊は山に食料がないから人の住むところに食べ物を探しに来るんだろうね。 
\\	人を恐れていればこそ、襲うらしいよ。	
\\	人を恐れていればこそ、襲うらしいよ。 
\\	各地で熊の被害が続いています。何が原因なのか、森大学の熊谷教授にインタビューをしてみました。	
\\	各地で熊の被害が続いています。何が原因なのか、森大学の熊谷教授にインタビューをしてみました。 
\\	熊が住宅地に出没するのは熊のえさが山にほとんどないからなんですよ。	
\\	熊が住宅地に出没するのは熊のえさが山にほとんどないからなんですよ。 
\\	温暖化や酸性雨のせいで木が枯れてきているのも原因の一つと言えますね。	
\\	温暖化や酸性雨のせいで木が枯れてきているのも原因の一つと言えますね。 
\\	つまり人間こそが全ての原因を作った…。	
\\	つまり人間こそが全ての原因を作った…。 
\\	そう。森が再生しないと、本当の意味での問題は解決しないでしょうね。	
\\	そう。森が再生しないと、本当の意味での問題は解決しないでしょうね。 
\\	熊も食料不足で大変なのね…。でも、怖いわ。あの公園、貞子の小学校から目と鼻の先でしょ。	
\\	熊も食料不足で大変なのね…。でも、怖いわ。あの公園、貞子の小学校から目と鼻の先でしょ。 
\\	辞令	
\\	辞令 
\\	海外営業部 佐藤元気殿	
\\	海外営業部 佐藤元気殿 
\\	香港支店転勤を命じる	
\\	香港支店転勤を命じる 
\\	皆さん、ご無沙汰してます。	
\\	皆さん、ご無沙汰してます。 
\\	突然ですが、会社から辞令が出て、来月から香港に転勤することになりました。	
\\	突然ですが、会社から辞令が出て、来月から香港に転勤することになりました。 
\\	出発まであと一ヶ月しかないのですが、皆さんさえよければ、飲みましょう。	
\\	出発まであと一ヶ月しかないのですが、皆さんさえよければ、飲みましょう。 
\\	えー。中学時代はアルファベットすら書けなかった元気が、海外に転勤することになりました。	
\\	えー。中学時代はアルファベットすら書けなかった元気が、海外に転勤することになりました。 
\\	元気が元気に行って帰ってくることを祈って乾杯したいと思います。行ってらっしゃーい。乾杯!	
\\	元気が元気に行って帰ってくることを祈って乾杯したいと思います。行ってらっしゃーい。乾杯! 
\\	乾杯!	
\\	乾杯! 
\\	お前すごいな。中国語まで話せるのか?	
\\	お前すごいな。中国語まで話せるのか? 
\\	いや、全然。でも、仕事は英語さえできれば大丈夫なんだよ。 
\\	いいなぁ。元気は大学時代に留学したから、英語ができて。	
\\	いいなぁ。元気は大学時代に留学したから、英語ができて。 
\\	俺には海外駐在員になる事はもちろん海外出張することすら夢の夢だよ。	
\\	俺には海外駐在員になる事はもちろん海外出張することすら夢の夢だよ。 
\\	でも、春樹の会社って最近アメリカの会社に買収されただろ?	
\\	でも、春樹の会社って最近アメリカの会社に買収されただろ? 
\\	見当
\\	友情
\\	きざ
\\	振る
\\	赴任
\\	期間
\\	飽きる
\\	色っぽい
\\	悩む
\\	理屈
\\	香港へは一人で行くの?	
\\	香港へは一人で行くの? 
\\	ああ。今、彼女いないからね。	
\\	ああ。今、彼女いないからね。 
\\	赴任期間はどのくらい?	
\\	赴任期間はどのくらい? 
\\	さあ。見当もつかないよ。	
\\	さあ。見当もつかないよ。 
\\	(男子トイレ) 
\\	志保ちゃんとは、うまくいっているのか?	
\\	志保ちゃんとは、うまくいっているのか? 
\\	あんなに女らしい子と付き合っているなんてうらやましいよ。	
\\	あんなに女らしい子と付き合っているなんてうらやましいよ。 
\\	別れたよ。俺、飽きっぽいからさ。	
\\	別れたよ。俺、飽きっぽいからさ。 
\\	なんでだよ!俺...相手がお前だから、志保ちゃんのことあきらめたんだぞ。	
\\	なんでだよ!俺...相手がお前だから、志保ちゃんのことあきらめたんだぞ。 
\\	やっと本当の事言ったな。お前らしいな。	
\\	やっと本当の事言ったな。お前らしいな。 
\\	でも安っぽい友情なんていらないんだよ。	
\\	でも安っぽい友情なんていらないんだよ。 
\\	は?	
\\	は? 
\\	俺、振られたんだ。中学の頃からずっと好きだった人がいて、そいつのことが忘れられないらしいぞ。	
\\	俺、振られたんだ。中学の頃からずっと好きだった人がいて、そいつのことが忘れられないらしいぞ。 
\\	え?	
\\	え? 
\\	お前以外にいないだろう。	
\\	お前以外にいないだろう。 
\\	でも...。	
\\	でも...。 
\\	何を悩んでいるんだよ。今夜が最後のチャンスだろ。当たって砕けろよ。	
\\	何を悩んでいるんだよ。今夜が最後のチャンスだろ。当たって砕けろよ。 
\\	おお。	
\\	おお。 
\\	(女子トイレ)	
\\	(女子トイレ) 
\\	志保、その服色っぽい。	
\\	志保、その服色っぽい。 
\\	ありがとう。元気君に会えるからおしゃれしたんだけど、残念。	
\\	ありがとう。元気君に会えるからおしゃれしたんだけど、残念。 
\\	中学のときは男らしくて素敵だったのに。	
\\	中学のときは男らしくて素敵だったのに。 
\\	吐き気
\\	食あたり
\\	リン
\\	毒
\\	占い師
\\	下痢
\\	吐く
\\	寒気
\\	めまい
\\	食中毒
\\	症状
\\	おいしそう。いただきます。	
\\	おいしそう。いただきます。 
\\	あれ?いたた。	
\\	あれ?いたた。 
\\	白雪さん、白雪姫さん、どうぞお入りください。	
\\	白雪さん、白雪姫さん、どうぞお入りください。 
\\	どうしましたか。	
\\	どうしましたか。 
\\	お昼を食べてからお腹の調子が悪いんです。	
\\	お昼を食べてからお腹の調子が悪いんです。 
\\	下痢と腹痛ですか。他に症状はありますか。	
\\	下痢と腹痛ですか。他に症状はありますか。 
\\	例えば、吐き気がするとか。熱があるとか。	
\\	例えば、吐き気がするとか。熱があるとか。 
\\	熱はわかりません。ここに来るまでに何回も吐いてしまいました。	
\\	熱はわかりません。ここに来るまでに何回も吐いてしまいました。 
\\	胃がまだむかむかします。寒気もしますし、めまいもします。	
\\	胃がまだむかむかします。寒気もしますし、めまいもします。 
\\	典型的な食中毒のようですが、ちょっと調べてみましょう。	
\\	典型的な食中毒のようですが、ちょっと調べてみましょう。 
\\	やっぱり食あたりですか?なんだか、目がちかちかしてきたんですが...。	
\\	やっぱり食あたりですか?なんだか、目がちかちかしてきたんですが...。 
\\	毒性のあるものを食べたと思うんです。何を食べましたか。	
\\	毒性のあるものを食べたと思うんです。何を食べましたか。 
\\	このリンゴを一口だけ食べたんです。	
\\	このリンゴを一口だけ食べたんです。 
\\	これは、どこで手に入れたんですか。	
\\	これは、どこで手に入れたんですか。 
\\	占い師風のおばあさんにもらいました。	
\\	占い師風のおばあさんにもらいました。 
\\	専門的なことは、調べてみないとわかりませんが、リン系の毒を食べてしまったのかも知れません。	
\\	専門的なことは、調べてみないとわかりませんが、リン系の毒を食べてしまったのかも知れません。 
\\	答案
\\	砂
\\	感動
\\	活躍
\\	だらけ
\\	期末テスト
\\	汗
\\	甲子園
\\	泥
\\	ばつ
\\	はぁ...。	
\\	はぁ...。 
\\	どうしたんですか?原先生?	
\\	どうしたんですか?原先生? 
\\	見てくださいよ。坂東君の期末テストの答案。	
\\	見てくださいよ。坂東君の期末テストの答案。 
\\	間違いだらけなんですよ。勉強したのかしら。	
\\	間違いだらけなんですよ。勉強したのかしら。 
\\	はは。こりゃひどいな。	
\\	はは。こりゃひどいな。 
\\	でも、うちの長島の答案もひどいですよ。 
\\	あら。バツだらけ。どんぐりの背比べね。	
\\	あら。バツだらけ。どんぐりの背比べね。 
\\	まあ、彼らは野球部だから仕方ないですよ。	
\\	まあ、彼らは野球部だから仕方ないですよ。 
\\	朝から晩まで砂まみれ、汗まみれで野球を練習しているんですから。	
\\	朝から晩まで砂まみれ、汗まみれで野球を練習しているんですから。 
\\	甲子園が終わるまでは、勉強どころじゃないでしょうね。	
\\	甲子園が終わるまでは、勉強どころじゃないでしょうね。 
\\	大目に見ましょうよ。	
\\	大目に見ましょうよ。 
\\	そうですね。	
\\	そうですね。 
\\	これで甲子園に出て活躍してプロに入れればいい事ずくめですからね。	
\\	これで甲子園に出て活躍してプロに入れればいい事ずくめですからね。 
\\	彼らはそんな理由で野球をしているんじゃありませんよ。	
\\	彼らはそんな理由で野球をしているんじゃありませんよ。 
\\	プロになれても芽が出るのは一部ですし、プロ選手達は天才だらけです。	
\\	プロになれても芽が出るのは一部ですし、プロ選手達は天才だらけです。 
\\	第一、坂東や長島のレベルではプロになることも無理です。	
\\	第一、坂東や長島のレベルではプロになることも無理です。 
\\	そうですか。	
\\	そうですか。 
\\	でも、先のことを考えずに泥だらけになって一生懸命頑張っている彼らを見ると、僕は、感動するんです。	
\\	でも、先のことを考えずに泥だらけになって一生懸命頑張っている彼らを見ると、僕は、感動するんです。 
\\	会話を通して聞く
\\	会話を通して聞く
\\	効果的
\\	評論家
\\	幻想的
\\	環境汚染
\\	夜景
\\	近未来
\\	風景
\\	都市
\\	強調
\\	人工的
\\	皆さんは「工場夜景」がブームになっていることをご存知ですか?	
\\	"皆さんは「工場夜景」がブームになっていることをご存知ですか? 
\\	今回は、夜景評論家の宮沢賢さんに工場を案内してもらいます。	
\\	今回は、夜景評論家の宮沢賢さんに工場を案内してもらいます。 
\\	よろしくお願いします。	
\\	よろしくお願いします。 
\\	さて、工場夜景ですが、どのような点が人気なのでしょうか。	
\\	さて、工場夜景ですが、どのような点が人気なのでしょうか。 
\\	百聞は一見にしかず。こちらへどうぞ。	
\\	百聞は一見にしかず。こちらへどうぞ。 
\\	うわ。きれい…。とても人工的で、幻想的な風景ですね。	
\\	うわ。きれい…。とても人工的で、幻想的な風景ですね。 
\\	そうでしょ。	
\\	そうでしょ。 
\\	近未来の都市にいるみたいです。	
\\	近未来の都市にいるみたいです。 
\\	人気の理由は色々あるでしょうが、工場夜景はSFのような風景をみることができるという点で人気があるのだと思います。	
\\	人気の理由は色々あるでしょうが、工場夜景はSFのような風景をみることができるという点で人気があるのだと思います。 
\\	なるほど。	
\\	なるほど。 
\\	工場ってマイナス面ばかりが強調されてきましたよね。環境汚染とか。	
\\	工場ってマイナス面ばかりが強調されてきましたよね。環境汚染とか。 
\\	はい。	
\\	はい。 
\\	今の工場は環境面を考えているから安全ですし、工場はわれわれの生活に必要なものです。	
\\	今の工場は環境面を考えているから安全ですし、工場はわれわれの生活に必要なものです。 
\\	工場全体のイメージをよくするという点でも、工場夜景見学は効果的なんですよ。	
\\	工場全体のイメージをよくするという点でも、工場夜景見学は効果的なんですよ。 
\\	なるほど。私も誰か素敵な人とこの夜景を見に来たいと思います。	
\\	なるほど。私も誰か素敵な人とこの夜景を見に来たいと思います。 
\\	すみませんねぇ。今日は私と一緒で…。	
\\	すみませんねぇ。今日は私と一緒で…。 
\\	あ、そういう意味じゃないです。	
\\	あ、そういう意味じゃないです。 
\\	-周年
\\	首相
\\	辞任
\\	失言
\\	浮気
\\	怠け者
\\	呼び出す
\\	緊急
\\	選挙
\\	税金
\\	支持率
\\	総理
\\	お疲れ。	
\\	お疲れ。 
\\	大変でしたね。急に呼び出されて。	
\\	大変でしたね。急に呼び出されて。 
\\	本当だよ。結婚10周年だからレストラン予約して舌鼓を打ってたのに、いきなり会社から呼び出しだよ。	
\\	本当だよ。結婚10周年だからレストラン予約して舌鼓を打ってたのに、いきなり会社から呼び出しだよ。 
\\	奥さん、大丈夫でした?	
\\	奥さん、大丈夫でした? 
\\	舌打ちされたよ。で、何?総理辞任?	
\\	こんな時間に緊急記者会見をするんだから、多分そうだと思いますよ。	
\\	こんな時間に緊急記者会見をするんだから、多分そうだと思いますよ。 
\\	前回の選挙で、「税金は上げない」って言ったのに、その舌の根の乾かぬうちに消費税を10パーセントにするなんていうから、支持率が下がったんだ。	
\\	前回の選挙で、「税金は上げない」って言ったのに、その舌の根の乾かぬうちに消費税を10パーセントにするなんていうから、支持率が下がったんだ。 
\\	それから、あの人、本当に話すの下手ですよね。	
\\	それから、あの人、本当に話すの下手ですよね。 
\\	舌足らずな話し方だし、失言も多いし。	
\\	舌足らずな話し方だし、失言も多いし。 
\\	「浮気は文化だ」でしたっけ?	
\\	"「浮気は文化だ」でしたっけ? 
\\	そうそう。「北海道の人は怠け者で、メロンも育てられない」とも言ったよな。	
\\	"そうそう。「北海道の人は怠け者で、メロンも育てられない」とも言ったよな。 
\\	あれは、まずかったな。	
\\	あれは、まずかったな。 
\\	言う事もころころ変えるから、「二枚舌」なんて外国メディアに悪口書かれちゃいましたしね。	
\\	"言う事もころころ変えるから、「二枚舌」なんて外国メディアに悪口書かれちゃいましたしね。 
\\	ま、舌先三寸なのは日本の首相だけじゃないけどな。	
\\	ま、舌先三寸なのは日本の首相だけじゃないけどな。 
\\	確かに。	
\\	確かに。 
\\	あ、来た!	
\\	あ、来た! 
\\	第一印象
\\	進学
\\	手法
\\	関心
\\	手土産
\\	研究室
\\	修士課程
\\	深める
\\	研究
\\	遠隔教育
\\	天下大学	
\\	天下大学 
\\	土屋博先生	
\\	突然のメールにて失礼いたします。私は、現在、羽田大学で
\\	ラーニングについて研究しておりますが、さらに研究を深めるべく、貴大学の修士課程への進学を希望しております。	
\\	突然のメールにて失礼いたします。私は、現在、羽田大学で
\\	ラーニングについて研究しておりますが、さらに研究を深めるべく、貴大学の修士課程への進学を希望しております。 
\\	土屋先生の研究テーマである「効果的な遠隔教育の手法」に関心を持っており、先生のご都合のよろしいときに、お会いして研究室や大学院での研究生活についてお話を伺いたいと思っております。	
\\	"土屋先生の研究テーマである「効果的な遠隔教育の手法」に関心を持っており、先生のご都合のよろしいときに、お会いして研究室や大学院での研究生活についてお話を伺いたいと思っております。 
\\	ご多忙のところ恐縮ではございますが、お時間をいただくことが可能かどうかご返信をいただければ幸いに存じます。	
\\	ご多忙のところ恐縮ではございますが、お時間をいただくことが可能かどうかご返信をいただければ幸いに存じます。 
\\	佐藤学	
\\	佐藤学 
\\	よし、書き終えた。でも念には念を入れて、送信する前に間違いがないか誰かにチェックしてもらうべきだな。	
\\	よし、書き終えた。でも念には念を入れて、送信する前に間違いがないか誰かにチェックしてもらうべきだな。 
\\	先輩に研究室を見に行くべきだって言われたけど、わからないことだらけだよ。	
\\	先輩に研究室を見に行くべきだって言われたけど、わからないことだらけだよ。 
\\	第一印象は大事だっていうから、研究室へはスーツで行くべきかなぁ?	
\\	第一印象は大事だっていうから、研究室へはスーツで行くべきかなぁ? 
\\	あと、手土産は持っていくべきなのか?	
\\	あと、手土産は持っていくべきなのか? 
\\	受験
\\	入試
\\	関して
\\	願書
\\	名門
\\	対策
\\	訪問
\\	肝心
\\	院生
\\	書類
\\	よう。研究室訪問どうだった。	
\\	よう。研究室訪問どうだった。 
\\	先輩のアドバイス通り、スーツで行って手土産を持って行きました。	
\\	先輩のアドバイス通り、スーツで行って手土産を持って行きました。 
\\	肝心な研究内容や入試についても聞いてきたんだろう。	
\\	肝心な研究内容や入試についても聞いてきたんだろう。 
\\	はい。自分がやりたいEラーニングのシステムの研究ができそうです。	
\\	はい。自分がやりたいEラーニングのシステムの研究ができそうです。 
\\	入試に関しては、天下大以外の大学からでも合格しにくいということはない…って言ってました。	
\\	入試に関しては、天下大以外の大学からでも合格しにくいということはない…って言ってました。 
\\	へー。	
\\	へー。 
\\	研究室の院生を何人か紹介してくれて、院生からも色々聞く事ができました。	
\\	研究室の院生を何人か紹介してくれて、院生からも色々聞く事ができました。 
\\	そりゃ良かった。	
\\	そりゃ良かった。 
\\	先生には直接聞きづらいことでも、院生には聞きやすいしな。	
\\	先生には直接聞きづらいことでも、院生には聞きやすいしな。 
\\	はい。院生に天下大の大学院に合格するのはかなり難しいから、天下大以外の大学院も受験しておいた方が良いって言われました。	
\\	はい。院生に天下大の大学院に合格するのはかなり難しいから、天下大以外の大学院も受験しておいた方が良いって言われました。 
\\	優秀な学生ばかりで選びがたい…って毎年教授も悩むそうです。	
\\	優秀な学生ばかりで選びがたい…って毎年教授も悩むそうです。 
\\	そっか。	
\\	そっか。 
\\	なので、帝国大学の大学院にも願書を出そうと思います。	
\\	なので、帝国大学の大学院にも願書を出そうと思います。 
\\	天下大と帝国大は甲乙つけがたい名門だな。	
\\	天下大と帝国大は甲乙つけがたい名門だな。 
\\	書類の準備と入試対策も頑張れよ。	
\\	書類の準備と入試対策も頑張れよ。 
\\	あ、何か相談あったら、連絡してこいよ。	
\\	あ、何か相談あったら、連絡してこいよ。 
\\	ありがとうございます。	
\\	ありがとうございます。 
\\	<会話を通して聞く>	
\\	<会話を通して聞く> 
\\	事情
\\	不備
\\	リストアップ
\\	速達
\\	筆記試験
\\	成績証明書
\\	志望
\\	書留
\\	研究計画書、志望理由書、成績証明書…。でも、不備があり得るかも。	
\\	研究計画書、志望理由書、成績証明書…。でも、不備があり得るかも。 
\\	念のため、もう一度、チェック…よし!	
\\	念のため、もう一度、チェック…よし! 
\\	次の方どうぞ。	
\\	次の方どうぞ。 
\\	これを書留で送ってください。あ…速達でお願いします。	
\\	これを書留で送ってください。あ…速達でお願いします。 
\\	明日には届きますよね。	
\\	明日には届きますよね。 
\\	ええ、明日の午前中には届く予定ですが、交通などの事情で遅れることもあり得ます。大丈夫ですか。	
\\	ええ、明日の午前中には届く予定ですが、交通などの事情で遅れることもあり得ます。大丈夫ですか。 
\\	はい。	
\\	はい。 
\\	660円です。	
\\	660円です。 
\\	健二先輩、今、時間、大丈夫ですか?	
\\	健二先輩、今、時間、大丈夫ですか? 
\\	おう。どうした?	
\\	おう。どうした? 
\\	先輩に見てもらった研究計画書を送ったんですけど…。	
\\	先輩に見てもらった研究計画書を送ったんですけど…。 
\\	あとは試験と面接だな。	
\\	あとは試験と面接だな。 
\\	はい…実は、面接が僕苦手で…。	
\\	はい…実は、面接が僕苦手で…。 
\\	じゃ、聞かれそうなことをすべてリストアップすることだな。	
\\	じゃ、聞かれそうなことをすべてリストアップすることだな。 
\\	リストアップ…。	
\\	リストアップ…。 
\\	ああ。面接では緊張して、頭が真っ白になる…ってことも起こり得る。	
\\	ああ。面接では緊張して、頭が真っ白になる…ってことも起こり得る。 
\\	答えを言う練習もしておいたほうがいいぞ。	
\\	答えを言う練習もしておいたほうがいいぞ。 
\\	なるほど。	
\\	なるほど。 
\\	筆記試験は大丈夫?	
\\	筆記試験は大丈夫? 
\\	出題されそうなものは一応全部勉強しました。	
\\	出題されそうなものは一応全部勉強しました。 
\\	作成
\\	連携
\\	システム構築
\\	充実
\\	具体的
\\	提示
\\	効率的
\\	活用
\\	簡潔
\\	理論
\\	把握
\\	失礼いたします。羽田大学より来ました佐藤学です。	
\\	失礼いたします。羽田大学より来ました佐藤学です。 
\\	よろしくお願いします。	
\\	よろしくお願いします。 
\\	まず、本大学院を志望する理由と研究テーマについて簡潔に述べてください。	
\\	まず、本大学院を志望する理由と研究テーマについて簡潔に述べてください。 
\\	私の研究テーマは、Eラーニングを活用した外国語学習に関するものです。	
\\	私の研究テーマは、Eラーニングを活用した外国語学習に関するものです。 
\\	効率的に外国語をマスターできる学習システムを作りたいと考えています。	
\\	効率的に外国語をマスターできる学習システムを作りたいと考えています。 
\\	大学院では、システム構築の理論を基礎から勉強し、実際にシステムを作る技術を身に付けたいため、教育の充実した天下大学院を志望しています。	
\\	大学院では、システム構築の理論を基礎から勉強し、実際にシステムを作る技術を身に付けたいため、教育の充実した天下大学院を志望しています。 
\\	どんな学習システムを具体的にイメージしていますか?	
\\	どんな学習システムを具体的にイメージしていますか? 
\\	学習者の弱点を把握しながら、学習コンテンツを提示する形です。	
\\	学習者の弱点を把握しながら、学習コンテンツを提示する形です。 
\\	要するに、フィードバック機能がついた学習システムということ?	
\\	要するに、フィードバック機能がついた学習システムということ? 
\\	はい、そうです。言い換えれば、テストと連携した学習システムです。	
\\	はい、そうです。言い換えれば、テストと連携した学習システムです。 
\\	卒業研究について教えてください。	
\\	はい。学習者が単語を効率的に記憶できるような学習システムを作成しました。	
\\	はい。学習者が単語を効率的に記憶できるような学習システムを作成しました。 
\\	ということは、基本的なプログラミングの知識はあるんですね。	
\\	ということは、基本的なプログラミングの知識はあるんですね。 
\\	はい。	
\\	はい。 
\\	受験番号
\\	雰囲気
\\	最中
\\	浪人
\\	一覧
\\	何となく
\\	競争率
\\	睨めっこ
\\	教授
\\	七転び八起き
\\	よう!パソコンとにらめっこか?	
\\	よう!パソコンとにらめっこか? 
\\	あ、先輩。今天下大学の合格者リストが11時から大学のホームページで発表される予定なんです。	
\\	あ、先輩。今天下大学の合格者リストが11時から大学のホームページで発表される予定なんです。 
\\	あと1分で11時…。首を長くして待っているところか。帝国大も受験したんだろ?どうだったんだ?	
\\	あと1分で11時…。首を長くして待っているところか。帝国大も受験したんだろ?どうだったんだ? 
\\	不合格でした。他にも2つ大学院を受験したんですが、それもダメでした。	
\\	不合格でした。他にも2つ大学院を受験したんですが、それもダメでした。 
\\	今年はどこの大学院も競争率が高かったらしいからな。	
\\	今年はどこの大学院も競争率が高かったらしいからな。 
\\	面接の雰囲気で何となく合格できないのは分かっていたんですよね。教授達は僕の話の最中に何度も首をひねってたんです。	
\\	面接の雰囲気で何となく合格できないのは分かっていたんですよね。教授達は僕の話の最中に何度も首をひねってたんです。 
\\	そうか…。	
\\	そうか…。 
\\	これで、天下大も落ちたらどうしよう。浪人することは親が首を縦に振らないだろうし…。	
\\	これで、天下大も落ちたらどうしよう。浪人することは親が首を縦に振らないだろうし…。 
\\	七転び八起きっていうじゃないか、くよくよすんな。あ、もう11時だぞ。あ、出てる出てる。合格者一覧。	
\\	"七転び八起きっていうじゃないか、くよくよすんな。あ、もう11時だぞ。あ、出てる出てる。合格者一覧。 
\\	え?俺の受験番号は…あった!ありました、先輩。	
\\	え?俺の受験番号は…あった!ありました、先輩。 
\\	おめでとう。	
\\	おめでとう。 
\\	首の皮一枚でつながりました。	
\\	首の皮一枚でつながりました。 
\\	俺も首を突っ込んだかいがあったよ。	
\\	俺も首を突っ込んだかいがあったよ。 
\\	年収
\\	立候補
\\	大臣
\\	批判
\\	報道
\\	投票
\\	選挙
\\	知事
\\	政策
\\	私生活
\\	10月8日は知事選挙の投票日です。	
\\	10月8日は知事選挙の投票日です。 
\\	投票時間は朝7時から夜8時までです。必ず投票に行きましょう。	
\\	投票時間は朝7時から夜8時までです。必ず投票に行きましょう。 
\\	10月8日は知事選挙の投票日です。投票時間は…。	
\\	10月8日は知事選挙の投票日です。投票時間は…。 
\\	来週の日曜日は選挙か。	
\\	来週の日曜日は選挙か。 
\\	立候補してるのは4人だったっけ。	
\\	立候補してるのは4人だったっけ。 
\\	自民党の阿倍山さんと民主党の菅谷さん、共産党の地位さん。	
\\	自民党の阿倍山さんと民主党の菅谷さん、共産党の地位さん。 
\\	政策面では阿倍山さんがいいと思うけど…親の七光りで議員になったお坊ちゃんだからな。	
\\	政策面では阿倍山さんがいいと思うけど…親の七光りで議員になったお坊ちゃんだからな。 
\\	前に大臣をしていたときに「もう疲れた」って言って辞めたんでしょ?	
\\	"前に大臣をしていたときに「もう疲れた」って言って辞めたんでしょ? 
\\	子供じゃあるまいし…。国民をバカにしているよね。	
\\	子供じゃあるまいし…。国民をバカにしているよね。 
\\	でも、何でもかんでもマスコミに批判されれば嫌になろうってもんだよ。	
\\	でも、何でもかんでもマスコミに批判されれば嫌になろうってもんだよ。 
\\	確かに、どうでもいい私生活とか奥さんのこととか面白おかしく報道するのは問題だと思うけど…。	
\\	確かに、どうでもいい私生活とか奥さんのこととか面白おかしく報道するのは問題だと思うけど…。 
\\	でも阿倍山さん「サラリーマンの平均年収がどれくらいか知ってるんですか?」という記者の質問に、「さあ、1千万位ですか?」って答えたんだって。	
\\	"でも阿倍山さん「サラリーマンの平均年収がどれくらいか知ってるんですか?」という記者の質問に、「さあ、1千万位ですか?」って答えたんだって。 
\\	一千万? はぁ…。金持ちのぼんぼんには俺らの気持ちは分かるまい。	
\\	一千万? はぁ…。金持ちのぼんぼんには俺らの気持ちは分かるまい。 
\\	避難訓練
\\	隠れる
\\	看板
\\	窓ガラス
\\	騒ぐ
\\	覆う
\\	揺れる
\\	対処
\\	確保
\\	慌てる
\\	「地震・雷・火事・親父」って聞いたことありますか。	
\\	"「地震・雷・火事・親父」って聞いたことありますか。 
\\	これは、昔の人が怖いと思っていた4つのものですが、この中の地震と火事の対処の仕方を覚えておきましょう。	
\\	これは、昔の人が怖いと思っていた4つのものですが、この中の地震と火事の対処の仕方を覚えておきましょう。 
\\	じゃ、まずグラッと揺れたらどうしますか。	
\\	じゃ、まずグラッと揺れたらどうしますか。 
\\	机の下に隠れます。	
\\	机の下に隠れます。 
\\	そうですね。机がなかったら頭を洋服やカバンで覆います。	
\\	そうですね。机がなかったら頭を洋服やカバンで覆います。 
\\	他には?	
\\	他には? 
\\	窓が割れるかもしれないので、窓から離れます。	
\\	窓が割れるかもしれないので、窓から離れます。 
\\	ドアを開けます。	
\\	ドアを開けます。 
\\	そのとおり。逃げ道を確保するためですよね。	
\\	そのとおり。逃げ道を確保するためですよね。 
\\	外に出ます。	
\\	外に出ます。 
\\	うーん。地震が起きている時にすぐに外に飛び出すのはかえって危険ですよ。	
\\	うーん。地震が起きている時にすぐに外に飛び出すのはかえって危険ですよ。 
\\	でも、建物が壊れるかもしれないし…。	
\\	でも、建物が壊れるかもしれないし…。 
\\	古い建物はそういうこともあります。	
\\	古い建物はそういうこともあります。 
\\	でも、日本の新しい建物や学校なら中にいた方が外に出るよりむしろ安全なんです。	
\\	でも、日本の新しい建物や学校なら中にいた方が外に出るよりむしろ安全なんです。 
\\	大きな街では地震で壊れた看板や窓ガラスなどが、落ちてきます。	
\\	大きな街では地震で壊れた看板や窓ガラスなどが、落ちてきます。 
\\	だから、まだ揺れている時に外に出るとかえって危ないんです。	
\\	だから、まだ揺れている時に外に出るとかえって危ないんです。 
\\	揺れが止まってから、慌てず、騒がずに避難してください。	
\\	揺れが止まってから、慌てず、騒がずに避難してください。 
\\	それでは、避難訓練を始めましょう。	
\\	それでは、避難訓練を始めましょう。 
\\	せい
\\	責める
\\	従兄弟
\\	嫁
\\	怖い
\\	辞める
\\	隅
\\	退職
\\	転職
\\	取引先
\\	退職?	
\\	退職? 
\\	はい。いろいろ考えて、それが一番いいかと…。	
\\	はい。いろいろ考えて、それが一番いいかと…。 
\\	君のような優秀な人が辞めてしまうのは残念でならないよ。	
\\	君のような優秀な人が辞めてしまうのは残念でならないよ。 
\\	何かあった?	
\\	何かあった? 
\\	いえ。何もありません。	
\\	いえ。何もありません。 
\\	参ったよ。ウチの課の子が退職したいって言い出してさ。	
\\	参ったよ。ウチの課の子が退職したいって言い出してさ。 
\\	大野綾乃だろ?	
\\	大野綾乃だろ? 
\\	なんだよ。知りあいか?お前も隅に置けないな。	
\\	なんだよ。知りあいか?お前も隅に置けないな。 
\\	実は俺の従兄弟の嫁さんなんだよ。	
\\	実は俺の従兄弟の嫁さんなんだよ。 
\\	彼女は五味のアシスタントだろ?	
\\	彼女は五味のアシスタントだろ? 
\\	五味は自分のミスを全部彼女のせいにして責めるから腹が立って仕方なかったらしい。	
\\	五味は自分のミスを全部彼女のせいにして責めるから腹が立って仕方なかったらしい。 
\\	五味か…。そう言えば、あいつのアシスタントはすぐ辞めるな。	
\\	五味か…。そう言えば、あいつのアシスタントはすぐ辞めるな。 
\\	五味は角のある言い方ばかりするから、彼女は会社に来るのが嫌でならなかったそうだよ。	
\\	五味は角のある言い方ばかりするから、彼女は会社に来るのが嫌でならなかったそうだよ。 
\\	なんで教えてくれなかったんだよ。	
\\	"なんで教えてくれなかったんだよ。 
\\	「誰にも言わないで」って言われてたんだ。	
\\	「誰にも言わないで」って言われてたんだ。 
\\	ま、でも、彼女の転職先はウチの取引先だ。	
\\	ま、でも、彼女の転職先はウチの取引先だ。 
\\	え?	
\\	え? 
\\	彼女、ウチのお客さんのとこに転職するんだ。	
\\	彼女、ウチのお客さんのとこに転職するんだ。 
\\	都心
\\	解約
\\	組む
\\	酔っぱらい
\\	郊外
\\	住宅ローン
\\	手狭
\\	更新
\\	購入
\\	問い合わせ
\\	この部屋の契約の更新もうすぐでしょ。どうする?	
\\	この部屋の契約の更新もうすぐでしょ。どうする? 
\\	更新していいんじゃないか。	
\\	更新していいんじゃないか。 
\\	引越してきたときは田舎だったけど、周りに住宅ができるに従ってお店も増えて、便利になってきたし。	
\\	引越してきたときは田舎だったけど、周りに住宅ができるに従ってお店も増えて、便利になってきたし。 
\\	でも、子供が大きくなるにつれて部屋が手狭になってきたし、それに、最近、駅前のガラが悪くなったと思わない?	
\\	でも、子供が大きくなるにつれて部屋が手狭になってきたし、それに、最近、駅前のガラが悪くなったと思わない? 
\\	まあ、確かに飲み屋が増えてくるにつれて、酔っぱらいをよく見かけるようになってきたな。	
\\	まあ、確かに飲み屋が増えてくるにつれて、酔っぱらいをよく見かけるようになってきたな。 
\\	でしょ?家賃もったいないし、そろそろ郊外に家を建てようよ。	
\\	でしょ?家賃もったいないし、そろそろ郊外に家を建てようよ。 
\\	住宅ローンを組むなら早ければ早いほどいいでしょ。	
\\	住宅ローンを組むなら早ければ早いほどいいでしょ。 
\\	そうだな。でも、どこに?	
\\	そうだな。でも、どこに? 
\\	都心から離れれば離れるほど土地は安くなるけど、その分通勤が...。	
\\	都心から離れれば離れるほど土地は安くなるけど、その分通勤が...。 
\\	あ、偶然なんだけど、うちの親が実家の隣の土地を購入したらしいの。	
\\	あ、偶然なんだけど、うちの親が実家の隣の土地を購入したらしいの。 
\\	お電話ありがとうございます。
\\	不動産です。	
\\	お電話ありがとうございます。
\\	不動産です。 
\\	音声に従って数字をご入力ください。	
\\	音声に従って数字をご入力ください。 
\\	契約の更新の方は1、契約の解約の方は2を、その他のお問い合わせの方は3を入力してください。	
\\	契約の更新の方は1、契約の解約の方は2を、その他のお問い合わせの方は3を入力してください。 
\\	2。(ピッ)はあ...。俺もマスオさんになるのか。	
\\	2。(ピッ)はあ...。俺もマスオさんになるのか。 
\\	放射能
\\	電力
\\	漏れ
\\	安全性
\\	ダム
\\	高める
\\	発電
\\	原子力
\\	二酸化炭素
\\	理想的
\\	「原子力発電の安全性を高めると共に、クリーンエネルギーにも取り組んでいく必要がある」か…。	
\\	"「原子力発電の安全性を高めると共に、クリーンエネルギーにも取り組んでいく必要がある」か…。 
\\	ねぇ。新聞読みながらご飯食べるのやめて。	
\\	ねぇ。新聞読みながらご飯食べるのやめて。 
\\	はいはい。	
\\	はいはい。 
\\	ねぇ、パパ。クリーンエネルギーって何?	
\\	"ねぇ、パパ。クリーンエネルギーって何? 
\\	ん?クリーンエネルギーっていうのは、簡単に言えば環境に優しいエネルギーってことだ。	
\\	"ん?クリーンエネルギーっていうのは、簡単に言えば環境に優しいエネルギーってことだ。 
\\	太陽光発電とか風力発電とかだな。	
\\	太陽光発電とか風力発電とかだな。 
\\	ダムも?	
\\	ダムも? 
\\	ダム?ああ、水力発電か。	
\\	ダム?ああ、水力発電か。 
\\	そうだな、水力もクリーンエネルギーだな。	
\\	"そうだな、水力もクリーンエネルギーだな。 
\\	放射能漏れの事故があったのに原発をやめることはできないの?	
\\	放射能漏れの事故があったのに原発をやめることはできないの? 
\\	原子力は火力や水力と並んで、日本ではメジャーな発電方法だから、今すぐやめるのは無理だろうなぁ。	
\\	原子力は火力や水力と並んで、日本ではメジャーな発電方法だから、今すぐやめるのは無理だろうなぁ。 
\\	割合はどのくらい?	
\\	割合はどのくらい? 
\\	確か、火力発電が約6割から7割。原子力が2割。	
\\	確か、火力発電が約6割から7割。原子力が2割。 
\\	水力は1割弱じゃなかったかな。	
\\	水力は1割弱じゃなかったかな。 
\\	え?じゃ、太陽光とか風力とかは?	
\\	え?じゃ、太陽光とか風力とかは? 
\\	全部あわせても、せいぜい1、2パーセントだろう。	
\\	全部あわせても、せいぜい1、2パーセントだろう。 
\\	えー。それだけ?	
\\	えー。それだけ? 
\\	事故さえ起きなければ、原子力は二酸化炭素も出ないし、得られる電力も大きいから、理想的なんだ。事故さえ起きなければね。	
\\	事故さえ起きなければ、原子力は二酸化炭素も出ないし、得られる電力も大きいから、理想的なんだ。事故さえ起きなければね。 
\\	宇宙
\\	組み立てる
\\	宇宙飛行士
\\	接近
\\	クレーター
\\	地球
\\	彗星
\\	天体望遠鏡
\\	人工衛星
\\	不良品
\\	60年に一度、地球に接近する
\\	彗星(すいせい)ですが、現在最も地球に接近しており、今夜は望遠鏡を使わなくてもはっきり見えると思われます。	
\\	60年に一度、地球に接近する
\\	彗星(すいせい)ですが、現在最も地球に接近しており、今夜は望遠鏡を使わなくてもはっきり見えると思われます。 
\\	ジャーン!買っちゃった。	
\\	ジャーン!買っちゃった。 
\\	え?	
\\	え? 
\\	値段もろくに見ずに買ったら、高くてさ、何と十万以上したんだよ、この天体望遠鏡。	
\\	値段もろくに見ずに買ったら、高くてさ、何と十万以上したんだよ、この天体望遠鏡。 
\\	宇宙にろくに興味もないくせに、どうしてそんなに高いのを買ったの?	
\\	宇宙にろくに興味もないくせに、どうしてそんなに高いのを買ったの? 
\\	インテリアにもなると思ってさ。このフォルムいいだろ。	
\\	インテリアにもなると思ってさ。このフォルムいいだろ。 
\\	あ、それから、これ、おまけでもらった。	
\\	あ、それから、これ、おまけでもらった。 
\\	何?	
\\	何? 
\\	宇宙飛行士と人工衛星のフィギュアと宇宙食。	
\\	宇宙飛行士と人工衛星のフィギュアと宇宙食。 
\\	うらやましいだろ。	
\\	うらやましいだろ。 
\\	もう!ろくなもの買わないんだから。	
\\	もう!ろくなもの買わないんだから。 
\\	第一、結婚式のために貯金するって昨日約束したばかりじゃん。	
\\	第一、結婚式のために貯金するって昨日約束したばかりじゃん。 
\\	カリカリしても、ろくなことがないぞ。	
\\	カリカリしても、ろくなことがないぞ。 
\\	さ、組み立てるぞ。
\\	さ、組み立てるぞ。
\\	よしできた。見てみよう。あれ?真っ暗で何も見えない…。	
\\	よしできた。見てみよう。あれ?真っ暗で何も見えない…。 
\\	説明書もろくに読まずに組み立てるからだよ。	
\\	説明書もろくに読まずに組み立てるからだよ。 
\\	おっかしいな。	
\\	おっかしいな。 
\\	この望遠鏡なら、月のクレーターや山が見えるって言ってたのに…。	
\\	この望遠鏡なら、月のクレーターや山が見えるって言ってたのに…。 
\\	不良品を買わされたんじゃない。いいカモにされたんだよ。	
\\	不良品を買わされたんじゃない。いいカモにされたんだよ。 
\\	宝くじ
\\	壁
\\	売り場
\\	火星
\\	詐欺
\\	障子
\\	権利書
\\	投資
\\	低金利
\\	並ぶ
\\	「不動産を買うなら、低金利の今が買い時」かぁ…。	
\\	"「不動産を買うなら、低金利の今が買い時」かぁ…。 
\\	でも、先立つものがなきゃ買えないよ。	
\\	でも、先立つものがなきゃ買えないよ。 
\\	ね、春樹。春樹?あれ?急に耳が遠くなっちゃった?	
\\	ね、春樹。春樹?あれ?急に耳が遠くなっちゃった? 
\\	耳が痛いなぁ。	
\\	耳が痛いなぁ。 
\\	天体望遠鏡なんて買ってないできちんと貯金してよ。	
\\	天体望遠鏡なんて買ってないできちんと貯金してよ。 
\\	はいはい。それは、耳にたこができるほど聞きました。	
\\	はいはい。それは、耳にたこができるほど聞きました。 
\\	…俺だって、色々考えているんだ。土地に投資したり…。	
\\	…俺だって、色々考えているんだ。土地に投資したり…。 
\\	うそ?土地買ったの?それは初耳。	
\\	うそ?土地買ったの?それは初耳。 
\\	ジャーン!これ火星の土地の権利書。	
\\	ジャーン!これ火星の土地の権利書。 
\\	は?こんなの詐欺じゃん。火星になんて住めっこないでしょ。	
\\	は?こんなの詐欺じゃん。火星になんて住めっこないでしょ。 
\\	そうだけど、夢があるじゃん。	
\\	そうだけど、夢があるじゃん。 
\\	夢ねぇ…。	
\\	夢ねぇ…。 
\\	自分だって、夢を買ってるだろ。宝くじ買ったって小耳に挟んだぜ。	
\\	自分だって、夢を買ってるだろ。宝くじ買ったって小耳に挟んだぜ。 
\\	どうして知ってんのよ。	
\\	どうして知ってんのよ。 
\\	志保が宝くじ売り場で並んでいるのを元気が見たって言ってたんだよ。	
\\	志保が宝くじ売り場で並んでいるのを元気が見たって言ってたんだよ。 
\\	当たったかどうか、調べてみようぜ。	
\\	当たったかどうか、調べてみようぜ。 
\\	当たるわけがないじゃない。	
\\	当たるわけがないじゃない。 
\\	…やだ…当たってる?…3億当たった!	
\\	…やだ…当たってる?…3億当たった! 
\\	シー。壁に耳あり障子に目あり。	
\\	シー。壁に耳あり障子に目あり。 
\\	誰がどこで聞き耳を立てているかわからないんだぞ。	
\\	誰がどこで聞き耳を立てているかわからないんだぞ。 
\\	このことは誰にも言うのはやめよう。	
\\	このことは誰にも言うのはやめよう。 
\\	公開
\\	感度
\\	様々
\\	数えきれないほど
\\	惹かれる
\\	標準語
\\	発音
\\	表現
\\	仕方
\\	まるで~ない
\\	不思議
\\	伝統的
\\	港町
\\	芸人
\\	ネタ
\\	ゴスロリ
\\	特殊
\\	仏閣
\\	増す
\\	神聖
\\	微妙
\\	横町
\\	肉体労働者
\\	定年
\\	フラリと
\\	根強い
\\	片隅
\\	接客する
\\	気さく
\\	魅了する
\\	下げる
\\	丸見え
\\	飲屋街
\\	終戦
\\	店舗
\\	密集する
\\	間もない
\\	火事
\\	焼失する
\\	開け放つ
\\	盛り合わせ
\\	日本酒
\\	朝方
\\	小鉢
\\	揚げ物
\\	煮物
\\	冷や奴
\\	枝豆
\\	乾き物
\\	塩辛
\\	飲み放題
\\	幅広い
\\	営業時間
\\	チェーン展開する
\\	揃う
\\	銘柄
\\	定番
\\	唐揚げ
\\	漬け物
\\	食べ放題
\\	店内装飾
\\	提供する
\\	豊富
\\	至るところ
\\	気軽に
\\	路地
\\	魅力
\\	感覚
\\	空間
\\	気さく
\\	国籍
\\	溢れ出す
\\	隠れ家
\\	開店する
\\	看板
\\	常連
\\	豪華
\\	ロココ調
\\	吊るす
\\	真っ赤
\\	その名の通り
\\	思う存分
\\	外国人バー	
\\	外国人バー 
\\	店内の壁は、その名の通り真っ赤で、天井には何十個ものシャンデリアが吊るされています。	
\\	店内の壁は、その名の通り真っ赤で、天井には何十個ものシャンデリアが吊るされています。 
\\	ブドウの形をしたもの、ロココ調のクリスタルの豪華なものなど、見ているだけでも楽しいものばかりです。	
\\	ブドウの形をしたもの、ロココ調のクリスタルの豪華なものなど、見ているだけでも楽しいものばかりです。 
\\	その他、鹿の顔のオブジェなどユニークなオブジェも展示されています。	
\\	その他、鹿の顔のオブジェなどユニークなオブジェも展示されています。 
\\	常連の間では、
\\	の看板ともいえる日本人バーテンダーもっさんに会うために訪れる人も少なくありません。	
\\	常連の間では、
\\	の看板ともいえる日本人バーテンダーもっさんに会うために訪れる人も少なくありません。 
\\	屋台
\\	開放的
\\	心身共に
\\	予想外
\\	屋内
\\	調理台
\\	台車
\\	木造
\\	立ち寄る
\\	高架下
\\	郊外
\\	沿線
\\	集う
\\	打ち解ける
\\	同性
\\	戸惑う
\\	合理的
\\	疑似恋愛
\\	指名する
\\	追加料金
\\	行きつけ
\\	灯す
\\	和製英語
\\	合成する
\\	看板
\\	地域
\\	象徴する
\\	政治
\\	行き交う
\\	志向
\\	うわべ
\\	斬新
\\	強烈
\\	支持
\\	圧倒的
\\	満載
\\	ズラリと
\\	直営店
\\	焦点
\\	派手
\\	装い
\\	ふんだんに
\\	担う
\\	役目
\\	煽る
\\	縁取る
\\	格好
\\	装い
\\	包む
\\	発信
\\	変化
\\	ごく
\\	整える
\\	薄化粧
\\	強調する
\\	主に
\\	急速
\\	特徴
\\	影響
\\	格好
\\	溶け込む
\\	~に目がない
\\	老舗
\\	築く
\\	~といっても過言ではない
\\	不景気
\\	久しい
\\	自己表現
\\	覗く
\\	眺める
\\	斬新
\\	発想
\\	敏感
\\	誇る
\\	設立する
\\	度肝を抜く
\\	上質
\\	肌触り
\\	施す
\\	斬新
\\	芸能分野
\\	追求する
\\	評価
\\	主旨
\\	容姿
\\	ひととき
\\	巻き込む
\\	発達する
\\	独自に
\\	滑稽
\\	傾向
\\	多様化
\\	独特
\\	世論
\\	お茶の間
\\	看板番組
\\	活動する
\\	受賞する
\\	知名度
\\	映画監督
\\	活躍
\\	前代未聞
\\	多才
\\	登場
\\	絶妙
\\	ひょうひょうとした
\\	無表情
\\	隙
\\	前提
\\	非難する
\\	役割分担
\\	稀
\\	支持する
\\	熱狂的
\\	司会者
\\	君臨する
\\	皮肉
\\	共感する
\\	発揮する
\\	屈辱
\\	採用試験
\\	睨みつける
\\	一躍
\\	連発する
\\	保母
\\	偽る
\\	標準語
\\	相当する
\\	平たん
\\	なだらか
\\	タレント
\\	来日する
\\	避ける
\\	耳にする
\\	方言
\\	頻繁に
\\	冗談めいた
\\	印象
\\	共通語
\\	歯切れがいい
\\	抑揚
\\	抱く
\\	例を挙げる
\\	敬遠する
\\	活躍
\\	含む
\\	炊く
\\	物事
\\	判断する
\\	勘定
\\	特有
\\	発言する
\\	中継
\\	国会
\\	著しい
\\	めでたい
\\	政治
\\	蒸す
\\	語尾
\\	省略する
\\	気質
\\	せっかち
\\	意識する
\\	辛抱
\\	無意識に
\\	経つ
\\	四六時中
\\	新鮮
\\	実家
\\	しみじみ
\\	耳にする
\\	省略する
\\	本殿
\\	そびえ立つ
\\	澄む
\\	神聖な
\\	拝む
\\	色とりどり
\\	壮大な
\\	祀る
\\	彫刻
\\	象徴する
\\	あいにく
\\	世界遺産
\\	存在感
\\	朱色
\\	汲む
\\	溢れ出す
\\	湧き水
\\	涼む
\\	吹き出る
\\	絶景
\\	てっぺん
\\	あちこち
\\	敷地
\\	見渡す
\\	清め
\\	天皇
\\	フラッと
\\	都会
\\	散策する
\\	静寂な
\\	賑わう
\\	最寄り
\\	参拝客
\\	恒例
\\	皇太后
\\	足を運ぶ
\\	八百万
\\	重要文化財
\\	妊娠
\\	心を込めて
\\	備え付け
\\	会釈する
\\	国宝
\\	空間
\\	神聖な
\\	縁結び
\\	くぐる
\\	振り返る
\\	保護者
\\	守護神
\\	~足らず
\\	ゆったりと
\\	何度となく
\\	急角度
\\	一連
\\	出展
\\	飾る
\\	上映会
\\	場所柄
\\	展示
\\	吊るす
\\	観客
\\	秘密結社
\\	働きかける
\\	根元
\\	好奇心
\\	映像作家
\\	創作意欲
\\	現代
\\	製作する
\\	純白
\\	シャレ
\\	滑稽
\\	作品
\\	秀逸
\\	盛況
\\	見事
\\	主催
\\	源
\\	ご存知
\\	発信
\\	大幅
\\	支持
\\	繊細
\\	代表格
\\	一躍
\\	ブーム
\\	水筒
\\	老舗
\\	繋がり
\\	主題歌
\\	かれこれ
\\	取材
\\	気さく
\\	魅力
\\	当たり前
\\	幻
\\	褒め称える
\\	以前
\\	大胆
\\	発祥
\\	渋い
\\	シミ
\\	うっすら
\\	風合い
\\	破天荒
\\	染色
\\	イカ墨
\\	ユニーク
\\	規模
\\	開催
\\	展示
\\	合同
\\	感心する
\\	チャーター
\\	身動き
\\	絡む
\\	歩行者
\\	徘徊する
\\	帯びる
\\	迫力
\\	程度
\\	実施する
\\	通常
\\	世紀
\\	煽る
\\	周知
\\	霊感
\\	動揺する
\\	舞台
\\	ご存知
\\	有力
\\	天体
\\	贅沢
\\	同郷
\\	空間
\\	瞬く間に
\\	珍しがる
\\	近頃
\\	余程
\\	ぼろ
\\	極端に
\\	世
\\	まとめる
\\	銘打つ
\\	建築
\\	非常に
\\	直接
\\	銭湯
\\	職人
\\	手際
\\	さすがに
\\	巧み
\\	刷毛
\\	仕上げる
\\	所要時間
\\	見事
\\	溢れる
\\	連なる
\\	縦
\\	くくり付ける
\\	器具
\\	代物
\\	接合
\\	展示会
\\	ギャラリー
\\	運営
\\	珍道具
\\	逆理
\\	肖像画
\\	慕う
\\	執筆
\\	境
\\	危篤状態
\\	胃潰瘍
\\	脱却する
\\	我輩
\\	言動
\\	定期的に
\\	巨頭
\\	励む
\\	流暢
\\	尽力
\\	社会復帰
\\	宿す
\\	苛まれる
\\	綴る
\\	雨宿り
\\	功績
\\	服毒自殺
\\	推敲する
\\	短編
\\	掲載する
\\	剥ぎ取る
\\	松明
\\	梯子
\\	特筆すべき
\\	生涯
\\	遺体
\\	心中
\\	断乎として
\\	余計者
\\	評する
\\	没後
\\	生誕
\\	秀才
\\	一線を画す
\\	傍ら
\\	着想を得る
\\	筋書き
\\	可憐
\\	一座
\\	孤児
\\	境遇
\\	遂には
\\	遂げる
\\	再建
\\	観念的
\\	炎上する
\\	湧き上がる
\\	願望
\\	間近に
\\	同人誌
\\	戦後
\\	僧侶
\\	抽象的
\\	肺炎
\\	批判
\\	分類
\\	奔走する
\\	従事
\\	肥料
\\	過酷な
\\	冒頭部
\\	文明
\\	身勝手な
\\	厳密に言えば
\\	近現代
\\	看過する
\\	仕草
\\	面影
\\	成就する
\\	思慕を募らせる
\\	容姿端麗
\\	好意を寄せる
\\	根強い
\\	受賞する
\\	脚本家
\\	演出家
\\	掻き立てる
\\	生き生きとしている
\\	一層
\\	一躍
\\	解釈
\\	支持
\\	比喩
\\	対極
\\	上梓する
\\	候補
\\	耳目を集める
\\	暗示する
\\	随所
\\	幻想的
\\	教壇
\\	舞台
\\	連想する
\\	柳
\\	水滴
\\	石碑
\\	堤防
\\	霞
\\	肩書き
\\	蚊取り線香
\\	湿原地帯
\\	心象風景
\\	芳しい
\\	群生する
\\	風物詩
\\	奥ゆかしい
\\	清楚
\\	癒される
\\	雄大な
\\	そこはかとない
\\	しみじみとした
\\	郷愁
\\	随筆
\\	作詞者
\\	古今東西
\\	慕わしい
\\	竿
\\	沁みる
\\	木枯らし
\\	身を立てる
\\	こつこつ
\\	絶つ
\\	見限る
\\	所在
\\	短調
\\	琴線に触れる
\\	文語
\\	賛美歌
\\	澄み渡る
\\	じんと
\\	小鮒
\\	掲載する
\\	着想を得る
\\	意訳
\\	故事
\\	馴染み深い
\\	立身出世
\\	由来
\\	励む
\\	公共施設
\\	年の瀬
\\	商業施設
\\	民謡
\\	溢れる
\\	旋律
\\	童謡
\\	まなざし
\\	物議をかもす
\\	擬音語
\\	母国語
\\	素朴な
\\	数詞
\\	いとおしむ
\\	解釈する
\\	口伝え
\\	婚礼
\\	属する
\\	効率
\\	身振り手振り
\\	掛け声
\\	漕ぎ出す
\\	単調さ
\\	極寒
\\	教訓
\\	出典
\\	主題
\\	尊ぶ
\\	起承転結
\\	色あせる
\\	旋律
\\	浸透する
\\	必然的に
\\	合唱曲
\\	哀愁
\\	幾多
\\	感傷
\\	綴る
\\	比較的
\\	豊作
\\	飾りつけ
\\	縄
\\	お供え
\\	節句
\\	畏敬
\\	軒下
\\	乾燥
\\	冒頭
\\	魔
\\	滅する
\\	炒る
\\	鰯
\\	柊
\\	邪気
\\	仏閣
\\	風習
\\	橘
\\	婚期
\\	装束
\\	戒める
\\	無病息災
\\	加熱する
\\	迷信
\\	造花
\\	鎧
\\	吹流し
\\	勇猛果敢
\\	災厄
\\	餡
\\	兜
\\	葛粉
\\	筒状
\\	立身出世
\\	カササギ
\\	水かさ
\\	短冊
\\	途端
\\	天帝
\\	隔てる
\\	架ける
\\	つるす
\\	風習
\\	霊
\\	焚く
\\	読経する
\\	僧侶
\\	霊魂
\\	法話
\\	帰省
\\	たどる
\\	短歌
\\	稲
\\	鑑賞
\\	命名
\\	次第に
\\	さかのぼる
\\	存在
\\	水面
\\	愛でる
\\	後継ぎ
\\	疫病
\\	奇数
\\	起源
\\	年中行事
\\	切実
\\	祈念
\\	習わし
\\	思想
\\	旧暦
\\	金粉
\\	金細工職人
\\	欲望
\\	所有
\\	憎む
\\	精神
\\	諸説
\\	仏教
\\	鐘
\\	深夜
\\	先駆け
\\	保存食
\\	食卓
\\	言わずもがな
\\	彩る
\\	酢飯
\\	発酵
\\	食する
\\	甘じょっぱい
\\	うま味
\\	成分
\\	レンコン
\\	明治時代
\\	漬ける
\\	扇形
\\	固形
\\	即席
\\	湯煎
\\	レトルト
\\	至って
\\	梅干
\\	神格化
\\	断言
\\	授かる
\\	差し支え
\\	シャケ
\\	携行
\\	具材
\\	ご加護
\\	こねる
\\	食感
\\	物質
\\	似て非なる
\\	和食
\\	両者
\\	ぬめり
\\	昆布
\\	芯
\\	消化
\\	ゆず
\\	絞り汁
\\	キス
\\	次いで
\\	庶民
\\	四季折々
\\	溶く
\\	液体
\\	串
\\	アジ
\\	風味
\\	変り種
\\	過言
\\	海草
\\	万能
\\	よだれ
\\	光景
\\	塩分
\\	製品
\\	軒
\\	縮れる
\\	消費
\\	一括り
\\	熱湯
\\	異論
\\	進化
\\	唱える
\\	形状
\\	出回る
\\	容器
\\	たんぱく質
\\	なおさら
\\	摂取
\\	粘り気
\\	和がらし
\\	食物繊維
\\	発泡スチロール
\\	語呂合わせ
\\	問う
\\	調える
\\	農民
\\	老若男女
\\	普及
\\	鉄板
\\	筆頭
\\	好む
\\	煮詰まる
\\	農作業
\\	渋み
\\	練る
\\	でんぷん
\\	種々
\\	粒
\\	穀物
\\	滑らか
\\	伝来
\\	潰す
\\	逸品
\\	礎
\\	先進国
\\	画期的
\\	世襲制
\\	血縁集団
\\	登用する
\\	引き合いに出す
\\	的確に
\\	過言
\\	逸話
\\	功績
\\	裏切り
\\	非凡
\\	同然
\\	悲劇
\\	味方
\\	同情する
\\	献身的
\\	英雄
\\	うなぎのぼり
\\	築く
\\	基本
\\	憲法
\\	思想
\\	補う
\\	手腕
\\	説く
\\	影響を受ける
\\	無用
\\	避ける
\\	労る
\\	合理的
\\	冒頭
\\	反抗する
\\	突飛
\\	冷酷
\\	成功
\\	原動力
\\	出身
\\	型破り
\\	成果
\\	雑用
\\	身分
\\	創意工夫
\\	出世
\\	懐
\\	外出する
\\	不可能
\\	かたき
\\	統一する
\\	同盟
\\	共感を覚える
\\	証
\\	差し出す
\\	武家政権
\\	ようやく
\\	創始者
\\	人質
\\	幼少期
\\	信頼
\\	背景
\\	狙い
\\	確保する
\\	必死
\\	聞く耳を持つ
\\	政策
\\	かっこうの
\\	迫る
\\	危惧する
\\	断つ
\\	設立する
\\	役職
\\	力を注ぐ
\\	反発する
\\	質素
\\	惜しい
\\	決意する
\\	人物
\\	愛称
\\	追い詰める
\\	知識
\\	独学
\\	肖像
\\	制度
\\	豊かな
\\	翻訳
\\	勧める
\\	入手する
\\	概念
\\	早口ことば
\\	一斉に
\\	体感する
\\	世界遺産
\\	湖
\\	雄大な
\\	足を運ぶ
\\	太平洋
\\	海岸線
\\	観光名所
\\	祭典
\\	厳密に言えば
\\	~に渡って
\\	掛け声
\\	雨天
\\	発祥
\\	開催する
\\	一説によると
\\	風習
\\	動機
\\	序文
\\	相対性理論
\\	代表作
\\	技
\\	描く
\\	コントラスト
\\	故人
\\	絵画
\\	湾
\\	粋
\\	取り上げる
\\	建築
\\	構造
\\	大国
\\	名所
\\	甚大
\\	最先端
\\	被害
\\	懸念する
\\	設置する
\\	またがる
\\	緩やか
\\	跡を絶たない
\\	標高
\\	与える
\\	古今東西
\\	心を動かす
\\	魅了する
\\	中継する
\\	堪能する
\\	慣用句
\\	神秘的
\\	平均
\\	領域
\\	眺める
\\	境
\\	並ぶ
\\	区別
\\	避暑
\\	修理
\\	称する
\\	ユネスコ
\\	途方もない
\\	国宝
\\	結晶
\\	現存
\\	価値観
\\	所蔵
\\	話題になる
\\	知恵
\\	シンボル
\\	吸収する
\\	独自に
\\	呼び名
\\	登録
\\	振動
\\	評価
\\	樹齢
\\	考え出す
\\	洞窟
\\	切り替える
\\	整備
\\	おすすめ
\\	遡る
\\	好評
\\	達する
\\	経つ
\\	珊瑚
\\	賑わう
\\	貫禄
\\	加える
\\	地形
\\	位置
\\	面積
\\	生態系
\\	片道
\\	湿る
\\	目当て
\\	観測
\\	弱冠
\\	先述
\\	この世を去る
\\	問う
\\	持ち主
\\	記録
\\	思い浮かぶ
\\	主演
\\	もてはやす
\\	わずか
\\	競技
\\	公演
\\	圧倒的
\\	幅広い
\\	解散
\\	支持
\\	更新
\\	武道
\\	行う
\\	奇抜
\\	反戦
\\	葬儀
\\	独特
\\	癌
\\	緊急
\\	決め台詞
\\	意訳
\\	永眠
\\	治療
\\	作詞
\\	権威
\\	早口
\\	テロップ
\\	与える
\\	業界
\\	当たり前
\\	作曲
\\	唯一無二
\\	用いる
\\	駆使
\\	前者
\\	草分け
\\	大掛かり
\\	後者
\\	想いを寄せる
\\	主旨
\\	支持する
\\	具体的
\\	魅了する
\\	歴代
\\	民間
\\	傷つく
\\	失恋
\\	唐突
\\	朗々
\\	ユーモア
\\	心理
\\	反面
\\	持ち主
\\	カバー
\\	初頭
\\	カリスマ
\\	挙句
\\	真っ只中
\\	癒える
\\	不条理
\\	喫煙
\\	大騒動
\\	留年
\\	前向きな
\\	まるで
\\	デビュー
\\	証券
\\	編成
\\	十代
\\	前進
\\	快挙
\\	シンクタンク
\\	株価
\\	充電
\\	爆発
\\	彗星
\\	切ない
\\	発信
\\	近況
\\	芸能界
\\	先駆け
\\	心待ちにする
\\	応募
\\	認める
\\	印象的
\\	周年
\\	代弁
\\	韻
\\	語彙
\\	口ずさむ
\\	緻密
\\	励ます
\\	人柄
\\	彫刻
\\	国宝
\\	睨む
\\	雰囲気
\\	役割
\\	防ぐ
\\	寺院
\\	迫力
\\	逞しい
\\	存在感
\\	輪郭
\\	和紙
\\	墨
\\	修行
\\	贋作
\\	輩出
\\	水墨画
\\	くっきり
\\	画法
\\	独自
\\	質素
\\	裕福
\\	関心
\\	生き方
\\	高価
\\	当時
\\	くつろぐ
\\	姿勢
\\	向き合う
\\	印象派
\\	主要
\\	御所
\\	将軍
\\	表情
\\	忽然と
\\	出版する
\\	意表を突く
\\	題材
\\	代表的な
\\	鮮やかな
\\	構図
\\	大胆な
\\	図柄
\\	画壇
\\	脳裏
\\	詠む
\\	元来
\\	感性
\\	繊細な
\\	鑑賞する
\\	創作する
\\	平易な
\\	冒頭
\\	流行する
\\	語句
\\	発祥
\\	俳諧師
\\	あそこ
\\	どこ
\\	マクドナルド
\\	エスカレーター
\\	はい、マクドナルド です。	
\\	はい、マクドナルド です。 
\\	マクドナルド は どこ に ありますか。	
\\	マクドナルド は どこ に ありますか。 
\\	マクドナルド は 2かい に あります。	
\\	マクドナルド は 2かい に あります。 
\\	2かい。。。わかりました。	
\\	2かい。。。わかりました。 
\\	すみません。エスカレーター は どこ に ありますか?	
\\	すみません。エスカレーター は どこ に ありますか? 
\\	エスカレーター?あそこ に あります。	
\\	エスカレーター?あそこ に あります。 
\\	通う
\\	通る
\\	代わりに
\\	早く
\\	以上
\\	以下
\\	以外
\\	以前
\\	以後
\\	遠慮
\\	もうすぐ
\\	やっと
\\	やっぱり
\\	確かに
\\	ちっとも
\\	便利
\\	住所
\\	見つける
\\	見つかる
\\	育つ
\\	予約
\\	約束
\\	壊れる
\\	故障する
\\	見学
\\	見物
\\	育てる
\\	釣る
\\	ふるさと
\\	がたい
\\	忘れる
\\	夢
\\	川
\\	ふな
\\	山
\\	追う
\\	うさぎ
\\	めぐる
\\	ねんね
\\	ぼうや
\\	おもり
\\	越える
\\	里
\\	土産
\\	でんでん太鼓
\\	笙の笛
\\	あたま
\\	雲
\\	四方
\\	見おろす
\\	かみなり
\\	青空
\\	そびえたつ
\\	かすみ
\\	すそ
\\	むすぶ
\\	ひらく
\\	うつ
\\	て
\\	また
\\	さくら
\\	やよい
\\	見渡す
\\	かすみ
\\	匂い
\\	いざ
\\	屋根
\\	真鯉
\\	泳ぐ
\\	お父さん
\\	子供達
\\	シャボン玉
\\	屋根
\\	壊れる
\\	消える
\\	うまれる
\\	風
\\	松虫
\\	鈴虫
\\	夜長
\\	こおろぎ
\\	鳴く
\\	くつわ虫
\\	秋
\\	鳴き出す
\\	鳴き通す
\\	茶つぼ
\\	抜ける
\\	俵
\\	井戸
\\	お茶わん
\\	欠く
\\	細道
\\	通す
\\	ご用
\\	お札
\\	おさめる
\\	参る
\\	いびきをかく
\\	開ける
\\	今日
\\	昨日
\\	デート
\\	明日
\\	閉まる
\\	ドア
\\	眠る
\\	起きる
\\	隣
\\	困る
\\	うるさい
\\	夜
\\	昼
\\	朝
\\	猫
\\	犬
\\	家
\\	匹
\\	私
\\	僕
\\	桜
\\	雑誌
\\	降る
\\	雨
\\	めくる
\\	ページ
\\	本
\\	咲く
\\	話す
\\	言う
\\	ぺらぺら話す
\\	ぺちゃくちゃ話す
\\	友達
\\	秘密
\\	他の人
\\	中
\\	授業
\\	外国語
\\	なつかしい
\\	サンリオ
\\	咲く
\\	授業中
\\	客
\\	買う
\\	家
\\	近く
\\	話す
\\	うわ~。なつかしい。このカエル、けろけろけろっぴ?	
\\	うわ~。なつかしい。このカエル、けろけろけろっぴ? 
\\	そうそう。サンリオのキャラクター。	
\\	そうそう。サンリオのキャラクター。 
\\	家の近くのワンニャンランドで買ったの。 
\\	へぇ~。お客さんがいた?	
\\	へぇ~。お客さんがいた? 
\\	うーん。パラパラ。	
\\	うーん。パラパラ。 
\\	そのうち
\\	クスクス笑う
\\	笑い出す
\\	ゲラゲラ笑う
\\	にこにこ笑う
\\	顔
\\	幸せ
\\	企む
\\	何で
\\	いい
\\	何か
\\	にやにや笑う
\\	感じ
\\	子供
\\	転ぶ
\\	泣く
\\	毎日
\\	怒る
\\	言う
\\	気持ち
\\	しつこい
\\	声
\\	文句
\\	叱る
\\	言う
\\	大変
\\	謝る
\\	最初
\\	忘れる
\\	誕生日
\\	先月
\\	家族
\\	明るい
\\	一緒に
\\	勉強する
\\	うちの母さん、勉強しろって、ガミガミうるさいんだよ。	
\\	うちの母さん、勉強しろって、ガミガミうるさいんだよ。 
\\	大変だな。	
\\	大変だな。 
\\	うちの母さんは、一緒にテレビを見て、ゲラゲラ笑ってるよ。	
\\	うちの母さんは、一緒にテレビを見て、ゲラゲラ笑ってるよ。 
\\	いいなぁ、だいきの家は。。。	
\\	いいなぁ、だいきの家は。。。 
\\	いつもニコニコ明るい家族だから。	
\\	いつもニコニコ明るい家族だから。 
\\	でも、先月大変だったよ。	
\\	でも、先月大変だったよ。 
\\	コンピューター
\\	何度も
\\	本当に
\\	隣の
\\	壁
\\	蹴る
\\	穴
\\	妻
\\	怒る
\\	社長
\\	言われる
\\	宝箱
\\	見つける
\\	サイコスリラー
\\	連続
\\	どきどきする
\\	わくわくする
\\	はらはらする
\\	目を覚ます
\\	思い出す
\\	汚れる
\\	真っ黒
\\	帰る
\\	どうやって
\\	自分
\\	部屋
\\	僕
\\	本当に
\\	母
\\	しょんぼりする
\\	がっくりくる
\\	犬
\\	庭
\\	一日中
\\	思う
\\	死ぬ
\\	父
\\	ぼんやりする
\\	知る
\\	超
\\	髪
\\	つかむ
\\	渡す
\\	見える
\\	手
\\	真っ白い
\\	止める
\\	続ける
\\	聞こえる
\\	声
\\	紙
\\	学校
\\	怒る
\\	ねぇねぇ、この話、知っている?	
\\	ねぇねぇ、この話、知っている? 
\\	ある人が学校のトイレに入った。	
\\	ある人が学校のトイレに入った。 
\\	そしたら、上から「・・・紙をください・・・」って声が聞こえた。 
\\	やめて。私、怖い話嫌いなの。ぞっとする。	
\\	やめて。私、怖い話嫌いなの。ぞっとする。 
\\	続けて!続けて!わくわくする!	
\\	続けて!続けて!わくわくする! 
\\	その人は、「ちょっと待って」って言って、トイレットペーパーを取って、上を見たんだ。	
\\	その人は、「ちょっと待って」って言って、トイレットペーパーを取って、上を見たんだ。 
\\	そしたら、真っ白い手が見えた・・・。	
\\	そしたら、真っ白い手が見えた・・・。 
\\	ぎょっとしたけど、「はい」って紙を渡したんだ。	
\\	ぎょっとしたけど、「はい」って紙を渡したんだ。 
\\	白い手がその人の頭をつかんで...	
\\	白い手がその人の頭をつかんで... 
\\	夕べ
\\	すやすや
\\	ぐうぐう
\\	うとうと
\\	ぐっすり
\\	赤ちゃん
\\	いびきをかく
\\	祖父
\\	眠る
\\	父
\\	皿
\\	男
\\	置く
\\	食べ始める
\\	食べ終わる
\\	犬
\\	猫
\\	いつまでも
\\	なめる
\\	さあ
\\	出て行く
\\	百円玉
\\	周り
\\	知り合い
\\	こっち
\\	何度も
\\	苦手
\\	初めて
\\	部屋
\\	拾う
\\	息子
\\	すぐに
\\	静かに
\\	危ない
\\	プラットホーム
\\	転ぶ
\\	酔っ払う
\\	最近
\\	歩く
\\	六ヶ月
\\	私の息子は、1才6ヶ月だ。最近、よちよち歩き始めたが、すぐに転ぶ。	
\\	私の息子は、1才6ヶ月だ。最近、よちよち歩き始めたが、すぐに転ぶ。 
\\	酔っ払った人が、プラットホームを、ふらふらと歩いている。とても危ない。	
\\	酔っ払った人が、プラットホームを、ふらふらと歩いている。とても危ない。 
\\	私は東京をふらふら歩くのが好きだ。	
\\	私は東京をふらふら歩くのが好きだ。 
\\	どたばた歩かないでください。静かに歩いてください。	
\\	どたばた歩かないでください。静かに歩いてください。 
\\	お腹
\\	外国語
\\	ラベル
\\	気持ち
\\	大変
\\	のど
\\	こんな
\\	ウイスキーボンボン
\\	目
\\	乾く
\\	あー、お腹ペコペコ。あ、チョコレートがある。	
\\	あー、お腹ペコペコ。あ、チョコレートがある。 
\\	(キョロキョロ)(ムシャムシャムシャ)	
\\	(キョロキョロ)(ムシャムシャムシャ) 
\\	あ、パパもチョコ食べる?	
\\	あ、パパもチョコ食べる? 
\\	いや、いいよ。パパは、のどがカラカラだ。	
\\	いや、いいよ。パパは、のどがカラカラだ。 
\\	おお、ジュースがある。(ごくごくごく)	
\\	おお、ジュースがある。(ごくごくごく) 
\\	ただいま。どうしたの?目がとろんとしてる・・・。	
\\	ただいま。どうしたの?目がとろんとしてる・・・。 
\\	やだ、このチョコレート食べたの?	
\\	やだ、このチョコレート食べたの? 
\\	うん。全部ぺろっと食べちゃった。	
\\	うん。全部ぺろっと食べちゃった。 
\\	そしたら・・・ふらふらするの。(ヒック)	
\\	そしたら・・・ふらふらするの。(ヒック) 
\\	これ、ウイスキーボンボンよ。パパは?	
\\	これ、ウイスキーボンボンよ。パパは? 
\\	トイレによろよろ歩いていったよ。(ヒック)	
\\	トイレによろよろ歩いていったよ。(ヒック) 
\\	あなた、メイが大変なの。あなた?(ドンドンドン)	
\\	あなた、メイが大変なの。あなた?(ドンドンドン) 
\\	開けるわよ(カチャ)やだ、こんなところで、何でぐうぐう寝ているの。	
\\	開けるわよ(カチャ)やだ、こんなところで、何でぐうぐう寝ているの。 
\\	ん?(あくび)さっき、のどが乾いて、ジュースをごくごく飲んだんだ。	
\\	ん?(あくび)さっき、のどが乾いて、ジュースをごくごく飲んだんだ。 
\\	その後、どんどん気持ちが悪くなって・・・。(ヒック)	
\\	その後、どんどん気持ちが悪くなって・・・。(ヒック) 
\\	あのラベル読まなかったの?(ジロッ)	
\\	あのラベル読まなかったの?(ジロッ) 
\\	・・・チラッと見たけど、(ヒック)外国語で・・・わからなかったから・・・。(ヒック)	
\\	・・・チラッと見たけど、(ヒック)外国語で・・・わからなかったから・・・。(ヒック) 
\\	風邪
\\	ストレス
\\	耳
\\	なる
\\	胃
\\	腹
\\	実は
\\	のど
\\	すう
\\	やめる
\\	アレルギー
\\	鼻
\\	部長
\\	(ハクション!ハクション!ハクション!)	
\\	(ハクション!ハクション!ハクション!) 
\\	風邪か?	
\\	風邪か? 
\\	いえ、目がしょぼしょぼして、鼻もずるずる出るので、アレルギーだと思います。(ゴホゴホ)	
\\	いえ、目がしょぼしょぼして、鼻もずるずる出るので、アレルギーだと思います。(ゴホゴホ) 
\\	ちょっと、タバコをやめたほうが、いいんじゃないか。	
\\	ちょっと、タバコをやめたほうが、いいんじゃないか。 
\\	そうかもしれないですね。	
\\	そうかもしれないですね。 
\\	タバコをすうと、ゲホゲホするし、のどもイガイガするんですよ。	
\\	タバコをすうと、ゲホゲホするし、のどもイガイガするんですよ。 
\\	実は、先週位から、腹もごろごろするんですよね。	
\\	実は、先週位から、腹もごろごろするんですよね。 
\\	おいおい、大丈夫かよ。	
\\	おいおい、大丈夫かよ。 
\\	時々きりきりと痛いんですよね。	
\\	時々きりきりと痛いんですよね。 
\\	ものを食べると、胃がむかむかするし、ときどき、耳もキーンとするし。	
\\	ものを食べると、胃がむかむかするし、ときどき、耳もキーンとするし。 
\\	医者に行けよ。	
\\	医者に行けよ。 
\\	毎日、へとへとになるまで、働いているので、ストレスですかね、部長。	
\\	毎日、へとへとになるまで、働いているので、ストレスですかね、部長。 
\\	くもる
\\	嫌
\\	外
\\	急ぐ
\\	光る
\\	鳴る
\\	雷
\\	降る
\\	晴れる
\\	暗い
\\	嫌な天気ですね、先輩。	
\\	嫌な天気ですね、先輩。 
\\	どんよりくもっているし、雷はごろごろ鳴っているし、外は暗いし。	
\\	どんよりくもっているし、雷はごろごろ鳴っているし、外は暗いし。 
\\	じめじめしているから、そのうち、ザーッと降り始めるんじゃないか。	
\\	じめじめしているから、そのうち、ザーッと降り始めるんじゃないか。 
\\	さっきまで、からっと晴れてたのに・・・	
\\	さっきまで、からっと晴れてたのに・・・ 
\\	あ、いま、ぴかっと光った。	
\\	あ、いま、ぴかっと光った。 
\\	きゃ! あー、今の大きかったですね。	
\\	きゃ! あー、今の大きかったですね。 
\\	あ、ぽつぽつ降ってきた。	
\\	あ、ぽつぽつ降ってきた。 
\\	うわー。ざあざあ降りだ。急げ!駅まで、走るぞ。	
\\	うわー。ざあざあ降りだ。急げ!駅まで、走るぞ。 
\\	ちょっと、まって、先輩。ここ、つるつるしてて。あ、(ドスン)イタタタタ、・・・	
\\	ちょっと、まって、先輩。ここ、つるつるしてて。あ、(ドスン)イタタタタ、・・・ 
\\	あーん。新しいスーツがびしょびしょ・・・。	
\\	あーん。新しいスーツがびしょびしょ・・・。 
\\	この前
\\	熱々
\\	上手
\\	最高
\\	お菓子
\\	なる
\\	経つ
\\	中
\\	外
\\	かたい
\\	あおいちゃん、この前は、ケーキありがとう。	
\\	あおいちゃん、この前は、ケーキありがとう。 
\\	どうだった?	
\\	どうだった? 
\\	なんか、外はカッチカチで中はパッサパサだったよ。	
\\	なんか、外はカッチカチで中はパッサパサだったよ。 
\\	え?本当?作ったときは、パサパサに見えなかったけど...。 
\\	時間が経って、堅くなったのかな...。	
\\	時間が経って、堅くなったのかな...。 
\\	うちのお母さん、お菓子作るの上手なんだ。	
\\	うちのお母さん、お菓子作るの上手なんだ。 
\\	母さんのケーキは、外は、ふんわり、中はしっとり。クッキーはさくさくでおいしいんだ。	
\\	母さんのケーキは、外は、ふんわり、中はしっとり。クッキーはさくさくでおいしいんだ。 
\\	へー。	
\\	へー。 
\\	素敵
\\	顔
\\	系
\\	結構です。
\\	筋肉
\\	俺
\\	焼ける
\\	見せる
\\	髪
\\	肌
\\	マジ
\\	あおいって、ビジュアル系好きなの?	
\\	あおいって、ビジュアル系好きなの? 
\\	うん。大好き。	
\\	うん。大好き。 
\\	みんな、きれいで、すらっとしてて。	
\\	みんな、きれいで、すらっとしてて。 
\\	見て、この写真、素敵じゃない?	
\\	見て、この写真、素敵じゃない? 
\\	ええ?これ、男?ガリガリじゃない。	
\\	ええ?これ、男?ガリガリじゃない。 
\\	なに、このつるつるの肌、サラサラの髪。女の子みたい。・・・	
\\	なに、このつるつるの肌、サラサラの髪。女の子みたい。・・・ 
\\	私は、もっとがっちりした人が好きだなぁ。	
\\	私は、もっとがっちりした人が好きだなぁ。 
\\	なになに、見せて見せて。・・・	
\\	なになに、見せて見せて。・・・ 
\\	うーん、顔はかわいいけど いまいち。俺はもう少し、ぽっちゃりした子のほうがいいな。	
\\	うーん、顔はかわいいけど いまいち。俺はもう少し、ぽっちゃりした子のほうがいいな。 
\\	・・・これ、男の人だよ。	
\\	・・・これ、男の人だよ。 
\\	えええええ?マジ?男?	
\\	えええええ?マジ?男? 
\\	でも、男は、こんがり焼けた肌に、むきむきがいいでしょ。	
\\	でも、男は、こんがり焼けた肌に、むきむきがいいでしょ。 
\\	見る?俺の筋肉?	
\\	見る?俺の筋肉? 
\\	結構です。	
\\	結構です。 
\\	きつい
\\	ぜいたく
\\	話し方
\\	隣
\\	部屋
\\	あいつ
\\	だらしない
\\	結構
\\	蹴飛ばす
\\	はっきり
\\	ゆうかちゃんっていいよね。ちゃきちゃき、さばさばしていて。	
\\	ゆうかちゃんっていいよね。ちゃきちゃき、さばさばしていて。 
\\	ゆうか?やめとけやめとけ。	
\\	ゆうか?やめとけやめとけ。 
\\	きついことをビシッと言うし、ちゃらちゃらしていると、蹴飛ばされるぞ。	
\\	きついことをビシッと言うし、ちゃらちゃらしていると、蹴飛ばされるぞ。 
\\	そうかな。	
\\	そうかな。 
\\	みんなは、ゆうかはてきぱきしてるって言うけど、俺は、せかせかしているだけだと思うね。	
\\	みんなは、ゆうかはてきぱきしてるって言うけど、俺は、せかせかしているだけだと思うね。 
\\	おっちょこちょいだしな。しかも、結構だらしないんだぜ。あいつの部屋ぐちゃぐちゃなんだ。	
\\	おっちょこちょいだしな。しかも、結構だらしないんだぜ。あいつの部屋ぐちゃぐちゃなんだ。 
\\	へー。よく知っているね。	
\\	へー。よく知っているね。 
\\	家が隣だからな。	
\\	家が隣だからな。 
\\	それより、あおいちゃんのほうが、いいだろう。ほのぼのした話し方だし、のほほんとして、かわいいし。	
\\	それより、あおいちゃんのほうが、いいだろう。ほのぼのした話し方だし、のほほんとして、かわいいし。 
\\	ぱっとしないんだよね。	
\\	ぱっとしないんだよね。 
\\	確かに、おっとりしてて、かわいいけど、はっきりしないし、ぐずぐずしていて好きじゃないんだ。	
\\	確かに、おっとりしてて、かわいいけど、はっきりしないし、ぐずぐずしていて好きじゃないんだ。 
\\	ぜいたくだなぁ。	
\\	ぜいたくだなぁ。 
\\	こんにちは。
\\	私
\\	どうぞ
\\	妻
\\	よろしくお願いします
\\	僕
\\	名前
\\	偶然
\\	主人
\\	こんにちは。わたしは、佐藤けい子です。	
\\	こんにちは。わたしは、佐藤けい子です。 
\\	どうぞ よろしく お願いします。	
\\	どうぞ よろしく お願いします。 
\\	けい子さん?(笑)妻の名前もけい子です。	
\\	けい子さん?(笑)妻の名前もけい子です。 
\\	あ、僕は田中明です。	
\\	あ、僕は田中明です。 
\\	それは、偶然!私の主人も明です。	
\\	それは、偶然!私の主人も明です。 
\\	英語
\\	通訳
\\	主人
\\	中国語
\\	昔
\\	勉強する
\\	忘れる
\\	でも
\\	私の主人は、通訳をしています。	
\\	私の主人は、通訳をしています。 
\\	へー。英語?	
\\	へー。英語? 
\\	いえ。主人は中国語ができます。	
\\	いえ。主人は中国語ができます。  
\\	けい子さんも中国語がわかる?	
\\	けい子さんも中国語がわかる? 
\\	私?私も昔、中国語を勉強しました。	
\\	私?私も昔、中国語を勉強しました。 
\\	でも・・・忘れました。(笑)	
\\	でも・・・忘れました。(笑) 
\\	皆さん
\\	校長先生
\\	先生
\\	新しい
\\	英語
\\	二年生
\\	教える
\\	こちら
\\	紹介
\\	みなさん、おはようございます。	
\\	みなさん、おはようございます。 
\\	おはようございます。	
\\	おはようございます。 
\\	新しい先生を紹介します。	
\\	新しい先生を紹介します。 
\\	こちらは、新しい英語の先生の佐藤けい子先生。	
\\	こちらは、新しい英語の先生の佐藤けい子先生。 
\\	佐藤先生は二年生の英語を教えます。	
\\	佐藤先生は二年生の英語を教えます。 
\\	よろしくお願いします。	
\\	よろしくお願いします。 
\\	お子さん
\\	子
\\	才
\\	男の子
\\	うちの子
\\	今週
\\	バーベキュー
\\	時間
\\	是非
\\	けい子さんは、お子さんがいますか。	
\\	けい子さんは、お子さんがいますか。 
\\	はい。4才の男の子がいます。	
\\	はい。4才の男の子がいます。 
\\	本当ですか?うちの子も4才。	
\\	本当ですか?うちの子も4才。 
\\	あ、今週の日曜日、時間がありますか。	
\\	あ、今週の日曜日、時間がありますか。 
\\	バーベキューをしませんか。	
\\	バーベキューをしませんか。 
\\	はい!是非。	
\\	はい!是非。 
\\	出る
\\	旅行
\\	かかる
\\	何分
\\	ここ
\\	空港
\\	行く
\\	番線
\\	ホーム(プラットホーム)
\\	次
\\	電車
\\	飛行機
\\	すみません。これは空港へ行きますか。	
\\	すみません。これは空港へ行きますか。 
\\	いえ、行きません。2番線のホームに行ってください。	
\\	いえ、行きません。2番線のホームに行ってください。 
\\	次の電車は2番線から出ます。	
\\	次の電車は2番線から出ます。 
\\	ここから、空港まで何分かかりますか。	
\\	ここから、空港まで何分かかりますか。 
\\	30分くらいです。・・・旅行?	
\\	30分くらいです。・・・旅行? 
\\	いえ、飛行機を見に行きます。	
\\	いえ、飛行機を見に行きます。 
\\	電話
\\	後
\\	すぐ
\\	着く
\\	いい
\\	今
\\	何時
\\	診察
\\	歯科
\\	受付
\\	はい、お電話ありがとうございます。中山歯科です。	
\\	はい、お電話ありがとうございます。中山歯科です。 
\\	すみません。診察は、何時から何時までですか。	
\\	すみません。診察は、何時から何時までですか。 
\\	九時から六時までです。	
\\	九時から六時までです。 
\\	今から、いいですか。	
\\	今から、いいですか。 
\\	あ、はい。何時に着きますか。	
\\	あ、はい。何時に着きますか。 
\\	すぐです。五分後に着きます。	
\\	すぐです。五分後に着きます。 
\\	公園
\\	バーベキュー
\\	近く
\\	橋
\\	入口
\\	渡る
\\	見える
\\	池
\\	どこ
\\	場所
\\	田中さん?佐藤です。場所がわかりません。	
\\	田中さん?佐藤です。場所がわかりません。 
\\	今、どこにいますか。	
\\	今、どこにいますか。 
\\	公園の入口にいます。	
\\	公園の入口にいます。 
\\	池が見えますか。	
\\	池が見えますか。 
\\	はい、見えます。	
\\	はい、見えます。 
\\	池の橋を渡ってください。	
\\	池の橋を渡ってください。 
\\	橋の近くでバーベキューをしています。	
\\	橋の近くでバーベキューをしています。 
\\	橋?ん?ちょっとわかりません。	
\\	橋?ん?ちょっとわかりません。 
\\	わかりました。じゃ、入り口で待っていてください。今行きます!	
\\	わかりました。じゃ、入り口で待っていてください。今行きます! 
\\	ソーセージ
\\	うちの子
\\	お水
\\	お茶
\\	飲む
\\	ジュース
\\	ワイン
\\	野菜
\\	ビール
\\	買う
\\	妻
\\	ソーセージと野菜を買ってきました。	
\\	ソーセージと野菜を買ってきました。 
\\	ありがとうございます。	
\\	ありがとうございます。 
\\	ビールやワインやジュースがあります。何を飲みますか。	
\\	ビールやワインやジュースがあります。何を飲みますか。 
\\	じゃ、お茶ありますか?	
\\	じゃ、お茶ありますか? 
\\	ありますよ。はいどうぞ。順君は?	
\\	ありますよ。はいどうぞ。順君は? 
\\	お水。	
\\	お水。 
\\	お水?	
\\	お水? 
\\	すみません。うちの子、ジュースもお茶も好きじゃなくて・・・。	
\\	すみません。うちの子、ジュースもお茶も好きじゃなくて・・・。 
\\	すぐ
\\	すごい
\\	作る
\\	誰
\\	上手
\\	あげる
\\	おばちゃん
\\	歯医者
\\	旦那
\\	来る
\\	佐藤さん、旦那さんは?	
\\	佐藤さん、旦那さんは? 
\\	あ、主人は、歯医者に行きました。すぐ、来ます。	
\\	あ、主人は、歯医者に行きました。すぐ、来ます。 
\\	はい、おばちゃん。これ、あげる。	
\\	はい、おばちゃん。これ、あげる。 
\\	これは何?	
\\	これは何? 
\\	つる!	
\\	つる! 
\\	うわー。上手。誰が作ったの?	
\\	うわー。上手。誰が作ったの? 
\\	ゆなが作ったの。	
\\	ゆなが作ったの。 
\\	すごいねぇ。	
\\	すごいねぇ。 
\\	さあ
\\	おしっこ
\\	ビール
\\	どうぞ
\\	冷たい
\\	もの
\\	必要
\\	こっち
\\	遅れる
\\	すぐそこ
\\	あ、来た来た。パパー!こっちこっち!	
\\	あ、来た来た。パパー!こっちこっち! 
\\	すみません。遅れました。はじめまして。佐藤です。	
\\	すみません。遅れました。はじめまして。佐藤です。 
\\	はじめまして、田中です。さあ、どうぞ。	
\\	はじめまして、田中です。さあ、どうぞ。 
\\	何を飲みますか。	
\\	何を飲みますか。 
\\	えっと・・・冷たいものが飲みたいです。	
\\	えっと・・・冷たいものが飲みたいです。 
\\	ビール、ありますか。	
\\	ビール、ありますか。 
\\	パパー。僕、おしっこがしたい。	
\\	パパー。僕、おしっこがしたい。 
\\	ええ?ママは?	
\\	ええ?ママは? 
\\	パパがいい。	
\\	パパがいい。 
\\	そうか。わかった。	
\\	そうか。わかった。 
\\	あ、すぐそこにトイレがあります。	
\\	あ、すぐそこにトイレがあります。 
\\	たぶん、お金が必要です。	
\\	たぶん、お金が必要です。 
\\	フットサル
\\	得意
\\	野球
\\	来る
\\	今度
\\	楽しい
\\	試合
\\	大変
\\	練習
\\	サッカー
\\	へー。田中さんはフットサルをするんですか。	
\\	へー。田中さんはフットサルをするんですか。 
\\	ええ。練習は大変ですが、試合は楽しいです。	
\\	ええ。練習は大変ですが、試合は楽しいです。 
\\	今度、練習に来ませんか。	
\\	今度、練習に来ませんか。 
\\	うーん。野球は得意ですが、サッカーはちょっと・・・。	
\\	うーん。野球は得意ですが、サッカーはちょっと・・・。 
\\	親
\\	子供
\\	電話
\\	友達
\\	塾
\\	家
\\	警察
\\	先生
\\	はい、田中です。	
\\	はい、田中です。 
\\	先生?うちの子供がいないんです。	
\\	先生?うちの子供がいないんです。 
\\	は?	
\\	は? 
\\	家にも塾にもいません。	
\\	家にも塾にもいません。 
\\	友達のうちへも、電話しましたが、いないんです。	
\\	友達のうちへも、電話しましたが、いないんです。 
\\	警察へは電話をしましたか。	
\\	警察へは電話をしましたか。 
\\	ええ?警察?	
\\	ええ?警察? 
\\	生徒
\\	勉強
\\	いい
\\	帰る
\\	ごめんなさい。
\\	悪い
\\	無事
\\	お母さん
\\	心配
\\	飛行機
\\	そうか。じゃ、空港で 飛行機を 見ていたんだね。	
\\	そうか。じゃ、空港で 飛行機を 見ていたんだね。 
\\	心配したんだぜ。	
\\	心配したんだぜ。 
\\	はい。すみません。	
\\	はい。すみません。 
\\	お母さん、心配したわ。	
\\	お母さん、心配したわ。 
\\	もう、勉強したくなかったんだよ。	
\\	もう、勉強したくなかったんだよ。 
\\	・・・お母さん、悪かったわ。	
\\	・・・お母さん、悪かったわ。 
\\	いつも「勉強、勉強」言って。	
\\	"いつも「勉強、勉強」言って。 
\\	母さんは 悪くないよ。・・・ごめんなさい。	
\\	母さんは 悪くないよ。・・・ごめんなさい。 
\\	ま、無事で よかったよ。さ、帰るぞ。	
\\	ま、無事で よかったよ。さ、帰るぞ。 
\\	はい。	
\\	はい。 
\\	無事
\\	反省する
\\	かわいそう
\\	だけ
\\	毎日
\\	遊ぶ
\\	友だち
\\	読む
\\	マンガ
\\	テレビ
\\	勉強
\\	ストレス
\\	お母さん
\\	無事だったの?よかったじゃん。	
\\	無事だったの?よかったじゃん。 
\\	そうだなぁ。	
\\	そうだなぁ。 
\\	勉強のストレス かな。	
\\	勉強のストレス かな。 
\\	そうだろうなぁ。	
\\	そうだろうなぁ。 
\\	「勉強しろ」「テレビみるな」「マンガ読むな」「友だちと遊ぶな」。。。	
\\	"「勉強しろ」「テレビみるな」「マンガ読むな」「友だちと遊ぶな」。。。 
\\	毎日 毎日 勉強だけ。。。ま、かわいそうだよな。	
\\	毎日 毎日 勉強だけ。。。ま、かわいそうだよな。 
\\	お母さんも 反省したかな?	
\\	お母さんも 反省したかな? 
\\	どうだろうね。	
\\	どうだろうね。 
\\	包丁
\\	痛い
\\	指
\\	切る
\\	手伝う
\\	かぼちゃ
\\	作る
\\	スープ
\\	イタッ!	
\\	イタッ! 
\\	どうした?	
\\	どうした? 
\\	包丁で 指 切っちゃった。	
\\	包丁で 指 切っちゃった。 
\\	手伝うよ。	
\\	手伝うよ。 
\\	じゃ、包丁で、そのかぼちゃ 切って。	
\\	じゃ、包丁で、そのかぼちゃ 切って。 
\\	何を 作るの?	
\\	何を 作るの? 
\\	かぼちゃで スープを 作るの。	
\\	かぼちゃで スープを 作るの。 
\\	クイズ
\\	大きい
\\	カナダ
\\	チーター
\\	はやぶさ
\\	中
\\	鳥
\\	動物
\\	速い
\\	ばつ
\\	まる
\\	アマゾン川
\\	長い
\\	川
\\	一番
\\	世界
\\	残念
\\	世界で 一番長い川は アマゾン川です。
\\	か、×か?	
\\	世界で 一番長い川は アマゾン川です。
\\	か、×か? 
\\	ナイル川!	
\\	ナイル川! 
\\	一番速い動物は チーター。	
\\	一番速い動物は チーター。 
\\	では、鳥の中で、何が 一番速いですか。	
\\	では、鳥の中で、何が 一番速いですか。 
\\	はやぶさが 一番速いです。	
\\	はやぶさが 一番速いです。 
\\	カナダと中国と、どちらが 大きいですか。	
\\	カナダと中国と、どちらが 大きいですか。 
\\	えっと・・・えっと・・・・中国のほうが 大きいです。	
\\	えっと・・・えっと・・・・中国のほうが 大きいです。 
\\	残念!中国より カナダの方が 大きいです。	
\\	残念!中国より カナダの方が 大きいです。 
\\	老人
\\	かさ
\\	ピンク
\\	降りる
\\	電車
\\	乗る
\\	昨日
\\	右
\\	夜
\\	曲がる
\\	角
\\	忘れ物
\\	駅
\\	忘れる
\\	すみません。忘れ物センターはどこですか?	
\\	すみません。忘れ物センターはどこですか? 
\\	あの角を曲がってください。右にあります。	
\\	あの角を曲がってください。右にあります。 
\\	あの、昨日の夜10時頃、上野で電車に乗って、赤坂で電車を降りました。	
\\	あの、昨日の夜10時頃、上野で電車に乗って、赤坂で電車を降りました。 
\\	ピンクのかさを忘れました。	
\\	ピンクのかさを忘れました。 
\\	これですか?	
\\	これですか? 
\\	あ、そうです。	
\\	あ、そうです。 
\\	クイズ
\\	座る
\\	始める
\\	テスト
\\	みんな
\\	勉強する
\\	塾
\\	面白い
\\	見る
\\	俺
\\	昨日のクイズみた?面白かったね。	
\\	昨日のクイズみた?面白かったね。 
\\	塾で、みられなかった。	
\\	塾で、みられなかった。 
\\	俺も、勉強していたからみられなかった。	
\\	俺も、勉強していたからみられなかった。 
\\	ええ?みんな、勉強していたの?	
\\	ええ?みんな、勉強していたの? 
\\	はーい。テストを始めますので、座ってください。	
\\	はーい。テストを始めますので、座ってください。 
\\	テスト
\\	大変
\\	お腹
\\	赤い
\\	顔
\\	何時間
\\	いつも
\\	頭
\\	簡単
\\	問題
\\	倒れる
\\	テストどうだった?	
\\	テストどうだった? 
\\	うーん。今日のテストは問題が簡単だった。	
\\	うーん。今日のテストは問題が簡単だった。 
\\	ひろみは、頭がいいなぁ。	
\\	ひろみは、頭がいいなぁ。 
\\	ひろみは、いつも何時間勉強するの?	
\\	ひろみは、いつも何時間勉強するの? 
\\	うーん。あれ?顔があかいよ。どうしたの?	
\\	うーん。あれ?顔があかいよ。どうしたの? 
\\	え?そう?。。。なんだか、お腹が痛いんだよね。	
\\	え?そう?。。。なんだか、お腹が痛いんだよね。 
\\	佐藤先生!大変!かおるが倒れた。	
\\	佐藤先生!大変!かおるが倒れた。 
\\	あなた
\\	落とす
\\	銀
\\	傘
\\	金
\\	違う
\\	あげる
\\	正直
\\	夢
\\	あなたが落とした傘は この、銀の傘ですか。それとも、この金の傘ですか?	
\\	あなたが落とした傘は この、銀の傘ですか。それとも、この金の傘ですか? 
\\	違います。	
\\	違います。 
\\	私の落とした傘は、ピンクの傘です。	
\\	私の落とした傘は、ピンクの傘です。 
\\	あなたは、正直ですね。	
\\	あなたは、正直ですね。 
\\	では、あなたに、銀の傘と 金の傘を あげます。	
\\	では、あなたに、銀の傘と 金の傘を あげます。 
\\	はっ・・・夢?	
\\	はっ・・・夢? 
\\	孫
\\	夢
\\	良い
\\	僕
\\	作る
\\	折り紙
\\	金色
\\	金婚式
\\	おじいさん
\\	おばあさん
\\	銀色
\\	おじいさん、良い夢を見ましたよ。	
\\	おじいさん、良い夢を見ましたよ。 
\\	ほう。そうですか。	
\\	ほう。そうですか。 
\\	おじいちゃーん、おばあちゃーん。こんにちは。	
\\	おじいちゃーん、おばあちゃーん。こんにちは。 
\\	金婚式、おめでとう。	
\\	金婚式、おめでとう。 
\\	おやおや、ありがとう。	
\\	おやおや、ありがとう。 
\\	これをあげる。金の傘と、銀の傘。	
\\	これをあげる。金の傘と、銀の傘。 
\\	おじいちゃんには金色の傘。おばあちゃんには銀色の傘。	
\\	おじいちゃんには金色の傘。おばあちゃんには銀色の傘。 
\\	だれが作ったの?	
\\	だれが作ったの? 
\\	僕が、折り紙で作ったんだ。	
\\	僕が、折り紙で作ったんだ。 
\\	びっくりする
\\	両親
\\	一緒に
\\	終わる
\\	仕事
\\	一人
\\	病気
\\	モデル
\\	なる
\\	きれい
\\	ご飯
\\	だめ
\\	海外
\\	びっくりしたわよ。だめよ。ご飯を食べなさい。	
\\	びっくりしたわよ。だめよ。ご飯を食べなさい。 
\\	でも、きれいになりたいんです。モデルになりたいんです。	
\\	でも、きれいになりたいんです。モデルになりたいんです。 
\\	何をいっているの。病気になるわよ。	
\\	何をいっているの。病気になるわよ。 
\\	今、一人で住んでいるの?	
\\	今、一人で住んでいるの? 
\\	いえ、兄と二人で住んでいます。	
\\	いえ、兄と二人で住んでいます。 
\\	両親は仕事で海外に住んでいます。	
\\	両親は仕事で海外に住んでいます。 
\\	わかった。仕事、5時で終わるから、一緒に家でご飯を食べましょう。	
\\	わかった。仕事、5時で終わるから、一緒に家でご飯を食べましょう。 
\\	え?先生の家へ行くんですか?	
\\	え?先生の家へ行くんですか? 
\\	お兄さんと二人で来て。	
\\	お兄さんと二人で来て。 
\\	兄
\\	後で
\\	手伝う
\\	青い
\\	取る
\\	お皿
\\	細い
\\	運動する
\\	申す
\\	先生、兄は後で来るといっていました。	
\\	先生、兄は後で来るといっていました。 
\\	あらそう。	
\\	あらそう。 
\\	手伝いましょうか。	
\\	手伝いましょうか。 
\\	じゃ、その、青いの取って。	
\\	じゃ、その、青いの取って。 
\\	この青いお皿ですか?	
\\	この青いお皿ですか? 
\\	そう。ありがとう。	
\\	そう。ありがとう。 
\\	細くなりたいのはわかるけど、食べないのはだめよ。	
\\	細くなりたいのはわかるけど、食べないのはだめよ。 
\\	はい。これからは運動しようと思います。 
\\	はじめまして。かおるの兄の鈴木広と申します。	
\\	はじめまして。かおるの兄の鈴木広と申します。 
\\	大学
\\	卒業
\\	決める
\\	大学院
\\	悩む
\\	何度も
\\	学費
\\	高い
\\	かかる
\\	広さんは、大学卒業後、どうするの?	
\\	広さんは、大学卒業後、どうするの? 
\\	まだ何をするか決めていません。	
\\	まだ何をするか決めていません。 
\\	大学院に行こうかどうか、悩んでいます。	
\\	大学院に行こうかどうか、悩んでいます。 
\\	大学院か・・・。	
\\	大学院か・・・。 
\\	私も何度も大学院に行こうと思った。	
\\	私も何度も大学院に行こうと思った。 
\\	でも、学費が高いから…。	
\\	でも、学費が高いから…。 
\\	そうですね。一年に200万円もかかります。	
\\	そうですね。一年に200万円もかかります。 
\\	校長先生
\\	思う
\\	たぶん
\\	前
\\	駅
\\	場所
\\	歓迎会
\\	慣れる
\\	一週間
\\	学校
\\	この学校に 来てから1週間ですね。	
\\	この学校に 来てから1週間ですね。 
\\	どうですか。慣れましたか。	
\\	どうですか。慣れましたか。 
\\	はい。	
\\	はい。 
\\	今日、7時から あなたの歓迎会を しますから、来てくださいね。	
\\	今日、7時から あなたの歓迎会を しますから、来てくださいね。 
\\	ありがとうございます。	
\\	ありがとうございます。 
\\	何時くらいまでですか。	
\\	何時くらいまでですか。 
\\	たぶん、7時から9時までだ と思いますよ。	
\\	たぶん、7時から9時までだ と思いますよ。 
\\	場所は、駅の前のバーイノベーティブです。	
\\	場所は、駅の前のバーイノベーティブです。 
\\	取捨選択
\\	心機一転
\\	喜怒哀楽
\\	危機一髪
\\	大器晩成
\\	単刀直入
\\	他人行儀
\\	大胆不敵
\\	絶体絶命
\\	前代未聞
\\	前途多難
\\	他力本願
\\	がっかり
\\	振り返る
\\	横切る
\\	何者か
\\	叫ぶ
\\	グランド
\\	校舎
\\	引っ張る
\\	何も言ず
\\	強引に
\\	半ば
\\	つかむ
\\	手首
\\	そのとき
\\	首
\\	寄っていく
\\	現れる
\\	取ろうとする
\\	開けようとする
\\	あご
\\	おでこ
\\	影
\\	咄嗟に
\\	苦しそうに
\\	覗きこんでみる
\\	運転席側のドア
\\	どうやら
\\	叩く
\\	閉める
\\	トイレの個室
\\	学校
\\	降りる
\\	混む
\\	結構
\\	道路
\\	やってくる
\\	深夜
\\	期待できそうだ
\\	病院
\\	変わった
\\	多数
\\	目撃者
\\	心霊
\\	建物
\\	探し始める
\\	意外に
\\	職員室
\\	理科室
\\	教室
\\	道路
\\	おかしなこと
\\	安全運転
\\	足あと
\\	警官
\\	検問
\\	叩く
\\	ひたすら
\\	瞬間
\\	見つめる
\\	白髪
\\	荷台
\\	つぶれる
\\	壊れている
\\	連れて行く
\\	迷う
\\	婆
\\	返す
\\	付き合う
\\	身代わり
\\	寺
\\	神社
\\	始末する
\\	預ける
\\	抱き抱える
\\	暴れる
\\	ダンボール
\\	気を失う
\\	お祓い
\\	住職
\\	いい加減
\\	悩む
\\	避ける
\\	酔っ払う
\\	寄る
\\	泊まる
\\	みらむ
\\	たんす
\\	日本人形
\\	むく
\\	気味悪い
\\	亡くなる
\\	噛む
\\	跡
\\	魂
\\	クレープ
\\	生
\\	僕
\\	どれどれ
\\	する
\\	決める
\\	食べる
\\	におい
\\	クレープ屋
\\	生クリーム
\\	おっ、クレープ屋だ。	
\\	おっ、クレープ屋だ。 
\\	すごくいいにおいだわ。ね、クレープ食べない?	
\\	すごくいいにおいだわ。ね、クレープ食べない? 
\\	いいねー。食べよう。食べよう。	
\\	いいねー。食べよう。食べよう。 
\\	何にしようかなあ。どれどれ。私は、バナナチョコ生クリームに決ーめたっ!	
\\	何にしようかなあ。どれどれ。私は、バナナチョコ生クリームに決ーめたっ! 
\\	じゃあ、僕はストロベリーチーズにしようっと。	
\\	じゃあ、僕はストロベリーチーズにしようっと。 
\\	ブロマイド写真
\\	だらけ
\\	すごい
\\	アイドル
\\	有名人
\\	こっち
\\	このお店、写真だらけだね。	
\\	このお店、写真だらけだね。 
\\	すごいわね。誰の写真かしら。	
\\	すごいわね。誰の写真かしら。 
\\	うーん。日本のアイドルとか有名人じゃない。	
\\	うーん。日本のアイドルとか有名人じゃない。 
\\	どこもいっしょね。	
\\	知ってる?こういう写真をこっちでは、ブロマイドって言う。	
\\	知ってる?こういう写真をこっちでは、ブロマイドって言う。 
\\	へー。	
\\	へー。 
\\	切符
\\	待つ
\\	スイカ
\\	昨日
\\	初めて
\\	読む
\\	確か
\\	買う
\\	切符買うから、ちょっと待ってて。	
\\	切符買うから、ちょっと待ってて。 
\\	えーっ、スイカじゃないのかよ。	
\\	えーっ、スイカじゃないのかよ。 
\\	スイカ?はあ?	
\\	スイカ?はあ? 
\\	あーそうか。昨日、初めて日本に来たんだっけ。	
\\	あーそうか。昨日、初めて日本に来たんだっけ。 
\\	新宿までいくら?読めないんだけど。	
\\	新宿までいくら?読めないんだけど。 
\\	確か、130円だったかなあ。150円かも。	
\\	確か、130円だったかなあ。150円かも。 
\\	踏む
\\	ガム
\\	ちょっと
\\	ティッシュ
\\	早く
\\	信号
\\	青
\\	渡る
\\	あっ、なんか踏んじゃったみたい。	
\\	あっ、なんか踏んじゃったみたい。 
\\	どうしたの?	
\\	どうしたの? 
\\	あー、ガム踏んじゃったよ。	
\\	あー、ガム踏んじゃったよ。 
\\	ちょっと待って。はい、ティッシュ。	
\\	ちょっと待って。はい、ティッシュ。 
\\	ありがとう。	
\\	ありがとう。 
\\	早く早く。信号が青のうちに渡りましょっ。	
\\	早く早く。信号が青のうちに渡りましょっ。 
\\	タクシー
\\	速い
\\	高い
\\	安い
\\	遅い
\\	電車
\\	満員
\\	嫌
\\	早い
\\	恵比寿までどうやって行こうか。	
\\	恵比寿までどうやって行こうか。 
\\	タクシーで行こうか。	
\\	タクシーで行こうか。 
\\	んん、、、いや、速いけど高いでしょ。バスは安いけど、ちょっと遅いし。	
\\	んん、、、いや、速いけど高いでしょ。バスは安いけど、ちょっと遅いし。 
\\	じゃあ、やっぱり電車で行こうか。	
\\	じゃあ、やっぱり電車で行こうか。 
\\	そうだね、やっぱり速くて安いしね。満員電車乗るの嫌だけど。。。	
\\	そうだね、やっぱり速くて安いしね。満員電車乗るの嫌だけど。。。 
\\	待ち人
\\	予感
\\	いやな
\\	心配
\\	つなぐ
\\	携帯
\\	遅れる
\\	経過
\\	遅い
\\	着物
\\	来ず
\\	実家
\\	(30分経過)	
\\	(30分経過)	
\\	携帯もつながらないし。心配だわ・・・なんかいやな予感がする。実家に電話しようかな。	
\\	携帯もつながらないし。心配だわ・・・なんかいやな予感がする。実家に電話しようかな。 
\\	お腹
\\	決定
\\	多い
\\	量
\\	そこ
\\	前に
\\	ランチ
\\	中華
\\	お昼ご飯
\\	そろそろ
\\	減る
\\	麻婆豆腐
\\	お腹減ったね。	
\\	お腹減ったね。 
\\	そろそろ、お昼ご飯にしようか。	
\\	そろそろ、お昼ご飯にしようか。 
\\	そこの中華にしない。ランチやってるし。	
\\	そこの中華にしない。ランチやってるし。 
\\	いいよ。前にそこで食べたことあるけど、量も多いし、美味しかったよ。	
\\	いいよ。前にそこで食べたことあるけど、量も多いし、美味しかったよ。 
\\	じゃあ、決定。私は麻婆豆腐にしよーっと。	
\\	じゃあ、決定。私は麻婆豆腐にしよーっと。 
\\	やっと
\\	着く
\\	看板
\\	方面
\\	出口
\\	閉鎖
\\	中
\\	大丈夫
\\	おっちょこちょい
\\	やっと駅に着いたね。	
\\	やっと駅に着いたね。 
\\	明治神宮はどっちの出口だろう。	
\\	明治神宮はどっちの出口だろう。 
\\	あそこに、看板があるよ。	
\\	あそこに、看板があるよ。 
\\	えっと、明治神宮は・・・代々木方面出口だって。	
\\	えっと、明治神宮は・・・代々木方面出口だって。 
\\	あれっ、でも一番出口は閉鎖中だって。どうしよう・・・。	
\\	あれっ、でも一番出口は閉鎖中だって。どうしよう・・・。 
\\	大丈夫だよ。明治神宮は二番出口って書いてあるじゃん。	
\\	大丈夫だよ。明治神宮は二番出口って書いてあるじゃん。 
\\	おれって、いつもおっちょこちょいだよなあ。	
\\	おれって、いつもおっちょこちょいだよなあ。 
\\	まっ、それがマモルの良い所でしょう。 
\\	写真
\\	今度
\\	もう一回
\\	ブス
\\	チーズ
\\	サンキュー
\\	デジカメ
\\	カメラマン
\\	とる
\\	感じ
\\	ねえ。せっかくだから、写真撮ろうか。	
\\	ねえ。せっかくだから、写真撮ろうか。 
\\	。私が撮ってあげる。なんたって、お父さんカメラマンだからね、デジカメかして。	
\\	。私が撮ってあげる。なんたって、お父さんカメラマンだからね、デジカメかして。 
\\	サンキュー。	
\\	サンキュー。 
\\	はい、チーズ。撮れたよ。どう?	
\\	はい、チーズ。撮れたよ。どう? 
\\	えーっ何これー。私こんなにブスじゃないわよ!	
\\	えーっ何これー。私こんなにブスじゃないわよ! 
\\	じゃあ、もう一回。はい、チーズ。今度はどう?	
\\	じゃあ、もう一回。はい、チーズ。今度はどう? 
\\	いい感じじゃない。私らしさがすごくでてるわ。さすが、ヒカルね。	
\\	いい感じじゃない。私らしさがすごくでてるわ。さすが、ヒカルね。 
\\	サンキュー。	
\\	サンキュー。 
\\	すごい
\\	酒樽
\\	銘柄
\\	全国
\\	蔵元
\\	約
\\	調べる
\\	嫌
\\	数える
\\	うわーすごい。何あれー。	
\\	うわーすごい。何あれー。 
\\	なんだろうね。行こう。	
\\	なんだろうね。行こう。 
\\	うわー、酒樽だあ。すごい数だね。数えよう。	
\\	うわー、酒樽だあ。すごい数だね。数えよう。 
\\	180個もあるよ。スッゲー。	
\\	180個もあるよ。スッゲー。 
\\	なんでここに、こんなにあるんだろう。	
\\	なんでここに、こんなにあるんだろう。 
\\	なんでだろうね。	
\\	なんでだろうね。 
\\	でも日本酒って、銘柄いくつあるんだろうね。	
\\	でも日本酒って、銘柄いくつあるんだろうね。 
\\	全国に、約2000の蔵元(酒を作る場所)があるらしいよ。だから、銘柄は数えられないくらいあるんじゃないかな。じゃあ、マモル調べてみてよ。	
\\	全国に、約2000の蔵元(酒を作る場所)があるらしいよ。だから、銘柄は数えられないくらいあるんじゃないかな。じゃあ、マモル調べてみてよ。 
\\	えーっ。嫌だよ。	
\\	えーっ。嫌だよ。 
\\	本日
\\	求める
\\	機会
\\	最大
\\	シューズ
\\	人気
\\	是非
\\	品
\\	最終日
\\	セール
\\	立ち寄る
\\	本日はセールの最終日でーす。	
\\	本日はセールの最終日でーす。 
\\	良い品がたくさんでていますよー。	
\\	良い品がたくさんでていますよー。 
\\	ぜひ、お立ち寄り下さいませー。	
\\	ぜひ、お立ち寄り下さいませー。 
\\	本日はセールの最終日でございます。	
\\	本日はセールの最終日でございます。 
\\	本日、人気のシューズが最大50
\\	でございます。	
\\	本日、人気のシューズが最大50
\\	でございます。 
\\	この機会にぜひお求めくださいませー。	
\\	この機会にぜひお求めくださいませー。 
\\	おいしい
\\	ちょっと
\\	ダメ
\\	すこし
\\	本当
\\	食いしん坊
\\	食べる
\\	お願い
\\	食べさせる
\\	このクレープおいしいーっ!	
\\	このクレープおいしいーっ! 
\\	ちょっと、食べさせてー。	
\\	ちょっと、食べさせてー。 
\\	えーっ。ダメーっ。	
\\	えーっ。ダメーっ。 
\\	私のも、すこし食べていいから。お願い!	
\\	私のも、すこし食べていいから。お願い! 
\\	じゃあ、ちょっとだけだからね。	
\\	じゃあ、ちょっとだけだからね。 
\\	わーい。	
\\	わーい。 
\\	もうっ。アイは、本当に食いしん坊なんだからーっ!	
\\	もうっ。アイは、本当に食いしん坊なんだからーっ! 
\\	パスタ
\\	食べ放題
\\	安い
\\	60分間
\\	お腹
\\	減る
\\	さっき
\\	中華
\\	ねえ、パスタ食べ放題だって。	
\\	ねえ、パスタ食べ放題だって。 
\\	いくら?	
\\	いくら? 
\\	980円だって。安いね。	
\\	980円だって。安いね。 
\\	時間は?	
\\	時間は? 
\\	えーっと、60分間。食べよっか?	
\\	えーっと、60分間。食べよっか? 
\\	お腹減ってるの?さっき、中華食べたばっかりじゃん。	
\\	お腹減ってるの?さっき、中華食べたばっかりじゃん。 
\\	パスタ
\\	大好き
\\	すごい
\\	並ぶ
\\	組
\\	嫌
\\	止める
\\	うん。パスタは大好きだし、食べたーい。	
\\	うん。パスタは大好きだし、食べたーい。 
\\	うわっ、すごく並んでるよ。 
\\	えーっ。どれくらい?	
\\	えーっ。どれくらい? 
\\	5組、6組くらいは待ってるよ。	
\\	5組、6組くらいは待ってるよ。 
\\	待つのは嫌ね。やっぱり、やめようか。	
\\	待つのは嫌ね。やっぱり、やめようか。 
\\	やめましょう。クレープでも食べに行きましょうよ。	
\\	やめましょう。クレープでも食べに行きましょうよ。 
\\	すみません
\\	下さい
\\	なる
\\	あと
\\	パン
\\	牛乳
\\	受け取る
\\	お返し
\\	どうも
\\	すみません。このパンとあのパンを下さい。	
\\	すみません。このパンとあのパンを下さい。 
\\	はい、210円になります。	
\\	はい、210円になります。 
\\	あ、あと、牛乳もお願いします。	
\\	あ、あと、牛乳もお願いします。 
\\	では320円になります。(500円受け取る)180円のお返しです。	
\\	では320円になります。(500円受け取る)180円のお返しです。 
\\	はい、どうも。	
\\	はい、どうも。 
\\	ありがとうございました。	
\\	ありがとうございました。 
\\	すみません
\\	新宿
\\	真っすぐ
\\	歩いて
\\	電車
\\	行く
\\	わかりました
\\	位
\\	だいたい
\\	すみません。新宿はどこですか?	
\\	すみません。新宿はどこですか? 
\\	新宿ですか。えっと、この道を真っすぐ行くと新宿に行きますよ。	
\\	新宿ですか。えっと、この道を真っすぐ行くと新宿に行きますよ。 
\\	歩いてどのくらいですか。	
\\	歩いてどのくらいですか。 
\\	そうですね、だいたい30分位です。電車の方が早いですよ。	
\\	そうですね、だいたい30分位です。電車の方が早いですよ。 
\\	わかりました。ありがとうございます。	
\\	わかりました。ありがとうございます。 
\\	男
\\	届く
\\	免許証
\\	長い
\\	黒い
\\	落とす
\\	財布
\\	どうしました
\\	警官
\\	もしかして
\\	すみません!!	
\\	すみません!! 
\\	どうしましたか。	
\\	どうしましたか。 
\\	財布を落としてしまって。。。届いてないですか。	
\\	財布を落としてしまって。。。届いてないですか。 
\\	どんな財布ですか。	
\\	どんな財布ですか。 
\\	黒くて、長くて、、、1万5千円ぐらい入ってたかな、、、	
\\	黒くて、長くて、、、1万5千円ぐらい入ってたかな、、、 
\\	もしかして、これですか。	
\\	もしかして、これですか。 
\\	すみません
\\	薬剤師
\\	客
\\	薬
\\	ひどい
\\	鼻水
\\	あと
\\	だるい
\\	頭が痛い
\\	熱っぽい
\\	症状
\\	風邪
\\	どうしました
\\	薬局
\\	すみません。	
\\	すみません。 
\\	どうしました?	
\\	どうしました? 
\\	風邪っぽいんですけど。。。	
\\	風邪っぽいんですけど。。。 
\\	どんな症状ですか。	
\\	どんな症状ですか。 
\\	熱っぽくて、頭が痛くて、だるくて。あと、鼻水がひどいです。	
\\	熱っぽくて、頭が痛くて、だるくて。あと、鼻水がひどいです。 
\\	そうですねぇ。。。この薬とこの薬をご飯の後に飲めば、すぐ治りますよ!	
\\	そうですねぇ。。。この薬とこの薬をご飯の後に飲めば、すぐ治りますよ! 
\\	遅い
\\	そんなに
\\	ナンバープレート
\\	黄色い
\\	四角い
\\	白い
\\	確か
\\	電話に出ない
\\	電話
\\	大きい
\\	あいつ、遅いなー。	
\\	あいつ、遅いなー。 
\\	電話してみた?	
\\	電話してみた? 
\\	してみたけど、出ないんだよ。確か、車で来るんだよね。	
\\	してみたけど、出ないんだよ。確か、車で来るんだよね。 
\\	そうそう。車は、白くて、四角くて、黄色いナンバープレートの。そんなに大きくないよ。	
\\	そうそう。車は、白くて、四角くて、黄色いナンバープレートの。そんなに大きくないよ。 
\\	あ、あれじゃない?	
\\	あ、あれじゃない? 
\\	竹下通り
\\	人の数
\\	ここ
\\	有名(な)
\\	ほら
\\	あそこ
\\	なるほど
\\	うわー、すごい人の数だなあ。	
\\	うわー、すごい人の数だなあ。 
\\	すごいね。なんでだろうね。	
\\	すごいね。なんでだろうね。 
\\	わかった!	
\\	わかった! 
\\	えっ、何?	
\\	えっ、何? 
\\	ここが、有名な竹下通りだよ。ほらっ、あそこに書いてあるよ。	
\\	ここが、有名な竹下通りだよ。ほらっ、あそこに書いてあるよ。 
\\	なるほどー。ここなのね。	
\\	なるほどー。ここなのね。 
\\	誰
\\	物知り
\\	踊る
\\	舞
\\	お守り
\\	対応
\\	参拝
\\	仕事
\\	働く
\\	神社
\\	詳しく
\\	知らない
\\	巫女
\\	あの人って、何をする人だろう?	
\\	あの人って、何をする人だろう? 
\\	えっ。知らないの。巫女さんだよ。	
\\	えっ。知らないの。巫女さんだよ。 
\\	どんな仕事してるんだろうね。	
\\	どんな仕事してるんだろうね。 
\\	参拝者への対応とか、お守りをあげるとか、舞を踊るとか・・・いろいろじゃないの。	
\\	参拝者への対応とか、お守りをあげるとか、舞を踊るとか・・・いろいろじゃないの。 
\\	さすが、マサル。物知りだなあ。	
\\	さすが、マサル。物知りだなあ。 
\\	今日
\\	お茶
\\	疲れる
\\	その前に
\\	久しぶり
\\	じゃあ
\\	先週
\\	また
\\	カラオケ
\\	賛成
\\	今日何しようか。。。カラオケ行かない?	
\\	今日何しようか。。。カラオケ行かない? 
\\	また?先週も行ったじゃん。	
\\	また?先週も行ったじゃん。 
\\	じゃあ、何したいんだよ。	
\\	じゃあ、何したいんだよ。 
\\	じゃあ、久しぶりにボーリング行こうよ。	
\\	じゃあ、久しぶりにボーリング行こうよ。 
\\	おお、良いね!けどその前に疲れたからお茶しようよ。	
\\	おお、良いね!けどその前に疲れたからお茶しようよ。 
\\	賛成!そうしよ。	
\\	賛成!そうしよ。 
\\	駄目(な)
\\	エスカレーター
\\	右
\\	左側
\\	関西
\\	立つ
\\	人
\\	うるさい
\\	ちょっと、雅人、ダメだよ。	
\\	ちょっと、雅人、ダメだよ。 
\\	ん?何がや。	
\\	ん?何がや。 
\\	東京では、エスカレーターの左側に立つの。	
\\	東京では、エスカレーターの左側に立つの。 
\\	何を言うとんねん。関西人は右側に立つんや。	
\\	何を言うとんねん。関西人は右側に立つんや。 
\\	ほらっ、人が来た。	
\\	ほらっ、人が来た。 
\\	メンズカットサロン
\\	配る
\\	困る
\\	ちょうど
\\	風邪をひく
\\	受け取る
\\	忘れる
\\	メンズカットサロン「サブロー」です。よろしくお願いしま~す。	
\\	メンズカットサロン「サブロー」です。よろしくお願いしま~す。 
\\	ティッシュ、ください。	
\\	ティッシュ、ください。 
\\	(男の人に配りたいんだけどなぁ。まぁ、いいや。)はい、どうぞ。	
\\	(男の人に配りたいんだけどなぁ。まぁ、いいや。)はい、どうぞ。 
\\	ティッシュ忘れちゃって、困っていたの。ちょうどよかったにゃー!	
\\	ティッシュ忘れちゃって、困っていたの。ちょうどよかったにゃー! 
\\	あ、はぁ。。(この季節は、みんな風邪ひいているから ティッシュを受け取るのか・・・。)	
\\	あ、はぁ。。(この季節は、みんな風邪ひいているから ティッシュを受け取るのか・・・。) 
\\	手水舍
\\	お参り
\\	ちゃんと
\\	お清め
\\	入れ物
\\	ひしゃく
\\	持ち替える
\\	すすぐ
\\	お母さん、どうしたらいいのー。	
\\	お母さん、どうしたらいいのー。 
\\	お手てとお口をきれいにするのよ。	
\\	お手てとお口をきれいにするのよ。 
\\	なんでー。	
\\	なんでー。 
\\	お参りする前は、ちゃんとお清めしないといけないの。	
\\	お参りする前は、ちゃんとお清めしないといけないの。 
\\	この入れ物はー?	
\\	この入れ物はー? 
\\	ひしゃくって言うのよ。	
\\	ひしゃくって言うのよ。 
\\	まず右手にひしゃくを持って、左手を清めて、	
\\	まず右手にひしゃくを持って、左手を清めて、 
\\	崩れ落ちる
\\	包む
\\	気配
\\	真っ暗
\\	病室
\\	即死
\\	下敷き
\\	直撃
\\	差し掛かる
\\	工事現場
\\	取り戻す
\\	意識
\\	垂れ下がる
\\	皮膚
\\	近づく
\\	揺れる
\\	救急車
\\	見回す
\\	完了
\\	大怪我
\\	自腹
\\	得体の知れない
\\	現象
\\	ビビる
\\	無駄金
\\	納得する
\\	たじろぐ
\\	得体の知れない
\\	担ぎ上げる
\\	乗車拒否
\\	待合室
\\	精神状態
\\	搬送する
\\	睨む
\\	お札
\\	業者
\\	引っ越す
\\	都合
\\	臭い
\\	臭い
\\	寒気
\\	図面
\\	担当者
\\	眼球
\\	すり抜ける
\\	連絡を取る
\\	一部始終
\\	尋問する
\\	白線
\\	突然
\\	ひき肉
\\	眼球
\\	悲鳴
\\	形相
\\	振り向く
\\	叫ぶ
\\	群がる
\\	活け作り
\\	堪らない
\\	覚める
\\	遇う
\\	薄暗い
\\	内側
\\	恐い
\\	静か
\\	職場
\\	鉄柵
\\	悲鳴
\\	気配
\\	暗闇
\\	異臭
\\	死角
\\	蛍光灯
\\	非常階段
\\	人身事故
\\	管理会社
\\	通知
\\	処理する
\\	発注
\\	鉄格子
\end{CJK}
\end{document}