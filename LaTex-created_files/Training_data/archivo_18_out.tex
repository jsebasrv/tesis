\documentclass[8pt]{extreport} 
\usepackage{hyperref}
\usepackage{CJKutf8}
\begin{document}
\begin{CJK}{UTF8}{min}
\\	でしょう	
\\	でしょう	
\\	でしょう	でしょう
\\	明日は雨が降るでしょう。	
\\	でしょう
\\	日本人が晩ごはんによく食べるものは寿司でしょう。	
\\	でしょう
\\	今夜は風が強いでしょう。	
\\	でしょう
\\	学校から帰ったら、すぐ宿題するって、お父さんと約束したでしょう。	
\\	でしょう
\\	宿題は後でできるでしょう?手伝いなさい。	
\\	でしょう
\\	この帽子はどこで買いましたか?	
\\	で
\\	このレストランで食事しましょう。	
\\	で
\\	家の中で遊ばないで。	
\\	で
\\	どこで日本語を勉強していますか?	
\\	で
\\	新宿駅南口で会いましょうか?	
\\	で
\\	被害者は公園で殺された。	
\\	で
\\	私はいつも図書館で本を読みます。	
\\	で
\\	ここでサッカーしないでください。	
\\	で
\\	教室には生徒がいる。	
\\	がいる
\\	図書館に人がいっぱいいる。	
\\	がいる
\\	台所には母がいます。	
\\	がいる
\\	お金がありません。	
\\	がある
\\	あなたの鞄はテーブルの上にある。	
\\	がある
\\	質問がありますか?	
\\	がある
\\	彼には家が二軒あります。	
\\	がある
\\	このアパートにはルームが二つあります。	
\\	がある
\\	言いたいことがありますか?	
\\	がある
\\	英会話をもっと練習したほうがいい。	
\\	ほうがいい
\\	私の言うとおりにしたほうがいいですよ。	
\\	ほうがいい
\\	願書を早く送るほうがいいですよ。	
\\	ほうがいい
\\	あなたの部屋は汚すぎる。毎日掃除したほうがいい。	
\\	ほうがいい
\\	気をつけたほうがいいよ。	
\\	ほうがいい
\\	ゆっくり休んだほうがいいですよ。	
\\	ほうがいい
\\	ちゃんと勉強するほうがいい。しないと成績が悪くなっちゃうよ。	
\\	ほうがいい
\\	彼のことを忘れたほうがいい。	
\\	ほうがいい
\\	彼女に謝ったほうがいいと思う。	
\\	ほうがいい
\\	約束を忘れないほうがいい。	
\\	ほうがいい
\\	彼は気が短いから、怒らせないほうがいい。	
\\	ほうがいい
\\	子供に「進撃の巨人」を読ませないほうがいい。	
\\	ほうがいい
\\	目に悪いから、ずっとパソコンを使わないほうがいい。	
\\	ほうがいい
\\	子供の前に悪い言葉を言わないほうがいいです。	
\\	ほうがいい
\\	彼女を泣かさないほうがいいよ。	
\\	ほうがいい
\\	やっといい仕事を見つけたから、辞めないほうがいい。	
\\	ほうがいい
\\	他人の物を勝手に使わないほうがいい。	
\\	ほうがいい
\\	子供はまだ3歳です。	
\\	まだ
\\	わたしはまだ結婚していません。	
\\	まだ
\\	まだ一時間もあるわよ。	
\\	まだ
\\	まだ、わたしには意見が述べられない。	
\\	まだ
\\	わたしは手紙の返事をまだ書いていません。	
\\	まだ
\\	あなたが最後ではありませんよ。中田がまだ来ていません。	
\\	まだ
\\	まだ決まってないの?	
\\	まだ
\\	まだ、バスが来る様子がない。	
\\	まだ
\\	海はまだ荒れていますよ。	
\\	まだ
\\	焼肉を食べましょう!	
\\	ましょう
\\	ここで遊びましょうか?	
\\	ましょう
\\	一緒に日本語を勉強しましょう!	
\\	ましょう
\\	テストを始めましょうか?	
\\	ましょう
\\	学校で会いましょう!	
\\	ましょう
\\	寒いから、熱いコーヒーを飲みましょうか?	
\\	ましょう
\\	行きましょう!	
\\	ましょう
\\	校歌を歌いましょう!	
\\	ましょう
\\	もう暗くなったから、帰りましょうか?	
\\	ましょう
\\	マラソン大会に参加しましょう!	
\\	ましょう
\\	弟は歌うのが上手だ。	
\\	のがじょうず
\\	私の彼氏はキスするのが上手ですよ。	
\\	のがじょうず
\\	ビールを飲みに行きましょう!	
\\	にいく
\\	私は彼を起こしに行く。	
\\	にいく
\\	薪を集めに行こう!	
\\	にいく
\\	父が新しいパソコンを買いに行った。	
\\	にいく
\\	ごみを捨てに行きます。	
\\	にいく
\\	お寿司を食べに行こう!	
\\	にいく
\\	駅の前にいるから、迎えに行ってくれる?	
\\	にいく
\\	私は絵を描くのがすごく下手です。	
\\	のがへた
\\	僕は走るのが下手です。	
\\	のがへた
\\	おれは泳ぐのが下手だ。	
\\	のがへた
\\	そのかばんは高すぎるから、買えない。	
\\	すぎる
\\	わたしの荷物は重すぎる。	
\\	すぎる
\\	この子達にギャンブルは早すぎる。	
\\	すぎる
\\	彼らは拍手のしすぎで手の感覚がなくなった。	
\\	すぎる
\\	暑すぎるのだ。泳ぎたい。	
\\	すぎる
\\	食べすぎると、太ってしまうよ。	
\\	すぎる
\\	昨日ビールを飲みすぎたから、頭が痛くなっちゃった。	
\\	すぎる
\\	あのサッカー選手が上手すぎたんだ。	
\\	すぎる
\\	あなたはおそすぎるわ。	
\\	すぎる
\\	この問題は簡単すぎます。	
\\	すぎる
\\	ここから新宿駅まで歩いてどれぐらいかかりますか。	
\\	から
\\	生まれた時から東京に住んでいる。	
\\	から
\\	この携帯電話は彼氏からもらった。	
\\	から
\\	これからもよろしくお願いします。	
\\	から
\\	パパはアイパッドをわたしから取り上げた。	
\\	から
\\	彼はやっと地獄から逃げ出した。	
\\	から
\\	「硫黄島からの手紙」という映画を見たことがありますか?	
\\	から
\\	テストの前に一生懸命勉強しました。	
\\	まえに
\\	彼の目の前に不機嫌な顔の男が立っていた。	
\\	まえに
\\	食事の前に手を洗ってください。	
\\	まえに
\\	男の子がケージの前に来て、じっと獅子を見つめた。	
\\	まえに
\\	日本へ行く前に日本語を勉強しました。	
\\	まえに
\\	学校に行く前に朝ごはんを食べなさい。	
\\	まえに
\\	おれの家の前に車が止まっている。	
\\	まえに
\\	私の目の前に彼が突然現れた。	
\\	まえに
\\	寿司を食べたことがありますか?	
\\	たことがある
\\	「幽霊を見たことがある」と彼が言った。	
\\	たことがある
\\	有名人に会ったことがある?	
\\	たことがある
\\	あの人は誰?何度も見たことがあるよ。	
\\	たことがある
\\	私がこの本を数え切れないほど読んだことがある。	
\\	たことがある
\\	先生に叱られたことがある。	
\\	たことがある
\\	日本語能力試験を受けたことがありますか?	
\\	たことがある
\\	騙されたことがありますか?	
\\	たことがある
\\	この話は聞いたことがある。	
\\	たことがある
\\	横浜八景島に行ったことがあります。	
\\	たことがある
\\	うそをついたことがある?	
\\	たことがある
\\	私の車が古すぎるから、新しいのを買いたい。	
\\	たい
\\	お酒を飲みたいです。一緒に居酒屋に行きませんか?	
\\	たい
\\	狭い部屋に住みたくない。	
\\	たい
\\	帰国したい。両親に会いたい。	
\\	たい
\\	うちの息子が大学に入りたくないと言いました。	
\\	たい
\\	将来有名人になりたいです。	
\\	たい
\\	日本語が上手く話せたいです。	
\\	たい
\\	あなたの言うことを聞きたくない。	
\\	たい
\\	彼女と結婚したいです。	
\\	たい
\\	寒いから、外に出たくない。	
\\	たい
\\	宿題をしてからテレビを見ました。	
\\	てから
\\	日本に来てからずっと仕事を探しています。	
\\	てから
\\	私の話を聞いてから、アドバスお願いします。	
\\	てから
\\	彼は親と離れてから独立になりました。	
\\	てから
\\	学校が終わってから家に帰りなさい。	
\\	てから
\\	ご飯を食べてから部屋を掃除してください。	
\\	てから
\\	彼女はここに来てからずっと寝ている。	
\\	てから
\\	休暇が始まってから帰国しました。	
\\	てから
\\	な
\\	い 
\\	くなる	
\\	になる	
\\	い
\\	になる	なる
\\	最近うちの娘が頑固になってきた。	
\\	なる
\\	孫悟空のように強くなりたい。	
\\	なる
\\	15時間も寝たので、頭が痛くなった。	
\\	なる
\\	彼女が美しくなりましたね。	
\\	なる
\\	病気にならないように毎日運動する。	
\\	なる
\\	暗くなる前に家に帰りたい。	
\\	なる
\\	家族のみんなが健康になれるように祈っています。	
\\	なる
\\	今日は寒くなるようだ。	
\\	なる
\\	彼の顔は真っ赤になった。	
\\	なる
\\	どうしてそうなったのか、わたしにはわかりません。	
\\	なる
\\	コーラを飲むのが好きです。	
\\	のがすき
\\	猫がネズミを追いかけるのが好きだ。	
\\	のがすき
\\	妻は不平を言うのが好き。	
\\	のがすき
\\	少年漫画を読むのが好きだ。	
\\	のがすき
\\	母は料理をするのが好きです。	
\\	のがすき
\\	彼女はみんなの注目を集めるのが好きなのだ。	
\\	のがすき
\\	あの子は自分の面倒を見てくれる親分にくっついているのが好きだ。	
\\	のがすき
\\	あいつはきれいな女性と付き合うのが好きだ。	
\\	のがすき
\\	サッカーするのが好きです。	
\\	のがすき
\\	日本語学校で日本語を勉強しています。	
\\	ている
\\	何を考えていますか?	
\\	ている
\\	姉は家を掃除しています。	
\\	ている
\\	雨が降っている。	
\\	ている
\\	働いているので、後で電話します。	
\\	ている
\\	何言ってるの?	
\\	ている
\\	私は教室で先生と話しています。	
\\	ている
\\	赤ちゃんはミルクを飲んでいます。	
\\	ている
\\	犬が走っている。	
\\	ている
\\	椅子がないので、みんなは床に座っている。	
\\	ている
\\	渋谷駅西口で待っています。	
\\	ている
\\	どっちのチームが勝っていますか?	
\\	ている
\\	彼女よりあなたの方がきれいですよ。	
\\	より 
\\	のほうが 
\\	ビールを飲むよりミルクを飲むの方が健康にいいですよ。	
\\	より 
\\	のほうが 
\\	英語より日本語の方が難しいと思う。	
\\	より 
\\	のほうが 
\\	サスケよりナルトの方が強い。	
\\	より 
\\	のほうが 
\\	日本よりロシアの方が寒い。	
\\	より 
\\	のほうが 
\\	他人に頼るより自分で問題を解決する方がいいよ。	
\\	より 
\\	のほうが 
\\	兄より僕の方が背が高い。	
\\	より 
\\	のほうが 
\\	給料の低いより、給料の高い仕事の方がいいじゃない?	
\\	より 
\\	のほうが 
\\	きみより彼の方が頭がいいよ。	
\\	より 
\\	のほうが 
\\	家族より仕事の方が大切だと思う人と結婚したくない。	
\\	より 
\\	のほうが 
\\	チョコレートを食べてもいいですか?	
\\	てもいい
\\	ここに座ってもいいかな。	
\\	てもいい
\\	エアコンをつけてもいいですか?	
\\	てもいい
\\	誰でもいいから恋愛したい。	
\\	てもいい
\\	写真を取ってもいいかい?	
\\	てもいい
\\	水をもらってもいいですか?	
\\	てもいい
\\	ヒントをあげてもいいですよ。	
\\	てもいい
\\	それじゃ、わたし、行ってもいいですか?	
\\	てもいい
\\	この指輪がほしいから、高くてもいいよ。	
\\	てもいい
\\	部屋が狭くてもいい。	
\\	てもいい
\\	友達と一緒に焼肉やキムチを食べました。	
\\	や
\\	ブックオフで
\\	や本を売った。	
\\	や
\\	大学生やサラリーマンがマラソン大会に参加しています。	
\\	や
\\	スーパーでりんごや野菜を買いました。	
\\	や
\\	冬休みは家族でイタリアやイギリスへ行きました。	
\\	や
\\	国へ帰ったら、会社をつくるつもりだ。	
\\	つもりだ
\\	あなたたちも行くつもりですか?	
\\	つもりだ
\\	おれを破滅させるつもりか。	
\\	つもりだ
\\	いったい全体あなたはどういうつもりなんですか。	
\\	つもりだ
\\	明日横浜へ行くつもりです。	
\\	つもりだ
\\	気に障るようなことを言うつもりはなかったんだ。	
\\	つもりだ
\\	プレッシャーをかけるつもりはない。	
\\	つもりだ
\\	ばれたらどうするつもりなの?	
\\	つもりだ
\\	ママ、お菓子とミルクを買ってくれる?	
\\	と
\\	漫画と小説を読みます。	
\\	と
\\	中国語と韓国語を勉強しています。	
\\	と
\\	犬と猫が好きです。	
\\	と
\\	コーヒーと紅茶がございますが、どちらになさいますか?	
\\	と
\\	図書館では大きい声を出してはいけません。	
\\	てはいけない
\\	うそをついてはいけない。	
\\	てはいけない
\\	道でタバコを吸ってはいけません。	
\\	てはいけない
\\	授業中携帯電話を使ってはいけない。	
\\	てはいけない
\\	高望みしてはいけない。	
\\	てはいけない
\\	お客さんに失礼してはいけない。	
\\	てはいけない
\\	両親のお金を盗んではいけない。	
\\	てはいけない
\\	廊下を走ってはいけない。先生に叱られるよ。	
\\	てはいけない
\\	今日テストがあるから、遅刻してはいけない。	
\\	てはいけない
\\	あの男には逆らってはいけないと思う。	
\\	てはいけない
\\	私は少女漫画を読むのが一番好きだ。	
\\	いちばん
\\	一番好きな映画は何ですか?	
\\	いちばん
\\	歴史は一番退屈な科目だと思わない?	
\\	いちばん
\\	私はあなたのことが一番好きです。	
\\	いちばん
\\	ゴーストが一番怖いです。	
\\	いちばん
\\	彼は漢字を覚えるのが一番上手だ。	
\\	いちばん
\\	のチッキンが一番おいしいと思う。	
\\	いちばん
\\	一番嫌いな男性のタイプは何ですか?	
\\	いちばん
\\	今まで読んだ本の中で、レ・ミゼラブルは一番面白かった。	
\\	いちばん
\\	この世で一番恐ろしいものはなんだろう?	
\\	いちばん
\\	明日東京ディズニーランドに行きませんか?	
\\	ませんか
\\	この問題について説明していただけませんか?	
\\	ませんか
\\	僕と踊りませんか?	
\\	ませんか
\\	私と一緒に晩ごはんを食べませんか?	
\\	ませんか
\\	池袋駅でお会いしませんか?	
\\	ませんか
\\	先生、医務室に行かせてもらえませんか?	
\\	ませんか
\\	私のパーティーに参加しませんか?	
\\	ませんか
\\	僕たちと一緒に座りませんか?	
\\	ませんか
\\	私の話を聞いてくれませんか?	
\\	ませんか
\\	寒いから、窓を閉めてください。	
\\	から
\\	宿題がいっぱいあるから、遊べない。	
\\	から
\\	天気がいいから、散歩しましょう。	
\\	から
\\	ここでちょっと待ってね。すぐ戻るから。	
\\	から
\\	財布を忘れたから、電車の切符を買えなかった。	
\\	から
\\	一生懸命勉強したから、大学入学試験に合格した。	
\\	から
\\	今忙しいから、後で電話する。	
\\	から
\\	来年アメリカへ行くから、英語を勉強している。	
\\	から
\\	危ないから気をつけてね。	
\\	から
\\	大学院に入ろうと思っています。	
\\	に
\\	どうやって学校に通っていますか?	
\\	に
\\	ここに名前を書いてください。	
\\	に
\\	かぞくと一緒に東京に住んでいます。	
\\	に
\\	昼ごはんの後で友達とお茶を飲みました。	
\\	あとで
\\	薬を飲んだ後で、一時間ほど寝ました。	
\\	あとで
\\	今度授業の後で映画に行かない?	
\\	あとで
\\	説明を聞いた後で、作業を始めましょう。	
\\	あとで
\\	パーティーが終わった後でゴミを集めました。	
\\	あとで
\\	「どうしてふとるのかしら」数分黙った後で、彼女がうめいた。	
\\	あとで
\\	彼が父親の名前だけでぎくりとした。	
\\	だけで
\\	彼は片足だけで立っている。	
\\	だけで
\\	一目見ただけで、俺に必要なことはすべてわかった。	
\\	だけで
\\	二人だけで少し話がしたい。	
\\	だけで
\\	見回しただけで、彼女の姿がないことがわかった。	
\\	だけで
\\	こちらは社長の山田でございます。	
\\	です)	
\\	でございます
\\	もちろんでございます。	
\\	です)	
\\	でございます
\\	こちらは前のところより静かでございます。	
\\	です)	
\\	でございます
\\	あれはヒントだったのでございます。	
\\	です)	
\\	でございます
\\	暇だからゲームでもしましょう。	
\\	でも
\\	疲れたから、コーヒーでも飲みましょう。	
\\	でも
\\	連休はどうする?郊外へでも行ってみようか。	
\\	でも
\\	夕食までもう少し時間があるから、パンでも食べなさい。	
\\	でも
\\	一緒にお茶でもいかがですか。	
\\	でも
\\	泣いたようですね。悪いことでもあったのですか。	
\\	でも
\\	よく運動してください。病気にでもなったら困るですから。	
\\	でも
\\	隣の家でピアノの音がしている。	
\\	がする
\\	いいにおいがする。	
\\	がする
\\	このピザは変な味がします。	
\\	がする
\\	このボタンを押せば、ドアが開きます。	
\\	(ば 
\\	ば
\\	毎日練習をすれば、上手になりますよ。	
\\	(ば 
\\	ば
\\	先生を待ってるふりをすればいいじゃないか。	
\\	(ば 
\\	ば
\\	今すぐ出発すれば間に合うでしょう。	
\\	(ば 
\\	ば
\\	あなたが行けば、わたしも行きます。	
\\	(ば 
\\	ば
\\	数週間もすれば、みんな忘れるよ。	
\\	(ば 
\\	ば
\\	あの美術館へ行けば、日本の古い絵を見ることができます。	
\\	(ば 
\\	ば
\\	このケーキは冷たくなければ、おいしくありません。	
\\	(ば 
\\	ば
\\	これ以上事を起こせば、政府とまずいことになってしまうよ。	
\\	(ば 
\\	ば
\\	イタリア大使館へ行きたいんですが、どう行けばいいですか。	
\\	(ば 
\\	ば
\\	わたしは兄弟がございません。	
\\	いる/ある)	
\\	ございます
\\	ネクタイ売り場は4階にございます。	
\\	いる/ある)	
\\	ございます
\\	どうぞこちらへ。お荷物の方は、もうお部屋に上げてございます。	
\\	いる/ある)	
\\	ございます
\\	まだ時間がございますので、どうぞごゆっくりお召し上がりください。	
\\	いる/ある)	
\\	ございます
\\	あちらに申込書がございます。	
\\	いる/ある)	
\\	ございます
\\	何かご用がございましたら、どうぞいつでもご遠慮なく。	
\\	いる/ある)	
\\	ございます
\\	お荷物はこちらにございます。	
\\	いる/ある)	
\\	ございます
\\	そんな難しい問題は小学生ができるはずがない。	
\\	はずがない
\\	あの人は入院しているので、明日の旅行に来るはずがない。	
\\	はずがない
\\	あの頑丈な建物が壊れるはずがありません。	
\\	はずがない
\\	やさしい山下さんがこんなひどいことをするはずがない。	
\\	はずがない
\\	田中さんが、病気の子供を一人で家においておくはずがない。	
\\	はずがない
\\	タイタニック号は「沈むはずがない」と言われた。	
\\	はずがない
\\	みんなは階下にいるのだ。部屋の中には、誰もいない。いるはずがない。	
\\	はずがない
\\	彼が今ここにいるはずがない。	
\\	はずがない
\\	彼女がわざと危険なものを僕に送ってくるはずがない。	
\\	はずがない
\\	彼は死ぬもんか。死ぬはずがない。	
\\	はずがない
\\	鶏肉はあまり好きじゃないです。	
\\	あまり~ない
\\	わたしはあまり期待していなかった。	
\\	あまり~ない
\\	あまりたくさんお酒を飲まないほうがいいですよ。	
\\	あまり~ない
\\	僕もチェスはあまり上手じゃない。	
\\	あまり~ない
\\	あまり眠くないです。	
\\	あまり~ない
\\	あまり勉強しなかったので、成績が悪くなった。	
\\	あまり~ない
\\	なんで彼がそんなことをする必要があるんだ?	
\\	ひつようがある
\\	彼女に納得させる必要がある。	
\\	ひつようがある
\\	記憶を取り戻す必要があるよ。	
\\	ひつようがある
\\	僕、大丈夫です。なんにもする必要がありません。	
\\	ひつようがある
\\	新人選手を教育する必要がある。	
\\	ひつようがある
\\	何が起こったのか、わたしは知る必要がある。	
\\	ひつようがある
\\	図書館に行く必要があります。	
\\	ひつようがある
\\	僕にはもう必要がないだろう。	
\\	ひつようがある
\\	わたしは、ナルトが知る必要があること以外に、この子に話してやるつもりはないよ。	
\\	ひつようがある
\\	この仕事はわたしがいたします。	
\\	する)	
\\	いたす
\\	お席にご案内いたします。	
\\	する)	
\\	いたす
\\	荷物はわたしが来週の月曜日にお届けいたします。	
\\	する)	
\\	いたす
\\	わたしは木村先生を6時までここでお待ちいたします。	
\\	する)	
\\	いたす
\\	必要な教材のリストを同封いたします。	
\\	する)	
\\	いたす
\\	新学期は九月五日に始まることをお知らせいたします。	
\\	する)	
\\	いたす
\\	飛行機の予約はわたしがいたします。	
\\	する)	
\\	いたす
\\	5千円からお預かりいたします。3千円のおつりでございます。	
\\	する)	
\\	いたす
\\	失礼いたします。	
\\	する)	
\\	いたす
\\	買い物を暗くならないうちにお済ませくださいますようお勧めいたします。	
\\	する)	
\\	いたす
\\	彼がこのプレゼントを喜ぶかどうか分かりません。	
\\	かどうか
\\	名前を書いたかどうか、もう一度チェックしてください。	
\\	かどうか
\\	弱いかどうかとはまったく関係ない。	
\\	かどうか
\\	田中さんが大学を卒業できたかどうか知っていますか。	
\\	かどうか
\\	やるべきことはしっかりわかった。それができるかどうかが問題だ。	
\\	かどうか
\\	行くかどうか後で電話で知らせます。	
\\	かどうか
\\	両親が死んだ時、自分が車の中にいたかどうかさえ思い出せない。	
\\	かどうか
\\	この計画は成功できるかどうか心配です。	
\\	かどうか
\\	おいしいかどうか食べなければわからない。	
\\	かどうか
\\	犯人が捕まったかどうか、知ってる?	
\\	かどうか
\\	怖がらないでください。	
\\	がる
\\	彼は人の失敗を面白がっている。	
\\	がる
\\	この猫、誰もほしがる人がいなかった。	
\\	がる
\\	恥ずかしがらないで、前に出て来てください。	
\\	がる
\\	息子は今の仕事をいやがっている。	
\\	がる
\\	雑誌のほうは彼に渡してくれ。おもしろがるだろう。	
\\	がる
\\	彼女はその話を聞いてちょっと恥ずかしがりました。	
\\	がる
\\	僕が何を怖がるっていうんだ?	
\\	がる
\\	子どもはみんなおもちゃを欲しがるもんだ。	
\\	がる
\\	この子がどこに行ったか知りたがるでしょう。	
\\	がる
\\	雨が急に降り出した。	
\\	だす
\\	「彼女はどうしてこんなふうになったの?」彼がゆっくりと切り出した。	
\\	だす
\\	母は5時ごろから夕食を作り出した。	
\\	だす
\\	「待て」ジョンがもう歩き出したとき、ブラッドが突然叫んだ。	
\\	だす
\\	あの赤ちゃんが泣き出すと、なかなか泣き止まない。	
\\	だす
\\	新宿駅に着く直前、彼はあることを思い出した。	
\\	だす
\\	非常ベルが急に鳴り出した。	
\\	だす
\\	あいつは今夜逃げ出したよ。	
\\	だす
\\	彼の話を聞いて、つい笑い出してしまった。	
\\	だす
\\	この小説は読み出すとやめられない。	
\\	だす
\\	お金が必要です。	
\\	ひつよう
\\	サッカーを学ぶ場所が必要です。	
\\	ひつよう
\\	少し考える時間が必要です。	
\\	ひつよう
\\	試合が近い。練習が必要だよ!	
\\	ひつよう
\\	夫婦はお互いに理解しあうことが必要だ。	
\\	ひつよう
\\	何と言えばいいかしら…	
\\	かしら
\\	早く来てくれないかしら。	
\\	かしら
\\	おかしいわね。どうしてあんたに気がつかなかったのかしら?	
\\	かしら
\\	学校まで連れて行ったらどうかしら…	
\\	かしら
\\	パパ、それほんとう? ちょっとまちがっているのではないかしら?	
\\	かしら
\\	危険じゃないかしら?	
\\	かしら
\\	でも、ママが何て言うかしら?	
\\	かしら
\\	キムさんの家は大学から遠いんじゃないか。	
\\	じゃないか
\\	いいじゃないか。	
\\	じゃないか
\\	顔色が悪いね。疲れているんじゃないか。	
\\	じゃないか
\\	だから言ったじゃないか!	
\\	じゃないか
\\	なんだ、空っぽじゃないか。	
\\	じゃないか
\\	あれは全財産だったじゃないか!	
\\	じゃないか
\\	ダヴィッドがわざと話題を変えたんじゃないか、とトムは勘ぐった。	
\\	じゃないか
\\	なかなかやるじゃないか。	
\\	じゃないか
\\	おれとダンスパーティーに行くかい?	
\\	かい
\\	田中を近ごろ見かけたかい?	
\\	かい
\\	心配事があるのかい?	
\\	かい
\\	大丈夫かい?	
\\	かい
\\	おれのせいだと思うかい?	
\\	かい
\\	話してくれるかい?いったい何があったんだ?	
\\	かい
\\	まだ終わってないのかい?	
\\	かい
\\	あそこに敵がいるのが見えないのかい?	
\\	かい
\\	このすばらしい申し出を僕が拒絶した場合はいかがなされます?	
\\	場合は
\\	スズキが言うには、カエデの場合は行方不明になっても、まったくおかしくはない。	
\\	場合は
\\	あの人の場合はそんなことはめったにないのではないですか。	
\\	場合は
\\	今後もし非行があった場合は、それで終りということになりましょう。	
\\	場合は
\\	この病院は日曜と祝日は休みだが、緊急の場合は診てもらえる。	
\\	場合は
\\	わたしはパソコンが欲しいです。	
\\	ほしい
\\	僕は少し助けが欲しい。	
\\	ほしい
\\	赤ちゃんが乳を欲しがって泣いている。	
\\	ほしい
\\	もっとベーコンが欲しいよ。	
\\	ほしい
\\	あなたの約束の言葉が欲しい!	
\\	ほしい
\\	寒くなったので、そろそろ暖房が欲しいです。	
\\	ほしい
\\	どうしてその本が欲しいの?	
\\	ほしい
\\	お前は何が欲しいのか?	
\\	ほしい
\\	彼は夕食のことなどすっかり忘れた。	
\\	など
\\	兄の部屋にはラジオやテレビなどがあります。	
\\	など
\\	これまで図書館で見つからないことなどなかったのだ。	
\\	など
\\	かばんの中に財布や鍵などがあります。	
\\	など
\\	みんなお祭り気分で盛り上がり、ほかの人が何をしているかなど気にも留めない。	
\\	など
\\	わたし、あなたの部屋に近づいたことなどありません。	
\\	など
\\	おまえはわたしの息子などではない!	
\\	など
\\	あいつの話など、一言も信じてはおられないでしょうな?	
\\	など
\\	この本屋に子供向けの漫画があるかもしれません。	
\\	かもしれない
\\	彼は邸宅へ戻ったのかもしれないわ。	
\\	かもしれない
\\	駅はあっちかもしれないね。	
\\	かもしれない
\\	彼女はいまにも戻ってくるかもしれない。	
\\	かもしれない
\\	鈴木さんは今日も学校に来なかった。病気かもしれない。	
\\	かもしれない
\\	なるほど、きみの言うとおりかもしれない。	
\\	かもしれない
\\	財布がない。駅で落としたかもしれない。	
\\	かもしれない
\\	彼は危険な人間であるかもしれない。	
\\	かもしれない
\\	途中で雨が降るかもしれないから、傘を持っていきましょう。	
\\	かもしれない
\\	今までのことは夢だったのかもしれない。	
\\	かもしれない
\\	この博物館では写真を撮ることができません。	
\\	ことができる
\\	昨日動物園に行ったら、先月生まれたばかりのライオンの赤ちゃんを見ることができました。	
\\	ことができる
\\	そのすきに彼女はこっそりシーツで涙を拭うことができた。	
\\	ことができる
\\	車の運転ができますか?	
\\	ことができる
\\	今日は何時まで本を借りることができますか。	
\\	ことができる
\\	彼はやっとみんなから離れることができた。	
\\	ことができる
\\	その映画館で日本の古い映画を見ることができます。	
\\	ことができる
\\	図書館でこの町の歴史を調べることができます。	
\\	ことができる
\\	このナイフはどんなものでも切ることができます。	
\\	ことができる
\\	彼には、わたしには決して持つことができない力があるよ。	
\\	ことができる
\\	今夜はテレビを見ないで勉強します。	
\\	ないで
\\	自分を危険に晒さないで。	
\\	ないで
\\	昨日何も着ないで寝た。	
\\	ないで
\\	バカなこと言わないで。	
\\	ないで
\\	「いいえ」彼女は僕の顔を見ないで言った。	
\\	ないで
\\	まだ諦めないで。	
\\	ないで
\\	彼は傘を持たないで出て行った。	
\\	ないで
\\	ここに立って、なんにもしないで見てるだけなのかい?	
\\	ないで
\\	もう、それ以上言わないで。	
\\	ないで
\\	弟は朝ごはんを食べないで、学校へ行きました。	
\\	ないで
\\	先生、何時まで学校にいらっしゃいますか。	
\\	いる, 来る, 行く)	
\\	いらっしゃる
\\	すみません、ご主人はいらっしゃいますか。	
\\	いる, 来る, 行く)	
\\	いらっしゃる
\\	明日のパーティーにいらっしゃいますか。	
\\	いる, 来る, 行く)	
\\	いらっしゃる
\\	申し訳ございません。ちょっと受付までいらっしゃてください。	
\\	いる, 来る, 行く)	
\\	いらっしゃる
\\	先生はあなたを信じていらっしゃる。	
\\	いる, 来る, 行く)	
\\	いらっしゃる
\\	憶えていらっしゃると思いますが、わたしはジョン・スミスです。	
\\	いる, 来る, 行く)	
\\	いらっしゃる
\\	校長先生は図書館にいらっしゃいます。	
\\	いる, 来る, 行く)	
\\	いらっしゃる
\\	お菓子を食べながら映画を見ました。	
\\	ながら
\\	音楽を聞きながら、運動します。	
\\	ながら
\\	歩きながら話しましょう。	
\\	ながら
\\	「さあ、どうする?」彼女が周りを見回しながら囁いた。	
\\	ながら
\\	アルバイトをしながら、弁護士の資格の勉強をしている。	
\\	ながら
\\	左右を見ながら横断しました。	
\\	ながら
\\	どこに行こう?彼は立ち止まり、ドキドキしながら考えた。	
\\	ながら
\\	田中さんはお茶を飲みながらしばらく旅行の話をした。	
\\	ながら
\\	彼はその女の人が泣いているのに気づいた。ほほえみながら、泣いている。	
\\	ながら
\\	このごろ寒くなってきた。	
\\	ごろ
\\	今日は夜9時ごろ帰ります。	
\\	ごろ
\\	映画は何時ごろ終わりますか。	
\\	ごろ
\\	山田さんと昼ごろ駅で別れました。	
\\	ごろ
\\	今日六時ごろ、お茶を飲みに来ませんか。	
\\	ごろ
\\	みんな5時ごろ来ると言っていました。	
\\	ごろ
\\	彼いまごろ荷物をまとめておるよ。	
\\	ごろ
\\	今度の試合は絶対に勝たなくてはいけない。	
\\	なくてはいけない/なくてはならない
\\	寝る前に必ず歯を磨かなくてはなりません。	
\\	なくてはいけない/なくてはならない
\\	早くゲームを終らせなくてはならない。	
\\	なくてはいけない/なくてはならない
\\	嫌でもテストは受けなくてはいけません。	
\\	なくてはいけない/なくてはならない
\\	皆さんが入る寮を決めなくてはなりません。	
\\	なくてはいけない/なくてはならない
\\	明日までにレポートを提出しなくてはならない。	
\\	なくてはいけない/なくてはならない
\\	星の名前や惑星の動きを勉強しなくてはならなかった。	
\\	なくてはいけない/なくてはならない
\\	宿題は必ずしなくてはいけません。	
\\	なくてはいけない/なくてはならない
\\	彼を殺すのはわたしでなくてはならない。	
\\	なくてはいけない/なくてはならない
\\	生花を習いはじめて、もう半年経った。	
\\	はじめる
\\	みんなは「いただきます」と言って食べはじめました。	
\\	はじめる
\\	彼は笛を唇にあてて吹きはじめた。	
\\	はじめる
\\	わたしは4歳から小学校に通いはじめた。	
\\	はじめる
\\	午前から雪がふりはじめました。	
\\	はじめる
\\	タバコを吸いはじめたのは、20歳のときでした。	
\\	はじめる
\\	全員がいっせいに大声で話しはじめた。	
\\	はじめる
\\	急に空が暗くなって風がふきはじめた。	
\\	はじめる
\\	このごろ甘いものをよく食べるので、太りはじめた。	
\\	はじめる
\\	彼女はゆっくりと封筒を開きはじめた。	
\\	はじめる
\\	田中さんは兄弟の中で上から2番目です。	
\\	のなかで
\\	三つの中で一番大きいボタンを押してください。	
\\	のなかで
\\	今朝、電車の中で足をふまれました。	
\\	のなかで
\\	言語の中で日本語が一番難しいと思う。	
\\	のなかで
\\	部屋の中で動物を飼ってはいけません。	
\\	のなかで
\\	笑いが、彼の耳の中で鳴り響いた。	
\\	のなかで
\\	辞書は図書館の中で使ってください。	
\\	のなかで
\\	ちょっと待ちなさい。	
\\	なさい
\\	弟のもっといいお手本になりなさい!	
\\	なさい
\\	食べなさい、ジョン。座ったまま死んでるような顔だよ。	
\\	なさい
\\	取り越し苦労はおやめなさい。	
\\	なさい
\\	アイスクリームは家にあるでしょう。家で食べなさい。	
\\	なさい
\\	いいかげんにおやめなさい。	
\\	なさい
\\	テレビ、消しなさいよ。	
\\	なさい
\\	トラブルに巻き込まれないよう十分気をつけなさい。	
\\	なさい
\\	二人とも私のところにおいでなさい!	
\\	なさい
\\	彼女の笑顔は太陽のように明るく輝いている。	
\\	のように 
\\	のような
\\	彼は突然、傷ついた犬のような声でワオーンと泣き出した。	
\\	のように 
\\	のような
\\	母ははじめて飛行機に乗って、子供のように喜んだ。	
\\	のように 
\\	のような
\\	彼は山のような菓子の箱を片づけている。	
\\	のように 
\\	のような
\\	みんなが子供のように元気に歌い始めた。	
\\	のように 
\\	のような
\\	彼らのテーブルから嵐のような歓声と足を踏み鳴らす音が上がった。	
\\	のように 
\\	のような
\\	この笛はこのように持ちます。	
\\	のように 
\\	のような
\\	彼女は山のように本を抱えた。	
\\	のように 
\\	のような
\\	リンさんは京都のような町へ行きたいって言っていました。	
\\	のように 
\\	のような
\\	彼女はランプのような目でわたし達を見たが、何もしなかった。	
\\	のように 
\\	のような
\\	みんながたくさん飲みましたから、もうお酒はありません。	
\\	もう
\\	アルコールはもう飲まないことにした。	
\\	もう
\\	もう我慢できない。	
\\	もう
\\	もうすこし小さいボールを貸してください。	
\\	もう
\\	僕たち、もう知ってた。	
\\	もう
\\	もう行っていいよ。	
\\	もう
\\	牛乳は全部飲みました。もうありません。	
\\	もう
\\	どの高校に入るかもう決めましたか。	
\\	もう
\\	料理を作るのに1時間かかります。	
\\	のに
\\	わたしたちは、店を始めるのにお金が必要だ。	
\\	のに
\\	パスポートを申請するのにいろいろな書類や写真を用意する必要がある。	
\\	のに
\\	やつを阻止するのに何をしてるんだ?	
\\	のに
\\	会社へ行くのに1時間かかります。	
\\	のに
\\	ここを人間が住むのにふさわしい場所にしている。	
\\	のに
\\	木を植えるのに水が必要だ。	
\\	のに
\\	彼にお目にかかるのには早すぎると思う?	
\\	のに
\\	この辞書は英単語の意味を調べるのに便利です。	
\\	のに
\\	それは証明しにくいことね。	
\\	にくい
\\	この質問には答えにくいです。	
\\	にくい
\\	手紙は湿っぽく、とても読みにくい。	
\\	にくい
\\	今勉強している文章は難しいから、覚えにくいです。	
\\	にくい
\\	東京は物価が高くて、住みにくいです。	
\\	にくい
\\	このペンはインクが時々出なくなって書きにくい。	
\\	にくい
\\	夜は暗くて歩いている人が見えにくいので、注意して運転します。	
\\	にくい
\\	この本は分かりにくいけれど、面白いです。	
\\	にくい
\\	この薬はとてもにがくての飲みにくいです。	
\\	にくい
\\	この傘は重くて、差しにくいです。	
\\	にくい
\\	な
\\	に見える	
\\	-て
\\	見える	
\\	い
\\	(-い) 
\\	く見える	にみえる
\\	先生が老いて見えるが、実際は50歳もないそうだ。	
\\	にみえる
\\	彼女は若く見えるので、学生だと思ったら、実は先生だった。	
\\	にみえる
\\	彼女は顔がきれいに見えるが、それは化粧したあとの姿だけだ。	
\\	にみえる
\\	雨の日は、木のみどりが美しく見える。	
\\	にみえる
\\	どうぞごゆっくりお召し上がりください。	
\\	お~ください
\\	少々お待ちください。	
\\	お~ください
\\	さぞお疲れになったことと存じます。どうぞしばらくお休みください。	
\\	お~ください
\\	わたくしをお雇いください。	
\\	お~ください
\\	ここがちょっと分からないんですが、ご説明ください。	
\\	お~ください
\\	あなたを悩ます事情というのを、どうぞ詳しくおきかせください。	
\\	お~ください
\\	お母さま、お父さまをお止めください。	
\\	お~ください
\\	小林先生はもうお帰りになりました。	
\\	お~になる
\\	今日の新聞をお読みになりましたか。	
\\	お~になる
\\	乗車券をお求めになるかたは、どうぞこちらへ。	
\\	お~になる
\\	お母さんはいつお帰りになりますか。	
\\	お~になる
\\	お電話でご注文になれば、すぐお届けいたします。	
\\	お~になる
\\	食べおわったら片付けてください。	
\\	おわる
\\	彼らは門を通過しおわるまで目を閉じていた。	
\\	おわる
\\	田中くんが書きおわるまで、ここでお待ちください。	
\\	おわる
\\	聞きおわると、彼は愕然としていた。	
\\	おわる
\\	わたしは学校から帰ってから小説を読みはじめた。3時間かかって読みおわった。	
\\	おわる
\\	学生達は黒板に書かれた宿題も写しおわっていた。	
\\	おわる
\\	新聞を読みおわったら寝なさい。	
\\	おわる
\\	彼女はトランクを詰めおわった。	
\\	おわる
\\	山田さんは、ご飯を食べおわると、すぐに勉強を始めました。	
\\	おわる
\\	話しおわれば、気持がすっきりするような予感がした。	
\\	おわる
\\	電灯が付いていない。田中さんはもう寝たみたいだ。	
\\	みたい
\\	父は子供みたいです。	
\\	みたい
\\	わたし、記憶喪失になったみたい。	
\\	みたい
\\	彼女の心は氷みたいに冷たい。	
\\	みたい
\\	なんだかそれが悪いことみたいな言い方ね。	
\\	みたい
\\	ドアをノックしたが、返事がなかったです。隣の家には誰もいないみたいです。	
\\	みたい
\\	この街は静かみたいよね。	
\\	みたい
\\	バカみたいに見えるのはわかってる。	
\\	みたい
\\	あなたみたいに日本語が上手に話したいです。	
\\	みたい
\\	こんな状況になってるのを大いに楽しんでるみたいじゃないか?	
\\	みたい
\\	彼女は財布を落としたことに気がつかなかった。	
\\	に気がつく
\\	姉はそれに気がつかなかった。	
\\	に気がつく
\\	ピーターは、彼女が彼の存在に気がついたのがわかるまで、この新しく出現した天使を、ちらちら横目で拝んでいた。	
\\	に気がつく
\\	そのとき彼は、彼女が泣いているのに気がついた。	
\\	に気がつく
\\	わたしは幼少のころから自分の性格が矛盾の多いものであることに気がついていた。	
\\	に気がつく
\\	注意して彼女をみたとき、彼はそれがやはりおなじ前の娘だったことに気がついた。	
\\	に気がつく
\\	日本語を習いたいのなら、この学校が安くていいですよ。	
\\	なら
\\	明日雨なら洗濯をしません。	
\\	なら
\\	この家賃は安いですね。東京なら、こんなに安い家賃で借りられませんよ。	
\\	なら
\\	夜ならいつでも家にいます。	
\\	なら
\\	君なら友を助けにくると思った。	
\\	なら
\\	事件の真相を知っているのなら、わたしに教えてください。	
\\	なら
\\	明日ハイキングに行くなら、早く寝なさいね。	
\\	なら
\\	やるならいまだ。	
\\	なら
\\	わたしの兄は、パソコンのことなら何でもわかります。	
\\	なら
\\	彼女を殺したいのなら、僕も殺しなさい!	
\\	なら
\\	お飲み物は何になさいますか?	
\\	する)	
\\	なさる
\\	田中さん、土曜日のパーティーに出席なさいますか。	
\\	する)	
\\	なさる
\\	あなたのおかあさまが、わたしに書類を直送なさいました。	
\\	する)	
\\	なさる
\\	今年の夏休みはどうなさるおつもりですか。	
\\	する)	
\\	なさる
\\	それからどうなさいました?	
\\	する)	
\\	なさる
\\	いつご結婚なさいますか。	
\\	する)	
\\	なさる
\\	どうかなさいましたか?	
\\	する)	
\\	なさる
\\	今年の暑さは普通ではない。	
\\	さ
\\	彼女の優しさに感動させられた。	
\\	さ
\\	みんなは寒さに震えている。	
\\	さ
\\	子供たちの元気さを見て安心した。	
\\	さ
\\	こいつはなんも悪さはしねえ。	
\\	さ
\\	彼は凍るような静けさに耳をすませていた。	
\\	さ
\\	いろいろ探したのに見つからないんです。	
\\	のに
\\	ずっとあなたの家に行ってたのに…どうして何にも言ってくれなかったの?	
\\	のに
\\	たくさん砂糖を入れたのに、まだあまくありません。	
\\	のに
\\	あいつ、夜中までの任務のはずだったのになんで行っちまったんだい。	
\\	のに
\\	田中さんは病気なのに、仕事をしています。	
\\	のに
\\	まだそれほどの歳ではないのに、彼は少し病気のような顔をしていた。	
\\	のに
\\	あの女の子は体が小さいのに、速いわね。	
\\	のに
\\	三浦さんは風邪を引いているのに、プールで泳いでいます。	
\\	のに
\\	冗談のつもりだったのに…	
\\	のに
\\	山本さんは試験が近いのに、まだ全然勉強していません。	
\\	のに
\\	子供のとき、わたしは掃除が嫌いでしたが、よく母に家の掃除をさせられました。	
\\	させられる
\\	父は会社を止めさせられてたいへん困った。	
\\	させられる
\\	息子は退学させられた。	
\\	させられる
\\	妻に月給を出させられるので、自由に使うことができない。	
\\	させられる
\\	山田さんは病気が非常に重かったので、医者にすぐ入院させられました。	
\\	させられる
\\	毎日母に野菜を食べさせられる。	
\\	させられる
\\	「三年間も彼女に会っていません。」「それでも君はその女と結婚しているんだよ。」	
\\	それでも
\\	それでも彼女はわたしを愛してはいたのです。	
\\	それでも
\\	ハルカは唇が震え目に涙がうかんだ。それでも、無理に快活さをよそおって、それを抑えておしゃべりをつづけた。	
\\	それでも
\\	彼女は食べることに集中しようとした。それでも手は震え、顔は怒りで火照りはじめた。	
\\	それでも
\\	さすがの三浦もあせったらしい。	
\\	さすが
\\	君があれを言うと、さすがの彼もへたへたとなってしまったね。	
\\	さすが
\\	彼の顔はいつものように無表情であるが、さすがにマスクのような冷たい顔にも、いくらか感情のうごきが見えた。	
\\	さすが
\\	マツダは強情で、容易に本心を見せない男だったが、いま聞いたニュースにはさすがに心中の動揺を包みかねた。	
\\	さすが
\\	二人は来年結婚するらしいです。	
\\	らしい
\\	妹が部屋にいないらしい。	
\\	らしい
\\	あの人は会社を中止して大学にいくらしい。	
\\	らしい
\\	彼はなかなかいたずらの好きな人間らしい。	
\\	らしい
\\	学生らしくもっと勉強しなさい。	
\\	らしい
\\	明日は、いい天気らしい。	
\\	らしい
\\	ジュリアン氏という人間はひじょうに変わっている人物らしい。	
\\	らしい
\\	彼がついに目を覚ましたらしい。	
\\	らしい
\\	たしかに、彼女はいま旅から帰ったばかりらしい。	
\\	らしい
\\	母は弟にもバイオリンかピアノを習わせたがっているんだ。	
\\	たがる
\\	妹はその馬に乗りたがっている。	
\\	たがる
\\	うちの娘が留学したがる。	
\\	たがる
\\	「ここで少し休みますか。」「はい。子供たちも休みたがっています。」	
\\	たがる
\\	寮に住みたがっている学生が多い。	
\\	たがる
\\	彼が一刻も早く立ち去りたがっているような気がする。	
\\	たがる
\\	クリス・ヘムズワースと結婚したがる女子が多すぎる。	
\\	たがる
\\	赤ちゃんがミルクを飲みたがっている。	
\\	たがる
\\	手紙がどこから来たのか、やけに知りたがっていたぞ、彼は。	
\\	たがる
\\	両親は学校で何が起きたのか話したがってた。	
\\	たがる
\\	いま朝食を食べたばかりでしょう?	
\\	たばかり
\\	日本に来たばかりです。	
\\	たばかり
\\	二人ともパーティーに誘った女の子に、断られたばかりだ。	
\\	たばかり
\\	あんなことが起こったばかりじゃないか。	
\\	たばかり
\\	先生と話をしてきたばかりです。	
\\	たばかり
\\	さっき起きたばかりで、まだ眠いです。	
\\	たばかり
\\	あの二人は結婚したばかりです。	
\\	たばかり
\\	彼が宝がどこにあるか、いま教えてくれたばかりだ。	
\\	たばかり
\\	昼ごはんを食べたばかりなので、今は何も食べられない。	
\\	たばかり
\\	参加者はそんなに少ないのですか。	
\\	そんなに
\\	日本語はそんなに難しくないよ。	
\\	そんなに
\\	そんなにがっかりするな。もう一度やり直しなさい。	
\\	そんなに
\\	肉はそんなに好きではありません。	
\\	そんなに
\\	そんなに強くないよ、彼は。	
\\	そんなに
\\	彼はそんなに太っていません。	
\\	そんなに
\\	何がそんなに緊急なんだ?	
\\	そんなに
\\	なんでそんなにわたしのこと好きなの?	
\\	そんなに
\\	そんなに知りたいのなら教えるよ。	
\\	そんなに
\\	飛行機が飛んでいきました。	
\\	ていく
\\	これからも頑張っていきたいと思います。	
\\	ていく
\\	彼は気をシャンと取りなおし、家に入っていった。	
\\	ていく
\\	遠足が終わって、みんな帰っていった。	
\\	ていく
\\	彼女は玄関へと歩いていった。	
\\	ていく
\\	日本語学校を卒業しても、日本語の勉強を続けていくつもりだ。	
\\	ていく
\\	船がどんどん遠くへ離れていく。	
\\	ていく
\\	結婚してからも仕事を続けていくつもりです。	
\\	ていく
\\	車は夜の闇へと消えていった。	
\\	ていく
\\	今日雑誌が出る日だから、本屋でちょっと見ていく。	
\\	ていく
\\	明日あなたは何時に学校に来られますか。	
\\	られる
\\	わたしから隠れられるものか。	
\\	られる
\\	店内でタバコが吸えません。	
\\	られる
\\	わたしは中国語が少し話せます。	
\\	られる
\\	タバコがやめられない。	
\\	られる
\\	どのくらい本が借りられますか。	
\\	られる
\\	このバナナは黄色くて、食べられます。	
\\	られる
\\	こんな難しい問題は誰にも答えられない。	
\\	られる
\\	僕も楽しみに待っていられるようなものが何かなくちゃ…	
\\	られる
\\	わたしは父に新しいラジオを買ってあげました。	
\\	てあげる
\\	弟の面倒見てあげてね。	
\\	てあげる
\\	何?財布を忘れてきたの?じゃ、今日、わたしがごちそうをしてあげよう。	
\\	てあげる
\\	もう彼女を許してあげたら?	
\\	てあげる
\\	わたしは、ジョンさんに銀行へ行く道を教えてあげた。	
\\	てあげる
\\	助けてあげるべきじゃないかな…	
\\	てあげる
\\	わたし、彼らに知りたいことは何でも教えてあげたのよ。	
\\	てあげる
\\	君が説明してあげて。	
\\	てあげる
\\	先生のかばんを持ってさしあげました。	
\\	てあげる
\\	外に放せば?自由にしてあげれば?	
\\	てあげる
\\	昨日、弟は母に叱られました。	
\\	られる
\\	この家はわたしが使うために作られた。	
\\	られる
\\	花子ちゃんは犬にかまれました。	
\\	られる
\\	僕、ばあちゃんに育てられたんだ。	
\\	られる
\\	羊は虎に食われた。	
\\	られる
\\	わたしは毎朝父に早く起こされます。	
\\	られる
\\	彼らは何も盗られたものはなかったと主張した。	
\\	られる
\\	彼はあっけにとられた。	
\\	られる
\\	わたしは人の前で社長に褒められました。	
\\	られる
\\	捕まったらどうするの?姿を見られたら?	
\\	られる
\\	この小説を読んだら、きっとこの作家が好きになりますよ。	
\\	たら
\\	寂しかったら電話をください。	
\\	たら
\\	あなたの言うとおりよ。わたしだったら、きっとカンカンだわ。	
\\	たら
\\	上海へ来たら、ぜひ連絡してください。	
\\	たら
\\	国へ帰ったら、会社を作るつもりです。	
\\	たら
\\	職場で彼に出会ったら、彼女がどうするかは知らない。	
\\	たら
\\	小林さんに道を聞いたら、親切に地図をかいてくれました。	
\\	たら
\\	何度言ったらわかるの。	
\\	たら
\\	こちらにいらっしゃるきかいがあったら、ぜひおよりください。	
\\	たら
\\	わたしが合図したら、すぐに噴射してね。	
\\	たら
\\	友達が掃除を手伝ってくれた。	
\\	てくれる
\\	ねえ、映画の時間調べてくれた?	
\\	てくれる
\\	彼はいつでも僕たちにすばらしいアドバイスをしてくれた。	
\\	てくれる
\\	友達がわたしたちに旅行の写真を見せてくれました。	
\\	てくれる
\\	先生はわたしに英語を教えてくださいました。	
\\	てくれる
\\	あなたはわたしを二度も救ってくれた。	
\\	てくれる
\\	父が僕を育ててくれた。	
\\	てくれる
\\	彼女が作ってくれた日本の料理はとてもおいしかったです。	
\\	てくれる
\\	彼が、何が起こったのかを、みんなに話してくれた。	
\\	てくれる
\\	あの人形は姉が買ってくれたんです。	
\\	てくれる
\\	冬休みは本を読んだり、運動したりするつもりです。	
\\	たり~たり
\\	週末はいつも家でテレビを見たり、本を読んだりします。	
\\	たり~たり
\\	天気がよければ、泳いだり、魚をつったりすることができる。	
\\	たり~たり
\\	休みの日は土曜日だったり、日曜日だったりです。	
\\	たり~たり
\\	彼は同じ場所を行ったり来たりして、誰かを待っているようです。	
\\	たり~たり
\\	昨日みんなの前で歌ったり踊ったりしました。	
\\	たり~たり
\\	走った後ですぐにとまったり、座ったりしないでください。	
\\	たり~たり
\\	図書館で本を読んだり勉強したりします。	
\\	たり~たり
\\	わからないことがある時は先輩に教えてもらいます。	
\\	てもらう
\\	君にはすぐに出発してもらいたい。	
\\	てもらう
\\	自転車が壊れてしまったので、兄に直してもらった。	
\\	てもらう
\\	なんで彼だけが質問に答えてもらえるんだ?	
\\	てもらう
\\	辞書を忘れたので、友達に貸してもらった。	
\\	てもらう
\\	個人的に言わせてもらうと、彼女はわがままよ。	
\\	てもらう
\\	娘に駅まで送ってもらった。	
\\	てもらう
\\	あの人たち、静かにしてもらわないと、この子を起こしてしまうよ。	
\\	てもらう
\\	姉に傘を貸してもらいました。	
\\	てもらう
\\	君に助けてもらいたいことがあるよ。	
\\	てもらう
\\	彼はお土産を持ってきました。	
\\	てくる
\\	窓の灯が徐々に大きくなってきた。	
\\	てくる
\\	彼はベッドの上に這い戻ってきた。	
\\	てくる
\\	この伝統は百年も続いてきた。	
\\	てくる
\\	ちょっと手紙を出してきます。	
\\	てくる
\\	ボールはまた床に落ちてきた。	
\\	てくる
\\	彼女がキッチンに入ってきた。	
\\	てくる
\\	家の中から足音が響いてきた。	
\\	てくる
\\	台風が近づいてきます。	
\\	てくる
\\	わたしが日本語で書いた作文を見ていただけませんか。	
\\	ていただけませんか
\\	本日は早退させていただけませんでしょうか?	
\\	ていただけませんか
\\	すみませんが、ちょっと静かにしてもらえませんか。	
\\	ていただけませんか
\\	明日用事があるので、仕事を休ませていただけませんか。	
\\	ていただけませんか
\\	すみませんが、そこどいてもらえませんか?	
\\	ていただけませんか
\\	鉛筆を忘れてしまったんですが、ちょっと貸してもらえませんか。	
\\	ていただけませんか
\\	早く帰ってほしいとお父さんに言ってくれる?	
\\	てほしい
\\	すみませんが、ここでタバコを吸わないでほしいんですが。	
\\	てほしい
\\	それ、直してほしいんです。	
\\	てほしい
\\	明日事務所に行ってほしい。	
\\	てほしい
\\	一応説明してほしいね、アニー。	
\\	てほしい
\\	人形買ってきてほしいのよ。	
\\	てほしい
\\	フランス語の家庭教師を探してほしいんですが。	
\\	てほしい
\\	職場で起こっている事は全て知らせてほしい。	
\\	てほしい
\\	マルコ、助けてほしいんだ。	
\\	てほしい
\\	しかしながら、この件はわたしに処理を任せてほしい。	
\\	てほしい
\\	誰かが階段を下りてくるところらしい。	
\\	ているところ
\\	わたしは今、家で日本語を勉強しているところです。	
\\	ているところ
\\	彼女はお茶を飲みながら雑誌を読んでいるところです。	
\\	ているところ
\\	姉は電話をかけているところだ。	
\\	ているところ
\\	卒業間近なのにまだ就職が決まってないんです。今迷っているところなんです。	
\\	ているところ
\\	数人が競技場に入ってくるところだった。	
\\	ているところ
\\	子供を産んだばかりなので、うちで休んでいるところです。	
\\	ているところ
\\	今対策を考えているところです。	
\\	ているところ
\\	あの二人が空っぽの教室でキスしてるところだ。	
\\	ているところ
\\	このごろ、わたしたちは会社の再建計画を作っているところなので、とても忙しいです。	
\\	ているところ
\\	わたしの今日があるのはまったく先生方々のおかげだと言ってもいいです。	
\\	といってもいい
\\	話し出してしまえば、続けて話してしまうほうが楽だった。ほっとすると言ってもよかった。	
\\	といってもいい
\\	わたしの家には家具らしい家具は一つもないと言ってもいい。	
\\	といってもいい
\\	彼の口調には、どこか変なところがあった。冷たいと言ってもいいくらいだった。	
\\	といってもいい
\\	これは彼の作ったものの中で一番いい作品だと言ってもいいだろう。	
\\	といってもいい
\\	天気予報によると、明日は大雪だそうです。	
\\	そうだ
\\	彼はいまにも涙をこぼしそうだった。	
\\	そうだ
\\	田中さんはまだ独身だそうです。	
\\	そうだ
\\	彼の声はちょっぴり心配そうだった。	
\\	そうだ
\\	君は昨夜、ずいぶん多くの命を救ったそうだね。	
\\	そうだ
\\	先生によると、今度の期末試験は難しいそうです。	
\\	そうだ
\\	あの時計、わたし、気が狂いそうだった。	
\\	そうだ
\\	田中君の話によると、山田さんはもう日本へ帰ったそうです。	
\\	そうだ
\\	去年よりはずっとましな夏休みになりそうだ。	
\\	そうだ
\\	わたしは息子を買い物に行かせました。	
\\	させる
\\	彼は手を上げてわたしを黙らせた。	
\\	させる
\\	社長は佐藤さんを東京へ出張させました。	
\\	させる
\\	もしかしたら、彼は君を現場に行かせたくなかったんだ。	
\\	させる
\\	母はわたしに自分の部屋を片付けさせた。	
\\	させる
\\	彼が咳払いをして皆を静かにさせた。	
\\	させる
\\	昨日の出来事が、僕に対する彼女の考えを変えさせたようだ。	
\\	させる
\\	先生は病気の生徒に早めに家に帰らせました。	
\\	させる
\\	なぜかその人たちは、わたしを不安な気持にさせた。	
\\	させる
\\	たけし、妹を泣かせてはいけないよ。	
\\	させる
\\	その辺りはとてもにぎやかだと聞いた。	
\\	ときいた
\\	木村さんも夕べのパーティーに出たと聞いた。	
\\	ときいた
\\	わたしはあなたが学校に戻ったと聞いたとき、すごく驚いた。	
\\	ときいた
\\	城のどこかに怪物が潜んでいると聞いた。	
\\	ときいた
\\	5メートルおきに木を植えました。	
\\	おきに
\\	彼女は、数分おきに窓の外に目をやっていた。	
\\	おきに
\\	空港へ向かうバスは20分おきに出ている。	
\\	おきに
\\	100メートルおきに電柱が立っている。	
\\	おきに
\\	一日おきにダイヤモンドが見つかると思っていたのか?	
\\	おきに
\\	彼は一週間おきに問題を起こす生徒だ。	
\\	おきに
\\	この道路には25キロおきにガソリンスタンドがある。	
\\	おきに
\\	今でも遅くないから、早く謝りなさい。	
\\	ても
\\	いくらお金があっても、金で買えないものがあります。	
\\	ても
\\	二人が何を話しているのかは、聞かなくてもわかる。	
\\	ても
\\	何回説明してもわかってくれない。	
\\	ても
\\	あいつはどんなことがあっても、僕たちを助けたりしないよ。	
\\	ても
\\	わたしがいなくてもだれも寂しがらなかった。	
\\	ても
\\	子供にでもできることよ。	
\\	ても
\\	何度失敗しても、やめずに頑張ります。	
\\	ても
\\	学校でもわたしは一人ぼっちだった。	
\\	ても
\\	-て
\\	(-て) 
\\	じゃう	
\\	-て
\\	しまう	
\\	-て
\\	(-て) 
\\	ちゃう	てしまう
\\	あ、大変。急いで来たから財布を忘れちゃった。	
\\	てしまう
\\	鍵をかけずに出かけてしまいました。	
\\	てしまう
\\	ワインは全部飲んでしまった。	
\\	てしまう
\\	小遣いはもう使ってしまったの。	
\\	てしまう
\\	この本はたった二日で読み終わってしまった。	
\\	てしまう
\\	僕、どうしてあんなことやっちゃったんだろう?	
\\	てしまう
\\	この手紙を書いてしまってから出かけます。	
\\	てしまう
\\	わたしは風邪を引いてしまいました。	
\\	てしまう
\\	昨日はメガネをかけたまま寝てしまった。	
\\	てしまう
\\	あいつ、どこ行っちゃったのかなあ?	
\\	てしまう
\\	スポーツをするのと、見るのとどちらが好きですか。	
\\	と~と、どちらが
\\	この大学とあの大学とどちらがいいですか。	
\\	と~と、どちらが
\\	バスと電車とどちらが速いですか。	
\\	と~と、どちらが
\\	コーヒーと紅茶とどちらになさいますか。	
\\	と~と、どちらが
\\	日本語と英語とどちらが難しいですか。	
\\	と~と、どちらが
\\	彼の顔とボールとどちらが赤いか判定が難しかった。	
\\	と~と、どちらが
\\	毎日、仕事とか、会議とかでとても忙しい。	
\\	とか~とか
\\	この大学では、ベトナムごとか東アジアのいろいろなことばが勉強できます。	
\\	とか~とか
\\	日本語は最も難しい言語だと言われている。	
\\	といわれている
\\	これからも中国はますます発展していくと言われている。	
\\	といわれている
\\	「ショーシャンクの空に」は世界の最も偉大な映画だと言われている。	
\\	といわれている
\\	それは伝説だ、存在しないものだと言われていました。	
\\	といわれている
\\	大都会は危険だと言われている。	
\\	といわれている
\\	あの人は頭がいいとみんなに言われている。	
\\	といわれている
\\	魚は健康にいいと言われています。	
\\	といわれている
\\	い
\\	と見える	
\\	と見える/と見えて	
\\	と見える	とみえる
\\	試験の問題が難しいと見えてみんな頭を抱えている。	
\\	とみえる
\\	この辺は静かだと見えて車も人も少ない。	
\\	とみえる
\\	雨が降ったと見えて、道が濡れています。	
\\	とみえる
\\	一日中雨が降りつづけそうです。	
\\	つづける
\\	彼と電話で1時間も喋りつづけた。	
\\	つづける
\\	歩き続けろ!	
\\	つづける
\\	あの人は30分ずっと話しつづけている。	
\\	つづける
\\	彼女は僕を見つめ続けていた。	
\\	つづける
\\	長い時間テレビを見つづけると、目が痛くなるよ。	
\\	つづける
\\	休暇が始まってから一週間無視し続けていた宿題を、思い出すときが来た。	
\\	つづける
\\	どうぞ読み続けてください。	
\\	つづける
\\	彼は何も心配する必要がないと主張し続ける。	
\\	つづける
\\	とにかくあなたが、彼女の言うことを聞いてくれてよかったです。	
\\	てよかった
\\	君に会えてよかった。	
\\	てよかった
\\	今こんな怪物と戦わなくてよかった。	
\\	てよかった
\\	とにかく、迎えにきてよかった。	
\\	てよかった
\\	救うことができてよかったです。	
\\	てよかった
\\	人の命は何より大切です。	
\\	より
\\	今朝はいつもより早く学校へ行きました。	
\\	より
\\	漢字はひらがなやかたかなより難しいです。	
\\	より
\\	ジョンはほかの誰よりピーターを信用した。	
\\	より
\\	東京は大阪より大きいですか。	
\\	より
\\	建物の中はいつもより暗かった。	
\\	より
\\	今朝は昨日の朝より寒いです。	
\\	より
\\	今年は去年より多くの留学生が日本へ来ました。	
\\	より
\\	弟はわたしより頭がいいです。	
\\	より
\\	一年日本語を勉強して、日本語が話せるようになった。	
\\	ようになる
\\	一ヶ月ぐらい運転の練習をしていたので、やっとできるようになりました。	
\\	ようになる
\\	娘が一人で服が着られるようになった。	
\\	ようになる
\\	英会話ができるようになりたいです。	
\\	ようになる
\\	以前はお酒は飲まなかったが、このごろ飲むようになりました。	
\\	ようになる
\\	毎日練習したので泳げるようになりました。	
\\	ようになる
\\	不思議なことに、会社をやめたら、よく眠れるようになった。	
\\	ようになる
\\	勉強する時は眼鏡をかけます。	
\\	とき
\\	車の運転をする時には、この薬を飲まないようにしてください。	
\\	とき
\\	角を曲がった時、彼が全速力でこちらに走ってくるのが見えた。顔中で笑っている。	
\\	とき
\\	暗いところで写真を撮る時は、どのボタンを押しますか。	
\\	とき
\\	この黄色い薬は熱が出た時に飲んでください。	
\\	とき
\\	彼らが教室に着いた時には、すでにそこで皆を待っていた。	
\\	とき
\\	もうここには戻って来れないんだ。楽しかった時のことを思い出さない?	
\\	とき
\\	爆弾が爆発したところだ。	
\\	たところ
\\	あなたのことを話してたところだよ。	
\\	たところ
\\	アメリカから日本に戻ったところです。	
\\	たところ
\\	あなたを起こしにいこうと思ってたところだ。	
\\	たところ
\\	彼が何か飲み物を取りにいったところよ。	
\\	たところ
\\	主人はちょうど今出かけたところです。	
\\	たところ
\\	図書館にあなたを探しにいってきたところです。	
\\	たところ
\\	わたしはたった今、うちへ帰ったところです。	
\\	たところ
\\	彼はわたしからのプレゼントを開けたところだった。	
\\	たところ
\\	「レポートはもう書き終わりましたか。」「ええ。今日、書き終わったところです。」	
\\	たところ
\\	わたしは留学をする予定です。	
\\	よていだ
\\	あなたはこのクラスで、今年何を教える予定ですか。	
\\	よていだ
\\	日本語を勉強をする予定です。	
\\	よていだ
\\	今日は全部で何人来る予定ですか。	
\\	よていだ
\\	これから一年、食べたり眠ったりする予定はあるの?	
\\	よていだ
\\	始めは飛行機で行く予定だったんだけど、友達がお金がないというから、船で行くことにしたんだ。	
\\	よていだ
\\	今年も北海道を旅する予定です。	
\\	よていだ
\\	わたしは先生を訪ねる予定だった。	
\\	よていだ
\\	誰かが来たようです。玄関で音がしましたよ。	
\\	ようだ
\\	桜の花が散って、地面はまるで雪が降ったようだ。	
\\	ようだ
\\	わたしたちは意見が違うようだが。	
\\	ようだ
\\	このホテルはサービスがよいので、お客は満足しているようだ。	
\\	ようだ
\\	火事の原因はタバコの火のようです。	
\\	ようだ
\\	彼はアイドルのようです。	
\\	ようだ
\\	話し方からすると、彼は、東京の人ではないようだ。	
\\	ようだ
\\	まだ10月のはじめなのにとても寒い。まるで冬になったかのようだ。	
\\	ようだ
\\	そのことは全然知りません。	
\\	ぜんぜん
\\	学校では遊んでばかりいて、全然勉強しませんでした。	
\\	ぜんぜん
\\	政治については全然興味がない。	
\\	ぜんぜん
\\	わたしは、1年前には、全然日本語が話せませんでした。	
\\	ぜんぜん
\\	お金は全然使わなかった。	
\\	ぜんぜん
\\	彼らには全然わかっていないらしい。	
\\	ぜんぜん
\\	全然気づいていなかった。	
\\	ぜんぜん
\\	この季節はたくさんの人が病気になりやすい。	
\\	やすい
\\	東京は生活しやすいところだと思いますか。	
\\	やすい
\\	それはだまされやすい者を怖がらせる作り話です。	
\\	やすい
\\	この問題は間違えやすい。	
\\	やすい
\\	このガラスが壊れやすいですよ。	
\\	やすい
\\	彼は影響を受けやすい人間だ。	
\\	やすい
\\	床は滑りやすいので、ご注意ください。	
\\	やすい
\\	この部屋は明るくて広いので、仕事がしやすいです。	
\\	やすい
\\	このペンは書きやすいですね。	
\\	やすい
\\	彼は来月結婚するということを、まだ誰にも知らせていない。	
\\	ということ
\\	いろいろ調べて、その駅が一番便利だということが分かった。	
\\	ということ
\\	教授になるということが彼にとってどんなにうれしいことなのか、わたしにはよくわかっている。	
\\	ということ
\\	この法律を知っている人が少ないということは、大きな問題だ。	
\\	ということ
\\	そう、わたしはこれが地図だということを知っている。	
\\	ということ
\\	部屋の電気がきえているということは、たぶん彼はもう寝たのでしょう。	
\\	ということ
\\	おまえがスパイだということを、なぜ初めから見抜けなかったのか。	
\\	ということ
\\	この封筒の中にあるもの、何だと思う?	
\\	とおもう
\\	仕事を続けるのはむりだと思う。	
\\	とおもう
\\	わたしたち、もうすぐ会えると思うよ。	
\\	とおもう
\\	あの人は山田さんの弟だと思う。	
\\	とおもう
\\	ママが、僕の忘れた物をいくつか送ってくれると思うよ。	
\\	とおもう
\\	みんな喜ぶと思うわ。	
\\	とおもう
\\	わたしは何も騒ぐ必要はないと思う。	
\\	とおもう
\\	買い物が1時間ぐらいかかると思う。	
\\	とおもう
\\	君は彼が父親代わりだと思う?	
\\	とおもう
\\	彼がそんなこと思うわけないわ!	
\\	とおもう
\\	これからパンを焼くところです。	
\\	ところ
\\	「晩御飯はもう食べましたか。」「いいえ、これから食べるところです。」	
\\	ところ
\\	わたしは今来るところです。	
\\	ところ
\\	これから食事に行くところなんですが、一緒にいかがですか。	
\\	ところ
\\	ちょうど電車に乗るところです。	
\\	ところ
\\	学生達は校門に戻るところです。	
\\	ところ
\\	うれしさのあまり、彼は涙を流しました。	
\\	あまり
\\	彼女は父親の死を知らされて、悲しみのあまり、寝込んでしまいました。	
\\	あまり
\\	怒りのあまり椅子を壊した。	
\\	あまり
\\	試験に合格したうれしさのあまり、電話で叫んだ。	
\\	あまり
\\	どうすればいいですか?	
\\	ばいい
\\	もう少し雨が降ればいいな。	
\\	ばいい
\\	新宿駅に行きたいんですが、どう行けばいいですか?	
\\	ばいい
\\	借りたお金は今後5年で帰せばいいです。	
\\	ばいい
\\	後で説明すればいい。	
\\	ばいい
\\	わざわざ来る必要はないです。電話で連絡すればいいです。	
\\	ばいい
\\	こうして運べばいい。	
\\	ばいい
\\	明日の9時に行けばいいです。	
\\	ばいい
\\	ただ謝ればいいというような問題じゃない。	
\\	ばいい
\\	どの出口に出ればいいの?	
\\	ばいい
\\	話したいことがあれば、どんどん話せばいい。	
\\	ばいい
\\	年を取れば取るほど、体の大切さが分かります。	
\\	ば~ほど
\\	経済が成長すればするほど貧富の差が広がっていく。	
\\	ば~ほど
\\	この本は読めば読むほど面白くなる。	
\\	ば~ほど
\\	高齢であればあるほど転職は厳しくなる。	
\\	ば~ほど
\\	甘いものを食べれば食べるほど太っていく。	
\\	ば~ほど
\\	朝まで待てばよかった。	
\\	ばよかった
\\	遅刻してしまった。もっと早く家を出ればよかった。	
\\	ばよかった
\\	漫画家になればよかった。	
\\	ばよかった
\\	わたしたちの手でドアを開ければよかったのですが、それを娘にさせたのです。	
\\	ばよかった
\\	早く連絡してくれればよかったのに。	
\\	ばよかった
\\	田中さんにあんなことを言わなければよかった。	
\\	ばよかった
\\	傘を持ってくればよかった。	
\\	ばよかった
\\	夕べの番組は面白かった。あなたも見ればよかったのに。	
\\	ばよかった
\\	新しいパソコンを買わないで、わたしのを使えばよかったのに。	
\\	ばよかった
\\	学生はもっと勉強す(る)べきだ。	
\\	べき
\\	電車の中では、若者は年寄りに席を譲るべきです。	
\\	べき
\\	何をすべきか、もう彼女の心は決まっていた。	
\\	べき
\\	わたしは彼に言うべきことを全部言った。	
\\	べき
\\	彼に会いに行くべきじゃないかな?	
\\	べき
\\	約束は守るべきだ。	
\\	べき
\\	人間は自分の行動に責任を持つべきだ。	
\\	べき
\\	拾った財布は警察に届けるべきですよ。	
\\	べき
\\	おもちゃは、まず安全であるべきだ。	
\\	べき
\\	このアパートは狭いばかりでなく、暗いです。	
\\	ばかりでなく
\\	彼は漢字が書けないばかりか、平仮名も片仮名も書けない。	
\\	ばかりでなく
\\	ピーターさんは日本に行くことは、友人ばかりでなく、家族でさえも知りませんでした。	
\\	ばかりでなく
\\	わたしは右手ばかりでなく左手でも字が書けます。	
\\	ばかりでなく
\\	このごろ、彼は遅刻が多いばかりか、授業中に居眠りすることもあります。	
\\	ばかりでなく
\\	この車は空気を汚さないばかりか、価格が高い。	
\\	ばかりでなく
\\	祖父は体が弱っていくばかりだ。	
\\	一方だ)	
\\	ばかりだ
\\	傷が深くなるばかりです。	
\\	一方だ)	
\\	ばかりだ
\\	「教えてよ!」わたしたちが促したが彼は笑って首を横に振るばかりだった。	
\\	一方だ)	
\\	ばかりだ
\\	手術が終わってから、父の病気は悪くなるばかりでした。	
\\	一方だ)	
\\	ばかりだ
\\	資源をめぐる争いで、両国の関係はますます悪化するばかりだ。	
\\	一方だ)	
\\	ばかりだ
\\	英語も数学も学校を出てからは忘れていくばかりだ。	
\\	一方だ)	
\\	ばかりだ
\\	彼女は何も言わなかった。ただわたしを睨むばかりだった。	
\\	一方だ)	
\\	ばかりだ
\\	戦争のために、国民の生活はますます苦しくなるばかりだ。	
\\	一方だ)	
\\	ばかりだ
\\	わたしは、いくらお酒を飲んでも顔色が変わらない。	
\\	どんなに~ても
\\	どんなに一生懸命働いても、生活は楽にならないだろうと思う。	
\\	どんなに~ても
\\	どんなに健康に自信があっても、油断していると風邪を引く。	
\\	どんなに~ても
\\	わたしはどんなに暑くても寝るときはクーラーを消して寝ます。	
\\	どんなに~ても
\\	父はどんなに苦しくてもあきらめませんでした。	
\\	どんなに~ても
\\	もう酒は飲まないと決めた以上は、どんなに誘われても絶対にそれを守りたい。	
\\	どんなに~ても
\\	ギターを習い始めたが、どんなに練習しても全然うまくならない。	
\\	どんなに~ても
\\	どんなに頑張っても、50点しかとれなかった。	
\\	どんなに~ても
\\	「いやだ!」しゃくりあげるふりをしながら彼女がわめいた。	
\\	ふりをする
\\	彼女は聞こえないふりをした。	
\\	ふりをする
\\	中田さんは独身のふりをしているが、結婚していて3人も子供がいる。	
\\	ふりをする
\\	彼は何かを探すふりをした。	
\\	ふりをする
\\	ケーキがまずかったけれど、二人ともおいしそうなふりをした。	
\\	ふりをする
\\	彼はそのことについて知っているふりをしているが、本当は知らないと思う。	
\\	ふりをする
\\	木村さんは息子などいないというふりをしていた。	
\\	ふりをする
\\	女の子が寝ているふりをしていた。	
\\	ふりをする
\\	おじさんが新聞を立てて開いて読むふりをした。	
\\	ふりをする
\\	ちょっとした油断をしたばかりに、大怪我をしてしまった。	
\\	ばかりに
\\	物価は上がるばかりに、生活は少し楽にならない。	
\\	ばかりに
\\	お金がないばかりに、今度の旅行には行けなかったのです。	
\\	ばかりに
\\	彼は競馬に熱中したばかりに、財産を全部失った。	
\\	ばかりに
\\	仕事の経験が少ないばかりに、就職口がなかなか見つからない。	
\\	ばかりに
\\	別にそれほど疲れていません。	
\\	べつに~ない
\\	わたしは別に彼女が大嫌いなわけではありませんよ。	
\\	べつに~ない
\\	彼は僕の失礼な言葉を別に気にする様子もない。	
\\	べつに~ない
\\	別に彼のために作ったわけではない。	
\\	べつに~ない
\\	その絵は別に貴重なものでもなんでもないです。	
\\	べつに~ない
\\	別にかまわないよ。	
\\	べつに~ない
\\	お金をたくさん持ってることは別に怪しくないでしょ。	
\\	べつに~ない
\\	あの子は甘いものばかり食べます。	
\\	ばかり
\\	主人は小説を読んでいるばかりで、家事を全然手伝ってくれない。	
\\	ばかり
\\	テレビを見てばかりいると、目が悪くなりますよ。	
\\	ばかり
\\	赤ちゃんが泣いてばかりいます。	
\\	ばかり
\\	この店のお客さんは、男性ばかりですね。	
\\	ばかり
\\	あの人は背が高いばかりで、あまり力はない。	
\\	ばかり
\\	嘘ばかりつく人が嫌いだ。	
\\	ばかり
\\	うちの子は毎日遊んでばかりで勉強をしない。	
\\	ばかり
\\	そんなに働いてばかりいては体を壊しますよ。	
\\	ばかり
\\	忙しくて、休みをとるどころか、食事をする時間もない。	
\\	どころか
\\	私は彼に役に立つどころか、迷惑もかけた。	
\\	どころか
\\	彼ほど失礼な人はいません。	
\\	ほど~ない
\\	英語の文法は日本語ほど難しくありません。	
\\	ほど~ない
\\	試験ほど嫌なものはない。	
\\	ほど~ない
\\	戦争ほど残酷で悲惨なものはない。	
\\	ほど~ない
\\	今年の合格率は去年ほど高くありません。	
\\	ほど~ない
\\	彼女ほど優しい人はいない。	
\\	ほど~ない
\\	東京ほど家賃の高いところはない。	
\\	ほど~ない
\\	あの家はいまや形容しがたいほどに汚れている。	
\\	がたい
\\	彼の話は信じがたい内容だった。	
\\	がたい
\\	女性の考え方には、我々男性には理解しがたいことがある。	
\\	がたい
\\	幼い子供に対する犯罪は許しがたい。	
\\	がたい
\\	信じがたい事かもしれませんが、うちの息子は20ヶ国語も話せます。	
\\	がたい
\\	その計画は成功したとは言いがたい。	
\\	がたい
\\	二人の顔に浮かんだ憎しみは、甲乙つけがたい激しさだった。	
\\	がたい
\\	過去を忘れるべきではありません。	
\\	べきではない
\\	そんなことをすべきではない。	
\\	べきではない
\\	暴力行為はどんな理由があっても絶対に許すべきではない。	
\\	べきではない
\\	親に対してはこんな失礼なことを言うべきではありません。	
\\	べきではない
\\	子供の安全を考えれば、夜遅くまで外で遊ばせるべきではない。	
\\	べきではない
\\	大学院は自分で研究を進めるところだから、先生に頼ろうとするべきではない。	
\\	べきではない
\\	なぜ、皆、僕が真実を知るべきではないと固く決意しているのだろうか?	
\\	べきではない
\\	いくら安くても、買いたくない。	
\\	いくら~ても
\\	いくら好きでもコーラーをたくさん飲むと、身体に悪いですよ。	
\\	いくら~ても
\\	いくら寝ても、まだ眠いです。	
\\	いくら~ても
\\	いくら調べてもわからなかったので、先生に聞いた。	
\\	いくら~ても
\\	彼女はいくら頼んでも聞いてくれない。	
\\	いくら~ても
\\	仕事だから、いくら嫌いでもやらなければならない。	
\\	いくら~ても
\\	必要なので、いくら高くても買います。	
\\	いくら~ても
\\	給料がいくら上がっても足りない。	
\\	いくら~ても
\\	日本語を教えてくれる代わりに、田中さんに中国語を教えています。	
\\	代わりに
\\	僕の代わりに、君がしてくれ。	
\\	代わりに
\\	今日は用事で来られなかった山田先生の代わりに、授業をします。	
\\	代わりに
\\	正月は海外旅行に行く代わりに、近くの温泉に行った。	
\\	代わりに
\\	手伝ってもらった代わりに晩御飯をおごらせてください。	
\\	代わりに
\\	映画を見に行く代わりに家でテレビを見る。	
\\	代わりに
\\	今とても忙しくてね。誰か代わりにやってくれる人いないかな。	
\\	代わりに
\\	寒いはずです。雪が降ってきました。	
\\	はずだ
\\	このブランドバッグは50万円以上かかるはずだ。	
\\	はずだ
\\	今日は祝日だから、会社は休みのはずだ。	
\\	はずだ
\\	今日は日曜日だからデパートはこんでいるはずです。	
\\	はずだ
\\	パパがそろそろ家に帰るはずだ。	
\\	はずだ
\\	彼のことを信じ切っています。	
\\	切る
\\	長い小説を、二日間で読み切った。	
\\	切る
\\	今日は忙しくて、もう疲れ切ってしまった。	
\\	切る
\\	お小遣いを使い切ってしまった。	
\\	切る
\\	彼女のわがままに困り切っている。	
\\	切る
\\	彼が何て言ったっけ?	
\\	っけ
\\	明日のパーティーのこと、話したっけ?	
\\	っけ
\\	自己紹介したっけ?	
\\	っけ
\\	国へ帰るのは、来週だったっけ?	
\\	っけ
\\	あなたの誕生日、いつだっけ?	
\\	っけ
\\	誰が僕たちに会いにくるって言ったっけ?	
\\	っけ
\\	今日は何曜日でしたっけ?	
\\	っけ
\\	夢を見た日はいつだって言ったっけ?	
\\	っけ
\\	あれ、わたし、さっき眼鏡をどこに置いたっけ?	
\\	っけ
\\	あれ、僕言わなかったっけ?	
\\	っけ
\\	わたし決して夢をあきらめません。	
\\	決して~ない
\\	この宝は決してきみのものにはならない。	
\\	決して~ない
\\	いろいろお世話になりました。親切にしていただいたことは決して忘れません。	
\\	決して~ない
\\	あなたは決して一人ではありません。	
\\	決して~ない
\\	うそは決して申しません。	
\\	決して~ない
\\	わたしは決して友人を裏切ったことはない。裏切るくらいなら、わたしが死ぬ方がましだ。	
\\	決して~ない
\\	あなたがここにいるのは決して安全じゃない。	
\\	決して~ない
\\	何があろうとも、決して家を離れてはいけない。	
\\	決して~ない
\\	お父さんは君を決して許さないよ。	
\\	決して~ない
\\	このことは決して言い漏らさないようにしてください。	
\\	決して~ない
\\	2時間も字を打ったので、ちょっと疲れ気味です。	
\\	ぎみ
\\	新入社員は緊張気味の顔をしていた。	
\\	ぎみ
\\	残業続きで疲れ気味だ。	
\\	ぎみ
\\	今日はどうも風邪気味で、頭が痛い。	
\\	ぎみ
\\	生活が不規則で睡眠不足気味だ。	
\\	ぎみ
\\	最近太り気味なので、夕食を抜くことにした。	
\\	ぎみ
\\	初めての演説なので、彼は少し緊張気味だ。	
\\	ぎみ
\\	今朝は寒くて、池に氷が張るほどだった。	
\\	ほど
\\	よく勉強する学生ほど成績がいい。	
\\	ほど
\\	年をとるほど体が弱くなる。	
\\	ほど
\\	伝統的なものは、古いほど価値がある。	
\\	ほど
\\	今年の日本語能力試験はそれほど簡単じゃなかった。	
\\	ほど
\\	今年ほど雪の降った年はなかった。	
\\	ほど
\\	駅に近いほど家賃が高い。	
\\	ほど
\\	飽きるほど食べた。	
\\	ほど
\\	ご飯の量が多くて、食切れないよ。	
\\	切れない
\\	洞窟内で埋められていた数え切れないほどの手紙が発見された。	
\\	切れない
\\	いっぱい頼まないで、二人では食べ切れないから。	
\\	切れない
\\	こんなに長い小説は、一日では読み切れない。	
\\	切れない
\\	たまにタクシーで通勤することがある。	
\\	ことがある
\\	会社員じゃないので、仕事のときはスーツを着ませんが、パーティーなどに行くときは、着ることがあります。	
\\	ことがある
\\	庭で食事をすることがあります。	
\\	ことがある
\\	たまに人の名前を忘れることがあります。	
\\	ことがある
\\	たいてい現金で払いますが、たまにカードを使うことがあります。	
\\	ことがある
\\	ときどき日本料理をすることがあります。	
\\	ことがある
\\	たまに父と一緒にテニスをしに行くことがあります。	
\\	ことがある
\\	父は自分に厳しい一方で、他人には優しい。	
\\	一方で
\\	わたしの仕事は夏は非常に忙しい一方、冬は暇になる。	
\\	一方で
\\	いい親は厳しく叱る一方で、ほめることも忘れない。	
\\	一方で
\\	収入が減る一方で、教育費などの支出は増えていくのだから、節約するしかない。	
\\	一方で
\\	このあたりは、静かな一方、不便である。	
\\	一方で
\\	老人の人口が増えている一方で、若年労働力の数がどんどん減っています。	
\\	一方で
\\	予算がないと言われているが、一方では予算がむだに使われている。	
\\	一方で
\\	仕事をする一方で、遊ぶことも忘れない、そんな若者が増えている。	
\\	一方で
\\	お金を大切にすること。	
\\	こと
\\	勝手に実験室に入らないこと。	
\\	こと
\\	学校の規則を守ること。	
\\	こと
\\	休むときは、必ず学校に連絡すること。	
\\	こと
\\	大声を出さないこと。	
\\	こと
\\	授業中は携帯電話を使わないこと。	
\\	こと
\\	病気でも、会議には必ず出席すること。	
\\	こと
\\	指定の場所以外に自転車を止めないこと。	
\\	こと
\\	身も心も傷だらけの彼は最後自ら自分の命を絶ったのです。	
\\	だらけ
\\	彼はそばかすだらけの顔が青白く見えた。	
\\	だらけ
\\	泥だらけの靴で家に入って怒られた。	
\\	だらけ
\\	彼女は学生の間違いだらけの作文を家へ持ち帰して、いちいち直してやりました。	
\\	だらけ
\\	部屋の中はおもちゃだらけで立つところもありません。	
\\	だらけ
\\	彼の部屋はゴミだらけだ。	
\\	だらけ
\\	彼女のベッドは包み紙だらけになっている。	
\\	だらけ
\\	机の上にほこりだらけです。	
\\	だらけ
\\	今度、東京に転勤することになりました。	
\\	ことになる
\\	よく話し合った結果、やはり離婚ということになりました。	
\\	ことになる
\\	来月出張でアメリカへ行くことになりました。	
\\	ことになる
\\	明日、新しい企画についての会議が行われることになっている。	
\\	ことになる
\\	来月からこの会社で働くことになりました。	
\\	ことになる
\\	急に国へ帰ることになりました。	
\\	ことになる
\\	今回のマラソンは、こちらの競技場からすることになっています。	
\\	ことになる
\\	卒業式は7月10日に行うことになりました。	
\\	ことになる
\\	わが国の首相が、来月、日本をお訪問することになった。	
\\	ことになる
\\	友達を車で空港まで送っていくことになった。	
\\	ことになる
\\	雪が降ると、電車は遅れがちだ。	
\\	がち
\\	先日、病気がちのおばを見舞いに行った。	
\\	がち
\\	結婚して何十年もたつと、結婚記念日さえ忘れがちになる。	
\\	がち
\\	何度も失敗をすると、また失敗をするのではないかと考えがちになる。	
\\	がち
\\	「写真を撮ってもいいですか?」カメラを持ち上げて、彼が遠慮がちに頼んだ。	
\\	がち
\\	そういう事故はとかく起こりがちです。	
\\	がち
\\	彼は最近、体調を崩して、学校を休みがちです。	
\\	がち
\\	愛情があるからこそ、しかるんです。	
\\	こそ
\\	難しい仕事だからこそ、面白いと友達が言った。	
\\	こそ
\\	明日こそ勉強するぞ!	
\\	こそ
\\	こちらこそごぶさたしております。	
\\	こそ
\\	政府は、今年こそ経済がよくなると予測していた。	
\\	こそ
\\	それはとても恐ろしいものだ。だからこそ注意深く扱わなければなるまい。	
\\	こそ
\\	な
\\	なことに	
\\	ことに	
\\	い
\\	ことに	ことに
\\	幸いなことに、やっと自分の好きな人と結婚できました。	
\\	ことに
\\	驚いたことに、あの子はもう大学のコースも独学で終了したそうです。	
\\	ことに
\\	残念なことに、努力して頑張りましたが、失敗しました。	
\\	ことに
\\	不幸なことに友人の母が亡くなったのです。	
\\	ことに
\\	驚いたことに、先生は相変わらず微笑んでいた。	
\\	ことに
\\	彼は一度わたしの命を救った。しかし、奇妙なことに、彼はわたしを毛嫌いしている。	
\\	ことに
\\	驚いたことに、町の様子が10年前に比べ、すっかり変わってしまった。	
\\	ことに
\\	今まで何度酒をやめようと思ったことか。	
\\	ことか
\\	家の近くに地下鉄が通って、どんなに便利なことか。	
\\	ことか
\\	何度注意したことか。	
\\	ことか
\\	合格できたら、どんなに嬉しいことか。	
\\	ことか
\\	学生時代、奨学金がもらえてどれほど助かったことか。	
\\	ことか
\\	顔がそっくりなことから、双子であることを知った。	
\\	ことから
\\	ここは桜の名所であることから、春には花見客が大勢やってくる。	
\\	ことから
\\	彼は父親が有名な作家であることから、範囲から特別な目で見られがちです。	
\\	ことから
\\	富士山が見えることから、この町は富士見町という名前がついた。	
\\	ことから
\\	ガラスが割れていることから、泥棒が入ったとわかった。	
\\	ことから
\\	エアコンをつけたまま出かけた。	
\\	まま
\\	テレビをつけたまま寝てしまいました。	
\\	まま
\\	年をとってもきれいなままでいたい。	
\\	まま
\\	彼はしばらく横になったまま、これからの試合のことを考えていた。	
\\	まま
\\	壁には「バカ」と書かれたままだ。	
\\	まま
\\	靴を履いたまま部屋に入ってしまった。	
\\	まま
\\	ああ、やっぱり会議室に置いたままだ。悪いけど、会議室に行ってノートあるかどうか見てきてくれる?	
\\	まま
\\	彼女に会ったのは1回きりです。	
\\	きり
\\	一度きりの人生を大切に生きることだ。	
\\	きり
\\	兄が朝出かけたっきりまだ帰ってこない。	
\\	きり
\\	一度会ったきりの男性からメールが来た。	
\\	きり
\\	彼の日本語はまるで日本人が話しているように聞こえる。	
\\	まるで
\\	あの二人はまるで兄弟のようによく似ている。	
\\	まるで
\\	まだ三月のはじめなのにとても暑い。まるで夏になったかのようだ。	
\\	まるで
\\	合格した!まるで夢のようだ!	
\\	まるで
\\	まるで彼の頭の中で、誰かが電灯をパチンと点けたようだった。	
\\	まるで
\\	この湖は水がきれいで、まるで透明なガラスのようだ。	
\\	まるで
\\	すごい!まるで魔法のようだ!	
\\	まるで
\\	彼がまるでわたしがいるのに初めて気づいたようにじっと見つめた。	
\\	まるで
\\	これから甘いものは食べないことにする。	
\\	ことにする
\\	冬休みに2週間、帰国することにします。	
\\	ことにする
\\	毎朝、30分ジョギングすることにしています。	
\\	ことにする
\\	外国へ行っていた友達が帰ってきた。パーティーを開くことにした。	
\\	ことにする
\\	寝る前に食べないことにしました。	
\\	ことにする
\\	彼は家の外でぶらぶらして過ごすことにした。	
\\	ことにする
\\	タバコをやめることにした。	
\\	ことにする
\\	今日はどこへも行かないで勉強することにした。	
\\	ことにする
\\	「あれ?買い物に行かないの?」「うん、明日行くことにした。」	
\\	ことにする
\\	今回の連休、どこへ行くことにした?	
\\	ことにする
\\	「どうして送別会に来なかったの。」「だって知らなかったのだもの。」	
\\	もの
\\	君の家族だもん。君が心配するのは当然だ。	
\\	もの
\\	「どうして食べないの?」「だって、まずいんだもん。」	
\\	もの
\\	今日の試験、できなかった。勉強しなかったんだもん、仕方がない。	
\\	もの
\\	「カラオケに行かないの?」「だってまだ仕事があるんだもん。」	
\\	もの
\\	「これ、男のものだぜ。」「だって、気に入ったんだもん。」	
\\	もの
\\	もしかしたら、明日行けないかもしれません。	
\\	もしかしたら
\\	もしかしたらあそこに座っている人は有名人かもしれない。	
\\	もしかしたら
\\	もしかしたら彼の話はうそかもしれない。	
\\	もしかしたら
\\	もしかしたら彼は会社をやめたかもしれない。	
\\	もしかしたら
\\	もしかしたら彼女とはまったく関係がないかもしれないわ。	
\\	もしかしたら
\\	もしかしたらほんとうに、学校に一人も友達がいなかったのかも。	
\\	もしかしたら
\\	もしかしたら今日彼女に会えるかもしれない。	
\\	もしかしたら
\\	もしかしたらみんな殺されてたかもしれないのに。	
\\	もしかしたら
\\	自分を責めることはないわ。	
\\	ことはない
\\	気にするな。あなたが謝ることはないよ。	
\\	ことはない
\\	おじさんの病気はすぐによくなるから、君は何も心配することはない。	
\\	ことはない
\\	彼は決してわたしを裏切ることはない。	
\\	ことはない
\\	わたしがここにいるかぎり、何物も君たちに危害を加えることはないと思いたまえ。	
\\	ことはない
\\	来ることはありません。郵送でいいですよ。	
\\	ことはない
\\	新しい車を買うことはないよ。貸してあげるから。	
\\	ことはない
\\	あの人は体が小さいながらも、なかなか力がある。	
\\	ながらも
\\	狭いながら、楽しいわが家。	
\\	ながらも
\\	体に悪いと知りながらも、タバコがやめられない人は多い。	
\\	ながらも
\\	わたしは残念ながら結婚式に出席できそうにありません。	
\\	ながらも
\\	の面接を受けた。緊張しながらも、質問にはしっかり答えられた。	
\\	ながらも
\\	彼はテストの時間を知っていながらわたしに教えませんでした。	
\\	ながらも
\\	彼は日本人のくせに漢字が読めない。	
\\	くせに
\\	この部屋は狭いくせに家賃がすごく高い。	
\\	くせに
\\	社長は会社のお金を盗んだのはあなたじゃないって分かってるくせに、それでもクビにしたなんて!	
\\	くせに
\\	全然勉強しなかったくせに、試験に合格した。	
\\	くせに
\\	彼女は暑いくせに、厚いコートを着ている。	
\\	くせに
\\	君は何もかも見ているくせに、見たものから推理するということをしないんだ。	
\\	くせに
\\	あいつが大嫌いなくせに、なんで守ってるの?	
\\	くせに
\\	彼は10年間韓国に住んでたくせに、韓国語が話せない。	
\\	くせに
\\	走れば間に合わないことはないよ。急ごう!	
\\	ないことはない
\\	スキーはできないことはないけど、もう何年もやってないから…	
\\	ないことはない
\\	納豆は、食べないことはないですが、あまり好きではありません。	
\\	ないことはない
\\	わたしも留学したことがありますから、あなたの苦労が分からないことはありません。	
\\	ないことはない
\\	車を運転できないことはないんですが、ほとんどしません。	
\\	ないことはない
\\	テーブルの上に食べかけのケーキが置いてある。	
\\	かけ
\\	この本はまだ読みかけだ。	
\\	かけ
\\	家の前で死にかけていた野良猫を拾って動物病院に連れて行った。	
\\	かけ
\\	おふろに入りかけたときに電話が鳴った。	
\\	かけ
\\	椅子の上には書きかけのレポートが置いてあった。	
\\	かけ
\\	明日はほかの仕事をしなければならないのだから、この仕事をやりかけのまま、帰るわけにはいかない。	
\\	かけ
\\	彼は冷めてしまった食べかけのパイをようやく口に入れた。	
\\	かけ
\\	無理をしないことです。	
\\	ことだ
\\	暖かくして、ゆっくり休むことだ。	
\\	ことだ
\\	体を丈夫にしたかったら、好き嫌いをしないで何でも食べることだ。	
\\	ことだ
\\	一度に全部は無理だ。毎日少しずつ勉強することだ。	
\\	ことだ
\\	読解力をつけたいのなら、毎日新聞を読むことだ。	
\\	ことだ
\\	忙しくてめったに休みが取れない。	
\\	めったに~ない
\\	真の予言者はめったにいません。	
\\	めったに~ない
\\	こんなチャンスはめったにないよ。	
\\	めったに~ない
\\	このところではめったに雪が降りません。	
\\	めったに~ない
\\	それは人の一生にめったには見られない美しさだった。	
\\	めったに~ない
\\	彼女はめったに泣かない。	
\\	めったに~ない
\\	こんなもの、めったに手に入らないよ。	
\\	めったに~ない
\\	能力から言って、彼がこの仕事に一番適切だと思う。	
\\	から言うと
\\	この作文は、日本語能力から言えば、まだまだだが、内容はいい。	
\\	から言うと
\\	この成績から言えば、国立大学は無理だと思う。	
\\	から言うと
\\	客の立場から言うと、この店は入り口が狭くて入りにくい。	
\\	から言うと
\\	現状から言って、直ちにその計画を実行するのは無理だ。	
\\	から言うと
\\	あの人たち、規則から言うとなんら悪いことをしていないよ。	
\\	から言うと
\\	財布は電車の中で盗まれたに違いない。	
\\	に違いない
\\	この車は高いに違いない。	
\\	に違いない
\\	これが合言葉だったに違いない。	
\\	に違いない
\\	真面目な山田さんのことだから、時間通りに来るに違いない。	
\\	に違いない
\\	彼が部屋に入ってきた。そのあとに続いて入ってきたのは父親に違いない。	
\\	に違いない
\\	教室にかばんがないから、田中さんはもう帰ったに違いない。	
\\	に違いない
\\	彼らはこのことを知らなかったに違いない。	
\\	に違いない
\\	試験の範囲は、教科書の最初から30ページまでだって。	
\\	んだって
\\	彼は東京大学に合格したんだって。	
\\	んだって
\\	明美さん、結婚してるんだって。	
\\	んだって
\\	彼女、休みには会いに来るんだって。	
\\	んだって
\\	彼はまた彼女と暮らしてるんだって。	
\\	んだって
\\	君たちがドラゴンに乗って逃げたんだって?誰もが話してるよ。	
\\	んだって
\\	この店のピザ、おいしいんだって。	
\\	んだって
\\	疲れて一歩も歩けないくらいだった。	
\\	くらい
\\	キミには感謝してもし切れないくらいだよ。	
\\	くらい
\\	子供じゃないんだから、自分のことぐらい自分で決めなさい。	
\\	くらい
\\	今日は朝から仕事が忙しくて、食事をする時間もないくらいだ。	
\\	くらい
\\	自分の服ぐらい自分で洗ってよ。	
\\	くらい
\\	痛みがひどかった。頭が二つに割れるかと思うくらいだった。	
\\	くらい
\\	50メートルぐらいならわたしも泳げるよ。	
\\	くらい
\\	明日は昼から夕方にかけて雨でしょう。	
\\	にかけて
\\	母は料理にかけては、誰にも負けない自身がある。	
\\	にかけて
\\	弟は勉強はできないが、泳ぎにかけては誰にも負けない。	
\\	にかけて
\\	日本では、大学の入学試験は、普通2月から3月にかけて行われる。	
\\	にかけて
\\	16日夜から17日未明にかけて関東地域の広い範囲で雪が降った。	
\\	にかけて
\\	2月から4月にかけて花粉が飛ぶそうです。	
\\	にかけて
\\	あの鳥が日本で見られるのは、11月から3月にかけてです。	
\\	にかけて
\\	絵を描くことにかけては、彼の右に出るものはいない。	
\\	にかけて
\\	九州から本州にかけて梅雨入りしました。	
\\	にかけて
\\	わたしは、2007年から2009年にかけて、東京に住んでいました。	
\\	にかけて
\\	東京に比べて、大阪のほうが物価が安い。	
\\	に比べて
\\	印刷技術は昔に比べてかなり進歩してきた。	
\\	に比べて
\\	今年は去年に比べて寒い。	
\\	に比べて
\\	十年前に比べて今の生活はなんと豊かでしょう。	
\\	に比べて
\\	電話は手紙に比べて、早く情報を伝えることができます。	
\\	に比べて
\\	結果はホームページにおいて発表されます。	
\\	において/ における
\\	わたしにおいてはどうも異存はない。	
\\	において/ における
\\	ロンドンにおける国際会議が行われた。	
\\	において/ における
\\	学校の外において呪文を行使することを許されておりません。	
\\	において/ における
\\	当時においては、海外旅行など夢のようなことだった。	
\\	において/ における
\\	中国における人口問題について会議を開きました。	
\\	において/ における
\\	現状において、当然、君は退学になりたいわけではなかろう?	
\\	において/ における
\\	初めてにしては、よくできました。	
\\	にしては
\\	今住んでいるところは都心にしては、静かで住み心地がいい。	
\\	にしては
\\	この子は小学生にしてはずいぶんしっかりしている。	
\\	にしては
\\	外国人にしては日本語が上手だ。	
\\	にしては
\\	彼は、タクシー運転手にしては道を知らない。	
\\	にしては
\\	洗ったの?それにしてはきれいじゃないね。	
\\	にしては
\\	「上田先生は何もおっしゃいませんでしたね。」「ええ、上田先生にしてはめずらしいことでしたね。」	
\\	にしては
\\	初めてケーキを作ったにしては、上手にできましたね。	
\\	にしては
\\	農村の生活様式について調べている。	
\\	について
\\	環境保護についてスピーチをしました。	
\\	について
\\	すみません、そちらのアルバイトの募集についてちょっと伺いたいんですが。	
\\	について
\\	労働条件の変更について会社から説明を受けた。	
\\	について
\\	あなたの国についていろいろと教えてください。	
\\	について
\\	この日本文化史についてのレポートは大変よくできている。	
\\	について
\\	この商品が腰の痛みにどのくらい効果があるのかについて詳しく知りたい。	
\\	について
\\	大学では日本経済について研究したいと思っています。	
\\	について
\\	水は人間の体にとって必要なものだ。	
\\	にとって
\\	ガソリンの値段が高くなった。毎日車を運転するわたしにとってこれは大きな問題だ。	
\\	にとって
\\	現在のわたしたちにとって、もっとも重要なのは、地球の環境を守ることであろう。	
\\	にとって
\\	どんな親にとっても、自分の子供は一番かわいいものだ。	
\\	にとって
\\	あの犬は彼にとって、子供のようなものだ。	
\\	にとって
\\	わたしは、病気は、人間にとって偶然おちかかってくる事故のようなものだと思っている。	
\\	にとって
\\	現代人にとって、携帯電話は生活の一部である。	
\\	にとって
\\	わたしにとって暗記は時間の無駄だ。	
\\	にとって
\\	これはわたしにとって難しすぎる問題です。	
\\	にとって
\\	これは小学生向きの童話集です。	
\\	向き
\\	このサイズの服は子供向きです。	
\\	向き
\\	日本語初心者向きの本はあの本屋にたくさんありますよ。	
\\	向き
\\	南向きの部屋があります。	
\\	向き
\\	若い女性向きの小説を書いています。	
\\	向き
\\	心臓病患者向きの本を探していますが。	
\\	向き
\\	留学生向きの大学案内書が刊行された。	
\\	向き
\\	海外向けの番組がますます多くなってきました。	
\\	向け
\\	ワンピースは少年向けの漫画です。	
\\	向け
\\	このデザイナーは若者向けのワンピースをたくさんデザインしました。	
\\	向け
\\	この本は外国人向けだが、日本人が読んでもとても面白く、ためになる。	
\\	向け
\\	この映画は大人向けなので、子供は見てもつまらない。	
\\	向け
\\	この会社では、子供向けのテレビ番組を作っている。	
\\	向け
\\	これは若いお母さん向けに書かれた本です。	
\\	向け
\\	納豆なんて嫌いだ。	
\\	なんか 
\\	なんて
\\	お化粧なんかしてはいけません。	
\\	なんか 
\\	なんて
\\	わたしは鼻が高いわ。なんて素敵な知らせでしょう。	
\\	なんか 
\\	なんて
\\	カラオケなんか行きたくない。	
\\	なんか 
\\	なんて
\\	彼は社長に辞表を提出するなんていうわけにはいかない。	
\\	なんか 
\\	なんて
\\	英語でスピーチなんかできません。	
\\	なんか 
\\	なんて
\\	あの女のやりそうなことだ。草むらに隠れて盗み聞きするなんて。	
\\	なんか 
\\	なんて
\\	「泣いてるの?」「泣いてなんかいないよ!」	
\\	なんか 
\\	なんて
\\	小学生の時、よく学校をサボって母に叱られたものだ。	
\\	ものだ
\\	学生時代は毎日図書館に通ったものだ。	
\\	ものだ
\\	うそは必ずばれるものだ。	
\\	ものだ
\\	子供のころはよく川で遊んだものだ。	
\\	ものだ
\\	前はよく秋葉原に行ったものだ。	
\\	ものだ
\\	人間は年を取ると記憶力が衰えるものだ。	
\\	ものだ
\\	時間の経つのは早いものですね。	
\\	ものだ
\\	昔はわたしが父に叱られたものだ。	
\\	ものだ
\\	来週、国に帰る予定です。なぜなら、親友の結婚式に出席するからです。	
\\	なぜなら
\\	この場所が大好きだ。なぜかというと、たくさんの思い出があるからだ。	
\\	なぜなら
\\	昔日本人は肉を食べなかった。なぜかというと、肉を食べることが仏教の教えに反したからだ。	
\\	なぜなら
\\	学校を変えた。なぜなら、僕のレベルのクラスがなかったからだ。	
\\	なぜなら
\\	彼は新しい背広とネクタイを買った。なぜかというと、就職の面接があるからだ。	
\\	なぜなら
\\	野球にかわり、サッカーがさかんになってきた。	
\\	にかわって
\\	トムがわたしにかわって君を監視してるよ。	
\\	にかわって
\\	入院した先生にかわって、今は新しい先生が授業をしている。	
\\	にかわって
\\	出張中の部長にかわって、わたしがご挨拶させていただきます。	
\\	にかわって
\\	父にかわって、一言お礼申し上げます。	
\\	にかわって
\\	ロボットは人間にかわって、危険な仕事をする。	
\\	にかわって
\\	にかわり、
\\	が使われるようになった。	
\\	にかわって
\\	留守の間に泥棒が入った。	
\\	の間に
\\	ママ、夜の間に誰が来たと思う。	
\\	の間に
\\	子供が寝ている間に洗濯をしました。	
\\	の間に
\\	この15年の間に何があったのでしょうか。	
\\	の間に
\\	四人の間に意見の相違が出てきた。	
\\	の間に
\\	わたしが友人と話している間に、彼は帰ってしまった。	
\\	の間に
\\	日本では、過去36年の間に都市の人口が増加した。	
\\	の間に
\\	大雪によって、電車は3時間以上も遅れました。	
\\	によって 
\\	による
\\	「ねえ、この古い書類はどうすればいいの?」「ものによって違うよ。」	
\\	によって 
\\	による
\\	このビルは有名な建築家によって設計された。	
\\	によって 
\\	による
\\	わたしたちの時間の感覚は、人によって、また立場によってもかなり違います。	
\\	によって 
\\	による
\\	人によって考え方が違う。	
\\	によって 
\\	による
\\	この問題は話し合いによって解決できると思います。	
\\	によって 
\\	による
\\	窓を開けっぱなしで出かけた。	
\\	っぱなし
\\	電車が混んでいて、東京まで立ちっぱなしだった。	
\\	っぱなし
\\	二人は開けっぱなしのドアの方に進んだ。	
\\	っぱなし
\\	水を出しっぱなしにしないでください。	
\\	っぱなし
\\	テレビをつけっぱなしで寝てしまった。	
\\	っぱなし
\\	その蛇口、壊れっぱなしよ。	
\\	っぱなし
\\	会議には中国をはじめ、アジアの国々が参加した。	
\\	をはじめ
\\	東京をはじめ、全国の主要都市で新製品の発売が決まった。	
\\	をはじめ
\\	わが国では、野球をはじめとして、サッカーやテニスなど、様々なスポーツがさかんである。	
\\	をはじめ
\\	薬を飲みさえすれば治ります。	
\\	さえ~ば
\\	申し訳ないんで、もし君さえ良ければ土曜日にも出て来てもらうかと思って。	
\\	さえ~ば
\\	天気さえよければ、よい旅行になるでしょう。	
\\	さえ~ば
\\	新聞に名前さえ出なければ、問題は起こらないだろう。	
\\	さえ~ば
\\	「昨日の試験、どうだった?」「時間さえあれば、あんな問題全部できたよ。	
\\	さえ~ば
\\	最近、自分さえよければいいという考えの人が増えている。	
\\	さえ~ば
\\	きみのおかげで私たちのチームは50点も獲得した。	
\\	おかげで
\\	両親のおかげで日本へ留学することができた。	
\\	おかげで
\\	あなたの情報のおかげで犯人を捕まえた。	
\\	おかげで
\\	駅員のおかげで私の大きい荷物が車内に運ばれた。	
\\	おかげで
\\	通行人のおかげで捨てられた赤ちゃんが救われた。	
\\	おかげで
\\	ロナウドのおかげで、ブラジルがワールドカップ優勝を果たした。	
\\	おかげで
\\	考えてみると、私たちの家族があなたのおかげで命拾いした。	
\\	おかげで
\\	先生のおかげで、僕の英語はすこしずつ上手になった。	
\\	おかげで
\\	い
\\	最中(に)	
\\	-て
\\	いる最中(に)	
\\	の最中(に)	最中に
\\	会議の最中に停電しました。	
\\	最中に
\\	電話している最中に、誰かが玄関に来た。	
\\	最中に
\\	授業の最中にノックする音がしました。	
\\	最中に
\\	テレビを見ると、アナウンサーが脱獄囚のニュースを読んでいる最中だった。	
\\	最中に
\\	今は食事の最中だから、タバコは遠慮したほうがいいですよ。	
\\	最中に
\\	みんなが大笑いの最中彼女がキッチンに戻ってきた。	
\\	最中に
\\	この日本語のクラスでは女子学生が15に対して、男子学生は3人しかいません。	
\\	に対して
\\	妻がスポーツ好きなのに対して、僕は読書好きだ。	
\\	に対して
\\	お客さんに対しては、そのような言葉遣いをしてはいけません。	
\\	に対して
\\	目上の人に対して敬意を示さなければなりません。	
\\	に対して
\\	わたしはその批判に対して二つのことを言いたい。	
\\	に対して
\\	この学校では教師に対する要求が厳しいです。	
\\	に対して
\\	先日の会議では彼の説明に対して質問が集中した。	
\\	に対して
\\	田中さんはもともと世話好きで、誰に対しても親切だ。	
\\	に対して
\\	先生の話によると、来年の大学受験はもっと難しくなるらしい。	
\\	によると 
\\	によれば
\\	医者の話によると、この病気はすぐによくなるというものではないらしい。	
\\	によると 
\\	によれば
\\	天気予報によると、明日は雪が降るそうだ。	
\\	によると 
\\	によれば
\\	発表によると、男女とも非常に長生きだそうです。	
\\	によると 
\\	によれば
\\	この本によると、昔ここにお寺があったということだ。	
\\	によると 
\\	によれば
\\	新聞によれば、近頃物価が上がるそうです。	
\\	によると 
\\	によれば
\\	さっきの地震はテレビの速報によると震度5だそうだ。	
\\	によると 
\\	によれば
\\	そちらにおつき次第、すぐお電話をください。	
\\	次第
\\	雨がやみ次第、出発しましょう。	
\\	次第
\\	来週のスケジュールが決まり次第、連絡してください。	
\\	次第
\\	わかり次第、すぐお知らせいたします。	
\\	次第
\\	成田空港に着き次第、電話します。	
\\	次第
\\	式が終了次第、ロビーに集合してください。	
\\	次第
\\	参加者の名前がわかり次第、教えていただけませんか。	
\\	次第
\\	姉はいつも子供っぽい服を着ている。	
\\	っぽい
\\	彼はいたずらっぽい目つきをした。	
\\	っぽい
\\	あの小学生は、大人っぽい。	
\\	っぽい
\\	たけしくんは飽きっぽくて何をやってもすぐやめてしまう。	
\\	っぽい
\\	この料理は油っぽくていやだ。	
\\	っぽい
\\	わたしは、自分でもおどろくほど忘れっぽいです。	
\\	っぽい
\\	どの寮に入るにしても、皆さん一人一人が寮にとって誇りとなるよう望みます。	
\\	にしても
\\	今回の試験は合格したにしても、点数はあまりよくないと思う。	
\\	にしても
\\	仕事で分からないことは、先輩に聞いたらいい。	
\\	たらいい/といい
\\	もう少し給料を上げてくれたらいいのだが。	
\\	たらいい/といい
\\	誰と行ったらいいですか。	
\\	たらいい/といい
\\	その問題についてはこの参考書を読むといい。	
\\	たらいい/といい
\\	もう遅いから、残りの仕事は明日にしたらいい。	
\\	たらいい/といい
\\	痩せたかったら、この薬を飲むといいですよ。	
\\	たらいい/といい
\\	ダイエットをしてから少し痩せてきたが、もっと体調が減ったらいいのに。	
\\	たらいい/といい
\\	僕、財布を無くしたんです。どうしたらいいですか。	
\\	たらいい/といい
\\	この問題に関して質問したいことがあります。	
\\	に関する 
\\	に関して
\\	大きい地震の際には、まず、被害に関する確かな情報を得ることが大切だ。	
\\	に関する 
\\	に関して
\\	彼女が先生に就職に関してアドバイスを受けています。	
\\	に関する 
\\	に関して
\\	その本屋には、留学に関する本だけを集めたコーナーがある。	
\\	に関する 
\\	に関して
\\	彼は経済だけでなく、法律に関しても詳しい。	
\\	に関する 
\\	に関して
\\	日本の歴史に関して研究しようと思います。	
\\	に関する 
\\	に関して
\\	コンピューターの使い方に関して、質問がある方は、わたしのところまでどうぞ。	
\\	に関する 
\\	に関して
\\	三ヶ月ダイエットを続けた。その結果、5キロやせた。	
\\	その結果
\\	全然勉強をしなかった。その結果、入学試験に落ちてしまった。	
\\	その結果
\\	父は、人の何倍も努力した。その結果、仕事で成功した。	
\\	その結果
\\	どんなに苦しくても彼はあきらめなかった。その結果、このような偉大な発明が生まれたのだ。	
\\	その結果
\\	-ない
\\	ため(に)	
\\	のため(に)	
\\	ため(に)	ために
\\	漫画を読むために、日本語を勉強している。	
\\	ために
\\	家族を養うために、一生懸命働いています。	
\\	ために
\\	わたしはパソコンを買うために貯金しています。	
\\	ために
\\	わたしは日本の歴史を研究するために、大学院に入りました。	
\\	ために
\\	健康のために、朝の散歩を日課としている。	
\\	ために
\\	彼は子供のためにタバコをやめました。	
\\	ために
\\	あなたのためなら死んでもいいわ。	
\\	ために
\\	世界平和のために、国際会議が開かれる。	
\\	ために
\\	勉強のために睡眠時間を削った。	
\\	ために
\\	わたしは彼を黙らせるためにあんな事を言ったのよ。	
\\	ために
\\	ひらがなさえ書けないんですから、漢字なんて書けません。	
\\	さえ
\\	わたしは忙しくて、散歩する暇さえない。	
\\	さえ
\\	そんなこと、子供でさえ知っている。	
\\	さえ
\\	リンさんは、日本人でさえ読めない漢字が読める。	
\\	さえ
\\	家族にさえ言えないこともある。	
\\	さえ
\\	山の上には夏でさえ雪が残っている。	
\\	さえ
\\	今のわたしの安い給料では、何年働いても自分の家は買えそうもない。	
\\	そうもない
\\	この雨は止みそうもないので、仕方なく雨の中帰りました。	
\\	そうもない
\\	この渋滞では約束の時間に間に合いそうもない。	
\\	そうもない
\\	もう一度言ってごらん。	
\\	てごらん
\\	二ページを見てごらん。	
\\	てごらん
\\	君はその理由を知っている。考えてごらん。	
\\	てごらん
\\	おいしいから、食べてごらん。	
\\	てごらん
\\	苦くないからちょっと飲んでごらん。	
\\	てごらん
\\	わからなかったら、先生に聞いてごらん。	
\\	てごらん
\\	やったことを思いだしてごらん。	
\\	てごらん
\\	二人とも自分を見てごらんなさいよ。	
\\	てごらん
\\	暑さのため、植物が枯れています。	
\\	ために
\\	一晩中よく眠れなかったため、元気がないのです。	
\\	ために
\\	風邪を引いたために、学校を休みました。	
\\	ために
\\	株価が暴騰したため、市場が混乱しています。	
\\	ために
\\	病気のため、会社を辞めました。	
\\	ために
\\	激しい雨が降ったため、試合が中止された。	
\\	ために
\\	お酒を飲んだとたん、顔が赤くなった。	
\\	たとたん
\\	暑くなるとたん、クーラーの売れ行きはよくなった。	
\\	たとたん
\\	窓を開けたとたん、強い風が入ってきた。	
\\	たとたん
\\	彼女を見たとたん、他の心配事はすべて心から飛び去った。	
\\	たとたん
\\	出発したとたんに囲まれたんだ。	
\\	たとたん
\\	彼は家を出たとたんに、警察に捕まえられた。	
\\	たとたん
\\	立ち上がったとたんに、目まいがした。	
\\	たとたん
\\	角を曲がったとたん、後ろから急ぎ足でやってくる音が聞こえた。	
\\	たとたん
\\	これは、日本の楽器で「尺八」というものです。	
\\	という
\\	彼は疲労ということを知らない。	
\\	という
\\	さっき、木村さんという人から電話がありましたよ。	
\\	という
\\	この「あいじょう」という言葉はどういう意味ですか。	
\\	という
\\	お金はいりません。無料ということです。	
\\	という
\\	この高校では、テニス、サッカー、野球といったスポーツが盛んです。	
\\	という
\\	箱の中から、ガリガリという音が聞こえる。	
\\	という
\\	サイゼリアというレストランを知っていますか。	
\\	という
\\	今週は忙しい、といっても先週ほどじゃない。	
\\	といっても
\\	料理ができるといっても、卵焼きぐらいです。	
\\	といっても
\\	僕は、酒が飲めないといっても、ぜんぜん飲めないわけではない。	
\\	といっても
\\	あの人は、こちらが何度だめだといっても、また頼みに来る。	
\\	といっても
\\	週末は旅行しました。旅行といっても、近くの温泉に行っただけですが。	
\\	といっても
\\	日本語の勉強を始めたといっても、まだ三ヶ月にすぎない。	
\\	といっても
\\	アルバイトの給料をもらったといっても、3万円だけですよ。	
\\	といっても
\\	女優といっても、有名ではありません。	
\\	といっても
\\	ちょっと暑いから、窓を開けてもかまいませんか。	
\\	てもかまわない
\\	テープを聞きながら、この問題用紙にメモをとってもかまいません。	
\\	てもかまわない
\\	体の調子が悪いなら、早めに帰ってもかまいません。	
\\	てもかまわない
\\	すぐに出ていってもかまわない。	
\\	てもかまわない
\\	この本、君には必要ないだろう?捨ててもかまわんな?	
\\	てもかまわない
\\	テレビを見てもかまいません。	
\\	てもかまわない
\\	雨のせいで、せっかく植えた苗が水に流された。	
\\	せいで
\\	熱があるせいで、何を食べてもおいしくない。	
\\	せいで
\\	もし君たちのせいで、僕たちが捕まるようなことになったら、どうする?	
\\	せいで
\\	雨のせいで、運動会が中止になりました。	
\\	せいで
\\	物価が上がったせいか、生活が苦しいです。	
\\	せいで
\\	朝寝坊したせいで、学校に遅れました。	
\\	せいで
\\	ブラッド、こいつのせいで、君はご両親を亡くしたんだぞ。	
\\	せいで
\\	お腹が痛くなったのは食べ過ぎたせいです。	
\\	せいで
\\	昨日、悪夢のせいで何度も目を覚ました。	
\\	せいで
\\	彼は狂ってるというのはほんとうですか?	
\\	というのは
\\	今日は家を出られないんです。というのは父の具合が悪くなりまして…	
\\	というのは
\\	東大というのは、東京大学の略だ。	
\\	というのは
\\	わたしは肉は食べません。というのは、菜食主義だからです。	
\\	というのは
\\	未来を予測するというのは、まさに非常に難しいことですよ。	
\\	というのは
\\	僕は卵を食べないんです。というのは、アレルギーがあるんですよ。	
\\	というのは
\\	法律というのは知っての通り、抜け穴がある。	
\\	というのは
\\	伝説というのは必ず事実に基づいているのではありませんか?	
\\	というのは
\\	試験の結果は70%、つまり合格ということだ。	
\\	ということだ
\\	三浦さんは、近く会社を辞めて留学するということです。	
\\	ということだ
\\	コーチの話では、彼が試合に出れば、優勝はまちがいないということだ。	
\\	ということだ
\\	田中さんから電話があって、少し遅れるということです。	
\\	ということだ
\\	物価は来月からさらに上がるということだ。	
\\	ということだ
\\	ドラゴンボールを読んではじめて、日本文化に興味を持った。	
\\	てはじめて
\\	子を持ってはじめて知る親の恩。	
\\	てはじめて
\\	先生に教えていただいてはじめて、数学の面白さが分かりました。	
\\	てはじめて
\\	こんなに腹が立ったのは生まれて初めてだった。	
\\	てはじめて
\\	入院してはじめて健康のありがたさが分かりました。	
\\	てはじめて
\\	弟はなく彼の方に注意が集まるのは、彼にとって初めての経験だ。	
\\	てはじめて
\\	彼女にとって初めての東京だった。	
\\	てはじめて
\\	わたしは生まれて初めて、彼に会えて嬉しいと思った。	
\\	てはじめて
\\	先生に注意されてはじめて、漢字の間違いに気がついた。	
\\	てはじめて
\\	僕、こんな素敵な家は生まれて初めてだ。	
\\	てはじめて
\\	もうすぐ、今年も終わりですね。ところで、お正月はどうなさいますか?	
\\	ところで
\\	ジョンさんのスピーチをお聞きするのが今からとても楽しみです。ところで、その後は何かご予定がありますか。	
\\	ところで
\\	ところで、どこに行くんだい?	
\\	ところで
\\	明日、試験でしょ。頑張ってね。ところで、来週の月曜日は空いてる?	
\\	ところで
\\	ところで、罰則、どうだった?何をさせられた?	
\\	ところで
\\	その話が本当だとしたら、嬉しいです。	
\\	としたら 
\\	とすれば
\\	飛行機で行くとしたら、いくらぐらいかかりますか?	
\\	としたら 
\\	とすれば
\\	本当にそんなことを彼がしたとしたら理由は何だろう。	
\\	としたら 
\\	とすれば
\\	報告の数字が間違っているとすれば、結論はまったく違うものになるだろう。	
\\	としたら 
\\	とすれば
\\	新しい家を建てるとしたら、かなりのお金が必要になる。	
\\	としたら 
\\	とすれば
\\	この文章に題をつけるとすれば、次のどれが適当か。	
\\	としたら 
\\	とすれば
\\	その大学受験をするとしたら、どんな準備が必要ですか。	
\\	としたら 
\\	とすれば
\\	わたしにはわたしとしての考えがあります。	
\\	として
\\	デパートの店員として商品をきちんとお客様に説明できることは当たり前です。	
\\	として
\\	京都は歴史の古い町として知られている。	
\\	として
\\	彼女は母としても、妻としてもすばらしい女性だ。	
\\	として
\\	わたしは3月から店長になります。店長としての責任感を持って頑張ります。	
\\	として
\\	グラフの説明として正しいものを選びなさい。	
\\	として
\\	年を取るにつれて、体のいろいろな機能が低下する。	
\\	につれて
\\	辺りが暗くなるにつれて、だんだん眠くなってきた。	
\\	につれて
\\	試験が近づくにつれて彼は不安になってきた。	
\\	につれて
\\	台風が近づくにつれて、雨がひどくなってきた。	
\\	につれて
\\	時が経つにつれて、愛が深くなる。	
\\	につれて
\\	年を取るにつれて、経験も豊富になる。	
\\	につれて
\\	町の発展につれて、自然が少なくなった。	
\\	につれて
\\	試合が近づくにつれて、天候は着実に悪くなっていった。	
\\	につれて
\\	みんな彼が勝つと思っていた。ところが、簡単に負けてしまった。	
\\	ところが
\\	山田さんはわたしより若いと思っていた。ところがわたしより5歳も年上だった。	
\\	ところが
\\	彼は非常に頭がいい。ところが、大学に入れなかった。	
\\	ところが
\\	昨夜はコンサートに行くつもりだった。ところが病気で行けなくなった。	
\\	ところが
\\	今朝の天気予報では午後から晴れると言っていた。ところが、予報に反して雨が降り続いている。	
\\	ところが
\\	確かに郵便で送った。ところが、届いていないと言われた。	
\\	ところが
\\	地球環境は、年々悪くなる一方だ。	
\\	一方だ
\\	日本に来る外国人の数は増える一方のようだ。	
\\	一方だ
\\	おじいさんの病気が悪化する一方だった。	
\\	一方だ
\\	景気が悪くて、ここ数年、失業率はあがる一方だ。	
\\	一方だ
\\	交通事故は増える一方だ。	
\\	一方だ
\\	最近の犯罪は悪質になる一方だ。	
\\	一方だ
\\	ここ数年、この町の人口は減る一方だ。	
\\	一方だ
\\	昨年の5月から、今年の春になるまで、雨も降っていなかったので、人々の生活は困難になる一方だ。	
\\	一方だ
\\	直せないから、新しいパソコンを買うしかなかった。	
\\	しかない
\\	彼を信じるしかないんじゃないか?	
\\	しかない
\\	最近の食べ物は安全だとはいえないので、もう自分で作るしかない。	
\\	しかない
\\	できるまで、やるしかない。	
\\	しかない
\\	終電がなくなったので、歩いて帰るしかない。	
\\	しかない
\\	逃げるしかない、すぐに。	
\\	しかない
\\	家内がいないあいだ、自分で料理を作るしかない。	
\\	しかない
\\	病気になったので、しばらく会社を休むしかない。	
\\	しかない
\\	急に雨が降ってきた。傘も持っていないし、荷物もたくさんある。これではタクシー乗るしかないだろう。	
\\	しかない
\\	とおり/どおり	
\\	とおり	
\\	とおり	とおり
\\	わたしの思ったとおりだ。	
\\	とおり
\\	雪が降ってきた。予報どうりだ。	
\\	とおり
\\	わたしの言うとおりにしてください。	
\\	とおり
\\	友達が描いてくれた地図のとおりに来たが、道に迷った。	
\\	とおり
\\	人生は思いどおりにいかないものだ。	
\\	とおり
\\	あなたが言ったとおり、簡単な計画だった。	
\\	とおり
\\	先生が言ったとおり、試験は難しかった。	
\\	とおり
\\	彼は言われたとおりにした。	
\\	とおり
\\	計画のとおりに進める。	
\\	とおり
\\	今日は涼しいというより寒いくらいだった。	
\\	というより
\\	水は冷たく、氷水というより、両足の肌をジリジリ焼く火のように感じられた。	
\\	というより
\\	この部屋は、部屋というより、むしろ地下牢のようだ。	
\\	というより
\\	本当にそうだと確信しているというよりは、そうあって欲しいと思っていた。	
\\	というより
\\	前の車は遅すぎて、走るというよりはっているようだ。	
\\	というより
\\	その音は、人というより動物のようだった。	
\\	というより
\\	あの学生はできないというより、やる気がないのでしょう。	
\\	というより
\\	この薬は説明書をよく読んだ上で、ご使用ください。	
\\	上で
\\	アパートは部屋の中をみた上で買うかどうかを決めたい。	
\\	上で
\\	この書類に署名の上、窓口に提出してください。	
\\	上で
\\	わたしが皆様のご意見をうかがった上で、明日ご報告いたします。	
\\	上で
\\	十分に調査した上でレポートを書き始めた。	
\\	上で
\\	彼らは顔をあわせるたびに喧嘩する。	
\\	たびに
\\	彼はその事件を思い出すたびに、いまでも涙が出るほど笑う。	
\\	たびに
\\	この本を読むたびに、彼のことを思い出す。	
\\	たびに
\\	誕生日のたびに、盛大なパーティーをやります。	
\\	たびに
\\	山田さんは会うたびに髪型が違う。	
\\	たびに
\\	この歌を聴くたびに、大学生時代を思い出す。	
\\	たびに
\\	おじさんは旅行に行くたびに、お土産を買ってきてくれる。	
\\	たびに
\\	最近、彼会うたびに忙しい、忙しいって言ってる。	
\\	たびに
\\	たとえ給料が高くてもあんな仕事はしたくない。	
\\	たとえ~ても
\\	たとえ元気じゃなくても、家族への手紙には元気だと書きます。	
\\	たとえ~ても
\\	たとえ、口外したとしても、誰が信じるというのです?	
\\	たとえ~ても
\\	たとえ両親に反対されても、彼と結婚します。	
\\	たとえ~ても
\\	たとえ貧しくても、親子一緒に暮らせるのが一番だ。	
\\	たとえ~ても
\\	たとえ仕事がつまらなくてもあまり文句を言わないほうがいい。	
\\	たとえ~ても
\\	たとえみんなに反対されても、わたしは絶対にこの計画を実行したい。	
\\	たとえ~ても
\\	たとえ、あなたが諦めても、僕は戦いつづける。	
\\	たとえ~ても
\\	たとえ雨が降っても参ります。	
\\	たとえ~ても
\\	もうすぐテストがあるので、一生懸命勉強しないわけにはいかない。	
\\	わけにはいかない
\\	彼女は「結婚しないと自殺する」と脅したから、結婚しないわけにはいかない。	
\\	わけにはいかない
\\	行くと約束したから、行かないわけにはいかない。	
\\	わけにはいかない
\\	明日は一年で一番忙しい日なので、会社を休むわけにはいかない。	
\\	わけにはいかない
\\	あとで車を運転するので、今お酒を飲むわけにはいきません。	
\\	わけにはいかない
\\	あなたの人生をサイコロを振って決めるわけにはいかないでしょう。	
\\	わけにはいかない
\\	あいつを逃すわけにはいかない!絶対に捕まえる!	
\\	わけにはいかない
\\	どうやって1分で学校に行くの?飛んで行くわけにはいかないし。	
\\	わけにはいかない
\\	彼は知りすぎている。自由にしてやるわけにはいかない。	
\\	わけにはいかない
\\	彼女は姉の娘。つまり、わたしにとっては姪です。	
\\	つまり
\\	それは盗聴器、つまり虫よ。	
\\	つまり
\\	山田さんは、携帯もパソコンも持っていない。つまり、メールで連絡はできないのだ。	
\\	つまり
\\	父の兄、つまりわたしの伯父は、医者をしている。	
\\	つまり
\\	わたしは君の名付親でもあるんだよ。つまり、君の両親が、わたしを君の後見人に決めたのだ。	
\\	つまり
\\	彼は獣医、つまり、動物のお医者さんです。	
\\	つまり
\\	彼は年のわりには若く見える。	
\\	わりに
\\	テストの勉強をしなかったわりには、よくできた。	
\\	わりに
\\	彼女は細いわりには体が丈夫だ。	
\\	わりに
\\	このレストランは値段が高いわりには、おいしくない。	
\\	わりに
\\	お金がないわりに、よく買い物をするね。	
\\	わりに
\\	わたしは甘いものばかり食べるわりにあまり太らない。	
\\	わりに
\\	このピザは安いわりにはおいしい。	
\\	わりに
\\	あの学生は試験の成績がいいわりには、日本語が通じない。	
\\	わりに
\\	このアパートは駅から近いわりには高くない。	
\\	わりに
\\	のついでに	
\\	ついでに	
\\	ついでに	ついでに
\\	郵便局へ行ったついでに、葉書を買ってきた。	
\\	ついでに
\\	仕事で京都に行ったついでに、高校時代の友人に会ってきた。	
\\	ついでに
\\	散歩のついでに、この手紙を出してきてくれない?	
\\	ついでに
\\	大佐、ついでにわたしのところへお寄りになって、お茶でもお飲みくださいませんか?	
\\	ついでに
\\	何か食べに行こう。ついでに買い物も先にしよう。	
\\	ついでに
\\	出張で大阪に行ったついでに、友達の家に寄ってみた。	
\\	ついでに
\\	母は駅まで客を送っていたついでに、買い物をしてきた。	
\\	ついでに
\\	年を取ると共に体が弱ってきたようだ。	
\\	と共に
\\	彼女は家族と共に、アメリカに行くことになった。	
\\	と共に
\\	代表に選ばれなくてくやしいと共に、ほっとする気持ちもあった。	
\\	と共に
\\	始発電車に間に合うように早く家を出た。	
\\	ように
\\	みんなに聞こえるようにもっと大きな声で話してください。	
\\	ように
\\	子供も読めるようにふりがなをつけた。	
\\	ように
\\	みなさん、少し、落ちつくように努力しましょう。	
\\	ように
\\	今年も良い年でありますように。	
\\	ように
\\	忘れないように、手帳に書いておこう。	
\\	ように
\\	みんなに、この時間にここに来るように言っておいたんだ。	
\\	ように
\\	風が入らないように窓を閉めておきましょう。	
\\	ように
\\	わけがない	
\\	わけがない	
\\	-ない
\\	わけがない	わけがない
\\	あの強い相手に勝てるわけがない。	
\\	わけがない
\\	いつも日本語で話をしないと、日本語が上手になるわけがない。	
\\	わけがない
\\	あの記者が自分について書いた記事を、わたしがそんなにすぐに忘れられるわけがない。	
\\	わけがない
\\	うそばかりついている彼が、みんなに信用されるわけがないでしょう。	
\\	わけがない
\\	こんなややこしい問題が小学生にわかるわけがない。	
\\	わけがない
\\	両親が生きていたらどう思うかなんて僕にわかるわけがないじゃないか。	
\\	わけがない
\\	このことはもう彼に話したのだから、彼が知らないわけがないですよ。	
\\	わけがない
\\	こんなに忙しい時期にスキーに行けるわけがない。	
\\	わけがない
\\	お金を盗むなんて、たとえどんな理由があっても両親が許すわけがない。	
\\	わけがない
\\	いくら丈夫でも、こんなに働かされては平気でいられるわけがない。	
\\	わけがない
\\	うちの息子は夢をあきらめようとしない。	
\\	ようとしない
\\	彼女は何も話そうとしない。	
\\	ようとしない
\\	ハチコ、どうしたんだろう。ご飯を食べようとしないんだよ。	
\\	ようとしない
\\	彼は飲酒の習慣を改めようとしない。	
\\	ようとしない
\\	どうして認めようとしないの?	
\\	ようとしない
\\	動けないのか動こうとしないのか、彼は動かない。	
\\	ようとしない
\\	彼は自分のことは何も言おうとしない。	
\\	ようとしない
\\	足が動こうとしないのが問題だ。	
\\	ようとしない
\\	日本の古都というと、京都、奈良でしょう。	
\\	というと
\\	日本代表の漫画といえばワンピースですよね。	
\\	というと
\\	昨日財布を忘れて出かけたが、困ったかといえば、それほどでもなかった。	
\\	というと
\\	手塚治虫に匹敵する漫画家といえば誰?	
\\	というと
\\	その景色の美しさといったら、とても言葉では表現できない。	
\\	というと
\\	ベンさんといえば、大学院に進学したそうですが、今どこに住んでいますか?	
\\	というと
\\	韓国ドラマというと「冬のソナタ」を思い出すという人が多い。	
\\	というと
\\	高いものが必ずしもいいものだとは限らない。	
\\	とは限らない
\\	親や先生の言うことが正しいとは限らない。	
\\	とは限らない
\\	落ち着くんだ。必ずしも君が考えているようなことだとは限らない。	
\\	とは限らない
\\	酒が好きな人は、必ずしも酒に強いとは限りません。	
\\	とは限らない
\\	お金持ちが必ずしも幸福だとは限らない。	
\\	とは限らない
\\	このような好景気がいつまでも続くとは限らない。	
\\	とは限らない
\\	日本人の日本語が必ずしもすべて正しいとは限らない。	
\\	とは限らない
\\	多数意見が常に正しいとは限らない。	
\\	とは限らない
\\	犬があなたの靴をかもうとしているよ。	
\\	ようとする
\\	電車に乗ろうとしたときに、ドアが閉まって乗れなかった。	
\\	ようとする
\\	子供達が迷路を抜けて外に出ようとしている。	
\\	ようとする
\\	彼は必死に逃げようとした。	
\\	ようとする
\\	三人が外に出ようとすると、先生が呼び止めた。	
\\	ようとする
\\	彼らはわたしを見つけようとするに違いない。	
\\	ようとする
\\	ハチコは、わたしが出かけようとすると、ほえる。	
\\	ようとする
\\	彼を助けようとする者は誰もいなかった。	
\\	ようとする
\\	息をしようとすると、頭がクラクラする。	
\\	ようとする
\\	おふろに入ろうとしたとき、誰かが来た。	
\\	ようとする
\\	夜は甘いものを食べないようにしています。	
\\	ようにする
\\	忘れ物をしないようにしてください。	
\\	ようにする
\\	健康のために野菜を食べるようにしています。	
\\	ようにする
\\	彼女は、わたしの部屋には近づかないようにしていた。	
\\	ようにする
\\	再び同じミスを犯さないようにしてください。	
\\	ようにする
\\	毎食後、歯をみがくようにしています。	
\\	ようにする
\\	毎日メールをチェックするようにしている。	
\\	ようにする
\\	彼が正体がばれないようにしている。	
\\	ようにする
\\	明日から、6時に起きるようにします。	
\\	ようにする
\\	日本語は、勉強すればするほど難しくなるような気がする。	
\\	ような気がする
\\	誰かがわたしのことを笑っているような気がします。	
\\	ような気がする
\\	彼は体が凍りついたような気がした。	
\\	ような気がする
\\	今年の試験は去年の試験より難しいような気がする。	
\\	ような気がする
\\	あの箱が今どこにあるのか、彼はそれがわかったような気がした。	
\\	ような気がする
\\	どうも今度の実験は成功できるような気がします。	
\\	ような気がする
\\	悪いことをしたような気がします。	
\\	ような気がする
\\	壁の間からヒソヒソ声が聞こえるような気がした。	
\\	ような気がする
\\	彼女の話の意味がやっと分かった気がします。	
\\	ような気がする
\\	全身の血液が逆流するような気がする。	
\\	ような気がする
\\	左右を見ずに道を渡ると危ないです。	
\\	ずに
\\	ドアに鍵をかけずに買い物に行ってしまった。	
\\	ずに
\\	相手の話も聞かずに自分の主張だけ通そうとするなんて、それはわがままというものだ。	
\\	ずに
\\	「何が見える?」彼が待ちきれずに聞いた。	
\\	ずに
\\	連絡せずに会社を休みました。	
\\	ずに
\\	自分が特別だなんて知らずに、赤ん坊は眠り続けている。	
\\	ずに
\\	昼ごはんを食べずに9時から5時までずっと働きました。	
\\	ずに
\\	フリーターは、自分で就職せずに、きままに働いて、自由を楽しんでいる。	
\\	ずに
\\	クーポンの有効期限が切れないうちに使ってください。	
\\	うちに
\\	雨が降らないうちに家に帰りましょう!	
\\	うちに
\\	まだ見つからないうちに証拠を隠さないといけない。	
\\	うちに
\\	赤ちゃんが寝るうちに昼ご飯を食べましょうか?	
\\	うちに
\\	ママが知らないうちに家を抜け出そう!	
\\	うちに
\\	若いうちに一生懸命勉強しなさい。	
\\	うちに
\\	ラーメンが冷たくならないうちに食べなさい。	
\\	うちに
\\	彼は「さっさといけ、おれの気が変わらないうちに」と私に言った。	
\\	うちに
\\	警察がこないうちに逃げよう!	
\\	うちに
\\	気がつかないうちに、三十歳になった。	
\\	うちに
\\	熱いうちに早く食べましょう。	
\\	うちに
\\	これは泣かずにはいられない感動の映画です。	
\\	ずにはいられない
\\	その無邪気さあふれる笑顔を見たら、誰でも愛せずにはいられないでしょう。	
\\	ずにはいられない
\\	接しているうち、彼女の魅力を感じせずにはいられません。	
\\	ずにはいられない
\\	彼は行儀作法も知らず、食べ物の皿を見れば投げつけずにはいられないようなやつです。	
\\	ずにはいられない
\\	彼に会うのは10年ぶりだ。	
\\	ぶりに
\\	今日、3ヶ月ぶりにまた九州へ撮影に行ってきました。	
\\	ぶりに
\\	一週間ぶりに晴れの日になった。	
\\	ぶりに
\\	彼は三日ぶりに意識を回復した。	
\\	ぶりに
\\	5年ぶりの帰国だから家族がとても喜んだ。	
\\	ぶりに
\\	学校で起きた出来事のおかげで、数週間ぶりに彼女のことなど頭から吹っ飛んだ。	
\\	ぶりに
\\	彼はスポーツの選手だけあって、体格がいい。	
\\	だけあって
\\	鈴木さんは歴史が好きなだけあって、歴史の点数はとても高い。	
\\	だけあって
\\	彼はさすが学生時代にやっていただけあって、今でもテニスが上手だ。	
\\	だけあって
\\	春子さんはアメリカに留学しただけあって、英語がうまいね。	
\\	だけあって
\\	このバッグは安いだけあってすぐに壊れてしまった。	
\\	だけあって
\\	彼は以前オーストラリアに住んでいただけあって、さすがにこの国のことをよく知っている。	
\\	だけあって
\\	彼はチームのキャプテンだけあってみんなに信頼されている。	
\\	だけあって
\\	今日中にファックス、あるいは、メールで送ってください。	
\\	あるいは
\\	あるいは、あなたの言うとおりかもしれません。	
\\	あるいは
\\	その男は不思議に思われるほど何も知らなかった。あるいは、話したくないのかもしれなかった。	
\\	あるいは
\\	来週の月曜日の午前はどうですか。あるいは火曜日の午後でもかまいませんが…	
\\	あるいは
\\	人間が未来を予想できないということは、あるいはいいことかもしれない。	
\\	あるいは
\\	彼女はわかっていた。あるいは、わかっていると思っていた。	
\\	あるいは
\\	な
\\	なだけでなく	
\\	だけでなく	
\\	だけでなく	だけでなく
\\	鈴木さんは字がきれいなだけでなく文章も上手だ。	
\\	だけでなく
\\	あの歌手は歌が上手なだけでなく、自分で曲も作る。	
\\	だけでなく
\\	君は慎み深い人だ。彼が気に入っているだけのことはある。	
\\	だけのことはある
\\	なるほど彼はすぐれた美男であった。ヨーロッパじゅうに美男の名をとどろかしただけのことはある。	
\\	だけのことはある
\\	だって、時差が12時間もあるんだもん。眠くて当たり前よ。	
\\	だって
\\	だって、あなたのガールフレンドはわたしよ!	
\\	だって
\\	だって、二人は友達だろう?	
\\	だって
\\	それはどうでもいいんだ。だって、どうやってそこに行くのかまだわからないんだし…	
\\	だって
\\	だって、あいつのために死ぬ価値なんてないぜ。	
\\	だって
\\	彼は都合が悪いと言っているが、それは口実でしかない。	
\\	でしかない
\\	彼は社長ですが、両親の目から見るといつまでも子供でしかない。	
\\	でしかない
\\	窓は、黒い岩のほんの小さなすきまでしかなかった。人がはいれるほど大きくなかった。	
\\	でしかない
\\	どんな素晴らしい人でも死ぬときは一人の人間でしかない。	
\\	でしかない
\\	どうやら下での論争は終わったみたいね。	
\\	どうやら
\\	彼らはどうやら地面に釘づけになっている。	
\\	どうやら
\\	どうやらあんまり選択の余地はなさそうですね?	
\\	どうやら
\\	どうやら分かれて座らないといけないね。	
\\	どうやら
\\	わたしはどうやら間違っていた。	
\\	どうやら
\\	どうせダメなんだなどと言っていては、何もできなくなるよ。	
\\	どうせ
\\	どうせ汚すのだから、子供に高価な服を着せるは必要ないだろう。	
\\	どうせ
\\	今から行ってもどうせ遅刻だから、行かないことにする。	
\\	どうせ
\\	白状しないと、どうせこの人たちはあなたから無理やり聞き出すじゃない。	
\\	どうせ
\\	僕が渡したとたん、どうせ皆殺しのつもりだろう!	
\\	どうせ
\\	あなたがどうせ信じないことがはっきりわかっている。	
\\	どうせ
\\	どんなふうに護られているのですか?	
\\	ふうに
\\	彼はどんなふうにお亡くなりになったの?	
\\	ふうに
\\	こんなふうにわたしを見放すなんて信じられない。	
\\	ふうに
\\	彼があなたをあんなふうに扱うのは、わたしたちのせいじゃない。	
\\	ふうに
\\	あの人にあんなふうに立ち向かうなんて、あなたはとても勇敢だった。	
\\	ふうに
\\	そんな不思議なこともこの世の中にはあり得る。	
\\	える 
\\	うる
\\	信頼しうる人材を集めることから始めましょう。	
\\	える 
\\	うる
\\	すべてが科学で説明し得るわけではない。	
\\	える 
\\	うる
\\	そんなことがありうるのだろうか?	
\\	える 
\\	うる
\\	彼の取った態度は、わたしには十分理解しうるものであった。	
\\	える 
\\	うる
\\	これがわたしの知りうるすべての情報です。	
\\	える 
\\	うる
\\	ブラジルとの試合が、こんな天候だということもありうる。	
\\	える 
\\	うる
\\	殺人者になりうるなんて、まったく思わなかった。	
\\	える 
\\	うる
\\	彼はまったくうろたえる様子もなく、逆にニッコリ笑った。	
\\	逆に
\\	ケンはソファーのところまで下がった。トニーは逆に前に出た。	
\\	逆に
\\	親は心配して子どもの様子を知りたがっていたし、逆に、家族は無事だと子どもに知らせて、安心させようとしていた。	
\\	逆に
\\	な
\\	なだけましだ	
\\	だけましだ	
\\	い
\\	だけましだ	だけましだ
\\	今度のアルバイト、自給は安いけど、交通費が出るだけましです。	
\\	だけましだ
\\	あなたは仕事があるだけましだよ。わたしはくびになってしまった。	
\\	だけましだ
\\	大学入試のころは雪の日が多い。今日は寒いけれど、雪が降らないだけましだ。	
\\	だけましだ
\\	給料が減ったけれど、首にならないだけましだ。	
\\	だけましだ
\\	覚えるだけは覚えたのだから、後は試験の日を待つだけだ。	
\\	だけは
\\	わたしは助けるだけは助けてやったのだ。	
\\	だけは
\\	恋愛のことは両親にも話すだけは話しました。	
\\	だけは
\\	指示のとおりにやるだけはやったが、いい結果が出るかどうか自信がない。	
\\	だけは
\\	な
\\	げ(な)	
\\	げ	
\\	い
\\	(-い) 
\\	げ(に)	げ
\\	彼女の笑顔にはどこか寂しげなところがあった。	
\\	げ
\\	彼女は苦しげにうめいて横たわった。	
\\	げ
\\	学生たちは楽しげにおしゃべりしていた。	
\\	げ
\\	「大丈夫?」彼は訝しげにもう一度聞いた。	
\\	げ
\\	老人が、何か言いたげに近づいて来た。	
\\	げ
\\	山田は片手で物憂げに髪を掻いた。	
\\	げ
\\	子供たちは庭でいかにも楽しげに遊んでいる。	
\\	げ
\\	「そうか」彼は考え深げに言った。	
\\	げ
\\	部屋中が再び静かになった。	
\\	ふたたび
\\	再びドアが開いた。	
\\	ふたたび
\\	彼が再び苛立ちを見せた。	
\\	ふたたび
\\	鳥は再び飛び去った。	
\\	ふたたび
\\	彼女は再び元気いっぱいになった。	
\\	ふたたび
\\	彼女は寝返りをうち、再び眠りに落ちた。	
\\	ふたたび
\\	いよいよ探していたものを見つけた。	
\\	いよいよ
\\	いよいよ、終わりだぜ。やっと、真実があらわれたんだ。	
\\	いよいよ
\\	いよいよ階段のてっぺんにたどり着いた。	
\\	いよいよ
\\	いよいよ、最後の場面になった。	
\\	いよいよ
\\	汽車がいよいよ速度を落としはじめた。	
\\	いよいよ
\\	一人でやると言った以上、他の人には頼めない。	
\\	以上
\\	学生である以上、学校の規則を守らなければならない。	
\\	以上
\\	試験を受ける以上、何とかして合格したい。	
\\	以上
\\	両親から独立した以上は、仕事を探さなければならない。	
\\	以上
\\	給料がこんなに安い以上は、転職を考える人がいるのも当然だ。	
\\	以上
\\	真相を知った以上、何かしなければならない。	
\\	以上
\\	ここに来てしまった以上、気になっていることは全部言ってしまおう。	
\\	以上
\\	僕以外はみんな知ってたんだ。	
\\	以外
\\	この駐車場は、関係者以外駐車できないことになっているんですよ。	
\\	以外
\\	死にたくない。死は、自分以外のひとびとを訪れるものなのだ。	
\\	以外
\\	それ以外考えられないよ。	
\\	以外
\\	更衣室に入ると、ロナウド以外の選手が全員揃っていた。	
\\	以外
\\	山田以外のみんなが午前中を寝て過ごした。	
\\	以外
\\	トム以外のだれも夫にはしません。	
\\	以外
\\	それ以外の者を連れて来たら二度とこの話はしない。	
\\	以外
\\	確かに、あの清純な、澄んだ目つきは、誠実以外のなにものでもあるはずがない。	
\\	以外
\\	仕事中は水以外の飲みものは禁ぜられていた。	
\\	以外
\\	みんな自分のこと以外は無関心なんだ!	
\\	以外
\\	この部屋は日当たりがいい反面、夏はかなり暑い。	
\\	反面
\\	この車は、空気を汚さない反面、価格が高い。	
\\	反面
\\	この薬はよく効く反面、副作用がある。	
\\	反面
\\	ノートパソコンは手軽である反面、壊れやすい。	
\\	反面
\\	家の近くに駅があると便利な反面、騒音に悩まされることもある。	
\\	反面
\\	織田さんは紳士である反面、子供っぽいところがある。	
\\	反面
\\	フェースブックは便利な反面、トラブルも多い。	
\\	反面
\\	彼はスポーツが上手な反面、勉強はまったくだめです。	
\\	反面
\\	この分野はコンピューターなしでは研究し得ない。	
\\	えない
\\	あり得ない理論ではない。	
\\	えない
\\	ドアを開けないとベルが聞こえないはずだ。	
\\	えない
\\	根性がないと、この目標を達成し得ないよ。	
\\	えない
\\	そんなことは到底あり得ないと思う。	
\\	えない
\\	あなたは完全に嫌疑をまぬがれ得ない。	
\\	えない
\\	学生たちは校庭に出て、今年はあと数日しか味わえないだろうと思われる陽の光を楽しんでいる。	
\\	えない
\\	オリンピック大会は4年ごとに開催されている。	
\\	ごとに
\\	彼女は、言葉の切れ目ごとに、ぶんなぐった。	
\\	ごとに
\\	列車が到着するごとに、ホームは人で手溢れている。	
\\	ごとに
\\	一足ごとに耳をそばだてていた。	
\\	ごとに
\\	この一か月間、彼は二十四時間ごとに気が変わっている。	
\\	ごとに
\\	彼女が一分ごとに時計をチェックする。	
\\	ごとに
\\	でも、あの人もいわゆるまともなくちじゃないだろう?	
\\	いわゆる
\\	野菜やポテトはいわゆる健康食品だ。	
\\	いわゆる
\\	この品物は品質がよいだけに値段も高い。	
\\	だけに
\\	期待が大きかっただけに、落選とわかったときの失望も大きかった。	
\\	だけに
\\	成績が悪いだけに、人一倍努力するよりほかはない。	
\\	だけに
\\	試験のために一生懸命勉強しただけに、不合格のショックは大きかった。	
\\	だけに
\\	彼女は、彼氏と別れたことをちっとも気にかけていない様子だ。	
\\	ちっとも~ない
\\	休みの日なのに、パパはテレビに夢中でちっとも遊んでくれない。	
\\	ちっとも~ない
\\	彼は、恥ずかしそうな顔などちっともしていない。	
\\	ちっとも~ない
\\	彼ならちっとも気にしなかったに違いないとわかっていた。	
\\	ちっとも~ない
\\	彼女は僕の気持なんかちっともわかっちゃいない。	
\\	ちっとも~ない
\\	今度は果たして本当に見えたのかどうか自信がない。	
\\	果たして
\\	果たして外出していいものかどうか完全に自信があったわけではない。	
\\	果たして
\\	わたしがあの店にいるとき、彼らがそこにやってきたのは果たして偶然だろうか?	
\\	果たして
\\	超能力というものは、果たしてあるのだろうか。	
\\	果たして
\\	わたしたちはジェームスが果たして彼女を殺したかどうか、証拠を握っていない。	
\\	果たして
\\	果たして本当に足音が聞こえたのかどうかはわからない。	
\\	果たして
\\	果たしてやってよかったのかどうか、わたしにはいまだに自信がない。	
\\	果たして
\\	みんなで遠足に行こうではないか。	
\\	ではないか
\\	日本語を身につけるために頑張ろうではないか。	
\\	ではないか
\\	ともに新しい世界をつくろうではないか。	
\\	ではないか
\\	家にばかりいないで映画でも見に行こうじゃないか。	
\\	ではないか
\\	さて、お茶にしようじゃないか。	
\\	ではないか
\\	環境汚染の問題について、真剣に考えようではありませんか。	
\\	ではないか
\\	どうやるのか、教えてやろうじゃないか。	
\\	ではないか
\\	もう一度話し合おうではないか。	
\\	ではないか
\\	な
\\	であるかのようだ	
\\	かのようだ	
\\	であるかのようだ	かのようだ
\\	賞をもらったのは弟だというのに、彼女の喜ぶ様子は自分が賞をもらったかのようだ。	
\\	かのようだ
\\	この人形はとてもよく作られていて、生きているかのようだ。	
\\	かのようだ
\\	彼はその間題は忘れたかのようだった。	
\\	かのようだ
\\	彼の話し方はその事件を実際に見ていたかのようだ。	
\\	かのようだ
\\	誰も呼吸をしていないかのようだった。	
\\	かのようだ
\\	彼はまるでこの家の主人であるかのように威張っている。	
\\	かのようだ
\\	急に冷え込んで、今朝はまるで冬が来たかのようだ。	
\\	かのようだ
\\	怖いものでも見たかのように、彼女は震えていた。	
\\	かのようだ
\\	彼は、いかにも全部わかっているかのように話している。	
\\	かのようだ
\\	顔つきから見ると、ほんとうの軍人であろう。	
\\	から見ると
\\	平凡なわたしから見ると、彼女はあらゆる才能に恵まれているように思える。	
\\	から見ると
\\	もっとも彼女の表情から見ると、あのピザはまったくおいしくなさそうだ。	
\\	から見ると
\\	あの様子から見て、彼は、昨晩飲みすぎたようだ。	
\\	から見ると
\\	ゴルフをしない者から見ると、何が面白いのか理解できない。	
\\	から見ると
\\	日本の習慣には、外国人から見ると妙なものもあるだろう。	
\\	から見ると
\\	失業していた池上さんは、今結婚どころではないと思うよ。	
\\	どころではない
\\	先週は試験勉強どころではなかった。毎日、送別会や歓迎会で夜遅くまで帰れなかった。	
\\	どころではない
\\	彼は腹が立って、こんな状況にユーモアを感じるどころではなかった。	
\\	どころではない
\\	風邪がひどくて、遊びに行くどころじゃない。	
\\	どころではない
\\	人の仕事を手伝うどころではありません。自分の仕事も間に合わないんです。	
\\	どころではない
\\	彼らはのんびりするどころではなかい。	
\\	どころではない
\\	車を買ったばかりでお金がないので、海外旅行どころではない。	
\\	どころではない
\\	みんなは眠くて喋るどころではない。	
\\	どころではない
\\	ご要望には応じかねますので、ご了承ください。	
\\	かねる
\\	わたしにはちょっとわかりかねますので、後で担当者からご返事させます。	
\\	かねる
\\	残念ながら、ご提案には賛成しかねます。	
\\	かねる
\\	当ホテルは盗難についての責任を負いかねます。貴重品は各自で保管してください。	
\\	かねる
\\	さあ、約束はできかねる。	
\\	かねる
\\	労働条件の変更について会社から説明を受けたが、わたしはどうも納得しかねる。	
\\	かねる
\\	即答はできかねます。	
\\	かねる
\\	約束したからにはその約束を守らなければならない。	
\\	からには
\\	中国に来たからには、中国文化を徹底的に知りたい。	
\\	からには
\\	何回も話し合って皆で決めたことだ。決めたからには成功するように頑張ろう。	
\\	からには
\\	やるからには最後までやりたいと思います。	
\\	からには
\\	彼が戻った?「戻った」と言うからには、生きていると?	
\\	からには
\\	韓国に住んでいるからには、韓国の法律に従わなければならない。	
\\	からには
\\	試合に出るからには、勝ちたい。	
\\	からには
\\	二ヶ月以上も特訓をしたからには、少しは進歩するものと思っていたのだが。	
\\	からには
\\	子供であるからには、親の世話をするのは当たり前だ。	
\\	からには
\\	必要以上にたくさんのお金を使わないでください。	
\\	以上に
\\	わしはあなたが思っている以上に注意深くあなたを見守ってきた。	
\\	以上に
\\	彼はわたしたちが思っていた以上の粘りを見せた。	
\\	以上に
\\	彼らは、わたしが考えた以上に強い反応を示した。	
\\	以上に
\\	わたしたちは相当いろいろ知っている。あいつの想像以上にいろいろ知っているんだ。	
\\	以上に
\\	彼はいきなり向こうをむいて、立ち去った。	
\\	いきなり
\\	ジェーンはいきなり、ベッドの上に座りなおした。	
\\	いきなり
\\	ほんとうの事故だったんです。彼女はいきなり飛び出してきたんです。	
\\	いきなり
\\	泥棒はいきなりナイフを取り出した。	
\\	いきなり
\\	マイケルはいきなり首を振りはじめた。	
\\	いきなり
\\	角を曲がったところで、前彼にいきなりぶつかった。	
\\	いきなり
\\	彼女はいきなり戸を開けた。	
\\	いきなり
\\	子供に残酷な映画を見せるのは、教育上良くない。	
\\	上
\\	他人の住所を勝手に公表することは、法律上、認められていない。	
\\	上
\\	彼は健康上の理由で仕事をやめました。	
\\	上
\\	男女差別は法律上も許されない。	
\\	上
\\	壁の汚れが気になって、上からペンキを塗ったら、かえって汚くなってしまった。	
\\	かえって
\\	彼女の動揺ぶりを見て、かえって彼は頭がさえてきたようだった。	
\\	かえって
\\	どうしても解けない数学の宿題を兄に説明してもらったら、かえって分からなくなった。	
\\	かえって
\\	彼は足を必死で引っ取ってはみたが、かえって深く沈むばかりだった。	
\\	かえって
\\	その制度を撤廃すれば、かえって悪影響が出てくる可能性がある。	
\\	かえって
\\	ちょっと暑いですから、アイスクリームか何かを食べましょう。	
\\	か何か
\\	それって、縁起の悪いものか何かみたいな気がするんだ。	
\\	か何か
\\	風で紙が飛んでしまうので、本か何か重いものを載せておこう。	
\\	か何か
\\	その女性は、事故か何かにあったのか?	
\\	か何か
\\	さとし、バッグか何か貸してくれない?	
\\	か何か
\\	帰る前に、トーストか何か、少し召しあがらない?	
\\	か何か
\\	父は元気な限りは、働きたいと言っている。	
\\	限り
\\	できる限りの努力をした。後は結果を待つだけだ。	
\\	限り
\\	わたしの見るかぎりでは彼は信頼できる人物だ。	
\\	限り
\\	ここは夜遅く一人で歩かない限り安全である。	
\\	限り
\\	赤ん坊は声のかぎりに泣き喚いている様子だった。	
\\	限り
\\	よほどのことがない限り学校を休まない。	
\\	限り
\\	社会人である限りは、自分の行動に責任を持つべきだ。	
\\	限り
\\	地獄が凍りつきでもしない限り、あなたの仲間にはならない。	
\\	限り
\\	彼はわたしの知るかぎり一番の忘れん坊だ。	
\\	限り
\\	彼から連絡がこないからといって、そんなに心配することないよ。	
\\	からといって
\\	日本人だからといって、すべて日本文化について知っているわけではない。	
\\	からといって
\\	一度ぐらい断られたからといって、そんなに簡単に諦めないでください。	
\\	からといって
\\	親だからといって、勝手に子供の人生を決めるのはあまりよくない。	
\\	からといって
\\	説明を聞いたからといってすぐにできるというものでもない。	
\\	からといって
\\	お金がたくさんあるからといって使いすぎるのはよくない。	
\\	からといって
\\	日本に住んでいるからといって日本語が話せるとは限らない。	
\\	からといって
\\	安いからといって、要らないものまで買うのはよくない。	
\\	からといって
\\	な
\\	ことは 
\\	な
\\	が	
\\	ことは 
\\	が	
\\	い
\\	ことは 
\\	い
\\	が	ことは~が
\\	レポートは最後まで書いたことは書いたが、まだ足りない部分がある。	
\\	ことは~が
\\	家賃が安いことは安いが、交通の便がよくない。	
\\	ことは~が
\\	彼女は美しいことは美しいが、少しそれを鼻にかけている。	
\\	ことは~が
\\	この薬は効くことは効くが、飲むと眠くなってしまう。	
\\	ことは~が
\\	彼女のことは好きなことは好きだけど、結婚したいほどじゃないよ。	
\\	ことは~が
\\	このレストランはおいしいことはおいしいが、値段が高すぎる。	
\\	ことは~が
\\	あんなところは二度と行くまい。	
\\	まい
\\	親に経済的な負担をかけまいとしてアルバイトで生活費を稼いだ。	
\\	まい
\\	何もいますぐ全員を移動させることはあるまい。	
\\	まい
\\	自分の目で確かめない限り、そんな恐ろしいことは誰も信じまい。	
\\	まい
\\	公園を作るのに反対する人はあるまい。	
\\	まい
\\	彼は絶対に彼女の方を見るまいと心を決めていた。	
\\	まい
\\	彼にはわたしの気持ちはわかるまい。	
\\	まい
\\	彼らは背後の低く響く声を聞くまいと努力した。	
\\	まい
\\	彼は何も言うまいと思っているらしい。	
\\	まい
\\	日本語学校の授業では、新聞も読めば作文も書く。	
\\	も~ば~も
\\	このレストランは、値段も安ければ味もいい。	
\\	も~ば~も
\\	あそこの家の父親は、毎日朝から酒を飲んで暴れている。息子は、仕事もせず賭けごとに夢中になっている。まったく、父も父なら子も子だ。	
\\	も~ば~も
\\	このバッグは値段も手頃なら、デザインもかわいくてすぐ買ってしまった。	
\\	も~ば~も
\\	父はお酒も飲めばたばこも吸うので、健康が心配だ。	
\\	も~ば~も
\\	雨も降っていれば、風も吹いています。	
\\	も~ば~も
\\	彼は歌も上手なら踊りもうまい、パーティーの人気者だ。	
\\	も~ば~も
\\	この手紙は筆跡からして彼が書いたものに違いない。	
\\	からして
\\	わたしは彼がきらいだ。その態度からして許せない。	
\\	からして
\\	タイトルからして、面白そうな本だ。	
\\	からして
\\	足音の重さからして、彼女ではない。	
\\	からして
\\	このレストランは雰囲気からして結構高そうだね。	
\\	からして
\\	彼は怒っていることは、その声からして明白だった。	
\\	からして
\\	彼は礼儀を知らない。挨拶からして、きちんとしていない。	
\\	からして
\\	息子さんの態度からして、自分の行為を何ら後悔していないことは明らかです。	
\\	からして
\\	ベルが鳴るかならないかのうちに、先生が教室に入ってきた。	
\\	か~ないのうちに
\\	彼が汽車が止まるか止まらないうちにホームに飛び降りた。	
\\	か~ないのうちに
\\	娘が出かけるか出かけないかのうちに雨が降り出した。	
\\	か~ないのうちに
\\	今朝まだ目が覚めるか覚めないかのうちに、友達が迎えに来た。	
\\	か~ないのうちに
\\	息子は「いってきます」と言い終わるか終わらないかのうちに玄関を飛び出していった。	
\\	か~ないのうちに
\\	彼らが向きを変えるか変えないうちに、小男がテーブルの下から突然現れた。	
\\	か~ないのうちに
\\	彼女は「いただきます」と言うか言わないかのうちに食べ始めた。	
\\	か~ないのうちに
\\	「もう着くころだ」彼の言葉が終わるか終わらないうちに、汽車が速度を落としはじめた。	
\\	か~ないのうちに
\\	ボーナスもくれない会社でもう働くものか。	
\\	ものか
\\	信じてなるものか。	
\\	ものか
\\	僕に何ができるかなど、君にわかるものか。	
\\	ものか
\\	お前が思うことを、誰が気にするものか?	
\\	ものか
\\	あの店のサービスは最低だ。二度と行くものか。	
\\	ものか
\\	わたしがお金など欲しいものか?	
\\	ものか
\\	もう二度と恋こそするものかと固く心に誓ったはずだったが、知らず知らずのうちに彼女のことが好きになっていた。	
\\	ものか
\\	かまうものか。	
\\	ものか
\\	逃げるものか!わたしは犯人じゃない!	
\\	ものか
\\	汚い手段を使わないからといって、使えないということにはならない。	
\\	ことにはならない
\\	本やインターネットの資料を写しただけではレポートを書いたことにはならない。	
\\	ことにはならない
\\	できるからといってやるべきだということにはならない。	
\\	ことにはならない
\\	体重が減っても体脂肪が増えているならダイエットに成功したことにはならない。	
\\	ことにはならない
\\	この問題を一分で解けるものなら解いてみろ。	
\\	ものなら
\\	それぐらいの練習で相手に勝てるものなら勝ちたいです。	
\\	ものなら
\\	あの日の記憶を消せるものなら消してしまいたい。	
\\	ものなら
\\	少年時代に戻れるものなら戻ってみたい。	
\\	ものなら
\\	スケジュールがつまっているが、兄が結婚するので、帰れるものなら、今すぐ国へ帰りたい。	
\\	ものなら
\\	いまのこの暗さを少しでも明るくしていられるものなら、彼にはありがたかった。	
\\	ものなら
\\	みんなダイエットしたがっているが、わたしは太れるものなら太りたい。	
\\	ものなら
\\	やれるものならやるがいい。	
\\	ものなら
\\	僕は、あいつを殺しはしない、そうだろ?だが、できるものならやってやる。	
\\	ものなら
\\	娘が勉強を始めたかと思えば、もう居間でテレビを見ている。	
\\	かと思ったら 
\\	かと思うと
\\	彼は家に着いたかと思うと玄関に倒れてしまった。	
\\	かと思ったら 
\\	かと思うと
\\	姉は出かけたかと思うとすぐ忘れ物を取りに帰ってきた。	
\\	かと思ったら 
\\	かと思うと
\\	弟は、さっき起きてきたかと思ったら、また寝ている。	
\\	かと思ったら 
\\	かと思うと
\\	急に空が暗くなったかと思うと、激しく雨が降り始めた。	
\\	かと思ったら 
\\	かと思うと
\\	彼女は忙しい人で、来たかと思ったら、もう帰ってしまった。	
\\	かと思ったら 
\\	かと思うと
\\	田中さんは「さようなら」と言ったかと思うと教室を飛び出していった。	
\\	かと思ったら 
\\	かと思うと
\\	こういう態度を取られるからこそ、わたしは彼が嫌いなのだ。	
\\	からこそ
\\	早く起きたからこそ始発電車に間に合ったのだ。	
\\	からこそ
\\	彼は悲劇を味わったからこそ、偉大な人間性を手に入れたのではないか。	
\\	からこそ
\\	その辛い一年があったからこそ、今のわたしがあると言えよう。	
\\	からこそ
\\	山本さんは20年間休むことなく会社に通った。	
\\	ことなく
\\	彼は何度失敗してもあきらめることなく、研究を続けた。	
\\	ことなく
\\	我々は苦悩の底に沈むことなく、抗い続けなければならない。	
\\	ことなく
\\	彼女は一日も休むことなく小説を書き続けている。	
\\	ことなく
\\	失敗を恐れることなく、行動してほしい。	
\\	ことなく
\\	時は止まることなく流れる。	
\\	ことなく
\\	彼は怒ることなく、冷静な態度で応対した。	
\\	ことなく
\\	あの人は毎年忘れることなく、誕生日プレゼントをくれる。	
\\	ことなく
\\	隣の人たちは近所の人の迷惑もかまわず毎晩遅くまで騒いでいる。	
\\	もかまわず
\\	電車の中で人目もかまわず、化粧をする女性がいる。	
\\	もかまわず
\\	彼は雨にぬれるのもかまわず歩き続けた。	
\\	もかまわず
\\	酔っ払った若者は、時間もかまわず大声で歌い続けた。	
\\	もかまわず
\\	人に迷惑もかまわず、電車の中で携帯電話で話している人がいる。	
\\	もかまわず
\\	彼女は自分の命もかまわず子供を救うため、火の中に飛び込んだ。	
\\	もかまわず
\\	彼は、怪我をした足が痛むのもかまわず、工事現場を見て回った。	
\\	もかまわず
\\	彼女は人目もかまわず子供のように泣いている。	
\\	もかまわず
\\	最近、中国の経済発展には目覚しいものがある。	
\\	ものがある
\\	校長のスピーチには人々を感動させるものがあった。	
\\	ものがある
\\	こんな単純な作業を3時間も続けるのはつらいものがある。	
\\	ものがある
\\	彼の話はどこか納得できないものがある。	
\\	ものがある
\\	彼女の演奏には、人の心を動かすものがある。	
\\	ものがある
\\	彼の説明にはどうも不自然なものがある。	
\\	ものがある
\\	満員電車で毎日通勤するのはつらいものがある。	
\\	ものがある
\\	そのドラマは観衆を引き付けるものがある。	
\\	ものがある
\\	今度の打ち合わせは土曜日です。なお、時間は後ほどお伝えします。	
\\	なお
\\	京都行きの列車にお乗りください。なお、旅費その他として五千円同封しました。	
\\	なお
\\	この件の説明は以上です。なお、詳細についてはプリントをご覧ください。	
\\	なお
\\	コンサートチケットをご希望の方はお申し込みください。なお、希望者多数の場合は先着順とさせていただきます。	
\\	なお
\\	働く気になれない。	
\\	気
\\	さっきの態度からすると、あの人は謝る気は全然なさそうだ。	
\\	気
\\	何もする気になれない。	
\\	気
\\	もともと勉強する気がないのなら、大学を受けてみたところで何の意味があるのか。	
\\	気
\\	たけしの事はどうする気?	
\\	気
\\	最近ムカムカとして食べる気にもなれない。	
\\	気
\\	この問題は難しくて、この子にはわかりっこない。	
\\	っこない
\\	こんなに弱い選手ばかりでは、次の試合に勝てっこない。	
\\	っこない
\\	あなたには理解できっこない。	
\\	っこない
\\	この子、まだ2歳だよ。漢字なんて読めっこないよ。	
\\	っこない
\\	あなたが言わなければ、他の人は知りっこないよ。	
\\	っこない
\\	彼があなたを傷つけるなんてできっこない。	
\\	っこない
\\	1日10時間も勉強するなんて、僕にはできっこないよ。	
\\	っこない
\\	走っても始発電車に間に合いっこない。	
\\	っこない
\\	安全運転をしている人だから、事故なんて起こしっこないよ。	
\\	っこない
\\	子供の能力からすると、どんなに頑張っても親の要求水準に答えることはできません。	
\\	からすると 
\\	からすれば
\\	症状からすると、心臓の病気かもしれません。	
\\	からすると 
\\	からすれば
\\	あの車は形からすると20年ぐらい前のものだと思う。	
\\	からすると 
\\	からすれば
\\	彼女の能力からすれば、東大に十分合格できるだろう。	
\\	からすると 
\\	からすれば
\\	このチームの今の実力からすると、勝つのはかなり難しい。	
\\	からすると 
\\	からすれば
\\	周りの態度からすると、あの人が有名人ではないでしょうか。	
\\	からすると 
\\	からすれば
\\	聞いた話からすると、彼が自分でこれを発明したらしい。	
\\	からすると 
\\	からすれば
\\	親からすると、何歳になっても子供は子供なのだ。	
\\	からすると 
\\	からすれば
\\	顔つきからするとみんな十七、八以上に見えた。	
\\	からすると 
\\	からすれば
\\	時間の正確な彼女のことだから、もうすぐ来ますよ。	
\\	ことだから
\\	有能な木村さんのことだから、きっといい仕事をするだろう。	
\\	ことだから
\\	お酒が好きな山田さんのことだから、このワインをあげたら、さぞ喜ぶだろう。	
\\	ことだから
\\	時間にルーズな彼のことだから、今日も遅れてくるだろう。	
\\	ことだから
\\	スポーツなら何でもやる鈴木さんのことだから、きっとスキーも上手でしょう。	
\\	ことだから
\\	あなたのことだからきっといい論文が書けると思います。	
\\	ことだから
\\	けちなあの人のことだから、こんな高い会費を払うわけはない。	
\\	ことだから
\\	あのチームのことだから、今年も優勝まちがいなしだ。	
\\	ことだから
\\	授業中は日本語だけを話すことになっている。	
\\	ことになっている
\\	三浦さんとの約束では、6時までに池袋駅に行くことになっている。	
\\	ことになっている
\\	日本では目上の人には敬語を使うことになっている。	
\\	ことになっている
\\	11時以降は外出禁止ということになっています。	
\\	ことになっている
\\	休むときは学校に連絡しなければならないことになっています。	
\\	ことになっている
\\	今度の日曜日には、彼氏と会うことになっている。	
\\	ことになっている
\\	70点以上が合格ということになっている。	
\\	ことになっている
\\	この部屋には、関係者以外入ってはいけないことになっている。	
\\	ことになっている
\\	彼だけが涼しい顔だ。むしろかなり楽しんでいるようにさえ見えた。	
\\	むしろ
\\	彼は死ぬことをいっこうに気にしていないようだった。むしろ、喜んでいるようでさえあった。	
\\	むしろ
\\	役に立つどころか、むしろ逆効果な警告だ。	
\\	むしろ
\\	幸せそうな顔など一つもない。むしろ、五歳児扱いされて、みな愕然とした顔だ。	
\\	むしろ
\\	これはむしろあなたの領域だ。	
\\	むしろ
\\	中を/中では	
\\	中を/中では	
\\	の中を/の中では	中を
\\	雨の降る中を3時間も歩き続いた。	
\\	中を
\\	わたしの聞いた中では、今までで最高の演奏だった。	
\\	中を
\\	本日はお忙しい中をお出でいただき、本当にありがとうございます。	
\\	中を
\\	男の子はかばんを頭の上に載せて雨の中を走り去ってしまった。	
\\	中を
\\	彼は人混みの中をすり抜けた。	
\\	中を
\\	彼はまったく彼女を傷つける様子がなかった。	
\\	まったく~ない
\\	もしかしたら彼とはまったく関係がないかもしれない。	
\\	まったく~ない
\\	ボールはわたしにはまったく見えない。	
\\	まったく~ない
\\	あなたには罰則がまったく効いていないようだ。	
\\	まったく~ない
\\	彼女はこのふざけをまったく認めなかった。	
\\	まったく~ない
\\	あの選手は子供のころ、野球に限らずスポーツならなんでも得意だったそうだ。	
\\	に限らず
\\	ナルトは、男性に限らず女性にも人気があるそうだ。	
\\	に限らず
\\	今に限らずこれまでずっと彼は学校の嫌われ者だ。	
\\	に限らず
\\	この大学は日本人に限らず、世界各国からの留学生を受け入れている。	
\\	に限らず
\\	見合い結婚は、韓国に限らず他の国でも行われている。	
\\	に限らず
\\	ディズニー映画は子供に限らず、大人にも人気がある。	
\\	に限らず
\\	自分の傘は人に貸してしまったものだから、わたしはしかたなく、濡れて帰った。	
\\	ものだから
\\	娘があまりに楽しみにしていたものだから、遊園地に行けなくなったことをすぐには言い出せなかった。	
\\	ものだから
\\	出かけにお客が来たものだから、遅れてしまいました。	
\\	ものだから
\\	英語が苦手なものだから、スチュワーデスになれなかった。	
\\	ものだから
\\	父が病気なものですから、わたしが変わりにうかがいました。	
\\	ものだから
\\	以前からほしかった本がやっと手に入ったものだから、早速徹夜しておわりまで読んでしまった。	
\\	ものだから
\\	言葉が通じないものだから、友達ができない。	
\\	ものだから
\\	家が狭いものですから、大きい家具は置けません。	
\\	ものだから
\\	今年こそヨーロッパへ行きたいと思っています。もっとも休みが取れたらの話ですが。	
\\	もっとも
\\	彼がお亡くなりになったと開いて残念だった。もっとも、もちろん、予期せぬことではなかった。	
\\	もっとも
\\	調査の前夜から飲食禁止です。もっとも水は飲んでもかまいません。	
\\	もっとも
\\	ウィキペディアに全部書いてあるよ。もっともこのサイトは完全には信用できないけど。	
\\	もっとも
\\	その話を聞いて、もう少しで泣くところだった。	
\\	もう少しで
\\	もう少しで倒れそうになった。	
\\	もう少しで
\\	もう少しで秘密を漏らしそうだった。	
\\	もう少しで
\\	もう少しでうまくいきそうだったのに、邪魔が入ってしまった。	
\\	もう少しで
\\	彼はもう少しで道路の柵に自動車をぶつけるところだった。	
\\	もう少しで
\\	もう少しでおぼれるところだった。	
\\	もう少しで
\\	もう少しで電車に乗り遅れそうになった。	
\\	もう少しで
\\	もう少しで大切な卒業論文をタクシーの中に忘れるところだった。	
\\	もう少しで
\\	計算を間違った理由は、コンピュータへの入力ミスにほかならない。	
\\	にほかならない
\\	今回の失敗の原因は、準備不足にほかならない。	
\\	にほかならない
\\	この大学の学生の質がいいのは選抜がきびしいからにほかならない。	
\\	にほかならない
\\	彼が成功したのは、長年の苦労の結果にほかならない。	
\\	にほかならない
\\	それはまったくの誤解にほかならない。	
\\	にほかならない
\\	彼があなたをからかうのは、まさに愛情表現にほかならない。	
\\	にほかならない
\\	戦争というものは、大量殺人にほかならない。	
\\	にほかならない
\\	事故が発生した原因は、運転手の前方不注意にほかならない。	
\\	にほかならない
\\	仕事を引き受けたものの、できるかどうか心配です。	
\\	ものの
\\	いつも勉強しようと考えているものの、つい遊んでしまう。	
\\	ものの
\\	三日でレポートを書くとは言ったものの三日ではとても書けなかった。	
\\	ものの
\\	スポーツクラブに入会したものの、仕事が忙しくてなかなか行けない。	
\\	ものの
\\	祖母は元気なものの、年を取ってあまり外に出なくなった。	
\\	ものの
\\	みんなに向かって強がりを言ったものの、試合が近づくにつれて彼は不安になってきた。	
\\	ものの
\\	やっと眠りに落ちたものの、夢にうなされた。	
\\	ものの
\\	英語は大丈夫だと言ったものの実は自信がない。	
\\	ものの
\\	申し込みはしたものの、試験を受けるかどうか未定だ。	
\\	ものの
\\	誰かを信じないではいられないよ。	
\\	ないではいられない
\\	父が病気だと聞いて、国に帰らないではいられない。	
\\	ないではいられない
\\	何をすべきか、もう心は決まっている。行かないではいられない。	
\\	ないではいられない
\\	試験の前だから、勉強しないではいられません。	
\\	ないではいられない
\\	その面白い話を聞いて、笑わないではいられない。	
\\	ないではいられない
\\	態度の悪い店員に、一言文句を言わないではいられなかった。	
\\	ないではいられない
\\	先輩は面接にあたっての注意事項を教えてくれた。	
\\	にあたり
\\	日本に入るにあたって、我々の法律に従うことを受け入れねばならない。	
\\	にあたり
\\	マンションを買うにあたって、わたしども夫婦はいろいろ調査をしました。	
\\	にあたり
\\	彼は最期にあたってわたしを守ろうとしたのだ。	
\\	にあたり
\\	アルバイトをするにあたっては、学業や体に無理のないようにすること。	
\\	にあたり
\\	新学期を迎えるにあたり、いくつかお知らせがあります。	
\\	にあたり
\\	新年を迎えるにあたって、一年の計画を立てた。	
\\	にあたり
\\	その問題に限ってわたしには分からない。	
\\	に限って
\\	遅刻してはいけない時に限って寝坊してしまう。	
\\	に限って
\\	彼に限って、嘘をつくようなことはしない。	
\\	に限って
\\	見たいドラマがある日に限って残業で帰りが遅くなってしまう。	
\\	に限って
\\	100名様に限りすてきな商品をプレゼントいたします。	
\\	に限って
\\	毎週火曜日に限って、映画券が半額になります。	
\\	に限って
\\	雨だ。今日に限って傘を持ってこなかった。	
\\	に限って
\\	彼は、いつも家にいるのに、今日に限って留守でした。	
\\	に限って
\\	うちの子に限って、ゲームが大嫌いだ。	
\\	に限って
\\	彼がやらねばならない。	
\\	ねばならない
\\	我々は今日は勝たねばならないのだ。	
\\	ねばならない
\\	ピストルはどうしても見つけねばならない。	
\\	ねばならない
\\	連行するつもりなら、きみを傷つけねばならなくなる。	
\\	ねばならない
\\	自信がなくはないが、ただちょっと緊張している。	
\\	なくはない
\\	野菜を食べなくはないんですが、あまり好きではありません。	
\\	なくはない
\\	あの人の性格を考えると、理解できなくはない。	
\\	なくはない
\\	それとはまったくべつの原因も考えられなくはないのです。	
\\	なくはない
\\	結婚するつもりはなくもないが、仕事が安定してからのほうがいいだろう。	
\\	なくはない
\\	あなたの気持ちは分からなくもないが、ただその行為には賛成できない。	
\\	なくはない
\\	それくらいのマンションなら、無理すれば買えなくもない。	
\\	なくはない
\\	兄は夜遅くまで起きているにもかかわらず、毎朝一番早く起きる。	
\\	にもかかわらず
\\	彼は、冬休みにもかかわらず、毎日図書館で勉強している。	
\\	にもかかわらず
\\	交通問題の深刻化にもかかわらず、改善される見込みが全然ない。	
\\	にもかかわらず
\\	野球大会の当日は、激しい雨が降っていた。それにもかかわらず大会は実施された。	
\\	にもかかわらず
\\	多くの人が不可能だと思っているにもかかわらず、あの人は新発明のための研究をあきらめようとしない。	
\\	にもかかわらず
\\	わたしたちの警告にもかかわらず、彼は森に戻ってきた。	
\\	にもかかわらず
\\	彼は熱があるにもかかわらず、サッカーの試合に出場した。	
\\	にもかかわらず
\\	まだ若いにもかかわらず、きみは特別であることを証明していた。	
\\	にもかかわらず
\\	なにも、皆の前でそんなに緊張しなくてもいいでしょう。	
\\	なにも~ない
\\	彼が冗談を言ったのだから、なにもそんなに怒ることはないじゃないか。	
\\	なにも~ない
\\	なにもこんな騒ぎに妹を引合いに出すことはなかろう!	
\\	なにも~ない
\\	友達と一緒に行くのだから、なにもそんなに心配する必要がないよ。	
\\	なにも~ない
\\	子供の言うことを馬鹿にするものではない。	
\\	ものではない
\\	お客さまにそんな態度をとるものではない。	
\\	ものではない
\\	この仕事は一時間で済ませるものではない。	
\\	ものではない
\\	消費者を騙すようなことをするものではない。	
\\	ものではない
\\	小さい子を一人で遠くに遊びに行かせるものではない。	
\\	ものではない
\\	外国語は一ヶ月で身に付けるものではない。	
\\	ものではない
\\	人の陰口と言うものではありません。	
\\	ものではない
\\	口の中にものを入れたまましゃべるものではない。	
\\	ものではない
\\	わたしはあなたがこれにかかわりがあるなどとは一言も言ってはいない。	
\\	にかかわる
\\	それは、個人の名誉にかかわる問題だ。	
\\	にかかわる
\\	睡眠剤は、量が多すぎると、生命にかかわるのだ。	
\\	にかかわる
\\	それから、母は病気になった―命にかかわる病気にね。	
\\	にかかわる
\\	国民の生活にかかわる大切な問題だから、早く解決しなければならない。	
\\	にかかわる
\\	天候にかかわらず、明日旅行に行きます。	
\\	にかかわらず
\\	テニスは年齢にかかわらず楽しめるスポーツだ。	
\\	にかかわらず
\\	住民の反対運動にかかわらず建設が始まった。	
\\	にかかわらず
\\	明日来る来ないにかかわらず、必ずわたしに電話してください。	
\\	にかかわらず
\\	好きか嫌いかにかかわらず、この仕事は必ずしなければならない。	
\\	にかかわらず
\\	経験のあるなしにかかわらず、誰でも参加することができます。	
\\	にかかわらず
\\	両親は賛成するかしないかにかかわらず、僕は君と結婚します。	
\\	にかかわらず
\\	国籍にかかわらず、歓迎します。	
\\	にかかわらず
\\	予算に応じて、旅行のコースを決める。	
\\	に応じて
\\	収入に応じて生活様式を変える。	
\\	に応じて
\\	みんな自分の努力に応じて成績を得たと思います。	
\\	に応じて
\\	この店では、買った量に応じて商品の値段が安くなります。	
\\	に応じて
\\	会社の変化に応じて様々な問題が生じてきた。	
\\	に応じて
\\	旅行のプランは、お客様のご希望に応じて変更できます。	
\\	に応じて
\\	ソフトは、必要に応じてダウンロードしてください。	
\\	に応じて
\\	この数字は、次の集会の日付けと時間に応じて変化します。	
\\	に応じて
\\	夜も寝ないで勉強したのだから、合格するに決まっている。	
\\	に決まっている
\\	一人で外国へ旅行するなんて、親に反対されるに決まっている。	
\\	に決まっている
\\	あの選手は今まで誰にも負けたことがないから、優勝するに決まっていますよ。	
\\	に決まっている
\\	無罪になるに決まっている。	
\\	に決まっている
\\	同じ値段なら、質がいいほうがたくさん売れるに決まっている。	
\\	に決まっている
\\	レオナルド・ダ・ヴィンチくらい聞いたことあるに決まってるだろ。	
\\	に決まっている
\\	あのチームが勝つに決まっています。	
\\	に決まっている
\\	子供にそんなお菓子を見せたらほしがるに決まっている。	
\\	に決まっている
\\	そんなうまい話はうそにきまっていますよ。	
\\	に決まっている
\\	予想に反して、試験の成績が悪かった。	
\\	に反して
\\	それは上司の意思に反してやったことだ。	
\\	に反して
\\	弟は、親の期待にはんすることばかりしている。	
\\	に反して
\\	「好きだ」その言葉は、彼の意志に反して、むしり取られて出てきたようだった。	
\\	に反して
\\	神の意向に反して、人間は自然を破壊している。	
\\	に反して
\\	予想に反して、彼女を好きになってきたのかな?	
\\	に反して
\\	我々の期待に反して、新商品はあまり売れなかった。	
\\	に反して
\\	予想に反して試験はとてもやさしかった。	
\\	に反して
\\	天気予報に反して雨が降り続いている。	
\\	に反して
\\	彼は父親の希望を反して、彼女と結婚した。	
\\	に反して
\\	若いにせよ、毎日残業したら病気になるよ。	
\\	にしろ 
\\	にせよ
\\	たった三日の旅行にせよ、準備は必要だ。	
\\	にしろ 
\\	にせよ
\\	熱が下がったにせよ、しばらく安静が必要です。	
\\	にしろ 
\\	にせよ
\\	上司に命令されてやっとことであるにしろ、責任はきみにもある。	
\\	にしろ 
\\	にせよ
\\	欠席するにせよ連絡してください。	
\\	にしろ 
\\	にせよ
\\	誰がやっているにせよ、取っ捕まえたいものだ。	
\\	にしろ 
\\	にせよ
\\	入試の結果が悪いにせよ通知だけはください。	
\\	にしろ 
\\	にせよ
\\	いくら忙しいにせよ食事をしないのはよくない。	
\\	にしろ 
\\	にせよ
\\	あの小説家はいつも事実に基づいて小説を書いている。	
\\	に基づいて
\\	調査結果に基づいて論文を書いた。	
\\	に基づいて
\\	調査した資料に基づいて、レポートを書いてください。	
\\	に基づいて
\\	これまでの研究に基づいてレポートをまとめた。	
\\	に基づいて
\\	この工事は市の計画に基づいて進められます。	
\\	に基づいて
\\	虚偽の証言に基づいたこの裁判を認めることはできない。	
\\	に基づいて
\\	この番組は、皆様からのご意見に基づいて構成されています。	
\\	に基づいて
\\	に限る	
\\	に限る	
\\	-ない
\\	に限る	に限る
\\	疲れたときはお風呂に入るに限る。	
\\	に限る
\\	調子が悪いときは、ゆっくり休むに限ります。	
\\	に限る
\\	うまい!やはり夏は冷たいビールに限るね。	
\\	に限る
\\	雨の日は外に出ずに一日中寝るに限ります。	
\\	に限る
\\	太りたくなければ、カロリーの高いものを食べないに限る。	
\\	に限る
\\	寒い日は、熱々の鍋に限る。	
\\	に限る
\\	風邪を引いた時は、部屋を暖かくして寝るに限る。	
\\	に限る
\\	行くにしても行かないにしても、早く決めてください。	
\\	にしても~にしても
\\	高いにしても安いにしても、必要なものは買わなければならない。	
\\	にしても~にしても
\\	借りるにしても買うにしても、この家の広さがちょうどいいです。	
\\	にしても~にしても
\\	中田にしても尾田にしてもこの仕事に向いているとは思えない。	
\\	にしても~にしても
\\	親にしたら、娘が立派な弁護士になったのは嬉しいことだろう。	
\\	にしたら
\\	留学は今すぐではなく、大学を卒業してからします。両親にしたら、そのほうが安心でしょう。	
\\	にしたら
\\	学生にすれば、宿題は少なければ少ないほどいいだろう。	
\\	にしたら
\\	あの人の立場にしたらそんなことはできなかっただろう。	
\\	にしたら
\\	彼氏に振られたから、彼女にしたら、かなり落ち込んでいるでしょう。	
\\	にしたら
\\	犬や猫はかわいいが、飼っていない人にすれば迷惑なこともある。	
\\	にしたら
\\	突然父の病気を告げられて、彼女にしたらきっとショックだっただろう。	
\\	にしたら
\\	彼女にしたら、わたしたちの親切はかえって迷惑かもしれません。	
\\	にしたら
\\	学生の希望にこたえて、図書館は夜9時まで開けられることになった。	
\\	に応えて
\\	両親の期待にこたえて、わたしはイギリスに留学した。	
\\	に応えて
\\	市民の希望にこたえて、あそこに公園を作ることになった。	
\\	に応えて
\\	社員の要求にこたえて、労働時間を短縮した。	
\\	に応えて
\\	先生の期待にこたえて、必死でピアノの練習をしている。	
\\	に応えて
\\	あの選手は国民の期待にこたえて、金メダルを手にした。	
\\	に応えて
\\	消費者の要求にこたえて、よい製品を作っていかなければならない。	
\\	に応えて
\\	客の意見にこたえて、営業時間を延長する。	
\\	に応えて
\\	社員の要求にこたえ、給料を高めた。	
\\	に応えて
\\	山道に沿って進んでいくと広い湖に着きますよ。	
\\	に沿って
\\	計画に沿って実施されているので、問題がないと思う。	
\\	に沿って
\\	彼は湖に沿って歩き出した。	
\\	に沿って
\\	先生のあとに従いて、二人は小部屋の列に沿って歩き、一番奥の部屋に行った。	
\\	に沿って
\\	通りに沿って食べ物を売る店が並んでいる。	
\\	に沿って
\\	テーブルに沿って進み、隣の部屋に入ってください。	
\\	に沿って
\\	市民の要望に沿って、古い家屋を取り壊した。	
\\	に沿って
\\	病気で病院へ行ったといっても、ただの風邪にすぎない。	
\\	に過ぎない
\\	いくら働いても一ヶ月の収入は20万円にすぎない。	
\\	に過ぎない
\\	日本語の勉強を始めたといっても、まだ六ヶ月にすぎない。	
\\	に過ぎない
\\	その声はわたしの想像にすぎなかった。	
\\	に過ぎない
\\	英語が話せるといっても、簡単な会話ができるにすぎません。	
\\	に過ぎない
\\	単に幸運だったにすぎない。	
\\	に過ぎない
\\	来年大地震があるというのは、うわさにすぎない。	
\\	に過ぎない
\\	田中さんは会社をやめた理由を病気のためだと言っているが、それはたんなる口実にすぎない。	
\\	に過ぎない
\\	この飛行機は現在大阪に向かっています。	
\\	に向かって
\\	選手はゴールに向かって全速力で走っている。	
\\	に向かって
\\	先生に向かって失礼なことを言わないでください。	
\\	に向かって
\\	目的にむかってまっすぐに進みなさい。	
\\	に向かって
\\	学生たちは黒板に向かって座っている。	
\\	に向かって
\\	経済の不況に伴って、失業した人が増えている。	
\\	に伴って
\\	人口が増えるに伴って、いろいろな問題が起こってきた。	
\\	に伴って
\\	現代の医学は進歩している。それに伴って、平均寿命が延びている。	
\\	に伴って
\\	自由には、それに伴う責任がある。	
\\	に伴って
\\	学生数が増えるに伴って、学生の質も多様化してきた。	
\\	に伴って
\\	人口の増加に伴い、資源の消費量も増える。	
\\	に伴って
\\	い
\\	につけ 
\\	い
\\	につけ 
\\	につけ 
\\	につけ
\\	その写真を見るにつけ、小学校のころを思い出す。	
\\	につけ
\\	嬉しいにつけ、悲しいにつけ、彼は音楽を聴く。	
\\	につけ
\\	良きにつけ悪しきにつけ、支持率というものがいつも話題になる。	
\\	につけ
\\	寒いにつけ、暑いにつけ、両親は離れて暮らす子供のことを案ずる。	
\\	につけ
\\	いいにつけわるいにつけ現状を報告しなければならない。	
\\	につけ
\\	彼は何事につけても注意深い人だ。	
\\	につけ
\\	雨につけ、雪につけ、あの人は毎日一時間のジョギングを続ける。	
\\	につけ
\\	いいにつけ悪いにつけ、子は親に似る。	
\\	につけ
\\	九月十日必着でメールにてのお返事をお待ちしております。	
\\	にて
\\	現地にて解散となります。	
\\	にて
\\	守衛室にてセキュリティ・チェックを受けてください。	
\\	にて
\\	電話にてご連絡ください。	
\\	にて
\\	帰国に先立ち、リンさんの送別会が行われた。	
\\	に先立ち
\\	会議の開始に先立って、新しいメンバーが紹介された。	
\\	に先立ち
\\	映画の一般公開に先立ち、プレミア試写会を行う予定だ。	
\\	に先立ち
\\	試合開始に先立って、会長からご挨拶をいただきます。	
\\	に先立ち
\\	新車の発売に先立ち、展示会が開かれた。	
\\	に先立ち
\\	わたしは留学するに先立って健康診断を行った。	
\\	に先立ち
\\	新薬の輸入に先立ち慎重な調査が行われている。	
\\	に先立ち
\\	彼は環境汚染の調査を5年にわたって続けてきた。	
\\	にわたって
\\	わが社は20年間にわたり、この薬の研究開発に努めてきた。	
\\	にわたって
\\	同じ家族が数代にわたって、一つの墓になっていることもあった。	
\\	にわたって
\\	相手はほぼ三十年にわたって逮捕を免れてきた輩ですぞ。	
\\	にわたって
\\	兄は7時間にわたる大手術を受けた。	
\\	にわたって
\\	大学入試は一週間にわたって行われます。	
\\	にわたって
\\	アルバイト料は昼は1時間につき800円ですが、深夜は1000円です。	
\\	につき
\\	会議中につき中に入ってはいけない。	
\\	につき
\\	水道代は一ヶ月につき4000円かかります。	
\\	につき
\\	昼休みにつき、事務所は2時まで休みです。	
\\	につき
\\	この幼稚園は子供10人につき一人の先生が配置されている。	
\\	につき
\\	雨につき、試合は延期いたします。	
\\	につき
\\	本日は祭日につき、休業させていただきます。	
\\	につき
\\	この子は音楽の才能に加えて、絵の才能も豊かだ。	
\\	に加えて
\\	のどの痛みに加えて、熱も出てきたので会社を休むしかない。	
\\	に加えて
\\	大雨に加えて、風も激しくなりました。	
\\	に加えて
\\	英語に加えて、フランス語も勉強しています。	
\\	に加えて
\\	彼女は美貌に加えて演技力がすばらしい。	
\\	に加えて
\\	彼女の両親の同意のもとに、来年彼女と結婚することになった。	
\\	のもとで
\\	人々は独立と自由の名のもとに戦っている。	
\\	のもとで
\\	子供の人権は法のもとに守られている。	
\\	のもとで
\\	わたしは大学に入るまで、両親のもとで暮らしていました。	
\\	のもとで
\\	あの夫婦が殺人を犯しているという仮定のもとに言ってるんだ。	
\\	のもとで
\\	三浦先生のもとで、日本の文学を研究しています。	
\\	のもとで
\\	車を買うに際しては、保険に入らなければならない。	
\\	に際して
\\	出発に際して、注意すべきことを皆に話した。	
\\	に際して
\\	面接を受けるに際して、基本的な礼儀を知っておいたほうがいい。	
\\	に際して
\\	お申し込みに際しては、履歴書が必要となります。	
\\	に際して
\\	図書館のご利用に際しては、以下の点にご注意ください。	
\\	に際して
\\	試験に際して、携帯電話の電源を切ってください。	
\\	に際して
\\	契約に際して必要な書類について説明します。	
\\	に際して
\\	東京で暮らすには、一ヶ月最低10万円は必要です。	
\\	には
\\	代々木駅に行くには、山手線に乗ったほうがいいですよ。	
\\	には
\\	外国語を習うには、聞くことと話すことが大切だ。	
\\	には
\\	辞書を見ずに新聞を読むには、相当の語学力が必要だ。	
\\	には
\\	海外旅行に行くにはパスポートが必要だ。	
\\	には
\\	この問題は初心者にはちょっと難しい。	
\\	には
\\	外国で暮らすにはその国の言葉を学ばなければならない。	
\\	には
\\	彼はどんな人か、実際会ってみないことには、よく分からない。	
\\	ないことには~ない
\\	この目で彼女をぜひとも見ないことには、彼女が生きているのが信じられない。	
\\	ないことには~ない
\\	病院へ見舞いに行かないことには、彼の病状は分からない。	
\\	ないことには~ない
\\	事実を確認しないことには、何とも申し上げられません。	
\\	ないことには~ない
\\	両方の話を聞いてみないことには、ケンかの原因が分からないでしょう。	
\\	ないことには~ない
\\	就職が見つからないことには、気持ちが落ち着かない。	
\\	ないことには~ない
\\	実際に読まないことには、この小説のおもしろさはわからないだろう。	
\\	ないことには~ない
\\	試験の日が近づくに従って、だんだん緊張してきた。	
\\	に従って
\\	夫に従ってフランスへ行く。	
\\	に従って
\\	時代の変化に従って、使わない言葉があるのだ。	
\\	に従って
\\	高く登るに従って、見晴らしがよくなった。	
\\	に従って
\\	先生のあとに従って、階段を上がった。	
\\	に従って
\\	会議での決定に従って、来月から新製品の生産を開始することになった。	
\\	に従って
\\	交通規則に従い車を運転してください。	
\\	に従って
\\	彼女の命令に従って目を閉じた。	
\\	に従って
\\	日が暮れるに従って、寒くなってきた。	
\\	に従って
\\	妻は外見のみならず心もきれいな女性だ。	
\\	のみならず
\\	最近、出生率の低下がいろいろ問題になっているが、この問題は、日本のみならず、他の国にもあるようだ。	
\\	のみならず
\\	改良されていたリンゴは、形のみならず味もよい。	
\\	のみならず
\\	学生のみならず、教師もスポーツ大会に参加することになっている。	
\\	のみならず
\\	彼女は成績優秀であるのみならず、人柄も申し分ない。	
\\	のみならず
\\	ワンピースは日本のみならず、外国でもよく知られている。	
\\	のみならず
\\	あの先生は教え方がうまいのみならず親切だ。	
\\	のみならず
\\	彼女は初めてマラソンに参加して最後まで走りぬいた。	
\\	抜く
\\	かなり辛かったが最後までやりぬいた。	
\\	抜く
\\	彼とは長年の友人だから、お互いを知り抜いている。	
\\	抜く
\\	これは考えぬいて、出した結論です。	
\\	抜く
\\	仕事を引き受けたからには、諦めず最後までやりぬくつもりだ。	
\\	抜く
\\	大変な仕事でも、やり抜く自信がある。	
\\	抜く
\\	悩みぬいたあげく、離婚することにした。	
\\	抜く
\\	な
\\	なのも(は)もっともだ	
\\	のも(は)もっともだ	
\\	い
\\	のも(は)もっともだ	のももっともだ
\\	親友に裏切られたんだから、彼が落ち込むのももっともだ。	
\\	のももっともだ
\\	その恐ろしいニュースを聞いておどろくのももっともだ。	
\\	のももっともだ
\\	あなたがそう言うのはもっともだ。	
\\	のももっともだ
\\	その情報を漏らしたのは彼女に相違ない。	
\\	に相違ない
\\	この地域の水汚染の問題を解決するのは難しいに相違ない。	
\\	に相違ない
\\	今日の判決は、彼にとって不満に相違ない。	
\\	に相違ない
\\	誰だろう。廊下をしずかに歩いているのだ。何ごとか企てているのに相違ない!	
\\	に相違ない
\\	石油を原料とする製品はたくさんある。	
\\	を~として
\\	彼女は辞書を枕として横になった。	
\\	を~として
\\	東京タワーを背景とする写真を撮りました。	
\\	を~として
\\	投票の結果、田中を会長さんとすることに決定した。	
\\	を~として
\\	ちょうさんを先生として、中国語の勉強会を開いた。	
\\	を~として
\\	彼の発言をきっかけとして、事件の真相が明らかになった。	
\\	をきっかけに
\\	留学をきっかけに、自分の国についていろいろ考えるようになった。	
\\	をきっかけに
\\	彼は大学入学をきっかけに親元を離れた。	
\\	をきっかけに
\\	山田さんは引っ越しをきっかけとして多くの古い本を捨てた。	
\\	をきっかけに
\\	子供が生まれたのをきっかけにわたしはタバコをやめた。	
\\	をきっかけに
\\	わたしは、3年前に腎臓結石を手術したのをきっかけに、健康に注意するようになった。	
\\	をきっかけに
\\	病気をきっかけに、酒をやめた。	
\\	をきっかけに
\\	冗談をぬきにして、本当のことを話しましょう。	
\\	抜きで/抜きにして
\\	こんな大切なこと、部長ぬきでは決められない。	
\\	抜きで/抜きにして
\\	あいつは抜きにしてさっそく会議を始めましょう。	
\\	抜きで/抜きにして
\\	この計画は彼を抜きにしては、とうてい実現できないだろう。	
\\	抜きで/抜きにして
\\	冗談抜きでまじめに考えてください。	
\\	抜きで/抜きにして
\\	時間がないので、昼食を抜きにして会議を続けることにした。	
\\	抜きで/抜きにして
\\	皆さん、仕事の話は抜きにして楽しく飲みましょう。	
\\	抜きで/抜きにして
\\	かたい挨拶は抜きにして、さっそく乾杯しましょう。	
\\	抜きで/抜きにして
\\	この論文は先生の指導抜きには完成できない。	
\\	抜きで/抜きにして
\\	赤ん坊が出来たことを契機に、夫婦の仲はさらに深まった。	
\\	を契機に
\\	会社名が変わるのを契機に社員の制服も新しくすることが決められた。	
\\	を契機に
\\	両親の離婚を契機に、子供は学校を辞めた。	
\\	を契機に
\\	アメリカの参戦を契機にヨーロッパの状況は変わっていた。	
\\	を契機に
\\	感謝の気持ちを込めてあなたにこの花を送ります。	
\\	をこめて
\\	彼女に愛をこめて手紙を書きました。	
\\	をこめて
\\	心をこめて、このカードを作った。	
\\	をこめて
\\	全快の祈りを込めて、入院している友達に花を贈りました。	
\\	をこめて
\\	彼は激しいあこがれをこめて彼女を見つめた。	
\\	をこめて
\\	次の機会があれば、ぜひご出席願いたいですね。	
\\	お~願う
\\	そろそろ出来上がりますので、しばらくお待ち願います。	
\\	お~願う
\\	間違いはないと思いますが、念のため、お調べ願います。	
\\	お~願う
\\	当日、東京の事務所まで出頭願います。	
\\	お~願う
\\	お静かに願います。	
\\	お~願う
\\	この作家の作品は、若い女性を中心に読まれている。	
\\	を中心に
\\	地球は太陽を中心として回っている。	
\\	を中心に
\\	午前中は文法を勉強します。午後は会話を中心に勉強することになっています。	
\\	を中心に
\\	弊社は関東地方を中心に事業を展開しております。	
\\	を中心に
\\	この店はスキー用品を中心としたスポーツ専門店です。	
\\	を中心に
\\	わたしは友人を通して彼女と知り合いました。	
\\	を通じて/ を通して
\\	その花は一年を通じて咲いている。	
\\	を通じて/ を通して
\\	入学試験の結果は先生を通じて聞いていました。	
\\	を通じて/ を通して
\\	日本の大学生と交流を通じて、日本文化に対する認識を深めた。	
\\	を通じて/ を通して
\\	そのときは、全国を通じてデモが起こりました。	
\\	を通じて/ を通して
\\	この地域は一年を通じて暖かくて、とても過ごしやすい。	
\\	を通じて/ を通して
\\	テレビのニュースを通してその事件を知った。	
\\	を通じて/ を通して
\\	社長は忙しいから、秘書を通じて頼んだほうがいい。	
\\	を通じて/ を通して
\\	学生のアンケートを通して留学生活の様子がわかる。	
\\	を通じて/ を通して
\\	事実をもとにして判断を下します。	
\\	をもとに
\\	このドラマは、実際にあった話をもとに作られたそうだ。	
\\	をもとに
\\	この戸棚は古い家具をもとにして作られたものです。	
\\	をもとに
\\	試験の結果をもとにしてクラス分けを行います。	
\\	をもとに
\\	飛行機の乗客はわたしを除いて皆日本人だった。	
\\	を除いて
\\	わたしは火曜日を除いてはいつでも空いています。	
\\	を除いて
\\	田中さんを除いて、全員会議に出ました。	
\\	を除いて
\\	彼を除いてみんなが笑った。	
\\	を除いて
\\	この仕事は楽だし、給料もいいし、通勤時間が長いことを除いては文句ない。	
\\	を除いて
\\	今シーズンも北海道を除いてはインフルエンザが流行しています。	
\\	を除いて
\\	うちの娘は食事の時を除いてずっと自分の部屋にいる。	
\\	を除いて
\\	日曜日を除いて、毎日働いている。	
\\	を除いて
\\	3週間も水をやらなかったのだから、花が枯れてしまうのも当然だ。	
\\	のも当然だ
\\	彼が怒ったように言い返すのも当然だと思った。	
\\	のも当然だ
\\	最初に気づかなかったのも当然だ。	
\\	のも当然だ
\\	大事な書類をなくしたんだから、上司に叱られるのも当然だ。	
\\	のも当然だ
\\	あんなひどいことを言われては、彼が怒るのも当然だ。	
\\	のも当然だ
\\	引き受けるにせよ引き受けないにせよ、なるべく早く決めたほうがいい。	
\\	にしろ~にしろ
\\	就職するにしろ進学するにしろ、自分でよく考えてください。	
\\	にしろ~にしろ
\\	勝つにしろ負けるにしろ、全力を尽くして戦いたいです。	
\\	にしろ~にしろ
\\	出席するにしろ欠席するにしろ、招待状の返事は早く出したほうがいいよ。	
\\	にしろ~にしろ
\\	行くにしろ行かないにしろ、連絡してください。	
\\	にしろ~にしろ
\\	あの人はおそらくパーティーには出席しないだろう。	
\\	おそらく
\\	おそらくこんな平凡な毎日こそが幸せなのだろう。	
\\	おそらく
\\	彼の顔に浮かんでいたのは、嫌悪と、そしておそらく失望だけだった。	
\\	おそらく
\\	おそらく彼女は、僕の気持ちを察したのだろう。	
\\	おそらく
\\	おそらく、そうではない。	
\\	おそらく
\\	「彼は、父親にそっくりだ…」「見かけは、おそらくそうだろう。」	
\\	おそらく
\\	非常の際には、この窓ガラスを割って逃げてください。	
\\	際に
\\	車を運転する際には免許証が必要だ。	
\\	際に
\\	外出する際は、必ず鍵をかけるようにしてください。	
\\	際に
\\	館内を見学する際には、写真撮影はご遠慮ください。	
\\	際に
\\	先週帰省した際に、小学校の先生を訪ねた。	
\\	際に
\\	パスポートを申請する際に必要なものを教えてください。	
\\	際に
\\	お帰りの際には忘れ物のないようご注意ください。	
\\	際に
\\	このスポーツクラブは年齢を問わず誰でも入れます。	
\\	を問わず
\\	この公園では、季節を問わず美しい花が見られます。	
\\	を問わず
\\	この試験は、年齢を問わず、誰でも受けられる。	
\\	を問わず
\\	そのドラマは世代を問わず多くの人々に好かれています。	
\\	を問わず
\\	老若男女を問わず、彼のファンは多い。	
\\	を問わず
\\	このカフェは昼夜を問わず賑わっている。	
\\	を問わず
\\	経験の有無を問わず、大募集!	
\\	を問わず
\\	この小説は、年齢や性別を問わず、多くの人々に読まれている。	
\\	を問わず
\\	彼女はいつもしかめっ面で、ろくに人と会話もしない。	
\\	ろくに~ない
\\	彼女は赤ちゃんの面倒を見るのに忙しく、仕事もろくにしていない。	
\\	ろくに~ない
\\	トモ子ちゃんだった。しかも珍しいことに、今度もたった一人だ。	
\\	しかも
\\	この辞書は説明がわかりやすく、しかも詳しい。	
\\	しかも
\\	僕達はまた宿題が溜まってるってこと知らないのかい?しかも、まだ二週目だぜ。	
\\	しかも
\\	彼女は美人で、しかも親切だ。	
\\	しかも
\\	あなたの恐れている事は、必ず起こるでしょう。しかも、おそらく、あなたの思っているより早く。	
\\	しかも
\\	の恐れがある	
\\	恐れがある	
\\	-ない
\\	恐れがある	恐れがある
\\	早く手術しないと、手遅れになるおそれがある。	
\\	恐れがある
\\	このテレビドラマは子供に悪い影響を与えるおそれがある。	
\\	恐れがある
\\	大雨が降ると、あの橋は壊れるおそれがある。	
\\	恐れがある
\\	この薬は副作用のおそれがあるので、医者の指示に従って飲んでください。	
\\	恐れがある
\\	こんな赤字が続くと、この会社は倒産のおそれがある。	
\\	恐れがある
\\	すぐに寮に戻らないと、マネージャに捕まって、罰される恐れがある。	
\\	恐れがある
\\	この病気は伝染のおそれがありません。	
\\	恐れがある
\\	盗み聞きされる恐れがあるので、彼女にパーティーのことを話せなかった。	
\\	恐れがある
\\	空港建設をめぐる争いが起こった。	
\\	をめぐって
\\	環境の問題をめぐって多くの意見が出された。	
\\	をめぐって
\\	山田さんの提案をめぐって、活発に討論した。	
\\	をめぐって
\\	今日は漢字の覚え方をめぐって話し合いましょう。	
\\	をめぐって
\\	審判の判定をめぐって両チームに摩擦があった。	
\\	をめぐって
\\	その事件をめぐって、さまざまなうわさが流れている。	
\\	をめぐって
\\	明美が手に持っている何かをめぐって、二人が口論していた。	
\\	をめぐって
\\	そのドラマは、一人の女性をめぐって、三人の男性が戦う話です。	
\\	をめぐって
\\	その問題をめぐって、みんなが方策を考えようとして、少し間があった。	
\\	をめぐって
\\	今年の夏は天候が非常に悪かった。したがって米の収穫がかなりすくなくなりそうだ。	
\\	したがって
\\	彼はまじめで誠実な人だ。したがってみんなから信頼されている。	
\\	したがって
\\	教授は急用で来られません。したがって講義は中止です。	
\\	したがって
\\	休暇はせいぜい一週間しかとれない。	
\\	せいぜい
\\	普通の猫は、せいぜい十五年の寿命なんですよ。	
\\	せいぜい
\\	彼はなんとか笑って見せようとしたが、せいぜい痛そうに顔をしかめたようにしか見えなかった。	
\\	せいぜい
\\	みんなには僕を悩ませていたことのせいぜい半分しかわかっていなかったのだ。	
\\	せいぜい
\\	せいぜい宿題の事しか心配しなくてもよいよ。	
\\	せいぜい
\\	新学期をせいぜい楽しめよ。	
\\	せいぜい
\\	結婚した相手次第で人生が決まってしまうこともある。	
\\	次第だ/ 次第で
\\	実力次第では社長になることも可能だろう。	
\\	次第だ/ 次第で
\\	明日ピクニックに行くかどうかは、お天気次第だ。	
\\	次第だ/ 次第で
\\	彼に急用ができたため、わたしが代わりに来た次第です。	
\\	次第だ/ 次第で
\\	先生のご都合次第では来週の講演は延期になります。	
\\	次第だ/ 次第で
\\	成績が伸びるかどうかはあなたの努力次第です。	
\\	次第だ/ 次第で
\\	頼み次第ではあの人もこの仕事を手伝ってくれるかもしれない。	
\\	次第だ/ 次第で
\\	英語が話せないわたしには無理な仕事と思い、お断りした次第です。	
\\	次第だ/ 次第で
\\	父が病気のため、実家の豆腐屋を引き継いだ次第だ。	
\\	次第だ/ 次第で
\\	近所で家事があったが、その家の人たちは幸い無事だった。	
\\	幸い
\\	幸いなことに、父の心臓の手術はうまくいった。	
\\	幸い
\\	警察が先週、僕たちの家を立ち入り検査した。幸い、たいした物は見つからなかったけど。	
\\	幸い
\\	試験当日、朝寝坊をしてしまったが、幸い試験を受けることはできた。	
\\	幸い
\\	幸い彼は気づかなかった。	
\\	幸い
\\	痛みもないけど、おまけになんにも感じないよ。	
\\	おまけに
\\	あのラーメン屋は高いしまずい。おまけにサービスも悪い。	
\\	おまけに
\\	おまけに宝の地図を取り戻すことにも失敗した。	
\\	おまけに
\\	そういえば、彼がどこに住んでいるのか聞いたことがない。	
\\	そういえば
\\	いい家ですね。そういえば、お兄さんは建築家さんでしたよね。	
\\	そういえば
\\	まてよ、そういえば、なんか以前に一回、そんなことを言ってた。	
\\	そういえば
\\	売り場が混んでいると思ったら、チョコレートか。ああ、そういえば、明日はバレンタインデーだった。	
\\	そういえば
\\	よく温泉に入ってから石鹸で体を洗う人がいますが、そうすると、温泉の成分が落ちてしまって、せっかく温泉に来たのに、意味がありません。	
\\	そうすると
\\	そうすると彼は、目の前にあったペンを手にとってなにやらメモし始めた。	
\\	そうすると
\\	そうすると、あなたもやはり医学のほうをおやりですか?	
\\	そうすると
\\	もともと家の周りじゅうに庭のスペースをとるのが夢だったんだ。でも、そうすると、家が小さくなりすぎて。	
\\	そうすると
\\	最近仕事がすごく忙しくて、パーティーに行けそうにないよ。	
\\	そうにない
\\	彼はとうてい眠れそうにないと思った。	
\\	そうにない
\\	この仕事、まだ半分もできていないから、今日中には終わりそうにない。	
\\	そうにない
\\	それにしても、なぜ、この地域の人そんなに長生きなんでしょうか。	
\\	それにしても
\\	それにしても、ここはどこなのだろう?	
\\	それにしても
\\	リーはアーサーが大嫌いだったが、このときばかりはちょっとかわいそうに思った。それにしても自分の方がもっとかわいそうだった。	
\\	それにしても
\\	それにしても、一体全体彼はどうやって生き延びたんでしょう?	
\\	それにしても
\\	母の父、すなわちわたしの祖父は現在63歳です。	
\\	すなわち
\\	このタンクには1000ミリリットル、すなわち1リットルの水が入っています。	
\\	すなわち
\\	母の兄、すなわちぼくの伯父は病院で働いています。	
\\	すなわち
\\	すなわち、来年にはワールドカップが開催されるのだ。	
\\	すなわち
\\	その上、雨が降りそうだな。	
\\	その上
\\	その上、彼はわたしの誕生日まで忘れている。	
\\	その上
\\	ただでさえ彼女と過ごす時間が少ないのに、その上練習で時間を取られるのが、特にうんざりだ。	
\\	その上
\\	その上彼女のやることも足しにならない。	
\\	その上
\\	わたしはあなたを少しも責めていない。	
\\	少しも~ない
\\	少しも見当がつかない。	
\\	少しも~ない
\\	彼女は少しも足音をたてなかった。影のように部屋から出ていった。	
\\	少しも~ない
\\	この町は少しも変わっていない。	
\\	少しも~ない
\\	彼は少しも後悔しなかった。	
\\	少しも~ない
\\	これに少しも驚かなかった。	
\\	少しも~ない
\\	少しも危険がないじゃないの。	
\\	少しも~ない
\\	母はどんなに辛くたって、決して顔に出さない人だ。	
\\	たって
\\	先生だって間違えることがあるよ。	
\\	たって
\\	うちのこはいくら注意したって、悪い癖が治らない。	
\\	たって
\\	彼はいつだって家にいません。	
\\	たって
\\	僕たちだって、この一ヶ月、あなたから聞き出そうとしてきた。	
\\	たって
\\	どの大学に入ろうかと、さんざん悩んだあげく、京都大学に決めた。	
\\	たあげく
\\	彼らは夫婦げんかを繰り返したあげくに、とうとう離婚した。	
\\	たあげく
\\	このチケットは8時間並んだあげくにやっと手にいれたものだ。	
\\	たあげく
\\	さんざんお酒を飲んだあげくに、健康を害してしまった。	
\\	たあげく
\\	わたしを呼びつけたあげく、またあいつに逃げられたと言うのか?	
\\	たあげく
\\	さんざん迷ったあげく卒業後は帰国することにした。	
\\	たあげく
\\	一時間探したあげく、あの本はなくなってしまったと結論するしかなくなった。	
\\	たあげく
\\	何度も手術したあげく、治らないと言われた。	
\\	たあげく
\\	長時間の議論のあげく、とうとう結論は出せなかった。	
\\	たあげく
\\	僕は、彼女の言うことが確かに正しいと思う。	
\\	確かに
\\	確かに、上に誰かいるぞ。	
\\	確かに
\\	彼は確かに立派な人だと思う。	
\\	確かに
\\	向こうで、何か動くのが、確かに見えた。	
\\	確かに
\\	叫び声がするのが確かに聞こえた気がした。	
\\	確かに
\\	水浸しの床がさらに水をかぶった。	
\\	さらに
\\	車はさらに速度を上げていった。	
\\	さらに
\\	何か聞こえはしないかと、さらに耳を澄ませた。何の物音もしない。	
\\	さらに
\\	さらに何人かの学生が笑った。	
\\	さらに
\\	汽車がさらに北へ進んだ。	
\\	さらに
\\	やつは何かをさせたがっている。それなのに、なかなかうまくいかない。	
\\	それなのに
\\	一生懸命勉強している。それなのに、成績はよくならない。	
\\	それなのに
\\	彼女はわたしにやさしくしてくれました。それなのに、なぜあんなことを言われなければならないのでしょう。	
\\	それなのに
\\	毎日節約している。それなのに、お金がちっともたまらない。	
\\	それなのに
\\	それなのに、わたしは、みんながわたしを信用したなんて思っていた。	
\\	それなのに
\\	他人に話したところで、夫婦の問題は解決できない。	
\\	たところで
\\	どんなに一生懸命働いたところで、生活は楽にならないだろうと思う。	
\\	たところで
\\	いくら後悔したところで、事故を起こしてからではどうにもならない。	
\\	たところで
\\	そんな多額の借金は、家を売ったところで、到底返せない。	
\\	たところで
\\	今から勉強したところで、明日のテストでいい点は取れないだろう。	
\\	たところで
\\	たとえ話したところで、世界広しといえども誰が信じるというのか?	
\\	たところで
\\	人生そんなに長くないのだから待ってばかりはいられない。	
\\	てばかりはいられない
\\	体の調子が悪いからといって、このまま会社を休んでばかりはいられない。	
\\	てばかりはいられない
\\	どんなに辛くても、泣いてばかりもいられない。	
\\	てばかりはいられない
\\	今日の試合に勝ったからといって喜んでばかりはいられない。	
\\	てばかりはいられない
\\	いくら食欲の秋とはいっても、食べてばかりはいられない。	
\\	てばかりはいられない
\\	それなら、どうしたら見つかるんだ?	
\\	それなら
\\	それなら、何が起きたのか説明せよ。	
\\	それなら
\\	それならいいわ、でも早くしてね。	
\\	それなら
\\	さて、それなら、ほかの手立てで探ってみるしかないということだ。	
\\	それなら
\\	「道がすごく渋滞しているようだよ。」「それなら、電車で行こう。」	
\\	それなら
\\	「本は好きだから買いたいけど、高いし、増え続けて邪魔だし…」「それなら、図書館で借りたらいいじゃないですか。」	
\\	それなら
\\	彼らに直ちにわたしの事務所に集まるようにと知らせてください。	
\\	直ちに
\\	参院予算委員会を直ちに開催すべきだ。	
\\	直ちに
\\	現在から言って、直ちにその計画を実行するのは無理だ。	
\\	直ちに
\\	午後新幹線で京都に移動しまして、到着後、直ちに会議場のあるホテルに向かいます。	
\\	直ちに
\\	家を売ってでも留学に行きたいです。	
\\	てでも
\\	引きずってでも田中を学校へ連れて行くぞ。	
\\	てでも
\\	人に憎まれてでも出世したいとは思わない。	
\\	てでも
\\	彼はどんな危険を冒してでも、必ずわたしに会いにきた。	
\\	てでも
\\	仕事を捨ててでも好きな彼女と結婚したいです。	
\\	てでも
\\	どんなことをしてでもこの謎を解く。	
\\	てでも
\\	大学を卒業した後どんな職業につくべきか、迷っている。会社員になるかそれとも学校の先生になるかだ。	
\\	それとも
\\	あなたが知りたいのは住所ですか、それとも電話番号ですか。	
\\	それとも
\\	わたしが、彼と結婚したくないだろうと思ったのですか?それとも、もしかして、そうなって欲しいと思いましたか?	
\\	それとも
\\	はじめましょうか。それとも、校長先生がお見えになるまでお待ちしましょうか?	
\\	それとも
\\	あの人が間違っていると思うのか?それとも俺がうまく騙したとでも?	
\\	それとも
\\	わたしは子供のころから物事を決めるのが遅かった。たとえば、友達に旅行に誘われても、行きたくないわけではないが、すぐに返事をしないで、いろいろ考えているうちに時間がたってしまう。いつのまにか友達もあきらめ、旅行はとりやめになってしまい、残念に思うことが多い。	
\\	たとえば
\\	世間には、犯人を罪に落とすことができない犯罪がある。たとえば、ブラウン夫婦の一件だ。	
\\	たとえば
\\	この本にはいい言葉がかなり多い。たとえば、「今日やらないと、明日は後悔!」。	
\\	たとえば
\\	これでいろんな謎が解けた。たとえば、どうして彼女は僕をあんなに憎んでるか。	
\\	たとえば
\\	説明したまえ。	
\\	たまえ
\\	こっちを向きたまえ。	
\\	たまえ
\\	それじゃ、自分の部屋に戻りたまえ。	
\\	たまえ
\\	さあ、目をつむりたまえ。	
\\	たまえ
\\	急ぎたまえ。	
\\	たまえ
\\	線路の手前を左に曲がってください。	
\\	手前
\\	約束した手前、たとえ雪が降っても行かないわけにはいかない。	
\\	手前
\\	生徒達の手前、この言葉を知らないとは言えない。	
\\	手前
\\	彼は扉の二メートルほど手前で立ち止まった。	
\\	手前
\\	毎日そんなに遅くまで残業をしていては体を壊しますよ。	
\\	ていては
\\	そんなにたくさん冷たいものを飲んでいては腹痛になりますよ。	
\\	ていては
\\	こんなに物価が上昇していては、生活が苦しくなるばかりです。	
\\	ていては
\\	そんな暗い部屋で本を読んでいては目が悪くなるから、電気をつけなさい。	
\\	ていては
\\	そんなに毎日甘いものばかり食べていては虫歯になりますよ。	
\\	ていては
\\	な
\\	で当然だ/で当たり前だ	
\\	-て
\\	当然だ/当たり前だ	
\\	い
\\	(-い) 
\\	くて当然だ/て当たり前だ	て当然だ
\\	ひどいことばかり言ったので、彼女に嫌われて当然だ。	
\\	て当然だ
\\	あいつは罰を受けて当然だ。	
\\	て当然だ
\\	相手のチームは弱い。勝って当たり前だ。	
\\	て当然だ
\\	無罪で当然なのよ。	
\\	て当然だ
\\	よく考えた末に、彼女と結婚することにした。	
\\	た末 
\\	の末
\\	8時間にわたる話し合いの末に、やっと契約を結んだ。	
\\	た末 
\\	の末
\\	さんざん悩んだ末、一番最初に見たソファーを買うことにした。	
\\	た末 
\\	の末
\\	激しい議論の末、ようやく結論を出した。	
\\	た末 
\\	の末
\\	考えた末に、会社を辞めて独立することにしました。	
\\	た末 
\\	の末
\\	この新しい薬は、何年にもわたる研究の末に作り出されるものだ。	
\\	た末 
\\	の末
\\	さんざん迷った末に、大学院への進学を諦めて就職することにした。	
\\	た末 
\\	の末
\\	住民と話し合いの末、その建物の建設は中止となった。	
\\	た末 
\\	の末
\\	100点がとれなくても、せめて90点はとりたい。	
\\	せめて
\\	わたしに話しかけてほしかったのに。せめてわたしを見てくれるだけでも。	
\\	せめて
\\	マリー、せめてこれぐらいしか、してあげられないけど。	
\\	せめて
\\	せめて軽く一杯飲んでから出かける事にしたらどう?	
\\	せめて
\\	お互いに率直に話し合ってこそ理解し合えるのです。	
\\	てこそ
\\	親になってこそ、親の苦労や気持ちがわかるものだ。	
\\	てこそ
\\	その国で生活してこそ文化が分かるというものだ。	
\\	てこそ
\\	社長になってこそ、社長の苦労がわかる。	
\\	てこそ
\\	大きな声で話しては魚が逃げてしまう。	
\\	ては
\\	休み中、食べては眠り、食べては眠りの連続で、すっかり太ってしまった。	
\\	ては
\\	料理がまずくては誰も来ませんよ。	
\\	ては
\\	書いては消し、書いては消し、やっと論文を完成した。	
\\	ては
\\	な
\\	であると同時に	
\\	と同時に	
\\	と同時に	と同時に
\\	ドアを開けると同時に電話のベルがなりました。	
\\	と同時に
\\	地震と同時に火事が起こった。	
\\	と同時に
\\	大学で英語を勉強すると同時に経済も勉強します。	
\\	と同時に
\\	彼は、恐れると同時に魅惑されているようだった。	
\\	と同時に
\\	わたしが乗り込むと同時に電車のドアが閉まった。	
\\	と同時に
\\	彼は大学の先生をしていると同時に小説も書いている。	
\\	と同時に
\\	扉がバタンと閉まると同時に、大きな音が、部屋の中に響いた。	
\\	と同時に
\\	両親と相談してからでないと、自分で決められません。	
\\	てからでないと
\\	証拠をつかんでからでないと、彼を逮捕することはできない。	
\\	てからでないと
\\	今日の宿題が終わってからでないと、遊びに行けない。	
\\	てからでないと
\\	君はまるで外科医のようだ。容態を詳しく聞いてからでないと、診断を下しなさらん。	
\\	てからでないと
\\	親の許可をもらってからでなければ申し込めません。	
\\	てからでないと
\\	もう少し具体的な説明を聞いてからでないと、その計画には賛成できません。	
\\	てからでないと
\\	手続きしてからでないと、図書館の本は借りられない。	
\\	てからでないと
\\	使ってからでなければ、その品物が本当にいいかどうか分からない。	
\\	てからでないと
\\	この仕事は訓練を受けてからでないとやられないよ。	
\\	てからでないと
\\	な
\\	でしょうがない/しかたがない	
\\	-て
\\	しょうがない/しかたがない	
\\	い
\\	(-い)くて 
\\	しょうがない/しかたがない	てしょうがない
\\	このアパートは部屋が広いが、駅から遠いので不便でしょうがない。	
\\	てしょうがない
\\	パソコンを使いすぎたせいか、このごろ目が疲れてしかたがない。	
\\	てしょうがない
\\	今日は何もすることがなくて、暇でしかたがない。	
\\	てしょうがない
\\	うちの会社は給料も低いし、労働時間も長いし、イヤでしかたがない。	
\\	てしょうがない
\\	大学入試に合格したので、嬉しくてしょうがない。	
\\	てしょうがない
\\	彼らは僕を養うのにお金がかかってしょうがないとあんなに愚痴を言っていたんだもの。	
\\	てしょうがない
\\	勉強中、眠くてしょうがないときは、濃いお茶を飲むといい。	
\\	てしょうがない
\\	新しい車が欲しくてしょうがない。	
\\	てしょうがない
\\	そんな方法では、時間がかかってしょうがない。	
\\	てしょうがない
\\	彼女は信じられないというふうに頭を振った。	
\\	というふうに
\\	彼は肩をすくめ、歳の功はあるものだというふうに、両手を広げた。	
\\	というふうに
\\	な
\\	でたまらない	
\\	-て
\\	たまらない	
\\	い
\\	(-い) 
\\	くてたまらない	てたまらない
\\	恋人ができて嬉しくてたまらない。	
\\	てたまらない
\\	昨日の歌の練習では何度も間違えてみんなに笑われ、恥ずかしくてたまらなかった。	
\\	てたまらない
\\	この仕事が大変なので、やめたくてたまらない。	
\\	てたまらない
\\	楽しみにしていたコンサートが中止になってしまった。残念でたまらない。	
\\	てたまらない
\\	母のことが心配でたまらない。	
\\	てたまらない
\\	わたしは10年も国へ帰っていないので、早く家族に会いたくてたまらない。	
\\	てたまらない
\\	買ったばかりのアイパッドをなくしてしまった、悲しくてたまらない。	
\\	てたまらない
\\	蚊に刺された足が痒くてたまらない。	
\\	てたまらない
\\	「絶対、大丈夫よ」彼女はそう言いながら、怖くてたまらない様子だ。	
\\	てたまらない
\\	彼は知りたくてたまらない気持ちが消えない。	
\\	てたまらない
\\	卒業以来、彼は母校で先生として働いている。	
\\	て以来
\\	彼とは5年前のクラス会で会って以来、一度も連絡を取っていない。	
\\	て以来
\\	結婚して以来、仕事と家事に追われる日々続いている。	
\\	て以来
\\	中国に来て以来、中国に対する考え方が少しずつ変わってきた。	
\\	て以来
\\	先週外食をして以来、おなかの調子が悪いんです。	
\\	て以来
\\	たしかに、彼はこの学校始まって以来、最高の秀才だったと言える。	
\\	て以来
\\	彼女はダイヤモンドが偽物だと知って以来、はじめて興奮していた。	
\\	て以来
\\	品物は、高ければいいというものではない。	
\\	というものでもない
\\	勝負は勝てばよいというものではない。どんな勝ち方をしたのかが重要である。	
\\	というものでもない
\\	会議に参加すればいいというものでもない。自分の意見を述べるべきだ。	
\\	というものでもない
\\	医者の話によると、この病気はすぐによくなるというものではないらしい。	
\\	というものでもない
\\	お金があれば幸せになれるというものでもない。	
\\	というものでもない
\\	親は子供をただ叱ればよいというものではない。	
\\	というものでもない
\\	性格は絶対に変わられないというものではない。	
\\	というものでもない
\\	有名な大学を卒業したからといって、それで成功できるというものでもない。	
\\	というものでもない
\\	やさしい問題だといって、油断をしてはならない。	
\\	てはならない
\\	自分の利益のために法律を変えることがあってはならない。	
\\	てはならない
\\	この悲惨な体験を決して忘れてはならない。	
\\	てはならない
\\	そんなことをさせてはならない。	
\\	てはならない
\\	「禁帯出」とは、外に持ち出してはならないという意味である。	
\\	てはならない
\\	自分の愚かさのせいで、友達にとんでもない代償を払わせてはならない。	
\\	てはならない
\\	な
\\	でならない	
\\	-て
\\	ならない	
\\	い
\\	(-い) 
\\	くてならない	てならない
\\	面接に失敗して悔やまれてならない。	
\\	てならない
\\	あの二人はどうして離婚したのか、知りたくてならない。	
\\	てならない
\\	もう十年も国に帰っていないので、両親に会いたくてならない。	
\\	てならない
\\	姉の結婚式に出席できないので、残念でならない。	
\\	てならない
\\	息子が大学に合格して、うれしくてならない。	
\\	てならない
\\	世界中を旅行して回れるなんてうらやましくてならない。	
\\	てならない
\\	大学の時もっと勉強しておけばよかったと悔やんでならない。	
\\	てならない
\\	母の病気が心配でならない。	
\\	てならない
\\	また公共料金が上がるとか。	
\\	とか
\\	今度の試験はかなり難しいとか。	
\\	とか
\\	先生は来週、お忙しいとか。	
\\	とか
\\	鈴木さんは今日は病気で仕事を休むとか。	
\\	とか
\\	昨日はこの夏一番の暑さだったとか。	
\\	とか
\\	桜子さんは来年結婚なさるとか。	
\\	とか
\\	今夜の花火大会は、雨で中止だとか。	
\\	とか
\\	店の前に長い行列ができているところを見ると、大変人気のラーメン屋のようだ。	
\\	ところを見ると
\\	電灯が消えているところを見ると、彼女はもう寝たようだ。	
\\	ところを見ると
\\	朝からニコニコしているところを見ると、何かいいことがあったにちがいない。	
\\	ところを見ると
\\	慌てたところを見ると、彼は事実を知っているにちがいない。	
\\	ところを見ると
\\	彼があんなに喜んでいるところを見ると、大学に合格したにちがいない。	
\\	ところを見ると
\\	悩んでいるところを見ると、彼は就職か進学かまだ決まっていないようだ。	
\\	ところを見ると
\\	彼女がショックを受けたような顔をしたところを見ると、本当のことを聞いたにちがいない。	
\\	ところを見ると
\\	このままでは、韓国の観光産業は進む一方だと考えれらる。	
\\	と考えられる
\\	自分の力では、環境を改善するのは不可能だと考えられる。	
\\	と考えられる
\\	新聞では、取りあげるほど重要ではないと考えられている、何人かの死について報告します。	
\\	と考えられる
\\	あなたが逃げだした直後に夫人は死んだものと考えられます。	
\\	と考えられる
\\	英語は世界の共通語だと考えられている。	
\\	と考えられる
\\	ちょうど山田さんのうわさをしているところに、本人がやってきた。	
\\	ところに/ところへ
\\	電話しようと思っていたところに、ちょうどその相手から電話がかかってきた。	
\\	ところに/ところへ
\\	いいところに来てくれたね。退屈していたんだよ。	
\\	ところに/ところへ
\\	ちょうど出かけようとしていたところへ友達が来た。	
\\	ところに/ところへ
\\	お弁当を持って、ピックニックに出かけようとしたところに、雨が降ってきた。	
\\	ところに/ところへ
\\	道に迷って困っていたところへ、警察が通りかかった。	
\\	ところに/ところへ
\\	お忙しいところをすみません。	
\\	ところに/ところへ
\\	旅行に行ったつもりでお金をためている。	
\\	つもりで
\\	これから一ヶ月、死んだつもりで働いてもらいます。	
\\	つもりで
\\	仕事が予定より遅れている。だから、今度の日曜日は、仕事を休んではいられない。	
\\	てはいられない
\\	明日は早く起きるので、いつものように遅くまでテレビを見てはいられない。	
\\	てはいられない
\\	来週は大事な期末試験なので、のんびりしてはいられない。	
\\	てはいられない
\\	大人になりたくないが、いつまでも子供ではいられない。	
\\	てはいられない
\\	試験終了時間まであと数分だから、この問題にそんなに時間をかけてはいられない。	
\\	てはいられない
\\	事の真相を知った以上、もう黙ってはいられない。	
\\	てはいられない
\\	早速彼女に聞かないではいられない。	
\\	てはいられない
\\	学校に来る途中で、財布を忘れたのに気づいた。	
\\	途中に/途中で
\\	彼らは恐怖で身動きできないらしく、階段の途中に立ちすくんでいた。	
\\	途中に/途中で
\\	さっき家へ帰る途中でちょっと事故に遭っちゃってさ。	
\\	途中に/途中で
\\	彼はセーターを着る途中で止めた。	
\\	途中に/途中で
\\	ここに来る途中で、山田先生に会ったんだ。	
\\	途中に/途中で
\\	わたしたちは帰る途中で別れた。	
\\	途中に/途中で
\\	彼女はテーブルまで行く途中で転んでしまった。	
\\	途中に/途中で
\\	日本では子供の数が減りつつある。	
\\	つつある
\\	彼の耳が今や真っ赤になりつつある。	
\\	つつある
\\	新しいビルはどんどん建てられつつある。	
\\	つつある
\\	彼女の足は回復しつつある。	
\\	つつある
\\	環境問題への関心が高くなり、車ではなく電車を利用する人が増えつつある。	
\\	つつある
\\	ジョンは自分に戻りつつある。	
\\	つつある
\\	政府は景気が緩やかに回復しつつあると発表した。	
\\	つつある
\\	問題は解決されつつあるが、時間がかかりそうだ。	
\\	つつある
\\	「わたしはもうお酒を飲めます。」「ということは二十歳を過ぎてるんだね。」	
\\	ということは
\\	「4位でした。」「ということは、メダルは取れなかったということですね、残念。」	
\\	ということは
\\	だから僕が知ってたとしても言わない。	
\\	としても
\\	たとえできたとしても、わたしは何もしないよ。	
\\	としても
\\	たとえ、口外したとしても、誰が信じるというのです?	
\\	としても
\\	あの牢獄に入れられたとき狂ってなかったとしても、いまは狂ってるだろうな。	
\\	としても
\\	もし誰かが何かを漏らしたとしても、そうするつもりでやったんじゃないと思う。	
\\	としても
\\	その男は年を取っている上に、体も弱っており、仕事をするのは無理だ。	
\\	上に
\\	そのスポーツクラブは入会金が要らない上に、わが家から近い。	
\\	上に
\\	この文章は漢字の間違いが多い上に文法の間違いも多い。	
\\	上に
\\	この手続きは、面倒な上に時間もかかるので、皆がいやがっている。	
\\	上に
\\	この本は内容が難しい上に、翻訳がよくないので読みにくい。	
\\	上に
\\	今庭に咲いている花は、きれいな上に香りもすごくいい。	
\\	上に
\\	彼は頭がいいうえに、ハンサムだ。	
\\	上に
\\	サッカー練習がある上、彼にはモニターとしての任務がある。	
\\	上に
\\	勉強が何より大事だと思いつつ、また授業をサボってしまった。	
\\	つつ
\\	資金のことを考えつつ、工事を進めていった。	
\\	つつ
\\	辞書を引きつつ英字新聞を読んでいます。	
\\	つつ
\\	人は皆、お互いに助け合いつつ生きている。	
\\	つつ
\\	彼は時々車を運転しつつ、電話をする。	
\\	つつ
\\	健康のためには階段を使ったほうがいいと思いつつ、ついエレベーターに乗ってしまう。	
\\	つつ
\\	悪いと知りつつ、うそをついてしまった。	
\\	つつ
\\	体に悪いと知りつつ、タバコを吸っている。	
\\	つつ
\\	ほかの科目は別として、物理はとても得意だ。	
\\	は別として
\\	うまくいくいかないは別として、とにかく何かをしたい。	
\\	は別として
\\	両親は別として他に誰かわたしの言う事を信じてくれるだろうか?	
\\	は別として
\\	おいしいかどうかは別として、一度納豆を食べてみたい。	
\\	は別として
\\	いまお話しくださったことは別として、ほかになにかこの男のことについてお聞きしておくことはありませんか?	
\\	は別として
\\	その仕事を引き受けた上は、最後まで責任を持ってやり遂げる。	
\\	上は
\\	社長が決断した上は、我々社員はやるしかない。	
\\	上は
\\	約束した上は、どんなことがあっても守ります。	
\\	上は
\\	キャプテンに選ばれた上は、頑張るしかない。	
\\	上は
\\	日本に来た上は日本語をマスターして帰りたい。	
\\	上は
\\	弁護士になると決めた上は、苦しくても頑張らなければならない。	
\\	上は
\\	結婚した上は、家族に対して責任を持つのが当然のことだ。	
\\	上は
\\	電気はもちろん水道もないから、生活するにはかなり不便です。	
\\	はもとより
\\	今の子供たちは学校はもとより塾へも通わせられている。	
\\	はもとより
\\	漫画の種類が増えている。子供のためのものはもとより大人が読むための歴史や経済の漫画もよく見かける。	
\\	はもとより
\\	わたしが生まれた村は、電車はもとより、バスも通っていない。	
\\	はもとより
\\	面接試験では、話し方はもちろん、服装などにも気をつける必要がある。	
\\	はもとより
\\	この製品は国内はもとより、海外でも高い評価を得ている。	
\\	はもとより
\\	うちには、車はもとより自転車もないんです。	
\\	はもとより
\\	床が濡れていることに気づかず、もう少しで倒れるところだった。	
\\	ところだった
\\	気がつくのが遅かったら、あやうく火事になるところだったよ。	
\\	ところだった
\\	あぶないところだった。	
\\	ところだった
\\	ちゃんと前を見て運転してよ。今、隣の車にぶつかるところだったよ。	
\\	ところだった
\\	女の子が急いで人形を拾いあげるところだった。	
\\	ところだった
\\	納豆や寿司といった日本の食べ物を好む外国人が増えてきた。	
\\	といった
\\	この人形は、「こんにちは」「さようなら」といった簡単な言葉を話します。	
\\	といった
\\	バスケットボール、サッカーといったスポーツは大学生に人気がある。	
\\	といった
\\	この日本語学校にはタイ、ベトナムといった東南アジアからの留学生が少ない。	
\\	といった
\\	い
\\	やら 
\\	い
\\	やら 
\\	やら	
\\	やら 
\\	やら~やら
\\	東京大学の学生は授業やら部活やらで毎日充実で忙しい。	
\\	やら~やら
\\	あの人は来るのやら来ないのやら、さっぱり分からない。	
\\	やら~やら
\\	初デートは恥ずかしいやら嬉しいやら、複雑な気分だった。	
\\	やら~やら
\\	ポケットには財布やら携帯電話やらが入っている。	
\\	やら~やら
\\	帰国前は飛行機の予約をするやらおみやげを買うやらで忙しい。	
\\	やら~やら
\\	今日は財布を落とすやらレポートを忘れるやら、散々な一日だった。	
\\	やら~やら
\\	専門学校の学生は宿題やらアルバイトやらで忙しい。	
\\	やら~やら
\\	驚くやら悲しむやら、そのニュースを聞いた人たちの反応は様々だった。	
\\	やら~やら
\\	値段はともかく、デザインのほうが気に入った。	
\\	はともかく
\\	彼が行くかどうかはともかく、初めの計画どおり旅行しよう。	
\\	はともかく
\\	挨拶はともかくとして、さっそく会議の本題に入りましょう。	
\\	はともかく
\\	わたしは、日本語を話すことはともかく、書く事は苦手だ。	
\\	はともかく
\\	フランス語はともかく、英語だけはぜひマスターしてください。	
\\	はともかく
\\	行くか行かないかはともかく、飛行機の予約だけはしておこう。	
\\	はともかく
\\	あの女優は、顔はともかくとして演技がすばらしい。	
\\	はともかく
\\	とも	い
\\	(-い) 
\\	くとも	
\\	-ない
\\	(-い) 
\\	くとも	とも
\\	母はどんなに辛くとも、決してぐちを言わなかった。	
\\	とも
\\	君が何をしようとも、わたしには責任を取りません。	
\\	とも
\\	何か目標を持って努力しているときは、どんなに苦しくとも最後まで頑張り通せるものだ。	
\\	とも
\\	言わなくともわかることだと思っていたが…	
\\	とも
\\	どんなに一生懸命やってもうまくいかない時もある。これが現実というものだ。	
\\	というものだ
\\	相手の話も聞かずに自分の主張だけ通そうとするなんて、それはわがままというものだ。	
\\	というものだ
\\	無断で他人の日記を読むのは、プライバシーの侵害というものだ。	
\\	というものだ
\\	彼がすぐそばにいるからには、そのような危険を冒すのは愚かというものだ。	
\\	というものだ
\\	相手の立場に立って考えることのできる人、それが大人というものです。	
\\	というものだ
\\	一日に20時間も働くのは無理というものだ。	
\\	というものだ
\\	それが公平というものだ。	
\\	というものだ
\\	いくら仲が良くても、夜中に電話をかけるなんて、それは非常識というものだ。	
\\	というものだ
\\	今年の調査で、クジラの生息数がわずかに増えていることがわかった。	
\\	わずかに
\\	暖炉にはまだわずかに残り火が燃えている。	
\\	わずかに
\\	彼はわずかに口を開けて眠り続けた。	
\\	わずかに
\\	屋敷の中には、わずかに明るいムードが漂いはじめた。	
\\	わずかに
\\	彼女がほんのわずかに首を横に振った。	
\\	わずかに
\\	みんなに知られたからには、白状するほかはない。	
\\	よりほかない
\\	バスがないので、歩いていくよりほかしかたがない。	
\\	よりほかない
\\	その人を雇うかどうかは書類だけでは決められないから、結局は実際に本人に会って判断するよりほかはない。	
\\	よりほかない
\\	わたしの不注意で壊したのだから、弁償するほかない。	
\\	よりほかない
\\	雨が降っているので、残念だが今日の遠足は延期するほかない。	
\\	よりほかない
\\	もう後には戻れない。前進するよりほかない。	
\\	よりほかない
\\	彼が生活に困っているというから、お金を貸したのに、そのまま姿をけしてしまい、今となってはあきらめるよりほかない。	
\\	よりほかない
\\	友達が余っていたコンサートの券を一枚くれた。それで、わたしは券を買わずにすんだ。	
\\	ずに済む
\\	彼の庇護で、僕は監獄に入らずにすんだ。	
\\	ずに済む
\\	隣の家の窓ガラスを割ってしまったが、「すみません」と謝ったらお金を払わずにすんだ。	
\\	ずに済む
\\	おしおきを受けずにすんでほっとした。	
\\	ずに済む
\\	黙っててくれたら捕まらずにすむよ。	
\\	ずに済む
\\	父は、普通ならとっくに引退している年齢だが、年をとっているわりには元気だ。	
\\	とっくに
\\	宝がどこにあるか、とっくに知っている。	
\\	とっくに
\\	僕の声?とっくに聞こえてるよ。彼女の声が!	
\\	とっくに
\\	前に持ってた携帯電話はとっくに失くした。	
\\	とっくに
\\	彼らはとっくに出かける準備ができていた。	
\\	とっくに
\\	彼女はとっくに眠っていた。	
\\	とっくに
\\	たとえ好きでもこのケーキは一人では食べようがない。	
\\	ようがない/ようもない
\\	彼女は美人で、言いようもないほどきれい。	
\\	ようがない/ようもない
\\	質問の意味が分からなくて、答えようがなかった。	
\\	ようがない/ようもない
\\	電気炊飯器は、コンセントがなければ使いようがない。	
\\	ようがない/ようもない
\\	わたしは、あそこで死ぬと思っていた。お礼の言いようがない。	
\\	ようがない/ようもない
\\	この複雑な気持ちは表しようがない。	
\\	ようがない/ようもない
\\	ゴミがこれほど散らかっていたら、一人で全部集めようもない。	
\\	ようがない/ようもない
\\	彼女の住所も電話番号も分からないので、連絡の取りようがない。	
\\	ようがない/ようもない
\\	わたしのパソコンが、直しようがないほど壊れた。	
\\	ようがない/ようもない
\\	「なぜ」と聞かれても、言葉では説明のしようがない。	
\\	ようがない/ようもない
\\	締め切り直前になってテーマを変えるようでは、いい論文は書けないだろう。	
\\	ようでは
\\	これぐらいの困難にぶつかって落ち込むようでは何事もできない。	
\\	ようでは
\\	この程度の練習で文句をいうようでは次の試合に勝てないぞ。	
\\	ようでは
\\	こんな簡単な問題が解決できないようでは、困る。	
\\	ようでは
\\	こんな質問をするようでは、まだ勉強が足りない。	
\\	ようでは
\\	やがて別れがくるのだから、それまで仲良く暮らそう。	
\\	やがて
\\	彼女が写真を一瞬睨みつけたが、やがて勢いよく立ち上がった。	
\\	やがて
\\	誰も動かなかった。やがて何人かが悲鳴を上げた。	
\\	やがて
\\	彼は一瞬ちょっと不安そうな顔をしたが、やがて口を開いた。「後で食べるよ。」	
\\	やがて
\\	彼が群れの中から息を切らして現れた。	
\\	やがて
\\	要するに、彼女は行きたくないって言ったってことでしょう。	
\\	要するに
\\	要するに、もしお前が彼女を虐待していると、おれが耳にしたら殺すぞ。	
\\	要するに
\\	要するに、彼は騙されてるってことだな?	
\\	要するに
\\	要するに、ジェン、あなたは多くの人にとって、希望の象徴なのだ。	
\\	要するに
\\	ほかに食べ物がなかったので、ポテトを食べざるを得なかった。	
\\	ざるを得ない
\\	今回の旅行を楽しみにしていたが、母が急に入院したので、行くのをあきらめざるを得ない。	
\\	ざるを得ない
\\	もし力づくでも、ということならば、やらざるを得ぬ。	
\\	ざるを得ない
\\	留学したい気持ちは分かるが、この病状では、延期せざるを得ないだろう。	
\\	ざるを得ない
\\	わたしも認めざるを得なかった。	
\\	ざるを得ない
\\	日本で生活をするのなら、漢字を覚えざるを得ない。	
\\	ざるを得ない
\\	こんなにひどい雨では運動会は中止せざるを得ない。	
\\	ざるを得ない
\\	仕事はまだ終わらないから今日は残業せざるを得ない。	
\\	ざるを得ない
\\	いやな仕事でも、生活のためには続けざるを得ない。	
\\	ざるを得ない
\\	この機械は水に弱い。水がかかればそれまでだ。	
\\	ばそれまでだ
\\	どんなに美しい花でも、散ってしまえばそれまでだ。	
\\	ばそれまでだ
\\	案の定、筆記試験は難しかった。	
\\	案の定
\\	あの堤防は、案の定今回の台風で決壊してしまった。	
\\	案の定
\\	案の定、山田さんだった。	
\\	案の定
\\	彼女は車庫へと駆け出した。案の定、父の車はなかった。	
\\	案の定
\\	わたしはあくまでも彼女を見つけるつもりです。	
\\	あくまでも
\\	あの人はあくまでも積極的で真剣だ。	
\\	あくまでも
\\	君はあくまでも彼を憎みたいんだね。	
\\	あくまでも
\\	ちょっと試してみただけです。あなたはあくまでも正しいことがわかりました。	
\\	あくまでも
\\	あくまでも美しい夏の日だった。	
\\	あくまでも
\\	その人の動機はあくまでもりっぱな、義侠的なものなのですが、名は出したくないというのです。	
\\	あくまでも
\\	それは、否定すべくもない。	
\\	べくもない
\\	この絵が偽者であることなど、素人のわたしは知るべくもなかい。	
\\	べくもない
\\	彼が衝撃を受けているのは疑うべくもない。	
\\	べくもない
\\	30年のベテランに、初心者のわたしがかなうべくもない。	
\\	べくもない
\\	いまのままでは、状況の改善は望むべくもない。	
\\	べくもない
\\	土地が高い東京では、家などそう簡単に手に入るべくもない。	
\\	べくもない
\\	主力が抜けているので、優勝は望むべくもない。	
\\	べくもない
\\	嵐とは比べるべくもないが、ビッグバンは日本ではかなり有名なバンドです。	
\\	べくもない
\\	どんな小さな成功も努力あってのことだ。	
\\	あっての
\\	日々の練習あっての勝利だ。	
\\	あっての
\\	つらい治療に耐え、病気を克服することができたのは、家族の励ましがあってのことだ。	
\\	あっての
\\	三浦監督はわたしの恩人です。今のわたしがあるのも監督あってのことです。	
\\	あっての
\\	僕の幸せは家族あってのものだ。みんな病気をせず元気でいてほしい。	
\\	あっての
\\	うるさいと感じるかもしれないが、親があれこれ言うのはあなたのことを心配していればこそだ。	
\\	ばこそ
\\	社長は親切であればこそ十万円くれたのだ。	
\\	ばこそ
\\	愛していればこそ、別れるのです。わたしの気持ち、わかってください。	
\\	ばこそ
\\	子供のためを思えばこそ、費用は子ども自身に用意させたのです。	
\\	ばこそ
\\	優勝できたのは、チーム全員の協力あればこそだ。	
\\	ばこそ
\\	あなたに勉強させているのは、あなたの将来を考えればこそだ。	
\\	ばこそ
\\	きみのためを思えばこそ、叱るのだ。	
\\	ばこそ
\\	経営者にとって決断力は欠くべからざるものである。	
\\	べからず
\\	ここまで来るはよいが、先へ進むべからず。	
\\	べからず
\\	関係者以外立ち入るべからず。	
\\	べからず
\\	『ここで釣りをするべからず』という立て札が立っているのに、何人か釣りをしている人がいる。	
\\	べからず
\\	入り口の看板に『危険、入るべからず。危ない』って書いてあるよ。	
\\	べからず
\\	芝生に立ち入るべからず。	
\\	べからず
\\	初心、忘るべからず。	
\\	べからず
\\	彼女はあえて反論しなかった。	
\\	あえて
\\	田中は、ほかのだれにも原稿を見せていない理由にはあえてふれなかった。	
\\	あえて
\\	最期のことばをあえて英語で書いた理由はわからない。	
\\	あえて
\\	ただ人形を惜しんだが、あえてそれを口に出しはしなかった。	
\\	あえて
\\	誰を信用していいか分からないので、あえて本心をうち明けようとはしないのだ。	
\\	あえて
\\	誰もあえてそれを語ろうとする者もなく、あえてそれを思いだしてみようとする者もなかった。	
\\	あえて
\\	場合によっては使用が正当化される。	
\\	場合によっては
\\	ここで事をおこすと、全国の新聞が騒ぎたて、警察や、場合によっては軍隊をも動かすことになると思う。	
\\	場合によっては
\\	事情はきわめて複雑なもので、場合によっては一大醜聞となるおそれがある。	
\\	場合によっては
\\	彼のはただ闘争を楽しむために闘争をするので、時と場合によってはどちらの味方にでもつこうという代物だ。	
\\	場合によっては
\\	注意しておくが、お前のいうことは記録にのこって、場合によっては後でお前に不利な材料となるかもしれないから、そのつもりで。	
\\	場合によっては
\\	彼はサッカー選手になるべく、毎日厳しい練習をしている。	
\\	べく
\\	兄は締め切りに間に合わせるべく、昼も夜も論文に取り組んでいる。	
\\	べく
\\	気を静めるべく、指輪の紫のアメジストに思いをめぐらせた。	
\\	べく
\\	彼は一日も早く借金を返すべく、必死で働いている。	
\\	べく
\\	ウイルスの感染経路を明らかにすべく調査が行われた。	
\\	べく
\\	恩師に会うべく、久し振りに母校を訪れた。	
\\	べく
\\	彼はこの十年間人生を立てなおし、暴力に満ちた過去を消すべくつとめてきた。	
\\	べく
\\	開幕式に間に合わせるべく突貫工事が行われている。	
\\	べく
\\	あなたがそれをいかに見るかによりけりですよ。ある角度から見れば偶然の符合でも何でもない。	
\\	でも何でもない
\\	わたしは気がちがっているんでも何でもありません。	
\\	でも何でもない
\\	彼は病気でも何でもない。ただずる休みしたいだけかもしれない。	
\\	でも何でもない
\\	これはただの石で、金目のものでも何でもありません。	
\\	でも何でもない
\\	その書類というのは大切なものでも何でもない。	
\\	でも何でもない
\\	わたしが何者であろうと、そんなことはどうでもよいのだ。	
\\	であれ/であろうと
\\	君のパスポートを盗んだのが誰であれ、理由があって盗んだのだ。	
\\	であれ/であろうと
\\	わたしの生きるかぎり、どんな人のためであろうと、あの人と別れることはできないわ。	
\\	であれ/であろうと
\\	天気がどうであれ、彼女の声ほど冷たくはなかった。	
\\	であれ/であろうと
\\	キッチンを動き回っているのが誰であれ、声をひそめようとしていないことだけは確かだ。	
\\	であれ/であろうと
\\	うそをつくとは、どんな理由であれ、許されない。	
\\	であれ/であろうと
\\	僕はお寺だの、名所見物だのには大して興味はないんですがね。	
\\	だの~だの
\\	昨日泊まったホテルは、眺めだのサービスだの、本当に満足のいくものだった。	
\\	だの~だの
\\	彼女は安っぽいお追従だの、お世辞だの、そういったものならなんでも好きだったんです。	
\\	だの~だの
\\	わたし、午後だの、午前だのって、どっちがどっちか、わからなくなってしまうのよ。	
\\	だの~だの
\\	彼は、風邪を引いただの、頭が痛いだのと言って、よく学校をサボる。	
\\	だの~だの
\\	みんな申し分のないほどの緊張ぶりである。	
\\	ぶり
\\	だいぶ驚かしてしまったようですね。まず、わたしの怠慢ぶりで。それから、わたしの考え方で。	
\\	ぶり
\\	自分のチームの信頼性を疑われたとでも言いたげな口ぶりだ。	
\\	ぶり
\\	「いや」ジョンは本来のイライラぶりをちらっと見せた。	
\\	ぶり
\\	社長の話しぶりでは、今年のボーナスはないようだ。	
\\	ぶり
\\	みんなが彼女の笑いっぶりを見ている。	
\\	ぶり
\\	こんなことになるとは、夢にだに思わなかった。	
\\	だに
\\	思い出すだに怒りで熱くなる。	
\\	だに
\\	科学の力で自分と全く同じ人間が作り出せるなんて、そんな事を考えるだに恐ろしい。	
\\	だに
\\	全国にこの病気が広まったら、百万人もの人が死ぬなど、想像するだに恐ろしい。	
\\	だに
\\	いま往時を回想するだに暗然たるものあり。	
\\	だに
\\	彼はやっとものをいったが、その答えは思いだすだに気持のよいものだった。	
\\	だに
\\	自分が政治家になろうとは、想像だにしていなかった。	
\\	だに
\\	彼の片手には屋根部屋のにちがいない古びたさびだらけの鍵をもっている。	
\\	びる
\\	市中にちかい割に田舎びた閑静な場所です。	
\\	びる
\\	彼は中学生になったとたん、大人びたことを言うようになった。	
\\	びる
\\	ドアは古びて弱っていた。二人で力を合せてぶつかるといちどでこわれた。	
\\	びる
\\	どういうつもりか知らないが、あなたは免職どころではすまないぞ。	
\\	ではすまない
\\	課長である以上、そんな大事なことを知らなかったではすまされないだろう。	
\\	ではすまない
\\	この事件で、これまでのように単なる不安感ではすまなくなり、パニック状態が起こった。	
\\	ではすまない
\\	あなたにはどうにもできないことよ!	
\\	どうにも~ない
\\	彼はどうにも身の置き場がないような様子だ。	
\\	どうにも~ない
\\	わたしはどうにも納得できないので、この問題はどこまでも黒白をつけてやろうと決心しました。	
\\	どうにも~ない
\\	僕はこれがどうにも理解できない。	
\\	どうにも~ない
\\	どうにも寝つかれないものですから、あきらめて読みかけの小説でも読むつもりで、起きて電気をつけました。	
\\	どうにも~ない
\\	祖父のところで、どうにも動きが取れなかっただろ?	
\\	どうにも~ない
\\	あなたは、彼がどうにもすることのできないごく少数の人の一人でした。	
\\	どうにも~ない
\\	この女はどこまで厚かましいのか、どうにも理解できない。	
\\	どうにも~ない
\\	寒気がこみあげたとき、ふと恐怖を覚えた。	
\\	ふと
\\	彼は、ふとぎくりとして、我に返った。	
\\	ふと
\\	ふと気がつくと、Tシャツが体に張りついていた。	
\\	ふと
\\	わたしたちは、昨日ふと会話を交して、互いに興味を抱いたのです。	
\\	ふと
\\	僕はふと、彼女の声があまり乗り気ではないのに気づいた。	
\\	ふと
\\	彼女がふとこちらを見てわたしと目が合った。	
\\	ふと
\\	彼女は虚栄心にそそのかされて勤勉でありましたが、それは同時にうぬぼれた学者ぶった態度を身につきまとわせました。	
\\	ぶる
\\	彼は謙遜ぶった様子をして満足そうに天井を見つめていた。	
\\	ぶる
\\	あの人はいつも金持ちぶったが、実は借金だらけだ。	
\\	ぶる
\\	わたしにはとても理解できない。彼の文体にはもったいぶったところがある。	
\\	ぶる
\\	彼は、興奮して、もったいぶった顔つきをしてはいるが、内心の喜びを、上品ぶった冷静さの態度でかくそうとしていた。	
\\	ぶる
\\	彼女の尊大ぶった鼻柱をおりおりくじいてやった。	
\\	ぶる
\\	僕は逃げ道をあらかじめ作りました。	
\\	あらかじめ
\\	わたしは邸宅の外側をまわって、あらかじめあけておいた食堂の窓から邸内に入った。	
\\	あらかじめ
\\	入会をご希望の方はあらかじめ配布した用紙にご記入の上、お申し込みください。	
\\	あらかじめ
\\	彼女は二人の男を家の外へおびき出しておいて、あらかじめ撒いておいたガソリンに火をつけた。	
\\	あらかじめ
\\	パトカーサイレンを鳴らしてナイフのごとく車の群れを切り裂きました。	
\\	ごとき/ごとく
\\	暑い日に草むしりをしていたら、汗が滝のごとく流れてきた。	
\\	ごとき/ごとく
\\	今回のごとき事件は二度と起こしてはならない。	
\\	ごとき/ごとく
\\	わたしのごとき未熟者にそんな重要な役割が果たせるでしょうか。	
\\	ごとき/ごとく
\\	君ごとき若者に、まだまだわたしは負けまないよ。	
\\	ごとき/ごとく
\\	彼のごとき人物は忘れられて当然だ。	
\\	ごとき/ごとく
\\	彼女はまるで自分が女王であるかのごとく振舞っている。	
\\	ごとき/ごとく
\\	泥棒が脱兎のごとく店を飛び出した。	
\\	ごとき/ごとく
\\	ひとつ、これをごらんになってください。	
\\	ひとつ
\\	ひとつ、今度も君か僕に、無礼な事をした奴がいるかどうか、調べてみよう。	
\\	ひとつ
\\	ひとつ、やってくれたまえ!	
\\	ひとつ
\\	今晩はひとつけいこしておかなくっちゃ。	
\\	ひとつ
\\	わたしは、みんなが自分のことばかり考えていたらどんなことになるか、ひとつ教訓に経験させて上げようと思ったんですよ。	
\\	ひとつ
\\	おりてくるよう呼ばせなさい。ひとつわたしの考えをきかせよう。	
\\	ひとつ
\\	あの家を家財道具ぐるみそっくり買いとる用意があったのです。	
\\	ぐるみ
\\	冠の金地の一角が、そこについていた三個の緑柱石ぐるみ折りとられて、なくなっています。	
\\	ぐるみ
\\	ゆうべあいつの店に手入れがあって、書類ぐるみ監獄へ持ってゆかれましたぜ。	
\\	ぐるみ
\\	彼女とは長年家族ぐるみの付き合いをしている。	
\\	ぐるみ
\\	ここまできて、やつらを財宝ぐるみ捕えられなかったとしたら、それこそよっぽどどうかしているというものだ。	
\\	ぐるみ
\\	信用というものは、いったん失ったが最後、取り戻すのは難しい。	
\\	が最後
\\	ああいうおとなしい人って、みんなそうらしいんですけど、いったん、癇癪を起こしたが最後、手がつけられないほど腹を立てるんですね。	
\\	が最後
\\	うちの息子はパソコンの前に座ったが最後、声をかけても返事もしない。	
\\	が最後
\\	あの子はいったんに遊びに出たが最後、暗くなるまで戻ってこない。	
\\	が最後
\\	田中さんはカラオケが大好きで、マイクを握ったが最後、誰にも渡さない。	
\\	が最後
\\	不倫をしたが最後、なかなか元の家庭に戻れない。	
\\	が最後
\\	それを言ったが最後あなたたち二人の友情は完全に壊れてしまうよ。	
\\	が最後
\\	ではあるまいし/じゃあるまいし	
\\	のではあるまいし	
\\	のではあるまいし	ではあるまいし
\\	子供じゃあるまいし、もう泣くのをよして。	
\\	ではあるまいし
\\	面接試験は初めてではあるまいし、どうしてそんなに緊張するの?	
\\	ではあるまいし
\\	子供ではあるいまいし、暗いところが怖いなんて、おかしいね。	
\\	ではあるまいし
\\	あなたが悪かったわけではあるいまいし、そんなに自分を責めることはないよ。	
\\	ではあるまいし
\\	役者でもあるまいし、こんなに腹を立てているのにニコニコなんかしていられるものですか。	
\\	ではあるまいし
\\	コンピュータではあるまいしそんな難しい計算はわたしには無理だ。	
\\	ではあるまいし
\\	野菜が体にいいのはわかるが、馬じゃあるまいしこんなにたくさんは食べられない。	
\\	ではあるまいし
\\	まったく、子供じゃあるまいし、これぐらい自分で解決したらどうなんだよ。	
\\	ではあるまいし
\\	その日は一日じゅう、誕生日祝いのことばすらかけてもらえなかった。	
\\	(で)すら
\\	彼は数学が大変得意で、まだ小学生のときに、高校生ですら習っていない問題を解くことができた。	
\\	(で)すら
\\	銃弾が心臓をそれたのは残酷ですらある。	
\\	(で)すら
\\	重い病気のため、一人では食事すらできない。	
\\	(で)すら
\\	結婚なんてとんでもない。わたしは男の人と話したことすらないのです。	
\\	(で)すら
\\	大学の教授ですら、その問題には答えられなかった。	
\\	(で)すら
\\	昔は、顔を見たことすらない相手と結婚する人もいたそうだ。	
\\	(で)すら
\\	この悩みは親友にすら言えない。	
\\	(で)すら
\\	もうほかにわからないことはないかい?	
\\	ほかに~ない
\\	ほかに説明のつけようはない。	
\\	ほかに~ない
\\	「僕にもいっしょに行けっていうのかい?」「ほかに用事がないのなら来るさ」	
\\	ほかに~ない
\\	わたしたちは、わたしたちだけで、ほかには誰も乗っていない一等車に腰をおろしていた。	
\\	ほかに~ない
\\	ほかにご用はございませんでしょうか?	
\\	ほかに~ない
\\	社長はほかには一言もいわずに立ちあがった。	
\\	ほかに~ない
\\	ほかに言うことはなかった。	
\\	ほかに~ない
\\	この地域には、ほかには誰もいないと思う。	
\\	ほかに~ない
\\	地震であれ火事であれ、緊急の場合に冷静になれる人は少ないだろう。	
\\	であれ~であれ
\\	正社員であるパートであれ、仕事に対する責任は変わりません。	
\\	であれ~であれ
\\	文学であれ音楽であれ、芸術には才能が必要なのだ。努力だけではだめなのだ。	
\\	であれ~であれ
\\	晴天であれ雨天であれ遠足は予定どおり行く。	
\\	であれ~であれ
\\	貧乏であれ、お金持ちであれ、自分の夢を失ってはいけない。	
\\	であれ~であれ
\\	男であれ女であれ料理はできた方が良いです。	
\\	であれ~であれ
\\	国会議員であろうと、公務員であろうと、税金は納めなければならない。	
\\	であれ~であれ
\\	日本人であれ外国人であれ、法律には従わなければならない。	
\\	であれ~であれ
\\	一度聞いただけの曲をかんぺきに演奏できるなんて、彼は天才でなくてなんだろう。	
\\	でなくてなんだろう
\\	戦争で多くの人が殺されているとは、悲劇でなくてなんだろう。	
\\	でなくてなんだろう
\\	このようなところに道路を作るのは、税金の無駄遣いでなくてなんだろう。	
\\	でなくてなんだろう
\\	彼女に対するこの気持ちは、愛でなくてなんだろう。	
\\	でなくてなんだろう
\\	泳げもしないのに海へ飛び込んだ。これが無謀でなくてなんだろう。	
\\	でなくてなんだろう
\\	自分の命を犠牲にして多くの人を救ったあの男が英雄でなくてなんだろう。	
\\	でなくてなんだろう
\\	彼はこれで落ちつくねぐらも愛する女も、いっぺんに失う羽目になった。	
\\	羽目になる
\\	うっかりしてデータを消してしまったので、もう一度入力する羽目になった。	
\\	羽目になる
\\	この家で何かがおかしくなると、やはり僕がとがめられる羽目になった。	
\\	羽目になる
\\	もしわたしのせいで、彼が刑務所に行く羽目になったら…	
\\	羽目になる
\\	彼女は自分の死刑執行命令書に署名する羽目になってしまった。	
\\	羽目になる
\\	な
\\	ではあるまいか	
\\	のではあるまいか	
\\	ではあるまいか	ではあるまいか
\\	あの男の父親がジョンソンのために没落したのではあるまいか。	
\\	ではあるまいか
\\	このごろはあまり彼の姿を見かけない。どこかへ行ってしまったのではあるまいか。	
\\	ではあるまいか
\\	彼女は変わり者の友だちを怒らせたのではあるまいかと心配になった。	
\\	ではあるまいか
\\	彼は夜になるまで起きないのではあるまいかと思い始めていた。	
\\	ではあるまいか
\\	彼女が何かを見誤ったのではあるまいかと首をひねっている。	
\\	ではあるまいか
\\	あの広告を見たとき、罠ではあるまいかとだいぶ迷いました。	
\\	ではあるまいか
\\	電車が着くが早いか、乗客は先を競って乗り込んだ。	
\\	が早いか
\\	彼女は手紙を取り上げるが早いか、封を切った。	
\\	が早いか
\\	わたしの顔を見るが早いか、彼はしゃべりはじめた。	
\\	が早いか
\\	授業終了のベルを聞くが早いか、生徒たちは教室を飛び出していった。	
\\	が早いか
\\	先生が地震だと叫ぶが早いか、子供たちは机の下にもぐりこんだ。	
\\	が早いか
\\	うちの子はいつも学校から帰るが早いか、遊びに行ってしまう。	
\\	が早いか
\\	あの人はいつも電車に乗り込むが早いか、席を確保しようとする。	
\\	が早いか
\\	大阪で警察に逮捕されたほうがましだったかもしれないと思う。	
\\	ほうがましだ
\\	あんなことになるなら、死んだほうがましだ。	
\\	ほうがましだ
\\	そんな奴なら、死んだ方がましだ。	
\\	ほうがましだ
\\	彼と結婚する?わたしなら、悪夢のほうがましだ!	
\\	ほうがましだ
\\	そんなことするくらいなら、五階から真逆さまに敷石の上に身を投げたほうがましだわ。	
\\	ほうがましだ
\\	いずれにせよ、きみには援護など必要ない。	
\\	いずれにせよ
\\	いずれにせよ、わたしは試験のことを考えたくない。	
\\	いずれにせよ
\\	いずれにせよ、やつの関心は、配下を集めることだけじゃない。	
\\	いずれにせよ
\\	いずれにせよ、あなたは物がわかっているようね。	
\\	いずれにせよ
\\	いずれにせよ時間がない。	
\\	いずれにせよ
\\	いずれにせよこのピストルは役に立つかもしれない。	
\\	いずれにせよ
\\	出席状況・学業成績のいかんでは、奨学金の支給を停止することもある。	
\\	いかんでは/いかんによっては
\\	筆記試験はパスした。明日の面接の結果いかんで採用が決まるそうだ。	
\\	いかんでは/いかんによっては
\\	試験の結果いかんでは、卒業できないこともある。	
\\	いかんでは/いかんによっては
\\	成功するかどうかはみんなの努力いかんだ。	
\\	いかんでは/いかんによっては
\\	検査の結果いかんでは、入院もありえます。	
\\	いかんでは/いかんによっては
\\	理由のいかんによらず、殺人は許されないことだ。	
\\	いかんによらず
\\	出席欠席のいかんによらず、メールで返事するようお願いします。	
\\	いかんによらず
\\	採否のいかんにかかわらず、結果は郵便でお知らせします。	
\\	いかんによらず
\\	理由のいかんによらず、試験開始後の入室は認めません。	
\\	いかんによらず
\\	ゆうべは、あたし、気ちがいじみた事をしてしまった。	
\\	じみた
\\	あの子はよく年寄りじみたことを言う。	
\\	じみた
\\	彼女はあんな子供じみたことは止めたんだなと思ったのです。	
\\	じみた
\\	彼の両眼は大きく見ひらかれて狂気じみた光りをおびている。	
\\	じみた
\\	いかにもそれは本気の沙汰だった。	
\\	いかにも
\\	姉がいかにも満足そうに、明るい様子に見えた。	
\\	いかにも
\\	小さな娘はいかにもうまく泣きだした。	
\\	いかにも
\\	いかにも鈴木さんはあまり口をききませんでした。	
\\	いかにも
\\	彼女の眼と口もとにはいかにもうれしくてならない輝かしさがあった。	
\\	いかにも
\\	世界中あちこちを旅行するなんて、うらやましい限りだ。	
\\	限りだ
\\	この近所で殺人事件があったらしい。恐ろしい限りだ。	
\\	限りだ
\\	彼女が遠くに引っ越すのは、寂しい限りだ。	
\\	限りだ
\\	みんな無事に帰ってきて、うれしい限りです。	
\\	限りだ
\\	あんなことを言ってしまって、お恥ずかしい限りです。	
\\	限りだ
\\	親友の結婚式に出られないとは、残念な限りだ。	
\\	限りだ
\\	自分の作品がこんなに大勢の人に評価されるとは、本当にうれしい限りだ。	
\\	限りだ
\\	いまいちど、被告とハンニバル・レクターとが同一人か否かということを証人たちに尋問したいのです。	
\\	か否か
\\	話を逸らすな。いまはショーンという男子とデート中か否か?	
\\	か否か
\\	彼のアリバイは真物か否か、それを判断しなければなりません。	
\\	か否か
\\	答えるべきか否か、彼はわたしの品定めをしているようだ。	
\\	か否か
\\	では、どうすればいいんです。おそかれ早かれ
\\	かれ~かれ
\\	今後は良かれ悪しかれ彼の家族に干渉しないほうがいいと思う。	
\\	かれ~かれ
\\	多かれ少なかれ、みんな悩みを抱えている。	
\\	かれ~かれ
\\	今となっては、わたしたちはあの子供の運命を善かれ悪しかれ支配する力がないのだから、あの子がどこに行ったかを娘にはけっして言ってはいけない。	
\\	かれ~かれ
\\	とにかく彼女は、遅かれ早かれ聞きださずにはおかないから、わたしの口から彼女の耳に入れても同じことである。	
\\	かれ~かれ
\\	何もそんなに急ぐほどのことはなかったとわかりましたよ。	
\\	ほどのことはない
\\	人に隠れてお話しするほどのことはないと思いました。	
\\	ほどのことはない
\\	待つほどの事もなく、約一分後に彼が戻ってきた。	
\\	ほどのことはない
\\	彼は教師の仕事をするかたわら小説を書いている。	
\\	かたわら
\\	彼女は会社勤めのかたわら、ボランティア活動に積極的に取り組んでいる。	
\\	かたわら
\\	わたしは学校に通うかたわら、夜はスーパーで働いている。	
\\	かたわら
\\	松本潤は歌手としての活動のかたわら、俳優としても活躍している。	
\\	かたわら
\\	彼がみんなの皿一杯にたっぷり肉を切って分けるかたわら、奥さんはおもしろおかしく話して皆を喜ばせました。	
\\	かたわら
\\	父は農業のかたわら、小さい店を経営している。	
\\	かたわら
\\	事故であったのか、自殺であったのか、いまだに明らかになっておりません。	
\\	いまだに
\\	その表情が映画のワンシーンのように、いまだに頭に残っています。	
\\	いまだに
\\	昔の記憶は、いまだに頭から離れない。	
\\	いまだに
\\	彼はいまだにまったく立場を変えていない。	
\\	いまだに
\\	いったい侵入者がどうやって侵入したのか、わたしにはいまだにわかりません。	
\\	いまだに
\\	いまだに彼女を恐れている。	
\\	いまだに
\\	友達を駅まで送りがてら本屋に行ってきた。	
\\	がてら
\\	週末にはドライブがてら、新しい博物館まで行ってみようと思う。	
\\	がてら
\\	散歩がてらちょっとタバコを買ってきます。	
\\	がてら
\\	桜を見がてら隣の駅まで歩いた。	
\\	がてら
\\	あの娘なら知らないものはありません。この付近きっての美人です。	
\\	きっての
\\	彼はこの地方きっての大地主です。	
\\	きっての
\\	めぐれ警部は警視庁きっての敏腕家とうたわれます。	
\\	きっての
\\	彼はニューヨークきっての凶悪漢です。	
\\	きっての
\\	ごあいさつかたがた、お伺いしました。	
\\	かたがた
\\	散歩かたがた、友達を訪ねた。	
\\	かたがた
\\	先日お世話になったお礼かたがた、社長のお宅にお寄りしました。	
\\	かたがた
\\	毎朝、犬の散歩かたがた公園のごみ拾いをしている。	
\\	かたがた
\\	本日は先日のお詫びかたがた、お伺いしました。	
\\	かたがた
\\	本日はお礼かたがたお伺いしました。	
\\	かたがた
\\	わたしはかつて、ある手術をうけたことがあった。	
\\	かつて
\\	目の前にあるのは、彼がかつて見たこともない異様な絵図だった。	
\\	かつて
\\	彼女は美しかったことはかつていちどもない。	
\\	かつて
\\	身元を明かさない使者はかつてこう言った。	
\\	かつて
\\	彼は無害な動物は全然殺さなかった。また彼はかつて小鳥を撃ったことがなかった。	
\\	かつて
\\	わたしは、彼の失礼きわまりない態度に我慢できなかった。	
\\	極まる/極まりない
\\	僕にとっては、彼女が生きていることは危険きわまることなんだ。	
\\	極まる/極まりない
\\	彼は首相として、無責任きわまりない。	
\\	極まる/極まりない
\\	あいつはおそらく、危険きわまる殺人者だろう!	
\\	極まる/極まりない
\\	「いま忙しい」迷惑きわまりないと言いたげに、彼女は答えた。	
\\	極まる/極まりない
\\	昨日見た映画は、平凡極まる内容だった。	
\\	極まる/極まりない
\\	こんな退屈極まりない仕事を辞めたい。	
\\	極まる/極まりない
\\	感極まって、彼女はナプキンに顔を埋めた。	
\\	極まる/極まりない
\\	いかに 
\\	いかに 
\\	いかに 
\\	いかに
\\	その絵がいかに価値のあるものかを、彼女は知っていたのだろうか。	
\\	いかに
\\	この計画を成功させることがいかに重要か分かっているのか。	
\\	いかに
\\	いかに強い男とはいえ、それはあまりにもひどすぎた。	
\\	いかに
\\	おれは、いかに勇敢さに価値があるかを知っている。	
\\	いかに
\\	いかに大事なものかはおまえも承知だろう。	
\\	いかに
\\	いかに考えてみても、みずからそれを説明することができなかった。	
\\	いかに
\\	彼女の燃えるようなひとみは、いかなる言葉よりも雄弁に彼女の胸中を語っていた。	
\\	いかなる
\\	信仰にはいかなる物質的な裏づけも必要無い。	
\\	いかなる
\\	彼がいかなる理由で、突然に考えを変えたかはうかがい知ることはできない。	
\\	いかなる
\\	面接の結果についてはいかなる点から考えても満足すべきものだ。	
\\	いかなる
\\	わたしにはいかなる動機も思いあたりません。	
\\	いかなる
\\	二日まえに極めて警戒を要する報道が耳に達しました。	
\\	極めて
\\	彼の態度は、きわめてやさしかった。	
\\	極めて
\\	これは、極めて異様な解明であって、それが真実の唯一の解明であるかどうかは、まだ断定はできない。	
\\	極めて
\\	その石板には、きわめて重要な秘密の隠された場所が記されている。	
\\	極めて
\\	ピーターと僕とは、きわめて仲良くやっていましたよ。	
\\	極めて
\\	彼女は極めて気軽に立って一同を迎えられました。	
\\	極めて
\\	今のところあの子の健康はきわめて心もとないです。	
\\	極めて
\\	きわめて悪い報告です。	
\\	極めて
\\	人はとかく自身に都合がいい意見にのみ耳を傾けるきらいがある。	
\\	嫌いがある
\\	彼はどうも物事を悲観的に考えるきらいがある。	
\\	嫌いがある
\\	父は自分と違う考え方を認めようとしないきらいがある。	
\\	嫌いがある
\\	人は年をとると、周りの人の忠告に耳を貸さなくなるきらいがある。	
\\	嫌いがある
\\	彼女は、何でもものごとを悪い方に考えるきらいがある。	
\\	嫌いがある
\\	あの人はものごとを大げさに言うきらいがある。	
\\	嫌いがある
\\	彼は物事を少し考えすぎるきらいがある。	
\\	嫌いがある
\\	インドではまだ女性を低く見るきらいがある。	
\\	嫌いがある
\\	皆さんお帰りになったことだし、そろそろ会場を片付けましょう。	
\\	ことだし
\\	木村さんのことだし、隠す必要もありませんが、いま胸中ある問題で悩んでいるのは事実ですよ。	
\\	ことだし
\\	時間もあることだし、お茶でも飲みましょうよ。	
\\	ことだし
\\	現在の良人とは結婚してから三年にもなることだし、もはや自分の地位は確実だと安心している。	
\\	ことだし
\\	雨もやんだことだし、そろそろ帰りましょうか。	
\\	ことだし
\\	まだよいのうちで、人通りもあることだし、金の像などもって歩いたら、人目につく。	
\\	ことだし
\\	あの人も謝ったことだし、許してあげよう。	
\\	ことだし
\\	からの 
\\	からある 
\\	からする 
\\	からある
\\	そのデモには20万人からの人が参加したそうだ。	
\\	からある
\\	100キロからある荷物を5階まで運ぶには、足腰の強い人が3人は必要だ。	
\\	からある
\\	今年のスピーチコンテストには1000人からの人が集まった。	
\\	からある
\\	長さ5メートルからある蛇が、突然、目の前に現れた。	
\\	からある
\\	彼女は5億円の遺産を相続したそうだ。	
\\	からある
\\	あなたの望みはかなう可能性がある。	
\\	可能性がある
\\	あなたと一緒に行ったら何がおこる可能性があるか、わたしたちが完璧によく知ってる。	
\\	可能性がある
\\	君は偉大になれる可能性があるんだよ。	
\\	可能性がある
\\	クレジットカードを使ったら、追跡される可能性がある。	
\\	可能性がある
\\	あの人は、自分の利益のためなら、子供みたいに平気で、嘘をつく可能性がある。	
\\	可能性がある
\\	彼にはまだ商談成立の可能性がある。	
\\	可能性がある
\\	君の愚かさを彼らが利用している可能性がある。	
\\	可能性がある
\\	そういう可能性があったかもしれないと思っただけのことですよ。	
\\	可能性がある
\\	彼の信頼を裏切るくらいなら、むしろ死を選ぶ。	
\\	くらいなら
\\	どうせ旅籠屋で泊るくらいなら、自分の家で一晩過すのもよかろうと思った。	
\\	くらいなら
\\	この秘密があばかれて、子供たちを汚らわしい家名に苦しめるくらいなら、いっそわたしが死刑になってもよいと思ったのです。	
\\	くらいなら
\\	あの人と結婚できないくらいなら死んじまうわ。	
\\	くらいなら
\\	父は、おれを自分のような貿易商にしたいんだ、ところがおれはそんなものになるくらいなら、弾に撃たれて死んだほうがいい。	
\\	くらいなら
\\	こんなところへ来るくらいなら、戦場にいたほうが遥かに安全でしたよ。	
\\	くらいなら
\\	おれが二度とお前を横目でだって見るくらいなら、からだも魂もとっくに地獄に行っちまう方がましだ。	
\\	くらいなら
\\	待ったかいもなく、彼はとうとう帰ってこなかった。	
\\	かいもなく
\\	必死の練習のかいもなく、オリンピックの代表選手には選ばれなかった。	
\\	かいもなく
\\	手術のかいもなく、ガンが移転してしまった。	
\\	かいもなく
\\	部屋にいたものはことごとく驚いた。	
\\	ことごとく
\\	彼は、友達に家に遊びに来られるのをことごとく嫌がる。	
\\	ことごとく
\\	彼女は、あらゆる不運の途をことごとくとおってきた。	
\\	ことごとく
\\	彼の推測はことごとく破れてしまった。	
\\	ことごとく
\\	彼らが考えていることはことごとく、正常なものには考えられないようなことだった。	
\\	ことごとく
\\	急なこととて、なにもご用意できなくて…	
\\	こととて
\\	なにぶん新入社員のこととて、失礼があれば、どうかお許しください。	
\\	こととて
\\	まだ検討中のこととて、はっきりとしたお返事はできません。	
\\	こととて
\\	何も知らぬこととて、失礼をいたしました。	
\\	こととて
\\	引っ越してきたばかりで、辺りの様子も分からないこととて、どうぞよろしくお願いいたします。	
\\	こととて
\\	連絡もなしにお客様がいらっしゃったが、急なこととて、なんのおもてなしもできなかった。	
\\	こととて
\\	年末年始は休業中のこととて、この時期の依頼にはすぐに対応ができません。	
\\	こととて
\\	あなたの中の彼の思い出を、あんな人たちに汚させるままにできるわね。	
\\	ままに
\\	扉はやはり音もなく、押されるままに動いた。	
\\	ままに
\\	契約のとき、言われるままにハンコを押している人が多い。	
\\	ままに
\\	わたしに立ち向かう代りに、お前のために友人たちを死なせるままにさせた。	
\\	ままに
\\	彼女がしゃべるままにしておいた。	
\\	ままに
\\	父のほうに目を向けて、妹が一晩中歌いつづけることのないよう干渉を求めました。	
\\	ことのないよう
\\	二度と同じミスを繰り返すことのないよう、指摘されたことをメモしておいた。	
\\	ことのないよう
\\	今後、このような犯罪と関わることのないように注意してください。	
\\	ことのないよう
\\	道に迷うことのないように、前もって地図で場所を確かめておいた。	
\\	ことのないよう
\\	パソコンはまるっきり初心者なので、いろいろ教えてください。	
\\	まるっきり
\\	面接のことはまるっきり忘れてしまった。	
\\	まるっきり
\\	今度の試合は状況がまるっきり違うのだ!	
\\	まるっきり
\\	歴史劇ってまるっきり興味ない。	
\\	まるっきり
\\	彼らはわれわれとはまるっきり違うんだから。	
\\	まるっきり
\\	社長の態度がまるっきり変わってしまいました。	
\\	まるっきり
\\	らきりがない	
\\	-ば
\\	きりがない	
\\	ときりがない	きりがない
\\	ベトナム戦争に関する本は数え上げたらきりがない。	
\\	きりがない
\\	彼の欠点を数えればきりがない。	
\\	きりがない
\\	邸宅はたしかにある富豪によって建てられたもので、贅沢きわまりないものであるといわれている。	
\\	もので
\\	そういう連中は、普通は昼間眠っているもので、夜になってかせぎに出るものらしい。	
\\	もので
\\	事件の真相を明らかにしたのは彼の天才によるもので、もし工藤・新一がいなかったら、果たして犯人をつきとめることができたかどうかは疑問である。	
\\	もので
\\	あの人はあなたが社長に信用されているのを知っているもので、いろいろと聞きだしに来たんだよ。	
\\	もので
\\	注文は、手紙で来たもので、郵便為替が同封してあった。	
\\	もので
\\	そういう失敗は誰にでもあるもので、その錯誤を自覚して、是正する人がえらいのだ。	
\\	もので
\\	これは信任されている秘書にもあるまじき行為ですが、こうせっぱ詰まっては止むを得ません。	
\\	まじき
\\	親を騙すなんて、子供にあるまじき行為だ。	
\\	まじき
\\	弱いものをいじめるなんて、許すまじきことです。	
\\	まじき
\\	彼のやったことは、人としてあるまじき残酷な行為だ。	
\\	まじき
\\	一言声をかけてくれれば手伝ったものを。	
\\	ものを
\\	よせばいいものを、彼は社長に文句を言った。そのため首になった。	
\\	ものを
\\	黙っていればいいものを、つい余計なことを言ってしまった。	
\\	ものを
\\	わたしに連絡してくれれば、迎えに行ってあげたものを。	
\\	ものを
\\	ヘルスチェックを受けていればすぐに治ったものを。	
\\	ものを
\\	やればできるものを、どうしてやらないんですか。	
\\	ものを
\\	あなたがお金にこまっていると知っていれば、何とか力になったものを。	
\\	ものを
\\	簡単でむずかしいとは、なんだかすこし逆説めいているぜ。	
\\	めく
\\	彼女はあなたに対する脅迫めいたことを言っていましたか?	
\\	めく
\\	あの女にたいして彼が恋愛めいた感情をいだいているというのではない。	
\\	めく
\\	利口な男はどこか気ちがいめいたところがある。	
\\	めく
\\	教授の謎めいた言葉にどぎもをぬかれて、彼は何を言っていいのか、次にどうしたらいいのか途方にくれているようであった。	
\\	めく
\\	箱の中に血まみれのナイフがある。	
\\	まみれ
\\	みんなは汗まみれの泥だらけで、体があちこち痛んだ。	
\\	まみれ
\\	彼は埃まみれのやかんを取り上げ、中を覗いた。	
\\	まみれ
\\	子供たちはインクまみれになった教科書を拾っている。	
\\	まみれ
\\	彼女の服は破れて泥まみれだった。	
\\	まみれ
\\	少なくともわたしには二人が泣いているものと思われました。	
\\	ものと思われる
\\	うたがいもなく、殺人者は怒りにまかせてやったものと思われます。	
\\	ものと思われる
\\	今回の調査で事故の原因が明らかになるものと思われる。	
\\	ものと思われる
\\	この話から察してみますと、彼は、周囲の同情が全然ないので、利己的な不快な子になったものと思われます。	
\\	ものと思われる
\\	彼は死んだものと思われていた。	
\\	ものと思われる
\\	彼はジャックの疑惑の眼もかえりみずにジョージ老人の生命を救った。	
\\	も顧みず/を顧みず
\\	おじゃま申し上げておゆるしいただきます。東京から何かおよろしいおしらせでもあったかと存じまして、失礼をかえりみずおうかがい申しあげますので。	
\\	も顧みず/を顧みず
\\	彼は自らの命もかえりみず戦場に向かった。	
\\	も顧みず/を顧みず
\\	わたしは落ちてるものを拾いあげたまでだ。	
\\	までだ
\\	就職が決まらなくても困らない。アルバイトをして生活するまでだ。	
\\	までだ
\\	ここじゃ日本の法律に従ってやるまでだ。	
\\	までだ
\\	君はこの土地は初めての人らしいから、よかれと思って注意したまでだよ。	
\\	までだ
\\	誰も協力してくれないのなら、一人でやってみるまでだ。	
\\	までだ
\\	条件さえ守ってくだされば、一週間に一回、ずっとこれと同じに払いますよ。いけなければ、それまでですがね。	
\\	までだ
\\	あなたは不運だったまでですよ。	
\\	までだ
\\	あなたのご意見は反対なのですから、事はそれまでです。	
\\	までだ
\\	試験まであと一週間しかない。もはや一日たりとも無駄にはできない。	
\\	もはや
\\	言いきれることなど、もはや何もない。	
\\	もはや
\\	僕はもはや自分が誰なのかわからない。	
\\	もはや
\\	助けが来るころに、自分の命はもはやあるまい。	
\\	もはや
\\	彼がもはや以前と同じ人間ではなく、彼の内部がすべて変化していたということである。	
\\	もはや
\\	彼女は、もはや誰に訴えるすべもなかった。	
\\	もはや
\\	その名前はもはや僕をおびやかさない。	
\\	もはや
\\	聞きもしないで、どうしてそんなことがわかるんだい?	
\\	もしないで
\\	彼女は好きなワルツが始まってもふりむきもしないで、窓ガラスに額をおしつけた。	
\\	もしないで
\\	彼女は書類をろくに読みもしないで、「もうわたしに用はないわね」と言って立ち上がった。	
\\	もしないで
\\	父は手紙を受けとって、読みもしないでポケットにおしこんだ。	
\\	もしないで
\\	彼はある書類を他の書類の間へ紛れこませてしまって、読みもしないで署名をした。	
\\	もしないで
\\	老人はその箱を手にとって、しばらく開きもしないでじっとながめた。	
\\	もしないで
\\	父もしくはわたしがお伺いします。	
\\	もしくは
\\	それでわたしはけさ、あなたにすべてを告白しにきたのです。すべてを、もしくはほとんどすべてをです。	
\\	もしくは
\\	それは、彼が、予想もしくは理解することができないことだった。	
\\	もしくは
\\	同封の許可証にご両親もしくは保護者の同意署名をもらってください。	
\\	もしくは
\\	息子は一流の音楽家になるといって家を出た。大変だが、きっと目的を達成しないではおかないだろう。	
\\	ないではおかない/ずにはおかない
\\	今度こそ、本当のことを言わせないではおかないぞ。	
\\	ないではおかない/ずにはおかない
\\	新企画の中止が決まろうとしているが、担当した者たちは反対しないではおかないだろう。	
\\	ないではおかない/ずにはおかない
\\	この映画は評判が高く、見るものを感動させずにはおかないだろう。	
\\	ないではおかない/ずにはおかない
\\	今度こんなことをしたら、警察に届けないではおかない。	
\\	ないではおかない/ずにはおかない
\\	あれほど注意したにもかかわらず試合でカンニングをするものには、厳しい罰を与えずにはおかない。	
\\	ないではおかない/ずにはおかない
\\	親友に裏切られたのだから、仕返しをしないではおかない。	
\\	ないではおかない/ずにはおかない
\\	人間は生まれながらにして平等である。	
\\	ながらに/ながらの
\\	彼は自らのつらい体験を涙ながらに語った。	
\\	ながらに/ながらの
\\	彼女は片手をのどへやってむせび泣きながらの話なので、終りのほうはよく聞きとれなかった。	
\\	ながらに/ながらの
\\	彼は生れながらにして、人をひきつける力をそなえている。	
\\	ながらに/ながらの
\\	今日は会社の同僚といつもながらのところで夕食を済ませた。	
\\	ながらに/ながらの
\\	ありがたいことに電話というものがある。わたしはいながらにして重要な情報は集められるのさ。	
\\	ながらに/ながらの
\\	その店は、昔ながらの製法で豆腐を作っている。	
\\	ながらに/ながらの
\\	人の心を傷つけてしまったなら、謝らないではすまない。	
\\	ないではすまない/ずにはすまない
\\	うそがばれてしまったら、本当のことを教えないではすまない。	
\\	ないではすまない/ずにはすまない
\\	お金を黙って持ち出したなんて、親に知られたら叱られないではすまないぞ。	
\\	ないではすまない/ずにはすまない
\\	あんな高価なものを壊したのだから、弁償しないではすまない。	
\\	ないではすまない/ずにはすまない
\\	あの社員は客の金を使ったのだから処罰されずにはすまないだろう。	
\\	ないではすまない/ずにはすまない
\\	そのパーティーにみんな出席するのなら、わたしも行かずにはすまない。	
\\	ないではすまない/ずにはすまない
\\	言うまでもないが、鍵をかけておいたほうがいい。	
\\	までもない
\\	そんなことは常識だ。きみに言われるまでもない。	
\\	までもない
\\	死ぬほど空腹で喉がかわいているのは、いうまでもないが、寒さで感覚がなかった。	
\\	までもない
\\	このぐらいの雨なら、傘をさすまでもない。	
\\	までもない
\\	予約したホテルは駅を降りたら探すまでもなく目の前にあった。	
\\	までもない
\\	年金にありつきたいのは言うまでもない。	
\\	までもない
\\	言うまでもなく、学生にとっては勉強が一番大切だ。	
\\	までもない
\\	比べるまでもない。	
\\	までもない
\\	電話番号なら調べるまでもない。ここに書いてある。	
\\	までもない
\\	そんな遠い店まで買いに行くまでもないよ。電話で注文すればすぐ届くんだから。	
\\	までもない
\\	プロにはなれないまでも、サッカーを続けていくつもりだ。	
\\	ないまでも
\\	全額でないまでも、せめて利子ぐらいは払ってくれ。	
\\	ないまでも
\\	彼が犯人だと断定できないまでも、いろいろと怪しいところがある。	
\\	ないまでも
\\	見舞いにこないまでも電話ぐらいはするものだ。	
\\	ないまでも
\\	昨日の演奏は、最高の出来とは言えないまでも、かなりよかったと思う。	
\\	ないまでも
\\	結婚式には出席できないまでも、お祝いのプレゼントくらいは贈ろう。	
\\	ないまでも
\\	プロのコックとは言わないまでも、彼の料理の腕はなかなかのものだ。	
\\	ないまでも
\\	彼は彼女に完全におちついたといえないまでも、少なくとも完全に礼儀正しい言葉で話しかけました。	
\\	ないまでも
\\	なにも知らないものとして教えてくれたまえ。	
\\	ものとして
\\	わたしはいないものとして話を進めてください。	
\\	ものとして
\\	木村さんはもう来ないものとして、始めましょう。	
\\	ものとして
\\	最近、運動不足だ。何しろ外出といったら近所のスーパーに行く程のもので、あとは朝から晩まで家にいるという生活なのだ。	
\\	何しろ
\\	何しろ僕たちには不思議に似たところがある。	
\\	何しろ
\\	誰一人としてテイラー一家のために悲しみにくれるような無駄はしなかった。何しろこの一家はこの上なく評判が悪かった。	
\\	何しろ
\\	わたしたちの部は今ものすごく忙しいんですよ。何しろワールドカップの手配何かがいろいろ。	
\\	何しろ
\\	ここで命令を出すのはご遠慮願いたいですな。なにしろ、わたしの家なのでね。	
\\	何しろ
\\	小学生ならいざ知らず、大学生がこんな簡単な計算ができないなんて信じられない。	
\\	ならいざしらず
\\	お泊り客ならいざ知らず、普通新宿通りへおいでになるお客様をわたしがお見かけするようなことはございません。	
\\	ならいざしらず
\\	安いホテルならいざ知らず、一流ホテルでこんなにサービスが悪いなんて許せない。	
\\	ならいざしらず
\\	祖父母の代ならいざ知らず、今の時代に「手ぬぐい」なんてあまり使わないよ。	
\\	ならいざしらず
\\	新入社員ならいざ知らず、入社10年にもなる君がこんなミスをするとは信じられない。	
\\	ならいざしらず
\\	20年前ならいざ知らず、今時、テレビを持っていない人は珍しい。	
\\	ならいざしらず
\\	子供ならいざ知らず、大人がこんなミスをするなんて、許せない。	
\\	ならいざしらず
\\	何とかしてこの束縛を断つ方法はないものか?	
\\	ないものか
\\	何とかして金だけでも取戻せないものか?	
\\	ないものか
\\	このにおい、なんとかならないものか?	
\\	ないものか
\\	何とかして超自然でない説明がつけられないものか?	
\\	ないものか
\\	アイパッドが壊れてしまった。何とかして直せないものだろうか。	
\\	ないものか
\\	疲れきっているのに、どうしてか眠れない。何とかして眠れないものか?	
\\	ないものか
\\	君は何も理由なくしてその場にいるということはない。	
\\	なくして
\\	「努力なくして成功はない」という言葉は、父の口癖だ。	
\\	なくして
\\	監督の熱心な指導なくしては、今回の優勝はなかっただろう。	
\\	なくして
\\	コミュニケーションなくしては、人間が関与する一切の活動は成立しえない。	
\\	なくして
\\	苦難に満ちた母の人生は、涙なくしては語れない。	
\\	なくして
\\	本人のやる気なくしては治療の意味がありません。	
\\	なくして
\\	アメリカにいたころ、子供同士の友達だったんです。こっちへ来てから、一二度出会ったくらいのもので、しみじみと会って、話したということはなかったです。	
\\	くらいのものだ
\\	わたしのほうでできることは、あなたと連絡を保つことくらいのものです。	
\\	くらいのものだ
\\	僕が買える家は、これぐらいのものだ。	
\\	くらいのものだ
\\	「承知しましたとも。それだけですか?」「あとはここでいっしょに食事していただくことくらいのものです。」	
\\	くらいのものだ
\\	彼が仕事を辞めないように説得できるのは、あなたぐらいのものです。	
\\	くらいのものだ
\\	わたしにはわたしなりの理由があっての事だったんです。	
\\	なりに/なりの
\\	あなたがどのぐらい彼女のことを好きなのか、彼女なりのやり方で試そうとしたのよ。	
\\	なりに/なりの
\\	僕なりの考えはある。	
\\	なりに/なりの
\\	わたしには、わたしなりに小さな独創があるのです。	
\\	なりに/なりの
\\	貧しいなら貧しいなりの生活をしなければならない。	
\\	なりに/なりの
\\	彼は誰に対しても、明らかにちっぽけで浅はかな相手でも、その人なりの価値を見出していた。	
\\	なりに/なりの
\\	忙しく誰かを追いかけるなり、何かをするなりしたいもんですな。	
\\	なり~なり
\\	電話をかけるなり、そのころ訪ねてくるなりしたらどうですか?	
\\	なり~なり
\\	彼はその陳述を確かめるなり、あるいは反証をあげるなりするまでは、何ともしようがなかった。	
\\	なり~なり
\\	メールなり電話なりで、彼女に連絡したまえ。	
\\	なり~なり
\\	分からない単語があったら、辞書を引くなり誰かに聞くなりして調べておきなさい。	
\\	なり~なり
\\	勉強ばかりしていないで、たまには外に遊びに行くなり買い物をするなりして、気分転換を図ったほうがいい。	
\\	なり~なり
\\	彼はニュースを聞くなり、家を飛び出していった。	
\\	なり
\\	課長は部屋に入ってくるなり、大声でどなった。	
\\	なり
\\	彼の顔を見るなり、わたしは半睡の状態から、いっきに目をさましてしまった。	
\\	なり
\\	太郎君は学校から帰るなり、インターネットゲームをし始めた。	
\\	なり
\\	彼女は僕に会うなり泣き出した。	
\\	なり
\\	写真をひと目見るなり、無言でその場を立ち去ったそうだ。	
\\	なり
\\	子供たちは布団に入るなりぐうぐう寝てしまった。	
\\	なり
\\	エレベーターのドアが開くなり彼は飛び乗った。	
\\	なり
\\	わたしを見るなり彼女はたずねた。「あの、娘は?」	
\\	なり
\\	姉はそのままいちども意識を回復することなしに、しだいに衰弱が加わってとうとう死んでしまいました。	
\\	ことなしに
\\	努力することなしに、成功を勝ち取ることは難しい。	
\\	ことなしに
\\	あなたにこんな恐ろしい経験をさせることなしに、救い得ていたろうと思いますよ。	
\\	ことなしに
\\	親友は細かい事情を聞くことなしにすぐにお金を貸してくれた。	
\\	ことなしに
\\	彼女は単刀直入に思うことを話しはじめた。何らためらうことなしに、言葉はすらすらと出てきた。	
\\	ことなしに
\\	担当教師の許可を得ることなしには、履修科目の変更はできません。	
\\	ことなしに
\\	タバコを吸うことなしには一日もいられない。	
\\	ことなしに
\\	人事部の許可を得ることなしに、勝手に休みを取ることはできない。	
\\	ことなしに
\\	大変な仕事だが、夜を徹して行えば、できないものでもない。	
\\	ないものでもない
\\	そんなに頼むのなら、その仕事を代わってやらないものでもない。	
\\	ないものでもない
\\	一生懸命働けばマイホームが手に入らないものでもない。	
\\	ないものでもない
\\	条件によっては、この仕事を引き受けないものでもない。	
\\	ないものでもない
\\	「わたしが悪かった」と素直に謝れば、許してやらないものでもない。	
\\	ないものでもない
\\	少し遠いが、歩いて行けないものでもない。	
\\	ないものでもない
\\	難しいけれど、何とか工夫すれば、できないものでもないだろう。	
\\	ないものでもない
\\	ちゃんと返すと約束するなら貸さないものでもない。	
\\	ないものでもない
\\	輪郭だけであっても、ヴェネツィアの眺望は一見に値した。	
\\	に値する
\\	この二人は罰則以上のものに値すると思います。	
\\	に値する
\\	信仰の篤い者は最大の敬意に値します。	
\\	に値する
\\	自分はもう一度パンチを食らうに値すると思う。	
\\	に値する
\\	彼は信頼に値しない者を信頼している。	
\\	に値する
\\	な
\\	な/である 
\\	のにひきかえ	
\\	のにひきかえ	
\\	(である)のにひきかえ	にひきかえ
\\	わたしがきれい好きなのにひきかえ、妹はいつも部屋を散らかしている。	
\\	にひきかえ
\\	停車場は奇妙に静かだ。それにひきかえ、車内はいやにざわついている。	
\\	にひきかえ
\\	先日までのひどい天気にひきかえ、当日は晴れている。	
\\	にひきかえ
\\	同じ兄弟でありながら、彼らほど性格が違うのも珍しい。いつも冷静でしっかり者の兄にひきかえ、弟の方はなんと落ち着きのないことか。	
\\	にひきかえ
\\	彼の給料は一ヶ月50万円だ。それにひきかえ僕の給料はなんと安いことか。	
\\	にひきかえ
\\	息子が社交的なタイプなのにひきかえ、娘は人前に出るのを嫌うタイプだ。	
\\	にひきかえ
\\	去年にひきかえ、今年は暖かい。	
\\	にひきかえ
\\	読書好きの姉にひきかえ、妹は本を読もうとしない。	
\\	にひきかえ
\\	美智子に至っては、うきうきしているようだった。	
\\	に至っては
\\	パーカーに至っては、彼女が殺したのでない事が決定的だ。	
\\	に至っては
\\	青山さんにいたってはなんら督励の必要もなく、だまっていようなどという気はさらさらなかったのでした。	
\\	に至っては
\\	ジェイミー・アダムスの事件にいたっては、弁護士もそれを論議すべきものをもたなかった。	
\\	に至っては
\\	全国各地にその地方ならではの名産がある。	
\\	ならでは
\\	あそこでは一流ホテルならではの豪華な雰囲気が味わえる。	
\\	ならでは
\\	世界各国にはその国ならではの伝統文化と風習がある。	
\\	ならでは
\\	あなたならではの発想で、この企画を考えてください。	
\\	ならでは
\\	いま思いだしたけど、あなたはぼくの住所を知ってなけりゃいけない。何かおこらないとも限らないからさ。	
\\	ないとも限らない
\\	彼らは最初の混乱のうちに同志打ちをしないとも限らない。	
\\	ないとも限らない
\\	しっかり鍵をかけないと、泥棒に入られないとも限らない。	
\\	ないとも限らない
\\	吸血鬼といったって、なにも死人と決まったわけじゃあるまい?生きながらそういう習性をもったものがいないとも限らない。	
\\	ないとも限らない
\\	そんな提案をしたら、社長の怒りを買わないとも限らない。	
\\	ないとも限らない
\\	ルーシーはこの不安定な状態にあってケーティよりもっと痛ましい苦悩にさいなまれたことはいうまでもない。	
\\	にあって
\\	彼はこの素気ない拒絶にあって少なからずめんくらった。	
\\	にあって
\\	彼女は、わたしの叱責にあっても、狂気じみた絶望的な眼でわたしを見つめるばかりで、一語も答えません。	
\\	にあって
\\	彼の鋭い視線にあって、彼女は少し顔を赤らめた。	
\\	にあって
\\	どのような困難な状況にあっても、あきらめてはいけない。	
\\	にあって
\\	その三ヶ月のあいだに、どんなことが起ったか? それは想像にかたくない。	
\\	にかたくない
\\	愛する人を失った彼女の悲しみは察するにかたくない。	
\\	にかたくない
\\	オスカー受賞者の喜びは、察するにかたくない。	
\\	にかたくない
\\	彼の実力を見れば、いかに努力したかは想像にかたくない。天才は1%のひらめきと99%の努力によって作られるというではないか。	
\\	にかたくない
\\	そのとき、彼がどんなにつらかったかは察するにかたくない。	
\\	にかたくない
\\	どうしてもおまえに、家族にまつわる真実を伝えなくては…	
\\	にまつわる
\\	この辺りには、この池にまつわる伝説が伝わっている。	
\\	にまつわる
\\	千年にわたってこの秘密にまつわるさまざまな伝説が伝えられてきた。	
\\	にまつわる
\\	アダムとイブにまつわる話は、たぶんだれもが聞いたことがあるはずだ。	
\\	にまつわる
\\	彼が何の予告もなしに立ち上って、話をはじめた。	
\\	なしに
\\	なにか確たる目的なしにああまで懸命に勉強したり、正確な知識の獲得に汲々するものはあるまい。	
\\	なしに
\\	ノックなしにわたしの部屋に入らないでください。	
\\	なしに
\\	あんなにひっきりなしに地図を見ていたのに、降りる駅を間違えた。	
\\	なしに
\\	教授の助言なしには、この研究の成功はなかった。	
\\	なしに
\\	彼は否応なしに店に入っていった。	
\\	なしに
\\	先生方のご指導なしには、論文を書き上げられなかっただろう。	
\\	なしに
\\	お茶なしには一日が始まらない。	
\\	なしに
\\	国会は、先例に則って運営されています。	
\\	に則って
\\	古くからの伝統に則って祭りが行われている。	
\\	に則って
\\	ルールに則って行動しましょう。	
\\	に則って
\\	スポーツマンシップに則って試合をします。	
\\	に則って
\\	あんたの言うことは、みんな本当でしょう。しかし、それにしたところで、わたしたちはそれをきりぬけなければならないんでしょう。	
\\	にしたって
\\	僕にしたってひとりよりか相棒のあったほうがいいからねえ。	
\\	にしたって
\\	全員が参加するにしたって、せいぜい20人だ。	
\\	にしたって
\\	しかし、それでも最近の不景気は株にひびかずにはいないでしょう。どんなに安全な株にしたところで。	
\\	にしたって
\\	わたしは動物がいじめられているのは見るにたえない。	
\\	にたえる/にたえない
\\	罪をあばかれたので、良心の呵責にたえられなかったのです。	
\\	にたえる/にたえない
\\	彼の歌はあまりにも下手で、聞くにたえません。	
\\	にたえる/にたえない
\\	このシャツは毎日の洗濯にたえる素材で作られています。	
\\	にたえる/にたえない
\\	その映画は面白くないから、見るにたえない。	
\\	にたえる/にたえない
\\	この展示は鑑賞にたえる作品が少ない。	
\\	にたえる/にたえない
\\	学校で最も人気があり、賞賛の的だった桜子は、一夜にして突然、一番の嫌われ者になっていた。	
\\	にして
\\	天才の彼にして解けない問題なのだから、凡才のわたしに解けるわけがない。	
\\	にして
\\	わずか七日間で、いまや一瞬にしてその彼女が、いとまを告げて帰ってしまったその場かぎりの客のように、彼の心から消えてしまったのだ。	
\\	にして
\\	日本語能力試験一級は非常に難しく、僕も5回目にしてようやく合格できた。	
\\	にして
\\	彼女は怖ろしい酒乱で結婚四年にして酒のために死んだ。	
\\	にして
\\	彼女は45歳にして母親になった。	
\\	にして
\\	今日に至るまで、彼の死因は判明しておらぬ。	
\\	に至るまで
\\	マイケル・ジャクソンは現代文化の象徴で、ポスターからマウスパッド、
\\	シャッに至るまで、世界じゅうで使われている。	
\\	に至るまで
\\	帰宅途中、夕立にあい、頭から足先に至るまで全身びしょ濡れになった。	
\\	に至るまで
\\	優勝にいたるまでの過程を話してくれませんか。	
\\	に至るまで
\\	あなたたちの結婚に至るまでのロマンスを教えてください。	
\\	に至るまで
\\	わたしの学校は服装に厳しい。制服の着方はもちろん、ヘアスタイルやスカートの長さに至るまで注意される。	
\\	に至るまで
\\	こんどの旅行のスケジュールは綿密だ。起床時間から飛行機内の食事開始時間に至るまで書いてある。	
\\	に至るまで
\\	わたしに言わせれば、あの娘は、ギャングの一味に違いありませんよ。	
\\	に言わせれば
\\	わたしに言わせれば、戦うべき相手は早く知れば知るほどよい。	
\\	に言わせれば
\\	僕に言わせれば、配下の者がいなければ、彼は危険ではないのでね。	
\\	に言わせれば
\\	俺に言わせればあいつはずっと人物が落ちる。	
\\	に言わせれば
\\	彼女は天才と言われているが、僕に言わせれば、単なる努力家だ。	
\\	に言わせれば
\\	わたしに言わせれば、田中さんの振舞いは非常に愚かでした。	
\\	に言わせれば
\\	僕がちょっと君を驚かすに足る材料を入手し得た。	
\\	に足る/に足りる
\\	山本さんは信頼するに足る人物です。	
\\	に足る/に足りる
\\	彼女の同情と好奇心とは、頼むに足るものがある。	
\\	に足る/に足りる
\\	彼が誇るに足る奴だ。	
\\	に足る/に足りる
\\	インターネットで得た情報は、信頼に足るものではない。	
\\	に足る/に足りる
\\	この子は、規則を犯していたと信じるに足る証拠をわたしはつかんでいます。	
\\	に足る/に足りる
\\	疑うに足る大きな理由があるよ。	
\\	に足る/に足りる
\\	会社の現状に即して経営計画を考えましょう。	
\\	に即して
\\	新聞には、事実に即して、正確な情報を提供してほしい。	
\\	に即して
\\	実状に即して計画を練らなければならない。	
\\	に即して
\\	違反者は、法律に即して処分されます。	
\\	に即して
\\	このような規則は、実態に即して柔軟に適用すべきだ。	
\\	に即して
\\	現状に即して物事を考えよう。	
\\	に即して
\\	彼女は前にもまして元気なくうなずいた。	
\\	にもまして
\\	けれども彼には、これにもまして悪い行いがあるのです。	
\\	にもまして
\\	彼女は以前にもまして美しい。	
\\	にもまして
\\	今回のイベントは、前回にもまして好評だった。	
\\	にもまして
\\	大学生の就職は、今年は去年にもまして、さらに厳しい状況になることが予想される。	
\\	にもまして
\\	あいつはまっ青な顔をして、こそこそ逃げていったが、そのときギロリと見かえした両の目は、どんな嚇し文句にもまして怖ろしく感じられた。	
\\	にもまして
\\	このノートパソコンは、価格もさることながら、デザインがよいので人気がある。	
\\	もさることながら
\\	歌のうまさもさることながら、彼女はピアノの名手なんです。	
\\	もさることながら
\\	彼は外見もさることながら性格もとてもいい。	
\\	もさることながら
\\	就職先を決めるときは仕事の内容もさることながら、条件や待遇も無視できない。	
\\	もさることながら
\\	さしあたっての気まりわるさもさることながら、彼女のまえにはいろいろと気の重くなることがひかえておりました。	
\\	もさることながら
\\	世界各国の小規模メーカーが、他社に先駆けて「空飛ぶ車」を発売しようとしのぎを削っている。	
\\	に先駆けて
\\	その映画は、世界に先駆けて、イギリスでの上映が決まった。	
\\	に先駆けて
\\	その時の言葉を多少なりとも覚えていらっしゃいますか。	
\\	なりとも
\\	ほかの男と結婚したいというのでありましたら、そうする自由を与えますが、わたしから一銭なりとも受ける理由はないのでありますから、そういう要求はしないはずであります。	
\\	なりとも
\\	お嬢さんに人目なりとも会わせていただけませんでしょうか。	
\\	なりとも
\\	それはひとつの苦痛にとどまらず、ひとつの苦役でもあった。	
\\	にとどまらず
\\	彼女の侮辱はあなたひとりだけにとどまらず、実に法を犯すものです。	
\\	にとどまらず
\\	ビッグバンは韓国だけにとどまらず、海外でも大人気がある。	
\\	にとどまらず
\\	彼はその店で、文句を言うにとどまらず、テーブルをひっくり返してしまった。	
\\	にとどまらず
\\	本研究室での研究は、単なる理論にとどまらず、その応用も重視している。	
\\	にとどまらず
\\	そんなことはべつにたいして驚くにはあたらないことだ。	
\\	にあたらない
\\	彼に振られたからといって、泣くにはあたらない。	
\\	にあたらない
\\	その程度のことで怒るにはあたらない。	
\\	にあたらない
\\	あのホテルのサービスは賞賛にはあたらない。ホテルならあのくらいは当然だ。	
\\	にあたらない
\\	彼女の発言は責めるにはあたらないと思う。	
\\	にあたらない
\\	彼なりにできるだけの努力をしたのだから、いい結果を出せなかったとしても、非難するにはあたらない。	
\\	にあたらない
\\	大統領にお目にかかれて、実に光栄の至りです。	
\\	の至り
\\	言葉遣いの間違いを指摘され、赤面の至りだった。	
\\	の至り
\\	こんな立派な賞をいただきまして、光栄の至りです。	
\\	の至り
\\	おほめの言葉をいただき、恐縮の至りでございます。	
\\	の至り
\\	あなたのような大スターとお会いできるなんて感激の至りです。	
\\	の至り
\\	彼らの言葉は一つとして信じるに足りない。	
\\	に足りない
\\	彼は全くあなたの信頼するにたりない男です。	
\\	に足りない
\\	彼の証言は、兄の証言のまったくのくり返しで、ほんとに取るに足りないものだった。	
\\	に足りない
\\	小田さんは頼むに足りない人だ。	
\\	に足りない
\\	あなたの実力をもってすれば、大学入試など恐れるに足りない。	
\\	に足りない
\\	人間は矛盾したものだという考え、長所とみえるもの、分別ありげにみえるものも信頼するに足りないものだという考えは日ごとにかたくなります。	
\\	に足りない
\\	宝石を盗んだ犯人は、法に照らして処罰できます。	
\\	に照らして
\\	自分の経験に照らして後輩にアドバイスした。	
\\	に照らして
\\	この事件は目下警察の手に移っているが、従来の前例に照らしてもその努力が報いられる見込みは少ない。	
\\	に照らして
\\	校長との怒鳴り合い試合のニュースは、この学校の基準に照らしても例外的な速さで伝わった。	
\\	に照らして
\\	なるべく危険は避けたい。	
\\	なるべく
\\	そこはみんながなるべく近寄らない場所である。	
\\	なるべく
\\	なるべく早くお返事をください。	
\\	なるべく
\\	なるべく外へは出ないようにしている。	
\\	なるべく
\\	なるべく早く帰りたい。	
\\	なるべく
\\	「お邪魔して本当にすみません」彼はなるべく小さな声で言った。	
\\	なるべく
\\	近況をなるべく早く知らせてね。	
\\	なるべく
\\	すべてはその偶然の機会にかかっている。	
\\	にかかっている
\\	あなたたちが成功するかどうかは、与えられたチャンスをどう使うかにかかっている。	
\\	にかかっている
\\	問題は、五時半から六時までの間に、誰か、店の近くで、あの男を見かけた人間がいるかいないかにかかっているわけです。	
\\	にかかっている
\\	すべては彼の一存にかかっていた。	
\\	にかかっている
\\	今年卒業できるかどうかは、これからの頑張りにかかっている。	
\\	にかかっている
\\	運命はただ一筋の糸にかかっていた。	
\\	にかかっている
\\	現在の状況を踏まえて、今後の計画を考え直す必要がある。	
\\	を踏まえて
\\	消費者の意見を踏まえて商品開発をしましょう。	
\\	を踏まえて
\\	先行研究を踏まえて、卒論のテーマを決める。	
\\	を踏まえて
\\	アンケート結果を踏まえて、業務を改善したいと思います。	
\\	を踏まえて
\\	被害の実態を踏まえて、対策を講じなければならない。	
\\	を踏まえて
\\	確実だとは申しかねますが、そこはお含みおき下さいまし、この情報はたくさんの手を経て来ておりますので。	
\\	を経て
\\	わたしたちは5年の交際期間を経て結婚した。	
\\	を経て
\\	ある一つの結果だけを与えられて、はたしてどんな段階をへてそういう結果にたち至ったかということを、論理的に推理できる人は、ほとんどいない。	
\\	を経て
\\	難民はオーストリアを経て、ドイツに列車で続々と到着している。	
\\	を経て
\\	先生がいないのをいいことに学校をサボった。	
\\	をいいことに
\\	親方はわしが年とってるのをいいことにしたんだ。	
\\	をいいことに
\\	廊下の突き当たりのドアが開いているのをいいことに、三階に抜け出した。	
\\	をいいことに
\\	彼女はわたしたちが黙っているのをいいことに、恐ろしい警告をひっきりなしに流し続けた。	
\\	をいいことに
\\	何によらず狡猾なものはいやしむべきものです。	
\\	によらず
\\	見かけによらずあの人はたくらみ深く険悪な人物ですね。	
\\	によらず
\\	何によらず物事を知るのは、わたしの商売です。	
\\	によらず
\\	職種によらず、賃金によらず、仕事があれば何でもやります。	
\\	によらず
\\	今日を限りにタバコをやめることにした。	
\\	を限りに
\\	観衆が声を限りに叫び拍手喝采している。	
\\	を限りに
\\	このバンドは今回のコンサートを限りに解散するこのになった。	
\\	を限りに
\\	昨日の講義を限りに、鈴木教授は大学を定年退官した。	
\\	を限りに
\\	実は、今日を限りに会社を辞めることにしたんです。	
\\	を限りに
\\	在庫限りで終了します。	
\\	を限りに
\\	新築の家を買った。今月を限りにこのマンションともお別れだ。	
\\	を限りに
\\	本人の前であんなことを言うなんて、非常識の極みだ。	
\\	の極み
\\	残業続きで、疲労の極みに達している。	
\\	の極み
\\	世界的に有名な俳優と握手できたなんて、感激の極みだ。	
\\	の極み
\\	おほめいただいて光栄の極みです。	
\\	の極み
\\	彼女は胸もはり裂けんばかりにわっと泣きくずれた。	
\\	んばかりに
\\	そうだろうか、と言わんばかりに彼女は目を見開いた。	
\\	んばかりに
\\	彼は今にも気も狂わんばかりの状態だった。	
\\	んばかりに
\\	頭はズキズキと割れんばかりだ。	
\\	んばかりに
\\	彼は、言わなくともわかるだろうと言わんばかりだった。	
\\	んばかりに
\\	彼が幸せのあまり泣かんばかりの様子で彼女に近づいてきた。	
\\	んばかりに
\\	神のつくりたまえるあらゆるものは、愛につかえんがためにできているのではないか。	
\\	んがため
\\	ライオンは生きんがために、ほかの動物を食べる。	
\\	んがため
\\	試験に合格せんがため、この一年間一生懸命勉強していました。	
\\	んがため
\\	真実を明らかにせんがための調査が行われました。	
\\	んがため
\\	夢をかなえんがため、日々努力している。	
\\	んがため
\\	僕は勝たんがためには、どんなひどい手段でもする。	
\\	んがため
\\	食べんがために生きるのではなく生きんがために食べるのだ。	
\\	んがため
\\	あの人は成功せんがために、ずいぶんひどいことをやってきた。	
\\	んがため
\\	かすり傷をしただけですから、心配におよびません。	
\\	には及ばない
\\	柱についている修道女にはお前の来たことを知らせるにはおよばない。	
\\	には及ばない
\\	時間があるから、急ぐには及ばない。	
\\	には及ばない
\\	彼はいつもよりいっそう怒っているように見えたが、わたしの怒りには及ばない。	
\\	には及ばない
\\	どんなに頑張っても、彼の成績には及ばない。	
\\	には及ばない
\\	面接の結果は電話で知らせますので、こちらに来るには及びません。	
\\	には及ばない
\\	ご案内には及びません。	
\\	には及ばない
\\	2月20日をもって願書受け付けを締め切ります。遅れないように出してください。	
\\	をもって
\\	当店は10月末日をもって閉店させていただきます。長い間のご利用ありがとうございました。	
\\	をもって
\\	この大会は本日をもちまして終了いたします。	
\\	をもって
\\	9月10日をもちまして、退職いたしました。	
\\	をもって
\\	彼は危険をものともせずに、行方不明の子の捜索を続けた。	
\\	をものともせず
\\	彼女は親の反対をものともせずに、彼と結婚した。	
\\	をものともせず
\\	エジソンは周囲の非難をものともせずに、実験を続けた。	
\\	をものともせず
\\	彼は度重なる困難をものともせず前に進んでいった。	
\\	をものともせず
\\	周囲の反対をものともせず、うちの息子はいつも自分の意思を通してきた。	
\\	をものともせず
\\	悪天候をものともせず、兄とサッカー観戦に出かけた。	
\\	をものともせず
\\	入院したのを機に、お酒をやめた。	
\\	を機に
\\	転勤を機に、何か新しい技術へチャレンジをしたいと考えています。	
\\	を機に
\\	結婚を機に新しい人生を迎える。	
\\	を機に
\\	当社は十周年を機に新商品を発表しました。	
\\	を機に
\\	ああ、恐ろしかったのなんのって。あの恐ろしさは、誰にも判りっこないわ。	
\\	のなんのって
\\	なにかいうことがあるかって? あるのなんのって、うんとありまさ。	
\\	のなんのって
\\	きたないのなんのって、やっと手だけは洗わせましたが、顔ときたらまるで鋳掛屋みたいにまっ黒ですよ。	
\\	のなんのって
\\	怖いのなんのって。	
\\	のなんのって
\\	ところが見ての通り、連中は機嫌が悪いのなんのって。	
\\	のなんのって
\\	彼をおいて、この仕事を任せられる人間はいないだろう。	
\\	をおいて
\\	まさか僕をおいてきぼりにするんじゃありますまいね。	
\\	をおいて
\\	わが社の危機を救える人は、あの人をおいて、ほかにはいないだろう。	
\\	をおいて
\\	わたしとしてはあなたをおいてほかに、助けていただける人はありません。	
\\	をおいて
\\	みんなから信頼されている彼をおいてほかに適当な人がいないだろう。	
\\	をおいて
\\	あの部屋でどんなことが行なわれたか?それを明確に語りうるのは、この男をおいてほかにはないのだ。	
\\	をおいて
\\	次の首相にふさわしい人物は、彼女をおいて、ほかにはいない。	
\\	をおいて
\\	この景色を一緒に見る人は、君をおいてほかにいないと思っていたんだ。	
\\	をおいて
\\	探偵ごっこはもうやめにしろ。さもないとママに手紙を書くぞ!	
\\	さもないと
\\	怒鳴り合いは、今後、更衣室だけに止めておきなさい。さもないとチームのキャプテンの座を失うことになります!	
\\	さもないと
\\	止まりなさい! さもないとこれを壊すわよ!	
\\	さもないと
\\	別の計画を立てねばならなくなりました。さもないと、殺人犯がまだ車内にひそんでいることが皆に知られてしまうからです。	
\\	さもないと
\\	指輪をよこせ、さもないと、こいつと同じ目に遭うぞ。	
\\	さもないと
\\	冗談にもほどがある。	
\\	にもほどがある
\\	情けないにもほどがあるぞ。	
\\	にもほどがある
\\	途中で仕事を投げ出すなんて、無責任にもほどがある。	
\\	にもほどがある
\\	ずうずうしいにもほどがある。	
\\	にもほどがある
\\	わたし、間違って、あの方の部屋のドアを開けてしまいました。本当に、恥ずかしい思いをしました。	
\\	思いをする
\\	辛い思いをさせてすまなかったね。	
\\	思いをする
\\	こんなにぞっとするような思いをしたことは、いままでありませんでした。	
\\	思いをする
\\	彼女はほんとに、みじめな思いをしている様子ね。	
\\	思いをする
\\	なるほどあなたは辛い思いをした、しかし、今あなたがやっていることは、その苦しみを長びかせるだけですよ。	
\\	思いをする
\\	彼はわたしをこんな所へつれて来て、退屈な思いをさせた。	
\\	思いをする
\\	痛みなら十分に経験ずみのロジャースは、わざわざこれ以上痛い思いをしたいとは思わなかった。	
\\	思いをする
\\	わたしたちはできるだけのことをしたつもりなのです。まごころをもってお仕えしました。	
\\	をもって
\\	敬意をもって接してくれるだろうな。	
\\	をもって
\\	彼は人一倍の努力をもってメダルを手にすることができました。	
\\	をもって
\\	最新の医療技術をもってすれば、人はさらに寿命を延ばせるだろう。	
\\	をもって
\\	やつは栄誉をもって再び迎え入れられる。	
\\	をもって
\\	彼らは恐懼の念をもってその死骸のまわりに集まった。	
\\	をもって
\\	彼女は名状しがたいある怨みの情をもって彼を見まもった。	
\\	をもって
\\	ばからしいと言うんです。およそ無意味です。	
\\	およそ
\\	ここに残っている足跡の主を全部洗いだすなんて、およそ不可能なことですよ。	
\\	およそ
\\	わたしは彼がどこに隠れているか、およその見当がついているのよ。	
\\	およそ
\\	ビンをさがすのに少し手間どりましたのでおよそ五分間もかかりました。	
\\	およそ
\\	およそこの世に宗教ほど推理を必要とするものはありません。	
\\	およそ
\\	駅の外はおよそ百台のタクシーが待っていた。	
\\	およそ
\\	父の退職祝いを兼ねて家族旅行に行った。	
\\	を兼ねて
\\	日本語の勉強を兼ねて漫画を読んでいる。	
\\	を兼ねて
\\	わたしは趣味と実益を兼ねてイラスト描いています。	
\\	を兼ねて
\\	観光を兼ねてイギリスへ研修に行った。	
\\	を兼ねて
\\	彼女の態度は僕にとっては嫌なものではあったがそれなりに理解できた。	
\\	それなりに
\\	彼のようなのが一人いると、それなりに価値がある。	
\\	それなりに
\\	倒れた男のほうはそれなりに動く気配もなく、いつまでも平野にそのままだった。	
\\	それなりに
\\	事情については何の説明もせずに、とにかくその場はそれなりになりました。	
\\	それなりに
\\	もしわしの探検の結果が何ら彼らの報告と変らなんだとしても、それはそれなりに価値があるのじゃ。	
\\	それなりに
\\	道路拡張の工事のために、この周辺の人々は引越しを余儀なくされた。	
\\	を余儀なくされる
\\	あのサッカー選手はまだ若いが、度重なるけがにより引退を余儀なくされた。	
\\	を余儀なくされる
\\	不正な取引が明らかになり、その取引に関わった会社役員は辞職を余儀なくされた。	
\\	を余儀なくされる
\\	災害のため、多くの住民が避難所での生活を余儀なくされている。	
\\	を余儀なくされる
\\	彼は病気で入院を余儀なくされた。	
\\	を余儀なくされる
\\	台風によって交通機関が止まってしまい、旅行の中止を余儀なくされた。	
\\	を余儀なくされる
\\	すぐ治るけがだと言われたのに無理をしたため、手術を余儀なくされた。	
\\	を余儀なくされる
\\	その名前をきいて彼はある不快な印象を禁じえなかった。	
\\	を禁じえない
\\	大災害で家族を失った人の話を聞いて、涙を禁じえなかった。	
\\	を禁じえない
\\	彼はおそれてはいなかった、しかしまさにおこらんとすることを考えると、戦慄を禁じえなかった。	
\\	を禁じえない
\\	大手企業が相次いで倒産すると聞いて、驚きを禁じえない。	
\\	を禁じえない
\\	わたしたちは、彼の突然の辞職に、戸惑いを禁じえない。	
\\	を禁じえない
\\	政治家の汚職を見るにつけ怒りを禁じえない。	
\\	を禁じえない
\\	地震の被害者の方々には、同情を禁じえない。	
\\	を禁じえない
\\	言ったそばから彼女は後悔した。	
\\	そばから
\\	片付けるそばから子供がおもちゃを散らかす。	
\\	そばから
\\	月末になると、払うそばからまた別の請求書が来る。	
\\	そばから
\\	もう遅刻しないと言ったそばからまた遅れるなんて、彼は何を考えているのだろう。	
\\	そばから
\\	祖父は、パソコンの使い方を教えても、教えるそばから忘れてしまう。	
\\	そばから
\\	こちらを向いた拍子にあらわれたその顔の表情は、彼を驚かせた。	
\\	た拍子に
\\	つまずいた拍子に、石にぶつかってしまったんで、ほんとに偶然だったんです。	
\\	た拍子に
\\	くしゃみをした拍子にメガネが落ちた。	
\\	た拍子に
\\	あの人がけとばした拍子に、長椅子の下に入ってしまったんです。	
\\	た拍子に
\\	ただ 
\\	い
\\	のみだ	ただ 
\\	のみだ	
\\	ただ 
\\	のみだ	ただ~のみだ
\\	ただ先へ進んでみるの一途あるのみです。	
\\	ただ~のみだ
\\	事故はあまりにも突然で、わたしは何もできず、ただ呆然とするのみだった。	
\\	ただ~のみだ
\\	いまはただ中央の拠点が残っているのみだった。	
\\	ただ~のみだ
\\	わたしのように哀れな、病弱な本の虫は、こんな打撃をうけるとどうしたらよいやら、ただ茫然とするのみです。	
\\	ただ~のみだ
\\	こうなったら、ただ一生懸命やるのみだ。	
\\	ただ~のみだ
\\	こんなに晩くまでお働きくださいまして、さぞお疲れでございましょう。	
\\	さぞ
\\	さぞおもしろいことがあったでしょうね。	
\\	さぞ
\\	彼女は男に生まれたらよかったのに。さぞかし偉大な航海者になったでもあろう。	
\\	さぞ
\\	みんなさぞ骨が折れることだろう。	
\\	さぞ
\\	さぞびっくりなすったでしょうが、その説明はごく簡単です。	
\\	さぞ
\\	ここに閉じこめられる間中、さぞ退屈に苦しめられるだろう。	
\\	さぞ
\\	わたしのことをさぞ無情な冷たい女だとお思いでございましょうねえ。	
\\	さぞ
\\	竹之内博士が東京に大きな診療所を持つ事を知ってからというもの、彼女は博士に対して愛想よくしたいと思っていた。	
\\	てからというもの
\\	彼女の料理の腕は、ネックレスをあげてからというもの劇的に進歩した。	
\\	てからというもの
\\	わたしはあの娘を雇ってからというもの、一度でもこの家の近所を独りで歩かしたことはありません。	
\\	てからというもの
\\	あなたが負傷してからというもの、彼女はいつも泣いてばかりいる。	
\\	てからというもの
\\	息子が大学に入り家を出ていってからというもの、寂しくなった。	
\\	てからというもの
\\	中田は大阪へ着いてからというもの、何者かによって厳重に尾行されている。	
\\	てからというもの
\\	メールを使うようになってからというもの、ほとんど手紙を書かなくなった。	
\\	てからというもの
\\	車掌が、自分で乗客の荷物の世話をしてくれるとは!聞いたためしがない!	
\\	ためしがない
\\	あなたは奇妙な男だねえ、何にも満足したためしがないんだからなあ。	
\\	ためしがない
\\	やつは時間どおりに来たためしがないんだよ。	
\\	ためしがない
\\	警察はいままで一度だって彼を見つけたためしがないじゃないか。	
\\	ためしがない
\\	宝くじをたくさん買ってきたが、いままで当たったためしがない。	
\\	ためしがない
\\	あれ、まだ家にいたんですか。てっきりもう出かけたのだと思っていました。	
\\	てっきり
\\	てっきり日本だけのものかと思っていたが、そうではないらしい。	
\\	てっきり
\\	これはてっきり乞食にちがいない。	
\\	てっきり
\\	そのドレスはてっきり黒と青だと思ったのに!	
\\	てっきり
\\	わたしが良人の意志に反してまで、心配をわかちたいと願っています。	
\\	てまで
\\	ヒステリーになってまで感情を表に現わすことは、嫌ですわ。	
\\	てまで
\\	彼は借金してまで遊びに行ったと聞いた。	
\\	てまで
\\	僕は人種の絆をきってまで彼女と結婚した。	
\\	てまで
\\	わたしたちとしてはべつだん骨折ってまでこんなものを隠す気なんかない。	
\\	てまで
\\	自分を犠牲にしてまで、彼の心を傷つけたくないと願っていますわたしは、あの人を愛していないのでしょうか?	
\\	てまで
\\	丸子は、親の心配をよそに、遊んでばかりいる。	
\\	をよそに
\\	住民たちの抗議行動をよそに、公園の建設計画が進められている。	
\\	をよそに
\\	彼女は親の心配をよそに、一人でイラクに行った。	
\\	をよそに
\\	親の期待をよそに、子供たちは毎日ゲームに熱中している。	
\\	をよそに
\\	あの子は教師の忠告をよそに、相変わらず悪い仲間と付き合っている。	
\\	をよそに
\\	彼女は自分の失敗をよそに、他人の批判ばかりしている。	
\\	をよそに
\\	田中さんは周囲の心配をよそにヨットで長い航海に出た。	
\\	をよそに
\\	これ以上言うのはよそう。だが、こんども必ず切りぬけてみせるよ!	
\\	てみせる
\\	僕は小説を書いて賞ををとってみせる。	
\\	てみせる
\\	彼が知っていさえすればおれがきっと吐かしてみせる。	
\\	てみせる
\\	今度きっと勝ってみせる。	
\\	てみせる
\\	ガミガミ言ってもどうにもならないよ。もうすんだことだ。	
\\	てもどうにもならない
\\	過ぎたことを後悔してもどうにもならない。	
\\	てもどうにもならない
\\	こんなこと相談してもどうにもならないと思う。	
\\	てもどうにもならない
\\	今更謝ってもどうにもならないよ。	
\\	てもどうにもならない
\\	いまさら、後悔してもはじまらない。	
\\	ても始まらない
\\	感傷的になってもはじまらないわ。	
\\	ても始まらない
\\	いまさら担当者を責めてもはじまらない。	
\\	ても始まらない
\\	くよくよしてもはじまらないと思う。	
\\	ても始まらない
\\	彼女の顔を見た瞬間にわかったんです。	
\\	た瞬間に
\\	いまこの瞬間に、わたしがなにを望んでいるか、おわかりですか?	
\\	た瞬間に
\\	わたしは書斎へ入った瞬間に、あの刀に気がつきました。	
\\	た瞬間に
\\	事実彼のはいってきた瞬間、僕は極度の危険をかんじた。	
\\	た瞬間に
\\	あなたを一目見た瞬間、恋に落ちました。	
\\	た瞬間に
\\	あの死体が車内から落ちたものでなく、屋根の上から落ちたのだと知った瞬間に、わたしにはわかった。	
\\	た瞬間に
\\	彼が二年前に日本へきた折に会いました。	
\\	折に
\\	御多忙の折に、わたしのために色々とお骨折り下さって、まことにありがとうございます。	
\\	折に
\\	婚約の折に夫に買ってもらった指輪をなくしてしまった。	
\\	折に
\\	来日の折には、わたしがご案内します。	
\\	折に
\\	ニヤニヤするのはやめろってば!	
\\	ってば
\\	はやく持ってきなさいってば。ねえ、聞こえないの?	
\\	ってば
\\	ほんとだってば。彼を見たんだ!	
\\	ってば
\\	分かってるってば!	
\\	ってば
\\	彼を殺すな!殺すなってば!	
\\	ってば
\\	わたしにおっしゃるほどのことでしたら、どんなことでもこの男に聞かせてさしつかえございません。	
\\	ても差し支えない
\\	少し待っても差し支えない、と彼は思った。	
\\	ても差し支えない
\\	手術後の経過が順調だったら、来週は散歩に出てもさしつかえない。	
\\	ても差し支えない
\\	多少なりとも、動機があることがわかれば、スミス氏を殺した犯人と見て差し支えないでしょう。	
\\	ても差し支えない
\\	お返事は明日でも差し支えありません。	
\\	ても差し支えない
\\	一分たりとも貴重だ! 君には分らないのか?	
\\	たりとも
\\	募金で集めたお金は1円たりとも無駄にできない。	
\\	たりとも
\\	生徒が一人たりとも寮の外に残っていないよう見まわってください。	
\\	たりとも
\\	どんな相手でも、試合が終わるまでは一瞬たりとも油断ができない。	
\\	たりとも
\\	あいつには、もう一瞬たりとも平和なときはない…	
\\	たりとも
\\	あなたのことは一日たりとも忘れたことはありません。	
\\	たりとも
\\	これから果たすべき重大なつとめのためには、残された時間を一秒たりとも無駄にできない。	
\\	たりとも
\\	医者は手術の間、一瞬たりとも気が抜けない。	
\\	たりとも
\\	多くの困難にも負けず、努力を続けている彼女はすばらしい。わたしは彼女の成功を願ってやまない。	
\\	てやまない
\\	親は子供の将来を期待してやまないものだ。	
\\	てやまない
\\	自分をおさえながらも、彼女は医者にいろいろなことをたずねてやまなかった。	
\\	てやまない
\\	あなたの幸せを願ってやみません。	
\\	てやまない
\\	なぜ彼女に本当の気持ちを正直に教えてあげなかったのかと後悔してやまない。	
\\	てやまない
\\	別れていても、彼女の幸福を心より祈ってやまない。	
\\	てやまない
\\	われわれは世界平和の実現を願ってやまない。	
\\	てやまない
\\	母が死んで10年以上たつのに、父はまだ母を愛してやまない。	
\\	てやまない
\\	望みとあれば来るがよかろう。	
\\	とあれば
\\	必要とあれば、行動に移るんだ。	
\\	とあれば
\\	いや、いたしますとも、お望みとあればね。	
\\	とあれば
\\	彼は、お金のためとあれば、どんな仕事でも引き受ける。	
\\	とあれば
\\	必要とあれば国家は全力をあげて君を後援するだろう。	
\\	とあれば
\\	あなたのためとあれば、僕はどんなことでもします。	
\\	とあれば
\\	山田さんの発言を皮切りにして、会議は大混乱となった。	
\\	を皮切りに
\\	東京を皮切りに全国20ヶ所でコンサートを開く。	
\\	を皮切りに
\\	花より男子に主演したのを皮切りに、小栗旬はスターへの道を歩み始めていた。	
\\	を皮切りに
\\	彼の発案に皮切りとして、熱心な討議が始まった。	
\\	を皮切りに
\\	彼の店は京都で成功したのを皮切りに、各地に次々と出店していずれも成功を収めた。	
\\	を皮切りに
\\	国防費を皮切りに種々の予算が見直され始めた。	
\\	を皮切りに
\\	大災害により財産のみか肉親までも失った。	
\\	のみか
\\	彼女は日本語のみか、中国語もぺらぺらだ。	
\\	のみか
\\	ヨンさまが日本にやってくるとあって、空港内で大勢のファンが待ち受けている。	
\\	とあって
\\	セールとあって、この店は大勢の客で混んでいる。	
\\	とあって
\\	久し振りの連休とあって、行楽地はどこも人でいっぱいだった。	
\\	とあって
\\	人気俳優がくるとあって、このイベントのチケットはあっという間に売り切れた。	
\\	とあって
\\	月曜日の昼とあって、映画館はがら空きでした。	
\\	とあって
\\	十年ぶりに再会したとあって、二人は深夜まで話していた。	
\\	とあって
\\	決勝戦とあって、その試合の入場券はすぐに売り切れた。	
\\	とあって
\\	な
\\	で(は)かなわない	
\\	-て
\\	(は)かなわない	
\\	い
\\	(-い) 
\\	くて(は)かなわない	てかなわない
\\	隣の工事がうるさくてかなわない。	
\\	てかなわない
\\	階段でドスンドスンされてたんじゃ、集中しにくくてかなわない。	
\\	てかなわない
\\	毎日こう寒くてはかなわない。	
\\	てかなわない
\\	一切を妻のせいにされてはかなわない。あれにはかかわりない事なんだから。	
\\	てかなわない
\\	彼が無口で、一緒にいると退屈でかなわない。	
\\	てかなわない
\\	彼は静かな部屋に隔離され、一時間休んでからテストを受ける始末だった。	
\\	始末だ
\\	うちの娘は、ダイエットのしすぎで病気になり、入院する始末だ。	
\\	始末だ
\\	体を鍛えようとジョギングを始めたが、走りすぎて膝を痛めてしまい、病院に通う始末だ。	
\\	始末だ
\\	あの人は遅刻はする、約束は忘れる、ついには居眠り運転で事故を起こす始末だ。	
\\	始末だ
\\	弟は昔から両親に手を焼かせていましたが、ついに家出をして、警察のやっかいにまでなる始末です。	
\\	始末だ
\\	ちょっと目を離したら、この始末だ。	
\\	始末だ
\\	兄は借金を返すためと言って家にあるものをお金に換え、母のパソコンまで売ってしまう始末だ。	
\\	始末だ
\\	弟はお金がなくて、電気や水道も止められる始末だった。	
\\	始末だ
\\	剽悍というか、火山のような気性の女です。	
\\	というか
\\	それから彼女はまた、普通というか、元に戻ったんです。	
\\	というか
\\	彼は、ひどく臭かった。というか、臭かったのは家かもしれない。	
\\	というか
\\	どんなに彼女が美しいか、彼は忘れていた、というか完全に気づいたことがなかったのだ。	
\\	というか
\\	成人の日に着物を着た彼女の美しさといったらなかった。	
\\	といったらない
\\	彼の得意げな顔といったらなかった。	
\\	といったらない
\\	このところ残業続きで疲れるといったらない。	
\\	といったらない
\\	一週間もお風呂に入らないんだ。汚いといったらありゃしない。	
\\	といったらない
\\	みんなの前で転んで、恥ずかしいったらなかった。	
\\	といったらない
\\	彼はうそをついている。腹立たしいといったらありゃしない。	
\\	といったらない
\\	夜、道を一人で歩いて帰る心細さといったらありはしない。	
\\	といったらない
\\	昨日見た映画は、つまらないといったらなかった。	
\\	といったらない
\\	わたしを見るときの眼つきったら、そりゃ気味がわるいったらないのよ。	
\\	といったらない
\\	一人であんな危険な場所へ行くとは、無茶というか、無知というか、とにかくわたしには理解できない。	
\\	というか~というか
\\	あなた、この特殊なつながりというか、関係というか、何でもいいけど、好きなの?	
\\	というか~というか
\\	課長に叱られたとき、恥ずかしいというか、悔しいというか、言い表せない気持ちだった。	
\\	というか~というか
\\	わたしは合格したが、親友は不合格だった。うれしいというか、残念というか、複雑な気持ちだ。	
\\	というか~というか
\\	平均睡眠時間は8時間といったところだ。	
\\	というところだ/といったところだ
\\	パーティーの参加者は、70人から80人といったところでしょう。	
\\	というところだ/といったところだ
\\	「毎日ここに行くの?」「毎日というわけではない。一周に、一回か二回というところだ。」	
\\	というところだ/といったところだ
\\	自給は800円から1000円というところだ。	
\\	というところだ/といったところだ
\\	勉強時間は3、4時間というところだ。	
\\	というところだ/といったところだ
\\	あなたときたら自分以外の者はどうでもいいとばかりに、しゃべりまくるんですもの、きいていてうんざりするわ。	
\\	とばかりに
\\	彼はお話にもならないとばかりに、肩をすくめてみせた。	
\\	とばかりに
\\	相手のチームに負けたのは山本さんのせいとばかりに、コーチは厳しく彼をしかった。	
\\	とばかりに
\\	彼は出て行けとばかりに、ドアを開けた。	
\\	とばかりに
\\	一つも買いそこねたくない、とばかりに彼女はどれも少しずつ買って、1万円を払った。	
\\	とばかりに
\\	彼は驚いて、こんな事は承知できないとばかりに身をこわばらせた。	
\\	とばかりに
\\	彼は、わたしに早く帰れとばかりに、書類を片付けはじめた。	
\\	とばかりに
\\	わたしのことがわかりはじめたというわけですね?	
\\	というわけだ
\\	その男は顔を見られないように隠していたというわけですね。	
\\	というわけだ
\\	トミーは容疑者というわけですね。	
\\	というわけだ
\\	何を信じるべきかわからないというわけだ。	
\\	というわけだ
\\	彼が一度もおまえを疑わなかったと信じろというわけか?	
\\	というわけだ
\\	いいかえればあの男と結婚するというわけだね。	
\\	というわけだ
\\	犯人は、男でもあれば、女でもありうるというわけですね?	
\\	というわけだ
\\	観察が鋭くて頭脳の働きが早いときているので、情報を集めてくる手先としては理想的だった。	
\\	ときている
\\	このレストランは安くておいしいときているから、いつも客でいっぱいだ。	
\\	ときている
\\	子供の頃から、とても感じ易い性ときているものだから、わたしは、ちらと血を見ただけでも、気が遠くなる。	
\\	ときている
\\	彼女は美人の上にお金持ちときているから、みんなにうらやましがられるのは当然だろう。	
\\	ときている
\\	あの連中ときたら、だれもかれもお金につまっていて、あの女からお金をせしめようとしていない者は、一人もいないんです。	
\\	ときたら
\\	アンソニーときたら、大ばかですよ!	
\\	ときたら
\\	うちの息子ときたら、夏休みになると部屋にこもってゲームばかりする。	
\\	ときたら
\\	あの辺ときたら、とてもたくさんの人が歩いていました。	
\\	ときたら
\\	ここの連中ときたら、何でも信じ込むんだから。	
\\	ときたら
\\	電車でお年寄りに席を譲ろうとしない高校生を見て、父は「近頃の若者ときたら、困ったものだ。」と嘆いていた。	
\\	ときたら
\\	このごろの新聞ときたら、どの新聞も、こんな人殺しのことばかりで、ほかには、なんにもない。	
\\	ときたら
\\	この辺の警察ときたら、この上もなく扱いにくい。	
\\	ときたら
\\	彼は大急ぎで野原のなかにつきすすみ、ゆきあたりばったりに、街道といわず小路といわず、むちゃくちゃにたどっていった。	
\\	といわず
\\	わたしたちの教室のものは、机といわず椅子といわず、めちゃくちゃに壊されていた。	
\\	といわず
\\	昼といわず夜といわず、為替相場は時々刻々と変わっている。	
\\	といわず
\\	山といわず野といわず、一面雪に覆われている。	
\\	といわず
\\	山田さんはウイスキーといわずビールといわず酒であれば、何にでも目がない。	
\\	といわず
\\	みな高橋留美子の愛読者だったところから、彼らは自分たちの集まりを犬夜叉・クラブと呼ぶことにした。	
\\	ところから
\\	金原先生はいつも赤い帽子をかぶっていたところから、「赤い帽子先生」と呼ばれていた。	
\\	ところから
\\	このなかで少なくとも十二、三人は、ものすごい顔つきや、安全灯を持っているところから、坑夫だとわかる。	
\\	ところから
\\	わたしたちには似たところがない。	
\\	ところがある
\\	その話に妙なところがあるのは疑いなかった。	
\\	ところがある
\\	彼には細心なところがあった。	
\\	ところがある
\\	彼女は、何か思うところがあるのか、とても心配そうだった。	
\\	ところがある
\\	あの人には、どこか変なところがある。	
\\	ところがある
\\	その顔だちには妙に複雑なところがあった。	
\\	ところがある
\\	彼といえども、我々と同様に弱みがある。	
\\	といえども
\\	大富豪といえども、この有名な絵を買うことはできない。	
\\	といえども
\\	帝王といえども一人の人間であることに変わりはない。	
\\	といえども
\\	白ひげといえども、まさか蘇ることは…	
\\	といえども
\\	医学がめざましく進歩したといえども、病人の数は減少傾向にはない。	
\\	といえども
\\	たとえ兄弟といえども、憎しみ会うこともある。	
\\	といえども
\\	子供といえども、公共の場で勝手なことをしてはならない。	
\\	といえども
\\	幼児といえども彼の名は知っている。	
\\	といえども
\\	軽い風邪といえども用心するに越したことはない。	
\\	といえども
\\	幅ひろい黒の帽子といい、袋のようにだぶだぶのズボンといい、白ネクタイといい、どう見ても名優ジョニー・ディップを除いてはくらぶべきものもないばかり巧みな変装ぶりであった。	
\\	といい~といい
\\	あの男は世にも珍しい好男子なうえに、心を蕩かすような態度といい、やさしい声音といい、女性には何よりの魅力になるロマンティックで甘いところをそなえているのです。	
\\	といい~といい
\\	彼は肩書きからいっても、資産からいっても、年ごろといい、性格といい、風采といい、まったく申しぶんのない人物だ。	
\\	といい~といい
\\	あの男の態度といい、威張った顔つきといい、また日やけしたところといい、たしかに軍人だ。	
\\	といい~といい
\\	その驚きようといい、悲しげなようすといい、全く自然と思われるほど巧みに演じられた。	
\\	といい~といい
\\	さも 
\\	な
\\	そうに/そうな	さも 
\\	さも 
\\	い
\\	(-い) 
\\	そうに/そうな	さも
\\	彼女は、さもいとしげな微笑を、夫に投げていた。	
\\	さも
\\	彼がさも驚いたというふうで息をきらしてとびこんできた。	
\\	さも
\\	不覚にも、さも悪さをしているような顔つきが、先生の目に止まってしまった。	
\\	さも
\\	彼はさも面白そうにわたしのいうことを聞いていた。	
\\	さも
\\	彼はさも珍しそうにしげしげとわたしのこぶしを見やりました。	
\\	さも
\\	彼女はさも意味ありげな目で僕を見つめた。	
\\	さも
\\	先生は見るともなくぼんやりと全生徒を見渡した。	
\\	ともなく
\\	夕べ、両親の話を聞くともなしに聞いていた。	
\\	ともなく
\\	本を引っ張り出し、開くともなく開いた。	
\\	ともなく
\\	休みの日にはどこへ行くともなしにドライブすることが多い。	
\\	ともなく
\\	「まあ、まあ」彼が誰に言うともなく言った。	
\\	ともなく
\\	ウェートレスは、聞くともなしに客の密談を聞いてしまった。	
\\	ともなく
\\	彼らはこの数カ月というもの、ワールドカップの準備に骨身を惜しまず尽力されてきた。	
\\	というもの
\\	彼女はここ一ヶ月というもの、授業を休んでいる。	
\\	というもの
\\	彼は、この一週間というもの、人の目が届かないところで襲撃する機会を待っていたに違いない。	
\\	というもの
\\	彼女は3年間というもの、完全に自分の絵をかくことに没頭した。	
\\	というもの
\\	それから数日というもの、学校中大会の話でもちきりだった。	
\\	というもの
\\	この一ヶ月間というもの、仕事が忙しくて朝ごはんをろくに食べていない。	
\\	というもの
\\	これはあなたが有罪だという決定的な証拠というわけでもない。	
\\	というわけではない
\\	「彼女とは、いつも夜、会うことになっていたのですか?」「毎晩というわけではないのです。」	
\\	というわけではない
\\	わたし自身も、特別に金持ちというわけではありませんが、兄が金持ちでした。	
\\	というわけではない
\\	ルフィは近くに海軍がいなくて残念だというわけではなかった。	
\\	というわけではない
\\	まったく絶望というわけではございません。	
\\	というわけではない
\\	わたしはいま白状しなければなりませんが、いつもあなたを愛してたというわけではなかった。	
\\	というわけではない
\\	人間の体なんて、厄介な物ですね、とりわけ、重くなりますとね。	
\\	とりわけ
\\	ダ・ヴィンチは、とりわけキリスト教の伝統からすると、つねに歴史学者を悩ませる存在だった。	
\\	とりわけ
\\	いたずらをしていた生徒たちは、教師が来たとみるやいっせいに逃げ出した。	
\\	とみるや
\\	夫が逮捕されるとみるや、彼女は現場を立ち去った。	
\\	とみるや
\\	士官がうなずくとみるや、彼は足の鉄輪についてた鎖を鉄槌の一撃でうちくだいた。	
\\	とみるや
\\	前進を続けると思いきや、リーダーは退却の命令を発した。	
\\	と思いきや
\\	彼は海辺の町で育ったと聞いていたので、さぞかし泳ぎがうまいだろうと思いきや、水に浮くこともできないらしい。	
\\	と思いきや
\\	テレビが直ったと思いきや、またすぐに壊れてしまった。	
\\	と思いきや
\\	試験問題は簡単だったので、満点を取れたと思いきや、名前を書くのを忘れてゼロ点にされてしまった。	
\\	と思いきや
\\	さぞや明るい男だと思いきや、実は、孤独で悩み多き男だった。	
\\	と思いきや
\\	やっと富士山の頂上に着いたと思いきや、まだ8合目だった。	
\\	と思いきや
\\	不合格になったと思いきや、意外なことに合格の通知が来た。	
\\	と思いきや
\\	被害者が生きているのを最後に見たのは誰?その人が嫌疑者と見られるんだ。	
\\	とみられる
\\	この地域では、今後も大きい地震が来ると見られている。	
\\	とみられる
\\	環境破壊と相まって、水や空気の汚染が進んでいる。	
\\	と相まって
\\	高齢化と相まって様々な悪影響が引き起こされている。	
\\	と相まって
\\	不況と相まって、会社員の給料は伸び悩んでいる。	
\\	と相まって
\\	経験不足が相まって、就職は非常に困難だ。	
\\	と相まって
\\	努力と運が相まって、合格することができた。	
\\	と相まって
\\	わたしは、あなたを幸福にできそうにもない女だと存じております。	
\\	そうにもない
\\	彼女はベッドに寝そべっていて、皮をはいだって起きそうにもなかった。	
\\	そうにもない
\\	でも、そんなこと、ありそうにもないでしょう?	
\\	そうにもない
\\	アメリカには、わたし自身が手をつけなければ、うまくゆきそうにもないと思う用事が、いろいろとあった。	
\\	そうにもない
\\	彼の性格からいえば、そんなことはしそうにもないのです。	
\\	そうにもない
\\	今君は自分で自分を監視できそうにもないよ。	
\\	そうにもない
\\	彼女の顔は、ひどく温和な、暴力などとは、およそ関係などありそうにもない顔をしていた。	
\\	そうにもない
\\	たとえわたしがあの人を殺そうとしたって、わたしに復讐する気にはなれますまいよ。	
\\	としたって/としたところで
\\	君が病気になるようなことがあったら、僕どんなことしたって自分がゆるせないよ!	
\\	としたって/としたところで
\\	そんなことを言って、僕に遠慮させようとしたってムダだ!	
\\	としたって/としたところで
\\	そうなったとしたって、わたしには何の責めもかからんです。	
\\	としたって/としたところで
\\	わたしがみんなの目をさまそうとしたって誰がわたしを信じるかしら?	
\\	としたって/としたところで
\\	あの頑固な人にタバコをやめさせようとしたってムダだよ。	
\\	としたって/としたところで
\\	秘密の計画が彼女にばれそうになり、とっさにうそをついてしまった。	
\\	とっさに
\\	言い過ぎた、とみんながとっさにそう思った。	
\\	とっさに
\\	彼はとっさに頭を働かせた。	
\\	とっさに
\\	「そんなことないよ。」彼はとっさに答えた。	
\\	とっさに
\\	どこかで聞いた名だ。しかし、どこだったか、とっさには思い出せない。	
\\	とっさに
\\	誰もが意気あがらず、ともすれば不平が出そうになった。	
\\	ともすれば
\\	彼がともすれば弱りがちな妹を支え半ば抱きかかえるようにして全速力で走った。	
\\	ともすれば
\\	このあいだ一日として家をあけたことなんかありません。	
\\	として~ない
\\	わたしの計画は一つとして狂わなかった。	
\\	として~ない
\\	僕とサミー以外は誰一人として知らない。	
\\	として~ない
\\	今回の期末テストでは、一人として満点をとる生徒はいなかった。	
\\	として~ない
\\	法廷で必要とするような証拠ということになると、まだ、何一つとして挙げられていない。	
\\	として~ない
\\	それ以来わたしには一日として心の休まるときもなかった。	
\\	として~ない
\\	一つとしてこのジャック・オースティンの悲劇事件ほど心を引かれるものはなかった。	
\\	として~ない
\\	不利な材料が一つとしてあがらないのです。	
\\	として~ない
\\	ゆっくり二人で、差しつ差されつ朝まで飲みましょう。	
\\	つ~つ
\\	冷えた体温を回復するために三、四十分間あちらこちら行きつ戻りつした後、子猫のように弱って書斎に移された。	
\\	つ~つ
\\	違反すれば、未成年者とて許すわけにはいかない。	
\\	とて
\\	病気だからとて、無断休むのはけしからん。	
\\	とて
\\	新聞記事だからとて、決していつも正確とは言えません。	
\\	とて
\\	母親とて、彼の本当の気持ちは理解できないだろう。	
\\	とて
\\	常に冷静な彼とてやはり人間だから、感情的になってしまうこともあるのだろう。	
\\	とて
\\	数字はさておき、あなたから見てこのメッセージの一番奇妙な点はどこですか?	
\\	はさておき
\\	実現できるかどうかはさておき、まずは新商品のアイデアをみんなで出してみよう。	
\\	はさておき
\\	ミランダ・カーは、他の事はさておき、とにかく美人だ。	
\\	はさておき
\\	かつてここで数々の式典が催された歴史はさておき、美術愛好家たちはまったく別の理由でこの地に敬意を払っている。	
\\	はさておき
\\	腰を痛めてしまい、歩くことはおろか立つことも難しい。	
\\	はおろか
\\	幼稚園のころの友達の名前はおろか、顔さえ忘れてしまった。	
\\	はおろか
\\	弟は内気で、人前でスピーチはおろか簡単なあいさつさえできない。	
\\	はおろか
\\	海外旅行はおろか国内さえもほとんど見て回ったことがない。	
\\	はおろか
\\	わたしは漢字はおろか、ひらがなも書けない。	
\\	はおろか
\\	病気にかかり、歩くことはおろか起き上がることすらできない。	
\\	はおろか
\\	うちの娘は手間がかかる料理はおろか、簡単な料理を作るのさえできない。	
\\	はおろか
\\	わたしは花粉症がひどくて、外ではおろか、家の中でさえマスクがはずせない。	
\\	はおろか
\\	社会人ともなると、服装や言葉遣いにも気をつけなければならない。	
\\	ともなると/ともなれば
\\	子供は小さいときは素直だが、中学生ともなると、親の言うことを聞かなくなる。	
\\	ともなると/ともなれば
\\	休日ともなると、朝から観光客の車で道路が渋滞する。	
\\	ともなると/ともなれば
\\	12月ともなると、町にはジングルベルのメロディーがあふれる。	
\\	ともなると/ともなれば
\\	部長ともなると、部下を育てることを考えなければならない。	
\\	ともなると/ともなれば
\\	息子の嫁ともなればその地位にはかならずやなみなみならない幸福がいろいろとつきそっておりましょう。	
\\	ともなると/ともなれば
\\	彼が家から出てゆくやいなや、例の旅の男がはいってきた。	
\\	や否や
\\	会社に着くや否や、またまたいいことが起こった。	
\\	や否や
\\	不運な娘が眼をとじるやいなやジョンは手紙をひらこうと思った。	
\\	や否や
\\	わたしはその人の顔を人目見るや否や、10年前に別れた恋人だと気がついた。	
\\	や否や
\\	彼が登場するや否や、会場から割れんばかりの拍手が沸き起こった。	
\\	や否や
\\	ドアが開くや、猫が飛び出してきた。	
\\	や否や
\\	ベルが鳴るや否や、学生たちはいっせいに書き始めた。	
\\	や否や
\\	授業が始まるや否や、彼は眠りを始めた。	
\\	や否や
\\	その男の人は部屋に入るや否や、いきなり大声で怒鳴り始めた。	
\\	や否や
\\	食卓に到着するや否や、彼女が詰め寄った。	
\\	や否や
\\	あんたには独自の行き届いた手段がおありのことと思うとりました。とはいうものの力を貸さぬと申すのではないがな。	
\\	とはいうものの
\\	大学時代はフランス文学専攻だった。とはいうものの、フランス語はほとんどしゃべれない。	
\\	とはいうものの
\\	組織にはそんな弱い男は一人だっていやしない。みんな筋金いりばかりだ。とはいうものの、ピーターのようなやつもいるからな。	
\\	とはいうものの
\\	あなたのいうことは、すべて証拠とされることを、ここに警告します。	
\\	とされる
\\	この新薬は、効果はあるが、副作用が強いとされている。	
\\	とされる
\\	あなたが必要とされる成績を絶対に取れるようにしてみせます!	
\\	とされる
\\	この物質は植物の成長を促進するとされる。	
\\	とされる
\\	彼の業績とされているものが本当に彼のものだったかというと、疑問が残るよ。	
\\	とされる
\\	友人たちの間では、作品についてあれこれと彼にきくのは禁物とされていた。	
\\	とされる
\\	月曜日まで、あなたはなにもできやしないのだ。	
\\	やしない
\\	わたしは一度も自分の部屋を出やしない。	
\\	やしない
\\	わたしたちは動かしようのない証拠をつかまれた。君だってごまかせやしない。	
\\	やしない
\\	何かやってみようとする人は一人もいやしない。	
\\	やしない
\\	僕は誰とだって、金のために結婚するような事はしやしない。	
\\	やしない
\\	あなたの見方は悲観的すぎやしないか。	
\\	やしない
\\	あなたが頼めばどんなことだって断りやしないだろ?	
\\	やしない
\\	あなたに気に入ろうが入るまいが、わたしは注意してあげるんですよ。	
\\	(よ)うが~まいが/(よ)うと~まいと
\\	彼女が結婚しようとしまいと、そんなことはわたしには関係ないことだ。	
\\	(よ)うが~まいが/(よ)うと~まいと
\\	わたしは他人の話に耳を傾けたわけではないんですが、むこうはきかれていようがいまいが、気にとめていなかったのです。	
\\	(よ)うが~まいが/(よ)うと~まいと
\\	雨が降ろうが降るまいが、わたしは出かけます。	
\\	(よ)うが~まいが/(よ)うと~まいと
\\	あいつらが死のうが死ぬまいが気にかける人がいると思うかい?	
\\	(よ)うが~まいが/(よ)うと~まいと
\\	あなたが反対しようとしまいと、わたしは一人で行くつもりです。	
\\	(よ)うが~まいが/(よ)うと~まいと
\\	そんなこと忘れようと忘れまいと、こっちは平気だ。	
\\	(よ)うが~まいが/(よ)うと~まいと
\\	彼に告白しようかするまいか悩んでいる。	
\\	(よ)うか~まいか
\\	事実を言おうか言うまいかと、廊下を行きつ戻りつ考えた。	
\\	(よ)うか~まいか
\\	彼女と結婚しようかするまいか悩んでいます。	
\\	(よ)うか~まいか
\\	大学に入ろうか入るまいか悩んだ末、いい仕事が見つかったので就職することにした。	
\\	(よ)うか~まいか
\\	やりようによっては、その仕事は簡単に済ませることができる。	
\\	ようによっては
\\	まだ見習いでバイト扱いだが、働きようによっては正社員にしてやってもいい。	
\\	ようによっては
\\	この古新聞も、使いようによっては、何かの役に立つのではないかと思いますが。	
\\	ようによっては
\\	考えようによっては彼の人生も幸せだったと言えるかもしれない。	
\\	ようによっては
\\	どんな道具も使いようによっては便利にも危険にもなります。	
\\	ようによっては
\\	インターポールはヨーロッパ内ならいつであろうと、だれがどこに泊まっているかを正確に特定することができる。	
\\	(よ)うが/(よ)うと
\\	誰が何と言おうと、わたしの決意は変わりません。	
\\	(よ)うが/(よ)うと
\\	彼がどんな人間であろうと、われわれのなすべきことは一つしかない!	
\\	(よ)うが/(よ)うと
\\	僕に全財産を残した?どれだけ突飛な想像をしようと、理由が思いあたらない。	
\\	(よ)うが/(よ)うと
\\	どんな悪人であろうと、心のどこかに良心が残っているはずだ。	
\\	(よ)うが/(よ)うと
\\	どんなに物理が得意だろうと、試験のときは油断してはならない。	
\\	(よ)うが/(よ)うと
\\	雨が降ろうと雪が降ろうと明日のパーティーには必ず行くよ。	
\\	(よ)うが/(よ)うと
\\	周囲がいかに反対しようが、自分でやると決めたことは最後までやりぬくつもりだ。	
\\	(よ)うが/(よ)うと
\\	彼女は、黒ずくめのめだたない服装をしている。	
\\	ずくめ
\\	もう十分に悪いことずくめなんだ。君が何でもかんでも自分のせいにしなくたって!	
\\	ずくめ
\\	今年は、息子の結婚、孫の誕生と、めでたいことずくめの一年だった。	
\\	ずくめ
\\	それらはみんな白色だったが、その白ずくめの中で彼女は光りかがやいていた。	
\\	ずくめ
\\	今日は朝からいいことずくめで幸せな気分だ。	
\\	ずくめ
\\	紫ずくめの肥った婦人は輝いて見える。	
\\	ずくめ
\\	彼は頭から足先まで黒ずくめで、その黒服はすりきれてはいたが、小ぎれいだった。	
\\	ずくめ
\\	この一年間はよいことずくめだったが、来年はどうだろうか。	
\\	ずくめ
\\	彼女はものを言うにも息がつけません。	
\\	(よ)うにも~ない
\\	新聞の募集広告に応募しようにも、切手代どころか封筒を買う金もないという始末です。	
\\	(よ)うにも~ない
\\	彼女は頭を上げようにも上げられなかった。	
\\	(よ)うにも~ない
\\	お金を借りようにも、貸してくれる金持ちの友人もいない。	
\\	(よ)うにも~ない
\\	逮捕令状の発給を請求しようにも、根拠となるものがなんにもないのだ。	
\\	(よ)うにも~ない
\\	たとえ飛んでいる飛行機が途中で故障なんかしたら、降りようにも降りられないだろう。	
\\	(よ)うにも~ない
\\	彼女は懇願するようなまなざしで父親を見上げた。そのまなざしはなにか言おうにも口から出てこない言葉以上に、彼女をとりなしてくれた。	
\\	(よ)うにも~ない
\\	ジェイクを呼ぼうにも声が出なかった。	
\\	(よ)うにも~ない
\\	とても 
\\	ではない	とても 
\\	-ない
\\	とても 
\\	はない	とても~ない
\\	とても信じられませんわ。	
\\	とても~ない
\\	たった一回の授業では、とてもこの本の内容を説明しきれない。	
\\	とても~ない
\\	外見から判断すれば、とてもありそうにもないことです。	
\\	とても~ない
\\	彼のような乱暴な男と、わたしのような弱い女とは、とてもおなじ屋根の下にいることはできない。	
\\	とても~ない
\\	一日あなたに会わずにいるの!いいえ、とてもわたし、そんなことできないわ。	
\\	とても~ない
\\	もうとても辛抱できません。	
\\	とても~ない
\\	わたしの夫はとても我慢のならない人なのです。わたしの名誉をきずつけるような目にあわせておりますの。	
\\	とても~ない
\\	優勝したサッカーチームの選手たちの顔という顔が喜びに満ち溢れていた。	
\\	という
\\	学校が始まる日の朝、道路という道路は車であふれていた。	
\\	という
\\	秋になると、山という山は紅葉で赤く染まっていた。	
\\	という
\\	子供のいたずらで、壁という壁は全部落書きされている。	
\\	という
\\	今度という今度は、僕もがまんができないぐらいなんです。	
\\	という
\\	ミルクをこの地域の家という家に配って歩いた。	
\\	という
\\	今日という今日は、警察をそっとしておかなきゃダメですよ。	
\\	という
\\	言われずとも最後はそうしてやろう。	
\\	ずとも
\\	彼は目が見えずともすばらしい演奏をしました。	
\\	ずとも
\\	あなたがわたしの立場だったら自殺せずとも生きようと思いますか?	
\\	ずとも
\\	彼があなたのことを愛するということは、言わずとも分かる。	
\\	ずとも
\\	そんな提案をしようものなら、わたしは大統領職から蹴り落とされる!	
\\	(よ)うものなら
\\	彼女はこの家のなかに僕がいることに気づきでもしようものなら、きっと警戒して、ふたたび僕からのがれようとするだろう。	
\\	(よ)うものなら
\\	昔は親に反抗しようものなら、すぐにたたかれたものだ。	
\\	(よ)うものなら
\\	夜、外にでも出ようものなら、そこらじゅうまっ暗なのです。	
\\	(よ)うものなら
\\	田中さんに対するみんなの偏見ははげしいんですもの。もし好ましい人として説明しようものなら、この町の善良な人々のなかばは困りきってしまってよ。	
\\	(よ)うものなら
\\	山の奥に邸宅を建てるなどとは、物好きな人間もいるものだ!	
\\	とは
\\	見ていたとは、どういうことですか?	
\\	とは
\\	このような事が起こるとは、思いもよらなかった!	
\\	とは
\\	二人はしいて何かききだそうなどとは、つゆ思っていませんでした。	
\\	とは
\\	ちょっとお尋ねしますが、あなたの注意をひいた彼女のお言葉とは、どんなものですか?	
\\	とは
\\	コーヒー一杯で3千円とは、いくらなんでも高すぎる。	
\\	とは
\\	こんなにはっきりしたことに気がつかないとは、わたしはのろまだった。	
\\	とは
\\	ああいった人物が、小説の主人公以外に実在しようとは、夢にも思わなかったなあ。	
\\	とは
\\	あなたの功績はひろく公表して、一般に認められてしかるべきだろう。	
\\	てしかるべきだ
\\	君のやりかたはもうすこし思慮があってしかるべきだった。	
\\	てしかるべきだ
\\	状況が変わったのだから、会社の経営計画も見直されてしかるべきだ。	
\\	てしかるべきだ
\\	退学する前に、親と相談してしかるべきだ。	
\\	てしかるべきだ
\\	悪天候ゆえに旅行は延期された。	
\\	ゆえに
\\	何ゆえにうそをついたのか。	
\\	ゆえに
\\	あの大統領は庶民性をそなえているがゆえに、人気を集めているという。	
\\	ゆえに
\\	動かぬ証拠があるゆえ、有罪が確定した。	
\\	ゆえに
\\	貧しいがゆえに十分な教育を受けられない人がいる。	
\\	ゆえに
\\	有名人ゆえの悩みがある。	
\\	ゆえに
\\	彼のボスが何とおっしゃったやら、正確には聞けずじまいだった。	
\\	ずじまい
\\	わたしたちは首をながくして、タイタニック号の消息のくるのを待ちわびていたのだが、それはついに聞かれずじまいであった。	
\\	ずじまい
\\	父に聞かずじまいになってしまったことが、あまりにも多い。	
\\	ずじまい
\\	アイデアはよかったが、実現せずじまいだった。	
\\	ずじまい
\\	結局宝石はどこにあるのか見つからずじまいだった。	
\\	ずじまい
\\	有名な観光地の近くまで行ったのに、忙しくてどこへもよらずじまいだった。	
\\	ずじまい
\\	想像するしかないだろう。本当のことはわからずじまいかもしれない。	
\\	ずじまい
\\	好きな彼女に告白できずじまいで卒業してしまった。	
\\	ずじまい
\\	とはいえ、今夜よい機会を与えてくださって感謝しています。	
\\	とはいえ
\\	12月に入ったとはいえ、まだ年末という気がしない。	
\\	とはいえ
\\	とはいえ、指示には従わなくてはならない。	
\\	とはいえ
\\	昔に比べて体力が衰えたとはいえ、まだまだ若い者には負けない。	
\\	とはいえ
\\	とはいえ、一連の疑わしい状況が存在します。	
\\	とはいえ
\\	まだ締め切りまで時間があるとはいえ、早めに完成させておいたほうがいい。	
\\	とはいえ
\\	とはいえ、入室する前から、彼女は必要なものが欠けていることに気づいていた。	
\\	とはいえ
\\	わざとじゃなかったとはいえ、迷惑をかけたのだから謝るべきだ。	
\\	とはいえ
\\	春とはいえ、まだまだ寒い日が続いている。	
\\	とはいえ
\\	招かれざる見学者が誰なのかを察した。	
\\	ざる
\\	彼はここから脱出することになる。本人が望むと望まざるとにかかわらず。	
\\	ざる
\\	知られざる異常者が、この屋敷のどこかに潜んでいるにちがいない!	
\\	ざる
\\	招かれざる客が尋ねてくることはないと確信できた。	
\\	ざる
\\	もうあなたは、ここでは歓迎されざる者だ。	
\\	ざる
\\	行[い]く	日曜日[にちようび]は 図書館[としょかん]に 行[い]きます。		
\\	見[み]る	仕事[しごと]のあと、 映画[えいが]を 見[み]た。		
\\	多[おお]い	この 道[みち]は 車[くるま]が 多[おお]い。		
\\	家[うち]	家[うち]に 遊び[あそび]にきてください。		
\\	これ	これをください。		
\\	私[わたし]	私[わたし]は 毎朝[まいあさ]コーヒーを 飲[の]みます。		
\\	仕事[しごと]	あたしは3 月[がつ]は 仕事[しごと]が 忙[いそが]しい。		
\\	いつ	彼女[かのじょ]はいつ 来[き]ますか。		
\\	する	友達[ともだち]と 一緒[いっしょ]に 宿題[しゅくだい]をした。		
\\	出[で]る	今日[きょう]は5 時[じ]に 会社[かいしゃ]を 出[で]ます。		
\\	使[つか]う	彼女[かのじょ]は 辞書[じしょ]をよく 使[つか]います。		
\\	所[ところ]	今年[ことし]の 夏休[なつやす]みには 静[しず]かな 所[ところ]へ 行[い]きたい。		
\\	作[つく]る	今[いま]、 朝[あさ]ご 飯[はん]を 作[つく]っています。		
\\	思[おも]う	あなたはどう 思[おも]いますか。		
\\	持[も]つ	私[わたし]は 車[くるま]を 持[も]っています。		
\\	買[か]う	本[ほん]を1 冊[さつ] 買[か]いました。		
\\	時間[じかん]	今[いま]は 時間[じかん]がありません。		
\\	知[し]る	誰[だれ]もその 話[はなし]を 知[し]らない。		
\\	同[おな]じ	彼[かれ]の 日本語[にほんご]のレベルは 私[わたし]と 同[おな]じくらいだ。		
\\	今[いま]	彼[かれ]は 今[いま]、 勉強[べんきょ]しています。		
\\	新[あたら]しい	新[あたら]しいテーブルを 買[か]いました。		
\\	なる	彼[かれ]は 医者[いしゃ]になりました。		
\\	まだ	宿題[しゅくだい]はまだ 終[お]わっていません。		
\\	あと	仕事[しごと]のあと、 映画[えいが]を 見[み]た。		
\\	聞[き]く	彼女[かのじょ]はラジオを 聞[き]いています。		
\\	言[い]う	上司[じょうし]が
\\	一杯[いっぱい]、 飲[の]もう」と 言[い]った。		
\\	少[すく]ない	今年[ことし]は 雨[あめ]が 少[すく]ないです。		
\\	高[たか]い	これがこの 町[まち]で 一番[いちばん] 高[たか]いビルです。		
\\	子供[こども]	電車[でんしゃ]で 子供[こども]が 騒[さわ]いでいた。		
\\	そう	彼女[かのじょ]はそう 言[い]いました。		
\\	もう	彼[かれ]はもう 帰[かえ]りました。		
\\	学生[がくせい]	彼[かれ]は 真面目[まじめ]な 学生[がくせい]です。		
\\	熱[あつ]い	このスープはとても 熱[あつ]い。		
\\	どうぞ	こちらへどうぞ。		
\\	午後[ごご]	明日[あす]の 午後[ごご]、お 客様[きゃくさま]が 来[く]る。		
\\	長
\\	[なが]い	彼女[かのじょ]の 髪[かみ]はとても 長[なが]い。		
\\	本[ほん]	本[ほん]を1 冊[さつ] 買[か]いました。		
\\	今年[ことし]	今年[ことし]はイタリアに 旅行[りょこう]したい。		
\\	よく	彼女[かのじょ]はよく 旅行[りょこう]に 行[い]きます。		
\\	彼女[かのじょ]	彼女[かのじょ]はいつも 元気[げんき]だ。		
\\	どう	あなたはどう 思[おも]いますか。		
\\	言葉[ことば]	この 言葉[ことば]の 意味[いみ]が 分[わ]かりません。		
\\	顔[かお]	早[はや]く 顔[かお]を 洗[あら]いなさい。		
\\	終[お]わる	やっと 仕事[しごと]が 終[お]わりました。		
\\	一[ひと]つ	それを 一[ひと]つください。		
\\	あげる	この 本[ほん]、あなたにあげます。		
\\	こう	こう 小[ちい]さい 字[じ]は 読[よ]めない。		
\\	学校[がっこう]	この 学校[がっこう]は1クラス30 人[にん]です。		
\\	くれる	父[ちち]が 本[ほん]をくれました。		
\\	始[はじ]める	テストを 始[はじ]めてください。		
\\	起[お]きる	今日[きょう]は7 時[じ]に 起[お]きました。		
\\	春[はる]	私[わたし]の 一番[いちばん] 好[す]きな 季節[きせつ]は 春[はる]です。		
\\	午前[ごぜん]	午前[ごぜん]9 時[じ]のニュースです。		
\\	別[べつ]	別[べつ]の 本[ほん]も 見[み]せてください。		
\\	どこ	あの 本[ほん]をどこに 置[お]きましたか。		
\\	部屋[へや]	私[わたし]の 部屋[へや]は2 階[かい]にあります。		
\\	若[わか]い	その 店[みせ]は 若[わか]い 客[きゃく]が 多[おお]いです。		
\\	車[くるま]	弟[おとうと]が 車[くるま]を 買[か]った。		
\\	置[お]く	彼[かれ]はかばんをいすの 上[うえ]に 置[お]きました。		
\\	住[す]む	彼[かれ]は 会社[かいしゃ]の 近[ちか]くに 住[す]んでいる。		
\\	働[はたら]く	姉[あね]は 銀行[ぎんこう]で 働[はたら]いています。		
\\	難[むずか]しい	外国語[がいこくご]を 習[なら]うのは 難[むずか]しい。		
\\	先生[せんせい]	私[わたし]は 日本語[にほんご]の 先生[せんせい]になりたいです。		
\\	立[た]つ	道[みち]の 真[ま]ん 中[な]に 人[ひと]が 立[た]っている。		
\\	呼[よ]ぶ	タクシーを 呼[よ]んでください。		
\\	大学[だいがく]	大学[だいがく]に 行[い]ってもっと 勉強[べんきょう]したいです。		
\\	安[やす]い	この 服[ふく]はとても 安[やす]かった。		
\\	もっと	もっとゆっくり 話[はな]してください。		
\\	帰[かえ]る	家[うち]に 帰[かえ]ろう。		
\\	分[わ]かる	その 子[こ]は 日本語[にほんご]が 分[わ]からない。		
\\	広[ひろ]い	彼[かれ]の 家[いえ]はとても 広[ひろ]い。		
\\	数[かず]	いすの 数[かず]を 数[かぞ]えてください。		
\\	近[ちか]い	駅[えき]はここから 近[ちか]いです。		
\\	そこ	そこに 座[すわ]ってください。		
\\	走[はし]る	彼[かれ]は 走[はし]るのが 速[はや]い。		
\\	入[い]れる	彼[かれ]はかばんに 手帳[てちょう]を 入[い]れた。		
\\	教[おし]える	彼[かれ]は 数学[すうがく]を 教[おし]えています。		
\\	歩[ある]く	駅[えき]まで 歩[ある]きましょう。		
\\	会[あ]う	昨日[きのう]おじさんに 会[あ]いました。		
\\	書[か]く	彼[かれ]に 手紙[てがみ]を 書[か]きました。		
\\	頭[あたま]	今朝[けさ]から 頭[あたま]が 痛[いた]い。		
\\	売[う]る	彼[かれ]は 家[いえ]を 売[う]った。		
\\	大好[だいす]き	私[わたし]は 犬[いぬ]が 大好[だいす]きだ。		
\\	体[からだ]	私[わたし]は 体[からだ]が 丈夫[じょうぶ]だ。		
\\	直[す]ぐ	直[す]ぐ 行[い]きます。		
\\	飛[と]ぶ	鳥[とり]が 飛[と]んでいます。		
\\	とても	この 本[ほん]はとてもおもしろい。		
\\	誰[だれ]	誰[だれ]と 会[あ]ってみたいですか。		
\\	好[す]き	私[わたし]はワインが 好[す]きです。		
\\	読[よ]む	テキストを 読[よ]んでください。		
\\	次[つぎ]	次[つぎ]の 駅[えき]で 降[お]ります。		
\\	あなた	あなたの 名前[なまえ]を 教[おし]えてください。		
\\	飲[の]む	友達[ともだち]とお 酒[おさけ]を 飲[の]んでいます。		
\\	古[ふる]い	私[わたし]は 古[ふる]い 車[くるま]が 好[す]きです。		
\\	質問[しつもん]	質問[しつもん]は 何[なん]ですか。		
\\	今日[きょう]	今日[きょう]は 仕事[しごと]がありません。		
\\	友達[ともだち]	友達[ともだち]とキャンプに 行[い]った。		
\\	早[はや]い	まだ 学校[がっこう]へ 行[い]くには 早[はや]い 時間[じかん]です。		
\\	どれ	ソースはどれですか。		
\\	それ	それ
\\	はとてもいい 話[はなし]だ。		
\\	美[うつく]しい	今夜[こんや]は 星[ほし]がとても 美[うつく]しい。		
\\	いつも	朝[あさ]ご 飯[はん]にはいつもパンを 食[た]べる。		
\\	足[あし]	彼[かれ]は 足[あし]が 長[なが]い。		
\\	起[お]こす	明日[あす]、6 時[じ]に 起[お]こしてください。		
\\	見[み]せる	その写真[しゃしん]を 見[み]せてください。		
\\	娘[むすめ]	私[わたし]の 娘[むすめ]はアメリカにいます。		
\\	楽[たの]しむ	日本[にほん]の 生活[せいかつ]を 楽[たの]しんでください。		
\\	色[いろ]	どっちの 色[いろ]が 好[す]きですか。		
\\	みんな	みんなにお 菓子[かし]をあげましょう。		
\\	取[と]る	テストでいい 点[てん]を 取[と]った。		
\\	勉強[べんきょう](~する)	私[わたし]は 日本語[にほんご]を 勉強[べんきょう]しています。		
\\	できる	彼女[かのじょ]はイタリア 語[ご]ができます。		
\\	短[みじか]い	彼[かれ]は 足[あし]が 短[みじか]い。		
\\	落[お]ちる	道[みち]に 何[なに]か 落[お]ちています。		
\\	息子[むすこ]	息子[むすこ]は 中学生[ちゅうがくせい]です。		
\\	白[しろ]い	彼[かれ]は 歯[は]が 白[しろ]い。		
\\	飛行機[ひこうき]	息子[むすこ]は 飛行機[ひこうき]のおもちゃが 好[す]きです。		
\\	病気[びょうき]	祖父[そふ]が 病気[びょうき]になった。		
\\	冬[ふゆ]	ロシアの 冬[ふゆ]はとても 寒[さむ]いです。		
\\	年[とし]	新[あたら]しい 年[とし]が 始[はじ]まりました。		
\\	重[おも]い	このかばんは 重[おも]いです。		
\\	胸[むね]	胸[むね]に 少[すこ]し 痛[いた]みがあります。		
\\	払[はら]う	別々[べつべつ]に 払[はら]いましょう。		
\\	軽[かる]い	この 靴[くつ]はとても 軽[かる]い。		
\\	見[み]つける	新[あたら]しい 仕事[しごと]を 見[み]つけました。		
\\	忘[わす]れる	電車[でんしゃ]に 傘[かさ]を 忘[わす]れた。		
\\	酒[さけ]	友達[ともだち]とお 酒[さけ]を 飲[のん]んでいます。		
\\	どちら	コーヒーと 紅茶[こうちゃ]、どちらにしますか。		
\\	姉[あね]	姉[あね]は 大学生[だいがくせい]です。		
\\	覚[おぼ]える	姉[あね]は 平仮名[ひらがな]を 全部[ぜんぶ] 覚[おぼ]えました。		
\\	狭[せま]い	私[わたし]の 部屋[へや]は 狭[せま]いです。		
\\	赤[あか]い	赤[あか]いバラを 買[か]いました。		
\\	着[き]る	このパジャマを 着[き]て。		
\\	笑[わら]う	赤[あか]ん 坊[ぼう]が 笑[わら]っています。		
\\	一番[いちばん]	彼[かれ]はクラスで 一番[いちばん] 背[せ]が 高[たか]い。		
\\	授業[じゅぎょう](~する)	今日[きょう]は 日本語[にほんご]の 授業[じゅぎょう]があります。		
\\	週
\\	[しゅう]	私[わたし]は 週[しゅう]に1 回[かい]、テニスをします。		
\\	漢字[かんじ]	漢字[かんじ]は 中国[ちゅうごく]からきました。		
\\	自転車[じてんしゃ]	毎日[まいにち]、 駅[えき]まで 自転車[じてんしゃ]で 行[い]きます。		
\\	電車[でんしゃ]	電車[でんしゃ]に 傘[かさ]を 忘[わす]れた。		
\\	探[さが]す	彼[かれ]は 郵便局[ゆうびんきょく]を 探[さが]していました。		
\\	紙[かみ]	紙[かみ]と 鉛筆[えんぴつ]はありますか。		
\\	歌[うた]う	母[はは]はいつもその 歌[うた]を 歌[うた]います。		
\\	遅[おそ]い	前[まえ]の 車[くるま]はとても 遅[おそ]い。		
\\	首[くび]	きりんの 首[くび]は 長[なが]い。		
\\	速[はや]い	彼[かれ]は 走[はし]るのが 速[はや]い。		
\\	一緒[いっしょ]に	一緒[いっしょ]に 帰[かえ]りましょう。		
\\	今月[こんげつ]	今月[こんげつ]はとても 忙[いそが]しい。		
\\	遊[あそ]ぶ	子供[こども]たちが 公園[こうえん]で 遊[あそ]んでいる。		
\\	遠[とお]い	家[いえ]から 学校[がっこう]までは 遠[とお]いです。		
\\	弱[よわ]い	その 子[こ]は 体[からだ]が 少[すこ]し 弱[よわ]い。		
\\	耳[みみ]	彼女[かのじょ]は 耳[みみ]がよく 聞[き]こえません。		
\\	座[すわ]る	そこに 座[すわ]ってください。		
\\	右[みぎ]	彼[かれ]は 右[みぎ]の 耳[みみ]がよく 聞[き]こえない。		
\\	浴[あ]びる	朝[あさ]、 急[いそ]いでシャワーを 浴[あ]びました。		
\\	肩[かた]	肩[かた]が 凝[こ]りました。		
\\	寝[ね]る	彼[かれ]はベッドで 寝[ね]ています。		
\\	消[け]す	昼間[ひるま]は 電気[でんき]を 消[け]してください。		
\\	元気[げんき]	彼女[かのじょ]はいつも 元気[げんき]だ。		
\\	全部[ぜんぶ]	私[わたし]は 平仮名[ひらがな]を 全部[ぜんぶ] 読[よ]めます。		
\\	去年[きょねん]	私[わたし]は 去年[きょねん]フランスへ 行[い]った。		
\\	引[ひ]く	このドアは 引[ひ]いてください。		
\\	図書館[としょかん]	図書館[としょかん]で 料理[りょうり]の 本[ほん]を 借[か]りた。		
\\	上[あ]げる	分[わ]かった 人[ひと]は 手[て]を 上[あ]げてください。		
\\	緑[みどり]	あの 町[まち]は 緑[みどり]を 増[ふ]やしています。		
\\	腕[うで]	彼[かれ]の 腕[うで]は 太[ふと]い。		
\\	ドア	電車[でんしゃ]のドアが 開[あ]きました。		
\\	女[おんな]の 子[こ]	あの 女[おんな]の 子[こ]を 知[し]っていますか。		
\\	男[おとこ]の 子[こ]	男[おとこ]の 子[こ]たちがサッカーをしている。		
\\	私[わたし]たち	私[わたし]たちはビデオゲームをした。		
\\	近[ちか]く	私[わたし]の 家[いえ]は 駅[えき]の 近[ちか]くです。		
\\	やる	一緒[いっしょ]に 宿題[しゅくだい]をやろう。		
\\	かなり	彼[かれ]はかなり 英語[えいご]が 上手[じょうず]です。		
\\	国[くに]	私[わたし]はこの 国[くに]の 文化[ぶんか]を 勉強[べんきょう]しています。		
\\	起[お]こる	大変[たいへん]なことが 起[お]こりました。		
\\	秋[あき]	秋[あき]はぶどうが 美味[おい]しいです。		
\\	送[おく]る	彼[かれ]の 家[いえ]に 荷物[にもつ]を 送[おく]りました。		
\\	死[し]ぬ	犬[いぬ]が 病気[びょうき]で 死[し]にました。		
\\	気持[きも]ち	彼[かれ]の 気持[きも]ちが 分[わ]からない。		
\\	乗[の]る	私[わたし]たちは 船[ふね]に 乗[の]った。		
\\	いる	父[ちち]は 家[いえ]の 中[なか]にいます。		
\\	木[き]	台風[たいふう]で 木[き]が 倒[たお]れた。		
\\	開[あ]ける	窓[まど]を 開[あ]けてください。		
\\	閉[し]める	ちゃんとドアを 閉[し]めてよ。		
\\	続[つづ]く	工事[こうじ]は3 月[がつ]まで 続[つづ]きます。		
\\	お 医者[いしゃ]さん	あたしは 熱[ねつ]があるのでお 医者[いしゃ]さんに 行[い]った。		
\\	円[えん]	この 靴[くつ]は1 万[まん] 円[えん]です。		
\\	ここ	ここに 本[ほん]があります。		
\\	待[ま]つ	あなたが 来[く]るのを 待[ま]っています。		
\\	低[ひく]い	彼[かれ]は 背[せ]が 低[ひく]い。		
\\	もらう
\\	彼女[かのじょ]からプレゼントをもらいました。		
\\	食[た]べる	昨日[きのう]の 夜[よる]は 肉[にく]をたくさん 食[た]べました。		
\\	兄[あに]	兄[あに]はゴルフを 始[はじ]めました。		
\\	名前[なまえ]	あなたの 名前[なまえ]を 教[おし]えてください。		
\\	夫[おっと]	夫[おっと]を「パパ」と 呼[よ]ぶ 女[おんな]の 人[ひと]は 少[すく]なくない。		
\\	一[いち]	日本語[にほんご]で 一[いち]から 十[じゅう]まで 数[かぞ]えてください。		
\\	結婚[けっこん](~する)	彼女[かのじょ]は 来月[らいげつ] 結婚[けっこん]します。		
\\	親[おや]	彼女[かのじょ]もそのうち、 親[おや]の 気持[きも]ちが 分[わ]かるだろう。		
\\	話[はな]す	母[はは]と 電話[でんわ]で 話[はな]しました。		
\\	少[すこ]し	ワインを 少[すこ]し 飲[のみ]みました。		
\\	閉[と]じる	教科書[きょうかしょ]を 閉[と]じてください。		
\\	時[とき]	あの 時[とき]、 私[わたし]はまだ 学生[がくせい]でした。		
\\	米[こめ]	日本人[にほんじん]はお 米[こめ]が 大好[だいす]きです。		
\\	切[き]る	この 紙[かみ]を 半分[はんぶん]に 切[き]ってください。		
\\	楽[たの]しい	彼[かれ]はとても 楽[たの]しい 人[ひと]です。		
\\	服[ふく]	この 服[ふく]はとても 安[やす]かった。		
\\	後[うし]ろ	後[うし]ろを 向[む]いて。		
\\	嬉[うれ]しい	彼[かれ]に 会[あ]えて 嬉[うれ]しかった。		
\\	腰[こし]	昨日[きのう]から 腰[こし]が 痛[いた]い。		
\\	日曜日[にちようび]	日曜日[にちようび]は 海[うみ]に 行[い]きました。		
\\	昼[ひる]	私[わたし]は 昼[ひる]のドラマを 毎日[まいにち] 見[み]ます。		
\\	お 母[かあ]さん	お 母[かあ]さんによろしくお 伝[おつ]えください。		
\\	大学生[だいがくせい]	来月[らいげつ]から 大学生[だいがくせい]になります。		
\\	終[お]わり	夏休[なつやす]みももう 終[お]わりだ。		
\\	背[せ]	彼女[かのじょ]は 背[せ]が 高[たか]い。		
\\	手伝[てつだ]う	私[わたし]が 手伝[てつだ]いましょう。		
\\	鼻[はな]	鼻[はな]がかゆいです。		
\\	起[お]きる
\\	事故[じこ]はここで 起[お]きました。		
\\	載[の]せる	皿[さら]にケーキを 載[の]せました。		
\\	悲[かな]しい	その 映画[えいが]はとても 悲[かな]しかった。		
\\	しゃべる	彼女[かのじょ]はよくしゃべるね。		
\\	近[ちか]く 
\\	近[ちか]く、 留学[りゅうがく]する 予定[よてい]です。		
\\	甘[あま]い	このいちごは 甘[あま]い。		
\\	テーブル	こっちの 四角[しかく]いテーブルを 買[か]おうよ。		
\\	食[た]べ 物[もの]	日本[にっぽん]の 食[た]べ 物[もの]はとても 美味[おい]しいです。		
\\	始[はじ]まる	新[あたら]しい 仕事[しごと]が 始[はじ]まりました。		
\\	ゲーム	このゲームのルールは 簡単[かんた]です。		
\\	十
\\	[じゅう]	その 子[こ]は 指[ゆび]で 十[じゅう] 数[かぞ]えました。		
\\	天気[てんき]	明日[あす]の 天気[てんき]は 晴[は]れです。		
\\	暑[あつ]い	今日[きょう]はとても 暑[あつ]い 天気[てんき]です。		
\\	太[ふと]い	彼女[かのじょ]は 足[あし]が 太[ふと]い。		
\\	晩[ばん]	彼[かれ]は 朝[あさ]から 晩[ばん]まで 働[はたら]いたの。		
\\	土曜日[どようび]	土曜日[どようび]の 夜[よる]はクラブに 行[い]きます。		
\\	痛[いた]い	今日[きょう]は 頭[あたま]が 痛[いた]いです。		
\\	お 父[とう]さん	お 父[とう]さんは 会社員[かいしゃいん]です。		
\\	多分[たぶん]	彼女[かのじょ]は 多分[たぶん] 家[いえ]で 寝[ね]ています。		
\\	時計[とけい]	あの 家[いえ]には 時計[とけい]が 八[やっ]つあります。		
\\	泊[と]まる	今日[きょう]はこのホテルに 泊[と]まります。		
\\	どうして	どうして 泣[な]いているの。		
\\	掛[か]ける	本[ほん]にカバーを 掛[か]けました。		
\\	曲
\\	[ま]がる	そこを 左[ひだり]に 曲[ま]がってください。		
\\	お 腹[なか]	お 腹[なか]が 空[す]きました。		
\\	ミーティング	朝[あさ]9 時[じ]からミーティングが 始[はじ]まった。		
\\	嫌[きら]い	私[わたし]はタバコが 嫌[きら]いです。		
\\	金曜日[きんようび]	金曜日[きんようび]の 夜[よる]は 友達[ともだち]と 出[で]かけます。		
\\	要[い]る	袋[ふくろ]は 要[い]りません。		
\\	無[な]い	ここには 何[なに]も 無[な]い。		
\\	黄色[きいろ]い	彼女[かのじょ]に 黄色[きいろい]いバラを 買[か]いました。		
\\	風邪[かぜ]	私[わたし]は 風邪[かぜ]をひきました。		
\\	優[やさ]しい	彼[かれ]はとても 優[やさ]しい 人[ひと]です。		
\\	晴[は]れる	今日[きょう]はよく 晴[は]れましたね。		
\\	汚[きたな]い	彼[かれ]の 部屋[へや]はとても 汚[きたな]い。		
\\	茶色[ちゃいろ]	彼女[かのじょ]は 茶色[ちゃいろ]の 靴[くつ]を 履[は]いています。		
\\	空[す]く	レストランは 空[す]いていました。		
\\	上[のぼ]る	猫[ねこ]が 屋根[やね]に 上[のぼ]っている。		
\\	ご 飯[はん]	私[わたし]はパンよりご 飯[はん]が 好[す]きだ。		
\\	日[にち]	彼女[かのじょ]の 誕生日[たんじょうび]は7 月[がつ]16 日[にち]です。		
\\	髪[かみ]の 毛[け]	彼[かれ]の 部屋[へや]に 長[なが]い 髪[かみ]の 毛[け]が 落[お]ちていました。		
\\	つける	電気[でんき]をつけてください。		
\\	月曜日[げつようび]	月曜日[げつようび]に 会[あ]いましょう。		
\\	入[はい]る	寒[さむい]いので 中[なか]に 入[はい]ってください。		
\\	カタカナ	お 名前[なまえ]をカタカナで 書[か]いてください。		
\\	今週[こんしゅう]	今週[こんしゅう]は 日本語[にほんご]のテストがあります。		
\\	開[ひら]く	32ページを 開[ひら]いてください。		
\\	水[みず]	水[みず]を 一杯[いっぱい]ください。		
\\	あれ	あれは 何[なん]ですか。		
\\	二[に]	兄[あに]は 二[に] 階[かい]にいます。		
\\	締[し]める	彼[かれ]はシートベルトを 締[し]めた。		
\\	まずい	ここの 料理[りょうり]はまずい。		
\\	平仮名[ひらがな]	姉[あね]は 平仮名[ひらがな]を 全部[ぜんぶ] 覚[おぼ]えました。		
\\	曇[くも]る	今日[きょう]は 朝[あさ]から 空[そら]が 曇[くも]っている。		
\\	触[さわ]る	絵[え]に 触[さわ]らないでください。		
\\	駄目[だめ](~な)	彼[かれ]は 駄目[だめ]な 男[おとこ]だ。		
\\	飲[の]み 物[もの]	何[なに]か 飲[の]み 物[もの]が 欲[ほ]しいな。		
\\	木曜日[もくようび]	木曜日[もくようび]は 仕事[しごと]が 休[やす]みです。		
\\	曜日[ようび]	曜日[ようび]を 間違[まちが]えました。		
\\	そば	そばにいてください。		
\\	こっち	こっちに 来[き]てください。		
\\	火曜日[かようび]	火曜日[かようび]は 日本語[にほんご]の 授業[じゅぎょう]があります。		
\\	渇[かわ]く	あたしは 渇[かわ]きました。		
\\	三[さん]	彼女[かのじょ]は 三[さん] 人[にん]の 子供[こども]の 母親[ははおや]だ。		
\\	水曜日[すいようび]	水曜日[すいようび]はバイトがあります。		
\\	二[ふた]つ	テーブルの 上[うえ]にケーキが 二[ふた]つあります。		
\\	今晩[こんばん]	今晩[こんばん]のパーティーは 何[なん] 時[じ]からですか。		
\\	千[せん]	千[せん] 円[えん] 貸[か]してください。		
\\	六日[むいか]	六日[むいか] 前[まえ]に 日本[にほん]に 帰[かえ]ってきました。		
\\	お 姉[ねえ]さん	昨日[きのう]、あなたのお 姉[ねえ]さんに 会[あ]ったよ。		
\\	直[なお]る	クーラーはまだ 直[なお]りません。		
\\	ちょっと	もうちょっと 塩[しお]を 入[い]れて。		
\\	四[よん]	パーティーがあるので、ワインを 四[よん] 本[ほん] 買[か]いました。		
\\	これから	これからはもっと 気[き]を 付[つ]けます。		
\\	考[かんが]える	今[いま]、 何[なに]を 考[かんが]えていたんですか。		
\\	戻[もど]る	今[いま]、 会社[かいしゃ]に 戻[もど]ります。		
\\	変[か]える	あたしは 旅行[りょこう]の 日程[にってい]を 変[か]えました。		
\\	朝[あさ]	朝[あさ]、カーテンを 開[あ]ける。		
\\	歯[は]	彼[かれ]は 歯[は]が 白[しろ]い。		
\\	頑張[がんば]る	明日[あした]の 試験[しけん]、 頑張[がんば]ってね。		
\\	携帯電話[けいたいでんわ]	この 携帯電話[けいたいでんわ]はもう 古[ふる]い。		
\\	雨[あめ]	雨[あめ]が 降[ふ]っています。		
\\	金[かね]	今月[こんげつ]はあまりお 金[かね]がない。		
\\	易[やさ]しい	この 問題[もんだい]はかなり 易[やさ]しいです。		
\\	お 兄[にい]さん	あなたのお 兄[にい]さんは 何[なん] 歳[さい]?		
\\	大[おお]きい	海[うみ]は 広[ひろ]くて 大[おお]きい。		
\\	小[ちい]さい	小[ちい]さい 花[はな]が 咲[さ]いています。		
\\	辛[から]い	彼[かれ]は 辛[から]いものが 好[す]きです。		
\\	八[はち]	りんごを 八[はち] 個[こ]ください。		
\\	あそこ	あそこにバス 停[てい]があります。		
\\	来[く]る	彼[かれ]は 昼[ひる]すぎに 来[き]ます。		
\\	前[まえ]	その 店[みせ]の 前[まえ]で 会[あ]いましょう。		
\\	五日[いつか]	五[ご] 月[がつ] 五日[いつか]は 祝日[しゅくじつ]です。		
\\	いっぱい	プールは 人[ひと]でいっぱいです。		
\\	九[きゅう]	野球[やきゅう]は 九[きゅう] 人[にん]で1チームです。		
\\	酸[す]っぱい	このぶどうは 酸[す]っぱいです。		
\\	違[ちが]う	答[こた]えが 違[ちが]います。		
\\	細[ほそ]い	彼女[かのじょ]は 指[ゆび]が 細[ほそ]いですね。		
\\	三[みっ]つ	コップを 三[みっ]つ 買[か]いました。		
\\	八日[ようか]	八日[ようか]からイギリスに 行[い]きます。		
\\	高校生[こうこうせい]	私[わたし]の 弟[おとうと]は 高校生[こうこうせい]です。		
\\	上手[じょうず](~な)	妹[いもうと]は 歌[うた]が 上手[じょうず]です。		
\\	強[つよ]い	今日[きょう]は 風[かぜ]が 強[つよ]い。		
\\	七
\\	[なな]	バナナが 七[なな] 本[ほん]あります。		
\\	二十日[はつか]	来月[らいげつ]の 二十日[はつか]は 弟[おとうと]の 誕生日[たんじょうび]です。		
\\	左[ひだり]	この 車[くるま]は 左[ひだり]ハンドルです。		
\\	二日[ふつか]	私[わたし]は 二日[ふつか] 待[ま]った。		
\\	四[よっ]つ	私[わたし]は 腕時計[うでどけい]を 四[よっ]つ 持[も]っています。		
\\	暖[あたた]かい	今年[ことし]の 春[はる]は 暖[あたた]かいね。		
\\	ある	私[わたし]の 机[つくえ]の 上[うえ]に 書類[しょるい]がたくさんあります。		
\\	いい	彼女[かのじょ]はいい 友達[ともだち]です。		
\\	上[うえ]	上[うえ]を 向[む]いて。		
\\	駅[えき]	駅[えき]はどこですか。		
\\	美味[おい]しい	夏[なつ]はすいかが 美味[おい]しい。		
\\	昨日[きのう]	昨日[きのう]、 映画館[えいがかん]に 行[い]った。		
\\	綺麗[きれい](~な)	彼女[かのじょ]はとても 綺麗[きれい]だ。		
\\	五[ご]	五[ご] 人[にん]で 旅行[りょこう]に 行[い]きました。		
\\	九[ここの]つ	娘[むすめ]は 明日[あす] 九[ここの]つになります。		
\\	お 願[ねが]い	支払[しはら]いはカードでお 願[ねが]いします。		
\\	答[こた]える	私[わたし]の 質問[しつもん]に 答[こた]えてください。		
\\	先[さき]	ここから 先[さき]は 入[はい]らないでください。		
\\	寒[さむ]い	この 部屋[へや]は 寒[さむ]いです。		
\\	四[し]	四[し] 月[がつ]に 大学[だいがく]に 入学[にゅうがく]しました。		
\\	三日[みっか]	手紙[てがみ] が 届[とど]くのに 三日[みっか]かかりました。		
\\	下[した]	財布[さいふ]は 机[つくえ]の 下[した]にあった。		
\\	大丈夫[だいじょうぶ]	大丈夫[だいじょうぶ]ですか。		
\\	大人[おとな]	お 酒[さけ]は 大人[おとな]になってから。		
\\	出[だ]す	彼[かれ]はかばんから 教科書[きょうかしょ]を 出[だ]した。		
\\	父[ちち]	私[わたし]は 父[ちち]が 大好[だいす]きです。		
\\	母[はは]	昨日[きのう]、あたしは 母[はは]と 話[はなし]をしました。		
\\	月[つき]	今夜[こんや]は 月[つき]がとてもきれいです。		
\\	妹[いもうと]	私[わたし]の 妹[いもうと]は 小学生[しょうがくせい]です。		
\\	冷[つめ]たい	冷[つめ]たい 飲[の]み 物[もの]をください。		
\\	弟[おとうと]	弟[おとうと]は 野球[やきゅう]が 好[す]きです。		
\\	手[て]	ハンカチで 手[て]をふきました。		
\\	十日[とおか]	十日[とおか] 後[ご]に 帰[かえ]ります。		
\\	口[くち]	口[くち]を 大[おお]きく 開[あ]けてください。		
\\	夏[なつ]	夏[なつ]はビールがとても 美味[おい]しい。		
\\	七[なな]つ	この 子[こ]は 今年[ことし] 七[なな]つになります。		
\\	時々[ときどき]	彼[かれ]は 時々[ときどき] 遅刻[ちこく]します。		
\\	何[なに]	晩[ばん]ご 飯[はん]のおかずは 何[なに]?		
\\	人[ひと]	プールは 人[ひと]でいっぱいです。		
\\	一人[ひとり]	今日[きょう]は 一人[ひとり]の 時間[じかん]を 楽[たの]しみたい。		
\\	一日[ついたち]	来月[らいげつ]の 一日[ついたち]は 空[あ]いていますか。		
\\	九日[ここのか]	九日[ここのか]に 荷物[にもつ]が 届[とど]きます。		
\\	方[ほう]	彼[かれ]は 私[わたし]の 方[ほう]を 見[み]ました。		
\\	他[ほか]	他[ほか]に 方法[ほうほう]がありません。		
\\	僕
\\	[ぼく]	僕[ぼく]は 新[あたら]しい 靴[くつ]が 欲[ほ]しいです。		
\\	欲[ほ]しい	何[なに]か 飲[の]み 物[もの]が 欲[ほ]しいな。		
\\	万[まん]	値段[ねだん]はちょうど3 万[まん] 円[えん]です。		
\\	見[み]える	遠[とお]くに 船[ふね]が 見[み]えます。		
\\	道[みち]	この 道[みち]は 車[くるま]が 多[おお]い。		
\\	五[いつ]つ	桃[もも]を 五[いつ]つください。		
\\	目[め]	彼女[かのじょ]は 青[あお]い 目[め]をしています。		
\\	八[やっ]つ	あの 家[いえ]には 時計[とけい]が 八[やっ]つあります。		
\\	止[や]める	話[はな]すのを 止[や]めてください。		
\\	四日[よっか]	新学期[しんがっき]は 来月[らいげつ]の 四日[よっか]からです。		
\\	夜[よる]	昨日[きのう]の 夜[よる]は 家[いえ]にいました。		
\\	来年[らいねん]	来年[らいねん] 一緒[いっしょ]に 旅行[りょこう]しましょう。		
\\	六[ろく]	彼[かれ]には 子供[こども]が 六[ろく] 人[にん]います。		
\\	悪[わる]い	彼[かれ]は 悪[わる]い 人[ひと]ではない。		
\\	お 手洗[てあら]い	お 手洗[てあら]いはどこですか。		
\\	ご 主人[しゅじん]	彼女[かのじょ]のご 主人[しゅじん]はアメリカ 人[じん]です。		
\\	本当[ほんとう]に	あなたが 本当[ほんとう]に 好[す]きです。		
\\	自分[じぶん]	彼[かれ]は 自分[じぶん]が 大好[だいす]きだ。		
\\	ため	これは 医者[いしゃ]のためのサイトです。		
\\	見[み]つかる	パスポートが 見[み]つからない。		
\\	休[やす]む	明日[あした]、 会社[かいしゃ]を 休[やす]みます。		
\\	ゆっくり	ゆっくり 行[い]きましょう。		
\\	六[むっ]つ	息子[むすこ]は 六[むっ]つになりました。		
\\	花[はな]	あたしは 母[はは]に 花[はな]を 贈[おく]った。		
\\	動[うご]く	お 腹[なか]が 痛[いた]くて 動[うご]けません。		
\\	線[せん]	赤[あか]い 線[せん]を2 本[ほん] 引[ひ]いてください。		
\\	七日[なのか]	先月[せんげつ]の 七日[なのか]に 孫[まご]が 生[う]まれました。		
\\	以外[いがい]	私[わたし]は 雨[あめ]の 日[ひ] 以外[いがい]は 自転車[じてんしゃ]で 学校[がっこう]に 行[い]く。		
\\	男[おとこ]	男[おとこ]の 人[ひと]が 私[わたし]たちに 話[はな]しかけた。		
\\	彼[かれ]	彼[かれ]の 父親[ちちおや]は 先生[せんせい]です。		
\\	女[おんな]	店員[てんいん]は 若[わか]い 女[おんな]の 人[ひと]でした。		
\\	妻[つま]	今日[きょう]は 妻[つま]の 誕生日[たんじょうび]だ。		
\\	百[ひゃく]	私[わたし]の 祖母[そぼ]は 百[ひゃく] 才[さい]です。		
\\	辺[へん]	この 辺[へん]は 田[た]んぼがたくさんあります。		
\\	店[みせ]	私[わたし]はこの 店[みせ]によく 来[き]ます。		
\\	閉[し]まる	お 店[みせ]はもう 閉[し]まっていました。		
\\	問題[もんだい]	この 問題[もんだい]はかなり 易[やさ]しいです。		
\\	必要[ひつよう](~な)	新[あたら]しい 歯[は]ブラシが 必要[ひつよう]だ。		
\\	もつ	この 車[くるま]はよくもっている。		
\\	開[あ]く	電車[でんしゃ]のドアが 開[あ]きました。		
\\	昨年[さくねん]	私[わたし]たちは 昨年[さくねん]、 富士山[ふじさん]に 登[のぼ]りました。		
\\	治[なお]る	けがはもう 治[なお]りましたか。		
\\	ドル	この 服[ふく]は300ドルしました。		
\\	システム	全[すべ]てのシステムが 停止[ていし]した。		
\\	以上[いじょう]	飛行機[ひこうき]が1 時間[じかん] 以上[いじょう] 遅[おく]れた。		
\\	最近[さいきん]	最近[さいきん]、 仕事[しごと]がかなりハードです。		
\\	世界[せかい]	エベレストは 世界[せかい]でいちばん 高[たか]い 山[やま]です。		
\\	コンピューター	新[あたら]しいコンピューターを 買[か]った。		
\\	やる
\\	あたしは 犬[いぬ]にえさをやった。		
\\	意味[いみ]	それはどういう 意味[いみ]ですか。		
\\	増[ふ]える	この 町[まち]は 人口[じんこう]が 増[ふ]えた。		
\\	選[えら]ぶ	良[よ]い 家[いえ]を 選[えら]ぶのは 難[むずか]しい。		
\\	生活[せいかつ]	日本[にほん]での 生活[せいかつ]は 楽[たの]しいです。		
\\	進[すす]める	早[はや]く 授業[じゅぎょう]を 進[すす]めましょう。		
\\	続[つづ]ける	仕事[しごと]を 続[つづ]けてください。		
\\	ほとんど	お 金[かね]がほとんどありません。		
\\	会社[かいしゃ]	彼[かれ]は 小[ちい]さな 会社[かいしゃ]に 勤[つと]めています。		
\\	家[いえ]	ここが 私[わたし]の 家[いえ]です。		
\\	多[おお]く	毎年[まいとし] 多[おお]くの 人[ひと]が 海外[かいがい]へ 旅行[りょこう]する。		
\\	話[はなし]	その 話[はなし]は 本当[ほんとう]ですか。		
\\	上[あ]がる 
\\	私[わたし]たちは2 階[かい]に 上[あ]がった。		
\\	もう
\\	コーヒーをもう 一杯[いっぱい]ください。		
\\	集[あつ]める	弟[おとうと]は 切手[きって]を 集[あつ]めています。		
\\	声[こえ]	彼[かれ]は 大[おお]きな 声[こえ]で 話[はな]した。		
\\	初[はじ]めて	彼女[かのじょ]は 寿司[すし]を 初[はじ]めて 食[た]べました。		
\\	変[か]わる	信号[しんごう]が 青[あお]に 変[か]わりました。		
\\	まず	帰[かえ]ったらまず 手[て]を 洗[あら]いましょう。		
\\	社会[しゃかい]	これは 大[おお]きな 社会[しゃかい] 問題[もんだい]になっている。		
\\	プログラム	受付[うけつけ]でプログラムを 受[う]け 取[と]った。		
\\	力[ちから]	私[わたし]にそんな 力[ちから]はありません。		
\\	今回[こんかい]	今回[こんかい]はビジネスクラスに 乗[の]りました。		
\\	予定[よてい]	今日[きょう]の 予定[よてい]を 教[おし]えてください。		
\\	まま	そのままのあなたでいてください。		
\\	テレビ	テレビのチャンネルを 変[か]えてください。		
\\	減[へ]る	彼女[かのじょ]は 体重[たいじゅう]がかなり 減[へ]りました。		
\\	消[き]える	突然[とつぜん]、 電気[でんき]が 消[き]えた。		
\\	家族[かぞく]	私[わたし]は 家族[かぞく]を 愛[あい]しています。		
\\	比[くら]べる	あたしは 今月[こんげつ]と 先月[せんげつ]の 売上[うりあげ]を 比[くら]べた。		
\\	生[う]まれる	祖母[そぼ]は 京都[きょうと]で 生[う]まれました。		
\\	ただ	今日[きょう]、ただでノートをもらった。		
\\	これら	今日[きょう]はこれらの 問題[もんだい]について 話[はな]し 合[あ]います。		
\\	調[しら]べる	この 単語[たんご]の 意味[いみ]を 辞書[じしょ]で 調[しら]べましょう。		
\\	事故[じこ]	彼[かれ]は 事故[じこ]で 怪我[けが]をしました。		
\\	電話[でんわ](~する)	あとで 電話[でんわ]します。		
\\	外国[がいこく]	私[わたし]は 外国[がいこく]に 住[す]みたいです。		
\\	銀行[ぎんこう]	銀行[ぎんこう]は3 時[じ]まで 開[あ]いています。		
\\	十分
\\	[じゅうぶん](~な)	お 金[かね]はまだ 十分[じゅうぶん]あります。		
\\	あまり	このビールはあまり 美味[おい]しくありません。		
\\	写真[しゃしん]	写真[しゃしん]は 良[い]い 思[おも]い 出[で]になります。		
\\	繰[く]り 返[え]す	彼女[かのじょ]は 同[おな]じ 間違[まちが]いを 繰[く]り 返[え]した。		
\\	種類[しゅるい]	バラにはいろいろな 種類[しゅるい]があります。		
\\	意見
\\	[いけん]	あなたの 意見[いけん]が 聞[き]きたいです。		
\\	新聞[しんぶん]	私[わたし]は 毎朝[まいあさ]、 電車[でんしゃ]で 新聞[しんぶん]を 読[よ]みます。		
\\	文章[ぶんしょう]	この 文章[ぶんしょう]の 意味[いみ]が 分[わ]かりません。		
\\	目立[めだ]つ	彼女[かのじょ]の 大[おお]きな 帽子[ぼうし]はとても 目立[めだ]つ。		
\\	相手[あいて]	試合[しあい]の 相手[あいて]は 誰[だれ]ですか。		
\\	病院[びょういん]	病院[びょういん]はどこですか。		
\\	厚[あつ]い	私[わたし]はあのお 店[みせ]の 厚[あつ]いステーキが 大好[だいす]きだ。		
\\	忙[いそが]しい	あたしは3 月[がつ]は 仕事[しごと]が 忙[いそが]しい。		
\\	薄[うす]い	この 電子[でんし] 辞書[じしょ]はとても 薄[うす]い。		
\\	川[かわ]	小[ちい]さな 川[かわ]を 渡[わた]りました。		
\\	暗[くら]い	東[ひがし]の 空[そら]が 暗[くら]いです。		
\\	クラス	この 学校[がっこう]は1クラス30 人[にん]です。		
\\	黒[くろ]い	彼女[かのじょ]は 黒[くろ]いドレスを 着[き]ています。		
\\	バス	バスで 行[い]こう。		
\\	青[あお]い	ここの 海[うみ]はとても 青[あお]い。		
\\	買[か]い 物[もの]	買[か]い 物[もの]をすっかり 忘[わす]れていた。		
\\	薬[くすり]	この 薬[くすり]を 必[かなら]ず 飲[の]んでください。		
\\	砂糖[さとう]	コーヒーに 砂糖[さとう]は 入[い]れますか。		
\\	休[やす]み	今日[きょう]は 休[やす]みです。		
\\	郵便局[ゆうびんきょく]	郵便局[ゆうびんきょく]はどこですか。		
\\	住所[じゅうしょ]	ここに 住所[じゅうしょ]と 氏名[しめい]を 書[か]いてください。		
\\	こちら	こちらへどうぞ。		
\\	財布[さいふ]	会社[かいしゃ]に 財布[さいふ]を 忘[わす]れた。		
\\	パスポート	私[わたし]のパスポートがありません。		
\\	椅子[いす]	その 椅子[いす]を 持[も]ってきてください。		
\\	可愛[かわい]い	彼女[かのじょ]の 赤[あか]ちゃんは 可愛[かわい]いです。		
\\	お 祖父[じい]さん	私[わたし]のお 祖父[じい]さんは 毎日[まいにち] 散歩[さんぽ]します。		
\\	切手[きって]	弟[おとうと]は 切手[きって]を 集[あつ]めています。		
\\	涼[すず]しい	今日[きょう]はすごく 涼[すず]しいですね。		
\\	いくつ	日本語[にほんご]の 単語[たんご]をいくつ 知[し]っていますか。		
\\	メニュー	キッズメニューはありますか。		
\\	電気[でんき]	部屋[へや]の 電気[でんき]が 付[つ]いています。		
\\	勝[か]つ	誰[だれ]も 彼[かれ]には 勝[か]てません。		
\\	負[ま]ける	私[わたし]たちのチームはその 試合[しあい]で 負[ま]けた。		
\\	建[た]てる	私[わたし]たちは 来年[らいねん]、 家[いえ]を 建[た]てます。		
\\	日記[にっき]	姉[あね]は 毎晩[まいばん] 日記[にっき]を 書[か]いています。		
\\	売[う]り 切[き]れ	チケットはもう 売[う]り 切[き]れだって。		
\\	お 巡[まわ]りさん	あのお 巡[まわ]りさんに 道[みち]を 聞[き]きましょう。		
\\	目覚[めざ]まし 時計[どけい]	
\\	時[じ]に 目覚[めざ]まし 時計[どけい]が 鳴[な]りました。		
\\	レシート	レシートを 財布[さいふ]に 入[い]れました。		
\\	ティッシュ	ティッシュを 取[と]ってください。		
\\	歯[は]ブラシ	新[あたら]しい 歯[は]ブラシが 必要[ひつよう]だ。		
\\	下[お]りる	そこの 階段[かいだん]を 下[お]りてください。		
\\	洗[あら]う	石鹸[せっけん]で 手[て]を 洗[あら]ってください。		
\\	パート	彼女[かのじょ]はパートで働[はたら]いています。		
\\	氏名[しめい]	ここに 住所[じゅうしょ]と 氏名[しめい]を 書[か]いてください。		
\\	今夜[こんや]	今夜[こんや]は 月[つき]がとてもきれいです。		
\\	夜中[よなか]	夜中[よなか]に 電話[でんわ]がありました。		
\\	来週[らいしゅう]	私[わたし]は 来週[らいしゅう]、すごく 忙[いそが]しいです。		
\\	誰[だれ]か	誰[だれ]かに 聞[き]いてみてください。		
\\	何[なん]	質問[しつもん]は 何[なん]ですか。		
\\	今朝[けさ]	あたしは 今朝[けさ]は 早[はや]く 家[いえ]を 出[で]ました。		
\\	寿司[すし]	あたしは 昼食[ちゅうしょく]に 寿司[すし]を 食[た]べました。		
\\	履[は]く	彼女[かのじょ]はブーツを 履[は]いています。		
\\	おじさん	おじさんがクリスマスプレゼントをくれました。		
\\	おばさん	おばさんと 一家[いっか]が 遊[あそ]びに 来[き]ました。		
\\	お 祖母[ばあ]さん	彼女[かのじょ]はお 祖母[ばあ]さんと 住[す]んでいる。		
\\	いとこ	いとこと 私[わたし]は 同[おな]じ 年[とし]です。		
\\	辞書[じしょ]	この 単語[たんご]の 意味[いみ]を 辞書[じしょ]で 調[しら]べましょう。		
\\	朝[あさ]ご 飯[はん]	今[いま]、 朝[あさ]ご 飯[はん]を 作[つく]っています。		
\\	白[しろ]	白[しろ]は 雪[ゆき]の 色[いろ]です。		
\\	どっち	どっちの 色[いろ]が 好[す]きですか。		
\\	そっち	そっちが 私[わたし]の 部屋[へや]です。		
\\	明日[あした]	明日[あした]は 理科[りか]のテストです。		
\\	明後日[あさって]	明後日[あさって]は 休日[きゅうじつ]です。		
\\	一昨日[おととい]	一昨日[おととい] 彼[かれ]から 電話[でんわ]がありました。		
\\	庭[にわ]	庭[にわ]に 白[しろ]いネコがいます。		
\\	左側[ひだりがわ]	画面[がめん]の 左側[ひだりがわ]を 見[み]てください。		
\\	右側[みぎがわ]	彼女[かのじょ]はいつも 私[わたし]の 右側[みぎがわ]を 歩[ある]きます。		
\\	指[ゆび]	彼[かれ]は 指[ゆび]が 太[ふと]い。		
\\	眼鏡[めがね]	彼[かれ]は 眼鏡[めがね]をかけています。		
\\	鞄[かばん]	その 黒[くろ]い 鞄[かばん]は 僕[ぼく]のです。		
\\	あっち	郵便局[ゆうびんきょく]はあっちです。		
\\	大人[おとな]しい	私[わたし]の 彼女[かのじょ]はとても 大人[おとな]しいです。		
\\	下手[へた](~な)	私[わたし]は 歌[うた]が 下手[へた]だ。		
\\	厳[きび]しい	私[わたし]の 上司[じょうし]はとても 厳[きび]しい。		
\\	一人[ひとり]で	今日[きょう]は 一人[ひとり]で 映画[えいが]を 見[み]ます。		
\\	答[こた]え	彼[かれ]はその 問題[もんだい]の 答[こた]えが 分[わ]からない。		
\\	この 頃[ごろ]	この 頃[ごろ]、よく 肩[かた]が 凝[こ]る。		
\\	残念[ざんねん](~な)	その 試合[しあい]は 残念[ざんねん]な 結果[けっか]になった。		
\\	仕舞[しま]う	財布[さいふ]をポケットに 仕舞[しま]った。		
\\	心配[しんぱい](~な)	明日[あす]のプレゼンテーションが 心配[しんぱい]だ。		
\\	外[そと]	外[そと]は 暑[あつ]いよ。		
\\	大切[たいせつ](~な)	これは 母[はは]が 大切[たいせつ]にしていた 指輪[ゆびわ]です。		
\\	ちょうど	値段[ねだん]はちょうど3 万[まん] 円[えん]です。		
\\	助[たす]ける	彼女[かのじょ]は 病気[びょうき]の 犬[いぬ]を 助[たす]けた。		
\\	勤[つと]める	私[わたし]は 銀行[ぎんこう]に 勤[つと]めています。		
\\	連[つ]れていく	私[わたし]も 連[つ]れていってください。		
\\	丈夫[じょうぶ](~な)	祖母[そぼ]は 体[からだ]が 丈夫[じょうぶ]だ。		
\\	賑[にぎ]やか(~な)	ここはいつも 賑[にぎ]やかです。		
\\	眠[ねむ]い	今日[きょう]はとても 眠[ねむ]いです。		
\\	山[やま]	山[やま]の 空気[くうき]はきれいだ。		
\\	橋[はし]	あの 橋[はし]は 日本[にほん]で 一番[いちばん] 長[なが]い。		
\\	止[と]まる	今朝[けさ]、 事故[じこ]で 電車[でんしゃ]が 止[と]まりました。		
\\	降[ふ]る	明日[あす]は 雨[あめ]が 降[ふ]るでしょう。		
\\	本当[ほんとう]	その 話[はなし]は 本当[ほんとう]ですか。		
\\	町[まち]	ここは 小[ちい]さな 町[まち]です。		
\\	お 菓子[かし]	あたしはお 母[かあ]さんと 一緒[いっしょ]にお 菓子[かし]を 作[つく]りました。		
\\	緩[ゆる]い	この 靴[くつ]は 少[すこ]し 緩[ゆる]いです。		
\\	良[よ]い	彼[かれ]は 良[よ]い 人[ひと]です。		
\\	ようこそ	日本[にっぽん]へようこそ。		
\\	お 土産[みやげ]	友達[ともだち]にイギリスのお 土産[みやげ]をもらいました。		
\\	両親[りょうしん]	私[わたし]の 両親[りょうしん]は 大阪[おおさか]に 住[す]んでいます。		
\\	ウェーター	ウェーターが 水[みず]を 運[はこ]んできました。		
\\	ウェートレス	ウェートレスを 呼[よ]んだ。		
\\	絶対[ぜったい]に	絶対[ぜったい]に 駄目[だめ]です。		
\\	ごちそう	今日[きょう]は 妻[つま]がごちそうを 作[つく]ってくれた。		
\\	フォーク	ケーキを 食[た]べるのにフォークが 要[い]ります。		
\\	スプーン	赤[あか]ちゃんがスプーンでご 飯[はん]を 食[た]べている。		
\\	瓶[びん]	瓶[びん]ビールを 注文[ちゅうもん]した。		
\\	付[つ]く	部屋[へや]の 電気[でんき]が 付[つ]いています。		
\\	醤油[しょうゆ]	もう 少[すこ]し 醤油[しょうゆ]を 足[た]してください。		
\\	茶碗[ちゃわん]	私[わたし]は 毎晩[まいばん]、 茶碗[ちゃわん]に2 杯[はい]ご 飯[はん]を 食[た]べます。		
\\	決[き]める	あたしは 帰国[きこく]することに 決[き]めました。		
\\	感[かん]じる	膝[ひざ]に 痛[いた]みを 感[かん]じます。		
\\	生[い]きる	このカメは30 年[ねん] 生[い]きている。		
\\	動[うご]かす	部屋[へや]の 家具[かぐ]を 動[うご]かしました。		
\\	壊[こわ]れる	会社[かいしゃ]のパソコンが 壊[こわ]れた。		
\\	復習[ふくしゅう](~する)	まず、 先週[せんしゅう]の 復習[ふくしゅう]から 始[はじ]めましょう。		
\\	眉[まゆ]	彼[かれ]は 眉[まゆ]が 濃[こ]いね。		
\\	客[きゃく]	その 店[みせ]は 若[わか]い 客[きゃく]が 多[おお]いです。		
\\	机[つくえ]	財布[さいふ]は 机[つくえ]の 下[した]にあった。		
\\	風呂[ふろ]	父[ちち]は 今[いま]お 風呂[ふろ]に 入[はい]っています。		
\\	湯[ゆ]	お 風呂[ふろ]のお 湯[ゆ]が 一杯[いっぱい]です。		
\\	ぬるい	風呂[ふろ]がぬるかった。		
\\	風邪薬[かぜぐすり]	この 風邪薬[かぜぐすり]を 飲[の]みなさい。		
\\	靴下[くつした]	靴下[くつした]に 穴[あな]が 開[あ]いている。		
\\	たばこ	たばこは 体[からだ]に 悪[わる]い。		
\\	アイスコーヒー	アイスコーヒーをください。		
\\	天[てん]ぷら	私[わたし]はえびの 天[てん]ぷらが 好[す]きです。		
\\	肉[にく]	肉[にく]と 魚[さかな]とどちらが 好[す]きですか。		
\\	昨夜[ゆうべ]	昨夜[ゆうべ]、 流[なが]れ 星[ぼし]を 見[み]ました。		
\\	流行[はや]る	今[いま]、 大[おお]きなサングラスが 流行[はや]っている。		
\\	連[つ]れてくる	息子[むすこ]が 友達[ともだち]を 連[つ]れてきました。		
\\	方[かた]	次[つぎ]の 方[かた]、どうぞ。		
\\	零[れい]	日本語[にほんご]のテストで 零[れい] 点[てん]を 取[と]った。		
\\	雲[くも]	あの 雲[くも]は 鳥[とり]みたいだ。		
\\	空[そら]	空[そら]の 写真[しゃしん]を 撮[と]るのが 好[す]きです。		
\\	人気[にんき]	このバンドはとても 人気[にんき]があるよ。		
\\	兄[にい]さん	兄[にい]さん、おめでとう。		
\\	姉[ねえ]さん	今日[きょう]は 姉[ねえ]さんが 朝
\\	あさごはんを 作[つく]った。		
\\	平成[へいせい]	平成[へいせい]は1989 年[ねん]1 月[がつ] 8日[ようか]に 始[はじ]まった。		
\\	毎月[まいつき]	私[わたし]は 毎月[まいつき] 貯金[ちょきん]をしています。		
\\	半日[はんにち]	今日[きょう]は 半日[はんにち]だけ 仕事[しごと]だ。
\\	半月[はんつき]	半月[はんつき] 前[まえ]に 日本[にほん]に 来[き]ました。		
\\	なるほど	なるほど、よく 分[わ]かりました。		
\\	つまり	つまり、あなたは 何[なに]も 知[し]らないのですね。		
\\	そのまま	これはそのまま 食[た]べられます。		
\\	はっきり	今日[きょう]は 山[やま]がはっきり 見[み]える。		
\\	大変[たいへん](~な)	大変[たいへん]なことが 起[お]こりました。		
\\	簡単[かんたん](~な)	この 料理[りょうり]はとても 簡単[かんたん]です。		
\\	似[に]ている	あたしは 母[はは]に 似[に]ています。		
\\	驚[おどろ]く	風[かぜ]の 強[つよ]さに 驚[おどろ]きました。		
\\	嫌[いや](~な)	彼[かれ]は 嫌[いや]な 男[おとこ]だ。		
\\	喧嘩[けんか](~する)	喧嘩[けんか]はやめて。		
\\	遅[おく]れる	今朝[けさ] 彼女[かのじょ]は 学校[がっこう]に 遅[おく]れました。		
\\	にんじん	あたしはスーパーマーケットでにんじんを 買[か]った。		
\\	ジャガイモ	ポテトサラダを 作[つく]るから、ジャガイモを 買[か]ってきて。		
\\	ナス	夕飯[ゆうはん]にナスの 天[てん]ぷらを 食[た]べました。		
\\	やかん	やかんに 水[みず]を 入[い]れてください。		
\\	話[はな]し 合[あ]う	私[わたし]たちはよく 話[はな]し 合[あ]いました。		
\\	残[のこ]す	彼女[かのじょ]はメッセージを 残[のこ]しました。		
\\	ごちそうする	友達[ともだち]が 昼[ひる]ご 飯[はん]をごちそうしてくれた。		
\\	合[あ]う	この 靴[くつ]は 私[わたし]の 足[あし]に 合[あ]っている。		
\\	当[あ]たる	ボールが 彼[かれ]の 頭[あたま]に 当[あ]たった。		
\\	集[あつ]まる	駅[えき] 前[まえ]に 人[ひと]が 集[あつ]まっています。		
\\	場所[ばしょ]	会社[かいしゃ]の 場所[ばしょ]を 教[おし]えてください。		
\\	海[うみ]	海[うみ]は 広[ひろ]くて 大[おお]きい。		
\\	少年[しょうねん]	私[わたし]はあの 少年[しょうねん]たちを 助[たす]けたい。		
\\	孫[まご]	彼[かれ]には 孫[まご]が8 人[にん]いる。		
\\	生徒[せいと]	このクラスの 生徒[せいと]は30 人[にん]です。		
\\	高校[こうこう]	彼[かれ]は 高校[こうこう] 教師[きょうし]だ。		
\\	年上[としうえ]	彼[かれ]は 年上[としうえ]の 女[おんな]の 人[ひと]が 好[す]きだ。		
\\	卒業[そつぎょう](~する)	姉[あね]は 大学[だいがく]を 卒業[そつぎょう]して、アメリカで 働[はたら]いています。		
\\	運動[うんどう](~する)	私[わたし]はもっと 運動[うんどう]が 必要[ひつよう]だ。		
\\	選手[せんしゅ]	彼[かれ]らはバスケットの 選手[せんしゅ]です。		
\\	映画[えいが]	彼[かれ]はよく 映画[えいが]を 見[み]ます。		
\\	英語[えいご]	あなたは 英語[えいご]が 話[はな]せますか。		
\\	手紙[てがみ]	あたしは 友人[ゆうじん]から 手紙[てがみ]をもらいました。		
\\	動物[どうぶつ]	私[わたし]は 動物[どうぶつ]が 大好[だいす]きです。		
\\	音[おと]	雨[あめ]の 音[おと]が 聞[き]こえる。		
\\	海外[かいがい]	私[わたし]は 毎年[まいとし]、 海外[かいがい] 旅行[りょこう]に 行[い]きます。		
\\	外国人[がいこくじん]	日本[にほん]に 住[す]む 外国人[がいこくじん]が 増[ふ]えています。		
\\	帰国[きこく](~する)	彼[かれ]は 帰国[きこく]しました。		
\\	彼[かれ]ら	彼[かれ]らはよくテニスをしています。		
\\	機械[きかい]	この 機械[きかい]は 何[なに]に 使[つか]うんですか。		
\\	基本[きほん]	彼[かれ]はサッカーの 基本[きほん]を 知[し]らない。		
\\	今度[こんど]	今度[こんど]はどこに 行[い]きたいですか。		
\\	最後[さいご]	今日[きょう]が 夏休[なつやす]み 最後[さいご]の 日[ひ]だ。		
\\	最初[さいしょ]	5ページの 最初[さいしょ]を 見[み]てください。		
\\	準備[じゅんび]	明日[あす]の 会議[かいぎ]の 準備[じゅんび]をした。		
\\	進[すす]む	会議[かいぎ]の 準備[じゅんび]は 進[すす]んでいますか。		
\\	直接[ちょくせつ]	彼[かれ]に 直接[ちょくせつ]お 願[ねが]いしなさい。		
\\	特[とく]に	私[わたし]は 特[とく]にチェコのビールが 好[す]きです。		
\\	届[とど]く	昨日[きのう]、 父[ちち]から 手紙[てがみ]が 届[とど]いた。		
\\	なぜ	なぜ 昨日[きのう]、 学校[がっこう]を 休[やす]んだんですか。		
\\	並[なら]ぶ	ここに 並[なら]んでください。		
\\	運[はこ]ぶ	いすを2 階[かい]に 運[はこ]んでください。		
\\	直[なお]す	あたしがそれを 直[なお]しました。		
\\	反対[はんたい](~する)	私[わたし]は 反対[はんたい]です。		
\\	場合[ばあい]	雨[あめ]の 場合[ばあい]は、 教室[きょうしつ]で 授業[じゅぎょう]をします。		
\\	詳[くわ]しい	もっと 詳[くわ]しく 説明[せつめい]してください。		
\\	いたずら(~する)	その 子[こ]はいたずらが 大好[だいす]きだ。		
\\	お 祝[いわ]い	彼[かれ]の 就職[しゅうしょく]のお 祝[いわ]いをしよう。		
\\	くし	彼[かれ]はくしで 髪[かみ]をとかした。		
\\	こぼれる	彼女[かのじょ]の 目[め]から 涙[なみだ]がこぼれた。		
\\	伝[つた]える	みんなにこのことを 伝[つた]えてください。		
\\	膝[ひざ]	スキーで 膝[ひざ]を 怪我[けが]しました。		
\\	肘[ひじ]	彼[かれ]は 肘[ひじ]に 怪我[けが]をした。		
\\	枕[まくら]	私[わたし]は 低[ひく]い 枕[まくら]が 好[す]きです。		
\\	建物[たてもの]	あの 大[おお]きい 建物[たてもの]は 何[なん]ですか。		
\\	道路[どうろ]	道路[どうろ]で 遊[あそ]ばないでください。		
\\	四[よ]つ 角[かど]	あそこの 四[よ]つ 角[かど]を 左[ひだり]に 曲[ま]がってください。		
\\	曲[ま]がり 角[かど]	ポストはそこの 曲[ま]がり 角[かど]にあります。		
\\	警察[けいさつ]	警察[けいさつ]を 呼[よ]んでください。		
\\	空気[くうき]	ここは 空気[くうき]がきれいです。		
\\	スポーツ	私[わたし]はスポーツは 何[なに]もしません。		
\\	チャンス	これは 素晴[すば]らしいチャンスだ。		
\\	クリーニング	スーツをクリーニングに 出[だ]しました。		
\\	サービス	この 店[みせ]はサービスがいい。		
\\	グループ	店[みせ]に 学生[がくせい]のグループが 来[き]た。		
\\	自宅[じたく]	ご 自宅[じたく]はどちらですか。		
\\	家庭[かてい]	彼[かれ]は 家庭[かてい]を 大切[たいせつ]にしている。		
\\	期間[きかん]	テスト 期間[きかん]は 10日[とおか]から15 日[にち]までだ。		
\\	年度[ねんど]	明日[あした]から 新[しん] 年度[ねんど]が 始[はじ]まります。		
\\	経験[けいけん]	今日[きょう]の 試合[しあい]はいい 経験[けいけん]になりました。		
\\	安全[あんぜん]	安全[あんぜん]が 第一[だいいち]です。		
\\	危険[きけん](~な)	その 地域[ちいき]は 今[いま]、 危険[きけん]だ。		
\\	注意[ちゅうい](~する)	車[くるま]に 注意[ちゅうい]してください。		
\\	成功[せいこう](~する)	ついに 実験[じっけん]が 成功[せいこう]した。		
\\	努力[どりょく](~する)	あなたはもっと 努力[どりょく]が 必要[ひつよう]です。		
\\	説明[せつめい](~する)	この 単語[たんご]の 意味[いみ]を 説明[せつめい]してください。		
\\	地震[じしん]	日本[にほん]は 地震[じしん]が 多[おお]いです。		
\\	手術[しゅじゅつ](~する)	父[ちち]は 胸[むね]の 手術[しゅじゅつ]をした。		
\\	火傷[やけど](~する)	彼[かれ]は 手[て]に 火傷[やけど]をしました。		
\\	課題[かだい]	夏休[なつやす]みの 課題[かだい]は 何[なん]ですか。		
\\	子[こ]	彼女[かのじょ]は2 才[さい]の 子[こ]の 母親[ははおや]です。		
\\	確認[かくにん](~する)	もう 一度[いちど]、 予約[よやく]を 確認[かくにん]した。		
\\	実際[じっさい]	彼[かれ]は 実際[じっさい]にはあまり 背[せ]が 高[たか]くない。		
\\	国際[こくさい]	ここで 国際[こくさい] 会議[かいぎ]が 開[ひら]かれます。		
\\	会議[かいぎ]	今日[きょう]の 午後[ごご]、 大事[だいじ]な 会議[かいぎ]があります。		
\\	提案[ていあん](~する)	そのアイデアは 彼[かれ]の 提案[ていあん]です。		
\\	事務所[じむしょ]	後[あと]で 事務所[じむしょ]に 来[き]てください。		
\\	教授[きょうじゅ]	彼[かれ]は 化学[かがく]の 教授[きょうじゅ]です。		
\\	世紀[せいき]	この 建物[たてもの]は17 世紀[せいき]に 建[た]てられました。		
\\	あちこち	私[わたし]たちは 朝[あさ]からあちこち 散歩[さんぽ]しました。		
\\	そちら	夕方[ゆうがた]そちらに 着[つ]きます。		
\\	あちら	あちらに 行[い]ってみよう。		
\\	もし	もし 雨[あめ]が 降[ふ]ったら、 行[い]きません。		
\\	うるさい	この 通[とお]りは 車[くるま]の 音[おと]がうるさい。		
\\	固[かた]い	私[わたし]の 上司[じょうし]は 頭[あたま]が 固[かた]い。		
\\	深[ふか]い	あの 池[いけ]はとても 深[ふか]い。		
\\	面白[おもしろ]い	この 本[ほん]はすごく 面白[おもしろ]かった。		
\\	全[まった]く	私[わたし]には 全[まった]く 分[わ]かりません。		
\\	半分[はんぶん]	あたしはお 菓子[かし]を 友達[ともだち]に 半分[はんぶん]あげた。		
\\	普通[ふつう]	彼女[かのじょ]は 普通[ふつう]の 女[おんな]の 子[こ]だ。		
\\	分[ぶん]	このケーキはあなたの 分[ぶん]です。		
\\	文化[ぶんか]	私[わたし]は 日本[にほん]の 文化[ぶんか]をもっと 知[し]りたいです。		
\\	毎日[まいにち]	私[わたし]たちは 毎日[まいにち] 散歩[さんぽ]をします。		
\\	気[き]を 付[つ]ける	体[からだ]に 気[き]を 付[つ]けてください。		
\\	守[まも]る	彼[かれ]は 約束[やくそく]を 守[まも]る 人[ひと]です。		
\\	もちろん	もちろん 一緒[いっしょ]に 行[い]きます。		
\\	やはり	彼[かれ]はやはり 遅刻[ちこく]しました。		
\\	いくら	この 靴[くつ]はいくらですか。		
\\	よろしく	お 母[かあ]さんによろしくお 伝[つた]えください。		
\\	どなた	失礼[しつれい]ですが、どなたですか。		
\\	許[ゆる]す	あたしは 彼[かれ]を 許[ゆる]しました。		
\\	分[わ]ける	お 菓子[かし]をみんなで 分[わ]けました。		
\\	自然[しぜん]	みんなで 自然[しぜん]を 守[まも]りましょう。		
\\	アパート	彼[かれ]はアパートに 住[す]んでいます。		
\\	ホテル	今[いま]あたしはホテルに 着[つ]きました。		
\\	パソコン	彼[かれ]はパソコンを2 台[だい] 持[も]っています。		
\\	うまい	彼[かれ]は 野球[やきゅう]がうまい。		
\\	明[あか]るい	彼女[かのじょ]は 明[あか]るい 性格[せいかく]です。		
\\	急[いそ]ぐ	私[わたし]たちは 駅[えき]へ 急[いそ]ぎました。		
\\	歌[うた]	あたしはその 歌[うた]を 知[し]らなかった。		
\\	中学校[ちゅうがっこう]	息子[むすこ]の 中学校[ちゅうがっこう]は 家[いえ]から5 分[ふん]です。		
\\	テスト(~する)	明日[あした]は 理科[りか]のテストです。		
\\	ポスト	あたしは 手紙[てがみ]をポストに 入[い]れました。		
\\	ハンカチ	右[みぎ]のポケットにハンカチが 入[はい]っています。		
\\	髪[かみ]	昨日[きのう]あたしは 髪[かみ]を 切[き]りました。		
\\	帽子[ぼうし]	彼女[かのじょ]の 大[おお]きな 帽子[ぼうし]はとても 目立[めだ]つ。		
\\	被[かぶ]る	あたしは 帽子[ぼうし]を 被[かぶ]って 外出[がいしゅつ]した。		
\\	ブラウス	彼女[かのじょ]は 白[しろ]いブラウスを 着[き]ている。		
\\	週末[しゅうまつ]	週末[しゅうまつ]は 部屋[へや]の 掃除[そうじ]をしました。		
\\	先週[せんしゅう]	先週[せんしゅう]は 海[うみ]に 行[い]った。		
\\	再来週[さらいしゅう]	再来週[さらいしゅう]は 忙[いそが]しいです。		
\\	いつか	僕[ぼく]はいつかアフリカに 行[い]きたい。		
\\	宿題[しゅくだい]	宿題[しゅくだい]、 手伝[てつだ]ってくれない?		
\\	鍵[かぎ]	今日[きょう]、 鍵[かぎ]を 無[な]くなりました。		
\\	傘[かさ]	今日[きょう]は 傘[かさ]を 持[も]っていきなさい。		
\\	乗[の]り 換[か]える	次[つぎ]の 駅[えき]で 地下鉄[ちかてつ]に 乗[の]り 換[か]えます。		
\\	向[む]かう	今[いま]、 会社[かいしゃ]に 向[む]かっています。		
\\	本屋[ほんや]	駅[えき] 前[まえ]に 本屋[ほんや]があります。		
\\	お 茶[ちゃ]	私[わたし]たちは 毎日[まいにち]お 茶[ちゃ]を 飲[の]みます。		
\\	改札口[かいさつぐち]	改札口[かいさつぐち]で 会[あ]いましょう。		
\\	晴[は]れ	明日[あす]の 天気[てんき]は 晴[は]れです。		
\\	バス 停[てい]	次[つぎ]のバス 停[てい]で 降[お]ります。		
\\	曇[くも]り	今日[きょう]は 一日[いちにち] 曇[くも]りでした。		
\\	塩[しお]	塩[しお]を 取[と]ってください。		
\\	たくさん	昨日[きのう]はたくさん 泳[およ]ぎました。		
\\	大嫌[だいきら]い(~な)	私[わたし]の 息子[むすこ]は 勉強[べんきょう]が 大嫌[だいきら]いだ。		
\\	中[なか]	寒[さむい]いので 中[なか]に 入[はい]ってください。		
\\	二階[にかい]	私[わたし]のうちはアパートの 二階[にかい]です。		
\\	無[な]くす	学校[がっこう]で 財布[さいふ]を 無[な]くした。		
\\	まあまあ	彼[かれ]の 成績[せいせき]はまあまあです。		
\\	黄色[きいろ]	信号[しんごう]は 黄色[きいろ]でした。		
\\	ランチ
\\	一緒[いっしょ]にランチに 行[い]きませんか。		
\\	魚[さかな]	魚[さかな]がたくさん 泳[およ]いでいます。		
\\	味[あじ]	この 料理[りょうり]は 味[あじ]が 薄[うす]い。		
\\	りんご	りんごを1つください。		
\\	みかん	うちの 庭[にわ]にみかんの 木[き]があります。		
\\	皿[さら]	空[あ]いたお 皿[さら]をお 下[さ]げします。		
\\	コーヒー	コーヒーに 砂糖[さとう]は 入[い]れますか。		
\\	コップ	私[わたし]は 毎朝[まいあさ]コップ 一杯[いっぱい]の 水[みず]を 飲[の]む。		
\\	二人[ふたり]	今日[きょう]は 妻[つま]と 二人[ふたり]で 食事[しょくじ]をします。		
\\	止[や]む	雨[あめ]が 止[や]みました。		
\\	九[く]	今日[きょう]は 九[く] 時[じ]までに 帰[かえ]ります。		
\\	昼間[ひるま]	昼間[ひるま]は 仕事[しごと]で 忙[いそが]しいです。		
\\	いつ 頃[ごろ]	いつ 頃[ごろ] 夏休[なつやす]みを 取[と]りますか。		
\\	字[じ]	もっと 大[おお]きく 字[じ]を 書[か]いてください。		
\\	七[しち]	七[しち] 時[じ]に 朝[あさ]ご 飯[ごはん]を 食[た]べました。		
\\	お 釣[つ]り	母[はは]はお 釣[つ]りを 確[たし]かめた。		
\\	名字[みょうじ]	あなたの 名字[みょうじ]は 何[なん]ですか。		
\\	おじ	おじは 銀行[ぎんこう]に 勤[つと]めています。		
\\	おば	おばは 小[ちい]さな 会社[かいしゃ]の 社長[しゃちょう]です。		
\\	祖父[そふ]	祖父[そふ]は 元気[げんき]です。		
\\	大事[だいじ](~な)	今日[きょう]の 午後[ごご]、 大事[だいじ]な 会議[かいぎ]があります。		
\\	見方[みかた]	彼[かれ]に 対[たい]する 見方[みかた]が 変[か]わりました。		
\\	鳥[とり]	あそこに 綺麗[きれい]な 鳥[とり]がいます。		
\\	犬[いぬ]	隣[となり]の 家[いえ]には 犬[いぬ]がいます。		
\\	返事[へんじ](~する)	手紙[てがみ]の 返事[へんじ]を 出[だ]しました。		
\\	また	明日[あした]、また 来[き]ます。		
\\	年間[ねんかん]	年間[ねんかん]5 万[まん] 人[にん]がここを 訪[おとず]れます。		
\\	青[あお]	私[わたし]の 好[す]きな 色[いろ]は 青[あお]です。		
\\	赤[あか]	信号[しんごう]が 赤[あか]に 変[か]わりました。		
\\	信号[しんごう]	信号[しんごう]は 黄色[きいろ]でした。		
\\	円[まる]	そこに 大[おお]きな 円[まる]を 描[か]いて。		
\\	非常[ひじょう]に	これは 非常[ひじょう]に 重要[じゅうよう]です。		
\\	複雑[ふくざつ](~な)	このプログラムはとても 複雑[ふくざつ]です。		
\\	平和[へいわ]	この 国[くに]は 平和[へいわ]です。		
\\	回[まわ]る	月[つき]は 地球[ちきゅう]のまわりを 回[まわ]っています。		
\\	若者[わかもの]	最近[さいきん]の 若者[わかもの]は 本[ほん]を 読[よ]まない。		
\\	雪[ゆき]	クリスマスに 雪[ゆき]が 降[ふ]りました。		
\\	祖母[そぼ]	祖母[そぼ]は 京都[きょうと]で 生[う]まれました。		
\\	うまい
\\	あそこのハンバーガーはすごくうまい。		
\\	思[おも]い 出[だ]す	大切[たいせつ]な 用事[ようじ]を 思[おも]い 出[だ]しました。		
\\	聞[き]こえる	隣[となり]の 部屋[へや]からテレビの 音[おと]が 聞[き]こえる。		
\\	借[か]りる	あたしは 彼[かれ]にビデオを 借[か]りました。		
\\	返[かえ]す	図書館[としょかん]に 本[ほん]を 返[かえ]した。		
\\	受[う]け 取[と]る	あたしは 彼[かれ]からメールを 受[う]け 取[と]りました。		
\\	捨[す]てる	ゴミを 捨[す]ててください。		
\\	一緒[いっしょ]	あの 二人[ふたり]はいつも 一緒[いっしょ]です。		
\\	遊[あそ]び	お 正月[しょうがつ]にはいろいろな 遊[あそ]びをします。		
\\	移[うつ]す	机[つくえ]を 窓[まど]のそばに 移[うつ]しました。		
\\	大[おお]きさ	あなたのスーツケースの 大[おお]きさを 教[おし]えてください。		
\\	考[かんが]え	それは 良[い]い 考[かんが]えです。		
\\	空港[くうこう]	空港[くうこう]までリムジンバスで 行[い]った。		
\\	出発[しゅっぱつ]	明日[あした]はいよいよ 出発[しゅっぱつ]の 日[ひ]です。		
\\	地図[ちず]	私[わたし]は 地図[ちず]が 読[よ]めません。		
\\	運転[うんてん](~する)	父[ちち]は 安全[あんぜん] 運転[うんてん]だ。		
\\	降[お]りる	次[つぎ]のバス 停[てい]で 降[お]ります。		
\\	ガス	地震[じしん]でガスが 止[と]まった。		
\\	必[かなら]ず	この 薬[くすり]を 必[かなら]ず 飲[の]んでください。		
\\	カメラ	カメラが 壊[こわ]れた。		
\\	通[かよ]う	妹[いもうと]は 高校[こうこう]に 通[かよ]っています。		
\\	急[きゅう]に	あたしは 急[きゅう]に 用事[ようじ]を 思[おも]い 出[だ]した。		
\\	サラリーマン	私[わたし]の 夫[おっと]はサラリーマンです。		
\\	給料[きゅうりょう]	来年[らいねん]から 給料[きゅうりょう]が 上[あ]がります。		
\\	曲[きょく]	私[わたし]はこの 曲[きょく]が 大好[だいす]きです。		
\\	切[き]れる	このはさみはよく 切[き]れますね。		
\\	正[ただ]しい	それは 正[ただ]しい 答[こた]えです。		
\\	苦[くる]しい	食[た]べ 過[す]ぎておなかが 苦[くる]しい。		
\\	細[こま]かい	細[こま]かいことはあとで 決[き]めましょう。		
\\	静[しず]か(~な)	今年[ことし]の 夏休[なつやす]みには 静[しず]かな 所[ところ]へ 行[い]きたい。		
\\	健康[けんこう](~な)	健康
\\	が 一番[いちばん] 大切[たいせつ]だ。		
\\	ゴルフ	兄[あに]はゴルフを 始[はじ]めました。		
\\	コース	私[わたし]は 日本語[にほんご]コースを 取[と]っています。		
\\	頼[たの]む	今夜[こんや]はピザを 頼[たの]みましょう。		
\\	困[こま]る	あたしは 携帯電話[けいたいでんわ]をなくして 困[こま]っています。		
\\	ずっと	父[ちち]は 休[やす]みの 日[ひ]はずっとテレビを 見[み]ている。		
\\	例[たと]えば	例[たと]えば、このソフトで 日本語[にほんご]を 勉強[べんきょう]することができます。		
\\	つもり	明日[あした]からタバコを 止[や]めるつもりです。		
\\	しばらく	そこでしばらく 休[やす]んでいます。		
\\	紹介[しょうかい](~する)	両親[りょうしん]に 彼女[かのじょ]を 紹介[しょうかい]した。		
\\	小学校[しょうがっこう]	家[いえ]の 近[ち]くに 小学校[しょうがっこう]があります。		
\\	公園[こうえん]	公園[こうえん]に 小[ちい]さな 池[いけ]があります。		
\\	中学[ちゅうがく]	息子[むすこ]は 中学[ちゅうがく]に 通[かよ]っています。		
\\	成績[せいせき]	成績[せいせき]が 上[あ]がりました。		
\\	教科書[きょうかしょ]	あたしは 日本語[にほんご]の 教科書[きょうかしょ]を 忘[わす]れた。		
\\	席[せき]	この 席[せき]、 空[あ]いてますか。		
\\	教室[きょうしつ]	私[わたし]の 教室[きょうしつ]は3 階[がい]にあります。		
\\	教師[きょうし]	父[ちち]は 英語[えいご] 教師[きょうし]です。		
\\	試験[しけん]	彼[かれ]は 採用[さいよう] 試験[しけん]に 合格[ごうかく]した。		
\\	合格[ごうかく](~する)	娘[むすめ]が 入学[にゅうがく] 試験[しけん]に 合格[ごうかく]しました。		
\\	数学[すうがく]	兄[あに]は 数学[すうがく]の 先生[せんせい]です。		
\\	数字[すうじ]	数字[すうじ]は 苦手[にがて]です。		
\\	音楽[おんがく]	私[わたし]は 音楽[おんがく]を 聞[き]くのが 好[す]きだ。		
\\	食事[しょくじ](~する)	一緒[いっしょ]に 食事[しょくじ]しようか。		
\\	壁[かべ]	壁[かべ]に 大[おお]きな 時計[とけい]が 掛[か]かっています。		
\\	信[しん]じる	私[わたし]は 彼女[かのじょ]の 話[はなし]を 信[しん]じていません。		
\\	育[そだ]てる	彼女[かのじょ]は 三[さん] 人[にん]の 子[こ]を 育[そだ]てました。		
\\	倒[たお]れる	強風[きょうふう]で 木[き]が 倒[たお]れた。		
\\	落[お]とす	あたしは 途中[とちゅう]で 財布[さいふ]を 落[お]としました。		
\\	代[か]わる	彼[かれ]に 代[か]わる 人[ひと]はいません。		
\\	タクシー	駅[えき]からはタクシーに 乗[の]ってください。		
\\	確[た]か(~な)	彼[かれ]の 昇進[しょうしん]は 確[たし]かだ。		
\\	立[た]てる	彼[かれ]はケーキにろうそくを 立[た]てた。 は ケーキ に ろうそく を たてた		
\\	中学生[ちゅうがくせい]	息子[むすこ]は 中学生[ちゅうがくせい]です。		
\\	売[う]れる	この 雑誌[ざっし]はよ く売[う]れています。		
\\	着[つ]く	今[いま]あたしはホテルに 着[つ]きました。		
\\	決[き]まる	旅行[りょこう]の 日程[にってい]が 決[き]まりました。		
\\	飾[かざ]る	玄関[げんかん]に 花[はな]を 飾[かざ]りました。		
\\	殺[ころ]す	私[わたし]は 生[い]き 物[もの]を 殺[ころ]すのが 嫌[きら]いだ。		
\\	下[さ]げる	少[すこ]し 音量[おんりょう]を 下[さ]げてください。		
\\	贈[おく]る	クラスのみんなで 先生[せんせい]にプレゼントを 贈[おく]りました。		
\\	訪[たず]ねる	彼[かれ]はニューヨークの 友達[ともだち]を 訪[たず]ねた。		
\\	打[う]つ	転[ころ]んでひざを 打[う]ちました。		
\\	相談[そうだん](~する)	私[わたし]は 医者[いしゃ]に 相談[そうだん]した。		
\\	玄関[げんかん]	玄関[げんかん]に 花[はな]を 飾[かざ]りました。		
\\	兄弟[きょうだい]	彼[かれ]は3 人[にん] 兄弟[きょうだい]です。		
\\	長男[ちょうなん]	彼[かれ]の 長男[ちょうなん]は8 才[さい]です。		
\\	高[たか]さ	富士山[ふじさん]の 高[たか]さは3,776メートルです。		
\\	用[よう]	今日[きょう]はちょっと 用[よう]があります。		
\\	時代[じだい]	今[いま]は 便利[べんり]さとスピードの 時代[じだい]だ。		
\\	位置[いち](~する)	私[わたし]の 町[まち]は 東京[とうきょう]の 北[きた]に 位置[いち]します。		
\\	季節[きせつ]	私[わたし]の 一番[いちばん] 好[す]きな 季節[きせつ]は 春[はる]です。		
\\	穴[あな]	靴下[くつした]に 穴[あな]が 開[あ]いている。		
\\	裏[うら]	会社[かいしゃ]の 裏[うら]に 公園[こうえん]があります。		
\\	島[しま]	日本[にっぽん]は 島国[しまぐに]です。		
\\	海岸[かいがん]	海岸[かいがん]を 散歩[さんぽ]しましょう。		
\\	ガラス	ガラスの 花瓶[かびん]を 買[か]いました。		
\\	自然[しぜん](~な)
\\	親[おや]が 子供[こども]を 守[まも]るのは 自然[しぜんな]なことだ。		
\\	風[かぜ]	今日[きょう]は 風[かぜ]が 強[つよ]いです。		
\\	科学[かがく]	科学[かがく]は 常[つね]に 進歩[しんぽ]している。		
\\	太陽[たいよう]	太陽[たいよう]が 雲[くも]に 隠[かく]れた。		
\\	台風[たいふう]	台風[たいふう]が 近[ちか]づいている。		
\\	北[きた]	ロシアは 日本[にっぽん]の 北[きた]にあります。		
\\	馬[うま]	彼[かれ]は 牧場[ぼくじょう]で 馬[うま]に 乗[の]った。		
\\	牛肉[ぎゅうにく]	夕食[ゆうしょく]に 牛肉[ぎゅうにく]を 買[か]った。		
\\	雑誌[ざっし]	彼女[かのじょ]は 雑誌[ざっし]を 読[よ]んでいます。		
\\	小説[しょうせつ]	私[わたし]は 月[つき]に3 冊[さつ]くらい 小説[しょうせつ]を 読[よ]みます。		
\\	大使館[たいしかん]	彼[かれ]は 大使館[たいしかん]に 勤[つと]めています。		
\\	故障
\\	[こしょう](~する)	冷蔵庫[れいぞうこ]が 故障[こしょう]しました。		
\\	温度[おんど]	今[いま]、 部屋[へや]の 温度[おんど]は25 度[ど]だ。		
\\	何[なに]か	道[みち]に 何[なに]か 落[お]ちています。		
\\	向[む]こう	友達[ともだち]は 向[む]こうにいます。		
\\	真[ま]ん 中[なか]	道[みち]の 真[ま]ん 中[な]に 人[ひと]が 立[た]っている。		
\\	遠[とお]く	遠[とお]くに 船[ふね]が 見[み]えます。		
\\	横[よこ]	横[よこ]の 長[なが]さは1メートルです。		
\\	つまらない	彼[かれ]の 話[はなし]はつまらないですね。		
\\	素晴[すば]らしい	素晴[すば]らしい 景色[けしき]ですね。		
\\	毎年[まいとし]	私[わたし]は 毎年[まいとし]、 海外[かいがい] 旅行[りょこう]に 行[い]きます。		
\\	来月[らいげつ]	私[わたし]たちは 来月[らいげつ] 結婚[けっこん]します。		
\\	日時[にちじ]	会[かい]の 日時[にちじ]を 教[おし]えてください。		
\\	夕方[ゆうがた]	夕方[ゆうがた]、 雨[あめ]が 降[ふ]りました。		
\\	通[とお]る	毎日[まいにち]、この 道[みち]を 通[とお]ります。		
\\	自動車[じどうしゃ]	彼[かれ]は 自動車[じどうしゃ] 会社[がいしゃ]で 働[はたら]いています。		
\\	慣[な]れる	新[あたら]しい 家[いえ]にはもう 慣[な]れましたか。		
\\	撮[と]る	あたしは 写真[しゃしん]をたくさん 撮[と]りました。		
\\	やっと	やっと 母[はは]に 会[あ]えました。		
\\	どんどん 
\\	隣[となり]からどんどん 言[い]う 音[おと]が 聞[き]こえる。		
\\	並[なら]べる	私[わたし]は 料理[りょうり]をテーブルに 並[なら]べた。		
\\	逃[に]げる	公園[こうえん]で 犬[いぬ]が 逃[に]げてしまった。		
\\	渡[わた]す	彼[かれ]に 手紙[てがみ]を 渡[わた]しました。		
\\	値段[ねだん]	このベルトの 値段[ねだん]は2500 円[えん]でした。		
\\	両方[りょうほう]	チョコレートとケーキを 両方[りょうほう]ください。		
\\	約束[やくそく](~する)	約束[やくそく]を 忘[わす]れないでください。		
\\	一部[いちぶ]	日本[にほん]はアジアの 一部[いちぶ]です。		
\\	ラジオ	私[わたし]は 毎朝[まいあさ]ラジオを 聞[き]きます。		
\\	入院[にゅういん](~する)	昨日[きのう]、 母[はは]が 入院[にゅういん]しました。		
\\	ニュース	夜[よる]のニュースを 見[み]ましたか。		
\\	旅行[りょこう](~する)	毎年[まいとし] 多[おお]くの 人[ひと]が 海外[かいがい]へ 旅行[りょこう]する。		
\\	用意[ようい](~する)	早[はや]く 明日[あした]の 用意[ようい]をしなさい。		
\\	伸[の]びる	髪[かみ]がだいぶ 伸[の]びたね。		
\\	パーティー	明日[あした]、うちでパーティーを 開[ひら]きます。		
\\	ビール	仕事[しごと]の 帰[かえ]りにビールを 飲[の]んだ。		
\\	早[はや]く	あたしは 今朝[けさ]は 早[はや]く 家[いえ]を 出[で]ました。		
\\	番組[ばんぐみ]	私[わたし]はこの 番組[ばんぐみ]が 好[す]きです。		
\\	ビデオ	私[わたし]はその 番組[ばんぐみ]をビデオに 撮[と]った。		
\\	増[ふ]やす	その 病院[びょういん]はベッドの 数[かず]を 増[ふ]やしました。		
\\	振[ふ]る	犬[いぬ]がしっぽを 振[ふ]っている。		
\\	迎[むか]える	みんな、 私[わたし]を 温[あたた]かく 迎[むか]えてくれました。		
\\	無理[むり](~な)	無理[むり]はしないでください。		
\\	珍[めずら]しい	昨日[きのう] 珍[めずら]しい 果物[くだもの]を 食[た]べました。		
\\	有名[ゆうめい](~な)	彼[かれ]は 有名[ゆうめい]な 歌手[かしゅ]です。		
\\	喜[よろこ]ぶ	彼女[かのじょ]はとても 喜[よろこ]びました。		
\\	留学[りゅうがく](~する)	
\\	年間[ねんかん]、アメリカに 留学[りゅうがく]しました。		
\\	料理[りょうり](~する)	この 料理[りょうり]はとても 簡単[かんたん]です。		
\\	野菜[やさい]	私[わたし]は 毎日[まいにち]たくさん 野菜[やさい]を 食[た]べます。		
\\	分[わ]かれる	グループの 中[なか]で 意見[いけん]が 分[わ]かれました。		
\\	特別[とくべつ]	あなたは 私[わたし]にとって 特別[とくべつ]な 人[ひと]です。		
\\	理由[りゆう]	遅[おく]れた 理由[りゆう]を 教[おし]えてください。		
\\	自由[じゆう](~な)	今日[きょう]は 自由[じゆう]な 時間[じかん]が 多[おお]い。		
\\	方向[ほうこう]	あの 人[ひと]たちは 皆[みな]、 同[おな]じ 方向[ほうこう]を 見[み]ている。		
\\	残[のこ]る	料理[りょうり]がたくさん 残[のこ]りました。		
\\	ビル	私[わたし]の 会社[かいしゃ]はあのビルの8 階[かい]です。		
\\	まとめる	彼女[かのじょ]は 荷物[にもつ]をまとめて 出[で]て 行[い]った。		
\\	流[なが]れる	この 町[まち]には 大[おお]きな 川[かわ]が 流[なが]れています。		
\\	セーター	このセーターはウールだ。		
\\	シャツ	このシャツはアイロンが 必要[ひつよう]だ。		
\\	洗濯[せんたく](~する)	一[いっ] 週間[しゅうかん]、 洗濯[せんたく]をしていない。		
\\	間違[まちが]える	あたしは 電話[でんわ] 番号[ばんごう]を 間違[まちが]えました。		
\\	アイスクリーム	弟[おとうと]はアイスクリームが 大好[だいす]きです。		
\\	乾[かわ]く	夏[なつ]は 洗濯物[せんたくもの]がすぐ 乾[かわ]く。		
\\	冷[さ]める	彼[かれ]への 気持[きも]ちが 冷[さ]めた。		
\\	色々[いろいろ](~な)	その 店[みせ]はいろいろな 品物[しなもの]を 売[う]っている。		
\\	持[も]って 行[い]く	水[みず]を 持[も]って 行[い]きましょう。		
\\	着替[きが]える	彼[かれ]はきれいな 服[ふく]に 着替[きが]えた。		
\\	石鹸[せっけん]	石鹸[せっけん]がないので 買[か]ってきてください。		
\\	野球[やきゅう]	彼[かれ]は 野球[やきゅう]がうまい。		
\\	昼食[ちゅうしょく]	昼食[ちゅうしょく]におにぎりを 食[た]べました。		
\\	朝食[ちょうしょく]	朝食[ちょうしょく]に 納豆[なっとう]を 食[た]べました。		
\\	眠[ねむ]る	昨日[きのう]はあたしが8 時間[じかん] 眠[ねむ]りました。		
\\	初[はじ]め	私[わたし]は 初[はじ]めは 自分[じぶん]の 仕事[しごと]が 好[す]きではありませんでした。		
\\	火[ひ]	その 家[いえ]は 火[ひ]の 海[うみ]となった。		
\\	西[にし]	太陽[たいよう]は 西[にし]に 沈[しず]みます。		
\\	東[ひがし]	日[ひ]は 東[ひがし]から 上[のぼ]ります。		
\\	南[みなみ]	私[わたし]の 家[いえ]は 町[まち]の 南[みなみ]にあります。		
\\	夕食[ゆうしょく]	夕食[ゆうしょく]は7 時[じ]です。		
\\	なかなか	彼女[かのじょ]は 絵[え]がなかなか 上手[じょうず]です。		
\\	励[はげ]ます	父[ちち]が 息子[むすこ]を 励[はげ]ましたの。		
\\	涙[なみだ]	母[はは]は 涙[なみだ]を 流[なが]していた。		
\\	夢[ゆめ]	私[わたし]の 夢[ゆめ]は 先生[せんせい]になることです。		
\\	職場[しょくば]	私[わたし]の 職場[しょくば]は 女性[じょせい]の 方[ほう]が 多[おお]いです。		
\\	隣[となり]	隣[となり]の 部屋[へや]からテレビの 音[おと]が 聞[き]こえる。		
\\	マンション	彼[かれ]はマンションに 住[す]んでいます。		
\\	エレベーター	エレベーターで 下[した]に 降[お]りましょう。		
\\	窓[まど]	鳥[とり]が 窓[まど]から 逃[に]げたよ。		
\\	押[お]す	ボタンを 押[お]してください。		
\\	入学[にゅうがく](~する)	四[し] 月[がつ]に 大学[だいがく]に 入学[にゅうがく]しました。		
\\	戸[と]	部屋[へや]の 戸[と]が 開[あ]いています。		
\\	通[とお]り	この 通[とお]りは 車[くるま]の 音[おと]がうるさい。		
\\	亡[な]くなる	おととい、 昔[むかし]の 友人[ゆうじん]が 亡[な]くなった。		
\\	夫婦[ふうふ]	その 夫婦[ふうふ]はとても 仲[なか]がいい。		
\\	女性[じょせい]	私[わたし]の 職場[しょくば]は 女性[じょせい]の 方[ほう]が 多[おお]いです。		
\\	森[もり]	私[わたし]は 森[もり]を 歩[ある]くのが 好[す]きです。		
\\	トラック	トラックを 運転[うんてん]できますか。		
\\	レコード	父[ちち]は 古[ふる]いレコードを 集[あつ]めています。		
\\	熱[ねつ]	昨日[きのう]の 夜[よる]、 熱[ねつ]が 出[で]ました。		
\\	ページ	教科書[きょうかしょ]の36ページを 開[ひら]いてください。		
\\	踊[おど]る	彼女[かのじょ]はクラブで 踊[おど]るのが 好[す]きです。		
\\	長[なが]さ	このケーブルの 長[なが]さは1メートルです。		
\\	厚[あつ]さ	私[わたし]は 板[いた]の 厚[あつ]さを 測[はか]った。		
\\	秘密[ひみつ]	これは 秘密[ひみつ]です。		
\\	凄[すご]い	凄[すご]い 雨[あめ]になった。		
\\	親切[しんせつ](~な)	あの 村[むら]の 人々[ひとびと]はとても 親切[しんせつ]です。		
\\	絶対[ぜったい]	そんなことは 絶対[ぜったい]できないよ。		
\\	先月[せんげつ]	先月[せんげつ]、 友達[ともだち]の 結婚[けっこん] 式[しき]があった。		
\\	一昨年[おととし]	私[わたし]は 一昨年[おととし] 日本[にほん]に 来[き]ました。		
\\	冬休[ふゆやす]み	冬休[ふゆやす]みにお 祖父[じい]ちゃんの 家[うち]に 行[い]きます。		
\\	休日[きゅうじつ]	休日[きゅうじつ]は 家[いえ]でよくテレビを 見[み]ます。		
\\	正月[しょうがつ]	お 正月[しょうがつ]にはいろいろな 遊[あそ]びをします。		
\\	思[おも]い 出[で]	写真[しゃしん]は 良[い]い 思[おも]い 出[で]になります。		
\\	たまに	彼[かれ]はたまに 料理[りょうり]をします。		
\\	おかしい	彼[かれ]の 様子[ようす]がおかしい。		
\\	浅[あさ]い	この 川[かわ]は 浅[あさ]いです。		
\\	君[きみ]	この 本[ほん]を 君[きみ]にあげます。		
\\	奥[おく]さん	彼[かれ]の 奥[おく]さんはきれいな 方[かた]です。		
\\	赤[あか]ちゃん	あたしは 赤[あか]ちゃんにミルクをあげた。		
\\	小学生[しょうがくせい]	私[わたし]の 妹[いもうと]は 小学生[しょうがくせい]です。		
\\	貸[か]す	私[わたし]は 彼[かれ]に 本[ほん]を 貸[か]しています。		
\\	数[かぞ]える	いすの 数[かず]を 数[かぞ]えてください。		
\\	生[う]む	うちのネコが 子[こ]ネコを 生[う]みました。		
\\	育[そだ]つ	野菜[やさい]がよく 育[そだ]っている。		
\\	是非[ぜひ]	是非[ぜひ]、うちに 来[き]てください。		
\\	黒[くろ]	黒[くろ]のボールペンはありますか。		
\\	おもちゃ	赤[あか]ちゃんが 自動車[じどうしゃ]のおもちゃで 遊[あそ]んでいる。		
\\	靴[くつ]	靴[くつ]が 汚[よご]れた。		
\\	連絡[れんらく](~する)	仕事[しごと]が 終[お]わったら 連絡[れんらく]します。		
\\	伝言[でんごん](~する)	課長[かちょう]から 伝言[でんごん]があります。		
\\	預[あず]ける	鍵[かぎ]を 彼[かれ]に 預[あず]けた。		
\\	下[さ]がる	やっと 熱[ねつ]が 下[さ]がった。		
\\	泳[およ]ぐ	魚[さかな]がたくさん 泳[およ]いでいます。		
\\	アルバイト	兄[あに]はアルバイトをしています。		
\\	コート
\\	寒[さむ]かったのでコートを 着[き]た。		
\\	趣味[しゅみ]	私[わたし]の 趣味[しゅみ]は 映画[えいが]とテニスです。		
\\	散歩[さんぽ](~する)	公園[こうえん]を 散歩[さんぽ]しようか。		
\\	幸[しあわ]せ(~な)	良[い]い 友達[ともだち]がいて 私[わたし]は 幸[しあわ]せだ。		
\\	安心[あんしん](~する)	それを 聞[き]いて 安心[あんしん]しました。		
\\	大勢[おおぜい]	大勢[おおぜい]で 食事[しょくじ]に 出[で]かけました。		
\\	医者[いしゃ]	私[わたし]は 医者[いしゃ]に 相談[そうだん]した。		
\\	運転手[うんてんしゅ]	彼[かれ]はタクシーの 運転手[うんてんしゅ]です。		
\\	新幹線[しんかんせん]	新幹線[しんかんせん]で 京都[きょうと]に 行[い]きました。		
\\	往復[おうふく]	往復[おうふく] 切符[きっぷ]をください。		
\\	コピー	会議[かいぎ]で 書類[しょるい]のコピーを 配[くば]った。		
\\	奥[おく]	鋏[はさみ]は 机[つくえ]の 奥[おく]にあった。		
\\	重[おも]さ	この 荷物[にもつ]の 重[おも]さを 量[はか]ってください。		
\\	親[した]しい	週末[しゅうまつ]、 親[した]しい 友達[ともだち]を 家[いえ]に 呼[よ]んだ。		
\\	修理[しゅうり](~する)	車[くるま]を 修理[しゅうり]に 出[だ]した。		
\\	カード	支払[しはら]いはカードでお 願[ねが]いします。		
\\	挨拶[あいさつ](~する)	彼女[かのじょ]は 笑顔[えがお]で 挨拶[あいさつ]した。		
\\	階段[かいだん]	そこの 階段[かいだん]を 下[お]りてください。		
\\	配[くば]る	会議[かいぎ]で 書類[しょるい]のコピーを 配[くば]った。		
\\	断[ことわ]る	あたしは 彼[かれ]のプロポーズを 断[ことわ]った。		
\\	グラム	ひき 肉[にく]を200グラムください。		
\\	濃[こ]い	私[わたし]は 濃[こ]い 味[あじ]が 好[す]きだ。		
\\	壊[こわ]す	彼女[かのじょ]が 私[わたし]のケータイを 壊[こわ]した。		
\\	桜[さくら]	桜[さくら]の 花[はな]が 咲[さ]きました。		
\\	仕方[しかた]	ファイルのダウンロードの 仕方[しかた]が 分[わ]かりません。		
\\	絵[え]	これは 有名[ゆうめい]な 画家[がか]の 絵[え]です。		
\\	気温[きおん]	今日[きょう]の 気温[きおん]は26 度[ど]です。		
\\	品物[しなもの]	その 店[みせ]はいろいろな 品物[しなもの]を 売[う]っている。		
\\	吸[す]う	彼[かれ]は 大[おお]きく 息[いき]を 吸[す]った。		
\\	案内[あんない](~する)	私[わたし]が 中[なか]をご 案内[あんない]します。		
\\	スキー	冬[ふゆ]はよくスキーに 行[い]きます。		
\\	金持[かねも]ち	私[わたし]はもっとお 金持[かねも]ちの 家[いえ]に 生[う]まれたかった。		
\\	尋[たず]ねる	彼女[かのじょ]は 道[みち]を 尋[たず]ねた。		
\\	叩[たた]く	知[し]らない 人[ひと]に 肩[かた]を 叩[たた]かれました。		
\\	帰[かえ]り	仕事[しごと]の 帰[かえ]りにビールを 飲[の]んだ。		
\\	卵[たまご]	ニワトリは 卵[たまご]を 産[う]みます。		
\\	氷[こおり]	グラスに 氷[こおり]を 入[い]れてください。		
\\	留学生[りゅうがくせい]	私[わたし]の 学校[がっこう]にアメリカから 留学生[りゅうがくせい]が 来[き]ます。		
\\	ソフト	このソフトにはいろいろな 機能[きのう]があります。		
\\	歯医者[はいしゃ]	私[わたし]は 歯医者[はいしゃ]が 嫌[きら]いです。		
\\	蹴[け]る	飛行機[ひこうき]で 前[まえ]の 席[せき]を 蹴[け]らないでください。		
\\	敷[し]く	生[う]まれて 初[はじ]めて 布団[ふとん]を 敷[し]いた。		
\\	ひととき	楽[たの]しいひとときをお 過[す]ごしください。		
\\	交通事故[こうつうじこ]	彼[かれ]は 交通事故[こうつうじこ]を 起[お]こした。		
\\	遭[あ]う	彼[かれ]は 交通事故[こうつうじこ]に 遭[あ]った。		
\\	招待[しょうたい](~する)	あたしは 高校[こうこう]のときの 先生[せんせい]を 結婚[けっこん] 式[しき]に 招待[しょうたい]した。		
\\	咲[さ]く	桜[さくら]の 花[はな]が 咲[さ]きました。		
\\	二十歳[はたち]	日本[にほん]では 二十歳[はたち]からお 酒[さけ]を 飲[の]めます。		
\\	銀[ぎん]	彼[かれ]は 銀[ぎん]メダルを 取[と]った。		
\\	内側[うちがわ]	カップの 内側[うちがわ]の 線[せん]までお 湯[ゆ]を 入[い]れてください。		
\\	池[いけ]	あの 池[いけ]はとても 深[ふか]い。		
\\	石[いし]	私[わたし]は 石[いし]につまずいた。		
\\	愛[あい](~する)	子供[こども]は 親[おや]の 愛[あい]を 求[もと]めます。		
\\	課[か]	今[いま]から 課[か]のミーティングがある。		
\\	大学院[だいがくいん]	彼[かれ]は 大学院[だいがくいん]に 進[すす]みました。		
\\	台所[だいどころ]	台所[だいどころ]にねずみが 出[で]た。		
\\	カラー	私[わたし]たちのチームのカラーは 青[あお]です。		
\\	すり	すりに 財布[さいふ]をとられた。		
\\	キャンプ(~する)	友達[ともだち]とキャンプに 行[い]った。		
\\	オートバイ	彼[かれ]はオートバイに 乗[の]っている。		
\\	建[た]つ	ここに 来年[らいねん]、 家[いえ]が 建[た]ちます。		
\\	刺[さ]す	蚊[か]に 足[あし]を 刺[さ]された。		
\\	言[い]い 方[かた]	そんな 言[い]い方[かた]をしてはいけません。		
\\	沈[しず]む	太陽[たいよう]は 西[にし]に 沈[しず]みます。		
\\	組[く]み 立[た]てる	日曜日[にちようび]に 本棚[ほんだな]を 組[く]み 立[た]てます。		
\\	恐[おそ]ろしい	昨夜[ゆうべ]あたしは 恐[おそ]ろしい 夢[ゆめ]を 見[み]た。		
\\	外人[がいじん]	この 町[まち]には 外人[がいじん]が 少[すく]ない。		
\\	夏休[なつやす]み	今日[きょう]が 夏休[なつやす]み 最後[さいご]の 日[ひ]だ。		
\\	牛[うし]	牛[うし]が 草[くさ]を 食[た]べています。		
\\	食堂[しょくどう]	大学[だいがく]の 食堂[しょくどう]は 安[やす]い。		
\\	毎年
\\	[まいねん]	毎年[まいねん] 給料[きゅうりょう]が 上[あ]がる。		
\\	毎週[まいしゅう]	私[わたし]は 毎週[まいしゅう] 母[はは]に 電話[でんわ]をします。		
\\	何度[なんど](~も)	あたしはあの 店[みせ]には 何度[なんど]も 行[い]きました。		
\\	なかなか
\\	荷物[にもつ]がなかなか 届[とど]きません。		
\\	さっき	さっきの 話[はなし]を 続[つづ]けましょう。		
\\	久[ひさ]しぶり	明日[あした]、 久[ひさ]しぶりに 友達[ともだち]に 会[あ]います。		
\\	便利[べんり](~な)	インターネットはとても 便利[べんり]です。		
\\	迷惑[めいわく](~な)(~する)	夜中[よなか]に 電話[でんわ]したら 迷惑[めいわく]ですよ。		
\\	そんなに	あのスーツはそんなに 高[たか]くない。		
\\	びっくりする	大[おお]きな 音[おと]にびっくりしました。		
\\	足[た]りる	千[せん] 円[えん]で 足[た]りました。		
\\	疲[つか]れる	今日[きょう]は 疲[つか]れました。		
\\	ひどい	今日[きょう]はひどい 雨[あめ]になりましたね。		
\\	全然[ぜんぜん]	この 本[ほん]は 全然[ぜんぜん] 面白[おもしろ]くなかった。		
\\	猫[ねこ]	私[わたし]は 猫[ねこ]が 大好[だいす]きです。		
\\	デパート	あたしはデパートで 靴[くつ]を 買[か]った。		
\\	トイレ	トイレを 掃除[そうじ]しましたか。		
\\	地下鉄[ちかてつ]	私[わたし]は 地下鉄[ちかてつ]で 通勤[つうきん]しています。		
\\	駐車場[ちゅうしゃじょう]	車[くるま]は 駐車場[ちゅうしゃじょう]に 止[と]めてください。		
\\	ぶつかる	車[くるま]が 電柱[でんちゅう]にぶつかった。		
\\	運賃[うんちん]	鉄道[てつどう]の 運賃[うんちん]が 値上[ねあ]げされたね。		
\\	通学[つうがく](~する)	毎朝[まいあさ]、 通学[つうがく]に1 時間[じかん]かかる。		
\\	見送[みおく]る	彼[かれ]は 外国[がいこく]に 行[い]くのを 見送[みおく]りました。		
\\	泣[な]く	どうして 泣[な]いているの。		
\\	レストラン	レストランを 予約[よやく]しました。		
\\	メモ(~する)	あたしは 用件[ようけん]をメモしておきました。		
\\	予約[よやく](~する)	予約[よやく]は 要[い]りません。		
\\	番号[ばんごう]	この 番号[ばんごう]に 電話[でんわ]してください。		
\\	注文[ちゅうもん](~する)	レストランでピザを 注文[ちゅうもん]しました。		
\\	パン	朝[あさ]ご 飯[はん]にはいつもパンを 食[た]べる。		
\\	ワイン	私[わたし]はワインが 好[す]きです。		
\\	材料[ざいりょう]	サラダの 材料[ざいりょう]をそろえました。		
\\	豚肉[ぶたにく]	今[いま]、 豚肉[ぶたにく]が 安[やす]くなっていますよ。		
\\	焼[や]く	今[いま]、 魚[さかな]を 焼[や]いています。		
\\	匂[にお]い	私[わたし]はバラの 匂[にお]いが 大好[だいす]きです。		
\\	付[つ]ける	プレゼントにリボンを 付[つ]けた。		
\\	痩[や]せる	お 医者[いしゃ]さんに 痩[や]せるように 言[い]われました。		
\\	届[とど]ける	これを 彼[かれ]に 届[とど]けてください。		
\\	間[ま]に 合[あ]う	あたしは 授業[じゅぎょう]に 間[ま]に 合[あ]いました。		
\\	向[む]く	こっちを 向[む]いてください。		
\\	強[つよ]さ	風[かぜ]の 強[つよ]さに 驚[おどろ]きました。		
\\	得意[とくい]	彼[かれ]は 歌[うた]が 得意[とくい]です。		
\\	塗[ぬ]る	壁[かべ]にペンキを 塗[ぬ]っています。		
\\	投[な]げる	彼[かれ]はボールを 投[な]げた。		
\\	習[なら]う	私[わたし]はダンスを 習[なら]っています。		
\\	注目[ちゅうもく](~する)	私[わたし]たちはその 会社[かいしゃ]に 注目[ちゅうもく]している。		
\\	未来[みらい]	お 二人[ふたり]の 未来[みらい]に 乾杯[かんぱい]しましょう。		
\\	箱[はこ]	この 箱[はこ]は 重[おも]い。		
\\	右手[みぎて]	彼[かれ]は 右手[みぎて]でボールを 取[と]りました。		
\\	左手[ひだりて]	私[わたし]はいつも 左手[ひだりて]に 時計[とけい]をします。		
\\	プール	あたしは 夏休[なつやす]みにプールに 行[い]った。		
\\	ピアノ	昔[むかし]あたしはピアノを 習[なら]っていました。		
\\	光[ひか]る	波[なみ]がきらきら 光[ひか]っていました。		
\\	踏[ふ]む	花[はな]を 踏[ふ]まないでください。		
\\	別々[べつべつ]	別々[べつべつ]に 払[はら]いましょう。		
\\	ボール	ボールが 彼[かれ]の 頭[あたま]に 当[あ]たった。		
\\	握[にぎ]る	少女[しょうじょ]は 母親[ははおや]の 手[て]を 握[にぎ]った。		
\\	真[ま]っ 赤[か]	彼[かれ]の 顔[かお]は 真[ま]っ 赤[か]でした。		
\\	拾[ひろ]う	あたしは 道[みち]で 財布[さいふ]を 拾[ひろ]った。		
\\	回[まわ]す	ねじは 左[ひだり]に 回[まわ]すと 外[はず]れます。		
\\	ノート	ノートを 開[ひら]いてください。		
\\	辞[や]める	彼[かれ]は 会社[かいしゃ]を 辞[や]めます。		
\\	揺[ゆ]れる	地震[じしん]で 家[いえ]がかなり 揺[ゆ]れました。		
\\	立派[りっぱ](~な)	彼[かれ]らは 立派[りっぱ]な 家[いえ]を 建[た]てました。		
\\	練習[れんしゅう](~する)	娘[むすめ]は 今[いま]、バイオリンを 練習[れんしゅう]しています。		
\\	割[わ
\\	る	皿[さら]を 落[お]として 割[わ]った。		
\\	部分[ぶぶん]	この 部分[ぶぶん]は 問題[もんだい]ないです。		
\\	拭[ふ]く	彼[かれ]はタオルで 顔[かお]を 拭[ふ]きました。		
\\	招[まね]く	両親[りょうしん]を 食事[しょくじ]に 招[まね]いた。		
\\	急[きゅう](~な)	急[きゅう]な 坂道[さかみち]を 上[のぼ]った。		
\\	汚[よご]れる	ズボンが 汚[よご]れた。		
\\	渡[わた]る	小[ちい]さな 川[かわ]を 渡[わた]りました。		
\\	途中[とちゅう]	あたしは 途中[とちゅう]で 財布[さいふ]を 落[お]としました。		
\\	インタビュー(~する)	彼[かれ]はインタビューに、はきはきと 答[こた]えてたよ。		
\\	盗[ぬす]む	誰[だれ]かが 私[わたし]のカバンを 盗[ぬす]みました。		
\\	普段[ふだん]	私[わたし]は 普段[ふだん]は
\\	シャツとジーンズを 着[き]ています。		
\\	旅館[りょかん]	京都[きょうと]では 旅館[りょかん]に 泊[と]まりました。		
\\	博物館[はくぶつかん]	昨日[きのう]、 車[くるま]の 博物館[はくぶつかん]に 行[い]った。		
\\	ホーム	もうすぐこのホームに 電車[でんしゃ]が 来[き]ます。		
\\	引[ひ]っ 越[こ]す	来月[らいげつ]、 大阪[おおさか]に 引[ひ]っ 越[こ]します。		
\\	荷物[にもつ]	九日[ここのか]に 荷物[にもつ]が 届[とど]きます。		
\\	家賃[やちん]	ここは 家賃[やちん]が 高[たか]いので 引[ひ]っ 越[こ]したいです。		
\\	計画[けいかく](~する)	彼[かれ]は 一人[ひとり] 旅[たび]の 計画[けいかく]を 立[た]てた。		
\\	廊下[ろうか]	廊下[ろうか]は 走[はし]らないでください。		
\\	天井[てんじょう]	この 部屋[へや]は 天井[てんじょう]が 高[たか]いですね。		
\\	ベッド	彼[かれ]はベッドで 寝[ね]ています。		
\\	ソファー	私[わたし]の 犬[いぬ]はいつもこのソファーに 座[すわ]ります。		
\\	屋根[やね]	猫[ねこ]が 屋根[やね]に 上[のぼ]っている。		
\\	祭[まつ]り	彼女[かのじょ]は 祭[まつ]りが 大好[だいす]きです。		
\\	離婚[りこん](~する)	友人[ゆうじん]が 離婚[りこん]しました。		
\\	葉[は]	これは 桜[さくら]の 葉[は]です。		
\\	ボート	池[いけ]でボートに 乗[の]りました。		
\\	燃[も]える	山[やま]が 燃[も]えています。		
\\	巻[ま]く	彼[かれ]は 頭[あたま]にタオルを 巻[ま]いていた。		
\\	プラスチック	このカップはプラスチックです。		
\\	訪[おとず]れる	私[わたし]は 夏[なつ]に 京都[きょうと]を 訪[おとず]れました。		
\\	破[やぶ]る	彼[かれ]は 約束[やくそく]を 破[やぶ]った。		
\\	関係[かんけい](~する)	彼[かれ]はその 事件[じけん]に 関係[かんけい]がない。		
\\	不幸[ふこう]	彼女[かのじょ]の 家族[かぞく]に 不幸[ふこう]があった。		
\\	録音[ろくおん](~する)	そのラジオ 番組[ばんぐみ]はもう 録音[ろくおん]しました。		
\\	外国語[がいこくご]	日本[にほん]では 外国語[がいこくご]として 英語[えいご]を 習[なら]います。		
\\	着物
\\	[きもの]	彼女[かのじょ]は 着物[きもの]がよく 似合[にあ]います。		
\\	牛乳[ぎゅうにゅう]	この 牛乳[ぎゅうにゅう]はもう 古[ふる]くなっています。		
\\	救急車[きゅうきゅうしゃ]	誰[だれ]か 救急車[きゅうきゅうしゃ]を 呼[よ]んでください。		
\\	果物[くだもの]	デザートに 果物[くだもの]を 食[た]べましょう。		
\\	間違[まちが]い	この 文[ぶん]には 間違[まちが]いがあります。		
\\	映画館[えいがかん]	この 映画館[えいがかん]のポップコーンは 美味[おい]しいですよ。		
\\	怖[こわ]い	私[わたし]は 怖[こわ]い 夢[ゆめ]を 見[み]ました。		
\\	鏡[かがみ]	彼女[かのじょ]は 鏡[かがみ]を 見[み]て 髪[かみ]を 直[なお]した。		
\\	草[くさ]	牛[うし]が 草[くさ]を 食[た]べています。		
\\	敬語[けいご]	お 客様[きゃくさま]には 敬語[けいご]を 使[つか]いなさい。		
\\	欠席[けっせき](~する)	風邪[かぜ]のため 今日[きょう]は 欠席[けっせき]します。		
\\	スイッチ	部屋[へや]の 電気[でんき]のスイッチがどこか 分[わ]かりません。		
\\	掃除[そうじ](~する)	週末[しゅうまつ]は 部屋[へや]の 掃除[そうじ]をしました。		
\\	畳[たたみ]	そのホテルには 畳[たたみ]の 部屋[へや]がある。		
\\	速[はや]さ	彼[かれ]はすごい 速[はや]さで 走[はし]っていきました。		
\\	林[はやし]	私[わたし]たちは 林[はやし]の 中[なか]に 入[はい]っていった。		
\\	プレゼント	友達[ともだち]が 誕生日[たんじょうび]プレゼントをくれた。		
\\	弁当[べんとう]	今日[きょう]は 弁当[べんとう]を 持[も]ってきました。		
\\	いつでも	いつでもうちに 来[き]てください。		
\\	寂[さび]しい	これは 寂[さび]しい 曲[きょく]ですね。		
\\	少[すこ]しも	あなたは 少[すこ]しも 悪[わる]くない。		
\\	尊敬[そんけい](~する)	あなたの 尊敬[そんけい]する 人[ひと]は 誰[だれ]ですか。		
\\	テニス	新[あたら]しいコートでテニスをしました。		
\\	広[ひろ]さ	その 家[いえ]の 広[ひろ]さはどれ 位[ぐらい]ですか。		
\\	読[よ]み	私[わたし]はその 漢字[かんじ]の 読[よ]みを 間違[まちが]えました。		
\\	我慢[がまん](~する)	彼[かれ]のわがままには 我慢[がまん]できません。		
\\	国籍[こくせき]	私[わたし]は 日本[にほん] 国籍[こくせき]です。		
\\	恥[は]ずかしい	とても 恥[は]ずかしかった。		
\\	ボタン	ボタンを 押[お]してください。		
\\	洋服[ようふく]	彼女[かのじょ]は 洋服[ようふく]をたくさん 持[も]っています。		
\\	油[あぶら]	水[みず]と 油[あぶら]は 混[ま]ざらない。		
\\	おかしい
\\	彼[かれ]の 話[はなし]はおかしかった。		
\\	余[あま]る	夕食[ゆうしょく]の 料理[りょうり]がたくさん 余[あま]った。		
\\	コート
\\	新[あたら]しいコートでテニスをしました。		
\\	正直[しょうじき](~な)	彼女[かのじょ]はとても 正直[しょうじき]だ。		
\\	ドラマ	私[わたし]は 昼[ひる]のドラマを 毎日[まいにち] 見[み]ます。		
\\	ビザ	その 国[くに]はビザなしで 入[はい]れますか。		
\\	水道[すいどう]	東京[とうきょう]は 水道[すいどう]の 水[みず]がまずい。		
\\	店員[てんいん]	あの 店員[てんいん]はとても 親切[しんせつ]です。		
\\	電池[でんち]	携帯電話[けいたいでんわ]の 電池[でんち]が 切[き]れている。		
\\	ポケット	財布[さいふ]をポケットに 仕舞[しま]った。		
\\	握手[あくしゅ]	卒業[そつぎょう] 式[しき]のあと、 先生[せんせい]と 握手[あくしゅ]しました。		
\\	そろそろ	そろそろ 始[はじ]めましょうか。		
\\	ぶどう	このぶどうは 酸[す]っぱいです。		
\\	迷[まよ]う	途中[とちゅう]で 迷[まよ]って 遅[おそ]くなってしまいました。		
\\	見物[けんぶつ](~する)	みんなで 東京[とうきょう] 見物[けんぶつ]をした。		
\\	知[し]らせる	友達[ともだち]がコンサートの 時間[じかん]を 知[し]らせてくれました。		
\\	神社[じんじゃ]	京都[きょうと]には 神社[じんじゃ]がたくさんある。		
\\	注射[ちゅうしゃ](~する)	彼[かれ]は 注射[ちゅうしゃ]があまり 好[す]きではありません。		
\\	めったに	彼女[かのじょ]はめったに 怒[おこ]りません。		
\\	作文[さくぶん]	あたしは 日本語[にほんご]で 作文[さくぶん]を 書[か]きました。		
\\	鳴[な]る	今朝[けさ]、5 時[じ]に 電話[でんわ]が 鳴[な]った。		
\\	真面目[まじめ](~な)	彼[かれ]は 真面目[まじめ]な 人[ひと]です。		
\\	申[もう]し 込[こ]む	日本語[にほんご]の 試験[しけん]を 受[う]けるなら、 来月[らいげつ]5 日[いつか]までに 申[もう]し 込[こ]んでください。		
\\	冷蔵庫[れいぞうこ]	飲[の]み 物[もの]は 冷蔵庫[れいぞうこ]にあります。		
\\	気[き]に 入[い]る	新[あたら]しい 靴[くつ]がとても 気[き]に 入[い]りました。		
\\	クリスマス	クリスマスに 雪[ゆき]が 降[ふ]りました。		
\\	ネクタイ	父[ちち]の 日[ひ]にネクタイをプレゼントした。		
\\	誕生日[たんじょうび]	彼女[かのじょ]の 誕生日[たんじょうび]は7 月[がつ]16 日[にち]です。		
\\	独身[どくしん]	私[わたし]の 兄[あに]はまだ 独身[どくしん]です。		
\\	苦[にが]い	この 薬[くすり]はとても 苦[にが]いです。		
\\	バイオリン	娘[むすめ]は 今[いま]、バイオリンを 練習[れんしゅう]しています。		
\\	バッグ	母[はは]は 高[たか]いバッグをたくさん 持[も]っています。		
\\	今[いま]まで	今[いま]までどこにいたのですか。		
\\	切符[きっぷ]	東京[とうきょう]までの 切符[きっぷ]を 買[か]った。		
\\	ゴム	ゴムが 伸[の]びてしまった。		
\\	少々[しょうしょう]	塩[しお]を 少々[しょうしょう] 入[い]れてください。		
\\	随分[ずいぶん]	随分[ずいぶん] 早[はや]く 終[お]わったね。		
\\	引[ひ]っ 張[ぱ]る	娘[むすめ]が 私[わたし]の 手[て]を 引[ひ]っ 張[ぱ]った。		
\\	焼[や]ける	肉[にく]が 焼[や]けました。		
\\	鉛筆[えんぴつ]	鉛筆[えんぴつ]を 貸[か]してください。		
\\	会話[かいわ](~する)	親子[おやこ]の 会話[かいわ]は 大切[たいせつ]です。		
\\	ギター	彼女[かのじょ]はギターが 得意[とくい]です。		
\\	茶[ちゃ]	私[わたし]たちは 毎日[まいにち]お 茶[ちゃ]を 飲[の]みます。		
\\	暑[あつ]さ	今年[ことし]の 夏[なつ]は 暑[あつ]さが 厳[きび]しい。		
\\	胃[い]	昨日[きのう]から 胃[い]が 痛[いた]い。		
\\	禁煙[きんえん](~する)	この 席[せき]は 禁煙[きんえん]です。		
\\	丸[まる]い	あの 子[こ]は 少[すこ]し 顔[かお]が 丸[まる]くなりましたね。		
\\	写[うつ]る	このカメラはよく 写[うつ]りますよ。		
\\	喫茶店[きっさてん]	いつもの 喫茶店[きっさてん]で3 時[じ]に 会[あ]いましょう。		
\\	やっぱり	やっぱり 旅[たび]が 大好[だいす]きだ。		
\\	週刊誌[しゅうかんし]	その 週刊誌[しゅうかんし]はゴシップばかりだ。		
\\	棚[たな]	棚[たな]からその 箱[はこ]を 下[お]ろしてください。		
\\	震[ふる]える	彼[かれ]は 寒[さむ]くて 震[ふる]えていました。		
\\	磨[みが]く	靴[くつ]を 磨[みが]いてください。		
\\	カーテン	暗[くら]くなったので、カーテンを 閉[し]めてください。		
\\	火事[かじ]	その 家[いえ]は 火事[かじ]で 焼[や]けてしまいました。		
\\	ナイフ	そのナイフはよく 切[き]れます。		
\\	皆[みな]さん	皆[みな]さんにお 話[はなし]があります。		
\\	受[う]ける	彼[かれ]は 就職[しゅうしょく] 試験[しけん]を 受[う]けた。		
\\	付[つ]く
\\	靴[くつ]に 泥[どろ]が 付[つ]いています。		
\\	私[わたくし]	私[わたくし]は 北海道[ほっかいどう]の 出身[しゅっしん]です。		
\\	結構[けっこう]	彼女[かのじょ]は 結構[けっこう]めがねが 似合[にあ]うね。		
\\	悪口[わるくち]	彼[かれ]は 決[けっ]して 人[ひと]の 悪口[わるくち]を 言[い]わないの。		
\\	広[ひろ]まる	その 噂[うわさ]はすぐに 広[ひろ]まったよ。		
\\	内緒[ないしょ]	この 話[はなし]は 課長[かちょう]には 内緒[ないしょ]ですよ。		
\\	つぶる	私[わたし]は 目[め]をつぶってよく 考[かんが]えました。		
\\	用件[ようけん]	あたしは 用件[ようけん]をメモしておきました。		
\\	ロマンチック	彼女[かのじょ]はロマンチックな 人[ひと]です。		
\\	貯金[ちょきん]	貯金[ちょきん]は 十分[じゅうぶん]にあります。		
\\	バイク	私[わたし]は 毎朝[まいあさ]バイクで 駅[えき]まで 行[い]きます。		
\\	預[あず]かる	荷物[にもつ]を 預[あず]かってください。		
\\	オレンジ	このオレンジは 酸[す]っぱいです。		
\\	噛[か]む	犬[いぬ]が 私[わたし]の 手[て]を 噛[か]みました。		
\\	きつい	このシャツは 少[すこ]しきついです。		
\\	越[こ]える	私[わたし]たちは 高[たか]い 山[やま]を 越[こ]えました。		
\\	小包[こづつみ]	フランスの 友達[ともだち]から 小包[こづつみ]が 届[とど]いた。		
\\	新年[しんねん]	私[わたし]たちは 毎年[まいとし]、 新年[しんねん]に 家族[かぞく]で 集[あつ]まります。		
\\	配達[はいたつ]	彼[かれ]は 新聞[しんぶん] 配達[はいたつ]をしている。		
\\	動物園[どうぶつえん]	昨日[きのう] 子供[こども]たちと 動物園[どうぶつえん]に 行[い]きました。		
\\	封筒[ふうとう]	その 手紙[てがみ]を 封筒[ふうとう]に 入[い]れた。		
\\	危[あぶ]ない	後[うし]ろから 押[お]すと 危[あぶ]ないですよ。		
\\	テキスト	テキストを 読[よ]んでください。		
\\	ビニール	ゴミはそのビニール 袋[ぶくろ]に 入[い]れてください。		
\\	アルバム	あたしは 彼[かれ]のアルバムを 見[み]た。		
\\	水泳[すいえい](~する)	母[はは]は 健康[けんこう]のために 水泳[すいえい]をしている。		
\\	混[こ]む	電車[でんしゃ]が 混[こ]んでいる。		
\\	スカート	彼女[かのじょ]はあまりスカートははかない。		
\\	取[と]れる	シャツのボタンが 取[と]れた。		
\\	ペン	ペンを 貸[か]してください。		
\\	物[もの]	彼女[かのじょ]の 家[いえ]には 物[もの]がたくさんあります。		
\\	郵便[ゆうびん]	この 小包[こづつみ]を 郵便[ゆうびん]で 送[おく]ってください。		
\\	割[わ]れる	コップが 落[お]ちて 割[わ]れた。		
\\	駅員[えきいん]	その 駅員[えきいん]は 英語[えいご]が 話[はな]せます。		
\\	おしゃべり 
\\	彼女[かのじょ]はとてもおしゃべりで、 秘密[ひみつ]を 守[まも]れません。		
\\	緑色[みどりいろ]	彼[かれ]は 緑色[みどりいろ]のジャケットを 着[き]ています。		
\\	毎朝[まいあさ]	私[わたし]は 毎朝[まいあさ]ラジオを 聞[き]きます。		
\\	曲[ま]げる	ひざを 曲[ま]げてください。		
\\	きっと	明日[あした]はきっと 雨[あめ]が 降[ふ]ります。		
\\	理科[りか]	私[わたし]は 理科[りか]が 得意[とくい]です。		
\\	下着[したぎ]	私[わたし]は 下着[したぎ]を 手[て]で 洗[あら]う。		
\\	太[ふと]る	私[わたし]は 去年[きょねん]から5キロ 太[ふと]りました。		
\\	冷房[れいぼう]	冷房[れいぼう]を 消[け]してください。 寒[さむ]いです。		
\\	別[わか]れる	彼女[かのじょ]とは3か 月前[げつまえ]に 別[わか]れました。		
\\	空[あ]く	後[うし]ろの 席[せき]が 空[あ]いています。		
\\	出掛[でか]ける	主人[しゅじん]はもう 出掛[でか]けました。		
\\	毎晩[まいばん]	姉[あね]は 毎晩[まいばん] 日記[にっき]を 書[か]いています。		
\\	指輪[ゆびわ]	彼女[かのじょ]に 指輪[ゆびわ]をプレゼントしました。		
\\	腐[くさ]る	リンゴが 腐[くさ]っている。		
\\	貧乏[びんぼう](~な)	彼[かれ]は 昔[むかし]は 貧乏[びんぼう]だった。		
\\	おしゃれ(~な)	彼[かれ]はいつもおしゃれな 服[ふく]を 着[き]ています。		
\\	休憩[きゅうけい]	このあと、10 分[ぷん] 休憩[きゅうけい]を 取[と]ります。		
\\	舌[した]	猫[ねこ]が 舌[した]で 自分[じぶん]の 体[からだ]をきれいにしています。		
\\	ジュース	このジュースは 甘[あま]すぎる。		
\\	ゼロ	私[わたし]は、 日本語[にほんご]の 勉強[べんきょう]をゼロから 始[はじ]めました。		
\\	朝刊[ちょうかん]	今日[きょう]の 朝刊[ちょうかん]に 面白[おもしろ]い 記事[きじ]があった。		
\\	丁寧[ていねい](~な)	彼女[かのじょ]はいつも 丁寧[ていねい]に 仕事[しごと]をします。		
\\	苦手[にがて]	私[わたし]は 料理[りょうり]が 苦手[にがて]です。		
\\	日[ひ]	夏[なつ]は 日[ひ]が 長[なが]い。		
\\	引[ひ]っ 越[こ]し(~する)	とうとう 引[ひ]っ 越[こ]しの 日[ひ]が 来[き]ました。		
\\	歩道[ほどう]	その 歩道[ほどう]は 狭[せま]くて 危[あぶ]ないです。		
\\	折[お]れる	強風[きょうふう]で 木[き]の 枝[えだ]が 折[お]れた。		
\\	見送[みおく]り	駅[えき]に 友人[ゆうじん]の 見送[みおく]りに 行[い]きました。		
\\	夕刊[ゆうかん]	そのニュースは 夕刊[ゆうかん]で 見[み]ました。		
\\	この 間[あいだ]	この 間彼[あいだ] 女[かのじょ]に 会[あ]った。		
\\	冷[ひ]える	クーラーで 体[からだ]が 冷[ひ]えた。		
\\	ピンク	妻[つま]はピンクの 花[はな]が 好[す]きです。		
\\	グラス	グラスに 氷[こおり]を 入[い]れてください。		
\\	毛[け]	猫[ねこ]の 毛[け]がセーターに 付[つ]いた。		
\\	黒板[こくばん]	答[こた]えを 黒板[こくばん]に 書[か]いてください。		
\\	積[つ]もる	雪[ゆき]が 積[つ]もっています。		
\\	半年[はんとし]	あたしは 半年[はんとし]に1 回[かい]、 旅行[りょこう]に 行[い]きます。		
\\	現在[げんざい]	この 電話[でんわ] 番号[ばんごう]は 現在[げんざい]、 使[つか]われておりません。		
\\	利用[りよう](~する)	私[わたし]はよく 図書館[としょかん]を 利用[りよう]します。		
\\	方法[ほうほう]	他[ほか]に 方法[ほうほう]がありません。		
\\	中心[ちゅうしん]	ここが 建物[たてもの]の 中心[ちゅうしん]です。		
\\	明[あき]らか(~な)	あの 人[ひと]がお 金[かね]を 取[と]ったのは 明[あき]らかだ。		
\\	ほぼ	仕事[しごと]がほぼ 終[お]わりました。		
\\	商品[しょうひん]	この 商品[しょうひん]はよく 売[う]れている。		
\\	大型[おおがた]	大型[おおがた]のテレビを 買[か]った。		
\\	限[かぎ]る	このサービスは 週末[しゅうまつ]に 限[かぎ]ります。		
\\	記録[きろく](~する)	会議[かいぎ]で 話[はな]したことを 記録[きろく]してください。		
\\	具体的[ぐたいてき](~な)	具体的[ぐたいてき]な 例[れい]をいくつか 見[み]せてください。		
\\	効果[こうか]	この 薬[くすり]には 胃[い]を 守[まも]る 効果[こうか]がある。		
\\	公開[こうかい](~する)	その 映画[えいが]は 今日[きょう]、 公開[こうかい]されます。		
\\	作品[さくひん]	この 絵[え]はゴッホの 作品[さくひん]だ。		
\\	迫[せま]る	締[し]め 切[き]りが 迫[せま]っています。		
\\	大会[たいかい]	彼[かれ]はマラソン 大会[たいかい]で1 位[い]になりました。		
\\	背景[はいけい]	私[わたし]は 山[やま]を 背景[はいけい]に 写真[しゃしん]を 撮[と]りました。		
\\	文字[もじ]	壁[かべ]に 文字[もじ]が 書[か]いてあった。		
\\	規則[きそく]	あの 会社[かいしゃ]の 規則[きそく]は 厳[きび]しいです。		
\\	近所[きんじょ](~に)	近所[きんじょ]にカナダ 人[じん]が 住[す]んでいる。		
\\	グラフ	彼[かれ]は 売上[うりあげ]をグラフにした。		
\\	首都[しゅと]	オーストラリアの 首都[しゅと]はシドニーではありません。		
\\	減[へ]らす	最近[さいきん]、 食事[しょくじ]を 減[へ]らしています。		
\\	隠[かく]れる	太陽[たいよう]が 雲[くも]に 隠[かく]れた。		
\\	美術館[びじゅつかん]	その 美術館[びじゅつかん]には、すばらしい 絵[え]がたくさんあります。		
\\	ばか(~な)	私[わたし]はばかだった。		
\\	暇[ひま]	今日[きょう]はお 昼[ひる]ご 飯[はん]を 食[た]べる 暇[ひま]もなかった。		
\\	絵葉書[えはがき]	友達[ともだち]から 絵葉書[えはがき]が 届[とど]きました。		
\\	あんなに	彼[かれ]は 日本語[にほんご]があんなに 上手[じょうず]だなんて 知[し]りませんでした。		
\\	行[い]き	沖縄[おきなわ]まで 行[い]きは3 時間[じかん]かかりました。		
\\	通勤[つうきん](~する)	毎朝[まいあさ]、 通勤[つうきん]に30 分[ぷん]かかります。		
\\	ちゃんと	朝食[ちょうしょく]はちゃんと 食[た]べましたか。		
\\	のんびり	休[やす]みの 日[ひ]は 家[うち]でのんびり 過[す]ごします。		
\\	緊張[きんちょう](~する)	彼[かれ]は 緊張[きんちょう]していたな。		
\\	気付[きづ]く	彼[かれ]はやっと 問題点[もんだいてん]に 気付[きづ]きました。		
\\	物語[ものがたり]	これは 日本[にほん]の 女[おんな]の 子[こ]の 物語[ものがたり]です。		
\\	社長[しゃちょう]	あの 会社[かいしゃ]は 社長[しゃちょう]がよく 変[か]わる。		
\\	出口[でぐち]	出口[でぐち]はどこですか。		
\\	登[のぼ]る	私[わたし]たちは 昨年[さくねん]、 富士山[ふじさん]に 登[のぼ]りました。		
\\	真っ白[まっしろ](~な)	外[そと]は 雪[ゆき]で 真っ白[まっしろ]だった。		
\\	下[お]ろす	棚[たな]からその 箱[はこ]を 下[お]ろしてください。		
\\	貸[か]し 出[だ]す	その 図書館[としょかん]は 海外[かいがい]の 雑誌[ざっし]も 貸[か]し 出[だ]している。		
\\	サッカー	彼[かれ]はサッカーの 選手[せんしゅ]です。		
\\	暖房[だんぼう]	冬[ふゆ]は 暖房[だんぼう]が 必要[ひつよう]です。		
\\	手袋[てぶくろ]	寒[さむ]いので 手袋[てぶくろ]をしました。		
\\	留守[るす]	父[ちち]は 今[いま]、 留守[るす]です。		
\\	嘘[うそ]	彼女[かのじょ]が 嘘[うそ]をついたので、 彼[かれ]は 怒[おこ]った。		
\\	遠慮[えんりょ](~する)	私[わたし]は 遠慮[えんりょ]します。		
\\	折[お]る	祖父[そふ]は 足[あし]の 骨[ほね]を 折[お]りました。		
\\	スープ	このスープはとても 熱[あつ]い。		
\\	バナナ	私[わたし]は 毎朝[まいあさ]バナナを 食[た]べます。		
\\	可哀相[かわいそう](~な)	その 可哀相[かわいそう]な 子供[こども]たちは 食[た]べるものがない。		
\\	パトカー	あそこにパトカーがいる。		
\\	ハンバーガー	あたしはハンバーガーを 頼[たの]みました。		
\\	毛布[もうふ]	この 毛布[もうふ]は 暖[あたた]かい。		
\\	エスカレーター	
\\	階[かい]までエスカレーターで 行[い]きましょう。		
\\	お 嬢[じょう]さん	彼[かれ]はお 嬢[じょう]さんと 一緒[いっしょ]でした。		
\\	手帳[てちょう]	新[あたら]しい 手帳[てちょう]を 買[か]いました。		
\\	タオル	あたしはタオルで 顔[かお]をふいた。		
\\	売店[ばいてん]	駅[えき]の 売店[ばいてん]で 雑誌[ざっし]を 買[か]った。		
\\	パチンコ	私[わたし]はパチンコの 面白[おもしろ]さが 分[わ]かりません。		
\\	謝[あやま]る	悪[わる]いのはあなたです。 謝[あやま]ってください。		
\\	ケーキ	誕生日[たんじょうび]にケーキを 食[た]べました。		
\\	天気予報[てんきよほう]	明日[あした]の 天気予報[てんきよほう]は 雨[あめ]です。		
\\	変[へん](~な)	変[へん]な 音[おと]が 聞[き]こえます。		
\\	一生懸命[いっしょうけんめい]	彼[かれ]は 毎日[まいにち] 一生懸命[いっしょうけんめい] 働[はたら]いている。		
\\	間違[まちが
\\	う	この 問題[もんだい]、 間違[まちが]っていますよ。		
\\	事務室[じむしつ]	事務室[じむしつ]でコピーを 取[と]ってきます。		
\\	スーツ	今日[きょう]はスーツを 着[き]ています。		
\\	チケット	この 遊園地[ゆうえんち]のチケットは3000 円[えん]です。		
\\	チョコレート	妹[いもうと]はチョコレートが 大好[だいす]きです。		
\\	脱[ぬ]ぐ	靴[くつ]を 脱[ぬ]いでください。		
\\	年賀状[ねんがじょう]	昨日[きのう]、 年賀状[ねんがじょう]を 出[だ]しました。		
\\	乗[の]り 物[もの]	自転車[じてんしゃ]は 便利[べんり]な 乗[の]り 物[もの]です。		
\\	布団[ふとん]	布団[ふとん]がとても 柔[やわ]らかい。		
\\	役[や]に 立[た]つ	この 辞書[じしょ]は 全[まった]く 役[や]に 立[た]ちません。		
\\	破[やぶ]れる	シャツが 破[やぶ]れている。		
\\	用事[ようじ]	父[ちち]は 用事[ようじ]で 出掛[でか]けています。		
\\	写[うつ]す	彼[かれ]は 友達[ともだち]の 答[こた]えを 写[うつ]した。		
\\	腕時計[うでどけい]	私[わたし]の 腕時計[うでどけい]は 遅[おく]れている。		
\\	贈[おく]り 物[もの]	すてきな 贈[おく]り 物[もの]をどうもありがとう。		
\\	チャンネル	テレビのチャンネルを 変[か]えてください。		
\\	冷[ひ]やす	頭[あたま]を 冷[ひ]やしなさい。		
\\	片付[かたづ]ける	早[はや]く 部屋[へや]を 片付[かたづ]けなさい。		
\\	乾杯[かんぱい](~する)	お 二人[ふたり]の 未来[みらい]に 乾杯[かんぱい]しましょう。		
\\	停車[ていしゃ](~する)	この 電車[でんしゃ]は 東京[とうきょう]まで 停車[ていしゃ]いたしません。		
\\	鳴[な]く	その 鶏[にわとり]は 毎朝[まいあさ] 鳴[な]きます。		
\\	ガソリンスタンド	あのガソリンスタンドにはコンビニもあります。		
\\	片道[かたみち]	東京[とうきょう]まで 片道[かたみち] 切符[きっぷ]を 買[か]った。		
\\	交番[こうばん]	あそこの 交番[こうばん]で 道[みち]を 聞[き]きましょう。		
\\	大分[だいぶ]	大分[だいぶ] 早[はや]く 着[つ]きましたね。		
\\	デート(~する)	デートで 遊園地[ゆうえんち]に 行[い]きました。		
\\	梅雨[つゆ]	梅雨[つゆ]は6 月[がつ] 頃[ごろ]です。		
\\	真[ま]っ 直[す]ぐ	この 道[みち]を 真[ま]っ 直[す]ぐ 行[い]ってください。		
\\	レモン	紅茶[こうちゃ]にレモンを 入[い]れて 飲[の]んだ。		
\\	上着[うわぎ]	暑[あつ]いので 上着[うわぎ]を 脱[ぬ]ぎました。		
\\	偉[えら]い	彼[かれ]は 偉[えら]い 学者[がくしゃ]です。		
\\	書[か]き 方[かた]	彼[かれ]はその 漢字[かんじ]の 書[か]き 方[かた]が 分[わ]からない。		
\\	炊[た]く	ご 飯[はん]と 一緒[いっしょ]にとり 肉[にく]を 炊[た]きました。		
\\	チーズ	チーズを 一切[ひとき]れ 食[た]べました。		
\\	ドライブ(~する)	今日[きょう]は 群馬[ぐんま]までドライブしました。		
\\	踏切[ふみきり]	その 踏切[ふみきり]は 長[なが]い。		
\\	ラケット	テニスのラケットを 買[か]いました。		
\\	受付[うけつけ]	受付[うけつけ]はこちらです。		
\\	怒[おこ]る	彼女[かのじょ]はめったに 怒[おこ]りません。		
\\	化粧[けしょう](~する)	彼女[かのじょ]は 化粧[けしょう]の 時間[じかん]が 長過[ながす]ぎる。		
\\	都合[つごう]	今日[きょう]は 都合[つごう]が 悪[わる]くて 行[い]けません。		
\\	ひげ	父[ちち]はひげをはやしています。		
\\	本棚[ほんだな]	これはとても 大[おお]きな 本棚[ほんだな]ですね。		
\\	真[ま]っ 暗[くら](~な)	彼[かれ]は 真[ま]っ 暗[くら]な 部屋[へや]でテレビを 見[み]ていました。		
\\	沸[わ]く	お 湯[ゆ]が 沸[わ]いたので、コーヒーを 入[い]れます。		
\\	売[う]り 切[き]れる	その 本[ほん]は 直[す]ぐ 売[う]り 切[き]れた。		
\\	押[お]し 入[い]れ	古[ふる]い 本[ほん]は 押[お]し 入[い]れに 入[はい]っています。		
\\	革[かわ]	革[かわ]のベルトを 買[か]いました。		
\\	ぐっすり	赤[あか]ちゃんがぐっすり 寝[ね]ている。		
\\	紅茶[こうちゃ]	温[あたた]かい 紅茶[こうちゃ]が 飲[の]みたい。		
\\	邪魔[じゃま](~な)	邪魔[じゃま]です、どいてください。		
\\	ソース	今日[きょう]はトマトソースをたくさん 作[つく]りました。		
\\	足[た]す	もう 少[すこ]し 醤油[しょうゆ]を 足[た]してください。		
\\	トマト	私[わたし]はトマトが 大好[だいす]きです。		
\\	バター	パンにバターをぬって 食[た]べました。		
\\	発車[はっしゃ](~する)	この 電車[でんしゃ]はあと5 分[ふん]で 発車[はっしゃ]します。		
\\	バレーボール	今日[きょう]、 学校[がっこう]の 休[やす]み 時間[じかん]にバレーボールをしました。		
\\	おかず	晩[ばん]ご 飯[はん]のおかずは 何[なに]?		
\\	カレンダー	カレンダーに 予定[よてい]を 書[か]いた。		
\\	とにかく	とにかく、 私[わたし]の 話[はなし]を 最後[さいご]まで 聞[き]いてください。		
\\	楽[たの]しみ	彼女[かのじょ]のいちばんの 楽[たの]しみはショッピングです。		
\\	おしゃべり 
\\	彼女[かのじょ]たちはおしゃべりに 夢中[むちゅう]です。		
\\	読書[どくしょ](~する)	私[わたし]の 趣味[しゅみ]は 読書[どくしょ]です。		
\\	寒気[さむけ]	何[なん]だか 寒気[さむけ]がします。		
\\	夕焼[ゆうや]け	今日[きょう]は 夕焼[ゆうや]けがきれいです。		
\\	痛[いた]める	彼[かれ]は 柔道[じゅうどう]で 腰[こし]を 痛[いた]めたんだ。		
\\	くすぐったい	くすぐったいから 足[あし]の 裏[うら]を 触[さわ]らないで。		
\\	泥棒[どろぼう](~する)	泥棒[どろぼう]にカーナビを 盗[ぬす]まれました。		
\\	襖[ふすま]	襖[ふすま]を 閉[し]めてください。		
\\	美人[びじん]	彼[かれ]のお 母[かあ]さんは 美人[びじん]です。		
\\	タイプライター	母[はは]は 古[ふる]いタイプライターを 持[も]っています。		
\\	再来年[さらいねん]	次[つぎ]のオリンピックは 再来年[さらいねん]です。		
\\	冗談[じょうだん]	冗談[じょうだん]は 止[や]めてください。		
\\	近[ちか]づく	試験[しけん]の 日[ひ]が 近[ちか]づいています。		
\\	電話帳[でんわちょう]	新[あたら]しい 電話帳[でんわちょう]が 届[とど]きました。		
\\	ベル	御用[ごよう]の 方[かた]はベルを 鳴[な]らしてください。		
\\	干[ほ]す	母[はは]は 洗濯物[せんたくもの]を 干[ほ]しています。		
\\	金額[きんがく]	レシートで 買[か]い 物[もの]の 金額[きんがく]を 確[たし]かめたよ。		
\\	蚊[か]	この 公園[こうえん]は 蚊[か]が 多[おお]いです。		
\\	悲[かな]しむ	父[ちち]は 友[とも]だちの 死[し]を 悲[かな]しんでいます。		
\\	クーラー	今年[ことし]の 夏[なつ]はクーラーがよく 売[う]れた。		
\\	失礼[しつれい](~な)(~する)	失礼[しつれい]ですが、どなたですか。		
\\	洗濯機[せんたくき]	汚[よご]れた 服[ふく]は 洗濯機[せんたくき]に 入[い]れましたよ。		
\\	豆腐[とうふ]	私[わたし]は 豆腐[とうふ]にしょうゆをかけて 食[た]べます。		
\\	昼寝[ひるね]	私[わたし]の 子供[こども]は 毎日[まいにち] 昼寝[ひるね]をします。		
\\	不便[ふべん](~な)	私[わたし]の 家[いえ]は 駅[えき]から 遠[とお]くて 不便[ふべん]です。		
\\	ミルク	コーヒーにミルクを 入[い]れてください。		
\\	向[む]こう 側[がわ]	私[わたし]の 家[いえ]は 川[かわ]の 向[む]こう 側[がわ]にあります。		
\\	億[おく]	世界[せかい] 人口[じんこう]は65 億[おく] 人[にん]だ。		
\\	お 礼[れい]	彼女[かのじょ]にお 礼[れい]の 手紙[てがみ]を 書[か]きました。		
\\	着[き]せる	娘[むすめ]に 可愛[かわい]いドレスを 着[き]せた。		
\\	サラダ	ポテトサラダを 作[つく]るから、ジャガイモを 買[か]ってきて。		
\\	代[か]える	社長[しゃちょう]に 代[か]えて 部長[ぶちょう]を 出席[しゅっせき]させます。		
\\	四角[しかく](~な)	名前[なまえ]はこの 四角[しかく]の 中[なか]に 書[か]いてください。		
\\	急行[きゅうこう]	急行[きゅうこう]はその 駅[えき]には 止[と]まりません。		
\\	交差点[こうさてん]	あの 交差点[こうさてん]で 大[おお]きな 事故[じこ]がありました。		
\\	覚[さ]める	これが 悪[わる]い 夢[ゆめ]なら 覚[さ]めてほしい。		
\\	昼休[ひるやす]み	昼休[ひるやす]みに 公園[こうえん]に 行[い]った。		
\\	ベルト	ベルトがきつくなりました。		
\\	真[ま]っ 黒[くろ](~な)	インクで 手[て]が 真[ま]っ 黒[くろ]になった。		
\\	皆様[みなさま]	皆様[みなさま]、こんにちは。		
\\	ラーメン	夕食[ゆうしょく]にラーメンを 食[た]べました。		
\\	ライター	店[みせ]にライターを 忘[わす]れました。		
\\	ワイシャツ	彼[かれ]は 白[しろ]いワイシャツを 着[き]ている。		
\\	転[ころ]ぶ	転[ころ]んでひざを 打[う]ちました。		
\\	シャワー	私[わたし]は 朝[あさ]、シャワーを 浴[あ]びます。		
\\	扇風機[せんぷうき]	暑[あつ]いから 扇風機[せんぷうき]をつけよう。		
\\	遅刻[ちこく](~する)	遅刻[ちこく]しないでください。		
\\	釣[つ]る	昨日[きのう] 大[おお]きな 魚[さかな]を 釣[つ]りました。		
\\	うどん	私[わたし]はお 昼[ひる]にうどんを 食[た]べました。		
\\	仮名[かな]	日本語[にほんご]では 漢字[かんじ]と 仮名[かな]で 文[ぶん]を 書[か]きます。		
\\	記念日[きねんび]	今日[きょう]は 両親[りょうしん]の 結婚[けっこん] 記念日[きねんび]です。		
\\	下[くだ]り	ここからは 急[きゅう]な 下[くだ]りが 続[つづ]きます。		
\\	掲示板[けいじばん]	私[わたし]は 掲示板[けいじばん]でアルバイトを 見[み]つけました。		
\\	凍[こお]る	寒[さむ]い 朝[あさ]は 道路[どうろ]が 凍[こお]ります。		
\\	ごみ	ごみはそこの 角[かど]に 出[だ]してください。		
\\	スーパーマーケット	このスーパーマーケットは10 時[じ]に 閉[し]まります。		
\\	ぶらぶら(~する)	朝[あさ]、 近所[きんじょ]をぶらぶらした。		
\\	アナウンサー	そのアナウンサーは 毎日[まいにち]テレビに 出[で]ています。		
\\	自炊[じすい](~する)	私[わたし]は 料理[りょうり]ができないので、 自炊[じすい]は 無理[むり]です。		
\\	止[と]める	車[くるま]を 止[と]めて。		
\\	年下[としした](~の)	彼[かれ]は 奥[おく]さんより 年下[としした]です。		
\\	治[なお]す	母[はは]の 病気[びょうき]を 治[なお]してください。		
\\	箸[はし]	お 箸[はし]の 使[つか]い 方[かた]を 直[なお]しなさい。		
\\	花見[はなみ]	友達[ともだち]と 花見[はなみ]をしました。		
\\	引[ひ]き 出[だ]し	財布[さいふ]は 引[ひ]き 出[だ]しの 中[なか]にあります。		
\\	訳[やく]	この 日本語[にほんご] 訳[やく]は 間違[まちが]っています。		
\\	いかが	お 加減[かげん]はいかがですか。		
\\	交通費[こうつうひ]	会社[かいしゃ]までの 交通費[こうつうひ]は1か 月[げつ]8,000 円[えん]です。		
\\	ステレオ	彼[かれ]はステレオで 音楽[おんがく]を 聞[き]いた。		
\\	再会[さいかい](~する)	彼[かれ]らは 互[たが]いに 再会[さいかい]を 喜[よろこ]んだの。		
\\	滑[すべ]る	雪[ゆき]で 道[みち]が 滑[すべ]ります。		
\\	早口[はやくち]	彼[かれ]は 早口[はやくち]で、 何[なに]を 言[い]っているか 分[わ]かりません。		
\\	裏返[うらがえ]す	彼[かれ]は 紙[かみ]を 裏返[うらがえ]しました。		
\\	角[かど]	次[つぎ]の 角[かど]で 左[ひだり]に 曲[ま]がってください。		
\\	きらきら(~する)	サッカーをしている 時[とき]、 彼[かれ]はきらきらしていますね。		
\\	三角[さんかく]	紙[かみ]を 三角[さんかく]に 切[き]りました。		
\\	激[はげ]しい	彼[かれ]らは 昨日[きのう]、 激[はげ]しい 夫婦喧嘩[ふうふげんか]をしました。		
\\	経営[けいえい](~する)	祖父[そふ]はタクシー 会社[がいしゃ]を 経営[けいえい]していました。		
\\	拡大[かくだい](~する)	この 図[ず]を 拡大[かくだい]コピーしてください。		
\\	姿勢[しせい]	あの 子[こ]はいつも 姿勢[しせい]が 悪[わる]い。		
\\	周辺[しゅうへん]	この 周辺[しゅうへん]には 大学[だいがく]が 多[おお]い。		
\\	他[た]	留学生[りゅうがくせい]はアジア 人[じん]が 多[おお]く、その 他[た]は3 割[わり]です。		
\\	達[たっ]する	そのサービスのユーザー 数[すう]は5,000 万[まん] 人[にん]に 達[たっ]しています。		
\\	当時[とうじ]	事故[じこ] 当時[とうじ]、ここにはマンションが 建[た]っていました。		
\\	当然[とうぜん]	当然[とうぜん]、 彼[かれ]も 来[く]ると 思[おも]っていました。		
\\	能力[のうりょく](~のある)	彼[かれ]は 能力[のうりょく]のある 社員[しゃいん]です。		
\\	判断[はんだん](~する)	彼[かれ]の 判断[はんだん]は 正[ただ]しい。		
\\	過[す]ぎる	もう 予定[よてい]の 時間[じかん]を 過[す]ぎています。		
\\	正確[せいかく](~な)	彼[かれ]の 計算[けいさん]は 正確[せいかく]です。		
\\	戦争[せんそう](~する)	
\\	年[ねん]にイラクで 戦争[せんそう]があった。		
\\	放送[ほうそう](~する)	その 番組[ばんぐみ]は 来週[らいしゅう] 放送[ほうそう]されます。		
\\	習慣[しゅうかん]	毎朝[まいあさ]コーヒーを 飲[の]むのが 習慣[しゅうかん]です。		
\\	フィルム	旅行[りょこう]のためにたくさんフィルムを 買[か]った。		
\\	受[う]け 付[つ]ける	郵便物[ゆうびんぶつ]は5 時[じ]まで 受[う]け 付[つ]けています。		
\\	ダンス	彼女[かのじょ]はダンスが 大好[だいす]きです。		
\\	こんなに	こんなに 面白[おもしろ]い 本[ほん]は 初[はじ]めて 読[よ]んだ。		
\\	吹[ふ]く	今日[きょう]は 北風[きたかぜ]が 吹[ふ]いている。		
\\	まあ	まあ 今回[こんかい]は 許[ゆる]してあげよう。		
\\	メール(~する)	彼[かれ]からメールを 受[う]け 取[と]りました。		
\\	インターネット	インターネットはとても 便利[べんり]です。		
\\	出身[しゅっしん]	私[わたくし]は 北海道[ほっかいどう]の 出身[しゅっしん]です。		
\\	職業[しょくぎょう]	あなたの 職業[しょくぎょう]を 教[おし]えてください。		
\\	スター	彼女[かのじょ]は 有名[ゆうめい]な 映画[えいが]スターです。		
\\	くだらない	あの 映画[えいが]は 全[まった]くくだらない。		
\\	名簿[めいぼ]	私[わたし]の 名前[なまえ]が 名簿[めいぼ]に 入[はい]っていません。		
\\	通常[つうじょう]	通常[つうじょう]、 授業[じゅぎょう]は8 時[じ] 半[はん]に 始[はじ]まります。		
\\	家内[かない]	家内[かない]は 今[いま]、ヨーロッパに 旅行[りょこう] 中[ちゅう]です。		
\\	消防署[しょうぼうしょ]	今日[きょう]、 授業[じゅぎょう]で 消防署[しょうぼうしょ]に 行[い]きました。		
\\	再[ふたた]び	彼[かれ]は 再[ふたた]びここに 戻[もど]ってきた。		
\\	大統領[だいとうりょう]	フランスの 大統領[だいとうりょう]は 誰[だれ]ですか。		
\\	入場券[にゅうじょうけん]	入場券[にゅうじょうけん]をお 持[も]ちですか。		
\\	ネックレス	彼女[かのじょ]は 金[きん]のネックレスをしています。		
\\	パジャマ	今日[きょう]はお 昼過[ひるす]ぎまでパジャマでいました。		
\\	ボールペン	黒[くろ]のボールペンはありますか。		
\\	かゆい	背中[せなか]がかゆいです。		
\\	覚[さ]ます	子供[こども]が 目[め]を 覚[さ]ました。		
\\	(目を覚ます); 
\\	すいか	すいかに 塩[しお]をかける 人[ひと]もいます。		
\\	そば 
\\	お 昼[ひる]にそばを 食[た]べた。		
\\	特急[とっきゅう]	特急[とっきゅう] 電車[でんしゃ]の 切符[きっぷ]はどこで 買[か]えますか。		
\\	のり	のりで2 枚[まい]の 紙[かみ]を 貼[は]り 合[あ]わせた。		
\\	大抵[たいてい]	朝食[ちょうしょく]は 大抵[たいてい]7 時[じ] 頃[ごろ]に 食[た]べます。		
\\	バスケットボール	姉[あね]はバスケットボールの 選手[せんしゅ]です。		
\\	酔[よ]っ 払[ぱら]う	彼女[かのじょ]は 酔[よ]っ 払[ぱら]うとすぐ 泣[な]く。		
\\	どきどき(~する)	彼女[かのじょ]に 会[あ]うとどきどきします。		
\\	欲[ほ]しがる	子供[こども]がジュースを 欲[ほ]しがっています。		
\\	早起[はやお]き(~する)	祖父[そふ]は 早起[はやお]きです。		
\\	味噌汁[みそしる]	私[わたし]は 毎日[まいにち] 味噌汁[みそしる]を 飲[の]みます。		
\\	読[よ]み 方[かた]	この 漢字[かんじ]の 読[よ]み 方[かた]を 教[おし]えてください。		
\\	入[い]り 口[ぐち]	入[い]り 口[ぐち]は 向[む]こうです。		
\\	降[お]ろす	彼[かれ]は 車[くるま]から 荷物[にもつ]を 降[お]ろした。		
\\	掛[か]かる	壁[かべ]に 絵[え]が 掛[か]かっている。		
\\	サンドイッチ	今日[きょう]のお 昼[ひる]はサンドイッチです。		
\\	四角[しかく]い	その 四角[しかく]いお 皿[さら]をサラダに 使[つか]ってください。		
\\	突[つ]き 当[あ]たる	彼[かれ]は 今[いま]、 大[おお]きな 壁[かべ]に 突[つ]き 当[あ]たっています。		
\\	乾[かわ]かす	髪[かみ]をよく 乾[かわ]かしなさい。		
\\	定期券[ていきけん]	私[わたし]は 定期券[ていきけん]を 持[も]っていないので、 切符[きっぷ]を 買[か]ってきます。		
\\	にこにこ(~する)	彼女[かのじょ]はいつもにこにこしています。		
\\	下[くだ]る	彼[かれ]らは 今[いま]、 山[やま]を 下[くだ]っています。		
\\	ウール	このウールのセーターは 暖[あたた]かい。		
\\	コーラ	私[わたし]の 兄[あに]はコーラが 好[す]きだ。		
\\	トイレットペーパー	トイレットペーパーを 交換[こうかん]しました。		
\\	乗[の]り 換[か]え	次[つぎ]の 駅[えき]で 乗[の]り 換[か]えです。		
\\	灰皿[はいざら]	父[ちち]はいつも 灰皿[はいざら]を 持[も]ち 歩[ある]いています。		
\\	ハム	ハムと 卵[たまご]でサンドイッチを 作[つ]りました。		
\\	床[ゆか]	彼[かれ]はかばんを 床[ゆか]に 置[お]きました。		
\\	ローマ 字[じ]	ローマ 字[じ]で 名前[なまえ]を 書[か]いてください。		
\\	いちご	このいちごはとても 甘[あま]い。		
\\	カレーライス	子供[こども]はカレーライスが 好[す]きです。		
\\	キャッシュカード	キャッシュカードを 無[な]くしたので、 銀行[ぎんこう]に 電話[でんわ]しました。		
\\	停留所[ていりゅうじょ]	バスの 停留所[ていりゅうじょ]で10 分[ぷん] 待[ま]ちました。		
\\	床屋[とこや]	昨日[きのう] 床屋[とこや]で 髪[かみ]を 切[き]った。		
\\	歯磨[はみが]き	朝[あさ] 晩[ばん]、 歯磨[はみが]きを 忘[わす]れないで。		
\\	夕飯[ゆうはん]	夕飯[ゆうはん]は 寿司[すし]でした。		
\\	酔[よ]っ 払[ぱら]い	彼[かれ]はただの 酔[よ]っ 払[ぱら]いです。		
\\	例[れい]	一[ひと]つ 例[れい]を 挙[あ]げてください。		
\\	お 辞儀[じぎ](~する)	お 客様[きゃくさま]には 深[ふか]くお 辞儀[じぎ]してください。		
\\	横断歩道[おうだんほどう]	あそこに 横断歩道[おうだんほどう]があります。		
\\	ぐらぐら(~する)	地震[じしん]で 家[いえ]がぐらぐらと 揺[ゆ]れた。		
\\	航空便[こうくうびん]	航空便[こうくうびん]で 書類[しょるい]が 届[とど]きました。		
\\	ジャム	このいちごでジャムを 作[つく]りましょう。		
\\	スリッパ	スリッパを 履[は]いてください。		
\\	畳[たた]む	布団[ふとん]を 畳[たた]んでください。		
\\	トランプ	お 正月[しょうがつ]に 家族[かぞく]でトランプをしました。		
\\	日陰[ひかげ](~の)	暑[あつ]いので 日陰[ひかげ]で 休[やす]みましょう。		
\\	文房具[ぶんぼうぐ]	新[あたら]しい 文房具[ぶんぼうぐ]を 買[か]いました。		
\\	免許証[めんきょしょう]	免許証[めんきょしょう]を 見[み]せてください。		
\\	訳[やく]す	父[ちち]は 英語[えいご]の 小説[しょうせつ]を 日本語[にほんご]に 訳[やく]しています。		
\\	よろしい	レポートはこれでよろしいですか。		
\\	両替[りょうがえ](~する)	アメリカドルを 日本[にほん] 円[えん]に 両替[りょうがえ]しました。		
\\	片[かた]づく	やっと 部屋[へや]が 片[かた]づきました。		
\\	カップ	水[みず]を1カップ 入[い]れてください。		
\\	消[け]しゴム	えんぴつや 消[け]しゴムにも 名前[なまえ]を 書[か]いてください。		
\\	スケート	湖[みずうみ]でスケートをした。		
\\	鍋[なべ]	鍋[なべ]にスープが 入[はい]っているよ。		
\\	寝坊[ねぼう](~する)	今朝[けさ]は 寝坊[ねぼう]しました。		
\\	鋏[はさみ]	この 鋏[はさみ]はよく 切[き]れる。		
\\	真[ま]っ 青[さお]	空[そら]が 真[ま]っ 青[さお]です。		
\\	汚[よご]す	彼[かれ]は 服[ふく]を 汚[よご]した。		
\\	忘[わす]れ 物[もの](~する)	うちの 子[こ]は 忘[わす]れ 物[もの]が 多[おお]いです。		
\\	掛[か]け 算[ざん]	弟[おとうと]は 掛[か]け 算[ざん]を 習[なら]っている。		
\\	臭[くさ]い	納豆[なっとう]は 臭[くさ]い。		
\\	突[つ]き 当[あ]たり	
\\	階[かい]の 突[つ]き 当[あ]たりが 私[わたし]の 部屋[へや]です。		
\\	マフラー	今日[きょう]は 寒[さむ]いので、 手袋[てぶくろ]とマフラーをしています。		
\\	桃[もも]	私[わたし]の 一番[いちばん] 好[す]きな 果物[くだもの]は 桃[もも]です。		
\\	アクセサリー	彼女[かのじょ]はアクセサリーを 自分[じぶん]で 作[つく]ります。		
\\	あくび	会議[かいぎ]であくびが 止[と]まりませんでした。		
\\	あめ	缶[かん]にあめが 入[はい]っています。		
\\	上[のぼ]り	この 山[やま]は 上[のぼ]りがかなりきついです。		
\\	なるべく	なるべく 早[はや]く 仕事[しごと]を 終[お]わらせてください。		
\\	じっと	あの 生徒[せいと]はじっと 先生[せんせい]の 話[はなし]を 聞[き]いていたね。		
\\	食器[しょっき]	結婚[けっこん]のお 祝[いわ]いに 食器[しょっき]を 頂[いただ]きました。		
\\	延[の]ばす	出発[しゅっぱつ]を 一週間[いっしゅうかん] 延[の]ばしたの。		
\\	薬局[やっきょく]	薬局[やっきょく]で 目薬[めぐすり]を 買[か]いました。		
\\	雄[おす]	うちの 猫[ねこ]は 雄[おす]です。		
\\	雌[めす]	あなたの 犬[いぬ]は 雌[めす]ですか、 雄[おす]ですか。		
\\	和風[わふう]	私[わたし]は 和風[わふう]の 家[いえ]に 住[す]みたいです。		
\\	いびき	自分[じぶん]のいびきで 目[め]が 覚[さ]めました。		
\\	ピクニック	今日[きょう]はピクニックに 行[い]きましょう。		
\\	地域[ちいき]	その 地域[ちいき]は 今[いま]、 危険[きけん]だ。		
\\	勘違[かんちが]い(~する)	あなたは 何[なに]か 勘違[かんちが]いしています。		
\\	重要[じゅうよう](~な)	これは 重要[じゅうよう]な 書類[しょるい]です。		
\\	主人[しゅじん]	主人[しゅじん]は 今[いま]、 留守[るす]です。		
\\	お 宅[たく]	先生[せんせい]のお 宅[たく]はどちらですか。		
\\	ハイキング	夏[なつ]に 山[やま]でハイキングをしました。		
\\	畑[はたけ]	母[はは]は 畑[はたけ] 仕事[しごと]が 好[す]きだ。		
\\	ボーイフレンド	あの 男[おとこ]の 人[ひと]はあなたのボーイフレンドですか。		
\\	イヤリング	妻[つま]の 誕生日[たんじょうび]に 銀[ぎん]のイヤリングを 買[か]いました。		
\\	ガールフレンド	あなたのガールフレンドは 可愛[かわい]いですね。		
\\	柔[やわ]らか(~な)	窓[まど]から 柔[やわ]らかな 風[かぜ]が 入[はい]ってきます。		
\\	納豆[なっとう]	私[わたし]は 納豆[なっとう]をよく 食[た]べます。		
\\	別[べつ]に	その 映画[えいが]は 別[べつ]に 見[み]たいと 思[おも]いません。		
\\	どうしても	どうしてもその 訳[わけ]を 知[し]りたい。		
\\	父親[ちちおや]	彼[かれ]の 父親[ちちおや]は 先生[せんせい]です。		
\\	誘[さそ]う	彼女[かのじょ]をデートに 誘[さそ]った。		
\\	母親[ははおや]	少女[しょうじょ]は 母親[ははおや]の 手[て]を 握[にぎ]った。		
\\	昼前[ひるまえ]	昼前[ひるまえ]に 会議[かいぎ]があった。		
\\	ふらふら(~する)	私[わたし]は 疲[つか]れてふらふらです。		
\\	こぼす	教科書[きょうかしょ]にコーヒーをこぼしてしまいました。		
\\	持[も]ってくる	そのいすを 持[も]ってきてください。		
\\	振[ふ]り 向[む]く	彼女[かのじょ]は 振[ふ]り 向[む]いて 俺[おれ]に 微笑[ほほえ]んだんだ。		
\\	洗面所[せんめんじょ]	洗面所[せんめんじょ]で 顔[かお]を 洗[あら]った。		
\\	外側[そとがわ]	窓[まど]の 外側[そとがわ]は 私[わたし]が 拭[ふ]きます。		
\\	段々[だんだん]	段々[だんだん] 仕事[しごと]が 楽[たの]しくなってきました。		
\\	包[つつ]む	お 皿[さら]を 新聞紙[しんぶんし]で 包[つつ]んでください。		
\\	無[な]くなる	もうお 金[かね]が 無[な]くなりました。		
\\	豚[ぶた]	豚[ぶた]の 鼻[はな]はおもしろいですね。		
\\	褒[ほ]める	頑張[がんば]ったので 褒[ほ]められました。		
\\	沸[わ]かす	お 湯[ゆ]を 沸[わ]かしてください。		
\\	暖[あたた]める	今[いま]、 車[くるま]を 暖[あたた]めています。		
\\	換[か]える	車[くるま]のタイヤを 換[か]えた。		
\\	安[やす]っぽい	あの 人[ひと]はいつも 安[やす]っぽい 服[ふく]を 着[き]ている。		
\\	ぎらぎら(~する)	太陽[たいよう]がぎらぎらしている。		
\\	何[なん]とか	何[なん]とか 会議[かいぎ]に 間[ま]に 合[あ]いました。		
\\	いよいよ	明日[あした]はいよいよ 出発[しゅっぱつ]の 日[ひ]です。		
\\	がらがら(~な)	映画館[えいがかん]はがらがらでした。		
\\	缶詰[かんづめ]	缶詰[かんづめ]のフルーツはとても 甘[あま]い。		
\\	怪我[けが](~する)	彼女[かのじょ]は 腕[うで]を 怪我[けが]した。		
\\	寒[さむ]さ	今日[きょう]は 厳[きび]しい 寒[さむ]さになるでしょう。		
\\	祝日[しゅくじつ]	
\\	月[がつ]11 日[にち]は
\\	山[やま]の 日[ひ]」という 祝日[しゅくじつ]です。		
\\	すき 焼[や]き	昨日[きのう]の 夜[よる]はすき 焼[や]きを 食[た]べた。		
\\	葉書[はがき]	家族[かぞく]に 葉書[はがき]を 書[か]いています。		
\\	掃[は]く	床[ゆか]をほうきで 掃[は]きました。		
\\	何[なに]しろ	彼[かれ]は 何[なに]しろよくしゃべります。		
\\	どんどん 
\\	彼[かれ]は 山道[やまみち]をどんどん 進[すす]んでいったの。		
\\	ぺこぺこ	私[わたし]はおなかがぺこぺこです。		
\\	ぺらぺら	彼女[かのじょ]はドイツ 語[ご]がぺらぺらです。		
\\	レインコート	雨[あめ]の 日[ひ]はレインコートを 着[き]て 自転車[じてんしゃ]に 乗[の]ります。		
\\	アイス	暑[あつ]いのでアイスを 食[た]べました。		
\\	塩辛[しおから]い	海[うみ]の 水[みず]は 塩辛[しおから]い。		
\\	セロテープ	セロテープはありますか。		
\\	大幅[おおはば](~な)(~に)	計画[けいかく]を 大幅[おおはば]に 変更[へんこう]した。		
\\	教育[きょういく]	彼[かれ]は 海外[かいがい]で 教育[きょういく]を 受[う]けました。		
\\	工事[こうじ]	工事[こうじ]の 音[おと]がうるさい。		
\\	担当[たんとう](~する)	あたしはセールスを 担当[たんとう]しています。		
\\	法律[ほうりつ](~の)	兄[あに]は 法律[ほうりつ] 事務所[じむしょ]で 働[はたら]いています。		
\\	丸[まる]	紙[かみ]に 大[おお]きな 丸[まる]を 描[か]きました。		
\\	発表[はっぴょう]	重大[じゅうだい]な 発表[はっぴょう]があります。		
\\	研究[けんきゅう](~する)	彼[かれ]はロボットの 研究[けんきゅう]をしています。		
\\	一方[いっぽう]	ここは 一方[いっぽう] 通行[つうこう]です。		
\\	国内[こくない]	この 携帯[けいたい] 電話[でんわ]が 使[つか]えるのは 国内[こくない]だけです。		
\\	全体[ぜんたい]	全体[ぜんたい]の80パーセントが 完成[かんせい]しました。		
\\	後[のち]	後[のち]に 彼[かれ]は 総理大臣[そうりだいじん]になりました。		
\\	人間[にんげん]	人間[にんげん]は 自然[しぜん]には 勝[か]てません。		
\\	刺身[さしみ]	私[わたし]は 刺身[さしみ]は 食[た]べません。		
\\	ばら	お 祝[いわ]いにばらの 花束[はなたば]を 贈[おく]りました。		
\\	ねずみ	台所[だいどころ]にねずみが 出[で]た。		
\\	以下[いか]	日本[にほん]では19 歳[さい] 以下[いか]の 人[ひと]はお 酒[さけ]を 飲[の]めません。		
\\	昔[むかし]	彼[かれ]は 昔[むかし]は 貧乏[びんぼう]だった。		
\\	訳[わけ]	そんな 訳[わけ]で、 私[わたし]は 仕事[しごと]を 辞[や]めました。		
\\	違[ちが]い	この 二[ふた]つには 大[おお]きな 違[ちが]いがある。		
\\	しっかり(~する)	お 兄[にい]さんなんだから、もっとしっかりしなさい。		
\\	将来[しょうらい]	私[わたし]は 子供[こども]の 将来[しょうらい]が 心配[しんぱい]です。		
\\	不思議[ふしぎ](~な)	それはとても 不思議[ふしぎ]な 話[はなし]だね。		
\\	決[けっ]して	このことを 決[けっ]して 忘[わす]れないでください。		
\\	失敗[しっぱい](~する)	失敗[しっぱい]は 成功[せいこう]の 母[はは]。		
\\	番[ばん]	次[つぎ]はあなたの 番[ばん]ですよ。		
\\	出会[であ]う	アメリカで 彼女[かのじょ]と 出会[であ]いました。		
\\	悩[なや]む	彼[かれ]は 受験[じゅけん]のことで 悩[なや]んでいます。		
\\	汗[あせ]	その 選手[せんしゅ]はタオルで 汗[あせ]を 拭[ふ]きました。		
\\	すっかり	買[か]い 物[もの]をすっかり 忘[わす]れていた。		
\\	代[か]わり	誰[だれ]も 母親[ははおや]の 代[か]わりはできません。		
\\	一言[ひとこと]	最後[さいご]に 一言[ひとこと] 言[い]わせてください。		
\\	豊[ゆた]か(~な)	彼[かれ]は 心[こころ]が 豊[ゆた]かな 人[ひと]です。		
\\	接続[せつぞく](~する)	コンピューターをネットワークに 接続[せつぞく]しました。		
\\	抱[かか]える	彼[かれ]は 大[おお]きな 荷物[にもつ]を 抱[かか]えているわ。		
\\	エンジン	車[くるま]のエンジンが 調子[ちょうし] 悪[わる]い。		
\\	少女[しょうじょ]	少女[しょうじょ]は 母親[ははおや]の 手[て]を 握[にぎ]った。		
\\	無駄[むだ](~な)	それは 時間[じかん]の 無駄[むだ]です。		
\\	自動[じどう]	このドアは 自動[じどう]よ。		
\\	先輩[せんぱい]	妻[つま]は 会社[かいしゃ]で 私[わたし]の 先輩[せんぱい]でした。		
\\	眺[なが]める	猫[ねこ]が 窓[まど]から 外[そと]を 眺[なが]めているよ。		
\\	書類[しょるい]	書類[しょるい]を10 枚[まい]コピー しました。		
\\	年月[ねんげつ]	昔[むかし]の 写真[しゃしん]を 見[み]ると 年月[ねんげつ]を 感[かん]じます。		
\\	燃[も]やす	古[ふる]い 手紙[てがみ]を 燃[も]やしたんだ。		
\\	線路[せんろ]	子猫[こねこ]が 線路[せんろ]に 迷[まよ]い 込[こ]んだぞ。		
\\	水着[みずぎ]	水着[みずぎ]に 着替[きが]えました。		
\\	可愛[かわい]らしい	娘[むすめ]さんは 可愛[かわい]らしい 顔[かお]をしていますね。		
\\	仲良[なかよ]し	あの3 人[にん] 組[ぐみ]は 仲良[なかよ]しですね。		
\\	梅干[うめぼ]し	朝[あさ]ご 飯[はん]は 梅干[うめぼ]しとご 飯[ごはん]だけでいいです。		
\\	ドレス	彼女[かのじょ]は 紫[むらさき]のドレスを 着[き]ていた。		
\\	尽[つ]くす	彼女[かのじょ]は 夫[おっと]に 尽[つ]くしている。		
\\	地球[ちきゅう]	月[つき]は 地球[ちきゅう]のまわりを 回[ま]っています。		
\\	不親切[ふしんせつ]	あの 医者[いしゃ]はいつも 不親切[ふしんせつ]だ。		
\\	からから	喉[のど]が 渇[かわ]いてからからです。		
\\	本箱[ほんばこ]	雑誌[ざっし]を 本箱[ほんばこ]に 入[い]れました。		
\\	横書[よこが]き	この 本[ほん]は 横書[よこが]きです。		
\\	そば 屋[や]	昼[ひる]はそば 屋[や]に 行[い]きました。		
\\	加工[かこう](~する)	この 工場[こうじょう]では 魚[さかな]を 加工[かこう]している。		
\\	実験[じっけん](~する)	科学[かがく]の 授業[じゅぎょう]で 実験[じっけん]をした。		
\\	愛[あい]する	私[わたし]は 家族[かぞく]を 愛[あい]しています。		
\\	あんまり	これはあんまり 好[す]きじゃないな。		
\\	いきなり	知[し]らない 人[ひと]からいきなりメールが 来[き]て 驚[おどろ]きました。		
\\	表[おもて]	表[おもて]に 救急車[きゅうきゅうしゃ]が 止[と]まっています。		
\\	汚[よご]れ	この 汚[よご]れはなかなか 落[お]ちません。		
\\	間[あいだ]	雲[くも]の 間[あいだ]から 月[つき]が 出[で]た。		
\\	紫[むらさき]	私[わたし]は 紫[むらさき]の 車[くるま]が 欲[ほ]しいが、 妻[つま]が 反対[はんたい]している。		
\\	灰色[はいいろ]	今日[きょう]の 空[そら]は 灰色[はいいろ]だ。		
\\	血[ち]	血[ち]が 出[で]ていますよ。		
\\	顎[あご]	顎[あご]が 痛[いた]い。		
\\	始[はじ]めに	始[はじ]めに 会長[かいちょう]が 挨拶[あいさつ]した。		
\\	暖[あたた]まる	まだ 部屋[へや]が 暖[あたた]まらない。		
\\	お 世辞[せじ]	彼[かれ]はお 世辞[せじ]を 言[い]うのが 上手[うま]い。		
\\	袋[ふくろ]	袋[ふくろ]は 要[い]りません。		
\\	長袖[ながそで]	今日[きょう]は 寒[さむ]いので 長袖[ながそで]を 着[き]ました。		
\\	半袖[はんそで]	今日[きょう]は、 半袖[はんそで]のシャツを 着[き]よう。		
\\	形[かたち]	娘[むすめ]が 色々[いろいろ]な 形[かたち]のクッキーを 焼[や]いた。		
\\	景色[けしき]	ここは 景色[けしき]が 綺麗[きれい]ですね。		
\\	ざあざあ	雨[あめ]がざあざあ 降[ふ]っている。		
\\	再来月[さらいげつ]	再来月[さらいげつ]まで 予約[よやく]でいっぱいです。		
\\	泊[と]める	友達[ともだち]をうちに 泊[と]めてあげました。		
\\	叱[しか]る	父親[ちちおや]が 子供[こども]を 叱[しか]っている。		
\\	騒[さわ]ぐ	電車[でんしゃ]の 中[なか]で 騒[さわ]がないでください。		
\\	車道[しゃどう]	車道[しゃどう]の 工事[こうじ]が 始[はじ]まりました。		
\\	掃除機[そうじき]	掃除機[そうじき]が 壊[こわ]れた。		
\\	剃[そ]る	父[ちち]は 毎日[まいにち] 髭[ひげ]を 剃[そ]ります。		
\\	貼[は]る	ここに 切手[きって]を 貼[は]ってください。		
\\	台[だい]	そこにちょうど 良[い]い 台[だい]がある。		
\\	主[おも]に	この 商品[しょうひん]は 主[おも]に 女性[じょせい]に 人気[にんき]がありまして。		
\\	倒[たお]す	そのスキーヤーはポールを 倒[たお]した。		
\\	星[ほし]	今夜[こんや]は 星[ほし]がよく 見[み]えます。		
\\	長女[ちょうじょ]	うちの 長女[ちょうじょ]は10 歳[さい]です。		
\\	爪[つめ]	爪[つめ]が 伸[の]びています。		
\\	缶[かん]	缶[かん]にあめが 入[はい]っています。		
\\	とんとん	誰[だれ]かがとんとんとドアを 叩[たた]きました。		
\\	袖[そで]	このシャツは 袖[そで]が 短[みじか]い。		
\\	濡[ぬ]れる	雨[あめ]で 濡[ぬ]れてしまった。		
\\	パーセント	この 村[むら]の80パーセントは 老人[ろうじん]です。		
\\	上[あ]がる 
\\	成績[せいせき]が 上[あ]がりました。		
\\	拍手[はくしゅ](~する)	大[おお]きな 拍手[はくしゅ]が 上[あ]がった。		
\\	弾[ひ]く	彼[かれ]はギターを 弾[ひ]きます。		
\\	ひも	うちの 猫[ねこ]はひもで 遊[あそ]ぶのが 好[す]きです。		
\\	昼過[ひるす]ぎ	彼[かれ]は 昼過[ひるす]ぎに 来[き]ます。		
\\	船[ふね]	私[わたし]たちは 船[ふね]に 乗[の]った。		
\\	まく	日本[にほん]では、 二月[にがつ]に 豆[まめ]をまく 行事[ぎょうじ]がある。		
\\	姪[めい]	私[わたし]の 姪[めい]は3 才[さい]です。		
\\	個人[こじん](~の)	これは 私[わたし]の 個人[こじん]の 意見[いけん]です。		
\\	歌手[かしゅ]	彼[かれ]は 有名[ゆうめい]な 歌手[かしゅ]です。		
\\	社員[しゃいん]	彼[かれ]は 優秀[ゆうしゅう]な 社員[しゃいん]です。		
\\	不動産屋[ふどうさんや]	私[わたし]の 父[ちち]は 不動産屋[ふどうさんや]です。		
\\	最大[さいだい](~の)	これは 日本[にほん] 最大[さいだい]の 博物館[はくぶつかん]です。		
\\	動[うご]き	雲[くも]の 動[うご]きを 見[み]てみましょう。		
\\	事件[じけん]	彼[かれ]はその 事件[じけん]に 関係[かんけい]がない。		
\\	過去[かこ]	それは 過去[かこ]の 話[はなし]だ。		
\\	協力[きょうりょく](~する)	皆様[みなさま]のご 協力[きょうりょく]、ありがとうございました。		
\\	時期[じき]	一年[いちねん]のこの 時期[じき]は 雨[あめ]が 多[おお]いです。		
\\	経[た]つ	あれから14 年[ねん]が 経[た]ちました。		
\\	雷[かみなり]	うちの 犬[いぬ]は 雷[かみなり]が 怖[こわ]い。		
\\	責任[せきにん]	彼[かれ]は 失敗[しっぱい]の 責任[せきにん]を 取[と]って、 会社[かいしゃ]を 辞[や]めた。		
\\	願[ねが]う	あなたのプロジェクトの 成功[せいこう]を 願[ねが]っています。		
\\	発見[はっけん](~する)	新[あたら]しい 星[ほし]が 発見[はっけん]された。		
\\	気[き]	彼[かれ]は 意外[いがい]に 気[き]が 小[ちい]さい。		
\\	美容院[びよういん]	父[ちち]は 美容院[びよういん]で 髪[かみ]を 切[き]ります。		
\\	状況[じょうきょう]	うちの 会社[かいしゃ]は 厳[きび]しい 状況[じょうきょう]にあります。		
\\	目的[もくてき]	彼[かれ]が 来[き]た 目的[もくてき]が 分[わ]かりません。		
\\	企業[きぎょう]	彼女[かのじょ]はアメリカの 企業[きぎょう]で 働[はたら]いています。		
\\	釣[つ]り	今日[きょう]は 海[うみ]に 釣[つ]りに 行[い]きます。		
\\	漫画[まんが]	妹[いもうと]は 漫画[まんが]が 好[す]きです。		
\\	温泉[おんせん]	この 温泉[おんせん]はただです。		
\\	トレーニング(~する)	彼[かれ]はサッカー 選手[せんしゅ]になるためにトレーニングしています。		
\\	噂[うわさ](~する)	その 噂[うわさ]は 本当[ほんとう]ですか。		
\\	本物[ほんもの]	これは 本物[ほんもの]のダイヤモンドです。		
\\	アイデア	そのアイデアは 彼[かれ]の 提案[ていあん]です。		
\\	地下[ちか]	スタジオは 地下[ちか]にあります。		
\\	マーケット	国内[こくない]マーケットにもチャンスはあります。		
\\	スケジュール	スケジュールの 調整[ちょうせい]は 君[きみ]に 任[まか]せるわ。		
\\	出張[しゅっちょう](~する)	部長[ぶちょう]は 京都[きょうと]に 出張[しゅっちょう] 中[ちゅう]です。		
\\	コック	彼[かれ]は 腕[うで]のいいコックです。		
\\	正午[しょうご]	昼休[ひるやす]みは 正午[しょうご]からです。		
\\	ブラシ	このブラシでコートの 汚[よご]れを 落[お]とせます。		
\\	書[か]き 取[と]る	話[はなし]のポイントを 書[か]き 取[と]った。		
\\	歓迎[かんげい](~する)	温[あたた]かい 歓迎[かんげい]を 受[う]けました。		
\\	開発[かいはつ](~する)	私[わたし]のチームは 今[いま]、 新[あたら]しいゲームを 開発[かいはつ]しています。		
\\	郵便屋[ゆうびんや]さん	郵便屋[ゆうびんや]さんはもう 来[き]ましたか。		
\\	会長[かいちょう]	始[はじ]めに 会長[かいちょう]が 挨拶[あいさつ]した。		
\\	金[きん]	彼女[かのじょ]は 金[きん]のネックレスをしています。		
\\	灰[はい]	火事[かじ]ですべて 灰[はい]になった。		
\\	骨[ほね]	彼[かれ]は 足[あし]の 骨[ほね]を 折[お]りました。		
\\	風呂屋[ふろや]	昨日[きのう]、 友達[ともだち]とお 風呂屋[おふろや]さんに 行[い]った。		
\\	細[こま]か(~な)	それは 細[こま]かな 問題[もんだい]です。		
\\	行[ゆ]き	東京[とうきょう] 行[ゆ]きの 列車[れっしゃ]に 乗[の]った。		
\\	人々[ひとびと]	あの 村[むら]の 人々[ひとびと]はとても 親切[しんせつ]です。		
\\	寺[てら]	あそこに 古[ふる]いお 寺[てら]があります。		
\\	鶏[にわとり]	その 鶏[にわとり]は 毎朝[まいあさ] 鳴[な]きます。		
\\	港[みなと]	港[みなと]に 船[ふね]が 着[つ]きました。		
\\	移[うつ]る	彼女[かのじょ]は 新[あたら]しい 会社[かいしゃ]に 移[うつ]った。		
\\	異[こと]なる	彼[かれ]と 私[わたし]はいつも 意見[いけん]が 異[こと]なる。		
\\	提供[ていきょう](~する)	もうすぐこのサービスの 提供[ていきょう]が 中止[ちゅうし]される。		
\\	市民[しみん]	市民[しみん]の 安全[あんぜん]は 大切[たいせつ]だ。		
\\	患者[かんじゃ]	この 病院[びょういん]は 患者[かんじゃ]さんにとても 親切[しんせつ]です。		
\\	元旦[がんたん]	元旦[がんたん]に 彼[かれ]から 年賀状[ねんがじょう]がきた。		
\\	柔[やわ]らかい	布団[ふとん]がとても 柔[やわ]らかい。		
\\	似合[にあ]う	彼女[かのじょ]は 着物[きもの]がよく 似合[にあ]います。		
\\	確[たし]かめる	レシートで 買[か]い 物[もの]の 金額[きんがく]を 確[たし]かめたよ。		
\\	背中[せなか]	背中[せなか]に 痛[いた]みがあります。		
\\	肌[はだ]	彼女[かのじょ]は 肌[はだ]がとても 白[しろ]いですね。		
\\	順番[じゅんばん]	順番[じゅんばん]を 守[まも]ってください。 私[わたし]が 先[さき]です。		
\\	まとも(~な)	彼[かれ]はまともな 人[ひと]には 見[み]えません。		
\\	ぴったり	この 服[ふく]はあなたにぴったりですね。		
\\	思[おも]わず	嬉[うれ]しくて 思[おも]わず 涙[なみだ]が 出[で]ました。		
\\	より	空港[くうこう]へのアクセスがより 便利[べんり]になりました。		
\\	もったいない	食[た]べ 物[もの]を 残[のこ]してはもったいないわよ。		
\\	情報[じょうほう]	学生[がくせい]たちはインターネットでいろいろな 情報[じょうほう]を 集[あつ]めた。		
\\	諦[あきら]める	彼[かれ]は 留学[りゅうがく]の 夢[ゆめ]を 諦[あきら]めていないよ。		
\\	懐[なつ]かしい	ここは 私[わたし]にとって 懐[なつ]かしい 場所[ばしょ]です。		
\\	怪[あや]しい	その 男[おとこ]の 行動[こうどう]は 怪[あや]しかったわ。		
\\	示[しめ]す	あのプレゼンテーションは 彼女[かのじょ]の 能力[のうりょく]が 高[たか]いことを 示[しめ]していた。		
\\	たっぷり	お 鍋[なべ]にたっぷりお 湯[ゆ]を 入[い]れてください。		
\\	早速[さっそく]	新[あたら]しいお 店[みせ]に 早速[さっそく] 行[い]ってきました。		
\\	偶然[ぐうぜん]	街[まち]で 偶然[ぐうぜん] 友人[ゆうじん]に 会[あ]いました。		
\\	続[つづ]き	話[はなし]の 続[つづ]きは 電話[でんわ]でしましょう。		
\\	合計[ごうけい](~する)	合計[ごうけい] 金額[きんがく]を 計算[けいさん]してください。		
\\	傷[きず]	この 傷[きず]はなかなか 治[なお]りません。		
\\	面倒[めんどう](~な)	会社[かいしゃ]で 面倒[めんどう]なことが 起[お]こったの。		
\\	具合[ぐあい]	今日[きょう]は 具合[ぐあい]はどうですか。		
\\	中止[ちゅうし]	雨[あめ]で 運動会[うんどうかい]が 中止[ちゅうし]になったの。		
\\	手続[てつづ]き	入国[にゅうこく] 手続[てつづ]きが 終[お]わりました。		
\\	結果[けっか]	試合[しあい]の 結果[けっか]を 早[はや]く 知[し]りたい。		
\\	建設[けんせつ](~する)	新[あたら]しいビルの 建設[けんせつ]が 始[はじ]まった。		
\\	販売[はんばい](~する)	前売[まえうり] 券[けん]は 窓口[まどぐち]で 販売[はんばい]しています。		
\\	機能[きのう]	このソフトにはいろいろな 機能[きのう]があります。		
\\	頃[ころ]	子供[こども]の 頃[ころ]は 勉強[べんきょう]が 大嫌[だいきら]いでした。		
\\	日中[にっちゅう]	日中[にっちゅう]はずっと 海[うみ]で 泳[およ]いでいました。		
\\	今後[こんご]	この 国[くに]は 今後[こんご]、どうなってしまうんでしょう。		
\\	影響[えいきょう](~する)	私[わたし]は 彼[かれ]から 大[おお]きな 影響[えいきょう]を 受[う]けました。		
\\	分野[ぶんや]	音楽[おんがく]は 彼[かれ]の 得意[とくい]な 分野[ぶんや]です。		
\\	関[かん]する	その 問題[もんだい]に 関[かん]する 記事[きじ]を 読[よ]みました。		
\\	状態[じょうたい]	ここは 道[みち]の 状態[じょうたい]がとても 悪[わる]いです。		
\\	価格[かかく]	この 地域[ちいき]は 土地[とち]の 価格[かかく]が 下[さ]がっています。		
\\	姿[すがた]	遠[とお]くに 彼女[かのじょ]の 姿[すがた]が 見[み]えた。		
\\	期待[きたい](~する)	私[わたし]にはみんなの 期待[きたい]が 重過[おもす]ぎました。		
\\	除[のぞ]く	部長[ぶちょう]を 除[のぞ]く 全員[ぜんいん]が 土曜日[どようび]も 働[はたら]いた。		
\\	一般[いっぱん](~に)	その 車[くるま]は 一般[いっぱん] 市民[しみん]には 高[たか]いです。		
\\	表現[ひょうげん](~する)	彼[かれ]は 歌[うた]で 自分[じぶん]の 気持[きも]ちを 表現[ひょうげん]した。		
\\	完成[かんせい](~する)	新[あたら]しいホームページが 完成[かんせい]した。		
\\	変化[へんか](~する)	今週[こんしゅう]は 気温[きおん]の 変化[へんか]が 激[はげ]しいです。		
\\	参加[さんか](~する)	彼女[かのじょ]はデモに 参加[さんか]したよ。		
\\	土地[とち]	ここは 父[ちち]の 土地[とち]です。		
\\	検討[けんとう](~する)	今[いま]、 家[いえ]を 買[か]うか 検討[けんとう] 中[ちゅう]です。		
\\	(~か) 
\\	開始[かいし](~する)	運動会[うんどうかい]は9 時[じ] 開始[かいし]です。		
\\	備[そな]える	災害[さいがい]に 備[そな]えて 大量[たいりょう]の 水[みず]を 買[か]い 込[こ]んだ。		
\\	内容[ないよう]	この 本[ほん]の 内容[ないよう]を 説明[せつめい]してください。		
\\	与[あた]える	その 先生[せんせい]は 生徒[せいと]に 大[おお]きな 影響[えいきょう]を 与[あた]えました。		
\\	広[ひろ]がる	留学[りゅうがく]してから 私[わたし]の 世界[せかい]が 広[ひろ]がった。		
\\	(~て から) 
\\	事実[じじつ]	計画[けいかく]が 失敗[しっぱい]したのは 事実[じじつ]です。		
\\	決定[けってい](~する)	会議[かいぎ]で 重要[じゅうよう]な 決定[けってい]がありました。		
\\	住宅[じゅうたく]	ここは 静[しず]かな 住宅[じゅうたく] 地[ち]だ。		
\\	代表[だいひょう](~する)	彼[かれ]がクラスの 代表[だいひょう]だ。		
\\	命令[めいれい](~する)	あの 人[ひと]の 命令[めいれい]は 聞[き]きたくありません。		
\\	立場[たちば]	彼女[かのじょ]は 自分[じぶん]の 立場[たちば]を 分[わ]かっていない。		
\\	なお	なお、 平日[へいじつ]のランチタイムの 予約[よやく]は 受[う]け 付[つ]けていません。		
\\	計算[けいさん](~する)	その 計算[けいさん]は 間違[まちが]っている。		
\\	可能[かのう](~な)	
\\	時[じ] 前[まえ]にチェックインすることは 可能[かのう]ですか。		
\\	報告[ほうこく](~する)	昨日[きのう]の 会議[かいぎ]について 報告[ほうこく]があります。		
\\	(~について) 
\\	都市[とし]	東京[とうきょう]は 日本[にほん] 一[いち] 大[おお]きな 都市[とし]です。		
\\	政治[せいじ]	私[わたし]は 政治[せいじ]に 関心[かんしん]がある。		
\\	競争[きょうそう](~する)	その 分野[ぶんや]は 競争[きょうそう]が 激[はげ]しい。		
\\	原因[げんいん]	この 事故[じこ]の 原因[げんいん]は 何[なん]ですか。		
\\	活動[かつどう](~する)	彼[かれ]は 地域[ちいき]の 活動[かつどう]に 参加[さんか]した。		
\\	歴史[れきし]	私[わたし]は 日本[にほん]の 歴史[れきし]を 勉強[べんきょう]したいです。		
\\	表[ひょう]	この 表[ひょう]を 見[み]てください。		
\\	サイト	会社[かいしゃ]のサイトが 新[あたら]しくなりました。		
\\	土産[みやげ]	土産[みやげ]に 日本酒[にほんしゅ]をもらった。		
\\	臭[にお]い	この 牛乳[ぎゅうにゅう]は 変[へん]な 臭[にお]いがします。		
\\	現金[げんきん]	代金[だいきん]は 現金[げんきん]でお 願[ねが]いします。		
\\	虫[むし]	庭[にわ]で 秋[あき]の 虫[むし]が 鳴[な]いているね。		
\\	寄[よ]る	帰[かえ]りに 叔母[おば]の 家[いえ]に 寄[よ]ります。		
\\	田舎[いなか]	私[わたし]は 毎年[まいとし] 夏[なつ]に 田舎[いなか]に 帰[かえ]ります。		
\\	帰宅[きたく](~する)	夜[よる]の11 時[じ]に 帰宅[きたく]しました。		
\\	重大[じゅうだい](~な)	重大[じゅうだい]な 発表[はっぴょう]があります。		
\\	マイナス(~する)	外[そと]の 温度[おんど]はマイナス3 度[ど]です。		
\\	稼[かせ]ぐ	私[わたし]より 妻[つま]の 方[ほう]が 稼[かせ]いでいます。		
\\	(~より
\\	~方) 
\\	左右[さゆう]	左右[さゆう]を 見[み]てから 横断歩道[おうだんほどう]を 渡[わた]りなさい。		
\\	(~て
\\	から) 
\\	痛[いた]み	膝[ひざ]に 痛[いた]みを 感[かん]じます。		
\\	誤解[ごかい](~する)	彼[かれ]はあまりしゃべらないので、よく 誤解[ごかい]されます。		
\\	文句[もんく]	彼女[かのじょ]はいつも 文句[もんく]ばかり 言[い]う。		
\\	欧米[おうべい]	その 会社[かいしゃ]は 欧米[おうべい]に 進出[しんしゅつ]しているよね。		
\\	面接[めんせつ]	新[あたら]しい 仕事[しごと]の 面接[めんせつ]に 行[い]ってきました。		
\\	サイン(~する)	有名[ゆうめい]な 歌手[かしゅ]のサインをもらいました。		
\\	年中[ねんじゅう]	叔母[おば]は 年中[ねんじゅう] 旅行[りょこう]しています。		
\\	機嫌[きげん]	今日[きょう]は 妻[つま]の 機嫌[きげん]が 悪[わる]い。		
\\	居間[いま]	父[ちち]は 居間[いま]でテレビを 見[み]ている。		
\\	度々[たびたび]	彼[かれ]から 度々[たびたび]メールがきます。		
\\	例文[れいぶん]	例文[れいぶん]を 3[みっ]つ 作[つく]ってください。		
\\	怠[なま]ける	怠[なま]けていないで、 手伝[てつだ]って。		
\\	転勤[てんきん](~する)	彼[かれ]は 大阪[おおさか]に 転勤[てんきん]しました。		
\\	政府[せいふ]	そのデモについて 政府[せいふ]は 何[なに]もしなかった。		
\\	(~について) 
\\	所有[しょゆう](~する)	彼[かれ]は 車[くるま]を3 台[だい] 所有[しょゆう]している。		
\\	いずれ	いずれまたお 会[あ]いしましょう。		
\\	対[たい]する	その 質問[しつもん]に 対[たい]する 答[こた]えが 見[み]つからなかった。		
\\	教会[きょうかい]	私[わたし]たちは 教会[きょうかい]で 結婚[けっこん] 式[しき]をしました。		
\end{CJK}
\end{document}