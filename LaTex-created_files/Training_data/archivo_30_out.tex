\documentclass[8pt]{extreport} 
\usepackage{hyperref}
\usepackage{CJKutf8}
\begin{document}
\begin{CJK}{UTF8}{min}
\\	それはとってもいい話だ。	それ	それはとってもいい 話[はなし]だ。	
\\	それを一つください。	一つ	それを 一[ひと]つください。	
\\	一から始めましょう。	一	一[いち]から 始[はじ]めましょう。	
\\	その人には二回会った。	二	その 人[ひと]には 二[に] 回会[かい あ]った。	
\\	ソフトクリームを二つください。	二つ	ソフトクリームを 二[ふた]つください。	
\\	彼女は三人の子供の母親だ。	三	彼女[かのじょ]は 三[さん] 人[にん]の 子供[こども]の 母親[ははおや]だ。	
\\	コップを三つ買いました。	三つ	コップを 三[みっ]つ 買[か]いました。	
\\	私は腕時計を四つ持っています。	四つ	私[わたし]は 腕時計[うでどけい]を 四[よっ]つ 持[も]っています。	
\\	四月に大学に入学しました。	四	四[し] 月[がつ]に 大学[だいがく]に 入学[にゅうがく]しました。	
\\	これをください。	これ	これをください。	
\\	ハワイは四回目です。	四	ハワイは 四[よん] 回目[かいめ]です。	
\\	桃を五つください。	五つ	桃[もも]を 五[いつ]つください。	
\\	五人で旅行に行きました。	五	五[ご] 人[にん]で 旅行[りょこう]に 行[い]きました。	
\\	息子は六つになりました。	六つ	息子[むすこ]は 六[むっ]つになりました。	
\\	彼には子供が六人います。	六	彼[かれ]には 子供[こども]が 六[ろく] 人[にん]います。	
\\	バナナが七本あります。	七	バナナが 七[なな] 本[ほん]あります。	
\\	この子は今年七つになります。	七つ	この 子[こ]は 今年[ことし] 七[なな]つになります。	
\\	りんごを八個ください。	八	りんごを 八[はち] 個[こ]ください。	
\\	あの家には時計が八つあります。	八つ	あの 家[いえ]には 時計[とけい]が 八[やっ]つあります。	
\\	彼は医者になりました。	なる	彼[かれ]は 医者[いしゃ]になりました。	
\\	野球は九人で1チームです。	九	野球[やきゅう]は 九[きゅう] 人[にん]で 1[ひと]チームです。	
\\	娘は明日九つになります。	九つ	娘[むすめ]は 明日[あす] 九[ここの]つになります。	
\\	その子は指で十数えました。	十	その 子[こ]は 指[ゆび]で 十[じゅう] 数[かぞ]えました。	
\\	私の祖母は百才です。	百	私[わたし]の 祖母[そぼ]は 百[ひゃく] 才[さい]です。	
\\	千円貸してください。	千	千[せん] 円貸[えん か]してください。	
\\	この靴は1万円です。	万	この 靴[くつ]は 1[いち] 万[まん] 円[えん]です。	
\\	そこに大きな円を描いて。	円	そこに 大[おお]きな 円[えん]を 描[か]いて。	
\\	カレーライスは700円です。	円	カレーライスは 700[ななひゃく] 円[えん]です。	
\\	時の経つのは早い。	時	時[とき]の 経[た]つのは 早[はや]い。	
\\	友達と一緒に宿題をした。	する	友達[ともだち]と 一緒[いっしょ]に 宿題[しゅくだい]をした。	
\\	彼は時々遅刻します。	時々	彼[かれ]は 時々[ときどき] 遅刻[ちこく]します。	
\\	私たちは先月11日に結婚しました。	日	私[わたし]たちは 先月11[せんげつ じゅういち] 日[にち]に 結婚[けっこん]しました。	
\\	六日前に日本に帰ってきました。	六日	六日[むいか] 前[まえ]に 日本[にほん]に 帰[かえ]ってきました。	
\\	手紙が届くのに三日かかりました。	三日	手紙[てがみ]が 届[とど]くのに 三日[みっか]かかりました。	
\\	五月五日は祝日です	五日	五月[ごがつ] 五日[いつか]は 祝日[しゅくじつ]です	
\\	八日からイギリスに行きます。	八日	八日[ようか]からイギリスに 行[い]きます。	
\\	来月の二十日は弟の誕生日です。	二十日	来月[らいげつ]の 二十日[はつか]は 弟[おとうと]の 誕生日[たんじょうび]です。	
\\	私は二日待った。	二日	私[わたし]は 二日[ふつか] 待[ま]った。	
\\	九日に荷物が届きます。	九日	九日[ここのか]に 荷物[にもつ]が 届[とど]きます。	
\\	私は友達のところに泊まった。	ところ	私[わたし]は 友達[ともだち]のところに 泊[と]まった。	
\\	来月の一日は空いていますか。	一日	来月[らいげつ]の 一日[ついたち]は 空[あ]いていますか。	
\\	十日後に帰ります。	十日	十日[とおか] 後[ご]に 帰[かえ]ります。	
\\	先月の七日に孫が生まれました。	七日	先月[せんげつ]の 七日[なのか]に 孫[まご]が 生[う]まれました。	
\\	新学期は来月の四日からです。	四日	新学期[しんがっき]は 来月[らいげつ]の 四日[よっか]からです。	
\\	今夜は月がとてもきれいです。	月	今夜[こんや]は 月[つき]がとてもきれいです。	
\\	水を一杯ください。	水	水[みず]を 一杯[いっぱい]ください。	
\\	これはかなり金がかかった。	金	これはかなり 金[かね]がかかった。	
\\	日曜日は海に行きました。	日曜日	日曜日[にちようび]は 海[うみ]に 行[い]きました。	
\\	土曜日の夜はクラブに行きます。	土曜日	土曜日[どようび]の 夜[よる]はクラブに 行[い]きます。	
\\	一緒に宿題をやろう。	やる	一緒[いっしょ]に 宿題[しゅくだい]をやろう。	
\\	金曜日の夜は友達と出かけます。	金曜日	金曜日[きんようび]の 夜[よる]は 友達[ともだち]と 出[で]かけます。	
\\	月曜日に会いましょう。	月曜日	月曜日[げつようび]に 会[あ]いましょう。	
\\	木曜日は仕事が休みです。	木曜日	木曜日[もくようび]は 仕事[しごと]が 休[やす]みです。	
\\	曜日を間違えました。	曜日	曜日[ようび]を 間違[まちが]えました。	
\\	火曜日に会議があります。	火曜日	火曜日[かようび]に 会議[かいぎ]があります。	
\\	水曜日はバイトがあります。	水曜日	水曜日[すいようび]はバイトがあります。	
\\	その次の週は空いてますか。	週	その 次[つぎ]の 週[しゅう]は 空[あ]いてますか。	
\\	新しい年が始まりました。	年	新[あたら]しい 年[とし]が 始[はじ]まりました。	
\\	質問の意味は分かりましたか。	分かる	質問[しつもん]の 意味[いみ]は 分[わ]かりましたか。	
\\	私もそう思います。	そう	私[わたし]もそう 思[おも]います。	
\\	夕食には何を食べたいですか。	何	夕食[ゆうしょく]には 何[なに]を 食[た]べたいですか。	
\\	お先にどうぞ。	先	お 先[さき]にどうぞ。	
\\	今年はイタリアに旅行したい。	今年	今年[ことし]はイタリアに 旅行[りょこう]したい。	
\\	彼は今、勉強しています。	今	彼[かれ]は 今[いま]、 勉強[べんきょう]しています。	
\\	今日は仕事がありません。	今日	今日[きょう]は 仕事[しごと]がありません。	
\\	今月はとても忙しい。	今月	今月[こんげつ]はとても 忙[いそが]しい。	
\\	今週は日本語のテストがあります。	今週	今週[こんしゅう]は 日本語[にほんご]のテストがあります。	
\\	彼は昼過ぎに来ます。	来る	彼[かれ]は 昼過[ひるす]ぎに 来[き]ます。	
\\	来年一緒に旅行しましょう。	来年	来年[らいねん] 一緒[いっしょ]に 旅行[りょこう]しましょう。	
\\	彼はもう帰りました。	もう	彼[かれ]はもう 帰[かえ]りました。	
\\	日曜日は図書館に行きます。	行く	日曜日[にちようび]は 図書館[としょかん]に 行[い]きます。	
\\	家に帰ろう。	帰る	家[うち]に 帰[かえ]ろう。	
\\	あの大きい建物は何ですか。	大きい	あの 大[おお]きい 建物[たてもの]は 何[なん]ですか。	
\\	小さい花が咲いています。	小さい	小[ちい]さい 花[はな]が 咲[さ]いています。	
\\	今年は雨が少ないです。	少ない	今年[ことし]は 雨[あめ]が 少[すく]ないです。	
\\	少し疲れました。	少し	少[すこ]し 疲[つか]れました。	
\\	京都にはお寺が多い。	多い	京都[きょうと]にはお 寺[てら]が 多[おお]い。	
\\	彼女は多分家で寝ています。	多分	彼女[かのじょ]は 多分[たぶん] 家[いえ]で 寝[ね]ています。	
\\	彼は荷物をあみだなに上げた。	上げる	彼[かれ]は 荷物[にもつ]をあみだなに 上[あ]げた。	
\\	彼女はよく旅行に行きます。	よく	彼女[かのじょ]はよく 旅行[りょこう]に 行[い]きます。	
\\	猫が屋根に上っている。	上る	猫[ねこ]が 屋根[やね]に 上[のぼ]っている。	
\\	上を向いて。	上	上[うえ]を 向[む]いて。	
\\	財布は机の下にあった。	下	財布[さいふ]は 机[つくえ]の 下[した]にあった。	
\\	右のポケットにハンカチが入っています。	右	右[みぎ]のポケットにハンカチが 入[はい]っています。	
\\	そこを左に曲がってください。	左	そこを 左[ひだり]に 曲[ま]がってください。	
\\	彼は私の方を見ました。	方	彼[かれ]は 私[わたし]の 方[ほう]を 見[み]ました。	
\\	お酒は大人になってから。	大人	お 酒[さけ]は 大人[おとな]になってから。	
\\	彼は優しい人です。	人	彼[かれ]は 優[やさ]しい 人[ひと]です。	
\\	そこには私一人しかいなかった。	一人	そこには 私[わたし] 一人[ひとり]しかいなかった。	
\\	あなたはどう思いますか。	どう	あなたはどう 思[おも]いますか。	
\\	彼はかばんに手帳を入れた。	入れる	彼[かれ]はかばんに 手帳[てちょう]を 入[い]れた。	
\\	寒いので中に入ってください。	入る	寒[さむ]いので 中[なか]に 入[はい]ってください。	
\\	今朝は早く家を出ました。	出る	今朝[けさ]は 早[はや]く 家[いえ]を 出[で]ました。	
\\	彼女はイタリア語が出来ます。	出来る	彼女[かのじょ]はイタリア 語[ご]が 出来[でき]ます。	
\\	彼はかばんから教科書を出した。	出す	彼[かれ]はかばんから 教科書[きょうかしょ]を 出[だ]した。	
\\	本を1冊買いました。	本	本[ほん]を 1冊買[いっさつ か]いました。	
\\	明日、会社を休みます。	休む	明日[あした]、 会社[かいしゃ]を 休[やす]みます。	
\\	私は体が丈夫だ。	体	私[わたし]は 体[からだ]が 丈夫[じょうぶ]だ。	
\\	彼女は青い目をしています。	目	彼女[かのじょ]は 青[あお]い 目[め]をしています。	
\\	あの本をどこに置きましたか。	どこ	あの 本[ほん]をどこに 置[お]きましたか。	
\\	口を大きく開けてください。	口	口[くち]を 大[おお]きく 開[あ]けてください。	
\\	彼女は耳がよく聞こえません。	耳	彼女[かのじょ]は 耳[みみ]がよく 聞[き]こえません。	
\\	妹は歌が上手です。	上手	妹[いもうと]は 歌[うた]が 上手[じょうず]です。	
\\	分かった人は手を上げてください。	手	分[わ]かった 人[ひと]は 手[て]を 上[あ]げてください。	
\\	彼は足が長い。	足	彼[かれ]は 足[あし]が 長[なが]い。	
\\	レストランは空いていました。	空く	レストランは 空[す]いていました。	
\\	男の人が私たちに話しかけた。	男	男[おとこ]の 人[ひと]が 私[わたし]たちに 話[はな]しかけた。	
\\	店員は若い女の人でした。	女	店員[てんいん]は 若[わか]い 女[おんな]の 人[ひと]でした。	
\\	電車で子供が騒いでいた。	子供	電車[でんしゃ]で 子供[こども]が 騒[さわ]いでいた。	
\\	この本、あなたにあげます。	あげる	この 本[ほん]、あなたにあげます。	
\\	私はワインが好きです。	好き	私[わたし]はワインが 好[す]きです。	
\\	私は犬が大好きだ。	大好き	私[わたし]は 犬[いぬ]が 大好[だいす]きだ。	
\\	私が行きましょう。	私	私[わたし]が 行[い]きましょう。	
\\	私には友達がたくさんいます。	友達	私[わたし]には 友達[ともだち]がたくさんいます。	
\\	家に遊びに来てください。	家	家[うち]に 遊[あそ]びに 来[き]てください。	
\\	彼は意外に気が小さい。	気	彼[かれ]は 意外[いがい]に 気[き]が 小[ちい]さい。	
\\	おかげさまで元気です。	元気	おかげさまで 元気[げんき]です。	
\\	今日はいい天気ですね。	天気	今日[きょう]はいい 天気[てんき]ですね。	
\\	明日は晴れるといいですね。	晴れる	明日[あした]は 晴[は]れるといいですね。	
\\	こう小さい字は読めない。	こう	こう 小[ちい]さい 字[じ]は 読[よ]めない。	
\\	昨日、友達に会った。	昨日	昨日[きのう]、 友達[ともだち]に 会[あ]った。	
\\	窓を開けてください。	開ける	窓[まど]を 開[あ]けてください。	
\\	32ページを開いてください。	開く	32[さんじゅうに]ページを 開[ひら]いてください。	
\\	教科書を閉じてください。	閉じる	教科書[きょうかしょ]を 閉[と]じてください。	
\\	ちゃんとドアを閉めてよ。	閉める	ちゃんとドアを 閉[し]めてよ。	
\\	お店はもう閉まっていました。	閉まる	お 店[みせ]はもう 閉[し]まっていました。	
\\	彼女はラジオを聞いています。	聞く	彼女[かのじょ]はラジオを 聞[き]いています。	
\\	今は時間がありません。	時間	今[いま]は 時間[じかん]がありません。	
\\	これがこの町で一番高いビルです。	高い	これがこの 町[まち]で 一番[いちばん] 高[たか]いビルです。	
\\	友達が誕生日プレゼントをくれた。	くれる	友達[ともだち]が 誕生日[たんじょうび]プレゼントをくれた。	
\\	この服はとても安かった。	安い	この 服[ふく]はとても 安[やす]かった。	
\\	彼は背が低い。	低い	彼[かれ]は 背[せ]が 低[ひく]い。	
\\	その店の前で会いましょう。	前	その 店[みせ]の 前[まえ]で 会[あ]いましょう。	
\\	仕事の後、映画を見た。	後	仕事[しごと]の 後[あと]、 映画[えいが]を 見[み]た。	
\\	後ろを向いて。	後ろ	後[うし]ろを 向[む]いて。	
\\	明日の午後、お客様が来る。	午後	明日[あす]の 午後[ごご]、お 客様[きゃくさま]が 来[く]る。	
\\	午前9時のニュースです。	午前	午前[ごぜん] 9時[くじ]のニュースです。	
\\	気持ちのいい朝です。	朝	気持[きも]ちのいい 朝[あさ]です。	
\\	私は昼のドラマを毎日見ます。	昼	私[わたし]は 昼[ひる]のドラマを 毎日見[まいにち み]ます。	
\\	彼はかなり英語が上手です。	かなり	彼[かれ]はかなり 英語[えいご]が 上手[じょうず]です。	
\\	晩ご飯は食べましたか。	晩	晩[ばん]ご 飯[はん]は 食[た]べましたか。	
\\	今晩のパーティーは何時からですか。	今晩	今晩[こんばん]のパーティーは 何時[なんじ]からですか。	
\\	きのうの夜は家にいました。	夜	きのうの 夜[よる]は 家[いえ]にいました。	
\\	昨日タイカレーを食べました。	食べる	昨日[きのう]タイカレーを 食[た]べました。	
\\	友達とお酒を飲んでいます。	飲む	友達[ともだち]とお 酒[さけ]を 飲[の]んでいます。	
\\	私はパンよりご飯が好きだ。	ご飯	私[わたし]はパンよりご 飯[はん]が 好[す]きだ。	
\\	郵便局で切手を買いました。	買う	郵便局[ゆうびんきょく]で 切手[きって]を 買[か]いました。	
\\	私は絵を見るのが好きです。	見る	私[わたし]は 絵[え]を 見[み]るのが 好[す]きです。	
\\	その写真を見せてください。	見せる	その 写真[しゃしん]を 見[み]せてください。	
\\	もっと近くに来てください。	もっと	もっと 近[ちか]くに 来[き]てください。	
\\	新しい仕事を見つけました。	見つける	新[あたら]しい 仕事[しごと]を 見[み]つけました。	
\\	ここから富士山がよく見えます。	見える	ここから 富士山[ふじさん]がよく 見[み]えます。	
\\	メガネが見つかりません。	見つかる	メガネが 見[み]つかりません。	
\\	上司が「一杯、飲もう。」と言った。	言う	上司[じょうし]が
\\	一杯[いっぱい]、 飲[の]もう。」と 言[い]った。	
\\	母と電話で話しました。	話す	母[はは]と 電話[でんわ]で 話[はな]しました。	
\\	彼女は雑誌を読んでいます。	読む	彼女[かのじょ]は 雑誌[ざっし]を 読[よ]んでいます。	
\\	漢字は中国から来ました。	漢字	漢字[かんじ]は 中国[ちゅうごく]から 来[き]ました。	
\\	彼に手紙を書きました。	書く	彼[かれ]に 手紙[てがみ]を 書[か]きました。	
\\	妹は平仮名を全部覚えました。	覚える	妹[いもうと]は 平仮名[ひらがな]を 全部[ぜんぶ] 覚[おぼ]えました。	
\\	そこに座ってください。	そこ	そこに 座[すわ]ってください。	
\\	また会いましょう。	会う	また 会[あ]いましょう。	
\\	3月は仕事が忙しい。	仕事	3月[さんがつ]は 仕事[しごと]が 忙[いそが]しい。	
\\	分からない場合は私に聞いてください。	場合	分[わ]からない 場合[ばあい]は 私[わたし]に 聞[き]いてください。	
\\	弟が車を買った。	車	弟[おとうと]が 車[くるま]を 買[か]った。	
\\	私は電車で通学しています。	電車	私[わたし]は 電車[でんしゃ]で 通学[つうがく]しています。	
\\	駅はどこですか。	駅	駅[えき]はどこですか。	
\\	この道を真っ直ぐ行くと駅です。	道	この 道[みち]を 真[ま]っ 直[す]ぐ 行[い]くと 駅[えき]です。	
\\	他に方法がありません。	他	他[ほか]に 方法[ほうほう]がありません。	
\\	話すのを止めてください。	止める	話[はな]すのを 止[や]めてください。	
\\	ここに本があります。	ここ	ここに 本[ほん]があります。	
\\	駅まで歩きましょう。	歩く	駅[えき]まで 歩[ある]きましょう。	
\\	彼は毎晩3キロ走っています。	走る	彼[かれ]は 毎晩3[まいばん さん]キロ 走[はし]っています。	
\\	駅の近くで食事をした。	近く	駅[えき]の 近[ちか]くで 食事[しょくじ]をした。	
\\	駅はここから近いです。	近い	駅[えき]はここから 近[ちか]いです。	
\\	私の家は駅の近くです。	近く	私[わたし]の 家[いえ]は 駅[えき]の 近[ちか]くです。	
\\	家から学校までは遠いです。	遠い	家[いえ]から 学校[がっこう]までは 遠[とお]いです。	
\\	彼女の髪はとても長い。	長い	彼女[かのじょ]の 髪[かみ]はとても 長[なが]い。	
\\	彼は足が短い。	短い	彼[かれ]は 足[あし]が 短[みじか]い。	
\\	彼の家はとても広い。	広い	彼[かれ]の 家[いえ]はとても 広[ひろ]い。	
\\	彼女からプレゼントをもらいました。	もらう	彼女[かのじょ]からプレゼントをもらいました。	
\\	それ、全部ください。	全部	それ、 全部[ぜんぶ]ください。	
\\	私の国について少しお話しましょう。	国	私[わたし]の 国[くに]について 少[すこ]しお 話[はなし]しましょう。	
\\	彼は歯が白い。	白い	彼[かれ]は 歯[は]が 白[しろ]い。	
\\	赤いバラを買いました。	赤い	赤[あか]いバラを 買[か]いました。	
\\	私の部屋は2階にあります。	部屋	私[わたし]の 部屋[へや]は 2階[にかい]にあります。	
\\	日本人はお米が大好きです。	米	日本人[にほんじん]はお 米[こめ]が 大好[だいす]きです。	
\\	宿題は未だ終わっていません。	未だ	宿題[しゅくだい]は 未[ま]だ 終[お]わっていません。	
\\	私の机の上に書類がたくさん有ります。	有る	私[わたし]の 机[つくえ]の 上[うえ]に 書類[しょるい]がたくさん 有[あ]ります。	
\\	ここには何も無い。	無い	ここには 何[なに]も 無[な]い。	
\\	この本はとてもおもしろい。	とても	この 本[ほん]はとてもおもしろい。	
\\	今、朝ご飯を作っています。	作る	今[いま]、 朝[あさ]ご 飯[はん]を 作[つく]っています。	
\\	このパソコンを使ってください。	使う	このパソコンを 使[つか]ってください。	
\\	昼間は電気を消してください。	消す	昼間[ひるま]は 電気[でんき]を 消[け]してください。	
\\	彼は家を売った。	売る	彼[かれ]は 家[いえ]を 売[う]った。	
\\	私はこの店によく来ます。	店	私[わたし]はこの 店[みせ]によく 来[き]ます。	
\\	今年の春は暖かいね。	春	今年[ことし]の 春[はる]は 暖[あたた]かいね。	
\\	私は夏が大好き。	夏	私[わたし]は 夏[なつ]が 大好[だいす]き。	
\\	彼女は秋に結婚します。	秋	彼女[かのじょ]は 秋[あき]に 結婚[けっこん]します。	
\\	カナダの冬はとても寒いです。	冬	カナダの 冬[ふゆ]はとても 寒[さむ]いです。	
\\	みんなにお菓子をあげましょう。	みんな	みんなにお 菓子[かし]をあげましょう。	
\\	今日はとても暑い。	暑い	今日[きょう]はとても 暑[あつ]い。	
\\	このスープはとても熱い。	熱い	このスープはとても 熱[あつ]い。	
\\	この部屋は寒いです。	寒い	この 部屋[へや]は 寒[さむ]いです。	
\\	このコートはとても暖かい。	暖かい	このコートはとても 暖[あたた]かい。	
\\	彼の車は新しい。	新しい	彼[かれ]の 車[くるま]は 新[あたら]しい。	
\\	私は古い車が好きです。	古い	私[わたし]は 古[ふる]い 車[くるま]が 好[す]きです。	
\\	彼女は良い友達です。	良い	彼女[かのじょ]は 良[い]い 友達[ともだち]です。	
\\	たばこは体に悪い。	悪い	たばこは 体[からだ]に 悪[わる]い。	
\\	私もそう思います。	思う	私[わたし]もそう 思[おも]います。	
\\	彼女はいつも元気だ。	いつも	彼女[かのじょ]はいつも 元気[げんき]だ。	
\\	約束を忘れないでください。	忘れる	約束[やくそく]を 忘[わす]れないでください。	
\\	よく考えてください。	考える	よく 考[かんが]えてください。	
\\	帰国することに決めました。	決める	帰国[きこく]することに 決[き]めました。	
\\	旅行の日程が決まりました。	決まる	旅行[りょこう]の 日程[にってい]が 決[き]まりました。	
\\	誰もその話を知らない。	知る	誰[だれ]もその 話[はなし]を 知[し]らない。	
\\	彼はクラスで一番背が高い。	一番	彼[かれ]はクラスで 一番[いちばん] 背[せ]が 高[たか]い。	
\\	彼は会社の近くに住んでいる。	住む	彼[かれ]は 会社[かいしゃ]の 近[ちか]くに 住[す]んでいる。	
\\	あなたの名前を教えてください。	名前	あなたの 名前[なまえ]を 教[おし]えてください。	
\\	日本の食べ物はとても美味しいです。	食べ物	日本[にっぽん]の 食[た]べ 物[もの]はとても 美味[おい]しいです。	
\\	彼女はいつ来ますか。	いつ	彼女[かのじょ]はいつ 来[き]ますか。	
\\	何か飲み物が欲しいな。	飲み物	何[なに]か 飲[の]み 物[もの]が 欲[ほ]しいな。	
\\	このかばんは重いです。	重い	このかばんは 重[おも]いです。	
\\	この靴はとても軽い。	軽い	この 靴[くつ]はとても 軽[かる]い。	
\\	彼の家に荷物を送りました。	送る	彼[かれ]の 家[いえ]に 荷物[にもつ]を 送[おく]りました。	
\\	テストでいい点を取った。	取る	テストでいい 点[てん]を 取[と]った。	
\\	あなたが来るのを待っています。	待つ	あなたが 来[く]るのを 待[ま]っています。	
\\	私は車を持っています。	持つ	私[わたし]は 車[くるま]を 持[も]っています。	
\\	彼の気持ちが分からない。	気持ち	彼[かれ]の 気持[きも]ちが 分[わ]からない。	
\\	皆一生懸命生きている。	生きる	皆一生懸命[みんな いっしょうけんめい] 生[い]きている。	
\\	肉と魚とどちらが好きですか。	どちら	肉[にく]と 魚[さかな]とどちらが 好[す]きですか。	
\\	私は日本語の先生になりたいです。	先生	私[わたし]は 日本語[にほんご]の 先生[せんせい]になりたいです。	
\\	大学に行ってもっと勉強したいです。	大学	大学[だいがく]に 行[い]ってもっと 勉強[べんきょう]したいです。	
\\	彼は真面目な学生です。	学生	彼[かれ]は 真面目[まじめ]な 学生[がくせい]です。	
\\	姉は大学生です。	大学生	姉[あね]は 大学生[だいがくせい]です。	
\\	学校は8時半に始まります。	学校	学校[がっこう]は 8時半[はちじはん]に 始[はじ]まります。	
\\	私の弟は高校生です。	高校生	私[わたし]の 弟[おとうと]は 高校生[こうこうせい]です。	
\\	彼は数学を教えています。	教える	彼[かれ]は 数学[すうがく]を 教[おし]えています。	
\\	私は日本語を勉強しています。	勉強	私[わたし]は 日本語[にほんご]を 勉強[べんきょう]しています。	
\\	今日は風が強い。	強い	今日[きょう]は 風[かぜ]が 強[つよ]い。	
\\	この中でどれが好きですか。	どれ	この 中[なか]でどれが 好[す]きですか。	
\\	その子は体が少し弱い。	弱い	その 子[こ]は 体[からだ]が 少[すこ]し 弱[よわ]い。	
\\	このドアは引いてください。	引く	このドアは 引[ひ]いてください。	
\\	質問のある方はどうぞ。	質問	質問[しつもん]のある 方[かた]はどうぞ。	
\\	この本は難しいですね。	難しい	この 本[ほん]は 難[むずか]しいですね。	
\\	グラスの数が足りません。	数	グラスの 数[かず]が 足[た]りません。	
\\	今日はヤンキースが勝った。	勝つ	今日[きょう]はヤンキースが 勝[か]った。	
\\	私たちのチームはその試合で負けた。	負ける	私[わたし]たちのチームはその 試合[しあい]で 負[ま]けた。	
\\	あなたが本当に好きです。	本当に	あなたが 本当[ほんとう]に 好[す]きです。	
\\	予約は要りません。	要る	予約[よやく]は 要[い]りません。	
\\	ドアを開けてください。	ドア	ドアを 開[あ]けてください。	
\\	時計を見たらちょうど3時だった。	時計	時計[とけい]を 見[み]たらちょうど 3時[さんじ]だった。	
\\	私が払いましょう。	払う	私[わたし]が 払[はら]いましょう。	
\\	この紙を半分に切ってください。	切る	この 紙[かみ]を 半分[はんぶん]に 切[き]ってください。	
\\	旅行の日程を変えました。	変える	旅行[りょこう]の 日程[にってい]を 変[か]えました。	
\\	駅からはタクシーに乗ってください。	乗る	駅[えき]からはタクシーに 乗[の]ってください。	
\\	今日はスーツを着ています。	着る	今日[きょう]はスーツを 着[き]ています。	
\\	彼はステージに立った。	立つ	彼[かれ]はステージに 立[た]った。	
\\	私は窓側の席に座った。	座る	私[わたし]は 窓側[まどがわ]の 席[せき]に 座[すわ]った。	
\\	次はいつ会いましょうか。	次	次[つぎ]はいつ 会[あ]いましょうか。	
\\	彼女はよくしゃべるね。	しゃべる	彼女[かのじょ]はよくしゃべるね。	
\\	動かないで。	動く	動[うご]かないで 。	
\\	姉は銀行で働いています。	働く	姉[あね]は 銀行[ぎんこう]で 働[はたら]いています。	
\\	まだ学校へ行くには早い時間です。	早い	まだ 学校[がっこう]へ 行[い]くには 早[はや]い 時間[じかん]です。	
\\	彼は走るのが速い。	速い	彼[かれ]は 走[はし]るのが 速[はや]い。	
\\	前の車はとても遅い。	遅い	前[まえ]の 車[くるま]はとても 遅[おそ]い。	
\\	テストを始めてください。	始める	テストを 始[はじ]めてください。	
\\	新しい仕事が始まりました。	始まる	新[あたら]しい 仕事[しごと]が 始[はじ]まりました。	
\\	会議は4時に終わります。	終わる	会議[かいぎ]は 4時[よじ]に 終[お]わります。	
\\	夏休みももう終わりだ。	終わり	夏休[なつやす]みももう 終[お]わりだ。	
\\	新しいテーブルを買いました。	テーブル	新[あたら]しいテーブルを 買[か]いました。	
\\	私は去年フランスへ行った。	去年	私[わたし]は 去年[きょねん]フランスへ 行[い]った。	
\\	紙と鉛筆はありますか。	紙	紙[かみ]と 鉛筆[えんぴつ]はありますか。	
\\	今日は一人の時間を楽しみたい。	楽しむ	今日[きょう]は 一人[ひとり]の 時間[じかん]を 楽[たの]しみたい。	
\\	彼はとても楽しい人です。	楽しい	彼[かれ]はとても 楽[たの]しい 人[ひと]です。	
\\	私たちは大きな声で歌いました。	歌う	私[わたし]たちは 大[おお]きな 声[こえ]で 歌[うた]いました。	
\\	僕は新しい靴が欲しいです。	欲しい	僕[ぼく]は 新[あたら]しい 靴[くつ]が 欲[ほ]しいです。	
\\	すてきな色のセーターですね。	色	すてきな 色[いろ]のセーターですね。	
\\	彼女は茶色の靴を履いています。	茶色	彼女[かのじょ]は 茶色[ちゃいろ]の 靴[くつ]を 履[は]いています。	
\\	直ぐ行きます。	直ぐ	直[す]ぐ 行[い]きます。	
\\	私たちはビデオゲームをした。	ゲーム	私[わたし]たちはビデオゲームをした。	
\\	この書類を書き直してください。	書き直す	この 書類[しょるい]を 書[か]き 直[なお]してください。	
\\	クーラーはまだ直りません。	直る	クーラーはまだ 直[なお]りません。	
\\	赤い線を2本引いて下さい。	線	赤[あか]い 線[せん]を 2本引[に ほん ひ]いて 下[くだ]さい。	
\\	そこを左に曲がってください。	曲がる	そこを 左[ひだり]に 曲[ま]がってください。	
\\	彼の日本語のレベルは私と同じ位だ。	同じ	彼[かれ]の 日本語[にほんご]のレベルは 私[わたし]と 同[おな]じ 位[くらい]だ。	
\\	答えが違います。	違う	答[こた]えが 違[ちが]います。	
\\	図書館で料理の本を借りた。	図書館	図書館[としょかん]で 料理[りょうり]の 本[ほん]を 借[か]りた。	
\\	今日はこのホテルに泊まります。	泊まる	今日[きょう]はこのホテルに 泊[と]まります。	
\\	子供たちが公園で遊んでいる。	遊ぶ	子供[こども]たちが 公園[こうえん]で 遊[あそ]んでいる。	
\\	どうして泣いているの。	どうして	どうして 泣[な]いているの。	
\\	昨日、新しい服を買った。	服	昨日[きのう]、 新[あたら]しい 服[ふく]を 買[か]った。	
\\	お父さんは会社員です。	お父さん	お 父[とう]さんは 会社員[かいしゃいん]です。	
\\	私は父が大好きです。	父	私[わたし]は 父[ちち]が 大好[だいす]きです。	
\\	お母さんによろしくお伝えください。	お母さん	お 母[かあ]さんによろしくお 伝[つた]えください。	
\\	昨日、母と話をしました。	母	昨日[きのう]、 母[はは]と 話[はなし]をしました。	
\\	親の愛は有り難い。	親	親[おや]の 愛[あい]は 有[あ]り 難[がた]い。	
\\	姉は大学生です。	姉	姉[あね]は 大学生[だいがくせい]です。	
\\	昨日、あなたのお姉さんに会ったよ。	お姉さん	昨日[きのう]、あなたのお 姉[ねえ]さんに 会[あ]ったよ。	
\\	私の妹は小学生です。	妹	私[わたし]の 妹[いもうと]は 小学生[しょうがくせい]です。	
\\	おなかが空きました。	おなか	おなかが 空[す]きました。	
\\	兄は水泳が得意です。	兄	兄[あに]は 水泳[すいえい]が 得意[とくい]です。	
\\	あなたのお兄さんは何歳?	お兄さん	あなたのお 兄[にい]さんは 何歳?[なんさい]	
\\	弟は野球が好きです。	弟	弟[おとうと]は 野球[やきゅう]が 好[す]きです。	
\\	私の娘はアメリカにいます。	娘	私[わたし]の 娘[むすめ]はアメリカにいます。	
\\	うちの息子は大学1年生です。	息子	うちの 息子[むすこ]は 大学1年生[だいがく いちねんせい]です。	
\\	彼はまだ若いです。	若い	彼[かれ]はまだ 若[わか]いです。	
\\	彼女は
\\	です。	彼女	彼女[かのじょ]は 
\\	[おーえる]です。	
\\	彼は私の上司です。	彼	彼[かれ]は 私[わたし]の 上司[じょうし]です。	
\\	彼女は来月結婚します。	結婚	彼女[かのじょ]は 来月[らいげつ] 結婚[けっこん]します。	
\\	朝9時からミーティングが始まった。	ミーティング	朝9時[あさ 
\\	じ]からミーティングが 始[はじ]まった。	
\\	毎日、駅まで自転車で行きます。	自転車	毎日[まいにち]、 駅[えき]まで 自転車[じてんしゃ]で 行[い]きます。	
\\	宿題は自分でやりなさい。	自分	宿題[しゅくだい]は 自分[じぶん]でやりなさい。	
\\	ご主人はお元気ですか。	ご主人	ご 主人[しゅじん]はお 元気[げんき]ですか。	
\\	私の質問に答えてください。	答える	私[わたし]の 質問[しつもん]に 答[こた]えてください。	
\\	別の本も見せてください。	別	別[べつ]の 本[ほん]も 見[み]せてください。	
\\	祖父が病気になった。	病気	祖父[そふ]が 病気[びょうき]になった。	
\\	犬が病気で死にました。	死ぬ	犬[いぬ]が 病気[びょうき]で 死[し]にました。	
\\	今日は頭が痛いです。	痛い	今日[きょう]は 頭[あたま]が 痛[いた]いです。	
\\	彼女は酒に強い。	酒	彼女[かのじょ]は 酒[さけ]に 強[つよ]い。	
\\	電気をつけてください。	つける	電気[でんき]をつけてください。	
\\	プールは人で一杯です。	一杯	プールは 人[ひと]で 一杯[いっぱい]です。	
\\	鳥が飛んでいます。	飛ぶ	鳥[とり]が 飛[と]んでいます。	
\\	息子は飛行機のおもちゃが好きです。	飛行機	息子[むすこ]は 飛行機[ひこうき]のおもちゃが 好[す]きです。	
\\	お願いがあります。	お願い	お 願[ねが]いがあります。	
\\	工事は3月まで続きます。	続く	工事[こうじ]は 3月[さんがつ]まで 続[つづ]きます。	
\\	私は毎日、日記を付けています。	日記	私[わたし]は 毎日[まいにち]、 日記[にっき]を 付[つ]けています。	
\\	きりんの首は長い。	首	きりんの 首[くび]は 長[なが]い。	
\\	今朝から頭が痛い。	頭	今朝[けさ]から 頭[あたま]が 痛[いた]い。	
\\	彼はタオルで顔を拭きました。	顔	彼[かれ]はタオルで 顔[かお]を 拭[ふ]きました。	
\\	あれは何ですか。	あれ	あれは 何[なん]ですか。	
\\	膝に痛みを感じます。	感じる	膝[ひざ]に 痛[いた]みを 感[かん]じます。	
\\	彼は郵便局を探していました。	探す	彼[かれ]は 郵便局[ゆうびんきょく]を 探[さが]していました。	
\\	猿も木から落ちる。	落ちる	猿[さる]も 木[き]から 落[お]ちる。	
\\	お手洗いはどこですか。	お手洗い	お 手洗[てあら]いはどこですか。	
\\	冷たい飲み物をください。	冷たい	冷[つめ]たい 飲[の]み 物[もの]をください。	
\\	彼の部屋はとても汚い。	汚い	彼[かれ]の 部屋[へや]はとても 汚[きたな]い。	
\\	彼女は足が太い。	太い	彼女[かのじょ]は 足[あし]が 太[ふと]い。	
\\	明日は昼頃から曇るでしょう。	曇る	明日[あす]は 昼頃[ひるごろ]から 曇[くも]るでしょう。	
\\	彼は神を信じている。	神	彼[かれ]は 神[かみ]を 信[しん]じている。	
\\	ここの料理はまずい。	まずい	ここの 料理[りょうり]はまずい。	
\\	私たちは来年、家を建てます。	建てる	私[わたし]たちは 来年[らいねん]、 家[いえ]を 建[た]てます。	
\\	彼はかばんをいすの上に置きました。	置く	彼[かれ]はかばんをいすの 上[うえ]に 置[お]きました。	
\\	彼はこの辺に住んでいます。	辺	彼[かれ]はこの 辺[へん]に 住[す]んでいます。	
\\	彼女に黄色いバラを買いました。	黄色い	彼女[かのじょ]に 黄色[きいろ]いバラを 買[か]いました。	
\\	一緒に帰りましょう。	一緒に	一緒[いっしょ]に 帰[かえ]りましょう。	
\\	この町には緑がたくさんあります。	緑	この 町[まち]には 緑[みどり]がたくさんあります。	
\\	この問題はかなり易しいです。	易しい	この 問題[もんだい]はかなり 易[やさ]しいです。	
\\	彼は留学生です。	留学生	彼[かれ]は 留学生[りゅうがくせい]です。	
\\	今、会社に戻ります。	戻る	今[いま]、 会社[かいしゃ]に 戻[もど]ります。	
\\	そばにいて下さい。	そば	そばにいて 下[くだ]さい。	
\\	私は毎朝6時に起きます。	起きる	私[わたし]は 毎朝6時[まいあさ ろくじ]に 起[お]きます。	
\\	町で大事件が起こりました。	起こる	町[まち]で 大事件[だいじけん]が 起[お]こりました。	
\\	明日、6時に起こしてください。	起こす	明日[あす]、 6時[ろくじ]に 起[お]こしてください。	
\\	近所で盗難事件が起きました。	起きる	近所[きんじょ]で 盗難事件[とうなん じけん]が 起[お]きました。	
\\	もう寝よう。	寝る	もう 寝[ね]よう。	
\\	彼女は指が細いですね。	細い	彼女[かのじょ]は 指[ゆび]が 細[ほそ]いですね。	
\\	皿にケーキを載せました。	載せる	皿[さら]にケーキを 載[の]せました。	
\\	彼はシートベルトを締めた。	締める	彼[かれ]はシートベルトを 締[し]めた。	
\\	このイチゴは甘い。	甘い	このイチゴは 甘[あま]い。	
\\	こっちに来て下さい。	こっち	こっちに 来[き]て 下[くだ]さい。	
\\	彼は辛いものが好きです。	辛い	彼[かれ]は 辛[から]いものが 好[す]きです。	
\\	彼はとても優しい人です。	優しい	彼[かれ]はとても 優[やさ]しい 人[ひと]です。	
\\	私の夫はサラリーマンです。	夫	私[わたし]の 夫[おっと]はサラリーマンです。	
\\	今日は妻の誕生日だ。	妻	今日[きょう]は 妻[つま]の 誕生日[たんじょうび]だ。	
\\	誰と会ってみたいですか。	誰	誰[だれ]と 会[あ]ってみたいですか。	
\\	私は家族を愛しています。	愛する	私[わたし]は 家族[かぞく]を 愛[あい]しています。	
\\	赤ん坊が笑っています。	笑う	赤[あか]ん 坊[ぼう]が 笑[わら]っています。	
\\	このぶどうは酸っぱいです。	酸っぱい	このぶどうは 酸[す]っぱいです。	
\\	この言葉の意味が分かりません。	言葉	この 言葉[ことば]の 意味[いみ]が 分[わ]かりません。	
\\	彼女は眠いようです。	よう	彼女[かのじょ]は 眠[ねむ]いようです。	
\\	ウェイターを呼びましょう。	呼ぶ	ウェイターを 呼[よ]びましょう。	
\\	胸に少し痛みがあります。	胸	胸[むね]に 少[すこ]し 痛[いた]みがあります。	
\\	昨日から腰が痛い。	腰	昨日[きのう]から 腰[こし]が 痛[いた]い。	
\\	彼女は背が高い。	背	彼女[かのじょ]は 背[せ]が 高[たか]い。	
\\	お名前を片仮名で書いてください。	片仮名	お 名前[なまえ]を 片仮名[かたかな]で 書[か]いてください。	
\\	私は平仮名を全部読めます。	平仮名	私[わたし]は 平仮名[ひらがな]を 全部読[ぜんぶ よ]めます。	
\\	その映画はとても悲しかった。	悲しい	その 映画[えいが]はとても 悲[かな]しかった。	
\\	この絵は美しいです。	美しい	この 絵[え]は 美[うつく]しいです。	
\\	今日は日本語の授業があります。	授業	今日[きょう]は 日本語[にほんご]の 授業[じゅぎょう]があります。	
\\	あそこにバス停があります。	あそこ	あそこにバス 停[てい]があります。	
\\	私が手伝いましょう。	手伝う	私[わたし]が 手伝[てつだ]いましょう。	
\\	私の部屋は狭いです。	狭い	私[わたし]の 部屋[へや]は 狭[せま]いです。	
\\	絵に触らないでください。	触る	絵[え]に 触[さわ]らないでください。	
\\	私はタバコが嫌いです。	嫌い	私[わたし]はタバコが 嫌[きら]いです。	
\\	私は朝、シャワーを浴びます。	浴びる	私[わたし]は 朝[あさ]、シャワーを 浴[あ]びます。	
\\	喉が渇きました。	渇く	喉[のど]が 渇[かわ]きました。	
\\	髪の毛が伸びたね。	髪の毛	髪[かみ]の 毛[け]が 伸[の]びたね。	
\\	肩が凝りました。	肩	肩[かた]が 凝[こ]りました。	
\\	鼻がかゆいです。	鼻	鼻[はな]がかゆいです。	
\\	このケーキはおいしいね。	おいしい	このケーキはおいしいね。	
\\	彼の腕は太い。	腕	彼[かれ]の 腕[うで]は 太[ふと]い。	
\\	夫の服をハンガーに掛けた。	掛ける	夫[おっと]の 服[ふく]をハンガーに 掛[か]けた。	
\\	僕は学生です。	僕	僕[ぼく]は 学生[がくせい]です。	
\\	彼は駄目な男だ。	駄目	彼[かれ]は 駄目[だめ]な 男[おとこ]だ。	
\\	大丈夫ですか。	大丈夫	大丈夫[だいじょうぶ]ですか。	
\\	私は風邪を引きました。	風邪	私[わたし]は 風邪[かぜ]を 引[ひ]きました。	
\\	彼女はとても奇麗だ。	奇麗	彼女[かのじょ]はとても 奇麗[きれい]だ。	
\\	彼に会えて嬉しかった。	嬉しい	彼[かれ]に 会[あ]えて 嬉[うれ]しかった。	
\\	柿の木に実がたくさんなっています。	なる	柿[かき]の 木[き]に 実[み]がたくさんなっています。	
\\	これは医者のためのサイトです。	ため	これは 医者[いしゃ]のためのサイトです。	
\\	これから、より一層努力します。	より	これから、より 一層努力[いっそう どりょく]します。	
\\	そのグループのメンバーは全部で七人だ。	七	そのグループのメンバーは 全部[ぜんぶ]で 七[しち] 人[にん]だ。	
\\	私は九月に行く予定です。	九	私[わたし]は 九[く] 月[がつ]に 行[い]く 予定[よてい]です。	
\\	世界人口は65億人だ。	億	世界人口[せかい じんこう]は 65[ろくじゅうご] 億[おく] 人[にん]だ。	
\\	あそこに古いお寺があります。	寺	あそこに 古[ふる]いお 寺[てら]があります。	
\\	この服は300ドルしました。	ドル	この 服[ふく]は 300[さんびゃく]ドルしました。	
\\	夏は日が長い。	日	夏[なつ]は 日[ひ]が 長[なが]い。	
\\	火を消して。	火	火[ひ]を 消[け]して。	
\\	台風で木が倒れた。	木	台風[たいふう]で 木[き]が 倒[たお]れた。	
\\	彼女は金のネックレスをしています。	金	彼女[かのじょ]は 金[きん]のネックレスをしています。	
\\	全てのシステムが停止した。	システム	全[すべ]てのシステムが 停止[ていし]した。	
\\	お金はまだ十分あります。	十分	お 金[かね]はまだ 十分[じゅうぶん]あります。	
\\	このケーキはあなたの分です。	分	このケーキはあなたの 分[ぶん]です。	
\\	お菓子をみんなで分けました。	分ける	お 菓子[かし]をみんなで 分[わ]けました。	
\\	グループの中で意見が分かれました。	分かれる	グループの 中[なか]で 意見[いけん]が 分[わ]かれました。	
\\	新しいコンピューターを買った。	コンピューター	新[あたら]しいコンピューターを 買[か]った。	
\\	道に何か落ちています。	何か	道[みち]に 何[なに]か 落[お]ちています。	
\\	質問は何ですか。	何	質問[しつもん]は 何[なん]ですか。	
\\	帰ったら先ず手を洗いましょう。	先ず	帰[かえ]ったら 先[ま]ず 手[て]を 洗[あら]いましょう。	
\\	先月、友達の結婚式があった。	先月	先月[せんげつ]、 友達[ともだち]の 結婚式[けっこんしき]があった。	
\\	犬にえさをやった。	やる	犬[いぬ]にえさをやった。	
\\	先週は海に行った。	先週	先週[せんしゅう]は 海[うみ]に 行[い]った。	
\\	今までどこにいたのですか。	今まで	今[いま]までどこにいたのですか。	
\\	来月から大学生になります。	来月	来月[らいげつ]から 大学生[だいがくせい]になります。	
\\	続きは来週やりましょう。	来週	続[つづ]きは 来週[らいしゅう]やりましょう。	
\\	彼はロボットの研究をしています。	ロボット	彼[かれ]はロボットの 研究[けんきゅう]をしています。	
\\	その会社は来月、キャンペーンを行う。	行う	その 会社[かいしゃ]は 来月[らいげつ]、キャンペーンを 行[おこな]う。	
\\	行きは新幹線で行った。	行き	行[い]きは 新幹線[しんかんせん]で 行[い]った。	
\\	東京行きの列車に乗った。	行き	東京[とうきょう] 行[ゆ]きの 列車[れっしゃ]に 乗[の]った。	
\\	仕事の帰りにビールを飲んだ。	帰り	仕事[しごと]の 帰[かえ]りにビールを 飲[の]んだ。	
\\	お金がほとんどありません。	ほとんど	お 金[かね]がほとんどありません。	
\\	この大きさの封筒が欲しいのですが。	大きさ	この 大[おお]きさの 封筒[ふうとう]が 欲[ほ]しいのですが。	
\\	大分ピアノが上手くなりました。	大分	大分[だいぶ]ピアノが 上手[うま]くなりました。	
\\	財布は引き出しの中にあります。	中	財布[さいふ]は 引[ひ]き 出[だ]しの 中[なか]にあります。	
\\	少年たちがサッカーをしている。	少年	少年[しょうねん]たちがサッカーをしている。	
\\	彼はパソコンを2台持っています。	パソコン	彼[かれ]はパソコンを 2台持[にだい も]っています。	
\\	あなたは少しも悪くない。	少しも	あなたは 少[すこ]しも 悪[わる]くない。	
\\	塩を少々入れてください。	少々	塩[しお]を 少々[しょうしょう] 入[い]れてください。	
\\	毎年多くの人が海外へ旅行する。	多く	毎年[まいとし] 多[おお]くの 人[ひと]が 海外[かいがい]へ 旅行[りょこう]する。	
\\	私たちは2階に上がった。	上がる	私[わたし]たちは 2階[にかい]に 上[あ]がった。	
\\	コーヒーをもう一杯ください。	もう	コーヒーをもう 一杯[いっぱい]ください。	
\\	彼は人前だと上がってしまう。	上がる	彼[かれ]は 人前[ひとまえ]だと 上[あ]がってしまう。	
\\	彼は私より年上です。	年上	彼[かれ]は 私[わたし]より 年上[としうえ]です。	
\\	これは上り電車です。	上り	これは 上[のぼ]り 電車[でんしゃ]です。	
\\	少し音量を下げてください。	下げる	少[すこ]し 音量[おんりょう]を 下[さ]げてください。	
\\	受付でプログラムを受け取った。	プログラム	受付[うけつけ]でプログラムを 受[う]け 取[と]った。	
\\	やっと熱が下がった。	下がる	やっと 熱[ねつ]が 下[さ]がった。	
\\	そこの階段を下りてください。	下りる	そこの 階段[かいだん]を 下[お]りてください。	
\\	棚からその箱を下ろしてください。	下ろす	棚[たな]からその 箱[はこ]を 下[お]ろしてください。	
\\	もうすぐ下りの電車が発車します。	下り	もうすぐ 下[くだ]りの 電車[でんしゃ]が 発車[はっしゃ]します。	
\\	よく答えが分かりましたね。	よく	よく 答[こた]えが 分[わ]かりましたね。	
\\	彼は奥さんより年下です。	年下	彼[かれ]は 奥[おく]さんより 年下[としした]です。	
\\	小さな船が川を下っています。	下る	小[ちい]さな 船[ふね]が 川[かわ]を 下[くだ]っています。	
\\	ここは一方通行です。	一方	ここは 一方[いっぽう] 通行[つうこう]です。	
\\	次の方、どうぞ。	方	次[つぎ]の 方[かた]、どうぞ。	
\\	電気がついたままですよ。	まま	電気[でんき]がついたままですよ。	
\\	今日は妻と二人で食事をします。	二人	今日[きょう]は 妻[つま]と 二人[ふたり]で 食事[しょくじ]をします。	
\\	私の彼女はとても大人しいです。	大人しい	私[わたし]の 彼女[かのじょ]はとても 大人[おとな]しいです。	
\\	あの村の人々はとても親切です。	人々	あの 村[むら]の 人々[ひとびと]はとても 親切[しんせつ]です。	
\\	今日は一人で映画を見ます。	一人で	今日[きょう]は 一人[ひとり]で 映画[えいが]を 見[み]ます。	
\\	私はテレビをあまり見ません。	テレビ	私[わたし]はテレビをあまり 見[み]ません。	
\\	この町には外人が少ない。	外人	この 町[まち]には 外人[がいじん]が 少[すく]ない。	
\\	外は暑いよ。	外	外[そと]は 暑[あつ]いよ。	
\\	休日は家でよくテレビを見ます。	休日	休日[きゅうじつ]は 家[いえ]でよくテレビを 見[み]ます。	
\\	木曜日は仕事が休みです。	休み	木曜日[もくようび]は 仕事[しごと]が 休[やす]みです。	
\\	このソフトで日本語を勉強することができます。	ソフト	このソフトで 日本語[にほんご]を 勉強[べんきょう]することができます。	
\\	お相撲さんは、みんな力持ちだ。	力	お 相撲[すもう]さんは、みんな 力[ちから] 持[も]ちだ。	
\\	このプロジェクトにはみんなの協力が必要です。	協力	このプロジェクトにはみんなの 協力[きょうりょく]が 必要[ひつよう]です。	
\\	その国の人口はどのくらいですか。	人口	その 国[くに]の 人口[じんこう]はどのくらいですか。	
\\	出口はあそこです。	出口	出口[でぐち]はあそこです。	
\\	この温泉はただです。	ただ	この 温泉[おんせん]はただです。	
\\	入り口は向こうです。	入り口	入[い]り 口[ぐち]は 向[む]こうです。	
\\	私は右手で字を書きます。	右手	私[わたし]は 右手[みぎて]で 字[じ]を 書[か]きます。	
\\	彼女は左手で字を書く。	左手	彼女[かのじょ]は 左手[ひだりて]で 字[じ]を 書[か]く。	
\\	私は歌が下手だ。	下手	私[わたし]は 歌[うた]が 下手[へた]だ。	
\\	今日はこれらの問題について話し合います。	これら	今日[きょう]はこれらの 問題[もんだい]について 話[はな]し 合[あ]います。	
\\	お金が足りなくて買えなかった。	足りる	お 金[かね]が 足[た]りなくて 買[か]えなかった。	
\\	母は味噌汁に水を足した。	足す	母[はは]は 味噌汁[みそしる]に 水[みず]を 足[た]した。	
\\	山の空気はきれいだ。	山	山[やま]の 空気[くうき]はきれいだ。	
\\	小さな川を渡りました。	川	小[ちい]さな 川[かわ]を 渡[わた]りました。	
\\	いずれまたお会いしましょう。	いずれ	いずれまたお 会[あ]いしましょう。	
\\	後ろの席が空いています。	空く	後[うし]ろの 席[せき]が 空[あ]いています。	
\\	彼は空手を習っています。	空手	彼[かれ]は 空手[からて]を 習[なら]っています。	
\\	空が真っ青です。	空	空[そら]が 真[ま]っ 青[さお]です。	
\\	彼は海外での生活が長いです。	海外	彼[かれ]は 海外[かいがい]での 生活[せいかつ]が 長[なが]いです。	
\\	このビールはあまり美味しくありません。	あまり	このビールはあまり 美味[おい]しくありません。	
\\	海は広くて大きい。	海	海[うみ]は 広[ひろ]くて 大[おお]きい。	
\\	私たちは毎日散歩をします。	毎日	私[わたし]たちは 毎日[まいにち] 散歩[さんぽ]をします。	
\\	私は毎年、海外旅行に行きます。	毎年	私[わたし]は 毎年[まいとし]、 海外旅行[かいがい りょこう]に 行[い]きます。	
\\	毎年給料が上がる。	毎年	毎年[まいねん] 給料[きゅうりょう]が 上[あ]がる。	
\\	なお、雨の場合は中止です。	なお	なお、 雨[あめ]の 場合[ばあい]は 中止[ちゅうし]です。	
\\	私は毎週母に電話をします。	毎週	私[わたし]は 毎週[まいしゅう] 母[はは]に 電話[でんわ]をします。	
\\	私は毎月貯金をしています。	毎月	私[わたし]は 毎月[まいつき] 貯金[ちょきん]をしています。	
\\	私は石につまづいた。	石	私[わたし]は 石[いし]につまづいた。	
\\	この辺は田んぼがたくさんあります。	田んぼ	この 辺[へん]は 田[た]んぼがたくさんあります。	
\\	仕事がほぼ終わりました。	ほぼ	仕事[しごと]がほぼ 終[お]わりました。	
\\	きれいな花が咲きました。	花	きれいな 花[はな]が 咲[さ]きました。	
\\	私たちは林の中に入っていった。	林	私[わたし]たちは 林[はやし]の 中[なか]に 入[はい]っていった。	
\\	私は森を歩くのが好きです。	森	私[わたし]は 森[もり]を 歩[ある]くのが 好[す]きです。	
\\	その子は日本語が分からない。	子	その 子[こ]は 日本語[にほんご]が 分[わ]からない。	
\\	この店はサービスがいい。	サービス	この 店[みせ]はサービスがいい。	
\\	あの女の子を知っていますか。	女の子	あの 女[おんな]の 子[こ]を 知[し]っていますか。	
\\	男の子たちがサッカーをしている。	男の子	男[おとこ]の 子[こ]たちがサッカーをしている。	
\\	私たちは来月結婚します。	私たち	私[わたし]たちは 来月結婚[らいげつ けっこん]します。	
\\	気温は35度に達した。	達する	気温[きおん]は 35度[さんじゅうごど]に 達[たっ]した。	
\\	店に学生のグループが来た。	グループ	店[みせ]に 学生[がくせい]のグループが 来[き]た。	
\\	ここが私の家です。	家	ここが 私[わたし]の 家[いえ]です。	
\\	家内は九州出身です。	家内	家内[かない]は 九州出身[きゅうしゅう しゅっしん]です。	
\\	その店は若い客が多いです。	客	その 店[みせ]は 若[わか]い 客[きゃく]が 多[おお]いです。	
\\	ここは空気がきれいです。	空気	ここは 空気[くうき]がきれいです。	
\\	今ホテルに着きました。	ホテル	今[いま]ホテルに 着[つ]きました。	
\\	新しい靴がとても気に入りました。	気に入る	新[あたら]しい 靴[くつ]がとても 気[き]に 入[い]りました。	
\\	このバンドはとても人気があるよ。	人気	このバンドはとても 人気[にんき]があるよ。	
\\	雨が降っています。	雨	雨[あめ]が 降[ふ]っています。	
\\	クリスマスに雪が降りました。	雪	クリスマスに 雪[ゆき]が 降[ふ]りました。	
\\	彼女は荷物をまとめて出て行った。	まとめる	彼女[かのじょ]は 荷物[にもつ]をまとめて 出[で]て 行[い]った。	
\\	ここの海はとても青い。	青い	ここの 海[うみ]はとても 青[あお]い。	
\\	私の好きな色は青です。	青	私[わたし]の 好[す]きな 色[いろ]は 青[あお]です。	
\\	明日の天気は晴れです。	晴れ	明日[あす]の 天気[てんき]は 晴[は]れです。	
\\	明らかに彼が悪い。	明らか	明[あき]らかに 彼[かれ]が 悪[わる]い。	
\\	彼はやはり遅刻しました。	やはり	彼[かれ]はやはり 遅刻[ちこく]しました。	
\\	彼女は明るい性格です。	明るい	彼女[かのじょ]は 明[あか]るい 性格[せいかく]です。	
\\	明日、会社を休みます。	明日	明日[あした]、 会社[かいしゃ]を 休[やす]みます。	
\\	東の空が暗いです。	暗い	東[ひがし]の 空[そら]が 暗[くら]いです。	
\\	昨年は地震が多い年でした。	昨年	昨年[さくねん]は 地震[じしん]が 多[おお]い 年[とし]でした。	
\\	今日は山がはっきり見える。	はっきり	今日[きょう]は 山[やま]がはっきり 見[み]える。	
\\	一昨年初めて京都へ旅行しました。	一昨年	一昨年[おととし] 初[はじ]めて 京都[きょうと]へ 旅行[りょこう]しました。	
\\	一昨日彼から電話がありました。	一昨日	一昨日[おととい] 彼[かれ]から 電話[でんわ]がありました。	
\\	東の空が暗いです。	東	東[ひがし]の 空[そら]が 暗[くら]いです。	
\\	太陽は西に沈みます。	西	太陽[たいよう]は 西[にし]に 沈[しず]みます。	
\\	つまり、あなたは何も知らないのですね。	つまり	つまり、あなたは 何[なに]も 知[し]らないのですね。	
\\	私の家は町の南にあります。	南	私[わたし]の 家[いえ]は 町[まち]の 南[みなみ]にあります。	
\\	ロシアは日本の北にあります。	北	ロシアは 日本[にっぽん]の 北[きた]にあります。	
\\	あの人たちは皆、同じ方向を見ている。	方向	あの 人[ひと]たちは 皆[みな]、 同[おな]じ 方向[ほうこう]を 見[み]ている。	
\\	今、会社に向かっています。	向かう	今[いま]、 会社[かいしゃ]に 向[む]かっています。	
\\	私の会社はあのビルの8階です。	ビル	私[わたし]の 会社[かいしゃ]はあのビルの 8階[はちかい]です。	
\\	友達は向こうにいます。	向こう	友達[ともだち]は 向[む]こうにいます。	
\\	こっちを向いてください。	向く	こっちを 向[む]いてください。	
\\	電車のドアが開きました。	開く	電車[でんしゃ]のドアが 開[あ]きました。	
\\	隣の部屋からテレビの音が聞こえる。	聞こえる	隣[となり]の 部屋[へや]からテレビの 音[おと]が 聞[き]こえる。	
\\	もちろん一緒に行きます。	もちろん	もちろん 一緒[いっしょ]に 行[い]きます。	
\\	年間5万人がここを訪れます。	年間	年間[ねんかん] 5万人[ごまんにん]がここを 訪[おとず]れます。	
\\	この間彼女に会った。	この間	この 間[あいだ] 彼女[かのじょ]に 会[あ]った。	
\\	雲の間から月が出た。	間	雲[くも]の 間[あいだ]から 月[つき]が 出[で]た。	
\\	人間の心は複雑です。	人間	人間[にんげん]の 心[こころ]は 複雑[ふくざつ]です。	
\\	かつて私が学生だった頃のことです。	かつて	かつて 私[わたし]が 学生[がくせい]だった 頃[ころ]のことです。	
\\	富士山の高さは3,776メートルです。	高さ	富士山[ふじさん]の 高[たか]さは 
\\	776[さんぜんななひゃく-ななじゅうろく]メートルです。	
\\	これは世界最大の船です。	最大	これは 世界[せかい] 最大[さいだい]の 船[ふね]です。	
\\	東京に来るのは初めてです。	初めて	東京[とうきょう]に 来[く]るのは 初[はじ]めてです。	
\\	5ページの最初を見てください。	最初	5[ご]ページの 最初[さいしょ]を 見[み]てください。	
\\	あなたは何かスポーツをしていますか。	スポーツ	あなたは 何[なに]かスポーツをしていますか。	
\\	初めは上手くできませんでした。	初め	初[はじ]めは 上手[うま]くできませんでした。	
\\	今後ともよろしくお願いします。	今後	今後[こんご]ともよろしくお 願[ねが]いします。	
\\	後に彼は総理大臣になりました。	後	後[のち]に 彼[かれ]は 総理大臣[そうり だいじん]になりました。	
\\	今日が夏休み最後の日だ。	最後	今日[きょう]が 夏休[なつやす]み 最後[さいご]の 日[ひ]だ。	
\\	なぜ来なかったの。	なぜ	なぜ 来[こ]なかったの。	
\\	明後日は休日です。	明後日	明後日[あさって]は 休日[きゅうじつ]です。	
\\	牛が草を食べています。	牛	牛[うし]が 草[くさ]を 食[た]べています。	
\\	お菓子を友達に半分あげた。	半分	お 菓子[かし]を 友達[ともだち]に 半分[はんぶん]あげた。	
\\	日本に来て半年になります。	半年	日本[にっぽん]に 来[き]て 半年[はんとし]になります。	
\\	そのままお待ちください。	そのまま	そのままお 待[ま]ちください。	
\\	半月前に日本に来ました。	半月	半月[はんつき] 前[まえ]に 日本[にほん]に 来[き]ました。	
\\	今日は半日だけ仕事だ。	半日	今日[きょう]は 半日[はんにち]だけ 仕事[しごと]だ。	
\\	私は毎朝ジョギングをします。	毎朝	私[わたし]は 毎朝[まいあさ]ジョギングをします。	
\\	今朝から頭が痛い。	今朝	今朝[けさ]から 頭[あたま]が 痛[いた]い。	
\\	もし雨が降ったら、行きません。	もし	もし 雨[あめ]が 降[ふ]ったら、 行[い]きません。	
\\	昼休みに公園に行った。	昼休み	昼休[ひるやす]みに 公園[こうえん]に 行[い]った。	
\\	昼前に会議があった。	昼前	昼前[ひるまえ]に 会議[かいぎ]があった。	
\\	昼間は仕事で忙しいです。	昼間	昼間[ひるま]は 仕事[しごと]で 忙[いそが]しいです。	
\\	姉は毎晩日記を書いています。	毎晩	姉[あね]は 毎晩[まいばん] 日記[にっき]を 書[か]いています。	
\\	明日からタバコを止めるつもりです。	つもり	明日[あした]からタバコを 止[や]めるつもりです。	
\\	今夜は月がとてもきれいです。	今夜	今夜[こんや]は 月[つき]がとてもきれいです。	
\\	昨夜、流れ星を見ました。	昨夜	昨夜[ゆうべ]、 流[なが]れ 星[ぼし]を 見[み]ました。	
\\	夜中に電話がありました。	夜中	夜中[よなか]に 電話[でんわ]がありました。	
\\	夕方そちらに着きます。	夕方	夕方[ゆうがた]そちらに 着[つ]きます。	
\\	やっと仕事が終わりました。	やっと	やっと 仕事[しごと]が 終[お]わりました。	
\\	昼食に寿司を食べました。	昼食	昼食[ちゅうしょく]に 寿司[すし]を 食[た]べました。	
\\	朝食に納豆を食べました。	朝食	朝食[ちょうしょく]に 納豆[なっとう]を 食[た]べました。	
\\	夕食は7時です。	夕食	夕食[ゆうしょく]は 7時[しちじ]です。	
\\	夕飯は寿司でした。	夕飯	夕飯[ゆうはん]は 寿司[すし]でした。	
\\	夜のニュースを見ましたか。	ニュース	夜[よる]のニュースを 見[み]ましたか。	
\\	彼に対する見方が変わりました。	見方	彼[かれ]に 対[たい]する 見方[みかた]が 変[か]わりました。	
\\	友達と花見をしました。	花見	友達[ともだち]と 花見[はなみ]をしました。	
\\	そんな言い方をしてはいけません。	言い方	そんな 言[い]い 方[かた]をしてはいけません。	
\\	あなたの話は面白いね。	話	あなたの 話[はなし]は 面白[おもしろ]いね。	
\\	父は休みの日はずっとテレビを見ている。	ずっと	父[ちち]は 休[やす]みの 日[ひ]はずっとテレビを 見[み]ている。	
\\	母に読み書きを習いました。	読み	母[はは]に 読[よ]み 書[か]きを 習[なら]いました。	
\\	この漢字の読み方を教えてください。	読み方	この 漢字[かんじ]の 読[よ]み 方[かた]を 教[おし]えてください。	
\\	今夜は大いに語りましょう。	語る	今夜[こんや]は 大[おお]いに 語[かた]りましょう。	
\\	彼はアジアの言語を研究している。	言語	彼[かれ]はアジアの 言語[げんご]を 研究[けんきゅう]している。	
\\	私はその番組をビデオに撮った。	ビデオ	私[わたし]はその 番組[ばんぐみ]をビデオに 撮[と]った。	
\\	あなたは英語が話せますか。	英語	あなたは 英語[えいご]が 話[はな]せますか。	
\\	壁に文字が書いてあった。	文字	壁[かべ]に 文字[もじ]が 書[か]いてあった。	
\\	ローマ字で名前を書いてください。	ローマ字	ローマ 字[じ]で 名前[なまえ]を 書[か]いてください。	
\\	もっと大きく字を書いてください。	字	もっと 大[おお]きく 字[じ]を 書[か]いてください。	
\\	彼はマンションに住んでいます。	マンション	彼[かれ]はマンションに 住[す]んでいます。	
\\	彼はその漢字の書き方が分からない。	書き方	彼[かれ]はその 漢字[かんじ]の 書[か]き 方[かた]が 分[わ]からない。	
\\	今朝は6時に目が覚めた。	覚める	今朝[けさ]は 6時[ろくじ]に 目[め]が 覚[さ]めた。	
\\	子供が目を覚ました。	覚ます	子供[こども]が 目[め]を 覚[さ]ました。	
\\	夏には川辺で花火大会があります。	大会	夏[なつ]には 川辺[かわべ]で 花火[はなび] 大会[たいかい]があります。	
\\	そこでしばらく休んでいます。	しばらく	そこでしばらく 休[やす]んでいます。	
\\	親子の会話は大切です。	会話	親子[おやこ]の 会話[かいわ]は 大切[たいせつ]です。	
\\	私たちはよく話し合いました。	話し合う	私[わたし]たちはよく 話[はな]し 合[あ]いました。	
\\	この靴は私の足に合っている。	合う	この 靴[くつ]は 私[わたし]の 足[あし]に 合[あ]っている。	
\\	授業に間に合いました。	間に合う	授業[じゅぎょう]に 間[ま]に 合[あ]いました。	
\\	地震でガスが止まった。	ガス	地震[じしん]でガスが 止[と]まった。	
\\	彼は小さな会社に勤めています。	会社	彼[かれ]は 小[ちい]さな 会社[かいしゃ]に 勤[つと]めています。	
\\	これは大きな社会問題になっている。	社会	これは 大[おお]きな 社会[しゃかい] 問題[もんだい]になっている。	
\\	彼は優秀な社員です。	社員	彼[かれ]は 優秀[ゆうしゅう]な 社員[しゃいん]です。	
\\	ファイルのダウンロードの仕方が分かりません。	仕方	ファイルのダウンロードの 仕方[しかた]が 分[わ]かりません。	
\\	彼は野球がうまい。	うまい	彼[かれ]は 野球[やきゅう]がうまい。	
\\	今日は上司と食事をする。	食事	今日[きょう]は 上司[じょうし]と 食事[しょくじ]をする。	
\\	火事です。119番に電話してください。	火事	火事[かじ]です。 119番[ひゃくじゅうきゅうばん]に 電話[でんわ]してください。	
\\	お体をお大事に。	大事	お 体[からだ]をお 大事[だいじ]に。	
\\	彼は事故で怪我をしました。	事故	彼[かれ]は 事故[じこ]で 怪我[けが]をしました。	
\\	父はサラリーマンです。	サラリーマン	父[ちち]はサラリーマンです。	
\\	工事の音がうるさい。	工事	工事[こうじ]の 音[おと]がうるさい。	
\\	彼は食品工場で働いています。	工場	彼[かれ]は 食品[しょくひん] 工場[こうじょう]で 働[はたら]いています。	
\\	あとで電話します。	電話	あとで 電話[でんわ]します。	
\\	電気をつけてください。	電気	電気[でんき]をつけてください。	
\\	バスで行こう。	バス	バスで 行[い]こう。	
\\	この道は車が多い。	車	この 道[みち]は 車[くるま]が 多[おお]い。	
\\	駅員に聞きましょう。	駅員	駅員[えきいん]に 聞[き]きましょう。	
\\	この通りはにぎやかですね。	通り	この 通[とお]りはにぎやかですね。	
\\	毎日、この道を通ります。	通る	毎日[まいにち]、この 道[みち]を 通[とお]ります。	
\\	この学校は1クラス30人です。	クラス	この 学校[がっこう]は1クラス30 人[にん]です。	
\\	私はジムに通っています。	通う	私[わたし]はジムに 通[かよ]っています。	
\\	彼は交通事故を起こした。	交通事故	彼[かれ]は 交通事故[こうつうじこ]を 起[お]こした。	
\\	東京は水道の水が不味い。	水道	東京[とうきょう]は 水道[すいどう]の 水[みず]が 不味[まず]い。	
\\	車道の工事が始まりました。	車道	車道[しゃどう]の 工事[こうじ]が 始[はじ]まりました。	
\\	トラックを運転できますか。	トラック	トラックを 運転[うんてん]できますか。	
\\	この道路は3年前にできました。	道路	この 道路[どうろ]は 3年前[さんねんまえ]にできました。	
\\	ここは父の土地です。	土地	ここは 父[ちち]の 土地[とち]です。	
\\	地図を見て来てください。	地図	地図[ちず]を 見[み]て 来[き]てください。	
\\	留学生はアジア人が多く、その他は3割です。	他	留学生[りゅうがくせい]はアジア 人[じん]が 多[おお]く、その 他[た]は 3割[さんわり]です。	
\\	明日、うちでパーティーを開きます。	パーティー	明日[あした]、うちでパーティーを 開[ひら]きます。	
\\	車を止めて。	止める	車[くるま]を 止[と]めて。	
\\	今朝、事故で電車が止まりました。	止まる	今朝[けさ]、 事故[じこ]で 電車[でんしゃ]が 止[と]まりました。	
\\	雨が止みました。	止む	雨[あめ]が 止[や]みました。	
\\	歩道を歩きましょう。	歩道	歩道[ほどう]を 歩[ある]きましょう。	
\\	うまい寿司屋を見つけたよ。	うまい	うまい 寿司屋[すしや]を 見[み]つけたよ。	
\\	彼に手紙を渡しました。	渡す	彼[かれ]に 手紙[てがみ]を 渡[わた]しました。	
\\	私たちは歩いて橋を渡った。	渡る	私[わたし]たちは 歩[ある]いて 橋[はし]を 渡[わた]った。	
\\	売り上げは年度によって違います。	年度	売[う]り 上[あ]げは 年度[ねんど]によって 違[ちが]います。	
\\	今度はどこに行きたいですか。	今度	今度[こんど]はどこに 行[い]きたいですか。	
\\	クリスタルガラスの花瓶を買いました。	ガラス	クリスタルガラスの 花瓶[かびん]を 買[か]いました。	
\\	あの店には何度も行きました。	何度	あの 店[みせ]には 何度[なんど]も 行[い]きました。	
\\	それは最近話題の本ですね。	最近	それは 最近[さいきん] 話題[わだい]の 本[ほん]ですね。	
\\	遠くに船が見えます。	遠く	遠[とお]くに 船[ふね]が 見[み]えます。	
\\	おばは小さな会社の社長です。	社長	おばは 小[ちい]さな 会社[かいしゃ]の 社長[しゃちょう]です。	
\\	私は日本語コースを取っています。	コース	私[わたし]は 日本語[にほんご]コースを 取[と]っています。	
\\	初めに会長が挨拶した。	会長	初[はじ]めに 会長[かいちょう]が 挨拶[あいさつ]した。	
\\	このケーブルの長さは1メートルです。	長さ	このケーブルの 長[なが]さは 1[いち]メートルです。	
\\	彼の長男は8才です。	長男	彼[かれ]の 長男[ちょうなん]は 8才[はっさい]です。	
\\	うちの長女は10歳です。	長女	うちの 長女[ちょうじょ]は 10歳[じゅっさい]です。	
\\	彼はアパートに住んでいます。	アパート	彼[かれ]はアパートに 住[す]んでいます。	
\\	留学してから私の世界が広がった。	広がる	留学[りゅうがく]してから 私[わたし]の 世界[せかい]が 広[ひろ]がった。	
\\	その家の広さはどれ位ですか。	広さ	その 家[いえ]の 広[ひろ]さはどれ 位[くらい]ですか。	
\\	全体の80パーセントが完成しました。	全体	全体[ぜんたい]の 80[はちじゅっ]パーセントが 完成[かんせい]しました。	
\\	私には全く分かりません。	全く	私[わたし]には 全[まった]く 分[わ]かりません。	
\\	ジャズのレコードをかけました。	レコード	ジャズのレコードをかけました。	
\\	安全が第一です。	安全	安全[あんぜん]が 第一[だいいち]です。	
\\	計画を一部変更しましょう。	一部	計画[けいかく]を 一部[いちぶ] 変更[へんこう]しましょう。	
\\	この部分は問題ないです。	部分	この 部分[ぶぶん]は 問題[もんだい]ないです。	
\\	この携帯電話が使えるのは国内だけです。	国内	この 携帯電話[けいたい でんわ]が 使[つか]えるのは 国内[こくない]だけです。	
\\	ドアをどんどんとたたいた。	どんどん	ドアをどんどんとたたいた。	
\\	次は全国のお天気です。	全国	次[つぎ]は 全国[ぜんこく]のお 天気[てんき]です。	
\\	母はまだ外国に行ったことがありません。	外国	母[はは]はまだ 外国[がいこく]に 行[い]ったことがありません。	
\\	国会が再開した。	国会	国会[こっかい]が 再開[さいかい]した。	
\\	彼は帰国しました。	帰国	彼[かれ]は 帰国[きこく]しました。	
\\	カメラが壊れた。	カメラ	カメラが 壊[こわ]れた。	
\\	日本に住む外国人が増えています。	外国人	日本[にほん]に 住[す]む 外国人[がいこくじん]が 増[ふ]えています。	
\\	外国語を習うのは難しい。	外国語	外国語[がいこくご]を 習[なら]うのは 難[むずか]しい。	
\\	私は世界旅行をしたい。	世界	私[わたし]は 世界[せかい] 旅行[りょこう]をしたい。	
\\	白は雪の色です。	白	白[しろ]は 雪[ゆき]の 色[いろ]です。	
\\	彼女はその会話をテープに録音した。	テープ	彼女[かのじょ]はその 会話[かいわ]をテープに 録音[ろくおん]した。	
\\	彼女は黒いドレスを着ています。	黒い	彼女[かのじょ]は 黒[くろ]いドレスを 着[き]ています。	
\\	黒のボールペンはありますか。	黒	黒[くろ]のボールペンはありますか。	
\\	ベッドで赤ちゃんが眠っています。	赤ちゃん	ベッドで 赤[あか]ちゃんが 眠[ねむ]っています。	
\\	信号が赤に変わりました。	赤	信号[しんごう]が 赤[あか]に 変[か]わりました。	
\\	夏はビールがとても美味しい。	ビール	夏[なつ]はビールがとても 美味[おい]しい。	
\\	銀行は3時まで開いています。	銀行	銀行[ぎんこう]は 3時[さんじ]まで 開[あ]いています。	
\\	彼は銀メダルを取った。	銀	彼[かれ]は 銀[ぎん]メダルを 取[と]った。	
\\	私は地下鉄で通勤しています。	地下鉄	私[わたし]は 地下鉄[ちかてつ]で 通勤[つうきん]しています。	
\\	夕食に牛肉を買った。	牛肉	夕食[ゆうしょく]に 牛肉[ぎゅうにく]を 買[か]った。	
\\	32ページを開いてください。	ページ	32[さんじゅうに]ページを 開[ひら]いてください。	
\\	肉が焼けました。	肉	肉[にく]が 焼[や]けました。	
\\	肉と魚とどちらが好きですか。	魚	肉[にく]と 魚[さかな]とどちらが 好[す]きですか。	
\\	音楽は彼の得意な分野です。	分野	音楽[おんがく]は 彼[かれ]の 得意[とくい]な 分野[ぶんや]です。	
\\	私は毎日たくさん野菜を食べます。	野菜	私[わたし]は 毎日[まいにち]たくさん 野菜[やさい]を 食[た]べます。	
\\	彼は売上をグラフにした。	グラフ	彼[かれ]は 売上[うりあげ]をグラフにした。	
\\	駅前に本屋があります。	本屋	駅前[えきまえ]に 本屋[ほんや]があります。	
\\	八百屋でみかんを買いました。	八百屋	八百屋[やおや]でみかんを 買[か]いました。	
\\	昼はそば屋に行きました。	そば屋	昼[ひる]はそば 屋[や]に 行[い]きました。	
\\	私はたばこ屋でライターを買った。	たばこ屋	私[わたし]はたばこ 屋[や]でライターを 買[か]った。	
\\	手紙をポストに入れました。	ポスト	手紙[てがみ]をポストに 入[い]れました。	
\\	私たちは毎日お茶を飲みます。	茶	私[わたし]たちは 毎日[まいにち]お 茶[ちゃ]を 飲[の]みます。	
\\	お茶を入れましょうか。	お茶	お 茶[ちゃ]を 入[い]れましょうか。	
\\	茶わんにご飯をよそいました。	茶わん	茶[ちゃ]わんにご 飯[はん]をよそいました。	
\\	この料理は味が薄い。	味	この 料理[りょうり]は 味[あじ]が 薄[うす]い。	
\\	テストを始めてください。	テスト	テストを 始[はじ]めてください。	
\\	未来は誰にも分からない。	未来	未来[みらい]は 誰[だれ]にも 分[わ]からない。	
\\	週末は家でゆっくりします。	週末	週末[しゅうまつ]は 家[うち]でゆっくりします。	
\\	母は料理が得意です。	料理	母[はは]は 料理[りょうり]が 得意[とくい]です。	
\\	5時までに家に帰るのは無理です。	無理	5時[ごじ]までに 家[いえ]に 帰[かえ]るのは 無理[むり]です。	
\\	私たちは朝からあちこち散歩しました。	あちこち	私[わたし]たちは 朝[あさ]からあちこち 散歩[さんぽ]しました。	
\\	今日、鍵を無くしました。	無くす	今日[きょう]、 鍵[かぎ]を 無[な]くしました。	
\\	もうお金が無くなりました。	無くなる	もうお 金[かね]が 無[な]くなりました。	
\\	日本語で作文を書きました。	作文	日本語[にほんご]で 作文[さくぶん]を 書[か]きました。	
\\	彼はその詩を用いて自分の気持ちを伝えた。	用いる	彼[かれ]はその 詩[し]を 用[もち]いて 自分[じぶん]の 気持[きも]ちを 伝[つた]えた。	
\\	兄はゴルフを始めました。	ゴルフ	兄[あに]はゴルフを 始[はじ]めました。	
\\	父は用事で出掛けています。	用事	父[ちち]は 用事[ようじ]で 出掛[でか]けています。	
\\	会社までの交通費は一ヶ月8,000円です。	交通費	会社[かいしゃ]までの 交通費[こうつうひ]は 一ヶ月8,000円[いっかげつ はっせんえん]です。	
\\	突然、電気が消えた。	消える	突然[とつぜん]、 電気[でんき]が 消[き]えた。	
\\	消しゴムを貸して下さい。	消しゴム	消[け]しゴムを 貸[か]して 下[くだ]さい。	
\\	彼女はラジオを聞いています。	ラジオ	彼女[かのじょ]はラジオを 聞[き]いています。	
\\	今年の夏はクーラーがよく売れた。	売れる	今年[ことし]の 夏[なつ]はクーラーがよく 売[う]れた。	
\\	くつ売り場はどこですか。	売り場	くつ 売[う]り 場[ば]はどこですか。	
\\	あの店員はとても親切です。	店員	あの 店員[てんいん]はとても 親切[しんせつ]です。	
\\	駅の売店で雑誌を買った。	売店	駅[えき]の 売店[ばいてん]で 雑誌[ざっし]を 買[か]った。	
\\	タクシーを呼んでください。	タクシー	タクシーを 呼[よ]んでください。	
\\	この商品はよく売れている。	商品	この 商品[しょうひん]はよく 売[う]れている。	
\\	この絵はゴッホの作品だ。	作品	この 絵[え]はゴッホの 作品[さくひん]だ。	
\\	前売券は窓口で販売しています。	販売	前売券[まえうりけん]は 窓口[まどぐち]で 販売[はんばい]しています。	
\\	兄は二階にいます。	二階	兄[あに]は 二階[にかい]にいます。	
\\	もっとゆっくり話してください。	ゆっくり	もっとゆっくり 話[はな]してください。	
\\	この段階では、決断するのはまだ早い。	段階	この 段階[だんかい]では、 決断[けつだん]するのはまだ 早[はや]い。	
\\	私たちは駅の階段をかけ上がった。	階段	私[わたし]たちは 駅[えき]の 階段[かいだん]をかけ 上[あ]がった。	
\\	段々仕事が楽しくなってきました。	段々	段々[だんだん] 仕事[しごと]が 楽[たの]しくなってきました。	
\\	このベルトの値段は2500円でした。	値段	このベルトの 値段[ねだん]は 2500円[にせんごひゃくえん]でした。	
\\	レストランでインド料理を食べました。	レストラン	レストランでインド 料理[りょうり]を 食[た]べました。	
\\	ガソリンの価格がどんどん上がっている。	価格	ガソリンの 価格[かかく]がどんどん 上[あ]がっている。	
\\	娘が入学試験に合格しました。	合格	娘[むすめ]が 入学試験[にゅうがく しけん]に 合格[ごうかく]しました。	
\\	今日が夏休み最後の日だ。	夏休み	今日[きょう]が 夏休[なつやす]み 最後[さいご]の 日[ひ]だ。	
\\	冬休みにお祖父ちゃんの家に行きます。	冬休み	冬休[ふゆやす]みにお 祖父[じい]ちゃんの 家[うち]に 行[い]きます。	
\\	支払いはカードでお願いします。	カード	支払[しはら]いはカードでお 願[ねが]いします。	
\\	日本には四季がある。	四季	日本[にほん]には 四季[しき]がある。	
\\	今年の夏は暑さが厳しい。	暑さ	今年[ことし]の 夏[なつ]は 暑[あつ]さが 厳[きび]しい。	
\\	昨日の夜、熱が出ました。	熱	昨日[きのう]の 夜[よる]、 熱[ねつ]が 出[で]ました。	
\\	今日は厳しい寒さになるでしょう。	寒さ	今日[きょう]は 厳[きび]しい 寒[さむ]さになるでしょう。	
\\	兄はアルバイトをしています。	アルバイト	兄[あに]はアルバイトをしています。	
\\	今、車を暖めています。	暖める	今[いま]、 車[くるま]を 暖[あたた]めています。	
\\	まだ部屋が暖まらない。	暖まる	まだ 部屋[へや]が 暖[あたた]まらない。	
\\	今、部屋の温度は25度だ。	温度	今[いま]、 部屋[へや]の 温度[おんど]は 25度[にじゅうごど]だ。	
\\	今日の気温は26度です。	気温	今日[きょう]の 気温[きおん]は 26度[にじゅうろくど]です。	
\\	会議で書類のコピーを配った。	コピー	会議[かいぎ]で 書類[しょるい]のコピーを 配[くば]った。	
\\	そこにちょうど良い台がある。	台	そこにちょうど 良[い]い 台[だい]がある。	
\\	今日は風が強いです。	風	今日[きょう]は 風[かぜ]が 強[つよ]いです。	
\\	台風が近づいている。	台風	台風[たいふう]が 近[ちか]づいている。	
\\	あなたの事情はよく分かりました。	事情	あなたの 事情[じじょう]はよく 分[わ]かりました。	
\\	車が電柱にぶつかった。	ぶつかる	車[くるま]が 電柱[でんちゅう]にぶつかった。	
\\	学生たちはインターネットでいろいろな情報を集めた。	情報	学生[がくせい]たちはインターネットでいろいろな 情報[じょうほう]を 集[あつ]めた。	
\\	昨日の会議について報告があります。	報告	昨日[きのう]の 会議[かいぎ]について 報告[ほうこく]があります。	
\\	今日の新聞、どこに置いた?	新聞	今日[きょう]の 新聞[しんぶん]、どこに 置[お]いた?	
\\	新年明けましておめでとうございます。	新年	新年[しんねん] 明[あ]けましておめでとうございます。	
\\	旅行のためにたくさんフィルムを買った。	フィルム	旅行[りょこう]のためにたくさんフィルムを 買[か]った。	
\\	彼は良いところだけを強調した。	良い	彼[かれ]は 良[よ]いところだけを 強調[きょうちょう]した。	
\\	ここが建物の中心です。	中心	ここが 建物[たてもの]の 中心[ちゅうしん]です。	
\\	それを聞いて安心しました。	安心	それを 聞[き]いて 安心[あんしん]しました。	
\\	大切な用事を思い出しました。	思い出す	大切[たいせつ]な 用事[ようじ]を 思[おも]い 出[だ]しました。	
\\	私はデパートで靴を買った。	デパート	私[わたし]はデパートで 靴[くつ]を 買[か]った。	
\\	旅行で楽しい思い出ができました。	思い出	旅行[りょこう]で 楽[たの]しい 思[おも]い 出[で]ができました。	
\\	それは良い考えです。	考え	それは 良[い]い 考[かんが]えです。	
\\	トラブルがやっと解決した。	解決	トラブルがやっと 解決[かいけつ]した。	
\\	皆に会議の日にちを知らせた。	知らせる	皆[みんな]に 会議[かいぎ]の 日[ひ]にちを 知[し]らせた。	
\\	彼はベッドで寝ています。	ベッド	彼[かれ]はベッドで 寝[ね]ています。	
\\	彼は能力のある社員です。	能力	彼[かれ]は 能力[のうりょく]のある 社員[しゃいん]です。	
\\	20キロのダイエットは可能だと思いますか。	可能	20[にじゅっ]キロのダイエットは 可能[かのう]だと 思[おも]いますか。	
\\	このアルバイトは「学生可」ですね。	可	このアルバイトは
\\	学生[がくせい] 可[か]」ですね。	
\\	さっき郵便が届きました。	郵便	さっき 郵便[ゆうびん]が 届[とど]きました。	
\\	寒かったのでコートを着た。	コート	寒[さむ]かったのでコートを 着[き]た。	
\\	私の家は駅から遠くて不便です。	不便	私[わたし]の 家[いえ]は 駅[えき]から 遠[とお]くて 不便[ふべん]です。	
\\	郵便屋さんはもう来ましたか。	郵便屋さん	郵便屋[ゆうびんや]さんはもう 来[き]ましたか。	
\\	郵便局で切手を買いました。	郵便局	郵便局[ゆうびんきょく]で 切手[きって]を 買[か]いました。	
\\	あそこの交番で道を聞きましょう。	交番	あそこの 交番[こうばん]で 道[みち]を 聞[き]きましょう。	
\\	ノートを開いてください。	ノート	ノートを 開[ひら]いてください。	
\\	その建物の番地は分かりますか。	番地	その 建物[たてもの]の 番地[ばんち]は 分[わ]かりますか。	
\\	この番号に電話してください。	番号	この 番号[ばんごう]に 電話[でんわ]してください。	
\\	会社の場所を教えてください。	場所	会社[かいしゃ]の 場所[ばしょ]を 教[おし]えてください。	
\\	近所にカナダ人が住んでいる。	近所	近所[きんじょ]にカナダ 人[じん]が 住[す]んでいる。	
\\	ワインを少し飲みました。	ワイン	ワインを 少[すこ]し 飲[の]みました。	
\\	お母さんは台所にいます。	台所	お 母[かあ]さんは 台所[だいどころ]にいます。	
\\	この住所に行ってください。	住所	この 住所[じゅうしょ]に 行[い]ってください。	
\\	便所はそこです。	便所	便所[べんじょ]はそこです。	
\\	ボルドーはワインの生産で有名だ。	有名	ボルドーはワインの 生産[せいさん]で 有名[ゆうめい]だ。	
\\	彼の様子がおかしい。	おかしい	彼[かれ]の 様子[ようす]がおかしい。	
\\	あなたの名字は何ですか。	名字	あなたの 名字[みょうじ]は 何[なん]ですか。	
\\	ここに住所と氏名を書いてください。	氏名	ここに 住所[じゅうしょ]と 氏名[しめい]を 書[か]いてください。	
\\	各国の代表がニューヨークに集まった。	各国	各国[かっこく]の 代表[だいひょう]がニューヨークに 集[あつ]まった。	
\\	あの町は緑を増やしています。	町	あの 町[まち]は 緑[みどり]を 増[ふ]やしています。	
\\	トイレを掃除しましたか。	トイレ	トイレを 掃除[そうじ]しましたか。	
\\	東京は日本一大きな都市です。	都市	東京[とうきょう]は 日本一大[にほんいち おお]きな 都市[とし]です。	
\\	今日は都合が悪くて行けません。	都合	今日[きょう]は 都合[つごう]が 悪[わる]くて 行[い]けません。	
\\	七時に朝御飯を食べました。	朝御飯	七時[しちじ]に 朝御飯[あさごはん]を 食[た]べました。	
\\	母は買い物に出かけています。	買い物	母[はは]は 買[か]い 物[もの]に 出[で]かけています。	
\\	友達とキャンプに行った。	キャンプ	友達[ともだち]とキャンプに 行[い]った。	
\\	彼の家に荷物を送りました。	荷物	彼[かれ]の 家[いえ]に 荷物[にもつ]を 送[おく]りました。	
\\	その店は色々な品物を売っている。	品物	その 店[みせ]は 色々[いろいろ]な 品物[しなもの]を 売[う]っている。	
\\	皆で東京見物をした。	見物	皆[みんな]で 東京[とうきょう] 見物[けんぶつ]をした。	
\\	彼女の家には物がたくさんあります。	物	彼女[かのじょ]の 家[いえ]には 物[もの]がたくさんあります。	
\\	このカップはプラスチックです。	プラスチック	このカップはプラスチックです。	
\\	学校に忘れ物をしました。	忘れ物	学校[がっこう]に 忘[わす]れ 物[もの]をしました。	
\\	この荷物の重さを計ってください。	重さ	この 荷物[にもつ]の 重[おも]さを 計[はか]ってください。	
\\	会議で書類を配った。	配る	会議[かいぎ]で 書類[しょるい]を 配[くば]った。	
\\	彼は新聞配達をしている。	配達	彼[かれ]は 新聞[しんぶん] 配達[はいたつ]をしている。	
\\	カラーコピーは一枚幾らですか。	カラー	カラーコピーは 一枚幾[いちまい いく]らですか。	
\\	明日のプレゼンテーションが心配だ。	心配	明日[あす]のプレゼンテーションが 心配[しんぱい]だ。	
\\	彼が外国に行くのを見送りました。	見送る	彼[かれ]が 外国[がいこく]に 行[い]くのを 見送[みおく]りました。	
\\	駅に友人の見送りに行きました。	見送り	駅[えき]に 友人[ゆうじん]の 見送[みおく]りに 行[い]きました。	
\\	彼は就職試験を受けた。	受ける	彼[かれ]は 就職試験[しゅうしょく しけん]を 受[う]けた。	
\\	昔、ピアノを習っていました。	ピアノ	昔[むかし]、ピアノを 習[なら]っていました。	
\\	彼からメールを受け取りました。	受け取る	彼[かれ]からメールを 受[う]け 取[と]りました。	
\\	シャツのボタンが取れた。	取れる	シャツのボタンが 取[と]れた。	
\\	話しのポイントを書き取った。	書き取る	話[はな]しのポイントを 書[か]き 取[と]った。	
\\	昨日、父から手紙が届いた。	届く	昨日[きのう]、 父[ちち]から 手紙[てがみ]が 届[とど]いた。	
\\	冬はよくスキーに行きます。	スキー	冬[ふゆ]はよくスキーに 行[い]きます。	
\\	これを彼に届けてください。	届ける	これを 彼[かれ]に 届[とど]けてください。	
\\	この車はよく持っているね。	持つ	この 車[くるま]はよく 持[も]っているね。	
\\	彼は金持ちです。	金持ち	彼[かれ]は 金持[かねも]ちです。	
\\	水を持って行きましょう。	持って行く	水[みず]を 持[も]って 行[い]きましょう。	
\\	荷物がなかなか届きません。	なかなか	荷物[にもつ]がなかなか 届[とど]きません。	
\\	そのいすを持って来てください。	持って来る	そのいすを 持[も]って 来[き]てください。	
\\	転んでひざを打ちました。	打つ	転[ころ]んでひざを 打[う]ちました。	
\\	ボールをこっちに投げてください。	投げる	ボールをこっちに 投[な]げてください。	
\\	姉夫婦に男の子が生まれました。	生まれる	姉夫婦[あね ふうふ]に 男[おとこ]の 子[こ]が 生[う]まれました。	
\\	私は夏休みにプールに行った。	プール	私[わたし]は 夏休[なつやす]みにプールに 行[い]った。	
\\	うちのネコが子猫を生みました。	生む	うちのネコが 子猫[こねこ]を 生[う]みました。	
\\	そのパーティーに女性は何人来ますか。	女性	そのパーティーに 女性[じょせい]は 何人来[なんにん き]ますか。	
\\	ボルドーはワインの生産で有名だ。	生産	ボルドーはワインの 生産[せいさん]で 有名[ゆうめい]だ。	
\\	父はお土産にお菓子を買ってきた。	お土産	父[ちち]はお 土産[みやげ]にお 菓子[かし]を 買[か]ってきた。	
\\	もうすぐこのホームに電車が来ます。	ホーム	もうすぐこのホームに 電車[でんしゃ]が 来[き]ます。	
\\	日本での生活は楽しいです。	生活	日本[にほん]での 生活[せいかつ]は 楽[たの]しいです。	
\\	このクラスの生徒は30人です。	生徒	このクラスの 生徒[せいと]は 30人[さんじゅうにん]です。	
\\	息子は中学に通っています。	中学	息子[むすこ]は 中学[ちゅうがく]に 通[かよ]っています。	
\\	妹は九月にアメリカの大学に入学します。	入学	妹[いもうと]は 九月[くがつ]にアメリカの 大学[だいがく]に 入学[にゅうがく]します。	
\\	エレベーターで下に降りましょう。	エレベーター	エレベーターで 下[した]に 降[お]りましょう。	
\\	息子は中学生です。	中学生	息子[むすこ]は 中学生[ちゅうがくせい]です。	
\\	うちの息子は来年、小学生になります。	小学生	うちの 息子[むすこ]は 来年[らいねん]、 小学生[しょうがくせい]になります。	
\\	今日、工場の見学に行きました。	見学	今日[きょう]、 工場[こうじょう]の 見学[けんがく]に 行[い]きました。	
\\	毎朝、通学に1時間かかる。	通学	毎朝[まいあさ]、 通学[つうがく]に 1時間[いちじかん]かかる。	
\\	メモを取ってください。	メモ	メモを 取[と]ってください。	
\\	妹は高校に通っています。	高校	妹[いもうと]は 高校[こうこう]に 通[かよ]っています。	
\\	家の近くに小学校があります。	小学校	家[いえ]の 近[ちか]くに 小学校[しょうがっこう]があります。	
\\	息子の中学校は家から5分です。	中学校	息子[むすこ]の 中学校[ちゅうがっこう]は 家[いえ]から 5分[ごふん]です。	
\\	あの人は高校の校長だ。	校長	あの 人[ひと]は 高校[こうこう]の 校長[こうちょう]だ。	
\\	朝ご飯にはいつもパンを食べる。	パン	朝[あさ]ご 飯[はん]にはいつもパンを 食[た]べる。	
\\	学校は今週は休校です。	休校	学校[がっこう]は 今週[こんしゅう]は 休校[きゅうこう]です。	
\\	私たちは教会で結婚式をしました。	教会	私[わたし]たちは 教会[きょうかい]で 結婚式[けっこんしき]をしました。	
\\	彼は海外で教育を受けました。	教育	彼[かれ]は 海外[かいがい]で 教育[きょういく]を 受[う]けました。	
\\	彼女は三人の子を育てました。	育てる	彼女[かのじょ]は 三人[さんにん]の 子[こ]を 育[そだ]てました。	
\\	大きな音にびっくりしました。	びっくりする	大[おお]きな 音[おと]にびっくりしました。	
\\	野菜がよく育っている。	育つ	野菜[やさい]がよく 育[そだ]っている。	
\\	来年から新しい制度が始まります。	制度	来年[らいねん]から 新[あたら]しい 制度[せいど]が 始[はじ]まります。	
\\	風の強さに驚きました。	強さ	風[かぜ]の 強[つよ]さに 驚[おどろ]きました。	
\\	私たちは中国の会社と取引しています。	取引	私[わたし]たちは 中国[ちゅうごく]の 会社[かいしゃ]と 取引[とりひき]しています。	
\\	ズボンが汚れた。	ズボン	ズボンが 汚[よご]れた。	
\\	財布は引き出しの中にあります。	引き出し	財布[さいふ]は 引[ひ]き 出[だ]しの 中[なか]にあります。	
\\	ドアを押さえてください。	押さえる	ドアを 押[お]さえてください。	
\\	布団を押し入れにしまいました。	押し入れ	布団[ふとん]を 押[お]し 入[い]れにしまいました。	
\\	娘は今、バイオリンを練習しています。	練習	娘[むすめ]は 今[いま]、バイオリンを 練習[れんしゅう]しています。	
\\	赤ちゃんが自動車のおもちゃで遊んでいる。	おもちゃ	赤[あか]ちゃんが 自動車[じどうしゃ]のおもちゃで 遊[あそ]んでいる。	
\\	彼は空手を習っています。	習う	彼[かれ]は 空手[からて]を 習[なら]っています。	
\\	新しい家にはもう慣れましたか。	慣れる	新[あたら]しい 家[いえ]にはもう 慣[な]れましたか。	
\\	毎朝コーヒーを飲むのが習慣です。	習慣	毎朝[まいあさ]コーヒーを 飲[の]むのが 習慣[しゅうかん]です。	
\\	彼は何年も地震の研究をしている。	研究	彼[かれ]は 何年[なんねん]も 地震[じしん]の 研究[けんきゅう]をしている。	
\\	ひき肉を200グラムください。	グラム	ひき 肉[にく]を 200[にひゃく]グラムください。	
\\	明日の試験、頑張ってね。	試験	明日[あした]の 試験[しけん]、 頑張[がんば]ってね。	
\\	問題が一つあります。	問題	問題[もんだい]が 一[ひと]つあります。	
\\	この料理はとても簡単です。	簡単	この 料理[りょうり]はとても 簡単[かんたん]です。	
\\	このプログラムはとても複雑です。	複雑	このプログラムはとても 複雑[ふくざつ]です。	
\\	私は毎朝コーヒーを飲みます。	コーヒー	私[わたし]は 毎朝[まいあさ]コーヒーを 飲[の]みます。	
\\	数字は苦手です。	数字	数字[すうじ]は 苦手[にがて]です。	
\\	兄は数学の先生です。	数学	兄[あに]は 数学[すうがく]の 先生[せんせい]です。	
\\	いすの数を数えてください。	数える	いすの 数[かず]を 数[かぞ]えてください。	
\\	まあ今回は許してあげよう。	今回	まあ 今回[こんかい]は 許[ゆる]してあげよう。	
\\	みんなでテントを張りました。	テント	みんなでテントを 張[は]りました。	
\\	月は地球のまわりを回っています。	回る	月[つき]は 地球[ちきゅう]のまわりを 回[まわ]っています。	
\\	ねじは左に回すと外れます。	回す	ねじは 左[ひだり]に 回[まわ]すと 外[はず]れます。	
\\	これは私個人の意見です。	個人	これは 私[わたし] 個人[こじん]の 意見[いけん]です。	
\\	私はセールスを担当しています。	担当	私[わたし]はセールスを 担当[たんとう]しています。	
\\	池でボートに乗りました。	ボート	池[いけ]でボートに 乗[の]りました。	
\\	ボールが彼の頭に当たった。	当たる	ボールが 彼[かれ]の 頭[あたま]に 当[あ]たった。	
\\	彼女は当時、まだ3才だった。	当時	彼女[かのじょ]は 当時[とうじ]、まだ 3才[さんさい]だった。	
\\	その話は本当ですか。	本当	その 話[はなし]は 本当[ほんとう]ですか。	
\\	彼女が怒るのも当然だ。	当然	彼女[かのじょ]が 怒[おこ]るのも 当然[とうぜん]だ。	
\\	彼はボールを投げた。	ボール	彼[かれ]はボールを 投[な]げた。	
\\	この本は全然面白くなかった。	全然	この 本[ほん]は 全然[ぜんぜん] 面白[おもしろ]くなかった。	
\\	いい方法を思いつきました。	方法	いい 方法[ほうほう]を 思[おも]いつきました。	
\\	新しい法律ができた。	法律	新[あたら]しい 法律[ほうりつ]ができた。	
\\	あの会社の規則は厳しいです。	規則	あの 会社[かいしゃ]の 規則[きそく]は 厳[きび]しいです。	
\\	彼はオートバイに乗っている。	オートバイ	彼[かれ]はオートバイに 乗[の]っている。	
\\	今日の試合はいい経験になりました。	経験	今日[きょう]の 試合[しあい]はいい 経験[けいけん]になりました。	
\\	あれから14年が経ちました。	経つ	あれから 14年[じゅうよねん]が 経[た]ちました。	
\\	大学で経済を勉強しました。	経済	大学[だいがく]で 経済[けいざい]を 勉強[べんきょう]しました。	
\\	我が社の経営はうまくいっています。	経営	我[わ]が 社[しゃ]の 経営[けいえい]はうまくいっています。	
\\	妹とひどい喧嘩をした。	ひどい	妹[いもうと]とひどい 喧嘩[けんか]をした。	
\\	最近株を始めました。	株	最近[さいきん] 株[かぶ]を 始[はじ]めました。	
\\	彼女はアメリカの企業で働いています。	企業	彼女[かのじょ]はアメリカの 企業[きぎょう]で 働[はたら]いています。	
\\	作業するにはもっと広いスペースが必要だ。	作業	作業[さぎょう]するにはもっと 広[ひろ]いスペースが 必要[ひつよう]だ。	
\\	日本の自動車産業は世界的に有名だ。	産業	日本[にほん]の 自動車[じどうしゃ] 産業[さんぎょう]は 世界的[せかいてき]に 有名[ゆうめい]だ。	
\\	この本、あなたにあげます。	あなた	この 本[ほん]、あなたにあげます。	
\\	そこは工業都市だ。	工業	そこは 工業[こうぎょう] 都市[とし]だ。	
\\	この町では商業が盛んだ。	商業	この 町[まち]では 商業[しょうぎょう]が 盛[さか]んだ。	
\\	私はよく図書館を利用します。	利用	私[わたし]はよく 図書館[としょかん]を 利用[りよう]します。	
\\	インターネットはとても便利です。	便利	インターネットはとても 便利[べんり]です。	
\\	彼はカーラジオのスイッチを入れた。	スイッチ	彼[かれ]はカーラジオのスイッチを 入[い]れた。	
\\	彼は非常に高い技術を持っている。	技術	彼[かれ]は 非常[ひじょう]に 高[たか]い 技術[ぎじゅつ]を 持[も]っている。	
\\	父は胸の手術をした。	手術	父[ちち]は 胸[むね]の 手術[しゅじゅつ]をした。	
\\	この工場ではエアコンを製造しています。	製造	この 工場[こうじょう]ではエアコンを 製造[せいぞう]しています。	
\\	必ずシートベルトを着けて下さい。	必ず	必[かなら]ずシートベルトを 着[つ]けて 下[くだ]さい。	
\\	誕生日にプレゼントをもらいました。	プレゼント	誕生日[たんじょうび]にプレゼントをもらいました。	
\\	私にはたくさんのお金が必要だ。	必要	私[わたし]にはたくさんのお 金[かね]が 必要[ひつよう]だ。	
\\	これは重要な書類です。	重要	これは 重要[じゅうよう]な 書類[しょるい]です。	
\\	彼は私の要求にこたえた。	要求	彼[かれ]は 私[わたし]の 要求[ようきゅう]にこたえた。	
\\	7時に目覚まし時計が鳴りました。	目覚まし時計	7時[しちじ]に 目覚[めざ]まし 時計[どけい]が 鳴[な]りました。	
\\	いつでもうちに来てください。	いつでも	いつでもうちに 来[き]てください。	
\\	その計算は間違っている。	計算	その 計算[けいさん]は 間違[まちが]っている。	
\\	娘は学校で引き算を習っている。	引き算	娘[むすめ]は 学校[がっこう]で 引[ひ]き 算[ざん]を 習[なら]っている。	
\\	娘は学校で足し算を習った。	足し算	娘[むすめ]は 学校[がっこう]で 足[た]し 算[ざん]を 習[なら]った。	
\\	次の交差点を左に曲がってください。	交差点	次[つぎ]の 交差点[こうさてん]を 左[ひだり]に 曲[ま]がってください。	
\\	彼らはよくテニスをしています。	テニス	彼[かれ]らはよくテニスをしています。	
\\	皿を落として割った。	割る	皿[さら]を 落[お]として 割[わ]った。	
\\	コップが落ちて割れた。	割れる	コップが 落[お]ちて 割[わ]れた。	
\\	割り算は小学校で習います。	割り算	割[わ]り 算[ざん]は 小学校[しょうがっこう]で 習[なら]います。	
\\	料理がたくさん残りました。	残る	料理[りょうり]がたくさん 残[のこ]りました。	
\\	受付はこちらです。	こちら	受付[うけつけ]はこちらです。	
\\	彼女はメッセージを残しました。	残す	彼女[かのじょ]はメッセージを 残[のこ]しました。	
\\	図書館に本を返した。	返す	図書館[としょかん]に 本[ほん]を 返[かえ]した。	
\\	手紙の返事を出しました。	返事	手紙[てがみ]の 返事[へんじ]を 出[だ]しました。	
\\	彼にビデオを借りました。	借りる	彼[かれ]にビデオを 借[か]りました。	
\\	ボタンを押してください。	ボタン	ボタンを 押[お]してください。	
\\	私は彼に本を貸しています。	貸す	私[わたし]は 彼[かれ]に 本[ほん]を 貸[か]しています。	
\\	その本は貸し出し中です。	貸し出す	その 本[ほん]は 貸[か]し 出[だ]し 中[ちゅう]です。	
\\	彼女はそのセミナーに申し込んだ。	申し込む	彼女[かのじょ]はそのセミナーに 申[もう]し 込[こ]んだ。	
\\	みんな私たちに期待しています。	期待	みんな 私[わたし]たちに 期待[きたい]しています。	
\\	彼の話はおかしかった。	おかしい	彼[かれ]の 話[はなし]はおかしかった。	
\\	テスト期間は10日から15日までだ。	期間	テスト 期間[きかん]は 10日[とおか]から 15日[じゅうごにち]までだ。	
\\	今はあなたにとって大事な時期です。	時期	今[いま]はあなたにとって 大事[だいじ]な 時期[じき]です。	
\\	このサービスは週末に限ります	限る	このサービスは 週末[しゅうまつ]に 限[かぎ]ります	
\\	私たちは駅へ急ぎました。	急ぐ	私[わたし]たちは 駅[えき]へ 急[いそ]ぎました。	
\\	新しいコートでテニスをしました。	コート	新[あたら]しいコートでテニスをしました。	
\\	急に用事を思い出した。	急に	急[きゅう]に 用事[ようじ]を 思[おも]い 出[だ]した。	
\\	急な坂道を上った。	急	急[きゅう]な 坂道[さかみち]を 上[のぼ]った。	
\\	ちょうど急行電車が来た。	急行	ちょうど 急行[きゅうこう] 電車[でんしゃ]が 来[き]た。	
\\	このはさみはよく切れますね。	切れる	このはさみはよく 切[き]れますね。	
\\	私はドラマを見るのが好きです。	ドラマ	私[わたし]はドラマを 見[み]るのが 好[す]きです。	
\\	郵便局で切手を買いました。	切手	郵便局[ゆうびんきょく]で 切手[きって]を 買[か]いました。	
\\	その本は直ぐ売り切れた。	売り切れる	その 本[ほん]は 直[す]ぐ 売[う]り 切[き]れた。	
\\	チケットはもう売り切れだって。	売り切れ	チケットはもう 売[う]り 切[き]れだって。	
\\	これは母が大切にしていた指輪です。	大切	これは 母[はは]が 大切[たいせつ]にしていた 指輪[ゆびわ]です。	
\\	学生ビザを持っています。	ビザ	学生[がくせい]ビザを 持[も]っています。	
\\	東京までの切符を買った。	切符	東京[とうきょう]までの 切符[きっぷ]を 買[か]った。	
\\	入場券は3000円です。	入場券	入場券[にゅうじょうけん]は 3000円[さんぜんえん]です。	
\\	ここの家賃は12万円です。	家賃	ここの 家賃[やちん]は 12万円[じゅうにまんえん]です。	
\\	今は便利さとスピードの時代だ。	時代	今[いま]は 便利[べんり]さとスピードの 時代[じだい]だ。	
\\	財布をポケットにしまった。	ポケット	財布[さいふ]をポケットにしまった。	
\\	上司に代わって会議に出た。	代わる	上司[じょうし]に 代[か]わって 会議[かいぎ]に 出[で]た。	
\\	社長に代えて部長を出席させます。	代える	社長[しゃちょう]に 代[か]えて 部長[ぶちょう]を 出席[しゅっせき]させます。	
\\	彼は指が太い。	指	彼[かれ]は 指[ゆび]が 太[ふと]い。	
\\	会議で重要な決定がありました。	決定	会議[かいぎ]で 重要[じゅうよう]な 決定[けってい]がありました。	
\\	そろそろ始めましょうか。	そろそろ	そろそろ 始[はじ]めましょうか。	
\\	申し込むには一定の資格が要る。	一定	申[もう]し 込[こ]むには 一定[いってい]の 資格[しかく]が 要[い]る。	
\\	定期券は1万2千円でした。	定期券	定期券[ていきけん]は 1万2千円[いちまんにせんえん]でした。	
\\	今日の予定を教えてください。	予定	今日[きょう]の 予定[よてい]を 教[おし]えてください。	
\\	明日の天気予報は雨です。	天気予報	明日[あした]の 天気予報[てんきよほう]は 雨[あめ]です。	
\\	私はぶどうが好きです。	ぶどう	私[わたし]はぶどうが 好[す]きです。	
\\	明日の予習をしましょう。	予習	明日[あした]の 予習[よしゅう]をしましょう。	
\\	レストランを予約しました。	予約	レストランを 予約[よやく]しました。	
\\	約束は守ります。	約束	約束[やくそく]は 守[まも]ります。	
\\	信号が青に変わりました。	変わる	信号[しんごう]が 青[あお]に 変[か]わりました。	
\\	彼女はめったに怒りません。	めったに	彼女[かのじょ]はめったに 怒[おこ]りません。	
\\	大変なことが起こりました。	大変	大変[たいへん]なことが 起[お]こりました。	
\\	変な音が聞こえます。	変	変[へん]な 音[おと]が 聞[き]こえます。	
\\	今年は変化の多い年でした。	変化	今年[ことし]は 変化[へんか]の 多[おお]い 年[とし]でした。	
\\	国は国語教育を強化しています。	強化	国[くに]は 国語教育[こくご きょういく]を 強化[きょうか]しています。	
\\	クリスマスにはケーキを食べます。	クリスマス	クリスマスにはケーキを 食[た]べます。	
\\	私はこの国の文化を勉強しています。	文化	私[わたし]はこの 国[くに]の 文化[ぶんか]を 勉強[べんきょう]しています。	
\\	この町は人口が増えた。	増える	この 町[まち]は 人口[じんこう]が 増[ふ]えた。	
\\	あの町は緑を増やしています。	増やす	あの 町[まち]は 緑[みどり]を 増[ふ]やしています。	
\\	体重がかなり減りました。	減る	体重[たいじゅう]がかなり 減[へ]りました。	
\\	父の日にネクタイをプレゼントした。	ネクタイ	父[ちち]の 日[ひ]にネクタイをプレゼントした。	
\\	最近、食事を減らしています。	減らす	最近[さいきん]、 食事[しょくじ]を 減[へ]らしています。	
\\	自転車は便利な乗り物です。	乗り物	自転車[じてんしゃ]は 便利[べんり]な 乗[の]り 物[もの]です。	
\\	次の駅で降ります。	降りる	次[つぎ]の 駅[えき]で 降[お]ります。	
\\	彼は車から荷物を降ろした。	降ろす	彼[かれ]は 車[くるま]から 荷物[にもつ]を 降[お]ろした。	
\\	彼女はバイオリンを習っています。	バイオリン	彼女[かのじょ]はバイオリンを 習[なら]っています。	
\\	激しい雨が降っています。	降る	激[はげ]しい 雨[あめ]が 降[ふ]っています。	
\\	午後8時に大阪に着きます。	着く	午後8時[ごご はちじ]に 大阪[おおさか]に 着[つ]きます。	
\\	彼女は着物がよく似合います。	着物	彼女[かのじょ]は 着物[きもの]がよく 似合[にあ]います。	
\\	私は下着を手で洗う。	下着	私[わたし]は 下着[したぎ]を 手[て]で 洗[あら]う。	
\\	パスポートが見つからない。	パスポート	パスポートが 見[み]つからない。	
\\	暑いので上着を脱ぎました。	上着	暑[あつ]いので 上着[うわぎ]を 脱[ぬ]ぎました。	
\\	娘に可愛いドレスを着せた。	着せる	娘[むすめ]に 可愛[かわい]いドレスを 着[き]せた。	
\\	靴を脱いでください。	脱ぐ	靴[くつ]を 脱[ぬ]いでください。	
\\	彼女は自分の立場を分かっていない。	立場	彼女[かのじょ]は 自分[じぶん]の 立場[たちば]を 分[わ]かっていない。	
\\	私は黒いバッグを持っています。	バッグ	私[わたし]は 黒[くろ]いバッグを 持[も]っています。	
\\	彼女の大きな帽子はとても目立つ。	目立つ	彼女[かのじょ]の 大[おお]きな 帽子[ぼうし]はとても 目立[めだ]つ。	
\\	彼はケーキにろうそくを立てた。	立てる	彼[かれ]はケーキにろうそくを 立[た]てた。	
\\	私は人々の役に立ちたいと思っています。	役に立つ	私[わたし]は 人々[ひとびと]の 役[やく]に 立[た]ちたいと 思[おも]っています。	
\\	午後は会議に出席します。	出席	午後[ごご]は 会議[かいぎ]に 出席[しゅっせき]します。	
\\	ゴムが伸びてしまった。	ゴム	ゴムが 伸[の]びてしまった。	
\\	この席、空いてますか。	席	この 席[せき]、 空[あ]いてますか。	
\\	風邪のため今日は欠席します。	欠席	風邪[かぜ]のため 今日[きょう]は 欠席[けっせき]します。	
\\	次男は今、海外にいます。	次男	次男[じなん]は 今[いま]、 海外[かいがい]にいます。	
\\	うちの次女は春から中学生です。	次女	うちの 次女[じじょ]は 春[はる]から 中学生[ちゅうがくせい]です。	
\\	彼女はギターが得意です。	ギター	彼女[かのじょ]はギターが 得意[とくい]です。	
\\	いすを二階に運んでください。	運ぶ	いすを 二階[にかい]に 運[はこ]んでください。	
\\	父は安全運転だ。	運転	父[ちち]は 安全[あんぜん] 運転[うんてん]だ。	
\\	彼はタクシーの運転手です。	運転手	彼[かれ]はタクシーの 運転手[うんてんしゅ]です。	
\\	お祖母ちゃんがお風呂場で転んだ。	転ぶ	お 祖母[ばあ]ちゃんがお 風呂場[ふろば]で 転[ころ]んだ。	
\\	このセーターはウールだ。	セーター	このセーターはウールだ。	
\\	彼女は新しい会社に移った。	移る	彼女[かのじょ]は 新[あたら]しい 会社[かいしゃ]に 移[うつ]った。	
\\	机を窓の傍に移しました。	移す	机[つくえ]を 窓[まど]の 傍[そば]に 移[うつ]しました。	
\\	まず相手の動きを見ましょう。	動き	まず 相手[あいて]の 動[うご]きを 見[み]ましょう。	
\\	彼はもっと運動した方がいい。	運動	彼[かれ]はもっと 運動[うんどう]した 方[ほう]がいい。	
\\	やっぱり旅が大好きだ。	やっぱり	やっぱり 旅[たび]が 大好[だいす]きだ。	
\\	彼は地域の活動に参加した。	活動	彼[かれ]は 地域[ちいき]の 活動[かつどう]に 参加[さんか]した。	
\\	私は動物が大好きです。	動物	私[わたし]は 動物[どうぶつ]が 大好[だいす]きです。	
\\	部屋の家具を動かしました。	動かす	部屋[へや]の 家具[かぐ]を 動[うご]かしました。	
\\	私の父は不動産屋です。	不動産屋	私[わたし]の 父[ちち]は 不動産屋[ふどうさんや]です。	
\\	朝、カーテンを開けた。	カーテン	朝[あさ]、カーテンを 開[あ]けた。	
\\	なるべく早く来て下さい。	早く	なるべく 早[はや]く 来[き]て 下[くだ]さい。	
\\	彼女は早口だ。	早口	彼女[かのじょ]は 早口[はやくち]だ。	
\\	新幹線の速さはどれ位ですか。	速さ	新幹線[しんかんせん]の 速[はや]さはどれ 位[くらい]ですか。	
\\	これを速達で送りたいのですが。	速達	これを 速達[そくたつ]で 送[おく]りたいのですが。	
\\	このシャツはアイロンが必要だ。	シャツ	このシャツはアイロンが 必要[ひつよう]だ。	
\\	今朝彼女は学校に遅れました。	遅れる	今朝彼女[けさ かのじょ]は 学校[がっこう]に 遅[おく]れました。	
\\	運動会は9時開始です。	開始	運動会[うんどうかい]は 9時[くじ] 開始[かいし]です。	
\\	始めにスープが出ます。	始めに	始[はじ]めにスープが 出[で]ます。	
\\	現在の気温は30度です。	現在	現在[げんざい]の 気温[きおん]は 30度[さんじゅうど]です。	
\\	ナイフで手を切った。	ナイフ	ナイフで 手[て]を 切[き]った。	
\\	夢を実現するには努力が必要です。	実現	夢[ゆめ]を 実現[じつげん]するには 努力[どりょく]が 必要[ひつよう]です。	
\\	現在、スペシャルキャンペーンを実施中です。	実施	現在[げんざい]、スペシャルキャンペーンを 実施[じっし] 中[ちゅう]です。	
\\	それは全て事実ですか。	事実	それは 全[すべ]て 事実[じじつ]ですか。	
\\	彼はその計画を実行した。	実行	彼[かれ]はその 計画[けいかく]を 実行[じっこう]した。	
\\	兄はバイクが大好きです。	バイク	兄[あに]はバイクが 大好[だいす]きです。	
\\	科学の授業で実験をした。	実験	科学[かがく]の 授業[じゅぎょう]で 実験[じっけん]をした。	
\\	次の駅は通過します。	通過	次[つぎ]の 駅[えき]は 通過[つうか]します。	
\\	時が過ぎるのは速い。	過ぎる	時[とき]が 過[す]ぎるのは 速[はや]い。	
\\	彼は昼過ぎに来ます。	昼過ぎ	彼[かれ]は 昼過[ひるす]ぎに 来[き]ます。	
\\	僕はいつかアフリカに行きたい。	いつか	僕[ぼく]はいつかアフリカに 行[い]きたい。	
\\	それは過去の話だ。	過去	それは 過去[かこ]の 話[はなし]だ。	
\\	ダムの開発に住民は反対しています。	開発	ダムの 開発[かいはつ]に 住民[じゅうみん]は 反対[はんたい]しています。	
\\	交差点で事故が発生した。	発生	交差点[こうさてん]で 事故[じこ]が 発生[はっせい]した。	
\\	会議で全員が発言した。	発言	会議[かいぎ]で 全員[ぜんいん]が 発言[はつげん]した。	
\\	私はオレンジが好きです。	オレンジ	私[わたし]はオレンジが 好[す]きです。	
\\	新しい星が発見された。	発見	新[あたら]しい 星[ほし]が 発見[はっけん]された。	
\\	あと15分で出発です。	出発	あと 15分[じゅうごふん]で 出発[しゅっぱつ]です。	
\\	バスが発車します。	発車	バスが 発車[はっしゃ]します。	
\\	合格者が発表された。	発表	合格者[ごうかくしゃ]が 発表[はっぴょう]された。	
\\	このシャツは少しきついです。	きつい	このシャツは 少[すこ]しきついです。	
\\	彼は歌で自分の気持ちを表現した。	表現	彼[かれ]は 歌[うた]で 自分[じぶん]の 気持[きも]ちを 表現[ひょうげん]した。	
\\	彼がクラスの代表だ。	代表	彼[かれ]がクラスの 代表[だいひょう]だ。	
\\	表に人が来ています。	表	表[おもて]に 人[ひと]が 来[き]ています。	
\\	この表を見てください。	表	この 表[ひょう]を 見[み]てください。	
\\	さっきの話を続けましょう。	さっき	さっきの 話[はなし]を 続[つづ]けましょう。	
\\	友人から手紙をもらいました。	手紙	友人[ゆうじん]から 手紙[てがみ]をもらいました。	
\\	これは有名な画家の絵です。	絵	これは 有名[ゆうめい]な 画家[がか]の 絵[え]です。	
\\	この雑誌はよく売れています。	雑誌	この 雑誌[ざっし]はよく 売[う]れています。	
\\	雨の音が聞こえる。	音	雨[あめ]の 音[おと]が 聞[き]こえる。	
\\	テキストを読んでください。	テキスト	テキストを 読[よ]んでください。	
\\	このピアノは音があまり良くない。	音	このピアノは 音[おと]があまり 良[よ]くない。	
\\	私は音楽を聞くのが好きだ。	音楽	私[わたし]は 音楽[おんがく]を 聞[き]くのが 好[す]きだ。	
\\	この薬を必ず飲んでください。	薬	この 薬[くすり]を 必[かなら]ず 飲[の]んでください。	
\\	私はその歌を知らなかった。	歌	私[わたし]はその 歌[うた]を 知[し]らなかった。	
\\	ゴミはそのビニール袋に入れてください。	ビニール	ゴミはそのビニール 袋[ぶくろ]に 入[い]れてください。	
\\	その歌手は歌が下手だ。	歌手	その 歌手[かしゅ]は 歌[うた]が 下手[へた]だ。	
\\	子供がジュースを欲しがっています。	欲しがる	子供[こども]がジュースを 欲[ほ]しがっています。	
\\	彼は一人旅の計画を立てた。	計画	彼[かれ]は 一人旅[ひとりたび]の 計画[けいかく]を 立[た]てた。	
\\	彼はよく映画を見ます。	映画	彼[かれ]はよく 映画[えいが]を 見[み]ます。	
\\	りんごを一つください。	りんご	りんごを 一[ひと]つください。	
\\	この本は全然面白くなかった。	面白い	この 本[ほん]は全然[ぜんぜん] 面白[おもしろ]くなかった。	
\\	このカメラはよく写りますよ。	写る	このカメラはよく 写[うつ]りますよ。	
\\	彼は友達の答えを写した。	写す	彼[かれ]は 友達[ともだち]の 答[こた]えを 写[うつ]した。	
\\	写真は良い思い出になります。	写真	写真[しゃしん]は 良[い]い 思[おも]い 出[で]になります。	
\\	私は彼のアルバムを見た。	アルバム	私[わたし]は 彼[かれ]のアルバムを 見[み]た。	
\\	彼の顔は真っ赤でした。	真っ赤	彼[かれ]の 顔[かお]は 真[ま]っ 赤[か]でした。	
\\	彼は真面目な人です。	真面目	彼[かれ]は 真面目[まじめ]な 人[ひと]です。	
\\	道の真ん中に人が立っている。	真ん中	道[みち]の 真[ま]ん 中[なか]に 人[ひと]が 立[た]っている。	
\\	外は雪で真っ白だった。	真っ白	外[そと]は 雪[ゆき]で 真[ま]っ 白[しろ]だった。	
\\	彼女はあまりスカートははかない。	スカート	彼女[かのじょ]はあまりスカートははかない。	
\\	外は真っ暗です。	真っ暗	外[そと]は 真[ま]っ 暗[くら]です。	
\\	インクで手が真っ黒になった	真っ黒	インクで 手[て]が 真[ま]っ 黒[くろ]になった	
\\	空が真っ青です。	真っ青	空[そら]が 真[ま]っ 青[さお]です。	
\\	彼は色々なことを知っている。	色々	彼[かれ]は 色々[いろいろ]なことを 知[し]っている。	
\\	ペンを貸してください。	ペン	ペンを 貸[か]してください。	
\\	彼女は人形をたくさん持っています。	人形	彼女[かのじょ]は 人形[にんぎょう]をたくさん 持[も]っています。	
\\	その椅子は変わった形をしている。	形	その 椅子[いす]は 変[か]わった 形[かたち]をしている。	
\\	大型のテレビを買った。	大型	大型[おおがた]のテレビを 買[か]った。	
\\	バラには色々な種類があります。	種類	バラには 色々[いろいろ]な 種類[しゅるい]があります。	
\\	妹はとてもおしゃべりです。	おしゃべり	妹[いもうと]はとてもおしゃべりです。	
\\	私がそれを直しました。	直す	私[わたし]がそれを 直[なお]しました。	
\\	この道を真っ直ぐ行ってください。	真っ直ぐ	この 道[みち]を 真[ま]っ 直[す]ぐ 行[い]ってください。	
\\	彼に直接お願いしなさい。	直接	彼[かれ]に 直接[ちょくせつ]お 願[ねが]いしなさい。	
\\	次の角で左に曲がってください。	角	次[つぎ]の 角[かど]で 左[ひだり]に 曲[ま]がってください。	
\\	明日はきっと雨が降ります。	きっと	明日[あした]はきっと 雨[あめ]が 降[ふ]ります。	
\\	紙を三角に切りました。	三角	紙[かみ]を 三角[さんかく]に 切[き]りました。	
\\	紙を四角に切ってください。	四角	紙[かみ]を 四角[しかく]に 切[き]ってください。	
\\	こっちの四角いテーブルを買おうよ。	四角い	こっちの 四角[しかく]いテーブルを 買[か]おうよ。	
\\	あそこの四つ角を左に曲がってください。	四つ角	あそこの 四[よ]つ 角[かど]を 左[ひだり]に 曲[ま]がってください。	
\\	メニューをください。	メニュー	メニューをください。	
\\	私はこの曲が大好きです。	曲	私[わたし]はこの 曲[きょく]が 大好[だいす]きです。	
\\	ひざを曲げてください。	曲げる	ひざを 曲[ま]げてください。	
\\	ポストはそこの曲がり角にあります。	曲がり角	ポストはそこの 曲[ま]がり 角[かど]にあります。	
\\	私たちは彼を家族同様に思っている。	同様	私[わたし]たちは 彼[かれ]を 家族[かぞく] 同様[どうよう]に 思[おも]っている。	
\\	弟はアイスクリームが大好きです。	アイスクリーム	弟[おとうと]はアイスクリームが 大好[だいす]きです。	
\\	この文には間違いがあります。	間違い	この 文[ぶん]には 間違[まちが]いがあります。	
\\	電話番号を間違えました。	間違える	電話番号[でんわ ばんごう]を 間違[まちが]えました。	
\\	あなたは間違っている。	間違う	あなたは 間違[まちが]っている。	
\\	私は母に似ています。	似ている	私[わたし]は 母[はは]に 似[に]ています。	
\\	あの子はおしゃれだね。	おしゃれ	あの 子[こ]はおしゃれだね。	
\\	飛行機が1時間以上遅れた。	以上	飛行機[ひこうき]が 1時間[いちじかん] 以上[いじょう] 遅[おく]れた。	
\\	彼女は旅行が好きです。	旅行	彼女[かのじょ]は 旅行[りょこう]が 好[す]きです。	
\\	彼は大使館に勤めています。	大使館	彼[かれ]は 大使館[たいしかん]に 勤[つと]めています。	
\\	京都では旅館に泊まりました。	旅館	京都[きょうと]では 旅館[りょかん]に 泊[と]まりました。	
\\	このジュースは甘すぎる。	ジュース	このジュースは 甘[あま]すぎる。	
\\	彼と近くの映画館に行きました。	映画館	彼[かれ]と 近[ちか]くの 映画館[えいがかん]に 行[い]きました。	
\\	友達と一緒に宿題をした。	宿題	友達[ともだち]と 一緒[いっしょ]に 宿題[しゅくだい]をした。	
\\	友達をうちに泊めてあげました。	泊める	友達[ともだち]をうちに 泊[と]めてあげました。	
\\	お正月には色々な遊びをします。	遊び	お 正月[しょうがつ]には 色々[いろいろ]な 遊[あそ]びをします。	
\\	今日の交通事故はゼロです。	ゼロ	今日[きょう]の 交通事故[こうつう じこ]はゼロです。	
\\	今日は洋服を買いに行きます。	洋服	今日[きょう]は 洋服[ようふく]を 買[か]いに 行[い]きます。	
\\	私の教室は3階にあります。	教室	私[わたし]の 教室[きょうしつ]は 3階[さんがい]にあります。	
\\	図書室で勉強した。	図書室	図書室[としょしつ]で 勉強[べんきょう]した。	
\\	窓を開けてください。	窓	窓[まど]を 開[あ]けてください。	
\\	娘がピンクのドレスを着ている。	ピンク	娘[むすめ]がピンクのドレスを 着[き]ている。	
\\	週末、親しい友達を家に呼んだ。	親しい	週末[しゅうまつ]、 親[した]しい 友達[ともだち]を 家[いえ]に 呼[よ]んだ。	
\\	親切にしてくださってどうもありがとうございます。	親切	親切[しんせつ]にしてくださってどうもありがとうございます。	
\\	その店員は不親切だった。	不親切	その 店員[てんいん]は 不親切[ふしんせつ]だった。	
\\	うちは五人家族です。	家族	うちは 五人[ごにん] 家族[かぞく]です。	
\\	これはきれいなグラスですね。	グラス	これはきれいなグラスですね。	
\\	勝った、万歳!	万歳	勝[か]った、 万歳[ばんざい]!	
\\	妹は来年、二十歳になります。	二十歳	妹[いもうと]は 来年[らいねん]、 二十歳[はたち]になります。	
\\	姉さん、ごめんね。	姉さん	姉[ねえ]さん、ごめんね。	
\\	兄さん、おめでとう。	兄さん	兄[にい]さん、おめでとう。	
\\	彼女はダンスが大好きです。	ダンス	彼女[かのじょ]はダンスが 大好[だいす]きです。	
\\	彼は3人兄弟です。	兄弟	彼[かれ]は 3人[さんにん] 兄弟[きょうだい]です。	
\\	引っ越しを業者に頼んだ。	業者	引[ひ]っ 越[こ]しを 業者[ぎょうしゃ]に 頼[たの]んだ。	
\\	最近の若者は本を読まない。	若者	最近[さいきん]の 若者[わかもの]は 本[ほん]を 読[よ]まない。	
\\	彼らはバスケットの選手です。	彼ら	彼[かれ]らはバスケットの 選手[せんしゅ]です。	
\\	おじは銀行に勤めています。	おじ	おじは 銀行[ぎんこう]に 勤[つと]めています。	
\\	ファイルを結合して1つにしました。	結合	ファイルを 結合[けつごう]して 1[ひと]つにしました。	
\\	試合の結果を早く知りたい。	結果	試合[しあい]の 結果[けっか]を 早[はや]く 知[し]りたい。	
\\	デザートに果物を食べましょう。	果物	デザートに 果物[くだもの]を 食[た]べましょう。	
\\	夏休みの課題は何ですか。	課題	夏休[なつやす]みの 課題[かだい]は 何[なん]ですか。	
\\	彼はサッカーの選手です。	サッカー	彼[かれ]はサッカーの 選手[せんしゅ]です。	
\\	今から課のミーティングがある。	課	今[いま]から 課[か]のミーティングがある。	
\\	この薬には胃を守る効果がある。	効果	この 薬[くすり]には 胃[い]を 守[まも]る 効果[こうか]がある。	
\\	彼は自動車会社に就職した。	自動車	彼[かれ]は 自動車[じどうしゃ] 会社[がいしゃ]に 就職[しゅうしょく]した。	
\\	みんなで自然を守りましょう。	自然	みんなで 自然[しぜん]を 守[まも]りましょう。	
\\	母がコーンスープを作っている。	スープ	母[はは]がコーンスープを 作[つく]っている。	
\\	親が子供を守るのは自然なことだ。	自然	親[おや]が 子供[こども]を 守[まも]るのは 自然[しぜん]なことだ。	
\\	明日は自習の時間がある。	自習	明日[あす]は 自習[じしゅう]の 時間[じかん]がある。	
\\	遅れた理由を教えてください。	理由	遅[おく]れた 理由[りゆう]を 教[おし]えてください。	
\\	今日は自由な時間が多い。	自由	今日[きょう]は 自由[じゆう]な 時間[じかん]が 多[おお]い。	
\\	私は毎朝バナナを食べます。	バナナ	私[わたし]は 毎朝[まいあさ]バナナを 食[た]べます。	
\\	彼はキリストを信じている。	信じる	彼[かれ]はキリストを 信[しん]じている。	
\\	信号が青になった。	信号	信号[しんごう]が 青[あお]になった。	
\\	私はハンバーガーを頼みました。	頼む	私[わたし]はハンバーガーを 頼[たの]みました。	
\\	市民の安全は大切だ。	市民	市民[しみん]の 安全[あんぜん]は 大切[たいせつ]だ。	
\\	この通りは車の音がうるさい。	うるさい	この 通[とお]りは 車[くるま]の 音[おと]がうるさい。	
\\	地域の住民が集まって話合いをした。	住民	地域[ちいき]の 住民[じゅうみん]が 集[あつ]まって 話合[はなしあ]いをした。	
\\	主人は今、留守です。	主人	主人[しゅじん]は 今[いま]、 留守[るす]です。	
\\	彼は国会議員です。	議員	彼[かれ]は 国会[こっかい] 議員[ぎいん]です。	
\\	今日の午後、大事な会議があります。	会議	今日[きょう]の 午後[ごご]、 大事[だいじ]な 会議[かいぎ]があります。	
\\	あそこにパトカーがいる。	パトカー	あそこにパトカーがいる。	
\\	その質問に対する答えが見つからなかった。	対する	その 質問[しつもん]に 対[たい]する 答[こた]えが 見[み]つからなかった。	
\\	その2社は対立しています。	対立	その 2社[にしゃ]は 対立[たいりつ]しています。	
\\	私は反対です。	反対	私[わたし]は 反対[はんたい]です。	
\\	彼はその問題の答えが分からない。	答え	彼[かれ]はその 問題[もんだい]の 答[こた]えが 分[わ]からない。	
\\	今日の昼ご飯はハンバーガーでした。	ハンバーガー	今日[きょう]の 昼[ひる]ご 飯[はん]はハンバーガーでした。	
\\	特に質問はありません。	特に	特[とく]に 質問[しつもん]はありません。	
\\	東京まで特急で3時間かかります。	特急	東京[とうきょう]まで 特急[とっきゅう]で 3時間[さんじかん]かかります。	
\\	あなたは私にとって特別な人です。	特別	あなたは 私[わたし]にとって 特別[とくべつ]な 人[ひと]です。	
\\	別々に払いましょう。	別々	別々[べつべつ]に 払[はら]いましょう。	
\\	3階までエスカレーターで行きましょう。	エスカレーター	3階[さんかい]までエスカレーターで 行[い]きましょう。	
\\	駅で友だちと別れました。	別れる	駅[えき]で 友[とも]だちと 別[わか]れました。	
\\	一般の方はこちらの席へどうぞ。	一般	一般[いっぱん]の 方[かた]はこちらの 席[せき]へどうぞ。	
\\	彼が来た目的が分かりません。	目的	彼[かれ]が 来[き]た 目的[もくてき]が 分[わ]かりません。	
\\	彼女は普通の女の子だ。	普通	彼女[かのじょ]は 普通[ふつう]の 女[おんな]の 子[こ]だ。	
\\	私はタオルで顔をふいた。	タオル	私[わたし]はタオルで 顔[かお]をふいた。	
\\	ここに並んでください。	並ぶ	ここに 並[なら]んでください。	
\\	私は料理をテーブルに並べた。	並べる	私[わたし]は 料理[りょうり]をテーブルに 並[なら]べた。	
\\	この国は平和です。	平和	この 国[くに]は 平和[へいわ]です。	
\\	弟が高等学校を卒業しました。	高等学校	弟[おとうと]が 高等学校[こうとうがっこう]を 卒業[そつぎょう]しました。	
\\	彼は毎日パチンコをしています。	パチンコ	彼[かれ]は 毎日[まいにち]パチンコをしています。	
\\	病院はどこですか。	病院	病院[びょういん]はどこですか。	
\\	昨日、母が入院しました。	入院	昨日[きのう]、 母[はは]が 入院[にゅういん]しました。	
\\	彼は大学院に進みました。	大学院	彼[かれ]は 大学院[だいがくいん]に 進[すす]みました。	
\\	私は医者に相談した。	医者	私[わたし]は 医者[いしゃ]に 相談[そうだん]した。	
\\	みかんを1つください。	みかん	みかんを 1[ひと]つください。	
\\	熱があるのでお医者さんに行った。	お医者さん	熱[ねつ]があるのでお 医者[いしゃ]さんに 行[い]った。	
\\	私は歯医者が嫌いです。	歯医者	私[わたし]は 歯医者[はいしゃ]が 嫌[きら]いです。	
\\	私の歯は丈夫です。	歯	私[わたし]の 歯[は]は 丈夫[じょうぶ]です。	
\\	新しい歯ブラシが必要だ。	歯ブラシ	新[あたら]しい 歯[は]ブラシが 必要[ひつよう]だ。	
\\	誕生日にケーキを食べました。	ケーキ	誕生日[たんじょうび]にケーキを 食[た]べました。	
\\	科学は常に進歩している。	科学	科学[かがく]は 常[つね]に 進歩[しんぽ]している。	
\\	日本語の教科書を忘れた。	教科書	日本語[にほんご]の 教科書[きょうかしょ]を 忘[わす]れた。	
\\	私は理科が得意です。	理科	私[わたし]は 理科[りか]が 得意[とくい]です。	
\\	おととい、昔の友人が亡くなった。	亡くなる	おととい、 昔[むかし]の 友人[ゆうじん]が 亡[な]くなった。	
\\	私は毎朝コップ一杯の水を飲む。	コップ	私[わたし]は 毎朝[まいあさ]コップ 一杯[いっぱい]の 水[みず]を 飲[の]む。	
\\	忙しいので手伝ってください。	忙しい	忙[いそが]しいので 手伝[てつだ]ってください。	
\\	私は疲れてふらふらです。	疲れる	私[わたし]は 疲[つか]れてふらふらです。	
\\	彼は今、禁煙しています。	禁煙	彼[かれ]は 今[いま]、 禁煙[きんえん]しています。	
\\	彼はただの酔っ払いです。	酔っ払い	彼[かれ]はただの 酔[よ]っ 払[ぱら]いです。	
\\	このジャケットはナイロンでできています。	ナイロン	このジャケットはナイロンでできています。	
\\	昨夜は酔っ払いました。	酔っ払う	昨夜[ゆうべ]は 酔[よ]っ 払[ぱら]いました。	
\\	その道は車が多くて危ない。	危ない	その 道[みち]は 車[くるま]が 多[おお]くて 危[あぶ]ない。	
\\	その地域は今、危険だ。	危険	その 地域[ちいき]は 今[いま]、 危険[きけん]だ。	
\\	宇宙人は存在すると思いますか。	存在	宇宙人[うちゅうじん]は 存在[そんざい]すると 思[おも]いますか。	
\\	あのスーツはそんなに高くない。	スーツ	あのスーツはそんなに 高[たか]くない。	
\\	私たちはその会社に注目している。	注目	私[わたし]たちはその 会社[かいしゃ]に 注目[ちゅうもく]している。	
\\	レストランでピザを注文しました。	注文	レストランでピザを 注文[ちゅうもん]しました。	
\\	それはどういう意味ですか。	意味	それはどういう 意味[いみ]ですか。	
\\	あなたの意見が聞きたいです。	意見	あなたの 意見[いけん]が 聞[き]きたいです。	
\\	この遊園地のチケットは3000円です。	チケット	この 遊園地[ゆうえんち]のチケットは 3000円[さんぜんえん]です。	
\\	車に注意してください。	注意	車[くるま]に 注意[ちゅうい]してください。	
\\	食事の用意ができました。	用意	食事[しょくじ]の 用意[ようい]ができました。	
\\	彼の昇進は確かだ。	確か	彼[かれ]の 昇進[しょうしん]は 確[たし]かだ。	
\\	もう一度、予約を確認した。	確認	もう 一度[いちど]、 予約[よやく]を 確認[かくにん]した。	
\\	妹はチョコレートが大好きです。	チョコレート	妹[いもうと]はチョコレートが 大好[だいす]きです。	
\\	このソフトにはいろいろな機能があります。	機能	このソフトにはいろいろな 機能[きのう]があります。	
\\	ジェット機が飛んでいる。	ジェット機	ジェット 機[き]が 飛[と]んでいる。	
\\	新しい機械が壊れた。	機械	新[あたら]しい 機械[きかい]が 壊[こわ]れた。	
\\	サラダの材料をそろえました。	材料	サラダの 材料[ざいりょう]をそろえました。	
\\	テレビのチャンネルを変えてください。	チャンネル	テレビのチャンネルを 変[か]えてください。	
\\	具体的な例をいくつか見せてください。	具体的	具体的[ぐたいてき]な 例[れい]をいくつか 見[み]せてください。	
\\	この話は真実に基づいています。	基づく	この 話[はなし]は 真実[しんじつ]に 基[もと]づいています。	
\\	今、ジャズダンスの基本を習っています。	基本	今[いま]、ジャズダンスの 基本[きほん]を 習[なら]っています。	
\\	判断の基準が示された。	基準	判断[はんだん]の 基準[きじゅん]が 示[しめ]された。	
\\	彼の話はつまらないですね。	つまらない	彼[かれ]の 話[はなし]はつまらないですね。	
\\	災害に備えて大量の水を買い込んだ。	備える	災害[さいがい]に 備[そな]えて 大量[たいりょう]の 水[みず]を 買[か]い 込[こ]んだ。	
\\	明日の会議の準備をした。	準備	明日[あす]の 会議[かいぎ]の 準備[じゅんび]をした。	
\\	兄は船の設計をしています。	設計	兄[あに]は 船[ふね]の 設計[せっけい]をしています。	
\\	そのホテルにはレジャー施設がたくさんある。	施設	そのホテルにはレジャー 施設[しせつ]がたくさんある。	
\\	マッチを持っていますか。	マッチ	マッチを 持[も]っていますか。	
\\	授業の終わりに復習の時間を設けた。	設ける	授業[じゅぎょう]の 終[お]わりに 復習[ふくしゅう]の 時間[じかん]を 設[もう]けた。	
\\	この単語の意味を説明してください。	説明	この 単語[たんご]の 意味[いみ]を 説明[せつめい]してください。	
\\	私は月に3冊くらい小説を読みます。	小説	私[わたし]は 月[つき]に 3冊[さんさつ]くらい 小説[しょうせつ]を 読[よ]みます。	
\\	その映画は今日、公開されます。	公開	その 映画[えいが]は 今日[きょう]、 公開[こうかい]されます。	
\\	従兄弟と私は同じ年です。	いとこ	従兄弟[いとこ]と 私[わたし]は 同[おな]じ 年[とし]です。	
\\	子供たちが公園で遊んでいる。	公園	子供[こども]たちが 公園[こうえん]で 遊[あそ]んでいる。	
\\	昨日子供たちと動物園に行きました。	動物園	昨日子供[きのう こども]たちと 動物園[どうぶつえん]に 行[い]きました。	
\\	彼女は祭りが大好きです。	祭り	彼女[かのじょ]は 祭[まつ]りが 大好[だいす]きです。	
\\	ここで国際会議が開かれます。	国際	ここで 国際[こくさい] 会議[かいぎ]が 開[ひら]かれます。	
\\	うちではまだストーブを使っています。	ストーブ	うちではまだストーブを 使[つか]っています。	
\\	彼は実際にはあまり背が高くない。	実際	彼[かれ]は 実際[じっさい]にはあまり 背[せ]が 高[たか]くない。	
\\	バスが飛行場に着きました。	飛行場	バスが 飛行場[ひこうじょう]に 着[つ]きました。	
\\	航空便で書類が届きました。	航空便	航空便[こうくうびん]で 書類[しょるい]が 届[とど]きました。	
\\	アメリカの友人から船便が届いた。	船便	アメリカの 友人[ゆうじん]から 船便[ふなびん]が 届[とど]いた。	
\\	この近くにガソリンスタンドはありますか。	ガソリンスタンド	この 近[ちか]くにガソリンスタンドはありますか。	
\\	私たちは船に乗った。	船	私[わたし]たちは 船[ふね]に 乗[の]った。	
\\	空港までリムジンバスで行った。	空港	空港[くうこう]までリムジンバスで 行[い]った。	
\\	港に船が着きました。	港	港[みなと]に 船[ふね]が 着[つ]きました。	
\\	日本は島国です。	島	日本[にっぽん]は 島[しま] 国[ぐに]です。	
\\	デートで遊園地に行きました。	デート	デートで 遊園地[ゆうえんち]に 行[い]きました。	
\\	新しいホームページが完成した。	完成	新[あたら]しいホームページが 完成[かんせい]した。	
\\	彼女は平成3年生まれです。	平成	彼女[かのじょ]は 平成[へいせい] 3年生[さんねんう]まれです。	
\\	ついに実験が成功した。	成功	ついに 実験[じっけん]が 成功[せいこう]した。	
\\	この事故の原因は何ですか。	原因	この 事故[じこ]の 原因[げんいん]は 何[なん]ですか。	
\\	ふすまを閉めてください。	ふすま	ふすまを 閉[し]めてください。	
\\	私たちは今、結婚資金を貯めています。	資金	私[わたし]たちは 今[いま]、 結婚[けっこん] 資金[しきん]を 貯[た]めています。	
\\	私は4つの会社に投資しています。	投資	私[わたし]は 4[よっ]つの 会社[かいしゃ]に 投資[とうし]しています。	
\\	彼が元気になるよう願っています。	願う	彼[かれ]が 元気[げんき]になるよう 願[ねが]っています。	
\\	彼の計算は正確です。	正確	彼[かれ]の 計算[けいさん]は 正確[せいかく]です。	
\\	紅茶にレモンを入れて飲んだ。	レモン	紅茶[こうちゃ]にレモンを 入[い]れて 飲[の]んだ。	
\\	それは正しい答えです。	正しい	それは 正[ただ]しい 答[こた]えです。	
\\	お正月にはたいてい、家族が集まる。	正月	お 正月[しょうがつ]にはたいてい、 家族[かぞく]が 集[あつ]まる。	
\\	彼女はとても正直だ。	正直	彼女[かのじょ]はとても 正直[しょうじき]だ。	
\\	彼と私はいつも意見が異なる。	異なる	彼[かれ]と 私[わたし]はいつも 意見[いけん]が 異[こと]なる。	
\\	チーズを一切れ食べました。	チーズ	チーズを 一切[ひとき]れ 食[た]べました。	
\\	通常は夜8時まで営業しています。	通常	通常[つうじょう]は 夜8時[よる はちじ]まで 営業[えいぎょう]しています。	
\\	これは非常に重要です。	非常に	これは 非常[ひじょう]に 重要[じゅうよう]です。	
\\	この単語の意味を辞書で調べましょう。	調べる	この 単語[たんご]の 意味[いみ]を 辞書[じしょ]で 調[しら]べましょう。	
\\	彼は良いところだけを強調した。	強調	彼[かれ]は 良[よ]いところだけを 強調[きょうちょう]した。	
\\	今日は群馬までドライブしました。	ドライブ	今日[きょう]は 群馬[ぐんま]までドライブしました。	
\\	私の一番好きな季節は春です。	季節	私[わたし]の 一番好[いちばん す]きな 季節[きせつ]は 春[はる]です。	
\\	私たちがその問題を調査しています。	調査	私[わたし]たちがその 問題[もんだい]を 調査[ちょうさ]しています。	
\\	彼がパーティー会場を提供してくれました。	提供	彼[かれ]がパーティー 会場[かいじょう]を 提供[ていきょう]してくれました。	
\\	そのアイデアは彼の提案です。	提案	そのアイデアは 彼[かれ]の 提案[ていあん]です。	
\\	テニスのラケットを買いました。	ラケット	テニスのラケットを 買[か]いました。	
\\	私が中をご案内します。	案内	私[わたし]が 中[なか]をご 案内[あんない]します。	
\\	彼は新製品に興味を示している。	示す	彼[かれ]は 新製品[しんせいひん]に 興味[きょうみ]を 示[しめ]している。	
\\	私も連れて行ってください。	連れて行く	私[わたし]も 連[つ]れて 行[い]ってください。	
\\	息子が友達を連れて来ました。	連れて来る	息子[むすこ]が 友達[ともだち]を 連[つ]れて 来[き]ました。	
\\	彼女はスチュワーデスになった。	スチュワーデス	彼女[かのじょ]はスチュワーデスになった。	
\\	仕事を続けてください。	続ける	仕事[しごと]を 続[つづ]けてください。	
\\	試合の相手は誰ですか。	相手	試合[しあい]の 相手[あいて]は 誰[だれ]ですか。	
\\	会談の内容が発表されました。	会談	会談[かいだん]の 内容[ないよう]が 発表[はっぴょう]されました。	
\\	相談したいことがあります。	相談	相談[そうだん]したいことがあります。	
\\	テープレコーダーで自分の声を録音しました。	テープレコーダー	テープレコーダーで 自分[じぶん]の 声[こえ]を 録音[ろくおん]しました。	
\\	彼は新聞記者です。	記者	彼[かれ]は 新聞[しんぶん] 記者[きしゃ]です。	
\\	マラソンで世界記録が出た。	記録	マラソンで 世界[せかい] 記録[きろく]が 出[で]た。	
\\	そのラジオ番組はもう録音しました。	録音	そのラジオ 番組[ばんぐみ]はもう 録音[ろくおん]しました。	
\\	私たちは昨年、富士山に登りました。	登る	私[わたし]たちは 昨年[さくねん]、 富士山[ふじさん]に 登[のぼ]りました。	
\\	父はひげをはやしています。	ひげ	父[ちち]はひげをはやしています。	
\\	その問題に関する記事を読みました。	関する	その 問題[もんだい]に 関[かん]する 記事[きじ]を 読[よ]みました。	
\\	関連のセクションへ連絡した。	関連	関連[かんれん]のセクションへ 連絡[れんらく]した。	
\\	彼はその事件に関係がない。	関係	彼[かれ]はその 事件[じけん]に 関係[かんけい]がない。	
\\	この状況では出発は難しいです。	状況	この 状況[じょうきょう]では 出発[しゅっぱつ]は 難[むずか]しいです。	
\\	赤ちゃんがぐっすり寝ている。	ぐっすり	赤[あか]ちゃんがぐっすり 寝[ね]ている。	
\\	ここは道の状態がとても悪いです。	状態	ここは 道[みち]の 状態[じょうたい]がとても 悪[わる]いです。	
\\	けがはもう治りましたか。	治る	けがはもう 治[なお]りましたか。	
\\	私は政治に関心がある。	政治	私[わたし]は 政治[せいじ]に 関心[かんしん]がある。	
\\	早く風邪を治してください。	治す	早[はや]く 風邪[かぜ]を 治[なお]してください。	
\\	ソースはどれですか。	ソース	ソースはどれですか。	
\\	そのデモについて政府は何もしなかった。	政府	そのデモについて 政府[せいふ]は 何[なに]もしなかった。	
\\	党の代表が質問に答えました。	党	党[とう]の 代表[だいひょう]が 質問[しつもん]に 答[こた]えました。	
\\	一緒に対策を考えましょう。	対策	一緒[いっしょ]に 対策[たいさく]を 考[かんが]えましょう。	
\\	新しい政策はあまり良いとは思えません。	政策	新[あたら]しい 政策[せいさく]はあまり 良[い]いとは 思[おも]えません。	
\\	母は古いタイプライターを持っています。	タイプライター	母[はは]は 古[ふる]いタイプライターを 持[も]っています。	
\\	良い家を選ぶのは難しい。	選ぶ	良[よ]い 家[いえ]を 選[えら]ぶのは 難[むずか]しい。	
\\	彼はプロのサッカー選手だ。	選手	彼[かれ]はプロのサッカー 選手[せんしゅ]だ。	
\\	彼は選挙に出るつもりだ。	選挙	彼[かれ]は 選挙[せんきょ]に 出[で]るつもりだ。	
\\	会長の候補は3人います。	候補	会長[かいちょう]の 候補[こうほ]は 3人[さんにん]います。	
\\	私はトマトが大好きです。	トマト	私[わたし]はトマトが 大好[だいす]きです。	
\\	今の首相はあまり力がない。	首相	今[いま]の 首相[しゅしょう]はあまり 力[ちから]がない。	
\\	東京は日本の首都です。	首都	東京[とうきょう]は 日本[にっぽん]の 首都[しゅと]です。	
\\	彼は行政を改革したいと思っている。	改革	彼[かれ]は 行政[ぎょうせい]を 改革[かいかく]したいと 思[おも]っている。	
\\	革のベルトを買いました。	革	革[かわ]のベルトを 買[か]いました。	
\\	パンにバターをぬって食べました。	バター	パンにバターをぬって 食[た]べました。	
\\	彼女は命令に従わなかった。	命令	彼女[かのじょ]は 命令[めいれい]に 従[したが]わなかった。	
\\	私はこの番組が好きです。	番組	私[わたし]はこの 番組[ばんぐみ]が 好[す]きです。	
\\	日曜日に本棚を組み立てます。	組み立てる	日曜日[にちようび]に 本棚[ほんだな]を 組[く]み 立[た]てます。	
\\	彼はある組織のリーダーだ。	組織	彼[かれ]はある 組織[そしき]のリーダーだ。	
\\	妹はバレーボールが得意です。	バレーボール	妹[いもうと]はバレーボールが 得意[とくい]です。	
\\	早く授業を進めましょう。	進める	早[はや]く 授業[じゅぎょう]を 進[すす]めましょう。	
\\	前に進んでください。	進む	前[まえ]に 進[すす]んでください。	
\\	彼女は来年、大学に進学する。	進学	彼女[かのじょ]は 来年[らいねん]、 大学[だいがく]に 進学[しんがく]する。	
\\	この図を拡大コピーしてください。	拡大	この 図[ず]を 拡大[かくだい]コピーしてください。	
\\	晩ご飯のおかずは何?	おかず	晩[ばん]ご 飯[はん]のおかずは 何?[なに]	
\\	彼は失敗の責任を取って、会社を辞めた。	責任	彼[かれ]は 失敗[しっぱい]の 責任[せきにん]を 取[と]って、 会社[かいしゃ]を 辞[や]めた。	
\\	彼は会社を辞めます。	辞める	彼[かれ]は 会社[かいしゃ]を 辞[や]めます。	
\\	彼女は辞書をよく使います。	辞書	彼女[かのじょ]は 辞書[じしょ]をよく 使[つか]います。	
\\	毎朝、通勤に30分かかります。	通勤	毎朝[まいあさ]、 通勤[つうきん]に 30分[さんじゅっぷん]かかります。	
\\	カレンダーに予定を書いた。	カレンダー	カレンダーに 予定[よてい]を 書[か]いた。	
\\	私は銀行に勤めています。	勤める	私[わたし]は 銀行[ぎんこう]に 勤[つと]めています。	
\\	後で事務所に来てください。	事務所	後[あと]で 事務所[じむしょ]に 来[き]てください。	
\\	事務室でコピーを取って来ます。	事務室	事務室[じむしつ]でコピーを 取[と]って 来[き]ます。	
\\	このプリンターは従来のものより速い。	従来	このプリンターは 従来[じゅうらい]のものより 速[はや]い。	
\\	ハンドバッグを忘れました。	ハンドバッグ	ハンドバッグを 忘[わす]れました。	
\\	成績が上がりました。	成績	成績[せいせき]が 上[あ]がりました。	
\\	弟は切手を集めています。	集める	弟[おとうと]は 切手[きって]を 集[あつ]めています。	
\\	駅前に人が集まっています。	集まる	駅前[えきまえ]に 人[ひと]が 集[あつ]まっています。	
\\	その会社は女性を多く採用している。	採用	その 会社[かいしゃ]は 女性[じょせい]を 多[おお]く 採用[さいよう]している。	
\\	玄関のベルが鳴った。	ベル	玄関[げんかん]のベルが 鳴[な]った。	
\\	来年から給料が上がります。	給料	来年[らいねん]から 給料[きゅうりょう]が 上[あ]がります。	
\\	私は去年、大学を卒業した。	卒業	私[わたし]は 去年[きょねん]、 大学[だいがく]を 卒業[そつぎょう]した。	
\\	最近、若い人たちの就職が難しくなっています。	就職	最近[さいきん]、 若[わか]い 人[ひと]たちの 就職[しゅうしょく]が 難[むずか]しくなっています。	
\\	母が今日退院します。	退院	母[はは]が 今日[きょう] 退院[たいいん]します。	
\\	プリンターのインクを買った。	インク	プリンターのインクを 買[か]った。	
\\	その選手は新しいチームと契約した。	契約	その 選手[せんしゅ]は 新[あたら]しいチームと 契約[けいやく]した。	
\\	今、値段を交渉しています。	交渉	今[いま]、 値段[ねだん]を 交渉[こうしょう]しています。	
\\	その事件の犯人はまだ捕まっていない。	事件	その 事件[じけん]の 犯人[はんにん]はまだ 捕[つか]まっていない。	
\\	この条件では厳し過ぎます。	条件	この 条件[じょうけん]では 厳[きび]し 過[す]ぎます。	
\\	暑いのでクーラーをつけました。	クーラー	暑[あつ]いのでクーラーをつけました。	
\\	明日は市民マラソンに参加します。	参加	明日[あす]は 市民[しみん]マラソンに 参加[さんか]します。	
\\	島の人口は年々増加しています。	増加	島[しま]の 人口[じんこう]は 年々[ねんねん] 増加[ぞうか]しています。	
\\	塩、コショウを加えてください。	加える	塩[しお]、コショウを 加[くわ]えてください。	
\\	この工場では魚を加工している。	加工	この 工場[こうじょう]では 魚[さかな]を 加工[かこう]している。	
\\	赤ちゃんにミルクをあげた。	ミルク	赤[あか]ちゃんにミルクをあげた。	
\\	今月と先月の売上を比べた。	比べる	今月[こんげつ]と 先月[せんげつ]の 売上[うりあげ]を 比[くら]べた。	
\\	彼は同僚を批判した。	批判	彼[かれ]は 同僚[どうりょう]を 批判[ひはん]した。	
\\	最近、彼の評価が上がった。	評価	最近[さいきん]、 彼[かれ]の 評価[ひょうか]が 上[あ]がった。	
\\	このアンケートは大学生が対象です。	対象	このアンケートは 大学生[だいがくせい]が 対象[たいしょう]です。	
\\	サラダをたくさん食べました。	サラダ	サラダをたくさん 食[た]べました。	
\\	冷蔵庫が故障しました。	故障	冷蔵庫[れいぞうこ]が 故障[こしょう]しました。	
\\	車を修理に出した。	修理	車[くるま]を 修理[しゅうり]に 出[だ]した。	
\\	次の駅で地下鉄に乗り換えます。	乗り換える	次[つぎ]の 駅[えき]で 地下鉄[ちかてつ]に 乗[の]り 換[か]えます。	
\\	次の駅で乗り換えです。	乗り換え	次[つぎ]の 駅[えき]で 乗[の]り 換[か]えです。	
\\	昨日おじさんに会いました。	おじさん	昨日[きのう]おじさんに 会[あ]いました。	
\\	車のタイヤを換えた。	換える	車[くるま]のタイヤを 換[か]えた。	
\\	彼はきれいな服に着替えた。	着替える	彼[かれ]はきれいな 服[ふく]に 着替[きが]えた。	
\\	帽子を被って外出した。	被る	帽子[ぼうし]を 被[かぶ]って 外出[がいしゅつ]した。	
\\	彼は約束を破った。	破る	彼[かれ]は 約束[やくそく]を 破[やぶ]った。	
\\	ベルトがきつくなりました。	ベルト	ベルトがきつくなりました。	
\\	シャツが破れている。	破れる	シャツが 破[やぶ]れている。	
\\	彼女が私のケータイを壊した。	壊す	彼女[かのじょ]が 私[わたし]のケータイを 壊[こわ]した。	
\\	会社のパソコンが壊れた。	壊れる	会社[かいしゃ]のパソコンが 壊[こわ]れた。	
\\	誰か救急車を呼んでください。	救急車	誰[だれ]か 救急車[きゅうきゅうしゃ]を 呼[よ]んでください。	
\\	夕食にラーメンを食べました。	ラーメン	夕食[ゆうしょく]にラーメンを 食[た]べました。	
\\	彼女は病気の犬を助けた。	助ける	彼女[かのじょ]は 病気[びょうき]の 犬[いぬ]を 助[たす]けた。	
\\	彼は立派な人です。	立派	彼[かれ]は 立派[りっぱ]な 人[ひと]です。	
\\	警察を呼んでください。	警察	警察[けいさつ]を 呼[よ]んでください。	
\\	彼女が私のスケジュールを管理している。	管理	彼女[かのじょ]が 私[わたし]のスケジュールを 管理[かんり]している。	
\\	店にライターを忘れました。	ライター	店[みせ]にライターを 忘[わす]れました。	
\\	誰かが私のカバンを盗みました。	盗む	誰[だれ]かが 私[わたし]のカバンを 盗[ぬす]みました。	
\\	私は生き物を殺すのが嫌いだ。	殺す	私[わたし]は 生[い]き 物[もの]を 殺[ころ]すのが 嫌[きら]いだ。	
\\	逃げていた犯人が逮捕されました。	逮捕	逃[に]げていた 犯人[はんにん]が 逮捕[たいほ]されました。	
\\	鳥が窓から逃げたよ。	逃げる	鳥[とり]が 窓[まど]から 逃[に]げたよ。	
\\	彼は白いワイシャツを着ている。	ワイシャツ	彼[かれ]は 白[しろ]いワイシャツを 着[き]ている。	
\\	2003年にイラクで戦争があった。	戦争	2003年[にせんさんねん]にイラクで 戦争[せんそう]があった。	
\\	2社は互いに競争している。	競争	2社[にしゃ]は 互[たが]いに 競争[きょうそう]している。	
\\	電車が込んでいる。	混む・込む	電車[でんしゃ]が 込[こ]んでいる。	
\\	フランスの大統領は誰ですか。	大統領	フランスの 大統領[だいとうりょう]は 誰[だれ]ですか。	
\\	朝、急いでシャワーを浴びました。	シャワー	朝[あさ]、 急[いそ]いでシャワーを 浴[あ]びました。	
\\	ゴミを捨ててください。	捨てる	ゴミを 捨[す]ててください。	
\\	道で財布を拾った。	拾う	道[みち]で 財布[さいふ]を 拾[ひろ]った。	
\\	池に鯉がいます。	池	池[いけ]に 鯉[こい]がいます。	
\\	新しい電池を入れましょう。	電池	新[あたら]しい 電池[でんち]を 入[い]れましょう。	
\\	私はお昼にうどんを食べました。	うどん	私[わたし]はお 昼[ひる]にうどんを 食[た]べました。	
\\	あの池はとても深い。	深い	あの 池[いけ]はとても 深[ふか]い。	
\\	このプールの深さは2メートルです。	深さ	このプールの 深[ふか]さは 2[に]メートルです。	
\\	この川は浅いです。	浅い	この 川[かわ]は 浅[あさ]いです。	
\\	途中で財布を落としました。	落とす	途中[とちゅう]で 財布[さいふ]を 落[お]としました。	
\\	今日はごみの日だ。	ごみ	今日[きょう]はごみの 日[ひ]だ。	
\\	彼女はダイエットのために泳いでいる。	泳ぐ	彼女[かのじょ]はダイエットのために 泳[およ]いでいる。	
\\	母は健康のために水泳をしている。	水泳	母[はは]は 健康[けんこう]のために 水泳[すいえい]をしている。	
\\	ラジオから美しい音楽が流れています。	流れる	ラジオから 美[うつく]しい 音楽[おんがく]が 流[なが]れています。	
\\	去年はスニーカーが流行りました。	流行る	去年[きょねん]はスニーカーが 流行[はや]りました。	
\\	スーパーマーケットで人参を買った。	スーパーマーケット	スーパーマーケットで 人参[にんじん]を 買[か]った。	
\\	早く顔を洗いなさい。	洗う	早[はや]く 顔[かお]を 洗[あら]いなさい。	
\\	洗面所で顔を洗った。	洗面所	洗面所[せんめんじょ]で 顔[かお]を 洗[あら]った。	
\\	水と油は混ざらない。	油	水[みず]と 油[あぶら]は 混[ま]ざらない。	
\\	ボートが川に沈んだ。	沈む	ボートが 川[かわ]に 沈[しず]んだ。	
\\	ハンカチで手をふきました。	ハンカチ	ハンカチで 手[て]をふきました。	
\\	明日、久しぶりに友達に会います。	久しぶり	明日[あした]、 久[ひさ]しぶりに 友達[ともだち]に 会[あ]います。	
\\	グラスに氷を入れてください。	氷	グラスに 氷[こおり]を 入[い]れてください。	
\\	クーラーで体が冷えた。	冷える	クーラーで 体[からだ]が 冷[ひ]えた。	
\\	頭を冷やしなさい。	冷やす	頭[あたま]を 冷[ひ]やしなさい。	
\\	私はビデオテープを30本持っています。	ビデオテープ	私[わたし]はビデオテープを 30本持[さんじゅっぽん も]っています。	
\\	彼への気持ちが冷めた。	冷める	彼[かれ]への 気持[きも]ちが 冷[さ]めた。	
\\	寒い朝は道路が凍ります。	凍る	寒[さむ]い 朝[あさ]は 道路[どうろ]が 凍[こお]ります。	
\\	飲み物は冷蔵庫にあります。	冷蔵庫	飲[の]み 物[もの]は 冷蔵庫[れいぞうこ]にあります。	
\\	夕方は涼しくなりますよ。	涼しい	夕方[ゆうがた]は 涼[すず]しくなりますよ。	
\\	朝、近所をぶらぶらした。	ぶらぶら	朝[あさ]、 近所[きんじょ]をぶらぶらした。	
\\	彼は服を汚した。	汚す	彼[かれ]は 服[ふく]を 汚[よご]した。	
\\	靴の汚れを落としました。	汚れ	靴[くつ]の 汚[よご]れを 落[お]としました。	
\\	エプロンをしないと服が汚れます。	汚れる	エプロンをしないと 服[ふく]が 汚[よご]れます。	
\\	ここは景色がきれいですね。	景色	ここは 景色[けしき]がきれいですね。	
\\	私はアナウンサーになりたい。	アナウンサー	私[わたし]はアナウンサーになりたい。	
\\	私は彼から大きな影響を受けました。	影響	私[わたし]は 彼[かれ]から 大[おお]きな 影響[えいきょう]を 受[う]けました。	
\\	波がきらきら光っていました。	光る	波[なみ]がきらきら 光[ひか]っていました。	
\\	私の姉はすぐ太ります。	太る	私[わたし]の 姉[あね]はすぐ 太[ふと]ります。	
\\	太陽が雲に隠れた。	太陽	太陽[たいよう]が 雲[くも]に 隠[かく]れた。	
\\	カレーライスはスプーンで食べます。	スプーン	カレーライスはスプーンで 食[た]べます。	
\\	今夜は星がよく見えます。	星	今夜[こんや]は 星[ほし]がよく 見[み]えます。	
\\	地球は丸い。	地球	地球[ちきゅう]は 丸[まる]い。	
\\	友達と野球をしました。	野球	友達[ともだち]と 野球[やきゅう]をしました。	
\\	今日は雲が多い。	雲	今日[きょう]は 雲[くも]が 多[おお]い。	
\\	彼女は白いブラウスを着ている。	ブラウス	彼女[かのじょ]は 白[しろ]いブラウスを 着[き]ている。	
\\	今日は一日曇りでした。	曇り	今日[きょう]は 一日[いちにち] 曇[くも]りでした。	
\\	日本は地震が多いです。	地震	日本[にほん]は 地震[じしん]が 多[おお]いです。	
\\	彼は寒くて震えていました。	震える	彼[かれ]は 寒[さむ]くて 震[ふる]えていました。	
\\	犬がしっぽを振っている。	振る	犬[いぬ]がしっぽを 振[ふ]っている。	
\\	プレゼントにリボンを付けた。	リボン	プレゼントにリボンを 付[つ]けた。	
\\	風で木が揺れています。	揺れる	風[かぜ]で 木[き]が 揺[ゆ]れています。	
\\	京都には神社がたくさんある。	神社	京都[きょうと]には 神社[じんじゃ]がたくさんある。	
\\	これは秘密です。	秘密	これは 秘密[ひみつ]です。	
\\	私の上司はとても厳しい。	厳しい	私[わたし]の 上司[じょうし]はとても 厳[きび]しい。	
\\	お加減はいかがですか。	いかが	お 加減[かげん]はいかがですか。	
\\	あの村にはお年寄りがたくさん住んでいます。	年寄り	あの 村[むら]にはお 年寄[としよ]りがたくさん 住[す]んでいます。	
\\	私は歴史に興味があります。	歴史	私[わたし]は 歴史[れきし]に 興味[きょうみ]があります。	
\\	新しい世紀の始まりです。	世紀	新[あたら]しい 世紀[せいき]の 始[はじ]まりです。	
\\	新しいビルの建設が始まった。	建設	新[あたら]しいビルの 建設[けんせつ]が 始[はじ]まった。	
\\	彼はステレオで音楽を聞いた。	ステレオ	彼[かれ]はステレオで 音楽[おんがく]を 聞[き]いた。	
\\	これは日本一古い建物です。	建物	これは 日本一古[にほんいち ふる]い 建物[たてもの]です。	
\\	ここに来年、家が建ちます。	建つ	ここに 来年[らいねん]、 家[いえ]が 建[た]ちます。	
\\	システムの構成を変えてみました。	構成	システムの 構成[こうせい]を 変[か]えてみました。	
\\	この建物の構造は複雑です。	構造	この 建物[たてもの]の 構造[こうぞう]は 複雑[ふくざつ]です。	
\\	フォークをもらえますか。	フォーク	フォークをもらえますか。	
\\	あの橋は日本で一番長い。	橋	あの 橋[はし]は 日本[にほん]で 一番長[いちばん なが]い。	
\\	この家の柱は太い。	柱	この 家[いえ]の 柱[はしら]は 太[ふと]い。	
\\	私の町は東京の北に位置します。	位置	私[わたし]の 町[まち]は 東京[とうきょう]の 北[きた]に 位置[いち]します。	
\\	友人が離婚しました。	離婚	友人[ゆうじん]が 離婚[りこん]しました。	
\\	おばは大阪に住んでいます。	おば	おばは 大阪[おおさか]に 住[す]んでいます。	
\\	この電車は東京まで停車致しません。	停車	この 電車[でんしゃ]は 東京[とうきょう]まで 停車[ていしゃ] 致[いた]しません。	
\\	次のバス停で降ります。	バス停	次[つぎ]のバス 停[てい]で 降[お]ります。	
\\	この周辺には大学が多い。	周辺	この 周辺[しゅうへん]には 大学[だいがく]が 多[おお]い。	
\\	隣の家には犬がいます。	隣	隣[となり]の 家[いえ]には 犬[いぬ]がいます。	
\\	星がきらきら光っている。	きらきら	星[ほし]がきらきら 光[ひか]っている。	
\\	信号は黄色でした。	黄色	信号[しんごう]は 黄色[きいろ]でした。	
\\	横の長さは1メートルです。	横	横[よこ]の 長[なが]さは 1[いち]メートルです。	
\\	この本は横書きです。	横書き	この 本[ほん]は 横書[よこが]きです。	
\\	彼の判断は正しい。	判断	彼[かれ]の 判断[はんだん]は 正[ただ]しい。	
\\	こちらへどうぞ。	どうぞ	こちらへどうぞ。	
\\	私は彼のプロポーズを断った。	断る	私[わたし]は 彼[かれ]のプロポーズを 断[ことわ]った。	
\\	あそこに横断歩道があります。	横断歩道	あそこに 横断歩道[おうだんほどう]があります。	
\\	計画を大幅に変更した。	大幅	計画[けいかく]を 大幅[おおはば]に 変更[へんこう]した。	
\\	私は夏に京都を訪れました。	訪れる	私[わたし]は 夏[なつ]に 京都[きょうと]を 訪[おとず]れました。	
\\	妻にネックレスをプレゼントしました。	ネックレス	妻[つま]にネックレスをプレゼントしました。	
\\	彼はニューヨークの友達を訪ねた。	訪ねる	彼[かれ]はニューヨークの 友達[ともだち]を 訪[たず]ねた。	
\\	冷房を入れてください。	冷房	冷房[れいぼう]を 入[い]れてください。	
\\	冬は暖房が必要です。	暖房	冬[ふゆ]は 暖房[だんぼう]が 必要[ひつよう]です。	
\\	新しい文房具を買いました。	文房具	新[あたら]しい 文房具[ぶんぼうぐ]を 買[か]いました。	
\\	このパジャマを着て。	パジャマ	このパジャマを 着[き]て。	
\\	彼はヨーロッパ諸国を旅行した。	諸国	彼[かれ]はヨーロッパ 諸国[しょこく]を 旅行[りょこう]した。	
\\	彼は緑色のジャケットを着ています。	緑色	彼[かれ]は 緑色[みどりいろ]のジャケットを 着[き]ています。	
\\	父は貿易の仕事をしています。	貿易	父[ちち]は 貿易[ぼうえき]の 仕事[しごと]をしています。	
\\	これはイタリアから輸入した服です。	輸入	これはイタリアから 輸入[ゆにゅう]した 服[ふく]です。	
\\	ボールペンで名前を書いてください。	ボールペン	ボールペンで 名前[なまえ]を 書[か]いてください。	
\\	彼の会社は車を輸出しています。	輸出	彼[かれ]の 会社[かいしゃ]は 車[くるま]を 輸出[ゆしゅつ]しています。	
\\	彼女に指輪をプレゼントしました。	指輪	彼女[かのじょ]に 指輪[ゆびわ]をプレゼントしました。	
\\	往復切符をください。	往復	往復[おうふく] 切符[きっぷ]をください。	
\\	昨日の復習をしましたか。	復習	昨日[きのう]の 復習[ふくしゅう]をしましたか。	
\\	夏はすいかが美味しい。	すいか	夏[なつ]はすいかが 美味[おい]しい。	
\\	彼女は同じ間違いを繰り返した。	繰り返す	彼女[かのじょ]は 同[おな]じ 間違[まちが]いを 繰[く]り 返[かえ]した。	
\\	1年間、アメリカに留学しました。	留学	1年間[いちねんかん]、アメリカに 留学[りゅうがく]しました。	
\\	バスの停留所で10分待ちました。	停留所	バスの 停留所[ていりゅうじょ]で 10分待[じゅっぷん ま]ちました。	
\\	これを書留で送りたいのですが。	書留	これを 書留[かきとめ]で 送[おく]りたいのですが。	
\\	夕方そちらに着きます。	そちら	夕方[ゆうがた]そちらに 着[つ]きます。	
\\	彼は約束を守る人です。	守る	彼[かれ]は 約束[やくそく]を 守[まも]る 人[ひと]です。	
\\	父は今、留守です。	留守	父[ちち]は 今[いま]、 留守[るす]です。	
\\	ここは静かな住宅地だ。	住宅	ここは 静[しず]かな 住宅[じゅうたく] 地[ち]だ。	
\\	自宅に電話を下さい。	自宅	自宅[じたく]に 電話[でんわ]を 下[くだ]さい。	
\\	お昼にそばを食べた。	そば	お 昼[ひる]にそばを 食[た]べた。	
\\	先生のお宅はどちらですか。	お宅	先生[せんせい]のお 宅[たく]はどちらですか。	
\\	祖父は早起きです。	早起き	祖父[そふ]は 早起[はやお]きです。	
\\	私の子供は毎日昼寝をします。	昼寝	私[わたし]の 子供[こども]は 毎日[まいにち] 昼寝[ひるね]をします。	
\\	今年の夏休みには静かな所へ行きたい。	静か	今年[ことし]の 夏休[なつやす]みには 静[しず]かな 所[ところ]へ 行[い]きたい。	
\\	どっちの色が好きですか。	どっち	どっちの 色[いろ]が 好[す]きですか。	
\\	明日は暇ですか。	暇	明日[あした]は 暇[ひま]ですか。	
\\	私の趣味は映画とテニスです。	趣味	私[わたし]の 趣味[しゅみ]は 映画[えいが]とテニスです。	
\\	チョコレートとケーキを両方ください。	両方	チョコレートとケーキを 両方[りょうほう]ください。	
\\	1万円を両替してください。	両替	1万円[いちまんえん]を 両替[りょうがえ]してください。	
\\	姉はバスケットボールの選手です。	バスケットボール	姉[あね]はバスケットボールの 選手[せんしゅ]です。	
\\	私の両親は大阪に住んでいます。	両親	私[わたし]の 両親[りょうしん]は 大阪[おおさか]に 住[す]んでいます。	
\\	東京まで片道切符を買った。	片道	東京[とうきょう]まで 片道[かたみち] 切符[きっぷ]を 買[か]った。	
\\	白線の内側に下がってください。	内側	白線[はくせん]の 内側[うちがわ]に 下[さ]がってください。	
\\	私の家は川の向こう側にあります。	向こう側	私[わたし]の 家[いえ]は 川[かわ]の 向[む]こう 側[がわ]にあります。	
\\	今日のお昼はサンドイッチです。	サンドイッチ	今日[きょう]のお 昼[ひる]はサンドイッチです。	
\\	白線の外側を歩かないでください。	外側	白線[はくせん]の 外側[そとがわ]を 歩[ある]かないでください。	
\\	画面の左側を見てください。	左側	画面[がめん]の 左側[ひだりがわ]を 見[み]てください。	
\\	彼女はいつも私の右側を歩きます。	右側	彼女[かのじょ]はいつも 私[わたし]の 右側[みぎがわ]を 歩[ある]きます。	
\\	会社の裏に公園があります。	裏	会社[かいしゃ]の 裏[うら]に 公園[こうえん]があります。	
\\	彼女はいつもにこにこしています。	にこにこ	彼女[かのじょ]はいつもにこにこしています。	
\\	彼は紙を裏返しました。	裏返す	彼[かれ]は 紙[かみ]を 裏返[うらがえ]しました。	
\\	その週刊誌はゴシップばかりだ。	週刊誌	その 週刊誌[しゅうかんし]はゴシップばかりだ。	
\\	今日の朝刊に面白い記事があった。	朝刊	今日[きょう]の 朝刊[ちょうかん]に 面白[おもしろ]い 記事[きじ]があった。	
\\	そのニュースは夕刊で見ました。	夕刊	そのニュースは 夕刊[ゆうかん]で 見[み]ました。	
\\	このウールのセーターは暖かい。	ウール	このウールのセーターは 暖[あたた]かい。	
\\	もっと詳しく説明してください。	詳しい	もっと 詳[くわ]しく 説明[せつめい]してください。	
\\	彼女は細かいことにうるさい。	細かい	彼女[かのじょ]は 細[こま]かいことにうるさい。	
\\	それは細かな問題です。	細か	それは 細[こま]かな 問題[もんだい]です。	
\\	雪が積もっています。	積もる	雪[ゆき]が 積[つ]もっています。	
\\	私の兄はコーラが好きだ。	コーラ	私[わたし]の 兄[あに]はコーラが 好[す]きだ。	
\\	その文の訳を読みました。	訳	その 文[ぶん]の 訳[やく]を 読[よ]みました。	
\\	この文を訳してください。	訳す	この 文[ぶん]を 訳[やく]してください。	
\\	今日中にこの問題を検討してください。	検討	今日中[きょう じゅう]にこの 問題[もんだい]を 検討[けんとう]してください。	
\\	壁にペンキを塗っています。	塗る	壁[かべ]にペンキを 塗[ぬ]っています。	
\\	トイレットペーパーを交換しました。	トイレットペーパー	トイレットペーパーを 交換[こうかん]しました。	
\\	靴に泥が付いています。	付く	靴[くつ]に 泥[どろ]が 付[つ]いています。	
\\	おまけを付けました。	付ける	おまけを 付[つ]けました。	
\\	郵便物は5時まで受け付けています。	受け付ける	郵便物[ゆうびんぶつ]は 5時[ごじ]まで 受[う]け 付[つ]けています。	
\\	早く部屋を片付けなさい。	片付ける	早[はや]く 部屋[へや]を 片付[かたづ]けなさい。	
\\	失礼ですが、どなたですか。	どなた	失礼[しつれい]ですが、どなたですか。	
\\	受付は9時からです。	受付	受付[うけつけ]は 9時[くじ]からです。	
\\	女性が近付いてきました。	近付く	女性[じょせい]が 近付[ちかづ]いてきました。	
\\	仕事が大分片付いた。	片付く	仕事[しごと]が 大分[だいぶ] 片付[かたづ]いた。	
\\	体に気を付けてください。	気を付ける	体[からだ]に 気[き]を 付[つ]けてください。	
\\	昨日は友人とハイキングに行きました。	ハイキング	昨日[きのう]は 友人[ゆうじん]とハイキングに 行[い]きました。	
\\	机にシールを貼った。	貼る	机[つくえ]にシールを 貼[は]った。	
\\	今日は両親の結婚記念日です。	記念日	今日[きょう]は 両親[りょうしん]の 結婚[けっこん] 記念日[きねんび]です。	
\\	その試合は残念な結果になった。	残念	その 試合[しあい]は 残念[ざんねん]な 結果[けっか]になった。	
\\	例えば、このソフトで日本語を勉強することができます。	例えば	例[たと]えば、このソフトで 日本語[にほんご]を 勉強[べんきょう]することができます。	
\\	ハムサンドをください。	ハム	ハムサンドをください。	
\\	例文を3つ作ってください。	例文	例文[れいぶん]を 3[みっ]つ 作[つく]ってください。	
\\	一つ例をあげてください。	例	一[ひと]つ 例[れい]をあげてください。	
\\	夕食の料理がたくさん余った。	余る	夕食[ゆうしょく]の 料理[りょうり]がたくさん 余[あま]った。	
\\	部長を除く全員が土曜日も働いた。	除く	部長[ぶちょう]を 除[のぞ]く 全員[ぜんいん]が 土曜日[どようび]も 働[はたら]いた。	
\\	私のボーイフレンドはハンサムな方でした。	ボーイフレンド	私[わたし]のボーイフレンドはハンサムな 方[かた]でした。	
\\	彼は家族のために仕事の時間を削った。	削る	彼[かれ]は 家族[かぞく]のために 仕事[しごと]の 時間[じかん]を 削[けず]った。	
\\	遅刻しないでください。	遅刻	遅刻[ちこく]しないでください。	
\\	このズボンは緩いです。	緩い	このズボンは 緩[ゆる]いです。	
\\	食べ過ぎておなかが苦しい。	苦しい	食[た]べ 過[す]ぎておなかが 苦[くる]しい。	
\\	彼女はシャツにアイロンをかけた。	アイロン	彼女[かのじょ]はシャツにアイロンをかけた。	
\\	私は苦いコーヒーが好きです。	苦い	私[わたし]は 苦[にが]いコーヒーが 好[す]きです。	
\\	私は料理が苦手です。	苦手	私[わたし]は 料理[りょうり]が 苦手[にがて]です。	
\\	ケータイをなくして困っています。	困る	ケータイをなくして 困[こま]っています。	
\\	彼は昔は貧乏だった。	貧乏	彼[かれ]は 昔[むかし]は 貧乏[びんぼう]だった。	
\\	子供はカレーライスが好きです。	カレーライス	子供[こども]はカレーライスが 好[す]きです。	
\\	彼女の家族に不幸があった。	不幸	彼女[かのじょ]の 家族[かぞく]に 不幸[ふこう]があった。	
\\	良い友達がいて私は幸せだ。	幸せ	良[い]い 友達[ともだち]がいて 私[わたし]は 幸[しあわ]せだ。	
\\	もうちょっと塩を入れて。	塩	もうちょっと 塩[しお]を 入[い]れて。	
\\	海の水は塩辛い。	塩辛い	海[うみ]の 水[みず]は 塩辛[しおから]い。	
\\	キャッシュカードでお金を下ろした。	キャッシュカード	キャッシュカードでお 金[かね]を 下[お]ろした。	
\\	コーヒーに砂糖は入れますか。	砂糖	コーヒーに 砂糖[さとう]は 入[い]れますか。	
\\	この動物園は日本一の規模です。	規模	この 動物園[どうぶつえん]は 日本一[にっぽんいち]の 規模[きぼ]です。	
\\	私は農業を勉強しています。	農業	私[わたし]は 農業[のうぎょう]を 勉強[べんきょう]しています。	
\\	私は濃い味が好きだ。	濃い	私[わたし]は 濃[こ]い 味[あじ]が 好[す]きだ。	
\\	彼女はすてきなイヤリングをしている。	イヤリング	彼女[かのじょ]はすてきなイヤリングをしている。	
\\	この電子辞書はとても薄い。	薄い	この 電子辞書[でんし じしょ]はとても 薄[うす]い。	
\\	その辞書はとても厚い。	厚い	その 辞書[じしょ]はとても 厚[あつ]い。	
\\	私は板の厚さを測った。	厚さ	私[わたし]は 板[いた]の 厚[あつ]さを 測[はか]った。	
\\	締め切りが迫っています。	迫る	締[し]め 切[き]りが 迫[せま]っています。	
\\	昔は彼女のことをガールフレンドと言っていました。	ガールフレンド	昔[むかし]は 彼女[かのじょ]のことをガールフレンドと 言[い]っていました。	
\\	髪がだいぶ伸びたね。	伸びる	髪[かみ]がだいぶ 伸[の]びたね。	
\\	来月、大阪に引っ越します。	引っ越す	来月[らいげつ]、 大阪[おおさか]に 引[ひ]っ 越[こ]します。	
\\	私たちは高い山を越えました。	越える	私[わたし]たちは 高[たか]い 山[やま]を 越[こ]えました。	
\\	去年、引っ越ししました。	引っ越し	去年[きょねん]、 引[ひ]っ 越[こ]ししました。	
\\	昔はカセットテープを使っていました。	カセットテープ	昔[むかし]はカセットテープを 使[つか]っていました。	
\\	大きなトラックが私たちを追い越した。	追い越す	大[おお]きなトラックが 私[わたし]たちを 追[お]い 越[こ]した。	
\\	地球の気温は上昇している。	上昇	地球[ちきゅう]の 気温[きおん]は 上昇[じょうしょう]している。	
\\	改札口で会いましょう。	改札口	改札口[かいさつぐち]で 会[あ]いましょう。	
\\	ではそろそろ失礼します。	失礼	ではそろそろ 失礼[しつれい]します。	
\\	背中がかゆいです。	かゆい	背中[せなか]がかゆいです。	
\\	彼女にお礼の手紙を書きました。	お礼	彼女[かのじょ]にお 礼[れい]の 手紙[てがみ]を 書[か]きました。	
\\	彼は直ぐに謝りました。	謝る	彼[かれ]は 直[す]ぐに 謝[あやま]りました。	
\\	彼は注射があまり好きではありません。	注射	彼[かれ]は 注射[ちゅうしゃ]があまり 好[す]きではありません。	
\\	この程度の怪我なら大丈夫です。	程度	この 程度[ていど]の 怪我[けが]なら 大丈夫[だいじょうぶ]です。	
\\	地震で家がぐらぐらと揺れた。	ぐらぐら	地震[じしん]で 家[いえ]がぐらぐらと 揺[ゆ]れた。	
\\	彼女をデートに誘った。	誘う	彼女[かのじょ]をデートに 誘[さそ]った。	
\\	会社で新しいシステムを導入した。	導入	会社[かいしゃ]で 新[あたら]しいシステムを 導入[どうにゅう]した。	
\\	もっと努力しよう。	努力	もっと 努力[どりょく]しよう。	
\\	彼女が嘘をついたので、彼は怒った。	怒る	彼女[かのじょ]が 嘘[うそ]をついたので、 彼[かれ]は 怒[おこ]った。	
\\	このイチゴでジャムを作りましょう。	ジャム	このイチゴでジャムを 作[つく]りましょう。	
\\	私の兄はまだ独身です。	独身	私[わたし]の 兄[あに]はまだ 独身[どくしん]です。	
\\	私に届くメールのうち、迷惑メールが7割を占めている。	占める	私[わたし]に 届[とど]くメールのうち、 迷惑[めいわく]メールが 7割[ななわり]を 占[し]めている。	
\\	事務的な処理に1週間かかります。	処理	事務的[じむてき]な 処理[しょり]に 1週間[いっしゅうかん]かかります。	
\\	両親に彼女を紹介した。	紹介	両親[りょうしん]に 彼女[かのじょ]を 紹介[しょうかい]した。	
\\	スリッパをはいてください。	スリッパ	スリッパをはいてください。	
\\	両親を食事に招いた。	招く	両親[りょうしん]を 食事[しょくじ]に 招[まね]いた。	
\\	高校の時の先生を結婚式に招待した。	招待	高校[こうこう]の 時[とき]の 先生[せんせい]を 結婚式[けっこんしき]に 招待[しょうたい]した。	
\\	その夫婦はとても仲がいい。	夫婦	その 夫婦[ふうふ]はとても 仲[なか]がいい。	
\\	はさみは机の奥にあった。	奥	はさみは 机[つくえ]の 奥[おく]にあった。	
\\	友達とトランプをして遊びました。	トランプ	友達[ともだち]とトランプをして 遊[あそ]びました。	
\\	彼の奥さんはきれいな方です。	奥さん	彼[かれ]の 奥[おく]さんはきれいな 方[かた]です。	
\\	皆さんにお話があります。	皆さん	皆[みな]さんにお 話[はなし]があります。	
\\	皆様、こんにちは。	皆様	皆様[みなさま]、こんにちは。	
\\	誰かに聞いてみてください。	誰か	誰[だれ]かに 聞[き]いてみてください。	
\\	レポートはこれでよろしいですか。	よろしい	レポートはこれでよろしいですか。	
\\	私は日本国籍です。	国籍	私[わたし]は 日本[にほん] 国籍[こくせき]です。	
\\	彼女は愛をこめて手紙を書いた。	愛	彼女[かのじょ]は 愛[あい]をこめて 手紙[てがみ]を 書[か]いた。	
\\	彼女の赤ちゃんは可愛いです。	可愛い	彼女[かのじょ]の 赤[あか]ちゃんは 可愛[かわい]いです。	
\\	彼は恋人を失った。	恋人	彼[かれ]は 恋人[こいびと]を 失[うしな]った。	
\\	私はラッシュアワーの電車が嫌いです。	ラッシュアワー	私[わたし]はラッシュアワーの 電車[でんしゃ]が 嫌[きら]いです。	
\\	彼女の誕生日は7月16日です。	誕生日	彼女[かのじょ]の 誕生日[たんじょうび]は 7月16日[しちがつ じゅうろくにち]です。	
\\	5月3日は祝日だ。	祝日	5月3日[ごがつ みっか]は 祝日[しゅくじつ]だ。	
\\	彼の就職のお祝いをしよう。	お祝い	彼[かれ]の 就職[しゅうしょく]のお 祝[いわ]いをしよう。	
\\	昨夜恐ろしい夢を見た。	夢	昨夜[ゆうべ] 恐[おそ]ろしい 夢[ゆめ]を 見[み]た。	
\\	水を1カップ入れてください。	カップ	水[みず]を 1[いち]カップ 入[い]れてください。	
\\	妹はすぐに泣く。	泣く	妹[いもうと]はすぐに 泣[な]く。	
\\	彼女の目から涙がこぼれた。	涙	彼女[かのじょ]の 目[め]から 涙[なみだ]がこぼれた。	
\\	彼女はとても喜びました。	喜ぶ	彼女[かのじょ]はとても 喜[よろこ]びました。	
\\	とても恥ずかしかった。	恥ずかしい	とても 恥[は]ずかしかった。	
\\	湖でスケートをした。	スケート	湖[みずうみ]でスケートをした。	
\\	今日は弁当を持ってきました。	弁当	今日[きょう]は 弁当[べんとう]を 持[も]ってきました。	
\\	母は看護婦です。	看護婦	母[はは]は 看護婦[かんごふ]です。	
\\	患者は眠っています。	患者	患者[かんじゃ]は 眠[ねむ]っています。	
\\	上司が意見を述べた。	述べる	上司[じょうし]が 意見[いけん]を 述[の]べた。	
\\	昨日はたくさん泳ぎました。	たくさん	昨日[きのう]はたくさん 泳[およ]ぎました。	
\\	彼女は会社を訴えた。	訴える	彼女[かのじょ]は 会社[かいしゃ]を 訴[うった]えた。	
\\	道に迷いました。	迷う	道[みち]に 迷[まよ]いました。	
\\	人に迷惑をかけてはいけません。	迷惑	人[ひと]に 迷惑[めいわく]をかけてはいけません。	
\\	この地域は雨が多い。	地域	この 地域[ちいき]は 雨[あめ]が 多[おお]い。	
\\	今日はピクニックに行きましょう。	ピクニック	今日[きょう]はピクニックに 行[い]きましょう。	
\\	政権が交代した。	政権	政権[せいけん]が 交代[こうたい]した。	
\\	母に花を贈った。	贈る	母[はは]に 花[はな]を 贈[おく]った。	
\\	すてきな贈り物をどうもありがとう。	贈り物	すてきな 贈[おく]り 物[もの]をどうもありがとう。	
\\	そのニュースは彼に大きなショックを与えた。	与える	そのニュースは 彼[かれ]に 大[おお]きなショックを 与[あた]えた。	
\\	あちらに行ってみよう。	あちら	あちらに 行[い]ってみよう。	
\\	貯金は十分にあります。	貯金	貯金[ちょきん]は 十分[じゅうぶん]にあります。	
\\	鍵を彼に預けた。	預ける	鍵[かぎ]を 彼[かれ]に 預[あず]けた。	
\\	荷物を預かってください。	預かる	荷物[にもつ]を 預[あず]かってください。	
\\	彼は歌が得意です。	得意	彼[かれ]は 歌[うた]が 得意[とくい]です。	
\\	彼女に会うとどきどきします。	どきどき	彼女[かのじょ]に 会[あ]うとどきどきします。	
\\	山が燃えています。	燃える	山[やま]が 燃[も]えています。	
\\	肉が焼けました。	焼ける	肉[にく]が 焼[や]けました。	
\\	昨日の夜はすき焼きを食べた。	すき焼き	昨日[きのう]の 夜[よる]はすき 焼[や]きを 食[た]べた。	
\\	今、魚を焼いています。	焼く	今[いま]、 魚[さかな]を 焼[や]いています。	
\\	彼の成績はまあまあです。	まあまあ	彼[かれ]の 成績[せいせき]はまあまあです。	
\\	夏は洗濯物がすぐ乾く。	乾く	夏[なつ]は 洗濯物[せんたくもの]がすぐ 乾[かわ]く。	
\\	お二人の未来に乾杯しましょう。	乾杯	お 二人[ふたり]の 未来[みらい]に 乾杯[かんぱい]しましょう。	
\\	ぬれた服を乾かした。	乾かす	ぬれた 服[ふく]を 乾[かわ]かした。	
\\	新幹線で京都に行きました。	新幹線	新幹線[しんかんせん]で 京都[きょうと]に 行[い]きました。	
\\	彼女は首にマフラーをまいていた。	マフラー	彼女[かのじょ]は 首[くび]にマフラーをまいていた。	
\\	素晴らしい景色ですね。	素晴らしい	素晴[すば]らしい 景色[けしき]ですね。	
\\	海岸を散歩しましょう。	海岸	海岸[かいがん]を 散歩[さんぽ]しましょう。	
\\	彼は家庭を大切にしている。	家庭	彼[かれ]は 家庭[かてい]を 大切[たいせつ]にしている。	
\\	庭にバラを植えました。	庭	庭[にわ]にバラを 植[う]えました。	
\\	このアクセサリーは素敵ね。	アクセサリー	このアクセサリーは 素敵[すてき]ね。	
\\	桜は三月か四月に咲きます。	桜	桜[さくら]は 三月[さんがつ]か 四月[しがつ]に 咲[さ]きます。	
\\	桜の花が咲きました。	咲く	桜[さくら]の 花[はな]が 咲[さ]きました。	
\\	今日は北風が吹いている。	吹く	今日[きょう]は 北風[きたかぜ]が 吹[ふ]いている。	
\\	私のお祖父さんは毎日散歩します。	散歩	私[わたし]のお 祖父[じい]さんは 毎日[まいにち] 散歩[さんぽ]します。	
\\	父があくびをした。	あくび	父[ちち]があくびをした。	
\\	庭にバラを植えました。	植える	庭[にわ]にバラを 植[う]えました。	
\\	屋根にカラスが止まっています。	屋根	屋根[やね]にカラスが 止[と]まっています。	
\\	答えを黒板に書いてください。	黒板	答[こた]えを 黒板[こくばん]に 書[か]いてください。	
\\	掲示板のお知らせを見ましたか。	掲示板	掲示板[けいじばん]のお 知[し]らせを 見[み]ましたか。	
\\	おしゃべりして喉がからからになりました。	からから	おしゃべりして 喉[のど]がからからになりました。	
\\	庭に草が生えている。	草	庭[にわ]に 草[くさ]が 生[は]えている。	
\\	これは桜の葉です。	葉	これは 桜[さくら]の 葉[は]です。	
\\	友達から絵葉書が届きました。	絵葉書	友達[ともだち]から 絵葉書[えはがき]が 届[とど]きました。	
\\	家族に葉書を書いています。	葉書	家族[かぞく]に 葉書[はがき]を 書[か]いています。	
\\	私は疲れてふらふらです。	ふらふら	私[わたし]は 疲[つか]れてふらふらです。	
\\	彼は大きく息を吸った。	吸う	彼[かれ]は 大[おお]きく 息[いき]を 吸[す]った。	
\\	ゴミのリサイクルが普及している。	普及	ゴミのリサイクルが 普及[ふきゅう]している。	
\\	昨日から胃が痛い。	胃	昨日[きのう]から 胃[い]が 痛[いた]い。	
\\	皿にケーキを載せました。	皿	皿[さら]にケーキを 載[の]せました。	
\\	ソファにワインをこぼした。	こぼす	ソファにワインをこぼした。	
\\	血が出ていますよ。	血	血[ち]が 出[で]ていますよ。	
\\	この本の内容を説明してください。	内容	この 本[ほん]の 内容[ないよう]を 説明[せつめい]してください。	
\\	事件の背景に何があったのだろう。	背景	事件[じけん]の 背景[はいけい]に 何[なに]があったのだろう。	
\\	彼は足の骨を折りました。	骨	彼[かれ]は 足[あし]の 骨[ほね]を 折[お]りました。	
\\	一緒にランチに行きませんか。	ランチ	一緒[いっしょ]にランチに 行[い]きませんか。	
\\	彼は雪の上を滑った。	滑る	彼[かれ]は 雪[ゆき]の 上[うえ]を 滑[すべ]った。	
\\	強風で木の枝が折れた。	折れる	強風[きょうふう]で 木[き]の 枝[えだ]が 折[お]れた。	
\\	祖父は足の骨を折りました。	折る	祖父[そふ]は 足[あし]の 骨[ほね]を 折[お]りました。	
\\	私は健康な生活を送っています。	健康	私[わたし]は 健康[けんこう]な 生活[せいかつ]を 送[おく]っています。	
\\	レシートを財布に入れました。	レシート	レシートを 財布[さいふ]に 入[い]れました。	
\\	昨日珍しい果物を食べました。	珍しい	昨日[きのう] 珍[めずら]しい 果物[くだもの]を 食[た]べました。	
\\	写真をたくさん撮りました。	撮る	写真[しゃしん]をたくさん 撮[と]りました。	
\\	彼は再びここに戻ってきた。	再び	彼[かれ]は 再[ふたた]びここに 戻[もど]ってきた。	
\\	次のオリンピックは再来年です。	再来年	次[つぎ]のオリンピックは 再来年[さらいねん]です。	
\\	チョークで黒板に字を書きました。	チョーク	チョークで 黒板[こくばん]に 字[じ]を 書[か]きました。	
\\	再来月まで予約で一杯です。	再来月	再来月[さらいげつ]まで 予約[よやく]で 一杯[いっぱい]です。	
\\	再来週は忙しいです。	再来週	再来週[さらいしゅう]は 忙[いそが]しいです。	
\\	その番組は来週放送されます。	放送	その 番組[ばんぐみ]は 来週[らいしゅう] 放送[ほうそう]されます。	
\\	実験にはこの装置を使います。	装置	実験[じっけん]にはこの 装置[そうち]を 使[つか]います。	
\\	ティッシュを取ってください。	ティッシュ	ティッシュを 取[と]ってください。	
\\	お名前に仮名を振ってください。	仮名	お 名前[なまえ]に 仮名[かな]を 振[ふ]ってください。	
\\	その送り仮名は間違っています。	送り仮名	その 送[おく]り 仮名[がな]は 間違[まちが]っています。	
\\	彼女は鏡を見て髪を直した。	鏡	彼女[かのじょ]は 鏡[かがみ]を 見[み]て 髪[かみ]を 直[なお]した。	
\\	父は友だちの死を悲しんでいます。	悲しむ	父[ちち]は 友[とも]だちの 死[し]を 悲[かな]しんでいます。	
\\	入る時はドアをノックしてください。	ノック	入[はい]る 時[とき]はドアをノックしてください。	
\\	私の上司は頭が固い。	固い	私[わたし]の 上司[じょうし]は 頭[あたま]が 固[かた]い。	
\\	昨日、美術館に行きました。	美術館	昨日[きのう]、 美術館[びじゅつかん]に 行[い]きました。	
\\	彼のお母さんは美人です。	美人	彼[かれ]のお 母[かあ]さんは 美人[びじん]です。	
\\	父は美容院で髪を切ります。	美容院	父[ちち]は 美容院[びよういん]で 髪[かみ]を 切[き]ります。	
\\	よろしくお願いします。	よろしく	よろしくお 願[ねが]いします。	
\\	昨日、車の博物館に行った。	博物館	昨日[きのう]、 車[くるま]の 博物館[はくぶつかん]に 行[い]った。	
\\	京都で国際会議が開催された。	開催	京都[きょうと]で 国際会議[こくさいかいぎ]が 開催[かいさい]された。	
\\	この曲線がこの車の特徴です。	特徴	この 曲線[きょくせん]がこの 車[くるま]の 特徴[とくちょう]です。	
\\	私は彼を許しました。	許す	私[わたし]は 彼[かれ]を 許[ゆる]しました。	
\\	映画館はがらがらでした。	がらがら	映画館[えいがかん]はがらがらでした。	
\\	免許証を見せてください。	免許証	免許証[めんきょしょう]を 見[み]せてください。	
\\	彼は高校教師だ。	教師	彼[かれ]は 高校[こうこう] 教師[きょうし]だ。	
\\	彼は化学の教授です。	教授	彼[かれ]は 化学[かがく]の 教授[きょうじゅ]です。	
\\	みんなにこのことを伝えてください。	伝える	みんなにこのことを 伝[つた]えてください。	
\\	太陽がぎらぎらしている。	ぎらぎら	太陽[たいよう]がぎらぎらしている。	
\\	鳥が飛んでいます。	鳥	鳥[とり]が 飛[と]んでいます。	
\\	今朝、5時に電話が鳴った。	鳴る	今朝[けさ]、 5時[ごじ]に 電話[でんわ]が 鳴[な]った。	
\\	どこかでネコが鳴いている。	鳴く	どこかでネコが 鳴[な]いている。	
\\	彼は大きな声で話した。	声	彼[かれ]は 大[おお]きな 声[こえ]で 話[はな]した。	
\\	そっちが私の部屋です。	そっち	そっちが 私[わたし]の 部屋[へや]です。	
\\	ニワトリは卵を産みます。	卵	ニワトリは 卵[たまご]を 産[う]みます。	
\\	この犬はとても賢い。	犬	この 犬[いぬ]はとても 賢[かしこ]い。	
\\	彼は牧場で馬に乗った。	馬	彼[かれ]は 牧場[ぼくじょう]で 馬[うま]に 乗[の]った。	
\\	車は駐車場に止めてください。	駐車場	車[くるま]は 駐車場[ちゅうしゃじょう]に 止[と]めてください。	
\\	昨日の夕方、友達とピンポンをした。	ピンポン	昨日[きのう]の 夕方[ゆうがた]、 友達[ともだち]とピンポンをした。	
\\	電車の中で騒がないでください。	騒ぐ	電車[でんしゃ]の 中[なか]で 騒[さわ]がないでください。	
\\	虫に腕を刺されました。	刺す	虫[むし]に 腕[うで]を 刺[さ]されました。	
\\	私は刺身は食べません。	刺身	私[わたし]は 刺身[さしみ]は 食[た]べません。	
\\	激しい雨が降っています。	激しい	激[はげ]しい 雨[あめ]が 降[ふ]っています。	
\\	私はおなかがぺこぺこです。	ぺこぺこ	私[わたし]はおなかがぺこぺこです。	
\\	彼は血を見て驚いた。	驚く	彼[かれ]は 血[ち]を 見[み]て 驚[おどろ]いた。	
\\	強風で木が倒れた。	倒れる	強風[きょうふう]で 木[き]が 倒[たお]れた。	
\\	そのスキーヤーはポールを倒した。	倒す	そのスキーヤーはポールを 倒[たお]した。	
\\	最近の若者は難しい本を読まない傾向がある。	傾向	最近[さいきん]の 若者[わかもの]は 難[むずか]しい 本[ほん]を 読[よ]まない 傾向[けいこう]がある。	
\\	彼女はドイツ語がぺらぺらです。	ぺらぺら	彼女[かのじょ]はドイツ 語[ご]がぺらぺらです。	
\\	私は柔道を習っています。	柔道	私[わたし]は 柔道[じゅうどう]を 習[なら]っています。	
\\	布団がとても柔らかい。	柔らかい	布団[ふとん]がとても 柔[やわ]らかい。	
\\	彼の声は柔らかだ。	柔らか	彼[かれ]の 声[こえ]は 柔[やわ]らかだ。	
\\	上司は私の主張を受け入れた。	主張	上司[じょうし]は 私[わたし]の 主張[しゅちょう]を 受[う]け 入[い]れた。	
\\	雨なのでレインコートを着ました。	レインコート	雨[あめ]なのでレインコートを 着[き]ました。	
\\	娘が私の手を引っ張った。	引っ張る	娘[むすめ]が 私[わたし]の 手[て]を 引[ひ]っ 張[ぱ]った。	
\\	突き当たったら右に曲がってください。	突き当たる	突[つ]き 当[あ]たったら 右[みぎ]に 曲[ま]がってください。	
\\	この先は突き当たりです。	突き当たり	この 先[さき]は 突[つ]き 当[あ]たりです。	
\\	壁に絵が掛かっている。	壁	壁[かべ]に 絵[え]が 掛[か]かっている。	
\\	暑いのでアイスを食べました。	アイス	暑[あつ]いのでアイスを 食[た]べました。	
\\	彼はギターを弾きます。	弾く	彼[かれ]はギターを 弾[ひ]きます。	
\\	地球は丸い。	丸い	地球[ちきゅう]は 丸[まる]い。	
\\	紙に大きな丸を書きました。	丸	紙[かみ]に 大[おお]きな 丸[まる]を 書[か]きました。	
\\	2003年にアメリカはイラクを攻撃した。	攻撃	2003年[にせんさんねん]にアメリカはイラクを 攻撃[こうげき]した。	
\\	いつごろ夏休みを取りますか。	いつごろ	いつごろ 夏休[なつやす]みを 取[と]りますか。	
\\	絶対に駄目です。	絶対に	絶対[ぜったい]に 駄目[だめ]です。	
\\	消防車が4台も来た。	消防車	消防車[しょうぼうしゃ]が 4台[よんだい]も 来[き]た。	
\\	私は待つのが嫌だ。	嫌	私[わたし]は 待[ま]つのが 嫌[いや]だ。	
\\	私はテストが大嫌い。	大嫌い	私[わたし]はテストが 大嫌[だいきら]い。	
\\	このウィスキーは強いよ。	ウィスキー	このウィスキーは 強[つよ]いよ。	
\\	朝食は大抵7時頃に食べます。	大抵	朝食[ちょうしょく]は 大抵[たいてい] 7時頃[しちじごろ]に 食[た]べます。	
\\	大勢で食事に出かけました。	大勢	大勢[おおぜい]で 食事[しょくじ]に 出[で]かけました。	
\\	遠くに彼女の姿が見えた。	姿	遠[とお]くに 彼女[かのじょ]の 姿[すがた]が 見[み]えた。	
\\	あの子はいつも姿勢が悪い。	姿勢	あの 子[こ]はいつも 姿勢[しせい]が 悪[わる]い。	
\\	彼にエアメールを送りました。	エアメール	彼[かれ]にエアメールを 送[おく]りました。	
\\	昨夜恐ろしい夢を見た。	恐ろしい	昨夜[ゆうべ] 恐[おそ]ろしい 夢[ゆめ]を 見[み]た。	
\\	私は犬が怖いです。	怖い	私[わたし]は 犬[いぬ]が 怖[こわ]いです。	
\\	母は糸を針に通した。	糸	母[はは]は 糸[いと]を 針[はり]に 通[とお]した。	
\\	昨日、孫が生まれました。	孫	昨日[きのう]、 孫[まご]が 生[う]まれました。	
\\	私のシャープペンシルがありません。	シャープペンシル	私[わたし]のシャープペンシルがありません。	
\\	彼女は木綿のシャツを着ています。	木綿	彼女[かのじょ]は 木綿[もめん]のシャツを 着[き]ています。	
\\	彼は偉い学者です。	偉い	彼[かれ]は 偉[えら]い 学者[がくしゃ]です。	
\\	爪が伸びています。	爪	爪[つめ]が 伸[の]びています。	
\\	新しい机を買ってもらいました。	机	新[あたら]しい 机[つくえ]を 買[か]ってもらいました。	
\\	セロテープはありますか。	セロテープ	セロテープはありますか。	
\\	大きな棚はとても便利です。	棚	大[おお]きな 棚[たな]はとても 便利[べんり]です。	
\\	これはとても大きな本棚ですね。	本棚	これはとても 大[おお]きな 本棚[ほんだな]ですね。	
\\	今後の方針が決まった。	方針	今後[こんご]の 方針[ほうしん]が 決[き]まった。	
\\	昨日大きな魚を釣りました。	釣る	昨日大[きのう おお]きな 魚[さかな]を 釣[つ]りました。	
\\	彼は手にやけどをしました。	やけど	彼[かれ]は 手[て]にやけどをしました。	
\\	母はお釣りを確かめた。	お釣り	母[はは]はお 釣[つ]りを 確[たし]かめた。	
\\	出かける時は鍵を掛けてください。	鍵	出[で]かける 時[とき]は 鍵[かぎ]を 掛[か]けてください。	
\\	鍋にスープが入っているよ。	鍋	鍋[なべ]にスープが 入[はい]っているよ。	
\\	彼女は寿司を初めて食べました。	寿司	彼女[かのじょ]は 寿司[すし]を 初[はじ]めて 食[た]べました。	
\\	アイスコーヒーをください。	アイスコーヒー	アイスコーヒーをください。	
\\	近所に泥棒が入った。	泥棒	近所[きんじょ]に 泥棒[どろぼう]が 入[はい]った。	
\\	お風呂のお湯が一杯です。	湯	お 風呂[ふろ]のお 湯[ゆ]が 一杯[いっぱい]です。	
\\	お風呂が沸きました。	沸く	お 風呂[ふろ]が 沸[わ]きました。	
\\	お湯を沸かしてください。	沸かす	お 湯[ゆ]を 沸[わ]かしてください。	
\\	郵便局はあっちです。	あっち	郵便局[ゆうびんきょく]はあっちです。	
\\	新しい洗濯機を買いました。	洗濯機	新[あたら]しい 洗濯機[せんたっき]を 買[か]いました。	
\\	一週間、洗濯をしていない。	洗濯	一週間[いっしゅうかん]、 洗濯[せんたく]をしていない。	
\\	雨で濡れてしまった。	濡れる	雨[あめ]で 濡[ぬ]れてしまった。	
\\	喫茶店でコーヒーを飲んだ。	喫茶店	喫茶店[きっさてん]でコーヒーを 飲[の]んだ。	
\\	あんなにいい人はいません。	あんなに	あんなにいい 人[ひと]はいません。	
\\	怠けていないで、手伝って。	怠ける	怠[なま]けていないで、 手伝[てつだ]って。	
\\	彼は毎日一生懸命働いている。	一生懸命	彼[かれ]は 毎日[まいにち] 一生懸命[いっしょうけんめい] 働[はたら]いている。	
\\	少し休憩しましょう。	休憩	少[すこ]し 休憩[きゅうけい]しましょう。	
\\	この部屋は天井が高いですね。	天井	この 部屋[へや]は 天井[てんじょう]が 高[たか]いですね。	
\\	その子はいたずらが大好きだ。	いたずら	その 子[こ]はいたずらが 大好[だいす]きだ。	
\\	私は納豆をよく食べます。	納豆	私[わたし]は 納豆[なっとう]をよく 食[た]べます。	
\\	彼女は化粧が上手い。	化粧	彼女[かのじょ]は 化粧[けしょう]が 上手[うま]い。	
\\	母は畑仕事が好きだ。	畑	母[はは]は 畑[はたけ] 仕事[しごと]が 好[す]きだ。	
\\	母は毎朝ご飯を炊く。	炊く	母[はは]は 毎朝[まいあさ]ご 飯[はん]を 炊[た]く。	
\\	ウェートレスを呼んだ。	ウェートレス	ウェートレスを 呼[よ]んだ。	
\\	私は毎日、自炊している。	自炊	私[わたし]は 毎日[まいにち]、 自炊[じすい]している。	
\\	灰皿をください。	灰皿	灰皿[はいざら]をください。	
\\	火事ですべて灰になった。	灰	火事[かじ]ですべて 灰[はい]になった。	
\\	今日の空は灰色だ。	灰色	今日[きょう]の 空[そら]は 灰色[はいいろ]だ。	
\\	おばさん一家が遊びに来ました。	おばさん	おばさん 一家[いっか]が 遊[あそ]びに 来[き]ました。	
\\	西暦1964年に東京でオリンピックがあった。	西暦	西暦[せいれき] 1964年[せんきゅうひゃくろくじゅうよねん]に 東京[とうきょう]でオリンピックがあった。	
\\	猫の毛がセーターに付いた。	毛	猫[ねこ]の 毛[け]がセーターに 付[つ]いた。	
\\	昨日、髪を切りました。	髪	昨日[きのう]、 髪[かみ]を 切[き]りました。	
\\	舌を出してください。	舌	舌[した]を 出[だ]してください。	
\\	彼はくしで髪をとかした。	くし	彼[かれ]はくしで 髪[かみ]をとかした。	
\\	納豆は臭い。	臭い	納豆[なっとう]は 臭[くさ]い。	
\\	いい匂いがする。	匂い	いい 匂[にお]いがする。	
\\	今日は一日中家に居ました。	居る	今日[きょう]は 一日中家[いちにちじゅう うち]に 居[い]ました。	
\\	彼女はブーツを履いています。	履く	彼女[かのじょ]はブーツを 履[は]いています。	
\\	スーツをクリーニングに出しました。	クリーニング	スーツをクリーニングに 出[だ]しました。	
\\	部屋の戸が開いています。	戸	部屋[へや]の 戸[と]が 開[あ]いています。	
\\	暑いから扇風機をつけよう。	扇風機	暑[あつ]いから 扇風機[せんぷうき]をつけよう。	
\\	豚はすごく鼻が良い。	豚	豚[ぶた]はすごく 鼻[はな]が 良[い]い。	
\\	豚肉は美味しいです。	豚肉	豚肉[ぶたにく]は 美味[おい]しいです。	
\\	テーブルの上にごちそうが並んでいる。	ごちそう	テーブルの 上[うえ]にごちそうが 並[なら]んでいる。	
\\	その鶏は毎朝鳴きます。	鶏	その 鶏[にわとり]は 毎朝鳴[まいあさ な]きます。	
\\	私の腕時計は遅れている。	腕時計	私[わたし]の 腕時計[うでどけい]は 遅[おく]れている。	
\\	みんなにお菓子をあげましょう。	お菓子	みんなにお 菓子[かし]をあげましょう。	
\\	皆、社長にお辞儀をした。	お辞儀	皆[みんな]、 社長[しゃちょう]にお 辞儀[じぎ]をした。	
\\	今日の夕食は私がごちそうします。	ごちそうする	今日[きょう]の 夕食[ゆうしょく]は 私[わたし]がごちそうします。	
\\	今朝は寝坊しました。	寝坊	今朝[けさ]は 寝坊[ねぼう]しました。	
\\	旗が風に揺れている。	旗	旗[はた]が 風[かぜ]に 揺[ゆ]れている。	
\\	鉛筆を貸して下さい。	鉛筆	鉛筆[えんぴつ]を 貸[か]して 下[くだ]さい。	
\\	父に万年筆をプレゼントしました。	万年筆	父[ちち]に 万年筆[まんねんひつ]をプレゼントしました。	
\\	彼女の目から涙がこぼれた。	こぼれる	彼女[かのじょ]の 目[め]から 涙[なみだ]がこぼれた。	
\\	この箱は重い。	箱	この 箱[はこ]は 重[おも]い。	
\\	雑誌を本箱に入れました。	本箱	雑誌[ざっし]を 本箱[ほんばこ]に 入[い]れました。	
\\	寒いので手袋をしました。	手袋	寒[さむ]いので 手袋[てぶくろ]をしました。	
\\	袋はいりません。	袋	袋[ふくろ]はいりません。	
\\	こんなに面白い本は初めて読んだ。	こんなに	こんなに 面白[おもしろ]い 本[ほん]は 初[はじ]めて 読[よ]んだ。	
\\	会社に財布を忘れた。	財布	会社[かいしゃ]に 財布[さいふ]を 忘[わす]れた。	
\\	この毛布は暖かい。	毛布	この 毛布[もうふ]は 暖[あたた]かい。	
\\	母が布団を干している。	布団	母[はは]が 布団[ふとん]を 干[ほ]している。	
\\	フランスの友達から小包が届いた。	小包	フランスの 友達[ともだち]から 小包[こづつみ]が 届[とど]いた。	
\\	雨がざあざあ降っている。	ざあざあ	雨[あめ]がざあざあ 降[ふ]っている。	
\\	プレゼントをきれいな紙で包みました。	包む	プレゼントをきれいな 紙[かみ]で 包[つつ]みました。	
\\	テーブルの上に花を飾りました。	飾る	テーブルの 上[うえ]に 花[はな]を 飾[かざ]りました。	
\\	新しい手帳を買いました。	手帳	新[あたら]しい 手帳[てちょう]を 買[か]いました。	
\\	電話帳を見せてください。	電話帳	電話帳[でんわちょう]を 見[み]せてください。	
\\	ポテトサラダを作るから、ジャガイモを買ってきて。	ジャガイモ	ポテトサラダを 作[つく]るから、ジャガイモを 買[か]ってきて。	
\\	私は毎朝、牛乳を飲む。	牛乳	私[わたし]は 毎朝[まいあさ]、 牛乳[ぎゅうにゅう]を 飲[の]む。	
\\	玄関に花を飾りました。	玄関	玄関[げんかん]に 花[はな]を 飾[かざ]りました。	
\\	彼は眼鏡をかけています。	眼鏡	彼[かれ]は 眼鏡[めがね]をかけています。	
\\	今日はとても眠いです。	眠い	今日[きょう]はとても 眠[ねむ]いです。	
\\	このソファーは気持ちがいい。	ソファー	このソファーは 気持[きも]ちがいい。	
\\	昨日は8時間眠りました。	眠る	昨日[きのう]は 8時間[はちじかん] 眠[ねむ]りました。	
\\	その手紙を封筒に入れた。	封筒	その 手紙[てがみ]を 封筒[ふうとう]に 入[い]れた。	
\\	主人はもう出掛けました。	出掛ける	主人[しゅじん]はもう 出掛[でか]けました。	
\\	壁に大きな時計が掛かっています。	掛かる	壁[かべ]に 大[おお]きな 時計[とけい]が 掛[か]かっています。	
\\	彼はたばこを吸いません。	たばこ	彼[かれ]はたばこを 吸[す]いません。	
\\	弟は掛け算を習っている。	掛け算	弟[おとうと]は 掛[か]け 算[ざん]を 習[なら]っている。	
\\	大きな拍手が上がった。	拍手	大[おお]きな 拍手[はくしゅ]が 上[あ]がった。	
\\	週末は部屋の掃除をしました。	掃除	週末[しゅうまつ]は 部屋[へや]の 掃除[そうじ]をしました。	
\\	床をほうきで掃きました。	掃く	床[ゆか]をほうきで 掃[は]きました。	
\\	値段はちょうど3万円です。	ちょうど	値段[ねだん]はちょうど 3万円[さんまんえん]です。	
\\	掃除機が壊れた。	掃除機	掃除機[そうじき]が 壊[こわ]れた。	
\\	少女は母親の手を握った。	握る	少女[しょうじょ]は 母親[ははおや]の 手[て]を 握[にぎ]った。	
\\	彼らは握手をした。	握手	彼[かれ]らは 握手[あくしゅ]をした。	
\\	姉が空港まで迎えに来てくれます。	迎える	姉[あね]が 空港[くうこう]まで 迎[むか]えに 来[き]てくれます。	
\\	もうちょっと塩を入れて。	ちょっと	もうちょっと 塩[しお]を 入[い]れて。	
\\	あのお巡りさんに道を聞きましょう。	お巡りさん	あのお 巡[まわ]りさんに 道[みち]を 聞[き]きましょう。	
\\	娘さんは幾つになりましたか。	幾つ	娘[むすめ]さんは 幾[いく]つになりましたか。	
\\	この靴は幾らですか。	幾ら	この 靴[くつ]は 幾[いく]らですか。	
\\	冗談は止めてください。	冗談	冗談[じょうだん]は 止[や]めてください。	
\\	部屋の電気がついています。	つく	部屋[へや]の 電気[でんき]がついています。	
\\	靴下に穴が開いている。	穴	靴下[くつした]に 穴[あな]が 開[あ]いている。	
\\	これは寂しい曲ですね。	寂しい	これは 寂[さび]しい 曲[きょく]ですね。	
\\	彼女はいつも丁寧に仕事をします。	丁寧	彼女[かのじょ]はいつも 丁寧[ていねい]に 仕事[しごと]をします。	
\\	その可哀相な子供たちは食べるものがない。	可哀相	その 可哀相[かわいそう]な 子供[こども]たちは 食[た]べるものがない。	
\\	私はえびのてんぷらが好きです。	てんぷら	私[わたし]はえびのてんぷらが 好[す]きです。	
\\	彼女は腕を怪我した。	怪我	彼女[かのじょ]は 腕[うで]を 怪我[けが]した。	
\\	彼のわがままには我慢できません。	我慢	彼[かれ]のわがままには 我慢[がまん]できません。	
\\	娘は幼稚園に通っています。	幼稚園	娘[むすめ]は 幼稚園[ようちえん]に 通[かよ]っています。	
\\	太陽が雲に隠れた。	隠れる	太陽[たいよう]が 雲[くも]に 隠[かく]れた。	
\\	母の肩をとんとんたたいた。	とんとん	母[はは]の 肩[かた]をとんとんたたいた。	
\\	暑いので日陰で休みましょう。	日陰	暑[あつ]いので 日陰[ひかげ]で 休[やす]みましょう。	
\\	随分早く終わったね。	随分	随分[ずいぶん] 早[はや]く 終[お]わったね。	
\\	明日の試験、頑張ってね。	頑張る	明日[あした]の 試験[しけん]、 頑張[がんば]ってね。	
\\	子供の頃、よくその公園で遊んだ。	頃	子供[こども]の 頃[ころ]、よくその 公園[こうえん]で 遊[あそ]んだ。	
\\	私はにんじんが嫌いだ。	にんじん	私[わたし]はにんじんが 嫌[きら]いだ。	
\\	この頃、よく肩が凝る。	この頃	この 頃[ごろ]、よく 肩[かた]が 凝[こ]る。	
\\	この町には消防署が1つしかありません。	消防署	この 町[まち]には 消防署[しょうぼうしょ]が 1[ひと]つしかありません。	
\\	彼女は道を尋ねた。	尋ねる	彼女[かのじょ]は 道[みち]を 尋[たず]ねた。	
\\	缶詰のフルーツはとても甘い。	缶詰	缶詰[かんづめ]のフルーツはとても 甘[あま]い。	
\\	風呂がぬるかった。	ぬるい	風呂[ふろ]がぬるかった。	
\\	コーヒーはその缶に入っています。	缶	コーヒーはその 缶[かん]に 入[はい]っています。	
\\	リンゴが腐っている。	腐る	リンゴが 腐[くさ]っている。	
\\	私は毎日豆腐を食べます。	豆腐	私[わたし]は 毎日[まいにち] 豆腐[とうふ]を 食[た]べます。	
\\	床がぬれている。	床	床[ゆか]がぬれている。	
\\	この村の80パーセントは老人です。	パーセント	この 村[むら]の 80[はちじゅっ]パーセントは 老人[ろうじん]です。	
\\	昨日床屋で髪を切った。	床屋	昨日[きのう] 床屋[とこや]で 髪[かみ]を 切[き]った。	
\\	そのホテルには畳の部屋がある。	畳	そのホテルには 畳[たたみ]の 部屋[へや]がある。	
\\	布団を畳んでください。	畳む	布団[ふとん]を 畳[たた]んでください。	
\\	母は洗濯物を干しています。	干す	母[はは]は 洗濯物[せんたくもの]を 干[ほ]しています。	
\\	スキーでひざを怪我しました。	ひざ	スキーでひざを 怪我[けが]しました。	
\\	暑いので帽子を被りましょう。	帽子	暑[あつ]いので 帽子[ぼうし]を 被[かぶ]りましょう。	
\\	是非、うちに来てください。	是非	是非[ぜひ]、うちに 来[き]てください。	
\\	お客様には敬語を使いなさい。	敬語	お 客様[きゃくさま]には 敬語[けいご]を 使[つか]いなさい。	
\\	祖父は家族みんなに尊敬されています。	尊敬	祖父[そふ]は 家族[かぞく]みんなに 尊敬[そんけい]されています。	
\\	彼はひじに怪我をした。	ひじ	彼[かれ]はひじに 怪我[けが]をした。	
\\	生まれて初めて布団を敷いた。	敷く	生[う]まれて 初[はじ]めて 布団[ふとん]を 敷[し]いた。	
\\	雷が鳴っています。	雷	雷[かみなり]が 鳴[な]っています。	
\\	今ちょうど零時です。	零	今[いま]ちょうど 零[れい] 時[じ]です。	
\\	彼女は大切な書類を机に仕舞った。	仕舞う	彼女[かのじょ]は 大切[たいせつ]な 書類[しょるい]を 机[つくえ]に 仕舞[しま]った。	
\\	靴のひもが切れた。	ひも	靴[くつ]のひもが 切[き]れた。	
\\	彼女はクラブで踊るのが好きです。	踊る	彼女[かのじょ]はクラブで 踊[おど]るのが 好[す]きです。	
\\	この踊りは易しいですよ。	踊り	この 踊[おど]りは 易[やさ]しいですよ。	
\\	運転手がブレーキを踏んだ。	踏む	運転手[うんてんしゅ]がブレーキを 踏[ふ]んだ。	
\\	その踏切は長い。	踏切	その 踏切[ふみきり]は 長[なが]い。	
\\	日本では、二月に豆をまく行事がある。	まく	日本[にほん]では、 二月[にがつ]に 豆[まめ]をまく 行事[ぎょうじ]がある。	
\\	ゴールキーパーがボールを蹴った。	蹴る	ゴールキーパーがボールを 蹴[け]った。	
\\	大学の食堂は安い。	食堂	大学[だいがく]の 食堂[しょくどう]は 安[やす]い。	
\\	私は猫が大好きです。	猫	私[わたし]は 猫[ねこ]が 大好[だいす]きです。	
\\	彼は文章がとてもうまい。	文章	彼[かれ]は 文章[ぶんしょう]がとてもうまい。	
\\	やかんでお湯を沸かしました。	やかん	やかんでお 湯[ゆ]を 沸[わ]かしました。	
\\	祖母は身体が丈夫だ。	丈夫	祖母[そぼ]は 身体[からだ]が 丈夫[じょうぶ]だ。	
\\	明日、また来ます。	又	明日[あした]、また 来[き]ます。	
\\	私のお祖父さんは毎日散歩します。	お祖父さん	私[わたし]のお 祖父[じい]さんは 毎日散歩[まいにち さんぽ]します。	
\\	彼女はお祖母さんと住んでいる。	お祖母さん	彼女[かのじょ]はお 祖母[ばあ]さんと 住[す]んでいる。	
\\	日本へようこそ。	ようこそ	日本[にっぽん]へようこそ。	
\\	祖父は元気です。	祖父	祖父[そふ]は 元気[げんき]です。	
\\	祖母は京都で生まれました。	祖母	祖母[そぼ]は 京都[きょうと]で 生[う]まれました。	
\\	邪魔です、どいてください。	邪魔	邪魔[じゃま]です、どいてください。	
\\	この風邪薬を飲みなさい。	風邪薬	この 風邪薬[かぜぐすり]を 飲[の]みなさい。	
\\	ウェーターが水を運んできました。	ウェーター	ウェーターが 水[みず]を 運[はこ]んできました。	
\\	昨日、友達とお風呂屋さんに行った。	風呂屋	昨日[きのう]、 友達[ともだち]とお 風呂屋[ふろや]さんに 行[い]った。	
\\	父は今お風呂に入っています。	風呂	父[ちち]は 今[いま]お 風呂[ふろ]に 入[はい]っています。	
\\	私の両親は昭和生まれです。	昭和	私[わたし]の 両親[りょうしん]は 昭和[しょうわ] 生[う]まれです。	
\\	彼女は紫のドレスを着ていた。	紫	彼女[かのじょ]は 紫[むらさき]のドレスを 着[き]ていた。	
\\	一度にそんなにたくさんはできない。	そんなに	一度[いちど]にそんなにたくさんはできない。	
\\	温かい紅茶が飲みたい。	紅茶	温[あたた]かい 紅茶[こうちゃ]が 飲[の]みたい。	
\\	梅雨は6月頃です。	梅雨	梅雨[つゆ]は 6月頃[ろくがつごろ]です。	
\\	私の一番好きな果物は桃です。	桃	私[わたし]の 一番好[いちばん す]きな 果物[くだもの]は 桃[もも]です。	
\\	彼は交通事故に遭った。	遭う	彼[かれ]は 交通事故[こうつう じこ]に 遭[あ]った。	
\\	私は低い枕が好きです。	枕	私[わたし]は 低[ひく]い 枕[まくら]が 好[す]きです。	
\\	嘘をついてはいけません。	嘘	嘘[うそ]をついてはいけません。	
\\	私は遠慮します。	遠慮	私[わたし]は 遠慮[えんりょ]します。	
\\	父親が子供を叱っている。	叱る	父親[ちちおや]が 子供[こども]を 叱[しか]っている。	
\\	電車に傘を忘れた。	傘	電車[でんしゃ]に 傘[かさ]を 忘[わす]れた。	
\\	お嬢さんはおいくつですか。	お嬢さん	お 嬢[じょう]さんはおいくつですか。	
\\	昨日、年賀状を出しました。	年賀状	昨日[きのう]、 年賀状[ねんがじょう]を 出[だ]しました。	
\\	浅草は賑やかな街です。	賑やか	浅草[あさくさ]は 賑[にぎ]やかな 街[まち]です。	
\\	蚊に足を刺された。	蚊	蚊[か]に 足[あし]を 刺[さ]された。	
\\	タオルで体を拭きました。	拭く	タオルで 体[からだ]を 拭[ふ]きました。	
\\	彼女は笑顔で挨拶した。	挨拶	彼女[かのじょ]は 笑顔[えがお]で 挨拶[あいさつ]した。	
\\	その仕事は危険を伴う。	伴う	その 仕事[しごと]は 危険[きけん]を 伴[ともな]う。	
\\	彼は頭にタオルを巻いていた。	巻く	彼[かれ]は 頭[あたま]にタオルを 巻[ま]いていた。	
\\	靴が汚れた。	靴	靴[くつ]が 汚[よご]れた。	
\\	この靴下は3足で1000円です。	靴下	この 靴下[くつした]は 3足[さんそく]で 1000円[せんえん]です。	
\\	靴を磨いてください。	磨く	靴[くつ]を 磨[みが]いてください。	
\\	歯磨きはしましたか。	歯磨き	歯磨[はみが]きはしましたか。	
\\	廊下は走らないでください。	廊下	廊下[ろうか]は 走[はし]らないでください。	
\\	瓶ビールを注文した。	瓶	瓶[びん]ビールを 注文[ちゅうもん]した。	
\\	頑張ったので褒められました。	褒める	頑張[がんば]ったので 褒[ほ]められました。	
\\	元旦に彼から年賀状が来た。	元旦	元旦[がんたん]に 彼[かれ]から 年賀状[ねんがじょう]が 来[き]た。	
\\	このシャツは袖が短い。	袖	このシャツは 袖[そで]が 短[みじか]い。	
\\	今日は寒いので長袖を着ました。	長袖	今日[きょう]は 寒[さむ]いので 長袖[ながそで]を 着[き]ました。	
\\	今日は、半袖のシャツを着よう。	半袖	今日[きょう]は、 半袖[はんそで]のシャツを 着[き]よう。	
\\	私は馬鹿だった。	馬鹿	私[わたし]は 馬鹿[ばか]だった。	
\\	凄い雨になった。	凄い	凄[すご]い 雨[あめ]になった。	
\\	父は毎日髭を剃ります。	剃る	父[ちち]は 毎日髭[まいにち ひげ]を 剃[そ]ります。	
\\	喧嘩はやめて。	喧嘩	喧嘩[けんか]はやめて。	
\\	彼は子供のおしりを叩いた。	叩く	彼[かれ]は 子供[こども]のおしりを 叩[たた]いた。	
\\	もっとよく噛みなさい。	噛む	もっとよく 噛[か]みなさい。	
\\	私は毎日味噌汁を飲みます。	味噌汁	私[わたし]は 毎日[まいにち] 味噌汁[みそしる]を 飲[の]みます。	
\\	私の姪は3才です。	姪	私[わたし]の 姪[めい]は 3才[さんさい]です。	
\\	この苺はとても甘い。	苺	この 苺[いちご]はとても 甘[あま]い。	
\\	夕飯に茄子の天ぷらを食べました。	茄子	夕飯[ゆうはん]に 茄子[なす]の 天[てん]ぷらを 食[た]べました。	
\\	ついに素晴らしい女性に逢えた。	逢う	ついに 素晴[すば]らしい 女性[じょせい]に 逢[あ]えた。	
\\	そのお年寄りは椅子に座った。	椅子	そのお 年寄[としよ]りは 椅子[いす]に 座[すわ]った。	
\\	私は少し痩せました。	痩せる	私[わたし]は 少[すこ]し 痩[や]せました。	
\\	箸を上手に使えるよ。	箸	箸[はし]を 上手[じょうず]に 使[つか]えるよ。	
\\	糊で2枚の紙を貼り合わせた。	糊	糊[のり]で 2枚[にまい]の 紙[かみ]を 貼[は]り 合[あ]わせた。	
\\	もう少し醤油を足してください。	醤油	もう 少[すこ]し 醤油[しょうゆ]を 足[た]してください。	
\\	この鋏はよく切れる。	鋏	この 鋏[はさみ]はよく 切[き]れる。	
\\	その黒い鞄は僕のです。	鞄	その 黒[くろ]い 鞄[かばん]は 僕[ぼく]のです。	
\\	顎が痛い。	顎	顎[あご]が 痛[いた]い。	
\\	缶に飴が入っています。	飴	缶[かん]に 飴[あめ]が 入[はい]っています。	
\\	石鹸で手を洗ってください。	石鹸	石鹸[せっけん]で 手[て]を 洗[あら]ってください。	
\\	お店は一時休業になったんだ。	一時	お 店[みせ]は 一時[いちじ] 休業[きゅうぎょう]になったんだ。	
\\	ここに生年月日を記入してください。	月日	ここに 生年[せいねん] 月日[がっぴ]を 記入[きにゅう]してください。	
\\	金もうけが彼の趣味だ。	金もうけ	金[かね]もうけが 彼[かれ]の 趣味[しゅみ]だ。	
\\	では、私たちは先に出発します。	先に	では、 私[わたし]たちは 先[さき]に 出発[しゅっぱつ]します。	
\\	先日の件はどうなりましたか。	先日	先日[せんじつ]の 件[けん]はどうなりましたか。	
\\	今日の日本の若者は欧米化している。	今日	今日[こんにち]の 日本[にほん]の 若者[わかもの]は 欧米化[おうべいか]している。	
\\	今ごろそんなこと言わないで。	今ごろ	今[いま]ごろそんなこと 言[い]わないで。	
\\	このケースには
\\	が50枚入ります。	ケース	このケースには
\\	が50 枚入[まい はい]ります。	
\\	今にも雨が降りそうですね。	今にも	今[いま]にも 雨[あめ]が 降[ふ]りそうですね。	
\\	あなたも今に分かるでしょう。	今に	あなたも 今[いま]に 分[わ]かるでしょう。	
\\	最近彼のところに行き来してないの。	行き来	最近彼[さいきん かれ]のところに 行[い]き 来[き]してないの。	
\\	5行以内で答えを書いてください。	行	5[ご] 行[ぎょう] 以内[いない]で 答[こた]えを 書[か]いてください。	
\\	学校は生徒たちを午前中に帰したね。	帰す	学校[がっこう]は 生徒[せいと]たちを 午前中[ごぜんちゅう]に 帰[かえ]したね。	
\\	大いに学び、大いに遊びなさい。	大いに	大[おお]いに 学[まな]び、 大[おお]いに 遊[あそ]びなさい。	
\\	彼の話は大げさだ。	大げさ	彼[かれ]の 話[はなし]は 大[おお]げさだ。	
\\	彼には清潔なイメージがあるね。	イメージ	彼[かれ]には 清潔[せいけつ]なイメージがあるね。	
\\	その年、この地域では大水がありました。	大水	その 年[とし]、この 地域[ちいき]では 大水[おおみず]がありました。	
\\	このカメラなら水中の写真が撮れますね。	水中	このカメラなら 水中[すいちゅう]の 写真[しゃしん]が 撮[と]れますね。	
\\	この箱の小をください。	小	この 箱[はこ]の 小[しょう]をください。	
\\	この仕事には少なくとも2週間必要でしょう。	少なくとも	この 仕事[しごと]には 少[すく]なくとも 2週間必要[にしゅうかん ひつよう]でしょう。	
\\	少々のことは我慢します。	少々	少々[しょうしょう]のことは 我慢[がまん]します。	
\\	今日は仕事が早く上がったんだ。	上がる	今日[きょう]は 仕事[しごと]が 早[はや]く 上[あ]がったんだ。	
\\	冷めないうちにどうぞお上がり下さい。	上がる	冷[さ]めないうちにどうぞお 上[あ]がり 下[くだ]さい。	
\\	自分の失敗を他人のせいにするな。	せい	自分[じぶん]の 失敗[しっぱい]を 他人[たにん]のせいにするな。	
\\	握り寿司の上を注文したよ。	上	握[にぎ]り 寿司[ずし]の 上[じょう]を 注文[ちゅうもん]したよ。	
\\	お上は何を考えてるんだろうね。	上	お 上[かみ]は 何[なに]を 考[かんが]えてるんだろうね。	
\\	先生が手紙を下さいました。	下さる	先生[せんせい]が 手紙[てがみ]を 下[くだ]さいました。	
\\	その道は今、下水の工事をしているよ。	下水	その 道[みち]は 今[いま]、 下水[げすい]の 工事[こうじ]をしているよ。	
\\	この服は上下セットで買いました。	上下	この 服[ふく]は 上下[じょうげ]セットで 買[か]いました。	
\\	彼は上下おそろいの服を着ていますね。	上下	彼[かれ]は 上下[うえした]おそろいの 服[ふく]を 着[き]ていますね。	
\\	彼女の成績は上の下です。	下	彼女[かのじょ]の 成績[せいせき]は 上[じょう]の 下[げ]です。	
\\	左右を見てから横断歩道を渡りなさい。	左右	左右[さゆう]を 見[み]てから 横断歩道[おうだん ほどう]を 渡[わた]りなさい。	
\\	彼女の様子がいつもと違う。	いつも	彼女[かのじょ]の 様子[ようす]がいつもと 違[ちが]う。	
\\	火が四方に広がったんだよ。	四方	火[ひ]が 四方[しほう]に 広[ひろ]がったんだよ。	
\\	大勢の方々にご出席いただきました。	方々	大勢[おおぜい]の 方々[かたがた]にご 出席[しゅっせき]いただきました。	
\\	あなたもだいぶ大人になったね。	大人	あなたもだいぶ 大人[おとな]になったね。	
\\	午後は外出の予定です。	外出	午後[ごご]は 外出[がいしゅつ]の 予定[よてい]です。	
\\	心の内をお話し下さい。	内	心[こころ]の 内[うち]をお 話[はな]し 下[くだ]さい。	
\\	うちの会社は週休2日です。	週休	うちの 会社[かいしゃ]は 週休[しゅうきゅう] 2日[ふつか]です。	
\\	一体何事ですか。	一体	一体[いったい] 何事[なにごと]ですか。	
\\	大学に通うのが上京のきっかけでした。	きっかけ	大学[だいがく]に 通[かよ]うのが 上京[じょうきょう]のきっかけでした。	
\\	このプリンターで写真を出力できます。	出力	このプリンターで 写真[しゃしん]を 出力[しゅつりょく]できます。	
\\	火力を弱めないと肉がこげるよ。	火力	火力[かりょく]を 弱[よわ]めないと 肉[にく]がこげるよ。	
\\	この島は発電を水力に頼っているんだ。	水力	この 島[しま]は 発電[はつでん]を 水力[すいりょく]に 頼[たよ]っているんだ。	
\\	きのう、弟と口げんかしました。	口げんか	きのう、 弟[おとうと]と 口[くち]げんかしました。	
\\	彼は大手のメーカーに勤めています。	大手	彼[かれ]は 大手[おおて]のメーカーに 勤[つと]めています。	
\\	暗いので足下に気を付けてください。	足下	暗[くら]いので 足下[あしもと]に 気[き]を 付[つ]けてください。	
\\	島で火山が噴火したよ。	火山	島[しま]で 火山[かざん]が 噴火[ふんか]したよ。	
\\	買い物をすっかり忘れていた。	すっかり	買[か]い 物[もの]をすっかり 忘[わす]れていた。	
\\	小川がさらさら流れています。	小川	小川[おがわ]がさらさら 流[なが]れています。	
\\	蝶々が空中を舞っています。	空中	蝶々[ちょうちょう]が 空中[くうちゅう]を 舞[ま]っています。	
\\	彼女はお年寄りのために席を空けたんだ。	空ける	彼女[かのじょ]はお 年寄[としよ]りのために 席[せき]を 空[あ]けたんだ。	
\\	僕の財布は空っぽだよ。	空っぽ	僕[ぼく]の 財布[さいふ]は 空[から]っぽだよ。	
\\	部屋の空きはありますか。	空き	部屋[へや]の 空[あ]きはありますか。	
\\	この瓶はもう空ね。	空	この 瓶[びん]はもう 空[から]ね。	
\\	海水から塩を作ります。	海水	海水[かいすい]から 塩[しお]を 作[つく]ります。	
\\	海上で衝突事故が発生した。	海上	海上[かいじょう]で 衝突事故[しょうとつ じこ]が 発生[はっせい]した。	
\\	このシリーズは50話もあるそうよ。	シリーズ	このシリーズは 50話[ごじゅうわ]もあるそうよ。	
\\	窓の外に水田が広がっていたよ。	水田	窓[まど]の 外[そと]に 水田[すいでん]が 広[ひろ]がっていたよ。	
\\	世界中で森林が失われています。	森林	世界中[せかいじゅう]で 森林[しんりん]が 失[うしな]われています。	
\\	彼は男らしさを心掛けているね。	男らしい	彼[かれ]は 男[おとこ]らしさを 心掛[こころが]けているね。	
\\	彼女は女らしい。	女らしい	彼女[かのじょ]は 女[おんな]らしい。	
\\	少女は母親の手を握った。	少女	少女[しょうじょ]は 母親[ははおや]の 手[て]を握[にぎ]った。	
\\	このクラスの女子は18人です。	女子	このクラスの 女子[じょし]は 18人[じゅうはちにん]です。	
\\	彼女は背の高い男性を好みますね。	好む	彼女[かのじょ]は 背[せ]の 高[たか]い 男性[だんせい]を 好[この]みますね。	
\\	その国ではエイズが大きな社会問題ね。	エイズ	その 国[くに]ではエイズが 大[おお]きな 社会問題[しゃかい もんだい]ね。	
\\	姉と私は服の好みが似ています。	好み	姉[あね]と 私[わたし]は 服[ふく]の 好[この]みが 似[に]ています。	
\\	短い時間にずいぶん上達しましたね。	上達	短[みじか]い 時間[じかん]にずいぶん 上達[じょうたつ]しましたね。	
\\	あの一家は仲がいい。	一家	あの 一家[いっか]は 仲[なか]がいい。	
\\	彼の息子が家出したそうよ。	家出	彼[かれ]の 息子[むすこ]が 家出[いえで]したそうよ。	
\\	ここの大家は近くに住んでいますよ。	大家	ここの 大家[おおや]は 近[ちか]くに 住[す]んでいますよ。	
\\	彼は元来、真面目な人です。	元来	彼[かれ]は 元来[がんらい]、 真面目[まじめ]な 人[ひと]です。	
\\	元日には初詣でに行きます。	元日	元日[がんじつ]には 初詣[はつもう]でに 行[い]きます。	
\\	車のエンジンが調子悪い。	エンジン	車[くるま]のエンジンが 調子悪[ちょうし わる]い。	
\\	お中元にビールが届きました。	お中元	お 中元[ちゅうげん]にビールが 届[とど]きました。	
\\	今夜は天の川が見えますね。	天の川	今夜[こんや]は 天[あま]の 川[がわ]が 見[み]えますね。	
\\	今日は最高にいい気分だよ。	気分	今日[きょう]は 最高[さいこう]にいい 気分[きぶん]だよ。	
\\	彼女は最近元気がないですね。	元気	彼女[かのじょ]は 最近[さいきん] 元気[げんき]がないですね。	
\\	水が沸騰して気体になったんだ。	気体	水[みず]が 沸騰[ふっとう]して 気体[きたい]になったんだ。	
\\	彼は気力にあふれていますね。	気力	彼[かれ]は 気力[きりょく]にあふれていますね。	
\\	各地で大雨が降っています。	大雨	各地[かくち]で 大雨[おおあめ]が 降[ふ]っています。	
\\	彼はビジネスクラスに乗ったの。	クラス	彼[かれ]はビジネスクラスに 乗[の]ったの。	
\\	小雨なので傘はいりません。	小雨	小雨[こさめ]なので 傘[かさ]はいりません。	
\\	試合は雨天中止です。	雨天	試合[しあい]は 雨天[うてん] 中止[ちゅうし]です。	
\\	10年振りの大雪です。	大雪	10年振[じゅうねん ぶ]りの 大雪[おおゆき]です。	
\\	町の青年たちはボランティア活動をしています。	青年	町[まち]の 青年[せいねん]たちはボランティア 活動[かつどう]をしています。	
\\	雲一つない青空ですね。	青空	雲一[くも ひと]つない 青空[あおぞら]ですね。	
\\	明日のプレゼンテーションが心配だ。	明日	明日[あす]のプレゼンテーションが 心配[しんぱい]だ。	
\\	もうすぐ夜が明けるね。	明ける	もうすぐ 夜[よ]が 明[あ]けるね。	
\\	部屋の明かりを点けましょう。	明かり	部屋[へや]の 明[あ]かりを 点[つ]けましょう。	
\\	カーブを曲がるときはスピードを落としましょう。	スピード	カーブを 曲[ま]がるときはスピードを 落[お]としましょう。	
\\	明け方に雨が降り始めましたね。	明け方	明[あ]け 方[がた]に 雨[あめ]が 降[ふ]り 始[はじ]めましたね。	
\\	大学卒業は一昨年です。	一昨年	大学卒業[だいがく そつぎょう]は 一昨年[いっさくねん]です。	
\\	一昨日、お電話を差し上げました。	一昨日	一昨日[いっさくじつ]、お 電話[でんわ]を 差[さ]し 上[あ]げました。	
\\	昨日は雨でしたね。	昨日	昨日[さくじつ]は 雨[あめ]でしたね。	
\\	全員で技術の向上に努めています。	向上	全員[ぜんいん]で 技術[ぎじゅつ]の 向上[こうじょう]に 努[つと]めています。	
\\	彼は一向に興味を示さないの。	一向に	彼[かれ]は 一向[いっこう]に 興味[きょうみ]を 示[しめ]さないの。	
\\	このブラウスは後ろ開きです。	開き	このブラウスは 後[うし]ろ 開[あ]きです。	
\\	彼に会えば、彼がいかに良い人か分かります。	いかに	彼[かれ]に 会[あ]えば、 彼[かれ]がいかに 良[い]い 人[ひと]か 分[わ]かります。	
\\	子供に昔話を聞かせてあげたの。	聞かす	子供[こども]に 昔話[むかしばなし]を 聞[き]かせてあげたの。	
\\	彼女はいつも聞き手にまわるね。	聞き手	彼女[かのじょ]はいつも 聞[き]き 手[て]にまわるね。	
\\	お客さんを客間にお通ししたわよ。	客間	お 客[きゃく]さんを 客間[きゃくま]にお 通[とお]ししたわよ。	
\\	その間に彼は居なくなっていました。	間	その 間[かん]に 彼[かれ]は 居[い]なくなっていました。	
\\	狭い空間に物がたくさん置いてあるね。	空間	狭[せま]い 空間[くうかん]に 物[もの]がたくさん 置[お]いてあるね。	
\\	円高の影響で海外製品が安く買えますよ。	円高	円高[えんだか]の 影響[えいきょう]で 海外製品[かいがい せいひん]が 安[やす]く 買[か]えますよ。	
\\	これまでで最高の結果が出たよ。	最高	これまでで 最高[さいこう]の 結果[けっか]が 出[で]たよ。	
\\	マラソンは雨の中でスタートしたの。	スタート	マラソンは 雨[あめ]の 中[なか]でスタートしたの。	
\\	これは今までで最低の記録だ。	最低	これは 今[いま]までで 最低[さいてい]の 記録[きろく]だ。	
\\	これは世界で最小のコンピューターです。	最小	これは 世界[せかい]で 最小[さいしょう]のコンピューターです。	
\\	このホテルでは最上のサービスが受けられます。	最上	このホテルでは 最上[さいじょう]のサービスが 受[う]けられます。	
\\	夕食の最中に電話がかかってきたの。	最中	夕食[ゆうしょく]の 最中[さいちゅう]に 電話[でんわ]がかかってきたの。	
\\	会議の初日に市長がスピーチをしたよ。	初日	会議[かいぎ]の 初日[しょにち]に 市長[しちょう]がスピーチをしたよ。	
\\	お前の言うことは信じられない。	お前	お 前[まえ]の 言[い]うことは 信[しん]じられない。	
\\	そちらに着くのは6時前後です。	前後	そちらに 着[つ]くのは 6時[ろくじ] 前後[ぜんご]です。	
\\	彼は後方の座席に着いたの。	後方	彼[かれ]は 後方[こうほう]の 座席[ざせき]に 着[つ]いたの。	
\\	このメニューはサラダと飲み物がセットになっています。	セット	このメニューはサラダと 飲[の]み 物[もの]がセットになっています。	
\\	彼は車を後ろ向きに駐車したの。	後ろ向き	彼[かれ]は 車[くるま]を 後[うし]ろ 向[む]きに 駐車[ちゅうしゃ]したの。	
\\	会議は明々後日に延期された。	明々後日	会議[かいぎ]は 明々後日[しあさって]に 延期[えんき]された。	
\\	ドラマの後半は来週放送されるんだ。	後半	ドラマの 後半[こうはん]は 来週放送[らいしゅう ほうそう]されるんだ。	
\\	朝日が昇りましたよ。	朝日	朝日[あさひ]が 昇[のぼ]りましたよ。	
\\	昨晩の雪がまだ庭に残っている。	昨晩	昨晩[さくばん]の 雪[ゆき]がまだ 庭[にわ]に 残[のこ]っている。	
\\	昨夜はテレビで喜劇を見たよ。	昨夜	昨夜[さくや]はテレビで 喜劇[きげき]を 見[み]たよ。	
\\	腹一杯食った。	食う	腹一杯[はらいっぱい] 食[く]った。	
\\	写真をアップで撮ろう。	アップ	写真[しゃしん]をアップで 撮[と]ろう。	
\\	たまには外食しましょう。	外食	たまには 外食[がいしょく]しましょう。	
\\	朝食に食パンを2枚食べました。	食パン	朝食[ちょうしょく]に 食[しょく]パンを 2枚食[にまい た]べました。	
\\	彼は一見サラリーマン風ですね。	一見	彼[かれ]は 一見[いっけん]サラリーマン 風[ふう]ですね。	
\\	突然何を言い出すのかと思った。	言い出す	突然何[とつぜん なに]を 言[い]い 出[だ]すのかと 思[おも]った。	
\\	外来語は一般にカタカナで書かれます。	外来語	外来語[がいらいご]は 一般[いっぱん]にカタカナで 書[か]かれます。	
\\	彼女は英文の手紙を書きました。	英文	彼女[かのじょ]は 英文[えいぶん]の 手紙[てがみ]を 書[か]きました。	
\\	漢語はもともと外来語です。	漢語	漢語[かんご]はもともと 外来語[がいらいご]です。	
\\	若者たちはエネルギーにあふれていますね。	エネルギー	若者[わかもの]たちはエネルギーにあふれていますね。	
\\	今、論文の下書きをしているところです。	下書き	今[いま]、 論文[ろんぶん]の 下書[したが]きをしているところです。	
\\	後書きをよく読んで下さい。	後書き	後書[あとが]きをよく 読[よ]んで 下[くだ]さい。	
\\	この子は覚えが早いね。	覚え	この 子[こ]は 覚[おぼ]えが 早[はや]いね。	
\\	運動会は9時に開会します。	開会	運動会[うんどうかい]は 9時[くじ]に 開会[かいかい]します。	
\\	会は午後9時に終わりました。	会	会[かい]は 午後9時[ごご くじ]に 終[お]わりました。	
\\	日時はご都合に合わせます。	合わせる	日時[にちじ]はご 都合[つごう]に 合[あ]わせます。	
\\	赤に黄色を合わすと何色になりますか。	合わす	赤[あか]に 黄色[きいろ]を 合[あ]わすと 何色[なにいろ]になりますか。	
\\	この春に彼は社会人になったよ。	社会人	この 春[はる]に 彼[かれ]は 社会人[しゃかいじん]になったよ。	
\\	そんなことをしたら、かえってよくないよ。	かえって	そんなことをしたら、かえってよくないよ。	
\\	彼女はクラブの会員です。	会員	彼女[かのじょ]はクラブの 会員[かいいん]です。	
\\	彼は野球部の一員です。	一員	彼[かれ]は 野球部[やきゅうぶ]の 一員[いちいん]です。	
\\	彼は論文の仕上げに入ったの。	仕上げ	彼[かれ]は 論文[ろんぶん]の 仕上[しあ]げに 入[はい]ったの。	
\\	彼はやっと報告書を仕上げたよ。	仕上げる	彼[かれ]はやっと 報告書[ほうこくしょ]を 仕上[しあ]げたよ。	
\\	彼は人事を担当しているんだ。	人事	彼[かれ]は 人事[じんじ]を 担当[たんとう]しているんだ。	
\\	今日は学校で行事がありました。	行事	今日[きょう]は 学校[がっこう]で 行事[ぎょうじ]がありました。	
\\	そんな事をしてはいけません。	事	そんな 事[こと]をしてはいけません。	
\\	コンクリートにひびが入っていますね。	コンクリート	コンクリートにひびが 入[はい]っていますね。	
\\	母は毎日てきぱきと家事をしているよ。	家事	母[はは]は 毎日[まいにち]てきぱきと 家事[かじ]をしているよ。	
\\	これは人工の湖です。	人工	これは 人工[じんこう]の 湖[みずうみ]です。	
\\	会場は人で一杯になりました。	会場	会場[かいじょう]は 人[ひと]で 一杯[いっぱい]になりました。	
\\	今日は8チームが出場しました。	出場	今日[きょう]は 8[はち]チームが 出場[しゅつじょう]しました。	
\\	車内に忘れ物があったよ。	車内	車内[しゃない]に 忘[わす]れ 物[もの]があったよ。	
\\	景色が良かったので途中下車したよ。	下車	景色[けしき]が 良[よ]かったので 途中[とちゅう] 下車[げしゃ]したよ。	
\\	駅前に交番が有ります。	駅前	駅前[えきまえ]に 交番[こうばん]が 有[あ]ります。	
\\	明日はいよいよ出発の日です。	いよいよ	明日[あす]はいよいよ 出発[しゅっぱつ]の 日[ひ]です。	
\\	政府は外交に力を入れているの。	外交	政府[せいふ]は 外交[がいこう]に 力[ちから]を 入[い]れているの。	
\\	彼は私と再会の約束を交わした。	交わす	彼[かれ]は 私[わたし]と 再会[さいかい]の 約束[やくそく]を 交[か]わした。	
\\	ここは交通の便がよいですね。	交通	ここは 交通[こうつう]の 便[べん]がよいですね。	
\\	大通りでパレードが始まるよ。	大通り	大通[おおどお]りでパレードが 始[はじ]まるよ。	
\\	書道をすると姿勢も良くなります。	書道	書道[しょどう]をすると 姿勢[しせい]も 良[よ]くなります。	
\\	十字路で車とバイクが衝突したぞ。	十字路	十字路[じゅうじろ]で 車[くるま]とバイクが 衝突[しょうとつ]したぞ。	
\\	空き地で工事が始まった。	空き地	空[あ]き 地[ち]で 工事[こうじ]が 始[はじ]まった。	
\\	オリンピックは4年に1度開催されます。	オリンピック	オリンピックは 4年[よねん]に 1度開催[いちど かいさい]されます。	
\\	図を描いて説明しましょう。	図	図[ず]を 描[か]いて 説明[せつめい]しましょう。	
\\	車掌が発車の合図をしたよ。	合図	車掌[しゃしょう]が 発車[はっしゃ]の 合図[あいず]をしたよ。	
\\	ここから先は行き止まりです。	行き止まり	ここから 先[さき]は 行[い]き 止[ど]まりです。	
\\	英語を初歩から勉強しています。	初歩	英語[えいご]を 初歩[しょほ]から 勉強[べんきょう]しています。	
\\	高度な技術を持つ人材を集めました。	高度	高度[こうど]な 技術[ぎじゅつ]を 持[も]つ 人材[じんざい]を 集[あつ]めました。	
\\	皆の顔と名前を一度には覚えられません。	一度に	皆[みんな]の 顔[かお]と 名前[なまえ]を 一度[いちど]には 覚[おぼ]えられません。	
\\	明日は遠足で動物園に行きます。	遠足	明日[あす]は 遠足[えんそく]で 動物園[どうぶつえん]に 行[い]きます。	
\\	彼は駅長です。	駅長	彼[かれ]は 駅長[えきちょう]です。	
\\	ここでいったんストップしてください。	ストップ	ここでいったんストップしてください。	
\\	チーム全員に名札が配られました。	全員	チーム 全員[ぜんいん]に 名札[なふだ]が 配[くば]られました。	
\\	彼に全てのことを伝えた。	全て	彼[かれ]に 全[すべ]てのことを 伝[つた]えた。	
\\	これは外部には秘密です。	外部	これは 外部[がいぶ]には 秘密[ひみつ]です。	
\\	僕が知っているのは一部分に過ぎない。	一部分	僕[ぼく]が 知[し]っているのは 一部分[いちぶぶん]に 過[す]ぎない。	
\\	首相は国家のリーダーだ。	国家	首相[しゅしょう]は 国家[こっか]のリーダーだ。	
\\	そのマラソン大会にはたくさんの国々から選手が集まったよ。	国々	そのマラソン 大会[たいかい]にはたくさんの 国々[くにぐに]から 選手[せんしゅ]が 集[あつ]まったよ。	
\\	犯人は国外に逃げたようです。	国外	犯人[はんにん]は 国外[こくがい]に 逃[に]げたようです。	
\\	問題についてきちんと話し合ったよ。	きちんと	問題[もんだい]についてきちんと 話[はな]し 合[あ]ったよ。	
\\	1週間後に出国します。	出国	1週間後[いっしゅうかん ご]に 出国[しゅっこく]します。	
\\	わが国の国土は70
\\	が森林です。	国土	わが 国[くに]の 国土[こくど]は 
\\	[ななじゅっぱーせんと]が 森林[しんりん]です。	
\\	今日の1時間目は国語です。	国語	今日[きょう]の 1時間目[いちじかんめ]は 国語[こくご]です。	
\\	あの国とは国交がない。	国交	あの 国[くに]とは 国交[こっこう]がない。	
\\	この道をまっすぐ進むと国道に出ます。	国道	この 道[みち]をまっすぐ 進[すす]むと 国道[こくどう]に 出[で]ます。	
\\	世間の反応は冷たかったよ。	世間	世間[せけん]の 反応[はんのう]は 冷[つめ]たかったよ。	
\\	旅行の間、犬の世話をしてください。	世話	旅行[りょこう]の 間[あいだ]、 犬[いぬ]の 世話[せわ]をしてください。	
\\	夏休みにはクラブの合宿に参加します。	クラブ	夏休[なつやす]みにはクラブの 合宿[がっしゅく]に 参加[さんか]します。	
\\	彼は出世するタイプだな。	出世	彼[かれ]は 出世[しゅっせ]するタイプだな。	
\\	彼女は青白い顔をしているね。	青白い	彼女[かのじょ]は 青白[あおじろ]い 顔[かお]をしているね。	
\\	わが社は今年度、黒字となりました。	黒字	わが 社[しゃ]は 今年度[こんねんど]、 黒字[くろじ]となりました。	
\\	赤道に沿って旅をしました。	赤道	赤道[せきどう]に 沿[そ]って 旅[たび]をしました。	
\\	私たちの会社は今月は赤字よ。	赤字	私[わたし]たちの 会社[かいしゃ]は 今月[こんげつ]は 赤字[あかじ]よ。	
\\	父は以前、国鉄に勤めていました。	国鉄	父[ちち]は 以前[いぜん]、 国鉄[こくてつ]に 勤[つと]めていました。	
\\	私は私鉄の職員です。	私鉄	私[わたし]は 私鉄[してつ]の 職員[しょくいん]です。	
\\	私は木造の家屋が好きです。	家屋	私[わたし]は 木造[もくぞう]の 家屋[かおく]が 好[す]きです。	
\\	あの生徒はじっと先生の話を聞いていたね。	じっと	あの 生徒[せいと]はじっと 先生[せんせい]の 話[はなし]を 聞[き]いていたね。	
\\	屋上から富士山が見えました。	屋上	屋上[おくじょう]から 富士山[ふじさん]が 見[み]えました。	
\\	母の手料理をゆっくり味わいました。	味わう	母[はは]の 手料理[てりょうり]をゆっくり 味[あじ]わいました。	
\\	今日、彼女は地味な服装をしていますね。	地味	今日[きょう]、 彼女[かのじょ]は 地味[じみ]な 服装[ふくそう]をしていますね。	
\\	月末までに申込書を送ってください。	月末	月末[げつまつ]までに 申込書[もうしこみしょ]を 送[おく]ってください。	
\\	彼は5人兄弟の末っ子です。	末っ子	彼[かれ]は 5人兄弟[ごにん きょうだい]の 末[すえ]っ 子[こ]です。	
\\	長い話合いの末、やっと同意に至った。	末	長[なが]い 話合[はなしあ]いの 末[すえ]、やっと 同意[どうい]に 至[いた]った。	
\\	被災地では飲料水が不足しているの。	飲料水	被災地[ひさいち]では 飲料水[いんりょうすい]が 不足[ふそく]しているの。	
\\	友達とジャズのコンサートに行きました。	コンサート	友達[ともだち]とジャズのコンサートに 行[い]きました。	
\\	食料はこの箱に入っています。	食料	食料[しょくりょう]はこの 箱[はこ]に 入[はい]っています。	
\\	あなたの見解を聞かせてください。	見解	あなたの 見解[けんかい]を 聞[き]かせてください。	
\\	有りのままを話して下さい。	有りのまま	有[あ]りのままを 話[はな]して 下[くだ]さい。	
\\	彼女は有名な作家です。	作家	彼女[かのじょ]は 有名[ゆうめい]な 作家[さっか]です。	
\\	この製品を使用する前に、説明書をお読みください。	使用	この 製品[せいひん]を 使用[しよう]する 前[まえ]に、 説明書[せつめいしょ]をお 読[よ]みください。	
\\	この薬は神経に作用します。	作用	この 薬[くすり]は 神経[しんけい]に 作用[さよう]します。	
\\	彼は使用人を首にしたよ。	使用人	彼[かれ]は 使用人[しようにん]を 首[くび]にしたよ。	
\\	勉強を怠けてはいけないよ。	いけない	勉強[べんきょう]を 怠[なま]けてはいけないよ。	
\\	私用で出かけなければなりません。	私用	私用[しよう]で 出[で]かけなければなりません。	
\\	会費が少し高いね。	会費	会費[かいひ]が 少[すこ]し 高[たか]いね。	
\\	彼は国費で留学しています。	国費	彼[かれ]は 国費[こくひ]で 留学[りゅうがく]しています。	
\\	男の子が3人もいるので食費がかさみます。	食費	男[おとこ]の 子[こ]が 3人[さんにん]もいるので 食費[しょくひ]がかさみます。	
\\	彼は私費で留学したんだ。	私費	彼[かれ]は 私費[しひ]で 留学[りゅうがく]したんだ。	
\\	日本は消費大国といわれています。	消費	日本[にっぽん]は 消費[しょうひ] 大国[たいこく]といわれています。	
\\	火事は無事消火されました。	消火	火事[かじ]は 無事[ぶじ] 消火[しょうか]されました。	
\\	この会社の売り上げは昨年の2倍ね。	売り上げ	この 会社[かいしゃ]の 売[う]り 上[あ]げは 昨年[さくねん]の 2倍[に ばい]ね。	
\\	あらかじめ必要な物を書き出して下さい。	あらかじめ	あらかじめ 必要[ひつよう]な 物[もの]を 書[か]き 出[だ]して 下[くだ]さい。	
\\	新しい車が売り出された。	売り出す	新[あたら]しい 車[くるま]が 売[う]り 出[だ]された。	
\\	駅前に新しい書店ができました。	書店	駅前[えきまえ]に 新[あたら]しい 書店[しょてん]ができました。	
\\	デパートは10時に開店しますよ。	開店	デパートは 10時[じゅうじ]に 開店[かいてん]しますよ。	
\\	この商品は小売店でも買えます。	小売店	この 商品[しょうひん]は 小売店[こうりてん]でも 買[か]えます。	
\\	兄は商社に勤めています。	商社	兄[あに]は 商社[しょうしゃ]に 勤[つと]めています。	
\\	この通りには商店が多いね。	商店	この 通[とお]りには 商店[しょうてん]が 多[おお]いね。	
\\	彼の商売は儲かっているな。	商売	彼[かれ]の 商売[しょうばい]は 儲[もう]かっているな。	
\\	課長はしばしば出張します。	しばしば	課長[かちょう]はしばしば 出張[しゅっちょう]します。	
\\	商人は数字に強いね。	商人	商人[しょうにん]は 数字[すうじ]に 強[つよ]いね。	
\\	食品は日曜日にまとめて買います。	食品	食品[しょくひん]は 日曜日[にちようび]にまとめて 買[か]います。	
\\	そちらの品は半額になっています。	品	そちらの 品[しな]は 半額[はんがく]になっています。	
\\	このドレスはとても上品なデザインですね。	上品	このドレスはとても 上品[じょうひん]なデザインですね。	
\\	彼女の下品なふるまいには我慢できない。	下品	彼女[かのじょ]の 下品[げひん]なふるまいには 我慢[がまん]できない。	
\\	彼は目的のためには手段を選ばなかったわね。	手段	彼[かれ]は 目的[もくてき]のためには 手段[しゅだん]を 選[えら]ばなかったわね。	
\\	物価は一段と上昇しそうだね。	一段と	物価[ぶっか]は 一段[いちだん]と 上昇[じょうしょう]しそうだね。	
\\	彼女はショックで口もきけなかったわ。	ショック	彼女[かのじょ]はショックで 口[くち]もきけなかったわ。	
\\	の値を求めなさい。	値	
\\	[えっくす]の 値[あたい]を 求[もと]めなさい。	
\\	その絵は一見に値しますよ。	値する	その 絵[え]は 一見[いっけん]に 値[あたい]しますよ。	
\\	とても価値のある話を聞いたよ。	価値	とても 価値[かち]のある 話[はなし]を 聞[き]いたよ。	
\\	彼女は高価な宝石を持っているわ。	高価	彼女[かのじょ]は 高価[こうか]な 宝石[ほうせき]を 持[も]っているわ。	
\\	私は彼の人格を尊敬しています。	人格	私[わたし]は 彼[かれ]の 人格[じんかく]を 尊敬[そんけい]しています。	
\\	格好のいい青年に会ったよ。	格好	格好[かっこう]のいい 青年[せいねん]に 会[あ]ったよ。	
\\	この曲を聞くと青春の頃を思い出します。	青春	この 曲[きょく]を 聞[き]くと 青春[せいしゅん]の 頃[ころ]を 思[おも]い 出[だ]します。	
\\	屋上にアンテナが立っていますね。	アンテナ	屋上[おくじょう]にアンテナが 立[た]っていますね。	
\\	春分の日は毎年3月20日頃です。	春分	春分[しゅんぶん]の 日[ひ]は 毎年3月20日頃[まいとし さんがつ はつかごろ]です。	
\\	初夏の高原は気持ちがいいですね。	初夏	初夏[しょか]の 高原[こうげん]は 気持[きも]ちがいいですね。	
\\	秋分の日は毎年9月23日頃です。	秋分	秋分[しゅうぶん]の 日[ひ]は 毎年9月23日頃[まいとし くがつ にじゅうさんにちごろ]です。	
\\	春夏秋冬の移り変わりを見るのが大好きです。	春夏秋冬	春夏秋冬[しゅんかしゅうとう]の 移[うつ]り 変[か]わりを 見[み]るのが 大好[だいす]きです。	
\\	夏季講習に申し込みした?	夏季	夏季[かき] 講習[こうしゅう]に 申[もう]し 込[こ]みした?	
\\	何だか寒気がします。	寒気	何[なん]だか 寒気[さむけ]がします。	
\\	最近は暖かです。	暖か	最近[さいきん]は 暖[あたた]かです。	
\\	金属は高温で溶かします。	高温	金属[きんぞく]は 高温[こうおん]で 溶[と]かします。	
\\	彼はあえて危険を冒したの。	あえて	彼[かれ]はあえて 危険[きけん]を 冒[おか]したの。	
\\	この地方は温暖で暮らしやすいな。	温暖	この 地方[ちほう]は 温暖[おんだん]で 暮[く]らしやすいな。	
\\	北風が冷たいです。	北風	北風[きたかぜ]が 冷[つめ]たいです。	
\\	秋風が気持ちいいね。	秋風	秋風[あきかぜ]が 気持[きも]ちいいね。	
\\	父は情熱を持って仕事に打ち込んでいます。	情熱	父[ちち]は 情熱[じょうねつ]を 持[も]って 仕事[しごと]に 打[う]ち 込[こ]んでいます。	
\\	その広告を新聞で見ました。	広告	その 広告[こうこく]を 新聞[しんぶん]で 見[み]ました。	
\\	新たな計画が進んでいます。	新た	新[あら]たな 計画[けいかく]が 進[すす]んでいます。	
\\	このビルは新聞社です。	新聞社	このビルは 新聞社[しんぶんしゃ]です。	
\\	本にカバーを掛けました。	カバー	本[ほん]にカバーを 掛[か]けました。	
\\	彼は今日入ったばかりの新人です。	新人	彼[かれ]は 今日入[きょう はい]ったばかりの 新人[しんじん]です。	
\\	何とか最悪の事態を避けることができました。	最悪	何[なん]とか 最悪[さいあく]の 事態[じたい]を 避[さ]けることができました。	
\\	彼は地位を悪用しています。	悪用	彼[かれ]は 地位[ちい]を 悪用[あくよう]しています。	
\\	彼は悪を憎んでいます。	悪	彼[かれ]は 悪[あく]を 憎[にく]んでいます。	
\\	彼は悪女に騙されたんだ。	悪女	彼[かれ]は 悪女[あくじょ]に 騙[だま]されたんだ。	
\\	顧客心理を理解することは重要です。	心理	顧客[こきゃく] 心理[しんり]を 理解[りかい]することは 重要[じゅうよう]です。	
\\	彼は素直な心を持っている。	心	彼[かれ]は 素直[すなお]な 心[こころ]を 持[も]っている。	
\\	彼女は有能なジャーナリストだ。	ジャーナリスト	彼女[かのじょ]は 有能[ゆうのう]なジャーナリストだ。	
\\	必死の思いで彼に頼んだよ。	思い	必死[ひっし]の 思[おも]いで 彼[かれ]に 頼[たの]んだよ。	
\\	嬉しくて思わず涙が出ました。	思わず	嬉[うれ]しくて 思[おも]わず 涙[なみだ]が 出[で]ました。	
\\	彼から思いがけないことを聞いた。	思いがけない	彼[かれ]から 思[おも]いがけないことを 聞[き]いた。	
\\	彼女の思いやりが嬉しかった。	思いやり	彼女[かのじょ]の 思[おも]いやりが 嬉[うれ]しかった。	
\\	このことを決して忘れないでください。	決して	このことを 決[けっ]して 忘[わす]れないでください。	
\\	今度こそタバコを止める決心をしました。	決心	今度[こんど]こそタバコを 止[や]める 決心[けっしん]をしました。	
\\	決まりを守ることは大切です。	決まり	決[き]まりを 守[まも]ることは 大切[たいせつ]です。	
\\	お二人はどこで知り合ったのですか。	知り合う	お 二人[ふたり]はどこで 知[し]り 合[あ]ったのですか。	
\\	彼女はモデルのようにスタイルがいいね。	スタイル	彼女[かのじょ]はモデルのようにスタイルがいいね。	
\\	他人に知れるとまずいことになるわね。	知れる	他人[たにん]に 知[し]れるとまずいことになるわね。	
\\	街で知り合いを見かけたよ。	知り合い	街[まち]で 知[し]り 合[あ]いを 見[み]かけたよ。	
\\	今日、合格の知らせをもらいました。	知らせ	今日[きょう]、 合格[ごうかく]の 知[し]らせをもらいました。	
\\	知らず知らずのうちに疲れがたまっていたよ。	知らず知らず	知[し]らず 知[し]らずのうちに 疲[つか]れがたまっていたよ。	
\\	彼は芸術的な才能にあふれているね。	才能	彼[かれ]は 芸術的[げいじゅつてき]な 才能[さいのう]にあふれているね。	
\\	ちょっと小便しに行って来る。	小便	ちょっと 小便[しょうべん]しに 行[い]って 来[く]る。	
\\	彼女はラジオ局で働いています。	局	彼女[かのじょ]はラジオ 局[きょく]で 働[はたら]いています。	
\\	彼はインタビューに、はきはきと答えてたよ。	インタビュー	彼[かれ]はインタビューに、はきはきと 答[こた]えてたよ。	
\\	私の住まいは東京にあります。	住まい	私[わたし]の 住[す]まいは 東京[とうきょう]にあります。	
\\	会長は田中氏に決定。	氏	会長[かいちょう]は 田中[たなか] 氏[し]に 決定。[けってい]	
\\	これは日本の人名ですか。	人名	これは 日本[にほん]の 人名[じんめい]ですか。	
\\	アジアの国名をいくつ知っていますか。	国名	アジアの 国名[こくめい]をいくつ 知[し]っていますか。	
\\	各地で大雨が降っています。	各地	各地[かくち]で 大雨[おおあめ]が 降[ふ]っています。	
\\	県の代表は2名です。	県	県[けん]の 代表[だいひょう]は 2名[にめい]です。	
\\	明日は市内を観光する予定です。	市内	明日[あした]は 市内[しない]を 観光[かんこう]する 予定[よてい]です。	
\\	その選手は今シーズンも好調だね。	シーズン	その 選手[せんしゅ]は 今[こん]シーズンも 好調[こうちょう]だね。	
\\	新しい市長が選ばれました。	市長	新[あたら]しい 市長[しちょう]が 選[えら]ばれました。	
\\	市場で新鮮な魚を買ってきました。	市場	市場[いちば]で 新鮮[しんせん]な 魚[さかな]を 買[か]ってきました。	
\\	デジカメ市場は急速に拡大している。	市場	デジカメ 市場[しじょう]は 急速[きゅうそく]に 拡大[かくだい]している。	
\\	祖父は市外の病院に通っているの。	市外	祖父[そふ]は 市外[しがい]の 病院[びょういん]に 通[かよ]っているの。	
\\	その市の人口は減り続けているの。	市	その 市[し]の 人口[じんこう]は 減[へ]り 続[つづ]けているの。	
\\	あの子は下町育ちだ。	下町	あの 子[こ]は 下町[したまち] 育[そだ]ちだ。	
\\	この表は年齢区分ごとの人口を表しています。	区分	この 表[ひょう]は 年齢[ねんれい] 区分[くぶん]ごとの 人口[じんこう]を 表[あらわ]しています。	
\\	東京には23の区がある。	区	東京[とうきょう]には23の 区[く]がある。	
\\	彼がいいアイデアを出したね。	アイデア	彼[かれ]がいいアイデアを 出[だ]したね。	
\\	その都市には様々な人種が集まっているわ。	様々	その 都市[とし]には 様々[さまざま]な 人種[じんしゅ]が 集[あつ]まっているわ。	
\\	ご注文の品は明日出荷致します。	出荷	ご 注文[ちゅうもん]の 品[しな]は 明日[あす] 出荷[しゅっか] 致[いた]します。	
\\	彼は会社の重要な人物です。	人物	彼[かれ]は 会社[かいしゃ]の 重要[じゅうよう]な 人物[じんぶつ]です。	
\\	見物人が大勢集まっているね。	見物人	見物人[けんぶつにん]が 大勢集[おおぜい あつ]まっているね。	
\\	荷物が多いので大きな入れ物が必要です。	入れ物	荷物[にもつ]が 多[おお]いので 大[おお]きな 入[い]れ 物[もの]が 必要[ひつよう]です。	
\\	米はアジアでは大切な作物だ。	作物	米[こめ]はアジアでは 大切[たいせつ]な 作物[さくもつ]だ。	
\\	人間にとって水は食物より大切なの。	食物	人間[にんげん]にとって 水[みず]は 食物[しょくもつ]より 大切[たいせつ]なの。	
\\	彼はいかにもスポーツマンらしいですね。	いかにも	彼[かれ]はいかにもスポーツマンらしいですね。	
\\	彼は書物に囲まれて生活しているの。	書物	彼[かれ]は 書物[しょもつ]に 囲[かこ]まれて 生活[せいかつ]しているの。	
\\	荷物はここに重ねてください。	重ねる	荷物[にもつ]はここに 重[かさ]ねてください。	
\\	高速道路で事故が重なった。	重なる	高速道路[こうそく どうろ]で 事故[じこ]が 重[かさ]なった。	
\\	重大な発表があります。	重大	重大[じゅうだい]な 発表[はっぴょう]があります。	
\\	雪の重みで枝が折れそうだ。	重み	雪[ゆき]の 重[おも]みで 枝[えだ]が 折[お]れそうだ。	
\\	この鞄は重たいです。	重たい	この 鞄[かばん]は 重[おも]たいです。	
\\	いつでも気軽に遊びに来て下さい。	気軽	いつでも 気軽[きがる]に 遊[あそ]びに 来[き]て 下[くだ]さい。	
\\	後ろからいきなり肩をたたかれた。	いきなり	後[うし]ろからいきなり 肩[かた]をたたかれた。	
\\	この荷物はかなりの重量ですね。	重量	この 荷物[にもつ]はかなりの 重量[じゅうりょう]ですね。	
\\	泡立てたクリームに少量のブランデーを加えます。	少量	泡立[あわだ]てたクリームに 少量[しょうりょう]のブランデーを 加[くわ]えます。	
\\	私はコーヒー豆を小量で買うようにしています。	小量	私[わたし]はコーヒー 豆[まめ]を 小量[しょうりょう]で 買[か]うようにしています。	
\\	私は彼の意見を受け入れました。	受け入れる	私[わたし]は 彼[かれ]の 意見[いけん]を 受[う]け 入[い]れました。	
\\	ボールが速過ぎて受け止められなかったの。	受け止める	ボールが 速過[はや す]ぎて 受[う]け 止[と]められなかったの。	
\\	第一志望の大学に受かりました。	受かる	第一志望[だいいち しぼう]の 大学[だいがく]に 受[う]かりました。	
\\	受け取りに判子をお願いします。	受け取り	受[う]け 取[と]りに 判子[はんこ]をお 願[ねが]いします。	
\\	いつまでもあなたを忘れません。	いつまでも	いつまでもあなたを 忘[わす]れません。	
\\	英語の聞き取り試験を受けたよ。	聞き取り	英語[えいご]の 聞[き]き 取[と]り 試験[しけん]を 受[う]けたよ。	
\\	僕たちは毎朝漢字の書き取りをします。	書き取り	僕[ぼく]たちは 毎朝漢字[まいあさ かんじ]の 書[か]き 取[と]りをします。	
\\	1年生を受け持っています。	受け持つ	1年生[いちねんせい]を 受[う]け 持[も]っています。	
\\	夏祭りで花火を打ち上げます。	打ち上げる	夏祭[なつまつ]りで 花火[はなび]を 打[う]ち 上[あ]げます。	
\\	午後に打ち合わせをしましょう。	打ち合わせ	午後[ごご]に 打[う]ち 合[あ]わせをしましょう。	
\\	親友に悩みを打ち明けたの。	打ち明ける	親友[しんゆう]に 悩[なや]みを 打[う]ち 明[あ]けたの。	
\\	来週の予定を打ち合わせましょう。	打ち合わせる	来週[らいしゅう]の 予定[よてい]を 打[う]ち 合[あ]わせましょう。	
\\	彼はうわさを打ち消したわよ。	打ち消し	彼[かれ]はうわさを 打[う]ち 消[け]したわよ。	
\\	彼は若いのにしっかりしてるね。	しっかり	彼[かれ]は 若[わか]いのにしっかりしてるね。	
\\	市役所で書類をもらって来たの。	市役所	市役所[しやくしょ]で 書類[しょるい]をもらって 来[き]たの。	
\\	区役所に書類を届けてください。	区役所	区役所[くやくしょ]に 書類[しょるい]を 届[とど]けてください。	
\\	明日、重役会議が開かれます。	重役	明日[あす]、 重役[じゅうやく] 会議[かいぎ]が 開[ひら]かれます。	
\\	両者の間に摩擦が生じています。	生じる	両者[りょうしゃ]の 間[あいだ]に 摩擦[まさつ]が 生[しょう]じています。	
\\	彼は自分の人生を振り返ったわ。	人生	彼[かれ]は 自分[じぶん]の 人生[じんせい]を 振[ふ]り 返[かえ]ったわ。	
\\	彼女は京都の生まれです。	生まれ	彼女[かのじょ]は 京都[きょうと]の 生[う]まれです。	
\\	一生のお願いがあります。	一生	一生[いっしょう]のお 願[ねが]いがあります。	
\\	彼女はアンケートに答えたよ。	アンケート	彼女[かのじょ]はアンケートに 答[こた]えたよ。	
\\	彼は数々の名作を生み出した。	生み出す	彼[かれ]は 数々[かずかず]の 名作[めいさく]を 生[う]み 出[だ]した。	
\\	自分らしい生き方をしなさい。	生き方	自分[じぶん]らしい 生[い]き 方[かた]をしなさい。	
\\	彼女は語学力を仕事に生かしているね。	生かす	彼女[かのじょ]は 語学力[ごがくりょく]を 仕事[しごと]に 生[い]かしているね。	
\\	生き物を大切にしましょう。	生き物	生[い]き 物[もの]を 大切[たいせつ]にしましょう。	
\\	彼女は生け花の先生です。	生け花	彼女[かのじょ]は 生[い]け 花[ばな]の 先生[せんせい]です。	
\\	ここに生年月日を記入してください。	生年月日	ここに 生年[せいねん] 月日[がっぴ]を 記入[きにゅう]してください。	
\\	昨日、生理が始まったの。	生理	昨日[きのう]、 生理[せいり]が 始[はじ]まったの。	
\\	スケジュールの調整は君に任せるわ。	スケジュール	スケジュールの 調整[ちょうせい]は 君[きみ]に 任[まか]せるわ。	
\\	彼女は生まれつき丈夫です。	生まれつき	彼女[かのじょ]は 生[う]まれつき 丈夫[じょうぶ]です。	
\\	私がこの世に生を受けて80年が過ぎたわ。	生	私[わたし]がこの 世[よ]に 生[せい]を 受[う]けて 80年[はちじゅうねん]が 過[す]ぎたわ。	
\\	海の底には不思議な生物がたくさんいるよ。	生物	海[うみ]の 底[そこ]には 不思議[ふしぎ]な 生物[せいぶつ]がたくさんいるよ。	
\\	子供たちが生き生きと遊んでいるね。	生き生きと	子供[こども]たちが 生[い]き 生[い]きと 遊[あそ]んでいるね。	
\\	今度のパソコンは性能がすごく良い。	性能	今度[こんど]のパソコンは 性能[せいのう]がすごく 良[い]い。	
\\	僕と姉の性格は正反対です。	性格	僕[ぼく]と 姉[あね]の 性格[せいかく]は 正反対[せいはんたい]です。	
\\	この会社では性による差別はありません。	性	この 会社[かいしゃ]では 性[せい]による 差別[さべつ]はありません。	
\\	このワインは国産です。	国産	このワインは 国産[こくさん]です。	
\\	車にガソリンを入れました。	ガソリン	車[くるま]にガソリンを 入[い]れました。	
\\	その地方はお茶の産地です。	産地	その 地方[ちほう]はお 茶[ちゃ]の 産地[さんち]です。	
\\	彼女は女の子を出産したんだ。	出産	彼女[かのじょ]は 女[おんな]の 子[こ]を 出産[しゅっさん]したんだ。	
\\	彼女はインターネットを活用しているの。	活用	彼女[かのじょ]はインターネットを 活用[かつよう]しているの。	
\\	新聞の活字が読みやすくなったね。	活字	新聞[しんぶん]の 活字[かつじ]が 読[よ]みやすくなったね。	
\\	彼は学会で論文を発表したよ。	学会	彼[かれ]は 学会[がっかい]で 論文[ろんぶん]を 発表[はっぴょう]したよ。	
\\	彼は私より一学年上です。	学年	彼[かれ]は 私[わたし]より 一[ひと] 学年[がくねん] 上[うえ]です。	
\\	彼は大学で工学を勉強しました。	工学	彼[かれ]は 大学[だいがく]で 工学[こうがく]を 勉強[べんきょう]しました。	
\\	市長がテープをカットした。	カット	市長[しちょう]がテープをカットした。	
\\	入学式で学長の挨拶がありました。	学長	入学式[にゅうがくしき]で 学長[がくちょう]の 挨拶[あいさつ]がありました。	
\\	海外で語学の勉強をします。	語学	海外[かいがい]で 語学[ごがく]の 勉強[べんきょう]をします。	
\\	彼は工学部の教授です。	工学部	彼[かれ]は 工学部[こうがくぶ]の 教授[きょうじゅ]です。	
\\	彼はアルバイトをして学費を稼いだんだ。	学費	彼[かれ]はアルバイトをして 学費[がくひ]を 稼[かせ]いだんだ。	
\\	彼は経済学部の学生です。	学部	彼[かれ]は 経済[けいざい] 学部[がくぶ]の 学生[がくせい]です。	
\\	学力を付けてその大学に進みたい。	学力	学力[がくりょく]を 付[つ]けてその 大学[だいがく]に 進[すす]みたい。	
\\	1年休学することにしました。	休学	1年[いちねん] 休学[きゅうがく]することにしました。	
\\	彼はその時スターだったよ。	スター	彼[かれ]はその 時[とき]スターだったよ。	
\\	彼は高校の教員です。	教員	彼[かれ]は 高校[こうこう]の 教員[きょういん]です。	
\\	この先にキリスト教の教会があります。	キリスト教	この 先[さき]にキリスト 教[きょう]の 教会[きょうかい]があります。	
\\	これはイスラム教の寺院です。	イスラム教	これはイスラム 教[きょう]の 寺院[じいん]です。	
\\	私は両親から多くを教わりました。	教わる	私[わたし]は 両親[りょうしん]から 多[おお]くを 教[おそ]わりました。	
\\	父の教えは「自分に厳しく」です。	教え	父[ちち]の 教[おし]えは
\\	自分[じぶん]に 厳[きび]しく」です。	
\\	ここのパネルで機械全体を制御できます。	制御	ここのパネルで 機械全体[きかい ぜんたい]を 制御[せいぎょ]できます。	
\\	これは強力な接着剤ね。	強力	これは 強力[きょうりょく]な 接着剤[せっちゃくざい]ね。	
\\	彼らは労働を強制されたんだ。	強制	彼[かれ]らは 労働[ろうどう]を 強制[きょうせい]されたんだ。	
\\	将来はプロのカメラマンになりたいです。	カメラマン	将来[しょうらい]はプロのカメラマンになりたいです。	
\\	彼は最強チームの一員です。	最強	彼[かれ]は 最強[さいきょう]チームの 一員[いちいん]です。	
\\	あなたが一緒にいてくれると心強い。	心強い	あなたが 一緒[いっしょ]にいてくれると 心強[こころづよ]い。	
\\	友人の強引な誘いを断れませんでした。	強引	友人[ゆうじん]の 強引[ごういん]な 誘[さそ]いを 断[ことわ]れませんでした。	
\\	論文にその本を引用したの。	引用	論文[ろんぶん]にその 本[ほん]を 引用[いんよう]したの。	
\\	この漢字を字引で引いてみて。	字引	この 漢字[かんじ]を 字引[じびき]で 引[ひ]いてみて。	
\\	今日は野外で学習した。	学習	今日[きょう]は 野外[やがい]で 学習[がくしゅう]した。	
\\	試合の結果を早く知りたい。	試合	試合[しあい]の 結果[けっか]を 早[はや]く 知[し]りたい。	
\\	彼はアルコールに強い体質です。	アルコール	彼[かれ]はアルコールに 強[つよ]い 体質[たいしつ]です。	
\\	彼は実験を試みたのよ。	試みる	彼[かれ]は 実験[じっけん]を 試[こころ]みたのよ。	
\\	日本語能力試験を受験したんだ。	受験	日本語能力試験[にほんご のうりょく しけん]を 受験[じゅけん]したんだ。	
\\	この犬は穏やかな性質だよ。	性質	この 犬[いぬ]は 穏[おだ]やかな 性質[せいしつ]だよ。	
\\	最近は悪質な事件が多いですね。	悪質	最近[さいきん]は 悪質[あくしつ]な 事件[じけん]が 多[おお]いですね。	
\\	量より質の方が大事です。	質	量[りょう]より 質[しつ]の 方[ほう]が 大事[だいじ]です。	
\\	彼は少年の頃から学問が好きでした。	学問	彼[かれ]は 少年[しょうねん]の 頃[ころ]から 学問[がくもん]が 好[す]きでした。	
\\	彼の助けは本当に有り難いな。	有り難い	彼[かれ]の 助[たす]けは 本当[ほんとう]に 有[あ]り 難[がた]いな。	
\\	せめてこれだけは約束してください。	せめて	せめてこれだけは 約束[やくそく]してください。	
\\	子供の自主性に重点を置いています。	重点	子供[こども]の 自主性[じしゅせい]に 重点[じゅうてん]を 置[お]いています。	
\\	彼の弱点はスタミナが足りないところです。	弱点	彼[かれ]の 弱点[じゃくてん]はスタミナが 足[た]りないところです。	
\\	その計画に反対の人はほんの少数だったよ。	少数	その 計画[けいかく]に 反対[はんたい]の 人[ひと]はほんの 少数[しょうすう]だったよ。	
\\	小数は切り捨てて計算して下さい。	小数	小数[しょうすう]は 切[き]り 捨[す]てて 計算[けいさん]して 下[くだ]さい。	
\\	コンピュータの電子回路が故障した。	回路	コンピュータの 電子[でんし] 回路[かいろ]が 故障[こしょう]した。	
\\	最近はテレビを見る回数が減りました。	回数	最近[さいきん]はテレビを 見[み]る 回数[かいすう]が 減[へ]りました。	
\\	前回の続きから始めます。	前回	前回[ぜんかい]の 続[つづ]きから 始[はじ]めます。	
\\	ストレスがいろいろな病気の元になっているの。	ストレス	ストレスがいろいろな 病気[びょうき]の 元[もと]になっているの。	
\\	おしゃべりは後回しにしましょう。	後回し	おしゃべりは 後回[あとまわ]しにしましょう。	
\\	彼女たちはそれぞれ個性が強いですね。	個性	彼女[かのじょ]たちはそれぞれ 個性[こせい]が 強[つよ]いですね。	
\\	個々の問題を解決しましょう。	個々	個々[ここ]の 問題[もんだい]を 解決[かいけつ]しましょう。	
\\	勝手なことばかり言わないでくれ。	勝手	勝手[かって]なことばかり 言[い]わないでくれ。	
\\	僕たちは頑張って決勝まで進んだよ。	決勝	僕[ぼく]たちは 頑張[がんば]って 決勝[けっしょう]まで 進[すす]んだよ。	
\\	歌合戦は赤組の勝ちでしたね。	勝ち	歌合戦[うたがっせん]は 赤組[あか ぐみ]の 勝[か]ちでしたね。	
\\	勝負はまだ始まったばかりよ。	勝負	勝負[しょうぶ]はまだ 始[はじ]まったばかりよ。	
\\	彼は大きな荷物を担いで来たの。	担ぐ	彼[かれ]は 大[おお]きな 荷物[にもつ]を 担[かつ]いで 来[き]たの。	
\\	コードが短くてコンセントに届きません。	コード	コードが 短[みじか]くてコンセントに 届[とど]きません。	
\\	彼はくじ引きで一等賞を当てたよ。	当てる	彼[かれ]はくじ 引[び]きで 一等賞[いっとう しょう]を 当[あ]てたよ。	
\\	あなたの成績が下がったのは当たり前です。	当たり前	あなたの 成績[せいせき]が 下[さ]がったのは 当[あ]たり 前[まえ]です。	
\\	この仕事には何日必要か見当もつかないね。	見当	この 仕事[しごと]には 何日必要[なんにち ひつよう]か 見当[けんとう]もつかないね。	
\\	彼の予想は大当たりでした。	当たり	彼[かれ]の 予想[よそう]は 大[おお] 当[あ]たりでした。	
\\	牛肉の輸入が厳しく規制されているわね。	規制	牛肉[ぎゅうにく]の 輸入[ゆにゅう]が 厳[きび]しく 規制[きせい]されているわね。	
\\	これからは経費を節約しましょう。	経費	これからは 経費[けいひ]を 節約[せつやく]しましょう。	
\\	父の手術が無事に済みました。	済む	父[ちち]の 手術[しゅじゅつ]が 無事[ぶじ]に 済[す]みました。	
\\	新しいコーチの指導は厳しかった。	コーチ	新[あたら]しいコーチの 指導[しどう]は 厳[きび]しかった。	
\\	昼ごはんは簡単に済ませましょう。	済ませる	昼[ひる]ごはんは 簡単[かんたん]に 済[す]ませましょう。	
\\	宿題を済ませてから遊びなさい。	済ます	宿題[しゅくだい]を 済[す]ませてから 遊[あそ]びなさい。	
\\	彼は株式の売買で多額の利益を得たのさ。	株式	彼[かれ]は 株式[かぶしき]の 売買[ばいばい]で 多額[たがく]の 利益[りえき]を 得[え]たのさ。	
\\	彼らは教会で式を挙げました。	式	彼[かれ]らは 教会[きょうかい]で 式[しき]を 挙[あ]げました。	
\\	私は
\\	業界で働いています。	業界	私[わたし]は
\\	業界[ぎょうかい]で 働[はたら]いています。	
\\	彼は営業を担当しています。	営業	彼[かれ]は 営業[えいぎょう]を 担当[たんとう]しています。	
\\	明日は臨時に休業します。	休業	明日[あした]は 臨時[りんじ]に 休業[きゅうぎょう]します。	
\\	彼はステージに立ったよ。	ステージ	彼[かれ]はステージに 立[た]ったよ。	
\\	彼は事業に失敗したの。	事業	彼[かれ]は 事業[じぎょう]に 失敗[しっぱい]したの。	
\\	最近、銀行の金利が少し上がった。	金利	最近[さいきん]、 銀行[ぎんこう]の 金利[きんり]が 少[すこ]し 上[あ]がった。	
\\	彼女は気が利いている。	利く	彼女[かのじょ]は 気[き]が 利[き]いている。	
\\	日本人の平均収入はどの位ですか。	収入	日本人[にほんじん]の 平均[へいきん] 収入[しゅうにゅう]はどの 位[くらい]ですか。	
\\	ここは芸術を愛する国です。	芸術	ここは 芸術[げいじゅつ]を 愛[あい]する 国[くに]です。	
\\	当社の製品は3年間の保障つきです。	製品	当社[とうしゃ]の 製品[せいひん]は 3年間[さんねんかん]の 保障[ほしょう]つきです。	
\\	この映画は香港で製作されました。	製作	この 映画[えいが]は 香港[ほんこん]で 製作[せいさく]されました。	
\\	合鍵の作製には2日ほどかかります。	作製	合鍵[あいかぎ]の 作製[さくせい]には 2日[ふつか]ほどかかります。	
\\	彼のアドバイスはいつも有り難いわね。	アドバイス	彼[かれ]のアドバイスはいつも 有[あ]り 難[がた]いわね。	
\\	親切は必ずしも喜ばれるわけではない。	必ずしも	親切[しんせつ]は 必[かなら]ずしも 喜[よろこ]ばれるわけではない。	
\\	彼は求人広告で仕事を見つけたんだ。	求人	彼[かれ]は 求人[きゅうじん] 広告[こうこく]で 仕事[しごと]を 見[み]つけたんだ。	
\\	写真を額に入れて飾ったんだ。	額	写真[しゃしん]を 額[がく]に 入[い]れて 飾[かざ]ったんだ。	
\\	レシートで買い物の金額を確かめたよ。	金額	レシートで 買[か]い 物[もの]の 金額[きんがく]を 確[たし]かめたよ。	
\\	3人分の代金は計6000円です。	計	3人分[さんにんぶん]の 代金[だいきん]は 計[けい] 6000円[ろくせんえん]です。	
\\	合計金額を計算してください。	合計	合計[ごうけい] 金額[きんがく]を 計算[けいさん]してください。	
\\	彼女は家計を任されているの。	家計	彼女[かのじょ]は 家計[かけい]を 任[まか]されているの。	
\\	このお鍋はアルミニウム製です。	アルミニウム	このお 鍋[なべ]はアルミニウム 製[せい]です。	
\\	会計を済ませて店を出たんだ。	会計	会計[かいけい]を 済[す]ませて 店[みせ]を 出[で]たんだ。	
\\	壁に寒暖計が掛かっていました。	寒暖計	壁[かべ]に 寒暖計[かんだんけい]が 掛[か]かっていました。	
\\	都心と地方では収入に大きな差があるね。	差	都心[としん]と 地方[ちほう]では 収入[しゅうにゅう]に 大[おお]きな 差[さ]があるね。	
\\	貧富の格差が大きくなっているな。	格差	貧富[ひんぷ]の 格差[かくさ]が 大[おお]きくなっているな。	
\\	彼は握手をしようと手を差し出したの。	差し出す	彼[かれ]は 握手[あくしゅ]をしようと 手[て]を 差[さ]し 出[だ]したの。	
\\	日本とフランスの時差は8時間です。	時差	日本[にほん]とフランスの 時差[じさ]は 8時間[はちじかん]です。	
\\	雨が降ってきたので傘を差しました。	差す	雨[あめ]が 降[ふ]ってきたので 傘[かさ]を 差[さ]しました。	
\\	彼はオーケストラを指揮しているの。	オーケストラ	彼[かれ]はオーケストラを 指揮[しき]しているの。	
\\	こちらを差し上げます。	差し上げる	こちらを 差[さ]し 上[あ]げます。	
\\	学割だとだいぶ安いな。	学割	学割[がくわり]だとだいぶ 安[やす]いな。	
\\	明日の授業は時間割り通りです。	時間割り	明日[あした]の 授業[じゅぎょう]は 時間割[じかんわ]り 通[どお]りです。	
\\	昨日は遅くまで残業しました。	残業	昨日[きのう]は 遅[おそ]くまで 残業[ざんぎょう]しました。	
\\	今年も残暑が厳しかった。	残暑	今年[ことし]も 残暑[ざんしょ]が 厳[きび]しかった。	
\\	彼は支店に転勤したよ。	支店	彼[かれ]は 支店[してん]に 転勤[てんきん]したよ。	
\\	彼は国民の支持を得たのよ。	支持	彼[かれ]は 国民[こくみん]の 支持[しじ]を 得[え]たのよ。	
\\	今月のわが家の支出は15万円です。	支出	今月[こんげつ]のわが 家[や]の 支出[ししゅつ]は 15万円[じゅうごまんえん]です。	
\\	今そこで隣のおじさんに会ったよ。	おじさん	今[いま]そこで 隣[となり]のおじさんに 会[あ]ったよ。	
\\	その権力者による支配は50年以上続いたんです。	支配	その 権力者[けんりょくしゃ]による 支配[しはい]は 50年以上続[ごじゅうねん いじょう つづ]いたんです。	
\\	家計の収支が合わないの。	収支	家計[かけい]の 収支[しゅうし]が 合[あ]わないの。	
\\	支度ができたら出かけましょう。	支度	支度[したく]ができたら 出[で]かけましょう。	
\\	父親には一家を支える責任がある。	支える	父親[ちちおや]には 一家[いっか]を 支[ささ]える 責任[せきにん]がある。	
\\	来月大阪に支社を開設します。	支社	来月大阪[らいげつ おおさか]に 支社[ししゃ]を 開設[かいせつ]します。	
\\	カウンターで料金を支払った。	支払う	カウンターで 料金[りょうきん]を 支払[しはら]った。	
\\	お支払いはカードもお使いいただけます。	支払い	お 支払[しはら]いはカードもお 使[つか]いいただけます。	
\\	来年のコンクールに出場するつもりです。	コンクール	来年[らいねん]のコンクールに 出場[しゅつじょう]するつもりです。	
\\	彼女は負けずに言い返したの。	言い返す	彼女[かのじょ]は 負[ま]けずに 言[い]い 返[かえ]したの。	
\\	もう一度原点に返って考えましょう。	返る	もう 一度原点[いちど げんてん]に 返[かえ]って 考[かんが]えましょう。	
\\	この借りは必ず返します。	借り	この 借[か]りは 必[かなら]ず 返[かえ]します。	
\\	彼は友達に借金をしたんだ。	借金	彼[かれ]は 友達[ともだち]に 借金[しゃっきん]をしたんだ。	
\\	その本は貸し出ししていません。	貸し出し	その 本[ほん]は 貸[か]し 出[だ]ししていません。	
\\	彼には貸しがあるの。	貸し	彼[かれ]には 貸[か]しがあるの。	
\\	これからパスポートの申請に行きます。	申請	これからパスポートの 申請[しんせい]に 行[い]きます。	
\\	彼はアマチュア音楽家です。	アマチュア	彼[かれ]はアマチュア 音楽家[おんがくか]です。	
\\	彼は感情を込めてその歌を歌ったの。	込める	彼[かれ]は 感情[かんじょう]を 込[こ]めてその 歌[うた]を 歌[うた]ったの。	
\\	彼は研究に打ち込んでいます。	打ち込む	彼[かれ]は 研究[けんきゅう]に 打[う]ち 込[こ]んでいます。	
\\	彼は騙されたと思い込んでいるようです。	思い込む	彼[かれ]は 騙[だま]されたと 思[おも]い 込[こ]んでいるようです。	
\\	鼻水は風邪の初期症状のひとつです。	初期	鼻水[はなみず]は 風邪[かぜ]の 初期[しょき] 症状[しょうじょう]のひとつです。	
\\	新学期が始まったね。	学期	新[しん] 学期[がっき]が 始[はじ]まったね。	
\\	後期の授業が始まりました。	後期	後期[こうき]の 授業[じゅぎょう]が 始[はじ]まりました。	
\\	前期の売上はとても良かったわ。	前期	前期[ぜんき]の 売上[うりあげ]はとても 良[よ]かったわ。	
\\	彼女は性格がきついよね。	きつい	彼女[かのじょ]は 性格[せいかく]がきついよね。	
\\	代金を期日までにお支払いください。	期日	代金[だいきん]を 期日[きじつ]までにお 支払[しはら]いください。	
\\	今日から新学期が始まります。	新学期	今日[きょう]から 新学期[しんがっき]が 始[はじ]まります。	
\\	限りある資源を大切にしよう。	限り	限[かぎ]りある 資源[しげん]を 大切[たいせつ]にしよう。	
\\	彼女は食事を制限しています。	制限	彼女[かのじょ]は 食事[しょくじ]を 制限[せいげん]しています。	
\\	もう我慢の限界です。	限界	もう 我慢[がまん]の 限界[げんかい]です。	
\\	期限までに申し込みました。	期限	期限[きげん]までに 申[もう]し 込[こ]みました。	
\\	物事には限度があります。	限度	物事[ものごと]には 限度[げんど]があります。	
\\	大急ぎでその仕事を仕上げたよ。	大急ぎ	大急[おおいそ]ぎでその 仕事[しごと]を 仕上[しあ]げたよ。	
\\	バーゲンコーナーで
\\	を買いました。	コーナー	バーゲンコーナーで 
\\	[でぃーぶいでぃー]を 買[か]いました。	
\\	彼は急用で帰りました。	急用	彼[かれ]は 急用[きゅうよう]で 帰[かえ]りました。	
\\	私は一切その問題とは関係がありません。	一切	私[わたし]は 一切[いっさい]その 問題[もんだい]とは 関係[かんけい]がありません。	
\\	フロアはパーティションで区切られています。	区切る	フロアはパーティションで 区切[くぎ]られています。	
\\	思い切って彼に相談します。	思い切って	思[おも]い 切[き]って 彼[かれ]に 相談[そうだん]します。	
\\	牛乳は品切れだったよ。	品切れ	牛乳[ぎゅうにゅう]は 品切[しなぎ]れだったよ。	
\\	カラオケで思い切り歌ったの。	思い切り	カラオケで 思[おも]い 切[き]り 歌[うた]ったの。	
\\	この包丁は切れが良いな。	切れ	この 包丁[ほうちょう]は 切[き]れが 良[い]いな。	
\\	会社の始業時間にぎりぎりで間に合ったよ。	ぎりぎり	会社[かいしゃ]の 始業時間[しぎょうじかん]にぎりぎりで 間[ま]に 合[あ]ったよ。	
\\	入場券をお持ちですか。	券	入場[にゅうじょう] 券[けん]をお 持[も]ちですか。	
\\	バスの回数券を買いました。	回数券	バスの 回数券[かいすうけん]を 買[か]いました。	
\\	私は彼と同じ世代です。	世代	私[わたし]は 彼[かれ]と 同[おな]じ 世代[せだい]です。	
\\	ごま油の代わりにオリーブ油を使いましょう。	代わり	ごま 油[あぶら]の 代[か]わりにオリーブ 油[ゆ]を 使[つか]いましょう。	
\\	近代の技術の発展はものすごいです。	近代	近代[きんだい]の 技術[ぎじゅつ]の 発展[はってん]はものすごいです。	
\\	古代の歴史について勉強しました。	古代	古代[こだい]の 歴史[れきし]について 勉強[べんきょう]しました。	
\\	皆が代わる代わる彼らを祝福したよ。	代わる代わる	皆[みな]が 代[か]わる 代[が]わる 彼[かれ]らを 祝福[しゅくふく]したよ。	
\\	ウサギとカメの話を知っていますか。	ウサギ	ウサギとカメの 話[はなし]を 知[し]っていますか。	
\\	時計が12時を指してる。	指す	時計[とけい]が 12時[じゅうにじ]を 指[さ]してる。	
\\	小指を切ってしまいました。	小指	小指[こゆび]を 切[き]ってしまいました。	
\\	彼は精神の安定が必要よ。	安定	彼[かれ]は 精神[せいしん]の 安定[あんてい]が 必要[ひつよう]よ。	
\\	代金には規定の手数料が含まれます。	規定	代金[だいきん]には 規定[きてい]の 手数料[てすうりょう]が 含[ふく]まれます。	
\\	税に関する新しい法律が定められたぞ。	定める	税[ぜい]に 関[かん]する 新[あたら]しい 法律[ほうりつ]が 定[さだ]められたぞ。	
\\	指定された席にお座りください。	指定	指定[してい]された 席[せき]にお 座[すわ]りください。	
\\	手の傷が悪化した。	悪化	手[て]の 傷[きず]が 悪化[あっか]した。	
\\	彼は化学の教授です。	化学	彼[かれ]は 化学[かがく]の 教授[きょうじゅ]です。	
\\	野球部はグラウンドで練習しています。	グラウンド	野球部[やきゅうぶ]はグラウンドで 練習[れんしゅう]しています。	
\\	彼は消化不良を起こしたんだ。	消化	彼[かれ]は 消化[しょうか] 不良[ふりょう]を 起[お]こしたんだ。	
\\	彼は更に質問を続けたの。	更に	彼[かれ]は 更[さら]に 質問[しつもん]を 続[つづ]けたの。	
\\	今更後悔しても、もう遅いよ。	今更	今更[いまさら] 後悔[こうかい]しても、もう 遅[おそ]いよ。	
\\	最近ヨガをやる人が急増しています。	急増	最近[さいきん]ヨガをやる 人[ひと]が 急増[きゅうぞう]しています。	
\\	この国は人口が減少しているわね。	減少	この 国[くに]は 人口[じんこう]が 減少[げんしょう]しているわね。	
\\	乗車券は無くさないように。	乗車券	乗車券[じょうしゃけん]は 無[な]くさないように。	
\\	このバスの運賃は乗車するときに払います。	乗車	このバスの 運賃[うんちん]は 乗車[じょうしゃ]するときに 払[はら]います。	
\\	彼女の言っていることがさっぱり分からないの。	さっぱり	彼女[かのじょ]の 言[い]っていることがさっぱり 分[わ]からないの。	
\\	乗客の一人の具合が悪くなったの。	乗客	乗客[じょうきゃく]の 一人[いち にん]の 具合[ぐあい]が 悪[わる]くなったの。	
\\	雨降りで月が見えなかったね。	雨降り	雨降[あめふ]りで 月[つき]が 見[み]えなかったね。	
\\	これは税金の無駄遣いだね。	税金	これは 税金[ぜいきん]の 無駄遣[むだづか]いだね。	
\\	これ以上税が上がると生活できないね。	税	これ 以上[いじょう] 税[ぜい]が 上[あ]がると 生活[せいかつ]できないね。	
\\	私立の大学は学費が高い。	私立	私立[しりつ]の 大学[だいがく]は 学費[がくひ]が 高[たか]い。	
\\	娘は市立の学校に通っています。	市立	娘[むすめ]は 市立[しりつ]の 学校[がっこう]に 通[かよ]っています。	
\\	彼は県立の高校に通っているよ。	県立	彼[かれ]は 県立[けんりつ]の 高校[こうこう]に 通[かよ]っているよ。	
\\	丸いボタンを押すとシャッターが閉まります。	シャッター	丸[まる]いボタンを 押[お]すとシャッターが 閉[し]まります。	
\\	新しい国立劇場が完成しました。	国立	新[あたら]しい 国立[こくりつ] 劇場[げきじょう]が 完成[かんせい]しました。	
\\	飛行機の座席はゆったりしていたよ。	座席	飛行機[ひこうき]の 座席[ざせき]はゆったりしていたよ。	
\\	私たちは客席に座ったんだ。	客席	私[わたし]たちは 客席[きゃくせき]に 座[すわ]ったんだ。	
\\	欠点のない人間はいません。	欠点	欠点[けってん]のない 人間[にんげん]はいません。	
\\	お気に入りのカップが欠けてしまいました。	欠ける	お 気[き]に 入[い]りのカップが 欠[か]けてしまいました。	
\\	彼の態度は誠意を欠いています。	欠く	彼[かれ]の 態度[たいど]は 誠意[せいい]を 欠[か]いています。	
\\	次回の会議は2週間後に行います。	次回	次回[じかい]の 会議[かいぎ]は 2週間後[に しゅうかん ご]に 行[おこな]います。	
\\	会計士に資金運用について相談したよ。	運用	会計士[かいけいし]に 資金[しきん] 運用[うんよう]について 相談[そうだん]したよ。	
\\	このスピーカーは音がいいですね。	スピーカー	このスピーカーは 音[おと]がいいですね。	
\\	その事業は国が運営しています。	運営	その 事業[じぎょう]は 国[くに]が 運営[うんえい]しています。	
\\	鉄道の運賃が値上げされたね。	運賃	鉄道[てつどう]の 運賃[うんちん]が 値上[ねあ]げされたね。	
\\	彼は運のいい男です。	運	彼[かれ]は 運[うん]のいい 男[おとこ]です。	
\\	彼は運送会社に勤めているの。	運送	彼[かれ]は 運送[うんそう] 会社[がいしゃ]に 勤[つと]めているの。	
\\	彼はボールに回転を掛けたんだ。	回転	彼[かれ]はボールに 回転[かいてん]を 掛[か]けたんだ。	
\\	猫は砂の上で転がったんだ。	転がる	猫[ねこ]は 砂[すな]の 上[うえ]で 転[ころ]がったんだ。	
\\	まず、肉をパン粉の上で転がします。	転がす	まず、 肉[にく]をパン 粉[こ]の 上[うえ]で 転[ころ]がします。	
\\	以前、セールスマンをしていたことがあります。	セールスマン	以前[いぜん]、セールスマンをしていたことがあります。	
\\	彼の会社は移転したよ。	移転	彼[かれ]の 会社[かいしゃ]は 移転[いてん]したよ。	
\\	私たちはレンタカーで移動しました。	移動	私[わたし]たちはレンタカーで 移動[いどう]しました。	
\\	彼の行動は理解できない。	行動	彼[かれ]の 行動[こうどう]は 理解[りかい]できない。	
\\	運動会で一位になったよ。	運動会	運動会[うんどうかい]で 一位[いちい]になったよ。	
\\	運動場でサッカーをしよう。	運動場	運動場[うんどうじょう]でサッカーをしよう。	
\\	運動場でサッカーをしよう。	運動場	運動場[うんどうじょう]でサッカーをしよう。	
\\	高速バスで東京に行ったんだ。	高速	高速[こうそく]バスで 東京[とうきょう]に 行[い]ったんだ。	
\\	彼女たちはおしゃべりに夢中です。	おしゃべり	彼女[かのじょ]たちはおしゃべりに 夢中[むちゅう]です。	
\\	あの国の経済は急速に発展しているのね。	急速	あの 国[くに]の 経済[けいざい]は 急速[きゅうそく]に 発展[はってん]しているのね。	
\\	高速道路は混んでたよ。	高速道路	高速道路[こうそく どうろ]は 混[こ]んでたよ。	
\\	では早速書類をお送りします。	早速	では 早速[さっそく] 書類[しょるい]をお 送[おく]りします。	
\\	新幹線の最高時速は300キロです。	時速	新幹線[しんかんせん]の 最高[さいこう] 時速[じそく]は 300[さんびゃく]キロです。	
\\	久しぶりに全速力で走りました。	全速力	久[ひさ]しぶりに 全速力[ぜんそくりょく]で 走[はし]りました。	
\\	電車が10分遅れで到着したんだ。	遅れ	電車[でんしゃ]が 10分[じゅっ ぷん] 遅[おく]れで 到着[とうちゃく]したんだ。	
\\	私たちは出発を1日遅らしたの。	遅らす	私[わたし]たちは 出発[しゅっぱつ]を 1日[いちにち] 遅[おく]らしたの。	
\\	カラオケで思い切り歌ったの。	カラオケ	カラオケで 思[おも]い 切[き]り 歌[うた]ったの。	
\\	やっと宿題を終えた。	終える	やっと 宿題[しゅくだい]を 終[お]えた。	
\\	東京行きの最終電車は何時ですか。	最終	東京行[とうきょうゆ]きの 最終[さいしゅう] 電車[でんしゃ]は 何時[なんじ]ですか。	
\\	電車の終点で降りました。	終点	電車[でんしゃ]の 終点[しゅうてん]で 降[お]りました。	
\\	その部屋は始終、人が出入りしているね。	始終	その 部屋[へや]は 始終[しじゅう]、 人[ひと]が 出入[でい]りしているね。	
\\	終電にやっと間に合った。	終電	終電[しゅうでん]にやっと 間[ま]に 合[あ]った。	
\\	現地の天気は雨です。	現地	現地[げんち]の 天気[てんき]は 雨[あめ]です。	
\\	現代の科学の進歩には驚くよ。	現代	現代[げんだい]の 科学[かがく]の 進歩[しんぽ]には 驚[おどろ]くよ。	
\\	代金は現金でお願いします。	現金	代金[だいきん]は 現金[げんきん]でお 願[ねが]いします。	
\\	彼女はこっそりダイエットを始めたの。	こっそり	彼女[かのじょ]はこっそりダイエットを 始[はじ]めたの。	
\\	雲の間から太陽が現れました。	現れる	雲[くも]の 間[あいだ]から 太陽[たいよう]が 現[あらわ]れました。	
\\	事故現場には入れません。	現場	事故[じこ] 現場[げんば]には 入[はい]れません。	
\\	ここには現住所を書いてください。	現住所	ここには 現住所[げんじゅうしょ]を 書[か]いてください。	
\\	彼はようやく姿を現しましたね。	現す	彼[かれ]はようやく 姿[すがた]を 現[あらわ]しましたね。	
\\	それは彼女の期待の現れです。	現れ	それは 彼女[かのじょ]の 期待[きたい]の 現[あらわ]れです。	
\\	姉は大学に在学しています。	在学	姉[あね]は 大学[だいがく]に 在学[ざいがく]しています。	
\\	現実は予想より厳しかった。	現実	現実[げんじつ]は 予想[よそう]より 厳[きび]しかった。	
\\	私は毎日ジョギングをしています。	ジョギング	私[わたし]は 毎日[まいにち]ジョギングをしています。	
\\	あれは実は私の勘違いでした。	実は	あれは 実[じつ]は 私[わたし]の 勘違[かんちが]いでした。	
\\	これはとても実用的なサイトだね。	実用	これはとても 実用[じつよう] 的[てき]なサイトだね。	
\\	二人の実力は互角です。	実力	二人[ふたり]の 実力[じつりょく]は 互角[ごかく]です。	
\\	今日は料理の実習があった。	実習	今日[きょう]は 料理[りょうり]の 実習[じっしゅう]があった。	
\\	何かの説明をする時は実物を使うとわかりやすいの。	実物	何[なに]かの 説明[せつめい]をする 時[とき]は 実物[じつぶつ]を 使[つか]うとわかりやすいの。	
\\	昨日のコンサートは実に素晴らしかったよ。	実に	昨日[さくじつ]のコンサートは 実[じつ]に 素晴[すば]らしかったよ。	
\\	夏休みを高原で過ごしました。	過ごす	夏休[なつやす]みを 高原[こうげん]で 過[す]ごしました。	
\\	服を買うのにあれこれ迷いました。	あれこれ	服[ふく]を 買[か]うのにあれこれ 迷[まよ]いました。	
\\	手術後の経過は順調です。	経過	手術後[しゅじゅつ ご]の 経過[けいか]は 順調[じゅんちょう]です。	
\\	同じ過ちを繰り返さないことだ。	過ち	同[おな]じ 過[あやま]ちを 繰[く]り 返[かえ]さないことだ。	
\\	去る者は追わず。	去る	去[さ]る 者[もの]は 追[お]わず。	
\\	活発な意見が交されたの。	活発	活発[かっぱつ]な 意見[いけん]が 交[かわ]されたの。	
\\	始発の電車に乗りました。	始発	始発[しはつ]の 電車[でんしゃ]に 乗[の]りました。	
\\	私たちは万歳をして喜びを表しました。	表す	私[わたし]たちは 万歳[ばんざい]をして 喜[よろこ]びを 表[あらわ]しました。	
\\	この気持ちをうまく言い表わせません。	言い表わす	この 気持[きも]ちをうまく 言[い]い 表[あら]わせません。	
\\	その子は絵本が大好きです。	絵本	その 子[こ]は 絵本[えほん]が 大好[だいす]きです。	
\\	家から駅までおよそ1キロあります。	およそ	家[いえ]から 駅[えき]までおよそ 1[いち]キロあります。	
\\	雑音がひどくて声が聞こえません。	雑音	雑音[ざつおん]がひどくて 声[こえ]が 聞[き]こえません。	
\\	廊下から大きな足音が聞こえた。	足音	廊下[ろうか]から 大[おお]きな 足音[あしおと]が 聞[き]こえた。	
\\	この漢字の音読みは何ですか。	音読み	この 漢字[かんじ]の 音読[おんよ]みは 何[なん]ですか。	
\\	名前は五十音順に並んでいます。	五十音	名前[なまえ]は 五十音[ごじゅうおん] 順[じゅん]に 並[なら]んでいます。	
\\	将来は気楽な生活がしたいです。	気楽	将来[しょうらい]は 気楽[きらく]な 生活[せいかつ]がしたいです。	
\\	秋は行楽にいい季節ですね。	行楽	秋[あき]は 行楽[こうらく]にいい 季節[きせつ]ですね。	
\\	彼女は薬指に指輪をはめています。	薬指	彼女[かのじょ]は 薬指[くすりゆび]に 指輪[ゆびわ]をはめています。	
\\	どのサークルに入るか迷っちゃった。	サークル	どのサークルに 入[はい]るか 迷[まよ]っちゃった。	
\\	今日は食欲がありません。	食欲	今日[きょう]は 食欲[しょくよく]がありません。	
\\	水面に月が映っているね。	映る	水面[みなも]に 月[つき]が 映[うつ]っているね。	
\\	彼女は自分の姿を鏡に映したの。	映す	彼女[かのじょ]は 自分[じぶん]の 姿[すがた]を 鏡[かがみ]に 映[うつ]したの。	
\\	独身者パーティーを企画した。	企画	独身者[どくしんしゃ]パーティーを 企画[きかく]した。	
\\	この絵はスペインの画家が描きました。	画家	この 絵[え]はスペインの 画家[がか]が 描[か]きました。	
\\	あそこの3区画は売り出し中です。	区画	あそこの 3[さん] 区画[くかく]は 売[う]り 出[だ]し 中[ちゅう]です。	
\\	テレビの画面が明るすぎる。	画面	テレビの 画面[がめん]が 明[あか]るすぎる。	
\\	キリストの母親の名前はマリアだ。	キリスト	キリストの 母親[ははおや]の 名前[なまえ]はマリアだ。	
\\	地面に何か絵が描いてあるぞ。	地面	地面[じめん]に 何[なに]か 絵[え]が 描[か]いてあるぞ。	
\\	湖の水面に小さく波が立っているな。	水面	湖[みずうみ]の 水面[すいめん]に 小[ちい]さく 波[なみ]が 立[た]っているな。	
\\	外は一面の雪景色だったね。	一面	外[そと]は 一面[いちめん]の 雪景色[ゆきげしき]だったね。	
\\	真実は一つです。	真実	真実[しんじつ]は 一[ひと]つです。	
\\	彼は人生の真理を求めて旅に出た。	真理	彼[かれ]は 人生[じんせい]の 真理[しんり]を 求[もと]めて 旅[たび]に 出[で]た。	
\\	あの寺の屋根は金色ですね。	金色	あの 寺[てら]の 屋根[やね]は 金色[きんいろ]ですね。	
\\	書類は形式を守って作ってください。	形式	書類[しょるい]は 形式[けいしき]を 守[まも]って 作[つく]ってください。	
\\	小型のスーツケースを買いました。	小型	小型[こがた]のスーツケースを 買[か]いました。	
\\	そうあっさり言わないで。	あっさり	そうあっさり 言[い]わないで。	
\\	新しい型のカメラを買いました。	型	新[あたら]しい 型[かた]のカメラを 買[か]いました。	
\\	それは果物の一種です。	一種	それは 果物[くだもの]の 一種[いっしゅ]です。	
\\	図書館には各種の雑誌が揃っています。	各種	図書館[としょかん]には 各種[かくしゅ]の 雑誌[ざっし]が 揃[そろ]っています。	
\\	これは一種独特の味がするね。	一種	これは 一種[いっしゅ] 独特[どくとく]の 味[あじ]がするね。	
\\	この国には様々な人種がいます。	人種	この 国[くに]には 様々[さまざま]な 人種[じんしゅ]がいます。	
\\	その選手は2種目でメダルを取ったわ。	種目	その 選手[せんしゅ]は 2[に] 種目[しゅもく]でメダルを 取[と]ったわ。	
\\	人類は2本の足で歩きます。	人類	人類[じんるい]は 2本[にほん]の 足[あし]で 歩[ある]きます。	
\\	彼は朝から晩までせっせと働いたの。	せっせと	彼[かれ]は 朝[あさ]から 晩[ばん]までせっせと 働[はたら]いたの。	
\\	書類を10枚コピーしました。	書類	書類[しょるい]を 10枚[じゅうまい]コピーしました。	
\\	台風が接近していますね。	接近	台風[たいふう]が 接近[せっきん]していますね。	
\\	それは間接的な原因の一つです。	間接	それは 間接[かんせつ] 的[てき]な 原因[げんいん]の 一[ひと]つです。	
\\	子供が動物に接するのは良いことです。	接する	子供[こども]が 動物[どうぶつ]に 接[せっ]するのは 良[よ]いことです。	
\\	この角度からは画面が見えにくいですね。	角度	この 角度[かくど]からは 画面[がめん]が 見[み]えにくいですね。	
\\	この三角形の面積を出しなさい。	三角形	この 三角形[さんかくけい]の 面積[めんせき]を 出[だ]しなさい。	
\\	この道は、ゆるい曲線を描いているよ。	曲線	この 道[みち]は、ゆるい 曲線[きょくせん]を 描[えが]いているよ。	
\\	カタログを見て注文したんだ。	カタログ	カタログを 見[み]て 注文[ちゅうもん]したんだ。	
\\	この曲は誰が作曲したのですか。	作曲	この 曲[きょく]は 誰[だれ]が 作曲[さっきょく]したのですか。	
\\	私たちは共通の趣味を持っています。	共通	私[わたし]たちは 共通[きょうつう]の 趣味[しゅみ]を 持[も]っています。	
\\	インドでは英語は共通語だよ。	共通語	インドでは 英語[えいご]は 共通語[きょうつうご]だよ。	
\\	2社が共同で新製品を開発したんだ。	共同	2社[にしゃ]が 共同[きょうどう]で 新製品[しんせいひん]を 開発[かいはつ]したんだ。	
\\	3社合同で新作の発表会を開きました。	合同	3社[さんしゃ] 合同[ごうどう]で 新作[しんさく]の 発表会[はっぴょうかい]を 開[ひら]きました。	
\\	一同顔を見合わせました。	一同	一同[いちどう] 顔[かお]を 見合[みあ]わせました。	
\\	以前彼はこの町に住んでいたんだ。	以前	以前[いぜん] 彼[かれ]はこの 町[まち]に 住[す]んでいたんだ。	
\\	昨日の夜、ジャズを聴きに行きました。	ジャズ	昨日[きのう]の 夜[よる]、ジャズを 聴[き]きに 行[い]きました。	
\\	数学が平均点以下だった。	以下	数学[すうがく]が 平均点[へいきんてん] 以下[いか]だった。	
\\	夕方以降にお電話を下さい。	以降	夕方[ゆうがた] 以降[いこう]にお 電話[でんわ]を 下[くだ]さい。	
\\	それ以来彼女に会ってない。	以来	それ 以来[いらい] 彼女[かのじょ]に 会[あ]ってない。	
\\	すみません、以後気を付けます。	以後	すみません、 以後[いご] 気[き]を 付[つ]けます。	
\\	親類の家に4年間下宿しました。	下宿	親類[しんるい]の 家[いえ]に 4年間[よねんかん] 下宿[げしゅく]しました。	
\\	漢和辞典で漢字の意味を調べたの。	漢和	漢和[かんわ] 辞典[じてん]で 漢字[かんじ]の 意味[いみ]を 調[しら]べたの。	
\\	英和辞書をよく使います。	英和	英和[えいわ] 辞書[じしょ]をよく 使[つか]います。	
\\	私は西洋の建築に興味があります。	西洋	私[わたし]は 西洋[せいよう]の 建築[けんちく]に 興味[きょうみ]があります。	
\\	ここにサインしてください。	サイン	ここにサインしてください。	
\\	その昔航海士達は未踏の地を求め海洋に乗り出しました。	海洋	その 昔航海士達[むかし こうかいしたち]は 未踏[みとう]の 地[ち]を 求[もと]め 海洋[かいよう]に 乗[の]り 出[だ]しました。	
\\	西洋人は正座が苦手よ。	西洋人	西洋人[せいようじん]は 正座[せいざ]が 苦手[にがて]よ。	
\\	私の学校には制服がありません。	制服	私[わたし]の 学校[がっこう]には 制服[せいふく]がありません。	
\\	雨の日は子供を室内で遊ばせます。	室内	雨[あめ]の 日[ひ]は 子供[こども]を 室内[しつない]で 遊[あそ]ばせます。	
\\	池に鴨の親子がいます。	親子	池[いけ]に 鴨[かも]の 親子[おやこ]がいます。	
\\	彼は幼い時に親類に預けられたんだ。	親類	彼[かれ]は 幼[おさな]い 時[とき]に 親類[しんるい]に 預[あず]けられたんだ。	
\\	ハイキングは自然に親しむ良い機会ですよ。	親しむ	ハイキングは 自然[しぜん]に 親[した]しむ 良[い]い 機会[きかい]ですよ。	
\\	彼は質問の答えをごまかしたね。	ごまかす	彼[かれ]は 質問[しつもん]の 答[こた]えをごまかしたね。	
\\	ご親切は決して忘れません。	親切	ご 親切[しんせつ]は 決[けっ]して 忘[わす]れません。	
\\	彼は私の親友です。	親友	彼[かれ]は 私[わたし]の 親友[しんゆう]です。	
\\	親指を怪我しました。	親指	親指[おやゆび]を 怪我[けが]しました。	
\\	彼には誰もが親しみを感じます。	親しみ	彼[かれ]には 誰[だれ]もが 親[した]しみを 感[かん]じます。	
\\	ここの水族館にはイルカがいます。	水族館	ここの 水族館[すいぞくかん]にはイルカがいます。	
\\	デパートの歳末大売出しが始まったよ。	歳末	デパートの 歳末[さいまつ] 大売出[おおうりだ]しが 始[はじ]まったよ。	
\\	うちは3人姉妹です。	姉妹	うちは 3人[さんにん] 姉妹[しまい]です。	
\\	彼女はインテリで、しかも美人ですね。	インテリ	彼女[かのじょ]はインテリで、しかも 美人[びじん]ですね。	
\\	その事件は次第に忘れられていったのさ。	次第に	その 事件[じけん]は 次第[しだい]に 忘[わす]れられていったのさ。	
\\	連絡があり次第出発します。	次第	連絡[れんらく]があり 次第[しだい] 出発[しゅっぱつ]します。	
\\	大きく息を吸ってください。	息	大[おお]きく 息[いき]を 吸[す]ってください。	
\\	休息をとることは大切です。	休息	休息[きゅうそく]をとることは 大切[たいせつ]です。	
\\	この法律は消費者を守るためのものです。	消費者	この 法律[ほうりつ]は 消費者[しょうひしゃ]を 守[まも]るためのものです。	
\\	彼は作家であり学者です。	学者	彼[かれ]は 作家[さっか]であり 学者[がくしゃ]です。	
\\	正しい答は後者です。	後者	正[ただ]しい 答[こたえ]は 後者[こうしゃ]です。	
\\	この本の作者は誰ですか。	作者	この 本[ほん]の 作者[さくしゃ]は 誰[だれ]ですか。	
\\	彼はいちいち私に指図する。	いちいち	彼[かれ]はいちいち 私[わたし]に 指図[さしず]する。	
\\	私は前者の方が優れていると思います。	前者	私[わたし]は 前者[ぜんしゃ]の 方[ほう]が 優[すぐ]れていると 思[おも]います。	
\\	彼は結局何を言いたかったのだろう。	結局	彼[かれ]は 結局[けっきょく] 何[なに]を 言[い]いたかったのだろう。	
\\	妹夫婦は新婚です。	新婚	妹夫婦[いもうとふうふ]は 新婚[しんこん]です。	
\\	二人は婚約しています。	婚約	二人[ふたり]は 婚約[こんやく]しています。	
\\	彼は課長に昇進しました。	課長	彼[かれ]は 課長[かちょう]に 昇進[しょうしん]しました。	
\\	この薬はあまりよく効かないよ。	効く	この 薬[くすり]はあまりよく 効[き]かないよ。	
\\	その薬は効力が強いよ。	効力	その 薬[くすり]は 効力[こうりょく]が 強[つよ]いよ。	
\\	従業員は明日からストライキです。	ストライキ	従業員[じゅうぎょういん]は 明日[あす]からストライキです。	
\\	この薬の効き目は素晴らしいの。	効き目	この 薬[くすり]の 効[き]き 目[め]は 素晴[すば]らしいの。	
\\	このドアは自動よ。	自動	このドアは 自動[じどう]よ。	
\\	自国の文化を大切にしましょう。	自国	自国[じこく]の 文化[ぶんか]を 大切[たいせつ]にしましょう。	
\\	ごみは各自で持ち帰ってください。	各自	ごみは 各自[かくじ]で 持[も]ち 帰[かえ]ってください。	
\\	硬くならないで、自然に話して下さい。	自然に	硬[かた]くならないで、 自然[しぜん]に 話[はな]して 下[くだ]さい。	
\\	メールを経由して広がるウィルスもあります。	経由	メールを 経由[けいゆ]して 広[ひろ]がるウィルスもあります。	
\\	あの人の顔には自信があふれているね。	自信	あの 人[ひと]の 顔[かお]には 自信[じしん]があふれているね。	
\\	これはあんまり好きじゃないな。	あんまり	これはあんまり 好[す]きじゃないな。	
\\	彼の言うことは信用できない。	信用	彼[かれ]の 言[い]うことは 信用[しんよう]できない。	
\\	私は部下を信頼しています。	信頼	私[わたし]は 部下[ぶか]を 信頼[しんらい]しています。	
\\	彼に協力を依頼しました。	依頼	彼[かれ]に 協力[きょうりょく]を 依頼[いらい]しました。	
\\	彼は人民のための政治を行った。	人民	彼[かれ]は 人民[じんみん]のための 政治[せいじ]を 行[おこな]った。	
\\	国民の安全が最も大切です。	国民	国民[こくみん]の 安全[あんぜん]が 最[もっと]も 大切[たいせつ]です。	
\\	産業はインドの主要産業よね。	主要	
\\	産業[あいてぃー さんぎょう]はインドの 主要[しゅよう] 産業[さんぎょう]よね。	
\\	この商品は主に女性に人気がありまして。	主に	この 商品[しょうひん]は 主[おも]に 女性[じょせい]に 人気[にんき]がありまして。	
\\	彼は学校の劇で主役に選ばれたよ。	主役	彼[かれ]は 学校[がっこう]の 劇[げき]で 主役[しゅやく]に 選[えら]ばれたよ。	
\\	あまりうるさく言わないで下さい。	うるさい	あまりうるさく 言[い]わないで 下[くだ]さい。	
\\	政治は国民主体であるべきです。	主体	政治[せいじ]は 国民[こくみん] 主体[しゅたい]であるべきです。	
\\	この小説の主題は反戦です。	主題	この 小説[しょうせつ]の 主題[しゅだい]は 反戦[はんせん]です。	
\\	その選手は自主トレーニングを始めたの。	自主	その 選手[せんしゅ]は 自主[じしゅ]トレーニングを 始[はじ]めたの。	
\\	アジアの主食は米です。	主食	アジアの 主食[しゅしょく]は 米[こめ]です。	
\\	彼女の主な仕事は接客だよ。	主	彼女[かのじょ]の 主[おも]な 仕事[しごと]は 接客[せっきゃく]だよ。	
\\	私は車は持たない主義です。	主義	私[わたし]は 車[くるま]は 持[も]たない 主義[しゅぎ]です。	
\\	その国は共産主義の国だよ。	共産主義	その 国[くに]は 共産主義[きょうさんしゅぎ]の 国[くに]だよ。	
\\	このカーブは気をつけて。	カーブ	このカーブは 気[き]をつけて。	
\\	彼は義理を大切にしています。	義理	彼[かれ]は 義理[ぎり]を 大切[たいせつ]にしています。	
\\	議会で区の来年の予定が話し合われたんだ。	議会	議会[ぎかい]で 区[く]の 来年[らいねん]の 予定[よてい]が 話[はな]し 合[あ]われたんだ。	
\\	委員会はその問題について協議しましたわ。	協議	委員会[いいんかい]はその 問題[もんだい]について 協議[きょうぎ]しましたわ。	
\\	議長が開会の挨拶をしました。	議長	議長[ぎちょう]が 開会[かいかい]の 挨拶[あいさつ]をしました。	
\\	今日の議題は面白そうです。	議題	今日[きょう]の 議題[ぎだい]は 面白[おもしろ]そうです。	
\\	父は議論好きです。	議論	父[ちち]は 議論[ぎろん] 好[ず]きです。	
\\	今日の話し合いでは結論が出なかった。	結論	今日[きょう]の 話[はな]し 合[あ]いでは 結論[けつろん]が 出[で]なかった。	
\\	ガイドつきの旅行を申し込んだの。	ガイド	ガイドつきの 旅行[りょこう]を 申[もう]し 込[こ]んだの。	
\\	言論の自由は国民の権利です。	言論	言論[げんろん]の 自由[じゆう]は 国民[こくみん]の 権利[けんり]です。	
\\	世論はあなたの味方です。	世論	世論[せろん]はあなたの 味方[みかた]です。	
\\	彼は校則に違反しました。	違反	彼[かれ]は 校則[こうそく]に 違反[いはん]しました。	
\\	私は彼の要望に応じました。	応じる	私[わたし]は 彼[かれ]の 要望[ようぼう]に 応[おう]じました。	
\\	このレシピはいろいろ応用できます。	応用	このレシピはいろいろ 応用[おうよう]できます。	
\\	彼にも一応知らせておいたほうがいいだろう。	一応	彼[かれ]にも 一応[いちおう] 知[し]らせておいたほうがいいだろう。	
\\	アンケートに回答しました。	回答	アンケートに 回答[かいとう]しました。	
\\	田んぼの中でかえるが合唱している。	かえる	田[た]んぼの 中[なか]でかえるが 合唱[がっしょう]している。	
\\	インターネットで試験の解答を確認したんだ。	解答	インターネットで 試験[しけん]の 解答[かいとう]を 確認[かくにん]したんだ。	
\\	あの二人はそっくりで区別できない。	区別	あの 二人[ふたり]はそっくりで 区別[くべつ]できない。	
\\	彼は差別をなくす運動をしています。	差別	彼[かれ]は 差別[さべつ]をなくす 運動[うんどう]をしています。	
\\	出席者を性別で分けてください。	性別	出席者[しゅっせきしゃ]を 性別[せいべつ]で 分[わ]けてください。	
\\	一般に老人は早起きです。	一般に	一般[いっぱん]に 老人[ろうじん]は 早起[はやお]きです。	
\\	経済的な効果は計り知れません。	経済的	経済的[けいざいてき]な 効果[こうか]は 計[はか]り 知[し]れません。	
\\	学生たちは自主的に勉強会を開いたわよ。	自主的	学生[がくせい]たちは 自主的[じしゅてき]に 勉強会[べんきょうかい]を 開[ひら]いたわよ。	
\\	彼女は合理的な考え方をする人です。	合理的	彼女[かのじょ]は 合理的[ごうりてき]な 考[かんが]え 方[かた]をする 人[ひと]です。	
\\	ポスターの位置が少しずれているよ。	ずれる	ポスターの 位置[いち]が 少[すこ]しずれているよ。	
\\	彼は形式的な質問をしただけだった。	形式的	彼[かれ]は 形式的[けいしきてき]な 質問[しつもん]をしただけだった。	
\\	彼の発明品はみな実用的だよ。	実用的	彼[かれ]の 発明品[はつめいひん]はみな 実用的[じつようてき]だよ。	
\\	彼は言葉遣いが少し女性的だね。	女性的	彼[かれ]は 言葉遣[ことばづか]いが 少[すこ]し 女性的[じょせいてき]だね。	
\\	この棚は水平になっていませんね。	水平	この 棚[たな]は 水平[すいへい]になっていませんね。	
\\	水平線に太陽が沈んでいった。	水平線	水平線[すいへいせん]に 太陽[たいよう]が 沈[しず]んでいった。	
\\	人間は高等な生物と言われている。	高等	人間[にんげん]は 高等[こうとう]な 生物[せいぶつ]と 言[い]われている。	
\\	上等なワインを飲んだの。	上等	上等[じょうとう]なワインを 飲[の]んだの。	
\\	彼の話はいつもオーバーです。	オーバー	彼[かれ]の 話[はなし]はいつもオーバーです。	
\\	知り合いが急病で倒れた。	急病	知[し]り 合[あ]いが 急病[きゅうびょう]で 倒[たお]れた。	
\\	彼は医学を学んでいます。	医学	彼[かれ]は 医学[いがく]を 学[まな]んでいます。	
\\	彼はとても有名な科学者だ。	科学者	彼[かれ]はとても 有名[ゆうめい]な 科学者[かがくしゃ]だ。	
\\	今日は3科目のテストを受けたよ。	科目	今日[きょう]は 3[さん] 科目[かもく]のテストを 受[う]けたよ。	
\\	彼の得意な学科は数学です。	学科	彼[かれ]の 得意[とくい]な 学科[がっか]は 数学[すうがく]です。	
\\	友人が骨折して外科に入院しました。	外科	友人[ゆうじん]が 骨折[こっせつ]して 外科[げか]に 入院[にゅういん]しました。	
\\	彼は歯科医師です。	歯科	彼[かれ]は 歯科[しか] 医師[いし]です。	
\\	部屋にエアコンを取り付けたんだ。	エアコン	部屋[へや]にエアコンを 取[と]り 付[つ]けたんだ。	
\\	姉は大学で自然科学を学んでいます。	自然科学	姉[あね]は 大学[だいがく]で 自然科学[しぜんかがく]を 学[まな]んでいます。	
\\	大学で人文科学を専攻しました。	人文科学	大学[だいがく]で 人文科学[じんぶんかがく]を 専攻[せんこう]しました。	
\\	社会科学の分野に関心があります。	社会科学	社会科学[しゃかいかがく]の 分野[ぶんや]に 関心[かんしん]があります。	
\\	死を恐れるのは自然なことです。	死	死[し]を 恐[おそ]れるのは 自然[しぜん]なことです。	
\\	その事故で30人の死者が出たの。	死者	その 事故[じこ]で 30人[さんじゅうにん]の 死者[ししゃ]が 出[で]たの。	
\\	公園で死体が見つかったの。	死体	公園[こうえん]で 死体[したい]が 見[み]つかったの。	
\\	これは人の生死にかかわる問題です。	生死	これは 人[ひと]の 生死[せいし]にかかわる 問題[もんだい]です。	
\\	その事故では2人死亡したの。	死亡	その 事故[じこ]では 2人[ふたり] 死亡[しぼう]したの。	
\\	彼はしょっちゅう遅刻しているな。	しょっちゅう	彼[かれ]はしょっちゅう 遅刻[ちこく]しているな。	
\\	背中に痛みがあります。	痛み	背中[せなか]に 痛[いた]みがあります。	
\\	虫歯がずきずき痛みます。	痛む	虫歯[むしば]がずきずき 痛[いた]みます。	
\\	交差点付近は駐車禁止です。	禁止	交差点付近[こうさてん ふきん]は 駐車[ちゅうしゃ] 禁止[きんし]です。	
\\	市は昨年から歩きタバコを禁じているの。	禁じる	市[し]は 昨年[さくねん]から 歩[ある]きタバコを 禁[きん]じているの。	
\\	煙突から煙がまっすぐ上っていたの。	煙	煙突[えんとつ]から 煙[けむり]がまっすぐ 上[あが]っていたの。	
\\	そこの酒屋さんでビールを買って来て。	酒屋	そこの 酒屋[さかや]さんでビールを 買[か]って 来[き]て。	
\\	彼は禁酒しています。	禁酒	彼[かれ]は 禁酒[きんしゅ]しています。	
\\	姉はスタンドの下で本を読んでいたんだ。	スタンド	姉[あね]はスタンドの 下[した]で 本[ほん]を 読[よ]んでいたんだ。	
\\	父の表情が険しくなったの。	険しい	父[ちち]の 表情[ひょうじょう]が 険[けわ]しくなったの。	
\\	証券を売って資金にしようと思うの。	証券	証券[しょうけん]を 売[う]って 資金[しきん]にしようと 思[おも]うの。	
\\	容疑者のアリバイが証明されました。	証明	容疑者[ようぎしゃ]のアリバイが 証明[しょうめい]されました。	
\\	乗客は全員生存しています。	生存	乗客[じょうきゃく]は 全員[ぜんいん] 生存[せいぞん]しています。	
\\	彼の名前をご存じですか。	ご存じ	彼[かれ]の 名前[なまえ]をご 存[ぞん]じですか。	
\\	両社が合併に合意しました。	合意	両社[りょうしゃ]が 合併[がっぺい]に 合意[ごうい]しました。	
\\	彼の意向を聞いてみましょう。	意向	彼[かれ]の 意向[いこう]を 聞[き]いてみましょう。	
\\	成績が落ちてがっかりした。	がっかり	成績[せいせき]が 落[お]ちてがっかりした。	
\\	彼は仕事に意欲を燃やしています。	意欲	彼[かれ]は 仕事[しごと]に 意欲[いよく]を 燃[も]やしています。	
\\	彼の決意は堅いな。	決意	彼[かれ]の 決意[けつい]は 堅[かた]いな。	
\\	あなたの意図はよく分かりました。	意図	あなたの 意図[いと]はよく 分[わ]かりました。	
\\	意外にも彼は独身です。	意外	意外[いがい]にも 彼[かれ]は 独身[どくしん]です。	
\\	この事業には大きな意義があります。	意義	この 事業[じぎょう]には 大[おお]きな 意義[いぎ]があります。	
\\	彼は彼女に好意をもっています。	好意	彼[かれ]は 彼女[かのじょ]に 好意[こうい]をもっています。	
\\	彼は時々意地悪な質問をする。	意地悪	彼[かれ]は 時々[ときどき] 意地悪[いじわる]な 質問[しつもん]をする。	
\\	優れた人材の確保は重要です。	確保	優[すぐ]れた 人材[じんざい]の 確保[かくほ]は 重要[じゅうよう]です。	
\\	このふたはスライドします。	スライド	このふたはスライドします。	
\\	北欧では社会保障が確立されています。	確立	北欧[ほくおう]では 社会保障[しゃかい ほしょう]が 確立[かくりつ]されています。	
\\	試合の代表メンバーが確定したの。	確定	試合[しあい]の 代表[だいひょう]メンバーが 確定[かくてい]したの。	
\\	彼女が将来、大統領になることは確実です。	確実	彼女[かのじょ]が 将来[しょうらい]、 大統領[だいとうりょう]になることは 確実[かくじつ]です。	
\\	私は彼の成功を確信しています。	確信	私[わたし]は 彼[かれ]の 成功[せいこう]を 確信[かくしん]しています。	
\\	作業の効率を上げるにはどうしたらいいですか。	効率	作業[さぎょう]の 効率[こうりつ]を 上[あ]げるにはどうしたらいいですか。	
\\	天気予報は当たる確率が高くなりましたね。	確率	天気予報[てんき よほう]は 当[あ]たる 確率[かくりつ]が 高[たか]くなりましたね。	
\\	軽率な発言はしないよう気をつけなさい。	軽率	軽率[けいそつ]な 発言[はつげん]はしないよう 気[き]をつけなさい。	
\\	あなたのおっしゃる通りです。	おっしゃる	あなたのおっしゃる 通[とお]りです。	
\\	家族で話し合う機会を持ちました。	機会	家族[かぞく]で 話[はな]し 合[あ]う 機会[きかい]を 持[も]ちました。	
\\	地球環境の危機が叫ばれているのよ。	危機	地球環境[ちきゅう かんきょう]の 危機[きき]が 叫[さけ]ばれているのよ。	
\\	機長の放送があったの。	機長	機長[きちょう]の 放送[ほうそう]があったの。	
\\	あせらずに時機を待つべきです。	時機	あせらずに 時機[じき]を 待[ま]つべきです。	
\\	私が好きな楽器はギターです。	楽器	私[わたし]が 好[す]きな 楽器[がっき]はギターです。	
\\	食器を全部新しくしました。	食器	食器[しょっき]を 全部新[ぜんぶ あたら]しくしました。	
\\	彼はかなり器用な人です。	器用	彼[かれ]はかなり 器用[きよう]な 人[ひと]です。	
\\	彼女はクラシック音楽が好きです。	クラシック	彼女[かのじょ]はクラシック 音楽[おんがく]が 好[す]きです。	
\\	受話器を取ってもらえますか。	受話器	受話器[じゅわき]を 取[と]ってもらえますか。	
\\	この器は上等ね。	器	この 器[うつわ]は 上等[じょうとう]ね。	
\\	消火器を交換しました。	消火器	消火器[しょうかき]を 交換[こうかん]しました。	
\\	体育館で器械を使って運動したの。	器械	体育館[たいいくかん]で 器械[きかい]を 使[つか]って 運動[うんどう]したの。	
\\	テレビも取材に来たほど有名なレストランです。	取材	テレビも 取材[しゅざい]に 来[き]たほど 有名[ゆうめい]なレストランです。	
\\	船から材木が降ろされていますね。	材木	船[ふね]から 材木[ざいもく]が 降[お]ろされていますね。	
\\	今日は体の具合が悪いです。	具合	今日[きょう]は 体[からだ]の 具合[ぐあい]が 悪[わる]いです。	
\\	こぼれた牛乳をさっと拭き取ったの。	さっと	こぼれた 牛乳[ぎゅうにゅう]をさっと 拭[ふ]き 取[と]ったの。	
\\	これはスポーツ施設用の器具です。	器具	これはスポーツ 施設用[しせつよう]の 器具[きぐ]です。	
\\	部屋の家具を動かしました。	家具	部屋[へや]の 家具[かぐ]を 動[うご]かしました。	
\\	雨具の用意を忘れないように。	雨具	雨具[あまぐ]の 用意[ようい]を 忘[わす]れないように。	
\\	ここは昔、軍事基地でした。	基地	ここは 昔[むかし]、 軍事[ぐんじ] 基地[きち]でした。	
\\	今年の応募作品は水準が高かったね。	水準	今年[ことし]の 応募作品[おうぼ さくひん]は 水準[すいじゅん]が 高[たか]かったね。	
\\	この建物は建築の基準に達していないよ。	基準	この 建物[たてもの]は 建築[けんちく]の 基準[きじゅん]に 達[たっ]していないよ。	
\\	新宿駅まで準急で行ったのよ。	準急	新宿駅[しんじゅくえき]まで 準急[じゅんきゅう]で 行[い]ったのよ。	
\\	当社は10年前に設立されたのよ。	設立	当社[とうしゃ]は 10年前[じゅうねんまえ]に 設立[せつりつ]されたのよ。	
\\	予約をキャンセルしたよ。	キャンセル	予約[よやく]をキャンセルしたよ。	
\\	この研究所には最新の設備が揃っています。	設備	この 研究所[けんきゅうじょ]には 最新[さいしん]の 設備[せつび]が 揃[そろ]っています。	
\\	私は彼の説が正しいと思う。	説	私[わたし]は 彼[かれ]の 説[せつ]が 正[ただ]しいと 思[おも]う。	
\\	新聞の解説欄を読みました。	解説	新聞[しんぶん]の 解説[かいせつ] 欄[らん]を 読[よ]みました。	
\\	この新聞の社説は面白いね。	社説	この 新聞[しんぶん]の 社説[しゃせつ]は 面白[おもしろ]いね。	
\\	それは最新の学説ね。	学説	それは 最新[さいしん]の 学説[がくせつ]ね。	
\\	今後の方針が公式に発表されました。	公式	今後[こんご]の 方針[ほうしん]が 公式[こうしき]に 発表[はっぴょう]されました。	
\\	公共の乗り物は誰でも利用できます。	公共	公共[こうきょう]の 乗[の]り 物[もの]は 誰[だれ]でも 利用[りよう]できます。	
\\	メイドがシーツを交換してくれたね。	シーツ	メイドがシーツを 交換[こうかん]してくれたね。	
\\	賞金をもらったら公平に分けましょう。	公平	賞金[しょうきん]をもらったら 公平[こうへい]に 分[わ]けましょう。	
\\	駅前に公立の図書館があるよ。	公立	駅前[えきまえ]に 公立[こうりつ]の 図書館[としょかん]があるよ。	
\\	彼女は郊外の学園に通っているね。	学園	彼女[かのじょ]は 郊外[こうがい]の 学園[がくえん]に 通[かよ]っているね。	
\\	うちの会社は祭日は休みです。	祭日	うちの 会社[かいしゃ]は 祭日[さいじつ]は 休[やす]みです。	
\\	彼女は国際的に有名な歌手です。	国際的	彼女[かのじょ]は 国際的[こくさいてき]に 有名[ゆうめい]な 歌手[かしゅ]です。	
\\	この大学も国際化してきたな。	国際化	この 大学[だいがく]も 国際化[こくさいか]してきたな。	
\\	彼はモデルと交際していたんだ。	交際	彼[かれ]はモデルと 交際[こうさい]していたんだ。	
\\	彼女はインフルエンザにかかっていますね。	かかる	彼女[かのじょ]はインフルエンザにかかっていますね。	
\\	この際はっきり言っておきます。	際	この 際[さい]はっきり 言[い]っておきます。	
\\	会議で航空の安全について話し合ったの。	航空	会議[かいぎ]で 航空[こうくう]の 安全[あんぜん]について 話[はな]し 合[あ]ったの。	
\\	去年は航空機の事故が多かった。	航空機	去年[きょねん]は 航空機[こうくうき]の 事故[じこ]が 多[おお]かった。	
\\	電話で航空券を予約しました。	航空券	電話[でんわ]で 航空券[こうくうけん]を 予約[よやく]しました。	
\\	この古い寺院は今でも完全な形を保っているんだ。	完全	この 古[ふる]い 寺院[じいん]は 今[いま]でも 完全[かんぜん]な 形[かたち]を 保[たも]っているんだ。	
\\	庭の木、大きく成長したわね。	成長	庭[にわ]の 木[き]、 大[おお]きく 成長[せいちょう]したわね。	
\\	あの会社との契約が成立しました。	成立	あの 会社[かいしゃ]との 契約[けいやく]が 成立[せいりつ]しました。	
\\	今は骨が形成される大切な時期です。	形成	今[いま]は 骨[ほね]が 形成[けいせい]される 大切[たいせつ]な 時期[じき]です。	
\\	彼は腕のいいコックです。	コック	彼[かれ]は 腕[うで]のいいコックです。	
\\	新しい代表チームが結成されました。	結成	新[あたら]しい 代表[だいひょう]チームが 結成[けっせい]されました。	
\\	厳しい練習が良い成果に結び付きました。	成果	厳[きび]しい 練習[れんしゅう]が 良[よ]い 成果[せいか]に 結[むす]び 付[つ]きました。	
\\	私は彼の意見に賛成だ。	賛成	私[わたし]は 彼[かれ]の 意見[いけん]に 賛成[さんせい]だ。	
\\	このソフトで画像を合成できます。	合成	このソフトで 画像[がぞう]を 合成[ごうせい]できます。	
\\	娘が今年成人します。	成人	娘[むすめ]が 今年[ことし] 成人[せいじん]します。	
\\	成年になると独立した戸籍を作れます。	成年	成年[せいねん]になると 独立[どくりつ]した 戸籍[こせき]を 作[つく]れます。	
\\	彼は地震で親を失いました。	失う	彼[かれ]は 地震[じしん]で 親[おや]を 失[うしな]いました。	
\\	私はえびのてんぷらが好きです。	えび	私[わたし]はえびのてんぷらが 好[す]きです。	
\\	友達のお父さんが突然失業したの。	失業	友達[ともだち]のお 父[とう]さんが 突然[とつぜん] 失業[しつぎょう]したの。	
\\	一度の失敗であきらめてはいけないよ。	失敗	一度[いちど]の 失敗[しっぱい]であきらめてはいけないよ。	
\\	このセットで勝敗が決まります。	勝敗	このセットで 勝敗[しょうはい]が 決[き]まります。	
\\	原則としてキャンセルできません。	原則	原則[げんそく]としてキャンセルできません。	
\\	石油はいろいろな製品の原料になります。	原料	石油[せきゆ]はいろいろな 製品[せいひん]の 原料[げんりょう]になります。	
\\	見学者に機械の動く原理を説明しました。	原理	見学者[けんがくしゃ]に 機械[きかい]の 動[うご]く 原理[げんり]を 説明[せつめい]しました。	
\\	原子力の安全な利用について考えましょう。	原子力	原子力[げんしりょく]の 安全[あんぜん]な 利用[りよう]について 考[かんが]えましょう。	
\\	そのグリーンのスカーフ、素敵ですね。	グリーン	そのグリーンのスカーフ、 素敵[すてき]ですね。	
\\	高原には気持ちのいい風が吹いていたよ。	高原	高原[こうげん]には 気持[きも]ちのいい 風[かぜ]が 吹[ふ]いていたよ。	
\\	図書館で資料を借りて来ました。	資料	図書館[としょかん]で 資料[しりょう]を 借[か]りて 来[き]ました。	
\\	大臣の資産が公表されたね。	資産	大臣[だいじん]の 資産[しさん]が 公表[こうひょう]されたね。	
\\	日本語を教える資格を取りたいです。	資格	日本語[にほんご]を 教[おし]える 資格[しかく]を 取[と]りたいです。	
\\	その国は次第に資本主義になったわね。	資本主義	その 国[くに]は 次第[しだい]に 資本主義[しほんしゅぎ]になったわね。	
\\	地球の資源を守りましょう。	資源	地球[ちきゅう]の 資源[しげん]を 守[まも]りましょう。	
\\	今日大学に願書を送りました。	願書	今日大学[きょう だいがく]に 願書[がんしょ]を 送[おく]りました。	
\\	正式な招待状を受け取りました。	正式	正式[せいしき]な 招待状[しょうたいじょう]を 受[う]け 取[と]りました。	
\\	彼はいつもジーパンをはいているね。	ジーパン	彼[かれ]はいつもジーパンをはいているね。	
\\	その家の正面には大きなバルコニーがあるの。	正面	その 家[いえ]の 正面[しょうめん]には 大[おお]きなバルコニーがあるの。	
\\	昼休みは正午からです。	正午	昼休[ひるやす]みは 正午[しょうご]からです。	
\\	この世に正義はないのだろうか。	正義	この 世[よ]に 正義[せいぎ]はないのだろうか。	
\\	受験生は正門から入って下さい。	正門	受験生[じゅけんせい]は 正門[せいもん]から 入[はい]って 下[くだ]さい。	
\\	10問中9問正解しました。	正解	10問中9問[じゅうもんちゅう きゅうもん] 正解[せいかい]しました。	
\\	正方形の紙を用意しましょう。	正方形	正方形[せいほうけい]の 紙[かみ]を 用意[ようい]しましょう。	
\\	書類は正と副の2通あります。	正	書類[しょるい]は 正[せい]と 副[ふく]の 2通[につう]あります。	
\\	彼はジャーナリズムを専攻している。	ジャーナリズム	彼[かれ]はジャーナリズムを 専攻[せんこう]している。	
\\	彼はきちんと正座して待っていたね。	正座	彼[かれ]はきちんと 正座[せいざ]して 待[ま]っていたね。	
\\	これは正当な要求です。	正当	これは 正当[せいとう]な 要求[ようきゅう]です。	
\\	息子はもう異性を意識している。	異性	息子[むすこ]はもう 異性[いせい]を 意識[いしき]している。	
\\	今年の夏は異常な暑さですね。	異常	今年[ことし]の 夏[なつ]は 異常[いじょう]な 暑[あつ]さですね。	
\\	患者の呼吸は正常です。	正常	患者[かんじゃ]の 呼吸[こきゅう]は 正常[せいじょう]です。	
\\	彼は意識を失いました。	意識	彼[かれ]は 意識[いしき]を 失[うしな]いました。	
\\	そんなの常識だよ。	常識	そんなの 常識[じょうしき]だよ。	
\\	彼のスピーチは素晴らしかった。	スピーチ	彼[かれ]のスピーチは 素晴[すば]らしかった。	
\\	警察の調べで女性の身元がわかりました。	調べ	警察[けいさつ]の 調[しら]べで 女性[じょせい]の 身元[みもと]がわかりました。	
\\	今月はエアコンの売り上げが好調です。	好調	今月[こんげつ]はエアコンの 売[う]り 上[あ]げが 好調[こうちょう]です。	
\\	まず第一に、しっかり下調べをしなさい。	下調べ	まず 第一[だいいち]に、しっかり 下調[したしら]べをしなさい。	
\\	車は整備に出しています。	整備	車[くるま]は 整備[せいび]に 出[だ]しています。	
\\	古い服を整理しました。	整理	古[ふる]い 服[ふく]を 整理[せいり]しました。	
\\	電気や水を節約しましょう。	節約	電気[でんき]や 水[みず]を 節約[せつやく]しましょう。	
\\	私は今日、目の検査を受けます。	検査	私[わたし]は 今日[きょう]、 目[め]の 検査[けんさ]を 受[う]けます。	
\\	ああうるさい人は苦手です。	ああ	ああうるさい 人[ひと]は 苦手[にがて]です。	
\\	もっと案を出し合いましょう。	案	もっと 案[あん]を 出[だ]し 合[あ]いましょう。	
\\	彼は案外いい人かも知れない。	案外	彼[かれ]は 案外[あんがい]いい 人[ひと]かも 知[し]れない。	
\\	案の定、彼は遅刻したな。	案の定	案[あん]の 定[じょう]、 彼[かれ]は 遅刻[ちこく]したな。	
\\	国連の本部はニューヨークにあります。	国連	国連[こくれん]の 本部[ほんぶ]はニューヨークにあります。	
\\	コンピューターをネットワークに接続しました。	接続	コンピューターをネットワークに 接続[せつぞく]しました。	
\\	外相は来週訪米の予定です。	外相	外相[がいしょう]は 来週訪米[らいしゅう ほうべい]の 予定[よてい]です。	
\\	最近、事件の真相が明らかになったよ。	真相	最近[さいきん]、 事件[じけん]の 真相[しんそう]が 明[あき]らかになったよ。	
\\	彼は相変わらず忙しいですね。	相変わらず	彼[かれ]は 相変[あいか]わらず 忙[いそが]しいですね。	
\\	アメリカ英語とイギリス英語ではアクセントが違うことがあるね。	アクセント	アメリカ 英語[えいご]とイギリス 英語[えいご]ではアクセントが 違[ちが]うことがあるね。	
\\	その先生はいつも授業の前に雑談をするの。	雑談	その 先生[せんせい]はいつも 授業[じゅぎょう]の 前[まえ]に 雑談[ざつだん]をするの。	
\\	環境問題に関する記事を読んだの。	記事	環境問題[かんきょう もんだい]に 関[かん]する 記事[きじ]を 読[よ]んだの。	
\\	地図にはいろいろな記号が使われているのね。	記号	地図[ちず]にはいろいろな 記号[きごう]が 使[つか]われているのね。	
\\	こちらにお名前をご記入ください。	記入	こちらにお 名前[なまえ]をご 記入[きにゅう]ください。	
\\	試験の前に英文を暗記したんだ。	暗記	試験[しけん]の 前[まえ]に 英文[えいぶん]を 暗記[あんき]したんだ。	
\\	当時のことはしっかり記憶しているよ。	記憶	当時[とうじ]のことはしっかり 記憶[きおく]しているよ。	
\\	彼は政治に関心が強いね。	関心	彼[かれ]は 政治[せいじ]に 関心[かんしん]が 強[つよ]いね。	
\\	彼女はテレビのコマーシャルに出ているわ。	コマーシャル	彼女[かのじょ]はテレビのコマーシャルに 出[で]ているわ。	
\\	毛皮を税関で没収されたんだ。	税関	毛皮[けがわ]を 税関[ぜいかん]で 没収[ぼっしゅう]されたんだ。	
\\	手首の関節をひねっちゃった。	関節	手首[てくび]の 関節[かんせつ]をひねっちゃった。	
\\	医師は人の命に関わる大切な職業だよ。	関わる	医師[いし]は 人[ひと]の 命[いのち]に 関[かか]わる 大切[たいせつ]な 職業[しょくぎょう]だよ。	
\\	台風で交通機関がストップしている。	機関	台風[たいふう]で 交通[こうつう] 機関[きかん]がストップしている。	
\\	彼女は会場整理の係だったの。	係	彼女[かのじょ]は 会場整理[かいじょう せいり]の 係[かかり]だったの。	
\\	問題を解決できないのが現状です。	現状	問題[もんだい]を 解決[かいけつ]できないのが 現状[げんじょう]です。	
\\	事態は深刻です。	事態	事態[じたい]は 深刻[しんこく]です。	
\\	しつこい迷惑メールに困っている。	しつこい	しつこい 迷惑[めいわく]メールに 困[こま]っている。	
\\	その会社の経営の実態を調査中だ。	実態	その 会社[かいしゃ]の 経営[けいえい]の 実態[じったい]を 調査中[ちょうさちゅう]だ。	
\\	年金問題は行政の最大の課題のひとつよ。	行政	年金問題[ねんきん もんだい]は 行政[ぎょうせい]の 最大[さいだい]の 課題[かだい]のひとつよ。	
\\	大きくなったら政治家になりたいです。	政治家	大[おお]きくなったら 政治家[せいじか]になりたいです。	
\\	咳が少し治まりました。	治まる	咳[せき]が 少[すこ]し 治[おさ]まりました。	
\\	選挙では3つの政党が争っています。	政党	選挙[せんきょ]では 3[みっ]つの 政党[せいとう]が 争[あらそ]っています。	
\\	その問題に対する策を皆で考えたの。	策	その 問題[もんだい]に 対[たい]する 策[さく]を 皆[みんな]で 考[かんが]えたの。	
\\	例を幾つか挙げてみましょう。	挙げる	例[れい]を 幾[いく]つか 挙[あ]げてみましょう。	
\\	日本の気候は温暖です。	気候	日本[にほん]の 気候[きこう]は 温暖[おんだん]です。	
\\	お年寄りが道端にしゃがんでいますね。	しゃがむ	お 年寄[としよ]りが 道端[みちばた]にしゃがんでいますね。	
\\	夏は水分を十分に補いましょう。	補う	夏[なつ]は 水分[すいぶん]を 十分[じゅうぶん]に 補[おぎな]いましょう。	
\\	彼は足首を痛めています。	足首	彼[かれ]は 足首[あしくび]を 痛[いた]めています。	
\\	東京で五カ国の首脳会談が開かれています。	首脳	東京[とうきょう]で 五[ご]カ 国[こく]の 首脳[しゅのう] 会談[かいだん]が 開[ひら]かれています。	
\\	彼は一家のお頭だったの。	頭	彼[かれ]は 一家[いっか]のお 頭[かしら]だったの。	
\\	今日は頭痛がします。	頭痛	今日[きょう]は 頭痛[ずつう]がします。	
\\	彼は顔色がよくありませんね。	顔色	彼[かれ]は 顔色[かおいろ]がよくありませんね。	
\\	紫の朝顔が咲きました。	朝顔	紫[むらさき]の 朝顔[あさがお]が 咲[さ]きました。	
\\	このコンテストに優勝すると車がもらえるんだ。	コンテスト	このコンテストに 優勝[ゆうしょう]すると 車[くるま]がもらえるんだ。	
\\	近く交通法が改正されます。	改正	近[ちか]く 交通法[こうつう ほう]が 改正[かいせい]されます。	
\\	日本では絶えず米の品種を改良しているの。	改良	日本[にほん]では 絶[た]えず 米[こめ]の 品種[ひんしゅ]を 改良[かいりょう]しているの。	
\\	彼は悪い習慣を改めようとしているわね。	改める	彼[かれ]は 悪[わる]い 習慣[しゅうかん]を 改[あらた]めようとしているわね。	
\\	首相は内閣の改造を行いました。	改造	首相[しゅしょう]は 内閣[ないかく]の 改造[かいぞう]を 行[おこな]いました。	
\\	改めてあなたのご意見を聞かせて下さい。	改めて	改[あらた]めてあなたのご 意見[いけん]を 聞[き]かせて 下[くだ]さい。	
\\	年号が改まりました。	改まる	年号[ねんごう]が 改[あらた]まりました。	
\\	それは歴史上の大きな革命です。	革命	それは 歴史上[れきしじょう]の 大[おお]きな 革命[かくめい]です。	
\\	キャベツの千切りを添えたよ。	キャベツ	キャベツの 千切[せんぎ]りを 添[そ]えたよ。	
\\	生命は海から始まったと言われている。	生命	生命[せいめい]は 海[うみ]から 始[はじ]まったと 言[い]われている。	
\\	命より大切なものは無いよ。	命	命[いのち]より 大切[たいせつ]なものは 無[な]いよ。	
\\	運命には逆らえないよ。	運命	運命[うんめい]には 逆[さか]らえないよ。	
\\	今日は組合の集まりがあるわ。	組合	今日[きょう]は 組合[くみあい]の 集[あつ]まりがあるわ。	
\\	この機械の仕組みは複雑だ。	仕組み	この 機械[きかい]の 仕組[しく]みは 複雑[ふくざつ]だ。	
\\	彼は1年2組の生徒です。	組	彼[かれ]は 1年2[いちねん に] 組[くみ]の 生徒[せいと]です。	
\\	このプロジェクトで私は彼と組んでいるんだ。	組む	このプロジェクトで 私[わたし]は 彼[かれ]と 組[く]んでいるんだ。	
\\	いろいろな花を組み合わせ花束を作りました。	組み合わせる	いろいろな 花[はな]を 組[く]み 合[あ]わせ 花束[はなたば]を 作[つく]りました。	
\\	彼女は若者のアイドルです。	アイドル	彼女[かのじょ]は 若者[わかもの]のアイドルです。	
\\	キャンペーンに新しいイベントを組み込みました。	組み込む	キャンペーンに 新[あたら]しいイベントを 組[く]み 込[こ]みました。	
\\	色の組み合わせで印象が変わりますよ。	組み合わせ	色[いろ]の 組[く]み 合[あ]わせで 印象[いんしょう]が 変[か]わりますよ。	
\\	その町は織物業で有名です。	織物	その 町[まち]は 織物[おりもの] 業[ぎょう]で 有名[ゆうめい]です。	
\\	大手チェーン店が進出している。	進出	大手[おおて]チェーン 店[てん]が 進出[しんしゅつ]している。	
\\	学会は予定通りに進行しています。	進行	学会[がっかい]は 予定通[よてい どお]りに 進行[しんこう]しています。	
\\	科学技術は目覚しく進歩しているの。	進歩	科学技術[かがく ぎじゅつ]は 目覚[めざま]しく 進歩[しんぽ]しているの。	
\\	一列に並んで前進して下さい。	前進	一列[いちれつ]に 並[なら]んで 前進[ぜんしん]して 下[くだ]さい。	
\\	彼は時々妹をからかいます。	からかう	彼[かれ]は 時々妹[ときどき いもうと]をからかいます。	
\\	卒業後の進路を迷っています。	進路	卒業後[そつぎょう ご]の 進路[しんろ]を 迷[まよ]っています。	
\\	開会式で選手たちが行進しました。	行進	開会式[かいかいしき]で 選手[せんしゅ]たちが 行進[こうしん]しました。	
\\	その会社はリサイクルを推進しているね。	推進	その 会社[かいしゃ]はリサイクルを 推進[すいしん]しているね。	
\\	彼女は会計主任です。	主任	彼女[かのじょ]は 会計[かいけい] 主任[しゅにん]です。	
\\	社長の辞任が決まりました。	辞任	社長[しゃちょう]の 辞任[じにん]が 決[き]まりました。	
\\	彼はお世辞を言うのが上手い。	お世辞	彼[かれ]はお 世辞[せじ]を 言[い]うのが 上手[うま]い。	
\\	明日、委員会が開かれます。	委員会	明日[あした]、 委員会[いいんかい]が 開[ひら]かれます。	
\\	よく眠ったら気分がすっきりした。	すっきり	よく 眠[ねむ]ったら 気分[きぶん]がすっきりした。	
\\	彼は委員に選ばれました。	委員	彼[かれ]は 委員[いいん]に 選[えら]ばれました。	
\\	毎朝7時5分に出勤します。	出勤	毎朝7時5分[まいあさ しち じ ご ふん]に 出勤[しゅっきん]します。	
\\	彼は勤勉な人です。	勤勉	彼[かれ]は 勤勉[きんべん]な 人[ひと]です。	
\\	私の一日の勤務時間は8時間です。	勤務	私[わたし]の 一日[いちにち]の 勤務[きんむ] 時間[じかん]は 8時間[はちじかん]です。	
\\	労働は国民の義務です。	義務	労働[ろうどう]は 国民[こくみん]の 義務[ぎむ]です。	
\\	事務の経験が3年あります。	事務	事務[じむ]の 経験[けいけん]が 3年[さんねん]あります。	
\\	私の父は公務員です。	公務員	私[わたし]の 父[ちち]は 公務員[こうむいん]です。	
\\	途中であいにく雨が降り出したの。	あいにく	途中[とちゅう]であいにく 雨[あめ]が 降[ふ]り 出[だ]したの。	
\\	新しい事務員が入りました。	事務員	新[あたら]しい 事務員[じむいん]が 入[はい]りました。	
\\	上司の指示に従った。	従う	上司[じょうし]の 指示[しじ]に 従[したが]った。	
\\	会社は従業員の数を増やす予定だ。	従業員	会社[かいしゃ]は 従業員[じゅうぎょういん]の 数[かず]を 増[ふ]やす 予定[よてい]だ。	
\\	彼はその会社と雇用契約を結んだの。	雇用	彼[かれ]はその 会社[かいしゃ]と 雇用[こよう] 契約[けいやく]を 結[むす]んだの。	
\\	彼は営業で実績を上げたんだ。	実績	彼[かれ]は 営業[えいぎょう]で 実績[じっせき]を 上[あ]げたんだ。	
\\	彼の今月の業績は素晴らしいです。	業績	彼[かれ]の 今月[こんげつ]の 業績[ぎょうせき]は 素晴[すば]らしいです。	
\\	求人に多数の応募があった。	応募	求人[きゅうじん]に 多数[たすう]の 応募[おうぼ]があった。	
\\	勉強に集中しなさい。	集中	勉強[べんきょう]に 集中[しゅうちゅう]しなさい。	
\\	おばさん、こんにちは。	おばさん	おばさん、こんにちは。	
\\	7時5分に駅で集合しましょう。	集合	7時5分[しち じ ご ふん]に 駅[えき]で 集合[しゅうごう]しましょう。	
\\	雨で集まりが悪いですね。	集まり	雨[あめ]で 集[あつ]まりが 悪[わる]いですね。	
\\	彼の趣味は切手収集です。	収集	彼[かれ]の 趣味[しゅみ]は 切手[きって] 収集[しゅうしゅう]です。	
\\	コストがこんなに高くては採算が取れません。	採算	コストがこんなに 高[たか]くては 採算[さいさん]が 取[と]れません。	
\\	先生は試験の採点が終わったようね。	採点	先生[せんせい]は 試験[しけん]の 採点[さいてん]が 終[お]わったようね。	
\\	彼の会社は電力を供給しています。	供給	彼[かれ]の 会社[かいしゃ]は 電力[でんりょく]を 供給[きょうきゅう]しています。	
\\	月給は毎月25日に支給されます。	月給	月給[げっきゅう]は 毎月25日[まいつき にじゅうごにち]に 支給[しきゅう]されます。	
\\	猿が高い木にジャンプした。	ジャンプ	猿[さる]が 高[たか]い 木[き]にジャンプした。	
\\	この仕事は時給1000円です。	時給	この 仕事[しごと]は 時給[じきゅう] 1000円[せんえん]です。	
\\	需要が多過ぎて生産が追い付きません。	需要	需要[じゅよう]が 多過[おおす]ぎて 生産[せいさん]が 追[お]い 付[つ]きません。	
\\	彼は新首相に就任しましたね。	就任	彼[かれ]は 新首相[しんしゅしょう]に 就任[しゅうにん]しましたね。	
\\	ここは職員専用の出入り口です。	職員	ここは 職員[しょくいん] 専用[せんよう]の 出入[でい]り 口[ぐち]です。	
\\	自宅から職場まで1時間かかります。	職場	自宅[じたく]から 職場[しょくば]まで 1時間[いちじかん]かかります。	
\\	あなたの職業を教えてください。	職業	あなたの 職業[しょくぎょう]を 教[おし]えてください。	
\\	2国間で条約が結ばれました。	条約	2国間[にこくかん]で 条約[じょうやく]が 結[むす]ばれました。	
\\	あちらこちらで紅葉がきれいですね。	あちらこちら	あちらこちらで 紅葉[こうよう]がきれいですね。	
\\	この本を参考にして下さい。	参考	この 本[ほん]を 参考[さんこう]にして 下[くだ]さい。	
\\	彼は参議院議員です。	参議院	彼[かれ]は 参議院[さんぎいん] 議員[ぎいん]です。	
\\	家族でお寺にお参りに行きました。	お参り	家族[かぞく]でお 寺[てら]にお 参[まい]りに 行[い]きました。	
\\	この参考書はとても役に立つよ。	参考書	この 参考書[さんこうしょ]はとても 役[やく]に 立[た]つよ。	
\\	私たちのチームに彼が加わった。	加わる	私[わたし]たちのチームに 彼[かれ]が 加[くわ]わった。	
\\	いい加減なことを言ってはいけません。	いい加減	いい 加減[かげん]なことを 言[い]ってはいけません。	
\\	警官は怪しい男の後を追ったよ。	追う	警官[けいかん]は 怪[あや]しい 男[おとこ]の 後[あと]を 追[お]ったよ。	
\\	彼は家から追い出された。	追い出す	彼[かれ]は 家[いえ]から 追[お]い 出[だ]された。	
\\	いつのまにか夜が明けていたね。	いつのまにか	いつのまにか 夜[よ]が 明[あ]けていたね。	
\\	あの子は感じの良い子です。	感じ	あの 子[こ]は 感[かん]じの 良[い]い 子[こ]です。	
\\	感情とは複雑なものです。	感情	感情[かんじょう]とは 複雑[ふくざつ]なものです。	
\\	冷えて指の感覚がない。	感覚	冷[ひ]えて 指[ゆび]の 感覚[かんかく]がない。	
\\	感動する映画でした。	感動	感動[かんどう]する 映画[えいが]でした。	
\\	子供が歩き始めたとき、子供の成長を実感した。	実感	子供[こども]が 歩[ある]き 始[はじ]めたとき、 子供[こども]の 成長[せいちょう]を 実感[じっかん]した。	
\\	彼の我慢強さには感心しました。	感心	彼[かれ]の 我慢強[がまんづよ]さには 感心[かんしん]しました。	
\\	人には思想の自由がある。	思想	人[ひと]には 思想[しそう]の 自由[じゆう]がある。	
\\	彼に食事をおごってもらった。	おごる	彼[かれ]に 食事[しょくじ]をおごってもらった。	
\\	ご感想をお聞かせ下さい。	感想	ご 感想[かんそう]をお 聞[き]かせ 下[くだ]さい。	
\\	彼は空想にふけっているの。	空想	彼[かれ]は 空想[くうそう]にふけっているの。	
\\	このフィルムを現像してください。	現像	このフィルムを 現像[げんぞう]してください。	
\\	村では最近、不思議な現象が起きています。	現象	村[むら]では 最近[さいきん]、 不思議[ふしぎ]な 現象[げんしょう]が 起[お]きています。	
\\	テレビで明日の気象情報を確認したよ。	気象	テレビで 明日[あす]の 気象[きしょう] 情報[じょうほう]を 確認[かくにん]したよ。	
\\	猫が障子を破った。	障子	猫[ねこ]が 障子[しょうじ]を 破[やぶ]った。	
\\	検討の結果、案を修正したよ。	修正	検討[けんとう]の 結果[けっか]、 案[あん]を 修正[しゅうせい]したよ。	
\\	今日のおやつはプリンだった。	おやつ	今日[きょう]のおやつはプリンだった。	
\\	足の傷が痛みます。	傷	足[あし]の 傷[きず]が 痛[いた]みます。	
\\	彼は柔道で腰を傷めたんだ。	傷める	彼[かれ]は 柔道[じゅうどう]で 腰[こし]を 傷[いた]めたんだ。	
\\	試合の相手とユニフォームを交換したよ。	交換	試合[しあい]の 相手[あいて]とユニフォームを 交換[こうかん]したよ。	
\\	彼は易しい言葉に言い換えたんだ。	言い換える	彼[かれ]は 易[やさ]しい 言葉[ことば]に 言[い]い 換[か]えたんだ。	
\\	着替えを旅行カバンに詰めたよ。	着替え	着替[きが]えを 旅行[りょこう]カバンに 詰[つ]めたよ。	
\\	家まで交替で荷物を持ちました。	交替	家[いえ]まで 交替[こうたい]で 荷物[にもつ]を 持[も]ちました。	
\\	火災の原因は放火だそうです。	火災	火災[かさい]の 原因[げんいん]は 放火[ほうか]だそうです。	
\\	旅先で思いがけない災難にあいました。	災難	旅先[たびさき]で 思[おも]いがけない 災難[さいなん]にあいました。	
\\	雷がごろごろ鳴っている。	ごろごろ	雷[かみなり]がごろごろ 鳴[な]っている。	
\\	まずは障害を取り除いてから計画を進めよう。	障害	まずは 障害[しょうがい]を 取[と]り 除[のぞ]いてから 計画[けいかく]を 進[すす]めよう。	
\\	地震は自然災害のひとつです。	災害	地震[じしん]は 自然[しぜん] 災害[さいがい]のひとつです。	
\\	私たちは公害を減らすよう努力しています。	公害	私[わたし]たちは 公害[こうがい]を 減[へ]らすよう 努力[どりょく]しています。	
\\	水害でたくさんの人が家を失ったの。	水害	水害[すいがい]でたくさんの 人[ひと]が 家[いえ]を 失[うしな]ったの。	
\\	お酒の飲み過ぎは健康に害があります。	害	お 酒[さけ]の 飲[の]み 過[す]ぎは 健康[けんこう]に 害[がい]があります。	
\\	本にカバーを被せたの。	被せる	本[ほん]にカバーを 被[かぶ]せたの。	
\\	彼女は通りがかりの人に救われたよ。	救う	彼女[かのじょ]は 通[とお]りがかりの 人[ひと]に 救[すく]われたよ。	
\\	ジョギング用のシューズは安くないな。	シューズ	ジョギング 用[よう]のシューズは 安[やす]くないな。	
\\	娘の存在が私の救いでした。	救い	娘[むすめ]の 存在[そんざい]が 私[わたし]の 救[すく]いでした。	
\\	資料は助手に預けておいてください。	助手	資料[しりょう]は 助手[じょしゅ]に 預[あず]けておいてください。	
\\	プールで男の子が救助されました。	救助	プールで 男[おとこ]の 子[こ]が 救助[きゅうじょ]されました。	
\\	彼の支援がなかったらどうなっていたか。	支援	彼[かれ]の 支援[しえん]がなかったらどうなっていたか。	
\\	その国には物資の援助が必要です。	援助	その 国[くに]には 物資[ぶっし]の 援助[えんじょ]が 必要[ひつよう]です。	
\\	大勢が応援に駆けつけてくれたよ。	応援	大勢[おおぜい]が 応援[おうえん]に 駆[か]けつけてくれたよ。	
\\	おじいちゃんにお小遣いをもらったよ。	小遣い	おじいちゃんにお 小遣[こづか]いをもらったよ。	
\\	初めてのキスは海岸でだったの。	キス	初[はじ]めてのキスは 海岸[かいがん]でだったの。	
\\	車に駐車違反の警告を貼られました。	警告	車[くるま]に 駐車違反[ちゅうしゃ いはん]の 警告[けいこく]を 貼[は]られました。	
\\	道で警官に呼び止められた。	警官	道[みち]で 警官[けいかん]に 呼[よ]び 止[と]められた。	
\\	ガス管が爆発しました。	管	ガス 管[かん]が 爆発[ばくはつ]しました。	
\\	彼は大きな過ちを犯している。	犯す	彼[かれ]は 大[おお]きな 過[あやま]ちを 犯[おか]している。	
\\	強盗がカメラに写っていました。	強盗	強盗[ごうとう]がカメラに 写[うつ]っていました。	
\\	犯人は警察に捕まる前に自殺しました。	自殺	犯人[はんにん]は 警察[けいさつ]に 捕[つか]まる 前[まえ]に 自殺[じさつ]しました。	
\\	その殺人事件は白昼に起こったんだ。	殺人	その 殺人[さつじん] 事件[じけん]は 白昼[はくちゅう]に 起[お]こったんだ。	
\\	猫は私の足元をくるくると回ったの。	くるくる	猫[ねこ]は 私[わたし]の 足元[あしもと]をくるくると 回[まわ]ったの。	
\\	その男は彼女のバッグを奪ったぞ。	奪う	その 男[おとこ]は 彼女[かのじょ]のバッグを 奪[うば]ったぞ。	
\\	戦後の日本は混乱していました。	戦後	戦後[せんご]の 日本[にほん]は 混乱[こんらん]していました。	
\\	彼らは次の試合のために作戦を立てたのよ。	作戦	彼[かれ]らは 次[つぎ]の 試合[しあい]のために 作戦[さくせん]を 立[た]てたのよ。	
\\	祖父は戦場に行ったことがあるそうです。	戦場	祖父[そふ]は 戦場[せんじょう]に 行[い]ったことがあるそうです。	
\\	戦前の生活は今と全く違いました。	戦前	戦前[せんぜん]の 生活[せいかつ]は 今[いま]と 全[まった]く 違[ちが]いました。	
\\	祖父は戦死しました。	戦死	祖父[そふ]は 戦死[せんし]しました。	
\\	その地域では争いが絶えません。	争い	その 地域[ちいき]では 争[あらそ]いが 絶[た]えません。	
\\	その2国は資源をめぐって争っています。	争う	その 2国[にこく]は 資源[しげん]をめぐって 争[あらそ]っています。	
\\	また仕事をサボっていますね。	サボる	また 仕事[しごと]をサボっていますね。	
\\	彼は個人競技のスポーツが好きです。	競技	彼[かれ]は 個人[こじん] 競技[きょうぎ]のスポーツが 好[す]きです。	
\\	今日はデパートが混雑していたよ。	混雑	今日[きょう]はデパートが 混雑[こんざつ]していたよ。	
\\	地震の後、町は大混乱だったよ。	混乱	地震[じしん]の 後[あと]、 町[まち]は 大[だい] 混乱[こんらん]だったよ。	
\\	警察がその男を捜しているの。	捜す	警察[けいさつ]がその 男[おとこ]を 捜[さが]しているの。	
\\	索引はたいてい本の後ろについていますよ。	索引	索引[さくいん]はたいてい 本[ほん]の 後[うし]ろについていますよ。	
\\	私の話を落ち着いて聞いてください。	落ち着く	私[わたし]の 話[はなし]を 落[お]ち 着[つ]いて 聞[き]いてください。	
\\	弟は落ち着きが足りません。	落ち着き	弟[おとうと]は 落[お]ち 着[つ]きが 足[た]りません。	
\\	彼女のいちばんの楽しみはショッピングです。	ショッピング	彼女[かのじょ]のいちばんの 楽[たの]しみはショッピングです。	
\\	財布の落とし物を拾いました。	落とし物	財布[さいふ]の 落[お]とし 物[もの]を 拾[ひろ]いました。	
\\	もっと他の町との交流を深めましょう。	交流	もっと 他[ほか]の 町[まち]との 交流[こうりゅう]を 深[ふか]めましょう。	
\\	彼は一流の選手です。	一流	彼[かれ]は 一流[いちりゅう]の 選手[せんしゅ]です。	
\\	上流には滝があります。	上流	上流[じょうりゅう]には 滝[たき]があります。	
\\	下流に小さな滝があるよ。	下流	下流[かりゅう]に 小[ちい]さな 滝[たき]があるよ。	
\\	ここで2つの海流が出合っている。	海流	ここで 2[ふた]つの 海流[かいりゅう]が 出合[であ]っている。	
\\	彼は三流大学を出たが、今は社長だよ。	三流	彼[かれ]は 三流[さんりゅう] 大学[だいがく]を 出[で]たが、 今[いま]は 社長[しゃちょう]だよ。	
\\	彼はいつもスマートに行動するね。	スマート	彼[かれ]はいつもスマートに 行動[こうどう]するね。	
\\	洪水でたくさんの家が流されたの。	洪水	洪水[こうずい]でたくさんの 家[いえ]が 流[なが]されたの。	
\\	大雨で崖が崩れたね。	崩れる	大雨[おおあめ]で 崖[がけ]が 崩[くず]れたね。	
\\	彼女は体調を崩しています。	崩す	彼女[かのじょ]は 体調[たいちょう]を 崩[くず]しています。	
\\	今はほとんどのトイレが水洗ですよ。	水洗	今[いま]はほとんどのトイレが 水洗[すいせん]ですよ。	
\\	母は台所で洗い物をしています。	洗い物	母[はは]は 台所[だいどころ]で 洗[あら]い 物[もの]をしています。	
\\	日本は石油のほとんどを輸入しています。	石油	日本[にっぽん]は 石油[せきゆ]のほとんどを 輸入[ゆにゅう]しています。	
\\	趣味で油絵を描いています。	油絵	趣味[しゅみ]で 油絵[あぶらえ]を 描[か]いています。	
\\	沖にボートが浮かんでいます。	浮かぶ	沖[おき]にボートが 浮[う]かんでいます。	
\\	玄関に靴がずらりと並んでいたな。	ずらり	玄関[げんかん]に 靴[くつ]がずらりと 並[なら]んでいたな。	
\\	お風呂に花を浮かべて入ったの。	浮かべる	お 風呂[ふろ]に 花[はな]を 浮[う]かべて 入[はい]ったの。	
\\	氷は水に浮きます。	浮く	氷[こおり]は 水[みず]に 浮[う]きます。	
\\	彼女はソファーに体を沈めたんだ。	沈める	彼女[かのじょ]はソファーに 体[からだ]を 沈[しず]めたんだ。	
\\	将来はパイロットになりたいです。	将来	将来[しょうらい]はパイロットになりたいです。	
\\	平和は人類の永遠のテーマです。	永遠	平和[へいわ]は 人類[じんるい]の 永遠[えいえん]のテーマです。	
\\	彼は永久に帰らぬ人となったのよ。	永久	彼[かれ]は 永久[えいきゅう]に 帰[かえ]らぬ 人[ひと]となったのよ。	
\\	この川の河口は太平洋に注いでいます。	河口	この 川[かわ]の 河口[かこう]は 太平洋[たいへいよう]に 注[そそ]いでいます。	
\\	お客様がいらっしゃいました。	いらっしゃる	お 客様[きゃくさま]がいらっしゃいました。	
\\	お茶がとても熱かったので冷ましてから飲んだの。	冷ます	お 茶[ちゃ]がとても 熱[あつ]かったので 冷[さ]ましてから 飲[の]んだの。	
\\	車を車庫に入れておきました。	車庫	車[くるま]を 車庫[しゃこ]に 入[い]れておきました。	
\\	ホテルの金庫に貴重品を入れたよ。	金庫	ホテルの 金庫[きんこ]に 貴重品[きちょうひん]を 入[い]れたよ。	
\\	祖母は心臓が悪いんだ。	心臓	祖母[そぼ]は 心臓[しんぞう]が 悪[わる]いんだ。	
\\	彼らは軽快なステップで踊り出したの。	軽快	彼[かれ]らは 軽快[けいかい]なステップで 踊[おど]り 出[だ]したの。	
\\	彼女は快い眠りについています。	快い	彼女[かのじょ]は 快[こころよ]い 眠[ねむ]りについています。	
\\	今日は快晴ですね。	快晴	今日[きょう]は 快晴[かいせい]ですね。	
\\	ウナギの蒲焼きは美味しいね。	ウナギ	ウナギの 蒲焼[かばや]きは 美味[おい]しいね。	
\\	ここは子育てに最適な環境です。	最適	ここは 子育[こそだ]てに 最適[さいてき]な 環境[かんきょう]です。	
\\	ご指摘いただきありがとうございます。	指摘	ご 指摘[してき]いただきありがとうございます。	
\\	日本の夏は湿度が高いです。	湿度	日本[にほん]の 夏[なつ]は 湿度[しつど]が 高[たか]いです。	
\\	雨の日が続いて家の中が湿っぽいな。	湿っぽい	雨[あめ]の 日[ひ]が 続[つづ]いて 家[いえ]の 中[なか]が 湿[しめ]っぽいな。	
\\	この部屋は暗くて湿気が多いね。	湿気	この 部屋[へや]は 暗[くら]くて 湿気[しっけ]が 多[おお]いね。	
\\	洗濯物がまだ湿っています。	湿る	洗濯物[せんたくもの]がまだ 湿[しめ]っています。	
\\	その川の水は汚染されています。	汚染	その 川[かわ]の 水[みず]は 汚染[おせん]されています。	
\\	景気が回復してきたね。	景気	景気[けいき]が 回復[かいふく]してきたね。	
\\	歩道に新しくガードが付きましたね。	ガード	歩道[ほどう]に 新[あたら]しくガードが 付[つ]きましたね。	
\\	窓に男性の影が映っています。	影	窓[まど]に 男性[だんせい]の 影[かげ]が 映[うつ]っています。	
\\	ここは隣の市との境界です。	境界	ここは 隣[となり]の 市[し]との 境界[きょうかい]です。	
\\	引っ越して環境が変わりました。	環境	引[ひ]っ 越[こ]して 環境[かんきょう]が 変[か]わりました。	
\\	あの山のすぐ近くが国境です。	国境	あの 山[やま]のすぐ 近[ちか]くが 国境[こっきょう]です。	
\\	2つの市の境に川が流れているの。	境	2[ふた]つの 市[し]の 境[さかい]に 川[かわ]が 流[なが]れているの。	
\\	花の成長を観察して日記を書きなさい。	観察	花[はな]の 成長[せいちょう]を 観察[かんさつ]して 日記[にっき]を 書[か]きなさい。	
\\	そのモダンな外観の建物が大使館です。	外観	そのモダンな 外観[がいかん]の 建物[たてもの]が 大使館[たいしかん]です。	
\\	かぼちゃのスープは美味しいです。	かぼちゃ	かぼちゃのスープは 美味[おい]しいです。	
\\	彼は自分の状況を客観的に見てみたのね。	客観的	彼[かれ]は 自分[じぶん]の 状況[じょうきょう]を 客観的[きゃっかんてき]に 見[み]てみたのね。	
\\	主観だけで物事を見てはいけない。	主観	主観[しゅかん]だけで 物事[ものごと]を 見[み]てはいけない。	
\\	それは主観的な意見だ。	主観的	それは 主観的[しゅかんてき]な 意見[いけん]だ。	
\\	観客は興奮していたよ。	観客	観客[かんきゃく]は 興奮[こうふん]していたよ。	
\\	京都で3日間観光しました。	観光	京都[きょうと]で 3日間[みっかかん] 観光[かんこう]しました。	
\\	先生が星の観測に連れて行ってくれました。	観測	先生[せんせい]が 星[ほし]の 観測[かんそく]に 連[つ]れて 行[い]ってくれました。	
\\	それは彼の推測にすぎない。	推測	それは 彼[かれ]の 推測[すいそく]にすぎない。	
\\	クイズ番組を見るのが好きです。	クイズ	クイズ 番組[ばんぐみ]を 見[み]るのが 好[す]きです。	
\\	宇宙の謎は限りなく大きいの。	宇宙	宇宙[うちゅう]の 謎[なぞ]は 限[かぎ]りなく 大[おお]きいの。	
\\	衛星が打ち上げられましたね。	衛星	衛星[えいせい]が 打[う]ち 上[あ]げられましたね。	
\\	衛生には十分気を付けましょう。	衛生	衛生[えいせい]には 十分気[じゅうぶん き]を 付[つ]けましょう。	
\\	このレストランは衛生的で安心です。	衛生的	このレストランは 衛生的[えいせいてき]で 安心[あんしん]です。	
\\	彼は球技が大好きです。	球	彼[かれ]は 球[きゅう] 技[ぎ]が 大好[だいす]きです。	
\\	今朝の地震は震度3でしたよ。	震度	今朝[けさ]の 地震[じしん]は 震度[しんど] 3[さん]でしたよ。	
\\	車の振動で棚の荷物が落ちた。	振動	車[くるま]の 振動[しんどう]で 棚[たな]の 荷物[にもつ]が 落[お]ちた。	
\\	私は洗い物をしたあと、手にクリームをぬります。	クリーム	私[わたし]は 洗[あら]い 物[もの]をしたあと、 手[て]にクリームをぬります。	
\\	彼は神経が細かいですね。	神経	彼[かれ]は 神経[しんけい]が 細[こま]かいですね。	
\\	神様にお願いしました。	神様	神様[かみさま]にお 願[ねが]いしました。	
\\	彼女の家は神道です。	神道	彼女[かのじょ]の 家[いえ]は 神道[しんとう]です。	
\\	彼女は今、精神が不安定だ。	精神	彼女[かのじょ]は 今[いま]、 精神[せいしん]が 不安定[ふあんてい]だ。	
\\	降りる駅で料金を精算して下さい。	精算	降[お]りる 駅[えき]で 料金[りょうきん]を 精算[せいさん]して 下[くだ]さい。	
\\	上司から厳重に注意されました。	厳重	上司[じょうし]から 厳重[げんじゅう]に 注意[ちゅうい]されました。	
\\	彼は貴重な体験をしたわね。	貴重	彼[かれ]は 貴重[きちょう]な 体験[たいけん]をしたわね。	
\\	貴重品は自分で持っていてください。	貴重品	貴重品[きちょうひん]は 自分[じぶん]で 持[も]っていてください。	
\\	犬が自分のしっぽを追いかけてぐるぐる回っている。	ぐるぐる	犬[いぬ]が 自分[じぶん]のしっぽを 追[お]いかけてぐるぐる 回[まわ]っている。	
\\	ここにタイヤの跡があるわ。	跡	ここにタイヤの 跡[あと]があるわ。	
\\	雪の上にうさぎの足跡があった。	足跡	雪[ゆき]の 上[うえ]にうさぎの 足跡[あしあと]があった。	
\\	子供は好奇心でいっぱいだね。	好奇心	子供[こども]は 好奇心[こうきしん]でいっぱいだね。	
\\	彼のマジックはまるで奇跡です。	奇跡	彼[かれ]のマジックはまるで 奇跡[きせき]です。	
\\	3は奇数です。	奇数	3[さん]は 奇数[きすう]です。	
\\	あなたの経歴をメールで送ってください。	経歴	あなたの 経歴[けいれき]をメールで 送[おく]ってください。	
\\	その職は大卒の学歴が必要だ。	学歴	その 職[しょく]は 大卒[だいそつ]の 学歴[がくれき]が 必要[ひつよう]だ。	
\\	それはさぞがっかりしたことでしょう。	さぞ	それはさぞがっかりしたことでしょう。	
\\	王の墓を見学したよ。	王	王[おう]の 墓[はか]を 見学[けんがく]したよ。	
\\	その国の王様はとても賢い。	王様	その 国[くに]の 王様[おうさま]はとても 賢[かしこ]い。	
\\	彼らは家を建築中です。	建築	彼[かれ]らは 家[いえ]を 建築[けんちく] 中[ちゅう]です。	
\\	彼は去年、家を新築しました。	新築	彼[かれ]は 去年[きょねん]、 家[いえ]を 新築[しんちく]しました。	
\\	彼は新しい小説の構想を練っているの。	構想	彼[かれ]は 新[あたら]しい 小説[しょうせつ]の 構想[こうそう]を 練[ね]っているの。	
\\	彼はあの通りに店を構えているの。	構える	彼[かれ]はあの 通[とお]りに 店[みせ]を 構[かま]えているの。	
\\	お腹がいっぱいなのでお代わりは結構です。	結構	お 腹[なか]がいっぱいなのでお 代[か]わりは 結構[けっこう]です。	
\\	彼はサングラスが似合うね。	サングラス	彼[かれ]はサングラスが 似合[にあ]うね。	
\\	彼女は結構めがねが似合うね。	結構	彼女[かのじょ]は 結構[けっこう]めがねが 似合[にあ]うね。	
\\	子供に構い過ぎてはいけない。	構う	子供[こども]に 構[かま]い 過[す]ぎてはいけない。	
\\	位が上がれば責任も増えるものだ。	位	位[くらい]が 上[あ]がれば 責任[せきにん]も 増[ふ]えるものだ。	
\\	お店に防犯カメラが設置されたわね。	設置	お 店[みせ]に 防犯[ぼうはん]カメラが 設置[せっち]されたわね。	
\\	彼らは長い距離を歩き続けたんだ。	距離	彼[かれ]らは 長[なが]い 距離[きょり]を 歩[ある]き 続[つづ]けたんだ。	
\\	私たちは庭園を一周しました。	一周	私[わたし]たちは 庭園[ていえん]を 一周[いっしゅう]しました。	
\\	辺りを見回したの。	辺り	辺[あた]りを 見回[みまわ]したの。	
\\	大声で話すと周囲の人に迷惑ですよ。	周囲	大声[おおごえ]で 話[はな]すと 周囲[しゅうい]の 人[ひと]に 迷惑[めいわく]ですよ。	
\\	彼女はきっちり3時に来たわ。	きっちり	彼女[かのじょ]はきっちり 3時[さんじ]に 来[き]たわ。	
\\	久しぶりに家族全員で食卓を囲みました。	囲む	久[ひさ]しぶりに 家族全員[かぞく ぜんいん]で 食卓[しょくたく]を 囲[かこ]みました。	
\\	道路を横断するときは注意して。	横断	道路[どうろ]を 横断[おうだん]するときは 注意[ちゅうい]して。	
\\	ものごとは根気よく継続することが大切です。	継続	ものごとは 根気[こんき]よく 継続[けいぞく]することが 大切[たいせつ]です。	
\\	その会社は欧米に進出しているよね。	欧米	その 会社[かいしゃ]は 欧米[おうべい]に 進出[しんしゅつ]しているよね。	
\\	来月、隣の州に引っ越します。	州	来月[らいげつ]、 隣[となり]の 州[しゅう]に 引[ひ]っ 越[こ]します。	
\\	私たちは舟を降りて、陸に上がったんだ。	陸	私[わたし]たちは 舟[ふね]を 降[お]りて、 陸[おか]に 上[あ]がったんだ。	
\\	これは極めて重要な問題です。	極めて	これは 極[きわ]めて 重要[じゅうよう]な 問題[もんだい]です。	
\\	きゅうりに味噌をつけて食べたの。	きゅうり	きゅうりに 味噌[みそ]をつけて 食[た]べたの。	
\\	消極的な人は成功しないよ。	消極的	消極的[しょうきょくてき]な 人[ひと]は 成功[せいこう]しないよ。	
\\	それは極端な意見だね。	極端	それは 極端[きょくたん]な 意見[いけん]だね。	
\\	そのカメラには最先端の技術が使われています。	最先端	そのカメラには 最先端[さいせんたん]の 技術[ぎじゅつ]が 使[つか]われています。	
\\	一緒に食事しようか。	一緒	一緒[いっしょ]に 食事[しょくじ]しようか。	
\\	外貨を両替しました。	外貨	外貨[がいか]を 両替[りょうがえ]しました。	
\\	このトランクは貨物で送ろう。	貨物	このトランクは 貨物[かもつ]で 送[おく]ろう。	
\\	このトラックの車輪は頑丈そうですね。	車輪	このトラックの 車輪[しゃりん]は 頑丈[がんじょう]そうですね。	
\\	言いたいことを言ったら胸がすっとした。	すっと	言[い]いたいことを 言[い]ったら 胸[むね]がすっとした。	
\\	体がすっかり回復した。	回復	体[からだ]がすっかり 回復[かいふく]した。	
\\	この柿は渋いね。	渋い	この 柿[かき]は 渋[しぶ]いね。	
\\	この国道はよく渋滞します。	渋滞	この 国道[こくどう]はよく 渋滞[じゅうたい]します。	
\\	この辺一帯はリンゴ畑です。	一帯	この 辺[へん] 一帯[いったい]はリンゴ 畑[ばたけ]です。	
\\	この帯は長過ぎます。	帯	この 帯[おび]は 長過[なが す]ぎます。	
\\	日本は温帯にあります。	温帯	日本[にっぽん]は 温帯[おんたい]にあります。	
\\	そのチームは守備が甘いね。	守備	そのチームは 守備[しゅび]が 甘[あま]いね。	
\\	夜の11時に帰宅しました。	帰宅	夜[よる]の 11時[じゅういちじ]に 帰宅[きたく]しました。	
\\	その小学生は難しい本をすらすら読んだの。	すらすら	その 小学生[しょうがくせい]は 難[むずか]しい 本[ほん]をすらすら 読[よ]んだの。	
\\	その住宅地は便利な場所にあるね。	住宅地	その 住宅地[じゅうたくち]は 便利[べんり]な 場所[ばしょ]にあるね。	
\\	宛先不明で手紙が戻ってきたの。	宛先	宛先[あてさき] 不明[ふめい]で 手紙[てがみ]が 戻[もど]ってきたの。	
\\	手紙に宛名を書き込んだよ。	宛名	手紙[てがみ]に 宛名[あてな]を 書[か]き 込[こ]んだよ。	
\\	彼女は途中で後戻りしました。	後戻り	彼女[かのじょ]は 途中[とちゅう]で 後戻[あともど]りしました。	
\\	彼は暴力事件を起こしたんだよ。	起こす	彼[かれ]は 暴力事件[ぼうりょく じけん]を 起[お]こしたんだよ。	
\\	言葉の起源に大変興味があります。	起源	言葉[ことば]の 起源[きげん]に 大変興味[たいへん きょうみ]があります。	
\\	弟はようやくベッドから起き上がった。	起き上がる	弟[おとうと]はようやくベッドから 起[お]き 上[あ]がった。	
\\	その男はロビーをしばらくうろうろしていたよ。	うろうろ	その 男[おとこ]はロビーをしばらくうろうろしていたよ。	
\\	寝室の壁紙を張り替えました。	寝室	寝室[しんしつ]の 壁紙[かべがみ]を 張[は]り 替[か]えました。	
\\	台風が去って風が静まったね。	静まる	台風[たいふう]が 去[さ]って 風[かぜ]が 静[しず]まったね。	
\\	今度の休暇にフィリピンに行きます。	休暇	今度[こんど]の 休暇[きゅうか]にフィリピンに 行[い]きます。	
\\	片手運転は危ないよ。	片手	片手[かたて] 運転[うんてん]は 危[あぶ]ないよ。	
\\	片方の目がかゆいです。	片方	片方[かたほう]の 目[め]がかゆいです。	
\\	この道は片側通行です。	片側	この 道[みち]は 片側[かたがわ] 通行[つうこう]です。	
\\	彼は仲間を裏切ったの。	裏切る	彼[かれ]は 仲間[なかま]を 裏切[うらぎ]ったの。	
\\	八百屋さんでたくさんおまけしてもらったよ。	おまけ	八百屋[やおや]さんでたくさんおまけしてもらったよ。	
\\	裏口へお回りください。	裏口	裏口[うらぐち]へお 回[まわ]りください。	
\\	シャツを裏返しに着ているよ。	裏返し	シャツを 裏返[うらがえ]しに 着[き]ているよ。	
\\	奴は裏門から出てきたぜ。	裏門	奴[やつ]は 裏門[うらもん]から 出[で]てきたぜ。	
\\	シャツを裏表に着ていますよ。	裏表	シャツを 裏表[うらおもて]に 着[き]ていますよ。	
\\	論文の項目を分かりやすく整理したわ。	項目	論文[ろんぶん]の 項目[こうもく]を 分[わ]かりやすく 整理[せいり]したわ。	
\\	彼から良い印象を受けたわ。	印象	彼[かれ]から 良[い]い 印象[いんしょう]を 受[う]けたわ。	
\\	間違いに印を付けておきました。	印	間違[まちが]いに 印[しるし]を 付[つ]けておきました。	
\\	クッキーと紅茶をいただきました。	クッキー	クッキーと 紅茶[こうちゃ]をいただきました。	
\\	年賀状を印刷したの。	印刷	年賀状[ねんがじょう]を 印刷[いんさつ]したの。	
\\	この週刊誌を毎週買っているの。	週刊	この 週刊[しゅうかん] 誌[し]を 毎週買[まいしゅう か]っているの。	
\\	この雑誌は月刊ですか。	月刊	この 雑誌[ざっし]は 月刊[げっかん]ですか。	
\\	この本は15年前に出版された。	出版	この 本[ほん]は 15年前[じゅうごねんまえ]に 出版[しゅっぱん]された。	
\\	彼女は出版社で働いています。	出版社	彼女[かのじょ]は 出版社[しゅっぱんしゃ]で 働[はたら]いています。	
\\	詳細はお気軽にお問い合わせ下さい。	詳細	詳細[しょうさい]はお 気軽[きがる]にお 問[と]い 合[あ]わせ 下[くだ]さい。	
\\	夜道の一人歩きは心細いね。	心細い	夜道[よみち]の 一人歩[ひとりある]きは 心細[こころぼそ]いね。	
\\	大会のスローガンを掲示したよ。	掲示	大会[たいかい]のスローガンを 掲示[けいじ]したよ。	
\\	プラグをコンセントに挿し込んだよ。	コンセント	プラグをコンセントに 挿[さ]し 込[こ]んだよ。	
\\	姉は何に対しても積極的です。	積極的	姉[あね]は 何[なに]に 対[たい]しても 積極的[せっきょくてき]です。	
\\	あなたは言い訳が多すぎます。	言い訳	あなたは 言[い]い 訳[わけ]が 多[おお]すぎます。	
\\	解答に誤りが3つ有りますよ。	誤り	解答[かいとう]に 誤[あやま]りが 3[みっ]つ 有[あ]りますよ。	
\\	彼は機械の操作を誤った。	誤る	彼[かれ]は 機械[きかい]の 操作[そうさ]を 誤[あやま]った。	
\\	私の気持ちを誤解しているんじゃない!	誤解	私[わたし]の 気持[きも]ちを 誤解[ごかい]しているんじゃない!	
\\	彼はやっと問題点に気付きました。	気付く	彼[かれ]はやっと 問題点[もんだいてん]に 気付[きづ]きました。	
\\	靴の底にガムがくっ付いてしまった。	くっ付く	靴[くつ]の 底[そこ]にガムがくっ 付[つ]いてしまった。	
\\	あなたのやり方はずるい。	ずるい	あなたのやり 方[かた]はずるい。	
\\	駅でようやく彼に追い付きました。	追い付く	駅[えき]でようやく 彼[かれ]に 追[お]い 付[つ]きました。	
\\	彼は恐い顔付きで話したの。	顔付き	彼[かれ]は 恐[こわ]い 顔付[かおつ]きで 話[はな]したの。	
\\	彼は名案を思い付いたの。	思い付く	彼[かれ]は 名案[めいあん]を 思[おも]い 付[つ]いたの。	
\\	食事の後片付けを手伝ったの。	後片付け	食事[しょくじ]の 後片付[あとかたづ]けを 手伝[てつだ]ったの。	
\\	私は彼女の言葉に傷付きました。	傷付く	私[わたし]は 彼女[かのじょ]の 言葉[ことば]に 傷付[きずつ]きました。	
\\	あなたを傷付けるつもりはありませんでした。	傷付ける	あなたを 傷付[きずつ]けるつもりはありませんでした。	
\\	彼は壁に耳をくっ付けて隣の話を聞いていたの。	くっ付ける	彼[かれ]は 壁[かべ]に 耳[みみ]をくっ 付[つ]けて 隣[となり]の 話[はなし]を 聞[き]いていたの。	
\\	彼はせっかちで困ります。	せっかち	彼[かれ]はせっかちで 困[こま]ります。	
\\	彼女への伝言を言付けたの。	言付ける	彼女[かのじょ]への 伝言[でんごん]を 言付[ことづ]けたの。	
\\	アルミニウムは金属の一種です。	金属	アルミニウムは 金属[きんぞく]の 一種[いっしゅ]です。	
\\	学校では音楽部に所属していました。	所属	学校[がっこう]では 音楽部[おんがくぶ]に 所属[しょぞく]していました。	
\\	彼は大蔵省に勤務しているんだよ。	大蔵省	彼[かれ]は 大蔵省[おおくらしょう]に 勤務[きんむ]しているんだよ。	
\\	ビザについて外務省に問い合わせた。	外務省	ビザについて 外務省[がいむしょう]に 問[と]い 合[あ]わせた。	
\\	来週、帰省します。	帰省	来週[らいしゅう]、 帰省[きせい]します。	
\\	地図内の小さい建物は省略してあります。	省略	地図内[ちずない]の 小[ちい]さい 建物[たてもの]は 省略[しょうりゃく]してあります。	
\\	インターネットは情報の概念を変えたよね。	概念	インターネットは 情報[じょうほう]の 概念[がいねん]を 変[か]えたよね。	
\\	父はうちわで扇いでいるの。	うちわ	父[ちち]はうちわで 扇[あお]いでいるの。	
\\	卒業の記念にみんなで旅行したよ。	記念	卒業[そつぎょう]の 記念[きねん]にみんなで 旅行[りょこう]したよ。	
\\	自分の信念に従いたいと思います。	信念	自分[じぶん]の 信念[しんねん]に 従[したが]いたいと 思[おも]います。	
\\	進み具合は全て順調です。	順調	進[すす]み 具合[ぐあい]は 全[すべ]て 順調[じゅんちょう]です。	
\\	あいうえお順に並べて下さい。	順	あいうえお 順[じゅん]に 並[なら]べて 下[くだ]さい。	
\\	これを順番通りに並べ替えて下さい。	順番	これを 順番[じゅんばん] 通[どお]りに 並[なら]べ 替[か]えて 下[くだ]さい。	
\\	今年は去年よりも順位が上がりました。	順位	今年[ことし]は 去年[きょねん]よりも 順位[じゅんい]が 上[あ]がりました。	
\\	子供たちは順々にお菓子を受け取ったよ。	順々に	子供[こども]たちは 順々[じゅんじゅん]にお 菓子[かし]を 受[う]け 取[と]ったよ。	
\\	大事なメールをうっかり消してしまった。	うっかり	大事[だいじ]なメールをうっかり 消[け]してしまった。	
\\	正しい順序で操作してください。	順序	正[ただ]しい 順序[じゅんじょ]で 操作[そうさ]してください。	
\\	駅は逆方向です。	逆	駅[えき]は 逆[ぎゃく] 方向[ほうこう]です。	
\\	親に逆らうのは良くない事です。	逆らう	親[おや]に 逆[さか]らうのは 良[よ]くない 事[こと]です。	
\\	絵を逆さにしてみてごらん。	逆さ	絵[え]を 逆[さか]さにしてみてごらん。	
\\	それじゃあ上と下が逆さまだよ。	逆さま	それじゃあ 上[うえ]と 下[した]が 逆[さか]さまだよ。	
\\	店の前に長い行列ができていますよ。	行列	店[みせ]の 前[まえ]に 長[なが]い 行列[ぎょうれつ]ができていますよ。	
\\	体育館に行って整列しなさい。	整列	体育館[たいいくかん]に 行[い]って 整列[せいれつ]しなさい。	
\\	くだらないおしゃべりはやめなさい。	くだらない	くだらないおしゃべりはやめなさい。	
\\	実例を使って説明してください。	実例	実例[じつれい]を 使[つか]って 説明[せつめい]してください。	
\\	余りは次回使いましょう。	余り	余[あま]りは 次回使[じかい つか]いましょう。	
\\	福祉予算は削減されないことに決まったのよ。	削減	福祉予算[ふくしよさん]は 削減[さくげん]されないことに 決[き]まったのよ。	
\\	要らないファイルは削除して下さい。	削除	要[い]らないファイルは 削除[さくじょ]して 下[くだ]さい。	
\\	そのことは既にみんな知っています。	既に	そのことは 既[すで]にみんな 知[し]っています。	
\\	彼の体型じゃ既製のサイズに合わないよ。	既製	彼[かれ]の 体型[たいけい]じゃ 既製[きせい]のサイズに 合[あ]わないよ。	
\\	これは深刻な問題です。	深刻	これは 深刻[しんこく]な 問題[もんだい]です。	
\\	玉ねぎを細かく刻んでください。	刻む	玉[たま]ねぎを 細[こま]かく 刻[きざ]んでください。	
\\	何てずうずうしい人なんだろう。	ずうずうしい	何[なん]てずうずうしい 人[ひと]なんだろう。	
\\	ただ今の時刻は6時35分です。	時刻	ただ 今[いま]の 時刻[じこく]は 6時35分[ろくじ さんじゅうごふん]です。	
\\	申し込みの締め切りはいつですか。	締め切り	申[もう]し 込[こ]みの 締[し]め 切[き]りはいつですか。	
\\	応募受付は締め切りました。	締め切る	応募受付[おうぼ うけつけ]は 締[し]め 切[き]りました。	
\\	レバーを右に回すと締まります。	締まる	レバーを 右[みぎ]に 回[まわ]すと 締[し]まります。	
\\	ここはかつてゴールドラッシュで栄えた町だよ。	栄える	ここはかつてゴールドラッシュで 栄[さか]えた 町[まち]だよ。	
\\	豆腐は栄養のある食べ物です。	栄養	豆腐[とうふ]は 栄養[えいよう]のある 食[た]べ 物[もの]です。	
\\	彼女はとても教養のある人ですね。	教養	彼女[かのじょ]はとても 教養[きょうよう]のある 人[ひと]ですね。	
\\	子供がすやすや眠っているね。	すやすや	子供[こども]がすやすや 眠[ねむ]っているね。	
\\	彼女は今、休養中です。	休養	彼女[かのじょ]は 今[いま]、 休養[きゅうよう] 中[ちゅう]です。	
\\	彼女はアレルギーに苦しんでいます。	苦しむ	彼女[かのじょ]はアレルギーに 苦[くる]しんでいます。	
\\	店に苦情の電話をかけました。	苦情	店[みせ]に 苦情[くじょう]の 電話[でんわ]をかけました。	
\\	苦心して絵を描き上げました。	苦心	苦心[くしん]して 絵[え]を 描[か]き 上[あ]げました。	
\\	誰も彼の苦しみを理解していなかったな。	苦しみ	誰[だれ]も 彼[かれ]の 苦[くる]しみを 理解[りかい]していなかったな。	
\\	私は人前で話すことが苦痛です。	苦痛	私[わたし]は 人前[ひとまえ]で 話[はな]すことが 苦痛[くつう]です。	
\\	強い日差しが選手たちを苦しめたの。	苦しめる	強[つよ]い 日差[ひざ]しが 選手[せんしゅ]たちを 苦[くる]しめたの。	
\\	この食堂はセルフサービスです。	セルフサービス	この 食堂[しょくどう]はセルフサービスです。	
\\	母は苦労して私たちを育てたの。	苦労	母[はは]は 苦労[くろう]して 私[わたし]たちを 育[そだ]てたの。	
\\	父は長い勤労生活を送ったんです。	勤労	父[ちち]は 長[なが]い 勤労[きんろう] 生活[せいかつ]を 送[おく]ったんです。	
\\	彼は過労のために倒れたの。	過労	彼[かれ]は 過労[かろう]のために 倒[たお]れたの。	
\\	困難にあってもあきらめてはいけないよ。	困難	困難[こんなん]にあってもあきらめてはいけないよ。	
\\	彼女の親切に甘えました。	甘える	彼女[かのじょ]の 親切[しんせつ]に 甘[あま]えました。	
\\	子供を甘やかしてはいけないの。	甘やかす	子供[こども]を 甘[あま]やかしてはいけないの。	
\\	部長は僕の仕事に辛い評価を出したんだ。	辛い	部長[ぶちょう]は 僕[ぼく]の 仕事[しごと]に 辛[から]い 評価[ひょうか]を 出[だ]したんだ。	
\\	この子はアルファベットを全部言えます。	アルファベット	この 子[こ]はアルファベットを 全部言[ぜんぶ い]えます。	
\\	幸い、電車に嵐の影響はなかった。	幸い	幸[さいわ]い、 電車[でんしゃ]に 嵐[あらし]の 影響[えいきょう]はなかった。	
\\	幸運にもチケットを手に入れました。	幸運	幸運[こううん]にもチケットを 手[て]に 入[い]れました。	
\\	彼女は幸福な日々を過ごしているわ。	幸福	彼女[かのじょ]は 幸福[こうふく]な 日々[ひび]を 過[す]ごしているわ。	
\\	靴に砂が入ってしまった。	砂	靴[くつ]に 砂[すな]が 入[はい]ってしまった。	
\\	砂漠ではほとんど雨が降らないのよ。	砂漠	砂漠[さばく]ではほとんど 雨[あめ]が 降[ふ]らないのよ。	
\\	彼らは漁業を営んでいるんだ。	漁業	彼[かれ]らは 漁業[ぎょぎょう]を 営[いとな]んでいるんだ。	
\\	外はもう薄暗くなりましたよ。	薄暗い	外[そと]はもう 薄暗[うすぐら]くなりましたよ。	
\\	スープを少し薄めましょうか。	薄める	スープを 少[すこ]し 薄[うす]めましょうか。	
\\	父のいびきは大きいんだ。	いびき	父[ちち]のいびきは 大[おお]きいんだ。	
\\	子供たちは冬でも薄着で通します。	薄着	子供[こども]たちは 冬[ふゆ]でも 薄着[うすぎ]で 通[とお]します。	
\\	普段から厚着しないようにしています。	厚着	普段[ふだん]から 厚着[あつぎ]しないようにしています。	
\\	相手会社から強い圧力が掛かった。	圧力	相手会社[あいてがいしゃ]から 強[つよ]い 圧力[あつりょく]が 掛[か]かった。	
\\	高い山は気圧が低いね。	気圧	高[たか]い 山[やま]は 気圧[きあつ]が 低[ひく]いね。	
\\	この暑さは高気圧のせいです。	高気圧	この 暑[あつ]さは 高気圧[こうきあつ]のせいです。	
\\	これを縮小して印刷してください。	縮小	これを 縮小[しゅくしょう]して 印刷[いんさつ]してください。	
\\	暑さは峠を越したね。	越す	暑[あつ]さは 峠[とうげ]を 越[こ]したね。	
\\	彼はいやらしい男ね。	いやらしい	彼[かれ]はいやらしい 男[おとこ]ね。	
\\	リレーで彼は2人追い抜いたね。	追い抜く	リレーで 彼[かれ]は 2人[ふたり] 追[お]い 抜[ぬ]いたね。	
\\	彼はカバンから札の束を取り出したんだ。	札	彼[かれ]はカバンから 札[さつ]の 束[たば]を 取[と]り 出[だ]したんだ。	
\\	改札で3時に会おう。	改札	改札[かいさつ]で 3時[さんじ]に 会[あ]おう。	
\\	家族に感謝しています。	感謝	家族[かぞく]に 感謝[かんしゃ]しています。	
\\	先生に月謝を渡しましたか。	月謝	先生[せんせい]に 月謝[げっしゃ]を 渡[わた]しましたか。	
\\	今日は久しぶりに日が射してるね。	射す	今日[きょう]は 久[ひさ]しぶりに 日[ひ]が 射[さ]してるね。	
\\	プログラムの過程に問題があるの。	過程	プログラムの 過程[かてい]に 問題[もんだい]があるの。	
\\	イントネーションの違いに気を付けて。	イントネーション	イントネーションの 違[ちが]いに 気[き]を 付[つ]けて。	
\\	一年生の課程を修了したの。	課程	一年生[いちねんせい]の 課程[かてい]を 修了[しゅうりょう]したの。	
\\	全部で6時間の行程です。	行程	全部[ぜんぶ]で 6時間[ろくじかん]の 行程[こうてい]です。	
\\	その件でしたら、先程上司から許可を得ております。	先程	その 件[けん]でしたら、 先程[さきほど] 上司[じょうし]から 許可[きょか]を 得[え]ております。	
\\	彼は非常に優れた選手です。	優れる	彼[かれ]は 非常[ひじょう]に 優[すぐ]れた 選手[せんしゅ]です。	
\\	彼女はずっと女優になるのが夢でした。	女優	彼女[かのじょ]はずっと 女優[じょゆう]になるのが 夢[ゆめ]でした。	
\\	湖の水は透き通っていたよ。	透き通る	湖[みずうみ]の 水[みず]は 透[す]き 通[とお]っていたよ。	
\\	彼は生徒の指導が上手ね。	指導	彼[かれ]は 生徒[せいと]の 指導[しどう]が 上手[じょうず]ね。	
\\	彼は本社で働くことを希望しています。	希望	彼[かれ]は 本社[ほんしゃ]で 働[はたら]くことを 希望[きぼう]しています。	
\\	この椅子はがたがたしているね。	がたがた	この 椅子[いす]はがたがたしているね。	
\\	彼女は結婚生活に失望していたの。	失望	彼女[かのじょ]は 結婚生活[けっこん せいかつ]に 失望[しつぼう]していたの。	
\\	彼は意志の強い人です。	意志	彼[かれ]は 意志[いし]の 強[つよ]い 人[ひと]です。	
\\	私は医者を志しています。	志す	私[わたし]は 医者[いしゃ]を 志[こころざ]しています。	
\\	彼ったら怒り爆発だったよ。	怒り	彼[かれ]ったら 怒[いか]り 爆発[ばくはつ]だったよ。	
\\	自分自身を疑ってはいけません。	自身	自分[じぶん] 自身[じしん]を 疑[うたが]ってはいけません。	
\\	家内は九州出身です。	出身	家内[かない]は 九州[きゅうしゅう] 出身[しゅっしん]です。	
\\	運動した翌日は全身の筋肉が痛い。	全身	運動[うんどう]した 翌日[よくじつ]は 全身[ぜんしん]の 筋肉[きんにく]が 痛[いた]い。	
\\	カンニングをした人は0点です。	カンニング	カンニングをした 人[ひと]は 0点[れいてん]です。	
\\	身長はどれくらいありますか。	身長	身長[しんちょう]はどれくらいありますか。	
\\	私は心身共に疲れていました。	心身	私[わたし]は 心身[しんしん] 共[とも]に 疲[つか]れていました。	
\\	彼はいつも受け身の姿勢で、自分からは何もしないんだ。	受け身	彼[かれ]はいつも 受[う]け 身[み]の 姿勢[しせい]で、 自分[じぶん]からは 何[なに]もしないんだ。	
\\	証拠を見つけるまで逮捕はできません。	証拠	証拠[しょうこ]を 見[み]つけるまで 逮捕[たいほ]はできません。	
\\	もっと照明を明るくしてください。	照明	もっと 照明[しょうめい]を 明[あか]るくしてください。	
\\	いろいろ工夫して仕事をやりとげたさ。	工夫	いろいろ 工夫[くふう]して 仕事[しごと]をやりとげたさ。	
\\	主婦の主な仕事は家事よ。	主婦	主婦[しゅふ]の 主[おも]な 仕事[しごと]は 家事[かじ]よ。	
\\	彼はあぐらをかいてしゃべっていますね。	あぐら	彼[かれ]はあぐらをかいてしゃべっていますね。	
\\	最近、産婦人科の不足が問題になっているね。	産婦人科	最近[さいきん]、 産婦人科[さんふじんか]の 不足[ふそく]が 問題[もんだい]になっているね。	
\\	社長の奥様はきれいな方です。	奥様	社長[しゃちょう]の 奥様[おくさま]はきれいな 方[かた]です。	
\\	お互いを信頼することが大事です。	お互い	お 互[たが]いを 信頼[しんらい]することが 大事[だいじ]です。	
\\	男女交互に並んでください。	交互	男女[だんじょ] 交互[こうご]に 並[なら]んでください。	
\\	彼は高齢を理由に社長を辞めたよ。	高齢	彼[かれ]は 高齢[こうれい]を 理由[りゆう]に 社長[しゃちょう]を 辞[や]めたよ。	
\\	子供はたくさんの愛情が必要です。	愛情	子供[こども]はたくさんの 愛情[あいじょう]が 必要[ひつよう]です。	
\\	彼女は娘に可愛らしい服を作りましたね。	可愛らしい	彼女[かのじょ]は 娘[むすめ]に 可愛[かわい]らしい 服[ふく]を 作[つく]りましたね。	
\\	彼女は猫を可愛がっています。	可愛がる	彼女[かのじょ]は 猫[ねこ]を 可愛[かわい]がっています。	
\\	その子はいやいや学校へ行ったの。	いやいや	その 子[こ]はいやいや 学校[がっこう]へ 行[い]ったの。	
\\	彼女は恋をしてきれいになったね。	恋	彼女[かのじょ]は 恋[こい]をしてきれいになったね。	
\\	彼は最近、失恋したらしいの。	失恋	彼[かれ]は 最近[さいきん]、 失恋[しつれん]したらしいの。	
\\	恋する気持ちを歌にしました。	恋する	恋[こい]する 気持[きも]ちを 歌[うた]にしました。	
\\	国会の会期が延長されたわね。	延長	国会[こっかい]の 会期[かいき]が 延長[えんちょう]されたわね。	
\\	運動会は雨で延期されました。	延期	運動会[うんどうかい]は 雨[あめ]で 延期[えんき]されました。	
\\	家族で彼の合格を祝ったよ。	祝う	家族[かぞく]で 彼[かれ]の 合格[ごうかく]を 祝[いわ]ったよ。	
\\	結婚のお祝いに食器を頂きました。	祝い	結婚[けっこん]のお 祝[いわ]いに 食器[しょっき]を 頂[いただ]きました。	
\\	最近ウエストが太くなったな。	ウエスト	最近[さいきん]ウエストが 太[ふと]くなったな。	
\\	子供の純粋な心を傷付けてはいけません。	純粋	子供[こども]の 純粋[じゅんすい]な 心[こころ]を 傷付[きずつ]けてはいけません。	
\\	もう一度慎重に見直しましょう。	慎重	もう 一度[いちど] 慎重[しんちょう]に 見直[みなお]しましょう。	
\\	赤ちゃんの笑顔が可愛いい。	笑顔	赤[あか]ちゃんの 笑顔[えがお]が 可愛[かわ]いい。	
\\	弟は新しい自転車に大喜びです。	大喜び	弟[おとうと]は 新[あたら]しい 自転車[じてんしゃ]に 大喜[おおよろこ]びです。	
\\	この病院は24時間看護です。	看護	この 病院[びょういん]は 24時間[にじゅうよじかん] 看護[かんご]です。	
\\	彼女は一晩中彼を看病したの。	看病	彼女[かのじょ]は 一晩中彼[ひとばんじゅう かれ]を 看病[かんびょう]したの。	
\\	僕は彼のしたことは肯定できないな。	肯定	僕[ぼく]は 彼[かれ]のしたことは 肯定[こうてい]できないな。	
\\	僕たちはオルガンに合わせて賛美歌を歌ったんだ。	オルガン	僕[ぼく]たちはオルガンに 合[あ]わせて 賛美歌[さんびか]を 歌[うた]ったんだ。	
\\	日本の古い料理法についての記述を読んだんだ。	記述	日本[にほん]の 古[ふる]い 料理法[りょうりほう]についての 記述[きじゅつ]を 読[よ]んだんだ。	
\\	裁判の様子はテレビで中継されたわよ。	裁判	裁判[さいばん]の 様子[ようす]はテレビで 中継[ちゅうけい]されたわよ。	
\\	彼らは裁判所の前で知らせを待っています。	裁判所	彼[かれ]らは 裁判所[さいばんしょ]の 前[まえ]で 知[し]らせを 待[ま]っています。	
\\	彼女の訴えは認められたわ。	訴え	彼女[かのじょ]の 訴[うった]えは 認[みと]められたわ。	
\\	ここは危険区域よ。	区域	ここは 危険[きけん] 区域[くいき]よ。	
\\	彼は盗みの疑いをかけられたの。	疑い	彼[かれ]は 盗[ぬす]みの 疑[うたが]いをかけられたの。	
\\	彼の言葉が本当かは疑問です。	疑問	彼[かれ]の 言葉[ことば]が 本当[ほんとう]かは 疑問[ぎもん]です。	
\\	彼はからからと笑いました。	からから	彼[かれ]はからからと 笑[わら]いました。	
\\	なぜあなたは私を疑うのですか。	疑う	なぜあなたは 私[わたし]を 疑[うたが]うのですか。	
\\	疑わしい場所は全部調べよう。	疑わしい	疑[うたが]わしい 場所[ばしょ]は 全部調[ぜんぶ しら]べよう。	
\\	彼の成長は著しいです。	著しい	彼[かれ]の 成長[せいちょう]は 著[いちじる]しいです。	
\\	この本では自然の大切さがよく著されているよ。	著す	この 本[ほん]では 自然[しぜん]の 大切[たいせつ]さがよく 著[あらわ]されているよ。	
\\	私たちには知る権利があります。	権利	私[わたし]たちには 知[し]る 権利[けんり]があります。	
\\	彼はこの国で大きな権力を持っているわ。	権力	彼[かれ]はこの 国[くに]で 大[おお]きな 権力[けんりょく]を 持[も]っているわ。	
\\	全ての人に人権がある。	人権	全[すべ]ての 人[ひと]に 人権[じんけん]がある。	
\\	彼は試合の途中で棄権したぞ。	棄権	彼[かれ]は 試合[しあい]の 途中[とちゅう]で 棄権[きけん]したぞ。	
\\	彼女はキッチンで夕食を作っています。	きっちん	彼女[かのじょ]はキッチンで 夕食[ゆうしょく]を 作[つく]っています。	
\\	他人の権利を侵してはなりません。	侵す	他人[たにん]の 権利[けんり]を 侵[おか]してはなりません。	
\\	彼の行為はみんなの誤解を招いたわ。	行為	彼[かれ]の 行為[こうい]はみんなの 誤解[ごかい]を 招[まね]いたわ。	
\\	このデザインは数々の賞を受けています。	賞	このデザインは 数々[かずかず]の 賞[しょう]を 受[う]けています。	
\\	パーティーのビンゴの賞品は何がいいだろう。	賞品	パーティーのビンゴの 賞品[しょうひん]は 何[なに]がいいだろう。	
\\	この鉢植えは観賞用です。	観賞	この 鉢植[はちう]えは 観賞[かんしょう] 用[よう]です。	
\\	国の財政はとても苦しい状態だな。	財政	国[くに]の 財政[ざいせい]はとても 苦[くる]しい 状態[じょうたい]だな。	
\\	彼は株で財産の半分を失ったの。	財産	彼[かれ]は 株[かぶ]で 財産[ざいさん]の 半分[はんぶん]を 失[うしな]ったの。	
\\	紙をくしゃくしゃに丸めました。	くしゃくしゃ	紙[かみ]をくしゃくしゃに 丸[まる]めました。	
\\	彼は金融関係の会社で働いています。	金融	彼[かれ]は 金融[きんゆう] 関係[かんけい]の 会社[かいしゃ]で 働[はたら]いています。	
\\	電車は5分間隔で来ますよ。	間隔	電車[でんしゃ]は 5分[ごふん] 間隔[かんかく]で 来[き]ますよ。	
\\	コンサートは夜7時5分に終了しました。	終了	コンサートは 夜7時5分[よる しち じ ご ふん]に 終了[しゅうりょう]しました。	
\\	仕事は全て完了しました。	完了	仕事[しごと]は 全[すべ]て 完了[かんりょう]しました。	
\\	先月、そのコースを修了しました。	修了	先月[せんげつ]、そのコースを 修了[しゅうりょう]しました。	
\\	これは政府の承認を受けた資格です。	承認	これは 政府[せいふ]の 承認[しょうにん]を 受[う]けた 資格[しかく]です。	
\\	そのことは承知しております。	承知	そのことは 承知[しょうち]しております。	
\\	彼は金持ちだけど、けちね。	けち	彼[かれ]は 金持[かねも]ちだけど、けちね。	
\\	自動車税を納めたよ。	納める	自動車税[じどうしゃぜい]を 納[おさ]めたよ。	
\\	なんとか親を説得してみるよ。	説得	なんとか 親[おや]を 説得[せっとく]してみるよ。	
\\	ここに去年の所得をご記入ください。	所得	ここに 去年[きょねん]の 所得[しょとく]をご 記入[きにゅう]ください。	
\\	彼は大金を得ましたよ。	得る	彼[かれ]は 大金[たいきん]を 得[え]ましたよ。	
\\	冬は空気が乾燥しますね。	乾燥	冬[ふゆ]は 空気[くうき]が 乾燥[かんそう]しますね。	
\\	あの会社の幹部は皆とても優秀だね。	幹部	あの 会社[かいしゃ]の 幹部[かんぶ]は 皆[みんな]とても 優秀[ゆうしゅう]だね。	
\\	私たちは酸素無しでは生きていけない。	酸素	私[わたし]たちは 酸素[さんそ] 無[な]しでは 生[い]きていけない。	
\\	彼女はとても素直で可愛いですね。	素直	彼女[かのじょ]はとても 素直[すなお]で 可愛[かわい]いですね。	
\\	砂ぼこりで机がざらざらしているね。	ざらざら	砂[すな]ぼこりで 机[つくえ]がざらざらしているね。	
\\	水は水素と酸素でできています。	水素	水[みず]は 水素[すいそ]と 酸素[さんそ]でできています。	
\\	彼は素早くあたりを見回したの。	素早い	彼[かれ]は 素早[すばや]くあたりを 見回[みまわ]したの。	
\\	小屋に石炭の山があります。	石炭	小屋[こや]に 石炭[せきたん]の 山[やま]があります。	
\\	あの岩まで泳ごう。	岩	あの 岩[いわ]まで 泳[およ]ごう。	
\\	船がやっと岸に着いたよ。	岸	船[ふね]がやっと 岸[きし]に 着[つ]いたよ。	
\\	陸上部は校庭で練習しています。	校庭	陸上部[りくじょうぶ]は 校庭[こうてい]で 練習[れんしゅう]しています。	
\\	来年、衆議院が解散されるだろう。	解散	来年[らいねん]、 衆議院[しゅうぎいん]が 解散[かいさん]されるだろう。	
\\	都会の空はスモッグで灰色ね。	スモッグ	都会[とかい]の 空[そら]はスモッグで 灰色[はいいろ]ね。	
\\	休日は植物の世話をして過ごします。	植物	休日[きゅうじつ]は 植物[しょくぶつ]の 世話[せわ]をして 過[す]ごします。	
\\	この国はイギリスの植民地でした。	植民地	この 国[くに]はイギリスの 植民地[しょくみんち]でした。	
\\	植木に水をやりました。	植木	植木[うえき]に 水[みず]をやりました。	
\\	植物園には珍しい花がたくさんありますね。	植物園	植物園[しょくぶつえん]には 珍[めずら]しい 花[はな]がたくさんありますね。	
\\	何を根拠にそんな事を言うのですか。	根拠	何[なに]を 根拠[こんきょ]にそんな 事[こと]を 言[い]うのですか。	
\\	問題の根本を見直しましょう。	根本	問題[もんだい]の 根本[こんぽん]を 見直[みなお]しましょう。	
\\	父は長い板を買って来たんだ。	板	父[ちち]は 長[なが]い 板[いた]を 買[か]って 来[き]たんだ。	
\\	すりに財布をとられた。	すり	すりに 財布[さいふ]をとられた。	
\\	店の看板を塗り替えました。	看板	店[みせ]の 看板[かんばん]を 塗[ぬ]り 替[か]えました。	
\\	草花を大切にしましょう。	草花	草花[くさばな]を 大切[たいせつ]にしましょう。	
\\	この庭は草木が枯れているね。	草木	この 庭[にわ]は 草木[くさき]が 枯[か]れているね。	
\\	先生にそんな言葉遣いをしてはいけません。	言葉遣い	先生[せんせい]にそんな 言葉遣[ことばづか]いをしてはいけません。	
\\	落ち葉の季節になりましたね。	落ち葉	落[お]ち 葉[ば]の 季節[きせつ]になりましたね。	
\\	秋には木の葉が赤くなります。	木の葉	秋[あき]には 木[こ]の 葉[は]が 赤[あか]くなります。	
\\	書き言葉と話し言葉はだいぶ違うことがあります。	書き言葉	書[か]き 言葉[ことば]と 話[はな]し 言葉[ことば]はだいぶ 違[ちが]うことがあります。	
\\	彼は知識の吸収が早いですね。	吸収	彼[かれ]は 知識[ちしき]の 吸収[きゅうしゅう]が 早[はや]いですね。	
\\	昼食におにぎりを食べました。	おにぎり	昼食[ちゅうしょく]におにぎりを 食[た]べました。	
\\	ゆっくり呼吸してください。	呼吸	ゆっくり 呼吸[こきゅう]してください。	
\\	ほこりを吸い込んじゃった。	吸い込む	ほこりを 吸[す]い 込[こ]んじゃった。	
\\	あなたにまで迷惑が及んでごめんなさい。	及ぶ	あなたにまで 迷惑[めいわく]が 及[およ]んでごめんなさい。	
\\	この荷物は丁寧に扱って下さい。	扱う	この 荷物[にもつ]は 丁寧[ていねい]に 扱[あつか]って 下[くだ]さい。	
\\	私たちは高級ホテルに泊まったの。	高級	私[わたし]たちは 高級[こうきゅう]ホテルに 泊[と]まったの。	
\\	彼は書道3級です。	級	彼[かれ]は 書道3[しょどう さん] 級[きゅう]です。	
\\	彼は上級のコースに上がったよ。	上級	彼[かれ]は 上級[じょうきゅう]のコースに 上[あ]がったよ。	
\\	かみそりの刃で指を切りました。	かみそり	かみそりの 刃[は]で 指[ゆび]を 切[き]りました。	
\\	これは初級の教科書です。	初級	これは 初級[しょきゅう]の 教科書[きょうかしょ]です。	
\\	年をとると血管が硬くなります。	血管	年[とし]をとると 血管[けっかん]が 硬[かた]くなります。	
\\	出血がひどいので、医者に行ったほうがいい。	出血	出血[しゅっけつ]がひどいので、 医者[いしゃ]に 行[い]ったほうがいい。	
\\	洗濯に液体の洗剤を使っています。	液体	洗濯[せんたく]に 液体[えきたい]の 洗剤[せんざい]を 使[つか]っています。	
\\	心臓は全身に血液を送り出している。	血液	心臓[しんぞう]は 全身[ぜんしん]に 血液[けつえき]を 送[おく]り 出[だ]している。	
\\	容器から液がこぼれていますよ。	液	容器[ようき]から 液[えき]がこぼれていますよ。	
\\	背中がかゆいです。	背中	背中[せなか]がかゆいです。	
\\	生徒たちががやがやと騒いでいますね。	がやがや	生徒[せいと]たちががやがやと 騒[さわ]いでいますね。	
\\	背広をクリーニングに出しました。	背広	背広[せびろ]をクリーニングに 出[だ]しました。	
\\	彼女はスキーで足を骨折したの。	骨折	彼女[かのじょ]はスキーで 足[あし]を 骨折[こっせつ]したの。	
\\	子供の健全な心を育てましょう。	健全	子供[こども]の 健全[けんぜん]な 心[こころ]を 育[そだ]てましょう。	
\\	健康が一番大切だ。	健康	健康[けんこう]が 一番大切[いちばん たいせつ]だ。	
\\	医師は異常なしと診断したんだ。	診断	医師[いし]は 異常[いじょう]なしと 診断[しんだん]したんだ。	
\\	今日、病院で診察してもらったの。	診察	今日[きょう]、 病院[びょういん]で 診察[しんさつ]してもらったの。	
\\	医療関係の仕事をしています。	医療	医療[いりょう] 関係[かんけい]の 仕事[しごと]をしています。	
\\	会場で子供がきょろきょろしているわよ。	きょろきょろ	会場[かいじょう]で 子供[こども]がきょろきょろしているわよ。	
\\	彼らは気の毒な生活をしている。	気の毒	彼[かれ]らは 気[き]の 毒[どく]な 生活[せいかつ]をしている。	
\\	足の傷を消毒したわ。	消毒	足[あし]の 傷[きず]を 消毒[しょうどく]したわ。	
\\	医者に症状を説明したよ。	症状	医者[いしゃ]に 症状[しょうじょう]を 説明[せつめい]したよ。	
\\	この手紙を清書してください。	清書	この 手紙[てがみ]を 清書[せいしょ]してください。	
\\	レストランは清潔が第一です。	清潔	レストランは 清潔[せいけつ]が 第一[だいいち]です。	
\\	その監督はアカデミー賞を受賞したよな。	監督	その 監督[かんとく]はアカデミー 賞[しょう]を 受賞[じゅしょう]したよな。	
\\	撮影は3ヶ月かけて行われました。	撮影	撮影[さつえい]は 3ヶ月[さんかげつ]かけて 行[おこな]われました。	
\\	彼は人物を描くのがうまいな。	描く	彼[かれ]は 人物[じんぶつ]を 描[えが]くのがうまいな。	
\\	風邪でくしゃみが止まりません。	くしゃみ	風邪[かぜ]でくしゃみが 止[と]まりません。	
\\	彼の活躍で優勝したよ。	活躍	彼[かれ]の 活躍[かつやく]で 優勝[ゆうしょう]したよ。	
\\	試合はすぐに再開されたの。	再開	試合[しあい]はすぐに 再開[さいかい]されたの。	
\\	留守番電話のメッセージを再生したの。	再生	留守番電話[るすばんでんわ]のメッセージを 再生[さいせい]したの。	
\\	彼らは互いに再会を喜んだの。	再会	彼[かれ]らは 互[たが]いに 再会[さいかい]を 喜[よろこ]んだの。	
\\	妹はマフラーを編みました。	編む	妹[いもうと]はマフラーを 編[あ]みました。	
\\	人質が解放されてよかった。	解放	人質[ひとじち]が 解放[かいほう]されてよかった。	
\\	その国は市場の開放を求められているの。	開放	その 国[くに]は 市場[しじょう]の 開放[かいほう]を 求[もと]められているの。	
\\	ぐずぐずしていると電車に乗り遅れますよ。	ぐずぐず	ぐずぐずしていると 電車[でんしゃ]に 乗[の]り 遅[おく]れますよ。	
\\	あの企業では学歴が重視されるよ。	重視	あの 企業[きぎょう]では 学歴[がくれき]が 重視[じゅうし]されるよ。	
\\	視点を変えて見てみましょう。	視点	視点[してん]を 変[か]えて 見[み]てみましょう。	
\\	彼は軽い近視です。	近視	彼[かれ]は 軽[かる]い 近視[きんし]です。	
\\	衣類の整理をしたよ。	衣類	衣類[いるい]の 整理[せいり]をしたよ。	
\\	彼の会社は衣服を扱っているよ。	衣服	彼[かれ]の 会社[かいしゃ]は 衣服[いふく]を 扱[あつか]っているよ。	
\\	その会社は衣料を扱っているんだ。	衣料	その 会社[かいしゃ]は 衣料[いりょう]を 扱[あつか]っているんだ。	
\\	土地によって衣食住は変化する。	衣食住	土地[とち]によって 衣食住[いしょくじゅう]は 変化[へんか]する。	
\\	犬に顔をなめられてくすぐったいよ。	くすぐったい	犬[いぬ]に 顔[かお]をなめられてくすぐったいよ。	
\\	仮にそれが事実だとしてももう遅いよ。	仮に	仮[かり]にそれが 事実[じじつ]だとしてももう 遅[おそ]いよ。	
\\	仮定の話だけでは結論は出ません。	仮定	仮定[かてい]の 話[はなし]だけでは 結論[けつろん]は 出[で]ません。	
\\	仮の申し込みをしました。	仮	仮[かり]の 申[もう]し 込[こ]みをしました。	
\\	祖父は古い仮名遣いで書くことがあるの。	仮名遣い	祖父[そふ]は 古[ふる]い 仮名遣[かなづか]いで 書[か]くことがあるの。	
\\	彼は地方で演説したのよ。	演説	彼[かれ]は 地方[ちほう]で 演説[えんぜつ]したのよ。	
\\	彼らの海外公演が発表されたよ。	公演	彼[かれ]らの 海外[かいがい] 公演[こうえん]が 発表[はっぴょう]されたよ。	
\\	彼はよく刑事の役を演じるね。	演じる	彼[かれ]はよく 刑事[けいじ]の 役[やく]を 演[えん]じるね。	
\\	彼女はその映画に出演しているよ。	出演	彼女[かのじょ]はその 映画[えいが]に 出演[しゅつえん]しているよ。	
\\	たくさん歩いてくたびれました。	くたびれる	たくさん 歩[ある]いてくたびれました。	
\\	小学校で子供たちの劇を見ました。	劇	小学校[しょうがっこう]で 子供[こども]たちの 劇[げき]を 見[み]ました。	
\\	午後8時に劇場の前で会いましょう。	劇場	午後8時[ごご はちじ]に 劇場[げきじょう]の 前[まえ]で 会[あ]いましょう。	
\\	彼女は演劇を学んでいます。	演劇	彼女[かのじょ]は 演劇[えんげき]を 学[まな]んでいます。	
\\	昨夜はテレビで喜劇を見たよ。	喜劇	昨夜[さくや]はテレビで 喜劇[きげき]を 見[み]たよ。	
\\	突然の悲しみが一家を襲いました。	悲しみ	突然[とつぜん]の 悲[かな]しみが 一家[いっか]を 襲[おそ]いました。	
\\	テロ集団が警察に捕まりました。	集団	テロ 集団[しゅうだん]が 警察[けいさつ]に 捕[つか]まりました。	
\\	棒をテープで固定しなさい。	固定	棒[ぼう]をテープで 固定[こてい]しなさい。	
\\	彼女は朝からいらいらしている。	いらいら	彼女[かのじょ]は 朝[あさ]からいらいらしている。	
\\	私はもう決心を固めたの。	固める	私[わたし]はもう 決心[けっしん]を 固[かた]めたの。	
\\	もうプリンは固まったかな。	固まる	もうプリンは 固[かた]まったかな。	
\\	氷は固体です。	固体	氷[こおり]は 固体[こたい]です。	
\\	これは日本固有の鳥です。	固有	これは 日本[にほん] 固有[こゆう]の 鳥[とり]です。	
\\	その芸術祭は市が主催しています。	主催	その 芸術祭[げいじゅつさい]は 市[し]が 主催[しゅさい]しています。	
\\	彼に集中するよう注意を促しといたよ。	促す	彼[かれ]に 集中[しゅうちゅう]するよう 注意[ちゅうい]を 促[うなが]しといたよ。	
\\	彼女に本を返すように催促したの。	催促	彼女[かのじょ]に 本[ほん]を 返[かえ]すように 催促[さいそく]したの。	
\\	エチケットを守ることは大切です。	エチケット	エチケットを 守[まも]ることは 大切[たいせつ]です。	
\\	私は古典を読むのが好きです。	古典	私[わたし]は 古典[こてん]を 読[よ]むのが 好[す]きです。	
\\	知らない言葉を辞典で調べました。	辞典	知[し]らない 言葉[ことば]を 辞典[じてん]で 調[しら]べました。	
\\	ロックは殊に若者に人気だ。	殊に	ロックは 殊[こと]に 若者[わかもの]に 人気[にんき]だ。	
\\	天皇は日本国の象徴です。	象徴	天皇[てんのう]は 日本国[にほんこく]の 象徴[しょうちょう]です。	
\\	階下から微かな音が聞こえた。	微か	階下[かいか]から 微[かす]かな 音[おと]が 聞[き]こえた。	
\\	遅れてご免。	ご免	遅[おく]れてご 免[めん]。	
\\	先生に許可をもらって早退しました。	許可	先生[せんせい]に 許可[きょか]をもらって 早退[そうたい]しました。	
\\	学校で避難訓練がありました。	訓練	学校[がっこう]で 避難[ひなん] 訓練[くんれん]がありました。	
\\	かっとなって余計なことを言ってしまったな。	かっと	かっとなって 余計[よけい]なことを 言[い]ってしまったな。	
\\	漢字には音読みと訓読みがあります。	訓読み	漢字[かんじ]には 音読[おんよ]みと 訓読[くんよ]みがあります。	
\\	漢字の読み方には、音と訓の2通りがあるの。	訓	漢字[かんじ]の 読[よ]み 方[かた]には、 音[おん]と 訓[くん]の 2通[ふたとお]りがあるの。	
\\	助詞を変えると文の意味が変わります。	助詞	助詞[じょし]を 変[か]えると 文[ぶん]の 意味[いみ]が 変[か]わります。	
\\	「大きい」は形容詞です。	形容詞	
\\	大[おお]きい」は 形容詞[けいようし]です。	
\\	上司に相談してみます。	上司	上司[じょうし]に 相談[そうだん]してみます。	
\\	彼は司会が上手ですね。	司会	彼[かれ]は 司会[しかい]が 上手[じょうず]ですね。	
\\	入学式の後、教科書を購入してください。	購入	入学式[にゅうがくしき]の 後[あと]、 教科書[きょうかしょ]を 購入[こうにゅう]してください。	
\\	誰かがガラガラとうがいをしているね。	がらがら	誰[だれ]かがガラガラとうがいをしているね。	
\\	彼の講演は評判がいい。	講演	彼[かれ]の 講演[こうえん]は 評判[ひょうばん]がいい。	
\\	彼の講義はとても分かりやすいですね。	講義	彼[かれ]の 講義[こうぎ]はとても 分[わ]かりやすいですね。	
\\	今日のフランス語の講座は休講です。	休講	今日[きょう]のフランス 語[ご]の 講座[こうざ]は 休講[きゅうこう]です。	
\\	医師に入院を勧められたの。	医師	医師[いし]に 入院[にゅういん]を 勧[すす]められたの。	
\\	専門家を講師に招いた。	講師	専門家[せんもんか]を 講師[こうし]に 招[まね]いた。	
\\	彼はレントゲン技師です。	技師	彼[かれ]はレントゲン 技師[ぎし]です。	
\\	師走に入ると忙しくなります。	師走	師走[しわす]に 入[はい]ると 忙[いそが]しくなります。	
\\	彼女はかんかんに怒っています。	かんかん	彼女[かのじょ]はかんかんに 怒[おこ]っています。	
\\	何かお手伝いしましょうか。	お手伝い	何[なに]かお 手伝[てつだ]いしましょうか。	
\\	誕生日に小鳥を買ってもらいました。	小鳥	誕生日[たんじょうび]に 小鳥[ことり]を 買[か]ってもらいました。	
\\	私たちは大声で歌を歌ったの。	大声	私[わたし]たちは 大声[おおごえ]で 歌[うた]を 歌[うた]ったの。	
\\	校舎から歌声が聞こえて来たよ。	歌声	校舎[こうしゃ]から 歌声[うたごえ]が 聞[き]こえて 来[き]たよ。	
\\	一体何の騒ぎですか。	騒ぎ	一体何[いったい なん]の 騒[さわ]ぎですか。	
\\	何だか表が騒がしい。	騒がしい	何[なん]だか 表[おもて]が 騒[さわ]がしい。	
\\	ペットを飼ったことはありますか。	飼う	ペットを 飼[か]ったことはありますか。	
\\	これまでの私とは違うんです。	これまで	これまでの 私[わたし]とは 違[ちが]うんです。	
\\	靴の底に釘が刺さってしまったんだ。	刺さる	靴[くつ]の 底[そこ]に 釘[くぎ]が 刺[さ]さってしまったんだ。	
\\	これから高齢化が急激に進みます。	急激	これから 高齢化[こうれいか]が 急激[きゅうげき]に 進[すす]みます。	
\\	そのクイズ番組は脳を刺激するね。	刺激	そのクイズ 番組[ばんぐみ]は 脳[のう]を 刺激[しげき]するね。	
\\	感激して泣いてしまいました。	感激	感激[かんげき]して 泣[な]いてしまいました。	
\\	私は歴史に興味があります。	興味	私[わたし]は 歴史[れきし]に 興味[きょうみ]があります。	
\\	彼女の優勝に感激し興奮しました。	興奮	彼女[かのじょ]の 優勝[ゆうしょう]に 感激[かんげき]し 興奮[こうふん]しました。	
\\	彼女は驚きを隠せませんでしたよ。	驚き	彼女[かのじょ]は 驚[おどろ]きを 隠[かく]せませんでしたよ。	
\\	あなたを驚かす話があります。	驚かす	あなたを 驚[おどろ]かす 話[はなし]があります。	
\\	梅雨時はじめじめする。	じめじめ	梅雨時[つゆどき]はじめじめする。	
\\	ようやく結論に至ったようね	至る	ようやく 結論[けつろん]に 至[いた]ったようね	
\\	コンビニは至る所にあります。	至る所	コンビニは 至[いた]る 所[ところ]にあります。	
\\	会社から「至急」との連絡があったんだ。	至急	会社[かいしゃ]から
\\	至急[しきゅう]」との 連絡[れんらく]があったんだ。	
\\	皆の意見が一致しました。	一致	皆[みんな]の 意見[いけん]が 一致[いっち]しました。	
\\	私からご連絡致します。	致す	私[わたし]からご 連絡[れんらく] 致[いた]します。	
\\	お日様が西に傾きましたね。	傾く	お 日様[ひさま]が 西[にし]に 傾[かたむ]きましたね。	
\\	彼女は首を少し傾けて笑うの。	傾ける	彼女[かのじょ]は 首[くび]を 少[すこ]し 傾[かたむ]けて 笑[わら]うの。	
\\	人をじろじろ見るものではありません。	じろじろ	人[ひと]をじろじろ 見[み]るものではありません。	
\\	この坂を上るのはすごくきついね。	坂	この 坂[さか]を 上[のぼ]るのはすごくきついね。	
\\	ここから道の幅が狭まっています。	狭まる	ここから 道[みち]の 幅[はば]が 狭[せば]まっています。	
\\	もっと範囲を狭めて探しましょう	狭める	もっと 範囲[はんい]を 狭[せば]めて 探[さが]しましょう	
\\	この先は徐行して下さい。	徐行	この 先[さき]は 徐行[じょこう]して 下[くだ]さい。	
\\	理科の時間に石の硬さを比べたよ。	硬さ	理科[りか]の 時間[じかん]に 石[いし]の 硬[かた]さを 比[くら]べたよ。	
\\	彼は柔軟に対応をした。	柔軟	彼[かれ]は 柔軟[じゅうなん]に 対応[たいおう]をした。	
\\	緊急事態が発生した。	緊急	緊急[きんきゅう] 事態[じたい]が 発生[はっせい]した。	
\\	ありの群れが砂糖にたかっている。	あり	ありの 群[む]れが 砂糖[さとう]にたかっている。	
\\	彼は緊張していたな。	緊張	彼[かれ]は 緊張[きんちょう]していたな。	
\\	その会社は店舗を拡張していますね。	拡張	その 会社[かいしゃ]は 店舗[てんぽ]を 拡張[かくちょう]していますね。	
\\	部長は京都に出張中です。	出張	部長[ぶちょう]は 京都[きょうと]に 出張[しゅっちょう] 中[ちゅう]です。	
\\	そこの角で車同士が衝突したのよ。	衝突	そこの 角[かど]で 車同士[くるま どうし]が 衝突[しょうとつ]したのよ。	
\\	工場の煙突から煙が上がっている。	煙突	工場[こうじょう]の 煙突[えんとつ]から 煙[けむり]が 上[あ]がっている。	
\\	彼女は昨日から私のことを避けているようなんだ。	避ける	彼女[かのじょ]は 昨日[きのう]から 私[わたし]のことを 避[さ]けているようなんだ。	
\\	彼は軍事に関わる仕事を続けてきたんだ。	軍事	彼[かれ]は 軍事[ぐんじ]に 関[かか]わる 仕事[しごと]を 続[つづ]けてきたんだ。	
\\	怪我人は軍の病院に運ばれたよ。	軍	怪我人[けがにん]は 軍[ぐん]の 病院[びょういん]に 運[はこ]ばれたよ。	
\\	彼の話に私はあいづちを打ったの。	あいづち	彼[かれ]の 話[はなし]に 私[わたし]はあいづちを 打[う]ったの。	
\\	自衛隊がイラクに派遣されたわ。	自衛隊	自衛隊[じえいたい]がイラクに 派遣[はけん]されたわ。	
\\	彼は軍隊に入ったよ。	軍隊	彼[かれ]は 軍隊[ぐんたい]に 入[はい]ったよ。	
\\	核戦争は絶対に防ぐべきよ。	核	核[かく] 戦争[せんそう]は 絶対[ぜったい]に 防[ふせ]ぐべきよ。	
\\	昔は結核でたくさんの人が亡くなった。	結核	昔[むかし]は 結核[けっかく]でたくさんの 人[ひと]が 亡[な]くなった。	
\\	大学では物理を専攻していました。	専攻	大学[だいがく]では 物理[ぶつり]を 専攻[せんこう]していました。	
\\	彼は積極的に攻めたが勝てなかったな。	攻める	彼[かれ]は 積極的[せっきょくてき]に 攻[せ]めたが 勝[か]てなかったな。	
\\	彼は拳銃で撃たれたわ。	撃つ	彼[かれ]は 拳銃[けんじゅう]で 撃[う]たれたわ。	
\\	池がかちかちに凍っています。	かちかち	池[いけ]がかちかちに 凍[こお]っています。	
\\	彼は悪酔いして暴れたんだ。	暴れる	彼[かれ]は 悪酔[わるよ]いして 暴[あば]れたんだ。	
\\	山で登山客が熊に襲われたよ。	襲う	山[やま]で 登山客[とざん きゃく]が 熊[くま]に 襲[おそ]われたよ。	
\\	そんなことは絶対できないよ。	絶対	そんなことは 絶対[ぜったい]できないよ。	
\\	彼女は驚いて気絶してしまったの。	気絶	彼女[かのじょ]は 驚[おどろ]いて 気絶[きぜつ]してしまったの。	
\\	消防士は勇敢でなければならない。	消防	消防[しょうぼう] 士[し]は 勇敢[ゆうかん]でなければならない。	
\\	父はラッシュアワーを嫌っています。	嫌う	父[ちち]はラッシュアワーを 嫌[きら]っています。	
\\	彼はタバコの煙を嫌がるの。	嫌がる	彼[かれ]はタバコの 煙[けむり]を 嫌[いや]がるの。	
\\	今日はたくさん歩いてくたくたです。	くたくた	今日[きょう]はたくさん 歩[ある]いてくたくたです。	
\\	彼女は大変機嫌がいいね。	機嫌	彼女[かのじょ]は 大変[たいへん] 機嫌[きげん]がいいね。	
\\	食べ物の好き嫌いは特にありません。	好き嫌い	食[た]べ 物[もの]の 好[す]き 嫌[きら]いは 特[とく]にありません。	
\\	彼の発言に対してたくさんの抗議があったよ。	抗議	彼[かれ]の 発言[はつげん]に 対[たい]してたくさんの 抗議[こうぎ]があったよ。	
\\	素敵なプレゼントをありがとう。	素敵	素敵[すてき]なプレゼントをありがとう。	
\\	脅かさないでよ。	脅かす	脅[おど]かさないでよ。	
\\	博士はその道の権威です。	権威	博士[はかせ]はその 道[みち]の 権威[けんい]です。	
\\	彼は威張ってなんかいません。	威張る	彼[かれ]は 威張[いば]ってなんかいません。	
\\	私は世界情勢を知るために毎日ニュースを見るわ。	情勢	私[わたし]は 世界[せかい] 情勢[じょうせい]を 知[し]るために 毎日[まいにち]ニュースを 見[み]るわ。	
\\	こしょうを入れ過ぎてスープが辛い。	こしょう	こしょうを 入[い]れ 過[す]ぎてスープが 辛[から]い。	
\\	この頃は、ローマが勢力を伸ばしていました。	勢力	この 頃[ころ]は、ローマが 勢力[せいりょく]を 伸[の]ばしていました。	
\\	そのチームには勢いがありますね。	勢い	そのチームには 勢[いきお]いがありますね。	
\\	叔父には心臓病の恐れがあります。	恐れ	叔父[おじ]には 心臓病[しんぞうびょう]の 恐[おそ]れがあります。	
\\	彼は受験の失敗を恐れています。	恐れる	彼[かれ]は 受験[じゅけん]の 失敗[しっぱい]を 恐[おそ]れています。	
\\	明日は恐らく晴れるでしょう。	恐らく	明日[あした]は 恐[おそ]らく 晴[は]れるでしょう。	
\\	彼女はクモを怖がります。	怖がる	彼女[かのじょ]はクモを 怖[こわ]がります。	
\\	あの巨大な建物は博物館です。	巨大	あの 巨大[きょだい]な 建物[たてもの]は 博物館[はくぶつかん]です。	
\\	彼女の肌はすべすべしているね。	すべすべ	彼女[かのじょ]の 肌[はだ]はすべすべしているね。	
\\	彼女は出席を拒否した。	拒否	彼女[かのじょ]は 出席[しゅっせき]を 拒否[きょひ]した。	
\\	高校卒業後、系列の大学に進みました。	系列	高校卒業後[こうこう そつぎょう ご]、 系列[けいれつ]の 大学[だいがく]に 進[すす]みました。	
\\	電気系統を図面で確かめました。	系統	電気[でんき] 系統[けいとう]を 図面[ずめん]で 確[たし]かめました。	
\\	彼は織田信長の子孫だよ。	子孫	彼[かれ]は 織田信長[おだのぶなが]の 子孫[しそん]だよ。	
\\	このシャツは絹でできています。	絹	このシャツは 絹[きぬ]でできています。	
\\	彼女は何とか健康を維持しているわね。	維持	彼女[かのじょ]は 何[なん]とか 健康[けんこう]を 維持[いじ]しているわね。	
\\	身なりのいい紳士が話しかけてきたの。	紳士	身[み]なりのいい 紳士[しんし]が 話[はな]しかけてきたの。	
\\	せっかく来たんだからゆっくりして行きなさい。	せっかく	せっかく 来[き]たんだからゆっくりして 行[い]きなさい。	
\\	彼は偉大な事業をなしとげました。	偉大	彼[かれ]は 偉大[いだい]な 事業[じぎょう]をなしとげました。	
\\	日本映画で刀を見た。	刀	日本映画[にほん えいが]で 刀[かたな]を 見[み]た。	
\\	彼は孤独な人生を送っていたんだ。	孤独	彼[かれ]は 孤独[こどく]な 人生[じんせい]を 送[おく]っていたんだ。	
\\	これまで話し合った事柄をまとめてください。	事柄	これまで 話[はな]し 合[あ]った 事柄[ことがら]をまとめてください。	
\\	彼は犬の柄の
\\	シャツを着ているよ。	柄	彼[かれ]は 犬[いぬ]の 柄[がら]の
\\	シャツを 着[き]ているよ。	
\\	この傘は柄が丈夫だな。	柄	この 傘[かさ]は 柄[え]が 丈夫[じょうぶ]だな。	
\\	強風で木の枝が折れた。	枝	強風[きょうふう]で 木[き]の 枝[えだ]が 折[お]れた。	
\\	彼の頑固さにはあきれました。	あきれる	彼[かれ]の 頑固[がんこ]さにはあきれました。	
\\	花瓶の花が枯れました。	枯れる	花瓶[かびん]の 花[はな]が 枯[か]れました。	
\\	外は木枯らしが吹いているよ。	木枯らし	外[そと]は 木枯[こが]らしが 吹[ふ]いているよ。	
\\	彼女の詩は世界中で有名になったんだ。	詩	彼女[かのじょ]の 詩[し]は 世界中[せかいじゅう]で 有名[ゆうめい]になったんだ。	
\\	彼は優れた詩人です。	詩人	彼[かれ]は 優[すぐ]れた 詩人[しじん]です。	
\\	彼はとても誠実な人です。	誠実	彼[かれ]はとても 誠実[せいじつ]な 人[ひと]です。	
\\	その都市は商業が盛んよ。	盛ん	その 都市[とし]は 商業[しょうぎょう]が 盛[さか]んよ。	
\\	そんなに謙遜しなくてもいい。	謙遜	そんなに 謙遜[けんそん]しなくてもいい。	
\\	彼女は秘書と事務の担当を兼ねています。	兼ねる	彼女[かのじょ]は 秘書[ひしょ]と 事務[じむ]の 担当[たんとう]を 兼[か]ねています。	
\\	彼のあだなは「りき」です。	あだな	彼[かれ]のあだなは「りき」です。	
\\	鋭い刃物で指を怪我した。	鋭い	鋭[するど]い 刃物[はもの]で 指[ゆび]を 怪我[けが]した。	
\\	犬を鎖でつなぎました。	鎖	犬[いぬ]を 鎖[くさり]でつなぎました。	
\\	今夜はオペラ鑑賞に行きます。	鑑賞	今夜[こんや]はオペラ 鑑賞[かんしょう]に 行[い]きます。	
\\	ここに受け取りの印鑑をお願いします。	印鑑	ここに 受[う]け 取[と]りの 印鑑[いんかん]をお 願[ねが]いします。	
\\	この鐘の音はとてもきれいですね。	鐘	この 鐘[かね]の 音[ね]はとてもきれいですね。	
\\	亀の寿命は長いんだ。	寿命	亀[かめ]の 寿命[じゅみょう]は 長[なが]いんだ。	
\\	夏休みには海水浴に行きます。	海水浴	夏休[なつやす]みには 海水浴[かいすいよく]に 行[い]きます。	
\\	バスタブからお湯があふれました。	あふれる	バスタブからお 湯[ゆ]があふれました。	
\\	今日は沿岸の波が荒いでしょう。	沿岸	今日[きょう]は 沿岸[えんがん]の 波[なみ]が 荒[あら]いでしょう。	
\\	朝早く砂浜を散歩しました。	砂浜	朝早[あさ はや]く 砂浜[すなはま]を 散歩[さんぽ]しました。	
\\	沖に小島が見えます。	沖	沖[おき]に 小島[こじま]が 見[み]えます。	
\\	森の中にきれいな泉があるの。	泉	森[もり]の 中[なか]にきれいな 泉[いずみ]があるの。	
\\	ここの温泉はよく効くそうです。	温泉	ここの 温泉[おんせん]はよく 効[き]くそうです。	
\\	山の空気は澄んでいますね。	澄む	山[やま]の 空気[くうき]は 澄[す]んでいますね。	
\\	彼女の心の叫びに誰も気付かなかったよ。	叫び	彼女[かのじょ]の 心[こころ]の 叫[さけ]びに 誰[だれ]も 気付[きづ]かなかったよ。	
\\	私はイカの刺し身が大好きです。	いか	私[わたし]はイカの 刺[さ]し 身[み]が 大好[だいす]きです。	
\\	彼女は助けを求めて大声で叫んだの。	叫ぶ	彼女[かのじょ]は 助[たす]けを 求[もと]めて 大声[おおごえ]で 叫[さけ]んだの。	
\\	ここでは喫煙できません。	喫煙	ここでは 喫煙[きつえん]できません。	
\\	彼は仕事に命を懸けているの。	懸ける	彼[かれ]は 仕事[しごと]に 命[いのち]を 懸[か]けているの。	
\\	このご恩は一生忘れません。	恩	このご 恩[おん]は 一生忘[いっしょう わす]れません。	
\\	夕方の市街は車が渋滞するよ。	市街	夕方[ゆうがた]の 市街[しがい]は 車[くるま]が 渋滞[じゅうたい]するよ。	
\\	その白い粉は赤ちゃんのミルクです。	粉	その 白[しろ]い 粉[こな]は 赤[あか]ちゃんのミルクです。	
\\	小麦は色々な食べ物に使われている。	小麦	小麦[こむぎ]は 色々[いろいろ]な 食[た]べ 物[もの]に 使[つか]われている。	
\\	うどんは小麦粉から作られます。	小麦粉	うどんは 小麦粉[こむぎこ]から 作[つく]られます。	
\\	彼はいつのまにいなくなったの。	いつのまに	彼[かれ]はいつのまにいなくなったの。	
\\	大麦はビールの原料になります。	大麦	大麦[おおむぎ]はビールの 原料[げんりょう]になります。	
\\	うちでは夫も炊事をします。	炊事	うちでは 夫[おっと]も 炊事[すいじ]をします。	
\\	暦の上では今日から冬ですね。	暦	暦[こよみ]の 上[うえ]では 今日[きょう]から 冬[ふゆ]ですね。	
\\	りんごの皮をむきましたよ。	皮	りんごの 皮[かわ]をむきましたよ。	
\\	彼女は毛皮のコートを着ていたの。	毛皮	彼女[かのじょ]は 毛皮[けがわ]のコートを 着[き]ていたの。	
\\	彼女は毛糸のセーターを編みました。	毛糸	彼女[かのじょ]は 毛糸[けいと]のセーターを 編[あ]みました。	
\\	尾の長い鳥が飛んでいますね。	尾	尾[お]の 長[なが]い 鳥[とり]が 飛[と]んでいますね。	
\\	市場にはインフレーションの影響が出ているね。	インフレーション	市場[しじょう]にはインフレーションの 影響[えいきょう]が 出[で]ているね。	
\\	祖父は白髪が少ないほうです。	白髪	祖父[そふ]は 白髪[しらが]が 少[すく]ないほうです。	
\\	今、耳鼻科にかかっています。	耳鼻科	今[いま]、 耳鼻科[じびか]にかかっています。	
\\	寒くて唇が青くなってしまった。	唇	寒[さむ]くて 唇[くちびる]が 青[あお]くなってしまった。	
\\	この詩を解釈してみましょう。	解釈	この 詩[し]を 解釈[かいしゃく]してみましょう。	
\\	父は居間でテレビを見ている。	居間	父[ちち]は 居間[いま]でテレビを 見[み]ている。	
\\	母は今、うちに居りません。	居る	母[はは]は 今[いま]、うちに 居[お]りません。	
\\	雨が一層激しくなったね。	一層	雨[あめ]が 一層[いっそう] 激[はげ]しくなったね。	
\\	今度のウィークエンドは映画を見ます。	ウィークエンド	今度[こんど]のウィークエンドは 映画[えいが]を 見[み]ます。	
\\	このあたりは高層ビルが増えましたね。	高層	このあたりは 高層[こうそう]ビルが 増[ふ]えましたね。	
\\	雲が空を覆っているね。	覆う	雲[くも]が 空[そら]を 覆[おお]っているね。	
\\	結婚すると新しい戸籍が作られます。	戸籍	結婚[けっこん]すると 新[あたら]しい 戸籍[こせき]が 作[つく]られます。	
\\	お店のガラス戸が割られたんだ。	ガラス戸	お 店[みせ]のガラス 戸[ど]が 割[わ]られたんだ。	
\\	この店では新鮮な野菜が買えますよ。	新鮮	この 店[みせ]では 新鮮[しんせん]な 野菜[やさい]が 買[か]えますよ。	
\\	山は鮮やかな緑でした。	鮮やか	山[やま]は 鮮[あざ]やかな 緑[みどり]でした。	
\\	数百人の群集が集まっているの。	群集	数百人[すうひゃくにん]の 群集[ぐんしゅう]が 集[あつ]まっているの。	
\\	この郡は来年、市になりますよ。	郡	この 郡[ぐん]は 来年[らいねん]、 市[し]になりますよ。	
\\	冬はまめにうがいをします。	うがい	冬[ふゆ]はまめにうがいをします。	
\\	この本を君にあげます。	君	この 本[ほん]を 君[きみ]にあげます。	
\\	この肉は脂が多い。	脂	この 肉[にく]は 脂[あぶら]が 多[おお]い。	
\\	お腹の脂肪を取りたい。	脂肪	お 腹[なか]の 脂肪[しぼう]を 取[と]りたい。	
\\	彼は胃腸が強くありません。	胃腸	彼[かれ]は 胃腸[いちょう]が 強[つよ]くありません。	
\\	何事も最初が肝心だ。	肝心	何事[なにごと]も 最初[さいしょ]が 肝心[かんじん]だ。	
\\	列車の窓から雄大な山脈が見えたんだよ。	山脈	列車[れっしゃ]の 窓[まど]から 雄大[ゆうだい]な 山脈[さんみゃく]が 見[み]えたんだよ。	
\\	将来は海の近くで暮らしたいな。	暮らす	将来[しょうらい]は 海[うみ]の 近[ちか]くで 暮[く]らしたいな。	
\\	もうすぐ夏休みなのでうきうきしています。	うきうき	もうすぐ 夏休[なつやす]みなのでうきうきしています。	
\\	彼女は毎日の暮らしを楽しんでいますね。	暮らし	彼女[かのじょ]は 毎日[まいにち]の 暮[く]らしを 楽[たの]しんでいますね。	
\\	暮れは用事が多くて忙しいです。	暮れ	暮[く]れは 用事[ようじ]が 多[おお]くて 忙[いそが]しいです。	
\\	日が暮れる前に帰りましょう。	暮れる	日[ひ]が 暮[く]れる 前[まえ]に 帰[かえ]りましょう。	
\\	デパートでお歳暮を送りました。	お歳暮	デパートでお 歳暮[せいぼ]を 送[おく]りました。	
\\	芝生がよく手入れされていますね。	芝生	芝生[しばふ]がよく 手入[てい]れされていますね。	
\\	チューリップは茎が長いです。	茎	チューリップは 茎[くき]が 長[なが]いです。	
\\	彼は会長に推薦されたよ。	推薦	彼[かれ]は 会長[かいちょう]に 推薦[すいせん]されたよ。	
\\	彼はうつむいて何かを考えているね。	うつむく	彼[かれ]はうつむいて 何[なに]かを 考[かんが]えているね。	
\\	台風で山も海も荒れています。	荒れる	台風[たいふう]で 山[やま]も 海[うみ]も 荒[あ]れています。	
\\	彼は金遣いが荒いな。	荒い	彼[かれ]は 金遣[かねづか]いが 荒[あら]いな。	
\\	うちの犬は気が荒い。	荒い	うちの 犬[いぬ]は 気[き]が 荒[あら]い。	
\\	彼は強盗に家の中を荒されたの。	荒す	彼[かれ]は 強盗[ごうとう]に 家[いえ]の 中[なか]を 荒[あら]されたの。	
\\	今日は慌ただしい一日でした。	慌ただしい	今日[きょう]は 慌[あわ]ただしい 一日[いちにち]でした。	
\\	そんなに慌ててどこに行くの。	慌てる	そんなに 慌[あわ]ててどこに 行[い]くの。	
\\	儀式には作法があります。	儀式	儀式[ぎしき]には 作法[さほう]があります。	
\\	彼女はうなずいて同意を示したの。	うなずく	彼女[かのじょ]はうなずいて 同意[どうい]を 示[しめ]したの。	
\\	この子は本当に行儀の良い子です。	行儀	この 子[こ]は 本当[ほんとう]に 行儀[ぎょうぎ]の 良[い]い 子[こ]です。	
\\	多くの人が災害の犠牲になったの。	犠牲	多[おお]くの 人[ひと]が 災害[さいがい]の 犠牲[ぎせい]になったの。	
\\	その頃彼はまだ赤ん坊だったよね。	赤ん坊	その 頃彼[ころ かれ]はまだ 赤[あか]ん 坊[ぼう]だったよね。	
\\	お寺からお坊さんが出て来ましたよ。	お坊さん	お 寺[てら]からお 坊[ぼう]さんが 出[で]て 来[き]ましたよ。	
\\	日本の国旗は描くのが簡単です。	国旗	日本[にほん]の 国旗[こっき]は 描[か]くのが 簡単[かんたん]です。	
\\	彼は最近筋肉を鍛えているよ。	筋肉	彼[かれ]は 最近[さいきん] 筋肉[きんにく]を 鍛[きた]えているよ。	
\\	筋道を立てて考えなさい。	筋道	筋道[すじみち]を 立[た]てて 考[かんが]えなさい。	
\\	転んで筋を痛めてしまいました。	筋	転[ころ]んで 筋[すじ]を 痛[いた]めてしまいました。	
\\	彼は人をおだてるのが上手です。	おだてる	彼[かれ]は 人[ひと]をおだてるのが 上手[じょうず]です。	
\\	釘に引っかけて袖が裂けてしまったな。	裂ける	釘[くぎ]に 引[ひ]っかけて 袖[そで]が 裂[さ]けてしまったな。	
\\	この座布団は座り心地がいいね。	座布団	この 座布団[ざぶとん]は 座[すわ]り 心地[ごこち]がいいね。	
\\	これからは自己の判断で行動してください。	自己	これからは 自己[じこ]の 判断[はんだん]で 行動[こうどう]してください。	
\\	彼は大きな荷物を抱えているわ。	抱える	彼[かれ]は 大[おお]きな 荷物[にもつ]を 抱[かか]えているわ。	
\\	少年よ大志を抱け。	抱く	少年[しょうねん]よ 大志[たいし]を 抱[いだ]け。	
\\	この句はどんな意味でしょうか。	句	この 句[く]はどんな 意味[いみ]でしょうか。	
\\	「手が空く」は慣用句です。	慣用句	
\\	手[て]が 空[す]く」は 慣用句[かんようく]です。	
\\	靴のかかとで彼の足を踏んでしまったの。	かかと	靴[くつ]のかかとで 彼[かれ]の 足[あし]を 踏[ふ]んでしまったの。	
\\	この仕事は来月の下旬には終わります。	下旬	この 仕事[しごと]は 来月[らいげつ]の 下旬[げじゅん]には 終[お]わります。	
\\	7月上旬に夏祭りがありますよ。	上旬	7月[しちがつ] 上旬[じょうじゅん]に 夏祭[なつまつ]りがありますよ。	
\\	その子はおもちゃに飽きたようだね。	飽きる	その 子[こ]はおもちゃに 飽[あ]きたようだね。	
\\	この部屋は飾りが多すぎるな。	飾り	この 部屋[へや]は 飾[かざ]りが 多[おお]すぎるな。	
\\	毎日金魚に餌をやります。	餌	毎日金魚[まいにち きんぎょ]に 餌[えさ]をやります。	
\\	旧ソビエトは今はロシアと呼ばれている。	旧	旧[きゅう]ソビエトは 今[いま]はロシアと 呼[よ]ばれている。	
\\	彼女は育児で忙しいよ。	育児	彼女[かのじょ]は 育児[いくじ]で 忙[いそが]しいよ。	
\\	玄関でがさがさと音がしましたよ。	がさがさ	玄関[げんかん]でがさがさと 音[おと]がしましたよ。	
\\	子供を小児科に連れて行くところです。	小児科	子供[こども]を 小児科[しょうにか]に 連[つ]れて 行[い]くところです。	
\\	あなたの姓名を教えてください。	姓名	あなたの 姓名[せいめい]を 教[おし]えてください。	
\\	結婚して姓が変わりました。	姓	結婚[けっこん]して 姓[せい]が 変[か]わりました。	
\\	彼女には幼い息子がいます。	幼い	彼女[かのじょ]には 幼[おさな]い 息子[むすこ]がいます。	
\\	眼科で視力検査をしました。	眼科	眼科[がんか]で 視力検査[しりょく けんさ]をしました。	
\\	彼女は近眼です。	近眼	彼女[かのじょ]は 近眼[きんがん]です。	
\\	彼はソファーで居眠りをしているよ。	居眠り	彼[かれ]はソファーで 居眠[いねむ]りをしているよ。	
\\	一瞬自分の耳を疑ったよ。	一瞬	一瞬[いっしゅん] 自分[じぶん]の 耳[みみ]を 疑[うたが]ったよ。	
\\	弟は美味しそうにりんごをかじっていたよ。	かじる	弟[おとうと]は 美味[おい]しそうにりんごをかじっていたよ。	
\\	その瞬間、猫が跳び出しました。	瞬間	その 瞬間[しゅんかん]、 猫[ねこ]が 跳[と]び 出[だ]しました。	
\\	1日7時間は睡眠を取るようにしています。	睡眠	1日7時間[いちにち しちじかん]は 睡眠[すいみん]を 取[と]るようにしています。	
\\	彼らは垂直のがけを登り始めたの。	垂直	彼[かれ]らは 垂直[すいちょく]のがけを 登[のぼ]り 始[はじ]めたの。	
\\	安全運転を心掛けてください。	心掛ける	安全運転[あんぜん うんてん]を 心掛[こころが]けてください。	
\\	私はベンチに腰掛けたの。	腰掛ける	私[わたし]はベンチに 腰掛[こしか]けたの。	
\\	パトカーがバイクを追いかけてるぞ。	追い掛ける	パトカーがバイクを 追[お]いかけてるぞ。	
\\	腰掛けはありませんか。	腰掛け	腰掛[こしか]けはありませんか。	
\\	彼は水をがぶがぶ飲んだんだ。	がぶがぶ	彼[かれ]は 水[みず]をがぶがぶ 飲[の]んだんだ。	
\\	お目に掛かれて嬉しいです。	お目に掛かる	お 目[め]に 掛[か]かれて 嬉[うれ]しいです。	
\\	仏像に手を合わせて拝みました。	拝む	仏像[ぶつぞう]に 手[て]を 合[あ]わせて 拝[おが]みました。	
\\	括弧の部分は省略できます。	括弧	括弧[かっこ]の 部分[ぶぶん]は 省略[しょうりゃく]できます。	
\\	彼がそのプロジェクトの指揮を取ったの。	指揮	彼[かれ]がそのプロジェクトの 指揮[しき]を 取[と]ったの。	
\\	彼女の瞳は喜びで輝いていますね。	輝く	彼女[かのじょ]の 瞳[ひとみ]は 喜[よろこ]びで 輝[かがや]いていますね。	
\\	彼は怒りを抑えていたの。	抑える	彼[かれ]は 怒[いか]りを 抑[おさ]えていたの。	
\\	信仰は心の支えです。	信仰	信仰[しんこう]は 心[こころ]の 支[ささ]えです。	
\\	木の上でからすが鳴いています。	からす	木[き]の 上[うえ]でからすが 鳴[な]いています。	
\\	街で偶然友人に会いました。	偶然	街[まち]で 偶然[ぐうぜん] 友人[ゆうじん]に 会[あ]いました。	
\\	偶数は2で割り切れます。	偶数	偶数[ぐうすう]は 2[に]で 割[わ]り 切[き]れます。	
\\	部屋の隅にいすが積んであったの。	隅	部屋[へや]の 隅[すみ]にいすが 積[つ]んであったの。	
\\	今回の旅行ではヨーロッパの城を見て回ります。	城	今回[こんかい]の 旅行[りょこう]ではヨーロッパの 城[しろ]を 見[み]て 回[まわ]ります。	
\\	この地方ではみかんの栽培が盛んです。	栽培	この 地方[ちほう]ではみかんの 栽培[さいばい]が 盛[さか]んです。	
\\	地震のあとは津波に警戒してください。	警戒	地震[じしん]のあとは 津波[つなみ]に 警戒[けいかい]してください。	
\\	幾ら呼んでも彼は返事をしなかったわ。	幾ら	幾[いく]ら 呼[よ]んでも 彼[かれ]は 返事[へんじ]をしなかったわ。	
\\	お話しを伺いたいのですが。	伺う	お 話[はな]しを 伺[うかが]いたいのですが。	
\\	祖父はがんで亡くなりました。	がん	祖父[そふ]はがんで 亡[な]くなりました。	
\\	後悔しても、しょうがない。	後悔	後悔[こうかい]しても、しょうがない。	
\\	試合に負けてとても悔しい。	悔しい	試合[しあい]に 負[ま]けてとても 悔[くや]しい。	
\\	俺の頼みを聞いてくれ。	俺	俺[おれ]の 頼[たの]みを 聞[き]いてくれ。	
\\	お相撲さんは、みんな力持ちだ。	相撲	お 相撲[すもう]さんは、みんな 力[ちから] 持[も]ちだ。	
\\	栄養が偏らないように食事に気を付けています。	偏る	栄養[えいよう]が 偏[かたよ]らないように 食事[しょくじ]に 気[き]を 付[つ]けています。	
\\	春は一遍に花が咲く。	一遍に	春[はる]は 一遍[いっぺん]に 花[はな]が 咲[さ]く。	
\\	宗教を持たない人もたくさんいるわ。	宗教	宗教[しゅうきょう]を 持[も]たない 人[ひと]もたくさんいるわ。	
\\	この箱にはぎっしり物が詰まっています。	ぎっしり	この 箱[はこ]にはぎっしり 物[もの]が 詰[つ]まっています。	
\\	その問題は審議中です。	審議	その 問題[もんだい]は 審議[しんぎ] 中[ちゅう]です。	
\\	憲法を改正することは難しいわね。	憲法	憲法[けんぽう]を 改正[かいせい]することは 難[むずか]しいわね。	
\\	年と共に体力が衰えています。	衰える	年[とし]と 共[とも]に 体力[たいりょく]が 衰[おとろ]えています。	
\\	私たちはみな彼の死を惜しんだの。	惜しむ	私[わたし]たちはみな 彼[かれ]の 死[し]を 惜[お]しんだの。	
\\	惜しい、もう少しで優勝だった。	惜しい	惜[お]しい、もう 少[すこ]しで 優勝[ゆうしょう]だった。	
\\	彼女は長年の恨みを晴らした。	恨み	彼女[かのじょ]は 長年[ながねん]の 恨[うら]みを 晴[は]らした。	
\\	彼を恨んではいけません。	恨む	彼[かれ]を 恨[うら]んではいけません。	
\\	電車がぎゅうぎゅうに込んでいますね。	ぎゅうぎゅう	電車[でんしゃ]がぎゅうぎゅうに 込[こ]んでいますね。	
\\	覚悟はできています。	覚悟	覚悟[かくご]はできています。	
\\	その男の行動は怪しかったわ。	怪しい	その 男[おとこ]の 行動[こうどう]は 怪[あや]しかったわ。	
\\	警察はそのグループを怪しんでいます。	怪しむ	警察[けいさつ]はそのグループを 怪[あや]しんでいます。	
\\	彼女はよく自分の成績を自慢するよね。	自慢	彼女[かのじょ]はよく 自分[じぶん]の 成績[せいせき]を 自慢[じまん]するよね。	
\\	食卓に花を飾りましょう。	食卓	食卓[しょくたく]に 花[はな]を 飾[かざ]りましょう。	
\\	温かい歓迎を受けました。	歓迎	温[あたた]かい 歓迎[かんげい]を 受[う]けました。	
\\	勧められて欲しくもないものを買っちゃったよ。	勧める	勧[すす]められて 欲[ほ]しくもないものを 買[か]っちゃったよ。	
\\	ここのぎょうざは美味しいよ。	ぎょうざ	ここのぎょうざは 美味[おい]しいよ。	
\\	焦点を絞って話しましょう。	焦点	焦点[しょうてん]を 絞[しぼ]って 話[はな]しましょう。	
\\	何をそんなに焦っているのですか。	焦る	何[なに]をそんなに 焦[あせ]っているのですか。	
\\	シチューが焦げたよー。	焦げる	シチューが 焦[こ]げたよー。	
\\	彼は国の英雄ですね。	英雄	彼[かれ]は 国[くに]の 英雄[えいゆう]ですね。	
\\	うちの猫は雄です。	雄	うちの 猫[ねこ]は 雄[おす]です。	
\\	その選手の獲得賞金は2億円だって。	獲得	その 選手[せんしゅ]の 獲得[かくとく] 賞金[しょうきん]は 2億円[におくえん]だって。	
\\	お米の収穫が始まったね。	収穫	お 米[こめ]の 収穫[しゅうかく]が 始[はじ]まったね。	
\\	日本は稲の品種が多いよ。	稲	日本[にっぽん]は 稲[いね]の 品種[ひんしゅ]が 多[おお]いよ。	
\\	余ったきれで人形を作りましょう。	きれ	余[あま]ったきれで 人形[にんぎょう]を 作[つく]りましょう。	
\\	彼はお金をだいぶ稼いだらしいよ。	稼ぐ	彼[かれ]はお 金[かね]をだいぶ 稼[かせ]いだらしいよ。	
\\	作文を書くのに原稿用紙を使ったの。	原稿用紙	作文[さくぶん]を 書[か]くのに 原稿用紙[げんこうようし]を 使[つか]ったの。	
\\	彼らは穏やかな暮らしをしているの。	穏やか	彼[かれ]らは 穏[おだ]やかな 暮[く]らしをしているの。	
\\	僕に隠していることは無いですか。	隠す	僕[ぼく]に 隠[かく]していることは 無[な]いですか。	
\\	先生のお陰で大学に合格できました。	お陰	先生[せんせい]のお 陰[かげ]で 大学[だいがく]に 合格[ごうかく]できました。	
\\	陰で少し休憩しましょう。	陰	陰[かげ]で 少[すこ]し 休憩[きゅうけい]しましょう。	
\\	道に土の塊ができてたよ。	塊	道[みち]に 土[つち]の 塊[かたまり]ができてたよ。	
\\	弟に足の裏をくすぐられた。	くすぐる	弟[おとうと]に 足[あし]の 裏[うら]をくすぐられた。	
\\	彼は金に卑しいね。	卑しい	彼[かれ]は 金[かね]に 卑[いや]しいね。	
\\	氷を細かく砕いてください。	砕く	氷[こおり]を 細[こま]かく 砕[くだ]いてください。	
\\	石が粉々に砕けましたね。	砕ける	石[いし]が 粉々[こなごな]に 砕[くだ]けましたね。	
\\	今ドイツ語の基礎を学んでいます。	基礎	今[いま]ドイツ 語[ご]の 基礎[きそ]を 学[まな]んでいます。	
\\	浜辺できれいな貝を拾いました。	貝	浜辺[はまべ]できれいな 貝[かい]を 拾[ひろ]いました。	
\\	彼は冷静で賢い男だね。	賢い	彼[かれ]は 冷静[れいせい]で 賢[かしこ]い 男[おとこ]だね。	
\\	彼は頻りに時計を気にしていた。	頻りに	彼[かれ]は 頻[しき]りに 時計[とけい]を 気[き]にしていた。	
\\	彼女はいつまでもげらげら笑っていたな。	げらげら	彼女[かのじょ]はいつまでもげらげら 笑[わら]っていたな。	
\\	なんて頑固な子だ。	頑固	なんて 頑固[がんこ]な 子[こ]だ。	
\\	お客様にお菓子を頂いた。	頂く	お 客様[きゃくさま]にお 菓子[かし]を 頂[いただ]いた。	
\\	彼らは一斉に拍手したね。	一斉に	彼[かれ]らは 一斉[いっせい]に 拍手[はくしゅ]したね。	
\\	床に洗剤をこぼしてしまいました。	洗剤	床[ゆか]に 洗剤[せんざい]をこぼしてしまいました。	
\\	彼は真剣に話を聞いていました。	真剣	彼[かれ]は 真剣[しんけん]に 話[はなし]を 聞[き]いていました。	
\\	彼は二度と刑務所から出ることができないの。	刑務所	彼[かれ]は 二度[にど]と 刑務所[けいむしょ]から 出[で]ることができないの。	
\\	刑事が現場を調べているわ。	刑事	刑事[けいじ]が 現場[げんば]を 調[しら]べているわ。	
\\	彼女は強烈な個性の持ち主ですよ。	強烈	彼女[かのじょ]は 強烈[きょうれつ]な 個性[こせい]の 持[も]ち 主[ぬし]ですよ。	
\\	ゴールデンウィークに海外旅行をします。	ゴールデンウィーク	ゴールデンウィークに 海外旅行[かいがい りょこう]をします。	
\\	彼は獣のような目をしていたな。	獣	彼[かれ]は 獣[けもの]のような 目[め]をしていたな。	
\\	子供たちは広場に駆けて行ったわよ。	駆ける	子供[こども]たちは 広場[ひろば]に 駆[か]けて 行[い]ったわよ。	
\\	駅まで駆け足で行ったよ。	駆け足	駅[えき]まで 駆[か]け 足[あし]で 行[い]ったよ。	
\\	丘の上にホテルが建ちましたね。	丘	丘[おか]の 上[うえ]にホテルが 建[た]ちましたね。	
\\	その古い建物は官庁です。	官庁	その 古[ふる]い 建物[たてもの]は 官庁[かんちょう]です。	
\\	あの白い建物が県庁です。	県庁	あの 白[しろ]い 建物[たてもの]が 県庁[けんちょう]です。	
\\	冷えた手を擦って温めた。	擦る	冷[ひ]えた 手[て]を 擦[こす]って 温[あたた]めた。	
\\	寒いのでこたつを出しました。	こたつ	寒[さむ]いのでこたつを 出[だ]しました。	
\\	彼は額に汗をかいていたの。	汗	彼[かれ]は 額[ひたい]に 汗[あせ]をかいていたの。	
\\	この線を軸にして図形を回転してください。	軸	この 線[せん]を 軸[じく]にして 図形[ずけい]を 回転[かいてん]してください。	
\\	彼は大学の後輩です。	後輩	彼[かれ]は 大学[だいがく]の 後輩[こうはい]です。	
\\	この花はいい香りがしますね。	香り	この 花[はな]はいい 香[かお]りがしますね。	
\\	香水のいい香りがした。	香水	香水[こうすい]のいい 香[かお]りがした。	
\\	私たちは校歌を合唱したの。	合唱	私[わたし]たちは 校歌[こうか]を 合唱[がっしょう]したの。	
\\	雪の結晶にはいろいろな形があります。	結晶	雪[ゆき]の 結晶[けっしょう]にはいろいろな 形[かたち]があります。	
\\	日本のことわざをいくつくらい知っていますか。	ことわざ	日本[にほん]のことわざをいくつくらい 知[し]っていますか。	
\\	両親を敬うことは大切です。	敬う	両親[りょうしん]を 敬[うやま]うことは 大切[たいせつ]です。	
\\	明日はお座敷での宴会になります。	座敷	明日[あした]はお 座敷[ざしき]での 宴会[えんかい]になります。	
\\	マンションの敷金を払いました。	敷金	マンションの 敷金[しききん]を 払[はら]いました。	
\\	私は体力では誰にも劣りません。	劣る	私[わたし]は 体力[たいりょく]では 誰[だれ]にも 劣[おと]りません。	
\\	お勘定をして下さい。	勘定	お 勘定[かんじょう]をして 下[くだ]さい。	
\\	女の勘を甘く見てはいけません。	勘	女[おんな]の 勘[かん]を 甘[あま]く 見[み]てはいけません。	
\\	待ち合わせは2時だと勘違いしていました。	勘違い	待[ま]ち 合[あ]わせは 2時[にじ]だと 勘違[かんちが]いしていました。	
\\	霧の深い夜のことでした。	霧	霧[きり]の 深[ふか]い 夜[よる]のことでした。	
\\	これからはもっと気を付けます。	これから	これからはもっと 気[き]を 付[つ]けます。	
\\	今朝は庭の草に霜が降りていたの。	霜	今朝[けさ]は 庭[にわ]の 草[くさ]に 霜[しも]が 降[お]りていたの。	
\\	彼は公衆電話を探したの。	公衆	彼[かれ]は 公衆[こうしゅう] 電話[でんわ]を 探[さが]したの。	
\\	予算案が衆議院を通過したな。	衆議院	予算案[よさんあん]が 衆議院[しゅうぎいん]を 通過[つうか]したな。	
\\	スタジアムは観衆で満員だったよ。	観衆	スタジアムは 観衆[かんしゅう]で 満員[まんいん]だったよ。	
\\	彼はマイクで群衆に話しかけたんだ。	群衆	彼[かれ]はマイクで 群衆[ぐんしゅう]に 話[はな]しかけたんだ。	
\\	先生に暑中見舞いを出しました。	暑中見舞い	先生[せんせい]に 暑中見舞[しょちゅうみま]いを 出[だ]しました。	
\\	話はこれでお仕舞いです。	お仕舞い	話[はなし]はこれでお 仕舞[しま]いです。	
\\	自転車に油をさしたの。	さす	自転車[じてんしゃ]に 油[あぶら]をさしたの。	
\\	金銭のトラブルには関わりたくありません。	金銭	金銭[きんせん]のトラブルには 関[かか]わりたくありません。	
\\	ここは児童の通学路です。	児童	ここは 児童[じどう]の 通学路[つうがくろ]です。	
\\	庭に穴を掘ってそれを埋めました。	埋める	庭[にわ]に 穴[あな]を 掘[ほ]ってそれを 埋[う]めました。	
\\	パレードと観衆が道を埋めていたよ。	埋める	パレードと 観衆[かんしゅう]が 道[みち]を 埋[うず]めていたよ。	
\\	服に墨がついちゃった。	墨	服[ふく]に 墨[すみ]がついちゃった。	
\\	全員、講堂に集まってください。	講堂	全員[ぜんいん]、 講堂[こうどう]に 集[あつ]まってください。	
\\	彼女は奨学金で大学に行きました。	奨学金	彼女[かのじょ]は 奨学金[しょうがくきん]で 大学[だいがく]に 行[い]きました。	
\\	私は通勤に
\\	を使うんだ。	
\\	私[わたし]は 通勤[つうきん]に 
\\	[じぇいあーる]を 使[つか]うんだ。	
\\	この時計はすぐ狂うの。	狂う	この 時計[とけい]はすぐ 狂[くる]うの。	
\\	山で猿の親子を見ました。	猿	山[やま]で 猿[さる]の 親子[おやこ]を 見[み]ました。	
\\	地震の後、街は地獄のようだったよ。	地獄	地震[じしん]の 後[あと]、 街[まち]は 地獄[じごく]のようだったよ。	
\\	この本の第6章が特に好きです。	章	この 本[ほん]の 第6[だいろく] 章[しょう]が 特[とく]に 好[す]きです。	
\\	錠を頑丈なものに替えました。	頑丈	錠[じょう]を 頑丈[がんじょう]なものに 替[か]えました。	
\\	祖父の家に親戚が集まったんだ。	親戚	祖父[そふ]の 家[いえ]に 親戚[しんせき]が 集[あつ]まったんだ。	
\\	皆が人質の無事を祈っているわよ。	祈る	皆[みんな]が 人質[ひとじち]の 無事[ぶじ]を 祈[いの]っているわよ。	
\\	足がしびれた。	しびれる	足[あし]がしびれた。	
\\	その日、国民は平和への祈りを捧げるの。	祈り	その 日[ひ]、 国民[こくみん]は 平和[へいわ]への 祈[いの]りを 捧[ささ]げるの。	
\\	襟の大きいコートを買いました。	襟	襟[えり]の 大[おお]きいコートを 買[か]いました。	
\\	神道では先祖を大切にします。	先祖	神道[しんとう]では 先祖[せんぞ]を 大切[たいせつ]にします。	
\\	その映画はどんな粗筋ですか。	粗筋	その 映画[えいが]はどんな 粗筋[あらすじ]ですか。	
\\	お酢を入れすぎて酸っぱい。	酢	お 酢[す]を 入[い]れすぎて 酸[す]っぱい。	
\\	私は郊外に家を買いました。	郊外	私[わたし]は 郊外[こうがい]に 家[いえ]を 買[か]いました。	
\\	彼らは東京近郊に住んでいる。	近郊	彼[かれ]らは 東京[とうきょう] 近郊[きんこう]に 住[す]んでいる。	
\\	邪魔です、どいてください。	邪魔	邪魔[じゃま]です、どいてください。	
\\	彼は顔をじゃぶじゃぶ洗ったの。	じゃぶじゃぶ	彼[かれ]は 顔[かお]をじゃぶじゃぶ 洗[あら]ったの。	
\\	甥と姪にお年玉をあげたの。	お年玉	甥[おい]と 姪[めい]にお 年玉[としだま]をあげたの。	
\\	彼は食生活を改善しました。	改善	彼[かれ]は 食生活[しょくせいかつ]を 改善[かいぜん]しました。	
\\	これからも両国の親善を深めましょう。	親善	これからも 両国[りょうこく]の 親善[しんぜん]を 深[ふか]めましょう。	
\\	少年は網でその蝶を捕まえた。	網	少年[しょうねん]は 網[あみ]でその 蝶[ちょう]を 捕[つか]まえた。	
\\	古新聞をひもで縛ったよ。	縛る	古新聞[ふるしんぶん]をひもで 縛[しば]ったよ。	
\\	布をもっと固く絞りなさい。	絞る	布[ぬの]をもっと 固[かた]く 絞[しぼ]りなさい。	
\\	日本の制服は紺が多いね。	紺	日本[にほん]の 制服[せいふく]は 紺[こん]が 多[おお]いね。	
\\	電話がじゃんじゃん掛かってきたな。	じゃんじゃん	電話[でんわ]がじゃんじゃん 掛[か]かってきたな。	
\\	この山は紅葉がとても美しい。	紅葉	この 山[やま]は 紅葉[こうよう]がとても 美[うつく]しい。	
\\	赤い口紅を買いました。	口紅	赤[あか]い 口紅[くちべに]を 買[か]いました。	
\\	梅の花が咲きました。	梅	梅[うめ]の 花[はな]が 咲[さ]きました。	
\\	うちでは、朝食には必ず梅干が出ます。	梅干	うちでは、 朝食[ちょうしょく]には 必[かなら]ず 梅干[うめぼし]が 出[で]ます。	
\\	アリは土の中に巣を作ります。	巣	アリは 土[つち]の 中[なか]に 巣[す]を 作[つく]ります。	
\\	彼の趣味は囲碁です。	囲碁	彼[かれ]の 趣味[しゅみ]は 囲碁[いご]です。	
\\	嘘つきは泥棒の始まりよ。	嘘つき	嘘[うそ]つきは 泥棒[どろぼう]の 始[はじ]まりよ。	
\\	スカートにしわがよっているよ。	しわ	スカートにしわがよっているよ。	
\\	あなたの事情を考慮して予定をたてました。	考慮	あなたの 事情[じじょう]を 考慮[こうりょ]して 予定[よてい]をたてました。	
\\	爪をかむ癖は直した方がいい。	癖	爪[つめ]をかむ 癖[くせ]は 直[なお]した 方[ほう]がいい。	
\\	古いお寿司を食べて下痢をしてしまったんだ。	下痢	古[ふる]いお 寿司[すし]を 食[た]べて 下痢[げり]をしてしまったんだ。	
\\	大雨で崖が崩れたんだ。	崖	大雨[おおあめ]で 崖[がけ]が 崩[くず]れたんだ。	
\\	嵐で庭の木が折れたよ。	嵐	嵐[あらし]で 庭[にわ]の 木[き]が 折[お]れたよ。	
\\	その海峡に橋が掛けられました。	海峡	その 海峡[かいきょう]に 橋[はし]が 掛[か]けられました。	
\\	その噂は本当ですか。	噂	その 噂[うわさ]は 本当[ほんとう]ですか。	
\\	私は毎年夏に田舎に帰ります。	田舎	私[わたし]は 毎年夏[まいとし なつ]に 田舎[いなか]に 帰[かえ]ります。	
\\	あいつはずるがしこい顔をしているね。	ずるがしこい	あいつはずるがしこい 顔[かお]をしているね。	
\\	古い校舎の修理が必要です。	校舎	古[ふる]い 校舎[こうしゃ]の 修理[しゅうり]が 必要[ひつよう]です。	
\\	彼はお嬢さんと一緒でした。	お嬢さん	彼[かれ]はお 嬢[じょう]さんと 一緒[いっしょ]でした。	
\\	テレビは彼のいちばんの娯楽です。	娯楽	テレビは 彼[かれ]のいちばんの 娯楽[ごらく]です。	
\\	彼は80年の生涯を閉じました。	生涯	彼[かれ]は 80年[はちじゅうねん]の 生涯[しょうがい]を 閉[と]じました。	
\\	汽車で街まで行った。	汽車	汽車[きしゃ]で 街[まち]まで 行[い]った。	
\\	彼は新しい仕事に人生を賭けている。	賭ける	彼[かれ]は 新[あたら]しい 仕事[しごと]に 人生[じんせい]を 賭[か]けている。	
\\	古い蛍光灯を取り替えてください。	蛍光灯	古[ふる]い 蛍光灯[けいこうとう]を 取[と]り 替[か]えてください。	
\\	どんなに頑張っても、せいぜい3位くらいにしかなれないだろうな。	せいぜい	どんなに 頑張[がんば]っても、せいぜい 3位[さんい]くらいにしかなれないだろうな。	
\\	車掌さんが車内を回って来たよ。	車掌	車掌[しゃしょう]さんが 車内[しゃない]を 回[まわ]って 来[き]たよ。	
\\	お芋の料理はお好きですか。	芋	お 芋[いも]の 料理[りょうり]はお 好[す]きですか。	
\\	菊の花を買ってきました。	菊	菊[きく]の 花[はな]を 買[か]ってきました。	
\\	彼女は彼を軽蔑していたの。	軽蔑	彼女[かのじょ]は 彼[かれ]を 軽蔑[けいべつ]していたの。	
\\	彼女は夕食に天ぷらを揚げました。	揚げる	彼女[かのじょ]は 夕食[ゆうしょく]に 天[てん]ぷらを 揚[あ]げました。	
\\	彼は留学の夢を諦めていないよ。	諦める	彼[かれ]は 留学[りゅうがく]の 夢[ゆめ]を 諦[あきら]めていないよ。	
\\	彼女のピアノの演奏は素晴らしいね。	演奏	彼女[かのじょ]のピアノの 演奏[えんそう]は 素晴[すば]らしいね。	
\\	彼女は一生ぜいたくに暮らしたんだ。	ぜいたく	彼女[かのじょ]は 一生[いっしょう]ぜいたくに 暮[く]らしたんだ。	
\\	この服は窮屈になったな。	窮屈	この 服[ふく]は 窮屈[きゅうくつ]になったな。	
\\	博物館で昔の貨幣を見ました。	貨幣	博物館[はくぶつかん]で 昔[むかし]の 貨幣[かへい]を 見[み]ました。	
\\	母に真珠のネックレスをもらいました。	真珠	母[はは]に 真珠[しんじゅ]のネックレスをもらいました。	
\\	彼女は久しぶりに故郷に帰りました。	故郷	彼女[かのじょ]は 久[ひさ]しぶりに 故郷[こきょう]に 帰[かえ]りました。	
\\	クリスタルガラスの花瓶を買いました。	花瓶	クリスタルガラスの 花瓶[かびん]を 買[か]いました。	
\\	疲れたでしょう、一旦休みましょう。	一旦	疲[つか]れたでしょう、 一旦[いったん] 休[やす]みましょう。	
\\	乞食が公園のベンチで寝ている。	乞食	乞食[こじき]が 公園[こうえん]のベンチで 寝[ね]ている。	
\\	この池には亀がいますね。	亀	この 池[いけ]には 亀[かめ]がいますね。	
\\	今学期は成績ががくんと落ちてしまった。	がくんと	今学期[こんがっき]は 成績[せいせき]ががくんと 落[お]ちてしまった。	
\\	彼の才能は羨ましい程だね。	羨ましい	彼[かれ]の 才能[さいのう]は 羨[うらや]ましい 程[ほど]だね。	
\\	彼らは交代でボートを漕いだんだ。	漕ぐ	彼[かれ]らは 交代[こうたい]でボートを 漕[こ]いだんだ。	
\\	相撲の稽古はとても厳しいよ。	稽古	相撲[すもう]の 稽古[けいこ]はとても 厳[きび]しいよ。	
\\	歌舞伎の芝居を見に行きました。	歌舞伎	歌舞伎[かぶき]の 芝居[しばい]を 見[み]に 行[い]きました。	
\\	「この会議は退屈だ」と同僚が私に囁いたの。	囁く	「この 会議[かいぎ]は 退屈[たいくつ]だ」と 同僚[どうりょう]が 私[わたし]に 囁[ささや]いたの。	
\\	咳が止まらないので病院に行ってきたの。	咳	咳[せき]が 止[と]まらないので 病院[びょういん]に 行[い]ってきたの。	
\\	犬が手に噛み付きました。	噛み付く	犬[いぬ]が 手[て]に 噛[か]み 付[つ]きました。	
\\	ダイヤモンドがきらりと光った。	きらりと	ダイヤモンドがきらりと 光[ひか]った。	
\\	彼の背広に糸屑がついているわ。	屑	彼[かれ]の 背広[せびろ]に 糸[いと] 屑[くず]がついているわ。	
\\	紙屑は屑入れに入れなさい。	紙屑	紙屑[かみくず]は 屑入[くず い]れに 入[い]れなさい。	
\\	背中をお母さんに掻いてもらったの。	掻く	背中[せなか]をお 母[かあ]さんに 掻[か]いてもらったの。	
\\	母は鍋のシチューを掻き回しているよ。	掻き回す	母[はは]は 鍋[なべ]のシチューを 掻[か]き 回[まわ]しているよ。	
\\	海外に住むのは私の憧れです。	憧れ	海外[かいがい]に 住[す]むのは 私[わたし]の 憧[あこが]れです。	
\\	彼はパイロットの職に憧れているんだ。	憧れる	彼[かれ]はパイロットの 職[しょく]に 憧[あこが]れているんだ。	
\\	彼は溺れている子を助けたんだ。	溺れる	彼[かれ]は 溺[おぼ]れている 子[こ]を 助[たす]けたんだ。	
\\	薬をごくんと飲み込んだんだ。	ごくんと	薬[くすり]をごくんと 飲[の]み 込[こ]んだんだ。	
\\	長いこと御無沙汰いたしました。	御無沙汰	長[なが]いこと 御無沙汰[ごぶさた]いたしました。	
\\	小さなバケツで水を汲んだの。	汲む	小[ちい]さなバケツで 水[みず]を 汲[く]んだの。	
\\	学校での苛めはなかなか減らない。	苛め	学校[がっこう]での 苛[いじ]めはなかなか 減[へ]らない。	
\\	動物を苛めてはいけません。	苛める	動物[どうぶつ]を 苛[いじ]めてはいけません。	
\\	彼女は曖昧な返事をしたね。	曖昧	彼女[かのじょ]は 曖昧[あいまい]な 返事[へんじ]をしたね。	
\\	日本では、大晦日にそばを食べます。	大晦日	日本[にっぽん]では、 大晦日[おおみそか]にそばを 食[た]べます。	
\\	弟は小さい頃は臆病だったんだ。	臆病	弟[おとうと]は 小[ちい]さい 頃[ころ]は 臆病[おくびょう]だったんだ。	
\\	彼女は車椅子に乗っています。	車椅子	彼女[かのじょ]は 車椅子[くるまいす]に 乗[の]っています。	
\\	庭に柿の実がなりました。	柿	庭[にわ]に 柿[かき]の 実[み]がなりました。	
\\	誤解があったことをお詫びします。	お詫び	誤解[ごかい]があったことをお 詫[わ]びします。	
\\	釘を踏んで怪我をしました。	釘	釘[くぎ]を 踏[ふ]んで 怪我[けが]をしました。	
\\	包丁が錆びてしまった。	錆びる	包丁[ほうちょう]が 錆[さ]びてしまった。	
\\	レモンの爽やかな香りがした。	爽やか	レモンの 爽[さわ]やかな 香[かお]りがした。	
\\	朝は雀の声で目が覚めます。	雀	朝[あさ]は 雀[すずめ]の 声[こえ]で 目[め]が 覚[さ]めます。	
\\	私の甥は3才です。	甥	私[わたし]の 甥[おい]は 3才[さんさい]です。	
\\	犬がお皿の匂いをクンクン嗅いでいるね。	嗅ぐ	犬[いぬ]がお 皿[さら]の 匂[にお]いをクンクン 嗅[か]いでいるね。	
\\	次に、野菜を炒めてください。	炒める	次[つぎ]に、 野菜[やさい]を 炒[いた]めてください。	
\\	一まず休憩しましょう。	一まず	一[ひと]まず 休憩[きゅうけい]しましょう。	
\\	万一のために保険に入ったの。	万一	万一[まんいち]のために 保険[ほけん]に 入[はい]ったの。	
\\	彼女には日ごろからお世話になっています。	日ごろ	彼女[かのじょ]には 日[ひ]ごろからお 世話[せわ]になっています。	
\\	当時、日ソ会談が開かれた。	日ソ	当時[とうじ]、 日[にっ]ソ 会談[かいだん]が 開[ひら]かれた。	
\\	日仏の共同研究が始まったな。	日	日[にち] 仏[ふつ]の 共同研究[きょうどう けんきゅう]が 始[はじ]まったな。	
\\	去年のデータを見せてください。	データ	去年[きょねん]のデータを 見[み]せてください。	
\\	試写会の日時を教えてください。	日時	試写会[ししゃかい]の 日時[にちじ]を 教[おし]えてください。	
\\	ミーティングの日日を間違えました。	日日	ミーティングの 日日[ひにち]を 間違[まちが]えました。	
\\	日々の努力が大切です。	日々	日々[ひび]の 努力[どりょく]が 大切[たいせつ]です。	
\\	空に三日月が見えました。	三日月	空[そら]に 三日月[みかづき]が 見[み]えました。	
\\	月日が経つのは早いものです。	月日	月日[つきひ]が 経[た]つのは 早[はや]いものです。	
\\	私はただ彼女と話したかっただけです。	ただ	私[わたし]はただ 彼女[かのじょ]と 話[はな]したかっただけです。	
\\	土を掘って木を植えました。	土	土[つち]を 掘[ほ]って 木[き]を 植[う]えました。	
\\	あれから長い年月が経ちました。	年月	あれから 長[なが]い 年月[としつき]が 経[た]ちました。	
\\	申請の年月日を西暦で書いてください。	年月日	申請[しんせい]の 年月日[ねんがっぴ]を 西暦[せいれき]で 書[か]いてください。	
\\	そのお寺は長い年月をかけて建てられた。	年月	そのお 寺[てら]は 長[なが]い 年月[ねんげつ]をかけて 建[た]てられた。	
\\	東京の人口は年々増えています。	年々	東京[とうきょう]の 人口[じんこう]は 年々[ねんねん] 増[ふ]えています。	
\\	彼の意見はもっともです。	もっとも	彼[かれ]の 意見[いけん]はもっともです。	
\\	何か質問はありますか。	何か	何[なに]か 質問[しつもん]はありますか。	
\\	何とかお願いします。	何とか	何[なん]とかお 願[ねが]いします。	
\\	ご無事で何よりです。	何より	ご 無事[ぶじ]で 何[なに]よりです。	
\\	何と彼らは結婚したそうです。	何と	何[なん]と 彼[かれ]らは 結婚[けっこん]したそうです。	
\\	パーティーに行くのは何となく気が進まないな。	何となく	パーティーに 行[い]くのは 何[なん]となく 気[き]が 進[すす]まないな。	
\\	茶柱が立って何だか少し幸せな気分です。	何だか	茶柱[ちゃばしら]が 立[た]って 何[なん]だか 少[すこ]し 幸[しあわ]せな 気分[きぶん]です。	
\\	彼は靴のメーカーで働いています。	メーカー	彼[かれ]は 靴[くつ]のメーカーで 働[はたら]いています。	
\\	彼女は何で来ないんだろう。	何で	彼女[かのじょ]は 何[なん]で 来[こ]ないんだろう。	
\\	その店に何十人も行列していたよ。	何十	その 店[みせ]に 何十[なんじゅう] 人[にん]も 行列[ぎょうれつ]していたよ。	
\\	何てきれいな人なんだ。	何て	何[なん]てきれいな 人[ひと]なんだ。	
\\	彼は何しろよくしゃべります。	何しろ	彼[かれ]は 何[なに]しろよくしゃべります。	
\\	新聞社に何千ものメールが寄せられたよ。	何千	新聞社[しんぶんしゃ]に 何千[なんぜん]ものメールが 寄[よ]せられたよ。	
\\	自分のレベルに合った授業を選んでください。	レベル	自分[じぶん]のレベルに 合[あ]った 授業[じゅぎょう]を 選[えら]んでください。	
\\	海で何百もの美しい魚を見ました。	何百	海[うみ]で 何百[なんびゃく]もの 美[うつく]しい 魚[さかな]を 見[み]ました。	
\\	フランスからサーカスがやって来ます。	やって来る	フランスからサーカスがやって 来[き]ます。	
\\	有名なバンドが来日していますね。	来日	有名[ゆうめい]なバンドが 来日[らいにち]していますね。	
\\	私たちは日帰りで京都に行きました。	日帰り	私[わたし]たちは 日帰[ひがえ]りで 京都[きょうと]に 行[い]きました。	
\\	彼は大して嬉しそうには見えなかったよね。	大して	彼[かれ]は 大[たい]して 嬉[うれ]しそうには 見[み]えなかったよね。	
\\	彼は建築家というよりむしろ芸術家ね。	むしろ	彼[かれ]は 建築家[けんちくか]というよりむしろ 芸術家[げいじゅつか]ね。	
\\	このかばんには大金が入っています。	大金	このかばんには 大金[たいきん]が 入[はい]っています。	
\\	チーズケーキの大を一つ下さい。	大	チーズケーキの 大[だい]を 一[ひと]つ 下[くだ]さい。	
\\	日中はずっと海で泳いでいました。	日中	日中[にっちゅう]はずっと 海[うみ]で 泳[およ]いでいました。	
\\	日中貿易は急激に伸びているわね。	日中	日中[にっちゅう] 貿易[ぼうえき]は 急激[きゅうげき]に 伸[の]びているわね。	
\\	これは、中年の男性によく見られる症状です。	中年	これは、 中年[ちゅうねん]の 男性[だんせい]によく 見[み]られる 症状[しょうじょう]です。	
\\	うちの子の身長はクラスで中くらいです。	中	うちの 子[こ]の 身長[しんちょう]はクラスで 中[ちゅう]くらいです。	
\\	状況はますます悪くなったね。	ますます	状況[じょうきょう]はますます 悪[わる]くなったね。	
\\	叔母は年中旅行しています。	年中	叔母[おば]は 年中[ねんじゅう] 旅行[りょこう]しています。	
\\	応募作品の大小は問いません。	大小	応募作品[おうぼ さくひん]の 大小[だいしょう]は 問[と]いません。	
\\	このソフトには多少問題がある。	多少	このソフトには 多少[たしょう] 問題[もんだい]がある。	
\\	猿が木にぶら下がってるな。	ぶら下がる	猿[さる]が 木[き]にぶら 下[さ]がってるな。	
\\	彼は首にカメラをぶら下げているね。	ぶら下げる	彼[かれ]は 首[くび]にカメラをぶら 下[さ]げているね。	
\\	彼女はこの小説の主人公のモデルです。	モデル	彼女[かのじょ]はこの 小説[しょうせつ]の 主人公[しゅじんこう]のモデルです。	
\\	私はその教授の下で論文を書いたの。	下	私[わたし]はその 教授[きょうじゅ]の 下[もと]で 論文[ろんぶん]を 書[か]いたの。	
\\	仕事のやり方を教えてください。	やり方	仕事[しごと]のやり 方[かた]を 教[おし]えてください。	
\\	彼の連絡先を方々に問い合わせたんだ。	方々	彼[かれ]の 連絡先[れんらくさき]を 方々[ほうぼう]に 問[と]い 合[あ]わせたんだ。	
\\	彼の行方が分かりません。	行方	彼[かれ]の 行方[ゆくえ]が 分[わ]かりません。	
\\	この学校では一人一人の生徒を大切にしているの。	一人一人	この 学校[がっこう]では 一人一人[ひとりひとり]の 生徒[せいと]を 大切[たいせつ]にしているの。	
\\	彼はこの国に平和をもたらしたのよ。	もたらす	彼[かれ]はこの 国[くに]に 平和[へいわ]をもたらしたのよ。	
\\	今日の日の入りは午後6時でした。	日の入り	今日[きょう]の 日[ひ]の 入[い]りは 午後6時[ごご 
\\	じ]でした。	
\\	彼の家は人の出入りが多いね。	出入り	彼[かれ]の 家[いえ]は 人[ひと]の 出入[でい]りが 多[おお]いね。	
\\	この家は来年出来上がります。	出来上がる	この 家[いえ]は 来年[らいねん] 出来上[できあ]がります。	
\\	今日も遊園地は大変な人出だったよ。	人出	今日[きょう]も 遊園地[ゆうえんち]は 大変[たいへん]な 人出[ひとで]だったよ。	
\\	日の出がとてもきれいですね。	日の出	日[ひ]の 出[で]がとてもきれいですね。	
\\	人はそれぞれ考え方が違います。	それぞれ	人[ひと]はそれぞれ 考[かんが]え 方[かた]が 違[ちが]います。	
\\	私は作品の出来上がりに満足しています。	出来上がり	私[わたし]は 作品[さくひん]の 出来上[できあ]がりに 満足[まんぞく]しています。	
\\	出来るだけ早く来てください。	出来るだけ	出来[でき]るだけ 早[はや]く 来[き]てください。	
\\	彼はメガネを外しました。	外す	彼[かれ]はメガネを 外[はず]しました。	
\\	びんのふたが外れません。	外れる	びんのふたが 外[はず]れません。	
\\	このくじは外れです。	外れ	このくじは 外[はず]れです。	
\\	彼女もその内、親の気持ちが分かるだろう。	その内	彼女[かのじょ]もその 内[うち]、 親[おや]の 気持[きも]ちが 分[わ]かるだろう。	
\\	彼はトップの成績で合格しました。	トップ	彼[かれ]はトップの 成績[せいせき]で 合格[ごうかく]しました。	
\\	彼女はプレッシャーから解放されて本来の自分に戻ったな。	本来	彼女[かのじょ]はプレッシャーから 解放[かいほう]されて 本来[ほんらい]の 自分[じぶん]に 戻[もど]ったな。	
\\	それは本人に聞いてください。	本人	それは 本人[ほんにん]に 聞[き]いてください。	
\\	会社の本年の目標が発表されたよ。	本年	会社[かいしゃ]の 本年[ほんねん]の 目標[もくひょう]が 発表[はっぴょう]されたよ。	
\\	本日のランチはハンバーグでございます。	本日	本日[ほんじつ]のランチはハンバーグでございます。	
\\	疲れた体を休めてください。	休める	疲[つか]れた 体[からだ]を 休[やす]めてください。	
\\	最近、日本は健康ブームです。	ブーム	最近[さいきん]、 日本[にほん]は 健康[けんこう]ブームです。	
\\	ここで一休みしましょう。	一休み	ここで 一休[ひとやす]みしましょう。	
\\	大体、初めから無理な計画だったのだ。	大体	大体[だいたい]、 初[はじ]めから 無理[むり]な 計画[けいかく]だったのだ。	
\\	彼女の仕事はデータの入力です。	入力	彼女[かのじょ]の 仕事[しごと]はデータの 入力[にゅうりょく]です。	
\\	若者は体力がありますね。	体力	若者[わかもの]は 体力[たいりょく]がありますね。	
\\	彼は目上の人にとても気を使います。	目上	彼[かれ]は 目上[めうえ]の 人[ひと]にとても 気[き]を 使[つか]います。	
\\	ようやくゴールが見えてきました。	ようやく	ようやくゴールが 見[み]えてきました。	
\\	この荷の目方は70キロってとこかね。	目方	この 荷[に]の 目方[めかた]は70キロってとこかね。	
\\	彼は目下の者にも優しいんだ。	目下	彼[かれ]は 目下[めした]の 者[もの]にも 優[やさ]しいんだ。	
\\	息子は目下勉強中です。	目下	息子[むすこ]は 目下[もっか] 勉強中[べんきょうちゅう]です。	
\\	出入り口に車を停めないでください。	出入り口	出入[でい]り 口[ぐち]に 車[くるま]を 停[と]めないでください。	
\\	彼はおまんじゅうを一口で食べたの。	一口	彼[かれ]はおまんじゅうを 一口[ひとくち]で 食[た]べたの。	
\\	人手が足りなくて忙しい。	人手	人手[ひとで]が 足[た]りなくて 忙[いそが]しい。	
\\	彼はチームの一員です。	チーム	彼[かれ]はチームの 一員[いちいん]です。	
\\	母は庭の手入れをしています。	手入れ	母[はは]は 庭[にわ]の 手入[てい]れをしています。	
\\	手本を見ながら習字をしました。	手本	手本[てほん]を 見[み]ながら 習字[しゅうじ]をしました。	
\\	あの人は手足が長い。	手足	あの 人[ひと]は 手足[てあし]が 長[なが]い。	
\\	今は田に水を入れる季節です。	田	今[いま]は 田[た]に 水[みず]を 入[い]れる 季節[きせつ]です。	
\\	昨日、花火を見に行ったよ。	花火	昨日[きのう]、 花火[はなび]を 見[み]に 行[い]ったよ。	
\\	同じタイプで色の違うものはありますか。	タイプ	同[おな]じタイプで 色[いろ]の 違[ちが]うものはありますか。	
\\	そのスポーツは男女一緒にします。	男女	そのスポーツは 男女[だんじょ] 一緒[いっしょ]にします。	
\\	男子はこっちに並んで。	男子	男子[だんし]はこっちに 並[なら]んで。	
\\	私は北海道の出身です。	私	私[わたくし]は 北海道[ほっかいどう]の 出身[しゅっしん]です。	
\\	彼は高校時代からの友人です。	友人	彼[かれ]は 高校時代[こうこう じだい]からの 友人[ゆうじん]です。	
\\	持つべきものは友よね。	友	持[も]つべきものは 友[とも]よね。	
\\	彼はビジネスクラスに乗ったの。	ビジネス	彼[かれ]はビジネスクラスに 乗[の]ったの。	
\\	あの人は卓球の元選手です。	元	あの 人[ひと]は 卓球[たっきゅう]の 元[もと] 選手[せんしゅ]です。	
\\	説明書は手元にありますか。	手元	説明書[せつめいしょ]は 手元[てもと]にありますか。	
\\	彼女は元々フランスに行くつもりだったの。	元々	彼女[かのじょ]は 元々[もともと]フランスに 行[い]くつもりだったの。	
\\	天から恵みの雨が降ったね。	天	天[てん]から 恵[めぐ]みの 雨[あめ]が 降[ふ]ったね。	
\\	いや、僕は本気なんだ。	本気	いや、 僕[ぼく]は 本気[ほんき]なんだ。	
\\	明日、会議を開きます。	明日	明日[みょうにち]、 会議[かいぎ]を 開[ひら]きます。	
\\	最近、社内ネットワークの調子が悪い。	ネットワーク	最近[さいきん]、 社内[しゃない]ネットワークの 調子[ちょうし]が 悪[わる]い。	
\\	東西に大きな道路が通っています。	東西	東西[とうざい]に 大[おお]きな 道路[どうろ]が 通[とお]っています。	
\\	ここから南西に進むと村があります。	南西	ここから 南西[なんせい]に 進[すす]むと 村[むら]があります。	
\\	南東の方角に月が見えるよ。	南東	南東[なんとう]の 方角[ほうがく]に 月[つき]が 見[み]えるよ。	
\\	南北に山が広がっています。	南北	南北[なんぼく]に 山[やま]が 広[ひろ]がっています。	
\\	台風は北西に進んでいます。	北西	台風[たいふう]は 北西[ほくせい]に 進[すす]んでいます。	
\\	この服、私にはやや小さいみたい。	やや	この 服[ふく]、 私[わたし]にはやや 小[ちい]さいみたい。	
\\	町の北東に山があります。	北東	町[まち]の 北東[ほくとう]に 山[やま]があります。	
\\	彼は上司に怒りの目を向けたんだよ。	向ける	彼[かれ]は 上司[じょうし]に 怒[いか]りの 目[め]を 向[む]けたんだよ。	
\\	花瓶の向きを変えたの。	向き	花瓶[かびん]の 向[む]きを 変[か]えたの。	
\\	向かいの席が空いていますよ。	向かい	向[む]かいの 席[せき]が 空[あ]いていますよ。	
\\	私は相撲部屋に入門しました。	入門	私[わたし]は 相撲部屋[すもう べや]に 入門[にゅうもん]しました。	
\\	彼女はデモに参加したよ。	デモ	彼女[かのじょ]はデモに 参加[さんか]したよ。	
\\	8時に学校の門が開きます。	門	
\\	時[じ]に 学校[がっこう]の 門[もん]が 開[ひら]きます。	
\\	霧が晴れて視界が開けたね。	開ける	霧[きり]が 晴[は]れて 視界[しかい]が 開[ひら]けたね。	
\\	そのふたつの町の中間には川が流れているの。	中間	そのふたつの 町[まち]の 中間[ちゅうかん]には 川[かわ]が 流[なが]れているの。	
\\	これはとても手間のかかる料理です。	手間	これはとても 手間[てま]のかかる 料理[りょうり]です。	
\\	彼は少し間を置いてから返事した。	間	彼[かれ]は 少[すこ]し 間[ま]を 置[お]いてから 返事[へんじ]した。	
\\	ワープロで報告書を打ったよ。	ワープロ	ワープロで 報告書[ほうこくしょ]を 打[う]ったよ。	
\\	その計画に対する反対の声が高まっているわ。	高まる	その 計画[けいかく]に 対[たい]する 反対[はんたい]の 声[こえ]が 高[たか]まっているわ。	
\\	自分を高めるのは大切なことです。	高める	自分[じぶん]を 高[たか]めるのは 大切[たいせつ]なことです。	
\\	そのシャツは安っぽいね。	安っぽい	そのシャツは 安[やす]っぽいね。	
\\	最近、教育レベルが低下しています。	低下	最近[さいきん]、 教育[きょういく]レベルが 低下[ていか]しています。	
\\	彼は低賃金で働いているんだ。	低	彼[かれ]は 低[てい] 賃金[ちんぎん]で 働[はたら]いているんだ。	
\\	彼は世界で最も早い男です。	最も	彼[かれ]は 世界[せかい]で 最[もっと]も 早[はや]い 男[おとこ]です。	
\\	グループのメンバーは10人です。	メンバー	グループのメンバーは10 人[にん]です。	
\\	いつも月初めに彼と会います。	月初め	いつも 月初[つきはじ]めに 彼[かれ]と 会[あ]います。	
\\	前年に比べて利益が上がりました。	前年	前年[ぜんねん]に 比[くら]べて 利益[りえき]が 上[あ]がりました。	
\\	前方に山が見えますね。	前方	前方[ぜんぽう]に 山[やま]が 見[み]えますね。	
\\	駅の手前に郵便局があります。	手前	駅[えき]の 手前[てまえ]に 郵便局[ゆうびんきょく]があります。	
\\	前向きに検討します。	前向き	前向[まえむ]きに 検討[けんとう]します。	
\\	この問題はハードウェアの故障が原因です。	ハードウェア	この 問題[もんだい]はハードウェアの 故障[こしょう]が 原因[げんいん]です。	
\\	彼女は人前に出ると緊張します。	人前	彼女[かのじょ]は 人前[ひとまえ]に 出[で]ると 緊張[きんちょう]します。	
\\	休む時は前もって連絡ください。	前もって	休[やす]む 時[とき]は 前[まえ]もって 連絡[れんらく]ください。	
\\	寿司の出前を頼んだよ。	出前	寿司[すし]の 出前[でまえ]を 頼[たの]んだよ。	
\\	相手チームのリードで前半が終わったよ。	前半	相手[あいて]チームのリードで 前半[ぜんはん]が 終[お]わったよ。	
\\	彼は30代の半ばです。	半ば	彼[かれ]は30 代[だい]の 半[なか]ばです。	
\\	彼はよくマスコミにも登場するね。	マスコミ	彼[かれ]はよくマスコミにも 登場[とうじょう]するね。	
\\	日本に来て半年になります。	半年	日本[にっぽん]に 来[き]て 半年[はんとし]になります。	
\\	明朝10時からまた会議です。	明朝	明朝[みょうちょう] 
\\	時[じ]からまた 会議[かいぎ]です。	
\\	彼は晩年を故郷で過ごしたんだ。	晩年	彼[かれ]は 晩年[ばんねん]を 故郷[こきょう]で 過[す]ごしたんだ。	
\\	夜間は裏口から入ってください。	夜間	夜間[やかん]は 裏口[うらぐち]から 入[はい]ってください。	
\\	ふたりで夜空を見上げたの。	夜空	ふたりで 夜空[よぞら]を 見上[みあ]げたの。	
\\	夜明けと共に目が覚めたんだ。	夜明け	夜明[よあ]けと 共[とも]に 目[め]が 覚[さ]めたんだ。	
\\	やがて雪も止むでしょう。	やがて	やがて 雪[ゆき]も 止[や]むでしょう。	
\\	散歩にいい月夜ですね。	月夜	散歩[さんぽ]にいい 月夜[つきよ]ですね。	
\\	あと1時間で夜が明けますね。	夜	あと1 時間[じかん]で 夜[よ]が 明[あ]けますね。	
\\	日本では7月7日は七夕の日です。	七夕	日本[にほん]では7 月7日[がつ 
\\	か]は 七夕[たなばた]の 日[ひ]です。	
\\	夕日が西の空に沈んだね。	夕日	夕日[ゆうひ]が 西[にし]の 空[そら]に 沈[しず]んだね。	
\\	昔は川の水が飲み水でした。	飲み水	昔[むかし]は 川[かわ]の 水[みず]が 飲[の]み 水[みず]でした。	
\\	彼はわずかな貯金で暮らしている。	わずか	彼[かれ]はわずかな 貯金[ちょきん]で 暮[く]らしている。	
\\	今朝は飯を食べたかい。	飯	今朝[けさ]は 飯[めし]を 食[た]べたかい。	
\\	私は空を見上げたの。	見上げる	私[わたし]は 空[そら]を 見上[みあ]げたの。	
\\	山の頂上からふもとを見下ろしたの。	見下ろす	山[やま]の 頂上[ちょうじょう]からふもとを 見下[みお]ろしたの。	
\\	新聞の見出しが彼の目を引いたんだ。	見出し	新聞[しんぶん]の 見出[みだ]しが 彼[かれ]の 目[め]を 引[ひ]いたんだ。	
\\	見本はこちらにございます。	見本	見本[みほん]はこちらにございます。	
\\	彼女の行動パターンは興味深いな。	パターン	彼女[かのじょ]の 行動[こうどう]パターンは 興味深[きょうみぶか]いな。	
\\	彼はその絵をずっと見つめていたの。	見つめる	彼[かれ]はその 絵[え]をずっと 見[み]つめていたの。	
\\	このベランダは見晴らしがいい。	見晴らし	このベランダは 見晴[みは]らしがいい。	
\\	9月には月見を楽しみます。	月見	
\\	月[がつ]には 月見[つきみ]を 楽[たの]しみます。	
\\	運転中はよそ見をしてはいけません。	よそ見	運転中[うんてんちゅう]はよそ 見[み]をしてはいけません。	
\\	彼は見かけによらず優しいね。	見かけ	彼[かれ]は 見[み]かけによらず 優[やさ]しいね。	
\\	彼は方言で話します。	方言	彼[かれ]は 方言[ほうげん]で 話[はな]します。	
\\	彼はついに弁護士の資格を取ったよ。	ついに	彼[かれ]はついに 弁護士[べんごし]の 資格[しかく]を 取[と]ったよ。	
\\	社長に一言お願いしたの。	一言	社長[しゃちょう]に 一言[ひとこと]お 願[ねが]いしたの。	
\\	多くの文明は川の近くで始まった。	文明	多[おお]くの 文明[ぶんめい]は 川[かわ]の 近[ちか]くで 始[はじ]まった。	
\\	本文をよく読んで答えてください。	本文	本文[ほんぶん]をよく 読[よ]んで 答[こた]えてください。	
\\	この文は意味が分かりません。	文	この 文[ぶん]は 意味[いみ]が 分[わ]かりません。	
\\	私の趣味は読書です。	読書	私[わたし]の 趣味[しゅみ]は 読書[どくしょ]です。	
\\	彼はまるで子供のようにはしゃいだよ。	まるで	彼[かれ]はまるで 子供[こども]のようにはしゃいだよ。	
\\	息子は学校で読み書きを勉強しています。	読み書き	息子[むすこ]は 学校[がっこう]で 読[よ]み 書[か]きを 勉強[べんきょう]しています。	
\\	あの国は目覚ましい発展を遂げたの。	目覚ましい	あの 国[くに]は 目覚[めざ]ましい 発展[はってん]を 遂[と]げたの。	
\\	この人に見覚えがありますか。	見覚え	この 人[ひと]に 見覚[みおぼ]えがありますか。	
\\	目覚ましにコーヒーを飲んだの。	目覚まし	目覚[めざ]ましにコーヒーを 飲[の]んだの。	
\\	会長が閉会の挨拶をしました。	閉会	会長[かいちょう]が 閉会[へいかい]の 挨拶[あいさつ]をしました。	
\\	彼は山道をどんどん進んで行ったの。	どんどん	彼[かれ]は 山道[やまみち]をどんどん 進[すす]んで 行[い]ったの。	
\\	アメリカで彼女と出会いました。	出会う	アメリカで 彼女[かのじょ]と 出会[であ]いました。	
\\	プロジェクトメンバーと話し合いをしたの。	話し合い	プロジェクトメンバーと 話[はな]し 合[あ]いをしたの。	
\\	うちの両親はお見合い結婚でした。	見合い	うちの 両親[りょうしん]はお 見合[みあ]い 結婚[けっこん]でした。	
\\	昼食はクッキーで間に合わせたの。	間に合わせる	昼食[ちゅうしょく]はクッキーで 間[ま]に 合[あ]わせたの。	
\\	今日は本社で会議があります。	本社	今日[きょう]は 本社[ほんしゃ]で 会議[かいぎ]があります。	
\\	彼がそれほど悩んでいたとは知らなかった。	それほど	彼[かれ]がそれほど 悩[なや]んでいたとは 知[し]らなかった。	
\\	彼は昨年入社したの。	入社	彼[かれ]は 昨年[さくねん] 入社[にゅうしゃ]したの。	
\\	彼は結果に満足したようです。	満足	彼[かれ]は 結果[けっか]に 満足[まんぞく]したようです。	
\\	彼は応募の条件を満たしていない。	満たす	彼[かれ]は 応募[おうぼ]の 条件[じょうけん]を 満[み]たしていない。	
\\	月が満ちてきましたね。	満ちる	月[つき]が 満[み]ちてきましたね。	
\\	このバスは満員です。	満員	このバスは 満員[まんいん]です。	
\\	今日は満月ですね。	満月	今日[きょう]は 満月[まんげつ]ですね。	
\\	もはや彼の助けは必要じゃないの。	もはや	もはや 彼[かれ]の 助[たす]けは 必要[ひつよう]じゃないの。	
\\	面白い出来事がありました。	出来事	面白[おもしろ]い 出来事[できごと]がありました。	
\\	彼は見事なジャンプを見せたね。	見事	彼[かれ]は 見事[みごと]なジャンプを 見[み]せたね。	
\\	私の父は大工です。	大工	私[わたし]の 父[ちち]は 大工[だいく]です。	
\\	この場でお礼を言わせてください。	場	この 場[ば]でお 礼[れい]を 言[い]わせてください。	
\\	選手の入場です。	入場	選手[せんしゅ]の 入場[にゅうじょう]です。	
\\	この服のデザインは素敵ですね。	デザイン	この 服[ふく]のデザインは 素敵[すてき]ですね。	
\\	このエアコンはあまり電力を使いません。	電力	このエアコンはあまり 電力[でんりょく]を 使[つか]いません。	
\\	電子辞書はとても便利です。	電子	電子[でんし] 辞書[じしょ]はとても 便利[べんり]です。	
\\	彼女は外国人に交じってダンスをしたの。	交じる	彼女[かのじょ]は 外国人[がいこくじん]に 交[ま]じってダンスをしたの。	
\\	人と交わって、多くのことを学びました。	交わる	人[ひと]と 交[まじ]わって、 多[おお]くのことを 学[まな]びました。	
\\	私も交ぜてください。	交ぜる	私[わたし]も 交[ま]ぜてください。	
\\	彼は大のサッカーファンです。	ファン	彼[かれ]は 大[だい]のサッカーファンです。	
\\	大人も子供も交ざって遊んだの。	交ざる	大人[おとな]も 子供[こども]も 交[ま]ざって 遊[あそ]んだの。	
\\	二つの国道はここで交わります。	交わる	二[ふた]つの 国道[こくどう]はここで 交[まじ]わります。	
\\	仕事の見通しがたたない。	見通し	仕事[しごと]の 見通[みとお]しがたたない。	
\\	その国では英語は通じますか。	通じる	その 国[くに]では 英語[えいご]は 通[つう]じますか。	
\\	針に糸を通してください。	通す	針[はり]に 糸[いと]を 通[とお]してください。	
\\	この道は通行できません。	通行	この 道[みち]は 通行[つうこう]できません。	
\\	お申し込みなさるのでしたら、こちらにお並びください。	なさる	お 申[もう]し 込[こ]みなさるのでしたら、こちらにお 並[なら]びください。	
\\	ここは人通りが激しいね。	人通り	ここは 人通[ひとどお]りが 激[はげ]しいね。	
\\	説明書を一通り読んだの。	一通り	説明書[せつめいしょ]を 一通[ひととお]り 読[よ]んだの。	
\\	通路の右側にトイレがありますよ。	通路	通路[つうろ]の 右側[みぎがわ]にトイレがありますよ。	
\\	この電車は地上を走ります。	地上	この 電車[でんしゃ]は 地上[ちじょう]を 走[はし]ります。	
\\	彼はその地で残りの生涯を過ごしたんだ。	地	彼[かれ]はその 地[ち]で 残[のこ]りの 生涯[しょうがい]を 過[す]ごしたんだ。	
\\	みんなの意見がまとまりました。	まとまる	みんなの 意見[いけん]がまとまりました。	
\\	地下道を通って行きましょう。	地下道	地下道[ちかどう]を 通[とお]って 行[い]きましょう。	
\\	スタジオは地下にあります。	地下	スタジオは 地下[ちか]にあります。	
\\	この地方は漁業が盛んです。	地方	この 地方[ちほう]は 漁業[ぎょぎょう]が 盛[さか]んです。	
\\	これからは経営の合理化を図りたいと思います。	図る	これからは 経営[けいえい]の 合理化[ごうりか]を 図[はか]りたいと 思[おも]います。	
\\	これは児童図書です。	図書	これは 児童[じどう] 図書[としょ]です。	
\\	このゲームのルールは簡単です。	ルール	このゲームのルールは 簡単[かんたん]です。	
\\	他方の視点からも見てみましょう。	他方	他方[たほう]の 視点[してん]からも 見[み]てみましょう。	
\\	私は他人に住所を教えたくない。	他人	私[わたし]は 他人[たにん]に 住所[じゅうしょ]を 教[おし]えたくない。	
\\	雨で運動会が中止になったの。	中止	雨[あめ]で 運動会[うんどうかい]が 中止[ちゅうし]になったの。	
\\	あの道路は通行止めだそうです。	通行止め	あの 道路[どうろ]は 通行止[つうこうど]めだそうです。	
\\	丘の上から草原を見渡したの。	見渡す	丘[おか]の 上[うえ]から 草原[そうげん]を 見渡[みわた]したの。	
\\	もうここには二度と来ません。	二度と	もうここには 二度[にど]と 来[き]ません。	
\\	事故現場の上空をヘリコプターが飛んでいます。	ヘリコプター	事故現場[じこ げんば]の 上空[じょうくう]をヘリコプターが 飛[と]んでいます。	
\\	彼から度々メールが来ます。	度々	彼[かれ]から 度々[たびたび]メールが 来[き]ます。	
\\	この度はご結婚おめでとうございます。	この度	この 度[たび]はご 結婚[けっこん]おめでとうございます。	
\\	毎度ありがとうございます。	毎度	毎度[まいど]ありがとうございます。	
\\	有名人を間近で見たよ。	間近	有名人[ゆうめいじん]を 間近[まぢか]で 見[み]たよ。	
\\	こっちが近道です。	近道	こっちが 近道[ちかみち]です。	
\\	ともかく一度考え直しましょう。	ともかく	ともかく 一度考[いちど かんが]え 直[なお]しましょう。	
\\	彼は長年の友人です。	長年	彼[かれ]は 長年[ながねん]の 友人[ゆうじん]です。	
\\	彼がこのグループで一番年長です。	年長	彼[かれ]がこのグループで 一番[いちばん] 年長[ねんちょう]です。	
\\	長らくお待たせしました。	長らく	長[なが]らくお 待[ま]たせしました。	
\\	母が電話で長話をしているんだ。	長話	母[はは]が 電話[でんわ]で 長話[ながばなし]をしているんだ。	
\\	妹は短大を卒業しました。	短大	妹[いもうと]は 短大[たんだい]を 卒業[そつぎょう]しました。	
\\	会議中、ついあくびをしてしまいました。	つい	会議中[かいぎちゅう]、ついあくびをしてしまいました。	
\\	彼は短気な人です。	短気	彼[かれ]は 短気[たんき]な 人[ひと]です。	
\\	電車の中では新聞を広げないで。	広げる	電車[でんしゃ]の 中[なか]では 新聞[しんぶん]を 広[ひろ]げないで。	
\\	広場に子供が沢山集まっていたよ。	広場	広場[ひろば]に 子供[こども]が 沢山集[たくさん あつ]まっていたよ。	
\\	その噂はすぐに広まったよ。	広まる	その 噂[うわさ]はすぐに 広[ひろ]まったよ。	
\\	誰が噂を広めたんだろう。	広める	誰[だれ]が 噂[うわさ]を 広[ひろ]めたんだろう。	
\\	トンネルを抜けると海が見えたよ。	トンネル	トンネルを 抜[ぬ]けると 海[うみ]が 見[み]えたよ。	
\\	全力で走れ。	全力	全力[ぜんりょく]で 走[はし]れ。	
\\	あの歌手は3部門で賞を取ったの。	部門	あの 歌手[かしゅ]は3 部門[ぶもん]で 賞[しょう]を 取[と]ったの。	
\\	その事件の直後、捜査本部が設置された。	本部	その 事件[じけん]の 直後[ちょくご]、 捜査[そうさ] 本部[ほんぶ]が 設置[せっち]された。	
\\	部長に仕事の相談をしたんだ。	部長	部長[ぶちょう]に 仕事[しごと]の 相談[そうだん]をしたんだ。	
\\	これは機械の内部の問題です。	内部	これは 機械[きかい]の 内部[ないぶ]の 問題[もんだい]です。	
\\	彼は優秀な部下を持っているわね。	部下	彼[かれ]は 優秀[ゆうしゅう]な 部下[ぶか]を 持[も]っているわね。	
\\	栄養バランスの良い食事をしよう。	バランス	栄養[えいよう]バランスの 良[い]い 食事[しょくじ]をしよう。	
\\	絵の大部分が水に濡れてしまったな。	大部分	絵[え]の 大部分[だいぶぶん]が 水[みず]に 濡[ぬ]れてしまったな。	
\\	その国は経済大国よ。	大国	その 国[くに]は 経済[けいざい] 大国[たいこく]よ。	
\\	私は留学生として日本に入国しました。	入国	私[わたし]は 留学生[りゅうがくせい]として 日本[にっぽん]に 入国[にゅうこく]しました。	
\\	彼女は本国に帰りました。	本国	彼女[かのじょ]は 本国[ほんごく]に 帰[かえ]りました。	
\\	死んだら天国に行きたいです。	天国	死[し]んだら 天国[てんごく]に 行[い]きたいです。	
\\	どうしてもその訳を知りたい。	どうしても	どうしてもその 訳[わけ]を 知[し]りたい。	
\\	世の中にはいろいろな人がいます。	世の中	世[よ]の 中[なか]にはいろいろな 人[ひと]がいます。	
\\	彼女は中世の音楽が好きです。	中世	彼女[かのじょ]は 中世[ちゅうせい]の 音楽[おんがく]が 好[す]きです。	
\\	やっと私の作品が世に出たの。	世	やっと 私[わたし]の 作品[さくひん]が 世[よ]に 出[で]たの。	
\\	彼が犯人なのは明白です。	明白	彼[かれ]が 犯人[はんにん]なのは 明白[めいはく]です。	
\\	日本は鉄道がとても発達しています。	鉄道	日本[にっぽん]は 鉄道[てつどう]がとても 発達[はったつ]しています。	
\\	どうかお許しください。	どうか	どうかお 許[ゆる]しください。	
\\	この鍋は鉄でできています。	鉄	この 鍋[なべ]は 鉄[てつ]でできています。	
\\	肉体はいつか滅びます。	肉体	肉体[にくたい]はいつか 滅[ほろ]びます。	
\\	彼女は白菜の漬物が好きです。	白菜	彼女[かのじょ]は 白菜[はくさい]の 漬物[つけもの]が 好[す]きです。	
\\	テレビで日米野球をやっていますよ。	日米	テレビで 日米[にちべい] 野球[やきゅう]をやっていますよ。	
\\	母はいつも私の味方です。	味方	母[はは]はいつも 私[わたし]の 味方[みかた]です。	
\\	年末のセールはいつも込んでいるね。	年末	年末[ねんまつ]のセールはいつも 込[こ]んでいるね。	
\\	彼はスケートのプロです。	プロ	彼[かれ]はスケートのプロです。	
\\	今月末にカナダに行きます。	末	今月[こんげつ] 末[まつ]にカナダに 行[い]きます。	
\\	まだ料金は払っていないけど。	料金	まだ 料金[りょうきん]は 払[はら]っていないけど。	
\\	彼は地理に詳しいの。	地理	彼[かれ]は 地理[ちり]に 詳[くわ]しいの。	
\\	理解してくれてありがとう。	理解	理解[りかい]してくれてありがとう。	
\\	この問題を解けますか。	解く	この 問題[もんだい]を 解[と]けますか。	
\\	これは素晴らしいチャンスだ。	チャンス	これは 素晴[すば]らしいチャンスだ。	
\\	機械を分解してみたの。	分解	機械[きかい]を 分解[ぶんかい]してみたの。	
\\	やっと難しい問題が解けました。	解ける	やっと 難[むずか]しい 問題[もんだい]が 解[と]けました。	
\\	あの都市はオリンピックの有力な候補地です。	有力	あの 都市[とし]はオリンピックの 有力[ゆうりょく]な 候補地[こうほち]です。	
\\	資格を有する人のみ応募できます。	有する	資格[しかく]を 有[ゆう]する 人[ひと]のみ 応募[おうぼ]できます。	
\\	このトイレは有料です。	有料	このトイレは 有料[ゆうりょう]です。	
\\	今日はどうも体の調子が悪い。	どうも	今日[きょう]はどうも 体[からだ]の 調子[ちょうし]が 悪[わる]い。	
\\	無理はしないでください。	無理	無理[むり]はしないでください。	
\\	お飲み物は無料でございます。	無料	お 飲[の]み 物[もの]は 無料[むりょう]でございます。	
\\	無事、家に着きました。	無事	無事[ぶじ]、 家[いえ]に 着[つ]きました。	
\\	彼女は無口な人ですね。	無口	彼女[かのじょ]は 無口[むくち]な 人[ひと]ですね。	
\\	あの飛行機は間も無く着陸しますね。	間も無く	あの 飛行機[ひこうき]は 間[ま]も 無[な]く 着陸[ちゃくりく]しますね。	
\\	彼は一日中無言だったな。	無言	彼[かれ]は 一日中[いちにちじゅう] 無言[むごん]だったな。	
\\	最近、仕事がかなりハードです。	ハード	最近[さいきん]、 仕事[しごと]がかなりハードです。	
\\	無茶をしないでくださいね。	無茶	無茶[むちゃ]をしないでくださいね。	
\\	荷物を無理やりかばんに詰めたよ。	無理やり	荷物[にもつ]を 無理[むり]やりかばんに 詰[つ]めたよ。	
\\	それは彼女が作り上げた話です。	作り上げる	それは 彼女[かのじょ]が 作[つく]り 上[あ]げた 話[はなし]です。	
\\	その子が言っていることは作り話です。	作り話	その 子[こ]が 言[い]っていることは 作[つく]り 話[ばなし]です。	
\\	彼は昔、ドイツの大使でした。	大使	彼[かれ]は 昔[むかし]、ドイツの 大使[たいし]でした。	
\\	外の温度はマイナス3度です。	マイナス	外[そと]の 温度[おんど]はマイナス3 度[ど]です。	
\\	ボーナスの使い道を考えているところです。	使い道	ボーナスの 使[つか]い 道[みち]を 考[かんが]えているところです。	
\\	主人の使いで市役所に行くところです。	使い	主人[しゅじん]の 使[つか]いで 市役所[しやくしょ]に 行[い]くところです。	
\\	私に何か用ですか。	用	私[わたし]に 何[なに]か 用[よう]ですか。	
\\	心配は無用です。	無用	心配[しんぱい]は 無用[むよう]です。	
\\	イタリア旅行の費用は30万円です。	費用	イタリア 旅行[りょこう]の 費用[ひよう]は30 万円[まんえん]です。	
\\	とにかく現場へ行ってみましょう。	とにかく	とにかく 現場[げんば]へ 行[い]ってみましょう。	
\\	私は語学の勉強にかなりの時間を費やしています。	費やす	私[わたし]は 語学[ごがく]の 勉強[べんきょう]にかなりの 時間[じかん]を 費[つい]やしています。	
\\	あの店で野菜の安売りをしていましたよ。	安売り	あの 店[みせ]で 野菜[やさい]の 安売[やすう]りをしていましたよ。	
\\	彼は不動産の売買をしています。	売買	彼[かれ]は 不動産[ふどうさん]の 売買[ばいばい]をしています。	
\\	この店は8時に閉店します。	閉店	この 店[みせ]は8 時[じ]に 閉店[へいてん]します。	
\\	ここはチェーン店の本店です。	本店	ここはチェーン 店[てん]の 本店[ほんてん]です。	
\\	このパイプは詰まっていますよ。	パイプ	このパイプは 詰[つ]まっていますよ。	
\\	車の部品を取り替えたんだ。	部品	車[くるま]の 部品[ぶひん]を 取[と]り 替[か]えたんだ。	
\\	今日は日用品の買い物をした。	日用品	今日[きょう]は 日用品[にちようひん]の 買[か]い 物[もの]をした。	
\\	この階段は18段ありますね。	段	この 階[かい] 段[だん]は18 段[だん]ありますね。	
\\	野菜が値上がりしていますね。	値上がり	野菜[やさい]が 値上[ねあ]がりしていますね。	
\\	バス代が値上げされました。	値上げ	バス 代[だい]が 値上[ねあ]げされました。	
\\	ガソリンが値下がりしました。	値下がり	ガソリンが 値下[ねさ]がりしました。	
\\	空がたちまち曇ってきたね。	たちまち	空[そら]がたちまち 曇[くも]ってきたね。	
\\	電話料金が値下げされた。	値下げ	電話料金[でんわ りょうきん]が 値下[ねさ]げされた。	
\\	東京の地価は上がり続けているんだ。	地価	東京[とうきょう]の 地価[ちか]は 上[あ]がり 続[つづ]けているんだ。	
\\	彼はとても体格がいいわ。	体格	彼[かれ]はとても 体格[たいかく]がいいわ。	
\\	次の冬季オリンピックは2年後です。	冬季	次[つぎ]の 冬季[とうき]オリンピックは2 年後[ねんご]です。	
\\	鉄は熱すると曲がるんだ。	熱する	鉄[てつ]は 熱[ねっ]すると 曲[ま]がるんだ。	
\\	この経験はあなたにとってプラスになるでしょう。	プラス	この 経験[けいけん]はあなたにとってプラスになるでしょう。	
\\	この製品は低温で保存してください。	低温	この 製品[せいひん]は 低温[ていおん]で 保存[ほぞん]してください。	
\\	今朝の体温は36度でした。	体温	今朝[けさ]の 体温[たいおん]は36 度[ど]でした。	
\\	彼らは強い友情で結ばれているな。	友情	彼[かれ]らは 強[つよ]い 友情[ゆうじょう]で 結[むす]ばれているな。	
\\	この町の人たちには人情があるね。	人情	この 町[まち]の 人[ひと]たちには 人情[にんじょう]があるね。	
\\	こんなことも知らないとは情けない。	情けない	こんなことも 知[し]らないとは 情[なさ]けない。	
\\	彼は彼女をリードしながら踊ったね。	リード	彼[かれ]は 彼女[かのじょ]をリードしながら 踊[おど]ったね。	
\\	夜中もテレビで台風の報道をしていた。	報道	夜中[よなか]もテレビで 台風[たいふう]の 報道[ほうどう]をしていた。	
\\	実家の母から電報が来ました。	電報	実家[じっか]の 母[はは]から 電報[でんぽう]が 来[き]ました。	
\\	その車は中古で買ったんだ。	中古	その 車[くるま]は 中古[ちゅうこ]で 買[か]ったんだ。	
\\	おととい古本を3冊買いました。	古本	おととい 古本[ふるほん]を3 冊買[さつ か]いました。	
\\	彼は昔は貧乏だった。	昔	彼[かれ]は 昔[むかし]は 貧乏[びんぼう]だった。	
\\	彼は決して人の悪口を言わないの。	悪口	彼[かれ]は 決[けっ]して 人[ひと]の 悪口[わるくち]を 言[い]わないの。	
\\	たとえ、嵐になっても絶対に行く。	たとえ	たとえ、 嵐[あらし]になっても 絶対[ぜったい]に 行[い]く。	
\\	彼女は4年間熱心に勉強したわ。	熱心	彼女[かのじょ]は4 年間[ねんかん] 熱心[ねっしん]に 勉強[べんきょう]したわ。	
\\	私は良心に従って行動します。	良心	私[わたし]は 良心[りょうしん]に 従[したが]って 行動[こうどう]します。	
\\	彼女は内心どきどきしていましたよ。	内心	彼女[かのじょ]は 内心[ないしん]どきどきしていましたよ。	
\\	夜道は用心して歩きましょう。	用心	夜道[よみち]は 用心[ようじん]して 歩[ある]きましょう。	
\\	明日は会社の忘年会があります。	忘年会	明日[あした]は 会社[かいしゃ]の 忘年会[ぼうねんかい]があります。	
\\	なるべく早く仕事を終わらせてください。	なるべく	なるべく 早[はや]く 仕事[しごと]を 終[お]わらせてください。	
\\	彼の名前を度忘れしたぞ。	度忘れ	彼[かれ]の 名前[なまえ]を 度忘[どわす]れしたぞ。	
\\	彼は有能な知事ね。	知事	彼[かれ]は 有能[ゆうのう]な 知事[ちじ]ね。	
\\	ここからは未知の領域です。	未知	ここからは 未知[みち]の 領域[りょういき]です。	
\\	明日、詳細を通知します。	通知	明日[あす]、 詳細[しょうさい]を 通知[つうち]します。	
\\	彼女は政治について無知でした。	無知	彼女[かのじょ]は 政治[せいじ]について 無知[むち]でした。	
\\	講演のテーマは何ですか。	テーマ	講演[こうえん]のテーマは 何[なん]ですか。	
\\	彼は昔からの知人です。	知人	彼[かれ]は 昔[むかし]からの 知人[ちじん]です。	
\\	彼は笑いの天才だね。	天才	彼[かれ]は 笑[わら]いの 天才[てんさい]だね。	
\\	動物は本能のまま動くね。	本能	動物[どうぶつ]は 本能[ほんのう]のまま 動[うご]くね。	
\\	彼女はとても有能な部下です。	有能	彼女[かのじょ]はとても 有能[ゆうのう]な 部下[ぶか]です。	
\\	あの子の知能はとても高いそうです。	知能	あの 子[こ]の 知能[ちのう]はとても 高[たか]いそうです。	
\\	彼は無能だ。	無能	彼[かれ]は 無能[むのう]だ。	
\\	部屋を暗くしてムードを出してみたよ。	ムード	部屋[へや]を 暗[くら]くしてムードを 出[だ]してみたよ。	
\\	先日、初めて能を見に行きました。	能	先日[せんじつ]、 初[はじ]めて 能[のう]を 見[み]に 行[い]きました。	
\\	彼は勉強するしか能のない人間だね。	能	彼[かれ]は 勉強[べんきょう]するしか 能[のう]のない 人間[にんげん]だね。	
\\	明日、病院に検査に行くので少し不安です。	不安	明日[あす]、 病院[びょういん]に 検査[けんさ]に 行[い]くので 少[すこ]し 不安[ふあん]です。	
\\	私は彼のやり方には不満です。	不満	私[わたし]は 彼[かれ]のやり 方[かた]には 不満[ふまん]です。	
\\	今年はひどい水不足よ。	不足	今年[ことし]はひどい 水[みず] 不足[ぶそく]よ。	
\\	そこをどいてください。	どく	そこをどいてください。	
\\	レポートを1日で仕上げるのは不可能です。	不可能	レポートを1 日[にち]で 仕上[しあ]げるのは 不可能[ふかのう]です。	
\\	その程度の努力では不十分です。	不十分	その 程度[ていど]の 努力[どりょく]では 不十分[ふじゅうぶん]です。	
\\	その病気は原因不明と言われているんだよ。	不明	その 病気[びょうき]は 原因[げんいん] 不明[ふめい]と 言[い]われているんだよ。	
\\	不良品を返品したよ。	不良	不良[ふりょう] 品[ひん]を 返品[へんぴん]したよ。	
\\	残念ながら試験は不合格でした。	不合格	残念[ざんねん]ながら 試験[しけん]は 不合格[ふごうかく]でした。	
\\	彼はイタリアのマーケットを開拓しました。	マーケット	彼[かれ]はイタリアのマーケットを 開拓[かいたく]しました。	
\\	今朝、停電で電車が不通になったよ。	不通	今朝[けさ]、 停電[ていでん]で 電車[でんしゃ]が 不通[ふつう]になったよ。	
\\	月に一度母から便りが来ます。	便り	月[つき]に 一度母[いちど はは]から 便[たよ]りが 来[き]ます。	
\\	病院で大便の検査をした。	大便	病院[びょういん]で 大便[だいべん]の 検査[けんさ]をした。	
\\	今日は私が皿を洗う番ですね。	番	今日[きょう]は 私[わたし]が 皿[さら]を 洗[あら]う 番[ばん]ですね。	
\\	君の長所は明るいところだね。	長所	君[きみ]の 長所[ちょうしょ]は 明[あか]るいところだね。	
\\	彼女はファッションの専門家だ。	ファッション	彼女[かのじょ]はファッションの 専門家[せんもんか]だ。	
\\	すぐにあきらめてしまうのが彼の短所です。	短所	すぐにあきらめてしまうのが 彼[かれ]の 短所[たんしょ]です。	
\\	この本はページが所々破れているね。	所々	この 本[ほん]はページが 所々[ところどころ] 破[やぶ]れているね。	
\\	彼は名の通った会社に就職しました。	名	彼[かれ]は 名[な]の 通[とお]った 会社[かいしゃ]に 就職[しゅうしょく]しました。	
\\	彼は釣りの名人です。	名人	彼[かれ]は 釣[つ]りの 名人[めいじん]です。	
\\	その地名は聞いたことがないなあ。	地名	その 地名[ちめい]は 聞[き]いたことがないなあ。	
\\	ここは桜の名所です。	名所	ここは 桜[さくら]の 名所[めいしょ]です。	
\\	赤信号でブレーキを踏んだの。	ブレーキ	赤信号[あかしんごう]でブレーキを 踏[ふ]んだの。	
\\	彼女は町外れの工場で働いているよ。	町外れ	彼女[かのじょ]は 町外[まちはず]れの 工場[こうじょう]で 働[はたら]いているよ。	
\\	町中で彼女に出会ったよ。	町中	町中[まちなか]で 彼女[かのじょ]に 出会[であ]ったよ。	
\\	私は隣の村から来ました。	村	私[わたし]は 隣[となり]の 村[むら]から 来[き]ました。	
\\	彼の職場は都内にあります。	都内	彼[かれ]の 職場[しょくば]は 都内[とない]にあります。	
\\	彼は都心に住んでいます。	都心	彼[かれ]は 都心[としん]に 住[す]んでいます。	
\\	ミスは誰にでもあります。	ミス	ミスは 誰[だれ]にでもあります。	
\\	彼女は都の職員です。	都	彼女[かのじょ]は 都[と]の 職員[しょくいん]です。	
\\	彼は都会での生活を楽しんでいるの。	都会	彼[かれ]は 都会[とかい]での 生活[せいかつ]を 楽[たの]しんでいるの。	
\\	ミラノはファッションの都です。	都	ミラノはファッションの 都[みやこ]です。	
\\	彼女の様子を見てきます。	様子	彼女[かのじょ]の 様子[ようす]を 見[み]てきます。	
\\	この国の物価はここ10年でだいぶ上がったね。	物価	この 国[くに]の 物価[ぶっか]はここ10 年[ねん]でだいぶ 上[あ]がったね。	
\\	ヨットでクルージングを楽しみましたわ。	ヨット	ヨットでクルージングを 楽[たの]しみましたわ。	
\\	謎の物体が空を飛んでいます。	物体	謎[なぞ]の 物体[ぶったい]が 空[そら]を 飛[と]んでいます。	
\\	これは本物のダイヤモンドです。	本物	これは 本物[ほんもの]のダイヤモンドです。	
\\	彼は物理を専攻している。	物理	彼[かれ]は 物理[ぶつり]を 専攻[せんこう]している。	
\\	彼はいつも物事を深く考えるの。	物事	彼[かれ]はいつも 物事[ものごと]を 深[ふか]く 考[かんが]えるの。	
\\	この町の名物はぶどうです。	名物	この 町[まち]の 名物[めいぶつ]はぶどうです。	
\\	あの老人はとても物知りだね。	物知り	あの 老人[ろうじん]はとても 物知[ものし]りだね。	
\\	彼女は絵がなかなか上手です。	なかなか	彼女[かのじょ]は 絵[え]がなかなか 上手[じょうず]です。	
\\	彼女は物覚えがいい。	物覚え	彼女[かのじょ]は 物覚[ものおぼ]えがいい。	
\\	私は毎朝体重を測っています。	体重	私[わたし]は 毎朝[まいあさ] 体重[たいじゅう]を 測[はか]っています。	
\\	手軽に作れる料理を教えてください。	手軽	手軽[てがる]に 作[つく]れる 料理[りょうり]を 教[おし]えてください。	
\\	その事故で多量のガス漏れがあったね。	多量	その 事故[じこ]で 多量[たりょう]のガス 漏[も]れがあったね。	
\\	はかりで小麦粉の分量を量ったよ。	量る	はかりで 小麦粉[こむぎこ]の 分量[ぶんりょう]を 量[はか]ったよ。	
\\	俺はロックを聞くのが好きだ。	ロック	俺[おれ]はロックを 聞[き]くのが 好[す]きだ。	
\\	最近、食事の量を減らしています。	量	最近[さいきん]、 食事[しょくじ]の 量[りょう]を 減[へ]らしています。	
\\	昨日大量のゴミが出たの。	大量	昨日[きのう] 大量[たいりょう]のゴミが 出[で]たの。	
\\	年配の人に席を譲りました。	年配	年配[ねんぱい]の 人[ひと]に 席[せき]を 譲[ゆず]りました。	
\\	取引先への送金を済ませました。	送金	取引先[とりひきさき]への 送金[そうきん]を 済[す]ませました。	
\\	書類を郵送してください。	郵送	書類[しょるい]を 郵送[ゆうそう]してください。	
\\	わざわざ来てくれてありがとう。	わざわざ	わざわざ 来[き]てくれてありがとう。	
\\	危ないので子供からライターを取り上げました。	取り上げる	危[あぶ]ないので 子供[こども]からライターを 取[と]り 上[あ]げました。	
\\	彼は新しいアイデアをすぐ取り入れます。	取り入れる	彼[かれ]は 新[あたら]しいアイデアをすぐ 取[と]り 入[い]れます。	
\\	これは会社間の取り決めです。	取り決め	これは 会社間[かいしゃかん]の 取[と]り 決[き]めです。	
\\	このナスは畑で取れたばかりです。	取れる	このナスは 畑[はたけ]で 取[と]れたばかりです。	
\\	ホテルの予約を取り消したよ。	取り消す	ホテルの 予約[よやく]を 取[と]り 消[け]したよ。	
\\	彼は免許取消しの処分を受けたよ。	取消し	彼[かれ]は 免許[めんきょ] 取消[とりけ]しの 処分[しょぶん]を 受[う]けたよ。	
\\	彼はパイロットです。	パイロット	彼[かれ]はパイロットです。	
\\	彼はポケットから財布を取り出したの。	取り出す	彼[かれ]はポケットから 財布[さいふ]を 取[と]り 出[だ]したの。	
\\	郵便局に引っ越しの届けを出したよ。	届け	郵便局[ゆうびんきょく]に 引[ひ]っ 越[こ]しの 届[とど]けを 出[だ]したよ。	
\\	明日の待ち合わせは11時です。	待ち合わせ	明日[あす]の 待[ま]ち 合[あ]わせは11 時[じ]です。	
\\	入学式が待ち遠しいです。	待ち遠しい	入学式[にゅうがくしき]が 待[ま]ち 遠[どお]しいです。	
\\	彼と新宿で待ち合わせました。	待ち合わせる	彼[かれ]と 新宿[しんじゅく]で 待[ま]ち 合[あ]わせました。	
\\	明日はどうやら雨らしいよ。	どうやら	明日[あした]はどうやら 雨[あめ]らしいよ。	
\\	持ち物には名前を書いてください。	持ち物	持[も]ち 物[もの]には 名前[なまえ]を 書[か]いてください。	
\\	このコートは長持ちしている。	長持ち	このコートは 長持[ながも]ちしている。	
\\	この絵はとても値打ちがあります。	値打ち	この 絵[え]はとても 値打[ねう]ちがあります。	
\\	その事件について新聞に投書したんだ。	投書	その 事件[じけん]について 新聞[しんぶん]に 投書[とうしょ]したんだ。	
\\	役所で住民票をもらってきた。	役所	役所[やくしょ]で 住民票[じゅうみんひょう]をもらってきた。	
\\	ふと昔の事を思い出したんだ。	ふと	ふと 昔[むかし]の 事[こと]を 思[おも]い 出[だ]したんだ。	
\\	叔父は役人として30年働きました。	役人	叔父[おじ]は 役人[やくにん]として30 年働[ねん はたら]きました。	
\\	私は無事に役目を終えたよ。	役目	私[わたし]は 無事[ぶじ]に 役目[やくめ]を 終[お]えたよ。	
\\	彼女は弁護士の役を演じているんだ。	役	彼女[かのじょ]は 弁護士[べんごし]の 役[やく]を 演[えん]じているんだ。	
\\	息子に新しい歯が生えました。	生える	息子[むすこ]に 新[あたら]しい 歯[は]が 生[は]えました。	
\\	この村の人たちは長生きです。	長生き	この 村[むら]の 人[ひと]たちは 長生[ながい]きです。	
\\	彼はベテランの運転手です。	ベテラン	彼[かれ]はベテランの 運転手[うんてんしゅ]です。	
\\	彼はヒゲを生やしています。	生やす	彼[かれ]はヒゲを 生[は]やしています。	
\\	彼は生の魚が食べられません。	生	彼[かれ]は 生[なま]の 魚[さかな]が 食[た]べられません。	
\\	岩は無生物に分類される。	無生物	岩[いわ]は 無生物[むせいぶつ]に 分類[ぶんるい]される。	
\\	彼は素敵な男性です。	男性	彼[かれ]は 素敵[すてき]な 男性[だんせい]です。	
\\	人間は理性を持つ動物です。	理性	人間[にんげん]は 理性[りせい]を 持[も]つ 動物[どうぶつ]です。	
\\	今年はボーナスがたくさん出ました。	ボーナス	今年[ことし]はボーナスがたくさん 出[で]ました。	
\\	土産に日本酒をもらった。	土産	土産[みやげ]に 日本酒[にほんしゅ]をもらった。	
\\	家から駅まで徒歩3分です。	徒歩	家[いえ]から 駅[えき]まで 徒歩[とほ]3 分[ぷん]です。	
\\	私は哲学を学んでいます。	学ぶ	私[わたし]は 哲学[てつがく]を 学[まな]んでいます。	
\\	彼女は文学に興味を持っているのよ。	文学	彼女[かのじょ]は 文学[ぶんがく]に 興味[きょうみ]を 持[も]っているのよ。	
\\	今日は体育の授業があります。	体育	今日[きょう]は 体育[たいいく]の 授業[じゅぎょう]があります。	
\\	あの親子はそっくりだね。	そっくり	あの 親子[おやこ]はそっくりだね。	
\\	政治の体制を変える必要があります。	体制	政治[せいじ]の 体制[たいせい]を 変[か]える 必要[ひつよう]があります。	
\\	雨はだんだん強まります。	強まる	雨[あめ]はだんだん 強[つよ]まります。	
\\	火を強めてください。	強める	火[ひ]を 強[つよ]めてください。	
\\	彼は力強い演技をするね。	力強い	彼[かれ]は 力強[ちからづよ]い 演技[えんぎ]をするね。	
\\	彼女は強気な女性ですね。	強気	彼女[かのじょ]は 強気[つよき]な 女性[じょせい]ですね。	
\\	眼鏡のレンズを替えました。	レンズ	眼鏡[めがね]のレンズを 替[か]えました。	
\\	夜になって風が弱まったね。	弱まる	夜[よる]になって 風[かぜ]が 弱[よわ]まったね。	
\\	火を弱めてください。	弱める	火[ひ]を 弱[よわ]めてください。	
\\	彼は病気で弱っているんだ。	弱る	彼[かれ]は 病気[びょうき]で 弱[よわ]っているんだ。	
\\	彼は私の弱みを握っているんだ。	弱み	彼[かれ]は 私[わたし]の 弱[よわ]みを 握[にぎ]っているんだ。	
\\	彼は少し弱気になっています。	弱気	彼[かれ]は 少[すこ]し 弱気[よわき]になっています。	
\\	新しい仕事を引き受けたよ。	引き受ける	新[あたら]しい 仕事[しごと]を 引[ひ]き 受[う]けたよ。	
\\	彼はとても早いペースで走っているね。	ペース	彼[かれ]はとても 早[はや]いペースで 走[はし]っているね。	
\\	沈んだ船を引き上げたんだ。	引き上げる	沈[しず]んだ 船[ふね]を 引[ひ]き 上[あ]げたんだ。	
\\	先生が私の能力を引き出してくれました。	引き出す	先生[せんせい]が 私[わたし]の 能力[のうりょく]を 引[ひ]き 出[だ]してくれました。	
\\	彼女は息子を引き取ったの。	引き取る	彼女[かのじょ]は 息子[むすこ]を 引[ひ]き 取[と]ったの。	
\\	あと1000円値引きしましょう。	値引き	あと1000 円[えん] 値引[ねび]きしましょう。	
\\	会議が長引いているようです。	長引く	会議[かいぎ]が 長引[ながび]いているようです。	
\\	彼はマラソンに出場したよ。	マラソン	彼[かれ]はマラソンに 出場[しゅつじょう]したよ。	
\\	彼はまだ失恋を引きずっています。	引きずる	彼[かれ]はまだ 失恋[しつれん]を 引[ひ]きずっています。	
\\	この勝負は引き分けです。	引き分け	この 勝負[しょうぶ]は 引[ひ]き 分[わ]けです。	
\\	帰ろうとする友達を引き止めたんだ。	引き止める	帰[かえ]ろうとする 友達[ともだち]を 引[ひ]き 止[と]めたんだ。	
\\	彼を見習ってもっと勉強します。	見習う	彼[かれ]を 見習[みなら]ってもっと 勉強[べんきょう]します。	
\\	仕事には慣れも必要です。	慣れ	仕事[しごと]には 慣[な]れも 必要[ひつよう]です。	
\\	今日の遠足はよほど楽しかったらしい。	よほど	今日[きょう]の 遠足[えんそく]はよほど 楽[たの]しかったらしい。	
\\	水の温度に体を慣らしてから、潜ったほうがいいぞ。	慣らす	水[みず]の 温度[おんど]に 体[からだ]を 慣[な]らしてから、 潜[もぐ]ったほうがいいぞ。	
\\	明日は高校の入試です。	入試	明日[あした]は 高校[こうこう]の 入試[にゅうし]です。	
\\	彼はそのソフトウェアを試したの。	試す	彼[かれ]はそのソフトウェアを 試[ため]したの。	
\\	試しにこの曲を弾いてみてください。	試し	試[ため]しにこの 曲[きょく]を 弾[ひ]いてみてください。	
\\	今日、珍しい体験をしました。	体験	今日[きょう]、 珍[めずら]しい 体験[たいけん]をしました。	
\\	この製品は有害な化学物質を含んでいるわよ。	物質	この 製品[せいひん]は 有害[ゆうがい]な 化学[かがく] 物質[ぶっしつ]を 含[ふく]んでいるわよ。	
\\	父は若いころラグビーの選手でした。	ラグビー	父[ちち]は 若[わか]いころラグビーの 選手[せんしゅ]でした。	
\\	このメーカーの製品は高品質だわね。	品質	このメーカーの 製品[せいひん]は 高[こう] 品質[ひんしつ]だわね。	
\\	彼は仕事の本質をよく理解しているわね。	本質	彼[かれ]は 仕事[しごと]の 本質[ほんしつ]をよく 理解[りかい]しているわね。	
\\	応募者の年齢は問いません。	問う	応募者[おうぼしゃ]の 年齢[ねんれい]は 問[と]いません。	
\\	保険会社に問い合わせます。	問い合わせる	保険会社[ほけん がいしゃ]に 問[と]い 合[あ]わせます。	
\\	この問いに答えられますか。	問い	この 問[と]いに 答[こた]えられますか。	
\\	私は毎朝ラッシュの電車に乗っています。	ラッシュ	私[わたし]は 毎朝[まいあさ]ラッシュの 電車[でんしゃ]に 乗[の]っています。	
\\	この街には家具の問屋がたくさんあります。	問屋	この 街[まち]には 家具[かぐ]の 問屋[とんや]がたくさんあります。	
\\	ここが話題のレストランです。	話題	ここが 話題[わだい]のレストランです。	
\\	この本の題名は「坊ちゃん」です。	題名	この 本[ほん]の 題名[だいめい]は
\\	坊[ぼっ]ちゃん」です。	
\\	その曲の題を思い出せません。	題	その 曲[きょく]の 題[だい]を 思[おも]い 出[だ]せません。	
\\	日本語の単語をいくつ知っていますか。	単語	日本語[にほんご]の 単語[たんご]をいくつ 知[し]っていますか。	
\\	ロープを使って崖を下りたんだ。	ロープ	ロープを 使[つか]って 崖[がけ]を 下[お]りたんだ。	
\\	心配しないで、単に眠いだけです。	単に	心配[しんぱい]しないで、 単[たん]に 眠[ねむ]いだけです。	
\\	もうすぐ目標の地点に到達します。	地点	もうすぐ 目標[もくひょう]の 地点[ちてん]に 到達[とうたつ]します。	
\\	国語のテストは満点でした。	満点	国語[こくご]のテストは 満点[まんてん]でした。	
\\	その点は心配ありません。	点	その 点[てん]は 心配[しんぱい]ありません。	
\\	その仕事に多数の応募があったよ。	多数	その 仕事[しごと]に 多数[たすう]の 応募[おうぼ]があったよ。	
\\	今、ダムの水が少ないね。	ダム	今[いま]、ダムの 水[みず]が 少[すく]ないね。	
\\	複数のレポートをひとつにまとめています。	複数	複数[ふくすう]のレポートをひとつにまとめています。	
\\	宇宙には無数の星があります。	無数	宇宙[うちゅう]には 無数[むすう]の 星[ほし]があります。	
\\	今月は出勤日数が多いです。	日数	今月[こんげつ]は 出勤[しゅっきん] 日数[にっすう]が 多[おお]いです。	
\\	今回はテストの点数が悪かった。	点数	今回[こんかい]はテストの 点数[てんすう]が 悪[わる]かった。	
\\	は単数で
\\	は複数です。	単数	
\\	[ぺん]は 単数[たんすう]で 
\\	[ぺんず]は 複数[ふくすう]です。	
\\	お手数ですがよろしくお願いします。	手数	お 手数[てすう]ですがよろしくお 願[ねが]いします。	
\\	サーバーがダウンしています。	ダウン	サーバーがダウンしています。	
\\	参加者の人数を教えてください。	人数	参加者[さんかしゃ]の 人数[にんずう]を 教[おし]えてください。	
\\	先生の周りに集まってください。	回り	先生[せんせい]の 周[まわ]りに 集[あつ]まってください。	
\\	今日は回り道して帰ろう。	回り道	今日[きょう]は 回[まわ]り 道[みち]して 帰[かえ]ろう。	
\\	コピーの枚数を数えてください。	枚数	コピーの 枚数[まいすう]を 数[かぞ]えてください。	
\\	この勝負はあなたの負けです。	負け	この 勝負[しょうぶ]はあなたの 負[ま]けです。	
\\	彼は毎日トレーニングをするの。	トレーニング	彼[かれ]は 毎日[まいにち]トレーニングをするの。	
\\	交通費は個人負担です。	負担	交通費[こうつうひ]は 個人[こじん] 負担[ふたん]です。	
\\	私たち夫婦は家事を分担しています。	分担	私[わたし]たち 夫婦[ふうふ]は 家事[かじ]を 分担[ぶんたん]しています。	
\\	当初の計画ではもっと早く終わるはずでした。	当初	当初[とうしょ]の 計画[けいかく]ではもっと 早[はや]く 終[お]わるはずでした。	
\\	その事件については当局が調査しています。	当局	その 事件[じけん]については 当局[とうきょく]が 調査[ちょうさ]しています。	
\\	入場券は当日でも買えますよ。	当日	入場券[にゅうじょうけん]は 当日[とうじつ]でも 買[か]えますよ。	
\\	母親は娘の髪をそっとなでたの。	そっと	母親[ははおや]は 娘[むすめ]の 髪[かみ]をそっとなでたの。	
\\	彼女は急いで怪我の手当てをしたよ。	手当て	彼女[かのじょ]は 急[いそ]いで 怪我[けが]の 手当[てあ]てをしたよ。	
\\	彼女は当分学校を休むそうです。	当分	彼女[かのじょ]は 当分[とうぶん] 学校[がっこう]を 休[やす]むそうです。	
\\	当人は意外に平気なようね。	当人	当人[とうにん]は 意外[いがい]に 平気[へいき]なようね。	
\\	今日は私が掃除の当番です。	当番	今日[きょう]は 私[わたし]が 掃除[そうじ]の 当番[とうばん]です。	
\\	この部屋は日当たりがいい。	日当たり	この 部屋[へや]は 日当[ひあ]たりがいい。	
\\	ここは天然の温泉です。	天然	ここは 天然[てんねん]の 温泉[おんせん]です。	
\\	このブランドのマークは可愛いですね。	マーク	このブランドのマークは 可愛[かわい]いですね。	
\\	今日は英語の文法を勉強します。	文法	今日[きょう]は 英語[えいご]の 文法[ぶんぽう]を 勉強[べんきょう]します。	
\\	彼女は法学部の学生です。	法学部	彼女[かのじょ]は 法学部[ほうがくぶ]の 学生[がくせい]です。	
\\	国民は法に従わなければならないよ。	法	国民[こくみん]は 法[ほう]に 従[したが]わなければならないよ。	
\\	勝利するには法則があるそうだ。	法則	勝利[しょうり]するには 法則[ほうそく]があるそうだ。	
\\	最近、不規則な生活をしている。	不規則	最近[さいきん]、 不規則[ふきそく]な 生活[せいかつ]をしている。	
\\	財布の中にたった1000円しかないよ。	たった	財布[さいふ]の 中[なか]にたった 1000円[せんえん]しかないよ。	
\\	彼は新聞記者を経て作家になりました。	経る	彼[かれ]は 新聞記者[しんぶん きしゃ]を 経[へ]て 作家[さっか]になりました。	
\\	資格があると就職に有利です。	有利	資格[しかく]があると 就職[しゅうしょく]に 有利[ゆうり]です。	
\\	彼は今、不利な立場にいます。	不利	彼[かれ]は 今[いま]、 不利[ふり]な 立場[たちば]にいます。	
\\	あの犬はとても利口ですね。	利口	あの 犬[いぬ]はとても 利口[りこう]ですね。	
\\	借金に利子をつけて返したの。	利子	借金[しゃっきん]に 利子[りし]をつけて 返[かえ]したの。	
\\	道でたまたま友達に会った。	たまたま	道[みち]でたまたま 友達[ともだち]に 会[あ]った。	
\\	私の息子は左利きです。	左利き	私[わたし]の 息子[むすこ]は 左利[ひだりき]きです。	
\\	先月の利益は200万円でした。	利益	先月[せんげつ]の 利益[りえき]は200 万円[まんえん]でした。	
\\	昨日の話し合いは有益でした。	有益	昨日[きのう]の 話[はな]し 合[あ]いは 有益[ゆうえき]でした。	
\\	彼は買収されたらしいわ。	買収	彼[かれ]は 買収[ばいしゅう]されたらしいわ。	
\\	税金の額は年収によって変わります。	年収	税金[ぜいきん]の 額[がく]は 年収[ねんしゅう]によって 変[か]わります。	
\\	隣に木造の家が建ったね。	木造	隣[となり]に 木造[もくぞう]の 家[いえ]が 建[た]ったね。	
\\	この曲はテンポが速いですね。	テンポ	この 曲[きょく]はテンポが 速[はや]いですね。	
\\	私と彼が出会ったのは必然だったの。	必然	私[わたし]と 彼[かれ]が 出会[であ]ったのは 必然[ひつぜん]だったの。	
\\	このビルは完成までに2年を要した。	要する	このビルは 完成[かんせい]までに2 年[ねん]を 要[よう]した。	
\\	不要になったパソコンを処分したんだ。	不要	不要[ふよう]になったパソコンを 処分[しょぶん]したんだ。	
\\	要するに時機を待つべきだ。	要するに	要[よう]するに 時機[じき]を 待[ま]つべきだ。	
\\	不必要なファイルは削除してください。	不必要	不必要[ふひつよう]なファイルは 削除[さくじょ]してください。	
\\	彼女と旅行のプランを考えました。	プラン	彼女[かのじょ]と 旅行[りょこう]のプランを 考[かんが]えました。	
\\	話の要点だけ教えてください。	要点	話[はなし]の 要点[ようてん]だけ 教[おし]えてください。	
\\	子供は親の愛を求めます。	求める	子供[こども]は 親[おや]の 愛[あい]を 求[もと]めます。	
\\	その国の政府は各国に支援を要請したのよ。	要請	その 国[くに]の 政府[せいふ]は 各国[かっこく]に 支援[しえん]を 要請[ようせい]したのよ。	
\\	額に汗をかいたよ。	額	額[ひたい]に 汗[あせ]をかいたよ。	
\\	100メートル走のタイムを計ったんだ。	計る	100メートル 走[そう]のタイムを 計[はか]ったんだ。	
\\	家に着いてほっとしたよ。	ほっと	家[いえ]に 着[つ]いてほっとしたよ。	
\\	体温計が壊れてしまった。	体温計	体温計[たいおんけい]が 壊[こわ]れてしまった。	
\\	今日は日差しが強いですね。	日差し	今日[きょう]は 日差[ひざ]しが 強[つよ]いですね。	
\\	30センチの物差しをください。	物差し	30センチの 物差[ものさ]しをください。	
\\	みんなで役割を分担しましょう。	役割	みんなで 役割[やくわり]を 分担[ぶんたん]しましょう。	
\\	二つの薬品を1対3の割合で混ぜたんだ。	割合	二[ふた]つの 薬品[やくひん]を1 対3[たい 
\\	の 割合[わりあい]で 混[ま]ぜたんだ。	
\\	なるほど、よく分かりました。	なるほど	なるほど、よく 分[わ]かりました。	
\\	チケットの割り当ては1人10枚です。	割り当て	チケットの 割[わ]り 当[あ]ては 1人10枚[ひとり じゅうまい]です。	
\\	全員に作業が割り当てられました。	割り当てる	全員[ぜんいん]に 作業[さぎょう]が 割[わ]り 当[あ]てられました。	
\\	定価から2000円割り引きますよ。	割り引く	定価[ていか]から2 000円[せんえん] 割[わ]り 引[び]きますよ。	
\\	この映画は割に面白いわ。	割に	この 映画[えいが]は 割[わり]に 面白[おもしろ]いわ。	
\\	3日に1度の割りで彼からメールが来るの。	割り	3日[みっか]に1 度[ど]の 割[わ]りで 彼[かれ]からメールが 来[く]るの。	
\\	今日は全品10
\\	割り引きです。	割り引き	今日[きょう]は 全品10[ぜんぴん 
\\	割[わ]り 引[び]きです。	
\\	今日は6時からバイトです。	バイト	今日[きょう]は6 時[じ]からバイトです。	
\\	今回のテストは割合に簡単でした。	割合に	今回[こんかい]のテストは 割合[わりあい]に 簡単[かんたん]でした。	
\\	仕事の残りは家でします。	残り	仕事[しごと]の 残[のこ]りは 家[いえ]でします。	
\\	ゴミを残らず拾ったよ。	残らず	ゴミを 残[のこ]らず 拾[ひろ]ったよ。	
\\	飲み屋の払いがたまっているんだ。	払い	飲[の]み 屋[や]の 払[はら]いがたまっているんだ。	
\\	雨が強かったので引き返したよ。	引き返す	雨[あめ]が 強[つよ]かったので 引[ひ]き 返[かえ]したよ。	
\\	車を壁にぶつけてしまいました。	ぶつける	車[くるま]を 壁[かべ]にぶつけてしまいました。	
\\	彼はチャンピオンのタイトルを取り返したね。	取り返す	彼[かれ]はチャンピオンのタイトルを 取[と]り 返[かえ]したね。	
\\	彼はプロジェクトへの参加を申し出たよ。	申し出る	彼[かれ]はプロジェクトへの 参加[さんか]を 申[もう]し 出[で]たよ。	
\\	結果を申し上げます。	申し上げる	結果[けっか]を 申[もう]し 上[あ]げます。	
\\	私は鈴木と申します。	申す	私[わたくし]は 鈴木[すずき]と 申[もう]します。	
\\	機内に荷物を持ち込んだの。	持ち込む	機内[きない]に 荷物[にもつ]を 持[も]ち 込[こ]んだの。	
\\	車のタイヤを換えた。	タイヤ	車[くるま]のタイヤを 換[か]えた。	
\\	3月に大学を卒業の見込みです。	見込み	
\\	月[がつ]に 大学[だいがく]を 卒業[そつぎょう]の 見込[みこ]みです。	
\\	今日、スポーツジムの申し込みをしました。	申し込み	今日[きょう]、スポーツジムの 申[もう]し 込[こ]みをしました。	
\\	授業料を学校に払い込みました。	払い込む	授業料[じゅぎょうりょう]を 学校[がっこう]に 払[はら]い 込[こ]みました。	
\\	彼は薬を一気に飲み込んだ。	飲み込む	彼[かれ]は 薬[くすり]を 一気[いっき]に 飲[の]み 込[こ]んだ。	
\\	車が前に割り込んできた。	割り込む	車[くるま]が 前[まえ]に 割[わ]り 込[こ]んできた。	
\\	ダイエットをしてお腹が引っ込みました。	引っ込む	ダイエットをしてお 腹[なか]が 引[ひ]っ 込[こ]みました。	
\\	彼女はマイクを持って話し始めたの。	マイク	彼女[かのじょ]はマイクを 持[も]って 話[はな]し 始[はじ]めたの。	
\\	私は人込みが好きではありません。	人込み	私[わたし]は 人込[ひとご]みが 好[す]きではありません。	
\\	今回は長期の滞在です。	長期	今回[こんかい]は 長期[ちょうき]の 滞在[たいざい]です。	
\\	明日から2週間、短期のアルバイトをします。	短期	明日[あした]から2 週間[しゅうかん]、 短期[たんき]のアルバイトをします。	
\\	彼女は短期大学で日本文学を勉強しました。	短期大学	彼女[かのじょ]は 短期大学[たんきだいがく]で 日本文学[にほん ぶんがく]を 勉強[べんきょう]しました。	
\\	宇宙は有限だと思いますか。	有限	宇宙[うちゅう]は 有限[ゆうげん]だと 思[おも]いますか。	
\\	最近、生活にゆとりがでてきました。	ゆとり	最近[さいきん]、 生活[せいかつ]にゆとりがでてきました。	
\\	資源は無限ではありません。	無限	資源[しげん]は 無限[むげん]ではありません。	
\\	このサイトでは音楽を無制限でダウンロードできる。	無制限	このサイトでは 音楽[おんがく]を 無制限[むせいげん]でダウンロードできる。	
\\	彼は値切るのが上手です。	値切る	彼[かれ]は 値切[ねぎ]るのが 上手[じょうず]です。	
\\	コンサートの前売り券を手に入れたんだ。	前売り券	コンサートの 前売[まえう]り 券[けん]を 手[て]に 入[い]れたんだ。	
\\	私と彼は同じ年代です。	年代	私[わたし]と 彼[かれ]は 同[おな]じ 年代[ねんだい]です。	
\\	彼は遠くをぼんやり見ていたの。	ぼんやり	彼[かれ]は 遠[とお]くをぼんやり 見[み]ていたの。	
\\	ここで代金をお支払いください。	代金	ここで 代金[だいきん]をお 支払[しはら]いください。	
\\	私は料理人を目指しています。	目指す	私[わたし]は 料理人[りょうりにん]を 目指[めざ]しています。	
\\	みんなが彼の指差す方を見たんだ。	指差す	みんなが 彼[かれ]の 指差[ゆびさ]す 方[ほう]を 見[み]たんだ。	
\\	彼女は人差し指を怪我したの。	人差し指	彼女[かのじょ]は 人差[ひとさ]し 指[ゆび]を 怪我[けが]したの。	
\\	中指をドアに挟んでしまった。	中指	中指[なかゆび]をドアに 挟[はさ]んでしまった。	
\\	彼は来年定年を迎える。	定年	彼[かれ]は 来年[らいねん] 定年[ていねん]を 迎[むか]える。	
\\	頭が痛くてたまらない。	たまらない	頭[あたま]が 痛[いた]くてたまらない。	
\\	最近、体調が少し不安定です。	不安定	最近[さいきん]、 体調[たいちょう]が 少[すこ]し 不安定[ふあんてい]です。	
\\	降りてください、定員オーバーです。	定員	降[お]りてください、 定員[ていいん]オーバーです。	
\\	定期演奏会は年に4回あります。	定期	定期[ていき] 演奏会[えんそうかい]は 年[ねん]に4 回[かい]あります。	
\\	この本の定価は525円です。	定価	この 本[ほん]の 定価[ていか]は525 円[えん]です。	
\\	この件の担当者は未定です。	未定	この 件[けん]の 担当者[たんとうしゃ]は 未定[みてい]です。	
\\	車のモーターを修理したぜ!	モーター	車[くるま]のモーターを 修理[しゅうり]したぜ!	
\\	昼の定食は3種類あります。	定食	昼[ひる]の 定食[ていしょく]は3 種類[しゅるい]あります。	
\\	この店は水曜が定休日です。	定休日	この 店[みせ]は 水曜[すいよう]が 定休日[ていきゅうび]です。	
\\	車の費用、予算オーバーだ。	予算	車[くるま]の 費用[ひよう]、 予算[よさん]オーバーだ。	
\\	予報では明日は雨ですね。	予報	予報[よほう]では 明日[あした]は 雨[あめ]ですね。	
\\	彼の予言は当たったことがないね。	予言	彼[かれ]の 予言[よげん]は 当[あ]たったことがないね。	
\\	他人でも悲しいのだから、まして本人はどれほどでしょう。	まして	他人[たにん]でも 悲[かな]しいのだから、まして 本人[ほんにん]はどれほどでしょう。	
\\	これは一束で300円です。	束	これは 一[ひと] 束[たば]で300 円[えん]です。	
\\	卒業式に花束をもらいました。	花束	卒業式[そつぎょうしき]に 花束[はなたば]をもらいました。	
\\	予定が変更になりました。	変更	予定[よてい]が 変更[へんこう]になりました。	
\\	最近の子供たちは夜更かしです。	夜更かし	最近[さいきん]の 子供[こども]たちは 夜更[よふ]かしです。	
\\	大雨で川の水かさが増しているな。	増す	大雨[おおあめ]で 川[かわ]の 水[みず]かさが 増[ま]しているな。	
\\	犯人はピストルを持っているわ。	ピストル	犯人[はんにん]はピストルを 持[も]っているわ。	
\\	生産コスト増大のため、値上げします。	増大	生産[せいさん]コスト 増大[ぞうだい]のため、 値上[ねあ]げします。	
\\	この数年、体重は増減していません。	増減	この 数年[すうねん]、 体重[たいじゅう]は 増減[ぞうげん]していません。	
\\	船が長い航海に乗り出したの。	乗り出す	船[ふね]が 長[なが]い 航海[こうかい]に 乗[の]り 出[だ]したの。	
\\	あの駅で学生がたくさん乗り込んだね。	乗り込む	あの 駅[えき]で 学生[がくせい]がたくさん 乗[の]り 込[こ]んだね。	
\\	この駅でたくさんの人が乗り降りしますね。	乗り降り	この 駅[えき]でたくさんの 人[ひと]が 乗[の]り 降[お]りしますね。	
\\	玄関に車を着けます。	着ける	玄関[げんかん]に 車[くるま]を 着[つ]けます。	
\\	たっぷりとマッサージしてもらいました。	たっぷり	たっぷりとマッサージしてもらいました。	
\\	水着に着替えました。	水着	水着[みずぎ]に 着替[きが]えました。	
\\	学校で勉強したことが役立った。	役立つ	学校[がっこう]で 勉強[べんきょう]したことが 役立[やくだ]った。	
\\	彼は急に立ち上がったの。	立ち上がる	彼[かれ]は 急[きゅう]に 立[た]ち 上[あ]がったの。	
\\	私は中立の立場を取っています。	中立	私[わたし]は 中立[ちゅうりつ]の 立場[たちば]を 取[と]っています。	
\\	ここは都立の病院です。	都立	ここは 都立[とりつ]の 病院[びょういん]です。	
\\	来てよかったとつくづく思います。	つくづく	来[き]てよかったとつくづく 思[おも]います。	
\\	この辺は住宅が立て込んで います。	立て込む	この 辺[へん]は 住宅[じゅうたく]が 立[た]て 込[こ]んでいます。	
\\	帰宅中、夕立にあったの。	夕立	帰宅中[きたくちゅう]、 夕立[ゆうだち]にあったの。	
\\	人々は立ち止まって上を見上げたの。	立ち止まる	人々[ひとびと]は 立[た]ち 止[ど]まって 上[うえ]を 見上[みあ]げたの。	
\\	みなさん、着席してください。	着席	みなさん、 着席[ちゃくせき]してください。	
\\	走者が次々にゴールしました。	次々に	走者[そうしゃ]が 次々[つぎつぎ]にゴールしました。	
\\	どうせ間に合わないならゆっくり行こう。	どうせ	どうせ 間[ま]に 合[あ]わないならゆっくり 行[い]こう。	
\\	彼はトップに次ぐ好成績でした。	次ぐ	彼[かれ]はトップに 次[つ]ぐ 好成績[こうせいせき]でした。	
\\	電話があったら取り次いでください。	取り次ぐ	電話[でんわ]があったら 取[と]り 次[つ]いでください。	
\\	読みたい章を目次で探しました。	目次	読[よ]みたい 章[しょう]を 目次[もくじ]で 探[さが]しました。	
\\	彼に不運な出来事が起こったの。	不運	彼[かれ]に 不運[ふうん]な 出来事[できごと]が 起[お]こったの。	
\\	今、経済の動向は読みにくいわ。	動向	今[いま]、 経済[けいざい]の 動向[どうこう]は 読[よ]みにくいわ。	
\\	彼は動作が機敏です。	動作	彼[かれ]は 動作[どうさ]が 機敏[きびん]です。	
\\	最近彼は生活のリズムが乱れています。	リズム	最近彼[さいきん かれ]は 生活[せいかつ]のリズムが 乱[みだ]れています。	
\\	この車の動力は電気です。	動力	この 車[くるま]の 動力[どうりょく]は 電気[でんき]です。	
\\	部下が素晴らしい働きをしたな。	働き	部下[ぶか]が 素晴[すば]らしい 働[はたら]きをしたな。	
\\	私は早朝のジョギングを日課にしています。	早朝	私[わたし]は 早朝[そうちょう]のジョギングを 日課[にっか]にしています。	
\\	集合時間を30分早めました。	早める	集合時間[しゅうごうじかん]を30 分[ぷん] 早[はや]めました。	
\\	早まらないでよく考えましょう。	早まる	早[はや]まらないでよく 考[かんが]えましょう。	
\\	この車は左ハンドルです。	ハンドル	この 車[くるま]は 左[ひだり]ハンドルです。	
\\	新幹線の速度はどれくらいですか。	速度	新幹線[しんかんせん]の 速度[そくど]はどれくらいですか。	
\\	速達便なら明日の午前中に届きます。	速達便	速達便[そくたつびん]なら 明日[あした]の 午前中[ごぜんちゅう]に 届[とど]きます。	
\\	寝坊して新幹線に乗り遅れたよ。	乗り遅れる	寝坊[ねぼう]して 新幹線[しんかんせん]に 乗[の]り 遅[おく]れたよ。	
\\	私たちの旅は始めはよかったんだ。	始め	私[わたし]たちの 旅[たび]は 始[はじ]めはよかったんだ。	
\\	いよいよ劇の始まりですね。	始まり	いよいよ 劇[げき]の 始[はじ]まりですね。	
\\	デジカメで撮った写真をプリントしたんだ。	プリント	デジカメで 撮[と]った 写真[しゃしん]をプリントしたんだ。	
\\	部下の方が年始の挨拶に見えましたよ。	年始	部下[ぶか]の 方[かた]が 年始[ねんし]の 挨拶[あいさつ]に 見[み]えましたよ。	
\\	妻の不在中、夫は毎日外食したんだ。	不在	妻[つま]の 不在[ふざい] 中[ちゅう]、 夫[おっと]は 毎日外食[まいにち がいしょく]したんだ。	
\\	やっと努力が実りました。	実る	やっと 努力[どりょく]が 実[みの]りました。	
\\	庭の木が赤い実をつけた。	実	庭[にわ]の 木[き]が 赤[あか]い 実[み]をつけた。	
\\	うっかり目的地を通り過ぎた。	通り過ぎる	うっかり 目的地[もくてきち]を 通[とお]り 過[す]ぎた。	
\\	新しい車が発売された。	発売	新[あたら]しい 車[くるま]が 発売[はつばい]された。	
\\	夏の日差しがまぶしかったよ。	まぶしい	夏[なつ]の 日差[ひざ]しがまぶしかったよ。	
\\	彼は明日メキシコへ発ちます。	発つ	彼[かれ]は 明日[あした]メキシコへ 発[た]ちます。	
\\	その雑誌は年に4回発行されているんだ。	発行	その 雑誌[ざっし]は 年[ねん]に4 回[かい] 発行[はっこう]されているんだ。	
\\	通信技術の発達は目覚ましいな。	発達	通信技術[つうしん ぎじゅつ]の 発達[はったつ]は 目覚[めざ]ましいな。	
\\	彼は偉大な発明王です。	発明	彼[かれ]は 偉大[いだい]な 発明[はつめい] 王[おう]です。	
\\	ここでは太陽エネルギーを使って発電しています。	発電	ここでは 太陽[たいよう]エネルギーを 使[つか]って 発電[はつでん]しています。	
\\	夕食はよそでごちそうになりました。	よそ	夕食[ゆうしょく]はよそでごちそうになりました。	
\\	すぐそこに発電所があります。	発電所	すぐそこに 発電所[はつでんしょ]があります。	
\\	娘が突然、発熱したんだ。	発熱	娘[むすめ]が 突然[とつぜん]、 発熱[はつねつ]したんだ。	
\\	彼はとても表情が豊かですね。	表情	彼[かれ]はとても 表情[ひょうじょう]が 豊[ゆた]かですね。	
\\	この用紙に名前を書いてください。	用紙	この 用紙[ようし]に 名前[なまえ]を 書[か]いてください。	
\\	その人気アイドルが今月の表紙だよ。	表紙	その 人気[にんき]アイドルが 今月[こんげつ]の 表紙[ひょうし]だよ。	
\\	お酒をおつぎしましょう。	つぐ	お 酒[さけ]をおつぎしましょう。	
\\	テストを白紙で出したの。	白紙	テストを 白紙[はくし]で 出[だ]したの。	
\\	秋は虫の音が心地良いです。	音	秋[あき]は 虫[むし]の 音[ね]が 心地良[ここちい]いです。	
\\	この単語を発音してください。	発音	この 単語[たんご]を 発音[はつおん]してください。	
\\	物音がしたので見に行った。	物音	物音[ものおと]がしたので 見[み]に 行[い]った。	
\\	旅行は父の老後の楽しみです。	楽しみ	旅行[りょこう]は 父[ちち]の 老後[ろうご]の 楽[たの]しみです。	
\\	昔はよくディスコに行ったな。	ディスコ	昔[むかし]はよくディスコに 行[い]ったな。	
\\	彼には楽な仕事が与えられたよ。	楽	彼[かれ]には 楽[らく]な 仕事[しごと]が 与[あた]えられたよ。	
\\	彼女は薬品を戸棚から出したんだ。	薬品	彼女[かのじょ]は 薬品[やくひん]を 戸棚[とだな]から 出[だ]したんだ。	
\\	薬局で目薬を買いました。	薬局	薬局[やっきょく]で 目薬[めぐすり]を 買[か]いました。	
\\	目が疲れたので目薬をさしたよ。	目薬	目[め]が 疲[つか]れたので 目薬[めぐすり]をさしたよ。	
\\	あまり欲を出しちゃだめだよ。	欲	あまり 欲[よく]を 出[だ]しちゃだめだよ。	
\\	時には自分の欲求を抑えることも必要です。	欲求	時[とき]には 自分[じぶん]の 欲求[よっきゅう]を 抑[おさ]えることも 必要[ひつよう]です。	
\\	この服はあなたにぴったりですね。	ぴったり	この 服[ふく]はあなたにぴったりですね。	
\\	月の表面にはクレーターがたくさんあるね。	表面	月[つき]の 表面[ひょうめん]にはクレーターがたくさんあるね。	
\\	ここがいちばん面白い場面です。	場面	ここがいちばん 面白[おもしろ]い 場面[ばめん]です。	
\\	私の家は川に面しています。	面する	私[わたし]の 家[いえ]は 川[かわ]に 面[めん]しています。	
\\	沖縄方面にお出かけの方は台風にご注意ください。	方面	沖縄[おきなわ] 方面[ほうめん]にお 出[で]かけの 方[かた]は 台風[たいふう]にご 注意[ちゅうい]ください。	
\\	テニスコートの半面を使って試合をしたの。	半面	テニスコートの 半面[はんめん]を 使[つか]って 試合[しあい]をしたの。	
\\	まさか車が当たるとは思わなかった。	まさか	まさか 車[くるま]が 当[あ]たるとは 思[おも]わなかった。	
\\	彼の意外な面を見た。	面	彼[かれ]の 意外[いがい]な 面[めん]を 見[み]た。	
\\	合格を真っ先に両親に伝えたの。	真っ先	合格[ごうかく]を 真[ま]っ 先[さき]に 両親[りょうしん]に 伝[つた]えたの。	
\\	真夏のビールはおいしい。	真夏	真夏[まなつ]のビールはおいしい。	
\\	彼は真夜中に帰ってきたの。	真夜中	彼[かれ]は 真夜中[まよなか]に 帰[かえ]ってきたの。	
\\	今は試験の真っ最中です。	真っ最中	今[いま]は 試験[しけん]の 真[ま]っ 最中[さいちゅう]です。	
\\	観衆がどっと笑ったんだ。	どっと	観衆[かんしゅう]がどっと 笑[わら]ったんだ。	
\\	彼女は真心をこめて彼にケーキを作ったよ。	真心	彼女[かのじょ]は 真心[まごころ]をこめて 彼[かれ]にケーキを 作[つく]ったよ。	
\\	アルコールは無色です。	無色	アルコールは 無色[むしょく]です。	
\\	彼女の人生はばら色だったの。	ばら色	彼女[かのじょ]の 人生[じんせい]はばら 色[いろ]だったの。	
\\	箱に水色のリボンがかかっていたの。	水色	箱[はこ]に 水色[みずいろ]のリボンがかかっていたの。	
\\	私の部屋は長方形です。	長方形	私[わたし]の 部屋[へや]は 長方形[ちょうほうけい]です。	
\\	ここはなだらかな地形です。	地形	ここはなだらかな 地形[ちけい]です。	
\\	先月は株で100万円もうけたの。	もうける	先月[せんげつ]は 株[かぶ]で100 万円[まん えん]もうけたの。	
\\	プランターにトマトの種をまきました。	種	プランターにトマトの 種[たね]をまきました。	
\\	これらの本を分類してください。	分類	これらの 本[ほん]を 分類[ぶんるい]してください。	
\\	彼は旅行の直前に熱が出たの。	直前	彼[かれ]は 旅行[りょこう]の 直前[ちょくぜん]に 熱[ねつ]が 出[で]たの。	
\\	直ちに会社に戻ってください。	直ちに	直[ただ]ちに 会社[かいしゃ]に 戻[もど]ってください。	
\\	彼のことを見直しました。	見直す	彼[かれ]のことを 見直[みなお]しました。	
\\	彼女はゆったりした服を着ていますね。	ゆったり	彼女[かのじょ]はゆったりした 服[ふく]を 着[き]ていますね。	
\\	食べた直後に寝ないほうがいいよ。	直後	食[た]べた 直後[ちょくご]に 寝[ね]ないほうがいいよ。	
\\	彼女は失敗から立ち直ったよ。	立ち直る	彼女[かのじょ]は 失敗[しっぱい]から 立[た]ち 直[なお]ったよ。	
\\	仕事がやり直しになった。	やり直し	仕事[しごと]がやり 直[なお]しになった。	
\\	急いでやり直します。	やり直す	急[いそ]いでやり 直[なお]します。	
\\	これは私の直通の電話番号です。	直通	これは 私[わたし]の 直通[ちょくつう]の 電話番号[でんわ ばんごう]です。	
\\	彼女が結婚したなんて、ちっとも知らなかった。	ちっとも	彼女[かのじょ]が 結婚[けっこん]したなんて、ちっとも 知[し]らなかった。	
\\	新しい仕事の面接に行ってきました。	面接	新[あたら]しい 仕事[しごと]の 面接[めんせつ]に 行[い]ってきました。	
\\	運転手が無線で話していたの。	無線	運転手[うんてんしゅ]が 無線[むせん]で 話[はな]していたの。	
\\	子猫が線路に迷い込んだぞ。	線路	子猫[こねこ]が 線路[せんろ]に 迷[まよ]い 込[こ]んだぞ。	
\\	直線を描いてください。	直線	直線[ちょくせん]を 描[か]いてください。	
\\	風で電線が揺れていますね。	電線	風[かぜ]で 電線[でんせん]が 揺[ゆ]れていますね。	
\\	地震で電車が脱線したよ。	脱線	地震[じしん]で 電車[でんしゃ]が 脱線[だっせん]したよ。	
\\	昨日素敵なドレスを買いました。	ドレス	昨日素敵[きのう すてき]なドレスを 買[か]いました。	
\\	白線の内側に下がってお待ちください。	白線	白線[はくせん]の 内側[うちがわ]に 下[さ]がってお 待[ま]ちください。	
\\	私と彼は帰る方角が同じです。	方角	私[わたし]と 彼[かれ]は 帰[かえ]る 方角[ほうがく]が 同[おな]じです。	
\\	ここで直角に曲がってください。	直角	ここで 直角[ちょっかく]に 曲[ま]がってください。	
\\	町角にカフェがありました。	町角	町角[まちかど]にカフェがありました。	
\\	彼の部屋には真四角な窓があったんだ。	真四角	彼[かれ]の 部屋[へや]には 真四角[ましかく]な 窓[まど]があったんだ。	
\\	自転車のタイヤがパンクしました。	パンク	自転車[じてんしゃ]のタイヤがパンクしました。	
\\	最後まで共にがんばりましょう。	共に	最後[さいご]まで 共[とも]にがんばりましょう。	
\\	二人の走者は同時にゴールしたよ。	同時	二人[ふたり]の 走者[そうしゃ]は 同時[どうじ]にゴールしたよ。	
\\	この人とその人は、同一人物ですか。	同一	この 人[ひと]とその 人[ひと]は、 同一[どういつ] 人物[じんぶつ]ですか。	
\\	友人は私に同情してくれたよ。	同情	友人[ゆうじん]は 私[わたし]に 同情[どうじょう]してくれたよ。	
\\	同性の友達より異性の友達のほうが多いよ。	同性	同性[どうせい]の 友達[ともだち]より 異性[いせい]の 友達[ともだち]のほうが 多[おお]いよ。	
\\	休みの日は家でのんびり過ごします。	のんびり	休[やす]みの 日[ひ]は 家[うち]でのんびり 過[す]ごします。	
\\	彼は成功するに違いないわ。	違いない	彼[かれ]は 成功[せいこう]するに 違[ちが]いないわ。	
\\	すみません、人違いでした。	人違い	すみません、 人違[ひとちが]いでした。	
\\	掃除をしたら部屋が見違えるようになったの。	見違える	掃除[そうじ]をしたら 部屋[へや]が 見違[みちが]えるようになったの。	
\\	この二つには大きな違いがある。	違い	この 二[ふた]つには 大[おお]きな 違[ちが]いがある。	
\\	女の子は話し方まで母親に似るわね。	似る	女[おんな]の 子[こ]は 話[はな]し 方[かた]まで 母親[ははおや]に 似[に]るわね。	
\\	彼女は指輪をたくさんはめていますね。	はめる	彼女[かのじょ]は 指輪[ゆびわ]をたくさんはめていますね。	
\\	彼女は着物がよく似合います。	似合う	彼女[かのじょ]は 着物[きもの]がよく 似合[にあ]います。	
\\	彼の字に似せて書きました。	似せる	彼[かれ]の 字[じ]に 似[に]せて 書[か]きました。	
\\	姉はよく旅をします。	旅	姉[あね]はよく 旅[たび]をします。	
\\	その便は外国人の旅客が多かったよ。	旅客	その 便[びん]は 外国人[がいこくじん]の 旅客[りょかく]が 多[おお]かったよ。	
\\	父が旅費を出してくれました。	旅費	父[ちち]が 旅費[りょひ]を 出[だ]してくれました。	
\\	ユーモアは場の雰囲気を和らげるわね。	和らげる	ユーモアは 場[ば]の 雰囲気[ふんいき]を 和[やわ]らげるわね。	
\\	箱にふたをしたよ。	ふた	箱[はこ]にふたをしたよ。	
\\	日本で生まれた言葉を和語といいます。	和語	日本[にっぽん]で 生[う]まれた 言葉[ことば]を 和語[わご]といいます。	
\\	夕食に和風パスタを作ったよ。	和風	夕食[ゆうしょく]に 和風[わふう]パスタを 作[つく]ったよ。	
\\	私は和食が好きです。	和食	私[わたし]は 和食[わしょく]が 好[す]きです。	
\\	彼らは和やかに食事をした。	和やか	彼[かれ]らは 和[なご]やかに 食事[しょくじ]をした。	
\\	この曲を聞くと気持ちが和らぎます。	和らぐ	この 曲[きょく]を 聞[き]くと 気持[きも]ちが 和[やわ]らぎます。	
\\	教授にレポートを提出しました。	レポート	教授[きょうじゅ]にレポートを 提出[ていしゅつ]しました。	
\\	あの家のトイレは和式です。	和式	あの 家[うち]のトイレは 和式[わしき]です。	
\\	私は和英辞書をよく使います。	和英	私[わたし]は 和英[わえい] 辞書[じしょ]をよく 使[つか]います。	
\\	彼は東洋文化を研究しているよ。	東洋	彼[かれ]は 東洋[とうよう] 文化[ぶんか]を 研究[けんきゅう]しているよ。	
\\	私は洋風の家に住んでいます。	洋風	私[わたし]は 洋風[ようふう]の 家[いえ]に 住[す]んでいます。	
\\	私は週に3本洋画を見ます。	洋画	私[わたし]は 週[しゅう]に3 本[ぼん] 洋画[ようが]を 見[み]ます。	
\\	インターネットでマネー情報をチェックしたよ。	マネー	インターネットでマネー 情報[じょうほう]をチェックしたよ。	
\\	彼の家のトイレは洋式です。	洋式	彼[かれ]の 家[いえ]のトイレは 洋式[ようしき]です。	
\\	昨日の晩御飯は洋食でした。	洋食	昨日[きのう]の 晩御飯[ばんごはん]は 洋食[ようしょく]でした。	
\\	彼の家には洋間があります。	洋間	彼[かれ]の 家[いえ]には 洋間[ようま]があります。	
\\	彼女は和服がよく似合う。	和服	彼女[かのじょ]は 和服[わふく]がよく 似合[にあ]う。	
\\	待合室はとても込んでいたよ。	待合室	待合室[まちあいしつ]はとても 込[こ]んでいたよ。	
\\	この和室の天井は低いね。	和室	この 和室[わしつ]の 天井[てんじょう]は 低[ひく]いね。	
\\	この岩は意外ともろい。	もろい	この 岩[いわ]は 意外[いがい]ともろい。	
\\	このテーブルは洋室に合わない。	洋室	このテーブルは 洋室[ようしつ]に 合[あ]わない。	
\\	お振込みは3番の窓口です。	窓口	お 振込[ふりこ]みは3 番[ばん]の 窓口[まどぐち]です。	
\\	学校から父母に連絡があったんだ。	父母	学校[がっこう]から 父母[ふぼ]に 連絡[れんらく]があったんだ。	
\\	彼女は2才の子の母親です。	母親	彼女[かのじょ]は2 才[さい]の 子[こ]の 母親[ははおや]です。	
\\	彼の父親は先生です。	父親	彼[かれ]の 父親[ちちおや]は 先生[せんせい]です。	
\\	彼女はユーモアのセンスがあります。	ユーモア	彼女[かのじょ]はユーモアのセンスがあります。	
\\	父兄の方々はこちらへどうぞ。	父兄	父兄[ふけい]の 方々[かたがた]はこちらへどうぞ。	
\\	私には仕事が第一です。	第一	私[わたし]には 仕事[しごと]が 第一[だいいち]です。	
\\	預金には利息がつきます。	利息	預金[よきん]には 利息[りそく]がつきます。	
\\	彼女はとても若々しい人です。	若々しい	彼女[かのじょ]はとても 若々[わかわか]しい 人[ひと]です。	
\\	新聞の読者の投書欄は好きです。	読者	新聞[しんぶん]の 読者[どくしゃ]の 投書欄[とうしょらん]は 好[す]きです。	
\\	車のギアをバックに入れたよ。	バック	車[くるま]のギアをバックに 入[い]れたよ。	
\\	歩行者は道の右側を歩いてください。	歩行者	歩行者[ほこうしゃ]は 道[みち]の 右側[みぎがわ]を 歩[ある]いてください。	
\\	彼は有名な文学者です。	文学者	彼[かれ]は 有名[ゆうめい]な 文学者[ぶんがくしゃ]です。	
\\	日曜日はこの通りが歩行者天国になります。	歩行者天国	日曜日[にちようび]はこの 通[とお]りが 歩行者天国[ほこうしゃてんごく]になります。	
\\	私は先ほど電話した者です。	者	私[わたし]は 先[さき]ほど 電話[でんわ]した 者[もの]です。	
\\	彼女は髪にリボンを結んだね。	結ぶ	彼女[かのじょ]は 髪[かみ]にリボンを 結[むす]んだね。	
\\	彼は会の結びとしてスピーチをしたよ。	結び	彼[かれ]は 会[かい]の 結[むす]びとしてスピーチをしたよ。	
\\	いくつもの高層ビルがそびえていたんだ。	そびえる	いくつもの 高層[こうそう]ビルがそびえていたんだ。	
\\	彼はまだ未婚です。	未婚	彼[かれ]はまだ 未婚[みこん]です。	
\\	彼はしっかりと責任を果たしました。	果たす	彼[かれ]はしっかりと 責任[せきにん]を 果[は]たしました。	
\\	果たして彼は現れるだろうか。	果たして	果[は]たして 彼[かれ]は 現[あらわ]れるだろうか。	
\\	犬の散歩は私の日課です。	日課	犬[いぬ]の 散歩[さんぽ]は 私[わたし]の 日課[にっか]です。	
\\	私の免許は来年まで有効です。	有効	私[わたし]の 免許[めんきょ]は 来年[らいねん]まで 有効[ゆうこう]です。	
\\	彼はタレントなのでよくテレビに出ています。	タレント	彼[かれ]はタレントなのでよくテレビに 出[で]ています。	
\\	このクーポンはもう無効です。	無効	このクーポンはもう 無効[むこう]です。	
\\	社長自らがその会社と交渉したよ。	自ら	社長[しゃちょう] 自[みずか]らがその 会社[かいしゃ]と 交渉[こうしょう]したよ。	
\\	彼の態度はどこか不自然だったな。	不自然	彼[かれ]の 態度[たいど]はどこか 不自然[ふしぜん]だったな。	
\\	彼は不自由な暮らしに慣れたようだね。	不自由	彼[かれ]は 不自由[ふじゆう]な 暮[く]らしに 慣[な]れたようだね。	
\\	通信教育はとても便利です。	通信	通信[つうしん] 教育[きょういく]はとても 便利[べんり]です。	
\\	彼はたまに料理をします。	たまに	彼[かれ]はたまに 料理[りょうり]をします。	
\\	人に頼らないで、自分でやってごらん。	頼る	人[ひと]に 頼[たよ]らないで、 自分[じぶん]でやってごらん。	
\\	あなたに頼みがあるんですけど。	頼み	あなたに 頼[たの]みがあるんですけど。	
\\	彼は頼もしい人です。	頼もしい	彼[かれ]は 頼[たの]もしい 人[ひと]です。	
\\	その土地は民間企業に売却されたんだ。	民間	その 土地[とち]は 民間[みんかん] 企業[きぎょう]に 売却[ばいきゃく]されたんだ。	
\\	私は民族の歴史に興味があります。	民族	私[わたし]は 民族[みんぞく]の 歴史[れきし]に 興味[きょうみ]があります。	
\\	床にペンキをこぼしてしまいました。	ペンキ	床[ゆか]にペンキをこぼしてしまいました。	
\\	この自転車の持ち主は誰ですか。	持ち主	この 自転車[じてんしゃ]の 持[も]ち 主[ぬし]は 誰[だれ]ですか。	
\\	民主主義について勉強しました。	民主	民主[みんしゅ] 主義[しゅぎ]について 勉強[べんきょう]しました。	
\\	家主は1階に住んでいます。	家主	家主[やぬし]は1 階[かい]に 住[す]んでいます。	
\\	美しさを定義してください。	定義	美[うつく]しさを 定義[ていぎ]してください。	
\\	それはとても不思議な話だね。	不思議	それはとても 不思議[ふしぎ]な 話[はなし]だね。	
\\	彼は大学で音楽の理論を勉強したんだよ。	理論	彼[かれ]は 大学[だいがく]で 音楽[おんがく]の 理論[りろん]を 勉強[べんきょう]したんだよ。	
\\	公園のベンチで本を読んでいます。	ベンチ	公園[こうえん]のベンチで 本[ほん]を 読[よ]んでいます。	
\\	数学は論理に基づく学問です。	論理	数学[すうがく]は 論理[ろんり]に 基[もと]づく 学問[がくもん]です。	
\\	論文を仕上げないと卒業できませんよ。	論文	論文[ろんぶん]を 仕上[しあ]げないと 卒業[そつぎょう]できませんよ。	
\\	仲間と地球の未来について論じたの。	論じる	仲間[なかま]と 地球[ちきゅう]の 未来[みらい]について 論[ろん]じたの。	
\\	ぼくは無論きみに賛成だ。	無論	ぼくは 無論[むろん]きみに 賛成[さんせい]だ。	
\\	世論は新しい法律に反対です。	世論	世論[よろん]は 新[あたら]しい 法律[ほうりつ]に 反対[はんたい]です。	
\\	交通費はめいめい支払ってください。	めいめい	交通費[こうつうひ]はめいめい 支払[しはら]ってください。	
\\	彼の人生論は独特だな。	論	彼[かれ]の 人生[じんせい] 論[ろん]は 独特[どくとく]だな。	
\\	彼は先生に反発していたよ。	反発	彼[かれ]は 先生[せんせい]に 反発[はんぱつ]していたよ。	
\\	新型車に研究の成果が反映されています。	反映	新型車[しんがたしゃ]に 研究[けんきゅう]の 成果[せいか]が 反映[はんえい]されています。	
\\	反論がある人はどうぞ。	反論	反論[はんろん]がある 人[ひと]はどうぞ。	
\\	彼の行いはモラルに反しています。	反する	彼[かれ]の 行[おこな]いはモラルに 反[はん]しています。	
\\	どうにか締め切りに間に合った。	どうにか	どうにか 締[し]め 切[き]りに 間[ま]に 合[あ]った。	
\\	その選手は反則で負けました。	反則	その 選手[せんしゅ]は 反則[はんそく]で 負[ま]けました。	
\\	親子の対話は大切だよ。	対話	親子[おやこ]の 対話[たいわ]は 大切[たいせつ]だよ。	
\\	このズボンは上着と対になっています。	対	このズボンは 上着[うわぎ]と 対[つい]になっています。	
\\	彼はいつも素早い対応をするね。	対応	彼[かれ]はいつも 素早[すばや]い 対応[たいおう]をするね。	
\\	彼女は良い反応を示したわ。	反応	彼女[かのじょ]は 良[よ]い 反応[はんのう]を 示[しめ]したわ。	
\\	あなたと問答している暇はないの。	問答	あなたと 問答[もんどう]している 暇[ひま]はないの。	
\\	朝食はちゃんと食べましたか。	ちゃんと	朝食[ちょうしょく]はちゃんと 食[た]べましたか。	
\\	警察は犯人を特定したらしいよ。	特定	警察[けいさつ]は 犯人[はんにん]を 特定[とくてい]したらしいよ。	
\\	その学校の教育は特色がありますね。	特色	その 学校[がっこう]の 教育[きょういく]は 特色[とくしょく]がありますね。	
\\	これは子供に特有の病気です。	特有	これは 子供[こども]に 特有[とくゆう]の 病気[びょうき]です。	
\\	私は別に気になりません。	別に	私[わたし]は 別[べつ]に 気[き]になりません。	
\\	別れはいつでも悲しいものです。	別れ	別[わか]れはいつでも 悲[かな]しいものです。	
\\	私の田舎はのどかな所です。	のどか	私[わたし]の 田舎[いなか]はのどかな 所[ところ]です。	
\\	送別の辞は誰に頼みましょうか。	送別	送別[そうべつ]の 辞[じ]は 誰[だれ]に 頼[たの]みましょうか。	
\\	来週、課長の送別会を開きます。	送別会	来週[らいしゅう]、 課長[かちょう]の 送別会[そうべつかい]を 開[ひら]きます。	
\\	教授はフランス文学の専門家。	専門家	教授[きょうじゅ]はフランス 文学[ぶんがく]の 専門家[せんもんか]。	
\\	法律は私の専門です。	専門	法律[ほうりつ]は 私[わたし]の 専門[せんもん]です。	
\\	これは女性専用の車両です。	専用	これは 女性[じょせい] 専用[せんよう]の 車両[しゃりょう]です。	
\\	このメロディーは聞いたことがある。	メロディー	このメロディーは 聞[き]いたことがある。	
\\	彼は植物全般に詳しいね。	全般	彼[かれ]は 植物[しょくぶつ] 全般[ぜんぱん]に 詳[くわ]しいね。	
\\	彼は絵を本格的に勉強しているんだ。	本格的	彼[かれ]は 絵[え]を 本格的[ほんかくてき]に 勉強[べんきょう]しているんだ。	
\\	彼が全面的に協力してくれるそうです。	全面的	彼[かれ]が 全面的[ぜんめんてき]に 協力[きょうりょく]してくれるそうです。	
\\	彼は論理的な人です。	論理的	彼[かれ]は 論理的[ろんりてき]な 人[ひと]です。	
\\	彼はすごく知的な人です。	知的	彼[かれ]はすごく 知的[ちてき]な 人[ひと]です。	
\\	やっと目的地に着いたよ。	目的地	やっと 目的地[もくてきち]に 着[つ]いたよ。	
\\	今日は水を2リットル以上飲んだわ。	リットル	今日[きょう]は 水[みず]を2リットル 以上飲[いじょう の]んだわ。	
\\	この国は文化的な事業に力を入れています。	文化的	この 国[くに]は 文化的[ぶんかてき]な 事業[じぎょう]に 力[ちから]を 入[い]れています。	
\\	彼はとても男性的な人です。	男性的	彼[かれ]はとても 男性的[だんせいてき]な 人[ひと]です。	
\\	私は普段は
\\	シャツとジーンズを着ています。	普段	私[わたし]は 普段[ふだん]は 
\\	[てぃー]シャツとジーンズを 着[き]ています。	
\\	彼は並みの人間ではありません。	並み	彼[かれ]は 並[な]みの 人間[にんげん]ではありません。	
\\	駅前の並木が台風で倒れたらしい。	並木	駅前[えきまえ]の 並木[なみき]が 台風[たいふう]で 倒[たお]れたらしい。	
\\	彼はわざと負けたように見えたな。	わざと	彼[かれ]はわざと 負[ま]けたように 見[み]えたな。	
\\	このメガネをかけると平面が立体に見えます。	平面	このメガネをかけると 平面[へいめん]が 立体[りったい]に 見[み]えます。	
\\	彼女は平気な顔をしていた。	平気	彼女[かのじょ]は 平気[へいき]な 顔[かお]をしていた。	
\\	その建物の屋根は平らだね。	平ら	その 建物[たてもの]の 屋根[やね]は 平[たい]らだね。	
\\	平行に線を引いてください。	平行	平行[へいこう]に 線[せん]を 引[ひ]いてください。	
\\	私はいつも同僚の不平を聞いているんだ。	不平	私[わたし]はいつも 同僚[どうりょう]の 不平[ふへい]を 聞[き]いているんだ。	
\\	彼ならとっくに帰りましたよ。	とっくに	彼[かれ]ならとっくに 帰[かえ]りましたよ。	
\\	地平線に夕日が沈むところだったの。	地平線	地平線[ちへいせん]に 夕日[ゆうひ]が 沈[しず]むところだったの。	
\\	広い平野が一面雪で真っ白でした。	平野	広[ひろ]い 平野[へいや]が 一面雪[いちめん ゆき]で 真[ま]っ 白[しろ]でした。	
\\	平たいお皿を一枚取って。	平たい	平[ひら]たいお 皿[さら]を 一枚取[いちまい と]って。	
\\	この土地の面積は約100平方メートルです。	平方	この 土地[とち]の 面積[めんせき]は 約100[やく 
\\	平方[へいほう]メートルです。	
\\	彼は平日がお休みです。	平日	彼[かれ]は 平日[へいじつ]がお 休[やす]みです。	
\\	私はベストを尽くしました。	ベスト	私[わたし]はベストを 尽[つ]くしました。	
\\	平均で一日に8時間ぐらい働いています。	平均	平均[へいきん]で 一日[いちにち]に8 時間[じかん]ぐらい 働[はたら]いています。	
\\	私は株の知識がないに等しいです。	等しい	私[わたし]は 株[かぶ]の 知識[ちしき]がないに 等[ひと]しいです。	
\\	あの先生は生徒をみな平等に扱います。	平等	あの 先生[せんせい]は 生徒[せいと]をみな 平等[びょうどう]に 扱[あつか]います。	
\\	彼には大学生と同等の学力があります。	同等	彼[かれ]には 大学生[だいがくせい]と 同等[どうとう]の 学力[がくりょく]があります。	
\\	その子供は大人と対等に話していたよ。	対等	その 子供[こども]は 大人[おとな]と 対等[たいとう]に 話[はな]していたよ。	
\\	職場での男女不平等はよく見られるね。	不平等	職場[しょくば]での 男女[だんじょ] 不平等[ふびょうどう]はよく 見[み]られるね。	
\\	彼女は白いワンピースを着ていたよ。	ワンピース	彼女[かのじょ]は 白[しろ]いワンピースを 着[き]ていたよ。	
\\	その病室はとても広くてきれいだった。	病室	その 病室[びょうしつ]はとても 広[ひろ]くてきれいだった。	
\\	病人が出たので電車が少し止まったんだ。	病人	病人[びょうにん]が 出[で]たので 電車[でんしゃ]が 少[すこ]し 止[と]まったんだ。	
\\	内科で胃の調子を診てもらったよ。	内科	内科[ないか]で 胃[い]の 調子[ちょうし]を 診[み]てもらったよ。	
\\	学生たちは授業についていくのに必死です。	必死	学生[がくせい]たちは 授業[じゅぎょう]についていくのに 必死[ひっし]です。	
\\	その作家は35歳という若さで病死した。	病死	その 作家[さっか]は35 歳[さい]という 若[わか]さで 病死[びょうし]した。	
\\	ダイヤモンドはとても高価ですね。	ダイヤモンド	ダイヤモンドはとても 高価[こうか]ですね。	
\\	必死に単語を暗記したよ。	必死に	必死[ひっし]に 単語[たんご]を 暗記[あんき]したよ。	
\\	昨年、友人を亡くしました。	亡くす	昨年[さくねん]、 友人[ゆうじん]を 亡[な]くしました。	
\\	彼女は多忙な人です。	多忙	彼女[かのじょ]は 多忙[たぼう]な 人[ひと]です。	
\\	最近疲れがたまっています。	疲れ	最近[さいきん] 疲[つか]れがたまっています。	
\\	ここは立ち入り禁止です。	立ち入り禁止	ここは 立[た]ち 入[い]り 禁止[きんし]です。	
\\	その車のナンバーを覚えていますか。	ナンバー	その 車[くるま]のナンバーを 覚[おぼ]えていますか。	
\\	珍しい日本酒が手に入りました。	日本酒	珍[めずら]しい 日本酒[にほんしゅ]が 手[て]に 入[はい]りました。	
\\	みんなかなり酔っていたの。	酔う	みんなかなり 酔[よ]っていたの。	
\\	彼女は若さを保とうと必死だ。	保つ	彼女[かのじょ]は 若[わか]さを 保[たも]とうと 必死[ひっし]だ。	
\\	あなたは保険に加入していますか。	保険	あなたは 保険[ほけん]に 加入[かにゅう]していますか。	
\\	この製品の保証期間は5年間となっております。	保証	この 製品[せいひん]の 保証[ほしょう] 期間[きかん]は5 年間[ねん かん]となっております。	
\\	彼が私の保証人になってくれました。	保証人	彼[かれ]が 私[わたし]の 保証人[ほしょうにん]になってくれました。	
\\	とうとう引っ越しの日がきました。	とうとう	とうとう 引[ひ]っ 越[こ]しの 日[ひ]がきました。	
\\	データを30分おきに保存してください。	保存	データを30 分[ぷん]おきに 保存[ほぞん]してください。	
\\	郵便局はどこかご存じですか。	存じる	郵便局[ゆうびんきょく]はどこかご 存[ぞん]じですか。	
\\	みんなのグラスにジュースを注いだよ。	注ぐ	みんなのグラスにジュースを 注[そそ]いだよ。	
\\	詳しくは注を読んでください。	注	詳[くわ]しくは 注[ちゅう]を 読[よ]んでください。	
\\	彼の意見には同意できません。	同意	彼[かれ]の 意見[いけん]には 同意[どうい]できません。	
\\	仕事中に帰るなんてとんでもない。	とんでもない	仕事中[しごとちゅう]に 帰[かえ]るなんてとんでもない。	
\\	そんなことをしても無意味よ。	無意味	そんなことをしても 無意味[むいみ]よ。	
\\	その事故は運転手の不注意が原因で起きたの。	不注意	その 事故[じこ]は 運転手[うんてんしゅ]の 不注意[ふちゅうい]が 原因[げんいん]で 起[お]きたの。	
\\	あの子は生意気だと思います。	生意気	あの 子[こ]は 生意気[なまいき]だと 思[おも]います。	
\\	母はお釣りを確かめたの。	確かめる	母[はは]はお 釣[つ]りを 確[たし]かめたの。	
\\	彼女には明確な目標があるね。	明確	彼女[かのじょ]には 明確[めいかく]な 目標[もくひょう]があるね。	
\\	彼らはにこやかに挨拶を交わしたけどね。	にこやか	彼[かれ]らはにこやかに 挨拶[あいさつ]を 交[か]わしたけどね。	
\\	人の記憶は不確かよ。	不確か	人[ひと]の 記憶[きおく]は 不確[ふたし]かよ。	
\\	この薬はまだ認可されていません。	認可	この 薬[くすり]はまだ 認可[にんか]されていません。	
\\	父が彼女との結婚を認めてくれました。	認める	父[ちち]が 彼女[かのじょ]との 結婚[けっこん]を 認[みと]めてくれました。	
\\	その手術の成功率は90
\\	だそうです。	率	その 手術[しゅじゅつ]の 成功[せいこう] 率[りつ]は90
\\	だそうです。	
\\	率直なご意見ありがとうございました。	率直	率直[そっちょく]なご 意見[いけん]ありがとうございました。	
\\	この方法だと能率がいいですね。	能率	この 方法[ほうほう]だと 能率[のうりつ]がいいですね。	
\\	荷物を駅のロッカーに入れた。	ロッカー	荷物[にもつ]を 駅[えき]のロッカーに 入[い]れた。	
\\	旅客機が墜落したよ。	旅客機	旅客機[りょかくき]が 墜落[ついらく]したよ。	
\\	妹は不器用で、料理も苦手なの。	不器用	妹[いもうと]は 不器用[ぶきよう]で、 料理[りょうり]も 苦手[にがて]なの。	
\\	道具は全部揃っていますか。	道具	道具[どうぐ]は 全部揃[ぜんぶ そろ]っていますか。	
\\	旅行には予備の靴を持って行きます。	予備	旅行[りょこう]には 予備[よび]の 靴[くつ]を 持[も]って 行[い]きます。	
\\	彼は非暴力を説きました。	説く	彼[かれ]は 非暴力[ひぼうりょく]を 説[と]きました。	
\\	彼はプライドが高い人です。	プライド	彼[かれ]はプライドが 高[たか]い 人[ひと]です。	
\\	彼のやり方は不公平です。	不公平	彼[かれ]のやり 方[かた]は 不公平[ふこうへい]です。	
\\	この遊園地のチケットは3000円です。	遊園地	この 遊園地[ゆうえんち]のチケットは 3000円[さんぜんえん]です。	
\\	娘を保育園に迎えに行きます。	保育園	娘[むすめ]を 保育園[ほいくえん]に 迎[むか]えに 行[い]きます。	
\\	彼は道路に飛び出したの。	飛び出す	彼[かれ]は 道路[どうろ]に 飛[と]び 出[だ]したの。	
\\	子供が紙飛行機を飛ばしています。	飛ばす	子供[こども]が 紙飛行機[かみひこうき]を 飛[と]ばしています。	
\\	ママに聞いてみよう。	ママ	ママに 聞[き]いてみよう。	
\\	大きな音にびっくりして飛び上がりました。	飛び上がる	大[おお]きな 音[おと]にびっくりして 飛[と]び 上[あ]がりました。	
\\	猫が屋根から飛び下りました。	飛び下りる	猫[ねこ]が 屋根[やね]から 飛[と]び 下[お]りました。	
\\	カエルが池に飛び込んだね。	飛び込む	カエルが 池[いけ]に 飛[と]び 込[こ]んだね。	
\\	このフライトの飛行時間は約3時間です。	飛行	このフライトの 飛行[ひこう] 時間[じかん]は 約3時間[やく 
\\	じかん]です。	
\\	私がこの船の船長です。	船長	私[わたし]がこの 船[ふね]の 船長[せんちょう]です。	
\\	椅子の荷物をよけて座ったけど。	よける	椅子[いす]の 荷物[にもつ]をよけて 座[すわ]ったけど。	
\\	子供が風船を膨らませているね。	風船	子供[こども]が 風船[ふうせん]を 膨[ふく]らませているね。	
\\	台風がその半島を通過したの。	半島	台風[たいふう]がその 半島[はんとう]を 通過[つうか]したの。	
\\	私たちはハワイのマウイ島に旅行したの。	島	私[わたし]たちはハワイのマウイ 島[とう]に 旅行[りょこう]したの。	
\\	このデータはまだ不完全ね。	不完全	このデータはまだ 不完全[ふかんぜん]ね。	
\\	彼は今月の売上目標を達成した。	達成	彼[かれ]は 今月[こんげつ]の 売上目標[うりあげ もくひょう]を 達成[たっせい]した。	
\\	この島は観光で成り立っています。	成り立つ	この 島[しま]は 観光[かんこう]で 成[な]り 立[た]っています。	
\\	彼は弁護士バッジを付けていたよ。	バッジ	彼[かれ]は 弁護士[べんごし]バッジを 付[つ]けていたよ。	
\\	未成年はお酒を飲めません。	未成年	未成年[みせいねん]はお 酒[さけ]を 飲[の]めません。	
\\	私のチームは1回戦で敗れたよ。	敗れる	私[わたし]のチームは1 回戦[かいせん]で 敗[やぶ]れたよ。	
\\	私たちは野原で花をつんだの。	野原	私[わたし]たちは 野原[のはら]で 花[はな]をつんだの。	
\\	クローバーの原でピクニックをしたよ。	原	クローバーの 原[はら]でピクニックをしたよ。	
\\	私たちはがんの要因を研究しています。	要因	私[わたし]たちはがんの 要因[よういん]を 研究[けんきゅう]しています。	
\\	彼女はいつも髪をピンで留めているね。	ピン	彼女[かのじょ]はいつも 髪[かみ]をピンで 留[と]めているね。	
\\	彼の病気は過労に因るものです。	因る	彼[かれ]の 病気[びょうき]は 過労[かろう]に 因[よ]るものです。	
\\	被災地に物資を送ったよ。	物資	被災地[ひさいち]に 物資[ぶっし]を 送[おく]ったよ。	
\\	世界の平和が私たちの願いです。	願い	世界[せかい]の 平和[へいわ]が 私[わたし]たちの 願[ねが]いです。	
\\	試験で不正が見つかったよ。	不正	試験[しけん]で 不正[ふせい]が 見[み]つかったよ。	
\\	祖母は大正生まれです。	大正	祖母[そぼ]は 大正[たいしょう] 生[う]まれです。	
\\	まあこれでいいだろう。	まあ	まあこれでいいだろう。	
\\	彼は正に英雄ね。	正に	彼[かれ]は 正[まさ]に 英雄[えいゆう]ね。	
\\	彼は常に姿勢がいい。	常に	彼[かれ]は 常[つね]に 姿勢[しせい]がいい。	
\\	音楽は私の日常の一部です。	日常	音楽[おんがく]は 私[わたし]の 日常[にちじょう]の 一部[いちぶ]です。	
\\	私は旅行から多くの知識を得た。	知識	私[わたし]は 旅行[りょこう]から 多[おお]くの 知識[ちしき]を 得[え]た。	
\\	その件は終わったと認識しています。	認識	その 件[けん]は 終[お]わったと 認識[にんしき]しています。	
\\	私は無意識に彼を傷つけてしまった。	無意識	私[わたし]は 無意識[むいしき]に 彼[かれ]を 傷[きず]つけてしまった。	
\\	愛がなければ人生はむなしいよ。	むなしい	愛[あい]がなければ 人生[じんせい]はむなしいよ。	
\\	国民は総理大臣を非難しているね。	非難	国民[こくみん]は 総理大臣[そうり だいじん]を 非難[ひなん]しているね。	
\\	非常事態です。	非常	非常[ひじょう] 事態[じたい]です。	
\\	彼は非常識な時間に電話してきたの。	非常識	彼[かれ]は 非常識[ひじょうしき]な 時間[じかん]に 電話[でんわ]してきたの。	
\\	ビルの非常口を確認したの。	非常口	ビルの 非常口[ひじょうぐち]を 確認[かくにん]したの。	
\\	体の調子がとても良いです。	調子	体[からだ]の 調子[ちょうし]がとても 良[い]いです。	
\\	ついでだから彼も呼ぼうよ。	ついで	ついでだから 彼[かれ]も 呼[よ]ぼうよ。	
\\	彼女は自然と調和した暮らし方をしているの。	調和	彼女[かのじょ]は 自然[しぜん]と 調和[ちょうわ]した 暮[く]らし 方[かた]をしているの。	
\\	今日は体調が悪いです。	体調	今日[きょう]は 体調[たいちょう]が 悪[わる]いです。	
\\	この曲は単調でつまらないな。	単調	この 曲[きょく]は 単調[たんちょう]でつまらないな。	
\\	塩、コショウはよく使われる調味料です。	調味料	塩[しお]、コショウはよく 使[つか]われる 調味料[ちょうみりょう]です。	
\\	今、スケジュールの調整をしています。	調整	今[いま]、スケジュールの 調整[ちょうせい]をしています。	
\\	バケツに水を汲んできてください。	バケツ	バケツに 水[みず]を 汲[く]んできてください。	
\\	彼はスピーチの前に服装を整えた。	整える	彼[かれ]はスピーチの 前[まえ]に 服装[ふくそう]を 整[ととの]えた。	
\\	パーティーの準備が整いました。	整う	パーティーの 準備[じゅんび]が 整[ととの]いました。	
\\	ここで部屋の温度が調節できます。	調節	ここで 部屋[へや]の 温度[おんど]が 調節[ちょうせつ]できます。	
\\	課題は7月5日までに提出してください。	提出	課題[かだい]は7 月5日[がつ 
\\	か]までに 提出[ていしゅつ]してください。	
\\	交渉が成立するという前提で話を進めます。	前提	交渉[こうしょう]が 成立[せいりつ]するという 前提[ぜんてい]で 話[はなし]を 進[すす]めます。	
\\	今から答案を集めます。	答案	今[いま]から 答案[とうあん]を 集[あつ]めます。	
\\	彼女の美しさにはっとしたよ。	はっと	彼女[かのじょ]の 美[うつく]しさにはっとしたよ。	
\\	私は朝早く投票を済ませました。	投票	私[わたし]は 朝早[あさ はや]く 投票[とうひょう]を 済[す]ませました。	
\\	私は父を目標にしています。	目標	私[わたし]は 父[ちち]を 目標[もくひょう]にしています。	
\\	ニュースでは標準語が使われるの。	標準	ニュースでは 標準[ひょうじゅん] 語[ご]が 使[つか]われるの。	
\\	親が君を一度連れて来いって言うんだ。	連れる	親[おや]が 君[きみ]を 一度[いちど] 連[つ]れて 来[こ]いって 言[い]うんだ。	
\\	展覧会は連日賑わいました。	連日	展覧会[てんらんかい]は 連日[れんじつ] 賑[にぎ]わいました。	
\\	彼女の運転がまずくて、はらはらした。	まずい	彼女[かのじょ]の 運転[うんてん]がまずくて、はらはらした。	
\\	今度の連休は実家に帰ります。	連休	今度[こんど]の 連休[れんきゅう]は 実家[じっか]に 帰[かえ]ります。	
\\	彼女は私の連れです。	連れ	彼女[かのじょ]は 私[わたし]の 連[つ]れです。	
\\	仕事が終わったら連絡します。	連絡	仕事[しごと]が 終[お]わったら 連絡[れんらく]します。	
\\	入国手続きが終わりました。	手続き	入国[にゅうこく] 手続[てつづ]きが 終[お]わりました。	
\\	彼女の人生は苦労の連続でした。	連続	彼女[かのじょ]の 人生[じんせい]は 苦労[くろう]の 連続[れんぞく]でした。	
\\	もしも彼女と結婚できたらどんなに嬉しいだろう。	もしも	もしも 彼女[かのじょ]と 結婚[けっこん]できたらどんなに 嬉[うれ]しいだろう。	
\\	お客さんが続々とやって来ましたよ。	続々	お 客[きゃく]さんが 続々[ぞくぞく]とやって 来[き]ましたよ。	
\\	話の続きは電話でしましょう。	続き	話[はなし]の 続[つづ]きは 電話[でんわ]でしましょう。	
\\	彼は仕事が長続きしません。	長続き	彼[かれ]は 仕事[しごと]が 長続[ながつづ]きしません。	
\\	この辺りの家賃の相場はいくらですか。	相場	この 辺[あた]りの 家賃[やちん]の 相場[そうば]はいくらですか。	
\\	1ポンドは454グラムに相当します。	相当	1[いち]ポンドは454グラムに 相当[そうとう]します。	
\\	スピーカーの音がやかましいな。	やかましい	スピーカーの 音[おと]がやかましいな。	
\\	その場に相応しい服装で来てください。	相応しい	その 場[ば]に 相応[ふさわ]しい 服装[ふくそう]で 来[き]てください。	
\\	雑誌にその女優の対談が載っていたよ。	対談	雑誌[ざっし]にその 女優[じょゆう]の 対談[たいだん]が 載[の]っていたよ。	
\\	好きな番組を録画したの。	録画	好[す]きな 番組[ばんぐみ]を 録画[ろくが]したの。	
\\	インタビューはこの録音テープに入っています。	録音テープ	インタビューはこの 録音[ろくおん]テープに 入[はい]っています。	
\\	彼女の登場で会場は盛り上がったね。	登場	彼女[かのじょ]の 登場[とうじょう]で 会場[かいじょう]は 盛[も]り 上[あ]がったね。	
\\	心理学の授業に登録しましたか。	登録	心理学[しんりがく]の 授業[じゅぎょう]に 登録[とうろく]しましたか。	
\\	このバットは金属で出来ています。	バット	このバットは 金属[きんぞく]で 出来[でき]ています。	
\\	夏休みには家族で登山をします。	登山	夏休[なつやす]みには 家族[かぞく]で 登山[とざん]をします。	
\\	生徒たちは朝8時ごろ登校します。	登校	生徒[せいと]たちは 朝8時[あさ 
\\	じ]ごろ 登校[とうこう]します。	
\\	明日は友達と山登りに行きます。	山登り	明日[あした]は 友達[ともだち]と 山登[やまのぼ]りに 行[い]きます。	
\\	彼女は政治に無関心です。	無関心	彼女[かのじょ]は 政治[せいじ]に 無関心[むかんしん]です。	
\\	彼はこの事件と無関係です。	無関係	彼[かれ]はこの 事件[じけん]と 無関係[むかんけい]です。	
\\	食事のマナーを守りましょう。	マナー	食事[しょくじ]のマナーを 守[まも]りましょう。	
\\	不況の影響で仕事が少ないね。	不況	不況[ふきょう]の 影響[えいきょう]で 仕事[しごと]が 少[すく]ないね。	
\\	あの男の態度にみんなあきれてたよ。	態度	あの 男[おとこ]の 態度[たいど]にみんなあきれてたよ。	
\\	祖父は明治の生まれです。	明治	祖父[そふ]は 明治[めいじ]の 生[う]まれです。	
\\	彼は大阪府に住んでいます。	府	彼[かれ]は 大阪[おおさか] 府[ふ]に 住[す]んでいます。	
\\	彼女は府立大学に通っています。	府立	彼女[かのじょ]は 府立[ふりつ] 大学[だいがく]に 通[かよ]っています。	
\\	生徒たちはばらばらに帰宅したね。	ばらばら	生徒[せいと]たちはばらばらに 帰宅[きたく]したね。	
\\	野党が与党を厳しく非難していましたね。	野党	野党[やとう]が 与党[よとう]を 厳[きび]しく 非難[ひなん]していましたね。	
\\	彼は選挙に当選しました。	当選	彼[かれ]は 選挙[せんきょ]に 当選[とうせん]しました。	
\\	彼の絵がコンクールに入選した。	入選	彼[かれ]の 絵[え]がコンクールに 入選[にゅうせん]した。	
\\	この5種類から選択できます。	選択	この5 種類[しゅるい]から 選択[せんたく]できます。	
\\	ここは天候の変化が激しいですね。	天候	ここは 天候[てんこう]の 変化[へんか]が 激[はげ]しいですね。	
\\	手首の関節をひねっちゃった。	手首	手首[てくび]の 関節[かんせつ]をひねっちゃった。	
\\	プラットホームで電車が来るのを待ったの。	プラットホーム	プラットホームで 電車[でんしゃ]が 来[く]るのを 待[ま]ったの。	
\\	そのクイズ番組は脳を刺激するね。	脳	そのクイズ 番組[ばんぐみ]は 脳[のう]を 刺激[しげき]するね。	
\\	彼は受験のことで悩んでいます。	悩む	彼[かれ]は 受験[じゅけん]のことで 悩[なや]んでいます。	
\\	彼は大きな悩みを抱えていました。	悩み	彼[かれ]は 大[おお]きな 悩[なや]みを 抱[かか]えていました。	
\\	先頭の人はプラカードを持ってください。	先頭	先頭[せんとう]の 人[ひと]はプラカードを 持[も]ってください。	
\\	急に出張を命じられました。	命じる	急[きゅう]に 出張[しゅっちょう]を 命[めい]じられました。	
\\	まぐろの刺身を食べました。	まぐろ	まぐろの 刺身[さしみ]を 食[た]べました。	
\\	ここから先は隣の国の領土です。	領土	ここから 先[さき]は 隣[となり]の 国[くに]の 領土[りょうど]です。	
\\	彼はアメリカ領事館に出かけたよ。	領事館	彼[かれ]はアメリカ 領事館[りょうじかん]に 出[で]かけたよ。	
\\	領収書をください。	領収書	領収書[りょうしゅうしょ]をください。	
\\	彼女はスペイン語に取り組んでいます。	取り組む	彼女[かのじょ]はスペイン 語[ご]に 取[と]り 組[く]んでいます。	
\\	この仕事は君に任せる。	任せる	この 仕事[しごと]は 君[きみ]に 任[まか]せる。	
\\	ゆうべはろくに寝ていない。	ろくに	ゆうべはろくに 寝[ね]ていない。	
\\	私は3年生のクラスを担任しています。	担任	私[わたし]は3 年生[ねんせい]のクラスを 担任[たんにん]しています。	
\\	無責任な行動は許されません。	無責任	無責任[むせきにん]な 行動[こうどう]は 許[ゆる]されません。	
\\	彼は大阪に転勤しました。	転勤	彼[かれ]は 大阪[おおさか]に 転勤[てんきん]しました。	
\\	私の勤め先にご連絡ください。	勤め先	私[わたし]の 勤[つと]め 先[さき]にご 連絡[れんらく]ください。	
\\	来月で勤めをやめます。	勤め	来月[らいげつ]で 勤[つと]めをやめます。	
\\	私が司会を務めます。	務める	私[わたし]が 司会[しかい]を 務[つと]めます。	
\\	お菓子でもつまんでください。	つまむ	お 菓子[かし]でもつまんでください。	
\\	彼は最後まで自分の任務を果たした。	任務	彼[かれ]は 最後[さいご]まで 自分[じぶん]の 任務[にんむ]を 果[は]たした。	
\\	人々を守るのが私の務めです。	務め	人々[ひとびと]を 守[まも]るのが 私[わたし]の 務[つと]めです。	
\\	新しい社員を雇いました。	雇う	新[あたら]しい 社員[しゃいん]を 雇[やと]いました。	
\\	私の会社で社員を募集しています。	募集	私[わたし]の 会社[かいしゃ]で 社員[しゃいん]を 募集[ぼしゅう]しています。	
\\	この山ではきのこが採れますよ	採る	この 山[やま]ではきのこが 採[と]れますよ	
\\	ボーイに荷物を運んでもらった。	ボーイ	ボーイに 荷物[にもつ]を 運[はこ]んでもらった。	
\\	今年から新しい仕事に就きます。	就く	今年[ことし]から 新[あたら]しい 仕事[しごと]に 就[つ]きます。	
\\	私は今月一杯で退職します。	退職	私[わたし]は 今月一杯[こんげつ いっぱい]で 退職[たいしょく]します。	
\\	弟は高校を退学しました。	退学	弟[おとうと]は 高校[こうこう]を 退学[たいがく]しました。	
\\	具合が悪かったので仕事を早退しました。	早退	具合[ぐあい]が 悪[わる]かったので 仕事[しごと]を 早退[そうたい]しました。	
\\	彼には理屈が通じない。	理屈	彼[かれ]には 理屈[りくつ]が 通[つう]じない。	
\\	コーヒーをホットでください。	ホット	コーヒーをホットでください。	
\\	校長の退屈な話が続いたんだよ。	退屈	校長[こうちょう]の 退屈[たいくつ]な 話[はなし]が 続[つづ]いたんだよ。	
\\	用件をメモしておきました。	用件	用件[ようけん]をメモしておきました。	
\\	さあ、参りましょうか。	参る	さあ、 参[まい]りましょうか。	
\\	彼は惨めな気持ちになったの。	惨め	彼[かれ]は 惨[みじ]めな 気持[きも]ちになったの。	
\\	オーダーを追加しました。	追加	オーダーを 追加[ついか]しました。	
\\	これでもないよりましだ。	まし	これでもないよりましだ。	
\\	業界は女性の比率が低い。	比率	
\\	業界[あいてぃーぎょうかい]は 女性[じょせい]の 比率[ひりつ]が 低[ひく]い。	
\\	この絵は赤と黒の対比が美しいですね。	対比	この 絵[え]は 赤[あか]と 黒[くろ]の 対比[たいひ]が 美[うつく]しいですね。	
\\	彼女は二人の顔を見比べたの。	見比べる	彼女[かのじょ]は 二人[ふたり]の 顔[かお]を 見比[みくら]べたの。	
\\	今年は比較的景気がいい。	比較的	今年[ことし]は 比較的[ひかくてき] 景気[けいき]がいい。	
\\	去年の売り上げと比較しましょう。	比較	去年[きょねん]の 売[う]り 上[あ]げと 比較[ひかく]しましょう。	
\\	判決が下された。	判決	判決[はんけつ]が 下[くだ]された。	
\\	彼女はぱっと目を開いた。	ぱっと	彼女[かのじょ]はぱっと 目[め]を 開[ひら]いた。	
\\	ここに判子を押してください。	判子	ここに 判子[はんこ]を 押[お]してください。	
\\	評判の良いレストランに行きました。	評判	評判[ひょうばん]の 良[い]いレストランに 行[い]きました。	
\\	その映画はよい批評を得ているんだ。	批評	その 映画[えいが]はよい 批評[ひひょう]を 得[え]ているんだ。	
\\	俺たちは彼の言動に反感を持ったね。	反感	俺[おれ]たちは 彼[かれ]の 言動[げんどう]に 反感[はんかん]を 持[も]ったね。	
\\	私の予想が当たった。	予想	私[わたし]の 予想[よそう]が 当[あ]たった。	
\\	インターネットは世界の人々をつなぎますね。	つなぐ	インターネットは 世界[せかい]の 人々[ひとびと]をつなぎますね。	
\\	彼は高い理想を持っているの。	理想	彼[かれ]は 高[たか]い 理想[りそう]を 持[も]っているの。	
\\	そんなことは想像できないよ。	想像	そんなことは 想像[そうぞう]できないよ。	
\\	象に乗ってみたいです。	象	象[ぞう]に 乗[の]ってみたいです。	
\\	彼は抽象的な絵が好きだね。	抽象的	彼[かれ]は 抽象的[ちゅうしょうてき]な 絵[え]が 好[す]きだね。	
\\	私があなたの安全を保障します。	保障	私[わたし]があなたの 安全[あんぜん]を 保障[ほしょう]します。	
\\	ひょっとしたら彼はそのことを知らないのかもしれない。	ひょっとしたら	ひょっとしたら 彼[かれ]はそのことを 知[し]らないのかもしれない。	
\\	その事故で多くの人が負傷したの。	負傷	その 事故[じこ]で 多[おお]くの 人[ひと]が 負傷[ふしょう]したの。	
\\	ひらがなをカタカナに変換しました。	変換	ひらがなをカタカナに 変換[へんかん]しました。	
\\	シーツを取り替えました。	取り替える	シーツを 取[と]り 替[か]えました。	
\\	お金がないなら私が立て替えておきます。	立て替える	お 金[かね]がないなら 私[わたし]が 立[た]て 替[か]えておきます。	
\\	天災を防ぐことはできません。	天災	天災[てんさい]を 防[ふせ]ぐことはできません。	
\\	この材料は有害だよ。	有害	この 材料[ざいりょう]は 有害[ゆうがい]だよ。	
\\	ドアがひとりでに閉まったな。	ひとりでに	ドアがひとりでに 閉[し]まったな。	
\\	この農薬は人には無害です。	無害	この 農薬[のうやく]は 人[ひと]には 無害[むがい]です。	
\\	彼の家は台風の被害にあったの。	被害	彼[かれ]の 家[いえ]は 台風[たいふう]の 被害[ひがい]にあったの。	
\\	彼は強敵を見事に破ったね。	破る	彼[かれ]は 強敵[きょうてき]を 見事[みごと]に 破[やぶ]ったね。	
\\	自然の破壊が進んでいるのよ。	破壊	自然[しぜん]の 破壊[はかい]が 進[すす]んでいるのよ。	
\\	彼はシートベルトをしていたので助かったんだ。	助かる	彼[かれ]はシートベルトをしていたので 助[たす]かったんだ。	
\\	彼は水道の蛇口をひねったの。	ひねる	彼[かれ]は 水道[すいどう]の 蛇口[じゃぐち]をひねったの。	
\\	雪道で派手に転んでしまったの。	派手	雪道[ゆきみち]で 派手[はで]に 転[ころ]んでしまったの。	
\\	派出所にだれもいないな。	派出所	派出所[はしゅつじょ]にだれもいないな。	
\\	彼はイギリスに派遣されました。	派遣	彼[かれ]はイギリスに 派遣[はけん]されました。	
\\	あいつが犯人です。	犯人	あいつが 犯人[はんにん]です。	
\\	最近犯罪が減っています。	犯罪	最近[さいきん] 犯罪[はんざい]が 減[へ]っています。	
\\	窓ガラスにひびが入ったよ。	ひび	窓[まど]ガラスにひびが 入[はい]ったよ。	
\\	彼の罪は重いな。	罪	彼[かれ]の 罪[つみ]は 重[おも]いな。	
\\	昨夜、自転車の盗難にあったよ。	盗難	昨夜[さくや]、 自転車[じてんしゃ]の 盗難[とうなん]にあったよ。	
\\	彼女はついに捕まりました。	捕まる	彼女[かのじょ]はついに 捕[つか]まりました。	
\\	少年は網でその蝶を捕まえた。	捕まえる	少年[しょうねん]は 網[あみ]でその 蝶[ちょう]を 捕[つか]まえた。	
\\	トラがおりから逃げ出しました。	逃げ出す	トラがおりから 逃[に]げ 出[だ]しました。	
\\	釣った魚を逃がしました。	逃がす	釣[つ]った 魚[さかな]を 逃[に]がしました。	
\\	主人公が悪者をやっつけた。	やっつける	主人公[しゅじんこう]が 悪者[わるもの]をやっつけた。	
\\	失敗した時の逃げ道を考えたの。	逃げ道	失敗[しっぱい]した 時[とき]の 逃[に]げ 道[みち]を 考[かんが]えたの。	
\\	彼は最後まで戦ったよ。	戦う	彼[かれ]は 最後[さいご]まで 戦[たたか]ったよ。	
\\	彼は新しいことに挑戦している。	挑戦	彼[かれ]は 新[あたら]しいことに 挑戦[ちょうせん]している。	
\\	敗戦の原因は何だろう。	敗戦	敗戦[はいせん]の 原因[げんいん]は 何[なん]だろう。	
\\	長い戦いが終わった。	戦い	長[なが]い 戦[たたか]いが 終[お]わった。	
\\	病院でレントゲンを撮ったよ。	レントゲン	病院[びょういん]でレントゲンを 撮[と]ったよ。	
\\	大戦で多くの人が亡くなりました。	大戦	大戦[たいせん]で 多[おお]くの 人[ひと]が 亡[な]くなりました。	
\\	カレーにヨーグルトを入れて、よく混ぜてください。	混ぜる	カレーにヨーグルトを 入[い]れて、よく 混[ま]ぜてください。	
\\	水と油は混ざりません。	混ざる	水[みず]と 油[あぶら]は 混[ま]ざりません。	
\\	電話の声に雑音が混じっていたの。	混じる	電話[でんわ]の 声[こえ]に 雑音[ざつおん]が 混[ま]じっていたの。	
\\	風で髪が乱れたね。	乱れる	風[かぜ]で 髪[かみ]が 乱[みだ]れたね。	
\\	彼女はスカートよりパンツが似合うね。	パンツ	彼女[かのじょ]はスカートよりパンツが 似合[にあ]うね。	
\\	彼は風紀を乱している。	乱す	彼[かれ]は 風紀[ふうき]を 乱[みだ]している。	
\\	三つの町が統合されて新しい市が誕生したのよ。	統合	三[みっ]つの 町[まち]が 統合[とうごう]されて 新[あたら]しい 市[し]が 誕生[たんじょう]したのよ。	
\\	文字の大きさは統一してください。	統一	文字[もじ]の 大[おお]きさは 統一[とういつ]してください。	
\\	これは昨年の売り上げの統計です。	統計	これは 昨年[さくねん]の 売[う]り 上[あ]げの 統計[とうけい]です。	
\\	来月の総会は東京で行われます。	総会	来月[らいげつ]の 総会[そうかい]は 東京[とうきょう]で 行[おこな]われます。	
\\	日焼けで皮がむけた。	むける	日焼[ひや]けで 皮[かわ]がむけた。	
\\	全員の意見を総合してみましょう。	総合	全員[ぜんいん]の 意見[いけん]を 総合[そうごう]してみましょう。	
\\	総理はヨーロッパを訪問中です。	総理	総理[そうり]はヨーロッパを 訪問中[ほうもんちゅう]です。	
\\	参加者の総数は705人でした。	総数	参加者[さんかしゃ]の 総数[そうすう]は705 人[にん]でした。	
\\	その殺人事件の捜査は2年間続きました。	捜査	その 殺人事件[さつじん じけん]の 捜査[そうさ]は2 年間続[ねんかん つづ]きました。	
\\	湖でスケートをした。	湖	湖[みずうみ]でスケートをした。	
\\	彼女は異文化に対する理解を深めたね。	深める	彼女[かのじょ]は 異文化[いぶんか]に 対[たい]する 理解[りかい]を 深[ふか]めたね。	
\\	彼はぶつぶつと独り言を言ったんだ。	ぶつぶつ	彼[かれ]はぶつぶつと 独[ひと]り 言[ごと]を 言[い]ったんだ。	
\\	二人の愛はますます深まっているね。	深まる	二人[ふたり]の 愛[あい]はますます 深[ふか]まっているね。	
\\	彼は注意深い人です。	注意深い	彼[かれ]は 注意深[ちゅういぶか]い 人[ひと]です。	
\\	次の段落を読んでください。	段落	次[つぎ]の 段落[だんらく]を 読[よ]んでください。	
\\	間違いを見落としたの。	見落とす	間違[まちが]いを 見落[みお]としたの。	
\\	このテストに失敗したら落第です。	落第	このテストに 失敗[しっぱい]したら 落第[らくだい]です。	
\\	彼はぶるぶる震えていたよ。	ぶるぶる	彼[かれ]はぶるぶる 震[ふる]えていたよ。	
\\	今日の海は波が穏やかです。	波	今日[きょう]の 海[うみ]は 波[なみ]が 穏[おだ]やかです。	
\\	川の上流は流れが速いよ。	流れ	川[かわ]の 上流[じょうりゅう]は 流[なが]れが 速[はや]いよ。	
\\	今日は流通の仕組みを勉強しましょう。	流通	今日[きょう]は 流通[りゅうつう]の 仕組[しく]みを 勉強[べんきょう]しましょう。	
\\	彼女は涙を流したんだ。	流す	彼女[かのじょ]は 涙[なみだ]を 流[なが]したんだ。	
\\	川の中流あたりにその村はあるよ。	中流	川[かわ]の 中流[ちゅうりゅう]あたりにその 村[むら]はあるよ。	
\\	ラジオのボリュームを上げてください。	ボリューム	ラジオのボリュームを 上[あ]げてください。	
\\	彼はまだまだ二流の芸人だね。	二流	彼[かれ]はまだまだ 二流[にりゅう]の 芸人[げいにん]だね。	
\\	この服は今の流行です。	流行	この 服[ふく]は 今[いま]の 流行[はやり]です。	
\\	このスタイルは今年の流行です。	流行	このスタイルは 今年[ことし]の 流行[りゅうこう]です。	
\\	洗面用具を忘れた。	洗面	洗面[せんめん] 用具[ようぐ]を 忘[わす]れた。	
\\	風邪をひかないように手洗いとうがいをしましょう。	手洗い	風邪[かぜ]をひかないように 手洗[てあら]いとうがいをしましょう。	
\\	洗面器でハンカチを洗ったの。	洗面器	洗面器[せんめんき]でハンカチを 洗[あら]ったの。	
\\	ちりも積もれば山となる。	ちり	ちりも 積[つ]もれば 山[やま]となる。	
\\	船は沈没しました。	沈没	船[ふね]は 沈没[ちんぼつ]しました。	
\\	明日から急に冷え込むそうです。	冷え込む	明日[あす]から 急[きゅう]に 冷[ひ]え 込[こ]むそうです。	
\\	友達はそのカップルを冷やかしたんだ。	冷やかす	友達[ともだち]はそのカップルを 冷[ひ]やかしたんだ。	
\\	残ったカレーを冷凍しました。	冷凍	残[のこ]ったカレーを 冷凍[れいとう]しました。	
\\	来週、内臓を検査します。	内臓	来週[らいしゅう]、 内臓[ないぞう]を 検査[けんさ]します。	
\\	壁のペンキがはげてきたな。	はげる	壁[かべ]のペンキがはげてきたな。	
\\	この場合、保険が適用されますか。	適用	この 場合[ばあい]、 保険[ほけん]が 適用[てきよう]されますか。	
\\	彼は適切な言葉で説明してくれました。	適切	彼[かれ]は 適切[てきせつ]な 言葉[ことば]で 説明[せつめい]してくれました。	
\\	彼はどんな環境にも適応できます。	適応	彼[かれ]はどんな 環境[かんきょう]にも 適応[てきおう]できます。	
\\	彼はこの仕事に適しています。	適する	彼[かれ]はこの 仕事[しごと]に 適[てき]しています。	
\\	その質問の適当な答えが見つかりません。	適当	その 質問[しつもん]の 適当[てきとう]な 答[こた]えが 見[み]つかりません。	
\\	私のパパは36歳です。	パパ	私[わたし]のパパは36 歳[さい]です。	
\\	健康のため、適度な運動が必要です。	適度	健康[けんこう]のため、 適度[てきど]な 運動[うんどう]が 必要[ひつよう]です。	
\\	髪を赤に染めてみた。	染める	髪[かみ]を 赤[あか]に 染[そ]めてみた。	
\\	その制度は廃止されました。	廃止	その 制度[せいど]は 廃止[はいし]されました。	
\\	私は山の風景が好きです。	風景	私[わたし]は 山[やま]の 風景[ふうけい]が 好[す]きです。	
\\	今、あの国は不景気らしいよ。	不景気	今[いま]、あの 国[くに]は 不景気[ふけいき]らしいよ。	
\\	彼の声はよく響きます。	響く	彼[かれ]の 声[こえ]はよく 響[ひび]きます。	
\\	彼女が僕の手をつかみました。	つかむ	彼女[かのじょ]が 僕[ぼく]の 手[て]をつかみました。	
\\	この部屋は日光がよく当たるね。	日光	この 部屋[へや]は 日光[にっこう]がよく 当[あ]たるね。	
\\	一筋の光が窓から差し込んだの。	光	一筋[ひとすじ]の 光[ひかり]が 窓[まど]から 差[さ]し 込[こ]んだの。	
\\	これから身長と体重を測定します。	測定	これから 身長[しんちょう]と 体重[たいじゅう]を 測定[そくてい]します。	
\\	このドアの高さを測ってください。	測る	このドアの 高[たか]さを 測[はか]ってください。	
\\	彼女はとても陽気です。	陽気	彼女[かのじょ]はとても 陽気[ようき]です。	
\\	食べ物を残してはもったいないわよ。	もったいない	食[た]べ 物[もの]を 残[のこ]してはもったいないわよ。	
\\	最近のプロ野球は面白くなってきたね。	プロ野球	最近[さいきん]のプロ 野球[やきゅう]は 面白[おもしろ]くなってきたね。	
\\	電球が切れたので交換しましょう。	電球	電球[でんきゅう]が 切[き]れたので 交換[こうかん]しましょう。	
\\	学生時代を懐かしく振り返ったんだ。	振り返る	学生時代[がくせい じだい]を 懐[なつ]かしく 振[ふ]り 返[かえ]ったんだ。	
\\	彼女は振り向いて俺に微笑んだんだ。	振り向く	彼女[かのじょ]は 振[ふ]り 向[む]いて 俺[おれ]に 微笑[ほほえ]んだんだ。	
\\	彼はバットの振りが大きすぎる。	振り	彼[かれ]はバットの 振[ふ]りが 大[おお]きすぎる。	
\\	ふざけるのは止めて。	ふざける	ふざけるのは 止[や]めて。	
\\	私の秘書はとても優秀です。	秘書	私[わたし]の 秘書[ひしょ]はとても 優秀[ゆうしゅう]です。	
\\	彼女の誕生日パーティーを密かに計画しています。	密か	彼女[かのじょ]の 誕生日[たんじょうび]パーティーを 密[ひそ]かに 計画[けいかく]しています。	
\\	その事故で貴い命が失われました。	貴い	その 事故[じこ]で 貴[とうと]い 命[いのち]が 失[うしな]われました。	
\\	パトカーが車を追跡しているわ。	追跡	パトカーが 車[くるま]を 追跡[ついせき]しているわ。	
\\	車を塀に寄せたよ。	寄せる	車[くるま]を 塀[へい]に 寄[よ]せたよ。	
\\	昨日は風邪で少しだるかったんだ。	だるい	昨日[きのう]は 風邪[かぜ]で 少[すこ]しだるかったんだ。	
\\	帰りに叔母の家に寄ります。	寄る	帰[かえ]りに 叔母[おば]の 家[いえ]に 寄[よ]ります。	
\\	彼に近寄らないで。	近寄る	彼[かれ]に 近寄[ちかよ]らないで。	
\\	今日は寄り道してから帰ります。	寄り道	今日[きょう]は 寄[よ]り 道[みち]してから 帰[かえ]ります。	
\\	ニューヨークには芸術家が寄り集まっているの。	寄り集まる	ニューヨークには 芸術家[げいじゅつか]が 寄[よ]り 集[あつ]まっているの。	
\\	今日は歴史的な日です。	歴史的	今日[きょう]は 歴史的[れきしてき]な 日[ひ]です。	
\\	私が一番好きな宝石はダイヤモンドなの。	宝石	私[わたし]が 一番好[いちばん す]きな 宝石[ほうせき]はダイヤモンドなの。	
\\	紙を細かくちぎってください。	ちぎる	紙[かみ]を 細[こま]かくちぎってください。	
\\	本音と建て前は違うことが多いよ。	建て前	本音[ほんね]と 建[た]て 前[まえ]は 違[ちが]うことが 多[おお]いよ。	
\\	メートルは長さの単位です。	単位	メートルは 長[なが]さの 単位[たんい]です。	
\\	彼女は会社で高い地位に就いているよ。	地位	彼女[かのじょ]は 会社[かいしゃ]で 高[たか]い 地位[ちい]に 就[つ]いているよ。	
\\	自民党から新しい党が分離したね。	分離	自民党[じみんとう]から 新[あたら]しい 党[とう]が 分離[ぶんり]したね。	
\\	実家を4年間離れていました。	離れる	実家[じっか]を4 年間[ねんかん] 離[はな]れていました。	
\\	みそ汁にねぎを入れました。	ねぎ	みそ 汁[しる]にねぎを 入[い]れました。	
\\	子供の手を離さないでください。	離す	子供[こども]の 手[て]を 離[はな]さないでください。	
\\	台風で停電したよ。	停電	台風[たいふう]で 停電[ていでん]したよ。	
\\	突然、機械が停止してしまったの。	停止	突然[とつぜん]、 機械[きかい]が 停止[ていし]してしまったの。	
\\	知っている範囲で教えてください。	範囲	知[し]っている 範囲[はんい]で 教[おし]えてください。	
\\	この町は山と海が隣り合っています。	隣り合う	この 町[まち]は 山[やま]と 海[うみ]が 隣[とな]り 合[あ]っています。	
\\	服が雨でびしょびしょになったよ。	びしょびしょ	服[ふく]が 雨[あめ]でびしょびしょになったよ。	
\\	目の前を猫が横切ったんだ。	横切る	目[め]の 前[まえ]を 猫[ねこ]が 横切[よこぎ]ったんだ。	
\\	彼女の横顔は素敵だ。	横顔	彼女[かのじょ]の 横顔[よこがお]は 素敵[すてき]だ。	
\\	停電のため仕事を中断しました。	中断	停電[ていでん]のため 仕事[しごと]を 中断[ちゅうだん]しました。	
\\	地震のために1週間、断水したの。	断水	地震[じしん]のために1 週間[しゅうかん]、 断水[だんすい]したの。	
\\	少しの油断が大きな事故につながります。	油断	少[すこ]しの 油断[ゆだん]が 大[おお]きな 事故[じこ]につながります。	
\\	まだ原因は断定できません。	断定	まだ 原因[げんいん]は 断定[だんてい]できません。	
\\	昨日バーゲンでスーツを買いました。	バーゲン	昨日[きのう]バーゲンでスーツを 買[か]いました。	
\\	彼は父親の店を継いだの。	継ぐ	彼[かれ]は 父親[ちちおや]の 店[みせ]を 継[つ]いだの。	
\\	京都から中継で放送しています。	中継	京都[きょうと]から 中継[ちゅうけい]で 放送[ほうそう]しています。	
\\	彼は幅広い知識を持っています。	幅広い	彼[かれ]は 幅広[はばひろ]い 知識[ちしき]を 持[も]っています。	
\\	この道は幅が狭いので気をつけて運転してください。	幅	この 道[みち]は 幅[はば]が 狭[せま]いので 気[き]をつけて 運転[うんてん]してください。	
\\	ユーラシアは世界で最も大きい大陸です。	大陸	ユーラシアは 世界[せかい]で 最[もっと]も 大[おお]きい 大陸[たいりく]です。	
\\	ガス会社がメーターを調べにきたぞ。	メーター	ガス 会社[がいしゃ]がメーターを 調[しら]べにきたぞ。	
\\	飛行機が無事着陸したわね。	着陸	飛行機[ひこうき]が 無事[ぶじ] 着陸[ちゃくりく]したわね。	
\\	まもなく飛行機が離陸します。	離陸	まもなく 飛行機[ひこうき]が 離陸[りりく]します。	
\\	彼は中国の内陸を旅行しました。	内陸	彼[かれ]は 中国[ちゅうごく]の 内陸[ないりく]を 旅行[りょこう]しました。	
\\	あの動物は陸上で生活しています。	陸上	あの 動物[どうぶつ]は 陸上[りくじょう]で 生活[せいかつ]しています。	
\\	ウミガメが陸に上がってきたんだ。	陸	ウミガメが 陸[りく]に 上[あ]がってきたんだ。	
\\	もしかするとあなたに一度お会いしてますか。	もしかすると	もしかするとあなたに 一度[いちど]お 会[あ]いしてますか。	
\\	南極でペンギンを見ました。	南極	南極[なんきょく]でペンギンを 見[み]ました。	
\\	北極にペンギンはいません。	北極	北極[ほっきょく]にペンギンはいません。	
\\	棒の先端を持って下さい。	先端	棒[ぼう]の 先端[せんたん]を 持[も]って 下[くだ]さい。	
\\	辞書は本棚の端にあります。	端	辞書[じしょ]は 本棚[ほんだな]の 端[はし]にあります。	
\\	来週、妻の実家を訪問します。	訪問	来週[らいしゅう]、 妻[つま]の 実家[じっか]を 訪問[ほうもん]します。	
\\	女房は実家に帰っています。	女房	女房[にょうぼう]は 実家[じっか]に 帰[かえ]っています。	
\\	荷物が重くて手がちぎれそうだ。	ちぎれる	荷物[にもつ]が 重[おも]くて 手[て]がちぎれそうだ。	
\\	この話は課長には内緒ですよ。	内緒	この 話[はなし]は 課長[かちょう]には 内緒[ないしょ]ですよ。	
\\	友達と百貨店で買い物をしました。	百貨店	友達[ともだち]と 百貨店[ひゃっかてん]で 買[か]い 物[もの]をしました。	
\\	この商品はトラックで輸送します。	輸送	この 商品[しょうひん]はトラックで 輸送[ゆそう]します。	
\\	その選手は怪我を乗り越えて復活したわね。	復活	その 選手[せんしゅ]は 怪我[けが]を 乗[の]り 越[こ]えて 復活[ふっかつ]したわね。	
\\	この携帯電話は操作が簡単です。	操作	この 携帯電話[けいたい でんわ]は 操作[そうさ]が 簡単[かんたん]です。	
\\	彼女はにっこりほほえんだ。	にっこり	彼女[かのじょ]はにっこりほほえんだ。	
\\	兄は体操の選手です。	体操	兄[あに]は 体操[たいそう]の 選手[せんしゅ]です。	
\\	縦2列に並んでください。	縦	縦[たて] 
\\	列[れつ]に 並[なら]んでください。	
\\	このボートは操縦が簡単です。	操縦	このボートは 操縦[そうじゅう]が 簡単[かんたん]です。	
\\	作文は縦書きで書いてください。	縦書き	作文[さくぶん]は 縦書[たてが]きで 書[か]いてください。	
\\	日本では北海道に滞在しました。	滞在	日本[にっぽん]では 北海道[ほっかいどう]に 滞在[たいざい]しました。	
\\	このパイナップルを切ってください。	ぱいなっぷる	このパイナップルを 切[き]ってください。	
\\	この都市は工業地帯です。	地帯	この 都市[とし]は 工業[こうぎょう] 地帯[ちたい]です。	
\\	店に熱帯の果物が並んでいますよ。	熱帯	店[みせ]に 熱帯[ねったい]の 果物[くだもの]が 並[なら]んでいますよ。	
\\	あの村は伝統を保守している。	保守	あの 村[むら]は 伝統[でんとう]を 保守[ほしゅ]している。	
\\	皆で暖かく見守りましょう。	見守る	皆[みんな]で 暖[あたた]かく 見守[みまも]りましょう。	
\\	私は留守番の者です。	留守番	私[わたし]は 留守番[るすばん]の 者[もの]です。	
\\	パンダは笹を食べます。	ぱんだ	パンダは 笹[ささ]を 食[た]べます。	
\\	話を戻しましょう。	戻す	話[はなし]を 戻[もど]しましょう。	
\\	緑を取り戻す必要があります。	取り戻す	緑[みどり]を 取[と]り 戻[もど]す 必要[ひつよう]があります。	
\\	飛行機の運賃が払い戻されたの。	払い戻す	飛行機[ひこうき]の 運賃[うんちん]が 払[はら]い 戻[もど]されたの。	
\\	いつも9時に子供を寝かします。	寝かす	いつも9 時[じ]に 子供[こども]を 寝[ね]かします。	
\\	うっかり寝過ごしてしまったんだ。	寝過ごす	うっかり 寝過[ねす]ごしてしまったんだ。	
\\	早寝は健康のためによいことです。	早寝	早寝[はやね]は 健康[けんこう]のためによいことです。	
\\	運動して汗びっしょりです。	びっしょり	運動[うんどう]して 汗[あせ]びっしょりです。	
\\	赤ちゃんをベッドに寝かせた。	寝かせる	赤[あか]ちゃんをベッドに 寝[ね]かせた。	
\\	土手に寝転んで空をながめました。	寝転ぶ	土手[どて]に 寝転[ねころ]んで 空[そら]をながめました。	
\\	冷静に話し合いましょう。	冷静	冷静[れいせい]に 話[はな]し 合[あ]いましょう。	
\\	両者の意見を聞きましょう。	両者	両者[りょうしゃ]の 意見[いけん]を 聞[き]きましょう。	
\\	勉強と仕事の両立は難しいよ。	両立	勉強[べんきょう]と 仕事[しごと]の 両立[りょうりつ]は 難[むずか]しいよ。	
\\	ペンチで針金を切ります。	ペンチ	ペンチで 針金[はりがね]を 切[き]ります。	
\\	彼は両手を握り締めたの。	両手	彼[かれ]は 両手[りょうて]を 握[にぎ]り 締[し]めたの。	
\\	紙の両面に字が書かれていたよ。	両面	紙[かみ]の 両面[りょうめん]に 字[じ]が 書[か]かれていたよ。	
\\	これが建物の側面の写真です。	側面	これが 建物[たてもの]の 側面[そくめん]の 写真[しゃしん]です。	
\\	道の両側にたくさんお店がありますよ。	両側	道[みち]の 両側[りょうがわ]にたくさんお 店[みせ]がありますよ。	
\\	募集要項をよくお読みください。	要項	募集[ぼしゅう] 要項[ようこう]をよくお 読[よ]みください。	
\\	あの建物はモダンだね。	モダン	あの 建物[たてもの]はモダンだね。	
\\	私の家は赤い屋根が目印です。	目印	私[わたし]の 家[いえ]は 赤[あか]い 屋根[やね]が 目印[めじるし]です。	
\\	その細長い棒を取ってください。	細長い	その 細長[ほそなが]い 棒[ぼう]を 取[と]ってください。	
\\	先生の論文が雑誌に載りましたよ。	載る	先生[せんせい]の 論文[ろんぶん]が 雑誌[ざっし]に 載[の]りましたよ。	
\\	車に荷物を積んでください。	積む	車[くるま]に 荷物[にもつ]を 積[つ]んでください。	
\\	この土地の面積はどれ位ですか。	面積	この 土地[とち]の 面積[めんせき]はどれ 位[くらい]ですか。	
\\	彼の翻訳は分かりやすいですよ。	翻訳	彼[かれ]の 翻訳[ほんやく]は 分[わ]かりやすいですよ。	
\\	ラジオ番組にリクエストを送ったの。	リクエスト	ラジオ 番組[ばんぐみ]にリクエストを 送[おく]ったの。	
\\	私は通訳です。	通訳	私[わたし]は 通訳[つうやく]です。	
\\	彼には申し訳ないことをした。	申し訳ない	彼[かれ]には 申[もう]し 訳[わけ]ないことをした。	
\\	そんな訳で、私は仕事を辞めました。	訳	そんな 訳[わけ]で、 私[わたし]は 仕事[しごと]を 辞[や]めました。	
\\	申し訳ございません。	申し訳	申[もう]し 訳[わけ]ございません。	
\\	間違いを訂正しました。	訂正	間違[まちが]いを 訂正[ていせい]しました。	
\\	ビーチで友達とサーフィンしたよ。	ビーチ	ビーチで 友達[ともだち]とサーフィンしたよ。	
\\	その問題についてクラスで討論しました。	討論	その 問題[もんだい]についてクラスで 討論[とうろん]しました。	
\\	贈り物に手書きのカードを添えました。	添える	贈[おく]り 物[もの]に 手書[てが]きのカードを 添[そ]えました。	
\\	あなたの期待に添えなくてすみません。	添う	あなたの 期待[きたい]に 添[そ]えなくてすみません。	
\\	事件現場付近で怪しい人物を見た。	付近	事件現場[じけん げんば] 付近[ふきん]で 怪[あや]しい 人物[じんぶつ]を 見[み]た。	
\\	今日は付き合いで帰りが遅くなります。	付き合い	今日[きょう]は 付[つ]き 合[あ]いで 帰[かえ]りが 遅[おそ]くなります。	
\\	ビタミンを毎日とりましょう。	ビタミン	ビタミンを 毎日[まいにち]とりましょう。	
\\	自分の考えを付け加えました。	付け加える	自分[じぶん]の 考[かんが]えを 付[つ]け 加[くわ]えました。	
\\	彼女と4年ほど付き合っています。	付き合う	彼女[かのじょ]と4 年[ねん]ほど 付[つ]き 合[あ]っています。	
\\	子猫にトラと名付けました。	名付ける	子猫[こねこ]にトラと 名付[なづ]けました。	
\\	今月の付録はアニメの
\\	です。	付録	今月[こんげつ]の 付録[ふろく]はアニメの 
\\	[でぃーぶぃーでぃー]です。	
\\	私は顔を窓に近付けたんだ。	近付ける	私[わたし]は 顔[かお]を 窓[まど]に 近付[ちかづ]けたんだ。	
\\	何か付け足すことはありますか。	付け足す	何[なに]か 付[つ]け 足[た]すことはありますか。	
\\	フライパンで目玉焼きを作ったよ。	フライパン	フライパンで 目玉焼[めだまや]きを 作[つく]ったよ。	
\\	今日の日付けは6月19日です。	日付け	今日[きょう]の 日付[ひづ]けは6 月19日[がつ 
\\	にち]です。	
\\	このソフトには色々なツールが付属しています。	付属	このソフトには 色々[いろいろ]なツールが 付属[ふぞく]しています。	
\\	私は市民オーケストラに属しています。	属する	私[わたし]は 市民[しみん]オーケストラに 属[ぞく]しています。	
\\	彼は十分反省しています。	反省	彼[かれ]は 十分[じゅうぶん] 反省[はんせい]しています。	
\\	時間がないので詳細は省きます。	省く	時間[じかん]がないので 詳細[しょうさい]は 省[はぶ]きます。	
\\	まあまあ、落ち着いてください。	まあまあ	まあまあ、 落[お]ち 着[つ]いてください。	
\\	国際連合を略して国連といいます。	略す	国際連合[こくさいれんごう]を 略[りゃく]して 国連[こくれん]といいます。	
\\	は何の略か知っていますか。	略	
\\	[あいてぃー]は 何[なん]の 略[りゃく]か 知[し]っていますか。	
\\	大概、風邪は寝ていれば治ります。	大概	大概[たいがい]、 風邪[かぜ]は 寝[ね]ていれば 治[なお]ります。	
\\	私を含めて10人が参加しました。	含める	私[わたし]を 含[ふく]めて10 人[にん]が 参加[さんか]しました。	
\\	その食品は有害物質を含んでいるぞ。	含む	その 食品[しょくひん]は 有害物質[ゆうがい ぶっしつ]を 含[ふく]んでいるぞ。	
\\	タコには足が8本ある。	たこ	タコには 足[あし]が8 本[ぽん]ある。	
\\	家から学校までの道順を教えてください。	道順	家[いえ]から 学校[がっこう]までの 道順[みちじゅん]を 教[おし]えてください。	
\\	列車がホームに入ってきた。	列車	列車[れっしゃ]がホームに 入[はい]ってきた。	
\\	日本は列島です。	列島	日本[にっぽん]は 列島[れっとう]です。	
\\	店の前に長い列ができていたね。	列	店[みせ]の 前[まえ]に 長[なが]い 列[れつ]ができていたね。	
\\	例外は認めません。	例外	例外[れいがい]は 認[みと]めません。	
\\	デザートにケーキを食べました。	デザート	デザートにケーキを 食[た]べました。	
\\	努力と結果が比例していないの。	比例	努力[どりょく]と 結果[けっか]が 比例[ひれい]していないの。	
\\	用例を使って説明してください。	用例	用例[ようれい]を 使[つか]って 説明[せつめい]してください。	
\\	彼は彼女をバラに例えたんだ。	例える	彼[かれ]は 彼女[かのじょ]をバラに 例[たと]えたんだ。	
\\	みんなで例題を解いてみましょう。	例題	みんなで 例題[れいだい]を 解[と]いてみましょう。	
\\	例年8月は雨が少ないね。	例年	例年[れいねん] 
\\	月[がつ]は 雨[あめ]が 少[すく]ないね。	
\\	例えを上げると話が分かり易くなる。	例え	例[たと]えを 上[あ]げると 話[はなし]が 分[わ]かり 易[やす]くなる。	
\\	とっさに彼の名前が出てこなかった。	とっさに	とっさに 彼[かれ]の 名前[なまえ]が 出[で]てこなかった。	
\\	彼に言い訳の余地はありません。	余地	彼[かれ]に 言[い]い 訳[わけ]の 余地[よち]はありません。	
\\	余計なことを言ってはだめよ。	余計	余計[よけい]なことを 言[い]ってはだめよ。	
\\	食事は余分に用意してあります。	余分	食事[しょくじ]は 余分[よぶん]に 用意[ようい]してあります。	
\\	途中で30分ほど休みましょう。	途中	途中[とちゅう]で30 分[ぷん]ほど 休[やす]みましょう。	
\\	この道具の用途を説明します。	用途	この 道具[どうぐ]の 用途[ようと]を 説明[せつめい]します。	
\\	うなぎはぬるぬるしていますね。	ぬるぬる	うなぎはぬるぬるしていますね。	
\\	私は中途採用で入社しました。	中途	私[わたし]は 中途[ちゅうと] 採用[さいよう]で 入社[にゅうしゃ]しました。	
\\	違法駐車の取り締まりが厳しくなったの。	取り締まり	違法駐車[いほう ちゅうしゃ]の 取[と]り 締[し]まりが 厳[きび]しくなったの。	
\\	警察が飲酒運転を取り締まっていますよ。	取り締まる	警察[けいさつ]が 飲酒運転[いんしゅ うんてん]を 取[と]り 締[し]まっていますよ。	
\\	食べ過ぎたのでベルトを緩めたの。	緩める	食[た]べ 過[す]ぎたのでベルトを 緩[ゆる]めたの。	
\\	ここから先は緩やかな坂になっています。	緩やか	ここから 先[さき]は 緩[ゆる]やかな 坂[さか]になっています。	
\\	この電車は本当にのろいですね。	のろい	この 電車[でんしゃ]は 本当[ほんとう]にのろいですね。	
\\	彼は気が緩んでいます。	緩む	彼[かれ]は 気[き]が 緩[ゆる]んでいます。	
\\	私は3人の子供を養っています。	養う	私[わたし]は3 人[にん]の 子供[こども]を 養[やしな]っています。	
\\	彼は心が豊かな人です。	豊か	彼[かれ]は 心[こころ]が 豊[ゆた]かな 人[ひと]です。	
\\	彼の人生は変化に富んでいるな。	富む	彼[かれ]の 人生[じんせい]は 変化[へんか]に 富[と]んでいるな。	
\\	この果物はビタミンが豊富です。	豊富	この 果物[くだもの]はビタミンが 豊富[ほうふ]です。	
\\	彼の運転は目茶苦茶です。	目茶苦茶	彼[かれ]の 運転[うんてん]は 目茶苦茶[めちゃくちゃ]です。	
\\	カエルがぴょんぴょんはねているね。	ぴょんぴょん	カエルがぴょんぴょんはねているね。	
\\	労働者がストをしたそうだよ。	労働者	労働者[ろうどうしゃ]がストをしたそうだよ。	
\\	彼らは1日7時間労働している。	労働	彼[かれ]らは1 日7時間[にち 
\\	じかん] 労働[ろうどう]している。	
\\	部下が疲労で倒れたの。	疲労	部下[ぶか]が 疲労[ひろう]で 倒[たお]れたの。	
\\	仕事には時間と労力が必要です。	労力	仕事[しごと]には 時間[じかん]と 労力[ろうりょく]が 必要[ひつよう]です。	
\\	彼は貧しい家庭で育ったんだ。	貧しい	彼[かれ]は 貧[まず]しい 家庭[かてい]で 育[そだ]ったんだ。	
\\	犬の毛をブラシでとかしてやったの。	ブラシ	犬[いぬ]の 毛[け]をブラシでとかしてやったの。	
\\	彼女はまだ仕事の経験が乏しい。	乏しい	彼女[かのじょ]はまだ 仕事[しごと]の 経験[けいけん]が 乏[とぼ]しい。	
\\	貧乏人が大金持ちになることもあるさ。	貧乏人	貧乏人[びんぼうにん]が 大金持[おおがねも]ちになることもあるさ。	
\\	この仕事は辛いです。	辛い	この 仕事[しごと]は 辛[つら]いです。	
\\	彼ほど不幸せな人はいないな。	不幸せ	彼[かれ]ほど 不幸[ふしあわ]せな 人[ひと]はいないな。	
\\	姉は福祉の勉強をしています。	福祉	姉[あね]は 福祉[ふくし]の 勉強[べんきょう]をしています。	
\\	サラダにマヨネーズをかけたの。	マヨネーズ	サラダにマヨネーズをかけたの。	
\\	彼女は水玉模様のスカートをはいているね。	模様	彼女[かのじょ]は 水玉[みずたま] 模様[もよう]のスカートをはいているね。	
\\	彼は全校生徒の模範です。	模範	彼[かれ]は 全校生徒[ぜんこう せいと]の 模範[もはん]です。	
\\	中世の農民は貧しかった。	農民	中世[ちゅうせい]の 農民[のうみん]は 貧[まず]しかった。	
\\	彼は農家に生まれました。	農家	彼[かれ]は 農家[のうか]に 生[う]まれました。	
\\	私は農村で育ちました。	農村	私[わたし]は 農村[のうそん]で 育[そだ]ちました。	
\\	低気圧が近づいています。	低気圧	低気圧[ていきあつ]が 近[ちか]づいています。	
\\	だらしない格好をしないでください。	だらしない	だらしない 格好[かっこう]をしないでください。	
\\	今日は授業を1時間に短縮します。	短縮	今日[きょう]は 授業[じゅぎょう]を1 時間[じかん]に 短縮[たんしゅく]します。	
\\	彼はタイムを1秒縮めたの。	縮める	彼[かれ]はタイムを1 秒[びょう] 縮[ちぢ]めたの。	
\\	この服の縮み具合はひどい。	縮む	この 服[ふく]の 縮[ちぢ]み 具合[ぐあい]はひどい。	
\\	この服の縮み具合はひどい。	縮み	この 服[ふく]の 縮[ちぢ]み 具合[ぐあい]はひどい。	
\\	兄との身長の差が縮まった。	縮まる	兄[あに]との 身長[しんちょう]の 差[さ]が 縮[ちぢ]まった。	
\\	このとうもろこしは甘くておいしい。	とうもろこし	このとうもろこしは 甘[あま]くておいしい。	
\\	ストレッチで筋肉を伸ばしましょう。	伸ばす	ストレッチで 筋肉[きんにく]を 伸[の]ばしましょう。	
\\	この写真を引き伸ばしてください。	引き伸ばす	この 写真[しゃしん]を 引[ひ]き 伸[の]ばしてください。	
\\	追伸、お兄さんはお元気ですか。	追伸	追伸[ついしん]、お 兄[にい]さんはお 元気[げんき]ですか。	
\\	以前の収入は今の倍はあった。	倍	以前[いぜん]の 収入[しゅうにゅう]は 今[いま]の 倍[ばい]はあった。	
\\	この荷物は重量超過です。	超過	この 荷物[にもつ]は 重量[じゅうりょう] 超過[ちょうか]です。	
\\	木の葉がひらひらと落ちたの。	ひらひら	木[こ]の 葉[は]がひらひらと 落[お]ちたの。	
\\	彼は悲しみを乗り越えて強く生きた。	乗り越える	彼[かれ]は 悲[かな]しみを 乗[の]り 越[こ]えて 強[つよ]く 生[い]きた。	
\\	乗り越しを機械で精算したの。	乗り越し	乗[の]り 越[こ]しを 機械[きかい]で 精算[せいさん]したの。	
\\	居眠りして降りる駅を乗り越した。	乗り越す	居眠[いねむ]りして 降[お]りる 駅[えき]を 乗[の]り 越[こ]した。	
\\	彼はグループから抜けました。	抜ける	彼[かれ]はグループから 抜[ぬ]けました。	
\\	ワインのコルクを抜きました。	抜く	ワインのコルクを 抜[ぬ]きました。	
\\	今日はぽかぽか暖かい日です。	ぽかぽか	今日[きょう]はぽかぽか 暖[あたた]かい 日[ひ]です。	
\\	太陽は東から昇ります。	昇る	太陽[たいよう]は 東[ひがし]から 昇[のぼ]ります。	
\\	桜のつぼみが膨らんだのね。	膨らむ	桜[さくら]のつぼみが 膨[ふく]らんだのね。	
\\	お腹が膨れたら眠くなった。	膨れる	お 腹[なか]が 膨[ふく]れたら 眠[ねむ]くなった。	
\\	店の外にまだ営業中の札がでているよ。	札	店[みせ]の 外[そと]にまだ 営業中[えいぎょう ちゅう]の 札[ふだ]がでているよ。	
\\	生徒たちは校内では名札をつけます。	名札	生徒[せいと]たちは 校内[こうない]では 名札[なふだ]をつけます。	
\\	先生にお礼の手紙を書きました。	礼	先生[せんせい]にお 礼[れい]の 手紙[てがみ]を 書[か]きました。	
\\	人は見かけによらないな。	よる	人[ひと]は 見[み]かけによらないな。	
\\	車のライトが反射していますね。	反射	車[くるま]のライトが 反射[はんしゃ]していますね。	
\\	試験の日程が発表されました。	日程	試験[しけん]の 日程[にってい]が 発表[はっぴょう]されました。	
\\	冗談にも程がある。	程	冗談[じょうだん]にも 程[ほど]がある。	
\\	私たちのチームが優勝しました。	優勝	私[わたし]たちのチームが 優勝[ゆうしょう]しました。	
\\	彼はとても優秀な生徒です。	優秀	彼[かれ]はとても 優秀[ゆうしゅう]な 生徒[せいと]です。	
\\	この服は大き過ぎてだぶだぶです。	だぶだぶ	この 服[ふく]は 大[おお]き 過[す]ぎてだぶだぶです。	
\\	ゴミは透明な袋に入れて出してください。	透明	ゴミは 透明[とうめい]な 袋[ふくろ]に 入[い]れて 出[だ]してください。	
\\	半導体は様々な製品に使われているよ。	半導体	半導体[はんどうたい]は 様々[さまざま]な 製品[せいひん]に 使[つか]われているよ。	
\\	先生は私たちを導いてくれます。	導く	先生[せんせい]は 私[わたし]たちを 導[みちび]いてくれます。	
\\	彼は私との結婚を望んでいます。	望む	彼[かれ]は 私[わたし]との 結婚[けっこん]を 望[のぞ]んでいます。	
\\	夜は10時までに寝るのが望ましいの。	望ましい	夜[よる]は10 時[じ]までに 寝[ね]るのが 望[のぞ]ましいの。	
\\	床がつるつるすべります。	つるつる	床[ゆか]がつるつるすべります。	
\\	お客様の要望を聞かせてください。	要望	お 客様[きゃくさま]の 要望[ようぼう]を 聞[き]かせてください。	
\\	私の望みは海外で暮らすことです。	望み	私[わたし]の 望[のぞ]みは 海外[かいがい]で 暮[く]らすことです。	
\\	彼は有望な社員です。	有望	彼[かれ]は 有望[ゆうぼう]な 社員[しゃいん]です。	
\\	彼は欲望が強い人です。	欲望	彼[かれ]は 欲望[よくぼう]が 強[つよ]い 人[ひと]です。	
\\	その国の人々は平和を待ち望んでいるの。	待ち望む	その 国[くに]の 人々[ひとびと]は 平和[へいわ]を 待[ま]ち 望[のぞ]んでいるの。	
\\	良い成績が取れるように努めます。	努める	良[い]い 成績[せいせき]が 取[と]れるように 努[つと]めます。	
\\	私は毎朝トーストを2枚食べます。	トースト	私[わたし]は 毎朝[まいあさ]トーストを2 枚食[まい た]べます。	
\\	先日会社から独立しました。	独立	先日会社[せんじつ かいしゃ]から 独立[どくりつ]しました。	
\\	彼の服のセンスは独特よね。	独特	彼[かれ]の 服[ふく]のセンスは 独特[どくとく]よね。	
\\	彼女はいつも独り言を言うんだ。	独り言	彼女[かのじょ]はいつも 独[ひと]り 言[ごと]を 言[い]うんだ。	
\\	旅行中は身の安全が第一です。	身	旅行中[りょこうちゅう]は 身[み]の 安全[あんぜん]が 第一[だいいち]です。	
\\	かばんの中身を見せてください。	中身	かばんの 中身[なかみ]を 見[み]せてください。	
\\	突然の質問にどきっとしたよ。	どきっと	突然[とつぜん]の 質問[しつもん]にどきっとしたよ。	
\\	私にとって動物は身近な存在です。	身近	私[わたし]にとって 動物[どうぶつ]は 身近[みぢか]な 存在[そんざい]です。	
\\	身の回りのお世話は私がします。	身の回り	身[み]の 回[まわ]りのお 世話[せわ]は 私[わたし]がします。	
\\	身なりのいい人が入ってきた。	身なり	身[み]なりのいい 人[ひと]が 入[はい]ってきた。	
\\	彼は身振りを交えて説明してくれたの。	身振り	彼[かれ]は 身振[みぶ]りを 交[まじ]えて 説明[せつめい]してくれたの。	
\\	この2社が市場を独占しています。	独占	この2 社[しゃ]が 市場[しじょう]を 独占[どくせん]しています。	
\\	車は雪道をのろのろと走ったの。	のろのろ	車[くるま]は 雪道[ゆきみち]をのろのろと 走[はし]ったの。	
\\	あの夫婦は仲良く暮らしているよ。	仲良く	あの 夫婦[ふうふ]は 仲良[なかよ]く 暮[く]らしているよ。	
\\	あの3人組は仲良しですね。	仲良し	あの3 人組[にんぐみ]は 仲良[なかよ]しですね。	
\\	けんかした友達と仲直りしました。	仲直り	けんかした 友達[ともだち]と 仲直[なかなお]りしました。	
\\	あなたたちは仲がいいですね。	仲	あなたたちは 仲[なか]がいいですね。	
\\	彼には仲間がたくさんいる。	仲間	彼[かれ]には 仲間[なかま]がたくさんいる。	
\\	あの夫婦は私たちの仲人です。	仲人	あの 夫婦[ふうふ]は 私[わたし]たちの 仲人[なこうど]です。	
\\	新車はぴかぴかですね。	ぴかぴか	新車[しんしゃ]はぴかぴかですね。	
\\	月が庭を照らしている。	照らす	月[つき]が 庭[にわ]を 照[て]らしている。	
\\	日差しが強く照りつけますね。	照る	日差[ひざ]しが 強[つよ]く 照[て]りつけますね。	
\\	スミス夫人がいらっしゃいました。	夫人	スミス 夫人[ふじん]がいらっしゃいました。	
\\	婦人服売り場は5階でございます。	婦人	婦人[ふじん] 服売[ふく う]り 場[ば]は5 階[かい]でございます。	
\\	昨日の夜、社長ご夫妻と食事をしました。	夫妻	昨日[きのう]の 夜[よる]、 社長[しゃちょう]ご 夫妻[ふさい]と 食事[しょくじ]をしました。	
\\	彼は上司にぺこぺこしている。	ぺこぺこ	彼[かれ]は 上司[じょうし]にぺこぺこしている。	
\\	チームの中では相互の信頼が大切です。	相互	チームの 中[なか]では 相互[そうご]の 信頼[しんらい]が 大切[たいせつ]です。	
\\	互いの話をよく聞きなさい。	互い	互[たが]いの 話[はなし]をよく 聞[き]きなさい。	
\\	私の昇進を皆が喜んでくれたの。	皆	私[わたし]の 昇進[しょうしん]を 皆[みな]が 喜[よろこ]んでくれたの。	
\\	我々の決意は固いです。	我々	我々[われわれ]の 決意[けつい]は 固[かた]いです。	
\\	彼は我が国を代表する作家です。	我が国	彼[かれ]は 我[わ]が 国[くに]を 代表[だいひょう]する 作家[さっか]です。	
\\	この本は薄くてぺらぺらですね。	ぺらぺら	この 本[ほん]は 薄[うす]くてぺらぺらですね。	
\\	ぜひ我が家に遊びに来てください。	我が家	ぜひ 我[わ]が 家[や]に 遊[あそ]びに 来[き]てください。	
\\	彼女の年齢は27です。	年齢	彼女[かのじょ]の 年齢[ねんれい]は27です。	
\\	彼女は恋愛にあこがれる年ごろです。	恋愛	彼女[かのじょ]は 恋愛[れんあい]にあこがれる 年[とし]ごろです。	
\\	私の初恋は小学生の時です。	初恋	私[わたし]の 初恋[はつこい]は 小学生[しょうがくせい]の 時[とき]です。	
\\	先月、娘が誕生しました。	誕生	先月[せんげつ]、 娘[むすめ]が 誕生[たんじょう]しました。	
\\	この部屋はほこりだらけですね。	ほこり	この 部屋[へや]はほこりだらけですね。	
\\	工事の予定が1ヶ月延びてしまった。	延びる	工事[こうじ]の 予定[よてい]が1 ヶ月[かげつ] 延[の]びてしまった。	
\\	出発を一週間延ばしたの。	延ばす	出発[しゅっぱつ]を 一週間[いっしゅうかん] 延[の]ばしたの。	
\\	司会者は話を引き延ばしたわ。	引き延ばす	司会者[しかいしゃ]は 話[はなし]を 引[ひ]き 延[の]ばしたわ。	
\\	彼は単純な人です。	単純	彼[かれ]は 単純[たんじゅん]な 人[ひと]です。	
\\	うちの子はその本に夢中です。	夢中	うちの 子[こ]はその 本[ほん]に 夢中[むちゅう]です。	
\\	彼女は泣き顔になったの。	泣き顔	彼女[かのじょ]は 泣[な]き 顔[がお]になったの。	
\\	お湯はポットに入っています。	ポット	お 湯[ゆ]はポットに 入[はい]っています。	
\\	あの家は笑いが絶えないね。	笑い	あの 家[いえ]は 笑[わら]いが 絶[た]えないね。	
\\	人々は喜びに沸いた。	喜び	人々[ひとびと]は 喜[よろこ]びに 沸[わ]いた。	
\\	私は人を喜ばすのが大好きです。	喜ばす	私[わたし]は 人[ひと]を 喜[よろこ]ばすのが 大好[だいす]きです。	
\\	間違えることは恥ではありません。	恥	間違[まちが]えることは 恥[はじ]ではありません。	
\\	弁論大会で優勝したことがあります。	弁論	弁論[べんろん] 大会[たいかい]で 優勝[ゆうしょう]したことがあります。	
\\	彼はソウルミュージックをよく聞くの。	ミュージック	彼[かれ]はソウルミュージックをよく 聞[き]くの。	
\\	みんなで環境を保護しましょう。	保護	みんなで 環境[かんきょう]を 保護[ほご]しましょう。	
\\	友人が私を弁護してくれました。	弁護	友人[ゆうじん]が 私[わたし]を 弁護[べんご]してくれました。	
\\	私はこの子の保護者です。	保護者	私[わたし]はこの 子[こ]の 保護者[ほごしゃ]です。	
\\	彼と私はいとこ同士です。	同士	彼[かれ]と 私[わたし]はいとこ 同士[どうし]です。	
\\	父は弁護士です。	弁護士	父[ちち]は 弁護士[べんごし]です。	
\\	彼女はロマンチックな人です。	ロマンチック	彼女[かのじょ]はロマンチックな 人[ひと]です。	
\\	彼、友達の意見を否定した。	否定	彼[かれ]、 友達[ともだち]の 意見[いけん]を 否定[ひてい]した。	
\\	あの人は日本銀行の総裁です。	総裁	あの 人[ひと]は 日本銀行[にっぽんぎんこう]の 総裁[そうさい]です。	
\\	村の人たちは迷信を信じています。	迷信	村[むら]の 人[ひと]たちは 迷信[めいしん]を 信[しん]じています。	
\\	うちの子が迷子になりました。	迷子	うちの 子[こ]が 迷子[まいご]になりました。	
\\	彼女は料理の腕が素人の領域を超えているわ。	領域	彼女[かのじょ]は 料理[りょうり]の 腕[うで]が 素人[しろうと]の 領域[りょういき]を 超[こ]えているわ。	
\\	ピラミッドには謎が多いんだ。	謎	ピラミッドには 謎[なぞ]が 多[おお]いんだ。	
\\	遠足が楽しみでわくわくしています。	わくわく	遠足[えんそく]が 楽[たの]しみでわくわくしています。	
\\	この本の著者はイギリス人です。	著者	この 本[ほん]の 著者[ちょしゃ]はイギリス 人[じん]です。	
\\	彼の新しい著書が出版されたね。	著書	彼[かれ]の 新[あたら]しい 著書[ちょしょ]が 出版[しゅっぱん]されたね。	
\\	彼らが売っていたのは偽物だ。	偽物	彼[かれ]らが 売[う]っていたのは 偽物[にせもの]だ。	
\\	日本人がノーベル賞を取ったよ。	ノーベル賞	日本人[にほんじん]がノーベル 賞[しょう]を 取[と]ったよ。	
\\	この建物は国の文化財です。	文化財	この 建物[たてもの]は 国[くに]の 文化財[ぶんかざい]です。	
\\	刺身にわさびは欠かせませんね。	わさび	刺身[さしみ]にわさびは 欠[か]かせませんね。	
\\	私はこの銀行に預金しています。	預金	私[わたし]はこの 銀行[ぎんこう]に 預金[よきん]しています。	
\\	彼の説明で納得できました。	納得	彼[かれ]の 説明[せつめい]で 納得[なっとく]できました。	
\\	私は数学が不得意です。	不得意	私[わたし]は 数学[すうがく]が 不得意[ふとくい]です。	
\\	ネットで得する情報を見つけたよ。	得する	ネットで 得[とく]する 情報[じょうほう]を 見[み]つけたよ。	
\\	この車を今買うとお得ですよ。	得	この 車[くるま]を 今買[いま か]うとお 得[とく]ですよ。	
\\	チキンソテーは私の大好物です。	チキン	チキンソテーは 私[わたし]の 大好物[だいこうぶつ]です。	
\\	パチンコで5000円損しました。	損	パチンコで5000 円[えん] 損[そん]しました。	
\\	町は台風で大きな損害を受けたんだ。	損害	町[まち]は 台風[たいふう]で 大[おお]きな 損害[そんがい]を 受[う]けたんだ。	
\\	わざわざ行って損した。	損する	わざわざ 行[い]って 損[そん]した。	
\\	車から燃料がもれていますよ。	燃料	車[くるま]から 燃料[ねんりょう]がもれていますよ。	
\\	古い手紙を燃やしたんだ。	燃やす	古[ふる]い 手紙[てがみ]を 燃[も]やしたんだ。	
\\	海で日焼けしたんだ。	日焼け	海[うみ]で 日焼[ひや]けしたんだ。	
\\	書類はこのファイルにとじてください。	とじる	書類[しょるい]はこのファイルにとじてください。	
\\	彼は焼き肉が大好きです。	焼き肉	彼[かれ]は 焼[や]き 肉[にく]が 大好[だいす]きです。	
\\	今日は夕焼けがきれいです。	夕焼け	今日[きょう]は 夕焼[ゆうや]けがきれいです。	
\\	お祭りで焼きそばを食べたよ。	焼きそば	お 祭[まつ]りで 焼[や]きそばを 食[た]べたよ。	
\\	この木の幹はとても太いよ。	幹	この 木[き]の 幹[みき]はとても 太[ふと]いよ。	
\\	その会社はいろいろな国に投資を分散しているね。	分散	その 会社[かいしゃ]はいろいろな 国[くに]に 投資[とうし]を 分散[ぶんさん]しているね。	
\\	なぜか彼はにやにやしています。	にやにや	なぜか 彼[かれ]はにやにやしています。	
\\	机の上に書類が散らばっている。	散らばる	机[つくえ]の 上[うえ]に 書類[しょるい]が 散[ち]らばっている。	
\\	風で桜の花が散ってるね。	散る	風[かぜ]で 桜[さくら]の 花[はな]が 散[ち]ってるね。	
\\	弟の部屋はいつも散らかっているんだ。	散らかる	弟[おとうと]の 部屋[へや]はいつも 散[ち]らかっているんだ。	
\\	部屋を散らかさないでください。	散らかす	部屋[へや]を 散[ち]らかさないでください。	
\\	5月は田植えの季節です。	田植え	
\\	月[がつ]は 田植[たう]えの 季節[きせつ]です。	
\\	お祝いにばらの花束を贈りました。	ばら	お 祝[いわ]いにばらの 花束[はなたば]を 贈[おく]りました。	
\\	この木の根はとても太いな。	根	この 木[き]の 根[ね]はとても 太[ふと]いな。	
\\	大根は白くて長い野菜です。	大根	大根[だいこん]は 白[しろ]くて 長[なが]い 野菜[やさい]です。	
\\	魚を切った後、まな板を洗ったの。	まな板	魚[さかな]を 切[き]った 後[あと]、まな 板[いた]を 洗[あら]ったの。	
\\	ほうれん草はビタミンが豊富です。	ほうれん草	ほうれん 草[そう]はビタミンが 豊富[ほうふ]です。	
\\	話し言葉と書き言葉は少し違いますね。	話し言葉	話[はな]し 言葉[ことば]と 書[か]き 言葉[ことば]は 少[すこ]し 違[ちが]いますね。	
\\	白いパンティーを買いました。	パンティー	白[しろ]いパンティーを 買[か]いました。	
\\	もみじの葉っぱが赤くなりましたね。	葉っぱ	もみじの 葉[は]っぱが 赤[あか]くなりましたね。	
\\	親が学校に呼び出されたんだ。	呼び出す	親[おや]が 学校[がっこう]に 呼[よ]び 出[だ]されたんだ。	
\\	当店ではお酒を取り扱っておりません。	取り扱う	当店[とうてん]ではお 酒[さけ]を 取[と]り 扱[あつか]っておりません。	
\\	この機械は取り扱いに注意してください。	取り扱い	この 機械[きかい]は 取[と]り 扱[あつか]いに 注意[ちゅうい]してください。	
\\	私たちは同級生です。	同級生	私[わたし]たちは 同級生[どうきゅうせい]です。	
\\	彼は中級レベルの日本語を習っています。	中級	彼[かれ]は 中級[ちゅうきゅう]レベルの 日本語[にほんご]を 習[なら]っています。	
\\	ふわふわの布団に寝たよ。	ふわふわ	ふわふわの 布団[ふとん]に 寝[ね]たよ。	
\\	腹が減って動けない。	腹	腹[はら]が 減[へ]って 動[うご]けない。	
\\	夕食を腹一杯食べた。	腹一杯	夕食[ゆうしょく]を 腹一杯[はらいっぱい] 食[た]べた。	
\\	彼は肺の病気にかかったのよ。	肺	彼[かれ]は 肺[はい]の 病気[びょうき]にかかったのよ。	
\\	暑さでアイスクリームが溶けてしまった。	溶ける	暑[あつ]さでアイスクリームが 溶[と]けてしまった。	
\\	春の太陽が雪を溶かしました。	溶かす	春[はる]の 太陽[たいよう]が 雪[ゆき]を 溶[と]かしました。	
\\	今日中に返答しなければまずい。	まずい	今日中[きょうじゅう]に 返答[へんとう]しなければまずい。	
\\	彼はその問題を容易に解決したわ。	容易	彼[かれ]はその 問題[もんだい]を 容易[ようい]に 解決[かいけつ]したわ。	
\\	容器のふたはきちんと閉めましょう。	容器	容器[ようき]のふたはきちんと 閉[し]めましょう。	
\\	この生地は滑らかな手触りが特長です。	滑らか	この 生地[きじ]は 滑[なめ]らかな 手触[てざわ]りが 特長[とくちょう]です。	
\\	今の経済の動きを分析しています。	分析	今[いま]の 経済[けいざい]の 動[うご]きを 分析[ぶんせき]しています。	
\\	保健室で少し休んだわ。	保健	保健[ほけん] 室[しつ]で 少[すこ]し 休[やす]んだわ。	
\\	肩をもんでください。	もむ	肩[かた]をもんでください。	
\\	今日、医者に診てもらいました。	診る	今日[きょう]、 医者[いしゃ]に 診[み]てもらいました。	
\\	今、歯を治療しています。	治療	今[いま]、 歯[は]を 治療[ちりょう]しています。	
\\	飲み過ぎは体に毒ですよ。	毒	飲[の]み 過[す]ぎは 体[からだ]に 毒[どく]ですよ。	
\\	そのホテルで食中毒が発生したんだ。	中毒	そのホテルで 食[しょく] 中毒[ちゅうどく]が 発生[はっせい]したんだ。	
\\	その工場は有毒ガスを出しているのね。	有毒	その 工場[こうじょう]は 有毒[ゆうどく]ガスを 出[だ]しているのね。	
\\	傷口にばい菌が入った。	ばい菌	傷口[きずぐち]にばい 菌[きん]が 入[はい]った。	
\\	このセンターで工場全体を管理しています。	センター	このセンターで 工場全体[こうじょう ぜんたい]を 管理[かんり]しています。	
\\	体を不潔にしているとかゆくなりますよ。	不潔	体[からだ]を 不潔[ふけつ]にしているとかゆくなりますよ。	
\\	私は雑誌の編集の仕事をしています。	編集	私[わたし]は 雑誌[ざっし]の 編集[へんしゅう]の 仕事[しごと]をしています。	
\\	彼は国外に追放されました。	追放	彼[かれ]は 国外[こくがい]に 追放[ついほう]されました。	
\\	この地区は放射能に汚染されたんだ。	放射能	この 地区[ちく]は 放射能[ほうしゃのう]に 汚染[おせん]されたんだ。	
\\	公園で犬を放したの。	放す	公園[こうえん]で 犬[いぬ]を 放[はな]したの。	
\\	そんなぞんざいな口のききかたはよくないよ。	ぞんざい	そんなぞんざいな 口[くち]のききかたはよくないよ。	
\\	ボールを空中に放ったの。	放る	ボールを 空中[くうちゅう]に 放[ほう]ったの。	
\\	彼の意見は無視されたよ。	無視	彼[かれ]の 意見[いけん]は 無視[むし]されたよ。	
\\	そのパーティーはカジュアルな服装で大丈夫です。	服装	そのパーティーはカジュアルな 服装[ふくそう]で 大丈夫[だいじょうぶ]です。	
\\	名前に振り仮名をつけてください。	振り仮名	名前[なまえ]に 振[ふ]り 仮名[がな]をつけてください。	
\\	10円玉は銅でできています。	銅	
\\	円玉[えんだま]は 銅[どう]でできています。	
\\	昨日、たんすを買いました。	たんす	昨日[きのう]、たんすを 買[か]いました。	
\\	あの悲劇を繰り返してはいけない。	悲劇	あの 悲劇[ひげき]を 繰[く]り 返[かえ]してはいけない。	
\\	サッカーは団体競技です。	団体	サッカーは 団体[だんたい] 競技[きょうぎ]です。	
\\	私の弟は団地に住んでいます。	団地	私[わたし]の 弟[おとうと]は 団地[だんち]に 住[す]んでいます。	
\\	弟は美術を専攻しています。	美術	弟[おとうと]は 美術[びじゅつ]を 専攻[せんこう]しています。	
\\	彼は数学の博士だそうです。	博士	彼[かれ]は 数学[すうがく]の 博士[はくし]だそうです。	
\\	彼は物理学の博士です。	博士	彼[かれ]は 物理学[ぶつりがく]の 博士[はかせ]です。	
\\	坂道でギアーをチェンジしたよ。	チェンジ	坂道[さかみち]でギアーをチェンジしたよ。	
\\	話の展開についていけない。	展開	話[はなし]の 展開[てんかい]についていけない。	
\\	あの会社は目覚しく発展しているね。	発展	あの 会社[かいしゃ]は 目覚[めざま]しく 発展[はってん]しているね。	
\\	静かな音楽で眠気を催したよ。	催す	静[しず]かな 音楽[おんがく]で 眠気[ねむけ]を 催[もよお]したよ。	
\\	学校の催しに参加しました。	催し	学校[がっこう]の 催[もよお]しに 参加[さんか]しました。	
\\	彼女は典型的なイタリア人です。	典型的	彼女[かのじょ]は 典型的[てんけいてき]なイタリア 人[じん]です。	
\\	しっかりとつかまっていてください。	つかまる	しっかりとつかまっていてください。	
\\	彼は職人の典型です。	典型	彼[かれ]は 職人[しょくにん]の 典型[てんけい]です。	
\\	めずらしい猫について百科事典で調べたの。	百科事典	めずらしい 猫[ねこ]について 百科事典[ひゃっかじてん]で 調[しら]べたの。	
\\	彼は特殊な能力を持っています。	特殊	彼[かれ]は 特殊[とくしゅ]な 能力[のうりょく]を 持[も]っています。	
\\	彼は会社で微妙な立場にあります。	微妙	彼[かれ]は 会社[かいしゃ]で 微妙[びみょう]な 立場[たちば]にあります。	
\\	この商品は免税です。	免税	この 商品[しょうひん]は 免税[めんぜい]です。	
\\	この文のつながりはおかしいです。	つながり	この 文[ぶん]のつながりはおかしいです。	
\\	やっと運転免許を手に入れたよ。	免許	やっと 運転[うんてん] 免許[めんきょ]を 手[て]に 入[い]れたよ。	
\\	「食べる」は動詞です。	動詞	
\\	食[た]べる」は 動詞[どうし]です。	
\\	「学校」は名詞です。	名詞	
\\	学校」[がっこう]は 名詞[めいし]です。	
\\	「ゆっくり歩く」の「ゆっくり」は副詞です。	副詞	「ゆっくり 歩[ある]く」の「ゆっくり」は 副詞[ふくし]です。	
\\	「彼」は代名詞の一つです。	代名詞	
\\	彼」[かれ]は 代名詞[だいめいし]の 一[ひと]つです。	
\\	電話がつながりません。	つながる	電話[でんわ]がつながりません。	
\\	私は日本の伝統を大切にします。	伝統	私[わたし]は 日本[にっぽん]の 伝統[でんとう]を 大切[たいせつ]にします。	
\\	漢字は中国から伝わった。	伝わる	漢字[かんじ]は 中国[ちゅうごく]から 伝[つた]わった。	
\\	彼は数々の伝説を残しました。	伝説	彼[かれ]は 数々[かずかず]の 伝説[でんせつ]を 残[のこ]しました。	
\\	会議の準備に手伝いが必要です。	手伝い	会議[かいぎ]の 準備[じゅんび]に 手伝[てつだ]いが 必要[ひつよう]です。	
\\	課長から伝言があります。	伝言	課長[かちょう]から 伝言[でんごん]があります。	
\\	その国ではコレラの伝染が蔓延しています。	伝染	その 国[くに]ではコレラの 伝染[でんせん]が 蔓延[まんえん]しています。	
\\	つばめが飛んでいますね。	つばめ	つばめが 飛[と]んでいますね。	
\\	焼き鳥はビールに合うな。	焼き鳥	焼[や]き 鳥[とり]はビールに 合[あ]うな。	
\\	お坊さんが鐘を鳴らしていますね。	鳴らす	お 坊[ぼう]さんが 鐘[かね]を 鳴[な]らしていますね。	
\\	外から悲鳴が聞こえたな。	悲鳴	外[そと]から 悲鳴[ひめい]が 聞[き]こえたな。	
\\	部屋の中から笑い声が聞こえた。	笑い声	部屋[へや]の 中[なか]から 笑[わら]い 声[ごえ]が 聞[き]こえた。	
\\	赤ちゃんの泣き声が聞こえますね。	泣き声	赤[あか]ちゃんの 泣[な]き 声[ごえ]が 聞[き]こえますね。	
\\	彼は一人で何かつぶやいていたの。	つぶやく	彼[かれ]は 一人[ひとり]で 何[なに]かつぶやいていたの。	
\\	隣の部屋から話し声が聞こえます。	話し声	隣[となり]の 部屋[へや]から 話[はな]し 声[ごえ]が 聞[き]こえます。	
\\	車はここに駐車してください。	駐車	車[くるま]はここに 駐車[ちゅうしゃ]してください。	
\\	窓から車の騒音が入って来ます。	騒音	窓[まど]から 車[くるま]の 騒音[そうおん]が 入[はい]って 来[き]ます。	
\\	警察がやって来て騒動を静めたんだ。	騒動	警察[けいさつ]がやって 来[き]て 騒動[そうどう]を 静[しず]めたんだ。	
\\	隣の家はいつも騒々しい。	騒々しい	隣[となり]の 家[いえ]はいつも 騒々[そうぞう]しい。	
\\	目をつぶってください。	つぶる	目[め]をつぶってください。	
\\	私たちは名刺を交換しました。	名刺	私[わたし]たちは 名刺[めいし]を 交換[こうかん]しました。	
\\	夜9時に東京に到着しました。	到着	夜9時[よる 
\\	じ]に 東京[とうきょう]に 到着[とうちゃく]しました。	
\\	ついに山頂に到達しました。	到達	ついに 山頂[さんちょう]に 到達[とうたつ]しました。	
\\	会社が倒産しました。	倒産	会社[かいしゃ]が 倒産[とうさん]しました。	
\\	会社で面倒なことが起こったの。	面倒	会社[かいしゃ]で 面倒[めんどう]なことが 起[お]こったの。	
\\	ここに斜めに線を引いてください。	斜め	ここに 斜[なな]めに 線[せん]を 引[ひ]いてください。	
\\	花のつぼみがたくさんついていますね。	つぼみ	花[はな]のつぼみがたくさんついていますね。	
\\	柔らかい日差しが気持ちいいね。	柔らかい	柔[やわ]らかい 日差[ひざ]しが 気持[きも]ちいいね。	
\\	疲れて肩が張っています。	張る	疲[つか]れて 肩[かた]が 張[は]っています。	
\\	母は張り切ってお弁当を用意したの。	張り切る	母[はは]は 張[は]り 切[き]ってお 弁当[べんとう]を 用意[ようい]したの。	
\\	彼女は欲張りです。	欲張り	彼女[かのじょ]は 欲張[よくば]りです。	
\\	彼は突然走り出したのよ。	突然	彼[かれ]は 突然[とつぜん] 走[はし]り 出[だ]したのよ。	
\\	石につまずきました。	つまずく	石[いし]につまずきました。	
\\	ポケットに財布を突っ込んじゃった。	突っ込む	ポケットに 財布[さいふ]を 突[つ]っ 込[こ]んじゃった。	
\\	彼はビリヤードの球を上手に突くね。	突く	彼[かれ]はビリヤードの 球[たま]を 上手[じょうず]に 突[つ]くね。	
\\	バスが乗用車に追突したよ。	追突	バスが 乗用車[じょうようしゃ]に 追突[ついとつ]したよ。	
\\	手と手が触れてどきどきしたよ。	触れる	手[て]と 手[て]が 触[ふ]れてどきどきしたよ。	
\\	危ないので避難してください。	避難	危[あぶ]ないので 避難[ひなん]してください。	
\\	湖にツルがいますよ。	ツル	湖[みずうみ]にツルがいますよ。	
\\	飛行機の墜落事故があったんだ。	墜落	飛行機[ひこうき]の 墜落[ついらく] 事故[じこ]があったんだ。	
\\	その爆撃で兵士が3人負傷しました。	兵士	その 爆撃[ばくげき]で 兵士[へいし]が 3人負傷[さんにん ふしょう]しました。	
\\	あの国は強力な兵器を持っている。	兵器	あの 国[くに]は 強力[きょうりょく]な 兵器[へいき]を 持[も]っている。	
\\	兵隊が銃をかまえていたよ。	兵隊	兵隊[へいたい]が 銃[じゅう]をかまえていたよ。	
\\	彼らは武器を取り、立ち上がった。	武器	彼[かれ]らは 武器[ぶき]を 取[と]り、 立[た]ち 上[あ]がった。	
\\	彼の家柄は武士でした。	武士	彼[かれ]の 家柄[いえがら]は 武士[ぶし]でした。	
\\	この道はでこぼこしていますね。	でこぼこ	この 道[みち]はでこぼこしていますね。	
\\	彼はピストルに弾を込めたんだ。	弾	彼[かれ]はピストルに 弾[たま]を 込[こ]めたんだ。	
\\	このボールはよく弾みますね。	弾む	このボールはよく 弾[はず]みますね。	
\\	日本の国旗は日の丸と呼ばれています。	日の丸	日本[にっぽん]の 国旗[こっき]は 日[ひ]の 丸[まる]と 呼[よ]ばれています。	
\\	今日は月が真ん丸です。	真ん丸	今日[きょう]は 月[つき]が 真[ま]ん 丸[まる]です。	
\\	ダイナマイトが爆発したんだ。	爆発	ダイナマイトが 爆発[ばくはつ]したんだ。	
\\	てのひらに汗をかきました。	てのひら	てのひらに 汗[あせ]をかきました。	
\\	その町に爆弾が落とされたの。	爆弾	その 町[まち]に 爆弾[ばくだん]が 落[お]とされたの。	
\\	昨日株価が暴落しました。	暴落	昨日株価[きのう かぶか]が 暴落[ぼうらく]しました。	
\\	暴力はいけません。	暴力	暴力[ぼうりょく]はいけません。	
\\	人に乱暴してはいけません。	乱暴	人[ひと]に 乱暴[らんぼう]してはいけません。	
\\	今日は絶えず電話が鳴りました。	絶えず	今日[きょう]は 絶[た]えず 電話[でんわ]が 鳴[な]りました。	
\\	この鉛筆はとがっていますね。	とがる	この 鉛筆[えんぴつ]はとがっていますね。	
\\	その文明は何千年も前に滅びてしまったの。	滅びる	その 文明[ぶんめい]は 何千年[なんぜんねん]も 前[まえ]に 滅[ほろ]びてしまったの。	
\\	怪我を防ぐためによくストレッチをしてください。	防ぐ	怪我[けが]を 防[ふせ]ぐためによくストレッチをしてください。	
\\	チャンピオンがタイトルを防衛した。	防衛	チャンピオンがタイトルを 防衛[ぼうえい]した。	
\\	警察は犯罪の防止に努めています。	防止	警察[けいさつ]は 犯罪[はんざい]の 防止[ぼうし]に 努[つと]めています。	
\\	虫歯は予防できます。	予防	虫歯[むしば]は 予防[よぼう]できます。	
\\	昼食にとんかつを食べました。	とんかつ	昼食[ちゅうしょく]にとんかつを 食[た]べました。	
\\	学校で防火訓練がありました。	防火	学校[がっこう]で 防火[ぼうか] 訓練[くんれん]がありました。	
\\	彼に営業を妨害されました。	妨害	彼[かれ]に 営業[えいぎょう]を 妨害[ぼうがい]されました。	
\\	この頃、あの人は不機嫌ですね。	不機嫌	この 頃[ごろ]、あの 人[ひと]は 不機嫌[ふきげん]ですね。	
\\	ゆうべは徹夜しました。	徹夜	ゆうべは 徹夜[てつや]しました。	
\\	コップの底が濡れていますよ。	底	コップの 底[そこ]が 濡[ぬ]れていますよ。	
\\	部屋の中を徹底的に探しました。	徹底的	部屋[へや]の 中[なか]を 徹底的[てっていてき]に 探[さが]しました。	
\\	彼女はどんなに嬉しかったろう。	どんなに	彼女[かのじょ]はどんなに 嬉[うれ]しかったろう。	
\\	今夜中に東京に着くのは到底無理でしょう。	到底	今夜中[こんやじゅう]に 東京[とうきょう]に 着[つ]くのは 到底[とうてい] 無理[むり]でしょう。	
\\	犯人は警察に抵抗したの。	抵抗	犯人[はんにん]は 警察[けいさつ]に 抵抗[ていこう]したの。	
\\	あそこに敵がひそんでいる。	敵	あそこに 敵[てき]がひそんでいる。	
\\	作業を始める態勢は整っています。	態勢	作業[さぎょう]を 始[はじ]める 態勢[たいせい]は 整[ととの]っています。	
\\	勤務の体系を見直すことにしました。	体系	勤務[きんむ]の 体系[たいけい]を 見直[みなお]すことにしました。	
\\	犬に顔をなめられました。	なめる	犬[いぬ]に 顔[かお]をなめられました。	
\\	彼は理系の学生です。	理系	彼[かれ]は 理系[りけい]の 学生[がくせい]です。	
\\	彼女は文系です。	文系	彼女[かのじょ]は 文系[ぶんけい]です。	
\\	私は綿のシャツをよく着ます。	綿	私[わたし]は 綿[めん]のシャツをよく 着[き]ます。	
\\	あの国は経済的に繁栄しているわね。	繁栄	あの 国[くに]は 経済的[けいざいてき]に 繁栄[はんえい]しているわね。	
\\	日韓合同のコンサートが開かれたよ。	日韓	日韓[にっかん] 合同[ごうどう]のコンサートが 開[ひら]かれたよ。	
\\	彼女は私をにらんだの。	にらむ	彼女[かのじょ]は 私[わたし]をにらんだの。	
\\	このビルは大地震に耐えられるでしょうか。	耐える	このビルは 大地震[おおじしん]に 耐[た]えられるでしょうか。	
\\	矢印にそって歩いてください。	矢印	矢印[やじるし]にそって 歩[ある]いてください。	
\\	爪切りを取ってください。	爪切り	爪切[つめき]りを 取[と]ってください。	
\\	枠の中に答えを書いてください。	枠	枠[わく]の 中[なか]に 答[こた]えを 書[か]いてください。	
\\	この棒は何に使うのですか。	棒	この 棒[ぼう]は 何[なに]に 使[つか]うのですか。	
\\	みんな彼の温かい人柄が大好きだよ。	人柄	みんな 彼[かれ]の 温[あたた]かい 人柄[ひとがら]が 大好[だいす]きだよ。	
\\	台所にねずみが出た。	ねずみ	台所[だいどころ]にねずみが 出[で]た。	
\\	かばんに荷物を詰めました。	詰める	かばんに 荷物[にもつ]を 詰[つ]めました。	
\\	プリンタのインクが詰まってしまいました。	詰まる	プリンタのインクが 詰[つ]まってしまいました。	
\\	上司が車を譲ってくれたの。	譲る	上司[じょうし]が 車[くるま]を 譲[ゆず]ってくれたの。	
\\	誠にありがとうございます。	誠に	誠[まこと]にありがとうございます。	
\\	応援団が試合を盛り上げたんだ。	盛り上げる	応援団[おうえんだん]が 試合[しあい]を 盛[も]り 上[あ]げたんだ。	
\\	ドアの隙間から中をのぞいたんだ。	のぞく	ドアの 隙間[すきま]から 中[なか]をのぞいたんだ。	
\\	昨夜のパーティはとても盛り上がったな。	盛り上がる	昨夜[さくや]のパーティはとても 盛[も]り 上[あ]がったな。	
\\	ご飯を盛ってください。	盛る	ご 飯[はん]を 盛[も]ってください。	
\\	母は糸を針に通した。	針	母[はは]は 糸[いと]を 針[はり]に 通[とお]した。	
\\	父は釣りが大好きです。	釣り	父[ちち]は 釣[つ]りが 大好[だいす]きです。	
\\	この紙飛行機は左右の釣り合いがとれていないな。	釣り合い	この 紙飛行機[かみひこうき]は 左右[さゆう]の 釣[つ]り 合[あ]いがとれていないな。	
\\	彼はずいぶんのんきなやつだ。	のんき	彼[かれ]はずいぶんのんきなやつだ。	
\\	収入と支出が釣り合っていない。	釣り合う	収入[しゅうにゅう]と 支出[ししゅつ]が 釣[つ]り 合[あ]っていない。	
\\	今日は頭の回転が鈍いです。	鈍い	今日[きょう]は 頭[あたま]の 回転[かいてん]が 鈍[にぶ]いです。	
\\	あの人は少し鈍感だと思います。	鈍感	あの 人[ひと]は 少[すこ]し 鈍感[どんかん]だと 思[おも]います。	
\\	その村は深い谷にあるの。	谷	その 村[むら]は 深[ふか]い 谷[たに]にあるの。	
\\	出発まで時間の余裕があります。	余裕	出発[しゅっぱつ]まで 時間[じかん]の 余裕[よゆう]があります。	
\\	私は毎日寝る前に入浴します。	入浴	私[わたし]は 毎日寝[まいにち ね]る 前[まえ]に 入浴[にゅうよく]します。	
\\	はえがうるさく飛んでいるね。	はえ	はえがうるさく 飛[と]んでいるね。	
\\	川に沿って歩いたんだ。	沿う	川[かわ]に 沿[そ]って 歩[ある]いたんだ。	
\\	今晩、浜で花火大会がありますよ。	浜	今晩[こんばん]、 浜[はま]で 花火大会[はなび たいかい]がありますよ。	
\\	浜辺できれいな貝がらを拾いました。	浜辺	浜辺[はまべ]できれいな 貝[かい]がらを 拾[ひろ]いました。	
\\	東京湾は重要な役割を果たしている。	湾	東京[とうきょう] 湾[わん]は 重要[じゅうよう]な 役割[やくわり]を 果[は]たしている。	
\\	彼は長い間海に潜っていたの。	潜る	彼[かれ]は 長[なが]い 間海[あいだ うみ]に 潜[もぐ]っていたの。	
\\	ばかばかしい映画だが面白かった。	ばかばかしい	ばかばかしい 映画[えいが]だが 面白[おもしろ]かった。	
\\	彼は彼女に冷淡な態度をとったね。	冷淡	彼[かれ]は 彼女[かのじょ]に 冷淡[れいたん]な 態度[たいど]をとったね。	
\\	靴が泥だらけになったよ。	泥	靴[くつ]が 泥[どろ]だらけになったよ。	
\\	その池の水は濁っているね。	濁る	その 池[いけ]の 水[みず]は 濁[にご]っているね。	
\\	湯飲みにはお茶を入れてお茶碗にはご飯をよそいます。	湯飲み	湯飲[ゆの]みにはお 茶[ちゃ]を 入[い]れてお 茶碗[ちゃわん]にはご 飯[はん]をよそいます。	
\\	やかんから湯気が出ていますよ。	湯気	やかんから 湯気[ゆげ]が 出[で]ていますよ。	
\\	ケーキを作る時ははかりを使います。	はかり	ケーキを 作[つく]る 時[とき]ははかりを 使[つか]います。	
\\	彼のヘッドフォンから音が漏れているね。	漏れる	彼[かれ]のヘッドフォンから 音[おと]が 漏[も]れているね。	
\\	秘密を漏らしたのは彼です。	漏らす	秘密[ひみつ]を 漏[も]らしたのは 彼[かれ]です。	
\\	天井から雨が漏るの。	漏る	天井[てんじょう]から 雨[あめ]が 漏[も]るの。	
\\	携帯電話を水で濡らしてしまったんだ。	濡らす	携帯電話[けいたい でんわ]を 水[みず]で 濡[ぬ]らしてしまったんだ。	
\\	車が溝にはまったの。	溝	車[くるま]が 溝[みぞ]にはまったの。	
\\	はしごを使って屋根に上りました。	はしご	はしごを 使[つか]って 屋根[やね]に 上[のぼ]りました。	
\\	彼は乱暴な言葉を吐いたぞ。	吐く	彼[かれ]は 乱暴[らんぼう]な 言葉[ことば]を 吐[は]いたぞ。	
\\	嘆いていても何も変わりません。	嘆く	嘆[なげ]いていても 何[なに]も 変[か]わりません。	
\\	虫歯は予防できます。	虫歯	虫歯[むしば]は 予防[よぼう]できます。	
\\	庭で秋の虫が鳴いているね。	虫	庭[にわ]で 秋[あき]の 虫[むし]が 鳴[な]いているね。	
\\	犬は飼い主に忠実です。	忠実	犬[いぬ]は 飼[か]い 主[ぬし]に 忠実[ちゅうじつ]です。	
\\	先生からの忠告を聞くべきだよ。	忠告	先生[せんせい]からの 忠告[ちゅうこく]を 聞[き]くべきだよ。	
\\	砂浜をはだしでかけ回ったんだ。	はだし	砂浜[すなはま]をはだしでかけ 回[まわ]ったんだ。	
\\	当日は天気に恵まれました。	恵まれる	当日[とうじつ]は 天気[てんき]に 恵[めぐ]まれました。	
\\	子供には道徳をきちんと教えていきたいと思います。	道徳	子供[こども]には 道徳[どうとく]をきちんと 教[おし]えていきたいと 思[おも]います。	
\\	この円は直径6センチです。	直径	この 円[えん]は 直径[ちょっけい]6センチです。	
\\	この円の半径は5センチです。	半径	この 円[えん]の 半径[はんけい]は5センチです。	
\\	彼はご飯を丼で食べたよ。	丼	彼[かれ]はご 飯[はん]を 丼[どんぶり]で 食[た]べたよ。	
\\	バツが三つあるので85点です。	バツ	バツが 三[みっ]つあるので85 点[てん]です。	
\\	今、豆を煮ています。	豆	今[いま]、 豆[まめ]を 煮[に]ています。	
\\	その子は大粒の涙を浮かべていたの。	粒	その 子[こ]は 大[おお] 粒[つぶ]の 涙[なみだ]を 浮[う]かべていたの。	
\\	この畑では麦を作っています。	麦	この 畑[はたけ]では 麦[むぎ]を 作[つく]っています。	
\\	この村には田畑がたくさんあります。	田畑	この 村[むら]には 田畑[たはた]がたくさんあります。	
\\	ストーブの灯油がなくなったわよ。	灯油	ストーブの 灯油[とうゆ]がなくなったわよ。	
\\	彼は車にはねられたけど無事だったの。	はねる	彼[かれ]は 車[くるま]にはねられたけど 無事[ぶじ]だったの。	
\\	午後5時には電灯がつきます。	電灯	午後5時[ごご 
\\	じ]には 電灯[でんとう]がつきます。	
\\	ろうそくの炎が部屋を照らしたんだ。	炎	ろうそくの 炎[ほのお]が 部屋[へや]を 照[て]らしたんだ。	
\\	彼は皮肉ばかり言います。	皮肉	彼[かれ]は 皮肉[ひにく]ばかり 言[い]います。	
\\	冬は皮膚が乾燥しますね。	皮膚	冬[ふゆ]は 皮膚[ひふ]が 乾燥[かんそう]しますね。	
\\	ここはいやな臭いがする。	臭い	ここはいやな 臭[にお]いがする。	
\\	まな板が生臭いです。	生臭い	まな 板[いた]が 生臭[なまぐさ]いです。	
\\	今朝は目がはれています。	はれる	今朝[けさ]は 目[め]がはれています。	
\\	流しが臭うのできれいにしたよ。	臭う	流[なが]しが 臭[にお]うのできれいにしたよ。	
\\	この計算は面倒臭いなあ。	面倒臭い	この 計算[けいさん]は 面倒臭[めんどうくさ]いなあ。	
\\	バラの花が甘く匂っているね。	匂う	バラの 花[はな]が 甘[あま]く 匂[にお]っているね。	
\\	私はまだ両親と同居しています。	同居	私[わたし]はまだ 両親[りょうしん]と 同居[どうきょ]しています。	
\\	ここに穴を掘りましょう。	掘る	ここに 穴[あな]を 掘[ほ]りましょう。	
\\	車にひかれないよう気をつけなさい。	ひく	車[くるま]にひかれないよう 気[き]をつけなさい。	
\\	猫が塀の上で寝ているぞ。	塀	猫[ねこ]が 塀[へい]の 上[うえ]で 寝[ね]ているぞ。	
\\	彼は大層喜んでいました。	大層	彼[かれ]は 大層[たいそう] 喜[よろこ]んでいました。	
\\	最近使ったファイルは、履歴からすぐ開けます。	履歴	最近使[さいきん つか]ったファイルは、 履歴[りれき]からすぐ 開[ひら]けます。	
\\	面接のために履歴書を書きました。	履歴書	面接[めんせつ]のために 履歴書[りれきしょ]を 書[か]きました。	
\\	履物は靴箱に入れてください。	履物	履物[はきもの]は 靴箱[くつばこ]に 入[い]れてください。	
\\	何かヒントをください。	ヒント	何[なに]かヒントをください。	
\\	この皿を戸棚にしまってください。	戸棚	この 皿[さら]を 戸棚[とだな]にしまってください。	
\\	彼は扉を開けたんだ。	扉	彼[かれ]は 扉[とびら]を 開[あ]けたんだ。	
\\	このペンは鳥の羽根で作られています。	羽根	このペンは 鳥[とり]の 羽根[はね]で 作[つく]られています。	
\\	翌日、彼に会いに行きました。	翌日	翌日[よくじつ]、 彼[かれ]に 会[あ]いに 行[い]きました。	
\\	翌朝は快晴だったね。	翌朝	翌朝[よくあさ]は 快晴[かいせい]だったね。	
\\	彼女は大学卒業の翌年に結婚したの。	翌年	彼女[かのじょ]は 大学卒業[だいがく そつぎょう]の 翌年[よくねん]に 結婚[けっこん]したの。	
\\	詳細はファックスで送ります。	ファックス	詳細[しょうさい]はファックスで 送[おく]ります。	
\\	湖に鳥の群れがいたよ。	群れ	湖[みずうみ]に 鳥[とり]の 群[む]れがいたよ。	
\\	彼は思わず友人を殴ったの。	殴る	彼[かれ]は 思[おも]わず 友人[ゆうじん]を 殴[なぐ]ったの。	
\\	会議の要旨は次の通りです。	要旨	会議[かいぎ]の 要旨[ようし]は 次[つぎ]の 通[とお]りです。	
\\	彼女は肌がとても白いですね。	肌	彼女[かのじょ]は 肌[はだ]がとても 白[しろ]いですね。	
\\	寒いので暖かい肌着を着ました。	肌着	寒[さむ]いので 暖[あたた]かい 肌着[はだぎ]を 着[き]ました。	
\\	彼は山のふもとに住んでいるよ。	ふもと	彼[かれ]は 山[やま]のふもとに 住[す]んでいるよ。	
\\	彼女は肌色のシャツを着ています。	肌色	彼女[かのじょ]は 肌色[はだいろ]のシャツを 着[き]ています。	
\\	彼は腸の手術を受けたの。	腸	彼[かれ]は 腸[ちょう]の 手術[しゅじゅつ]を 受[う]けたの。	
\\	彼女はずいぶん大胆なことを言うね。	大胆	彼女[かのじょ]はずいぶん 大胆[だいたん]なことを 言[い]うね。	
\\	ステージの幕が上がったんだ。	幕	ステージの 幕[まく]が 上[あ]がったんだ。	
\\	夕暮れの空がきれいですね。	夕暮れ	夕暮[ゆうぐ]れの 空[そら]がきれいですね。	
\\	おいしそうなフルーツゼリーだね。	フルーツ	おいしそうなフルーツゼリーだね。	
\\	祖父の墓は近くにあります。	墓	祖父[そふ]の 墓[はか]は 近[ちか]くにあります。	
\\	寺のとなりに墓地があります。	墓地	寺[てら]のとなりに 墓地[ぼち]があります。	
\\	明日は家族で墓参りに行きます。	墓参り	明日[あした]は 家族[かぞく]で 墓参[はかまい]りに 行[い]きます。	
\\	チューリップの芽が出ました。	芽	チューリップの 芽[め]が 出[で]ました。	
\\	彼の葬式は明日です。	葬式	彼[かれ]の 葬式[そうしき]は 明日[あした]です。	
\\	カエルにはへそがないんだ。	へそ	カエルにはへそがないんだ。	
\\	彼女は礼儀が身に付いています。	礼儀	彼女[かのじょ]は 礼儀[れいぎ]が 身[み]に 付[つ]いています。	
\\	彼はもう彼女を憎んではいないよ。	憎む	彼[かれ]はもう 彼女[かのじょ]を 憎[にく]んではいないよ。	
\\	彼の心は憎しみに満ちていたんだ。	憎しみ	彼[かれ]の 心[こころ]は 憎[にく]しみに 満[み]ちていたんだ。	
\\	妹は時々憎らしいことを言うんだ。	憎らしい	妹[いもうと]は 時々[ときどき] 憎[にく]らしいことを 言[い]うんだ。	
\\	彼が憎いですか。	憎い	彼[かれ]が 憎[にく]いですか。	
\\	仏の教えを勉強しました。	仏	仏[ほとけ]の 教[おし]えを 勉強[べんきょう]しました。	
\\	カップルがべたべたしているね。	べたべた	カップルがべたべたしているね。	
\\	お葬式は仏教で行うことが多いです。	仏教	お 葬式[そうしき]は 仏教[ぶっきょう]で 行[おこな]うことが 多[おお]いです。	
\\	私の友人はお坊さんをしているの。	坊さん	私[わたし]の 友人[ゆうじん]はお 坊[ぼう]さんをしているの。	
\\	この笛は竹でできている。	竹	この 笛[ふえ]は 竹[たけ]でできている。	
\\	筆者の趣旨を考えてください。	筆者	筆者[ひっしゃ]の 趣旨[しゅし]を 考[かんが]えてください。	
\\	その会社の筆記試験は難しかったよ。	筆記試験	その 会社[かいしゃ]の 筆記試験[ひっきしけん]は 難[むずか]しかったよ。	
\\	ほうきで庭を掃除しました。	ほうき	ほうきで 庭[にわ]を 掃除[そうじ]しました。	
\\	今日は筆記用具を忘れました。	筆記用具	今日[きょう]は 筆記用具[ひっきようぐ]を 忘[わす]れました。	
\\	彼は筆で手紙を書いたの。	筆	彼[かれ]は 筆[ふで]で 手紙[てがみ]を 書[か]いたの。	
\\	彼は笛を吹くのが上手いね。	笛	彼[かれ]は 笛[ふえ]を 吹[ふ]くのが 上手[うま]いね。	
\\	これが参加者の名簿です。	名簿	これが 参加者[さんかしゃ]の 名簿[めいぼ]です。	
\\	会社が二つの派閥に分裂しているんだ。	分裂	会社[かいしゃ]が 二[ふた]つの 派閥[はばつ]に 分裂[ぶんれつ]しているんだ。	
\\	電柱にポスターが貼ってあった。	ポスター	電柱[でんちゅう]にポスターが 貼[は]ってあった。	
\\	この布はカーテンに使えます。	布	この 布[ぬの]はカーテンに 使[つか]えます。	
\\	包丁で指を切った。	包丁	包丁[ほうちょう]で 指[ゆび]を 切[き]った。	
\\	彼女は手に包帯を巻いていたんだ。	包帯	彼女[かのじょ]は 手[て]に 包帯[ほうたい]を 巻[ま]いていたんだ。	
\\	この包みは誰のですか。	包み	この 包[つつ]みは 誰[だれ]のですか。	
\\	包み紙はたたんでください。	包み紙	包[つつ]み 紙[がみ]はたたんでください。	
\\	祖母がうちの猫を抱いているわよ。	抱く	祖母[そぼ]がうちの 猫[ねこ]を 抱[だ]いているわよ。	
\\	靴のひもをほどいたの。	ほどく	靴[くつ]のひもをほどいたの。	
\\	彼女はいつも文句ばかり言う。	文句	彼女[かのじょ]はいつも 文句[もんく]ばかり 言[い]う。	
\\	来月中旬にフランスへ行きます。	中旬	来月[らいげつ] 中旬[ちゅうじゅん]にフランスへ 行[い]きます。	
\\	彼はごく平凡な人です。	平凡	彼[かれ]はごく 平凡[へいぼん]な 人[ひと]です。	
\\	銀行の通帳を持ってきてください。	通帳	銀行[ぎんこう]の 通帳[つうちょう]を 持[も]ってきてください。	
\\	彼女が妊娠したそうです。	妊娠	彼女[かのじょ]が 妊娠[にんしん]したそうです。	
\\	彼女は真っ赤なほほをしているね。	ほほ	彼女[かのじょ]は 真[ま]っ 赤[か]なほほをしているね。	
\\	日本では夫婦同姓が一般的よ。	同姓	日本[にっぽん]では 夫婦[ふうふ] 同姓[どうせい]が 一般的[いっぱんてき]よ。	
\\	姉が嫁に行ったの。	嫁	姉[あね]が 嫁[よめ]に 行[い]ったの。	
\\	花嫁が父親と一緒に入ってきましたね。	花嫁	花嫁[はなよめ]が 父親[ちちおや]と 一緒[いっしょ]に 入[はい]ってきましたね。	
\\	花婿と花嫁が並んで登場したんだ。	花婿	花婿[はなむこ]と 花嫁[はなよめ]が 並[なら]んで 登場[とうじょう]したんだ。	
\\	幼児は入場無料です。	幼児	幼児[ようじ]は 入場無料[にゅうじょう むりょう]です。	
\\	彼女は私にほほえんだよ。	ほほえむ	彼女[かのじょ]は 私[わたし]にほほえんだよ。	
\\	まだ8時なのに、もう眠たいです。	眠たい	まだ8 時[じ]なのに、もう 眠[ねむ]たいです。	
\\	彼女はいつもより早く眠りに着いたよ。	眠り	彼女[かのじょ]はいつもより 早[はや]く 眠[ねむ]りに 着[つ]いたよ。	
\\	猫が窓から外を眺めているよ。	眺める	猫[ねこ]が 窓[まど]から 外[そと]を 眺[なが]めているよ。	
\\	ここからの眺めは最高です。	眺め	ここからの 眺[なが]めは 最高[さいこう]です。	
\\	手紙に写真が同封されていたよ。	同封	手紙[てがみ]に 写真[しゃしん]が 同封[どうふう]されていたよ。	
\\	手紙に封をしたよ。	封	手紙[てがみ]に 封[ふう]をしたよ。	
\\	庭に花の種をまきました。	まく	庭[にわ]に 花[はな]の 種[たね]をまきました。	
\\	友人達に協力を呼び掛けたの。	呼び掛ける	友人達[ゆうじんたち]に 協力[きょうりょく]を 呼[よ]び 掛[か]けたの。	
\\	魚の骨がのどに引っ掛かった。	引っ掛かる	魚[さかな]の 骨[ほね]がのどに 引[ひ]っ 掛[か]かった。	
\\	知らない人が話し掛けてきた。	話し掛ける	知[し]らない 人[ひと]が 話[はな]し 掛[か]けてきた。	
\\	その店を通り掛かったらバーゲンをやっていた。	通り掛かる	その 店[みせ]を 通[とお]り 掛[か]かったらバーゲンをやっていた。	
\\	その選手はハードルに足を引っ掛けたんだ。	引っ掛ける	その 選手[せんしゅ]はハードルに 足[あし]を 引[ひ]っ 掛[か]けたんだ。	
\\	論文のまとめを書いています。	まとめ	論文[ろんぶん]のまとめを 書[か]いています。	
\\	彼はフェンスに寄り掛かったの。	寄り掛かる	彼[かれ]はフェンスに 寄[よ]り 掛[か]かったの。	
\\	被災者を救うための特別な措置が取られました。	措置	被災者[ひさいしゃ]を 救[すく]うための 特別[とくべつ]な 措置[そち]が 取[と]られました。	
\\	あなたの著書を拝見しました。	拝見	あなたの 著書[ちょしょ]を 拝見[はいけん]しました。	
\\	最近甘いものを控えています。	控える	最近[さいきん] 甘[あま]いものを 控[ひか]えています。	
\\	ここはお客様用の控え室です。	控え室	ここはお 客様用[きゃくさま よう]の 控[ひか]え 室[しつ]です。	
\\	娘はよく私のまねをします。	まね	娘[むすめ]はよく 私[わたし]のまねをします。	
\\	彼は両手を握り締めたの。	握り締める	彼[かれ]は 両手[りょうて]を 握[にぎ]り 締[し]めたの。	
\\	朝は家族全員が揃って食事します。	揃う	朝[あさ]は 家族全員[かぞく ぜんいん]が 揃[そろ]って 食事[しょくじ]します。	
\\	花は長さを揃えて花瓶にさしましょう。	揃える	花[はな]は 長[なが]さを 揃[そろ]えて 花瓶[かびん]にさしましょう。	
\\	空港へ彼女を出迎えに行きました。	出迎え	空港[くうこう]へ 彼女[かのじょ]を 出迎[でむか]えに 行[い]きました。	
\\	駅まで迎えに来て下さい。	迎え	駅[えき]まで 迎[むか]えに 来[き]て 下[くだ]さい。	
\\	私の動きをまねてください。	まねる	私[わたし]の 動[うご]きをまねてください。	
\\	この会社は急成長を遂げました。	遂げる	この 会社[かいしゃ]は 急成長[きゅうせいちょう]を 遂[と]げました。	
\\	明日から温泉を巡る旅に出ます。	巡る	明日[あした]から 温泉[おんせん]を 巡[めぐ]る 旅[たび]に 出[で]ます。	
\\	社員たちは待遇の改善を要求している。	待遇	社員[しゃいん]たちは 待遇[たいぐう]の 改善[かいぜん]を 要求[ようきゅう]している。	
\\	「さくら」という言葉を使って俳句を書いたよ。	俳句	「さくら」という 言葉[ことば]を 使[つか]って 俳句[はいく]を 書[か]いたよ。	
\\	彼は俳優です。	俳優	彼[かれ]は 俳優[はいゆう]です。	
\\	彼女は流行に敏感だね。	敏感	彼女[かのじょ]は 流行[りゅうこう]に 敏感[びんかん]だね。	
\\	眠くてまぶたが重くなってきた。	まぶた	眠[ねむ]くてまぶたが 重[おも]くなってきた。	
\\	この街はロンドンに倣って作られました。	倣う	この 街[まち]はロンドンに 倣[なら]って 作[つく]られました。	
\\	彼は会社の寮に住んでいます。	寮	彼[かれ]は 会社[かいしゃ]の 寮[りょう]に 住[す]んでいます。	
\\	彼は世の中の偏見と戦ったんだ。	偏見	彼[かれ]は 世[よ]の 中[なか]の 偏見[へんけん]と 戦[たたか]ったんだ。	
\\	この映画は何遍も見ました。	何遍	この 映画[えいが]は 何遍[なんべん]も 見[み]ました。	
\\	その商品の宣伝をよく見かけます。	宣伝	その 商品[しょうひん]の 宣伝[せんでん]をよく 見[み]かけます。	
\\	彼が仕事を休むのはまれです。	まれ	彼[かれ]が 仕事[しごと]を 休[やす]むのはまれです。	
\\	昨日、絵の展覧会に行ってきました。	展覧会	昨日[きのう]、 絵[え]の 展覧会[てんらんかい]に 行[い]ってきました。	
\\	国民の6割が総理大臣を支持しています。	総理大臣	国民[こくみん]の6 割[わり]が 総理大臣[そうりだいじん]を 支持[しじ]しています。	
\\	大臣が汚職で逮捕されました。	大臣	大臣[だいじん]が 汚職[おしょく]で 逮捕[たいほ]されました。	
\\	選手は最高の状態で大会に臨んだわ。	臨む	選手[せんしゅ]は 最高[さいこう]の 状態[じょうたい]で 大会[たいかい]に 臨[のぞ]んだわ。	
\\	彼女は臨時の店員です。	臨時	彼女[かのじょ]は 臨時[りんじ]の 店員[てんいん]です。	
\\	資料にミスプリントがありました。	ミスプリント	資料[しりょう]にミスプリントがありました。	
\\	新しい内閣が誕生したね。	内閣	新[あたら]しい 内閣[ないかく]が 誕生[たんじょう]したね。	
\\	彼は病気と闘ったの。	闘う	彼[かれ]は 病気[びょうき]と 闘[たたか]ったの。	
\\	あの大学には派閥がたくさんあります。	派閥	あの 大学[だいがく]には 派閥[はばつ]がたくさんあります。	
\\	ここは私にとって懐かしい場所です。	懐かしい	ここは 私[わたし]にとって 懐[なつ]かしい 場所[ばしょ]です。	
\\	島で火山が噴火したよ。	噴火	島[しま]で 火山[かざん]が 噴火[ふんか]したよ。	
\\	公園の噴水の前で会いましょう。	噴水	公園[こうえん]の 噴水[ふんすい]の 前[まえ]で 会[あ]いましょう。	
\\	人参の皮をむきました。	むく	人参[にんじん]の 皮[かわ]をむきました。	
\\	彼の話を聞いて不愉快になった。	不愉快	彼[かれ]の 話[はなし]を 聞[き]いて 不愉快[ふゆかい]になった。	
\\	彼らはとても愉快な人たちです。	愉快	彼[かれ]らはとても 愉快[ゆかい]な 人[ひと]たちです。	
\\	妹は漫画が好きです。	漫画	妹[いもうと]は 漫画[まんが]が 好[す]きです。	
\\	この電卓はポケットに入ります。	電卓	この 電卓[でんたく]はポケットに 入[はい]ります。	
\\	雄大な景色に感動したの。	雄大	雄大[ゆうだい]な 景色[けしき]に 感動[かんどう]したの。	
\\	彼はページをめくったの。	めくる	彼[かれ]はページをめくったの。	
\\	この猫は雌です。	雌	この 猫[ねこ]は 雌[めす]です。	
\\	その考えは少し幼稚だと思う。	幼稚	その 考[かんが]えは 少[すこ]し 幼稚[ようち]だと 思[おも]う。	
\\	みんなで製品の名称を考えました。	名称	みんなで 製品[せいひん]の 名称[めいしょう]を 考[かんが]えました。	
\\	その国の社会秩序は乱れているわ。	秩序	その 国[くに]の 社会[しゃかい] 秩序[ちつじょ]は 乱[みだ]れているわ。	
\\	私たちは物陰に隠れたんだ。	物陰	私[わたし]たちは 物陰[ものかげ]に 隠[かく]れたんだ。	
\\	それはめでたいことだ。	めでたい	それはめでたいことだ。	
\\	彼女の明るさに魅力を感じました。	魅力	彼女[かのじょ]の 明[あか]るさに 魅力[みりょく]を 感[かん]じました。	
\\	醜い争いはやめましょう。	醜い	醜[みにく]い 争[あらそ]いはやめましょう。	
\\	彼は悪賢い男だな。	悪賢い	彼[かれ]は 悪賢[わるがしこ]い 男[おとこ]だな。	
\\	あと少しで山の頂上です。	頂上	あと 少[すこ]しで 山[やま]の 頂上[ちょうじょう]です。	
\\	近頃、彼女の様子がおかしい。	近頃	近頃[ちかごろ]、 彼女[かのじょ]の 様子[ようす]がおかしい。	
\\	彼の言っていることは矛盾しています。	矛盾	彼[かれ]の 言[い]っていることは 矛盾[むじゅん]しています。	
\\	彼の商売はもうかっているらしいね。	もうかる	彼[かれ]の 商売[しょうばい]はもうかっているらしいね。	
\\	この荷物を倉庫に運びましょう。	倉庫	この 荷物[にもつ]を 倉庫[そうこ]に 運[はこ]びましょう。	
\\	この学校は1962年に創立されました。	創立	この 学校[がっこう]は1962 年[ねん]に 創立[そうりつ]されました。	
\\	罰金を3万円もとられたよ。	罰金	罰金[ばっきん]を3 万円[まんえん]もとられたよ。	
\\	彼は違法駐車で罰せられたんだ。	罰する	彼[かれ]は 違法駐車[いほう ちゅうしゃ]で 罰[ばっ]せられたんだ。	
\\	彼は悪いことをしたので罰を受けたんだ。	罰	彼[かれ]は 悪[わる]いことをしたので 罰[ばつ]を 受[う]けたんだ。	
\\	今日のもうけは2万円でした。	もうけ	今日[きょう]のもうけは2 万円[まんえん]でした。	
\\	子供が老人の手を引いて歩いていたの。	老人	子供[こども]が 老人[ろうじん]の 手[て]を 引[ひ]いて 歩[ある]いていたの。	
\\	今、豆を煮ています。	煮る	今[いま]、 豆[まめ]を 煮[に]ています。	
\\	もうすぐじゃがいもが煮えます。	煮える	もうすぐじゃがいもが 煮[に]えます。	
\\	今日は蒸し暑いね。	蒸し暑い	今日[きょう]は 蒸[む]し 暑[あつ]いね。	
\\	彼はいつも黙って仕事をします。	黙る	彼[かれ]はいつも 黙[だま]って 仕事[しごと]をします。	
\\	もしかしたら来年転勤になるかも知れない。	もしかしたら	もしかしたら 来年転勤[らいねん てんきん]になるかも 知[し]れない。	
\\	それは時間の無駄です。	無駄	それは 時間[じかん]の 無駄[むだ]です。	
\\	これは税金の無駄遣いだね。	無駄遣い	これは 税金[ぜいきん]の 無駄遣[むだづか]いだね。	
\\	水が沸騰したよ。	沸騰	水[みず]が 沸騰[ふっとう]したよ。	
\\	図書館に本を返却したよ。	返却	図書館[としょかん]に 本[ほん]を 返却[へんきゃく]したよ。	
\\	彼は県庁で働いています。	庁	彼[かれ]は 県[けん] 庁[ちょう]で 働[はたら]いています。	
\\	彼は非常にもてるね。	もてる	彼[かれ]は 非常[ひじょう]にもてるね。	
\\	あの高い建物が都庁です。	都庁	あの 高[たか]い 建物[たてもの]が 都庁[とちょう]です。	
\\	彼は府庁で働いています。	府庁	彼[かれ]は 府庁[ふちょう]で 働[はたら]いています。	
\\	その二つの国に摩擦が生じたんだ。	摩擦	その 二[ふた]つの 国[くに]に 摩擦[まさつ]が 生[しょう]じたんだ。	
\\	日本人は先輩、後輩の関係をとても大切に考えます。	先輩	日本人[にっぽんじん]は 先輩[せんぱい]、 後輩[こうはい]の 関係[かんけい]をとても 大切[たいせつ]に 考[かんが]えます。	
\\	昔の子供たちは冒険小説をよく読みました。	冒険	昔[むかし]の 子供[こども]たちは 冒険[ぼうけん] 小説[しょうせつ]をよく 読[よ]みました。	
\\	叔父は牧場を持っているんだ。	牧場	叔父[おじ]は 牧場[ぼくじょう]を 持[も]っているんだ。	
\\	今、野菜をゆでています。	ゆでる	今[いま]、 野菜[やさい]をゆでています。	
\\	取り敢えずお知らせしておきます。	取り敢えず	取[と]り 敢[あ]えずお 知[し]らせしておきます。	
\\	彼は勇気があるね。	勇気	彼[かれ]は 勇気[ゆうき]があるね。	
\\	彼女はとても勇敢でした。	勇敢	彼女[かのじょ]はとても 勇敢[ゆうかん]でした。	
\\	父が息子を励ましたの。	励ます	父[ちち]が 息子[むすこ]を 励[はげ]ましたの。	
\\	花に露がついていました。	露	花[はな]に 露[つゆ]がついていました。	
\\	彼は飛んできたボールをよけたの。	よける	彼[かれ]は 飛[と]んできたボールをよけたの。	
\\	今日は零下の寒さでしたね。	零下	今日[きょう]は 零下[れいか]の 寒[さむ]さでしたね。	
\\	とても雰囲気のいいお店ですね。	雰囲気	とても 雰囲気[ふんいき]のいいお 店[みせ]ですね。	
\\	盆地は夏、とても暑いんだ。	盆地	盆地[ぼんち]は 夏[なつ]、とても 暑[あつ]いんだ。	
\\	お盆は実家に帰りました。	盆	お 盆[ぼん]は 実家[じっか]に 帰[かえ]りました。	
\\	母が料理をお盆に乗せたの。	盆	母[はは]が 料理[りょうり]をお 盆[ぼん]に 乗[の]せたの。	
\\	父が長い手紙をよこしたよ。	よこす	父[ちち]が 長[なが]い 手紙[てがみ]をよこしたよ。	
\\	能は元々大衆の文化でした。	大衆	能[のう]は 元々[もともと] 大衆[たいしゅう]の 文化[ぶんか]でした。	
\\	その物語は京都が舞台だ。	舞台	その 物語[ものがたり]は 京都[きょうと]が 舞台[ぶたい]だ。	
\\	昨日友人を見舞ったの。	見舞う	昨日友人[きのう ゆうじん]を 見舞[みま]ったの。	
\\	友達が見舞いに来てくれたよ。	見舞い	友達[ともだち]が 見舞[みま]いに 来[き]てくれたよ。	
\\	みんなで盆踊りに行きました。	盆踊り	みんなで 盆踊[ぼんおど]りに 行[い]きました。	
\\	ウサギが雪の上を跳ねていったよ。	跳ねる	ウサギが 雪[ゆき]の 上[うえ]を 跳[は]ねていったよ。	
\\	人をからかうのはよしなさい。	よす	人[ひと]をからかうのはよしなさい。	
\\	銭湯は昔より少なくなったわね。	銭湯	銭湯[せんとう]は 昔[むかし]より 少[すく]なくなったわね。	
\\	いつも古里を懐かしく思い出すの。	古里	いつも 古里[ふるさと]を 懐[なつ]かしく 思[おも]い 出[だ]すの。	
\\	来年は優勝を狙います。	狙う	来年[らいねん]は 優勝[ゆうしょう]を 狙[ねら]います。	
\\	彼は木の像を彫ったの。	彫る	彼[かれ]は 木[き]の 像[ぞう]を 彫[ほ]ったの。	
\\	その公園は町のほぼ中央に位置するんだ。	中央	その 公園[こうえん]は 町[まち]のほぼ 中央[ちゅうおう]に 位置[いち]するんだ。	
\\	町内会のレクリエーションに参加したの。	レクリエーション	町内会[ちょうないかい]のレクリエーションに 参加[さんか]したの。	
\\	友達に双子が生まれたよ。	双子	友達[ともだち]に 双子[ふたご]が 生[う]まれたよ。	
\\	彼らは裸のつきあいをしているよ。	裸	彼[かれ]らは 裸[はだか]のつきあいをしているよ。	
\\	犬も猫も祖先は同じ動物らしいよ。	祖先	犬[いぬ]も 猫[ねこ]も 祖先[そせん]は 同[おな]じ 動物[どうぶつ]らしいよ。	
\\	その仕事の報酬として20万円もらいました。	報酬	その 仕事[しごと]の 報酬[ほうしゅう]として20 万円[まんえん]もらいました。	
\\	子供たちが無邪気に遊んでいるね。	無邪気	子供[こども]たちが 無邪気[むじゃき]に 遊[あそ]んでいるね。	
\\	バースデーケーキにろうそくを立てました。	ろうそく	バースデーケーキにろうそくを 立[た]てました。	
\\	私は玉ねぎが嫌いです。	玉ねぎ	私[わたし]は 玉[たま]ねぎが 嫌[きら]いです。	
\\	彼の顔に玉のような汗が流れていたの。	玉	彼[かれ]の 顔[かお]に 玉[たま]のような 汗[あせ]が 流[なが]れていたの。	
\\	日本には天皇がいます。	天皇	日本[にっぽん]には 天皇[てんのう]がいます。	
\\	私は風呂場で歯を磨きます。	風呂場	私[わたし]は 風呂場[ふろば]で 歯[は]を 磨[みが]きます。	
\\	班のメンバーは5人です。	班	班[はん]のメンバーは5 人[にん]です。	
\\	どうぞ召し上がってください。	召し上がる	どうぞ 召[め]し 上[あ]がってください。	
\\	子供のわがままを叱ったの。	わがまま	子供[こども]のわがままを 叱[しか]ったの。	
\\	哲学は興味深い学問です。	哲学	哲学[てつがく]は 興味深[きょうみぶか]い 学問[がくもん]です。	
\\	この綱は直径20
\\	あるそうです。	綱	この 綱[つな]は 直径20
\\	[ちょっけい 
\\	せんち]あるそうです。	
\\	枝を縄でしばったよ。	縄	枝[えだ]を 縄[なわ]でしばったよ。	
\\	彼女は子供のスカートを縫ったんだ。	縫う	彼女[かのじょ]は 子供[こども]のスカートを 縫[ぬ]ったんだ。	
\\	雨が上がって美しい虹が出たね。	虹	雨[あめ]が 上[あ]がって 美[うつく]しい 虹[にじ]が 出[で]たね。	
\\	友達に励まされて勇気がわいたよ。	わく	友達[ともだち]に 励[はげ]まされて 勇気[ゆうき]がわいたよ。	
\\	梅雨明けは来週だそうです。	梅雨明け	梅雨明[つゆあ]けは 来週[らいしゅう]だそうです。	
\\	梅雨入りは6月17日でした。	梅雨入り	梅雨入[つゆい]りは6 月17日[がつ 
\\	にち]でした。	
\\	お名前は上の欄にお書きください。	欄	お 名前[なまえ]は 上[うえ]の 欄[らん]にお 書[か]きください。	
\\	栓抜きはどこですか。	栓抜き	栓抜[せんぬ]きはどこですか。	
\\	この峠を越えると村があります。	峠	この 峠[とうげ]を 越[こ]えると 村[むら]があります。	
\\	小さなことにくよくよするのはばからしいよ。	ばからしい	小[ちい]さなことにくよくよするのはばからしいよ。	
\\	彼女は元気なお婆さんね。	婆さん	彼女[かのじょ]は 元気[げんき]なお 婆[ばあ]さんね。	
\\	魚を味噌に漬けました。	漬ける	魚[さかな]を 味噌[みそ]に 漬[つ]けました。	
\\	祖母は漬け物を自分で作ります。	漬け物	祖母[そぼ]は 漬[つ]け 物[もの]を 自分[じぶん]で 作[つく]ります。	
\\	ペットボトルは潰して捨てましょう。	潰す	ペットボトルは 潰[つぶ]して 捨[す]てましょう。	
\\	箱の角が潰れていますよ。	潰れる	箱[はこ]の 角[かど]が 潰[つぶ]れていますよ。	
\\	今日はなぜか体がだるい。	なぜか	今日[きょう]はなぜか 体[からだ]がだるい。	
\\	年賀葉書が売り出されたね。	年賀	年賀[ねんが] 葉書[はがき]が 売[う]り 出[だ]されたね。	
\\	ここは正月はたくさんの人で賑わいます。	賑わう	ここは 正月[しょうがつ]はたくさんの 人[ひと]で 賑[にぎ]わいます。	
\\	蛇がカエルを捕まえたの。	蛇	蛇[へび]がカエルを 捕[つか]まえたの。	
\\	彼女は華やかな女性ですね。	華やか	彼女[かのじょ]は 華[はな]やかな 女性[じょせい]ですね。	
\\	ドアに指を挟んだ。	挟む	ドアに 指[ゆび]を 挟[はさ]んだ。	
\\	あの塔の高さはどれくらいですか。	塔	あの 塔[とう]の 高[たか]さはどれくらいですか。	
\\	空がにわかに暗くなったな。	にわかに	空[そら]がにわかに 暗[くら]くなったな。	
\\	二人は一生を共にすることを誓ったの。	誓う	二人[ふたり]は 一生[いっしょう]を 共[とも]にすることを 誓[ちか]ったの。	
\\	友人が慰めてくれました。	慰める	友人[ゆうじん]が 慰[なぐさ]めてくれました。	
\\	彼はとても朗らかな人です。	朗らか	彼[かれ]はとても 朗[ほが]らかな 人[ひと]です。	
\\	見て、きれいな蝶蝶が飛んでいる。	蝶蝶	見[み]て、きれいな 蝶蝶[ちょうちょう]が 飛[と]んでいる。	
\\	公園の鳩にえさをやったの。	鳩	公園[こうえん]の 鳩[はと]にえさをやったの。	
\\	風がぴたりと止んだね。	ぴたりと	風[かぜ]がぴたりと 止[や]んだね。	
\\	3月3日は雛祭りです。	雛祭	
\\	月[がつ]3 日[か]は 雛祭[ひなまつ]りです。	
\\	夕方、物凄い雨が降ったね。	物凄い	夕方[ゆうがた]、 物凄[ものすご]い 雨[あめ]が 降[ふ]ったね。	
\\	風邪をひいて喉が痛いです。	喉	風邪[かぜ]をひいて 喉[のど]が 痛[いた]いです。	
\\	彼は地面に唾を吐いたよ。	唾	彼[かれ]は 地面[じめん]に 唾[つば]を 吐[は]いたよ。	
\\	どこかで犬が吠えていますね。	吠える	どこかで 犬[いぬ]が 吠[ほ]えていますね。	
\\	子供が水たまりをぴょんと飛びこえたね。	ぴょんと	子供[こども]が 水[みず]たまりをぴょんと 飛[と]びこえたね。	
\\	木にブランコを吊るしました。	吊るす	木[き]にブランコを 吊[つ]るしました。	
\\	きゅうりに味噌をつけて食べたんだ。	味噌	きゅうりに 味噌[みそ]をつけて 食[た]べたんだ。	
\\	傷口はもう塞がりました。	塞がる	傷口[きずぐち]はもう 塞[ふさ]がりました。	
\\	あまりにうるさかったので耳を塞いだわ。	塞ぐ	あまりにうるさかったので 耳[みみ]を 塞[ふさ]いだわ。	
\\	猫の頭を撫でてあげた。	撫でる	猫[ねこ]の 頭[あたま]を 撫[な]でてあげた。	
\\	彼らは卑怯な方法で勝利したの。	卑怯	彼[かれ]らは 卑怯[ひきょう]な 方法[ほうほう]で 勝利[しょうり]したの。	
\\	桜の花びらがひらりと舞い落ちたな。	ひらりと	桜[さくら]の 花[はな]びらがひらりと 舞[ま]い 落[お]ちたな。	
\\	母は溜め息をついたの。	溜め息	母[はは]は 溜[た]め 息[いき]をついたの。	
\\	彼はストレスが溜まっているの。	溜まる	彼[かれ]はストレスが 溜[た]まっているの。	
\\	お風呂に水を溜めておいてください。	溜める	お 風呂[ふろ]に 水[みず]を 溜[た]めておいてください。	
\\	旅館の朝食にのりが出ました。	海苔	旅館[りょかん]の 朝食[ちょうしょく]にのりが 出[で]ました。	
\\	遥か向こうに目的地が見えてきた。	遥か	遥[はる]か 向[む]こうに 目的地[もくてきち]が 見[み]えてきた。	
\\	彼はもうじき父親になります。	もうじき	彼[かれ]はもうじき 父親[ちちおや]になります。	
\\	自転車の乗り過ぎで股が痛いです。	股	自転車[じてんしゃ]の 乗[の]り 過[す]ぎで 股[また]が 痛[いた]いです。	
\\	薪を集めて火を焚いたよ。	焚く	薪[まき]を 集[あつ]めて 火[ひ]を 焚[た]いたよ。	
\\	彼は話し方が明瞭ね。	明瞭	彼[かれ]は 話[はな]し 方[かた]が 明瞭[めいりょう]ね。	
\\	彼は眉が濃いね。	眉	彼[かれ]は 眉[まゆ]が 濃[こ]いね。	
\\	彼女の眉毛は太いな。	眉毛	彼女[かのじょ]の 眉毛[まゆげ]は 太[ふと]いな。	
\\	彼はやたらに話しかけてきた。	やたらに	彼[かれ]はやたらに 話[はな]しかけてきた。	
\\	彼女は割り箸を使わず、自分の箸を使うの。	割り箸	彼女[かのじょ]は 割[わ]り 箸[ばし]を 使[つか]わず、 自分[じぶん]の 箸[はし]を 使[つか]うの。	
\\	蛋白質は大切な栄養です。	蛋白質	蛋白質[たんぱくしつ]は 大切[たいせつ]な 栄養[えいよう]です。	
\\	友達が蜂に刺されたんだ。	蜂	友達[ともだち]が 蜂[はち]に 刺[さ]されたんだ。	
\\	それ、ひとつ頂戴。	頂戴	それ、ひとつ 頂戴[ちょうだい]。	
\\	近くの神社に初詣でに行きました。	初詣で	近[ちか]くの 神社[じんじゃ]に 初詣[はつもう]でに 行[い]きました。	
\\	彼は彼女に心から詫びたんだ。	詫びる	彼[かれ]は 彼女[かのじょ]に 心[こころ]から 詫[わ]びたんだ。	
\\	彼女はほおを赤く染めたわ。	頬	彼女[かのじょ]はほおを 赤[あか]く 染[そ]めたわ。	
\\	正月には餅を食べますよ。	餅	正月[しょうがつ]には 餅[もち]を 食[た]べますよ。	
\\	人を騙してはいけません。	騙す	人[ひと]を 騙[だま]してはいけません。	
\\	友達に手紙を書こうと便箋を買ったんだ。	便箋	友達[ともだち]に 手紙[てがみ]を 書[か]こうと 便箋[びんせん]を 買[か]ったんだ。	
\end{CJK}
\end{document}