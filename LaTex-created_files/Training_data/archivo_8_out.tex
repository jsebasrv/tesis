\documentclass[8pt]{extreport} 
\usepackage{hyperref}
\usepackage{CJKutf8}
\begin{document}
\begin{CJK}{UTF8}{min}
\\	10歳のときに母は亡くなったんです。	
\\	際[さい]のときに 母[はは]は 亡[な]くなったんです。	
\\	血液検査をしてもらいたいんです。	
\\	血液[けつえき] 検査[けんさ]をしてもらいたいんです。	
\\	何を食べたらいいですか。	
\\	何[なに]を 食[た]べたらいいですか。	
\\	どこに座ったらいいですか。	
\\	どこに 座[すわ]ったらいいですか。	
\\	誰に知らせたらいいですか。	
\\	誰[だれ]に 知[し]らせたらいいですか。	
\\	あなた自身のご経験から、実例をいくつか挙げていただけませんか。	
\\	あなた 自身[じしん]のご 経験[けいけん]から、 実例[じつれい]をいくつか 挙[あ]げていただけませんか。	
\\	少し時間をいただけないでしょうか。	
\\	少[すこ]し 時間[じかん]をいただけないでしょうか。	
\\	彼は病気で来られません。	
\\	彼[かれ]は 病気[びょうき]で 来[こ]られません。	
\\	塩は目方で売られる。	
\\	塩[しお]は 目方[めかた]で 売[う]られる。	
\\	彼女は男性に見られるのが好きだ。	
\\	彼女[かのじょ]は 男性[だんせい]に 見[み]られるのが 好[す]きだ。	
\\	人生はしばしば旅に例えられる。	
\\	人生[じんせい]はしばしば 旅[たび]に 例[たと]えられる。	
\\	今しかない。	
\\	今[いま]しかない。	
\\	彼は役に立ちそうなものしか買わない。	
\\	彼[かれ]は 役に立[やくにた]ちそうなものしか 買[か]わない。	
\\	それは遺憾ながら本当だ。	
\\	それは 遺憾[いかん]ながら 本当[ほんとう]だ。	
\\	彼女は食べながら話を続けた。	
\\	彼女[かのじょ]は 食[た]べながら 話[はなし]を 続[つづ]けた。	
\\	頭痛はするし、せきでも苦しんでいます。	
\\	頭痛[ずつう]はするし、せきでも 苦[くる]しんでいます。	
\\	その辞書は役に立つし、おまけに高くない。	
\\	その 辞書[じしょ]は 役に立[やくにた]つし、おまけに 高[たか]くない。	
\\	私はその町に不案内だった、それにその言葉も一言も話せなかった。	
\\	私[わたし]はその 町[まち]に 不案内[ふあんない]だった、それにその 言葉[ことば]も 一言[ひとこと]も 話[はな]せなかった。	
\\	ラベルには一回2錠と書いてある。	
\\	ラベルには 一回[いっかい]2 錠[じょう]と 書[か]いてある。	
\\	5人用の席を予約してあります。	
\\	人[にん] 用[よう]の 席[せき]を 予約[よやく]してあります。	
\\	あなたの時計は明日までには直しておきますよ。	
\\	あなたの 時計[とけい]は 明日[あした]までには 直[なお]しておきますよ。	
\\	私は、山へ行こうと思っています。	
\\	私[わたし]は、 山[やま]へ 行[い]こうと 思[おも]っています。	
\\	彼は技師になりたく思っています。	
\\	彼[かれ]は 技師[ぎし]になりたく 思[おも]っています。	
\\	禁煙するつもりだ。	
\\	禁煙[きんえん]するつもりだ。	
\\	私達は警察を呼んだほうがいい。	
\\	私[わたし] 達[たち]は 警察[けいさつ]を 呼[よ]んだほうがいい。	
\\	そのことについては気にしないほうがいい。	
\\	そのことについては 気[き]にしないほうがいい。	
\\	明日は恐らく雨が降るでしょう。	
\\	明日[あした]は 恐[おそ]らく 雨[あめ]が 降[ふ]るでしょう。	
\\	手紙は一週間かそこらで着くでしょう。	
\\	手紙[てがみ]は 一週間[いっしゅうかん]かそこらで 着[つ]くでしょう。	
\\	彼らはその列車に乗り遅れたかもしれない。	
\\	彼[かれ]らはその 列車[れっしゃ]に 乗り遅[のりおく]れたかもしれない。	
\\	彼女はその質問に答えることができるかもしれない。	
\\	彼女[かのじょ]はその 質問[しつもん]に 答[こた]えることができるかもしれない。	
\\	もしかしたら、またすぐ日本へ行くかもしれません。	
\\	もしかしたら、またすぐ 日本[にほん]へ 行[い]くかもしれません。	
\\	もしかしたら彼は気が変わるかもしれない。	
\\	もしかしたら 彼[かれ]は 気[き]が 変[か]わるかもしれない。	
\\	早くビールを持って来い、のどが乾いているんだ。	
\\	早[はや]くビールを 持[も]って 来[こ]い、のどが 乾[かわ]いているんだ。	
\\	明日家へ来い。	
\\	明日[あした] 家[うち]へ 来[こ]い。	
\\	エレベーターを使うな。	
\\	エレベーターを 使[つか]うな。	
\\	医者の言うとおりにしなさい。	
\\	医者[いしゃ]の 言[い]うとおりにしなさい。	
\\	私よりうまい言い方をしてくれましたね。	
\\	私[わたし]よりうまい 言い方[いいかた]をしてくれましたね。	
\\	私は、しょうゆ、みそよりも、とんかつラーメンが好きです。	
\\	私[わたし]は、しょうゆ、みそよりも、とんかつラーメンが 好[す]きです。	
\\	この車は思っていたよりもはるかに良い。	
\\	この 車[くるま]は 思[おも]っていたよりもはるかに 良[よ]い。	
\\	彼を待つより仕方がない。	
\\	彼[かれ]を 待[ま]つより 仕方[しかた]がない。	
\\	まだ始まったばかりです。	
\\	まだ 始[はじ]まったばかりです。	
\\	この党は2年前に作られたばかりだ。	
\\	この 党[とう]は2 年[ねん] 前[まえ]に 作[つく]られたばかりだ。	
\\	自信を持てるようになるために努力することが大切だと私は思います。	
\\	自信[じしん]を 持[も]てるようになるために 努力[どりょく]することが 大切[たいせつ]だと 私[わたし]は 思[おも]います。	
\\	5年前、初めての海外旅行で韓国へ行ったのがきっかけでこの国にとても興味を持つようになった。	
\\	年[ねん] 前[まえ]、 初[はじ]めての 海外[かいがい] 旅行[りょこう]で 韓国[かんこく]へ 行[い]ったのがきっかけでこの 国[くに]にとても 興味[きょうみ]を 持[も]つようになった。	
\\	この部屋では喫煙してはいけないことになっている。	
\\	この 部屋[へや]では 喫煙[きつえん]してはいけないことになっている。	
\\	すべてのイスラム教徒は、金曜日正午の礼拝のためにモスクに行くことになっている。	
\\	すべてのイスラム 教徒[きょうと]は、 金曜日[きんようび] 正午[しょうご]の 礼拝[れいはい]のためにモスクに 行[い]くことになっている。	正午=しょうご= 
\\	礼拝=れいはい= 
\\	その会議は2年ごとに開催されることになっている。	
\\	その 会議[かいぎ]は2 年[ねん]ごとに 開催[かいさい]されることになっている。	
\\	この放送により、別の翻訳本も米国で発売することになった。	
\\	この 放送[ほうそう]により、 別[べつ]の 翻訳[ほんやく] 本[ほん]も 米国[べいこく]で 発売[はつばい]することになった。	
\\	雑誌とかパンフレットとかいろんなもの、君、片付けてほしい。	
\\	雑誌[ざっし]とかパンフレットとかいろんなもの、 君[きみ]、 片付[かたづ]けてほしい。	
\\	なんとかかんとかしてこの仕事を1月間で終えなくてはならない。	
\\	なんとかかんとかしてこの 仕事[しごと]を 一月間[いちがつかん]で 終[お]えなくてはならない。	
\\	あなたと一緒にいると自分の一番いいところが引き出されるような気がする。	
\\	あなたと 一緒[いっしょ]にいると 自分[じぶん]の 一番[いちばん]いいところが 引き出[ひきだ]されるような 気[き]がする。	
\\	あなたのこと、ずっと前から知っているような気がするの。	
\\	あなたのこと、ずっと 前[まえ]から 知[し]っているような 気[き]がするの。	
\\	あなたはいつも頼りそうな気がする。	
\\	あなたはいつも 頼[たよ]りそうな 気[き]がする。	
\\	およそ70年前まで、絶滅したと考えられていた。	
\\	およそ70 年[ねん] 前[まえ]まで、 絶滅[ぜつめつ]したと 考[かんが]えられていた。	
\\	ちょっとボケていた。	
\\	ちょっとボケていた。	
\\	それにはコツがあるんだよ。	
\\	それにはコツがあるんだよ。	
\\	何かコツはありますか。	
\\	何[なに]かコツはありますか。	
\\	これは私には初めてです。	
\\	これは 私[わたし]には 初[はじ]めてです。	
\\	こんな嬉しい気持ちは初めてです。	
\\	こんな 嬉[うれ]しい 気持[きも]ちは 初[はじ]めてです。	
\\	1年間同棲したあと、私達は結婚することにした。	
\\	年間[ねんかん] 同棲[どうせい]したあと、 私[わたし] 達[たち]は 結婚[けっこん]することにした。	
\\	たくさんあって、とても食べ切れません。	
\\	たくさんあって、とても 食[た]べ 切[き]れません。	
\\	15年ぶりに息子に会った時、彼女は感動でいっぱいになった。	
\\	年[ねん]ぶりに 息子[むすこ]に 会[あ]った 時[とき]、 彼女[かのじょ]は 感動[かんどう]でいっぱいになった。	
\\	私は、感情を表すことができればいいのに。	
\\	私[わたし]は、 感情[かんじょう]を 表[あらわ]すことができればいいのに。	
\\	つまり、二千人に一人は
\\	に感染しているんです。	
\\	つまり、 二千[にせん] 人[にん]に 一人[ひとり]は 
\\	に 感染[かんせん]しているんです。	
\\	つまり、あなたは合計75ドルを払う。	
\\	つまり、あなたは 合計[ごうけい]75ドルを 払[はら]う。	
\\	お正月休みの間は主婦も少しゆっくりできるというわけです。	
\\	お 正月[しょうがつ] 休[やす]みの 間[ま]は 主婦[しゅふ]も 少[すこ]しゆっくりできるというわけです。	
\\	同じブランド名でも生産国によって品質が違うことがある。	
\\	同[おな]じブランド 名[めい]でも 生産[せいさん] 国[こく]によって 品質[ひんしつ]が 違[ちが]うことがある。	
\\	一ドルは、何円に当たりますか。	
\\	一ドルは、 何[なん] 円[えん]に 当[あ]たりますか。	
\\	「どうも」のような便利なフレーズは、英語にはないだろう。	
\\	「どうも」のような 便利[べんり]なフレーズは、 英語[えいご]にはないだろう。	
\\	基本的なものから始めましょう。	
\\	基本[きほん] 的[てき]なものから 始[はじ]めましょう。	
\\	彼らには罪の意識がない。	
\\	彼[かれ]らには 罪[つみ]の 意識[いしき]がない。	
\\	彼はポケットから中身を出した。	
\\	彼[かれ]はポケットから 中身[なかみ]を 出[だ]した。	
\\	この主張には根拠がありません。	
\\	この 主張[しゅちょう]には 根拠[こんきょ]がありません。	
\\	彼は自分の体験を述べた。	
\\	彼[かれ]は 自分[じぶん]の 体験[たいけん]を 述[の]べた。	
\\	彼はその話題について述べた。	
\\	彼[かれ]はその 話題[わだい]について 述[の]べた。	
\\	私は彼の招待に応じた。	
\\	私[わたし]は 彼[かれ]の 招待[しょうたい]に 応[おう]じた。	
\\	働きに応じて支払われます。	
\\	働[はたら]きに 応[おう]じて 支払[しはら]われます。	
\\	彼は我々の求めに応じて歌った。	
\\	彼[かれ]は 我々[われわれ]の 求[もと]めに 応[おう]じて 歌[うた]った。	
\\	その国は産業によって支えられている。	
\\	その 国[くに]は 産業[さんぎょう]によって 支[ささ]えられている。	
\\	その話をすればするほど腹が立つ。	
\\	その 話[はなし]をすればするほど 腹[はら]が 立[た]つ。	
\\	それがあの男の欠点ですが、だいたいにおいて働き者です。	
\\	それがあの 男[おとこ]の 欠点[けってん]ですが、だいたいにおいて 働き者[はたらきもの]です。	
\\	家に帰る時間は大体、午前2時か3時頃です。	
\\	家[いえ]に 帰[かえ]る 時間[じかん]は 大体[だいたい]、 午前[ごぜん]2 時[じ]か3 時[じ] 頃[ごろ]です。	
\\	のどが乾いた感じです。	
\\	のどが 乾[かわ]いた 感[かん]じです。	
\\	私は英語も日本語もうまく話せますから、彼女と同じ仕事ができると思ったのです。	
\\	私[わたし]は 英語[えいご]も 日本語[にほんご]もうまく 話[はな]せますから、 彼女[かのじょ]と 同[おな]じ 仕事[しごと]ができると 思[おも]ったのです。	
\\	3月から4月にかけて桜が満開になる頃、日本中の人々が花見に出掛けます。	
\\	3月[さんがつ]から 4月[しがつ]にかけて 桜[さくら]が 満開[まんかい]になる 頃[ころ]、 日本 中[ちゅう]の 人々が 花見[はなみ]に 出掛[でか]けます。	
\\	その水は飲むのに不適当だ。	
\\	その 水[みず]は 飲[の]むのに 不[ふ] 適当[てきとう]だ。	
\\	その農家は日の出とともに起き、日没まで働いた。	
\\	その 農家[のうか]は 日の出[ひので]とともに 起[お]き、 日没[にちぼつ]まで 働[はたら]いた。	
\\	主人とともに世界の国々を旅したことによって、私の視野が広がりました。	
\\	主人[しゅじん]とともに 世界[せかい]の 国々[くにぐに]を 旅[たび]したことによって、 私[わたし]の 視野[しや]が 広[ひろ]がりました。	
\\	彼にとって3回目のオリンピック出場だ。	
\\	彼[かれ]にとって3 回[かい] 目[め]のオリンピック 出場[しゅつじょう]だ。	
\\	ほかのアジアの国に比べて、繰り返しこの国を訪問する観光客が一番多いのも事実です。	
\\	ほかのアジアの 国[くに]に 比[くら]べて、 繰り返[くりかえ]しこの 国[くに]を 訪問[ほうもん]する 観光[かんこう] 客[きゃく]が一番 多[おお]いのも 事実[じじつ]です。	
\\	マンションを借りるのは簡単だが、買うのに比べて割高だ。	
\\	マンションを 借[か]りるのは 簡単[かんたん]だが、 買[か]うのに 比[くら]べて 割高[わりだか]だ。	割高=わりだか= 
\\	ただ、アメリカと比べて、女性が差別されていると感じた。	
\\	ただ、アメリカと 比[くら]べて、 女性[じょせい]が 差別[さべつ]されていると 感[かん]じた。	
\\	我々は経験によって学ぶ。	
\\	我々[われわれ]は 経験[けいけん]によって 学[まな]ぶ。	
\\	紙は中国人によって発明された。	
\\	紙[かみ]は 中国人[ちゅうごくじん]によって 発明[はつめい]された。	
\\	需要が増すに連れて、値段が上がる。	
\\	需要[じゅよう]が 増[ま]すに 連[つ]れて、 値段[ねだん]が 上[あ]がる。	
\\	時間が経つにつれて我々の希望は消えた。	
\\	時間[じかん]が 経[た]つにつれて 我々[われわれ]の 希望[きぼう]は 消[き]えた。	
\\	それは単なる冗談だよ。	
\\	それは 単[たん]なる 冗談[じょうだん]だよ。	
\\	私はそれが単なる偶然だと思う。	
\\	私[わたし]はそれが 単[たん]なる 偶然[ぐうぜん]だと 思[おも]う。	
\\	彼はしばしば、怠けていると私を非難する。	
\\	彼[かれ]はしばしば、 怠[なま]けていると 私[わたし]を 非難[ひなん]する。	
\\	やっぱり、世界は狭くなった。	
\\	やっぱり、 世界[せかい]は 狭[せま]くなった。	
\\	やっぱりここにいると思いましたよ。	
\\	やっぱりここにいると 思[おも]いましたよ。	
\\	私たちの家は快適ですが、やっぱり前の家が懐かしい。	
\\	私[わたし]たちの 家[いえ]は 快適[かいてき]ですが、やっぱり 前[まえ]の 家[いえ]が 懐[なつ]かしい。	
\\	やはり生活をするためには働かなければなりません。	
\\	やはり 生活[せいかつ]をするためには 働[はたら]かなければなりません。	
\\	双方の親同士にも違いがあります。	
\\	双方[そうほう]の 親[おや] 同士[どうし]にも 違[ちが]いがあります。	双方=そうほう= 
\\	グローバル化が進む今日、国際語として英語の重要性は高まっています。	
\\	グローバル 化[か]が 進[すす]む 今日[こんにち]、 国際[こくさい] 語[ご]として 英語[えいご]の 重要[じゅうよう] 性[せい]は 高[たか]まっています。	
\\	日本では少子化が進む一方、ペットの数は急増している。	
\\	日本[にっぽん]では 少子化[しょうしか]が 進[すす]む 一方[いっぽう]、ペットの 数[かず]は 急増[きゅうぞう]している。	
\\	その男には同情心といった人間的感情はなかった。	
\\	その 男[おとこ]には 同情[どうじょう] 心[しん]といった 人間[にんげん] 的[てき] 感情[かんじょう]はなかった。	
\\	「泣いた」「感動した!」といったたくさんの手紙が作者に届いた。	
\\	泣[な]いた」
\\	感動[かんどう]した!」といったたくさんの 手紙[てがみ]が 作者[さくしゃ]に 届[とど]いた。	
\\	そこに共通するものは、東洋的考えといえるかもしれない。	
\\	そこに 共通[きょうつう]するものは、 東洋[とうよう] 的[てき] 考[かんが]えといえるかもしれない。	
\\	革新的な日本製品に共通する特徴は、使いよさと性能の良さである。	
\\	革新[かくしん] 的[てき]な 日本[にほん] 製品[せいひん]に 共通[きょうつう]する 特徴[とくちょう]は、 使[つか]いよさと 性能[せいのう]の 良[よ]さである。	
\\	「お手伝いましょうか?」	
\\	「お 手伝[てつだ]いましょうか?」	
\\	完全な中立というものはあり得ない。	
\\	完全[かんぜん]な 中立[ちゅうりつ]というものはあり 得[え]ない。	
\\	お互い平等でなければ、完全な友情などあり得ない。	
\\	お 互[たが]い 平等[びょうどう]でなければ、 完全[かんぜん]な 友情[ゆうじょう]などあり 得[え]ない。	
\\	そうはいい考えだけど、とりあえず、使い捨てカメラを買おうよ。	
\\	そうはいい 考[かんが]えだけど、とりあえず、 使い捨[つかいす]てカメラを 買[か]おうよ。	
\\	家具の作り方をご存じなんですか。	
\\	家具[かぐ]の 作り方[つくりかた]をご 存[ぞん]じなんですか。	
\\	際立った鼻が特徴です。	
\\	際立[きわだ]った 鼻[はな]が 特徴[とくちょう]です。	
\\	彼女はできるだけお金を貯めようとしている。	
\\	彼女[かのじょ]はできるだけお 金[かね]を 貯[た]めようとしている。	
\\	日本では、まだまだアフリカやアフリカの抱える問題に対する認識が薄いです。	
\\	日本[にほん]では、まだまだアフリカやアフリカの 抱[かか]える 問題[もんだい]に 対[たい]する 認識[にんしき]が 薄[うす]いです。	
\\	このキャンペーンは、地球温暖化についての関心を高めるためのものです。	
\\	このキャンペーンは、 地球[ちきゅう] 温暖[おんだん] 化[か]についての 関心[かんしん]を 高[たか]めるためのものです。	
\\	その行為は彼の評判を悪くした。	
\\	その 行為[こうい]は 彼[かれ]の 評判[ひょうばん]を 悪[わる]くした。	
\\	その山頂に立っていると、気分爽快だった。	
\\	その 山頂[さんちょう]に 立[た]っていると、 気分[きぶん] 爽快[そうかい]だった。	
\\	さっきフェイスブックであなたの写真いいね!したよ。	
\\	さっきフェイスブックであなたの 写真[しゃしん]いいね!したよ。	
\\	さっき君がいっていたことに戻るけど、あの映画は一見価値ありだね。	
\\	さっき 君[きみ]がいっていたことに 戻[もど]るけど、あの 映画[えいが]は 一見[いっけん] 価値[かち]ありだね。	
\\	そのあと、執拗な電話はぴたりとなくなった。	
\\	そのあと、 執拗[しつよう]な 電話[でんわ]はぴたりとなくなった。	
\\	それは必ずしも私が意図していることではない。	
\\	それは 必[かなら]ずしも 私[わたし]が 意図[いと]していることではない。	
\\	ただ、事を収拾するには、少し時間が要る。	
\\	ただ、 事[こと]を 収拾[しゅうしゅう]するには、 少[すこ]し 時間[じかん]が 要[い]る。	
\\	予想以上に、年を取ると骨が弱くなるようである。	
\\	予想[よそう] 以上[いじょう]に、 年[とし]を 取[と]ると 骨[ほね]が 弱[よわ]くなるようである。	
\\	私は部長にお土産をいただきました。	
\\	私[わたし]は 部長[ぶちょう]にお 土産[みやげ]をいただきました。	
\\	自己紹介させていただきます。	
\\	自己[じこ] 紹介[しょうかい]させていただきます。	
\\	早速ですが本題に入らせていただきます。	
\\	早速[さっそく]ですが 本題[ほんだい]に 入[はい]らせていただきます。	
\\	これちょっと持っててくれますか。	
\\	これちょっと 持[も]っててくれますか。	
\\	これにサインしてくれますか。	
\\	これにサインしてくれますか。	
\\	ここで何が起こっているのですか。	
\\	ここで 何[なに]が 起[お]こっているのですか。	
\\	有名人に感想を書いてもらいます。	
\\	有名人[ゆうめいじん]に 感想[かんそう]を 書[か]いてもらいます。	
\\	ガーナでは、大抵、服は仕立ててもらいます。	
\\	ガーナでは、 大抵[たいてい]、 服[ふく]は 仕立[した]ててもらいます。	
\\	このドレスをドライクリーニングしてもらうよ。	
\\	このドレスをドライクリーニングしてもらうよ。	
\\	これを取り換えてくださいますか。	
\\	これを 取り換[とりか]えてくださいますか。	
\\	塩を取ってくださいますか。	
\\	塩[しお]を 取[と]ってくださいますか。	
\\	もちろん、私を支えてくださった方々のおかげです。	
\\	もちろん、 私[わたし]を 支[ささ]えてくださった 方々[かたがた]のおかげです。	
\\	あなたと一緒にお仕事させていただくのを楽しみにしています。	
\\	あなたと 一緒[いっしょ]にお 仕事[しごと]させていただくのを 楽[たの]しみにしています。	
\\	お返しをしていただく必要はありません。	
\\	お 返[かえ]しをしていただく 必要[ひつよう]はありません。	
\\	私のために少し演奏してくださいませんか。	
\\	私[わたし]のために 少[すこ]し 演奏[えんそう]してくださいませんか。	
\\	それをもう少し詳しく説明してくださいませんか。	
\\	それをもう 少[すこ]し 詳[くわ]しく 説明[せつめい]してくださいませんか。	
\\	ベロベロの酔っ払っちゃった。	
\\	ベロベロの 酔っ払[よっぱら]っちゃった。	
\\	寝過ごしちゃった。	
\\	寝過[ねす]ごしちゃった。	
\\	ごめん帰らなきゃ。	
\\	ごめん 帰[かえ]らなきゃ。	
\\	「日本人ばっかりで嫌だ」なんて言いながらも、旅行者の約4割は再訪者だ。	
\\	日本人[にほんじん]ばっかりで 嫌[いや]だ」なんて 言[い]いながらも、 旅行[りょこう] 者[しゃ]の 約[やく]4 割[わり]は 再訪[さいほう] 者[しゃ]だ。	
\\	13は不吉な数字とされている。	
\\	13は 不吉[ふきつ]な 数字[すうじ]とされている。	
\\	「ニャ」は日本で猫の鳴き声とされている。	
\\	「ニャ」は 日本[にっぽん]で 猫[ねこ]の 鳴き声[なきごえ]とされている。	
\\	人は幼少期に、基礎的な社会的スキルを身につける。	
\\	人[ひと]は 幼少[ようしょう] 期[き]に、 基礎[きそ] 的[てき]な 社会[しゃかい] 的[てき]スキルを 身[み]につける。	
\\	勇気は天性のもの。自分で身に付けることはできない。	
\\	勇気[ゆうき]は 天性[てんせい]のもの。 自分[じぶん]で 身[み]に 付[つ]けることはできない。	
\\	3年前に日本へ来てから、私は同じ交番で50回以上呼び止められました。	
\\	年[ねん] 前[まえ]に 日本[にほん]へ 来[き]てから、 私[わたし]は 同[おな]じ 交番[こうばん]で50 回[かい] 以上[いじょう] 呼び止[よびと]められました。	
\\	なんてうれしい驚きでしょう!	
\\	なんてうれしい 驚[おどろ]きでしょう!	
\\	何てケチなんだ。	
\\	何[なに]てケチなんだ。	
\\	は大きなチャンズを逸し、悔しがっているに違いない。	
\\	は 大[おお]きなチャンズを 逸[いっ]し、 悔[くや]しがっているに 違[ちが]いない。	
\\	すぐに残虐行為のじょうほうが外部に漏れ始めました。	
\\	すぐに 残虐[ざんぎゃく] 行為[こうい]のじょうほうが 外部[がいぶ]に 漏[も]れ 始[はじ]めました。	
\\	そのアルバムは、熱狂的な反響を呼んだ。	
\\	そのアルバムは、 熱狂[ねっきょう] 的[てき]な 反響[はんきょう]を 呼[よ]んだ。	
\\	クラスの皆さんに言いたいのは、やるべきことをためらうなということです。	
\\	クラスの 皆[みな]さんに 言[い]いたいのは、やるべきことをためらうなということです。	
\\	彼は一方進みでて、ためらうと、私の手に触れました。	
\\	彼[かれ]は 一方[いっぽう] 進[すす]みでて、ためらうと、 私[わたし]の 手[て]に 触[ふ]れました。	
\\	必死でいつもベストを尽くそうと頑張ってる。	
\\	必死[ひっし]でいつもベストを 尽[つ]くそうと 頑張[がんば]ってる。	
\\	たまたまその国から来ている人と恋に落ちたということでしょう。	
\\	たまたまその 国[くに]から 来[き]ている 人[ひと]と 恋[こい]に 落[お]ちたということでしょう。	
\\	要点はそんなところです。	
\\	要点[ようてん]はそんなところです。	
\\	お年寄りに席を譲るのは当たり前のことだった。	
\\	お 年寄[としよ]りに 席[せき]を 譲[ゆず]るのは 当たり前[あたりまえ]のことだった。	
\\	言い訳するつもりはないけど、今週は本当に忙しかったのです。	
\\	言い訳[いいわけ]するつもりはないけど、 今週[こんしゅう]は 本当[ほんとう]に 忙[いそが]しかったのです。	
\\	そして横になるとすぐ眠りが私をおそった。	
\\	そして 横[よこ]になるとすぐ 眠[ねむ]りが 私[わたし]をおそった。	
\\	象は鼻が長いです。	
\\	象[ぞう]は 鼻[はな]が 長[なが]いです。	
\\	竹本さんは性格が優しいです。	
\\	竹本[たけもと]さんは 性格[せいかく]が 優[やさ]しいです。	
\\	あそこに私のボールペンがありますか。	
\\	あそこに私のボールペンがありますか。	
\\	私の会社には、女の人が50人以上います。	
\\	私[わたし]の 会社[かいしゃ]には、 女[おんな]の 人[ひと]が50 人[にん] 以上[いじょう]います。	
\\	あの山を見てください。まだ雪が残っていますよ。	
\\	あの 山[やま]を 見[み]てください。まだ 雪[ゆき]が 残[のこ]っていますよ。	
\\	こちらが山田さんです。	
\\	こちらが山田さんです。	
\\	昨日友達のジャックがあなたに会いたいと言っていました。	
\\	昨日[きのう] 友達[ともだち]のジャックがあなたに 会[あ]いたいと 言[い]っていました。	
\\	ご紹介します。こちらが山田さんで、こちらは鈴木さんです。	
\\	ご 紹介[しょうかい]します。こちらが 山田[やまだ]さんで	
\\	誰が一番早く来ましたか。	
\\	誰[だれ]が 一番[いちばん] 早[はや]く 来[き]ましたか。	
\\	寿司と天ぷらと、どちらが好きですか。	
\\	寿司[すし]と 天[てん]ぷらと、どちらが 好[す]きですか。	
\\	中国語と日本語と、どちらが難しいですか。	
\\	中国[ちゅうごく] 語[ご]と 日本語[にほんご]と、どちらが 難[むずか]しいですか。	
\\	なぜ彼がそんなことをやったか、分かりません。	
\\	なぜ 彼[かれ]がそんなことをやったか、 分[わ]かりません。	
\\	母が来たとき、私はごちそうを作った。	
\\	母[はは]が 来[き]たとき、 私[わたし]はごちそうを 作[つく]った。	
\\	あの人が行くんだったら、私は行かない。	
\\	あの 人[ひと]が 行[い]くんだったら、 私[わたし]は 行[い]かない。	
\\	朝の台所は、コーヒーの香りがします。	
\\	朝[あさ]の 台所[だいどころ]は、コーヒーの 香[かお]りがします。	
\\	私はお金が要る。	
\\	私はお金が 要[い]る。	
\\	交通の安全のため、厳しい規則が必要です。	
\\	交通[こうつう]の 安全[あんぜん]のため、 厳[きび]しい 規則[きそく]が 必要[ひつよう]です。	
\\	友達が作ってくれたので、あまりおいしくなくても料理は全部食べました。	
\\	友達が作ってくれたので、あまりおいしくなくても 料理[りょうり]は 全部[ぜんぶ] 食べました。	
\\	武田さんは、いくら飲んでも酔わないんですよ。	
\\	武田[たけだ]さんは、いくら 飲[の]んでも 酔[よ]わないんですよ。	
\\	2000年に、彼女は単身でロサンゼルに渡った。	
\\	年[ねん]に、 彼女[かのじょ]は 単身[たんしん]でロサンゼルに 渡[わた]った。	
\\	「大好きな料理という趣味を通して楽しく英会話をマスターできた」と強調する。	
\\	「大好きな 料理[りょうり]という 趣味[しゅみ]を 通[とお]して 楽[たの]しく 英会話[えいかいわ]をマスターできた」と 強調[きょうちょう]する。	
\\	こうした安全の手引きの重要性について強調したいです。	
\\	こうした 安全[あんぜん]の 手引[てび]きの 重要[じゅうよう] 性[せい]について 強調[きょうちょう]したいです。	
\\	最も人気が高かった商品はゲームソフトでした。	
\\	最[もっと]も 人気[にんき]が 高[たか]かった 商品[しょうひん]はゲームソフトでした。	
\\	しかし、両国は、検査方法をめぐって対立しています。	
\\	しかし、 両国[りょうこく]は、 検査[けんさ] 方法[ほうほう]をめぐって 対立[たいりつ]しています。	
\\	その政党は移民の問題をめぐって分裂しました。	
\\	その 政党[せいとう]は 移民[いみん]の 問題[もんだい]をめぐって 分裂[ぶんれつ]しました。	
\\	減税をめぐる論争が続いている。	
\\	減税[げんぜい]をめぐる 論争[ろんそう]が 続[つづ]いている。	
\\	25日に行きたかったけど、席がないんだったら仕方がない。	
\\	日[にち]に 行[い]きたかったけど、 席[せき]がないんだったら 仕方[しかた]がない。	
\\	人に失礼なことをすれば、失礼なことをされても仕方がない。	
\\	人[ひと]に 失礼[しつれい]なことをすれば、 失礼[しつれい]なことをされても 仕方[しかた]がない。	
\\	私は自分がどう見えるかが気になって仕方がない。	
\\	私[わたし]は 自分[じぶん]がどう 見[み]えるかが 気[き]になって 仕方[しかた]がない。	
\\	あの音、どこから聞こえてきてるの?!うるさくてたまらないわ!	
\\	あの 音[おと]、どこから 聞[き]こえてきてるの?!うるさくてたまらないわ!	
\\	うれしくってたまらない。	
\\	うれしくってたまらない。	
\\	お酒が一杯飲みたくてたまらない。	
\\	お 酒[さけ]が 一杯[いっぱい] 飲[の]みたくてたまらない。	
\\	ポテトチップスが食べたくてしょうがない。	
\\	ポテトチップスが 食[た]べたくてしょうがない。	
\\	あそこで君を見掛けたことがあるような気がするんだけど。	
\\	あそこで 君[きみ]を 見掛[みか]けたことがあるような 気[き]がするんだけど。	
\\	5時ぴったりだよ。	
\\	時[じ]ぴったりだよ。	
\\	あなたにぴったりの人がいるの。	
\\	あなたにぴったりの 人[ひと]がいるの。	
\\	心と体の健康はつながっている。	
\\	心[こころ]と 体[からだ]の 健康[けんこう]はつながっている。	
\\	喫煙と肺がんの間には関連がある。	
\\	喫煙[きつえん]と 肺[はい]がんの 間[あいだ]には 関連[かんれん]がある。	
\\	まるでふるさとに帰ってきたような気がする。	
\\	まるでふるさとに 帰[かえ]ってきたような 気[き]がする。	
\\	あら、まるで私たちのことみたい!	
\\	あら、まるで 私[わたし]たちのことみたい!	
\\	あなたのことを既に知っているような感じがします。	
\\	あなたのことを 既[すで]に 知[し]っているような 感[かん]じがします。	
\\	この30年間、日本人は外国人に冷たくなった感じがします。	
\\	この30 年間[ねんかん]、 日本人[にほんじん]は 外国[がいこく] 人[じん]に 冷[つめ]たくなった 感[かん]じがします。	
\\	それはどのような感じがするのでしょうか。	
\\	それはどのような 感[かん]じがするのでしょうか。	
\\	こんなふうに感じるなんて、きっと私はおかしいのだろう。	
\\	こんなふうに 感[かん]じるなんて、きっと 私[わたし]はおかしいのだろう。	
\\	こんなふうに手を動かすと痛い。	
\\	こんなふうに 手[て]を 動[うご]かすと 痛[いた]い。	
\\	そんなふうにしていたら、絶対に終わりませんよ。	
\\	そんなふうにしていたら、 絶対[ぜったい]に 終[お]わりませんよ。	
\\	そんなふうに考えたことは全くありませんでした。	
\\	そんなふうに 考[かんが]えたことは 全[まった]くありませんでした。	
\\	そんなふうに見ないで。	
\\	そんなふうに 見[み]ないで。	
\\	あんなふうに振る舞ってすみません。	
\\	あんなふうに 振る舞[ふるま]ってすみません。	
\\	おそらく彼はあんなふうに答えるべきではなかったのだろう。	
\\	おそらく 彼[かれ]はあんなふうに 答[こた]えるべきではなかったのだろう。	
\\	ぐすぐすしないでついてきなさい。	
\\	ぐすぐすしないでついてきなさい。	
\\	24時間スポーツチャンネルをつけている。	
\\	時間[じかん]スポーツチャンネルをつけている。	
\\	今、アジア諸国がものすごい勢いで力をつけてきている。	
\\	今[いま]、アジア 諸国[しょこく]がものすごい 勢[いきお]いで 力[ちから]をつけてきている。	
\\	世界には200を超える国があるが、特定の国を好きになる人が多い。	
\\	世界[せかい]には200を 超[こ]える 国[くに]があるが、 特定[とくてい]の 国[くに]を 好[す]きになる 人[ひと]が 多[おお]い。	
\\	世界には六つの大陸がある。	
\\	世界[せかい]には 六[むっ]つの 大陸[たいりく]がある。	
\\	特に近年は、全国で地震や洪水が数多く発生しています。	
\\	特[とく]に 近年[きんねん]は、 全国[ぜんこく]で 地震[じしん]や 洪水[こうずい]が 数多[かずおお]く 発生[はっせい]しています。	
\\	この現象の背景には、世界的な結婚に対する考え方の大きな変化がある。	
\\	この 現象[げんしょう]の 背景[はいけい]には、 世界[せかい] 的[てき]な 結婚[けっこん]に 対[たい]する 考え方[かんがえかた]の 大[おお]きな 変化[へんか]がある。	
\\	この仕事を得るためには、東京に引っ越さなくてはなりません。	
\\	この 仕事[しごと]を 得[え]るためには、 東京[とうきょう]に 引っ越[ひっこ]さなくてはなりません。	
\\	こんなこと、日本でしかあり得ないことですよ。	
\\	こんなこと、 日本[にっぽん]でしかあり 得[え]ないことですよ。	
\\	その話にハッピーエンドはあり得ない。	
\\	その 話[はなし]にハッピーエンドはあり 得[え]ない。	
\\	イスラエルにいる450人の難民のうち、およそ150人は拘束されている。	
\\	イスラエルにいる450 人[にん]の 難民[なんみん]のうち、およそ150 人[にん]は 拘束[こうそく]されている。	
\\	私は日本語を習うために日本に来ました。ですから、できるだけ日本語を使いたいと思っています。	
\\	私は日本語を 習[なら]うために日本に来ました。ですから、できるだけ日本語を使いたいと思っています。	
\\	前の日に試験を受けさせていただけないでしょうか。	
\\	前[まえ]の 日[ひ]に 試験[しけん]を 受[う]けさせていただけないでしょうか。	
\\	先生、授業を休ませていただけないでしょうか。	
\\	先生[せんせい]、 授業[じゅぎょう]を 休[やす]ませていただけないでしょうか。	
\\	「勉強すればするほど、私の知らないことがあることに気が付きます」と、彼は笑う。	
\\	勉強[べんきょう]すればするほど、 私[わたし]の 知[し]らないことがあることに 気が付[きがつ]きます」と、 彼[かれ]は 笑[わら]う。	
\\	ある問題を議論すればするほど複雑に見えてくる。	
\\	ある 問題[もんだい]を 議論[ぎろん]すればするほど 複雑[ふくざつ]に 見[み]えてくる。	
\\	お互いに合えば合うほど、より親しみを感じるはず。	
\\	お 互[たが]いに 合[あ]えば 合[あ]うほど、より 親[した]しみを 感[かん]じるはず。	
\\	「ご老体は釣り以外気にならないんだ。」	
\\	「ご 老体[ろうたい]は 釣[つ]り 以外[いがい] 気[き]にならないんだ。」	
\\	日本以外で日本語を教えている国はあまりない。	
\\	日本 以外[いがい]で 日本語[にほんご]を 教[おし]えている 国[くに]はあまりない。	
\\	あなたには家族以外の人々のサポートが必要かもしれません。	
\\	あなたには 家族[かぞく] 以外[いがい]の 人々[ひとびと]のサポートが 必要[ひつよう]かもしれません。	
\\	必ずしも日本へ行けば日本語が上手になるというわけではありません。	
\\	必[かなら]ずしも 日本[にっぽん]へ 行[い]けば 日本語[にほんご]が 上手[じょうず]になるというわけではありません。	
\\	必ずしも高いものがいいというわけではない。	
\\	必[かなら]ずしも 高[たか]いものがいいというわけではない。	
\\	何も変化していないというわけではない。	
\\	何[なに]も 変化[へんか]していないというわけではない。	
\\	誰もが一緒になる人を見つけられるほど幸運というわけではないのです。	
\\	誰[だれ]もが 一緒[いっしょ]になる 人[ひと]を 見[み]つけられるほど 幸運[こううん]というわけではないのです。	
\\	文句ばかり言っていても解決にはなりません。	
\\	文句[もんく]ばかり 言[い]っていても 解決[かいけつ]にはなりません。	
\\	それは災難ばかり引き起こす。	
\\	それは 災難[さいなん]ばかり 引き起[ひきお]こす。	
\\	黒人女性は文句ばかり言って僕に何もさせてくれない。	
\\	黒人[こくじん] 女性[じょせい]は 文句[もんく]ばかり 言[い]って 僕[ぼく]に 何[なに]もさせてくれない。	
\\	なるべく早く配達してください。	
\\	なるべく 早[はや]く 配達[はいたつ]してください。	
\\	そのことはなるべく考えないようにしている。	
\\	そのことはなるべく 考[かんが]えないようにしている。	
\\	私は、この借金をなるべく早く完済したいです。	
\\	私[わたし]は、この 借金[しゃっきん]をなるべく 早[はや]く 完済[かんさい]したいです。	
\\	日本語の新聞が読みたかったら、漢字を勉強すべきだ。	
\\	日本語[にほんご]の 新聞[しんぶん]が 読[よ]みたかったら、 漢字[かんじ]を 勉強[べんきょう]すべきだ。	
\\	国籍という概念はなくしていくべきだ。	
\\	国籍[こくせき]という 概念[がいねん]はなくしていくべきだ。	
\\	と言ったかったのです。	
\\	と 言[い]ったかったのです。	
\\	その出来事が彼にとってどのような意味があったのか、私には分からない。	
\\	その 出来事[できごと]が 彼[かれ]にとってどのような 意味[いみ]があったのか、 私[わたし]には 分[わ]からない。	
\\	深い意味はない。	
\\	深[ふか]い 意味[いみ]はない。	
\\	呼吸は大抵無意識に行われる。	
\\	呼吸[こきゅう]は 大抵[たいてい] 無意識[むいしき]に 行[おこな]われる。	
\\	たいていの人には決まった習慣がある。	
\\	たいていの 人[ひと]には 決[き]まった 習慣[しゅうかん]がある。	
\\	たいていの日本の台所は食器洗い機を置くには狭過ぎる。	
\\	たいていの日本の 台所[だいどころ]は 食器洗[しょっきあら]い 機[き]を 置[お]くには 狭[せま] 過[す]ぎる。	
\\	それはたいていの場合、よい兆候です。	
\\	それはたいていの 場合[ばあい]、よい 兆候[ちょうこう]です。	
\\	ちゃんと箸を持ちなさい。	
\\	ちゃんと 箸[はし]を 持[も]ちなさい。	
\\	ちゃんとできないんだったら、最初からやめておけ。	
\\	ちゃんとできないんだったら、 最初[さいしょ]からやめておけ。	
\\	ちゃんと家に帰れる?	
\\	ちゃんと 家[いえ]に 帰[かえ]れる?	
\\	アルバイトと学業もちゃんと両立させています。	
\\	アルバイトと 学業[がくぎょう]もちゃんと 両立[りょうりつ]させています。	
\\	きちんとした事務所が必要です。	
\\	きちんとした 事務所[じむしょ]が 必要[ひつよう]です。	
\\	あなたはもう少しきちんとする必要があります。	
\\	あなたはもう 少[すこ]しきちんとする 必要[ひつよう]があります。	
\\	その先生は、生徒ときちんと向き合っています。	
\\	その 先生[せんせい]は、 生徒[せいと]ときちんと 向き合[むきあ]っています。	
\\	きっとうまくいくよ。	
\\	きっとうまくいくよ。	
\\	きっと事故があったに違いない。	
\\	きっと 事故[じこ]があったに 違[ちが]いない。	
\\	きっと楽しい時間が過ごせるでしょう。	
\\	きっと 楽[たの]しい 時間[じかん]が 過[す]ごせるでしょう。	
\\	のんびりと、一日中テレビを見て過ごすことにした。	
\\	のんびりと、一 日[にち] 中[ちゅう]テレビを 見[み]て 過[す]ごすことにした。	
\\	あの男は実にのんびりしているよ。	
\\	あの 男[おとこ]は 実[じつ]にのんびりしているよ。	
\\	これと比べると日本語は、よく批判されるように、確かに曖昧に見える。	
\\	これと 比[くら]べると 日本語[にほんご]は、よく 批判[ひはん]されるように、 確[たし]かに 曖昧[あいまい]に 見[み]える。	
\\	その政治家は、曖昧な発言をした。	
\\	その 政治[せいじ] 家[か]は、 曖昧[あいまい]な 発言[はつげん]をした。	
\\	それは私には非常に印象的な出来事だった。	
\\	それは 私[わたし]には 非常[ひじょう]に 印象[いんしょう] 的[てき]な 出来事[できごと]だった。	
\\	彼はキャンパスの名物男です。ほとんど知らない人はいない。	
\\	彼[かれ]はキャンパスの 名物[めいぶつ] 男[おとこ]です。ほとんど 知[し]らない 人[ひと]はいない。	
\\	厄介なのは、それにはお金がかかることだ。	
\\	厄介[やっかい]なのは、それにはお 金[かね]がかかることだ。	
\\	これは外国人に限らず、日本人にとってもとても厄介だ。	
\\	これは 外国[がいこく] 人[じん]に 限[かぎ]らず、 日本人[にっぽんじん]にとってもとても 厄介[やっかい]だ。	
\\	私の計画は裏目に出てしまいました。	
\\	私[わたし]の 計画[けいかく]は 裏目[うらめ]に 出[で]てしまいました。	
\\	果たして本当にそうなのか。	
\\	果[は]たして 本当[ほんとう]にそうなのか。	
\\	お金を使い果たしてしまった。	
\\	お 金[かね]を 使い果[つかいは]たしてしまった。	
\\	この小さな村では教会が重要な役割を果たしている。	
\\	この 小[ちい]さな 村[むら]では 教会[きょうかい]が 重要[じゅうよう]な 役割[やくわり]を 果[は]たしている。	
\\	彼は果たして成功するでしょうか。	
\\	彼[かれ]は 果[は]たして 成功[せいこう]するでしょうか。	
\\	彼女は自分の役割を果たしていないね。	
\\	彼女[かのじょ]は 自分[じぶん]の 役割[やくわり]を 果[は]たしていないね。	
\\	人は誰でも変身願望を持っている。	
\\	人[ひと]は 誰[だれ]でも 変身[へんしん] 願望[がんぼう]を 持[も]っている。	
\\	夜はディスコの街に変身する。	
\\	夜[よる]はディスコの 街[まち]に 変身[へんしん]する。	
\\	普段、彼女は恋愛小説しか読まない。	
\\	普段[ふだん]、 彼女[かのじょ]は 恋愛[れんあい] 小説[しょうせつ]しか 読[よ]まない。	
\\	普段、職場の同僚とは日本語で話しています。	
\\	普段[ふだん]、 職場[しょくば]の 同僚[どうりょう]とは 日本語[にほんご]で 話[はな]しています。	
\\	たっぷり6時間はかかるよ。	
\\	たっぷり6 時間[じかん]はかかるよ。	
\\	たっぷりと水を飲むことは減量に役立つ。	
\\	たっぷりと 水[みず]を 飲[の]むことは 減量[げんりょう]に 役立[やくだ]つ。	
\\	「分かってるわ」と、サラは自信たっぷりに言った。	
\\	分[わ]かってるわ」と、サラは 自信[じしん]たっぷりに 言[い]った。	
\\	それで私は気持ちが整理されるのです。	
\\	それで 私[わたし]は 気持[きも]ちが 整理[せいり]されるのです。	
\\	デジタル写真を整理するには、これが一番の方法です。	
\\	デジタル 写真[しゃしん]を 整理[せいり]するには、これが 一番[いちばん]の 方法[ほうほう]です。	
\\	いくら遅くてもかまいません。	
\\	いくら 遅[おそ]くてもかまいません。	
\\	すみません、ここでたばこを吸ってもかまいませんか?	
\\	すみません、ここでたばこを 吸[す]ってもかまいませんか?	
\\	1997年2月半ばのある夜、私たちはバンコクで会いました。	
\\	年[ねん] 2月[にがつ] 半[なか]ばのある 夜[よる]、 私[わたし]たちはバンコクで 会[あ]いました。	
\\	これらは1990年代半ばに初めて登場した。	
\\	これらは1990 年代[ねんだい] 半[なか]ばに 初[はじ]めて 登場[とうじょう]した。	
\\	この現象について研究した歴史家が数多くいます。	
\\	この 現象[げんしょう]について 研究[けんきゅう]した 歴史[れきし] 家[か]が 数多[かずおお]くいます。	
\\	この現象はなぜ起きるのか?	
\\	この 現象[げんしょう]はなぜ 起[お]きるのか?	
\\	訴えのほとんどは妻に対する夫の暴力がらみのものでした。	
\\	訴[うった]えのほとんどは 妻[つま]に 対[たい]する 夫[おっと]の 暴力[ぼうりょく]がらみのものでした。	がらみ= 
\\	確か1950年のことだから、もう50年も経っている。	
\\	確[たし]か1950 年[ねん]のことだから、もう50 年[ねん]も 経[た]っている。	
\\	確かではないですけど、それは人気のある催し物のはずです。	
\\	確[たし]かではないですけど、それは 人気[にんき]のある 催し物[もよおしもの]のはずです。	催し物= 
\\	確かに、多国籍企業なら英語は不可欠だろう。	
\\	確[たし]かに、 多[た] 国籍[こくせき] 企業[きぎょう]なら 英語[えいご]は 不可欠[ふかけつ]だろう。	
\\	あなたと私の間でさえ、時にはコミュニケーションに問題が起きることがある。	
\\	あなたと 私[わたし]の 間[ま]でさえ、 時[とき]にはコミュニケーションに 問題[もんだい]が 起[お]きることがある。	
\\	それは考えることさえばかばかしいことです。	
\\	それは 考[かんが]えることさえばかばかしいことです。	
\\	ついには水と電気さえ止められる。	
\\	ついには 水[みず]と 電気[でんき]さえ 止[と]められる。	
\\	そして、いつの間にか私にとってもそれは習慣となっていた。	
\\	そして、いつの 間[ま]にか 私[わたし]にとってもそれは 習慣[しゅうかん]となっていた。	
\\	外はいつのまにか暗くなっていました。	
\\	外[そと]はいつのまにか 暗[くら]くなっていました。	
\\	つまらない会議でも仕事ですから出なければなりません。	
\\	つまらない 会議[かいぎ]でも 仕事[しごと]ですから 出[で]なければなりません。	
\\	嫌いな食べ物でも、体によければ食べた方がいいですね。	
\\	嫌[きら]いな 食べ物[たべもの]でも、 体[からだ]によければ 食[た]べた 方[ほう]がいいですね。	
\\	その仕事は私でもできましたからあなたならすぐできますよ。	
\\	その 仕事[しごと]は 私[わたし]でもできましたからあなたならすぐできますよ。	
\\	動物でも人間の心がわかります。	
\\	動物[どうぶつ]でも 人間[にんげん]の 心[こころ]がわかります。	
\\	ジョンさんは、日本料理なら何でも食べます。	
\\	ジョンさんは、 日本[にっぽん] 料理[りょうり]なら 何[なに]でも 食[た]べます。	
\\	私は、夜だったらいつでもいいですよ。	
\\	私[わたし]は、 夜[よる]だったらいつでもいいですよ。	
\\	ヨーロッパへ行ったら、どんな美術館でも見てみたい。	
\\	ヨーロッパへ 行[い]ったら、どんな 美術館[びじゅつかん]でも 見[み]てみたい。	
\\	英語のできる人なら、どんな人でもかまいません。	
\\	英語[えいご]のできる 人[ひと]なら、どんな 人[ひと]でもかまいません。	
\\	松本さんは運動神経がいいので、テニスでも、ゴルフでもできますよ。	
\\	松本[まつもと]さんは 運動[うんどう] 神経[しんけい]がいいので、テニスでも、ゴルフでもできますよ。	
\\	片岡さんは、外国語に興味を持っているから、フランス語でも中国語でも、すぐ覚えてしまう。	
\\	片岡[かたおか]さんは、 外国[がいこく] 語[ご]に 興味[きょうみ]を 持[も]っているから、 フランス語[ふらんすご]でも 中国語[ちゅうごくご]でも、すぐ 覚[おぼ]えてしまう。	
\\	映画でも見に行きませんか。	
\\	映画[えいが]でも 見[み]に 行[い]きませんか。	
\\	レコードでも聞きましょうか。	
\\	レコードでも 聞[き]きましょうか。	
\\	白いゆりと赤いばらの花を買いましょう。	
\\	白[しろ]いゆりと 赤[あか]いばらの 花[はな]を 買[か]いましょう。	
\\	あなたにあまりプレッシャーをかけたくない。	
\\	あなたにあまりプレッシャーをかけたくない。	
\\	しかし、状態を改善する動きがある。	
\\	しかし、 状態[じょうたい]を 改善[かいぜん]する 動[うご]きがある。	
\\	このようなことがあったので、私はいくぶん外国人恐怖症だ。	
\\	このようなことがあったので、 私[わたし]はいくぶん 外国[がいこく] 人[じん] 恐怖症[きょうふしょう]だ。	
\\	いつも、元気な姿を見せることができます。	
\\	いつも、 元気[げんき]な 姿[すがた]を 見[み]せることができます。	
\\	この厳しい時にこそ、団結を見せることが重要です。	
\\	この 厳[きび]しい 時[とき]にこそ、 団結[だんけつ]を 見[み]せることが 重要[じゅうよう]です。	
\\	本格的なケバブは炭火で焼かれます。	
\\	本格[ほんかく] 的[てき]なケバブは 炭火[すみび]で 焼[や]かれます。	
\\	本格的に日本語を勉強し始めたのは、早稲田大学に来てからですね。	
\\	本格[ほんかく] 的[てき]に 日本語[にほんご]を 勉強[べんきょう]し 始[はじ]めたのは、 早稲田大学[わせだだいがく]に 来[き]てからですね。	
\\	私は今何をするべきなのでしょうか。	
\\	私[わたし]は 今[いま] 何[なに]をするべきなのでしょうか。	
\\	じゃあ、私はどうしたらいいの?	
\\	じゃあ、私はどうしたらいいの?	
\\	アメリカの大学生はたくさん勉強しなければいけませんが、勉強ばかりしているわけではありません。	
\\	アメリカの 大学生[だいがくせい]はたくさん 勉強[べんきょう]しなければいけませんが、 勉強[べんきょう]ばかりしているわけではありません。	
\\	彼は雑誌を買わないですむよう誌面を撮影しました。	
\\	彼[かれ]は 雑誌[ざっし]を 買[か]わないですむよう 誌面[しめん]を 撮影[さつえい]しました。	
\\	それは二度と同じことをしないで済むよううまく仕事をすることなのだ。	
\\	それは 二度[にど]と 同[おな]じことをしないで 済[す]むよううまく 仕事[しごと]をすることなのだ。	
\\	多くの人が夏にエアコンをあまり使用しないで済むように扇風機を購入した。	
\\	多[おお]くの 人[ひと]が 夏[なつ]にエアコンをあまり 使用[しよう]しないで 済[す]むように 扇風機[せんぷうき]を 購入[こうにゅう]した。	
\\	1錠ずつ毎日3回服用してください。	
\\	錠[じょう]ずつ 毎日[まいにち]3 回[かい] 服用[ふくよう]してください。	
\\	4時間ごとに1粒ずつ飲んでください。	
\\	時間[じかん]ごとに 1粒[ひとつぶ]ずつ 飲[の]んでください。	一粒=ひとつぶ= 
\\	世界人口は1年で1億人ずつ増えている。	
\\	世界[せかい] 人口[じんこう]は1 年[ねん]で1 億[おく] 人[にん]ずつ 増[ふ]えている。	
\\	「彼はとてもよく気がつくわ。」	
\\	彼[かれ]はとてもよく 気[き]がつくわ。」	
\\	間違いに気がつきませんでした。	
\\	間違[まちが]いに 気[き]がつきませんでした。	
\\	それから時計に気がつきました。	
\\	それから 時計[とけい]に 気[き]がつきました。	
\\	せっかくのご招待ですが、残念ながらお受けできません。	
\\	せっかくのご 招待[しょうたい]ですが、 残念[ざんねん]ながらお 受[う]けできません。	
\\	せっかくの努力も全て水の泡だった。	
\\	せっかくの 努力[どりょく]も 全[すべ]て 水の泡[みずのあわ]だった。	水の泡= 
\\	法廷でうそをつくと、深刻な結果を招きます。	
\\	法廷[ほうてい]でうそをつくと、 深刻[しんこく]な 結果[けっか]を 招[まね]きます。	法廷= 
\\	彼女は私の言うことを信じようとしないんです。	
\\	彼女[かのじょ]は 私[わたし]の 言[い]うことを 信[しん]じようとしないんです。	
\\	どうして誰も、その件を何とかしようとしないんだろう。	
\\	どうして 誰[だれ]も、その 件[けん]を 何[なん]とかしようとしないんだろう。	
\\	ほとんどの日本人は外見で人を判断する。その人の内面を見ようとしない。	
\\	ほとんどの日本人は 外見[がいけん]で 人[ひと]を 判断[はんだん]する。その 人[ひと]の 内面[ないめん]を 見[み]ようとしない。	
\\	政治家は権力を利用して金もうけをしようとする。	
\\	政治[せいじ] 家[か]は 権力[けんりょく]を 利用[りよう]して 金[かね]もうけをしようとする。	金もうけ= 
\\	これらは、国民を管理しようとする目的が見え見えです。	
\\	これらは、 国民[こくみん]を 管理[かんり]しようとする 目的[もくてき]が 見[み]え 見[み]えです。	見え見え= 
\\	人権の推進が、私たちの活動の中心です。	
\\	人権[じんけん]の 推進[すいしん]が、 私[わたし]たちの 活動[かつどう]の 中心[ちゅうしん]です。	
\\	人権侵害に関する多くの事例がある。	
\\	人権[じんけん] 侵害[しんがい]に 関[かん]する 多[おお]くの 事例[じれい]がある。	人権侵害=
\\	権利の平等は、私たちの社会で重要な価値観です。	
\\	権利[けんり]の 平等[びょうどう]は、 私[わたし]たちの 社会[しゃかい]で 重要[じゅうよう]な 価値[かち] 観[かん]です。	
\\	他人の命を奪うのは重罪です。	
\\	他人[たにん]の 命[いのち]を 奪[うば]うのは 重罪[じゅうざい]です。	
\\	エイズは多くの人々の命を奪う病気である。	
\\	エイズは 多[おお]くの 人々[ひとびと]の 命[いのち]を 奪[うば]う 病気[びょうき]である。	
\\	いつまでもこんなふうに扱われるのはおかしいと思います。	
\\	いつまでもこんなふうに 扱[あつか]われるのはおかしいと 思[おも]います。	
\\	その民族は、自分たちの伝統を維持する方法を見いだした。	
\\	その 民族[みんぞく]は、 自分[じぶん]たちの 伝統[でんとう]を 維持[いじ]する 方法[ほうほう]を 見[み]いだした。	
\\	その車は高価で、維持するのが難しい。	
\\	その 車[くるま]は 高価[こうか]で、 維持[いじ]するのが 難[むずか]しい。	
\\	それを維持するかどうかは私たち次第です。	
\\	それを 維持[いじ]するかどうかは 私[わたし]たち 次第[しだい]です。	
\\	だから、途上国の方がアメリカと関係を維持することを望んでいる。	
\\	だから、 途上[とじょう] 国[こく]の 方[ほう]がアメリカと 関係[かんけい]を 維持[いじ]することを 望[のぞ]んでいる。	
\\	芸能界の仕事は麻薬に似ている。	
\\	芸能[げいのう] 界[かい]の 仕事[しごと]は 麻薬[まやく]に 似[に]ている。	
\\	芸能界への入り方を知りたい。	
\\	芸能[げいのう] 界[かい]への 入[はい]り 方[かた]を 知[し]りたい。	
\\	日本を嫌悪している人もいます。	
\\	日本[にっぽん]を 嫌悪[けんお]している 人[ひと]もいます。	
\\	泣いては自己嫌悪の毎日だった。	
\\	泣[な]いては 自己[じこ] 嫌悪[けんお]の 毎日[まいにち]だった。	
\\	きっとうまく溶け込むだろう。	
\\	きっとうまく 溶け込[とけこ]むだろう。	
\\	お子さんたちは新しい言葉や文化にうまく溶け込めていますか。	
\\	お 子[こ]さんたちは 新[あたら]しい 言葉[ことば]や 文化[ぶんか]にうまく 溶[と]け 込[こ]めていますか。	
\\	青少年が自動販売機を利用するのは夜間が多い。	
\\	青少年[せいしょうねん]が 自動[じどう] 販売[はんばい] 機[き]を 利用[りよう]するのは 夜間[やかん]が 多[おお]い。	
\\	あなたをこの件で罪に問わないわけにはいかない。	
\\	あなたをこの 件[けん]で 罪[つみ]に 問[と]わないわけにはいかない。	
\\	あなただけに、こんな苦労を掛けるわけにはいかない。	
\\	あなただけに、こんな 苦労[くろう]を 掛[か]けるわけにはいかない。	
\\	こんなチャンスを見逃すわけにはいかない。	
\\	こんなチャンスを 見逃[みのが]すわけにはいかない。	見逃す=
\\	私はけちではないが、そうかといって浪費家でもない。	
\\	私[わたし]はけちではないが、そうかといって 浪費[ろうひ] 家[か]でもない。	
\\	14歳の時に彼は米国に渡り、マジシャンになるための訓練を開始した。	
\\	歳[さい]の 時[とき]に 彼[かれ]は 米国[べいこく]に 渡[わた]り、マジシャンになるための 訓練[くんれん]を 開始[かいし]した。	訓練= 
\\	あの救援活動のための寄付は、どこの銀行でも受け付けている。	
\\	あの 救援[きゅうえん] 活動[かつどう]のための 寄付[きふ]は、どこの 銀行[ぎんこう]でも 受け付[うけつ]けている。	救援活動= 
\\	この箱は食料の寄付のためのものです。	
\\	この 箱[はこ]は 食料[しょくりょう]の 寄付[きふ]のためのものです。	
\\	天気予報によると、今日は午後雨が降るそうだ。	
\\	天気[てんき] 予報[よほう]によると、 今日[きょう]は 午後[ごご] 雨[あめ]が 降[ふ]るそうだ。	
\\	今日の新聞によると、昨日カリフォルニアで地震があったそうだ。	
\\	今日[きょう]の 新聞[しんぶん]によると、 昨日[きのう]カリフォルニアで 地震[じしん]があったそうだ。	
\\	長い間待たされるのは誰でも嫌だ。	
\\	長[なが]い 間[あいだ] 待[ま]たされるのは 誰[だれ]でも 嫌[いや]だ。	
\\	大抵一学期に一つは論文を書かされる。	
\\	大抵[たいてい]一 学期[がっき]に 一[ひと]つは 論文[ろんぶん]を 書[か]かされる。	
\\	一般的に言えば、日本人の喫煙マナーが悪いと思います。	
\\	一般[いっぱん] 的[てき]に 言[い]えば、 日本人[にっぽんじん]の 喫煙[きつえん]マナーが 悪[わる]いと 思[おも]います。	
\\	明白な事実を否定するわけにはいきません。	
\\	明白[めいはく]な 事実[じじつ]を 否定[ひてい]するわけにはいきません。	
\\	その質問に対する答えは明白です。	
\\	その 質問[しつもん]に 対[たい]する 答[こた]えは 明白[めいはく]です。	
\\	明らかに、それには十分な根拠があります。	
\\	明[あき]らかに、それには 十分[じゅうぶん]な 根拠[こんきょ]があります。	
\\	うわさは根拠のないものだった。	
\\	うわさは 根拠[こんきょ]のないものだった。	
\\	そのジャーナリストの主張には根拠がなかった。	
\\	そのジャーナリストの 主張[しゅちょう]には 根拠[こんきょ]がなかった。	
\\	このリーグはわずか6チームで構成されていました。	
\\	このリーグはわずか6チームで 構成[こうせい]されていました。	
\\	りんごとみかんとどちらが好きですか。	
\\	りんごとみかんとどちらが 好[す]きですか。	
\\	明日この問題について、先生と話すつもりです。	
\\	明日[あした]この 問題[もんだい]について、 先生[せんせい]と 話[はな]すつもりです。	
\\	オリンピックの開会式の日となった。	
\\	オリンピックの 開会[かいかい] 式[しき]の 日[ひ]となった。	
\\	あの山に登るには、2時間とかかりません。	
\\	あの 山[やま]に 登[のぼ]るには、2 時間[じかん]とかかりません。	[と+量 
\\	山本さんが、後で電話するとおっしゃいました。	
\\	山本[やまもと]さんが、 後[あと]で 電話[でんわ]するとおっしゃいました。	
\\	母が先生によろしくと申しておりました。	
\\	母[はは]が 先生[せんせい]によろしくと 申[もう]しておりました。	
\\	渡辺さんが9時までに事務所に来るように、と言っていました。	
\\	渡辺[わたなべ]さんが9 時[じ]までに 事務所[じむしょ]に 来[く]るように、と 言[い]っていました。	
\\	来年は、アメリカへ行こうと考えています。	
\\	来年[らいねん]は、アメリカへ 行[い]こうと 考[かんが]えています。	
\\	電車は9時に出ると思いましたが、10時でした。	
\\	電車[でんしゃ]は9 時[じ]に 出[で]ると 思[おも]いましたが、10 時[じ]でした。	
\\	まず、田中という部長に書類をもらって下さい。	
\\	まず、 田中[たなか]という 部長[ぶちょう]に 書類[しょるい]をもらって 下[くだ]さい。	
\\	小川がさらさらと、道のそばを流れていた。	
\\	小川[おがわ]がさらさらと、 道[みち]のそばを 流[なが]れていた。	
\\	星がきらきらと輝いています。	
\\	星[ほし]がきらきらと 輝[かがや]いています。	
\\	昨日は会社の仕事が終わると、まっすぐ家に帰った。	
\\	昨日[きのう]は 会社[かいしゃ]の 仕事[しごと]が 終[お]わると、まっすぐ 家[いえ]に 帰[かえ]った。	
\\	日本では春になると桜が咲きます。	
\\	日本では 春[はる]になると 桜[さくら]が 咲[さ]きます。	
\\	車が多くなると交通事故が増えます。	
\\	車[くるま]が 多[おお]くなると 交通[こうつう] 事故[じこ]が 増[ふ]えます。	
\\	不景気になると失業者が増えます。	
\\	不景気[ふけいき]になると 失業[しつぎょう] 者[しゃ]が 増[ふ]えます。	
\\	山田さんが来ないと会議が始められません。	
\\	山田[やまだ]さんが 来[こ]ないと 会議[かいぎ]が 始[はじ]められません。	
\\	明日、天気がいいと野球ができます。	
\\	明日[あした]、 天気[てんき]がいいと 野球[やきゅう]ができます。	
\\	銀行へ行くと、もう閉まっていた。	
\\	銀行[ぎんこう]へ 行[い]くと、もう 閉[し]まっていた。	
\\	交番で道を聞くと、その会社はすぐ見つかった。	
\\	交番[こうばん]で 道[みち]を 聞[き]くと、その 会社[かいしゃ]はすぐ 見[み]つかった。	
\\	彼女が一人でパーティーに行こうと行くまいと、私はかまいません。	
\\	彼女[かのじょ]が一 人[にん]でパーティーに 行[い]こうと 行[い]くまいと、 私[わたし]はかまいません。	
\\	円が強くなろうと弱くなろうと、私の生活には関係ありません。	
\\	円[えん]が 強[つよ]くなろうと 弱[よわ]くなろうと、 私[わたし]の 生活[せいかつ]には 関係[かんけい]ありません。	
\\	とは、国連のことです。	
\\	とは、 国連[こくれん]のことです。	
\\	リーダーの条件とは何でしょうか。	
\\	リーダーの 条件[じょうけん]とは 何[なに]でしょうか。	
\\	政府を信用していないとはいえ、政府のやり方に従わないわけにはいかない。	
\\	政府[せいふ]を 信用[しんよう]していないとはいえ、 政府[せいふ]のやり 方[かた]に 従[したが]わないわけにはいかない。	
\\	私の部屋には、コンピューターやステレオが置いてあります。	
\\	私[わたし]の 部屋[へや]には、コンピューターやステレオが 置[お]いてあります。	
\\	駅に着くやいなや、電車が出てしまった。	
\\	駅[えき]に 着[つ]くやいなや、 電車[でんしゃ]が 出[で]てしまった。	
\\	お風呂に入るやいなや、電話が鳴った。	
\\	お 風呂[ふろ]に 入[はい]るやいなや、 電話[でんわ]が 鳴[な]った。	
\\	休みにはジョギングをするとか、テニスをするとかしています。	
\\	休[やす]みにはジョギングをするとか、テニスをするとかしています。	
\\	川口さんは、あの銀行に勤めるとか勤めないとか言っていました、どうなりましたか。	
\\	川口[かわぐち]さんは、あの 銀行[ぎんこう]に 勤[つと]めるとか 勤[つと]めないとか 言[い]っていました、どうなりましたか。	
\\	あの人はその時によって、仕事が面白いとか面白くないとか言うので、どちらなのかわかりません。	
\\	あの 人[ひと]はその 時[とき]によって、 仕事[しごと]が 面白[おもしろ]いとか 面白[おもしろ]くないとか 言[い]うので、どちらなのかわかりません。	
\\	私は、みかんやりんごやバナナなど、果物なら何でも好きです。	
\\	私[わたし]は、みかんやりんごやバナナなど、 果物[くだもの]なら 何[なに]でも 好[す]きです。	
\\	来週の旅行は箱根などどうですか。	
\\	来週[らいしゅう]の 旅行[りょこう]は 箱根[はこね]などどうですか。	
\\	プレゼントを買うなら、真珠のブローチなんかいいんじゃないんですか。	
\\	プレゼントを 買[か]うなら、 真珠[しんじゅ]のブローチなんかいいんじゃないんですか。	真珠=しんじゅ= 
\\	私など、そんな難しい試験にはとても合格できません。	
\\	私[わたし]など、そんな 難[むずか]しい 試験[しけん]にはとても 合格[ごうかく]できません。	
\\	橋本先生のご兄弟と違って、私の兄弟なぞは、頭が悪い者ばかりです。	
\\	橋本[はしもと] 先生[せんせい]のご 兄弟[きょうだい]と 違[ちが]って、 私[わたし]の 兄弟[きょうだい]なぞは、 頭[あたま]が 悪[わる]い 者[もの]ばかりです。	
\\	田中さんなどは、とても社長にはなれない。	
\\	田中[たなか]さんなどは、とても 社長[しゃちょう]にはなれない。	
\\	あの人なんか、選挙に出てもだめですよ。	
\\	あの 人[ひと]なんか、 選挙[せんきょ]に 出[で]てもだめですよ。	
\\	家なんか、とても買えない。	
\\	家[いえ]なんか、とても 買[か]えない。	
\\	山本さんがそのコンサートがとてもよかったなどと言っていましたよ。	
\\	山本[やまもと]さんがそのコンサートがとてもよかったなどと 言[い]っていましたよ。	
\\	彼はその仕事を自分がやったなどと言っている。	
\\	彼[かれ]はその 仕事[しごと]を 自分[じぶん]がやったなどと 言[い]っている。	
\\	これは誰の傘ですか。	
\\	これは 誰[だれ]の 傘[かさ]ですか。	
\\	佐藤さんに聞いてみたらどうですか。	
\\	佐藤[さとう]さんに 聞[き]いてみたらどうですか。	
\\	こんなにきれいな所が、ほかにあるだろうか。	
\\	こんなにきれいな 所[ところ]が、ほかにあるだろうか。	
\\	そんな悪い人がいるものですか。	
\\	そんな 悪[わる]い 人[ひと]がいるものですか。	
\\	また今日も、遅れて来たんですか。	
\\	また 今日[きょう]も、 遅[おく]れて 来[き]たんですか。	
\\	コーヒーか紅茶か飲みたいですね。	
\\	コーヒーか 紅茶[こうちゃ]か 飲[の]みたいですね。	
\\	旅行に行くか行かないか、まだ決めていません。	
\\	旅行[りょこう]に 行[い]くか 行[い]かないか、まだ 決[き]めていません。	
\\	広田さんはお酒が飲めるかどうか聞いてみましょう。	
\\	広田[ひろた]さんはお 酒[さけ]が 飲[の]めるかどうか 聞[き]いてみましょう。	
\\	風邪をひいたのか、頭が痛いんです。	
\\	風邪[かぜ]をひいたのか、 頭[あたま]が 痛[いた]いんです。	
\\	試験があるのか、みんな図書館で勉強していますよ。	
\\	試験[しけん]があるのか、みんな 図書館[としょかん]で 勉強[べんきょう]していますよ。	
\\	誰か山田さんの電話番号を知っていますか。	
\\	誰[だれ]か 山田[やまだ]さんの 電話[でんわ] 番号[ばんごう]を 知[し]っていますか。	
\\	何か冷たいものが飲みたい。	
\\	何[なに]か 冷[つめ]たいものが 飲[の]みたい。	
\\	山田さんとかいう人から電話がありました。	
\\	山田[やまだ]さんとかいう 人[ひと]から 電話[でんわ]がありました。	
\\	彼女は、デパートでかブティックでか、どちらかで買物をしたいと言っていました。	
\\	彼女[かのじょ]は、デパートでかブティックでか、どちらかで 買物[かいもの]をしたいと 言[い]っていました。	
\\	駅に着くか着かないうちに電車が来た。	
\\	駅[えき]に 着[つ]くか 着[つ]かないうちに 電車[でんしゃ]が 来[き]た。	
\\	お風呂に入るか入らないうちに電話が鳴った。	
\\	お 風呂[ふろ]に 入[はい]るか 入[はい]らないうちに 電話[でんわ]が 鳴[な]った。	
\\	昨日銀座のレストランで晩ご飯を食べました。	
\\	昨日[きのう] 銀座[ぎんざ]のレストランで 晩[ばん]ご 飯[はん]を 食[た]べました。	
\\	私は日本へ船で来ました。	
\\	私[わたし]は 日本[にっぽん]へ 船[ふね]で 来[き]ました。	
\\	このケーキは、卵と砂糖で作ります。	
\\	このケーキは、 卵[たまご]と 砂糖[さとう]で 作[つく]ります。	
\\	昔、日本人は木と紙で作った家に住んでいました。	
\\	昔[むかし]、 日本人[にっぽんじん]は 木[き]と 紙[かみ]で 作[つく]った 家[いえ]に 住[す]んでいました。	
\\	これはこの村で一番古いお寺です。	
\\	これはこの 村[むら]で 一番[いちばん] 古[ふる]いお 寺[てら]です。	
\\	この本は1時間で読めますよ。	
\\	この 本[ほん]は1 時間[じかん]で 読[よ]めますよ。	
\\	あのテレビは10万円で買える。	
\\	あのテレビは10 万[まん] 円[えん]で 買[か]える。	
\\	山田さんはアパートに一人で住んでいます。	
\\	山田[やまだ]さんはアパートに一 人[にん]で 住[す]んでいます。	
\\	家族中でハワイへ旅行した。	
\\	家族[かぞく] 中[ちゅう]でハワイへ 旅行[りょこう]した。	
\\	あの詩人は25歳で死んだ。	
\\	あの 詩人[しじん]は25 歳[さい]で 死[し]んだ。	
\\	戦争が終わって来年で50年になる。	
\\	戦争[せんそう]が 終[お]わって 来年[らいねん]で50 年[ねん]になる。	
\\	病気で旅行に行けなかった。	
\\	病気[びょうき]で 旅行[りょこう]に 行[い]けなかった。	
\\	台風電車が止まった。	
\\	台風[たいふう] 電車[でんしゃ]が 止[と]まった。	
\\	山田先生は、今図書館にいらっしゃいます。	
\\	山田[やまだ] 先生[せんせい]は、 今[こん] 図書館[としょかん]にいらっしゃいます。	
\\	電話帳は机の上にあります。	
\\	電話[でんわ] 帳[ちょう]は 机[つくえ]の 上[うえ]にあります。	
\\	課長は今会議に出席しています。	
\\	課長[かちょう]は 今[こん] 会議[かいぎ]に 出席[しゅっせき]しています。	
\\	彼は今でも演劇界に君臨しています。	
\\	彼[かれ]は 今[いま]でも 演劇[えんげき] 界[かい]に 君臨[くんりん]しています。	
\\	寺田さんは新宿の銀行に勤めています。	
\\	寺田[てらだ]さんは 新宿[しんじゅく]の 銀行[ぎんこう]に 勤[つと]めています。	
\\	あのいすに座って本を読んでいる人は、誰ですか。	
\\	あのいすに 座[すわ]って 本[ほん]を 読[よ]んでいる 人[ひと]は、 誰[だれ]ですか。	
\\	山の上に雪が積もっていますね。	
\\	山の上[やまのうえ]に 雪[ゆき]が 積[つ]もっていますね。	
\\	すみませんが、壁に掛かっている私のコートを取ってくれますか。	
\\	すみませんが、 壁[かべ]に 掛[か]かっている 私[わたし]のコートを 取[と]ってくれますか。	
\\	新聞は机の上に置いて下さい。	
\\	新聞[しんぶん]は 机[つくえ]の 上[うえ]に 置[お]いて 下[くだ]さい。	
\\	会社は9時に始まります。	
\\	会社[かいしゃ]は9 時[じ]に 始[はじ]まります。	
\\	月曜日に大阪へ行きます。	
\\	月曜日[げつようび]に 大阪[おおさか]へ 行[い]きます。	
\\	1週間に1度テニスをします。	
\\	週間[しゅうかん]に1 度[ど]テニスをします。	
\\	このバスは30分おきに来ます。	
\\	このバスは30分おきに 来[き]ます。	
\\	東京駅の前でバスに乗って下さい。	
\\	東京[とうきょう] 駅[えき]の 前[まえ]でバスに 乗[の]って 下[くだ]さい。	
\\	オフィスに入ったら、タバコは吸わないで下さい。	
\\	オフィスに 入[はい]ったら、タバコは 吸[す]わないで 下[くだ]さい。	
\\	去年渡辺さんは歴史学会に入った。	
\\	去年[きょねん] 渡辺[わたなべ]さんは 歴史[れきし] 学会[がっかい]に 入[はい]った。	
\\	あなたは僕の夢の中に何度も出てきました。	
\\	あなたは 僕[ぼく]の 夢[ゆめ]の 中[なか]に 何[なん] 度[ど]も 出[で]てきました。	
\\	明日は歌舞伎に行くつもりです。	
\\	明日[あした]は 歌舞伎[かぶき]に行くつもりです。	
\\	もうお昼ですから、食事をしに行きませんか。	
\\	もうお 昼[ひる]ですから、 食事[しょくじ]をしに 行[い]きませんか。	
\\	木下さんは友達を迎えに成田まで出かけました。	
\\	木下[きのした]さんは 友達[ともだち]を 迎[むか]えに 成田[なりた]まで 出[で]かけました。	
\\	昨日フランスにいるナンシーに手紙を出してあげた。	
\\	昨日[きのう]フランスにいるナンシーに 手紙[てがみ]を 出[だ]してあげた。	
\\	ジョンさんは大学を卒業して、医者になった。	
\\	ジョンさんは 大学[だいがく]を 卒業[そつぎょう]して、 医者[いしゃ]になった。	
\\	このケーキを3つに分けて下さい。	
\\	このケーキを 3[みっ]つに 分[わ]けて 下[くだ]さい。	
\\	渡辺さんは仕事のしすぎて病気になった。	
\\	渡辺[わたなべ]さんは 仕事[しごと]のしすぎて 病気[びょうき]になった。	
\\	この建物の右側が教室になっています。	
\\	この 建物[たてもの]の 右側[みぎがわ]が 教室[きょうしつ]になっています。	
\\	ホテルの前がビーチになっています。	
\\	ホテルの 前[まえ]がビーチになっています。	
\\	電車の中で、すりにお金を取られた。	
\\	電車[でんしゃ]の 中[なか]で、すりにお 金[かね]を 取[と]られた。	掏摸=すり= 
\\	家に帰る途中で雨に降られた。	
\\	家[いえ]に 帰[かえ]る 途中[とちゅう]で 雨[あめ]に 降[ふ]られた。	
\\	先生は学生に漢字を書かせました。	
\\	先生[せんせい]は 学生[がくせい]に 漢字[かんじ]を 書[か]かせました。	
\\	子供たちに本を読ませることはとても大切だ。	
\\	子供[こども]たちに 本[ほん]を 読[よ]ませることはとても 大切[たいせつ]だ。	
\\	学生は先生に漢字を書かされました。	
\\	学生[がくせい]は 先生[せんせい]に 漢字[かんじ]を 書[か]かされました。	
\\	私は子供のとき、母に嫌いなものも食べさせられました。	
\\	私[わたし]は 子供[こども]のとき、 母[はは]に 嫌[きら]いなものも 食[た]べさせられました。	
\\	その会議に出席した人は、中国人に、韓国人に、日本人だった。	
\\	その 会議[かいぎ]に 出席[しゅっせき]した 人[ひと]は、 中国人[ちゅうごくじん]に、 韓国[かんこく] 人[じん]に、 日本人[にっぽんじん]だった。	
\\	ロメオにジュリエット。	
\\	ロメオにジュリエット。	「に」
\\	あの映画は有名な小説に基づいて作られました。	
\\	あの 映画[えいが]は 有名[ゆうめい]な 小説[しょうせつ]に 基[もと]づいて 作[つく]られました。	
\\	テレビの普及によって、外国の様子がよくわかるようになった。	
\\	テレビの 普及[ふきゅう]によって、 外国[がいこく]の 様子[ようす]がよくわかるようになった。	
\\	この飛行機は、6時に成田空港へ到着しました。	
\\	この 飛行機[ひこうき]は、6 時[じ]に 成田空港[なりたくうこう]へ 到着[とうちゃく]しました。	
\\	外国にいる友達へ手紙を書いた。	
\\	外国[がいこく]にいる 友達[ともだち]へ 手紙[てがみ]を 書[か]いた。	
\\	夕方川田さんへ電話をかけたが、いなかった。	
\\	夕方[ゆうがた] 川田[かわた]さんへ 電話[でんわ]をかけたが、いなかった。	
\\	日本語のクラスは、1時から4時までです。	
\\	日本語[にほんご]のクラスは、1 時[じ]から4 時[じ]までです。	
\\	マラソンはここから出発します。	
\\	マラソンはここから 出発[しゅっぱつ]します。	
\\	社長はパリから飛行機でスペインへ行きます。	
\\	社長[しゃちょう]はパリから 飛行機[ひこうき]でスペインへ 行[い]きます。	
\\	新聞を隅から隅まで読んだ。	
\\	新聞[しんぶん]を 隅[すみ]から 隅[すみ]まで 読[よ]んだ。	"「隅から隅まで」= 
\\	女の人の目から見れば、日本にはまだ差別がたくさんある。	
\\	女[おんな]の 人[ひと]の 目[め]から 見[み]れば、 日本[にっぽん]にはまだ 差別[さべつ]がたくさんある。	
\\	昨日私は仕事が終わってから買物をしました。	
\\	昨日[きのう] 私[わたし]は 仕事[しごと]が 終[お]わってから 買物[かいもの]をしました。	
\\	明日の夜、食事をしてから映画を見ませんか。	
\\	明日[あした]の 夜[よる]、 食事[しょくじ]をしてから 映画[えいが]を 見[み]ませんか。	
\\	山田さんが大学を卒業してから5年になります。	
\\	山田[やまだ]さんが 大学[だいがく]を 卒業[そつぎょう]してから5 年[ねん]になります。	
\\	あの2人が結婚してから20年だそうです。	
\\	あの2 人[にん]が 結婚[けっこん]してから20 年[ねん]だそうです。	
\\	ワインはぶどうから作ります。	
\\	ワインはぶどうから 作[つく]ります。	
\\	豆腐は何から作るか知っていますか。	
\\	豆腐[とうふ]は 何[なに]から 作[つく]るか 知[し]っていますか。	
\\	この望みをかなえてくれる最適なパートナーを探し求めます。	
\\	この 望[のぞ]みをかなえてくれる 最適[さいてき]なパートナーを 探し求[さがしもと]めます。	
\\	これはお土産に最適ですね。	
\\	これはお 土産[みやげ]に 最適[さいてき]ですね。	
\\	新生児は母親の声を識別できる。	
\\	新生児[しんせいじ]は 母親[ははおや]の 声[こえ]を 識別[しきべつ]できる。	
\\	私は、野生のキノコの幾種類かを識別できる。	
\\	私は、 野生[やせい]のキノコの 幾[いく] 種類[しゅるい]かを 識別[しきべつ]できる。	
\\	あなたは、経験の結果よりも過程の方を大事に思っています。	
\\	あなたは、 経験[けいけん]の 結果[けっか]よりも 過程[かてい]の 方[ほう]を 大事[だいじ]に 思[おも]っています。	
\\	この過程があまりにも早過ぎた。	
\\	この 過程[かてい]があまりにも 早[はや] 過[す]ぎた。	
\\	この過程は、見た目ほど複雑ではない。	
\\	この 過程[かてい]は、 見た目[みため]ほど 複雑[ふくざつ]ではない。	
\\	とりあえずそれで結構です。	
\\	とりあえずそれで 結構[けっこう]です。	
\\	とりあえず今は目の前の問題に集中しよう。	
\\	とりあえず 今[いま]は 目[め]の 前[まえ]の 問題[もんだい]に 集中[しゅうちゅう]しよう。	
\\	それはいい考えだけど、とりあえず、使い捨てカメラを買おうよ。	
\\	それはいい 考[かんが]えだけど、とりあえず、 使い捨[つかいす]てカメラを 買[か]おうよ。	
\\	ミーティングを5月に延期してもよろしいでしょうか。	
\\	ミーティングを 5月[ごがつ]に 延期[えんき]してもよろしいでしょうか。	
\\	彼らは彼が海外から帰国するまで結婚式を延期しました。	
\\	彼[かれ]らは 彼[かれ]が 海外[かいがい]から 帰国[きこく]するまで 結婚式[けっこんしき]を 延期[えんき]しました。	
\\	彼は階段から落ちて手を骨折しました。	
\\	彼[かれ]は 階段[かいだん]から 落[お]ちて 手[て]を 骨折[こっせつ]しました。	
\\	右目がぼやける。	
\\	右目[みぎめ]がぼやける。	ぼやける= 
\\	私は大使からパーティーに招待されました。	
\\	私[わたし]は 大使[たいし]からパーティーに 招待[しょうたい]されました。	
\\	そんなことばかり言っているとみんなに嫌われるから。。。	
\\	そんなことばかり 言[い]っているとみんなに 嫌[きら]われるから。。。	
\\	この会社の社員は9時から5時まで働きます。	
\\	この 会社[かいしゃ]の 社員[しゃいん]は9 時[じ]から5 時[じ]まで 働[はたら]きます。	
\\	このデパートは、土曜日と日曜日は8時までです。	
\\	このデパートは、 土曜日[どようび]と 日曜日[にちようび]は8 時[じ]までです。	
\\	この飛行機は東京からホノルルまで行きます。	
\\	この 飛行機[ひこうき]は 東京[とうきょう]からホノルルまで 行[い]きます。	
\\	ここから京都まで何時間かかりますか。	
\\	ここから 京都[きょうと]まで 何[なん] 時間[じかん]かかりますか。	
\\	子供だけでなく大人まで、そのゲームを楽しんだ。	
\\	子供[こども]だけでなく 大人[おとな]まで、そのゲームを 楽[たの]しんだ。	
\\	その日山の上はとても寒くて、夕方には雪まで降ってきた。	
\\	その 日山[ひやま]の 上[うえ]はとても 寒[さむ]くて、 夕方[ゆうがた]には 雪[ゆき]まで 降[ふ]ってきた。	
\\	斉藤さんは、あの男の人と結婚できなければ死のうとまで思いつめたそうです。	
\\	斉藤[さいとう]さんは、あの 男[おとこ]の 人[ひと]と 結婚[けっこん]できなければ 死[し]のうとまで 思[おも]いつめたそうです。	
\\	その両親は子供の病気が治るなら、全財産を捨ててもいいとまで考えていた。	
\\	その 両親[りょうしん]は 子供[こども]の 病気[びょうき]が 治[なお]るなら、 全[ぜん] 財産[ざいさん]を 捨[す]ててもいいとまで 考[かんが]えていた。	
\\	明日のパーティーにはわざわざ行くまでもない。	
\\	明日[あした]のパーティーにはわざわざ 行[い]くまでもない。	
\\	言うまでもないことですが、この会社の経営状態は、かなり悪化しています。	
\\	言[い]うまでもないことですが、この 会社[かいしゃ]の 経営[けいえい] 状態[じょうたい]は、かなり 悪化[あっか]しています。	
\\	早稲田大学学長の今井氏が演説をしています。	
\\	早稲田大学[わせだだいがく] 学長[がくちょう]の 今井[いまい] 氏[し]が 演説[えんぜつ]をしています。	
\\	こちらが佐山さんのお姉さんの千香子さんです。	
\\	こちらが 佐山[さやま]さんのお 姉[ねえ]さんの 千香子[ちかこ]さんです。	
\\	これは坂本さんの描いた油絵です。	
\\	これは 坂本[さかもと]さんの 描[えが]いた 油絵[あぶらえ]です。	
\\	昨日あなたの話していたレストランはどこですか。	
\\	昨日[きのう]あなたの 話[はな]していたレストランはどこですか。	
\\	天気が悪いですから、ドライブに行くのはやめましょう。	
\\	天気[てんき]が 悪[わる]いですから、ドライブに 行[い]くのはやめましょう。	
\\	このビルの屋上から、車が走っているのがよく見えます。	
\\	このビルの 屋上[おくじょう]から、 車[くるま]が 走[はし]っているのがよく 見[み]えます。	
\\	女の人が歌っているのが聞こえますね。	
\\	女[おんな]の 人[ひと]が 歌[うた]っているのが 聞[き]こえますね。	
\\	会社、本当にやめるの?	
\\	会社[かいしゃ]、 本当[ほんとう]にやめるの?	
\\	私は昨日、電車の中でお金とパスポートを盗まれました。	
\\	私[わたし]は 昨日[きのう]、 電車[でんしゃ]の 中[なか]でお 金[かね]とパスポートを 盗[ぬす]まれました。	
\\	彼女は犬に手を噛まれた。	
\\	彼女[かのじょ]は 犬[いぬ]に 手[て]を 噛[か]まれた。	
\\	その政治家は、財界人のパーティーに秘書を出席させた。	
\\	その 政治[せいじ] 家[か]は、 財界[ざいかい] 人[じん]のパーティーに 秘書[ひしょ]を 出席[しゅっせき]させた。	
\\	部長は部下を出張させた。	
\\	部長[ぶちょう]は 部下[ぶか]を 出張[しゅっちょう]させた。	
\\	山本さんのお父さんは、医者をしている。	
\\	山本[やまもと]さんのお 父[とう]さんは、 医者[いしゃ]をしている。	
\\	私の兄は、新聞記者をしています。	
\\	私[わたし]の 兄[あに]は、 新聞[しんぶん] 記者[きしゃ]をしています。	
\\	ジョンさんはお寿司を食べたがっていますよ。	
\\	ジョンさんはお 寿司[すし]を 食[た]べたがっていますよ。	
\\	毎日新宿駅で地下鉄を降ります。	
\\	毎日[まいにち] 新宿[しんじゅく] 駅[えき]で 地下鉄[ちかてつ]を 降[お]ります。	
\\	山本さんは夕方5時半に会社を出ます。	
\\	山本[やまもと]さんは 夕方[ゆうがた]5 時半[じはん]に 会社[かいしゃ]を 出[で]ます。	
\\	沖氏は、70歳になった年に経済界を引退した。	
\\	沖[おき] 氏[し]は、70 歳[さい]になった 年[とし]に 経済[けいざい] 界[かい]を 引退[いんたい]した。	
\\	車で新しい橋を渡った。	
\\	車[くるま]で 新[あたら]しい 橋[はし]を 渡[わた]った。	
\\	私の国では、車は道の左側を走ります。	
\\	私[わたし]の 国[くに]では、 車[くるま]は 道[みち]の 左側[ひだりがわ]を 走[はし]ります。	
\\	このバスは、デパートの前を通りますか。	
\\	このバスは、デパートの 前[まえ]を 通[とお]りますか。	
\\	社長は火曜日の午後6時に成田を出発します。	
\\	社長[しゃちょう]は 火曜日[かようび]の 午後[ごご]6 時[じ]に 成田[なりた]を 出発[しゅっぱつ]します。	
\\	この電車は8時に東京駅を出ますから遅れないで来て下さい。	
\\	この 電車[でんしゃ]は8 時[じ]に 東京[とうきょう] 駅[えき]を 出[で]ますから 遅[おく]れないで 来[き]て 下[くだ]さい。	
\\	昨日のパーティーに来た人は、100人ぐらいだったと思います。	
\\	昨日[きのう]のパーティーに 来[き]た 人[ひと]は、100 人[にん]ぐらいだったと 思[おも]います。	
\\	安田さんの旅行の話は面白くて、時間の経つのも忘れたくらいだった。	
\\	安田[やすだ]さんの 旅行[りょこう]の 話[はなし]は 面白[おもしろ]くて、 時間[じかん]の 経[た]つのも 忘[わす]れたくらいだった。	
\\	恥ずかしくて穴があったら入りたいぐらいだった。	
\\	恥[は]ずかしくて 穴[あな]があったら 入[はい]りたいぐらいだった。	
\\	山下さんの新しい家の庭は、ゴルフ場ぐらいの大きさだ。	
\\	山下[やました]さんの 新[あたら]しい 家[いえ]の 庭[にわ]は、ゴルフ 場[じょう]ぐらいの 大[おお]きさだ。	
\\	自分の家くらい、いい場所はない。	
\\	自分[じぶん]の 家[いえ]くらい、いい 場所[ばしょ]はない。	
\\	来月は、一週間ほど九州へ出張します。	
\\	来月[らいげつ]は、一 週間[しゅうかん]ほど 九州[きゅうしゅう]へ 出張[しゅっちょう]します。	
\\	今度の事故で、100人ほどの人が死んだそうです。	
\\	今度[こんど]の 事故[じこ]で、100 人[にん]ほどの 人[ひと]が 死[し]んだそうです。	
\\	今年は去年ほど寒くないです。	
\\	今年[ことし]は 去年[きょねん]ほど 寒[さむ]くないです。	
\\	あの人ほど頭のいい人はいないでしょう。	
\\	あの 人[ひと]ほど 頭[あたま]のいい 人[ひと]はいないでしょう。	
\\	今日は勉強ができないほど疲れた。	
\\	今日[きょう]は 勉強[べんきょう]ができないほど 疲[つか]れた。	
\\	試験に合格したので、うれしくて眠れないほどです。	
\\	試験[しけん]に 合格[ごうかく]したので、うれしくて 眠[ねむ]れないほどです。	
\\	北へ行けば行くほど寒くなります。	
\\	北[きた]へ 行[い]けば 行[い]くほど 寒[さむ]くなります。	
\\	年を取れば取るほど、体が弱くなります。	
\\	年[とし]を 取[と]れば 取[と]るほど、 体[からだ]が 弱[よわ]くなります。	
\\	明日から2日ばかり旅行に行ってきます。	
\\	明日[あした]から 2日[ふつか]ばかり 旅行[りょこう]に 行[い]ってきます。	
\\	一万円ばかり貸していただけませんか。	
\\	一 万[まん] 円[えん]ばかり 貸[か]していただけませんか。	
\\	原田さんはピアノばかりでなく、歌もうまいんですよ。	
\\	原田[はらだ]さんはピアノばかりでなく、 歌[うた]もうまいんですよ。	
\\	英語ばかりでなく、フランス語も勉強したいんです。	
\\	英語[えいご]ばかりでなく、 フランス語[ふらんすご]も 勉強[べんきょう]したいんです。	
\\	課長はこの頃ウイスキーばかり飲んでいますね。	
\\	課長[かちょう]はこの 頃[ころ]ウイスキーばかり 飲[の]んでいますね。	
\\	課長はこの頃ウイスキーを飲んでばかりいますね。	
\\	課長[かちょう]はこの 頃[ころ]ウイスキーを 飲[の]んでばかりいますね。	
\\	課長はこの頃ウイスキー飲んでいるばかりですね。	
\\	課長[かちょう]はこの 頃[ころ]ウイスキー 飲[の]んでいるばかりですね。	
\\	テレビばかり見ていると目を悪くしますよ。	
\\	テレビばかり 見[み]ていると 目[め]を 悪[わる]くしますよ。	
\\	父は今帰ってきたばかりです。	
\\	父[ちち]は 今[いま] 帰[かえ]ってきたばかりです。	
\\	順ちゃんは、ご飯を食べたばかりなのに、もうおやつを欲しがっています。	
\\	順[じゅん]ちゃんは、ご 飯[はん]を 食[た]べたばかりなのに、もうおやつを 欲[ほ]しがっています。	
\\	渡辺さんはステレオを買いたいばかりに、一生懸命にアルバイトをしている。	
\\	渡辺[わたなべ]さんはステレオを 買[か]いたいばかりに、 一生懸命[いっしょうけんめい]にアルバイトをしている。	
\\	山田さんは政治家と結婚したばかりに、苦労している。	
\\	山田[やまだ]さんは 政治[せいじ] 家[か]と 結婚[けっこん]したばかりに、 苦労[くろう]している。	
\\	昨日クラスに来た学生は、5人だけでした。	
\\	昨日[きのう]クラスに 来[き]た 学生[がくせい]は、5 人[にん]だけでした。	
\\	今日は1時間だけテレビを見ました。	
\\	今日[きょう]は1 時間[じかん]だけテレビを 見[み]ました。	
\\	どうぞお好きなだけお飲みください。	
\\	どうぞお 好[す]きなだけお 飲[の]みください。	
\\	できるだけ早く行きます。	
\\	できるだけ 早[はや]く行きます。	
\\	あの大学に合格できたから、勉強しただけのことはあった。	
\\	あの 大学[だいがく]に 合格[ごうかく]できたから、 勉強[べんきょう]しただけのことはあった。	
\\	寺田さんは私のプレゼントを喜んでくれたので、無理して買っただけのことはあった。	
\\	寺田[てらだ]さんは 私[わたし]のプレゼントを 喜[よろこ]んでくれたので、 無理[むり]して 買[か]っただけのことはあった。	
\\	彼は彼女に夢中だっただけに、失恋のショックはとても大きかった。	
\\	彼[かれ]は 彼女[かのじょ]に 夢中[むちゅう]だっただけに、 失恋[しつれん]のショックはとても 大[おお]きかった。	
\\	一生懸命に勉強しただけに、不合格の通知を受け取ったとき、山本さんは非常にがっかりした。	
\\	一生懸命[いっしょうけんめい]に 勉強[べんきょう]しただけに、 不[ふ] 合格[ごうかく]の 通知[つうち]を 受け取[うけと]ったとき、 山本[やまもと]さんは 非常[ひじょう]にがっかりした。	
\\	佐藤さんは英国の大学で勉強しただけあって、英語がうまいですね。	
\\	佐藤[さとう]さんは 英国[えいこく]の 大学[だいがく]で 勉強[べんきょう]しただけあって、 英語[えいご]がうまいですね。	
\\	ジョンさんは、京都に15年も住んでいるだけあって、お寺のことをよく知っています。	
\\	ジョンさんは、 京都[きょうと]に15 年[ねん]も 住[す]んでいるだけあって、お 寺[てら]のことをよく 知[し]っています。	
\\	一流のピアニストだけに、素晴らしい演奏をしますね。	
\\	一流[いちりゅう]のピアニストだけに、 素晴[すば]らしい 演奏[えんそう]をしますね。	
\\	ここは北海道だけに、寒さが厳しいです。	
\\	ここは 北海道[ほっかいどう]だけに、 寒[さむ]さが 厳[きび]しいです。	
\\	あの店には、この雑誌しかありませんでした。	
\\	あの 店[みせ]には、この 雑誌[ざっし]しかありませんでした。	
\\	今は、1300円きりしか持っていないから、とてもフランス料理など食べられないよ。	
\\	今[いま]は、1300 円[えん]きりしか 持[も]っていないから、とてもフランス 料理[りょうり]など 食[た]べられないよ。	
\\	嫌だけれど、出張だから行くしかない。	
\\	嫌[いや]だけれど、 出張[しゅっちょう]だから 行[い]くしかない。	
\\	このレポートは、明日までだから、今日中に終わらせるしかない。	
\\	このレポートは、 明日[あした]までだから、 今日[きょう] 中[ちゅう]に 終[お]わらせるしかない。	
\\	この会議には、4つの国の代表のみが出席した。	
\\	この 会議[かいぎ]には、 4[よっ]つの 国[くに]の 代表[だいひょう]のみが 出席[しゅっせき]した。	
\\	以前、この大学には男性のみしか入れなかった。	
\\	以前[いぜん]、この 大学[だいがく]には 男性[だんせい]のみしか 入[はい]れなかった。	
\\	この大学の文学部の学生は、英語のみならずフランス語も勉強しなければならない。	
\\	この 大学[だいがく]の 文学部[ぶんがくぶ]の 学生[がくせい]は、 英語[えいご]のみならず フランス語[ふらんすご]も 勉強[べんきょう]しなければならない。	
\\	シェークスピアは戯曲のみならず詩もたくさん書いた。	
\\	シェークスピアは 戯曲[ぎきょく]のみならず 詩[し]もたくさん 書[か]いた。	
\\	あのおじいさんは、一人きりで大きな家に住んでいる。	
\\	あのおじいさんは、一 人[にん]きりで 大[おお]きな 家[いえ]に 住[す]んでいる。	
\\	その子供は、黙ったきりで何も言わなかった。	
\\	その 子供[こども]は、 黙[だま]ったきりで 何[なに]も 言[い]わなかった。	
\\	あと発車まで2分きりだから、山本さんはとても間に合わないだろう。	
\\	あと 発車[はっしゃ]まで 2分[にふん]きりだから、 山本[やまもと]さんはとても 間に合[まにあ]わないだろう。	
\\	お金は1万円きりしかないから、あのコンピューターを買うのは無理だ。	
\\	お 金[かね]は1 万[まん] 円[えん]きりしかないから、あのコンピューターを 買[か]うのは 無理[むり]だ。	
\\	それはもうどうしようもないよ。	
\\	それはもうどうしようもないよ。	
\\	上司が最後決定したんだ。今更どうしようもないだろう。	
\\	上司[じょうし]が 最後[さいご] 決定[けってい]したんだ。 今更[いまさら]どうしようもないだろう。	
\\	例えようもなく美しい物語です。	
\\	例[たと]えようもなく 美[うつく]しい 物語[ものがたり]です。	例えようもない= 
\\	これでは儲けようがない。	
\\	これでは 儲[もう]けようがない。	
\\	これが今の自分だし、変えようがない。	
\\	これが 今[いま]の 自分[じぶん]だし、 変[か]えようがない。	
\\	これ以上悪くなりようがないよ。	
\\	これ 以上[いじょう] 悪[わる]くなりようがないよ。	
\\	しかし、誰からも援助のしようがないと言われてしまった。	
\\	しかし、 誰[だれ]からも 援助[えんじょ]のしようがないと 言[い]われてしまった。	しようがない= 
\\	「それどころか」	
\\	「それどころか」	
\\	凶暴になるどころかリラックスできる。	
\\	凶暴[きょうぼう]になるどころかリラックスできる。	
\\	同性愛者は危険どころか本当の愛を知っている人たちだという意見もある。	
\\	同性愛[どうせいあい] 者[しゃ]は 危険[きけん]どころか 本当[ほんとう]の 愛[あい]を 知[し]っている 人[ひと]たちだという 意見[いけん]もある。	
\\	犯罪者の数は減らないどころか増えている。	
\\	犯罪[はんざい] 者[しゃ]の 数[かず]は 減[へ]らないどころか 増[ふ]えている。	
\\	このインクは温度変化に応じて変色します。	
\\	このインクは 温度[おんど] 変化[へんか]に 応[おう]じて 変色[へんしょく]します。	
\\	しかし、北朝鮮は応じていない。	
\\	しかし、 北朝鮮[きたちょうせん]は 応[おう]じていない。	
\\	イラクは、国連の査察を受け入れることで、この演説に応じた。	
\\	イラクは、 国連[こくれん]の 査察[ささつ]を 受け入[うけい]れることで、この 演説[えんぜつ]に 応[おう]じた。	
\\	ジョンさんは「ありがとう」と日本語で応じた。	
\\	ジョンさんは「ありがとう」と 日本語[にほんご]で 応[おう]じた。	
\\	あなたの指図に従って彼を首にしました。	
\\	あなたの 指図[さしず]に 従[したが]って 彼[かれ]を 首[くび]にしました。	首にする= 
\\	どうぞ以下の手順に従って下さい。	
\\	どうぞ 以下[いか]の 手順[てじゅん]に 従[したが]って 下[くだ]さい。	
\\	木材の縁はザラザラになりがちだ。	
\\	木材[もくざい]の 縁[へり]はザラザラになりがちだ。	縁=へり= 
\\	アランさんからは、去年クリスマス・カードが来たきりで、そのあと手紙が来ません。	
\\	アランさんからは、 去年[きょねん]クリスマス・カードが 来[き]たきりで、そのあと 手紙[てがみ]が 来[き]ません。	
\\	岸さんとは、先月のクラス会で会ったきりです。	
\\	岸[きし]さんとは、 先月[せんげつ]のクラス 会[かい]で 会[あ]ったきりです。	
\\	車の事故があったので、道が混んでいます。	
\\	車[くるま]の 事故[じこ]があったので、 道[みち]が 混[こ]んでいます。	
\\	雪がたくさん降ったので、電車が遅れるそうです。	
\\	雪[ゆき]がたくさん 降[ふ]ったので、 電車[でんしゃ]が 遅[おく]れるそうです。	
\\	病気なので、旅行に行くのは無理です。	
\\	病気[びょうき]なので、 旅行[りょこう]に 行[い]くのは 無理[むり]です。	
\\	東京は物価が高いもので、生活が大変です。	
\\	東京[とうきょう]は 物価[ぶっか]が 高[たか]いもので、 生活[せいかつ]が 大変[たいへん]です。	
\\	私は体が弱いもので、長い旅行は無理です。	
\\	私[わたし]は 体[からだ]が 弱[よわ]いもので、 長[なが]い 旅行[りょこう]は 無理[むり]です。	
\\	天気予報で今日は雨は降らないと言ったんですけれども、夕方から降ってきましたね。	
\\	天気[てんき] 予報[よほう]で 今日[きょう]は 雨[あめ]は 降[ふ]らないと 言[い]ったんですけれども、 夕方[ゆうがた]から 降[ふ]ってきましたね。	
\\	竹内さんに電話をかけたけれど留守でした。	
\\	竹内[たけうち]さんに 電話[でんわ]をかけたけれど 留守[るす]でした。	
\\	たまには旅行にも行きたいと思っているんですけど・・・	
\\	たまには 旅行[りょこう]にも 行[い]きたいと 思[おも]っているんですけど・・・	
\\	私はゴルフをしないわけではないんですけど・・・	
\\	私[わたし]はゴルフをしないわけではないんですけど・・・	
\\	まだ発車まで1時間もありますけど、どうしましょうか。	
\\	まだ 発車[はっしゃ]まで1 時間[じかん]もありますけど、どうしましょうか。	
\\	谷ですけど、知子さんいらっしゃいますか。	
\\	谷[たに]ですけど、 知子[ともこ]さんいらっしゃいますか。	
\\	早く暖かくなるといいんだけど・・・	
\\	早[はや]く 暖[あたた]かくなるといいんだけど・・・	
\\	もう少し大きいのが欲しいんだけれど・・・	
\\	もう 少[すこ]し 大[おお]きいのが 欲[ほ]しいんだけれど・・・	
\\	東京は雪が降ったところで、そんなに積もることはありません。	
\\	東京[とうきょう]は 雪[ゆき]が 降[ふ]ったところで、そんなに 積[つ]もることはありません。	
\\	「ところで」
\\	「ーた」
\\	あの人ならいくら頑張ったところで、この程度の仕事しかできないでしょう。	
\\	あの 人[ひと]ならいくら 頑張[がんば]ったところで、この 程度[ていど]の 仕事[しごと]しかできないでしょう。	
\\	「ところで」
\\	「ーた」
\\	今から急いで行ったところで、1時間の新幹線には間に合いませんよ。	
\\	今[いま]から 急[いそ]いで 行[い]ったところで、1 時間[じかん]の 新幹線[しんかんせん]には 間に合[まにあ]いませんよ。	
\\	「ところで」
\\	「ーた」
\\	高岡さんに頼んだところで、やってくれるはずがないでしょう。	
\\	高岡[たかおか]さんに 頼[たの]んだところで、やってくれるはずがないでしょう。	
\\	「ところで」
\\	「ーた」
\\	ここでお勉強なんかできないでしょ。	
\\	ここでお 勉強[べんきょう]なんかできないでしょ。	
\\	あなた、何かあるでしょ。	
\\	あなた、 何[なに]かあるでしょ。	
\\	あなた、こういうのは好きじゃなかったんでしょ。	
\\	あなた、こういうのは 好[す]きじゃなかったんでしょ。	
\\	あなた、数学を専攻したんでしょ。それ、暗算でできないの?	
\\	あなた、 数学[すうがく]を 専攻[せんこう]したんでしょ。それ、 暗算[あんざん]でできないの?	
\\	あなたにも、この気持ちは分かるでしょ。	
\\	あなたにも、この 気持[きも]ちは 分[わ]かるでしょ。	
\\	池田さんは風邪で咳が出るのに、タバコばかり吸っています。	
\\	池田[いけだ]さんは 風邪[かぜ]で 咳[せき]が 出[で]るのに、タバコばかり 吸[す]っています。	
\\	山本さんのパーティーには行かないと言ったのに、どうして行くんですか。	
\\	山本[やまもと]さんのパーティーには 行[い]かないと 言[い]ったのに、どうして 行[い]くんですか。	
\\	ここから湖へ行くのに何時間ぐらいかかりますか。	
\\	ここから 湖[みずうみ]へ 行[い]くのに 何[なん] 時間[じかん]ぐらいかかりますか。	
\\	漢字を覚えるのにいい方法を教えて下さい。	
\\	漢字[かんじ]を 覚[おぼ]えるのにいい 方法[ほうほう]を 教[おし]えて 下[くだ]さい。	
\\	子供のくせに、大人の話に口を出してはいけません。	
\\	子供[こども]のくせに、 大人[おとな]の 話[はなし]に 口[くち]を 出[だ]してはいけません。	
\\	あなただってできないくせに・・・	
\\	あなただってできないくせに・・・	
\\	自分でも分からないくせに・・・	
\\	自分[じぶん]でも 分[わ]からないくせに・・・	
\\	その国は独立したものの、まだ経済的な問題がたくさんある。	
\\	その 国[くに]は 独立[どくりつ]したものの、まだ 経済[けいざい] 的[てき]な 問題[もんだい]がたくさんある。	
\\	アメリカへ留学することに決めたものの、奨学金を取るのが難しい。	
\\	アメリカへ 留学[りゅうがく]することに 決[き]めたものの、 奨学[しょうがく] 金[きん]を 取[と]るのが 難[むずか]しい。	
\\	銀行へ行ったところが、もう閉まっていた。	
\\	銀行[ぎんこう]へ 行[い]ったところが、もう 閉[し]まっていた。	
\\	「ところが」
\\	「ーた」
\\	あの人に会いに行ったところが、会議中で会えなかった。	
\\	あの 人[ひと]に 会[あ]いに 行[い]ったところが、 会議[かいぎ] 中[ちゅう]で 会[あ]えなかった。	
\\	「ところが」
\\	「ーた」
\\	早稲田大学には合格できないと思っていたところが、合格通知が来た。	
\\	早稲田大学[わせだだいがく]には 合格[ごうかく]できないと 思[おも]っていたところが、 合格[ごうかく] 通知[つうち]が 来[き]た。	
\\	「ところが」
\\	「ーた」
\\	あまり期待していなかったところが、そのコンサートは素晴らしかった。	
\\	あまり 期待[きたい]していなかったところが、そのコンサートは 素晴[すば]らしかった。	
\\	「ところが」
\\	「ーた」
\\	明日天気がよければ、ドライブに行きましょう。	
\\	明日[あした] 天気[てんき]がよければ、ドライブに 行[い]きましょう。	
\\	六甲山に登れば、神戸の街がきれいに見えますよ。	
\\	六甲山[ろっこうざん]に 登[のぼ]れば、 神戸[こうべ]の 街[まち]がきれいに 見[み]えますよ。	
\\	よく練習すれば、このピアノ曲が弾けるようになります。	
\\	よく 練習[れんしゅう]すれば、このピアノ 曲[きょく]が 弾[ひ]けるようになります。	
\\	よろしければ、お菓子を召し上がって下さい。	
\\	よろしければ、お 菓子[かし]を 召し上[めしあ]がって 下[くだ]さい。	
\\	簡単に言えば、それは無理だということでしょう。	
\\	簡単[かんたん]に 言[い]えば、それは 無理[むり]だということでしょう。	
\\	できれば明後日の方が私は都合がいいんですが・・・	
\\	できれば 明後日[あさって]の 方[ほう]が 私[わたし]は 都合[つごう]がいいんですが・・・	
\\	今日は、天気もよければ風もないで、お花見には最適です。	
\\	今日[きょう]は、 天気[てんき]もよければ 風[かぜ]もないで、お 花見[はなみ]には 最適[さいてき]です。	
\\	戦後は米もなければ野菜もないで、大変でしたよ。	
\\	戦後[せんご]は 米[こめ]もなければ 野菜[やさい]もないで、 大変[たいへん]でしたよ。	
\\	本を借りるには、ここに名前を書けばいいんです。	
\\	本[ほん]を 借[か]りるには、ここに 名前[なまえ]を 書[か]けばいいんです。	
\\	確かに宿題を忘れたのは君の責任ですが、先生に謝りさえすればいいんです。	
\\	確[たし]かに 宿題[しゅくだい]を 忘[わす]れたのは 君[きみ]の 責任[せきにん]ですが、 先生[せんせい]に 謝[あやま]りさえすればいいんです。	
\\	いつ、どこで会いましょうか?	
\\	いつ、どこで 会[あ]いましょうか?	
\\	その料理があまり辛かったら、私は食べないわ。	
\\	その 料理[りょうり]があまり 辛[つら]かったら、 私[わたし]は 食[た]べないわ。	
\\	彼に会ったら、よろしくと言って下さい。	
\\	彼[かれ]に 会[あ]ったら、よろしくと 言[い]って 下[くだ]さい。	
\\	山田さんの都合が悪かったら、誰にワープロを頼みましょうか。	
\\	山田[やまだ]さんの 都合[つごう]が 悪[わる]かったら、 誰[だれ]にワープロを 頼[たの]みましょうか。	
\\	もう遅いから、その仕事明日になさったら。	
\\	もう 遅[おそ]いから、その 仕事[しごと] 明日[あした]になさったら。	「たら」
\\	それは小さいから、こちらの大きいのをお買いになったら。	
\\	それは 小[ちい]さいから、こちらの 大[おお]きいのをお 買[か]いになったら。	「たら」
\\	友達の家へ行ったら、彼は留守だった。	
\\	友達[ともだち]の 家[いえ]へ 行[い]ったら、 彼[かれ]は 留守[るす]だった。	
\\	ホテルに電話をしたら、部屋はいっぱいだった。	
\\	ホテルに 電話[でんわ]をしたら、 部屋[へや]はいっぱいだった。	
\\	私が声をかけたら来て下さい。	
\\	私[わたし]が 声[こえ]をかけたら 来[き]て 下[くだ]さい。	
\\	この仕事が終わったら、そちらへ行きます。	
\\	この 仕事[しごと]が 終[お]わったら、そちらへ 行[い]きます。	
\\	明日雨なら、ゴルフに行かないつもりです。	
\\	明日[あした] 雨[あめ]なら、ゴルフに 行[い]かないつもりです。	
\\	乗るなら飲むな。飲んだら乗るな。	
\\	乗[の]るなら 飲[の]むな。 飲[の]んだら 乗[の]るな。	
\\	その問題なら、もう解決しました。	
\\	その 問題[もんだい]なら、もう 解決[かいけつ]しました。	
\\	あなたにできるものなら、やってみて下さい。	
\\	あなたにできるものなら、やってみて 下[くだ]さい。	
\\	初めてのデートでは手をつなぐだけだ。	
\\	初[はじ]めてのデートでは 手[て]をつなぐだけだ。	
\\	彼らは人前で手さえつなぎません。	
\\	彼[かれ]らは 人前[ひとまえ]で 手[て]さえつなぎません。	
\\	彼につなぎます。	
\\	(電話で)	彼[かれ]につなぎます。	
\\	それは見た目ほど悪くない。	
\\	それは 見た目[みため]ほど 悪[わる]くない。	見た目= 
\\	比較して言えば、私は昨日よりもずっと調子がいい。	
\\	比較[ひかく]して 言[い]えば、 私[わたし]は 昨日[きのう]よりもずっと 調子[ちょうし]がいい。	
\\	20〜30年前と比較して、試合はどのようにかわりましたか。	
\\	20〜30 年[ねん] 前[まえ]と 比較[ひかく]して、 試合[しあい]はどのようにかわりましたか。	
\\	ビジネス業界そのものが変わってきているのです。	
\\	ビジネス 業界[ぎょうかい]そのものが 変[か]わってきているのです。	
\\	そのもの 
\\	彼は情報そのものが間違っていると言った。	
\\	彼[かれ]は 情報[じょうほう]そのものが 間違[まちが]っていると 言[い]った。	
\\	いかにもあなたらしいね。	
\\	いかにもあなたらしいね。	
\\	じっと我慢するしかないでしょう。	
\\	じっと 我慢[がまん]するしかないでしょう。	
\\	あなたは退屈に耐えられません。	
\\	あなたは 退屈[たいくつ]に 耐[た]えられません。	
\\	今日は交通に耐えられる気分じゃない。うちにいようよ。	
\\	今日[きょう]は 交通[こうつう]に 耐[た]えられる 気分[きぶん]じゃない。うちにいようよ。	
\\	これらの花は寒さに耐えられなくて冬には枯れます。	
\\	これらの 花[はな]は 寒[さむ]さに 耐[た]えられなくて 冬[ふゆ]には 枯[か]れます。	枯れる= 
\\	私は不器用です。	
\\	私[わたし]は 不器用[ぶきよう]です。	
\\	その子犬は歩くとき、とてもぎこちなかった。	
\\	その 子犬[こいぬ]は 歩[ある]くとき、とてもぎこちなかった。	
\\	たまに贅沢をするのは少しも悪いことではない。	
\\	たまに 贅沢[ぜいたく]をするのは 少[すこ]しも 悪[わる]いことではない。	
\\	たまには彼女も感謝の気持ちを表してほしいものです。	
\\	たまには 彼女[かのじょ]も 感謝[かんしゃ]の 気持[きも]ちを 表[あらわ]してほしいものです。	
\\	お互いを知るようになるにつれて、私たちは親友になった。	
\\	お 互[たが]いを 知[し]るようになるにつれて、 私[わたし]たちは 親友[しんゆう]になった。	
\\	これは自分の直感を信頼することを学ぶにつれて、楽にできるようになっていく。	
\\	これは 自分[じぶん]の 直感[ちょっかん]を 信頼[しんらい]することを 学[まな]ぶにつれて、 楽[らく]にできるようになっていく。	直感= 
\\	信頼= 
\\	政治家がそんなばかなことをしようものなら、国民は黙っていませんよ。	
\\	政治[せいじ] 家[か]がそんなばかなことをしようものなら、 国民[こくみん]は 黙[だま]っていませんよ。	
\\	課長がそんなやり方をしようものなら、部下は課長を全然信頼しなくなるでしょうね。	
\\	課長[かちょう]がそんなやり 方[かた]をしようものなら、 部下[ぶか]は 課長[かちょう]を 全然[ぜんぜん] 信頼[しんらい]しなくなるでしょうね。	
\\	これ以上仕事を続けようものなら、あなたは死んでしまいますよ。	
\\	これ 以上[いじょう] 仕事[しごと]を 続[つづ]けようものなら、あなたは 死[し]んでしまいますよ。	
\\	空を飛べるものなら飛んでみたい。	
\\	空[そら]を 飛[と]べるものなら 飛[と]んでみたい。	
\\	会社へ行かなくていいものなら行きたくない。	
\\	会社[かいしゃ]へ 行[い]かなくていいものなら 行[い]きたくない。	
\\	これから出かけるところなので、ゆっくり話をする時間はありません。	
\\	これから 出[で]かけるところなので、ゆっくり 話[はなし]をする 時間[じかん]はありません。	
\\	山田さんに電話をするところですが、何か伝言はありませんか。	
\\	山田[やまだ]さんに 電話[でんわ]をするところですが、 何[なに]か 伝言[でんごん]はありませんか。	
\\	今手紙を書いているところです。	
\\	今[こん] 手紙[てがみ]を 書[か]いているところです。	
\\	役員は、今その問題を検討しているところです。	
\\	役員[やくいん]は、 今[いま]その 問題[もんだい]を 検討[けんとう]しているところです。	
\\	彼は今成田に着いたところです。	
\\	彼[かれ]は 今[こん] 成田[なりた]に 着[つ]いたところです。	
\\	これは5年連続の減少です。	
\\	これは5 年[ねん] 連続[れんぞく]の 減少[げんしょう]です。	
\\	広田さんは、今日フランスから帰国したところですよ。	
\\	広田[ひろた]さんは、 今日[きょう]フランスから 帰国[きこく]したところですよ。	
\\	デパートに問い合わせてみたところ、その品物は売り切れだった。	
\\	デパートに 問い合[といあ]わせてみたところ、その 品物[しなもの]は 売り切[うりき]れだった。	
\\	大学の図書館で調べたところ、その作家は詩も書いていたことが分かった。	
\\	大学[だいがく]の 図書館[としょかん]で 調[しら]べたところ、その 作家[さっか]は 詩[し]も 書[か]いていたことが 分[わ]かった。	
\\	お知らせ下されば、病院へお見舞いに参りましたものを・・・	
\\	お 知[し]らせ 下[くだ]されば、 病院[びょういん]へお 見舞[みま]いに 参[まい]りましたものを・・・	
\\	私にできることでしたら、お手伝いしましたものを・・・	
\\	私[わたし]にできることでしたら、お 手伝[てつだ]いしましたものを・・・	
\\	毎朝テレビを見ながら、朝ご飯を食べますよ。	
\\	毎朝[まいあさ]テレビを 見[み]ながら、 朝[あさ]ご 飯[はん]を 食[た]べますよ。	
\\	青木さんはいつも音楽を聞きながら勉強している。	
\\	青木[あおき]さんはいつも 音楽[おんがく]を 聞[き]きながら 勉強[べんきょう]している。	
\\	高木さんは体が弱いと言いながら、よく夜遅くまで酒を飲んでいる。	
\\	高木[たかぎ]さんは 体[からだ]が 弱[よわ]いと 言[い]いながら、よく 夜[よる] 遅[おそ]くまで 酒[さけ]を 飲[の]んでいる。	
\\	あの先生は学生には遅刻をしないようにと言いながら、自分はいつも遅れて学校へ来る。	
\\	あの 先生[せんせい]は 学生[がくせい]には 遅刻[ちこく]をしないようにと 言[い]いながら、 自分[じぶん]はいつも 遅[おく]れて 学校[がっこう]へ 来[く]る。	
\\	日曜日は大抵友達とテニスをしたり、映画を見に行ったりします。	
\\	日曜日[にちようび]は 大抵[たいてい] 友達[ともだち]とテニスをしたり、 映画[えいが]を 見[み]に 行[い]ったりします。	
\\	旅行中は美術館に行ったりお土産を買ったりで、あっという間に時間がなくなりました。	
\\	旅行[りょこう] 中[ちゅう]は 美術館[びじゅつかん]に 行[い]ったりお 土産[みやげ]を 買[か]ったりで、あっという 間[ま]に 時間[じかん]がなくなりました。	
\\	天気の悪い日には、家で音楽を聞いたりします。	
\\	天気[てんき]の 悪[わる]い 日[ひ]には、 家[いえ]で 音楽[おんがく]を 聞[き]いたりします。	
\\	午前中は漢字を書いたり読んだりする。	
\\	午前[ごぜん] 中[ちゅう]は 漢字[かんじ]を 書[か]いたり 読[よ]んだりする。	
\\	この頃その俳優のことをテレビや雑誌で聞いたり見たりします。	
\\	この 頃[ころ]その 俳優[はいゆう]のことをテレビや 雑誌[ざっし]で 聞[き]いたり 見[み]たりします。	
\\	そんなにテレビをつけたり消したりしないでちょうだい。	
\\	そんなにテレビをつけたり 消[け]したりしないでちょうだい。	
\\	今週、株は上がったり下がったりしています。	
\\	今週[こんしゅう]、 株[かぶ]は 上[あ]がったり 下[さ]がったりしています。	
\\	父は体の調子によって、ゴルフに行ったり行かなかったりします。	
\\	父[ちち]は 体[からだ]の 調子[ちょうし]によって、ゴルフに 行[い]ったり 行[い]かなかったりします。	
\\	お金を少々貸してもらえませんか。	
\\	お 金[かね]を 少々[しょうしょう] 貸[か]してもらえませんか。	
\\	ペンを貸してもらえますか?	
\\	ペンを 貸[か]してもらえますか?	
\\	友達がこの本を貸してくれました。	
\\	友達[ともだち]がこの 本[ほん]を 貸[か]してくれました。	
\\	手を貸してくれませんか?	
\\	手[て]を 貸[か]してくれませんか?	
\\	その本を一気に読んだ。	
\\	その 本[ほん]を 一気[いっき]に 読[よ]んだ。	一気= 
\\	急に雨が降りだしました。	
\\	急[きゅう]に 雨[あめ]が 降[ふ]りだしました。	
\\	急に食欲がなくなった。	
\\	急[きゅう]に 食欲[しょくよく]がなくなった。	
\\	これらの問題を一挙に解決する良い方法はないか?	
\\	これらの 問題[もんだい]を 一挙[いっきょ]に 解決[かいけつ]する 良[よ]い 方法[ほうほう]はないか?	
\\	やるからには、ちゃんと頑張れよ。	
\\	やるからには、ちゃんと 頑張[がんば]れよ。	
\\	高い給料をもらうからにはきつい仕事をするのは当然です。	
\\	高[たか]い 給料[きゅうりょう]をもらうからにはきつい 仕事[しごと]をするのは 当然[とうぜん]です。	きつい= 
\\	「外国人」という言葉がとにかく嫌い。	
\\	外国[がいこく] 人[じん]」という 言葉[ことば]がとにかく 嫌[きら]い。	
\\	とにかくありがとう。	
\\	とにかくありがとう。	
\\	その詩人は、美しい隠喩で有名です。	
\\	その 詩人[しじん]は、 美[うつく]しい 隠喩[いんゆ]で 有名[ゆうめい]です。	隠喩=いんゆ= 
\\	このプログラムの目的は、二国間の文化交流を促進することです。	
\\	このプログラムの 目的[もくてき]は、二 国[こく] 間[かん]の 文化[ぶんか] 交流[こうりゅう]を 促進[そくしん]することです。	
\\	「バブル経済」が崩壊した。	
\\	「バブル 経済[けいざい]」が 崩壊[ほうかい]した。	崩壊=ほうかい= 
\\	そのビルは崩壊する危険がある。	
\\	そのビルは 崩壊[ほうかい]する 危険[きけん]がある。	崩壊=ほうかい= 
\\	こうしたアイヌ民族の受難は、世界の他の先住民族に共通するものです。	
\\	こうしたアイヌ 民族[みんぞく]の 受難[じゅなん]は、 世界[せかい]の 他[ほか]の 先住民[せんじゅうみん] 族[ぞく]に 共通[きょうつう]するものです。	受難=
\\	この気持ちは、多くの日本人に共通しているでしょう。	
\\	この 気持[きも]ちは、 多[おお]くの日本人に 共通[きょうつう]しているでしょう。	
\\	これらの人々に共通するものとは何でしょうか。	
\\	これらの 人々[ひとびと]に 共通[きょうつう]するものとは 何[なに]でしょうか。	
\\	放課後はサッカーをしています。	
\\	放課後[ほうかご]はサッカーをしています。	
\\	一層のご努力をお願いします。	
\\	一層[いっそう]のご 努力[どりょく]をお 願[ねが]いします。	
\\	これらの問題は一層の作業を必要とする。	
\\	これらの 問題[もんだい]は 一層[いっそう]の 作業[さぎょう]を 必要[ひつよう]とする。	
\\	休みの間には1日何回メールチェックをしていましたか。	
\\	休[やす]みの 間[ま]には1 日[にち] 何[なん] 回[かい]メールチェックをしていましたか。	
\\	滞在している間は何をするつもりですか?	
\\	滞在[たいざい]している 間[あいだ]は 何[なに]をするつもりですか?	滞在= 
\\	私がいない間にあなたがしたことを全て話してください。	
\\	私[わたし]がいない 間[ま]にあなたがしたことを 全[すべ]て 話[はな]してください。	
\\	あなたがいない間、彼が全ての仕事を引き受けます。	
\\	あなたがいない 間[あいだ]、 彼[かれ]が 全[すべ]ての 仕事[しごと]を 引き受[ひきう]けます。	引き受ける= 
\\	ルーシーがうそをつくたびに、1ドル請求することにしたの。	
\\	ルーシーがうそをつくたびに、1ドル 請求[せいきゅう]することにしたの。	請求= 
\\	会うたびにかわいくなるね。	
\\	会[あ]うたびにかわいくなるね。	
\\	「的に」は主語の後に加えられる。	
\\	的[てき]に」は 主語[しゅご]の 後[のち]に 加[くわ]えられる。	
\\	その会議の後に軽食が出されます。	
\\	その 会議[かいぎ]の 後[のち]に 軽食[けいしょく]が 出[だ]されます。	
\\	なぜ外国人男性は道を聞いた後に電話番号を聞くの?	
\\	なぜ 外国[がいこく] 人[じん] 男性[だんせい]は 道[みち]を 聞[き]いた 後[あと]に 電話[でんわ] 番号[ばんごう]を 聞[き]くの?	
\\	彼と会った後に手を洗った?	
\\	彼[かれ]と 会[あ]った 後[のち]に 手[て]を 洗[あら]った?	
\\	2005年に初めて訪れて以来、 平井[ひらい]さんはカンボジアへ9回も行っている。	
\\	年[ねん]に 初[はじ]めて 訪[おとず]れて 以来[いらい]、 平井[ひらい]さんはカンボジアへ9 回[かい]も 行[い]っている。	訪れる=おとずれる= (訪問する) 
\\	(到来する) 
\\	この映画は4月に公開されて以来、英国で大ヒットしています。	
\\	この 映画[えいが]は 4月[しがつ]に 公開[こうかい]されて 以来[いらい]、 英国[えいこく]で 大[だい]ヒットしています。	公開= 
\\	この本は、2008年1月に発売されて以来、100万部以上を売り上げています。	
\\	この 本[ほん]は、2008 年[ねん] 1月[いちがつ]に 発売[はつばい]されて 以来[いらい]、100 万[まん] 部[ぶ] 以上[いじょう]を 売り上[うりあ]げています。	
\\	うわあ、買ったばかりのコンピューターが壊れちゃったよ。	
\\	うわあ、 買[か]ったばかりのコンピューターが 壊[こわ]れちゃったよ。	
\\	エゴがあるうちは、人は自分のことばかり考える。	
\\	エゴがあるうちは、 人[ひと]は 自分[じぶん]のことばかり 考[かんが]える。	
\\	明日は、朝のうちは晴れますが、午後には曇が出てきます。	
\\	明日[あした]は、 朝[あさ]のうちは 晴[は]れますが、 午後[ごご]には 曇[くも]が 出[で]てきます。	
\\	最初のうちは、みんなお互いに何度も顔を見合わせた。	
\\	最初[さいしょ]のうちは、みんなお 互[たが]いに 何[なん] 度[ど]も 顔[かお]を 見合[みあ]わせた。	
\\	良い気持ちで眠りかけていたところに、彼から電話がかかってきた。	
\\	良[よ]い 気持[きも]ちで 眠[ねむ]りかけていたところに、 彼[かれ]から 電話[でんわ]がかかってきた。	
\\	あなたの場合、ショートヘアも似合うと思いますけど。	
\\	あなたの 場合[ばあい]、ショートヘアも 似合[にあ]うと 思[おも]いますけど。	
\\	それは、大抵の場合、とてもうまく機能します。	
\\	それは、 大抵[たいてい]の 場合[ばあい]、とてもうまく 機能[きのう]します。	
\\	また、少年の場合、刑は極めて軽い。	
\\	また、 少年[しょうねん]の 場合[ばあい]、 刑[けい]は 極[きわ]めて 軽[かる]い。	
\\	彼女に真実を伝える以外に道はないようだ。	
\\	彼女[かのじょ]に 真実[しんじつ]を 伝[つた]える 以外[いがい]に 道[みち]はないようだ。	
\\	それ以外に説明のしようがない。	
\\	それ 以外[いがい]に 説明[せつめい]のしようがない。	しようがない= 
\\	これ以外には文字も発音もない。	
\\	これ 以外[いがい]には 文字[もじ]も 発音[はつおん]もない。	
\\	自分以外に非難する人は誰もいない。	
\\	自分[じぶん] 以外[いがい]に 非難[ひなん]する 人[ひと]は 誰[だれ]もいない。	
\\	食べたいだけ食べなさい。	
\\	食[た]べたいだけ 食[た]べなさい。	
\\	居たいだけ居てください。	
\\	居[い]たいだけ 居[い]てください。	
\\	眠りたいだけ眠らせてあげましょう。	
\\	眠[ねむ]りたいだけ 眠[ねむ]らせてあげましょう。	
\\	お好きなだけご覧ください。	
\\	お 好[す]きなだけご 覧[らん]ください。	
\\	パパは好きなだけ遊ばせてくれるよ。	
\\	パパは 好[す]きなだけ 遊[あそ]ばせてくれるよ。	
\\	彼は1日中寝てばかりいる。	
\\	彼[かれ]は1 日[にち] 中[ちゅう] 寝[ね]てばかりいる。	
\\	彼は人を批判してばかりいる。	
\\	彼[かれ]は 人[ひと]を 批判[ひはん]してばかりいる。	
\\	怒ってばかりいるのは不健康だ。	
\\	怒[おこ]ってばかりいるのは 不健康[ふけんこう]だ。	
\\	彼は怒鳴ってばかりいます。	
\\	彼[かれ]は 怒鳴[どな]ってばかりいます。	
\\	あまり考え過ぎると、頭がおかしくなってしまうよ。	
\\	あまり 考[かんが]え 過[す]ぎると、 頭[あたま]がおかしくなってしまうよ。	
\\	あまり心配し過ぎると早く年を取るよ。	
\\	あまり 心配[しんぱい]し 過[す]ぎると 早[はや]く 年[とし]を 取[と]るよ。	
\\	あなたは頭が切れて、怖いくらいだ。	
\\	あなたは 頭[あたま]が 切[き]れて、 怖[こわ]いくらいだ。	頭が切れる= 
\\	うちの犬はすごく私に似ていて、見た目もそっくりだと言われるくらいです。	
\\	うちの 犬[いぬ]はすごく 私[わたし]に 似[に]ていて、 見た目[みため]もそっくりだと 言[い]われるくらいです。	そっくり= 
\\	彼はひどく反宗教的で、教会を嫌うほどだ。	
\\	彼[かれ]はひどく 反[はん] 宗教[しゅうきょう] 的[てき]で、 教会[きょうかい]を 嫌[きら]うほどだ。	
\\	彼は仕事に慣れきっているので、眠りながらでも十分働けるほどだ。	
\\	彼[かれ]は 仕事[しごと]に 慣[な]れきっているので、 眠[ねむ]りながらでも 十分[じゅうぶん] 働[はたら]けるほどだ。	
\\	私はこの本を何度も何度も読んだので、眠りながら暗唱できるほどです。	
\\	私[わたし]はこの 本[ほん]を 何[なん] 度[ど]も 何[なん] 度[ど]も 読[よ]んだので、 眠[ねむ]りながら 暗唱[あんしょう]できるほどです。	暗唱= 
\\	このとき彼は40代半ばで、これから50年も生きるはずだった。	
\\	このとき 彼[かれ]は40 代[だい] 半[なか]ばで、これから50 年[ねん]も 生[い]きるはずだった。	
\\	これは起こらないはずだった。	
\\	これは 起[お]こらないはずだった。	
\\	5月末までに荷物を受け取るはずだったのですが。	
\\	5月[ごがつ] 末[すえ]までに 荷物[にもつ]を 受け取[うけと]るはずだったのですが。	
\\	こんなことは私の町では起こるはずがない。	
\\	こんなことは 私[わたし]の 町[まち]では 起[お]こるはずがない。	
\\	こんなとんでもない時間に彼は来るはずがない。	
\\	こんなとんでもない 時間[じかん]に 彼[かれ]は 来[く]るはずがない。	とんでもない= 
\\	それは真実であるはずがない。	
\\	それは 真実[しんじつ]であるはずがない。	
\\	そんなことが自分に起こるはずはない。	
\\	そんなことが 自分[じぶん]に 起[お]こるはずはない。	
\\	他の人はともなく、あなたに限って盗みを働くはずがないと思っていました。	
\\	他[た]の 人[ひと]はともなく、あなたに 限[かぎ]って 盗[ぬす]みを 働[はたら]くはずがないと 思[おも]っていました。	盗みを働く= 
\\	私のお願いはまだ一つもかなっていないが、まだ時間がある。	
\\	私[わたし]のお 願[ねが]いはまだ 一[ひと]つもかなっていないが、まだ 時間[じかん]がある。	かなう= 
\\	周りには他に外国人が一人もいない。	
\\	周[まわ]りには 他[ほか]に 外国[がいこく] 人[じん]が 一人[ひとり]もいない。	
\\	くしゃみを止められそうにない。	
\\	くしゃみを 止[と]められそうにない。	
\\	その手の哲学の問題は難解過ぎて、とても理解できそうにない。	
\\	その 手[て]の 哲学[てつがく]の 問題[もんだい]は 難解[なんかい] 過[す]ぎて、とても 理解[りかい]できそうにない。	その手= 
\\	この雪はやみそうにない。	
\\	この 雪[ゆき]はやみそうにない。	
\\	かなり酔っ払っているので、今夜は眠れそうもない。	
\\	かなり 酔っ払[よっぱら]っているので、 今夜[こんや]は 眠[ねむ]れそうもない。	
\\	そのプロジェクトは予定通りに修了そうもない。	
\\	そのプロジェクトは 予定[よてい] 通[どお]りに 修了[しゅうりょう]そうもない。	
\\	彼は驚きのあまり心臓が胸から飛び出しそうになった。	
\\	彼[かれ]は 驚[おどろ]きのあまり 心臓[しんぞう]が 胸[むね]から 飛び出[とびだ]しそうになった。	
\\	私はこのパンでのどが詰まりそうになった。	
\\	私[わたし]はこのパンでのどが 詰[つ]まりそうになった。	
\\	車にひかれそうになった。	
\\	車[くるま]にひかれそうになった。	
\\	ひどくおなかがすいているから、あの食べ物がとても美味しそうに見える。	
\\	ひどくおなかがすいているから、あの 食べ物[たべもの]がとても 美味[おい]しそうに 見[み]える。	
\\	彼は、最後に会ったときと比べてずっと幸せそうに見える。	
\\	彼[かれ]は、 最後[さいご]に 会[あ]ったときと 比[くら]べてずっと 幸[しあわ]せそうに 見[み]える。	
\\	君の睡眠を妨げたくなかったんだ。あまりにも幸せそうに見えたから。	
\\	君[きみ]の 睡眠[すいみん]を 妨[さまた]げたくなかったんだ。あまりにも 幸[しあわ]せそうに 見[み]えたから。	睡眠= 
\\	睡眠を妨げる= 
\\	その話は本当のような気がする。	
\\	その 話[はなし]は 本当[ほんとう]のような 気[き]がする。	
\\	お金のためだけに働くのではない。	
\\	お 金[かね]のためだけに 働[はたら]くのではない。	
\\	時代が変わったのではない。	
\\	時代[じだい]が 変[か]わったのではない。	
\\	あなたのために残ったのではない。	
\\	あなたのために 残[のこ]ったのではない。	
\\	実は、私が決めたのではないんです。	
\\	実[じつ]は、 私[わたし]が 決[き]めたのではないんです。	
\\	おじぎは謝るときだけにするのではありません。	
\\	おじぎは 謝[あやま]るときだけにするのではありません。	
\\	これは、発展途上国だけで起こるのではありません。	
\\	これは、 発展[はってん] 途上[とじょう] 国[こく]だけで 起[お]こるのではありません。	
\\	学校のことで電話をかけているんじゃないんです。	
\\	学校[がっこう]のことで 電話[でんわ]をかけているんじゃないんです。	
\\	エッセーは先生が指定した長さに満たないんですが、これで大丈夫でしょうか?	
\\	エッセーは 先生[せんせい]が 指定[してい]した 長[なが]さに 満[み]たないんですが、これで 大丈夫[だいじょうぶ]でしょうか?	
\\	ダイエットしないといけないんですが、していません。	
\\	ダイエットしないといけないんですが、していません。	
\\	主人の携帯にかけたんですが、スイッチを切っているんです。	
\\	主人[しゅじん]の 携帯[けいたい]にかけたんですが、スイッチを 切[き]っているんです。	
\\	2人なんですけど、テーブルはありますか。	
\\	人[にん]なんですけど、テーブルはありますか。	
\\	いつかパレスチンに戻りたいんです。	
\\	いつかパレスチンに 戻[もど]りたいんです。	
\\	日本にファクスを送りたいんですけど、部屋のファクス機の使い方がよく分からないのですが。	
\\	日本[にほん]にファクスを 送[おく]りたいんですけど、 部屋[へや]のファクス 機[き]の 使い方[つかいかた]がよく 分[わ]からないのですが。	
\\	かくなる上は、真実を話すしかない。	
\\	かくなる 上[うえ]は、 真実[しんじつ]を 話[はな]すしかない。	かくなる上は= 
\\	こうした違いは当たり前なので、受け入れるしかない。	
\\	こうした 違[ちが]いは 当たり前[あたりまえ]なので、 受け入[うけい]れるしかない。	
\\	あなたのアドバイスを聞いておけばよかったです。	
\\	あなたのアドバイスを 聞[き]いておけばよかったです。	
\\	お役に立てればよかったのですが。	
\\	お 役[やく]に 立[た]てればよかったのですが。	
\\	それらの株を売っておけばよかった。	
\\	それらの 株[かぶ]を 売[う]っておけばよかった。	
\\	もっと早くあなたに聞けばよかった。	
\\	もっと 早[はや]くあなたに 聞[き]けばよかった。	
\\	もっと若い時から水泳を楽しんでいればよかった。	
\\	もっと 若[わか]い 時[とき]から 水泳[すいえい]を 楽[たの]しんでいればよかった。	
\\	カメラを持ってくればよかった。	
\\	カメラを 持[も]ってくればよかった。	
\\	傘を持ってくればよかった。	
\\	傘[かさ]を 持[も]ってくればよかった。	
\\	アメリカならせいぜい30分も待てばいいのに、日本では4時間ばかり待たされることもある。	
\\	アメリカならせいぜい30 分[ふん]も 待[ま]てばいいのに、日本では4 時間[じかん]ばかり 待[ま]たされることもある。	
\\	分からないことがあればその場で聞けばいいのに、聞かないで後で図書館で何時間もかけて調べる。	
\\	分[わ]からないことがあればその 場[ば]で 聞[き]けばいいのに、 聞[き]かないで 後[あと]で 図書館[としょかん]で 何[なん] 時間[じかん]もかけて 調[しら]べる。	
\\	それが真実だといいのに。	
\\	それが 真実[しんじつ]だといいのに。	
\\	あんなふうになれたらいいのにと思います。	
\\	あんなふうになれたらいいのにと 思[おも]います。	
\\	うちの娘があの学校に行けたらいいのに・・・	
\\	うちの 娘[むすめ]があの 学校[がっこう]に 行[い]けたらいいのに・・・	
\\	そこにいられたらいいのになあ。	
\\	そこにいられたらいいのになあ。	
\\	そんなふうに踊れたらいいのになあ。	
\\	そんなふうに 踊[おど]れたらいいのになあ。	
\\	あなたを傷つけるつもりはない。	
\\	あなたを 傷[きず]つけるつもりはない。	
\\	お邪魔をするつもりはないんだけど。	
\\	お 邪魔[じゃま]をするつもりはないんだけど。	
\\	これを独りで処理するつもりはない。	
\\	これを 独[ひと]りで 処理[しょり]するつもりはない。	
\\	やめるつもりはないね。	
\\	やめるつもりはないね。	
\\	まだあきらめるつもりはない。	
\\	まだあきらめるつもりはない。	
\\	2階で物音がする。	
\\	階[かい]で 物音[ものおと]がする。	
\\	このいすは座ると変な音がするね。	
\\	このいすは 座[すわ]ると 変[へん]な 音[おと]がするね。	
\\	車から変な音がするのですが。	
\\	車[くるま]から 変[へん]な 音[おと]がするのですが。	
\\	そのことをきっかけに彼は写真に夢中になった。	
\\	そのことをきっかけに 彼[かれ]は 写真[しゃしん]に 夢中[むちゅう]になった。	
\\	妻の病気をきっかけに、家族の結束は再び固まりました。	
\\	妻[つま]の 病気[びょうき]をきっかけに、 家族[かぞく]の 結束[けっそく]は 再[ふたた]び 固[かた]まりました。	結束= 
\\	固まる= 
\\	彼のことを知れば知るほどますます好きになる。	
\\	彼[かれ]のことを 知[し]れば 知[し]るほどますます 好[す]きになる。	
\\	考えれば考えるほど、怒りが増しました。	
\\	考[かんが]えれば 考[かんが]えるほど、 怒[いか]りが 増[ま]しました。	
\\	大抵の人が、その件について意見を持っている。	
\\	大抵[たいてい]の 人[ひと]が、その 件[けん]について 意見[いけん]を 持[も]っている。	
\\	学べば学ぶほど、私はいかに自分が何も知らないかをますます実感します。	
\\	学[まな]べば 学[まな]ぶほど、 私[わたし]はいかに 自分[じぶん]が 何[なに]も 知[し]らないかをますます 実感[じっかん]します。	
\\	要点は何でしょうか。	
\\	要点[ようてん]は 何[なに]でしょうか。	
\\	要点はまさにそこだ。	
\\	要点[ようてん]はまさにそこだ。	まさに= 
\\	私は携帯電話の電源を切るのを忘れた。	
\\	私[わたし]は 携帯[けいたい] 電話[でんわ]の 電源[でんげん]を 切[き]るのを 忘[わす]れた。	
\\	その概念は少し抽象的です。	
\\	その 概念[がいねん]は 少[すこ]し 抽象[ちゅうしょう] 的[てき]です。	概念= 
\\	抽象的= 
\\	その詩は、抽象的で分かりにくかった。	
\\	その 詩[し]は、 抽象[ちゅうしょう] 的[てき]で 分[わ]かりにくかった。	抽象的= 
\\	その場合には説得が必要だ。	
\\	その 場合[ばあい]には 説得[せっとく]が 必要[ひつよう]だ。	
\\	アメリカに行かせてくれるよう家族を説得するのは大変だった。	
\\	アメリカに 行[い]かせてくれるよう 家族[かぞく]を 説得[せっとく]するのは 大変[たいへん]だった。	
\\	被害は広がり、東アフリカにまで及びました。	
\\	被害[ひがい]は 広[ひろ]がり、 東[ひがし]アフリカにまで 及[およ]びました。	
\\	私はある日本企業で6か月間働いています。	
\\	私[わたし]はある 日本[にっぽん] 企業[きぎょう]で6か 月[げつ] 間[かん] 働[はたら]いています。	
\\	苦しみや悩みを経験したことのない者は、決して幸せになることはない。	
\\	苦[くる]しみや 悩[なや]みを 経験[けいけん]したことのない 者[もの]は、 決[けっ]して 幸[しあわ]せになることはない。	決して= 
\\	両方、中辛にしてください。	
\\	両方[りょうほう]、 中[ちゅう] 辛[からし]にしてください。	
\\	両方とも20代後半です。	
\\	両方[りょうほう]とも20 代[だい] 後半[こうはん]です。	両方= 
\\	乗組員の到着直後に記者会見が開かれた。	
\\	乗組[のりくみ] 員[いん]の 到着[とうちゃく] 直後[ちょくご]に 記者[きしゃ] 会見[かいけん]が 開[ひら]かれた。	乗組員= 
\\	記者会見= 
\\	これにはいくつかとても面白い性質がある。	
\\	これにはいくつかとても 面白[おもしろ]い 性質[せいしつ]がある。	
\\	10か月ぶりにペンフレンドから手紙が来た。	
\\	10か 月[げつ]ぶりにペンフレンドから 手紙[てがみ]が 来[き]た。	
\\	2か月ぶりのごぶさたです。	
\\	2か 月[げつ]ぶりのごぶさたです。	ごぶさた= 
\\	2年ぶりのフルマラソンとなります。	
\\	年[ねん]ぶりのフルマラソンとなります。	
\\	10分おきに電車がまいります。	
\\	10分[じっぷん]おきに 電車[でんしゃ]がまいります。	
\\	この調査は5年おきに実施されています。	
\\	この 調査[ちょうさ]は5 年[ねん]おきに 実施[じっし]されています。	
\\	二人は、インターネットでほぼ一日おきに会話を続けている。	
\\	二人[ふたり]は、インターネットでほぼ 一日[いちにち]おきに 会話[かいわ]を 続[つづ]けている。	ほぼ= 
\\	曲がったネクタイを直しなさい。	
\\	曲[ま]がったネクタイを 直[なお]しなさい。	
\\	このまま真っすぐ行くと橋があるので、それを渡って左に曲がってください。	
\\	このまま 真[ま]っすぐ 行[い]くと 橋[はし]があるので、それを 渡[わた]って 左[ひだり]に 曲[ま]がってください。	
\\	その店は私の職場から角を一つ曲がった所です。	
\\	その 店[みせ]は 私[わたし]の 職場[しょくば]から 角[かど]を 一[ひと]つ 曲[ま]がった 所[ところ]です。	
\\	お箸でよくかき回してください。	
\\	お 箸[はし]でよくかき 回[まわ]してください。	かき回す= 
\\	彼はそわそわと辺りを見回していた。	
\\	彼[かれ]はそわそわと 辺[あた]りを 見回[みまわ]していた。	
\\	私と同じように彼も脅えていたのです。	
\\	私[わたし]と 同[おな]じように 彼[かれ]も 脅[おび]えていたのです。	脅える= 
\\	そのような事故が起こる予兆はなかったのか。	
\\	そのような 事故[じこ]が 起[お]こる 予兆[よちょう]はなかったのか。	予兆= 
\\	これらは絶対に確実な日本の伝統料理の作り方です。	
\\	これらは 絶対[ぜったい]に 確実[かくじつ]な 日本[にっぽん]の 伝統[でんとう] 料理[りょうり]の 作り方[つくりかた]です。	
\\	その時、日本の勝利は確実なように思われた。	
\\	その 時[とき]、 日本[にっぽん]の 勝利[しょうり]は 確実[かくじつ]なように 思[おも]われた。	
\\	それぞれが相手のためにお茶を立てる。	
\\	それぞれが 相手[あいて]のためにお 茶[ちゃ]を 立[た]てる。	
\\	それぞれの国にはそれぞれの文化や習慣があります。	
\\	それぞれの 国[くに]にはそれぞれの 文化[ぶんか]や 習慣[しゅうかん]があります。	
\\	じっくり考えることができる。	
\\	じっくり 考[かんが]えることができる。	じっくり= 
\\	彼は、日本に来るのは今回が初めてです。	
\\	彼[かれ]は、 日本[にっぽん]に 来[く]るのは 今回[こんかい]が 初[はじ]めてです。	
\\	初めてお会いしますね。	
\\	初[はじ]めてお 会[あ]いしますね。	
\\	初めてたばこを吸ったのはいつですか?	
\\	初[はじ]めてたばこを 吸[す]ったのはいつですか?	
\\	初めての海外でした。	
\\	初[はじ]めての 海外[かいがい]でした。	
\\	すると、そのディスカッションが始まったのです。	
\\	すると、そのディスカッションが 始[はじ]まったのです。	
\\	通りは静かというよりも、かえってにわかに活気づいていたほどだ。	
\\	通[とお]りは 静[しず]かというよりも、かえってにわかに 活気[かっき]づいていたほどだ。	にわか= 
\\	それなのに、なぜ日本の若者はブランド品が買えるのか?	
\\	それなのに、なぜ 日本[にっぽん]の 若者[わかもの]はブランド 品[ひん]が 買[か]えるのか?	
\\	要するに、それが通訳という技術なのです。	
\\	要[よう]するに、それが 通訳[つうやく]という 技術[ぎじゅつ]なのです。	
\\	要するに、押しが強いんです。	
\\	要[よう]するに、 押[お]しが 強[つよ]いんです。	押し= 
\\	その時、彼の人生の目標がはっきりした。	
\\	その 時[とき]、 彼[かれ]の 人生[じんせい]の 目標[もくひょう]がはっきりした。	
\\	1時間待たないといけないのよ。	
\\	時間[じかん] 待[ま]たないといけないのよ。	
\\	人間は生きていくために困難を乗り越えないといけない。	
\\	人間[にんげん]は 生[い]きていくために 困難[こんなん]を 乗り越[のりこ]えないといけない。	
\\	あとどのくらいで出発しないといけないのですか。	
\\	あとどのくらいで 出発[しゅっぱつ]しないといけないのですか。	
\\	彼は、朝5時には早くも働き始めます。	
\\	彼[かれ]は、 朝[あさ]5 時[じ]には 早[はや]くも 働[はたら]き 始[はじ]めます。	
\\	赤は火の色、太陽の色を表し、昔から縁起の良い色とされている。	
\\	赤[あか]は 火[ひ]の 色[いろ]、 太陽[たいよう]の 色[いろ]を 表[あらわ]し、 昔[むかし]から 縁起[えんぎ]の 良[よ]い 色[いろ]とされている。	縁起=えんぎ= 
\\	あなたが元気になったのを見てとても安心しました。	
\\	あなたが 元気[げんき]になったのを 見[み]てとても 安心[あんしん]しました。	
\\	今日は何時に閉店する?	
\\	今日[きょう]は 何[なん] 時[じ]に 閉店[へいてん]する?	
\\	ここは何時に閉まりますか。	
\\	ここは 何[なん] 時[じ]に 閉[し]まりますか。	
\\	古いシステムは複雑でした。一方、その新しいシステムはとてもシンプルです。	
\\	古[ふる]いシステムは 複雑[ふくざつ]でした。 一方[いっぽう]、その 新[あたら]しいシステムはとてもシンプルです。	
\\	みんなに聞いてほしいよい知らせがあります。	
\\	みんなに 聞[き]いてほしいよい 知[し]らせがあります。	
\\	100歳を超えたお年寄りも、日本ではそう珍しいことではなくなった。人生80年は当然のこと。	
\\	歳[さい]を 超[こ]えたお 年寄[としよ]りも、日本ではそう 珍[めずら]しいことではなくなった。 人生[じんせい]80 年[ねん]は 当然[とうぜん]のこと。	
\\	実際のところ、日本人にとって海外旅行は、もはや特別なものではなくなっていると言えよう。	
\\	実際[じっさい]のところ、日本人にとって 海外[かいがい] 旅行[りょこう]は、もはや 特別[とくべつ]なものではなくなっていると 言[い]えよう。	
\\	ええ、でも彼女は私の言うことを信じようとしないんです。	
\\	ええ、でも 彼女[かのじょ]は 私[わたし]の 言[い]うことを 信[しん]じようとしないんです。	
\\	お金があれば外国の若者だってブランド品を買うはずです。	
\\	お 金[かね]があれば 外国[がいこく]の 若者[わかもの]だってブランド 品[ひん]を 買[か]うはずです。	
\\	「だって」
\\	11月15日で結構です。	
\\	11月[じゅういちがつ] 15日[じゅうごにち]で 結構[けっこう]です。	
\\	お金持ちの人が必ずしも幸せだとは限らない。	
\\	お 金持[かねも]ちの 人[ひと]が 必[かなら]ずしも 幸[しあわ]せだとは 限[かぎ]らない。	
\\	すべての人がこの結論に同意しているとは限らない。	
\\	すべての 人[ひと]がこの 結論[けつろん]に 同意[どうい]しているとは 限[かぎ]らない。	
\\	それは常に楽しいとは限らないでしょう。	
\\	それは 常[つね]に 楽[たの]しいとは 限[かぎ]らないでしょう。	
\\	ボリビアの都市には何軒もの日本食レストランがあります。	
\\	ボリビアの 都市[とし]には 何[なん] 軒[けん]もの 日本食[にほんしょく]レストランがあります。	
\\	とても暑かったので、外で作業をしたら倒れないかと心配だった。	
\\	とても 暑[あつ]かったので、 外[そと]で 作業[さぎょう]をしたら 倒[たお]れないかと 心配[しんぱい]だった。	
\\	君には一時はどうなるかと心配させられた。	
\\	君[きみ]には 一時[いちじ]はどうなるかと 心配[しんぱい]させられた。	
\\	買おうかどうしようかと迷っているところだ。	
\\	買[か]おうかどうしようかと 迷[まよ]っているところだ。	
\\	サービスが悪い時は、チップを払わなくてもいいのではないでしょうか。	
\\	サービスが 悪[わる]い 時[とき]は、チップを 払[はら]わなくてもいいのではないでしょうか。	
\\	最近、日本語のできる外国人も増えてきているのではないかと思います。	
\\	最近[さいきん]、 日本語[にほんご]のできる 外国[がいこく] 人[じん]も 増[ふ]えてきているのではないかと 思[おも]います。	
\\	夏風邪は冬の風邪よりむしろ治りにくいから、気をつけてください。	
\\	夏[なつ] 風邪[かぜ]は 冬[ふゆ]の 風邪[かぜ]よりむしろ 治[なお]りにくいから、 気[き]をつけてください。	
\\	会話は、日本語の方が英語よりむしろやさしいかもしれない。	
\\	会話[かいわ]は、 日本語[にほんご]の 方[ほう]が 英語[えいご]よりむしろやさしいかもしれない。	
\\	今度の試験は意外にやさしかった。	
\\	今度[こんど]の 試験[しけん]は 意外[いがい]にやさしかった。	
\\	日本へ行く飛行機の切符は意外に安かった。	
\\	日本[にっぽん]へ 行[い]く 飛行機[ひこうき]の 切符[きっぷ]は 意外[いがい]に 安[やす]かった。	
\\	なぜなら、この国は喫煙者に対してあまりにも甘いからだ。	
\\	なぜなら、この国は 喫煙[きつえん] 者[しゃ]に 対[たい]してあまりにも 甘[あま]いからだ。	
\\	あえて君の両親にはまだ伝えていないよ。	
\\	あえて 君[きみ]の 両親[りょうしん]にはまだ 伝[つた]えていないよ。	
\\	いけないことになっているのにあえて飲むという方が楽しめる。	
\\	いけないことになっているのにあえて 飲[の]むという 方[ほう]が 楽[たの]しめる。	あえて= 
\\	このささやかな楽しみを奪おうというのか。	
\\	このささやかな 楽[たの]しみを 奪[うば]おうというのか。	
\\	日本では、水面下での差別が多い。	
\\	日本[にっぽん]では、 水面[すいめん] 下[か]での 差別[さべつ]が 多[おお]い。	
\\	この話から得られる教訓は何か?	
\\	この 話[はなし]から 得[え]られる 教訓[きょうくん]は 何[なに]か?	
\\	あなたの存在自体が嫌だ。	
\\	あなたの 存在[そんざい] 自体[じたい]が 嫌[いや]だ。	
\\	お金を稼ぐことそれ自体が目的なのではありません。	
\\	お 金[かね]を 稼[かせ]ぐことそれ 自体[じたい]が 目的[もくてき]なのではありません。	
\\	あれ、トイレットペーパーがなくなりそうだ。	
\\	あれ、トイレットペーパーがなくなりそうだ。	
\\	調査によると、香港市民の半数以上がこの計画に賛成だそうです。	
\\	調査[ちょうさ]によると、 香港[ほんこん] 市民[しみん]の 半数[はんすう] 以上[いじょう]がこの 計画[けいかく]に 賛成[さんせい]だそうです。	
\\	2年目はロボット工学や宇宙科学の発達について学びます。	
\\	年[ねん] 目[め]はロボット 工学[こうがく]や 宇宙[うちゅう] 科学[かがく]の 発達[はったつ]について 学[まな]びます。	
\\	当時コンピューターはどこの国でも発達途中にありました。	
\\	当時[とうじ]コンピューターはどこの 国[くに]でも 発達[はったつ] 途中[とちゅう]にありました。	当時= 
\\	2人は、いつしか友達からカップルへと発展していた。	
\\	人[にん]は、いつしか 友達[ともだち]からカップルへと 発展[はってん]していた。	
\\	70年代の頃は、石油は国を急速に発展させる確実な方法と見なされていた。	
\\	年代[ねんだい]の 頃[ころ]は、 石油[せきゆ]は 国[くに]を 急速[きゅうそく]に 発展[はってん]させる 確実[かくじつ]な 方法[ほうほう]と 見[み]なされていた。	
\\	進展があればお知らせします。	
\\	進展[しんてん]があればお 知[し]らせします。	
\\	2002年以来、進展がない。	
\\	年[ねん] 以来[いらい]、 進展[しんてん]がない。	
\\	3ヶ月後、私の日本語はかなり進歩しました。	
\\	ヶ月[かげつ] 後[ご]、 私[わたし]の 日本語[にほんご]はかなり 進歩[しんぽ]しました。	
\\	そろそろお昼にしましょう。	
\\	そろそろお 昼[ひる]にしましょう。	
\\	この分では、日本で宗教対立は永遠に起こりそうにない。	
\\	この 分[ぶん]では、 日本[にっぽん]で 宗教[しゅうきょう] 対立[たいりつ]は 永遠[えいえん]に 起[お]こりそうにない。	
\\	日本の会社はハードだけれど、反面、社員への思いやりにあふれています。	
\\	日本[にっぽん]の 会社[かいしゃ]はハードだけれど、 反面[はんめん]、 社員[しゃいん]への 思[おも]いやりにあふれています。	
\\	その問題について考えるのにもっと時間が必要です。	
\\	その 問題[もんだい]について 考[かんが]えるのにもっと 時間[じかん]が 必要[ひつよう]です。	
\\	献血者をもっと増やす必要がある。	
\\	献血[けんけつ] 者[しゃ]をもっと 増[ふ]やす 必要[ひつよう]がある。	
\\	危険を冒す覚悟が必要です。	
\\	危険[きけん]を 冒[おか]す 覚悟[かくご]が 必要[ひつよう]です。	危険を冒す= 
\\	あなたの生活水準を保つには、かなりのお金が必要です。	
\\	あなたの 生活[せいかつ] 水準[すいじゅん]を 保[たも]つには、かなりのお 金[かね]が 必要[ひつよう]です。	
\\	この学校は教育水準が高い。	
\\	この 学校[がっこう]は 教育[きょういく] 水準[すいじゅん]が 高[たか]い。	
\\	日本は単一民族国家というのは本当ですか。	
\\	日本[にっぽん]は 単一[たんいつ] 民族[みんぞく] 国家[こっか]というのは 本当[ほんとう]ですか。	
\\	私は、現代のアートは現代を表現すべきだと考えています。	
\\	私[わたし]は、 現代[げんだい]のアートは 現代[げんだい]を 表現[ひょうげん]すべきだと 考[かんが]えています。	
\\	すべての弁護士はより強いプロ意識を持つべきである。	
\\	すべての 弁護士[べんごし]はより 強[つよ]いプロ 意識[いしき]を 持[も]つべきである。	
\\	その赤ちゃんは、いい音楽の良さが分かるようだ。	
\\	その 赤[あか]ちゃんは、いい 音楽[おんがく]の 良[よ]さが 分[わ]かるようだ。	
\\	「誰でも間違いはする」ということを覚えておくことは重要です。	
\\	誰[だれ]でも 間違[まちが]いはする」ということを 覚[おぼ]えておくことは 重要[じゅうよう]です。	
\\	私が一番驚かされたのは若い女性です。	
\\	私[わたし]が 一番[いちばん] 驚[おどろ]かされたのは 若[わか]い 女性[じょせい]です。	
\\	多くの家庭では食料を買うための十分なお金を所持していません。	
\\	多[おお]くの 家庭[かてい]では 食料[しょくりょう]を 買[か]うための 十分[じゅうぶん]なお 金[かね]を 所持[しょじ]していません。	
\\	その感情は、言葉では表現しようがない。	
\\	その 感情[かんじょう]は、 言葉[ことば]では 表現[ひょうげん]しようがない。	しようがない= 
\\	けんかなんて、ここでは日常茶飯事です。	
\\	けんかなんて、ここでは 日常[にちじょう] 茶飯事[さはんじ]です。	
\\	大丈夫だよ。僕に任せておいて。	
\\	大丈夫[だいじょうぶ]だよ。 僕[ぼく]に 任[まか]せておいて。	
\\	昨日のこと、そんなに謝らなくていいよ。	
\\	昨日[きのう]のこと、そんなに 謝[あやま]らなくていいよ。	
\\	ああ、また寝坊しちゃった。	
\\	ああ、また 寝坊[ねぼう]しちゃった。	
\\	利益は地元に還元されるべきである。	
\\	利益[りえき]は 地元[じもと]に 還元[かんげん]されるべきである。	
\\	あなたが留守の間、私が犬の世話をします。	
\\	あなたが 留守[るす]の 間[あいだ]、私が 犬[いぬ]の 世話[せわ]をします。	
\\	日本政府によると、約310万人がこの戦争で死亡しました。	
\\	日本[にっぽん] 政府[せいふ]によると、 約[やく]310 万[まん] 人[にん]がこの 戦争[せんそう]で 死亡[しぼう]しました。	
\\	さらに、ほとんどの日本人は学校で学んでいるのでかなりの英語の意識は持っている。	
\\	さらに、ほとんどの 日本人[にっぽんじん]は 学校[がっこう]で 学[まな]んでいるのでかなりの 英語[えいご]の 意識[いしき]は 持[も]っている。	
\\	そこで、結婚があれば離婚があるのもやむを得ないことかと思います。	
\\	そこで、 結婚[けっこん]があれば 離婚[りこん]があるのもやむを 得[え]ないことかと 思[おも]います。	
\\	タイ人数百人が避難を余儀なくさせられました。	
\\	タイ 人数[にんずう] 百[ひゃく] 人[にん]が 避難[ひなん]を 余儀[よぎ]なくさせられました。	
\\	ビアガーデンも雨の日が多いために何日も休業を余儀なくされています。	
\\	ビアガーデンも 雨[あめ]の 日[ひ]が 多[おお]いために 何[なん] 日[にち]も 休業[きゅうぎょう]を 余儀[よぎ]なくされています。	
\\	中国政府はとても難しい立場に追い込まれるはずです。	
\\	中国[ちゅうごく] 政府[せいふ]はとても 難[むずか]しい 立場[たちば]に 追い込[おいこ]まれるはずです。	
\\	それらの犯罪は同一人物によって犯された。	
\\	それらの 犯罪[はんざい]は 同一人物[どういつじんぶつ]によって 犯[おか]された。	同一人物= 
\\	痛むのはちょうどこの箇所です。	
\\	痛[いた]むのはちょうどこの 箇所[かしょ]です。	
\\	それ、本当に気に障るんだ。	
\\	それ、 本当[ほんとう]に 気[き]に 障[さわ]るんだ。	
\\	何がお気に障るのでしょうか?	
\\	何[なに]がお 気[き]に 障[さわ]るのでしょうか?	
\\	当社は印刷業を営む会社です。	
\\	当社[とうしゃ]は 印刷[いんさつ] 業[ぎょう]を 営[いとな]む 会社[かいしゃ]です。	
\\	子どもの頃、両親は和菓子点を営んでいました。	
\\	子[こ]どもの 頃[ころ]、 両親[りょうしん]は 和菓子[わがし] 点[てん]を 営[いとな]んでいました。	和菓子点=
\\	液体が完全に固まるまで待ちなさい。	
\\	液体[えきたい]が 完全[かんぜん]に 固[かた]まるまで 待[ま]ちなさい。	
\\	65歳以上の人が人口の19%を占めます。	
\\	歳[さい] 以上[いじょう]の 人[ひと]が 人口[じんこう]の19 
\\	[ぱーせんと]を 占[し]めます。	
\\	このテーブルは、部屋の3分の1ぐらいを占めている。	
\\	このテーブルは、 部屋[へや]の3 分[ぶん]の1ぐらいを 占[し]めている。	
\\	お客様の口座は残高がマイナスになっております。	
\\	お 客様[きゃくさま]の 口座[こうざ]は 残高[ざんだか]がマイナスになっております。	
\\	私は口座に5000ドルの残高があります。	
\\	私[わたし]は 口座[こうざ]に5000ドルの 残高[ざんだか]があります。	
\\	京都の町には忘れかけていた美しい伝統のぬくもりを感じます。	
\\	京都[きょうと]の 町[まち]には 忘[わす]れかけていた 美[うつく]しい 伝統[でんとう]のぬくもりを 感[かん]じます。	
\\	その王国には戦いではなく平和、憎しみではなく愛がある	
\\	その 王国[おうこく]には 戦[たたか]いではなく 平和[へいわ]、 憎[にく]しみではなく 愛[あい]がある	
\\	二日酔いです。	
\\	二日酔[ふつかよ]いです。	
\\	シルビアは、献身的な母親です。	
\\	シルビアは、 献身[けんしん] 的[てき]な 母親[ははおや]です。	
\\	彼女は、地震救援活動に献身しました。	
\\	彼女[かのじょ]は、 地震[じしん] 救援[きゅうえん] 活動[かつどう]に 献身[けんしん]しました。	救援活動= 
\\	気がめいっちゃう!雨の日は大嫌い。	
\\	気[き]がめいっちゃう! 雨[あめ]の 日[ひ]は 大嫌[だいきら]い。	
\\	その芸術家は後ろに下がって、自分の作品をじっくり見た。	
\\	その 芸術[げいじゅつ] 家[か]は 後[うし]ろに 下[さ]がって、 自分[じぶん]の 作品[さくひん]をじっくり 見[み]た。	
\\	それ故に、結婚することは絶対に必要なことではない。	
\\	それ 故[ゆえ]に、 結婚[けっこん]することは 絶対[ぜったい]に 必要[ひつよう]なことではない。	
\\	あなたは昨夜、素晴らしいパーティーを逃したんだよ。	
\\	あなたは 昨夜[さくや]、 素晴[すば]らしいパーティーを 逃[のが]したんだよ。	
\\	たとえ就職しても、いつか転職を強いられるかもしれません。	
\\	たとえ 就職[しゅうしょく]しても、いつか 転職[てんしょく]を 強[し]いられるかもしれません。	強いる=しいる= 
\\	一度約束したことは、自分が損をしても必ず守ります。	
\\	一度[いちど] 約束[やくそく]したことは、 自分[じぶん]が 損[そん]をしても 必[かなら]ず 守[まも]ります。	
\\	あなたは友達のキューピッド役を買って出たいということですか。	
\\	あなたは 友達[ともだち]のキューピッド 役[やく]を 買[か]って 出[で]たいということですか。	買って出る= 
\\	このことから私たちはどのような結論を導き出せるだろうか?	
\\	このことから 私[わたし]たちはどのような 結論[けつろん]を 導き出[みちびきだ]せるだろうか?	導き出す= 
\\	その結果、その数は劇的に減少している。	
\\	その 結果[けっか]、その 数[かず]は 劇的[げきてき]に 減少[げんしょう]している。	
\\	今の標準からすれば、このコンピューターは時代遅れです。	
\\	今[いま]の 標準[ひょうじゅん]からすれば、このコンピューターは 時代遅[じだいおく]れです。	
\\	少なくとも100歳以上でしょう。	
\\	少[すく]なくとも100 歳[さい] 以上[いじょう]でしょう。	
\\	少なくとも180人の死亡が確認され、255以上が重傷を負いました。	
\\	少[すく]なくとも180 人[にん]の 死亡[しぼう]が 確認[かくにん]され、255 以上[いじょう]が 重傷[じゅうしょう]を 負[お]いました。	
\\	漢字の読み書きをマスターしようとしたら、大変だ。	
\\	漢字[かんじ]の 読み書[よみか]きをマスターしようとしたら、 大変[たいへん]だ。	
\\	日本語には「曖昧文化」を好む日本人の性格が反映されていると言えよう。	
\\	日本語には
\\	曖昧[あいまい] 文化[ぶんか]」を 好[この]む日本人の 性格[せいかく]が 反映[はんえい]されていると 言[い]えよう。	
\\	新しい愛は、失われた愛を補うことができるのだろうか。	
\\	新[あたら]しい 愛[あい]は、 失[うしな]われた 愛[あい]を 補[おぎな]うことができるのだろうか。	
\\	自信のない人間は、それを補うため、外見を飾り立てようとします。	
\\	自信[じしん]のない 人間[にんげん]は、それを 補[おぎな]うため、 外見[がいけん]を 飾り立[かざりた]てようとします。	
\\	思いがけないことが起こりました。	
\\	思[おも]いがけないことが 起[お]こりました。	
\\	思いがけなく機会が訪れた。	
\\	思[おも]いがけなく 機会[きかい]が 訪[おとず]れた。	訪れる=おとずれる= (訪問する) 
\\	(到来する) 
\\	鼓動が聞こえない。	
\\	鼓動[こどう]が 聞[き]こえない。	
\\	しかし、周辺の自然環境が心配されている。	
\\	しかし、 周辺[しゅうへん]の 自然[しぜん] 環境[かんきょう]が 心配[しんぱい]されている。	
\\	その映画の結末はさまざまなやり方で解釈できます。	
\\	その 映画[えいが]の 結末[けつまつ]はさまざまなやり 方[かた]で 解釈[かいしゃく]できます。	
\\	どちらの方角から風が吹いてきていますか。	
\\	どちらの 方角[ほうがく]から 風[かぜ]が 吹[ふ]いてきていますか。	
\\	彼女の容態はどんどん悪化していた。	
\\	彼女[かのじょ]の 容態[ようだい]はどんどん 悪化[あっか]していた。	容態= 
\\	治療は結果がなく、私の容態は徐々に悪化しました。	
\\	治療[ちりょう]は 結果[けっか]がなく、 私[わたし]の 容態[ようだい]は 徐々[じょじょ]に 悪化[あっか]しました。	治療= 
\\	容態= 
\\	あなたと彼女の間の距離がだんだん遠くなっていくのが私には分かる。	
\\	あなたと 彼女[かのじょ]の 間[ま]の 距離[きょり]がだんだん 遠[とお]くなっていくのが 私[わたし]には 分[わ]かる。	
\\	ここから東京までどれくらい距離がありますか。	
\\	ここから 東京[とうきょう]までどれくらい 距離[きょり]がありますか。	
\\	あなたの人生って音楽一色なんですね。	
\\	あなたの 人生[じんせい]って 音楽[おんがく] 一色[いっしょく]なんですね。	
\\	最近私たちはその話題一色です。	
\\	最近[さいきん] 私[わたし]たちはその 話題[わだい] 一色[いっしょく]です。	
\\	「お願いします」は一体どうしたの。	
\\	「お 願[ねが]いします」は 一体[いったい]どうしたの。	
\\	これらは表裏一体なのです。	
\\	これらは 表裏一体[ひょうりいったい]なのです。	表裏一体= 
\\	これは一体どういうこと?	
\\	これは 一体[いったい]どういうこと?	
\\	しかし、私たちは一体誰のために生きているのだろう。	
\\	しかし、 私[わたし]たちは 一体[いったい] 誰[だれ]のために 生[い]きているのだろう。	
\\	実際の順序が本に載っている順序と必ずしも同じとは限らない。	
\\	実際[じっさい]の 順序[じゅんじょ]が 本[ほん]に 載[の]っている 順序[じゅんじょ]と 必[かなら]ずしも 同[おな]じとは 限[かぎ]らない。	載る(のる)= 
\\	もしかすると、この議論には良いおまけがついてくるかもしれません。	
\\	もしかすると、この 議論[ぎろん]には 良[よ]いおまけがついてくるかもしれません。	
\\	もしかすると、それは愛だと自分に信じ込ませているだけかも。	
\\	もしかすると、それは 愛[あい]だと 自分[じぶん]に 信じ込[しんじこ]ませているだけかも。	
\\	これを警鐘と見なしなさい。	
\\	これを 警鐘[けいしょう]と 見[み]なしなさい。	
\\	妻は私の日本語よりずっとうまく英語を話しますが、毎日私は彼女と日本語で、彼女は私と英語で会話をしています。	
\\	妻[つま]は 私[わたし]の 日本語[にほんご]よりずっとうまく 英語[えいご]を 話[はな]しますが、 毎日[まいにち] 私[わたし]は 彼女[かのじょ]と 日本語[にほんご]で、 彼女[かのじょ]は 私[わたし]と 英語[えいご]で 会話[かいわ]をしています。	
\\	「昨日よりかなり良くなっています」と彼は言いました。	
\\	昨日[きのう]よりかなり 良[よ]くなっています」と 彼[かれ]は 言[い]いました。	
\\	テロリストたちは、いずれにせよ再びアメリカを攻撃するつもりです。	
\\	テロリストたちは、いずれにせよ 再[ふたた]びアメリカを 攻撃[こうげき]するつもりです。	いずれにせよ= 
\\	攻撃=こうげき= 
\\	まあ、とにかく私はもう行かないといけません。	
\\	まあ、とにかく 私[わたし]はもう 行[い]かないといけません。	
\\	このポップ・ミュージックの旋律が頭から離れない!	
\\	このポップ・ミュージックの 旋律[せんりつ]が 頭[あたま]から 離[はな]れない!	旋律=せんりつ= 
\\	この日から彼女のことが始終頭から離れなくなった。	
\\	この 日[ひ]から 彼女[かのじょ]のことが 始終[しじゅう] 頭[あたま]から 離[はな]れなくなった。	
\\	どこに行っても生徒達のことは頭から離れない。	
\\	どこに 行[い]っても 生徒[せいと] 達[たち]のことは 頭[あたま]から 離[はな]れない。	
\\	ひろうは肩をすくめた。	
\\	ひろうは 肩[かた]をすくめた。	
\\	ジョンは立ち上がり、肩をすくめる。	
\\	ジョンは 立ち上[たちあ]がり、 肩[かた]をすくめる。	
\\	そんな私がまるで子供みたいだと友達は言いました。	
\\	そんな 私[わたし]がまるで 子供[こども]みたいだと 友達[ともだち]は 言[い]いました。	
\\	日曜日には、テレビを見たり見なかったりします。	
\\	日曜日[にちようび]には、テレビを 見[み]たり 見[み]なかったりします。	
\\	ハワイは気候はいいし、花も美しいし、いいところですよ。	
\\	ハワイは 気候[きこう]はいいし、 花[はな]も 美[うつく]しいし、いいところですよ。	
\\	新しい課長は頑固だし、仕事もできないですよ。	
\\	新[あたら]しい 課長[かちょう]は 頑固[がんこ]だし、 仕事[しごと]もできないですよ。	頑固=がんこ= 
\\	竹村さんは、みんなから信頼されていないし、人気もないし、グループをまとめるのは無理でしょう。	
\\	竹村[たけむら]さんは、みんなから 信頼[しんらい]されていないし、 人気[にんき]もないし、グループをまとめるのは 無理[むり]でしょう。	
\\	あのレストランは、サービスは悪いし、料理はまずいし、行かない方がいいですよ。	
\\	あのレストランは、サービスは 悪[わる]いし、 料理[りょうり]はまずいし、 行[い]かない 方[ほう]がいいですよ。	
\\	ここにいる人たちは、3人とも大学で言語学を勉強しました。	
\\	ここにいる 人[ひと]たちは、3 人[にん]とも 大学[だいがく]で 言語[げんご] 学[がく]を 勉強[べんきょう]しました。	
\\	このセーターは、2枚とも
\\	サイズですか。	
\\	このセーターは、2 枚[まい]とも 
\\	サイズですか。	
\\	この家なら、少なくとも一億円はするでしょう。	
\\	この 家[いえ]なら、 少[すく]なくとも一 億[おく] 円[えん]はするでしょう。	
\\	この事故で死んだ人は、多くとも100人ぐらいだろう。	
\\	この 事故[じこ]で 死[し]んだ 人[ひと]は、 多[おお]くとも100 人[にん]ぐらいだろう。	
\\	あなたの靴のサイズはいくつですか。	
\\	あなたの 靴[くつ]のサイズはいくつですか。	
\\	サービス料とも合計1万5000円です。	
\\	サービス 料[りょう]とも 合計[ごうけい]1 万[まん]5000 円[えん]です。	
\\	運賃ともで、5万円になりますが。	
\\	運賃[うんちん]ともで、5 万[まん] 円[えん]になりますが。	
\\	あの人ならどんな事があろうとも、最後まで頑張るだろう。	
\\	あの 人[ひと]ならどんな 事[こと]があろうとも、 最後[さいご]まで 頑張[がんば]るだろう。	
\\	「とも」
\\	明日は雪が降ろうとも、行くつもりだ。	
\\	明日[あした]は 雪[ゆき]が 降[くだ]ろうとも、 行[い]くつもりだ。	
\\	「とも」
\\	首相ともあろう人が、そんなことをして平気だとは信じられない。	
\\	首相[しゅしょう]ともあろう 人[ひと]が、そんなことをして 平気[へいき]だとは 信[しん]じられない。	
\\	大学の学長ともあろう人が、あんなにビジョンがないのでは困る。	
\\	大学[だいがく]の 学長[がくちょう]ともあろう 人[ひと]が、あんなにビジョンがないのでは 困[こま]る。	
\\	多田さんは、あの映画はいいとも悪いとも言えないと言ってました。	
\\	多田[ただ]さんは、あの 映画[えいが]はいいとも 悪[わる]いとも 言[い]えないと 言[い]ってました。	
\\	「言えない」, 「とも」= 
\\	その値段は、高いとも安いとも言えませんね。	
\\	その 値段[ねだん]は、 高[たか]いとも 安[やす]いとも 言[い]えませんね。	
\\	「この本を借りていいですか」 
\\	「いいとも」	
\\	「この 本[ほん]を 借[か]りていいですか」 
\\	「いいとも」	
\\	「とも」
\\	「明日の試合に行きますか」 
\\	「行くとも」	
\\	明日[あした]の 試合[しあい]に 行[い]きますか」 
\\	行[い]くとも」	
\\	「とも」
\\	あの国の人たちは、戦争やら、インフレやらで大変でしょうね。	
\\	あの 国[くに]の 人[ひと]たちは、 戦争[せんそう]やら、インフレやらで 大変[たいへん]でしょうね。	
\\	その大学には、イギリス人やらフランス人やら、いろいろな国の人がいますよ。	
\\	その 大学[だいがく]には、イギリス 人[じん]やらフランス 人[じん]やら、いろいろな 国[くに]の 人[ひと]がいますよ。	
\\	久美ちゃんはこのお菓子が好きなのやら、嫌いなのやら何も言わなんです。	
\\	久美[くみ]ちゃんはこのお 菓子[かし]が 好[す]きなのやら、 嫌[きら]いなのやら 何[なに]も 言[い]わなんです。	
\\	パーティーに行くのやら行かないのやら、はっきりしてください。	
\\	パーティーに 行[い]くのやら 行[い]かないのやら、はっきりしてください。	
\\	あの犬はどこへ行ったのやら、分からないんですよ。	
\\	あの 犬[いぬ]はどこへ 行[い]ったのやら、 分[わ]からないんですよ。	
\\	研くんはどの大学に入れるやら、本当に心配です。	
\\	研[けん]くんはどの 大学[だいがく]に 入[はい]れるやら、 本当[ほんとう]に 心配[しんぱい]です。	
\\	来年はどんな年になるやら・・・	
\\	来年[らいねん]はどんな 年[とし]になるやら・・・	
\\	日本の将来はどうなるやら・・・	
\\	日本の 将来[しょうらい]はどうなるやら・・・	
\\	クリスマスには、友達から本だの、カレンダーだの、チョコレートだのをもらいました。	
\\	クリスマスには、 友達[ともだち]から 本[ほん]だの、カレンダーだの、チョコレートだのをもらいました。	
\\	学生のときは、ショパンだのリストだののピアノ曲をよく弾きました。	
\\	学生[がくせい]のときは、ショパンだのリストだののピアノ 曲[きょく]をよく 弾[はじ]きました。	
\\	行くだの行かないだのと言わないで、どちらかに決めて下さい。	
\\	行[い]くだの 行[い]かないだのと 言[い]わないで、どちらかに 決[き]めて 下[くだ]さい。	
\\	オーストラリアは、今寒いだの、暑いだのと言う人がいて、どちらなのか分かりません。	
\\	オーストラリアは、 今[いま] 寒[さむ]いだの、 暑[あつ]いだのと 言[い]う 人[ひと]がいて、どちらなのか 分[わ]かりません。	
\\	その問題について、松田先生なり、高山先生なりに聞いてみて下さい。	
\\	その 問題[もんだい]について、 松田[まつだ] 先生[せんせい]なり、 高山[たかやま] 先生[せんせい]なりに 聞[き]いてみて 下[くだ]さい。	
\\	明日行くなり、あさって行くなり、早く決めましょう。	
\\	明日[あした] 行[い]くなり、あさって 行[い]くなり、 早[はや]く 決[き]めましょう。	
\\	まだ時間がありますから、喫茶店でお茶を飲むなりしましょうか。	
\\	まだ 時間[じかん]がありますから、 喫茶店[きっさてん]でお 茶[ちゃ]を 飲[の]むなりしましょうか。	
\\	健康のためにテニスなり、ゴルフなり、何か運動をした方がいいですよ。	
\\	健康[けんこう]のためにテニスなり、ゴルフなり、 何[なに]か 運動[うんどう]をした 方[ほう]がいいですよ。	
\\	信号が青になるなり、待っていた車が走り出した。	
\\	信号[しんごう]が 青[あお]になるなり、 待[ま]っていた 車[くるま]が 走り出[はしりだ]した。	
\\	山本さんが電車に飛び乗るなり、ドアが閉まった。	
\\	山本[やまもと]さんが 電車[でんしゃ]に 飛び乗[とびの]るなり、ドアが 閉[し]まった。	
\\	ご不満なことがあったら、何なりとおっしゃって下さい。	
\\	ご 不満[ふまん]なことがあったら、 何[なん]なりとおっしゃって 下[くだ]さい。	
\\	どこにでも、大なり小なり問題はあるでしょう。	
\\	どこにでも、 大[だい]なり 小[しょう]なり 問題[もんだい]はあるでしょう。	「大なり小なり」= 
\\	デビッドさんは、漢字を書いては消し、書いては消ししています。	
\\	デビッドさんは、 漢字[かんじ]を 書[か]いては 消[け]し、 書[か]いては 消[け]ししています。	
\\	海岸には波が、寄せては返し、返しては寄せています。	
\\	海岸[かいがん]には 波[なみ]が、 寄[よ]せては 返[かえ]し、 返[かえ]しては 寄[よ]せています。	
\\	日曜日にしては、デパートがすいていますね。	
\\	日曜日[にちようび]にしては、デパートがすいていますね。	
\\	あの人は日本人にしては、英語がうまいですね。	
\\	あの 人[ひと]は 日本人[にっぽんじん]にしては、 英語[えいご]がうまいですね。	
\\	「腹が減っては戦ができぬ」と昔の人は言いました。	
\\	腹[はら]が 減[へ]っては 戦[いくさ]ができぬ」と 昔[むかし]の 人[ひと]は 言[い]いました。	
\\	ここに車は止めては困ります。	
\\	ここに 車[くるま]は 止[と]めては 困[こま]ります。	
\\	公園で花を取ってはいけません。	
\\	公園[こうえん]で 花[はな]を 取[と]ってはいけません。	
\\	ここでタバコを吸ってはならない。	
\\	ここでタバコを 吸[す]ってはならない。	
\\	この問題については、私は黙ってはいられません。	
\\	この 問題[もんだい]については、 私[わたし]は 黙[だま]ってはいられません。	
\\	この頃は仕事が忙しくて、旅行に行くどころではない。	
\\	この頃[このごろ]は 仕事[しごと]が 忙[いそが]しくて、 旅行[りょこう]に 行[い]くどころではない。	
\\	あの人は秀才どころの話じゃなくて、まるで天才ですね。	
\\	あの 人[ひと]は 秀才[しゅうさい]どころの 話[はなし]じゃなくて、まるで 天才[てんさい]ですね。	
\\	普通は1時間ぐらいかかるが、昨日は道が混んでいて、1時間どころか3時間もかかった。	
\\	普通[ふつう]は1 時間[じかん]ぐらいかかるが、 昨日[きのう]は 道[みち]が 混[こ]んでいて、1 時間[じかん]どころか3 時間[じかん]もかかった。	
\\	あの人は英語どころか、フランス語もよくできますよ。	
\\	あの 人[ひと]は 英語[えいご]どころか、 フランス語[ふらんすご]もよくできますよ。	
\\	ショパンは、ポーランドの作曲家として有名です。	
\\	ショパンは、ポーランドの 作曲[さっきょく] 家[か]として 有名[ゆうめい]です。	
\\	こちらが、交換学生として日本へ来たホワイトさんです。	
\\	こちらが、 交換[こうかん] 学生[がくせい]として 日本[にっぽん]へ 来[き]たホワイトさんです。	
\\	あの人達のする仕事は、一つとしていいものがないんです。	
\\	あの 人達[ひとたち]のする 仕事[しごと]は、 一[ひと]つとしていいものがないんです。	
\\	課長はマージャンに、一度として勝ったことがありません。	
\\	課長[かちょう]はマージャンに、一 度[ど]として 勝[か]ったことがありません。	
\\	吉田さんは小出さんより背が高いです。	
\\	吉田[よしだ]さんは 小出[こいで]さんより 背[せ]が 高[たか]いです。	
\\	ニューヨークより東京の方が、人口が多いです。	
\\	ニューヨークより 東京[とうきょう]の 方[ほう]が、 人口[じんこう]が 多[おお]いです。	
\\	あの人は指導者というより、独裁者です。	
\\	あの 人[ひと]は 指導[しどう] 者[しゃ]というより、 独裁[どくさい] 者[しゃ]です。	
\\	大学の学長は、学者というより政治家だ。	
\\	大学[だいがく]の 学長[がくちょう]は、 学者[がくしゃ]というより 政治[せいじ] 家[か]だ。	
\\	この店のフランス料理は、どこよりおいしいと思います。	
\\	この 店[みせ]のフランス 料理[りょうり]は、どこよりおいしいと 思[おも]います。	
\\	彼は私にプロポーズしたとき、「僕は誰よりも君を愛している」と言ったのよ。	
\\	彼[かれ]は 私[わたし]にプロポーズしたとき、
\\	僕[ぼく]は 誰[だれ]よりも 君[きみ]を 愛[あい]している」と 言[い]ったのよ。	
\\	3時より閣議が行われます。	
\\	時[じ]より 閣議[かくぎ]が 行[おこな]われます。	
\\	この国境より向こうが中国です。	
\\	この 国境[こっきょう]より 向[む]こうが 中国[ちゅうごく]です。	
\\	電車が参りますから、白線より後ろに下がってお待ち下さい。	
\\	電車[でんしゃ]が 参[まい]りますから、 白線[はくせん]より 後[うし]ろに 下[さ]がってお 待[ま]ち 下[くだ]さい。	
\\	私の家は駅より手前にある。	
\\	私[わたし]の 家[いえ]は 駅[えき]より 手前[てまえ]にある。	
\\	先月の市場調査より次のような結果が明らかになった。	
\\	先月[せんげつ]の 市場[しじょう] 調査[ちょうさ]より 次[つぎ]のような 結果[けっか]が 明[あき]らかになった。	
\\	河川の汚染より伝染病が発生した。	
\\	河川[かせん]の 汚染[おせん]より 伝染病[でんせんびょう]が 発生[はっせい]した。	
\\	ここまでやったんですから、終わりまでやるよりほかないでしょう。	
\\	ここまでやったんですから、 終[お]わりまでやるよりほかないでしょう。	
\\	その問題は先生でさえ答えられなかった。	
\\	その 問題[もんだい]は 先生[せんせい]でさえ 答[こた]えられなかった。	
\\	そんな簡単なことは、子供でさえ知っていますよ。	
\\	そんな 簡単[かんたん]なことは、 子供[こども]でさえ 知[し]っていますよ。	
\\	もう5分さえあったら、飛行機に間に合ったのに・・・	
\\	もう 5分[ごふん]さえあったら、 飛行機[ひこうき]に 間に合[まにあ]ったのに・・・	
\\	健ちゃんは頭がいいんですから、勉強さえすればいい大学に入れますよ。	
\\	健[けん]ちゃんは 頭[あたま]がいいんですから、 勉強[べんきょう]さえすればいい 大学[だいがく]に 入[い]れますよ。	
\\	山田さんは英語の教師なのに、日常会話すらできない。	
\\	山田[やまだ]さんは 英語[えいご]の 教師[きょうし]なのに、 日常[にちじょう] 会話[かいわ]すらできない。	
\\	山で救助された人たちは、疲労で動くことすらできなかった。	
\\	山[やま]で 救助[きゅうじょ]された 人[ひと]たちは、 疲労[ひろう]で 動[うご]くことすらできなかった。	
\\	来年こそ、ヨーロッパへ旅行したいと思っています。	
\\	来年[らいねん]こそ、ヨーロッパへ 旅行[りょこう]したいと 思[おも]っています。	
\\	今度こそ、頑張りましょう。	
\\	今度[こんど]こそ、 頑張[がんば]りましょう。	
\\	本当にそんなこと起こったんでしょうかね。	
\\	本当[ほんとう]にそんなこと 起[お]こったんでしょうかね。	
\\	あの人たち、何を考えているのか分かりませんね。	
\\	あの 人[ひと]たち、 何[なに]を 考[かんが]えているのか 分[わ]かりませんね。	
\\	もうだいぶ歩いたから、この辺でちょっと休もうよ。	
\\	もうだいぶ 歩[ある]いたから、この 辺[あたり]でちょっと 休[やす]もうよ。	
\\	あの展覧会へ行ってみましょうよ。	
\\	あの 展覧[てんらん] 会[かい]へ 行[い]ってみましょうよ。	
\\	この仕事はあなたしかできませんから、ぜひお願いします。	
\\	この 仕事[しごと]はあなたしかできませんから、ぜひお 願[ねが]いします。	
\\	いえ、恵子は小学校を去年出ましたから、もう13歳ですよ。	
\\	いえ、 恵子[けいこ]は 小学校[しょうがっこう]を 去年[きょねん] 出[で]ましたから、もう13 歳[さい]ですよ。	
\\	谷さん、そんな悪いことをしてはいけませんよ。	
\\	谷[たに]さん、そんな 悪[わる]いことをしてはいけませんよ。	
\\	課長、何時に来るかな。	
\\	課長[かちょう]、 何[なん] 時[じ]に 来[く]るかな。	
\\	多賀君は、この仕事できるかな。	
\\	多賀[たが] 君[くん]は、この 仕事[しごと]できるかな。	
\\	この仕事、頼んでいいかな。	
\\	この 仕事[しごと]、 頼[たの]んでいいかな。	
\\	明日の朝早く会社に来てもらえるかな。	
\\	明日[あした]の 朝[あさ] 早[はや]く 会社[かいしゃ]に 来[き]てもらえるかな。	
\\	奇麗な星だなあ。	
\\	奇麗[きれい]な 星[ほし]だなあ。	
\\	あの車は新車だよな。	
\\	あの 車[くるま]は 新車[しんしゃ]だよな。	
\\	あの人はなかなか立派な人だと思うな。	
\\	あの 人[ひと]はなかなか 立派[りっぱ]な 人[ひと]だと 思[おも]うな。	
\\	明日必ず来いな。	
\\	明日[あした] 必[かなら]ず 来[こ]いな。	
\\	絶対にあいつに会うな。	
\\	絶対[ぜったい]にあいつに 会[あ]うな。	
\\	もうあのバーに行くな。	
\\	もうあのバーに 行[い]くな。	
\\	明日の高橋さんのパーティーには、もちろん行くさ。	
\\	明日[あした]の 高橋[たかはし]さんのパーティーには、もちろん 行[い]くさ。	
\\	それより、こっちのセーターの方が大きいさ。	
\\	それより、こっちのセーターの 方[ほう]が 大[おお]きいさ。	
\\	あんな無能な社員を入れるから、会社が伸びないのさ。	
\\	あんな 無能[むのう]な 社員[しゃいん]を 入[い]れるから、 会社[かいしゃ]が 伸[の]びないのさ。	
\\	あの人のやりそうなことさ。	
\\	あの 人[ひと]のやりそうなことさ。	
\\	あと10年たつと、日本人だって「日本語ができるのが国際人の条件」っていうようになるかもしれませんよ。	
\\	あと10 年[ねん]たつと、 日本人[にっぽんじん]だって
\\	日本語[にほんご]ができるのが 国際[こくさい] 人[じん]の 条件[じょうけん]」っていうようになるかもしれませんよ。	
\\	あなたの成績が悪かったからパーティーはなしだってママが言っていたよ。	
\\	あなたの 成績[せいせき]が 悪[わる]かったからパーティーはなしだってママが 言[い]っていたよ。	
\\	あの子、夜トイレに行かないのよ。彼女、怖いんだって。	
\\	あの 子[こ]、 夜[よる]トイレに 行[い]かないのよ。 彼女[かのじょ]、 怖[こわ]いんだって。	
\\	この花の色の美しいこと。	
\\	この 花[はな]の 色[いろ]の 美[うつく]しいこと。	「こと」
\\	美味しいお料理ですこと。	
\\	美味[おい]しいお 料理[りょうり]ですこと。	「こと」
\\	どこかへお花見に行きませんこと。	
\\	どこかへお 花見[はなみ]に 行[い]きませんこと。	「こと」
\\	一度クイーンエリザベス号に乗ってみませんこと。	
\\	一度[いちど]クイーンエリザベス 号[ごう]に 乗[の]ってみませんこと。	「こと」
\\	明日の結婚式は、何時に始まるんでしたっけ。	
\\	明日[あした]の 結婚式[けっこんしき]は、 何[なん] 時[じ]に 始[はじ]まるんでしたっけ。	
\\	あなたの家はどこだったっけ。	
\\	あなたの 家[いえ]はどこだったっけ。	
\\	この辺に学校があったっけ。	
\\	この 辺[あたり]に 学校[がっこう]があったっけ。	
\\	あの人とよく酒を飲んだっけ。	
\\	あの 人[ひと]とよく 酒[さけ]を 飲[の]んだっけ。	
\\	明日までにできなければ困るってば。	
\\	明日[あした]までにできなければ 困[こま]るってば。	
\\	来年では遅すぎるってば。	
\\	来年[らいねん]では 遅[おそ]すぎるってば。	
\\	そんなことをしたら、だめだってば。	
\\	そんなことをしたら、だめだってば。	
\\	コンピューターを使わなければ、できないってば。	
\\	コンピューターを 使[つか]わなければ、できないってば。	
\\	昨日どこで飲んだんだい。	
\\	昨日[きのう]どこで 飲[の]んだんだい。	「ーい」
\\	ーだ 
\\	ーか 
\\	またアメリカに出張かい、大変だな。	
\\	またアメリカに 出張[しゅっちょう]かい、 大変[たいへん]だな。	「ーい」
\\	ーだ 
\\	ーか 
\\	あの人、元気だったかい。	
\\	あの 人[ひと]、 元気[げんき]だったかい。	「ーい」
\\	ーだ 
\\	ーか 
\\	あの映画は面白くないんですもの。だから、行かなかったのよ。	
\\	あの 映画[えいが]は 面白[おもしろ]くないんですもの。だから、 行[い]かなかったのよ。	"「もの」
\\	どうして食べないんだい。 
\\	この料理、嫌いなんですもの。	
\\	どうして 食[た]べないんだい。 
\\	この 料理[りょうり]、 嫌[きら]いなんですもの。	"「もの」
\\	課長の仕事はやりたくないわ。下の者に冷たいんですもの。	
\\	課長[かちょう]の 仕事[しごと]はやりたくないわ。 下[した]の 者[もの]に 冷[つめ]たいんですもの。	"「もの」
\\	竹内さんとは一緒に仕事をしたくないのよ。ちっとも働かないんだもの。	
\\	竹内[たけうち]さんとは 一緒[いっしょ]に 仕事[しごと]をしたくないのよ。ちっとも 働[はたら]かないんだもの。	"「もの」
\\	出かけましょうよ。たまには外で食事がしたいんですもの。	
\\	出[で]かけましょうよ。たまには 外[そと]で 食事[しょくじ]がしたいんですもの。	"「もの」
\\	あれ欲しいですもの。買ってもいいでしょう。	
\\	あれ 欲[ほ]しいですもの。 買[か]ってもいいでしょう。	"「もの」
\\	先に行くぜ。	
\\	先[さき]に 行[い]くぜ。	
\\	その仕事、君に頼んだぜ。	
\\	その 仕事[しごと]、 君[きみ]に 頼[たの]んだぜ。	
\\	頑張るぜ。	
\\	頑張[がんば]るぜ。	
\\	そろそろ会議を始めるぞ。	
\\	そろそろ 会議[かいぎ]を 始[はじ]めるぞ。	
\\	今度そんなことしたら、絶対に許さないぞ。	
\\	今度[こんど]そんなことしたら、 絶対[ぜったい]に 許[ゆる]さないぞ。	
\\	今度こそ成功するぞ。	
\\	今度[こんど]こそ 成功[せいこう]するぞ。	
\\	あんな所、もう行くもんか。	
\\	あんな 所[ところ]、もう 行[い]くもんか。	「ものか/もんか」
\\	あんな人と一緒に仕事ができるもんですか。	
\\	あんな 人[ひと]と 一緒[いっしょ]に 仕事[しごと]ができるもんですか。	「ものか/もんか」
\\	ほかの人がやったら、もっと早くできたでしょうに。	
\\	ほかの 人[ひと]がやったら、もっと 早[はや]くできたでしょうに。	
\\	もう少し待っていたら、雨がやんだろうに。	
\\	もう 少[すこ]し 待[ま]っていたら、 雨[あめ]がやんだろうに。	
\\	それは廊下のつきあたりです。	
\\	それは 廊下[ろうか]のつきあたりです。	
\\	もしかして今日仕事に戻る見込みはある?	
\\	もしかして 今日[きょう] 仕事[しごと]に 戻[もど]る 見込[みこ]みはある?	
\\	トムが回復する見込みはほとんどない。	
\\	トムが 回復[かいふく]する 見込[みこ]みはほとんどない。	
\\	成功の見込みはどうでしょうか?	
\\	成功[せいこう]の 見込[みこ]みはどうでしょうか?	
\\	地元の経済が回復に向けて前進することを希望する。	
\\	地元[じもと]の 経済[けいざい]が 回復[かいふく]に 向[む]けて 前進[ぜんしん]することを 希望[きぼう]する。	
\\	御用はございますか?	
\\	御用[ごよう]はございますか?	
\\	親から、外人という言葉には外国からきた人という意味の他に嫌な意味があるんだみたいなことを聞いた覚えがあります。	
\\	親[おや]から、 外人[がいじん]という 言葉[ことば]には 外国[がいこく]からきた 人[ひと]という 意味[いみ]の 他[ほか]に 嫌[いや]な 意味[いみ]があるんだみたいなことを 聞[き]いた 覚[おぼ]えがあります。	
\\	ここがきつい感じです。	
\\	ここがきつい 感[かん]じです。	
\\	ウエストがちょっときついです。	
\\	ウエストがちょっときついです。	
\\	仕事はきついが、自分の役割をこなしている。	
\\	仕事[しごと]はきついが、 自分[じぶん]の 役割[やくわり]をこなしている。	
\\	あなたの指示通りに仕事をしました。	
\\	あなたの 指示[しじ] 通[どお]りに 仕事[しごと]をしました。	
\\	この方法は常識的なものなので、指示はあまり必要ないだろう。	
\\	この 方法[ほうほう]は 常識[じょうしき] 的[てき]なものなので、 指示[しじ]はあまり 必要[ひつよう]ないだろう。	
\\	彼らの中には、彼を殺そうとする者もいた。	
\\	彼[かれ]らの 中[なか]には、 彼[かれ]を 殺[ころ]そうとする 者[もの]もいた。	
\\	あいにくそれは切らしております。	
\\	あいにくそれは 切[き]らしております。	"「あいにく」って 
\\	という意味。
\\	あいにく全室が予約済みでございます。	
\\	あいにく 全[ぜん] 室[しつ]が 予約[よやく] 済[ず]みでございます。	"「あいにく」って 
\\	という意味。
\\	彼は、即刻辞職すべきだ。	
\\	彼[かれ]は、 即刻[そっこく] 辞職[じしょく]すべきだ。	
\\	僕たちは皆、毎日、危険を冒しながら選択しているのだ。	
\\	僕[ぼく]たちは 皆[みな]、 毎日[まいにち]、 危険[きけん]を 冒[おか]しながら 選択[せんたく]しているのだ。	
\\	命の危険を冒してでもあなたを救う。	
\\	命[いのち]の 危険[きけん]を 冒[おか]してでもあなたを 救[すく]う。	
\\	彼女は彼の娘だと言い張った。	
\\	彼女[かのじょ]は 彼[かれ]の 娘[むすめ]だと 言い張[いいは]った。	
\\	例えば、白人の喫煙率は、中国系と南アジア系より3倍も高かったのです。	
\\	例[たと]えば、 白人[はくじん]の 喫煙[きつえん] 率[りつ]は、 中国[ちゅうごく] 系[けい]と 南[みなみ]アジア 系[けい]より3 倍[ばい]も 高[たか]かったのです。	
\\	そのピアニストは、最新アルバムの宣伝のため、現在ヨーロッパに行っている。	
\\	そのピアニストは、 最新[さいしん]アルバムの 宣伝[せんでん]のため、 現在[げんざい]ヨーロッパに 行[い]っている。	
\\	この症状は家族に代々伝わっている。	
\\	この 症状[しょうじょう]は 家族[かぞく]に 代々[だいだい] 伝[つた]わっている。	
\\	こんなことが起きると前もって知るすべはなかった。	
\\	こんなことが 起[お]きると 前[まえ]もって 知[し]るすべはなかった。	
\\	その問題に緊急に対処する必要がある。	
\\	その 問題[もんだい]に 緊急[きんきゅう]に 対処[たいしょ]する 必要[ひつよう]がある。	
\\	民主主義の根幹は、自由な言論を保障することにあります。	
\\	民主[みんしゅ] 主義[しゅぎ]の 根幹[こんかん]は、 自由[じゆう]な 言論[げんろん]を 保障[ほしょう]することにあります。	
\\	遠慮しとくよ。	
\\	遠慮[えんりょ]しとくよ。	
\\	50ドルというのは今の時代大した金額ではない。	
\\	50ドルというのは 今[いま]の 時代[じだい] 大[たい]した 金額[きんがく]ではない。	
\\	お引き出し金額を入力して下さい。	
\\	お 引き出[ひきだ]し 金額[きんがく]を 入力[にゅうりょく]して 下[くだ]さい。	
\\	採用の時点では、英語が話せるか否かは重要ではない。	
\\	採用[さいよう]の 時点[じてん]では、 英語[えいご]が 話[はな]せるか 否[いな]かは 重要[じゅうよう]ではない。	採用= 
\\	1枚のチケットで一つのアトラクションに有効です。	
\\	枚[まい]のチケットで 一[ひと]つのアトラクションに 有効[ゆうこう]です。	
\\	あなたのカードの有効期限は切れています。	
\\	あなたのカードの 有効[ゆうこう] 期限[きげん]は 切[き]れています。	
\\	このチケットの有効期限は30日間です。	
\\	このチケットの 有効[ゆうこう] 期限[きげん]は30 日間[にちかん]です。	
\\	これは、もはや避けて通れない道です。	
\\	これは、もはや 避[さ]けて 通[とお]れない 道[みち]です。	避けて通れない= 
\\	そういう話題は議論を避けた方がいいですよ。	
\\	そういう 話題[わだい]は 議論[ぎろん]を 避[さ]けた 方[ほう]がいいですよ。	
\\	その件について語るのはいつも避けている。	
\\	その 件[けん]について 語[かた]るのはいつも 避[さ]けている。	
\\	これは世界に平和をもたらす素質です。	
\\	これは 世界[せかい]に 平和[へいわ]をもたらす 素質[そしつ]です。	
\\	従来のその言葉の定義は誤っている。	
\\	従来[じゅうらい]のその 言葉[ことば]の 定義[ていぎ]は 誤[あやま]っている。	
\\	従来のテレビ放送の終わりが近づいている。	
\\	従来[じゅうらい]のテレビ 放送[ほうそう]の 終[お]わりが 近[ちか]づいている。	
\\	ターンテーブルは従来、楽器とは認められていませんでした。	
\\	ターンテーブルは 従来[じゅうらい]、 楽器[がっき]とは 認[みと]められていませんでした。	
\\	こんな急な呼び出しで駆け付けてくれてありがとう。	
\\	こんな 急[きゅう]な 呼び出[よびだ]しで 駆け付[かけつ]けてくれてありがとう。	
\\	スピーカーで呼び出された。	
\\	スピーカーで 呼び出[よびだ]された。	
\\	現在も依然として、それほど多くの貧しい人がいるとは信じにくい。	
\\	現在[げんざい]も 依然[いぜん]として、それほど 多[おお]くの 貧[まず]しい 人[ひと]がいるとは 信[しん]じにくい。	
\\	当時に比べて成長した。	
\\	当時[とうじ]に 比[くら]べて 成長[せいちょう]した。	
\\	あなたはとても気楽な人で、同時に、とても独立心が強い人です。	
\\	あなたはとても 気楽[きらく]な 人[ひと]で、 同時[どうじ]に、とても 独立[どくりつ] 心[しん]が 強[つよ]い 人[ひと]です。	
\\	学生なのに、彼女は勉強しない。	
\\	学生[がくせい]なのに、 彼女[かのじょ]は 勉強[べんきょう]しない。	
\\	デパートに行きましたが、何も欲しくなかったです。	
\\	デパートに 行[い]きましたが、 何[なに]も 欲[ほ]しくなかったです。	
\\	友達に聞いたけど、知らなかった。	
\\	友達[ともだち]に 聞[き]いたけど、 知[し]らなかった。	
\\	この大学の授業は簡単だったり、難しかったりする。	
\\	この 大学[だいがく]の 授業[じゅぎょう]は 簡単[かんたん]だったり、 難[むずか]しかったりする。	
\\	準備は、もうしてあるよ。	
\\	準備[じゅんび]は、もうしてあるよ。	
\\	日本語をずっと前から勉強してきて、結局はやめた。	
\\	日本語[にほんご]をずっと 前[まえ]から 勉強[べんきょう]してきて、 結局[けっきょく]はやめた。	
\\	ここに入ってはいけません。	
\\	ここに 入[はい]ってはいけません。	
\\	それを食べてはだめ!	
\\	それを 食[た]べてはだめ!	
\\	夜、遅くまで電話してはならない。	
\\	夜[よる]、 遅[おそ]くまで 電話[でんわ]してはならない。	
\\	毎日学校に行かなくてはなりません。	
\\	毎日[まいにち] 学校[がっこう]に 行[い]かなくてはなりません。	
\\	宿題をしなくてはいけなかった。	
\\	宿題[しゅくだい]をしなくてはいけなかった。	
\\	毎日学校に行かないとだめです。	
\\	毎日[まいにち] 学校[がっこう]に 行[い]かないとだめです。	
\\	死んじゃだめだよ!	
\\	死[し]んじゃだめだよ!	「じゃ」
\\	「では」
\\	「寒い」とアリスが田中に言った。	
\\	寒[さむ]い」とアリスが 田中[たなか]に 言[い]った。	
\\	これが私のお目当てものだ。	
\\	これが 私[わたし]のお 目当[めあ]てものだ。	
\\	それを目当てに訪れる観光客もいます。	
\\	それを 目当[めあ]てに 訪[おとず]れる 観光[かんこう] 客[きゃく]もいます。	目当て= 
\\	訪れる=おとずれる= (訪問する) 
\\	(到来する) 
\\	何が目当てなの?	
\\	何[なに]が 目当[めあ]てなの?	
\\	この行列を見て!入れっこないわ。	
\\	この 行列[ぎょうれつ]を 見[み]て! 入[はい]れっこないわ。	「っこない」
\\	-ます 
\\	これは何の行列ですか?	
\\	これは 何[なに]の 行列[ぎょうれつ]ですか?	
\\	これは、日本語で何と言いますか。	
\\	これは、 日本語[にほんご]で 何[なに]と 言[い]いますか。	
\\	カレーを食べようと思ったけど、食べる時間がなかった。	
\\	カレーを 食[た]べようと 思[おも]ったけど、 食[た]べる 時間[じかん]がなかった。	
\\	今、どこに行こうかと考えている。	
\\	今[いま]、どこに 行[い]こうかと 考[かんが]えている。	
\\	彼は、これは何だと言いましたか。	
\\	彼[かれ]は、これは 何[なに]だと 言[い]いましたか。	
\\	智子は来年、海外に行くんだって。	
\\	智子[ともこ]は 来年[らいねん]、 海外[かいがい]に 行[い]くんだって。	
\\	もうお金がないって。	
\\	もうお 金[かね]がないって。	
\\	明日って、雨が降るんだって。	
\\	明日[あした]って、 雨[あめ]が 降[ふ]るんだって。	
\\	ルミネというデパートはどこにあるか、知っていますか?	
\\	ルミネというデパートはどこにあるか、 知[し]っていますか?	
\\	みきちゃんは、あんたの彼女でしょう? 
\\	うーん、彼女というか、友達というか、何というか・・・	
\\	みきちゃんは、あんたの 彼女[かのじょ]でしょう? 
\\	うーん、 彼女[かのじょ]というか、 友達[ともだち]というか、 何[なん]というか・・・	
\\	お酒は好きというか、ないと生きていけない。	
\\	お 酒[さけ]は 好[す]きというか、ないと 生[い]きていけない。	
\\	多分行かないと思う。というか、お金がないから行けない。	
\\	多分[たぶん] 行[い]かないと 思[おも]う。というか、お 金[かね]がないから 行[い]けない。	
\\	みきちゃんが洋介と別れたんだって。 
\\	ということは、みきちゃんは、今彼氏がいないということ? 
\\	そう。そういうこと。	
\\	みきちゃんが 洋介[ようすけ]と 別[わか]れたんだって。 
\\	ということは、みきちゃんは、 今[こん] 彼氏[かれし]がいないということ? 
\\	そう。そういうこと。	
\\	お好み焼きを初めて食べてみたけど、とてもおいしかった!	
\\	お 好み焼[このみや]きを 初[はじ]めて 食[た]べてみたけど、とてもおいしかった!	
\\	新しいデパートに行ってみる。	
\\	新[あたら]しいデパートに 行[い]ってみる。	
\\	広島のお好み焼きを食べてみたい!	
\\	広島[ひろしま]のお 好み焼[このみや]きを 食[た]べてみたい!	
\\	毎日、勉強を避けようとする。	
\\	毎日[まいにち]、 勉強[べんきょう]を 避[さ]けようとする。	
\\	早く寝ようとしたけど、結局は徹夜した。	
\\	早[はや]く 寝[ね]ようとしたけど、 結局[けっきょく]は 徹夜[てつや]した。	
\\	勉強をなるべく避けようと思った。	
\\	勉強[べんきょう]をなるべく 避[さ]けようと 思[おも]った。	
\\	毎日ジムに行こうと決めた。	
\\	毎日[まいにち]ジムに 行[い]こうと 決[き]めた。	
\\	自分の子供に「駄目」と言うとき、悪者になったような気がします。	
\\	自分[じぶん]の 子供[こども]に
\\	駄目[だめ]」と 言[い]うとき、 悪者[わるもの]になったような 気[き]がします。	
\\	今日は酔いたい気分です。	
\\	今日[きょう]は 酔[よ]いたい 気分[きぶん]です。	
\\	場違いな感じがするよ。	
\\	場違[ばちが]いな 感[かん]じがするよ。	
\\	特に電車に乗っている時に、おかしなガイジンになったような気がします。	
\\	特[とく]に 電車[でんしゃ]に 乗[の]っている 時[とき]に、おかしなガイジンになったような 気[き]がします。	
\\	日本に長くいればいるほど、私は自分が「変なガイジン」だという気がしてきます。	
\\	日本[にっぽん]に 長[なが]くいればいるほど、 私[わたし]は 自分[じぶん]が
\\	変[へん]なガイジン」だという 気[き]がしてきます。	
\\	代わりに行ってあげる。	
\\	代[か]わりに 行[い]ってあげる。	
\\	犬に餌をやった。	
\\	犬[いぬ]に 餌[えさ]をやった。	
\\	友達が私にプレゼントをくれた。	
\\	友達[ともだち]が 私[わたし]にプレゼントをくれた。	
\\	車を買ってくれるの?	
\\	車[くるま]を 買[か]ってくれるの?	
\\	代わりに行ってくれる?	
\\	代[か]わりに 行[い]ってくれる?	
\\	私が友達にプレゼントをもらった。	
\\	私[わたし]が 友達[ともだち]にプレゼントをもらった。	
\\	千円を貸してくれる?	
\\	千[せん] 円[えん]を 貸[か]してくれる?	
\\	千円を貸してもらえる?	
\\	千[せん] 円[えん]を 貸[か]してもらえる?	
\\	ちょっと静かにしてくれない?	
\\	ちょっと 静[しず]かにしてくれない?	
\\	漢字で書いてもらえませんか。	
\\	漢字[かんじ]で 書[か]いてもらえませんか。	
\\	全部食べないでくれますか。	
\\	全部[ぜんぶ] 食[た]べないでくれますか。	
\\	高い物を買わないでくれる?	
\\	高[たか]い 物[もの]を 買[か]わないでくれる?	
\\	それをください。	
\\	それをください。	
\\	お父さんがくれた時計が壊れた。	
\\	お 父[とう]さんがくれた 時計[とけい]が 壊[こわ]れた。	
\\	好きにしろ。	
\\	好[す]きにしろ。	
\\	速く酒を持ってきてくれ。	
\\	速[はや]く 酒[さけ]を 持[も]ってきてくれ。	
\\	変なことを言うな!	
\\	変[へん]なことを 言[い]うな!	
\\	四万三千七十六	
\\	四万三千七十六[よんまんさんぜんななじゅうろく]	
\\	二時間四十分	
\\	二時間[にじかん] 四十分[よんじゅっぷん]	
\\	二十日間	
\\	二十日間[はつかかん]	
\\	十五日間	
\\	十五日間[じゅうごにちかん]	
\\	サラリーマンだから、残業はたくさんするんじゃない?	
\\	サラリーマンだから、 残業[ざんぎょう]はたくさんするんじゃない?	
\\	ほら、やっぱりレポートを書かないとだめじゃん。	
\\	ほら、やっぱりレポートを 書[か]かないとだめじゃん。	
\\	誰もいないからここで着替えてもいいじゃん。	
\\	誰[だれ]もいないからここで 着替[きが]えてもいいじゃん。	
\\	先生が学生に宿題をたくさんさせた。	
\\	先生[せんせい]が 学生[がくせい]に 宿題[しゅくだい]をたくさんさせた。	
\\	先生が質問をたくさん聞かせてくれた。	
\\	先生[せんせい]が 質問[しつもん]をたくさん 聞[き]かせてくれた。	
\\	今日は仕事を休ませてください。	
\\	今日[きょう]は 仕事[しごと]を 休[やす]ませてください。	
\\	その部長は、よく長時間働かせる。	
\\	その 部長[ぶちょう]は、よく 長時間[ちょうじかん] 働[はたら]かせる。	
\\	ポリッジが誰かに食べられた!	
\\	ポリッジが 誰[だれ]かに 食[た]べられた!	
\\	みんなに変だと言われます。	
\\	みんなに 変[へん]だと 言[い]われます。	
\\	光の速さを超えるのは、不可能だと思われる。	
\\	光[ひかり]の 速[はや]さを 超[こ]えるのは、 不可能[ふかのう]だと 思[おも]われる。	
\\	この教科書は多くの人に読まれている。	
\\	この 教科書[きょうかしょ]は 多[おお]くの 人[ひと]に 読[よ]まれている。	
\\	外国人に質問を聞かれたが、答えられなかった。	
\\	外国[がいこく] 人[じん]に 質問[しつもん]を 聞[き]かれたが、 答[こた]えられなかった。	
\\	このパッケージには、あらゆるものが含まれている。	
\\	このパッケージには、あらゆるものが 含[ふく]まれている。	
\\	朝ご飯は食べたくなかったのに、食べさせられた。	
\\	朝[あさ]ご 飯[はん]は 食[た]べたくなかったのに、 食[た]べさせられた。	
\\	日本では、お酒を飲ませられることが多い。	
\\	日本[にっぽん]では、お 酒[さけ]を 飲[の]ませられることが 多[おお]い。	
\\	親に毎日宿題をさせられる。	
\\	親[おや]に 毎日[まいにち] 宿題[しゅくだい]をさせられる。	
\\	学生が廊下に立たされた。	
\\	学生[がくせい]が 廊下[ろうか]に 立[た]たされた。	
\\	あいつに二時間も待たされた。	
\\	あいつに二 時間[じかん]も 待[ま]たされた。	
\\	アリスさん、もう召し上がりましたか。	
\\	アリスさん、もう 召し上[めしあ]がりましたか。	
\\	仕事で何をなさっているんですか。	
\\	仕事[しごと]で 何[なに]をなさっているんですか。	
\\	推薦状を書いてくださるんですか。	
\\	推薦[すいせん] 状[じょう]を 書[か]いてくださるんですか。	
\\	どちらからいらっしゃいましたか。	
\\	どちらからいらっしゃいましたか。	
\\	今日は、どちらへいらっしゃいますか。	
\\	今日[きょう]は、どちらへいらっしゃいますか。	
\\	私が書いたレポートを見ていただけますか。	
\\	私[わたし]が 書[か]いたレポートを 見[み]ていただけますか。	
\\	お手洗いはこのビルの二階にございます。	
\\	お 手洗[てあら]いはこのビルの二 階[かい]にございます。	
\\	こちらにご覧下さい。	
\\	こちらにご 覧[らん] 下[くだ]さい。	
\\	毎日ケーキを食べて、2キロ太ってしまいました。	
\\	毎日[まいにち]ケーキを 食[た]べて、2キロ 太[ふと]ってしまいました。	
\\	ちゃんと食べないと、痩せてしまいますよ。	
\\	ちゃんと 食[た]べないと、 痩[や]せてしまいますよ。	
\\	ごめん、待たせてしまって!	
\\	ごめん、 待[ま]たせてしまって!	
\\	そろそろ遅くなっちゃうよ。	
\\	そろそろ 遅[おそ]くなっちゃうよ。	
\\	徹夜して、宿題することはある。	
\\	徹夜[てつや]して、 宿題[しゅくだい]することはある。	
\\	一人で行くことはありません。	
\\	一 人[にん]で 行[い]くことはありません。	
\\	ヨーロッパに行ったことがあったらいいな。	
\\	ヨーロッパに 行[い]ったことがあったらいいな。	
\\	そういうのを見たことがなかった。	
\\	そういうのを 見[み]たことがなかった。	
\\	一度行ったこともないんです。	
\\	一度[いちど] 行[い]ったこともないんです。	
\\	早く来て。映画は、今ちょうどいいところだよ。	
\\	早[はや]く 来[き]て。 映画[えいが]は、 今[いま]ちょうどいいところだよ。	
\\	彼は、優しいところもあるよ。	
\\	彼[かれ]は、 優[やさ]しいところもあるよ。	
\\	今は授業が終わったところです。	
\\	今[いま]は 授業[じゅぎょう]が 終[お]わったところです。	
\\	スミスさんは食堂に行ったかもしれません。	
\\	スミスさんは 食堂[しょくどう]に 行[い]ったかもしれません。	
\\	雨で試合は中止になるかもしれないね。	
\\	雨[あめ]で 試合[しあい]は 中止[ちゅうし]になるかもしれないね。	
\\	あそこが代々木公園かもしれない。	
\\	あそこが 代々木公園[よよぎこうえん]かもしれない。	
\\	休ませていただけますでしょうか。	
\\	休[やす]ませていただけますでしょうか。	
\\	アリスはどこだ? 
\\	もう寝ているだろう。	
\\	アリスはどこだ? 
\\	もう 寝[ね]ているだろう。	
\\	もう家に帰るんだろう。 
\\	そうよ。	
\\	もう 家[いえ]に 帰[かえ]るんだろう。 
\\	そうよ。	
\\	その人だけが好きだったんだ。	
\\	その 人[ひと]だけが 好[す]きだったんだ。	
\\	この販売機だけでは、500円玉が使えない。	
\\	この 販売[はんばい] 機[き]だけでは、500 円[えん] 玉[だま]が 使[つか]えない。	
\\	小林さんからだけは、返事が来なかった。	
\\	小林[こばやし]さんからだけは、 返事[へんじ]が 来[こ]なかった。	
\\	準備が終わったから、これからは食べるだけだ。	
\\	準備[じゅんび]が 終[お]わったから、これからは 食[た]べるだけだ。	
\\	この乗車券は発売当日のみ有効です。	
\\	この 乗車[じょうしゃ] 券[けん]は 発売[はつばい] 当日[とうじつ]のみ 有効[ゆうこう]です。	「のみ」
\\	「だけ」
\\	アンケート対象は大学生のみです。	
\\	アンケート 対象[たいしょう]は 大学生[だいがくせい]のみです。	
\\	今日は忙しくて、朝ご飯しか食べられなかった。	
\\	今日[きょう]は 忙[いそが]しくて、 朝[あさ]ご 飯[はん]しか 食[た]べられなかった。	
\\	こうなったら、逃げるしかない。	
\\	こうなったら、 逃[に]げるしかない。	
\\	もう腐っているから、捨てるしかないよ。	
\\	もう 腐[くさ]っているから、 捨[す]てるしかないよ。	
\\	これは買うっきゃない!	
\\	これは 買[か]うっきゃない!	
\\	「っきゃ」
\\	「しか」
\\	彼は麻雀ばかりです。	
\\	彼[かれ]は 麻雀[まーじゃん]ばかりです。	
\\	直美ちゃんと遊ぶばっかりでしょう!	
\\	直美[なおみ]ちゃんと 遊[あそ]ぶばっかりでしょう!	
\\	最近は仕事ばっかだよ。	
\\	最近[さいきん]は 仕事[しごと]ばっかだよ。	「ばっか」
\\	「ばかり」
\\	お酒を飲み過ぎないように気をつけてね。	
\\	お 酒[さけ]を 飲[の]み 過[す]ぎないように 気[き]をつけてね。	
\\	大きすぎるからトランクに入らないぞ。	
\\	大[おお]きすぎるからトランクに 入[はい]らないぞ。	
\\	時間が足りなさ過ぎて、何もできなかった。	
\\	時間[じかん]が 足[た]りなさ 過[す]ぎて、 何[なに]もできなかった。	
\\	彼には、彼女がもったいなさすぎるよ。	
\\	彼[かれ]には、 彼女[かのじょ]がもったいなさすぎるよ。	
\\	昨日、電話三回もしたよ!	
\\	昨日[きのう]、 電話[でんわ]三 回[かい]もしたよ!	
\\	今年、十キロも太っちゃった!	
\\	今年[ことし]、十キロも 太[ふと]っちゃった!	
\\	今日の天気はそれほど寒くない。	
\\	今日[きょう]の 天気[てんき]はそれほど 寒[さむ]くない。	
\\	韓国料理は食べれば食べるほど、おいしくなる。	
\\	韓国[かんこく] 料理[りょうり]は 食[た]べれば 食[た]べるほど、おいしくなる。	
\\	歩いたら歩くほど、迷ってしまった。	
\\	歩[ある]いたら 歩[ある]くほど、 迷[まよ]ってしまった。	
\\	は、ハードディスクの容量が大きければ大きいほどもっとたくさんの曲が保存できます。	
\\	は、ハードディスクの 容量[ようりょう]が 大[おお]きければ 大[おお]きいほどもっとたくさんの 曲[きょく]が 保存[ほぞん]できます。	
\\	文章は、短ければ短いほど、簡単なら簡単なほどよいです。	
\\	文章[ぶんしょう]は、 短[みじか]ければ 短[みじか]いほど、 簡単[かんたん]なら 簡単[かんたん]なほどよいです。	
\\	ここには、誰もいないようだ。	
\\	ここには、 誰[だれ]もいないようだ。	
\\	彼は学生のような雰囲気ですね。	
\\	彼[かれ]は 学生[がくせい]のような 雰囲気[ふんいき]ですね。	
\\	ちょっと怒ったように聞こえた。	
\\	ちょっと 怒[おこ]ったように 聞[き]こえた。	
\\	何も起こらなかったように言った。	
\\	何[なに]も 起[お]こらなかったように 言[い]った。	
\\	もう売り切れみたい。	
\\	もう 売り切[うりき]れみたい。	
\\	このピザはお好み焼きのように見える。	
\\	このピザはお 好み焼[このみや]きのように 見[み]える。	
\\	この辺りにありそうだけどな。	
\\	この 辺[あた]りにありそうだけどな。	
\\	これも結構よさそうだけど、やっぱり高いよね。	
\\	これも 結構[けっこう]よさそうだけど、やっぱり 高[たか]いよね。	
\\	お前なら、金髪の女が好きそうだな。	
\\	お 前[まえ]なら、 金髪[きんぱつ]の 女[おんな]が 好[す]きそうだな。	
\\	今日、田中さんは来ないの? 
\\	だそうです。	
\\	今日[きょう]、 田中[たなか]さんは 来[こ]ないの? 
\\	だそうです。	
\\	あの子は子供らしくない。	
\\	あの 子[こ]は 子供[こども]らしくない。	
\\	みんなで、もう全部食べてしまったっぽいよ。	
\\	みんなで、もう 全部[ぜんぶ] 食[た]べてしまったっぽいよ。	
\\	恭子は全然女っぽくないね。	
\\	恭子[きょうこ]は 全然[ぜんぜん] 女[おんな]っぽくないね。	
\\	赤ちゃんは、静かな方が好き。	
\\	赤[あか]ちゃんは、 静[しず]かな 方[ほう]が 好[す]き。	
\\	ゆっくり食べた方が健康にいいよ。	
\\	ゆっくり 食[た]べた 方[ほう]が 健康[けんこう]にいいよ。	
\\	そんなに飲まなかった方がよかった。	
\\	そんなに 飲[の]まなかった 方[ほう]がよかった。	
\\	キムさんより鈴木さんの方が若い。	
\\	キムさんより 鈴木[すずき]さんの 方[ほう]が 若[わか]い。	
\\	商品の品質を何より大切にしています。	
\\	商品[しょうひん]の 品質[ひんしつ]を 何[なに]より 大切[たいせつ]にしています。	
\\	この仕事は誰よりも早くできます。	
\\	この 仕事[しごと]は 誰[だれ]よりも 早[はや]くできます。	
\\	新宿の行き方は分かりますか。	
\\	新宿[しんじゅく]の 行[い]き 方[かた]は 分[わ]かりますか。	
\\	季節によって果物はおいしくなったり、まずくなったりする。	
\\	季節[きせつ]によって 果物[くだもの]はおいしくなったり、まずくなったりする。	
\\	友達の話によると、朋子はやっとボーイフレンドを見つけたらしい。	
\\	友達[ともだち]の 話[はなし]によると、 朋子[ともこ]はやっとボーイフレンドを 見[み]つけたらしい。	「らしい」
\\	この字は読みにくい。	
\\	この 字[じ]は 読[よ]みにくい。	
\\	待ち合わせは、分かりづらい場所にしないでね。	
\\	待ち合[まちあ]わせは、 分[わ]かりづらい 場所[ばしょ]にしないでね。	
\\	何も食べないで寝ました。	
\\	何[なに]も 食[た]べないで 寝[ね]ました。	
\\	歯を磨かないで、学校に行っちゃいました。	
\\	歯[は]を 磨[みが]かないで、 学校[がっこう]に 行[い]っちゃいました。	
\\	先生と相談しないで、この授業を取ることは出来ない。	
\\	先生[せんせい]と 相談[そうだん]しないで、この 授業[じゅぎょう]を 取[と]ることは 出来[でき]ない。	
\\	今から行くとしたら、9時に着くと思います。	
\\	今[いま]から 行[い]くとしたら、9 時[じ]に 着[つ]くと 思[おも]います。	
\\	朝ご飯を食べたとしても、もう昼だからお腹が空いたでしょう。	
\\	朝[あさ]ご 飯[はん]を 食[た]べたとしても、もう 昼[ひる]だからお 腹[なか]が 空[す]いたでしょう。	
\\	すみません、今食べたばかりなので、お腹がいっぱいです。	
\\	すみません、 今[いま] 食[た]べたばかりなので、お 腹[なか]がいっぱいです。	
\\	今、家に帰ったばかりです。	
\\	今[いま]、 家[いえ]に 帰[かえ]ったばかりです。	
\\	まさか、今起きたばっかなの?	
\\	まさか、 今[いま] 起[お]きたばっかなの?	
\\	相手に何も言わないながら、自分の気持ちを分かってほしいのは単なるわがままだと思わない?	
\\	相手[あいて]に 何[なに]も 言[い]わないながら、 自分[じぶん]の 気持[きも]ちを 分[わ]かってほしいのは 単[たん]なるわがままだと 思[おも]わない?	
\\	仕事がいっぱい入って、残念ながら、今日は行けなくなりました。	
\\	仕事[しごと]がいっぱい 入[はい]って、 残念[ざんねん]ながら、 今日[きょう]は 行[い]けなくなりました。	
\\	貧乏ながらも、高級なバッグを買っちゃったよ。	
\\	貧乏[びんぼう]ながらも、 高級[こうきゅう]なバッグを 買[か]っちゃったよ。	
\\	ゲームにはまっちゃって、最近パソコンを使いまくっているよ。	
\\	ゲームにはまっちゃって、 最近[さいきん]パソコンを 使[つか]いまくっているよ。	「まくる」 
\\	アメリカにいた時はコーラを飲みまくっていた。	
\\	アメリカにいた 時[とき]はコーラを 飲[の]みまくっていた。	
\\	半分しか食べてないままで捨てちゃダメ!	
\\	半分[はんぶん]しか 食[た]べてないままで 捨[す]てちゃダメ!	
\\	テレビを付けっぱなしにしなければ眠れない人は、結構いる。	
\\	テレビを 付[つ]けっぱなしにしなければ 眠[ねむ]れない 人[ひと]は、 結構[けっこう]いる。	
\\	窓が開けっ放しだったので、蚊がいっぱい入った。	
\\	窓[まど]が 開けっ放[あけっぱな]しだったので、 蚊[か]がいっぱい 入[はい]った。	
\\	彼は漫画マニアだから、これらをもう全部読んだはずだよ。	
\\	彼[かれ]は 漫画[まんが]マニアだから、これらをもう 全部[ぜんぶ] 読[よ]んだはずだよ。	
\\	この料理はおいしいはずだったが、焦げちゃって、まずくなった。	
\\	この 料理[りょうり]はおいしいはずだったが、 焦[こ]げちゃって、まずくなった。	
\\	色々予定してあるから、今年は楽しいクリスマスのはず。	
\\	色々[いろいろ] 予定[よてい]してあるから、 今年[ことし]は 楽[たの]しいクリスマスのはず。	
\\	そう簡単に直せるはずがないよ。	
\\	そう 簡単[かんたん]に 直[なお]せるはずがないよ。	
\\	打ち合わせは毎週2時から始まるはずじゃないですか。	
\\	打ち合[うちあ]わせは 毎週[まいしゅう]2 時[じ]から 始[はじ]まるはずじゃないですか。	
\\	何かを買う前に本当に必要かどうかをよく考えるべきだ。	
\\	何[なに]かを 買[か]う 前[まえ]に 本当[ほんとう]に 必要[ひつよう]かどうかをよく 考[かんが]えるべきだ。	
\\	例え国のためであっても、国民を騙すべきではないと思う。	
\\	例[たと]え 国[こく]のためであっても、 国民[こくみん]を 騙[だま]すべきではないと 思[おも]う。	
\\	試験に合格すべく、皆一生懸命に勉強している。	
\\	試験[しけん]に 合格[ごうかく]すべく、 皆[みな] 一生懸命[いっしょうけんめい]に 勉強[べんきょう]している。	
\\	ゴミ捨てるべからず。	
\\	ゴミ 捨[す]てるべからず。	
\\	安全措置を忘れるべからず。	
\\	安全[あんぜん] 措置[そち]を 忘[わす]れるべからず。	
\\	私でさえ出来れば、あんたには楽ちんでしょう。	
\\	私[わたし]でさえ 出来[でき]れば、あんたには 楽[らく]ちんでしょう。	「楽ちん」= 
\\	ビタミンを食べさえすれば、健康が保証されますよ。	
\\	ビタミンを 食[た]べさえすれば、 健康[けんこう]が 保証[ほしょう]されますよ。	
\\	自分の過ちを認めさえしなければ、問題は解決しないよ。	
\\	自分[じぶん]の 過[あやま]ちを 認[みと]めさえしなければ、 問題[もんだい]は 解決[かいけつ]しないよ。	「過ち」=あやまち= 
\\	この天才の私ですら分からなかった。	
\\	この 天才[てんさい]の 私[わたし]ですら 分[わ]からなかった。	「すら」=「さえ」
\\	私は緊張しすぎて、ちらっと見ることすら出来ませんでした。	
\\	私[わたし]は 緊張[きんちょう]しすぎて、ちらっと 見[み]ることすら 出来[でき]ませんでした。	「すら」=「さえ」
\\	「人」の漢字すら知らない生徒は、いないでしょ!	
\\	「人」
\\	人[ひと]」の 漢字[かんじ]すら 知[し]らない 生徒[せいと]は、いないでしょ!	「すら」=「さえ」
\\	漢字はおろか、ひらがなさえ読めないよ!	
\\	漢字[かんじ]はおろか、ひらがなさえ 読[よ]めないよ!	「おろか」= 
\\	結婚はおろか、2ヶ月付き合って、結局別れてしまった。	
\\	結婚[けっこん]はおろか、2 ヶ月[かげつ] 付き合[つきあ]って、 結局[けっきょく] 別[わか]れてしまった。	「おろか」= 
\\	大学はおろか、高校すら卒業しなかった。	
\\	大学[だいがく]はおろか、 高校[こうこう]すら 卒業[そつぎょう]しなかった。	「おろか」= 
\\	早く来てよ!何を恥ずかしがっているの?	
\\	早[はや]く 来[き]てよ! 何[なに]を 恥[は]ずかしがっているの?	
\\	うちの子供はプールに入るのを理由もなく怖がる。	
\\	うちの 子供[こども]はプールに 入[はい]るのを 理由[りゆう]もなく 怖[こわ]がる。	
\\	家に帰ったら、すぐパソコンを使いたがる。	
\\	家[いえ]に 帰[かえ]ったら、すぐパソコンを 使[つか]いたがる。	
\\	妻はルイヴィトンのバッグを欲しがっているんだけど、そんなもん、買えるわけないでしょう!	
\\	妻[つま]はルイヴィトンのバッグを 欲[ほ]しがっているんだけど、そんなもん、 買[か]えるわけないでしょう!	
\\	私は寒がり屋だから、ミネソタで暮らすのは辛かった。	
\\	私[わたし]は 寒[さむ]がり 屋[や]だから、ミネソタで 暮[く]らすのは 辛[つら]かった。	「がり屋」= 
\\	紅葉が始まり、すっかり秋めいた空気になってきた。	
\\	紅葉[こうよう]が 始[はじ]まり、すっかり 秋[あき]めいた 空気[くうき]になってきた。	
\\	「めく」
\\	/-な 
\\	いつも皮肉めいた言い方をしたら、みんなを嫌がらせるよ。	
\\	いつも 皮肉[ひにく]めいた 言い方[いいかた]をしたら、みんなを 嫌[いや]がらせるよ。	
\\	「めく」
\\	/-な 
\\	このテレビがこれ以上壊れたら、新しいのを買わざるを得ないな。	
\\	このテレビがこれ 以上[いじょう] 壊[こわ]れたら、 新[あたら]しいのを 買[か]わざるを 得[え]ないな。	
\\	ずっと我慢してきたが、この状態だと歯医者さんに行かざるを得ない。	
\\	ずっと 我慢[がまん]してきたが、この 状態[じょうたい]だと 歯医者[はいしゃ]さんに 行[い]かざるを 得[え]ない。	
\\	上司の話を聞くと、どうしても海外に出張をせざるを得ないようです。	
\\	上司[じょうし]の 話[はなし]を 聞[き]くと、どうしても 海外[かいがい]に 出張[しゅっちょう]をせざるを 得[え]ないようです。	
\\	やむを得ない事由により手続きが遅れた場合、必ずご連絡下さい。	
\\	やむを 得[え]ない 事由[じゆう]により 手続[てつづ]きが 遅[おく]れた 場合[ばあい]、 必[かなら]ずご 連絡[れんらく] 下[くだ]さい。	
\\	「やむを得ない」= 
\\	この場ではちょっと決めかねますので、また別途会議を設けましょう。	
\\	この 場[ば]ではちょっと 決[き]めかねますので、また 別途[べっと] 会議[かいぎ]を 設[もう]けましょう。	-かねる= 
\\	このままでは、個人情報が漏洩しかねないので、速やかに対応をお願い致します。	
\\	このままでは、 個人[こじん] 情報[じょうほう]が 漏洩[ろうえい]しかねないので、 速[すみ]やかに 対応[たいおう]をお 願[ねが]い 致[いた]します。	漏洩= 
\\	速やか= 
\\	確定申告は忘れがちな手続の一つだ。	
\\	確定[かくてい] 申告[しんこく]は 忘[わす]れがちな 手続[てつづき]の 一[ひと]つだ。	
\\	留守がちなご家庭には、犬よりも、猫の方がオススメです。	
\\	留守[るす]がちなご 家庭[かてい]には、 犬[いぬ]よりも、 猫[ねこ]の 方[ほう]がオススメです。	
\\	父親は病気がちで、みんなが心配している。	
\\	父親[ちちおや]は 病気[びょうき]がちで、みんなが 心配[しんぱい]している。	
\\	多くの大学生は締切日ぎりぎりまで、宿題をやらないきらいがある。	
\\	多[おお]くの 大学生[だいがくせい]は 締切日[しめきりび]ぎりぎりまで、 宿題[しゅくだい]をやらないきらいがある。	「きらいがある」
\\	相手は剣の達人だ。そう簡単には勝てまい。	
\\	相手[あいて]は 剣[つるぎ]の 達人[たつじん]だ。そう 簡単[かんたん]には 勝[か]てまい。	剣=つるぎ= 
\\	達人= 
\\	そんな無茶な手段は認めますまい!	
\\	そんな 無茶[むちゃ]な 手段[しゅだん]は 認[みと]めますまい!	
\\	明日に行くのをやめよう。	
\\	明日[あした]に 行[い]くのをやめよう。	
\\	肉を食べないようにしている。	
\\	肉[にく]を 食[た]べないようにしている。	
\\	あいつが大学に入ろうが入るまいが、俺とは関係ないよ。	
\\	あいつが 大学[だいがく]に 入[にゅう]ろうが 入[はい]るまいが、 俺[おれ]とは 関係[かんけい]ないよ。	
\\	めっちゃムラムラしてエッチしたくてたまんないよ!	
\\	めっちゃムラムラしてエッチしたくてたまんないよ!	
\\	彼はエッチなことばかり考えている人だ。	
\\	彼[かれ]はエッチなことばかり 考[かんが]えている 人[ひと]だ。	
\\	携帯を2年間使ってたら、傷だらけになった。	
\\	携帯[けいたい]を2 年間[ねんかん] 使[つか]ってたら、 傷[きず]だらけになった。	
\\	この埃だらけのテレビをちゃんと拭いてくれない?	
\\	この 埃[ほこり]だらけのテレビをちゃんと 拭[ふ]いてくれない?	
\\	彼は油まみれになりながら、車の修理に頑張りました。	
\\	彼[かれ]は 油[あぶら]まみれになりながら、 車[くるま]の 修理[しゅうり]に 頑張[がんば]りました。	「まみれ」
\\	たった1キロを走っただけで、汗まみれになるのは情けない。	
\\	たった1キロを 走[はし]っただけで、 汗[あせ]まみれになるのは 情[なさ]けない。	「まみれ」
\\	彼女は、教授の姿を見るが早いか、教室から逃げ出した。	
\\	彼女[かのじょ]は、 教授[きょうじゅ]の 姿[すがた]を 見[み]るが 早[はや]いか、 教室[きょうしつ]から 逃げ出[にげだ]した。	「が早いか」
\\	「食べてみよう」と言うが早いか、口の中に放り込んだ。	
\\	食[た]べてみよう」と 言[い]うが 早[はや]いか、 口[くち]の 中[なか]に 放り込[ほうりこ]んだ。	「が早いか」
\\	子供が掃除するそばから散らかすから、もうあきらめたくなった。	
\\	子供[こども]が 掃除[そうじ]するそばから 散[ち]らかすから、もうあきらめたくなった。	散らかす= 
\\	「そばから」
\\	教科書を読んだそばから忘れてしまうので勉強できない。	
\\	教科書[きょうかしょ]を 読[よ]んだそばから 忘[わす]れてしまうので 勉強[べんきょう]できない。	「そばから」
\\	昼間だから絶対込んでいると思いきや、一人もいなかった。	
\\	昼間[ひるま]だから 絶対[ぜったい] 込[こ]んでいると 思[おも]いきや、一 人[にん]もいなかった。	「思いきや」
\\	このレストランは安いと思いきや、会計は5千円以上だった!	
\\	このレストランは 安[やす]いと 思[おも]いきや、 会計[かいけい]は5 千[せん] 円[えん] 以上[いじょう]だった!	「思いきや」
\\	先生と相談のあげく、退学をしないことにした。	
\\	先生[せんせい]と 相談[そうだん]のあげく、 退学[たいがく]をしないことにした。	「挙句 (あげく)」
\\	気晴らしには何をしますか?	
\\	気晴[きば]らしには 何[なに]をしますか?	
\\	彼女のお気に入りの気晴らしはテレビを見ることだ。	
\\	彼女[かのじょ]のお 気に入[きにい]りの 気晴[きば]らしはテレビを 見[み]ることだ。	
\\	政府は安定しているという体裁を保とうとした。	
\\	政府[せいふ]は 安定[あんてい]しているという 体裁[ていさい]を 保[たも]とうとした。	
\\	この経済制裁は、イラクに原油の輸出を禁じるものでした。	
\\	この 経済[けいざい] 制裁[せいさい]は、イラクに 原油[げんゆ]の 輸出[ゆしゅつ]を 禁[きん]じるものでした。	経済制裁=けいざいせいさい= 
\\	原油=げんゆ= 
\\	その国に対する経済制裁は解除されました。	
\\	その 国[くに]に 対[たい]する 経済[けいざい] 制裁[せいさい]は 解除[かいじょ]されました。	解除する=かいじょする= 
\\	あなたが密かに好きな相手は誰ですか。	
\\	あなたが 密[ひそ]かに 好[す]きな 相手[あいて]は 誰[だれ]ですか。	
\\	南はひそかに達也にひかれている。	
\\	南[みなみ]はひそかに 達也[たつや]にひかれている。	ひかれる= 
\\	その作業の膨大さは、私を圧倒しました。	
\\	その 作業[さぎょう]の 膨大[ぼうだい]さは、 私[わたし]を 圧倒[あっとう]しました。	
\\	ディズニーランドの新しい乗り物は客を圧倒しました。	
\\	ディズニーランドの 新[あたら]しい 乗り物[のりもの]は 客[きゃく]を 圧倒[あっとう]しました。	
\\	世の中には自分の都合だけを考えている人が圧倒的に多いです。	
\\	世の中[よのなか]には 自分[じぶん]の 都合[つごう]だけを 考[かんが]えている 人[ひと]が 圧倒的[あっとうてき]に 多[おお]いです。	
\\	あなたはグループに拒絶されることを恐れています。	
\\	あなたはグループに 拒絶[きょぜつ]されることを 恐[おそ]れています。	
\\	ホテルの部屋に戻ると、すぐにお風呂にお湯を入れた。	
\\	ホテルの 部屋[へや]に 戻[もど]ると、すぐにお 風呂[ふろ]にお 湯[ゆ]を 入[い]れた。	
\\	帰ったらすぐにお風呂に入りたいから、お風呂を沸かしておいてね。	
\\	帰[かえ]ったらすぐにお 風呂[ふろ]に 入[はい]りたいから、お 風呂[ふろ]を 沸[わ]かしておいてね。	
\\	こんな寒い日は、熱いお風呂にゆっくりつかりたい。	
\\	こんな 寒[さむ]い 日[ひ]は、 熱[あつ]いお 風呂[ふろ]にゆっくりつかりたい。	
\\	ルーシーはそろそろお風呂から上がる?	
\\	ルーシーはそろそろお 風呂[ふろ]から 上[あ]がる?	
\\	一風呂浴びてビールにしよう。	
\\	一風呂[ひとふろ] 浴[あ]びてビールにしよう。	一風呂=ひとふろ= 
\\	それを冷ましたくない。	
\\	それを 冷[さ]ましたくない。	
\\	お茶が冷めてしまった。	
\\	お 茶[ちゃ]が 冷[さ]めてしまった。	
\\	コーヒーが冷めるよ。	
\\	コーヒーが 冷[さ]めるよ。	
\\	ほとぼりが冷めたら戻ってきます。	
\\	ほとぼりが 冷[さ]めたら 戻[もど]ってきます。	ほとぼり= 
\\	ビールがちょっとぬるいです。	
\\	ビールがちょっとぬるいです。	温い=ぬるい= 
\\	なまぬるい関係に満足している?	
\\	なまぬるい 関係[かんけい]に 満足[まんぞく]している?	生温い=なまぬるい= 
\\	朝起きると、顔を洗うより先に猫に餌をやります。	
\\	朝[あさ] 起[お]きると、 顔[かお]を 洗[あら]うより 先[さき]に 猫[ねこ]に 餌[えさ]をやります。	
\\	便器におしっこしちゃったら、ちゃんと拭くのよ!	
\\	便器[べんき]におしっこしちゃったら、ちゃんと 拭[ふ]くのよ!	
\\	彼は額を拭くために、ハンカチを使いました。		彼[かれ]は 額[ひたい]を 拭[ふ]くために、ハンカチを 使[つか]いました。	額=ひたい= 
\\	週に一度、顔の産毛を剃ります。	
\\	週[しゅう]に一 度[ど]、 顔[かお]の 産毛[うぶげ]を 剃[そ]ります。	
\\	夕べ髪をよく乾かさずに寝たので、ひどい寝ぐせがついてしまった。	
\\	夕[ゆう]べ 髪[かみ]をよく 乾[かわ]かさずに 寝[ね]たので、ひどい 寝[ね]ぐせがついてしまった。	寝癖=ねぐせ= 
\\	一時間も寝坊してあわてて家を出たので、髪をとかすひまもなかった。	
\\	一 時間[じかん]も 寝坊[ねぼう]してあわてて 家[いえ]を 出[で]たので、 髪[かみ]をとかすひまもなかった。	あわてる= 
\\	美容院でやってくれるように、うまく髪をブローすることができない。	
\\	美容[びよう] 院[いん]でやってくれるように、うまく 髪[かみ]をブローすることができない。	
\\	もう少し肌のお手入れをするようにしないといけない。	
\\	もう 少[すこ]し 肌[はだ]のお 手入[てい]れをするようにしないといけない。	手入れをする 
\\	車の手入れをするのと同じくらい、自分の体調もきちんと管理するべきです。	
\\	車[くるま]の 手入[てい]れをするのと 同[おな]じくらい、 自分[じぶん]の 体調[たいちょう]もきちんと 管理[かんり]するべきです。	
\\	朝はいつもシャワーを浴びてはじめて目が覚める。	
\\	朝[あさ]はいつもシャワーを 浴[あ]びてはじめて 目[め]が 覚[さ]める。	
\\	昨日の夜は断水でシャワーを使えなかった。	
\\	昨日[きのう]の 夜[よる]は 断水[だんすい]でシャワーを 使[つか]えなかった。	断水=だんすい=
\\	美容院でシャンプーしてもらうのが好きです。	
\\	美容[びよう] 院[いん]でシャンプーしてもらうのが 好[す]きです。	
\\	どの銘柄のシャンプーを使っていますか?	
\\	どの 銘柄[めいがら]のシャンプーを 使[つか]っていますか?	銘柄=めいがら= 
\\	シャンプーを切らしてしまったんだけど、借りてもいい?	
\\	シャンプーを 切[き]らしてしまったんだけど、 借[か]りてもいい?	切らす= 
\\	油の汚れは、せっけんを使わないとなかなか落ちない。	
\\	油[あぶら]の 汚[よご]れは、せっけんを 使[つか]わないとなかなか 落[お]ちない。	
\\	インフルエンザの予防には、手をよくせっけんで洗うことが重要だ。	
\\	インフルエンザの 予防[よぼう]には、 手[て]をよくせっけんで 洗[あら]うことが 重要[じゅうよう]だ。	
\\	しっかり泡立ててね!	
\\	しっかり 泡立[あわだ]ててね!	
\\	せっけんを泡立てた。	
\\	せっけんを 泡立[あわだ]てた。	
\\	散歩の途中、コンビニに寄ってトイレを借りた。	
\\	散歩[さんぽ]の 途中[とちゅう]、コンビニに 寄[よ]ってトイレを 借[か]りた。	
\\	休憩時間にトイレに行ったが、すべてのトイレがふさがっていた。	
\\	休憩[きゅうけい] 時間[じかん]にトイレに 行[い]ったが、すべてのトイレがふさがっていた。	
\\	今は手が塞がっている。	
\\	今[いま]は 手[て]が 塞[ふさ]がっている。	
\\	「便所」と書いてあるでしょう。あそこがトイレです。	
\\	便所[べんじょ]」と 書[か]いてあるでしょう。あそこがトイレです。	
\\	私は毎食後に歯を磨きます。	
\\	私[わたし]は 毎食[まいしょく] 後[ご]に 歯[は]を 磨[みが]きます。	
\\	あなたに理解できない事に口を挟まないでください。	
\\	あなたに 理解[りかい]できない 事[こと]に 口[くち]を 挟[はさ]まないでください。	口を挟まる=くちをはさまる= 
\\	冬になると肌が乾燥して、かゆくなることがある。	
\\	冬[ふゆ]になると 肌[はだ]が 乾燥[かんそう]して、かゆくなることがある。	
\\	彼女はお風呂上がりに、肌の手入れを書かしません。	
\\	彼女[かのじょ]はお 風呂[ふろ] 上[あ]がりに、 肌[はだ]の 手入[てい]れを 書[か]かしません。	
\\	最近は、夏でも肌を焼きたくないという人が、多くなってきている。	
\\	最近[さいきん]は、 夏[なつ]でも 肌[はだ]を 焼[や]きたくないという 人[ひと]が、 多[おお]くなってきている。	
\\	日焼けは私を健康な気分にさせてくれる。	
\\	日焼[ひや]けは 私[わたし]を 健康[けんこう]な 気分[きぶん]にさせてくれる。	
\\	髭を剃っていたら、誤って少し切ってしまった。	
\\	髭[ひげ]を 剃[す]っていたら、 誤[あやま]って 少[すこ]し 切[き]ってしまった。	
\\	彼は髭を剃り落としていて、まるで別人のようだった。	
\\	彼[かれ]は 髭[ひげ]を 剃り落[そりお]としていて、まるで 別人[べつじん]のようだった。	
\\	その国では、ほとんどの成人男性は髭をたくわえている。	
\\	その 国[くに]では、ほとんどの 成人[せいじん] 男性[だんせい]は 髭[ひげ]をたくわえている。	髭をたくわえる= 
\\	生まれて初めて髭を伸ばしてみることにした。	
\\	生[う]まれて 初[はじ]めて 髭[ひげ]を 伸[の]ばしてみることにした。	髭を伸ばす=ひげをのばす= 
\\	解放されたとき、人質は髭が伸びて疲れ切った様子だった。	
\\	解放[かいほう]されたとき、 人質[ひとじち]は 髭[ひげ]が 伸[の]びて 疲[つか]れ 切[き]った 様子[ようす]だった。	人質=ひとじち= 
\\	もし私が男なら、ひげを生やすわ。	
\\	もし 私[わたし]が 男[おとこ]なら、ひげを 生[は]やすわ。	ひげを生やす= 
\\	彼は、口ひげを生やして、格好良く見せようとしました。	
\\	彼[かれ]は、 口[くち]ひげを 生[は]やして、 格好[かっこう] 良[よ]く 見[み]せようとしました。	口ひげ= 
\\	まぶた全体に、明るい色のアイシャドウをのせます。	
\\	まぶた 全体[ぜんたい]に、 明[あか]るい 色[いろ]のアイシャドウをのせます。	瞼=まぶた= 
\\	うちの犬は、まるでアイラインを入れたかのような目をしている。	
\\	うちの 犬[いぬ]は、まるでアイラインを 入[い]れたかのような 目[め]をしている。	
\\	アイラインがいつもうまく引けない。	
\\	アイラインがいつもうまく 引[ひ]けない。	
\\	口紅はどこで買えますか?	
\\	口紅[くちべに]はどこで 買[か]えますか?	
\\	着替えたとき、白いシャツに口紅がついてしまった。	
\\	着替[きが]えたとき、 白[しろ]いシャツに 口紅[くちべに]がついてしまった。	
\\	食事をすると、どうしても口紅が落ちてしまう。	
\\	食事[しょくじ]をすると、どうしても 口紅[くちべに]が 落[お]ちてしまう。	
\\	普段は、ほとんど化粧をしません。	
\\	普段[ふだん]は、ほとんど 化粧[けしょう]をしません。	
\\	そのタレントは、ちょっと化粧が濃すぎる。	
\\	そのタレントは、ちょっと 化粧[けしょう]が 濃[こ]すぎる。	濃い=こい= 
\\	濃いコーヒーを一杯飲みたい。	
\\	濃[こ]いコーヒーを 一杯[いっぱい] 飲[の]みたい。	濃い=こい= 
\\	ここでは、宗教的伝統が濃いようだ。	
\\	ここでは、 宗教[しゅうきょう] 的[てき] 伝統[でんとう]が 濃[こ]いようだ。	濃い=こい= 
\\	家に帰ると、化粧も落とさずにそのまま寝てしまった。	
\\	家[いえ]に 帰[かえ]ると、 化粧[けしょう]も 落[お]とさずにそのまま 寝[ね]てしまった。	
\\	一足早いけど誕生日おめでとう。	
\\	一足[いっそく] 早[はや]いけど 誕生[たんじょう] 日[び]おめでとう。	
\\	彼女は、
\\	陽性の男に恋をして感染したのです。	
\\	彼女[かのじょ]は、 
\\	陽性[ようせい]の 男[おとこ]に 恋[こい]をして 感染[かんせん]したのです。	
\\	彼女は20代の時に、自分が 
\\	陽性者だと分かりました。	
\\	彼女[かのじょ]は20 代[だい]の 時[とき]に、 自分[じぶん]が 
\\	陽性[ようせい] 者[しゃ]だと 分[わ]かりました。	
\\	その映画のテーマはとても普遍的なものである。	
\\	その 映画[えいが]のテーマはとても 普遍[ふへん] 的[てき]なものである。	普遍=ふへん= 
\\	喜びと悲しみは普遍的な人間の感情です。	
\\	喜[よろこ]びと 悲[かな]しみは 普遍[ふへん] 的[てき]な 人間[にんげん]の 感情[かんじょう]です。	
\\	それは学校の方針です。	
\\	それは 学校[がっこう]の 方針[ほうしん]です。	
\\	これは順序が間違っている。	
\\	これは 順序[じゅんじょ]が 間違[まちが]っている。	
\\	すべて順調だった?	
\\	すべて 順調[じゅんちょう]だった?	
\\	期待と不安との板挟みになっている。	
\\	期待[きたい]と 不安[ふあん]との 板挟[いたばさ]みになっている。	板挟み=いたばさみ= 
\\	私たちカンボジア人は板挟みになりました。	
\\	私[わたし]たちカンボジア 人[じん]は 板挟[いたばさ]みになりました。	板挟み=いたばさみ= 
\\	その犯罪者は、捕まるまで何年間も逃亡していた。	
\\	その 犯罪[はんざい] 者[しゃ]は、 捕[つか]まるまで 何[なん] 年間[ねんかん]も 逃亡[とうぼう]していた。	逃亡=とうぼう= 
\\	探検隊はキング・コングを捕獲して、ニューヨークに連れ帰りました。	
\\	探検[たんけん] 隊[たい]はキング・コングを 捕獲[ほかく]して、ニューヨークに 連れ帰[つれかえ]ります。	
\\	対処しなければならない問題がたくさんある。	
\\	対処[たいしょ]しなければならない 問題[もんだい]がたくさんある。	
\\	あなたが自分でそれに対処するのは無理です。	
\\	あなたが 自分[じぶん]でそれに 対処[たいしょ]するのは 無理[むり]です。	
\\	この問題は上司に相談するのが最善の対処法だと思います。	
\\	この 問題[もんだい]は 上司[じょうし]に 相談[そうだん]するのが 最善[さいぜん]の 対処[たいしょ] 法[ほう]だと 思[おも]います。	最善= 
\\	レストランで上着を脱いで、そのまま置いてきてしまった。	
\\	レストランで 上着[うわぎ]を 脱[ぬ]いで、そのまま 置[お]いてきてしまった。	
\\	シャツをハンガーにかけた。	
\\	シャツをハンガーにかけた。	
\\	クロークで上着を預けましょう。	
\\	クロークで 上着[うわぎ]を 預[あづ]けましょう。	
\\	彼は大雨の中を傘もささずに歩いていた。	
\\	彼[かれ]は 大雨[おおあめ]の 中[なか]を 傘[かさ]もささずに 歩[ある]いていた。	
\\	天気予報では夕方から雨らしいから、傘を持って行きなさい。	
\\	天気[てんき] 予報[よほう]では 夕方[ゆうがた]から 雨[あめ]らしいから、 傘[かさ]を 持[も]って 行[い]きなさい。	
\\	また電車に傘を置き忘れてしまった。	
\\	また 電車[でんしゃ]に 傘[かさ]を 置き忘[おきわす]れてしまった。	
\\	私はすぐ傘をなくすから、安いビニール傘で十分だ。	
\\	私[わたし]はすぐ 傘[かさ]をなくすから、 安[やす]いビニール 傘[かさ]で 十分[じゅうぶん]だ。	
\\	可愛い女の子が傘をさしかけてくれた。	
\\	可愛[かわい]い 女の子[おんなのこ]が 傘[かさ]をさしかけてくれた。	差し掛ける=さしかける= 
\\	彼女は傘を持っていなかったので、駅まで傘に入れてあげた。	
\\	彼女[かのじょ]は 傘[かさ]を 持[も]っていなかったので、 駅[えき]まで 傘[かさ]に 入[い]れてあげた。	
\\	彼は大きなかばんを持っていたので、どこか旅行に行ったのかもしれない。	
\\	彼[かれ]は 大[おお]きなかばんを 持[も]っていたので、どこか 旅行[りょこう]に 行[い]ったのかもしれない。	
\\	彼はかばんから、分厚い本を取り出した。	
\\	彼[かれ]はかばんから、 分厚[ぶあつ]い 本[ほん]を 取り出[とりだ]した。	分厚い=ぶあつい= 
\\	大事な書類なので、なくさないようにすぐにかばんにしまった。	
\\	大事[だいじ]な 書類[しょるい]なので、なくさないようにすぐにかばんにしまった。	
\\	しまう= 
\\	電車の網棚に、かばんを置き忘れた。	
\\	電車[でんしゃ]の 網棚[あみだな]に、かばんを 置き忘[おきわす]れた。	網棚=あみだな= 
\\	寒がりなので、朝起きたらすぐに靴下を履きます。	
\\	寒[さむ]がりなので、 朝[あさ] 起[お]きたらすぐに 靴下[くつした]を 履[は]きます。	〜がり= 
\\	男性はほとんどブーツを履きません、でも女性は履きます。	
\\	男性[だんせい]はほとんどブーツを 履[は]きません、でも 女性[じょせい]は 履[は]きます。	
\\	靴下を脱いだら、脱ぎっぱなしにしないで洗濯機に入れてください。	
\\	靴下[くつした]を 脱[ぬ]いだら、 脱[ぬ]ぎっぱなしにしないで 洗濯[せんたく] 機[き]に 入[い]れてください。	
\\	息子はまだ自分で靴のひもが結べない。	
\\	息子[むすこ]はまだ 自分[じぶん]で 靴[くつ]のひもが 結[むす]べない。	
\\	この間、自動で靴を磨く機械を見かけました。	
\\	この 間[かん]、 自動[じどう]で 靴[くつ]を 磨[みが]く 機械[きかい]を 見[み]かけました。	
\\	普段はいつもズボンで、めったにスカートは履きません。	
\\	普段[ふだん]はいつもズボンで、めったにスカートは 履[は]きません。	めったに=
\\	めったにたばこを吸わない。	
\\	めったにたばこを 吸[す]わない。	めったに= 
\\	これほど優秀な学生はめったにいない。	
\\	これほど 優秀[ゆうしゅう]な 学生[がくせい]はめったにいない。	
\\	こんなことはめったに起こらない。	
\\	こんなことはめったに 起[お]こらない。	
\\	パーティーに着ていくスーツのズボンに、アイロンをかけた。	
\\	パーティーに 着[き]ていくスーツのズボンに、アイロンをかけた。	
\\	公衆浴場では、絶対に下着をつけたまま入浴してはいけません。	
\\	公衆[こうしゅう] 浴場[よくじょう]では、 絶対[ぜったい]に 下着[したぎ]をつけたまま 入浴[にゅうよく]してはいけません。	公衆浴場=こうしゅうよくじょう= 
\\	セーターを後ろ前に着ていたことに、一日中気づかなかった。	
\\	セーターを 後ろ前[うしろまえ]に 着[き]ていたことに、一 日[にち] 中[ちゅう] 気[き]づかなかった。	
\\	最近、シャツのすそを出すのが、流行っているみたい。	
\\	最近[さいきん]、シャツのすそを 出[だ]すのが、 流行[はや]っているみたい。	すそ= 
\\	その店ではジーンズのすそをあげてもらうのに、三十分もかからない。	
\\	その 店[みせ]ではジーンズのすそをあげてもらうのに、 三十分[さんじゅっぷん]もかからない。	すそ= 
\\	このジャケット、サイズはぴったりだけど、ちょっとそでが短いみたい。	
\\	このジャケット、サイズはぴったりだけど、ちょっとそでが 短[みじか]いみたい。	
\\	腕相撲をしようと言うと、彼は早速そでをまくり上げて応じた。	
\\	腕相撲[うでずもう]をしようと 言[い]うと、 彼[かれ]は 早速[さっそく]そでをまくり 上[あ]げて 応[おう]じた。	腕相撲=うでずもう= 
\\	まくり上げる= 
\\	初めてそのユニフォームにそでを通したときの喜びは、忘れられない。	
\\	初[はじ]めてそのユニフォームにそでを 通[とお]したときの 喜[よろこ]びは、 忘[わす]れられない。	そでを通す= 
\\	彼女は洗い物をするとき、必ずゴム手袋をしている。	
\\	彼女[かのじょ]は 洗い物[あらいもの]をするとき、 必[かなら]ずゴム 手袋[てぶくろ]をしている。	
\\	手袋をはめたままで握手をするのは、失礼ですよ。	
\\	手袋[てぶくろ]をはめたままで 握手[あくしゅ]をするのは、 失礼[しつれい]ですよ。	
\\	サングラスをはずした。	
\\	サングラスをはずした。	はずす= 
\\	シートベルトをはずしました。	
\\	シートベルトをはずしました。	はずす= 
\\	ネクタイをするのは年に数回だ。	
\\	ネクタイをするのは 年[ねん]に 数[すう] 回[かい]だ。	
\\	クライアントとの打ち合わせのときは、ネクタイを締めることになっている。	
\\	クライアントとの 打ち合[うちあ]わせのときは、ネクタイを 締[し]めることになっている。	ネクタイを締める/する= 
\\	その会社は夏の間、服装規定を緩めました。	
\\	その 会社[かいしゃ]は 夏[なつ]の 間[あいだ]、 服装[ふくそう] 規定[きてい]を 緩[ゆる]めました。	服装規定=ふくそうきてい= 
\\	緩める=ゆるめる=1)(締まっているものを) 
\\	(結び目などを) 
\\	(気持ち・表情などを) 
\\	(取り締まりや規則などを) 
\\	(力などを) 
\\	(速度) 
\\	もし今、気を緩めたら、この試験に落ちるよ。	
\\	もし 今[いま]、 気[き]を 緩[ゆる]めたら、この 試験[しけん]に 落[お]ちるよ。	気を緩める=き を ゆるめる= 
\\	背中のファスナーが、自分で上げられない。	
\\	背中[せなか]のファスナーが、 自分[じぶん]で 上[あ]げられない。	
\\	かばんのファスナーが壊れて閉まらなくなった。	
\\	かばんのファスナーが 壊[こわ]れて 閉[し]まらなくなった。	
\\	ファスナーが引っかかった。	
\\	ファスナーが 引[ひ]っかかった。	引っかかる= 
\\	彼女はいつも自分によく似合う服を着ている。	
\\	彼女[かのじょ]はいつも 自分[じぶん]によく 似合[にあ]う 服[ふく]を 着[き]ている。	
\\	服を着替えるのに、こちらの部屋を使ってください。	
\\	服[ふく]を 着替[きが]えるのに、こちらの 部屋[へや]を 使[つか]ってください。	
\\	友達の結婚式に着ていく服を選ぶのに、付き合ってもらえますか?	
\\	友達[ともだち]の 結婚式[けっこんしき]に 着[き]ていく 服[ふく]を 選[えら]ぶのに、 付き合[つきあ]ってもらえますか?	
\\	このベルトをしていると、必ず空港の金属探知機に引っかかってしまう。	
\\	このベルトをしていると、 必[かなら]ず 空港[くうこう]の 金属[きんぞく] 探知[たんち] 機[き]に 引[ひ]っかかってしまう。	ベルトをする= 
\\	金属探知機=きん ぞく たん ち き 
\\	引っかかる= 
\\	食べ放題の店で食べ過ぎて苦しくなり、ベルトを緩めた。	
\\	食[た]べ 放題[ほうだい]の 店[みせ]で 食[た]べ 過[す]ぎて 苦[くる]しくなり、ベルトを 緩[ゆる]めた。	緩める=ゆるめる= 
\\	その町の男性は、みな独特の帽子をかぶっていた。	
\\	その 町[まち]の 男性[だんせい]は、みな 独特[どくとく]の 帽子[ぼうし]をかぶっていた。	独特=どくとく= 
\\	帽子をかぶる= 
\\	挨拶するときには、帽子をとるのがエチケットだ。	
\\	挨拶[あいさつ]するときには、 帽子[ぼうし]をとるのがエチケットだ。	
\\	女性は室内でも帽子を脱ぐ必要はない。	
\\	女性[じょせい]は 室内[しつない]でも 帽子[ぼうし]を 脱[ぬ]ぐ 必要[ひつよう]はない。	
\\	利き手をけがしたので、ボタンを留めるのが大変です。	
\\	利き手[ききて]をけがしたので、ボタンを 留[と]めるのが 大変[たいへん]です。	利き手= 
\\	ボタンを留める= 
\\	今朝は大急ぎで着替えたので、ボタンをかけ違えていた。	
\\	今朝[けさ]は 大急[おおいそ]ぎで 着替[きが]えたので、ボタンをかけ 違[ちが]えていた。	
\\	前から二列目の、眼鏡をかけている人が私の姉です。	
\\	前[まえ]から 二列目[にれつめ]の、 眼鏡[めがね]をかけている 人[ひと]が 私[わたし]の 姉[あね]です。	
\\	眼鏡を取ると1メートル先の人の顔も、よくわからない。	
\\	眼鏡[めがね]を 取[と]ると1メートル 先[さき]の 人[ひと]の 顔[かお]も、よくわからない。	
\\	ラーメンを食べるとき、いつも眼鏡が曇るのがうっとうしい。	
\\	ラーメンを 食[た]べるとき、いつも 眼鏡[めがね]が 曇[くも]るのがうっとうしい。	うっとうしい= 
\\	彼女は左手の薬指に指輪をしていた。	
\\	彼女[かのじょ]は 左手[ひだりて]の 薬指[くすりゆび]に 指輪[ゆびわ]をしていた。	薬指=くすり ゆび= 
\\	指輪をする= 
\\	結婚していますが、普段は指輪をつけていません。	
\\	結婚[けっこん]していますが、 普段[ふだん]は 指輪[ゆびわ]をつけていません。	
\\	それは私におあつらえ向きだ。	
\\	それは 私[わたし]におあつらえ 向[む]きだ。	おあつらえ向き= 
\\	私が着ているこのスーツは、あつらえたようにぴったりだ。	
\\	私[わたし]が 着[き]ているこのスーツは、あつらえたようにぴったりだ。	あつらえる= 
\\	私たちは、女性と子供の生命を救うための知識を持っています。	
\\	私[わたし]たちは、 女性[じょせい]と 子供[こども]の 生命[せいめい]を 救[すく]うための 知識[ちしき]を 持[も]っています。	
\\	この制度は女性を救うためにできたのです。	
\\	この 制度[せいど]は 女性[じょせい]を 救[すく]うためにできたのです。	
\\	宗教は人の悩みを救うばかりでなく、生活様式まで変えてしまう。	
\\	宗教[しゅうきょう]は 人[ひと]の 悩[なや]みを 救[すく]うばかりでなく、 生活[せいかつ] 様式[ようしき]まで 変[か]えてしまう。	生活様式=せいかつようしき= 
\\	貯蔵期間は約5年です。	
\\	貯蔵[ちょぞう] 期間[きかん]は 約[やく]5 年[ねん]です。	貯蔵=ちょぞう= 
\\	うちでは、料理を作るのと洗い物をするのを、交代でやっています。	
\\	うちでは、 料理[りょうり]を 作[つく]るのと 洗い物[あらいもの]をするのを、 交代[こうたい]でやっています。	交代=こうたい= 
\\	シンクに洗い物がたまっている状況が嫌なので、食器はすぐ洗う。	
\\	シンクに 洗い物[あらいもの]がたまっている 状況[じょうきょう]が 嫌[いや]なので、 食器[しょっき]はすぐ 洗[あら]う。	
\\	毎朝ゴミを出すのは、私の役目だ。	
\\	毎朝[まいあさ]ゴミを 出[だ]すのは、 私[わたし]の 役目[やくめ]だ。	
\\	きちんとゴミを分別するように、大家さんから注意された。	
\\	きちんとゴミを 分別[ふんべつ]するように、 大家[おおや]さんから 注意[ちゅうい]された。	
\\	この料理は大根の皮まで使うので、ゴミが出ません。	
\\	この 料理[りょうり]は 大根[だいこん]の 皮[かわ]まで 使[つか]うので、ゴミが 出[で]ません。	
\\	せっけんが目にしみだ。	
\\	せっけんが 目[め]にしみだ。	しみ= 
\\	ワインをこぼして、テーブルクロスに思いっきりしみをつけてしまった。	
\\	ワインをこぼして、テーブルクロスに 思[おも]いっきりしみをつけてしまった。	こぼす= 
\\	しみ= 
\\	このクリーナーを使うと、簡単にしみが落ちます。	
\\	このクリーナーを 使[つか]うと、 簡単[かんたん]にしみが 落[お]ちます。	しみ= 
\\	しみになるといけないから、すぐに洗ったほうがいい。	
\\	しみになるといけないから、すぐに 洗[あら]ったほうがいい。	しみになる= 
\\	上の階の住人は、よく夜遅くに洗濯をする。	
\\	上[うえ]の 階[かい]の 住人[じゅうにん]は、よく 夜[よる] 遅[おそ]くに 洗濯[せんたく]をする。	住人= 
\\	このジャケットは、家で洗濯できますか?	
\\	このジャケットは、 家[いえ]で 洗濯[せんたく]できますか?	
\\	やっと晴れたので、洗濯物を外に干すことができる。	
\\	やっと 晴[は]れたので、 洗濯[せんたく] 物[もの]を 外[そと]に 干[ほ]すことができる。	干す= 
\\	夕方からは雨みたいだから、昼すぎには洗濯物を取り込んでおいてね。	
\\	夕方[ゆうがた]からは 雨[あめ]みたいだから、 昼[ひる]すぎには 洗濯[せんたく] 物[もの]を 取り込[とりこ]んでおいてね。	
\\	やむを得ない事情で当レストランをたたむことになりました。	
\\	やむを 得[え]ない 事情[じじょう]で 当[とう]レストランをたたむことになりました。	やむを得ない= 
\\	たたむ= 
\\	私は服を全部たたんだ。	
\\	私[わたし]は 服[ふく]を 全部[ぜんぶ]たたんだ。	たたむ= 
\\	雨降りの日が続いたので、洗濯物がたまっています。	
\\	雨降[あめふ]りの 日[ひ]が 続[つづ]いたので、 洗濯[せんたく] 物[もの]がたまっています。	
\\	これは、ひどく腹立たしいことです。	
\\	これは、ひどく 腹立[はらだ]たしいことです。	
\\	私はスパムメールがどれほど腹立たしいものかを知っている。	
\\	私[わたし]はスパムメールがどれほど 腹立[はらだ]たしいものかを 知[し]っている。	
\\	変人かもしれない人をどうやって見極めたのですか?	
\\	変人[へんじん]かもしれない 人[ひと]をどうやって 見極[みきわ]めたのですか?	見極める=みきわめる= 
\\	以降、ロシアからの移住は絶えず続いている。	
\\	以降[いこう]、ロシアからの 移住[いじゅう]は 絶[た]えず 続[つづ]いている。	以降=いこう= 
\\	18歳未満も午後10時以降は入場が禁止されます。	
\\	歳[さい] 未満[みまん]も 午後[ごご]10 時[じ] 以降[いこう]は 入場[にゅうじょう]が 禁止[きんし]されます。	〜歳未満= さい・み・まん= 
\\	以降=いこう= 
\\	その日以降、二人のデートは始まった。	
\\	その 日[ひ] 以降[いこう]、 二人[ふたり]のデートは 始[はじ]まった。	以降=いこう= 
\\	このバーを手掛けて6年あまりになります。	
\\	このバーを 手掛[てが]けて6 年[ねん]あまりになります。	
\\	プリンスは作詞・作曲を自分で手掛けます。	
\\	プリンスは 作詞[さくし]・ 作曲[さっきょく]を 自分[じぶん]で 手掛[てが]けます。	
\\	ちなみに、アメリカでは「ビルマ」と呼び続けています。	
\\	ちなみに、アメリカでは「ビルマ」と 呼び続[よびつづ]けています。	
\\	仏教は6世紀に中国・朝鮮を経て日本に伝えられた。	
\\	仏教[ぶっきょう]は6 世紀[せいき]に中国・ 朝鮮[ちょうせん]を 経[へ]て日本に 伝[つた]えられた。	〜を経て=〜を へて 
\\	ここのショッピングセンターは10代の子たちのたまり場です。	
\\	ここのショッピングセンターは10 代[だい]の 子[こ]たちのたまり 場[ば]です。	たまり場= 
\\	渋谷と原宿は、とともに若者のたまり場だ。	
\\	渋谷[しぶや]と 原宿[はらじゅく]は、とともに 若者[わかもの]のたまり 場[ば]だ。	たまり場= 
\\	こちらの銀行では外国為替は扱っていますか?	
\\	こちらの 銀行[ぎんこう]では 外国[がいこく] 為替[かわせ]は 扱[あつか]っていますか?	
\\	ドル建ての郵便為替は受け付けます。	
\\	ドル 建[だ]ての 郵便[ゆうびん] 為替[かわせ]は 受け付[うけつ]けます。	建て=だて= 
\\	郵便為替=ゆうびんかわせ= 
\\	こちらで郵便為替は購入できますか?	
\\	こちらで 郵便[ゆうびん] 為替[かわせ]は 購入[こうにゅう]できますか?	郵便為替=ゆうびんかわせ= 
\\	購入=こうにゅう= 
\\	そこではダイエット炭酸飲料だけが購入できました。	
\\	そこではダイエット 炭酸[たんさん] 飲料[いんりょう]だけが 購入[こうにゅう]できました。	炭酸飲料=たんさんいんりょう= 
\\	購入=こうにゅう= 
\\	その店の販売員によると、多くの人々が贈り物としてそれらを購入している。	
\\	その 店[みせ]の 販売[はんばい] 員[いん]によると、 多[おお]くの 人々[ひとびと]が 贈り物[おくりもの]としてそれらを 購入[こうにゅう]している。	購入=こうにゅう= 
\\	われわれは断固として賃上げを要求する。	
\\	われわれは 断固[だんこ]として 賃上[ちんあ]げを 要求[ようきゅう]する。	断固=だんこ= 
\\	賃上げ=ちんあげ= 
\\	要求=ようきゅう= 
\\	最寄りのバス停まで歩いて行った。	
\\	最寄[もよ]りの バス停[ばすてい]まで 歩[ある]いて 行[い]った。	最寄り=もより= 
\\	最寄りの病院はどこでしょうか?	
\\	最寄[もよ]りの 病院[びょういん]はどこでしょうか?	最寄り=もより= 
\\	最寄りの駅を教えていただけますか?	
\\	最寄[もよ]りの 駅[えき]を 教[おし]えていただけますか?	最寄り=もより= 
\\	当リンクでは、最寄りのレストランを手早くお探しいただけます。	
\\	当[とう]リンクでは、 最寄[もよ]りのレストランを 手早[てばや]くお 探[さが]しいただけます。	最寄り=もより= 
\\	手早い=てばやい= 
\\	これらの壊れやすい彫刻を移動する際には、注意してください。	
\\	これらの 壊[こわ]れやすい 彫刻[ちょうこく]を 移動[いどう]する 際[さい]には、 注意[ちゅうい]してください。	彫刻=ちょうこく= 
\\	移動=いどう= 
\\	際=さい= 
\\	それが当時ビザを申請する際に私が挙げた理由でした。	
\\	それが 当時[とうじ]ビザを 申請[しんせい]する 際[さい]に 私[わたし]が 挙[あ]げた 理由[りゆう]でした。	際に=さいに= 
\\	それを決定する際はどのようなことを判断基準にするのですか?	
\\	それを 決定[けってい]する 際[さい]はどのようなことを 判断[はんだん] 基準[きじゅん]にするのですか?	際=さい= 
\\	ノートを取っておけば、論文を書こうとする際に役立つ。	
\\	ノートを 取[と]っておけば、 論文[ろんぶん]を 書[か]こうとする 際[さい]に 役立[やくだ]つ。	際に=さいに= 
\\	3時までにご用意致します。	
\\	時[じ]までにご 用意[ようい] 致[いた]します。	用意=ようい= 
\\	いつ用意が整いますか?	
\\	いつ 用意[ようい]が 整[ととの]いますか?	用意=ようい= 
\\	整う=ととのう= 
\\	お部屋が用意できるまでお待ちいただくことになります。	
\\	お 部屋[へや]が 用意[ようい]できるまでお 待[ま]ちいただくことになります。	用意=ようい= 
\\	ご飯の用意してくれる?	
\\	ご 飯[はん]の 用意[ようい]してくれる?	用意=ようい= 
\\	これをする用意はありますか?	
\\	これをする 用意[ようい]はありますか?	用意=ようい= 
\\	冬物のコートやジャケットを、全部まとめてクリーニングに出した。	
\\	冬物[ふゆもの]のコートやジャケットを、 全部[ぜんぶ]まとめてクリーニングに 出[だ]した。	全部まとめて= 
\\	そのワンピースは、まだクリーニングから戻ってきていません。	
\\	そのワンピースは、まだクリーニングから 戻[もど]ってきていません。	
\\	新聞をいくつかの束にまとめて、ひもで縛った。	
\\	新聞[しんぶん]をいくつかの 束[たば]にまとめて、ひもで 縛[しば]った。	束=たば= 
\\	縛る=しばる= 
\\	そのほうきを手に取って掃きなさい。	
\\	そのほうきを 手[て]に 取[と]って 掃[は]きなさい。	ほうき= 
\\	掃く=はく= 
\\	地元のボランティアたちが、ほうきで落ち葉を掃き集めていた。	
\\	地元[じもと]のボランティアたちが、ほうきで 落ち葉[おちば]を 掃[は]き 集[あつ]めていた。	
\\	棚の上には、分厚くほこりがたまっている。	
\\	棚[たな]の 上[うえ]には、 分厚[ぶあつ]くほこりがたまっている。	
\\	ツリーをしまう前に、よくほこりを払いなさい。	
\\	ツリーをしまう 前[まえ]に、よくほこりを 払[はら]いなさい。	ほこりを払う= 
\\	その植物の葉っぱは、すっかりほこりをかぶっていた。	
\\	その 植物[しょくぶつ]の 葉[は]っぱは、すっかりほこりをかぶっていた。	ほこりをかぶる= 
\\	高層ビルの窓ガラスを磨く仕事は、私には絶対無理です。	
\\	高層[こうそう]ビルの 窓[まど]ガラスを 磨[みが]く 仕事[しごと]は、 私[わたし]には 絶対[ぜったい] 無理[むり]です。	
\\	電車の窓はすっかり曇っていて、外が見えなかった。	
\\	電車[でんしゃ]の 窓[まど]はすっかり 曇[くも]っていて、 外[そと]が 見[み]えなかった。	
\\	窓ガラスが1枚、何者かに割らされていた。	
\\	窓[まど]ガラスが1 枚[まい]、 何者[なにもの]かに 割[わ]らされていた。	
\\	私は割れた窓ガラスを新しいものに取り換えた。	
\\	私[わたし]は 割[わ]れた 窓[まど]ガラスを 新[あたら]しいものに 取り換[とりか]えた。	
\\	窓を閉めてくれる?	
\\	窓[まど]を 閉[し]めてくれる?	
\\	白だと汚れが目立つから、別の色にしよう。	
\\	白[しろ]だと 汚[よご]れが 目立[めだ]つから、 別[べつ]の 色[いろ]にしよう。	
\\	この洗剤は、とにかく汚れがよく落ちる。	
\\	この 洗剤[せんざい]は、とにかく 汚[よご]れがよく 落[お]ちる。	
\\	このチーズは、ハーブとガーリックの味がする。	
\\	このチーズは、ハーブとガーリックの 味[あじ]がする。	
\\	このスープは、私には少し味が濃すぎる。	
\\	このスープは、 私[わたし]には 少[すこ]し 味[あじ]が 濃[こ]すぎる。	
\\	風邪をひいて、何を食べても味がわからない。	
\\	風邪[かぜ]をひいて、 何[なに]を 食[た]べても 味[あじ]がわからない。	
\\	あのレストランは、最近味が落ちたといううわさだ。	
\\	あのレストランは、 最近[さいきん] 味[み]が 落[お]ちたといううわさだ。	
\\	まず、鍋に油を引いて、弱火でニンニクを炒めます。	
\\	まず、 鍋[なべ]に 油[あぶら]を 引[ひ]いて、 弱火[よわび]でニンニクを 炒[いた]めます。	
\\	今、オーブンでローストチキンを焼いているところです。	
\\	今[いま]、オーブンでローストチキンを 焼[や]いているところです。	
\\	さらに2〜3分焼いたら、フライパンから取り出します。	
\\	さらに2〜3 分[ぷん] 焼[や]いたら、フライパンから 取り出[とりだ]します。	
\\	お昼に残り物のピザをオーブンで温めた。	
\\	お 昼[ひる]に 残り物[のこりもの]のピザをオーブンで 温[あたた]めた。	
\\	しゃもじで、ご飯をよそってもらえますか?	
\\	しゃもじで、ご 飯[はん]をよそってもらえますか?	しゃもじ= 
\\	肉にはやや強めにこしょうをしてください。	
\\	肉[にく]にはやや 強[つよ]めにこしょうをしてください。	やや=少し
\\	ラーメンにこしょうをかけようとしたら、ふたが取れてしまった。	
\\	ラーメンにこしょうをかけようとしたら、ふたが 取[と]れてしまった。	
\\	炊飯ジャーではなく、鍋でご飯を炊いています。	
\\	炊飯[すいはん]ジャーではなく、 鍋[なべ]でご 飯[はん]を 炊[た]いています。	炊飯ジャー=すいはん ジャー 
\\	卵を加えるときは、コンロから下ろすこと。	
\\	卵[たまご]を 加[くわ]えるときは、コンロから 下[お]ろすこと。	コンロ= 
\\	もう少しで、コンロの火を消すのを忘れて出かけるところだった。	
\\	もう 少[すこ]しで、コンロの 火[ひ]を 消[け]すのを 忘[わす]れて 出[で]かけるところだった。	
\\	台所に魚のにおいがこもるので、家では魚を焼きません。	
\\	台所[だいどころ]に 魚[さかな]のにおいがこもるので、 家[いえ]では 魚[さかな]を 焼[や]きません。	こもる= 
\\	単に塩とこしょうをしただけのステーキが、ものすごくおいしかった。	
\\	単[たん]に 塩[しお]とこしょうをしただけのステーキが、ものすごくおいしかった。	
\\	スープの味を見て、足りなければ塩を加えてください。	
\\	スープの 味[あじ]を 見[み]て、 足[た]りなければ 塩[しお]を 加[くわ]えてください。	
\\	健康のために、少し塩を控えるようにしています。	
\\	健康[けんこう]のために、 少[すこ]し 塩[しお]を 控[ひか]えるようにしています。	控える=ひかえる= 
\\	朝は大体卵を焼いて、あとはトーストとコーヒーです。	
\\	朝[あさ]は 大体[だいたい] 卵[たまご]を 焼[や]いて、あとはトーストとコーヒーです。	
\\	彼女は私に卵をゆでてくれた。	
\\	彼女[かのじょ]は 私[わたし]に 卵[たまご]をゆでてくれた。	
\\	鍋を火にかけている間は、火のそばから離れないように。	
\\	鍋[なべ]を 火[ひ]にかけている 間[ま]は、 火[ひ]のそばから 離[はな]れないように。	
\\	肉を炒めてから、野菜を入れてください。	
\\	肉[にく]を 炒[いた]めてから、 野菜[やさい]を 入[い]れてください。	
\\	日本郵便は年賀はがきのデザインの種類を増やした。	
\\	日本[にっぽん] 郵便[ゆうびん]は 年賀[ねんが]はがきのデザインの 種類[しゅるい]を 増[ふ]やした。	
\\	友達に年賀状を送った。	
\\	友達[ともだち]に 年賀状[ねんがじょう]を 送[おく]った。	
\\	いくつかの爆弾がそこで爆破された。	
\\	いくつかの 爆弾[ばくだん]がそこで 爆破[ばくは]された。	
\\	そのアクション映画では、バスが爆破されました。	
\\	そのアクション 映画[えいが]では、バスが 爆破[ばくは]されました。	
\\	イラクではテロ爆破がよく起こるようになっている。	
\\	イラクではテロ 爆破[ばくは]がよく 起[お]こるようになっている。	
\\	エルサレムで爆破事件があったのは2004年以来のことである。	
\\	エルサレムで 爆破[ばくは] 事件[じけん]があったのは2004 年[ねん] 以来[いらい]のことである。	
\\	国連のイラク本部は8月に爆破されました。	
\\	国連[こくれん]のイラク 本部[ほんぶ]は 8月[はちがつ]に 爆破[ばくは]されました。	
\\	動物園では入園者が900万人を突破しました。	
\\	動物[どうぶつ] 園[えん]では 入園[にゅうえん] 者[しゃ]が900 万[まん] 人[にん]を 突破[とっぱ]しました。	
\\	ちょうど突破口を見つけたよ。	
\\	ちょうど 突破口[とっぱこう]を 見[み]つけたよ。	突破口=とっぱこう= 
\\	友達はみな、ケイトの、社長になりたいという願いを後押しした。	
\\	友達[ともだち]はみな、ケイトの、 社長[しゃちょう]になりたいという 願[ねが]いを 後押[あとお]しした。	
\\	彼女の成功は、人生を変えたいという気持ちに後押しされた。	
\\	彼女[かのじょ]の 成功[せいこう]は、 人生[じんせい]を 変[か]えたいという 気持[きも]ちに 後押[あとお]しされた。	
\\	良い成績が、私の自信を後押ししました。	
\\	良[よ]い 成績[せいせき]が、 私[わたし]の 自信[じしん]を 後押[あとお]ししました。	
\\	誕生日パーティーに遅れちゃったけど来ないよりはいいでしょ?	
\\	誕生[たんじょう] 日[び]パーティーに 遅[おく]れちゃったけど 来[こ]ないよりはいいでしょ?	
\\	近々ここよりも大きなオフィスに移動しなくてはなりません。	
\\	近々[ちかぢか]ここよりも 大[おお]きなオフィスに 移動[いどう]しなくてはなりません。	
\\	近々ご連絡致します。	
\\	近々[ちかぢか]ご 連絡[れんらく] 致[いた]します。	
\\	近々また話そうね。	
\\	近々[ちかぢか]また 話[はな]そうね。	
\\	「誰だったの?」と、妻が何げなく聞きました。	
\\	誰[だれ]だったの?」と、 妻[つま]が 何[なに]げなく 聞[き]きました。	何気なく=なにげなく= 
\\	彼は彼女を何げなく見た。	
\\	彼[かれ]は 彼女[かのじょ]を 何[なに]げなく 見[み]た。	何気なく=なにげなく= 
\\	三日間の休暇でフランスに行くなんてとんでもない。	
\\	三日間[みっかかん]の 休暇[きゅうか]でフランスに 行[い]くなんてとんでもない。	
\\	その国はとんでもない方向に向かっている。	
\\	その 国[くに]はとんでもない 方向[ほうこう]に 向[む]かっている。	
\\	それはとんでもない嘘だ。	
\\	それはとんでもない 嘘[うそ]だ。	
\\	それはとんでもない失敗でした。	
\\	それはとんでもない 失敗[しっぱい]でした。	
\\	太陽は約一分半の間、完全に隠れます。	
\\	太陽[たいよう]は 約[やく] 一分半[いっぷんはん]の 間[あいだ]、 完全[かんぜん]に 隠[かく]れます。	
\\	彼は隠れてタバコを吸うために外へ出た。	
\\	彼[かれ]は 隠[かく]れてタバコを 吸[す]うために 外[そと]へ 出[で]た。	
\\	あなたを待っている間、ある男性が私になれなれしく話してかけてきた。	
\\	あなたを 待[ま]っている 間[あいだ]、ある 男性[だんせい]が 私[わたし]になれなれしく 話[はな]してかけてきた。	馴れ馴れしい=なれなれしい= 
\\	この青年は入国を拒否された。	
\\	この 青年[せいねん]は 入国[にゅうこく]を 拒否[きょひ]された。	
\\	これを拒否とは受け取らないでください。	
\\	これを 拒否[きょひ]とは 受け取[うけと]らないでください。	
\\	ひき肉が売っていなかったので、自分で肉をミンチにした。	
\\	ひき 肉[にく]が 売[う]っていなかったので、 自分[じぶん]で 肉[にく]をミンチにした。	ひき肉= 
\\	アスパラガスに薄切りの肉を巻いた。	
\\	アスパラガスに 薄切[うすぎ]りの 肉[にく]を 巻[ま]いた。	巻く=まく= 
\\	タマネギを茶色になるまでバターで炒めなさい。	
\\	タマネギを 茶色[ちゃいろ]になるまでバターで 炒[いた]めなさい。	
\\	ホットケーキにバターを塗って、メープルシロップをたっぷりかけた。	
\\	ホットケーキにバターを 塗[ぬ]って、メープルシロップをたっぷりかけた。	塗る=ぬる= 
\\	パンが残ったら、冷凍しておけばいい。	
\\	パンが 残[のこ]ったら、 冷凍[れいとう]しておけばいい。	冷凍する=れいとうする= 
\\	包丁でジャガイモの皮をむくのは、難しい。	
\\	包丁[ほうちょう]でジャガイモの 皮[かわ]をむくのは、 難[むずか]しい。	包丁=ほうちょう= 
\\	ジャガイモ= 
\\	皮をむく= 
\\	材料をすべてボウルに入れて、よく混ぜてください。	
\\	材料[ざいりょう]をすべてボウルに 入[い]れて、よく 混[ま]ぜてください。	
\\	やかんでお湯を沸かすより、この電気ポットの方が速い。	
\\	やかんでお 湯[ゆ]を 沸[わ]かすより、この 電気[でんき]ポットの 方[ほう]が 速[はや]い。	
\\	鍋物は、肉や野菜を切るだけで準備できるので、楽だ。	
\\	鍋物[なべもの]は、 肉[にく]や 野菜[やさい]を 切[き]るだけで 準備[じゅんび]できるので、 楽[らく]だ。	
\\	普段は外食ばかりで、家ではほとんど料理をしません。	
\\	普段[ふだん]は 外食[がいしょく]ばかりで、 家[いえ]ではほとんど 料理[りょうり]をしません。	
\\	彼は人に教えられるくらい料理が上手だ。	
\\	彼[かれ]は 人[ひと]に 教[おし]えられるくらい 料理[りょうり]が 上手[じょうず]だ。	
\\	一休みしてお茶にしましょうか?	
\\	一休[ひとやす]みしてお 茶[ちゃ]にしましょうか?	一休み=ひとやすみ= 
\\	お客さんが来ると、まずお茶を出してもてなすのが習慣だ。	
\\	お 客[きゃく]さんが 来[く]ると、まずお 茶[ちゃ]を 出[だ]してもてなすのが 習慣[しゅうかん]だ。	もてなす= 
\\	今朝は朝ご飯を食べそこなったので、おなかがすいた。	
\\	今朝[けさ]は 朝[あさ]ご 飯[はん]を 食[た]べそこなったので、おなかがすいた。	損なう=そこなう= 
\\	おなかがぺこぺこで死にそうです。	
\\	おなかがぺこぺこで 死[し]にそうです。	
\\	おなかがぺこぺこだ。	
\\	おなかがぺこぺこだ。	
\\	シーンとしたエレベーターの中で、おなかが鳴って恥ずかしかった。	
\\	シーンとしたエレベーターの 中[なか]で、おなかが 鳴[な]って 恥[は]ずかしかった。	シーンと= 
\\	練習を始める前に、何かおなかに入れておかないともたないよ。	
\\	練習[れんしゅう]を 始[はじ]める 前[まえ]に、 何[なに]かおなかに 入[い]れておかないともたないよ。	
\\	仕事中についつい、おやつをつまんでしまう。	
\\	仕事[しごと] 中[ちゅう]についつい、おやつをつまんでしまう。	ついつい= 
\\	つまむ= 
\\	あまりに大盛りだったので、ご飯を残してしまった。	
\\	あまりに 大盛[おおも]りだったので、ご 飯[はん]を 残[のこ]してしまった。	大盛り=おおもり= 
\\	子供たちはみな、きれいにご飯を平らげた。	
\\	子供[こども]たちはみな、きれいにご 飯[はん]を 平[たい]らげた。	平らげる=たいらげる= 
\\	その店では、無料でご飯をお代わりすることができる。	
\\	その 店[みせ]では、 無料[むりょう]でご 飯[はん]をお 代[か]わりすることができる。	
\\	私は娘に罰として朝ご飯を抜きにすることにしました。	
\\	私[わたし]は 娘[むすめ]に 罰[ばち]として 朝[あさ]ご 飯[はん]を 抜[ぬ]きにすることにしました。	
\\	私はグチをこぼすようなタイプではありません。	
\\	私[わたし]はグチをこぼすようなタイプではありません。	愚痴をこぼす=ぐち を こぼす= 
\\	私がお皿を洗うから、それを拭いてもらえますか。	
\\	私[わたし]がお 皿[さら]を 洗[あら]うから、それを 拭[ふ]いてもらえますか。	
\\	高級な六枚組のセットのお皿を一枚割ってしまった。	
\\	高級[こうきゅう]な 六枚組[ろくまいぐみ]のセットのお 皿[さら]を 一枚[いちまい] 割[わ]ってしまった。	高級な=こうきゅう(な)= 
\\	何か食事で気を付けていることはありますか?	
\\	何[なに]か 食事[しょくじ]で 気[き]を 付[つ]けていることはありますか?	
\\	ここにあった英和辞典、知らない?	
\\	ここにあった 英和[えいわ] 辞典[じてん]、 知[し]らない?	
\\	昨日山田が訪ねて来た。この男は私の高校時代のクラスメートだった。	
\\	昨日[きのう] 山田[やまだ]が 訪[たず]ねて 来[き]た。この 男[おとこ]は 私[わたし]の 高校[こうこう] 時代[じだい]のクラスメートだった。	訪ねる=たずねる= 
\\	昨日神戸の小学校で火事があった。警察が今原因を調べている。	
\\	昨日[きのう] 神戸[こうべ]の 小学校[しょうがっこう]で 火事[かじ]があった。 警察[けいさつ]が 今[こん] 原因[げんいん]を 調[しら]べている。	
\\	この大学は学生数はどのぐらいですか。	
\\	この 大学[だいがく]は 学生[がくせい] 数[すう]はどのぐらいですか。	
\\	小林さんが来たら知らせて下さい。	
\\	小林[こばやし]さんが 来[き]たら 知[し]らせて 下[くだ]さい。	
\\	私がそこにいた時には異常はなかった。	
\\	私[わたし]がそこにいた 時[とき]には 異常[いじょう]はなかった。	異常=いじょう= 
\\	友達が訪ねて来るのでうちを空けるわけにいかない。	
\\	友達[ともだち]が 訪[たず]ねて 来[く]るのでうちを 空[あ]けるわけにいかない。	訪ねる=たずねる= 
\\	みんながよく聞こえるように、マイクを使って下さい。	
\\	みんながよく 聞[き]こえるように、マイクを 使[つか]って 下[くだ]さい。	
\\	ジョージは漢字は難しくないと言っている。	
\\	ジョージは 漢字[かんじ]は 難[むずか]しくないと 言[い]っている。	
\\	質問されてもだまっていて下さい。	
\\	質問[しつもん]されてもだまっていて 下[くだ]さい。	
\\	自分の頑固さに我ながらあきれた。	
\\	自分[じぶん]の 頑固[がんこ]さに 我[わが]ながらあきれた。	頑固=がんこ= 
\\	あきれる= 
\\	底の見えない日本の不況を描く。	
\\	底[そこ]の 見[み]えない日本の 不況[ふきょう]を 描[えが]く。	底=そこ= 
\\	あなたたち全員を心の底から信頼している。	
\\	あなたたち 全員[ぜんいん]を 心[こころ]の 底[そこ]から 信頼[しんらい]している。	心の底=こころ の そこ= 
\\	私は全く徹底した人間です。	
\\	私[わたし]は 全[まった]く 徹底[てってい]した 人間[にんげん]です。	
\\	この部屋は徹底的に掃除する必要がある。	
\\	この 部屋[へや]は 徹底的[てっていてき]に 掃除[そうじ]する 必要[ひつよう]がある。	徹底=てってい= 
\\	すべてのことが徹底的に考え抜かれた。	
\\	すべてのことが 徹底的[てっていてき]に 考え抜[かんがえぬ]かれた。	考え抜く=かんがえぬく= 
\\	警察はその事件の徹底的な調査を行った。	
\\	警察[けいさつ]はその 事件[じけん]の 徹底的[てっていてき]な 調査[ちょうさ]を 行[おこな]った。	
\\	彼はどん底の生活を味わった。	
\\	彼[かれ]はどん 底[ぞこ]の 生活[せいかつ]を 味[あじ]わった。	どん底=どんぞこ= 
\\	その劇の根底にあるテーマは伝統です。	
\\	その 劇[げき]の 根底[こんてい]にあるテーマは 伝統[でんとう]です。	劇=げき= 
\\	根底=こんてい= 
\\	心の奥底では、それが嘘だと彼には分かっていた。	
\\	心[こころ]の 奥底[おくそこ]では、それが 嘘[うそ]だと 彼[かれ]には 分[わ]かっていた。	心の奥底=こころ の おくそこ= 
\\	戦争は私たちの心の奥底に根づいている。	
\\	戦争[せんそう]は 私[わたし]たちの 心[こころ]の 奥底[おくそこ]に 根[ね]づいている。	心の奥底=こころ の おくそこ= 
\\	根づく=ねづく= 
\\	真のインスピレーションとは、心の奥底から得られるものだ。	
\\	真[しん]のインスピレーションとは、 心[こころ]の 奥底[おくそこ]から 得[え]られるものだ。	心の奥底=こころ の おくそこ= 
\\	真の=しんの= 
\\	その昔、マフィアは、多くの酒屋を支配していました。	
\\	その 昔[むかし]、マフィアは、 多[おお]くの 酒屋[さかや]を 支配[しはい]していました。	
\\	私は部屋を真っ暗にしないと眠れない。	
\\	私[わたし]は 部屋[へや]を 真っ暗[まっくら]にしないと 眠[ねむ]れない。	
\\	秋子は恐怖のあまり声も出なかった。	
\\	秋子[あきこ]は 恐怖[きょうふ]のあまり 声[ごえ]も 出[で]なかった。	
\\	私は喜びのあまり思わず隣の人に抱きついてしまった。	
\\	私[わたし]は 喜[よろこ]びのあまり 思[おも]わず 隣[となり]の 人[ひと]に 抱[だ]きついてしまった。	抱きつく=だきつく= 
\\	今度の会合は形式を重んずるあまり内容が乏しくなってしまった。	
\\	今度[こんど]の 会合[かいごう]は 形式[けいしき]を 重[おも]んずるあまり 内容[ないよう]が 乏[とぼ]しくなってしまった。	会合=かいごう= 
\\	形式=けいしき= 
\\	重んずる=おもんずる= 
\\	乏しい=とぼしい= 
\\	彼らは心配のあまり食事も喉を通らない様子だった。	
\\	彼[かれ]らは 心配[しんぱい]のあまり 食事[しょくじ]も 喉[のど]を 通[とお]らない 様子[ようす]だった。	
\\	昨日ビールを飲むあまり今日頭が痛い。	
\\	昨日[きのう]ビールを 飲[の]むあまり 今日[きょう] 頭[あたま]が 痛[いた]い。	
\\	大学はよければよいほど入るのが難しいです。	
\\	大学[だいがく]はよければよいほど 入[はい]るのが 難[むずか]しいです。	
\\	玄米はかめばかむほど味が出る。	
\\	玄米[げんまい]はかめばかむほど 味[あじ]が 出[で]る。	玄米=げんまい= 
\\	かむ= 
\\	日本では子供ばかりか大人さえ漫画を読んでいる。	
\\	日本[にっぽん]では 子供[こども]ばかりか 大人[おとな]さえ 漫画[まんが]を 読[よ]んでいる。	
\\	あの人は絵を見て楽しむばかりか、自分でも絵を描く。	
\\	あの 人[ひと]は 絵[え]を 見[み]て 楽[たの]しむばかりか、 自分[じぶん]でも 絵[え]を 描[えが]く。	
\\	アメリカでは大学生ばかりか、中学生、高校生さえ日本語を勉強している。	
\\	アメリカでは 大学生[だいがくせい]ばかりか、 中学生[ちゅうがくせい]、 高校生[こうこうせい]さえ 日本語[にほんご]を 勉強[べんきょう]している。	
\\	僕の寮の部屋は狭いばかりか、窓さえないんです。	
\\	僕[ぼく]の 寮[りょう]の 部屋[へや]は 狭[せま]いばかりか、 窓[まど]さえないんです。	
\\	トムは漢字が読めないばかりか、平仮名さえ読めない。	
\\	トムは 漢字[かんじ]が 読[よ]めないばかりか、 平仮名[ひらがな]さえ 読[よ]めない。	
\\	父は食べるのが大好きなばかりか、料理をするのも大好きです。	
\\	父[ちち]は 食[た]べるのが 大好[だいす]きなばかりか、 料理[りょうり]をするのも 大好[だいす]きです。	
\\	あの人は勉強だけでなくスポーツもよく出来る。	
\\	あの 人[ひと]は 勉強[べんきょう]だけでなくスポーツもよく 出来[でき]る。	
\\	この本は面白いばかりでなく、とてもためになる。	
\\	この 本[ほん]は 面白[おもしろ]いばかりでなく、とてもためになる。	ためになる= 
\\	私は日本語が話せるどころか、一度も勉強したことがありません。	
\\	私[わたし]は 日本語[にほんご]が 話[はな]せるどころか、一 度[ど]も 勉強[べんきょう]したことがありません。	
\\	スミスさんは日本語が書けないどころか、日本語で小説さえ書ける。	
\\	スミスさんは 日本語[にほんご]が 書[か]けないどころか、 日本語[にほんご]で 小説[しょうせつ]さえ 書[か]ける。	
\\	クラークさんは日本語が話せるどころか、韓国語さえ話せる。	
\\	クラークさんは 日本語[にほんご]が 話[はな]せるどころか、 韓国[かんこく] 語[ご]さえ 話[はな]せる。	
\\	ジムは日本語で会話出来ないばかりか、簡単な挨拶も出来ない。	
\\	ジムは 日本語[にほんご]で 会話[かいわ] 出来[でき]ないばかりか、 簡単[かんたん]な 挨拶[あいさつ]も 出来[でき]ない。	
\\	そんなことを人に言うべきじゃありません。	
\\	そんなことを 人[ひと]に 言[い]うべきじゃありません。	
\\	君も来るべきでしたよ。	
\\	君[きみ]も 来[きた]るべきでしたよ。	
\\	山田には話すべきじゃなかった。	
\\	山田[やまだ]には 話[はな]すべきじゃなかった。	
\\	話すべきことは全部話しました。	
\\	話[はな]すべきことは 全部[ぜんぶ] 話[はな]しました。	
\\	自分のことは自分ですべきだ。	
\\	自分[じぶん]のことは 自分[じぶん]ですべきだ。	
\\	今、家を買うべきじゃないよ。	
\\	今[いま]、 家[いえ]を 買[か]うべきじゃないよ。	
\\	それは課長にも言っておくべきだったね。	
\\	それは 課長[かちょう]にも 言[い]っておくべきだったね。	
\\	彼は結婚なんかすべきじゃなかったんだ。	
\\	彼[かれ]は 結婚[けっこん]なんかすべきじゃなかったんだ。	
\\	我々はもっと創造的であるべきだ。	
\\	我々[われわれ]はもっと 創造[そうぞう] 的[てき]であるべきだ。	
\\	調査の結果、驚くべきことが分かった。	
\\	調査[ちょうさ]の 結果[けっか]、 驚[おどろ]くべきことが 分[わ]かった。	
\\	田中は全く軽蔑すべき男だ。	
\\	田中[たなか]は 全[まった]く 軽蔑[けいべつ]すべき 男[おとこ]だ。	軽蔑=けいべつ= 
\\	あるべきところに記述がない。	
\\	あるべきところに 記述[きじゅつ]がない。	
\\	学生は勉強するものだ。	
\\	学生[がくせい]は 勉強[べんきょう]するものだ。	「ものだ」
\\	「べきだ」
\\	お金と良心は両立し得ないものだ。	
\\	お 金[かね]と 良心[りょうしん]は 両立[りょうりつ]し 得[え]ないものだ。	「得ない」
\\	-ます 
\\	このレポートは山田さんが書き直すはずだ。	
\\	このレポートは 山田[やまだ]さんが 書き直[かきなお]すはずだ。	
\\	この本はここの図書館にあるはずだ。	
\\	この 本[ほん]はここの 図書館[としょかん]にあるはずだ。	
\\	この本はここの図書館にあるべきだ。	
\\	この 本[ほん]はここの 図書館[としょかん]にあるべきだ。	
\\	ガソリンを十ドル分入れておきました。	
\\	ガソリンを十ドル 分[ぶん] 入[い]れておきました。	
\\	私は今日三日分の仕事を片付けた。	
\\	私[わたし]は 今日[きょう] 三日[みっか] 分[ぶん]の 仕事[しごと]を 片付[かたづ]けた。	
\\	現金の不足分は小切手で払います。	
\\	現金[げんきん]の 不足[ふそく] 分[ぶん]は 小切手[こぎって]で 払[はら]います。	小切手=こぎって= 
\\	会議の資料を六人分用意しておいて下さい。	
\\	会議[かいぎ]の 資料[しりょう]を 六人[ろくにん] 分[ぶん] 用意[ようい]しておいて 下[くだ]さい。	
\\	ここは後でサインをしますので二行分あけておいて下さい。	
\\	ここは 後[あと]でサインをしますので二 行[こう] 分[ぶん]あけておいて 下[くだ]さい。	
\\	私達は四ヶ月分のボーナスをもらった。	
\\	私[わたし] 達[たち]は 四ヶ月[よんかげつ] 分[ぶん]のボーナスをもらった。	
\\	トラック三台分のごみが出た。	
\\	トラック三 台[だい] 分[ぶん]のごみが 出[で]た。	
\\	政府は十万人分の食糧を被災地に送った。	
\\	政府[せいふ]は 十万[じゅうまん] 人[にん] 分[ぶん]の 食糧[しょくりょう]を 被災[ひさい] 地[ち]に 送[おく]った。	食糧=しょくりょう= 
\\	吉田は寿司を五人前平らげた。	
\\	吉田[よしだ]は 寿司[すし]を五 人前[にんまえ] 平[たい]らげた。	平らげる=たいらげる= 
\\	いつでも規律を維持しなさい。	
\\	いつでも 規律[きりつ]を 維持[いじ]しなさい。	規律=きりつ= 
\\	維持=いじ= 
\\	13人が死亡し、10人が今も行方不明となっている。	
\\	人[にん]が 死亡[しぼう]し、10 人[にん]が 今[いま]も 行方[ゆくえ] 不明[ふめい]となっている。	行方不明=ゆくえふめい= 
\\	2年間、その兵士は敵の捕虜となっていた。	
\\	年間[ねんかん]、その 兵士[へいし]は 敵[てき]の 捕虜[ほりょ]となっていた。	捕虜=ほりょ= 
\\	は特に若者たちの間で人気となっている。	
\\	は 特[とく]に 若者[わかもの]たちの 間[ま]で 人気[にんき]となっている。	
\\	神社やお寺は日本人にとって、生活の一部となっています。	
\\	神社[じんじゃ]やお 寺[てら]は 日本人[にっぽんじん]にとって、 生活[せいかつ]の 一部[いちぶ]となっています。	
\\	この料理は2人前となっていますが、よろしいでしょうか?	
\\	この 料理[りょうり]は2 人前[にんまえ]となっていますが、よろしいでしょうか?	
\\	すしレストランは、今や世界中の都市で普通の光景となっている。	
\\	すしレストランは、 今[いま]や 世界中[せかいじゅう]の 都市[とし]で 普通[ふつう]の 光景[こうけい]となっている。	今や= 
\\	光景=こうけい= 
\\	この島をめぐる紛争は二国間で長くわだかまりとなっています。	
\\	この 島[しま]をめぐる 紛争[ふんそう]は二 国[こく] 間[かん]で 長[なが]くわだかまりとなっています。	紛争=ふんそう= 
\\	わだかまり= 
\\	彼はなんとも母国愛の強い人だ。	
\\	彼[かれ]はなんとも 母国[ぼこく] 愛[あい]の 強[つよ]い 人[ひと]だ。	なんとも= 
\\	その会社は、高成長率を維持することができた。	
\\	その 会社[かいしゃ]は、 高[こう] 成長[せいちょう] 率[りつ]を 維持[いじ]することができた。	成長率=せいちょうりつ= 
\\	日本の経済力は伸びている。だが、いつまで続くかは分からない。	
\\	日本[にっぽん]の 経済[けいざい] 力[りょく]は 伸[の]びている。だが、いつまで 続[つづ]くかは 分[わ]からない。	
\\	都会の生活は便利だ。だが、ストレスが多すぎる。	
\\	都会[とかい]の 生活[せいかつ]は 便利[べんり]だ。だが、ストレスが 多[おお]すぎる。	
\\	私は彼女とは初めて会った。だが、前から知っていたような親しみを感じた。	
\\	私[わたし]は 彼女[かのじょ]とは 初[はじ]めて 会[あ]った。だが、 前[まえ]から 知[し]っていたような 親[した]しみを 感[かん]じた。	
\\	山本は医者に何度もたばこをやめるように言われた。だが、やめる気はないらしい。	
\\	山本[やまもと]は 医者[いしゃ]に 何[なん] 度[ど]もたばこをやめるように 言[い]われた。だが、やめる 気[き]はないらしい。	「らしい」
\\	あの人には才能がある。だが、その才能を使っていない。	
\\	あの 人[ひと]には 才能[さいのう]がある。だが、その 才能[さいのう]を 使[つか]っていない。	
\\	今日の試験のために寝ないで勉強した。だが、さっぱりできなかった。	
\\	今日[きょう]の 試験[しけん]のために 寝[ね]ないで 勉強[べんきょう]した。だが、さっぱりできなかった。	さっぱり= 
\\	さっぱり分からない。	
\\	さっぱり 分[わ]からない。	さっぱり= 
\\	あなたたちが何を話しているのかまださっぱり分かりません。	
\\	あなたたちが 何[なに]を 話[はな]しているのかまださっぱり 分[わ]かりません。	さっぱり= 
\\	ちょっとさっぱりさせてね。	
\\	ちょっとさっぱりさせてね。	さっぱり= 
\\	夕べはよく眠れたので今朝はさっぱりした気分です。	
\\	夕[ゆう]べはよく 眠[ねむ]れたので 今朝[けさ]はさっぱりした 気分[きぶん]です。	
\\	妹はよく勉強するし、頭もいい。だが、成績はなぜかよくない。	
\\	妹[いもうと]はよく 勉強[べんきょう]するし、 頭[あたま]もいい。だが、 成績[せいせき]はなぜかよくない。	
\\	自らの命を絶つという行為は決して許されることではない。	
\\	自[みずか]らの 命[いのち]を 絶[た]つという 行為[こうい]は 決[けっ]して 許[ゆる]されることではない。	自ら=おのずから= 
\\	みずから= 
\\	絶つ=たつ= 
\\	自らの喜びをダンスで表現する女性の姿に感動しました。	
\\	自[みずか]らの 喜[よろこ]びをダンスで 表現[ひょうげん]する 女性[じょせい]の 姿[すがた]に 感動[かんどう]しました。	自ら=みずから= 
\\	おのずから= 
\\	自らの経験からこのことを学びました。	
\\	自[みずか]らの 経験[けいけん]からこのことを 学[まな]びました。	自ら=みずから= 
\\	おのずから= 
\\	自らを「運のいい男」と話す。	
\\	自[みずか]らを
\\	運[うん]のいい 男[おとこ]」と 話[はな]す。	自ら=みずから= 
\\	おのずから= 
\\	その大臣は自らの発言を公式に謝罪しました。	
\\	その 大臣[だいじん]は 自[みずか]らの 発言[はつげん]を 公式[こうしき]に 謝罪[しゃざい]しました。	自ら=みずから= 
\\	おのずから= 
\\	公式= 
\\	あなたが犯したこの罪に対してどんな罰がふさわしいと思いますか。	
\\	あなたが 犯[おか]したこの 罪[つみ]に 対[たい]してどんな 罰[ばつ]がふさわしいと 思[おも]いますか。	ふさわしい= 
\\	あらゆる機会、あらゆる状況にふさわしいお茶がある。	
\\	あらゆる 機会[きかい]、あらゆる 状況[じょうきょう]にふさわしいお 茶[ちゃ]がある。	ふさわしい= 
\\	あらゆる= 
\\	この本は子供が読むのにふさわしい。	
\\	この 本[ほん]は 子供[こども]が 読[よ]むのにふさわしい。	ふさわしい= 
\\	すべてのことにはふさわしい時と場所がある。	
\\	すべてのことにはふさわしい 時[とき]と 場所[ばしょ]がある。	ふさわしい= 
\\	今この瞬間ほどふさわしい時は決して巡って来ないでしょう。	
\\	今[いま]この 瞬間[しゅんかん]ほどふさわしい 時[とき]は 決[けっ]して 巡[めぐ]って 来[こ]ないでしょう。	ふさわしい= 
\\	巡る=めぐる= 
\\	今のところ、その会社には彼にふさわしい仕事がない。	
\\	今[いま]のところ、その 会社[かいしゃ]には 彼[かれ]にふさわしい 仕事[しごと]がない。	ふさわしい= 
\\	会ったときからあなたは私にふさわしい人だと思っていた。	
\\	会[あ]ったときからあなたは 私[わたし]にふさわしい 人[ひと]だと 思[おも]っていた。	ふさわしい= 
\\	これは初めのうち、かなり困惑させられました。	
\\	これは 初[はじ]めのうち、かなり 困惑[こんわく]させられました。	
\\	私はとても困惑する。	
\\	私[わたし]はとても 困惑[こんわく]する。	
\\	同僚の言葉に私は困惑した。	
\\	同僚[どうりょう]の 言葉[ことば]に 私[わたし]は 困惑[こんわく]した。	
\\	アメリカはこの2国の侵略を後押ししていた。	
\\	アメリカはこの2 国[こく]の 侵略[しんりゃく]を 後押[あとお]ししていた。	後押し=あとおし= 
\\	兵士たちはその村を侵略しました。	
\\	兵士[へいし]たちはその 村[むら]を 侵略[しんりゃく]しました。	
\\	ビートルズは1970年に解散しました。	
\\	ビートルズは1970 年[ねん]に 解散[かいさん]しました。	
\\	彼のお客のほとんどは、高い教養があり、見る目がある。	
\\	彼[かれ]のお 客[きゃく]のほとんどは、 高[たか]い 教養[きょうよう]があり、 見[み]る 目[め]がある。	教養=きょうよう= 
\\	見る目=みるめ= 
\\	彼女たちの無教養にひどくショックを受けました。	
\\	彼女[かのじょ]たちの 無[む] 教養[きょうよう]にひどくショックを 受[う]けました。	
\\	彼女と話すといつも、自分に教養がないことを感じさせられます。	
\\	彼女[かのじょ]と 話[はな]すといつも、 自分[じぶん]に 教養[きょうよう]がないことを 感[かん]じさせられます。	
\\	9人家族を養うのは大変です。	
\\	人[にん] 家族[かぞく]を 養[やしな]うのは 大変[たいへん]です。	養う=やしなう= 
\\	読書は理論的思考力を養うための最善の方法です。	
\\	読書[どくしょ]は 理論[りろん] 的[てき] 思考[しこう] 力[りょく]を 養[やしな]うための 最善[さいぜん]の 方法[ほうほう]です。	思考力=しこうりょく= 
\\	養う=やしなう= 
\\	この養成プログラムを終了するには3年かかる予定だ。	
\\	この 養成[ようせい]プログラムを 終了[しゅうりょう]するには3 年[ねん]かかる 予定[よてい]だ。	養成=ようせい= 
\\	攻撃は自爆攻撃だと伝えられている。	
\\	攻撃[こうげき]は 自爆[じばく] 攻撃[こうげき]だと 伝[つた]えられている。	自爆攻撃=じ ばく こう げき= 
\\	からかわれると、かなり攻撃的になります。	
\\	からかわれると、かなり 攻撃[こうげき] 的[てき]になります。	からかう= 
\\	このような攻撃は決して正当化されるものではない。	
\\	このような 攻撃[こうげき]は 決[けっ]して 正当[せいとう] 化[か]されるものではない。	正当か=せいとうか= 
\\	ジュラバはモロッコの男女用の伝統的な衣装です。	
\\	ジュラバはモロッコの 男女[だんじょ] 用[よう]の 伝統[でんとう] 的[てき]な 衣装[いしょう]です。	
\\	ハロウィーンの衣装を持ってきました。	
\\	ハロウィーンの 衣装[いしょう]を 持[も]ってきました。	
\\	リオのカーニバルの衣装は、世界一きらびやかだ。	
\\	リオのカーニバルの 衣装[いしょう]は、 世界一[せかいいち]きらびやかだ。	世界一=せかいいち= 
\\	きらびやか(な)= 
\\	単にストレス発散をしたいだけなのかもしれない。	
\\	単[たん]にストレス 発散[はっさん]をしたいだけなのかもしれない。	発散=はっさん= 
\\	体を動かすことによって、怒りを発散させることができます。	
\\	体[からだ]を 動[うご]かすことによって、 怒[いか]りを 発散[はっさん]させることができます。	怒り=いかり= 
\\	発散=はっさん= 
\\	この言葉は少々からかった意味で使う。	
\\	この 言葉[ことば]は 少々[しょうしょう]からかった 意味[いみ]で 使[つか]う。	からかう= 
\\	マイクは人をからかってばかりいる。	
\\	マイクは 人[ひと]をからかってばかりいる。	からかう= 
\\	彼はあなたをからかっています。	
\\	彼[かれ]はあなたをからかっています。	からかう= 
\\	それは発見者の鈴木直さんにちなんで名付けられた。	
\\	それは 発見[はっけん] 者[しゃ]の 鈴木[すずき] 直[ただし]さんにちなんで 名付[なづ]けられた。	
\\	メアリーは彼女の祖母の名を取って名付けられた。	
\\	メアリーは 彼女[かのじょ]の 祖母[そぼ]の 名[な]を 取[と]って 名付[なづ]けられた。	
\\	チーム名は県の名物にちなんでいます。	
\\	チーム 名[な]は 県[けん]の 名物[めいぶつ]にちなんでいます。	
\\	私たちの町は、町の創立者にちなんで名付けられました。	
\\	私[わたし]たちの 町[まち]は、 町[まち]の 創立[そうりつ] 者[しゃ]にちなんで 名付[なづ]けられました。	創立者=そうりつしゃ= 
\\	ハーベイ・ミルク高校は、米国で初めて同性愛者であることを公言した政治家にちなんで名付けられました。	
\\	ハーベイ・ミルク 高校[こうこう]は、 米国[べいこく]で 初[はじ]めて 同性愛[どうせいあい] 者[しゃ]であることを 公言[こうげん]した 政治[せいじ] 家[か]にちなんで 名付[なづ]けられました。	公言=こうげん= 
\\	この映画は事実に忠実だと思いますか?	
\\	この 映画[えいが]は 事実[じじつ]に 忠実[ちゅうじつ]だと 思[おも]いますか?	忠実=ちゅうじつ= 
\\	その小説家には、忠実な信奉者がいます。	
\\	その 小説[しょうせつ] 家[か]には、 忠実[ちゅうじつ]な 信奉[しんぽう] 者[しゃ]がいます。	忠実=ちゅうじつ= 
\\	信奉者=しんぽうしゃ= 
\\	その訳は原文に忠実である。	
\\	その 訳[やく]は 原文[げんぶん]に 忠実[ちゅうじつ]である。	忠実=ちゅうじつ= 
\\	その君主は、国民に絶対的な忠誠を要求しました。	
\\	その 君主[くんしゅ]は、 国民[こくみん]に 絶対[ぜったい] 的[てき]な 忠誠[ちゅうせい]を 要求[ようきゅう]しました。	君主=くんしゅ= 
\\	要求=ようきゅう= 
\\	弊社の顧客は弊社のブランドに忠誠心があります。	
\\	弊社[へいしゃ]の 顧客[こきゃく]は 弊社[へいしゃ]のブランドに 忠誠[ちゅうせい] 心[しん]があります。	弊社=へいしゃ= 
\\	顧客=こきゃく= 
\\	忠告してあげましょうか?	
\\	忠告[ちゅうこく]してあげましょうか?	忠告=ちゅうこく= 
\\	忠告しましたよ。	
\\	忠告[ちゅうこく]しましたよ。	忠告=ちゅうこく= 
\\	いくつか忠告があります。	
\\	いくつか 忠告[ちゅうこく]があります。	忠告=ちゅうこく= 
\\	あなたに一言忠告があります。	
\\	あなたに 一言[ひとこと] 忠告[ちゅうこく]があります。	
\\	その数学の先生は、とても融通が利いて、英語も教えている。	
\\	その 数学[すうがく]の 先生[せんせい]は、とても 融通[ゆうずう]が 利[き]いて、 英語[えいご]も 教[おし]えている。	融通が利く=ゆうずう が きく= 
\\	週末のスケジュールならかなり融通が利きます。	
\\	週末[しゅうまつ]のスケジュールならかなり 融通[ゆうずう]が 利[き]きます。	融通が利く=ゆうずう が きく= 
\\	金融危機の結果、その会社は倒産しました。	
\\	金融[きんゆう] 危機[きき]の 結果[けっか]、その 会社[かいしゃ]は 倒産[とうさん]しました。	金融危機=きんゆうきき= 
\\	倒産=とうさん= 
\\	金融取引は非常に複雑な場合もあります。	
\\	金融[きんゆう] 取引[とりひき]は 非常[ひじょう]に 複雑[ふくざつ]な 場合[ばあい]もあります。	金融=きんゆう= 
\\	取引=とりひき= 
\\	彼のスピーチは心に響く、意義深いものだった。	
\\	彼[かれ]のスピーチは 心[こころ]に 響[ひび]く、 意義[いぎ] 深[ぶか]いものだった。	響く=ひびく= 
\\	意義深い=いぎ ぶかい= 
\\	救急車のサイレンが鳴り響く。	
\\	救急[きゅうきゅう] 車[しゃ]のサイレンが 鳴り響[なりひび]く。	鳴り響く=なりひびく= 
\\	その主張を裏付ける証拠は何一つありません。	
\\	その 主張[しゅちょう]を 裏付[うらづ]ける 証拠[しょうこ]は 何一[なにひと]つありません。	何一つ=なに ひとつ= 
\\	その説を裏付ける証拠がいくつかある。	
\\	その 説[せつ]を 裏付[うらづ]ける 証拠[しょうこ]がいくつかある。	説=せつ= 
\\	この電車は代田橋に止まりますか?	
\\	この 電車[でんしゃ]は 代田橋[だいたばし]に 止[と]まりますか?	
\\	この地下鉄はウォールストリートで停まりますか?	
\\	この 地下鉄[ちかてつ]はウォールストリートで 停[とま]まりますか?	
\\	どんなうそも裏切りだと思う?	
\\	どんなうそも 裏切[うらぎ]りだと 思[おも]う?	裏切る=うらぎる= 
\\	仲間を裏切るな。	
\\	仲間[なかま]を 裏切[うらぎ]るな。	裏切る=うらぎる= 
\\	彼はガールフレンドを裏切って他の女と浮気をした。	
\\	彼[かれ]はガールフレンドを 裏切[うらぎ]って 他[た]の 女[おんな]と 浮気[うわき]をした。	
\\	どうして日本の5円と50円硬貨には穴があいているのですか?	
\\	どうして日本の5 円[えん]と50 円[えん] 硬貨[こうか]には 穴[あな]があいているのですか?	
\\	地面に硬貨が落ちています。	
\\	地面[じめん]に 硬貨[こうか]が 落[お]ちています。	
\\	そのフェリーは貨物と乗客の両方を運ぶ。	
\\	そのフェリーは 貨物[かもつ]と 乗客[じょうきゃく]の 両方[りょうほう]を 運[はこ]ぶ。	
\\	たくさんの貨物が、浜に流れ着いた。	
\\	たくさんの 貨物[かもつ]が、 浜[はま]に 流れ着[ながれつ]いた。	浜=はま= 
\\	タイの国内通貨はバーツです。	
\\	タイの 国内[こくない] 通貨[つうか]はバーツです。	
\\	ドルは恐らく世界で最も強い通貨です。	
\\	ドルは 恐[おそ]らく 世界[せかい]で 最[もっと]も 強[つよ]い 通貨[つうか]です。	恐らく=おそらく= 
\\	その赤ん坊はちょいちょい母親を困らせる。	
\\	その 赤ん坊[あかんぼう]はちょいちょい 母親[ははおや]を 困[こま]らせる。	ちょいちょい= 
\\	あの猫は、いつもこの辺りをうろついている。	
\\	あの 猫[ねこ]は、いつもこの 辺[あた]りをうろついている。	うろつく= 
\\	これらの狐は、鳥の卵を探してその辺りをうろつき回る。	
\\	これらの 狐[きつね]は、 鳥[とり]の 卵[たまご]を 探[さが]してその 辺[あた]りをうろつき 回[まわ]る。	うろつき回る= 
\\	あなたは試験結果が返ってきてから、ずっと機嫌がいい。	
\\	あなたは 試験[しけん] 結果[けっか]が 返[かえ]ってきてから、ずっと 機嫌[きげん]がいい。	機嫌がいい=きげん が いい= 
\\	ややこしそうだわ。	
\\	ややこしそうだわ。	ややこしい= 
\\	その名前は紛らわしい。	
\\	その 名前[なまえ]は 紛[まぎ]らわしい。	紛らわしい=まぎらわしい= 
\\	私のあなたへの気持ちは、要するに、混乱している。	
\\	私[わたし]のあなたへの 気持[きも]ちは、 要[よう]するに、 混乱[こんらん]している。	混乱=こんらん= 
\\	彼には、人をイライラさせる変な癖がいくつかある。	
\\	彼[かれ]には、 人[ひと]をイライラさせる 変[へん]な 癖[くせ]がいくつかある。	癖=くせ= 
\\	それにはマジでヒビッタよ。	
\\	それにはマジでヒビッタよ。	ひびる= 
\\	もう機嫌は直りました。	
\\	もう 機嫌[きげん]は 直[なお]りました。	機嫌=きげん= 
\\	上司の機嫌を取った。	
\\	上司[じょうし]の 機嫌[きげん]を 取[と]った。	機嫌を取る=きげん を とる= 
\\	今日は機嫌が悪いのね。	
\\	今日[きょう]は 機嫌[きげん]が 悪[わる]いのね。	機嫌=きげん= 
\\	彼は朝、機嫌が悪い。	
\\	彼[かれ]は 朝[あさ]、 機嫌[きげん]が 悪[わる]い。	機嫌=きげん= 
\\	このパソコンは市販されるのではなく、途上国に販売されます。	
\\	このパソコンは 市販[しはん]されるのではなく、 途上[とじょう] 国[こく]に 販売[はんばい]されます。	市販=しはん= 
\\	私は素晴らしい市販薬を使っている。	
\\	私[わたし]は 素晴[すば]らしい 市販[しはん] 薬[やく]を 使[つか]っている。	市販薬=しはんやく= 
\\	この会議の目的は販売計画を立てることだ。	
\\	この 会議[かいぎ]の 目的[もくてき]は 販売[はんばい] 計画[けいかく]を 立[た]てることだ。	計画を立てる=けいかく を たてる= 
\\	この販売促進は七月に修了します。	
\\	この 販売[はんばい] 促進[そくしん]は 七月[しちがつ]に 修了[しゅうりょう]します。	
\\	そのアンケートは女性視聴者を対象にしたものだった。	
\\	そのアンケートは 女性[じょせい] 視聴[しちょう] 者[しゃ]を 対象[たいしょう]にしたものだった。	
\\	統計は社会の動向を分析するのに有益な手段である。	
\\	統計[とうけい]は 社会[しゃかい]の 動向[どうこう]を 分析[ぶんせき]するのに 有益[ゆうえき]な 手段[しゅだん]である。	動向=どうこう= 
\\	利益の上昇が見込まれている。	
\\	利益[りえき]の 上昇[じょうしょう]が 見込[みこ]まれている。	上昇=じょうしょう= 
\\	見込む=みこむ= 
\\	利益を上げることがビジネスの主要な目的です。	
\\	利益[りえき]を 上[あ]げることがビジネスの 主要[しゅよう]な 目的[もくてき]です。	主要=しゅよう= 
\\	その会社は非常に利益が多い。	
\\	その 会社[かいしゃ]は 非常[ひじょう]に 利益[りえき]が 多[おお]い。	
\\	仮に宝くじに当たったとしたら、そのお金で何を買いますか?	
\\	仮[かり]に 宝[たから]くじに 当[あ]たったとしたら、そのお 金[かね]で 何[なに]を 買[か]いますか?	仮に=かりに= 
\\	仮に私が死んだとしたら、誰が私の子供の面倒を見てくれるだろう。	
\\	仮[かり]に 私[わたし]が 死[し]んだとしたら、 誰[だれ]が 私[わたし]の 子供[こども]の 面倒[めんどう]を 見[み]てくれるだろう。	仮に=かりに= 
\\	面倒を見る=めんどう を みる= 
\\	例えばの話で質問させてください。	
\\	例[たと]えばの 話[はなし]で 質問[しつもん]させてください。	
\\	仮定上の質問をさせてください。	
\\	仮定[かてい] 上[じょう]の 質問[しつもん]をさせてください。	
\\	単なる仮定上の話です。	
\\	単[たん]なる 仮定[かてい] 上[じょう]の 話[はなし]です。	
\\	あなたを訴えてやる!	
\\	あなたを 訴[うった]えてやる!	
\\	それはこれから起こることの前兆だよ。	
\\	それはこれから 起[お]こることの 前兆[ぜんちょう]だよ。	
\\	それは良い兆しだね。	
\\	それは 良[よ]い 兆[きざ]しだね。	
\\	日本経済は回復の兆しを見せています。	
\\	日本[にっぽん] 経済[けいざい]は 回復[かいふく]の 兆[きざ]しを 見[み]せています。	
\\	人々は脱出しようと必死になっていた。	
\\	人々[ひとびと]は 脱出[だっしゅつ]しようと 必死[ひっし]になっていた。	
\\	彼には今まで貧困から脱出する機会がありませんでした。	
\\	彼[かれ]には 今[いま]まで 貧困[ひんこん]から 脱出[だっしゅつ]する 機会[きかい]がありませんでした。	
\\	女性はいつも男性に従属しなくてはならないと考えますか?	
\\	女性[じょせい]はいつも 男性[だんせい]に 従属[じゅうぞく]しなくてはならないと 考[かんが]えますか?	
\\	食事を抜くのは減量の悪いやり方だ。	
\\	食事[しょくじ]を 抜[ぬ]くのは 減量[げんりょう]の 悪[わる]いやり 方[かた]だ。	減量=げんりょう= 
\\	うちでは姉と交代で食事を作っています。	
\\	うちでは 姉[あね]と 交代[こうたい]で 食事[しょくじ]を 作[つく]っています。	
\\	お食事に招いていただいて、どうもありがとうございます。	
\\	お 食事[しょくじ]に 招[まね]いていただいて、どうもありがとうございます。	
\\	すみません、彼は今食事に出ています。	
\\	すみません、 彼[かれ]は 今[こん] 食事[しょくじ]に 出[で]ています。	
\\	仕事に行く支度をしなければならない。	
\\	仕事[しごと]に 行[い]く 支度[したく]をしなければならない。	支度=したく= 
\\	いつもの通り、妻は身支度に時間がかかっている。	
\\	いつもの 通[とお]り、 妻[つま]は 身支度[みじたく]に 時間[じかん]がかかっている。	身支度=みじたく= 
\\	風邪をひいていて食欲がない。	
\\	風邪[かぜ]をひいていて 食欲[しょくよく]がない。	
\\	熱が下がって、食欲が出てきた。	
\\	熱[ねつ]が 下[さ]がって、 食欲[しょくよく]が 出[で]てきた。	
\\	この映画は食欲を減退させるから、食事前には見ない方がいい。	
\\	この 映画[えいが]は 食欲[しょくよく]を 減退[げんたい]させるから、 食事[しょくじ] 前[まえ]には 見[み]ない 方[ほう]がいい。	
\\	普段使っている食器を、食器棚にしまうことはほとんどない。	
\\	普段[ふだん] 使[つか]っている 食器[しょっき]を、 食器[しょっき] 棚[たな]にしまうことはほとんどない。	
\\	いつも会社の近くのマクドナルドで、朝食をとっています。	
\\	いつも 会社[かいしゃ]の 近[ちか]くのマクドナルドで、 朝食[ちょうしょく]をとっています。	
\\	明日は午後から健康診断なので、朝食を食べないようにと言われた。	
\\	明日[あした]は 午後[ごご]から 健康[けんこう] 診断[しんだん]なので、 朝食[ちょうしょく]を 食[た]べないようにと 言[い]われた。	
\\	すごく疲れたから、料理したくないわ。 
\\	お寿司の出前でもとろうか?	
\\	すごく 疲[つか]れたから、 料理[りょうり]したくないわ。 
\\	お 寿司[すし]の 出前[でまえ]でもとろうか?	
\\	日本食は今や世界中でとてもポピュラーなので、はしを使いこなす外国人はたくさんいる。	
\\	日本[にっぽん] 食[しょく]は 今[いま]や 世界中[せかいじゅう]でとてもポピュラーなので、はしを 使[つか]いこなす 外国[がいこく] 人[じん]はたくさんいる。	使いこなす= 
\\	この肉は、はしで切れるくらい柔らかい。	
\\	この 肉[にく]は、はしで 切[き]れるくらい 柔[やわ]らかい。	
\\	ちょうどコーヒーを入れようと、お湯を沸かしていたところです。	
\\	ちょうどコーヒーを 入[い]れようと、お 湯[ゆ]を 沸[わ]かしていたところです。	
\\	スパゲッティは、たっぷりのお湯でゆでてください。	
\\	スパゲッティは、たっぷりのお 湯[ゆ]でゆでてください。	
\\	私の趣味はインターネットをすることです。	
\\	私[わたし]の 趣味[しゅみ]はインターネットをすることです。	
\\	インターネットに接続さえできれば、家から楽に仕事をすることができる。	
\\	インターネットに 接続[せつぞく]さえできれば、 家[いえ]から 楽[らく]に 仕事[しごと]をすることができる。	
\\	を買うより、ネットで配信されている音楽をダウンロードする方が、圧倒的に多い。	
\\	を 買[か]うより、ネットで 配信[はいしん]されている 音楽[おんがく]をダウンロードする 方[ほう]が、 圧倒的[あっとうてき]に 多[おお]い。	圧倒的=あっとうてき= 
\\	最近は、本や
\\	はたいていネットで買っている。	
\\	最近[さいきん]は、 本[ほん]や 
\\	はたいていネットで 買[か]っている。	
\\	大体のことはネットで調べられる。	
\\	大体[だいたい]のことはネットで 調[しら]べられる。	
\\	キッチンから、コーヒーを入れるいいにおいがしてきた。	
\\	キッチンから、コーヒーを 入[い]れるいいにおいがしてきた。	
\\	朝、コーヒーを飲まないと一日が始まらない。	
\\	朝[あさ]、コーヒーを 飲[の]まないと一 日[にち]が 始[はじ]まらない。	
\\	メキシコ料理のいいレストランを見つけたら、教えてね。	
\\	メキシコ 料理[りょうり]のいいレストランを 見[み]つけたら、 教[おし]えてね。	
\\	子供が英語は嫌いだと言ったら、親は勉強を強いるべきでしょうか?	
\\	子供[こども]が 英語[えいご]は 嫌[きら]いだと 言[い]ったら、 親[おや]は 勉強[べんきょう]を 強[し]いるべきでしょうか?	強いる=しいる= 
\\	モザンビークは、次から次へと起こる自然災害にあえいでいます。	
\\	モザンビークは、 次[つぎ]から 次[つぎ]へと 起[お]こる 自然[しぜん] 災害[さいがい]にあえいでいます。	喘ぐ=あえぐ= 
\\	日本は今不況に喘いでいる。	
\\	日本は 今[いま] 不況[ふきょう]に 喘[あえ]いでいる。	喘ぐ=あえぐ= 
\\	息をしようとしてあえいだ!	
\\	息[いき]をしようとしてあえいだ!	息をする=いき を する= 
\\	喘ぐ=あえぐ= 
\\	狂ってしまったのかも。	
\\	狂[くる]ってしまったのかも。	狂う=くるう= 
\\	この時計は狂っています。	
\\	この 時計[とけい]は 狂[くる]っています。	狂う=くるう= 
\\	姉は私が狂ってしまったんだと思っています。	
\\	姉[あね]は 私[わたし]が 狂[くる]ってしまったんだと 思[おも]っています。	狂う=くるう= 
\\	彼は気が狂っている。	
\\	彼[かれ]は 気[き]が 狂[くる]っている。	気が狂う= 
\\	彼は怒り狂っている。	
\\	彼[かれ]は 怒り狂[いかりくる]っている。	怒り狂う=いかり くるう= 
\\	彼は怒り狂って跳び上がり、火を指した。	
\\	彼[かれ]は 怒り狂[いかりくる]って 跳[と]び 上[あ]がり、 火[ひ]を 指[さ]した。	怒り狂う=いかり くるう= 
\\	跳び上がる=とびあがる= 
\\	いつ、そしてなぜブログを立ち上げようと思ったのですか?	
\\	いつ、そしてなぜブログを 立[た]ち 上[あ]げようと 思[おも]ったのですか?	
\\	そのうち自分の会社を立ち上げたいと思っています。	
\\	そのうち 自分[じぶん]の 会社[かいしゃ]を 立[た]ち 上[あ]げたいと 思[おも]っています。	そのうち= 
\\	シャツをズボンの中にしまいなさい。だらしなく見えるわよ。	
\\	シャツをズボンの 中[なか]にしまいなさい。だらしなく 見[み]えるわよ。	だらしない= 
\\	上司が見ていないときは、従業員たちはだらしなくしていた。	
\\	上司[じょうし]が 見[み]ていないときは、 従業[じゅうぎょう] 員[いん]たちはだらしなくしていた。	だらしない= 
\\	従業員=じゅうぎょういん= 
\\	日本女性がだらしないのではありません。	
\\	日本[にっぽん] 女性[じょせい]がだらしないのではありません。	だらしない= 
\\	バーテンが彼らの飲み物を持ってきた。	
\\	バーテンが 彼[かれ]らの 飲み物[のみもの]を 持[も]ってきた。	
\\	そのウエーターはあまりにも無礼なので、私は支配人に文句を言うつもりだ。	
\\	そのウエーターはあまりにも 無礼[ぶれい]なので、 私[わたし]は 支配人[しはいにん]に 文句[もんく]を 言[い]うつもりだ。	
\\	ウエーターに注文しました。	
\\	ウエーターに 注文[ちゅうもん]しました。	
\\	チップをウエーターにあげました。	
\\	チップをウエーターにあげました。	
\\	それって何か変な感じしない?	
\\	それって 何[なに]か 変[へん]な 感[かん]じしない?	
\\	長く気まずい沈黙が続いた。	
\\	長[なが]く 気[き]まずい 沈黙[ちんもく]が 続[つづ]いた。	沈黙=ちんもく= 
\\	あなたを困らせるようなことはしたくありません。	
\\	あなたを 困[こま]らせるようなことはしたくありません。	
\\	私は、お見合いデートのぎこちなさが嫌いです。	
\\	私[わたし]は、お 見合[みあ]いデートのぎこちなさが 嫌[きら]いです。	ぎこちない= 
\\	席に着いたとたん、私はたまらない気持ちになって、思わず泣き出してしまった。	
\\	席[せき]に 着[つ]いたとたん、 私[わたし]はたまらない 気持[きも]ちになって、 思[おも]わず 泣[な]き 出[だ]してしまった。	
\\	旅行の計画をたてたとき、ガソリン費も計算に入れましたか。	
\\	旅行[りょこう]の 計画[けいかく]をたてたとき、ガソリン 費[ひ]も 計算[けいさん]に 入[い]れましたか。	
\\	日本の方々は、旅行を計画する際に距離を重要視しますね。	
\\	日本の 方々[かたがた]は、 旅行[りょこう]を 計画[けいかく]する 際[さい]に 距離[きょり]を 重要[じゅうよう] 視[し]しますね。	重要視=じゅうようし= 
\\	アメリカは、日本よりもほかのアジア諸国を重要視し始めた。	
\\	アメリカは、 日本[にっぽん]よりもほかのアジア 諸国[しょこく]を 重要[じゅうよう] 視[し]し 始[はじ]めた。	重要視=じゅうようし= 
\\	その高校ではスポーツが非常に重要視されている。	
\\	その 高校[こうこう]ではスポーツが 非常[ひじょう]に 重要[じゅうよう] 視[し]されている。	重要視=じゅうようし= 
\\	ひげを生やすつもりなの?	
\\	ひげを 生[は]やすつもりなの?	
\\	また無精髭を生やしていました。	
\\	また 無精髭[ぶしょうひげ]を 生[は]やしていました。	無精髭=ぶしょうひげ= 
\\	生やす= 
\\	とても充実した学校生活を送っていました。	
\\	とても 充実[じゅうじつ]した 学校[がっこう] 生活[せいかつ]を 送[おく]っていました。	充実=じゅうじつ= 
\\	もちろんホテルの設備も充実しています。	
\\	もちろんホテルの 設備[せつび]も 充実[じゅうじつ]しています。	充実=じゅうじつ= 
\\	今私は、とても充実した生活を送っている。	
\\	今[いま] 私[わたし]は、とても 充実[じゅうじつ]した 生活[せいかつ]を 送[おく]っている。	充実=じゅうじつ= 
\\	ガールフレンドと別れた後、人生が空っぽのように感じた。	
\\	ガールフレンドと 別[わか]れた 後[のち]、 人生[じんせい]が 空[から]っぽのように 感[かん]じた。	空っぽ=からっぽ= 
\\	ポケットが空っぽだ。	
\\	ポケットが 空[から]っぽだ。	空っぽ=からっぽ= 
\\	犬のエサの皿は空っぽだ。	
\\	犬[いぬ]のエサの 皿[さら]は 空[から]っぽだ。	空っぽ=からっぽ= 
\\	どのようにして依存症を克服するのだろう?	
\\	どのようにして 依存[いぞん] 症[しょう]を 克服[こくふく]するのだろう?	依存症=いぞんしょう= 
\\	克服=こくふく= 
\\	妻のことを忘れられず、彼はアルコールに依存するようになっていた。	
\\	妻[つま]のことを 忘[わす]れられず、 彼[かれ]はアルコールに 依存[いぞん]するようになっていた。	依存=いぞん= 
\\	日本は食糧供給を輸入品に大いに依存している。	
\\	日本[にっぽん]は 食糧[しょくりょう] 供給[きょうきゅう]を 輸入[ゆにゅう] 品[ひん]に 大[おお]いに 依存[いぞん]している。	依存=いぞん= 
\\	供給=きょうきゅう= 
\\	その話は一千年も遡る。	
\\	その 話[はなし]は 一千[いっせん] 年[ねん]も 遡[さかのぼ]る。	遡る=さかのぼる= 
\\	多くの英単語の語源はラテン語に遡ることができる。	
\\	多[おお]くの 英単語[えいたんご]の 語源[ごげん]は ラテン語[らてんご]に 遡[さかのぼ]ることができる。	遡る=さかのぼる= 
\\	この出来事は私にとって大変衝撃的でした。	
\\	この 出来事[できごと]は 私[わたし]にとって 大変[たいへん] 衝撃[しょうげき] 的[てき]でした。	衝撃的=しょうげきてき= 
\\	この衝撃的な話は全米に広がりました。	
\\	この 衝撃[しょうげき] 的[てき]な 話[はなし]は 全米[ぜんべい]に 広[ひろ]がりました。	衝撃的=しょうげきてき= 
\\	その地震は彼女の人生最大の忘れられないほど衝撃的な体験だった。	
\\	その 地震[じしん]は 彼女[かのじょ]の 人生[じんせい] 最大[さいだい]の 忘[わす]れられないほど 衝撃[しょうげき] 的[てき]な 体験[たいけん]だった。	衝撃的=しょうげきてき= 
\\	事実は小説よりも衝撃的である。	
\\	事実[じじつ]は 小説[しょうせつ]よりも 衝撃[しょうげき] 的[てき]である。	衝撃的=しょうげきてき= 
\\	彼の作品は絶大な人気を集め、アトリエは訪問者であふれた。	
\\	彼[かれ]の 作品[さくひん]は 絶大[ぜつだい]な 人気[にんき]を 集[あつ]め、アトリエは 訪問[ほうもん] 者[しゃ]であふれた。	絶大=ぜつだい= 
\\	アトリエ= 
\\	あふれる= 
\\	新宿駅の地下はまるで巨大な迷路のようです。	
\\	新宿[しんじゅく] 駅[えき]の 地下[ちか]はまるで 巨大[きょだい]な 迷路[めいろ]のようです。	迷路=めいろ= 
\\	東京の道路は迷路のようだ。	
\\	東京[とうきょう]の 道路[どうろ]は 迷路[めいろ]のようだ。	迷路=めいろ= 
\\	その本には、歴史に関する記述が多い。	
\\	その 本[ほん]には、 歴史[れきし]に 関[かん]する 記述[きじゅつ]が 多[おお]い。	記述=きじゅつ= 
\\	予算はどのくらいですか?	
\\	予算[よさん]はどのくらいですか?	予算=よさん= 
\\	ご予算はおいくらぐらいですか?	
\\	ご 予算[よさん]はおいくらぐらいですか?	予算=よさん= 
\\	それは予算的に手が届かないよ。	
\\	それは 予算[よさん] 的[てき]に 手[て]が 届[とど]かないよ。	予算=よさん= 
\\	典型的なトルコの贈り物をいくつか見てみましょう。	
\\	典型[てんけい] 的[てき]なトルコの 贈り物[おくりもの]をいくつか 見[み]てみましょう。	典型的=てんけいてき= 
\\	これは片思いの典型的な例だ。	
\\	これは 片思[かたおも]いの 典型[てんけい] 的[てき]な 例[れい]だ。	典型的=てんけいてき= 
\\	片思い=かたおもい= 
\\	アメリカ人が持つ夢の典型的なものの一つは新しい家である。	
\\	アメリカ 人[じん]が 持[も]つ 夢[ゆめ]の 典型[てんけい] 的[てき]なものの 一[ひと]つは 新[あたら]しい 家[いえ]である。	典型的=てんけいてき= 
\\	あなたの就労時間について、話がしたいです。	
\\	あなたの 就労[しゅうろう] 時間[じかん]について、 話[はなし]がしたいです。	
\\	女性がこの種の仕事に就くのはそんなに珍しいことではない。	
\\	女性[じょせい]が この種[このしゅ]の 仕事[しごと]に 就[つ]くのはそんなに 珍[めずら]しいことではない。	
\\	これを、逆さから言える?	
\\	これを、 逆[さか]さから 言[い]える?	逆さ=さかさ= 
\\	悲鳴が聞こえませんでしたか?	
\\	悲鳴[ひめい]が 聞[き]こえませんでしたか?	
\\	彼女はその怪物が画面に現れると悲鳴をあげた。	
\\	彼女[かのじょ]はその 怪物[かいぶつ]が 画面[がめん]に 現[あらわ]れると 悲鳴[ひめい]をあげた。	
\\	夜更かしは体に障りますよ。	
\\	夜更[よふ]かしは 体[からだ]に 障[さわ]りますよ。	夜更かし=よふかし= 
\\	明日は夜更かししていい?	
\\	明日[あした]は 夜[よ] 更[ふ]かししていい?	夜更かし=よふかし= 
\\	彼女は本を読みながら夜更かしをすることがある。	
\\	彼女[かのじょ]は 本[ほん]を 読[よ]みながら 夜更[よふ]かしをすることがある。	夜更かし=よふかし= 
\\	周囲にある風景も今までと変わりがない。	
\\	周囲[しゅうい]にある 風景[ふうけい]も 今[いま]までと 変[か]わりがない。	
\\	その島は周囲20
\\	である。	
\\	その 島[しま]は 周囲[しゅうい]20 
\\	である。	
\\	彼には周囲に配慮する気質がない。	
\\	彼[かれ]には 周囲[しゅうい]に 配慮[はいりょ]する 気質[きしつ]がない。	
\\	彼は周囲の乗客に全く注意を払っていなかった。	
\\	彼[かれ]は 周囲[しゅうい]の 乗客[じょうきゃく]に 全[まった]く 注意[ちゅうい]を 払[はら]っていなかった。	注意を払う= 
\\	あなたはささいなことに注意を払い、結果的に心配しすぎます。	
\\	あなたはささいなことに 注意[ちゅうい]を 払[はら]い、 結果[けっか] 的[てき]に 心配[しんぱい]しすぎます。	注意を払う= 
\\	この事件対してマスコミは明らかに注意を払っていない。	
\\	この 事件[じけん] 対[たい]してマスコミは 明[あき]らかに 注意[ちゅうい]を 払[はら]っていない。	注意を払う= 
\\	そのトラは飼い慣らされていたが、それでも私たちは十分な注意を払った。	
\\	そのトラは 飼い慣[かいな]らされていたが、それでも 私[わたし]たちは 十分[じゅうぶん]な 注意[ちゅうい]を 払[はら]った。	飼い慣らす=かいならす= 
\\	注意を払う= 
\\	日本に行っている友人を訪ねることにした。	
\\	日本に 行[い]っている 友人[ゆうじん]を 訪[たず]ねることにした。	訪ねる=たずねる= 
\\	今年の5月、私はアメリカに住む友人を訪ねた。	
\\	今年[ことし]の 5月[ごがつ]、 私[わたし]はアメリカに 住[す]む 友人[ゆうじん]を 訪[たず]ねた。	訪ねる=たずねる= 
\\	暴動が各地で発生しました。	
\\	暴動[ぼうどう]が 各地[かくち]で 発生[はっせい]しました。	
\\	彼は暴行を受けた。	
\\	彼[かれ]は 暴行[ぼうこう]を 受[う]けた。	
\\	彼女はその暴行事件をでっち上げた。	
\\	彼女[かのじょ]はその 暴行[ぼうこう] 事件[じけん]をでっち 上[あ]げた。	でっち上げる= 
\\	彼女は子供が死亡するまで暴行を止めなかった。	
\\	彼女[かのじょ]は 子供[こども]が 死亡[しぼう]するまで 暴行[ぼうこう]を 止[と]めなかった。	
\\	5月12日のジャカルタの暴動では約1200人が死亡しました。	
\\	5月[ごがつ]12 日[にち]のジャカルタの 暴動[ぼうどう]では 約[やく]1200 人[にん]が 死亡[しぼう]しました。	
\\	昨日、大学で暴動が起きた。	
\\	昨日[きのう]、 大学[だいがく]で 暴動[ぼうどう]が 起[お]きた。	
\\	抗議運動は死者が出るほどの暴動へと発展しました。	
\\	抗議[こうぎ] 運動[うんどう]は 死者[ししゃ]が 出[で]るほどの 暴動[ぼうどう]へと 発展[はってん]しました。	抗議運動=こうぎ うんどう= 
\\	ついに彼らの罪を暴く証拠を手にした。	
\\	ついに 彼[かれ]らの 罪[つみ]を 暴[あば]く 証拠[しょうこ]を 手[て]にした。	暴く=あばく= 
\\	あなたの事を暴いてやる。	
\\	あなたの 事[こと]を 暴[あば]いてやる。	暴く=あばく= 
\\	そのレストランには、厳格なドレスコードがある。	
\\	そのレストランには、 厳格[げんかく]なドレスコードがある。	
\\	とても厳格な家庭で育った。	
\\	とても 厳格[げんかく]な 家庭[かてい]で 育[そだ]った。	
\\	フランは子供の扱いに関しては、厳格だった。	
\\	フランは 子供[こども]の 扱[あつか]いに 関[かん]しては、 厳格[げんかく]だった。	
\\	両親がとても厳格なので、彼女は8時までに帰宅しなければならない。	
\\	両親[りょうしん]がとても 厳格[げんかく]なので、 彼女[かのじょ]は8 時[じ]までに 帰宅[きたく]しなければならない。	
\\	その事件で保安チェックがものすごく厳重になった。	
\\	その 事件[じけん]で 保安[ほあん]チェックがものすごく 厳重[げんじゅう]になった。	保安=ほあん= 
\\	日本では銃は厳重に取り締まられている。	
\\	日本では 銃[じゅう]は 厳重[げんじゅう]に 取り締[とりし]まられている。	銃=じゅう= 
\\	取り締まる= 
\\	警備は極めて厳重だった。	
\\	警備[けいび]は 極[きわ]めて 厳重[げんじゅう]だった。	警備=けいび= 
\\	雲行きが怪しいです。	
\\	雲行[くもゆ]きが 怪[あや]しいです。	雲行きが怪しい= 
\\	雲行きが怪しくなってきた。	
\\	雲行[くもゆ]きが 怪[あや]しくなってきた。	雲行きが怪しい= 
\\	この状況から逃れる道はなかった。	
\\	この 状況[じょうきょう]から 逃[のが]れる 道[みち]はなかった。	
\\	彼は立ち上がって逃げ出そうとしました。	
\\	彼[かれ]は 立ち上[たちあ]がって 逃げ出[にげだ]そうとしました。	
\\	見逃すことはないですよ。	
\\	見逃[みのが]すことはないですよ。	
\\	それを見逃すつもりですか?	
\\	それを 見逃[みのが]すつもりですか?	
\\	この習慣の起源については諸説があります。	
\\	この 習慣[しゅうかん]の 起源[きげん]については 諸説[しょせつ]があります。	諸説=しょせつ= 
\\	ものすごい嵐で、たくさんの家が倒壊しました。	
\\	ものすごい 嵐[あらし]で、たくさんの 家[いえ]が 倒壊[とうかい]しました。	嵐=あらし= 
\\	倒壊=とうかい= 
\\	理想的には、子供は学ぶことに意欲的であるべきです。	
\\	理想[りそう] 的[てき]には、 子供[こども]は 学[まな]ぶことに 意欲[いよく] 的[てき]であるべきです。	
\\	要請ではなく命令です。	
\\	要請[ようせい]ではなく 命令[めいれい]です。	
\\	雑誌をぱらぱらめくっていたら、知っている顔が目に飛び込んできた。	
\\	雑誌[ざっし]をぱらぱらめくっていたら、 知[し]っている 顔[かお]が 目[め]に 飛び込[とびこ]んできた。	ぱらぱら= 
\\	時間があるときは、本屋に寄って、雑誌をぱらぱらめくる。	
\\	時間[じかん]があるときは、 本屋[ほんや]に 寄[よ]って、 雑誌[ざっし]をぱらぱらめくる。	ぱらぱら= 
\\	雑誌はあまり買わずに、立ち読みすることが多い。	
\\	雑誌[ざっし]はあまり 買[か]わずに、 立ち読[たちよ]みすることが 多[おお]い。	
\\	購読は半年と一年のどちらになさいますか?	
\\	購読[こうどく]は 半年[はんとし]と一 年[ねん]のどちらになさいますか?	購読=こうどく= 
\\	あなたの購読期間は今月で終わります。	
\\	あなたの 購読[こうどく] 期間[きかん]は 今月[こんげつ]で 終[お]わります。	購読=こうどく= 
\\	このごろは、以前より公園を散歩する人が増えた。	
\\	このごろは、 以前[いぜん]より 公園[こうえん]を 散歩[さんぽ]する 人[ひと]が 増[ふ]えた。	
\\	朝晩二回、犬を散歩させています。	
\\	朝晩[あさばん] 二回[にかい]、 犬[いぬ]を 散歩[さんぽ]させています。	
\\	犬に餌をやって散歩させなくちゃならない。	
\\	犬[いぬ]に 餌[えさ]をやって 散歩[さんぽ]させなくちゃならない。	
\\	このごろは、時間がたつのが恐ろしく速い。	
\\	このごろは、 時間[じかん]がたつのが 恐[おそ]ろしく 速[はや]い。	
\\	商店街をぶらぶらして、時間をつぶした。	
\\	商店[しょうてん] 街[がい]をぶらぶらして、 時間[じかん]をつぶした。	ぶらぶら= 
\\	彼らは海岸をぶらぶらと歩いた。	
\\	彼[かれ]らは 海岸[かいがん]をぶらぶらと 歩[ある]いた。	ぶらぶら= 
\\	近頃は忙しくて、映画を観に行く時間がほとんどありません。	
\\	近頃[ちかごろ]は 忙[いそが]しくて、 映画[えいが]を 観[み]に 行[い]く 時間[じかん]がほとんどありません。	
\\	もはや彼がすることと言えば、ぶらぶらしてテレビを見ることぐらいだ。	
\\	もはや 彼[かれ]がすることと 言[い]えば、ぶらぶらしてテレビを 見[み]ることぐらいだ。	もはや= 
\\	ぶらぶら= 
\\	台所と居間を掃除するのに、すごく時間がかかった。	
\\	台所[だいどころ]と 居間[いま]を 掃除[そうじ]するのに、すごく 時間[じかん]がかかった。	
\\	わずか400メートルを歩くのに1時間費やします。	
\\	わずか400メートルを 歩[ある]くのに1 時間[じかん] 費[つい]やします。	
\\	ジムで2時間を費やし、彼はひどい疲れを感じた。	
\\	ジムで2 時間[じかん]を 費[つい]やし、 彼[かれ]はひどい 疲[つか]れを 感[かん]じた。	
\\	これをいつもニューヨーク時間に合わせておくね。	
\\	これをいつもニューヨーク 時間[じかん]に 合[あ]わせておくね。	
\\	私は彼に電話をかけて待ち合わせの時間を決めました。	
\\	私[わたし]は 彼[かれ]に 電話[でんわ]をかけて 待ち合[まちあ]わせの 時間[じかん]を 決[き]めました。	
\\	彼は会議の時間の変更を提案しました。	
\\	彼[かれ]は 会議[かいぎ]の 時間[じかん]の 変更[へんこう]を 提案[ていあん]しました。	
\\	この新システムは時間を大幅に節約すると期待されています。	
\\	この 新[しん]システムは 時間[じかん]を 大幅[おおはば]に 節約[せつやく]すると 期待[きたい]されています。	
\\	時間に追われる毎日から、しばらく逃げ出したい。	
\\	時間[じかん]に 追[お]われる 毎日[まいにち]から、しばらく 逃げ出[にげだ]したい。	
\\	その国について、できるだけ情報を集めないといけない。	
\\	その 国[くに]について、できるだけ 情報[じょうほう]を 集[あつ]めないといけない。	
\\	友人たちと定期的に会って、情報を交換しています。	
\\	友人[ゆうじん]たちと 定期[ていき] 的[てき]に 会[あ]って、 情報[じょうほう]を 交換[こうかん]しています。	
\\	彼がどこからそんな情報を得ているのかは、知りません。	
\\	彼[かれ]がどこからそんな 情報[じょうほう]を 得[え]ているのかは、 知[し]りません。	
\\	どうやってこの情報を手に入れたの?	
\\	どうやってこの 情報[じょうほう]を 手[て]に 入[い]れたの?	
\\	その情報は、あっという間に世界中に広まった。	
\\	その 情報[じょうほう]は、あっという 間[ま]に 世界中[せかいじゅう]に 広[ひろ]まった。	
\\	会者定離	
\\	会者定離[えしゃじょうり]	
\\	七転び八起き	
\\	七転び八起[ななころびやお]き	
\\	猿も木から落ちる	
\\	猿[さる]も 木[き]から 落[お]ちる	
\\	彼は筋肉質です。	
\\	彼[かれ]は 筋肉質[きんにくしつ]です。	
\\	彼の腕は細いが、筋肉質です。	
\\	彼[かれ]の 腕[うで]は 細[ほそ]いが、 筋肉質[きんにくしつ]です。	
\\	記録によると、あなたは借りたビデオを返していません。	
\\	記録[きろく]によると、あなたは 借[か]りたビデオを 返[かえ]していません。	
\\	この曲は現在数週間にわたってチャートの一位を記録しています。	
\\	この 曲[きょく]は 現在[げんざい] 数[すう] 週間[しゅうかん]にわたってチャートの一 位[い]を 記録[きろく]しています。	
\\	この記録が破られることはないだろうと考えられていました。	
\\	この 記録[きろく]が 破[やぶ]られることはないだろうと 考[かんが]えられていました。	
\\	私は知らない世界、中国に興味津々でした。	
\\	私[わたし]は 知[し]らない 世界[せかい]、 中国[ちゅうごく]に 興味津々[きょうみしんしん]でした。	
\\	私は興味津々で尋ねた。	
\\	私[わたし]は 興味津々[きょうみしんしん]で 尋[たず]ねた。	
\\	人々は、その展覧会に興味津々だった。	
\\	人々[ひとびと]は、その 展覧[てんらん] 会[かい]に 興味津々[きょうみしんしん]だった。	
\\	我々全員が、彼の勇敢さを称賛しました。	
\\	我々[われわれ] 全員[ぜんいん]が、 彼[かれ]の 勇敢[ゆうかん]さを 称賛[しょうさん]しました。	
\\	人は誰しも称賛を求めている。	
\\	人[ひと]は 誰[だれ]しも 称賛[しょうさん]を 求[もと]めている。	誰しも= 
\\	多数の人々が納豆は健康に良いと称賛しました。	
\\	多数[たすう]の 人々[ひとびと]が 納豆[なっとう]は 健康[けんこう]に 良[よ]いと 称賛[しょうさん]しました。	
\\	この提案でご賛同いただけますか。	
\\	この 提案[ていあん]でご 賛同[さんどう]いただけますか。	賛同=さんどう= 
\\	彼は2度世界一周の航海をした。	
\\	彼[かれ]は2 度[ど] 世界[せかい] 一周[いっしゅう]の 航海[こうかい]をした。	世界一周=せかいいっしゅう= 
\\	航海=こうかい= 
\\	彼女は一日に30〜50キロを航海する。	
\\	彼女[かのじょ]は一 日[にち]に30〜50キロを 航海[こうかい]する。	航海=こうかい= 
\\	その交渉は難航している。	
\\	その 交渉[こうしょう]は 難航[なんこう]している。	難航=なんこう= 
\\	1933年に日本は国際連盟を脱退しました。	
\\	年[ねん]に 日本[にっぽん]は 国際[こくさい] 連盟[れんめい]を 脱退[だったい]しました。	連盟=れんめい= 
\\	脱退=だったい= 
\\	クロアチアは2007年に
\\	加盟を目指しています。	
\\	クロアチアは2007 年[ねん]に 
\\	加盟[かめい]を 目指[めざ]しています。	加盟=かめい= 
\\	国際連合は50カ国の加盟国で始まりました。	
\\	国際[こくさい] 連合[れんごう]は50 カ国[かこく]の 加盟[かめい] 国[こく]で 始[はじ]まりました。	加盟=かめい= 
\\	日本は1953年に、正式に加盟国になりました。	
\\	日本[にっぽん]は1953 年[ねん]に、 正式[せいしき]に 加盟[かめい] 国[こく]になりました。	加盟=かめい= 
\\	同盟国の支援が期待できると思いますか?	
\\	同盟[どうめい] 国[こく]の 支援[しえん]が 期待[きたい]できると 思[おも]いますか?	同盟=どうめい= 
\\	この戦争において、私たちは同盟国です。	
\\	この 戦争[せんそう]において、 私[わたし]たちは 同盟[どうめい] 国[こく]です。	同盟=どうめい= 
\\	彼の国は忠実な同盟国であり続けると彼は強調しました。	
\\	彼[かれ]の 国[くに]は 忠実[ちゅうじつ]な 同盟[どうめい] 国[こく]であり 続[つづ]けると 彼[かれ]は 強調[きょうちょう]しました。	同盟=どうめい= 
\\	2年後の1975年、米国に亡命を求めた。	
\\	年[ねん] 後[ご]の1975 年[ねん]、 米国[べいこく]に 亡命[ぼうめい]を 求[もと]めた。	亡命=ぼうめい= 
\\	子供の頃、スイスに亡命しました。	
\\	子供[こども]の 頃[ころ]、スイスに 亡命[ぼうめい]しました。	亡命=ぼうめい= 
\\	彼は20年間の亡命生活を終え祖国へ帰ってきました。	
\\	彼[かれ]は20 年間[ねんかん]の 亡命[ぼうめい] 生活[せいかつ]を 終[お]え 祖国[そこく]へ 帰[かえ]ってきました。	亡命=ぼうめい= 
\\	祖国=そこく= 
\\	それからはバレエー一筋ですか?	
\\	それからはバレエー 一筋[ひとすじ]ですか?	一筋=ひとすじ= 
\\	世の中は何事も一筋縄ではいかない。	
\\	世の中[よのなか]は 何事[なにごと]も 一筋縄[ひとすじなわ]ではいかない。	一筋縄ではいかない=ひとすじなわ ではいかない= 
\\	あいつは一筋縄でいく男ではない。	
\\	あいつは 一筋縄[ひとすじなわ]でいく 男[おとこ]ではない。	一筋縄ではいかない=ひとすじなわ ではいかない= 
\\	筋がややこしくて分かりませんでした。	
\\	筋[すじ]がややこしくて 分[わ]かりませんでした。	筋=すじ= 
\\	ややこしい= 
\\	筋を違えたに違いない。	
\\	筋[すじ]を 違[ちが]えたに 違[ちが]いない。	筋を違える=すじ を ちがえる= 
\\	臨時雇用者の需要は落ち込んでしまった。	
\\	臨時[りんじ] 雇用[こよう] 者[しゃ]の 需要[じゅよう]は 落ち込[おちこ]んでしまった。	臨時=りんじ= 
\\	雇用者=こようしゃ= 
\\	その祭りのために地下鉄は臨時列車を出した。	
\\	その 祭[まつ]りのために 地下鉄[ちかてつ]は 臨時[りんじ] 列車[れっしゃ]を 出[だ]した。	臨時=りんじ= 
\\	番組の途中ですが、ここで臨時ニュースをお伝えします。	
\\	番組[ばんぐみ]の 途中[とちゅう]ですが、ここで 臨時[りんじ]ニュースをお 伝[つた]えします。	臨時ニュース= りんじ ニュース= 
\\	誠に申し訳なく存じます。	
\\	誠[まこと]に 申し訳[もうしわけ]なく 存[ぞん]じます。	
\\	あなたの夫は誠実ですか?	
\\	あなたの 夫[おっと]は 誠実[せいじつ]ですか?	誠実=せいじつ= 
\\	あなたの誠実さに非常に感謝しています。	
\\	あなたの 誠実[せいじつ]さに 非常[ひじょう]に 感謝[かんしゃ]しています。	誠実=せいじつ= 
\\	トンプソン氏は、誠実なことで評判です。	
\\	トンプソン 氏[し]は、 誠実[せいじつ]なことで 評判[ひょうばん]です。	誠実=せいじつ= 
\\	女性に対して一度も誠実ではなかった。	
\\	女性[じょせい]に 対[たい]して一 度[ど]も 誠実[せいじつ]ではなかった。	誠実=せいじつ= 
\\	おまえを撃つぞ。	
\\	おまえを 撃[う]つぞ。	
\\	その軍曹は部下たちに撃つように命じた。	
\\	その 軍曹[ぐんそう]は 部下[ぶか]たちに 撃[う]つように 命[めい]じた。	軍曹=ぐんそう= 
\\	彼は両親を敬うように育てられた。	
\\	彼[かれ]は 両親[りょうしん]を 敬[うやま]うように 育[そだ]てられた。	敬う=うやまう= 
\\	私たちは、祖先を敬うべきです。	
\\	私[わたし]たちは、 祖先[そせん]を 敬[うやま]うべきです。	敬う=うやまう= 
\\	これは環境に対して敬意を払っている日本人の一例です。	
\\	これは 環境[かんきょう]に 対[たい]して 敬意[けいい]を 払[はら]っている 日本人[にっぽんじん]の 一例[いちれい]です。	敬意を払う= 
\\	彼にしかるべく敬意を払ってください。	
\\	彼[かれ]にしかるべく 敬意[けいい]を 払[はら]ってください。	しかるべく= 
\\	敬意を払う= 
\\	彼は私に対してわずかの敬意も抱いていません。	
\\	彼[かれ]は 私[わたし]に 対[たい]してわずかの 敬意[けいい]も 抱[だ]いていません。	
\\	その車には外部の損傷はなかった。	
\\	その 車[くるま]には 外部[がいぶ]の 損傷[そんしょう]はなかった。	損傷=そんしょう= 
\\	それらの自動車は修理できないほど損傷していた。	
\\	それらの 自動車[じどうしゃ]は 修理[しゅうり]できないほど 損傷[そんしょう]していた。	損傷=そんしょう= 
\\	コンドームは快感を損なう。	
\\	コンドームは 快感[かいかん]を 損[そこ]なう。	
\\	手遅れにならないようにしなさい。	
\\	手遅[ておく]れにならないようにしなさい。	手遅れ=ておくれ= 
\\	もう手遅れです。	
\\	もう 手遅[ておく]れです。	手遅れ=ておくれ= 
\\	今さらそんなこと言っても手遅れだ。	
\\	今[いま]さらそんなこと 言[い]っても 手遅[ておく]れだ。	今さら= 
\\	手遅れ=ておくれ= 
\\	気付いたときはもう手遅れだった。	
\\	気付[きづ]いたときはもう 手遅[ておく]れだった。	手遅れ=ておくれ= 
\\	バスに乗り遅れそうだ。	
\\	バスに 乗り遅[のりおく]れそうだ。	
\\	彼はバスに乗り遅れたので、学校に遅刻しました。	
\\	彼[かれ]はバスに 乗り遅[のりおく]れたので、 学校[がっこう]に 遅刻[ちこく]しました。	
\\	私は決定を遅らせた。	
\\	私[わたし]は 決定[けってい]を 遅[おく]らせた。	遅らせる=おくらせる= 
\\	遅かれ早かれ、すべてのデジタル情報は無料になります。	
\\	遅[おそ]かれ 早[はや]かれ、すべてのデジタル 情報[じょうほう]は 無料[むりょう]になります。	遅かれ早かれ=おそかれ はやかれ= 
\\	遅かれ早かれこの仕事をやらなければならない。	
\\	遅[おそ]かれ 早[はや]かれこの 仕事[しごと]をやらなければならない。	遅かれ早かれ=おそかれ はやかれ= 
\\	遅かれ早かれ運は尽きる。	
\\	遅[おそ]かれ 早[はや]かれ 運[うん]は 尽[つ]きる。	遅かれ早かれ=おそかれ はやかれ= 
\\	尽きる=つきる= 
\\	この本は絶版になっています。	
\\	この 本[ほん]は 絶版[ぜっぱん]になっています。	
\\	何新聞を取っていますか。	
\\	何[なに] 新聞[しんぶん]を 取[と]っていますか。	
\\	今朝は、新聞に目を通す暇がなかった。	
\\	今朝[けさ]は、 新聞[しんぶん]に 目[め]を 通[とお]す 暇[ひま]がなかった。	
\\	やや強い地震があったので、すぐにテレビをつけた。	
\\	やや 強[つよ]い 地震[じしん]があったので、すぐにテレビをつけた。	
\\	あのタレントは、最近よくテレビに出ている。	
\\	あのタレントは、 最近[さいきん]よくテレビに 出[で]ている。	
\\	朝は、ラジオでニュースを聞きながら、支度をします。	
\\	朝[あさ]は、ラジオでニュースを 聞[き]きながら、 支度[したく]をします。	
\\	どのチャンネルでも、その事故のニュースをやっていた。	
\\	どのチャンネルでも、その 事故[じこ]のニュースをやっていた。	
\\	このハードディスクレコーダーは、自動で自分好みの番組を録画できる。	
\\	このハードディスクレコーダーは、 自動[じどう]で 自分[じぶん] 好[この]みの 番組[ばんぐみ]を 録画[ろくが]できる。	録画=ろくが= 
\\	仕事をやめてから、暇を持て余しています。	
\\	仕事[しごと]をやめてから、 暇[ひま]を 持て余[もてあま]しています。	
\\	彼はこの一ヶ月の間、一日も休まずにブログを更新している。	
\\	彼[かれ]はこの一 ヶ月[かげつ]の 間[ま]、一 日[にち]も 休[やす]まずにブログを 更新[こうしん]している。	
\\	昔読んだ本を今読み直すと、新しい発見があるものだ。	
\\	昔[むかし] 読[よ]んだ 本[ほん]を 今[いま] 読み直[よみなお]すと、 新[あたら]しい 発見[はっけん]があるものだ。	
\\	子供の頃、母は毎日のように本を読んで聞かせてくれた。	
\\	子供[こども]の 頃[ころ]、 母[はは]は 毎日[まいにち]のように 本[ほん]を 読[よ]んで 聞[き]かせてくれた。	
\\	アマゾンで本を注文したら、翌日届いた。	
\\	アマゾンで 本[ほん]を 注文[ちゅうもん]したら、 翌日[よくじつ] 届[とど]いた。	
\\	小学校のときの友達から、メールが来て驚いた。	
\\	小学校[しょうがっこう]のときの 友達[ともだち]から、メールが 来[き]て 驚[おどろ]いた。	
\\	メールを携帯に転送するように、パソコンを設定してある。	
\\	メールを 携帯[けいたい]に 転送[てんそう]するように、パソコンを 設定[せってい]してある。	
\\	あなたはお父さんにそっくりですね。	
\\	あなたはお 父[とう]さんにそっくりですね。	そっくり= 
\\	お母さんと声がそっくりね。	
\\	お 母[かあ]さんと 声[こえ]がそっくりね。	そっくり= 
\\	その家は夜に明かりがつくとお城そっくりに見える。	
\\	その 家[いえ]は 夜[よる]に 明[あ]かりがつくとお 城[しろ]そっくりに 見[み]える。	そっくり= 
\\	明かり=あかり= 
\\	催しには、そっくりさんコンテストやパレードなどがあります。	
\\	催[もよお]しには、そっくりさんコンテストやパレードなどがあります。	そっくり= 
\\	催し=もよおし= 
\\	彼の目は君にそっくりだ。	
\\	彼[かれ]の 目[め]は 君[きみ]にそっくりだ。	そっくり= 
\\	無駄口をたたくな。	
\\	無駄口[むだぐち]をたたくな。	
\\	お世辞を言っても無駄です。	
\\	お 世辞[せじ]を 言[い]っても 無駄[むだ]です。	
\\	彼がそう言っていたのは、ただのお世辞です。	
\\	彼[かれ]がそう 言[い]っていたのは、ただのお 世辞[せじ]です。	
\\	きっかけはとても単純なことだった。	
\\	きっかけはとても 単純[たんじゅん]なことだった。	
\\	その質問への答えは単純だ。	
\\	その 質問[しつもん]への 答[こた]えは 単純[たんじゅん]だ。	
\\	人生は必ずしも単純ではなく、答えは一つとは限らない。	
\\	人生[じんせい]は 必[かなら]ずしも 単純[たんじゅん]ではなく、 答[こた]えは 一[ひと]つとは 限[かぎ]らない。	
\\	純粋に楽しむためだったと彼女は言います。	
\\	純粋[じゅんすい]に 楽[たの]しむためだったと 彼女[かのじょ]は 言[い]います。	純粋=じゅんすい= 
\\	単なる純粋なオファーである。	
\\	単[たん]なる 純粋[じゅんすい]なオファーである。	純粋=じゅんすい= 
\\	彼は日本女性の純粋なところが何よりも好きだと言う。	
\\	彼は日本 女性[じょせい]の 純粋[じゅんすい]なところが 何[なん]よりも 好[す]きだと 言[い]う。	純粋=じゅんすい= 
\\	期待したほど均一ではありません。	
\\	期待[きたい]したほど 均一[きんいつ]ではありません。	
\\	政府は雇用の機会均等を保証すべきです。	
\\	政府[せいふ]は 雇用[こよう]の 機会[きかい] 均等[きんとう]を 保証[ほしょう]すべきです。	
\\	今でも、この沖縄旅行のことは時折、夢に見る。	
\\	今[いま]でも、この 沖縄[おきなわ] 旅行[りょこう]のことは 時折[ときおり]、 夢[ゆめ]に 見[み]る。	
\\	学校生活について、子どもよりも熱心な親を時折見掛ける。	
\\	学校[がっこう] 生活[せいかつ]について、 子[こ]どもよりも 熱心[ねっしん]な 親[おや]を 時折[ときおり] 見掛[みか]ける。	
\\	彼女は、テレビを見ながら、時折、寝入ってしまいます。	
\\	彼女[かのじょ]は、テレビを 見[み]ながら、 時折[ときおり]、 寝入[ねい]ってしまいます。	寝入る=ねいる= 
\\	この頃物価がめちゃくちゃ上昇しています。	
\\	この 頃[ころ] 物価[ぶっか]がめちゃくちゃ 上昇[じょうしょう]しています。	
\\	その株価は既にかなり上昇している。	
\\	その 株価[かぶか]は 既[すで]にかなり 上昇[じょうしょう]している。	株価=かぶか= 
\\	その販売促進キャンペーンは売上高の上昇につながりました。	
\\	その 販売[はんばい] 促進[そくしん]キャンペーンは 売上[うりあげ] 高[だか]の 上昇[じょうしょう]につながりました。	売上高=うりあげだか= 
\\	デルの株価はほぼ7%上昇しました。	
\\	デルの 株価[かぶか]はほぼ7 
\\	[ぱーせんと] 上昇[じょうしょう]しました。	株価=かぶか= 
\\	ほぼ= 
\\	それはとても描写的でした。	
\\	それはとても 描写[びょうしゃ] 的[てき]でした。	
\\	私は現実をありのままに描写したい。	
\\	私[わたし]は 現実[げんじつ]をありのままに 描写[びょうしゃ]したい。	ありのまま= 
\\	ありのままのあなたが好きです。	
\\	ありのままのあなたが 好[す]きです。	ありのまま= 
\\	ありのままの私を受け入れてほしい。	
\\	ありのままの 私[わたし]を 受け入[うけい]れてほしい。	ありのまま= 
\\	この小説は、すべてありのままの真実を伝えているようだ。	
\\	この 小説[しょうせつ]は、すべてありのままの 真実[しんじつ]を 伝[つた]えているようだ。	ありのまま= 
\\	何でも話せて、ありのままの自分でいられるような友人を選びなさい。	
\\	何[なに]でも 話[はな]せて、ありのままの 自分[じぶん]でいられるような 友人[ゆうじん]を 選[えら]びなさい。	ありのまま= 
\\	本当に私を愛しているなら、ありのままを話して。	
\\	本当[ほんとう]に 私[わたし]を 愛[あい]しているなら、ありのままを 話[はな]して。	ありのまま= 
\\	会社を作ることは絵を描くことに似ています。	
\\	会社[かいしゃ]を 作[つく]ることは 絵[え]を 描[えが]くことに 似[に]ています。	
\\	コメントは全部読んでいますが、返信する時間がありません。	
\\	コメントは 全部[ぜんぶ] 読[よ]んでいますが、 返信[へんしん]する 時間[じかん]がありません。	
\\	彼はぐったりと椅子に腰を下ろした。	
\\	彼[かれ]はぐったりと 椅子[いす]に 腰[こし]を 下[お]ろした。	腰を下ろす=こし を おろす= 
\\	ぐったり= 
\\	私がちょっと腰を下ろしたところで、不意に電話が鳴った。	
\\	私がちょっと 腰[こし]を 下[お]ろしたところで、 不意[ふい]に 電話[でんわ]が 鳴[な]った。	腰を下ろす=こし を おろす= 
\\	不意に=ふいに= 
\\	笑いすぎて、もう少しで椅子から転げ落ちるところだった。	
\\	笑[わら]いすぎて、もう 少[すこ]しで 椅子[いす]から 転げ落[ころげお]ちるところだった。	
\\	部屋に入ると、彼は椅子から立ち上がって挨拶してくれた。	
\\	部屋[へや]に 入[はい]ると、 彼[かれ]は 椅子[いす]から 立ち上[たちあ]がって 挨拶[あいさつ]してくれた。	
\\	そこはウェイターが椅子を引いて座らせてくれる高級レストランだった。	
\\	そこはウェイターが 椅子[いす]を 引[ひ]いて 座[すわ]らせてくれる 高級[こうきゅう]レストランだった。	
\\	彼は彼女に椅子を勧めた。	
\\	彼[かれ]は 彼女[かのじょ]に 椅子[いす]を 勧[すす]めた。	
\\	そのウォーターベッドには、温度調節機能が付いている。	
\\	そのウォーターベッドには、 温度[おんど] 調節[ちょうせつ] 機能[きのう]が 付[つ]いている。	
\\	ふとんは毎日たたんで、押し入れにしまっています。	
\\	ふとんは 毎日[まいにち]たたんで、 押し入[おしい]れにしまっています。	たたむ= 
\\	押し入れ=おしいれ= 
\\	部屋中に散乱していた洋服や雑誌を、全部押し入れに押し込んだ。	
\\	部屋中[へやじゅう]に 散乱[さんらん]していた 洋服[ようふく]や 雑誌[ざっし]を、 全部[ぜんぶ] 押し入[おしい]れに 押し込[おしこ]んだ。	散乱=さんらん= 
\\	たまに押し入れを換気しないと、中がかび臭くなる。	
\\	たまに 押し入[おしい]れを 換気[かんき]しないと、 中[なか]がかび 臭[くさ]くなる。	換気=かんき= 
\\	カビ臭い= 
\\	まだ引っ越したばかりで、カーテンさえかけていません。	
\\	まだ 引っ越[ひっこ]したばかりで、カーテンさえかけていません。	
\\	郊外にある住宅は、同じような構造になりがちだ。	
\\	郊外[こうがい]にある 住宅[じゅうたく]は、 同[おな]じような 構造[こうぞう]になりがちだ。	構造=こうぞう= 
\\	そのコンサート会場は東京郊外にあります。	
\\	そのコンサート 会場[かいじょう]は 東京[とうきょう] 郊外[こうがい]にあります。	会場=かいじょう= 
\\	リチャードさんはロンドン郊外の出身だ。	
\\	リチャードさんはロンドン 郊外[こうがい]の 出身[しゅっしん]だ。	
\\	子どもが生まれると、郊外に引っ越したがる夫婦もいます。	
\\	子[こ]どもが 生[う]まれると、 郊外[こうがい]に 引っ越[ひっこ]したがる 夫婦[ふうふ]もいます。	
\\	家賃は一般的に郊外へ行けば行くほど安くなる。	
\\	家賃[やちん]は 一般[いっぱん] 的[てき]に 郊外[こうがい]へ 行[い]けば 行[い]くほど 安[やす]くなる。	
\\	彼は郊外に住んでいて、毎日電車で通勤している。	
\\	彼[かれ]は 郊外[こうがい]に 住[す]んでいて、 毎日[まいにち] 電車[でんしゃ]で 通勤[つうきん]している。	通勤=つうきん= 
\\	新しいショッピングセンターが市の郊外で建設中です。	
\\	新[あたら]しいショッピングセンターが 市[し]の 郊外[こうがい]で 建設[けんせつ] 中[ちゅう]です。	建設中=けん せつ ちゅう= 
\\	これはどうやって保存すればいいのですか?	
\\	これはどうやって 保存[ほぞん]すればいいのですか?	
\\	その写真家は自分の最高作を作品集に保存していた。	
\\	その 写真[しゃしん] 家[か]は 自分[じぶん]の 最高[さいこう] 作[さく]を 作品[さくひん] 集[しゅう]に 保存[ほぞん]していた。	
\\	その情報はコンピューターに保存される。	
\\	その 情報[じょうほう]はコンピューターに 保存[ほぞん]される。	
\\	その製品は長期保存が可能です。	
\\	その 製品[せいひん]は 長期[ちょうき] 保存[ほぞん]が 可能[かのう]です。	
\\	いいかげんなことを言うな。	
\\	いいかげんなことを 言[い]うな。	いい加減=いい かげん= 
\\	いいかげんにしないと本当に怒るよ。	
\\	いいかげんにしないと 本当[ほんとう]に 怒[おこ]るよ。	いいかげんにする= 
\\	その無礼な言動が、彼女を怒らせました。	
\\	その 無礼[ぶれい]な 言動[げんどう]が、 彼女[かのじょ]を 怒[おこ]らせました。	言動=げんどう= 
\\	それは全く無礼だと思う。	
\\	それは 全[まった]く 無礼[ぶれい]だと 思[おも]う。	
\\	彼の無礼さにショックを受けた。	
\\	彼[かれ]の 無礼[ぶれい]さにショックを 受[う]けた。	
\\	障害のある人も参加することができる。	
\\	障害[しょうがい]のある 人[ひと]も 参加[さんか]することができる。	
\\	あなたに出会えてどれほど感激しているかを伝えたい。	
\\	あなたに 出会[であ]えてどれほど 感激[かんげき]しているかを 伝[つた]えたい。	感激=かんげき= 
\\	ストライキは一般的に過激な手段と見なされる。	
\\	ストライキは 一般[いっぱん] 的[てき]に 過激[かげき]な 手段[しゅだん]と 見[み]なされる。	過激=かげき= 
\\	見なす= 
\\	彼の出版物はあまりにも過激だったため、何度も告訴された。	
\\	彼[かれ]の 出版[しゅっぱん] 物[ぶつ]はあまりにも 過激[かげき]だったため、 何[なん] 度[ど]も 告訴[こくそ]された。	過激=かげき= 
\\	それ以来、アフリカの人口は急激に増加しています。	
\\	それ 以来[いらい]、アフリカの 人口[じんこう]は 急激[きゅうげき]に 増加[ぞうか]しています。	急激=きゅうげき= 
\\	インドは新興経済国の一つです。	
\\	インドは 新興[しんこう] 経済[けいざい] 国[こく]の 一[ひと]つです。	
\\	即興で何時間もピアノを弾くとき、決してあきませんでした。	
\\	即興[そっきょう]で 何[なん] 時間[じかん]もピアノを 弾[ひ]くとき、 決[けっ]してあきませんでした。	即興=そっきょう= 
\\	ジャズの本質は、即興演奏にある。	
\\	ジャズの 本質[ほんしつ]は、 即興[そっきょう] 演奏[えんそう]にある。	即興=そっきょう= 
\\	私たちの音楽の先生は、楽器を即興で演奏することを勧める。	
\\	私[わたし]たちの 音楽[おんがく]の 先生[せんせい]は、 楽器[がっき]を 即興[そっきょう]で 演奏[えんそう]することを 勧[すす]める。	即興=そっきょう= 
\\	その新製品は世界を席巻しました。	
\\	その 新[しん] 製品[せいひん]は 世界[せかい]を 席巻[せっけん]しました。	席巻=せっけん= 
\\	このプロジェクトは全米で議論を巻き起こしています。	
\\	このプロジェクトは 全米[ぜんべい]で 議論[ぎろん]を 巻き起[まきお]こしています。	
\\	ラップを巻いたまま、寿司を8等分に切ります。	
\\	ラップを 巻[ま]いたまま、 寿司[すし]を8 等分[とうぶん]に 切[き]ります。	等分=とうぶん= 
\\	包帯を巻いておきましょう。	
\\	包帯[ほうたい]を 巻[ま]いておきましょう。	包帯=ほうたい= 
\\	巻き込まれたくない。	
\\	巻き込[まきこ]まれたくない。	巻き込む=まきこむ= 
\\	あなたを巻き込みたくない。	
\\	あなたを 巻き込[まきこ]みたくない。	巻き込む=まきこむ= 
\\	あなたは厄介な状況に巻き込まれてしまいましたね。	
\\	あなたは 厄介[やっかい]な 状況[じょうきょう]に 巻き込[まきこ]まれてしまいましたね。	巻き込む=まきこむ= 
\\	厄介=やっかい= 
\\	トラブルに巻き込まないでね。	
\\	トラブルに 巻き込[まきこ]まないでね。	巻き込む=まきこむ= 
\\	事故に巻き込まれた。	
\\	事故[じこ]に 巻き込[まきこ]まれた。	巻き込む=まきこむ= 
\\	ジョンは溜息をついた。	
\\	ジョンは 溜息[ためいき]をついた。	溜息をつく=ためいき を つく= 
\\	彼は安堵のため息をつきました。	
\\	彼[かれ]は 安堵[あんど]のため 息[いき]をつきました。	安堵=あんど= 
\\	溜息をつく=ためいき を つく= 
\\	彼女はちょっと休んで、ため息をついた。	
\\	彼女[かのじょ]はちょっと 休[やす]んで、ため 息[いき]をついた。	溜息をつく=ためいき を つく= 
\\	彼女は安堵と悲しみが入り交じった気持ちになった。	
\\	彼女[かのじょ]は 安堵[あんど]と 悲[かな]しみが 入り交[いりま]じった 気持[きも]ちになった。	安堵=あんど= 
\\	入り交じる=いりまじる= 
\\	怠惰とは単に仕事を避けることではない。	
\\	怠惰[たいだ]とは 単[たん]に 仕事[しごと]を 避[さ]けることではない。	怠惰=たいだ= 
\\	彼女がその試験に落ちたのは怠惰の結果だった。	
\\	彼女[かのじょ]がその 試験[しけん]に 落[お]ちたのは 怠惰[たいだ]の 結果[けっか]だった。	怠惰=たいだ= 
\\	私は本当に野球に打ち込んできた。	
\\	私[わたし]は 本当[ほんとう]に 野球[やきゅう]に 打ち込[うちこ]んできた。	
\\	しまった!本に夢中になってて乗り過ごしちゃった。	
\\	しまった! 本[ほん]に 夢中[むちゅう]になってて 乗り過[のりす]ごしちゃった。	
\\	その新しいテレビ番組に、私は夢中になりました。	
\\	その 新[あたら]しいテレビ 番組[ばんぐみ]に、 私[わたし]は 夢中[むちゅう]になりました。	
\\	この本は優れた作品です。	
\\	この 本[ほん]は 優[すぐ]れた 作品[さくひん]です。	
\\	その大学には優れた研究施設がある。	
\\	その 大学[だいがく]には 優[すぐ]れた 研究[けんきゅう] 施設[しせつ]がある。	
\\	ホッとしました!	
\\	ホッとしました!	
\\	これを聞いて私はホッとしました。	
\\	これを 聞[き]いて 私[わたし]はホッとしました。	
\\	全部終わってホッとしている。	
\\	全部[ぜんぶ] 終[お]わってホッとしている。	
\\	真実を知って心からホッとした。	
\\	真実[しんじつ]を 知[し]って 心[こころ]からホッとした。	
\\	私がどれほどホッとしているかは言葉で言い表せません。	
\\	私[わたし]がどれほどホッとしているかは 言葉[ことば]で 言い表[いいあらわ]せません。	
\\	ときには自分の欲求に身を任せるのもいいことだ。	
\\	ときには 自分[じぶん]の 欲求[よっきゅう]に 身[み]を 任[まか]せるのもいいことだ。	身を任せる=みをまかせる= 
\\	彼は運命に身を任せた。	
\\	彼[かれ]は 運命[うんめい]に 身[み]を 任[まか]せた。	身を任せる=みをまかせる= 
\\	彼らはただ流れに身を任せてやっている。	
\\	彼[かれ]らはただ 流[なが]れに 身[み]を 任[まか]せてやっている。	身を任せる=みをまかせる= 
\\	怠惰に身を任せてはいけない。	
\\	怠惰[たいだ]に 身[み]を 任[まか]せてはいけない。	身を任せる=みをまかせる= 
\\	彼は部屋に行って眠ったふりをしました。	
\\	彼[かれ]は 部屋[へや]に 行[い]って 眠[ねむ]ったふりをしました。	フリをする= 
\\	私は日本語が分からないふりをします。	
\\	私[わたし]は 日本語[にほんご]が 分[わ]からないふりをします。	フリをする= 
\\	彼は病気のふりをした。	
\\	彼[かれ]は 病気[びょうき]のふりをした。	フリをする= 
\\	砂糖は歯を傷める可能性があります。	
\\	砂糖[さとう]は 歯[は]を 傷[いた]める 可能[かのう] 性[せい]があります。	傷める=いためる= 
\\	あなたのふくらはぎの片方はもう一方より大きい。	
\\	あなたのふくらはぎの 片方[かたほう]はもう 一方[いっぽう]より 大[おお]きい。	ふくらはぎ= 
\\	片方=かたほう= 
\\	コンタクトレンズの片方が目から飛び出した。	
\\	コンタクトレンズの 片方[かたほう]が 目[め]から 飛び出[とびだ]した。	片方=かたほう= 
\\	彼は片方の耳が不自由です。	
\\	彼[かれ]は 片方[かたほう]の 耳[みみ]が 不自由[ふじゆう]です。	片方=かたほう= 
\\	不自由=ふじゆう= 
\\	彼は身体が不自由です。	
\\	彼[かれ]は 身体[しんたい]が 不自由[ふじゆう]です。	不自由=ふじゆう= 
\\	私は生まれてずっと不自由なく暮らしてきた。	
\\	私[わたし]は 生[う]まれてずっと 不自由[ふじゆう]なく 暮[く]らしてきた。	不自由=ふじゆう= 
\\	彼の話のすべての断片がうまくまとまり始めた。	
\\	彼[かれ]の 話[はなし]のすべての 断片[だんぺん]がうまくまとまり 始[はじ]めた。	断片= 
\\	まとまる= 
\\	私たちの会話の断片しか思い出せない。	
\\	私[わたし]たちの 会話[かいわ]の 断片[だんぺん]しか 思い出[おもいだ]せない。	断片= 
\\	片思いほどつらいものはない。	
\\	片思[かたおも]いほどつらいものはない。	片思い= 
\\	出会った日からずっと彼に片思いしている。	
\\	出会[であ]った 日[ひ]からずっと 彼[かれ]に 片思[かたおも]いしている。	
\\	初恋は片思いでした。	
\\	初恋[はつこい]は 片思[かたおも]いでした。	片手=かたて= 
\\	お互い言葉が片言でも意気投合することがあります。	
\\	お 互[たが]い 言葉[ことば]が 片言[かたこと]でも 意気投合[いきとうごう]することがあります。	片言=かたこと= 
\\	意気投合=いきとうごう= 
\\	ボブとメグは昨日パーティーで出会って意気投合した。	
\\	ボブとメグは 昨日[きのう]パーティーで 出会[であ]って 意気投合[いきとうごう]した。	意気投合=いきとうごう= 
\\	最初から意気投合したわ。	
\\	最初[さいしょ]から 意気投合[いきとうごう]したわ。	意気投合=いきとうごう= 
\\	私たちは知り合ったとたんに意気投合しました。	
\\	私[わたし]たちは 知り合[しりあ]ったとたんに 意気投合[いきとうごう]しました。	意気投合=いきとうごう= 
\\	彼は片言のスペイン語を話す。	
\\	彼[かれ]は 片言[かたこと]のスペイン 語[ご]を 話[はな]す。	片言=かたこと= 
\\	彼女は片言の日本語が話せるようになったところだ。	
\\	彼女[かのじょ]は 片言[かたこと]の 日本語[にほんご]が 話[はな]せるようになったところだ。	片言=かたこと= 
\\	私は片言のスペイン語を使って、彼に気持ちを伝えました。	
\\	私[わたし]は 片言[かたこと]のスペイン 語[ご]を 使[つか]って、 彼[かれ]に 気持[きも]ちを 伝[つた]えました。	片言=かたこと= 
\\	創立50周年おめでとうございます。	
\\	創立[そうりつ]50 周年[しゅうねん]おめでとうございます。	創立=そうりつ= 
\\	周年=しゅうねん= 
\\	先週、私たちは学校の創立記念日を祝った。	
\\	先週[せんしゅう]、 私[わたし]たちは 学校[がっこう]の 創立[そうりつ] 記念[きねん] 日[び]を 祝[いわ]った。	創立記念日=そうりつ きねんび= 
\\	ご自分の創作の刺激となる芸術家はいますか?	
\\	ご 自分[じぶん]の 創作[そうさく]の 刺激[しげき]となる 芸術[げいじゅつ] 家[か]はいますか?	創作= 
\\	刺激= 
\\	アナベラの創作方法は、まずスケッチを描くことから始まる。	
\\	アナベラの 創作[そうさく] 方法[ほうほう]は、まずスケッチを 描[えが]くことから 始[はじ]まる。	創作= 
\\	彼女は、芸術的な創作が人に与える影響の大きさに気付いた。	
\\	彼女[かのじょ]は、 芸術[げいじゅつ] 的[てき]な 創作[そうさく]が 人[ひと]に 与[あた]える 影響[えいきょう]の 大[おお]きさに 気付[きづ]いた。	創作= 
\\	その雑誌は2000年に創刊された。	
\\	その 雑誌[ざっし]は2000 年[ねん]に 創刊[そうかん]された。	創刊= 
\\	新聞); 
\\	英語版の「少年ジャンプ」が米国で創刊されました。	
\\	英語[えいご] 版[ばん]の
\\	少年ジャンプ[しょうねんじゃんぷ]」が 米国[べいこく]で 創刊[そうかん]されました。	創刊= 
\\	新聞); 
\\	新しい雑誌の創刊が相次いでいる。	
\\	新[あたら]しい 雑誌[ざっし]の 創刊[そうかん]が 相次[あいつ]いでいる。	創刊= 
\\	新聞); 
\\	相次いで= 
\\	その企業に対する訴訟が相次いで起こされている。	
\\	その 企業[きぎょう]に 対[たい]する 訴訟[そしょう]が 相次[あいつ]いで 起[お]こされている。	相次いで= 
\\	訴訟= 
\\	この会社では、創業者の精神が生き続けている。	
\\	この 会社[かいしゃ]では、 創業[そうぎょう] 者[しゃ]の 精神[せいしん]が 生[い]き 続[つづ]けている。	創業=そうぎょう= 
\\	独創力を教えるのは不可能です。	
\\	独創[どくそう] 力[りょく]を 教[おし]えるのは 不可能[ふかのう]です。	
\\	もっと独創的なものが読みたいです。	
\\	もっと 独創[どくそう] 的[てき]なものが 読[よ]みたいです。	
\\	今日にいたるまで、その作品は極めて独創的であり続けている。	
\\	今日[きょう]にいたるまで、その 作品[さくひん]は 極[きわ]めて 独創[どくそう] 的[てき]であり 続[つづ]けている。	
\\	その会社は銀行からの借り入れを減らす必要があります。	
\\	その 会社[かいしゃ]は 銀行[ぎんこう]からの 借り入[かりい]れを 減[へ]らす 必要[ひつよう]があります。	
\\	その借家人は建物の所有者とうまくいっていた。	
\\	その 借家[しゃくや] 人[じん]は 建物[たてもの]の 所有[しょゆう] 者[しゃ]とうまくいっていた。	
\\	いつも生き物を傷つけないように注意していた。	
\\	いつも 生き物[いきもの]を 傷[きず]つけないように 注意[ちゅうい]していた。	
\\	その会社の評判はスキャンダルによって傷ついた。	
\\	その 会社[かいしゃ]の 評判[ひょうばん]はスキャンダルによって 傷[きず]ついた。	
\\	彼は会社を中傷してやると脅かした。	
\\	彼[かれ]は 会社[かいしゃ]を 中傷[ちゅうしょう]してやると 脅[おど]かした。	中傷= 
\\	脅かす= 
\\	彼女は私についての中傷的なうそを隣人に言いふらしました。	
\\	彼女[かのじょ]は 私[わたし]についての 中傷[ちゅうしょう] 的[てき]なうそを 隣人[りんじん]に 言[い]いふらしました。	中傷= 
\\	言いふらす= 
\\	感傷的なことを言っている場合ではありません。	
\\	感傷[かんしょう] 的[てき]なことを 言[い]っている 場合[ばあい]ではありません。	感傷= 
\\	色とりどりの花火が東京の夏の夜空を照らす。	
\\	色[いろ]とりどりの 花火[はなび]が 東京[とうきょう]の 夏[なつ]の 夜空[よぞら]を 照[て]らす。	色とりどり= 
\\	その作家の明るい性格は彼女の暗い小説と対照的です。	
\\	その 作家[さっか]の 明[あか]るい 性格[せいかく]は 彼女[かのじょ]の 暗[くら]い 小説[しょうせつ]と 対照[たいしょう] 的[てき]です。	対照的= 
\\	背景との対照が面白い色を使いなさい。	
\\	背景[はいけい]との 対照[たいしょう]が 面白[おもしろ]い 色[いろ]を 使[つか]いなさい。	
\\	彼女の名前をリストと照合した。	
\\	彼女[かのじょ]の 名前[なまえ]をリストと 照合[しょうごう]した。	
\\	ずっとカンカン照りの天気が続いています。	
\\	ずっと カンカン照[かんかんで]りの 天気[てんき]が 続[つづ]いています。	カンカン= 
\\	照り= 
\\	午後になると太陽が暑くみんなの顔に照りつけました。	
\\	午後[ごご]になると 太陽[たいよう]が 暑[あつ]くみんなの 顔[かお]に 照[て]りつけました。	照りつける= 
\\	どうすれば彼の愛を取り戻せるでしょうか?	
\\	どうすれば 彼[かれ]の 愛[あい]を 取り戻[とりもど]せるでしょうか?	取り戻す= 
\\	彼が意識を取り戻すのに8時間かかった。	
\\	彼[かれ]が 意識[いしき]を 取り戻[とりもど]すのに8 時間[じかん]かかった。	取り戻す= 
\\	彼は失った時間を取り戻そうとしている。	
\\	彼[かれ]は 失[うしな]った 時間[じかん]を 取り戻[とりもど]そうとしている。	取り戻す= 
\\	彼女のおかげで、人間への信頼を取り戻せた。	
\\	彼女[かのじょ]のおかげで、 人間[にんげん]への 信頼[しんらい]を 取り戻[とりもど]せた。	取り戻す= 
\\	状況は、もう後戻りできないものです。	
\\	状況[じょうきょう]は、もう 後戻[あともど]りできないものです。	
\\	彼はギターだけでなくピアノも弾ける。	
\\	彼[かれ]はギターだけでなくピアノも 弾[ひ]ける。	
\\	天皇の住居である皇居は、東京都の千代田区にあります。	
\\	天皇[てんのう]の 住居[じゅうきょ]である 皇居[こうきょ]は、 東京[とうきょう] 都[と]の 千代田[ちよだ] 区[く]にあります。	住居= 
\\	皇居= 
\\	ウエストを測ってもらえますか?	
\\	ウエストを 測[はか]ってもらえますか?	測る・計る・量る=はかる= 
\\	体重計で体重を測った。	
\\	体重[たいじゅう] 計[けい]で 体重[たいじゅう]を 測[はか]った。	測る・計る・量る=はかる= 
\\	体重計=たいじゅうけい= 
\\	熱を測った方が良い。	
\\	熱[ねつ]を 測[はか]った 方[ほう]が 良[よ]い。	熱を測る=ねつ を はかる= 
\\	体温を測りましょう。	
\\	体温[たいおん]を 測[はか]りましょう。	
\\	笑いは測定できる。	
\\	笑[わら]いは 測定[そくてい]できる。	
\\	速度を測定する方法はいくつかある。	
\\	速度[そくど]を 測定[そくてい]する 方法[ほうほう]はいくつかある。	
\\	それは、単なる希望的観測でした。	
\\	それは、 単[たん]なる 希望[きぼう] 的[てき] 観測[かんそく]でした。	希望的観測=きぼうてき かんそく= 
\\	まさに政界のようです。	
\\	まさに 政界[せいかい]のようです。	まさに= 
\\	ここはまさに来る価値のある場所です。	
\\	ここはまさに 来[く]る 価値[かち]のある 場所[ばしょ]です。	まさに= 
\\	そのオペラはまさに傑作である。	
\\	そのオペラはまさに 傑作[けっさく]である。	まさに= 
\\	傑作=けっさく= 
\\	彼の入念な準備が功を奏した。	
\\	彼[かれ]の 入念[にゅうねん]な 準備[じゅんび]が 功[こう]を 奏[そう]した。	功を奏する=こう を そうする= 
\\	入念(な)= 
\\	その仕事はいつ完了するのか予測がつきますか?	
\\	その 仕事[しごと]はいつ 完了[かんりょう]するのか 予測[よそく]がつきますか?	予測=よそく= 
\\	その数字は2倍になると予測されている。	
\\	その 数字[すうじ]は2 倍[ばい]になると 予測[よそく]されている。	予測=よそく= 
\\	その映画の製作費はいくらだと予測しますか?	
\\	その 映画[えいが]の 製作[せいさく] 費[ひ]はいくらだと 予測[よそく]しますか?	予測=よそく= 
\\	中国の人口は毎年、1000万人ずつ増加すると予測されています。	
\\	中国[ちゅうごく]の 人口[じんこう]は 毎年[まいとし]、1000 万[まん] 人[にん]ずつ 増加[ぞうか]すると 予測[よそく]されています。	予測=よそく= 
\\	僕が一番心配なのは核戦争です。	
\\	僕[ぼく]が 一番[いちばん] 心配[しんぱい]なのは 核[かく] 戦争[せんそう]です。	
\\	核兵器は人類がかつて発見した中で最悪のものだ。	
\\	核兵器[かくへいき]は 人類[じんるい]がかつて 発見[はっけん]した 中[なか]で 最悪[さいあく]のものだ。	かつて= 
\\	かつて、地球は平らだと広く信じられていました。	
\\	かつて、 地球[ちきゅう]は 平[たい]らだと 広[ひろ]く 信[しん]じられていました。	かつて= 
\\	そのレストランはかつては倉庫だった。	
\\	そのレストランはかつては 倉庫[そうこ]だった。	かつて= 
\\	彼はかつて私の彼氏だった。	
\\	彼[かれ]はかつて 私[わたし]の 彼氏[かれし]だった。	かつて= 
\\	あなたは核心をついていらっしゃると思います。	
\\	あなたは 核心[かくしん]をついていらっしゃると 思[おも]います。	核心= 
\\	それがこの映画の核心です。	
\\	それがこの 映画[えいが]の 核心[かくしん]です。	核心= 
\\	私には、彼の理論の核心が理解できなかった。	
\\	私[わたし]には、 彼[かれ]の 理論[りろん]の 核心[かくしん]が 理解[りかい]できなかった。	核心= 
\\	私は問題の核心に迫りたい。	
\\	私[わたし]は 問題[もんだい]の 核心[かくしん]に 迫[せま]りたい。	核心= 
\\	迫る=せまる= 
\\	協議は北朝鮮の核兵器開発を止めさせることを目的としています。	
\\	協議[きょうぎ]は 北朝鮮[きたちょうせん]の 核兵器[かくへいき] 開発[かいはつ]を 止[と]めさせることを 目的[もくてき]としています。	協議=きょうぎ= 
\\	核兵器= 
\\	北朝鮮は10月9日に核実験を行ったと発表した。	
\\	北朝鮮[きたちょうせん]は 10月[じゅうがつ] 9日[ここのか]に 核[かく] 実験[じっけん]を 行[おこな]ったと 発表[はっぴょう]した。	核実験=かくじっけん= 
\\	日本は今「脱皮」の時期に来ている。	
\\	日本[にっぽん]は 今[いま]
\\	脱皮[だっぴ]」の 時期[じき]に 来[き]ている。	
\\	皮はパリパリしていて、肉は柔らかいです。	
\\	皮[かわ]はパリパリしていて、 肉[にく]は 柔[やわ]らかいです。	
\\	物忘れしやすいんだ。	
\\	物忘[ものわす]れしやすいんだ。	
\\	物忘れは誰でも経験することである。	
\\	物忘[ものわす]れは 誰[だれ]でも 経験[けいけん]することである。	
\\	年を取るにつれて物忘れがひどくなってきた。	
\\	年[とし]を 取[と]るにつれて 物忘[ものわす]れがひどくなってきた。	
\\	私は物忘れがひどい。	
\\	私[わたし]は 物忘[ものわす]れがひどい。	
\\	お名前を度忘れしてしまいました。	
\\	お 名前[なまえ]を 度忘[どわす]れしてしまいました。	
\\	頭が真っ白になった。	
\\	頭[あたま]が 真っ白[まっしろ]になった。	
\\	頭が真っ白になって名前を度忘れして、すみません。	
\\	頭[あたま]が 真っ白[まっしろ]になって 名前[なまえ]を 度忘[どわす]れして、すみません。	
\\	その国には民主主義の入り込む余地はないと彼は宣言しました。	
\\	その 国[くに]には 民主[みんしゅ] 主義[しゅぎ]の 入り込[はいりこ]む 余地[よち]はないと 彼[かれ]は 宣言[せんげん]しました。	宣言= 
\\	その国は核兵器保有国であることを宣言しました。	
\\	その 国[くに]は 核兵器[かくへいき] 保有[ほゆう] 国[こく]であることを 宣言[せんげん]しました。	宣言= 
\\	保有= 
\\	彼女は自分がフェミニストだと宣言しました。	
\\	彼女[かのじょ]は 自分[じぶん]がフェミニストだと 宣言[せんげん]しました。	宣言= 
\\	この国ではアルコール飲料の宣伝は禁止されている。	
\\	この 国[くに]ではアルコール 飲料[いんりょう]の 宣伝[せんでん]は 禁止[きんし]されている。	
\\	これ以上の宣伝はない。	
\\	これ 以上[いじょう]の 宣伝[せんでん]はない。	
\\	その会社はニュースを通じて無料で宣伝することができた。	
\\	その 会社[かいしゃ]はニュースを 通[つう]じて 無料[むりょう]で 宣伝[せんでん]することができた。	
\\	その製品はテレビで何度も宣伝されています。	
\\	その 製品[せいひん]はテレビで 何[なん] 度[ど]も 宣伝[せんでん]されています。	
\\	アルコール飲料はテレビで宣伝されるべきではないと考える人たちもいる。	
\\	アルコール 飲料[いんりょう]はテレビで 宣伝[せんでん]されるべきではないと 考[かんが]える 人[ひと]たちもいる。	
\\	彼は死刑宣告を受けた。	
\\	彼[かれ]は 死刑[しけい] 宣告[せんこく]を 受[う]けた。	
\\	彼は終身刑を宣告された。	
\\	彼[かれ]は 終身[しゅうしん] 刑[けい]を 宣告[せんこく]された。	終身刑=しゅうしんけい= 
\\	宣教師たちは彼らの宗教を先住民に押し付けた。	
\\	宣教師[せんきょうし]たちは 彼[かれ]らの 宗教[しゅうきょう]を 先住民[せんじゅうみん]に 押し付[おしつ]けた。	宣教師= 
\\	宣教師たちは、宗教の他にも多くのものを日本にもたらした。	
\\	宣教師[せんきょうし]たちは、 宗教[しゅうきょう]の 他[ほか]にも 多[おお]くのものを日本にもたらした。	宣教師= 
\\	もたらす= 
\\	この店ではたばこを取り扱っています。	
\\	この 店[みせ]ではたばこを 取り扱[とりあつか]っています。	取り扱う= 
\\	この薬品は取り扱いに注意を要します。	
\\	この 薬品[やくひん]は 取り扱[とりあつか]いに 注意[ちゅうい]を 要[よう]します。	取り扱う= 
\\	薬品= 
\\	これは適切に取り扱わないと本当に大変なことになる。	
\\	これは 適切[てきせつ]に 取り扱[とりあつか]わないと 本当[ほんとう]に 大変[たいへん]なことになる。	取り扱う= 
\\	こんなふうに特別扱いを受けるなんて変な感じです。	
\\	こんなふうに 特別[とくべつ] 扱[あつか]いを 受[う]けるなんて 変[へん]な 感[かん]じです。	
\\	麻薬や酒に依存している人たちが路上にいる。	
\\	麻薬[まやく]や 酒[さけ]に 依存[いぞん]している 人[ひと]たちが 路上[ろじょう]にいる。	路上= 
\\	そのスーツケースの検査で麻薬が発見された。	
\\	そのスーツケースの 検査[けんさ]で 麻薬[まやく]が 発見[はっけん]された。	
\\	アルコールと麻薬は脳に多くの影響を与える。	
\\	アルコールと 麻薬[まやく]は 脳[のう]に 多[おお]くの 影響[えいきょう]を 与[あた]える。	
\\	先生は麻薬を使ってはいけない理由を説明しました。	
\\	先生[せんせい]は 麻薬[まやく]を 使[つか]ってはいけない 理由[りゆう]を 説明[せつめい]しました。	
\\	危険だから、麻薬に手を出さないようにしなさい。	
\\	危険[きけん]だから、 麻薬[まやく]に 手[て]を 出[だ]さないようにしなさい。	
\\	まひの兆候はありません。	
\\	まひの 兆候[ちょうこう]はありません。	
\\	彼は麻薬中毒から逃れようとした。	
\\	彼[かれ]は 麻薬[まやく] 中毒[ちゅうどく]から 逃[のが]れようとした。	逃れる=のがれる= 
\\	緊急ですか?	
\\	緊急[きんきゅう]ですか?	
\\	その国の人々を助けるために医療品が緊急に必要です。	
\\	その 国[くに]の 人々[ひとびと]を 助[たす]けるために 医療[いりょう] 品[ひん]が 緊急[きんきゅう]に 必要[ひつよう]です。	
\\	緊迫した5時間にわたる会議がやっと終わった。	
\\	緊迫[きんぱく]した5 時間[じかん]にわたる 会議[かいぎ]がやっと 終[お]わった。	緊迫=きんぱく= 
\\	スリランカの状況はいまだ非常に緊迫している。	
\\	スリランカの 状況[じょうきょう]はいまだ 非常[ひじょう]に 緊迫[きんぱく]している。	緊迫=きんぱく= 
\\	アメリカは韓国と緊密に協力すると誓約しています。	
\\	アメリカは 韓国[かんこく]と 緊密[きんみつ]に 協力[きょうりょく]すると 誓約[せいやく]しています。	誓約=せいやく= 
\\	この会談は、三国間の関係がより緊密になっていること示しています。	
\\	この 会談[かいだん]は、三 国[こく] 間[かん]の 関係[かんけい]がより 緊密[きんみつ]になっていること 示[しめ]しています。	
\\	日本女性の肌が若くみずみずしいことは世界的にもよく知られている。	
\\	日本 女性[じょせい]の 肌[はだ]が 若[わか]くみずみずしいことは 世界[せかい] 的[てき]にもよく 知[し]られている。	
\\	学生ローンによる負債は、非常に大きな問題です。	
\\	学生[がくせい]ローンによる 負債[ふさい]は、 非常[ひじょう]に 大[おお]きな 問題[もんだい]です。	
\\	親たちは子供たちに道徳上の指導をしなければならない。	
\\	親[おや]たちは 子供[こども]たちに 道徳[どうとく] 上[じょう]の 指導[しどう]をしなければならない。	道徳上=どうとくじょう= 
\\	特に婚前の性交渉は不道徳であると考えています。	
\\	特[とく]に 婚前[こんぜん]の 性[せい] 交渉[こうしょう]は 不道徳[ふどうとく]であると 考[かんが]えています。	性交渉=せいこうしょう= 
\\	結婚前の男女が一緒に暮らすというのは、道徳的に間違っている。	
\\	結婚[けっこん] 前[まえ]の 男女[だんじょ]が 一緒[いっしょ]に 暮[く]らすというのは、 道徳[どうとく] 的[てき]に 間違[まちが]っている。	
\\	調査によると、食べ過ぎが悪徳の一位になりました。	
\\	調査[ちょうさ]によると、 食[た]べ 過[す]ぎが 悪徳[あくとく]の一 位[い]になりました。	悪徳=あくとく= 
\\	位=い= 
\\	勤勉は美徳です。	
\\	勤勉[きんべん]は 美徳[びとく]です。	
\\	日本には「謙譲の美徳」という言葉があります。	
\\	日本には
\\	謙譲[けんじょう]の 美徳[びとく]」という 言葉[ことば]があります。	謙譲の美徳=けんじょう の びとく
\\	ジェーンは車の鍵をさがそうとかばんの中をごそごそと探った。	
\\	ジェーンは 車[くるま]の 鍵[かぎ]をさがそうとかばんの 中[なか]をごそごそと 探[さぐ]った。	ごそごそ= 
\\	私は仕事についてはまだ手探りの状態である。	
\\	私[わたし]は 仕事[しごと]についてはまだ 手探[てさぐ]りの 状態[じょうたい]である。	手探り= 
\\	この実験は、日本の宇宙探査の始まりを象徴していた。	
\\	この 実験[じっけん]は、 日本[にっぽん]の 宇宙[うちゅう] 探査[たんさ]の 始[はじ]まりを 象徴[しょうちょう]していた。	
\\	多くの女性は、40歳以降の出産には若干のリスクが伴うことを知っています。	
\\	多[おお]くの 女性[じょせい]は、40 歳[さい] 以降[いこう]の 出産[しゅっさん]には 若干[じゃっかん]のリスクが 伴[ともな]うことを 知[し]っています。	以降=いこう= 
\\	若干=じゃっかん= 
\\	伴う=ともなう= 
\\	感情的な危険を伴う決定から身を引きがちです。	
\\	感情[かんじょう] 的[てき]な 危険[きけん]を 伴[ともな]う 決定[けってい]から 身[み]を 引[ひ]きがちです。	伴う=ともなう= 
\\	身を引く= 
\\	政治決断には常にリスクが伴う。	
\\	政治[せいじ] 決断[けつだん]には 常[つね]にリスクが 伴[ともな]う。	政治決断=せいじ けつだん= 
\\	このような力の行使には責任が伴います。	
\\	このような 力[ちから]の 行使[こうし]には 責任[せきにん]が 伴[ともな]います。	行使=こうし= 
\\	今回は息子も同伴するんです。	
\\	今回[こんかい]は 息子[むすこ]も 同伴[どうはん]するんです。	同伴=どうはん= 
\\	彼女の弟が彼女の旅行に同伴しました。	
\\	彼女[かのじょ]の 弟[おとうと]が 彼女[かのじょ]の 旅行[りょこう]に 同伴[どうはん]しました。	同伴=どうはん= 
\\	それに伴って日本人読者も急増した。	
\\	それに 伴[ともな]って日本人 読者[どくしゃ]も 急増[きゅうぞう]した。	
\\	グローバリゼーションに伴って、私たちは今とても複雑な世界に生きている。	
\\	グローバリゼーションに 伴[ともな]って、 私[わたし]たちは 今[いま]とても 複雑[ふくざつ]な 世界[せかい]に 生[い]きている。	
\\	私たちは今、夫の転勤に伴って、アメリカに住んでいます。	
\\	私[わたし]たちは 今[いま]、 夫[おっと]の 転勤[てんきん]に 伴[ともな]って、アメリカに 住[す]んでいます。	
\\	バンドの伴奏がないと、私は歌えません。	
\\	バンドの 伴奏[ばんそう]がないと、 私[わたし]は 歌[うた]えません。	伴走=ばんそう= 
\\	伴侶を選ぶときは慎重にしなさい。	
\\	伴侶[はんりょ]を 選[えら]ぶときは 慎重[しんちょう]にしなさい。	伴侶= 
\\	慎重= 
\\	私には自分の人生の伴侶を選ぶ権利がある。	
\\	私[わたし]には 自分[じぶん]の 人生[じんせい]の 伴侶[はんりょ]を 選[えら]ぶ 権利[けんり]がある。	伴侶= 
\\	口を挟まずに、彼の言い分を聞きなさい。	
\\	口[くち]を 挟[はさ]まずに、 彼[かれ]の 言い分[いいぶん]を 聞[き]きなさい。	口を挟む=くち を はさむ= 
\\	口を挟まないでください。	
\\	口[くち]を 挟[はさ]まないでください。	口を挟む=くち を はさむ= 
\\	口を挟んですみませんが、あなたは事実を取り違えていると思います。	
\\	口[くち]を 挟[はさ]んですみませんが、あなたは 事実[じじつ]を 取り違[とりちが]えていると 思[おも]います。	口を挟む=くち を はさむ= 
\\	プライベートには口を挟まないでください。	
\\	プライベートには 口[くち]を 挟[はさ]まないでください。	口を挟む=くち を はさむ= 
\\	祖父母ですら私たちの生涯の伴侶選びに口を挟んでいました。	
\\	祖父母[そふぼ]ですら 私[わたし]たちの 生涯[しょうがい]の 伴侶[はんりょ] 選[えら]びに 口[くち]を 挟[はさ]んでいました。	口を挟む=くち を はさむ= 
\\	生涯=しょうがい= 
\\	私の母は、出勤前に洗濯ができるよう、朝6時に起きます。	
\\	私[わたし]の 母[はは]は、 出勤[しゅっきん] 前[まえ]に 洗濯[せんたく]ができるよう、 朝[あさ]6 時[じ]に 起[お]きます。	出勤=しゅっきん= 
\\	彼は6時に出勤します。	
\\	彼[かれ]は6 時[じ]に 出勤[しゅっきん]します。	出勤=しゅっきん= 
\\	彼は時間通りに出勤することがない。	
\\	彼[かれ]は 時間[じかん] 通[どお]りに 出勤[しゅっきん]することがない。	出勤=しゅっきん= 
\\	毎朝6時に起きて、出勤前に1時間はしっかり運動するようにしている。	
\\	毎朝[まいあさ]6 時[じ]に 起[お]きて、 出勤[しゅっきん] 前[まえ]に1 時間[じかん]はしっかり 運動[うんどう]するようにしている。	出勤=しゅっきん= 
\\	私の勤め先は都心にあります。	
\\	私[わたし]の 勤め先[つとめさき]は 都心[としん]にあります。	
\\	彼は、医者という職業を満喫しているが、勤め先の病院が嫌いです。	
\\	彼[かれ]は、 医者[いしゃ]という 職業[しょくぎょう]を 満喫[まんきつ]しているが、 勤め先[つとめさき]の 病院[びょういん]が 嫌[きら]いです。	満喫=まんきつ= 
\\	夜勤が終わったばかりですか?	
\\	夜勤[やきん]が 終[お]わったばかりですか?	
\\	彼は病院で夜勤をしています。	
\\	彼[かれ]は 病院[びょういん]で 夜勤[やきん]をしています。	
\\	非常勤の労働者は労働者全体の約4分の1を占めていました。	
\\	非常勤[ひじょうきん]の 労働[ろうどう] 者[しゃ]は 労働[ろうどう] 者[しゃ] 全体[ぜんたい]の 約[やく]4 分[ぶん]の1を 占[し]めていました。	占める=しめる= 
\\	彼はまた無断欠勤をしました。	
\\	彼[かれ]はまた 無断[むだん] 欠勤[けっきん]をしました。	無断=むだん= 
\\	欠勤=けっきん= 
\\	大学へ行くか、家業を継ぐかで迷った。	
\\	大学[だいがく]へ 行[い]くか、 家業[かぎょう]を 継[つ]ぐかで 迷[まよ]った。	継ぐ=つぐ= 
\\	父親の死後、家業を継ぐためにクリーブランドに戻りました。	
\\	父親[ちちおや]の 死後[しご]、 家業[かぎょう]を 継[つ]ぐためにクリーブランドに 戻[もど]りました。	継ぐ=つぐ= 
\\	彼が父親の事業を引き継ぐのは避けられないだろう。	
\\	彼[かれ]が 父親[ちちおや]の 事業[じぎょう]を 引き継[ひきつ]ぐのは 避[さ]けられないだろう。	引き継ぐ= 
\\	休暇中、私の仕事を引き継いでやってくれる。	
\\	休暇[きゅうか] 中[ちゅう]、 私[わたし]の 仕事[しごと]を 引き継[ひきつ]いでやってくれる。	引き継ぐ= 
\\	会長は娘を後継者に任命しました。	
\\	会長[かいちょう]は 娘[むすめ]を 後継[こうけい] 者[しゃ]に 任命[にんめい]しました。	後継者=こうけいしゃ= 
\\	任命=にんめい= 
\\	彼なら私の後継者として賛成です。	
\\	彼[かれ]なら 私[わたし]の 後継[こうけい] 者[しゃ]として 賛成[さんせい]です。	後継者=こうけいしゃ= 
\\	彼らはスミスさんの後継者となる新入社員を雇った。	
\\	彼[かれ]らはスミスさんの 後継[こうけい] 者[しゃ]となる 新入[しんにゅう] 社員[しゃいん]を 雇[やと]った。	後継者=こうけいしゃ= 
\\	この大会は3時間にわたって全米に生中継されました。	
\\	この 大会[たいかい]は3 時間[じかん]にわたって 全米[ぜんべい]に 生[なま] 中継[ちゅうけい]されました。	生中継=なまちゅうけい= 
\\	レースはテレビで世界中に生中継もされます。	
\\	レースはテレビで 世界中[せかいじゅう]に 生[なま] 中継[ちゅうけい]もされます。	生中継=なまちゅうけい= 
\\	あの店は親から子へ3代にわたって受け継がれた。	
\\	あの 店[みせ]は 親[おや]から 子[こ]へ3 代[だい]にわたって 受け継[うけつ]がれた。	受け継ぐ= 
\\	それは父から受け継いだものだと思う。	
\\	それは 父[ちち]から 受け継[うけつ]いだものだと 思[おも]う。	受け継ぐ= 
\\	複数のプロジェクトに同時に取り掛かっています。	
\\	複数[ふくすう]のプロジェクトに 同時[どうじ]に 取り掛[とりか]かっています。	複数= 
\\	取り掛かる= 
\\	このプログラムは複数のことをこなす。	
\\	このプログラムは 複数[ふくすう]のことをこなす。	複数= 
\\	このソフトウェアを複製することは違法です。	
\\	このソフトウェアを 複製[ふくせい]することは 違法[いほう]です。	複製= 
\\	ここはちょっと気味悪い。	
\\	ここはちょっと 気味悪[きみわる]い。	気味悪い=きみ わるい= 
\\	そして、一分かそこら、とても居心地の悪い黙りが続きました。	
\\	そして、 一分[いっぷん]かそこら、とても 居心地[いごこち]の 悪[わる]い 黙[だんま]りが 続[つづ]きました。	かそこら= 
\\	黙り=だんまり= 
\\	みんなが出掛けてしまえば、東京は居心地のいい静かな場所になる。	
\\	みんなが 出掛[でか]けてしまえば、 東京[とうきょう]は 居心地[いごこち]のいい 静[しず]かな 場所[ばしょ]になる。	
\\	ネコは椅子の上で居心地がよさそうに眠っていた。	
\\	ネコは 椅子[いす]の 上[うえ]で 居心地[いごこち]がよさそうに 眠[ねむ]っていた。	
\\	新しい環境で居心地が悪い?	
\\	新[あたら]しい 環境[かんきょう]で 居心地[いごこち]が 悪[わる]い?	
\\	主任教師は学校を辞めました。	
\\	主任[しゅにん] 教師[きょうし]は 学校[がっこう]を 辞[や]めました。	
\\	ウェブデザイナーには、この研修会が役立ちます。	
\\	ウェブデザイナーには、この 研修[けんしゅう] 会[かい]が 役立[やくだ]ちます。	研修会=けんしゅうかい= 
\\	就労ビザと研修ビザは、全く性質の異なるものです。	
\\	就労[しゅうろう]ビザと 研修[けんしゅう]ビザは、 全[まった]く 性質[せいしつ]の 異[こと]なるものです。	
\\	彼は、研修プログラムに参加しました。	
\\	彼[かれ]は、 研修[けんしゅう]プログラムに 参加[さんか]しました。	
\\	彼はちょっとした外出のときは、鍵をかけない。	
\\	彼[かれ]はちょっとした 外出[がいしゅつ]のときは、 鍵[かぎ]をかけない。	
\\	このドアは自動的に鍵がかかるので、閉め出されないように気をつけること。	
\\	このドアは 自動的[じどうてき]に 鍵[かぎ]がかかるので、 閉め出[しめだ]されないように 気[き]をつけること。	
\\	その引出しは鍵がかかっていて、何が入っているのか分からない。	
\\	その 引出[ひきだ]しは 鍵[かぎ]がかかっていて、 何[なに]が 入[はい]っているのか 分[わ]からない。	
\\	畳の部屋にじゅうたんを敷くのは、あまり勧められない。	
\\	畳[たたみ]の 部屋[へや]にじゅうたんを 敷[し]くのは、あまり 勧[すす]められない。	じゅうたんを敷く= 
\\	誰かがワインをこぼして、じゅうたんにしみをつけてしまった。	
\\	誰[だれ]かがワインをこぼして、じゅうたんにしみをつけてしまった。	
\\	じゅうたんに掃除機をかけていました。	
\\	じゅうたんに 掃除[そうじ] 機[き]をかけていました。	
\\	私がソファに座ると、すぐに犬がやってきて隣に飛び乗る。	
\\	私[わたし]がソファに 座[すわ]ると、すぐに 犬[いぬ]がやってきて 隣[となり]に 飛び乗[とびの]る。	
\\	ラジオを聞きながら砂浜に寝そべっていた。	
\\	ラジオを 聞[き]きながら 砂浜[すなはま]に 寝[ね]そべっていた。	砂浜=すなはま= 
\\	寝そべる= 
\\	ソファに横になったら、そのまま朝まで眠ってしまった。	
\\	ソファに 横[よこ]になったら、そのまま 朝[あさ]まで 眠[ねむ]ってしまった。	
\\	彼はだるい気分に負けてソファで眠り込んでしまった。	
\\	彼[かれ]はだるい 気分[きぶん]に 負[ま]けてソファで 眠り込[ねむりこ]んでしまった。	だるい= 
\\	みんながテーブルについたら、乾杯しよう。	
\\	みんながテーブルについたら、 乾杯[かんぱい]しよう。	
\\	テーブルに料理を並べるのを手伝ってくれる?	
\\	テーブルに 料理[りょうり]を 並[なら]べるのを 手伝[てつだ]ってくれる?	
\\	ドアが閉まらないように、押さえていてもらえますか?	
\\	ドアが 閉[し]まらないように、 押[お]さえていてもらえますか?	押さえる= 
\\	誰かが隣の部屋のドアをノックする音が聞こえた。	
\\	誰[だれ]かが 隣[となり]の 部屋[へや]のドアをノックする 音[おと]が 聞[き]こえた。	
\\	彼はチラチラ時計を見ていたので、何か用事があったのかもしれない。	
\\	彼[かれ]はチラチラ 時計[とけい]を 見[み]ていたので、 何[なに]か 用事[ようじ]があったのかもしれない。	チラチラ= 
\\	チラチラと雪が降ってて、とってもきれいですよ。	
\\	チラチラと 雪[ゆき]が 降[ふ]ってて、とってもきれいですよ。	チラチラ= 
\\	友達が通り掛ったちょうどその時、私は窓から外をチラチラ見ていた。	
\\	友達[ともだち]が 通り掛[とおりかか]ったちょうどその 時[とき]、 私[わたし]は 窓[まど]から 外[そと]をチラチラ 見[み]ていた。	チラチラ= 
\\	明かりがずっとチラチラしているのですが。	
\\	明[あ]かりがずっとチラチラしているのですが。	チラチラ= 
\\	遠くで小さな明かりがチラチラしていた。	
\\	遠[とお]くで 小[ちい]さな 明[あ]かりがチラチラしていた。	
\\	ちょっとしたことが起こった。	
\\	ちょっとしたことが 起[お]こった。	
\\	あなたにちょっとしたものを持ってきました。	
\\	あなたにちょっとしたものを 持[も]ってきました。	
\\	あなたの上司といい関係を築くことは、大変重要です。	
\\	あなたの 上司[じょうし]といい 関係[かんけい]を 築[きず]くことは、 大変[たいへん] 重要[じゅうよう]です。	築く= 
\\	オーストリアで、私は他人と良い関係を築く方法を学びました。	
\\	オーストリアで、 私[わたし]は 他人[たにん]と 良[よ]い 関係[かんけい]を 築[きず]く 方法[ほうほう]を 学[まな]びました。	築く= 
\\	僕の両親はアメリカで今の地位を築くために大変な努力をした。	
\\	僕[ぼく]の 両親[りょうしん]はアメリカで 今[いま]の 地位[ちい]を 築[きず]くために 大変[たいへん]な 努力[どりょく]をした。	築く= 
\\	ビーン氏は独自の方法を用いて、少ない予算で強いチームを築いている。	
\\	ビーン 氏[し]は 独自[どくじ]の 方法[ほうほう]を 用[もち]いて、 少[すく]ない 予算[よさん]で 強[つよ]いチームを 築[きず]いている。	築く= 
\\	来日後、彼は地域の人々と親しい関係を築いてきた。	
\\	来日[らいにち] 後[ご]、 彼[かれ]は 地域[ちいき]の 人々[ひとびと]と 親[した]しい 関係[かんけい]を 築[きず]いてきた。	築く= 
\\	その高額な改築で、我々は負債を負った。	
\\	その 高額[こうがく]な 改築[かいちく]で、 我々[われわれ]は 負債[ふさい]を 負[お]った。	
\\	その国々は、より密接な関係の構築を望んでいる。	
\\	その 国々[くにぐに]は、より 密接[みっせつ]な 関係[かんけい]の 構築[こうちく]を 望[のぞ]んでいる。	構築=こうちく= 
\\	今、私は自分を再構築している。	
\\	今[いま]、 私[わたし]は 自分[じぶん]を 再[さい] 構築[こうちく]している。	再構築=さい こう ちく= 
\\	彼は戦略の構築について語った。	
\\	彼[かれ]は 戦略[せんりゃく]の 構築[こうちく]について 語[かた]った。	構築=こうちく= 
\\	戦略=せんりゃく= 
\\	建築ファンにも人気のスポットとして知られている。	
\\	建築[けんちく]ファンにも 人気[にんき]のスポットとして 知[し]られている。	
\\	建築家の目から見て日本をどう思われますか?	
\\	建築[けんちく] 家[か]の 目[め]から 見[み]て 日本[にっぽん]をどう 思[おも]われますか?	
\\	この建物は、現代建築のスタイルの一例です。	
\\	この 建物[たてもの]は、 現代[げんだい] 建築[けんちく]のスタイルの一 例[れい]です。	
\\	この建築様式の起源はギリシャだ。	
\\	この 建築[けんちく] 様式[ようしき]の 起源[きげん]はギリシャだ。	
\\	新築祝いのパーティーの後、彼女は友人からお返しの招待を受けた。	
\\	新築[しんちく] 祝[いわ]いのパーティーの 後[のち]、 彼女[かのじょ]は 友人[ゆうじん]からお 返[かえ]しの 招待[しょうたい]を 受[う]けた。	新築祝い=しんちくいわい= 
\\	3階部分が現在の建物に増築される。	
\\	階[かい] 部分[ぶぶん]が 現在[げんざい]の 建物[たてもの]に 増築[ぞうちく]される。	
\\	我が社は世界の多くの有名企業と協力関係を築き上げています。	
\\	我[わ]が 社[しゃ]は 世界[せかい]の 多[おお]くの 有名[ゆうめい] 企業[きぎょう]と 協力[きょうりょく] 関係[かんけい]を 築き上[きずきあ]げています。	築き上げる= 
\\	我が社= わがしゃ= 
\\	人間は英知で文明を築き上げてきました。	
\\	人間[にんげん]は 英知[えいち]で 文明[ぶんめい]を 築き上[きずきあ]げてきました。	築き上げる= 
\\	英知= 
\\	文明=ぶんめい= 
\\	いつ天災が起こるか分からない。	
\\	いつ 天災[てんさい]が 起[お]こるか 分[わ]からない。	
\\	あなたの写真は返却されませんのでご了承下さい。	
\\	あなたの 写真[しゃしん]は 返却[へんきゃく]されませんのでご 了承[りょうしょう] 下[くだ]さい。	了承= 
\\	お席には限りがありますのでご了承下さい。	
\\	お 席[せき]には 限[かぎ]りがありますのでご 了承[りょうしょう] 下[くだ]さい。	了承= 
\\	その大臣は辞表を提出して、了承された。	
\\	その 大臣[だいじん]は 辞表[じひょう]を 提出[ていしゅつ]して、 了承[りょうしょう]された。	了承= 
\\	辞表= 
\\	何とぞご了承ください。	
\\	何[なに]とぞご 了承[りょうしょう]ください。	何とぞ= 
\\	了承= 
\\	お客様の電子メールによるご注文、確かに承りました。	
\\	お 客様[きゃくさま]の 電子[でんし]メールによるご 注文[ちゅうもん]、 確[たし]かに 承[うけたまわ]りました。	承る= 
\\	お客様のご注文を承りましたことを取り急ぎご連絡致します。	
\\	お 客様[きゃくさま]のご 注文[ちゅうもん]を 承[うけたまわ]りましたことを 取り急[とりいそ]ぎご 連絡[れんらく] 致[いた]します。	承る= 
\\	でございます。ご用件を承ります。	
\\	でございます。ご 用件[ようけん]を 承[うけたまわ]ります。	承る= 
\\	伝言を承りましょうか?	
\\	伝言[でんごん]を 承[うけたまわ]りましょうか?	承る= 
\\	私の休暇中は太郎が代わりに承ります。	
\\	私[わたし]の 休暇[きゅうか] 中[ちゅう]は 太郎[たろう]が 代[か]わりに 承[うけたまわ]ります。	承る= 
\\	彼女が支払ったことを承諾と見て取った。	
\\	彼女[かのじょ]が 支払[しはら]ったことを 承諾[しょうだく]と 見て取[みてと]った。	承諾=しょうだく= 
\\	父は私たちの結婚を承諾してくれた。	
\\	父[ちち]は 私[わたし]たちの 結婚[けっこん]を 承諾[しょうだく]してくれた。	承諾=しょうだく= 
\\	私は心ならずも承諾しました。	
\\	私[わたし]は 心[こころ]ならずも 承諾[しょうだく]しました。	心ならずも= 
\\	承諾=しょうだく= 
\\	この情報をあなたの部署の人たちに伝えていただけませんか。	
\\	この 情報[じょうほう]をあなたの 部署[ぶしょ]の 人[ひと]たちに 伝[つた]えていただけませんか。	部署= 
\\	この部署は、月例会議を行っています。	
\\	この 部署[ぶしょ]は、 月例[げつれい] 会議[かいぎ]を 行[おこな]っています。	部署= 
\\	月例= 
\\	矢印が右を差していれば右に進んでください。	
\\	矢印[やじるし]が 右[みぎ]を 差[さ]していれば 右[みぎ]に 進[すす]んでください。	矢印=やじるし= 
\\	その矢印は左向きです。	
\\	その 矢印[やじるし]は 左向[ひだりむ]きです。	矢印=やじるし= 
\\	その政治家は非難の矢面に立たされて、もう少しで泣くところだった。	
\\	その 政治[せいじ] 家[か]は 非難[ひなん]の 矢面[やおもて]に 立[た]たされて、もう 少[すこ]しで 泣[な]くところだった。	矢面=やおもて= 
\\	非難=ひなん= 
\\	幼い子供たちは、聞こえた音を真似することがある。	
\\	幼[おさな]い 子供[こども]たちは、 聞[き]こえた 音[おと]を 真似[まね]することがある。	
\\	幼い男の子が電話に出た。	
\\	幼[おさな]い 男の子[おとこのこ]が 電話[でんわ]に 出[で]た。	
\\	幼稚なことはやめて年相応にしっかりしてください。	
\\	幼稚[ようち]なことはやめて 年[とし] 相応[そうおう]にしっかりしてください。	幼稚=ようち= 
\\	年相応=としそうおう= 
\\	彼はあの年齢にしては少し幼稚です。	
\\	彼[かれ]はあの 年齢[ねんれい]にしては 少[すこ]し 幼稚[ようち]です。	幼稚=ようち= 
\\	留学生の目から見ると、日本の大学生は幼稚なんです。	
\\	留学生[りゅうがくせい]の 目[め]から 見[み]ると、 日本[にっぽん]の 大学生[だいがくせい]は 幼稚[ようち]なんです。	幼稚=ようち= 
\\	幼稚園の先生は、子供たち全員を自分の周りに集めた。	
\\	幼稚園[ようちえん]の 先生[せんせい]は、 子供[こども]たち 全員[ぜんいん]を 自分[じぶん]の 周[まわ]りに 集[あつ]めた。	
\\	幼稚園の影響が大きい。	
\\	幼稚園[ようちえん]の 影響[えいきょう]が 大[おお]きい。	
\\	汚職を根絶しようと取り組んで、彼は大勢の敵を作りました。	
\\	汚職[おしょく]を 根絶[こんぜつ]しようと 取り組[とりく]んで、 彼[かれ]は 大勢[おおぜい]の 敵[てき]を 作[つく]りました。	根絶=こんぜつ= 
\\	ロシアは汚職と貧困に満ちている。	
\\	ロシアは 汚職[おしょく]と 貧困[ひんこん]に 満[み]ちている。	満ちる= 
\\	それは間違いなく日本史における汚点の一つだ。	
\\	それは 間違[まちが]いなく 日本[にっぽん] 史[し]における 汚点[おてん]の 一[ひと]つだ。	汚点=おてん= 
\\	汚れている皿を片付けた。	
\\	汚[よご]れている 皿[さら]を 片付[かたづ]けた。	
\\	宇宙は刻々と変化している。	
\\	宇宙[うちゅう]は 刻々[こっこく]と 変化[へんか]している。	刻々=こっこく= 
\\	時が刻々と過ぎる。	
\\	時[とき]が 刻々[こっこく]と 過[す]ぎる。	刻々=こっこく= 
\\	タマネギを刻んだ。	
\\	タマネギを 刻[きざ]んだ。	
\\	定刻通りに空港に着いた。	
\\	定刻[ていこく] 通[どお]りに 空港[くうこう]に 着[つ]いた。	定刻= 
\\	きちんとした身なりで、定刻に来るように言われました。	
\\	きちんとした 身[み]なりで、 定刻[ていこく]に 来[く]るように 言[い]われました。	身なり= 
\\	定刻= 
\\	列車は定刻に出ますか?	
\\	列車[れっしゃ]は 定刻[ていこく]に 出[で]ますか?	
\\	こちらの貸し出しをお願いします。	
\\	こちらの 貸し出[かしだ]しをお 願[ねが]いします。	
\\	私の部屋の賃貸契約は切れました。	
\\	私[わたし]の 部屋[へや]の 賃貸[ちんたい] 契約[けいやく]は 切[き]れました。	賃貸=ちんたい= 
\\	契約=けいやく= 
\\	私のアパートの賃貸契約は来月、期限が切れる。	
\\	私[わたし]のアパートの 賃貸[ちんたい] 契約[けいやく]は 来月[らいげつ]、 期限[きげん]が 切[き]れる。	賃貸=ちんたい= 
\\	契約=けいやく= 
\\	そのレストランの部屋を貸し切りにした。	
\\	そのレストランの 部屋[へや]を 貸し切[かしき]りにした。	貸し切り= 
\\	私は昨日彼らが喫茶店でいちゃいちゃしているのを見た。	
\\	私[わたし]は 昨日[きのう] 彼[かれ]らが 喫茶店[きっさてん]でいちゃいちゃしているのを 見[み]た。	いちゃいちゃ= 
\\	この本はぼろぼろになるまで読んだ。	
\\	この 本[ほん]はぼろぼろになるまで 読[よ]んだ。	ぼろぼろ= 
\\	これらの中古本のほとんどは、ぼろぼろに傷んで、破れている。	
\\	これらの 中古本[ちゅうこほん]のほとんどは、ぼろぼろに 傷[いた]んで、 破[やぶ]れている。	ぼろぼろ= 
\\	頭痛がとうとう治まった。	
\\	頭痛[ずつう]がとうとう 治[おさ]まった。	
\\	いよいよ私の番です。	
\\	いよいよ 私[わたし]の 番[ばん]です。	いよいよ= 
\\	寒さはいよいよ厳しくなった。	
\\	寒[さむ]さはいよいよ 厳[きび]しくなった。	いよいよ= 
\\	いよいよという時に言葉が出ない。	
\\	いよいよという 時[とき]に 言葉[ことば]が 出[で]ない。	いよいよ= 
\\	これをこっそりとやるべきだった。	
\\	これをこっそりとやるべきだった。	こっそり= 
\\	彼女は恋人とはこっそり会わなければならなかった。	
\\	彼女[かのじょ]は 恋人[こいびと]とはこっそり 会[あ]わなければならなかった。	こっそり= 
\\	ダイエット本を人に隠れてこっそり読んでいる。	
\\	ダイエット 本[ほん]を 人[ひと]に 隠[かく]れてこっそり 読[よ]んでいる。	こっそり= 
\\	彼が彼女を殺した理由は、彼女がこっそりと他の男と付き合っていたからだ。	
\\	彼[かれ]が 彼女[かのじょ]を 殺[ころ]した 理由[りゆう]は、 彼女[かのじょ]がこっそりと 他[た]の 男[おとこ]と 付き合[つきあ]っていたからだ。	こっそり= 
\\	彼は夜中によくこっそり家を抜け出したものだった。	
\\	彼[かれ]は 夜中[やちゅう]によくこっそり 家[いえ]を 抜け出[ぬけだ]したものだった。	こっそり= 
\\	彼はこっそりささやいた。	
\\	彼[かれ]はこっそりささやいた。	こっそり= 
\\	彼は僕に彼女の写真をこっそり見せてくれた。	
\\	彼[かれ]は 僕[ぼく]に 彼女[かのじょ]の 写真[しゃしん]をこっそり 見[み]せてくれた。	こっそり= 
\\	多くの人は健康的、また魅力的に見せるため人工的に日焼けをします。	
\\	多[おお]くの 人[ひと]は 健康[けんこう] 的[てき]、また 魅力[みりょく] 的[てき]に 見[み]せるため 人工[じんこう] 的[てき]に 日焼[ひや]けをします。	人工的= 
\\	妊婦は激しい運動をしない方が良い。	
\\	妊婦[にんぷ]は 激[はげ]しい 運動[うんどう]をしない 方[ほう]が 良[よ]い。	妊婦=にんぷ= 
\\	今日の出張中の食事手当は、45ドルです。	
\\	今日[きょう]の 出張[しゅっちょう] 中[ちゅう]の 食事[しょくじ] 手当[てあて]は、45ドルです。	手当= 
\\	彼は失業手当で生活しています。	
\\	彼[かれ]は 失業[しつぎょう] 手当[てあて]で 生活[せいかつ]しています。	手当= 
\\	選手はまだ選出されていない。	
\\	選手[せんしゅ]はまだ 選出[せんしゅつ]されていない。	
\\	あなたは、マリファナを合法化すべきだと思いますか?	
\\	あなたは、マリファナを 合法[ごうほう] 化[か]すべきだと 思[おも]いますか?	
\\	安楽死は合法化されるべきだと思いますか?	
\\	安楽[あんらく] 死[し]は 合法[ごうほう] 化[か]されるべきだと 思[おも]いますか?	
\\	安楽死に対する賛成意見と反対意見は、どのようなものですか。	
\\	安楽[あんらく] 死[し]に 対[たい]する 賛成[さんせい] 意見[いけん]と 反対[はんたい] 意見[いけん]は、どのようなものですか。	安楽死=あん らく し= 
\\	近年、日本と中国の間で不信感が高まっている。	
\\	近年[きんねん]、 日本[にっぽん]と 中国[ちゅうごく]の 間[ま]で 不信[ふしん] 感[かん]が 高[たか]まっている。	不信感=ふ しん かん= 
\\	その会議の焦点はいかに利益を挙げるかということだった。	
\\	その 会議[かいぎ]の 焦点[しょうてん]はいかに 利益[りえき]を 挙[あ]げるかということだった。	
\\	どこに焦点を置いていますか?	
\\	どこに 焦点[しょうてん]を 置[お]いていますか?	
\\	イラク戦争は11月の大統領選挙の焦点です。	
\\	イラク 戦争[せんそう]は 11月[じゅういちがつ]の 大統領[だいとうりょう] 選挙[せんきょ]の 焦点[しょうてん]です。	
\\	貧富の差が大きい。	
\\	貧富[ひんぷ]の 差[さ]が 大[おお]きい。	貧富の差=ひんぷ の さ= 
\\	貧富の差は急激に拡大しています。	
\\	貧富[ひんぷ]の 差[さ]は 急激[きゅうげき]に 拡大[かくだい]しています。	貧富の差=ひんぷ の さ= 
\\	大丈夫、その時計は十分進んでいるから、終電には間に合うよ。	
\\	大丈夫[だいじょうぶ]、その 時計[とけい]は 十分[じゅっぷん] 進[すす]んでいるから、 終電[しゅうでん]には 間に合[まにあ]うよ。	
\\	明日から夏時間になるから、全部の時計を一時間進めなきゃ。	
\\	明日[あした]から 夏時間[なつじかん]になるから、 全部[ぜんぶ]の 時計[とけい]を一 時間[じかん] 進[すす]めなきゃ。	
\\	私の腕時計は5分遅れているから、今は5時ちょうどだ。	
\\	私[わたし]の 腕時計[うでどけい]は5 分[ふん] 遅[おく]れているから、 今[いま]は5 時[じ]ちょうどだ。	
\\	娘は3歳ですが、もう時計が読めます。	
\\	娘[むすめ]は3 歳[さい]ですが、もう 時計[とけい]が 読[よ]めます。	
\\	脱いだ上着をハンガーにかけて、しばらく風をあてた。	
\\	脱[ぬ]いだ 上着[うわぎ]をハンガーにかけて、しばらく 風[かぜ]をあてた。	
\\	私は畳の部屋に、ふとんを敷いて寝ています。	
\\	私[わたし]は 畳[たたみ]の 部屋[へや]に、ふとんを 敷[し]いて 寝[ね]ています。	
\\	ベランダでふとんを干した。	
\\	ベランダでふとんを 干[ほ]した。	
\\	全身が緊張感に包まれていたって感じです。	
\\	全身[ぜんしん]が 緊張[きんちょう] 感[かん]に 包[つつ]まれていたって 感[かん]じです。	包まる=くるまる= 
\\	休日にはふとんに寝転んで、のんびり本を読む。	
\\	休日[きゅうじつ]にはふとんに 寝転[ねころ]んで、のんびり 本[ほん]を 読[よ]む。	寝転ぶ= 
\\	あの窓はいつもブラインドが下ろされている。	
\\	あの 窓[まど]はいつもブラインドが 下[お]ろされている。	
\\	子どものころに、自分でベッドを整えるようにしつけられた。	
\\	子[こ]どものころに、 自分[じぶん]でベッドを 整[ととの]えるようにしつけられた。	しつける= 
\\	とても疲れていたので、家に着くとそのままベッドに潜り込んだ。	
\\	とても 疲[つか]れていたので、 家[いえ]に 着[つ]くとそのままベッドに 潜り込[もぐりこ]んだ。	ベッドに潜り込む= 
\\	いつもテレビの前でゴロゴロしていてはいけません。	
\\	いつもテレビの 前[まえ]でゴロゴロしていてはいけません。	ごろごろ= 
\\	この猫ちゃんナデナデすると、ゴロゴロ言うわよ。	
\\	この 猫[ねこ]ちゃんナデナデすると、ゴロゴロ 言[い]うわよ。	ごろごろ= 
\\	撫でる=なでる= 
\\	ただふとんの中でゴロゴロしていた。	
\\	ただふとんの 中[なか]でゴロゴロしていた。	ごろごろ= 
\\	彼女は赤ちゃんのほおを優しく撫でた。	
\\	彼女[かのじょ]は 赤[あか]ちゃんのほおを 優[やさ]しく 撫[な]でた。	撫でる=なでる= 
\\	彼は手であごを撫でた。	
\\	彼[かれ]は 手[て]であごを 撫[な]でた。	撫でる=なでる= 
\\	ケビンは午前中、家でゴロゴロしていた。	
\\	ケビンは 午前[ごぜん] 中[ちゅう]、 家[いえ]でゴロゴロしていた。	ごろごろ= 
\\	今日は家でゴロゴロしていたい。	
\\	今日[きょう]は 家[いえ]でゴロゴロしていたい。	ごろごろ= 
\\	日本人の男は、畳の上でゴロゴロしているのが好きなようだ。	
\\	日本人[にっぽんじん]の 男[おとこ]は、 畳[たたみ]の 上[うえ]でゴロゴロしているのが 好[す]きなようだ。	ごろごろ= 
\\	彼女の体重でその椅子は、軋んだ。	
\\	彼女[かのじょ]の 体重[たいじゅう]でその 椅子[いす]は、 軋[きし]んだ。	きしむ= 
\\	目が覚めて時計を見たとたん、ベッドから飛び起きた。	
\\	目[め]が 覚[さ]めて 時計[とけい]を 見[み]たとたん、ベッドから 飛び起[とびお]きた。	
\\	6時に目覚ましをかけたはずなのに、鳴らなかった。	
\\	時[じ]に 目覚[めざ]ましをかけたはずなのに、 鳴[な]らなかった。	
\\	知らないうちに目覚ましを止めていたらしく、起きたらもう11時だった。	
\\	知[し]らないうちに 目覚[めざ]ましを 止[と]めていたらしく、 起[お]きたらもう11 時[じ]だった。	
\\	ふだん着るものには、ほとんどアイロンをかけない。	
\\	ふだん 着[き]るものには、ほとんどアイロンをかけない。	
\\	この素材はしわにならないので、アイロンがいりません。	
\\	この 素材[そざい]はしわにならないので、アイロンがいりません。	
\\	エアコンをつけたまま寝たら、のどが痛くなった。	
\\	エアコンをつけたまま 寝[ね]たら、のどが 痛[いた]くなった。	
\\	1時間後に切れるように、エアコンのタイマーをかけた。	
\\	時間[じかん] 後[ご]に 切[き]れるように、エアコンのタイマーをかけた。	
\\	このセーターを乾燥機にかけたら、縮んでしまった。	
\\	このセーターを 乾燥[かんそう] 機[き]にかけたら、 縮[ちぢ]んでしまった。	乾燥機=かん そう き= 
\\	縮む=ちぢむ= 
\\	蛍光灯がちかちかするようになってきたら、そろそろ取り換え時だ。	
\\	蛍光[けいこう] 灯[とう]がちかちかするようになってきたら、そろそろ 取り換[とりか]え 時[どき]だ。	蛍光灯=けい こう とう= 
\\	蛍光灯はまぶしすぎるので、居間では白熱灯を使っている。	
\\	蛍光[けいこう] 灯[とう]はまぶしすぎるので、 居間[いま]では 白熱[はくねつ] 灯[とう]を 使[つか]っている。	蛍光灯=けい こう とう= 
\\	暗闇の中、手探りで電気のスイッチを入れた。	
\\	暗闇[くらやみ]の 中[なか]、 手探[てさぐ]りで 電気[でんき]のスイッチを 入[い]れた。	暗闇=くら やみ= 
\\	この
\\	レコーダーは、ダビングが終わると自動的にスイッチが切れる。	
\\	この 
\\	レコーダーは、ダビングが 終[お]わると 自動的[じどうてき]にスイッチが 切[き]れる。	
\\	昨日の夜、炊飯器のタイマーをセットするのを忘れていた。	
\\	昨日[きのう]の 夜[よる]、 炊飯[すいはん] 器[き]のタイマーをセットするのを 忘[わす]れていた。	炊飯器=すい はん き= 
\\	このジャケットは、洗濯機で洗える。	
\\	このジャケットは、 洗濯[せんたく] 機[き]で 洗[あら]える。	
\\	洗濯物が大量にたまっていたので、今日は洗濯機を三回も回した。	
\\	洗濯[せんたく] 物[もの]が 大量[たいりょう]にたまっていたので、 今日[きょう]は 洗濯[せんたく] 機[き]を三 回[かい]も 回[まわ]した。	溜まる=たまる= 
\\	犬を飼っているので、毎日掃除機をかけないと、部屋が大変なことになる。	
\\	犬[いぬ]を 飼[か]っているので、 毎日[まいにち] 掃除[そうじ] 機[き]をかけないと、 部屋[へや]が 大変[たいへん]なことになる。	
\\	タイマーを9分にセットして、スパゲッティをゆで始めた。	
\\	タイマーを9 分[ふん]にセットして、スパゲッティをゆで 始[はじ]めた。	
\\	トイレの電球が切れたので、新しい電球を買いに行った。	
\\	トイレの 電球[でんきゅう]が 切[き]れたので、 新[あたら]しい 電球[でんきゅう]を 買[か]いに 行[い]った。	
\\	私は天井に手が届かないのですが、電球を取り換えてもらえませんか?	
\\	私[わたし]は 天井[てんじょう]に 手[て]が 届[とど]かないのですが、 電球[でんきゅう]を 取り換[とりか]えてもらえませんか?	
\\	使っていない部屋の電気を、こまめに消すようにしよう。	
\\	使[つか]っていない 部屋[へや]の 電気[でんき]を、こまめに 消[け]すようにしよう。	こまめ= 
\\	電気をつけっぱなしにしたまま、眠り込んでしまった。	
\\	電気[でんき]をつけっぱなしにしたまま、 眠り込[ねむりこ]んでしまった。	
\\	顔は知っているが一度も話したことのない同僚から、電話がかかってきた。	
\\	顔[かお]は 知[し]っているが一 度[ど]も 話[はな]したことのない 同僚[どうりょう]から、 電話[でんわ]がかかってきた。	
\\	電話を取ると、元気な声が聞こえてくる。	
\\	電話[でんわ]を 取[と]ると、 元気[げんき]な 声[こえ]が 聞[き]こえてくる。	
\\	近くが、運転中に携帯電話を取ると回答しています。	
\\	[ぱーせんと] 近[ちか]くが、 運転[うんてん] 中[ちゅう]に 携帯[けいたい] 電話[でんわ]を 取[と]ると 回答[かいとう]しています。	
\\	日本の職場で働けば当然電話を取ることになる。	
\\	日本[にっぽん]の 職場[しょくば]で 働[はたら]けば 当然[とうぜん] 電話[でんわ]を 取[と]ることになる。	
\\	私はこの電話を取らなくてはなりません。	
\\	私[わたし]はこの 電話[でんわ]を 取[と]らなくてはなりません。	
\\	忙しそうだね。また後で電話をかけ直すよ。	
\\	忙[いそが]しそうだね。また 後[あと]で 電話[でんわ]をかけ 直[なお]すよ。	
\\	電話が遠い。	
\\	電話[でんわ]が 遠[とお]い。	
\\	ちょっと電話が遠いのですが。	
\\	ちょっと 電話[でんわ]が 遠[とお]いのですが。	
\\	シャワーを浴びていて、電話が鳴っているのに気がつかなかった。	
\\	シャワーを 浴[あ]びていて、 電話[でんわ]が 鳴[な]っているのに 気[き]がつかなかった。	
\\	今、手が離せないの。代わりに電話に出てくれる?	
\\	今[いま]、 手[て]が 離[はな]せないの。 代[か]わりに 電話[でんわ]に 出[で]てくれる?	手が離せない= 
\\	すみません。今手が離せないんです。	
\\	すみません。 今[こん] 手[て]が 離[はな]せないんです。	手が離せない= 
\\	今ちょっと仕事で手が離せないんだ。	
\\	今[いま]ちょっと 仕事[しごと]で 手[て]が 離[はな]せないんだ。	手が離せない= 
\\	昨日からずっとかけていたのだが、ようやく彼女に電話がつながった。	
\\	昨日[きのう]からずっとかけていたのだが、ようやく 彼女[かのじょ]に 電話[でんわ]がつながった。	
\\	山田君から電話があった。	
\\	山田[やまだ] 君[くん]から 電話[でんわ]があった。	
\\	この辺には公衆電話がない。	
\\	この 辺[あたり]には 公衆[こうしゅう] 電話[でんわ]がない。	
\\	留守番電話が応答しました。	
\\	留守番[るすばん] 電話[でんわ]が 応答[おうとう]しました。	留守番電話= 
\\	応答=おうとう= 
\\	留守番電話に任せておく。	
\\	留守番[るすばん] 電話[でんわ]に 任[まか]せておく。	留守番電話= 
\\	留守番電話のメッセージをチェックしました。	
\\	留守番[るすばん] 電話[でんわ]のメッセージをチェックしました。	留守番電話= 
\\	ピザが残ったら、ラップをして冷蔵庫に入れておいてね。	
\\	ピザが 残[のこ]ったら、ラップをして 冷蔵庫[れいぞうこ]に 入[い]れておいてね。	
\\	相手のチームが強過ぎて、全く歯が立たない。	
\\	相手[あいて]のチームが 強[つよ] 過[す]ぎて、 全[まった]く 歯[は]が 立[た]たない。	歯が立たない= 
\\	父が肩たたきにあって、失業した。	
\\	父[ちち]が 肩[かた]たたきにあって、 失業[しつぎょう]した。	肩たたき= 
\\	エアコンがつかないんですが、修理に来てもらえますか。	
\\	エアコンがつかないんですが、 修理[しゅうり]に 来[き]てもらえますか。	
\\	入居するまでには必ずエアコンを修理しておいてください。	
\\	入居[にゅうきょ]するまでには 必[かなら]ずエアコンを 修理[しゅうり]しておいてください。	
\\	どこで鍵をなくしたのか、全く分かりません。	
\\	どこで 鍵[かぎ]をなくしたのか、 全[まった]く 分[わ]かりません。	
\\	部屋を借りる時は、ドアの鍵を替えてもらった方がいい。	
\\	部屋[へや]を 借[か]りる 時[とき]は、ドアの 鍵[かぎ]を 替[か]えてもらった 方[ほう]がいい。	
\\	私のアパートは大きな通りに面しているため、自動車の騒音がひどい。	
\\	私[わたし]のアパートは 大[おお]きな 通[とお]りに 面[めん]しているため、 自動車[じどうしゃ]の 騒音[そうおん]がひどい。	面する= 
\\	騒音=そうおん= 
\\	エアコンとオーブンとドライヤーを同時に使うと、電気のブレーカーが落ちる。	
\\	エアコンとオーブンとドライヤーを 同時[どうじ]に 使[つか]うと、 電気[でんき]のブレーカーが 落[お]ちる。	
\\	オイルヒーターは、思ったより電気を食う。	
\\	オイルヒーターは、 思[おも]ったより 電気[でんき]を 食[く]う。	
\\	天井から雨漏りする。	
\\	天井[てんじょう]から 雨漏[あまも]りする。	
\\	屋根があちこち雨漏りする。	
\\	屋根[やね]があちこち 雨漏[あまも]りする。	
\\	トイレが詰まっちゃったみたいです。	
\\	トイレが 詰[つ]まっちゃったみたいです。	
\\	トイレが詰まりました。	
\\	トイレが 詰[つ]まりました。	
\\	トイレの水が流れないんです。	
\\	トイレの 水[みず]が 流[なが]れないんです。	
\\	トイレの水が流れっぱなしなんです。	
\\	トイレの 水[みず]が 流[なが]れっぱなしなんです。	
\\	昨日からトイレが水漏れしています。	
\\	昨日[きのう]からトイレが 水[みず] 漏[も]れしています。	
\\	映画館を出るころには、雨は上がっていた。	
\\	映画[えいが] 館[かん]を 出[で]るころには、 雨[あめ]は 上[あ]がっていた。	雨が上がる= 
\\	雨が上がって虹が出た。	
\\	雨[あめ]が 上[あ]がって 虹[にじ]が 出[で]た。	雨が上がる= 
\\	私たちは雨が上がるのを待ちました。	
\\	私[わたし]たちは 雨[あめ]が 上[あ]がるのを 待[ま]ちました。	雨が上がる= 
\\	あと少ししたら、雨は止むだろう。	
\\	あと 少[すこ]ししたら、 雨[あめ]は 止[や]むだろう。	
\\	ここ一週間、ずっと雨が続いています。	
\\	ここ一 週間[しゅうかん]、ずっと 雨[あめ]が 続[つづ]いています。	
\\	シアトルは雨が多い。	
\\	シアトルは 雨[あめ]が 多[おお]い。	
\\	雨に洗われた木々の緑が美しい。	
\\	雨[あめ]に 洗[あら]われた 木々[きぎ]の 緑[みどり]が 美[うつく]しい。	
\\	風が強くて、洗濯物が飛ばされてしまった。	
\\	風[かぜ]が 強[つよ]くて、 洗濯[せんたく] 物[もの]が 飛[と]ばされてしまった。	
\\	午後になって、だんだん風が強まってきた。	
\\	午後[ごご]になって、だんだん 風[かぜ]が 強[つよ]まってきた。	
\\	いつも手を洗うたびにあなたのことを思う。	
\\	いつも 手[て]を 洗[あら]うたびにあなたのことを 思[おも]う。	
\\	その後、彼と顔を合わせるたびに心がときめくようになりました。	
\\	その 後[ご]、 彼[かれ]と 顔[かお]を 合[あ]わせるたびに 心[こころ]がときめくようになりました。	ときめく= 
\\	去年の旅行を思い出すたびに、いつも幸せな気分になる。	
\\	去年[きょねん]の 旅行[りょこう]を 思い出[おもいだ]すたびに、いつも 幸[しあわ]せな 気分[きぶん]になる。	
\\	彼の歌をラジオで聞くたびにすごくどきどきする。	
\\	彼[かれ]の 歌[うた]をラジオで 聞[き]くたびにすごくどきどきする。	
\\	彼女はぽっちゃりしています。	
\\	彼女[かのじょ]はぽっちゃりしています。	ぽっちゃり= 
\\	落ち着いたら電話してね。	
\\	落ち着[おちつ]いたら 電話[でんわ]してね。	落ち着く= 
\\	馬は餌をもらったら落ち着くだろう。	
\\	馬[うま]は 餌[えさ]をもらったら 落ち着[おちつ]くだろう。	落ち着く= 
\\	落ち着けよ。	
\\	落ち着[おちつ]けよ。	落ち着く= 
\\	この曲で心を落ち着かせてください。	
\\	この 曲[きょく]で 心[こころ]を 落ち着[おちつ]かせてください。	落ち着く= 
\\	ここにいるのはどうも気分が落ち着かない。	
\\	ここにいるのはどうも 気分[きぶん]が 落ち着[おちつ]かない。	落ち着く= 
\\	この部屋に入ると、いつも僕は落ち着かなくなる。	
\\	この 部屋[へや]に 入[はい]ると、いつも 僕[ぼく]は 落ち着[おちつ]かなくなる。	落ち着く= 
\\	彼は、彼の結婚式の日にまつわる、滑稽な話をしました。	
\\	彼[かれ]は、 彼[かれ]の 結婚式[けっこんしき]の 日[ひ]にまつわる、 滑稽[こっけい]な 話[はなし]をしました。	滑稽=こっけい= 
\\	湿度が低いから、気温が高いわりに快適だ。	
\\	湿度[しつど]が 低[ひく]いから、 気温[きおん]が 高[たか]いわりに 快適[かいてき]だ。	湿度=しつど= 
\\	わりに= 
\\	ここ数日で、ぐっと気温が下がってきた。	
\\	ここ 数[すう] 日[にち]で、ぐっと 気温[きおん]が 下[さ]がってきた。	ぐっと= 
\\	青森では今日、気温が39度に達したらしい。	
\\	青森[あおもり]では 今日[きょう]、 気温[きおん]が39 度[ど]に 達[たっ]したらしい。	「らしい」
\\	最近は気温の変動が激しいので、体調管理が難しい。	
\\	最近[さいきん]は 気温[きおん]の 変動[へんどう]が 激[はげ]しいので、 体調[たいちょう] 管理[かんり]が 難[むずか]しい。	
\\	朝、この辺りはよく霧が出ます。	
\\	朝[あさ]、この 辺[あた]りはよく 霧[きり]が 出[で]ます。	
\\	霧がかかっていたので、残念ながら富士山を見ることはできなかった。	
\\	霧[きり]がかかっていたので、 残念[ざんねん]ながら 富士山[ふじさん]を 見[み]ることはできなかった。	
\\	霧がかかっているね。	
\\	霧[きり]がかかっているね。	
\\	霧が濃くなってきたから、運転には気をつけて。	
\\	霧[きり]が 濃[こ]くなってきたから、 運転[うんてん]には 気[き]をつけて。	
\\	私たちは霧が晴れるのを待ちました。	
\\	私[わたし]たちは 霧[きり]が 晴[は]れるのを 待[ま]ちました。	
\\	昼前には、すっきりと霧が晴れた。	
\\	昼前[ひるまえ]には、すっきりと 霧[きり]が 晴[は]れた。	
\\	あたり一面、霧が立ちこめていた。	
\\	あたり 一面[いちめん]、 霧[きり]が 立[た]ちこめていた。	あたり一面= 
\\	立ちこめる= 
\\	昨日、フィリピンのほうで、台風が発生したらしい。	
\\	昨日[きのう]、フィリピンのほうで、 台風[たいふう]が 発生[はっせい]したらしい。	「らしい」
\\	台風が近づいているから、今日は早く家に帰ったほうがいい。	
\\	台風[たいふう]が 近[ちか]づいているから、 今日[きょう]は 早[はや]く 家[いえ]に 帰[かえ]ったほうがいい。	
\\	台風は今朝、紀伊半島に上陸した。	
\\	台風[たいふう]は 今朝[けさ]、紀伊半島[きいはんとう]に 上陸[じょうりく]した。	紀伊半島=きいはんとう= 
\\	台風が過ぎた後は、晴天になることが多い。	
\\	台風[たいふう]が 過[す]ぎた 後[のち]は、 晴天[せいてん]になることが 多[おお]い。	
\\	今日は太陽が出ていないから、少し肌寒い。	
\\	今日[きょう]は 太陽[たいよう]が 出[で]ていないから、 少[すこ]し 肌寒[はださむ]い。	肌寒い=はだ さむい= 
\\	先週、梅雨に入ったらしい。	
\\	先週[せんしゅう]、 梅雨[つゆ]に 入[はい]ったらしい。	梅雨=つゆ= 
\\	「らしい」
\\	やっと梅雨が明けた。	
\\	やっと 梅雨[つゆ]が 明[あ]けた。	梅雨=つゆ= 
\\	今年はいつもの年より、梅雨が長引いている。	
\\	今年[ことし]はいつもの 年[とし]より、 梅雨[つゆ]が 長引[ながび]いている。	梅雨=つゆ= 
\\	明日、天気になりますように。	
\\	明日[あした]、 天気[てんき]になりますように。	天気になる= 
\\	予報によると、天気は下り坂だそうだ。	
\\	予報[よほう]によると、 天気[てんき]は 下り坂[くだりざか]だそうだ。	下り坂=くだりざか= 
\\	今日は目まぐるしく天気が変わった。	
\\	今日[きょう]は 目[め]まぐるしく 天気[てんき]が 変[か]わった。	目まぐるしい= 
\\	天気がぐずついて、洗濯物がなかなか乾かない。	
\\	天気[てんき]がぐずついて、 洗濯[せんたく] 物[もの]がなかなか 乾[かわ]かない。	ぐずつく= 
\\	山では天気が急変することがあるから、気をつけなさい。	
\\	山[やま]では 天気[てんき]が 急変[きゅうへん]することがあるから、 気[き]をつけなさい。	
\\	なんとか明日まで天気がもってくれるといいのです。	
\\	なんとか 明日[あした]まで 天気[てんき]がもってくれるといいのです。	
\\	今回の旅行は、天気に恵まれて楽しかった。	
\\	今回[こんかい]の 旅行[りょこう]は、 天気[てんき]に 恵[めぐ]まれて 楽[たの]しかった。	
\\	天気予報では、明日は雪になると言っています。	
\\	天気[てんき] 予報[よほう]では、 明日[あした]は 雪[ゆき]になると 言[い]っています。	
\\	ありがたいことに天気予報がはずれて、雨は降らなかった。	
\\	ありがたいことに 天気[てんき] 予報[よほう]がはずれて、 雨[あめ]は 降[ふ]らなかった。	外れる=はずれる= 
\\	この部屋は午前中、よく日が差し込む。	
\\	この 部屋[へや]は 午前[ごぜん] 中[ちゅう]、よく 日[ひ]が 差し込[さしこ]む。	差し込む= 
\\	十分に日が当たるように、植物を窓際に移した。	
\\	十分[じゅうぶん]に 日[ひ]が 当[あ]たるように、 植物[しょくぶつ]を 窓際[まどぎわ]に 移[うつ]した。	窓際=まど ぎわ= 
\\	その本の表紙は日に焼けていて、タイトルさえ読み取れなかった。	
\\	その 本[ほん]の 表紙[ひょうし]は 日[ひ]に 焼[や]けていて、タイトルさえ 読み取[よみと]れなかった。	
\\	日が暮れないうちに、テントを立てなければ。	
\\	日[ひ]が 暮[く]れないうちに、テントを 立[た]てなければ。	日が暮れる= 
\\	だいぶ日が短くなってきた。	
\\	だいぶ 日[ひ]が 短[みじか]くなってきた。	
\\	今日、この冬初めての雪が降った。	
\\	今日[きょう]、この 冬[ふゆ] 初[はじ]めての 雪[ゆき]が 降[ふ]った。	
\\	目が覚めたら、10センチも雪が積もっていた。	
\\	目[め]が 覚[さ]めたら、10センチも 雪[ゆき]が 積[つ]もっていた。	
\\	山頂付近は、すっぽりと雪に覆われている。	
\\	山頂[さんちょう] 付近[ふきん]は、すっぽりと 雪[ゆき]に 覆[おお]われている。	付近=ふきん= 
\\	すっぽり= 
\\	覆う=おおう= 
\\	私の出身地の町では、冬の間雪に閉じ込められる。	
\\	私[わたし]の 出身[しゅっしん] 地[ち]の 町[まち]では、 冬[ふゆ]の 間[ま] 雪[ゆき]に 閉じ込[とじこ]められる。	
\\	今年は例年に比べて、ずいぶん雪が少ない。	
\\	今年[ことし]は 例年[れいねん]に 比[くら]べて、ずいぶん 雪[ゆき]が 少[すく]ない。	
\\	今夜は雪になりそうだ。	
\\	今夜[こんや]は 雪[ゆき]になりそうだ。	
\\	お昼頃に、雨が雪に変わった。	
\\	お 昼[ひる] 頃[ごろ]に、 雨[あめ]が 雪[ゆき]に 変[か]わった。	
\\	80円の切手を10枚、買ってきてもらえる?	
\\	円[えん]の 切手[きって]を10 枚[まい]、 買[か]ってきてもらえる?	
\\	速達で送るのに、いくらの切手を貼ればいいか知っていますか?	
\\	速達[そくたつ]で 送[おく]るのに、いくらの 切手[きって]を 貼[は]ればいいか 知[し]っていますか?	
\\	しまった!今投函したはがきに、切手を貼り忘れた!	
\\	しまった! 今[いま] 投函[とうかん]したはがきに、 切手[きって]を 貼[は]り 忘[わす]れた!	投函=とうかん= 
\\	コンビニでも切手を売っていますよ。	
\\	コンビニでも 切手[きって]を 売[う]っていますよ。	
\\	父はここ20年ほど、切手を集めています。	
\\	父[ちち]はここ20 年[ねん]ほど、 切手[きって]を 集[あつ]めています。	
\\	アメリカに小包を送りたいのですが。	
\\	アメリカに 小包[こづつみ]を 送[おく]りたいのですが。	小包=こ づつみ= 
\\	あなた宛に小包が届いています。	
\\	あなた 宛[あて]に 小包[こづつみ]が 届[とど]いています。	小包=こ づつみ= 
\\	あなた宛の荷物が今日の午後届いた。	
\\	あなた 宛[あて]の 荷物[にもつ]が 今日[きょう]の 午後[ごご] 届[とど]いた。	
\\	あなた宛の荷物を預かっています。	
\\	あなた 宛[あて]の 荷物[にもつ]を 預[あず]かっています。	
\\	この小包はだれ宛のものですか?	
\\	この 小包[こづつみ]はだれ 宛[あて]のものですか?	小包=こ づつみ= 
\\	京都の姉から小包が届いた。	
\\	京都[きょうと]の 姉[あね]から 小包[こづつみ]が 届[とど]いた。	
\\	途中でこの手紙を出してもらえる?	
\\	途中[とちゅう]でこの 手紙[てがみ]を 出[だ]してもらえる?	
\\	英語を教えた生徒から、手紙をもらいました。	
\\	英語[えいご]を 教[おし]えた 生徒[せいと]から、 手紙[てがみ]をもらいました。	
\\	旅先で知り合った友人に、手紙を書いた。	
\\	旅先[たびさき]で 知り合[しりあ]った 友人[ゆうじん]に、 手紙[てがみ]を 書[か]いた。	
\\	人の手紙を読んじゃだめだよ。	
\\	人[ひと]の 手紙[てがみ]を 読[よ]んじゃだめだよ。	
\\	私宛に郵便は届いていませんか?	
\\	私[わたし] 宛[あて]に 郵便[ゆうびん]は 届[とど]いていませんか?	
\\	うちには普通、午前11時ごろ郵便が来る。	
\\	うちには 普通[ふつう]、 午前[ごぜん]11 時[じ]ごろ 郵便[ゆうびん]が 来[く]る。	
\\	コンビニに寄って
\\	で一万円下ろした。	
\\	コンビニに 寄[よ]って 
\\	で一 万[まん] 円[えん] 下[お]ろした。	
\\	カードを忘れてきたので、お金を引き出せない。	
\\	カードを 忘[わす]れてきたので、お 金[かね]を 引き出[ひきだ]せない。	
\\	日本中を旅行して回るために、お金を貯めています。	
\\	日本[にっぽん] 中[ちゅう]を 旅行[りょこう]して 回[まわ]るために、お 金[かね]を 貯[た]めています。	
\\	クレジットで新しいパソコンを買った。	
\\	クレジットで 新[あたら]しいパソコンを 買[か]った。	
\\	公共料金の多くは、クレジットカードで払うことができます。	
\\	公共[こうきょう] 料金[りょうきん]の 多[おお]くは、クレジットカードで 払[はら]うことができます。	公共料金=こうきょう りょうきん= 
\\	ほとんどの人は、複数の口座をもっている。	
\\	ほとんどの 人[ひと]は、 複数[ふくすう]の 口座[こうざ]をもっている。	
\\	もう使わなくなった口座を解約した。	
\\	もう 使[つか]わなくなった 口座[こうざ]を 解約[かいやく]した。	
\\	スポーツクラブの会費は、口座引き落としになっている。	
\\	スポーツクラブの 会費[かいひ]は、 口座[こうざ] 引き落[ひきお]としになっている。	
\\	私の普通預金口座に振り込みたいのですが。	
\\	私[わたし]の 普通[ふつう] 預金[よきん] 口座[こうざ]に 振り込[ふりこ]みたいのですが。	
\\	給料が口座に振り込まれた。	
\\	給料[きゅうりょう]が 口座[こうざ]に 振り込[ふりこ]まれた。	
\\	この金額には、手数料は含まれていますか?	
\\	この 金額[きんがく]には、 手数料[てすうりょう]は 含[ふく]まれていますか?	手数料=てすうりょう= 
\\	今、預金が100万円ほどあります。	
\\	今[いま]、 預金[よきん]が100 万[まん] 円[えん]ほどあります。	
\\	お金を払うときになって、財布を忘れてきたことに気がついた。	
\\	お 金[かね]を 払[はら]うときになって、 財布[さいふ]を 忘[わす]れてきたことに 気[き]がついた。	
\\	お金が足りなくて、最新型の掃除機は買えなかった。	
\\	お 金[かね]が 足[た]りなくて、 最新[さいしん] 型[がた]の 掃除[そうじ] 機[き]は 買[か]えなかった。	
\\	スーパーのレジで、お釣りを間違えられた。	
\\	スーパーのレジで、お 釣[つ]りを 間違[まちが]えられた。	
\\	バスに乗る時は、お釣りがいらないように小銭を準備するといい。	
\\	バスに 乗[の]る 時[とき]は、お 釣[つ]りがいらないように 小銭[こぜに]を 準備[じゅんび]するといい。	
\\	日曜日にはよく渋谷に買い物に行きます。	
\\	日曜日[にちようび]にはよく 渋谷[しぶや]に 買い物[かいもの]に 行[い]きます。	
\\	ミキは買い物に出かけていて、留守です。	
\\	ミキは 買い物[かいもの]に 出[で]かけていて、 留守[るす]です。	
\\	買い物から戻るとすぐに、夕食の準備をした。	
\\	買い物[かいもの]から 戻[もど]るとすぐに、 夕食[ゆうしょく]の 準備[じゅんび]をした。	
\\	クリスマスの買い物はもう済ませましたか?	
\\	クリスマスの 買い物[かいもの]はもう 済[す]ませましたか?	済ませる= 
\\	この絵は少なくとも50万円の価値がある。	
\\	この 絵[え]は 少[すく]なくとも50 万[まん] 円[えん]の 価値[かち]がある。	
\\	この本には金を払っただけの価値がない。	
\\	この 本[ほん]には 金[きん]を 払[はら]っただけの 価値[かち]がない。	
\\	そのサインも、あと10年くらいたてば価値が出るかもしれない。	
\\	そのサインも、あと10 年[ねん]くらいたてば 価値[かち]が 出[で]るかもしれない。	
\\	いつか価値が出るので、この本は大事に持っていなさい。	
\\	いつか 価値[かち]が 出[で]るので、この 本[ほん]は 大事[だいじ]に 持[も]っていなさい。	
\\	彼はいつも財布をズボンの尻ポケットに入れて持ち歩いている。	
\\	彼[かれ]はいつも 財布[さいふ]をズボンの 尻[しり]ポケットに 入[い]れて 持ち歩[もちある]いている。	
\\	いつどこで財布をなくしたのか、見当がつかない。	
\\	いつどこで 財布[さいふ]をなくしたのか、 見当[けんとう]がつかない。	見当がつかない= 
\\	彼が何を考えているのか見当がつかない。	
\\	彼[かれ]が 何[なに]を 考[かんが]えているのか 見当[けんとう]がつかない。	見当がつかない= 
\\	私はその問題の答が何なのか全く見当がつかなかった。	
\\	私[わたし]はその 問題[もんだい]の 答[こたえ]が 何[なに]なのか 全[まった]く 見当[けんとう]がつかなかった。	見当がつかない= 
\\	家から駅までの間で、財布を落としたに違いない。	
\\	家[いえ]から 駅[えき]までの 間[ま]で、 財布[さいふ]を 落[お]としたに 違[ちが]いない。	
\\	電車の中で、財布を盗まれた。	
\\	電車[でんしゃ]の 中[なか]で、 財布[さいふ]を 盗[ぬす]まれた。	
\\	バスの中でお金をすられました。	
\\	バスの 中[なか]でお 金[かね]をすられました。	
\\	私が店に行ったときには、棚にはわずかな商品しか並んでいなかった。	
\\	私[わたし]が 店[みせ]に 行[い]ったときには、 棚[たな]にはわずかな 商品[しょうひん]しか 並[なら]んでいなかった。	
\\	何軒かの店を回ったが、どこもその商品を扱っていなかった。	
\\	何[なん] 軒[けん]かの 店[みせ]を 回[まわ]ったが、どこもその 商品[しょうひん]を 扱[あつか]っていなかった。	
\\	週に一回スーパーに行って、1週間分の食料を買います。	
\\	週[しゅう]に一 回[かい]スーパーに 行[い]って、1 週間[しゅうかん] 分[ぶん]の 食料[しょくりょう]を 買[か]います。	
\\	帰りにスーパーに寄って、牛乳を買ってきて。	
\\	帰[かえ]りにスーパーに 寄[よ]って、 牛乳[ぎゅうにゅう]を 買[か]ってきて。	
\\	品質のいいものは、やはり値段が高い。	
\\	品質[ひんしつ]のいいものは、やはり 値段[ねだん]が 高[たか]い。	
\\	ガソリンの値段が上がり続いている。	
\\	ガソリンの 値段[ねだん]が 上[あ]がり 続[つづ]いている。	
\\	これらのアンティークは、値段がつけられないほど貴重だ。	
\\	これらのアンティークは、 値段[ねだん]がつけられないほど 貴重[きちょう]だ。	
\\	ロンドンは物価が高い。	
\\	ロンドンは 物価[ぶっか]が 高[たか]い。	
\\	ここ一年でずいぶん物価が下がった。	
\\	ここ一 年[ねん]でずいぶん 物価[ぶっか]が 下[さ]がった。	
\\	いくつか部品を買おうと思ったら、店が閉まっていて買えなかった。	
\\	いくつか 部品[ぶひん]を 買[か]おうと 思[おも]ったら、 店[みせ]が 閉[し]まっていて 買[か]えなかった。	
\\	ペットを飼いたいなら、店で買う以外にも選択肢はある。	
\\	ペットを 飼[か]いたいなら、 店[みせ]で 買[か]う 以外[いがい]にも 選択肢[せんたくし]はある。	選択肢=せん たく し= 
\\	友人に贈るプレゼントを探して、いろいろな店を見て回った。	
\\	友人[ゆうじん]に 贈[おく]るプレゼントを 探[さが]して、いろいろな 店[みせ]を 見[み]て 回[まわ]った。	
\\	大勢の客がレジに並んでいた。	
\\	大勢[おおぜい]の 客[きゃく]がレジに 並[なら]んでいた。	
\\	この時間帯は、いつもレジが混む。	
\\	この 時間[じかん] 帯[たい]は、いつもレジが 混[こ]む。	
\\	日本語は発音と文字が完全に一致しています。	
\\	日本語[にほんご]は 発音[はつおん]と 文字[もじ]が 完全[かんぜん]に 一致[いっち]しています。	一致= 
\\	これに関しては意見は一致しているかい?	
\\	これに 関[かん]しては 意見[いけん]は 一致[いっち]しているかい?	一致= 
\\	その二つのサインを比べてみたらぴったり一致しました。	
\\	その 二[ふた]つのサインを 比[くら]べてみたらぴったり 一致[いっち]しました。	一致= 
\\	キムさんは韓国から拉致された。	
\\	キムさんは 韓国[かんこく]から 拉致[らち]された。	拉致=らち= 
\\	理論と実践とは必ずしも合致しない。	
\\	理論[りろん]と 実践[じっせん]とは 必[かなら]ずしも 合致[がっち]しない。	合致=がっち= 
\\	残念なことに、この病気は例外なく致死性です。	
\\	残念[ざんねん]なことに、この 病気[びょうき]は 例外[れいがい]なく 致死[ちし] 性[せい]です。	致死(性)=ちし(せい)= 
\\	触れたくない話題なら、無理に話す必要ありません。	
\\	触[ふ]れたくない 話題[わだい]なら、 無理[むり]に 話[はな]す 必要[ひつよう]ありません。	触れる= 
\\	おずおずと私は手を伸ばし、彼のほおに触れました。	
\\	おずおずと 私[わたし]は 手[て]を 伸[の]ばし、 彼[かれ]のほおに 触[ふ]れました。	触れる= 
\\	これらの爆発性の物質に触れないでください。	
\\	これらの 爆発[ばくはつ] 性[せい]の 物質[ぶっしつ]に 触[ふ]れないでください。	触れる= 
\\	その新聞は、彼の死因の詳細については触れていなかった。	
\\	その 新聞[しんぶん]は、 彼[かれ]の 死因[しいん]の 詳細[しょうさい]については 触[ふ]れていなかった。	触れる= 
\\	ピアノに手を触れないでください。	
\\	ピアノに 手[て]を 触[ふ]れないでください。	触れる= 
\\	作品には手を触れないでください。	
\\	作品[さくひん]には 手[て]を 触[ふ]れないでください。	
\\	彼はおずおずとほほ笑みました。	
\\	彼[かれ]はおずおずとほほ 笑[え]みました。	おずおず= 
\\	ほほ笑む=ほほえむ= 
\\	それに触っていいと言ったか?	
\\	それに 触[さわ]っていいと 言[い]ったか?	触る=さわる= 
\\	自分の物でもないのに触ってはいけませんよ。	
\\	自分[じぶん]の 物[もの]でもないのに 触[さわ]ってはいけませんよ。	触る=さわる= 
\\	彼は彼女を触るのをやめなかった。	
\\	彼[かれ]は 彼女[かのじょ]を 触[さわ]るのをやめなかった。	触る=さわる= 
\\	彼は彼女の情緒的な欠点を見過ごした。	
\\	彼[かれ]は 彼女[かのじょ]の 情緒[じょうちょ] 的[てき]な 欠点[けってん]を 見過[みす]ごした。	情緒=じょうちょ= 
\\	異国情緒あふれる雰囲気の都市である。	
\\	異国[いこく] 情緒[じょうちょ]あふれる 雰囲気[ふんいき]の 都市[とし]である。	異国情緒=いこく じょうちょ= 
\\	内緒にしておいた方がいいのかな?	
\\	内緒[ないしょ]にしておいた 方[ほう]がいいのかな?	
\\	内緒にしておいてね。	
\\	内緒[ないしょ]にしておいてね。	
\\	私がここに来ているのは彼には内緒なんです。	
\\	私[わたし]がここに 来[き]ているのは 彼[かれ]には 内緒[ないしょ]なんです。	
\\	私が言ったってことは内緒ですよ。	
\\	私[わたし]が 言[い]ったってことは 内緒[ないしょ]ですよ。	
\\	柱に犬をつないだ。	
\\	柱[はしら]に 犬[いぬ]をつないだ。	柱=はしら= 
\\	コメは日本農業の一番大きな柱です。	
\\	コメは 日本[にっぽん] 農業[のうぎょう]の 一番[いちばん] 大[おお]きな 柱[はしら]です。	柱=はしら= 
\\	その車はスリップして電柱に激突しました。	
\\	その 車[くるま]はスリップして 電柱[でんちゅう]に 激突[げきとつ]しました。	電柱=でんちゅう= 
\\	激突=げきとつ= 
\\	日本では、ポルノ雑誌を堂々と見ている男性がたくさんいますよ。	
\\	日本[にっぽん]では、ポルノ 雑誌[ざっし]を 堂々[どうどう]と 見[み]ている 男性[だんせい]がたくさんいますよ。	堂々=どうどう= 
\\	と)
\\	彼は本当に正々堂々としたやつだ。	
\\	彼[かれ]は 本当[ほんとう]に 正々堂々[せいせいどうどう]としたやつだ。	正々堂々=せいせい どうどう= 
\\	彼が二枚目なのはみんなが認めるところだ。	
\\	彼[かれ]が 二枚目[にまいめ]なのはみんなが 認[みと]めるところだ。	二枚目=にまいめ= 
\\	彼は現行犯で捕まりました。	
\\	彼[かれ]は 現行[げんこう] 犯[はん]で 捕[つか]まりました。	現行犯で=げん こう はん で= 
\\	レジ係はわずかなお金を盗んだところを現行犯逮捕された。	
\\	レジ 係[がかり]はわずかなお 金[かね]を 盗[ぬす]んだところを 現行[げんこう] 犯[はん] 逮捕[たいほ]された。	現行犯で=げん こう はん で= 
\\	日常生活への応用のリストとなると枚挙にいとまがないほどである。	
\\	日常[にちじょう] 生活[せいかつ]への 応用[おうよう]のリストとなると 枚挙[まいきょ]にいとまがないほどである。	枚挙にいとまがない= 
\\	この部屋は居間を兼ねています。	
\\	この 部屋[へや]は 居間[いま]を 兼[か]ねています。	兼ねる= 
\\	長い出張でしたので、仕事の他に、遊びも兼ねることができました。	
\\	長[なが]い 出張[しゅっちょう]でしたので、 仕事[しごと]の 他[ほか]に、 遊[あそ]びも 兼[か]ねることができました。	兼ねる= 
\\	この帽子は、男女兼用です。	
\\	この 帽子[ぼうし]は、 男女[だんじょ] 兼用[けんよう]です。	兼用=けんよう= 
\\	気兼ねせずに何でも言ってください。	
\\	気兼[きが]ねせずに 何[なに]でも 言[い]ってください。	気兼ね=きがね= 
\\	オリーブ油は料理以外の用途もある。	
\\	オリーブ油[おりーぶゆ]は 料理[りょうり] 以外[いがい]の 用途[ようと]もある。	用途=ようと= 
\\	言葉は途切れた。	
\\	言葉[ことば]は 途切[とぎ]れた。	途切れる=とぎれる= 
\\	これからどうしたらよいのか途方に暮れている。	
\\	これからどうしたらよいのか 途方[とほう]に 暮[く]れている。	途方に暮れる= 
\\	彼は何をすべきか分からず、途方に暮れているようだった。	
\\	彼[かれ]は 何[なに]をすべきか 分[わ]からず、 途方[とほう]に 暮[く]れているようだった。	途方に暮れる= 
\\	彼は宝くじで途方もない額のお金を得た。	
\\	彼[かれ]は 宝[たから]くじで 途方[とほう]もない 額[がく]のお 金[かね]を 得[え]た。	途方もない= 
\\	先生は生卵をご飯に混ぜて食べ始めました。	
\\	先生[せんせい]は 生[なま] 卵[たまご]をご 飯[はん]に 混[ま]ぜて 食[た]べ 始[はじ]めました。	
\\	ひき肉とジャガイモとタマネギを混ぜた。	
\\	ひき 肉[にく]とジャガイモとタマネギを 混[ま]ぜた。	
\\	円とドルを混ぜて払ってもいいですか?	
\\	円[えん]とドルを 混[ま]ぜて 払[はら]ってもいいですか?	
\\	よくかき混ぜてお召し上がりください。	
\\	よくかき 混[ま]ぜてお 召し上[めしあ]がりください。	
\\	京都には伝統と目新しさが混在している。	
\\	京都[きょうと]には 伝統[でんとう]と 目新[めあたら]しさが 混在[こんざい]している。	目新しい=めあたらしい= 
\\	混在= 
\\	君の作品にはあまり目新しさがない。	
\\	君[きみ]の 作品[さくひん]にはあまり 目新[めあたら]しさがない。	目新しい=めあたらしい= 
\\	ここはいつ来ても混んでいることはありません。	
\\	ここはいつ 来[き]ても 混[こ]んでいることはありません。	
\\	その中はかなり混んでいる。	
\\	その 中[なか]はかなり 混[こ]んでいる。	
\\	激痛がします。	
\\	激痛[げきつう]がします。	激痛=げき つう= 
\\	子供たちは食欲旺盛です。	
\\	子供[こども]たちは 食欲[しょくよく] 旺盛[おうせい]です。	旺盛=おうせい= 
\\	私は好奇心が旺盛なんですね。	
\\	私[わたし]は 好奇[こうき] 心[しん]が 旺盛[おうせい]なんですね。	旺盛=おうせい= 
\\	しょっちゅう金欠になる。	
\\	しょっちゅう 金欠[きんけつ]になる。	しょっちゅう= 
\\	金欠=きんけつ= 
\\	この国では外国人に部屋を貸さないというのはしょっちゅうあることだ。	
\\	この 国[くに]では 外国[がいこく] 人[じん]に 部屋[へや]を 貸[か]さないというのはしょっちゅうあることだ。	しょっちゅう= 
\\	そんなにしょっちゅう起きることではない。	
\\	そんなにしょっちゅう 起[お]きることではない。	しょっちゅう= 
\\	見知らぬ世界で裸になった気分でした。	
\\	見知[みし]らぬ 世界[せかい]で 裸[はだか]になった 気分[きぶん]でした。	見知らぬ= 
\\	彼女は、見知らぬ人に話し掛けることを嫌がらない。	
\\	彼女[かのじょ]は、 見知[みし]らぬ 人[ひと]に 話し掛[はなしか]けることを 嫌[いや]がらない。	見知らぬ= 
\\	彼は結婚式で満面の笑みを浮かべていた。	
\\	彼[かれ]は 結婚式[けっこんしき]で 満面[まんめん]の 笑[え]みを 浮[う]かべていた。	満面=まんめん= 
\\	彼女は彼に満面の笑みを投げ掛けた。	
\\	彼女[かのじょ]は 彼[かれ]に 満面[まんめん]の 笑[え]みを 投げ掛[なげか]けた。	満面=まんめん= 
\\	投げ掛ける= 
\\	私たちは全員、今日のおすすめランチを注文した。	
\\	私[わたし]たちは 全員[ぜんいん]、 今日[きょう]のおすすめランチを 注文[ちゅうもん]した。	
\\	若いウェイターが注文を取りに来た。	
\\	若[わか]いウェイターが 注文[ちゅうもん]を 取[と]りに 来[き]た。	
\\	デザートを頼んだのですが、食べられそうにありません。注文を取り消せますか。	
\\	デザートを 頼[たの]んだのですが、 食[た]べられそうにありません。 注文[ちゅうもん]を 取り消[とりけ]せますか。	
\\	すみません、メニューをもらえますか?	
\\	すみません、メニューをもらえますか?	
\\	メニューを見ても、どんな料理なのかさっぱり分からなかった。	
\\	メニューを 見[み]ても、どんな 料理[りょうり]なのかさっぱり 分[わ]からなかった。	
\\	整形手術をしたの?	
\\	整形[せいけい] 手術[しゅじゅつ]をしたの?	
\\	私は日本に来てから至る所でたばこの煙を浴びています。	
\\	私[わたし]は 日本[にっぽん]に 来[き]てから 至[いた]る 所[ところ]でたばこの 煙[けむり]を 浴[あ]びています。	至る所= 
\\	コンピューターは教育の至る所に普及した。	
\\	コンピューターは 教育[きょういく]の 至[いた]る 所[ところ]に 普及[ふきゅう]した。	至る所= 
\\	今日では日本の至る所で外国人労働者の姿を見掛けることができる。	
\\	今日[きょう]では 日本[にっぽん]の 至[いた]る 所[ところ]で 外国[がいこく] 人[じん] 労働[ろうどう] 者[しゃ]の 姿[すがた]を 見掛[みか]けることができる。	至る所= 
\\	私はボブとトムとパリで合流しました。	
\\	私[わたし]はボブとトムとパリで 合流[ごうりゅう]しました。	合流=ごうりゅう= 
\\	私は友人たちに合流して、街をブラブラしました。	
\\	私[わたし]は 友人[ゆうじん]たちに 合流[ごうりゅう]して、 街[まち]をブラブラしました。	合流=ごうりゅう= 
\\	以前は現役の間はずっと同じところで働き続けるのが普通でした。	
\\	以前[いぜん]は 現役[げんえき]の 間[ま]はずっと 同[おな]じところで 働[はたら]き 続[つづ]けるのが 普通[ふつう]でした。	現役=げんえき= 
\\	私のピアノの先生は、現役のミュージシャンだ。	
\\	私[わたし]のピアノの 先生[せんせい]は、 現役[げんえき]のミュージシャンだ。	現役=げんえき= 
\\	私たちの結婚は破綻しました。	
\\	私[わたし]たちの 結婚[けっこん]は 破綻[はたん]しました。	破綻=はたん= 
\\	ホテルにチェックインした後、町に出てレストランを探した。	
\\	ホテルにチェックインした 後[のち]、 町[まち]に 出[で]てレストランを 探[さが]した。	
\\	今夜七時に、レストランを予約してあります。	
\\	今夜[こんや] 七時[しちじ]に、レストランを 予約[よやく]してあります。	
\\	彼は私たちと夕食をとる。	
\\	彼[かれ]は 私[わたし]たちと 夕食[ゆうしょく]をとる。	
\\	車にガソリンを入れないといけないから、少し早めに出ましょう。	
\\	車[くるま]にガソリンを 入[い]れないといけないから、 少[すこ]し 早[はや]めに 出[で]ましょう。	
\\	この車はガソリンを食うので、エコカーに買い替えるつもりだ。	
\\	この 車[くるま]はガソリンを 食[く]うので、エコカーに 買い替[かいか]えるつもりだ。	
\\	自動券売機で切符を買う方法が分かりません。	
\\	自動[じどう] 券売[けんばい] 機[き]で 切符[きっぷ]を 買[か]う 方法[ほうほう]が 分[わ]かりません。	
\\	出張がキャンセルになったので、切符を払い戻してもらった。	
\\	出張[しゅっちょう]がキャンセルになったので、 切符[きっぷ]を 払い戻[はらいもど]してもらった。	
\\	免許を持っていますが、ふだんは車を運転しません。	
\\	免許[めんきょ]を 持[も]っていますが、ふだんは 車[くるま]を 運転[うんてん]しません。	
\\	全社員のうち半分は、車で通勤しています。	
\\	全[ぜん] 社員[しゃいん]のうち 半分[はんぶん]は、 車[くるま]で 通勤[つうきん]しています。	
\\	車は駅の近くの駐車場に、停めてあります。	
\\	車[くるま]は 駅[えき]の 近[ちか]くの 駐車[ちゅうしゃ] 場[じょう]に、 停[と]めてあります。	
\\	タイヤがパンクしました。	
\\	タイヤがパンクしました。	
\\	家に帰る途中でタイヤがパンクした。	
\\	家[いえ]に 帰[かえ]る 途中[とちゅう]でタイヤがパンクした。	
\\	電車では、いつも高齢者に席をゆずるようにしている。	
\\	電車[でんしゃ]では、いつも 高齢[こうれい] 者[しゃ]に 席[せき]をゆずるようにしている。	
\\	あなたのために席を取っておいた。	
\\	あなたのために 席[せき]を 取[と]っておいた。	
\\	特に知りたいことがあったら教えてください。	
\\	特[とく]に 知[し]りたいことがあったら 教[おし]えてください。	
\\	あなたの席を取っておきましたよ。	
\\	あなたの 席[せき]を 取[と]っておきましたよ。	
\\	窓側に座っていた人が、席を代わってくれた。	
\\	窓側[まどがわ]に 座[すわ]っていた 人[ひと]が、 席[せき]を 代[か]わってくれた。	
\\	渋滞に巻き込まれてしまい、予定より1時間も遅れて目的地に着いた。	
\\	渋滞[じゅうたい]に 巻き込[まきこ]まれてしまい、 予定[よてい]より1 時間[じかん]も 遅[おく]れて 目的[もくてき] 地[ち]に 着[つ]いた。	渋滞=じゅうたい= 
\\	渋滞を避けるために別の道を選んだが、たいして変わらなかった。	
\\	渋滞[じゅうたい]を 避[さ]けるために 別[べつ]の 道[みち]を 選[えら]んだが、たいして 変[か]わらなかった。	渋滞=じゅうたい= 
\\	大して=たいして= 
\\	渋滞にはまって、6時間も車の中に閉じ込められた。	
\\	渋滞[じゅうたい]にはまって、6 時間[じかん]も 車[くるま]の 中[なか]に 閉じ込[とじこ]められた。	渋滞=じゅうたい= 
\\	定期を忘れたので、切符を買わなければならない。	
\\	定期[ていき]を 忘[わす]れたので、 切符[きっぷ]を 買[か]わなければならない。	定期= 
\\	新しく買ったばかりだったのに、不注意にも定期をなくしてしまった。	
\\	新[あたら]しく 買[か]ったばかりだったのに、 不注意[ふちゅうい]にも 定期[ていき]をなくしてしまった。	定期= 
\\	あと1週間で定期が切れる。	
\\	あと1 週間[しゅうかん]で 定期[ていき]が 切[き]れる。	定期= 
\\	その店の前には、客の長い列ができていた。	
\\	その 店[みせ]の 前[まえ]には、 客[きゃく]の 長[なが]い 列[れつ]ができていた。	
\\	列に並んでまで、そのドーナツを買いたいとは思わない。	
\\	列[れつ]に 並[なら]んでまで、そのドーナツを 買[か]いたいとは 思[おも]わない。	
\\	中年のおばさん二人組が、平気で列に割り込んできた。	
\\	中年[ちゅうねん]のおばさん 二人組[ふたりぐみ]が、 平気[へいき]で 列[れつ]に 割り込[わりこ]んできた。	
\\	列に割り込まないでください。	
\\	列[れつ]に 割り込[わりこ]まないでください。	
\\	列はとても長く、私のところからは、列の最後尾が見えない。	
\\	列[れつ]はとても 長[なが]く、 私[わたし]のところからは、 列[れつ]の 最[さい] 後尾[こうび]が 見[み]えない。	
\\	私はその文化が死に絶えるとは思いません。	
\\	私[わたし]はその 文化[ぶんか]が 死に絶[しにた]えるとは 思[おも]いません。	
\\	相変わらずぶしつけなやつだ。	
\\	相変[あいか]わらずぶしつけなやつだ。	相変わらず= 
\\	ぶしつけ= 
\\	カタカナが相変わらず使われているのはなぜでしょうか?	
\\	カタカナが 相変[あいか]わらず 使[つか]われているのはなぜでしょうか?	相変わらず= 
\\	本は相変わらず大切な役割を果たしている。	
\\	本[ほん]は 相変[あいか]わらず 大切[たいせつ]な 役割[やくわり]を 果[は]たしている。	相変わらず= 
\\	私は相変わらず内気だが、昔に比べればずっとましだ。	
\\	私[わたし]は 相変[あいか]わらず 内気[うちき]だが、 昔[むかし]に 比[くら]べればずっとましだ。	相変わらず= 
\\	内気=うちき= 
\\	まし= 
\\	しばらく待って様子を見るしかない。	
\\	しばらく 待[ま]って 様子[ようす]を 見[み]るしかない。	
\\	気合いを入れてもう一度。	
\\	気合[きあ]いを 入[い]れてもう 一度[いちど]。	
\\	日本人は普段おとなしい。	
\\	日本人[にっぽんじん]は 普段[ふだん]おとなしい。	大人しい=おとなしい= 
\\	ミックは教室では生意気だが、家ではとても大人しい。	
\\	ミックは 教室[きょうしつ]では 生意気[なまいき]だが、 家[いえ]ではとても 大人[おとな]しい。	大人しい=おとなしい= 
\\	生意気=なまいき= 
\\	彼女はやかましい姉に比べてとても大人しい。	
\\	彼女[かのじょ]はやかましい 姉[あね]に 比[くら]べてとても 大人[おとな]しい。	大人しい=おとなしい= 
\\	やかましい= 
\\	彼は無口なたちだ。	
\\	彼[かれ]は 無口[むくち]なたちだ。	質=たち= 
\\	あなたは努力家で、挑戦を受けるのを恐れません。	
\\	あなたは 努力[どりょく] 家[か]で、 挑戦[ちょうせん]を 受[う]けるのを 恐[おそ]れません。	努力家=どりょくか= 
\\	在日外国人ビジネスマンは結構努力家だ。	
\\	在日[ざいにち] 外国[がいこく] 人[じん]ビジネスマンは 結構[けっこう] 努力[どりょく] 家[か]だ。	努力家=どりょくか= 
\\	あなたは、食べ物について神経質ですか?	
\\	あなたは、 食べ物[たべもの]について 神経質[しんけいしつ]ですか?	
\\	彼は神経質なので、すぐにイライラする。	
\\	彼[かれ]は 神経質[しんけいしつ]なので、すぐにイライラする。	
\\	彼は厳格に見えるかもしれないが、実はとても思いやりがある。	
\\	彼[かれ]は 厳格[げんかく]に 見[み]えるかもしれないが、 実[じつ]はとても 思[おも]いやりがある。	思いやりがある= 
\\	私は日本人は大変思いやりがあると感じます。	
\\	私[わたし]は 日本人[にっぽんじん]は 大変[たいへん] 思[おも]いやりがあると 感[かん]じます。	思いやりがある= 
\\	あなたの負けず嫌いの態度がけんかを引き起こすかも。	
\\	あなたの 負けず嫌[まけずぎら]いの 態度[たいど]がけんかを 引き起[ひきお]こすかも。	負けず嫌い= 
\\	ワインを何杯か飲んだ後は、彼はいつも社交的になる。	
\\	ワインを 何[なん] 杯[はい]か 飲[の]んだ 後[のち]は、 彼[かれ]はいつも 社交[しゃこう] 的[てき]になる。	社交的= 
\\	彼女は非常に社交的な性格をしている。	
\\	彼女[かのじょ]は 非常[ひじょう]に 社交[しゃこう] 的[てき]な 性格[せいかく]をしている。	社交的= 
\\	私はあまり社交的な性格ではありません。	
\\	私[わたし]はあまり 社交[しゃこう] 的[てき]な 性格[せいかく]ではありません。	社交的= 
\\	日本人というのは、どうしてこんなに気が短いのだろう?	
\\	日本人[にっぽんじん]というのは、どうしてこんなに 気[き]が 短[みじか]いのだろう?	
\\	それは前向きな徴候です。	
\\	それは 前向[まえむ]きな 兆候[ちょうこう]です。	前向き= 
\\	私は自分の将来を前向きに見ている。	
\\	私[わたし]は 自分[じぶん]の 将来[しょうらい]を 前向[まえむ]きに 見[み]ている。	前向き= 
\\	男と女は結婚しても和気あいあいと暮らせるの?	
\\	男[おとこ]と 女[おんな]は 結婚[けっこん]しても 和気[わき]あいあいと 暮[く]らせるの?	和気あいあい= 
\\	これからの社員は、英語とパソコンを駆使できなければ仕事ができません。	
\\	これからの 社員[しゃいん]は、 英語[えいご]とパソコンを 駆使[くし]できなければ 仕事[しごと]ができません。	駆使=くし= 
\\	彼は人見知りをする。	
\\	彼[かれ]は 人見知[ひとみし]りをする。	
\\	泥酔するか、お金がなくなるまで飲むか、あるいはその両方。	
\\	泥酔[でいすい]するか、お 金[かね]がなくなるまで 飲[の]むか、あるいはその 両方[りょうほう]。	泥酔=でいすい= 
\\	彼は泥酔していました。	
\\	彼[かれ]は 泥酔[でいすい]していました。	泥酔=でいすい= 
\\	彼は大酒を飲んで泥酔しました。	
\\	彼[かれ]は 大酒[おおざけ]を 飲[の]んで 泥酔[でいすい]しました。	大酒=おおざけ= 
\\	泥酔=でいすい= 
\\	これらの殺人事件の大半は未解決のままです。	
\\	これらの 殺人[さつじん] 事件[じけん]の 大半[たいはん]は 未[み] 解決[かいけつ]のままです。	大半=たいはん= 
\\	その船の積荷の大半は回収された。	
\\	その 船[ふね]の 積荷[つみに]の 大半[たいはん]は 回収[かいしゅう]された。	大半=たいはん= 
\\	積荷=つみに= 
\\	回収=かいしゅう= 
\\	ビールは麦から造られます。	
\\	ビールは 麦[むぎ]から 造[つく]られます。	
\\	昔日本の家は木で造られました。	
\\	昔[むかし] 日本[にっぽん]の 家[いえ]は 木[き]で 造[つく]られました。	
\\	私は犬に噛まれました。	
\\	私[わたし]は 犬[いぬ]に 噛[か]まれました。	
\\	日本の車は世界中へ輸出されています。	
\\	日本の 車[くるま]は 世界中[せかいじゅう]へ 輸出[ゆしゅつ]されています。	
\\	会議は神戸で開かれました。	
\\	会議[かいぎ]は 神戸[こうべ]で 開[ひら]かれました。	
\\	電話はベルによって発明されました。	
\\	電話[でんわ]はベルによって 発明[はつめい]されました。	
\\	東京の人は歩くのが速いです。	
\\	東京[とうきょう]の 人[ひと]は 歩[ある]くのが 速[はや]いです。	
\\	鈴木さんが来月結婚するのを知っていますか。	
\\	鈴木[すずき]さんが 来月[らいげつ] 結婚[けっこん]するのを 知[し]っていますか。	
\\	木村さんに赤ちゃんが生まれたのを知っていますか。	
\\	木村[きむら]さんに 赤[あか]ちゃんが 生[う]まれたのを 知[し]っていますか。	
\\	家族に会えなくて、寂しいです。	
\\	家族[かぞく]に 会[あ]えなくて、 寂[さび]しいです。	
\\	事故があって、バスが遅れてしまいました。	
\\	事故[じこ]があって、バスが 遅[おく]れてしまいました。	
\\	私たちが初めて会ったのはいつか、覚えていますか。	
\\	私[わたし]たちが 初[はじ]めて 会[あ]ったのはいつか、 覚[おぼ]えていますか。	
\\	部長が娘にお土産をくださいました。	
\\	部長[ぶちょう]が 娘[むすめ]にお 土産[みやげ]をくださいました。	
\\	部長が私にお土産をくださいました。	
\\	部長[ぶちょう]が 私[わたし]にお 土産[みやげ]をくださいました。	
\\	私は犬を散歩に連れて行ってやりました。	
\\	私[わたし]は 犬[いぬ]を 散歩[さんぽ]に 連[つ]れて 行[い]ってやりました。	
\\	私は課長に手紙の間違いを直していただきました。	
\\	私[わたし]は 課長[かちょう]に 手紙[てがみ]の 間違[まちが]いを 直[なお]していただきました。	
\\	部長の奥さんはお茶を教えてくださいました。	
\\	部長[ぶちょう]の 奥[おく]さんはお 茶[ちゃ]を 教[おし]えてくださいました。	
\\	部長はレポートを直してくださいました。	
\\	部長[ぶちょう]はレポートを 直[なお]してくださいました。	
\\	引っ越しのために、車を借ります。	
\\	引っ越[ひっこ]しのために、 車[くるま]を 借[か]ります。	
\\	日本語が上手になるように、毎日勉強しています。	
\\	日本語[にほんご]が 上手[じょうず]になるように、 毎日[まいにち] 勉強[べんきょう]しています。	
\\	このはさみは花を切るのに使います。	
\\	このはさみは 花[はな]を 切[き]るのに 使[つか]います。	
\\	このかばんは大きくて、旅行に便利です。	
\\	このかばんは 大[おお]きくて、 旅行[りょこう]に 便利[べんり]です。	
\\	早く届くように、速達で出します。	
\\	早[はや]く 届[とど]くように、 速達[そくたつ]で 出[だ]します。	
\\	忘れないように、メモします。	
\\	忘[わす]れないように、メモします。	
\\	近くに店がなくて、買い物に不便です。	
\\	近[ちか]くに 店[みせ]がなくて、 買い物[かいもの]に 不便[ふべん]です。	
\\	日本では結婚式をするのに200万円は要ります。	
\\	日本[にっぽん]では 結婚式[けっこんしき]をするのに200 万[まん] 円[えん]は 要[い]ります。	
\\	駅まで行くのに2時間もかかりました。	
\\	駅[えき]まで 行[い]くのに2 時間[じかん]もかかりました。	
\\	うちを建てるのに3000万円も必要なんですか。	
\\	うちを 建[た]てるのに3000 万[まん] 円[えん]も 必要[ひつよう]なんですか。	
\\	今にも雨が降りそうです。	
\\	今[いま]にも 雨[あめ]が 降[ふ]りそうです。	
\\	シャンプーがなくなりそうです。	
\\	シャンプーがなくなりそうです。	
\\	もうすぐ桜が咲きそうです。	
\\	もうすぐ 桜[さくら]が 咲[さ]きそうです。	
\\	これから寒くなりそうです。	
\\	これから 寒[さむ]くなりそうです。	
\\	彼女は頭がよさそうです。	
\\	彼女[かのじょ]は 頭[あたま]がよさそうです。	
\\	この机は丈夫そうです。	
\\	この 机[つくえ]は 丈夫[じょうぶ]そうです。	
\\	ちょっとたばこを買ってきます。	
\\	ちょっとたばこを 買[か]ってきます。	
\\	スーパーで牛乳を買ってきます。	
\\	スーパーで 牛乳[ぎゅうにゅう]を 買[か]ってきます。	
\\	台所からコップを取ってきます。	
\\	台所[だいどころ]からコップを 取[と]ってきます。	
\\	会議に間に合わない場合は、連絡して下さい。	
\\	会議[かいぎ]に 間に合[まにあ]わない 場合[ばあい]は、 連絡[れんらく]して 下[くだ]さい。	
\\	ファクスの調子が悪い場合は、どうしたらいいですか。	
\\	ファクスの 調子[ちょうし]が 悪[わる]い 場合[ばあい]は、どうしたらいいですか。	
\\	領収書が必要な場合は、係に言ってください。	
\\	領収[りょうしゅう] 書[しょ]が 必要[ひつよう]な 場合[ばあい]は、 係[かかり]に 言[い]ってください。	
\\	たった今バスが出たところです。	
\\	たった 今[いま]バスが 出[で]たところです。	
\\	このビデオは先週買ったばかりなのに、調子がおかしいです。	
\\	このビデオは 先週[せんしゅう] 買[か]ったばかりなのに、 調子[ちょうし]がおかしいです。	
\\	天気予報によると、明日は寒くなるそうです。	
\\	天気[てんき] 予報[よほう]によると、 明日[あした]は 寒[さむ]くなるそうです。	
\\	クララさんは子供のとき、フランスに住んでいたそうです。	
\\	クララさんは 子供[こども]のとき、フランスに 住[す]んでいたそうです。	
\\	バリ島はとてもきれいだそうです。	
\\	バリ島[ばりとう]はとてもきれいだそうです。	
\\	この料理はおいしいそうです。	
\\	この 料理[りょうり]はおいしいそうです。	
\\	ミラーさんは明日京都へ行くそうです。	
\\	ミラーさんは 明日[あした] 京都[きょうと]へ 行[い]くそうです。	
\\	せきも出るし、頭も痛い。どうも風邪をひいたようだ。	
\\	せきも 出[で]るし、 頭[あたま]も 痛[いた]い。どうも 風邪[かぜ]をひいたようだ。	
\\	先生は生徒に自由に意見を言わせました。	
\\	先生[せんせい]は 生徒[せいと]に 自由[じゆう]に 意見[いけん]を 言[い]わせました。	
\\	朝は忙しいですから、娘に朝ご飯の準備を手伝わせます。	
\\	朝[あさ]は 忙[いそが]しいですから、 娘[むすめ]に 朝[あさ]ご 飯[はん]の 準備[じゅんび]を 手伝[てつだ]わせます。	
\\	社長はもうお帰りになりました。	
\\	社長[しゃちょう]はもうお 帰[かえ]りになりました。	
\\	重そうですね。お持ちしましょうか。	
\\	重[おも]そうですね。お 持[も]ちしましょうか。	
\\	社長の奥様にお目にかかりました。	
\\	社長[しゃちょう]の 奥様[おくさま]にお 目[め]にかかりました。	
\\	このパンフレットをいただいてもよろしいでしょうか。	
\\	このパンフレットをいただいてもよろしいでしょうか。	
\\	私は飽きっぽい性格ですが、演劇だけは止めずに続けてきました。	
\\	私[わたし]は 飽[あ]きっぽい 性格[せいかく]ですが、 演劇[えんげき]だけは 止[と]めずに 続[つづ]けてきました。	
\\	甘えん坊だな。	
\\	甘えん坊[あまえんぼう]だな。	甘えん坊=あまえんぼう= 
\\	あなたは我慢強いですか?	
\\	あなたは 我慢強[がまんづよ]いですか?	
\\	覚えるのが遅い人には、とても我慢強い先生が必要です。	
\\	覚[おぼ]えるのが 遅[おそ]い 人[ひと]には、とても 我慢強[がまんづよ]い 先生[せんせい]が 必要[ひつよう]です。	
\\	彼は泣き虫だ。	
\\	彼[かれ]は 泣き虫[なきむし]だ。	
\\	お金にだらしない人は将来のパートナーにふさわしくありません。	
\\	お 金[かね]にだらしない 人[ひと]は 将来[しょうらい]のパートナーにふさわしくありません。	だらしない= 
\\	彼は服装がだらしない。	
\\	彼[かれ]は 服装[ふくそう]がだらしない。	だらしない= 
\\	明日も熱が下がらないようなら、医者に診てもらったほうがいいよ。	
\\	明日[あした]も 熱[ねつ]が 下[さ]がらないようなら、 医者[いしゃ]に 診[み]てもらったほうがいいよ。	
\\	アレルギー体質なので、その薬を飲む前に医者に相談した。	
\\	アレルギー 体質[たいしつ]なので、その 薬[くすり]を 飲[の]む 前[まえ]に 医者[いしゃ]に 相談[そうだん]した。	
\\	私は持病のぜんそくで、医者にかかっています。	
\\	私[わたし]は 持病[じびょう]のぜんそくで、 医者[いしゃ]にかかっています。	持病の= 
\\	喘息=ぜんそく= 
\\	医者を呼びましょうか?	
\\	医者[いしゃ]を 呼[よ]びましょうか?	
\\	インフルエンザにかかってしまったかもしれない。	
\\	インフルエンザにかかってしまったかもしれない。	
\\	今年の冬は、インフルエンザが大流行すると報道されている。	
\\	今年[ことし]の 冬[ふゆ]は、インフルエンザが 大[だい] 流行[りゅうこう]すると 報道[ほうどう]されている。	報道=ほうどう= 
\\	インフルエンザを予防する最も効果的な方法は、うがいと手洗いだ。	
\\	インフルエンザを 予防[よぼう]する 最[もっと]も 効果[こうか] 的[てき]な 方法[ほうほう]は、うがいと 手洗[てあら]いだ。	
\\	昨日、足を骨折して入院している友人のお見舞いに行った。	
\\	昨日[きのう]、 足[あし]を 骨折[こっせつ]して 入院[にゅういん]している 友人[ゆうじん]のお 見舞[みま]いに 行[い]った。	
\\	彼女にお見舞いを渡すよう、頼まれました。	
\\	彼女[かのじょ]にお 見舞[みま]いを 渡[わた]すよう、 頼[たの]まれました。	
\\	今、私の周りのたくさんの人が風邪をひいています。	
\\	今[いま]、 私[わたし]の 周[まわ]りのたくさんの 人[ひと]が 風邪[かぜ]をひいています。	
\\	今年の風邪はしつこくて、なかなか治らない。	
\\	今年[ことし]の 風邪[かぜ]はしつこくて、なかなか 治[なお]らない。	しつこい= 
\\	風邪をこじらせるといけないから、もう一日休みを取ったら?	
\\	風邪[かぜ]をこじらせるといけないから、もう一 日[にち] 休[やす]みを 取[と]ったら?	こじらせる= 
\\	風邪がはやっているので、気をつけてください。	
\\	風邪[かぜ]がはやっているので、 気[き]をつけてください。	
\\	家族全員が花粉症です。	
\\	家族[かぞく] 全員[ぜんいん]が 花粉[かふん] 症[しょう]です。	
\\	どうやら今年から花粉症になったようだ。	
\\	どうやら 今年[ことし]から 花粉[かふん] 症[しょう]になったようだ。	
\\	救急隊の一人が、けがの手当をしてくれた。	
\\	救急[きゅうきゅう] 隊[たい]の一 人[にん]が、けがの 手当[てあて]をしてくれた。	救急隊= 
\\	けがの手当をする= 
\\	完全にけがが治るまで、運動は禁止です。	
\\	完全[かんぜん]にけがが 治[なお]るまで、 運動[うんどう]は 禁止[きんし]です。	
\\	オリーブ油は健康にいい。	
\\	オリーブ油[おりーぶゆ]は 健康[けんこう]にいい。	
\\	一度病気をしてから、健康に気をつけるようになった。	
\\	一度[いちど] 病気[びょうき]をしてから、 健康[けんこう]に 気[き]をつけるようになった。	
\\	彼は若さと健康を保つために、毎朝ジョギングをしている。	
\\	彼[かれ]は 若[わか]さと 健康[けんこう]を 保[たも]つために、 毎朝[まいあさ]ジョギングをしている。	
\\	夕べから頭痛がします。	
\\	夕[ゆう]べから 頭痛[ずつう]がします。	
\\	この薬を飲んで少ししたら、頭痛は治まった。	
\\	この 薬[くすり]を 飲[の]んで 少[すこ]ししたら、 頭痛[ずつう]は 治[おさ]まった。	
\\	せきをしているけど、風邪ですか?	
\\	せきをしているけど、 風邪[かぜ]ですか?	
\\	夜になるとせきが止まらなくなって、とても苦しい。	
\\	夜[よる]になるとせきが 止[と]まらなくなって、とても 苦[くる]しい。	
\\	毎朝のジョギングを始めたら、最近とても体調がいい。	
\\	毎朝[まいあさ]のジョギングを 始[はじ]めたら、 最近[さいきん]とても 体調[たいちょう]がいい。	
\\	今日は体調が悪いので、パーティーはパスします。	
\\	今日[きょう]は 体調[たいちょう]が 悪[わる]いので、パーティーはパスします。	
\\	来週富士山に登るので、体調を整えておかなければ。	
\\	来週[らいしゅう] 富士山[ふじさん]に 登[のぼ]るので、 体調[たいちょう]を 整[ととの]えておかなければ。	
\\	彼は体調が崩して、先月から休職しています。	
\\	彼[かれ]は 体調[たいちょう]が 崩[くず]して、 先月[せんげつ]から 休職[きゅうしょく]しています。	体調を崩す= 
\\	彼は精いっぱい働いたが、のちに体調を崩しました。	
\\	彼[かれ]は 精[せい]いっぱい 働[はたら]いたが、のちに 体調[たいちょう]を 崩[くず]しました。	体調を崩す= 
\\	一時間ボクシングの練習をして、すっかり体力を消耗してしまった。	
\\	一 時間[じかん]ボクシングの 練習[れんしゅう]をして、すっかり 体力[たいりょく]を 消耗[しょうもう]してしまった。	消耗=しょうもう= 
\\	明日は忙しくなりそうだから、体力を蓄えておこう。	
\\	明日[あした]は 忙[いそが]しくなりそうだから、 体力[たいりょく]を 蓄[たくわ]えておこう。	蓄える= 
\\	体力を蓄える= 
\\	体力があるので、一日や二日徹夜しても全く平気です。	
\\	体力[たいりょく]があるので、 一日[いちにち]や 二日[ふつか] 徹夜[てつや]しても 全[まった]く 平気[へいき]です。	
\\	彼は自転車で転んで、ひざから血が出ていた。	
\\	彼[かれ]は 自転車[じてんしゃ]で 転[ころ]んで、ひざから 血[ち]が 出[で]ていた。	
\\	はっと気がついたら、額から血が流れていた。	
\\	はっと 気[き]がついたら、 額[ひたい]から 血[ち]が 流[なが]れていた。	額=ひたい= 
\\	傷口をしばらくの間押さえておくと、血は止まります。	
\\	傷口[きずぐち]をしばらくの 間[あいだ] 押[お]さえておくと、 血[ち]は 止[と]まります。	
\\	鼻血が出て、シャツに血がついてしまった。	
\\	鼻血[はなぢ]が 出[で]て、シャツに 血[ち]がついてしまった。	
\\	顔が赤いけど、熱があるんじゃない?	
\\	顔[かお]が 赤[あか]いけど、 熱[ねつ]があるんじゃない?	
\\	高い熱が出た時は、インフルエンザの可能性が高い。	
\\	高い 熱[ねつ]が出た時は、インフルエンザの 可能[かのう] 性[せい]が高い。	
\\	熱を測ったら、37
\\	8度ありました。		熱[ねつ]を 測[はか]ったら、 
\\	8[さんじゅうななてんはち] 度[ど]ありました。	
\\	翌日になってようやく熱が下がってきた。	
\\	翌日[よくじつ]になってようやく 熱[ねつ]が 下[さ]がってきた。	
\\	この季節は、毎日のように洟が出て止まらない。	
\\	この 季節[きせつ]は、毎日のように 洟[はな]が出て止まらない。	洟=はな= 
\\	待合室のあちこちから、洟をすする音が聞こえてきた。	
\\	待合室[まちあいしつ]のあちこちから、 洟[はな]をすする 音[おと]が 聞[き]こえてきた。	洟=はな= 
\\	すする= 
\\	洟をすする= 
\\	この5年間、病院に行っていません。	
\\	この5 年間[ねんかん]、 病院[びょういん]に行っていません。	
\\	アトピーの治療のため、病院に通っています。	
\\	アトピーの 治療[ちりょう]のため、 病院[びょういん]に 通[かよ]っています。	アトピー= 
\\	治療=ちりょう= 
\\	病院はとても混んでいて、1時間以上待たされた。	
\\	病院[びょういん]はとても 混[こ]んでいて、1 時間[じかん] 以上[いじょう] 待[ま]たされた。	
\\	そんな不摂生を続けていたら、病気になるよ。	
\\	そんな 不摂生[ふせっせい]を 続[つづ]けていたら、 病気[びょうき]になるよ。	摂生=せっせい= 
\\	不摂生=ふせっせい= 
\\	病気から回復しました。	
\\	病気[びょうき]から 回復[かいふく]しました。	
\\	彼は持病が悪化して、先週入院した。	
\\	彼[かれ]は 持病[じびょう]が 悪化[あっか]して、 先週[せんしゅう] 入院[にゅういん]した。	持病=じびょう= 
\\	きちんと病気を治して、また仕事に復帰したい。	
\\	きちんと 病気[びょうき]を 治[なお]して、また 仕事[しごと]に 復帰[ふっき]したい。	復帰=ふっき= 
\\	これらのハーブは、いろいろな病気に効くとされています。	
\\	これらのハーブは、いろいろな 病気[びょうき]に 効[き]くとされています。	
\\	その病気をうつさないで!	
\\	その 病気[びょうき]をうつさないで!	
\\	私の父は78歳のとき、病気で亡くなりました。	
\\	私[わたし]の 父[ちち]は 78歳[ななじゅうはっさい]のとき、 病気[びょうき]で 亡[な]くなりました。	
\\	薬局は近くにありますか?	
\\	薬局[やっきょく]は 近[ちか]くにありますか?	薬局=やっきょく= 
\\	薬剤師は、この薬に有害な副作用はないと言った。	
\\	薬剤師[やくざいし]は、この 薬[くすり]に 有害[ゆうがい]な 副作用[ふくさよう]はないと言った。	薬剤師=やくざいし= 
\\	有害=ゆうがい= 
\\	副作用=ふくさよう= 
\\	ひんやりした風が気持ちいい。	
\\	ひんやりした 風邪[かぜ]が 気持[きも]ちいい。	ひんやり= 
\\	ひんやりとした風が顔に心地良かったです。	
\\	ひんやりとした 風[かぜ]が 顔[かお]に 心地[ここち] 良[よ]かったです。	ひんやり= 
\\	心地良い=ここちよい= 
\\	この部屋はひんやりしている。	
\\	この 部屋[へや]はひんやりしている。	ひんやり= 
\\	普段は眼鏡ですが、運動するときはコンタクトをしています。	
\\	普段[ふだん]は 眼鏡[めがね]ですが、 運動[うんどう]するときはコンタクトをしています。	
\\	初めてコンタクトを入れたときは、なかなかうまく入れられなかった。	
\\	初[はじ]めてコンタクトを 入[い]れたときは、なかなかうまく 入[い]れられなかった。	
\\	コンタクトをはずすのを忘れて寝てしまい、翌朝は目が痛くて大変だった。	
\\	コンタクトをはずすのを 忘[わす]れて 寝[ね]てしまい、 翌朝[よくあさ]は 目[め]が 痛[いた]くて 大変[たいへん]だった。	
\\	私はしょっちゅうコンタクトを亡くすので、使い捨ての物に変えた。	
\\	私[わたし]はしょっちゅうコンタクトを 亡[な]くすので、 使い捨[つかいす]ての 物[もの]に 変[か]えた。	しょっちゅう= 
\\	いくら毎日歯磨きをしても、歯石がつくのは避けられない。	
\\	いくら毎日 歯磨[はみが]きをしても、 歯石[しせき]がつくのは 避[さ]けられない。	歯磨き= 
\\	歯石=しせき= 
\\	一年に一回、歯医者で歯石を取ってもらうことにしています。	
\\	一 年[ねん]に一 回[かい]、 歯医者[はいしゃ]で 歯石[しせき]を 取[と]ってもらうことにしています。	歯石=しせき= 
\\	先日、眼科で視力を測ってもらった。	
\\	先日[せんじつ]、 眼科[がんか]で 視力[しりょく]を 測[はか]ってもらった。	眼科= 
\\	目を使う仕事を始めてから、視力が落ちた。	
\\	目[め]を 使[つか]う 仕事[しごと]を 始[はじ]めてから、 視力[しりょく]が 落[お]ちた。	
\\	彼女は小さいころに、病気で視力を失った。	
\\	彼女[かのじょ]は 小[ちい]さいころに、 病気[びょうき]で 視力[しりょく]を 失[うしな]った。	
\\	来週、虫歯を一本抜くことになった。	
\\	来週[らいしゅう]、 虫歯[むしば]を 一本[いっぽん] 抜[ぬ]くことになった。	
\\	歯が痛くて、夕べは一睡もできなかった。	
\\	歯[は]が 痛[いた]くて、 夕[ゆう]べは 一睡[いっすい]もできなかった。	一睡=いっすい= 
\\	アイスのように冷たい物を食べると、歯にしみる。	
\\	アイスのように 冷[つめ]たい 物[もの]を 食[た]べると、 歯[は]にしみる。	しみる= 
\\	痛っ、冷たい水が歯にしみる。歯医者に行かなくちゃ。	
\\	痛[いた]っ、 冷[つめ]たい 水[みず]が 歯[は]にしみる。 歯医者[はいしゃ]に 行[い]かなくちゃ。	
\\	トウモロコシの皮が歯にはさまってしまった。	
\\	トウモロコシの 皮[かわ]が 歯[は]にはさまってしまった。	トウモロコシ= 
\\	挟まる=はさまる= 
\\	アメリカでは、小さいときに歯の矯正をするのが、一般的だ。	
\\	アメリカでは、 小[ちい]さいときに 歯[は]の 矯正[きょうせい]をするのが、 一般[いっぱん] 的[てき]だ。	矯正=きょうせい= 
\\	あなたの生きがいは何ですか?	
\\	あなたの 生[い]きがいは 何[なに]ですか?	生き甲斐=いきがい= 
\\	仕事は私の生きがいです。	
\\	仕事[しごと]は 私[わたし]の 生[い]きがいです。	生き甲斐=いきがい= 
\\	私はとうとう生きがいを見つけました。	
\\	私はとうとう 生[い]きがいを 見[み]つけました。	生き甲斐=いきがい= 
\\	とうとう= 
\\	ブレア氏は最近、困難に直面しています。	
\\	ブレア 氏[し]は最近、 困難[こんなん]に 直面[ちょくめん]しています。	
\\	彼は終身刑に直面している。	
\\	彼[かれ]は 終身[しゅうしん] 刑[けい]に 直面[ちょくめん]している。	
\\	日本は犯罪率が低い点で世界のトップにある。	
\\	日本は 犯罪[はんざい] 率[りつ]が 低[ひく]い 点[てん]で 世界[せかい]のトップにある。	
\\	日本のエネルギー消費総量は、アメリカ、中国、ロシアに次いで世界第4位です。	
\\	日本のエネルギー 消費[しょうひ] 総量[そうりょう]は、アメリカ、 中国[ちゅうごく]、ロシアに 次[つ]いで 世界[せかい] 第[だい]4 位[い]です。	総量=そうりょう= 
\\	エクアドルは、バナナの輸出量第1位の国であることをご存じですか?	
\\	エクアドルは、バナナの 輸出[ゆしゅつ] 量[りょう] 第[だい]1 位[い]の 国[くに]であることをご 存[ぞん]じですか?	
\\	米国は、死刑に背を向けつつある。	
\\	米国[べいこく]は、 死刑[しけい]に 背[せ]を 向[む]けつつある。	背を向ける= 
\\	親友でさえ私に背を向け始めた。	
\\	親友[しんゆう]でさえ 私[わたし]に 背[せ]を 向[む]け 始[はじ]めた。	背を向ける= 
\\	いずれ近いうちにニューヨークでお会いしたいと思います。	
\\	いずれ 近[ちか]いうちにニューヨークでお 会[あ]いしたいと 思[おも]います。	
\\	この出会いは画期的なものだった。	
\\	この 出会[であ]いは 画期的[かっきてき]なものだった。	画期的=かっき てき= 
\\	ひょっとすると、これは私が今まで見た中で最高の映画かもしれません。	
\\	ひょっとすると、これは私が 今[いま]まで 見[み]た 中[なか]で 最高[さいこう]の 映画[えいが]かもしれません。	ひょっとすると= 
\\	ひょっとするとあなたの言っていることが彼に聞こえるかもしれません。	
\\	ひょっとするとあなたの言っていることが 彼[かれ]に聞こえるかもしれません。	ひょっとすると= 
\\	その方針にはあからさまな例外がある。	
\\	その 方針[ほうしん]にはあからさまな 例外[れいがい]がある。	あからさま= 
\\	アメリカではエイズ患者に対してあからさまな差別がある。	
\\	アメリカではエイズ 患者[かんじゃ]に 対[たい]してあからさまな 差別[さべつ]がある。	あからさま= 
\\	彼女はあからさまに彼を会話から締め出しました。	
\\	彼女[かのじょ]はあからさまに 彼[かれ]を 会話[かいわ]から 締め出[しめだ]しました。	あからさま= 
\\	締め出す= 
\\	フリーターの数は過去10年で2倍に増えました。	
\\	フリーターの 数[かず]は 過去[かこ]10 年[ねん]で2 倍[ばい]に 増[ふ]えました。	フリーター= 
\\	15歳から34歳の人の約20%がフリーターです。	
\\	歳[さい]から34 歳[さい]の 人[ひと]の 約[やく]20 
\\	[ぱーせんと]がフリーターです。	フリーター= 
\\	現在、労働者の5人に1人がフリーターです。	
\\	現在[げんざい]、 労働[ろうどう] 者[しゃ]の5 人[にん]に 1人[ひとり]がフリーターです。	フリーター= 
\\	毎日欠かさずリンゴを食べる。	
\\	毎日[まいにち] 欠[か]かさずリンゴを 食[た]べる。	欠かさず= 
\\	私はこの番組を毎週欠かさず見る。	
\\	私はこの 番組[ばんぐみ]を 毎週[まいしゅう] 欠[か]かさず 見[み]る。	欠かさず= 
\\	それは私にとって解決できない問題のように思える。	
\\	それは 私[わたし]にとって 解決[かいけつ]できない 問題[もんだい]のように 思[おも]える。	
\\	あなたには選択の余地がありそうです。	
\\	あなたには 選択[せんたく]の 余地[よち]がありそうです。	
\\	残念ですが、明日の午後は歯医者を予約しています。	
\\	残念[ざんねん]ですが、 明日[あした]の 午後[ごご]は 歯医者[はいしゃ]を 予約[よやく]しています。	
\\	子どものころは、歯医者に行くのが嫌でたまらなかった。	
\\	子[こ]どものころは、 歯医者[はいしゃ]に 行[い]くのが 嫌[いや]でたまらなかった。	たまる= 
\\	甘い物を食べると虫歯になるというのは、必ずしも正しくない。	
\\	甘[あま]い 物[もの]を 食[た]べると 虫歯[むしば]になるというのは、 必[かなら]ずしも 正[ただ]しくない。	
\\	虫歯があると、そこから菌に感染することがある。	
\\	虫歯[むしば]があると、そこから 菌[きん]に 感染[かんせん]することがある。	菌=きん= 
\\	感染=かんせん= 
\\	虫歯を予防するには、毎食後の歯磨きが大切だ。	
\\	虫歯[むしば]を 予防[よぼう]するには、 毎食[まいしょく] 後[ご]の 歯磨[はみが]きが 大切[たいせつ]だ。	
\\	目がいい人たちが、うらやましい。	
\\	目[め]がいい 人[ひと]たちが、うらやましい。	
\\	パソコンの画面をずっと見ていると、目が疲れる。	
\\	パソコンの 画面[がめん]をずっと 見[み]ていると、 目[め]が 疲[つか]れる。	
\\	朝起きたら、目が少し腫れていた。	
\\	朝[あさ] 起[お]きたら、 目[め]が 少[すこ]し 腫[は]れていた。	腫れる=はれる= 
\\	そんな暗い所で本を読んでいると、目が悪くなるよ。	
\\	そんな 暗[くら]い 所[ところ]で 本[ほん]を 読[よ]んでいると、 目[め]が 悪[わる]くなるよ。	
\\	最近、夕方になると目がかすむ。	
\\	最近[さいきん]、 夕方[ゆうがた]になると 目[め]がかすむ。	かすむ= 
\\	目にごみが入った。	
\\	目[め]にごみが 入[はい]った。	
\\	写真の真ん中にいる、眼鏡をかけた人が私の兄です。	
\\	写真[しゃしん]の 真ん中[まんなか]にいる、 眼鏡[めがね]をかけた人が私の 兄[あに]です。	
\\	そんなに目が疲れるというのは、眼鏡の度が合っていないのかもしれない。	
\\	そんなに 目[め]が 疲[つか]れるというのは、 眼鏡[めがね]の 度[たび]が 合[あ]っていないのかもしれない。	
\\	「度」= 
\\	眼鏡をとると、まるで別人のようだね。	
\\	眼鏡[めがね]をとると、まるで 別人[べつじん]のようだね。	
\\	私は一ヶ月に一回、髪を切ります。	
\\	私は 一ヶ月[いっかげつ]に 一回[いっかい]、 髪[かみ]を 切[き]ります。	
\\	気分転換に髪にパーマをかけた。	
\\	気分[きぶん] 転換[てんかん]に 髪[かみ]にパーマをかけた。	
\\	最近では、髪を染めるのがごく当たり前になってきた。	
\\	最近[さいきん]では、 髪[かみ]を 染[そ]めるのがごく 当たり前[あたりまえ]になってきた。	染める= 
\\	ごく= 
\\	私は美容院よりも、床屋で散髪するほうが好きだ。	
\\	私は 美容[びよう] 院[いん]よりも、 床屋[とこや]で 散髪[さんぱつ]するほうが 好[す]きだ。	散髪=さんぱつ= 
\\	どのくらいの割合で、ネイルサロンに行きますか?	
\\	どのくらいの 割合[わりあい]で、ネイルサロンに 行[い]きますか?	
\\	明日は友達の結婚パーティーなので、ネイルをした。	
\\	明日[あした]は 友達[ともだち]の 結婚[けっこん]パーティーなので、ネイルをした。	
\\	ネイルを付けた状態で、料理をしたり洗い物をしたりするのは難しい。	
\\	ネイルを 付[つ]けた 状態[じょうたい]で、 料理[りょうり]をしたり 洗い物[あらいもの]をしたりするのは 難[むずか]しい。	
\\	来週水曜日に、美容院を予約した。	
\\	来週[らいしゅう] 水曜日[すいようび]に、 美容[びよう] 院[いん]を 予約[よやく]した。	
\\	私の住んでいるアパートの隣の部屋が、空き巣に入られた。	
\\	私の 住[す]んでいるアパートの 隣[となり]の 部屋[へや]が、 空き巣[あきす]に 入[はい]られた。	空き巣=あきす= 
\\	このあたりは空き巣が多い。	
\\	このあたりは 空き巣[あきす]が 多[おお]い。	空き巣=あきす= 
\\	今年に入ってから、様々なアクシデントに見舞われています。	
\\	今年[ことし]に 入[はい]ってから、 様々[さまざま]なアクシデントに 見舞[みま]われています。	見舞われる= 
\\	20台もの車がその事故に巻き込まれた。	
\\	台[だい]もの 車[くるま]がその 事故[じこ]に 巻き込[まきこ]まれた。	
\\	これらの動物は、絶滅の危険にさらされています。	
\\	これらの 動物[どうぶつ]は、 絶滅[ぜつめつ]の 危険[きけん]にさらされています。	絶滅= 
\\	さらす= 
\\	ほとんどの手術には、ある程度の危険が伴うものだ。	
\\	ほとんどの 手術[しゅじゅつ]には、ある 程度[ていど]の 危険[きけん]が 伴[ともな]うものだ。	
\\	人間には本来、危険を察知する力が備わっている。	
\\	人間[にんげん]には 本来[ほんらい]、 危険[きけん]を 察知[さっち]する 力[ちから]が 備[そな]わっている。	察知=さっち= 
\\	備わる=そなわる= 
\\	彼は、自分の仕事を失う危険を冒して、私を助けてくれた。	
\\	彼[かれ]は、 自分[じぶん]の 仕事[しごと]を 失[うしな]う 危険[きけん]を 冒[おか]して、 私[わたし]を 助[たす]けてくれた。	冒す= 
\\	その店員は危険をかえりみず、強盗と格闘した。	
\\	その 店員[てんいん]は 危険[きけん]をかえりみず、 強盗[ごうとう]と 格闘[かくとう]した。	危険をかえりみない= 
\\	強盗=ごうとう= 
\\	格闘=かくとう= 
\\	道で財布を拾ったので、警察に届けた。	
\\	道[みち]で 財布[さいふ]を 拾[ひろ]ったので、 警察[けいさつ]に 届[とど]けた。	
\\	誰かが彼のことを警察に通報しました。	
\\	誰[だれ]かが 彼[かれ]のことを 警察[けいさつ]に 通報[つうほう]しました。	通報=つうほう= 
\\	何か変わったことがあったら、すぐに警察に知らせてください。	
\\	何[なに]か 変[か]わったことがあったら、すぐに 警察[けいさつ]に 知[し]らせてください。	
\\	隣の部屋の騒音があまりにひどかったので、警察に通報した。	
\\	隣[となり]の 部屋[へや]の 騒音[そうおん]があまりにひどかったので、 警察[けいさつ]に 通報[つうほう]した。	
\\	彼は一度酔っ払いとけんかして、警察沙汰になったことがある。	
\\	彼[かれ]は 一度[いちど] 酔っ払[よっぱら]いとけんかして、 警察[けいさつ] 沙汰[ざた]になったことがある。	警察沙汰=けいさつざた= 
\\	電子レンジから煙が出ているよ!	
\\	電子[でんし]レンジから 煙[けむり]が 出[で]ているよ!	
\\	ホールにはあっという間に煙が立ち込めたそうだ。	
\\	ホールにはあっという 間[ま]に 煙[けむり]が 立ち込[たちこ]めたそうだ。	煙が立ち込める= 
\\	その火事では、煙に巻かれて二人の客が亡くなった。	
\\	その 火事[かじ]では、 煙[けむり]に 巻[ま]かれて二 人[にん]の 客[きゃく]が 亡[な]くなった。	
\\	子どものころは、しょっちゅう兄とけんかをしたものだ。	
\\	子[こ]どものころは、しょっちゅう 兄[あに]とけんかをしたものだ。	
\\	ここまで私を連れ出してくれてありがとう。	
\\	ここまで私を 連れ出[つれだ]してくれてありがとう。	
\\	こんな寒いところに連れ出して申し訳ない。	
\\	こんな 寒[さむ]いところに 連れ出[つれだ]して 申し訳[もうしわけ]ない。	
\\	人生は真剣に受け止めるにはあまりにも深刻すぎる。	
\\	人生[じんせい]は 真剣[しんけん]に 受け止[うけと]めるにはあまりにも 深刻[しんこく]すぎる。	真剣=しんけん= (本物の剣) 
\\	(〜な) (まじめな) 
\\	彼の批評は真剣に受け止める価値はない。	
\\	彼[かれ]の 批評[ひひょう]は 真剣[しんけん]に 受け止[うけと]める 価値[かち]はない。	批評=ひひょう= 
\\	真剣=しんけん= (本物の剣) 
\\	(〜な) (まじめな) 
\\	この料理法はわが家に代々伝わるものです。	
\\	この 料理[りょうり] 法[ほう]はわが 家[や]に 代々[だいだい] 伝[つた]わるものです。	
\\	我が家に代々伝わる迷信がある。	
\\	我が家[わがや]に 代々[だいだい] 伝[つた]わる 迷信[めいしん]がある。	迷信=めいしん= 
\\	代々木上原で降りるとおっしゃってませんでしたか?	
\\	代々木上原[よよぎうえはら]で 降[お]りるとおっしゃってませんでしたか?	
\\	この解決法が見つかるまで眠れない。	
\\	この 解決[かいけつ] 法[ほう]が 見[み]つかるまで 眠[ねむ]れない。	
\\	結婚生活に満足している。	
\\	結婚[けっこん] 生活[せいかつ]に 満足[まんぞく]している。	
\\	今の自分に満足していますか?	
\\	今[いま]の 自分[じぶん]に 満足[まんぞく]していますか?	
\\	仕事に満足している?	
\\	仕事[しごと]に 満足[まんぞく]している?	
\\	どうやってバランスを取るのですか?	
\\	どうやってバランスを 取[と]るのですか?	
\\	何らかの代案があるに違いない。	
\\	何[なん]らかの 代案[だいあん]があるに 違[ちが]いない。	何らか= 
\\	代案= 
\\	何らかの希望の光は、誰にとっても必要です。	
\\	何[なん]らかの 希望[きぼう]の 光[ひかり]は、 誰[だれ]にとっても 必要[ひつよう]です。	何らか= 
\\	外国人が日本で成功するために必要なことは何でしょう?	
\\	外国[がいこく] 人[じん]が 日本[にっぽん]で 成功[せいこう]するために 必要[ひつよう]なことは 何[なに]でしょう?	
\\	どうしたらよいかよくわからないが、本当に行き詰まるまでは臨機応変にやろう。	
\\	どうしたらよいかよくわからないが、 本当[ほんとう]に 行き詰[いきづ]まるまでは 臨機応変[りんきおうへん]にやろう。	行き詰まる= 
\\	臨機応変= 
\\	何かがこつこつと音を立てている。	
\\	何[なに]かがこつこつと 音[おと]を 立[た]てている。	こつこつ= 
\\	音を立てる= 
\\	ペンで机をこつこつとたたいた。	
\\	ペンで 机[つくえ]をこつこつとたたいた。	こつこつ= 
\\	それはあなたが決めることではありません。	
\\	それはあなたが 決[き]めることではありません。	
\\	彼は、よっぽど運が強いのだろう。	
\\	彼[かれ]は、よっぽど 運[うん]が 強[つよ]いのだろう。	よっぽど= 
\\	私たちに耳を傾けてくれる人はいませんでした。	
\\	私たちに 耳[みみ]を 傾[かたむ]けてくれる 人[ひと]はいませんでした。	耳を傾ける=みみ を かたむける= 
\\	何も言わず、ただ耳を傾けてください。	
\\	何[なに]も 言[い]わず、ただ 耳[みみ]を 傾[かたむ]けてください。	耳を傾ける=みみ を かたむける= 
\\	医者は患者の言うことに耳を傾けなければならない。	
\\	医者[いしゃ]は 患者[かんじゃ]の 言[い]うことに 耳[みみ]を 傾[かたむ]けなければならない。	耳を傾ける=みみ を かたむける= 
\\	彼は常に私の話に耳を傾けてくれた。	
\\	彼[かれ]は 常[つね]に 私[わたし]の 話[はなし]に 耳[みみ]を 傾[かたむ]けてくれた。	耳を傾ける=みみ を かたむける= 
\\	意見が合わないときは、いつも私が折れるので、あまりけんかにならない。	
\\	意見[いけん]が 合[あ]わないときは、いつも 私[わたし]が 折[お]れるので、あまりけんかにならない。	
\\	君たち二人のけんかに巻き込まれたくない。	
\\	君[きみ]たち二 人[にん]のけんかに 巻き込[まきこ]まれたくない。	
\\	けんかを売る気か?	
\\	けんかを 売[う]る 気[き]か?	けんかを売る= 
\\	私にけんかを売っているのですか。	
\\	私[わたし]にけんかを 売[う]っているのですか。	けんかを売る= 
\\	私はさっさとこのけんかを止めさせて、事件を終わらせようとしました。	
\\	私[わたし]はさっさとこのけんかを 止[と]めさせて、 事件[じけん]を 終[お]わらせようとしました。	さっさと= 
\\	彼は売られたけんかを買う人だ。	
\\	彼[かれ]は 売[う]られたけんかを 買[か]う 人[ひと]だ。	けんかを売る= 
\\	けんかを買う= 
\\	誰かに後をつけられているような気がして、思わず交番に駆け込んだ。	
\\	誰[だれ]かに 後[あと]をつけられているような 気[き]がして、 思[おも]わず 交番[こうばん]に 駆け込[かけこ]んだ。	後を付ける= 
\\	駆け込む= 
\\	すみません(その場所は)分かりません。あそこの交番で聞いてみてください。	
\\	すみません(その 場所[ばしょ]は) 分[わ]かりません。あそこの 交番[こうばん]で 聞[き]いてみてください。	
\\	そのピアニストは努力することで、いくつもの困難を乗り越えてきた。	
\\	そのピアニストは 努力[どりょく]することで、いくつもの 困難[こんなん]を 乗り越[のりこ]えてきた。	
\\	今、人生で最大の困難に陥っています。	
\\	今[いま]、 人生[じんせい]で 最大[さいだい]の 困難[こんなん]に 陥[おちい]っています。	陥る= 
\\	困難に陥る= 
\\	彼女はどんな困難に見舞われようと、決してくじけなかった。	
\\	彼女[かのじょ]はどんな 困難[こんなん]に 見舞[みま]われようと、 決[けっ]してくじけなかった。	見舞われる= 
\\	くじける= 
\\	彼は経済的に大変な困難に直面している。	
\\	彼[かれ]は 経済[けいざい] 的[てき]に 大変[たいへん]な 困難[こんなん]に 直面[ちょくめん]している。	
\\	彼らは戦争中、多くの困難に耐えてきた。	
\\	彼[かれ]らは 戦争[せんそう] 中[ちゅう]、 多[おお]くの 困難[こんなん]に 耐[た]えてきた。	耐える= 
\\	その人は詐欺にあって、大金を騙し取られてしまった。	
\\	その 人[ひと]は 詐欺[さぎ]にあって、 大金[たいきん]を 騙[だま]し 取[と]られてしまった。	詐欺=さぎ= 
\\	騙しとる= 
\\	その容疑者は違法な株の売買で、詐欺を働いたとされている。	
\\	その 容疑[ようぎ] 者[しゃ]は 違法[いほう]な 株[かぶ]の 売買[ばいばい]で、 詐欺[さぎ]を 働[はたら]いたとされている。	容疑者= 
\\	売買= 
\\	詐欺を働く= 
\\	私は彼に100万円の借金がある。	
\\	私は彼に 100万[ひゃくまん] 円[えん]の 借金[しゃっきん]がある。	
\\	彼は銀行に対して、3千万円の借金を抱えている。	
\\	彼は 銀行[ぎんこう]に 対[たい]して、 3千[さんぜん] 万[まん] 円[えん]の 借金[しゃっきん]を 抱[かか]えている。	
\\	借金を完済するのに、3年かかった。	
\\	借金[しゃっきん]を 完済[かんさい]するのに、3 年[ねん]かかった。	
\\	その夫婦は借金を踏み倒して逃げた。	
\\	その 夫婦[ふうふ]は 借金[しゃっきん]を 踏み倒[ふみたお]して 逃[に]げた。	借金を踏み倒す= 
\\	彼は彼女を借金を踏み倒したことで告発しました。	
\\	彼[かれ]は 彼女[かのじょ]を 借金[しゃっきん]を 踏み倒[ふみたお]したことで 告発[こくはつ]しました。	踏み倒す= 
\\	告発=こくはつ= 
\\	来年、私たちは住宅ローンを完済するだろう。	
\\	来年[らいねん]、 私[わたし]たちは 住宅[じゅうたく]ローンを 完済[かんさい]するだろう。	完済=かんさい= 
\\	3ヶ月前に奨学金を完済した。	
\\	ヶ月[かげつ] 前[まえ]に 奨学[しょうがく] 金[きん]を 完済[かんさい]した。	完済=かんさい= 
\\	世界は新しい平和の時代に向かっていると彼は私に断言した。	
\\	世界[せかい]は 新[あたら]しい 平和[へいわ]の 時代[じだい]に 向[む]かっていると 彼[かれ]は 私[わたし]に 断言[だんげん]した。	
\\	その国では多くの人が入れ墨をしている。	
\\	その 国[くに]では 多[おお]くの 人[ひと]が 入れ墨[いれずみ]をしている。	
\\	その入れ墨、どのくらい時間かかった?	
\\	その 入れ墨[いれずみ]、どのくらい 時間[じかん]かかった?	
\\	入れ墨にはいまだに偏見がまとわりついている。	
\\	入れ墨[いれずみ]にはいまだに 偏見[へんけん]がまとわりついている。	偏見=へんけん= 
\\	まとわりつく= 
\\	エベレスト登頂を試みるつもりです。	
\\	エベレスト 登頂[とうちょう]を 試[こころ]みるつもりです。	登頂=とうちょう= 
\\	試みる=こころみる= 
\\	私は20年近く前に初めて禁煙を試みた。	
\\	私[わたし]は20 年[ねん] 近[ちか]く 前[まえ]に 初[はじ]めて 禁煙[きんえん]を 試[こころ]みた。	試みる=こころみる= 
\\	私はこのゲームの難易度に10段階評価で9をつけた。	
\\	私[わたし]はこのゲームの 難易[なんい] 度[ど]に10 段階[だんかい] 評価[ひょうか]で9をつけた。	難易度=なんいど= 
\\	段階評価=だんかいひょうか= 
\\	厳しい道だろうが努力は報われる。	
\\	厳[きび]しい 道[みち]だろうが 努力[どりょく]は 報[むく]われる。	報う=むくう= 
\\	その市では、ここ5年間で20%人口が増えた。	
\\	その 市[し]では、ここ5 年間[ねんかん]で 
\\	[にじゅっぱーせんと] 人口[じんこう]が 増[ふ]えた。	
\\	オーストラリアでは、沿岸地域に人口が集中している。	
\\	オーストラリアでは、 沿岸[えんがん] 地域[ちいき]に 人口[じんこう]が 集中[しゅうちゅう]している。	沿岸=えんがん= 
\\	その事件の真相を知る人は、もはやこの世にいない。	
\\	その 事件[じけん]の 真相[しんそう]を 知[し]る 人[ひと]は、もはやこの 世[よ]にいない。	真相=しんそう= 
\\	犯人が捕まって、ようやく真相が明らかになった。	
\\	犯人[はんにん]が 捕[つか]まって、ようやく 真相[しんそう]が 明[あき]らかになった。	真相=しんそう= 
\\	今は多くの人が、生活が苦しいと感じている。	
\\	今[いま]は 多[おお]くの 人[ひと]が、 生活[せいかつ]が 苦[くる]しいと 感[かん]じている。	
\\	私たちは横浜に引っ越して、新しい生活を始めました。	
\\	私[わたし]たちは 横浜[よこはま]に 引っ越[ひっこ]して、 新[あたら]しい 生活[せいかつ]を 始[はじ]めました。	
\\	その男は生活に困って、コンビニで盗みを働いた。	
\\	その 男[おとこ]は 生活[せいかつ]に 困[こま]って、コンビニで 盗[ぬす]みを 働[はたら]いた。	生活に困る= 
\\	私はなかなか新しい生活になじめなかった。	
\\	私[わたし]はなかなか 新[あたら]しい 生活[せいかつ]になじめなかった。	なじむ= 
\\	彼は長野の山奥で、自給自足の生活を営んでいます。	
\\	彼[かれ]は 長野[ながの]の 山奥[やまおく]で、 自給自足[じきゅうじそく]の 生活[せいかつ]を 営[いとな]んでいます。	山奥= 
\\	自給自足= 
\\	生活を営む= 
\\	この部屋に越してきてから、ずっと騒音に悩まされている。	
\\	この 部屋[へや]に 越[こ]してきてから、ずっと 騒音[そうおん]に 悩[なや]まされている。	
\\	アパートの上の階の住人に騒音を注意した。	
\\	アパートの 上[うえ]の 階[かい]の 住人[じゅうにん]に 騒音[そうおん]を 注意[ちゅうい]した。	
\\	ゴミ出しをめぐって、近隣の住人たちの間で騒動が起きた。	
\\	ゴミ 出[だ]しをめぐって、 近隣[きんりん]の 住人[じゅうにん]たちの 間[ま]で 騒動[そうどう]が 起[お]きた。	近隣=きんりん= 
\\	騒動=そうどう= 
\\	彼がこの騒動を引き起こした張本人だ。	
\\	彼[かれ]がこの 騒動[そうどう]を 引き起[ひきお]こした 張本人[ちょうほんにん]だ。	騒動=そうどう= 
\\	張本人= 
\\	一般に、日本は治安がいいとされている。	
\\	一般[いっぱん]に、 日本[にっぽん]は 治安[ちあん]がいいとされている。	治安=ちあん= 
\\	治安の悪化も目立ち始めた。	
\\	治安[ちあん]の 悪化[あっか]も 目立[めだ]ち 始[はじ]めた。	治安=ちあん= 
\\	その地域は最近、急速に治安が悪化している。	
\\	その 地域[ちいき]は 最近[さいきん]、 急速[きゅうそく]に 治安[ちあん]が 悪化[あっか]している。	治安=ちあん= 
\\	その国の治安が回復するのに、まだかなり時間がかかるだろう。	
\\	その 国[くに]の 治安[ちあん]が 回復[かいふく]するのに、まだかなり 時間[じかん]がかかるだろう。	治安=ちあん= 
\\	日本は自給自足を保護する必要がある。	
\\	日本は 自給自足[じきゅうじそく]を 保護[ほご]する 必要[ひつよう]がある。	自給自足= 
\\	村人は自給自足の生活をしている。	
\\	村人[むらびと]は 自給自足[じきゅうじそく]の 生活[せいかつ]をしている。	自給自足= 
\\	このテーマパークは2010年に開業する予定です。	
\\	このテーマパークは2010 年[ねん]に 開業[かいぎょう]する 予定[よてい]です。	開業=かいぎょう= 
\\	この新しいビルは2014年春に全面開業予定だ。	
\\	この 新[あたら]しいビルは2014 年[ねん] 春[はる]に 全面[ぜんめん] 開業[かいぎょう] 予定[よてい]だ。	開業=かいぎょう= 
\\	ビルの残りの部分は来春開業予定だ。	
\\	ビルの 残[のこ]りの 部分[ぶぶん]は 来春[らいしゅん] 開業[かいぎょう] 予定[よてい]だ。	開業=かいぎょう= 
\\	103階建てのエンパイアステートビルは高さ443メートルです。	
\\	階[かい] 建[だ]てのエンパイアステートビルは 高[たか]さ 443メートルです。	階建て=かいだて= 
\\	23階建てのビルと同じ高さです。	
\\	階[かい] 建[だ]てのビルと 同[おな]じ 高[たか]さです。	階建て=かいだて= 
\\	この2階建てバスは、日本で最初のオープンバスとなります。	
\\	この 
\\	階[かい] 建[だ]てバスは、 日本[にっぽん]で 最初[さいしょ]のオープンバスとなります。	階建て=かいだて= 
\\	このビルは101階建てになります。	
\\	このビルは 
\\	階[かい] 建[だ]てになります。	階建て=かいだて= 
\\	既に確立されたビジネスの運営にも、新たなアイデアは必要です。	
\\	既[すで]に 確立[かくりつ]されたビジネスの 運営[うんえい]にも、 新[あら]たなアイデアは 必要[ひつよう]です。	確立=かくりつ= 
\\	運営=うんえい= 
\\	その事業がどのように運営されているのか知りたい。	
\\	その 事業[じぎょう]がどのように 運営[うんえい]されているのか 知[し]りたい。	運営=うんえい= 
\\	その公園は約9000平方キロの面積を持つ。	
\\	その 公園[こうえん]は 約[やく] 
\\	平方キロ[へいほうきろ]の 面積[めんせき]を 持[も]つ。	面積=めんせき= 
\\	平方=へいほう= 
\\	中国に比べると、日本は面積の小さな国である。	
\\	中国[ちゅうごく]に 比[くら]べると、日本は 面積[めんせき]の 小[ちい]さな国である。	面積=めんせき= 
\\	あなたがもっと長く滞在できないのは残念です。	
\\	あなたがもっと 長[なが]く 滞在[たいざい]できないのは 残念[ざんねん]です。	滞在=たいざい= 
\\	友人のアパートに先週、泥棒が入った。	
\\	友人[ゆうじん]のアパートに 先週[せんしゅう]、 泥棒[どろぼう]が 入[はい]った。	
\\	警察は自動車泥棒を捕まえた。	
\\	警察[けいさつ]は 自動車[じどうしゃ] 泥棒[どろぼう]を 捕[つか]まえた。	捕まえる= 
\\	店主はあと少しのところまで追いつめたが、結局泥棒を取り逃がした。	
\\	店主[てんしゅ]はあと 少[すこ]しのところまで 追[お]いつめたが、 結局[けっきょく] 泥棒[どろぼう]を 取り逃[とりに]がした。	追いつめる= 
\\	取り逃がす= 
\\	その居酒屋の厨房から火が出て、あっという間に燃え広がった。	
\\	その 居酒屋[いざかや]の 厨房[ちゅうぼう]から 火[ひ]が 出[で]て、あっという 間[ま]に 燃え広[もえひろ]がった。	厨房=ちゅうぼう= 
\\	昨日の雨で薪が湿っているので、なかなか火がつかない。	
\\	昨日[きのう]の 雨[あめ]で 薪[たきぎ]が 湿[しめ]っているので、なかなか 火[ひ]がつかない。	薪=まき= 
\\	湿る=しめる= 
\\	ろうそくに火をつけてください。	
\\	ろうそくに 火[ひ]をつけてください。	蝋燭=ろうそく= 
\\	花火をした後は、火が完全に消えているかどうか必ず確かめること。	
\\	花火[はなび]をした 後[のち]は、 火[ひ]が 完全[かんぜん]に 消[き]えているかどうか 必[かなら]ず 確[たし]かめること。	
\\	消防隊、一時間後にようやく火を消し止めた。	
\\	消防[しょうぼう] 隊[たい]、一 時間[じかん] 後[ご]にようやく 火[ひ]を 消し止[けしと]めた。	消防隊=しょうぼうたい= 
\\	その手の詐欺の被害にあっているのは、ほとんどが高齢者だ。	
\\	その 手[て]の 詐欺[さぎ]の 被害[ひがい]にあっているのは、ほとんどが 高齢[こうれい] 者[しゃ]だ。	詐欺=さぎ= 
\\	先月の台風は、中国地方に大きな被害をもたらした。	
\\	先月[せんげつ]の 台風[たいふう]は、 中国[ちゅうごく] 地方[ちほう]に 大[おお]きな 被害[ひがい]をもたらした。	
\\	すぐにクレジットカードを止める手続を取ったので、被害は最小限に食い止められた。	
\\	すぐにクレジットカードを 止[と]める 手続[てつづき]を 取[と]ったので、 被害[ひがい]は 最小限[さいしょうげん]に 食い止[くいと]められた。	食い止める= 
\\	天井から水が漏れています。	
\\	天井[てんじょう]から 水[みず]が 漏[も]れています。	
\\	地下の水道管が破裂して、道路に水があふれた。	
\\	地下[ちか]の 水道[すいどう] 管[かん]が 破裂[はれつ]して、 道路[どうろ]に 水[みず]があふれた。	水道管= 
\\	破裂=はれつ= 
\\	あふれる= 
\\	あふれた水が引くまで、ほぼ一日かかった。	
\\	あふれた 水[みず]が 引[ひ]くまで、ほぼ一 日[にち]かかった。	あふれる= 
\\	私の留守中に何か問題が起こったら、この番号に電話してください。	
\\	私[わたし]の 留守[るす] 中[ちゅう]に 何[なに]か 問題[もんだい]が 起[お]こったら、この 番号[ばんごう]に 電話[でんわ]してください。	
\\	彼は何か問題を起こして、会社を首になったといううわさだ。	
\\	彼[かれ]は 何[なに]か 問題[もんだい]を 起[お]こして、 会社[かいしゃ]を 首[くび]になったといううわさだ。	
\\	その問題を解決するには、プロの弁護士に相談したほうがいい。	
\\	その 問題[もんだい]を 解決[かいけつ]するには、プロの 弁護士[べんごし]に 相談[そうだん]したほうがいい。	
\\	その程度の失敗は、全然問題にならない。	
\\	その 程度[ていど]の 失敗[しっぱい]は、 全然[ぜんぜん] 問題[もんだい]にならない。	
\\	私が誰からその話を聞いたかは、問題ではない。	
\\	私が 誰[だれ]からその 話[はなし]を 聞[き]いたかは、 問題[もんだい]ではない。	
\\	その会社に投資するのは、かなりのリスクを伴う。	
\\	その 会社[かいしゃ]に 投資[とうし]するのは、かなりのリスクを 伴[ともな]う。	投資=とうし= 
\\	それなりのリスクを背負う覚悟は、できていますか?	
\\	それなりのリスクを 背負[せお]う 覚悟[かくご]は、できていますか?	それなり= 
\\	覚悟=かくご= 
\\	それほどよく知らない人と一緒に事業を始めるのは、あまりにもリスクが大きい。	
\\	それほどよく 知[し]らない 人[ひと]と 一緒[いっしょ]に 事業[じぎょう]を 始[はじ]めるのは、あまりにもリスクが 大[おお]きい。	
\\	急がば回れ。	
\\	急[いそ]がば 回[まわ]れ。	
\\	「急がば回れ」ということわざは逆説のように聞こえる。	
\\	急[いそ]がば 回[まわ]れ」ということわざは 逆説[ぎゃくせつ]のように 聞[き]こえる。	
\\	彼にゴマをするな。	
\\	彼[かれ]にゴマをするな。	ごまをする= 
\\	私は日本人の友人からアパートを借りています。	
\\	私は日本人の 友人[ゆうじん]からアパートを 借[か]りています。	
\\	住み慣れた家を出て、小さなアパートに引っ越した。	
\\	住み慣[すみな]れた 家[いえ]を 出[で]て、 小[ちい]さなアパートに 引っ越[ひっこ]した。	
\\	先週彼を訪ねたが、既にアパートを引き払った後だった。	
\\	先週[せんしゅう] 彼[かれ]を 訪[たず]ねたが、 既[すで]にアパートを 引き払[ひきはら]った 後[のち]だった。	訪ねる=たずねる= 
\\	~引き払う= 
\\	もっと職場に近いところで、アパートを探すことにした。	
\\	もっと 職場[しょくば]に 近[ちか]いところで、アパートを 探[さが]すことにした。	
\\	その選手はアーセナルと契約を結んだ。	
\\	その 選手[せんしゅ]はアーセナルと 契約[けいやく]を 結[むす]んだ。	契約=けいやく= 
\\	この契約は、毎年自動的に更新される。	
\\	この 契約[けいやく]は、 毎年[まいとし] 自動的[じどうてき]に 更新[こうしん]される。	
\\	この部屋の賃貸契約は、来月切れる。	
\\	この 部屋[へや]の 賃貸[ちんたい] 契約[けいやく]は、 来月[らいげつ] 切[き]れる。	賃貸=ちんたい= 
\\	契約=けいやく= 
\\	直に会って話したい。	
\\	直[じか]に 会[あ]って 話[はな]したい。	直に=じかに= 
\\	地球温暖化は確かに起きている。	
\\	地球[ちきゅう] 温暖[おんだん] 化[か]は 確[たし]かに 起[お]きている。	
\\	あなたの全てが私の関心事です。	
\\	あなたの 全[すべ]てが 私[わたし]の 関心事[かんしんじ]です。	関心事= 
\\	夫は妻の関心事について認識不足である。	
\\	夫[おっと]は 妻[つま]の 関心[かんしん] 事[ごと]について 認識[にんしき] 不足[ふそく]である。	関心事= 
\\	犯罪の増加が国民の関心事です。	
\\	犯罪[はんざい]の 増加[ぞうか]が 国民[こくみん]の 関心事[かんしんじ]です。	関心事= 
\\	普段は、コンビニで光熱費を払っています。	
\\	普段[ふだん]は、コンビニで 光熱[こうねつ] 費[ひ]を 払[はら]っています。	光熱費=こう ねつ ひ= 
\\	石油の値上がりの影響で、来月から光熱費が上がる。	
\\	石油[せきゆ]の 値上[ねあ]がりの 影響[えいきょう]で、 来月[らいげつ]から 光熱[こうねつ] 費[ひ]が 上[あ]がる。	光熱費=こう ねつ ひ= 
\\	おとといこの町に引っ越してきて、今日、市役所で住民登録した。	
\\	おとといこの 町[まち]に 引っ越[ひっこ]してきて、 今日[きょう]、 市役所[しやくしょ]で 住民[じゅうみん] 登録[とうろく]した。	市役所= 
\\	たばこには高い税金がかかっている。	
\\	たばこには 高[たか]い 税金[ぜいきん]がかかっている。	
\\	宝くじの当選金には、税金はかかりません。	
\\	宝[たから]くじの 当選[とうせん] 金[きん]には、 税金[ぜいきん]はかかりません。	当選= 
\\	その俳優は稼いだ金額の6割近くを、税金に持っていかれると言った。	
\\	その 俳優[はいゆう]は 稼[かせ]いだ 金額[きんがく]の 6割[ろくわり] 近[ちか]くを、 税金[ぜいきん]に 持[も]っていかれると 言[い]った。	
\\	今日の午後2時から4時の間に、宅配便がくる予定だ。	
\\	今日[きょう]の 午後[ごご] 2時[にじ]から 4時[よじ]の 間[あいだ]に、 宅配[たくはい] 便[びん]がくる 予定[よてい]だ。	
\\	一定の期間、電気代を滞納していると、電気を止められる。	
\\	一定[いってい]の 期間[きかん]、 電気[でんき] 代[だい]を 滞納[たいのう]していると、 電気[でんき]を 止[と]められる。	一定= 
\\	滞納=たいのう= 
\\	就労ビザを取るためには、いろいろな条件を満たしていなければならない。	
\\	就労[しゅうろう]ビザを 取[と]るためには、いろいろな 条件[じょうけん]を 満[み]たしていなければならない。	
\\	中国への入国ビザを申請しました。	
\\	中国[ちゅうごく]への 入国[にゅうこく]ビザを 申請[しんせい]しました。	
\\	来月、ビザを書き換えなければならない。	
\\	来月[らいげつ]、ビザを 書き換[かきか]えなければならない。	
\\	しまった!ビザが切れてた!	
\\	しまった!ビザが 切[き]れてた!	
\\	あらゆる面で、できるだけ費用を切り詰めるようにしないといけない。	
\\	あらゆる 面[めん]で、できるだけ 費用[ひよう]を 切り詰[きりつ]めるようにしないといけない。	費用=ひよう= 
\\	切り詰める= 
\\	その旅行の費用を一人当たり75,000円と見積もった。	
\\	その 旅行[りょこう]の 費用[ひよう]を一 人[にん] 当[あ]たり 
\\	000円[ななまんごせんえん]と 見積[みつ]もった。	費用=ひよう= 
\\	見積もる= 
\\	彼らは私たちに修理の見積もりを出しました。	
\\	彼[かれ]らは 私[わたし]たちに 修理[しゅうり]の 見積[みつ]もりを 出[だ]しました。	見積もる= 
\\	その絵は数百万ドルの価値があると見積もられます。	
\\	その 絵[え]は 数[すう] 百[ひゃく] 万[まん]ドルの 価値[かち]があると 見積[みつ]もられます。	見積もる= 
\\	ありがたいことに、かかった費用はすべて会社が負担してくれた。	
\\	ありがたいことに、かかった 費用[ひよう]はすべて 会社[かいしゃ]が 負担[ふたん]してくれた。	費用=ひよう= 
\\	フェンスの修理にかかる費用は折半することになった。	
\\	フェンスの 修理[しゅうり]にかかる 費用[ひよう]は 折半[せっぱん]することになった。	費用=ひよう= 
\\	折半=せっぱん= 
\\	愛犬の治療には、いくらかかっても費用を惜しまない。	
\\	愛犬[あいけん]の 治療[ちりょう]には、いくらかかっても 費用[ひよう]を 惜[お]しまない。	愛犬=あいけん= 
\\	治療=ちりょう= 
\\	惜しむ=おしむ= 
\\	東京にいる間は、この部屋を自由に使ってくれていいよ。	
\\	東京[とうきょう]にいる 間[ま]は、この 部屋[へや]を 自由[じゆう]に 使[つか]ってくれていいよ。	
\\	このあたりで部屋を探しているのですが、何かいい物件はありますか?	
\\	このあたりで 部屋[へや]を 探[さが]しているのですが、 何[なに]かいい 物件[ぶっけん]はありますか?	物件=ぶっけん= 
\\	私は中国人一人、カナダ人二人と部屋をシェアしている。	
\\	私[わたし]は 中国人[ちゅうごくじん]一 人[にん]、カナダ 人[じん]二 人[にん]と 部屋[へや]をシェアしている。	
\\	車に乗るなら、保険には必ず入っておくべきだ。	
\\	車[くるま]に 乗[の]るなら、 保険[ほけん]には 必[かなら]ず 入[はい]っておくべきだ。	
\\	アメリカ人の多くは、高校生のときに運転免許を取ります。	
\\	アメリカ 人[じん]の 多[おお]くは、 高校生[こうこうせい]のときに 運転[うんてん] 免許[めんきょ]を 取[と]ります。	
\\	私は調理師の免許を持っています。	
\\	私は 調理[ちょうり] 師[し]の 免許[めんきょ]を 持[も]っています。	調理師= 
\\	免許を更新するのに、わざわざ遠くの警察署まで行かなければならない。	
\\	免許[めんきょ]を 更新[こうしん]するのに、わざわざ 遠[とお]くの 警察[けいさつ] 署[しょ]まで 行[い]かなければならない。	
\\	不動産屋の人が大家と家賃の交渉をしてくれた。	
\\	不動産[ふどうさん] 屋[や]の 人[ひと]が 大家[おおや]と 家賃[やちん]の 交渉[こうしょう]をしてくれた。	不動産屋=ふどうさんや= 
\\	彼は2か月分の家賃を滞納したまま、突然行方をくらました。	
\\	彼[かれ]は2か 月[げつ] 分[ぶん]の 家賃[やちん]を 滞納[たいのう]したまま、 突然[とつぜん] 行方[ゆくえ]をくらました。	滞納=たいのう= 
\\	行方をくらます= 
\\	この協定は公正取引の重要性を示している。	
\\	この 協定[きょうてい]は 公正[こうせい] 取引[とりひき]の 重要[じゅうよう] 性[せい]を 示[しめ]している。	協定=きょうてい= 
\\	公正取引=こうせいとりひき= 
\\	人々は公正で民主的な選挙を望んでいた。	
\\	人々[ひとびと]は 公正[こうせい]で 民主[みんしゅ] 的[てき]な 選挙[せんきょ]を 望[のぞ]んでいた。	公正=こうせい= 
\\	そのモデルは脚が長くてすらりとしている。	
\\	そのモデルは 脚[あし]が 長[なが]くてすらりとしている。	すらり= 
\\	彼は足が大きいから、この靴は入らないだろう。	
\\	彼[かれ]は 足[あし]が 大[おお]きいから、この 靴[くつ]は 入[はい]らないだろう。	
\\	彼は背丈の割に、足が小さいほうだ。	
\\	彼[かれ]は 背丈[せたけ]の 割[わり]に、 足[あし]が 小[ちい]さいほうだ。	背丈=せたけ= 
\\	彼女は脚が太いことを気にしている。	
\\	彼女[かのじょ]は 脚[あし]が 太[ふと]いことを 気[き]にしている。	
\\	私もあなたみたいにウエストが細かったらなあ。	
\\	私[わたし]もあなたみたいにウエストが 細[ほそ]かったらなあ。	
\\	毎日の運動の成果が出ているのか、だんだんウエストが締まってきた。	
\\	毎日[まいにち]の 運動[うんどう]の 成果[せいか]が 出[で]ているのか、だんだんウエストが 締[し]まってきた。	成果=せいか= 
\\	締まる= 
\\	髪が伸びてきたから、そろそろ散髪に行かなくちゃ。	
\\	髪[かみ]が 伸[の]びてきたから、そろそろ 散髪[さんぱつ]に 行[い]かなくちゃ。	散髪=さんぱつ= 
\\	夏向きに髪を短くしてもらいたいのです。	
\\	夏向[なつむ]きに 髪[かみ]を 短[みじか]くしてもらいたいのです。	
\\	私は髪が多いほうだ。	
\\	私は 髪[かみ]が 多[おお]いほうだ。	
\\	この髪型は、髪が硬いとうまくいかない。	
\\	この 髪型[かみがた]は、 髪[かみ]が 硬[かた]いとうまくいかない。	
\\	私の姉はここ数年で、ずいぶんしわが増えてきた。	
\\	私[わたし]の 姉[あね]はここ 数[すう] 年[ねん]で、ずいぶんしわが 増[ふ]えてきた。	
\\	そのアイドルは、抜群にスタイルがいい。	
\\	そのアイドルは、 抜群[ばつぐん]にスタイルがいい。	抜群=ばつぐん= 
\\	スタイルを保つ秘訣は何ですか?	
\\	スタイルを 保[たも]つ 秘訣[ひけつ]は 何[なに]ですか?	秘訣=ひけつ= 
\\	秘訣は時間をかけることです。	
\\	秘訣[ひけつ]は 時間[じかん]をかけることです。	秘訣=ひけつ= 
\\	ギャリーさんは日本人と商売をする上での秘訣を一つだけ教えてくれた。	
\\	ギャリーさんは日本人と 商売[しょうばい]をする 上[うえ]での 秘訣[ひけつ]を一つだけ教えてくれた。	商売=しょうばい= 
\\	秘訣=ひけつ= 
\\	パチンコで勝ち続ける秘訣は?	
\\	パチンコで 勝[か]ち 続[つづ]ける 秘訣[ひけつ]は?	秘訣=ひけつ= 
\\	ケイトは私が思っていたより、背が高かった。	
\\	ケイトは 私[わたし]が 思[おも]っていたより、 背[せ]が 高[たか]かった。	
\\	弟は昨年の春から、ずいぶん背が伸びました。	
\\	弟[おとうと]は 昨年[さくねん]の 春[はる]から、ずいぶん 背[せ]が 伸[の]びました。	
\\	兄は父に似て、目が大きい。	
\\	兄[あに]は 父[ちち]に 似[に]て、 目[め]が 大[おお]きい。	
\\	彼女は目が一重だということを気にしている。	
\\	彼女[かのじょ]は 目[め]が 一重[ひとえ]だということを 気[き]にしている。	
\\	この子はとても指が長いから、ピアニストになるといいよ。	
\\	この 子[こ]はとても 指[ゆび]が 長[なが]いから、ピアニストになるといいよ。	
\\	今度の部長は、非常に頭が切れると評判だ。	
\\	今度[こんど]の 部長[ぶちょう]は、 非常[ひじょう]に 頭[あたま]が 切[き]れると 評判[ひょうばん]だ。	
\\	料理人として腕が立つようになるまでには、何年もの修業が必要だ。	
\\	料理人[りょうりにん]として 腕[うで]が 立[た]つようになるまでには、 何[なん] 年[ねん]もの 修業[しゅうぎょう]が 必要[ひつよう]だ。	腕が立つ= 
\\	修業= 
\\	ビデオの撮影なら、腕に覚えがあります。	
\\	ビデオの 撮影[さつえい]なら、 腕[うで]に 覚[おぼ]えがあります。	腕の覚えがある= 
\\	腕のいい歯医者を私に紹介してくれませんか?	
\\	腕[うで]のいい 歯医者[はいしゃ]を私に 紹介[しょうかい]してくれませんか?	腕がいい= 
\\	彼は勘が鋭いから、そんなうそは通らないよ。	
\\	彼[かれ]は 勘[かん]が 鋭[するど]いから、そんなうそは 通[とお]らないよ。	勘=かん= 
\\	鋭い=するどい= 
\\	勘が鋭い= 
\\	君は相変わらず勘が鈍いなあ。	
\\	君[きみ]は 相変[あいか]わらず 勘[かん]が 鈍[にぶ]いなあ。	勘=かん= 
\\	鈍い=にぶい= 
\\	勘が鈍い= 
\\	君の勘が当たったようだ。やっぱりあいつはぺてん師だったよ。	
\\	君[きみ]の 勘[かん]が 当[あ]たったようだ。やっぱりあいつはぺてん 師[し]だったよ。	勘=かん= 
\\	勘が当たる= 
\\	ぺてん師= 
\\	その日に限って、私は崩落したトンネルを通らなかったのです。何かがおかしいという勘が働いたとしか思えません。	
\\	その 日[ひ]に 限[かぎ]って、 私[わたし]は 崩落[ほうらく]したトンネルを 通[とお]らなかったのです。 何[なに]かがおかしいという 勘[かん]が 働[はたら]いたとしか 思[おも]えません。	崩落=ほうらく= 
\\	勘=かん= 
\\	勘が働く= 
\\	彼の声は高くて、神経に障る。	
\\	彼[かれ]の 声[こえ]は 高[たか]くて、 神経[しんけい]に 障[さわ]る。	神経に障る= 
\\	彼女は女性にしては、かなり声が低い。	
\\	彼女[かのじょ]は 女性[じょせい]にしては、かなり 声[ごえ]が 低[ひく]い。	
\\	ユカは合唱部で一番声がきれいだ。	
\\	ユカは 合唱[がっしょう] 部[ぶ]で 一番[いちばん] 声[ごえ]がきれいだ。	合唱=がっしょう= 
\\	この地方の人々は、一般に信仰心が篤い。	
\\	この 地方[ちほう]の 人々[ひとびと]は、 一般[いっぱん]に 信仰[しんこう] 心[しん]が 篤[あつ]い。	篤い=あつい= 
\\	信仰心=しんこうしん= 
\\	私は信念を曲げてまで、その地位を得たいとは思いません	
\\	私は 信念[しんねん]を 曲[ま]げてまで、その 地位[ちい]を 得[え]たいとは 思[おも]いません	信念= 
\\	信念を曲げる= 
\\	我々が何といおうと、彼は自分の信念を貫くだろう。	
\\	我々[われわれ]が 何[なん]といおうと、 彼[かれ]は 自分[じぶん]の 信念[しんねん]を 貫[つらぬ]くだろう。	
\\	彼女は夫への愛を貫いた。	
\\	彼女[かのじょ]は 夫[おっと]への 愛[あい]を 貫[つらぬ]いた。	
\\	一条の光が闇を貫いた。	
\\	一 条[じょう]の 光[ひかり]が 闇[やみ]を 貫[つらぬ]いた。	闇=やみ= 
\\	どの店も売り上げを伸ばそうと知恵をしぼっている。	
\\	どの 店[みせ]も 売り上[うりあ]げを 伸[の]ばそうと 知恵[ちえ]をしぼっている。	知恵を絞る=ちえ を しぼる= 
\\	そんな奇抜なアイディアを思いつくなんて、よく知恵が働くものだね。	
\\	そんな 奇抜[きばつ]なアイディアを 思[おも]いつくなんて、よく 知恵[ちえ]が 働[はたら]くものだね。	奇抜な=きばつな= 
\\	知恵が働く= 
\\	入浴中にいい知恵が浮かんだ。	
\\	入浴[にゅうよく] 中[ちゅう]にいい 知恵[ちえ]が 浮[う]かんだ。	入浴=にゅうよく= 
\\	知恵が浮かぶ= 
\\	甥はまだ5歳だというのに、よく知恵が回る。	
\\	甥[おい]はまだ5 歳[さい]だというのに、よく 知恵[ちえ]が 回[まわ]る。	甥=おい= 
\\	知恵が回る= 
\\	彼は、日本の伝統工芸に関する知識が豊富だ。	
\\	彼[かれ]は、日本の 伝統[でんとう] 工芸[こうげい]に 関[かん]する 知識[ちしき]が 豊富[ほうふ]だ。	工芸=こうげい= 
\\	この本は、科学の知識が乏しい人のために、やさしく書かれている。	
\\	この 本[ほん]は、 科学[かがく]の 知識[ちしき]が 乏[とぼ]しい 人[ひと]のために、やさしく 書[か]かれている。	
\\	コンピューターに関する知識はありますか。	
\\	コンピューターに 関[かん]する 知識[ちしき]はありますか。	
\\	その問題について、早まった判断を下すべきではありません。	
\\	その 問題[もんだい]について、 早[はや]まった 判断[はんだん]を 下[くだ]すべきではありません。	判断を下す= 
\\	ご判断に任せます。	
\\	ご 判断[はんだん]に 任[まか]せます。	
\\	景気予測はなかなか判断が難しい。	
\\	景気[けいき] 予測[よそく]はなかなか 判断[はんだん]が 難[むずか]しい。	
\\	私たちは彼を言葉の癖から関西人と判断した。	
\\	私[わたし]たちは 彼[かれ]を 言葉[ことば]の 癖[くせ]から 関西[かんさい] 人[じん]と 判断[はんだん]した。	
\\	彼は頭をかく癖がある。	
\\	彼[かれ]は 頭[あたま]をかく 癖[くせ]がある。	
\\	その薬は飲みつけると癖になる。	
\\	その 薬[くすり]は 飲[の]みつけると 癖[くせ]になる。	
\\	判断に迷ったら、いつでも相談してください。	
\\	判断[はんだん]に 迷[まよ]ったら、いつでも 相談[そうだん]してください。	
\\	彼の仕事の申し出を受けるべきかどうか、判断がつかない。	
\\	彼[かれ]の 仕事[しごと]の 申し出[もうしで]を 受[う]けるべきかどうか、 判断[はんだん]がつかない。	
\\	彼は物事の理解が速いので、どんな仕事も任せられる。	
\\	彼[かれ]は 物事[ものごと]の 理解[りかい]が 速[はや]いので、どんな 仕事[しごと]も 任[まか]せられる。	
\\	私たちの会社は子育てに理解があり、女性にとっては働きやすい。	
\\	私たちの会社は 子育[こそだ]てに 理解[りかい]があり、女性にとっては 働[はたら]きやすい。	
\\	リサイクルの重要性に、理解がない人が多すぎる。	
\\	リサイクルの 重要[じゅうよう] 性[せい]に、 理解[りかい]がない 人[ひと]が 多[おお]すぎる。	
\\	彼は信じられないほど頭の回転が速い。私にはとてもついて行けない。	
\\	彼[かれ]は 信[しん]じられないほど 頭[あたま]の 回転[かいてん]が 速[はや]い。 私[わたし]にはとてもついて 行[い]けない。	
\\	父は頭が固いので、説得するのにいつも苦労する。	
\\	父[ちち]は 頭[あたま]が 固[かた]いので、 説得[せっとく]するのにいつも 苦労[くろう]する。	
\\	こんな意見をブログに載せるなんて、彼はよほど頭が古い人なんだろう。	
\\	こんな 意見[いけん]をブログに 載[の]せるなんて、 彼[かれ]はよほど 頭[あたま]が 古[ふる]い 人[ひと]なんだろう。	よほど= 
\\	彼は酒を飲むと、時々気が大きくなる。	
\\	彼[かれ]は 酒[さけ]を 飲[の]むと、 時々[ときどき] 気[き]が 大[おお]きくなる。	気が大きい= 
\\	その話し方から、彼は気が小さいと分かる。	
\\	その 話し方[はなしかた]から、 彼[かれ]は 気[き]が 小[ちい]さいと 分[わ]かる。	
\\	彼女は気が短いのが、唯一の欠点だ。	
\\	彼女[かのじょ]は 気[き]が 短[みじか]いのが、 唯一[ゆいいつ]の 欠点[けってん]だ。	
\\	リカはいかにも気が強そうな顔をしている。	
\\	リカはいかにも 気[き]が 強[つよ]そうな 顔[かお]をしている。	
\\	彼は一見気が弱そうだが、はっきりと自分の意見を言う。	
\\	彼[かれ]は 一見[いっけん] 気[き]が 弱[よわ]そうだが、はっきりと 自分[じぶん]の 意見[いけん]を 言[い]う。	
\\	あなたは本当に気がきかない人ね。雨が降り出したら、洗濯物くらい取り込んでよ。	
\\	あなたは 本当[ほんとう]に 気[き]がきかない 人[ひと]ね。 雨[あめ]が 降り出[ふりだ]したら、 洗濯[せんたく] 物[もの]くらい 取り込[とりこ]んでよ。	
\\	気の置けない友人たちを呼んで、パーティーをした。	
\\	気[き]の 置[お]けない 友人[ゆうじん]たちを 呼[よ]んで、パーティーをした。	気が置けない= 
\\	彼は口がうまいから、あまり信用するな。	
\\	彼[かれ]は 口[くち]がうまいから、あまり 信用[しんよう]するな。	
\\	彼に相談してみたらどう?彼は口がかたいよ。	
\\	彼[かれ]に 相談[そうだん]してみたらどう? 彼[かれ]は 口[くち]がかたいよ。	口がかたい= 
\\	遅かれ早かれ、彼は尻に敷かれた夫の典型になるだろう。	
\\	遅[おそ]かれ 早[はや]かれ、 彼[かれ]は 尻[しり]に 敷[し]かれた 夫[おっと]の 典型[てんけい]になるだろう。	尻に敷かれている= 
\\	典型=てんけい= 
\\	中には尻の長い人もいる。	
\\	中[なか]には 尻[しり]の 長[なが]い 人[ひと]もいる。	尻が長い= 
\\	彼は私の方へ尻を向けて座った。	
\\	彼[かれ]は 私[わたし]の 方[ほう]へ 尻[しり]を 向[む]けて 座[すわ]った。	
\\	彼女は男勝りで夫を尻に敷くタイプだ。	
\\	彼女[かのじょ]は 男勝[おとこまさ]りで 夫[おっと]を 尻[しり]に 敷[し]くタイプだ。	男勝り=おとこまさり= 
\\	彼女は性格がいいので、みんなから愛されている。	
\\	彼女[かのじょ]は 性格[せいかく]がいいので、みんなから 愛[あい]されている。	
\\	トムとジェイクは双子の兄弟が、性格はずいぶん違う。	
\\	トムとジェイクは 双子[ふたご]の 兄弟[きょうだい]が、 性格[せいかく]はずいぶん 違[ちが]う。	双子=ふたご= 
\\	その問題では皆の意見がまちまちだ。	
\\	その 問題[もんだい]では 皆[みな]の 意見[いけん]がまちまちだ。	
\\	過ぎた事を嘆くのはやめましょう。	
\\	過[す]ぎた 事[こと]を 嘆[なげ]くのはやめましょう。	過ぎた事= 
\\	どんなに嘆いても死んだ人は戻ってこない。	
\\	どんなに 嘆[なげ]いても 死[し]んだ 人[ひと]は 戻[もど]ってこない。	
\\	さぞお腹がすいたことでしょう。	
\\	さぞお 腹[なか]がすいたことでしょう。	さぞ、さぞかし、さぞや= 
\\	さぞかし喜ぶことでしょう。	
\\	さぞかし 喜[よろこ]ぶことでしょう。	さぞ、さぞかし、さぞや= 
\\	この資料を全部読むのはしんどい。	
\\	この 資料[しりょう]を 全部[ぜんぶ] 読[よ]むのはしんどい。	しんどい= 
\\	開演時間は午後7時です。	
\\	開演[かいえん] 時間[じかん]は 午後[ごご]7 時[じ]です。	開演時間= 
\\	この問題を理解するためには若干の専門的知識が必要である。	
\\	この 問題[もんだい]を 理解[りかい]するためには 若干[じゃっかん]の 専門[せんもん] 的[てき] 知識[ちしき]が 必要[ひつよう]である。	若干=じゃっかん= 
\\	委員会の提案は若干手直しする必要があろう。	
\\	委員[いいん] 会[かい]の 提案[ていあん]は 若干[じゃっかん] 手直[てなお]しする 必要[ひつよう]があろう。	若干=じゃっかん= 
\\	きちんと期限を決めておかないとずるずるになる。	
\\	きちんと 期限[きげん]を 決[き]めておかないとずるずるになる。	ずるずる= 
\\	性格が明るいと、人間関係を築く上で大いにプラスになると思います。	
\\	性格[せいかく]が 明[あか]るいと、 人間[にんげん] 関係[かんけい]を 築[きず]く 上[うえ]で 大[おお]いにプラスになると 思[おも]います。	築く=きずく= 
\\	新しい上司とは、どうも性格が合わない。	
\\	新[あたら]しい 上司[じょうし]とは、どうも 性格[せいかく]が 合[あ]わない。	
\\	この機械の操作は難しくて、私の手に余る。	
\\	この 機械[きかい]の 操作[そうさ]は 難[むずか]しくて、 私[わたし]の 手[て]に 余[あま]る。	操作=そうさ= 
\\	手に余る= 
\\	その問題は私の手に余る。	
\\	その 問題[もんだい]は 私[わたし]の 手[て]に 余[あま]る。	手に余る= 
\\	私は母の手に余るわんぱくな子どもだった。	
\\	私[わたし]は 母[はは]の 手[て]に 余[あま]るわんぱくな 子[こ]どもだった。	手に余る= 
\\	腕白=わんぱく= 
\\	自分の手に余るようなことにはかかわりたくない。	
\\	自分[じぶん]の 手[て]に 余[あま]るようなことにはかかわりたくない。	手に余る= 
\\	かかわる= 
\\	あの少年はやんちゃ過ぎて、私の手に負えない。	
\\	あの 少年[しょうねん]はやんちゃ 過[す]ぎて、 私[わたし]の 手[て]に 負[お]えない。	やんちゃ(な)= 
\\	手に負えない= 
\\	だんだん手に負えなくなってきた。	
\\	だんだん 手[て]に 負[お]えなくなってきた。	手に負えない= 
\\	時として私たちには手に負えないことがあります。	
\\	時[とき]として 私[わたし]たちには 手[て]に 負[お]えないことがあります。	時として= 
\\	手に負えない= 
\\	状況は全く手に負えなくなっているようです。	
\\	状況[じょうきょう]は 全[まった]く 手[て]に 負[お]えなくなっているようです。	手に負えない= 
\\	あいつはいつも学歴を鼻にかけるから嫌いだ。	
\\	あいつはいつも 学歴[がくれき]を 鼻[はな]にかけるから 嫌[きら]いだ。	鼻にかける= 
\\	彼は、車のこととなると、ひどく知識を鼻にかける人です。	
\\	彼[かれ]は、 車[くるま]のこととなると、ひどく 知識[ちしき]を 鼻[はな]にかける 人[ひと]です。	鼻にかける= 
\\	彼は金髪美女を前にして、鼻の下が長くなっていた。	
\\	彼[かれ]は 金髪[きんぱつ] 美女[びじょ]を 前[まえ]にして、 鼻[はな]の 下[した]が 長[なが]くなっていた。	鼻の下が長い= 
\\	彼女はおいしいレストランには、鼻が利く。	
\\	彼女[かのじょ]はおいしいレストランには、 鼻[はな]が 利[き]く。	鼻が利く= 
\\	あの人はいつも親切そうにしているのが鼻につく。	
\\	あの 人[ひと]はいつも 親切[しんせつ]そうにしているのが 鼻[はな]につく。	鼻につく= 
\\	私の兄は趣味で絵を描くので、絵画には目がきく。	
\\	私の 兄[あに]は 趣味[しゅみ]で 絵[え]を 描[か]くので、 絵画[かいが]には 目[め]がきく。	絵画=かいが= 
\\	彼女は甘い物、特にチョコレートには目がない。	
\\	彼女[かのじょ]は 甘[あま]い 物[もの]、 特[とく]にチョコレートには 目[め]がない。	目がない= 
\\	ボブが日本酒に目がないのは有名だ。	
\\	ボブが 日本[にっぽん] 酒[しゅ]に 目[め]がないのは 有名[ゆうめい]だ。	目がない= 
\\	彼はシーフードに目がない。	
\\	彼[かれ]はシーフードに 目[め]がない。	目がない= 
\\	甘いものには目がない。	
\\	甘[あま]いものには 目[め]がない。	目がない= 
\\	この本の価値が分かるとは、なかなかお目が高いですね。	
\\	この 本[ほん]の 価値[かち]が 分[わ]かるとは、なかなかお 目[め]が 高[たか]いですね。	目が高い= 
\\	我が社の新製品は、目が肥えた日本の消費者を満足させるでしょう。	
\\	我[わ]が 社[しゃ]の 新[しん] 製品[せいひん]は、 目[め]が 肥[こ]えた 日本[にっぽん]の 消費[しょうひ] 者[しゃ]を 満足[まんぞく]させるでしょう。	目が肥えている= 
\\	彼女は服を買うことになると、目が肥えていた。	
\\	彼女[かのじょ]は 服[ふく]を 買[か]うことになると、 目[め]が 肥[こ]えていた。	目が肥えている= 
\\	彼からのメールを読んで猛烈に頭にきたので、返信をしなかった。	
\\	彼[かれ]からのメールを 読[よ]んで 猛烈[もうれつ]に 頭[あたま]にきたので、 返信[へんしん]をしなかった。	猛烈=もうれつ= 
\\	頭にくる= 
\\	あいつが君にあんな口の利き方をすると頭にくる。	
\\	あいつが 君[きみ]にあんな 口[くち]の 利[き]き 方[かた]をすると 頭[あたま]にくる。	口の利き方= 
\\	頭にくる= 
\\	ママには本当に頭にくる。	
\\	ママには 本当[ほんとう]に 頭[あたま]にくる。	頭にくる= 
\\	来週の就職面接試験のことを考えると、頭が痛い。	
\\	来週[らいしゅう]の 就職[しゅうしょく] 面接[めんせつ] 試験[しけん]のことを 考[かんが]えると、 頭[あたま]が 痛[いた]い。	
\\	彼はいつも偉そうなことを言っているくせに、部長に頭が上がらない。	
\\	彼[かれ]はいつも 偉[えら]そうなことを 言[い]っているくせに、 部長[ぶちょう]に 頭[あたま]が 上[あ]がらない。	頭が上がらない= 
\\	この問題に関しては検討中です。	
\\	この 問題[もんだい]に 関[かん]しては 検討[けんとう] 中[ちゅう]です。	
\\	この論証は、詳細な検討に耐えるものではありません。	
\\	この 論証[ろんしょう]は、 詳細[しょうさい]な 検討[けんとう]に 耐[た]えるものではありません。	論証= 
\\	耐える= 
\\	その案は検討してみる価値があります。	
\\	その 案[あん]は 検討[けんとう]してみる 価値[かち]があります。	
\\	耐えられないほど頭が痛い。	
\\	耐[た]えられないほど 頭[あたま]が 痛[いた]い。	耐える= 
\\	彼女には、とても耐えられない。	
\\	彼女[かのじょ]には、とても 耐[た]えられない。	耐える= 
\\	日本の高層建築は地震に耐えられますか?	
\\	日本[にっぽん]の 高層[こうそう] 建築[けんちく]は 地震[じしん]に 耐[た]えられますか?	高層建築=こうそうけんちく= 
\\	耐える= 
\\	私はこのような無礼には耐えられません。	
\\	私[わたし]はこのような 無礼[ぶれい]には 耐[た]えられません。	無礼=ぶれい= 
\\	耐える= 
\\	私はたばこの煙の匂いに耐えられない。	
\\	私はたばこの 煙[けむり]の 匂[にお]いに 耐[た]えられない。	耐える= 
\\	まだ二人は入るだけの余裕がある。	
\\	まだ二 人[にん]は 入[はい]るだけの 余裕[よゆう]がある。	余裕=よゆう= 
\\	発車にはまだ30分の余裕がある。	
\\	発車[はっしゃ]にはまだ30 分[ぶん]の 余裕[よゆう]がある。	余裕=よゆう= 
\\	もう一円の余裕もない。	
\\	もう一 円[えん]の 余裕[よゆう]もない。	余裕=よゆう= 
\\	テスト前なのにテレビ見てるなんてずいぶん余裕だね。	
\\	テスト 前[まえ]なのにテレビ 見[み]てるなんてずいぶん 余裕[よゆう]だね。	余裕=よゆう= 
\\	あれだけの男はめったにいない。	
\\	あれだけの 男[おとこ]はめったにいない。	めったに= 
\\	あれだけ勉強したのに合格できなかった。	
\\	あれだけ 勉強[べんきょう]したのに 合格[ごうかく]できなかった。	
\\	彼は、めったに残業しない。	
\\	彼[かれ]は、めったに 残業[ざんぎょう]しない。	めったに= 
\\	彼はいつも忙しくてめったに家にいません。	
\\	彼[かれ]はいつも 忙[いそが]しくてめったに 家[いえ]にいません。	めったに= 
\\	彼はめったに家で食事をしない。	
\\	彼[かれ]はめったに 家[いえ]で 食事[しょくじ]をしない。	めったに= 
\\	彼女はめったにスカートをはかない。	
\\	彼女[かのじょ]はめったにスカートをはかない。	めったに= 
\\	彼女ほど広い心を持った人にはめったに出会えない。	
\\	彼女[かのじょ]ほど 広[ひろ]い 心[こころ]を 持[も]った 人[ひと]にはめったに 出会[であ]えない。	めったに= 
\\	うっかりするな!	
\\	うっかりするな!	うっかり= 
\\	標識をうっかり見過ごしてしまった。	
\\	標識[ひょうしき]をうっかり 見過[みす]ごしてしまった。	うっかり= 
\\	うっかりして切手を貼らずに手紙を出した。	
\\	うっかりして 切手[きって]を 貼[は]らずに 手紙[てがみ]を 出[だ]した。	うっかり= 
\\	彼に伝言するのをうっかり忘れてしまった。	
\\	彼[かれ]に 伝言[でんごん]するのをうっかり 忘[わす]れてしまった。	うっかり= 
\\	彼の服はぼろぼろだった。	
\\	彼[かれ]の 服[ふく]はぼろぼろだった。	
\\	引退直前、彼の肉体はぼろぼろになっていた。	
\\	引退[いんたい] 直前[ちょくぜん]、 彼[かれ]の 肉体[にくたい]はぼろぼろになっていた。	
\\	なんとなくうれしい。	
\\	なんとなくうれしい。	
\\	何となく泣きたいような気持ちだ。	
\\	何[なん]となく 泣[な]きたいような 気持[きも]ちだ。	
\\	私は何となく海にあこがれを持っていた。	
\\	私[わたし]は 何[なん]となく 海[うみ]にあこがれを 持[も]っていた。	
\\	彼女には何となく人を引き付けるものがある。	
\\	彼女[かのじょ]には 何[なん]となく 人[ひと]を 引き付[ひきつ]けるものがある。	
\\	あの人といると何となく落ち着かない。	
\\	あの 人[ひと]といると 何[なん]となく 落ち着[おちつ]かない。	
\\	どのくらいのアメリカ人が、第2国語が話せるのでしょうか。	
\\	どのくらいのアメリカ 人[じん]が、 第[だい]2 国語[こくご]が 話[はな]せるのでしょうか。	
\\	これまでに何カ国を訪れましたか?	
\\	これまでに 何[なん] カ国[かこく]を 訪[おとず]れましたか?	訪れる=おとずれる= (訪問する) 
\\	(到来する) 
\\	一日に何時間勉強しますか?	
\\	一 日[にち]に 何[なん] 時間[じかん] 勉強[べんきょう]しますか?	
\\	週に何時間テレビを見ますか?	
\\	週[しゅう]に 何[なん] 時間[じかん]テレビを 見[み]ますか?	
\\	何カ国語を話せますか?	
\\	何[なん] カ国[かこく] 語[ご]を 話[はな]せますか?	
\\	彼らに協力を仰ぐことができます。	
\\	彼[かれ]らに 協力[きょうりょく]を 仰[あお]ぐことができます。	仰ぐ= 
\\	それは私をびっくり仰天させた。	
\\	それは私をびっくり 仰天[ぎょうてん]させた。	仰天=ぎょうてん= 
\\	初めて地震を経験した時は、びっくり仰天しました。	
\\	初[はじ]めて 地震[じしん]を 経験[けいけん]した 時[とき]は、びっくり 仰天[ぎょうてん]しました。	仰天=ぎょうてん= 
\\	実践や練習こそが最大の教師である。	
\\	実践[じっせん]や 練習[れんしゅう]こそが 最大[さいだい]の 教師[きょうし]である。	実践= 
\\	彼女は、まさに実践型の人間です。	
\\	彼女[かのじょ]は、まさに 実践[じっせん] 型[がた]の 人間[にんげん]です。	実践= 
\\	クラシック音楽を演奏する時は、制約がある。	
\\	クラシック 音楽[おんがく]を 演奏[えんそう]する 時[とき]は、 制約[せいやく]がある。	制約= 
\\	それは正当な批判ではありません。	
\\	それは 正当[せいとう]な 批判[ひはん]ではありません。	正当= 
\\	テロはいかなる理由によっても決して正当化されない。	
\\	テロはいかなる 理由[りゆう]によっても 決[けっ]して 正当[せいとう] 化[か]されない。	いかなる= 
\\	正当化= 
\\	ブッシュ大統領はイラク戦争の正当性を強調しました。	
\\	ブッシュ 大統領[だいとうりょう]はイラク 戦争[せんそう]の 正当[せいとう] 性[せい]を 強調[きょうちょう]しました。	正当= 
\\	彼の懸念は正当でした。	
\\	彼[かれ]の 懸念[けねん]は 正当[せいとう]でした。	懸念= 
\\	正当= 
\\	批判が正当ならば受け入れなさい。	
\\	批判[ひはん]が 正当[せいとう]ならば 受け入[うけい]れなさい。	正当= 
\\	広告は若者に大きな影響を与える。	
\\	広告[こうこく]は 若者[わかもの]に 大[おお]きな 影響[えいきょう]を 与[あた]える。	
\\	一生あなたのことを思っている。	
\\	一生[いっしょう]あなたのことを 思[おも]っている。	
\\	その独裁者は、国民の大量虐殺を実行しました。	
\\	その 独裁[どくさい] 者[しゃ]は、 国民[こくみん]の 大量[たいりょう] 虐殺[ぎゃくさつ]を 実行[じっこう]しました。	独裁者=どくさいしゃ= 
\\	大量虐殺=たいりょう ぎゃくさつ= 
\\	実行=じっこう= 
\\	始終意見が変わるので、そのコンサルタントは信頼できない。	
\\	始終[しじゅう] 意見[いけん]が 変[か]わるので、そのコンサルタントは 信頼[しんらい]できない。	始終=しじゅう= 
\\	そういう話は始終耳にするよ。	
\\	そういう 話[はなし]は 始終[しじゅう] 耳[みみ]にするよ。	始終=しじゅう= 
\\	どんなルールにも始終あることだが例外がある。	
\\	どんなルールにも 始終[しじゅう]あることだが 例外[れいがい]がある。	始終=しじゅう= 
\\	私たちは始終多くの情報にさらされている。	
\\	私[わたし]たちは 始終[しじゅう] 多[おお]くの 情報[じょうほう]にさらされている。	始終=しじゅう= 
\\	さらす= 
\\	中村さんは縁起をかついで、試合前には必ずこのサラダ食べます。	
\\	中村[なかむら]さんは 縁起[えんぎ]をかついで、 試合[しあい] 前[まえ]には 必[かなら]ずこのサラダ 食[た]べます。	縁起=えんぎ= 
\\	担ぐ=かつぐ= 
\\	年が明けてから年越しそばを食べるのは、縁起が悪いとされています。	
\\	年[とし]が 明[あ]けてから 年越[としこ]しそばを 食[た]べるのは、 縁起[えんぎ]が 悪[わる]いとされています。	年が明ける= 
\\	年越しそば= 
\\	縁起=えんぎ= 
\\	彼は縁起をかついで息子の写真を財布に入れ、持ち歩いている。	
\\	彼[かれ]は 縁起[えんぎ]をかついで 息子[むすこ]の 写真[しゃしん]を 財布[さいふ]に 入[い]れ、 持ち歩[もちある]いている。	縁起=えんぎ= 
\\	担ぐ=かつぐ= 
\\	6時に起床するのは平気です。	
\\	時[じ]に 起床[きしょう]するのは 平気[へいき]です。	起床=きしょう= 
\\	あなたはそれに平気なの?	
\\	あなたはそれに 平気[へいき]なの?	
\\	失礼で下品なことを平気で言います。	
\\	失礼[しつれい]で 下品[げひん]なことを 平気[へいき]で 言[い]います。	下品=げひん= 
\\	僕は平気で人をだませるようなやつだ。	
\\	僕[ぼく]は 平気[へいき]で 人[ひと]をだませるようなやつだ。	騙す=だます= 
\\	南アジアの国の人の中で、女性にも平気で年齢を聞く人がいます。	
\\	南[みなみ]アジアの 国[くに]の 人[ひと]の 中[なか]で、 女性[じょせい]にも 平気[へいき]で 年齢[ねんれい]を 聞[き]く 人[ひと]がいます。	
\\	これが当たり前だと思ってはいけません。	
\\	これが 当たり前[あたりまえ]だと 思[おも]ってはいけません。	
\\	やれやれ、やっと終わりました。	
\\	やれやれ、やっと 終[お]わりました。	やれやれ= 
\\	君と離れていたことで、どのくらい君を愛しているか分かった。	
\\	君[きみ]と 離[はな]れていたことで、どのくらい 君[きみ]を 愛[あい]しているか 分[わ]かった。	
\\	空気や水のように、私たちは電気があるのを当然のことと思いがちだ。	
\\	空気[くうき]や 水[みず]のように、 私[わたし]たちは 電気[でんき]があるのを 当然[とうぜん]のことと 思[おも]いがちだ。	
\\	たとえ関係が長くても、お互いを疎かにしていいということではありません。	
\\	たとえ 関係[かんけい]が 長[なが]くても、お 互[たが]いを 疎[おろ]かにしていいということではありません。	疎かにする=おろそか にする= 
\\	豊かな生活を当たり前に思っている人も多いと思います。	
\\	豊[ゆた]かな 生活[せいかつ]を 当たり前[あたりまえ]に 思[おも]っている 人[ひと]も 多[おお]いと 思[おも]います。	
\\	欧米の男性は、デートする時、割り勘が当然だと考えている人が非常に多いです。	
\\	欧米[おうべい]の 男性[だんせい]は、デートする 時[とき]、 割り勘[わりかん]が 当然[とうぜん]だと 考[かんが]えている 人[ひと]が 非常[ひじょう]に 多[おお]いです。	割り勘=わりかん= 
\\	インスタント食品は、お母さん達の負担を大幅に減らした。	
\\	インスタント 食品[しょくひん]は、お 母[かあ]さん 達[たち]の 負担[ふたん]を 大幅[おおはば]に 減[へ]らした。	
\\	私たちは自然環境と野生生物を保護しなければならない。	
\\	私[わたし]たちは 自然[しぜん] 環境[かんきょう]と 野生[やせい] 生物[せいぶつ]を 保護[ほご]しなければならない。	野生生物= 
\\	沸騰したら、アクを取ります。	
\\	沸騰[ふっとう]したら、アクを 取[と]ります。	沸騰=ふっとう= 
\\	あく= 
\\	沸騰した湯に野菜を入れます。	
\\	沸騰[ふっとう]した 湯[ゆ]に 野菜[やさい]を 入[い]れます。	沸騰=ふっとう= 
\\	やかんの湯が沸騰しています。	
\\	やかんの 湯[ゆ]が 沸騰[ふっとう]しています。	沸騰=ふっとう= 
\\	ほら。もうちょっとで沸騰するよ。	
\\	ほら。もうちょっとで 沸騰[ふっとう]するよ。	沸騰=ふっとう= 
\\	大体日本の賞味期限は短過ぎますよ。	
\\	大体[だいたい] 日本[にっぽん]の 賞味[しょうみ] 期限[きげん]は 短[みじか] 過[す]ぎますよ。	賞味期限= 
\\	「知恵には賞味期限がありません」と彼は言います。	
\\	知恵[ちえ]には 賞味[しょうみ] 期限[きげん]がありません」と 彼[かれ]は 言[い]います。	賞味期限= 
\\	ほんの少しレモン果汁を加えてください。	
\\	ほんの 少[すこ]しレモン 果汁[かじゅう]を 加[くわ]えてください。	果汁=かじゅう= 
\\	率直なあなたは容易に友達を作ります。	
\\	率直[そっちょく]なあなたは 容易[ようい]に 友達[ともだち]を 作[つく]ります。	率直=そっちょく= 
\\	容易=ようい= 
\\	率直に言ってください。	
\\	率直[そっちょく]に 言[い]ってください。	率直=そっちょく= 
\\	率直に言わせていただきます。	
\\	率直[そっちょく]に 言[い]わせていただきます。	率直=そっちょく= 
\\	ごく率直に言うなら、これは不可能だよ。	
\\	ごく 率直[そっちょく]に 言[い]うなら、これは 不可能[ふかのう]だよ。	率直=そっちょく= 
\\	その男はギャングの一員だと率直に認めている。	
\\	その 男[おとこ]はギャングの 一員[いちいん]だと 率直[そっちょく]に 認[みと]めている。	率直=そっちょく= 
\\	10年越しで夢がかなったわけですね?	
\\	年[ねん] 越[ご]しで 夢[ゆめ]がかなったわけですね?	かなう= 
\\	あなたに隠していたわけではありません。	
\\	あなたに 隠[かく]していたわけではありません。	
\\	それじゃあ、婚約を解消したいっていうわけ?	
\\	それじゃあ、 婚約[こんやく]を 解消[かいしょう]したいっていうわけ?	解消=かいしょう= 
\\	ということは、休暇中ずっと苦しんでいたわけ?	
\\	ということは、 休暇[きゅうか] 中[ちゅう]ずっと 苦[くる]しんでいたわけ?	
\\	警察がそう言っているだけで、まだそれが事実と決まったわけではない。	
\\	警察[けいさつ]がそう 言[い]っているだけで、まだそれが 事実[じじつ]と 決[き]まったわけではない。	
\\	夢を見るのにお金は要りません。	
\\	夢[ゆめ]を 見[み]るのにお 金[かね]は 要[い]りません。	
\\	この夢を実現するきっかけとなったのは何ですか?	
\\	この 夢[ゆめ]を 実現[じつげん]するきっかけとなったのは 何[なに]ですか?	実現=じつげん= 
\\	でも、母が、夢を追うように私を励ましてくれました。	
\\	でも、 母[はは]が、 夢[ゆめ]を 追[お]うように 私[わたし]を 励[はげ]ましてくれました。	
\\	私は間違った印象を与えている。	
\\	私[わたし]は 間違[まちが]った 印象[いんしょう]を 与[あた]えている。	
\\	あの映画は、原作の良さを十分に表していない。	
\\	あの 映画[えいが]は、 原作[げんさく]の 良[よ]さを 十分[じゅうぶん]に 表[あらわ]していない。	
\\	自分で精いっぱいやったと思いますか?	
\\	自分[じぶん]で 精[せい]いっぱいやったと 思[おも]いますか?	精いっぱい= 
\\	精いっぱいやりさえすれば、勝敗は問題ではない。	
\\	精[せい]いっぱいやりさえすれば、 勝敗[しょうはい]は 問題[もんだい]ではない。	精いっぱい= 
\\	勝敗=しょうはい= 
\\	この値段で精一杯です。	
\\	この 値段[ねだん]で 精一杯[せいいっぱい]です。	精いっぱい= 
\\	該当するものすべてにチェックマークを付けてください。	
\\	該当[がいとう]するものすべてにチェックマークを 付[つ]けてください。	該当=がいとう= 
\\	次の条件に該当することが必要だ。	
\\	次[つぎ]の 条件[じょうけん]に 該当[がいとう]することが 必要[ひつよう]だ。	該当=がいとう= 
\\	落書きは犯罪行為ですか?	
\\	落書[らくが]きは 犯罪[はんざい] 行為[こうい]ですか?	
\\	その壁のほとんどが、落書きで覆われています。	
\\	その 壁[かべ]のほとんどが、 落書[らくが]きで 覆[おお]われています。	覆う=おおう= 
\\	その壁は落書きだらけです。	
\\	その 壁[かべ]は 落書[らくが]きだらけです。	
\\	その落書きには「ジョンはあほだ」と書いてあります。	
\\	その 落書[らくが]きには「ジョンはあほだ」と 書[か]いてあります。	
\\	彼女はその難局をどう乗り切るかで、頭がいっぱいだ。	
\\	彼女[かのじょ]はその 難局[なんきょく]をどう 乗り切[のりき]るかで、 頭[あたま]がいっぱいだ。	難局=なんきょく= 
\\	乗り切る= 
\\	彼の説明を聞いているうちに、私は何だか頭がこんがらがってきた。	
\\	彼[かれ]の 説明[せつめい]を 聞[き]いているうちに、 私[わたし]は 何[なん]だか 頭[あたま]がこんがらがってきた。	何だか= 
\\	こんがらがる= 
\\	別れた彼女のことが頭から離れない。	
\\	別[わか]れた 彼女[かのじょ]のことが 頭[あたま]から 離[はな]れない。	
\\	怒りがこみ上げてきて、思わず拳を握りしめた。	
\\	怒[いか]りがこみ 上[あ]げてきて、 思[おも]わず 拳[こぶし]を 握[にぎ]りしめた。	こみ上げる= 
\\	拳=こぶし= 
\\	握る= 
\\	彼女は怒りに震えながら、「出て行って!」と男に言った。	
\\	彼女[かのじょ]は 怒[いか]りに 震[ふる]えながら、
\\	出[で]て 行[い]って!」と 男[おとこ]に 言[い]った。	
\\	彼の身勝手な態度に、彼女の怒りは今にも爆発しそうだった。	
\\	彼[かれ]の 身勝手[みがって]な 態度[たいど]に、 彼女[かのじょ]の 怒[いか]りは 今[いま]にも 爆発[ばくはつ]しそうだった。	身勝手=みがって= 
\\	彼はオフィスに戻ってくるなり、同僚達に怒りをぶちまけた。	
\\	彼[かれ]はオフィスに 戻[もど]ってくるなり、 同僚[どうりょう] 達[たち]に 怒[いか]りをぶちまけた。	ぶちまける= 
\\	母親は不満をぶちまけた。	
\\	母親[ははおや]は 不満[ふまん]をぶちまけた。	ぶちまける= 
\\	彼は怒りに駆られて、目の前のごみ箱を蹴飛ばした。	
\\	彼[かれ]は 怒[いか]りに 駆[か]られて、 目[め]の 前[まえ]のごみ 箱[ばこ]を 蹴飛[けと]ばした。	駆られる= 
\\	蹴飛ばす= 
\\	私は彼を思い切り、蹴飛ばした。	
\\	私[わたし]は 彼[かれ]を 思い切[おもいき]り、 蹴飛[けと]ばした。	思い切り= 
\\	蹴飛ばす= 
\\	今回の理不尽な判決に、私たちは皆怒りを覚えています。	
\\	今回[こんかい]の 理不尽[りふじん]な 判決[はんけつ]に、 私[わたし]たちは 皆[みな] 怒[いか]りを 覚[おぼ]えています。	理不尽=りふじん= 
\\	怒りを覚える= 
\\	怒りを抑えられなくなり、私は彼の顔を殴りつけた。	
\\	怒[いか]りを 抑[おさ]えられなくなり、 私[わたし]は 彼[かれ]の 顔[かお]を 殴[なぐ]りつけた。	
\\	今さら言い訳しても遅いよ。彼女の怒りを買うだけさ。	
\\	今[いま]さら 言い訳[いいわけ]しても 遅[おそ]いよ。 彼女[かのじょ]の 怒[いか]りを 買[か]うだけさ。	怒りを買う= 
\\	驚いたことに、この質問は販売員の激しい怒りを買ったらしい。	
\\	驚[おどろ]いたことに、この 質問[しつもん]は 販売[はんばい] 員[いん]の 激[はげ]しい 怒[いか]りを 買[か]ったらしい。	怒りを買う= 
\\	「らしい」
\\	あなたの礼儀正しい話し方は、面接官にいい印象を与えるでしょう。	
\\	あなたの 礼儀[れいぎ] 正[ただ]しい 話し方[はなしかた]は、 面接[めんせつ] 官[かん]にいい 印象[いんしょう]を 与[あた]えるでしょう。	
\\	そのチームはよくまとまっているという印象を受けました。	
\\	そのチームはよくまとまっているという 印象[いんしょう]を 受[う]けました。	
\\	彼は非常に人情に厚い人だという印象を持っています。	
\\	彼[かれ]は 非常[ひじょう]に 人情[にんじょう]に 厚[あつ]い 人[ひと]だという 印象[いんしょう]を 持[も]っています。	
\\	彼女は信頼の置ける社員だが、やや印象が薄い。	
\\	彼女[かのじょ]は 信頼[しんらい]の 置[お]ける 社員[しゃいん]だが、やや 印象[いんしょう]が 薄[うす]い。	印象が薄い= 
\\	その映画で一番印象に残ったシーンはどこですか?	
\\	その 映画[えいが]で 一番[いちばん] 印象[いんしょう]に 残[のこ]ったシーンはどこですか?	
\\	生徒の一人から間違いを指摘されて、顔から火が出る思いでした。	
\\	生徒[せいと]の一 人[にん]から 間違[まちが]いを 指摘[してき]されて、 顔[かお]から 火[ひ]が 出[で]る 思[おも]いでした。	顔から火が出る= 
\\	あなたは彼に強い印象を受けましたか?	
\\	あなたは 彼[かれ]に 強[つよ]い 印象[いんしょう]を 受[う]けましたか?	
\\	その詩からどんな印象を受けましたか?	
\\	その 詩[し]からどんな 印象[いんしょう]を 受[う]けましたか?	
\\	うちのパーティーでのケイトの振る舞いは、悪い印象を与えた。	
\\	うちのパーティーでのケイトの 振る舞[ふるま]いは、 悪[わる]い 印象[いんしょう]を 与[あた]えた。	振る舞い=ふるまい= 
\\	これらの言葉はたばこが無害であるという印象を与える。	
\\	これらの 言葉[ことば]はたばこが 無害[むがい]であるという 印象[いんしょう]を 与[あた]える。	
\\	医者を呼ぼうか?	
\\	医者[いしゃ]を 呼[よ]ぼうか?	
\\	後でまた来ましょうか?	
\\	後[あと]でまた 来[き]ましょうか?	
\\	こちらの書類は明日持参した方がよろしいですか。	
\\	こちらの 書類[しょるい]は 明日[あした] 持参[じさん]した 方[ほう]がよろしいですか。	持参=じさん= 
\\	それは高価ですが、高いだけのことはあります。	
\\	それは 高価[こうか]ですが、 高[たか]いだけのことはあります。	
\\	4日前から食欲がありません。	
\\	4日[よっか] 前[まえ]から 食欲[しょくよく]がありません。	
\\	4日前に気が変わりました。	
\\	4日[よっか] 前[まえ]に 気[き]が 変[か]わりました。	
\\	4日後に、彼女は腹痛を訴え始めた。	
\\	4日[よっか] 後[ご]に、 彼女[かのじょ]は 腹痛[ふくつう]を 訴[うった]え 始[はじ]めた。	
\\	二日酔いになったのは今回が初めてですか?	
\\	二日酔[ふつかよ]いになったのは 今回[こんかい]が 初[はじ]めてですか?	
\\	二日酔いにはこれが一番効く。	
\\	二日酔[ふつかよ]いにはこれが 一番[いちばん] 効[き]く。	
\\	3日後にパリを発って日本へ向かいます。	
\\	3日[みっか] 後[ご]にパリを 発[た]って 日本[にっぽん]へ 向[む]かいます。	発つ=たつ= 
\\	3日続けて雨が降った。	
\\	3日[みっか] 続[つづ]けて 雨[あめ]が 降[ふ]った。	
\\	3日間にわたって祭りが行われる。	
\\	3日間[みっかかん]にわたって 祭[まつ]りが 行[おこな]われる。	
\\	1日3時間の日本語の授業を週に5日取っているんだ。	
\\	1日[いちにち]3 時間[じかん]の日本語の 授業[じゅぎょう]を 週[しゅう]に 5日[いつか] 取[と]っているんだ。	
\\	被害総額は4660億円以上に及びました。	
\\	被害[ひがい] 総額[そうがく]は4660 億[おく] 円[えん] 以上[いじょう]に 及[およ]びました。	総額= 
\\	賞金総額は1700万円である。	
\\	賞金[しょうきん] 総額[そうがく]は1700 万[まん] 円[えん]である。	総額= 
\\	彼の集めた美術品の総額は約250億円です。	
\\	彼[かれ]の 集[あつ]めた 美術[びじゅつ] 品[ひん]の 総額[そうがく]は 約[やく]250 億[おく] 円[えん]です。	総額= 
\\	彼はいつも焦っている。	
\\	彼[かれ]はいつも 焦[あせ]っている。	焦る= 
\\	その会社は銀座に本拠を置き、6台のタクシーを所有していました。	
\\	その 会社[かいしゃ]は 銀座[ぎんざ]に 本拠[ほんきょ]を 置[お]き、6 台[だい]のタクシーを 所有[しょゆう]していました。	本拠= 
\\	所有=しょゆう= 
\\	この町はソニーの工場の本拠地です。	
\\	この 町[まち]はソニーの 工場[こうじょう]の 本拠地[ほんきょち]です。	本拠= 
\\	隠そうとしてもだめだよ。ちゃんと顔に書いてある。	
\\	隠[かく]そうとしてもだめだよ。ちゃんと 顔[かお]に 書[か]いてある。	
\\	その英雄の死に、国中が深い悲しみに包まれた。	
\\	その 英雄[えいゆう]の 死[し]に、 国中[くにじゅう]が 深[ふか]い 悲[かな]しみに 包[つつ]まれた。	英雄=えいゆう= 
\\	その夫婦は娘を亡くした悲しみに、打ちひしがれていた。	
\\	その 夫婦[ふうふ]は 娘[むすめ]を 亡[な]くした 悲[かな]しみに、 打[う]ちひしがれていた。	打ちひしがれる= 
\\	打ちひしがれてはいけないのです。	
\\	打[う]ちひしがれてはいけないのです。	打ちひしがれる= 
\\	彼は親友を事故でなくして、悲しみに沈んでいる。	
\\	彼は 親友[しんゆう]を 事故[じこ]でなくして、 悲[かな]しみに 沈[しず]んでいる。	沈む=しずむ= 
\\	教師は生徒達に、感情を込めて歌うように言った。	
\\	教師[きょうし]は 生徒[せいと] 達[たち]に、 感情[かんじょう]を 込[こ]めて 歌[うた]うように 言[い]った。	
\\	私の言ったことで感情を害されたのなら、謝ります。	
\\	私の言ったことで 感情[かんじょう]を 害[がい]されたのなら、 謝[あやま]ります。	
\\	誰でも故郷には、特別な感情を抱いているものだ。	
\\	誰[だれ]でも 故郷[こきょう]には、 特別[とくべつ]な 感情[かんじょう]を 抱[だ]いているものだ。	
\\	彼の弱点は、感情に流されやすいということだ。	
\\	彼[かれ]の 弱点[じゃくてん]は、 感情[かんじょう]に 流[なが]されやすいということだ。	
\\	演説には政策に関する具体的な話はほとんどなく、主に有権者の感情に訴えるものだった。	
\\	演説[えんぜつ]には 政策[せいさく]に 関[かん]する 具体[ぐたい] 的[てき]な 話[はなし]はほとんどなく、 主[おも]に 有権者[ゆうけんしゃ]の 感情[かんじょう]に 訴[うった]えるものだった。	演説= 
\\	有権者= 
\\	カナダに出発する前の晩、彼女は感情が高ぶってよく眠れなかった。	
\\	カナダに 出発[しゅっぱつ]する 前[まえ]の 晩[ばん]、 彼女[かのじょ]は 感情[かんじょう]が 高[たか]ぶってよく 眠[ねむ]れなかった。	
\\	彼は人前では、めったに感情を表すことがない。	
\\	彼[かれ]は 人前[ひとまえ]では、めったに 感情[かんじょう]を 表[あらわ]すことがない。	
\\	カズオは私の同僚だが、どうも気が合わない。	
\\	カズオは 私[わたし]の 同僚[どうりょう]だが、どうも 気[き]が 合[あ]わない。	
\\	今後の住宅ローン返済のことを考えると、気が重い。	
\\	今後[こんご]の 住宅[じゅうたく]ローン 返済[へんさい]のことを 考[かんが]えると、 気[き]が 重[おも]い。	
\\	私たちの乗った飛行機に、今にも雷が落ちるのではないかと気が気でなかった。	
\\	私たちの乗った 飛行機[ひこうき]に、 今[いま]にも 雷[かみなり]が 落[お]ちるのではないかと 気[き]が 気[き]でなかった。	雷=かみなり= 
\\	気が気でない= 
\\	彼女を待たずに先に行くのは、気が進まなかった。	
\\	彼女[かのじょ]を 待[ま]たずに 先[さき]に 行[い]くのは、 気[き]が 進[すす]まなかった。	気が進まない= 
\\	彼女は映画で裸になることについて気が進まない。	
\\	彼女[かのじょ]は 映画[えいが]で 裸[はだか]になることについて 気[き]が 進[すす]まない。	気が進まない= 
\\	彼との約束を破ってしまったことで、非常に気がとがめた。	
\\	彼[かれ]との 約束[やくそく]を 破[やぶ]ってしまったことで、 非常[ひじょう]に 気[き]がとがめた。	気がとがめる= 
\\	この問題について上司に掛け合うのは、少々気が引ける。	
\\	この 問題[もんだい]について 上司[じょうし]に 掛け合[かけあ]うのは、 少々[しょうしょう] 気[き]が 引[ひ]ける。	
\\	マスターベーションに対して罪悪感を抱きますか?	
\\	マスターベーションに 対[たい]して 罪悪[ざいあく] 感[かん]を 抱[いだ]きますか?	罪悪感=ざいあくかん= 
\\	あなたが罪悪感を覚える必要などありません。	
\\	あなたが 罪悪[ざいあく] 感[かん]を 覚[おぼ]える 必要[ひつよう]などありません。	罪悪感=ざいあくかん= 
\\	アイルランドでは公共の場所で喫煙は禁止されている。	
\\	アイルランドでは 公共[こうきょう]の 場所[ばしょ]で 喫煙[きつえん]は 禁止[きんし]されている。	公共= 
\\	人前で話すのは私にとって全く苦痛なんです。	
\\	人前[ひとまえ]で 話[はな]すのは 私[わたし]にとって 全[まった]く 苦痛[くつう]なんです。	
\\	人前でイチャイチャするな。	
\\	人前[ひとまえ]でイチャイチャするな。	イチャイチャ= 
\\	彼は人前で、私を怠惰だと叱った。	
\\	彼[かれ]は 人前[ひとまえ]で、 私[わたし]を 怠惰[たいだ]だと 叱[しか]った。	
\\	これはガス漏れを防ぐために必要です。	
\\	これはガス 漏[も]れを 防[ふせ]ぐために 必要[ひつよう]です。	ガス漏れ= 
\\	天井が水漏れしているよ。	
\\	天井[てんじょう]が 水[みず] 漏[も]れしているよ。	水漏れ=みずもれ= 
\\	後悔するようなことは、言わないように。	
\\	後悔[こうかい]するようなことは、 言[い]わないように。	後悔= 
\\	後悔はない。	
\\	後悔[こうかい]はない。	後悔= 
\\	仕事をやめたこと、ものすごく後悔しています。	
\\	仕事[しごと]をやめたこと、ものすごく 後悔[こうかい]しています。	後悔= 
\\	裁判官はすべての告訴において彼を有罪とした。	
\\	裁判官[さいばんかん]はすべての 告訴[こくそ]において 彼[かれ]を 有罪[ゆうざい]とした。	裁判官=さいばんかん= 
\\	告訴=こくそ= 
\\	有罪=ゆうざい= 
\\	法学部の学生として、私は、裁判官になることを夢見ていた。	
\\	法学部[ほうがくぶ]の 学生[がくせい]として、 私[わたし]は、 裁判官[さいばんかん]になることを 夢見[ゆめみ]ていた。	裁判官=さいばんかん= 
\\	その部屋にはビデオカメラが設置してある。	
\\	その 部屋[へや]にはビデオカメラが 設置[せっち]してある。	設置=せっち= 
\\	それについては誰も何も知りません。	
\\	それについては 誰[だれ]も 何[なに]も 知[し]りません。	
\\	この製品が、生産承認される可能性は低い。	
\\	この 製品[せいひん]が、 生産[せいさん] 承認[しょうにん]される 可能[かのう] 性[せい]は 低[ひく]い。	生産=せいさん= 
\\	承認= 
\\	その計画は株主たちに承認されました。	
\\	その 計画[けいかく]は 株主[かぶぬし]たちに 承認[しょうにん]されました。	株主=かぶぬし= 
\\	承認= 
\\	給与や仕事に不満を持つ。	
\\	給与[きゅうよ]や 仕事[しごと]に 不満[ふまん]を 持[も]つ。	給与= 
\\	あなたの方がもっと気の毒ですね。	
\\	あなたの 方[ほう]がもっと 気の毒[きのどく]ですね。	気の毒= 
\\	私は天皇をとても気の毒に思います。	
\\	私[わたし]は 天皇[てんのう]をとても 気の毒[きのどく]に 思[おも]います。	気の毒= 
\\	それはお気の毒な話です。	
\\	それはお 気の毒[きのどく]な 話[はなし]です。	気の毒= 
\\	おっしゃることがよくわかりません。	
\\	おっしゃることがよくわかりません。	
\\	もちろんすべての利用者の秘密は厳守されます。	
\\	もちろんすべての 利用[りよう] 者[しゃ]の 秘密[ひみつ]は 厳守[げんしゅ]されます。	
\\	彼はその指示を厳守しました。	
\\	彼[かれ]はその 指示[しじ]を 厳守[げんしゅ]しました。	
\\	それぞれの書類内で、文体を統一することが重要です。	
\\	それぞれの 書類[しょるい] 内[ない]で、 文体[ぶんたい]を 統一[とういつ]することが 重要[じゅうよう]です。	統一= 
\\	日本の都市は総じて清潔です。	
\\	日本の 都市[とし]は 総[そう]じて 清潔[せいけつ]です。	総じて=そうじて= 
\\	清潔=せいけつ= 
\\	私の家の中はすべてがきちんとしていて清潔です。	
\\	私[わたし]の 家[いえ]の 中[なか]はすべてがきちんとしていて 清潔[せいけつ]です。	清潔=せいけつ= 
\\	その犯罪の激しさは、映画の中では和らげられた。	
\\	その 犯罪[はんざい]の 激[はげ]しさは、 映画[えいが]の 中[なか]では 和[やわ]らげられた。	和らげる= 
\\	ときには音楽が苦しみや悲しみを和らげてくれる。	
\\	ときには 音楽[おんがく]が 苦[くる]しみや 悲[かな]しみを 和[やわ]らげてくれる。	和らげる= 
\\	彼の説明は、その患者の不安を和らげた。	
\\	彼[かれ]の 説明[せつめい]は、その 患者[かんじゃ]の 不安[ふあん]を 和[やわ]らげた。	和らげる= 
\\	言い方を和らげようとほほ笑んだ。	
\\	言い方[いいかた]を 和[やわ]らげようとほほ 笑[え]んだ。	和らげる= 
\\	資金不足が彼らの最大問題だった。	
\\	資金[しきん] 不足[ぶそく]が 彼[かれ]らの 最大[さいだい] 問題[もんだい]だった。	資金=しきん= 
\\	この行動は大きな反発を招きました。	
\\	この 行動[こうどう]は 大[おお]きな 反発[はんぱつ]を 招[まね]きました。	反発= 
\\	ある種の精神的葛藤が存在する。	
\\	ある 種[たね]の 精神[せいしん] 的[てき] 葛藤[かっとう]が 存在[そんざい]する。	ある種= 
\\	葛藤= 
\\	彼の頭の中で、どんな葛藤があったのだろうか?	
\\	彼[かれ]の 頭[あたま]の 中[なか]で、どんな 葛藤[かっとう]があったのだろうか?	葛藤= 
\\	「外国人の先生を幼稚園に派遣しています。	
\\	外国[がいこく] 人[じん]の 先生[せんせい]を 幼稚園[ようちえん]に 派遣[はけん]しています。	派遣= 
\\	お金を寄付するよう頼まれた時、彼は難色を示しました。	
\\	お 金[かね]を 寄付[きふ]するよう 頼[たの]まれた 時[とき]、 彼[かれ]は 難色[なんしょく]を 示[しめ]しました。	難色= 
\\	競馬にのめり込んで全財産を使い果たした。	
\\	競馬[けいば]にのめり 込[こ]んで 全[ぜん] 財産[ざいさん]を 使い果[つかいは]たした。	のめり込む= 
\\	全財産= 
\\	使い果たす= 
\\	競馬に関心のない人でも彼の名前は知っている。	
\\	競馬[けいば]に 関心[かんしん]のない 人[ひと]でも 彼[かれ]の 名前[なまえ]は 知[し]っている。	
\\	スキンシップのない文化で育った私はドキドキです。	
\\	スキンシップのない 文化[ぶんか]で 育[そだ]った 私[わたし]はドキドキです。	スキンシップ= 
\\	子どもにとって、母親とのスキンシップはとても大切だ。	
\\	子[こ]どもにとって、 母親[ははおや]とのスキンシップはとても 大切[たいせつ]だ。	スキンシップ= 
\\	日本では、欧米のようにスキンシップが多くない。	
\\	日本では、 欧米[おうべい]のようにスキンシップが 多[おお]くない。	スキンシップ= 
\\	ただ荷物をまとめて去ることはできない。	
\\	ただ 荷物[にもつ]をまとめて 去[さ]ることはできない。	去る= 
\\	逮捕された男は両親の死に対する復讐を企てていました。	
\\	逮捕[たいほ]された 男[おとこ]は 両親[りょうしん]の 死[し]に 対[たい]する 復讐[ふくしゅう]を 企[くわだ]てていました。	逮捕される=たいほ= 
\\	復讐=ふくしゅう= 
\\	企てる= 
\\	自分の子どもの「修理」を他人に委ねてはいけない。	
\\	自分[じぶん]の 子[こ]どもの
\\	修理[しゅうり]」を 他人[たにん]に 委[ゆだ]ねてはいけない。	委ねる=ゆだねる= 
\\	私を犬のように扱わないでください。	
\\	私[わたし]を 犬[いぬ]のように 扱[あつか]わないでください。	
\\	彼らは人を羊みたいに扱う。	
\\	彼[かれ]らは 人[ひと]を 羊[ひつじ]みたいに 扱[あつか]う。	
\\	日本の社会が女性をもっと大人として扱うべきです。	
\\	日本[にっぽん]の 社会[しゃかい]が 女性[じょせい]をもっと 大人[おとな]として 扱[あつか]うべきです。	
\\	彼女をドライブに誘おうと思っていたが、気が変わった。	
\\	彼女[かのじょ]をドライブに 誘[さそ]おうと 思[おも]っていたが、 気[き]が 変[か]わった。	
\\	彼女は昨日具合が悪そうだったので、気になっていたんです。	
\\	彼女[かのじょ]は 昨日[きのう] 具合[ぐあい]が 悪[わる]そうだったので、 気[き]になっていたんです。	
\\	彼は受付の女の子のことが気になるようだ。	
\\	彼[かれ]は 受付[うけつけ]の 女の子[おんなのこ]のことが 気[き]になるようだ。	
\\	ビールの気が抜けている。	
\\	ビールの 気[き]が 抜[ぬ]けている。	
\\	くじでテレビが当たったので、今日は気分がいい。	
\\	くじでテレビが 当[あ]たったので、 今日[きょう]は 気分[きぶん]がいい。	
\\	湖畔でも散歩すれば、気分が晴れるよ。	
\\	湖畔[こはん]でも 散歩[さんぽ]すれば、 気分[きぶん]が 晴[は]れるよ。	湖畔=こはん= 
\\	彼は最近、気分がふさいでいるようだね。どうしたんだろう。	
\\	彼[かれ]は 最近[さいきん]、 気分[きぶん]がふさいでいるようだね。どうしたんだろう。	
\\	コンサートが進むうちに、私たちはますます気分が盛り上がってきた。	
\\	コンサートが 進[すす]むうちに、 私[わたし]たちはますます 気分[きぶん]が 盛り上[もりあ]がってきた。	
\\	彼の不用意な発言に、彼女は気分を害した様子だった。	
\\	彼[かれ]の 不用意[ふようい]な 発言[はつげん]に、 彼女[かのじょ]は 気分[きぶん]を 害[がい]した 様子[ようす]だった。	
\\	シャワーを浴びたら、気分がすっきりした。	
\\	シャワーを 浴[あ]びたら、 気分[きぶん]がすっきりした。	
\\	二日酔いで今朝は気分が最悪だ。	
\\	二日酔[ふつかよ]いで 今朝[けさ]は 気分[きぶん]が 最悪[さいあく]だ。	
\\	飢餓のどん底にある人々のことを思うと、心が痛む。	
\\	飢餓[きが]のどん 底[ぞこ]にある 人々[ひとびと]のことを 思[おも]うと、 心[こころ]が 痛[いた]む。	飢餓=きが= 
\\	どん底= 
\\	彼女からメールをもらうと、心が弾みます。	
\\	彼女[かのじょ]からメールをもらうと、 心[こころ]が 弾[はず]みます。	
\\	彼は心にもないことを言って、彼女を怒らせた。	
\\	彼[かれ]は 心[こころ]にもないことを 言[い]って、 彼女[かのじょ]を 怒[おこ]らせた。	
\\	その映画のラストシーンには、見るたびに心を打たれる。	
\\	その 映画[えいが]のラストシーンには、 見[み]るたびに 心[こころ]を 打[う]たれる。	
\\	親友が列車事故で亡くなったと聞いて、彼はショックを受けた。	
\\	親友[しんゆう]が 列車[れっしゃ] 事故[じこ]で 亡[な]くなったと 聞[き]いて、 彼[かれ]はショックを 受[う]けた。	
\\	その有名歌手の逮捕のニュースは、彼の熱心なファンにショックを与えた。	
\\	その 有名[ゆうめい] 歌手[かしゅ]の 逮捕[たいほ]のニュースは、 彼[かれ]の 熱心[ねっしん]なファンにショックを 与[あた]えた。	
\\	彼女はその事件のショックから、立ち直りかけている。	
\\	彼女[かのじょ]はその 事件[じけん]のショックから、 立ち直[たちなお]りかけている。	
\\	まだ改善の余地がある。	
\\	まだ 改善[かいぜん]の 余地[よち]がある。	
\\	常に改善の余地はあります。	
\\	常[つね]に 改善[かいぜん]の 余地[よち]はあります。	
\\	彼はここでの教え方は記憶学習ばかりだと気付きました。	
\\	彼[かれ]はここでの 教[おし]え 方[かた]は 記憶[きおく] 学習[がくしゅう]ばかりだと 気付[きづ]きました。	
\\	当校はまた、自分の力で考えるように生徒を促しています。	
\\	当校[とうこう]はまた、 自分[じぶん]の 力[ちから]で 考[かんが]えるように 生徒[せいと]を 促[うなが]しています。	
\\	私たちはその本を楽しくかつ実用的なものにしようとした。	
\\	私[わたし]たちはその 本[ほん]を 楽[たの]しくかつ 実用[じつよう] 的[てき]なものにしようとした。	かつ= 
\\	私も”実用性”のない学校の教育にうんざりしていた。	
\\	私[わたし]も” 実用[じつよう] 性[せい]”のない 学校[がっこう]の 教育[きょういく]にうんざりしていた。	うんざり=物事に飽きること
\\	英語教育に関する限り、日本の教育制度は、語学習得のための論理的な学習法のほとんどと逆行している。	
\\	英語 教育[きょういく]に 関[かん]する 限[かぎ]り、日本の 教育[きょういく] 制度[せいど]は、 語学[ごがく] 習得[しゅうとく]のための 論理[ろんり] 的[てき]な 学習[がくしゅう] 法[ほう]のほとんどと 逆行[ぎゃっこう]している。	
\\	アメリカで経済学を学べば、大いに得るところがあるだろう。	
\\	アメリカで 経済[けいざい] 学[がく]を 学[まな]べば、 大[おお]いに 得[え]るところがあるだろう。	
\\	血流を止めるために圧力をかけなさい。	
\\	血[ち] 流[りゅう]を 止[と]めるために 圧力[あつりょく]をかけなさい。	圧力をかける= 
\\	試験では、ユニークな方法で学生の独創力をテストします。	
\\	試験[しけん]では、ユニークな 方法[ほうほう]で 学生[がくせい]の 独創[どくそう] 力[りょく]をテストします。	
\\	彼女は東京の裕福な家庭の出身である。	
\\	彼女[かのじょ]は 東京[とうきょう]の 裕福[ゆうふく]な 家庭[かてい]の 出身[しゅっしん]である。	
\\	これは修得にとても時間のかかる技術です。	
\\	これは 修得[しゅうとく]にとても 時間[じかん]のかかる 技術[ぎじゅつ]です。	修得=しゅうとく= 
\\	その会社にはアジアへの拡張計画がある。	
\\	その 会社[かいしゃ]にはアジアへの 拡張[かくちょう] 計画[けいかく]がある。	
\\	それも一種のセクハラです。	
\\	それも 一種[いっしゅ]のセクハラです。	
\\	結論を出すのはまだ早過ぎる。	
\\	結論[けつろん]を 出[だ]すのはまだ 早[はや] 過[す]ぎる。	
\\	あの傲慢な部長の下で仕事をするのは、さぞ神経が疲れるでしょう。	
\\	あの 傲慢[ごうまん]な 部長[ぶちょう]の 下[した]で 仕事[しごと]をするのは、さぞ 神経[しんけい]が 疲[つか]れるでしょう。	傲慢=ごうまん= 
\\	さぞ= 
\\	さぞかしつらかったでしょう。	
\\	さぞかしつらかったでしょう。	さぞかし= 
\\	そのピッチャーは登板を前にして、神経が高ぶっているようだ。	
\\	そのピッチャーは 登板[とうばん]を 前[まえ]にして、 神経[しんけい]が 高[たか]ぶっているようだ。	登板= 
\\	あの騒々しい音楽は、神経にさわる。	
\\	あの 騒々[そうぞう]しい 音楽[おんがく]は、 神経[しんけい]にさわる。	
\\	厳しい訓練で、彼は神経が参ってしまうのではないかと心配だ。	
\\	厳[きび]しい 訓練[くんれん]で、 彼[かれ]は 神経[しんけい]が 参[まい]ってしまうのではないかと 心配[しんぱい]だ。	訓練=くんれん= 
\\	神経が参る= 
\\	劣悪な労働環境の中で、神経をすり減らしている人は多い。	
\\	劣悪[れつあく]な 労働[ろうどう] 環境[かんきょう]の 中[なか]で、 神経[しんけい]をすり 減[へ]らしている 人[ひと]は 多[おお]い。	すり減らす= 
\\	政府は首相退陣を求める世論の高まりに、神経をとがらせている。	
\\	政府[せいふ]は 首相[しゅしょう] 退陣[たいじん]を 求[もと]める 世論[せろん]の 高[たか]まりに、 神経[しんけい]をとがらせている。	退陣=たいじん= 
\\	彼は目の前のモニターの画面に、神経を集中させた。	
\\	彼[かれ]は 目[め]の 前[まえ]のモニターの 画面[がめん]に、 神経[しんけい]を 集中[しゅうちゅう]させた。	
\\	彼の配慮に欠けた発言は、被害者の神経を逆なでした。	
\\	彼[かれ]の 配慮[はいりょ]に 欠[か]けた 発言[はつげん]は、 被害[ひがい] 者[しゃ]の 神経[しんけい]を 逆[さか]なでした。	神経を逆なでする= 
\\	私のことで、あれこれ神経を使っていただかなくても大丈夫ですよ。	
\\	私のことで、あれこれ 神経[しんけい]を 使[つか]っていただかなくても 大丈夫[だいじょうぶ]ですよ。	
\\	ストレスがたまると、精神が不安定になりやすい。	
\\	ストレスがたまると、 精神[せいしん]が 不安定[ふあんてい]になりやすい。	
\\	バッターは投手に精神を集中させて打席にたった。	
\\	バッターは 投手[とうしゅ]に 精神[せいしん]を 集中[しゅうちゅう]させて 打席[だせき]にたった。	投手= 
\\	打席= 
\\	精神を鍛えるために、兄は剣道を始めた。	
\\	精神[せいしん]を 鍛[きた]えるために、 兄[あに]は 剣道[けんどう]を 始[はじ]めた。	鍛える= 
\\	あまり腹が立ったので、私は電話をガチャンと切ってしまった。	
\\	あまり 腹[はら]が 立[た]ったので、 私[わたし]は 電話[でんわ]をガチャンと 切[き]ってしまった。	
\\	彼の傍若無人の振る舞いは、腹にすえかねる。	
\\	彼[かれ]の 傍若無人[ぼうじゃくぶじん]の 振る舞[ふるま]いは、 腹[はら]にすえかねる。	傍若無人= 
\\	彼を一発殴ったくらいでは、とても腹の虫が治まらない。	
\\	彼[かれ]を一 発[はつ] 殴[なぐ]ったくらいでは、とても 腹の虫[はらのむし]が 治[おさ]まらない。	
\\	闇にうごめく人影を見て、恐怖で身が縮む思いだった。	
\\	闇[やみ]にうごめく 人影[ひとかげ]を 見[み]て、 恐怖[きょうふ]で 身[み]が 縮[ちぢ]む 思[おも]いだった。	闇=やみ= 
\\	縮む= 
\\	身が縮む= 
\\	被害者が私の娘と同じ年だと知って、身につまされる思いだった。	
\\	被害[ひがい] 者[しゃ]が 私[わたし]の 娘[むすめ]と 同[おな]じ 年[とし]だと 知[し]って、 身[み]につまされる 思[おも]いだった。	身につまされる= 
\\	宿題を忘れたことに気がついたのは、授業が始まってからだった。	
\\	宿題[しゅくだい]を 忘[わす]れたことに 気[き]がついたのは、 授業[じゅぎょう]が 始[はじ]まってからだった。	
\\	人がせっかく作ってくれた料理を食べないのは失礼だ。	
\\	人[ひと]がせっかく 作[つく]ってくれた 料理[りょうり]を 食[た]べないのは 失礼[しつれい]だ。	
\\	せっかく習った漢字は忘れないようにしましょう。	
\\	せっかく 習[なら]った 漢字[かんじ]は 忘[わす]れないようにしましょう。	
\\	これは先生の本だから、あなたに貸すわけにはいきません。	
\\	これは 先生[せんせい]の 本[ほん]だから、あなたに 貸[か]すわけにはいきません。	
\\	キャロルは、試験の日を変えてもらうために先生の研究室へ行った。	
\\	キャロルは、 試験[しけん]の 日[ひ]を 変[か]えてもらうために 先生[せんせい]の 研究[けんきゅう] 室[しつ]へ 行[い]った。	
\\	アメリカンフットボールは、雪が降ってもやるそうです。	
\\	アメリカンフットボールは、 雪[ゆき]が 降[ふ]ってもやるそうです。	
\\	日本料理は、たくさん食べても太らないそうです。	
\\	日本[にっぽん] 料理[りょうり]は、たくさん 食[た]べても 太[ふと]らないそうです。	
\\	戦後日本はずいぶん西洋化した。	
\\	戦後[せんご] 日本[にっぽん]はずいぶん 西洋[せいよう] 化[か]した。	
\\	映画化された小説が多い。	
\\	映画[えいが] 化[か]された 小説[しょうせつ]が 多[おお]い。	
\\	日本の高校生がみんな勉強ばかりしているとは限らない。	
\\	日本の 高校生[こうこうせい]がみんな勉強ばかりしているとは 限[かぎ]らない。	
\\	高いものが必ずしもみんなよいとは限らない。	
\\	高[たか]いものが 必[かなら]ずしもみんなよいとは 限[かぎ]らない。	
\\	日本語を勉強している学生がみんな日本へ行くとは限らない。	
\\	日本語を勉強している 学生[がくせい]がみんな日本へ 行[い]くとは 限[かぎ]らない。	
\\	この辞書は、日本へ行った時に買いました。	
\\	この 辞書[じしょ]は、 日本[にっぽん]へ 行[い]った 時[とき]に 買[か]いました。	
\\	この辞書は、日本へ行く時に買いました。	
\\	この 辞書[じしょ]は、日本へ 行[い]く 時[とき]に 買[か]いました。	
\\	辞書は、日本へ行った時に買います。	
\\	辞書[じしょ]は、日本へ 行[い]った 時[とき]に 買[か]います。	
\\	日本へは何度も行ったことがあります。	
\\	日本へは 何[なん] 度[ど]も 行[おこな]ったことがあります。	
\\	中華料理屋は何軒もあります。	
\\	中華[ちゅうか] 料理[りょうり] 屋[や]は 何軒[なんげん]もあります。	
\\	翻訳をする人は、辞書を何十冊も持っているそうです。	
\\	翻訳[ほんやく]をする 人[ひと]は、 辞書[じしょ]を 何十冊[なんじゅうさつ]も 持[も]っているそうです。	
\\	日本へ行ったことは、一度もありません。	
\\	日本へ 行[い]ったことは、一 度[ど]もありません。	
\\	お寿司にしようか天ぷらにしようかと迷った。	
\\	お 寿司[すし]にしようか 天[てん]ぷらにしようかと 迷[まよ]った。	
\\	それに関する賛否両論にはどんなものがありますか?	
\\	それに 関[かん]する 賛否[さんぴ] 両論[りょうろん]にはどんなものがありますか?	賛否両論= 
\\	その赤ちゃんはよだれまみれの顔で笑っている。	
\\	その 赤[あか]ちゃんはよだれまみれの 顔[かお]で 笑[わら]っている。	「まみれ」
\\	自分の収入の範囲内で暮らしていかないと、借金まみれになるよ。	
\\	自分[じぶん]の 収入[しゅうにゅう]の 範囲[はんい] 内[ない]で 暮[く]らしていかないと、 借金[しゃっきん]まみれになるよ。	「まみれ」
\\	顔中汗まみれになった。	
\\	顔[かお] 中[ちゅう] 汗[あせ]まみれになった。	顔中= 
\\	「まみれ」
\\	せっかち過ぎるよ。	
\\	せっかち 過[す]ぎるよ。	
\\	恐れていたことがすべて現実になった。	
\\	恐[おそ]れていたことがすべて 現実[げんじつ]になった。	
\\	これは我々が最も恐れていたことを裏付けていると思います。	
\\	これは 我々[われわれ]が 最[もっと]も 恐[おそ]れていたことを 裏付[うらづ]けていると 思[おも]います。	裏付ける= 
\\	そう聞かれるのではないかと恐れていましたよ。	
\\	そう 聞[き]かれるのではないかと 恐[おそ]れていましたよ。	
\\	友人の大切さが身にしみてわかった。	
\\	友人[ゆうじん]の 大切[たいせつ]さが 身[み]にしみてわかった。	
\\	彼女のメールを読んで、胸が痛んだ。	
\\	彼女[かのじょ]のメールを 読[よ]んで、 胸[むね]が 痛[いた]んだ。	
\\	支援者からたくさんの励ましの手紙をもらい、彼は感謝で胸がいっぱいになった。	
\\	支援[しえん] 者[しゃ]からたくさんの 励[はげ]ましの 手紙[てがみ]をもらい、 彼[かれ]は 感謝[かんしゃ]で 胸[むね]がいっぱいになった。	
\\	病院のベッドに横たわる母親を見て、彼女は胸がつまった。	
\\	病院[びょういん]のベッドに 横[よこ]たわる 母親[ははおや]を 見[み]て、 彼女[かのじょ]は 胸[むね]がつまった。	
\\	事故現場を訪れるたびに、私は胸が張り裂けそうです。	
\\	事故[じこ] 現場[げんば]を 訪[おとず]れるたびに、 私[わたし]は 胸[むね]が 張り裂[はりさ]けそうです。	張り裂ける=はりさける= 
\\	訪れる=おとずれる= (訪問する) 
\\	(到来する) 
\\	彼女が金メダルを獲得したというニュースに、国中が大きな喜びに包まれた。	
\\	彼女[かのじょ]が 金メダル[きんめだる]を 獲得[かくとく]したというニュースに、 国[くに] 中[ちゅう]が 大[おお]きな 喜[よろこ]びに 包[つつ]まれた。	獲得=かくとく= 
\\	試合から一週間経って、ようやく優勝の喜びをかみしめています。	
\\	試合[しあい]から 一週間[いっしゅうかん] 経[た]って、ようやく 優勝[ゆうしょう]の 喜[よろこ]びをかみしめています。	
\\	そのメールを読むうちに、喜びがこみ上げてきた。	
\\	そのメールを 読[よ]むうちに、 喜[よろこ]びがこみ 上[あ]げてきた。	
\\	地元のチームが優勝して、町は今喜びに沸いている。	
\\	地元[じもと]のチームが 優勝[ゆうしょう]して、 町[まち]は 今[こん] 喜[よろこ]びに 沸[わ]いている。	
\\	散歩をしている最中に、彼は新製品のアイディアが頭に浮かんだ。	
\\	散歩[さんぽ]をしている 最中[さいちゅう]に、 彼[かれ]は 新[しん] 製品[せいひん]のアイディアが 頭[あたま]に 浮[う]かんだ。	
\\	彼らは、電気を節約するための斬新なアイディアを、いくつか思いついた。	
\\	彼[かれ]らは、 電気[でんき]を 節約[せつやく]するための 斬新[ざんしん]なアイディアを、いくつか 思[おも]いついた。	斬新な=ざんしん= 
\\	この新製品を売り込むための何かいいアイディアを出してくれませんか?	
\\	この 新[しん] 製品[せいひん]を 売り込[うりこ]むための 何[なに]かいいアイディアを 出[だ]してくれませんか?	
\\	彼の頭には次から次へと、斬新なアイディアがわいてくるようだ。	
\\	彼[かれ]の 頭[あたま]には 次[つぎ]から 次[つぎ]へと、 斬新[ざんしん]なアイディアがわいてくるようだ。	斬新な=ざんしん= 
\\	私たちは、互いの足を引っぱり合うことをやめて、解決法を考えるべきだ。	
\\	私[わたし]たちは、 互[たが]いの 足[あし]を 引[ひ]っぱり 合[あ]うことをやめて、 解決[かいけつ] 法[ほう]を 考[かんが]えるべきだ。	足を引っ張る= 
\\	消費税率の引き上げは景気回復の足を引っ張ることになりかねない。	
\\	消費[しょうひ] 税率[ぜいりつ]の 引き上[ひきあ]げは 景気[けいき] 回復[かいふく]の 足[あし]を 引っ張[ひっぱ]ることになりかねない。	足を引っ張る= 
\\	〜かねない= 
\\	私は、チームの足を引っ張りたくありません。	
\\	私[わたし]は、チームの 足[あし]を 引っ張[ひっぱ]りたくありません。	足を引っ張る= 
\\	引っ越しの最中に本をなくした。	
\\	引っ越[ひっこ]しの 最中[さいちゅう]に 本[ほん]をなくした。	
\\	会話の最中に私たちの電話は切れてしまいました。	
\\	会話[かいわ]の 最中[さいちゅう]に 私[わたし]たちの 電話[でんわ]は 切[き]れてしまいました。	
\\	何かしている最中だったら、ごめんなさい。	
\\	何[なに]かしている 最中[さいちゅう]だったら、ごめんなさい。	
\\	混浴というものがあるなんて知りませんでした。	
\\	混浴[こんよく]というものがあるなんて 知[し]りませんでした。	
\\	蚊に刺された。	
\\	蚊[か]に 刺[さ]された。	
\\	蚊に刺されただけだな。掻いちゃ駄目だ。ほっておきなさい。	
\\	蚊[か]に 刺[さ]されただけだな。 掻[か]いちゃ 駄目[だめ]だ。ほっておきなさい。	掻く=かく= 
\\	ほっておく= 
\\	いびきをかくな。	
\\	いびきをかくな。	いびき を かく= 
\\	いびきをかくのをやめてもらいたいんですけど。	
\\	いびきをかくのをやめてもらいたいんですけど。	いびき を かく= 
\\	恥をかいたのは私の方だった。	
\\	恥[はじ]をかいたのは 私[わたし]の 方[ほう]だった。	恥をかく= 
\\	彼に恥をかかせてはいけません。	
\\	彼[かれ]に 恥[はじ]をかかせてはいけません。	恥をかく= 
\\	汗をかいたから水分を補給しなきゃ。	
\\	汗[あせ]をかいたから 水分[すいぶん]を 補給[ほきゅう]しなきゃ。	汗をかく= 
\\	電車に乗るために走って、大汗をかいた。	
\\	電車[でんしゃ]に 乗[の]るために 走[はし]って、 大[だい] 汗[あせ]をかいた。	
\\	そのことはきれいさっぱり忘れました。	
\\	そのことはきれいさっぱり 忘[わす]れました。	奇麗さっぱり= 
\\	コンピューターのことはさっぱり分からない。	
\\	コンピューターのことはさっぱり 分[わ]からない。	
\\	彼が彼女に何か言っているのを立ち聞きした。	
\\	彼[かれ]が 彼女[かのじょ]に 何[なに]か 言[い]っているのを 立ち聞[たちぎ]きした。	立ち聞き= 
\\	盗み聞きするつもりはなかったのですが。	
\\	盗み聞[ぬすみぎ]きするつもりはなかったのですが。	
\\	それはあべこべだと思いますよ。	
\\	それはあべこべだと 思[おも]いますよ。	
\\	あなたはシャツをあべこべに着ている。	
\\	あなたはシャツをあべこべに 着[き]ている。	
\\	私は日本の政治にはうんざりしています。	
\\	私[わたし]は 日本[にっぽん]の 政治[せいじ]にはうんざりしています。	
\\	あいつには本当にうんざりする。	
\\	あいつには 本当[ほんとう]にうんざりする。	
\\	あなたの愚痴にはうんざりだ。	
\\	あなたの 愚痴[ぐち]にはうんざりだ。	愚痴=ぐち= 
\\	こんなたわ言はうんざりだ。	
\\	こんなたわ 言[ごと]はうんざりだ。	たわ言= 
\\	彼は仕事にうんざりしている。	
\\	彼[かれ]は 仕事[しごと]にうんざりしている。	
\\	地元のイタリアの女の子が買えなくてうろうろしていたわ。	
\\	地元[じもと]のイタリアの 女の子[おんなのこ]が 買[か]えなくてうろうろしていたわ。	
\\	私がドレスを選んでいる間、彼は店の中をうろうろ歩き回っていた。	
\\	私[わたし]がドレスを 選[えら]んでいる 間[ま]、 彼[かれ]は 店[みせ]の 中[なか]をうろうろ 歩き回[あるきまわ]っていた。	
\\	膝が、がくがくしている。	
\\	膝[ひざ]が、がくがくしている。	膝=ひざ= 
\\	彼女の脚はがくがくしていた。	
\\	彼女[かのじょ]の 脚[あし]はがくがくしていた。	
\\	彼は歯をカチカチ鳴らした。	
\\	彼[かれ]は 歯[は]をカチカチ 鳴[な]らした。	
\\	聞こえる音は時計のカチカチという音だけだった。	
\\	聞[き]こえる 音[おと]は 時計[とけい]のカチカチという 音[おと]だけだった。	
\\	歯がガチガチ言ってるよ。	
\\	歯[は]がガチガチ 言[い]ってるよ。	
\\	ボブの指はキーボードをカチャカチャと素早くたたいていた。	
\\	ボブの 指[ゆび]はキーボードをカチャカチャと 素早[すばや]くたたいていた。	素早い= 
\\	そんなにガツガツ食べたら、しゃっくりが出るのがオチだよ。	
\\	そんなにガツガツ 食[た]べたら、しゃっくりが 出[で]るのがオチだよ。	落ち=おち= 
\\	ボブはまるで一ヶ月何も食べていなかったかのように夕食をガツガツ食べた。	
\\	ボブはまるで一 ヶ月[かげつ] 何[なに]も 食[た]べていなかったかのように 夕食[ゆうしょく]をガツガツ 食[た]べた。	
\\	彼はフライドチキンをガツガツ食べている。	
\\	彼[かれ]はフライドチキンをガツガツ 食[た]べている。	
\\	彼は大きなチーズバーガーをガツガツ食べている。	
\\	彼[かれ]は 大[おお]きなチーズバーガーをガツガツ 食[た]べている。	
\\	彼は、その会社へ、匿名の苦情の手紙を送った。	
\\	彼[かれ]は、その 会社[かいしゃ]へ、 匿名[とくめい]の 苦情[くじょう]の 手紙[てがみ]を 送[おく]った。	
\\	その絵は匿名の買い手によって購入された。	
\\	その 絵[え]は 匿名[とくめい]の 買い手[かいて]によって 購入[こうにゅう]された。	
\\	言うまでもないことだが、日本人は勤勉な国民である。	
\\	言[い]うまでもないことだが、 日本人[にっぽんじん]は 勤勉[きんべん]な 国民[こくみん]である。	
\\	言うまでもなく、パーティーは大盛況でした。	
\\	言[い]うまでもなく、パーティーは 大[だい] 盛況[せいきょう]でした。	大盛況=だい せい きょう= 
\\	それは言うまでもないでしょう。	
\\	それは 言[い]うまでもないでしょう。	
\\	彼は、自らが説き勧めることを実践しない。	
\\	彼[かれ]は、 自[みずか]らが 説[と]き 勧[すす]めることを 実践[じっせん]しない。	
\\	口説いているわけじゃありませんよ。	
\\	口説[くど]いているわけじゃありませんよ。	口説く= 
\\	日本は、自国の化石燃料の埋蔵量がほとんどない。	
\\	日本は、 自国[じこく]の 化石[かせき] 燃料[ねんりょう]の 埋蔵[まいぞう] 量[りょう]がほとんどない。	化石燃料= 
\\	埋蔵量= 
\\	お茶を入れるためにお湯を沸かし始めた。	
\\	お 茶[ちゃ]を 入[い]れるためにお 湯[ゆ]を 沸[わ]かし 始[はじ]めた。	
\\	天然資源が過度の発展で消費されてしまった。	
\\	天然[てんねん] 資源[しげん]が 過度[かど]の 発展[はってん]で 消費[しょうひ]されてしまった。	過度=かど= 
\\	天才と気違いは紙一重です。	
\\	天才[てんさい]と 気違[きちが]いは 紙一重[かみひとえ]です。	紙一重=かみひとえ= 
\\	妥協する用意があります。	
\\	妥協[だきょう]する 用意[ようい]があります。	妥協=だきょう= 
\\	妥協は一切許されない。	
\\	妥協[だきょう]は 一切[いっさい] 許[ゆる]されない。	妥協=だきょう= 
\\	人生は妥協だらけである。	
\\	人生[じんせい]は 妥協[だきょう]だらけである。	妥協=だきょう= 
\\	長い会議の後、両者はようやく妥協に至りました。	
\\	長[なが]い 会議[かいぎ]の 後[のち]、 両者[りょうしゃ]はようやく 妥協[だきょう]に 至[いた]りました。	妥協=だきょう= 
\\	食べ物に関しては、妥協したくない。	
\\	食べ物[たべもの]に 関[かん]しては、 妥協[だきょう]したくない。	妥協=だきょう= 
\\	人種差別が学べるものなら、学ばずにいることもできるはずだ。	
\\	人種[じんしゅ] 差別[さべつ]が 学[まな]べるものなら、 学[まな]ばずにいることもできるはずだ。	
\\	この名門大学からこの国の最精鋭の人材が輩出されてきた。	
\\	この 名門[めいもん] 大学[だいがく]からこの 国[くに]の 最[さい] 精鋭[せいえい]の 人材[じんざい]が 輩出[はいしゅつ]されてきた。	名門大学= 
\\	精鋭=せいえい= 
\\	輩出= 
\\	その新しい税方針で恩恵を受ける人は、ほとんどないだろう。	
\\	その 新[あたら]しい 税[ぜい] 方針[ほうしん]で 恩恵[おんけい]を 受[う]ける 人[ひと]は、ほとんどないだろう。	恩恵=おんけい= 
\\	子どもたちは、日光と新鮮な空気の恩恵を受けます。	
\\	子[こ]どもたちは、 日光[にっこう]と 新鮮[しんせん]な 空気[くうき]の 恩恵[おんけい]を 受[う]けます。	日光=にっこう= 
\\	恩恵=おんけい= 
\\	その残酷な男はいつも飼い犬を蹴っていた。	
\\	その 残酷[ざんこく]な 男[おとこ]はいつも 飼い犬[かいいぬ]を 蹴[け]っていた。	
\\	人々は自分たちと違うように見える人々に対して残酷になることがある。	
\\	人々[ひとびと]は 自分[じぶん]たちと 違[ちが]うように 見[み]える 人々[ひとびと]に 対[たい]して 残酷[ざんこく]になることがある。	
\\	彼は残酷な独裁者だった。	
\\	彼[かれ]は 残酷[ざんこく]な 独裁[どくさい] 者[しゃ]だった。	独裁者=どくさいしゃ= 
\\	教育者は、歴史のより残酷な出来事に触れたがらない。	
\\	教育[きょういく] 者[しゃ]は、 歴史[れきし]のより 残酷[ざんこく]な 出来事[できごと]に 触[ふ]れたがらない。	
\\	あの政治家は、雄弁だが具体的な行動を取らない。	
\\	あの 政治[せいじ] 家[か]は、 雄弁[ゆうべん]だが 具体[ぐたい] 的[てき]な 行動[こうどう]を 取[と]らない。	行動を取る= 
\\	弁護士は裁判官に対して雄弁な演説をした。	
\\	弁護士[べんごし]は 裁判官[さいばんかん]に 対[たい]して 雄弁[ゆうべん]な 演説[えんぜつ]をした。	
\\	彼は若い。経験も浅い。だからと言って教えられないとは限らない。	
\\	彼[かれ]は 若[わか]い。 経験[けいけん]も 浅[あさ]い。だからと 言[い]って 教[おし]えられないとは 限[かぎ]らない。	「だからと言って」
\\	ジョンは奥さんのことをちっとも褒めない。時々口をきかないこともある。だからと言って、奥さんを愛していないわけではない。	
\\	ジョンは 奥[おく]さんのことをちっとも 褒[ほ]めない。 時々[ときどき] 口[くち]をきかないこともある。だからと 言[い]って、 奥[おく]さんを 愛[あい]していないわけではない。	「だからと言って」
\\	僕は毎日運動をしている。食べ物にも注意している。しかし、だからと言って、長生きする保障はない。	
\\	僕[ぼく]は 毎日[まいにち] 運動[うんどう]をしている。 食べ物[たべもの]にも 注意[ちゅうい]している。しかし、だからと 言[い]って、 長生[ながい]きする 保障[ほしょう]はない。	「だからと言って」
\\	日本人は集団行動が好きだと言われる。何をするにも一緒にやる。しかし、だからと言って、個人行動が全くないわけではない。	
\\	日本人[にっぽんじん]は 集団[しゅうだん] 行動[こうどう]が 好[す]きだと 言[い]われる。 何[なに]をするにも 一緒[いっしょ]にやる。しかし、だからと 言[い]って、 個人[こじん] 行動[こうどう]が 全[まった]くないわけではない。	「だからと言って」
\\	魚は健康にいい。しかし、だからと言って、魚ばかり食べていたら、体に悪いはずだ。	
\\	魚[さかな]は 健康[けんこう]にいい。しかし、だからと 言[い]って、 魚[さかな]ばかり 食[た]べていたら、 体[からだ]に 悪[わる]いはずだ。	「だからと言って」
\\	日本語はよく難しい言語だと言われる。文法が複雑だし、漢字を覚えるのも大変だ。だからと言って、外国人が学べないわけではない。	
\\	日本語[にほんご]はよく 難[むずか]しい 言語[げんご]だと 言[い]われる。 文法[ぶんぽう]が 複雑[ふくざつ]だし、 漢字[かんじ]を 覚[おぼ]えるのも 大変[たいへん]だ。だからと 言[い]って、 外国[がいこく] 人[じん]が 学[まな]べないわけではない。	「だからと言って」
\\	そのイラストは、物語に命を吹き込みました。	
\\	そのイラストは、 物語[ものがたり]に 命[いのち]を 吹き込[ふきこ]みました。	
\\	マイクに息を吹き込まないでください。	
\\	マイクに 息[いき]を 吹き込[ふきこ]まないでください。	
\\	1960年代初めには黒人による公民権運動が全米で吹き荒れていた。	
\\	年代[ねんだい] 初[はじ]めには 黒人[こくじん]による 公民[こうみん] 権[けん] 運動[うんどう]が 全米[ぜんべい]で 吹き荒[ふきあ]れていた。	吹き荒れる= 
\\	竜巻が車を道路の外に吹き飛ばした。	
\\	竜巻[たつまき]が 車[くるま]を 道路[どうろ]の 外[そと]に 吹き飛[ふきと]ばした。	竜巻=たつまき= 
\\	私は中流の家庭育ちで、父は大学の教授をしていた。	
\\	私[わたし]は 中流[ちゅうりゅう]の 家庭[かてい] 育[そだ]ちで、 父[ちち]は 大学[だいがく]の 教授[きょうじゅ]をしていた。	
\\	その政党は中流階級の人々からの支持を得ている。	
\\	その 政党[せいとう]は 中流[ちゅうりゅう] 階級[かいきゅう]の 人々[ひとびと]からの 支持[しじ]を 得[え]ている。	
\\	政府は中流階級に対する減税を約束しました。	
\\	政府[せいふ]は 中流[ちゅうりゅう] 階級[かいきゅう]に 対[たい]する 減税[げんぜい]を 約束[やくそく]しました。	
\\	5時間連続で運転した後でしたが、私は依然元気でした。	
\\	時間[じかん] 連続[れんぞく]で 運転[うんてん]した 後[のち]でしたが、 私[わたし]は 依然[いぜん] 元気[げんき]でした。	依然= 
\\	事件は以前謎に包まれたままだ。	
\\	事件[じけん]は 以前[いぜん] 謎[なぞ]に 包[つつ]まれたままだ。	依然= 
\\	その命令は依然として有効です。	
\\	その 命令[めいれい]は 依然[いぜん]として 有効[ゆうこう]です。	依然= 
\\	その国では女性が依然として差別されている。	
\\	その 国[くに]では 女性[じょせい]が 依然[いぜん]として 差別[さべつ]されている。	依然= 
\\	炎が、私の家を飲み込んだ。	
\\	炎[ほのお]が、私の 家[いえ]を 飲み込[のみこ]んだ。	
\\	休憩時間は一斉に与えなければならない。	
\\	休憩[きゅうけい] 時間[じかん]は 一斉[いっせい]に 与[あた]えなければならない。	一斉に= 
\\	乗客が一斉に乗ってきた。	
\\	乗客[じょうきゃく]が 一斉[いっせい]に 乗[の]ってきた。	一斉に= 
\\	学生は皆一斉に立ち上がった。	
\\	学生[がくせい]は 皆[みな] 一斉[いっせい]に 立ち上[たちあ]がった。	一斉に= 
\\	彼は代理で賞を受けてほしいと私に頼みました。	
\\	彼[かれ]は 代理[だいり]で 賞[しょう]を 受[う]けてほしいと 私[わたし]に 頼[たの]みました。	代理=だいり= 
\\	友達が外で野球をしようと誘ってきた。	
\\	友達[ともだち]が 外[そと]で 野球[やきゅう]をしようと 誘[さそ]ってきた。	
\\	彼はいろいろと尋ねたが、私には答えられなかった。	
\\	彼[かれ]はいろいろと 尋[たず]ねたが、 私[わたし]には 答[こた]えられなかった。	
\\	クラブ会員の大半と面識がある。	
\\	クラブ 会員[かいいん]の 大半[たいはん]と 面識[めんしき]がある。	大半=たいはん= 
\\	面識= 
\\	私は彼に面識がない。	
\\	私は 彼[かれ]に 面識[めんしき]がない。	面識= 
\\	私はスミス氏と面識がある。	
\\	私はスミス 氏[し]と 面識[めんしき]がある。	面識= 
\\	私は筆跡を見ただけで誰が書いたか分かった。	
\\	私[わたし]は 筆跡[ひっせき]を 見[み]ただけで 誰[だれ]が 書[か]いたか 分[わ]かった。	
\\	この植木は二週間に一度水をやるだけでよい。	
\\	この 植木[うえき]は 二週間[にしゅうかん]に一 度[ど] 水[すい]をやるだけでよい。	植木=うえき= 
\\	この用紙にサインしていただくだけで結構です。	
\\	この 用紙[ようし]にサインしていただくだけで 結構[けっこう]です。	
\\	本当に来るだけでいいんですか。	
\\	本当[ほんとう]に 来[く]るだけでいいんですか。	
\\	その会議では私はただ座っているだけでよかった。	
\\	その 会議[かいぎ]では 私[わたし]はただ 座[すわ]っているだけでよかった。	
\\	村上君は一週間勉強しただけであの試験に通ったそうだ。	
\\	村上[むらかみ] 君[くん]は一 週間[しゅうかん] 勉強[べんきょう]しただけであの 試験[しけん]に 通[とお]ったそうだ。	
\\	お金を入れてボタンを押すだけで温かいラーメンが出てくる自動販売機がある。	
\\	お 金[かね]を 入[い]れてボタンを 押[お]すだけで 温[あたた]かいラーメンが 出[で]てくる 自動[じどう] 販売[はんばい] 機[き]がある。	
\\	頭金一万円を払うだけで品物をお届けします。	
\\	頭金[あたまきん]一 万[まん] 円[えん]を 払[はら]うだけで 品物[しなもの]をお 届[とど]けします。	頭金=あたまきん= 
\\	聞くだけで胸が悪くなるような話だ。	
\\	聞[き]くだけで 胸[むね]が 悪[わる]くなるような 話[はなし]だ。	
\\	スミスさんはいつも泥だらけの靴を履いています。	
\\	スミスさんはいつも 泥[どろ]だらけの 靴[くつ]を 履[は]いています。	
\\	長いこと掃除をしていなかったらしく、床も机の上も埃だらけだった。	
\\	長[なが]いこと 掃除[そうじ]をしていなかったらしく、 床[ゆか]も 机[つくえ]の 上[うえ]も 埃[ほこり]だらけだった。	床=ゆか= 
\\	病院に担ぎ込んだ時、その男の顔は血だらけだった。	
\\	病院[びょういん]に 担[かつ]ぎ 込[こ]んだ 時[とき]、その 男[おとこ]の 顔[かお]は 血[ち]だらけだった。	担ぎ込む= 
\\	泥だらけの足で入って来ないで。	
\\	泥[どろ]だらけの 足[あし]で 入[はい]って 来[こ]ないで。	
\\	借金だらけの生活をしています。	
\\	借金[しゃっきん]だらけの 生活[せいかつ]をしています。	
\\	この貝は砂だらけで食べにくい。	
\\	この 貝[かい]は 砂[すな]だらけで 食[た]べにくい。	
\\	床の上には札束がゴロゴロしていた。	
\\	床[ゆか]の 上[うえ]には 札束[さつたば]がゴロゴロしていた。	札束=さつたば= 
\\	もうヘトヘトだ。	
\\	もうヘトヘトだ。	
\\	この暑さと湿気のせいで私はもうヘトヘトだ。	
\\	この 暑[あつ]さと 湿気[しっけ]のせいで 私[わたし]はもうヘトヘトだ。	湿気=しっけ= 
\\	もうヘトヘトに疲れたよ。今日はこれくらいで終わりにしよう。	
\\	もうヘトヘトに 疲[つか]れたよ。 今日[きょう]はこれくらいで 終[お]わりにしよう。	
\\	ストレスでへとへとみたいですね。	
\\	ストレスでへとへとみたいですね。	
\\	徹夜で仕事をして、もうへとへとだ。	
\\	徹夜[てつや]で 仕事[しごと]をして、もうへとへとだ。	
\\	迷子になったアルツハイマー病の患者を探し出さなければならない。	
\\	迷子[まいご]になったアルツハイマー 病[びょう]の 患者[かんじゃ]を 探し出[さがしだ]さなければならない。	迷子になる= 
\\	うちの子どもが迷子になっているんです。	
\\	うちの 子[こ]どもが 迷子[まいご]になっているんです。	迷子になる= 
\\	つい元ガールフレンドのアパートに、足が向いてしまった。	
\\	つい 元[もと]ガールフレンドのアパートに、 足[あし]が 向[む]いてしまった。	
\\	彼は、暴力団から足を洗う決心をした。	
\\	彼[かれ]は、 暴力団[ぼうりょくだん]から 足[あし]を 洗[あら]う 決心[けっしん]をした。	足を洗う= 
\\	彼は私が入院していた病院に、何度か足を運んでくれた。	
\\	彼[かれ]は 私[わたし]が 入院[にゅういん]していた 病院[びょういん]に、 何[なん] 度[ど]か 足[あし]を 運[はこ]んでくれた。	足を運ぶ= 
\\	彼女は25歳の時、政治の世界に足を踏み入れた。	
\\	彼女[かのじょ]は 25歳[にじゅうごさい]の 時[とき]、 政治[せいじ]の 世界[せかい]に 足[あし]を 踏み入[ふみい]れた。	足を踏み入れる= 
\\	彼の謙虚な姿勢には、全く頭が下がる。	
\\	彼[かれ]の 謙虚[けんきょ]な 姿勢[しせい]には、 全[まった]く 頭[あたま]が 下[さ]がる。	謙虚=けんきょ= 
\\	頭が下がる= 
\\	一晩頭を冷やしてから、決断を下すことにしよう。	
\\	一 晩[ばん] 頭[あたま]を 冷[ひ]やしてから、 決断[けつだん]を 下[くだ]すことにしよう。	頭を冷やす= 
\\	決断を下す= 
\\	時間切れ寸前に、いい考えが頭に浮かんだ。	
\\	時間切[じかんぎ]れ 寸前[すんぜん]に、いい 考[かんが]えが 頭[あたま]に 浮[う]かんだ。	寸前に= 
\\	頭に浮かぶ= 
\\	彼らは何か解決策はないかと、頭をひねった。	
\\	彼[かれ]らは 何[なに]か 解決[かいけつ] 策[さく]はないかと、 頭[あたま]をひねった。	頭をひねる= 
\\	その支援団体のメンバーは、資金難に頭を抱えている。	
\\	その 支援[しえん] 団体[だんたい]のメンバーは、 資金[しきん] 難[なん]に 頭[あたま]を 抱[かか]えている。	頭を抱える= 
\\	私が言っていることの意味が分かりますか。	
\\	私[わたし]が 言[い]っていることの 意味[いみ]が 分[わ]かりますか。	
\\	この単語の意味を辞書で調べてごらん。	
\\	この 単語[たんご]の 意味[いみ]を 辞書[じしょ]で 調[しら]べてごらん。	
\\	私は彼が残したメッセージの意味を考えた。	
\\	私[わたし]は 彼[かれ]が 残[のこ]したメッセージの 意味[いみ]を 考[かんが]えた。	
\\	私たちの多くは、その国についていいイメージを持っていない。	
\\	私[わたし]たちの 多[おお]くは、その 国[くに]についていいイメージを 持[も]っていない。	
\\	茶髪にして、彼女はすっかりイメージが変わった。	
\\	茶髪[ちゃぱつ]にして、 彼女[かのじょ]はすっかりイメージが 変[か]わった。	
\\	私には、その遠く離れた国がどんなところなのか、イメージが沸かない。	
\\	私[わたし]には、その 遠[とお]く 離[はな]れた 国[くに]がどんなところなのか、イメージが 沸[わ]かない。	イメージが沸く= 
\\	このアルバイトは1時間で千円払ってくれます。	
\\	このアルバイトは 1時間[いちじかん]で 千[せん] 円[えん] 払[はら]ってくれます。	
\\	昨日は1日で本を五百ページ読んだ。	
\\	昨日[きのう]は 1日[いちにち]で 本[ほん]を 五百[ごひゃく]ページ 読[よ]んだ。	
\\	食べて飲んで、五人で、七万円ぐらいでした。	
\\	食[た]べて 飲[の]んで、五 人[にん]で、七 万[まん] 円[えん]ぐらいでした。	
\\	このりんごは一山で二百円です。	
\\	このりんごは 一山[ひとやま]で 二百[にひゃく] 円[えん]です。	
\\	このようなブームはもう二度と起こらないであろう。	
\\	このようなブームはもう 二度[にど]と 起[お]こらないであろう。	
\\	この次に起こる地震は非常に大きいであろうと予想される。	
\\	この 次[つぎ]に 起[お]こる 地震[じしん]は 非常[ひじょう]に 大[おお]きいであろうと 予想[よそう]される。	
\\	その交渉は極めて困難であろう。	
\\	その 交渉[こうしょう]は 極[きわ]めて 困難[こんなん]であろう。	
\\	私達はいつか起こるであろう大地震に対して備えておかなければならない。	
\\	私[わたし] 達[たち]はいつか 起[お]こるであろう 大[だい] 地震[じしん]に 対[たい]して 備[そな]えておかなければならない。	備える=そなえる= 
\\	彼はもうここへは来るまい。	
\\	彼[かれ]はもうここへは 来[く]るまい。	「まい」
\\	今年の水不足は極めて深刻である。	
\\	今年[ことし]の 水不足[みずぶそく]は 極[きわ]めて 深刻[しんこく]である。	
\\	大型車に一人で乗るのは不経済である。	
\\	大型[おおがた] 車[しゃ]に一 人[にん]で 乗[の]るのは 不経済[ふけいざい]である。	
\\	彼に何度も手紙を書いたが無駄であった。	
\\	彼[かれ]に 何[なん] 度[ど]も 手紙[てがみ]を 書[か]いたが 無駄[むだ]であった。	
\\	この方法の方が効果的であるらしい。	
\\	この 方法[ほうほう]の 方[ほう]が 効果[こうか] 的[てき]であるらしい。	
\\	「だ」、「である」
\\	「らしい」&「かもしれない」 
\\	「らしい」
\\	食事はご馳走どころか、豚の餌みたいだった。	
\\	食事[しょくじ]はご 馳走[ちそう]どころか、 豚[ぶた]の 餌[えさ]みたいだった。	
\\	私の父は丈夫どころか、寝たきりです。	
\\	私[わたし]の 父[ちち]は 丈夫[じょうぶ]どころか、 寝[ね]たきりです。	
\\	鈴木さんはどうも京都大学に入りたいらしい。	
\\	鈴木[すずき]さんはどうも 京都[きょうと] 大学[だいがく]に 入[はい]りたいらしい。	「どうも」
\\	「らしい」
\\	あの先生の授業はどうも面白くない。	
\\	あの 先生[せんせい]の 授業[じゅぎょう]はどうも 面白[おもしろ]くない。	「どうも」
\\	このごろどうも体の調子がよくないんです。	
\\	このごろどうも 体[からだ]の 調子[ちょうし]がよくないんです。	
\\	あの人の日本語はどうも聞きにくい。	
\\	あの 人[ひと]の 日本語[にほんご]はどうも 聞[き]きにくい。	
\\	先生、この問題の意味がどうもつかめないんです。	
\\	先生[せんせい]、この 問題[もんだい]の 意味[いみ]がどうもつかめないんです。	つかむ= 
\\	こんな田舎に住むのはどうも不便だ。	
\\	こんな 田舎[いなか]に 住[す]むのはどうも 不便[ふべん]だ。	
\\	うちの子はテレビばかり見て、どうも本を読まない。	
\\	うちの 子[こ]はテレビばかり 見[み]て、どうも 本[ほん]を 読[よ]まない。	
\\	あの先生はどうも厳しいようだ。	
\\	あの 先生[せんせい]はどうも 厳[きび]しいようだ。	
\\	ジョンは日本語を話すのは上手だが、読むのはどうも下手なようです。	
\\	ジョンは 日本語[にほんご]を 話[はな]すのは 上手[じょうず]だが、 読[よ]むのはどうも 下手[へた]なようです。	
\\	その男がどうも犯人に違いないと思っていたが、やっぱりそうだった。	
\\	その 男[おとこ]がどうも 犯人[はんにん]に 違[ちが]いないと 思[おも]っていたが、やっぱりそうだった。	
\\	父は症状からしてどうもがんになったらしい。	
\\	父[ちち]は 症状[しょうじょう]からしてどうもがんになったらしい。	「らしい」
\\	彼女が今朝電車の中で僕に言ったことがどうも気になった。	
\\	彼女[かのじょ]が 今朝[けさ] 電車[でんしゃ]の 中[なか]で 僕[ぼく]に 言[い]ったことがどうも 気[き]になった。	
\\	兄は病気が治ってどんなにうれしかったことか。	
\\	兄[あに]は 病気[びょうき]が 治[なお]ってどんなにうれしかったことか。	「どんなに〜(こと)か」
\\	いたずら電話がよくかかってくるんだ。	
\\	いたずら 電話[でんわ]がよくかかってくるんだ。	悪戯=いたずら= 
\\	彼の目はいかにもいたずらっぽかった。	
\\	彼[かれ]の 目[め]はいかにもいたずらっぽかった。	
\\	そのお祭りのときは本当に盛り上がります。	
\\	そのお 祭[まつ]りのときは 本当[ほんとう]に 盛り上[もりあ]がります。	
\\	友達をたくさん呼んでみんなで盛り上がろう。	
\\	友達[ともだち]をたくさん 呼[よ]んでみんなで 盛り上[もりあ]がろう。	
\\	受けても、どうせ駄目だから、文部省の留学生試験を受けないことにしました。	
\\	受[う]けても、どうせ 駄目[だめ]だから、 文部省[もんぶしょう]の 留学生[りゅうがくせい] 試験[しけん]を 受[う]けないことにしました。	
\\	人間はどうせ死ぬんだから、あくせく働いても仕方がない。	
\\	人間[にんげん]はどうせ 死[し]ぬんだから、あくせく 働[はたら]いても 仕方[しかた]がない。	あくせく= 
\\	私の家の庭は日本風に大きい石が置いてあります。	
\\	私[わたし]の 家[いえ]の 庭[にわ]は 日本[にっぽん] 風[ふう]に 大[おお]きい 石[いし]が 置[お]いてあります。	
\\	あんな風に勉強していたらいい成績は取れないだろう。	
\\	あんな 風[かぜ]に 勉強[べんきょう]していたらいい 成績[せいせき]は 取[と]れないだろう。	
\\	あんな風に毎日飲んでいたら、きっと病気になるでしょう。	
\\	あんな 風[かぜ]に 毎日[まいにち] 飲[の]んでいたら、きっと 病気[びょうき]になるでしょう。	
\\	このカレーライスはインド風に、とても辛くしてあります。	
\\	このカレーライスはインド 風[ふう]に、とても 辛[つら]くしてあります。	
\\	ジョンが日本風のお辞儀をした時にはびっくりした。	
\\	ジョンが 日本[にっぽん] 風[ふう]のお 辞儀[じぎ]をした 時[とき]にはびっくりした。	
\\	お母さんが入院なさったという風に人から伺いましたが、いかがですか。	
\\	お 母[かあ]さんが 入院[にゅういん]なさったという 風[かぜ]に 人[ひと]から 伺[うかが]いましたが、いかがですか。	
\\	私は小さい時病気がちでした。	
\\	私は 小[ちい]さい 時[とき] 病気[びょうき]がちでした。	
\\	夏はややもすると塩分が不足しがちだ。	
\\	夏[なつ]はややもすると 塩分[えんぶん]が 不足[ふそく]しがちだ。	ややもすると= 
\\	人はともすると自分の都合のいいように物事を考えがちだ。	
\\	人[ひと]はともすると 自分[じぶん]の 都合[つごう]のいいように 物事[ものごと]を 考[かんが]えがちだ。	ともすると= 
\\	若いうちはとかく物事を一途に考えがちだ。	
\\	若[わか]いうちはとかく 物事[ものごと]を 一途[いちず]に 考[かんが]えがちだ。	とかく= (ややもすれば) 
\\	[(よくないことを)あれこれ] 
\\	一途=いちず= 
\\	私は最近週末もうちを空けがちです。	
\\	私は 最近[さいきん] 週末[しゅうまつ]もうちを 空[あ]けがちです。	
\\	これはアメリカ人の学生が犯しがちな間違いだ。	
\\	これはアメリカ 人[じん]の 学生[がくせい]が 犯[おか]しがちな 間違[まちが]いだ。	
\\	明日は曇りがちの天気でしょう。	
\\	明日[あした]は 曇[くも]りがちの 天気[てんき]でしょう。	
\\	彼女は遠慮がちに話した。	
\\	彼女[かのじょ]は 遠慮[えんりょ]がちに 話[はな]した。	
\\	便秘ぎみです。	
\\	便秘[べんぴ]ぎみです。	ぎみ= 
\\	彼のしたことは許しがたい。	
\\	彼[かれ]のしたことは 許[ゆる]しがたい。	
\\	この旅行は私にとって忘れがたい思い出になるだろう。	
\\	この 旅行[りょこう]は 私[わたし]にとって 忘[わす]れがたい 思い出[おもいで]になるだろう。	
\\	彼の行為は理解しがたい。	
\\	彼[かれ]の 行為[こうい]は 理解[りかい]しがたい。	
\\	あの先生は偉すぎて私には近寄りがたい。	
\\	あの 先生[せんせい]は 偉[えら]すぎて 私[わたし]には 近寄[ちかよ]りがたい。	近寄る= 
\\	この二つの作品は甲乙付けがたい。	
\\	この 二[ふた]つの 作品[さくひん]は 甲乙[こうおつ] 付[つ]けがたい。	甲乙=こうおつ= 
\\	このプロジェクトは成功したとは言いがたい。	
\\	このプロジェクトは 成功[せいこう]したとは 言[い]いがたい。	
\\	ジョーンズ氏は得がたい人物だ。	
\\	ジョーンズ 氏[し]は 得[え]がたい 人物[じんぶつ]だ。	
\\	我々は彼の犯行に関する動かしがたい証拠をつかんだ。	
\\	我々[われわれ]は 彼[かれ]の 犯行[はんこう]に 関[かん]する 動[うご]かしがたい 証拠[しょうこ]をつかんだ。	
\\	薬を飲んだら、逆に熱が出た。	
\\	薬[くすり]を 飲[の]んだら、 逆[ぎゃく]に 熱[ねつ]が 出[で]た。	
\\	叱られると思ったのに、逆にほめられた。	
\\	叱[しか]られると 思[おも]ったのに、 逆[ぎゃく]にほめられた。	
\\	しばらく練習しなかったら、逆に成績が伸びた。	
\\	しばらく 練習[れんしゅう]しなかったら、 逆[ぎゃく]に 成績[せいせき]が 伸[の]びた。	
\\	寝すぎると、元気にならないで、逆に疲れてしまう。	
\\	寝[ね]すぎると、 元気[げんき]にならないで、 逆[ぎゃく]に 疲[つか]れてしまう。	
\\	ガールフレンドを喜ばせようとしたのに、逆に怒らせてしまった。	
\\	ガールフレンドを 喜[よろこ]ばせようとしたのに、 逆[ぎゃく]に 怒[おこ]らせてしまった。	
\\	その試験に落ちるだろうと思っていたのに、逆に一番で通ってしまった。	
\\	その 試験[しけん]に 落[お]ちるだろうと 思[おも]っていたのに、 逆[ぎゃく]に 一番[いちばん]で 通[かよ]ってしまった。	
\\	いい演技をするためには緊張しすぎてはいけない。しかし、逆にリラックスしすぎてもいい演技は出来ない。	
\\	いい 演技[えんぎ]をするためには 緊張[きんちょう]しすぎてはいけない。しかし、 逆[ぎゃく]にリラックスしすぎてもいい 演技[えんぎ]は 出来[でき]ない。	
\\	上空に行くほど酸素が薄くなる。	
\\	上空[じょうくう]に 行[い]くほど 酸素[さんそ]が 薄[うす]くなる。	
\\	私は難しい仕事ほどやる気が出てくる。	
\\	私は 難[むずか]しい 仕事[しごと]ほどやる 気[き]が 出[で]てくる。	
\\	子供は小言を言うほど反発するものだ。	
\\	子供[こども]は 小言[こごと]を 言[い]うほど 反発[はんぱつ]するものだ。	小言=こごと= 
\\	反発= 
\\	私は静かなほど落ち着かない。	
\\	私[わたし]は 静[しず]かなほど 落ち着[おちつ]かない。	
\\	駅に近くなるほど家賃が高くなる。	
\\	駅[えき]に 近[ちか]くなるほど 家賃[やちん]が 高[たか]くなる。	
\\	運動するほど体の調子が変になる。	
\\	運動[うんどう]するほど 体[からだ]の 調子[ちょうし]が 変[へん]になる。	
\\	君はいつも不平を言っている。	
\\	君[きみ]はいつも 不平[ふへい]を 言[い]っている。	不平=ふへい= 
\\	山田さん以外の人はみんなそのことを知っています。	
\\	山田[やまだ]さん 以外[いがい]の 人[ひと]はみんなそのことを 知[し]っています。	
\\	日本語以外に何か外国語が話せますか。	
\\	日本語 以外[いがい]に 何[なに]か 外国[がいこく] 語[ご]が 話[はな]せますか。	
\\	私は日本酒以外の酒は飲まない。	
\\	私[わたし]は 日本[にっぽん] 酒[しゅ] 以外[いがい]の 酒[さけ]は 飲[の]まない。	
\\	我々の会社では現在オーストラリア以外の国と取引はない。	
\\	我々[われわれ]の 会社[かいしゃ]では 現在[げんざい]オーストラリア 以外[いがい]の 国[くに]と 取引[とりひき]はない。	取引= 
\\	アメリカ以外の国からもたくさん研究者が来た。	
\\	アメリカ 以外[いがい]の 国[くに]からもたくさん 研究[けんきゅう] 者[しゃ]が 来[き]た。	
\\	原因はこれ以外に考えられない。	
\\	原因[げんいん]はこれ 以外[いがい]に 考[かんが]えられない。	
\\	私は散歩以外にも毎日軽い運動をしている。	
\\	私は 散歩[さんぽ] 以外[いがい]にも 毎日[まいにち] 軽[かる]い 運動[うんどう]をしている。	
\\	安い以外に何かいいことがありますか。	
\\	安[やす]い 以外[いがい]に 何[なに]かいいことがありますか。	
\\	この文は少し漢字の間違いがあるが、それ以外は完全だ。	
\\	この 文[ぶん]は 少[すこ]し 漢字[かんじ]の 間違[まちが]いがあるが、それ 以外[いがい]は 完全[かんぜん]だ。	
\\	お金以外であれば何でも貸してあげるよ。	
\\	お 金[かね] 以外[いがい]であれば 何[なに]でも 貸[か]してあげるよ。	
\\	金持ち以外とは付き合わないことにしている。	
\\	金持[かねも]ち 以外[いがい]とは 付き合[つきあ]わないことにしている。	
\\	この部屋以外で物を食べないで下さい。	
\\	この 部屋[へや] 以外[いがい]で 物[もの]を 食[た]べないで 下[くだ]さい。	
\\	日本に来た以上は、日本語をしっかり勉強したい。	
\\	日本に 来[き]た 以上[いじょう]は、 日本語[にほんご]をしっかり 勉強[べんきょう]したい。	
\\	学生である以上は、勉強すべきだ。	
\\	学生である 以上[いじょう]は、 勉強[べんきょう]すべきだ。	
\\	日本語を始めた以上、よく話せて、聞けて、読めて、書けるようになるまで頑張ります。	
\\	日本語を 始[はじ]めた 以上[いじょう]、よく 話[はな]せて、 聞[き]けて、 読[よ]めて、 書[か]けるようになるまで 頑張[がんば]ります。	
\\	新車を買う以上は、出来るだけ燃費のいいのを買いたいです。	
\\	新車[しんしゃ]を 買[か]う 以上[いじょう]は、 出来[でき]るだけ 燃費[ねんぴ]のいいのを 買[か]いたいです。	燃費=ねんぴ= 
\\	親である以上、子供の教育に関心があるのは当然でしょう。	
\\	親[おや]である 以上[いじょう]、 子供[こども]の 教育[きょういく]に 関心[かんしん]があるのは 当然[とうぜん]でしょう。	
\\	もらった以上は、あなたが何と言おうと、私の物です。	
\\	もらった 以上[いじょう]は、あなたが 何[なん]と 言[い]おうと、 私[わたし]の 物[もの]です。	
\\	体をよく動かしている以上は、人間の体は衰えないらしい。	
\\	体[からだ]をよく 動[うご]かしている 以上[いじょう]は、 人間[にんげん]の 体[からだ]は 衰[おとろ]えないらしい。	衰える= 
\\	「らしい」
\\	人と約束した以上は、それを守らなければならない。	
\\	人[ひと]と 約束[やくそく]した 以上[いじょう]は、それを 守[まも]らなければならない。	
\\	僕が生きている以上、お前に不自由をさせない。	
\\	僕[ぼく]が 生[い]きている 以上[いじょう]、お 前[まえ]に 不自由[ふじゆう]をさせない。	
\\	酒を飲み続けている以上、病気は治らないよ。	
\\	酒[さけ]を 飲[の]み 続[つづ]けている 以上[いじょう]、 病気[びょうき]は 治[なお]らないよ。	
\\	日本語のラジオを聞いている以上、日本語を聞く力は低下しないでしょうね。	
\\	日本語[にほんご]のラジオを 聞[き]いている 以上[いじょう]、 日本語[にほんご]を 聞[き]く 力[ちから]は 低下[ていか]しないでしょうね。	
\\	アメリカ人の目から見ると、日本の社会はいかにも閉鎖的だ。	
\\	アメリカ 人[じん]の 目[め]から 見[み]ると、 日本[にっぽん]の 社会[しゃかい]はいかにも 閉鎖[へいさ] 的[てき]だ。	
\\	外は雪が降っていて、いかにも寒そうだ。	
\\	外[そと]は 雪[ゆき]が 降[ふ]っていて、いかにも 寒[さむ]そうだ。	いかにも= 
\\	彼の書斎の本棚には古今東西の本が詰まっていて、いかにも学者の部屋らしい。	
\\	彼の 書斎[しょさい]の 本棚[ほんだな]には 古今[ここん] 東西[とうざい]の 本[ほん]が 詰[つ]まっていて、いかにも 学者[がくしゃ]の 部屋[へや]らしい。	書斎=しょさい= 
\\	古今東西= 
\\	いかにも= 
\\	先生は最近いかにもお忙しいようだ。	
\\	先生[せんせい]は 最近[さいきん]いかにもお 忙[いそが]しいようだ。	いかにも= 
\\	日本人は集団行動がいかにも好きではあるが、個人行動をしないわけではない。	
\\	日本人[にっぽんじん]は 集団[しゅうだん] 行動[こうどう]がいかにも 好[す]きではあるが、 個人[こじん] 行動[こうどう]をしないわけではない。	
\\	彼はいかにも紳士であるかのように振る舞っているが、なかなかの策士だ。	
\\	彼[かれ]はいかにも 紳士[しんし]であるかのように 振る舞[ふるま]っているが、なかなかの 策士[さくし]だ。	いかにも= 
\\	振る舞う= 
\\	策士= 
\\	彼の発想はいかにも日本的だ。	
\\	彼[かれ]の 発想[はっそう]はいかにも 日本[にっぽん] 的[てき]だ。	発想=はっそう= 
\\	いかにも= 
\\	その教授の知識はいかにも百科全書的だ。	
\\	その 教授[きょうじゅ]の 知識[ちしき]はいかにも 百科全書[ひゃっかぜんしょ] 的[てき]だ。	いかにも= 
\\	百科全書= 
\\	父は退院して、いかにも元気そうになった。	
\\	父[ちち]は 退院[たいいん]して、いかにも 元気[げんき]そうになった。	いかにも= 
\\	彼女は明るく、陽気で、いかにもアメリカ人らしい。	
\\	彼女[かのじょ]は 明[あか]るく、 陽気[ようき]で、いかにもアメリカ 人[じん]らしい。	陽気=ようき= 
\\	いかにも= 
\\	僕の大学の友人はいかにも金持ちらしく、いつもしゃれた物を着ている。	
\\	僕[ぼく]の 大学[だいがく]の 友人[ゆうじん]はいかにも 金持[かねも]ちらしく、いつもしゃれた 物[もの]を 着[き]ている。	いかにも= 
\\	ジョンは恋人と別れて、いかにも落ち込んでいるようだった。	
\\	ジョンは 恋人[こいびと]と 別[わか]れて、いかにも 落ち込[おちこ]んでいるようだった。	
\\	みゆきは母を失って、いかにも悲しんでいる様子だった。	
\\	みゆきは 母[はは]を 失[うしな]って、いかにも 悲[かな]しんでいる 様子[ようす]だった。	
\\	いかにもおっしゃる通りです。	
\\	いかにもおっしゃる 通[とお]りです。	いかにも= 
\\	日本語はいかにも難しい言語ではあるが、マスター出来ないわけではない。	
\\	日本語はいかにも 難[むずか]しい 言語[げんご]ではあるが、マスター 出来[でき]ないわけではない。	
\\	彼はいかにも全部分かっているかのように話しているが、その実何も分かっていない。	
\\	彼[かれ]はいかにも 全部[ぜんぶ] 分[わ]かっているかのように 話[はな]しているが、その 実[じつ] 何[なに]も 分[わ]かっていない。	その実= 
\\	私は世界の果てまで行って戻ってきました。	
\\	私は 世界[せかい]の 果[は]てまで 行[い]って 戻[もど]ってきました。	
\\	もうよせよ。	
\\	もうよせよ。	よせよ= 
\\	ここには十二月三十一日までに払えと書いてある。	
\\	ここには 十二月[じゅうにがつ] 三十一日[さんじゅういちにち]までに 払[はら]えと 書[か]いてある。	
\\	書け!	
\\	書[か]け!	
\\	黙れ!	
\\	黙[だま]れ!	
\\	動くな!	
\\	動[うご]くな!	
\\	山中首相は即時退陣せよ!	
\\	山中[やまなか] 首相[しゅしょう]は 即時[そくじ] 退陣[たいじん]せよ!	即時=そくじ= 
\\	退陣=たいじん= 
\\	次の文を英訳せよ。	
\\	次[つぎ]の 文[ぶん]を 英訳[えいやく]せよ。	
\\	次の質問に答えよ。	
\\	次[つぎ]の 質問[しつもん]に 答[こた]えよ。	
\\	現金は送るなと書いてある。	
\\	現金[げんきん]は 送[おく]るなと 書[か]いてある。	
\\	課長にあまりタクシーは使うなと言われた。	
\\	課長[かちょう]にあまりタクシーは 使[つか]うなと 言[い]われた。	
\\	神田先生は、一方では大学で物理学を教えながら、他方では日本語の研究をなさっている。	
\\	神田[かんだ]先生は、 一方[いっぽう]では 大学[だいがく]で 物理[ぶつり] 学[がく]を 教[おし]えながら、 他方[たほう]では 日本語[にほんご]の 研究[けんきゅう]をなさっている。	「一方...他方...」
\\	この薬は、一方では症状を軽くするが、他方では強い副作用がある。	
\\	この 薬[くすり]は、 一方[いっぽう]では 症状[しょうじょう]を 軽[かる]くするが、 他方[たほう]では 強[つよ]い 副作用[ふくさよう]がある。	「一方...他方...」
\\	あの男は、一方では静かな日本画を描いたりしているが、他方ではサッカーのような激しいスポーツをしている。	
\\	あの 男[おとこ]は、 一方[いっぽう]では 静[しず]かな 日本[にほん] 画[が]を 描[か]いたりしているが、 他方[たほう]ではサッカーのような 激[はげ]しいスポーツをしている。	「一方...他方...」
\\	山田氏は、一方で慈善事業をやりながら、他方でかなりあくどい商売をしているという噂だ。	
\\	山田[やまだ] 氏[し]は、 一方[いっぽう]で 慈善[じぜん] 事業[じぎょう]をやりながら、 他方[たほう]でかなりあくどい 商売[しょうばい]をしているという 噂[うわさ]だ。	慈善事業= 
\\	あくどい= 
\\	商売= 
\\	あの大統領は、一方では減税を約束しておきながら、他方では側近の税金の無駄使いをあまり重要視していない。	
\\	あの 大統領[だいとうりょう]は、 一方[いっぽう]では 減税[げんぜい]を 約束[やくそく]しておきながら、 他方[たほう]では 側近[そっきん]の 税金[ぜいきん]の 無駄[むだ] 使[づか]いをあまり 重要[じゅうよう] 視[し]していない。	側近=そっきん= 
\\	この映画は教育上よくない。	
\\	この 映画[えいが]は 教育[きょういく] 上[じょう]よくない。	
\\	便宜上私がこの部屋の鍵を預かっているんです。	
\\	便宜上[べんぎじょう] 私[わたし]がこの 部屋[へや]の 鍵[かぎ]を 預[あず]かっているんです。	便宜=べんぎ= 
\\	時間の制約上細かい説明は省略させていただきます。	
\\	時間[じかん]の 制約[せいやく] 上[じょう] 細[こま]かい 説明[せつめい]は 省略[しょうりゃく]させていただきます。	
\\	計算上はこれで正しい。	
\\	計算[けいさん] 上[じょう]はこれで 正[ただ]しい。	
\\	健康上の理由で引退することにした。	
\\	健康[けんこう] 上[じょう]の 理由[りゆう]で 引退[いんたい]することにした。	
\\	このような行為は道義上許せない。	
\\	このような 行為[こうい]は 道義[どうぎ] 上[じょう] 許[ゆる]せない。	動議=どうぎ= 
\\	この条件はこれからの取引上極めて不利だ。	
\\	この 条件[じょうけん]はこれからの 取引[とりひき] 上[じょう] 極[きわ]めて 不利[ふり]だ。	不利=ふり= 
\\	仕事の都合上こんな高いマンションに住んでいるんです。	
\\	仕事[しごと]の 都合[つごう] 上[じょう]こんな 高[たか]いマンションに 住[す]んでいるんです。	
\\	仕事の関係上、今この町を離れるわけにはいかないんです。	
\\	仕事[しごと]の 関係[かんけい] 上[じょう]、 今[いま]この 町[まち]を 離[はな]れるわけにはいかないんです。	
\\	理論上はこうなるはずなのだが、実際どうなるかは分からない。	
\\	理論[りろん] 上[じょう]はこうなるはずなのだが、 実際[じっさい]どうなるかは 分[わ]からない。	
\\	法律上は彼の行為は罪にならない。	
\\	法律[ほうりつ] 上[じょう]は 彼[かれ]の 行為[こうい]は 罪[つみ]にならない。	
\\	彼女は一身上の都合で会社を辞めることになった。	
\\	彼女[かのじょ]は 一身上[いっしんじょう]の 都合[つごう]で 会社[かいしゃ]を 辞[や]めることになった。	一身=いっしん= 
\\	この部品は製作上いくつかの問題がある。	
\\	この 部品[ぶひん]は 製作[せいさく] 上[じょう]いくつかの 問題[もんだい]がある。	
\\	薬を飲んだら、かえって病気がひどくなった。	
\\	薬[くすり]を 飲[の]んだら、かえって 病気[びょうき]がひどくなった。	
\\	日本へ行ったら日本語が上手になるかと思って、日本へ行ったんですが、日本人と英語でばかり話していたので、かえって、下手になって帰って来ました。	
\\	日本へ 行[い]ったら 日本語[にほんご]が 上手[じょうず]になるかと 思[おも]って、日本へ 行[い]ったんですが、日本人と 英語[えいご]でばかり 話[はな]していたので、かえって、 下手[へた]になって 帰[かえ]って 来[き]ました。	
\\	就職のことで三人の先生にアドバイスをしてもらったのですが、アドバイスが全然違うので、かえって、分からなくなってしまいました。	
\\	就職[しゅうしょく]のことで三 人[にん]の 先生[せんせい]にアドバイスをしてもらったのですが、アドバイスが 全然[ぜんぜん] 違[ちが]うので、かえって、 分[わ]からなくなってしまいました。	
\\	いわゆる一流大学で勉強するより小さな私立大学で勉強する方が、かえって、いい教育を受けることが出来る。	
\\	いわゆる 一流[いちりゅう] 大学[だいがく]で 勉強[べんきょう]するより 小[ちい]さな 私立[しりつ] 大学[だいがく]で 勉強[べんきょう]する 方[ほう]が、かえって、いい 教育[きょういく]を 受[う]けることが 出来[でき]る。	いわゆる= 
\\	アルコールも適量飲めば、かえって、体にいいそうだ。	
\\	アルコールも 適量[てきりょう] 飲[の]めば、かえって、 体[からだ]にいいそうだ。	
\\	あの人は学者というよりはむしろ教育者だと思う。	
\\	あの 人[ひと]は 学者[がくしゃ]というよりはむしろ 教育[きょういく] 者[しゃ]だと 思[おも]う。	
\\	京都へ行くのは冬よりむしろ春の方がよくありませんか。	
\\	京都[きょうと]へ 行[い]くのは 冬[ふゆ]よりむしろ 春[はる]の 方[ほう]がよくありませんか。	
\\	私がここにいる限り心配は無用です。	
\\	私がここにいる 限[かぎ]り 心配[しんぱい]は 無用[むよう]です。	無用=むよう= 
\\	田中さんが来ない限りこの会議は始められない。	
\\	田中[たなか]さんが 来[こ]ない 限[かぎ]りこの 会議[かいぎ]は 始[はじ]められない。	
\\	私の知っている限り彼は正直者です。	
\\	私の 知[し]っている 限[かぎ]り 彼[かれ]は 正直[しょうじき] 者[しゃ]です。	
\\	その書類は私が読んだ限り誤りはなかった。	
\\	その 書類[しょるい]は 私[わたし]が 読[よ]んだ 限[かぎ]り 誤[あやま]りはなかった。	
\\	教育者である限りそんなことは口にすべきではない。	
\\	教育[きょういく] 者[しゃ]である 限[かぎ]りそんなことは 口[くち]にすべきではない。	
\\	この試験に通らない限り上級クラスには入れません。	
\\	この 試験[しけん]に 通[とお]らない 限[かぎ]り 上級[じょうきゅう]クラスには 入[はい]れません。	
\\	事態が変わらない限り今以上の援助は不可能です。	
\\	事態[じたい]が 変[か]わらない 限[かぎ]り 今[いま] 以上[いじょう]の 援助[えんじょ]は 不可能[ふかのう]です。	
\\	アメリカ人でない限りこの仕事には就けない。	
\\	アメリカ 人[じん]でない 限[かぎ]りこの 仕事[しごと]には 就[つ]けない。	
\\	私達は力の続く限り漕いだ。	
\\	私 達[たち]は 力[ちから]の 続[つづ]く 限[かぎ]り 漕[こ]いだ。	
\\	出来る限りやってみます。	
\\	出来[でき]る 限[かぎ]りやってみます。	
\\	今度限りで彼のパーティーには行かないつもりだ。	
\\	今度[こんど] 限[かぎ]りで 彼[かれ]のパーティーには 行[い]かないつもりだ。	
\\	切符は一人二枚限りです。	
\\	切符[きっぷ]は 一人[ひとり] 二枚[にまい] 限[かぎ]りです。	
\\	この映画館は今月限りで閉館されます。	
\\	この 映画[えいが] 館[かん]は 今月[こんげつ] 限[かぎ]りで 閉館[へいかん]されます。	
\\	今日限りで酒もたばこもやめます。	
\\	今日[きょう] 限[かぎ]りで 酒[さけ]もたばこもやめます。	
\\	セールは明日限りです。	
\\	セールは 明日[あす] 限[かぎ]りです。	
\\	その場限りの約束はしない方がいい。	
\\	その 場[ば] 限[かぎ]りの 約束[やくそく]はしない 方[ほう]がいい。	
\\	貸し出しは一回三冊限りです。	
\\	貸[か]し 出[だ]しは 一回[いっかい] 三冊[さんさつ] 限[かぎ]りです。	
\\	書き直しは一回限りです。	
\\	書き直[かきなお]しは 一回[いっかい] 限[かぎ]りです。	
\\	来年日本へ行けるか行けないかまだ分かりません。	
\\	来年[らいねん] 日本[にほん]へ 行[い]けるか 行[い]けないかまだ 分[わ]かりません。	
\\	傘を駅に置き忘れたのか事務所に忘れたのか、はっきり覚えていません。	
\\	傘[かさ]を 駅[えき]に 置き忘[おきわす]れたのか 事務所[じむしょ]に 忘[わす]れたのか、はっきり 覚[おぼ]えていません。	
\\	大学を出てから就職するか大学院に入るかまだ決めていません。	
\\	大学[だいがく]を 出[で]てから 就職[しゅうしょく]するか 大学院[だいがくいん]に 入[はい]るかまだ 決[き]めていません。	
\\	お客さんが肉が好きか魚が好きか、聞いておいて下さい。	
\\	お 客[きゃく]さんが 肉[にく]が 好[す]きか 魚[さかな]が 好[す]きか、 聞[き]いておいて 下[くだ]さい。	
\\	夏休みにはヨーロッパを旅行するか、ソウルで仕事をするか、まだ決めていません。	
\\	夏休[なつやす]みにはヨーロッパを 旅行[りょこう]するか、ソウルで 仕事[しごと]をするか、まだ 決[き]めていません。	
\\	会議が木曜日だったか金曜日だったか、忘れてしまいました。	
\\	会議[かいぎ]が 木曜日[もくようび]だったか 金曜日[きんようび]だったか、 忘[わす]れてしまいました。	
\\	木村さんが大学で経済を専攻したのか、政治を専攻したのか、知っていますか。	
\\	木村[きむら]さんが 大学[だいがく]で 経済[けいざい]を 専攻[せんこう]したのか、 政治[せいじ]を 専攻[せんこう]したのか、 知[し]っていますか。	
\\	会議は月曜か水曜にして下さい。	
\\	会議[かいぎ]は 月曜[げつよう]か 水曜[すいよう]にして 下[くだ]さい。	
\\	今週末には何をしようかな。	
\\	今週[こんしゅう] 末[まつ]には 何[なに]をしようかな。	
\\	この問題、君に分かるかな。	
\\	この 問題[もんだい]、 君[きみ]に 分[わ]かるかな。	
\\	今日は何曜日だったかな。	
\\	今日[きょう]は 何[なに] 曜日[ようび]だったかな。	
\\	来学期から日本語を始めようかな。	
\\	来[らい] 学期[がっき]から 日本語[にほんご]を 始[はじ]めようかな。	
\\	日本での生活はどうかな。	
\\	日本[にほん]での 生活[せいかつ]はどうかな。	
\\	頭のいい人が必ずしも成功するとは限らない。	
\\	頭[あたま]のいい 人[ひと]が 必[かなら]ずしも 成功[せいこう]するとは 限[かぎ]らない。	必ずしも= 
\\	高い料理が必ずしもおいしいわけではない。	
\\	高[たか]い 料理[りょうり]が 必[かなら]ずしもおいしいわけではない。	必ずしも= 
\\	記憶力のいい人が必ずしも外国語が上手だとは言えない。	
\\	記憶[きおく] 力[りょく]のいい 人[ひと]が 必[かなら]ずしも 外国[がいこく] 語[ご]が 上手[じょうず]だとは 言[い]えない。	必ずしも= 
\\	日本人が必ずしもいい日本語の先生だとは思わない。	
\\	日本人[にほんじん]が 必[かなら]ずしもいい 日本語[にほんご]の 先生[せんせい]だとは 思[おも]わない。	必ずしも= 
\\	結婚しても必ずしも幸福になるとは言えない。	
\\	結婚[けっこん]しても 必[かなら]ずしも 幸福[こうふく]になるとは 言[い]えない。	必ずしも= 
\\	運動をよくする人が必ずしも長生きするとは限らない。	
\\	運動[うんどう]をよくする 人[ひと]が 必[かなら]ずしも 長生[ながい]きするとは 限[かぎ]らない。	必ずしも= 
\\	お金は人を必ずしも幸福にはしない。	
\\	お 金[かね]は 人[ひと]を 必[かなら]ずしも 幸福[こうふく]にはしない。	必ずしも= 
\\	優れた研究者が必ずしも優れた教育者であるわけではない。	
\\	優[すぐ]れた 研究[けんきゅう] 者[しゃ]が 必[かなら]ずしも 優[すぐ]れた 教育[きょういく] 者[しゃ]であるわけではない。	
\\	眼鏡をかけて、カメラを下げて、集団で歩いている東洋人が必ずしも日本人ではない。	
\\	眼鏡[めがね]をかけて、カメラを 下[さ]げて、 集団[しゅうだん]で 歩[ある]いている 東洋[とうよう] 人[じん]が 必[かなら]ずしも日本人ではない。	必ずしも= 
\\	光る物が全部金ではない。	
\\	光[ひか]る 物[もの]が 全部[ぜんぶ] 金[きん]ではない。	
\\	友達はみんな結婚している。	
\\	友達[ともだち]はみんな 結婚[けっこん]している。	
\\	来週はロンドンへ出張しますので、誠に申し訳ございませんが、名古屋での会議には出席できかねます。	
\\	来週[らいしゅう]はロンドンへ 出張[しゅっちょう]しますので、 誠[まこと]に 申し訳[もうしわけ]ございませんが、 名古屋[なごや]での 会議[かいぎ]には 出席[しゅっせき]できかねます。	「ーかねる」
\\	これだけの書類を一月では処理いたしかねますが。	
\\	これだけの 書類[しょるい]を 一月[ひとつき]では 処理[しょり]いたしかねますが。	-かねる= 
\\	大変遺憾に存じますが、ご依頼には応じかねます。	
\\	大変[たいへん] 遺憾[いかん]に 存[ぞん]じますが、ご 依頼[いらい]には 応[おう]じかねます。	「ーかねる」
\\	こんな高価な贈り物、いただきかねます。	
\\	こんな 高価[こうか]な 贈り物[おくりもの]、いただきかねます。	-かねる= 
\\	彼は私の気持ちを量りかねているようだ。	
\\	彼[かれ]は 私[わたし]の 気持[きも]ちを 量[はか]りかねているようだ。	「ーかねる」
\\	測る・計る・量る=はかる= 
\\	あいつはとんでもないことを言いかねないから、注意した方がいいよ。	
\\	あいつはとんでもないことを 言[い]いかねないから、 注意[ちゅうい]した 方[ほう]がいいよ。	「ーかねない」= 
\\	あの男はちょっとしたことで暴力を振るいかねない。	
\\	あの 男[おとこ]はちょっとしたことで 暴力[ぼうりょく]を 振[ふ]るいかねない。	「ーかねない」= 
\\	振う=ふるう= 
\\	そいつはちょっと賛成しにくいな。	
\\	そいつはちょっと 賛成[さんせい]しにくいな。	
\\	スミスさんは週刊誌から学術書に至るまで、幅広い日本語が読める。	
\\	スミスさんは 週刊[しゅうかん] 誌[し]から 学術[がくじゅつ] 書[しょ]に 至[いた]るまで、 幅広[はばひろ]い 日本語[にほんご]が 読[よ]める。	「〜から〜に至るまで」= 
\\	その新聞記者は首相の公の生活から私生活に至るまで、何でも知っている。	
\\	その 新聞[しんぶん] 記者[きしゃ]は 首相[しゅしょう]の 公[おおやけ]の 生活[せいかつ]から 私生活[しせいかつ]に 至[いた]るまで、 何[なに]でも 知[し]っている。	「〜から〜に至るまで」= 
\\	公=おおやけ= 
\\	私が日本へ行った時、友人の山田さんは空港への出迎えからホテルの予約に至るまで、実に親切にしてくれた。	
\\	私が日本へ 行[い]った 時[とき]、 友人[ゆうじん]の 山田[やまだ]さんは 空港[くうこう]への 出迎[でむか]えからホテルの 予約[よやく]に 至[いた]るまで、 実[じつ]に 親切[しんせつ]にしてくれた。	「〜から〜に至るまで」= 
\\	その女の人は私に家族のことから自分の悩みに至るまで、細かに話した。	
\\	その 女[おんな]の 人[ひと]は 私[わたし]に 家族[かぞく]のことから 自分[じぶん]の 悩[なや]みに 至[いた]るまで、 細[こま]かに 話[はな]した。	「〜から〜に至るまで」= 
\\	社長が現れた時には、守衛から副社長に至るまで、門の前で待っていた。	
\\	社長[しゃちょう]が 現[あらわ]れた 時[とき]には、 守衛[しゅえい]から 副[ふく] 社長[しゃちょう]に 至[いた]るまで、 門[もん]の 前[まえ]で 待[ま]っていた。	「〜から〜に至るまで」= 
\\	ルーシーは靴から帽子に至るまで、緑の装束だった。	
\\	ルーシーは 靴[くつ]から 帽子[ぼうし]に 至[いた]るまで、 緑[みどり]の 装束[しょうぞく]だった。	「〜から〜に至るまで」= 
\\	今年の夏、七月から八月にかけて中国大陸を旅行した。	
\\	今年[ことし]の 夏[なつ]、 七月[しちがつ]から 八月[はちがつ]にかけて 中国[ちゅうごく] 大陸[たいりく]を 旅行[りょこう]した。	「〜から〜にかけて」 
\\	今週は木曜から金曜にかけて雪が降るでしょう。	
\\	今週[こんしゅう]は 木曜[もくよう]から 金曜[きんよう]にかけて 雪[ゆき]が 降[ふ]るでしょう。	
\\	日本は六月から七月にかけて梅雨が続く。	
\\	日本は 六月[ろくがつ]から 七月[しちがつ]にかけて 梅雨[つゆ]が 続[つづ]く。	
\\	高気圧が朝鮮半島から九州にかけて張り出している。	
\\	高気圧[こうきあつ]が 朝鮮半島[ちょうせんはんとう]から 九州[きゅうしゅう]にかけて 張り出[はりだ]している。	
\\	午前九時から午後五時まで水道が止まります。	
\\	午前[ごぜん] 九時[くじ]から 午後五[ごごご] 時[じ]まで 水道[すいどう]が 止[と]まります。	
\\	子供だからと言って許すわけにはいかない。	
\\	子供[こども]だからと 言[い]って 許[ゆる]すわけにはいかない。	「からと言って」= 
\\	何も不平を言わないからと言って現状に満足しているわけではない。	
\\	何[なに]も 不平[ふへい]を 言[い]わないからと 言[い]って 現状[げんじょう]に 満足[まんぞく]しているわけではない。	「からと言って」= 
\\	試験に受からなかったからと言ってそんなに悲観することはない。	
\\	試験[しけん]に 受[う]からなかったからと 言[い]ってそんなに 悲観[ひかん]することはない。	「からと言って」= 
\\	弁償したからと言って済む問題ではない。	
\\	弁償[べんしょう]したからと 言[い]って 済[す]む 問題[もんだい]ではない。	弁償=
\\	「からと言って」= 
\\	上司の命令だからと言って黙って従うわけにはいかない。	
\\	上司[じょうし]の 命令[めいれい]だからと 言[い]って 黙[だま]って 従[したが]うわけにはいかない。	「からと言って」= 
\\	アメリカへ行ったからと言って勝手に英語が上手になるものではない。	
\\	アメリカへ 行[い]ったからと 言[い]って 勝手[かって]に 英語[えいご]が 上手[じょうず]になるものではない。	「からと言って」= 
\\	毎日授業に出ているからと言ってまじめに勉強していることにはならない。	
\\	毎日[まいにち] 授業[じゅぎょう]に 出[で]ているからと 言[い]ってまじめに 勉強[べんきょう]していることにはならない。	「からと言って」= 
\\	こんなことを言うからと言って別に批判しているわけではない。	
\\	こんなことを 言[い]うからと 言[い]って 別[べつ]に 批判[ひはん]しているわけではない。	「からと言って」= 
\\	自分の問題じゃないからと言って知らん顔をしているのはよくない。	
\\	自分[じぶん]の 問題[もんだい]じゃないからと 言[い]って 知らん顔[しらんかお]をしているのはよくない。	「からと言って」= 
\\	女だからと言って侮ってはいけない。	
\\	女[おんな]だからと 言[い]って 侮[あなど]ってはいけない。	侮る=あなどる= 
\\	ちょっと出来るからって、そんなに威張らなくてもいいでしょう。	
\\	ちょっと 出来[でき]るからって、そんなに 威張[いば]らなくてもいいでしょう。	威張る=いばる= 
\\	「からって」
\\	「からと言って」= 
\\	十年後の自分を予想するのは難しかろう。	
\\	十年[じゅうねん] 後[ご]の 自分[じぶん]を 予想[よそう]するのは 難[むずか]しかろう。	かろう= 
\\	林田先生は厳しい教え方をする。	
\\	林田[はやしだ] 先生[せんせい]は 厳[きび]しい 教[おし]え 方[かた]をする。	
\\	昭は変わったものの見方をする。	
\\	昭[あきら]は 変[か]わったものの 見方[みかた]をする。	
\\	ブラウンさんは日本人のような考え方をする。	
\\	ブラウンさんは 日本人[にほんじん]のような 考え方[かんがえかた]をする。	
\\	誰にでも分かるような書き方をして下さい。	
\\	誰[だれ]にでも 分[わ]かるような 書き方[かきかた]をして 下[くだ]さい。	
\\	あのピッチャーは面白い投げ方をする。	
\\	あのピッチャーは 面白[おもしろ]い 投[な]げ 方[かた]をする。	
\\	吉田君は乱暴な運転のし方をするので乗せてもらうのが怖い。	
\\	吉田[よしだ] 君[くん]は 乱暴[らんぼう]な 運転[うんてん]のし 方[かた]をするので 乗[の]せてもらうのが 怖[こわ]い。	
\\	正はほかの学生と違った勉強のし方をしているようだ。	
\\	正[ただし]はほかの 学生[がくせい]と 違[ちが]った 勉強[べんきょう]のし 方[かた]をしているようだ。	
\\	私には野村先生のような考え方は出来ない。	
\\	私[わたし]には 野村[のむら] 先生[せんせい]のような 考え方[かんがえかた]は 出来[でき]ない。	
\\	勉強が楽しくなるような教え方をしてほしい。	
\\	勉強[べんきょう]が 楽[たの]しくなるような 教[おし]え 方[かた]をしてほしい。	
\\	この問題について私と同じような考え方をする人は多くないでしょう。	
\\	この 問題[もんだい]について 私[わたし]と 同[おな]じような 考え方[かんがえかた]をする 人[ひと]は 多[おお]くないでしょう。	
\\	田村はその大臣をよく知っているような話し方をする。	
\\	田村[たむら]はその 大臣[だいじん]をよく 知[し]っているような 話し方[はなしかた]をする。	
\\	私達は丸く座った。	
\\	私[わたし] 達[たち]は 丸[まる]く 座[すわ]った。	
\\	日本人はみんな寿司が好きかと言うと、そうではない。嫌いな人もいる。	
\\	日本人[にほんじん]はみんな 寿司[すし]が 好[す]きかと 言[い]うと、そうではない。 嫌[きら]いな 人[ひと]もいる。	かと言うと= 
\\	がんは治らない病気かと言うと、そうではない。早期発見をすれば治ると言われている。	
\\	がんは 治[なお]らない 病気[びょうき]かと 言[い]うと、そうではない。 早期[そうき] 発見[はっけん]をすれば 治[なお]ると 言[い]われている。	かと言うと= 
\\	日本に行って、二、三年住めば日本語が上手になるかと言うと、そうでもないようだ。かえって下手になることもある。	
\\	日本に 行[い]って、 二[に]、 三年[さんねん] 住[す]めば 日本語[にほんご]が 上手[じょうず]になるかと 言[い]うと、そうでもないようだ。かえって 下手[へた]になることもある。	かと言うと= 
\\	大学の時にいい成績の学生が社会で成功するかと言うと、必ずしもそうではないようだ。	
\\	大学[だいがく]の 時[とき]にいい 成績[せいせき]の 学生[がくせい]が 社会[しゃかい]で 成功[せいこう]するかと 言[い]うと、 必[かなら]ずしもそうではないようだ。	かと言うと= 
\\	毎日運動をすれば長生きをするかと言うと、そうでもなさそうだ。	
\\	毎日[まいにち] 運動[うんどう]をすれば 長生[ながい]きをするかと 言[い]うと、そうでもなさそうだ。	かと言うと= 
\\	日本語は難しいかと言うと、話したり聞いたりすることはそんなに難しくない。	
\\	日本語[にほんご]は 難[むずか]しいかと 言[い]うと、 話[はな]したり 聞[き]いたりすることはそんなに 難[むずか]しくない。	かと言うと= 
\\	ボストンでの車なしの生活が不便だったかと言うと、全然そうではなかったんです。	
\\	ボストンでの 車[くるま]なしの 生活[せいかつ]が 不便[ふべん]だったかと 言[い]うと、 全然[ぜんぜん]そうではなかったんです。	かと言うと= 
\\	漫画はくだらないかと言うと、中にはとてもいい漫画もある。	
\\	漫画[まんが]はくだらないかと 言[い]うと、 中[なか]にはとてもいい 漫画[まんが]もある。	下らない= 
\\	かと言うと= 
\\	その医師は患者に一連の治療を施しました。	
\\	その 医師[いし]は 患者[かんじゃ]に 一連[いちれん]の 治療[ちりょう]を 施[ほどこ]しました。	一連=いちれん= 
\\	書き始める前に構想を綿密に練った方がいいですよ。	
\\	書[か]き 始[はじ]める 前[まえ]に 構想[こうそう]を 綿密[めんみつ]に 練[ね]った 方[ほう]がいいですよ。	構想=こうそう= 
\\	綿密=めんみつ= 
\\	練る= 
\\	その祭りでは大勢の人が踊りながら通りを練り歩く。	
\\	その 祭[まつ]りでは 大勢[おおぜい]の 人[ひと]が 踊[おど]りながら 通[とお]りを 練[ね]り 歩[ある]く。	練り歩く=ねりあるく= 
\\	昼間遊ぶ代わりに夜勉強するつもりだ。	
\\	昼間[ひるま] 遊[あそ]ぶ 代[か]わりに 夜[よる] 勉強[べんきょう]するつもりだ。	
\\	山田さんにはちょっと余分に働いてもらった代わりに特別手当てを出した。	
\\	山田[やまだ]さんにはちょっと 余分[よぶん]に 働[はたら]いてもらった 代[か]わりに 特別[とくべつ] 手当[てあ]てを 出[だ]した。	余分=よぶん= 
\\	手当て=てあて= 
\\	前のアパートは設備が悪かった代わりに家賃が安かった。	
\\	前[まえ]のアパートは 設備[せつび]が 悪[わる]かった 代[か]わりに 家賃[やちん]が 安[やす]かった。	
\\	今朝は、コーヒーの代わりにココアを飲んだ。	
\\	今朝[けさ]は、コーヒーの 代[か]わりにココアを 飲[の]んだ。	
\\	今日は図書館で勉強する代わりに寮の部屋で勉強した。	
\\	今日[きょう]は 図書館[としょかん]で 勉強[べんきょう]する 代[か]わりに 寮[りょう]の 部屋[へや]で 勉強[べんきょう]した。	
\\	私はトムに日本語を教えてあげた代わりに彼に英語を教えてもらった。	
\\	私はトムに 日本語[にほんご]を 教[おし]えてあげた 代[か]わりに 彼[かれ]に 英語[えいご]を 教[おし]えてもらった。	
\\	高い長距離電話をかける代わりに、手紙をよく書いています。	
\\	高[たか]い 長距離[ちょうきょり] 電話[でんわ]をかける 代[か]わりに、 手紙[てがみ]をよく 書[か]いています。	
\\	私のアパートは家賃が高い代わりに、駅に近くてとても便利です。	
\\	私のアパートは 家賃[やちん]が 高[たか]い 代[か]わりに、 駅[えき]に 近[ちか]くてとても 便利[べんり]です。	
\\	私達の日本語の先生は厳しい代わりに学生の面倒見がいい。	
\\	私[わたし] 達[たち]の 日本語[にほんご]の 先生[せんせい]は 厳[きび]しい 代[か]わりに 学生[がくせい]の 面倒[めんどう] 見[み]がいい。	面倒見がいい= 
\\	大学の先生は給料が低い代わりに自由がある。	
\\	大学[だいがく]の 先生[せんせい]は 給料[きゅうりょう]が 低[ひく]い 代[か]わりに 自由[じゆう]がある。	
\\	父は体が弱い代わりに意志がとても強い。	
\\	父[ちち]は 体[からだ]が 弱[よわ]い 代[か]わりに 意志[いし]がとても 強[つよ]い。	
\\	この辺は静かな代わりに、店も遠くて不便です。	
\\	この 辺[あたり]は 静[しず]かな 代[か]わりに、 店[みせ]も 遠[とお]くて 不便[ふべん]です。	
\\	日本語の授業にいつもの山田先生の代わりに田中という新しい先生がいらっしゃった。	
\\	日本語[にほんご]の 授業[じゅぎょう]にいつもの 山田[やまだ] 先生[せんせい]の 代[か]わりに 田中[たなか]という 新[あたら]しい 先生[せんせい]がいらっしゃった。	
\\	昼間遊んだ代わりに夜勉強した。	
\\	昼間[ひるま] 遊[あそ]んだ 代[か]わりに 夜[よる] 勉強[べんきょう]した。	
\\	昼間遊んだ代わりに夜勉強するつもりだ。	
\\	昼間[ひるま] 遊[あそ]んだ 代[か]わりに 夜[よる] 勉強[べんきょう]するつもりだ。	
\\	昼間遊ぶ代わりに夜遊んだ。	
\\	昼間[ひるま] 遊[あそ]ぶ 代[か]わりに 夜[よる] 遊[あそ]んだ。	
\\	この運動を推し進めている本当の力は金ではない。	
\\	この 運動[うんどう]を 推し進[おしすす]めている 本当[ほんとう]の 力[ちから]は 金[きん]ではない。	推し進める= 
\\	そのワンピースを着ると、彼女は洗練されて見える。	
\\	そのワンピースを 着[き]ると、 彼女[かのじょ]は 洗練[せんれん]されて 見[み]える。	洗練=せんれん= 
\\	ワンピース= 
\\	その著者は独特な洗練された文体で知られる。	
\\	その 著者[ちょしゃ]は 独特[どくとく]な 洗練[せんれん]された 文体[ぶんたい]で 知[し]られる。	洗練=せんれん= 
\\	知識人たちは洗練された会話を交わした。	
\\	知識[ちしき] 人[じん]たちは 洗練[せんれん]された 会話[かいわ]を 交[か]わした。	知識人= 
\\	洗練=せんれん= 
\\	交わす= 
\\	この薬のせいで胃がむかつくかもしれない。	
\\	この 薬[くすり]のせいで 胃[い]がむかつくかもしれない。	むかつく= 
\\	それがむかつくところだ。	
\\	それがむかつくところだ。	むかつく= 
\\	一番むかつくのは彼女の態度です。	
\\	一番[いちばん]むかつくのは 彼女[かのじょ]の 態度[たいど]です。	むかつく= 
\\	彼は息子の消息に関する報告を待っている。	
\\	彼[かれ]は 息子[むすこ]の 消息[しょうそく]に 関[かん]する 報告[ほうこく]を 待[ま]っている。	消息=しょうそく= 
\\	誰か彼の消息を知っていますか?	
\\	誰[だれ]か 彼[かれ]の 消息[しょうそく]を 知[し]っていますか?	消息=しょうそく= 
\\	志願者の中にはトレーニングの途中で脱落する者もいる。	
\\	志願[しがん] 者[しゃ]の 中[なか]にはトレーニングの 途中[とちゅう]で 脱落[だつらく]する 者[もの]もいる。	志願=しがん= 
\\	志願者= 
\\	志願者全員の選考が慎重に行われました。	
\\	志願[しがん] 者[しゃ] 全員[ぜんいん]の 選考[せんこう]が 慎重[しんちょう]に 行[おこな]われました。	志願=しがん= 
\\	志願者= 
\\	慎重=しんちょう= 
\\	彼らはその志願者を詳しく面接しました。	
\\	彼[かれ]らはその 志願[しがん] 者[しゃ]を 詳[くわ]しく 面接[めんせつ]しました。	
\\	どんなに追いつこうと頑張っても、勉強が後れてしまう。	
\\	どんなに 追[お]いつこうと 頑張[がんば]っても、 勉強[べんきょう]が 後[おく]れてしまう。	追いつく= 
\\	その企業は、競争で後れを取らないために工場を近代化しました。	
\\	その 企業[きぎょう]は、 競争[きょうそう]で 後[おく]れを 取[と]らないために 工場[こうじょう]を 近代[きんだい] 化[か]しました。	後れを取る= 
\\	近代化= 
\\	ヨーロッパは科学技術に関してアメリカに約3年後れを取っている。	
\\	ヨーロッパは 科学[かがく] 技術[ぎじゅつ]に 関[かん]してアメリカに 約[やく]3 年[ねん] 後[おく]れを 取[と]っている。	後れを取る= 
\\	親は、子供が学習曲線から後れを取ると心配します。	
\\	親[おや]は、 子供[こども]が 学習[がくしゅう] 曲線[きょくせん]から 後[おく]れを 取[と]ると 心配[しんぱい]します。	曲線=きょくせん= 
\\	後れを取る= 
\\	ご愁傷のこととお察し致します。	
\\	ご 愁傷[しゅうしょう]のこととお 察[さっ]し 致[いた]します。	愁傷=しゅうしょう= 
\\	察す=さっす= 
\\	彼女は彼の優しさから、彼が自分のことを好きだと察した。	
\\	彼女[かのじょ]は 彼[かれ]の 優[やさ]しさから、 彼[かれ]が 自分[じぶん]のことを 好[す]きだと 察[さっ]した。	察す=さっす= 
\\	この情報は極秘です。	
\\	この 情報[じょうほう]は 極秘[ごくひ]です。	極秘=ごくひ= 
\\	くれぐれも機密情報を漏らさないようにして下さい。	
\\	くれぐれも 機密[きみつ] 情報[じょうほう]を 漏[も]らさないようにして 下[くだ]さい。	くれぐれ= 
\\	機密情報= 
\\	あなたの個人情報は常に極秘に保管されます。	
\\	あなたの 個人[こじん] 情報[じょうほう]は 常[つね]に 極秘[ごくひ]に 保管[ほかん]されます。	極秘=ごくひ= 
\\	彼を、会社のベストスーパーバイザー賞に水洗しよう。	
\\	彼[かれ]を、 会社[かいしゃ]のベストスーパーバイザー 賞[しょう]に 水洗[すいせん]しよう。	
\\	彼は引退後に自分の職を引き継ぐ者を指名しなければなりませんでした。	
\\	彼[かれ]は 引退[いんたい] 後[ご]に 自分[じぶん]の 職[しょく]を 引き継[ひきつ]ぐ 者[もの]を 指名[しめい]しなければなりませんでした。	引き継ぐ= 
\\	指名=しめい= 
\\	その映画は、今年のアカデミー賞の多くの部門で候補に挙がった。	
\\	その 映画[えいが]は、 今年[ことし]のアカデミー 賞[しょう]の 多[おお]くの 部門[ぶもん]で 候補[こうほ]に 挙[あ]がった。	部門=ぶもん= 
\\	候補=こうほ= 
\\	洞察力を養うことは英知を育てるための一つの方法である。	
\\	洞察[どうさつ] 力[りょく]を 養[やしな]うことは 英知[えいち]を 育[そだ]てるための 一[ひと]つの 方法[ほうほう]である。	洞察=どうさつ= 
\\	英知=えいち= 
\\	そのジャーナリストは、軍事問題についての素晴らしい洞察で有名です。	
\\	そのジャーナリストは、 軍事[ぐんじ] 問題[もんだい]についての 素晴[すば]らしい 洞察[どうさつ]で 有名[ゆうめい]です。	軍事=ぐんじ= 
\\	洞察=どうさつ= 
\\	我々の社長は、洞察力と強い意志がある人です。	
\\	我々[われわれ]の 社長[しゃちょう]は、 洞察[どうさつ] 力[りょく]と 強[つよ]い 意志[いし]がある 人[ひと]です。	洞察=どうさつ= 
\\	業界の重役たちは否定的な報道に留意している。	
\\	業界[ぎょうかい]の 重役[じゅうやく]たちは 否定[ひてい] 的[てき]な 報道[ほうどう]に 留意[りゅうい]している。	重役=じゅうやく= 
\\	報道=ほうどう= 
\\	あなたの言ったことは、心に留めておくべきとても大切なことだと思う。	
\\	あなたの 言[い]ったことは、 心[こころ]に 留[と]めておくべきとても 大切[たいせつ]なことだと 思[おも]う。	心に留める= 
\\	その歌手は、批判する人たちを気に留めていないようだ。	
\\	その 歌手[かしゅ]は、 批判[ひはん]する 人[ひと]たちを 気[き]に 留[と]めていないようだ。	気に留める= 
\\	どうして我々がここにいるのか心に留めておきましょう。	
\\	どうして 我々[われわれ]がここにいるのか 心[こころ]に 留[と]めておきましょう。	心に留める= 
\\	トミー、シャツのボタンを留めなさい。	
\\	トミー、シャツのボタンを 留[と]めなさい。	
\\	あなたを引き留めさせないで下さい。	
\\	あなたを 引き留[ひきと]めさせないで 下[くだ]さい。	引き留める= 
\\	いつになったら仕事に取り掛かるつもりですか?	
\\	いつになったら 仕事[しごと]に 取り掛[とりか]かるつもりですか?	
\\	注文した商品について、お電話させていただきました。	
\\	注文[ちゅうもん]した 商品[しょうひん]について、お 電話[でんわ]させていただきました。	
\\	新聞の求人広告のことでお電話しているのですが。	
\\	新聞[しんぶん]の 求人[きゅうじん] 広告[こうこく]のことでお 電話[でんわ]しているのですが。	求人広告= 
\\	列車の発着時刻についてお聞きしたくて電話しています。	
\\	列車[れっしゃ]の 発着[はっちゃく] 時刻[じこく]についてお 聞[き]きしたくて 電話[でんわ]しています。	発着=はっちゃく= 
\\	彼女のダンスは素晴らしいが、名声を博することはできないだろう。	
\\	彼女[かのじょ]のダンスは 素晴[すば]らしいが、 名声[めいせい]を 博[はく]することはできないだろう。	名声= 
\\	博する= 
\\	その本は大変な好評を博しました。	
\\	その 本[ほん]は 大変[たいへん]な 好評[こうひょう]を 博[はく]しました。	好評= 
\\	博する= 
\\	アメリカで好評を博した者は、しばらくすると日本にやって来る。	
\\	アメリカで 好評[こうひょう]を 博[はく]した 者[もの]は、しばらくすると日本にやって 来[く]る。	好評= 
\\	博する= 
\\	やってくる= 
\\	セーターを編んであげるね。何色がいい?	
\\	セーターを 編[あ]んであげるね。 何色[なんいろ]がいい?	編む= 
\\	彼女は私の誕生日のためにセーターを編んでくれた。	
\\	彼女[かのじょ]は 私[わたし]の 誕生[たんじょう] 日[び]のためにセーターを 編[あ]んでくれた。	編む= 
\\	彼女は髪の毛を編んでもらった。	
\\	彼女[かのじょ]は 髪の毛[かみのけ]を 編[あ]んでもらった。	編む= 
\\	そのレストランを借りて内輪だけのパーティーをやった。	
\\	そのレストランを 借[か]りて 内輪[うちわ]だけのパーティーをやった。	内輪= 
\\	我が国が軽率に破滅的な戦争に突入するのを防がなければならない。	
\\	我が国[わがくに]が 軽率[けいそつ]に 破滅[はめつ] 的[てき]な 戦争[せんそう]に 突入[とつにゅう]するのを 防[ふせ]がなければならない。	破滅=はめつ= 
\\	突入=とつにゅう= 
\\	レースは大接戦だった、ずっと互角だった。	
\\	レースは 大[だい] 接戦[せっせん]だった、ずっと 互角[ごかく]だった。	接戦=せっせん= 
\\	互角=ごかく= 
\\	父と息子は背丈が互角です。	
\\	父[ちち]と 息子[むすこ]は 背丈[せたけ]が 互角[ごかく]です。	背丈=せたけ= 
\\	互角=ごかく= 
\\	彼女は彼の紳士らしい体裁にだまされてしまった。	
\\	彼女[かのじょ]は 彼[かれ]の 紳士[しんし]らしい 体裁[ていさい]にだまされてしまった。	体裁= 
\\	だます= 
\\	友達に電話したけれど、あいにく、家にいなかった。	
\\	友達[ともだち]に 電話[でんわ]したけれど、あいにく、 家[いえ]にいなかった。	
\\	投票の結果、その提案は反対多数で否決された。	
\\	投票[とうひょう]の 結果[けっか]、その 提案[ていあん]は 反対[はんたい] 多数[たすう]で 否決[ひけつ]された。	
\\	妻と相談した結果、家を買うことにした。	
\\	妻[つま]と 相談[そうだん]した 結果[けっか]、 家[いえ]を 買[か]うことにした。	
\\	相談の結果、今回の旅行は延期することになった。	
\\	相談[そうだん]の 結果[けっか]、 今回[こんかい]の 旅行[りょこう]は 延期[えんき]することになった。	
\\	検査の結果、妻の体はどこにも異状がないことが分かった。	
\\	検査[けんさ]の 結果[けっか]、 妻[つま]の 体[からだ]はどこにも 異状[いじょう]がないことが 分[わ]かった。	
\\	調査の結果、新しい事実が発見された。	
\\	調査[ちょうさ]の 結果[けっか]、 新[あたら]しい 事実[じじつ]が 発見[はっけん]された。	
\\	警察で調べた結果、原因はたばこの火の不始末と分かった。	
\\	警察[けいさつ]で 調[しら]べた 結果[けっか]、 原因[げんいん]はたばこの 火[ひ]の 不[ふ] 始末[しまつ]と 分[わ]かった。	
\\	特別のダイエットをした結果、十キロの減量に成功した。	
\\	特別[とくべつ]のダイエットをした 結果[けっか]、 十[じっ]キロの 減量[げんりょう]に 成功[せいこう]した。	
\\	新しい教科書を使った結果、学生の成績が著しく伸びた。	
\\	新[あたら]しい 教科書[きょうかしょ]を 使[つか]った 結果[けっか]、 学生[がくせい]の 成績[せいせき]が 著[いちじる]しく 伸[の]びた。	著しい=いちじるしい= (はっきりしている) 
\\	(程度が大きい) 
\\	ゴルフの個人指導を受けた結果、自分の問題点が明らかになった。	
\\	ゴルフの 個人[こじん] 指導[しどう]を 受[う]けた 結果[けっか]、 自分[じぶん]の 問題[もんだい] 点[てん]が 明[あき]らかになった。	
\\	試験の結果は来週知らせます。	
\\	試験[しけん]の 結果[けっか]は 来週[らいしゅう] 知[し]らせます。	
\\	ここの寿司は、結構おいしいね。	
\\	ここの 寿司[すし]は、 結構[けっこう]おいしいね。	
\\	この車は古いんだけど、結構よく走りますよ。	
\\	この 車[くるま]は 古[ふる]いんだけど、 結構[けっこう]よく 走[はし]りますよ。	
\\	あの人は間抜けに見えるでしょう。でも、結構頭がいいんです。	
\\	あの 人[ひと]は 間抜[まぬ]けに 見[み]えるでしょう。でも、 結構[けっこう] 頭[あたま]がいいんです。	間抜け= 
\\	四月だというのに、結構寒いね。	
\\	四月[しがつ]だというのに、 結構[けっこう] 寒[さむ]いね。	
\\	この料理は量は少ないけど、結構胃にもたれるね。	
\\	この 料理[りょうり]は 量[りょう]は 少[すく]ないけど、 結構[けっこう] 胃[い]にもたれるね。	もたれる= 
\\	今日は日曜日なのに高速が結構混んでいるね。	
\\	今日[きょう]は 日曜日[にちようび]なのに 高速[こうそく]が 結構[けっこう] 混[こ]んでいるね。	
\\	父は楽天的な人でしたが、失職した時には結構悩んだようです。	
\\	父[ちち]は 楽天的[らくてんてき]な 人[ひと]でしたが、 失職[しっしょく]した 時[とき]には 結構[けっこう] 悩[なや]んだようです。	
\\	案外あの二人は結婚するかもしれないよ。	
\\	案外[あんがい]あの 二人[ふたり]は 結婚[けっこん]するかもしれないよ。	
\\	案外、彼の方が先に着いているかもしれませんよ。	
\\	案外[あんがい]、 彼[かれ]の 方[ほう]が 先[さき]に 着[つ]いているかもしれませんよ。	
\\	ここの寿司は、割合とおいしいね。	
\\	ここの 寿司[すし]は、 割合[わりあい]とおいしいね。	割合= 
\\	秋山さんはこの春結婚します。	
\\	秋山[あきやま]さんはこの 春[はる] 結婚[けっこん]します。	
\\	山下君はこの一週間授業を休んでいます。	
\\	山下[やました] 君[くん]はこの 一週間[いっしゅうかん] 授業[じゅぎょう]を 休[やす]んでいます。	
\\	このたびこの会の会員に加えていただきました。	
\\	このたびこの 会[かい]の 会員[かいいん]に 加[くわ]えていただきました。	
\\	この辺で妥協したらどうですか。	
\\	この 辺[へん]で 妥協[だきょう]したらどうですか。	
\\	この夏は日本の女流作家の研究をしています。	
\\	この 夏[なつ]は日本の 女流[じょりゅう] 作家[さっか]の 研究[けんきゅう]をしています。	
\\	この次はいつお目にかかれますか。	
\\	この 次[つぎ]はいつお 目[め]にかかれますか。	
\\	こうした問題はこの国では聞かれないようである。	
\\	こうした 問題[もんだい]はこの 国[くに]では 聞[き]かれないようである。	
\\	こうした行為がどのような結果を招くかは誰の目にも明らかだ。	
\\	こうした 行為[こうい]がどのような 結果[けっか]を 招[まね]くかは 誰[だれ]の 目[め]にも 明[あき]らかだ。	
\\	こうした経験は日本へ行ったことのある者なら誰にでもあるはずだ。	
\\	こうした 経験[けいけん]は 日本[にっぽん]へ 行[い]ったことのある 者[もの]なら 誰[だれ]にでもあるはずだ。	
\\	私はこうした話には耳を貸さないことにしている。	
\\	私[わたし]はこうした 話[はなし]には 耳[みみ]を 貸[か]さないことにしている。	
\\	小高い山がその村を囲んでいる。	
\\	小高[こだか]い 山[やま]がその 村[むら]を 囲[かこ]んでいる。	小高い= 
\\	囲む= 
\\	車の周りを人が取り囲んだ。	
\\	車[くるま]の 周[まわ]りを 人[ひと]が 取り囲[とりかこ]んだ。	
\\	正解の番号を丸で囲みなさい。	
\\	正解[せいかい]の 番号[ばんごう]を 丸[まる]で 囲[かこ]みなさい。	正解= 
\\	囲む= 
\\	暴徒はその建物を囲み、石を投げ始めた。	
\\	暴徒[ぼうと]はその 建物[たてもの]を 囲[かこ]み、 石[いし]を 投[な]げ 始[はじ]めた。	暴徒=ぼうと= 
\\	囲む= 
\\	いくつかの前菜に加えて、スープも出されました。	
\\	いくつかの 前菜[ぜんさい]に 加[くわ]えて、スープも 出[だ]されました。	前菜=ぜんさい= 
\\	その先生は、自分の体験談を講義に加えた。	
\\	その 先生[せんせい]は、 自分[じぶん]の 体験[たいけん] 談[だん]を 講義[こうぎ]に 加[くわ]えた。	体験談=たいけんだん= 
\\	あなたは、なかなか度胸がありますね。	
\\	あなたは、なかなか 度胸[どきょう]がありますね。	度胸=どきょう= 
\\	なかなかいい度胸をしていますね。	
\\	なかなかいい 度胸[どきょう]をしていますね。	度胸=どきょう= 
\\	泥沼にならないうちに脱出して下さい。	
\\	泥沼[どろぬま]にならないうちに 脱出[だっしゅつ]して 下[くだ]さい。	泥沼=どろぬま= 
\\	脱出=だっしゅつ= 
\\	当社の主眼は日本ではなく、中国で車を売ることだ。	
\\	当社[とうしゃ]の 主眼[しゅがん]は 日本[にっぽん]ではなく、 中国[ちゅうごく]で 車[くるま]を 売[う]ることだ。	主眼=しゅがん= 
\\	彼のことは眼中にない。	
\\	彼[かれ]のことは 眼中[がんちゅう]にない。	眼中にない= 
\\	彼のことはまるで眼中になかった。	
\\	彼[かれ]のことはまるで 眼中[がんちゅう]になかった。	眼中にない= 
\\	この新薬は効果が永続的である。	
\\	この 新薬[しんやく]は 効果[こうか]が 永続[えいぞく] 的[てき]である。	
\\	その国は米国との外交関係を断絶した。	
\\	その 国[くに]は 米国[べいこく]との 外交[がいこう] 関係[かんけい]を 断絶[だんぜつ]した。	外交= 
\\	外交関係= 
\\	彼のユーモアセンスは抜群です。	
\\	彼[かれ]のユーモアセンスは 抜群[ばつぐん]です。	
\\	彼は運動神経が抜群です。	
\\	彼[かれ]は 運動[うんどう] 神経[しんけい]が 抜群[ばつぐん]です。	
\\	その試合に勝つために彼が考えた戦略は抜群だった。	
\\	その 試合[しあい]に 勝[か]つために 彼[かれ]が 考[かんが]えた 戦略[せんりゃく]は 抜群[ばつぐん]だった。	戦略= 
\\	いつ仕事の依頼が途絶えるか分からない。	
\\	いつ 仕事[しごと]の 依頼[いらい]が 途絶[とだ]えるか 分[わ]からない。	
\\	中国政府は1989年6月4日、デモ参加者を弾圧しました。	
\\	中国[ちゅうごく] 政府[せいふ]は1989 年[ねん] 6月[ろくがつ] 4日[よっか]、デモ 参加[さんか] 者[しゃ]を 弾圧[だんあつ]しました。	弾圧= 
\\	設計上の欠陥が一連の爆発につながりました。	
\\	設計[せっけい] 上[じょう]の 欠陥[けっかん]が 一連[いちれん]の 爆発[ばくはつ]につながりました。	欠陥= 
\\	一連= 
\\	企業が独占を長期間享受することは困難です。	
\\	企業[きぎょう]が 独占[どくせん]を 長期間[ちょうきかん] 享受[きょうじゅ]することは 困難[こんなん]です。	享受= 
\\	喫煙者はたばこをやめることで利益を享受できる。	
\\	喫煙[きつえん] 者[しゃ]はたばこをやめることで 利益[りえき]を 享受[きょうじゅ]できる。	享受= 
\\	疑う根拠はどこにもない。	
\\	疑[うたが]う 根拠[こんきょ]はどこにもない。	疑う= 
\\	我が耳を疑う話だ。	
\\	我[わ]が 耳[みみ]を 疑[うたが]う 話[はなし]だ。	わが= 
\\	耳を疑う= 
\\	主人の決断を疑うのは彼の職分ではない。	
\\	主人[しゅじん]の 決断[けつだん]を 疑[うたが]うのは 彼[かれ]の 職分[しょくぶん]ではない。	疑う= 
\\	それを一度も疑ったことがない。	
\\	それを一 度[ど]も 疑[うたが]ったことがない。	疑う= 
\\	いまだに疑っているんだ。	
\\	いまだに 疑[うたが]っているんだ。	疑う= 
\\	キャメロンは妻と上司が関係を持っているのではないかと疑った。	
\\	キャメロンは 妻[つま]と 上司[じょうし]が 関係[かんけい]を 持[も]っているのではないかと 疑[うたが]った。	関係を持つ= 
\\	疑う= 
\\	彼女はその話が本当かどうか疑っていた。	
\\	彼女[かのじょ]はその 話[はなし]が 本当[ほんとう]かどうか 疑[うたが]っていた。	疑う= 
\\	警察はその目撃者の話を疑っている。	
\\	警察[けいさつ]はその 目撃[もくげき] 者[しゃ]の 話[はなし]を 疑[うたが]っている。	目撃者=もくげきしゃ= 
\\	疑う= 
\\	この建物は、電気と水の消費を節約できるように設計されている。	
\\	この 建物[たてもの]は、 電気[でんき]と 水[みず]の 消費[しょうひ]を 節約[せつやく]できるように 設計[せっけい]されている。	設計= 
\\	その橋は日本の技術を使って設計された。	
\\	その 橋[はし]は 日本[にっぽん]の 技術[ぎじゅつ]を 使[つか]って 設計[せっけい]された。	設計= 
\\	気まぐれと変わっているというのは違うものです。	
\\	気[き]まぐれと 変[か]わっているというのは 違[ちが]うものです。	気まぐれ= 
\\	気まぐれなダイエットはめったにうまくいかない。	
\\	気[き]まぐれなダイエットはめったにうまくいかない。	気まぐれ= 
\\	イギリスの天気、特に春はあまりにも気まぐれだ。	
\\	イギリスの 天気[てんき]、 特[とく]に 春[はる]はあまりにも 気[き]まぐれだ。	気まぐれ= 
\\	彼が彼女の気まぐれに我慢しているのは、彼女を愛しているからだ。	
\\	彼[かれ]が 彼女[かのじょ]の 気[き]まぐれに 我慢[がまん]しているのは、 彼女[かのじょ]を 愛[あい]しているからだ。	気まぐれ= 
\\	運命の女神は気まぐれです。	
\\	運命[うんめい]の 女神[めがみ]は 気[き]まぐれです。	女神=めがみ= 
\\	気まぐれ= 
\\	あなたはどうやってテレビから映画への進出を果たしたのですか?	
\\	あなたはどうやってテレビから 映画[えいが]への 進出[しんしゅつ]を 果[は]たしたのですか?	進出= 
\\	果たす= 
\\	この委員会は本来の機能を果たしていない。	
\\	この 委員[いいん] 会[かい]は 本来[ほんらい]の 機能[きのう]を 果[は]たしていない。	委員会= 
\\	果たす= 
\\	ご自分の目的はすべて果たされたと思われますか?	
\\	ご 自分[じぶん]の 目的[もくてき]はすべて 果[は]たされたと 思[おも]われますか?	果たす= 
\\	ジャーナリストが果たすべき役割の一つは、言論の自由を守ることだ。	
\\	ジャーナリストが 果[は]たすべき 役割[やくわり]の 一[ひと]つは、 言論[げんろん]の 自由[じゆう]を 守[まも]ることだ。	果たす= 
\\	言論の自由= 
\\	トムはたとえ病気でも、会社では自分の職務をきちんと果たしている。	
\\	トムはたとえ 病気[びょうき]でも、 会社[かいしゃ]では 自分[じぶん]の 職務[しょくむ]をきちんと 果[は]たしている。	職務= 
\\	果たす= 
\\	パリに着いた最初の週に、私はお金を使い果たした。	
\\	パリに 着[つ]いた 最初[さいしょ]の 週[しゅう]に、 私[わたし]はお 金[かね]を 使い果[つかいは]たした。	使い果たす= 
\\	僕は責任を果たすよ。	
\\	僕[ぼく]は 責任[せきにん]を 果[は]たすよ。	果たす= 
\\	誰か幹事をやってくれる人はいますか?	
\\	誰[だれ]か 幹事[かんじ]をやってくれる 人[ひと]はいますか?	幹事=かんじ= 
\\	私、幹事をやるのが初めてだから要領が悪いのかもしれません。	
\\	私[わたし]、 幹事[かんじ]をやるのが 初[はじ]めてだから 要領[ようりょう]が 悪[わる]いのかもしれません。	幹事=かんじ= 
\\	要領= 
\\	これを補充していただけますか?	
\\	これを 補充[ほじゅう]していただけますか?	補充= 
\\	補償は、米ドルで支払われるだろう。	
\\	補償[ほしょう]は、 米ドル[あめりかどる]で 支払[しはら]われるだろう。	補償= 
\\	不足を借金で補充した。	
\\	不足[ふそく]を 借金[しゃっきん]で 補充[ほじゅう]した。	補充= 
\\	あの研究への参加者は財政的に補償されるでしょう。	
\\	あの 研究[けんきゅう]への 参加[さんか] 者[しゃ]は 財政[ざいせい] 的[てき]に 補償[ほしょう]されるでしょう。	財政的= 
\\	補償= 
\\	その運転手は、けがに対する補償を要求しました。	
\\	その 運転[うんてん] 手[しゅ]は、けがに 対[たい]する 補償[ほしょう]を 要求[ようきゅう]しました。	補償= 
\\	彼女は事故の補償金として200万ドルを受け取った。	
\\	彼女[かのじょ]は 事故[じこ]の 補償[ほしょう] 金[きん]として200 万[まん]ドルを 受け取[うけと]った。	補償= 
\\	これこそ我々が探し求めていたものだ。	
\\	これこそ 我々[われわれ]が 探し求[さがしもと]めていたものだ。	
\\	どうもすみませんでした。 
\\	いいえ、こちらこそすみませんでした。	
\\	どうもすみませんでした。 
\\	いいえ、こちらこそすみませんでした。	
\\	一人でやってこそ勉強になるのだ。	
\\	一人[ひとり]でやってこそ 勉強[べんきょう]になるのだ。	
\\	君が正直に話してくれたからこそ問題は最小で済んだんだ。	
\\	君[きみ]が 正直[しょうじき]に 話[はな]してくれたからこそ 問題[もんだい]は 最小[さいしょう]で 済[す]んだんだ。	
\\	この件には様々な要素が混じり合っている。	
\\	この 件[けん]には 様々[さまざま]な 要素[ようそ]が 混[ま]じり 合[あ]っている。	要素= 
\\	これはわが国の国力の極めて重要な要素です。	
\\	これはわが 国[くに]の 国力[こくりょく]の 極[きわ]めて 重要[じゅうよう]な 要素[ようそ]です。	要素= 
\\	チームワークは、ビジネスの大切な要素です。	
\\	チームワークは、ビジネスの 大切[たいせつ]な 要素[ようそ]です。	要素= 
\\	上司は、売上高に難色を示した。	
\\	上司[じょうし]は、 売上[うりあげ] 高[だか]に 難色[なんしょく]を 示[しめ]した。	
\\	この山は神聖な場所として先住民の人々によってあがめられている。	
\\	この 山[やま]は 神聖[しんせい]な 場所[ばしょ]として 先住民[せんじゅうみん]の 人々[ひとびと]によってあがめられている。	神聖= 
\\	崇める=あがめる= 
\\	この神聖な神社に入る前には、手を洗わなければならない。	
\\	この 神聖[しんせい]な 神社[じんじゃ]に 入[はい]る 前[まえ]には、 手[て]を 洗[あら]わなければならない。	神聖= 
\\	彼らは山々に対し神聖な敬意を持っていた	
\\	彼[かれ]らは 山々[やまやま]に 対[たい]し 神聖[しんせい]な 敬意[けいい]を 持[も]っていた	神聖= 
\\	敬意= 
\\	彼は生まれつき温和な気質だ。	
\\	彼[かれ]は 生[う]まれつき 温和[おんわ]な 気質[きしつ]だ。	生まれつき= 
\\	一般的に言えば、日本の気候は温和だ。	
\\	一般[いっぱん] 的[てき]に 言[い]えば、日本の 気候[きこう]は 温和[おんわ]だ。	
\\	その戦争犯罪人たちを裁くため、法廷が設置された。	
\\	その 戦争[せんそう] 犯[はん] 罪人[ざいにん]たちを 裁[さば]くため、 法廷[ほうてい]が 設置[せっち]された。	戦争犯罪人= 
\\	最近私たちは最高裁から満場一致の判決を勝ち取りました。	
\\	最近[さいきん] 私[わたし]たちは 最高裁[さいこうさい]から 満場一致[まんじょういっち]の 判決[はんけつ]を 勝ち取[かちと]りました。	最高裁= 
\\	満場一致=まんじょういっち= 
\\	判決= 
\\	勝ち取る= 
\\	冷静になろうよ。	
\\	冷静[れいせい]になろうよ。	冷静= 
\\	冷静を保つことはとても重要です。	
\\	冷静[れいせい]を 保[たも]つことはとても 重要[じゅうよう]です。	冷静= 
\\	不満な気持ちは分かるけれど、ここは冷静に考えましょう。	
\\	不満[ふまん]な 気持[きも]ちは 分[わ]かるけれど、ここは 冷静[れいせい]に 考[かんが]えましょう。	冷静= 
\\	事態が緊迫している時は、冷静な態度を保つことは難しいです。	
\\	事態[じたい]が 緊迫[きんぱく]している 時[とき]は、 冷静[れいせい]な 態度[たいど]を 保[たも]つことは 難[むずか]しいです。	緊迫=きんぱく= 
\\	冷静= 
\\	道順は分かる?	
\\	道順[みちじゅん]は 分[わ]かる?	道順=みちじゅん= 
\\	あなたの家までの道順を教えてくれない?	
\\	あなたの 家[いえ]までの 道順[みちじゅん]を 教[おし]えてくれない?	道順=みちじゅん= 
\\	ジェーンの家までの道順を覚えていますか?	
\\	ジェーンの 家[いえ]までの 道順[みちじゅん]を 覚[おぼ]えていますか?	道順=みちじゅん= 
\\	すべてが順調に進んでいます。	
\\	すべてが 順調[じゅんちょう]に 進[すす]んでいます。	順調= 
\\	とりあえず彼の返答を待つしかありません。	
\\	とりあえず 彼[かれ]の 返答[へんとう]を 待[ま]つしかありません。	
\\	私たちは顧客からの苦情にすぐに返答しなければなりません。	
\\	私[わたし]たちは 顧客[こきゃく]からの 苦情[くじょう]にすぐに 返答[へんとう]しなければなりません。	顧客= 
\\	苦情= 
\\	正式な数字はまだ発表されてない。	
\\	正式[せいしき]な 数字[すうじ]はまだ 発表[はっぴょう]されてない。	正式= 
\\	正式には何も述べられていない。	
\\	正式[せいしき]には 何[なに]も 述[の]べられていない。	正式= 
\\	ボブとメグは正式に付き合っています。	
\\	ボブとメグは 正式[せいしき]に 付き合[つきあ]っています。	正式= 
\\	合衆国は、1776年に正式に独立しました。	
\\	合衆国[がっしゅうこく]は、 
\\	年[ねん]に 正式[せいしき]に 独立[どくりつ]しました。	正式= 
\\	人間は、様々な多くの状況に順応できます。	
\\	人間[にんげん]は、 様々[さまざま]な 多[おお]くの 状況[じょうきょう]に 順応[じゅんのう]できます。	順応= 
\\	子どもは新しい物事に大人よりも早く順応する傾向がある。	
\\	子[こ]どもは 新[あたら]しい 物事[ものごと]に 大人[おとな]よりも 早[はや]く 順応[じゅんのう]する 傾向[けいこう]がある。	物事= 
\\	順応= 
\\	傾向= 
\\	彼女は日本での新しい生活にすぐに順応しました。	
\\	彼女[かのじょ]は 日本[にっぽん]での 新[あたら]しい 生活[せいかつ]にすぐに 順応[じゅんのう]しました。	順応= 
\\	私たちの体は、数分後には、冷たい水に順応しました。	
\\	私[わたし]たちの 体[からだ]は、 数[すう] 分[ふん] 後[ご]には、 冷[つめ]たい 水[みず]に 順応[じゅんのう]しました。	数分後=すうふんご= 
\\	順応= 
\\	ついさっき彼と会いました。	
\\	ついさっき 彼[かれ]と 会[あ]いました。	つい= 
\\	つい先日、正式に離婚しました。	
\\	つい 先日[せんじつ]、 正式[せいしき]に 離婚[りこん]しました。	つい= 
\\	その映画には日本語の字幕がついています。	
\\	その 映画[えいが]には 日本語[にほんご]の 字幕[じまく]がついています。	字幕=じまく= 
\\	君の命を彼女に委ねていいのか。	
\\	君[きみ]の 命[いのち]を 彼女[かのじょ]に 委[ゆだ]ねていいのか。	委ねる= 
\\	彼女は子どもを叔母の世話に委ねた。	
\\	彼女[かのじょ]は 子[こ]どもを 叔母[おば]の 世話[せわ]に 委[ゆだ]ねた。	叔母=おば= 
\\	委ねる= 
\\	あなたには自分の才能を発揮できる機会がありました。	
\\	あなたには 自分[じぶん]の 才能[さいのう]を 発揮[はっき]できる 機会[きかい]がありました。	発揮= 
\\	独創性を発揮する余地があまりない。	
\\	独創[どくそう] 性[せい]を 発揮[はっき]する 余地[よち]があまりない。	独創性= 
\\	発揮= 
\\	私たちは十二分にこの件を処理できる。	
\\	私[わたし]たちは 十二分[じゅうにぶん]にこの 件[けん]を 処理[しょり]できる。	十二分= 
\\	私はようやく本領を発揮し始めています。	
\\	私[わたし]はようやく 本領[ほんりょう]を 発揮[はっき]し 始[はじ]めています。	本領を発揮= 
\\	勝ちたいという欲望を持っている選手は頭角を現すだろう。	
\\	勝[か]ちたいという 欲望[よくぼう]を 持[も]っている 選手[せんしゅ]は 頭角[とうかく]を 現[あらわ]すだろう。	欲望=よくぼう= 
\\	頭角を現す= 
\\	責めを負うべきなのはあなたです。	
\\	責[せ]めを 負[お]うべきなのはあなたです。	責めを負う= 
\\	その事故のことで彼を責めちゃいけないよ。	
\\	その 事故[じこ]のことで 彼[かれ]を 責[せ]めちゃいけないよ。	責める= 
\\	彼らは責任を認めようとしないことで責められた。	
\\	彼[かれ]らは 責任[せきにん]を 認[みと]めようとしないことで 責[せ]められた。	責める= 
\\	私はひそかに自分を責めた。	
\\	私[わたし]はひそかに 自分[じぶん]を 責[せ]めた。	
\\	自分を責めても役には立たない。	
\\	自分[じぶん]を 責[せ]めても 役[やく]には 立[た]たない。	
\\	その試練は、夫と子どもにとって非常につらいものだった。	
\\	その 試練[しれん]は、 夫[おっと]と 子[こ]どもにとって 非常[ひじょう]につらいものだった。	試練= 
\\	乗り越えられない試練はない。	
\\	乗り越[のりこ]えられない 試練[しれん]はない。	試練= 
\\	全国民にとって試練の時であった。	
\\	全[ぜん] 国民[こくみん]にとって 試練[しれん]の 時[とき]であった。	試練= 
\\	彼女はいくつかの指針を採用しました。	
\\	彼女[かのじょ]はいくつかの 指針[ししん]を 採用[さいよう]しました。	指針= 
\\	こんな時こそ全員で力を合わせて問題を解決しなければならない。	
\\	こんな 時[とき]こそ 全員[ぜんいん]で 力[ちから]を 合[あ]わせて 問題[もんだい]を 解決[かいけつ]しなければならない。	
\\	今年こそこの試験に通ってみせる。	
\\	今年[ことし]こそこの 試験[しけん]に 通[かよ]ってみせる。	
\\	それでこそ我々のリーダーだ。	
\\	それでこそ 我々[われわれ]のリーダーだ。	
\\	親友だからこそこんなことまで君に言うんだよ。	
\\	親友[しんゆう]だからこそこんなことまで 君[きみ]に 言[い]うんだよ。	
\\	出来ないからこそ人より余計に練習しなければならないのだ。	
\\	出来[でき]ないからこそ 人[ひと]より 余計[よけい]に 練習[れんしゅう]しなければならないのだ。	余計= 
\\	発表は十五分以内で行うこと。	
\\	発表[はっぴょう]は 十五分[じゅうごふん] 以内[いない]で 行[おこな]うこと。	
\\	プールサイドを走らないこと。	
\\	プールサイドを 走[はし]らないこと。	
\\	詳細は二十三ページ参照のこと。	
\\	詳細[しょうさい]は二十三ページ 参照[さんしょう]のこと。	参照= 
\\	それは情けない仕事だね。	
\\	それは 情[なさ]けない 仕事[しごと]だね。	情けない= 
\\	そんなに情けない顔をするなよ。	
\\	そんなに 情[なさ]けない 顔[かお]をするなよ。	情けない= 
\\	自分自身を情けないと思わないで。	
\\	自分[じぶん] 自身[じしん]を 情[なさ]けないと 思[おも]わないで。	情けない= 
\\	その家は鉄道の駅まで徒歩5分です。	
\\	その 家[いえ]は 鉄道[てつどう]の 駅[えき]まで 徒歩[とほ]5 分[ふん]です。	鉄道= 
\\	徒歩= 
\\	彼は、徒歩で通学している。	
\\	彼[かれ]は、 徒歩[とほ]で 通学[つうがく]している。	徒歩= 
\\	どなたかお代わりはいかがですか。	
\\	どなたかお 代[か]わりはいかがですか。	お代わり= 
\\	どなたか=誰か
\\	どなたかご在宅ですか?	
\\	どなたかご 在宅[ざいたく]ですか?	
\\	どなたか、スマートフォンで使える良い和英辞書アプリを紹介してくださいませんか?	
\\	どなたか、スマートフォンで 使[つか]える 良[よ]い 和英[かずひで] 辞書[じしょ]アプリを 紹介[しょうかい]してくださいませんか?	
\\	案の定彼は遅れて来た。	
\\	案の定[あんのじょう] 彼[かれ]は 遅[おく]れて 来[き]た。	案の定= 
\\	そのチームはタイガーズに圧勝しました。	
\\	そのチームはタイガーズに 圧勝[あっしょう]しました。	圧勝=あっしょう= 
\\	彼は選挙でライバルの共産党候補に圧勝した。	
\\	彼[かれ]は 選挙[せんきょ]でライバルの 共産党[きょうさんとう] 候補[こうほ]に 圧勝[あっしょう]した。	共産党=きょうさんとう= 
\\	圧勝=あっしょう= 
\\	首相は辞任要求を拒絶しました。	
\\	首相[しゅしょう]は 辞任[じにん] 要求[ようきゅう]を 拒絶[きょぜつ]しました。	辞任=じにん= 
\\	拒絶=きょぜつ= 
\\	シャンパンを飲むと、乱れてしまう。	
\\	シャンパンを 飲[の]むと、 乱[みだ]れてしまう。	乱れる= 
\\	心が千々に乱れて仕事に集中できません。	
\\	心[こころ]が 千々[ちじ]に 乱[みだ]れて 仕事[しごと]に 集中[しゅうちゅう]できません。	千々に=ちぢに= 
\\	乱れる= 
\\	その会社は巨額の負債を抱えて倒産しました。	
\\	その 会社[かいしゃ]は 巨額[きょがく]の 負債[ふさい]を 抱[かか]えて 倒産[とうさん]しました。	負債=ふさい= 
\\	抱える= 
\\	その国は深刻な食糧難を抱えている。	
\\	その 国[くに]は 深刻[しんこく]な 食糧難[しょくりょうなん]を 抱[かか]えている。	抱える= 
\\	その決定に頭を抱えた。	
\\	その 決定[けってい]に 頭[あたま]を 抱[かか]えた。	頭を抱える= 
\\	グーグルの検索によると、この問題を抱えているのは私だけではないようだ。	
\\	グーグルの 検索[けんさく]によると、この 問題[もんだい]を 抱[かか]えているのは 私[わたし]だけではないようだ。	抱える= 
\\	弱々しい老人がつえを突きながら道路をわたった。	
\\	弱々[よわよわ]しい 老人[ろうじん]がつえを 突[つ]きながら 道路[どうろ]をわたった。	弱々しい= 
\\	つえ= 
\\	彼は私の鼻に頭突きを食らわせた。	
\\	彼[かれ]は 私[わたし]の 鼻[はな]に 頭突[ずつ]きを 食[く]らわせた。	頭突き=ずつき= 
\\	食らわせる= 
\\	打ち明けなければならないことがあります。	
\\	打ち明[うちあ]けなければならないことがあります。	
\\	友達にも打ち明けられません。	
\\	友達[ともだち]にも 打ち明[うちあ]けられません。	打ち明ける= 
\\	あなたに打ち明けるように彼にうるさく言うべきではない。	
\\	あなたに 打ち明[うちあ]けるように 彼[かれ]にうるさく 言[い]うべきではない。	
\\	「うるさい」
\\	それら二本の直線は平行である。	
\\	それら二 本[ほん]の 直線[ちょくせん]は 平行[へいこう]である。	直線= 
\\	平行= 
\\	海と平行しているので、私はこの道が好きだ。	
\\	海[うみ]と 平行[へいこう]しているので、 私[わたし]はこの 道[みち]が 好[す]きだ。	平行= 
\\	土壌が汚染された後、その土地は不毛になった。	
\\	土壌[どじょう]が 汚染[おせん]された 後[のち]、その 土地[とち]は 不毛[ふもう]になった。	土壌=どじょう= 
\\	汚染= 
\\	不毛=ふもう= 
\\	農民たちは不毛の地を捨てて、他の地域へ移動しました。	
\\	農民[のうみん]たちは 不毛[ふもう]の 地[ち]を 捨[す]てて、 他[た]の 地域[ちいき]へ 移動[いどう]しました。	農民=のうみん= 
\\	不毛=ふもう= 
\\	私は彼の結婚のことで大騒ぎをする気は毛頭ない。	
\\	私[わたし]は 彼[かれ]の 結婚[けっこん]のことで 大騒[おおさわ]ぎをする 気[き]は 毛頭[もうとう]ない。	大騒ぎ=おおさわぎ= 
\\	毛頭ない= 
\\	私は引退するつもりは毛頭ない。	
\\	私[わたし]は 引退[いんたい]するつもりは 毛頭[もうとう]ない。	毛頭ない= 
\\	ある日は春らしい陽気なのに、次の日にはまた冬に逆戻りする。	
\\	ある 日[ひ]は 春[はる]らしい 陽気[ようき]なのに、 次[つぎ]の 日[ひ]にはまた 冬[ふゆ]に 逆戻[ぎゃくもど]りする。	陽気= 
\\	逆戻り=
\\	この陽気な音楽を聴くのは楽しい。	
\\	この 陽気[ようき]な 音楽[おんがく]を 聴[き]くのは 楽[たの]しい。	陽気= 
\\	子どもたちは、砂場で陽気に遊んでいた。	
\\	子[こ]どもたちは、 砂場[すなば]で 陽気[ようき]に 遊[あそ]んでいた。	砂場=すなば= 
\\	陽気= 
\\	彼女は、とても陽気な性格です。	
\\	彼女[かのじょ]は、とても 陽気[ようき]な 性格[せいかく]です。	陽気= 
\\	あからさまにそのミスに言及しました。	
\\	あからさまにそのミスに 言及[げんきゅう]しました。	あからさま= 
\\	そのインタビューで、作家は自身の本について言及しました。	
\\	そのインタビューで、 作家[さっか]は 自身[じしん]の 本[ほん]について 言及[げんきゅう]しました。	
\\	私は3年前に
\\	陽性であると診断された。	
\\	私は 3年[さんねん] 前[まえ]に 
\\	陽性[ようせい]であると 診断[しんだん]された。	陽性= 
\\	診断= 
\\	その薬物の乱用は非常に危険で、死亡する場合もある。	
\\	その 薬物[やくぶつ]の 乱用[らんよう]は 非常[ひじょう]に 危険[きけん]で、 死亡[しぼう]する 場合[ばあい]もある。	乱用= 
\\	彼は、薬物の乱用で逮捕された。	
\\	彼[かれ]は、 薬物[やくぶつ]の 乱用[らんよう]で 逮捕[たいほ]された。	乱用= 
\\	逮捕される= 
\\	子どもたちがけんかを始めたときは、私が介入してやめさせます。	
\\	子[こ]どもたちがけんかを 始[はじ]めたときは、 私[わたし]が 介入[かいにゅう]してやめさせます。	介入=かいにゅう= 
\\	当時の経済理論は、政府介入を否定していた。	
\\	当時[とうじ]の 経済[けいざい] 理論[りろん]は、 政府[せいふ] 介入[かいにゅう]を 否定[ひてい]していた。	介入=かいにゅう= 
\\	その山は実に険しい。	
\\	その 山[やま]は 実[じつ]に 険[けわ]しい。	険しい= 
\\	シングルマザーの道は険しい。	
\\	シングルマザーの 道[みち]は 険[けわ]しい。	険しい= 
\\	彼は険しい顔をしていた。	
\\	彼[かれ]は 険[けわ]しい 顔[かお]をしていた。	険しい= 
\\	その道路には広告板が並んでいます。	
\\	その 道路[どうろ]には 広告[こうこく] 板[ばん]が 並[なら]んでいます。	
\\	スキー板が、私の車の後ろから突き出ています。	
\\	スキー 板[いた]が、 私[わたし]の 車[くるま]の 後[うし]ろから 突き出[つきで]ています。	スキー板= 
\\	突き出る= 
\\	おなかが突き出てしまってますね。	
\\	おなかが 突き出[つきで]てしまってますね。	突き出る= 
\\	これを突き破るものは何もありません。	
\\	これを 突き破[つきやぶ]るものは 何[なに]もありません。	突き破る= 
\\	事故の原因を突き止める調査が進行中である。	
\\	事故[じこ]の 原因[げんいん]を 突き止[つきと]める 調査[ちょうさ]が 進行[しんこう] 中[ちゅう]である。	突き止める= 
\\	彼は彼女に銃を突きつけた。	
\\	彼[かれ]は 彼女[かのじょ]に 銃[じゅう]を 突[つ]きつけた。	突きつける= 
\\	彼は私の喉にナイフを突きつけた。	
\\	彼[かれ]は 私[わたし]の 喉[のど]にナイフを 突[つ]きつけた。	突きつける= 
\\	なんだか陳腐に聞こえるかもしれないが、それが真実です。	
\\	なんだか 陳腐[ちんぷ]に 聞[き]こえるかもしれないが、それが 真実[しんじつ]です。	陳腐=ちんぷ= 
\\	その本は時たま陳腐に思われる。	
\\	その 本[ほん]は 時[とき]たま 陳腐[ちんぷ]に 思[おも]われる。	陳腐=ちんぷ= 
\\	時たま= 
\\	低い失業率は、必ずしも堅調な経済を示すものではありません。	
\\	低[ひく]い 失業[しつぎょう] 率[りつ]は、 必[かなら]ずしも 堅調[けんちょう]な 経済[けいざい]を 示[しめ]すものではありません。	堅調=けんちょう= 
\\	文字通りの意味ではなかった。	
\\	文字通[もじどお]りの 意味[いみ]ではなかった。	文字通り= 
\\	日本は47の都道府県から成り立っている。	
\\	日本は47の 都道府県[とどうふけん]から 成り立[なりた]っている。	都道府県=とどうふけん= 
\\	よりによって何でジョンなんかと結婚したんだい。	
\\	よりによって 何[なん]でジョンなんかと 結婚[けっこん]したんだい。	よりによって= 
\\	当社は従業員を処遇する方法を常に改善していかなければならない。	
\\	当社[とうしゃ]は 従業[じゅうぎょう] 員[いん]を 処遇[しょぐう]する 方法[ほうほう]を 常[つね]に 改善[かいぜん]していかなければならない。	処遇= 
\\	ファッション雑誌によると、今シーズンは黒の装いが再び流行している。	
\\	ファッション 雑誌[ざっし]によると、 今[こん]シーズンは 黒[くろ]の 装[よそお]いが 再[ふたた]び 流行[りゅうこう]している。	装う= 
\\	私は今日、冷蔵庫から消費期限切れの食べ物を全部処分しました。	
\\	私は 今日[きょう]、 冷蔵庫[れいぞうこ]から 消費[しょうひ] 期限切[きげんぎ]れの 食べ物[たべもの]を 全部[ぜんぶ] 処分[しょぶん]しました。	処分= 
\\	私は古い車を処分しました。	
\\	私[わたし]は 古[ふる]い 車[くるま]を 処分[しょぶん]しました。	処分= 
\\	ひどく気分が悪かったが、彼女はつとめて平静を装った。	
\\	ひどく 気分[きぶん]が 悪[わる]かったが、 彼女[かのじょ]はつとめて 平静[へいせい]を 装[よそお]った。	平静=へいせい= 
\\	装う= 
\\	彼は隠れみのとしてカメラマンを装った。	
\\	彼[かれ]は 隠[かく]れみのとしてカメラマンを 装[よそお]った。	隠れみの= 
\\	装う= 
\\	彼女は妊娠を装って彼をだまし、結婚に持ち込んだ。	
\\	彼女[かのじょ]は 妊娠[にんしん]を 装[よそお]って 彼[かれ]をだまし、 結婚[けっこん]に 持ち込[もちこ]んだ。	妊娠= 
\\	装う= 
\\	これは何もかもが私には初めてなんです。	
\\	これは 何[なに]もかもが 私[わたし]には 初[はじ]めてなんです。	何もかも= 
\\	何かも変わってしまった。	
\\	何[なに]かも 変[か]わってしまった。	何もかも= 
\\	お年寄りの世話をするのは、やりがいのある仕事です。	
\\	お 年寄[としよ]りの 世話[せわ]をするのは、やりがいのある 仕事[しごと]です。	やりがい= 
\\	教職はやりがいのある職業です。	
\\	教職[きょうしょく]はやりがいのある 職業[しょくぎょう]です。	やりがい= 
\\	読み書きのできない大人に読み書きができるよう教えるのはやりがいのある経験です。	
\\	読み書[よみか]きのできない 大人[おとな]に 読み書[よみか]きができるよう 教[おし]えるのはやりがいのある 経験[けいけん]です。	やりがい= 
\\	イスラム教徒は一生に少なくとも一回はメッカへ巡礼しなければならない。	
\\	イスラム 教徒[きょうと]は 一生[いっしょう]に 少[すく]なくとも一 回[かい]はメッカへ 巡礼[じゅんれい]しなければならない。	巡礼=じゅんれい= 
\\	先日、ものすごく現実離れした体験をした。	
\\	先日[せんじつ]、ものすごく 現実離[げんじつばな]れした 体験[たいけん]をした。	
\\	愛は永遠に続くのか、それともはかないのだろうか。	
\\	愛[あい]は 永遠[えいえん]に 続[つづ]くのか、それともはかないのだろうか。	はかない= 
\\	私たちの友情は信頼に基づいている。	
\\	私[わたし]たちの 友情[ゆうじょう]は 信頼[しんらい]に 基[もと]づいている。	
\\	彼は怒りっぽい。	
\\	彼[かれ]は 怒[いか]りっぽい。	
\\	我々は正々堂々と闘うことを誓います。	
\\	我々[われわれ]は 正々堂々[せいせいどうどう]と 闘[たたか]うことを 誓[ちか]います。	正々堂々= 
\\	闘う= 
\\	誓う= 
\\	消防士たちにとって火災と闘うことは大変な仕事である。	
\\	消防[しょうぼう] 士[し]たちにとって 火災[かさい]と 闘[たたか]うことは 大変[たいへん]な 仕事[しごと]である。	闘う= 
\\	これまでの人生、私はずっとこのために闘い続けてきました。	
\\	これまでの 人生[じんせい]、 私[わたし]はずっとこのために 闘[たたか]い 続[つづ]けてきました。	闘う= 
\\	医者たちは時間と闘いながら彼女の命を救おうとした。	
\\	医者[いしゃ]たちは 時間[じかん]と 闘[たたか]いながら 彼女[かのじょ]の 命[いのち]を 救[すく]おうとした。	闘う= 
\\	市は、古い建物の取り壊しを決定しました。	
\\	市[し]は、 古[ふる]い 建物[たてもの]の 取り壊[とりこわ]しを 決定[けってい]しました。	取り壊す= 
\\	新しい家のスペースを作るため、その古い家は取り壊された。	
\\	新[あたら]しい 家[いえ]のスペースを 作[つく]るため、その 古[ふる]い 家[いえ]は 取り壊[とりこわ]された。	取り壊す= 
\\	私にそんな見え透いたうそをついても、無駄だよ。	
\\	私[わたし]にそんな 見え透[みえす]いたうそをついても、 無駄[むだ]だよ。	見え透く= 
\\	うそがばれて、彼は同僚の信用をなくした。	
\\	うそがばれて、 彼[かれ]は 同僚[どうりょう]の 信用[しんよう]をなくした。	ばれる= 
\\	彼女はその薬の効能に、疑いを抱き始めた。	
\\	彼女[かのじょ]はその 薬[くすり]の 効能[こうのう]に、 疑[うたが]いを 抱[いだ]き 始[はじ]めた。	
\\	あらぬ疑いをかけられて、本当に迷惑しています。	
\\	あらぬ 疑[うたが]いをかけられて、 本当[ほんとう]に 迷惑[めいわく]しています。	あらぬ= 
\\	彼女の証言が、私にかけられた疑いを晴らしてくれるだろう。	
\\	彼女[かのじょ]の 証言[しょうげん]が、 私[わたし]にかけられた 疑[うたが]いを 晴[は]らしてくれるだろう。	証言= 
\\	疑いを晴らす= 
\\	あなたがこれ以上言い訳しても、周囲の疑いを招くだけです。	
\\	あなたがこれ 以上[いじょう] 言い訳[いいわけ]しても、 周囲[しゅうい]の 疑[うたが]いを 招[まね]くだけです。	
\\	彼に音楽の才能があるということには、疑いの余地がない。	
\\	彼[かれ]に 音楽[おんがく]の 才能[さいのう]があるということには、 疑[うたが]いの 余地[よち]がない。	
\\	彼は腕を組んで、じっと考え事をしていた。	
\\	彼[かれ]は 腕[うで]を 組[く]んで、じっと 考え事[かんがえごと]をしていた。	腕を組む= 
\\	考え事= 
\\	彼は最近、めきめきゴルフの腕を上げている。	
\\	彼[かれ]は 最近[さいきん]、めきめきゴルフの 腕[うで]を 上[あ]げている。	腕を上げる= 
\\	彼は日本で、ホテルの経営者として腕をふるっている。	
\\	彼[かれ]は日本で、ホテルの 経営[けいえい] 者[しゃ]として 腕[うで]をふるっている。	腕をふるう= 
\\	彼女は毎週、料理学校に通って、料理の腕を磨いた。	
\\	彼女[かのじょ]は 毎週[まいしゅう]、 料理[りょうり] 学校[がっこう]に 通[かよ]って、 料理[りょうり]の 腕[うで]を 磨[みが]いた。	腕を磨く= 
\\	母は腕によりをかけて、クリスマスディナーを作った。	
\\	母[はは]は 腕[うで]によりをかけて、クリスマスディナーを 作[つく]った。	腕によりをかける= 
\\	彼らの横暴を、ただ腕をこまねいて見ているわけにはいきません。	
\\	彼[かれ]らの 横暴[おうぼう]を、ただ 腕[うで]をこまねいて 見[み]ているわけにはいきません。	腕をこまねく= 
\\	横暴= 
\\	うわさをすれば影で、ほら、アキラがやって来た。	
\\	うわさをすれば 影[かげ]で、ほら、アキラがやって 来[き]た。	やって来る= 
\\	クラスメートたちはみな、その新任の先生のうわさをしている。	
\\	クラスメートたちはみな、その 新任[しんにん]の 先生[せんせい]のうわさをしている。	
\\	そんな根も葉もないうわさを立てたのは誰なんだ?	
\\	そんな 根[ね]も 葉[は]もないうわさを 立[た]てたのは 誰[だれ]なんだ?	根も葉もない= 
\\	うわさを立てる= 
\\	その新しいフランス料理店のことは、うわさには聞いているが、まだ試したことはない。	
\\	その 新[あたら]しいフランス 料理[りょうり] 店[てん]のことは、うわさには 聞[き]いているが、まだ 試[ため]したことはない。	うわさに聞く= 
\\	試す= 
\\	私の家は泥棒に入られたことがない。	
\\	私[わたし]の 家[いえ]は 泥棒[どろぼう]に 入[はい]られたことがない。	泥棒=どろぼう= 
\\	人々は、政府に反対して街頭デモを行った。	
\\	人々[ひとびと]は、 政府[せいふ]に 反対[はんたい]して 街頭[がいとう]デモを 行[おこな]った。	街頭= 
\\	彼は街頭で殴られた。	
\\	彼[かれ]は 街頭[がいとう]で 殴[なぐ]られた。	街頭= 
\\	殴る= 
\\	反戦運動家たちは、街頭でデモをしました。	
\\	反戦[はんせん] 運動[うんどう] 家[か]たちは、 街頭[がいとう]でデモをしました。	反戦= 
\\	街頭= 
\\	秘めた動機はない。	
\\	秘[ひ]めた 動機[どうき]はない。	秘める= 
\\	動機= 
\\	感情を内に秘めておくのはよくない。	
\\	感情[かんじょう]を 内[うち]に 秘[ひ]めておくのはよくない。	秘める= 
\\	私は、自分の秘めた願望を、決して誰にも話さなかった。	
\\	私は、 自分[じぶん]の 秘[ひ]めた 願望[がんぼう]を、 決[けっ]して 誰[だれ]にも 話[はな]さなかった。	秘める= 
\\	納得していないようですね。	
\\	納得[なっとく]していないようですね。	納得= 
\\	その件についてお互いに納得のいく解決を見いだす必要があります。	
\\	その 件[けん]についてお 互[たが]いに 納得[なっとく]のいく 解決[かいけつ]を 見[み]いだす 必要[ひつよう]があります。	納得がいく= 
\\	捕まったとき、120キロで走っていた。	
\\	捕[つか]まったとき、120キロで 走[はし]っていた。	捕まる= 
\\	タクシーを捕まえるのが大変だった。	
\\	タクシーを 捕[つか]まえるのが 大変[たいへん]だった。	捕まえる= 
\\	タクシーを捕まえるのにひどい目に遭った。	
\\	タクシーを 捕[つか]まえるのにひどい 目[め]に 遭[あ]った。	捕まえる= 
\\	目に遭う= 
\\	今日、会社に行く途中で交通渋滞に捕まりました。	
\\	今日[きょう]、 会社[かいしゃ]に 行[い]く 途中[とちゅう]で 交通[こうつう] 渋滞[じゅうたい]に 捕[つか]まりました。	交通渋滞= 
\\	彼はスピード違反で捕まりました。	
\\	彼[かれ]はスピード 違反[いはん]で 捕[つか]まりました。	違反= 
\\	彼女は電車に乗ってるとき痴漢を捕まえた。	
\\	彼女[かのじょ]は 電車[でんしゃ]に 乗[の]ってるとき 痴漢[ちかん]を 捕[つか]まえた。	
\\	彼は、音楽と数学に秀でています。	
\\	彼[かれ]は、 音楽[おんがく]と 数学[すうがく]に 秀[ひい]でています。	秀でる= 
\\	一匹の猿が動物園から逃げ出したが、後に捕らえられました。	
\\	一 匹[ひき]の 猿[さる]が 動物[どうぶつ] 園[えん]から 逃げ出[にげだ]したが、 後[ご]に 捕[と]らえられました。	捕らえる= 
\\	彼は敵に捕らえられ拷問された。	
\\	彼[かれ]は 敵[てき]に 捕[と]らえられ 拷問[ごうもん]された。	捕らえる= 
\\	拷問= 
\\	スタッフは優秀で、技術は最先端です。	
\\	スタッフは 優秀[ゆうしゅう]で、 技術[ぎじゅつ]は 最先端[さいせんたん]です。	最先端= 
\\	当社は国籍・性別に関わらず、優秀な人材を採用します。	
\\	当社[とうしゃ]は 国籍[こくせき]・ 性別[せいべつ]に 関[かか]わらず、 優秀[ゆうしゅう]な 人材[じんざい]を 採用[さいよう]します。	
\\	彼女の心は彼の求婚に激しく揺れ動いた。	
\\	彼女[かのじょ]の 心[こころ]は 彼[かれ]の 求婚[きゅうこん]に 激[はげ]しく 揺れ動[ゆれうご]いた。	求婚= 
\\	揺れ動く= 
\\	彼女の心は二人の男性の間で揺れ動いた。	
\\	彼女[かのじょ]の 心[こころ]は二 人[にん]の 男性[だんせい]の 間[ま]で 揺れ動[ゆれうご]いた。	揺れ動く= 
\\	希少動物の捕獲を禁止する法律は、既に施行されている。	
\\	希少[きしょう] 動物[どうぶつ]の 捕獲[ほかく]を 禁止[きんし]する 法律[ほうりつ]は、 既[すで]に 施行[しこう]されている。	希少= 
\\	捕獲= 
\\	施行= 
\\	我々は、希少動物を発見する6日間の旅を引き受けました。	
\\	我々[われわれ]は、 希少[きしょう] 動物[どうぶつ]を 発見[はっけん]する 6日間[むいかかん]の 旅[たび]を 引き受[ひきう]けました。	希少= 
\\	引き受ける= 
\\	汚染されていない自然はどんどん希少なものになっていく。	
\\	汚染[おせん]されていない 自然[しぜん]はどんどん 希少[きしょう]なものになっていく。	希少= 
\\	いつの日か、
\\	チップが人体に内蔵される日が来るだろう。	
\\	いつの 日[ひ]か、 
\\	チップが 人体[じんたい]に 内蔵[ないぞう]される 日[ひ]が 来[く]るだろう。	内蔵= 
\\	このノートパソコンは内蔵バッテリで6時間作動する。	
\\	このノートパソコンは 内蔵[ないぞう]バッテリで6 時間[じかん] 作動[さどう]する。	内蔵= 
\\	作動= 
\\	僕のノートパソコンにはカメラが内蔵されている。	
\\	僕[ぼく]のノートパソコンにはカメラが 内蔵[ないぞう]されている。	内蔵= 
\\	理論的には良さそうだが、実際はうまくいかないだろう。	
\\	理論[りろん] 的[てき]には 良[よ]さそうだが、 実際[じっさい]はうまくいかないだろう。	
\\	彼女をうつぶせに寝かした。	
\\	彼女をうつぶせに 寝[ね]かした。	うつぶせ= 
\\	寝かす= 
\\	私はうつぶせに寝るのが好きだ。	
\\	私はうつぶせに寝るのが好きだ。	うつぶせ= 
\\	彼は複雑な心境だった。	
\\	彼は 複雑[ふくざつ]な 心境[しんきょう]だった。	心境= 
\\	彼は眼鏡を掛けて変装しました。	
\\	彼は 眼鏡[めがね]を 掛[か]けて 変装[へんそう]しました。	変装= 
\\	彼女の変装に、母親以外の皆がだまされた。	
\\	彼女の 変装[へんそう]に、 母親[ははおや] 以外[いがい]の 皆[みな]がだまされた。	変装= 
\\	だます= 
\\	顔を隠す変装が必要です。	
\\	顔[かお]を 隠[かく]す 変装[へんそう]が必要です。	変装= 
\\	蔵書の大半は英語の本だ。	
\\	蔵書[ぞうしょ]の 大半[たいはん]は英語の本だ。	蔵書= 
\\	これらの本は、あるアーティストの個人の蔵書です。	
\\	これらの本は、あるアーティストの 個人[こじん]の 蔵書[ぞうしょ]です。	蔵書= 
\\	私の蔵書は二千冊に及ぶ。	
\\	私の 蔵書[ぞうしょ]は 二千[にせん] 冊[さつ]に 及[およ]ぶ。	蔵書= 
\\	及ぶ= 
\\	あなたの提案の要点を簡潔に述べてくれますか?	
\\	あなたの 提案[ていあん]の 要点[ようてん]を 簡潔[かんけつ]に 述[の]べてくれますか?	簡潔= 
\\	この報告書は簡潔で要を得ている。	
\\	この 報告[ほうこく] 書[しょ]は 簡潔[かんけつ]で 要[よう]を 得[え]ている。	簡潔= 
\\	要を得る= 
\\	その辞書は簡潔な定義を載せている。	
\\	その 辞書[じしょ]は 簡潔[かんけつ]な 定義[ていぎ]を 載[の]せている。	簡潔= 
\\	最後に、私の考えを簡潔に要約したいと思います。	
\\	最後[さいご]に、 私[わたし]の 考[かんが]えを 簡潔[かんけつ]に 要約[ようやく]したいと 思[おも]います。	簡潔= 
\\	俳句は、とても簡潔な詩の形態です。	
\\	俳句[はいく]は、とても 簡潔[かんけつ]な 詩[し]の 形態[けいたい]です。	簡潔= 
\\	形態= 
\\	話をする時には、彼は簡潔に話しました。	
\\	話[はなし]をする 時[とき]には、 彼[かれ]は 簡潔[かんけつ]に 話[はな]しました。	簡潔= 
\\	気象状態に合った服装をする必要がある。	
\\	気象[きしょう] 状態[じょうたい]に 合[あ]った 服装[ふくそう]をする 必要[ひつよう]がある。	気象= 
\\	気象庁は北海道の東部海岸沿いに津波警報を出しました。	
\\	気象庁[きしょうちょう]は 北海道[ほっかいどう]の 東部[とうぶ] 海岸[かいがん] 沿[ぞ]いに 津波[つなみ] 警報[けいほう]を 出[だ]しました。	気象庁= 
\\	東部= 
\\	津波警報= 
\\	おびえた犬は後ろ足の間に尾を入れた。	
\\	おびえた 犬[いぬ]は 後ろ足[うしろあし]の 間[あいだ]に 尾[お]を 入[い]れた。	おびえる= 
\\	猫は浮かない気分の時に尾を振る。	
\\	猫[ねこ]は 浮[う]かない 気分[きぶん]の 時[とき]に 尾[お]を 振[ふ]る。	浮く= 
\\	貯蔵室にはどんな種類の物を保管していますか?	
\\	貯蔵[ちょぞう] 室[しつ]にはどんな 種類[しゅるい]の 物[もの]を 保管[ほかん]していますか?	貯蔵室= 
\\	念のため、傘を持っていった方がいいですよ。	
\\	念[ねん]のため、 傘[かさ]を 持[も]っていった 方[ほう]がいいですよ。	念のため= 
\\	念のため、携帯の番号を教えていただけますか?	
\\	念[ねん]のため、 携帯[けいたい]の 番号[ばんごう]を 教[おし]えていただけますか?	念のため= 
\\	念のためにシートベルトを締めてください。	
\\	念[ねん]のためにシートベルトを 締[し]めてください。	念のため= 
\\	私はいつも念のため着替えを持って行きます。	
\\	私[わたし]はいつも 念[ねん]のため 着替[きが]えを 持[も]って 行[い]きます。	念のため= 
\\	着替え= 
\\	このメールを興味のありそうな方に転送していただけますか?	
\\	このメールを 興味[きょうみ]のありそうな 方[ほう]に 転送[てんそう]していただけますか?	転送= 
\\	彼女はおしゃれに服を着こなすが、最新の流行は拒否します。	
\\	彼女[かのじょ]はおしゃれに 服[ふく]を 着[き]こなすが、 最新[さいしん]の 流行[りゅうこう]は 拒否[きょひ]します。	着こなす= 
\\	拒否= 
\\	彼女はいつもセンスのいい服装をしているね。	
\\	彼女[かのじょ]はいつもセンスのいい 服装[ふくそう]をしているね。	
\\	苦境にある時には音楽が必要です。	
\\	苦境[くきょう]にある 時[とき]には 音楽[おんがく]が 必要[ひつよう]です。	苦境= 
\\	このことは、既に苦境にある政府にとってとどめとなるでしょう。	
\\	このことは、 既[すで]に 苦境[くきょう]にある 政府[せいふ]にとってとどめとなるでしょう。	苦境= 
\\	とどめ= 
\\	樹木は単なる物ではなく、生きて機能している生命体です。	
\\	樹木[じゅもく]は 単[たん]なる 物[もの]ではなく、 生[い]きて 機能[きのう]している 生命[せいめい] 体[たい]です。	樹木= 
\\	生命体= 
\\	この辺りは樹木が豊富です。	
\\	この 辺[あた]りは 樹木[じゅもく]が 豊富[ほうふ]です。	樹木= 
\\	彼は逆境に打ち勝って、トーナメントで優勝しました。	
\\	彼[かれ]は 逆境[ぎゃっきょう]に 打ち勝[うちか]って、トーナメントで 優勝[ゆうしょう]しました。	逆境= 
\\	打ち勝つ= 
\\	三日間にわたる審議後評決が下された。	
\\	三日間[みっかかん]にわたる 審議[しんぎ] 後[ご] 評決[ひょうけつ]が 下[くだ]された。	審議= 
\\	評決= 
\\	私は忘れないうちに夢を書き留めた。	
\\	私[わたし]は 忘[わす]れないうちに 夢[ゆめ]を 書き留[かきと]めた。	書き留める= 
\\	誰かがありがたく思ったら、あなたの行いは善行と見なされる。	
\\	誰[だれ]かがありがたく 思[おも]ったら、あなたの 行[おこな]いは 善行[ぜんこう]と 見[み]なされる。	行い= 
\\	善行= 
\\	見なす= 
\\	彼はいつか悪行の報いを受けるだろう。	
\\	彼[かれ]はいつか 悪行[あくぎょう]の 報[むく]いを 受[う]けるだろう。	悪行= 
\\	人生の最後に、彼は、自分のすべての悪事を詫びました。	
\\	人生[じんせい]の 最後[さいご]に、 彼[かれ]は、 自分[じぶん]のすべての 悪事[あくじ]を 詫[わ]びました。	最期= 
\\	悪事= 
\\	詫びる= 
\\	彼は病気で、最期の時が近づいていた。	
\\	彼[かれ]は 病気[びょうき]で、 最期[さいご]の 時[とき]が 近[ちか]づいていた。	最期= 
\\	日本人はお互いに出会うとしばしば会釈する。	
\\	日本人[にっぽんじん]はお 互[たが]いに 出会[であ]うとしばしば 会釈[えしゃく]する。	しばしば= 
\\	会釈= 
\\	夕日が眩しい。	
\\	夕日[せきじつ]が 眩[まぶ]しい。	夕日= 
\\	眩しい= 
\\	水平線のかなたに沈んでいく夕日ほど美しいものは他にはない。	
\\	水平[すいへい] 線[せん]のかなたに 沈[しず]んでいく 夕日[せきじつ]ほど 美[うつく]しいものは 他[た]にはない。	水平線= 
\\	かなた= 
\\	沈む= 
\\	夕日= 
\\	パニック障害者は、はっきりとした理由もなく発作を起こす。	
\\	パニック 障害[しょうがい] 者[しゃ]は、はっきりとした 理由[りゆう]もなく 発作[ほっさ]を 起[お]こす。	パニック障害= 
\\	発作= 
\\	彼を強欲だと見なすべきではありません。	
\\	彼[かれ]を 強欲[ごうよく]だと 見[み]なすべきではありません。	強欲= 
\\	見なす= 
\\	中国は、台湾を自国の省の一つと見なしている。	
\\	中国[ちゅうごく]は、 台湾[たいわん]を 自国[じこく]の 省[しょう]の 一[ひと]つと 見[み]なしている。	省= 
\\	見なす= 
\\	彼はその状況が重大であると見なした。	
\\	彼[かれ]はその 状況[じょうきょう]が 重大[じゅうだい]であると 見[み]なした。	重大= 
\\	見なす= 
\\	生産性と効率は密接に関係しています。	
\\	生産[せいさん] 性[せい]と 効率[こうりつ]は 密接[みっせつ]に 関係[かんけい]しています。	生産性= 
\\	密接= 
\\	言語と文化には密接な関係がある。	
\\	言語[げんご]と 文化[ぶんか]には 密接[みっせつ]な 関係[かんけい]がある。	密接= 
\\	気味の悪い話は止せ。	
\\	気味[きみ]の 悪[わる]い 話[はなし]は 止[よ]せ。	気味= 
\\	止す= 
\\	ちょっと下痢気味です。	
\\	ちょっと 下痢[げり] 気味[ぎみ]です。	-気味= 
\\	ちょっと風邪気味です。	
\\	ちょっと 風邪[かぜ] 気味[ぎみ]です。	風邪気味= 
\\	彼は痩せ気味である。	
\\	彼[かれ]は 痩[や]せ 気味[ぎみ]である。	-気味= 
\\	彼は自信過剰気味です。	
\\	彼[かれ]は 自信[じしん] 過剰[かじょう] 気味[ぎみ]です。	自信過剰= 
\\	-気味= 
\\	暑い夏の間はかなりだらけ気味だった。	
\\	暑[あつ]い 夏[なつ]の 間[ま]はかなりだらけ 気味[ぎみ]だった。	だらける= 
\\	-気味= 
\\	そこまでいくとかわいいというより不気味です。	
\\	そこまでいくとかわいいというより 不気味[ぶきみ]です。	不気味= 
\\	それらは不気味なほど現実的だ。	
\\	それらは 不気味[ぶきみ]なほど 現実[げんじつ] 的[てき]だ。	不気味= 
\\	詳細は未定です。	
\\	詳細[しょうさい]は 未定[みてい]です。	
\\	その男性は不審な状況下で死んでいるのを発見された。	
\\	その 男性[だんせい]は 不審[ふしん]な 状況[じょうきょう] 下[か]で 死[し]んでいるのを 発見[はっけん]された。	
\\	ホワイト氏の不審な行動についてご報告したいと思います。	
\\	ホワイト 氏[し]の 不審[ふしん]な 行動[こうどう]についてご 報告[ほうこく]したいと 思[おも]います。	
\\	世界では、毎日劇的な変化が起こりつつある。	
\\	世界[せかい]では、 毎日[まいにち] 劇的[げきてき]な 変化[へんか]が 起[お]こりつつある。	
\\	世界の人口は20世紀に劇的に増加しました。	
\\	世界[せかい]の 人口[じんこう]は20 世紀[せいき]に 劇的[げきてき]に 増加[ぞうか]しました。	
\\	彼は3週間アメリカに行き、戻ってきた時には劇的に変化していた。	
\\	彼[かれ]は3 週間[しゅうかん]アメリカに 行[い]き、 戻[もど]ってきた 時[とき]には 劇的[げきてき]に 変化[へんか]していた。	
\\	技術革新は私たちの生活を劇的に変えてきた。	
\\	技術[ぎじゅつ] 革新[かくしん]は 私[わたし]たちの 生活[せいかつ]を 劇的[げきてき]に 変[か]えてきた。	
\\	自分勝手な行動が彼の評判を劇的に落とした。	
\\	自分勝手[じぶんかって]な 行動[こうどう]が 彼[かれ]の 評判[ひょうばん]を 劇的[げきてき]に 落[お]とした。	自分勝手= 
\\	需要は年々、劇的に高まっている。	
\\	需要[じゅよう]は 年々[ねんねん]、 劇的[げきてき]に 高[たか]まっている。	
\\	彼らの関係は悲劇的な結末を迎えるだろう。	
\\	彼[かれ]らの 関係[かんけい]は 悲劇[ひげき] 的[てき]な 結末[けつまつ]を 迎[むか]えるだろう。	悲劇的= 
\\	悲劇的な事故が、家族をさらに結束させた。	
\\	悲劇[ひげき] 的[てき]な 事故[じこ]が、 家族[かぞく]をさらに 結束[けっそく]させた。	悲劇的= 
\\	結束= 
\\	初期の恐竜は、ほぼ確実に捕食動物だった。	
\\	初期[しょき]の 恐竜[きょうりゅう]は、ほぼ 確実[かくじつ]に 捕食[ほしょく] 動物[どうぶつ]だった。	初期= 
\\	恐竜= 
\\	捕食= 
\\	イルカは世界で最も賢い動物だと思う。	
\\	イルカは 世界[せかい]で 最[もっと]も 賢[かしこ]い 動物[どうぶつ]だと 思[おも]う。	賢い= 
\\	彼は、実際よりも賢いと思われることが多い。	
\\	彼[かれ]は、 実際[じっさい]よりも 賢[かしこ]いと 思[おも]われることが 多[おお]い。	賢い= 
\\	犬は忠実で賢い動物だと思う。	
\\	犬[いぬ]は 忠実[ちゅうじつ]で 賢[かしこ]い 動物[どうぶつ]だと 思[おも]う。	賢い= 
\\	彼は正直そうに見えるが、実はかなり悪賢い。	
\\	彼[かれ]は 正直[しょうじき]そうに 見[み]えるが、 実[じつ]はかなり 悪賢[わるがしこ]い。	悪賢い= 
\\	彼らはお互いに友好的な間柄である。	
\\	彼[かれ]らはお 互[たが]いに 友好[ゆうこう] 的[てき]な 間柄[あいだがら]である。	友好的= 
\\	間柄=あいだがら= (血縁の) 
\\	(交際の) 
\\	私たちはお互いに口も利きたくない間柄です。	
\\	私[わたし]たちはお 互[たが]いに 口[くち]も 利[き]きたくない 間柄[あいだがら]です。	口をきく= 
\\	間柄=あいだがら= (血縁の) 
\\	(交際の) 
\\	孤独感は真夜中に最も強まる。	
\\	孤独[こどく] 感[かん]は 真夜中[まよなか]に 最[もっと]も 強[つよ]まる。	孤独=こどく= 
\\	真夜中= 
\\	その物質は酸素のような作用がある。	
\\	その 物質[ぶっしつ]は 酸素[さんそ]のような 作用[さよう]がある。	物質= 
\\	作用= 
\\	この会議の目的は、新しいオフィスに移転する計画を再検討することです。	
\\	この 会議[かいぎ]の 目的[もくてき]は、 新[あたら]しいオフィスに 移転[いてん]する 計画[けいかく]を 再[さい] 検討[けんとう]することです。	再検討= 
\\	彼は、そのスイミングクラブで救命員として働いている。	
\\	彼[かれ]は、そのスイミングクラブで 救命[きゅうめい] 員[いん]として 働[はたら]いている。	救命= 
\\	我々は現在、事故原因の究明に取り組んでいます。	
\\	我々[われわれ]は 現在[げんざい]、 事故[じこ] 原因[げんいん]の 究明[きゅうめい]に 取り組[とりく]んでいます。	究明= 
\\	捜査官たちは、いまだにその事故の原因を究明していない。	
\\	捜査[そうさ] 官[かん]たちは、いまだにその 事故[じこ]の 原因[げんいん]を 究明[きゅうめい]していない。	捜査官= 
\\	究明= 
\\	間違いなく私のせいじゃないんだから、八つ当たりしないでよ。	
\\	間違[まちが]いなく 私[わたし]のせいじゃないんだから、 八[や]つ 当[あ]たりしないでよ。	八つ当たり= 
\\	何を言い出すんだ?	
\\	何[なに]を 言い出[いいだ]すんだ?	
\\	英気を養うために、私は週末を田舎で過ごした。	
\\	英気[えいき]を 養[やしな]うために、 私[わたし]は 週末[しゅうまつ]を 田舎[いなか]で 過[す]ごした。	英気を養う= 
\\	ここにはビルがびっしりと立ち並んでいる。	
\\	ここにはビルがびっしりと 立ち並[たちなら]んでいる。	びっしり= 
\\	スケジュールがびっしりだ。	
\\	スケジュールがびっしりだ。	びっしり= 
\\	華やかなドレスが彼女の美しさを引き立てた。	
\\	華[はな]やかなドレスが 彼女[かのじょ]の 美[うつく]しさを 引き立[ひきた]てた。	華やか= 
\\	引き立てる= 
\\	卒業式には、ある程度の華やかさがあるべきです。	
\\	卒業[そつぎょう] 式[しき]には、ある 程度[ていど]の 華[はな]やかさがあるべきです。	華やか= 
\\	学校にしてはとても豪華です。	
\\	学校[がっこう]にしてはとても 豪華[ごうか]です。	にしては= 
\\	豪華= 
\\	彼女は華麗で人気者です。	
\\	彼女[かのじょ]は 華麗[かれい]で 人気[にんき] 者[もの]です。	華麗= 
\\	人気者=にんきもの
\\	彼女は高嶺の花です。	
\\	彼女[かのじょ]は 高嶺[たかね]の 花[はな]です。	高嶺の花= 
\\	彼の成功は華々しくとても有名になった。	
\\	彼[かれ]の 成功[せいこう]は 華々[はなばな]しくとても 有名[ゆうめい]になった。	華々しい= 
\\	あり合わせのもので手早くお弁当を作った。	
\\	あり 合[あ]わせのもので 手早[てばや]くお 弁当[べんとう]を 作[つく]った。	あり合わせ= 
\\	いつか田中にお引き合わせしたいと思っておりました。	
\\	いつか 田中[たなか]にお 引[ひ]き 合[あ]わせしたいと 思[おも]っておりました。	引き合わせる= 
\\	この記事は戦争の忘れられがちな側面に焦点を合わせている。	
\\	この 記事[きじ]は 戦争[せんそう]の 忘[わす]れられがちな 側面[そくめん]に 焦点[しょうてん]を 合[あ]わせている。	焦点= 
\\	合わせる= 
\\	こんな巡り合わせでなければあなたの妻になっていたかもしれません。	
\\	こんな 巡り合[めぐりあ]わせでなければあなたの 妻[つま]になっていたかもしれません。	巡り合わせ= 
\\	彼は試験に失敗して、自暴自棄になった。	
\\	彼[かれ]は 試験[しけん]に 失敗[しっぱい]して、 自暴自棄[じぼうじき]になった。	
\\	試験でのカンニングなど言語道断です。	
\\	試験[しけん]でのカンニングなど 言語道断[ごんごどうだん]です。	
\\	自画自賛したくはありませんが、私はまさに適任です。	
\\	自画[じが] 自賛[じさん]したくはありませんが、 私[わたし]はまさに 適任[てきにん]です。	適任= 
\\	試行錯誤しながらソフトを開発した。	
\\	試行錯誤[しこうさくご]しながらソフトを 開発[かいはつ]した。	
\\	私は、料理の仕方を試行錯誤で学びました。	
\\	私[わたし]は、 料理[りょうり]の 仕方[しかた]を 試行錯誤[しこうさくご]で 学[まな]びました。	
\\	思い通りの効果を上げるには、ある程度の試行錯誤が必要だろう。	
\\	思い通[おもいどお]りの 効果[こうか]を 上[あ]げるには、ある 程度[ていど]の 試行錯誤[しこうさくご]が 必要[ひつよう]だろう。	
\\	彼の身勝手さは身から出た錆となるだろう。	
\\	彼[かれ]の 身勝手[みがって]さは 身[み]から 出[で]た 錆[さび]となるだろう。	身から出た錆= 
\\	人々は、その計画を現実させるために力を合わせた。	
\\	人々[ひとびと]は、その 計画[けいかく]を 現実[げんじつ]させるために 力[ちから]を 合[あ]わせた。	力を合わせる= 
\\	今日の練習では、基本的な技術に焦点を合わせます。	
\\	今日[きょう]の 練習[れんしゅう]では、 基本[きほん] 的[てき]な 技術[ぎじゅつ]に 焦点[しょうてん]を 合[あ]わせます。	焦点= 
\\	合わせる= 
\\	日本では外国のレシピはいつだって日本人の味覚に合わせられている。	
\\	日本では 外国[がいこく]のレシピはいつだって 日本人[にほんじん]の 味覚[みかく]に 合[あ]わせられている。	いつだって= 
\\	合わせる= 
\\	服装に合わせた良い靴がないのが残念です。	
\\	服装[ふくそう]に 合[あ]わせた 良[よ]い 靴[くつ]がないのが 残念[ざんねん]です。	合わせる= 
\\	本を脇に置いて、私の話を聞いてください。	
\\	本[ほん]を 脇[わき]に 置[お]いて、 私[わたし]の 話[はなし]を 聞[き]いてください。	脇に置く= 
\\	私たちは、相違点を脇に置いて、力を合わせなければならない。	
\\	私[わたし]たちは、 相違[そうい] 点[てん]を 脇[わき]に 置[お]いて、 力[ちから]を 合[あ]わせなければならない。	相違点= 
\\	脇に置く= 
\\	つまらない講義に閉口しました。	
\\	つまらない 講義[こうぎ]に 閉口[へいこう]しました。	閉口= 
\\	隣のピアノの音に閉口している。	
\\	隣[となり]のピアノの 音[おと]に 閉口[へいこう]している。	閉口= 
\\	その会社は東京市場へ食い込む方法を模索している。	
\\	その 会社[かいしゃ]は 東京[とうきょう] 市場[しじょう]へ 食い込[くいこ]む 方法[ほうほう]を 模索[もさく]している。	食い込む= 
\\	模索= 
\\	私たちは、常に無駄を減らす方法を模索している。	
\\	私[わたし]たちは、 常[つね]に 無駄[むだ]を 減[へ]らす 方法[ほうほう]を 模索[もさく]している。	模索= 
\\	この絵画は、ヴァン・ゴッホのスタイルを模倣している。	
\\	この 絵画[かいが]は、ヴァン・ゴッホのスタイルを 模倣[もほう]している。	模倣= 
\\	子どもは模倣することによって学ぶ。	
\\	子[こ]どもは 模倣[もほう]することによって 学[まな]ぶ。	模倣= 
\\	美術を学ぶ上で、傑作を模倣することは有益です。	
\\	美術[びじゅつ]を 学[まな]ぶ 上[うえ]で、 傑作[けっさく]を 模倣[もほう]することは 有益[ゆうえき]です。	模倣= 
\\	有益= 
\\	彼は自分の職業については口を閉ざしている。	
\\	彼[かれ]は 自分[じぶん]の 職業[しょくぎょう]については 口[くち]を 閉[と]ざしている。	口を閉ざす= 
\\	おなかが減っていたが、運の悪いことに、最寄りの食堂は閉まっていました。	
\\	おなかが 減[へ]っていたが、 運[うん]の 悪[わる]いことに、 最寄[もよ]りの 食堂[しょくどう]は 閉[し]まっていました。	
\\	その酒屋は半日閉まっているようだ。	
\\	その 酒屋[さかや]は 半日[はんにち] 閉[し]まっているようだ。	
\\	ドアはバタンと閉まった。	
\\	ドアはバタンと 閉[し]まった。	バタン(と)= 
\\	水道の蛇口を閉めた。	
\\	水道[すいどう]の 蛇口[じゃぐち]を 閉[し]めた。	蛇口= 
\\	彼は鍵を掛けて私を家から閉め出しました。	
\\	彼[かれ]は 鍵[かぎ]を 掛[か]けて 私[わたし]を 家[いえ]から 閉め出[しめだ]しました。	
\\	その小説は世界中の人々の心に訴えました。	
\\	その 小説[しょうせつ]は 世界中[せかいじゅう]の 人々[ひとびと]の 心[こころ]に 訴[うった]えました。	訴える= 
\\	工事現場近くの住民は、騒音について苦情を訴えた。	
\\	工事[こうじ] 現場[げんば] 近[ちか]くの 住民[じゅうみん]は、 騒音[そうおん]について 苦情[くじょう]を 訴[うった]えた。	工事現場= 
\\	訴える= 
\\	彼らはすべての核兵器の廃絶を訴えている。	
\\	彼[かれ]らはすべての 核兵器[かくへいき]の 廃絶[はいぜつ]を 訴[うった]えている。	訴える= 
\\	彼の目はしっかりと閉じていた。	
\\	彼[かれ]の 目[め]はしっかりと 閉[と]じていた。	
\\	目を閉じるとあなたの顔が浮かびます。	
\\	目[め]を 閉[と]じるとあなたの 顔[かお]が 浮[う]かびます。	
\\	私は部屋に閉じこもって本を読んでいた。	
\\	私[わたし]は 部屋[へや]に 閉[と]じこもって 本[ほん]を 読[よ]んでいた。	閉じこもる= 
\\	為替レートの急上昇は避けられません。	
\\	為替[かわせ]レートの 急上昇[きゅうじょうしょう]は 避[さ]けられません。	為替レート= 
\\	その火災が人為的原因によって起こったという可能性は十分ある。	
\\	その 火災[かさい]が 人為[じんい] 的[てき] 原因[げんいん]によって 起[お]こったという 可能[かのう] 性[せい]は 十分[じゅうぶん]ある。	人為的= 
\\	戦争は究極のテロリズムだ。	
\\	戦争[せんそう]は 究極[きゅうきょく]のテロリズムだ。	究極= 
\\	私の究極の夢は、世界中を航海することだ。	
\\	私[わたし]の 究極[きゅうきょく]の 夢[ゆめ]は、 世界中[せかいじゅう]を 航海[こうかい]することだ。	究極= 
\\	航海= 
\\	参加者は無作為に2グループに分けられる。	
\\	参加[さんか] 者[しゃ]は 無[む] 作為[さくい]に2グループに 分[わ]けられる。	無作為= 
\\	その仕事は固定給ですか?それとも時給制ですか?	
\\	その 仕事[しごと]は 固定[こてい] 給[きゅう]ですか?それとも 時給[じきゅう] 制[せい]ですか?	固定給= 
\\	時給= 
\\	医者は折れた足をギプスで固定した。	
\\	医者[いしゃ]は 折[お]れた 足[あし]をギプスで 固定[こてい]した。	ギプス= 
\\	固定= 
\\	新しいアイデアを考え出す際に難しいのは、固定観念から抜け出すことだ。	
\\	新[あたら]しいアイデアを 考え出[かんがえだ]す 際[さい]に 難[むずか]しいのは、 固定[こてい] 観念[かんねん]から 抜け出[ぬけだ]すことだ。	固定観念= 
\\	彼はイギリス出身だが、日本固有のものに興味を持っている。	
\\	彼[かれ]はイギリス 出身[しゅっしん]だが、 日本[にっぽん] 固有[こゆう]のものに 興味[きょうみ]を 持[も]っている。	固有= 
\\	宗教は皆それぞれに固有の儀式を持っている。	
\\	宗教[しゅうきょう]は 皆[みな]それぞれに 固有[こゆう]の 儀式[ぎしき]を 持[も]っている。	固有= 
\\	儀式= 
\\	彼は正式に研究所から除名された。	
\\	彼[かれ]は 正式[せいしき]に 研究所[けんきゅうじょ]から 除名[じょめい]された。	
\\	どのような可能性でも除外したくはない。	
\\	どのような 可能[かのう] 性[せい]でも 除外[じょがい]したくはない。	除外= 
\\	堅いことは抜きにしよう。	
\\	堅[かた]いことは 抜[ぬ]きにしよう。	堅い= 
\\	ダンスを始める前は、とても体が堅くて柔軟性ゼロだった。	
\\	ダンスを 始[はじ]める 前[まえ]は、とても 体[からだ]が 堅[かた]くて 柔軟[じゅうなん] 性[せい]ゼロだった。	堅い= 
\\	5キロ減量するというエミリーの決意は、とても固い。	
\\	5キロ 減量[げんりょう]するというエミリーの 決意[けつい]は、とても 固[かた]い。	決意= 
\\	固い= 
\\	そこまで決意が固いのなら、私も力を貸すよ。	
\\	そこまで 決意[けつい]が 固[かた]いのなら、 私[わたし]も 力[ちから]を 貸[か]すよ。	決意が固い= 
\\	力を貸す= 
\\	彼女の決心はとても固いので、気持ちを変えさせることはできない。	
\\	彼女[かのじょ]の 決心[けっしん]はとても 固[かた]いので、 気持[きも]ちを 変[か]えさせることはできない。	固い= 
\\	監督は、その小説を映画化することに決めた。	
\\	監督[かんとく]は、その 小説[しょうせつ]を 映画[えいが] 化[か]することに 決[き]めた。	監督= 
\\	監督はシーンごとに多くテイクを撮るのが好きだ。	
\\	監督[かんとく]はシーンごとに 多[おお]くテイクを 撮[と]るのが 好[す]きだ。	監督= 
\\	ごとに= 
\\	監督者の方とお話ししたいのですが。	
\\	監督[かんとく] 者[しゃ]の 方[ほう]とお 話[はな]ししたいのですが。	監督者= 
\\	これまでに、その監督は10本の映画を作った。	
\\	これまでに、その 監督[かんとく]は 10本[じっぽん]の 映画[えいが]を 作[つく]った。	監督= 
\\	監視カメラは万引きをする人たちを思いとどまらせるはずだ。	
\\	監視[かんし]カメラは 万引[まんび]きをする 人[ひと]たちを 思[おも]いとどまらせるはずだ。	監視カメラ= 
\\	万引き= 
\\	思いとどまる= 
\\	ホームレスの人々を社会ののけ者にしてはならない。	
\\	ホームレスの 人々[ひとびと]を 社会[しゃかい]ののけ 者[もの]にしてはならない。	除け者= 
\\	あの党はいつも中流階級に迎合しています。	
\\	あの 党[とう]はいつも 中流[ちゅうりゅう] 階級[かいきゅう]に 迎合[げいごう]しています。	中流階級= 
\\	迎合= 
\\	この議題についてのこれ以上の議論は、次回の会議まで持ち越しになります。	
\\	この 議題[ぎだい]についてのこれ 以上[いじょう]の 議論[ぎろん]は、 次回[じかい]の 会議[かいぎ]まで 持ち越[もちこ]しになります。	議題= 
\\	持ち越す= 
\\	他人に対して優越感を持ってはいけない。	
\\	他人[たにん]に 対[たい]して 優越[ゆうえつ] 感[かん]を 持[も]ってはいけない。	優越感= 
\\	事前に予約が必要です。	
\\	事前[じぜん]に 予約[よやく]が 必要[ひつよう]です。	
\\	もっと事前にお知らせできなくて申し訳なく存じます。	
\\	もっと 事前[じぜん]にお 知[し]らせできなくて 申し訳[もうしわけ]なく 存[ぞん]じます。	
\\	この点では、コーヒーの値段がむやみに高い。	
\\	この 点[てん]では、コーヒーの 値段[ねだん]がむやみに 高[たか]い。	むやみに= 
\\	そのネズミは、10秒足らずの間にこの迷路を突破できます。	
\\	そのネズミは、10 秒[びょう] 足[た]らずの 間[あいだ]にこの 迷路[めいろ]を 突破[とっぱ]できます。	足らず= 
\\	突破= 
\\	彼は私の正体をすぐに見破っただろう。	
\\	彼[かれ]は 私[わたし]の 正体[しょうたい]をすぐに 見破[みやぶ]っただろう。	正体= 
\\	見破る= 
\\	ガラスの破片で切って、血が止まらない。	
\\	ガラスの 破片[はへん]で 切[き]って、 血[ち]が 止[と]まらない。	破片= 
\\	そのことは私には分かりかねますから、部長にお聞きになって下さい。	
\\	そのことは 私[わたし]には 分[わ]かりかねますから、 部長[ぶちょう]にお 聞[き]きになって 下[くだ]さい。	-かねる= 
\\	私は週末にゴルフをすることで気分転換を図っています。	
\\	私[わたし]は 週末[しゅうまつ]にゴルフをすることで 気分[きぶん] 転換[てんかん]を 図[はか]っています。	図る= 
\\	私が十万円出すことで問題は解決した。	
\\	私[わたし]が十 万[まん] 円[えん] 出[だ]すことで 問題[もんだい]は 解決[かいけつ]した。	
\\	朝晩簡単な体操をするだけのことですばらしい健康が保てます。	
\\	朝晩[あさばん] 簡単[かんたん]な 体操[たいそう]をするだけのことですばらしい 健康[けんこう]が 保[たも]てます。	
\\	私が参加することであなたに迷惑はかかりませんか。	
\\	私[わたし]が 参加[さんか]することであなたに 迷惑[めいわく]はかかりませんか。	
\\	彼のパーティーに行かなかったことで彼の気持ちを害したのでなければよいが。	
\\	彼[かれ]のパーティーに 行[い]かなかったことで 彼[かれ]の 気持[きも]ちを 害[がい]したのでなければよいが。	
\\	私は英語が下手なことで時々損をしている。	
\\	私[わたし]は 英語[えいご]が 下手[へた]なことで 時々[ときどき] 損[そん]をしている。	
\\	この国では外国人であることで得をすることがある。	
\\	この 国[くに]では 外国[がいこく] 人[じん]であることで 得[とく]をすることがある。	
\\	車が動き出した。	
\\	車[くるま]が 動き出[うごきだ]した。	~出す= 
\\	一歳になって初めて歩き出した。	
\\	一 歳[さい]になって 初[はじ]めて 歩[ある]き 出[だ]した。	~出す= 
\\	そのアイディアは誰が考え出したんですか。	
\\	そのアイディアは 誰[だれ]が 考え出[かんがえだ]したんですか。	~出す= 
\\	一時ぐらいかけてとうとうその本屋を探し出した。	
\\	一時[いちじ]ぐらいかけてとうとうその 本屋[ほんや]を 探し出[さがしだ]した。	~出す= 
\\	どうしてか分からなかったが、男は急に怒り出した?	
\\	どうしてか 分[わ]からなかったが、 男[おとこ]は 急[きゅう]に 怒[おこ]り 出[だ]した?	~出す= 
\\	ベッツィーはそのレポートを一日で書いたそうだ。	
\\	ベッツィーはそのレポートを一 日[にち]で 書[か]いたそうだ。	
\\	父は交通事故で入院しました。	
\\	父[ちち]は 交通[こうつう] 事故[じこ]で 入院[にゅういん]しました。	
\\	アメリカに来てから今日で三年になる。	
\\	アメリカに 来[き]てから 今日[きょう]で 三年[さんねん]になる。	
\\	父へ手紙を出したが、まだ返事が来ない。	
\\	父[ちち]へ 手紙[てがみ]を 出[だ]したが、まだ 返事[へんじ]が 来[こ]ない。	
\\	深田さんは犬を怖がっている。	
\\	深田[ふかた]さんは 犬[いぬ]を 怖[こわ]がっている。	
\\	怖い・欲しい 
\\	-がる 
\\	「が」
\\	「を」
\\	一男はスポーツカーを欲しがっている。	
\\	一男[かずお]はスポーツカーを 欲[ほ]しがっている。	
\\	怖い・欲しい 
\\	-がる 
\\	「が」
\\	「を」
\\	私は三時間ごとに薬を飲んだ。	
\\	私[わたし]は 三時間[さんじかん]ごとに 薬[くすり]を 飲[の]んだ。	
\\	一課ごとに試験がある。	
\\	一課[いっか]ごとに 試験[しけん]がある。	
\\	木村さんは会う人ごとに挨拶している。	
\\	木村[きむら]さんは 会[あ]う 人[ひと]ごとに 挨拶[あいさつ]している。	
\\	三日ごとにテニスをしています。	
\\	三日[みっか]ごとにテニスをしています。	
\\	マーサも来ますか。 
\\	はい、そのはずです。	
\\	マーサも 来[き]ますか。 
\\	はい、そのはずです。	
\\	私の方が上田さんよりよく食べる。	
\\	私[わたし]の 方[ほう]が 上田[うえだ]さんよりよく 食[た]べる。	
\\	石田さんの方が私より若いです。	
\\	石田[いしだ]さんの 方[ほう]が 私[わたし]より 若[わか]いです。	
\\	車で行く方がバスで行くより安い。	
\\	車[くるま]で 行[い]く 方[ほう]がバスで 行[い]くより 安[やす]い。	
\\	子供は元気な方が静かなより安心だ。	
\\	子供[こども]は 元気[げんき]な 方[ほう]が 静[しず]かなより 安心[あんしん]だ。	
\\	私は女である方が男であるより楽しいと思う。	
\\	私[わたし]は 女[おんな]である 方[ほう]が 男[おとこ]であるより 楽[たの]しいと 思[おも]う。	
\\	私はビールより酒の方をよく飲む。	
\\	私[わたし]はビールより 酒[さけ]の 方[ほう]をよく 飲[の]む。	
\\	学生は川田先生より木村先生の方によく質問に行く。	
\\	学生[がくせい]は 川田[かわた] 先生[せんせい]より 木村[きむら] 先生[せんせい]の 方[ほう]によく 質問[しつもん]に 行[い]く。	
\\	あの事件の後、皆、少しピリピリしていました。	
\\	あの 事件[じけん]の 後[あと]、 皆[みな]、 少[すこ]しピリピリしていました。	
\\	プレゼンテーションをするときにはいつも、私は極端にピリピリする。	
\\	プレゼンテーションをするときにはいつも、 私[わたし]は 極端[きょくたん]にピリピリする。	
\\	彼女は今朝ピリピリしているが、困ったことがあるに違いない。	
\\	彼女[かのじょ]は 今朝[けさ]ピリピリしているが、 困[こま]ったことがあるに 違[ちが]いない。	
\\	神経がピリピリしている。	
\\	神経[しんけい]がピリピリしている。	
\\	シクシクと痛みます。	
\\	シクシクと 痛[いた]みます。	
\\	そうカリカリするな!	
\\	そうカリカリするな!	
\\	私は彼女のことでカリカリしたりしません。	
\\	私[わたし]は 彼女[かのじょ]のことでカリカリしたりしません。	
\\	迷子の子が交番でシクシク泣いていた。	
\\	迷子[まいご]の 子[こ]が 交番[こうばん]でシクシク 泣[な]いていた。	しくしく泣く= 
\\	今後は、数字を念には念を入れて確認してもらいたい。	
\\	今後[こんご]は、 数字[すうじ]を 念[ねん]には 念[ねん]を 入[い]れて 確認[かくにん]してもらいたい。	念には念を入れる= 
\\	難しい語は用語解説に収録された。	
\\	難[むずか]しい 語[ご]は 用語[ようご] 解説[かいせつ]に 収録[しゅうろく]された。	用語解説= 
\\	収録= 
\\	クラッカーやナッツみたいな、バリバリ食べるものが何かある?	
\\	クラッカーやナッツみたいな、バリバリ 食[た]べるものが 何[なに]かある?	
\\	彼は今、編集者としてバリバリ働いている。	
\\	彼[かれ]は 今[いま]、 編集[へんしゅう] 者[しゃ]としてバリバリ 働[はたら]いている。	編集者= 
\\	魚をバリバリに焦がしてしまいました。	
\\	魚[さかな]をバリバリに 焦[こ]がしてしまいました。	焦がす= 
\\	顔をチラッと見ただけで、彼女こそ運命の人だと分かった。	
\\	顔[かお]をチラッと 見[み]ただけで、 彼女[かのじょ]こそ 運命[うんめい]の 人[ひと]だと 分[わ]かった。	
\\	正気ですか?	
\\	正気[しょうき]ですか?	正気=しょうき= 
\\	彼女にふられて正気を失った。	
\\	彼女[かのじょ]にふられて 正気[しょうき]を 失[うしな]った。	振られる= 
\\	正気=しょうき= 
\\	ミスがあると、部長はひどく冷ややかだ。	
\\	ミスがあると、 部長[ぶちょう]はひどく 冷[ひ]ややかだ。	冷ややか= 
\\	彼の意見は穏やかだが、上司の意見は極端です。	
\\	彼[かれ]の 意見[いけん]は 穏[おだ]やかだが、 上司[じょうし]の 意見[いけん]は 極端[きょくたん]です。	穏やか= 
\\	湖面はとても穏やかだったので、鏡のようだった。	
\\	湖面[こめん]はとても 穏[おだ]やかだったので、 鏡[かがみ]のようだった。	湖面=こめん= 
\\	穏やか= 
\\	思ったよりも喉が乾いてた。	
\\	思[おも]ったよりも 喉[のど]が 乾[かわ]いてた。	
\\	この花は長い間鮮やかに咲いています。	
\\	この 花[はな]は 長[なが]い 間[あいだ] 鮮[あざ]やかに 咲[さ]いています。	鮮やか= 
\\	私の故郷は海岸沿いにあり、そのため気候はいつも穏やかです。	
\\	私[わたし]の 故郷[こきょう]は 海岸[かいがん] 沿[ぞ]いにあり、そのため 気候[きこう]はいつも 穏[おだ]やかです。	穏やか= 
\\	鮮明な記憶力を持ってるの?	
\\	鮮明[せんめい]な 記憶[きおく] 力[りょく]を 持[も]ってるの?	鮮明= 
\\	彼がその話をとても鮮明に話したため、彼女は泣き出した。	
\\	彼[かれ]がその 話[はなし]をとても 鮮明[せんめい]に 話[はな]したため、 彼女[かのじょ]は 泣[な]き 出[だ]した。	鮮明= 
\\	私は祖父のことを鮮明に記憶している。	
\\	私[わたし]は 祖父[そふ]のことを 鮮明[せんめい]に 記憶[きおく]している。	鮮明= 
\\	第一印象はいつまでも鮮明に残る。	
\\	第[だい]一 印象[いんしょう]はいつまでも 鮮明[せんめい]に 残[のこ]る。	鮮明= 
\\	理論的な反論を見つけるのに苦労しているんです。	
\\	理論[りろん] 的[てき]な 反論[はんろん]を 見[み]つけるのに 苦労[くろう]しているんです。	
\\	ささやかなプレゼントですが、ご笑納ください。	
\\	ささやかなプレゼントですが、ご 笑納[しょうのう]ください。	笑納= 
\\	アルコールは速やかな睡眠を誘発する。	
\\	アルコールは 速[すみ]やかな 睡眠[すいみん]を 誘発[ゆうはつ]する。	速やか= 
\\	誘発= 
\\	インターネットは、膨大な量の情報への速やかなアクセスを提供する。	
\\	インターネットは、 膨大[ぼうだい]な 量[りょう]の 情報[じょうほう]への 速[すみ]やかなアクセスを 提供[ていきょう]する。	膨大= 
\\	速やか= 
\\	ローズとジョンの結婚式は、ささやかながら、威厳のあるものでした。	
\\	ローズとジョンの 結婚式[けっこんしき]は、ささやかながら、 威厳[いげん]のあるものでした。	威厳= 
\\	瞑想のおかげで彼女はより心穏やかな人物になった。	
\\	瞑想[めいそう]のおかげで 彼女[かのじょ]はより 心[こころ] 穏[おだ]やかな 人物[じんぶつ]になった。	瞑想=めいそう= 
\\	速やかなご回復を願っております。	
\\	速[すみ]やかなご 回復[かいふく]を 願[ねが]っております。	速やか= 
\\	あの子、なんかムッとしてるよな。どうしたんだろう?	
\\	あの 子[こ]、なんかムッとしてるよな。どうしたんだろう?	
\\	あれにはゾッとしたよ。	
\\	あれには ゾッ[ぞっ]としたよ。	
\\	この本はアッという間にベストセラーになった。	
\\	この 本[ほん]は アッという間[あっというま]にベストセラーになった。	
\\	この注射は少しチクッとします。	
\\	この 注射[ちゅうしゃ]は 少[すこ]しチクッとします。	注射= 
\\	チクッと= 
\\	これにザッと目を通してくれる?	
\\	これにザッと 目[め]を 通[とお]してくれる?	
\\	これをザッと振り返りましょう。	
\\	これをザッと 振り返[ふりかえ]りましょう。	
\\	その名前を聞くだけでゾッとするよ。	
\\	その 名前[なまえ]を 聞[き]くだけで ゾッ[ぞっ]とするよ。	
\\	その記事はゾッとするほど敵意に満ちていた。	
\\	その 記事[きじ]は ゾッ[ぞっ]とするほど 敵意[てきい]に 満[み]ちていた。	敵意= 
\\	もし彼女が好きならば、ズバッと言うしかない。	
\\	もし 彼女[かのじょ]が 好[す]きならば、ズバッと 言[い]うしかない。	スバッと= 
\\	アッという間に1週間が過ぎた。	
\\	アッという間[あっというま]に 1週間[いっしゅうかん]が 過[す]ぎた。	
\\	カッとなってしまったことを謝ります。	
\\	カッとなってしまったことを 謝[あやま]ります。	カッとなる= 
\\	カラッとしているなら暑い天気は大好きですが、湿気には参ります。	
\\	カラッとしているなら 暑[あつ]い 天気[てんき]は 大好[だいす]きですが、 湿気[しっけ]には 参[まい]ります。	カラッと= 
\\	湿気= 
\\	参る= 
\\	彼の家族の安否は依然として不明である。	
\\	彼[かれ]の 家族[かぞく]の 安否[あんぴ]は 依然[いぜん]として 不明[ふめい]である。	安否= 
\\	依然として= 
\\	大学入試の合否結果をまだ待っているところです。	
\\	大学[だいがく] 入試[にゅうし]の 合否[ごうひ] 結果[けっか]をまだ 待[ま]っているところです。	合否= 
\\	この素材はよく伸び縮みします。	
\\	この 素材[そざい]はよく 伸[の]び 縮[ちぢ]みします。	
\\	彼は、人に罪悪感を抱かせるのが上手です。	
\\	彼[かれ]は、 人[ひと]に 罪悪[ざいあく] 感[かん]を 抱[いだ]かせるのが 上手[じょうず]です。	罪悪感= 
\\	これは極秘です。漏れないようにしろ。	
\\	これは 極秘[ごくひ]です。 漏[も]れないようにしろ。	漏れる= 
\\	トイレが水漏れしています。	
\\	トイレが 水[みず] 漏[も]れしています。	水漏れ= 
\\	彼はその情報がどのように外部に漏れていたのか突き止めました。	
\\	彼[かれ]はその 情報[じょうほう]がどのように 外部[がいぶ]に 漏[も]れていたのか 突き止[つきと]めました。	外部= 
\\	漏れる= 
\\	突き止める= 
\\	私は燃料がタンクから漏れているのを見つけた。	
\\	私[わたし]は 燃料[ねんりょう]がタンクから 漏[も]れているのを 見[み]つけた。	燃料= 
\\	漏れる= 
\\	ガスの漏れる匂いがした。	
\\	ガスの 漏[も]れる 匂[にお]いがした。	漏れる= 
\\	クリスマスはアッという間にやってきます。	
\\	クリスマスは アッという間[あっというま]にやってきます。	やって来る= 
\\	ハッと息を呑んだ。	
\\	ハッと 息[いき]を 呑[の]んだ。	ハッと息を呑む= 
\\	今朝は、朝刊をチラッと見るぐらいの時間しかなかった。	
\\	今朝[けさ]は、 朝刊[ちょうかん]をチラッと 見[み]るぐらいの 時間[じかん]しかなかった。	朝刊= 
\\	外の空気はひんやりとしていて、私はブルッと震えた。	
\\	外[そと]の 空気[くうき]はひんやりとしていて、 私[わたし]はブルッと 震[ふる]えた。	ひんやり= 
\\	ブルッと= 
\\	震える= 
\\	工場をザッとご案内致します。	
\\	工場[こうじょう]をザッとご 案内[あんない] 致[いた]します。	
\\	彼はとても疲れていたので、ボーッとしていた。	
\\	彼[かれ]はとても 疲[つか]れていたので、ボーッとしていた。	
\\	彼は私をギュッと抱き締めました。	
\\	彼[かれ]は 私[わたし]をギュッと 抱き締[だきし]めました。	抱き締める= 
\\	彼女はその知らせを聞いてワッと泣き出した。	
\\	彼女[かのじょ]はその 知[し]らせを 聞[き]いてワッと 泣[な]き 出[だ]した。	
\\	私は彼を見たときビビビッと感じました。	
\\	私[わたし]は 彼[かれ]を 見[み]たときビビビッと 感[かん]じました。	
\\	空気が冷たくてピリッとしています。	
\\	空気[くうき]が 冷[つめ]たくてピリッとしています。	
\\	胃がギュッと締め付けられている。	
\\	胃[い]がギュッと 締め付[しめつ]けられている。	
\\	背筋がゾッとした。	
\\	背筋[せすじ]が ゾッ[ぞっ]とした。	
\\	顔に水をパシャッとかけた。	
\\	顔[かお]に 水[みず]をパシャッとかけた。	
\\	彼女は雑用をみんな済ませたとうそをついた。	
\\	彼女[かのじょ]は 雑用[ざつよう]をみんな 済[す]ませたとうそをついた。	雑用= 
\\	済ませる= 
\\	全般的に、私は彼の意見に賛成です。	
\\	全般[ぜんぱん] 的[てき]に、 私[わたし]は 彼[かれ]の 意見[いけん]に 賛成[さんせい]です。	
\\	全般的に見て、状況は改善してきている。	
\\	全般[ぜんぱん] 的[てき]に 見[み]て、 状況[じょうきょう]は 改善[かいぜん]してきている。	
\\	私はあなたに英語を教えてほしいです。	
\\	私[わたし]はあなたに 英語[えいご]を 教[おし]えてほしいです。	
\\	あなたは誰に来てほしいですか。	
\\	あなたは 誰[だれ]に 来[き]てほしいですか。	
\\	私は吉田先生に来てほしい。	
\\	私[わたし]は 吉田[よしだ] 先生[せんせい]に 来[き]てほしい。	
\\	アダムスさんはフランシスにこの仕事をしてもらいたがっている。	
\\	アダムスさんはフランシスにこの 仕事[しごと]をしてもらいたがっている。	
\\	私はあなたに英語を教えてもらいたい。	
\\	私[わたし]はあなたに 英語[えいご]を 教[おし]えてもらいたい。	
\\	私は吉田先生に来ていただきたい。	
\\	私[わたし]は 吉田[よしだ] 先生[せんせい]に 来[き]ていただきたい。	
\\	「〜来てほしい」
\\	「来てもらいたい」
\\	クラス(の中)で大川さんが一番頭がいい。	
\\	クラス(の 中[なか])で 大川[おおかわ]さんが 一番頭[いちばんあたま]がいい。	
\\	あなたにもその知らせは行きましたか。	
\\	あなたにもその 知[し]らせは 行[い]きましたか。	
\\	これからは毎日本を一冊読んで行くつもりです。	
\\	これからは 毎日[まいにち] 本[ほん]を 一冊[いっさつ] 読[よ]んで 行[い]くつもりです。	
\\	その頃から日本の経済は強くなって行った。	
\\	その 頃[ころ]から 日本[にほん]の 経済[けいざい]は 強[つよ]くなって 行[い]った。	
\\	分からないことをノートに書いて行った。	
\\	分[わ]からないことをノートに 書[か]いて 行[い]った。	
\\	これからは暖かくなって来ますよ。	
\\	これからは 暖[あたた]かくなって 来[き]ますよ。	
\\	来る 
\\	行く 
\\	夫の死を知らされたとき、彼女は悲しみのどん底に沈んだ。	
\\	夫[おっと]の 死[し]を 知[し]らされたとき、 彼女[かのじょ]は 悲[かな]しみのどん 底[ぞこ]に 沈[しず]んだ。	どん底= 
\\	沈む= 
\\	今は不景気のどん底だ。	
\\	今[いま]は 不景気[ふけいき]のどん 底[ぞこ]だ。	
\\	荒っぽい着陸になるかもしれません。	
\\	荒[あら]っぽい 着陸[ちゃくりく]になるかもしれません。	
\\	彼らの荒っぽい態度に私はおびえた。	
\\	彼[かれ]らの 荒[あら]っぽい 態度[たいど]に 私[わたし]はおびえた。	おびえる= 
\\	少年の父親は彼に荒々しい殴打を与えた。	
\\	少年[しょうねん]の 父親[ちちおや]は 彼[かれ]に 荒々[あらあら]しい 殴打[おうだ]を 与[あた]えた。	殴打= 
\\	この城は、持ち主が死んだ後、荒れ果てたままになった。	
\\	この 城[しろ]は、 持ち主[もちぬし]が 死[し]んだ 後[のち]、 荒れ果[あれは]てたままになった。	持ち主= 
\\	荒れ果てる= 
\\	彼の顔が怒りで紅潮しました。	
\\	彼[かれ]の 顔[かお]が 怒[いか]りで 紅潮[こうちょう]しました。	紅潮= 
\\	この子にはいい家庭教師がいる。	
\\	この 子[こ]にはいい 家庭[かてい] 教師[きょうし]がいる。	
\\	土田は幸子が自分を愛していることを知らなかった。	
\\	土田[つちた]は 幸子[さちこ]が 自分[じぶん]を 愛[あい]していることを 知[し]らなかった。	
\\	中川は自分が京大に入れると思っていなかった。	
\\	中川[なかがわ]は 自分[じぶん]が 京大[きょうだい]に 入[はい]れると 思[おも]っていなかった。	
\\	日本人は自分の国の文化をユニークだと思っている。	
\\	日本人[にほんじん]は 自分[じぶん]の 国[くに]の 文化[ぶんか]をユニークだと 思[おも]っている。	
\\	一男は自分を励ました。	
\\	一男[かずお]は 自分[じぶん]を 励[はげ]ました。	
\\	先生はご自分の家で私に合って下さった。	
\\	先生[せんせい]はご 自分[じぶん]の 家[いえ]で 私[わたし]に 合[あ]って 下[くだ]さった。	
\\	トムが行くかメアリーが行くかどちらかです。	
\\	トムが 行[い]くかメアリーが 行[い]くかどちらかです。	
\\	それはボブかマークがします。	
\\	それはボブかマークがします。	
\\	手紙を書くか電話をかけるかどちらかして下さい。	
\\	手紙[てがみ]を 書[か]くか 電話[でんわ]をかけるかどちらかして 下[くだ]さい。	
\\	私と一緒に来ますか。それともここにいますか。	
\\	私[わたし]と 一緒[いっしょ]に 来[き]ますか。それともここにいますか。	
\\	私はテリーにナンシーが日本へ行くかと聞いた。	
\\	私はテリーにナンシーが日本へ 行[い]くかと 聞[き]いた。	
\\	酒は米から作る。	
\\	酒[さけ]は 米[べい]から 作[つく]る。	
\\	つまらないことから喧嘩になった。	
\\	つまらないことから 喧嘩[けんか]になった。	
\\	土曜日に仕事をする代わりに月曜日は休みます。	
\\	土曜日[どようび]に 仕事[しごと]をする 代[か]わりに 月曜日[げつようび]は 休[やす]みます。	
\\	その車は安かった代わりによく故障した。	
\\	その 車[くるま]は 安[やす]かった 代[か]わりによく 故障[こしょう]した。	
\\	僕は耳が聞こえない。	
\\	僕[ぼく]は 耳[みみ]が 聞[き]こえない。	
\\	アメリカで日本への土産を買うとそれが日本製であることがよくある。	
\\	アメリカで 日本[にほん]への 土産[みやげ]を 買[か]うとそれが 日本[にほん] 製[せい]であることがよくある。	
\\	私は来年大阪に転勤することになりました。	
\\	私[わたし]は 来年[らいねん] 大阪[おおさか]に 転勤[てんきん]することになりました。	
\\	私は毎日三十分ぐらい運動をすることにしています。	
\\	私[わたし]は 毎日[まいにち] 三十分[さんじゅっぷん]ぐらい 運動[うんどう]をすることにしています。	
\\	毎日漢字を十覚えることにしました。	
\\	毎日[まいにち] 漢字[かんじ]を 十[とお] 覚[おぼ]えることにしました。	
\\	ジョーンズさんは日本語を話すことは話しますが、簡単なことしか言えません。	
\\	ジョーンズさんは 日本語[にほんご]を 話[はな]すことは 話[はな]しますが、 簡単[かんたん]なことしか 言[い]えません。	
\\	今日の試験は難しかったことは難しかったがよく出来た。	
\\	今日[きょう]の 試験[しけん]は 難[むずか]しかったことは 難[むずか]しかったがよく 出来[でき]た。	
\\	その女の子が好きだったことは好きでしたが、結婚はしなかったんです。	
\\	その 女の子[おんなのこ]が 好[す]きだったことは 好[す]きでしたが、 結婚[けっこん]はしなかったんです。	
\\	大川さんは (私に) 本をくれました。	
\\	大川[おおかわ]さんは
\\	私[わたし]に) 本[ほん]をくれました。	
\\	ビルは (君に) 何をくれましたか。	
\\	ビルは
\\	君[きみ]に) 何[なに]をくれましたか。	
\\	川村さんは私の娘にレコードをくれた。	
\\	川村[かわむら]さんは 私[わたし]の 娘[むすめ]にレコードをくれた。	
\\	彼らは人権侵害の例をいくつか挙げた。	
\\	彼[かれ]らは 人権[じんけん] 侵害[しんがい]の 例[れい]をいくつか 挙[あ]げた。	人権侵害= 
\\	その囚人たちの扱いは、目に余る人権侵害です。	
\\	その 囚人[しゅうじん]たちの 扱[あつか]いは、 目[め]に 余[あま]る 人権[じんけん] 侵害[しんがい]です。	囚人= 
\\	目に余る= 
\\	人権侵害= 
\\	陪審員は、彼が重大な人権侵害を犯したとして有罪を宣告しました。	
\\	陪審[ばいしん] 員[いん]は、 彼[かれ]が 重大[じゅうだい]な 人権[じんけん] 侵害[しんがい]を 犯[おか]したとして 有罪[ゆうざい]を 宣告[せんこく]しました。	人権侵害= 
\\	宣告= 
\\	彼女は事故以来、全く口をきかない。	
\\	彼女[かのじょ]は 事故[じこ] 以来[いらい]、 全[まった]く 口[くち]をきかない。	口をきく= 
\\	自分の行動を謝罪しない限り、私はあなたとは二度と口をきかない。	
\\	自分[じぶん]の 行動[こうどう]を 謝罪[しゃざい]しない 限[かぎ]り、 私[わたし]はあなたとは 二度[にど]と 口[くち]をきかない。	口をきく= 
\\	この章の根底にある理論はマルクスに基づいている。	
\\	この 章[しょう]の 根底[こんてい]にある 理論[りろん]はマルクスに 基[もと]づいている。	根底= 
\\	問題の根底は力の不均衡にある。	
\\	問題[もんだい]の 根底[こんてい]は 力[ちから]の 不[ふ] 均衡[きんこう]にある。	根底= 
\\	不均衡= 
\\	これは良いことだと心底信じています。	
\\	これは 良[よ]いことだと 心底[しんそこ] 信[しん]じています。	心底=しんそこ= 
\\	彼の死は私にとって心底ショックでした。	
\\	彼[かれ]の 死[し]は 私[わたし]にとって 心底[しんそこ]ショックでした。	心底=しんそこ= 
\\	彼は心底その職に就きたがっています。	
\\	彼[かれ]は 心底[しんそこ]その 職[しょく]に 就[つ]きたがっています。	心底=しんそこ= 
\\	ぴったりの言葉が思い浮かばない。	
\\	ぴったりの 言葉[ことば]が 思[おも]い 浮[う]かばない。	思い浮かぶ= 
\\	彼は情景をありありと思い浮かべることが出来ました。	
\\	彼[かれ]は 情景[じょうけい]をありありと 思い浮[おもいう]かべることが 出来[でき]ました。	情景= 
\\	ありあり= 
\\	思い浮かべる= 
\\	喫煙というとがんを思い浮かべる人は多い。	
\\	喫煙[きつえん]というとがんを 思い浮[おもいう]かべる 人[ひと]は 多[おお]い。	思い浮かべる= 
\\	世界平和に対する主要な脅威として、宗教テロが浮上している。	
\\	世界[せかい] 平和[へいわ]に 対[たい]する 主要[しゅよう]な 脅威[きょうい]として、 宗教[しゅうきょう]テロが 浮上[ふじょう]している。	脅威= 
\\	浮上= 
\\	彼はボクシング界の最有力人物として浮上していた。	
\\	彼[かれ]はボクシング 界[かい]の 最[さい] 有力[ゆうりょく] 人物[じんぶつ]として 浮上[ふじょう]していた。	浮上= 
\\	道男は私を慰めてくれた。	
\\	道男[みちお]は 私[わたし]を 慰[なぐさ]めてくれた。	慰める=なぐさめる= 
\\	母は(私に)ケーキを焼いてくれた。	
\\	母[はは]は
\\	私[わたし]に)ケーキを 焼[や]いてくれた。	
\\	先生は私に本を貸して下さった。	
\\	先生[せんせい]は 私[わたし]に 本[ほん]を 貸[か]して 下[くだ]さった。	
\\	私はコンピューターが少し分かって来た。	
\\	私[わたし]はコンピューターが 少[すこ]し 分[わ]かって 来[き]た。	
\\	私はいろいろ日本の歴史書を読んで来ました。	
\\	私[わたし]はいろいろ 日本[にほん]の 歴史[れきし] 書[しょ]を 読[よ]んで 来[き]ました。	
\\	テニスをしていたら急に雨が降って来た。	
\\	テニスをしていたら 急[きゅう]に 雨[あめ]が 降[ふ]って 来[き]た。	
\\	「~て来る」
\\	午後から頭が痛くなって来ました。	
\\	午後[ごご]から 頭[あたま]が 痛[いた]くなって 来[き]ました。	
\\	「~て来る」
\\	あの子はこの頃ずいぶんきれいになって来たね。	
\\	あの 子[こ]はこの 頃[ごろ]ずいぶんきれいになって 来[き]たね。	
\\	「~て来る」
\\	今までたくさん本を読んで来ましたが、これからも読んで行くつもりです。	
\\	今[いま]までたくさん 本[ほん]を 読[よ]んで 来[き]ましたが、これからも 読[よ]んで 行[い]くつもりです。	
\\	「~て来る」
\\	今まで遊んで来ましたが、これからは一生懸命勉強するつもりです。	
\\	今[いま]まで 遊[あそ]んで 来[き]ましたが、これからは 一生懸命[いっしょうけんめい] 勉強[べんきょう]するつもりです。	
\\	「~て来る」
\\	私が行くまで家で待っていてください。	
\\	私[わたし]が 行[い]くまで 家[いえ]で 待[ま]っていてください。	
\\	私は10時までに帰る。	
\\	私[わたし]は 10時[じゅうじ]までに 帰[かえ]る。	
\\	学校が始まるまでにこの本を読んでおいてください。	
\\	学校[がっこう]が 始[はじ]まるまでにこの 本[ほん]を 読[よ]んでおいてください。	
\\	この部屋は昨日のままです。	
\\	この 部屋[へや]は 昨日[きのう]のままです。	
\\	高山さんはアメリカへ行ったまま帰らなかった。	
\\	高山[たかやま]さんはアメリカへ 行[い]ったまま 帰[かえ]らなかった。	
\\	さようならの挨拶もしないまま行ってしまった。	
\\	さようならの 挨拶[あいさつ]もしないまま 行[おこな]ってしまった。	
\\	村田さんはコーヒーを飲もうと言った。	
\\	村田[むらた]さんはコーヒーを 飲[の]もうと 言[い]った。	
\\	学会には上田先生も見えた。	
\\	学会[がっかい]には 上田[うえだ] 先生[せんせい]も 見[み]えた。	
\\	「見える」
\\	来る
\\	山川さんは酒もタバコもやりません。	
\\	山川[やまかわ]さんは 酒[さけ]もタバコもやりません。	
\\	昔はよく映画を見たものだ。	
\\	昔[むかし]はよく 映画[えいが]を 見[み]たものだ。	
\\	もの 
\\	どうして食べないの? 
\\	だって、ますいもの。	
\\	どうして 食[た]べないの? 
\\	だって、ますいもの。	
\\	よくそんなばかなことをしたものだ!	
\\	よくそんなばかなことをしたものだ!	
\\	人の家に行く時はお土産を持って行くものです。	
\\	人[ひと]の 家[いえ]に 行[い]く 時[とき]はお 土産[みやげ]を 持[も]って 行[い]くものです。	
\\	もう帰って来るな!	
\\	もう 帰[かえ]って 来[く]るな!	
\\	酒をあまり飲むなよ。	
\\	酒[さけ]をあまり 飲[の]むなよ。	
\\	よ 
\\	ものを食べながら話してはいけません。	
\\	ものを 食[た]べながら 話[はな]してはいけません。	
\\	電話しないで欲しい。	
\\	電話[でんわ]しないで 欲[ほ]しい。	
\\	やかましくしないでもらいたい。	
\\	やかましくしないでもらいたい。	やかましい= 
\\	むかつくのは、あいつのきざったらしい態度です。	
\\	むかつくのは、あいつのきざったらしい 態度[たいど]です。	むかつく= 
\\	気障= 
\\	ったらしい= 
\\	あいつのメールの書き方にはマジでむかつく。	
\\	あいつのメールの 書き方[かきかた]にはマジでむかつく。	むかつく= 
\\	七面鳥を食べるといつも胃がむかつく。	
\\	七面鳥[しちめんちょう]を 食[た]べるといつも 胃[い]がむかつく。	七面鳥=しちめんちょう= 
\\	むかつく= 
\\	考えるだけでもむかつくよ。	
\\	考[かんが]えるだけでもむかつくよ。	むかつく= 
\\	その正反対が真実です。	
\\	その 正反対[せいはんたい]が 真実[しんじつ]です。	正反対= 
\\	ボブとメグはまるで正反対の性格です。	
\\	ボブとメグはまるで 正反対[せいはんたい]の 性格[せいかく]です。	正反対= 
\\	彼らは正反対でありながら、非常に似通っている。	
\\	彼[かれ]らは 正反対[せいはんたい]でありながら、 非常[ひじょう]に 似通[にかよ]っている。	正反対= 
\\	似通う= 
\\	彼は眉間にしわを寄せてレポートを読んでいる。	
\\	彼[かれ]は 眉間[みけん]にしわを 寄[よ]せてレポートを 読[よ]んでいる。	眉間= 
\\	しわを寄せる= 
\\	塾は日本の教育にしっかり定着している。	
\\	塾[じゅく]は日本の 教育[きょういく]にしっかり 定着[ていちゃく]している。	定着= 
\\	お時間を割いていただきましてどうもありがとうございました。	
\\	お 時間[じかん]を 割[さ]いていただきましてどうもありがとうございました。	時間を割く= 
\\	彼はどんなに忙しくても私に電話をする時間を割いてくれる。	
\\	彼[かれ]はどんなに 忙[いそが]しくても 私[わたし]に 電話[でんわ]をする 時間[じかん]を 割[さ]いてくれる。	時間を割く= 
\\	彼はわざわざ時間を割いて東京を案内してくれました。	
\\	彼[かれ]はわざわざ 時間[じかん]を 割[さ]いて 東京[とうきょう]を 案内[あんない]してくれました。	時間を割く= 
\\	年配の人たちの中にも、流行の服を着たいと思っている人がいる。	
\\	年配[ねんぱい]の 人[ひと]たちの 中[なか]にも、 流行[はやり]の 服[ふく]を 着[き]たいと 思[おも]っている 人[ひと]がいる。	年配= 
\\	当たらずといえども遠からずだ。	
\\	当[あ]たらずといえども 遠[とお]からずだ。	当たらずといえども遠からず= 
\\	私はこの仕事が好きになりかけています。	
\\	私[わたし]はこの 仕事[しごと]が 好[す]きになりかけています。	なりかける= 
\\	私はワインが好きになりかけています。	
\\	私[わたし]はワインが 好[す]きになりかけています。	なりかける= 
\\	彼の観察力には脱帽する。	
\\	彼[かれ]の 観察[かんさつ] 力[りょく]には 脱帽[だつぼう]する。	脱帽= 
\\	彼の社会貢献には、脱帽せざるを得ない。	
\\	彼[かれ]の 社会[しゃかい] 貢献[こうけん]には、 脱帽[だつぼう]せざるを 得[え]ない。	社会貢献= 
\\	脱帽= 
\\	彼女の表情は私に賛成していないことを暗示していた。	
\\	彼女[かのじょ]の 表情[ひょうじょう]は 私[わたし]に 賛成[さんせい]していないことを 暗示[あんじ]していた。	暗示= 
\\	あなたと競うつもりは、全くありません。	
\\	あなたと 競[きそ]うつもりは、 全[まった]くありません。	競う= 
\\	一人っ子は親からの愛情や注目を兄弟と競う必要がありません。	
\\	一人っ子[ひとりっこ]は 親[おや]からの 愛情[あいじょう]や 注目[ちゅうもく]を 兄弟[きょうだい]と 競[きそ]う 必要[ひつよう]がありません。	競う= 
\\	彼らは競い合ってばかりいた。	
\\	彼[かれ]らは 競い合[きそいあ]ってばかりいた。	競い合う= 
\\	明かりが薄暗くて何も見えない。	
\\	明[あ]かりが 薄暗[うすぐら]くて 何[なに]も 見[み]えない。	
\\	いつか、ヒンディー語を流ちょうに話せるようになりたいです。	
\\	いつか、ヒンディー 語[ご]を 流[りゅう]ちょうに 話[はな]せるようになりたいです。	流暢=りゅうちょう= 
\\	流ちょう}}
\\	ビルが他の学生と違う点は、フランス語を流ちょうに話せることだ。	
\\	ビルが 他[た]の 学生[がくせい]と 違[ちが]う 点[てん]は、 フランス語[ふらんすご]を 流[りゅう]ちょうに 話[はな]せることだ。	流暢=りゅうちょう= 
\\	流ちょう}}
\\	彼は英語を話す国には一度も行ったことがなかったが、英語が流ちょうだった。	
\\	彼[かれ]は 英語[えいご]を 話[はな]す 国[くに]には一 度[ど]も 行[おこな]ったことがなかったが、 英語[えいご]が 流[りゅう]ちょうだった。	流暢=りゅうちょう= 
\\	流ちょう}}
\\	その先生の親しみやすい態度は、生徒を安心させた。	
\\	その 先生[せんせい]の 親[した]しみやすい 態度[たいど]は、 生徒[せいと]を 安心[あんしん]させた。	親しみやすい= 
\\	その動物シェルターは1週間に約20匹の犬を安楽死させた。	
\\	その 動物[どうぶつ]シェルターは 1週間[いっしゅうかん]に 約[やく]20 匹[ぴき]の 犬[いぬ]を 安楽[あんらく] 死[し]させた。	安楽死= 
\\	その母親は娘に部屋を掃除させた。	
\\	その 母親[ははおや]は 娘[むすめ]に 部屋[へや]を 掃除[そうじ]させた。	
\\	政界で競り合うのは大変な仕事に違いない。	
\\	政界[せいかい]で 競り合[せりあ]うのは 大変[たいへん]な 仕事[しごと]に 違[ちが]いない。	競り合う= 
\\	換気扇をつけっ放しにするな。	
\\	換気扇[かんきせん]をつけっ 放[ぱな]しにするな。	換気扇= 
\\	その窓では十分な換気が出来ないかもしれない。	
\\	その 窓[まど]では 十分[じゅうぶん]な 換気[かんき]が 出来[でき]ないかもしれない。	換気= 
\\	お昼かなにかをごちそうしてくれるつもり?	
\\	お 昼[ひる]かなにかをごちそうしてくれるつもり?	
\\	本校の主要な目的は、生徒に幅広い教養を身に付けさせることです。	
\\	本校[ほんこう]の 主要[しゅよう]な 目的[もくてき]は、 生徒[せいと]に 幅広[はばひろ]い 教養[きょうよう]を 身[み]に 付[つ]けさせることです。	教養= 
\\	やさしい漢字も書けなくなった。	
\\	やさしい 漢字[かんじ]も 書[か]けなくなった。	
\\	前は酒をよく飲んでいたが、この頃は飲まなくなりました。	
\\	前[まえ]は 酒[さけ]をよく 飲[の]んでいたが、この 頃[ころ]は 飲[の]まなくなりました。	
\\	彼は前よく電話をかけて来ましたが、もうかけて来なくなりました。	
\\	彼[かれ]は 前[まえ]よく 電話[でんわ]をかけて 来[き]ましたが、もうかけて 来[こ]なくなりました。	
\\	日本語はもう難しくなくなりました。	
\\	日本語[にほんご]はもう 難[むずか]しくなくなりました。	
\\	この郊外も地下鉄が来て不便ではなくなった。	
\\	この 郊外[こうがい]も 地下鉄[ちかてつ]が 来[き]て 不便[ふべん]ではなくなった。	
\\	日本語が話せなくなった。	
\\	日本語[にほんご]が 話[はな]せなくなった。	
\\	日本語がもう話せない。	
\\	日本語[にほんご]がもう 話[はな]せない。	
\\	先生の説明が分からなくて困りました。	
\\	先生[せんせい]の 説明[せつめい]が 分[わ]からなくて 困[こま]りました。	
\\	試験は難しくなくてよかったですね。	
\\	試験[しけん]は 難[むずか]しくなくてよかったですね。	
\\	松田が来るのなら僕は行かない。	
\\	松田[まつだ]が 来[く]るのなら 僕[ぼく]は 行[い]かない。	
\\	食べろ!	
\\	食[た]べろ!	
\\	子供が学校に行っている間に手紙を書いた。	
\\	子供[こども]が 学校[がっこう]に 行[い]っている 間[あいだ]に 手紙[てがみ]を 書[か]いた。	
\\	田中は大学にいる時 (に) 今の奥さんと出会った。	
\\	田中[たなか]は 大学[だいがく]にいる 時[とき](に) 今[いま]の 奥[おく]さんと 出会[であ]った。	
\\	一男は友達に手紙を読まれた。	
\\	一男[かずお]は 友達[ともだち]に 手紙[てがみ]を 読[よ]まれた。	
\\	秋子は浩にご飯を作らせた。	
\\	秋子[あきこ]は 浩[ひろし]にご 飯[はん]を 作[つく]らせた。	
\\	オーバーはハンガーにかけてください。	
\\	オーバーはハンガーにかけてください。	
\\	下田さんは今日のことを忘れたに違いません。	
\\	下田[しもだ]さんは 今日[きょう]のことを 忘[わす]れたに 違[ちが]いません。	
\\	あの先生は話しにくいです。	
\\	あの 先生[せんせい]は 話[はな]しにくいです。	
\\	この靴は走りにくいです。	
\\	この 靴[くつ]は 走[はし]りにくいです。	
\\	高山さんは日本人にしては大きい。	
\\	高山[たかやま]さんは 日本人[にほんじん]にしては 大[おお]きい。	にしては= 
\\	ボブは日本語をよく勉強しているにしては下手だ。	
\\	ボブは 日本語[にほんご]をよく 勉強[べんきょう]しているにしては 下手[へた]だ。	にしては= 
\\	八月にしては涼しいですね。	
\\	八月[はちがつ]にしては 涼[すず]しいですね。	にしては= 
\\	青木さんはアメリカに十年いたにしては英語があまり上手じゃない。	
\\	青木[あおき]さんはアメリカに 十年[じゅうねん]いたにしては 英語[えいご]があまり 上手[じょうず]じゃない。	にしては= 
\\	石田先生は英語でお話しになりました。	
\\	石田[いしだ] 先生[せんせい]は 英語[えいご]でお 話[はな]しになりました。	
\\	お飲物は何がよろしいですか。	
\\	お 飲物[のみもの]は 何[なに]がよろしいですか。	
\\	私は五番街を歩いた。	
\\	私[わたし]は 五番[ごばん] 街[がい]を 歩[ある]いた。	
\\	公園を通って帰りましょう。	
\\	公園[こうえん]を 通[とお]って 帰[かえ]りましょう。	
\\	日本を離れて外国で暮らしている。	
\\	日本[にほん]を 離[はな]れて 外国[がいこく]で 暮[く]らしている。	
\\	次郎は父の死を悲しんだ。	
\\	次郎[じろう]は 父[ちち]の 死[し]を 悲[かな]しんだ。	「を」
\\	ヨーロッパ人はまた戦争が起きることを恐れている。	
\\	ヨーロッパ 人[じん]はまた 戦争[せんそう]が 起[お]きることを 恐[おそ]れている。	「を」
\\	信子は京都での一年を懐かしんだ。	
\\	信子[のぶこ]は 京都[きょうと]での 一年[いちねん]を 懐[なつ]かしんだ。	「を」
\\	田中先生はもうお帰りになりました。	
\\	田中[たなか] 先生[せんせい]はもうお 帰[かえ]りになりました。	
\\	私は先生のスーツケースをお持ちしました。	
\\	私[わたし]は 先生[せんせい]のスーツケースをお 持[も]ちしました。	
\\	やっと論文を書き終わった。	
\\	やっと 論文[ろんぶん]を 書[か]き 終[お]わった。	
\\	一郎は花子にだまされた。	
\\	一郎[いちろう]は 花子[はなこ]にだまされた。	
\\	ジェーンはフレッドに夜遅くアパートに来られた。	
\\	ジェーンはフレッドに 夜[よる] 遅[おそ]くアパートに 来[きた]られた。	
\\	この本は1965年にアメリカで出版された。	
\\	この 本[ほん]は1965 年[ねん]にアメリカで 出版[しゅっぱん]された。	
\\	私は2年前妻に死なれた。	
\\	私[わたし]は 2年[にねん] 前[まえ] 妻[つま]に 死[し]なれた。	
\\	メアリーはビルが好きじゃないらしい。	
\\	メアリーはビルが 好[す]きじゃないらしい。	「らしい」
\\	神様を信じますか。	
\\	神様[かみさま]を 信[しん]じますか。	
\\	せっかくいい大学に入ったのだからよく勉強するつもりです。	
\\	せっかくいい 大学[だいがく]に 入[はい]ったのだからよく 勉強[べんきょう]するつもりです。	せっかく= 
\\	せっかくの日曜日なのに働いた。	
\\	せっかくの 日曜日[にちようび]なのに 働[はたら]いた。	せっかく= 
\\	せっかくアメリカまで行ったのにニューヨークに行けなくて残念だった。	
\\	せっかくアメリカまで 行[い]ったのにニューヨークに 行[い]けなくて 残念[ざんねん]だった。	せっかく= 
\\	せっかくの旅行が病気でだめになりました。	
\\	せっかくの 旅行[りょこう]が 病気[びょうき]でだめになりました。	せっかく= 
\\	彼の財産がねたましかった。	
\\	彼[かれ]の 財産[ざいさん]がねたましかった。	ねたましい= 
\\	それは江口さんにしか話していません。	
\\	それは 江口[えぐち]さんにしか 話[はな]していません。	
\\	私はご飯をいっぱいしか食べなかった。	
\\	私[わたし]はご 飯[はん]をいっぱいしか 食[た]べなかった。	
\\	この本はこの図書館 (に) しかありません。	
\\	この 本[ほん]はこの 図書館[としょかん](に)しかありません。	
\\	子供たちをしっかり養育するために、両親は力を合わせた。	
\\	子供[こども]たちをしっかり 養育[よういく]するために、 両親[りょうしん]は 力[ちから]を 合[あ]わせた。	養育= 
\\	安定した家庭環境は子供の養育に不可欠です。	
\\	安定[あんてい]した 家庭[かてい] 環境[かんきょう]は 子供[こども]の 養育[よういく]に 不可欠[ふかけつ]です。	養育= 
\\	適切な衣服を身に着けてください。	
\\	適切[てきせつ]な 衣服[いふく]を 身[み]に 着[つ]けてください。	身に着ける= 
\\	池田君は三日でその本を読んでしまった。	
\\	池田[いけだ] 君[くん]は 三日[みっか]でその 本[ほん]を 読[よ]んでしまった。	
\\	私はルームメートのミルクを飲んでしまった。	
\\	私[わたし]はルームメートのミルクを 飲[の]んでしまった。	
\\	シチューを作りすぎてしまいました。	
\\	シチューを 作[つく]りすぎてしまいました。	
\\	清水さんはお酒を飲まないそうです。	
\\	清水[しみず]さんはお 酒[さけ]を 飲[の]まないそうです。	そうだ 
\\	日本の肉はとても高いそうだ。	
\\	日本[にほん]の 肉[にく]はとても 高[たか]いそうだ。	そうだ 
\\	利子さんは英語がとても上手だそうです。	
\\	利子[としこ]さんは 英語[えいご]がとても 上手[じょうず]だそうです。	そうだ 
\\	あの車は高そうです。	
\\	あの 車[くるま]は 高[たか]そうです。	そうだ 
\\	この家は強い風が吹いたら倒れそうだ。	
\\	この 家[いえ]は 強[つよ]い 風[かぜ]が 吹[ふ]いたら 倒[たお]れそうだ。	そうだ 
\\	この辺りは静かそうだ。	
\\	この 辺[あた]りは 静[しず]かそうだ。	そうだ 
\\	加藤さんは学生じゃなさそうだ。	
\\	加藤[かとう]さんは 学生[がくせい]じゃなさそうだ。	そうだ 
\\	僕はこのケーキを残しそうだ。	
\\	僕[ぼく]はこのケーキを 残[のこ]しそうだ。	そうだ 
\\	私はとても疲れていて倒れそうだ。	
\\	私[わたし]はとても 疲[つか]れていて 倒[たお]れそうだ。	そうだ 
\\	昨日は風邪を引きました。それで学校を休んだんです。	
\\	昨日[きのう]は 風邪[かぜ]を 引[ひ]きました。それで 学校[がっこう]を 休[やす]んだんです。	それで 
\\	ちょっと大阪で用事がありました。それで昨日いなかったんです。	
\\	ちょっと 大阪[おおさか]で 用事[ようじ]がありました。それで 昨日[きのう]いなかったんです。	それで 
\\	昨日は二時間ぐらい友達と飲んでそれからうちに帰った。	
\\	昨日[きのう]は 二時間[にじかん]ぐらい 友達[ともだち]と 飲[の]んでそれからうちに 帰[かえ]った。	
\\	山田は停学になった。	
\\	山田[やまだ]は 停学[ていがく]になった。	
\\	洋子は長い足をしている。	
\\	洋子[ようこ]は 長[なが]い 足[あし]をしている。	
\\	一男は丈夫な体をしています。	
\\	一男[かずお]は 丈夫[じょうぶ]な 体[からだ]をしています。	
\\	子供達の声がした。	
\\	子供[こども] 達[たち]の 声[こえ]がした。	
\\	私は寒気がします。	
\\	私[わたし]は 寒気[さむけ]がします。	
\\	この時計は十万円する。	
\\	この 時計[とけい]は 十万[じゅうまん] 円[えん]する。	
\\	あと一年したら大学を出ます。	
\\	あと 一年[いちねん]したら 大学[だいがく]を 出[で]ます。	
\\	今日は月曜日ですよ。 
\\	するとあのデパートは休みですね。	
\\	今日[きょう]は 月曜日[げつようび]ですよ。 
\\	するとあのデパートは 休[やす]みですね。	
\\	私は自転車を買いました。すると弟も欲しがりました。	
\\	私[わたし]は 自転車[じてんしゃ]を 買[か]いました。すると 弟[おとうと]も 欲[ほ]しがりました。	
\\	和男はとても行きたかった。	
\\	和男[かずお]はとても 行[い]きたかった。	~たい 
\\	一郎も行きたいと行っている。	
\\	一郎[いちろう]も 行[い]きたいと 行[おこな]っている。	~たい 
\\	利子は日本へ帰りたいそうだ。	
\\	利子[としこ]は 日本[にほん]へ 帰[かえ]りたいそうだ。	~たい 
\\	野村さんはあなたと話したいんですよ。	
\\	野村[のむら]さんはあなたと 話[はな]したいんですよ。	~たい 
\\	村山さんはのり子と踊りたいらしい。	
\\	村山[むらやま]さんはのり 子[こ]と 踊[おど]りたいらしい。	~たい 
\\	早田さんは早く家族に会いたそうだ。	
\\	早田[そうだ]さんは 早[はや]く 家族[かぞく]に 会[あ]いたそうだ。	~たい 
\\	ビクビクしないで。	
\\	ビクビクしないで。	ビクビク= 
\\	何をそんなにビクビクしているの?	
\\	何[なに]をそんなにビクビクしているの?	ビクビク= 
\\	あなたはいつもぼんやりしている。	
\\	あなたはいつもぼんやりしている。	
\\	将来についてはまだぼんやりした考えしか持っていない。	
\\	将来[しょうらい]についてはまだぼんやりした 考[かんが]えしか 持[も]っていない。	ぼんやり= (はっきりしないようす) 
\\	(放心したようす) 
\\	(注意力を欠いたようす); (無為に時を過ごすようす) 
\\	彼は睡眠薬で頭がぼんやりしていた。	
\\	彼[かれ]は 睡眠薬[すいみんやく]で 頭[あたま]がぼんやりしていた。	ぼんやり= (はっきりしないようす) 
\\	(放心したようす) 
\\	(注意力を欠いたようす); (無為に時を過ごすようす) 
\\	子供の頃のことはぼんやりとしか覚えていない。	
\\	子供[こども]の 頃[ころ]のことはぼんやりとしか 覚[おぼ]えていない。	ぼんやり= (はっきりしないようす) 
\\	(放心したようす) 
\\	(注意力を欠いたようす); (無為に時を過ごすようす) 
\\	彼はそこにぼんやり立っていた。	
\\	彼[かれ]はそこにぼんやり 立[た]っていた。	ぼんやり= (はっきりしないようす) 
\\	(放心したようす) 
\\	(注意力を欠いたようす); (無為に時を過ごすようす) 
\\	また計算が違ってるぞ、ぼんやりするな!	
\\	また 計算[けいさん]が 違[ちが]ってるぞ、ぼんやりするな!	ぼんやり= (はっきりしないようす) 
\\	(放心したようす) 
\\	(注意力を欠いたようす); (無為に時を過ごすようす) 
\\	ぼんやりしていて電車を間違えた。	
\\	ぼんやりしていて 電車[でんしゃ]を 間違[まちが]えた。	ぼんやり= (はっきりしないようす) 
\\	(放心したようす) 
\\	(注意力を欠いたようす); (無為に時を過ごすようす) 
\\	今日も一日ぼんやり過ごしてしまった。	
\\	今日[きょう]も 一日[いちにち]ぼんやり 過[す]ごしてしまった。	ぼんやり= (はっきりしないようす) 
\\	(放心したようす) 
\\	(注意力を欠いたようす); (無為に時を過ごすようす) 
\\	子供が親を探して泣いているのを彼はただぼんやり見ていた。	
\\	子供[こども]が 親[おや]を 探[さが]して 泣[な]いているのを 彼[かれ]はただぼんやり 見[み]ていた。	ぼんやり= (はっきりしないようす) 
\\	(放心したようす) 
\\	(注意力を欠いたようす); (無為に時を過ごすようす) 
\\	数学が嫌いでたまりません。	
\\	数学[すうがく]が 嫌[きら]いでたまりません。	
\\	この本は面白くてたまりません。	
\\	この 本[ほん]は 面白[おもしろ]くてたまりません。	
\\	親が甘かったため子供がだめになった。	
\\	親[おや]が 甘[あま]かったため(に) 子供[こども]がだめになった。	
\\	字が下手なため (に) 人に笑われた。	
\\	字[じ]が 下手[へた]なため(に) 人[ひと]に 笑[わら]われた。	
\\	フランスに行ったのは香水を買うためだ。	
\\	フランスに 行[い]ったのは 香水[こうすい]を 買[か]うためだ。	
\\	嫌いだったら残してください。	
\\	嫌[きら]いだったら 残[のこ]してください。	
\\	もっと日本の本を読んだらどうですか。	
\\	もっと 日本[にほん]の 本[ほん]を 読[よ]んだらどうですか。	
\\	サラダも食べたらどうですか。	
\\	サラダも 食[た]べたらどうですか。	
\\	山村先生に聞いたらどうですか。	
\\	山村[さんそん] 先生[せんせい]に 聞[き]いたらどうですか。	
\\	お金があったって車は買いたくない。	
\\	お 金[かね]があったって 車[くるま]は 買[か]いたくない。	たって 
\\	僕はその切符を高くたって買います。	
\\	僕[ぼく]はその 切符[きっぷ]を 高[たか]くたって 買[か]います。	たって 
\\	汚くたってかまいません。	
\\	汚[きたな]くたってかまいません。	たって 
\\	どんなにいい先生だって時々間違います。	
\\	どんなにいい 先生[せんせい]だって 時々[ときどき] 間違[まちが]います。	たって 
\\	子供達はバタバタと走り回った。	
\\	子供[こども] 達[たち]はバタバタと 走り回[はしりまわ]った。	
\\	ベンはむっつりと座っている。	
\\	ベンはむっつりと 座[すわ]っている。	
\\	学生だと割引があります。	
\\	学生[がくせい]だと 割引[わりびき]があります。	
\\	花子はじっと待っていた。	
\\	花子[はなこ]はじっと 待[ま]っていた。	
\\	もっと頑張らなくてはならないという気持ちがあります。	
\\	もっと 頑張[がんば]らなくてはならないという 気持[きも]ちがあります。	
\\	友達が今日来るということをすっかり忘れていた。	
\\	友達[ともだち]が 今日[きょう] 来[く]るということをすっかり 忘[わす]れていた。	
\\	これは出発のときに渡します。	
\\	これは 出発[しゅっぱつ]のときに 渡[わた]します。	
\\	私はもう少しで宿題を忘れるところだった。	
\\	私[わたし]はもう 少[すこ]しで 宿題[しゅくだい]を 忘[わす]れるところだった。	
\\	私は危ないところをジーンに助けてもらった。	
\\	私[わたし]は 危[あぶ]ないところをジーンに 助[たす]けてもらった。	
\\	お仕事中のところをすみません。	
\\	お 仕事[しごと] 中[ちゅう]のところをすみません。	
\\	僕は今出かけるところです。	
\\	僕[ぼく]は 今[いま] 出[で]かけるところです。	
\\	テリーと踊っているところをマーサに見られてしまった。	
\\	テリーと 踊[おど]っているところをマーサに 見[み]られてしまった。	
\\	田中さんはセールスマンとして採用されました。	
\\	田中[たなか]さんはセールスマンとして 採用[さいよう]されました。	
\\	この部屋は物置として使っている。	
\\	この 部屋[へや]は 物置[ものおき]として 使[つか]っている。	
\\	ジョンソンさんは日本語の一年生としては日本語が上手だ。	
\\	ジョンソンさんは 日本語[にほんご]の 一年生[いちねんせい]としては 日本語[にほんご]が 上手[じょうず]だ。	としては 
\\	これはお礼のつもりです。	
\\	これはお 礼[れい]のつもりです。	
\\	父はまだ若いつもりだ。	
\\	父[ちち]はまだ 若[わか]いつもりだ。	
\\	私はよく読んだつもりです。	
\\	私[わたし]はよく 読[よ]んだつもりです。	
\\	日本人ってよく写真を撮りますね。	
\\	日本人[にほんじん]ってよく 写真[しゃしん]を 撮[と]りますね。	
\\	外国で暮らすって難しいね。	
\\	外国[がいこく]で 暮[く]らすって 難[むずか]しいね。	
\\	ジェーンは踊らないって。	
\\	ジェーンは 踊[おど]らないって。	
\\	僕も行こうかって思いました。	
\\	僕[ぼく]も 行[い]こうかって 思[おも]いました。	
\\	あんな生意気なやつらの言うことを真に受けるな。	
\\	あんな 生意気[なまいき]なやつらの 言[い]うことを 真[しん]に 受[う]けるな。	生意気= 
\\	真に受ける= 
\\	聞いたことを真に受けてはいけない。	
\\	聞[き]いたことを 真[しん]に 受[う]けてはいけない。	真に受ける= 
\\	あいつなんかいつだってやっつけてやる。	
\\	あいつなんかいつだってやっつけてやる。	やっつける= 
\\	彼は厳しいスケジュールを抱えていても、友達のためならいつだって時間を作る。	
\\	彼[かれ]は 厳[きび]しいスケジュールを 抱[かか]えていても、 友達[ともだち]のためならいつだって 時間[じかん]を 作[つく]る。	
\\	この辺りには若者がエネルギーを発散するところがない。	
\\	この 辺[あた]りには 若者[わかもの]がエネルギーを 発散[はっさん]するところがない。	発散= 
\\	たまには発散しないとストレスがたまってしょうがない。	
\\	たまには 発散[はっさん]しないとストレスがたまってしょうがない。	発散= 
\\	警官は群衆を解散させた。	
\\	警官[けいかん]は 群衆[ぐんしゅう]を 解散[かいさん]させた。	群衆= 
\\	解散= 
\\	今日はこれで解散にしよう。	
\\	今日[きょう]はこれで 解散[かいさん]にしよう。	解散= 
\\	そのデモは解散させられた。	
\\	そのデモは 解散[かいさん]させられた。	
\\	私たちのバンドは、2回のコンサートの後、解散しました。	
\\	私[わたし]たちのバンドは、 2回[にかい]のコンサートの 後[あと]、 解散[かいさん]しました。	解散= 
\\	彼はそのプロジェクトの概略を話してくれました。	
\\	彼[かれ]はそのプロジェクトの 概略[がいりゃく]を 話[はな]してくれました。	概略= 
\\	私は水に関する問題の概略について簡単にお話しするつもりです。	
\\	私[わたし]は 水[みず]に 関[かん]する 問題[もんだい]の 概略[がいりゃく]について 簡単[かんたん]にお 話[はな]しするつもりです。	概略= 
\\	前田さんはアメリカにいるうちに英語が上手になりました。	
\\	前田[まえだ]さんはアメリカにいるうちに 英語[えいご]が 上手[じょうず]になりました。	
\\	何もしないうちに今年も終わった。	
\\	何[なに]もしないうちに 今年[ことし]も 終[お]わった。	
\\	走っているうちにおなかが痛くなった。	
\\	走[はし]っているうちにおなかが 痛[いた]くなった。	
\\	若いうちに本をたくさん読みなさい。	
\\	若[わか]いうちに 本[ほん]をたくさん 読[よ]みなさい。	
\\	福祉によってホームレス問題が解決するだろう。	
\\	福祉[ふくし]によってホームレス 問題[もんだい]が 解決[かいけつ]するだろう。	
\\	雨が降らないうちにテニスをして来ます。	
\\	雨[あめ]が 降[ふ]らないうちにテニスをして 来[き]ます。	
\\	忘れないうちに言っておきたいことがある。	
\\	忘[わす]れないうちに 言[い]っておきたいことがある。	
\\	働けるうちに出来るだけ働きたい。	
\\	働[はたら]けるうちに 出来[でき]るだけ 働[はたら]きたい。	
\\	温かいうちに飲んでください。	
\\	温[あたた]かいうちに 飲[の]んでください。	
\\	今度の試合はワシントン大学とだ。	
\\	今度[こんど]の 試合[しあい]はワシントン 大学[だいがく]とだ。	
\\	毎日三時間も日本語を勉強しているんですか。よく出来るわけですね。	
\\	毎日[まいにち] 三時間[さんじかん]も 日本語[にほんご]を 勉強[べんきょう]しているんですか。よく 出来[でき]るわけですね。	わけだ 
\\	明日試験ですか。じゃあ今晩忙しいわけですね。	
\\	明日[あした] 試験[しけん]ですか。じゃあ 今晩[こんばん] 忙[いそが]しいわけですね。	わけだ 
\\	スミスさんは十年間もテニスをしたのだから上手なわけだ。	
\\	スミスさんは 十年間[じゅうねんかん]もテニスをしたのだから 上手[じょうず]なわけだ。	わけだ 
\\	毎日プールで泳いでいるんですか。丈夫なわけですね。	
\\	毎日[まいにち]プールで 泳[およ]いでいるんですか。 丈夫[じょうぶ]なわけですね。	わけだ 
\\	昨日は三時間しか寝ていない。道理で眠いわけだ。	
\\	昨日[きのう]は 三時間[さんじかん]しか 寝[ね]ていない。 道理[どうり]で 眠[ねむ]いわけだ。	わけだ 
\\	えっ?足立さんが入院したんですか。パーティーに来なかったわけだ。	
\\	えっ? 足立[あだち]さんが 入院[にゅういん]したんですか。パーティーに 来[こ]なかったわけだ。	わけだ 
\\	父の言うことが分からないわけではないが、どうしても医者になりたくない。	
\\	父[ちち]の 言[い]うことが 分[わ]からないわけではないが、どうしても 医者[いしゃ]になりたくない。	わけだ 
\\	上田さんはボクシングが好きなようです。	
\\	上田[うえだ]さんはボクシングが 好[す]きなようです。	
\\	あの人は田中先生のようだ。	
\\	あの 人[ひと]は 田中[たなか] 先生[せんせい]のようだ。	
\\	石井さんはもう帰りましたか。 
\\	はい、そのようです。	
\\	石井[いしい]さんはもう 帰[かえ]りましたか。 
\\	はい、そのようです。	
\\	読めるように字をきれいに書いてください。	
\\	読[よ]めるように 字[じ]をきれいに 書[か]いてください。	
\\	僕が分かるようにスミスさんはゆっくり英語を話してくれた。	
\\	僕[ぼく]が 分[わ]かるようにスミスさんはゆっくり 英語[えいご]を 話[はな]してくれた。	
\\	遅れないようにタクシーで行きました。	
\\	遅[おく]れないようにタクシーで 行[い]きました。	
\\	子供が本を読むように面白そうな本を買って来た。	
\\	子供[こども]が 本[ほん]を 読[よ]むように 面白[おもしろ]そうな 本[ほん]を 買[か]って 来[き]た。	
\\	坂本さんは雪江に図書館の前で待っているように言った。	
\\	坂本[さかもと]さんは 雪江[ゆきえ]に 図書館[としょかん]の 前[まえ]で 待[ま]っているように 言[い]った。	
\\	パットは私と話さないようになった。	
\\	パットは 私[わたし]と 話[はな]さないようになった。	
\\	分からないことは先生に聞くようにしている。	
\\	分[わ]からないことは 先生[せんせい]に 聞[き]くようにしている。	
\\	メキシコは赤道より北にあります。	
\\	メキシコは 赤道[せきどう]より 北[きた]にあります。	
\\	3時より前に来てください。	
\\	3時[さんじ]より 前[まえ]に 来[き]てください。	
\\	八十点より上は合格です。	
\\	八十点[はちじゅってん]より 上[うえ]は 合格[ごうかく]です。	
\\	私はもう酒を飲むまいと思います。	
\\	私[わたし]はもう 酒[さけ]を 飲[の]むまいと 思[おも]います。	
\\	僕はあの人とはもう話すまいと思う。	
\\	僕[ぼく]はあの 人[ひと]とはもう 話[はな]すまいと 思[おも]う。	
\\	私は漢字を毎日五つずつ覚えます。	
\\	私[わたし]は 漢字[かんじ]を 毎日[まいにち] 五[いつ]つずつ 覚[おぼ]えます。	
\\	私は子供達に本を二冊ずつやった。	
\\	私[わたし]は 子供[こども] 達[たち]に 本[ほん]を 二冊[にさつ]ずつやった。	
\\	私は中学に入った時に父が買ってくれた小さな辞書をまだ使っている。	
\\	私[わたし]は 中学[ちゅうがく]に 入[はい]った 時[とき]に 父[ちち]が 買[か]ってくれた 小[ちい]さな 辞書[じしょ]をまだ 使[つか]っている。	
\\	ある意味では、非常に道理にかなっている。	
\\	ある 意味[いみ]では、 非常[ひじょう]に 道理[どうり]にかなっている。	道理にかなう= 
\\	私は、これが道理にかなっているのかどうか分からない。	
\\	私[わたし]は、これが 道理[どうり]にかなっているのかどうか 分[わ]からない。	道理にかなう= 
\\	人に恵んでもらうほど落ちぶれてはいない。	
\\	人[ひと]に 恵[めぐ]んでもらうほど 落[お]ちぶれてはいない。	恵む= 
\\	落ちぶれる= 
\\	この店舗の売り上げはたった1年間で7倍になった。	
\\	この 店舗[てんぽ]の 売り上[うりあ]げはたった 1年間[いちねんかん]で 7倍[ななばい]になった。	店舗=てんぽ= 
\\	レモンとライムは大抵のレシピで代替可能だ。	
\\	レモンとライムは 大抵[たいてい]のレシピで 代替[だいたい] 可能[かのう]だ。	代替=だいたい= 
\\	極貧の男は自殺を決意しました。	
\\	極貧[ごくひん]の 男[おとこ]は 自殺[じさつ]を 決意[けつい]しました。	極貧= 
\\	彼女は、妊娠中絶手術を受けることを決意しました。	
\\	彼女[かのじょ]は、 妊娠[にんしん] 中絶[ちゅうぜつ] 手術[しゅじゅつ]を 受[う]けることを 決意[けつい]しました。	妊娠中絶= 
\\	その領土は敵によって侵略された。	
\\	その 領土[りょうど]は 敵[てき]によって 侵略[しんりゃく]された。	領土= 
\\	侵略= 
\\	彼は略式の服装で宴会に現れた。	
\\	彼[かれ]は 略式[りゃくしき]の 服装[ふくそう]で 宴会[えんかい]に 現[あらわ]れた。	宴会= 
\\	彼の講義は退屈極まりない。	
\\	彼[かれ]の 講義[こうぎ]は 退屈[たいくつ] 極[きわ]まりない。	極まりない= 
\\	彼はできるだけ多くの女性と寝ようという淫乱極まりない男です。	
\\	彼[かれ]はできるだけ 多[おお]くの 女性[じょせい]と 寝[ね]ようという 淫乱[いんらん] 極[きわ]まりない 男[おとこ]です。	淫乱=いんらん= 
\\	極まりない= 
\\	彼の話の際どさに私達はハラハラしながら聞いていた。	
\\	彼[かれ]の 話[はなし]の 際[きわ]どさに 私[わたし] 達[たち]はハラハラしながら 聞[き]いていた。	際どい= 
\\	ウィンブルドンの決勝では 
\\	で際どく勝った。	
\\	ウィンブルドンの 決勝[けっしょう]では 
\\	2で 際[きわ]どく 勝[か]った。	決勝=けっしょう= 
\\	際どい= 
\\	際どかったな。もう少しで引かれるところだった。	
\\	際[きわ]どかったな。もう 少[すこ]しで 引[ひ]かれるところだった。	際どい= 
\\	もう少しで赤字企業に転落という際どい状況にある。	
\\	もう 少[すこ]しで 赤字[あかじ] 企業[きぎょう]に 転落[てんらく]という 際[きわ]どい 状況[じょうきょう]にある。	転落=てんらく= 
\\	際どい= 
\\	この機械は操作がややこしい。	
\\	この 機械[きかい]は 操作[そうさ]がややこしい。	操作=そうさ= 
\\	ややこしい= 
\\	この問題は少しややこしい。	
\\	この 問題[もんだい]は 少[すこ]しややこしい。	ややこしい= 
\\	彼は何をやってもだらしない。	
\\	彼[かれ]は 何[なに]をやってもだらしない。	だらしない= 
\\	あの男は酒を飲むと急にだらしなくなる。	
\\	あの 男[おとこ]は 酒[さけ]を 飲[の]むと 急[きゅう]にだらしなくなる。	だらしない= 
\\	なんてずうずうしいやつだ。	
\\	なんてずうずうしいやつだ。	ずうずうしい= 
\\	あいつのずうずうしさにはあきれる。	
\\	あいつのずうずうしさにはあきれる。	ずうずうしい= 
\\	すさまじい音を立てて爆発した。	
\\	すさまじい 音[おと]を 立[た]てて 爆発[ばくはつ]した。	凄まじい=すさまじい= 恐ろしい/
\\	ものすごい/
\\	この本の売れ行きは実にすさまじい。	
\\	この 本[ほん]の 売れ行[うれゆ]きは 実[じつ]にすさまじい。	売れ行き=うれゆき= 
\\	凄まじい=すさまじい= 恐ろしい/
\\	ものすごい/
\\	外国語も知らないで海外旅行とはすさまじい。	
\\	外国[がいこく] 語[ご]も 知[し]らないで 海外[かいがい] 旅行[りょこう]とはすさまじい。	凄まじい=すさまじい= 恐ろしい/
\\	ものすごい/
\\	この国は末期患者に対する医療が著しく立ち後れている。	
\\	この 国[くに]は 末期[まっき] 患者[かんじゃ]に 対[たい]する 医療[いりょう]が 著[いちじる]しく 立ち後[たちおく]れている。	末期患者=まっきかんじゃ= 
\\	著しい=いちじるしい= (はっきりしている) 
\\	(程度が大きい) 
\\	この業界の給与水準は一般に比べて著しく高い。	
\\	この 業界[ぎょうかい]の 給与[きゅうよ] 水準[すいじゅん]は 一般[いっぱん]に 比[くら]べて 著[いちじる]しく 高[たか]い。	著しい=いちじるしい= (はっきりしている) 
\\	(程度が大きい) 
\\	若い人たちの政治離れが近年ますます著しい。	
\\	若[わか]い 人[ひと]たちの 政治[せいじ] 離[ばな]れが 近年[きんねん]ますます 著[いちじる]しい。	著しい=いちじるしい= (はっきりしている) 
\\	(程度が大きい) 
\\	島によって方言の差が著しい。	
\\	島[しま]によって 方言[ほうげん]の 差[さ]が 著[いちじる]しい。	著しい=いちじるしい= (はっきりしている) 
\\	(程度が大きい) 
\\	最近では女性の喫煙率の伸びが著しい。	
\\	最近[さいきん]では 女性[じょせい]の 喫煙[きつえん] 率[りつ]の 伸[の]びが 著[いちじる]しい。	著しい=いちじるしい= (はっきりしている) 
\\	(程度が大きい) 
\\	喫煙は望ましいことではない。	
\\	喫煙[きつえん]は 望[のぞ]ましいことではない。	望ましい= 
\\	強制ではないが全員参加が望ましい。	
\\	強制[きょうせい]ではないが 全員[ぜんいん] 参加[さんか]が 望[のぞ]ましい。	強制= 
\\	望ましい= 
\\	あなたは年を取って骨がもろくなっている。	
\\	あなたは 年[とし]を 取[と]って 骨[ほね]がもろくなっている。	脆い= 
\\	過保護は子供を精神的にもろくする。	
\\	過[か] 保護[ほご]は 子供[こども]を 精神[せいしん] 的[てき]にもろくする。	脆い= 
\\	ひもじくて死にそうだ。	
\\	ひもじくて 死[し]にそうだ。	ひもじい= 
\\	日本は面積ではドイツにほぼ等しい。	
\\	日本は 面積[めんせき]ではドイツにほぼ 等[ひと]しい。	
\\	彼の要求はほとんど脅迫に等しかった。	
\\	彼[かれ]の 要求[ようきゅう]はほとんど 脅迫[きょうはく]に 等[ひと]しかった。	脅迫=きょうはく= 
\\	末子は特にいとおしいものだ。	
\\	末子[まっし]は 特[とく]にいとおしいものだ。	末子=ばっし・まっし= 
\\	愛おしい=いとおしい= 
\\	かわいい; 
\\	その手続きは煩わしかった。	
\\	その 手続[てつづ]きは 煩[わずら]わしかった。	煩わしい=わずらわしい= 
\\	年を取るとともに人との付き合いがだんだん煩わしくなる。	
\\	年[とし]を 取[と]るとともに 人[ひと]との 付き合[つきあ]いがだんだん 煩[わずら]わしくなる。	煩わしい=わずらわしい= 
\\	この紛争の原因は根深いところにある。	
\\	この 紛争[ふんそう]の 原因[げんいん]は 根深[ねぶか]いところにある。	根深い=ねぶかい= 
\\	私はこの世で人命以上に尊いものはないと思っている。	
\\	私[わたし]はこの 世[よ]で 人命[じんめい] 以上[いじょう]に 尊[とうと]いものはないと 思[おも]っている。	尊い・貴い=とうとい= 
\\	そんな心細くなるようなことを言うな。	
\\	そんな 心細[こころぼそ]くなるようなことを 言[い]うな。	心細い= 
\\	我が社の将来を考えると心細くなる。	
\\	我[わ]が 社[しゃ]の 将来[しょうらい]を 考[かんが]えると 心細[こころぼそ]くなる。	心細い= 
\\	身寄りもなく心細い身の上です。	
\\	身寄[みよ]りもなく 心細[こころぼそ]い 身の上[みのうえ]です。	身寄り=みより= 
\\	心細い= 
\\	身の上= 
\\	親しい友もなく私はとても行く末が心細い。	
\\	親[した]しい 友[とも]もなく 私[わたし]はとても 行く末[ゆくすえ]が 心細[こころぼそ]い。	行く末=ゆくすえ= 
\\	心細い= 
\\	彼のアイデアには、何かしら使えるはずの良いものがあると思う。	
\\	彼[かれ]のアイデアには、 何[なに]かしら 使[つか]えるはずの 良[よ]いものがあると 思[おも]う。	何かしら= 
\\	これ以上聞くのは彼を余計に苦しめることになる。	
\\	これ 以上[いじょう] 聞[き]くのは 彼[かれ]を 余計[よけい]に 苦[くる]しめることになる。	
\\	今これをしておかなかったら後で大変なことになる。	
\\	今[いま]これをしておかなかったら 後[あと]で 大変[たいへん]なことになる。	
\\	今怠けていると試験の時ひどいことになるよ。	
\\	今[いま] 怠[なま]けていると 試験[しけん]の 時[とき]ひどいことになるよ。	
\\	すべてがうまくいけば来年卒業ということになります。	
\\	すべてがうまくいけば 来年[らいねん] 卒業[そつぎょう]ということになります。	
\\	葉書が戻って来たということは彼はもうこの住所には住んでいないことになる。	
\\	葉書[はがき]が 戻[もど]って 来[き]たということは 彼[かれ]はもうこの 住所[じゅうしょ]には 住[す]んでいないことになる。	
\\	今回の失敗は事前の話し合いが不十分だったことによる。	
\\	今回[こんかい]の 失敗[しっぱい]は 事前[じぜん]の 話し合[はなしあ]いが 不十分[ふじゅうぶん]だったことによる。	ことによる 
\\	消極的な姿勢は、あなたのキャリアアップを妨げるだろう。	
\\	消極[しょうきょく] 的[てき]な 姿勢[しせい]は、あなたのキャリアアップを 妨[さまた]げるだろう。	消極的= 
\\	原則として指定席券の払い戻しはできません。	
\\	原則[げんそく]として 指定[してい] 席[せき] 券[けん]の 払い戻[はらいもど]しはできません。	原則= 
\\	原則として図書の貸し出しは2週間5冊まで。	
\\	原則[げんそく]として 図書[としょ]の 貸し出[かしだ]しは 2週間[にしゅうかん] 5冊[ごさつ]まで。	原則= 
\\	原則として臓器移植のドナーは常に匿名である。	
\\	原則[げんそく]として 臓器[ぞうき] 移植[いしょく]のドナーは 常[つね]に 匿名[とくめい]である。	
\\	原則には例外がつきものだ。	
\\	原則[げんそく]には 例外[れいがい]がつきものだ。	原則= 
\\	表現の自由は民主主義の基本原則だ。	
\\	表現[ひょうげん]の 自由[じゆう]は 民主[みんしゅ] 主義[しゅぎ]の 基本[きほん] 原則[げんそく]だ。	原則= 
\\	この施設は、地元の困窮者を援助するために設立された。	
\\	この 施設[しせつ]は、 地元[じもと]の 困窮[こんきゅう] 者[しゃ]を 援助[えんじょ]するために 設立[せつりつ]された。	困窮者= 
\\	彼の英語は変則だ。	
\\	彼[かれ]の 英語[えいご]は 変則[へんそく]だ。	変則= 
\\	あいつにあんな言い方されると不愉快です。	
\\	あいつにあんな 言い方[いいかた]されると 不愉快[ふゆかい]です。	不愉快= 
\\	私は、彼ほど不愉快な人に出会ったことがない。	
\\	私[わたし]は、 彼[かれ]ほど 不愉快[ふゆかい]な 人[ひと]に 出会[であ]ったことがない。	不愉快= 
\\	20ドル札でお釣りをもらえますか?	
\\	20ドル 札[さつ]でお 釣[つ]りをもらえますか?	お釣り= 
\\	小銭で2000円持っている。	
\\	小銭[こぜに]で2000 円[えん] 持[も]っている。	小銭=こぜに= 
\\	小銭を持っていないのですが。	
\\	小銭[こぜに]を 持[も]っていないのですが。	小銭=こぜに= 
\\	あなた、セーターを裏返しに着ていますよ。	
\\	あなた、セーターを 裏返[うらがえ]しに 着[き]ていますよ。	裏返し= 
\\	理沙が友達に三人バラの花束をあげた。	
\\	理沙[りさ]が 友達[ともだち]に 三人[さんにん]バラの 花束[はなたば]をあげた。	花束=はなたば= 
\\	妊娠中絶は、その国で最も論議を呼んでいる話題の一つである。	
\\	妊娠[にんしん] 中絶[ちゅうぜつ]は、その 国[くに]で 最[もっと]も 論議[ろんぎ]を 呼[よ]んでいる 話題[わだい]の 一[ひと]つである。	妊娠中絶= 
\\	心にわだかまりがある。	
\\	心[こころ]にわだかまりがある。	わだかまり= 
\\	現代芸術に対する人々の反応は、多岐にわたる。	
\\	現代[げんだい] 芸術[げいじゅつ]に 対[たい]する 人々[ひとびと]の 反応[はんのう]は、 多岐[たき]にわたる。	多岐にわたる= 
\\	彼女の著作は政治・文化・教育など多岐に及んでいる。	
\\	彼女[かのじょ]の 著作[ちょさく]は 政治[せいじ]・ 文化[ぶんか]・ 教育[きょういく]など 多岐[たき]に 及[およ]んでいる。	著作=ちょさく= 
\\	多岐=たき= 
\\	彼は自分の主張が正しいことを論証した。	
\\	彼[かれ]は 自分[じぶん]の 主張[しゅちょう]が 正[ただ]しいことを 論証[ろんしょう]した。	論証= 
\\	そんなのは全くのデタラメだ。	
\\	そんなのは 全[まった]くのデタラメだ。	デタラメ= 
\\	その新聞は4月ばかの日にデタラメのニュース記事を載せた。	
\\	その 新聞[しんぶん]は 4月[しがつ]ばかの 日[ひ]にデタラメのニュース 記事[きじ]を 載[の]せた。	デタラメ= 
\\	何らかの抜本的な対策が取られるべきです。	
\\	何[なん]らかの 抜本[ばっぽん] 的[てき]な 対策[たいさく]が 取[と]られるべきです。	何らか=なんらか= 
\\	抜本的= 
\\	彼は、エイズ患者を取り巻く沈黙の壁を破ろうと努めてきた。	
\\	彼[かれ]は、エイズ 患者[かんじゃ]を 取り巻[とりま]く 沈黙[ちんもく]の 壁[かべ]を 破[やぶ]ろうと 努[つと]めてきた。	取り巻く= 
\\	沈黙=ちんもく= 
\\	彼は孫たちに取り巻かれて幸福に暮らしている。	
\\	彼[かれ]は 孫[まご]たちに 取り巻[とりま]かれて 幸福[こうふく]に 暮[く]らしている。	取り巻く= 
\\	部活動が若いエネルギーの格好のはけ口になっている。	
\\	部[ぶ] 活動[かつどう]が 若[わか]いエネルギーの 格好[かっこう]のはけ 口[ぐち]になっている。	はけ口= 
\\	彼女は夫への不満のはけ口を娘に求めた。	
\\	彼女[かのじょ]は 夫[おっと]への 不満[ふまん]のはけ 口[ぐち]を 娘[むすめ]に 求[もと]めた。	はけ口= 
\\	彼はネットの書き込みに愚痴のはけ口を見出した。	
\\	彼[かれ]はネットの 書き込[かきこ]みに 愚痴[ぐち]のはけ 口[ぐち]を 見出[みいだ]した。	愚痴=ぐち= 
\\	はけ口= 
\\	人々は時々遅いインターネット接続に欲求不満になる。	
\\	人々[ひとびと]は 時々[ときどき] 遅[おそ]いインターネット 接続[せつぞく]に 欲求[よっきゅう] 不満[ふまん]になる。	欲求不満= 
\\	その行事の主な趣旨は何ですか?	
\\	その 行事[ぎょうじ]の 主[おも]な 趣旨[しゅし]は 何[なに]ですか?	行事=ぎょうじ= 
\\	趣旨=しゅし= 
\\	考えさせられる話だ。	
\\	考[かんが]えさせられる 話[はなし]だ。	
\\	あなたの言葉には考えさせられた。	
\\	あなたの 言葉[ことば]には 考[かんが]えさせられた。	
\\	この本は、一言で言えば「考えさせる」。	
\\	この 本[ほん]は、 一言[ひとこと]で 言[い]えば
\\	考[かんが]えさせる」。	
\\	彼は通りで凍え死んだ。	
\\	彼[かれ]は 通[とお]りで 凍[こご]え 死[し]んだ。	凍え死ぬ=こごえしぬ= 
\\	刑務所にいたことがある人々は雇用に適さないと見なされることがある。	
\\	刑務所[けいむしょ]にいたことがある 人々[ひとびと]は 雇用[こよう]に 適[てき]さないと 見[み]なされることがある。	適さない= 
\\	その囚人は3年後に刑務所を出所しました。	
\\	その 囚人[しゅうじん]は3 年[ねん] 後[ご]に 刑務所[けいむしょ]を 出所[でどころ]しました。	出所=でどころ= 
\\	彼らは対テロ戦争の一環として国民へのスパイ行為を続けました。	
\\	彼[かれ]らは 対[たい]テロ 戦争[せんそう]の 一環[いっかん]として 国民[こくみん]へのスパイ 行為[こうい]を 続[つづ]けました。	対テロ戦争= 
\\	日光浴もこの病気の治療の一環である。	
\\	日光浴[にっこうよく]もこの 病気[びょうき]の 治療[ちりょう]の 一環[いっかん]である。	日光浴=にっこうよく= 
\\	防災訓練の一環として消火訓練が行われた。	
\\	防災[ぼうさい] 訓練[くんれん]の 一環[いっかん]として 消火[しょうか] 訓練[くんれん]が 行[おこな]われた。	
\\	会社の経営方針が一貫していない。	
\\	会社[かいしゃ]の 経営[けいえい] 方針[ほうしん]が 一貫[いっかん]していない。	一貫= 
\\	うまく溶け込めるといいね。	
\\	うまく 溶[と]け 込[こ]めるといいね。	溶け込む= 
\\	子どもたちは直に新しい言葉や文化に溶け込みました。	
\\	子[こ]どもたちは 直[じか]に 新[あたら]しい 言葉[ことば]や 文化[ぶんか]に 溶け込[とけこ]みました。	直に=じきに= 
\\	溶け込む= 
\\	皆さん、これは無意味な議論です。	
\\	皆[みな]さん、これは 無意味[むいみ]な 議論[ぎろん]です。	
\\	金は気になるわけではないということを肝に銘じておきなさい。	
\\	金[きん]は 気[き]になるわけではないということを 肝[きも]に 銘[めい]じておきなさい。	肝に銘じる= 
\\	全くの早合点だった。	
\\	全[まった]くの 早合点[はやがてん]だった。	早合点=はやがてん= 
\\	その理論は事実とよく合致します。	
\\	その 理論[りろん]は 事実[じじつ]とよく 合致[がっち]します。	合致=がっち= 
\\	オリンピック大会はその本来の精神と合致しなくなっている。	
\\	オリンピック 大会[たいかい]はその 本来[ほんらい]の 精神[せいしん]と 合致[がっち]しなくなっている。	合致=がっち= 
\\	私の意図したところは達成された。	
\\	私[わたし]の 意図[いと]したところは 達成[たっせい]された。	意図=いと= 
\\	これは彼の意図した結果と違う。	
\\	これは 彼[かれ]の 意図[いと]した 結果[けっか]と 違[ちが]う。	意図=いと= 
\\	彼の沈黙にはどんな意図が隠されているのか。	
\\	彼[かれ]の 沈黙[ちんもく]にはどんな 意図[いと]が 隠[かく]されているのか。	意図=いと= 
\\	その夜食事に誘った彼の意図が彼女にはすぐ分かった。	
\\	その 夜[よる] 食事[しょくじ]に 誘[さそ]った 彼[かれ]の 意図[いと]が 彼女[かのじょ]にはすぐ 分[わ]かった。	意図=いと= 
\\	あなたみたいな男前になろうとしてる。	
\\	あなたみたいな 男前[おとこまえ]になろうとしてる。	男前=おとこまえ= 
\\	いろいろなコースの中から好きなものを取捨選択できる。	
\\	いろいろなコースの 中[なか]から 好[す]きなものを 取捨選択[しゅしゃせんたく]できる。	取捨選択= 
\\	彼女は無口だから話の糸口を見つけるのが大変だ。	
\\	彼女[かのじょ]は 無口[むくち]だから 話[はなし]の 糸口[いとぐち]を 見[み]つけるのが 大変[たいへん]だ。	糸口= 
\\	話の糸口を見つけようと、まず相手の会社についてちょっと尋ねてみました。	
\\	話[はなし]の 糸口[いとぐち]を 見[み]つけようと、まず 相手[あいて]の 会社[かいしゃ]についてちょっと 尋[たず]ねてみました。	糸口= 
\\	まだ偏見はたくさん残っている。	
\\	まだ 偏見[へんけん]はたくさん 残[のこ]っている。	
\\	あの役者は人気が実力に先行している。	
\\	あの 役者[やくしゃ]は 人気[にんき]が 実力[じつりょく]に 先行[せんこう]している。	役者=やくしゃ= 
\\	先行=せんこう= 
\\	彼女は上流志向だから子どもを名門小学校に入学させたがっている。	
\\	彼女[かのじょ]は 上流[じょうりゅう] 志向[しこう]だから 子[こ]どもを 名門[めいもん] 小学校[しょうがっこう]に 入学[にゅうがく]させたがっている。	志向=しこう= 
\\	名門=めいもん= 
\\	彼は上昇志向が強い。	
\\	彼[かれ]は 上昇[じょうしょう] 志向[しこう]が 強[つよ]い。	志向=しこう= 
\\	彼の現在の成功は平生の努力を怠らなかったことによる。	
\\	彼[かれ]の 現在[げんざい]の 成功[せいこう]は 平生[へいぜい]の 努力[どりょく]を 怠[おこた]らなかったことによる。	平生=へいぜい=普段。いつも。常日頃。
\\	彼らの離婚は二人の価値観があまりに違っていたことによる。	
\\	彼[かれ]らの 離婚[りこん]は二 人[にん]の 価値[かち] 観[かん]があまりに 違[ちが]っていたことによる。	
\\	彼の才能がこれ程までに開発されたのは鈴木氏に師事したことによる。	
\\	彼[かれ]の 才能[さいのう]がこれ 程[ほど]までに 開発[かいはつ]されたのは 鈴木[すずき] 氏[し]に 師事[しじ]したことによる。	師事=しじ=師として尊敬し、教えを受けること。
\\	彼がパーティーに来なかったのは忙しかったためだ。	
\\	彼[かれ]がパーティーに 来[こ]なかったのは 忙[いそが]しかったためだ。	
\\	あんな男の言うことを聞くことはない。	
\\	あんな 男[おとこ]の 言[い]うことを 聞[き]くことはない。	ことはない 
\\	彼女が今日の会議を忘れることはないと思います。	
\\	彼女[かのじょ]が 今日[きょう]の 会議[かいぎ]を 忘[わす]れることはないと 思[おも]います。	ことはない 
\\	何もそんなに慌てることはない。	
\\	何[なに]もそんなに 慌[あわ]てることはない。	ことはない 
\\	もうお目にかかることはないかもしれませんね。	
\\	もうお 目[め]にかかることはないかもしれませんね。	ことはない 
\\	君が来ることはないと思います。	
\\	君[きみ]が 来[く]ることはないと 思[おも]います。	ことはない 
\\	加奈子の性格は明るく、無邪気だった。	
\\	加奈子[かなこ]の 性格[せいかく]は 明[あか]るく、 無邪気[むじゃき]だった。	
\\	ニューヨークタイムズは質が高く、購読者の数も多い。	
\\	ニューヨークタイムズは 質[しつ]が 高[たか]く、 購読[こうどく] 者[しゃ]の 数[かず]も 多[おお]い。	
\\	あの哲学者の思想は分かりやすく、文章も簡潔だ。	
\\	あの 哲学[てつがく] 者[しゃ]の 思想[しそう]は 分[わ]かりやすく、 文章[ぶんしょう]も 簡潔[かんけつ]だ。	
\\	私は失敗が恐ろしく、新しいことが何もできない。	
\\	私[わたし]は 失敗[しっぱい]が 恐[おそ]ろしく、 新[あたら]しいことが 何[なに]もできない。	
\\	山田さんくらいよく物を忘れる人はいない。	
\\	山田[やまだ]さんくらいよく 物[もの]を 忘[わす]れる 人[ひと]はいない。	
\\	私は料理は下手ですが、ご飯くらい (は) 炊けます。	
\\	私[わたし]は 料理[りょうり]は 下手[へた]ですが、ご 飯[はん]くらい(は) 炊[た]けます。	
\\	その家は直しようがないくらい傷んでいた。	
\\	その 家[いえ]は 直[なお]しようがないくらい 傷[いた]んでいた。	
\\	次郎はひどく酔っていて立っていられないくらいだった。	
\\	次郎[じろう]はひどく 酔[よ]っていて 立[た]っていられないくらいだった。	
\\	信頼していた人に裏切られることくらい辛いことはない。	
\\	信頼[しんらい]していた 人[ひと]に 裏切[うらぎ]られることくらい 辛[つら]いことはない。	
\\	これは簡単な計算だから計算器を使うまでもない。	
\\	これは 簡単[かんたん]な 計算[けいさん]だから 計算[けいさん] 器[き]を 使[つか]うまでもない。	までもない= 
\\	言うまでもなくジョージ・ワシントンはアメリカの初代大統領だ。	
\\	言[い]うまでもなくジョージ・ワシントンはアメリカの 初代[しょだい] 大統領[だいとうりょう]だ。	までもない= 
\\	彼がみんなに尊敬されたのは言うまでもない。	
\\	彼[かれ]がみんなに 尊敬[そんけい]されたのは 言[い]うまでもない。	
\\	それは説明するまでもなく明らかなことだ。	
\\	それは 説明[せつめい]するまでもなく 明[あき]らかなことだ。	までもない= 
\\	もう橋本には何も頼むまい。	
\\	もう 橋本[はしもと]には 何[なに]も 頼[たの]むまい。	まい 
\\	これは恐らく誰も気が付くまい。	
\\	これは 恐[おそ]らく 誰[だれ]も 気[き]が 付[つ]くまい。	まい 
\\	この教え方はあまり効果的ではあるまい。	
\\	この 教[おし]え 方[かた]はあまり 効果[こうか] 的[てき]ではあるまい。	まい 
\\	これは何かの間違いではあるまいか。	
\\	これは 何[なに]かの 間違[まちが]いではあるまいか。	まい 
\\	日本へ行こうか行くまいか迷った。	
\\	日本[にほん]へ 行[い]こうか 行[い]くまいか 迷[まよ]った。	まい 
\\	彼は多分誰の言うことも聞くまい。	
\\	彼[かれ]は 多分[たぶん] 誰[だれ]の 言[い]うことも 聞[き]くまい。	まい 
\\	彼はもう英語は教えまい。	
\\	彼[かれ]はもう 英語[えいご]は 教[おし]えまい。	まい 
\\	彼女はそんなことはすまい。	
\\	彼女[かのじょ]はそんなことはすまい。	まい 
\\	こんな機会は二度と来まい。	
\\	こんな 機会[きかい]は 二度[にど]と 来[こ]まい。	
\\	彼は私のハンドバッグを取ろうとしたが、私は取られまいとして脇の下に強く挟んだ。	
\\	彼[かれ]は 私[わたし]のハンドバッグを 取[と]ろうとしたが、 私[わたし]は 取[と]られまいとして 脇の下[わきのした]に 強[つよ]く 挟[はさ]んだ。	"脇の下=わき の した= 
\\	まいとする= 
\\	私は負けまいと頑張った。	
\\	私[わたし]は 負[ま]けまいと 頑張[がんば]った。	
\\	まいとする= 
\\	まさか美智子があんな男と結婚するとは思わなかった。	
\\	まさか 美智子[みちこ]があんな 男[おとこ]と 結婚[けっこん]するとは 思[おも]わなかった。	まさか 
\\	まさかあたしの誕生日を忘れたんじゃないでしょうね。	
\\	まさかあたしの 誕生[たんじょう] 日[び]を 忘[わす]れたんじゃないでしょうね。	まさか 
\\	まさか自分が交通事故に巻き込まれるとは思いませんでしたよ。	
\\	まさか 自分[じぶん]が 交通[こうつう] 事故[じこ]に 巻き込[まきこ]まれるとは 思[おも]いませんでしたよ。	まさか 
\\	青い顔をしているけど、まさか病気じゃないでしょうね。	
\\	青[あお]い 顔[かお]をしているけど、まさか 病気[びょうき]じゃないでしょうね。	まさか 
\\	もう五月なのだから、まさか雪は降るまい。	
\\	もう 五月[ごがつ]なのだから、まさか 雪[ゆき]は 降[ふ]るまい。	まさか 
\\	スミスさんは日本語を一年しか勉強していないのに、ぺらぺらですよ。 
\\	まさか。	
\\	スミスさんは 日本語[にほんご]を一 年[ねん]しか 勉強[べんきょう]していないのに、ぺらぺらですよ。 
\\	まさか。	
\\	まさかのときとか何かに備えておけよ。	
\\	まさかのときとか 何[なに]かに 備[そな]えておけよ。	まさかの時= 
\\	こんな給料をもらうなら辞めた方がましだ。	
\\	こんな 給料[きゅうりょう]をもらうなら 辞[や]めた 方[ほう]がましだ。	
\\	今度の日本語の先生は前の先生よりずっとましだ。	
\\	今度[こんど]の 日本語[にほんご]の 先生[せんせい]は 前[まえ]の 先生[せんせい]よりずっとましだ。	
\\	その薬はまだ市販されていない。	
\\	その 薬[くすり]はまだ 市販[しはん]されていない。	市販= 
\\	こめかみの付近が痛い。	
\\	こめかみの 付近[ふきん]が 痛[いた]い。	こめかみ= 
\\	どこかこの付近に郵便局はありませんか。	
\\	どこかこの 付近[ふきん]に 郵便[ゆうびん] 局[きょく]はありませんか。	
\\	彼はどこか新宿付近に住んでいる。	
\\	彼[かれ]はどこか 新宿[しんじゅく] 付近[ふきん]に 住[す]んでいる。	
\\	いちいちお名前は挙げないが、大変お世話になった。	
\\	いちいちお 名前[なまえ]は 挙[あ]げないが、 大変[たいへん]お 世話[せわ]になった。	
\\	彼は私のすることにいちいち難癖をつける。	
\\	彼[かれ]は 私[わたし]のすることにいちいち 難癖[なんくせ]をつける。	難癖=なんくせ= 
\\	天候悪化により登頂は断念せざるを得なかった。	
\\	天候[てんこう] 悪化[あっか]により 登頂[とうちょう]は 断念[だんねん]せざるを 得[え]なかった。	断念= 
\\	その職場は同調圧力が強い。	
\\	その 職場[しょくば]は 同調[どうちょう] 圧力[あつりょく]が 強[つよ]い。	同調圧力= 
\\	窓口の職員の慇懃無礼な応対に腹が立った。	
\\	窓口[まどぐち]の 職員[しょくいん]の 慇懃無礼[いんぎんぶれい]な 応対[おうたい]に 腹[はら]が 立[た]った。	応対= 
\\	彼は気が利かない男だが、指示に対しては素直だ。	
\\	彼[かれ]は 気[き]が 利[き]かない 男[おとこ]だが、 指示[しじ]に 対[たい]しては 素直[すなお]だ。	素直=すなお= 
\\	素直に金を出せ。	
\\	素直[すなお]に 金[きん]を 出[だ]せ。	素直=すなお= 
\\	この歌詞は当時の彼女の気持ちを素直に表現している。	
\\	この 歌詞[かし]は 当時[とうじ]の 彼女[かのじょ]の 気持[きも]ちを 素直[すなお]に 表現[ひょうげん]している。	素直=すなお= 
\\	国家の命運はまさに危機一髪であった。	
\\	国家[こっか]の 命運[めいうん]はまさに 危機一髪[ききいっぱつ]であった。	
\\	それでは不正を奨励するようなものだ。	
\\	それでは 不正[ふせい]を 奨励[しょうれい]するようなものだ。	奨励=しょうれい= 
\\	それは化学研究の奨励となるだろう。	
\\	それは 化学[かがく] 研究[けんきゅう]の 奨励[しょうれい]となるだろう。	奨励=しょうれい= 
\\	雨が多くて年内の完成は望み薄になった。	
\\	雨[あめ]が 多[おお]くて 年内[ねんない]の 完成[かんせい]は 望[のぞ]み 薄[うす]になった。	望み薄=のぞみうす= 
\\	そのビルの建設は望み薄だ。	
\\	そのビルの 建設[けんせつ]は 望[のぞ]み 薄[うす]だ。	望み薄=のぞみうす= 
\\	彼女のおかげでことなきを得た。	
\\	彼女[かのじょ]のおかげでことなきを 得[え]た。	事なきを得る= 
\\	出掛ける前にドアに鍵を掛け忘れたが、幸いにも事なきを得た。	
\\	出掛[でか]ける 前[まえ]にドアに 鍵[かぎ]を 掛[か]け 忘[わす]れたが、 幸[さいわ]いにも 事[こと]なきを 得[え]た。	事なきを得る= 
\\	もう少しましなコーヒーはありませんか。	
\\	もう 少[すこ]しましなコーヒーはありませんか。	
\\	こんな苦しい生活をするくらいなら、死んだ方がましだ。	
\\	こんな 苦[くる]しい 生活[せいかつ]をするくらいなら、 死[し]んだ 方[ほう]がましだ。	
\\	給料は二万円でも、ないよりましだ。	
\\	給料[きゅうりょう]は 二万[にまん] 円[えん]でも、ないよりましだ。	
\\	現金、または小切手でお払い下さい。	
\\	現金[げんきん]、または 小切手[こぎって]でお 払[はら]い 下[くだ]さい。	
\\	電話番号を聞き違えたか、またはもうこの電話は使われていないのだろう。	
\\	電話[でんわ] 番号[ばんごう]を 聞[き]き 違[ちが]えたか、またはもうこの 電話[でんわ]は 使[つか]われていないのだろう。	
\\	私の車は右から二台目です。	
\\	私[わたし]の 車[くるま]は 右[みぎ]から 二台[にだい] 目[め]です。	
\\	それは上から三つ目の引き出しに入っています。	
\\	それは 上[うえ]から 三[みっ]つ 目[め]の 引き出[ひきだ]しに 入[はい]っています。	
\\	山本先生は前から二列目、左から三人目の人だ。	
\\	山本[やまもと] 先生[せんせい]は 前[まえ]から 二列[にれつ] 目[め]、 左[ひだり]から 三人[さんにん] 目[め]の 人[ひと]だ。	
\\	一回目は失敗した。	
\\	一回[いっかい] 目[め]は 失敗[しっぱい]した。	
\\	五週目からは林先生がこのクラスをお教えになります。	
\\	五週[ごしゅう] 目[め]からは 林[はやし] 先生[せんせい]がこのクラスをお 教[おし]えになります。	
\\	上野さんは一番目に演奏する。	
\\	上野[うえの]さんは 一番目[いちばんめ]に 演奏[えんそう]する。	
\\	この食堂は衛生面によく気を配っている。	
\\	この 食堂[しょくどう]は 衛生[えいせい] 面[めん]によく 気[き]を 配[くば]っている。	
\\	藤田さんは技術面からのみものを見る傾向がある。	
\\	藤田[ふじた]さんは 技術[ぎじゅつ] 面[めん]からのみものを 見[み]る 傾向[けいこう]がある。	
\\	あの候補者の演説は政策面での説得力に欠ける。	
\\	あの 候補[こうほ] 者[しゃ]の 演説[えんぜつ]は 政策[せいさく] 面[めん]での 説得[せっとく] 力[りょく]に 欠[か]ける。	
\\	東京に来て足掛け5年になる。	
\\	東京[とうきょう]に 来[き]て 足掛[あしか]け 5年[ごねん]になる。	足掛け= 
\\	足掛け10年にわたる研究の結果が世に出た。	
\\	足掛[あしか]け10 年[ねん]にわたる 研究[けんきゅう]の 結果[けっか]が 世[よ]に 出[で]た。	足掛け= 
\\	世に出る= 
\\	その漁船はアフリカ沖で操業している。	
\\	その 漁船[ぎょせん]はアフリカ 沖[おき]で 操業[そうぎょう]している。	漁船=ぎょせん= 
\\	操業=そうぎょう= 
\\	この沖に島がある。	
\\	この 沖[おき]に 島[しま]がある。	
\\	台風は鹿児島のすぐ沖まで来ている。	
\\	台風[たいふう]は 鹿児島[かごしま]のすぐ 沖[おき]まで 来[き]ている。	
\\	彼の顔には一週間剃らなかったひげがもじゃもじゃと生えていた。	
\\	彼[かれ]の 顔[かお]には 一週間[いっしゅうかん] 剃[そ]らなかったひげがもじゃもじゃと 生[は]えていた。	もじゃもじゃ= 
\\	子犬は私の手をペロペロなめた。	
\\	子犬[こいぬ]は 私[わたし]の 手[て]をペロペロなめた。	
\\	日焼けした背中の皮がぺろぺろむけてきた。	
\\	日焼[ひや]けした 背中[せなか]の 皮[かわ]がぺろぺろむけてきた。	剥ける=むける= 
\\	彼は二人前の料理をぺろりと平らげた。	
\\	彼[かれ]は 二人前[ふたりまえ]の 料理[りょうり]をぺろりと 平[たい]らげた。	
\\	彼は小銭をポケットの中でチャラチャラいわせた。	
\\	彼[かれ]は 小銭[こぜに]をポケットの 中[なか]でチャラチャラいわせた。	
\\	それを聞いて彼はちょっとムッとしたようだった。	
\\	それを 聞[き]いて 彼[かれ]はちょっとムッとしたようだった。	
\\	ムッとする暑さであった。	
\\	ムッとする 暑[あつ]さであった。	
\\	納豆好きの人はあのネバネバがいいんだと言う。	
\\	納豆[なっとう] 好[す]きの 人[ひと]はあのネバネバがいいんだと 言[い]う。	
\\	溶けたあめがネバネバ手にくっついた。	
\\	溶[と]けたあめがネバネバ 手[しゅ]にくっついた。	飴=あめ= 
\\	くっつく=
\\	霜柱をさくさく踏みつけながら歩いた。	
\\	霜柱[しもばしら]をさくさく 踏[ふ]みつけながら 歩[ある]いた。	霜柱=しもばしら= 
\\	踏みつける= 
\\	遺族は大いに困窮している。	
\\	遺族[いぞく]は 大[おお]いに 困窮[こんきゅう]している。	遺族=いぞく= 
\\	ある老人に易を見てもらった。	
\\	ある 老人[ろうじん]に 易[えき]を 見[み]てもらった。	易=えき= 
\\	そんなことをして何の益になるのか。	
\\	そんなことをして 何[なに]の 益[えき]になるのか。	
\\	彼女の発明は当社に大きな益をもたらした。	
\\	彼女[かのじょ]の 発明[はつめい]は 当社[とうしゃ]に 大[おお]きな 益[えき]をもたらした。	
\\	社会に益をもたらす活動に参加したい。	
\\	社会[しゃかい]に 益[えき]をもたらす 活動[かつどう]に 参加[さんか]したい。	
\\	新会社の事業計画案を練っているところだ。	
\\	新[しん] 会社[かいしゃ]の 事業[じぎょう] 計画[けいかく] 案[あん]を 練[ね]っているところだ。	事業計画= 
\\	私はあのハンサムな男の子と結婚してみせるわ。	
\\	私[わたし]はあのハンサムな 男の子[おとこのこ]と 結婚[けっこん]してみせるわ。	「〜てみせる」
\\	今年こそは修士論文を書き上げてみせる。	
\\	今年[ことし]こそは 修士[しゅうし] 論文[ろんぶん]を 書き上[かきあ]げてみせる。	
\\	この本は有益で、その上、面白くもある。	
\\	この 本[ほん]は 有益[ゆうえき]で、その 上[うえ]、 面白[おもしろ]くもある。	
\\	この映画は面白くないし、特に教育的でもない。	
\\	この 映画[えいが]は 面白[おもしろ]くないし、 特[とく]に 教育[きょういく] 的[てき]でもない。	
\\	由利子は最近廊下ですれ違っても見向きもしない。	
\\	由利子[ゆりこ]は 最近[さいきん] 廊下[ろうか]ですれ 違[ちが]っても 見向[みむ]きもしない。	すれ違う= 
\\	田口さんなんて友達でもないのに、どうしてそんなにしてあげるの。	
\\	田口[たぐち]さんなんて 友達[ともだち]でもないのに、どうしてそんなにしてあげるの。	
\\	この美術館は三時間もあれば全部見られる。	
\\	この 美術館[びじゅつかん]は 三時間[さんじかん]もあれば 全部[ぜんぶ] 見[み]られる。	〜も〜ば 
\\	二万円も持って行けば足りるでしょう。	
\\	二万[にまん] 円[えん]も 持[も]って 行[い]けば 足[た]りるでしょう。	〜も〜ば 
\\	ビールは二ダーズも買っておけば大丈夫だ。	
\\	ビールは 二[に]ダーズも 買[か]っておけば 大丈夫[だいじょうぶ]だ。	〜も〜ば 
\\	腰が痛くて立ちも座りも出来ない。	
\\	腰[こし]が 痛[いた]くて 立[た]ちも 座[すわ]りも 出来[でき]ない。	
\\	私は俳句が大好きで、よく読みもするし自分で作りもする。	
\\	私[わたし]は 俳句[はいく]が 大好[だいす]きで、よく 読[よ]みもするし 自分[じぶん]で 作[つく]りもする。	
\\	新しい入れ歯が気になる。	
\\	新[あたら]しい 入れ歯[いれば]が 気[き]になる。	
\\	デザイナーという仕事柄、街行く人の服装がつい気になる。	
\\	デザイナーという 仕事[しごと] 柄[がら]、 街[まち] 行[い]く 人[ひと]の 服装[ふくそう]がつい 気[き]になる。	仕事柄= 
\\	別れ際の一言が気になって彼に電話した。	
\\	別れ際[わかれぎわ]の 一言[ひとこと]が 気[き]になって 彼[かれ]に 電話[でんわ]した。	別れ際=わかれぎわ= 
\\	若者の言葉遣いが気になって仕方がない。	
\\	若者[わかもの]の 言葉[ことば] 遣[づか]いが 気[き]になって 仕方[しかた]がない。	
\\	ニンニクを食べた後、牛乳を飲めば匂いも気にならない。	
\\	ニンニクを 食[た]べた 後[のち]、 牛乳[ぎゅうにゅう]を 飲[の]めば 匂[にお]いも 気[き]にならない。	
\\	彼は成績が悪くても全然気にならないらしい。	
\\	彼[かれ]は 成績[せいせき]が 悪[わる]くても 全然[ぜんぜん] 気[き]にならないらしい。	
\\	鈴木 (仮名) はその場で逮捕された。	
\\	鈴木[すずき]
\\	仮名[かめい])はその 場[ば]で 逮捕[たいほ]された。	
\\	最初の仮説が間違っていたのだ。	
\\	最初[さいしょ]の 仮説[かせつ]が 間違[まちが]っていたのだ。	
\\	親の愛情をひしひしと感じた。	
\\	親[おや]の 愛情[あいじょう]をひしひしと 感[かん]じた。	ひしひしと= 
\\	手紙からは彼の思いがひしひしと伝わってきた。	
\\	手紙[てがみ]からは 彼[かれ]の 思[おも]いがひしひしと 伝[つた]わってきた。	
\\	四季折々の眺めがある。	
\\	四季[しき] 折々[おりおり]の 眺[なが]めがある。	四季折々= 
\\	今年は優れた作品に恵まれず、文壇は不毛な一年だった。	
\\	今年[ことし]は 優[すぐ]れた 作品[さくひん]に 恵[めぐ]まれず、 文壇[ぶんだん]は 不毛[ふもう]な 一年[いちねん]だった。	文壇=ぶんだん= 
\\	トラはネコ科に属する。	
\\	トラはネコ 科[か]に 属[ぞく]する。	
\\	高校の時はテニス部に属していた。	
\\	高校[こうこう]の 時[とき]はテニス 部[ぶ]に 属[ぞく]していた。	
\\	その国は当時スペインに属していた。	
\\	その 国[くに]は 当時[とうじ]スペインに 属[ぞく]していた。	
\\	どこに配属されるか分からない。	
\\	どこに 配属[はいぞく]されるか 分[わ]からない。	
\\	この試案を討議の叩き台として提出します。	
\\	この 試案[しあん]を 討議[とうぎ]の 叩き台[たたきだい]として 提出[ていしゅつ]します。	試案= 
\\	叩き台=たたきだい= 
\\	我が党はあくまでも憲法を守る姿勢を貫く方針である。	
\\	我[わ]が 党[とう]はあくまでも 憲法[けんぽう]を 守[まも]る 姿勢[しせい]を 貫[つらぬ]く 方針[ほうしん]である。	あくまで= 
\\	貫く=つらぬく= 
\\	どんなに体を揺さぶっても彼女は目を覚まさなかった。	
\\	どんなに 体[からだ]を 揺[ゆ]さぶっても 彼女[かのじょ]は 目[め]を 覚[さ]まさなかった。	揺さぶる=ゆさぶる= (揺する) 
\\	(動揺させる) 
\\	それは科学界を根底から揺さぶる発見だった。	
\\	それは 科学[かがく] 界[かい]を 根底[こんてい]から 揺[ゆ]さぶる 発見[はっけん]だった。	揺さぶる=ゆさぶる= (揺する) 
\\	(動揺させる) 
\\	彼は私の発言に対して必ずしも肯定的ではなかった。	
\\	彼[かれ]は 私[わたし]の 発言[はつげん]に 対[たい]して 必[かなら]ずしも 肯定[こうてい] 的[てき]ではなかった。	肯定的=こうていてき= 
\\	彼女の提案はおおむね肯定的に受け止められた。	
\\	彼女[かのじょ]の 提案[ていあん]はおおむね 肯定[こうてい] 的[てき]に 受け止[うけと]められた。	おおむね= 
\\	あんな奴に見下されて悔しくないのか。	
\\	あんな 奴[やつ]に 見下[みくだ]されて 悔[くや]しくないのか。	
\\	この機械にはもはや性能向上の余地はほとんどない。	
\\	この 機械[きかい]にはもはや 性能[せいのう] 向上[こうじょう]の 余地[よち]はほとんどない。	
\\	この機械を使いこなせる人は少ない。	
\\	この 機械[きかい]を 使[つか]いこなせる 人[ひと]は 少[すく]ない。	
\\	彼女は5カ国語を使いこなす。	
\\	彼女[かのじょ]は5 カ国[かこく] 語[ご]を 使[つか]いこなす。	
\\	彼は英語を上手に使いこなす。	
\\	彼[かれ]は 英語[えいご]を 上手[じょうず]に 使[つか]いこなす。	
\\	うまい趣向が浮かんだ。	
\\	うまい 趣向[しゅこう]が 浮[う]かんだ。	趣向= 
\\	彼女にとって昨年は大きく飛躍した一年だった。	
\\	彼女[かのじょ]にとって 昨年[さくねん]は 大[おお]きく 飛躍[ひやく]した 一年[いちねん]だった。	飛躍= 
\\	この部分に論理の飛躍がある。	
\\	この 部分[ぶぶん]に 論理[ろんり]の 飛躍[ひやく]がある。	飛躍= 
\\	この修理にはだいぶ手間がかかりそうだ。	
\\	この 修理[しゅうり]にはだいぶ 手間[てま]がかかりそうだ。	手間=てま= (時) 
\\	(苦労) 
\\	この方法だとかなり手間が省ける。	
\\	この 方法[ほうほう]だとかなり 手間[てま]が 省[はぶ]ける。	手間=てま= 
\\	省く=はぶく= 
\\	その映画は場面が火星に設定されている。	
\\	その 映画[えいが]は 場面[ばめん]が 火星[かせい]に 設定[せってい]されている。	設定= 
\\	アメリカは京都定義書からの離脱を表明した。	
\\	アメリカは 京都[きょうと] 定義[ていぎ] 書[しょ]からの 離脱[りだつ]を 表明[ひょうめい]した。	離脱= 
\\	表明= 
\\	もうこれで危険は脱した。	
\\	もうこれで 危険[きけん]は 脱[だっ]した。	
\\	これは日本の将来にとってよい前兆だ。	
\\	これは 日本[にほん]の 将来[しょうらい]にとってよい 前兆[ぜんちょう]だ。	前兆= 
\\	黒雲はたいてい嵐の前兆だ。	
\\	黒雲[くろくも]はたいてい 嵐[あらし]の 前兆[ぜんちょう]だ。	前兆= 
\\	彼の作品は面白くもあり読みやすくもある。	
\\	彼[かれ]の 作品[さくひん]は 面白[おもしろ]くもあり 読[よ]みやすくもある。	
\\	彼女の英語は特に上手でも下手でもない。	
\\	彼女[かのじょ]の 英語[えいご]は 特[とく]に 上手[じょうず]でも 下手[へた]でもない。	
\\	この教科書は難しすぎもせずやさしすぎもせず、ちょうどいい。	
\\	この 教科書[きょうかしょ]は 難[むずか]しすぎもせずやさしすぎもせず、ちょうどいい。	
\\	エレクトロニクスの世界は今後も急速な進歩を続けていくものと予想される。	
\\	エレクトロニクスの 世界[せかい]は 今後[こんご]も 急速[きゅうそく]な 進歩[しんぽ]を 続[つづ]けていくものと 予想[よそう]される。	
\\	まだ間に合うかなあ。	
\\	まだ 間に合[まにあ]うかなあ。	
\\	日本の文化はユニークだなどと言う人がいるが、私はそうは思わない。	
\\	日本[にほん]の 文化[ぶんか]はユニークだなどと 言[い]う 人[ひと]がいるが、 私[わたし]はそうは 思[おも]わない。	
\\	お金がないから、日本へ行くなどということは夢です。	
\\	お 金[かね]がないから、 日本[にほん]へ 行[い]くなどということは 夢[ゆめ]です。	
\\	日本語は難しくありませんか。 
\\	いえ、難しくないこともないんですが、日本語の難しさは強調されすぎていると思いますよ。	
\\	日本語[にほんご]は 難[むずか]しくありませんか。 
\\	いえ、 難[むずか]しくないこともないんですが、 日本語[にほんご]の 難[むずか]しさは 強調[きょうちょう]されすぎていると 思[おも]いますよ。	
\\	この生け花はなかなか見事に生けてある。	
\\	この 生け花[いけばな]はなかなか 見事[みごと]に 生[い]けてある。	見事=みごと= 
\\	生ける= 
\\	昨日見た映画は実に面白くなく、途中で寝てしまった。	
\\	昨日[きのう] 見[み]た 映画[えいが]は 実[じつ]に 面白[おもしろ]くなく、 途中[とちゅう]で 寝[ね]てしまった。	
\\	彼はその吉報に対して半信半疑であった。	
\\	彼[かれ]はその 吉報[きっぽう]に 対[たい]して 半信半疑[はんしんはんぎ]であった。	吉報=きっぽう= 
\\	吉報があるぞ。	
\\	吉報[きっぽう]があるぞ。	吉報=きっぽう= 
\\	この部分をもう少し砕けた言葉に直してくれ。	
\\	この 部分[ぶぶん]をもう 少[すこ]し 砕[くだ]けた 言葉[ことば]に 直[なお]してくれ。	砕けた=くだけた= 
\\	すぐに討議は砕けた会話になった。	
\\	すぐに 討議[とうぎ]は 砕[くだ]けた 会話[かいわ]になった。	砕けた=くだけた= 
\\	社長が死んだらその会社はがたがたになった。	
\\	社長[しゃちょう]が 死[し]んだらその 会社[かいしゃ]はがたがたになった。	
\\	「おお寒い!」と彼は歯をがたがたさせながら言った。	
\\	「おお 寒[さむ]い!」と 彼[かれ]は 歯[は]をがたがたさせながら 言[い]った。	
\\	つまらないことでがたがた言うな。	
\\	つまらないことでがたがた 言[い]うな。	
\\	地震の前には何らかの予兆があると主張する人もいる。	
\\	地震[じしん]の 前[まえ]には 何[なん]らかの 予兆[よちょう]があると 主張[しゅちょう]する 人[ひと]もいる。	予兆= 
\\	日増しに暑くなりつつある。	
\\	日増[ひま]しに 暑[あつ]くなりつつある。	日増しに= 
\\	赤ん坊が日増しにかわいくなってきています。	
\\	赤ん坊[あかんぼう]が 日増[ひま]しにかわいくなってきています。	日増しに= 
\\	彼はこの地域の軍事情勢に詳しい。	
\\	彼[かれ]はこの 地域[ちいき]の 軍事[ぐんじ] 情勢[じょうせい]に 詳[くわ]しい。	情勢=じょうせい= 
\\	首相の急死で情勢は一変した。	
\\	首相[しゅしょう]の 急死[きゅうし]で 情勢[じょうせい]は 一変[いっぺん]した。	情勢=じょうせい= 
\\	一変=いっぺん= 
\\	戦争は避けられない情勢だ。	
\\	戦争[せんそう]は 避[さ]けられない 情勢[じょうせい]だ。	情勢=じょうせい= 
\\	有権者の半数は態度を決めておらず、情勢はなお流動的だ。	
\\	有権者[ゆうけんしゃ]の 半数[はんすう]は 態度[たいど]を 決[き]めておらず、 情勢[じょうせい]はなお 流動的[りゅうどうてき]だ。	情勢=じょうせい= 
\\	なお= 
\\	流動的= 
\\	これには莫大な費用がかかったに違いない。	
\\	これには 莫大[ばくだい]な 費用[ひよう]がかかったに 違[ちが]いない。	莫大=ばくだい= 
\\	費用=ひよう= 
\\	この手術は莫大な金がかかる。	
\\	この 手術[しゅじゅつ]は 莫大[ばくだい]な 金[きん]がかかる。	莫大=ばくだい= 
\\	この間いただいた百科事典は実に重宝しています。	
\\	この 間[あいだ]いただいた 百科[ひゃっか] 事典[じてん]は 実[じつ]に 重宝[ちょうほう]しています。	重宝=ちょうほう= 
\\	地下に財宝が眠っている。	
\\	地下[ちか]に 財宝[ざいほう]が 眠[ねむ]っている。	
\\	洪水被災者のほとんどは仮設の避難所に滞在している。	
\\	洪水[こうずい] 被災[ひさい] 者[しゃ]のほとんどは 仮設[かせつ]の 避難[ひなん] 所[しょ]に 滞在[たいざい]している。	仮設= 
\\	避難所=ひなんじょ= 
\\	予算が大幅に削減された。	
\\	予算[よさん]が 大幅[おおはば]に 削減[さくげん]された。	
\\	鉄道運賃が大幅に引き上げられた。	
\\	鉄道[てつどう] 運賃[うんちん]が 大幅[おおはば]に 引き上[ひきあ]げられた。	引き上げる= 
\\	彼はマラソンで自己最高記録を大幅に上回るタイムで優勝した。	
\\	彼[かれ]はマラソンで 自己[じこ] 最高[さいこう] 記録[きろく]を 大幅[おおはば]に 上回[うわまわ]るタイムで 優勝[ゆうしょう]した。	上回る=うわまわる= 
\\	二人の意見は大幅に食い違っていた。	
\\	二 人[にん]の 意見[いけん]は 大幅[おおはば]に 食い違[くいちが]っていた。	食い違う=くいちがう= 
\\	薬のせいか体がだるくてしょうがない。	
\\	薬[くすり]のせいか 体[からだ]がだるくてしょうがない。	だるい= 
\\	だるくて働くのが嫌だ。	
\\	だるくて 働[はたら]くのが 嫌[いや]だ。	だるい= 
\\	彼はだるそうな目をしていた。	
\\	彼[かれ]はだるそうな 目[め]をしていた。	だるい= 
\\	彼女はだるそうに返事をした。	
\\	彼女[かのじょ]はだるそうに 返事[へんじ]をした。	だるい= 
\\	その酔っぱらい運転で告訴された警官は職を失った。	
\\	その 酔[よ]っぱらい 運転[うんてん]で 告訴[こくそ]された 警官[けいかん]は 職[しょく]を 失[うしな]った。	告訴=こくそ= 
\\	その男は詐欺で告訴された。	
\\	その 男[おとこ]は 詐欺[さぎ]で 告訴[こくそ]された。	詐欺=さぎ= 
\\	告訴=こくそ= 
\\	彼は児童のポルノ写真を持っていた罪で告訴された。	
\\	彼[かれ]は 児童[じどう]のポルノ 写真[しゃしん]を 持[も]っていた 罪[つみ]で 告訴[こくそ]された。	児童=じどう= 
\\	告訴=こくそ= 
\\	内線の続くソマリアで見た光景は実に悲惨であった。	
\\	内線[ないせん]の 続[つづ]くソマリアで 見[み]た 光景[こうけい]は 実[じつ]に 悲惨[ひさん]であった。	
\\	試験の結果は悲惨なものだった。	
\\	試験[しけん]の 結果[けっか]は 悲惨[ひさん]なものだった。	
\\	半年も掃除していないから部屋は悲惨なことになっている。	
\\	半年[はんとし]も 掃除[そうじ]していないから 部屋[へや]は 悲惨[ひさん]なことになっている。	
\\	遠くて女性の悲鳴が聞こえた。	
\\	遠[とお]くて 女性[じょせい]の 悲鳴[ひめい]が 聞[き]こえた。	
\\	その映像に観衆から悲鳴が上がった。	
\\	その 映像[えいぞう]に 観衆[かんしゅう]から 悲鳴[ひめい]が 上[あ]がった。	観衆= 
\\	小さい虫にいちいち悲鳴を上げるな。	
\\	小[ちい]さい 虫[むし]にいちいち 悲鳴[ひめい]を 上[あ]げるな。	
\\	注文が殺到してうれしい悲鳴を上げている。	
\\	注文[ちゅうもん]が 殺到[さっとう]してうれしい 悲鳴[ひめい]を 上[あ]げている。	
\\	その後ろ姿には悲哀が漂っていた。	
\\	その 後ろ姿[うしろすがた]には 悲哀[ひあい]が 漂[ただよ]っていた。	悲哀=ひあい= 
\\	漂う=ただよう= 
\\	俺に逆らうつもりか。	
\\	俺[おれ]に 逆[さか]らうつもりか。	逆らう=さからう= 
\\	(命令などに) 
\\	こちらの論法をうまく逆手に取られた。	
\\	こちらの 論法[ろんぽう]をうまく 逆手[さかて]に 取[と]られた。	論法=ろんぽう= 
\\	それは順番が逆さまだ。	
\\	それは 順番[じゅんばん]が 逆[さか]さまだ。	逆さま= 
\\	息子に叱られるなんて逆さまだ。	
\\	息子[むすこ]に 叱[しか]られるなんて 逆[さか]さまだ。	逆さま= 
\\	彼女は有名な先生に就いてピアノを習っている。	
\\	彼女[かのじょ]は 有名[ゆうめい]な 先生[せんせい]に 就[つ]いてピアノを 習[なら]っている。	就く= 
\\	当時は大学を出た者でも職に就くのが難しかった。	
\\	当時[とうじ]は 大学[だいがく]を 出[で]た 者[もの]でも 職[しょく]に 就[つ]くのが 難[むずか]しかった。	就く= 
\\	彼はまだ定職に就いていない。	
\\	彼[かれ]はまだ 定職[ていしょく]に 就[つ]いていない。	就く= 
\\	大学では加藤先生に就いていた。	
\\	大学[だいがく]では 加藤[かとう] 先生[せんせい]に 就[つ]いていた。	就く= 
\\	あの人は何一つ成就できない。	
\\	あの 人[ひと]は 何一[なにひと]つ 成就[じょうじゅ]できない。	成就=じょうじゅ= 
\\	駅の階段を上がるのは年寄りには重労働だ。	
\\	駅[えき]の 階段[かいだん]を 上[あ]がるのは 年寄[としよ]りには 重労働[じゅうろうどう]だ。	重労働= 
\\	この保険には素晴らしい特典がついている。	
\\	この 保険[ほけん]には 素晴[すば]らしい 特典[とくてん]がついている。	特典= 
\\	会員の方には全商品5%引きという特典があります。	
\\	会員[かいいん]の 方[かた]には 全[ぜん] 商品[しょうひん]5 
\\	[ぱーせんと] 引[び]きという 特典[とくてん]があります。	特典= 
\\	これは最新の学問を踏まえた英国社会史である。	
\\	これは 最新[さいしん]の 学問[がくもん]を 踏[ふ]まえた 英国[えいこく] 社会[しゃかい] 史[し]である。	踏まえる=ふまえる= 
\\	これまでの実績を踏まえてオリンピック出場選手を決める。	
\\	これまでの 実績[じっせき]を 踏[ふ]まえてオリンピック 出場[しゅつじょう] 選手[せんしゅ]を 決[き]める。	実績=じっせき= 
\\	踏まえる=ふまえる= 
\\	このグラフは縦軸が平均体重、横軸が年齢を表している。	
\\	このグラフは 縦[たて] 軸[じく]が 平均[へいきん] 体重[たいじゅう]、 横[よこ] 軸[じく]が 年齢[ねんれい]を 表[あらわ]している。	縦軸=たてじく= 
\\	横軸=よこじく= 
\\	創業当時の経営方針が旧態依然として今も続いている。	
\\	創業[そうぎょう] 当時[とうじ]の 経営[けいえい] 方針[ほうしん]が 旧態[きゅうたい] 依然[いぜん]として 今[いま]も 続[つづ]いている。	創業= 
\\	旧態依然= 
\\	事件は依然として謎だ。	
\\	事件[じけん]は 依然[いぜん]として 謎[なぞ]だ。	依然= 
\\	彼は依然としてその悪癖をやめていない。	
\\	彼[かれ]は 依然[いぜん]としてその 悪癖[あくへき]をやめていない。	依然= 
\\	悪癖=あくへき= 
\\	依然としてその説明に私は納得できない。	
\\	依然[いぜん]としてその 説明[せつめい]に 私[わたし]は 納得[なっとく]できない。	依然= 
\\	彼からは依然として何の連絡もない。	
\\	彼[かれ]からは 依然[いぜん]として 何[なに]の 連絡[れんらく]もない。	依然= 
\\	何でも山本さんは奥さんと別れて、一人で暮らしているそうですよ。	
\\	何[なに]でも 山本[やまもと]さんは 奥[おく]さんと 別[わか]れて、一 人[にん]で 暮[く]らしているそうですよ。	"「何でも」 
\\	何でもあの人は株でだいぶ儲けたようですよ。	
\\	何[なに]でもあの 人[ひと]は 株[かぶ]でだいぶ 儲[もう]けたようですよ。	"「何でも」 
\\	何でも日本とアメリカの西海岸を五時間ぐらいで飛ぶ飛行機を開発しているという話ですよ。	
\\	何[なに]でも 日本[にほん]とアメリカの 西海岸[にしかいがん]を五 時間[じかん]ぐらいで 飛[と]ぶ 飛行機[ひこうき]を 開発[かいはつ]しているという 話[はなし]ですよ。	"「何でも」 
\\	一日ボスにがなり立てられるんで、何しろ、ストレスが多いんだ。	
\\	一 日[にち]ボスにがなり 立[た]てられるんで、 何[なに]しろ、ストレスが 多[おお]いんだ。	何しろ 
\\	がなり立てる= 
\\	あの人はよくたばこを吸いますよ。何しろ一日六十本ぐらい吸うんですから。	
\\	あの 人[ひと]はよくたばこを 吸[す]いますよ。 何[なに]しろ一 日[にち] 六十本[ろくじっぽん]ぐらい 吸[す]うんですから。	何しろ 
\\	何しろ、忙しいんだ。寝る時間もないんだよ。	
\\	何[なに]しろ、 忙[いそが]しいんだ。 寝[ね]る 時間[じかん]もないんだよ。	何しろ 
\\	病気の母のことが心配でならない。	
\\	病気[びょうき]の 母[はは]のことが 心配[しんぱい]でならない。	ならない= 
\\	一人で住んでいる母親のことが気になってならない。	
\\	一 人[にん]で 住[す]んでいる 母親[ははおや]のことが 気[き]になってならない。	ならない= 
\\	夫が単身赴任しているので、寂しくてなりません。	
\\	夫[おっと]が 単身[たんしん] 赴任[ふにん]しているので、 寂[さび]しくてなりません。	単身赴任=たんしんふにん= 
\\	ならない= 
\\	仕事がうまく行っていないので、気が滅入ってならない。	
\\	仕事[しごと]がうまく 行[い]っていないので、 気[き]が 滅入[めい]ってならない。	滅入る=めいる= 
\\	ならない= 
\\	彼が帰宅するまでを見届けた。	
\\	彼[かれ]が 帰宅[きたく]するまでを 見届[みとど]けた。	見届ける= 
\\	教え子の成功を見届けて安心した。	
\\	教え子[おしえご]の 成功[せいこう]を 見届[みとど]けて 安心[あんしん]した。	見届ける= 
\\	そのびんには「毒薬」という札が貼ってあった。	
\\	そのびんには
\\	毒薬[どくやく]」という 札[さつ]が 貼[は]ってあった。	
\\	何か札が貼ってあるけど何て書いてあるの。	
\\	何[なに]か 札[さつ]が 貼[は]ってあるけど 何[なに]て 書[か]いてあるの。	
\\	酔いつぶれた兄は友達に抱えられて買ってきた。	
\\	酔[よ]いつぶれた 兄[あに]は 友達[ともだち]に 抱[かか]えられて 買[か]ってきた。	酔いつぶれる= 
\\	彼は7人の家族を抱えて昼も夜も働いた。	
\\	彼[かれ]は 7人[しちにん]の 家族[かぞく]を 抱[かか]えて 昼[ひる]も 夜[よる]も 働[はたら]いた。	
\\	彼女は逆境にあっても屈しなかった。	
\\	彼女[かのじょ]は 逆境[ぎゃっきょう]にあっても 屈[くっ]しなかった。	逆境= 
\\	屈する=くっする= 
\\	逆境は人物を作る。	
\\	逆境[ぎゃっきょう]は 人物[じんぶつ]を 作[つく]る。	逆境= 
\\	普通に歩いていて転倒する老人が多い。	
\\	普通[ふつう]に 歩[ある]いていて 転倒[てんとう]する 老人[ろうじん]が 多[おお]い。	
\\	彼女にも取り柄がないわけじゃない。	
\\	彼女[かのじょ]にも 取り柄[とりえ]がないわけじゃない。	取り柄=とりえ= 
\\	まじめなだけが彼の取り柄だ。	
\\	まじめなだけが 彼[かれ]の 取り柄[とりえ]だ。	取り柄=とりえ= 
\\	デパートはかつてファッションの発信源だった。	
\\	デパートはかつてファッションの 発信[はっしん] 源[げん]だった。	発信源=はっしんげん= 
\\	容疑者は遺体で発見された。	
\\	容疑[ようぎ] 者[しゃ]は 遺体[いたい]で 発見[はっけん]された。	遺体=いたい= 
\\	地元漁民の要請を受けて県が水質汚染の調査に乗り出した。	
\\	地元[じもと] 漁民[ぎょみん]の 要請[ようせい]を 受[う]けて 県[けん]が 水質[すいしつ] 汚染[おせん]の 調査[ちょうさ]に 乗り出[のりだ]した。	漁民=ぎょみん= 
\\	水質汚染=すいしつおせん= 
\\	乗り出す= 
\\	学校案内は請求すれば送ってくれる。	
\\	学校[がっこう] 案内[あんない]は 請求[せいきゅう]すれば 送[おく]ってくれる。	
\\	キャンセルしたら高額な違約金を請求された。	
\\	キャンセルしたら 高額[こうがく]な 違約[いやく] 金[きん]を 請求[せいきゅう]された。	違約金=いやくきん= 
\\	5万円の請求が来た。	
\\	5万[ごまん] 円[えん]の 請求[せいきゅう]が 来[き]た。	
\\	彼は彼女について悪い風説を立てた。	
\\	彼[かれ]は 彼女[かのじょ]について 悪[わる]い 風説[ふうせつ]を 立[た]てた。	風説= 
\\	風説を立てる= 
\\	それを見るとたちまち気が変わった。	
\\	それを 見[み]るとたちまち 気[き]が 変[か]わった。	たちまち= 
\\	その建物はたちまちのうちに猛火になめ尽くされた。	
\\	その 建物[たてもの]はたちまちのうちに 猛火[もうか]になめ 尽[つ]くされた。	たちまち= 
\\	猛火=もうか= 
\\	彼女はたちまちのうちにパソコンの腕が上達した。	
\\	彼女[かのじょ]はたちまちのうちにパソコンの 腕[うで]が 上達[じょうたつ]した。	たちまち= 
\\	彼は衣類をことごとく質に入れてしまった。	
\\	彼[かれ]は 衣類[いるい]をことごとく 質[しち]に 入[い]れてしまった。	ことごとく= 
\\	質に入れる=しちにいれる= 
\\	所持金をことごとく使い果たした。	
\\	所持[しょじ] 金[きん]をことごとく 使い果[つかいは]たした。	所持金=しょじきん= 
\\	ことごとく= 
\\	彼らはことごとく戦死した。	
\\	彼[かれ]らはことごとく 戦死[せんし]した。	ことごとく= 
\\	彼が立てた計画はことごとく失敗に終わった。	
\\	彼[かれ]が 立[た]てた 計画[けいかく]はことごとく 失敗[しっぱい]に 終[お]わった。	ことごとく= 
\\	夜ドライブしていて車がぬかるみにのめり込んでしまった。	
\\	夜[よる]ドライブしていて 車[くるま]がぬかるみにのめり 込[こ]んでしまった。	のめり込む= 
\\	僕は高校時代はサッカーにのめり込んでいた。	
\\	僕[ぼく]は 高校[こうこう] 時代[じだい]はサッカーにのめり 込[こ]んでいた。	のめり込む= 
\\	作品を読んで、私はますます三島由紀夫にのめり込んでしまった。	
\\	作品[さくひん]を 読[よ]んで、 私[わたし]はますます 三島[みしま] 由紀夫[ゆきお]にのめり 込[こ]んでしまった。	のめり込む= 
\\	両親も彼を持て余している。	
\\	両親[りょうしん]も 彼[かれ]を 持て余[もてあま]している。	持て余す=もてあます= 
\\	定年後私は時間を持て余していた。	
\\	定年[ていねん] 後[ご] 私[わたし]は 時間[じかん]を 持て余[もてあま]していた。	持て余す=もてあます= 
\\	二人の仲はだんだん疎遠になった。	
\\	二人[ふたり]の 仲[なか]はだんだん 疎遠[そえん]になった。	疎遠=そえん= 
\\	どうしてあの二人の仲はあんなに疎遠になったのか。	
\\	どうしてあの 二人[ふたり]の 仲[なか]はあんなに 疎遠[そえん]になったのか。	疎遠=そえん= 
\\	今さら過去のことを暴き出したって仕方がない。	
\\	今[いま]さら 過去[かこ]のことを 暴き出[あばきだ]したって 仕方[しかた]がない。	暴き出す=あばきだす= 
\\	彼は口を閉じて黙り込んでいた。	
\\	彼[かれ]は 口[くち]を 閉[と]じて 黙り込[だまりこ]んでいた。	
\\	厳選した旅をご用意しております。	
\\	厳選[げんせん]した 旅[たび]をご 用意[ようい]しております。	厳選=げんせん= 
\\	私は子どもに約束違反だと責められた。	
\\	私[わたし]は 子[こ]どもに 約束[やくそく] 違反[いはん]だと 責[せ]められた。	違反=いはん= 
\\	私はどうして子供の気持ちに早く気が付いてあげられなかったんだろうと自分を責めた。	
\\	私[わたし]はどうして 子供[こども]の 気持[きも]ちに 早[はや]く 気[き]が 付[つ]いてあげられなかったんだろうと 自分[じぶん]を 責[せ]めた。	
\\	事前に書類を用意した。	
\\	事前[じぜん]に 書類[しょるい]を 用意[ようい]した。	事前=じぜん= 
\\	事前に準備することがとても大切だ。	
\\	事前[じぜん]に 準備[じゅんび]することがとても 大切[たいせつ]だ。	事前=じぜん= 
\\	彼らは会社の中核を成している。	
\\	彼[かれ]らは 会社[かいしゃ]の 中核[ちゅうかく]を 成[な]している。	成す= 
\\	イギリスと日本は海に囲まれた島国だ。	
\\	イギリスと日本は 海[うみ]に 囲[かこ]まれた 島国[しまぐに]だ。	
\\	彼女は友達を部屋に通した。	
\\	彼女[かのじょ]は 友達[ともだち]を 部屋[へや]に 通[とお]した。	
\\	ちなみに、彼はまだ独身です。	
\\	ちなみに、 彼[かれ]はまだ 独身[どくしん]です。	
\\	で車両の位置を特定する。	
\\	で 車両[しゃりょう]の 位置[いち]を 特定[とくてい]する。	車両=しゃりょう= 
\\	特定する= 
\\	彼はデータを表にまとめた。	
\\	彼[かれ]はデータを 表[ひょう]にまとめた。	
\\	女の子が母親のすぐ側に座っている。	
\\	女の子[おんなのこ]が 母親[ははおや]のすぐ 側[そば]に 座[すわ]っている。	
\\	彼女自らが失敗を招いた。	
\\	彼女[かのじょ] 自[みずか]らが 失敗[しっぱい]を 招[まね]いた。	
\\	彼は相当英語がうまい。	
\\	彼[かれ]は 相当[そうとう] 英語[えいご]がうまい。	相当= 
\\	法律には従わなければならない。	
\\	法律[ほうりつ]には 従[したが]わなければならない。	
\\	それがまさに聞きたかったことなんです。	
\\	それがまさに 聞[き]きたかったことなんです。	
\\	ずいぶん長い間実家に戻っていない。	
\\	ずいぶん 長[なが]い 間[あいだ] 実家[じっか]に 戻[もど]っていない。	実家= 
\\	量よりむしろ質の方が大切だ。	
\\	量[りょう]よりむしろ 質[しつ]の 方[ほう]が 大切[たいせつ]だ。	
\\	最近自分のウェブサイトを作成しました。	
\\	最近[さいきん] 自分[じぶん]のウェブサイトを 作成[さくせい]しました。	
\\	政府は国民の安全を確保しなければならない。	
\\	政府[せいふ]は 国民[こくみん]の 安全[あんぜん]を 確保[かくほ]しなければならない。	
\\	この荷物を車に運んでください。	
\\	この 荷物[にもつ]を 車[くるま]に 運[はこ]んでください。	
\\	私は毎日ほぼ同じ時間に起きます。	
\\	私[わたし]は 毎日[まいにち]ほぼ 同[おな]じ 時間[じかん]に 起[お]きます。	
\\	掃除やら洗濯やらやることが多い。	
\\	掃除[そうじ]やら 洗濯[せんたく]やらやることが 多[おお]い。	
\\	この新しいモデルの方が機能が多い。	
\\	この 新[あたら]しいモデルの 方[ほう]が 機能[きのう]が 多[おお]い。	
\\	彼の作文は多少の間違いを除いて素晴らしかった。	
\\	彼[かれ]の 作文[さくぶん]は 多少[たしょう]の 間違[まちが]いを 除[のぞ]いて 素晴[すば]らしかった。	
\\	それはまだ準備の段階にある。	
\\	それはまだ 準備[じゅんび]の 段階[だんかい]にある。	
\\	現職候補に過半数の票が集まった。	
\\	現職[げんしょく] 候補[こうほ]に 過半数[かはんすう]の 票[ひょう]が 集[あつ]まった。	現職=現在、ある職務に就いていること。また、その職業や職務。
\\	かなりの票が対立候補に流れた。	
\\	かなりの 票[ひょう]が 対立[たいりつ] 候補[こうほ]に 流[なが]れた。	
\\	あの失言で票が逃げたのだ。	
\\	あの 失言[しつげん]で 票[ひょう]が 逃[に]げたのだ。	失言=言うべきではないことを、うっかり言ってしまうこと。また、その言葉。 うっかり= 
\\	今回の選挙は票が読みにくい。	
\\	今回[こんかい]の 選挙[せんきょ]は 票[ひょう]が 読[よ]みにくい。	
\\	殉難者追悼式は昨日大変厳かに執り行われた。	
\\	殉難[じゅんなん] 者[しゃ] 追悼[ついとう] 式[しき]は 昨日[きのう] 大変[たいへん] 厳[おごそ]かに 執り行[とりおこな]われた。	殉難者=じゅんなんしゃ= 
\\	追悼式=ついとうしき= 
\\	執り行う=とりおこなう= 
\\	急な仕事が入ってしまった。	
\\	急[きゅう]な 仕事[しごと]が 入[はい]ってしまった。	
\\	彼女は挨拶もなしに部屋に入った。	
\\	彼女[かのじょ]は 挨拶[あいさつ]もなしに 部屋[へや]に 入[はい]った。	
\\	その絵を壁に掛けてくれませんか。	
\\	その 絵[え]を 壁[かべ]に 掛[か]けてくれませんか。	
\\	その事実が知れたら大問題になるだろう。	
\\	その 事実[じじつ]が 知[し]れたら 大[だい] 問題[もんだい]になるだろう。	知れる= 
\\	彼は画家でありかつ作家でもある。	
\\	彼[かれ]は 画家[がか]でありかつ 作家[さっか]でもある。	かつ= 
\\	彼女はいきなり会社をクビになった。	
\\	彼女[かのじょ]はいきなり 会社[かいしゃ]をクビになった。	
\\	伯父さんからお小遣いをもらった。	
\\	伯父[おじ]さんからお 小遣[こづか]いをもらった。	
\\	彼らは良い結果を生んだ。	
\\	彼[かれ]らは 良[よ]い 結果[けっか]を 生[う]んだ。	
\\	この絵本は本来子供向けに書かれた。	
\\	この 絵本[えほん]は 本来[ほんらい] 子供[こども] 向[む]けに 書[か]かれた。	
\\	窓から山々を眺めることが出来る。	
\\	窓[まど]から 山々[やまやま]を 眺[なが]めることが 出来[でき]る。	
\\	音楽に合わせて手をたたこう!	
\\	音楽[おんがく]に 合[あ]わせて 手[て]をたたこう!	
\\	何とかして時間までに行きます。	
\\	何[なん]とかして 時間[じかん]までに 行[い]きます。	
\\	通常は8時から営業しています。	
\\	通常[つうじょう]は8 時[じ]から 営業[えいぎょう]しています。	通常=つうじょう= 
\\	会場に怒号が飛び交った。	
\\	会場[かいじょう]に 怒号[どごう]が 飛び交[とびか]った。	怒号=どごう= 
\\	飛び交う=とびかう= 
\\	開票が進むにつれ、野党の優勢が色濃くなってきた。	
\\	開票[かいひょう]が 進[すす]むにつれ、 野党[やとう]の 優勢[ゆうせい]が 色濃[いろこ]くなってきた。	開票= 
\\	野党=やとう= 
\\	優勢=ゆうせい= 
\\	色濃い=いろこい= 
\\	開票の結果、鈴木氏が当選した。	
\\	開票[かいひょう]の 結果[けっか]、 鈴木[すずき] 氏[し]が 当選[とうせん]した。	開票= 
\\	私は色恋の話は興味がない。	
\\	私[わたし]は 色恋[いろこい]の 話[はなし]は 興味[きょうみ]がない。	色恋=いろこい= 
\\	彼の方が得票が少なかった。	
\\	彼[かれ]の 方[ほう]が 得票[とくひょう]が 少[すく]なかった。	得票= 
\\	彼女の得票は10万票であった。	
\\	彼女[かのじょ]の 得票[とくひょう]は10 万[まん] 票[ひょう]であった。	得票= 
\\	学校ではうわさが飛び交っている。	
\\	学校[がっこう]ではうわさが 飛び交[とびか]っている。	飛び交う= 
\\	ウエーターから伝票を受け取った。	
\\	ウエーターから 伝票[でんぴょう]を 受け取[うけと]った。	
\\	馬は物音に驚いて急に暴れ出すことがある。	
\\	馬[うま]は 物音[ものおと]に 驚[おどろ]いて 急[きゅう]に 暴れ出[あばれだ]すことがある。	暴れる=あばれる= 
\\	学生が暴れて数台のバスに火をつけた。	
\\	学生[がくせい]が 暴[あば]れて 数[すう] 台[だい]のバスに 火[ひ]をつけた。	暴れる=あばれる= 
\\	この子は思うようにならないと、すぐ泣いて暴れる。	
\\	この 子[こ]は 思[おも]うようにならないと、すぐ 泣[な]いて 暴[あば]れる。	暴れる=あばれる= 
\\	いまさら恋人の過去を暴き立てて何になる?	
\\	いまさら 恋人[こいびと]の 過去[かこ]を 暴[あば]き 立[た]てて 何[なに]になる?	暴く=あばく= 
\\	その雑誌は、社会の不正・虚偽を暴き出すことを目的として創刊された。	
\\	その 雑誌[ざっし]は、 社会[しゃかい]の 不正[ふせい]・ 虚偽[きょぎ]を 暴き出[あばきだ]すことを 目的[もくてき]として 創刊[そうかん]された。	不正=ふせい= 
\\	虚偽=きょぎ= 
\\	暴く=あばく= 
\\	手頃な値段でおいしい料理が味わえる。	
\\	手頃[てごろ]な 値段[ねだん]でおいしい 料理[りょうり]が 味[あじ]わえる。	手頃=てごろ= 
\\	駅へ通じる道はいつも混雑している。	
\\	駅[えき]へ 通[つう]じる 道[みち]はいつも 混雑[こんざつ]している。	
\\	日々の生活にストレスを感じる人は多い。	
\\	日々[ひび]の 生活[せいかつ]にストレスを 感[かん]じる 人[ひと]は 多[おお]い。	
\\	私たちは共通の友人を通じて知り合いました。	
\\	私[わたし]たちは 共通[きょうつう]の 友人[ゆうじん]を 通[つう]じて 知り合[しりあ]いました。	
\\	この貴重な経験を生かすつもりです。	
\\	この 貴重[きちょう]な 経験[けいけん]を 生[い]かすつもりです。	
\\	何年もの交渉の末、二人の離婚は成立した。	
\\	何[なん] 年[ねん]もの 交渉[こうしょう]の 末[すえ]、二 人[にん]の 離婚[りこん]は 成立[せいりつ]した。	
\\	冗談にしてもそれはひどすぎる。	
\\	冗談[じょうだん]にしてもそれはひどすぎる。	
\\	靴のひもは自分で結べるよね。	
\\	靴[くつ]のひもは 自分[じぶん]で 結[むす]べるよね。	
\\	これは従来の方法よりも効果的だろう。	
\\	これは 従来[じゅうらい]の 方法[ほうほう]よりも 効果[こうか] 的[てき]だろう。	
\\	家族の写真を額に入れた。	
\\	家族[かぞく]の 写真[しゃしん]を 額[がく]に 入[い]れた。	額=がく= 
\\	彼は顔の汗を拭った。	
\\	彼[かれ]は 顔[かお]の 汗[あせ]を 拭[ぬぐ]った。	拭う=ぬぐう= 
\\	天候の変化が農業に深刻な被害をもたらしている。	
\\	天候[てんこう]の 変化[へんか]が 農業[のうぎょう]に 深刻[しんこく]な 被害[ひがい]をもたらしている。	
\\	この機会を活用すべきだ。	
\\	この 機会[きかい]を 活用[かつよう]すべきだ。	
\\	正直な気持ちを聞かせてほしい。	
\\	正直[しょうじき]な 気持[きも]ちを 聞[き]かせてほしい。	
\\	彼はこの仕事は初めてだというがそれなりにうまくやっている。	
\\	彼[かれ]はこの 仕事[しごと]は 初[はじ]めてだというがそれなりにうまくやっている。	それなり= 
\\	彼女の言うこともそれなりに説得力がある。	
\\	彼女[かのじょ]の 言[い]うこともそれなりに 説得[せっとく] 力[りょく]がある。	それなり= 
\\	そのニュースで会社の株価は暴落しました。	
\\	そのニュースで 会社[かいしゃ]の 株価[かぶか]は 暴落[ぼうらく]しました。	暴落= 
\\	この問題は彼の裁量に委ねることにしよう。	
\\	この 問題[もんだい]は 彼[かれ]の 裁量[さいりょう]に 委[ゆだ]ねることにしよう。	裁量=さいりょう= 
\\	委ねる= 
\\	丘の上から一筋の川が見えた。	
\\	丘[おか]の 上[うえ]から 一筋[ひとすじ]の 川[かわ]が 見[み]えた。	一筋=ひとすじ= 
\\	その部屋で見つかった一筋の髪の毛が犯人を特定する決め手となった。	
\\	その 部屋[へや]で 見[み]つかった 一筋[ひとすじ]の 髪の毛[かみのけ]が 犯人[はんにん]を 特定[とくてい]する 決め手[きめて]となった。	一筋=ひとすじ= 
\\	特定する= 
\\	決め手= 
\\	涙が一筋彼女の頬を伝った。	
\\	涙[なみだ]が 一筋[ひとすじ] 彼女[かのじょ]の 頬[ほお]を 伝[つた]った。	一筋=ひとすじ= 
\\	頬=ほお= 
\\	父は教育一筋に生きた。	
\\	父[ちち]は 教育[きょういく] 一筋[ひとすじ]に 生[い]きた。	一筋=ひとすじ= 
\\	彼はただ学問一筋に生きてきた人だ。	
\\	彼[かれ]はただ 学問[がくもん] 一筋[ひとすじ]に 生[い]きてきた 人[ひと]だ。	一筋=ひとすじ= 
\\	彼女が舞台に現れると観客から拍手が沸き起こった。	
\\	彼女[かのじょ]が 舞台[ぶたい]に 現[あらわ]れると 観客[かんきゃく]から 拍手[はくしゅ]が 沸き起[わきお]こった。	
\\	現地の言葉を知っていたら旅行はもっと楽しくなる。	
\\	現地[げんち]の 言葉[ことば]を 知[し]っていたら 旅行[りょこう]はもっと 楽[たの]しくなる。	
\\	人がミスをするのはしようがない。	
\\	人[ひと]がミスをするのはしようがない。	
\\	日本人にとって米は最も身近な食べ物だ。	
\\	日本人[にほんじん]にとって 米[べい]は 最[もっと]も 身近[みぢか]な 食べ物[たべもの]だ。	身近=みぢか= 
\\	その車は詳細な検査に合格した。	
\\	その 車[くるま]は 詳細[しょうさい]な 検査[けんさ]に 合格[ごうかく]した。	
\\	すべての人は対等の権利を有する。	
\\	すべての 人[ひと]は 対等[たいとう]の 権利[けんり]を 有[ゆう]する。	対等=たいとう= 
\\	有する= 
\\	彼はワインの瓶からコルクを抜いた。	
\\	彼[かれ]はワインの 瓶[びん]からコルクを 抜[ぬ]いた。	
\\	ストレスのせいで髪が抜けてしまった。	
\\	ストレスのせいで 髪[かみ]が 抜[ぬ]けてしまった。	抜ける= 
\\	この車は日本の仕様に合わせて作られた。	
\\	この 車[くるま]は 日本[にほん]の 仕様[しよう]に 合[あ]わせて 作[つく]られた。	仕様=しよう= 
\\	話し合いの場を設ける必要がある。	
\\	話し合[はなしあ]いの 場[ば]を 設[もう]ける 必要[ひつよう]がある。	設ける=もうける= 
\\	このコンピューターはデータの処理が遅い。	
\\	このコンピューターはデータの 処理[しょり]が 遅[おそ]い。	
\\	これはあくまで私個人の意見です。	
\\	これはあくまで 私[わたし] 個人[こじん]の 意見[いけん]です。	
\\	彼は首を吊って自殺した。	
\\	彼[かれ]は 首[くび]を 吊[つ]って 自殺[じさつ]した。	吊る=つる= 
\\	足がつった。	
\\	足[あし]がつった。	攣る=つる= 
\\	彼の夢がついに実現した。	
\\	彼[かれ]の 夢[ゆめ]がついに 実現[じつげん]した。	
\\	大地震が起きたが、家族は皆無事だった。	
\\	大[だい] 地震[じしん]が 起[お]きたが、 家族[かぞく]は 皆[みな] 無事[ぶじ]だった。	
\\	君の言うことは筋道が立たない。	
\\	君[きみ]の 言[い]うことは 筋道[すじみち]が 立[た]たない。	筋道=すじみち= 
\\	彼のやることは少しも筋道が立っていない。	
\\	彼[かれ]のやることは 少[すこ]しも 筋道[すじみち]が 立[た]っていない。	筋道=すじみち= 
\\	万事筋書き通りに運んだ。	
\\	万事[ばんじ] 筋書[すじが]き 通[どお]りに 運[はこ]んだ。	万事=ばんじ= 
\\	筋書き通りに= 
\\	人生は筋書き通りに行くもんじゃない。	
\\	人生[じんせい]は 筋書[すじが]き 通[どお]りに 行[い]くもんじゃない。	筋書き通りに= 
\\	サッカーは筋書きのないドラマだ。	
\\	サッカーは 筋書[すじが]きのないドラマだ。	筋書き= 
\\	床の細かい砂は洗い流した方が掃いてとるより簡単だ。	
\\	床[ゆか]の 細[こま]かい 砂[すな]は 洗い流[あらいなが]した 方[ほう]が 掃[は]いてとるより 簡単[かんたん]だ。	洗い流す= 
\\	過去の諍いは過去のこととして洗い流し、これからは協力やって行きましょう。	
\\	過去[かこ]の 諍[いさか]いは 過去[かこ]のこととして 洗い流[あらいなが]し、これからは 協力[きょうりょく]やって 行[い]きましょう。	諍い=いさかい= 
\\	洗い流す= 
\\	3日続いた大雨のため川の水が増し、水の勢いで橋が洗い流された。	
\\	3日[みっか] 続[つづ]いた 大雨[おおあめ]のため 川[かわ]の 水[みず]が 増[ま]し、 水[みず]の 勢[いきお]いで 橋[はし]が 洗い流[あらいなが]された。	洗い流す= 
\\	このカードの中から好きなのを4枚選び取ってください。	
\\	このカードの 中[なか]から 好[す]きなのを 4枚[よんまい] 選[えら]び 取[と]ってください。	
\\	真ん中は完全に禿げ上がって毛が一本もない。	
\\	真ん中[まんなか]は 完全[かんぜん]に 禿[は]げ 上[あ]がって 毛[け]が 一本[いっぽん]もない。	禿げ上がる=はげあがる= 
\\	電車の中で若いハンサムな男性に話しかけられてどぎまぎしてしまった。	
\\	電車[でんしゃ]の 中[なか]で 若[わか]いハンサムな 男性[だんせい]に 話[はな]しかけられてどぎまぎしてしまった。	どぎまぎ= 
\\	彼女は日本か韓国人か中国人か分からなかったが、試しに日本語で話しかけてみた。	
\\	彼女[かのじょ]は 日本[にほん]か 韓国[かんこく] 人[じん]か 中国人[ちゅうごくじん]か 分[わ]からなかったが、 試[ため]しに 日本語[にほんご]で 話[はな]しかけてみた。	試し=ためし= 
\\	代金はスイスの銀行口座に直接払い込む仕組みになっている。	
\\	代金[だいきん]はスイスの 銀行[ぎんこう] 口座[こうざ]に 直接[ちょくせつ] 払い込[はらいこ]む 仕組[しく]みになっている。	代金=だいきん= 
\\	仕組み=しくみ= 
\\	賛成する人が少ないと見ると、彼は自分の案を簡単に引き下げた。	
\\	賛成[さんせい]する 人[ひと]が 少[すく]ないと 見[み]ると、 彼[かれ]は 自分[じぶん]の 案[あん]を 簡単[かんたん]に 引き下[ひきさ]げた。	引き下げる= 
\\	テレビは若い人を引きつけておくだけの魅力がない。	
\\	テレビは 若[わか]い 人[ひと]を 引[ひ]きつけておくだけの 魅力[みりょく]がない。	
\\	今日午後クラスの最中に、生徒の一人が突然引きつけを起こしてみんなが驚いた。	
\\	今日[きょう] 午後[ごご]クラスの 最中[さいちゅう]に、 生徒[せいと]の一 人[にん]が 突然[とつぜん] 引[ひ]きつけを 起[お]こしてみんなが 驚[おどろ]いた。	引きつける= 
\\	てんかんの患者はこの薬のおかげで引きつけを起こす回数がずっと少なくなった。	
\\	てんかんの 患者[かんじゃ]はこの 薬[くすり]のおかげで 引[ひ]きつけを 起[お]こす 回数[かいすう]がずっと 少[すく]なくなった。	癲癇=てんかん= 
\\	引きつける= 
\\	叔父がたばこの煙を8の字型にして吹き出すのに見とれていた。	
\\	叔父[おじ]がたばこの 煙[けむり]を 8[はち]の 字[じ] 型[がた]にして 吹き出[ふきだ]すのに 見[み]とれていた。	叔父=おじ= 
\\	見とれる= 
\\	彼の額に汗が吹き出た。	
\\	彼[かれ]の 額[がく]に 汗[あせ]が 吹き出[ふきで]た。	
\\	家の息子に危険思想を吹き込まないでください。	
\\	家[うち]の 息子[むすこ]に 危険[きけん] 思想[しそう]を 吹き込[ふきこ]まないでください。	
\\	一歩中へ踏み込んだところをドアの後ろに隠れていた男にこん棒で殴られた。	
\\	一歩[いっぽ] 中[なか]へ 踏み込[ふみこ]んだところをドアの 後[うし]ろに 隠[かく]れていた 男[おとこ]にこん 棒[ぼう]で 殴[なぐ]られた。	踏み込む= 
\\	こん棒=こんぼう= 
\\	たしかでないことを言い張るのはよくない。	
\\	たしかでないことを 言い張[いいは]るのはよくない。	言い張る= 
\\	世の中には「白だ」「黒だ」とはっきり言い切ることが出来ないことがたくさんある。	
\\	世の中[よのなか]には
\\	白[しろ]だ」
\\	黒[くろ]だ」とはっきり 言い切[いいき]ることが 出来[でき]ないことがたくさんある。	
\\	彼を「有罪だ」と言い切る絶対的証拠がない。	
\\	彼[かれ]を
\\	有罪[ゆうざい]だ」と 言[い]い 切[き]る 絶対[ぜったい] 的[てき] 証拠[しょうこ]がない。	
\\	寝坊してあわてて学校へ駆けつけた時はもう授業が始まっていた。	
\\	寝坊[ねぼう]してあわてて 学校[がっこう]へ 駆[か]けつけた 時[とき]はもう 授業[じゅぎょう]が 始[はじ]まっていた。	
\\	あの学生は教授の講義を一字一句もらさずノートに書き取る。	
\\	あの 学生[がくせい]は 教授[きょうじゅ]の 講義[こうぎ]を 一字[いちじ] 一句[いっく]もらさずノートに 書き取[かきと]る。	一字一句=いちじいっく= 
\\	漏らす= 
\\	宝くじの当選番号を読み上げますから、書き取ってください。	
\\	宝[たから]くじの 当選[とうせん] 番号[ばんごう]を 読み上[よみあ]げますから、 書き取[かきと]ってください。	
\\	声は十分大きいのだが、発音が不明瞭なので、何を言っているのか聞き取りにくい。	
\\	声[こえ]は 十分[じゅうぶん] 大[おお]きいのだが、 発音[はつおん]が 不明瞭[ふめいりょう]なので、 何[なに]を 言[い]っているのか 聞き取[ききと]りにくい。	不明瞭=ふめいりょう= 
\\	最近の就職難は不況の影響が色濃い。	
\\	最近[さいきん]の 就職[しゅうしょく] 難[なん]は 不況[ふきょう]の 影響[えいきょう]が 色濃[いろこ]い。	色濃い=いろこい= 
\\	自殺者の増加は不安な時代の様相を色濃く反映している。	
\\	自殺[じさつ] 者[しゃ]の 増加[ぞうか]は 不安[ふあん]な 時代[じだい]の 様相[ようそう]を 色濃[いろこ]く 反映[はんえい]している。	様相=ようそう= 
\\	色濃い=いろこい= 
\\	彼はとても疑い深い性格だ。	
\\	彼[かれ]はとても 疑い深[うたがいぶか]い 性格[せいかく]だ。	疑い深い=うたがいぶかい= 
\\	親友に裏切られて彼女は疑い深くなった。	
\\	親友[しんゆう]に 裏切[うらぎ]られて 彼女[かのじょ]は 疑い深[うたがいぶか]くなった。	疑い深い=うたがいぶかい= 
\\	あの人が本当にそんなことをしたかどうか疑わしい。	
\\	あの 人[ひと]が 本当[ほんとう]にそんなことをしたかどうか 疑[うたが]わしい。	疑わしい= 
\\	彼の挙動を疑わしく思った。	
\\	彼[かれ]の 挙動[きょどう]を 疑[うたが]わしく 思[おも]った。	挙動=きょどう= 
\\	疑わしい= 
\\	その効果のほどは疑わしい。	
\\	その 効果[こうか]のほどは 疑[うたが]わしい。	疑わしい= 
\\	「本当ですか」と彼は疑わしげに言った。	
\\	本当[ほんとう]ですか」と 彼[かれ]は 疑[うたが]わしげに 言[い]った。	疑わしげに= 
\\	彼の話に疑わしさは残るが、犯罪の立証も出来ない。	
\\	彼[かれ]の 話[はなし]に 疑[うたが]わしさは 残[のこ]るが、 犯罪[はんざい]の 立証[りっしょう]も 出来[でき]ない。	疑わしい= 
\\	立証=りっしょう= 
\\	これは久しく思い出さなかったことだ。	
\\	これは 久[ひさ]しく 思い出[おもいだ]さなかったことだ。	久しい=ひさしい= 
\\	彼女のうわさは久しく聞かない。	
\\	彼女[かのじょ]のうわさは 久[ひさ]しく 聞[き]かない。	久しい=ひさしい= 
\\	ずいぶん久しくお目にかかりませんでしたね。	
\\	ずいぶん 久[ひさ]しくお 目[め]にかかりませんでしたね。	久しい=ひさしい= 
\\	この不景気の中、仕事にありつける可能性は低い。	
\\	この 不景気[ふけいき]の 中[なか]、 仕事[しごと]にありつける 可能[かのう] 性[せい]は 低[ひく]い。	ありつく= 
\\	歌手と言ってもピンからキリまでだ。	
\\	歌手[かしゅ]と 言[い]ってもピンからキリまでだ。	ピンからキリまで= 
\\	なぜ、あの男はうっかりと秘密を漏らしたのか。	
\\	なぜ、あの 男[おとこ]はうっかりと 秘密[ひみつ]を 漏[も]らしたのか。	うっかり= 
\\	そんなやり方では一時しのぎしかならない。	
\\	そんなやり 方[かた]では 一時[いちじ]しのぎしかならない。	一時しのぎ= 
\\	雨漏りの個所には一時しのぎに青いビニールシートがかぶせてあった。	
\\	雨漏[あまも]りの 個所[かしょ]には 一時[いちじ]しのぎに 青[あお]いビニールシートがかぶせてあった。	個所=かしょ= 
\\	一時しのぎ= 
\\	かぶせる= 
\\	一時しのぎのつもりで借りた金が雪だるま式にふくらんでしまった。	
\\	一時[いちじ]しのぎのつもりで 借[か]りた 金[きむ]が 雪[ゆき]だるま 式[しき]にふくらんでしまった。	一時しのぎ= 
\\	雪だるま式= 
\\	膨らむ=ふくらむ= 
\\	普段何気なくしていることも、文化を異にする人から見ると新鮮に映るらしい。	
\\	普段[ふだん] 何気[なにげ]なくしていることも、 文化[ぶんか]を 異[こと]にする 人[ひと]から 見[み]ると 新鮮[しんせん]に 映[うつ]るらしい。	異にする=こと にする= 
\\	映る=うつる= 
\\	何気なく入ったのだが、意外に感じのいい店だった。	
\\	何気[なにげ]なく 入[はい]ったのだが、 意外[いがい]に 感[かん]じのいい 店[みせ]だった。	
\\	何気なく目を上げると見知らぬ男がじっと私を見ていた。	
\\	何気[なにげ]なく 目[め]を 上[あ]げると 見知[みし]らぬ 男[おとこ]がじっと 私[わたし]を 見[み]ていた。	
\\	誰かが何気なく言った一言がいつまでも心に残ることがある。	
\\	誰[だれ]かが 何気[なにげ]なく 言[い]った 一言[ひとこと]がいつまでも 心[こころ]に 残[のこ]ることがある。	
\\	ここまで来てむざむざ引き返すわけにはいかない。	
\\	ここまで 来[き]てむざむざ 引き返[ひきかえ]すわけにはいかない。	引き返す= 
\\	その計画は実施寸前になってむざむざ取りやめとなった。	
\\	その 計画[けいかく]は 実施[じっし] 寸前[すんぜん]になってむざむざ 取[と]りやめとなった。	寸前=すんぜん= 
\\	その映画館は名作映画を日替わりで上映している。	
\\	その 映画[えいが] 館[かん]は 名作[めいさく] 映画[えいが]を 日替[ひが]わりで 上映[じょうえい]している。	名作=めいさく= 
\\	日替わり= 
\\	(~の); 
\\	上映=じょうえい= 
\\	目的地天候不良につき 
\\	便は本日欠航となります。	
\\	目的[もくてき] 地[ち] 天候[てんこう] 不良[ふりょう]につき 
\\	便[びん]は 本日[ほんじつ] 欠航[けっこう]となります。	
\\	荒天のため、欠航する便が相次いだ。	
\\	荒天[こうてん]のため、 欠航[けっこう]する 便[びん]が 相次[あいつ]いだ。	荒天=こうてん= 
\\	この国の人口問題には果たして解答が出るのだろうか。	
\\	この 国[くに]の 人口[じんこう] 問題[もんだい]には 果[は]たして 解答[かいとう]が 出[で]るのだろうか。	解答=かいとう= (問題に答えること) 
\\	(問題に対する答え) 
\\	(答え)
\\	この種の少年非行は容易に解答が出ない問題である。	
\\	この 種[しゅ]の 少年[しょうねん] 非行[ひこう]は 容易[ようい]に 解答[かいとう]が 出[で]ない 問題[もんだい]である。	非行=ひこう= 
\\	解答=かいとう= (問題に答えること) 
\\	(問題に対する答え) 
\\	(答え)
\\	市民からの申し入れに対し未だに何の回答もない。	
\\	市民[しみん]からの 申し入[もうしい]れに 対[たい]し 未[いま]だに 何[なに]の 回答[かいとう]もない。	申し入れ= 
\\	回答=かいとう= (問い合わせに対する) 
\\	(要求に対する) 
\\	(解答)
\\	会社側は組合からの賃上げ要求に対する回答として3%を提示した。	
\\	会社[かいしゃ] 側[がわ]は 組合[くみあい]からの 賃上[ちんあ]げ 要求[ようきゅう]に 対[たい]する 回答[かいとう]として3 
\\	[ぱーせんと]を 提示[ていじ]した。	回答=かいとう= (問い合わせに対する) 
\\	(要求に対する) 
\\	(解答)
\\	枝葉末節の問題は切り捨てるべきだ。	
\\	枝葉[しよう] 末節[まっせつ]の 問題[もんだい]は 切り捨[きりす]てるべきだ。	
\\	その国の外交政策は支離滅裂だった。	
\\	その 国[くに]の 外交[がいこう] 政策[せいさく]は 支離滅裂[しりめつれつ]だった。	
\\	その問題で党内は四分五裂している。	
\\	その 問題[もんだい]で 党内[とうない]は 四分五裂[しぶんごれつ]している。	四分五裂=しぶんごれつ= 
\\	高齢化社会に相応した雇用対策が必要だ。	
\\	高齢[こうれい] 化[か] 社会[しゃかい]に 相応[そうおう]した 雇用[こよう] 対策[たいさく]が 必要[ひつよう]だ。	相応=そうおう= 
\\	営業成績が上がればそれに相応した賃金が支払われる。	
\\	営業[えいぎょう] 成績[せいせき]が 上[あ]がればそれに 相応[そうおう]した 賃金[ちんぎん]が 支払[しはら]われる。	相応=そうおう= 
\\	賃金=ちんぎん= 
\\	この記録は前代未聞だ。	
\\	この 記録[きろく]は 前代未聞[ぜんだいみもん]だ。	
\\	国内秩序の乱れは前代未聞であった。	
\\	国内[こくない] 秩序[ちつじょ]の 乱[みだ]れは 前代未聞[ぜんだいみもん]であった。	秩序=ちつじょ=1)物事を行う場合の正しい順序・筋道。         
\\	その社会・団体などが、望ましい状態を保つための順序や決まり。
\\	3年前に退職して今は悠々自適の毎日です。	
\\	年[ねん] 前[まえ]に 退職[たいしょく]して 今[いま]は 悠々自適[ゆうゆうじてき]の 毎日[まいにち]です。	悠々自適=ゆうゆうじてき= 
\\	私たち兄弟は懸命に働く両親の姿を見て育った。	
\\	私[わたし]たち 兄弟[きょうだい]は 懸命[けんめい]に 働[はたら]く 両親[りょうしん]の 姿[すがた]を 見[み]て 育[そだ]った。	
\\	彼のこれまでの懸命の努力が実を結んだ。	
\\	彼[かれ]のこれまでの 懸命[けんめい]の 努力[どりょく]が 実[み]を 結[むす]んだ。	実を結ぶ=み を むすぶ= 
\\	人生ははかないものだ。	
\\	人生[じんせい]ははかないものだ。	はかない= 
\\	老後は故郷に安住できたらいいな。	
\\	老後[ろうご]は 故郷[こきょう]に 安住[あんじゅう]できたらいいな。	安住= 
\\	運転免許証の期限が来週で切れる。	
\\	運転[うんてん] 免許[めんきょ] 証[しょう]の 期限[きげん]が 来週[らいしゅう]で 切[き]れる。	
\\	そもそもそんなことをするべきではなかった。	
\\	そもそもそんなことをするべきではなかった。	
\\	少なくとも毎月一冊は本を読む。	
\\	少[すく]なくとも 毎月[まいつき] 一冊[いっさつ]は 本[ほん]を 読[よ]む。	
\\	その家は周囲を森に囲まれていた。	
\\	その 家[いえ]は 周囲[しゅうい]を 森[もり]に 囲[かこ]まれていた。	
\\	彼女は誰とでもすぐに仲良くなる。	
\\	彼女[かのじょ]は 誰[だれ]とでもすぐに 仲良[なかよ]くなる。	
\\	この写真の猫を見かけたら、お電話ください。	
\\	この 写真[しゃしん]の 猫[ねこ]を 見[み]かけたら、お 電話[でんわ]ください。	
\\	たとえ両親が反対しても私は行く。	
\\	たとえ 両親[りょうしん]が 反対[はんたい]しても 私[わたし]は 行[い]く。	
\\	その規則はこの場合には適用されない。	
\\	その 規則[きそく]はこの 場合[ばあい]には 適用[てきよう]されない。	
\\	私はあまり外国人と接する機会がない。	
\\	私[わたし]はあまり 外国[がいこく] 人[じん]と 接[せっ]する 機会[きかい]がない。	
\\	最近、この国では出生率が低下している。	
\\	最近[さいきん]、この 国[くに]では 出生[しゅっしょう] 率[りつ]が 低下[ていか]している。	
\\	この計画は何ら問題はない。	
\\	この 計画[けいかく]は 何[なん]ら 問題[もんだい]はない。	なんら= 
\\	予算に余裕がある。	
\\	予算[よさん]に 余裕[よゆう]がある。	
\\	メールにて結果をお知らせ下さい。	
\\	メールにて 結果[けっか]をお 知[し]らせ 下[くだ]さい。	
\\	彼の英語力は大きく向上した。	
\\	彼[かれ]の 英語[えいご] 力[りょく]は 大[おお]きく 向上[こうじょう]した。	
\\	この生地でスカートを作るつもりだ。	
\\	この 生地[きじ]でスカートを 作[つく]るつもりだ。	生地=きじ= 
\\	必要なものはすべて揃った。	
\\	必要[ひつよう]なものはすべて 揃[そろ]った。	揃う=そろう= 
\\	彼は助けを求めて叫んだ。	
\\	彼[かれ]は 助[たす]けを 求[もと]めて 叫[さけ]んだ。	
\\	教育の目的は人格の形成である。	
\\	教育[きょういく]の 目的[もくてき]は 人格[じんかく]の 形成[けいせい]である。	
\\	いよいよ私の番が来た。	
\\	いよいよ 私[わたし]の 番[ばん]が 来[き]た。	
\\	彼女はジャケットのボタンを外した。	
\\	彼女[かのじょ]はジャケットのボタンを 外[はず]した。	
\\	ドアに指を挟まれてしまった。	
\\	ドアに 指[ゆび]を 挟[はさ]まれてしまった。	
\\	鼻がつまっている。	
\\	鼻[はな]がつまっている。	
\\	パンにバターを塗った。	
\\	パンにバターを 塗[ぬ]った。	塗る=ぬる= 
\\	警官は彼の腕をつかんだ。	
\\	警官[けいかん]は 彼[かれ]の 腕[うで]をつかんだ。	
\\	ちょっとじっとしていてください。	
\\	ちょっとじっとしていてください。	
\\	彼は時々うなずきながら私の話を聞いた。	
\\	彼[かれ]は 時々[ときどき]うなずきながら 私[わたし]の 話[はなし]を 聞[き]いた。	うなずく= 
\\	コップを床に落としてしまった。	
\\	コップを 床[ゆか]に 落[お]としてしまった。	
\\	私たちは二人きりになった。	
\\	私[わたし]たちは 二人[ふたり]きりになった。	
\\	彼は 1992年に弁護士の資格を取った。	
\\	彼[かれ]は 
\\	年[ねん]に 弁護士[べんごし]の 資格[しかく]を 取[と]った。	
\\	ハンドルを両手でしっかり握ってください。	
\\	ハンドルを 両手[りょうて]でしっかり 握[にぎ]ってください。	握る=にぎる= 
\\	何か楽器をやってる?	
\\	何[なに]か 楽器[がっき]をやってる?	
\\	もっとびーるをどうですか。 
\\	じゃ、ほんの少しお願いします。	
\\	もっとびーるをどうですか。 
\\	じゃ、ほんの 少[すこ]しお 願[ねが]いします。	
\\	彼女のスピーチは見事だった。	
\\	彼女[かのじょ]のスピーチは 見事[みごと]だった。	
\\	彼女は上司をセクハラで訴えた。	
\\	彼女[かのじょ]は 上司[じょうし]をセクハラで 訴[うった]えた。	
\\	まだ状況が把握できていない。	
\\	まだ 状況[じょうきょう]が 把握[はあく]できていない。	把握=はあく= 
\\	この問題を別の観点から見てみましょう。	
\\	この 問題[もんだい]を 別[べつ]の 観点[かんてん]から 見[み]てみましょう。	
\\	彼は 70歳だがまだ両親が揃っている。	
\\	彼[かれ]は 
\\	歳[さい]だがまだ 両親[りょうしん]が 揃[そろ]っている。	揃う=そろう= 
\\	必要な書類が揃っていますか。	
\\	必要[ひつよう]な 書類[しょるい]が 揃[そろ]っていますか。	揃う=そろう= 
\\	全ての条件が揃った。	
\\	全[すべ]ての 条件[じょうけん]が 揃[そろ]った。	揃う=そろう= 
\\	全員同じ答えに揃った。	
\\	全員[ぜんいん] 同[おな]じ 答[こた]えに 揃[そろ]った。	揃う=そろう= 
\\	あの本屋には経済書が揃っている。	
\\	あの 本屋[ほんや]には 経済[けいざい] 書[しょ]が 揃[そろ]っている。	揃う=そろう= 
\\	追いつめられてとうとう彼の生地が出た。	
\\	追[お]いつめられてとうとう 彼[かれ]の 生地[きじ]が 出[で]た。	追いつめる= 
\\	生地=きじ= 
\\	それについては史書に何も記載してない。	
\\	それについては 史書[ししょ]に 何[なに]も 記載[きさい]してない。	記載=きさい= 
\\	ともかく準備だけはしておこう。	
\\	ともかく 準備[じゅんび]だけはしておこう。	
\\	被害の全体を把握するのに時間がかかった。	
\\	被害[ひがい]の 全体[ぜんたい]を 把握[はあく]するのに 時間[じかん]がかかった。	把握=はあく= 
\\	このような結果になり、誠に遺憾に存じます。	
\\	このような 結果[けっか]になり、 誠[まこと]に 遺憾[いかん]に 存[ぞん]じます。	遺憾=いかん= 
\\	このたびの他国による領空侵犯を大変遺憾に思う。	
\\	このたびの 他国[たこく]による 領空[りょうくう] 侵犯[しんぱん]を 大変[たいへん] 遺憾[いかん]に 思[おも]う。	領空侵犯=りょうくうしんぱん= 
\\	遺憾=いかん= 
\\	次の電車は8両編成で参ります。	
\\	次[つぎ]の 電車[でんしゃ]は8 両[りょう] 編成[へんせい]で 参[まい]ります。	両=りょう= 
\\	エレベーターは高速で下降した。	
\\	エレベーターは 高速[こうそく]で 下降[かこう]した。	下降=かこう= 
\\	株価はみるみる下降した。	
\\	株価[かぶか]はみるみる 下降[かこう]した。	みるみる= 
\\	下降=かこう= 
\\	室温はマイナス15度まで下降した。	
\\	室温[しつおん]はマイナス15 度[ど]まで 下降[かこう]した。	下降=かこう= 
\\	今月は販売成績が下降した。	
\\	今月[こんげつ]は 販売[はんばい] 成績[せいせき]が 下降[かこう]した。	下降=かこう= 
\\	最近彼は人気が下降ぎみだ。	
\\	最近[さいきん] 彼[かれ]は 人気[にんき]が 下降[かこう]ぎみだ。	下降=かこう= 
\\	気味=ぎみ= 
\\	その事件以来大統領の支持率は下降の一途をたどっている。	
\\	その 事件[じけん] 以来[いらい] 大統領[だいとうりょう]の 支持[しじ] 率[りつ]は 下降[かこう]の 一途[いっと]をたどっている。	下降=かこう= 
\\	一途=いっと= 
\\	たどる= 
\\	白旗は降伏の印だ。	
\\	白旗[しらはた]は 降伏[こうふく]の 印[しるし]だ。	降伏=こうふく= 
\\	白旗=しらはた= 
\\	彼女の死はフランス文学界にとって大きな損失だ。	
\\	彼女[かのじょ]の 死[し]はフランス 文学[ぶんがく] 界[かい]にとって 大[おお]きな 損失[そんしつ]だ。	
\\	彼の急逝は国家的損失と言っても過言でない。	
\\	彼[かれ]の 急逝[きゅうせい]は 国家[こっか] 的[てき] 損失[そんしつ]と 言[い]っても 過言[かごん]でない。	急逝=きゅうせい= 
\\	彼の信用は少しも損なわれなかった。	
\\	彼[かれ]の 信用[しんよう]は 少[すこ]しも 損[そこ]なわれなかった。	
\\	僕は留学し損なったんです。	
\\	僕[ぼく]は 留学[りゅうがく]し 損[そこ]なったんです。	
\\	50万円の欠損になっている。	
\\	万[まん] 円[えん]の 欠損[けっそん]になっている。	欠損= 
\\	(損失); 
\\	今年の天候不順が農家にとって打撃となった。	
\\	今年[ことし]の 天候[てんこう] 不順[ふじゅん]が 農家[のうか]にとって 打撃[だげき]となった。	
\\	母の死は彼には大きな打撃であった。	
\\	母[はは]の 死[し]は 彼[かれ]には 大[おお]きな 打撃[だげき]であった。	
\\	病院側からは誠意のある説明は聞かれなかった。	
\\	病院[びょういん] 側[がわ]からは 誠意[せいい]のある 説明[せつめい]は 聞[き]かれなかった。	
\\	彼は優勝を狙っている。	
\\	彼[かれ]は 優勝[ゆうしょう]を 狙[ねら]っている。	狙う=ねらう= 
\\	申し込みの締め切り日が迫っている。	
\\	申し込[もうしこ]みの 締め切[しめき]り 日[び]が 迫[せま]っている。	
\\	最近は食べることが唯一の楽しみだ。	
\\	最近[さいきん]は 食[た]べることが 唯一[ゆいいつ]の 楽[たの]しみだ。	
\\	彼は悪い連中と付き合っている。	
\\	彼[かれ]は 悪[わる]い 連中[れんちゅう]と 付き合[つきあ]っている。	連中= 
\\	本日をもって退職致します。	
\\	本日[ほんじつ]をもって 退職[たいしょく] 致[いた]します。	をもって= 
\\	全ての講義は予習を前提に進めます。	
\\	全[すべ]ての 講義[こうぎ]は 予習[よしゅう]を 前提[ぜんてい]に 進[すす]めます。	
\\	家に帰る途中でふと約束を思い出した。	
\\	家[いえ]に 帰[かえ]る 途中[とちゅう]でふと 約束[やくそく]を 思い出[おもいだ]した。	ふと= 
\\	彼女はいつもジーンズを履いている。	
\\	彼女[かのじょ]はいつもジーンズを 履[は]いている。	
\\	プレゼント用に包んでもらえますか。	
\\	プレゼント 用[よう]に 包[つつ]んでもらえますか。	
\\	両手でしっかり持って。	
\\	両手[りょうて]でしっかり 持[も]って。	
\\	かゆくてもかいてはいけない。	
\\	かゆくてもかいてはいけない。	かく= 
\\	いろいろな方法を試してみたが、全部だめだった。	
\\	いろいろな 方法[ほうほう]を 試[ため]してみたが、 全部[ぜんぶ]だめだった。	
\\	遥か昔のようだ。	
\\	遥[はる]か 昔[むかし]のようだ。	遥か=はるか= 
\\	長男はフランスで芸術を学んでいる。	
\\	長男[ちょうなん]はフランスで 芸術[げいじゅつ]を 学[まな]んでいる。	
\\	私はごく普通の家庭で育った。	
\\	私[わたし]はごく 普通[ふつう]の 家庭[かてい]で 育[そだ]った。	ごく= 
\\	頭痛を堪えきれずに鎮痛剤を飲んだ。	
\\	頭痛[ずつう]を 堪[こた]えきれずに 鎮痛[ちんつう] 剤[ざい]を 飲[の]んだ。	堪える=こらえる= 
\\	我慢する 鎮痛剤=ちんつうざい= 
\\	泣き出したいのをじっとこらえた。	
\\	泣[な]き 出[だ]したいのをじっとこらえた。	
\\	彼は震え上がっていた。	
\\	彼[かれ]は 震え上[ふるえあ]がっていた。	震え上がる=ふるえあがる= 
\\	その大声に彼は震え上がった。	
\\	その 大声[おおごえ]に 彼[かれ]は 震え上[ふるえあ]がった。	震え上がる=ふるえあがる= 
\\	彼は勇気を奮い起こして敵に立ち向かっていった。	
\\	彼[かれ]は 勇気[ゆうき]を 奮い起[ふるいお]こして 敵[てき]に 立ち向[たちむ]かっていった。	奮い起こす=ふるいおこす= 
\\	私たちは互いに「久しぶり」「元気?」などと言い交わした。	
\\	私[わたし]たちは 互[たが]いに
\\	久[ひさ]しぶり」
\\	元気[げんき]?」などと 言い交[いいか]わした。	
\\	彼女の人柄にみんな惚れ込んだ。	
\\	彼女[かのじょ]の 人柄[ひとがら]にみんな 惚れ込[ほれこ]んだ。	人柄=ひとがら= 
\\	惚れ込む=ほれこむ= 
\\	この色に惚れ込んだんです。	
\\	この 色[いろ]に 惚れ込[ほれこ]んだんです。	惚れ込む=ほれこむ= 
\\	惚れた欲目か彼女はずいぶんきれいに見えた。	
\\	惚[ほ]れた 欲目[よくめ]か 彼女[かのじょ]はずいぶんきれいに 見[み]えた。	惚れる=ほれる= 
\\	欲目=よくめ= 
\\	彼らは他の5人と一緒にその部屋に放り込まれた。	
\\	彼[かれ]らは 他[た]の5 人[にん]と 一緒[いっしょ]にその 部屋[へや]に 放り込[ほうりこ]まれた。	放り込む=ほうりこむ= 
\\	何もかも一緒くたにして段ボール箱に放り込んだ。	
\\	何[なに]もかも 一緒[いっしょ]くたにして 段ボール[だんぼーる] 箱[ばこ]に 放り込[ほうりこ]んだ。	何もかも= 
\\	一緒くた=いっしょくた= 
\\	放り込む=ほうりこむ= 
\\	言うことを聞かないと表へ放り出すぞ。	
\\	言[い]うことを 聞[き]かないと 表[おもて]へ 放り出[ほうりだ]すぞ。	放り出す=ほうりだす= 
\\	彼は悔しそうに手持ちのカードを放り出した。	
\\	彼[かれ]は 悔[くや]しそうに 手持[ても]ちのカードを 放り出[ほうりだ]した。	放り出す=ほうりだす= 
\\	彼は小さい時に親に死なれてそのまま世の中へ放り出された。	
\\	彼[かれ]は 小[ちい]さい 時[とき]に 親[おや]に 死[し]なれてそのまま 世の中[よのなか]へ 放り出[ほうりだ]された。	放り出す=ほうりだす= 
\\	ストの計画は会社に捻り潰された。	
\\	ストの 計画[けいかく]は 会社[かいしゃ]に 捻[ねじ]り 潰[つぶ]された。	捻り潰す=ひねりつぶす= 
\\	ランニングで足首を捻った。	
\\	ランニングで 足首[あしくび]を 捻[ひね]った。	捻る=ひねる= 
\\	あんな相手は軽くひねってやる。	
\\	あんな 相手[あいて]は 軽[かる]くひねってやる。	捻る=ひねる= 
\\	強豪チームにあっさりひねられた。	
\\	強豪[きょうごう]チームにあっさりひねられた。	強豪=きょうごう= 
\\	捻る=ひねる= 
\\	ようやく俳句を一句ひねり出した。	
\\	ようやく 俳句[はいく]を 一句[いっく]ひねり 出[だ]した。	ひねり出す= 
\\	この匂いがゴキブリを引き寄せる。	
\\	この 匂[にお]いがゴキブリを 引き寄[ひきよ]せる。	
\\	恐怖で顔が引きつった。	
\\	恐怖[きょうふ]で 顔[かお]が 引[ひ]きつった。	引き攣る=ひきつる= 
\\	彼女の音楽が僕の気分をいつも引き立ててくれる。	
\\	彼女[かのじょ]の 音楽[おんがく]が 僕[ぼく]の 気分[きぶん]をいつも 引き立[ひきた]ててくれる。	
\\	この絵が部屋全体を引き立てている。	
\\	この 絵[え]が 部屋[へや] 全体[ぜんたい]を 引き立[ひきた]てている。	
\\	彼女は英語は言うに及ばず、スペイン語や中国語まで話せる。	
\\	彼女[かのじょ]は 英語[えいご]は 言[い]うに 及[およ]ばず、スペイン 語[ご]や 中国語[ちゅうごくご]まで 話[はな]せる。	言うに及ばず= 
\\	一人暮らしは気ままでいいよ。	
\\	一人暮[ひとりぐ]らしは 気[き]ままでいいよ。	
\\	勝手気ままにさせておいたので彼女は自分のことしか考えないようになった。	
\\	勝手[かって] 気[き]ままにさせておいたので 彼女[かのじょ]は 自分[じぶん]のことしか 考[かんが]えないようになった。	
\\	気ままに歌の練習が出来る場所が欲しい。	
\\	気[き]ままに 歌[うた]の 練習[れんしゅう]が 出来[でき]る 場所[ばしょ]が 欲[ほ]しい。	
\\	クラスメートが賞を取ったのを見て奮起一番、彼も猛勉強を始めた。	
\\	クラスメートが 賞[しょう]を 取[と]ったのを 見[み]て 奮起[ふんき] 一番[いちばん]、 彼[かれ]も 猛[もう] 勉強[べんきょう]を 始[はじ]めた。	奮起=ふんき= 
\\	猛勉強=もうべんきょう= 
\\	奮起せよ!	
\\	奮起[ふんき]せよ!	奮起=ふんき= 
\\	それは極めて広義に解釈できる。	
\\	それは 極[きわ]めて 広義[こうぎ]に 解釈[かいしゃく]できる。	広義=こうぎ= 
\\	解釈= 
\\	その地域は広義のオリエントに含まれる。	
\\	その 地域[ちいき]は 広義[こうぎ]のオリエントに 含[ふく]まれる。	広義=こうぎ= 
\\	課長の入院中は、臨時に部長が業務を引き受けることになった。	
\\	課長[かちょう]の 入院[にゅういん] 中[ちゅう]は、 臨時[りんじ]に 部長[ぶちょう]が 業務[ぎょうむ]を 引き受[ひきう]けることになった。	
\\	そのビルは東京湾に臨んでいる。	
\\	そのビルは 東京[とうきょう] 湾[わん]に 臨[のぞ]んでいる。	臨む=のぞむ= 
\\	その家は町の大通りに臨んで建っている。	
\\	その 家[いえ]は 町[まち]の 大通[おおどお]りに 臨[のぞ]んで 建[た]っている。	大通り=おおどおり= 
\\	臨む=のぞむ= 
\\	通貨偽造には厳罰をもって臨む。	
\\	通貨[つうか] 偽造[ぎぞう]には 厳罰[げんばつ]をもって 臨[のぞ]む。	偽造=ぎぞう= 
\\	臨む=のぞむ= 
\\	予習もしないで授業に臨むとは、お前やる気があるのか。	
\\	予習[よしゅう]もしないで 授業[じゅぎょう]に 臨[のぞ]むとは、お 前[まえ]やる 気[き]があるのか。	臨む=のぞむ= 
\\	試合に臨んで一言お願いします。	
\\	試合[しあい]に 臨[のぞ]んで 一言[ひとこと]お 願[ねが]いします。	臨む=のぞむ= 
\\	大使は臨機応変に不測の事態を処理した。	
\\	大使[たいし]は 臨機応変[りんきおうへん]に 不測[ふそく]の 事態[じたい]を 処理[しょり]した。	臨機応変= 
\\	このところ寝不足が何日も続いている。	
\\	このところ 寝不足[ねぶそく]が 何[なん] 日[にち]も 続[つづ]いている。	
\\	寝不足がたたって今日は職場でミスの連続だった。	
\\	寝不足[ねぶそく]がたたって 今日[きょう]は 職場[しょくば]でミスの 連続[れんぞく]だった。	
\\	通勤者には寝不足の人が多いらしい。	
\\	通勤[つうきん] 者[しゃ]には 寝不足[ねぶそく]の 人[ひと]が 多[おお]いらしい。	
\\	寝不足の頭では会議の進行についていけなかった。	
\\	寝不足[ねぶそく]の 頭[あたま]では 会議[かいぎ]の 進行[しんこう]についていけなかった。	
\\	寝不足のドライバーには事故が多い。	
\\	寝不足[ねぶそく]のドライバーには 事故[じこ]が 多[おお]い。	
\\	寝不足を訴える主婦が多い。	
\\	寝不足[ねぶそく]を 訴[うった]える 主婦[しゅふ]が 多[おお]い。	
\\	その件は泣き寝入りしなければなるまい。	
\\	その 件[けん]は 泣き寝入[なきねい]りしなければなるまい。	
\\	再三言ったが聞き入れなかった。	
\\	再三[さいさん] 言[い]ったが 聞き入[ききい]れなかった。	再三=さいさん= 
\\	ここはすきま風が入る。	
\\	ここはすきま 風[かぜ]が 入[はい]る。	隙間風=すきまかぜ= 
\\	そのことがあってから、二人の間にすきま風が吹き始めた。	
\\	そのことがあってから、二 人[にん]の 間[ま]にすきま 風[かぜ]が 吹[ふ]き 始[はじ]めた。	隙間風=すきまかぜ= 
\\	それは今もなお記憶に生々しい。	
\\	それは 今[いま]もなお 記憶[きおく]に 生々[なまなま]しい。	
\\	出て行かないと叩き出すぞ!	
\\	出[で]て 行[い]かないと 叩[はた]き 出[だ]すぞ!	
\\	ヤクザをこの街から叩き出そう。	
\\	ヤクザをこの 街[まち]から 叩[はた]き 出[で]そう。	
\\	池に石を投げたら、波紋が広がった。	
\\	池[いけ]に 石[いし]を 投[な]げたら、 波紋[はもん]が 広[ひろ]がった。	波紋=はもん= 
\\	(あるきっかけが次々と他に影響を及ぼすこと) 
\\	彼女はずっとクラシック一辺倒だったが、そのバンドの演奏を聞いてからロックにも興味を持つようになった。	
\\	彼女[かのじょ]はずっとクラシック 一辺倒[いっぺんとう]だったが、そのバンドの 演奏[えんそう]を 聞[き]いてからロックにも 興味[きょうみ]を 持[も]つようになった。	一辺倒=いっぺんとう= 
\\	中等教育は学力一辺倒から個性重視へと変わりつつある。	
\\	中等[ちゅうとう] 教育[きょういく]は 学力[がくりょく] 一辺倒[いっぺんとう]から 個性[こせい] 重視[じゅうし]へと 変[か]わりつつある。	中等教育= 
\\	一辺倒=いっぺんとう= 
\\	日はずんずん傾いていた。	
\\	日[ひ]はずんずん 傾[かたむ]いていた。	傾く=かたむく= 
\\	彼が社長になって商売が傾いた。	
\\	彼[かれ]が 社長[しゃちょう]になって 商売[しょうばい]が 傾[かたむ]いた。	傾く=かたむく= 
\\	彼女の心は彼よりも君の方に傾いているようだ。	
\\	彼女[かのじょ]の 心[こころ]は 彼[かれ]よりも 君[きみ]の 方[ほう]に 傾[かたむ]いているようだ。	傾く=かたむく= 
\\	その事件で世論は死刑廃止論に大きく傾いた。	
\\	その 事件[じけん]で 世論[せろん]は 死刑[しけい] 廃止[はいし] 論[ろん]に 大[おお]きく 傾[かたむ]いた。	傾く=かたむく= 
\\	彼の気持ちはだんだん計画に賛成の方に傾いてきた。	
\\	彼[かれ]の 気持[きも]ちはだんだん 計画[けいかく]に 賛成[さんせい]の 方[ほう]に 傾[かたむ]いてきた。	傾く=かたむく= 
\\	彼はどちらかというと君の意見に傾いている。	
\\	彼[かれ]はどちらかというと 君[きみ]の 意見[いけん]に 傾[かたむ]いている。	傾く=かたむく= 
\\	雨露をしのぐ家もない。	
\\	雨露[あめつゆ]をしのぐ 家[いえ]もない。	雨露=あめつゆ= 
\\	凌ぐ=しのぐ= 
\\	私はかばんで雨をしのぎながら迎えの車を待っていた。	
\\	私[わたし]はかばんで 雨[あめ]をしのぎながら 迎[むか]えの 車[くるま]を 待[ま]っていた。	凌ぐ=しのぐ= 
\\	ここをしのげば、後は何とかなる。	
\\	ここをしのげば、 後[あと]は 何[なん]とかなる。	凌ぐ=しのぐ= 
\\	その市は人口において名古屋をしのいでいる。	
\\	その 市[し]は 人口[じんこう]において 名古屋[なごや]をしのいでいる。	凌ぐ=しのぐ= 
\\	この専門技術で彼をしのぐものはいない。	
\\	この 専門[せんもん] 技術[ぎじゅつ]で 彼[かれ]をしのぐものはいない。	凌ぐ=しのぐ= 
\\	その地域に原理主義が再び台頭し始めた。	
\\	その 地域[ちいき]に 原理[げんり] 主義[しゅぎ]が 再[ふたた]び 台頭[たいとう]し 始[はじ]めた。	台頭=たいとう= 
\\	電子メディアの台頭で書籍の売り上げが落ち込んでいる。	
\\	電子[でんし]メディアの 台頭[たいとう]で 書籍[しょせき]の 売り上[うりあ]げが 落ち込[おちこ]んでいる。	台頭=たいとう= 
\\	その少年は川を渉ろうとして深みに陥り、溺死した。	
\\	その 少年[しょうねん]は 川[かわ]を 渉[わたる]ろうとして 深[ふか]みに 陥[おちい]り、 溺死[できし]した。	陥る=おちいる= はまり込む, 
\\	溺死する=できしする= 
\\	わが家は父の会社が倒産して貧乏のどん底に陥った。	
\\	わが 家[や]は 父[ちち]の 会社[かいしゃ]が 倒産[とうさん]して 貧乏[びんぼう]のどん 底[ぞこ]に 陥[おちい]った。	陥る=おちいる= はまり込む, 
\\	まじめな人ほどスランプに陥りやすい。	
\\	まじめな 人[ひと]ほどスランプに 陥[おちい]りやすい。	陥る=おちいる= はまり込む, 
\\	子供の数より老人の数の方が多いという事態に陥っている。	
\\	子供[こども]の 数[かず]より 老人[ろうじん]の 数[かず]の 方[ほう]が 多[おお]いという 事態[じたい]に 陥[おちい]っている。	陥る=おちいる= はまり込む, 
\\	その計略には乗らないぞ。	
\\	その 計略[けいりゃく]には 乗[の]らないぞ。	計略=けいりゃく= 
\\	私は法律に抵触するようなことは何もしていない。	
\\	私[わたし]は 法律[ほうりつ]に 抵触[ていしょく]するようなことは 何[なに]もしていない。	
\\	その点では君と意見が相違する。	
\\	その 点[てん]では 君[きみ]と 意見[いけん]が 相違[そうい]する。	
\\	職場の同僚との折り合いはうまくいってるか。	
\\	職場[しょくば]の 同僚[どうりょう]との 折り合[おりあ]いはうまくいってるか。	折り合い= 
\\	彼は上司と折り合いが悪くてとうとう辞職した。	
\\	彼[かれ]は 上司[じょうし]と 折り合[おりあ]いが 悪[わる]くてとうとう 辞職[じしょく]した。	折り合い= 
\\	彼女は両親との折り合いが非常に悪い。	
\\	彼女[かのじょ]は 両親[りょうしん]との 折り合[おりあ]いが 非常[ひじょう]に 悪[わる]い。	折り合い= 
\\	妻は僕の母とあまり折り合いがよくない。	
\\	妻[つま]は 僕[ぼく]の 母[はは]とあまり 折り合[おりあ]いがよくない。	折り合い= 
\\	ここはなんとか折り合いがつかないものかね。	
\\	ここはなんとか 折り合[おりあ]いがつかないものかね。	折り合い= 
\\	3年契約ということであの会社とうまく折り合いがついた。	
\\	年[ねん] 契約[けいやく]ということであの 会社[かいしゃ]とうまく 折り合[おりあ]いがついた。	折り合い= 
\\	あの先生は脱線が多い。	
\\	あの 先生[せんせい]は 脱線[だっせん]が 多[おお]い。	
\\	均衡が取れていない。	
\\	均衡[きんこう]が 取[と]れていない。	
\\	僕は省みて悔いるところがない。	
\\	僕[ぼく]は 省[かえり]みて 悔[く]いるところがない。	顧みる・省みる=かえりみる= {顧みる} (振り向く) 
\\	(回想する) 
\\	(顧慮する) 
\\	{省みる} (反省する) 
\\	悔いる=くいる= 
\\	彼の助言は誰からも顧みられなかった。	
\\	彼[かれ]の 助言[じょげん]は 誰[だれ]からも 顧[かえり]みられなかった。	顧みる・省みる=かえりみる= {顧みる} (振り向く) 
\\	(回想する) 
\\	(顧慮する) 
\\	{省みる} (反省する) 
\\	国内に不穏な空気が醸成されつつある。	
\\	国内[こくない]に 不穏[ふおん]な 空気[くうき]が 醸成[じょうせい]されつつある。	不穏=ふおん= 
\\	醸成=じょうせい= 
\\	その結果、彼の行為に対して新たな疑惑が生じてきた。	
\\	その 結果[けっか]、 彼[かれ]の 行為[こうい]に 対[たい]して 新[あら]たな 疑惑[ぎわく]が 生[しょう]じてきた。	
\\	事件を初めから再度検証してみよう。	
\\	事件[じけん]を 初[はじ]めから 再度[さいど] 検証[けんしょう]してみよう。	再度=さいど= 
\\	この論文では当時の教育制度を検証する。	
\\	この 論文[ろんぶん]では 当時[とうじ]の 教育[きょういく] 制度[せいど]を 検証[けんしょう]する。	
\\	我が国では高齢化が加速している。	
\\	我が国[わがくに]では 高齢[こうれい] 化[か]が 加速[かそく]している。	加速=かそく= 
\\	次の瞬間彼の車はぐっと加速して前の車を追い越した。	
\\	次[つぎ]の 瞬間[しゅんかん] 彼[かれ]の 車[くるま]はぐっと 加速[かそく]して 前[まえ]の 車[くるま]を 追い越[おいこ]した。	加速=かそく= 
\\	今や待望のチャンスが到来した。	
\\	今[いま]や 待望[たいぼう]のチャンスが 到来[とうらい]した。	今や= 
\\	今こそ 待望=たいぼう= 
\\	彼を乗せた飛行機の到着を今や遅しと待ちわびていた。	
\\	彼[かれ]を 乗[の]せた 飛行機[ひこうき]の 到着[とうちゃく]を 今[いま]や 遅[おそ]しと 待[ま]ちわびていた。	今や遅しと=いまやおそしと= 
\\	待ちわびる= 
\\	紛う方なき父の筆跡だ。	
\\	紛[まご]う 方[かた]なき 父[ちち]の 筆跡[ひっせき]だ。	紛うかたなき= 
\\	筆跡=ひっせき= 
\\	建築費用の大雑把な見積もりを出してくれ。	
\\	建築[けんちく] 費用[ひよう]の 大雑把[おおざっぱ]な 見積[みつ]もりを 出[だ]してくれ。	大雑把=おおざっぱ= 
\\	大雑把な当て推量だけで重大な決定を下してはならない。	
\\	大雑把[おおざっぱ]な 当て推量[あてずいりょう]だけで 重大[じゅうだい]な 決定[けってい]を 下[くだ]してはならない。	大雑把=おおざっぱ= 
\\	当て推量=あてずいりょう= 
\\	君の説明は大雑把すぎる。	
\\	君[きみ]の 説明[せつめい]は 大雑把[おおざっぱ]すぎる。	大雑把=おおざっぱ= 
\\	ここには彼の英語力に匹敵する力を持つ者はいない。	
\\	ここには 彼[かれ]の 英語[えいご] 力[りょく]に 匹敵[ひってき]する 力[ちから]を 持[も]つ 者[もの]はいない。	匹敵=ひってき= 比べてみて能力や価値などが同じ程度であること。肩を並べること。
\\	夜明けのアルプスの美しさに匹敵するものはない。	
\\	夜明[よあ]けのアルプスの 美[うつく]しさに 匹敵[ひってき]するものはない。	夜明け=よあけ= 
\\	匹敵=ひってき= 比べてみて能力や価値などが同じ程度であること。肩を並べること。
\\	その賞金は私の年収に匹敵する。	
\\	その 賞金[しょうきん]は 私[わたし]の 年収[ねんしゅう]に 匹敵[ひってき]する。	匹敵=ひってき= 比べてみて能力や価値などが同じ程度であること。肩を並べること。
\\	ことさらにやった訳ではない。	
\\	ことさらにやった 訳[わけ]ではない。	
\\	不登校問題が、「楽しくない学校」の現状をあぶり出した。	
\\	不[ふ] 登校[とうこう] 問題[もんだい]が、
\\	楽[たの]しくない 学校[がっこう]」の 現状[げんじょう]をあぶり 出[だ]した。	あぶり出す= 
\\	彼は今どきの若者にしては信じられないほど純粋だ。	
\\	彼[かれ]は 今[いま]どきの 若者[わかもの]にしては 信[しん]じられないほど 純粋[じゅんすい]だ。	今どき= 
\\	この島の人々は純粋で心優しい。	
\\	この 島[しま]の 人々[ひとびと]は 純粋[じゅんすい]で 心[こころ] 優[やさ]しい。	
\\	失業率はここ4ヶ月続けて上昇している。	
\\	失業[しつぎょう] 率[りつ]はここ 4ヶ月[よんかげつ] 続[つづ]けて 上昇[じょうしょう]している。	
\\	昇進は腕次第だ。	
\\	昇進[しょうしん]は 腕[うで] 次第[しだい]だ。	
\\	ひとすじの煙が空へ昇っていった。	
\\	ひとすじの 煙[けむり]が 空[そら]へ 昇[のぼ]っていった。	昇る=のぼる= 
\\	太陽が昇るところだ。	
\\	太陽[たいよう]が 昇[のぼ]るところだ。	昇る=のぼる= 
\\	心が引き裂かれるような気がした。	
\\	心[こころ]が 引き裂[ひきさ]かれるような 気[き]がした。	引き裂く=ひきさく= 
\\	あの夫婦の仲を引き裂くものなんて何もないはずだった。	
\\	あの 夫婦[ふうふ]の 仲[なか]を 引き裂[ひきさ]くものなんて 何[なに]もないはずだった。	引き裂く=ひきさく= 
\\	彼は黙って引き下がっていった。	
\\	彼[かれ]は 黙[だま]って 引き下[ひきさ]がっていった。	引き下がる= 
\\	それで君はすごすごと引き下がってきたのか。	
\\	それで 君[きみ]はすごすごと 引き下[ひきさ]がってきたのか。	すごすご= 
\\	引き下がる= 
\\	彼女はライバル会社に引き抜かれた。	
\\	彼女[かのじょ]はライバル 会社[かいしゃ]に 引き抜[ひきぬ]かれた。	引き抜く= 
\\	今日は子供に一日中動物園を引き回されたよ。	
\\	今日[きょう]は 子供[こども]に一 日[にち] 中[ちゅう] 動物[どうぶつ] 園[えん]を 引き回[ひきまわ]されたよ。	引き回す= 
\\	あなたは採用条件に該当しない。	
\\	あなたは 採用[さいよう] 条件[じょうけん]に 該当[がいとう]しない。	該当=がいとう= 
\\	該当欄にチェックを入れてください。	
\\	該当[がいとう] 欄[らん]にチェックを 入[い]れてください。	
\\	首相は周辺に辞意を漏らしている。	
\\	首相[しゅしょう]は 周辺[しゅうへん]に 辞意[じい]を 漏[も]らしている。	辞意=じい= 
\\	昨夜は飲み過ぎて潰れてしまった。	
\\	昨夜[さくや]は 飲[の]み 過[す]ぎて 潰[つぶ]れてしまった。	潰れる=つぶれる= 
\\	会社がつぶれてしまった。	
\\	会社[かいしゃ]がつぶれてしまった。	潰れる=つぶれる= 
\\	指先のまめがつぶれた。	
\\	指先[ゆびさき]のまめがつぶれた。	潰れる=つぶれる= 
\\	私は彼に負い目があるので、その頼みを断れなかった。	
\\	私[わたし]は 彼[かれ]に 負い目[おいめ]があるので、その 頼[たの]みを 断[ことわ]れなかった。	負い目=おいめ= 
\\	それを君が負い目に思うことはない。	
\\	それを 君[きみ]が 負い目[おいめ]に 思[おも]うことはない。	負い目=おいめ= 
\\	彼女は彼らをうそつきだとののしった。	
\\	彼女[かのじょ]は 彼[かれ]らをうそつきだとののしった。	罵る=ののしる= 
\\	彼は私をくそみそにののしった。	
\\	彼[かれ]は 私[わたし]をくそみそにののしった。	くそみそ= 
\\	罵る=ののしる= 
\\	その問題については自信をもって断じることが出来る。	
\\	その 問題[もんだい]については 自信[じしん]をもって 断[だん]じることが 出来[でき]る。	断じる= 
\\	この映画、二人の再会のシーンが圧巻だね。	
\\	この 映画[えいが]、 二人[ふたり]の 再会[さいかい]のシーンが 圧巻[あっかん]だね。	圧巻=あっかん= 
\\	ゲリラ事件は、国家へのあからさまな挑戦である。	
\\	ゲリラ 事件[じけん]は、 国家[こっか]へのあからさまな 挑戦[ちょうせん]である。	あからさま= 
\\	その言葉に、彼はあからさまに不満の表情をした。	
\\	その 言葉[ことば]に、 彼[かれ]はあからさまに 不満[ふまん]の 表情[ひょうじょう]をした。	あからさま= 
\\	医師の怠慢によって助かる命が助からなかった。	
\\	医師[いし]の 怠慢[たいまん]によって 助[たす]かる 命[いのち]が 助[たす]からなかった。	怠慢な=たいまんな= 
\\	(不注意な) 
\\	私が特に力説したいのはこの点だ。	
\\	私[わたし]が 特[とく]に 力説[りきせつ]したいのはこの 点[てん]だ。	力説=りきせつ= 
\\	議論を活発するために資料を用意しました。	
\\	議論[ぎろん]を 活発[かっぱつ]するために 資料[しりょう]を 用意[ようい]しました。	
\\	これ以上赤字を膨らませてはいけない。	
\\	これ 以上[いじょう] 赤字[あかじ]を 膨[ふく]らませてはいけない。	膨らませる=ふくらませる= 
\\	(息で)
\\	脚注への参照は数字で示してある。	
\\	脚注[きゃくちゅう]への 参照[さんしょう]は 数字[すうじ]で 示[しめ]してある。	脚注=きゃくちゅう= 
\\	注を参照したらそれが分かった。	
\\	注[ちゅう]を 参照[さんしょう]したらそれが 分[わ]かった。	注=ちゅう= 
\\	土地の境界をめぐって隣家同士のいがみ合いが続いている。	
\\	土地[とち]の 境界[きょうかい]をめぐって 隣家[りんか] 同士[どうし]のいがみ 合[あ]いが 続[つづ]いている。	いがみ合い= 
\\	賛成派と反対派が互いにいがみ合っていた。	
\\	賛成[さんせい] 派[は]と 反対[はんたい] 派[は]が 互[たが]いにいがみ 合[あ]っていた。	いがみ合う= (犬などが) 
\\	両家はいがみ合っている。	
\\	両家[りょうけ]はいがみ 合[あ]っている。	両家=りょうけいがみ合う= (犬などが) 
\\	5年の結婚生活に終止符を打った。	
\\	年[ねん]の 結婚[けっこん] 生活[せいかつ]に 終止符[しゅうしふ]を 打[う]った。	終止符=しゅうしふ= 
\\	終止符を打つ= 
\\	やっとこの問題にも一区切りがついた。	
\\	やっとこの 問題[もんだい]にも 一区切[ひとくぎ]りがついた。	区切り= 
\\	働く母親の増加に伴い、保育施設の拡充が急務だ。	
\\	働[はたら]く 母親[ははおや]の 増加[ぞうか]に 伴[ともな]い、 保育[ほいく] 施設[しせつ]の 拡充[かくじゅう]が 急務[きゅうむ]だ。	拡充=かくじゅう= 
\\	急務=きゅうむ= 
\\	彼らの心を込めたもてなしには感激した。	
\\	彼[かれ]らの 心[こころ]を 込[こ]めたもてなしには 感激[かんげき]した。	感激= 
\\	感激で胸がいっぱいです。	
\\	感激[かんげき]で 胸[むね]がいっぱいです。	
\\	何かよい芝居があれば行きたいね。	
\\	何[なに]かよい 芝居[しばい]があれば 行[い]きたいね。	芝居=しばい= 
\\	あの役者は芝居が下手だ。	
\\	あの 役者[やくしゃ]は 芝居[しばい]が 下手[へた]だ。	芝居=しばい= 
\\	その芝居は当たった。	
\\	その 芝居[しばい]は 当[あ]たった。	芝居=しばい= 
\\	彼女の芝居にまんまとだまされた。	
\\	彼女[かのじょ]の 芝居[しばい]にまんまとだまされた。	芝居=しばい= 
\\	腹が痛いという芝居をして授業をさぼった。	
\\	腹[はら]が 痛[いた]いという 芝居[しばい]をして 授業[じゅぎょう]をさぼった。	芝居=しばい= 
\\	下手な芝居は止せ。	
\\	下手[へた]な 芝居[しばい]は 止[よ]せ。	芝居=しばい= 
\\	その映画は大成功を収めた。	
\\	その 映画[えいが]は 大[だい] 成功[せいこう]を 収[おさ]めた。	
\\	この資料にはジョンは1800年ごろ生まれたと記されている。	
\\	この 資料[しりょう]にはジョンは1800 年[ねん]ごろ 生[う]まれたと 記[しる]されている。	
\\	マラソン大会のため、交通が規制される。	
\\	マラソン 大会[たいかい]のため、 交通[こうつう]が 規制[きせい]される。	
\\	この試験の結果が私の将来を左右する。	
\\	この 試験[しけん]の 結果[けっか]が 私[わたし]の 将来[しょうらい]を 左右[さゆう]する。	左右=さゆう= 
\\	彼女は爪をかむ癖がある。	
\\	彼女[かのじょ]は 爪[つめ]をかむ 癖[くせ]がある。	
\\	その国の政治の仕組みを理解するのは難しい。	
\\	その 国[くに]の 政治[せいじ]の 仕組[しく]みを 理解[りかい]するのは 難[むずか]しい。	
\\	慌ててやると間違えてしまいますよ。	
\\	慌[あわ]ててやると 間違[まちが]えてしまいますよ。	慌てる= 
\\	政府は適切な措置をとらなかった。	
\\	政府[せいふ]は 適切[てきせつ]な 措置[そち]をとらなかった。	
\\	全員の視線が彼に向けられた。	
\\	全員[ぜんいん]の 視線[しせん]が 彼[かれ]に 向[む]けられた。	
\\	彼は岸まで必死で泳いだ。	
\\	彼[かれ]は 岸[きし]まで 必死[ひっし]で 泳[およ]いだ。	
\\	がんは恐ろしい病気だ。	
\\	がんは 恐[おそ]ろしい 病気[びょうき]だ。	
\\	この箱の中身は何だろう。	
\\	この 箱[はこ]の 中身[なかみ]は 何[なに]だろう。	
\\	私は2歳で母と引き離された。	
\\	私[わたし]は2 歳[さい]で 母[はは]と 引き離[ひきはな]された。	引き離す= 
\\	経済の面ではアメリカがロシアを断然引き離している。	
\\	経済[けいざい]の 面[めん]ではアメリカがロシアを 断然[だんぜん] 引き離[ひきはな]している。	断然= 
\\	引き離す= 
\\	今投資すれば必ず引き合う。	
\\	今[いま] 投資[とうし]すれば 必[かなら]ず 引き合[ひきあ]う。	引き合う= 
\\	今夜は冷え込みそうだ。	
\\	今夜[こんや]は 冷え込[ひえこ]みそうだ。	冷え込む=ひえこむ= 
\\	米国と世界との関係がいつになく冷え込んでいる。	
\\	米国[べいこく]と 世界[せかい]との 関係[かんけい]がいつになく 冷え込[ひえこ]んでいる。	冷え込む=ひえこむ= 
\\	敵の攻撃が間欠的に続いた。	
\\	敵[てき]の 攻撃[こうげき]が 間欠[かんけつ] 的[てき]に 続[つづ]いた。	
\\	彼らはロシア側と昨年の3月以来断続的に交渉してきた。	
\\	彼[かれ]らはロシア 側[がわ]と 昨年[さくねん]の 3月[さんがつ] 以来[いらい] 断続[だんぞく] 的[てき]に 交渉[こうしょう]してきた。	断続的= 
\\	彼の声は家中に響き渡った。	
\\	彼[かれ]の 声[こえ]は 家中[いえじゅう]に 響[ひび]き 渡[わた]った。	家中=うちじゅう響き渡る=ひびきわたる= 
\\	(知れ渡る)
\\	彼の名声は全国に響き渡った。	
\\	彼[かれ]の 名声[めいせい]は 全国[ぜんこく]に 響[ひび]き 渡[わた]った。	名声=めいせい= 
\\	響き渡る=ひびきわたる= 
\\	(知れ渡る)
\\	除夜の鐘の音が遠く響き渡った。	
\\	除夜の鐘[じょやのかね]の 音[おと]が 遠[とお]く 響[ひび]き 渡[わた]った。	除夜の鐘=じょやのかね= 
\\	響き渡る=ひびきわたる= 
\\	(知れ渡る)
\\	仲間からはじき出された。	
\\	仲間[なかま]からはじき 出[だ]された。	はじき出す= 
\\	その費用をやっとはじき出した。	
\\	その 費用[ひよう]をやっとはじき 出[だ]した。	はじき出す= 
\\	この分には推敲の跡が見える。	
\\	この 分[ぶん]には 推敲[すいこう]の 跡[あと]が 見[み]える。	推敲=すいこう= 
\\	跡=あと= 
\\	法律の恣意的な解釈は許されない。	
\\	法律[ほうりつ]の 恣意[しい] 的[てき]な 解釈[かいしゃく]は 許[ゆる]されない。	恣意的=しいてき= 
\\	その法案は内容に曖昧な点があり、運用が恣意的になる恐れがある。	
\\	その 法案[ほうあん]は 内容[ないよう]に 曖昧[あいまい]な 点[てん]があり、 運用[うんよう]が 恣意[しい] 的[てき]になる 恐[おそ]れがある。	恣意的=しいてき= 
\\	ささいな犯罪をどのように防止すべきかについて人々の姿勢は二極化した。	
\\	ささいな 犯罪[はんざい]をどのように 防止[ぼうし]すべきかについて 人々[ひとびと]の 姿勢[しせい]は二 極[きょく] 化[か]した。	二極化=にきょくか= 
\\	印象主義は絵画の歴史において重要な役割を果たした。	
\\	印象[いんしょう] 主義[しゅぎ]は 絵画[かいが]の 歴史[れきし]において 重要[じゅうよう]な 役割[やくわり]を 果[は]たした。	
\\	それは私が保証する。	
\\	それは 私[わたし]が 保証[ほしょう]する。	
\\	彼の人柄は私が保証する。	
\\	彼[かれ]の 人柄[ひとがら]は 私[わたし]が 保証[ほしょう]する。	
\\	彼は来るよ、私が保証する。	
\\	彼[かれ]は 来[く]るよ、 私[わたし]が 保証[ほしょう]する。	
\\	それが事実かどうかは保証できない。	
\\	それが 事実[じじつ]かどうかは 保証[ほしょう]できない。	
\\	この計画がうまくいくという保証は何もない。	
\\	この 計画[けいかく]がうまくいくという 保証[ほしょう]は 何[なに]もない。	
\\	大統領は国民の自由と安全を保障した。	
\\	大統領[だいとうりょう]は 国民[こくみん]の 自由[じゆう]と 安全[あんぜん]を 保障[ほしょう]した。	
\\	この会議日程で支障がある方はお申し出下さい。	
\\	この 会議[かいぎ] 日程[にってい]で 支障[ししょう]がある 方[かた]はお 申し出[もうしで] 下[くだ]さい。	支障= 
\\	そのロケット打ち上げ失敗は我が国の宇宙開発に支障をきたすことになるだろう。	
\\	そのロケット 打ち上[うちあ]げ 失敗[しっぱい]は 我が国[わがくに]の 宇宙[うちゅう] 開発[かいはつ]に 支障[ししょう]をきたすことになるだろう。	支障= 
\\	来す=きたす= 
\\	お気に障りましたらごめんなさい。	
\\	お 気[き]に 障[さわ]りましたらごめんなさい。	
\\	右手にけがをしたので書き物に差し支える。	
\\	右手[みぎて]にけがをしたので 書き物[かきもの]に 差し支[さしつか]える。	差し支える=さしつかえる= 
\\	十分眠っておかないと明日の仕事に差し支える。	
\\	十分[じゅうぶん] 眠[ねむ]っておかないと 明日[あした]の 仕事[しごと]に 差し支[さしつか]える。	差し支える=さしつかえる= 
\\	運転に差し支えるから私は飲みません。	
\\	運転[うんてん]に 差し支[さしつか]えるから 私[わたし]は 飲[の]みません。	差し支える=さしつかえる= 
\\	彼女はお気に入りのドレスを着てパーティに出掛けた。	
\\	彼女[かのじょ]はお 気に入[きにい]りのドレスを 着[き]てパーティに 出掛[でか]けた。	
\\	彼はその山の初登頂の歴史的瞬間をカメラに収めた。	
\\	彼[かれ]はその 山[やま]の 初[はつ] 登頂[とうちょう]の 歴史[れきし] 的[てき] 瞬間[しゅんかん]をカメラに 収[おさ]めた。	
\\	亡き父の遺品は寺に納めて供養してもらうことにした。	
\\	亡[な]き 父[ちち]の 遺品[いひん]は 寺[てら]に 納[おさ]めて 供養[くよう]してもらうことにした。	供養=くよう= 
\\	彼は深く恥じ入った。	
\\	彼[かれ]は 深[ふか]く 恥じ入[はじい]った。	恥じ入る=はじいる= 
\\	彼女は彼に手紙を渡すと何も言わずに走り去った。	
\\	彼女[かのじょ]は 彼[かれ]に 手紙[てがみ]を 渡[わた]すと 何[なに]も 言[い]わずに 走り去[はしりさ]った。	走り去る= 
\\	犯行現場から白い車が走り去るのが目撃されている。	
\\	犯行[はんこう] 現場[げんば]から 白[しろ]い 車[くるま]が 走り去[はしりさ]るのが 目撃[もくげき]されている。	走り去る= 
\\	目撃=もくげき= 
\\	胸も張り裂けそうな気持ちです。	
\\	胸[むね]も 張り裂[はりさ]けそうな 気持[きも]ちです。	張り裂ける=はりさける= 
\\	日本は世界有数の漁業国である。	
\\	日本は 世界[せかい] 有数[ゆうすう]の 漁業[ぎょぎょう] 国[こく]である。	有数=ゆうすう= 
\\	世界でも有数のデザイナーたちが勢揃いした。	
\\	世界[せかい]でも 有数[ゆうすう]のデザイナーたちが 勢揃[せいぞろ]いした。	有数=ゆうすう= 
\\	勢揃いする=せいぞろいする= 
\\	彼は言行が一致しない。	
\\	彼[かれ]は 言行[げんこう]が 一致[いっち]しない。	
\\	我が社の現行賃金制度は改定されるべきだ。	
\\	我[わ]が 社[しゃ]の 現行[げんこう] 賃金[ちんぎん] 制度[せいど]は 改定[かいてい]されるべきだ。	現行= 
\\	改定= 
\\	必然的にそうなるしかないのだ。	
\\	必然[ひつぜん] 的[てき]にそうなるしかないのだ。	必然=ひつぜん= 
\\	弁護士になろうと思いを定めた。	
\\	弁護士[べんごし]になろうと 思[おも]いを 定[さだ]めた。	定める=さだめる= 
\\	昨今は小学生でも携帯を持っている。	
\\	昨今[さっこん]は 小学生[しょうがくせい]でも 携帯[けいたい]を 持[も]っている。	昨今=さっこん= 
\\	焼き芋屋さんを昨今見かけなくなったね。	
\\	焼き芋[やきいも] 屋[や]さんを 昨今[さっこん] 見[み]かけなくなったね。	昨今=さっこん= 
\\	準備不足は否めない。	
\\	準備[じゅんび] 不足[ふそく]は 否[いな]めない。	否めない= 
\\	私は地域の一人暮らしのお年寄りに給食を配達しています。	
\\	私[わたし]は 地域[ちいき]の 一人暮[ひとりぐ]らしのお 年寄[としよ]りに 給食[きゅうしょく]を 配達[はいたつ]しています。	
\\	インターネットを使えば大量のデータを瞬時に世界中に送ることが出来る。	
\\	インターネットを 使[つか]えば 大量[たいりょう]のデータを 瞬時[しゅんじ]に 世界中[せかいじゅう]に 送[おく]ることが 出来[でき]る。	
\\	首相は女性スキャンダルで辞任に追い込まれた。	
\\	首相[しゅしょう]は 女性[じょせい]スキャンダルで 辞任[じにん]に 追い込[おいこ]まれた。	
\\	この皿はちょっといびつに見える。	
\\	この 皿[さら]はちょっといびつに 見[み]える。	歪な=いびつ= 
\\	いびつな社会構造が経済を混乱させた。	
\\	いびつな 社会[しゃかい] 構造[こうぞう]が 経済[けいざい]を 混乱[こんらん]させた。	歪な=いびつ= 
\\	弁当箱がつぶれてひどくいびつになった。	
\\	弁当[べんとう] 箱[ばこ]がつぶれてひどくいびつになった。	歪な=いびつ= 
\\	死んだふりをして彼女を脅かしてやろう。	
\\	死[し]んだふりをして 彼女[かのじょ]を 脅[おど]かしてやろう。	脅かす=おどかす= 
\\	その計画に100万円かかるって?脅かすなよ。	
\\	その 計画[けいかく]に100 万[まん] 円[えん]かかるって? 脅[おど]かすなよ。	脅かす=おどかす= 
\\	大量破壊兵器が全世界を脅かしている。	
\\	大量[たいりょう] 破壊[はかい] 兵器[へいき]が 全[ぜん] 世界[せかい]を 脅[おびや]かしている。	大量破壊兵器=たいりょうはかいへいき= 
\\	脅かす=おびやかす= (不安にさせる) 
\\	強い紫外線は我々の健康を脅かす。	
\\	強[つよ]い 紫外線[しがいせん]は 我々[われわれ]の 健康[けんこう]を 脅[おびや]かす。	紫外線=しがいせん= 
\\	脅かす=おびやかす= (不安にさせる) 
\\	上着の背中が裂けていますよ。	
\\	上着[うわぎ]の 背中[せなか]が 裂[さ]けていますよ。	裂ける=さける= 
\\	落雷で幹が裂けた。	
\\	落雷[らくらい]で 幹[みき]が 裂[さ]けた。	落雷=らくらい= 
\\	裂ける=さける= 
\\	その冠は黄金でできていた。	
\\	その 冠[かんむり]は 黄金[おうごん]でできていた。	冠=かんむり= 
\\	問題点が出尽くしたところで討論に入った。	
\\	問題[もんだい] 点[てん]が 出尽[でつ]くしたところで 討論[とうろん]に 入[はい]った。	出尽くす=でつくす= 
\\	この記事で朝日新聞は他社を出し抜いた。	
\\	この 記事[きじ]で 朝日新聞[あさひしんぶん]は 他社[たしゃ]を 出し抜[だしぬ]いた。	出し抜く= (先回りする) 
\\	彼は私を出し抜いて先に彼女とデートした。	
\\	彼[かれ]は 私[わたし]を 出し抜[だしぬ]いて 先[さき]に 彼女[かのじょ]とデートした。	出し抜く= (先回りする) 
\\	母親は戦場から無事に帰ってきた我が子をひしと抱き寄せた。	
\\	母親[ははおや]は 戦場[せんじょう]から 無事[ぶじ]に 帰[かえ]ってきた 我[わ]が 子[こ]をひしと 抱[いだ]き 寄[よ]せた。	抱き寄せる=だきよせる= 
\\	彼は患者が転びそうになるたびに抱き留めてあげた。	
\\	彼[かれ]は 患者[かんじゃ]が 転[ころ]びそうになるたびに 抱き留[だきと]めてあげた。	抱き留める= 
\\	赤ん坊を抱きかかえたまま倒れていた。	
\\	赤ん坊[あかんぼう]を 抱[いだ]きかかえたまま 倒[たお]れていた。	抱きかかえる= 
\\	弟は弱虫と嘲られ、よく泣きながら帰ってきたものだ。	
\\	弟[おとうと]は 弱虫[よわむし]と 嘲[あざけ]られ、よく 泣[な]きながら 帰[かえ]ってきたものだ。	嘲る=あざける= 
\\	日程に無理がある。	
\\	日程[にってい]に 無理[むり]がある。	
\\	無理を言うんじゃない。	
\\	無理[むり]を 言[い]うんじゃない。	
\\	この石を持ち上げるのは子供には無理だ。	
\\	この 石[いし]を 持ち上[もちあ]げるのは 子供[こども]には 無理[むり]だ。	
\\	彼はいやだという私に無理に酒を飲ませた。	
\\	彼[かれ]はいやだという 私[わたし]に 無理[むり]に 酒[さけ]を 飲[の]ませた。	
\\	かなり無理をして新車を買った。	
\\	かなり 無理[むり]をして 新車[しんしゃ]を 買[か]った。	
\\	ちょっと無理してこのダイヤ買っちゃった。	
\\	ちょっと 無理[むり]してこのダイヤ 買[か]っちゃった。	
\\	ニンジンを適宜な大きさに切って鍋に入れなさい。	
\\	ニンジンを 適宜[てきぎ]な 大[おお]きさに 切[き]って 鍋[なべ]に 入[い]れなさい。	適宜=てきぎ= (適当)
\\	(好きなように) 
\\	(事情により) 
\\	砂糖は適宜自分の好みで入れてください。	
\\	砂糖[さとう]は 適宜[てきぎ] 自分[じぶん]の 好[この]みで 入[い]れてください。	適宜=てきぎ= (適当)
\\	(好きなように) 
\\	(事情により) 
\\	この本は校正が雑で誤植が多い。	
\\	この 本[ほん]は 校正[こうせい]が 雑[ざつ]で 誤植[ごしょく]が 多[おお]い。	校正=こうせい= 
\\	誤植=ごしょく= 
\\	赤ん坊は両親の遺伝子を受け継いで生まれてくる。	
\\	赤ん坊[あかんぼう]は 両親[りょうしん]の 遺伝子[いでんし]を 受け継[うけつ]いで 生[う]まれてくる。	遺伝子=いでんし= 
\\	警察官を見たとたん、彼は逃げ出した。	
\\	警察官[けいさつかん]を 見[み]たとたん、 彼[かれ]は 逃げ出[にげだ]した。	
\\	その国への入国はビザの取得が必要です。	
\\	その 国[くに]への 入国[にゅうこく]はビザの 取得[しゅとく]が 必要[ひつよう]です。	
\\	私より妻の方が収入が多い。	
\\	私[わたし]より 妻[つま]の 方[ほう]が 収入[しゅうにゅう]が 多[おお]い。	
\\	これは仮のタイトルです。	
\\	これは 仮[かり]のタイトルです。	
\\	彼はいわば日本のエジソンだ。	
\\	彼[かれ]はいわば 日本[にっぽん]のエジソンだ。	
\\	試合開始から十分が経過しました。	
\\	試合[しあい] 開始[かいし]から 十分[じゅっぷん]が 経過[けいか]しました。	
\\	この市役所には英語が話せる職員が2名おります。	
\\	この 市役所[しやくしょ]には 英語[えいご]が 話[はな]せる 職員[しょくいん]が2 名[めい]おります。	
\\	世間の目なんて気にしない。	
\\	世間[せけん]の 目[め]なんて 気[き]にしない。	
\\	防犯カメラの映像が証拠として使われた。	
\\	防犯[ぼうはん]カメラの 映像[えいぞう]が 証拠[しょうこ]として 使[つか]われた。	
\\	美人は見慣れると飽きるよ。	
\\	美人[びじん]は 見慣[みな]れると 飽[あ]きるよ。	
\\	こんな大金は見慣れていない。	
\\	こんな 大金[たいきん]は 見慣[みな]れていない。	大金=たいきん・おおがね
\\	見慣れぬ光景であった。	
\\	見慣[みな]れぬ 光景[こうけい]であった。	
\\	この程度の散らかりは見慣れているので何とも思わない。	
\\	この 程度[ていど]の 散[ち]らかりは 見慣[みな]れているので 何[なん]とも 思[おも]わない。	
\\	私はこうした仕事には不慣れです。	
\\	私[わたし]はこうした 仕事[しごと]には 不慣[ふな]れです。	
\\	散歩するなり、泳ぐなり、何か運動をした方がいいですよ。	
\\	散歩[さんぽ]するなり、 泳[およ]ぐなり、 何[なに]か 運動[うんどう]をした 方[ほう]がいいですよ。	
\\	動物は動物なりのコミュニケーションができる。	
\\	動物[どうぶつ]は 動物[どうぶつ]なりのコミュニケーションができる。	「なりに」= 
\\	私は私なりに、人生観を持っています。	
\\	私[わたし]は 私[わたし]なりに、 人生[じんせい] 観[かん]を 持[も]っています。	「なりに」= 
\\	自転車は自転車なりに、車は車なりに、長所、短所がある。	
\\	自転車[じてんしゃ]は 自転車[じてんしゃ]なりに、 車[くるま]は 車[くるま]なりに、 長所[ちょうしょ]、 短所[たんしょ]がある。	「なりに」= 
\\	ジャズ音楽はジャズ音楽なりの魅力がある。	
\\	ジャズ 音楽[おんがく]はジャズ 音楽[おんがく]なりの 魅力[みりょく]がある。	「なりに」= 
\\	あなたなしでは生きていけない。	
\\	あなたなしでは 生[い]きていけない。	「なしでは」= 
\\	基礎研究なしでは科学は発展しない。	
\\	基礎[きそ] 研究[けんきゅう]なしでは 科学[かがく]は 発展[はってん]しない。	「なしでは」= 
\\	人種偏見をなくさねばならない。	
\\	人種[じんしゅ] 偏見[へんけん]をなくさねばならない。	
\\	昨日来れば夏子に会えただろうに。	
\\	昨日[きのう] 来[く]れば 夏子[なつこ]に 会[あ]えただろうに。	
\\	後一年ぐらい日本にいたら日本語がもっと上手になるでしょうに。	
\\	後[こう] 一年[いちねん]ぐらい 日本[にほん]にいたら 日本語[にほんご]がもっと 上手[じょうず]になるでしょうに。	
\\	日本の経済を研究するに当たって、国会図書館で資料集めをした。	
\\	日本[にほん]の 経済[けいざい]を 研究[けんきゅう]するに 当[あ]たって、 国会図書館[こっかいとしょかん]で 資料[しりょう] 集[あつ]めをした。	「に当たって・当たり」
\\	小説家は時代小説を書くにあたり、その時代の歴史を詳しく調べた。	
\\	小説[しょうせつ] 家[か]は 時代[じだい] 小説[しょうせつ]を 書[か]くにあたり、その 時代[じだい]の 歴史[れきし]を 詳[くわ]しく 調[しら]べた。	「に当たって・当たり」
\\	私は就寝に当たって少量の洋酒を喫することを習慣にしている。	
\\	私[わたし]は 就寝[しゅうしん]に 当[あ]たって 少量[しょうりょう]の 洋酒[ようしゅ]を 喫[きっ]することを 習慣[しゅうかん]にしている。	「に当たって・当たり」
\\	喫する=きっする= 
\\	浩は両親の期待に反して高校を出てからコックになった。	
\\	浩[ひろし]は 両親[りょうしん]の 期待[きたい]に 反[はん]して 高校[こうこう]を 出[で]てからコックになった。	
\\	奥村さんのうちはご主人が無口なのに反して奥さんが人一倍のおしゃべりだ。	
\\	奥村[おくむら]さんのうちはご 主人[しゅじん]が 無口[むくち]なのに 反[はん]して 奥[おく]さんが 人一倍[ひといちばい]のおしゃべりだ。	
\\	日本語を勉強しているのは将来日本で仕事をしたいからにほかならない。	
\\	日本語[にほんご]を 勉強[べんきょう]しているのは 将来[しょうらい] 日本[にほん]で 仕事[しごと]をしたいからにほかならない。	「にほかならない」
\\	外国語学習はほかの国の人の考え方を学ぶことにほかならない。	
\\	外国[がいこく] 語[ご] 学習[がくしゅう]はほかの 国[くに]の 人[ひと]の 考え方[かんがえかた]を 学[まな]ぶことにほかならない。	「にほかならない」
\\	結婚は人生の墓場にほかならない。	
\\	結婚[けっこん]は 人生[じんせい]の 墓場[はかば]にほかならない。	「にほかならない」
\\	墓場=はかば= 
\\	見合い結婚は日本に限らずほかの国でも行われている。	
\\	見合[みあ]い 結婚[けっこん]は 日本[にほん]に 限[かぎ]らずほかの 国[くに]でも 行[おこな]われている。	に限らず= 
\\	音楽はクラシックに限らず何でも聞きます。	
\\	音楽[おんがく]はクラシックに 限[かぎ]らず 何[なに]でも 聞[き]きます。	に限らず= 
\\	このバーは男性だけに限らず女性の間にも人気がある。	
\\	このバーは 男性[だんせい]だけに 限[かぎ]らず 女性[じょせい]の 間[あいだ]にも 人気[にんき]がある。	に限らず= 
\\	果物は何に限らず好きです。	
\\	果物[くだもの]は 何[なに]に 限[かぎ]らず 好[す]きです。	に限らず= 
\\	何事に限らず仕事は誠意をもって行うことが大切だ。	
\\	何事[なにごと]に 限[かぎ]らず 仕事[しごと]は 誠意[せいい]をもって 行[おこな]うことが 大切[たいせつ]だ。	に限らず= 
\\	傘を持って来ない日に限って雨が降るんですよ。	
\\	傘[かさ]を 持[も]って 来[こ]ない 日[ひ]に 限[かぎ]って 雨[あめ]が 降[ふ]るんですよ。	「〜に限って」= 
\\	急ぐ時に限って、バスがなかなか来ない。	
\\	急[いそ]ぐ 時[とき]に 限[かぎ]って、バスがなかなか 来[こ]ない。	「〜に限って」= 
\\	病気の山田先生に代わって、鈴木先生が教えて下さった。	
\\	病気[びょうき]の 山田[やまだ] 先生[せんせい]に 代[か]わって、 鈴木[すずき] 先生[せんせい]が 教[おし]えて 下[くだ]さった。	
\\	懸命な努力にもかかわらず、健一は大学入試に失敗した。	
\\	懸命[けんめい]な 努力[どりょく]にもかかわらず、 健一[けんいち]は 大学[だいがく] 入試[にゅうし]に 失敗[しっぱい]した。	
\\	あの人はよく運動をする(の)にもかかわらず、太っている。	
\\	あの 人[ひと]はよく 運動[うんどう]をする(の)にもかかわらず、 太[ふと]っている。	
\\	キャロルは日本に三年も住んでいたにもかかわらず、日本語は大変下手だ。	
\\	キャロルは 日本[にほん]に 三年[さんねん]も 住[す]んでいたにもかかわらず、 日本語[にほんご]は 大変[たいへん] 下手[へた]だ。	
\\	事実に基づいてお話しします。	
\\	事実[じじつ]に 基[もと]づいてお 話[はな]しします。	
\\	これは五百年前の資料に基づく研究だ。	
\\	これは五 百[ひゃく] 年[ねん] 前[まえ]の 資料[しりょう]に 基[もと]づく 研究[けんきゅう]だ。	
\\	私は夜十一時になると、頭が働かなくなる。	
\\	私[わたし]は 夜[よる] 十一時[じゅういちじ]になると、 頭[あたま]が 働[はたら]かなくなる。	
\\	子供の頃、夏になると、両親は僕を海に連れて行ってくれた。	
\\	子供[こども]の 頃[ころ]、 夏[なつ]になると、 両親[りょうしん]は 僕[ぼく]を 海[うみ]に 連[つ]れて 行[い]ってくれた。	
\\	コンピュータは近い将来においてほとんどの家庭に行き渡るだろう。	
\\	コンピュータは 近[ちか]い 将来[しょうらい]においてほとんどの 家庭[かてい]に 行き渡[ゆきわた]るだろう。	
\\	この作文は文法においてはあまり問題はない。	
\\	この 作文[さくぶん]は 文法[ぶんぽう]においてはあまり 問題[もんだい]はない。	
\\	日本の経済力が強くなるに従って、日本語学習者が増えてきた。	
\\	日本[にほん]の 経済[けいざい] 力[りょく]が 強[つよ]くなるに 従[したが]って、 日本語[にほんご] 学習[がくしゅう] 者[しゃ]が 増[ふ]えてきた。	
\\	年を取るに従い、体力が衰える。	
\\	年[とし]を 取[と]るに 従[したが]い、 体力[たいりょく]が 衰[おとろ]える。	
\\	これは私の私見に過ぎない。	
\\	これは 私[わたし]の 私見[しけん]に 過[す]ぎない。	
\\	これは数ある中のほんの一例に過ぎない。	
\\	これは 数[かず]ある 中[なか]のほんの 一例[いちれい]に 過[す]ぎない。	
\\	あの子はまだ十五に過ぎないが、なかなかしっかりしている。	
\\	あの 子[こ]はまだ 十五[じゅうご]に 過[す]ぎないが、なかなかしっかりしている。	
\\	日本は外国に対して市場をもっと開放すべきだ。	
\\	日本[にほん]は 外国[がいこく]に 対[たい]して 市場[しじょう]をもっと 開放[かいほう]すべきだ。	
\\	私は政治に対して強い関心がある。	
\\	私[わたし]は 政治[せいじ]に 対[たい]して 強[つよ]い 関心[かんしん]がある。	
\\	この仕事は一時間につき六ドルもらえる。	
\\	この 仕事[しごと]は 一時間[いちじかん]につき 六[ろく]ドルもらえる。	
\\	間違い一つにつき一点減点します。	
\\	間違[まちが]い 一[ひと]つにつき 一点[いってん] 減点[げんてん]します。	
\\	日本語が上達するにつれて、日本人の友達が増えた。	
\\	日本語[にほんご]が 上達[じょうたつ]するにつれて、 日本人[にほんじん]の 友達[ともだち]が 増[ふ]えた。	
\\	病気が治ってくるにつれて、食欲が出てきた。	
\\	病気[びょうき]が 治[なお]ってくるにつれて、 食欲[しょくよく]が 出[で]てきた。	
\\	子供は成長するにつれて、親から離れていく。	
\\	子供[こども]は 成長[せいちょう]するにつれて、 親[おや]から 離[はな]れていく。	
\\	ハイヒールはバイキングには不向きだ。	
\\	ハイヒールはバイキングには 不向[ふむ]きだ。	
\\	戦争によって父を亡くした。	
\\	戦争[せんそう]によって 父[ちち]を 亡[な]くした。	
\\	この研究所は文部省によって設立された。	
\\	この 研究所[けんきゅうじょ]は 文部省[もんぶしょう]によって 設立[せつりつ]された。	
\\	何を食べるかによって健康状態は変わる。	
\\	何[なに]を 食[た]べるかによって 健康[けんこう] 状態[じょうたい]は 変[か]わる。	
\\	僕はその日の気分によって、違う音楽を聞きます。	
\\	僕[ぼく]はその 日[ひ]の 気分[きぶん]によって、 違[ちが]う 音楽[おんがく]を 聞[き]きます。	
\\	定年は会社によって違う。	
\\	定年[ていねん]は 会社[かいしゃ]によって 違[ちが]う。	
\\	ホールさんは今仕事の関係で東京に行っています。	
\\	ホールさんは 今[いま] 仕事[しごと]の 関係[かんけい]で 東京[とうきょう]に 行[い]っています。	
\\	要点のみ話して下さい。	
\\	要点[ようてん]のみ 話[はな]して 下[くだ]さい。	
\\	姓のみ記入のこと。	
\\	姓[せい]のみ 記入[きにゅう]のこと。	
\\	私は鈴木先生の指導の下で修士論文を書き上げた。	
\\	私[わたし]は 鈴木[すずき] 先生[せんせい]の 指導[しどう]の 下[もと]で 修士[しゅうし] 論文[ろんぶん]を 書き上[かきあ]げた。	「の下で」=のもとで= 
\\	囚人達は厳しい監視の下で強制労働をさせられた。	
\\	囚人[しゅうじん] 達[たち]は 厳[きび]しい 監視[かんし]の 下[もと]で 強制[きょうせい] 労働[ろうどう]をさせられた。	「の下で」=のもとで= 
\\	彼の行為は法律の上では罰しようがない。	
\\	彼[かれ]の 行為[こうい]は 法律[ほうりつ]の 上[うえ]では 罰[ばっ]しようがない。	
\\	日本語の面白さが分かり始めたのはごく最近のことだ。	
\\	日本語[にほんご]の 面白[おもしろ]さが 分[わ]かり 始[はじ]めたのはごく 最近[さいきん]のことだ。	
\\	知らぬことを知らぬと言うには勇気が要る。	
\\	知[し]らぬことを 知[し]らぬと 言[い]うには 勇気[ゆうき]が 要[い]る。	
\\	私は親しい友達を通してそのピアニストと知り合いになった。	
\\	私[わたし]は 親[した]しい 友達[ともだち]を 通[とお]してそのピアニストと 知り合[しりあ]いになった。	
\\	私は一年を通して五回ぐらい海外に行っている。	
\\	私[わたし]は 一年[いちねん]を 通[とお]して五 回[かい]ぐらい 海外[かいがい]に 行[い]っている。	
\\	その事件のことは新聞の記事を通して知っていた。	
\\	その 事件[じけん]のことは 新聞[しんぶん]の 記事[きじ]を 通[とお]して 知[し]っていた。	
\\	現場に残された指紋を通して犯人が割れた。	
\\	現場[げんば]に 残[のこ]された 指紋[しもん]を 通[とお]して 犯人[はんにん]が 割[わ]れた。	
\\	新幹線が満員で東京から京都までずっと立ちっぱなしだった。	
\\	新幹線[しんかんせん]が 満員[まんいん]で 東京[とうきょう]から 京都[きょうと]までずっと 立[た]ちっぱなしだった。	
\\	友達にまだお金を借りっぱなしだ。	
\\	友達[ともだち]にまだお 金[かね]を 借[か]りっぱなしだ。	
\\	ブラジルのバレーボール・チームは今日まで勝ちっぱなしだ。	
\\	ブラジルのバレーボール・チームは 今日[きょう]まで 勝[か]ちっぱなしだ。	
\\	この学校は創立以来30年になった。	
\\	この 学校[がっこう]は 創立[そうりつ] 以来[いらい]30 年[ねん]になった。	
\\	彼には独創的なところがある。	
\\	彼[かれ]には 独創[どくそう] 的[てき]なところがある。	
\\	そんなにいすを引きずると床を傷めるじゃないか。	
\\	そんなにいすを 引[ひ]きずると 床[ゆか]を 傷[いた]めるじゃないか。	
\\	彼らの言うことは全くの中傷だ。	
\\	彼[かれ]らの 言[い]うことは 全[まった]くの 中傷[ちゅうしょう]だ。	
\\	友人から妙に感傷的な内容の手紙が届いた。	
\\	友人[ゆうじん]から 妙[みょう]に 感傷[かんしょう] 的[てき]な 内容[ないよう]の 手紙[てがみ]が 届[とど]いた。	
\\	良心に照らして恥じるところがない。	
\\	良心[りょうしん]に 照[て]らして 恥[は]じるところがない。	照らす= 
\\	そう言われると照れるなあ。	
\\	そう 言[い]われると 照[て]れるなあ。	照れる= 
\\	先生にほめられたその子は照れてうつむいた。	
\\	先生[せんせい]にほめられたその 子[こ]は 照[て]れてうつむいた。	照れる= 
\\	うつむく= 
\\	ヘッドライトが道に倒れている人を照らし出した。	
\\	ヘッドライトが 道[みち]に 倒[たお]れている 人[ひと]を 照[て]らし 出[だ]した。	
\\	あまり急いだのでそこに着いたときは息が弾んでいた。	
\\	あまり 急[いそ]いだのでそこに 着[つ]いたときは 息[いき]が 弾[はず]んでいた。	弾む=はずむ= 
\\	そのことから話が弾んだ。	
\\	そのことから 話[はなし]が 弾[はず]んだ。	弾む=はずむ= 
\\	ボールが弾まなくなった。	
\\	ボールが 弾[はず]まなくなった。	弾む=はずむ= 
\\	名古屋は素通りして大阪へ直行した。	
\\	名古屋[なごや]は 素通[すどお]りして 大阪[おおさか]へ 直行[ちょっこう]した。	素通り=すどおり= 
\\	これは60年代を語るとき素通りできない問題だ。	
\\	これは60 年代[ねんだい]を 語[かた]るとき 素通[すどお]りできない 問題[もんだい]だ。	素通り=すどおり= 
\\	その日から彼女のピアニストとしての歩みが始まった。	
\\	その 日[ひ]から 彼女[かのじょ]のピアニストとしての 歩[あゆ]みが 始[はじ]まった。	歩み=あゆみ= (歩くこと) 
\\	(歩調) 
\\	(物事の進行) 
\\	(過程) 
\\	(沿革) 
\\	我が社の50年の歩みをこの一冊にまとめました。	
\\	我[わ]が 社[しゃ]の50 年[ねん]の 歩[あゆ]みをこの 一冊[いっさつ]にまとめました。	歩み=あゆみ= (歩くこと) 
\\	(歩調) 
\\	(物事の進行) 
\\	(過程) 
\\	(沿革) 
\\	ついに両者の歩み寄りが見られた。	
\\	ついに 両者[りょうしゃ]の 歩み寄[あゆみよ]りが 見[み]られた。	歩み寄り=あゆみより= 
\\	関係諸国の歩み寄りが求められている。	
\\	関係[かんけい] 諸国[しょこく]の 歩み寄[あゆみよ]りが 求[もと]められている。	歩み寄り=あゆみより= 
\\	我々にはもう歩み寄りの余地はない。	
\\	我々[われわれ]にはもう 歩み寄[あゆみよ]りの 余地[よち]はない。	歩み寄り=あゆみより= 
\\	その知らせを聞いて気勢が上がった。	
\\	その 知[し]らせを 聞[き]いて 気勢[きせい]が 上[あ]がった。	気勢=きせい= 
\\	その報告書は初めから終わりまで戯言に等しい。	
\\	その 報告[ほうこく] 書[しょ]は 初[はじ]めから 終[お]わりまで 戯言[たわごと]に 等[ひと]しい。	戯言=たわごと= 
\\	この紙の長さと幅を測ってみて下さい。	
\\	この 紙[かみ]の 長[なが]さと 幅[はば]を 測[はか]ってみて 下[くだ]さい。	測る・計る・量る=はかる= 
\\	あの人の真意は量りかねる。	
\\	あの 人[ひと]の 真意[しんい]は 量[はか]りかねる。	測る・計る・量る=はかる= 
\\	観測史上の最高気温を記録した。	
\\	観測[かんそく] 史上[しじょう]の 最高[さいこう] 気温[きおん]を 記録[きろく]した。	観測=かんそく= 
\\	この決定に異を差し挟もうとは思わない。	
\\	この 決定[けってい]に 異[い]を 差し挟[さしはさ]もうとは 思[おも]わない。	差し挟む=さしはさむ= 
\\	その演説は教育界の物議をかもした。	
\\	その 演説[えんぜつ]は 教育[きょういく] 界[かい]の 物議[ぶつぎ]をかもした。	物議=ぶつぎ= 
\\	物議をかもす= 
\\	その政治家は人種差別に関する発言で物議を醸した。	
\\	その 政治[せいじ] 家[か]は 人種[じんしゅ] 差別[さべつ]に 関[かん]する 発言[はつげん]で 物議[ぶつぎ]を 醸[かも]した。	物議を醸す=ぶつぎをかもす= 
\\	母親は息子を大学に進学させようと必死になっていた。一方息子はと言えば、学校にも行かず遊び回っていた。	
\\	母親[ははおや]は 息子[むすこ]を 大学[だいがく]に 進学[しんがく]させようと 必死[ひっし]になっていた。 一方[いっぽう] 息子[むすこ]はと 言[い]えば、 学校[がっこう]にも 行[い]かず 遊び回[あそびまわ]っていた。	一方= 
\\	(~ばかり), 
\\	人口は増える一方だ。	
\\	人口[じんこう]は 増[ふ]える 一方[いっぽう]だ。	一方= 
\\	(~ばかり), 
\\	物価は上がる一方だ。	
\\	物価[ぶっか]は 上[あ]がる 一方[いっぽう]だ。	一方= 
\\	(~ばかり), 
\\	彼は学生のレポートの欠点を厳しく指摘する一方で、よいところをほめるのも忘れない。	
\\	彼[かれ]は 学生[がくせい]のレポートの 欠点[けってん]を 厳[きび]しく 指摘[してき]する 一方[いっぽう]で、よいところをほめるのも 忘[わす]れない。	一方= 
\\	(~ばかり), 
\\	その時期、彼はマルクス主義に打ち込んでいた。	
\\	その 時期[じき]、 彼[かれ]は マルクス主義[まるくすしゅぎ]に 打ち込[うちこ]んでいた。	
\\	親は子どもに、朝食を取るよう促すべきです。	
\\	親[おや]は 子[こ]どもに、 朝食[ちょうしょく]を 取[と]るよう 促[うなが]すべきです。	
\\	あいつのことなんか放っておけ。	
\\	あいつのことなんか 放[はな]っておけ。	放っておく=ほうっておく= 
\\	この問題は放っておけない。	
\\	この 問題[もんだい]は 放[はな]っておけない。	放っておく=ほうっておく= 
\\	あんなに困っているのを放ってはおけない。	
\\	あんなに 困[こま]っているのを 放[はな]ってはおけない。	放っておく=ほうっておく= 
\\	彼がいじめられているのを放ってはおけない。	
\\	彼[かれ]がいじめられているのを 放[はな]ってはおけない。	放っておく=ほうっておく= 
\\	このくらいのけがなら放っておいてもそのうち治るさ。	
\\	このくらいのけがなら 放[はな]っておいてもそのうち 治[なお]るさ。	放っておく=ほうっておく= 
\\	火勢がつのった。	
\\	火勢[かせい]がつのった。	募る=つのる= 
\\	夕刻から風が吹きつのってきた。	
\\	夕刻[ゆうこく]から 風[かぜ]が 吹[ふ]きつのってきた。	募る=つのる= 
\\	悲しみは募る一方であった。	
\\	悲[かな]しみは 募[つの]る 一方[いっぽう]であった。	募る=つのる= 
\\	首相は記者会見で次々に厳しい質問を浴びせられた。	
\\	首相[しゅしょう]は 記者[きしゃ] 会見[かいけん]で 次々[つぎつぎ]に 厳[きび]しい 質問[しつもん]を 浴[あ]びせられた。	
\\	それを聞いて彼への敵意を一段と募らせた。	
\\	それを 聞[き]いて 彼[かれ]への 敵意[てきい]を 一段[いちだん]と 募[つの]らせた。	募る=つのる= 
\\	彼は私への敵意を隠そうともしなかった。	
\\	彼[かれ]は 私[わたし]への 敵意[てきい]を 隠[かく]そうともしなかった。	
\\	私は彼に敵意は持っていない。	
\\	私[わたし]は 彼[かれ]に 敵意[てきい]は 持[も]っていない。	
\\	佐々木さんは、9月から東京本社に移る予定です。	
\\	佐々木[ささき]さんは、 9月[くがつ]から 東京[とうきょう] 本社[ほんしゃ]に 移[うつ]る 予定[よてい]です。	
\\	時世はすでに移っている。	
\\	時世[じせい]はすでに 移[うつ]っている。	
\\	世界経済の中心はロンドンからニューヨークに移った。	
\\	世界[せかい] 経済[けいざい]の 中心[ちゅうしん]はロンドンからニューヨークに 移[うつ]った。	
\\	話題は環境問題に移った。	
\\	話題[わだい]は 環境[かんきょう] 問題[もんだい]に 移[うつ]った。	
\\	彼は口ばっかりで、なかなか行動に移らない。	
\\	彼[かれ]は 口[くち]ばっかりで、なかなか 行動[こうどう]に 移[うつ]らない。	
\\	一日その会議に出ると、たばこの匂いがすっかり服に移ってしまう。	
\\	一 日[にち]その 会議[かいぎ]に 出[で]ると、たばこの 匂[にお]いがすっかり 服[ふく]に 移[うつ]ってしまう。	
\\	あくびは移るものだと思いませんか。	
\\	あくびは 移[うつ]るものだと 思[おも]いませんか。	
\\	隣家に火が移った。	
\\	隣家[りんか]に 火[ひ]が 移[うつ]った。	
\\	山々が湖水に映っていた。	
\\	山々[やまやま]が 湖水[こすい]に 映[うつ]っていた。	
\\	その子は水たまりに映った自分の顔をじっと眺めていた。	
\\	その 子[こ]は 水[みず]たまりに 映[うつ]った 自分[じぶん]の 顔[かお]をじっと 眺[なが]めていた。	水たまり= 
\\	彼女にはこの色がよく映る。	
\\	彼女[かのじょ]にはこの 色[いろ]がよく 映[うつ]る。	映る・写る= 
\\	(写真・映像など) 
\\	あの着物は彼女にはさっぱり映らない。	
\\	あの 着物[きもの]は 彼女[かのじょ]にはさっぱり 映[うつ]らない。	映る・写る= 
\\	(写真・映像など) 
\\	この写真はとてもよく写っている。	
\\	この 写真[しゃしん]はとてもよく 写[うつ]っている。	映る・写る= 
\\	(写真・映像など) 
\\	この写真に写っている女の子は僕の妹です。	
\\	この 写真[しゃしん]に 写[うつ]っている 女の子[おんなのこ]は 僕[ぼく]の 妹[いもうと]です。	映る・写る= 
\\	(写真・映像など) 
\\	その映画には山々の美しい景色が映っていた。	
\\	その 映画[えいが]には 山々[やまやま]の 美[うつく]しい 景色[けしき]が 映[うつ]っていた。	
\\	この使い捨てカメラは安くてよく写る。	
\\	この 使い捨[つかいす]てカメラは 安[やす]くてよく 写[うつ]る。	映る・写る= 
\\	(写真・映像など) 
\\	動物園でサルの写真を写した。	
\\	動物[どうぶつ] 園[えん]でサルの 写真[しゃしん]を 写[うつ]した。	写す= 
\\	図案をそっくり写した。	
\\	図案[ずあん]をそっくり 写[うつ]した。	写す= 
\\	看板から電話番号を写した。	
\\	看板[かんばん]から 電話[でんわ] 番号[ばんごう]を 写[うつ]した。	写す= 
\\	友人のノートを写させてもらった。	
\\	友人[ゆうじん]のノートを 写[うつ]させてもらった。	写す= 
\\	彼は鏡に自分の姿を映して見た。	
\\	彼[かれ]は 鏡[かがみ]に 自分[じぶん]の 姿[すがた]を 映[うつ]して 見[み]た。	映す= 
\\	彼が私に風邪をうつしたのです。	
\\	彼[かれ]が 私[わたし]に 風邪[かぜ]をうつしたのです。	移す= 
\\	新聞を読んでいた彼は視線をテレビに映した。	
\\	新聞[しんぶん]を 読[よ]んでいた 彼[かれ]は 視線[しせん]をテレビに 映[うつ]した。	移す= 
\\	その仕事を後任に移すことになるでしょう。	
\\	その 仕事[しごと]を 後任[こうにん]に 移[うつ]すことになるでしょう。	移す= 
\\	口の周りにジャムがついてるよ。	
\\	口[くち]の 周[まわ]りにジャムがついてるよ。	周り=まわり= 
\\	彼女は読書に夢中になって周りのことが全然目に入っていないようだった。	
\\	彼女[かのじょ]は 読書[どくしょ]に 夢中[むちゅう]になって 周[まわ]りのことが 全然[ぜんぜん] 目[め]に 入[はい]っていないようだった。	周り=まわり= 
\\	あの人は頭の回りが速い。	
\\	あの 人[ひと]は 頭[あたま]の 回[まわ]りが 速[はや]い。	
\\	蛇は脱皮できないと死ぬ。	
\\	蛇[へび]は 脱皮[だっぴ]できないと 死[し]ぬ。	脱皮=だっぴ= 
\\	会社もここら辺りで脱皮する必要がある。	
\\	会社[かいしゃ]もここら 辺[あた]りで 脱皮[だっぴ]する 必要[ひつよう]がある。	ここら辺り=ここらあたり= 
\\	脱皮=だっぴ= 
\\	日焼けで背中の皮がむけた。	
\\	日焼[ひや]けで 背中[せなか]の 皮[かわ]がむけた。	
\\	やけどで手の皮がむけそうだ。	
\\	やけどで 手[て]の 皮[かわ]がむけそうだ。	
\\	これは保険金目当ての殺人だと報じられている。	
\\	これは 保険[ほけん] 金[きん] 目当[めあ]ての 殺人[さつじん]だと 報[ほう]じられている。	
\\	「火事だ」という叫び声に皆我勝ちに外へ出ようとした。	
\\	火事[かじ]だ」という 叫び声[さけびごえ]に 皆[かい] 我勝[われが]ちに 外[そと]へ 出[で]ようとした。	我勝ちに・・・しようとする= 
\\	こういう時は誰でも我勝ちになるものだ。	
\\	こういう 時[とき]は 誰[だれ]でも 我勝[われが]ちになるものだ。	我勝ちに・・・しようとする= 
\\	彼はあやふやなことは言わない人だ。	
\\	彼[かれ]はあやふやなことは 言[い]わない 人[ひと]だ。	あやふや= 
\\	この節は車内のマナーもルールもない。	
\\	この 節[せつ]は 車内[しゃない]のマナーもルールもない。	節=せつ= 
\\	その節は大変お世話になりました。	
\\	その 節[せつ]は 大変[たいへん]お 世話[せわ]になりました。	節=せつ= 
\\	こちらへお出かけの節はぜひお立ち寄り下さい。	
\\	こちらへお 出[で]かけの 節[せつ]はぜひお 立ち寄[たちよ]り 下[くだ]さい。	節=せつ= 
\\	和歌を読むにも特有の節がある。	
\\	和歌[わか]を 読[よ]むにも 特有[とくゆう]の 節[ふし]がある。	節=ふし= 
\\	その辺はどこかで聞いたことのある節だ。	
\\	その 辺[へん]はどこかで 聞[き]いたことのある 節[ふし]だ。	節=ふし= 
\\	初戦に勝ってチームの士気は高揚した。	
\\	初戦[しょせん]に 勝[か]ってチームの 士気[しき]は 高揚[こうよう]した。	士気=しき= 
\\	高揚・昂揚=こうよう= 
\\	物価が甚だしく高騰している。	
\\	物価[ぶっか]が 甚[はなは]だしく 高騰[こうとう]している。	甚だしい=はなはだしい= 
\\	高騰=こうとう= 
\\	ほかに何とかしようがありそうなものだ。	
\\	ほかに 何[なん]とかしようがありそうなものだ。	
\\	住所が分からないので連絡のしようがない。	
\\	住所[じゅうしょ]が 分[わ]からないので 連絡[れんらく]のしようがない。	仕様=しよう= 
\\	(仕方); 
\\	そんなことを聞かれても返事のしようがない。	
\\	そんなことを 聞[き]かれても 返事[へんじ]のしようがない。	仕様=しよう= 
\\	(仕方); 
\\	多少の行動の自由は寛容するがよい。	
\\	多少[たしょう]の 行動[こうどう]の 自由[じゆう]は 寛容[かんよう]するがよい。	寛容=かんよう= 
\\	彼らのあの振る舞いは寛容できない。	
\\	彼[かれ]らのあの 振る舞[ふるま]いは 寛容[かんよう]できない。	寛容=かんよう= 
\\	ご寛容のほどをお願いします。	
\\	ご 寛容[かんよう]のほどをお 願[ねが]いします。	寛容=かんよう= 
\\	この商品は、1個いくらの価格で売ればもうけが出ますか?	
\\	この 商品[しょうひん]は、 1個[いっこ]いくらの 価格[かかく]で 売[う]ればもうけが 出[で]ますか?	価格=かかく= 値段; 
\\	もっと安い価格で手に入れる方法がある。	
\\	もっと 安[やす]い 価格[かかく]で 手[て]に 入[い]れる 方法[ほうほう]がある。	価格=かかく= 値段; 
\\	リンゴはちょっとした傷でも価格に響く。	
\\	リンゴはちょっとした 傷[きず]でも 価格[かかく]に 響[ひび]く。	価格=かかく= 値段; 
\\	天候が農作物の価格を左右する。	
\\	天候[てんこう]が 農作物[のうさくもつ]の 価格[かかく]を 左右[さゆう]する。	価格=かかく= 値段; 
\\	初めてパソコンが売り出されたころは僕の月給の3倍くらいの価格だった。	
\\	初[はじ]めてパソコンが 売り出[うりだ]されたころは 僕[ぼく]の 月給[げっきゅう]の3 倍[ばい]くらいの 価格[かかく]だった。	価格=かかく= 値段; 
\\	私の乗っているバスが乗っ取られた。	
\\	私[わたし]の 乗[の]っているバスが 乗っ取[のっと]られた。	
\\	彼女は彼に1、2歩歩み寄った。	
\\	彼女[かのじょ]は 彼[かれ]に1、2 歩[ほ] 歩み寄[あゆみよ]った。	歩み寄る=あゆみよる= (近寄る) 
\\	(折れ合う) 
\\	この交渉では結局双方が歩み寄る形となった。	
\\	この 交渉[こうしょう]では 結局[けっきょく] 双方[そうほう]が 歩み寄[あゆみよ]る 形[かたち]となった。	歩み寄る=あゆみよる= (近寄る) 
\\	(折れ合う) 
\\	彼は僕に当てつけて英字新聞を読んでいたのだ。	
\\	彼[かれ]は 僕[ぼく]に 当[あ]てつけて 英字[えいじ] 新聞[しんぶん]を 読[よ]んでいたのだ。	当てつける= 
\\	この川柳はあの政治家の汚職を当てこすっている。	
\\	この 川柳[せんりゅう]はあの 政治[せいじ] 家[か]の 汚職[おしょく]を 当[あ]てこすっている。	川柳=せんりゅう= 
\\	当てこする= 
\\	暑い夏を当て込んでビールの生産量を20%増しにした。	
\\	暑[あつ]い 夏[なつ]を 当て込[あてこ]んでビールの 生産[せいさん] 量[りょう]を20 
\\	[ぱーせんと] 増[ま]しにした。	当て込む= 
\\	ボーナスを当て込んで家具を注文した。	
\\	ボーナスを 当て込[あてこ]んで 家具[かぐ]を 注文[ちゅうもん]した。	当て込む= 
\\	彼は不機嫌なとき家族に当たり散らす。	
\\	彼[かれ]は 不機嫌[ふきげん]なとき 家族[かぞく]に 当たり散[あたりち]らす。	当たり散らす= 
\\	以前はがんを宣告されることは死の宣告を受けるに等しかった。	
\\	以前[いぜん]はがんを 宣告[せんこく]されることは 死[し]の 宣告[せんこく]を 受[う]けるに 等[ひと]しかった。	宣告=せんこく= 
\\	医者は彼にもう6ヶ月しか命がないと宣告した。	
\\	医者[いしゃ]は 彼[かれ]にもう6 ヶ月[かげつ]しか 命[いのち]がないと 宣告[せんこく]した。	宣告=せんこく= 
\\	葬式から戻ったら玄関先で塩をかけてもらって清めることになっている。	
\\	葬式[そうしき]から 戻[もど]ったら 玄関[げんかん] 先[さき]で 塩[しお]をかけてもらって 清[きよ]めることになっている。	清める=きよめる= 
\\	その歌声を聞くと心が清められる思いがする。	
\\	その 歌声[うたごえ]を 聞[き]くと 心[こころ]が 清[きよ]められる 思[おも]いがする。	清める=きよめる= 
\\	表彰台から仰ぎ見た日の丸は涙でにじんだ。	
\\	表彰台[ひょうしょうだい]から 仰[あお]ぎ 見[み]た 日の丸[ひのまる]は 涙[なみだ]でにじんだ。	表彰台=ひょうしょうだい= 
\\	仰ぎ見る=あおぎみる= 
\\	滲む=にじむ= 
\\	この紙は墨がにじむ。	
\\	この 紙[かみ]は 墨[すみ]がにじむ。	にじむ= 
\\	彼の腕の包帯には血がにじんでいた。	
\\	彼[かれ]の 腕[うで]の 包帯[ほうたい]には 血[ち]がにじんでいた。	にじむ= 
\\	彼の服のわきの下のところに汗がにじんでいた。	
\\	彼[かれ]の 服[ふく]のわきの 下[した]のところに 汗[あせ]がにじんでいた。	にじむ= 
\\	彼女の目には涙がにじんでいた。	
\\	彼女[かのじょ]の 目[め]には 涙[なみだ]がにじんでいた。	にじむ= 
\\	3月で契約が切れるのでそれまでに部屋を明け渡してほしい。	
\\	3月[さんがつ]で 契約[けいやく]が 切[き]れるのでそれまでに 部屋[へや]を 明け渡[あけわた]してほしい。	明け渡す= 
\\	政権を野党に明け渡すわけにはいかない。	
\\	政権[せいけん]を 野党[やとう]に 明け渡[あけわた]すわけにはいかない。	明け渡す= 
\\	高齢のため佐々木氏は社長の席を田中氏に明け渡して相談役になった。	
\\	高齢[こうれい]のため 佐々木[ささき] 氏[し]は 社長[しゃちょう]の 席[せき]を 田中[たなか] 氏[し]に 明け渡[あけわた]して 相談役[そうだんやく]になった。	明け渡す= 
\\	壁画には彩色が施してあった。	
\\	壁画[へきが]には 彩色[さいしき]が 施[ほどこ]してあった。	彩色=さいしき= 
\\	旬の野菜は値段も安く栄養も豊富だ。	
\\	旬[しゅん]の 野菜[やさい]は 値段[ねだん]も 安[やす]く 栄養[えいよう]も 豊富[ほうふ]だ。	旬=しゅん= 
\\	あの役者は今が旬だ。	
\\	あの 役者[やくしゃ]は 今[いま]が 旬[しゅん]だ。	旬=しゅん= 
\\	カキは今が旬だ。	
\\	カキは 今[いま]が 旬[しゅん]だ。	旬=しゅん= 
\\	私たちは出費を切り詰めるようにしないとね。	
\\	私[わたし]たちは 出費[しゅっぴ]を 切り詰[きりつ]めるようにしないとね。	出費=しゅっぴ= 
\\	切り詰める= 切って短くする; 
\\	警察は犯人をとうとう袋小路に追い詰めた。	
\\	警察[けいさつ]は 犯人[はんにん]をとうとう 袋小路[ふくろこうじ]に 追い詰[おいつ]めた。	袋小路=ふくろこうじ= 
\\	君のせいで負けたんじゃない、あまり自分を追い詰めない方がいいよ。	
\\	君[きみ]のせいで 負[ま]けたんじゃない、あまり 自分[じぶん]を 追い詰[おいつ]めない 方[ほう]がいいよ。	
\\	彼は立ち止まってその像をじっと見詰めていた。	
\\	彼[かれ]は 立ち止[たちど]まってその 像[ぞう]をじっと 見詰[みつ]めていた。	見詰める=みつめる= (目前を) 
\\	(過去や未来を) 
\\	(自己を) 
\\	少女は地面を見つめたまま顔を上げなかった。	
\\	少女[しょうじょ]は 地面[じめん]を 見[み]つめたまま 顔[かお]を 上[あ]げなかった。	見詰める=みつめる= (目前を) 
\\	(過去や未来を) 
\\	(自己を) 
\\	バブル崩壊でその銀行は経営が行き詰まった。	
\\	バブル 崩壊[ほうかい]でその 銀行[ぎんこう]は 経営[けいえい]が 行き詰[いきづ]まった。	行き詰まる= 
\\	3章まで書いたところで行き詰まった。	
\\	章[しょう]まで 書[か]いたところで 行き詰[いきづ]まった。	行き詰まる= 
\\	開け放たれたケージから何百羽もの鳩がいっせいに飛び立った。	
\\	開け放[あけはな]たれたケージから 何[なん] 百[ひゃく] 羽[わ]もの 鳩[はと]がいっせいに 飛び立[とびた]った。	羽=わ= 
\\	鳩=はと= 
\\	開け放す=あけはなす(「あけはなつ」も 
\\	開け放したままのドアからにぎやかな笑い声が聞こえた。	
\\	開け放[あけはな]したままのドアからにぎやかな 笑い声[わらいごえ]が 聞[き]こえた。	開け放す=あけはなす(「あけはなつ」も 
\\	寄りかかるのはやめろ。	
\\	寄[よ]りかかるのはやめろ。	
\\	彼は成人しても親に寄りかかって暮らしている。	
\\	彼[かれ]は 成人[せいじん]しても 親[おや]に 寄[よ]りかかって 暮[く]らしている。	成人=せいじん= 
\\	追悼式では事故の犠牲者全員の名前が厳かに読み上げられた。	
\\	追悼[ついとう] 式[しき]では 事故[じこ]の 犠牲[ぎせい] 者[しゃ] 全員[ぜんいん]の 名前[なまえ]が 厳[おごそ]かに 読み上[よみあ]げられた。	追悼式=ついとうしき= 
\\	犠牲者=ぎせいしゃ= 事件や事故により、生命を失うなどの重大な損害を受けた人。
\\	この問題を理屈で割り切ることはできない。	
\\	この 問題[もんだい]を 理屈[りくつ]で 割り切[わりき]ることはできない。	理屈=りくつ= 
\\	割り切る= 
\\	何をやらせても彼女は割り切って行動している。	
\\	何[なに]をやらせても 彼女[かのじょ]は 割り切[わりき]って 行動[こうどう]している。	割り切る= 
\\	その判決に私は割り切れない気持ちを抱いている。	
\\	その 判決[はんけつ]に 私[わたし]は 割り切[わりき]れない 気持[きも]ちを 抱[いだ]いている。	割り切れる= 
\\	彼の説明にはどこか割り切れないところがある。	
\\	彼[かれ]の 説明[せつめい]にはどこか 割り切[わりき]れないところがある。	割り切れる= 
\\	16は5では割り切れない。	
\\	16は5では 割り切[わりき]れない。	
\\	30は6で割り切れる。	
\\	30は6で 割り切[わりき]れる。	割り切れる= 
\\	階段で転んでむこうずねをひどく打ち付けてしまった。	
\\	階段[かいだん]で 転[ころ]んでむこうずねをひどく 打ち付[うちつ]けてしまった。	向こう脛=むこうずね= 
\\	スキャンダルが報道された日、その会社の株は過去最安値まで売り込まれた。	
\\	スキャンダルが 報道[ほうどう]された 日[ひ]、その 会社[かいしゃ]の 株[かぶ]は 過去[かこ] 最[さい] 安値[やすね]まで 売り込[うりこ]まれた。	最安値=さいやすね売り込む= 
\\	君のことは僕が部長に売り込んでおいてやるよ。	
\\	君[きみ]のことは 僕[ぼく]が 部長[ぶちょう]に 売り込[うりこ]んでおいてやるよ。	売り込む= 
\\	もっと積極的に自分を売り込んだ方がいいよ。	
\\	もっと 積極[せっきょく] 的[てき]に 自分[じぶん]を 売り込[うりこ]んだ 方[ほう]がいいよ。	売り込む= 
\\	この機会に当店の名前を建設業界に売り込んでおこう。	
\\	この 機会[きかい]に 当店[とうてん]の 名前[なまえ]を 建設[けんせつ] 業界[ぎょうかい]に 売り込[うりこ]んでおこう。	売り込む= 
\\	彼はライバル会社である我が社に自社の機密情報を売り込んできた。	
\\	彼[かれ]はライバル 会社[かいしゃ]である 我[わ]が 社[しゃ]に 自社[じしゃ]の 機密[きみつ] 情報[じょうほう]を 売り込[うりこ]んできた。	売り込む= 
\\	彼はちょうど画家として売り出してきたときに亡くなった。	
\\	彼[かれ]はちょうど 画家[がか]として 売り出[うりだ]してきたときに 亡[な]くなった。	売り出す= 
\\	その製品は売り出してからわずか3ヶ月後に製造中止となった。	
\\	その 製品[せいひん]は 売り出[うりだ]してからわずか3 ヶ月[かげつ] 後[ご]に 製造[せいぞう] 中止[ちゅうし]となった。	売り出す= 
\\	彼は君の言葉を個人攻撃と受け止めたのだ。	
\\	彼[かれ]は 君[きみ]の 言葉[ことば]を 個人[こじん] 攻撃[こうげき]と 受け止[うけと]めたのだ。	
\\	政府は今度の事件に対し、冷静な受け止め方をしている。	
\\	政府[せいふ]は 今度[こんど]の 事件[じけん]に 対[たい]し、 冷静[れいせい]な 受け止[うけと]め 方[かた]をしている。	
\\	この支店は関西地区での販売を受け持っている。	
\\	この 支店[してん]は 関西[かんさい] 地区[ちく]での 販売[はんばい]を 受け持[うけも]っている。	
\\	田中先生は物理と化学の両方の授業を受け持っている。	
\\	田中[たなか] 先生[せんせい]は 物理[ぶつり]と 化学[かがく]の 両方[りょうほう]の 授業[じゅぎょう]を 受け持[うけも]っている。	
\\	平日の炊事は私が受け持っている。	
\\	平日[へいじつ]の 炊事[すいじ]は 私[わたし]が 受け持[うけも]っている。	炊事=すいじ= 
\\	日本から派遣された平和維持部隊はとても危険な地域を受け持っている。	
\\	日本から 派遣[はけん]された 平和[へいわ] 維持[いじ] 部隊[ぶたい]はとても 危険[きけん]な 地域[ちいき]を 受け持[うけも]っている。	
\\	彼の努力も空しかった。	
\\	彼[かれ]の 努力[どりょく]も 空[むな]しかった。	空しい・虚しい=むなしい= 
\\	実験の成果は空しかった。	
\\	実験[じっけん]の 成果[せいか]は 空[むな]しかった。	成果= 
\\	空しい・虚しい=むなしい= 
\\	黙って兄のグラブを使ったらバットであざができるほどひどく打たれた。	
\\	黙[だま]って 兄[あに]のグラブを 使[つか]ったらバットであざができるほどひどく 打[う]たれた。	痣=あざ= 
\\	(目の周りの) 
\\	彼女は右頬に痣があって化粧で隠している。	
\\	彼女[かのじょ]は 右[みぎ] 頬[ほお]に 痣[あざ]があって 化粧[けしょう]で 隠[かく]している。	痣=あざ= 
\\	(目の周りの) 
\\	きのう机の角にぶつけたところがあざになった。	
\\	きのう 机[つくえ]の 角[かく]にぶつけたところがあざになった。	痣=あざ= 
\\	(目の周りの) 
\\	マリちゃん、このあざはどうしたの?転んだの?ぶたれたの?	
\\	マリちゃん、このあざはどうしたの? 転[ころ]んだの?ぶたれたの?	痣=あざ= 
\\	(目の周りの) 
\\	一生懸命つくっても、レトルトの方がおいしいと言われては立つ瀬がない。	
\\	一生懸命[いっしょうけんめい]つくっても、レトルトの 方[ほう]がおいしいと 言[い]われては 立つ瀬[たつせ]がない。	レトルト= 
\\	立つ瀬=たつせ= 
\\	物事を半端でやめるのは僕は嫌いだ。	
\\	物事[ものごと]を 半端[はんぱ]でやめるのは 僕[ぼく]は 嫌[きら]いだ。	
\\	あなたは何か勘違いしているんじゃない?	
\\	あなたは 何[なに]か 勘違[かんちが]いしているんじゃない?	勘違い(~する)= 
\\	君は私を誰かと勘違いしているね。	
\\	君[きみ]は 私[わたし]を 誰[だれ]かと 勘違[かんちが]いしているね。	勘違い(~する)= 
\\	このスタジアムの整備には時間がかかる。	
\\	このスタジアムの 整備[せいび]には 時間[じかん]がかかる。	整備=せいび= 
\\	その工場には最新の機械が整備されている。	
\\	その 工場[こうじょう]には 最新[さいしん]の 機械[きかい]が 整備[せいび]されている。	整備=せいび= 
\\	事故は整備不良が原因だった。	
\\	事故[じこ]は 整備[せいび] 不良[ふりょう]が 原因[げんいん]だった。	整備=せいび= 
\\	ただ今その飛行機は整備点検中です。	
\\	ただ 今[いま]その 飛行機[ひこうき]は 整備[せいび] 点検[てんけん] 中[ちゅう]です。	整備=せいび= 
\\	彼女は結婚に対して否定的だ。	
\\	彼女[かのじょ]は 結婚[けっこん]に 対[たい]して 否定[ひてい] 的[てき]だ。	否定的= 
\\	君は何でも否定的に考えすぎる。	
\\	君[きみ]は 何[なん]でも 否定[ひてい] 的[てき]に 考[かんが]えすぎる。	否定的= 
\\	老人が悪徳商法によって被害を被ることが多い。	
\\	老人[ろうじん]が 悪徳[あくとく] 商法[しょうほう]によって 被害[ひがい]を 被[こうむ]ることが 多[おお]い。	悪徳商法= 
\\	いつまでも悔やんでいては亡くなったご主人も浮かばれませんよ。	
\\	いつまでも 悔[く]やんでいては 亡[な]くなったご 主人[しゅじん]も 浮[う]かばれませんよ。	悔やむ= 
\\	浮かばれない= 
\\	今さら悔やんでも始まらない。	
\\	今[いま]さら 悔[く]やんでも 始[はじ]まらない。	悔やむ= 
\\	始まらない= 
\\	このハンドバッグは本物をそっくり模倣したものだ。	
\\	このハンドバッグは 本物[ほんもの]をそっくり 模倣[もほう]したものだ。	模倣=もほう= (まねること) 
\\	(まねたもの) 
\\	子供は生まれつき模倣を好むものだ。	
\\	子供[こども]は 生[う]まれつき 模倣[もほう]を 好[この]むものだ。	生まれつき= 
\\	模倣=もほう= (まねること) 
\\	(まねたもの) 
\\	彼らは模倣ばかりで創造力に乏しい。	
\\	彼[かれ]らは 模倣[もほう]ばかりで 創造[そうぞう] 力[りょく]に 乏[とぼ]しい。	模倣=もほう= (まねること) 
\\	(まねたもの) 
\\	弘君がお前と仲直りしようと言ってきたよ。	
\\	弘[ひろし] 君[くん]がお 前[まえ]と 仲直[なかなお]りしようと 言[い]ってきたよ。	
\\	子供たちは喧嘩をしたがすぐ仲直りした。	
\\	子供[こども]たちは 喧嘩[けんか]をしたがすぐ 仲直[なかなお]りした。	
\\	いいかげんに仲直りしたらどうだ。	
\\	いいかげんに 仲直[なかなお]りしたらどうだ。	
\\	二人とも意地っぱりでなかなか仲直りできない。	
\\	二人[ふたり]とも 意地[いじ]っぱりでなかなか 仲直[なかなお]りできない。	意地っ張り=いじっぱり= 
\\	仲直りに握手しよう。	
\\	仲直[なかなお]りに 握手[あくしゅ]しよう。	
\\	私は今年は5年生を預かっています。	
\\	私[わたし]は 今年[ことし]は5 年生[ねんせい]を 預[あず]かっています。	預かる=あずかる= 
\\	私が家計を預かっている。	
\\	私[わたし]が 家計[かけい]を 預[あず]かっている。	預かる=あずかる= 
\\	(教育機関などで) 預かっている子供に事故があってはならない。	
\\	教育[きょういく] 機関[きかん]などで) 預[あず]かっている 子供[こども]に 事故[じこ]があってはならない。	預かる=あずかる= 
\\	留守の間うちのカナリヤを預かってもらえませんか。	
\\	留守[るす]の 間[ま]うちのカナリヤを 預[あず]かってもらえませんか。	預かる=あずかる= 
\\	私たちの市では寝たきりの老人を日中預かるデイサービスが始まった。	
\\	私[わたし]たちの 市[し]では 寝[ね]たきりの 老人[ろうじん]を 日中[にっちゅう] 預[あず]かるデイサービスが 始[はじ]まった。	寝たきり= 
\\	日中=にっちゅう 預かる=あずかる= 
\\	姪が病気なので、今彼女の子を預かっています。	
\\	姪[めい]が 病気[びょうき]なので、 今[いま] 彼女[かのじょ]の 子[こ]を 預[あず]かっています。	姪=めい= 
\\	預かる=あずかる= 
\\	この金を預かっておいてくれ。	
\\	この 金[きん]を 預[あず]かっておいてくれ。	
\\	母から先生に手紙を預かってきました。	
\\	母[はは]から 先生[せんせい]に 手紙[てがみ]を 預[あず]かってきました。	預かる=あずかる= 
\\	今日の気温は35度を超えるでしょう。	
\\	今日[きょう]の 気温[きおん]は35 度[ど]を 超[こ]えるでしょう。	
\\	その事故による死者は100名を超えた。	
\\	その 事故[じこ]による 死者[ししゃ]は100 名[めい]を 超[こ]えた。	
\\	この台風による被害は10億円を超えたという。	
\\	この 台風[たいふう]による 被害[ひがい]は10 億[おく] 円[えん]を 超[こ]えたという。	
\\	あの候補者は世代を超えた圧倒的支持を集めている。	
\\	あの 候補[こうほ] 者[しゃ]は 世代[せだい]を 超[こ]えた 圧倒的[あっとうてき] 支持[しじ]を 集[あつ]めている。	
\\	事件はようやく山を越した観がある。	
\\	事件[じけん]はようやく 山[やま]を 越[こ]した 観[かん]がある。	
\\	あの病人は年を越せるだろうか。	
\\	あの 病人[びょうにん]は 年[とし]を 越[こ]せるだろうか。	
\\	こういうことにかけては彼を越す者はない。	
\\	こういうことにかけては 彼[かれ]を 越[こ]す 者[もの]はない。	
\\	隣の家に越してまいりました。	
\\	隣[となり]の 家[いえ]に 越[こ]してまいりました。	
\\	応募者は千人を越すと思われる。	
\\	応募[おうぼ] 者[しゃ]は 千[せん] 人[にん]を 越[こ]すと 思[おも]われる。	
\\	彼の演説は所定の時間を越した。	
\\	彼[かれ]の 演説[えんぜつ]は 所定[しょてい]の 時間[じかん]を 越[こ]した。	
\\	それは一利一害だ。	
\\	それは 一利一害[いちりいちがい]だ。	一利一害=いちりいちがい= 
\\	生徒たちの主張にも一理ある。	
\\	生徒[せいと]たちの 主張[しゅちょう]にも 一理[いちり]ある。	
\\	それも一理ある。	
\\	それも 一理[いちり]ある。	
\\	君の言うことにも一理ある。	
\\	君[きみ]の 言[い]うことにも 一理[いちり]ある。	
\\	日本経済の不振の影響は単に日本国内に止まるものではない。	
\\	日本 経済[けいざい]の 不振[ふしん]の 影響[えいきょう]は 単[たん]に日本 国内[こくない]に 止[と]まるものではない。	留まる・止まる=とどまる= 
\\	彼女の欲望は止まることがなかった。	
\\	彼女[かのじょ]の 欲望[よくぼう]は 止[と]まることがなかった。	留まる・止まる=とどまる= 
\\	白書は何らの改善案も示さず、現状を述べるに止まっている。	
\\	白書[はくしょ]は 何[なん]らの 改善[かいぜん] 案[あん]も 示[しめ]さず、 現状[げんじょう]を 述[の]べるに 止[と]まっている。	何ら=なんら留まる・止まる=とどまる= 
\\	それは単なるうわさに止まった。	
\\	それは 単[たん]なるうわさに 止[と]まった。	留まる・止まる=とどまる= 
\\	日本選手の獲得したメダルは、わずか2個に止まった。	
\\	日本 選手[せんしゅ]の 獲得[かくとく]したメダルは、わずか2 個[こ]に 止[とど]まった。	留まる・止まる=とどまる= 
\\	ぜひ我が社に留まってもらいたい。	
\\	ぜひ 我[わ]が 社[しゃ]に 留[と]まってもらいたい。	留まる・止まる=とどまる= 
\\	戦時中も東京に留まっていた。	
\\	戦時[せんじ] 中[ちゅう]も 東京[とうきょう]に 留[と]まっていた。	留まる・止まる=とどまる= 
\\	彼女は定年後も顧問として会社に留まった。	
\\	彼女[かのじょ]は 定年[ていねん] 後[ご]も 顧問[こもん]として 会社[かいしゃ]に 留[と]まった。	顧問=こもん= 
\\	留まる・止まる=とどまる= 
\\	そこに一ヶ月ほど留まった。	
\\	そこに 一ヶ月[いっかげつ]ほど 留[と]まった。	留まる・止まる=とどまる= 
\\	販売成績は着実に伸びている。	
\\	販売[はんばい] 成績[せいせき]は 着実[ちゃくじつ]に 伸[の]びている。	
\\	我が社の業績は着実に伸びている。	
\\	我[わ]が 社[しゃ]の 業績[ぎょうせき]は 着実[ちゃくじつ]に 伸[の]びている。	
\\	彼女はゆっくりだが着実に仕事を進める。	
\\	彼女[かのじょ]はゆっくりだが 着実[ちゃくじつ]に 仕事[しごと]を 進[すす]める。	
\\	この種のビデオは青少年の性犯罪を助長する恐れがある。	
\\	この 種[たね]のビデオは 青少年[せいしょうねん]の 性[せい] 犯罪[はんざい]を 助長[じょちょう]する 恐[おそ]れがある。	助長=じょちょう= 
\\	衛生は洋上に落下した。	
\\	衛生[えいせい]は 洋上[ようじょう]に 落下[らっか]した。	洋上=ようじょう= 
\\	崖の上から岩が落下した。	
\\	崖[がけ]の 上[うえ]から 岩[いわ]が 落下[らっか]した。	崖=がけ= 
\\	その教師は生徒に対していつも一貫した姿勢で臨んでいた。	
\\	その 教師[きょうし]は 生徒[せいと]に 対[たい]していつも 一貫[いっかん]した 姿勢[しせい]で 臨[のぞ]んでいた。	一貫した= 
\\	臨む=のぞむ= 
\\	彼女は一貫して容疑を否認し続けた。	
\\	彼女[かのじょ]は 一貫[いっかん]して 容疑[ようぎ]を 否認[ひにん]し 続[つづ]けた。	
\\	物事を一貫して考えることのできない男だ。	
\\	物事[ものごと]を 一貫[いっかん]して 考[かんが]えることのできない 男[おとこ]だ。	一貫した= 
\\	議論は白熱した。	
\\	議論[ぎろん]は 白熱[はくねつ]した。	
\\	計画は挫折した。	
\\	計画[けいかく]は 挫折[ざせつ]した。	挫折=ざせつ= 
\\	計画); 
\\	独りで禁煙を誓っても挫折しがちだ。	
\\	独[ひと]りで 禁煙[きんえん]を 誓[ちか]っても 挫折[ざせつ]しがちだ。	誓う=ちかう= 
\\	挫折=ざせつ= 
\\	計画); 
\\	そのレストランでは客から丸見えのところで調理をしていた。	
\\	そのレストランでは 客[きゃく]から 丸見[まるみ]えのところで 調理[ちょうり]をしていた。	丸見え=まるみえ= 
\\	この部屋は通りから丸見えだ。	
\\	この 部屋[へや]は 通[とお]りから 丸見[まるみ]えだ。	丸見え=まるみえ= 
\\	あの山はなんとなく神秘的だ。	
\\	あの 山[やま]はなんとなく 神秘[しんぴ] 的[てき]だ。	神秘的=しんぴてき= 
\\	その件については総合的に判断する。	
\\	その 件[けん]については 総合[そうごう] 的[てき]に 判断[はんだん]する。	総合= 
\\	(~する) 
\\	生まれつき探究心の強い人がいる。	
\\	生[う]まれつき 探究[たんきゅう] 心[しん]の 強[つよ]い 人[ひと]がいる。	探究=たんきゅう= 
\\	手探りで非常口にたどり着いた。	
\\	手探[てさぐ]りで 非常口[ひじょうぐち]にたどり 着[つ]いた。	手探り= 
\\	たどり着く= 
\\	目をつむって手探りで箱の中身が何か当ててください。	
\\	目[め]をつむって 手探[てさぐ]りで 箱[はこ]の 中身[なかみ]が 何[なに]か 当[あ]ててください。	手探り= 
\\	この病気の研究はいまだに手探り状態だ。	
\\	この 病気[びょうき]の 研究[けんきゅう]はいまだに 手探[てさぐ]り 状態[じょうたい]だ。	手探り= 
\\	地震予知にはまだ手探りの部分が多い。	
\\	地震[じしん] 予知[よち]にはまだ 手探[てさぐ]りの 部分[ぶぶん]が 多[おお]い。	予知=よち= 
\\	手探り= 
\\	この店を開いて1年、まだ毎日が手探りの状態です。	
\\	この 店[みせ]を 開[あ]いて1 年[ねん]、まだ 毎日[まいにち]が 手探[てさぐ]りの 状態[じょうたい]です。	開く=あく= 
\\	手探り= 
\\	まったく君の無知には驚いて物が言えないよ。	
\\	まったく 君[きみ]の 無知[むち]には 驚[おどろ]いて 物[もの]が 言[い]えないよ。	
\\	科学技術の進歩に慣れてしまって、私たちはもうどんな新製品にも驚かなくなった。	
\\	科学[かがく] 技術[ぎじゅつ]の 進歩[しんぽ]に 慣[な]れてしまって、 私[わたし]たちはもうどんな 新[しん] 製品[せいひん]にも 驚[おどろ]かなくなった。	
\\	その程度の贈り物じゃ彼女は驚かないよ。	
\\	その 程度[ていど]の 贈り物[おくりもの]じゃ 彼女[かのじょ]は 驚[おどろ]かないよ。	
\\	この薬を飲んでも驚くような効果は期待できないだろう。	
\\	この 薬[くすり]を 飲[の]んでも 驚[おどろ]くような 効果[こうか]は 期待[きたい]できないだろう。	
\\	犬の声に驚いて賊は何も取らずに逃げた。	
\\	犬[いぬ]の 声[こえ]に 驚[おどろ]いて 賊[ぞく]は 何[なに]も 取[と]らずに 逃[に]げた。	賊=ぞく= 
\\	銃声を聞いても彼は少しも驚いたようすを見せなかった。	
\\	銃声[じゅうせい]を 聞[き]いても 彼[かれ]は 少[すこ]しも 驚[おどろ]いた 様子[ようす]を 見[み]せなかった。	銃声=じゅうせい= 
\\	驚いたことには彼は一夜にして巨万の富をなした。	
\\	驚[おどろ]いたことには 彼[かれ]は 一夜[いちや]にして 巨万[きょまん]の 富[とみ]をなした。	
\\	両親を驚かせたくて連絡せずに帰省した。	
\\	両親[りょうしん]を 驚[おどろ]かせたくて 連絡[れんらく]せずに 帰省[きせい]した。	帰省=きせい= 
\\	君を大いに驚かせることがある。	
\\	君[きみ]を 大[おお]いに 驚[おどろ]かせることがある。	
\\	ケネディ大統領暗殺のニュースは全世界を驚かせた。	
\\	ケネディ 大統領[だいとうりょう] 暗殺[あんさつ]のニュースは 全[ぜん] 世界[せかい]を 驚[おどろ]かせた。	
\\	私の心はのどかであった。	
\\	私[わたし]の 心[こころ]はのどかであった。	長閑=のどか= (~な) 
\\	田舎はのどかだねえ。	
\\	田舎[いなか]はのどかだねえ。	長閑=のどか= (~な) 
\\	レポートの提出は一応来週の月曜日を目安としよう。	
\\	レポートの 提出[ていしゅつ]は 一応[いちおう] 来週[らいしゅう]の 月曜日[げつようび]を 目安[めやす]としよう。	目安=めやす=目当て。目標。おおよその基準。また、おおよその見当。
\\	あの人は話を誇張する癖がある。	
\\	あの 人[ひと]は 話[はなし]を 誇張[こちょう]する 癖[くせ]がある。	誇張=こちょう= 
\\	これは誇張ではありませんよ。	
\\	これは 誇張[こちょう]ではありませんよ。	誇張=こちょう= 
\\	不法入国の手引きをする組織がある。	
\\	不法[ふほう] 入国[にゅうこく]の 手引[てび]きをする 組織[そしき]がある。	手引(き)=てびき= 
\\	この本は初めて釣りをする人の良い手引きになる。	
\\	この 本[ほん]は 初[はじ]めて 釣[つ]りをする 人[ひと]の 良[よ]い 手引[てび]きになる。	手引(き)=てびき= 
\\	彼女はこの森の保護活動に携わって十年になる。	
\\	彼女[かのじょ]はこの 森[もり]の 保護[ほご] 活動[かつどう]に 携[たずさ]わって 十年[じゅうねん]になる。	携わる=たずさわる=1)ある物事に関係する。従事する。           
\\	手を取り合う。連れ立つ。
\\	彼女にはうかつな話はするな。	
\\	彼女[かのじょ]にはうかつな 話[はなし]はするな。	迂闊=うかつ=(な) 
\\	不注意
\\	あの人とは同席したくない。	
\\	あの 人[ひと]とは 同席[どうせき]したくない。	
\\	当社は今年20人の技術者を採る。	
\\	当社[とうしゃ]は 今年[ことし]20 人[にん]の 技術[ぎじゅつ] 者[しゃ]を 採[と]る。	採る・摂る=とる= 
\\	採るべき道はただ一つだ。	
\\	採[と]るべき 道[みち]はただ 一[ひと]つだ。	採る・摂る=とる= 
\\	彼は弟子に人望が厚い。	
\\	彼[かれ]は 弟子[でし]に 人望[じんぼう]が 厚[あつ]い。	人望=じんぼう= 
\\	あの先生は生徒に人望がある。	
\\	あの 先生[せんせい]は 生徒[せいと]に 人望[じんぼう]がある。	人望=じんぼう= 
\\	この方がはるかによい。	
\\	この 方[ほう]がはるかによい。	
\\	人間が登場するはるか以前に恐竜は滅びていた。	
\\	人間[にんげん]が 登場[とうじょう]するはるか 以前[いぜん]に 恐竜[きょうりゅう]は 滅[ほろ]びていた。	恐竜=きょうりゅう= 
\\	滅びる=ほろびる= 
\\	2人の中では彼女の方がはるかに成績がいい。	
\\	2人[ふたり]の 中[なか]では 彼女[かのじょ]の 方[ほう]がはるかに 成績[せいせき]がいい。	
\\	戦局はわが方に有利である。	
\\	戦局[せんきょく]はわが 方[ほう]に 有利[ゆうり]である。	戦局=せんきょく= 
\\	有利=ゆうり= (利益のある), 
\\	何か資格を持っていると就職のとき有利になる。	
\\	何[なに]か 資格[しかく]を 持[も]っていると 就職[しゅうしょく]のとき 有利[ゆうり]になる。	有利=ゆうり= (利益のある), 
\\	彼は今や一本立ちすべきだと決心した。	
\\	彼[かれ]は 今[いま]や 一本立[いっぽんだ]ちすべきだと 決心[けっしん]した。	一本立ち=いっぽんだち= 
\\	一本立ちする前、彼は有名な料亭で10年近く修業していた。	
\\	一本立[いっぽんだ]ちする 前[まえ]、 彼[かれ]は 有名[ゆうめい]な 料亭[りょうてい]で10 年[ねん] 近[ちか]く 修業[しゅうぎょう]していた。	一本立ち=いっぽんだち= 
\\	料亭=りょうてい= 
\\	仕事も決まって、やっと親から離れて一本立ちした。	
\\	仕事[しごと]も 決[き]まって、やっと 親[おや]から 離[はな]れて 一本立[いっぽんだ]ちした。	一本立ち=いっぽんだち= 
\\	景気はゆるやかな回復に向かっている。	
\\	景気[けいき]はゆるやかな 回復[かいふく]に 向[む]かっている。	緩やかな=ゆるやかな=1)『緩慢な』
\\	『なだらかな』
\\	『寛大な』
\\	『きつくない』
\\	私は思い切ってその話を持ち出した。	
\\	私[わたし]は 思い切[おもいき]ってその 話[はなし]を 持ち出[もちだ]した。	思い切って= 
\\	さんざん迷った末、彼女に思い切って別れ話を持ち出した。	
\\	さんざん 迷[まよ]った 末[すえ]、 彼女[かのじょ]に 思い切[おもいき]って 別れ話[わかればなし]を 持ち出[もちだ]した。	さんざん= 
\\	思い切って= 
\\	私は思い切って一人暮らしを始めることにした。	
\\	私[わたし]は 思い切[おもいき]って 一人暮[ひとりぐ]らしを 始[はじ]めることにした。	思い切って= 
\\	学校にいる間はさんざん試験に苦しめられた。	
\\	学校[がっこう]にいる 間[あいだ]はさんざん 試験[しけん]に 苦[くる]しめられた。	散々=さんざん= (激しく) 
\\	雨がざんざん降っている。	
\\	雨[あめ]がざんざん 降[ふ]っている。	
\\	彼は公開の席に出ることを好まない。	
\\	彼[かれ]は 公開[こうかい]の 席[せき]に 出[で]ることを 好[この]まない。	
\\	この映画はただいま公開中です。	
\\	この 映画[えいが]はただいま 公開[こうかい] 中[ちゅう]です。	
\\	閣僚は資産を公開しなければならない。	
\\	閣僚[かくりょう]は 資産[しさん]を 公開[こうかい]しなければならない。	
\\	この町に新しく診療所が開設された。	
\\	この 町[まち]に 新[あたら]しく 診療[しんりょう] 所[しょ]が 開設[かいせつ]された。	
\\	電話の開設費用はどれくらいですか。	
\\	電話[でんわ]の 開設[かいせつ] 費用[ひよう]はどれくらいですか。	
\\	今回の円高について昨日の新聞に解説が載った。	
\\	今回[こんかい]の 円[えん] 高[だか]について 昨日[きのう]の 新聞[しんぶん]に 解説[かいせつ]が 載[の]った。	円高=えんだか
\\	彼は父からたぐいまれな絵の才能を受け継いでいる。	
\\	彼[かれ]は 父[ちち]からたぐいまれな 絵[え]の 才能[さいのう]を 受け継[うけつ]いでいる。	たぐいまれ(な)= 
\\	叔父の家には跡継ぎがいない。	
\\	叔父[おじ]の 家[いえ]には 跡継[あとつ]ぎがいない。	跡継ぎ=あとつぎ= 
\\	海綿は水を吸収する。	
\\	海綿[かいめん]は 水[みず]を 吸収[きゅうしゅう]する。	海綿=かいめん・うみわた= 
\\	その大会社は二つの小さな会社を吸収した。	
\\	その 大会[だいがい] 社[しゃ]は 二[ふた]つの 小[ちい]さな 会社[かいしゃ]を 吸収[きゅうしゅう]した。	大会社=だいがいしゃ
\\	リストの中に多くの重複が見つかった。	
\\	リストの 中[なか]に 多[おお]くの 重複[ちょうふく]が 見[み]つかった。	重複=ちょうふく・じゅうふく
\\	伝票は4枚複写になっているのでボールペンで強く書いて下さい。	
\\	伝票[でんぴょう]は 4[よん] 枚[まい] 複写[ふくしゃ]になっているのでボールペンで 強[つよ]く 書[か]いて 下[くだ]さい。	
\\	11時ご出勤とはたいそうなご身分だね。	
\\	時[じ]ご 出勤[しゅっきん]とはたいそうなご 身分[みぶん]だね。	大層=たいそう= 非常に;甚だしい;大量の;大規模;立派な;大げさな
\\	こりゃまたたいそうなご発言だな。	
\\	こりゃまたたいそうなご 発言[はつげん]だな。	大層=たいそう= 非常に;甚だしい;大量の;大規模;立派な;大げさな
\\	たいそうな人出だった。	
\\	たいそうな 人出[ひとで]だった。	大層=たいそう= 非常に;甚だしい;大量の;大規模;立派な;大げさな
\\	たいそう大きくなったね。	
\\	たいそう 大[おお]きくなったね。	大層=たいそう= 非常に;甚だしい;大量の;大規模;立派な;大げさな
\\	ご両親はたいそうお喜びでしょうね。	
\\	ご 両親[りょうしん]はたいそうお 喜[よろこ]びでしょうね。	大層=たいそう= 非常に;甚だしい;大量の;大規模;立派な;大げさな
\\	彼、たいそう悔しがってたよ。	
\\	彼[かれ]、たいそう 悔[くや]しがってたよ。	大層=たいそう= 非常に;甚だしい;大量の;大規模;立派な;大げさな
\\	そうたいそうに考えることはない。	
\\	そうたいそうに 考[かんが]えることはない。	大層=たいそう= 非常に;甚だしい;大量の;大規模;立派な;大げさな
\\	そんなことを言われるとなんだかくすぐったいね。	
\\	そんなことを 言[い]われるとなんだかくすぐったいね。	くすぐったい= 
\\	くすぐったいからやめて。	
\\	くすぐったいからやめて。	くすぐったい= 
\\	行方不明だった父親が帰ってきて彼女に笑顔がよみがえった。	
\\	行方[ゆくえ] 不明[ふめい]だった 父親[ちちおや]が 帰[かえ]ってきて 彼女[かのじょ]に 笑顔[えがお]がよみがえった。	蘇る・甦る=よみがえる= 
\\	戦争が終わり、人々の暮らしに平和がよみがえった。	
\\	戦争[せんそう]が 終[お]わり、 人々[ひとびと]の 暮[く]らしに 平和[へいわ]がよみがえった。	蘇る・甦る=よみがえる= 
\\	にらめっこしましょ。笑うと負けよ。	
\\	にらめっこしましょ。 笑[わら]うと 負[ま]けよ。	にらめっこ= 
\\	会社では一日中パソコンとにらめっこしています。	
\\	会社[かいしゃ]では 一日[いちにち] 中[ちゅう]パソコンとにらめっこしています。	にらめっこ= 
\\	世相が悪い。	
\\	世相[せそう]が 悪[わる]い。	世相=せそう= 
\\	この事件の中に現代の世相がよく現れている。	
\\	この 事件[じけん]の 中[なか]に 現代[げんだい]の 世相[せそう]がよく 現[あらわ]れている。	世相=せそう= 
\\	報復などまったく私の念頭にない。	
\\	報復[ほうふく]などまったく 私[わたし]の 念頭[ねんとう]にない。	報復=ほうふく= 
\\	そんなことは念頭に置くな。	
\\	そんなことは 念頭[ねんとう]に 置[お]くな。	
\\	このことをまず念頭に置くべきだ。	
\\	このことをまず 念頭[ねんとう]に 置[お]くべきだ。	
\\	そのことが絶えず念頭から離れない。	
\\	そのことが 絶[た]えず 念頭[ねんとう]から 離[はな]れない。	
\\	どうやったらキノコの善し悪しが分かるのですか。	
\\	どうやったらキノコの 善し悪[よしあ]しが 分[わ]かるのですか。	善し悪し=よしあし= 良いことと悪いこと; 
\\	学校が近いのも善し悪しだ。	
\\	学校[がっこう]が 近[ちか]いのも 善し悪[よしあ]しだ。	善し悪し=よしあし= 良いことと悪いこと; 
\\	どの山と比べても富士山の美しさは際立っている。	
\\	どの 山[やま]と 比[くら]べても 富士山[ふじさん]の 美[うつく]しさは 際立[きわだ]っている。	際立つ=きわだつ= 
\\	(目に付く) 
\\	和服を着ると彼女の美しさがひときわ際立つ。	
\\	和服[わふく]を 着[き]ると 彼女[かのじょ]の 美[うつく]しさがひときわ 際立[きわだ]つ。	一際=ひときわ= 
\\	際立つ=きわだつ= 
\\	(目に付く) 
\\	この2つには際立った違いはない。	
\\	この 2[ふた]つには 際立[きわだ]った 違[ちが]いはない。	際立つ=きわだつ= 
\\	(目に付く) 
\\	私は田舎が好きです。とりわけ春の田舎が。	
\\	私[わたし]は 田舎[いなか]が 好[す]きです。とりわけ 春[はる]の 田舎[いなか]が。	とりわけ= 
\\	(否定の場合) 
\\	その一言が取り分け印象に残った。	
\\	その 一言[ひとこと]が 取り分[とりわ]け 印象[いんしょう]に 残[のこ]った。	とりわけ= 
\\	(否定の場合) 
\\	それはつい見落とした。	
\\	それはつい 見落[みお]とした。	
\\	彼は長年苦労して今日の地位を築き上げた。	
\\	彼[かれ]は 長年[ながねん] 苦労[くろう]して 今日[きょう]の 地位[ちい]を 築き上[きずきあ]げた。	
\\	彼は一代財産を築き上げた。	
\\	彼[かれ]は 一代[いちだい] 財産[ざいさん]を 築き上[きずきあ]げた。	一代=いちだい= 
\\	前日の徹夜が災いして心臓発作を起こした。	
\\	前日[ぜんじつ]の 徹夜[てつや]が 災[わざわ]いして 心臓[しんぞう] 発作[ほっさ]を 起[お]こした。	災い=わざわい= 
\\	(災難) 
\\	(~する) 
\\	心臓発作=しんぞうほっさ= 
\\	新任の課長はあの性格が災いして部下に人気がない。	
\\	新任[しんにん]の 課長[かちょう]はあの 性格[せいかく]が 災[わざわ]いして 部下[ぶか]に 人気[にんき]がない。	災い=わざわい= 
\\	(災難) 
\\	(~する) 
\\	一家に次々と災いが降り掛かった。	
\\	一家[いっか]に 次々[つぎつぎ]と 災[わざわ]いが 降り掛[ふりか]かった。	災い=わざわい= 
\\	(災難) 
\\	(~する) 
\\	彼は美貌が災いとなった。	
\\	彼[かれ]は 美貌[びぼう]が 災[わざわ]いとなった。	美貌=びぼう= 美しい顔かたち 災い=わざわい= 
\\	(災難) 
\\	(~する) 
\\	相手が未成年であることを承知の上で酒を飲ませるのは違法だ。	
\\	相手[あいて]が 未成年[みせいねん]であることを 承知[しょうち]の 上[うえ]で 酒[さけ]を 飲[の]ませるのは 違法[いほう]だ。	承知=しょうち= (知っていること) 
\\	(承諾) 
\\	(承認) 
\\	(容認) 
\\	(容赦)
\\	以上のリスクをご承知の上、ご利用ください。	
\\	以上[いじょう]のリスクをご 承知[しょうち]の 上[うえ]、ご 利用[りよう]ください。	承知=しょうち= (知っていること) 
\\	(承諾) 
\\	(承認) 
\\	(容認) 
\\	(容赦)
\\	両親の家で同居することを妻は快く承知してくれた。	
\\	両親[りょうしん]の 家[いえ]で 同居[どうきょ]することを 妻[つま]は 快[こころよ]く 承知[しょうち]してくれた。	快く=こころよく= 
\\	承知=しょうち= (知っていること) 
\\	(承諾) 
\\	(承認) 
\\	(容認) 
\\	(容赦)
\\	妻が外で働くのを承知しない男性はまだまだ多い。	
\\	妻[つま]が 外[そと]で 働[はたら]くのを 承知[しょうち]しない 男性[だんせい]はまだまだ 多[おお]い。	承知=しょうち= (知っていること) 
\\	(承諾) 
\\	(承認) 
\\	(容認) 
\\	(容赦)
\\	言う通りにしないと承知しないぞ。	
\\	言[い]う 通[とお]りにしないと 承知[しょうち]しないぞ。	承知=しょうち= (知っていること) 
\\	(承諾) 
\\	(承認) 
\\	(容認) 
\\	(容赦)
\\	雇い主が使用人を搾取した。	
\\	雇い主[やといぬし]が 使用人[しようにん]を 搾取[さくしゅ]した。	使用人=しようにん= 
\\	搾取=さくしゅ= 
\\	(人を) 
\\	何か超自然的な力が作用しているのかもしれない。	
\\	何[なに]か 超[ちょう] 自然[しぜん] 的[てき]な 力[ちから]が 作用[さよう]しているのかもしれない。	作用=さよう= (働き) 
\\	(機能) 
\\	(影響) 
\\	アルコールと共に服用すると、薬の作用に影響を及ぼすことがある。	
\\	アルコールと 共[とも]に 服用[ふくよう]すると、 薬[くすり]の 作用[さよう]に 影響[えいきょう]を 及[およ]ぼすことがある。	作用=さよう= (働き) 
\\	(機能) 
\\	(影響) 
\\	その分だと仕事は予定よりずっと早く終わるだろう。	
\\	その 分[ぶん]だと 仕事[しごと]は 予定[よてい]よりずっと 早[はや]く 終[お]わるだろう。	
\\	承ればお父様がお亡くなりになったそうで、心中お察し申し上げます。	
\\	承[うけたまわ]ればお 父様[とうさま]がお 亡[な]くなりになったそうで、 心中[しんじゅう]お 察[さっ]し 申し上[もうしあ]げます。	
\\	その計画の詳細を承りたい。	
\\	その 計画[けいかく]の 詳細[しょうさい]を 承[うけたまわ]りたい。	
\\	お所とお名前を承っておきましょう。	
\\	お 所[ところ]とお 名前[なまえ]を 承[うけたまわ]っておきましょう。	
\\	彼の承諾がなければこの書類を見せるわけにはいかない。	
\\	彼[かれ]の 承諾[しょうだく]がなければこの 書類[しょるい]を 見[み]せるわけにはいかない。	
\\	未成年者の参加には親もしくは保護者の承諾が必要だ。	
\\	未成年[みせいねん] 者[しゃ]の 参加[さんか]には 親[おや]もしくは 保護[ほご] 者[しゃ]の 承諾[しょうだく]が 必要[ひつよう]だ。	
\\	この村には変わった踊りが伝承されている。	
\\	この 村[むら]には 変[か]わった 踊[おど]りが 伝承[でんしょう]されている。	伝承=でんしょう= (次の世代に伝えること) 
\\	(伝説) 
\\	こちらから電話しようとしていた矢先に電話が鳴ったのです。	
\\	こちらから 電話[でんわ]しようとしていた 矢先[やさき]に 電話[でんわ]が 鳴[な]ったのです。	・・・しようとした矢先に= 
\\	俺もそろそろ独立しなくちゃなと思っていた矢先に親父が入院した。	
\\	俺[おれ]もそろそろ 独立[どくりつ]しなくちゃなと 思[おも]っていた 矢先[やさき]に 親父[おやじ]が 入院[にゅういん]した。	・・・しようとした矢先に= 
\\	この矢印の方向にお進み下さい。	
\\	この 矢印[やじるし]の 方向[ほうこう]にお 進[すす]み 下[くだ]さい。	
\\	彼を新聞記者とにらんだが君はどう思う?	
\\	彼[かれ]を 新聞[しんぶん] 記者[きしゃ]とにらんだが 君[きみ]はどう 思[おも]う?	にらむ= (目を怒らして見つめる) 
\\	そんなことをしたら先生ににらまれるぞ。	
\\	そんなことをしたら 先生[せんせい]ににらまれるぞ。	にらむ= (目を怒らして見つめる) 
\\	それ以来私は社長ににらまれている。	
\\	それ 以来[いらい] 私[わたし]は 社長[しゃちょう]ににらまれている。	にらむ= (目を怒らして見つめる) 
\\	問題をいくらにらんでも解けなかった。	
\\	問題[もんだい]をいくらにらんでも 解[と]けなかった。	にらむ= (目を怒らして見つめる) 
\\	先生ににらまれて、いたずらっ子は黙ってしまった。	
\\	先生[せんせい]ににらまれて、いたずらっ 子[こ]は 黙[だま]ってしまった。	にらむ= (目を怒らして見つめる) 
\\	彼は怒った目で私をにらんだ。	
\\	彼[かれ]は 怒[おこ]った 目[め]で 私[わたし]をにらんだ。	にらむ= (目を怒らして見つめる) 
\\	ジョンさんとトムさんの間に軋轢がある。	
\\	ジョンさんとトムさんの 間[あいだ]に 軋轢[あつれき]がある。	軋轢=あつれき= 
\\	党員の間に絶えず軋轢がある。	
\\	党員[とういん]の 間[あいだ]に 絶[た]えず 軋轢[あつれき]がある。	軋轢=あつれき= 
\\	両者の軋轢が激しくなっている。	
\\	両者[りょうしゃ]の 軋轢[あつれき]が 激[はげ]しくなっている。	軋轢=あつれき= 
\\	彼らの間に軋轢が生じた。	
\\	彼[かれ]らの 間[あいだ]に 軋轢[あつれき]が 生[しょう]じた。	軋轢=あつれき= 
\\	配送料は全国一律料金です。	
\\	配送[はいそう] 料[りょう]は 全国[ぜんこく] 一律[いちりつ] 料金[りょうきん]です。	一律=いちりつ= (均等) 
\\	(無差別)
\\	政府は公共投資額の一律カットを決めた。	
\\	政府[せいふ]は 公共[こうきょう] 投資[とうし] 額[がく]の 一律[いちりつ]カットを 決[き]めた。	一律=いちりつ= (均等) 
\\	(無差別)
\\	レンズにごみが付着した。	
\\	レンズにごみが 付着[ふちゃく]した。	
\\	集合したら直ちに出発する。	
\\	集合[しゅうごう]したら 直[ただ]ちに 出発[しゅっぱつ]する。	直ちに=ただちに= 
\\	直ちにご返事いたします。	
\\	直[ただ]ちにご 返事[へんじ]いたします。	直ちに=ただちに= 
\\	機動隊が直ちに出動した。	
\\	機動[きどう] 隊[たい]が 直[ただ]ちに 出動[しゅつどう]した。	機動隊=きどうたい= 
\\	出動=しゅつどう= 
\\	帰京後直ちに彼の家へ駆けつけた。	
\\	帰京[ききょう] 後[ご] 直[ただ]ちに 彼[かれ]の 家[いえ]へ 駆[か]けつけた。	直ちに=ただちに= 
\\	その病気は直ちに死につながるというものではない。	
\\	その 病気[びょうき]は 直[ただ]ちに 死[し]につながるというものではない。	直ちに=ただちに= 
\\	デモ隊は暴走と化し、機動隊が出動する事態となった。	
\\	デモ 隊[たい]は 暴走[ぼうそう]と 化[か]し、 機動[きどう] 隊[たい]が 出動[しゅつどう]する 事態[じたい]となった。	出動= 
\\	方向性がはっきりしない。	
\\	方向[ほうこう] 性[せい]がはっきりしない。	
\\	新思想が人々の心に浸透し始めた。	
\\	新[しん] 思想[しそう]が 人々[ひとびと]の 心[こころ]に 浸透[しんとう]し 始[はじ]めた。	浸透=しんとう= 
\\	水が徐々に土壌の中に浸透してゆく。	
\\	水[みず]が 徐々[じょじょ]に 土壌[どじょう]の 中[なか]に 浸透[しんとう]してゆく。	浸透=しんとう= 
\\	彼の名は冒険の代名詞となった。	
\\	彼[かれ]の 名[な]は 冒険[ぼうけん]の 代名詞[だいめいし]となった。	代名詞=だいめいし= 
\\	彼は柄はでかいが、まだ小学生だ。	
\\	彼[かれ]は 柄[がら]はでかいが、まだ 小学生[しょうがくせい]だ。	柄=がら= (模様) 
\\	(体格) 
\\	(品格) 
\\	(その人の性質や身分・分際)
\\	柄でもないことはしない方がいいよ。	
\\	柄[がら]でもないことはしない 方[ほう]がいいよ。	柄=がら= (模様) 
\\	(体格) 
\\	(品格) 
\\	(その人の性質や身分・分際)
\\	そんな仕事は私の柄でない。	
\\	そんな 仕事[しごと]は 私[わたし]の 柄[がら]でない。	柄=がら= (模様) 
\\	(体格) 
\\	(品格) 
\\	(その人の性質や身分・分際)
\\	あれは会長という柄じゃないね。	
\\	あれは 会長[かいちょう]という 柄[がら]じゃないね。	柄=がら= (模様) 
\\	(体格) 
\\	(品格) 
\\	(その人の性質や身分・分際)
\\	一部では性の話はまだ汚らわしいものと思われている。	
\\	一部[いちぶ]では 性[せい]の 話[はなし]はまだ 汚[けが]らわしいものと 思[おも]われている。	汚らわしい=けがらわしい= (汚い) 
\\	そんな汚らわしい金は受け取れない。	
\\	そんな 汚[けが]らわしい 金[きん]は 受け取[うけと]れない。	汚らわしい=けがらわしい= (汚い) 
\\	見るも汚らわしい。	
\\	見[み]るも 汚[けが]らわしい。	汚らわしい=けがらわしい= (汚い) 
\\	そのスキャンダルは彼の経歴に汚点を残した。	
\\	そのスキャンダルは 彼[かれ]の 経歴[けいれき]に 汚点[おてん]を 残[のこ]した。	
\\	彼は父親の汚名をそそぐためにあらゆる努力をした。	
\\	彼[かれ]は 父親[ちちおや]の 汚名[おめい]をそそぐためにあらゆる 努力[どりょく]をした。	
\\	そんな話を聞いたら耳が汚れる。	
\\	そんな 話[はなし]を 聞[き]いたら 耳[みみ]が 汚[よご]れる。	
\\	お前の不行跡でわが家の家名は汚れた。	
\\	お 前[まえ]の 不行跡[ふぎょうせき]でわが 家[や]の 家名[かめい]は 汚[けが]れた。	不行跡=ふぎょうせき= 
\\	彼女のあの言葉は私の心に深く刻み込まれている。	
\\	彼女[かのじょ]のあの 言葉[ことば]は 私[わたし]の 心[こころ]に 深[ふか]く 刻み込[きざみこ]まれている。	
\\	刻々と別れの時が迫っていた。	
\\	刻々[こっこく]と 別[わか]れの 時[とき]が 迫[せま]っていた。	
\\	彼は一刻なところがある。	
\\	彼[かれ]は 一刻[いっこく]なところがある。	一刻=いっこく= 
\\	(頑固な) 
\\	一刻も早く解決しなければならない。	
\\	一刻[いっこく]も 早[はや]く 解決[かいけつ]しなければならない。	一刻早く=いっこくはやく= 
\\	この病気は一刻も早く治療しなければならない。	
\\	この 病気[びょうき]は 一刻[いっこく]も 早[はや]く 治療[ちりょう]しなければならない。	一刻も早く= 
\\	クラスの全員に招集がかかった。	
\\	クラスの 全員[ぜんいん]に 招集[しょうしゅう]がかかった。	
\\	谷は深いしじまに包まれていた。	
\\	谷[たに]は 深[ふか]いしじまに 包[つつ]まれていた。	黙・静寂=しじま= 
\\	あたりはまったく静寂であった。	
\\	あたりはまったく 静寂[せいじゃく]であった。	静寂=せいじゃく= 
\\	貴重で脆弱な自然環境をみんなで守らなければならない。	
\\	貴重[きちょう]で 脆弱[ぜいじゃく]な 自然[しぜん] 環境[かんきょう]をみんなで 守[まも]らなければならない。	脆弱=ぜいじゃく= (~な) 
\\	問題の一端をうかがうにはこれで十分だ。	
\\	問題[もんだい]の 一端[いったん]をうかがうにはこれで 十分[じゅうぶん]だ。	一端=いったん= (片端) 
\\	(一部) 
\\	彼もこの仕事の一端を担っている。	
\\	彼[かれ]もこの 仕事[しごと]の 一端[いったん]を 担[にな]っている。	一端=いったん= (片端) 
\\	(一部) 
\\	一端を担う= 
\\	責任の一端は君にもある。	
\\	責任[せきにん]の 一端[いったん]は 君[きみ]にもある。	一端=いったん= (片端) 
\\	(一部) 
\\	いったん家に帰って荷物を置いてきます。	
\\	いったん 家[いえ]に 帰[かえ]って 荷物[にもつ]を 置[お]いてきます。	一旦=いったん= 
\\	男は現場からいったん逃走したが、しばらくして自首してきた。	
\\	男[おとこ]は 現場[げんば]からいったん 逃走[とうそう]したが、しばらくして 自首[じしゅ]してきた。	一旦=いったん= 
\\	願書に履歴書を添えて提出すること。	
\\	願書[がんしょ]に 履歴[りれき] 書[しょ]を 添[そ]えて 提出[ていしゅつ]すること。	添える・副える=そえる= (付け足す) 
\\	(必要な要素を合わせて一括する) 
\\	彼女はパスタにバジルを添えた。	
\\	彼女[かのじょ]はパスタにバジルを 添[そ]えた。	添える・副える=そえる= (付け足す) 
\\	(必要な要素を合わせて一括する) 
\\	彼に対する民衆の好奇心はやがて感嘆の念へと変わっていった。	
\\	彼[かれ]に 対[たい]する 民衆[みんしゅう]の 好奇[こうき] 心[しん]はやがて 感嘆[かんたん]の 念[ねん]へと 変[か]わっていった。	
\\	彼の演奏を聴いて感嘆した。	
\\	彼[かれ]の 演奏[えんそう]を 聴[き]いて 感嘆[かんたん]した。	
\\	教えこの何人かは今でもまめに便りをくれる。	
\\	教[おし]えこの 何[なん] 人[にん]かは 今[いま]でもまめに 便[たよ]りをくれる。	
\\	経営側との交渉は一向らちがあかない。	
\\	経営[けいえい] 側[がわ]との 交渉[こうしょう]は 一向[いっこう]らちがあかない。	らちがあく= (決着する) 
\\	彼の証言は事実と合致する。	
\\	彼[かれ]の 証言[しょうげん]は 事実[じじつ]と 合致[がっち]する。	
\\	兄が大学に合格したのに触発されて、私も真剣に勉強に取り組み始めた。	
\\	兄[あに]が 大学[だいがく]に 合格[ごうかく]したのに 触発[しょくはつ]されて、 私[わたし]も 真剣[しんけん]に 勉強[べんきょう]に 取り組[とりく]み 始[はじ]めた。	真剣=しんけん= (本物の剣) 
\\	(〜な) (まじめな) 
\\	手に触れるな。	
\\	手[て]に 触[ふ]れるな。	
\\	陳列品に手を触れるな。	
\\	陳列[ちんれつ] 品[ひん]に 手[て]を 触[ふ]れるな。	陳列品=ちんれつひん= 
\\	私は傘の先でちょっと触れてみた。	
\\	私[わたし]は 傘[かさ]の 先[さき]でちょっと 触[ふ]れてみた。	
\\	触れると爆発するぞ!	
\\	触[ふ]れると 爆発[ばくはつ]するぞ!	
\\	彼はこの問題にはちょっと触れただけであった。	
\\	彼[かれ]はこの 問題[もんだい]にはちょっと 触[ふ]れただけであった。	
\\	誰もが触れられたくない傷を心に持っている。	
\\	誰[だれ]もが 触[ふ]れられたくない 傷[きず]を 心[こころ]に 持[も]っている。	
\\	足に何かが触るのを感じた。	
\\	足[あし]に 何[なに]かが 触[さわ]るのを 感[かん]じた。	
\\	その子の額に触ると熱かった。	
\\	その 子[こ]の 額[ひたい]に 触[さわ]ると 熱[あつ]かった。	
\\	火は強風に煽られて四方に広がった。	
\\	火[ひ]は 強風[きょうふう]に 煽[あふ]られて 四方[しほう]に 広[ひろ]がった。	煽る=あおる= (風が) 
\\	旗が風に激しく煽られていた。	
\\	旗[はた]が 風[かぜ]に 激[はげ]しく 煽[あふ]られていた。	煽る=あおる= (風が) 
\\	政治家が国民の危機感をあおる発言を繰り返す。	
\\	政治[せいじ] 家[か]が 国民[こくみん]の 危機[きき] 感[かん]をあおる 発言[はつげん]を 繰り返[くりかえ]す。	煽る=あおる= (風が) 
\\	麻シャツのサラサラした感触が肌に快い。	
\\	麻[あさ]シャツのサラサラした 感触[かんしょく]が 肌[はだ]に 快[こころよ]い。	麻=あさ= 
\\	感触= 
\\	この手袋は合成皮革だが、本物そっくりの感触だ。	
\\	この 手袋[てぶくろ]は 合成[ごうせい] 皮革[ひかく]だが、 本物[ほんもの]そっくりの 感触[かんしょく]だ。	皮革=ひかく= 
\\	感触= 
\\	手と手が触れ合った。	
\\	手[て]と 手[て]が 触れ合[ふれあ]った。	
\\	このチームの顔触れでは優勝は難しいだろう。	
\\	このチームの 顔触[かおぶ]れでは 優勝[ゆうしょう]は 難[むずか]しいだろう。	顔触れ=かおぶれ= 
\\	委員の顔触れはだいぶ新しくなった。	
\\	委員[いいん]の 顔触[かおぶ]れはだいぶ 新[あたら]しくなった。	顔触れ=かおぶれ= 
\\	ここは由緒ある町です。	
\\	ここは 由緒[ゆいしょ]ある 町[まち]です。	由緒=ゆいしょ= 
\\	テントの柱が倒れた。	
\\	テントの 柱[はしら]が 倒[たお]れた。	
\\	彼女は3年間チームの柱として頑張った。	
\\	彼女[かのじょ]は3 年間[ねんかん]チームの 柱[はしら]として 頑張[がんば]った。	
\\	彼の主張は三つの柱からなっている。	
\\	彼[かれ]の 主張[しゅちょう]は 三[みっ]つの 柱[はしら]からなっている。	
\\	彼女の早すぎる死をみなで嘆いた。	
\\	彼女[かのじょ]の 早[はや]すぎる 死[し]をみなで 嘆[なげ]いた。	
\\	波がどうどうと岩を打った。	
\\	波[なみ]がどうどうと 岩[いわ]を 打[う]った。	
\\	青年は人生いかに生きるべきか、大いに思い悩んだ。	
\\	青年[せいねん]は 人生[じんせい]いかに 生[い]きるべきか、 大[おお]いに 思い悩[おもいなや]んだ。	
\\	彼女は人生の岐路に立って、どちらの道を選ぶべきか思い悩んでいた。	
\\	彼女[かのじょ]は 人生[じんせい]の 岐路[きろ]に 立[た]って、どちらの 道[みち]を 選[えら]ぶべきか 思い悩[おもいなや]んでいた。	岐路=きろ= 
\\	現代人には過去にない様々な疾病が見られる。	
\\	現代[げんだい] 人[じん]には 過去[かこ]にない 様々[さまざま]な 疾病[しっぺい]が 見[み]られる。	疾病=しっぺい= (病気) 
\\	株価の横ばいが続いている。	
\\	株価[かぶか]の 横[よこ]ばいが 続[つづ]いている。	横ばい= 
\\	ジャガイモは北海道の主な産物の一つである。	
\\	ジャガイモは 北海道[ほっかいどう]の 主[おも]な 産物[さんぶつ]の 一[ひと]つである。	
\\	その作家の作品は短編が主だ。	
\\	その 作家[さっか]の 作品[さくひん]は 短編[たんぺん]が 主[おも]だ。	
\\	彼としゃべるときは英語が主だ。	
\\	彼[かれ]としゃべるときは 英語[えいご]が 主[おも]だ。	
\\	仕事は営業が主です。	
\\	仕事[しごと]は 営業[えいぎょう]が 主[おも]です。	
\\	ほんのわずかの差で2位だった。	
\\	ほんのわずかの 差[さ]で 2位[にい]だった。	
\\	彼は車を売ってわずかの金を得た。	
\\	彼[かれ]は 車[くるま]を 売[う]ってわずかの 金[きん]を 得[え]た。	
\\	応募者はわずか4人にとどまった。	
\\	応募[おうぼ] 者[しゃ]はわずか4 人[にん]にとどまった。	
\\	彼らの実力の差はわずかなものだ。	
\\	彼[かれ]らの 実力[じつりょく]の 差[さ]はわずかなものだ。	
\\	黒いドレスを着た彼女は悩ましいほど美しかった。	
\\	黒[くろ]いドレスを 着[き]た 彼女[かのじょ]は 悩[なや]ましいほど 美[うつく]しかった。	悩ましい=なやましい= 
\\	その会社は売り上げが伸び悩んでいる。	
\\	その 会社[かいしゃ]は 売り上[うりあ]げが 伸び悩[のびなや]んでいる。	
\\	私は彼のストーカー行為に悩まされている。	
\\	私[わたし]は 彼[かれ]のストーカー 行為[こうい]に 悩[なや]まされている。	
\\	私はよく腹痛に悩まされる。	
\\	私[わたし]はよく 腹痛[はらいた]に 悩[なや]まされる。	
\\	私の言うことが間違っていたら訂正して下さい。	
\\	私[わたし]の 言[い]うことが 間違[まちが]っていたら 訂正[ていせい]して 下[くだ]さい。	訂正=ていせい= 
\\	非常によく書けていて訂正の余地がない。	
\\	非常[ひじょう]によく 書[か]けていて 訂正[ていせい]の 余地[よち]がない。	訂正=ていせい= 
\\	戦時中、米は統制品だった。	
\\	戦時[せんじ] 中[ちゅう]、 米[こめ]は 統制[とうせい] 品[ひん]だった。	
\\	この業界は慢性の人手不足に悩んでいる。	
\\	この 業界[ぎょうかい]は 慢性[まんせい]の 人手[ひとで] 不足[ぶそく]に 悩[なや]んでいる。	人手不足=ひとでぶそく= 
\\	ペンキがまだぬれているからそこに座らないでください。	
\\	ペンキがまだぬれているからそこに 座[すわ]らないでください。	濡れる=ぬれる= 
\\	夕立にぐっしょり濡れながら歩いた。	
\\	夕立[ゆうだち]にぐっしょり 濡[ぬ]れながら 歩[ある]いた。	夕立=ゆうだち= 
\\	濡れる=ぬれる= 
\\	彼女の頬は涙でぬれていた。	
\\	彼女[かのじょ]の 頬[ほほ]は 涙[なみだ]でぬれていた。	頬=ほほ・ほお濡れる=ぬれる= 
\\	この子はおむつがぬれている。	
\\	この 子[こ]はおむつがぬれている。	おむつ= 
\\	濡れる=ぬれる= 
\\	彼はぬれた服のまま立っていた。	
\\	彼[かれ]はぬれた 服[ふく]のまま 立[た]っていた。	濡れる=ぬれる= 
\\	猫は体がぬれるのが嫌いだ。	
\\	猫[ねこ]は 体[からだ]がぬれるのが 嫌[きら]いだ。	濡れる=ぬれる= 
\\	うれしさは包み切れずに顔に出た。	
\\	うれしさは 包[つつ]み 切[き]れずに 顔[かお]に 出[で]た。	包む=つつむ= (包装) 
\\	(店員に) これ包んでください。	
\\	店員[てんいん]に)これ 包[つつ]んでください。	
\\	(店員に) プレゼント用に包んでください。	
\\	店員[てんいん]に)プレゼント 用[よう]に 包[つつ]んでください。	包む=つつむ= (包装) 
\\	これはずば抜けてよい。	
\\	これはずば 抜[ぬ]けてよい。	ずば抜ける= 
\\	あの生徒はクラスの中でずば抜けてよくできる。	
\\	あの 生徒[せいと]はクラスの 中[なか]でずば 抜[ぬ]けてよくできる。	ずば抜ける= 
\\	あいつはずば抜けて背が高い。	
\\	あいつはずば 抜[ぬ]けて 背[せ]が 高[たか]い。	ずば抜ける= 
\\	今幾分でもやっておけば後で楽だろう。	
\\	今[いま] 幾[いく] 分[ふん]でもやっておけば 後[あと]で 楽[らく]だろう。	幾分=いくぶん= (一部分) 
\\	(ある程度) 
\\	今日は幾分気分が良い。	
\\	今日[きょう]は 幾分[いくぶん] 気分[きぶん]が 良[よ]い。	幾分=いくぶん= (一部分) 
\\	(ある程度) 
\\	彼には幾分詩人らしいところがある。	
\\	彼[かれ]には 幾分[いくぶん] 詩人[しじん]らしいところがある。	幾分=いくぶん= (一部分) 
\\	(ある程度) 
\\	彼の人気は幾分衰え気味である。	
\\	彼[かれ]の 人気[にんき]は 幾分[いくぶん] 衰[おとろ]え 気味[ぎみ]である。	幾分=いくぶん= (一部分) 
\\	(ある程度) 
\\	自分と同じ病気の人がいるということが幾分の慰めであった。	
\\	自分[じぶん]と 同[おな]じ 病気[びょうき]の 人[ひと]がいるということが 幾分[いくぶん]の 慰[なぐさ]めであった。	幾分=いくぶん= (一部分) 
\\	(ある程度) 
\\	慰め=なぐさめ= 
\\	私にも費用の幾分かを負担させてください。	
\\	私[わたし]にも 費用[ひよう]の 幾分[いくぶん]かを 負担[ふたん]させてください。	幾分=いくぶん= (一部分) 
\\	(ある程度) 
\\	音楽は孤独を慰めてくれる。	
\\	音楽[おんがく]は 孤独[こどく]を 慰[なぐさ]めてくれる。	慰める=なぐさめる= 
\\	彼は愛する人を失った悲しみを旅で慰めようとした。	
\\	彼[かれ]は 愛[あい]する 人[ひと]を 失[うしな]った 悲[かな]しみを 旅[たび]で 慰[なぐさ]めようとした。	慰める=なぐさめる= 
\\	同室者の不潔さに耐えられない。	
\\	同室[どうしつ] 者[しゃ]の 不潔[ふけつ]さに 耐[た]えられない。	
\\	そのトイレの不潔なことはお話しにならなかった。	
\\	そのトイレの 不潔[ふけつ]なことはお 話[はな]しにならなかった。	
\\	不潔な手で触るな。	
\\	不潔[ふけつ]な 手[て]で 触[さわ]るな。	
\\	彼も君のチームに交ぜてやれよ。	
\\	彼[かれ]も 君[きみ]のチームに 交[ま]ぜてやれよ。	交ぜる・混ぜる=まぜる= (一体にする) 
\\	少年たちがサッカーの試合に混ぜてくれた。	
\\	少年[しょうねん]たちがサッカーの 試合[しあい]に 混[ま]ぜてくれた。	交ぜる・混ぜる=まぜる= (一体にする) 
\\	聴衆には外国人もかなり交じっていた。	
\\	聴衆[ちょうしゅう]には 外国[がいこく] 人[じん]もかなり 交[ま]じっていた。	交じる・混じる=まじる= (一体になる) 
\\	この報告文には英単語がたくさん交じっている。	
\\	この 報告[ほうこく] 文[ぶん]には 英単語[えいたんご]がたくさん 交[ま]じっている。	交じる・混じる=まじる= (一体になる) 
\\	録音には雑音が混じっていた。	
\\	録音[ろくおん]には 雑音[ざつおん]が 混[ま]じっていた。	交じる・混じる=まじる= (一体になる) 
\\	米に砂が混じり込んだ。	
\\	米[こめ]に 砂[すな]が 混[ま]じり 込[こ]んだ。	交じる・混じる=まじる= (一体になる) 
\\	尿に血が混じっていた。	
\\	尿[にょう]に 血[ち]が 混[ま]じっていた。	交じる・混じる=まじる= (一体になる) 
\\	水と油はよく混じらない。	
\\	水[みず]と 油[あぶら]はよく 混[ま]じらない。	交じる・混じる=まじる= (一体になる) 
\\	財布にはいろいろの国の通貨が交ざっていた。	
\\	財布[さいふ]にはいろいろの 国[くに]の 通貨[つうか]が 交[ま]ざっていた。	交ざる・混ざる=まざる= (一体になる) 
\\	まじる)
\\	くしゃみをしたとき尿が漏れることがある。	
\\	くしゃみをしたとき 尿[にょう]が 漏[も]れることがある。	
\\	尿に血が混じったら検査を受けた方がよい。	
\\	尿[にょう]に 血[ち]が 混[ま]じったら 検査[けんさ]を 受[う]けた 方[ほう]がよい。	
\\	母親に死なれて彼女は途方に暮れてしまった。	
\\	母親[ははおや]に 死[し]なれて 彼女[かのじょ]は 途方[とほう]に 暮[く]れてしまった。	途方に暮れる= 
\\	2人は宿が見つからず途方に暮れていた。	
\\	2人[ふたり]は 宿[やど]が 見[み]つからず 途方[とほう]に 暮[く]れていた。	途方に暮れる= 
\\	途方もないことを言う人だね、君は。	
\\	途方[とほう]もないことを 言[い]う 人[ひと]だね、 君[きみ]は。	
\\	仕事一途の父はこれといった趣味がない。	
\\	仕事[しごと] 一途[いちず]の 父[ちち]はこれといった 趣味[しゅみ]がない。	一途=いちず= 
\\	彼は一途に妻を信じている。	
\\	彼[かれ]は 一途[いちず]に 妻[つま]を 信[しん]じている。	一途=いちず= 
\\	都市環境は悪化の一途である。	
\\	都市[とし] 環境[かんきょう]は 悪化[あっか]の 一途[いっと]である。	一途=いっと= 
\\	帰宅途上で何者かに拉致された。	
\\	帰宅[きたく] 途上[とじょう]で 何者[なにもの]かに 拉致[らち]された。	
\\	公的年金の前途が危うくなっている。	
\\	公的[こうてき] 年金[ねんきん]の 前途[ぜんと]が 危[あや]うくなっている。	前途=ぜんと= (将来) 
\\	その会社は前途が思わしくない。	
\\	その 会社[かいしゃ]は 前途[ぜんと]が 思[おも]わしくない。	前途=ぜんと= (将来) 
\\	前途にあまり希望が持てない。	
\\	前途[ぜんと]にあまり 希望[きぼう]が 持[も]てない。	前途=ぜんと= (将来) 
\\	私たちの前途は多難だ。	
\\	私[わたし]たちの 前途[ぜんと]は 多難[たなん]だ。	前途=ぜんと= (将来) 
\\	近所の人に気兼ねしてピアノの練習ができない。	
\\	近所[きんじょ]の 人[ひと]に 気兼[きが]ねしてピアノの 練習[れんしゅう]ができない。	
\\	気兼ねをせずにいつまででもいらっしゃい。	
\\	気兼[きが]ねをせずにいつまででもいらっしゃい。	
\\	彼は女子高校を兼務している。	
\\	彼[かれ]は 女子[じょし] 高校[こうこう]を 兼務[けんむ]している。	
\\	この部屋は居間と食堂の兼用です。	
\\	この 部屋[へや]は 居間[いま]と 食堂[しょくどう]の 兼用[けんよう]です。	
\\	この旅行は仕事を兼ねている。	
\\	この 旅行[りょこう]は 仕事[しごと]を 兼[か]ねている。	
\\	ぴちゃぴちゃと音を立ててジュースを飲んではいけません。	
\\	ぴちゃぴちゃと 音[おと]を 立[た]ててジュースを 飲[の]んではいけません。	ぴちゃぴちゃと= 
\\	故あって今は妻子と離れて暮らしている。	
\\	故[ゆえ]あって 今[いま]は 妻子[さいし]と 離[はな]れて 暮[く]らしている。	
\\	老齢の故をもって会長を辞任した。	
\\	老齢[ろうれい]の 故[ゆえ]をもって 会長[かいちょう]を 辞任[じにん]した。	故をもって= 
\\	私と彼女は同時に大学に入った。	
\\	私[わたし]と 彼女[かのじょ]は 同時[どうじ]に 大学[だいがく]に 入[はい]った。	
\\	彼の長所は認めるが同時に短所にも目をつぶるわけにいかない。	
\\	彼[かれ]の 長所[ちょうしょ]は 認[みと]めるが 同時[どうじ]に 短所[たんしょ]にも 目[め]をつぶるわけにいかない。	
\\	あいつは面の皮が厚い。	
\\	あいつは 面[つら]の 皮[かわ]が 厚[あつ]い。	面の皮が厚い= 
\\	いい夫にはとうていなれそうもない。	
\\	いい 夫[おっと]にはとうていなれそうもない。	到底=とうてい= 
\\	(まったく) 
\\	解決はとうてい不可能だ。	
\\	解決[かいけつ]はとうてい 不可能[ふかのう]だ。	到底=とうてい= 
\\	(まったく) 
\\	彼にはとうていできない仕事だ。	
\\	彼[かれ]にはとうていできない 仕事[しごと]だ。	到底=とうてい= 
\\	(まったく) 
\\	法案が成立した。	
\\	法案[ほうあん]が 成立[せいりつ]した。	成立= 
\\	(組織されること) 
\\	源氏物語は1000年ごろに成立した。	
\\	源氏物語[げんじものがたり]は1000 年[ねん]ごろに 成立[せいりつ]した。	成立= 
\\	(組織されること) 
\\	二人の間に婚約が成立した。	
\\	二人[ふたり]の 間[あいだ]に 婚約[こんやく]が 成立[せいりつ]した。	成立= 
\\	(組織されること) 
\\	中東情勢が新しい展開を見せている。	
\\	中東[ちゅうとう] 情勢[じょうせい]が 新[あたら]しい 展開[てんかい]を 見[み]せている。	情勢=じょうせい= 
\\	戦後の経済発展は日本の農村を大きく変貌させた。	
\\	戦後[せんご]の 経済[けいざい] 発展[はってん]は日本の 農村[のうそん]を 大[おお]きく 変貌[へんぼう]させた。	変貌=へんぼう= 
\\	町の変貌が著しかった。	
\\	町[まち]の 変貌[へんぼう]が 著[いちじる]しかった。	変貌=へんぼう= 
\\	私は彼の記録にははるかに及ばなかった。	
\\	私[わたし]は 彼[かれ]の 記録[きろく]にははるかに 及[およ]ばなかった。	
\\	疲労回復には風呂に及ぶものはない。	
\\	疲労[ひろう] 回復[かいふく]には 風呂[ふろ]に 及[およ]ぶものはない。	
\\	世界記録にはわずかに及ばなかった。	
\\	世界[せかい] 記録[きろく]にはわずかに 及[およ]ばなかった。	
\\	油と水は混合しない。	
\\	油[あぶら]と 水[みず]は 混合[こんごう]しない。	
\\	日本社会には新旧両方の文化が混在している。	
\\	日本[にほん] 社会[しゃかい]には 新旧[しんきゅう] 両方[りょうほう]の 文化[ぶんか]が 混在[こんざい]している。	
\\	その国の政界は混沌としている。	
\\	その 国[くに]の 政界[せいかい]は 混沌[こんとん]としている。	
\\	そのコーヒーにはシアン化合物が混入されていた。	
\\	そのコーヒーにはシアン 化合[かごう] 物[ぶつ]が 混入[こんにゅう]されていた。	シアン化合物= 
\\	{化合物= 
\\	このミルクには毒物が混入している。	
\\	このミルクには 毒物[どくぶつ]が 混入[こんにゅう]している。	毒物=どくぶつ
\\	現場写真のあまりの痛ましさに言葉を失った。	
\\	現場[げんば] 写真[しゃしん]のあまりの 痛[いた]ましさに 言葉[ことば]を 失[うしな]った。	
\\	日本は当時敗戦の痛手から懸命に立ち直ろうとしていた。	
\\	日本[にほん]は 当時[とうじ] 敗戦[はいせん]の 痛手[いたで]から 懸命[けんめい]に 立ち直[たちなお]ろうとしていた。	立ち直る= 
\\	長い間の鬱病からようやく立ち直った。	
\\	長[なが]い 間[あいだ]の 鬱病[うつびょう]からようやく 立ち直[たちなお]った。	
\\	夫を失った痛手から容易に立ち直れなかった。	
\\	夫[おっと]を 失[うしな]った 痛手[いたで]から 容易[ようい]に 立ち直[たちなお]れなかった。	
\\	景気はまだまだ立ち直りそうにない。	
\\	景気[けいき]はまだまだ 立ち直[たちなお]りそうにない。	
\\	秀才たちに囲まれて実力のなさを痛感させられた。	
\\	秀才[しゅうさい]たちに 囲[かこ]まれて 実力[じつりょく]のなさを 痛感[つうかん]させられた。	
\\	どの歯が痛むのですか。	
\\	どの 歯[は]が 痛[いた]むのですか。	
\\	ここを押すと痛みますか。	
\\	ここを 押[お]すと 痛[いた]みますか。	
\\	胸がきりきり痛むんです。	
\\	胸[むね]がきりきり 痛[いた]むんです。	
\\	うんと痛むの?	
\\	うんと 痛[いた]むの?	
\\	その話を聞くと胸が痛む。	
\\	その 話[はなし]を 聞[き]くと 胸[むね]が 痛[いた]む。	
\\	梅雨時は食べ物が傷みやすいので注意しましょう。	
\\	梅雨[つゆ] 時[どき]は 食べ物[たべもの]が 傷[いた]みやすいので 注意[ちゅうい]しましょう。	梅雨時=つゆどき
\\	このオレンジは傷みかけている。	
\\	このオレンジは 傷[いた]みかけている。	
\\	最近の豪雨で道路はかなり傷んだ。	
\\	最近[さいきん]の 豪雨[ごうう]で 道路[どうろ]はかなり 傷[いた]んだ。	豪雨=ごうう= 
\\	世界は輝きに満ちている。	
\\	世界[せかい]は 輝[かがや]きに 満[み]ちている。	輝き=かがやき= 
\\	そんな嫌みを言うな。	
\\	そんな 嫌[いや]みを 言[い]うな。	
\\	彼の嫌みな言い草がしゃくにさわった。	
\\	彼[かれ]の 嫌[いや]みな 言い草[いいぐさ]がしゃくにさわった。	言い草=いいぐさ= 
\\	癪に障る=しゃくにさわる= 
\\	隣家の奥さんは私に嫌みたっぷりに毎日ピアノの練習大変ねと言った。	
\\	隣家[りんか]の 奥[おく]さんは 私[わたし]に 嫌[いや]みたっぷりに 毎日[まいにち]ピアノの 練習[れんしゅう] 大変[たいへん]ねと 言[い]った。	
\\	これが精一杯です。	
\\	これが 精一杯[せいいっぱい]です。	
\\	着のみ着のままで焼け出されました。	
\\	着のみ着[きのみき]のままで 焼け出[やけだ]されました。	着のみ着のまま=きのみきのまま= 
\\	私は彼が気の毒になって、できるだけの援助を約束した。	
\\	私[わたし]は 彼[かれ]が 気の毒[きのどく]になって、できるだけの 援助[えんじょ]を 約束[やくそく]した。	気の毒=きのどく= 
\\	(遺憾) 
\\	(不幸) 
\\	誠にお気の毒なことです。	
\\	誠[まこと]にお 気の毒[きのどく]なことです。	気の毒=きのどく= 
\\	(遺憾) 
\\	(不幸) 
\\	彼は実に気の毒な暮らしをしている。	
\\	彼[かれ]は 実[じつ]に 気の毒[きのどく]な 暮[く]らしをしている。	気の毒=きのどく= 
\\	(遺憾) 
\\	(不幸) 
\\	あんなに勉強して不合格とは気の毒だ。	
\\	あんなに 勉強[べんきょう]して 不[ふ] 合格[ごうかく]とは 気の毒[きのどく]だ。	気の毒=きのどく= 
\\	(遺憾) 
\\	(不幸) 
\\	済んだら私に貸して下さい。	
\\	済[す]んだら 私[わたし]に 貸[か]して 下[くだ]さい。	済む=すむ= (終了する) 
\\	(人が主語) 
\\	済んだことは仕方がない。	
\\	済[す]んだことは 仕方[しかた]がない。	済む=すむ= (終了する) 
\\	(人が主語) 
\\	ローンが済まないうちは節約しなければならない。	
\\	ローンが 済[す]まないうちは 節約[せつやく]しなければならない。	済む=すむ= (終了する) 
\\	(人が主語) 
\\	韓国旅行なら10万円あれば済む。	
\\	韓国[かんこく] 旅行[りょこう]なら 
\\	万[まん] 円[えん]あれば 済[す]む。	済む=すむ= (終了する) 
\\	(人が主語) 
\\	借りずに済むなら借りはしない。	
\\	借[か]りずに 済[す]むなら 借[か]りはしない。	済む=すむ= (終了する) 
\\	(人が主語) 
\\	自家用車がなくても済んでいる。	
\\	自家用車[じかようしゃ]がなくても 済[す]んでいる。	自家用車= 
\\	済む=すむ= (終了する) 
\\	(人が主語) 
\\	罰金だけで済んだ。	
\\	罰金[ばっきん]だけで 済[す]んだ。	済む=すむ= (終了する) 
\\	(人が主語) 
\\	謝罪して済む問題ではない。	
\\	謝罪[しゃざい]して 済[す]む 問題[もんだい]ではない。	済む=すむ= (終了する) 
\\	(人が主語) 
\\	バーゲン売り場に人が群がっていた。	
\\	バーゲン 売り場[うりば]に 人[ひと]が 群[むら]がっていた。	群がる=むらがる= 
\\	記者団が彼のまわりに群がり質問を浴びせかけた。	
\\	記者[きしゃ] 団[だん]が 彼[かれ]のまわりに 群[むら]がり 質問[しつもん]を 浴[あ]びせかけた。	群がる=むらがる= 
\\	広場に群衆が集まり出した。	
\\	広場[ひろば]に 群衆[ぐんしゅう]が 集[あつ]まり 出[だ]した。	群衆=ぐんしゅう= 
\\	広場=ひろば= 
\\	群衆が街にあふれ出た。	
\\	群衆[ぐんしゅう]が 街[まち]にあふれ 出[で]た。	群衆=ぐんしゅう= 
\\	イルカの群れが泳いでいる。	
\\	イルカの 群[む]れが 泳[およ]いでいる。	
\\	野党は結束してその政策を批判した。	
\\	野党[やとう]は 結束[けっそく]してその 政策[せいさく]を 批判[ひはん]した。	野党=やとう= 
\\	音が大きくなるにつれ恐怖が募った。	
\\	音[おと]が 大[おお]きくなるにつれ 恐怖[きょうふ]が 募[つの]った。	
\\	彼女は恐怖のあまりその場に立ちすくんだ。	
\\	彼女[かのじょ]は 恐怖[きょうふ]のあまりその 場[ば]に 立[た]ちすくんだ。	
\\	彼は少しも恐怖の色を見せなかった。	
\\	彼[かれ]は 少[すこ]しも 恐怖[きょうふ]の 色[いろ]を 見[み]せなかった。	
\\	彼女から誤りを指摘されて大いに恐縮した。	
\\	彼女[かのじょ]から 誤[あやま]りを 指摘[してき]されて 大[おお]いに 恐縮[きょうしゅく]した。	恐縮=きょうしゅく= 
\\	恐縮ですが、たばこの火を貸していただけませんか。	
\\	恐縮[きょうしゅく]ですが、たばこの 火[ひ]を 貸[か]していただけませんか。	恐縮=きょうしゅく= 
\\	彼には恐らく不可能でしょう。	
\\	彼[かれ]には 恐[おそ]らく 不可能[ふかのう]でしょう。	
\\	恐らく彼は気にしないだろう。	
\\	恐[おそ]らく 彼[かれ]は 気[き]にしないだろう。	
\\	恐らく彼は二度と日本に帰らないでしょう。	
\\	恐[おそ]らく 彼[かれ]は 二度[にど]と日本に 帰[かえ]らないでしょう。	
\\	恐らく彼は 2、3日すれば帰るだろう。	
\\	恐[おそ]らく 彼[かれ]は 2、 
\\	日[にち]すれば 帰[かえ]るだろう。	
\\	恐らく彼はそれを知らないだろう。	
\\	恐[おそ]らく 彼[かれ]はそれを 知[し]らないだろう。	
\\	恐らくこの絵は彼のコレクションの中でも最高のものだろう。	
\\	恐[おそ]らくこの 絵[え]は 彼[かれ]のコレクションの 中[なか]でも 最高[さいこう]のものだろう。	
\\	恐らくそんなことだろうと思っていたよ。	
\\	恐[おそ]らくそんなことだろうと 思[おも]っていたよ。	
\\	彼の知識の豊富さには恐るべきものがある。	
\\	彼[かれ]の 知識[ちしき]の 豊富[ほうふ]さには 恐[おそれ]るべきものがある。	
\\	インターネットの普及は恐るべき速さで進んでいる。	
\\	インターネットの 普及[ふきゅう]は 恐[おそ]るべき 速[はや]さで 進[すす]んでいる。	
\\	その程度のリスクは恐るるに足らない。	
\\	その 程度[ていど]のリスクは 恐[おそ]るるに 足[た]らない。	
\\	試験の結果をおそるおそる先生に尋ねた。	
\\	試験[しけん]の 結果[けっか]をおそるおそる 先生[せんせい]に 尋[たず]ねた。	
\\	彼らは君に恐れをなしている。	
\\	彼[かれ]らは 君[きみ]に 恐[おそ]れをなしている。	恐れをなす= 
\\	兵士たちは大勢の敵に恐れをなした。	
\\	兵士[へいし]たちは 大勢[おおぜい]の 敵[てき]に 恐[おそ]れをなした。	恐れをなす= 
\\	失敗の恐れがある。	
\\	失敗[しっぱい]の 恐[おそ]れがある。	
\\	提出された法案は憲法に抵触する恐れがある。	
\\	提出[ていしゅつ]された 法案[ほうあん]は 憲法[けんぽう]に 抵触[ていしょく]する 恐[おそ]れがある。	
\\	この容疑者は海外に逃亡する恐れがある。	
\\	この 容疑[ようぎ] 者[しゃ]は 海外[かいがい]に 逃亡[とうぼう]する 恐[おそ]れがある。	
\\	彼を恐れない者はない。	
\\	彼[かれ]を 恐[おそ]れない 者[もの]はない。	
\\	遅れはしないかと恐れた。	
\\	遅[おく]れはしないかと 恐[おそ]れた。	
\\	君には何も恐れることはない。	
\\	君[きみ]には 何[なに]も 恐[おそ]れることはない。	
\\	失敗を恐れてばかりいては何もできない。	
\\	失敗[しっぱい]を 恐[おそ]れてばかりいては 何[なに]もできない。	
\\	その部隊は恐れ気もなく敵に向かって進んだ。	
\\	その 部隊[ぶたい]は 恐[おそ]れ 気[げ]もなく 敵[てき]に 向[む]かって 進[すす]んだ。	恐れ気もなく=おそれげもなく= 
\\	恐ろしく寒い晩だった。	
\\	恐[おそ]ろしく 寒[さむ]い 晩[ばん]だった。	
\\	物価が恐ろしく上がっている。	
\\	物価[ぶっか]が 恐[おそ]ろしく 上[あ]がっている。	
\\	トラが恐ろしいうなり声を立てた。	
\\	トラが 恐[おそ]ろしいうなり 声[ごえ]を 立[た]てた。	うなり声= 
\\	あの人は恐ろしい人だ。	
\\	あの 人[ひと]は 恐[おそ]ろしい 人[ひと]だ。	
\\	当市ではお年寄りや体の不自由な方のための施設が大変よく整っている。	
\\	当[とう] 市[し]ではお 年寄[としよ]りや 体[からだ]の 不自由[ふじゆう]な 方[ほう]のための 施設[しせつ]が 大変[たいへん]よく 整[ととの]っている。	
\\	日本は四季の移ろいがはっきりしている。	
\\	日本[にっぽん]は 四季[しき]の 移[うつ]ろいがはっきりしている。	
\\	世界では三千にも及ぶ多種多様な言語が使われている。	
\\	世界[せかい]では 三千[さんぜん]にも 及[およ]ぶ 多種[たしゅ] 多様[たよう]な 言語[げんご]が 使[つか]われている。	多種多様=たしゅたよう= 
\\	そのとき彼女はもうすっかり死を覚悟していた。	
\\	そのとき 彼女[かのじょ]はもうすっかり 死[し]を 覚悟[かくご]していた。	覚悟= (心の準備) 
\\	(決心) 
\\	彼女は成功するまでは帰らない覚悟で国を出た。	
\\	彼女[かのじょ]は 成功[せいこう]するまでは 帰[かえ]らない 覚悟[かくご]で 国[くに]を 出[で]た。	覚悟= (心の準備) 
\\	(決心) 
\\	多少の危険は覚悟のうえだ。	
\\	多少[たしょう]の 危険[きけん]は 覚悟[かくご]のうえだ。	覚悟= (心の準備) 
\\	(決心) 
\\	彼は命をかける覚悟である。	
\\	彼[かれ]は 命[いのち]をかける 覚悟[かくご]である。	覚悟= (心の準備) 
\\	(決心) 
\\	光が強くて眩しかった。	
\\	光[ひかり]が 強[つよ]くて 眩[まぶ]しかった。	眩しい=まぶしい= 
\\	その無邪気な笑顔が私には眩しかった。	
\\	その 無邪気[むじゃき]な 笑顔[えがお]が 私[わたし]には 眩[まぶ]しかった。	眩しい=まぶしい= 
\\	被告は恐れ入りましたと言った。	
\\	被告[ひこく]は 恐れ入[おそれい]りましたと 言[い]った。	恐れ入る=おそれいる= (驚きあきれる) 
\\	(申し訳なく思う) 
\\	(ありがたく思う) 
\\	(負ける) 
\\	(非を認める) 
\\	第2問には恐れ入った。	
\\	第[だい]2 問[もん]には 恐れ入[おそれい]った。	恐れ入る=おそれいる= (驚きあきれる) 
\\	(申し訳なく思う) 
\\	(ありがたく思う) 
\\	(負ける) 
\\	(非を認める) 
\\	お邪魔をして恐れ入りますが、これをどこに置きましょうか。	
\\	お 邪魔[じゃま]をして 恐れ入[おそれい]りますが、これをどこに 置[お]きましょうか。	恐れ入る=おそれいる= (驚きあきれる) 
\\	(申し訳なく思う) 
\\	(ありがたく思う) 
\\	(負ける) 
\\	(非を認める) 
\\	恐れ入りますがその窓を開けて下さいますか。	
\\	恐れ入[おそれい]りますがその 窓[まど]を 開[あ]けて 下[くだ]さいますか。	開ける=あける 恐れ入る=おそれいる= (驚きあきれる) 
\\	(申し訳なく思う) 
\\	(ありがたく思う) 
\\	(負ける) 
\\	(非を認める) 
\\	恐れ入りますが何時ですか。	
\\	恐れ入[おそれい]りますが 何[なん] 時[じ]ですか。	恐れ入る=おそれいる= (驚きあきれる) 
\\	(申し訳なく思う) 
\\	(ありがたく思う) 
\\	(負ける) 
\\	(非を認める) 
\\	彼は医師の勧告を無視してサッカーを続けた。	
\\	彼[かれ]は 医師[いし]の 勧告[かんこく]を 無視[むし]してサッカーを 続[つづ]けた。	
\\	彼の演技は気迫にあふれている。	
\\	彼[かれ]の 演技[えんぎ]は 気迫[きはく]にあふれている。	演技=えんぎ= 
\\	お腹がせり出した。	
\\	お 腹[なか]がせり 出[だ]した。	迫り出す=せりだす= 
\\	試験が迫っている。	
\\	試験[しけん]が 迫[せま]っている。	
\\	臨終が迫っている。	
\\	臨終[りんじゅう]が 迫[せま]っている。	臨終=りんじゅう= 
\\	試験は3日後に迫っている。	
\\	試験[しけん]は 3日[みっか] 後[ご]に 迫[せま]っている。	
\\	その映像は迫力がない。	
\\	その 映像[えいぞう]は 迫力[はくりょく]がない。	迫力=はくりょく= 
\\	彼の言葉には大して迫力がない。	
\\	彼[かれ]の 言葉[ことば]には 大[たい]して 迫力[はくりょく]がない。	迫力=はくりょく= 
\\	「だまっていろ、さもないと命がないぞ」という脅迫状が舞い込んだ。	
\\	「だまっていろ、さもないと 命[いのち]がないぞ」という 脅迫[きょうはく] 状[じょう]が 舞い込[まいこ]んだ。	さもないと= 
\\	舞い込む= 
\\	私は彼女に脅迫されている。	
\\	私[わたし]は 彼女[かのじょ]に 脅迫[きょうはく]されている。	
\\	彼は訴えるぞと私を脅迫した。	
\\	彼[かれ]は 訴[うった]えるぞと 私[わたし]を 脅迫[きょうはく]した。	
\\	脅迫するつもりか。	
\\	脅迫[きょうはく]するつもりか。	
\\	靴のひもが解けた。	
\\	靴[くつ]のひもが 解[ほど]けた。	紐=ひも= 
\\	解ける=ほどける= 
\\	息子はまだ靴のひもが結べない。	
\\	息子[むすこ]はまだ 靴[くつ]のひもが 結[むす]べない。	紐=ひも= 
\\	これで事件の全容が解明されるだろう。	
\\	これで 事件[じけん]の 全容[ぜんよう]が 解明[かいめい]されるだろう。	解明= 
\\	どのようにして月が誕生したか、まだ解明されていない。	
\\	どのようにして 月[つき]が 誕生[たんじょう]したか、まだ 解明[かいめい]されていない。	解明= 
\\	彼はこれ以上老醜をさらしたくなかったのであろう。	
\\	彼[かれ]はこれ 以上[いじょう] 老醜[ろうしゅう]をさらしたくなかったのであろう。	老醜=ろうしゅう= 
\\	あの新聞は報道が迅速だ。	
\\	あの 新聞[しんぶん]は 報道[ほうどう]が 迅速[じんそく]だ。	迅速=じんそく= 
\\	面倒な仕事は後回しにして、できることから先に片づけよう。	
\\	面倒[めんどう]な 仕事[しごと]は 後回[あとまわ]しにして、できることから 先[さき]に 片[かた]づけよう。	後回し=あとまわし= 
\\	それは後回しにしてよい。	
\\	それは 後回[あとまわ]しにしてよい。	後回し=あとまわし= 
\\	彼はいつも自分のことを後回しにして人の世話をする。	
\\	彼[かれ]はいつも 自分[じぶん]のことを 後回[あとまわ]しにして 人[ひと]の 世話[せわ]をする。	後回し=あとまわし= 
\\	あの会社は最悪の状況に立ち至っている。	
\\	あの 会社[かいしゃ]は 最悪[さいあく]の 状況[じょうきょう]に 立ち至[たちいた]っている。	立ち至る= 
\\	洪水のため、村は跡形もなく消えてしまった。	
\\	洪水[こうずい]のため、 村[むら]は 跡形[あとかた]もなく 消[き]えてしまった。	跡形=あとかた= (形跡・痕跡) 
\\	(証跡) 
\\	跡形もなく=あとかたもなく= 
\\	雪は昼までに跡形もなく消えた。	
\\	雪[ゆき]は 昼[ひる]までに 跡形[あとかた]もなく 消[き]えた。	跡形=あとかた= (形跡・痕跡) 
\\	(証跡) 
\\	跡形もなく=あとかたもなく= 
\\	彼女の赤い帽子は、群衆の中でも目に付く。	
\\	彼女[かのじょ]の 赤[あか]い 帽子[ぼうし]は、 群衆[ぐんしゅう]の 中[なか]でも 目[め]に 付[つ]く。	
\\	彼女の視線をたどると、その先にはハンサムな青年がいた。	
\\	彼女[かのじょ]の 視線[しせん]をたどると、その 先[さき]にはハンサムな 青年[せいねん]がいた。	辿る=たどる= 
\\	その会社はひたすら破産に至る道をたどっていた。	
\\	その 会社[かいしゃ]はひたすら 破産[はさん]に 至[いた]る 道[みち]をたどっていた。	ひたすら= 
\\	辿る=たどる= 
\\	こんな落ちぶれた浅ましい格好を人に見られたくない。	
\\	こんな 落[お]ちぶれた 浅[あさ]ましい 格好[かっこう]を 人[ひと]に 見[み]られたくない。	浅ましい= (みじめな) 
\\	(卑しい) 
\\	(恥ずべき) 
\\	落ちぶれる= 
\\	教師が、女生徒の更衣室をのぞき見したという浅ましい行為で捕まった。	
\\	教師[きょうし]が、 女生徒[じょせいと]の 更衣[こうい] 室[しつ]をのぞき 見[み]したという 浅[あさ]ましい 行為[こうい]で 捕[つか]まった。	更衣室=こういしつ= 
\\	覗き見する=のぞきみする= 
\\	浅ましい= (みじめな) 
\\	(卑しい) 
\\	(恥ずべき) 
\\	姫を救ったのは身分の卑しい男だった。	
\\	姫[ひめ]を 救[すく]ったのは 身分[みぶん]の 卑[いや]しい 男[おとこ]だった。	卑しい=いやしい= 
\\	彼女は口が卑しい。	
\\	彼女[かのじょ]は 口[くち]が 卑[いや]しい。	卑しい=いやしい= 
\\	彼は金に卑しい。	
\\	彼[かれ]は 金[かね]に 卑[いや]しい。	卑しい=いやしい= 
\\	いやしくも戦うからには精一杯戦え。	
\\	いやしくも 戦[たたか]うからには 精一杯[せいいっぱい] 戦[たたか]え。	
\\	私のことには一切干渉してくれるな。	
\\	私[わたし]のことには 一切[いっさい] 干渉[かんしょう]してくれるな。	
\\	自分の結婚については、誰からの干渉も受けたくない。	
\\	自分[じぶん]の 結婚[けっこん]については、 誰[だれ]からの 干渉[かんしょう]も 受[う]けたくない。	
\\	現代の世の中では肩書きが物を言う。	
\\	現代[げんだい]の 世の中[よのなか]では 肩書[かたが]きが 物[もの]を 言[い]う。	肩書き=かたがき= 
\\	物を言う= 
\\	市況は低迷を続けている。	
\\	市況[しきょう]は 低迷[ていめい]を 続[つづ]けている。	
\\	この一節は両様に解釈ができる。	
\\	この 一節[いっせつ]は 両様[りょうよう]に 解釈[かいしゃく]ができる。	
\\	祖母は退院して自宅で療養中だ。	
\\	祖母[そぼ]は 退院[たいいん]して 自宅[じたく]で 療養[りょうよう] 中[ちゅう]だ。	
\\	授業をうつらうつらしながら聞く学生が必ずいる。	
\\	授業[じゅぎょう]をうつらうつらしながら 聞[き]く 学生[がくせい]が 必[かなら]ずいる。	
\\	うとうとしている間に約束の時間が過ぎてしまった。	
\\	うとうとしている 間[あいだ]に 約束[やくそく]の 時間[じかん]が 過[す]ぎてしまった。	
\\	息子は昨日からの高熱でウンウンなされている。	
\\	息子[むすこ]は 昨日[きのう]からの 高熱[こうねつ]でウンウンなされている。	
\\	けが人は痛くてウンウン言っている。	
\\	けが 人[にん]は 痛[いた]くてウンウン 言[い]っている。	けが人=けがにん
\\	私はからからの喉で、やっと「水」と言った。	
\\	私[わたし]はからからの 喉[のど]で、やっと
\\	水[みず]」と 言[い]った。	
\\	毎日からからの天気が続いています。	
\\	毎日[まいにち]からからの 天気[てんき]が 続[つづ]いています。	
\\	喉がからからだ。	
\\	喉[のど]がからからだ。	
\\	喉がからからになって声が出ない。	
\\	喉[のど]がからからになって 声[こえ]が 出[で]ない。	
\\	二日酔いで頭ががんがんする。	
\\	二日酔[ふつかよ]いで 頭[あたま]ががんがんする。	がんがん= (連続する鈍い金属的な音) 
\\	(勢いのよいさま)
\\	そんな大声を出さないでくれ。頭にがんがん響く。	
\\	そんな 大声[おおごえ]を 出[だ]さないでくれ。 頭[あたま]にがんがん 響[ひび]く。	がんがん= (連続する鈍い金属的な音) 
\\	(勢いのよいさま)
\\	息子はロックをがんがんかけながら勉強するんです。	
\\	息子[むすこ]はロックをがんがんかけながら 勉強[べんきょう]するんです。	がんがん= (連続する鈍い金属的な音) 
\\	(勢いのよいさま)
\\	よーし、今夜はがんがん飲むぞ!	
\\	よーし、 今夜[こんや]はがんがん 飲[の]むぞ!	がんがん= (連続する鈍い金属的な音) 
\\	(勢いのよいさま)
\\	腹がぐうぐう鳴った。	
\\	腹[はら]がぐうぐう 鳴[な]った。	
\\	お腹が減ってぐうぐう鳴っている。	
\\	お 腹[なか]が 減[へ]ってぐうぐう 鳴[な]っている。	
\\	煮すぎて野菜がぐたぐたになってしまった。	
\\	煮[に]すぎて 野菜[やさい]がぐたぐたになってしまった。	
\\	暑い日はいつも家でぐたぐたしている。	
\\	暑[あつ]い 日[ひ]はいつも 家[いえ]でぐたぐたしている。	
\\	長い戦争もようやく終わりを告げた。	
\\	長[なが]い 戦争[せんそう]もようやく 終[お]わりを 告[つ]げた。	告げる=つげる= (知らせる) 
\\	(指示する) 
\\	(広告する) 
\\	(気づかせる) 
\\	(その状態になる) 終わりを告げる= 
\\	来客を告げるベルが鳴った。	
\\	来客[らいきゃく]を 告[つ]げるベルが 鳴[な]った。	告げる=つげる= (知らせる) 
\\	(指示する) 
\\	(広告する) 
\\	(気づかせる) 
\\	(その状態になる)
\\	彼女は恋人に別れを告げた。	
\\	彼女[かのじょ]は 恋人[こいびと]に 別[わか]れを 告[つ]げた。	告げる=つげる= (知らせる) 
\\	(指示する) 
\\	(広告する) 
\\	(気づかせる) 
\\	(その状態になる)
\\	彼は行き先を告げずに出かけた。	
\\	彼[かれ]は 行き先[ゆきさき]を 告[つ]げずに 出[で]かけた。	告げる=つげる= (知らせる) 
\\	(指示する) 
\\	(広告する) 
\\	(気づかせる) 
\\	(その状態になる)
\\	彼女は菜食主義者だ。焼き肉なんか食べっこない。	
\\	彼女[かのじょ]は 菜食[さいしょく] 主義[しゅぎ] 者[しゃ]だ。 焼き肉[やきにく]なんか 食[た]べっこない。	〜っこない= するわけがない
\\	彼が女装をするなんてそんなことありっこない。	
\\	彼[かれ]が 女装[じょそう]をするなんてそんなことありっこない。	〜っこない= するわけがない
\\	こんな大事なことはメモしないでも忘れっこない。	
\\	こんな 大事[だいじ]なことはメモしないでも 忘[わす]れっこない。	〜っこない= するわけがない
\\	その話はもう耳にタコができるほど聞いた。	
\\	その 話[はなし]はもう 耳[みみ]にタコができるほど 聞[き]いた。	耳にタコができる= 
\\	マツタケは庶民の口にはなかなか入らない。	
\\	マツタケは 庶民[しょみん]の 口[くち]にはなかなか 入[はい]らない。	松茸=まつたけ= 
\\	庶民=しょみん= 
\\	この町の庶民的なところが好きだ。	
\\	この 町[まち]の 庶民[しょみん] 的[てき]なところが 好[す]きだ。	庶民=しょみん= 
\\	問題はこの一点に集約される。	
\\	問題[もんだい]はこの一 点[てん]に 集約[しゅうやく]される。	集約=しゅうやく= (寄せ集める) 
\\	(要約する) 
\\	(一つにまとめ上げる) 
\\	その城の跡は今は立派な公園になっている。	
\\	その 城[しろ]の 跡[あと]は 今[いま]は 立派[りっぱ]な 公園[こうえん]になっている。	跡=あと= (印) 
\\	(汚点) 
\\	商品の評判が良くて、客の注文が跡を絶たない。	
\\	商品[しょうひん]の 評判[ひょうばん]が 良[よ]くて、 客[きゃく]の 注文[ちゅうもん]が 跡[あと]を 絶[た]たない。	跡=あと= (印) 
\\	(汚点) 
\\	跡を絶つ= 
\\	犯罪が跡を絶たない。	
\\	犯罪[はんざい]が 跡[あと]を 絶[た]たない。	跡=あと= (印) 
\\	(汚点) 
\\	跡を絶つ= 
\\	手術の跡が消えずに残っている。	
\\	手術[しゅじゅつ]の 跡[あと]が 消[き]えずに 残[のこ]っている。	跡=あと= (印) 
\\	(汚点) 
\\	手には苦労の跡が刻まれている。	
\\	手[て]には 苦労[くろう]の 跡[あと]が 刻[きざ]まれている。	跡=あと= (印) 
\\	(汚点) 
\\	犯行現場の床には、血の跡が点々とついていた。	
\\	犯行[はんこう] 現場[げんば]の 床[ゆか]には、 血[ち]の 跡[あと]が 点々[てんてん]とついていた。	跡=あと= (印) 
\\	(汚点) 
\\	床=ゆか 点々=てんてん= 
\\	窓ガラスに汚れた指の跡がついている。	
\\	窓[まど]ガラスに 汚[よご]れた 指[ゆび]の 跡[あと]がついている。	跡=あと= (印) 
\\	(汚点) 
\\	健康は幸福の根本である。	
\\	健康[けんこう]は 幸福[こうふく]の 根本[こんぽん]である。	
\\	君の議論は根本から誤っている。	
\\	君[きみ]の 議論[ぎろん]は 根本[こんぽん]から 誤[あやま]っている。	
\\	彼女は団体行動を厭っていつも独りでいる。	
\\	彼女[かのじょ]は 団体[だんたい] 行動[こうどう]を 厭[いと]っていつも 独[ひと]りでいる。	厭う=いとう= 
\\	私は少しぐらいの危険は厭わない。	
\\	私[わたし]は 少[すこ]しぐらいの 危険[きけん]は 厭[いと]わない。	厭う=いとう= 
\\	彼は寒さを厭わず釣りに出かけた。	
\\	彼[かれ]は 寒[さむ]さを 厭[いと]わず 釣[つ]りに 出[で]かけた。	厭う=いとう= 
\\	この鳥は絶滅の危険にさらされている。	
\\	この 鳥[とり]は 絶滅[ぜつめつ]の 危険[きけん]にさらされている。	晒す・曝す=さらす= [曝す]
\\	[晒す] 
\\	風雨にさらされて、自転車がさびてしまった。	
\\	風雨[ふうう]にさらされて、 自転車[じてんしゃ]がさびてしまった。	晒す・曝す=さらす= [曝す]
\\	[晒す] 
\\	彼女は腰まで髪を伸ばしていた。	
\\	彼女[かのじょ]は 腰[こし]まで 髪[かみ]を 伸[の]ばしていた。	伸ばす・延ばす= [伸ばす] 
\\	[延ばす] 
\\	木は太陽の方向に枝を延ばす。	
\\	木[き]は 太陽[たいよう]の 方向[ほうこう]に 枝[えだ]を 延[の]ばす。	伸ばす・延ばす= [伸ばす] 
\\	[延ばす] 
\\	彼は手を伸ばしてナイフを取ろうとした。	
\\	彼[かれ]は 手[て]を 伸[の]ばしてナイフを 取[と]ろうとした。	伸ばす・延ばす= [伸ばす] 
\\	[延ばす] 
\\	売り上げを2割ほど伸ばしたい。	
\\	売り上[うりあ]げを2 割[わり]ほど 伸[の]ばしたい。	伸ばす・延ばす= [伸ばす] 
\\	[延ばす] 
\\	返済はもう一ヶ月延ばして下さい。	
\\	返済[へんさい]はもう 一ヶ月[いっかげつ] 延[の]ばして 下[くだ]さい。	伸ばす・延ばす= [伸ばす] 
\\	[延ばす] 
\\	その問題についての採決は次会まで延ばすことにした。	
\\	その 問題[もんだい]についての 採決[さいけつ]は 次会[じかい]まで 延[の]ばすことにした。	採決=さいけつ= 
\\	伸ばす・延ばす= [伸ばす] 
\\	[延ばす] 
\\	締め切りはもうこれ以上延ばせません。	
\\	締め切[しめき]りはもうこれ 以上[いじょう] 延[の]ばせません。	伸ばす・延ばす= [伸ばす] 
\\	[延ばす] 
\\	相手を一発で延ばすだけの力を持っている。	
\\	相手[あいて]を一 発[はつ]で 延[の]ばすだけの 力[ちから]を 持[も]っている。	伸ばす・延ばす= [伸ばす] 
\\	[延ばす] 
\\	もう辛抱し切れない。	
\\	もう 辛抱[しんぼう]し 切[き]れない。	辛抱=しんぼう= (我慢) 
\\	今度の雇い主は厳しくてとても辛抱できません。	
\\	今度[こんど]の 雇い主[やといぬし]は 厳[きび]しくてとても 辛抱[しんぼう]できません。	辛抱=しんぼう= (我慢) 
\\	もう少しの辛抱だ。	
\\	もう 少[すこ]しの 辛抱[しんぼう]だ。	辛抱=しんぼう= (我慢) 
\\	つらいだろうがしばらくの辛抱だよ。	
\\	つらいだろうがしばらくの 辛抱[しんぼう]だよ。	辛抱=しんぼう= (我慢) 
\\	それは何の変哲もない石ころのように見えた。	
\\	それは 何[なん]の 変哲[へんてつ]もない 石[いし]ころのように 見[み]えた。	変哲もない= 
\\	(平凡) 
\\	(普通の) 
\\	二人の結婚式は春に延期された。	
\\	二人[ふたり]の 結婚式[けっこんしき]は 春[はる]に 延期[えんき]された。	
\\	展覧会が好評で明日の終了が延期になった。	
\\	展覧[てんらん] 会[かい]が 好評[こうひょう]で 明日[あした]の 終了[しゅうりょう]が 延期[えんき]になった。	
\\	田舎の友人がひょっこり訪ねてきた。	
\\	田舎[いなか]の 友人[ゆうじん]がひょっこり 訪[たず]ねてきた。	訪ねる=たずねる= 
\\	彼はちょいちょい訪ねてくる。	
\\	彼[かれ]はちょいちょい 訪[たず]ねてくる。	訪ねる=たずねる= 
\\	修学旅行以来奈良を訪ねるのはこれが初めてだ。	
\\	修学旅行[しゅうがくりょこう] 以来[いらい] 奈良[なら]を 訪[たず]ねるのはこれが 初[はじ]めてだ。	訪ねる=たずねる= 
\\	待ちに待ったチャンスが訪れた。	
\\	待[ま]ちに 待[ま]ったチャンスが 訪[おとず]れた。	待ちに待った= 
\\	訪れる=おとずれる= (訪問する) 
\\	(到来する) 
\\	長い内戦が終わり、ようやくその地に平和が訪れた。	
\\	長[なが]い 内戦[ないせん]が 終[お]わり、ようやくその 地[ち]に 平和[へいわ]が 訪[おとず]れた。	訪れる=おとずれる= (訪問する) 
\\	(到来する) 
\\	京都を訪れたいという外国人観光客は多い。	
\\	京都[きょうと]を 訪[おとず]れたいという 外国[がいこく] 人[じん] 観光[かんこう] 客[きゃく]は 多[おお]い。	訪れる=おとずれる= (訪問する) 
\\	(到来する) 
\\	この家は掃除が行き届いている。	
\\	この 家[いえ]は 掃除[そうじ]が 行き届[いきとど]いている。	行き届く= (注意が届く) 
\\	社長の秘書は万事に行き届いている。	
\\	社長[しゃちょう]の 秘書[ひしょ]は 万事[ばんじ]に 行き届[いきとど]いている。	行き届く= (注意が届く) 
\\	目上の人の前では起ち居振る舞いに気をつけなさい。	
\\	目上[めうえ]の 人[ひと]の 前[まえ]では 起[た]ち 居[い] 振る舞[ふるま]いに 気[き]をつけなさい。	起ち居振る舞い・立ち居振る舞い=たちいふるまい= 
\\	毎朝起き抜けにまずコーヒーを飲む。	
\\	毎朝[まいあさ] 起き抜[おきぬ]けにまずコーヒーを 飲[の]む。	起き抜けに= 
\\	軍隊は規律が厳重だ。	
\\	軍隊[ぐんたい]は 規律[きりつ]が 厳重[げんじゅう]だ。	規律=きりつ= (秩序) 
\\	(きまり) 
\\	規律が乱れている。	
\\	規律[きりつ]が 乱[みだ]れている。	規律=きりつ= (秩序) 
\\	(きまり) 
\\	年を取ると新しいものへの適応が難しくなる。	
\\	年[とし]を 取[と]ると 新[あたら]しいものへの 適応[てきおう]が 難[むずか]しくなる。	適応=てきおう= 
\\	この突然の環境の変化に適応し得たものだけが生き残った。	
\\	この 突然[とつぜん]の 環境[かんきょう]の 変化[へんか]に 適応[てきおう]し 得[え]たものだけが 生き残[いきのこ]った。	適応=てきおう= 
\\	彼の性格は祖父母によって形成された。	
\\	彼[かれ]の 性格[せいかく]は 祖父母[そふぼ]によって 形成[けいせい]された。	形成=けいせい= 
\\	形勢が好転した。	
\\	形勢[けいせい]が 好転[こうてん]した。	形勢=けいせい= (情勢) 
\\	形勢は2対1で負けている。	
\\	形勢[けいせい]は2 対[たい]1で 負[ま]けている。	形勢=けいせい= (情勢) 
\\	彼は声の調子を和らげて言った。	
\\	彼[かれ]は 声[こえ]の 調子[ちょうし]を 和[やわ]らげて 言[い]った。	
\\	暑さが和らいだ。	
\\	暑[あつ]さが 和[やわ]らいだ。	
\\	彼の怒りもようやく和らいできたようだ。	
\\	彼[かれ]の 怒[いか]りもようやく 和[やわ]らいできたようだ。	
\\	彼らは仕事の出来高に応じて賃金を支給されている。	
\\	彼[かれ]らは 仕事[しごと]の 出来高[できだか]に 応[おう]じて 賃金[ちんぎん]を 支給[しきゅう]されている。	出来高=できだか= 
\\	支給= 
\\	旅行に携帯するには柔らかい本の方がいい。	
\\	旅行[りょこう]に 携帯[けいたい]するには 柔[やわ]らかい 本[ほん]の 方[ほう]がいい。	
\\	この猫は毛が柔らかい。	
\\	この 猫[ねこ]は 毛[け]が 柔[やわ]らかい。	
\\	僕は柔らかいご飯は嫌いだ。	
\\	僕[ぼく]は 柔[やわ]らかいご 飯[はん]は 嫌[きら]いだ。	
\\	薄暗がりで読書するのは目に毒だ。	
\\	薄暗[うすくら]がりで 読書[どくしょ]するのは 目[め]に 毒[どく]だ。	薄暗がり=うすくらがり= 
\\	毒=どく= 
\\	(害) 
\\	(悪意) 
\\	飲み過ぎは体の毒だよ。	
\\	飲[の]み 過[す]ぎは 体[からだ]の 毒[どく]だよ。	毒=どく= 
\\	(害) 
\\	(悪意) 
\\	彼女の声は毒を含んでいた。	
\\	彼女[かのじょ]の 声[こえ]は 毒[どく]を 含[ふく]んでいた。	毒=どく= 
\\	(害) 
\\	(悪意) 
\\	彼は毒にも薬にもならない男だ。	
\\	彼[かれ]は 毒[どく]にも 薬[くすり]にもならない 男[おとこ]だ。	毒にも薬にもならない= 
\\	いくら抵抗してもだめだ。	
\\	いくら 抵抗[ていこう]してもだめだ。	抵抗=ていこう= 
\\	(受け入れがたいという気持ち)
\\	彼の考えは農民たちの激しい抵抗にあった。	
\\	彼[かれ]の 考[かんが]えは 農民[のうみん]たちの 激[はげ]しい 抵抗[ていこう]にあった。	抵抗=ていこう= 
\\	(受け入れがたいという気持ち)
\\	無駄な抵抗はやめろ。	
\\	無駄[むだ]な 抵抗[ていこう]はやめろ。	抵抗=ていこう= 
\\	(受け入れがたいという気持ち)
\\	彼から金銭的援助を受けることには抵抗がある。	
\\	彼[かれ]から 金銭[きんせん] 的[てき] 援助[えんじょ]を 受[う]けることには 抵抗[ていこう]がある。	金銭的=きんせんてき= 
\\	抵抗=ていこう= 
\\	(受け入れがたいという気持ち)
\\	刺身は初めは少し抵抗があるかもしれませんが、慣れればおいしいですよ。	
\\	刺身[さしみ]は 初[はじ]めは 少[すこ]し 抵抗[ていこう]があるかもしれませんが、 慣[な]れればおいしいですよ。	抵抗=ていこう= 
\\	(受け入れがたいという気持ち)
\\	今日は本当に充実した一日だった。	
\\	今日[きょう]は 本当[ほんとう]に 充実[じゅうじつ]した 一日[いちにち]だった。	充実=じゅうじつ= 
\\	(充足) 
\\	(完備) 
\\	今年のカンヌ映画祭は充実している。	
\\	今年[ことし]のカンヌ 映画[えいが] 祭[さい]は 充実[じゅうじつ]している。	充実=じゅうじつ= 
\\	(充足) 
\\	(完備) 
\\	このサイトは検索システムが充実している。	
\\	このサイトは 検索[けんさく]システムが 充実[じゅうじつ]している。	充実=じゅうじつ= 
\\	(充足) 
\\	(完備) 
\\	このスーパーは野菜売場が充実している。	
\\	このスーパーは 野菜[やさい] 売場[うりば]が 充実[じゅうじつ]している。	充実=じゅうじつ= 
\\	(充足) 
\\	(完備) 
\\	株でだいぶ当てたそうじゃないか。	
\\	株[かぶ]でだいぶ 当[あ]てたそうじゃないか。	当てる・充てる= 
\\	(打ち付ける) 
\\	(さらす) 
\\	(成功する) 
\\	彼はこの車の値段をピタリと当てた。	
\\	彼[かれ]はこの 車[くるま]の 値段[ねだん]をピタリと 当[あ]てた。	当てる・充てる= 
\\	(打ち付ける) 
\\	(さらす) 
\\	(成功する) 
\\	手に何を持ってるか当ててごらん。	
\\	手[て]に 何[なに]を 持[も]ってるか 当[あ]ててごらん。	当てる・充てる= 
\\	(打ち付ける) 
\\	(さらす) 
\\	(成功する) 
\\	5階は事務所に充ててある。	
\\	階[かい]は 事務所[じむしょ]に 充[あ]ててある。	当てる・充てる= 
\\	(打ち付ける) 
\\	(さらす) 
\\	(成功する) 
\\	その収益は地震被災者救援に充てる。	
\\	その 収益[しゅうえき]は 地震[じしん] 被災[ひさい] 者[しゃ] 救援[きゅうえん]に 充[あ]てる。	当てる・充てる= 
\\	(打ち付ける) 
\\	(さらす) 
\\	(成功する) 
\\	この基金は慈善事業に充ててある。	
\\	この 基金[ききん]は 慈善[じぜん] 事業[じぎょう]に 充[あ]ててある。	当てる・充てる= 
\\	(打ち付ける) 
\\	(さらす) 
\\	(成功する) 
\\	昔は「イギリス」に「英吉利」という漢字を当てていた。	
\\	英吉利 
\\	昔[むかし]は「イギリス」に
\\	英吉利[いぎりす]」という 漢字[かんじ]を 当[あ]てていた。	当てる・充てる= 
\\	(打ち付ける) 
\\	(さらす) 
\\	(成功する) 
\\	彼女の顔にじっと目を当てたまま彼はそう言った。	
\\	彼女[かのじょ]の 顔[かお]にじっと 目[め]を 当[あ]てたまま 彼[かれ]はそう 言[い]った。	当てる・充てる= 
\\	(打ち付ける) 
\\	(さらす) 
\\	(成功する) 
\\	彼は親に反抗して家出をした。	
\\	彼[かれ]は 親[おや]に 反抗[はんこう]して 家出[いえで]をした。	
\\	軍隊では一切反抗は許されなかった。	
\\	軍隊[ぐんたい]では 一切[いっさい] 反抗[はんこう]は 許[ゆる]されなかった。	
\\	当局はこの惨事を不可抗力であると称している。	
\\	当局[とうきょく]はこの 惨事[さんじ]を 不可抗力[ふかこうりょく]であると 称[しょう]している。	当局= 
\\	不可抗力=ふかこうりょく= 
\\	彼は1年生の時、大学の英語クラブに勧誘された。	
\\	彼[かれ]は 1年生[いちねんせい]の 時[とき]、 大学[だいがく]の 英語[えいご]クラブに 勧誘[かんゆう]された。	勧誘=かんゆう= 
\\	その新興宗教に入るようしつこく勧誘された。	
\\	その 新興[しんこう] 宗教[しゅうきょう]に 入[はい]るようしつこく 勧誘[かんゆう]された。	勧誘=かんゆう= 
\\	私達は彼の家族から暖かい歓迎を受けた。	
\\	私[わたし] 達[たち]は 彼[かれ]の 家族[かぞく]から 暖[あたた]かい 歓迎[かんげい]を 受[う]けた。	
\\	最近の日ロ関係には歓迎すべき動きが見られる。	
\\	最近[さいきん]の 日ロ[にちろ] 関係[かんけい]には 歓迎[かんげい]すべき 動[うご]きが 見[み]られる。	日ロ=にちろ
\\	駅は歓迎する人々でいっぱいだった。	
\\	駅[えき]は 歓迎[かんげい]する 人々[ひとびと]でいっぱいだった。	
\\	この記事は当局には歓迎されないだろう。	
\\	この 記事[きじ]は 当局[とうきょく]には 歓迎[かんげい]されないだろう。	当局= 
\\	率直なご意見を歓迎いたします。	
\\	率直[そっちょく]なご 意見[いけん]を 歓迎[かんげい]いたします。	
\\	新首相の構想は一般から大歓迎を受けた。	
\\	新[しん] 首相[しゅしょう]の 構想[こうそう]は 一般[いっぱん]から 大[だい] 歓迎[かんげい]を 受[う]けた。	構想=こうそう= (計画) 
\\	あなたならいつでも歓迎ですよ。	
\\	あなたならいつでも 歓迎[かんげい]ですよ。	
\\	売れ行きが渋った。	
\\	売れ行[うれゆ]きが 渋[しぶ]った。	渋る=しぶる= 
\\	新図書館は渡辺氏が設計した。	
\\	新[しん] 図書館[としょかん]は 渡辺[わたなべ] 氏[し]が 設計[せっけい]した。	
\\	あの人には跡取りがない。	
\\	あの 人[ひと]には 跡取[あとと]りがない。	跡取り=あととり= (跡継ぎ) 
\\	あの目つき、彼はただ者ではないと私は踏んだ。	
\\	あの 目[め]つき、 彼[かれ]はただ 者[もの]ではないと 私[わたし]は 踏[ふ]んだ。	踏む=ふむ= 
\\	(その場に立つ) 
\\	(訪れる) 
\\	(経験する) 
\\	(評価する・見積もる) 
\\	誰かが踏むと危ないからハサミを片付けなさい。	
\\	誰[だれ]かが 踏[ふ]むと 危[あぶ]ないからハサミを 片付[かたづ]けなさい。	踏む=ふむ= 
\\	(その場に立つ) 
\\	(訪れる) 
\\	(経験する) 
\\	(評価する・見積もる) 
\\	私はどこかでガムを踏んだらしい。	
\\	私[わたし]はどこかでガムを 踏[ふ]んだらしい。	踏む=ふむ= 
\\	(その場に立つ) 
\\	(訪れる) 
\\	(経験する) 
\\	(評価する・見積もる) 
\\	彼女が結婚に踏み切ったのは彼が誰よりも優しかったからだ。	
\\	彼女[かのじょ]が 結婚[けっこん]に 踏み切[ふみき]ったのは 彼[かれ]が 誰[だれ]よりも 優[やさ]しかったからだ。	
\\	政府はついに消費税の導入に踏み切った。	
\\	政府[せいふ]はついに 消費[しょうひ] 税[ぜい]の 導入[どうにゅう]に 踏み切[ふみき]った。	
\\	せっせと働かないと口が干上がる。	
\\	せっせと 働[はたら]かないと 口[くち]が 干上[ひあ]がる。	干上がる=ひあがる= (乾き切る) 
\\	(生活費が途絶える)
\\	田んぼが干上がった。	
\\	田[た]んぼが 干上[ひあ]がった。	田んぼ=たんぼ= 
\\	干上がる=ひあがる= (乾き切る) 
\\	(生活費が途絶える)
\\	私は何事も不徹底にしておくことは嫌いだ。	
\\	私[わたし]は 何事[なにごと]も 不徹底[ふてってい]にしておくことは 嫌[きら]いだ。	
\\	今年の同窓会は連絡が不徹底で集まりが悪かった。	
\\	今年[ことし]の 同窓会[どうそうかい]は 連絡[れんらく]が 不[ふ] 徹底[てってい]で 集[あつ]まりが 悪[わる]かった。	
\\	ドレスが薄くて腕が透き通って見える。	
\\	ドレスが 薄[うす]くて 腕[うで]が 透き通[すきとお]って 見[み]える。	透き通る=すきとおる= 
\\	この湖は底まで透き通って見える。	
\\	この 湖[みずうみ]は 底[そこ]まで 透き通[すきとお]って 見[み]える。	透き通る=すきとおる= 
\\	彼は手の内を見透かされた。	
\\	彼[かれ]は 手の内[てのうち]を 見透[みす]かされた。	手の内=てのうち= 
\\	見透かす=みすかす= (見抜く)
\\	この電車はすいている。	
\\	この 電車[でんしゃ]は 空[あ]いている。	空く=すく= 
\\	(空席ができる) 
\\	正月は道路が空いていて走りやすい。	
\\	正月[しょうがつ]は 道路[どうろ]が 空[す]いていて 走[はし]りやすい。	空く=すく= 
\\	(空席ができる) 
\\	ギャンブルは好かない。	
\\	ギャンブルは 好[す]かない。	
\\	彼は「オリンピックは金持ちの祭典だ」と切り捨てた。	
\\	彼[かれ]は「オリンピックは 金持[かねも]ちの 祭典[さいてん]だ」と 切り捨[きりす]てた。	
\\	乗組員は難破船を乗り捨てた。	
\\	乗組[のりくみ] 員[いん]は 難破[なんぱ] 船[せん]を 乗り捨[のりす]てた。	乗り捨てる= 
\\	盗難自転車は空き地に乗り捨ててあった。	
\\	盗難[とうなん] 自転車[じてんしゃ]は 空き地[あきち]に 乗り捨[のりす]ててあった。	乗り捨てる= 
\\	強盗は車を乗り捨てて山へ逃げ込んだらしい。	
\\	強盗[ごうとう]は 車[くるま]を 乗り捨[のりす]てて 山[やま]へ 逃げ込[にげこ]んだらしい。	乗り捨てる= 
\\	セーターを脱ぎ捨てておいてはいけません。	
\\	セーターを 脱ぎ捨[ぬぎす]てておいてはいけません。	
\\	この森には誰も足を踏み入れたことがない。	
\\	この 森[もり]には 誰[だれ]も 足[あし]を 踏み入[ふみい]れたことがない。	
\\	彼はその踏切でひかれて死んだ。	
\\	彼[かれ]はその 踏切[ふみきり]でひかれて 死[し]んだ。	踏切=ふみきり= (鉄道の) 
\\	自殺を考えたが、どたん場で踏み止まった。	
\\	自殺[じさつ]を 考[かんが]えたが、どたん 場[じょう]で 踏み止[ふみとど]まった。	土壇場=どたんば= 
\\	踏み止まる=ふみとどまる= 
\\	あくまで踏み止まる決心だ。	
\\	あくまで 踏み止[ふみとど]まる 決心[けっしん]だ。	踏み止まる=ふみとどまる= 
\\	足を踏み外して岩場から転落した。	
\\	足[あし]を 踏み外[ふみはず]して 岩場[いわば]から 転落[てんらく]した。	踏み外す=ふみはずす= 
\\	一歩踏み外せばまさに命がないところだった。	
\\	一 歩[ほ] 踏み外[ふみはず]せばまさに 命[いのち]がないところだった。	踏み外す=ふみはずす= 
\\	自己管理がなっていないから遅刻ばかりするんだ。	
\\	自己[じこ] 管理[かんり]がなっていないから 遅刻[ちこく]ばかりするんだ。	
\\	失恋したからといって自己嫌悪することはない。	
\\	失恋[しつれん]したからといって 自己[じこ] 嫌悪[けんお]することはない。	
\\	自己顕示欲が強い人だ。	
\\	自己[じこ] 顕示[けんじ] 欲[よく]が 強[つよ]い 人[ひと]だ。	自己顕示=じこけんじ= 
\\	自己顕示欲= 
\\	彼は自己主張ばかりで周囲への配慮がない。	
\\	彼[かれ]は 自己[じこ] 主張[しゅちょう]ばかりで 周囲[しゅうい]への 配慮[はいりょ]がない。	
\\	株で得をするも損をするも全て自己責任だ。	
\\	株[かぶ]で 得[とく]をするも 損[そん]をするも 全[すべ]て 自己[じこ] 責任[せきにん]だ。	得=とく
\\	日本人は一般に自己宣伝が下手だ。	
\\	日本人[にほんじん]は 一般[いっぱん]に 自己[じこ] 宣伝[せんでん]が 下手[へた]だ。	
\\	講演者の話は自己宣伝ばかりでちっとも面白くなかった。	
\\	講演[こうえん] 者[しゃ]の 話[はなし]は 自己[じこ] 宣伝[せんでん]ばかりでちっとも 面白[おもしろ]くなかった。	
\\	最近自己中心の大人が目立って多い。	
\\	最近[さいきん] 自己[じこ] 中心[ちゅうしん]の 大人[おとな]が 目立[めだ]って 多[おお]い。	
\\	彼女は自己中心的になりやすい。	
\\	彼女[かのじょ]は 自己[じこ] 中心[ちゅうしん] 的[てき]になりやすい。	
\\	その送料は自己負担です。	
\\	その 送料[そうりょう]は 自己[じこ] 負担[ふたん]です。	
\\	彼は何事をするにも自己本位だ。	
\\	彼[かれ]は 何事[なにごと]をするにも 自己[じこ] 本位[ほんい]だ。	自己本位=じこほんい= 
\\	彼は自己流の弾き方でギターの腕前をここまで伸ばした。	
\\	彼[かれ]は 自己流[じこりゅう]の 弾[はじ]き 方[かた]でギターの 腕前[うでまえ]をここまで 伸[の]ばした。	腕前=うでまえ= 
\\	彼は何をするにも自己流だ。	
\\	彼[かれ]は 何[なに]をするにも 自己流[じこりゅう]だ。	
\\	よく猫は利己的な動物だと言われる。	
\\	よく 猫[ねこ]は 利己[りこ] 的[てき]な 動物[どうぶつ]だと 言[い]われる。	
\\	ネット犯罪を厳重に取り締まる必要がある。	
\\	ネット 犯罪[はんざい]を 厳重[げんじゅう]に 取り締[とりし]まる 必要[ひつよう]がある。	
\\	人々の貯蓄意欲が高まっている。	
\\	人々[ひとびと]の 貯蓄[ちょちく] 意欲[いよく]が 高[たか]まっている。	貯蓄=ちょちく= 
\\	彼らは私の言うことを一向に聞いてくれない。	
\\	彼[かれ]らは 私[わたし]の 言[い]うことを 一向[いっこう]に 聞[き]いてくれない。	一向に=いっこうに= (まったく) 
\\	(少しも) 
\\	あそこでたばこを吸うことは一向にかまわない。	
\\	あそこでたばこを 吸[す]うことは 一向[いっこう]にかまわない。	一向に=いっこうに= (まったく) 
\\	(少しも) 
\\	彼女は貯める一方で一向に使わない。	
\\	彼女[かのじょ]は 貯[た]める 一方[いっぽう]で 一向[いっこう]に 使[つか]わない。	一向に=いっこうに= (まったく) 
\\	(少しも) 
\\	これは父が一生かけて貯めた金だ。	
\\	これは 父[ちち]が 一生[いっしょう]かけて 貯[た]めた 金[きん]だ。	
\\	そういう人はついぞ見掛けたことがない。	
\\	そういう 人[ひと]はついぞ 見掛[みか]けたことがない。	ついぞ= 
\\	彼が今どこにいるのか何も手がかりがない。	
\\	彼[かれ]が 今[いま]どこにいるのか 何[なに]も 手[て]がかりがない。	手掛かり= 
\\	(犯人などの) 
\\	床に残された靴跡が犯人特定の手がかりとなった。	
\\	床[ゆか]に 残[のこ]された 靴[くつ] 跡[あと]が 犯人[はんにん] 特定[とくてい]の 手[て]がかりとなった。	靴跡=くつあと= 
\\	手掛かり= 
\\	(犯人などの) 
\\	喉に魚の小骨が引っかかった。	
\\	喉[のど]に 魚[さかな]の 小骨[こぼね]が 引[ひ]っかかった。	
\\	コートの袖が釘に引っかかった。	
\\	コートの 袖[そで]が 釘[くぎ]に 引[ひ]っかかった。	釘=くぎ= 
\\	一問目で引っかかって時間をだいぶ取られた。	
\\	一問[いちもん] 目[め]で 引[ひ]っかかって 時間[じかん]をだいぶ 取[と]られた。	
\\	思いがけず優勝しました。	
\\	思[おも]いがけず 優勝[ゆうしょう]しました。	
\\	旅行中に思いがけず昔の友達にばったり会った。	
\\	旅行[りょこう] 中[ちゅう]に 思[おも]いがけず 昔[むかし]の 友達[ともだち]にばったり 会[あ]った。	
\\	思いがけずおじさんが訪ねてきた。	
\\	思[おも]いがけずおじさんが 訪[たず]ねてきた。	
\\	原油高が世界経済を揺さぶっている。	
\\	原油[げんゆ] 高[だか]が 世界[せかい] 経済[けいざい]を 揺[ゆ]さぶっている。	揺さぶる=ゆさぶる= (揺する) 
\\	(動揺させる) 
\\	彼女の決意が揺らぐことはなかった。	
\\	彼女[かのじょ]の 決意[けつい]が 揺[ゆ]らぐことはなかった。	
\\	コートが釘にかかっている。	
\\	コートが 釘[くぎ]にかかっている。	
\\	やかんがストーブにかかっている。	
\\	やかんがストーブにかかっている。	
\\	どの本にもカバーがかかっている。	
\\	どの 本[ほん]にもカバーがかかっている。	
\\	空に雨雲がかかり始めた。	
\\	空[そら]に 雨雲[あまぐも]がかかり 始[はじ]めた。	
\\	雲が山の頂上にかかっている。	
\\	雲[くも]が 山[やま]の 頂上[ちょうじょう]にかかっている。	
\\	あの喫茶店ではいつもジャズがかかっていた。	
\\	あの 喫茶店[きっさてん]ではいつもジャズがかかっていた。	
\\	目覚まし時計は7時にかかっていた。	
\\	目覚[めざ]まし 時計[とけい]は 7時[しちじ]にかかっていた。	
\\	大きな魚が網にかかった。	
\\	大[おお]きな 魚[さかな]が 網[あみ]にかかった。	
\\	雨の後には魚がよくかかる。	
\\	雨[あめ]の 後[のち]には 魚[さかな]がよくかかる。	
\\	それはいくらかかったか?	
\\	それはいくらかかったか?	
\\	この子の教育にはずいぶん金がかかった。	
\\	この 子[こ]の 教育[きょういく]にはずいぶん 金[かね]がかかった。	
\\	歩くと駅までどのくらいかかりますか。	
\\	歩[ある]くと 駅[えき]までどのくらいかかりますか。	
\\	その仕事は2時までかかるでしょう。	
\\	その 仕事[しごと]は2 時[じ]までかかるでしょう。	
\\	世話のかかる老人を抱えています。	
\\	世話[せわ]のかかる 老人[ろうじん]を 抱[かか]えています。	
\\	まだその事業にはかかっていない。	
\\	まだその 事業[じぎょう]にはかかっていない。	
\\	嫌疑が被害者の隣人にかかった。	
\\	嫌疑[けんぎ]が 被害[ひがい] 者[しゃ]の 隣人[りんじん]にかかった。	嫌疑=けんぎ= 
\\	近所に迷惑がかからないよう気をつけた。	
\\	近所[きんじょ]に 迷惑[めいわく]がかからないよう 気[き]をつけた。	
\\	そろそろ昼休みにかかる時刻だ。	
\\	そろそろ 昼休[ひるやす]みにかかる 時刻[じこく]だ。	
\\	将来の成功は今の君の努力に懸かっている。	
\\	将来[しょうらい]の 成功[せいこう]は 今[いま]の 君[きみ]の 努力[どりょく]に 懸[か]かっている。	
\\	どこの病院にかかっていますか。	
\\	どこの 病院[びょういん]にかかっていますか。	
\\	離婚した妻のことがいつも心にかかっている。	
\\	離婚[りこん]した 妻[つま]のことがいつも 心[こころ]にかかっている。	
\\	空に虹が架かっている。	
\\	空[そら]に 虹[にじ]が 架[か]かっている。	架かる=かかる= 
\\	こちらの席におかけください。	
\\	こちらの 席[せき]におかけください。	
\\	息をかけて手を温めた。	
\\	息[いき]をかけて 手[て]を 温[あたた]めた。	
\\	(店員に)プレゼントなのでリボンをかけて下さい。	
\\	店員[てんいん]に)プレゼントなのでリボンをかけて 下[くだ]さい。	
\\	洗濯物を乾燥機にかけた。	
\\	洗濯[せんたく] 物[もの]を 乾燥[かんそう] 機[き]にかけた。	
\\	目覚まし時計を5時にかけた。	
\\	目覚[めざ]まし 時計[とけい]を5 時[じ]にかけた。	
\\	彼は新居の家具にだいぶ金をかけた。	
\\	彼[かれ]は 新居[しんきょ]の 家具[かぐ]にだいぶ 金[かね]をかけた。	
\\	二人は十分に時間をかけて話し合った。	
\\	二人[ふたり]は 十分[じゅうぶん]に 時間[じかん]をかけて 話し合[はなしあ]った。	
\\	君のことはいつも心にかけているよ。	
\\	君[きみ]のことはいつも 心[こころ]にかけているよ。	
\\	今年の受験で彼女は早稲田と慶応をかけている。	
\\	今年[ことし]の 受験[じゅけん]で 彼女[かのじょ]は 早稲田[わせだ]と 慶応[けいおう]をかけている。	
\\	この紙は油をよく吸い取る。	
\\	この 紙[かみ]は 油[あぶら]をよく 吸い取[すいと]る。	
\\	未成年の飲酒は非行を誘発する。	
\\	未成年[みせいねん]の 飲酒[いんしゅ]は 非行[ひこう]を 誘発[ゆうはつ]する。	誘発する= 
\\	その爆発が雪崩を誘発した。	
\\	その 爆発[ばくはつ]が 雪崩[なだれ]を 誘発[ゆうはつ]した。	雪崩=なだれ= 
\\	誘発する= 
\\	あの声はまさしく田中だ。	
\\	あの 声[こえ]はまさしく 田中[たなか]だ。	正しく=まさしく= (ちょうど) 
\\	(確かに) 
\\	(明らかに) 
\\	(真に) 
\\	日本の経済はまさしく危機に直面している。	
\\	日本[にほん]の 経済[けいざい]はまさしく 危機[きき]に 直面[ちょくめん]している。	正しく=まさしく= (ちょうど) 
\\	(確かに) 
\\	(明らかに) 
\\	(真に) 
\\	彼女に初めて出会ったのはまさしくこの場所だ。	
\\	彼女[かのじょ]に 初[はじ]めて 出会[であ]ったのはまさしくこの 場所[ばしょ]だ。	正しく=まさしく= (ちょうど) 
\\	(確かに) 
\\	(明らかに) 
\\	(真に) 
\\	私が大学を出たのは90年代初頭、時はまさしくバブルの絶頂期だった。	
\\	私[わたし]が 大学[だいがく]を 出[で]たのは90 年代[ねんだい] 初頭[しょとう]、 時[とき]はまさしくバブルの 絶頂[ぜっちょう] 期[き]だった。	正しく=まさしく= (ちょうど) 
\\	(確かに) 
\\	(明らかに) 
\\	(真に) 
\\	絶頂=ぜっちょう= (頂点) 
\\	次の文中の誤りを正せ。	
\\	次[つぎ]の 文中[ぶんちゅう]の 誤[あやま]りを 正[ただ]せ。	
\\	工事の進行は完全に滞っていた。	
\\	工事[こうじ]の 進行[しんこう]は 完全[かんぜん]に 滞[とどこお]っていた。	滞る=とどこおる= 
\\	仕事が滞っている。	
\\	仕事[しごと]が 滞[とどこお]っている。	滞る=とどこおる= 
\\	私は家賃を滞らせたことはない。	
\\	私[わたし]は 家賃[やちん]を 滞[とどこお]らせたことはない。	滞る=とどこおる= 
\\	彼女は支払いが滞っている。	
\\	彼女[かのじょ]は 支払[しはら]いが 滞[とどこお]っている。	滞る=とどこおる= 
\\	たいていの人はカメラを向けられると構えてしまう。	
\\	たいていの 人[ひと]はカメラを 向[む]けられると 構[かま]えてしまう。	構える=かまえる= 
\\	妹は横浜に新居を構えた。	
\\	妹[いもうと]は 横浜[よこはま]に 新居[しんきょ]を 構[かま]えた。	構える=かまえる= 
\\	この情勢では慎重に構えた方がよい。	
\\	この 情勢[じょうせい]では 慎重[しんちょう]に 構[かま]えた 方[ほう]がよい。	構える=かまえる= 
\\	彼女はどんな時も慌てずじっくり構えている。	
\\	彼女[かのじょ]はどんな 時[とき]も 慌[あわ]てずじっくり 構[かま]えている。	構える=かまえる= 
\\	兵士は私に向かって銃を構えていた。	
\\	兵士[へいし]は 私[わたし]に 向[む]かって 銃[じゅう]を 構[かま]えていた。	構える=かまえる= 
\\	彼は夫人を亡くしたばかりだ。	
\\	彼[かれ]は 夫人[ふじん]を 亡[な]くしたばかりだ。	夫人=ふじん=貴人の妻。また、他人の妻を敬っていう語。
\\	夫人は小学校の先生で、彼の高校時代の同級生だそうだ。	
\\	夫人[ふじん]は 小学校[しょうがっこう]の 先生[せんせい]で、 彼[かれ]の 高校[こうこう] 時代[じだい]の 同級生[どうきゅうせい]だそうだ。	夫人=ふじん=貴人の妻。また、他人の妻を敬っていう語。
\\	その寺は創建当時の姿に修復された。	
\\	その 寺[てら]は 創建[そうけん] 当時[とうじ]の 姿[すがた]に 修復[しゅうふく]された。	修復=しゅうふく= 
\\	これらの壁画は修復が必要だ。	
\\	これらの 壁画[へきが]は 修復[しゅうふく]が 必要[ひつよう]だ。	壁画=へきが 修復=しゅうふく= 
\\	この体操は激しい動作を伴う。	
\\	この 体操[たいそう]は 激[はげ]しい 動作[どうさ]を 伴[ともな]う。	動作=どうさ= (体の動き) 
\\	(振る舞い) 
\\	(コンピューターなどの作動) 
\\	あの男は動作が敏捷だ。	
\\	あの 男[おとこ]は 動作[どうさ]が 敏捷[びんしょう]だ。	動作=どうさ= (体の動き) 
\\	(振る舞い) 
\\	(コンピューターなどの作動) 
\\	敏捷=びんしょう= 
\\	エンジンが作動している。	
\\	エンジンが 作動[さどう]している。	作動=さどう= 
\\	機械や装置の運動部分が働くこと。
\\	店内では防犯カメラが常時作動している。	
\\	店内[てんない]では 防犯[ぼうはん]カメラが 常時[じょうじ] 作動[さどう]している。	作動=さどう= 
\\	機械や装置の運動部分が働くこと。
\\	このシャワーは水温の調節が利かない。	
\\	このシャワーは 水温[すいおん]の 調節[ちょうせつ]が 利[き]かない。	調節=ちょうせつ= 
\\	人は発汗によって体温の調節を行う。	
\\	人[ひと]は 発汗[はっかん]によって 体温[たいおん]の 調節[ちょうせつ]を 行[おこな]う。	発汗=はっかん= 
\\	調節=ちょうせつ= 
\\	彼は人生の節目を迎えている。	
\\	彼[かれ]は 人生[じんせい]の 節目[ふしめ]を 迎[むか]えている。	節目=ふしめ= 
\\	彼の話には多少疑わしい節がある。	
\\	彼[かれ]の 話[はなし]には 多少[たしょう] 疑[うたが]わしい 節[ふし]がある。	節=ふし= 
\\	(曲) 
\\	(箇所、点) 
\\	よくぞ言ってくれた。	
\\	よくぞ 言[い]ってくれた。	よくぞ= 
\\	メディアは政府によって抑圧を受けた。	
\\	メディアは 政府[せいふ]によって 抑圧[よくあつ]を 受[う]けた。	抑圧=よくあつ= 
\\	(精神医) 
\\	景気の回復とともに状況が好転してきた。	
\\	景気[けいき]の 回復[かいふく]とともに 状況[じょうきょう]が 好転[こうてん]してきた。	状況・情況=じょうきょう=移り変わる物事の、その時々のありさま。
\\	現場の状況がわかったら知らせてくれ。	
\\	現場[げんば]の 状況[じょうきょう]がわかったら 知[し]らせてくれ。	状況・情況=じょうきょう=移り変わる物事の、その時々のありさま。
\\	状況は刻々と変わった。	
\\	状況[じょうきょう]は 刻々[こっこく]と 変[か]わった。	状況・情況=じょうきょう=移り変わる物事の、その時々のありさま。
\\	結局マスコミの空騒ぎに終わった。	
\\	結局[けっきょく]マスコミの 空[そら] 騒[さわ]ぎに 終[お]わった。	空騒ぎ=からさわぎ= 
\\	全力を尽くしたが結果は今ひとつだった。	
\\	全力[ぜんりょく]を 尽[つ]くしたが 結果[けっか]は 今[いま]ひとつだった。	今ひとつ= (もう一つ) 
\\	(満足な状態に少し足りない) 
\\	君の文は説得力が今ひとつ足りない。	
\\	君[きみ]の 文[ぶん]は 説得[せっとく] 力[りょく]が 今[いま]ひとつ 足[た]りない。	今ひとつ= (もう一つ) 
\\	(満足な状態に少し足りない) 
\\	彼の言葉は今ひとつ信用できない。	
\\	彼[かれ]の 言葉[ことば]は 今[いま]ひとつ 信用[しんよう]できない。	今ひとつ= (もう一つ) 
\\	(満足な状態に少し足りない) 
\\	彼女の真意が今ひとつ理解できない。	
\\	彼女[かのじょ]の 真意[しんい]が 今[いま]ひとつ 理解[りかい]できない。	今ひとつ= (もう一つ) 
\\	(満足な状態に少し足りない) 
\\	ミカンを今ひとついかがですか。	
\\	ミカンを 今[いま]ひとついかがですか。	今ひとつ= (もう一つ) 
\\	(満足な状態に少し足りない) 
\\	つい野次馬根性で現場に行ってしまった。	
\\	つい 野次馬[やじうま] 根性[こんじょう]で 現場[げんば]に 行[い]ってしまった。	野次馬根性=やじうまこんじょう= 
\\	隣の奥さんが野次馬根性丸出しでこっちをのぞき込んでいた。	
\\	隣[となり]の 奥[おく]さんが 野次馬[やじうま] 根性[こんじょう] 丸出[まるだ]しでこっちをのぞき 込[こ]んでいた。	野次馬根性=やじうまこんじょう= 
\\	丸出し=まるだし= 
\\	覗き込む=のぞきこむ= 
\\	ビートルズ再結成といううわさは多くあったが、実現することはなかった。	
\\	ビートルズ 再[さい] 結成[けっせい]といううわさは 多[おお]くあったが、 実現[じつげん]することはなかった。	結成=けっせい= 
\\	私達は新しいバンドを結成した。	
\\	私[わたし] 達[たち]は 新[あたら]しいバンドを 結成[けっせい]した。	結成=けっせい= 
\\	この技術は他の追随を許さないものがある。	
\\	この 技術[ぎじゅつ]は 他[た]の 追随[ついずい]を 許[ゆる]さないものがある。	追随=ついずい= 
\\	この分野では我が社は他の追随を許さない。	
\\	この 分野[ぶんや]では 我[わ]が 社[しゃ]は 他[た]の 追随[ついずい]を 許[ゆる]さない。	追随=ついずい= 
\\	ロケットの発射は予定通り行われた。	
\\	ロケットの 発射[はっしゃ]は 予定[よてい] 通[どお]り 行[おこな]われた。	
\\	月は太陽の光を反射して輝く。	
\\	月[つき]は 太陽[たいよう]の 光[ひかり]を 反射[はんしゃ]して 輝[かがや]く。	反射=はんしゃ= 
\\	彼は反射神経が鈍い。	
\\	彼[かれ]は 反射[はんしゃ] 神経[しんけい]が 鈍[にぶ]い。	反射=はんしゃ= 
\\	彼の姿が見えた時、私は反射的に目をそらした。	
\\	彼[かれ]の 姿[すがた]が 見[み]えた 時[とき]、 私[わたし]は 反射[はんしゃ] 的[てき]に 目[め]をそらした。	反射=はんしゃ= 
\\	朝刊の大見出しが目を射た。	
\\	朝刊[ちょうかん]の 大[だい] 見出[みだ]しが 目[め]を 射[い]た。	射る=いる= (矢などを) 
\\	彼女は射るような眼差しで夫を見た。	
\\	彼女[かのじょ]は 射[い]るような 眼差[まなざ]しで 夫[おっと]を 見[み]た。	射る=いる= (矢などを) 
\\	眼差し=まなざし= 
\\	矢は見事に的の真ん中を射抜いた。	
\\	矢[や]は 見事[みごと]に 的[まと]の 真ん中[まんなか]を 射抜[いぬ]いた。	的=まと= (射撃などの標的) 
\\	(対象) 
\\	(焦点) 
\\	(中心) 
\\	(急所) 
\\	射抜く=いぬく= 
\\	当時の彼は全校女生徒の憧れの的だった。	
\\	当時[とうじ]の 彼[かれ]は 全校[ぜんこう] 女[じょ] 生徒[せいと]の 憧[あこが]れの 的[まと]だった。	的=まと= (射撃などの標的) 
\\	(対象) 
\\	(焦点) 
\\	(中心) 
\\	(急所) 
\\	君の意見は大いに的を射ている。	
\\	君[きみ]の 意見[いけん]は 大[おお]いに 的[まと]を 射[い]ている。	的=まと= (射撃などの標的) 
\\	(対象) 
\\	(焦点) 
\\	(中心) 
\\	(急所) 
\\	的を射る=まとをいる= 
\\	矢は的をはずれた。	
\\	矢[や]は 的[まと]をはずれた。	的=まと= (射撃などの標的) 
\\	(対象) 
\\	(焦点) 
\\	(中心) 
\\	(急所) 
\\	私は矢を的に当てた。	
\\	私[わたし]は 矢[や]を 的[まと]に 当[あ]てた。	的=まと= (射撃などの標的) 
\\	(対象) 
\\	(焦点) 
\\	(中心) 
\\	(急所) 
\\	その歌手は女性の憧れの的だ。	
\\	その 歌手[かしゅ]は 女性[じょせい]の 憧[あこが]れの 的[まと]だ。	的=まと= (射撃などの標的) 
\\	(対象) 
\\	(焦点) 
\\	(中心) 
\\	(急所) 
\\	彼はスーツのえりにバラの花を差していた。	
\\	彼[かれ]はスーツのえりにバラの 花[はな]を 差[さ]していた。	差す・射す=さす= 
\\	(色) 
\\	それを実行するのはちょっと気が差す。	
\\	それを 実行[じっこう]するのはちょっと 気[き]が 差[さ]す。	気が差す= 
\\	急に眠気が差してきた。	
\\	急[きゅう]に 眠気[ねむけ]が 差[さ]してきた。	差す・射す=さす= 
\\	(色) 
\\	潮が差している。	
\\	潮[しお]が 差[さ]している。	潮=しお= 
\\	差す・射す=さす= 
\\	(色) 
\\	心に不安の影が差した。	
\\	心[こころ]に 不安[ふあん]の 影[かげ]が 差[さ]した。	差す・射す=さす= 
\\	(色) 
\\	雲が切れて日がさしてきた。	
\\	雲[くも]が 切[き]れて 日[ひ]がさしてきた。	
\\	少女ははにかんで頬をほんのり染めた。	
\\	少女[しょうじょ]ははにかんで 頬[ほお]をほんのり 染[そ]めた。	はにかむ= 
\\	ほんのり= 
\\	染める=そめる= (染色) 
\\	(変色させる) 
\\	負傷者は衣装を血で染めていた。	
\\	負傷[ふしょう] 者[しゃ]は 衣装[いしょう]を 血[ち]で 染[そ]めていた。	染める=そめる= (染色) 
\\	(変色させる) 
\\	(赤くする) 
\\	昇る朝日が水平線を染めた。	
\\	昇[のぼ]る 朝日[あさひ]が 水平[すいへい] 線[せん]を 染[そ]めた。	染める=そめる= (染色) 
\\	(変色させる) 
\\	(赤くする) 
\\	どんな色に染めましょうか。	
\\	どんな 色[いろ]に 染[そ]めましょうか。	染める=そめる= (染色) 
\\	(変色させる) 
\\	(赤くする) 
\\	その病気はヒトには伝染しないといわれている。	
\\	その 病気[びょうき]はヒトには 伝染[でんせん]しないといわれている。	伝染=でんせん= 
\\	彼の病気は子供から伝染したものだ。	
\\	彼[かれ]の 病気[びょうき]は 子供[こども]から 伝染[でんせん]したものだ。	伝染=でんせん= 
\\	日焼けを放っておくとしみになる。	
\\	日焼[ひや]けを 放[ほう]っておくとしみになる。	染み=しみ= 
\\	近ごろめっきり顔に染みが増えた。	
\\	近[ちか]ごろめっきり 顔[かお]に 染[し]みが 増[ふ]えた。	めっきり= 
\\	染み=しみ= 
\\	晴れ着に染みをつけないよう気をつけるのよ。	
\\	晴れ着[はれぎ]に 染[し]みをつけないよう 気[き]をつけるのよ。	染み=しみ= 
\\	どうしても染みが抜けない。	
\\	どうしても 染[し]みが 抜[ぬ]けない。	染み=しみ= 
\\	天井には雨漏りの染みができていた。	
\\	天井[てんじょう]には 雨漏[あまも]りの 染[し]みができていた。	染み=しみ= 
\\	シャツの胸に何かしみがついているよ。	
\\	シャツの 胸[むね]に 何[なに]かしみがついているよ。	染み=しみ= 
\\	シャツにしみ込んだ油汚れがどうしても取れない。	
\\	シャツにしみ 込[こ]んだ 油[あぶら] 汚[よご]れがどうしても 取[と]れない。	染み込む・沁み込む・浸み込む=しみこむ= 
\\	記念館で見た写真は今も私の胸に深くしみ込んでいる。	
\\	記念[きねん] 館[かん]で 見[み]た 写真[しゃしん]は 今[いま]も 私[わたし]の 胸[むね]に 深[ふか]くしみ 込[こ]んでいる。	染み込む・沁み込む・浸み込む=しみこむ= 
\\	彼には人種的偏見が深くしみ込んでいた。	
\\	彼[かれ]には 人種[じんしゅ] 的[てき] 偏見[へんけん]が 深[ふか]くしみ 込[こ]んでいた。	染み込む・沁み込む・浸み込む=しみこむ= 
\\	政府は新たな産業育成政策を打ち出した。	
\\	政府[せいふ]は 新[あら]たな 産業[さんぎょう] 育成[いくせい] 政策[せいさく]を 打ち出[うちだ]した。	育成=いくせい= 
\\	(選手や人材の) 
\\	この会社は若手社員の育成に力を入れている。	
\\	この 会社[かいしゃ]は 若手[わかて] 社員[しゃいん]の 育成[いくせい]に 力[ちから]を 入[い]れている。	育成=いくせい= 
\\	(選手や人材の) 
\\	彼女の大胆なドレスには目のやり場に困った。	
\\	彼女[かのじょ]の 大胆[だいたん]なドレスには 目[め]のやり 場[ば]に 困[こま]った。	大胆(な)=だいたん= (恐れを知らない) 
\\	やり場= 
\\	原稿無しの講演とは大胆だ。	
\\	原稿[げんこう] 無[な]しの 講演[こうえん]とは 大胆[だいたん]だ。	大胆(な)=だいたん= (恐れを知らない) 
\\	彼女はアルコールが入ると大胆になる。	
\\	彼女[かのじょ]はアルコールが 入[はい]ると 大胆[だいたん]になる。	大胆(な)=だいたん= (恐れを知らない) 
\\	自分に厳しく、他人に寛大でありなさい。	
\\	自分[じぶん]に 厳[きび]しく、 他人[たにん]に 寛大[かんだい]でありなさい。	寛大(な)=かんだい= 
\\	窓ガラスを割った僕に対して、隣のおじさんはすこぶる寛大であった。	
\\	窓[まど]ガラスを 割[わ]った 僕[ぼく]に 対[たい]して、 隣[となり]のおじさんはすこぶる 寛大[かんだい]であった。	寛大(な)=かんだい= 
\\	すこぶる= 非常に
\\	その紛争は古くからの宗教対立に根差している。	
\\	その 紛争[ふんそう]は 古[ふる]くからの 宗教[しゅうきょう] 対立[たいりつ]に 根差[ねざ]している。	根差す=ねざす= 
\\	環境保護の考えが一般の人の心にも根差し始めている。	
\\	環境[かんきょう] 保護[ほご]の 考[かんが]えが 一般[いっぱん]の 人[ひと]の 心[こころ]にも 根差[ねざ]し 始[はじ]めている。	根差す=ねざす= 
\\	彼は彼女の美しさにとらえられた。	
\\	彼[かれ]は 彼女[かのじょ]の 美[うつく]しさにとらえられた。	捉える・捕らえる=とらえる= (捕まる) 
\\	犯人はまだとらえられていない。	
\\	犯人[はんにん]はまだとらえられていない。	捉える・捕らえる=とらえる= (捕まる) 
\\	兵士たちは敵にとらえられた。	
\\	兵士[へいし]たちは 敵[てき]にとらえられた。	捉える・捕らえる=とらえる= (捕まる) 
\\	そのことで彼らの間に間隙が生じた。	
\\	そのことで 彼[かれ]らの 間[あいだ]に 間隙[かんげき]が 生[しょう]じた。	間隙=かんげき= (すきま) 
\\	さまざまな情報が錯綜していて事件の真相がつかめない。	
\\	さまざまな 情報[じょうほう]が 錯綜[さくそう]していて 事件[じけん]の 真相[しんそう]がつかめない。	錯綜=さくそう= 
\\	錯綜する= 
\\	真相=しんそう= 
\\	その小説は人間関係が錯綜していてわかりにくいところがある。	
\\	その 小説[しょうせつ]は 人間[にんげん] 関係[かんけい]が 錯綜[さくそう]していてわかりにくいところがある。	錯綜=さくそう= 
\\	錯綜する= 
\\	あの人に限ってそんなことはしない。	
\\	あの 人[ひと]に 限[かぎ]ってそんなことはしない。	
\\	彼が休みを取っている日に限って厄介な問題が起こるんだ。	
\\	彼[かれ]が 休[やす]みを 取[と]っている 日[ひ]に 限[かぎ]って 厄介[やっかい]な 問題[もんだい]が 起[お]こるんだ。	
\\	うちの息子に限ってそんな悪さをするなんて考えられない。	
\\	うちの 息子[むすこ]に 限[かぎ]ってそんな 悪[わる]さをするなんて 考[かんが]えられない。	
\\	他民族の文化をいたずらに野蛮視すべきではない。	
\\	他[た] 民族[みんぞく]の 文化[ぶんか]をいたずらに 野蛮[やばん] 視[し]すべきではない。	野蛮=やばん= 
\\	戦争ほど野蛮な行為はない。	
\\	戦争[せんそう]ほど 野蛮[やばん]な 行為[こうい]はない。	野蛮=やばん= 
\\	この食文化は他国の目には野蛮に映るらしい。	
\\	この 食[しょく] 文化[ぶんか]は 他国[たこく]の 目[め]には 野蛮[やばん]に 映[うつ]るらしい。	野蛮=やばん= 
\\	その演奏に魅了された。	
\\	その 演奏[えんそう]に 魅了[みりょう]された。	
\\	彼は突然涙声になった。	
\\	彼[かれ]は 突然[とつぜん] 涙声[なみだごえ]になった。	
\\	空に稲妻が走った。	
\\	空[そら]に 稲妻[いなづま]が 走[はし]った。	稲妻=いなずま= 
\\	名案が稲妻のようにひらめいた。	
\\	名案[めいあん]が 稲妻[いなづま]のようにひらめいた。	名案=めいあん= 
\\	稲妻=いなずま= 
\\	ひらめく= 
\\	法の尊重が我々の社会の基本だ。	
\\	法[ほう]の 尊重[そんちょう]が 我々[われわれ]の 社会[しゃかい]の 基本[きほん]だ。	
\\	両親は私の意志を尊重して音楽の道に進ませてくれました。	
\\	両親[りょうしん]は 私[わたし]の 意志[いし]を 尊重[そんちょう]して 音楽[おんがく]の 道[みち]に 進[すす]ませてくれました。	
\\	このような行為は人間の尊厳を犯すものだ。	
\\	このような 行為[こうい]は 人間[にんげん]の 尊厳[そんげん]を 犯[おか]すものだ。	
\\	彼女は自尊心が強い。	
\\	彼女[かのじょ]は 自尊心[じそんしん]が 強[つよ]い。	
\\	そのテロ攻撃で多くの貴い命が奪われた。	
\\	そのテロ 攻撃[こうげき]で 多[おお]くの 貴[とうと]い 命[いのち]が 奪[うば]われた。	尊い・貴い=とうとい= 
\\	人の命ほど貴いものはない。	
\\	人[ひと]の 命[いのち]ほど 貴[とうと]いものはない。	尊い・貴い=とうとい= 
\\	テレビのニュースは台風が接近していることを報じた。	
\\	テレビのニュースは 台風[たいふう]が 接近[せっきん]していることを 報[ほう]じた。	
\\	彼は人から批判されて屈服するような男ではない。	
\\	彼[かれ]は 人[ひと]から 批判[ひはん]されて 屈服[くっぷく]するような 男[おとこ]ではない。	屈服=くっぷく= 
\\	テロに屈服することがあってはならない。	
\\	テロに 屈服[くっぷく]することがあってはならない。	屈服=くっぷく= 
\\	彼は老齢を理由に雇用を拒否された。	
\\	彼[かれ]は 老齢[ろうれい]を 理由[りゆう]に 雇用[こよう]を 拒否[きょひ]された。	
\\	彼は老齢で腰が曲がっている。	
\\	彼[かれ]は 老齢[ろうれい]で 腰[こし]が 曲[ま]がっている。	
\\	今日の彼女は化粧がすごく厚い。	
\\	今日[きょう]の 彼女[かのじょ]は 化粧[けしょう]がすごく 厚[あつ]い。	
\\	あの女優はファンの年齢層が厚い。	
\\	あの 女優[じょゆう]はファンの 年齢[ねんれい] 層[そう]が 厚[あつ]い。	
\\	職を失った友人がわが家で居候生活を始めた。	
\\	職[しょく]を 失[うしな]った 友人[ゆうじん]がわが 家[や]で 居候[いそうろう] 生活[せいかつ]を 始[はじ]めた。	居候=居候= (状態) 
\\	(人) 
\\	彼女は30を超えているのに、まだ家族に居候している。	
\\	彼女[かのじょ]は30を 超[こ]えているのに、まだ 家族[かぞく]に 居候[いそうろう]している。	居候=居候= (状態) 
\\	(人) 
\\	なんでそんなにもめてるのだ。	
\\	なんでそんなにもめてるのだ。	揉める=もめる= (ごたごたが起こる) 
\\	雇い主と従業員との間がもめた。	
\\	雇い主[やといぬし]と 従業[じゅうぎょう] 員[いん]との 間[あいだ]がもめた。	揉める=もめる= (ごたごたが起こる) 
\\	兄にならってジョギングを始めた。	
\\	兄[あに]にならってジョギングを 始[はじ]めた。	倣う=ならう= 
\\	他の大都市もこれに倣っている。	
\\	他[た]の 大都市[だいとし]もこれに 倣[なら]っている。	倣う=ならう= 
\\	私にはもう策の施しようがない。	
\\	私[わたし]にはもう 策[さく]の 施[ほどこ]しようがない。	策を施す=さくをほどこす= 
\\	捜査が難航している。	
\\	捜査[そうさ]が 難航[なんこう]している。	
\\	警察では犯人を捜索中だ。	
\\	警察[けいさつ]では 犯人[はんにん]を 捜索[そうさく] 中[ちゅう]だ。	
\\	日が暮れた。	
\\	日[ひ]が 暮[く]れた。	
\\	6時ごろ日が暮れます。	
\\	時[じ]ごろ 日[び]が 暮[く]れます。	
\\	秋は日の暮れるのが速い。	
\\	秋[あき]は 日[ひ]の 暮[く]れるのが 速[はや]い。	
\\	今年も暮れた。	
\\	今年[ことし]も 暮[く]れた。	
\\	秋が暮れて冬になった。	
\\	秋[あき]が 暮[く]れて 冬[ふゆ]になった。	
\\	その頃の私は毎日研究に明け暮れていた。	
\\	その 頃[ころ]の 私[わたし]は 毎日[まいにち] 研究[けんきゅう]に 明け暮[あけく]れていた。	明け暮れる= 
\\	大学時代はアルバイトに明け暮れていた。	
\\	大学[だいがく] 時代[じだい]はアルバイトに 明け暮[あけく]れていた。	明け暮れる= 
\\	私を盾にする気?	
\\	私[わたし]を 盾[たて]にする 気[き]?	盾・楯=たて= 
\\	それは盾の半面にすぎない。	
\\	それは 盾[たて]の 半面[はんめん]にすぎない。	盾・楯=たて= 
\\	ものには盾の両面がある。	
\\	ものには 盾[たて]の 両面[りょうめん]がある。	盾・楯=たて= 
\\	この仕事は気が進まない。	
\\	この 仕事[しごと]は 気[き]が 進[すす]まない。	進む= 
\\	(昇進する) 
\\	日本はインドネシアより2時間進んでいる。	
\\	日本[にっぽん]はインドネシアより2 時間[じかん] 進[すす]んでいる。	進む= 
\\	(昇進する) 
\\	漬け物があるとご飯が進む。	
\\	漬[つ]け 物[もの]があるとご 飯[はん]が 進[すす]む。	進む= 
\\	(昇進する) 
\\	こう暑いと食が進まない。	
\\	こう 暑[あつ]いと 食[しょく]が 進[すす]まない。	進む= 
\\	(昇進する) 
\\	犯罪の低年齢化が進んでいる。	
\\	犯罪[はんざい]の 低[てい] 年齢[ねんれい] 化[か]が 進[すす]んでいる。	進む= 
\\	(昇進する) 
\\	先週の授業は第4課まで進んだ。	
\\	先週[せんしゅう]の 授業[じゅぎょう]は 第[だい]4 課[か]まで 進[すす]んだ。	進む= 
\\	(昇進する) 
\\	科学技術は日々進んでいる。	
\\	科学[かがく] 技術[ぎじゅつ]は 日々[ひび] 進[すす]んでいる。	進む= 
\\	(昇進する) 
\\	彼は環境問題について進んだ考えを持っている。	
\\	彼[かれ]は 環境[かんきょう] 問題[もんだい]について 進[すす]んだ 考[かんが]えを 持[も]っている。	進む= 
\\	(昇進する) 
\\	彼は一躍最高の地位に昇った。	
\\	彼[かれ]は 一躍[いちやく] 最高[さいこう]の 地位[ちい]に 昇[のぼ]った。	一躍=いちやく= 
\\	彼はいわゆる団塊の世代に属する。	
\\	彼[かれ]はいわゆる 団塊[だんかい]の 世代[せだい]に 属[ぞく]する。	所謂=いわゆる= 
\\	団塊の世代=だんかい の せだい= 
\\	彼女はいわゆる教養人だ。	
\\	彼女[かのじょ]はいわゆる 教養[きょうよう] 人[じん]だ。	所謂=いわゆる= 
\\	犯人が自首して事件は決着した。	
\\	犯人[はんにん]が 自首[じしゅ]して 事件[じけん]は 決着[けっちゃく]した。	決着=けっちゃく= 
\\	(解決) 
\\	討論はようやく決着を見た。	
\\	討論[とうろん]はようやく 決着[けっちゃく]を 見[み]た。	決着=けっちゃく= 
\\	(解決) 
\\	100メートル走のタイムを計ってくれませんか。	
\\	100メートル 走[そう]のタイムを 計[はか]ってくれませんか。	測る・計る・量る=はかる= 
\\	デパートへ行くついでがある。	
\\	デパートへ 行[い]くついでがある。	序で=ついで= 
\\	ついでがあったらこれを彼に渡しておいてください。	
\\	ついでがあったらこれを 彼[かれ]に 渡[わた]しておいてください。	序で=ついで= 
\\	話のついでに別の秘密も教えてあげよう。	
\\	話[はなし]のついでに 別[べつ]の 秘密[ひみつ]も 教[おし]えてあげよう。	序でに=ついでに= 
\\	翻訳を読んだのだから、ついでに原文にも挑戦しなさい。	
\\	翻訳[ほんやく]を 読[よ]んだのだから、ついでに 原文[げんぶん]にも 挑戦[ちょうせん]しなさい。	序でに=ついでに= 
\\	東京へ来たら、ついでにわが家にも寄ってください。	
\\	東京[とうきょう]へ 来[き]たら、ついでにわが 家[や]にも 寄[よ]ってください。	序でに=ついでに= 
\\	郵便局に行くついでに切手を買ってきてくれ。	
\\	郵便[ゆうびん] 局[きょく]に 行[い]くついでに 切手[きって]を 買[か]ってきてくれ。	序でに=ついでに= 
\\	立っているついでに冷蔵庫からビールを持ってきてくれ。	
\\	立[た]っているついでに 冷蔵庫[れいぞうこ]からビールを 持[も]ってきてくれ。	序でに=ついでに= 
\\	ここ数年で学内の規律がかなり緩んだようだ。	
\\	ここ 数[すう] 年[ねん]で 学内[がくない]の 規律[きりつ]がかなり 緩[ゆる]んだようだ。	緩む=ゆるむ= (締まっていたものが) 
\\	(気持ち・表情などが) 
\\	結び目が緩んだ。	
\\	結び目[むすびめ]が 緩[ゆる]んだ。	緩む=ゆるむ= (締まっていたものが) 
\\	(気持ち・表情などが) 
\\	機械のボルトが緩んだ。	
\\	機械[きかい]のボルトが 緩[ゆる]んだ。	緩む=ゆるむ= (締まっていたものが) 
\\	(気持ち・表情などが) 
\\	脚の痛みを鎮める薬か何かありませんか。	
\\	脚[あし]の 痛[いた]みを 鎮[しず]める 薬[くすり]か 何[なに]かありませんか。	静める・鎮める=しずめる= [静める] 
\\	[鎮める] 
\\	人生には浮き沈みがあるものだ。	
\\	人生[じんせい]には 浮き沈[うきしず]みがあるものだ。	浮き沈み=うきしずみ= (上下運動) 
\\	あの人は感情の浮き沈みが激しい。	
\\	あの 人[ひと]は 感情[かんじょう]の 浮き沈[うきしず]みが 激[はげ]しい。	浮き沈み=うきしずみ= (上下運動) 
\\	彼女は当時作家として大いに鳴らしていた。	
\\	彼女[かのじょ]は 当時[とうじ] 作家[さっか]として 大[おお]いに 鳴[な]らしていた。	
\\	それは自ら明らかなことだ。	
\\	それは 自[おのずか]ら 明[あき]らかなことだ。	自ら=おのずから= 
\\	それは自らの首を絞めるようなものだ。	
\\	それは 自[みずか]らの 首[くび]を 絞[し]めるようなものだ。	自ら=みずから絞める=しめる= 
\\	これが発表されれば世界中が大騒ぎになるだろう。	
\\	これが 発表[はっぴょう]されれば 世界中[せかいじゅう]が 大騒[おおさわ]ぎになるだろう。	
\\	彼らは大騒ぎをして僕らを迎えてくれた。	
\\	彼[かれ]らは 大騒[おおさわ]ぎをして 僕[ぼく]らを 迎[むか]えてくれた。	
\\	私の家は子供が5人もいてとても騒がしい。	
\\	私[わたし]の 家[いえ]は 子供[こども]が5 人[にん]もいてとても 騒[さわ]がしい。	騒がしい=さわがしい=1)盛んに声や物音がしてうるさい。騒々しい。            
\\	事件などが起こって世情が落ち着かない。平静。平穏でない。            
\\	ことが多く忙しい。あわただしい。            
\\	ごたごたしている。乱雑である。
\\	彼らの演奏は騒がしいだけで音楽になってない。	
\\	彼[かれ]らの 演奏[えんそう]は 騒[さわ]がしいだけで 音楽[おんがく]になってない。	騒がしい=さわがしい=1)盛んに声や物音がしてうるさい。騒々しい。            
\\	事件などが起こって世情が落ち着かない。平静。平穏でない。            
\\	ことが多く忙しい。あわただしい。            
\\	ごたごたしている。乱雑である。
\\	そのスターが離婚するのではないかとマスコミが騒がしい。	
\\	そのスターが 離婚[りこん]するのではないかとマスコミが 騒[さわ]がしい。	騒がしい=さわがしい=1)盛んに声や物音がしてうるさい。騒々しい。            
\\	事件などが起こって世情が落ち着かない。平静。平穏でない。            
\\	ことが多く忙しい。あわただしい。            
\\	ごたごたしている。乱雑である。
\\	中学生の一団が騒がしく列車に乗り込んできた。	
\\	中学生[ちゅうがくせい]の 一団[いちだん]が 騒[さわ]がしく 列車[れっしゃ]に 乗り込[のりこ]んできた。	騒がしい=さわがしい=1)盛んに声や物音がしてうるさい。騒々しい。            
\\	事件などが起こって世情が落ち着かない。平静。平穏でない。            
\\	ことが多く忙しい。あわただしい。            
\\	ごたごたしている。乱雑である。
\\	図書館で騒がしくしていて係の人に叱られた。	
\\	図書館[としょかん]で 騒[さわ]がしくしていて 係[かかり]の 人[ひと]に 叱[しか]られた。	騒がしい=さわがしい=1)盛んに声や物音がしてうるさい。騒々しい。            
\\	事件などが起こって世情が落ち着かない。平静。平穏でない。            
\\	ことが多く忙しい。あわただしい。            
\\	ごたごたしている。乱雑である。
\\	彼女が騒ぎ立てたので泥棒は何も取らずに逃げた。	
\\	彼女[かのじょ]が 騒ぎ立[さわぎた]てたので 泥棒[どろぼう]は 何[なに]も 取[と]らずに 逃[に]げた。	騒ぎ立てる= 
\\	彼は生活に疲れ果てているようだ。	
\\	彼[かれ]は 生活[せいかつ]に 疲[つか]れ 果[は]てているようだ。	
\\	彼女は眠れない夜が続いて疲れ切っている。	
\\	彼女[かのじょ]は 眠[ねむ]れない 夜[よる]が 続[つづ]いて 疲[つか]れ 切[き]っている。	
\\	訴訟を起こせば多少の出費は免れない。	
\\	訴訟[そしょう]を 起[お]こせば 多少[たしょう]の 出費[しゅっぴ]は 免[まぬか]れない。	免れる=まぬかれる・まぬがれる= 
\\	彼女の冷静な判断のおかげで我々は事故を免れた。	
\\	彼女[かのじょ]の 冷静[れいせい]な 判断[はんだん]のおかげで 我々[われわれ]は 事故[じこ]を 免[まぬか]れた。	免れる=まぬかれる・まぬがれる= 
\\	政府はこの事件の責任を免れない。	
\\	政府[せいふ]はこの 事件[じけん]の 責任[せきにん]を 免[まぬか]れない。	免れる=まぬかれる・まぬがれる= 
\\	一度この病気にかかると後は免疫になる。	
\\	一度[いちど]この 病気[びょうき]にかかると 後[のち]は 免疫[めんえき]になる。	免疫=めんえき= 
\\	このワクチンによる免疫はわずかに5年しか続かない。	
\\	このワクチンによる 免疫[めんえき]はわずかに5 年[ねん]しか 続[つづ]かない。	免疫=めんえき= 
\\	いきなり犬が腕に噛み付いてきた。	
\\	いきなり 犬[いぬ]が 腕[うで]に 噛み付[かみつ]いてきた。	
\\	腕を挙げると肩が痛む。	
\\	腕[うで]を 挙[あ]げると 肩[かた]が 痛[いた]む。	
\\	二人はしっかりと腕を絡み合わせて歩いていた。	
\\	二人[ふたり]はしっかりと 腕[うで]を 絡み合[からみあ]わせて 歩[ある]いていた。	
\\	あの人はいわゆる腕一本で叩き上げた人です。	
\\	あの 人[ひと]はいわゆる 腕[うで]一 本[ほん]で 叩き上[たたきあ]げた 人[ひと]です。	
\\	明日の試合を考えると腕が鳴るよ。	
\\	明日[あした]の 試合[しあい]を 考[かんが]えると 腕[うで]が 鳴[な]るよ。	腕が鳴る= 
\\	写真を一枚見れば、彼のカメラの腕がわかる。	
\\	写真[しゃしん]を 一枚[いちまい] 見[み]れば、 彼[かれ]のカメラの 腕[うで]がわかる。	
\\	彼は腕が立つから会社に引き留めたい。	
\\	彼[かれ]は 腕[うで]が 立[た]つから 会社[かいしゃ]に 引き留[ひきと]めたい。	
\\	ここが腕の見せどころだ。	
\\	ここが 腕[うで]の 見[み]せどころだ。	
\\	彼のゴルフの腕はたいしたことない。	
\\	彼[かれ]のゴルフの 腕[うで]はたいしたことない。	
\\	その問題は厚生労働省の縄張りだ。	
\\	その 問題[もんだい]は 厚生[こうせい] 労働省[ろうどうしょう]の 縄張[なわば]りだ。	縄張り=なわばり= 
\\	新宿は俺の縄張りだ。	
\\	新宿[しんじゅく]は 俺[おれ]の 縄張[なわば]りだ。	縄張り=なわばり= 
\\	俺の縄張りから今すぐ出て行け。	
\\	俺[おれ]の 縄張[なわば]りから 今[いま]すぐ 出[で]て 行[い]け。	縄張り=なわばり= 
\\	風が炎を煽った。	
\\	風[かぜ]が 炎[ほのお]を 煽[あお]った。	炎・焔=ほのお= 
\\	(激しい感情);
\\	落雷で寺が炎上した。	
\\	落雷[らくらい]で 寺[てら]が 炎上[えんじょう]した。	炎上=えんじょう= 
\\	帽子をかぶらず炎天を歩いた。	
\\	帽子[ぼうし]をかぶらず 炎天[えんてん]を 歩[ある]いた。	炎天=えんてん= 
\\	小学生は調査対象から除外した。	
\\	小学生[しょうがくせい]は 調査[ちょうさ] 対象[たいしょう]から 除外[じょがい]した。	
\\	息子が就職して私も重荷が下りだ。	
\\	息子[むすこ]が 就職[しゅうしょく]して 私[わたし]も 重荷[おもに]が 下[くだ]りだ。	重荷=おもに= (荷物) 
\\	(負担) 
\\	夫の看病は大変な重荷になっている。	
\\	夫[おっと]の 看病[かんびょう]は 大変[たいへん]な 重荷[おもに]になっている。	看病=かんびょう= 
\\	重荷=おもに= (荷物) 
\\	(負担) 
\\	そのとき初めて私は家族の存在を重荷に感じた。	
\\	そのとき 初[はじ]めて 私[わたし]は 家族[かぞく]の 存在[そんざい]を 重荷[おもに]に 感[かん]じた。	重荷=おもに= (荷物) 
\\	(負担) 
\\	母親は息子の重荷になりたくなかった。	
\\	母親[ははおや]は 息子[むすこ]の 重荷[おもに]になりたくなかった。	重荷=おもに= (荷物) 
\\	(負担) 
\\	昨夜小島氏宅から出火した。	
\\	昨夜[さくや] 小島[こじま] 氏[し] 宅[たく]から 出火[しゅっか]した。	
\\	私からの寿を受けください。	
\\	私[わたし]からの 寿[ことぶき]を 受[う]けください。	寿=ことぶき= 
\\	この分譲マンションは収納場所が少ない。	
\\	この 分譲[ぶんじょう]マンションは 収納[しゅうのう] 場所[ばしょ]が 少[すく]ない。	分譲マンション=ぶんじょう〜= 
\\	君は音楽の才能がある。	
\\	君[きみ]は 音楽[おんがく]の 才能[さいのう]がある。	
\\	彼女には商売の才能がある。	
\\	彼女[かのじょ]には 商売[しょうばい]の 才能[さいのう]がある。	
\\	彼には芸術的才能が突如花開いた。	
\\	彼[かれ]には 芸術[げいじゅつ] 的[てき] 才能[さいのう]が 突如[とつじょ] 花開[はなひら]いた。	突如(として)=とつじょ= 突然花開く=はなひらく= 
\\	私は語学の才能に乏しい。	
\\	私[わたし]は 語学[ごがく]の 才能[さいのう]に 乏[とぼ]しい。	
\\	彼は息子に数学の才能を見出した。	
\\	彼[かれ]は 息子[むすこ]に 数学[すうがく]の 才能[さいのう]を 見出[みだ]した。	見出す=みだす・みいだす
\\	彼女は絵の才能を生かして画家になった。	
\\	彼女[かのじょ]は 絵[え]の 才能[さいのう]を 生[い]かして 画家[がか]になった。	
\\	そのテレビ番組は日本語で吹き替えになっている。	
\\	そのテレビ 番組[ばんぐみ]は 日本語[にほんご]で 吹き替[ふきか]えになっている。	吹き替え=ふきかえ= 
\\	その晩の夢に、死んだ母が現れた。	
\\	その 晩[ばん]の 夢[ゆめ]に、 死[し]んだ 母[はは]が 現[あらわ]れた。	
\\	人類は数百万年前に地球上に現れた。	
\\	人類[じんるい]は 数[すう] 百[ひゃく] 万[まん] 年[ねん] 前[まえ]に 地球[ちきゅう] 上[じょう]に 現[あらわ]れた。	
\\	男性社員にも教育休暇を認める会社が現れてきた。	
\\	男性[だんせい] 社員[しゃいん]にも 教育[きょういく] 休暇[きゅうか]を 認[みと]める 会社[かいしゃ]が 現[あらわ]れてきた。	
\\	酒を飲むと本性が現れる。	
\\	酒[さけ]を 飲[の]むと 本性[ほんしょう]が 現[あらわ]れる。	
\\	驚きの色が彼の表情に現れた。	
\\	驚[おどろ]きの 色[いろ]が 彼[かれ]の 表情[ひょうじょう]に 現[あらわ]れた。	
\\	彼女にはその薬の効き目がなかった。	
\\	彼女[かのじょ]にはその 薬[くすり]の 効き目[ききめ]がなかった。	効き目=ききめ= 
\\	薬はすぐ効き目があった。	
\\	薬[くすり]はすぐ 効き目[ききめ]があった。	効き目=ききめ= 
\\	薬の効き目が徐々に現れてきた。	
\\	薬[くすり]の 効き目[ききめ]が 徐々[じょじょ]に 現[あらわ]れてきた。	効き目=ききめ= 
\\	秘密が顕れた。	
\\	秘密[ひみつ]が 顕[あらわ]れた。	顕れる=現れる
\\	計画は順調に滑り出した。	
\\	計画[けいかく]は 順調[じゅんちょう]に 滑り出[すべりだ]した。	滑り出す=すべりだす= (滑り始める) 
\\	(発足する) 
\\	発車間際の電車に滑り込んだ。	
\\	発車[はっしゃ] 間際[まぎわ]の 電車[でんしゃ]に 滑り込[すべりこ]んだ。	滑り込む=すべりこむ= 
\\	たばこの火でカーペットを焦がした。	
\\	たばこの 火[ひ]でカーペットを 焦[こ]がした。	
\\	焦がさないように注意して焼きなさい。	
\\	焦[こ]がさないように 注意[ちゅうい]して 焼[や]きなさい。	
\\	ご飯が焦げ付いた。	
\\	ご 飯[はん]が 焦げ付[こげつ]いた。	
\\	アメリカは全てが同居する地球の縮図だ。	
\\	アメリカは 全[すべ]てが 同居[どうきょ]する 地球[ちきゅう]の 縮図[しゅくず]だ。	
\\	インターネットはまさに社会の縮図だ。	
\\	インターネットはまさに 社会[しゃかい]の 縮図[しゅくず]だ。	
\\	自分の優柔不断を後悔した。	
\\	自分[じぶん]の 優柔不断[ゆうじゅうふだん]を 後悔[こうかい]した。	優柔不断=ゆうじゅうふだん= 
\\	そういう甘い話には必ず落とし穴がある。	
\\	そういう 甘[あま]い 話[はなし]には 必[かなら]ず 落とし穴[おとしあな]がある。	落とし穴=おとしあな= 
\\	コンピューターの過信は現代社会の危険な落とし穴である。	
\\	コンピューターの 過信[かしん]は 現代[げんだい] 社会[しゃかい]の 危険[きけん]な 落とし穴[おとしあな]である。	落とし穴=おとしあな= 
\\	今日は面倒をかけてしまったので、この穴埋めに今度何かごちそうするよ。	
\\	今日[きょう]は 面倒[めんどう]をかけてしまったので、この 穴埋[あなう]めに 今度[こんど] 何[なに]かごちそうするよ。	穴埋め=あなうめ= 
\\	彼は働きすぎて命を縮めた。	
\\	彼[かれ]は 働[はたら]きすぎて 命[いのち]を 縮[ちぢ]めた。	
\\	その報告書を5ページに縮めてください。	
\\	その 報告[ほうこく] 書[しょ]を5ページに 縮[ちぢ]めてください。	
\\	洗濯したらソックスが縮んだ。	
\\	洗濯[せんたく]したらソックスが 縮[ちぢ]んだ。	
\\	これは洗っても縮まない。	
\\	これは 洗[あら]っても 縮[ちぢ]まない。	
\\	年をとったら身長が縮んでしまった。	
\\	年[とし]をとったら 身長[しんちょう]が 縮[ちぢ]んでしまった。	
\\	彼は彼女の表情の変化をじっと観察していた。	
\\	彼[かれ]は 彼女[かのじょ]の 表情[ひょうじょう]の 変化[へんか]をじっと 観察[かんさつ]していた。	
\\	彼女は観察が鋭い。	
\\	彼女[かのじょ]は 観察[かんさつ]が 鋭[するど]い。	
\\	「どうしたの」と声をかけても、その子は泣きじゃくるばかりだった。	
\\	「どうしたの」と 声[こえ]をかけても、その 子[こ]は 泣[な]きじゃくるばかりだった。	泣きじゃくる= 
\\	子供は母親を捜して泣き叫んだ。	
\\	子供[こども]は 母親[ははおや]を 捜[さが]して 泣き叫[なきさけ]んだ。	
\\	彼女はその知らせを聞いて泣き崩れた。	
\\	彼女[かのじょ]はその 知[し]らせを 聞[き]いて 泣き崩[なきくず]れた。	
\\	君の靴下に穴があいている。	
\\	君[きみ]の 靴下[くつした]に 穴[あな]があいている。	
\\	彼女の死によって私の心にぽっかり穴があいた。	
\\	彼女[かのじょ]の 死[し]によって 私[わたし]の 心[こころ]にぽっかり 穴[あな]があいた。	
\\	恥ずかしさのあまり私は穴があれば入りたい気がした。	
\\	恥[は]ずかしさのあまり 私[わたし]は 穴[あな]があれば 入[はい]りたい 気[き]がした。	
\\	あのチームはディフェンスに穴がある。	
\\	あのチームはディフェンスに 穴[あな]がある。	
\\	その計画は穴だらけだ。	
\\	その 計画[けいかく]は 穴[あな]だらけだ。	
\\	彼の議論は穴だらけである。	
\\	彼[かれ]の 議論[ぎろん]は 穴[あな]だらけである。	
\\	彼は海外勤務のために英語の特訓を受けた。	
\\	彼[かれ]は 海外[かいがい] 勤務[きんむ]のために 英語[えいご]の 特訓[とっくん]を 受[う]けた。	
\\	この物語には教訓がある。	
\\	この 物語[ものがたり]には 教訓[きょうくん]がある。	教訓=きょうくん= 
\\	今回の失敗から得た教訓を生かさなければならない。	
\\	今回[こんかい]の 失敗[しっぱい]から 得[え]た 教訓[きょうくん]を 生[い]かさなければならない。	教訓=きょうくん= 
\\	英語が生かせる仕事に就きたい。	
\\	英語[えいご]が 生[い]かせる 仕事[しごと]に 就[つ]きたい。	生かす・活かす=いかす= 
\\	廃物を生かして家具を作った。	
\\	廃物[はいぶつ]を 生[い]かして 家具[かぐ]を 作[つく]った。	生かす・活かす=いかす= 
\\	素材の味を生かした料理が食べたい。	
\\	素材[そざい]の 味[あじ]を 生[い]かした 料理[りょうり]が 食[た]べたい。	生かす・活かす=いかす= 
\\	お前を生かすも殺すも俺の胸一つだ。	
\\	お 前[まえ]を 生[い]かすも 殺[ころ]すも 俺[おれ]の 胸[むね] 一[ひと]つだ。	生かす・活かす=いかす= 
\\	胸一つ=むねひとつ= 
\\	心停止の患者を応急処置で生かした。	
\\	心[しん] 停止[ていし]の 患者[かんじゃ]を 応急[おうきゅう] 処置[しょち]で 生[い]かした。	心停止=しんていし= 
\\	生かす・活かす=いかす= 
\\	彼は2回目のデートで彼女に求婚した。	
\\	彼[かれ]は2 回[かい] 目[め]のデートで 彼女[かのじょ]に 求婚[きゅうこん]した。	
\\	彼は不用意な発言をして自ら墓穴を掘った。	
\\	彼[かれ]は 不用意[ふようい]な 発言[はつげん]をして 自[みずか]ら 墓穴[ぼけつ]を 掘[ほ]った。	
\\	その条約には大きな抜け穴がある。	
\\	その 条約[じょうやく]には 大[おお]きな 抜け穴[ぬけあな]がある。	
\\	全員地下の抜け穴を通って脱出した。	
\\	全員[ぜんいん] 地下[ちか]の 抜け穴[ぬけあな]を 通[とお]って 脱出[だっしゅつ]した。	
\\	どの法律にも抜け穴はある。	
\\	どの 法律[ほうりつ]にも 抜け穴[ぬけあな]はある。	
\\	考えようによっては彼は英雄とも言える。	
\\	考[かんが]えようによっては 彼[かれ]は 英雄[えいゆう]とも 言[い]える。	因る・依る・拠る=よる= (原因する) 
\\	(基づく) 
\\	よるべき先例がない。	
\\	よるべき 先例[せんれい]がない。	因る・依る・拠る=よる= (原因する) 
\\	(基づく) 
\\	彼は大学で抜群の成績を収めた。	
\\	彼[かれ]は 大学[だいがく]で 抜群[ばつぐん]の 成績[せいせき]を 収[おさ]めた。	
\\	しようとしてもとても太刀打ちできない。	
\\	しようとしてもとても 太刀打[たちう]ちできない。	太刀打ち=たちうち= 
\\	(対抗) 
\\	あの会社の技術にはとても太刀打ちできない。	
\\	あの 会社[かいしゃ]の 技術[ぎじゅつ]にはとても 太刀打[たちう]ちできない。	太刀打ち=たちうち= 
\\	(対抗) 
\\	将棋では我々の誰も彼に太刀打ちできない。	
\\	将棋[しょうぎ]では 我々[われわれ]の 誰[だれ]も 彼[かれ]に 太刀打[たちう]ちできない。	太刀打ち=たちうち= 
\\	(対抗) 
\\	剃刀で指を切ってしまった。	
\\	剃刀[かみそり]で 指[ゆび]を 切[き]ってしまった。	剃刀=かみそり= 
\\	その数学者は剃刀のように切れる頭脳を持っている。	
\\	その 数学[すうがく] 者[しゃ]は 剃刀[かみそり]のように 切[き]れる 頭脳[ずのう]を 持[も]っている。	剃刀=かみそり= 
\\	彼は仕事で2年間タイに駐在したことがある。	
\\	彼[かれ]は 仕事[しごと]で2 年間[ねんかん]タイに 駐在[ちゅうざい]したことがある。	駐在=ちゅうざい= 
\\	燃料が漏れている。	
\\	燃料[ねんりょう]が 漏[も]れている。	
\\	このエンジンは多くの燃料を消費する。	
\\	このエンジンは 多[おお]くの 燃料[ねんりょう]を 消費[しょうひ]する。	
\\	戦争が再燃した。	
\\	戦争[せんそう]が 再燃[さいねん]した。	再燃=さいねん= (火の) 
\\	二人の恋は再燃したようだ。	
\\	二人[ふたり]の 恋[こい]は 再燃[さいねん]したようだ。	再燃=さいねん= (火の) 
\\	問題が再燃した。	
\\	問題[もんだい]が 再燃[さいねん]した。	再燃=さいねん= (火の) 
\\	歩くと床がきしむ。	
\\	歩[ある]くと 床[ゆか]がきしむ。	床=ゆか= 
\\	私は床が変わると眠れない。	
\\	私[わたし]は 床[ゆか]が 変[か]わると 眠[ねむ]れない。	床=とこ= (寝床) 
\\	(病床) 
\\	彼は風邪で床についている。	
\\	彼[かれ]は 風邪[かぜ]で 床[ゆか]についている。	床=とこ= (寝床) 
\\	(病床) 
\\	彼はまだ床を離れることができない。	
\\	彼[かれ]はまだ 床[ゆか]を 離[はな]れることができない。	床=とこ= (寝床) 
\\	(病床) 
\\	君のそういうとこが嫌いだ。	
\\	君[きみ]のそういうとこが 嫌[きら]いだ。	とこ= 
\\	所
\\	彼はその女性の前でいいとこを見せようとした。	
\\	彼[かれ]はその 女性[じょせい]の 前[まえ]でいいとこを 見[み]せようとした。	とこ= 
\\	所
\\	まったく予測がつかないから出たとこ勝負で行くしかない。	
\\	まったく 予測[よそく]がつかないから 出たとこ勝負[でたとこしょうぶ]で 行[い]くしかない。	出たとこ勝負で行く=でた とこ しょうぶ で いく= 
\\	考えていてもしようがない。出たとこ勝負で行こう。	
\\	考[かんが]えていてもしようがない。 出たとこ勝負[でたとこしょうぶ]で 行[い]こう。	出たとこ勝負で行く=でた とこ しょうぶ で いく= 
\\	寝床から「おおい」と父が呼んだ	
\\	寝床[ねどこ]から「おおい」と 父[ちち]が 呼[よ]んだ	寝床・寝所=ねどこ= 
\\	今日は休日なので父はまだ寝床でぐずぐずしている。	
\\	今日[きょう]は 休日[きゅうじつ]なので 父[ちち]はまだ 寝床[ねどこ]でぐずぐずしている。	寝床・寝所=ねどこ= 
\\	ぐずぐず= (のろいようす) 
\\	(ぶつぶつ言うようす)
\\	社会的不満がテロの温床になる。	
\\	社会[しゃかい] 的[てき] 不満[ふまん]がテロの 温床[おんしょう]になる。	温床=おんしょう= 
\\	貧民街は病気と犯罪の温床である。	
\\	貧民[ひんみん] 街[がい]は 病気[びょうき]と 犯罪[はんざい]の 温床[おんしょう]である。	温床=おんしょう= 
\\	彼は坂本氏のために祝杯をあげようと発議した。	
\\	彼[かれ]は 坂本[さかもと] 氏[し]のために 祝杯[しゅくはい]をあげようと 発議[ほつぎ]した。	発議=はつぎ= 
\\	今夜はここに宿泊することになっている。	
\\	今夜[こんや]はここに 宿泊[しゅくはく]することになっている。	
\\	猫がえさを催促して鳴いている。	
\\	猫[ねこ]がえさを 催促[さいそく]して 鳴[な]いている。	催促=さいそく= 
\\	催促がましいことは言いたくないんだが、あの金はいつ返してくれるんだい。	
\\	催促[さいそく]がましいことは 言[い]いたくないんだが、あの 金[きん]はいつ 返[かえ]してくれるんだい。	催促=さいそく= 
\\	〜がましい= 
\\	図書館に本を返却するよう督促された。	
\\	図書館[としょかん]に 本[ほん]を 返却[へんきゃく]するよう 督促[とくそく]された。	返却=へんきゃく= 
\\	督促=とくそく= 
\\	夕飯の献立がなかなか決まらない。	
\\	夕飯[ゆうはん]の 献立[こんだて]がなかなか 決[き]まらない。	献立=こんだて= 
\\	朝食の献立は味噌汁、卵焼き、海苔と香の物だ。	
\\	朝食[ちょうしょく]の 献立[こんだて]は 味噌汁[みそしる]、 卵焼[たまごや]き、 海苔[のり]と 香の物[こうのもの]だ。	献立=こんだて= 
\\	海苔=のり= 
\\	香の物=こうのもの= 
\\	味噌汁=みそしる
\\	彼は日本の政界に大いに貢献するところがあった。	
\\	彼[かれ]は 日本[にっぽん]の 政界[せいかい]に 大[おお]いに 貢献[こうけん]するところがあった。	貢献=こうけん= 
\\	君は会社への貢献が足りない。	
\\	君[きみ]は 会社[かいしゃ]への 貢献[こうけん]が 足[た]りない。	貢献=こうけん= 
\\	地球温暖化の結果、アルプス地方の氷河がかなり後退した。	
\\	地球[ちきゅう] 温暖[おんだん] 化[か]の 結果[けっか]、アルプス 地方[ちほう]の 氷河[ひょうが]がかなり 後退[こうたい]した。	氷河=ひょうが= 
\\	時代はまだ氷河期を脱したとは言えない。	
\\	時代[じだい]はまだ 氷河期[ひょうがき]を 脱[だっ]したとは 言[い]えない。	"氷河期=ひょうがき= 
\\	(就職難の) 
\\	今、女子大生の就職は氷河期だ。	
\\	今[いま]、 女子大[じょしだい] 生[せい]の 就職[しゅうしょく]は 氷河期[ひょうがき]だ。	"氷河期=ひょうがき= 
\\	(就職難の) 
\\	これですっかり疑惑が氷解した。	
\\	これですっかり 疑惑[ぎわく]が 氷解[ひょうかい]した。	氷解=ひょうかい= 
\\	港は目下氷結している。	
\\	港[みなと]は 目下[もっか] 氷結[ひょうけつ]している。	目下=もっか= 
\\	氷結=ひょうけつ= 
\\	二人は氷炭相容れない仲だ。	
\\	二人[ふたり]は 氷炭[ひょうたん] 相容[あいい]れない 仲[なか]だ。	氷炭=ひょうたん= 
\\	氷炭相容れない=ひょうたんあいいれない= 
\\	宗教と科学は本来氷炭相容れないものである。	
\\	宗教[しゅうきょう]と 科学[かがく]は 本来[ほんらい] 氷炭[ひょうたん] 相容[あいい]れないものである。	氷炭=ひょうたん= 
\\	氷炭相容れない=ひょうたんあいいれない= 
\\	氷柱ができた。	
\\	氷柱[つらら]ができた。	氷柱=つらら= 
\\	外気は氷点下に下がっていた。	
\\	外気[がいき]は 氷点下[ひょうてんか]に 下[さ]がっていた。	
\\	湖面の半分が結氷した。	
\\	湖面[こめん]の 半分[はんぶん]が 結氷[けっぴょう]した。	結氷=けっぴょう= 
\\	湖が20年ぶりに全面結氷した。	
\\	湖[みずうみ]が20 年[ねん]ぶりに 全面[ぜんめん] 結氷[けっぴょう]した。	結氷=けっぴょう= 
\\	共産主義と民族主義には相容れないものがある。	
\\	共産[きょうさん] 主義[しゅぎ]と 民族[みんぞく] 主義[しゅぎ]には 相容[あいい]れないものがある。	相容れない=あいいれない= 
\\	犯人を取り逃したのは警察の大失態だった。	
\\	犯人[はんにん]を 取り逃[とりにが]したのは 警察[けいさつ]の 大[だい] 失態[しったい]だった。	取り逃がす=とりにがす= 
\\	網が破れて魚を大半取り逃がした。	
\\	網[あみ]が 破[やぶ]れて 魚[さかな]を 大半[たいはん] 取り逃[とりに]がした。	取り逃がす=とりにがす= 
\\	やつを取り逃がしてしまった。	
\\	やつを 取り逃[とりに]がしてしまった。	取り逃がす=とりにがす= 
\\	小学校建設のために献金をお願いします。	
\\	小学校[しょうがっこう] 建設[けんせつ]のために 献金[けんきん]をお 願[ねが]いします。	献金=けんきん= 
\\	彼は津波被災者救援活動の支援に1000万円を献金した。	
\\	彼[かれ]は 津波[つなみ] 被災[ひさい] 者[しゃ] 救援[きゅうえん] 活動[かつどう]の 支援[しえん]に1000 万[まん] 円[えん]を 献金[けんきん]した。	献金=けんきん= 
\\	ここ数年患ったことがない。	
\\	ここ 数[すう] 年[ねん] 患[わずら]ったことがない。	患う=わずらう= 
\\	チームは得点力が落ちている。かなりの重症だ。	
\\	チームは 得点[とくてん] 力[りょく]が 落[お]ちている。かなりの 重症[じゅうしょう]だ。	重症=じゅうしょう= 
\\	死刑を廃止せよ。	
\\	死刑[しけい]を 廃止[はいし]せよ。	廃止=はいし= 
\\	(法律などの) 
\\	この古い規則は廃止すべきだ。	
\\	この 古[ふる]い 規則[きそく]は 廃止[はいし]すべきだ。	廃止=はいし= 
\\	(法律などの) 
\\	いくら言っても馬耳東風だった。	
\\	いくら 言[い]っても 馬耳東風[ばじとうふう]だった。	馬耳東風=ばじとうふう= 
\\	彼らの立場はわれわれと大同小異である。	
\\	彼[かれ]らの 立場[たちば]はわれわれと 大同小異[だいどうしょうい]である。	大同小異=だいどうしょうい= 
\\	これらの製品は使い勝手の点では大同小異だ。	
\\	これらの 製品[せいひん]は 使い勝手[つかいがって]の 点[てん]では 大同小異[だいどうしょうい]だ。	使い勝手=つかいがって= 
\\	大同小異=だいどうしょうい= 
\\	あの人は何事も他力本願だ。	
\\	あの 人[ひと]は 何事[なにごと]も 他力本願[たりきほんがん]だ。	他力本願=たりきほんがん= 
\\	本末転倒も甚だしい。	
\\	本末転倒[ほんまつてんとう]も 甚[はなは]だしい。	本末転倒=ほんまつてんとう= 
\\	リサーチする前にペーパーを書くのは本末転倒だ。	
\\	リサーチする 前[まえ]にペーパーを 書[か]くのは 本末転倒[ほんまつてんとう]だ。	本末転倒=ほんまつてんとう= 
\\	冬ごもり前の熊はひたすら食べる。	
\\	冬[ふゆ]ごもり 前[まえ]の 熊[くま]はひたすら 食[た]べる。	ひたすら= (いちずに) 
\\	(そればかり) 
\\	私は今はひたすら勉強するだけだ。	
\\	私[わたし]は 今[いま]はひたすら 勉強[べんきょう]するだけだ。	ひたすら= (いちずに) 
\\	(そればかり) 
\\	私は彼女にひたすら頭を下げるしかなかった。	
\\	私[わたし]は 彼女[かのじょ]にひたすら 頭[あたま]を 下[さ]げるしかなかった。	ひたすら= (いちずに) 
\\	(そればかり) 
\\	頭を下げる= 
\\	彼はひたすら経済学の研究に没頭した。	
\\	彼[かれ]はひたすら 経済[けいざい] 学[がく]の 研究[けんきゅう]に 没頭[ぼっとう]した。	ひたすら= (いちずに) 
\\	(そればかり) 
\\	没頭=ぼっとう= 
\\	就職以来ただひたすら働き続けてきた。	
\\	就職[しゅうしょく] 以来[いらい]ただひたすら 働[はたら]き 続[つづ]けてきた。	ひたすら= (いちずに) 
\\	(そればかり) 
\\	友人が困っているのに知らん顔はできない。	
\\	友人[ゆうじん]が 困[こま]っているのに 知らん顔[しらんかお]はできない。	知らん顔=しらんかお= (知らないふり) 
\\	彼女に話しかけたが知らん顔をされた。	
\\	彼女[かのじょ]に 話[はな]しかけたが 知らん顔[しらんかお]をされた。	知らん顔=しらんかお= (知らないふり) 
\\	ぐずぐずしてはいられない。	
\\	ぐずぐずしてはいられない。	ぐずぐず= (のろいようす) 
\\	(ぶつぶつ言うようす)
\\	ぐずぐずしてると電車に乗り遅れるぞ。	
\\	ぐずぐずしてると 電車[でんしゃ]に 乗り遅[のりおく]れるぞ。	ぐずぐず= (のろいようす) 
\\	(ぶつぶつ言うようす)
\\	ぐずぐずするな。	
\\	ぐずぐずするな。	ぐずぐず= (のろいようす) 
\\	(ぶつぶつ言うようす)
\\	何をぐずぐずしてるんだ。	
\\	何[なに]をぐずぐずしてるんだ。	ぐずぐず= (のろいようす) 
\\	(ぶつぶつ言うようす)
\\	人間は知らぬ間に独りよがりに陥ることがある。	
\\	人間[にんげん]は 知[し]らぬ 間[あいだ]に 独[ひと]りよがりに 陥[おちい]ることがある。	独り善がり=ひとりよがり= 
\\	彼の独りよがりには我慢がならない。	
\\	彼[かれ]の 独[ひと]りよがりには 我慢[がまん]がならない。	独り善がり=ひとりよがり= 
\\	私の言ったことを取り違えないでください。	
\\	私[わたし]の 言[い]ったことを 取り違[とりちが]えないでください。	
\\	子供の足ではどんなに急いでもたかが知れている。	
\\	子供[こども]の 足[あし]ではどんなに 急[いそ]いでもたかが 知[し]れている。	高が知れている=たか が・・・= 
\\	いくら器用だと言っても素人のやることではたかが知れている。	
\\	いくら 器用[きよう]だと 言[い]っても 素人[しろうと]のやることではたかが 知[し]れている。	高が知れている=たか が・・・= 
\\	あまりたかをくくるもんじゃない。	
\\	あまりたかをくくるもんじゃない。	たかをくくる= 
\\	たかをくくっていると負ける。	
\\	たかをくくっていると 負[ま]ける。	たかをくくる= 
\\	彼はあの会社に有力なコネがある。	
\\	彼[かれ]はあの 会社[かいしゃ]に 有力[ゆうりょく]なコネがある。	コネ= 
\\	彼はコネで採用されたんだ。	
\\	彼[かれ]はコネで 採用[さいよう]されたんだ。	コネ= 
\\	たった一度の失敗にめげるんじゃない!	
\\	たった一 度[ど]の 失敗[しっぱい]にめげるんじゃない!	めげる= 
\\	彼は絶対めげない人だ。	
\\	彼[かれ]は 絶対[ぜったい]めげない 人[ひと]だ。	めげる= 
\\	世間のうわさなど聞き流しておけ。	
\\	世間[せけん]のうわさなど 聞き流[ききなが]しておけ。	聞き流す= 
\\	彼女はその冗談を聞き流した。	
\\	彼女[かのじょ]はその 冗談[じょうだん]を 聞き流[ききなが]した。	聞き流す= 
\\	ただの思いつきです。聞き流してください。	
\\	ただの 思[おも]いつきです。 聞き流[ききなが]してください。	聞き流す= 
\\	息子は結婚したばかりで幸福の絶頂にいます。	
\\	息子[むすこ]は 結婚[けっこん]したばかりで 幸福[こうふく]の 絶頂[ぜっちょう]にいます。	絶頂=ぜっちょう= (山頂) 
\\	(頂点) 
\\	この研究のねらいは序文で述べられている。	
\\	この 研究[けんきゅう]のねらいは 序文[じょぶん]で 述[の]べられている。	狙い=ねらい= 
\\	(意図・目算) 
\\	この作品のねらいははっきりしている。	
\\	この 作品[さくひん]のねらいははっきりしている。	狙い=ねらい= 
\\	(意図・目算) 
\\	あいつの本当のねらいは何だろう。	
\\	あいつの 本当[ほんとう]のねらいは 何[なに]だろう。	狙い=ねらい= 
\\	(意図・目算) 
\\	腕が震えてねらいが定まらなかった。	
\\	腕[うで]が 震[ふる]えてねらいが 定[さだ]まらなかった。	狙い=ねらい= 
\\	(意図・目算) 
\\	ハワイはよく太平洋の楽園と称される。	
\\	ハワイはよく 太平洋[たいへいよう]の 楽園[らくえん]と 称[しょう]される。	
\\	亡父の娘だと称する女性が現れた。	
\\	亡父[ぼうふ]の 娘[むすめ]だと 称[しょう]する 女性[じょせい]が 現[あらわ]れた。	
\\	ことはトントン拍子に運んだ。	
\\	ことは トントン拍子[とんとんびょうし]に 運[はこ]んだ。	トントン拍子(びょうし)= 
\\	話し合いは順調でトントン拍子に話がまとまった。	
\\	話し合[はなしあ]いは 順調[じゅんちょう]で トントン拍子[とんとんびょうし]に 話[はなし]がまとまった。	トントン拍子(びょうし)= 
\\	万事トントン拍子にうまく行った。	
\\	万事[ばんじ] トントン拍子[とんとんびょうし]にうまく 行[い]った。	トントン拍子(びょうし)= 
\\	少子化によって社会の高齢化に一層拍車がかかっている。	
\\	少子[しょうし] 化[か]によって 社会[しゃかい]の 高齢[こうれい] 化[か]に 一層[いっそう] 拍車[はくしゃ]がかかっている。	拍車=はくしゃ= 
\\	拍車がかかる= (進行が早まる) 
\\	この事件が環境保護をめぐる議論に一層の拍車をかけることになった。	
\\	この 事件[じけん]が 環境[かんきょう] 保護[ほご]をめぐる 議論[ぎろん]に 一層[いっそう]の 拍車[はくしゃ]をかけることになった。	拍車=はくしゃ= 
\\	拍車をかける= 
\\	そろそろ作業に拍車をかけなければ、納期に間に合わないぞ。	
\\	そろそろ 作業[さぎょう]に 拍車[はくしゃ]をかけなければ、 納期[のうき]に 間に合[まにあ]わないぞ。	拍車=はくしゃ= 
\\	拍車をかける= 
\\	あいつもいよいよ年貢の納め時だ。	
\\	あいつもいよいよ 年貢[ねんぐ]の 納[おさ]め 時[じ]だ。	年貢の納め時=ねんぐのおさめどき= 
\\	あのプレイボーイも彼女に出会ってようやく年貢を納める気になったらしい。	
\\	あのプレイボーイも 彼女[かのじょ]に 出会[であ]ってようやく 年貢[ねんぐ]を 納[おさ]める 気[き]になったらしい。	年貢の納め時=ねんぐのおさめどき= 
\\	方針がなし崩しに変更されていった。	
\\	方針[ほうしん]がなし 崩[くず]しに 変更[へんこう]されていった。	なし崩し=なしくずし= 
\\	明日の演奏会にはたくさんお友達を誘って来てね。	
\\	明日[あした]の 演奏[えんそう] 会[かい]にはたくさんお 友達[ともだち]を 誘[さそ]って 来[き]てね。	
\\	睡眠薬が眠りを誘ってくれた。	
\\	睡眠薬[すいみんやく]が 眠[ねむ]りを 誘[さそ]ってくれた。	
\\	春の陽気が眠気を誘った。	
\\	春[はる]の 陽気[ようき]が 眠気[ねむけ]を 誘[さそ]った。	
\\	甘言に誘われて承諾してしまった。	
\\	甘言[かんげん]に 誘[さそ]われて 承諾[しょうだく]してしまった。	甘言=かんげん= 
\\	毎日通勤でこの坂を上ったり下りたりしている。	
\\	毎日[まいにち] 通勤[つうきん]でこの 坂[さか]を 上[のぼ]ったり 下[お]りたりしている。	上る=登る 下りる=おりる
\\	彼は少しずつ木を登っていった。	
\\	彼[かれ]は 少[すこ]しずつ 木[き]を 登[のぼ]っていった。	
\\	私は富士山に百回登った。	
\\	私[わたし]は 富士山[ふじさん]に 百回[ひゃっかい] 登[のぼ]った。	
\\	今屋根に上っている。	
\\	今[いま] 屋根[やね]に 上[のぼ]っている。	
\\	サケが川を上るのを見たことがありますか。	
\\	サケが 川[かわ]を 上[のぼ]るのを 見[み]たことがありますか。	上る=登る
\\	被害者は5千人に上った。	
\\	被害[ひがい] 者[しゃ]は5 千[せん] 人[にん]に 上[のぼ]った。	上る=登る
\\	核兵器が国を守るというのは幻想にすぎない。	
\\	核兵器[かくへいき]が 国[くに]を 守[まも]るというのは 幻想[げんそう]にすぎない。	幻想=げんそう= 
\\	友達の家に泊まり込んで一緒にレポートを書いた。	
\\	友達[ともだち]の 家[いえ]に 泊まり込[とまりこ]んで 一緒[いっしょ]にレポートを 書[か]いた。	
\\	病院に何日も泊まり込んで父親の看病をした。	
\\	病院[びょういん]に 何[なん] 日[にち]も 泊まり込[とまりこ]んで 父親[ちちおや]の 看病[かんびょう]をした。	
\\	スポーツ選手たちは、国際交流の一翼を担っている。	
\\	スポーツ 選手[せんしゅ]たちは、 国際[こくさい] 交流[こうりゅう]の 一翼[いちよく]を 担[にな]っている。	一翼=いちよく= 
\\	(一つの役割) 
\\	インチキ商品を売りつけられないように気をつけろよ。	
\\	インチキ 商品[しょうひん]を 売[う]りつけられないように 気[き]をつけろよ。	インチキ= 
\\	売りつける= 
\\	友達に芝居のチケットを売りつけられた。	
\\	友達[ともだち]に 芝居[しばい]のチケットを 売[う]りつけられた。	売りつける= 
\\	激しい戦闘が生じた。	
\\	激[はげ]しい 戦闘[せんとう]が 生[しょう]じた。	戦闘=せんとう= 
\\	長い間の戦闘がやっと終わった。	
\\	長[なが]い 間[あいだ]の 戦闘[せんとう]がやっと 終[お]わった。	戦闘=せんとう= 
\\	我が社では副社長が事実上の社長です。	
\\	我[わ]が 社[しゃ]では 副[ふく] 社長[しゃちょう]が 事実[じじつ] 上[じょう]の 社長[しゃちょう]です。	事実上=じじつじょう= 
\\	戦争は事実上終わっていた。	
\\	戦争[せんそう]は 事実[じじつ] 上[じょう] 終[お]わっていた。	事実上=じじつじょう= 
\\	その銀行は事実上倒産している。	
\\	その 銀行[ぎんこう]は 事実[じじつ] 上[じょう] 倒産[とうさん]している。	事実上=じじつじょう= 
\\	反乱が勃発した。	
\\	反乱[はんらん]が 勃発[ぼっぱつ]した。	勃発=ぼっぱつ= 事件などが突然に起こること。
\\	大事件が勃発した。	
\\	大[だい] 事件[じけん]が 勃発[ぼっぱつ]した。	勃発=ぼっぱつ= 事件などが突然に起こること。
\\	ある日私の母親だと名乗る女性が現れた。	
\\	ある 日[ひ] 私[わたし]の 母親[ははおや]だと 名乗[なの]る 女性[じょせい]が 現[あらわ]れた。	名乗る=なのる= 
\\	彼女は職場では旧姓を名乗っている。	
\\	彼女[かのじょ]は 職場[しょくば]では 旧姓[きゅうせい]を 名乗[なの]っている。	名乗る=なのる= 
\\	彼は名乗ることを拒んだ。	
\\	彼[かれ]は 名乗[なの]ることを 拒[こば]んだ。	名乗る=なのる= 
\\	二人は互いに名乗り合った。	
\\	二人[ふたり]は 互[たが]いに 名乗[なの]り 合[あ]った。	名乗る=なのる= 
\\	帽子を黒にすればドレスが一段と引き立ちますよ。	
\\	帽子[ぼうし]を 黒[くろ]にすればドレスが 一段[いちだん]と 引き立[ひきた]ちますよ。	一段= 
\\	(さらに・ひときわ) 
\\	彼女は一段と美しくなった。	
\\	彼女[かのじょ]は 一段[いちだん]と 美[うつく]しくなった。	一段= 
\\	(さらに・ひときわ) 
\\	彼女はこのところ舞台俳優として一段の進歩を見せている。	
\\	彼女[かのじょ]はこのところ 舞台[ぶたい] 俳優[はいゆう]として一 段[だん]の 進歩[しんぽ]を 見[み]せている。	一段= 
\\	(さらに・ひときわ) 
\\	ゲリラが空港を制圧した。	
\\	ゲリラが 空港[くうこう]を 制圧[せいあつ]した。	制圧=せいあつ= 
\\	その町もついに敵の占領するところとなった。	
\\	その 町[まち]もついに 敵[てき]の 占領[せんりょう]するところとなった。	占領=せんりょう= 
\\	(独占) 
\\	君はこの広い部屋を一人で占領しているのか。	
\\	君[きみ]はこの 広[ひろ]い 部屋[へや]を 一人[ひとり]で 占領[せんりょう]しているのか。	占領=せんりょう= 
\\	(独占) 
\\	彼の頭の中は彼女のことで占領されていた。	
\\	彼[かれ]の 頭[あたま]の 中[なか]は 彼女[かのじょ]のことで 占領[せんりょう]されていた。	占領=せんりょう= 
\\	(独占) 
\\	警察はようやくその事件の捜査に乗り出した。	
\\	警察[けいさつ]はようやくその 事件[じけん]の 捜査[そうさ]に 乗り出[のりだ]した。	
\\	都市部での投票率の伸びが彼の勝利を決定づけた。	
\\	都市[とし] 部[ぶ]での 投票[とうひょう] 率[りつ]の 伸[の]びが 彼[かれ]の 勝利[しょうり]を 決定[けってい]づけた。	決定づける=けっていづける= 
\\	さしあたっての用はこれで足りるだろう。	
\\	さしあたっての 用[よう]はこれで 足[た]りるだろう。	差し当たって・差し当たり= 
\\	さしあたっての問題は場所の確保だ。	
\\	さしあたっての 問題[もんだい]は 場所[ばしょ]の 確保[かくほ]だ。	差し当たって・差し当たり= 
\\	さしあたりそれはいらない。	
\\	さしあたりそれはいらない。	差し当たって・差し当たり= 
\\	彼らは反対の方向へ向かって歩き出した。	
\\	彼[かれ]らは 反対[はんたい]の 方向[ほうこう]へ 向[む]かって 歩[ある]き 出[だ]した。	
\\	私もそっちの方向へ参ります。	
\\	私[わたし]もそっちの 方向[ほうこう]へ 参[まい]ります。	
\\	彼は音のする方向へ進んでいった。	
\\	彼[かれ]は 音[おと]のする 方向[ほうこう]へ 進[すす]んでいった。	
\\	駅はどちらの方向ですか。	
\\	駅[えき]はどちらの 方向[ほうこう]ですか。	
\\	我が社の将来の方向はまだ決まっていない。	
\\	我[わ]が 社[しゃ]の 将来[しょうらい]の 方向[ほうこう]はまだ 決[き]まっていない。	
\\	これは一刻を争う問題だ。	
\\	これは 一刻[いっこく]を 争[あらそ]う 問題[もんだい]だ。	争う=あらそう= (敵対する) 
\\	(急を要する)
\\	一分一秒を争うときだ。	
\\	一分[いっぷん] 一秒[いちびょう]を 争[あらそ]うときだ。	争う=あらそう= (敵対する) 
\\	(急を要する)
\\	彼と僕は彼女を争った仲だ。	
\\	彼[かれ]と 僕[ぼく]は 彼女[かのじょ]を 争[あらそ]った 仲[なか]だ。	争う=あらそう= (敵対する) 
\\	(急を要する)
\\	彼女の遺体には争った跡があった。	
\\	彼女[かのじょ]の 遺体[いたい]には 争[あらそ]った 跡[あと]があった。	争う=あらそう= (敵対する) 
\\	(急を要する)
\\	その国は領土問題で近隣諸国と争っている。	
\\	その 国[くに]は 領土[りょうど] 問題[もんだい]で 近隣[きんりん] 諸国[しょこく]と 争[あらそ]っている。	争う=あらそう= (敵対する) 
\\	(急を要する)
\\	二つの党派が結合して新党を結成した。	
\\	二[ふた]つの 党派[とうは]が 結合[けつごう]して 新党[しんとう]を 結成[けっせい]した。	結成=けっせい= 
\\	私はその手紙を幾たびも書き直した。	
\\	私[わたし]はその 手紙[てがみ]を 幾[いく]たびも 書き直[かきなお]した。	
\\	私も幾度か挑戦したが、禁煙は難しいものだ。	
\\	私[わたし]も 幾度[いくど]か 挑戦[ちょうせん]したが、 禁煙[きんえん]は 難[むずか]しいものだ。	
\\	彼は幾度もやってみたが成功しなかった。	
\\	彼[かれ]は 幾度[いくど]もやってみたが 成功[せいこう]しなかった。	
\\	その事業は彼の生活を意義あらしめるものだ。	
\\	その 事業[じぎょう]は 彼[かれ]の 生活[せいかつ]を 意義[いぎ]あらしめるものだ。	意義=いぎ= (意味内容) 
\\	(他との関連における重要性・価値) 
\\	日本では大学教育の意義が薄れてきた。	
\\	日本[にほん]では 大学[だいがく] 教育[きょういく]の 意義[いぎ]が 薄[うす]れてきた。	意義=いぎ= (意味内容) 
\\	(他との関連における重要性・価値) 
\\	この条約は両国の友好に重要な意義をもつ。	
\\	この 条約[じょうやく]は 両国[りょうこく]の 友好[ゆうこう]に 重要[じゅうよう]な 意義[いぎ]をもつ。	意義=いぎ= (意味内容) 
\\	(他との関連における重要性・価値) 
\\	「文明」という語の意義は極めて広い。	
\\	文明[ぶんめい]」という 語[かたり]の 意義[いぎ]は 極[きわ]めて 広[ひろ]い。	意義=いぎ= (意味内容) 
\\	(他との関連における重要性・価値) 
\\	日米同盟は日本の外交政策の基軸だ。	
\\	日[にち] 米[べい] 同盟[どうめい]は日本の 外交[がいこう] 政策[せいさく]の 基軸[きじく]だ。	基軸=きじく= 
\\	中東の紛争がようやく終結した。	
\\	中東[ちゅうとう]の 紛争[ふんそう]がようやく 終結[しゅうけつ]した。	
\\	裁判が終結した。	
\\	裁判[さいばん]が 終結[しゅうけつ]した。	
\\	この契約は30日前の通知をもって終結させることができる。	
\\	この 契約[けいやく]は30 日[にち] 前[まえ]の 通知[つうち]をもって 終結[しゅうけつ]させることができる。	
\\	まさかの落選に候補者はがっくりと肩を落とした。	
\\	まさかの 落選[らくせん]に 候補[こうほ] 者[しゃ]はがっくりと 肩[かた]を 落[お]とした。	まさか= 
\\	まさかの友は真の友。	
\\	まさかの 友[とも]は 真[しん]の 友[とも]。	まさか= 
\\	まさかそんなことはあるまい。	
\\	まさかそんなことはあるまい。	まさか= 
\\	まさか火事になるとは思わなかった。	
\\	まさか 火事[かじ]になるとは 思[おも]わなかった。	まさか= 
\\	まさか彼が負けるとは思わなかった。	
\\	まさか 彼[かれ]が 負[ま]けるとは 思[おも]わなかった。	まさか= 
\\	まさか、かばんをバスに忘れてきたんじゃないでしょうね。	
\\	まさか、かばんをバスに 忘[わす]れてきたんじゃないでしょうね。	まさか= 
\\	私はどんな宗教活動にも加わらない。	
\\	私[わたし]はどんな 宗教[しゅうきょう] 活動[かつどう]にも 加[くわ]わらない。	加わる=くわわる= 
\\	(仲間に入る) 
\\	日増しに暑さが加わってきた。	
\\	日増[ひま]しに 暑[あつ]さが 加[くわ]わってきた。	加わる=くわわる= 
\\	(仲間に入る) 
\\	豪雨に風さえ加わった。	
\\	豪雨[ごうう]に 風[かぜ]さえ 加[くわ]わった。	豪雨=ごうう= 激しい勢いで大量に降る雨。 加わる=くわわる= 
\\	(仲間に入る) 
\\	実に油断のならない世の中だ。	
\\	実[じつ]に 油断[ゆだん]のならない 世の中[よのなか]だ。	油断=ゆだん= 
\\	(不注意) 
\\	あの男には油断をするな。	
\\	あの 男[おとこ]には 油断[ゆだん]をするな。	油断=ゆだん= 
\\	(不注意) 
\\	このパウンドケーキがぱさぱさだ。	
\\	このパウンドケーキがぱさぱさだ。	
\\	ぱさぱさしたご飯が嫌いだ。	
\\	ぱさぱさしたご 飯[はん]が 嫌[きら]いだ。	
\\	この痛み止めの効力はおよそ12時間持続します。	
\\	この 痛[いた]み 止[ど]めの 効力[こうりょく]はおよそ12 時間[じかん] 持続[じぞく]します。	痛み止め=いたみどめ持続=じぞく= 
\\	彼は毎年禁煙を決意するが、それを持続するだけの精神力がない。	
\\	彼[かれ]は 毎年[まいとし] 禁煙[きんえん]を 決意[けつい]するが、それを 持続[じぞく]するだけの 精神[せいしん] 力[りょく]がない。	持続=じぞく= 
\\	この暑さで集中力が持続できない。	
\\	この 暑[あつ]さで 集中[しゅうちゅう] 力[りょく]が 持続[じぞく]できない。	持続=じぞく= 
\\	机の上がごちゃごちゃだ。	
\\	机[つくえ]の 上[うえ]がごちゃごちゃだ。	
\\	いろんな人の意見を聞いて頭の中がごちゃごちゃだ。	
\\	いろんな 人[ひと]の 意見[いけん]を 聞[き]いて 頭[あたま]の 中[なか]がごちゃごちゃだ。	
\\	お前とは関係ないんだからごちゃごちゃ文句を言うな。	
\\	お 前[まえ]とは 関係[かんけい]ないんだからごちゃごちゃ 文句[もんく]を 言[い]うな。	
\\	彼は兄弟と断絶したままだ。	
\\	彼[かれ]は 兄弟[きょうだい]と 断絶[だんぜつ]したままだ。	
\\	それと取り替えてください。	
\\	それと 取り替[とりか]えてください。	
\\	この品物を取り替えてくれませんか。	
\\	この 品物[しなもの]を 取り替[とりか]えてくれませんか。	
\\	大きいサイズのと取り替えていただけませんか。	
\\	大[おお]きいサイズのと 取り替[とりか]えていただけませんか。	
\\	それが汚れたらすぐほかのと取り替えます。	
\\	それが 汚[よご]れたらすぐほかのと 取り替[とりか]えます。	
\\	彼は連行されるのを激しく拒んだ。	
\\	彼[かれ]は 連行[れんこう]されるのを 激[はげ]しく 拒[こば]んだ。	連行=れんこう= 
\\	拒む=こばむ= (拒絶する) 
\\	(妨げる) 
\\	私は入場を拒まれた。	
\\	私[わたし]は 入場[にゅうじょう]を 拒[こば]まれた。	拒む=こばむ= (拒絶する) 
\\	(妨げる) 
\\	あと100メートルのところで彼に追い抜かれた。	
\\	あと100メートルのところで 彼[かれ]に 追い抜[おいぬ]かれた。	追い抜く=おいぬく= 
\\	今年の交通事故による死亡者数は、去年の1万人を追い抜く勢いで増加している。	
\\	今年[ことし]の 交通[こうつう] 事故[じこ]による 死亡[しぼう] 者[しゃ] 数[すう]は、 去年[きょねん]の1 万[まん] 人[にん]を 追い抜[おいぬ]く 勢[いきお]いで 増加[ぞうか]している。	追い抜く=おいぬく= 
\\	彼女はこの先大成しまい。	
\\	彼女[かのじょ]はこの 先[さき] 大成[たいせい]しまい。	大成=たいせい= (大物になること) 
\\	(完成)
\\	彼は画家として立派に大成した。	
\\	彼[かれ]は 画家[がか]として 立派[りっぱ]に 大成[たいせい]した。	大成=たいせい= (大物になること) 
\\	(完成)
\\	彼は大成して巨万の富を築いた。	
\\	彼[かれ]は 大成[たいせい]して 巨万[きょまん]の 富[とみ]を 築[きず]いた。	大成=たいせい= (大物になること) 
\\	(完成)
\\	今年のプロ野球の新人には大物が多い。	
\\	今年[ことし]のプロ 野球[やきゅう]の 新人[しんじん]には 大物[おおもの]が 多[おお]い。	大物=おおもの= (実力者), (有力者) 
\\	彼女は芸能界の大物だ。	
\\	彼女[かのじょ]は 芸能[げいのう] 界[かい]の 大物[おおもの]だ。	大物=おおもの= (実力者), (有力者) 
\\	あの力士は大物食いだ。	
\\	あの 力士[りきし]は 大物食[おおものぐ]いだ。	力士=りきし大物=おおもの= (実力者), (有力者) 
\\	一連の発言には彼の野心が見え隠れしていた。	
\\	一連[いちれん]の 発言[はつげん]には 彼[かれ]の 野心[やしん]が 見え隠[みえかく]れしていた。	見え隠れ=みえがくれ= 
\\	男が見え隠れに跡をつけてきた。	
\\	男[おとこ]が 見え隠[みえかく]れに 跡[あと]をつけてきた。	見え隠れ=みえがくれ= 
\\	跡をつける= 
\\	3連勝してタイガーズが首位に返り咲いた。	
\\	連勝[れんしょう]してタイガーズが 首位[しゅい]に 返り咲[かえりざ]いた。	首位=しゅい= 
\\	返り咲く= 
\\	女優としては遅咲きだが、彼女はなかなか実力がある。	
\\	女優[じょゆう]としては 遅咲[おそざ]きだが、 彼女[かのじょ]はなかなか 実力[じつりょく]がある。	
\\	47歳で小説家デビューとは、ずいぶんと遅咲きだな。	
\\	歳[さい]で 小説[しょうせつ] 家[か]デビューとは、ずいぶんと 遅咲[おそざ]きだな。	
\\	これにはいろいろな要因が絡まっている。	
\\	これにはいろいろな 要因[よういん]が 絡[から]まっている。	
\\	夫が自分の親友と浮気していることを知った妻は半狂乱になった。	
\\	夫[おっと]が 自分[じぶん]の 親友[しんゆう]と 浮気[うわき]していることを 知[し]った 妻[つま]は 半[はん] 狂乱[きょうらん]になった。	
\\	この本で著者は歴史に埋もれていた新事実を掘り起こした。	
\\	この 本[ほん]で 著者[ちょしゃ]は 歴史[れきし]に 埋[う]もれていた 新[しん] 事実[じじつ]を 掘り起[ほりお]こした。	埋もれる=うもれる= 
\\	掘り起こす= 
\\	その建物は周りの風景によく溶け込んでいる。	
\\	その 建物[たてもの]は 周[まわ]りの 風景[ふうけい]によく 溶け込[とけこ]んでいる。	
\\	彼女はすぐ新しい職場に溶け込んだ。	
\\	彼女[かのじょ]はすぐ 新[あたら]しい 職場[しょくば]に 溶け込[とけこ]んだ。	
\\	年ごとに互いの心が解け合ってきた。	
\\	年[とし]ごとに 互[たが]いの 心[こころ]が 解け合[とけあ]ってきた。	解け合う=とけあう= 
\\	これで言いたいことは言い尽くした。	
\\	これで 言[い]いたいことは 言い尽[いいつ]くした。	言い尽くす=いいつくす= 
\\	それは一言で言い尽くせる。	
\\	それは 一言[ひとこと]で 言[い]い 尽[つ]くせる。	言い尽くす=いいつくす= 
\\	ご親切は言葉では言い尽くせません。	
\\	ご 親切[しんせつ]は 言葉[ことば]では 言い尽[いいつ]くせません。	言い尽くす=いいつくす= 
\\	彼の名前はリストから削除された。	
\\	彼[かれ]の 名前[なまえ]はリストから 削除[さくじょ]された。	
\\	この一節を削ってほしい。	
\\	この 一節[いっせつ]を 削[けず]ってほしい。	削る=けずる= 
\\	(削除する) 
\\	予算をばっさり削る必要がある。	
\\	予算[よさん]をばっさり 削[けず]る 必要[ひつよう]がある。	削る=けずる= 
\\	(削除する) 
\\	川の流れが岸辺を削っていく。	
\\	川[かわ]の 流[なが]れが 岸辺[きしべ]を 削[けず]っていく。	岸辺=きしべ= 
\\	削る=けずる= 
\\	(削除する) 
\\	もう1ミリ削ってくれ。	
\\	もう1ミリ 削[けず]ってくれ。	削る=けずる= 
\\	(削除する) 
\\	この鉛筆はよく削れている。	
\\	この 鉛筆[えんぴつ]はよく 削[けず]れている。	
\\	米ソ両国は核戦争の瀬戸際まで行って引き返した。	
\\	米[べい]ソ 両国[りょうこく]は 核[かく] 戦争[せんそう]の 瀬戸際[せとぎわ]まで 行[い]って 引き返[ひきかえ]した。	瀬戸際=せとぎわ= 
\\	我々は生きるか死ぬかの瀬戸際にいる。	
\\	我々[われわれ]は 生[い]きるか 死[し]ぬかの 瀬戸際[せとぎわ]にいる。	瀬戸際=せとぎわ= 
\\	彼は倒産の瀬戸際にいる。	
\\	彼[かれ]は 倒産[とうさん]の 瀬戸際[せとぎわ]にいる。	瀬戸際=せとぎわ= 
\\	この大学には学生は自動車で通学できないという不文律がある。	
\\	この 大学[だいがく]には 学生[がくせい]は 自動車[じどうしゃ]で 通学[つうがく]できないという 不文[ふぶん] 律[りつ]がある。	不文律=ふぶんりつ= 
\\	親の価値観で子供を律するべきではない。	
\\	親[おや]の 価値[かち] 観[かん]で 子供[こども]を 律[りっ]するべきではない。	
\\	この商店街はすっかり廃れてしまった。	
\\	この 商店[しょうてん] 街[がい]はすっかり 廃[すた]れてしまった。	廃れる=すたれる= (流行しなくなる) 
\\	あの国では道義心が廃れている。	
\\	あの 国[くに]では 道義[どうぎ] 心[しん]が 廃[すた]れている。	廃れる=すたれる= (流行しなくなる) 
\\	一部の敬語は既に廃れかかっている。	
\\	一部[いちぶ]の 敬語[けいご]は 既[すで]に 廃[すた]れかかっている。	廃れる=すたれる= (流行しなくなる) 
\\	その習慣は今は廃れかかっている。	
\\	その 習慣[しゅうかん]は 今[いま]は 廃[すた]れかかっている。	廃れる=すたれる= (流行しなくなる) 
\\	一度廃れたがまた流行し出した。	
\\	一度[いちど] 廃[すた]れたがまた 流行[りゅうこう]し 出[だ]した。	廃れる=すたれる= (流行しなくなる) 
\\	人を指差すものではない。	
\\	人[ひと]を 指[ゆび] 差[さ]すものではない。	
\\	近ごろあの男は増長している。	
\\	近[ちか]ごろあの 男[おとこ]は 増長[ぞうちょう]している。	増長=ぞうちょう= 
\\	これ以上彼の増長を許すわけにはいかない。	
\\	これ 以上[いじょう] 彼[かれ]の 増長[ぞうちょう]を 許[ゆる]すわけにはいかない。	増長=ぞうちょう= 
\\	彼の考えはどちらかといえば穏健だ。	
\\	彼[かれ]の 考[かんが]えはどちらかといえば 穏健[おんけん]だ。	穏健=おんけん= 
\\	平和実現の構図がなかなか見えてこない。	
\\	平和[へいわ] 実現[じつげん]の 構図[こうず]がなかなか 見[み]えてこない。	構図=こうず= 
\\	この法律は善良な市民の自由を拘束するものではない。	
\\	この 法律[ほうりつ]は 善良[ぜんりょう]な 市民[しみん]の 自由[じゆう]を 拘束[こうそく]するものではない。	拘束=こうそく= 
\\	映画俳優が麻薬所持の疑いで空港で身柄を拘束された。	
\\	映画[えいが] 俳優[はいゆう]が 麻薬[まやく] 所持[しょじ]の 疑[うたが]いで 空港[くうこう]で 身柄[みがら]を 拘束[こうそく]された。	身柄=みがら= 
\\	拘束=こうそく= 
\\	大家さんに談判して家賃を下げてもらおう。	
\\	大家[おおや]さんに 談判[だんぱん]して 家賃[やちん]を 下[さ]げてもらおう。	談判=だんぱん= (交渉) 
\\	(高圧的な要求) 
\\	談判が決裂した。	
\\	談判[だんぱん]が 決裂[けつれつ]した。	談判=だんぱん= (交渉) 
\\	(高圧的な要求) 
\\	君の趣味は仕事の延長だな。	
\\	君[きみ]の 趣味[しゅみ]は 仕事[しごと]の 延長[えんちょう]だな。	延長=えんちょう= (延ばすこと) 
\\	滑走路が延長された。	
\\	滑走[かっそう] 路[ろ]が 延長[えんちょう]された。	延長=えんちょう= (延ばすこと) 
\\	交渉は円満に妥結した。	
\\	交渉[こうしょう]は 円満[えんまん]に 妥結[だけつ]した。	妥結=だけつ= 
\\	(協定) 
\\	家庭は円満が第一だ。	
\\	家庭[かてい]は 円満[えんまん]が 第[だい]一だ。	円満=えんまん= 
\\	夫婦円満の秘訣を教えよう。	
\\	夫婦[ふうふ] 円満[えんまん]の 秘訣[ひけつ]を 教[おし]えよう。	円満=えんまん= 
\\	この件は円満に収めたい。	
\\	この 件[けん]は 円満[えんまん]に 収[おさ]めたい。	円満=えんまん= 
\\	あの人たちはあまり円満に行っていない。	
\\	あの 人[ひと]たちはあまり 円満[えんまん]に 行[い]っていない。	円満=えんまん= 
\\	彼は飛んできた石に当たった。	
\\	彼[かれ]は 飛[と]んできた 石[いし]に 当[あ]たった。	当たる= (ぶつかる) 
\\	(触れる) 
\\	(日・光が) 
\\	(的中する) 
\\	(成功する) 
\\	(対戦する) 
\\	(相当する・対応する) 
\\	(適当である・当てはまる) 
\\	魚がしきりに当たっていた。	
\\	魚[さかな]がしきりに 当[あ]たっていた。	しきりに= 
\\	このモモはあたっている。	
\\	このモモはあたっている。	当たる= (ぶつかる) 
\\	(触れる) 
\\	(日・光が) 
\\	(的中する) 
\\	(成功する) 
\\	(対戦する) 
\\	(相当する・対応する) 
\\	(適当である・当てはまる) 
\\	(果物などがいたむ)
\\	当たったのは昨夜のエビに違いない。	
\\	当[あ]たったのは 昨夜[さくや]のエビに 違[ちが]いない。	当たる= (ぶつかる) 
\\	(触れる) 
\\	(日・光が) 
\\	(的中する) 
\\	(成功する) 
\\	(対戦する) 
\\	(相当する・対応する) 
\\	(適当である・当てはまる) 
\\	(果物などがいたむ)
\\	何かそこで食べたものが当たったのだ。	
\\	何[なに]かそこで 食[た]べたものが 当[あ]たったのだ。	当たる= (ぶつかる) 
\\	(触れる) 
\\	(日・光が) 
\\	(的中する) 
\\	(成功する) 
\\	(対戦する) 
\\	(相当する・対応する) 
\\	(適当である・当てはまる) 
\\	(果物などがいたむ)
\\	私はあの先生の授業でよく当たる。	
\\	私[わたし]はあの 先生[せんせい]の 授業[じゅぎょう]でよく 当[あ]たる。	当たる= (ぶつかる) 
\\	(触れる) 
\\	(日・光が) 
\\	(的中する) 
\\	(成功する) 
\\	(対戦する) 
\\	(相当する・対応する) 
\\	(適当である・当てはまる) 
\\	(果物などがいたむ)
\\	掃除係に当たった。	
\\	掃除[そうじ] 係[がかり]に 当[あ]たった。	当たる= (ぶつかる) 
\\	(触れる) 
\\	(日・光が) 
\\	(的中する) 
\\	(成功する) 
\\	(対戦する) 
\\	(相当する・対応する) 
\\	(適当である・当てはまる) 
\\	(果物などがいたむ)
\\	その役目は私に当たった。	
\\	その 役目[やくめ]は 私[わたし]に 当[あ]たった。	当たる= (ぶつかる) 
\\	(触れる) 
\\	(日・光が) 
\\	(的中する) 
\\	(成功する) 
\\	(対戦する) 
\\	(相当する・対応する) 
\\	(適当である・当てはまる) 
\\	(果物などがいたむ)
\\	宝くじというものはなかなか当たらない。	
\\	宝[たから]くじというものはなかなか 当[あ]たらない。	当たる= (ぶつかる) 
\\	(触れる) 
\\	(日・光が) 
\\	(的中する) 
\\	(成功する) 
\\	(対戦する) 
\\	(相当する・対応する) 
\\	(適当である・当てはまる) 
\\	(果物などがいたむ)
\\	その語はここには当たらない。	
\\	その 語[ご]はここには 当[あ]たらない。	当たる= (ぶつかる) 
\\	(触れる) 
\\	(日・光が) 
\\	(的中する) 
\\	(成功する) 
\\	(対戦する) 
\\	(相当する・対応する) 
\\	(適当である・当てはまる) 
\\	(果物などがいたむ)
\\	私を金持ちなどというのは当たらない。	
\\	私[わたし]を 金持[かねも]ちなどというのは 当[あ]たらない。	当たる= (ぶつかる) 
\\	(触れる) 
\\	(日・光が) 
\\	(的中する) 
\\	(成功する) 
\\	(対戦する) 
\\	(相当する・対応する) 
\\	(適当である・当てはまる) 
\\	(果物などがいたむ)
\\	彼を英雄と呼ぶのは当たらない。	
\\	彼[かれ]を 英雄[えいゆう]と 呼[よ]ぶのは 当[あ]たらない。	当たる= (ぶつかる) 
\\	(触れる) 
\\	(日・光が) 
\\	(的中する) 
\\	(成功する) 
\\	(対戦する) 
\\	(相当する・対応する) 
\\	(適当である・当てはまる) 
\\	(果物などがいたむ)
\\	あの人はあなたの何に当たりますか。	
\\	あの 人[ひと]はあなたの 何[なに]に 当[あ]たりますか。	当たる= (ぶつかる) 
\\	(触れる) 
\\	(日・光が) 
\\	(的中する) 
\\	(成功する) 
\\	(対戦する) 
\\	(相当する・対応する) 
\\	(適当である・当てはまる) 
\\	(果物などがいたむ)
\\	この罪は罰金または禁固に当たる。	
\\	この 罪[つみ]は 罰金[ばっきん]または 禁固[きんこ]に 当[あ]たる。	当たる= (ぶつかる) 
\\	(触れる) 
\\	(日・光が) 
\\	(的中する) 
\\	(成功する) 
\\	(対戦する) 
\\	(相当する・対応する) 
\\	(適当である・当てはまる) 
\\	(果物などがいたむ)
\\	この語がちょうどそれに当たる。	
\\	この 語[ご]がちょうどそれに 当[あ]たる。	当たる= (ぶつかる) 
\\	(触れる) 
\\	(日・光が) 
\\	(的中する) 
\\	(成功する) 
\\	(対戦する) 
\\	(相当する・対応する) 
\\	(適当である・当てはまる) 
\\	(果物などがいたむ)
\\	"英語の 
\\	にぴったり当たるフランス語がありますか。
\\	"英語[えいご]の
\\	""にぴったり 当[あ]たる フランス語[ふらんすご]がありますか。
\\	当たる= (ぶつかる) 
\\	(触れる) 
\\	(日・光が) 
\\	(的中する) 
\\	(成功する) 
\\	(対戦する) 
\\	(相当する・対応する) 
\\	(適当である・当てはまる) 
\\	(果物などがいたむ)
\\	それに当たる英語はない。	
\\	それに 当[あ]たる 英語[えいご]はない。	当たる= (ぶつかる) 
\\	(触れる) 
\\	(日・光が) 
\\	(的中する) 
\\	(成功する) 
\\	(対戦する) 
\\	(相当する・対応する) 
\\	(適当である・当てはまる) 
\\	(果物などがいたむ)
\\	私の育った時代は日本の高度経済成長期に当たっていた。	
\\	私[わたし]の 育[そだ]った 時代[じだい]は 日本[にっぽん]の 高度[こうど] 経済[けいざい] 成長[せいちょう] 期[き]に 当[あ]たっていた。	当たる= (ぶつかる) 
\\	(触れる) 
\\	(日・光が) 
\\	(的中する) 
\\	(成功する) 
\\	(対戦する) 
\\	(相当する・対応する) 
\\	(適当である・当てはまる) 
\\	(果物などがいたむ)
\\	彼は会社でうまくいかないといつも家族に当たる。	
\\	彼[かれ]は 会社[かいしゃ]でうまくいかないといつも 家族[かぞく]に 当[あ]たる。	当たる= (ぶつかる) 
\\	(触れる) 
\\	(日・光が) 
\\	(的中する) 
\\	(成功する) 
\\	(対戦する) 
\\	(相当する・対応する) 
\\	(適当である・当てはまる) 
\\	(果物などがいたむ)
\\	大阪は東京の西に当たる。	
\\	大阪[おおさか]は 東京[とうきょう]の 西[にし]に 当[あ]たる。	当たる= (ぶつかる) 
\\	(触れる) 
\\	(日・光が) 
\\	(的中する) 
\\	(成功する) 
\\	(対戦する) 
\\	(相当する・対応する) 
\\	(適当である・当てはまる) 
\\	(果物などがいたむ)
\\	一回戦で強敵と当たった。	
\\	一回戦[いっかいせん]で 強敵[きょうてき]と 当[あ]たった。	当たる= (ぶつかる) 
\\	(触れる) 
\\	(日・光が) 
\\	(的中する) 
\\	(成功する) 
\\	(対戦する) 
\\	(相当する・対応する) 
\\	(適当である・当てはまる) 
\\	(果物などがいたむ)
\\	この仕事が当たれば大金持ちになれるぞ。	
\\	この 仕事[しごと]が 当[あ]たれば 大金持[おおがねも]ちになれるぞ。	当たる= (ぶつかる) 
\\	(触れる) 
\\	(日・光が) 
\\	(的中する) 
\\	(成功する) 
\\	(対戦する) 
\\	(相当する・対応する) 
\\	(適当である・当てはまる) 
\\	(果物などがいたむ)
\\	そのミュージカルはヨーロッパでは当たらなかった。	
\\	そのミュージカルはヨーロッパでは 当[あ]たらなかった。	当たる= (ぶつかる) 
\\	(触れる) 
\\	(日・光が) 
\\	(的中する) 
\\	(成功する) 
\\	(対戦する) 
\\	(相当する・対応する) 
\\	(適当である・当てはまる) 
\\	(果物などがいたむ)
\\	その映画はまるっきり当たらなかった。	
\\	その 映画[えいが]はまるっきり 当[あ]たらなかった。	当たる= (ぶつかる) 
\\	(触れる) 
\\	(日・光が) 
\\	(的中する) 
\\	(成功する) 
\\	(対戦する) 
\\	(相当する・対応する) 
\\	(適当である・当てはまる) 
\\	(果物などがいたむ)
\\	その芝居は大いに当たった。	
\\	その 芝居[しばい]は 大[おお]いに 当[あ]たった。	当たる= (ぶつかる) 
\\	(触れる) 
\\	(日・光が) 
\\	(的中する) 
\\	(成功する) 
\\	(対戦する) 
\\	(相当する・対応する) 
\\	(適当である・当てはまる) 
\\	(果物などがいたむ)
\\	あの手相見の言うことはよく当たる。	
\\	あの 手相[てそう] 見[み]の 言[い]うことはよく 当[あ]たる。	当たる= (ぶつかる) 
\\	(触れる) 
\\	(日・光が) 
\\	(的中する) 
\\	(成功する) 
\\	(対戦する) 
\\	(相当する・対応する) 
\\	(適当である・当てはまる) 
\\	(果物などがいたむ)
\\	君の予想、当たったよ。	
\\	君[きみ]の 予想[よそう]、 当[あ]たったよ。	当たる= (ぶつかる) 
\\	(触れる) 
\\	(日・光が) 
\\	(的中する) 
\\	(成功する) 
\\	(対戦する) 
\\	(相当する・対応する) 
\\	(適当である・当てはまる) 
\\	(果物などがいたむ)
\\	弾丸が当たらなかった。	
\\	弾丸[だんがん]が 当[あ]たらなかった。	当たる= (ぶつかる) 
\\	(触れる) 
\\	(日・光が) 
\\	(的中する) 
\\	(成功する) 
\\	(対戦する) 
\\	(相当する・対応する) 
\\	(適当である・当てはまる) 
\\	(果物などがいたむ)
\\	この部屋はとてもよく日が当たる。	
\\	この 部屋[へや]はとてもよく 日[ひ]が 当[あ]たる。	当たる= (ぶつかる) 
\\	(触れる) 
\\	(日・光が) 
\\	(的中する) 
\\	(成功する) 
\\	(対戦する) 
\\	(相当する・対応する) 
\\	(適当である・当てはまる) 
\\	(果物などがいたむ)
\\	日がまともに彼の顔に当たっていた。	
\\	日[ひ]がまともに 彼[かれ]の 顔[かお]に 当[あ]たっていた。	当たる= (ぶつかる) 
\\	(触れる) 
\\	(日・光が) 
\\	(的中する) 
\\	(成功する) 
\\	(対戦する) 
\\	(相当する・対応する) 
\\	(適当である・当てはまる) 
\\	(果物などがいたむ)
\\	私の部屋は東に窓があって朝は日が当たる。	
\\	私[わたし]の 部屋[へや]は 東[ひがし]に 窓[まど]があって 朝[あさ]は 日[ひ]が 当[あ]たる。	当たる= (ぶつかる) 
\\	(触れる) 
\\	(日・光が) 
\\	(的中する) 
\\	(成功する) 
\\	(対戦する) 
\\	(相当する・対応する) 
\\	(適当である・当てはまる) 
\\	(果物などがいたむ)
\\	この靴はかかとに当たる。	
\\	この 靴[くつ]はかかとに 当[あ]たる。	かかと= 
\\	当たる= (ぶつかる) 
\\	(触れる) 
\\	(日・光が) 
\\	(的中する) 
\\	(成功する) 
\\	(対戦する) 
\\	(相当する・対応する) 
\\	(適当である・当てはまる) 
\\	(果物などがいたむ)
\\	歓声が起こった。	
\\	歓声[かんせい]が 起[お]こった。	歓声=かんせい= 
\\	聴衆の間に大歓声があがった。	
\\	聴衆[ちょうしゅう]の 間[あいだ]に 大[だい] 歓声[かんせい]があがった。	歓声=かんせい= 
\\	大学生になって半年、やっと気持ちの余裕が出てきた。	
\\	大学生[だいがくせい]になって 半年[はんとし]、やっと 気持[きも]ちの 余裕[よゆう]が 出[で]てきた。	余裕=よゆう= 
\\	(余地・ゆとり) 
\\	(落ち着き) 
\\	うちはそんな余裕ないでしょ。	
\\	うちはそんな 余裕[よゆう]ないでしょ。	余裕=よゆう= 
\\	(余地・ゆとり) 
\\	(落ち着き) 
\\	そんなものを買う余裕はない。	
\\	そんなものを 買[か]う 余裕[よゆう]はない。	余裕=よゆう= 
\\	(余地・ゆとり) 
\\	(落ち着き) 
\\	今の彼は数年前に比べずっと裕福だ。	
\\	今[いま]の 彼[かれ]は 数[すう] 年[ねん] 前[まえ]に 比[くら]べずっと 裕福[ゆうふく]だ。	裕福=ゆうふく= 
\\	彼女の勇ましさには並み居る男たちも舌を巻いた。	
\\	彼女[かのじょ]の 勇[いさ]ましさには 並み居[なみい]る 男[おとこ]たちも 舌[した]を 巻[ま]いた。	勇ましい=いさましい= (勇敢) 
\\	並み居る=なみいる= 
\\	舌を巻く=したをまく= 
\\	彼らは負けるとわかっていたが勇ましく戦った。	
\\	彼[かれ]らは 負[ま]けるとわかっていたが 勇[いさ]ましく 戦[たたか]った。	勇ましい=いさましい= (勇敢) 
\\	洪水の恐れがあるので住民に避難命令が出された。	
\\	洪水[こうずい]の 恐[おそ]れがあるので 住民[じゅうみん]に 避難[ひなん] 命令[めいれい]が 出[だ]された。	
\\	まず子供たちを避難させた。	
\\	まず 子供[こども]たちを 避難[ひなん]させた。	
\\	私は現実から逃避するために読書をしていた。	
\\	私[わたし]は 現実[げんじつ]から 逃避[とうひ]するために 読書[どくしょ]をしていた。	
\\	ストライキは不可避だ。	
\\	ストライキは 不可避[ふかひ]だ。	
\\	増税は不可避だと首相は考えている。	
\\	増税[ぞうぜい]は 不可避[ふかひ]だと 首相[しゅしょう]は 考[かんが]えている。	
\\	この二つの問題を切り離して論じることはできない。	
\\	この 二[ふた]つの 問題[もんだい]を 切り離[きりはな]して 論[ろん]じることはできない。	切り離す= (分離する) 
\\	(分けて考える) 
\\	点線に沿ってクーポンを切り離してください。	
\\	点線[てんせん]に 沿[そ]ってクーポンを 切り離[きりはな]してください。	切り離す= (分離する) 
\\	(分けて考える) 
\\	納期が迫っているので休んではいられない。	
\\	納期[のうき]が 迫[せま]っているので 休[やす]んではいられない。	
\\	この難関を越えれば後は楽だ。	
\\	この 難関[なんかん]を 越[こ]えれば 後[あと]は 楽[らく]だ。	難関=なんかん= (障害) 
\\	(困難) 
\\	この計画実現の前に多くの難関が立ちはだかった。	
\\	この 計画[けいかく] 実現[じつげん]の 前[まえ]に 多[おお]くの 難関[なんかん]が 立[た]ちはだかった。	難関=なんかん= (障害) 
\\	(困難) 
\\	立ちはだかる= 
\\	弁護士になるには司法試験という難関がある。	
\\	弁護士[べんごし]になるには 司法[しほう] 試験[しけん]という 難関[なんかん]がある。	難関=なんかん= (障害) 
\\	(困難) 
\\	私は過去に縛られたくない。	
\\	私[わたし]は 過去[かこ]に 縛[しば]られたくない。	縛る=しばる= 
\\	仕事と家事に縛られてがんじがらめになっている。	
\\	仕事[しごと]と 家事[かじ]に 縛[しば]られてがんじがらめになっている。	縛る=しばる= 
\\	がんじがらめ= 
\\	勤務時間に縛られない職業に就きたい。	
\\	勤務[きんむ] 時間[じかん]に 縛[しば]られない 職業[しょくぎょう]に 就[つ]きたい。	縛る=しばる= 
\\	何事にも縛られない生活をしたいものだ。	
\\	何事[なにごと]にも 縛[しば]られない 生活[せいかつ]をしたいものだ。	縛る=しばる= 
\\	彼は柱に縛られていた。	
\\	彼[かれ]は 柱[はしら]に 縛[しば]られていた。	縛る=しばる= 
\\	彼女は余裕を漂わせて記者たちの質問に応じた。	
\\	彼女[かのじょ]は 余裕[よゆう]を 漂[ただよ]わせて 記者[きしゃ]たちの 質問[しつもん]に 応[おう]じた。	漂う=ただよう= 
\\	(雰囲気などが満ちる) 余裕=よゆう= 
\\	(余地・ゆとり) 
\\	(落ち着き) 
\\	チャンピオンには余裕が漂っていた。	
\\	チャンピオンには 余裕[よゆう]が 漂[ただよ]っていた。	漂う=ただよう= 
\\	(雰囲気などが満ちる) 余裕=よゆう= 
\\	(余地・ゆとり) 
\\	(落ち着き) 
\\	ドアを開けると異臭が漂ってきた。	
\\	ドアを 開[あ]けると 異臭[いしゅう]が 漂[ただよ]ってきた。	開ける=あける漂う=ただよう= 
\\	(雰囲気などが満ちる)
\\	若いころ彼は東南アジア諸国を漂い歩いていた。	
\\	若[わか]いころ 彼[かれ]は 東南アジア[とうなんあじあ] 諸国[しょこく]を 漂[ただよ]い 歩[ある]いていた。	漂う=ただよう= 
\\	(雰囲気などが満ちる)
\\	雲が空を漂っていく。	
\\	雲[くも]が 空[そら]を 漂[ただよ]っていく。	漂う=ただよう= 
\\	(雰囲気などが満ちる)
\\	救命ボートで一周間海を漂った。	
\\	救命[きゅうめい]ボートで 一周[いっしゅう] 間[かん] 海[うみ]を 漂[ただよ]った。	漂う=ただよう= 
\\	(雰囲気などが満ちる)
\\	この仕事はやっと半分まで漕ぎ着けた。	
\\	この 仕事[しごと]はやっと 半分[はんぶん]まで 漕ぎ着[こぎつ]けた。	漕ぎ着ける=こぎつける= 
\\	その夫婦の年齢はかなり隔たっていた。	
\\	その 夫婦[ふうふ]の 年齢[ねんれい]はかなり 隔[へだ]たっていた。	隔たる=へだたる= (離れる) 
\\	(差がある) 
\\	彼女の文才は我々から遠く隔たっている。	
\\	彼女[かのじょ]の 文才[ぶんさい]は 我々[われわれ]から 遠[とお]く 隔[へだ]たっている。	隔たる=へだたる= (離れる) 
\\	(差がある) 
\\	テレビなんていつもくだらない情報を垂れ流しているだけじゃないか。	
\\	テレビなんていつもくだらない 情報[じょうほう]を 垂[た]れ 流[なが]しているだけじゃないか。	垂れ流す=たれながす= 
\\	飼い犬が郵便配達人に噛み付いた。	
\\	飼い犬[かいいぬ]が 郵便[ゆうびん] 配達[はいたつ] 人[じん]に 噛み付[かみつ]いた。	噛み付く=かみつく= 
\\	彼女はひがみっぽい。	
\\	彼女[かのじょ]はひがみっぽい。	僻む=ひがむ= 
\\	兄だけに何か買ってやれば弟がひがむ。	
\\	兄[あに]だけに 何[なに]か 買[か]ってやれば 弟[おとうと]がひがむ。	僻む=ひがむ= 
\\	甘えっ子は自分の好きなようにならないとすぐすねる。	
\\	甘[あま]えっ 子[こ]は 自分[じぶん]の 好[す]きなようにならないとすぐすねる。	拗ねる=すねる= 
\\	これは私がスパイスにこだわって作ったチキンカレーです。	
\\	これは 私[わたし]がスパイスにこだわって 作[つく]ったチキンカレーです。	こだわる= (過度に気にする) 
\\	(妥協せず追求する) 
\\	過去にこだわるのをやめて、未来に目を向けなさい。	
\\	過去[かこ]にこだわるのをやめて、 未来[みらい]に 目[め]を 向[む]けなさい。	こだわる= (過度に気にする) 
\\	(妥協せず追求する) 
\\	重視するのは能力で、学歴にはこだわりません。	
\\	重視[じゅうし]するのは 能力[のうりょく]で、 学歴[がくれき]にはこだわりません。	こだわる= (過度に気にする) 
\\	(妥協せず追求する) 
\\	私は彼の良識を疑う。	
\\	私[わたし]は 彼[かれ]の 良識[りょうしき]を 疑[うたが]う。	
\\	暴走を抑えるために警察が出動した。	
\\	暴走[ぼうそう]を 抑[おさ]えるために 警察[けいさつ]が 出動[しゅつどう]した。	抑える=おさえる= 
\\	出動=しゅつどう= 
\\	彼は内部告発を未然に抑えた。	
\\	彼[かれ]は 内部[ないぶ] 告発[こくはつ]を 未然[みぜん]に 抑[おさ]えた。	内部告発= 
\\	未然=みぜん= 
\\	抑える=おさえる= 
\\	彼女はやっとのことで涙を抑えた。	
\\	彼女[かのじょ]はやっとのことで 涙[なみだ]を 抑[おさ]えた。	抑える=おさえる= 
\\	私は政府のやり方に怒りを抑え切れず反対運動を起こした。	
\\	私[わたし]は 政府[せいふ]のやり 方[かた]に 怒[いか]りを 抑[おさ]え 切[き]れず 反対[はんたい] 運動[うんどう]を 起[お]こした。	抑える=おさえる= 
\\	ついカッとなって自分を抑え切れず、彼を殴ってしまった。	
\\	ついカッとなって 自分[じぶん]を 抑[おさ]え 切[き]れず、 彼[かれ]を 殴[なぐ]ってしまった。	かっとなる= 
\\	抑える=おさえる= 
\\	私はずっと自分を抑えて良妻の仮面をかぶっていた。	
\\	私[わたし]はずっと 自分[じぶん]を 抑[おさ]えて 良妻[りょうさい]の 仮面[かめん]をかぶっていた。	抑える=おさえる= 
\\	良妻=りょうさい= 
\\	二酸化炭素の排出量をもっと抑えなかればならない。	
\\	二酸化炭素[にさんかたんそ]の 排出[はいしゅつ] 量[りょう]をもっと 抑[おさ]えなかればならない。	抑える=おさえる= 
\\	このごろ少し酒量を抑えている。	
\\	このごろ 少[すこ]し 酒量[しゅりょう]を 抑[おさ]えている。	抑える=おさえる= 
\\	贈賄側、収賄側のいずれも有罪となった。	
\\	贈賄[ぞうわい] 側[がわ]、 収賄[しゅうわい] 側[がわ]のいずれも 有罪[ゆうざい]となった。	
\\	人々は景気回復について悲観的な見通しを持っている。	
\\	人々[ひとびと]は 景気[けいき] 回復[かいふく]について 悲観[ひかん] 的[てき]な 見通[みとお]しを 持[も]っている。	
\\	調査の結果不正経理が発覚した。	
\\	調査[ちょうさ]の 結果[けっか] 不正[ふせい] 経理[けいり]が 発覚[はっかく]した。	発覚=はっかく= 隠していた悪事・陰謀などが明るみに出ること。
\\	彼はうそをついていたことが発覚した。	
\\	彼[かれ]はうそをついていたことが 発覚[はっかく]した。	発覚=はっかく= 隠していた悪事・陰謀などが明るみに出ること。
\\	入会手続きはご存じですか。	
\\	入会[にゅうかい] 手続[てつづ]きはご 存[ぞん]じですか。	
\\	退学の手続きはどうしたらいいのですか。	
\\	退学[たいがく]の 手続[てつづ]きはどうしたらいいのですか。	
\\	そんな面倒な手続きはやめにしようよ。	
\\	そんな 面倒[めんどう]な 手続[てつづ]きはやめにしようよ。	
\\	そんなくだらない心配はするな。	
\\	そんなくだらない 心配[しんぱい]はするな。	
\\	くだらないことに金を使うな。	
\\	くだらないことに 金[きん]を 使[つか]うな。	
\\	霧で見通しが利かなかった。	
\\	霧[きり]で 見通[みとお]しが 利[き]かなかった。	
\\	この道はくねくね曲がっていて見通しが悪い。	
\\	この 道[みち]はくねくね 曲[ま]がっていて 見通[みとお]しが 悪[わる]い。	
\\	どうもまだはっきりと見通しがつかない。	
\\	どうもまだはっきりと 見通[みとお]しがつかない。	
\\	いま失職したら将来の見通しが立たない。	
\\	いま 失職[しっしょく]したら 将来[しょうらい]の 見通[みとお]しが 立[た]たない。	
\\	見通しが崩れた。	
\\	見通[みとお]しが 崩[くず]れた。	
\\	見通しは明るい。	
\\	見通[みとお]しは 明[あか]るい。	
\\	核兵器廃絶への見通しは暗い。	
\\	核兵器[かくへいき] 廃絶[はいぜつ]への 見通[みとお]しは 暗[くら]い。	
\\	大雪で都内一円の交通が麻痺した。	
\\	大雪[おおゆき]で 都内[とない] 一円[いちえん]の 交通[こうつう]が 麻痺[まひ]した。	一円=いちえん= (一帯) 
\\	週末ともなると原宿には関東一円から若者が集まってくる。	
\\	週末[しゅうまつ]ともなると 原宿[はらじゅく]には 関東[かんとう] 一円[いちえん]から 若者[わかもの]が 集[あつ]まってくる。	一円=いちえん= (一帯) 
\\	会議は円滑に進行した。	
\\	会議[かいぎ]は 円滑[えんかつ]に 進行[しんこう]した。	円滑=えんかつ= 
\\	(支障のない) 
\\	両者の関係は円滑に行っていない。	
\\	両者[りょうしゃ]の 関係[かんけい]は 円滑[えんかつ]に 行[い]っていない。	円滑=えんかつ= 
\\	(支障のない) 
\\	単に形だけのことだよ。	
\\	単[たん]に 形[かたち]だけのことだよ。	
\\	地球温暖化の影響がいろいろな形で現れてきている。	
\\	地球[ちきゅう] 温暖[おんだん] 化[か]の 影響[えいきょう]がいろいろな 形[かたち]で 現[あらわ]れてきている。	
\\	こんな形で再会するとは思いませんでした。	
\\	こんな 形[かたち]で 再会[さいかい]するとは 思[おも]いませんでした。	
\\	机を互いに向かい合う形に並べてください。	
\\	机[つくえ]を 互[たが]いに 向かい合[むかいあ]う 形[かたち]に 並[なら]べてください。	
\\	鬱病はいろいろな形を取って現れる。	
\\	鬱病[うつびょう]はいろいろな 形[かたち]を 取[と]って 現[あらわ]れる。	
\\	人によって幸せの形はさまざまだ。	
\\	人[ひと]によって 幸[しあわ]せの 形[かたち]はさまざまだ。	
\\	彼の技術は円熟の域に達している。	
\\	彼[かれ]の 技術[ぎじゅつ]は 円熟[えんじゅく]の 域[いき]に 達[たっ]している。	円熟=えんじゅく= 
\\	あなたの舞台はすっかり円熟の境地ですね。	
\\	あなたの 舞台[ぶたい]はすっかり 円熟[えんじゅく]の 境地[きょうち]ですね。	円熟=えんじゅく= 
\\	麻薬密輸組織が摘発された。	
\\	麻薬[まやく] 密輸[みつゆ] 組織[そしき]が 摘発[てきはつ]された。	摘発=てきはつ= 
\\	最近密入国者の摘発が相次いでいる。	
\\	最近[さいきん] 密入国[みつにゅうこく] 者[しゃ]の 摘発[てきはつ]が 相次[あいつ]いでいる。	摘発=てきはつ= 
\\	その指摘は実に的を射ていた。	
\\	その 指摘[してき]は 実[じつ]に 的[まと]を 射[い]ていた。	的を射る=まと を いる= 
\\	彼女はその研究の成果を公にした。	
\\	彼女[かのじょ]はその 研究[けんきゅう]の 成果[せいか]を 公[おおやけ]にした。	公=おおやけ= 
\\	自分の名前は公にしたくない。	
\\	自分[じぶん]の 名前[なまえ]は 公[おおやけ]にしたくない。	公=おおやけ= 
\\	彼女の過去の経歴は既に公にされている。	
\\	彼女[かのじょ]の 過去[かこ]の 経歴[けいれき]は 既[すで]に 公[おおやけ]にされている。	公=おおやけ= 
\\	それはまだ公には発表されていない。	
\\	それはまだ 公[おおやけ]には 発表[はっぴょう]されていない。	公=おおやけ= 
\\	先端にダイヤモンドを植えたドリルで壁に穴をあけた。	
\\	先端[せんたん]にダイヤモンドを 植[う]えたドリルで 壁[かべ]に 穴[あな]をあけた。	植える=うえる= 
\\	父は家庭菜園に植える野菜の苗を買いに行った。	
\\	父[ちち]は 家庭[かてい] 菜園[さいえん]に 植[う]える 野菜[やさい]の 苗[なえ]を 買[か]いに 行[い]った。	苗=なえ= 
\\	植える=うえる= 
\\	この小説の作者は誰ですか。	
\\	この 小説[しょうせつ]の 作者[さくしゃ]は 誰[だれ]ですか。	
\\	この劇の作者はわからない。	
\\	この 劇[げき]の 作者[さくしゃ]はわからない。	
\\	落雷で巨木が真っ二つに裂けた。	
\\	落雷[らくらい]で 巨木[きょぼく]が 真[ま]っ 二[ぷた]つに 裂[さ]けた。	巨木=きょぼく= 大きな木
\\	国論は真っ二つに割れた。	
\\	国論[こくろん]は 真[ま]っ 二[ぷた]つに 割[わ]れた。	
\\	宿題を早く終わらせて遊びに行こう。	
\\	宿題[しゅくだい]を 早[はや]く 終[お]わらせて 遊[あそ]びに 行[い]こう。	
\\	私達の付き合いは半年前に終わりました。	
\\	私[わたし] 達[たち]の 付き合[つきあ]いは 半年[はんとし] 前[まえ]に 終[お]わりました。	
\\	戦争も終局に近づいていた。	
\\	戦争[せんそう]も 終局[しゅうきょく]に 近[ちか]づいていた。	終局=しゅうきょく= (終わり) 
\\	妻の不満は、その表情からも十分にうかがえた。	
\\	妻[つま]の 不満[ふまん]は、その 表情[ひょうじょう]からも 十分[じゅうぶん]にうかがえた。	窺う=うかがう= (のぞき見る) 
\\	(機会を待つ) 
\\	(察知する) 
\\	彼の服装は場所柄を弁えないものだ。	
\\	彼[かれ]の 服装[ふくそう]は 場所[ばしょ] 柄[がら]を 弁[わきま]えないものだ。	弁える=わきまえる= (弁別する) 
\\	彼は道理をわきまえた人だ。	
\\	彼[かれ]は 道理[どうり]をわきまえた 人[ひと]だ。	弁える=わきまえる= (弁別する) 
\\	その金は私が請け合って必ず返済させます。	
\\	その 金[かね]は 私[わたし]が 請け合[うけあ]って 必[かなら]ず 返済[へんさい]させます。	請け合う=うけあう= (引き受ける) 
\\	(保証する) 
\\	それは私が請け合います。	
\\	それは 私[わたし]が 請け合[うけあ]います。	請け合う=うけあう= (引き受ける) 
\\	(保証する) 
\\	彼が正直なことは私が請け合う。	
\\	彼[かれ]が 正直[しょうじき]なことは 私[わたし]が 請け合[うけあ]う。	請け合う=うけあう= (引き受ける) 
\\	(保証する) 
\\	箱が山と積み上げてある。	
\\	箱[はこ]が 山[やま]と 積み上[つみあ]げてある。	
\\	この道を行くと花屋に突き当たります。	
\\	この 道[みち]を 行[い]くと 花屋[はなや]に 突き当[つきあ]たります。	突き当たる= (衝突する) 
\\	(直面する) 
\\	まっすぐに行って突き当たったら右へ曲がりなさい。	
\\	まっすぐに 行[い]って 突き当[つきあ]たったら 右[みぎ]へ 曲[ま]がりなさい。	突き当たる= (衝突する) 
\\	(直面する) 
\\	我々の計画はいま壁に突き当たっている。	
\\	我々[われわれ]の 計画[けいかく]はいま 壁[かべ]に 突き当[つきあ]たっている。	突き当たる= (衝突する) 
\\	(直面する) 
\\	「ただし確かではない」と彼は付け加えた。	
\\	「ただし 確[たし]かではない」と 彼[かれ]は 付け加[つけくわ]えた。	
\\	病気で休んだ分を取り返さなければならない。	
\\	病気[びょうき]で 休[やす]んだ 分[ぶん]を 取り返[とりかえ]さなければならない。	
\\	日本語はこれまで多くの外来語を採り入れてきた。	
\\	日本語[にほんご]はこれまで 多[おお]くの 外来[がいらい] 語[ご]を 採り入[とりい]れてきた。	
\\	台風は直に通り過ぎるだろう。	
\\	台風[たいふう]は 直[じき]に 通り過[とおりす]ぎるだろう。	直に=じきに= 
\\	彼は何の挨拶もしないで通り過ぎた。	
\\	彼[かれ]は 何[なに]の 挨拶[あいさつ]もしないで 通り過[とおりす]ぎた。	
\\	ちょっと通りかかったもので寄ってみました。	
\\	ちょっと 通[とお]りかかったもので 寄[よ]ってみました。	通りかかる= 
\\	詳しいことは幹事に問い合わせてください。	
\\	詳[くわ]しいことは 幹事[かんじ]に 問い合[といあ]わせてください。	
\\	問い合わせてみたらその情報は事実無根だった。	
\\	問い合[といあ]わせてみたらその 情報[じょうほう]は 事実無根[じじつむこん]だった。	無根=むこん= 
\\	子供は父親に飛びついた。	
\\	子供[こども]は 父親[ちちおや]に 飛[と]びついた。	
\\	彼女はその願ってもない話に飛びついた。	
\\	彼女[かのじょ]はその 願[ねが]ってもない 話[はなし]に 飛[と]びついた。	願ってもない= 
\\	部屋の中へスズメが飛び込んできた。	
\\	部屋[へや]の 中[なか]へスズメが 飛び込[とびこ]んできた。	雀=すずめ= 
\\	その時、友人の逮捕というとんでもない知らせが飛び込んできた。	
\\	その 時[とき]、 友人[ゆうじん]の 逮捕[たいほ]というとんでもない 知[し]らせが 飛び込[とびこ]んできた。	
\\	レバーを引くとピンポン玉が飛び出した。	
\\	レバーを 引[ひ]くとピンポン 玉[だま]が 飛び出[とびだ]した。	
\\	子供がいきなり通りへ飛び出してきた。	
\\	子供[こども]がいきなり 通[とお]りへ 飛び出[とびだ]してきた。	
\\	そこ、釘が飛び出しているから気をつけて。	
\\	そこ、 釘[くぎ]が 飛び出[とびだ]しているから 気[き]をつけて。	
\\	友人に入場料を立て替えてもらった。	
\\	友人[ゆうじん]に 入場[にゅうじょう] 料[りょう]を 立て替[たてか]えてもらった。	立て替える= 
\\	雨が屋根に叩き付けるように降っていた。	
\\	雨[あめ]が 屋根[やね]に 叩き付[たたきつ]けるように 降[ふ]っていた。	
\\	試行錯誤の結果、フィラメントとして竹を使うアイデアにたどり着いた。	
\\	試行錯誤[しこうさくご]の 結果[けっか]、フィラメントとして 竹[たけ]を 使[つか]うアイデアにたどり 着[つ]いた。	辿り着く=たどりつく= 
\\	彼らは警察の手を逃れて国境を越え、メキシコにたどり着いた。	
\\	彼[かれ]らは 警察[けいさつ]の 手[て]を 逃[のが]れて 国境[こっきょう]を 越[こ]え、メキシコにたどり 着[つ]いた。	辿り着く=たどりつく= 
\\	靴下が擦り切れて穴が開いた。	
\\	靴下[くつした]が 擦り切[すりき]れて 穴[あな]が 開[ひら]いた。	擦り切れる=すりきれる= 
\\	几帳面な彼女は心が擦り切れて自殺してしまった。	
\\	几帳面[きちょうめん]な 彼女[かのじょ]は 心[こころ]が 擦り切[すりき]れて 自殺[じさつ]してしまった。	几帳面=きちょうめん= 
\\	擦り切れる=すりきれる= 
\\	点差が2点に開いた。	
\\	点差[てんさ]が2 点[てん]に 開[ひら]いた。	開く=ひらく= 
\\	(見えるようにする) 
\\	(広げる) 
\\	(開催する) 
\\	(開業する) (創立する); (口座を設ける) 
\\	(切り開く) 
\\	この技術が新しい時代を開いたのだ。	
\\	この 技術[ぎじゅつ]が 新[あたら]しい 時代[じだい]を 開[ひら]いたのだ。	開く=ひらく= 
\\	(見えるようにする) 
\\	(広げる) 
\\	(開催する) 
\\	(開業する) (創立する); (口座を設ける) 
\\	(切り開く) 
\\	彼女は自宅で英会話教室を開いている。	
\\	彼女[かのじょ]は 自宅[じたく]で 英会話[えいかいわ] 教室[きょうしつ]を 開[ひら]いている。	開く=ひらく= 
\\	(見えるようにする) 
\\	(広げる) 
\\	(開催する) 
\\	(開業する) (創立する); (口座を設ける) 
\\	(切り開く) 
\\	青木さんの送別会が開かれた。	
\\	青木[あおき]さんの 送別[そうべつ] 会[かい]が 開[ひら]かれた。	開く=ひらく= 
\\	(見えるようにする) 
\\	(広げる) 
\\	(開催する) 
\\	(開業する) (創立する); (口座を設ける) 
\\	(切り開く) 
\\	進行方向左側のとびらが開きます。	
\\	進行[しんこう] 方向[ほうこう] 左側[ひだりがわ]のとびらが 開[ひら]きます。	開く=ひらく= 
\\	(見えるようにする) 
\\	(広げる) 
\\	(開催する) 
\\	(開業する) (創立する); (口座を設ける) 
\\	(切り開く) 
\\	候補者は汗だくで支持を訴えていた。	
\\	候補[こうほ] 者[しゃ]は 汗[あせ]だくで 支持[しじ]を 訴[うった]えていた。	
\\	この運動は5分もやったらもう汗だくだ。	
\\	この 運動[うんどう]は5 分[ふん]もやったらもう 汗[あせ]だくだ。	
\\	私は汗っかきだ。	
\\	私[わたし]は 汗[あせ]っかきだ。	
\\	字を書いていると手がじっとり汗ばんでくる。	
\\	字[じ]を 書[か]いていると 手[て]がじっとり 汗[あせ]ばんでくる。	汗ばむ=あせばむ= 
\\	これは私が汗水流して働いてやっと設けた金だ。	
\\	これは 私[わたし]が 汗水[あせみず] 流[なが]して 働[はたら]いてやっと 設[もう]けた 金[きん]だ。	
\\	ウエートレスが、空いた皿をさっと片付けた。	
\\	ウエートレスが、 空[あ]いた 皿[さら]をさっと 片付[かたづ]けた。	開く・空く・明く=あく= 
\\	(始まる) 
\\	空いている所に詰めて座ってください。	
\\	空[あ]いている 所[ところ]に 詰[つ]めて 座[すわ]ってください。	開く・空く・明く=あく= 
\\	(始まる) 
\\	お隣(の席)空いていますか。	
\\	お 隣[となり](の 席[せき]) 空[あ]いていますか。	開く・空く・明く=あく= 
\\	(始まる) 
\\	明日は少し時間が空いている。	
\\	明日[あした]は 少[すこ]し 時間[じかん]が 空[あ]いている。	開く・空く・明く=あく= 
\\	(始まる) 
\\	明日午後空いていますか。	
\\	明日[あした] 午後[ごご] 空[あ]いていますか。	開く・空く・明く=あく= 
\\	(始まる) 
\\	そのボールペンが空いたら貸して下さい。	
\\	そのボールペンが 空[あ]いたら 貸[か]して 下[くだ]さい。	開く・空く・明く=あく= 
\\	(始まる) 
\\	いま手が空いています。	
\\	いま 手[て]が 空[あ]いています。	開く・空く・明く=あく= 
\\	(始まる) 
\\	ストレスで胃に穴があいた。	
\\	ストレスで 胃[い]に 穴[あな]があいた。	開く・空く・明く=あく= 
\\	(始まる) 
\\	その店は一晩中開いている。	
\\	その 店[みせ]は 一晩中[ひとばんじゅう] 開[あ]いている。	一晩中=ひとばんじゅう開く・空く・明く=あく= 
\\	(始まる) 
\\	日本では銀行は3時まで開いている。	
\\	日本[にほん]では 銀行[ぎんこう]は3 時[じ]まで 開[あ]いている。	開く・空く・明く=あく= 
\\	(始まる) 
\\	びっくり箱のふたが開いてピエロが跳び出した。	
\\	びっくり 箱[ばこ]のふたが 開[あ]いてピエロが 跳[と]び 出[だ]した。	開く・空く・明く=あく= 
\\	(始まる) 
\\	(ズボンの)チャックが開いてるよ。	
\\	(ズボンの)チャックが 開[あ]いてるよ。	開く・空く・明く=あく= 
\\	(始まる) 
\\	鍵が開かない。	
\\	鍵[かぎ]が 開[あ]かない。	開く・空く・明く=あく= 
\\	(始まる) 
\\	栓が開かない。	
\\	栓[せん]が 開[あ]かない。	栓=せん= 
\\	開く・空く・明く=あく= 
\\	(始まる) 
\\	ドアが少し開いた。	
\\	ドアが 少[すこ]し 開[あ]いた。	開く・空く・明く=あく= 
\\	(始まる) 
\\	明日は明日の風が吹く	
\\	(ことわざ)	明日[あした]は 明日[あした]の 風[かぜ]が 吹[ふ]く	
\\	果たして事実がその通りであるとしても、それは彼の責任ではない。	
\\	果[は]たして 事実[じじつ]がその 通[とお]りであるとしても、それは 彼[かれ]の 責任[せきにん]ではない。	果たして= (思った通り) 
\\	(本当に) 
\\	(疑問) 
\\	果たして事実だった。	
\\	果[は]たして 事実[じじつ]だった。	果たして= (思った通り) 
\\	(本当に) 
\\	(疑問) 
\\	果たして言ったとおりだろ。	
\\	果[は]たして 言[い]ったとおりだろ。	果たして= (思った通り) 
\\	(本当に) 
\\	(疑問) 
\\	彼は果たして失敗した。	
\\	彼[かれ]は 果[は]たして 失敗[しっぱい]した。	果たして= (思った通り) 
\\	(本当に) 
\\	(疑問) 
\\	その店ならこの道の突き当たりにあります。	
\\	その 店[みせ]ならこの 道[みち]の 突[つ]き 当[あ]たりにあります。	
\\	交番はあの突き当たりを右に曲がったところです。	
\\	交番[こうばん]はあの 突[つ]き 当[あ]たりを 右[みぎ]に 曲[ま]がったところです。	
\\	あの会社の社長は女性秘書のつまみ食いをするという噂だ。	
\\	あの 会社[かいしゃ]の 社長[しゃちょう]は 女性[じょせい] 秘書[ひしょ]のつまみ 食[ぐ]いをするという 噂[うわさ]だ。	つまみ食い=つまみぐい= 
\\	彼女の初めての詩集が刊行された。	
\\	彼女[かのじょ]の 初[はじ]めての 詩集[ししゅう]が 刊行[かんこう]された。	刊行=かんこう= 
\\	彼はいろいろの本を著した。	
\\	彼[かれ]はいろいろの 本[ほん]を 著[あらわ]した。	
\\	あの人には著書が多い。	
\\	あの 人[ひと]には 著書[ちょしょ]が 多[おお]い。	
\\	彼女は彫刻家として著名だ。	
\\	彼女[かのじょ]は 彫刻[ちょうこく] 家[か]として 著名[ちょめい]だ。	
\\	次の著作に取りかかっている。	
\\	次[つぎ]の 著作[ちょさく]に 取[と]りかかっている。	
\\	ほとんどの時間を著作に費やしている。	
\\	ほとんどの 時間[じかん]を 著作[ちょさく]に 費[つい]やしている。	
\\	彼は物理学の分野で顕著な業績を残した。	
\\	彼[かれ]は 物理[ぶつり] 学[がく]の 分野[ぶんや]で 顕著[けんちょ]な 業績[ぎょうせき]を 残[のこ]した。	顕著=けんちょ= 
\\	二つの車の性能に顕著な相違は見られない。	
\\	二[ふた]つの 車[くるま]の 性能[せいのう]に 顕著[けんちょ]な 相違[そうい]は 見[み]られない。	顕著=けんちょ= 
\\	女性の晩婚の傾向がますます顕著になってきている。	
\\	女性[じょせい]の 晩婚[ばんこん]の 傾向[けいこう]がますます 顕著[けんちょ]になってきている。	顕著=けんちょ= 
\\	彼は政治的には左翼である。	
\\	彼[かれ]は 政治[せいじ] 的[てき]には 左翼[さよく]である。	
\\	子供たちはテーブル一杯のご馳走を瞬く間に平らげた。	
\\	子供[こども]たちはテーブル一 杯[はい]のご 馳走[ちそう]を 瞬[またた]く 間[ま]に 平[たい]らげた。	瞬く間に=またたくまに= 
\\	瞬く間に楽しい一週間が過ぎた。	
\\	瞬[またた]く 間[ま]に 楽[たの]しい 一週間[いっしゅうかん]が 過[す]ぎた。	瞬く間に=またたくまに= 
\\	サッカーの試合は瞬時も目が離せない。	
\\	サッカーの 試合[しあい]は 瞬時[しゅんじ]も 目[め]が 離[はな]せない。	瞬時=しゅんじ= 
\\	その地震でその地域の建物は瞬時にして倒壊した。	
\\	その 地震[じしん]でその 地域[ちいき]の 建物[たてもの]は 瞬時[しゅんじ]にして 倒壊[とうかい]した。	瞬時=しゅんじ= 
\\	それは瞬時の出来事だった。	
\\	それは 瞬時[しゅんじ]の 出来事[できごと]だった。	瞬時=しゅんじ= 
\\	毛虫を足で踏みつぶした。	
\\	毛虫[けむし]を 足[あし]で 踏[ふ]みつぶした。	毛虫=けむし= 
\\	足に合わない靴を一日履いていたら、腰までいたくなった。	
\\	足[あし]に 合[あ]わない 靴[くつ]を 一日[いちにち] 履[は]いていたら、 腰[こし]までいたくなった。	
\\	靴がやっと足になじんできた。	
\\	靴[くつ]がやっと 足[あし]になじんできた。	なじむ= 
\\	左足にまめができてしまった。	
\\	左足[ひだりあし]にまめができてしまった。	
\\	プールに入る人は足を洗ってシャワーを浴びてください。	
\\	プールに 入[はい]る 人[ひと]は 足[あし]を 洗[あら]ってシャワーを 浴[あ]びてください。	
\\	右足をくじいた。	
\\	右足[みぎあし]をくじいた。	
\\	彼は腕組みし、足を机に載せて座っていた。	
\\	彼[かれ]は 腕組[うでぐ]みし、 足[あし]を 机[つくえ]に 載[の]せて 座[すわ]っていた。	
\\	川岸で遊んでいた男の子が足を滑らせて川に落ちた。	
\\	川岸[かわぎし]で 遊[あそ]んでいた 男の子[おとこのこ]が 足[あし]を 滑[すべ]らせて 川[かわ]に 落[お]ちた。	
\\	痛いのは右左どちらの足ですか。	
\\	痛[いた]いのは 右左[みぎひだり]どちらの 足[あし]ですか。	
\\	正座していたからすっかり足がしびれた。	
\\	正座[せいざ]していたからすっかり 足[あし]がしびれた。	しびれる= 
\\	不合格で、家に帰る足が重い。	
\\	不[ふ] 合格[ごうかく]で、 家[いえ]に 帰[かえ]る 足[あし]が 重[おも]い。	
\\	彼は足を傷めている。	
\\	彼[かれ]は 足[あし]を 傷[いた]めている。	
\\	多くの人が地雷で足を失った。	
\\	多[おお]くの 人[ひと]が 地雷[じらい]で 足[あし]を 失[うしな]った。	地雷=じらい= 
\\	足を切断しなければ彼は死んでしまう。	
\\	足[あし]を 切断[せつだん]しなければ 彼[かれ]は 死[し]んでしまう。	
\\	ダックスフントは足が短い犬だ。	
\\	ダックスフントは 足[あし]が 短[みじか]い 犬[いぬ]だ。	ダックスフント= 
\\	猫が汚れた足で家の中に上がってきた。	
\\	猫[ねこ]が 汚[よご]れた 足[あし]で 家[いえ]の 中[なか]に 上[あ]がってきた。	
\\	カメラマンは三脚の足を広げて撮影の準備に取りかかった。	
\\	カメラマンは 三脚[さんきゃく]の 足[あし]を 広[ひろ]げて 撮影[さつえい]の 準備[じゅんび]に 取[と]りかかった。	
\\	チーターは足が速い。	
\\	チーターは 足[あし]が 速[はや]い。	
\\	通勤の足として何を使っていますか。	
\\	通勤[つうきん]の 足[あし]として 何[なに]を 使[つか]っていますか。	
\\	寝る前は満腹になるまで食べない方がいい。	
\\	寝[ね]る 前[まえ]は 満腹[まんぷく]になるまで 食[た]べない 方[ほう]がいい。	満腹=まんぷく= 
\\	〜する= 
\\	すっかり満腹だ。	
\\	すっかり 満腹[まんぷく]だ。	満腹=まんぷく= 
\\	〜する= 
\\	男は空腹に耐えられなかった。	
\\	男[おとこ]は 空腹[くうふく]に 耐[た]えられなかった。	
\\	子供たちはしきりに空腹を訴えた。	
\\	子供[こども]たちはしきりに 空腹[くうふく]を 訴[うった]えた。	
\\	今回の実験では我々の仮説とまったく裏腹な結果が出た。	
\\	今回[こんかい]の 実験[じっけん]では 我々[われわれ]の 仮説[かせつ]とまったく 裏腹[うらはら]な 結果[けっか]が 出[で]た。	裏腹=うらはら= 
\\	彼はいつも言うこととやることが裏腹だ。	
\\	彼[かれ]はいつも 言[い]うこととやることが 裏腹[うらはら]だ。	裏腹=うらはら= 
\\	夕刊各紙は一斉にその事件を一面トップで報じた。	
\\	夕刊[ゆうかん] 各紙[かくし]は 一斉[いっせい]にその 事件[じけん]を 一面[いちめん]トップで 報[ほう]じた。	一斉に=いっせいに= 
\\	観客は一斉に立ち上がった。	
\\	観客[かんきゃく]は 一斉[いっせい]に 立ち上[たちあ]がった。	一斉に=いっせいに= 
\\	宿泊先に着いたら真っ先にシャワーを浴びた。	
\\	宿泊[しゅくはく] 先[さき]に 着[つ]いたら 真っ先[まっさき]にシャワーを 浴[あ]びた。	真っ先に=まっさきに= (最初に) 
\\	(何より前に) 
\\	退院したら真っ先に何食べたい?	
\\	退院[たいいん]したら 真っ先[まっさき]に 何[なに] 食[た]べたい?	真っ先に=まっさきに= (最初に) 
\\	(何より前に) 
\\	真っ先に起きるのは決まって母だ。	
\\	真っ先[まっさき]に 起[お]きるのは 決[き]まって 母[はは]だ。	真っ先に=まっさきに= (最初に) 
\\	(何より前に) 
\\	彼女は宝石を模造品とすり替えた。	
\\	彼女[かのじょ]は 宝石[ほうせき]を 模造[もぞう] 品[ひん]とすり 替[か]えた。	
\\	子供たちは一斉に校庭へ駆け出していった。	
\\	子供[こども]たちは 一斉[いっせい]に 校庭[こうてい]へ 駆け出[かけだ]していった。	駆け出す=かけだす= 
\\	彼女はエレベーターへ向かって駆け出した。	
\\	彼女[かのじょ]はエレベーターへ 向[む]かって 駆け出[かけだ]した。	駆け出す=かけだす= 
\\	始業1分前に教室に駆け込んだ。	
\\	始業[しぎょう] 1分[いっぷん] 前[まえ]に 教室[きょうしつ]に 駆け込[かけこ]んだ。	駆け込む=かけこむ= 
\\	彼らは大使館に駆け込んだ。	
\\	彼[かれ]らは 大使館[たいしかん]に 駆け込[かけこ]んだ。	駆け込む=かけこむ= 
\\	アムステルダムには街中に運河が張り巡らされている。	
\\	アムステルダムには 街[まち] 中[じゅう]に 運河[うんが]が 張り巡[はりめぐ]らされている。	街中=まちじゅう
\\	彼の料理には何とも言えない野趣がある。	
\\	彼[かれ]の 料理[りょうり]には 何[なん]とも 言[い]えない 野趣[やしゅ]がある。	野趣=やしゅ= 
\\	いたって無趣味な生活を送っています。	
\\	いたって 無趣味[むしゅみ]な 生活[せいかつ]を 送[おく]っています。	いたって= 
\\	微震があった。	
\\	微震[びしん]があった。	
\\	地雷と聞いただけで身震いする。	
\\	地雷[じらい]と 聞[き]いただけで 身震[みぶる]いする。	身震い=みぶるい= 
\\	考えただけでも身震いがする。	
\\	考[かんが]えただけでも 身震[みぶる]いがする。	身震い=みぶるい= 
\\	その殺人のニュースに町中が恐怖で身震いした。	
\\	その 殺人[さつじん]のニュースに 町中[まちじゅう]が 恐怖[きょうふ]で 身震[みぶる]いした。	身震い=みぶるい= 
\\	町中=まちじゅう= 
\\	町中=まちなか= 
\\	このビルは震度7にも耐えられるよう設計されています。	
\\	このビルは 震度[しんど]7にも 耐[た]えられるよう 設計[せっけい]されています。	
\\	体に感じる余震が10回観測された。	
\\	体[からだ]に 感[かん]じる 余震[よしん]が 10回[じゅうかい] 観測[かんそく]された。	
\\	この車は震動がひどい。	
\\	この 車[くるま]は 震動[しんどう]がひどい。	震動=しんどう= 
\\	名古屋では震動がことに強かった。	
\\	名古屋[なごや]では 震動[しんどう]がことに 強[つよ]かった。	震動=しんどう= 
\\	ことに= 
\\	彼の唇は震えていた。	
\\	彼[かれ]の 唇[くちびる]は 震[ふる]えていた。	震える=ふるえる= 
\\	(声が) 
\\	手が震えて字が書けない。	
\\	手[て]が 震[ふる]えて 字[じ]が 書[か]けない。	震える=ふるえる= 
\\	(声が) 
\\	字が震えていた。	
\\	字[じ]が 震[ふる]えていた。	震える=ふるえる= 
\\	(声が) 
\\	彼の声は感動で震えていた。	
\\	彼[かれ]の 声[こえ]は 感動[かんどう]で 震[ふる]えていた。	震える=ふるえる= 
\\	(声が) 
\\	そう言う彼の声は震えていた。	
\\	そう 言[い]う 彼[かれ]の 声[こえ]は 震[ふる]えていた。	震える=ふるえる= 
\\	(声が) 
\\	その映画は震えるほど怖い。	
\\	その 映画[えいが]は 震[ふる]えるほど 怖[こわ]い。	震える=ふるえる= 
\\	(声が) 
\\	その報告書には事実を隠蔽しようとする作為の跡が見える。	
\\	その 報告[ほうこく] 書[しょ]には 事実[じじつ]を 隠蔽[いんぺい]しようとする 作為[さくい]の 跡[あと]が 見[み]える。	作為=さくい= (作り事) 
\\	(見せかけ) 
\\	データに作為の跡が見られる。	
\\	データに 作為[さくい]の 跡[あと]が 見[み]られる。	作為=さくい= (作り事) 
\\	(見せかけ) 
\\	次の協議事項は何ですか。	
\\	次[つぎ]の 協議[きょうぎ] 事項[じこう]は 何[なに]ですか。	
\\	ちょっと背比べをしてみよう。	
\\	ちょっと 背[せ] 比[くら]べをしてみよう。	背比べ=せいくらべ= 
\\	女手一つで5人の子供を育て上げた。	
\\	女手[おんなで] 一[ひと]つで 5人[ごにん]の 子供[こども]を 育て上[そだてあ]げた。	
\\	注文のお品はしあさってまでにお届けできます。	
\\	注文[ちゅうもん]のお 品[しな]はしあさってまでにお 届[とど]けできます。	お品=おしな明々後日=しあさって= 
\\	支払いはしあさってまで待ってください。	
\\	支払[しはら]いはしあさってまで 待[ま]ってください。	明々後日=しあさって= 
\\	一週間以内に仕上げてもらいたい。	
\\	一週間[いっしゅうかん] 以内[いない]に 仕上[しあ]げてもらいたい。	仕上げる=しあげる= (仕事などを完了させる) 
\\	草の根を分けても彼を捜し出してみせる。	
\\	草の根[くさのね]を 分[わ]けても 彼[かれ]を 捜し出[さがしだ]してみせる。	
\\	相場が落ち着いた。	
\\	相場[そうば]が 落ち着[おちつ]いた。	落ち着く・落ち付く=おちつく= 
\\	(新居などに) 
\\	(決着する) 
\\	騒動がやっと落ち着きました。	
\\	騒動[そうどう]がやっと 落ち着[おちつ]きました。	落ち着く・落ち付く=おちつく= 
\\	(新居などに) 
\\	(決着する) 
\\	少しは痛みが落ち着きましたか。	
\\	少[すこ]しは 痛[いた]みが 落ち着[おちつ]きましたか。	落ち着く・落ち付く=おちつく= 
\\	(新居などに) 
\\	(決着する) 
\\	今度の会社には海外への転勤がないから、しばらくは国内に落ち着いていられると思う。	
\\	今度[こんど]の 会社[かいしゃ]には 海外[かいがい]への 転勤[てんきん]がないから、しばらくは 国内[こくない]に 落ち着[おちつ]いていられると 思[おも]う。	落ち着く・落ち付く=おちつく= 
\\	(新居などに) 
\\	(決着する) 
\\	やっと私達は新居に落ち着きました。	
\\	やっと 私[わたし] 達[たち]は 新居[しんきょ]に 落ち着[おちつ]きました。	落ち着く・落ち付く=おちつく= 
\\	(新居などに) 
\\	(決着する) 
\\	この部屋では心が落ち着かない。	
\\	この 部屋[へや]では 心[こころ]が 落ち着[おちつ]かない。	落ち着く・落ち付く=おちつく= 
\\	(新居などに) 
\\	(決着する) 
\\	私には自分の部屋が一番落ち着ける場所です。	
\\	私[わたし]には 自分[じぶん]の 部屋[へや]が 一番[いちばん] 落ち着[おちつ]ける 場所[ばしょ]です。	落ち着く・落ち付く=おちつく= 
\\	(新居などに) 
\\	(決着する) 
\\	新しい眼鏡がどうも落ち着かない。	
\\	新[あたら]しい 眼鏡[めがね]がどうも 落ち着[おちつ]かない。	落ち着く・落ち付く=おちつく= 
\\	(新居などに) 
\\	(決着する) 
\\	チェックのブラウスにチェックのズボンでは落ち着かないよ。	
\\	チェックのブラウスにチェックのズボンでは 落ち着[おちつ]かないよ。	落ち着く・落ち付く=おちつく= 
\\	(新居などに) 
\\	(決着する) 
\\	彼はいつもシックで落ち着いた感じの服装をしている。	
\\	彼[かれ]はいつもシックで 落ち着[おちつ]いた 感[かん]じの 服装[ふくそう]をしている。	落ち着く・落ち付く=おちつく= 
\\	(新居などに) 
\\	(決着する) 
\\	ずいぶん落ち着いたいい感じの美術館ですね。	
\\	ずいぶん 落ち着[おちつ]いたいい 感[かん]じの 美術館[びじゅつかん]ですね。	落ち着く・落ち付く=おちつく= 
\\	(新居などに) 
\\	(決着する) 
\\	クラブ活動もなくなってこれで落ち着いて勉強ができる。	
\\	クラブ 活動[かつどう]もなくなってこれで 落ち着[おちつ]いて 勉強[べんきょう]ができる。	落ち着く・落ち付く=おちつく= 
\\	(新居などに) 
\\	(決着する) 
\\	ここはうるさくて落ち着いて話もできない。	
\\	ここはうるさくて 落ち着[おちつ]いて 話[はなし]もできない。	落ち着く・落ち付く=おちつく= 
\\	(新居などに) 
\\	(決着する) 
\\	彼らは約束の時刻に駅で落ち合った。	
\\	彼[かれ]らは 約束[やくそく]の 時刻[じこく]に 駅[えき]で 落ち合[おちあ]った。	落ち合う=おちあう= 
\\	人を押しのけて満員電車に乗り込んだ。	
\\	人[ひと]を 押[お]しのけて 満員[まんいん] 電車[でんしゃ]に 乗り込[のりこ]んだ。	押し除ける=おしのける= 
\\	人を押しのけてまで成功しようとは思わない。	
\\	人[ひと]を 押[お]しのけてまで 成功[せいこう]しようとは 思[おも]わない。	押し除ける=おしのける= 
\\	大人たちは変革を求める若者の気持ちをいつも押さえつけようとする。	
\\	大人[おとな]たちは 変革[へんかく]を 求[もと]める 若者[わかもの]の 気持[きも]ちをいつも 押[お]さえつけようとする。	押さえつける・抑えつける=おさえつける= 
\\	(抑圧する) 
\\	ワンマン社長に頭を押さえつけられていて、社員は自由に意見を述べることができなかった。	
\\	ワンマン 社長[しゃちょう]に 頭[あたま]を 押[お]さえつけられていて、 社員[しゃいん]は 自由[じゆう]に 意見[いけん]を 述[の]べることができなかった。	ワンマン= 
\\	押さえつける・抑えつける=おさえつける= 
\\	(抑圧する) 
\\	はやる馬を押さえつけるのに苦労した。	
\\	はやる 馬[うま]を 押[お]さえつけるのに 苦労[くろう]した。	押さえつける・抑えつける=おさえつける= 
\\	(抑圧する) 
\\	その説は広く採用された。	
\\	その 説[せつ]は 広[ひろ]く 採用[さいよう]された。	採用=さいよう= (方法や意見の) 
\\	(任用) 
\\	(雇用) 
\\	この教授法は一般に採用されている。	
\\	この 教授[きょうじゅ] 法[ほう]は 一般[いっぱん]に 採用[さいよう]されている。	採用=さいよう= (方法や意見の) 
\\	(任用) 
\\	(雇用) 
\\	来年は新卒採用の予定はない。	
\\	来年[らいねん]は 新卒[しんそつ] 採用[さいよう]の 予定[よてい]はない。	採用=さいよう= (方法や意見の) 
\\	(任用) 
\\	(雇用) 
\\	彼の企てたことは何でも成功する。	
\\	彼[かれ]の 企[くわだ]てたことは 何[なに]でも 成功[せいこう]する。	
\\	その非難は不当だ。	
\\	その 非難[ひなん]は 不当[ふとう]だ。	
\\	彼女は不当な差別を受けた。	
\\	彼女[かのじょ]は 不当[ふとう]な 差別[さべつ]を 受[う]けた。	
\\	彼は何も非難すべきことはしていない。	
\\	彼[かれ]は 何[なに]も 非難[ひなん]すべきことはしていない。	批難・非難=ひなん= 
\\	彼は首相を非難する演説を行った。	
\\	彼[かれ]は 首相[しゅしょう]を 非難[ひなん]する 演説[えんぜつ]を 行[おこな]った。	批難・非難=ひなん= 
\\	各紙とも政府の外交政策を一斉に非難した。	
\\	各紙[かくし]とも 政府[せいふ]の 外交[がいこう] 政策[せいさく]を 一斉[いっせい]に 非難[ひなん]した。	批難・非難=ひなん= 
\\	首相の発言に対し激しい非難がわき上がった。	
\\	首相[しゅしょう]の 発言[はつげん]に 対[たい]し 激[はげ]しい 非難[ひなん]がわき 上[あ]がった。	批難・非難=ひなん= 
\\	増税に対する激しい非難の声が上がった。	
\\	増税[ぞうぜい]に 対[たい]する 激[はげ]しい 非難[ひなん]の 声[こえ]が 上[あ]がった。	批難・非難=ひなん= 
\\	彼女は消極的な性格だ。	
\\	彼女[かのじょ]は 消極[しょうきょく] 的[てき]な 性格[せいかく]だ。	消極的=しょうきょくてき= 
\\	会社は新システムの導入には消極的だ。	
\\	会社[かいしゃ]は 新[しん]システムの 導入[どうにゅう]には 消極[しょうきょく] 的[てき]だ。	消極的=しょうきょくてき= 
\\	私は大学に合格できて有頂天だった。	
\\	私[わたし]は 大学[だいがく]に 合格[ごうかく]できて 有頂天[うちょうてん]だった。	有頂天=うちょうてん= 
\\	彼女はおだてられて有頂天になった。	
\\	彼女[かのじょ]はおだてられて 有頂天[うちょうてん]になった。	おだてる= 
\\	有頂天=うちょうてん= 
\\	それを聞いたら彼は有頂天になって喜ぶだろう。	
\\	それを 聞[き]いたら 彼[かれ]は 有頂天[うちょうてん]になって 喜[よろこ]ぶだろう。	有頂天=うちょうてん= 
\\	たき火が煙くて涙が出た。	
\\	たき 火[び]が 煙[けむ]くて 涙[なみだ]が 出[で]た。	たき火=たきび= 
\\	煙い=けむい= 
\\	彼女は口うるさいので職場では煙たがられている。	
\\	彼女[かのじょ]は 口[くち]うるさいので 職場[しょくば]では 煙[けむ]たがられている。	煙たがる=けむたがる= 
\\	この店は不完全分煙だ。	
\\	この 店[みせ]は 不完全[ふかんぜん] 分煙[ぶんえん]だ。	分煙=ぶんえん= 
\\	煙突が煙を吐いていた。	
\\	煙突[えんとつ]が 煙[けむり]を 吐[は]いていた。	煙突=えんとつ= 
\\	私はずっと日陰の身だった。	
\\	私[わたし]はずっと 日陰[ひかげ]の 身[み]だった。	日陰・日蔭・日影=ひかげ= 
\\	ここは昼過ぎには日陰になる。	
\\	ここは 昼過[ひるす]ぎには 日陰[ひかげ]になる。	日陰・日蔭・日影=ひかげ= 
\\	今日は日陰で25度の気温がある。	
\\	今日[きょう]は 日陰[ひかげ]で25 度[ど]の 気温[きおん]がある。	日陰・日蔭・日影=ひかげ= 
\\	地震がすべてを一瞬にして奪い去った。	
\\	地震[じしん]がすべてを 一瞬[いっしゅん]にして 奪[うば]い 去[さ]った。	奪い去る=うばいさる= 
\\	その点は特に考慮したいと思う。	
\\	その 点[てん]は 特[とく]に 考慮[こうりょ]したいと 思[おも]う。	考慮=こうりょ= 
\\	これを第一に考慮しなければならない。	
\\	これを 第一[だいいち]に 考慮[こうりょ]しなければならない。	考慮=こうりょ= 
\\	この決定に際し彼の年齢は考慮に入っていません。	
\\	この 決定[けってい]に 際[さい]し 彼[かれ]の 年齢[ねんれい]は 考慮[こうりょ]に 入[はい]っていません。	考慮=こうりょ= 
\\	採否に関し国籍は考慮の対象にしない。	
\\	採否[さいひ]に 関[かん]し 国籍[こくせき]は 考慮[こうりょ]の 対象[たいしょう]にしない。	考慮=こうりょ= 
\\	車窓から美しい田園風景が見える。	
\\	車窓[しゃそう]から 美[うつく]しい 田園[でんえん] 風景[ふうけい]が 見[み]える。	田園=でんえん= 
\\	(田舎) 
\\	数年前までこの辺りは一面の田んぼでした。	
\\	数[すう] 年[ねん] 前[まえ]までこの 辺[あた]りは 一面[いちめん]の 田[た]んぼでした。	田んぼ=たんぼ= 
\\	銀行まで行くのが手間なので私はテレホンバンキングを利用している。	
\\	銀行[ぎんこう]まで 行[い]くのが 手間[てま]なので 私[わたし]はテレホンバンキングを 利用[りよう]している。	手間=てま= (時) 
\\	(苦労) 
\\	お手間は取らせません。	
\\	お 手間[てま]は 取[と]らせません。	手間=てま= (時) 
\\	(苦労) 
\\	お手間をおかけして申し訳ありません。	
\\	お 手間[てま]をおかけして 申し訳[もうしわけ]ありません。	手間=てま= (時) 
\\	(苦労) 
\\	作る手間を考えるとつい冷凍食品を買ってしまう。	
\\	作[つく]る 手間[てま]を 考[かんが]えるとつい 冷凍[れいとう] 食品[しょくひん]を 買[か]ってしまう。	手間=てま= (時) 
\\	(苦労) 
\\	このウェディングドレスにはなかなか手間がかかっている。	
\\	このウェディングドレスにはなかなか 手間[てま]がかかっている。	手間=てま= (時) 
\\	(苦労) 
\\	二人はかつて愛を交わした間柄だった。	
\\	二人[ふたり]はかつて 愛[あい]を 交[か]わした 間柄[あいだがら]だった。	交わす=かわす= (やりとりする) 
\\	その日私達は一言も交わさなかった。	
\\	その 日[ひ] 私[わたし] 達[たち]は 一言[ひとこと]も 交[か]わさなかった。	交わす=かわす= (やりとりする) 
\\	終わった後で何を言ってもむなしい。	
\\	終[お]わった 後[のち]で 何[なに]を 言[い]ってもむなしい。	空しい=むなしい= 
\\	その地震によって多くの家屋が倒壊した。	
\\	その 地震[じしん]によって 多[おお]くの 家屋[かおく]が 倒壊[とうかい]した。	家屋=かおく= 
\\	日本の家屋は火災に弱い。	
\\	日本の 家屋[かおく]は 火災[かさい]に 弱[よわ]い。	家屋=かおく= 
\\	その車は性能面ではこの車よりはるかに下だ。	
\\	その 車[くるま]は 性能[せいのう] 面[めん]ではこの 車[くるま]よりはるかに 下[した]だ。	
\\	あの人の運も下がり目だ。	
\\	あの 人[ひと]の 運[うん]も 下がり目[さがりめ]だ。	下がり目=さがりめ= 
\\	物価は下がり目だ。	
\\	物価[ぶっか]は 下がり目[さがりめ]だ。	下がり目=さがりめ= 
\\	店の前には「営業中」の札が下がっていた。	
\\	店[みせ]の 前[まえ]には
\\	営業[えいぎょう] 中[ちゅう]」の 札[さつ]が 下[さ]がっていた。	
\\	このところ毎日気温が下がっている。	
\\	このところ 毎日[まいにち] 気温[きおん]が 下[さ]がっている。	
\\	男優の人気は結婚で下がることもある。	
\\	男優[だんゆう]の 人気[にんき]は 結婚[けっこん]で 下[さ]がることもある。	
\\	新しい段落は1字下げて書き始める。	
\\	新[あたら]しい 段落[だんらく]は 1字[いちじ] 下[さ]げて 書[か]き 始[はじ]める。	
\\	今月に入って株価がようやく下げ止まった。	
\\	今月[こんげつ]に 入[はい]って 株価[かぶか]がようやく 下[さ]げ 止[と]まった。	
\\	わざわざ空港まで迎えに来てくださった。	
\\	わざわざ 空港[くうこう]まで 迎[むか]えに 来[き]てくださった。	
\\	昨日から腹を下している。	
\\	昨日[きのう]から 腹[はら]を 下[くだ]している。	下す=くだす= (言い渡す); (最終的に行う); (下痢をする) 
\\	最終決定を下す時が来た。	
\\	最終[さいしゅう] 決定[けってい]を 下[くだ]す 時[とき]が 来[き]た。	下す=くだす= (言い渡す); (最終的に行う); (下痢をする) 
\\	私なんか会社では下っ端ですから発言権なんてないですよ。	
\\	私[わたし]なんか 会社[かいしゃ]では 下っ端[したっぱ]ですから 発言[はつげん] 権[けん]なんてないですよ。	下っ端=したっぱ= 
\\	急いで階段を下りた。	
\\	急[いそ]いで 階段[かいだん]を 下[お]りた。	下りる・降りる=おりる= 
\\	子猫が屋根から下りられないでいる。	
\\	子猫[こねこ]が 屋根[やね]から 下[お]りられないでいる。	下りる・降りる=おりる= 
\\	その歌手は舞台から下りてきて客と握手した。	
\\	その 歌手[かしゅ]は 舞台[ぶたい]から 下[お]りてきて 客[きゃく]と 握手[あくしゅ]した。	下りる・降りる=おりる= 
\\	ワシが餌をさらいに下りてきた。	
\\	ワシが 餌[えさ]をさらいに 下[お]りてきた。	下りる・降りる=おりる= 
\\	今年から年金が下りるようになった。	
\\	今年[ことし]から 年金[ねんきん]が 下[お]りるようになった。	下りる・降りる=おりる= 
\\	許可証は簡単に下りる。	
\\	許可[きょか] 証[しょう]は 簡単[かんたん]に 下[お]りる。	下りる・降りる=おりる= 
\\	降りる方からお先に願います。	
\\	降[お]りる 方[かた]からお 先[さき]に 願[ねが]います。	下りる・降りる=おりる= 
\\	降りる駅を間違えた。	
\\	降[お]りる 駅[えき]を 間違[まちが]えた。	下りる・降りる=おりる= 
\\	私は車に酔ったので途中で降りて歩いて帰った。	
\\	私[わたし]は 車[くるま]に 酔[よ]ったので 途中[とちゅう]で 降[お]りて 歩[ある]いて 帰[かえ]った。	下りる・降りる=おりる= 
\\	ヒースロー空港に降り立った時、私は不安でいっぱいだった。	
\\	ヒースロー 空港[くうこう]に 降[お]り 立[た]った 時[とき]、 私[わたし]は 不安[ふあん]でいっぱいだった。	降り立つ=おりたつ= (乗り物から) 
\\	被告に禁固10年の判決が下った。	
\\	被告[ひこく]に 禁固[きんこ]10 年[ねん]の 判決[はんけつ]が 下[くだ]った。	禁固=きんこ= 
\\	死者は300人を下らない。	
\\	死者[ししゃ]は 300人[さんぴゃくにん]を 下[くだ]らない。	
\\	子猫を木から下ろしてやった。	
\\	子猫[こねこ]を 木[き]から 下[お]ろしてやった。	
\\	彼はおばあさんを車いすから降ろしてあげた。	
\\	彼[かれ]はおばあさんを 車[くるま]いすから 降[お]ろしてあげた。	車いす=くるまいす= 
\\	あの秀才がどうしたわけか今学期は下位を低迷している。	
\\	あの 秀才[しゅうさい]がどうしたわけか 今[こん] 学期[がっき]は 下位[かい]を 低迷[ていめい]している。	
\\	火事はもうほとんど下火になった。	
\\	火事[かじ]はもうほとんど 下火[したび]になった。	
\\	大流行していたそのゲーム・ソフトの人気も下火になりかけてきた。	
\\	大[だい] 流行[りゅうこう]していたそのゲーム・ソフトの 人気[にんき]も 下火[したび]になりかけてきた。	
\\	入場者数は予想をはるかに下回った。	
\\	入場[にゅうじょう] 者[しゃ] 数[すう]は 予想[よそう]をはるかに 下回[したまわ]った。	
\\	出発する前にこのことはすっかり始末をつけておかねばならない。	
\\	出発[しゅっぱつ]する 前[まえ]にこのことはすっかり 始末[しまつ]をつけておかねばならない。	始末=しまつ= (処理) 
\\	(事の次第・事情) 
\\	放射性廃棄物はその処理を誤ると始末に負えない。	
\\	放射[ほうしゃ] 性[せい] 廃棄[はいき] 物[ぶつ]はその 処理[しょり]を 誤[あやま]ると 始末[しまつ]に 負[お]えない。	始末=しまつ= (処理) 
\\	(事の次第・事情) 
\\	始末に負えない= 
\\	こんな始末になってしまって大変申し訳ありません。	
\\	こんな 始末[しまつ]になってしまって 大変[たいへん] 申し訳[もうしわけ]ありません。	始末=しまつ= (処理) 
\\	(事の次第・事情) 
\\	彼は年を取って自分の始末もできなくなった。	
\\	彼[かれ]は 年[とし]を 取[と]って 自分[じぶん]の 始末[しまつ]もできなくなった。	始末=しまつ= (処理) 
\\	(事の次第・事情) 
\\	彼は行儀が悪い。	
\\	彼[かれ]は 行儀[ぎょうぎ]が 悪[わる]い。	行儀=ぎょうぎ= 
\\	人の前であくびをするなんて行儀が悪いね。	
\\	人[ひと]の 前[まえ]であくびをするなんて 行儀[ぎょうぎ]が 悪[わる]いね。	行儀=ぎょうぎ= 
\\	以後あんな所へ行ってはならない。	
\\	以後[いご]あんな 所[ところ]へ 行[い]ってはならない。	
\\	酒は以後慎みなさい。	
\\	酒[さけ]は 以後[いご] 慎[つつし]みなさい。	慎む=つつしむ= 
\\	それ以後何の音沙汰もない。	
\\	それ 以後[いご] 何[なん]の 音沙汰[おとさた]もない。	音沙汰=おとさた= 
\\	小学生の時に犬に噛まれてそれ以後犬が嫌いになった。	
\\	小学生[しょうがくせい]の 時[とき]に 犬[いぬ]に 噛[か]まれてそれ 以後[いご] 犬[いぬ]が 嫌[きら]いになった。	
\\	それ以後の彼を誰も知らない。	
\\	それ 以後[いご]の 彼[かれ]を 誰[だれ]も 知[し]らない。	
\\	石油で大儲けをして以後彼は安楽に暮らした。	
\\	石油[せきゆ]で 大儲[おおもう]けをして 以後[いご] 彼[かれ]は 安楽[あんらく]に 暮[く]らした。	大儲け=おおもうけ= 非常に大きな利益を得ること。
\\	恐ろしい光景を目撃した。	
\\	恐[おそ]ろしい 光景[こうけい]を 目撃[もくげき]した。	目撃=もくげき= 
\\	男が女性のバッグから財布を抜き取るところを目撃した。	
\\	男[おとこ]が 女性[じょせい]のバッグから 財布[さいふ]を 抜き取[ぬきと]るところを 目撃[もくげき]した。	目撃=もくげき= 
\\	ボイラー屋の暑さで目眩がした。	
\\	ボイラー 屋[や]の 暑[あつ]さで 目眩[めまい]がした。	目眩=めまい= 
\\	天井がぐるぐる回るような目眩がする。	
\\	天井[てんじょう]がぐるぐる 回[まわ]るような 目眩[めまい]がする。	目眩=めまい= 
\\	財布は見つかったが、中身が抜き取られていた。	
\\	財布[さいふ]は 見[み]つかったが、 中身[なかみ]が 抜き取[ぬきと]られていた。	中身・中味=なかみ= (内容) 
\\	(実質) 
\\	あの男は中身がない。	
\\	あの 男[おとこ]は 中身[なかみ]がない。	中身・中味=なかみ= (内容) 
\\	(実質) 
\\	あの人の話は中身が濃い。	
\\	あの 人[ひと]の 話[はなし]は 中身[なかみ]が 濃[こ]い。	中身・中味=なかみ= (内容) 
\\	(実質) 
\\	かばんの中身は何ですか。	
\\	かばんの 中身[なかみ]は 何[なに]ですか。	中身・中味=なかみ= (内容) 
\\	(実質) 
\\	こんな身形で人前に出ることはできない。	
\\	こんな 身形[みなり]で 人前[ひとまえ]に 出[で]ることはできない。	身形=みなり= 
\\	私は身形などは構わない。	
\\	私[わたし]は 身形[みなり]などは 構[かま]わない。	身形=みなり= 
\\	中東情勢は暗雲に覆われている。	
\\	中東[ちゅうとう] 情勢[じょうせい]は 暗雲[あんうん]に 覆[おお]われている。	覆う=おおう= 
\\	(おおい隠す) 
\\	(包む) 
\\	冷ややかな空気が会場を覆っていた。	
\\	冷[ひ]ややかな 空気[くうき]が 会場[かいじょう]を 覆[おお]っていた。	覆う=おおう= 
\\	(おおい隠す) 
\\	(包む) 
\\	それは覆うべからざる事実だ。	
\\	それは 覆[おお]うべからざる 事実[じじつ]だ。	覆う=おおう= 
\\	(おおい隠す) 
\\	(包む) 
\\	彼は新聞記者に顔を見られまいとして帽子で顔を覆った。	
\\	彼[かれ]は 新聞[しんぶん] 記者[きしゃ]に 顔[かお]を 見[み]られまいとして 帽子[ぼうし]で 顔[かお]を 覆[おお]った。	
\\	それは目を覆いたくなるような光景だった。	
\\	それは 目[め]を 覆[おお]いたくなるような 光景[こうけい]だった。	覆う=おおう= 
\\	(おおい隠す) 
\\	(包む) 
\\	太陽は厚い雲に覆われていた。	
\\	太陽[たいよう]は 厚[あつ]い 雲[くも]に 覆[おお]われていた。	覆う=おおう= 
\\	(おおい隠す) 
\\	(包む) 
\\	空は一面厚い雲に覆われていた。	
\\	空[そら]は 一面[いちめん] 厚[あつ]い 雲[くも]に 覆[おお]われていた。	覆う=おおう= 
\\	(おおい隠す) 
\\	(包む) 
\\	この町では夜間の外出は控えた方がいい。	
\\	この 町[まち]では 夜間[やかん]の 外出[がいしゅつ]は 控[ひか]えた 方[ほう]がいい。	
\\	この季節、日中は暖かいが夜間は冷え込む。	
\\	この 季節[きせつ]、 日中[にっちゅう]は 暖[あたた]かいが 夜間[やかん]は 冷え込[ひえこ]む。	日中=にっちゅう
\\	この難局の打開策を見いださねばならない。	
\\	この 難局[なんきょく]の 打開[だかい] 策[さく]を 見[み]いださねばならない。	打開=だかい= 
\\	所期の成果を収められなかった。	
\\	所期[しょき]の 成果[せいか]を 収[おさ]められなかった。	
\\	日本はさまざまな外国文化を同化してきた。	
\\	日本はさまざまな 外国[がいこく] 文化[ぶんか]を 同化[どうか]してきた。	同化=どうか= 
\\	(順応) 
\\	彼らは自由を熱望している。	
\\	彼[かれ]らは 自由[じゆう]を 熱望[ねつぼう]している。	熱望=ねつぼう= 
\\	日本のファンは彼の日本での公演を熱望している。	
\\	日本のファンは 彼[かれ]の 日本[にっぽん]での 公演[こうえん]を 熱望[ねつぼう]している。	熱望=ねつぼう= 
\\	炭火が消えかかっている。	
\\	炭火[すみび]が 消[き]えかかっている。	炭火=すみび= 
\\	社長が代わって経営方針が新たになった。	
\\	社長[しゃちょう]が 代[か]わって 経営[けいえい] 方針[ほうしん]が 新[あら]たになった。	
\\	年が明けて気分が新たになった。	
\\	年[とし]が 明[あ]けて 気分[きぶん]が 新[あら]たになった。	
\\	息子の写真を見て父は悲しみを新たにした。	
\\	息子[むすこ]の 写真[しゃしん]を 見[み]て 父[ちち]は 悲[かな]しみを 新[あら]たにした。	
\\	彼女の意外な一面を見て認識を新たにしたよ。	
\\	彼女[かのじょ]の 意外[いがい]な 一面[いちめん]を 見[み]て 認識[にんしき]を 新[あら]たにしたよ。	
\\	昇進を機に仕事への思いを新たにした。	
\\	昇進[しょうしん]を 機[き]に 仕事[しごと]への 思[おも]いを 新[あら]たにした。	
\\	彼の胸に新たな希望がわいた。	
\\	彼[かれ]の 胸[むね]に 新[あら]たな 希望[きぼう]がわいた。	
\\	世界記録を5秒更新した。	
\\	世界[せかい] 記録[きろく]を 5秒[ごびょう] 更新[こうしん]した。	更新=こうしん= 
\\	(改新) 
\\	それは永続の見込みがない。	
\\	それは 永続[えいぞく]の 見込[みこ]みがない。	
\\	たばこの火の不始末から火事になった。	
\\	たばこの 火[ひ]の 不[ふ] 始末[しまつ]から 火事[かじ]になった。	
\\	入社の志望動機を聞かれて「大きいからです」などと間抜けな答えをしてしまった。	
\\	入社[にゅうしゃ]の 志望[しぼう] 動機[どうき]を 聞[き]かれて
\\	大[おお]きいからです」などと 間抜[まぬ]けな 答[こた]えをしてしまった。	
\\	そんな間抜け面をして立っているな。	
\\	そんな 間抜[まぬ]け 面[づら]をして 立[た]っているな。	間抜け面=まぬけづら= 
\\	彼には少しも妥協する姿勢がない。	
\\	彼[かれ]には 少[すこ]しも 妥協[だきょう]する 姿勢[しせい]がない。	妥協=だきょう= 
\\	妥協が成立した。	
\\	妥協[だきょう]が 成立[せいりつ]した。	妥協=だきょう= 
\\	妥協の余地がない。	
\\	妥協[だきょう]の 余地[よち]がない。	妥協=だきょう= 
\\	経営側は妥協の用意があることを示唆した。	
\\	経営[けいえい] 側[がわ]は 妥協[だきょう]の 用意[ようい]があることを 示唆[しさ]した。	妥協=だきょう= 
\\	示唆=しさ= 
\\	非妥協的な姿勢を堅持している。	
\\	非[ひ] 妥協[だきょう] 的[てき]な 姿勢[しせい]を 堅持[けんじ]している。	妥協=だきょう= 
\\	堅持=けんじ= 
\\	そのなんとかいう温泉はどこにあるんですか。	
\\	そのなんとかいう 温泉[おんせん]はどこにあるんですか。	
\\	そのいまわしい事件は歴史の教科書に載ってない。	
\\	そのいまわしい 事件[じけん]は 歴史[れきし]の 教科書[きょうかしょ]に 載[の]ってない。	
\\	言うだけのことを言ったら気が清々した。	
\\	言[い]うだけのことを 言[い]ったら 気[き]が 清々[せいせい]した。	清々=せいせい= (すっきりする) 
\\	(はればれとする) 
\\	借金を払って清々した気持ちになった。	
\\	借金[しゃっきん]を 払[はら]って 清々[せいせい]した 気持[きも]ちになった。	清々=せいせい= (すっきりする) 
\\	(はればれとする) 
\\	産業廃棄物が環境破壊の元凶となった。	
\\	産業[さんぎょう] 廃棄[はいき] 物[ぶつ]が 環境[かんきょう] 破壊[はかい]の 元凶[げんきょう]となった。	元凶=げんきょう= 
\\	不況の元凶は不良債権の増大である。	
\\	不況[ふきょう]の 元凶[げんきょう]は 不良[ふりょう] 債権[さいけん]の 増大[ぞうだい]である。	元凶=げんきょう= 
\\	この賞は今年科学の分野で活躍された方に贈られるものです。	
\\	この 賞[しょう]は 今年[ことし] 科学[かがく]の 分野[ぶんや]で 活躍[かつやく]された 方[ほう]に 贈[おく]られるものです。	
\\	佐藤先生はこの分野の草分け的存在である。	
\\	佐藤[さとう] 先生[せんせい]はこの 分野[ぶんや]の 草分[くさわ]け 的[てき] 存在[そんざい]である。	草分け=くさわけ= 
\\	彼は臨床心理学の草分けだ。	
\\	彼[かれ]は 臨床[りんしょう] 心理[しんり] 学[がく]の 草分[くさわ]けだ。	草分け=くさわけ= 
\\	頭上からクモが下がってきた。	
\\	頭上[ずじょう]からクモが 下[さ]がってきた。	
\\	これは、折り畳めばソファー、広げればベッドになります。	
\\	これは、 折り畳[おりたた]めばソファー、 広[ひろ]げればベッドになります。	
\\	この日記は日々のことを思い付くままに書いたものです。	
\\	この 日記[にっき]は 日々[ひび]のことを 思い付[おもいつ]くままに 書[か]いたものです。	思い付く=おもいつく= 
\\	いいことを思いついた。	
\\	いいことを 思[おも]いついた。	思い付く=おもいつく= 
\\	私は久しぶりに旧友を訪ねようと思いついた。	
\\	私[わたし]は 久[ひさ]しぶりに 旧友[きゅうゆう]を 訪[たず]ねようと 思[おも]いついた。	思い付く=おもいつく= 
\\	何か思いついたことがあったらおっしゃってください。	
\\	何[なに]か 思[おも]いついたことがあったらおっしゃってください。	思い付く=おもいつく= 
\\	靖国神社参拝に関しては閣僚の自主的な判断に任せられている。	
\\	靖国神社[やすくにじんじゃ] 参拝[さんぱい]に 関[かん]しては 閣僚[かくりょう]の 自主[じしゅ] 的[てき]な 判断[はんだん]に 任[まか]せられている。	自主的=じしゅてき= 
\\	学園祭では学生たちが自主的に企画したものがたくさん催されます。	
\\	学園[がくえん] 祭[さい]では 学生[がくせい]たちが 自主[じしゅ] 的[てき]に 企画[きかく]したものがたくさん 催[もよお]されます。	自主的=じしゅてき= 
\\	彼女の胸中がまったくわからなかった。	
\\	彼女[かのじょ]の 胸中[きょうちゅう]がまったくわからなかった。	胸中=きょうちゅう= 
\\	彼女からきた結婚通知状を読んで私は胸中複雑なものがあった。	
\\	彼女[かのじょ]からきた 結婚[けっこん] 通知[つうち] 状[じょう]を 読[よ]んで 私[わたし]は 胸中[きょうちゅう] 複雑[ふくざつ]なものがあった。	胸中=きょうちゅう= 
\\	失せろ!	
\\	失[う]せろ!	失せる=うせる= (なくなる) 
\\	(姿を消す; 行ってしまう) 
\\	暑さですっかりやる気が失せた。	
\\	暑[あつ]さですっかりやる 気[き]が 失[う]せた。	失せる=うせる= (なくなる) 
\\	(姿を消す; 行ってしまう) 
\\	彼の顔から血の気が失せた。	
\\	彼[かれ]の 顔[かお]から 血の気[ちのけ]が 失[う]せた。	失せる=うせる= (なくなる) 
\\	(姿を消す; 行ってしまう) 
\\	「とっとと失せろ」と彼は叫んだ。	
\\	「とっとと 失[う]せろ」と 彼[かれ]は 叫[さけ]んだ。	とっとと= 
\\	失せる=うせる= (なくなる) 
\\	(姿を消す; 行ってしまう) 
\\	こんな不景気な時に会社を始めるのにはよほどの度胸がいる。	
\\	こんな 不景気[ふけいき]な 時[とき]に 会社[かいしゃ]を 始[はじ]めるのにはよほどの 度胸[どきょう]がいる。	度胸=どきょう= 
\\	(勇気) 
\\	私にはそれをやってみるだけの度胸がない。	
\\	私[わたし]にはそれをやってみるだけの 度胸[どきょう]がない。	度胸=どきょう= 
\\	(勇気) 
\\	ここが君の度胸の見せ所だ。	
\\	ここが 君[きみ]の 度胸[どきょう]の 見[み]せ 所[どころ]だ。	度胸=どきょう= 
\\	(勇気) 
\\	見せ所=みせどころ= 
\\	彼女は度胸を据えて夫に離婚話を持ち出した。	
\\	彼女[かのじょ]は 度胸[どきょう]を 据[す]えて 夫[おっと]に 離婚[りこん] 話[ばなし]を 持ち出[もちだ]した。	度胸=どきょう= 
\\	(勇気) 
\\	度胸を据える=どきょうをすえる= 
\\	お前、いい度胸しているね。	
\\	お 前[まえ]、いい 度胸[どきょう]しているね。	度胸=どきょう= 
\\	(勇気) 
\\	君は本気でそう言ったのか。	
\\	君[きみ]は 本気[ほんき]でそう 言[い]ったのか。	
\\	本気で言ってるのか。	
\\	本気[ほんき]で 言[い]ってるのか。	
\\	会社を辞めようなんて本気で考えてるの?	
\\	会社[かいしゃ]を 辞[や]めようなんて 本気[ほんき]で 考[かんが]えてるの?	
\\	遊びのつもりが本気になった。	
\\	遊[あそ]びのつもりが 本気[ほんき]になった。	
\\	私が冗談で言ったことを彼は本気にした。	
\\	私[わたし]が 冗談[じょうだん]で 言[い]ったことを 彼[かれ]は 本気[ほんき]にした。	
\\	本気になれば彼を負かすことなど何でもない。	
\\	本気[ほんき]になれば 彼[かれ]を 負[ま]かすことなど 何[なん]でもない。	
\\	本気になりさえすればできないことはない。	
\\	本気[ほんき]になりさえすればできないことはない。	
\\	彼女は胸がない。	
\\	彼女[かのじょ]は 胸[むね]がない。	
\\	胸がドキドキした。	
\\	胸[むね]がドキドキした。	
\\	胸がとどろいた。	
\\	胸[むね]がとどろいた。	轟く=とどろく= 
\\	見ていて胸が悪くなった。	
\\	見[み]ていて 胸[むね]が 悪[わる]くなった。	
\\	胸をえぐられるような思いだった。	
\\	胸[むね]をえぐられるような 思[おも]いだった。	
\\	そのことを思うと胸が痛みます。	
\\	そのことを 思[おも]うと 胸[むね]が 痛[いた]みます。	
\\	翌日から始まる念願のヨーロッパ旅行を思うと胸が騒いで眠れなかった。	
\\	翌日[よくじつ]から 始[はじ]まる 念願[ねんがん]のヨーロッパ 旅行[りょこう]を 思[おも]うと 胸[むね]が 騒[さわ]いで 眠[ねむ]れなかった。	
\\	言うだけ言ったら胸がすっとした。	
\\	言[い]うだけ 言[い]ったら 胸[むね]がすっとした。	胸がすっとする= 
\\	もう彼らに会えないと思うと胸が詰まった。	
\\	もう 彼[かれ]らに 会[あ]えないと 思[おも]うと 胸[むね]が 詰[つ]まった。	胸が詰まる・胸がふさがる= 
\\	胸が詰まって、それ以上何も言えなかった。	
\\	胸[むね]が 詰[つ]まって、それ 以上[いじょう] 何[なに]も 言[い]えなかった。	胸が詰まる・胸がふさがる= 
\\	悲しみに胸がふさがった。	
\\	悲[かな]しみに 胸[むね]がふさがった。	胸が詰まる・胸がふさがる= 
\\	明日から旅行だと思うと胸が弾む。	
\\	明日[あした]から 旅行[りょこう]だと 思[おも]うと 胸[むね]が 弾[はず]む。	
\\	悲しみに胸が張り裂けそうだった。	
\\	悲[かな]しみに 胸[むね]が 張り裂[はりさ]けそうだった。	
\\	それは実に胸に迫る光景であった。	
\\	それは 実[じつ]に 胸[むね]に 迫[せま]る 光景[こうけい]であった。	胸に迫る=むねにせまる= 
\\	そのことがずっと胸につかえていた。	
\\	そのことがずっと 胸[むね]につかえていた。	
\\	あの時の彼女の一言が今も胸にわだかまっている。	
\\	あの 時[とき]の 彼女[かのじょ]の 一言[ひとこと]が 今[いま]も 胸[むね]にわだかまっている。	
\\	自分のしたことを胸に手を当ててよく考えなさい。	
\\	自分[じぶん]のしたことを 胸[むね]に 手[て]を 当[あ]ててよく 考[かんが]えなさい。	胸に手を当てて考える= 
\\	このプロジェクトの可否は銀行の胸一つで決まる。	
\\	このプロジェクトの 可否[かひ]は 銀行[ぎんこう]の 胸[むね] 一[ひと]つで 決[き]まる。	
\\	それを見て彼女は胸を痛めた。	
\\	それを 見[み]て 彼女[かのじょ]は 胸[むね]を 痛[いた]めた。	
\\	全員無事との知らせにホット胸を撫でおろした。	
\\	全員[ぜんいん] 無事[ぶじ]との 知[し]らせにホット 胸[むね]を 撫[な]でおろした。	胸を撫で下ろす=むねをなでおろす= 
\\	社長はこれはうちだけの技術だと胸を張った。	
\\	社長[しゃちょう]はこれはうちだけの 技術[ぎじゅつ]だと 胸[むね]を 張[は]った。	胸を張る=むねをはる= 
\\	堂々と胸を張っていろ。	
\\	堂々[どうどう]と 胸[むね]を 張[は]っていろ。	胸を張る=むねをはる= 
\\	この漬物は塩が甘い。	
\\	この 漬物[つけもの]は 塩[しお]が 甘[あま]い。	甘い=あまい= 
\\	採点が甘い先生もいます。	
\\	採点[さいてん]が 甘[あま]い 先生[せんせい]もいます。	甘い=あまい= 
\\	汚職官吏の処分が甘すぎる。	
\\	汚職[おしょく] 官吏[かんり]の 処分[しょぶん]が 甘[あま]すぎる。	甘い=あまい= 
\\	私の読みが甘かったことは認める。	
\\	私[わたし]の 読[よ]みが 甘[うま]かったことは 認[みと]める。	甘い=あまい= 
\\	この包丁は切れ味が甘い。	
\\	この 包丁[ほうちょう]は 切れ味[きれあじ]が 甘[あま]い。	甘い=あまい= 
\\	ねじが甘くなっているぞ。絞め直せ。	
\\	ねじが 甘[あま]くなっているぞ。 絞[し]め 直[なお]せ。	甘い=あまい= 
\\	俺を甘く見るな。	
\\	俺[おれ]を 甘[あま]く 見[み]るな。	甘く見る= 
\\	は1993年に発足した。	
\\	は1993 年[ねん]に 発足[ほっそく]した。	発足=ほっそく= 
\\	何となく疎外されている感じがする。	
\\	何[なん]となく 疎外[そがい]されている 感[かん]じがする。	疎外=そがい= (のけものにすること) 
\\	この運転免許証は既に失効している。	
\\	この 運転[うんてん] 免許[めんきょ] 証[しょう]は 既[すで]に 失効[しっこう]している。	失効=しっこう= 
\\	彼女に冷たい眼差しで見られた。	
\\	彼女[かのじょ]に 冷[つめ]たい 眼差[まなざ]しで 見[み]られた。	
\\	この研究論文は着眼点がよい。	
\\	この 研究[けんきゅう] 論文[ろんぶん]は 着眼[ちゃくがん] 点[てん]がよい。	着眼=ちゃくがん= 
\\	彼には私の存在なんかまるで眼中になかった。	
\\	彼[かれ]には 私[わたし]の 存在[そんざい]なんかまるで 眼中[がんちゅう]になかった。	眼中にない=がんちゅうにない= 
\\	彼の眼中には政府もなく世論もない。	
\\	彼[かれ]の 眼中[がんちゅう]には 政府[せいふ]もなく 世論[せろん]もない。	眼中にない=がんちゅうにない= 
\\	この政策の主眼は経済成長を促すことにある。	
\\	この 政策[せいさく]の 主眼[しゅがん]は 経済[けいざい] 成長[せいちょう]を 促[うなが]すことにある。	
\\	この携帯は使いやすさに主眼を置いてデザインされた。	
\\	この 携帯[けいたい]は 使[つか]いやすさに 主眼[しゅがん]を 置[お]いてデザインされた。	
\\	話し合いは泥沼の様相を呈してきた。	
\\	話し合[はなしあ]いは 泥沼[どろぬま]の 様相[ようそう]を 呈[てい]してきた。	泥沼=どろぬま= 
\\	(なかなか抜けられない状態) 呈す(る)=ていす= 
\\	アフガニスタン情勢は泥沼化している。	
\\	アフガニスタン 情勢[じょうせい]は 泥沼[どろぬま] 化[か]している。	泥沼=どろぬま= 
\\	(なかなか抜けられない状態)
\\	私の順番が3番から4番に繰り下がった。	
\\	私[わたし]の 順番[じゅんばん]が3 番[ばん]から4 番[ばん]に 繰[く]り 下[さ]がった。	繰り下がる=くりさがる= 
\\	会合の日程が1日繰り下がった。	
\\	会合[かいごう]の 日程[にってい]が 1日[いちにち] 繰[く]り 下[さ]がった。	繰り下がる=くりさがる= 
\\	デモ隊が公園に繰り込んだ。	
\\	デモ 隊[たい]が 公園[こうえん]に 繰り込[くりこ]んだ。	
\\	開店と同時に客が繰り込んできた。	
\\	開店[かいてん]と 同時[どうじ]に 客[きゃく]が 繰り込[くりこ]んできた。	
\\	湯が煮立ったら野菜を入れます。	
\\	湯[ゆ]が 煮立[にた]ったら 野菜[やさい]を 入[い]れます。	煮立つ=にたつ= 
\\	スップを煮立たせてはいけない。	
\\	スップを 煮立[にた]たせてはいけない。	煮立つ=にたつ= 
\\	ニンジンは煮えたかな。	
\\	ニンジンは 煮[に]えたかな。	煮える=にえる= (食べられるようになる) 
\\	煮えたかどうか一口食べてみてごらん。	
\\	煮[に]えたかどうか 一口[ひとくち] 食[た]べてみてごらん。	一口=ひとくち=飲食物を一回口に入れること。 煮える=にえる= (食べられるようになる) 
\\	敵軍が我々の領土に襲来した。	
\\	敵[てき] 軍[ぐん]が 我々[われわれ]の 領土[りょうど]に 襲来[しゅうらい]した。	襲来=しゅうらい= 
\\	彼女は試合に負けてしょんぼりしていた。	
\\	彼女[かのじょ]は 試合[しあい]に 負[ま]けてしょんぼりしていた。	
\\	彼はしょんぼりした様子で立ち去った。	
\\	彼[かれ]はしょんぼりした 様子[ようす]で 立ち去[たちさ]った。	
\\	さしもの名医もさじを投げた。	
\\	さしもの 名医[めいい]もさじを 投[な]げた。	"さしもの= 
\\	匙=さじ= 
\\	匙を投げる=さじをなげる= 
\\	それはいかに達者な訳者でもさじを投げたくなるような難解な文章だった。	
\\	それはいかに 達者[たっしゃ]な 訳者[やくしゃ]でもさじを 投[な]げたくなるような 難解[なんかい]な 文章[ぶんしょう]だった。	達者=たっしゃ= 
\\	匙=さじ= 
\\	匙を投げる=さじをなげる= 
\\	こうして達者でいられるのも妻のおかげです。	
\\	こうして 達者[たっしゃ]でいられるのも 妻[つま]のおかげです。	達者=たっしゃ= (健康である) 
\\	(巧みである) 
\\	彼は達者なスペイン語でウエーターに話しかけた。	
\\	彼[かれ]は 達者[たっしゃ]なスペイン 語[ご]でウエーターに 話[はな]しかけた。	達者=たっしゃ= (健康である) 
\\	(巧みである) 
\\	あの人は口だけは達者だが実践が伴わない。	
\\	あの 人[ひと]は 口[くち]だけは 達者[たっしゃ]だが 実践[じっせん]が 伴[ともな]わない。	達者=たっしゃ= (健康である) 
\\	(巧みである) 
\\	どうもタクシーにぼられたらしい。	
\\	どうもタクシーにぼられたらしい。	ぼる= 
\\	彼らが感じた衝撃はとてもないものだっただろう。	
\\	彼[かれ]らが 感[かん]じた 衝撃[しょうげき]はとてもないものだっただろう。	とてつもない= 
\\	(途方もない)
\\	彼女はときどきとてつもなく変なことを言い出す。	
\\	彼女[かのじょ]はときどきとてつもなく 変[へん]なことを 言い出[いいだ]す。	とてつもない= 
\\	(途方もない)
\\	これはとてつもなく金がかかるぞ。	
\\	これはとてつもなく 金[きん]がかかるぞ。	とてつもない= 
\\	(途方もない)
\\	眠気に勝てず、私はとうとう目を閉じてしまった。	
\\	眠気[ねむけ]に 勝[か]てず、 私[わたし]はとうとう 目[め]を 閉[と]じてしまった。	
\\	この薬をのむと眠気を催す。	
\\	この 薬[くすり]をのむと 眠気[ねむけ]を 催[もよお]す。	
\\	教師の怒声が生徒たちの眠気を覚ました。	
\\	教師[きょうし]の 怒声[どせい]が 生徒[せいと]たちの 眠気[ねむけ]を 覚[さ]ました。	
\\	この機械の操作は至って簡単です。	
\\	この 機械[きかい]の 操作[そうさ]は 至[いた]って 簡単[かんたん]です。	至って=いたって= (極めて) 
\\	祖父は今80歳ですが至って元気です。	
\\	祖父[そふ]は 今[いま]80 歳[さい]ですが 至[いた]って 元気[げんき]です。	至って=いたって= (極めて) 
\\	彼らはいずれも愛国の至情に燃えていた。	
\\	彼[かれ]らはいずれも 愛国[あいこく]の 至情[しじょう]に 燃[も]えていた。	至情=しじょう= 
\\	至急ファクスしてください。	
\\	至急[しきゅう]ファクスしてください。	至急=しきゅう= 
\\	至急回答お願いします。	
\\	至急[しきゅう] 回答[かいとう]お 願[ねが]いします。	至急=しきゅう= 
\\	回答=かいとう= (問い合わせに対する) 
\\	(要求に対する) 
\\	(解答)
\\	この傷は至急手当をしなくてはならない。	
\\	この 傷[きず]は 至急[しきゅう] 手当[てあて]をしなくてはならない。	至急=しきゅう= 
\\	彼女は同僚から執拗ないやがらせを受けている。	
\\	彼女[かのじょ]は 同僚[どうりょう]から 執拗[しつよう]ないやがらせを 受[う]けている。	
\\	母と妻との間に確執が絶えず、気の休まる時がない。	
\\	母[はは]と 妻[つま]との 間[ま]に 確執[かくしつ]が 絶[た]えず、 気[き]の 休[やす]まる 時[とき]がない。	確執=かくしつ= 
\\	遺産をめぐって兄弟の間に確執が生じた。	
\\	遺産[いさん]をめぐって 兄弟[きょうだい]の 間[ま]に 確執[かくしつ]が 生[しょう]じた。	確執=かくしつ= 
\\	彼は金銭に執着している。	
\\	彼[かれ]は 金銭[きんせん]に 執着[しゅうちゃく]している。	執着=しゅうちゃく= (愛着) 
\\	彼女は結婚に執着している。	
\\	彼女[かのじょ]は 結婚[けっこん]に 執着[しゅうちゃく]している。	執着=しゅうちゃく= (愛着) 
\\	彼の執った態度は立派だった。	
\\	彼[かれ]の 執[と]った 態度[たいど]は 立派[りっぱ]だった。	
\\	いい景色!ここまで登ってきた甲斐があったわね。	
\\	いい 景色[けしき]!ここまで 登[のぼ]ってきた 甲斐[かい]があったわね。	甲斐=かい= (効果) 
\\	(価値) 
\\	(益) 
\\	努力の甲斐がなかった。	
\\	努力[どりょく]の 甲斐[かい]がなかった。	甲斐=かい= (効果) 
\\	(価値) 
\\	(益) 
\\	彼女がいなくては生きる甲斐がない。	
\\	彼女[かのじょ]がいなくては 生[い]きる 甲斐[かい]がない。	甲斐=かい= (効果) 
\\	(価値) 
\\	(益) 
\\	彼の車がタクシーに接触した。	
\\	彼[かれ]の 車[くるま]がタクシーに 接触[せっしょく]した。	接触=せっしょく= (触れること) 
\\	(関係・交渉を持つこと)
\\	金属ネックレスが接触した部分の肌が赤くなった。	
\\	金属[きんぞく]ネックレスが 接触[せっしょく]した 部分[ぶぶん]の 肌[はだ]が 赤[あか]くなった。	接触=せっしょく= (触れること) 
\\	(関係・交渉を持つこと)
\\	この病気は患者との接触によって感染する。	
\\	この 病気[びょうき]は 患者[かんじゃ]との 接触[せっしょく]によって 感染[かんせん]する。	接触=せっしょく= (触れること) 
\\	(関係・交渉を持つこと)
\\	彼はその晩安眠できなかった。	
\\	彼[かれ]はその 晩[ばん] 安眠[あんみん]できなかった。	安眠=あんみん= 
\\	その薬を飲んだら安眠できた。	
\\	その 薬[くすり]を 飲[の]んだら 安眠[あんみん]できた。	安眠=あんみん= 
\\	住民のほとんどがダム工事に従事している。	
\\	住民[じゅうみん]のほとんどがダム 工事[こうじ]に 従事[じゅうじ]している。	従事=じゅうじ= 
\\	祖父は70歳になるまで農業に従事していた。	
\\	祖父[そふ]は70 歳[さい]になるまで 農業[のうぎょう]に 従事[じゅうじ]していた。	従事=じゅうじ= 
\\	彼はひざまずいて十字を切った。	
\\	彼[かれ]はひざまずいて 十字[じゅうじ]を 切[き]った。	十字=じゅうじ= 
\\	彼女はがんの研究に従事している。	
\\	彼女[かのじょ]はがんの 研究[けんきゅう]に 従事[じゅうじ]している。	従事=じゅうじ= 
\\	問題が続々起こってきた。	
\\	問題[もんだい]が 続々[ぞくぞく] 起[お]こってきた。	続々=ぞくぞく= 
\\	注文が続々と舞い込んでいる。	
\\	注文[ちゅうもん]が 続々[ぞくぞく]と 舞い込[まいこ]んでいる。	続々=ぞくぞく= 
\\	彼に会えると思うとうれしさにぞくぞくする。	
\\	彼[かれ]に 会[あ]えると 思[おも]うとうれしさにぞくぞくする。	ぞくぞく= 
\\	(心踊るさま)
\\	風邪をひいたのか、体がぞくぞくする。	
\\	風邪[かぜ]をひいたのか、 体[からだ]がぞくぞくする。	ぞくぞく= 
\\	(心踊るさま)
\\	寒気で体がぞくぞくした。	
\\	寒気[さむけ]で 体[からだ]がぞくぞくした。	ぞくぞく= 
\\	(心踊るさま)
\\	退職を契機に俳句を始めた。	
\\	退職[たいしょく]を 契機[けいき]に 俳句[はいく]を 始[はじ]めた。	契機=けいき= 
\\	君の犯した罪はいくら償っても償いきれるものではない。	
\\	君[きみ]の 犯[おか]した 罪[つみ]はいくら 償[つぐな]っても 償[つぐな]いきれるものではない。	償う=つぐなう= 
\\	(弁償する) 
\\	損害は君に償ってもらわなければならない。	
\\	損害[そんがい]は 君[きみ]に 償[つぐな]ってもらわなければならない。	償う=つぐなう= 
\\	(弁償する) 
\\	過去の過ちを償わなければならない。	
\\	過去[かこ]の 過[あやま]ちを 償[つぐな]わなければならない。	償う=つぐなう= 
\\	(弁償する) 
\\	その酔漢は上体をふらふら泳がせながら歩いていった。	
\\	その 酔漢[すいかん]は 上体[じょうたい]をふらふら 泳[およ]がせながら 歩[ある]いていった。	泳がせる=およがせる= 
\\	(体を前にのめらせる); (被疑者を自由に行動させる) 
\\	酔漢=すいかん= ひどく酒に酔った男。
\\	彼は面会謝絶だ。	
\\	彼[かれ]は 面会[めんかい] 謝絶[しゃぜつ]だ。	面会=めんかい= 
\\	謝絶=しゃぜつ= 
\\	大臣は公に失言を陳謝した。	
\\	大臣[だいじん]は 公[おおやけ]に 失言[しつげん]を 陳謝[ちんしゃ]した。	陳謝=ちんしゃ= 
\\	会社は不祥事を陳謝した。	
\\	会社[かいしゃ]は 不祥事[ふしょうじ]を 陳謝[ちんしゃ]した。	不祥事=ふしょうじ= 
\\	陳謝=ちんしゃ= 
\\	患者はその病院に対して総額2000万円の慰謝料を求める訴訟を起こした。	
\\	患者[かんじゃ]はその 病院[びょういん]に 対[たい]して 総額[そうがく]2000 万[まん] 円[えん]の 慰謝[いしゃ] 料[りょう]を 求[もと]める 訴訟[そしょう]を 起[お]こした。	慰謝料=いしゃりょう= 
\\	彼は妻との離婚の際、莫大な慰謝料を払わされた。	
\\	彼[かれ]は 妻[つま]との 離婚[りこん]の 際[さい]、 莫大[ばくだい]な 慰謝[いしゃ] 料[りょう]を 払[はら]わされた。	慰謝料=いしゃりょう= 
\\	彼女は慰謝料として1000万円受け取った。	
\\	彼女[かのじょ]は 慰謝[いしゃ] 料[りょう]として1000 万[まん] 円[えん] 受け取[うけと]った。	慰謝料=いしゃりょう= 
\\	私は彼女の言葉をつゆ疑わなかった。	
\\	私[わたし]は 彼女[かのじょ]の 言葉[ことば]をつゆ 疑[うたが]わなかった。	つゆ疑わずに= 
\\	当地では入梅は6月です。	
\\	当地[とうち]では 入梅[にゅうばい]は 6月[ろくがつ]です。	入梅=にゅうばい= 
\\	梅雨入り宣言が出された。	
\\	梅雨入[つゆい]り 宣言[せんげん]が 出[だ]された。	梅雨入り=つゆいり= 
\\	私はそのひどい侮辱に耐えられず、会社を辞めた。	
\\	私[わたし]はそのひどい 侮辱[ぶじょく]に 耐[た]えられず、 会社[かいしゃ]を 辞[や]めた。	侮辱=ぶじょく= 
\\	僕の宿題はお父さんにやってもらっただなんて、すごい侮辱だ。	
\\	僕[ぼく]の 宿題[しゅくだい]はお 父[とう]さんにやってもらっただなんて、すごい 侮辱[ぶじょく]だ。	侮辱=ぶじょく= 
\\	彼は自分でやると言ったのに実行していない。	
\\	彼[かれ]は 自分[じぶん]でやると 言[い]ったのに 実行[じっこう]していない。	
\\	それは計画もよく実行のしかたも良かった。	
\\	それは 計画[けいかく]もよく 実行[じっこう]のしかたも 良[よ]かった。	
\\	確かにいい考えだが実行は難しい。	
\\	確[たし]かにいい 考[かんが]えだが 実行[じっこう]は 難[むずか]しい。	
\\	机上のプランとしては良さそうだが、実行上問題がある。	
\\	机上[きじょう]のプランとしては 良[よ]さそうだが、 実行[じっこう] 上[じょう] 問題[もんだい]がある。	
\\	政府はその事実を歴史から抹殺しようとした。	
\\	政府[せいふ]はその 事実[じじつ]を 歴史[れきし]から 抹殺[まっさつ]しようとした。	抹殺=まっさつ= (消し去ること) 
\\	((史実などの) 否定) 
\\	遺体は腐敗が激しく、性別もわからない。	
\\	遺体[いたい]は 腐敗[ふはい]が 激[はげ]しく、 性別[せいべつ]もわからない。	腐敗=ふはい= (腐ること) 
\\	卵は腐敗しやすい。	
\\	卵[たまご]は 腐敗[ふはい]しやすい。	腐敗=ふはい= (腐ること) 
\\	権力は必ず腐敗する。	
\\	権力[けんりょく]は 必[かなら]ず 腐敗[ふはい]する。	腐敗=ふはい= (腐ること) 
\\	近ごろこの近辺ではひったくりが多発している。	
\\	近[ちか]ごろこの 近辺[きんぺん]ではひったくりが 多発[たはつ]している。	ひったくり= 
\\	多発=たはつ= 
\\	脈拍が乱調になってきた。	
\\	脈拍[みゃくはく]が 乱調[らんちょう]になってきた。	乱調=らんちょう= (混乱) 
\\	(不整) 
\\	今年は我が社にとって躍進の年であった。	
\\	今年[ことし]は 我[わ]が 社[しゃ]にとって 躍進[やくしん]の 年[とし]であった。	躍進=やくしん= (進出) 
\\	(発展) 
\\	ここ数年インターネット産業が大きな躍進を遂げた。	
\\	ここ 数[すう] 年[ねん]インターネット 産業[さんぎょう]が 大[おお]きな 躍進[やくしん]を 遂[と]げた。	躍進=やくしん= (進出) 
\\	(発展) 
\\	この地域は近年目覚ましい経済的躍進を遂げている。	
\\	この 地域[ちいき]は 近年[きんねん] 目覚[めざ]ましい 経済[けいざい] 的[てき] 躍進[やくしん]を 遂[と]げている。	躍進=やくしん= (進出) 
\\	(発展) 
\\	両者の争いはまだ決着がつかない。	
\\	両者[りょうしゃ]の 争[あらそ]いはまだ 決着[けっちゃく]がつかない。	決着=けっちゃく= 
\\	(解決) 
\\	双方の利益の対立で未だに交渉の決着がつかない。	
\\	双方[そうほう]の 利益[りえき]の 対立[たいりつ]で 未[いま]だに 交渉[こうしょう]の 決着[けっちゃく]がつかない。	決着=けっちゃく= 
\\	(解決) 
\\	予算の削減は小幅にとどまった。	
\\	予算[よさん]の 削減[さくげん]は 小幅[こはば]にとどまった。	
\\	その会合は国歌斉唱で始まった。	
\\	その 会合[かいごう]は 国歌[こっか] 斉唱[せいしょう]で 始[はじ]まった。	斉唱=せいしょう= 
\\	みんなそろって校歌を合唱した。	
\\	みんなそろって 校歌[こうか]を 合唱[がっしょう]した。	合唱=がっしょう= (声を合わせて歌ったり唱えたりすること) 
\\	万歳の合唱が群衆の間から起こった。	
\\	万歳[ばんざい]の 合唱[がっしょう]が 群衆[ぐんしゅう]の 間[あいだ]から 起[お]こった。	合唱=がっしょう= (声を合わせて歌ったり唱えたりすること) 
\\	魔女たちは彼に向かって魔法の呪文を唱えた。	
\\	魔女[まじょ]たちは 彼[かれ]に 向[む]かって 魔法[まほう]の 呪文[じゅもん]を 唱[とな]えた。	魔女=まじょ= 
\\	魔法=まほう= 
\\	呪文=じゅもん= 
\\	唱える=となえる= 
\\	(高く叫ぶ) 
\\	(唱道する) 
\\	彼女は国防費を削減するように唱えた。	
\\	彼女[かのじょ]は 国防[こくぼう] 費[ひ]を 削減[さくげん]するように 唱[とな]えた。	唱える=となえる= 
\\	(高く叫ぶ) 
\\	(唱道する) 
\\	沖縄料理の筆頭格はヤギ料理だ。	
\\	沖縄[おきなわ] 料理[りょうり]の 筆頭[ひっとう] 格[かく]はヤギ 料理[りょうり]だ。	筆頭=ひっとう= (筆の先) 
\\	(首席) 
\\	筆頭格=ひっとうかく= 
\\	彼女はそのレースの優勝候補の筆頭だ。	
\\	彼女[かのじょ]はそのレースの 優勝[ゆうしょう] 候補[こうほ]の 筆頭[ひっとう]だ。	筆頭=ひっとう= (筆の先) 
\\	(首席) 
\\	この日は世界史上特筆すべき日である。	
\\	この 日[ひ]は 世界[せかい] 史上[しじょう] 特筆[とくひつ]すべき 日[ひ]である。	特筆=とくひつ= 
\\	どんな代償を払ってもこの計画を成功させたい。	
\\	どんな 代償[だいしょう]を 払[はら]ってもこの 計画[けいかく]を 成功[せいこう]させたい。	代償=だいしょう= (補償) 
\\	(代価) 
\\	今怠けていると後でその代償を払うことになる。	
\\	今[いま] 怠[なま]けていると 後[あと]でその 代償[だいしょう]を 払[はら]うことになる。	代償=だいしょう= (補償) 
\\	(代価) 
\\	小中学生には教科書が無償で与えられる。	
\\	小中学生[しょうちゅうがくせい]には 教科書[きょうかしょ]が 無償[むしょう]で 与[あた]えられる。	
\\	これをなくしたら弁償だぞ。	
\\	これをなくしたら 弁償[べんしょう]だぞ。	
\\	割った窓ガラスは弁償します。	
\\	割[わ]った 窓[まど]ガラスは 弁償[べんしょう]します。	
\\	尊い人命という代価を払ってそのトンネルは完成した。	
\\	尊[とうと]い 人命[じんめい]という 代価[だいか]を 払[はら]ってそのトンネルは 完成[かんせい]した。	
\\	二人の主張には大きな開きがある。	
\\	二 人[にん]の 主張[しゅちょう]には 大[おお]きな 開[ひら]きがある。	開き=ひらき= 
\\	(差) 
\\	家族の間では何でも開けっ広げに話し合える雰囲気が大事だと思う。	
\\	家族[かぞく]の 間[あいだ]では 何[なに]でも 開[あ]けっ 広[ひろ]げに 話し合[はなしあ]える 雰囲気[ふんいき]が 大事[だいじ]だと 思[おも]う。	開けっ広げ=あけっぴろげ= 
\\	女の子が股を開けっ広げにして座るもんじゃないよ。	
\\	女の子[おんなのこ]が 股[また]を 開[あ]けっ 広[ひろ]げにして 座[すわ]るもんじゃないよ。	股=もも= 
\\	開けっ広げ=あけっぴろげ= 
\\	ちょっとのあいだ手が空けられますか。	
\\	ちょっとのあいだ 手[て]が 空[あ]けられますか。	開ける・明ける・空ける=あける= 
\\	(時間を) 
\\	一晩でウイスキーを一瓶皆で空けてしまった。	
\\	一晩[ひとばん]でウイスキーを 一瓶[ひとびん] 皆[みな]で 空[あ]けてしまった。	一瓶=ひとびん= 
\\	開ける・明ける・空ける=あける= 
\\	(時間を) 
\\	私はわくわくしながら包みを開けた。	
\\	私[わたし]はわくわくしながら 包[つつ]みを 開[あ]けた。	包み=つつみ= 
\\	開ける・明ける・空ける=あける= 
\\	(時間を) 
\\	弁当を開けたら大好きな卵焼きが入っていた。	
\\	弁当[べんとう]を 開[あ]けたら 大好[だいす]きな 卵焼[たまごや]きが 入[はい]っていた。	開ける・明ける・空ける=あける= 
\\	(時間を) 
\\	3ページを開けなさい。	
\\	3ページを 開[あ]けなさい。	開ける・明ける・空ける=あける= 
\\	(時間を) 
\\	窓を少し開けてくれませんか。	
\\	窓[まど]を 少[すこ]し 開[あ]けてくれませんか。	開ける・明ける・空ける=あける= 
\\	(時間を) 
\\	ドアを開けると廊下だ。	
\\	ドアを 開[あ]けると 廊下[ろうか]だ。	開ける・明ける・空ける=あける= 
\\	(時間を) 
\\	祖母は非常に開けた人だった。	
\\	祖母[そぼ]は 非常[ひじょう]に 開[ひら]けた 人[ひと]だった。	開ける=ひらける= (見渡せる) 
\\	(好転する) 
\\	(発展する) 
\\	(理解がある)
\\	父にはもう少し開けてほしいと思う。	
\\	父[ちち]にはもう 少[すこ]し 開[ひら]けてほしいと 思[おも]う。	開ける=ひらける= (見渡せる) 
\\	(好転する) 
\\	(発展する) 
\\	(理解がある)
\\	彼は中々開けている。	
\\	彼[かれ]は 中[なか]々 開[ひら]けている。	開ける=ひらける= (見渡せる) 
\\	(好転する) 
\\	(発展する) 
\\	(理解がある)
\\	鉄道の駅ができて町が開けた。	
\\	鉄道[てつどう]の 駅[えき]ができて 町[まち]が 開[ひら]けた。	開ける=ひらける= (見渡せる) 
\\	(好転する) 
\\	(発展する) 
\\	(理解がある)
\\	霧が晴れて視界が開けた。	
\\	霧[きり]が 晴[は]れて 視界[しかい]が 開[ひら]けた。	開ける=ひらける= (見渡せる) 
\\	(好転する) 
\\	(発展する) 
\\	(理解がある)
\\	ドアを開け閉てする音がうるさい。	
\\	ドアを 開け閉[あけた]てする 音[おと]がうるさい。	開け閉て=あけたて= 
\\	病人が寝ていますからドアの開け閉めは静かに願います。	
\\	病人[びょうにん]が 寝[ね]ていますからドアの 開[あ]け 閉[し]めは 静[しず]かに 願[ねが]います。	開け閉め=あけしめ= 
\\	水族館は10時に開園する。	
\\	水族館[すいぞくかん]は10 時[じ]に 開園[かいえん]する。	
\\	気象庁が今年の桜の開花予想を発表した。	
\\	気象庁[きしょうちょう]が 今年[ことし]の 桜[さくら]の 開花[かいか] 予想[よそう]を 発表[はっぴょう]した。	開花=かいか= (植物の花が咲くこと) 
\\	(能力や努力が成果となって現れること)
\\	東京辺りの桜の開花期は4月上旬です。	
\\	東京[とうきょう] 辺[あた]りの 桜[さくら]の 開花[かいか] 期[き]は 4月[しがつ] 上旬[じょうじゅん]です。	開花=かいか= (植物の花が咲くこと) 
\\	(能力や努力が成果となって現れること)
\\	還暦を過ぎてからその才能が一気に開花した。	
\\	還暦[かんれき]を 過[す]ぎてからその 才能[さいのう]が 一気[いっき]に 開花[かいか]した。	還暦=かんれき= 
\\	開花=かいか= (植物の花が咲くこと) 
\\	(能力や努力が成果となって現れること)
\\	歯科医はビルの3階で開業した。	
\\	歯科[しか] 医[い]はビルの3 階[かい]で 開業[かいぎょう]した。	歯科医=しかい
\\	私が到着すると彼は開口一番「また遅刻だな」と言った。	
\\	私[わたし]が 到着[とうちゃく]すると 彼[かれ]は 開口一番[かいこういちばん]「また 遅刻[ちこく]だな」と 言[い]った。	
\\	開口一番まず政府の無能を攻撃した。	
\\	開口一番[かいこういちばん]まず 政府[せいふ]の 無能[むのう]を 攻撃[こうげき]した。	
\\	彼女のことが信じ切れなかった。	
\\	彼女[かのじょ]のことが 信[しん]じ 切[き]れなかった。	
\\	これを一週間で覚え切れるだろうか。	
\\	これを 一週間[いっしゅうかん]で 覚[おぼ]え 切[き]れるだろうか。	
\\	その借金は彼に払い切れる額ではない。	
\\	その 借金[しゃっきん]は 彼[かれ]に 払[はら]い 切[き]れる 額[がく]ではない。	
\\	この批評の内容はあら捜しばかりじゃないか。	
\\	この 批評[ひひょう]の 内容[ないよう]はあら 捜[さが]しばかりじゃないか。	あら捜し=あらさがし= 
\\	彼は私のあら探しばかりしている。	
\\	彼[かれ]は 私[わたし]のあら 探[さが]しばかりしている。	あら捜し・あら探し=あらさがし= 
\\	子供を養子に出したい。	
\\	子供[こども]を 養子[ようし]に 出[だ]したい。	
\\	この機械は地震の時は自動的に停止する。	
\\	この 機械[きかい]は 地震[じしん]の 時[とき]は 自動的[じどうてき]に 停止[ていし]する。	停止=ていし= (車両など、止まること) 
\\	(機能・活動など、止まること) 
\\	(禁止) 
\\	車はスピードを落として停止した。	
\\	車[くるま]はスピードを 落[お]として 停止[ていし]した。	停止=ていし= (車両など、止まること) 
\\	(機能・活動など、止まること) 
\\	(禁止) 
\\	列車はトンネル内で停止した。	
\\	列車[れっしゃ]はトンネル 内[ない]で 停止[ていし]した。	停止=ていし= (車両など、止まること) 
\\	(機能・活動など、止まること) 
\\	(禁止) 
\\	彼女の心臓の鼓動が停止した。	
\\	彼女[かのじょ]の 心臓[しんぞう]の 鼓動[こどう]が 停止[ていし]した。	停止=ていし= (車両など、止まること) 
\\	(機能・活動など、止まること) 
\\	(禁止) 
\\	社会人になったんだからもう家族に甘えるわけにはいかない。	
\\	社会[しゃかい] 人[じん]になったんだからもう 家族[かぞく]に 甘[あま]えるわけにはいかない。	甘える=あまえる= (甘ったれる) 
\\	その甘えた姿勢を改めろ。	
\\	その 甘[あま]えた 姿勢[しせい]を 改[あらた]めろ。	甘える=あまえる= (甘ったれる) 
\\	彼は第4位に甘んじなければならなかった。	
\\	彼[かれ]は 第[だい]4 位[い]に 甘[あま]んじなければならなかった。	甘んじる=あまんじる= (満足する) 
\\	(我慢する); 
\\	(あきらめる)
\\	こうした差別に甘んじることができない。	
\\	こうした 差別[さべつ]に 甘[あま]んじることができない。	甘んじる=あまんじる= (満足する) 
\\	(我慢する); 
\\	(あきらめる)
\\	そんな甘口には乗らないよ。	
\\	そんな 甘口[あまくち]には 乗[の]らないよ。	
\\	僕は甘党だ。	
\\	僕[ぼく]は 甘党[あまとう]だ。	甘党=あまとう= 甘いものが好きな人
\\	我が社は不景気で金が回らずあっぷあっぷしている。	
\\	我[わ]が 社[しゃ]は 不景気[ふけいき]で 金[きん]が 回[まわ]らずあっぷあっぷしている。	あっぷあっぷする= 
\\	魚が水面であっぷあっぷしている。	
\\	魚[さかな]が 水面[すいめん]であっぷあっぷしている。	あっぷあっぷする= 
\\	戦闘に巻き込まれて多数の民間が死んだ。	
\\	戦闘[せんとう]に 巻き込[まきこ]まれて 多数[たすう]の 民間[みんかん]が 死[し]んだ。	戦闘=せんとう= 
\\	慣れない新型機械を使いこなそうと連日悪戦苦闘している。	
\\	慣[な]れない 新型[しんがた] 機械[きかい]を 使[つか]いこなそうと 連日[れんじつ] 悪戦苦闘[あくせんくとう]している。	悪戦苦闘=あくせんくとう= 
\\	子育てに奮闘中と彼女からメールが届きました。	
\\	子育[こそだ]てに 奮闘[ふんとう] 中[ちゅう]と 彼女[かのじょ]からメールが 届[とど]きました。	奮闘=ふんとう= 
\\	彼の今日の成功は奮闘努力の賜物である。	
\\	彼[かれ]の 今日[きょう]の 成功[せいこう]は 奮闘[ふんとう] 努力[どりょく]の 賜物[たまもの]である。	奮闘=ふんとう= 
\\	賜物=たまもの= 
\\	祭りの翌日には市内は平静に戻っていた。	
\\	祭[まつ]りの 翌日[よくじつ]には 市内[しない]は 平静[へいせい]に 戻[もど]っていた。	平静=へいせい= (穏やかさ) 
\\	(落ち着き) 
\\	彼女はいっとき激しく泣いたが、じきに平静を取り戻した。	
\\	彼女[かのじょ]はいっとき 激[はげ]しく 泣[な]いたが、じきに 平静[へいせい]を 取り戻[とりもど]した。	いっとき= 
\\	平静=へいせい= (穏やかさ) 
\\	(落ち着き) 
\\	舟は横浜沖に停泊中である。	
\\	舟[ふね]は 横浜[よこはま] 沖[おき]に 停泊[ていはく] 中[ちゅう]である。	停泊=ていはく= 
\\	駅まで友人の車に便乗させてもらった。	
\\	駅[えき]まで 友人[ゆうじん]の 車[くるま]に 便乗[びんじょう]させてもらった。	
\\	彼は180センチの身長がある。	
\\	彼[かれ]は180センチの 身長[しんちょう]がある。	
\\	身長はいくらありますか。	
\\	身長[しんちょう]はいくらありますか。	
\\	食品は表示をよく見てから買います。	
\\	食品[しょくひん]は 表示[ひょうじ]をよく 見[み]てから 買[か]います。	表示=ひょうじ= (知らせるための) 
\\	(画面上の) 
\\	この靴のサイズはヨーロッパ式に表示されている。	
\\	この 靴[くつ]のサイズはヨーロッパ 式[しき]に 表示[ひょうじ]されている。	表示=ひょうじ= (知らせるための) 
\\	(画面上の) 
\\	テレビの画面サイズはインチで表示されている。	
\\	テレビの 画面[がめん]サイズはインチで 表示[ひょうじ]されている。	表示=ひょうじ= (知らせるための) 
\\	(画面上の) 
\\	メニューに料理ごとのカロリーが表示されている。	
\\	メニューに 料理[りょうり]ごとのカロリーが 表示[ひょうじ]されている。	表示=ひょうじ= (知らせるための) 
\\	(画面上の) 
\\	パソコンの画面上にエラーメッセージが表示された。	
\\	パソコンの 画面[がめん] 上[じょう]にエラーメッセージが 表示[ひょうじ]された。	表示=ひょうじ= (知らせるための) 
\\	(画面上の) 
\\	そのような大規模の噴火は想定外だった。	
\\	そのような 大[だい] 規模[きぼ]の 噴火[ふんか]は 想定[そうてい] 外[がい]だった。	噴火=ふんか= 
\\	想定=そうてい= (仮定) 
\\	(予想) 
\\	(見積もり) 
\\	そういう質問は想定内だった。	
\\	そういう 質問[しつもん]は 想定[そうてい] 内[ない]だった。	想定=そうてい= (仮定) 
\\	(予想) 
\\	(見積もり) 
\\	彼女が来ることはまったく想定していなかった。	
\\	彼女[かのじょ]が 来[く]ることはまったく 想定[そうてい]していなかった。	想定=そうてい= (仮定) 
\\	(予想) 
\\	(見積もり) 
\\	栃木は、茨城、群馬、埼玉、福島の4県と隣接している。	
\\	栃木[とちぎ]は、 茨城[いばらき]、 群馬[ぐんま]、 埼玉[さいたま]、 福島[ふくしま]の4 県[けん]と 隣接[りんせつ]している。	隣接=りんせつ= 
\\	大会当日は風邪を引いて実力を十分に発揮できなかった。	
\\	大会[たいかい] 当日[とうじつ]は 風邪[かぜ]を 引[ひ]いて 実力[じつりょく]を 十分[じゅうぶん]に 発揮[はっき]できなかった。	
\\	そのホテルは港に面している。	
\\	そのホテルは 港[みなと]に 面[めん]している。	面する=めんする= 
\\	私の部屋は庭に面しています。	
\\	私[わたし]の 部屋[へや]は 庭[にわ]に 面[めん]しています。	面する=めんする= 
\\	大型店が古くからの小売店を締め出してしまった。	
\\	大型[おおがた] 店[てん]が 古[ふる]くからの 小売[こうり] 店[てん]を 締め出[しめだ]してしまった。	閉め出す・締め出す=しめだす= (閉じて入らせない) 
\\	(追い出す) 
\\	小売店=こうりてん= 小売りをする店。
\\	私は昨夜門限に遅れて寮から締め出されてしまった。	
\\	私[わたし]は 昨夜[さくや] 門限[もんげん]に 遅[おく]れて 寮[りょう]から 締め出[しめだ]されてしまった。	閉め出す・締め出す=しめだす= (閉じて入らせない) 
\\	(追い出す) 
\\	彼女の話を聞いて目頭が熱くなった。	
\\	彼女[かのじょ]の 話[はなし]を 聞[き]いて 目頭[めがしら]が 熱[あつ]くなった。	目頭=めがしら= 
\\	彼女はハンカチで目頭をぬぐった。	
\\	彼女[かのじょ]はハンカチで 目頭[めがしら]をぬぐった。	目頭=めがしら= 
\\	新方式の問題点が徐々に顕在化し始めている。	
\\	新[しん] 方式[ほうしき]の 問題[もんだい] 点[てん]が 徐々[じょじょ]に 顕在[けんざい] 化[か]し 始[はじ]めている。	顕在化する=けんざいかする= 
\\	4月から給料がアップした。	
\\	4月[しがつ]から 給料[きゅうりょう]がアップした。	アップ= (上昇) 
\\	{映} (クローズアップ); {電算} (アップロード)
\\	その女優の出演で視聴率がアップした。	
\\	その 女優[じょゆう]の 出演[しゅつえん]で 視聴[しちょう] 率[りつ]がアップした。	アップ= (上昇) 
\\	{映} (クローズアップ); {電算} (アップロード)
\\	彼は世間から隔離されて育った。	
\\	彼[かれ]は 世間[せけん]から 隔離[かくり]されて 育[そだ]った。	隔離=かくり= 
\\	そこは山奥の一軒家で外界から完全に隔離されている。	
\\	そこは 山奥[やまおく]の 一軒家[いっけんや]で 外界[がいかい]から 完全[かんぜん]に 隔離[かくり]されている。	一軒家=いっけんや= 
\\	隔離=かくり= 
\\	彼女は今、人生のテーマを暗中模索している。	
\\	彼女[かのじょ]は 今[いま]、 人生[じんせい]のテーマを 暗中模索[あんちゅうもさく]している。	暗中模索=あんちゅうもさく= 
\\	彼女は平泳ぎの世界記録を保持している。	
\\	彼女[かのじょ]は 平泳[ひらおよ]ぎの 世界[せかい] 記録[きろく]を 保持[ほじ]している。	
\\	畏敬すべき神をもたない人々は不幸せである。	
\\	畏敬[いけい]すべき 神[かみ]をもたない 人々[ひとびと]は 不幸[ふしあわ]せである。	畏敬=いけい= 
\\	不幸せ=ふしあわせ= 
\\	彼の作品には自然への愛と畏敬の念があふれている。	
\\	彼[かれ]の 作品[さくひん]には 自然[しぜん]への 愛[あい]と 畏敬[いけい]の 念[ねん]があふれている。	畏敬=いけい= 
\\	富士山は人々に畏敬の念を起こさせる姿の美しい山だ。	
\\	富士山[ふじさん]は 人々[ひとびと]に 畏敬[いけい]の 念[ねん]を 起[お]こさせる 姿[すがた]の 美[うつく]しい 山[やま]だ。	畏敬=いけい= 
\\	そのような隕石が地球に落ちれば一瞬にして文明は壊滅するであろう。	
\\	そのような隕石が 地球[ちきゅう]に 落[お]ちれば 一瞬[いっしゅん]にして 文明[ぶんめい]は 壊滅[かいめつ]するであろう。	壊滅=かいめつ= 
\\	隕石=いんせき= 
\\	大規模な摘発によって密輸組織は壊滅した。	
\\	大[だい] 規模[きぼ]な 摘発[てきはつ]によって 密輸[みつゆ] 組織[そしき]は 壊滅[かいめつ]した。	壊滅=かいめつ= 
\\	一発の原子爆弾によって広島は壊滅した。	
\\	一発[いっぱつ]の 原子[げんし] 爆弾[ばくだん]によって 広島[ひろしま]は 壊滅[かいめつ]した。	壊滅=かいめつ= 
\\	警官は泥棒と格闘した末に逮捕した。	
\\	警官[けいかん]は 泥棒[どろぼう]と 格闘[かくとう]した 末[すえ]に 逮捕[たいほ]した。	格闘=かくとう= (取っ組み合い) 
\\	(乱闘) 
\\	(難問への) 
\\	その建物は全壊した。	
\\	その 建物[たてもの]は 全壊[ぜんかい]した。	
\\	その台風で多数の家屋が損壊した。	
\\	その 台風[たいふう]で 多数[たすう]の 家屋[かおく]が 損壊[そんかい]した。	
\\	経営陣は退陣が近い。	
\\	経営[けいえい] 陣[じん]は 退陣[たいじん]が 近[ちか]い。	経営陣=けいえいじん= 
\\	退陣=たいじん= (軍の) 
\\	(比喩的に) 
\\	首相は就任わずか8ヶ月で退陣を表明した。	
\\	首相[しゅしょう]は 就任[しゅうにん]わずか 8ヶ月[はちかげつ]で 退陣[たいじん]を 表明[ひょうめい]した。	退陣=たいじん= (軍の) 
\\	(比喩的に) 
\\	新監督は初陣を勝利で飾った。	
\\	新[しん] 監督[かんとく]は 初陣[ういじん]を 勝利[しょうり]で 飾[かざ]った。	初陣=ういじん= (初めて戦に出ること) 
\\	この贈り物は誰にも喜ばれるでしょう。	
\\	この 贈り物[おくりもの]は 誰[だれ]にも 喜[よろこ]ばれるでしょう。	
\\	彼らは信仰のためなら喜んで命を捨てる。	
\\	彼[かれ]らは 信仰[しんこう]のためなら 喜[よろこ]んで 命[いのち]を 捨[す]てる。	
\\	そんなうわさが流れていることは薄々感づいていた。	
\\	そんなうわさが 流[なが]れていることは 薄々[うすうす] 感[かん]づいていた。	薄々=うすうす= (少し) 
\\	(かすかに) 
\\	病気をしてから髪が薄くなった。	
\\	病気[びょうき]をしてから 髪[かみ]が 薄[うす]くなった。	薄い=うすい= 
\\	(少ない) 
\\	キュウリは薄く切ってください。	
\\	キュウリは 薄[うす]く 切[き]ってください。	薄い=うすい= 
\\	(少ない) 
\\	東大に合格する見込みは非常に薄い。	
\\	東大[とうだい]に 合格[ごうかく]する 見込[みこ]みは 非常[ひじょう]に 薄[うす]い。	薄い=うすい= 
\\	(少ない) 
\\	インクが薄すぎて字が読めない。	
\\	インクが 薄[うす]すぎて 字[じ]が 読[よ]めない。	薄い=うすい= 
\\	(少ない) 
\\	そのホテルは壁が薄っぺらで隣の部屋の音が筒抜けだ。	
\\	そのホテルは 壁[かべ]が 薄[うす]っぺらで 隣[となり]の 部屋[へや]の 音[おと]が 筒抜[つつぬ]けだ。	薄っぺら=うすっぺら= 
\\	筒抜け=つつぬけ= 
\\	お湯を入れすぎてスープが薄まってしまった。	
\\	お 湯[ゆ]を 入[い]れすぎてスープが 薄[うす]まってしまった。	
\\	暑さが次第に薄らいだ。	
\\	暑[あつ]さが 次第[しだい]に 薄[うす]らいだ。	薄らぐ=うすらぐ= 
\\	時を経るとともに悲しみも薄らいでくるだろう。	
\\	時[とき]を 経[へ]るとともに 悲[かな]しみも 薄[うす]らいでくるだろう。	薄らぐ=うすらぐ= 
\\	最近は英会話に対する興味がすっかり薄れてしまった。	
\\	最近[さいきん]は 英会話[えいかいわ]に 対[たい]する 興味[きょうみ]がすっかり 薄[うす]れてしまった。	
\\	薄気味悪い男がこっちをじっと見つめていた。	
\\	薄気味悪[うすきみわる]い 男[おとこ]がこっちをじっと 見[み]つめていた。	薄気味(の)悪い=うすきみ〜= 
\\	私は薄給で食うや食わずの有り様だ。	
\\	私[わたし]は 薄給[はっきゅう]で 食[く]うや 食[く]わずの 有り様[ありさま]だ。	薄給=はっきゅう= 
\\	食うや食わず= 
\\	私にはそんな薄情なことはできない。	
\\	私[わたし]にはそんな 薄情[はくじょう]なことはできない。	薄情=はくじょう= (無情な) 
\\	あんたって薄情ね!	
\\	あんたって 薄情[はくじょう]ね!	薄情=はくじょう= (無情な) 
\\	あの子はなかなかのお茶目さんだ	
\\	あの 子[こ]はなかなかのお 茶[ちゃ] 目[め]さんだ	お茶目=おちゃめ= 
\\	(人) 
\\	彼女は喜怒哀楽が激しい。	
\\	彼女[かのじょ]は 喜怒哀楽[きどあいらく]が 激[はげ]しい。	喜怒哀楽=きどあいらく= (感情) 
\\	彼の演説は怒号でかき消された。	
\\	彼[かれ]の 演説[えんぜつ]は 怒号[どごう]でかき 消[け]された。	怒号=どごう= 
\\	かき消す= 
\\	彼は彼女の無礼な態度に激怒した。	
\\	彼[かれ]は 彼女[かのじょ]の 無礼[ぶれい]な 態度[たいど]に 激怒[げきど]した。	激怒=げきど= 
\\	彼は激怒のあまり机をたたいた。	
\\	彼[かれ]は 激怒[げきど]のあまり 机[つくえ]をたたいた。	激怒=げきど= 
\\	激怒した男は刃物を振り上げて襲いかかってきた。	
\\	激怒[げきど]した 男[おとこ]は 刃物[はもの]を 振り上[ふりあ]げて 襲[おそ]いかかってきた。	激怒=げきど= 
\\	刃物=はもの= 
\\	襲いかかる=おそいかかる= 
\\	ボールは壁に当たってはね返った。	
\\	ボールは 壁[かべ]に 当[あ]たってはね 返[かえ]った。	はね返る= 
\\	ベルリンは壁によって隔てられていた。	
\\	ベルリンは 壁[かべ]によって 隔[へだ]てられていた。	隔てる=へだてる= 
\\	留学しても言葉の壁が乗り越えられなかった。	
\\	留学[りゅうがく]しても 言葉[ことば]の 壁[かべ]が 乗り越[のりこ]えられなかった。	
\\	君の論拠は薄弱だ。	
\\	君[きみ]の 論拠[ろんきょ]は 薄弱[はくじゃく]だ。	
\\	その手の商品は在庫が手薄になっている。	
\\	その 手[て]の 商品[しょうひん]は 在庫[ざいこ]が 手薄[てうす]になっている。	在庫=ざいこ= 
\\	手薄=てうす= 
\\	品薄になった。	
\\	品薄[しなうす]になった。	品薄=しなうす
\\	横浜と言うとすぐ中華街を連想する。	
\\	横浜[よこはま]と 言[い]うとすぐ 中華[ちゅうか] 街[がい]を 連想[れんそう]する。	連想=れんそう= 
\\	リンカーンというと奴隷解放を連想する。	
\\	リンカーンというと 奴隷[どれい] 解放[かいほう]を 連想[れんそう]する。	奴隷解放=どれいかいほう= 
\\	連想=れんそう= 
\\	彼は飲み進めるにつれて目が据わってきた。	
\\	彼[かれ]は 飲[の]み 進[すす]めるにつれて 目[め]が 据[す]わってきた。	目が据わる=めがすわる= 
\\	レストランで勘定書を見て目が点になった。	
\\	レストランで 勘定[かんじょう] 書[しょ]を 見[み]て 目[め]が 点[てん]になった。	勘定書=かんじょうしょ= 
\\	目が点になる= 
\\	彼は細かなところまで目が届く。	
\\	彼[かれ]は 細[こま]かなところまで 目[め]が 届[とど]く。	
\\	子供は親の目が届かないところでいたずらをするものだ。	
\\	子供[こども]は 親[おや]の 目[め]が 届[とど]かないところでいたずらをするものだ。	
\\	彼女は甘いものに目がない。	
\\	彼女[かのじょ]は 甘[あま]いものに 目[め]がない。	目がない= 
\\	その患者からは片時も目が離せない。	
\\	その 患者[かんじゃ]からは 片時[かたとき]も 目[め]が 離[はな]せない。	片時=かたとき・へんじ= 
\\	開店の準備で目が回るような忙しさだった。	
\\	開店[かいてん]の 準備[じゅんび]で 目[め]が 回[まわ]るような 忙[いそが]しさだった。	
\\	ジェットコースターから降りたら目が回って立っていられなかった。	
\\	ジェットコースターから 降[お]りたら 目[め]が 回[まわ]って 立[た]っていられなかった。	
\\	ここから駅は目と鼻の先だ。	
\\	ここから 駅[えき]は 目[め]と 鼻[はな]の 先[さき]だ。	目と鼻の先= 
\\	新しい仕事に奮闘しているあなたの姿が目に浮かぶようです。	
\\	新[あたら]しい 仕事[しごと]に 奮闘[ふんとう]しているあなたの 姿[すがた]が 目[め]に 浮[う]かぶようです。	目に浮かぶ= 
\\	人を嫌いになるとその人の欠点ばかりが目につくようになる。	
\\	人[ひと]を 嫌[きら]いになるとその 人[ひと]の 欠点[けってん]ばかりが 目[め]につくようになる。	
\\	ある小さな新聞記事が彼女の目に留まった。	
\\	ある 小[ちい]さな 新聞[しんぶん] 記事[きじ]が 彼女[かのじょ]の 目[め]に 留[と]まった。	目に留まる= 
\\	彼女の病状は目に見えてよくなっている。	
\\	彼女[かのじょ]の 病状[びょうじょう]は 目[め]に 見[み]えてよくなっている。	目に見える= 
\\	彼が事業に失敗するのは目に見えていた。	
\\	彼[かれ]が 事業[じぎょう]に 失敗[しっぱい]するのは 目[め]に 見[み]えていた。	目に見える= 
\\	試験が近づいて、彼女の目の色が変わった。	
\\	試験[しけん]が 近[ちか]づいて、 彼女[かのじょ]の 目[め]の 色[いろ]が 変[か]わった。	目の色を変える= 
\\	この漫画は子供には目の毒だ。	
\\	この 漫画[まんが]は 子供[こども]には 目[め]の 毒[どく]だ。	目の毒= 
\\	あのネックレスは妻には目の毒だよ。	
\\	あのネックレスは 妻[つま]には 目[め]の 毒[どく]だよ。	目の毒= 
\\	娘は目の中に入れても痛くないほどかわいい。	
\\	娘[むすめ]は 目[め]の 中[なか]に 入[い]れても 痛[いた]くないほどかわいい。	目の中に入れても痛くない= 
\\	被災地は目も当てられない惨状だ。	
\\	被災[ひさい] 地[ち]は 目[め]も 当[あ]てられない 惨状[さんじょう]だ。	
\\	彼女は仕事以外のことには目もくれない。	
\\	彼女[かのじょ]は 仕事[しごと] 以外[いがい]のことには 目[め]もくれない。	目もくれない= 
\\	彼女の華やかな装いに目を奪われた。	
\\	彼女[かのじょ]の 華[はな]やかな 装[よそお]いに 目[め]を 奪[うば]われた。	
\\	映画の暴力シーンに目を覆った。	
\\	映画[えいが]の 暴力[ぼうりょく]シーンに 目[め]を 覆[おお]った。	
\\	部長は彼に目をかけている。	
\\	部長[ぶちょう]は 彼[かれ]に 目[め]をかけている。	目をかける= 
\\	子供たちに事故がないように目を配ってください。	
\\	子供[こども]たちに 事故[じこ]がないように 目[め]を 配[くば]ってください。	
\\	目を覚まして現実を直視しなさい。	
\\	目[め]を 覚[さ]まして 現実[げんじつ]を 直視[ちょくし]しなさい。	
\\	彼女は目を三角にして息子を叱った。	
\\	彼女[かのじょ]は 目[め]を 三角[さんかく]にして 息子[むすこ]を 叱[しか]った。	目を三角にする= 
\\	彼は数ヶ月前からテロリストとして目を付けられていた。	
\\	彼[かれ]は 数[すう] ヶ月[かげつ] 前[まえ]からテロリストとして 目[め]を 付[つ]けられていた。	
\\	彼は彼女の指輪に目を留めた。	
\\	彼[かれ]は 彼女[かのじょ]の 指輪[ゆびわ]に 目[め]を 留[と]めた。	
\\	ちょっと目を離したすきにかばんがなくなっていた。	
\\	ちょっと 目[め]を 離[はな]したすきにかばんがなくなっていた。	
\\	テレビの画面から目を離すことができなかった。	
\\	テレビの 画面[がめん]から 目[め]を 離[はな]すことができなかった。	
\\	彼のピアノの急速な進歩に先生は目を丸くした。	
\\	彼[かれ]のピアノの 急速[きゅうそく]な 進歩[しんぽ]に 先生[せんせい]は 目[め]を 丸[まる]くした。	目を丸くする= 
\\	彼は木から落ちた時、頭を打って目を回した。	
\\	彼[かれ]は 木[き]から 落[お]ちた 時[とき]、 頭[あたま]を 打[う]って 目[め]を 回[まわ]した。	
\\	折からの大雨にもかかわらず我々は出発した。	
\\	折[おり]からの 大雨[おおあめ]にもかかわらず 我々[われわれ]は 出発[しゅっぱつ]した。	折から=おりから= 
\\	折から雪が降り出した。	
\\	折[おり]から 雪[ゆき]が 降り出[ふりだ]した。	折から=おりから= 
\\	私達は公園にテントを張って、被災者のための応急の宿舎とした。	
\\	私[わたし] 達[たち]は 公園[こうえん]にテントを 張[は]って、 被災[ひさい] 者[しゃ]のための 応急[おうきゅう]の 宿舎[しゅくしゃ]とした。	応急=おうきゅう= 
\\	破裂したパイプを応急修理した。	
\\	破裂[はれつ]したパイプを 応急[おうきゅう] 修理[しゅうり]した。	応急=おうきゅう= 
\\	その医者は患者に応急手当を施した。	
\\	その 医者[いしゃ]は 患者[かんじゃ]に 応急[おうきゅう] 手当[てあて]を 施[ほどこ]した。	応急=おうきゅう= 
\\	彼らは直接あるいは間接に西欧の思想の影響を受けた。	
\\	彼[かれ]らは 直接[ちょくせつ]あるいは 間接[かんせつ]に 西欧[せいおう]の 思想[しそう]の 影響[えいきょう]を 受[う]けた。	
\\	彼の思想はその時代よりも100年進んでいた。	
\\	彼[かれ]の 思想[しそう]はその 時代[じだい]よりも100 年[ねん] 進[すす]んでいた。	
\\	訪問者は名前を告げずに立ち去った。	
\\	訪問[ほうもん] 者[しゃ]は 名前[なまえ]を 告[つ]げずに 立ち去[たちさ]った。	告げる=つげる= (知らせる) 
\\	(指示する) 
\\	(広告する) 
\\	(気づかせる) 
\\	(その状態になる)
\\	機長は間もなく空港に着陸すると告げた。	
\\	機長[きちょう]は 間[ま]もなく 空港[くうこう]に 着陸[ちゃくりく]すると 告[つ]げた。	告げる=つげる= (知らせる) 
\\	(指示する) 
\\	(広告する) 
\\	(気づかせる) 
\\	(その状態になる)
\\	彼の突然の引退表明にファンは驚いた。	
\\	彼[かれ]の 突然[とつぜん]の 引退[いんたい] 表明[ひょうめい]にファンは 驚[おどろ]いた。	表明=ひょうめい= 
\\	彼はその候補者への支持を表明した。	
\\	彼[かれ]はその 候補[こうほ] 者[しゃ]への 支持[しじ]を 表明[ひょうめい]した。	表明=ひょうめい= 
\\	労組はストライキの決行を会社側に通告した。	
\\	労組[ろうそ]はストライキの 決行[けっこう]を 会社[かいしゃ] 側[がわ]に 通告[つうこく]した。	通告=つうこく= 
\\	彼は母の病気を治したいという動機から医者になった。	
\\	彼[かれ]は 母[はは]の 病気[びょうき]を 治[なお]したいという 動機[どうき]から 医者[いしゃ]になった。	動機=どうき= 
\\	ちょっとしたことが戦争を引き起こす動機となることもある。	
\\	ちょっとしたことが 戦争[せんそう]を 引き起[ひきお]こす 動機[どうき]となることもある。	動機=どうき= 
\\	フランス語を習い始めた動機は何ですか。	
\\	フランス語[ふらんすご]を 習[なら]い 始[はじ]めた 動機[どうき]は 何[なに]ですか。	動機=どうき= 
\\	その薬は私に少しも効能がなかった。	
\\	その 薬[くすり]は 私[わたし]に 少[すこ]しも 効能[こうのう]がなかった。	効能=こうのう= (ききめ・効力) 
\\	(効果)
\\	この薬は私には非常に効能があった。	
\\	この 薬[くすり]は 私[わたし]には 非常[ひじょう]に 効能[こうのう]があった。	効能=こうのう= (ききめ・効力) 
\\	(効果)
\\	決して他言をしないと誓っていただけますか。	
\\	決[けっ]して 他言[たごん]をしないと 誓[ちか]っていただけますか。	誓う=ちかう= 
\\	互いに助け合うことを誓った。	
\\	互[たが]いに 助け合[たすけあ]うことを 誓[ちか]った。	誓う=ちかう= 
\\	独身で通すと誓った。	
\\	独身[どくしん]で 通[とお]すと 誓[ちか]った。	誓う=ちかう= 
\\	誓ってそんなことはない。	
\\	誓[ちか]ってそんなことはない。	誓う=ちかう= 
\\	人間は腹が減ると怒りっぽくなる。	
\\	人間[にんげん]は 腹[はら]が 減[へ]ると 怒[いか]りっぽくなる。	
\\	万事円滑に運んだ。	
\\	万事[ばんじ] 円滑[えんかつ]に 運[はこ]んだ。	円滑=えんかつ= 
\\	(支障のない) 
\\	ねらいを定めて引き金を引いた。	
\\	ねらいを 定[さだ]めて 引き金[ひきがね]を 引[ひ]いた。	定める=さだめる= 
\\	給与の支払いは毎月25日と契約書に定められている。	
\\	給与[きゅうよ]の 支払[しはら]いは 毎月[まいつき]25 日[にち]と 契約[けいやく] 書[しょ]に 定[さだ]められている。	定める=さだめる= 
\\	彼らは停戦に合意する前にいくつかの条件を定めた。	
\\	彼[かれ]らは 停戦[ていせん]に 合意[ごうい]する 前[まえ]にいくつかの 条件[じょうけん]を 定[さだ]めた。	定める=さだめる= 
\\	命令には逆らえなかった。	
\\	命令[めいれい]には 逆[さか]らえなかった。	逆らう=さからう= 
\\	(命令などに) 
\\	公衆電話が廃止の方向にある。	
\\	公衆[こうしゅう] 電話[でんわ]が 廃止[はいし]の 方向[ほうこう]にある。	廃止=はいし= 
\\	(法律などの) 
\\	彼らは訴訟合戦を展開した。	
\\	彼[かれ]らは 訴訟[そしょう] 合戦[かっせん]を 展開[てんかい]した。	合戦=かっせん= 
\\	こんな物を持っていては足手まといだ。	
\\	こんな 物[もの]を 持[も]っていては 足[あし] 手[しゅ]まといだ。	足手纏い=あしてまとい・あしでまとい= 
\\	私はあなたの足手まといになりたくない。	
\\	私[わたし]はあなたの 足[あし] 手[しゅ]まといになりたくない。	足手纏い=あしてまとい・あしでまとい= 
\\	捜索隊は行方不明者に関する何の手がかりもつかめなかった。	
\\	捜索[そうさく] 隊[たい]は 行方[ゆくえ] 不明[ふめい] 者[しゃ]に 関[かん]する 何[なに]の 手[て]がかりもつかめなかった。	手掛かり= 
\\	(犯人などの) 
\\	これが暗号を解く手がかりだ。	
\\	これが 暗号[あんごう]を 解[と]く 手[て]がかりだ。	手掛かり= 
\\	(犯人などの) 
\\	その手がかりを追って真相解明にたどり着いた。	
\\	その 手[て]がかりを 追[お]って 真相[しんそう] 解明[かいめい]にたどり 着[つ]いた。	手掛かり= 
\\	(犯人などの) 
\\	真相=しんそう= 
\\	解明=かいめい= 
\\	私たちは手分けしてその仕事をした。	
\\	私[わたし]たちは 手分[てわ]けしてその 仕事[しごと]をした。	手分け=てわけ= 
\\	私たちは手分けして行方不明の少年を捜した。	
\\	私[わたし]たちは 手分[てわ]けして 行方[ゆくえ] 不明[ふめい]の 少年[しょうねん]を 捜[さが]した。	手分け=てわけ= 
\\	彼女はその紛争の処理に鮮やかな手際を見せた。	
\\	彼女[かのじょ]はその 紛争[ふんそう]の 処理[しょり]に 鮮[あざ]やかな 手際[てぎわ]を 見[み]せた。	手際=てぎわ= 
\\	このパソコンは手頃な価格だ。	
\\	このパソコンは 手頃[てごろ]な 価格[かかく]だ。	手頃=てごろ= 
\\	我々は自動車に過度に依存している。	
\\	我々[われわれ]は 自動車[じどうしゃ]に 過度[かど]に 依存[いぞん]している。	
\\	確かめるすべがない。	
\\	確[たし]かめるすべがない。	術=すべ= (方法) 
\\	食中毒が怖いので用心して私は夏には生ものは食べない。	
\\	食中毒[しょくちゅうどく]が 怖[こわ]いので 用心[ようじん]して 私[わたし]は 夏[なつ]には 生[なま]ものは 食[た]べない。	用心=ようじん= (注意) 
\\	用心に傘を持って行きなさい。	
\\	用心[ようじん]に 傘[かさ]を 持[も]って 行[い]きなさい。	用心=ようじん= (注意) 
\\	割れたガラスに用心して!	
\\	割[わ]れたガラスに 用心[ようじん]して!	用心=ようじん= (注意) 
\\	にわかに断定はできない。	
\\	にわかに 断定[だんてい]はできない。	にわかに= 
\\	断定=だんてい= (はっきりした下した判断) 
\\	彼女が犯人だという疑いがあるがまだ断定はできない。	
\\	彼女[かのじょ]が 犯人[はんにん]だという 疑[うたが]いがあるがまだ 断定[だんてい]はできない。	断定=だんてい= (はっきりした下した判断) 
\\	この記事がどこまで本当かを断定するのは難しい。	
\\	この 記事[きじ]がどこまで 本当[ほんとう]かを 断定[だんてい]するのは 難[むずか]しい。	断定=だんてい= (はっきりした下した判断) 
\\	この調査は問題の深刻さを浮き彫りにした。	
\\	この 調査[ちょうさ]は 問題[もんだい]の 深刻[しんこく]さを 浮き彫[うきぼ]りにした。	浮き彫り=うきぼり= 
\\	この事件は両国の関係改善にとって障壁となるだろう。	
\\	この 事件[じけん]は 両国[りょうこく]の 関係[かんけい] 改善[かいぜん]にとって 障壁[しょうへき]となるだろう。	障壁=しょうへき= (しきり) 
\\	(さまたげ) 
\\	それが在宅老人介護の死角である。	
\\	それが 在宅[ざいたく] 老人[ろうじん] 介護[かいご]の 死角[しかく]である。	死角=しかく= 
\\	高成長を続ける中国経済に死角はないのか。	
\\	高[こう] 成長[せいちょう]を 続[つづ]ける 中国[ちゅうごく] 経済[けいざい]に 死角[しかく]はないのか。	死角=しかく= 
\\	彼は擦れ違っても挨拶一つしない。	
\\	彼[かれ]は 擦れ違[すれちが]っても 挨拶[あいさつ] 一[ひと]つしない。	擦れ違う=すれちがう= (行き違う) 
\\	(論点が合わない)
\\	共働きで夫とは擦れ違いの毎日です。	
\\	共働[ともばたら]きで 夫[おっと]とは 擦[す]れ 違[ちが]いの 毎日[まいにち]です。	擦れ違い=すれちがい= (行き違うこと) 
\\	(間が悪くて会えないこと); (論点が合わないこと)
\\	道が細くて対向車との擦れ違いができなかった。	
\\	道[みち]が 細[ほそ]くて 対向[たいこう] 車[しゃ]との 擦[す]れ 違[ちが]いができなかった。	擦れ違い=すれちがい= (行き違うこと) 
\\	(間が悪くて会えないこと); (論点が合わないこと)
\\	彼は目撃者を脅して黙らせた。	
\\	彼[かれ]は 目撃[もくげき] 者[しゃ]を 脅[おど]して 黙[だま]らせた。	脅す=おどす= (脅かす), (威嚇する) 
\\	彼らは言う通りにしなければ殺すぞと私を脅した。	
\\	彼[かれ]らは 言[い]う 通[とお]りにしなければ 殺[ころ]すぞと 私[わたし]を 脅[おど]した。	脅す=おどす= (脅かす), (威嚇する) 
\\	彼は彼女をピストルで脅して金を奪った。	
\\	彼[かれ]は 彼女[かのじょ]をピストルで 脅[おど]して 金[きん]を 奪[うば]った。	脅す=おどす= (脅かす), (威嚇する) 
\\	あわてて電車を乗り違えた。	
\\	あわてて 電車[でんしゃ]を 乗[の]り 違[ちが]えた。	慌てる=あわてる= (落ち着きを失う) 
\\	(ひどく急ぐ) 
\\	(狼狽する)
\\	期限ぎりぎりになってあわててレポートを書き始めた。	
\\	期限[きげん]ぎりぎりになってあわててレポートを 書[か]き 始[はじ]めた。	慌てる=あわてる= (落ち着きを失う) 
\\	(ひどく急ぐ) 
\\	(狼狽する)
\\	友人が交通事故にあったという知らせをもらって、あわてて病院に駆けつけた。	
\\	友人[ゆうじん]が 交通[こうつう] 事故[じこ]にあったという 知[し]らせをもらって、あわてて 病院[びょういん]に 駆[か]けつけた。	慌てる=あわてる= (落ち着きを失う) 
\\	(ひどく急ぐ) 
\\	(狼狽する)
\\	このスーツケースは、急に海外出張が決まった時に慌てて買いに行ったものだ。	
\\	このスーツケースは、 急[きゅう]に 海外[かいがい] 出張[しゅっちょう]が 決[き]まった 時[とき]に 慌[あわ]てて 買[か]いに 行[い]ったものだ。	慌てる=あわてる= (落ち着きを失う) 
\\	(ひどく急ぐ) 
\\	(狼狽する)
\\	彼は危険に臨んで慌てなかった。	
\\	彼[かれ]は 危険[きけん]に 臨[のぞ]んで 慌[あわ]てなかった。	慌てる=あわてる= (落ち着きを失う) 
\\	(ひどく急ぐ) 
\\	(狼狽する)
\\	一瞬慌てたが、すぐに落ち着きを取り戻した。	
\\	一瞬[いっしゅん] 慌[あわ]てたが、すぐに 落ち着[おちつ]きを 取り戻[とりもど]した。	慌てる=あわてる= (落ち着きを失う) 
\\	(ひどく急ぐ) 
\\	(狼狽する)
\\	彼はなかなか本心を明かさない。	
\\	彼[かれ]はなかなか 本心[ほんしん]を 明[あ]かさない。	明かす=あかす= (夜を過ごす) 
\\	(明らかにする) 
\\	名前も顔も明かさない約束で、私は取材に応じた。	
\\	名前[なまえ]も 顔[かお]も 明[あ]かさない 約束[やくそく]で、 私[わたし]は 取材[しゅざい]に 応[おう]じた。	明かす=あかす= (夜を過ごす) 
\\	(明らかにする) 
\\	取材=しゅざい= 
\\	トンネルの向こうに明かりが見えてきた。	
\\	トンネルの 向[む]こうに 明[あ]かりが 見[み]えてきた。	明かり=あかり= (灯火) 
\\	(光) 
\\	ようやく雨がやみ、雲間から一筋の明かりがさしてきた。	
\\	ようやく 雨[あめ]がやみ、 雲間[くもま]から 一筋[ひとすじ]の 明[あ]かりがさしてきた。	明かり=あかり= (灯火) 
\\	(光) 
\\	雲間=くもま= 雲の切れているところ。
\\	明かりを消して!	
\\	明[あ]かりを 消[け]して!	明かり=あかり= (灯火) 
\\	(光) 
\\	長距離レースではスタミナの配分が大事だ。	
\\	長距離[ちょうきょり]レースではスタミナの 配分[はいぶん]が 大事[だいじ]だ。	配分=はいぶん= 
\\	犯罪と貧困の間には密接な関係がある。	
\\	犯罪[はんざい]と 貧困[ひんこん]の 間[あいだ]には 密接[みっせつ]な 関係[かんけい]がある。	密接(〜な)=みっせつ= (関係の深い) 
\\	二つの考えは密接に結ぶ付いている。	
\\	二[ふた]つの 考[かんが]えは 密接[みっせつ]に 結[むす]ぶ 付[つ]いている。	密接(〜な)=みっせつ= (関係の深い) 
\\	気骨のない男だ。	
\\	気骨[きこつ]のない 男[おとこ]だ。	気骨=きこつ= 
\\	(気概)
\\	最近は気概のある政治家が見当たらない。	
\\	最近[さいきん]は 気概[きがい]のある 政治[せいじ] 家[か]が 見当[みあ]たらない。	気概=きがい= 
\\	見当たらない=みあたらない= 
\\	それは小説の良い題材になる。	
\\	それは 小説[しょうせつ]の 良[よ]い 題材[だいざい]になる。	題材=だいざい= 
\\	画家は故郷の景色に題材を求めた。	
\\	画家[がか]は 故郷[こきょう]の 景色[けしき]に 題材[だいざい]を 求[もと]めた。	題材=だいざい= 
\\	この製品には最先端技術が集約されている。	
\\	この 製品[せいひん]には 最先端[さいせんたん] 技術[ぎじゅつ]が 集約[しゅうやく]されている。	最先端=さいせんたん= 
\\	集約=しゅうやく= (寄せ集める) 
\\	(要約する) 
\\	(一つにまとめ上げる) 
\\	この教科書は数校で採用している。	
\\	この 教科書[きょうかしょ]は 数[すう] 校[こう]で 採用[さいよう]している。	採用=さいよう= (方法や意見の) 
\\	(任用) 
\\	(雇用) 
\\	ファンはビートルズの再結成を待ち望んだ。	
\\	ファンはビートルズの 再[さい] 結成[けっせい]を 待ち望[まちのぞ]んだ。	結成=けっせい= 
\\	犯人逮捕は時間の問題だ。	
\\	犯人[はんにん] 逮捕[たいほ]は 時間[じかん]の 問題[もんだい]だ。	
\\	大統領の演説はテレビ放送のために録画された。	
\\	大統領[だいとうりょう]の 演説[えんぜつ]はテレビ 放送[ほうそう]のために 録画[ろくが]された。	
\\	その機種は製造中止になっている。	
\\	その 機種[きしゅ]は 製造[せいぞう] 中止[ちゅうし]になっている。	製造=せいぞう= 
\\	その工場では年間5万台の自動車を製造している。	
\\	その 工場[こうじょう]では 年間[ねんかん]5 万[まん] 台[だい]の 自動車[じどうしゃ]を 製造[せいぞう]している。	製造=せいぞう= 
\\	候補者選びが難航している。	
\\	候補[こうほ] 者[しゃ] 選[えら]びが 難航[なんこう]している。	
\\	政府はこの問題の対応に苦慮している。	
\\	政府[せいふ]はこの 問題[もんだい]の 対応[たいおう]に 苦慮[くりょ]している。	対応=たいおう= (相当すること) 
\\	(釣り合うこと) 
\\	{電算} 
\\	(処理・応対) 
\\	問い合わせへの対応に追われた。	
\\	問い合[といあ]わせへの 対応[たいおう]に 追[お]われた。	対応=たいおう= (相当すること) 
\\	(釣り合うこと) 
\\	{電算} 
\\	(処理・応対) 
\\	「頂きます」に正確に対応する英語はない。	
\\	頂[いただ]きます」に 正確[せいかく]に 対応[たいおう]する 英語[えいご]はない。	対応=たいおう= (相当すること) 
\\	(釣り合うこと) 
\\	{電算} 
\\	(処理・応対) 
\\	サミットは各国首脳が顔を合わせ対話する良い機会だ。	
\\	サミットは 各国[かっこく] 首脳[しゅのう]が 顔[かお]を 合[あ]わせ 対話[たいわ]する 良[よ]い 機会[きかい]だ。	対話=たいわ= 
\\	彼はイタリアからの移民の子孫だ。	
\\	彼[かれ]はイタリアからの 移民[いみん]の 子孫[しそん]だ。	移民=いみん= 
\\	子孫=しそん= 
\\	アメリカは移民の国である。	
\\	アメリカは 移民[いみん]の 国[くに]である。	移民=いみん= 
\\	その国にはアジアからの移民が多い。	
\\	その 国[くに]にはアジアからの 移民[いみん]が 多[おお]い。	移民=いみん= 
\\	アフガニスタン戦争の結果、大量の難民が発生した。	
\\	アフガニスタン 戦争[せんそう]の 結果[けっか]、 大量[たいりょう]の 難民[なんみん]が 発生[はっせい]した。	難民=なんみん= 
\\	大量の難民が近隣諸国に押し寄せた。	
\\	大量[たいりょう]の 難民[なんみん]が 近隣[きんりん] 諸国[しょこく]に 押し寄[おしよ]せた。	難民=なんみん= 
\\	押し寄せる=おしよせる= 
\\	あの試合は八百長だったに違いない。	
\\	あの 試合[しあい]は 八百長[やおちょう]だったに 違[ちが]いない。	
\\	イライラしてつい子供に八つ当たりしてしまった。	
\\	イライラしてつい 子供[こども]に 八[やっ]つ 当[あ]たりしてしまった。	八つ当たり=やつあたり= 
\\	仕事のことで私に八つ当たりしないでよ。	
\\	仕事[しごと]のことで 私[わたし]に 八[やっ]つ 当[あ]たりしないでよ。	八つ当たり=やつあたり= 
\\	勝算があると踏んだからこそ私は立候補したのだ。	
\\	勝算[しょうさん]があると 踏[ふ]んだからこそ 私[わたし]は 立候補[りっこうほ]したのだ。	踏む=ふむ= 
\\	(その場に立つ) 
\\	(訪れる) 
\\	(経験する) 
\\	(評価する・見積もる) 
\\	その土は踏んだ感じが柔らかかった。	
\\	その 土[ど]は 踏[ふ]んだ 感[かん]じが 柔[やわ]らかかった。	踏む=ふむ= 
\\	(その場に立つ) 
\\	(訪れる) 
\\	(経験する) 
\\	(評価する・見積もる) 
\\	彼の言い方は脅迫めいていた。	
\\	彼[かれ]の 言い方[いいかた]は 脅迫[きょうはく]めいていた。	ーめく= 
\\	あの家は先週から雨戸を立てたままだ。	
\\	あの 家[いえ]は 先週[せんしゅう]から 雨戸[あまど]を 立[た]てたままだ。	立てる=たてる= (直立させる) 
\\	(定める) 
\\	(表明する) 
\\	(発生させる) 
\\	(はっきりさせる) 
\\	彼女はいびきを立てて眠っていた。	
\\	彼女[かのじょ]はいびきを 立[た]てて 眠[ねむ]っていた。	立てる=たてる= (直立させる) 
\\	(定める) 
\\	(表明する) 
\\	(発生させる) 
\\	(はっきりさせる) 
\\	彼女は声を立てて笑った。	
\\	彼女[かのじょ]は 声[こえ]を 立[た]てて 笑[わら]った。	立てる=たてる= (直立させる) 
\\	(定める) 
\\	(表明する) 
\\	(発生させる) 
\\	(はっきりさせる) 
\\	彼らはいつも私を立ててくれる。	
\\	彼[かれ]らはいつも 私[わたし]を 立[た]ててくれる。	立てる=たてる= (直立させる) 
\\	(定める) 
\\	(表明する) 
\\	(発生させる) 
\\	(はっきりさせる) 
\\	彼は親指を立てて車を止めようとした。	
\\	彼[かれ]は 親指[おやゆび]を 立[た]てて 車[くるま]を 止[と]めようとした。	立てる=たてる= (直立させる) 
\\	(定める) 
\\	(表明する) 
\\	(発生させる) 
\\	(はっきりさせる) 
\\	それは明治維新に先立つこと10年であった。	
\\	それは 明治維新[めいじいしん]に 先立[さきだ]つこと10 年[ねん]であった。	先立つ=さきだつ= (先行する) 
\\	(先に死ぬ) 
\\	彼は自分の感情を制することができなかった。	
\\	彼[かれ]は 自分[じぶん]の 感情[かんじょう]を 制[せい]することができなかった。	制する=せいする= (抑制する) 
\\	(制圧する) 
\\	(自分のものとする・支配する) 
\\	わずかにタイミングが外れた。	
\\	わずかにタイミングが 外[はず]れた。	外れる=はずれる= (中心・定位置などから離れる) 
\\	(標準と異なる) 
\\	(除外される) 
\\	私の目算は外れた。	
\\	私[わたし]の 目算[もくさん]は 外[はず]れた。	外れる=はずれる= (中心・定位置などから離れる) 
\\	(標準と異なる) 
\\	(除外される) 
\\	目算=もくさん= 
\\	期待が外れた。	
\\	期待[きたい]が 外[はず]れた。	外れる=はずれる= (中心・定位置などから離れる) 
\\	(標準と異なる) 
\\	(除外される) 
\\	天気予報は外れることもある。	
\\	天気[てんき] 予報[よほう]は 外[はず]れることもある。	外れる=はずれる= (中心・定位置などから離れる) 
\\	(標準と異なる) 
\\	(除外される) 
\\	弾丸は急所を外れた。	
\\	弾丸[だんがん]は 急所[きゅうしょ]を 外[はず]れた。	急所=きゅうしょ= 
\\	外れる=はずれる= (中心・定位置などから離れる) 
\\	(標準と異なる) 
\\	(除外される) 
\\	犬の首輪が外れた。	
\\	犬[いぬ]の 首輪[くびわ]が 外[はず]れた。	外れる=はずれる= (中心・定位置などから離れる) 
\\	(標準と異なる) 
\\	(除外される) 
\\	第1ボタンが外れているよ。	
\\	第[だい]1ボタンが 外[はず]れているよ。	外れる=はずれる= (中心・定位置などから離れる) 
\\	(標準と異なる) 
\\	(除外される) 
\\	肩の関節が外れた。	
\\	肩[かた]の 関節[かんせつ]が 外[はず]れた。	外れる=はずれる= (中心・定位置などから離れる) 
\\	(標準と異なる) 
\\	(除外される) 
\\	車は道を外れ、塀に激突した。	
\\	車[くるま]は 道[みち]を 外[はず]れ、 塀[へい]に 激突[げきとつ]した。	外れる=はずれる= (中心・定位置などから離れる) 
\\	(標準と異なる) 
\\	(除外される) 
\\	モモはまだ実っていない。	
\\	モモはまだ 実[みの]っていない。	実る=みのる= (実が熟す) 
\\	(結果が出る) 
\\	君の努力は実る日がきっと来る。	
\\	君[きみ]の 努力[どりょく]は 実[みの]る 日[ひ]がきっと 来[く]る。	実る=みのる= (実が熟す) 
\\	(結果が出る) 
\\	彼の恋は実らなかった。	
\\	彼[かれ]の 恋[こい]は 実[みの]らなかった。	実る=みのる= (実が熟す) 
\\	(結果が出る) 
\\	長年の苦労がやっと実った。	
\\	長年[ながねん]の 苦労[くろう]がやっと 実[みの]った。	実る=みのる= (実が熟す) 
\\	(結果が出る) 
\\	思考力の方が暗記力よりずっと価値が高い。	
\\	思考[しこう] 力[りょく]の 方[ほう]が 暗記[あんき] 力[りょく]よりずっと 価値[かち]が 高[たか]い。	
\\	立候補の意思を明らかにしているのは今のところ4人である。	
\\	立候補[りっこうほ]の 意思[いし]を 明[あき]らかにしているのは 今[いま]のところ4 人[にん]である。	意思=いし= 
\\	彼女は手を挙げて賛成の意思を示した。	
\\	彼女[かのじょ]は 手[て]を 挙[あ]げて 賛成[さんせい]の 意思[いし]を 示[しめ]した。	意思=いし= 
\\	英語で自分の意思を伝えられるようになるのに僕は半年かかった。	
\\	英語[えいご]で 自分[じぶん]の 意思[いし]を 伝[つた]えられるようになるのに 僕[ぼく]は 半年[はんとし]かかった。	意思=いし= 
\\	言葉は意思を伝達する重要な手段である。	
\\	言葉[ことば]は 意思[いし]を 伝達[でんたつ]する 重要[じゅうよう]な 手段[しゅだん]である。	意思=いし= 
\\	彼はまだ意思が固まっていない。	
\\	彼[かれ]はまだ 意思[いし]が 固[かた]まっていない。	意思=いし= 
\\	お互い無言のうちに了解していた。	
\\	お 互[たが]い 無言[むごん]のうちに 了解[りょうかい]していた。	無言=むごん= 
\\	二人は無言のまま外へ出た。	
\\	二人[ふたり]は 無言[むごん]のまま 外[そと]へ 出[で]た。	無言=むごん= 
\\	彼は終始無言であった。	
\\	彼[かれ]は 終始[しゅうし] 無言[むごん]であった。	無言=むごん= 
\\	私の目は右は正常ですが、左が近視です。	
\\	私[わたし]の 目[め]は 右[みぎ]は 正常[せいじょう]ですが、 左[ひだり]が 近視[きんし]です。	正常=せいじょう= 
\\	列車のダイヤが正常に戻った。	
\\	列車[れっしゃ]のダイヤが 正常[せいじょう]に 戻[もど]った。	ダイヤ= 
\\	正常=せいじょう= 
\\	この劇の主人公は貧乏な絵描きだ。	
\\	この 劇[げき]の 主人公[しゅじんこう]は 貧乏[びんぼう]な 絵描[えか]きだ。	
\\	その手紙は間違って配達された。	
\\	その 手紙[てがみ]は 間違[まちが]って 配達[はいたつ]された。	配達=はいたつ= 
\\	配達料を800円取られた。	
\\	配達[はいたつ] 料[りょう]を 800円[はっぴゃくえん] 取[と]られた。	配達=はいたつ= 
\\	その点に関しては何ら法律上の規定はない。	
\\	その 点[てん]に 関[かん]しては 何[なん]ら 法律[ほうりつ] 上[じょう]の 規定[きてい]はない。	規定=きてい= (定め) 
\\	(条項) 
\\	(規則) 
\\	我々は規定によりその会議に出席する義務がある。	
\\	我々[われわれ]は 規定[きてい]によりその 会議[かいぎ]に 出席[しゅっせき]する 義務[ぎむ]がある。	規定=きてい= (定め) 
\\	(条項) 
\\	(規則) 
\\	企業がセクハラを防止するための必要措置を講じることが法で規定されている。	
\\	企業[きぎょう]がセクハラを 防止[ぼうし]するための 必要[ひつよう] 措置[そち]を 講[こう]じることが 法[ほう]で 規定[きてい]されている。	規定=きてい= (定め) 
\\	(条項) 
\\	(規則) 
\\	会議の日程の調整が必要だ。	
\\	会議[かいぎ]の 日程[にってい]の 調整[ちょうせい]が 必要[ひつよう]だ。	調整=ちょうせい= 
\\	彼の死は労災と認定された。	
\\	彼[かれ]の 死[し]は 労災[ろうさい]と 認定[にんてい]された。	認定=にんてい= 
\\	(承認) 
\\	(認可) 
\\	(許可) 
\\	当局はそれを認可しないだろう。	
\\	当局[とうきょく]はそれを 認可[にんか]しないだろう。	認可=にんか= 
\\	乗り心地の良さにかけてはこの車が最高だ。	
\\	乗[の]り 心地[ごこち]の 良[よ]さにかけてはこの 車[くるま]が 最高[さいこう]だ。	ーにかけて= (範囲の終わりを示して) 
\\	(一事をとりあげて) 
\\	泳ぐことにかけては私はクラスの誰にも負けない。	
\\	泳[およ]ぐことにかけては 私[わたし]はクラスの 誰[だれ]にも 負[ま]けない。	ーにかけて= (範囲の終わりを示して) 
\\	(一事をとりあげて) 
\\	車修理にかけては彼が一番だ。	
\\	車[くるま] 修理[しゅうり]にかけては 彼[かれ]が 一番[いちばん]だ。	ーにかけて= (範囲の終わりを示して) 
\\	(一事をとりあげて) 
\\	肩から胸にかけて切られた。	
\\	肩[かた]から 胸[むね]にかけて 切[き]られた。	ーにかけて= (範囲の終わりを示して) 
\\	(一事をとりあげて) 
\\	十代後半から二十代にかけて私はスキーに熱中していた。	
\\	十代[じゅうだい] 後半[こうはん]から 二十代[にじゅうだい]にかけて 私[わたし]はスキーに 熱中[ねっちゅう]していた。	ーにかけて= (範囲の終わりを示して) 
\\	(一事をとりあげて) 
\\	道に迷ったあげくに財布を落としてしまい、今日は最悪だった。	
\\	道[みち]に 迷[まよ]ったあげくに 財布[さいふ]を 落[お]としてしまい、 今日[きょう]は 最悪[さいあく]だった。	挙句・揚げ句=あげく= (結果・末); 
\\	これがずいぶん考え抜いたあげくの私の結論です。	
\\	これがずいぶん 考え抜[かんがえぬ]いたあげくの 私[わたし]の 結論[けつろん]です。	挙句・揚げ句=あげく= (結果・末); 
\\	私は進学か就職かよくよく考えたあげく、就職することにした。	
\\	私[わたし]は 進学[しんがく]か 就職[しゅうしょく]かよくよく 考[かんが]えたあげく、 就職[しゅうしょく]することにした。	挙句・揚げ句=あげく= (結果・末); 
\\	彼は映画が好きで好きで、挙げ句の果てに自分で映画館を作ってしまった。	
\\	彼[かれ]は 映画[えいが]が 好[す]きで 好[す]きで、 挙げ句[あげく]の 果[は]てに 自分[じぶん]で 映画[えいが] 館[かん]を 作[つく]ってしまった。	挙句・揚げ句=あげく= (結果・末); 
\\	挙げ句の果てに=あげくのはてに= 
\\	人妻を恋した挙げ句の果て、彼は自殺してしまった。	
\\	人妻[ひとづま]を 恋[こい]した 挙げ句[あげく]の 果[は]て、 彼[かれ]は 自殺[じさつ]してしまった。	挙句・揚げ句=あげく= (結果・末); 
\\	挙げ句の果てに=あげくのはてに= 
\\	2時間も待たされたあげく、翌日来てくれと言われた。	
\\	時間[じかん]も 待[ま]たされたあげく、 翌日[よくじつ] 来[き]てくれと 言[い]われた。	挙句・揚げ句=あげく= (結果・末); 
\\	お冷やをください。	
\\	お 冷[ひ]やをください。	冷や=ひや= 
\\	時代の歩みはそれを生きている人間にはわからぬものである。	
\\	時代[じだい]の 歩[あゆ]みはそれを 生[い]きている 人間[にんげん]にはわからぬものである。	歩み=あゆみ= (歩くこと) 
\\	(歩調) 
\\	(物事の進行) 
\\	(過程) 
\\	(沿革) 
\\	私の少年期は高度経済成長と歩みを共にしていた。	
\\	私[わたし]の 少年[しょうねん] 期[き]は 高度[こうど] 経済[けいざい] 成長[せいちょう]と 歩[あゆ]みを 共[とも]にしていた。	歩み=あゆみ= (歩くこと) 
\\	(歩調) 
\\	(物事の進行) 
\\	(過程) 
\\	(沿革) 
\\	警察に呼ばれた彼の歩みは重かった。	
\\	警察[けいさつ]に 呼[よ]ばれた 彼[かれ]の 歩[あゆ]みは 重[おも]かった。	歩み=あゆみ= (歩くこと) 
\\	(歩調) 
\\	(物事の進行) 
\\	(過程) 
\\	(沿革) 
\\	核兵器廃絶への歩みを止めてはならない。	
\\	核兵器[かくへいき] 廃絶[はいぜつ]への 歩[あゆ]みを 止[と]めてはならない。	歩み=あゆみ= (歩くこと) 
\\	(歩調) 
\\	(物事の進行) 
\\	(過程) 
\\	(沿革) 
\\	彼は立ち上がって私の方に歩み寄ってきた。	
\\	彼[かれ]は 立ち上[たちあ]がって 私[わたし]の 方[ほう]に 歩み寄[あゆみよ]ってきた。	歩み寄る=あゆみよる= (近寄る) 
\\	(折れ合う) 
\\	我々はテロリストに一切歩み寄るつもりはない。	
\\	我々[われわれ]はテロリストに 一切[いっさい] 歩み寄[あゆみよ]るつもりはない。	歩み寄る=あゆみよる= (近寄る) 
\\	(折れ合う) 
\\	狐が一匹わなにかかった。	
\\	狐[きつね]が 一匹[いっぴき]わなにかかった。	罠=わな= (捕獲のしかけ) 
\\	(はかりごと) 
\\	彼はまんまと敵の罠にかかった。	
\\	彼[かれ]はまんまと 敵[てき]の 罠[わな]にかかった。	罠=わな= (捕獲のしかけ) 
\\	(はかりごと) 
\\	そのような医療行為は倫理的に問題がある。	
\\	そのような 医療[いりょう] 行為[こうい]は 倫理[りんり] 的[てき]に 問題[もんだい]がある。	倫理=りんり= 
\\	長引く不況をどう乗り切るか、首相の手腕が問われている。	
\\	長引[ながび]く 不況[ふきょう]をどう 乗り切[のりき]るか、 首相[しゅしょう]の 手腕[しゅわん]が 問[と]われている。	手腕=しゅわん= 
\\	彼女の交渉手腕には定評がある。	
\\	彼女[かのじょ]の 交渉[こうしょう] 手腕[しゅわん]には 定評[ていひょう]がある。	手腕=しゅわん= 
\\	新監督はその手腕を十分に発揮できないままで解雇された。	
\\	新[しん] 監督[かんとく]はその 手腕[しゅわん]を 十分[じゅうぶん]に 発揮[はっき]できないままで 解雇[かいこ]された。	手腕=しゅわん= 
\\	彼の冗談にどっと場内がわいた。	
\\	彼[かれ]の 冗談[じょうだん]にどっと 場内[じょうない]がわいた。	
\\	場内から盛大な拍手が湧き起こった。	
\\	場内[じょうない]から 盛大[せいだい]な 拍手[はくしゅ]が 湧[わ]き 起[お]こった。	
\\	ザッと場内を見渡したが彼の姿はなかった。	
\\	ザッと 場内[じょうない]を 見渡[みわた]したが 彼[かれ]の 姿[すがた]はなかった。	
\\	2題目の問題には間違った解答をした。	
\\	題目[だいめ]の 問題[もんだい]には 間違[まちが]った 解答[かいとう]をした。	解答=かいとう= (問題に答えること) 
\\	(問題に対する答え) 
\\	(答え)
\\	1題目の問題には正しい解答をした。	
\\	題目[だいめ]の 問題[もんだい]には 正[ただ]しい 解答[かいとう]をした。	解答=かいとう= (問題に答えること) 
\\	(問題に対する答え) 
\\	(答え)
\\	その問題の解答は一つではない。	
\\	その 問題[もんだい]の 解答[かいとう]は 一[ひと]つではない。	解答=かいとう= (問題に答えること) 
\\	(問題に対する答え) 
\\	(答え)
\\	あの会社は人使いが荒い。	
\\	あの 会社[かいしゃ]は 人使[ひとづか]いが 荒[あら]い。	荒い=あらい= (勢いが強い) 
\\	(節度がない) 
\\	この馬は気性が荒い。	
\\	この 馬[うま]は 気性[きしょう]が 荒[あら]い。	気性=きしょう= 生まれつきの性質。荒い=あらい= (勢いが強い) 
\\	(節度がない) 
\\	今日は波が荒い。	
\\	今日[きょう]は 波[なみ]が 荒[あら]い。	荒い=あらい= (勢いが強い) 
\\	(節度がない) 
\\	緊張して息が荒くなってきた。	
\\	緊張[きんちょう]して 息[いき]が 荒[あら]くなってきた。	荒い=あらい= (勢いが強い) 
\\	(節度がない) 
\\	台風が接近しているので波が荒い。	
\\	台風[たいふう]が 接近[せっきん]しているので 波[なみ]が 荒[あら]い。	荒い=あらい= (勢いが強い) 
\\	(節度がない) 
\\	彼の顔にはまだ童顔が残っている。	
\\	彼[かれ]の 顔[かお]にはまだ 童顔[どうがん]が 残[のこ]っている。	童顔=どうがん= 
\\	彼は死ぬまで童貞であった。	
\\	彼[かれ]は 死[し]ぬまで 童貞[どうてい]であった。	童貞=どうてい= 
\\	それはいろいろの点から考察する必要がある。	
\\	それはいろいろの 点[てん]から 考察[こうさつ]する 必要[ひつよう]がある。	
\\	その問題はあらゆる面から考察されなければならない。	
\\	その 問題[もんだい]はあらゆる 面[めん]から 考察[こうさつ]されなければならない。	
\\	気性の激しかった父も老いてからはだいぶ穏やかになった。	
\\	気性[きしょう]の 激[はげ]しかった 父[ちち]も 老[お]いてからはだいぶ 穏[おだ]やかになった。	気性=きしょう= 生まれつきの性質穏やか=おだやか= (平穏) 
\\	(穏当) 
\\	(妥協的) 
\\	(円満な) 
\\	静岡は気候が穏やかだ。	
\\	静岡[しずおか]は 気候[きこう]が 穏[おだ]やかだ。	穏やか=おだやか= (平穏) 
\\	(穏当) 
\\	(妥協的) 
\\	(円満な) 
\\	彼は話し方が穏やかだ。	
\\	彼[かれ]は 話し方[はなしかた]が 穏[おだ]やかだ。	穏やか=おだやか= (平穏) 
\\	(穏当) 
\\	(妥協的) 
\\	(円満な) 
\\	コンピューターの故障で電話料金が預金から二重に引き落とされた。	
\\	コンピューターの 故障[こしょう]で 電話[でんわ] 料金[りょうきん]が 預金[よきん]から 二重[にじゅう]に 引き落[ひきお]とされた。	二重=にじゅう= 
\\	この運動には自然保護と資源の再利用という二重の目的がある。	
\\	この 運動[うんどう]には 自然[しぜん] 保護[ほご]と 資源[しげん]の 再[さい] 利用[りよう]という 二重[にじゅう]の 目的[もくてき]がある。	二重=にじゅう= 
\\	いずれにせよ、食事の時に君と話すよ。	
\\	いずれにせよ、 食事[しょくじ]の 時[とき]に 君[きみ]と 話[はな]すよ。	いずれにせよ= (とにかく) 
\\	いずれにせよ、それはもう終わったことだ。	
\\	いずれにせよ、それはもう 終[お]わったことだ。	いずれにせよ= (とにかく) 
\\	何度かやってみたがその都度失敗した。	
\\	何[なん] 度[ど]かやってみたがその 都度[つど] 失敗[しっぱい]した。	その都度=そのつど= 
\\	意味が分からなかったらその都度辞書を引きなさい。	
\\	意味[いみ]が 分[わ]からなかったらその 都度[つど] 辞書[じしょ]を 引[ひ]きなさい。	その都度=そのつど= 
\\	外国に行くと彼はその都度絵はがきを暮れる。	
\\	外国[がいこく]に 行[い]くと 彼[かれ]はその 都度[つど] 絵[え]はがきを 暮[く]れる。	その都度=そのつど= 
\\	彼女だってそれなりに努力はしていますよ。	
\\	彼女[かのじょ]だってそれなりに 努力[どりょく]はしていますよ。	それなり= 
\\	彼の言うことは少し過激だが、それなりに筋は通っている。	
\\	彼[かれ]の 言[い]うことは 少[すこ]し 過激[かげき]だが、それなりに 筋[すじ]は 通[かよ]っている。	それなり= 
\\	十代の子供ならともかくあの年であんなことをするなんて。	
\\	十代[じゅうだい]の 子供[こども]ならともかくあの 年[とし]であんなことをするなんて。	ともかく= (いずれにしても) 
\\	(どうであろうと) 
\\	食事はともかく、お茶を一杯どうぞ。	
\\	食事[しょくじ]はともかく、お 茶[ちゃ]を 一杯[いっぱい]どうぞ。	ともかく= (いずれにしても) 
\\	(どうであろうと) 
\\	このことは外の人にはともかく妻にだけは知られたくない。	
\\	このことは 外[そと]の 人[ひと]にはともかく 妻[つま]にだけは 知[し]られたくない。	ともかく= (いずれにしても) 
\\	(どうであろうと) 
\\	善し悪しはともかくとして、それが事実だ。	
\\	善し悪[よしあ]しはともかくとして、それが 事実[じじつ]だ。	ともかく= (いずれにしても) 
\\	(どうであろうと) 
\\	ともかくやってみよう。	
\\	ともかくやってみよう。	ともかく= (いずれにしても) 
\\	(どうであろうと) 
\\	数字はともかく、全体としてテストはまあまあだった。	
\\	数字[すうじ]はともかく、 全体[ぜんたい]としてテストはまあまあだった。	ともかく= (いずれにしても) 
\\	(どうであろうと) 
\\	うまくいかないかもしれないが、ともかく始めよう。	
\\	うまくいかないかもしれないが、ともかく 始[はじ]めよう。	ともかく= (いずれにしても) 
\\	(どうであろうと) 
\\	あのホテルは設備も料理もいいが、それにもまして素晴らしいのはあの眺望だ。	
\\	あのホテルは 設備[せつび]も 料理[りょうり]もいいが、それにもまして 素晴[すば]らしいのはあの 眺望[ちょうぼう]だ。	増して・況して=まして= (言うまでもなく) 
\\	(なおさら・・・でない) 
\\	(一層) 
\\	歩くだけで精一杯なのに、まして走るなんてとんでもない。	
\\	歩[ある]くだけで 精一杯[せいいっぱい]なのに、まして 走[はし]るなんてとんでもない。	増して・況して=まして= (言うまでもなく) 
\\	(なおさら・・・でない) 
\\	(一層) 
\\	家賃も払えないのに、まして旅行なんか行けないよ。	
\\	家賃[やちん]も 払[はら]えないのに、まして 旅行[りょこう]なんか 行[い]けないよ。	増して・況して=まして= (言うまでもなく) 
\\	(なおさら・・・でない) 
\\	(一層) 
\\	当局はその危険を百も承知している。	
\\	当局[とうきょく]はその 危険[きけん]を 百[ひゃく]も 承知[しょうち]している。	承知=しょうち= (知っていること) 
\\	(承諾) 
\\	(承認) 
\\	(容認) 
\\	(容赦)
\\	君もくどいね。そんなことは百も承知だよ。	
\\	君[きみ]もくどいね。そんなことは 百[ひゃく]も 承知[しょうち]だよ。	くどい= 
\\	承知=しょうち= (知っていること) 
\\	(承諾) 
\\	(承認) 
\\	(容認) 
\\	(容赦)
\\	返品はできませんのであらかじめご承知おきください。	
\\	返品[へんぴん]はできませんのであらかじめご 承知[しょうち]おきください。	承知=しょうち= (知っていること) 
\\	(承諾) 
\\	(承認) 
\\	(容認) 
\\	(容赦)
\\	彼はなかなか承知しないだろう。	
\\	彼[かれ]はなかなか 承知[しょうち]しないだろう。	承知=しょうち= (知っていること) 
\\	(承諾) 
\\	(承認) 
\\	(容認) 
\\	(容赦)
\\	ご承知のように日本は貿易立国です。	
\\	ご 承知[しょうち]のように 日本[にほん]は 貿易[ぼうえき] 立国[りっこく]です。	承知=しょうち= (知っていること) 
\\	(承諾) 
\\	(承認) 
\\	(容認) 
\\	(容赦)
\\	借金を頼んだら彼は承知してくれた。	
\\	借金[しゃっきん]を 頼[たの]んだら 彼[かれ]は 承知[しょうち]してくれた。	承知=しょうち= (知っていること) 
\\	(承諾) 
\\	(承認) 
\\	(容認) 
\\	(容赦)
\\	彼らの好みはある程度一致していた。	
\\	彼[かれ]らの 好[この]みはある 程度[ていど] 一致[いっち]していた。	
\\	君にも多少は責任がある。	
\\	君[きみ]にも 多少[たしょう]は 責任[せきにん]がある。	
\\	洪水発生時の被害についてはある程度の推測ができる。	
\\	洪水[こうずい] 発生[はっせい] 時[じ]の 被害[ひがい]についてはある 程度[ていど]の 推測[すいそく]ができる。	
\\	そんなにくどく聞くものでない。	
\\	そんなにくどく 聞[き]くものでない。	くどい= (しつこい) 
\\	(文章などが) 
\\	(濃厚な) (味が) 
\\	(色が) 
\\	この料理は味がくどいな。	
\\	この 料理[りょうり]は 味[あじ]がくどいな。	くどい= (しつこい) 
\\	(文章などが) 
\\	(濃厚な) (味が) 
\\	(色が) 
\\	同じことをくどいほど言った。	
\\	同[おな]じことをくどいほど 言[い]った。	くどい= (しつこい) 
\\	(文章などが) 
\\	(濃厚な) (味が) 
\\	(色が) 
\\	彼の説明はくどすぎた。	
\\	彼[かれ]の 説明[せつめい]はくどすぎた。	くどい= (しつこい) 
\\	(文章などが) 
\\	(濃厚な) (味が) 
\\	(色が) 
\\	くどいようだけど7時に必ずここに来てくれよ。	
\\	くどいようだけど7 時[じ]に 必[かなら]ずここに 来[き]てくれよ。	くどい= (しつこい) 
\\	(文章などが) 
\\	(濃厚な) (味が) 
\\	(色が) 
\\	その日彼女は妙によそよそしかった。	
\\	その 日[ひ] 彼女[かのじょ]は 妙[みょう]によそよそしかった。	よそよそしい= 
\\	彼女が急によそよそしくなった。どうしたんだろう。	
\\	彼女[かのじょ]が 急[きゅう]によそよそしくなった。どうしたんだろう。	よそよそしい= 
\\	このごろ彼女の態度がどこかよそよそしい。	
\\	このごろ 彼女[かのじょ]の 態度[たいど]がどこかよそよそしい。	よそよそしい= 
\\	このビール生温いよ。	
\\	このビール 生温[なまぬる]いよ。	生温い=なまぬるい= 
\\	(不徹底な) 
\\	(中途半端な); (穏やかすぎる)
\\	彼女の質問はいかにも唐突だった。	
\\	彼女[かのじょ]の 質問[しつもん]はいかにも 唐突[とうとつ]だった。	唐突な=とうとつな= (突然の) 
\\	(予期しない) 
\\	彼女の立候補は唐突な印象を受けた。	
\\	彼女[かのじょ]の 立候補[りっこうほ]は 唐突[とうとつ]な 印象[いんしょう]を 受[う]けた。	唐突な=とうとつな= (突然の) 
\\	(予期しない) 
\\	これらの猿と人間との間には多くの著しい類似点がある。	
\\	これらの 猿[さる]と 人間[にんげん]との 間[あいだ]には 多[おお]くの 著[いちじる]しい 類似[るいじ] 点[てん]がある。	類似点=るいじてん= 
\\	彼女は勉強に余念がない。	
\\	彼女[かのじょ]は 勉強[べんきょう]に 余念[よねん]がない。	余念なく=よねんなく= 
\\	彼は余念なくプレゼンテーションの準備をしている。	
\\	彼[かれ]は 余念[よねん]なくプレゼンテーションの 準備[じゅんび]をしている。	余念なく=よねんなく= 
\\	それが彼女が天才と言われるゆえんである。	
\\	それが 彼女[かのじょ]が 天才[てんさい]と 言[い]われるゆえんである。	所以=ゆえん= (理由・いわれ) 
\\	これがその名のゆえんである。	
\\	これがその 名[な]のゆえんである。	所以=ゆえん= (理由・いわれ) 
\\	彼はロンドンの爆破テロはイギリス政府の自作自演だと非難した。	
\\	彼[かれ]はロンドンの 爆破[ばくは]テロはイギリス 政府[せいふ]の 自作[じさく] 自演[じえん]だと 非難[ひなん]した。	自作自演=じさくじえん= 
\\	眼鏡無しでは2メートル先も見えない。	
\\	眼鏡[めがね] 無[な]しでは2メートル 先[さき]も 見[み]えない。	
\\	断りもなしに休まれては困るね。	
\\	断[ことわ]りもなしに 休[やす]まれては 困[こま]るね。	
\\	お互い隠し事なしにしましょう。	
\\	お 互[たが]い 隠し事[かくしごと]なしにしましょう。	
\\	これでビールがあれば文句なしだ。	
\\	これでビールがあれば 文句[もんく]なしだ。	
\\	彼は安定志向だ。	
\\	彼[かれ]は 安定[あんてい] 志向[しこう]だ。	安定=あんてい= 
\\	(均衡) 
\\	(落ち着き) 
\\	志向=しこう= 
\\	病状が安定している。	
\\	病状[びょうじょう]が 安定[あんてい]している。	安定=あんてい= 
\\	(均衡) 
\\	(落ち着き) 
\\	これは血圧を安定させる薬です。	
\\	これは 血圧[けつあつ]を 安定[あんてい]させる 薬[くすり]です。	安定=あんてい= 
\\	(均衡) 
\\	(落ち着き) 
\\	彼女には安定した実力がある。	
\\	彼女[かのじょ]には 安定[あんてい]した 実力[じつりょく]がある。	安定=あんてい= 
\\	(均衡) 
\\	(落ち着き) 
\\	その国の経済は内戦のため安定しない。	
\\	その 国[くに]の 経済[けいざい]は 内戦[ないせん]のため 安定[あんてい]しない。	安定=あんてい= 
\\	(均衡) 
\\	(落ち着き) 
\\	長期政権が政治の安定をもたらした。	
\\	長期[ちょうき] 政権[せいけん]が 政治[せいじ]の 安定[あんてい]をもたらした。	安定=あんてい= 
\\	(均衡) 
\\	(落ち着き) 
\\	彼らの生活環境は著しく改善された。	
\\	彼[かれ]らの 生活[せいかつ] 環境[かんきょう]は 著[いちじる]しく 改善[かいぜん]された。	著しい=いちじるしい= (はっきりしている) 
\\	(程度が大きい) 
\\	両者の間には著しい相違がある。	
\\	両者[りょうしゃ]の 間[あいだ]には 著[いちじる]しい 相違[そうい]がある。	著しい=いちじるしい= (はっきりしている) 
\\	(程度が大きい) 
\\	経済の好調に伴い、物価の上昇が著しい。	
\\	経済[けいざい]の 好調[こうちょう]に 伴[ともな]い、 物価[ぶっか]の 上昇[じょうしょう]が 著[いちじる]しい。	著しい=いちじるしい= (はっきりしている) 
\\	(程度が大きい) 
\\	わずかな海水温の上昇も生態系に著しい影響を及ぼす。	
\\	わずかな 海水温[かいすいおん]の 上昇[じょうしょう]も 生態[せいたい] 系[けい]に 著[いちじる]しい 影響[えいきょう]を 及[およ]ぼす。	生態系=せいたいけい= 
\\	著しい=いちじるしい= (はっきりしている) 
\\	(程度が大きい) 
\\	彼はその事件への関与を否定した。	
\\	彼[かれ]はその 事件[じけん]への 関与[かんよ]を 否定[ひてい]した。	関与=かんよ= 
\\	彼女はテロ活動への関与を疑われている。	
\\	彼女[かのじょ]はテロ 活動[かつどう]への 関与[かんよ]を 疑[うたが]われている。	関与=かんよ= 
\\	地球温暖化対策を成功させる上で発展途上国の関与は不可欠である。	
\\	地球[ちきゅう] 温暖[おんだん] 化[か] 対策[たいさく]を 成功[せいこう]させる 上[うえ]で 発展[はってん] 途上[とじょう] 国[こく]の 関与[かんよ]は 不可欠[ふかけつ]である。	関与=かんよ= 
\\	私はその件には関与していない。	
\\	私[わたし]はその 件[けん]には 関与[かんよ]していない。	関与=かんよ= 
\\	若い者はとかくそういう風に考えるものだ。	
\\	若[わか]い 者[もの]はとかくそういう 風[ふう]に 考[かんが]えるものだ。	とかく= (ややもすれば) 
\\	[(よくないことを)あれこれ] 
\\	あの人にはとかくのうわさがある。	
\\	あの 人[ひと]にはとかくのうわさがある。	とかく= (ややもすれば) 
\\	[(よくないことを)あれこれ] 
\\	私たちはややもすれば自分の欠点を忘れがちだ。	
\\	私[わたし]たちはややもすれば 自分[じぶん]の 欠点[けってん]を 忘[わす]れがちだ。	ややもすると[すれば]= 
\\	表向きには許可されていないが黙認されている。	
\\	表向[おもてむ]きには 許可[きょか]されていないが 黙認[もくにん]されている。	表向き=おもてむき= (うわべ) 
\\	(表面上) 
\\	(形式上) 
\\	彼が会社を辞めた表向きの理由は病気ということになっている。	
\\	彼[かれ]が 会社[かいしゃ]を 辞[や]めた 表向[おもてむ]きの 理由[りゆう]は 病気[びょうき]ということになっている。	表向き=おもてむき= (うわべ) 
\\	(表面上) 
\\	(形式上) 
\\	表向きの社長は彼だが、会社を動かしているのは女房のほうさ。	
\\	表向[おもてむ]きの 社長[しゃちょう]は 彼[かれ]だが、 会社[かいしゃ]を 動[うご]かしているのは 女房[にょうぼう]のほうさ。	表向き=おもてむき= (うわべ) 
\\	(表面上) 
\\	(形式上) 
\\	そりゃ表向きの話だ。実は裏がある。	
\\	そりゃ 表向[おもてむ]きの 話[はなし]だ。 実[じつ]は 裏[うら]がある。	表向き=おもてむき= (うわべ) 
\\	(表面上) 
\\	(形式上) 
\\	あの二人、表向きは仲の良い夫婦だ。	
\\	あの二 人[にん]、 表向[おもてむ]きは 仲[なか]の 良[よ]い 夫婦[ふうふ]だ。	表向き=おもてむき= (うわべ) 
\\	(表面上) 
\\	(形式上) 
\\	1週間の滞在ではこの国のうわべを見たに過ぎないよ。	
\\	1週間[いっしゅうかん]の 滞在[たいざい]ではこの 国[くに]のうわべを 見[み]たに 過[す]ぎないよ。	上辺=うわべ= (外面・外部) 
\\	(表面) 
\\	(外側) 
\\	(外観) 
\\	君は物のうわべだけしか見ないね。	
\\	君[きみ]は 物[もの]のうわべだけしか 見[み]ないね。	上辺=うわべ= (外面・外部) 
\\	(表面) 
\\	(外側) 
\\	(外観) 
\\	彼女はうわべほどたくましい女性ではない。	
\\	彼女[かのじょ]はうわべほどたくましい 女性[じょせい]ではない。	上辺=うわべ= (外面・外部) 
\\	(表面) 
\\	(外側) 
\\	(外観) 
\\	あれはうわべばかりの親切だ。	
\\	あれはうわべばかりの 親切[しんせつ]だ。	上辺=うわべ= (外面・外部) 
\\	(表面) 
\\	(外側) 
\\	(外観) 
\\	人はうわべを見ただけではわからない。	
\\	人[ひと]はうわべを 見[み]ただけではわからない。	上辺=うわべ= (外面・外部) 
\\	(表面) 
\\	(外側) 
\\	(外観) 
\\	社長はうわべ通りの派手好きな性格をしている。	
\\	社長[しゃちょう]はうわべ 通[どお]りの 派手[はで] 好[す]きな 性格[せいかく]をしている。	上辺=うわべ= (外面・外部) 
\\	(表面) 
\\	(外側) 
\\	(外観) 
\\	派手好き=はでずき=派手好み=はでごのみ= 
\\	あのひ弱だった少年はたくましい若者に成長した。	
\\	あのひ 弱[よわ]だった 少年[しょうねん]はたくましい 若者[わかもの]に 成長[せいちょう]した。	ひ弱=ひよわ= 
\\	逞しい=たくましい= (がっしりと力強い) 
\\	(活発な) 
\\	(精神が強靭な) 
\\	(力強くて屈しない) 
\\	彼は有り余るほど金を持っている。	
\\	彼[かれ]は 有り余[ありあま]るほど 金[きん]を 持[も]っている。	
\\	私は新聞をあちこち拾い読みした。	
\\	私[わたし]は 新聞[しんぶん]をあちこち 拾い読[ひろいよ]みした。	
\\	私は自分の死後、臓器の提供をしても良いと思っている。	
\\	私[わたし]は 自分[じぶん]の 死後[しご]、 臓器[ぞうき]の 提供[ていきょう]をしても 良[よ]いと 思[おも]っている。	提供=ていきょう= 
\\	私は客になりすましてライバル店の様子を探った。	
\\	私[わたし]は 客[きゃく]になりすましてライバル 店[てん]の 様子[ようす]を 探[さぐ]った。	成り済ます=なりすます= 
\\	彼女は学生になりすました。	
\\	彼女[かのじょ]は 学生[がくせい]になりすました。	成り済ます=なりすます= 
\\	ライオンは獲物にそっと忍び寄った。	
\\	ライオンは 獲物[えもの]にそっと 忍び寄[しのびよ]った。	忍び寄る=しのびよる= 
\\	老いは気づかぬうちに忍び寄ってくる。	
\\	老[お]いは 気[き]づかぬうちに 忍び寄[しのびよ]ってくる。	忍び寄る=しのびよる= 
\\	赤ん坊は母親に寄り添って眠った。	
\\	赤ん坊[あかんぼう]は 母親[ははおや]に 寄り添[よりそ]って 眠[ねむ]った。	寄り添う=よりそう= 
\\	彼女は夫の入院中、常に付き添って世話をした。	
\\	彼女[かのじょ]は 夫[おっと]の 入院[にゅういん] 中[ちゅう]、 常[つね]に 付き添[つきそ]って 世話[せわ]をした。	付き添う=つきそう= 
\\	彼は彼女に付き添ってパーティーへ行った。	
\\	彼[かれ]は 彼女[かのじょ]に 付き添[つきそ]ってパーティーへ 行[い]った。	付き添う=つきそう= 
\\	異様な静けさが町全体を支配していた。	
\\	異様[いよう]な 静[しず]けさが 町[まち] 全体[ぜんたい]を 支配[しはい]していた。	支配=しはい= (統治) 
\\	(指揮) 
\\	(管理) 
\\	(人や物事に影響し束縛すること) 
\\	独裁者による支配が終わった。	
\\	独裁[どくさい] 者[しゃ]による 支配[しはい]が 終[お]わった。	支配=しはい= (統治) 
\\	(指揮) 
\\	(管理) 
\\	(人や物事に影響し束縛すること) 
\\	天災のたび人間には自然する力などないことを思い知らされる。	
\\	天災[てんさい]のたび 人間[にんげん]には 自然[しぜん]する 力[ちから]などないことを 思い知[おもいし]らされる。	支配=しはい= (統治) 
\\	(指揮) 
\\	(管理) 
\\	(人や物事に影響し束縛すること) 
\\	軍部が政府を支配下に置いた。	
\\	軍部[ぐんぶ]が 政府[せいふ]を 支配[しはい] 下[か]に 置[お]いた。	支配=しはい= (統治) 
\\	(指揮) 
\\	(管理) 
\\	(人や物事に影響し束縛すること) 
\\	彼は支持を得ようと東奔西走した。	
\\	彼[かれ]は 支持[しじ]を 得[え]ようと 東奔西走[とうほんせいそう]した。	東奔西走=とうほんせいそう= 
\\	彼女は開店資金を集めるために東奔西走していた。	
\\	彼女[かのじょ]は 開店[かいてん] 資金[しきん]を 集[あつ]めるために 東奔西走[とうほんせいそう]していた。	東奔西走=とうほんせいそう= 
\\	もう堪忍袋の緒が切れた。	
\\	もう 堪忍袋[かんにんぶくろ]の 緒[お]が 切[き]れた。	堪忍=かんにん= (我慢) 
\\	(許すこと) 
\\	堪忍袋の緒が切れる=かんにんぶくろのおがきれる= 
\\	(事が主語) 
\\	あいつの無礼な態度は堪忍できない。	
\\	あいつの 無礼[ぶれい]な 態度[たいど]は 堪忍[かんにん]できない。	堪忍=かんにん= (我慢) 
\\	(許すこと) 
\\	もう堪忍もこれまでだ。	
\\	もう 堪忍[かんにん]もこれまでだ。	堪忍=かんにん= (我慢) 
\\	(許すこと) 
\\	それについて参考になる文献はない。	
\\	それについて 参考[さんこう]になる 文献[ぶんけん]はない。	参考=さんこう= 
\\	後日の参考のために取っておきなさい。	
\\	後日[ごじつ]の 参考[さんこう]のために 取[と]っておきなさい。	後日=ごじつ= その日より後の日。 参考=さんこう= 
\\	アドバイスありがとう。とても参考になりました。	
\\	アドバイスありがとう。とても 参考[さんこう]になりました。	参考=さんこう= 
\\	彼女の背後には有力政治家が控えている。	
\\	彼女[かのじょ]の 背後[はいご]には 有力[ゆうりょく] 政治[せいじ] 家[か]が 控[ひか]えている。	背後=はいご= 
\\	君の背後に誰かいる。	
\\	君[きみ]の 背後[はいご]に 誰[だれ]かいる。	背後=はいご= 
\\	部長はあえて社長に反対意見を述べた。	
\\	部長[ぶちょう]はあえて 社長[しゃちょう]に 反対[はんたい] 意見[いけん]を 述[の]べた。	
\\	私はあえて大掛かりな研究計画を立てたいと思う。	
\\	私[わたし]はあえて 大掛[おおが]かりな 研究[けんきゅう] 計画[けいかく]を 立[た]てたいと 思[おも]う。	大掛かり=おおがかり= 
\\	ストレスで胃が荒れた。	
\\	ストレスで 胃[い]が 荒[あ]れた。	荒れる=あれる= 
\\	(暴れる) 
\\	(荒廃する) 
\\	風邪をひいてのどが荒れた。	
\\	風邪[かぜ]をひいてのどが 荒[あ]れた。	荒れる=あれる= 
\\	(暴れる) 
\\	(荒廃する) 
\\	庭は荒れて草がぼうぼうとしている。	
\\	庭[にわ]は 荒[あ]れて 草[くさ]がぼうぼうとしている。	荒れる=あれる= 
\\	(暴れる) 
\\	(荒廃する) 
\\	海が荒れている。	
\\	海[うみ]が 荒[あ]れている。	荒れる=あれる= 
\\	(暴れる) 
\\	(荒廃する) 
\\	外は荒れているから外出はやめよう。	
\\	外[そと]は 荒[あ]れているから 外出[がいしゅつ]はやめよう。	荒れる=あれる= 
\\	(暴れる) 
\\	(荒廃する) 
\\	酒を飲んでほおが紅潮していた。	
\\	酒[さけ]を 飲[の]んでほおが 紅潮[こうちょう]していた。	紅潮=こうちょう= 
\\	恥ずかしくて彼女の顔が紅潮した。	
\\	恥[は]ずかしくて 彼女[かのじょ]の 顔[かお]が 紅潮[こうちょう]した。	紅潮=こうちょう= 
\\	彼女が正に紅一点であった。	
\\	彼女[かのじょ]が 正[まさ]に 紅一点[こういってん]であった。	紅一点=こういってん= 
\\	北海道ではすでに紅葉が始まっている。	
\\	北海道[ほっかいどう]ではすでに 紅葉[こうよう]が 始[はじ]まっている。	
\\	あの事件は細部にいたるまではっきり思い出せる。	
\\	あの 事件[じけん]は 細部[さいぶ]にいたるまではっきり 思い出[おもいだ]せる。	至る=いたる= 
\\	(ある時点になる); 
\\	長い会議の末、ようやく合意を見るに至った。	
\\	長[なが]い 会議[かいぎ]の 末[すえ]、ようやく 合意[ごうい]を 見[み]るに 至[いた]った。	至る=いたる= 
\\	(ある時点になる); 
\\	何時間もの話し合いの末、我々は結論に到達した。	
\\	何[なん] 時間[じかん]もの 話し合[はなしあ]いの 末[すえ]、 我々[われわれ]は 結論[けつろん]に 到達[とうたつ]した。	到達=とうたつ= (到着) 
\\	(達成) 
\\	間もなく津波が到達すると予測される。	
\\	間[ま]もなく 津波[つなみ]が 到達[とうたつ]すると 予測[よそく]される。	到達=とうたつ= (到着) 
\\	(達成) 
\\	駅まで10分で行くのは到底無理だ。	
\\	駅[えき]まで 10分[じゅっぷん]で 行[い]くのは 到底[とうてい] 無理[むり]だ。	到底=とうてい= 
\\	(まったく) 
\\	彼女はいつも用意が周到だ。	
\\	彼女[かのじょ]はいつも 用意[ようい]が 周到[しゅうとう]だ。	周到=しゅうとう= 
\\	彼は全快するのに長い間かかった。	
\\	彼[かれ]は 全快[ぜんかい]するのに 長[なが]い 間[ま]かかった。	全快=ぜんかい= 
\\	私はその光景を見てひどく不快に感じた。	
\\	私[わたし]はその 光景[こうけい]を 見[み]てひどく 不快[ふかい]に 感[かん]じた。	不快=ふかい= 
\\	彼に嫌疑がかかった。	
\\	彼[かれ]に 嫌疑[けんぎ]がかかった。	嫌疑=けんぎ= (疑い), 
\\	彼女は夫が受けている殺人の嫌疑を晴らそうとした。	
\\	彼女[かのじょ]は 夫[おっと]が 受[う]けている 殺人[さつじん]の 嫌疑[けんぎ]を 晴[は]らそうとした。	嫌疑=けんぎ= (疑い), 
\\	ふと彼の胸に疑念が浮かんだ。	
\\	ふと 彼[かれ]の 胸[むね]に 疑念[ぎねん]が 浮[う]かんだ。	疑念=ぎねん= 
\\	(嫌疑) 
\\	(不信用) 
\\	ふとある考えが彼の胸に浮かんだ。	
\\	ふとある 考[かんが]えが 彼[かれ]の 胸[むね]に 浮[う]かんだ。	ふと= (突然) 
\\	(偶然) 
\\	(思いがけず) 
\\	(何気なく) 
\\	ふと振り返ると見慣れた顔があった。	
\\	ふと 振り返[ふりかえ]ると 見慣[みな]れた 顔[かお]があった。	ふと= (突然) 
\\	(偶然) 
\\	(思いがけず) 
\\	(何気なく) 
\\	彼とはその時が初対面でした。	
\\	彼[かれ]とはその 時[とき]が 初対面[しょたいめん]でした。	初対面=しょたいめん= 
\\	この薬によって彼の病気は一時的に軽快した。	
\\	この 薬[くすり]によって 彼[かれ]の 病気[びょうき]は 一時[いちじ] 的[てき]に 軽快[けいかい]した。	軽快=けいかい= 
\\	彼は動作が軽快だ。	
\\	彼[かれ]は 動作[どうさ]が 軽快[けいかい]だ。	軽快=けいかい= 
\\	朝、公園のまわりを走るのは爽快だ。	
\\	朝[あさ]、 公園[こうえん]のまわりを 走[はし]るのは 爽快[そうかい]だ。	爽快=そうかい= 
\\	おかげでまったく不愉快な旅になった。	
\\	おかげでまったく 不愉快[ふゆかい]な 旅[たび]になった。	不愉快=ふゆかい= 
\\	彼女の態度は不愉快だ。	
\\	彼女[かのじょ]の 態度[たいど]は 不愉快[ふゆかい]だ。	不愉快=ふゆかい= 
\\	あいつは本当に不愉快なやつだ。	
\\	あいつは 本当[ほんとう]に 不愉快[ふゆかい]なやつだ。	不愉快=ふゆかい= 
\\	今晩は実に愉快でした。	
\\	今晩[こんばん]は 実[じつ]に 愉快[ゆかい]でした。	愉快=ゆかい= 
\\	彼女はいつもまじめな顔つきをしているけれど、実は愉快な人物だ。	
\\	彼女[かのじょ]はいつもまじめな 顔[かお]つきをしているけれど、 実[じつ]は 愉快[ゆかい]な 人物[じんぶつ]だ。	愉快=ゆかい= 
\\	連日の残業に加えて子供の看病でくたくただ。	
\\	連日[れんじつ]の 残業[ざんぎょう]に 加[くわ]えて 子供[こども]の 看病[かんびょう]でくたくただ。	加える=くわえる= 
\\	日照り続きで川が干上がった。	
\\	日照[ひで]り 続[つづ]きで 川[かわ]が 干上[ひあ]がった。	干上がる=ひあがる= (乾き切る) 
\\	(生活費が途絶える)
\\	くたばってしまえ!	
\\	くたばってしまえ!	くたばる= 
\\	(へたばる) 
\\	炎天下の練習でみんなくたばってしまった。	
\\	炎天下[えんてんか]の 練習[れんしゅう]でみんなくたばってしまった。	くたばる= 
\\	(へたばる) 
\\	試合から戻った娘はソファの上にへたばっていた。	
\\	試合[しあい]から 戻[もど]った 娘[むすめ]はソファの 上[うえ]にへたばっていた。	へたばる= (疲れ切る) 
\\	汗で体がべたべたする。	
\\	汗[あせ]で 体[からだ]がべたべたする。	
\\	蒸し暑くてシャツがべたべたする。	
\\	蒸し暑[むしあつ]くてシャツがべたべたする。	
\\	この子はキャラメルで手がべたべただ。	
\\	この 子[こ]はキャラメルで 手[て]がべたべただ。	
\\	若い男女が電車の中でべたべたしているのを見ると目のやり場に困る。	
\\	若[わか]い 男女[だんじょ]が 電車[でんしゃ]の 中[なか]でべたべたしているのを 見[み]ると 目[め]のやり 場[ば]に 困[こま]る。	
\\	洗っていないから髪がベタベタだ。	
\\	洗[あら]っていないから 髪[かみ]がベタベタだ。	
\\	全身汗でべたべただ。	
\\	全身[ぜんしん] 汗[あせ]でべたべただ。	
\\	収益の一部は文化施設などへの寄付の形で社会に還元されます。	
\\	収益[しゅうえき]の 一部[いちぶ]は 文化[ぶんか] 施設[しせつ]などへの 寄付[きふ]の 形[かたち]で 社会[しゃかい]に 還元[かんげん]されます。	還元=かんげん= (元に戻すこと) 
\\	あなたと私とは流儀が違う。	
\\	あなたと 私[わたし]とは 流儀[りゅうぎ]が 違[ちが]う。	流儀=りゅうぎ= (やり方) 
\\	(流派) 
\\	人にはそれぞれ流儀があるものだ。	
\\	人[ひと]にはそれぞれ 流儀[りゅうぎ]があるものだ。	流儀=りゅうぎ= (やり方) 
\\	(流派) 
\\	これが私の流儀だ。	
\\	これが 私[わたし]の 流儀[りゅうぎ]だ。	流儀=りゅうぎ= (やり方) 
\\	(流派) 
\\	この会議は単なる儀式にすぎない。	
\\	この 会議[かいぎ]は 単[たん]なる 儀式[ぎしき]にすぎない。	儀式=ぎしき= (式典) 
\\	(礼法) 
\\	(宗教上の) 
\\	リストラで社員の4割が解雇された。	
\\	リストラで 社員[しゃいん]の4 割[わり]が 解雇[かいこ]された。	解雇=かいこ= 
\\	彼は予告なしに解雇された。	
\\	彼[かれ]は 予告[よこく]なしに 解雇[かいこ]された。	解雇=かいこ= 
\\	会社は職務怠慢を理由に彼を解雇した。	
\\	会社[かいしゃ]は 職務[しょくむ] 怠慢[たいまん]を 理由[りゆう]に 彼[かれ]を 解雇[かいこ]した。	解雇=かいこ= 
\\	こういう傷は炎症を起こしやすい。	
\\	こういう 傷[きず]は 炎症[えんしょう]を 起[お]こしやすい。	炎症=えんしょう= 
\\	正月はハワイで過ごすのがわが家の慣例になっている。	
\\	正月[しょうがつ]はハワイで 過[す]ごすのがわが 家[や]の 慣例[かんれい]になっている。	慣例=かんれい= 
\\	(先例) 
\\	正月には子供たちにお年玉をやるのが慣例である。	
\\	正月[しょうがつ]には 子供[こども]たちにお 年玉[としだま]をやるのが 慣例[かんれい]である。	慣例=かんれい= 
\\	(先例) 
\\	それは慣用上認められている言い方だ。	
\\	それは 慣用[かんよう] 上[じょう] 認[みと]められている 言い方[いいかた]だ。	慣用=かんよう= 
\\	日本人は概して酔っぱらいに寛容だ。	
\\	日本人[にほんじん]は 概[がい]して 酔[よ]っぱらいに 寛容[かんよう]だ。	概して=がいして= 
\\	寛容=かんよう= 
\\	概して日本人は外国語を話すのが苦手である。	
\\	概[がい]して 日本人[にほんじん]は 外国[がいこく] 語[ご]を 話[はな]すのが 苦手[にがて]である。	概して=がいして= 
\\	ここでは物価は概して高い。	
\\	ここでは 物価[ぶっか]は 概[がい]して 高[たか]い。	概して=がいして= 
\\	何事も断続することが肝要だ。	
\\	何事[なにごと]も 断続[だんぞく]することが 肝要[かんよう]だ。	肝要な=かんような= (重要な) 
\\	(不可欠の) 
\\	最後まであきらめないことが肝要だ。	
\\	最後[さいご]まであきらめないことが 肝要[かんよう]だ。	肝要な=かんような= (重要な) 
\\	(不可欠の) 
\\	この品種の馬をならすことは難しい。	
\\	この 品種[ひんしゅ]の 馬[うま]をならすことは 難[むずか]しい。	慣らす・馴らす=ならす= (慣らす) (違和感をなくす) 
\\	(使用のための準備をする) 
\\	(馴らす) 
\\	この熊はよくならしてある。	
\\	この 熊[くま]はよくならしてある。	慣らす・馴らす=ならす= (慣らす) (違和感をなくす) 
\\	(使用のための準備をする) 
\\	(馴らす) 
\\	妻の料理の味にすっかり慣らされてしまった。	
\\	妻[つま]の 料理[りょうり]の 味[あじ]にすっかり 慣[な]らされてしまった。	慣らす・馴らす=ならす= (慣らす) (違和感をなくす) 
\\	(使用のための準備をする) 
\\	(馴らす) 
\\	この程度の騒音には慣らされている。	
\\	この 程度[ていど]の 騒音[そうおん]には 慣[な]らされている。	慣らす・馴らす=ならす= (慣らす) (違和感をなくす) 
\\	(使用のための準備をする) 
\\	(馴らす) 
\\	彼は来年還暦を迎える。	
\\	彼[かれ]は 来年[らいねん] 還暦[かんれき]を 迎[むか]える。	
\\	この事故は主催者側の怠慢から起こったと言える。	
\\	この 事故[じこ]は 主催[しゅさい] 者[しゃ] 側[がわ]の 怠慢[たいまん]から 起[お]こったと 言[い]える。	怠慢な=たいまんな= 
\\	(不注意な) 
\\	不審な文章にはしるしを付けておきなさい。	
\\	不審[ふしん]な 文章[ぶんしょう]にはしるしを 付[つ]けておきなさい。	不審な=ふしんな= (疑わしい)
\\	(いぶかしい) 
\\	(不思議な) 
\\	不審な車が犯行現場から去っていったのが見られた。	
\\	不審[ふしん]な 車[くるま]が 犯行[はんこう] 現場[げんば]から 去[さ]っていったのが 見[み]られた。	不審な=ふしんな= (疑わしい)
\\	(いぶかしい) 
\\	(不思議な) 
\\	何かご不審がありますか。	
\\	何[なに]かご 不審[ふしん]がありますか。	不審な=ふしんな= (疑わしい)
\\	(いぶかしい) 
\\	(不思議な) 
\\	そのご説明で不審が晴れました。	
\\	そのご 説明[せつめい]で 不審[ふしん]が 晴[は]れました。	不審な=ふしんな= (疑わしい)
\\	(いぶかしい) 
\\	(不思議な) 
\\	不審な人物や荷物を見掛けたらただちに係の者までお知らせ下さい。	
\\	不審[ふしん]な 人物[じんぶつ]や 荷物[にもつ]を 見掛[みか]けたらただちに 係[かかり]の 者[もの]までお 知[し]らせ 下[くだ]さい。	不審な=ふしんな= (疑わしい)
\\	(いぶかしい) 
\\	(不思議な) 
\\	当方の手落ちで大変ご迷惑をおかけし、申し訳ございません。	
\\	当方[とうほう]の 手落[てお]ちで 大変[たいへん]ご 迷惑[めいわく]をおかけし、 申し訳[もうしわけ]ございません。	手落ち=ておち= 
\\	それは私の手落ちだ。	
\\	それは 私[わたし]の 手落[てお]ちだ。	手落ち=ておち= 
\\	この職種は不況とは無縁だ。	
\\	この 職種[しょくしゅ]は 不況[ふきょう]とは 無縁[むえん]だ。	無縁の=むえんの= (無関係の) 
\\	(縁者のない) 
\\	それまで凶暴犯罪とは無縁だった静かな村で殺人事件が起こった。	
\\	それまで 凶暴[きょうぼう] 犯罪[はんざい]とは 無縁[むえん]だった 静[しず]かな 村[むら]で 殺人[さつじん] 事件[じけん]が 起[お]こった。	無縁の=むえんの= (無関係の) 
\\	(縁者のない) 
\\	これは縁起のために持って歩いているのだ。	
\\	これは 縁起[えんぎ]のために 持[も]って 歩[ある]いているのだ。	縁起=えんぎ= (さいさき) 
\\	(起源) 
\\	(由来) 
\\	靴ひもが切れた。縁起が悪いな。	
\\	靴[くつ]ひもが 切[き]れた。 縁起[えんぎ]が 悪[わる]いな。	縁起=えんぎ= (さいさき) 
\\	(起源) 
\\	(由来) 
\\	あのコートは縁起が良くないので捨てたよ。	
\\	あのコートは 縁起[えんぎ]が 良[よ]くないので 捨[す]てたよ。	縁起=えんぎ= (さいさき) 
\\	(起源) 
\\	(由来) 
\\	英国では黒猫は縁起が良いと言われている。	
\\	英国[えいこく]では 黒[くろ] 猫[ねこ]は 縁起[えんぎ]が 良[よ]いと 言[い]われている。	縁起=えんぎ= (さいさき) 
\\	(起源) 
\\	(由来) 
\\	彼女は泣いたため目の縁が赤い。	
\\	彼女[かのじょ]は 泣[な]いたため 目[め]の 縁[ふち]が 赤[あか]い。	縁=ふち= (辺・端) 
\\	(川の) 
\\	(茶碗・眼鏡などの) 
\\	カップの縁に口紅がついてしまった。	
\\	カップの 縁[ふち]に 口紅[くちべに]がついてしまった。	縁=ふち= (辺・端) 
\\	(川の) 
\\	(茶碗・眼鏡などの) 
\\	グラスの縁まで水を入れた。	
\\	グラスの 縁[ふち]まで 水[みず]を 入[い]れた。	縁=ふち= (辺・端) 
\\	(川の) 
\\	(茶碗・眼鏡などの) 
\\	彼の言っていることに対して心に疑惑の念が生じた。	
\\	彼[かれ]の 言[い]っていることに 対[たい]して 心[こころ]に 疑惑[ぎわく]の 念[ねん]が 生[しょう]じた。	疑惑=ぎわく= 
\\	その力士に八百長疑惑が浮上した。	
\\	その 力士[りきし]に 八百長[やおちょう] 疑惑[ぎわく]が 浮上[ふじょう]した。	疑惑=ぎわく= 
\\	彼はあの女の魅惑に打ち勝てなかったのだ。	
\\	彼[かれ]はあの 女[おんな]の 魅惑[みわく]に 打ち勝[うちか]てなかったのだ。	魅惑=みわく= 
\\	打ち勝つ=うちかつ= (勝利する) 
\\	(克服する) 
\\	長い戦いの末、敵の大軍に打ち勝った。	
\\	長[なが]い 戦[たたか]いの 末[すえ]、 敵[てき]の 大軍[たいぐん]に 打ち勝[うちか]った。	打ち勝つ=うちかつ= (勝利する) 
\\	(克服する) 
\\	彼女は強い精神力でとうとう病気に打ち勝った。	
\\	彼女[かのじょ]は 強[つよ]い 精神[せいしん] 力[りょく]でとうとう 病気[びょうき]に 打ち勝[うちか]った。	打ち勝つ=うちかつ= (勝利する) 
\\	(克服する) 
\\	人間は自然の猛威に打ち勝つことはできない。	
\\	人間[にんげん]は 自然[しぜん]の 猛威[もうい]に 打ち勝[うちか]つことはできない。	打ち勝つ=うちかつ= (勝利する) 
\\	(克服する) 
\\	どうしたらいいのか当惑してしまった。	
\\	どうしたらいいのか 当惑[とうわく]してしまった。	当惑=とうわく= (困惑) 
\\	(混乱) 
\\	予定が突然変更されて彼らは当惑気味だった。	
\\	予定[よてい]が 突然[とつぜん] 変更[へんこう]されて 彼[かれ]らは 当惑[とうわく] 気味[ぎみ]だった。	当惑=とうわく= (困惑) 
\\	(混乱) 
\\	彼女は当惑顔で質問に答えた。	
\\	彼女[かのじょ]は 当惑[とうわく] 顔[がお]で 質問[しつもん]に 答[こた]えた。	当惑=とうわく= (困惑) 
\\	(混乱) 
\\	そのイベントは思わくどおり大成功を収めた。	
\\	そのイベントは 思[おも]わくどおり 大[だい] 成功[せいこう]を 収[おさ]めた。	思惑=おもわく= (予想) 
\\	(ねらい) 
\\	その決定はわれわれの思わくから外れたものだった。	
\\	その 決定[けってい]はわれわれの 思[おも]わくから 外[はず]れたものだった。	思惑=おもわく= (予想) 
\\	(ねらい) 
\\	彼は他人の思わくを気にせず思うままにやっている。	
\\	彼[かれ]は 他人[たにん]の 思[おも]わくを 気[き]にせず 思[おも]うままにやっている。	思惑=おもわく= (予想) 
\\	(ねらい) 
\\	戦争は国の財源を枯渇させる。	
\\	戦争[せんそう]は 国[くに]の 財源[ざいげん]を 枯渇[こかつ]させる。	枯渇=こかつ= (水の) 
\\	(消尽・欠乏) 
\\	彼はすでにアイデアが枯渇している。	
\\	彼[かれ]はすでにアイデアが 枯渇[こかつ]している。	枯渇=こかつ= (水の) 
\\	(消尽・欠乏) 
\\	天然資源の枯渇が心配だ。	
\\	天然[てんねん] 資源[しげん]の 枯渇[こかつ]が 心配[しんぱい]だ。	枯渇=こかつ= (水の) 
\\	(消尽・欠乏) 
\\	日照り続きで水源が枯渇した。	
\\	日照[ひで]り 続[つづ]きで 水源[すいげん]が 枯渇[こかつ]した。	枯渇=こかつ= (水の) 
\\	(消尽・欠乏) 
\\	地球の天然資源は急速に枯渇しつつある。	
\\	地球[ちきゅう]の 天然[てんねん] 資源[しげん]は 急速[きゅうそく]に 枯渇[こかつ]しつつある。	枯渇=こかつ= (水の) 
\\	(消尽・欠乏) 
\\	想像力が枯渇してしまった。	
\\	想像[そうぞう] 力[りょく]が 枯渇[こかつ]してしまった。	枯渇=こかつ= (水の) 
\\	(消尽・欠乏) 
\\	ミネラルが欠乏するとこのような症状が出る。	
\\	ミネラルが 欠乏[けつぼう]するとこのような 症状[しょうじょう]が 出[で]る。	欠乏=けつぼう= 
\\	(不足) 
\\	被災地では医薬品が欠乏している。	
\\	被災[ひさい] 地[ち]では 医薬品[いやくひん]が 欠乏[けつぼう]している。	欠乏=けつぼう= 
\\	(不足) 
\\	汗は蒸発するときに体の熱を奪う。	
\\	汗[あせ]は 蒸発[じょうはつ]するときに 体[からだ]の 熱[ねつ]を 奪[うば]う。	蒸発=じょうはつ= (液体の) 
\\	(人が姿を消すこと) 
\\	熱は水を蒸発させる。	
\\	熱[ねつ]は 水[みず]を 蒸発[じょうはつ]させる。	蒸発=じょうはつ= (液体の) 
\\	(人が姿を消すこと) 
\\	彼は2年前に蒸発した。	
\\	彼[かれ]は2 年[ねん] 前[まえ]に 蒸発[じょうはつ]した。	蒸発=じょうはつ= (液体の) 
\\	(人が姿を消すこと) 
\\	あの人は折衝がうまい。	
\\	あの 人[ひと]は 折衝[せっしょう]がうまい。	折衝=せっしょう= 
\\	折衝を重ねること3ヶ月に及んだが解決しない。	
\\	折衝[せっしょう]を 重[かさ]ねること3 ヶ月[かげつ]に 及[およ]んだが 解決[かいけつ]しない。	折衝=せっしょう= 
\\	秘密裏に折衝が行われた。	
\\	秘密[ひみつ] 裏[り]に 折衝[せっしょう]が 行[おこな]われた。	秘密裏=ひみつり= 物事が秘密の状態で行われること。折衝=せっしょう= 
\\	折衝を重ねてようやく合意に達した。	
\\	折衝[せっしょう]を 重[かさ]ねてようやく 合意[ごうい]に 達[たっ]した。	折衝=せっしょう= 
\\	その頃のことは漠然としていてあまり思い出せない。	
\\	その 頃[ころ]のことは 漠然[ばくぜん]としていてあまり 思い出[おもいだ]せない。	漠然と=ばくぜんと= 
\\	何か仕事が見つかるだろうという漠然たる考えで上京した。	
\\	何[なに]か 仕事[しごと]が 見[み]つかるだろうという 漠然[ばくぜん]たる 考[かんが]えで 上京[じょうきょう]した。	漠然と=ばくぜんと= 
\\	海外で働きたいと漠然と考えている。	
\\	海外[かいがい]で 働[はたら]きたいと 漠然[ばくぜん]と 考[かんが]えている。	漠然と=ばくぜんと= 
\\	その事故については漠然とした記憶しかない。	
\\	その 事故[じこ]については 漠然[ばくぜん]とした 記憶[きおく]しかない。	漠然と=ばくぜんと= 
\\	彼女の考えに共感した人々が集まった。	
\\	彼女[かのじょ]の 考[かんが]えに 共感[きょうかん]した 人々[ひとびと]が 集[あつ]まった。	共感=きょうかん= 
\\	彼の人生観に共感を覚える。	
\\	彼[かれ]の 人生[じんせい] 観[かん]に 共感[きょうかん]を 覚[おぼ]える。	共感=きょうかん= 
\\	彼女はその映画のヒロインに共感した。	
\\	彼女[かのじょ]はその 映画[えいが]のヒロインに 共感[きょうかん]した。	共感=きょうかん= 
\\	その会合は非常に有意義だった。	
\\	その 会合[かいごう]は 非常[ひじょう]に 有意義[ゆういぎ]だった。	有意義な=ゆういぎな= (意味のある) 
\\	(有益な) 
\\	(価値のある) 
\\	限りある資源はもっと有意義に使うべきだ。	
\\	限[かぎ]りある 資源[しげん]はもっと 有意義[ゆういぎ]に 使[つか]うべきだ。	有意義な=ゆういぎな= (意味のある) 
\\	(有益な) 
\\	(価値のある) 
\\	彼とは絶縁した。	
\\	彼[かれ]とは 絶縁[ぜつえん]した。	絶縁=ぜつえん= 
\\	『電』
\\	両親とは絶縁状態です。	
\\	両親[りょうしん]とは 絶縁[ぜつえん] 状態[じょうたい]です。	絶縁=ぜつえん= 
\\	『電』
\\	彼女は看護師だから病院一般の内幕に通じている。	
\\	彼女[かのじょ]は 看護[かんご] 師[し]だから 病院[びょういん] 一般[いっぱん]の 内幕[うちまく]に 通[つう]じている。	"内幕=うちまく= 
\\	あの人が政界の黒幕だ。	
\\	あの 人[ひと]が 政界[せいかい]の 黒幕[くろまく]だ。	黒幕=くろまく= 
\\	これには誰か黒幕がいるに違いない。	
\\	これには 誰[だれ]か 黒幕[くろまく]がいるに 違[ちが]いない。	黒幕=くろまく= 
\\	ここは親の出る幕ではない。子供たちに任せよう。	
\\	ここは 親[おや]の 出[で]る 幕[まく]ではない。 子供[こども]たちに 任[まか]せよう。	幕=まく= (舞台の) 
\\	(芝居の一区切り) 
\\	・・・の出る幕ではない= 
\\	君なんかの出る幕じゃない。	
\\	君[きみ]なんかの 出[で]る 幕[まく]じゃない。	幕=まく= (舞台の) 
\\	(芝居の一区切り) 
\\	・・・の出る幕ではない= 
\\	この発見が、後に続く宇宙開発競争の幕を切って落とすことになった。	
\\	この 発見[はっけん]が、 後[ご]に 続[つづ]く 宇宙[うちゅう] 開発[かいはつ] 競争[きょうそう]の 幕[まく]を 切[き]って 落[お]とすことになった。	幕=まく= (舞台の) 
\\	(芝居の一区切り) 
\\	・・・の幕を切って落とす= 
\\	ついに高速通信時代の幕が切って落とされた。	
\\	ついに 高速[こうそく] 通信[つうしん] 時代[じだい]の 幕[まく]が 切[き]って 落[お]とされた。	幕=まく= (舞台の) 
\\	(芝居の一区切り) 
\\	・・・の幕を切って落とす= 
\\	社長の引責辞任で事件はあっけなく幕が引かれた。	
\\	社長[しゃちょう]の 引責[いんせき] 辞任[じにん]で 事件[じけん]はあっけなく 幕[まく]が 引[ひ]かれた。	引責辞任=いんせきじにん= 責任を自分の身に引き受けて、就いていた任務を自らやめること。幕=まく= (舞台の) 
\\	(芝居の一区切り) 
\\	幕が下りても拍手は鳴りやまなかった。	
\\	幕[まく]が 下[お]りても 拍手[はくしゅ]は 鳴[な]りやまなかった。	幕=まく= (舞台の) 
\\	(芝居の一区切り) 
\\	幕が上がってオペラが始まった。	
\\	幕[まく]が 上[あ]がってオペラが 始[はじ]まった。	幕=まく= (舞台の) 
\\	(芝居の一区切り) 
\\	息子に縁談が持ち上がっている。	
\\	息子[むすこ]に 縁談[えんだん]が 持ち上[もちあ]がっている。	縁談=えんだん= 
\\	雪子はその青年との縁談を断った。	
\\	雪子[ゆきこ]はその 青年[せいねん]との 縁談[えんだん]を 断[ことわ]った。	縁談=えんだん= 
\\	娘の縁談がまとまった。	
\\	娘[むすめ]の 縁談[えんだん]がまとまった。	縁談=えんだん= 
\\	ここで引っ込んでは幕切れが悪い。	
\\	ここで 引っ込[ひっこ]んでは 幕切[まくぎ]れが 悪[わる]い。	幕切れ=まくぎれ= 
\\	(終わり) 
\\	事件はあっけない幕切れとなった。	
\\	事件[じけん]はあっけない 幕切[まくぎ]れとなった。	幕切れ=まくぎれ= 
\\	(終わり) 
\\	物語はいよいよ終幕を迎えた。	
\\	物語[ものがたり]はいよいよ 終幕[しゅうまく]を 迎[むか]えた。	終幕=しゅうまく= 
\\	(芝居の) 
\\	この大きな箱は書き物机の代用になる。	
\\	この 大[おお]きな 箱[はこ]は 書き物[かきもの] 机[つくえ]の 代用[だいよう]になる。	書き物机=かきものづくえ= 
\\	代用=だいよう= 
\\	〜する 
\\	この座布団は枕の代用にもなる。	
\\	この 座布団[ざぶとん]は 枕[まくら]の 代用[だいよう]にもなる。	枕=まくら= 寝るときに頭を載せる寝具。 代用=だいよう= 
\\	〜する 
\\	バターがない時はマーガリンで代用することもできます。	
\\	バターがない 時[とき]はマーガリンで 代用[だいよう]することもできます。	代用=だいよう= 
\\	〜する 
\\	彼のわがままは許容範囲を超えている。	
\\	彼[かれ]のわがままは 許容[きょよう] 範囲[はんい]を 超[こ]えている。	許容=きょよう= (認容) 
\\	(容赦) 
\\	土壌からから許容量を超える水銀が検出された。	
\\	土壌[どじょう]からから 許容[きょよう] 量[りょう]を 超[こ]える 水銀[すいぎん]が 検出[けんしゅつ]された。	許容=きょよう= (認容) 
\\	(容赦) 
\\	ご病人の経過は順調で、もう安心です。	
\\	ご 病人[びょうにん]の 経過[けいか]は 順調[じゅんちょう]で、もう 安心[あんしん]です。	経過=けいか= (時間・月日の) 
\\	(事態の) 
\\	この患者の術後の経過は良好だ。	
\\	この 患者[かんじゃ]の 術後[じゅつご]の 経過[けいか]は 良好[りょうこう]だ。	経過=けいか= (時間・月日の) 
\\	(事態の) 
\\	経過を見て今後の処置を決めましょう。	
\\	経過[けいか]を 見[み]て 今後[こんご]の 処置[しょち]を 決[き]めましょう。	経過=けいか= (時間・月日の) 
\\	(事態の) 
\\	遺体は死後一周間経過している。	
\\	遺体[いたい]は 死後[しご] 一周[いっしゅう] 間[かん] 経過[けいか]している。	経過=けいか= (時間・月日の) 
\\	(事態の) 
\\	事件から3ヶ月が経過した。	
\\	事件[じけん]から3 ヶ月[かげつ]が 経過[けいか]した。	経過=けいか= (時間・月日の) 
\\	(事態の) 
\\	高校を卒業して5年の歳月が経過した。	
\\	高校[こうこう]を 卒業[そつぎょう]して5 年[ねん]の 歳月[さいげつ]が 経過[けいか]した。	経過=けいか= (時間・月日の) 
\\	(事態の) 
\\	デンマークはヨーロッパ有数のチーズ生産国だ。	
\\	デンマークはヨーロッパ 有数[ゆうすう]のチーズ 生産[せいさん] 国[こく]だ。	有数=ゆうすう= 
\\	清水は日本有数の漁港だ。	
\\	清水[しみず]は 日本[にっぽん] 有数[ゆうすう]の 漁港[ぎょこう]だ。	有数=ゆうすう= 
\\	開封後はなるべく早くお召し上がり下さい。	
\\	開封[かいふう] 後[ご]はなるべく 早[はや]くお 召し上[めしあ]がり 下[くだ]さい。	開封=かいふう= (封を開くこと) 
\\	小包は開封すると爆発する仕組みになっている。	
\\	小包[こづつみ]は 開封[かいふう]すると 爆発[ばくはつ]する 仕組[しく]みになっている。	開封=かいふう= (封を開くこと) 
\\	ここは厳重な封鎖線が布かれている。	
\\	ここは 厳重[げんじゅう]な 封鎖[ふうさ] 線[せん]が 布[し]かれている。	封鎖=ふうさ= 
\\	(凍結) 
\\	布く=しく
\\	パレードの間、警察は道路を封鎖した。	
\\	パレードの 間[あいだ]、 警察[けいさつ]は 道路[どうろ]を 封鎖[ふうさ]した。	封鎖=ふうさ= 
\\	(凍結) 
\\	両国間の摩擦は避けられない情勢だ。	
\\	両国[りょうこく] 間[かん]の 摩擦[まさつ]は 避[さ]けられない 情勢[じょうせい]だ。	摩擦=まさつ= (こすること) 
\\	(あつれき) 
\\	習慣の違いが文化的摩擦を生むことがある。	
\\	習慣[しゅうかん]の 違[ちが]いが 文化[ぶんか] 的[てき] 摩擦[まさつ]を 生[う]むことがある。	摩擦=まさつ= (こすること) 
\\	(あつれき) 
\\	この地域は民族間の摩擦により紛争が多発している。	
\\	この 地域[ちいき]は 民族[みんぞく] 間[かん]の 摩擦[まさつ]により 紛争[ふんそう]が 多発[たはつ]している。	摩擦=まさつ= (こすること) 
\\	(あつれき) 
\\	両派の間には絶えず摩擦がある。	
\\	両派[りょうは]の 間[あいだ]には 絶[た]えず 摩擦[まさつ]がある。	摩擦=まさつ= (こすること) 
\\	(あつれき) 
\\	摩擦によって熱が起きる。	
\\	摩擦[まさつ]によって 熱[ねつ]が 起[お]きる。	摩擦=まさつ= (こすること) 
\\	(あつれき) 
\\	もし摩擦がなかったら、バイオリンも鳴らず、車も進まない。	
\\	もし 摩擦[まさつ]がなかったら、バイオリンも 鳴[な]らず、 車[くるま]も 進[すす]まない。	摩擦=まさつ= (こすること) 
\\	(あつれき) 
\\	努力なしには何事も成し遂げられない。	
\\	努力[どりょく]なしには 何事[なにごと]も 成し遂[なしと]げられない。	成し遂げる=なしとげる= 
\\	(成功する) 
\\	私たちはピアノとバイオリンで合奏した。	
\\	私[わたし]たちはピアノとバイオリンで 合奏[がっそう]した。	合奏=がっそう= 
\\	(〜する) 
\\	彼女は広範な支持を得て当選した。	
\\	彼女[かのじょ]は 広範[こうはん]な 支持[しじ]を 得[え]て 当選[とうせん]した。	広範な=こうはんな= 
\\	不況の広範な影響が懸念される。	
\\	不況[ふきょう]の 広範[こうはん]な 影響[えいきょう]が 懸念[けねん]される。	広範な=こうはんな= 
\\	彼は高い道徳的規範を守っている。	
\\	彼[かれ]は 高[たか]い 道徳[どうとく] 的[てき] 規範[きはん]を 守[まも]っている。	規範=きはん= (基準) 
\\	病床では死を見詰めた日々だった。	
\\	病床[びょうしょう]では 死[し]を 見詰[みつ]めた 日々[ひび]だった。	見詰める=みつめる= (目前を) 
\\	(過去や未来を) 
\\	(自己を) 
\\	彼女は目を丸くしてその光景を見つめた。	
\\	彼女[かのじょ]は 目[め]を 丸[まる]くしてその 光景[こうけい]を 見[み]つめた。	見詰める=みつめる= (目前を) 
\\	(過去や未来を) 
\\	(自己を) 
\\	彼女は鏡に映る自分の姿を見つめた。	
\\	彼女[かのじょ]は 鏡[かがみ]に 映[うつ]る 自分[じぶん]の 姿[すがた]を 見[み]つめた。	見詰める=みつめる= (目前を) 
\\	(過去や未来を) 
\\	(自己を) 
\\	彼女は彼に見つめられて赤くなった。	
\\	彼女[かのじょ]は 彼[かれ]に 見[み]つめられて 赤[あか]くなった。	見詰める=みつめる= (目前を) 
\\	(過去や未来を) 
\\	(自己を) 
\\	彼は静かに水平線を見つめていた。	
\\	彼[かれ]は 静[しず]かに 水平[すいへい] 線[せん]を 見[み]つめていた。	見詰める=みつめる= (目前を) 
\\	(過去や未来を) 
\\	(自己を) 
\\	暴風雪がやまず、現在山小屋に缶詰状態だ。	
\\	暴風[ぼうふう] 雪[ゆき]がやまず、 現在[げんざい] 山小屋[やまごや]に 缶詰[かんづめ] 状態[じょうたい]だ。	暴風雪=ぼうふうせつ= 激しい風を伴った降雪 缶詰=かんづめ= (食品などを缶に詰めたもの) 
\\	(人を隔離状態にして閉じ込めること) 
\\	事故で立ち往生した電車の中に8時間缶詰になった。	
\\	事故[じこ]で 立ち往生[たちおうじょう]した 電車[でんしゃ]の 中[なか]に8 時間[じかん] 缶詰[かんづめ]になった。	缶詰=かんづめ= (食品などを缶に詰めたもの) 
\\	(人を隔離状態にして閉じ込めること) 
\\	彼は演説の途中で言葉を忘れて立ち往生した。	
\\	彼[かれ]は 演説[えんぜつ]の 途中[とちゅう]で 言葉[ことば]を 忘[わす]れて 立ち往生[たちおうじょう]した。	立ち往生=たちおうじょう= (立ったままで死ぬこと) 
\\	(動けなくなるさま) 
\\	(当惑し何もできなくなるさま) 
\\	資金不足で計画は立ち往生した。	
\\	資金[しきん] 不足[ふそく]で 計画[けいかく]は 立ち往生[たちおうじょう]した。	立ち往生=たちおうじょう= (立ったままで死ぬこと) 
\\	(動けなくなるさま) 
\\	(当惑し何もできなくなるさま) 
\\	車が泥にはまって立ち往生した。	
\\	車[くるま]が 泥[どろ]にはまって 立ち往生[たちおうじょう]した。	立ち往生=たちおうじょう= (立ったままで死ぬこと) 
\\	(動けなくなるさま) 
\\	(当惑し何もできなくなるさま) 
\\	大臣は野党議員の質問に立ち往生した。	
\\	大臣[だいじん]は 野党[やとう] 議員[ぎいん]の 質問[しつもん]に 立ち往生[たちおうじょう]した。	立ち往生=たちおうじょう= (立ったままで死ぬこと) 
\\	(動けなくなるさま) 
\\	(当惑し何もできなくなるさま) 
\\	大詰めであっと驚くどんでん返しが待っていた。	
\\	大詰[おおづ]めであっと 驚[おどろ]くどんでん 返[がえ]しが 待[ま]っていた。	大詰め=おおづめ= (物事の最終階段) 
\\	(芝居の最終の幕) 
\\	どんでん返し=どんでんがえし= (芝居の) 
\\	(逆転) 
\\	その法案の国会審議が大詰めを迎えた。	
\\	その 法案[ほうあん]の 国会[こっかい] 審議[しんぎ]が 大詰[おおづ]めを 迎[むか]えた。	大詰め=おおづめ= (物事の最終階段) 
\\	(芝居の最終の幕) 
\\	トーナメントは大詰めを迎えた。	
\\	トーナメントは 大詰[おおづ]めを 迎[むか]えた。	大詰め=おおづめ= (物事の最終階段) 
\\	(芝居の最終の幕) 
\\	両国は目下敵対関係にある。	
\\	両国[りょうこく]は 目下[もっか] 敵対[てきたい] 関係[かんけい]にある。	敵対=てきたい= (〜する) 
\\	あえて彼女に適する者はいなかった。	
\\	あえて 彼女[かのじょ]に 適[てき]する 者[もの]はいなかった。	適する=てきする= 
\\	君は彼女にはとうてい適し得まい。	
\\	君[きみ]は 彼女[かのじょ]にはとうてい 適[てき]し 得[え]まい。	適する=てきする= 
\\	彼女の言い方には私への敵意が感じられる。	
\\	彼女[かのじょ]の 言い方[いいかた]には 私[わたし]への 敵意[てきい]が 感[かん]じられる。	
\\	彼女に対する思いは年を経るにしたがってつのった。	
\\	彼女[かのじょ]に 対[たい]する 思[おも]いは 年[とし]を 経[へ]るにしたがってつのった。	募る=つのる= 
\\	英語スピーチコンテストの参加者を募っています。	
\\	英語[えいご]スピーチコンテストの 参加[さんか] 者[しゃ]を 募[つの]っています。	募る=つのる= 
\\	時が経つにつれて望郷の念が募った。	
\\	時[とき]が 経[た]つにつれて 望郷[ぼうきょう]の 念[ねん]が 募[つの]った。	募る=つのる= 
\\	次第に不安が募ってきた。	
\\	次第[しだい]に 不安[ふあん]が 募[つの]ってきた。	募る=つのる= 
\\	政府に対する国民の不信感が募った。	
\\	政府[せいふ]に 対[たい]する 国民[こくみん]の 不信[ふしん] 感[かん]が 募[つの]った。	募る=つのる= 
\\	彼女は彼に対する恋心を日々募らせた。	
\\	彼女[かのじょ]は 彼[かれ]に 対[たい]する 恋心[こいごころ]を 日々[ひび] 募[つの]らせた。	募る=つのる= 
\\	コンサートには多彩な顔ぶれがそろった。	
\\	コンサートには 多彩[たさい]な 顔[かお]ぶれがそろった。	多彩な=たさいな= 
\\	彼女は多彩な経歴の持ち主だ。	
\\	彼女[かのじょ]は 多彩[たさい]な 経歴[けいれき]の 持ち主[もちぬし]だ。	多彩な=たさいな= 
\\	その会は政党的色彩のない団体だった。	
\\	その 会[かい]は 政党[せいとう] 的[てき] 色彩[しきさい]のない 団体[だんたい]だった。	色彩=しきさい= (彩り) 
\\	(調子・傾向・気味) 
\\	その寄付は売名行為的な色彩が濃い。	
\\	その 寄付[きふ]は 売名[ばいめい] 行為[こうい] 的[てき]な 色彩[しきさい]が 濃[こ]い。	色彩=しきさい= (彩り) 
\\	(調子・傾向・気味) 
\\	その慣習には宗教的色彩がある。	
\\	その 慣習[かんしゅう]には 宗教[しゅうきょう] 的[てき] 色彩[しきさい]がある。	色彩=しきさい= (彩り) 
\\	(調子・傾向・気味) 
\\	熱帯の鳥や魚は色彩が豊かだ。	
\\	熱帯[ねったい]の 鳥[とり]や 魚[さかな]は 色彩[しきさい]が 豊[ゆた]かだ。	色彩=しきさい= (彩り) 
\\	(調子・傾向・気味) 
\\	アマスポーツの祭典であるオリンピックも、現在では強く商業主義に彩られている。	
\\	アマスポーツの 祭典[さいてん]であるオリンピックも、 現在[げんざい]では 強[つよ]く 商業[しょうぎょう] 主義[しゅぎ]に 彩[いろど]られている。	彩る=いろどる= (彩色する) 
\\	(化粧する) 
\\	(飾る) 
\\	彼の地盤は固く、今回も当選確実だ。	
\\	彼[かれ]の 地盤[じばん]は 固[かた]く、 今回[こんかい]も 当選[とうせん] 確実[かくじつ]だ。	地盤=じばん= (地面) 
\\	(土台)
\\	(活動のための足場・地歩) 
\\	(選挙における勢力範囲) 
\\	(選挙区) 
\\	彼はふるさとに固い地盤を持っている。	
\\	彼[かれ]はふるさとに 固[かた]い 地盤[じばん]を 持[も]っている。	地盤=じばん= (地面) 
\\	(土台)
\\	(活動のための足場・地歩) 
\\	(選挙における勢力範囲) 
\\	(選挙区) 
\\	彼女は県内に確固とした地盤がある。	
\\	彼女[かのじょ]は 県内[けんない]に 確固[かっこ]とした 地盤[じばん]がある。	地盤=じばん= (地面) 
\\	(土台)
\\	(活動のための足場・地歩) 
\\	(選挙における勢力範囲) 
\\	(選挙区) 
\\	雨で地盤が緩んでいるので土砂崩れにご注意下さい。	
\\	雨[あめ]で 地盤[じばん]が 緩[ゆる]んでいるので 土砂崩[どしゃくず]れにご 注意[ちゅうい] 下[くだ]さい。	地盤=じばん= (地面) 
\\	(土台)
\\	(活動のための足場・地歩) 
\\	(選挙における勢力範囲) 
\\	(選挙区) 
\\	土砂崩れ=どしゃくずれ= 
\\	この家は地盤がしっかりしているから揺れない。	
\\	この 家[いえ]は 地盤[じばん]がしっかりしているから 揺[ゆ]れない。	地盤=じばん= (地面) 
\\	(土台)
\\	(活動のための足場・地歩) 
\\	(選挙における勢力範囲) 
\\	(選挙区) 
\\	教育の普及こそが民主主義のよって立つ基盤である。	
\\	教育[きょういく]の 普及[ふきゅう]こそが 民主[みんしゅ] 主義[しゅぎ]のよって 立[た]つ 基盤[きばん]である。	基盤=きばん= (物事の基礎・土台) 
\\	土台から作り直す必要がある。	
\\	土台[どだい]から 作り直[つくりなお]す 必要[ひつよう]がある。	土台=どだい= (建築の) 
\\	(物事の基礎) 
\\	この家は土台がしっかりしている。	
\\	この 家[いえ]は 土台[どだい]がしっかりしている。	土台=どだい= (建築の) 
\\	(物事の基礎) 
\\	その本は11月下旬に発売される。	
\\	その 本[ほん]は 11月[じゅういちがつ] 下旬[げじゅん]に 発売[はつばい]される。	
\\	安いのが取り柄ですね。	
\\	安[やす]いのが 取り柄[とりえ]ですね。	取り柄=とりえ= 
\\	彼にはこれといって取り柄がない。	
\\	彼[かれ]にはこれといって 取り柄[とりえ]がない。	取り柄=とりえ= 
\\	彼とは会えば会釈する程度の間柄である。	
\\	彼[かれ]とは 会[あ]えば 会釈[えしゃく]する 程度[ていど]の 間柄[あいだがら]である。	間柄=あいだがら= (血縁の) 
\\	(交際の) 
\\	君とあの人とはどういう間柄なのだ。	
\\	君[きみ]とあの 人[ひと]とはどういう 間柄[あいだがら]なのだ。	間柄=あいだがら= (血縁の) 
\\	(交際の) 
\\	達也は彼女とごく親しい間柄だ。	
\\	達也[たつや]は 彼女[かのじょ]とごく 親[した]しい 間柄[あいだがら]だ。	間柄=あいだがら= (血縁の) 
\\	(交際の) 
\\	彼とはお前・俺の間柄だ。	
\\	彼[かれ]とはお 前[まえ]・ 俺[おれ]の 間柄[あいだがら]だ。	間柄=あいだがら= (血縁の) 
\\	(交際の) 
\\	彼女たちはおばと姪の間柄だ。	
\\	彼女[かのじょ]たちはおばと 姪[めい]の 間柄[あいだがら]だ。	間柄=あいだがら= (血縁の) 
\\	(交際の) 
\\	おじおいの間柄です。	
\\	おじおいの 間柄[あいだがら]です。	間柄=あいだがら= (血縁の) 
\\	(交際の) 
\\	それだけで彼女の人柄がわかる。	
\\	それだけで 彼女[かのじょ]の 人柄[ひとがら]がわかる。	人柄=ひとがら= (性質) 
\\	彼女の奥ゆかしい人柄に引かれました。	
\\	彼女[かのじょ]の 奥[おく]ゆかしい 人柄[ひとがら]に 引[ひ]かれました。	奥ゆかしい=おくゆかしい= 
\\	(女性が) 
\\	人柄=ひとがら= (性質) 
\\	私は彼の裏表のない人柄を買ったのです。	
\\	私[わたし]は 彼[かれ]の 裏表[うらおもて]のない 人柄[ひとがら]を 買[か]ったのです。	人柄=ひとがら= (性質) 
\\	彼はどんな人柄ですか。	
\\	彼[かれ]はどんな 人柄[ひとがら]ですか。	人柄=ひとがら= (性質) 
\\	私は人に指図するなんて柄じゃない。	
\\	私[わたし]は 人[ひと]に 指図[さしず]するなんて 柄[がら]じゃない。	柄=がら= (模様) 
\\	(体格) 
\\	(品格) 
\\	(その人の性質や身分・分際)
\\	私はまだ未熟者であなたに教えるなんて柄じゃありません。	
\\	私[わたし]はまだ 未[み] 熟[じゅく] 者[しゃ]であなたに 教[おし]えるなんて 柄[がら]じゃありません。	柄=がら= (模様) 
\\	(体格) 
\\	(品格) 
\\	(その人の性質や身分・分際)
\\	彼は柄は悪いが、いい奴だ。	
\\	彼[かれ]は 柄[がら]は 悪[わる]いが、いい 奴[やっこ]だ。	柄=がら= (模様) 
\\	(体格) 
\\	(品格) 
\\	(その人の性質や身分・分際)
\\	彼は教師の柄ではない。	
\\	彼[かれ]は 教師[きょうし]の 柄[え]ではない。	柄=がら= (模様) 
\\	(体格) 
\\	(品格) 
\\	(その人の性質や身分・分際)
\\	彼は柄にもなくこんな本を読んでいる。	
\\	彼[かれ]は 柄[え]にもなくこんな 本[ほん]を 読[よ]んでいる。	柄=がら= (模様) 
\\	(体格) 
\\	(品格) 
\\	(その人の性質や身分・分際)
\\	選手たちは沿道の声援に励まされて走った。	
\\	選手[せんしゅ]たちは 沿道[えんどう]の 声援[せいえん]に 励[はげ]まされて 走[はし]った。	沿道=えんどう= 
\\	パレードを待つ沿道は熱気にあふれていた。	
\\	パレードを 待[ま]つ 沿道[えんどう]は 熱気[ねっき]にあふれていた。	沿道=えんどう= 
\\	この赤い線に沿って進んでください。	
\\	この 赤[あか]い 線[せん]に 沿[そ]って 進[すす]んでください。	沿う=そう= (長く続いているものから離れないでいる); (方針などに従う)
\\	川に沿って温泉宿が並んでいる。	
\\	川[かわ]に 沿[そ]って 温泉[おんせん] 宿[やど]が 並[なら]んでいる。	沿う=そう= (長く続いているものから離れないでいる); (方針などに従う)
\\	最近の子供にはうっかりしたことは言えない。	
\\	最近[さいきん]の 子供[こども]にはうっかりしたことは 言[い]えない。	うっかり= 
\\	すみません、私がうっかりしていました。	
\\	すみません、 私[わたし]がうっかりしていました。	うっかり= 
\\	うっかりしてバスを乗り間違えた。	
\\	うっかりしてバスを 乗[の]り 間違[まちが]えた。	うっかり= 
\\	空気がからからだから肌ががさがさだ。	
\\	空気[くうき]がからからだから 肌[はだ]ががさがさだ。	
\\	地面はからからに乾いている。	
\\	地面[じめん]はからからに 乾[かわ]いている。	
\\	今日のレースでは最初からがんがん飛ばしていく作戦だ。	
\\	今日[きょう]のレースでは 最初[さいしょ]からがんがん 飛[と]ばしていく 作戦[さくせん]だ。	がんがん= (連続する鈍い金属的な音) 
\\	(勢いのよいさま)
\\	彼の怒声が部屋中にがんがん響いた。	
\\	彼[かれ]の 怒声[どせい]が 部屋[へや] 中[ちゅう]にがんがん 響[ひび]いた。	がんがん= (連続する鈍い金属的な音) 
\\	(勢いのよいさま)
\\	ぐずぐずしている場合ではない。	
\\	ぐずぐずしている 場合[ばあい]ではない。	ぐずぐず= (のろいようす) 
\\	(ぶつぶつ言うようす)
\\	ぐずぐずしていないで君の考えを言いなさい。	
\\	ぐずぐずしていないで 君[きみ]の 考[かんが]えを 言[い]いなさい。	ぐずぐず= (のろいようす) 
\\	(ぶつぶつ言うようす)
\\	彼はぐっすり眠っている。	
\\	彼[かれ]はぐっすり 眠[ねむ]っている。	ぐっすり= 
\\	夕べはぐっすり眠れましたか。	
\\	夕[ゆう]べはぐっすり 眠[ねむ]れましたか。	ぐっすり= 
\\	一晩ぐっすり眠った後は気分がよい。	
\\	一晩[ひとばん]ぐっすり 眠[ねむ]った 後[のち]は 気分[きぶん]がよい。	ぐっすり= 
\\	仕事と遊びがごちゃごちゃになっている。	
\\	仕事[しごと]と 遊[あそ]びがごちゃごちゃになっている。	
\\	問い合わせの電話が殺到した。	
\\	問い合[といあ]わせの 電話[でんわ]が 殺到[さっとう]した。	殺到=さっとう= 
\\	(〜する) 
\\	(押し掛ける) 
\\	抗議が殺到した。	
\\	抗議[こうぎ]が 殺到[さっとう]した。	殺到=さっとう= 
\\	(〜する) 
\\	(押し掛ける) 
\\	女の子たちがサインを求めて彼の所へ殺到した。	
\\	女の子[おんなのこ]たちがサインを 求[もと]めて 彼[かれ]の 所[ところ]へ 殺到[さっとう]した。	殺到=さっとう= 
\\	(〜する) 
\\	(押し掛ける) 
\\	元気がもりもり回復した。	
\\	元気[げんき]がもりもり 回復[かいふく]した。	
\\	彼は筋肉もりもりだ。	
\\	彼[かれ]は 筋肉[きんにく]もりもりだ。	
\\	彼は押されてよろよろした。	
\\	彼[かれ]は 押[お]されてよろよろした。	よろよろ= 
\\	彼は病人のようによろよろと通りを歩いていった。	
\\	彼[かれ]は 病人[びょうにん]のようによろよろと 通[とお]りを 歩[ある]いていった。	よろよろ= 
\\	彼は酔って足下がよろよろしていた。	
\\	彼[かれ]は 酔[よ]って 足下[あしもと]がよろよろしていた。	よろよろ= 
\\	彼はどちらかというと小柄なほうでした。	
\\	彼[かれ]はどちらかというと 小柄[こがら]なほうでした。	小柄な=こがらな= (体格が) 
\\	彼は小柄だががっしりしている。	
\\	彼[かれ]は 小柄[こがら]だががっしりしている。	小柄な=こがらな= (体格が) 
\\	彼女はバスケットボール選手としては小柄だ。	
\\	彼女[かのじょ]はバスケットボール 選手[せんしゅ]としては 小柄[こがら]だ。	小柄な=こがらな= (体格が) 
\\	私の上司は目下の者には横柄だ。	
\\	私[わたし]の 上司[じょうし]は 目下[めした]の 者[もの]には 横柄[おうへい]だ。	横柄な=おうへいな= (傲慢・尊大な) 
\\	(見下すような) 
\\	(高圧的な) 
\\	こうすれば能率が上がるだろう。	
\\	こうすれば 能率[のうりつ]が 上[あ]がるだろう。	能率=のうりつ= (はかどり具合) 
\\	やり方を変えたらかえって能率が落ちてしまった。	
\\	やり 方[かた]を 変[か]えたらかえって 能率[のうりつ]が 落[お]ちてしまった。	能率=のうりつ= (はかどり具合) 
\\	私は朝の方が仕事の能率が上がる。	
\\	私[わたし]は 朝[あさ]の 方[ほう]が 仕事[しごと]の 能率[のうりつ]が 上[あ]がる。	能率=のうりつ= (はかどり具合) 
\\	そのデータを処理するのにもっと能率的な方法はないのか。	
\\	そのデータを 処理[しょり]するのにもっと 能率[のうりつ] 的[てき]な 方法[ほうほう]はないのか。	能率=のうりつ= (はかどり具合) 
\\	日本と米国では国柄が違う。	
\\	日本[にっぽん]と 米国[べいこく]では 国柄[くにがら]が 違[ちが]う。	国柄=くにがら= 
\\	イタリアは陽気なお国柄だ。	
\\	イタリアは 陽気[ようき]なお 国柄[くにがら]だ。	国柄=くにがら= 
\\	こんな政府に国の命運を託すことはできない。	
\\	こんな 政府[せいふ]に 国[くに]の 命運[めいうん]を 託[たく]すことはできない。	託す=たくす= (預ける) 
\\	(任せる) 
\\	(他の形を借りて表現する) 
\\	すべては田中君に託してある。	
\\	すべては 田中[たなか] 君[くん]に 託[たく]してある。	託す=たくす= (預ける) 
\\	(任せる) 
\\	(他の形を借りて表現する) 
\\	この件は君に託しておけば安心だ。	
\\	この 件[けん]は 君[きみ]に 託[たく]しておけば 安心[あんしん]だ。	託す=たくす= (預ける) 
\\	(任せる) 
\\	(他の形を借りて表現する) 
\\	この手紙をお渡しするよう託されて参りました。	
\\	この 手紙[てがみ]をお 渡[わた]しするよう 託[たく]されて 参[まい]りました。	託す=たくす= (預ける) 
\\	(任せる) 
\\	(他の形を借りて表現する) 
\\	友人に両親への伝言を託した。	
\\	友人[ゆうじん]に 両親[りょうしん]への 伝言[でんごん]を 託[たく]した。	託す=たくす= (預ける) 
\\	(任せる) 
\\	(他の形を借りて表現する) 
\\	僕は何の屈託もない。	
\\	僕[ぼく]は 何[なに]の 屈託[くったく]もない。	屈託=くったく= (心配) 
\\	(疲れて飽きること) 
\\	彼は物事に屈託しない性分だ。	
\\	彼[かれ]は 物事[ものごと]に 屈託[くったく]しない 性分[しょうぶん]だ。	屈託=くったく= (心配) 
\\	(疲れて飽きること) 
\\	性分=しょうぶん= (性質) 
\\	(気質) 
\\	夫ののんきな顔を見てるとむかむかと腹が立ってきた。	
\\	夫[おっと]ののんきな 顔[かお]を 見[み]てるとむかむかと 腹[はら]が 立[た]ってきた。	むかむか= (吐き気を催すようす) 
\\	(腹立たしいようす) 
\\	飲み過ぎて胃がむかむかする。	
\\	飲[の]み 過[す]ぎて 胃[い]がむかむかする。	むかむか= (吐き気を催すようす) 
\\	(腹立たしいようす) 
\\	それを見ているうちになんだかむかむかしてきた。	
\\	それを 見[み]ているうちになんだかむかむかしてきた。	むかむか= (吐き気を催すようす) 
\\	(腹立たしいようす) 
\\	それを見ているだけで胸がむかむかする。	
\\	それを 見[み]ているだけで 胸[むね]がむかむかする。	むかむか= (吐き気を催すようす) 
\\	(腹立たしいようす) 
\\	車酔いで胸がむかむかする。	
\\	車[くるま] 酔[よ]いで 胸[むね]がむかむかする。	むかむか= (吐き気を催すようす) 
\\	(腹立たしいようす) 
\\	ぼんやりしちゃいられない。	
\\	ぼんやりしちゃいられない。	ぼんやり= (はっきりしないようす) 
\\	(放心したようす) 
\\	(注意力を欠いたようす); (無為に時を過ごすようす) 
\\	ぼんやりつったっていないで、何か手伝え。	
\\	ぼんやりつったっていないで、 何[なに]か 手伝[てつだ]え。	ぼんやり= (はっきりしないようす) 
\\	(放心したようす) 
\\	(注意力を欠いたようす); (無為に時を過ごすようす) 
\\	ぼんやりしていてすみませんでした。	
\\	ぼんやりしていてすみませんでした。	ぼんやり= (はっきりしないようす) 
\\	(放心したようす) 
\\	(注意力を欠いたようす); (無為に時を過ごすようす) 
\\	何をぼんやり考えているんだ。	
\\	何[なに]をぼんやり 考[かんが]えているんだ。	ぼんやり= (はっきりしないようす) 
\\	(放心したようす) 
\\	(注意力を欠いたようす); (無為に時を過ごすようす) 
\\	霧で富士山もぼんやりとしか見えなかった。	
\\	霧[きり]で 富士山[ふじさん]もぼんやりとしか 見[み]えなかった。	ぼんやり= (はっきりしないようす) 
\\	(放心したようす) 
\\	(注意力を欠いたようす); (無為に時を過ごすようす) 
\\	飲み過ぎて頭がぼんやりしてきた。	
\\	飲[の]み 過[す]ぎて 頭[あたま]がぼんやりしてきた。	ぼんやり= (はっきりしないようす) 
\\	(放心したようす) 
\\	(注意力を欠いたようす); (無為に時を過ごすようす) 
\\	授業中にぼんやりしていて先生に叱られた。	
\\	授業[じゅぎょう] 中[ちゅう]にぼんやりしていて 先生[せんせい]に 叱[しか]られた。	ぼんやり= (はっきりしないようす) 
\\	(放心したようす) 
\\	(注意力を欠いたようす); (無為に時を過ごすようす) 
\\	今日は寝不足で頭がぼんやりしている。	
\\	今日[きょう]は 寝不足[ねぶそく]で 頭[あたま]がぼんやりしている。	ぼんやり= (はっきりしないようす) 
\\	(放心したようす) 
\\	(注意力を欠いたようす); (無為に時を過ごすようす) 
\\	彼はぼんやりした目で私を見た。	
\\	彼[かれ]はぼんやりした 目[め]で 私[わたし]を 見[み]た。	ぼんやり= (はっきりしないようす) 
\\	(放心したようす) 
\\	(注意力を欠いたようす); (無為に時を過ごすようす) 
\\	関係者がぼろぼろ逮捕されていった。	
\\	関係[かんけい] 者[しゃ]がぼろぼろ 逮捕[たいほ]されていった。	ぼろぼろ= (涙や粒がこぼれるようす); (乾いてパサパサになるようす); (物が傷んでいるようす) 
\\	(心身が疲れ切っているようす); (次々に表れ出るようす)
\\	もう身も心もぼろぼろです。	
\\	もう 身[み]も 心[こころ]もぼろぼろです。	ぼろぼろ= (涙や粒がこぼれるようす); (乾いてパサパサになるようす); (物が傷んでいるようす) 
\\	(心身が疲れ切っているようす); (次々に表れ出るようす)
\\	精神的にぼろぼろになりそうだ。	
\\	精神[せいしん] 的[てき]にぼろぼろになりそうだ。	ぼろぼろ= (涙や粒がこぼれるようす); (乾いてパサパサになるようす); (物が傷んでいるようす) 
\\	(心身が疲れ切っているようす); (次々に表れ出るようす)
\\	辞書がぼろぼろになった。	
\\	辞書[じしょ]がぼろぼろになった。	ぼろぼろ= (涙や粒がこぼれるようす); (乾いてパサパサになるようす); (物が傷んでいるようす) 
\\	(心身が疲れ切っているようす); (次々に表れ出るようす)
\\	力を入れてこすったら垢がぼろぼろ落ちた。	
\\	力[ちから]を 入[い]れてこすったら 垢[あか]がぼろぼろ 落[お]ちた。	こする= 
\\	垢=あか= 
\\	ぼろぼろ= (涙や粒がこぼれるようす); (乾いてパサパサになるようす); (物が傷んでいるようす) 
\\	(心身が疲れ切っているようす); (次々に表れ出るようす)
\\	彼女は麻薬で身も心もぼろぼろになった。	
\\	彼女[かのじょ]は 麻薬[まやく]で 身[み]も 心[こころ]もぼろぼろになった。	ぼろぼろ= (涙や粒がこぼれるようす); (乾いてパサパサになるようす); (物が傷んでいるようす) 
\\	(心身が疲れ切っているようす); (次々に表れ出るようす)
\\	試験の結果はぼろぼろだった。	
\\	試験[しけん]の 結果[けっか]はぼろぼろだった。	ぼろぼろ= (涙や粒がこぼれるようす); (乾いてパサパサになるようす); (物が傷んでいるようす) 
\\	(心身が疲れ切っているようす); (次々に表れ出るようす)
\\	竜巻が発生して大きな被害が出た。	
\\	竜巻[たつまき]が 発生[はっせい]して 大[おお]きな 被害[ひがい]が 出[で]た。	竜巻=たつまき= 
\\	芥川賞・直木賞は文壇への登竜門だ。	
\\	芥川賞[あくたがわしょう]・ 直木賞[なおきしょう]は 文壇[ぶんだん]への 登竜門[とうりゅうもん]だ。	文壇=ぶんだん= 作家・批判かなどの社会。文学界。登竜門=とうりゅうもん= 
\\	この学校からはたくさんの俊才が出ている。	
\\	この 学校[がっこう]からはたくさんの 俊才[しゅんさい]が 出[で]ている。	俊才=しゅんさい= 
\\	子供はさといから親の隠し事をすぐ感じ取る。	
\\	子供[こども]はさといから 親[おや]の 隠し事[かくしごと]をすぐ 感じ取[かんじと]る。	敏い・聡い=さとい= (賢い) 
\\	(鋭い) 
\\	その言葉は彼の神経の一番敏感なところに触れた。	
\\	その 言葉[ことば]は 彼[かれ]の 神経[しんけい]の 一番[いちばん] 敏感[びんかん]なところに 触[ふ]れた。	敏感な=びんかんな= 
\\	彼は敏感なのか鈍感なのかわからない。	
\\	彼[かれ]は 敏感[びんかん]なのか 鈍感[どんかん]なのかわからない。	敏感な=びんかんな= 
\\	医者は患者の病状の変化に敏感でなければならない。	
\\	医者[いしゃ]は 患者[かんじゃ]の 病状[びょうじょう]の 変化[へんか]に 敏感[びんかん]でなければならない。	敏感な=びんかんな= 
\\	彼は他人の言葉に敏感だ。	
\\	彼[かれ]は 他人[たにん]の 言葉[ことば]に 敏感[びんかん]だ。	敏感な=びんかんな= 
\\	犬は人間よりも音に敏感だ。	
\\	犬[いぬ]は 人間[にんげん]よりも 音[おと]に 敏感[びんかん]だ。	敏感な=びんかんな= 
\\	彼女は流行に敏感だ。	
\\	彼女[かのじょ]は 流行[りゅうこう]に 敏感[びんかん]だ。	敏感な=びんかんな= 
\\	お宅の坊ちゃんはおいくつですか。	
\\	お 宅[たく]の 坊[ぼっ]ちゃんはおいくつですか。	坊ちゃん=ぼっちゃん= 
\\	かわいらしい坊ちゃんですね。	
\\	かわいらしい 坊[ぼっ]ちゃんですね。	坊ちゃん=ぼっちゃん= 
\\	坊ちゃん、お名前は?	
\\	坊[ぼっ]ちゃん、お 名前[なまえ]は?	坊ちゃん=ぼっちゃん= 
\\	部屋の中がぼうっと明るくなった。	
\\	部屋[へや]の 中[なか]がぼうっと 明[あか]るくなった。	ぼうっと= (ぼんやりしているようす) 
\\	(はっきり見えないようす) 
\\	(火が燃え立つようす) 
\\	霧が深くて向かい側の家もぼうっとしか見えなかった。	
\\	霧[きり]が 深[ふか]くて 向かい側[むかいがわ]の 家[いえ]もぼうっとしか 見[み]えなかった。	ぼうっと= (ぼんやりしているようす) 
\\	(はっきり見えないようす) 
\\	(火が燃え立つようす) 
\\	ぼうっと海を眺めていた。	
\\	ぼうっと 海[うみ]を 眺[なが]めていた。	ぼうっと= (ぼんやりしているようす) 
\\	(はっきり見えないようす) 
\\	(火が燃え立つようす) 
\\	風邪のせいか頭がぼうっとする。	
\\	風邪[かぜ]のせいか 頭[あたま]がぼうっとする。	ぼうっと= (ぼんやりしているようす) 
\\	(はっきり見えないようす) 
\\	(火が燃え立つようす) 
\\	あいつはいつもぼうっとした顔をしている。	
\\	あいつはいつもぼうっとした 顔[かお]をしている。	ぼうっと= (ぼんやりしているようす) 
\\	(はっきり見えないようす) 
\\	(火が燃え立つようす) 
\\	ぼうっとしているうちに夏休みも終わってしまった。	
\\	ぼうっとしているうちに 夏休[なつやす]みも 終[お]わってしまった。	ぼうっと= (ぼんやりしているようす) 
\\	(はっきり見えないようす) 
\\	(火が燃え立つようす) 
\\	時には何もせずぼうっとして過ごすのも悪くない。	
\\	時[とき]には 何[なに]もせずぼうっとして 過[す]ごすのも 悪[わる]くない。	ぼうっと= (ぼんやりしているようす) 
\\	(はっきり見えないようす) 
\\	(火が燃え立つようす) 
\\	街の明かりが遠くにぼうっと浮かんでいる。	
\\	街[まち]の 明[あ]かりが 遠[とお]くにぼうっと 浮[う]かんでいる。	ぼうっと= (ぼんやりしているようす) 
\\	(はっきり見えないようす) 
\\	(火が燃え立つようす) 
\\	二日酔いで頭がぼうっとしている。	
\\	二日酔[ふつかよ]いで 頭[あたま]がぼうっとしている。	ぼうっと= (ぼんやりしているようす) 
\\	(はっきり見えないようす) 
\\	(火が燃え立つようす) 
\\	手元が不確かだとこの仕事は無理だ。	
\\	手元[てもと]が 不確[ふたし]かだとこの 仕事[しごと]は 無理[むり]だ。	手元=てもと= (手の届く近い所); 
\\	(手で握るところ) 
\\	ナイフを持つ手元が危なっかしい。	
\\	ナイフを 持[も]つ 手元[てもと]が 危[あぶ]なっかしい。	手元=てもと= (手の届く近い所); 
\\	(手で握るところ) 
\\	買いたいが残念ながら手元不如意だ。	
\\	買[か]いたいが 残念[ざんねん]ながら 手元[てもと] 不如意[ふにょい]だ。	手元=てもと= (手の届く近い所); 
\\	(手で握るところ) 
\\	お手元の資料をご覧ください。	
\\	お 手元[てもと]の 資料[しりょう]をご 覧[らん]ください。	手元=てもと= (手の届く近い所); 
\\	(手で握るところ) 
\\	今手元にいくら持っていますか。	
\\	今[いま] 手元[てもと]にいくら 持[も]っていますか。	手元=てもと= (手の届く近い所); 
\\	(手で握るところ) 
\\	父親は娘をできるだけ長く手元に置きたがるものだ。	
\\	父親[ちちおや]は 娘[むすめ]をできるだけ 長[なが]く 手元[てもと]に 置[お]きたがるものだ。	手元=てもと= (手の届く近い所); 
\\	(手で握るところ) 
\\	あの本は今手元にない。	
\\	あの 本[ほん]は 今[いま] 手元[てもと]にない。	手元=てもと= (手の届く近い所); 
\\	(手で握るところ) 
\\	彼は祖父母の手元で育てられた。	
\\	彼[かれ]は 祖父母[そふぼ]の 手元[てもと]で 育[そだ]てられた。	手元=てもと= (手の届く近い所); 
\\	(手で握るところ) 
\\	彼はその話題に過敏すぎる。	
\\	彼[かれ]はその 話題[わだい]に 過敏[かびん]すぎる。	過敏な=かびんな= 
\\	住民は地震に対して過敏になっている。	
\\	住民[じゅうみん]は 地震[じしん]に 対[たい]して 過敏[かびん]になっている。	過敏な=かびんな= 
\\	われわれは緊急事態に機敏に対処しなければならない。	
\\	われわれは 緊急[きんきゅう] 事態[じたい]に 機敏[きびん]に 対処[たいしょ]しなければならない。	機敏な=きびんな= 
\\	救助隊には冷静さと機敏さが要求されている。	
\\	救助[きゅうじょ] 隊[たい]には 冷静[れいせい]さと 機敏[きびん]さが 要求[ようきゅう]されている。	機敏な=きびんな= 
\\	彼の機敏な対応のおかげで火事にならずにすんだ。	
\\	彼[かれ]の 機敏[きびん]な 対応[たいおう]のおかげで 火事[かじ]にならずにすんだ。	機敏な=きびんな= 
\\	彼は頭が鋭敏に働く。	
\\	彼[かれ]は 頭[あたま]が 鋭敏[えいびん]に 働[はたら]く。	鋭敏な=えいびんな= (感じ方が) 
\\	(寒さ・痛みなどに) 
\\	犬はわずかな匂いにも鋭敏に反応する。	
\\	犬[いぬ]はわずかな 匂[にお]いにも 鋭敏[えいびん]に 反応[はんのう]する。	鋭敏な=えいびんな= (感じ方が) 
\\	(寒さ・痛みなどに) 
\\	彼は鼻が鋭敏だ。	
\\	彼[かれ]は 鼻[はな]が 鋭敏[えいびん]だ。	鋭敏な=えいびんな= (感じ方が) 
\\	(寒さ・痛みなどに) 
\\	彼は大いに敏腕をふるった。	
\\	彼[かれ]は 大[おお]いに 敏腕[びんわん]をふるった。	敏腕な=びんわんな= 
\\	あの選手は動きが敏捷だ。	
\\	あの 選手[せんしゅ]は 動[うご]きが 敏捷[びんしょう]だ。	敏捷な=びんしょうな= 
\\	机に足を載せるな。	
\\	机[つくえ]に 足[あし]を 載[の]せるな。	載せる=のせる= (記す・掲載する) 
\\	(置く) 
\\	乗用車の屋根にスキー板を載せている。	
\\	乗用車[じょうようしゃ]の 屋根[やね]にスキー 板[いた]を 載[の]せている。	載せる=のせる= (記す・掲載する) 
\\	(置く) 
\\	この辞書は科学用語をたくさん載せている。	
\\	この 辞書[じしょ]は 科学[かがく] 用語[ようご]をたくさん 載[の]せている。	載せる=のせる= (記す・掲載する) 
\\	(置く) 
\\	彼女はこの雑誌に毎号記事を載せている。	
\\	彼女[かのじょ]はこの 雑誌[ざっし]に 毎号[まいごう] 記事[きじ]を 載[の]せている。	載せる=のせる= (記す・掲載する) 
\\	(置く) 
\\	私の名は電話帳に載せないでください。	
\\	私[わたし]の 名[な]は 電話[でんわ] 帳[ちょう]に 載[の]せないでください。	載せる=のせる= (記す・掲載する) 
\\	(置く) 
\\	ある地方紙が市長へのインタビューを載せていた。	
\\	ある 地方紙[ちほうし]が 市長[しちょう]へのインタビューを 載[の]せていた。	載せる=のせる= (記す・掲載する) 
\\	(置く) 
\\	今日はまたべらぼうに暑い。	
\\	今日[きょう]はまたべらぼうに 暑[あつ]い。	べらぼうな= (馬鹿げた) 
\\	(法外な) 
\\	そんなべらぼうな話があるか。	
\\	そんなべらぼうな 話[はなし]があるか。	べらぼうな= (馬鹿げた) 
\\	(法外な) 
\\	腹がぺこぺこで目が回る。	
\\	腹[はら]がぺこぺこで 目[め]が 回[まわ]る。	ぺこぺこ= (何度も頭を下げるさま) 
\\	(空腹であること)
\\	「お腹すいた?」 
\\	「もうぺこぺこだよ」	
\\	「お 腹[なか]すいた?」 
\\	「もうぺこぺこだよ」	ぺこぺこ= (何度も頭を下げるさま) 
\\	(空腹であること)
\\	私は人にぺこぺこ頭を下げるのが嫌いだ。	
\\	私[わたし]は 人[ひと]にぺこぺこ 頭[あたま]を 下[さ]げるのが 嫌[きら]いだ。	ぺこぺこ= (何度も頭を下げるさま) 
\\	(空腹であること)
\\	老人がぺこぺこ頭を下げながら部屋に入ってきた。	
\\	老人[ろうじん]がぺこぺこ 頭[あたま]を 下[さ]げながら 部屋[へや]に 入[はい]ってきた。	ぺこぺこ= (何度も頭を下げるさま) 
\\	(空腹であること)
\\	彼は電話口でぺこぺこ頭を下げていた。	
\\	彼[かれ]は 電話[でんわ] 口[ぐち]でぺこぺこ 頭[あたま]を 下[さ]げていた。	ぺこぺこ= (何度も頭を下げるさま) 
\\	(空腹であること)
\\	彼にぺこぺこする必要なんてないよ。	
\\	彼[かれ]にぺこぺこする 必要[ひつよう]なんてないよ。	ぺこぺこ= (何度も頭を下げるさま) 
\\	(空腹であること)
\\	ぶるぶる手が震えて字が書けなかった。	
\\	ぶるぶる 手[て]が 震[ふる]えて 字[じ]が 書[か]けなかった。	ぶるぶる= 
\\	(震える)
\\	川から上がった犬はぶるぶると体を震った。	
\\	川[かわ]から 上[あ]がった 犬[いぬ]はぶるぶると 体[からだ]を 震[ふ]った。	ぶるぶる= 
\\	(震える)
\\	箱の中の子犬は寒いのか、ぶるぶる震えていた。	
\\	箱[はこ]の 中[なか]の 子犬[こいぬ]は 寒[さむ]いのか、ぶるぶる 震[ふる]えていた。	ぶるぶる= 
\\	(震える)
\\	彼は寒さでぶるぶる震えていた。	
\\	彼[かれ]は 寒[さむ]さでぶるぶる 震[ふる]えていた。	ぶるぶる= 
\\	(震える)
\\	この船は一度の積載で1200台の自動車を運ぶ。	
\\	この 船[ふね]は 一度[いちど]の 積載[せきさい]で1200 台[だい]の 自動車[じどうしゃ]を 運[はこ]ぶ。	積載=せきさい= 
\\	(〜する) 
\\	その軍艦は核兵器を搭載している。	
\\	その 軍艦[ぐんかん]は 核兵器[かくへいき]を 搭載[とうさい]している。	搭載=とうさい= 
\\	(〜する) 
\\	この飛行機は新型エンジンを搭載している。	
\\	この 飛行機[ひこうき]は 新型[しんがた]エンジンを 搭載[とうさい]している。	搭載=とうさい= 
\\	(〜する) 
\\	このパソコンは2
\\	のメモリーを搭載している。	
\\	このパソコンは2 
\\	のメモリーを 搭載[とうさい]している。	搭載=とうさい= 
\\	(〜する) 
\\	バーコードのしたに記載されている13桁の数字を入力してください。	
\\	バーコードのしたに 記載[きさい]されている13 桁[けた]の 数字[すうじ]を 入力[にゅうりょく]してください。	記載=きさい= (文書・書類などへの) 
\\	この卒業アルバムには全卒業生の名前が記載されている。	
\\	この 卒業[そつぎょう]アルバムには 全[ぜん] 卒業生[そつぎょうせい]の 名前[なまえ]が 記載[きさい]されている。	記載=きさい= (文書・書類などへの) 
\\	彼の死亡記事は昨日の新聞に掲載された。	
\\	彼[かれ]の 死亡[しぼう] 記事[きじ]は 昨日[きのう]の 新聞[しんぶん]に 掲載[けいさい]された。	掲載=けいさい= 
\\	(〜する) 
\\	(事実などを記入する) 
\\	彼の原稿の掲載は見合わせることになった。	
\\	彼[かれ]の 原稿[げんこう]の 掲載[けいさい]は 見合[みあ]わせることになった。	掲載=けいさい= 
\\	(〜する) 
\\	(事実などを記入する) 
\\	その会社は新聞に広告を掲載した。	
\\	その 会社[かいしゃ]は 新聞[しんぶん]に 広告[こうこく]を 掲載[けいさい]した。	掲載=けいさい= 
\\	(〜する) 
\\	(事実などを記入する) 
\\	その新聞は彼女の手紙の全文を掲載した。	
\\	その 新聞[しんぶん]は 彼女[かのじょ]の 手紙[てがみ]の 全文[ぜんぶん]を 掲載[けいさい]した。	掲載=けいさい= 
\\	(〜する) 
\\	(事実などを記入する) 
\\	この記事は明日の新聞の文化欄に掲載される。	
\\	この 記事[きじ]は 明日[あした]の 新聞[しんぶん]の 文化[ぶんか] 欄[らん]に 掲載[けいさい]される。	掲載=けいさい= 
\\	(〜する) 
\\	(事実などを記入する) 
\\	先ほどの地震は
\\	4と推定されます。	
\\	先[さき]ほどの 地震[じしん]は 
\\	4と 推定[すいてい]されます。	推定=すいてい= 
\\	被害総額は推定10億円だ。	
\\	被害[ひがい] 総額[そうがく]は 推定[すいてい]10 億[おく] 円[えん]だ。	推定=すいてい= 
\\	出火の原因は放火と推定される。	
\\	出火[しゅっか]の 原因[げんいん]は 放火[ほうか]と 推定[すいてい]される。	推定=すいてい= 
\\	このパソコンにはソフトがたくさん載っている。	
\\	このパソコンにはソフトがたくさん 載[の]っている。	載る=のる= (記される) 
\\	(掲載される) 
\\	(置かれる) 
\\	(積載される) 
\\	(搭載される) 
\\	私の車では荷物が載り切らない。	
\\	私[わたし]の 車[くるま]では 荷物[にもつ]が 載[の]り 切[き]らない。	載る=のる= (記される) 
\\	(掲載される) 
\\	(置かれる) 
\\	(積載される) 
\\	(搭載される) 
\\	皿にはオムレツが載っていた。	
\\	皿[さら]にはオムレツが 載[の]っていた。	載る=のる= (記される) 
\\	(掲載される) 
\\	(置かれる) 
\\	(積載される) 
\\	(搭載される) 
\\	その事件は新聞に載るほどではなかった。	
\\	その 事件[じけん]は 新聞[しんぶん]に 載[の]るほどではなかった。	載る=のる= (記される) 
\\	(掲載される) 
\\	(置かれる) 
\\	(積載される) 
\\	(搭載される) 
\\	この語はどの辞書にも載っていない。	
\\	この 語[ご]はどの 辞書[じしょ]にも 載[の]っていない。	載る=のる= (記される) 
\\	(掲載される) 
\\	(置かれる) 
\\	(積載される) 
\\	(搭載される) 
\\	彼の顔写真が新聞に載った。	
\\	彼[かれ]の 顔[かお] 写真[しゃしん]が 新聞[しんぶん]に 載[の]った。	載る=のる= (記される) 
\\	(掲載される) 
\\	(置かれる) 
\\	(積載される) 
\\	(搭載される) 
\\	向こうの棚にジャムのびんが載っている。	
\\	向[む]こうの 棚[たな]にジャムのびんが 載[の]っている。	載る=のる= (記される) 
\\	(掲載される) 
\\	(置かれる) 
\\	(積載される) 
\\	(搭載される) 
\\	彼の名前はリストに載っている。	
\\	彼[かれ]の 名前[なまえ]はリストに 載[の]っている。	載る=のる= (記される) 
\\	(掲載される) 
\\	(置かれる) 
\\	(積載される) 
\\	(搭載される) 
\\	このサイトには役立つレシピがたくさん載っている。	
\\	このサイトには 役立[やくだ]つレシピがたくさん 載[の]っている。	載る=のる= (記される) 
\\	(掲載される) 
\\	(置かれる) 
\\	(積載される) 
\\	(搭載される) 
\\	細かいことでいがみ合うな。	
\\	細[こま]かいことでいがみ 合[あ]うな。	細かい=こまかい= 
\\	(詳しい) 
\\	(精密な) 
\\	(厳密な) 
\\	そんな細かいことはどうでもいい。	
\\	そんな 細[こま]かいことはどうでもいい。	細かい=こまかい= 
\\	(詳しい) 
\\	(精密な) 
\\	(厳密な) 
\\	彼女の仕事ぶりはとても細かい。	
\\	彼女[かのじょ]の 仕事[しごと]ぶりはとても 細[こま]かい。	細かい=こまかい= 
\\	(詳しい) 
\\	(精密な) 
\\	(厳密な) 
\\	あの人は細かいところに気がつく。	
\\	あの 人[ひと]は 細[こま]かいところに 気[き]がつく。	細かい=こまかい= 
\\	(詳しい) 
\\	(精密な) 
\\	(厳密な) 
\\	彼は細かいことにうるさい。	
\\	彼[かれ]は 細[こま]かいことにうるさい。	細かい=こまかい= 
\\	(詳しい) 
\\	(精密な) 
\\	(厳密な) 
\\	細かいことは後で決めよう。	
\\	細[こま]かいことは 後[あと]で 決[き]めよう。	細かい=こまかい= 
\\	(詳しい) 
\\	(精密な) 
\\	(厳密な) 
\\	細かい持ち合わせがない。	
\\	細[こま]かい 持ち合[もちあ]わせがない。	細かい=こまかい= 
\\	(詳しい) 
\\	(精密な) 
\\	(厳密な) 
\\	細かい字を読む時は眼鏡が必要だ。	
\\	細[こま]かい 字[じ]を 読[よ]む 時[とき]は 眼鏡[めがね]が 必要[ひつよう]だ。	細かい=こまかい= 
\\	(詳しい) 
\\	(精密な) 
\\	(厳密な) 
\\	この説明書きは活字が細かい。	
\\	この 説明[せつめい] 書[が]きは 活字[かつじ]が 細[こま]かい。	細かい=こまかい= 
\\	(詳しい) 
\\	(精密な) 
\\	(厳密な) 
\\	この一万円札を細かくしてくれませんか。	
\\	この 一万[いちまん] 円[えん] 札[さつ]を 細[こま]かくしてくれませんか。	細かい=こまかい= 
\\	(詳しい) 
\\	(精密な) 
\\	(厳密な) 
\\	今は細かい話は抜きにしよう。	
\\	今[いま]は 細[こま]かい 話[はなし]は 抜[ぬ]きにしよう。	細かい=こまかい= 
\\	(詳しい) 
\\	(精密な) 
\\	(厳密な) 
\\	君は細かいことを気にしすぎだ。	
\\	君[きみ]は 細[こま]かいことを 気[き]にしすぎだ。	細かい=こまかい= 
\\	(詳しい) 
\\	(精密な) 
\\	(厳密な) 
\\	相次ぐ不運で家運はみるみる傾いた。	
\\	相次[あいつ]ぐ 不運[ふうん]で 家運[かうん]はみるみる 傾[かたむ]いた。	傾く=かたむく= 
\\	突然機体が大きく傾いたので、コーヒーがこぼれた。	
\\	突然[とつぜん] 機体[きたい]が 大[おお]きく 傾[かたむ]いたので、コーヒーがこぼれた。	機体=きたい= 飛行機の本体。翼以外、またはエンジン以外の部分。また、飛行機の全体。 傾く=かたむく= 
\\	古い小屋は右に傾いていた。	
\\	古[ふる]い 小屋[こや]は 右[みぎ]に 傾[かたむ]いていた。	傾く=かたむく= 
\\	その土地は海の方に傾いている。	
\\	その 土地[とち]は 海[うみ]の 方[ほう]に 傾[かたむ]いている。	傾く=かたむく= 
\\	その柱は左に傾いている。	
\\	その 柱[はしら]は 左[ひだり]に 傾[かたむ]いている。	傾く=かたむく= 
\\	国の財政が傾いている。	
\\	国[くに]の 財政[ざいせい]が 傾[かたむ]いている。	傾く=かたむく= 
\\	彼女はつんつんしていた。	
\\	彼女[かのじょ]はつんつんしていた。	
\\	丘はゆるやかに南に向かって傾斜している。	
\\	丘[おか]はゆるやかに 南[みなみ]に 向[む]かって 傾斜[けいしゃ]している。	傾斜=けいしゃ= 
\\	(坂などの) 
\\	(〜する) 
\\	(上へ) 
\\	(下へ) 
\\	陸は海へと傾斜している。	
\\	陸[りく]は 海[うみ]へと 傾斜[けいしゃ]している。	傾斜=けいしゃ= 
\\	(坂などの) 
\\	(〜する) 
\\	(上へ) 
\\	(下へ) 
\\	彼はその仕事に全力を傾注した。	
\\	彼[かれ]はその 仕事[しごと]に 全力[ぜんりょく]を 傾注[けいちゅう]した。	傾注=けいちゅう= 精神や力を一つのことに集中すること。
\\	彼女は村上春樹に傾倒している。	
\\	彼女[かのじょ]は 村上[むらかみ] 春樹[はるき]に 傾倒[けいとう]している。	傾倒=けいとう= ある物事に深く心を引かれ、夢中になること。また、ある人を心から尊敬し、慕うこと。
\\	彼女はジャズに傾倒している。	
\\	彼女[かのじょ]はジャズに 傾倒[けいとう]している。	傾倒=けいとう= ある物事に深く心を引かれ、夢中になること。また、ある人を心から尊敬し、慕うこと。
\\	平均株価は5000円台を往来している。	
\\	平均[へいきん] 株価[かぶか]は5000 円[えん] 台[だい]を 往来[おうらい]している。	往来=おうらい= 
\\	(道路) 
\\	(相場などの小幅な値動き) 
\\	家は往来から少し引っ込んでいる。	
\\	家[いえ]は 往来[おうらい]から 少[すこ]し 引っ込[ひっこ]んでいる。	往来=おうらい= 
\\	(道路) 
\\	(相場などの小幅な値動き) 
\\	雨で往来が途絶えた。	
\\	雨[あめ]で 往来[おうらい]が 途絶[とだ]えた。	往来=おうらい= 
\\	(道路) 
\\	(相場などの小幅な値動き) 
\\	両国間の物資の往来が再開した。	
\\	両国[りょうこく] 間[かん]の 物資[ぶっし]の 往来[おうらい]が 再開[さいかい]した。	往来=おうらい= 
\\	(道路) 
\\	(相場などの小幅な値動き) 
\\	手術後体調が悪くてぶらぶらしている。	
\\	手術[しゅじゅつ] 後[ご] 体調[たいちょう]が 悪[わる]くてぶらぶらしている。	ぶらぶら= (揺れ動くさま) 
\\	(あてどないさま); (無為なさま) 
\\	公園をぶらぶら歩きましょう。	
\\	公園[こうえん]をぶらぶら 歩[ある]きましょう。	ぶらぶら= (揺れ動くさま) 
\\	(あてどないさま); (無為なさま) 
\\	途中でぶらぶらしていたから遅くなってしまった。	
\\	途中[とちゅう]でぶらぶらしていたから 遅[おそ]くなってしまった。	ぶらぶら= (揺れ動くさま) 
\\	(あてどないさま); (無為なさま) 
\\	提灯が風でぶらぶらしている。	
\\	提灯[ちょうちん]が 風[かぜ]でぶらぶらしている。	提灯=ちょうちん= 
\\	ぶらぶら= (揺れ動くさま) 
\\	(あてどないさま); (無為なさま) 
\\	彼は椅子に座って足をぶらぶらさせていた。	
\\	彼[かれ]は 椅子[いす]に 座[すわ]って 足[あし]をぶらぶらさせていた。	ぶらぶら= (揺れ動くさま) 
\\	(あてどないさま); (無為なさま) 
\\	甘い言葉に誘われてふらふらっとついていってしまった。	
\\	甘[あま]い 言葉[ことば]に 誘[さそ]われてふらふらっとついていってしまった。	ふらふら= (物や身体が安定しないさま) 
\\	(心や態度が安定しないさま) 
\\	(無自覚に行動するさま) 
\\	私はふらふらとその店に入っていった。	
\\	私[わたし]はふらふらとその 店[みせ]に 入[はい]っていった。	ふらふら= (物や身体が安定しないさま) 
\\	(心や態度が安定しないさま) 
\\	(無自覚に行動するさま) 
\\	定職にも就かずふらふらしている若者が増えた。	
\\	定職[ていしょく]にも 就[つ]かずふらふらしている 若者[わかもの]が 増[ふ]えた。	ふらふら= (物や身体が安定しないさま) 
\\	(心や態度が安定しないさま) 
\\	(無自覚に行動するさま) 
\\	彼はあっちにふらふらこっちにふらふらぶつかりながら歩いていった。	
\\	彼[かれ]はあっちにふらふらこっちにふらふらぶつかりながら 歩[ある]いていった。	ふらふら= (物や身体が安定しないさま) 
\\	(心や態度が安定しないさま) 
\\	(無自覚に行動するさま) 
\\	彼は足がふらふらする。	
\\	彼[かれ]は 足[あし]がふらふらする。	ふらふら= (物や身体が安定しないさま) 
\\	(心や態度が安定しないさま) 
\\	(無自覚に行動するさま) 
\\	私は頭がふらふらする。	
\\	私[わたし]は 頭[あたま]がふらふらする。	ふらふら= (物や身体が安定しないさま) 
\\	(心や態度が安定しないさま) 
\\	(無自覚に行動するさま) 
\\	彼女は空腹でふらふらだった。	
\\	彼女[かのじょ]は 空腹[くうふく]でふらふらだった。	ふらふら= (物や身体が安定しないさま) 
\\	(心や態度が安定しないさま) 
\\	(無自覚に行動するさま) 
\\	彼女はまだ決心がつかなくてふらふらしている。	
\\	彼女[かのじょ]はまだ 決心[けっしん]がつかなくてふらふらしている。	ふらふら= (物や身体が安定しないさま) 
\\	(心や態度が安定しないさま) 
\\	(無自覚に行動するさま) 
\\	その方針はふらふらしている。	
\\	その 方針[ほうしん]はふらふらしている。	ふらふら= (物や身体が安定しないさま) 
\\	(心や態度が安定しないさま) 
\\	(無自覚に行動するさま) 
\\	ご飯がふっくらと炊きあがった。	
\\	ご 飯[はん]がふっくらと 炊[た]きあがった。	
\\	列車は吹雪のため立ち往生した。	
\\	列車[れっしゃ]は 吹雪[ふぶき]のため 立ち往生[たちおうじょう]した。	立ち往生=たちおうじょう= (立ったままで死ぬこと) 
\\	(動けなくなるさま) 
\\	(当惑し何もできなくなるさま) 
\\	停電で電車が何台も立ち往生していた。	
\\	停電[ていでん]で 電車[でんしゃ]が 何[なん] 台[だい]も 立ち往生[たちおうじょう]していた。	立ち往生=たちおうじょう= (立ったままで死ぬこと) 
\\	(動けなくなるさま) 
\\	(当惑し何もできなくなるさま) 
\\	シロナガスクジラは体長が往々にして30メートルにもなる。	
\\	シロナガスクジラは 体長[たいちょう]が 往々[おうおう]にして30メートルにもなる。	往々=おうおう= 
\\	学者は往々にして世間を知らない。	
\\	学者[がくしゃ]は 往々[おうおう]にして 世間[せけん]を 知[し]らない。	往々=おうおう= 
\\	こういう間違いは往々にして起こるものだ。	
\\	こういう 間違[まちが]いは 往々[おうおう]にして 起[お]こるものだ。	往々=おうおう= 
\\	映画と原作の小説がかなり違うということは往々にしてある。	
\\	映画[えいが]と 原作[げんさく]の 小説[しょうせつ]がかなり 違[ちが]うということは 往々[おうおう]にしてある。	往々=おうおう= 
\\	修学旅行生たちは、その寺を探して右往左往していた。	
\\	修学旅行[しゅうがくりょこう] 生[せい]たちは、その 寺[てら]を 探[さが]して 右往左往[うおうさおう]していた。	右往左往=うおうさおう= (〜する) 
\\	クリスマスツリーに電飾がぴかぴか光っている。	
\\	クリスマスツリーに 電飾[でんしょく]がぴかぴか 光[ひか]っている。	電飾=でんしょく= イルミネーションぴかぴか=(〜する)ー 光る
\\	稲妻がぴかぴか光った。	
\\	稲妻[いなづま]がぴかぴか 光[ひか]った。	ぴかぴか=(〜する)ー 光る
\\	父に買ってもらったばかりの自転車はぴかぴかに光っていた。	
\\	父[ちち]に 買[か]ってもらったばかりの 自転車[じてんしゃ]はぴかぴかに 光[ひか]っていた。	ぴかぴか=(〜する)ー 光る
\\	彼女は靴がぴかぴかになるまで磨いた。	
\\	彼女[かのじょ]は 靴[くつ]がぴかぴかになるまで 磨[みが]いた。	ぴかぴか=(〜する)ー 光る
\\	その日はだれにも会いたくなかったので居留守を使った。	
\\	その 日[ひ]はだれにも 会[あ]いたくなかったので 居留守[いるす]を 使[つか]った。	居留守を使う=いるすをつかう= 
\\	その知らせを聞いてどきっとした。	
\\	その 知[し]らせを 聞[き]いてどきっとした。	どきっと= (〜する) <驚いて> 
\\	<ときめいて> 
\\	彼の目を見た瞬間どきっとした。	
\\	彼[かれ]の 目[め]を 見[み]た 瞬間[しゅんかん]どきっとした。	どきっと= (〜する) <驚いて> 
\\	<ときめいて> 
\\	彼女の名前が言われただけでどきっとする。	
\\	彼女[かのじょ]の 名前[なまえ]が 言[い]われただけでどきっとする。	どきっと= (〜する) <驚いて> 
\\	<ときめいて> 
\\	彼女の言葉には時々どきっとさせられる。	
\\	彼女[かのじょ]の 言葉[ことば]には 時々[ときどき]どきっとさせられる。	どきっと= (〜する) <驚いて> 
\\	<ときめいて> 
\\	大きな音にどきっとした。	
\\	大[おお]きな 音[おと]にどきっとした。	どきっと= (〜する) <驚いて> 
\\	<ときめいて> 
\\	往路は船だった。	
\\	往路[おうろ]は 船[ふね]だった。	往路=おうろ= 
\\	往路は新幹線を使い、復路は飛行機にした。	
\\	往路[おうろ]は 新幹線[しんかんせん]を 使[つか]い、 復路[ふくろ]は 飛行機[ひこうき]にした。	往路=おうろ= 
\\	彼女は往年の名女優である。	
\\	彼女[かのじょ]は 往年[おうねん]の 名[めい] 女優[じょゆう]である。	往年=おうねん= 
\\	彼女の歌声には往年の輝きがない。	
\\	彼女[かのじょ]の 歌声[うたごえ]には 往年[おうねん]の 輝[かがや]きがない。	往年=おうねん= 
\\	あいつのしつこさには往生したものだった。	
\\	あいつのしつこさには 往生[おうじょう]したものだった。	往生する=おうじょうする= <仏教> 
\\	(死ぬこと) 
\\	(抵抗をあきらめること) 
\\	(閉口する) 
\\	(困る) 
\\	あの時は彼女に泣かれて私はほとほと往生した。	
\\	あの 時[とき]は 彼女[かのじょ]に 泣[な]かれて 私[わたし]はほとほと 往生[おうじょう]した。	往生する=おうじょうする= <仏教> 
\\	(死ぬこと) 
\\	(抵抗をあきらめること) 
\\	(閉口する) 
\\	(困る) 
\\	私は議長就任を渋る彼を説きつけてやっと往生させた。	
\\	私[わたし]は 議長[ぎちょう] 就任[しゅうにん]を 渋[しぶ]る 彼[かれ]を 説[と]きつけてやっと 往生[おうじょう]させた。	往生する=おうじょうする= <仏教> 
\\	(死ぬこと) 
\\	(抵抗をあきらめること) 
\\	(閉口する) 
\\	(困る) 
\\	高速道路で車が故障して往生した。	
\\	高速[こうそく] 道路[どうろ]で 車[くるま]が 故障[こしょう]して 往生[おうじょう]した。	往生する=おうじょうする= <仏教> 
\\	(死ぬこと) 
\\	(抵抗をあきらめること) 
\\	(閉口する) 
\\	(困る) 
\\	この窓ガラスは内側が汚れているんだ。	
\\	この 窓[まど]ガラスは 内側[うちがわ]が 汚[よご]れているんだ。	内側=うちがわ= 
\\	袋の内側が濡れていた。	
\\	袋[ふくろ]の 内側[うちがわ]が 濡[ぬ]れていた。	内側=うちがわ= 
\\	上着の内側にポケットがある。	
\\	上着[うわぎ]の 内側[うちがわ]にポケットがある。	内側=うちがわ= 
\\	靴下をひっくり返して内側を外にして履いていた。	
\\	靴下[くつした]をひっくり 返[かえ]して 内側[うちがわ]を 外[そと]にして 履[は]いていた。	内側=うちがわ= 
\\	この箱は外側は金色に、内側は漆黒に塗ってある。	
\\	この 箱[はこ]は 外側[そとがわ]は 金色[きんいろ]に、 内側[うちがわ]は 漆黒[しっこく]に 塗[ぬ]ってある。	内側=うちがわ= 
\\	私は山手線の内側に住んでいる。	
\\	私[わたし]は 山手[やまのて] 線[せん]の 内側[うちがわ]に 住[す]んでいる。	内側=うちがわ= 
\\	その箱は外側が白く塗ってある。	
\\	その 箱[はこ]は 外側[そとがわ]が 白[しろ]く 塗[ぬ]ってある。	外側=そとがわ= 
\\	冷蔵庫をあさって一人で一杯やっていたら、妻が帰ってきた。	
\\	冷蔵庫[れいぞうこ]をあさって一 人[にん]で一 杯[はい]やっていたら、 妻[つま]が 帰[かえ]ってきた。	漁る=あさる= 
\\	掘り出し物を捜して古本屋をあさるのも楽しいものだ。	
\\	掘り出し物[ほりだしもの]を 捜[さが]して 古本屋[ふるほんや]をあさるのも 楽[たの]しいものだ。	掘り出し物=ほりだしもの= 思いがけなく手に入った珍しいもの。また、思いがけなく安い値段で手に入れたもの。漁る=あさる= 
\\	カラスがゴミをあさりに来るので困っている。	
\\	カラスがゴミをあさりに 来[く]るので 困[こま]っている。	漁る=あさる= 
\\	野良猫はゴミをあさって生きている。	
\\	野良猫[のらねこ]はゴミをあさって 生[い]きている。	野良猫=のらねこ= 飼い主のない猫。 
\\	漁る=あさる= 
\\	旧制度が復活した。	
\\	旧[きゅう] 制度[せいど]が 復活[ふっかつ]した。	復活=ふっかつ= (再生・蘇生) 
\\	(再興) 
\\	町は洪水から立派に復活した。	
\\	町[まち]は 洪水[こうずい]から 立派[りっぱ]に 復活[ふっかつ]した。	復活=ふっかつ= (再生・蘇生) 
\\	(再興) 
\\	詩に対する興味が復活してきた。	
\\	詩[し]に 対[たい]する 興味[きょうみ]が 復活[ふっかつ]してきた。	復活=ふっかつ= (再生・蘇生) 
\\	(再興) 
\\	その選手はけがから見事に復活した。	
\\	その 選手[せんしゅ]はけがから 見事[みごと]に 復活[ふっかつ]した。	復活=ふっかつ= (再生・蘇生) 
\\	(再興) 
\\	この物語にははらはらさせる場面がたくさんある。	
\\	この 物語[ものがたり]にははらはらさせる 場面[ばめん]がたくさんある。	はらはら= (散り落ちるさま); (危ぶむさま) (〜する) 
\\	彼女のほおを涙がはらはらと落ちた。	
\\	彼女[かのじょ]のほおを 涙[なみだ]がはらはらと 落[お]ちた。	はらはら= (散り落ちるさま); (危ぶむさま) (〜する) 
\\	彼の車の運転ははらはらするよ。	
\\	彼[かれ]の 車[くるま]の 運転[うんてん]ははらはらするよ。	はらはら= (散り落ちるさま); (危ぶむさま) (〜する) 
\\	卒業したら皆ちりぢりばらばらになるんだね。	
\\	卒業[そつぎょう]したら 皆[かい]ちりぢりばらばらになるんだね。	ばらばら= (たくさんのものが勢いよく落ちたり散らばるさま) 
\\	(雨などが) 
\\	(それぞれに分かれるさま) 
\\	(それぞれ違うさま) 
\\	彼は組合がばらばらにならぬよう最大限の努力をした。	
\\	彼[かれ]は 組合[くみあい]がばらばらにならぬよう 最大限[さいだいげん]の 努力[どりょく]をした。	ばらばら= (たくさんのものが勢いよく落ちたり散らばるさま) 
\\	(雨などが) 
\\	(それぞれに分かれるさま) 
\\	(それぞれ違うさま) 
\\	乱暴に運んできたが、ケーキはばらばらにはなっていなかった。	
\\	乱暴[らんぼう]に 運[はこ]んできたが、ケーキはばらばらにはなっていなかった。	ばらばら= (たくさんのものが勢いよく落ちたり散らばるさま) 
\\	(雨などが) 
\\	(それぞれに分かれるさま) 
\\	(それぞれ違うさま) 
\\	ドラマーは手と足とをばらばらに動かさなければならない。	
\\	ドラマーは 手[て]と 足[あし]とをばらばらに 動[うご]かさなければならない。	ばらばら= (たくさんのものが勢いよく落ちたり散らばるさま) 
\\	(雨などが) 
\\	(それぞれに分かれるさま) 
\\	(それぞれ違うさま) 
\\	彼は金を家の中のあちこちにばらばらに隠した。	
\\	彼[かれ]は 金[きん]を 家[いえ]の 中[なか]のあちこちにばらばらに 隠[かく]した。	ばらばら= (たくさんのものが勢いよく落ちたり散らばるさま) 
\\	(雨などが) 
\\	(それぞれに分かれるさま) 
\\	(それぞれ違うさま) 
\\	行きは一緒で帰りはばらばらに帰った。	
\\	行[い]きは 一緒[いっしょ]で 帰[かえ]りはばらばらに 帰[かえ]った。	ばらばら= (たくさんのものが勢いよく落ちたり散らばるさま) 
\\	(雨などが) 
\\	(それぞれに分かれるさま) 
\\	(それぞれ違うさま) 
\\	警官がばらばらと駆けつけて来た。	
\\	警官[けいかん]がばらばらと 駆[か]けつけて 来[き]た。	ばらばら= (たくさんのものが勢いよく落ちたり散らばるさま) 
\\	(雨などが) 
\\	(それぞれに分かれるさま) 
\\	(それぞれ違うさま) 
\\	うちの家族の帰宅時間はばらばらだ。	
\\	うちの 家族[かぞく]の 帰宅[きたく] 時間[じかん]はばらばらだ。	ばらばら= (たくさんのものが勢いよく落ちたり散らばるさま) 
\\	(雨などが) 
\\	(それぞれに分かれるさま) 
\\	(それぞれ違うさま) 
\\	彼らは言うことがでんでんばらばらだ。	
\\	彼[かれ]らは 言[い]うことがでんでんばらばらだ。	ばらばら= (たくさんのものが勢いよく落ちたり散らばるさま) 
\\	(雨などが) 
\\	(それぞれに分かれるさま) 
\\	(それぞれ違うさま) 
\\	我が校からの合格者はほんの数人、ぱらぱらという状態だ。	
\\	我[わ]が 校[こう]からの 合格[ごうかく] 者[しゃ]はほんの 数[すう] 人[にん]、ぱらぱらという 状態[じょうたい]だ。	ぱらぱら= (まばらに落ちるさま); (点在するさま); (本などをすばやくめくるさま) 
\\	(水分や粘り気がなく粒がくっつかないさま)
\\	そこは草がぱらぱらと生えているだけの荒れ地だった。	
\\	そこは 草[くさ]がぱらぱらと 生[は]えているだけの 荒れ地[あれち]だった。	ぱらぱら= (まばらに落ちるさま); (点在するさま); (本などをすばやくめくるさま) 
\\	(水分や粘り気がなく粒がくっつかないさま)
\\	雨がぱらぱらと屋根に当たる音が聞こえる。	
\\	雨[あめ]がぱらぱらと 屋根[やね]に 当[あ]たる 音[おと]が 聞[き]こえる。	ぱらぱら= (まばらに落ちるさま); (点在するさま); (本などをすばやくめくるさま) 
\\	(水分や粘り気がなく粒がくっつかないさま)
\\	あられが窓にぱらぱらと当たる音が聞こえる。	
\\	あられが 窓[まど]にぱらぱらと 当[あ]たる 音[おと]が 聞[き]こえる。	ぱらぱら= (まばらに落ちるさま); (点在するさま); (本などをすばやくめくるさま) 
\\	(水分や粘り気がなく粒がくっつかないさま)
\\	雨がぱらぱらと降ってきた。	
\\	雨[あめ]がぱらぱらと 降[ふ]ってきた。	ぱらぱら= (まばらに落ちるさま); (点在するさま); (本などをすばやくめくるさま) 
\\	(水分や粘り気がなく粒がくっつかないさま)
\\	その電車には乗客がぱらぱらとしか乗っていなかった。	
\\	その 電車[でんしゃ]には 乗客[じょうきゃく]がぱらぱらとしか 乗[の]っていなかった。	ぱらぱら= (まばらに落ちるさま); (点在するさま); (本などをすばやくめくるさま) 
\\	(水分や粘り気がなく粒がくっつかないさま)
\\	この店のチャーハンはぱらぱらしていておいしい。	
\\	この 店[みせ]のチャーハンはぱらぱらしていておいしい。	ぱらぱら= (まばらに落ちるさま); (点在するさま); (本などをすばやくめくるさま) 
\\	(水分や粘り気がなく粒がくっつかないさま)
\\	このところしけ続きで漁がない。	
\\	このところしけ 続[つづ]きで 漁[りょう]がない。	時化=しけ= 
\\	漁=りょう= 
\\	(旅漁) 
\\	(漁獲) 
\\	このしけでは漁ができない。	
\\	このしけでは 漁[りょう]ができない。	時化=しけ= 
\\	漁=りょう= 
\\	(旅漁) 
\\	(漁獲) 
\\	この沖ではニシンの漁獲が多い。	
\\	この 沖[おき]ではニシンの 漁獲[ぎょかく]が 多[おお]い。	漁獲=ぎょかく= (水産物をとること) 
\\	2000年の世界のマグロの漁獲量は約200万トンだった。	
\\	年[ねん]の 世界[せかい]のマグロの 漁獲[ぎょかく] 量[りょう]は 約[やく]200 万[まん]トンだった。	漁獲=ぎょかく= (水産物をとること) 
\\	猫に小判。	
\\	猫[ねこ]に 小判[こばん]。	
\\	花より団子	
\\	花[はな]より 団子[だんご]	
\\	見ぬが花	
\\	見[み]ぬが 花[はな]	
\\	知らぬが仏	
\\	知[し]らぬが 仏[ほとけ]	
\\	案ずるより産むが易し	
\\	案[あん]ずるより 産[う]むが 易[やす]し	
\\	出る杭は打たれる。	
\\	出[で]る 杭[くい]は 打[う]たれる。	
\\	1万円のドレスを衝動的に買ってしまった。	
\\	1万[いちまん] 円[えん]のドレスを 衝動[しょうどう] 的[てき]に 買[か]ってしまった。	衝動=しょうどう= 
\\	それでは究極的な解決にならない。	
\\	それでは 究極[きゅうきょく] 的[てき]な 解決[かいけつ]にならない。	
\\	彼はテレビに釘付けだった。	
\\	彼[かれ]はテレビに 釘付[くぎづ]けだった。	釘付け=くぎづけ= (釘を打ち付けること) 
\\	(値段などの変動のないこと)
\\	(動きのとれない状態) 
\\	全員の視線がテレビ画面に釘付けになった。	
\\	全員[ぜんいん]の 視線[しせん]がテレビ 画面[がめん]に 釘付[くぎづ]けになった。	釘付け=くぎづけ= (釘を打ち付けること) 
\\	(値段などの変動のないこと)
\\	(動きのとれない状態) 
\\	驚きのあまりその場に釘付けになった。	
\\	驚[おどろ]きのあまりその 場[ば]に 釘付[くぎづ]けになった。	釘付け=くぎづけ= (釘を打ち付けること) 
\\	(値段などの変動のないこと)
\\	(動きのとれない状態) 
\\	相場は釘付け状態である。	
\\	相場[そうば]は 釘付[くぎづ]け 状態[じょうたい]である。	釘付け=くぎづけ= (釘を打ち付けること) 
\\	(値段などの変動のないこと)
\\	(動きのとれない状態) 
\\	その光景を見て私はその場に釘付けになった。	
\\	その 光景[こうけい]を 見[み]て 私[わたし]はその 場[ば]に 釘付[くぎづ]けになった。	釘付け=くぎづけ= (釘を打ち付けること) 
\\	(値段などの変動のないこと)
\\	(動きのとれない状態) 
\\	一口には言えない経緯がある。	
\\	一口[ひとくち]には 言[い]えない 経緯[いきさつ]がある。	経緯=いきさつ= (事情) 
\\	(過程) 
\\	彼女は退職までの細かい経緯を述べた。	
\\	彼女[かのじょ]は 退職[たいしょく]までの 細[こま]かい 経緯[いきさつ]を 述[の]べた。	経緯=いきさつ= (事情) 
\\	(過程) 
\\	事件発生の経緯を解明してほしい。	
\\	事件[じけん] 発生[はっせい]の 経緯[いきさつ]を 解明[かいめい]してほしい。	経緯=いきさつ= (事情) 
\\	(過程) 
\\	公にはできない経緯もあった。	
\\	公[おおやけ]にはできない 経緯[いきさつ]もあった。	経緯=いきさつ= (事情) 
\\	(過程) 
\\	どういう経緯で口論になったのですか。	
\\	どういう 経緯[いきさつ]で 口論[こうろん]になったのですか。	経緯=いきさつ= (事情) 
\\	(過程) 
\\	彼は秘書として誠に申し分がない。	
\\	彼[かれ]は 秘書[ひしょ]として 誠[まこと]に 申し分[もうしぶん]がない。	申し分=もうしぶん= 
\\	これは申し分がない、誠に結構だ。	
\\	これは 申し分[もうしぶん]がない、 誠[まこと]に 結構[けっこう]だ。	申し分=もうしぶん= 
\\	あの人ならクラブの会長として申し分がない。	
\\	あの 人[ひと]ならクラブの 会長[かいちょう]として 申し分[もうしぶん]がない。	申し分=もうしぶん= 
\\	この部屋は広いから集会所として申し分がない。	
\\	この 部屋[へや]は 広[ひろ]いから 集会[しゅうかい] 所[しょ]として 申し分[もうしぶん]がない。	申し分=もうしぶん= 
\\	山登りには申し分のない天気だ。	
\\	山登[やまのぼ]りには 申し分[もうしぶん]のない 天気[てんき]だ。	申し分=もうしぶん= 
\\	彼女の料理の腕は申し分ない。	
\\	彼女[かのじょ]の 料理[りょうり]の 腕[うで]は 申し分[もうしぶん]ない。	申し分=もうしぶん= 
\\	ばっちりだったよ。	
\\	ばっちりだったよ。	
\\	これで今日の試験はばっちりだ。	
\\	これで 今日[きょう]の 試験[しけん]はばっちりだ。	
\\	あなたの主張はわが国の将来を誤るものだ。	
\\	あなたの 主張[しゅちょう]はわが 国[くに]の 将来[しょうらい]を 誤[あやま]るものだ。	誤る=あやまる= (しそこなう) 
\\	(悪いほうを選ぶ) 
\\	私は選択を誤らなかった。	
\\	私[わたし]は 選択[せんたく]を 誤[あやま]らなかった。	誤る=あやまる= (しそこなう) 
\\	(悪いほうを選ぶ) 
\\	この写真は日本について誤った印象を与えかねない。	
\\	この 写真[しゃしん]は 日本[にほん]について 誤[あやま]った 印象[いんしょう]を 与[あた]えかねない。	誤る=あやまる= (しそこなう) 
\\	(悪いほうを選ぶ) 
\\	それは明らかに誤っている。	
\\	それは 明[あき]らかに 誤[あやま]っている。	誤る=あやまる= (しそこなう) 
\\	(悪いほうを選ぶ) 
\\	君の認識はまったく誤っている。	
\\	君[きみ]の 認識[にんしき]はまったく 誤[あやま]っている。	誤る=あやまる= (しそこなう) 
\\	(悪いほうを選ぶ) 
\\	子供が誤って硬貨を飲み込んでしまった。	
\\	子供[こども]が 誤[あやま]って 硬貨[こうか]を 飲み込[のみこ]んでしまった。	誤る=あやまる= (しそこなう) 
\\	(悪いほうを選ぶ) 
\\	そのメールを誤って山田さんに送ってしまった。	
\\	そのメールを 誤[あやま]って 山田[やまだ]さんに 送[おく]ってしまった。	誤る=あやまる= (しそこなう) 
\\	(悪いほうを選ぶ) 
\\	誤って花びんを割ってしまった。	
\\	誤[あやま]って 花[か]びんを 割[わ]ってしまった。	誤る=あやまる= (しそこなう) 
\\	(悪いほうを選ぶ) 
\\	は誤差の範囲内だ。	
\\	[ぱーせんと]は 誤差[ごさ]の 範囲[はんい] 内[ない]だ。	誤差=ごさ= (数・統計) 
\\	彼女の優勝は監督にとってうれしい誤算だった。	
\\	彼女[かのじょ]の 優勝[ゆうしょう]は 監督[かんとく]にとってうれしい 誤算[ごさん]だった。	誤算=ごさん= (計算違い) 
\\	(見込み違い) 
\\	あの事件が株式市況に及ぼす影響に関して投資者側に誤算があった。	
\\	あの 事件[じけん]が 株式[かぶしき] 市況[しきょう]に 及[およ]ぼす 影響[えいきょう]に 関[かん]して 投資[とうし] 者[しゃ] 側[がわ]に 誤算[ごさん]があった。	誤算=ごさん= (計算違い) 
\\	(見込み違い) 
\\	彼女が賛成するだろうと思ったのは彼の誤算だった。	
\\	彼女[かのじょ]が 賛成[さんせい]するだろうと 思[おも]ったのは 彼[かれ]の 誤算[ごさん]だった。	誤算=ごさん= (計算違い) 
\\	(見込み違い) 
\\	警察は彼を誤認逮捕した。	
\\	警察[けいさつ]は 彼[かれ]を 誤認[ごにん] 逮捕[たいほ]した。	誤認=ごにん= 
\\	その新聞は時々誤報がある。	
\\	その 新聞[しんぶん]は 時々[ときどき] 誤報[ごほう]がある。	誤報=ごほう= (報告・情報の) 
\\	(警報の) 
\\	それは誤報だった。	
\\	それは 誤報[ごほう]だった。	誤報=ごほう= (報告・情報の) 
\\	(警報の) 
\\	さっき鳴った火災報知器のベルは誤報だった。	
\\	さっき 鳴[な]った 火災報知器[かさいほうちき]のベルは 誤報[ごほう]だった。	誤報=ごほう= (報告・情報の) 
\\	(警報の) 
\\	彼が死んだというのは誤報だった。	
\\	彼[かれ]が 死[し]んだというのは 誤報[ごほう]だった。	誤報=ごほう= (報告・情報の) 
\\	(警報の) 
\\	彼のレポートには誤字が多い。	
\\	彼[かれ]のレポートには 誤字[ごじ]が 多[おお]い。	誤字=ごじ= 
\\	この慣用句はよく誤用される。	
\\	この 慣用[かんよう] 句[く]はよく 誤用[ごよう]される。	誤用=ごよう= 
\\	ここでは政治の話は慎んでください。	
\\	ここでは 政治[せいじ]の 話[はなし]は 慎[つつし]んでください。	慎む=つつしむ= (自重する) 
\\	(節制する) 
\\	二度とこのようなことのないよう慎みます。	
\\	二度[にど]とこのようなことのないよう 慎[つつし]みます。	慎む=つつしむ= (自重する) 
\\	(節制する) 
\\	言葉を慎みなさい。	
\\	言葉[ことば]を 慎[つつし]みなさい。	慎む=つつしむ= (自重する) 
\\	(節制する) 
\\	身を慎みなさい。	
\\	身[み]を 慎[つつし]みなさい。	慎む=つつしむ= (自重する) 
\\	(節制する) 
\\	取り扱いは慎重を要する。	
\\	取り扱[とりあつか]いは 慎重[しんちょう]を 要[よう]する。	慎重な=しんちょうな= 
\\	目上の人と話すときには、言葉選びに慎重さが求められる。	
\\	目上[めうえ]の 人[ひと]と 話[はな]すときには、 言葉[ことば] 選[えら]びに 慎重[しんちょう]さが 求[もと]められる。	慎重な=しんちょうな= 
\\	「考えておきます」と彼女は慎重に答えた。	
\\	考[かんが]えておきます」と 彼女[かのじょ]は 慎重[しんちょう]に 答[こた]えた。	慎重な=しんちょうな= 
\\	自宅謹慎を命じられた。	
\\	自宅[じたく] 謹慎[きんしん]を 命[めい]じられた。	謹慎=きんしん= 
\\	彼女は1週間の自宅謹慎を命じられた。	
\\	彼女[かのじょ]は1 週間[しゅうかん]の 自宅[じたく] 謹慎[きんしん]を 命[めい]じられた。	謹慎=きんしん= 
\\	あの夫婦はまだごたごたしている。	
\\	あの 夫婦[ふうふ]はまだごたごたしている。	ごたごた= (もめごと) 
\\	(〜する) (もめる) 
\\	隣人との間にごたごたが起きている。	
\\	隣人[りんじん]との 間[あいだ]にごたごたが 起[お]きている。	ごたごた= (もめごと) 
\\	(〜する) (もめる) 
\\	両国間にごたごたが起きかかっている。	
\\	両国[りょうこく] 間[かん]にごたごたが 起[お]きかかっている。	ごたごた= (もめごと) 
\\	(〜する) (もめる) 
\\	長く続いたごたごたがようやく治まった。	
\\	長[なが]く 続[つづ]いたごたごたがようやく 治[おさ]まった。	ごたごた= (もめごと) 
\\	(〜する) (もめる) 
\\	両親が離婚した場合、その子供はいわば被害者である。	
\\	両親[りょうしん]が 離婚[りこん]した 場合[ばあい]、その 子供[こども]はいわば 被害[ひがい] 者[しゃ]である。	言わば=いわば= 
\\	彼はいわばその業界の帝王のような存在だ。	
\\	彼[かれ]はいわばその 業界[ぎょうかい]の 帝王[ていおう]のような 存在[そんざい]だ。	言わば=いわば= 
\\	この国はいわば私の第二の祖国だ。	
\\	この 国[くに]はいわば 私[わたし]の 第二[だいに]の 祖国[そこく]だ。	言わば=いわば= 
\\	最近発覚した収賄はいわば氷山の一角でしかない。	
\\	最近[さいきん] 発覚[はっかく]した 収賄[しゅうわい]はいわば 氷山[ひょうざん]の 一角[いっかく]でしかない。	言わば=いわば= 
\\	このいすはパパが日曜大工で作ってくれたの。	
\\	このいすはパパが 日曜[にちよう] 大工[だいく]で 作[つく]ってくれたの。	日曜大工=にちようだいく= 
\\	肝心なことは機会を逃さないことだ。	
\\	肝心[かんじん]なことは 機会[きかい]を 逃[のが]さないことだ。	肝心な・肝腎な=かんじんな= 
\\	いつも肝心な時にあいつはいなくなる。	
\\	いつも 肝心[かんじん]な 時[とき]にあいつはいなくなる。	肝心な・肝腎な=かんじんな= 
\\	そこが肝心なところだ。	
\\	そこが 肝心[かんじん]なところだ。	肝心な・肝腎な=かんじんな= 
\\	彼の話は肝心なところが抜けている。	
\\	彼[かれ]の 話[はなし]は 肝心[かんじん]なところが 抜[ぬ]けている。	肝心な・肝腎な=かんじんな= 
\\	人生なにごとも辛抱が肝心だ。	
\\	人生[じんせい]なにごとも 辛抱[しんぼう]が 肝心[かんじん]だ。	肝心な・肝腎な=かんじんな= 
\\	肝心なのはすぐにそれを実行することだ。	
\\	肝心[かんじん]なのはすぐにそれを 実行[じっこう]することだ。	肝心な・肝腎な=かんじんな= 
\\	労働争議はいよいよ先鋭化してきた。	
\\	労働[ろうどう] 争議[そうぎ]はいよいよ 先鋭[せんえい] 化[か]してきた。	先鋭化=せんえいか= 
\\	憂慮すべき事態が発生している。	
\\	憂慮[ゆうりょ]すべき 事態[じたい]が 発生[はっせい]している。	憂慮=ゆうりょ= 
\\	このままオゾン層が破壊され続けるとこの先どうなるかまことに憂慮にたえない。	
\\	このままオゾン 層[そう]が 破壊[はかい]され 続[つづ]けるとこの 先[さき]どうなるかまことに 憂慮[ゆうりょ]にたえない。	憂慮=ゆうりょ= 
\\	多くの人々が国の将来を憂慮している。	
\\	多[おお]くの 人々[ひとびと]が 国[くに]の 将来[しょうらい]を 憂慮[ゆうりょ]している。	憂慮=ゆうりょ= 
\\	その事態をだれも正しく把握していなかった。	
\\	その 事態[じたい]をだれも 正[まさ]しく 把握[はあく]していなかった。	把握=はあく= 
\\	事態の正確な把握には時間がかかりそうだ。	
\\	事態[じたい]の 正確[せいかく]な 把握[はあく]には 時間[じかん]がかかりそうだ。	把握=はあく= 
\\	親たちは子供たちの状況を十分把握していなかった。	
\\	親[おや]たちは 子供[こども]たちの 状況[じょうきょう]を 十分[じゅうぶん] 把握[はあく]していなかった。	把握=はあく= 
\\	彼は会議の参加人数を把握していなかった。	
\\	彼[かれ]は 会議[かいぎ]の 参加[さんか] 人数[にんずう]を 把握[はあく]していなかった。	把握=はあく= 
\\	別に悪意があって言ったわけではない。	
\\	別[べつ]に 悪意[あくい]があって 言[い]ったわけではない。	悪意=あくい= (悪気) 
\\	(悪い意味) 
\\	悪意があってしたことではない。	
\\	悪意[あくい]があってしたことではない。	悪意=あくい= (悪気) 
\\	(悪い意味) 
\\	彼の行為には悪意が感じられる。	
\\	彼[かれ]の 行為[こうい]には 悪意[あくい]が 感[かん]じられる。	悪意=あくい= (悪気) 
\\	(悪い意味) 
\\	彼は私に対して悪意を抱いている。	
\\	彼[かれ]は 私[わたし]に 対[たい]して 悪意[あくい]を 抱[だ]いている。	悪意=あくい= (悪気) 
\\	(悪い意味) 
\\	私は彼には何の悪意も抱いていない。	
\\	私[わたし]は 彼[かれ]には 何[なん]の 悪意[あくい]も 抱[だ]いていない。	悪意=あくい= (悪気) 
\\	(悪い意味) 
\\	両親を安心させるため悪意のないうそをついた。	
\\	両親[りょうしん]を 安心[あんしん]させるため 悪意[あくい]のないうそをついた。	悪意=あくい= (悪気) 
\\	(悪い意味) 
\\	猫はネズミを餌食にした。	
\\	猫[ねこ]はネズミを 餌食[えじき]にした。	餌食=えじき= (餌として食われるもの) 
\\	(肉食動物などの) 
\\	彼女は暴漢の餌食になった。	
\\	彼女[かのじょ]は 暴漢[ぼうかん]の 餌食[えじき]になった。	餌食=えじき= (餌として食われるもの) 
\\	(肉食動物などの) 
\\	あの会社は大企業の餌食にされたのだ。	
\\	あの 会社[かいしゃ]は 大[だい] 企業[きぎょう]の 餌食[えじき]にされたのだ。	餌食=えじき= (餌として食われるもの) 
\\	(肉食動物などの) 
\\	彼らは悪徳商法の餌食になった。	
\\	彼[かれ]らは 悪徳[あくとく] 商法[しょうほう]の 餌食[えじき]になった。	餌食=えじき= (餌として食われるもの) 
\\	(肉食動物などの) 
\\	人の来る気配がした。	
\\	人[ひと]の 来[く]る 気配[けはい]がした。	気配=けはい= (何となく感じられるようす) 
\\	人の気配に、犬たちは一斉に吠え立てた。	
\\	人[ひと]の 気配[けはい]に、 犬[いぬ]たちは 一斉[いっせい]に 吠[ほ]え 立[た]てた。	気配=けはい= (何となく感じられるようす) 
\\	異様な気配に目を覚ますと、すでに室内に煙が充満していた。	
\\	異様[いよう]な 気配[けはい]に 目[め]を 覚[さ]ますと、すでに 室内[しつない]に 煙[けむり]が 充満[じゅうまん]していた。	気配=けはい= (何となく感じられるようす) 
\\	この試合で日本は復活の気配を見せた。	
\\	この 試合[しあい]で 日本[にっぽん]は 復活[ふっかつ]の 気配[けはい]を 見[み]せた。	気配=けはい= (何となく感じられるようす) 
\\	その人気は一向に衰える気配を見せていない。	
\\	その 人気[にんき]は 一向[いっこう]に 衰[おとろ]える 気配[けはい]を 見[み]せていない。	気配=けはい= (何となく感じられるようす) 
\\	背後に人の気配を感じて振り返った。	
\\	背後[はいご]に 人[ひと]の 気配[けはい]を 感[かん]じて 振り返[ふりかえ]った。	気配=けはい= (何となく感じられるようす) 
\\	その家に人の住んでいる気配はなかった。	
\\	その 家[いえ]に 人[ひと]の 住[す]んでいる 気配[けはい]はなかった。	気配=けはい= (何となく感じられるようす) 
\\	この街の犯罪件数は一向に減る気配もない。	
\\	この 街[まち]の 犯罪[はんざい] 件数[けんすう]は 一向[いっこう]に 減[へ]る 気配[けはい]もない。	気配=けはい= (何となく感じられるようす) 
\\	彼がその試合に勝ったとき新聞は彼のことで持ち切りだった。	
\\	彼[かれ]がその 試合[しあい]に 勝[か]ったとき 新聞[しんぶん]は 彼[かれ]のことで 持ち切[もちき]りだった。	
\\	社内は田中さんの受賞の話で持ち切りだ。	
\\	社内[しゃない]は 田中[たなか]さんの 受賞[じゅしょう]の 話[はなし]で 持ち切[もちき]りだ。	
\\	(お見合いで)顔立ちが気に入った。	
\\	(お 見合[みあ]いで) 顔立[かおだ]ちが 気に入[きにい]った	顔立ち=かおだち= 
\\	彼も中学生になる頃には大人の顔立ちになっていた。	
\\	彼[かれ]も 中学生[ちゅうがくせい]になる 頃[ころ]には 大人[おとな]の 顔立[かおだ]ちになっていた。	顔立ち=かおだち= 
\\	彼女は日本人形のような顔立ちの美人です。	
\\	彼女[かのじょ]は 日本[にほん] 人形[にんぎょう]のような 顔立[かおだ]ちの 美人[びじん]です。	顔立ち=かおだち= 
\\	彼はなかなか理知的な顔立ちをしている。	
\\	彼[かれ]はなかなか 理知的[りちてき]な 顔立[かおだ]ちをしている。	顔立ち=かおだち= 
\\	彼は鼻が高くて西洋人みたいな顔立ちだ。	
\\	彼[かれ]は 鼻[はな]が 高[たか]くて 西洋[せいよう] 人[じん]みたいな 顔立[かおだ]ちだ。	顔立ち=かおだち= 
\\	彼ははっきりした顔立ちをしている。	
\\	彼[かれ]ははっきりした 顔立[かおだ]ちをしている。	顔立ち=かおだち= 
\\	その法律は即日施行された。	
\\	その 法律[ほうりつ]は 即日[そくじつ] 施行[しこう]された。	即日=そくじつ= 
\\	従業員が一人即日解雇を言い渡された。	
\\	従業[じゅうぎょう] 員[いん]が一 人[にん] 即日[そくじつ] 解雇[かいこ]を 言い渡[いいわた]された。	即日=そくじつ= 
\\	午前中の注文は即日発送いたします。	
\\	午前[ごぜん] 中[ちゅう]の 注文[ちゅうもん]は 即日[そくじつ] 発送[はっそう]いたします。	即日=そくじつ= 
\\	この計画は現実に即していない。	
\\	この 計画[けいかく]は 現実[げんじつ]に 即[そく]していない。	即する=そくする= (ぴったり合う) 
\\	(基づく) 
\\	この小説はあくまで事実に即して書かれている。	
\\	この 小説[しょうせつ]はあくまで 事実[じじつ]に 即[そく]して 書[か]かれている。	即する=そくする= (ぴったり合う) 
\\	(基づく) 
\\	この問題は即時解決を要する。	
\\	この 問題[もんだい]は 即時[そくじ] 解決[かいけつ]を 要[よう]する。	即時=そくじ= 
\\	10名が即死した。	
\\	名[めい]が 即死[そくし]した。	即死=そくし= 
\\	彼は即死だった。	
\\	彼[かれ]は 即死[そくし]だった。	即死=そくし= 
\\	体育館に即席の舞台が作られた。	
\\	体育館[たいいくかん]に 即席[そくせき]の 舞台[ぶたい]が 作[つく]られた。	即席の=そくせきの= 
\\	両国間の関係は一触即発の状態だった。	
\\	両国[りょうこく] 間[かん]の 関係[かんけい]は 一触即発[いっしょくそくはつ]の 状態[じょうたい]だった。	一触即発=いっしょくそくはつ=ちょっと触れればすぐに爆発しそうなこと。危機に直面していること。危機一髪。
\\	その地域は一触即発の危機をはらんでいる。	
\\	その 地域[ちいき]は 一触即発[いっしょくそくはつ]の 危機[きき]をはらんでいる。	一触即発=いっしょくそくはつ=ちょっと触れればすぐに爆発しそうなこと。危機に直面していること。危機一髪。
\\	あの二人は今一触即発の険悪な状態だ。	
\\	あの二 人[にん]は 今[こん] 一触即発[いっしょくそくはつ]の 険悪[けんあく]な 状態[じょうたい]だ。	一触即発=いっしょくそくはつ=ちょっと触れればすぐに爆発しそうなこと。危機に直面していること。危機一髪。
\\	彼らは予算の削減に頭を悩ました。	
\\	彼[かれ]らは 予算[よさん]の 削減[さくげん]に 頭[あたま]を 悩[なや]ました。	悩ます=なやます= (悩ませる) 
\\	(負担をかける) 
\\	(困らせる) 
\\	子供のころはよく両親を悩ませたものだ。	
\\	子供[こども]のころはよく 両親[りょうしん]を 悩[なや]ませたものだ。	悩ます・悩ませる=なやます・なやませる= 
\\	(負担をかける) 
\\	(困らせる) 
\\	おぞましい記憶に悩ませられている。	
\\	おぞましい 記憶[きおく]に 悩[なや]ませられている。	悩ます・悩ませる=なやます・なやませる= 
\\	(負担をかける) 
\\	(困らせる) 
\\	家庭内のごたごたが彼を悩ましている。	
\\	家庭[かてい] 内[ない]のごたごたが 彼[かれ]を 悩[なや]ましている。	悩ます・悩ませる=なやます・なやませる= 
\\	(負担をかける) 
\\	(困らせる) 
\\	彼女は家計のやりくりに頭を悩ませている。	
\\	彼女[かのじょ]は 家計[かけい]のやりくりに 頭[あたま]を 悩[なや]ませている。	悩ます・悩ませる=なやます・なやませる= 
\\	(負担をかける) 
\\	(困らせる) 
\\	地元住民は工場の騒音に悩まされている。	
\\	地元[じもと] 住民[じゅうみん]は 工場[こうじょう]の 騒音[そうおん]に 悩[なや]まされている。	悩ます・悩ませる=なやます・なやませる= 
\\	(負担をかける) 
\\	(困らせる) 
\\	その問題の解決に頭を悩ませた。	
\\	その 問題[もんだい]の 解決[かいけつ]に 頭[あたま]を 悩[なや]ませた。	悩ます・悩ませる=なやます・なやませる= 
\\	(負担をかける) 
\\	(困らせる) 
\\	彼は非常な苦悩を味わった。	
\\	彼[かれ]は 非常[ひじょう]な 苦悩[くのう]を 味[あじ]わった。	苦悩=くのう= 
\\	苦悩の色が彼の顔に表れた。	
\\	苦悩[くのう]の 色[いろ]が 彼[かれ]の 顔[かお]に 表[あらわ]れた。	苦悩=くのう= 
\\	ドイツ軍のポーランド侵攻が第二次世界大戦の発端となった。	
\\	ドイツ 軍[ぐん]のポーランド 侵攻[しんこう]が 第[だい]二 次[じ] 世界[せかい] 大戦[たいせん]の 発端[ほったん]となった。	発端=ほったん= (起こり) 
\\	この事件の発端は何だったのか。	
\\	この 事件[じけん]の 発端[ほったん]は 何[なに]だったのか。	発端=ほったん= (起こり) 
\\	その会社の脱税が発覚した。	
\\	その 会社[かいしゃ]の 脱税[だつぜい]が 発覚[はっかく]した。	発覚=はっかく= 隠していた悪事・陰謀などが明るみに出ること。
\\	その歌手の麻薬使用が発覚した。	
\\	その 歌手[かしゅ]の 麻薬[まやく] 使用[しよう]が 発覚[はっかく]した。	発覚=はっかく= 隠していた悪事・陰謀などが明るみに出ること。
\\	大統領の暗殺計画が発覚した。	
\\	大統領[だいとうりょう]の 暗殺[あんさつ] 計画[けいかく]が 発覚[はっかく]した。	発覚=はっかく= 隠していた悪事・陰謀などが明るみに出ること。
\\	右の一番端に写っているのが私です。	
\\	右[みぎ]の 一番[いちばん] 端[はし]に 写[うつ]っているのが 私[わたし]です。	端=はし= (末端) 
\\	(縁) 
\\	(片端) 
\\	(順序のはじめ) 
\\	(重要でない部分) 
\\	(切り離した部分) 
\\	彼はその通りの端から端まで往復した。	
\\	彼[かれ]はその 通[とお]りの 端[はし]から 端[はし]まで 往復[おうふく]した。	端=はし= (末端) 
\\	(縁) 
\\	(片端) 
\\	(順序のはじめ) 
\\	(重要でない部分) 
\\	(切り離した部分) 
\\	空港はその島の北の端にある。	
\\	空港[くうこう]はその 島[しま]の 北[きた]の 端[はし]にある。	端=はし= (末端) 
\\	(縁) 
\\	(片端) 
\\	(順序のはじめ) 
\\	(重要でない部分) 
\\	(切り離した部分) 
\\	バスが道の端に寄って停車した。	
\\	バスが 道[みち]の 端[はし]に 寄[よ]って 停車[ていしゃ]した。	端=はし= (末端) 
\\	(縁) 
\\	(片端) 
\\	(順序のはじめ) 
\\	(重要でない部分) 
\\	(切り離した部分) 
\\	新しい言葉は聞いた端から忘れてしまう。	
\\	新[あたら]しい 言葉[ことば]は 聞[き]いた 端[はし]から 忘[わす]れてしまう。	端=はし= (末端) 
\\	(縁) 
\\	(片端) 
\\	(順序のはじめ) 
\\	(重要でない部分) 
\\	(切り離した部分) 
\\	見たとたんに彼だとわかった。	
\\	見[み]たとたんに 彼[かれ]だとわかった。	
\\	名前を変えたとたんにその商品は売れ出した。	
\\	名前[なまえ]を 変[か]えたとたんにその 商品[しょうひん]は 売[う]れ 出[だ]した。	
\\	その話を持ち出したらとたんに彼女の機嫌が悪くなった。	
\\	その 話[はなし]を 持ち出[もちだ]したらとたんに 彼女[かのじょ]の 機嫌[きげん]が 悪[わる]くなった。	
\\	彼女が現れた途端割れるような拍手が起こった。	
\\	彼女[かのじょ]が 現[あらわ]れた 途端[とたん] 割[わ]れるような 拍手[はくしゅ]が 起[お]こった。	
\\	その覚醒剤は末端価格で1グラム3万円する。	
\\	その 覚醒剤[かくせいざい]は 末端[まったん] 価格[かかく]で1グラム3 万[まん] 円[えん]する。	末端価格=まったんかかく= 
\\	(麻薬などの) 
\\	事件解決の鍵を握っているのは彼だ。	
\\	事件[じけん] 解決[かいけつ]の 鍵[かぎ]を 握[にぎ]っているのは 彼[かれ]だ。	握る=にぎる= (つかむ) 
\\	(握って作る); (権力・金・情報・証拠などを手に入れる、つかむ) 
\\	わが家の財布を握っているのは女房だ。	
\\	わが 家[や]の 財布[さいふ]を 握[にぎ]っているのは 女房[にょうぼう]だ。	握る=にぎる= (つかむ) 
\\	(握って作る); (権力・金・情報・証拠などを手に入れる、つかむ) 
\\	実力は彼が握っている。	
\\	実力[じつりょく]は 彼[かれ]が 握[にぎ]っている。	握る=にぎる= (つかむ) 
\\	(握って作る); (権力・金・情報・証拠などを手に入れる、つかむ) 
\\	彼はあの取引で相当の金を握った。	
\\	彼[かれ]はあの 取引[とりひき]で 相当[そうとう]の 金[きん]を 握[にぎ]った。	握る=にぎる= (つかむ) 
\\	(握って作る); (権力・金・情報・証拠などを手に入れる、つかむ) 
\\	僕はまだ彼女の手を握ったことすらない。	
\\	僕[ぼく]はまだ 彼女[かのじょ]の 手[て]を 握[にぎ]ったことすらない。	握る=にぎる= (つかむ) 
\\	(握って作る); (権力・金・情報・証拠などを手に入れる、つかむ) 
\\	何を握っているか当ててごらん。	
\\	何[なに]を 握[にぎ]っているか 当[あ]ててごらん。	握る=にぎる= (つかむ) 
\\	(握って作る); (権力・金・情報・証拠などを手に入れる、つかむ) 
\\	力をこめて彼女は私の手を握り返した。	
\\	力[ちから]をこめて 彼女[かのじょ]は 私[わたし]の 手[て]を 握[にぎ]り 返[かえ]した。	握る=にぎる= (つかむ) 
\\	(握って作る); (権力・金・情報・証拠などを手に入れる、つかむ) 
\\	彼は私の手をギュッと握った。	
\\	彼[かれ]は 私[わたし]の 手[て]をギュッと 握[にぎ]った。	握る=にぎる= (つかむ) 
\\	(握って作る); (権力・金・情報・証拠などを手に入れる、つかむ) 
\\	その子は母親のそでを握っていた。	
\\	その 子[こ]は 母親[ははおや]のそでを 握[にぎ]っていた。	握る=にぎる= (つかむ) 
\\	(握って作る); (権力・金・情報・証拠などを手に入れる、つかむ) 
\\	私は彼に弱みを握られている。	
\\	私[わたし]は 彼[かれ]に 弱[よわ]みを 握[にぎ]られている。	握る=にぎる= (つかむ) 
\\	(握って作る); (権力・金・情報・証拠などを手に入れる、つかむ) 
\\	彼は私の手をギュッと握り締めた。	
\\	彼[かれ]は 私[わたし]の 手[て]をギュッと 握り締[にぎりし]めた。	握り締める=にぎりしめる= 
\\	彼は私の手を握り締めて離さなかった。	
\\	彼[かれ]は 私[わたし]の 手[て]を 握り締[にぎりし]めて 離[はな]さなかった。	握り締める=にぎりしめる= 
\\	彼は刀のつかを握り締めた。	
\\	彼[かれ]は 刀[かたな]のつかを 握り締[にぎりし]めた。	握り締める=にぎりしめる= 
\\	歴史に名をとどめるのはほんの一握りの人間だけである。	
\\	歴史[れきし]に 名[な]をとどめるのはほんの 一握[ひとにぎ]りの 人間[にんげん]だけである。	一握り=ひとにぎり= 
\\	(わずかな数量)
\\	この減税で恩恵を受けるのは一握りの人間だけだ。	
\\	この 減税[げんぜい]で 恩恵[おんけい]を 受[う]けるのは 一握[ひとにぎ]りの 人間[にんげん]だけだ。	一握り=ひとにぎり= 
\\	(わずかな数量)
\\	日本チームは最終戦にも勝って有終の美を飾った。	
\\	日本[にほん]チームは 最終[さいしゅう] 戦[せん]にも 勝[か]って 有終の美[ゆうしゅうのび]を 飾[かざ]った。	有終の美=ゆうしゅうのび= 
\\	彼は若い女性にもてもてだ。	
\\	彼[かれ]は 若[わか]い 女性[じょせい]にもてもてだ。	
\\	彼はクラスの女子にもてもてだ。	
\\	彼[かれ]はクラスの 女子[じょし]にもてもてだ。	
\\	彼女は美しくて愛嬌があるので、どこのパーティーでも大もてに持てる。	
\\	彼女[かのじょ]は 美[うつく]しくて 愛嬌[あいきょう]があるので、どこのパーティーでも 大[おお]もてに 持[も]てる。	大持て=おおもて= (大いに人気がある) 
\\	(歓迎される) 
\\	その歌手は日本中どこへ行っても大もてだ。	
\\	その 歌手[かしゅ]は 日本[にほん] 中[じゅう]どこへ 行[い]っても 大[おお]もてだ。	大持て=おおもて= (大いに人気がある) 
\\	(歓迎される) 
\\	日本製のビデオカメラはヨーロッパでも大もてだ。	
\\	日本[にっぽん] 製[せい]のビデオカメラはヨーロッパでも 大[おお]もてだ。	大持て=おおもて= (大いに人気がある) 
\\	(歓迎される) 
\\	謹んで新年のご挨拶を申し上げます。	
\\	謹[つつし]んで 新年[しんねん]のご 挨拶[あいさつ]を 申し上[もうしあ]げます。	謹んで=つつしんで= 
\\	謹んで哀悼の意を表します。	
\\	謹[つつし]んで 哀悼[あいとう]の 意[い]を 表[あらわ]します。	謹んで=つつしんで= 
\\	謹んでお悔やみ申し上げます。	
\\	謹[つつし]んでお 悔[く]やみ 申し上[もうしあ]げます。	謹んで=つつしんで= 
\\	謹んで申し出をお受けいたします。	
\\	謹[つつし]んで 申し出[もうしで]をお 受[う]けいたします。	謹んで=つつしんで= 
\\	世間ではとやかく言うが、彼女はいい人だった。	
\\	世間[せけん]ではとやかく 言[い]うが、 彼女[かのじょ]はいい 人[ひと]だった。	とやかく= 
\\	すんでしまったことをとやかく言ってみても始まらない。	
\\	すんでしまったことをとやかく 言[い]ってみても 始[はじ]まらない。	とやかく言う= 
\\	そんなことをするとかえってインフレを助長することになる。	
\\	そんなことをするとかえってインフレを 助長[じょちょう]することになる。	助長=じょちょう= 
\\	この本からは有益な示唆を得た。	
\\	この 本[ほん]からは 有益[ゆうえき]な 示唆[しさ]を 得[え]た。	示唆=しさ= 
\\	(〜する) 
\\	市長の発言は辞任の可能性を示唆している。	
\\	市長[しちょう]の 発言[はつげん]は 辞任[じにん]の 可能[かのう] 性[せい]を 示唆[しさ]している。	示唆=しさ= 
\\	(〜する) 
\\	この小国は今、国際政治の中心へと変貌を遂げつつある。	
\\	この 小国[しょうこく]は 今[いま]、 国際[こくさい] 政治[せいじ]の 中心[ちゅうしん]へと 変貌[へんぼう]を 遂[と]げつつある。	変貌=へんぼう= 
\\	この国は目覚ましい変貌を遂げつつある。	
\\	この 国[くに]は 目覚[めざ]ましい 変貌[へんぼう]を 遂[と]げつつある。	変貌=へんぼう= 
\\	女性はいまだに昇進などで差別待遇を受けている。	
\\	女性[じょせい]はいまだに 昇進[しょうしん]などで 差別[さべつ] 待遇[たいぐう]を 受[う]けている。	待遇=たいぐう= (取り扱い) 
\\	(処遇) 
\\	(接待) 
\\	待遇に不満がある。	
\\	待遇[たいぐう]に 不満[ふまん]がある。	待遇=たいぐう= (取り扱い) 
\\	(処遇) 
\\	(接待) 
\\	この会社は待遇が悪い。	
\\	この 会社[かいしゃ]は 待遇[たいぐう]が 悪[わる]い。	待遇=たいぐう= (取り扱い) 
\\	(処遇) 
\\	(接待) 
\\	あんなにひどい待遇を受けようとは思いもよらなかった。	
\\	あんなにひどい 待遇[たいぐう]を 受[う]けようとは 思[おも]いもよらなかった。	待遇=たいぐう= (取り扱い) 
\\	(処遇) 
\\	(接待) 
\\	使節団は冷ややかな待遇を受けた。	
\\	使節[しせつ] 団[だん]は 冷[ひ]ややかな 待遇[たいぐう]を 受[う]けた。	待遇=たいぐう= (取り扱い) 
\\	(処遇) 
\\	(接待) 
\\	あのホテルは待遇がとてもよかった。	
\\	あのホテルは 待遇[たいぐう]がとてもよかった。	待遇=たいぐう= (取り扱い) 
\\	(処遇) 
\\	(接待) 
\\	女性は男性に比べると握力がない。	
\\	女性[じょせい]は 男性[だんせい]に 比[くら]べると 握力[あくりょく]がない。	握力=あくりょく= 
\\	彼の背後には大物政治家が控えている。	
\\	彼[かれ]の 背後[はいご]には 大物[おおもの] 政治[せいじ] 家[か]が 控[ひか]えている。	控える=ひかえる= (見合わせる) 
\\	(制限する) 
\\	(少なめにする) (発言を) 
\\	(メモする) 
\\	(予定を目前にしている) 
\\	(待機する) 
\\	秘書は社長の後ろに控えていた。	
\\	秘書[ひしょ]は 社長[しゃちょう]の 後[うし]ろに 控[ひか]えていた。	控える=ひかえる= (見合わせる) 
\\	(制限する) 
\\	(少なめにする) (発言を) 
\\	(メモする) 
\\	(予定を目前にしている) 
\\	(待機する) 
\\	妻は出産をじきに控えている。	
\\	妻[つま]は 出産[しゅっさん]をじきに 控[ひか]えている。	控える=ひかえる= (見合わせる) 
\\	(制限する) 
\\	(少なめにする) (発言を) 
\\	(メモする) 
\\	(予定を目前にしている) 
\\	(待機する) 
\\	投票を明日に控えて候補者は有権者に最後の訴えをしている。	
\\	投票[とうひょう]を 明日[あした]に 控[ひか]えて 候補[こうほ] 者[しゃ]は 有権者[ゆうけんしゃ]に 最後[さいご]の 訴[うった]えをしている。	控える=ひかえる= (見合わせる) 
\\	(制限する) 
\\	(少なめにする) (発言を) 
\\	(メモする) 
\\	(予定を目前にしている) 
\\	(待機する) 
\\	友達の誕生日は全て手帳に控えてある。	
\\	友達[ともだち]の 誕生[たんじょう] 日[び]は 全[すべ]て 手帳[てちょう]に 控[ひか]えてある。	控える=ひかえる= (見合わせる) 
\\	(制限する) 
\\	(少なめにする) (発言を) 
\\	(メモする) 
\\	(予定を目前にしている) 
\\	(待機する) 
\\	年長者に対してそのような口のきき方は控えなさい。	
\\	年長[ねんちょう] 者[しゃ]に 対[たい]してそのような 口[くち]のきき 方[かた]は 控[ひか]えなさい。	控える=ひかえる= (見合わせる) 
\\	(制限する) 
\\	(少なめにする) (発言を) 
\\	(メモする) 
\\	(予定を目前にしている) 
\\	(待機する) 
\\	双方の言い分を聞くまで判断を控えます。	
\\	双方[そうほう]の 言い分[いいぶん]を 聞[き]くまで 判断[はんだん]を 控[ひか]えます。	控える=ひかえる= (見合わせる) 
\\	(制限する) 
\\	(少なめにする) (発言を) 
\\	(メモする) 
\\	(予定を目前にしている) 
\\	(待機する) 
\\	現在はその地域への旅行は控えるように外務省は進めている。	
\\	現在[げんざい]はその 地域[ちいき]への 旅行[りょこう]は 控[ひか]えるように 外務省[がいむしょう]は 進[すす]めている。	控える=ひかえる= (見合わせる) 
\\	(制限する) 
\\	(少なめにする) (発言を) 
\\	(メモする) 
\\	(予定を目前にしている) 
\\	(待機する) 
\\	この件に関するコメントは控えさせていただきます。	
\\	この 件[けん]に 関[かん]するコメントは 控[ひか]えさせていただきます。	控える=ひかえる= (見合わせる) 
\\	(制限する) 
\\	(少なめにする) (発言を) 
\\	(メモする) 
\\	(予定を目前にしている) 
\\	(待機する) 
\\	彼女は秋に結婚を控えている。	
\\	彼女[かのじょ]は 秋[あき]に 結婚[けっこん]を 控[ひか]えている。	控える=ひかえる= (見合わせる) 
\\	(制限する) 
\\	(少なめにする) (発言を) 
\\	(メモする) 
\\	(予定を目前にしている) 
\\	(待機する) 
\\	国民が消費を手控える傾向が続いている。	
\\	国民[こくみん]が 消費[しょうひ]を 手控[てびか]える 傾向[けいこう]が 続[つづ]いている。	手控える=てびかえる= (メモする) 
\\	(予備に取っておく) 
\\	(控え目にする)
\\	太陽がさんさんと輝いている。	
\\	太陽[たいよう]がさんさんと 輝[かがや]いている。	輝く=かがやく= 
\\	彼女の顔は喜びに輝いていた。	
\\	彼女[かのじょ]の 顔[かお]は 喜[よろこ]びに 輝[かがや]いていた。	輝く=かがやく= 
\\	それを聞いて彼女の顔はぱっと輝いた。	
\\	それを 聞[き]いて 彼女[かのじょ]の 顔[かお]はぱっと 輝[かがや]いた。	輝く=かがやく= 
\\	トラはらんらんと輝く目で鹿を見据えた。	
\\	トラはらんらんと 輝[かがや]く 目[め]で 鹿[しか]を 見据[みす]えた。	輝く=かがやく= 
\\	星が空一面に輝いている。	
\\	星[ほし]が 空[そら] 一面[いちめん]に 輝[かがや]いている。	輝く=かがやく= 
\\	彼女の指にはダイヤモンドの指輪が輝いている。	
\\	彼女[かのじょ]の 指[ゆび]にはダイヤモンドの 指輪[ゆびわ]が 輝[かがや]いている。	輝く=かがやく= 
\\	彼は今最も輝いているタレントの一人だ。	
\\	彼[かれ]は 今[いま] 最[もっと]も 輝[かがや]いているタレントの一 人[にん]だ。	輝く=かがやく= 
\\	月が明るく輝いていた。	
\\	月[つき]が 明[あか]るく 輝[かがや]いていた。	輝く=かがやく= 
\\	彼の目は興奮で輝いていた。	
\\	彼[かれ]の 目[め]は 興奮[こうふん]で 輝[かがや]いていた。	輝く=かがやく= 
\\	彼女の顔は希望に輝いている。	
\\	彼女[かのじょ]の 顔[かお]は 希望[きぼう]に 輝[かがや]いている。	輝く=かがやく= 
\\	子供たちの目は光り輝いていた。	
\\	子供[こども]たちの 目[め]は 光り輝[ひかりかがや]いていた。	光り輝く=ひかりかがやく= 
\\	(輝く)
\\	われわれの事業は好成績をあげています。	
\\	われわれの 事業[じぎょう]は 好[こう] 成績[せいせき]をあげています。	
\\	今度の実験の成功は彼女に取って大変な業績になる。	
\\	今度[こんど]の 実験[じっけん]の 成功[せいこう]は 彼女[かのじょ]に 取[と]って 大変[たいへん]な 業績[ぎょうせき]になる。	業績=ぎょうせき= (個人の) 
\\	(会社などの) 
\\	当社の業績は順調に伸びている。	
\\	当社[とうしゃ]の 業績[ぎょうせき]は 順調[じゅんちょう]に 伸[の]びている。	業績=ぎょうせき= (個人の) 
\\	(会社などの) 
\\	繊維業界はここ数年業績不振である。	
\\	繊維[せんい] 業界[ぎょうかい]はここ 数[すう] 年[ねん] 業績[ぎょうせき] 不振[ふしん]である。	業績=ぎょうせき= (個人の) 
\\	(会社などの) 
\\	(釣りで)何か大物がかかったぞ。	
\\	釣[つ]りで) 何[なに]か 大物[おおもの]がかかったぞ。	大物=おおもの= (実力者), (有力者) 
\\	あの選手は磨けばきっと大物になる。	
\\	あの 選手[せんしゅ]は 磨[みが]けばきっと 大物[おおもの]になる。	大物=おおもの= (実力者), (有力者) 
\\	オーディションの応募者はどんぐりの背比べだった。	
\\	オーディションの 応募[おうぼ] 者[しゃ]はどんぐりの 背[せい] 比[くら]べだった。	どんぐりの背比べ=どんぐりのせいくらべ= 
\\	最近のパソコンは、機能面ではどんぐりの背比べだ。	
\\	最近[さいきん]のパソコンは、 機能[きのう] 面[めん]ではどんぐりの 背[せい] 比[くら]べだ。	どんぐりの背比べ=どんぐりのせいくらべ= 
\\	彼らはどんぐりの背比べだ。	
\\	彼[かれ]らはどんぐりの 背[せい] 比[くら]べだ。	どんぐりの背比べ=どんぐりのせいくらべ= 
\\	そこは芋を洗うような混雑だった。	
\\	そこは 芋[いも]を 洗[あら]うような 混雑[こんざつ]だった。	芋を洗うような雑踏=いもをあらうようなざっとう= 
\\	プールはイモを洗うような込み方だった。	
\\	プールはイモを 洗[あら]うような 込[こ]み 方[かた]だった。	芋を洗うような雑踏=いもをあらうようなざっとう= 
\\	そんな帽子のかぶり方をするとイモにいちゃんみたいだぞ。	
\\	そんな 帽子[ぼうし]のかぶり 方[かた]をするとイモにいちゃんみたいだぞ。	イモにいちゃん・ねえちゃん= 
\\	ドン・ホセを演じた俳優、イモだな。	
\\	ドン・ホセを 演[えん]じた 俳優[はいゆう]、イモだな。	イモ= 
\\	(やぼったい人) 
\\	あいつはイモだからこういう服は似合わないだろう。	
\\	あいつはイモだからこういう 服[ふく]は 似合[にあ]わないだろう。	イモ= 
\\	(やぼったい人) 
\\	その話なら耳にたこができるほど聞かされた。	
\\	その 話[はなし]なら 耳[みみ]にたこができるほど 聞[き]かされた。	耳にたこができる= 
\\	良心なんてものは爪の垢ほども持ち合わせていない。	
\\	良心[りょうしん]なんてものは 爪[つめ]の 垢[あか]ほども 持ち合[もちあ]わせていない。	爪の垢=つめのあか= 
\\	爪の垢ほどもない= 
\\	彼には他人を気遣う気持ちが爪の垢ほどもない。	
\\	彼[かれ]には 他人[たにん]を 気遣[きづか]う 気持[きも]ちが 爪[つめ]の 垢[あか]ほどもない。	爪の垢=つめのあか= 
\\	爪の垢ほどもない= 
\\	少しは師匠の爪の垢でも煎じて飲んだらどうだ。	
\\	少[すこ]しは 師匠[ししょう]の 爪[つめ]の 垢[あか]でも 煎[せん]じて 飲[の]んだらどうだ。	爪の垢=つめのあか= 
\\	爪の垢を煎じて飲む=つめのあかをせんじてのむ= 
\\	彼女は毎日2時間ピアノの練習をする。少しは彼女の爪の垢でも煎じて飲みなさい。	
\\	彼女[かのじょ]は 毎日[まいにち]2 時間[じかん]ピアノの 練習[れんしゅう]をする。 少[すこ]しは 彼女[かのじょ]の 爪[つめ]の 垢[あか]でも 煎[せん]じて 飲[の]みなさい。	爪の垢=つめのあか= 
\\	爪の垢を煎じて飲む=つめのあかをせんじてのむ= 
\\	それは蛇足だ。	
\\	それは 蛇足[だそく]だ。	蛇足=だそく= 
\\	蛇足ながらもう一言。	
\\	蛇足[だそく]ながらもう 一言[ひとこと]。	蛇足=だそく= 
\\	客の応対に追われて猫の手も借りたいほど忙しかった。	
\\	客[きゃく]の 応対[おうたい]に 追[お]われて 猫[ねこ]の 手[て]も 借[か]りたいほど 忙[いそが]しかった。	猫の手も借りたい= 
\\	彼は拾った財布をねこばばした。	
\\	彼[かれ]は 拾[ひろ]った 財布[さいふ]をねこばばした。	猫ばばする=ねこばばする= 
\\	あいつは猫をかぶっているが実は大変な陰謀家なんだ。	
\\	あいつは 猫[ねこ]をかぶっているが 実[じつ]は 大変[たいへん]な 陰謀[いんぼう] 家[か]なんだ。	猫をかぶる= 
\\	彼女は上司の前では猫をかぶっている。	
\\	彼女[かのじょ]は 上司[じょうし]の 前[まえ]では 猫[ねこ]をかぶっている。	猫をかぶる= 
\\	6つの町村を併せて1つの市にした。	
\\	6[むっ]つの 町村[ちょうそん]を 併[あわ]せて 1[ひと]つの 市[し]にした。	併せる=あわせる= 
\\	彼女はその政治運動の推移に関心を持った。	
\\	彼女[かのじょ]はその 政治[せいじ] 運動[うんどう]の 推移[すいい]に 関心[かんしん]を 持[も]った。	推移=すいい= 
\\	(〜する) 
\\	株価は8000円前後で推移している。	
\\	株価[かぶか]は8000 円[えん] 前後[ぜんご]で 推移[すいい]している。	推移=すいい= 
\\	(〜する) 
\\	情勢が目まぐるしく推移した。	
\\	情勢[じょうせい]が 目[め]まぐるしく 推移[すいい]した。	推移=すいい= 
\\	(〜する) 
\\	両会社は合併した。	
\\	両[りょう] 会社[かいしゃ]は 合併[がっぺい]した。	合併=がっぺい= 
\\	(政治などの) 
\\	それは新設の会社に合併された。	
\\	それは 新設[しんせつ]の 会社[かいしゃ]に 合併[がっぺい]された。	合併=がっぺい= 
\\	(政治などの) 
\\	その銀行は他の銀行と合併した。	
\\	その 銀行[ぎんこう]は 他[た]の 銀行[ぎんこう]と 合併[がっぺい]した。	合併=がっぺい= 
\\	(政治などの) 
\\	意思疎通は英語と日本語の併用で行われている。	
\\	意思[いし] 疎通[そつう]は 英語[えいご]と 日本語[にほんご]の 併用[へいよう]で 行[おこな]われている。	意思疎通=いしそつう= 
\\	併用=へいよう= 
\\	(〜する) 
\\	余病が併発した。	
\\	余病[よびょう]が 併発[へいはつ]した。	余病=よびょう= 一つの病気に伴って起こる別の病気。併発=へいはつ= 
\\	(病の) 
\\	彼は流感から肺炎を併発した。	
\\	彼[かれ]は 流感[りゅうかん]から 肺炎[はいえん]を 併発[へいはつ]した。	流感=りゅうかん= 流行性感冒 
\\	インフルエンザ併用=へいよう= 
\\	(〜する) 
\\	彼女はそれを見て嫌悪の叫びをあげた。	
\\	彼女[かのじょ]はそれを 見[み]て 嫌悪[けんお]の 叫[さけ]びをあげた。	嫌悪=けんお= 
\\	私はそろそろこの仕事に嫌気がさしてきた。	
\\	私[わたし]はそろそろこの 仕事[しごと]に 嫌気[いやけ]がさしてきた。	嫌気=いやけ= 
\\	〜に嫌気がさす= 
\\	その一言で彼は恋人に嫌気がさした。	
\\	その 一言[ひとこと]で 彼[かれ]は 恋人[こいびと]に 嫌気[いやけ]がさした。	嫌気=いやけ= 
\\	〜に嫌気がさす= 
\\	サラリーマン生活にはほとほと嫌気がした。	
\\	サラリーマン 生活[せいかつ]にはほとほと 嫌気[いやけ]がした。	嫌気=いやけ= 
\\	〜に嫌気がさす= 
\\	うちは先祖代々浄土宗だ。	
\\	うちは 先祖[せんぞ] 代々[だいだい] 浄土宗[じょうどしゅう]だ。	先祖=せんぞ= 
\\	先祖代々の=せんぞだいだいの= 
\\	プレスリーはロックンロールの元祖だ。	
\\	プレスリーはロックンロールの 元祖[がんそ]だ。	元祖=がんそ= (一族の祖先) 
\\	(一宗の開祖) 
\\	(物事の創始者) 
\\	(発明者) 
\\	この店が広島風お好み焼きの元祖と言われている。	
\\	この 店[みせ]が 広島[ひろしま] 風[ふう]お 好み焼[このみや]きの 元祖[がんそ]と 言[い]われている。	元祖=がんそ= (一族の祖先) 
\\	(一宗の開祖) 
\\	(物事の創始者) 
\\	(発明者) 
\\	老人施設はどこも満杯で、入所希望者は1年待ちがざらだ。	
\\	老人[ろうじん] 施設[しせつ]はどこも 満杯[まんぱい]で、 入所[にゅうしょ] 希望[きぼう] 者[しゃ]は1 年[ねん] 待[ま]ちがざらだ。	満杯=まんぱい= 
\\	冷蔵庫は食料品で満杯だ。	
\\	冷蔵庫[れいぞうこ]は 食料[しょくりょう] 品[ひん]で 満杯[まんぱい]だ。	満杯=まんぱい= 
\\	書棚はみんな本で満杯だ。	
\\	書棚[しょだな]はみんな 本[ほん]で 満杯[まんぱい]だ。	満杯=まんぱい= 
\\	事件の特殊性を考慮して対策を立てる必要がある。	
\\	事件[じけん]の 特殊[とくしゅ] 性[せい]を 考慮[こうりょ]して 対策[たいさく]を 立[た]てる 必要[ひつよう]がある。	特殊な=とくしゅな= 
\\	これは極めて特殊なケースだ。	
\\	これは 極[きわ]めて 特殊[とくしゅ]なケースだ。	特殊な=とくしゅな= 
\\	われわれは敵の追跡をかわして逃げた。	
\\	われわれは 敵[てき]の 追跡[ついせき]をかわして 逃[に]げた。	追跡=ついせき= (跡を追うこと) 
\\	(逃亡者を追うこと) 
\\	(その後をたどる) 
\\	警察犬に犯人を追跡させた。	
\\	警察[けいさつ] 犬[けん]に 犯人[はんにん]を 追跡[ついせき]させた。	警察犬=けいさつけん= 
\\	追跡=ついせき= (跡を追うこと) 
\\	(逃亡者を追うこと) 
\\	(その後をたどる) 
\\	警官はただちに犯人を追跡した。	
\\	警官[けいかん]はただちに 犯人[はんにん]を 追跡[ついせき]した。	追跡=ついせき= (跡を追うこと) 
\\	(逃亡者を追うこと) 
\\	(その後をたどる) 
\\	様々な国家機関が東京に過度に集中している。	
\\	様々[さまざま]な 国家[こっか] 機関[きかん]が 東京[とうきょう]に 過度[かど]に 集中[しゅうちゅう]している。	過度=かど= 
\\	(〜に) 
\\	ファンの過度の期待は若手の選手にプレッシャーになる。	
\\	ファンの 過度[かど]の 期待[きたい]は 若手[わかて]の 選手[せんしゅ]にプレッシャーになる。	過度=かど= 
\\	(〜に) 
\\	過度の運動は体に良くないこともある。	
\\	過度[かど]の 運動[うんどう]は 体[からだ]に 良[よ]くないこともある。	過度=かど= 
\\	(〜に) 
\\	あいつの飲み方は桁が外れている。	
\\	あいつの 飲[の]み 方[かた]は 桁[けた]が 外[はず]れている。	桁=けた= (建物の) 
\\	(鉄製の) 
\\	(印刷物の) 
\\	(単位) 
\\	(位取り) 
\\	桁が外れる=けたがはずれる= 
\\	モスクワの冬の寒さは日本とは桁が違う。	
\\	モスクワの 冬[ふゆ]の 寒[さむ]さは 日本[にほん]とは 桁[けた]が 違[ちが]う。	桁=けた= (建物の) 
\\	(鉄製の) 
\\	(印刷物の) 
\\	(単位) 
\\	(位取り) 
\\	桁が違う=けたがちがう= 
\\	番組の視聴率が二桁を超えた。	
\\	番組[ばんぐみ]の 視聴[しちょう] 率[りつ]が 二桁[ふたけた]を 超[こ]えた。	桁=けた= (建物の) 
\\	(鉄製の) 
\\	(印刷物の) 
\\	(単位) 
\\	(位取り) 
\\	二桁=ふたけた= 
\\	失業率がついに二桁に達した。	
\\	失業[しつぎょう] 率[りつ]がついに二 桁[けた]に 達[たっ]した。	桁=けた= (建物の) 
\\	(鉄製の) 
\\	(印刷物の) 
\\	(単位) 
\\	(位取り) 
\\	二桁=ふたけた= 
\\	今月の交通事故件数はなんとか一桁にとどまった。	
\\	今月[こんげつ]の 交通[こうつう] 事故[じこ] 件数[けんすう]はなんとか 一桁[ひとけた]にとどまった。	桁=けた= (建物の) 
\\	(鉄製の) 
\\	(印刷物の) 
\\	(単位) 
\\	(位取り) 
\\	一桁=ひとけた= 
\\	一万円なら安いと思ったら一桁違った。	
\\	一 万[まん] 円[えん]なら 安[やす]いと 思[おも]ったら 一桁[ひとけた] 違[ちが]った。	桁=けた= (建物の) 
\\	(鉄製の) 
\\	(印刷物の) 
\\	(単位) 
\\	(位取り) 
\\	一桁=ひとけた= 
\\	彼のボーナスは我々とは桁が違う。	
\\	彼[かれ]のボーナスは 我々[われわれ]とは 桁[けた]が 違[ちが]う。	桁=けた= (建物の) 
\\	(鉄製の) 
\\	(印刷物の) 
\\	(単位) 
\\	(位取り) 
\\	桁が違う=けたがちがう= 
\\	今日の株式市場は取引開始から大荒れで、株価が乱高下した。	
\\	今日[きょう]の 株式[かぶしき] 市場[しじょう]は 取引[とりひき] 開始[かいし]から 大[おお] 荒[あ]れで、 株価[かぶか]が 乱高下[らんこうげ]した。	大荒れ=おおあれ= (大嵐); (混乱した状況、またひどく暴れるさま) 
\\	今年の政界は大荒れだった。	
\\	今年[ことし]の 政界[せいかい]は 大[おお] 荒[あ]れだった。	大荒れ=おおあれ= (大嵐); (混乱した状況、またひどく暴れるさま) 
\\	このチンピラどもは酔っぱらって街路上で大荒れに荒れた。	
\\	このチンピラどもは 酔[よ]っぱらって 街路[がいろ] 上[じょう]で 大[おお] 荒[あ]れに 荒[あ]れた。	チンピラ= 
\\	大荒れ=おおあれ= (大嵐); (混乱した状況、またひどく暴れるさま) 
\\	台風の接近で海は大荒れだ。	
\\	台風[たいふう]の 接近[せっきん]で 海[うみ]は 大[おお] 荒[あ]れだ。	大荒れ=おおあれ= (大嵐); (混乱した状況、またひどく暴れるさま) 
\\	試合は初めから大荒れだった。	
\\	試合[しあい]は 初[はじ]めから 大[だい] 荒[あ]れだった。	大荒れ=おおあれ= (大嵐); (混乱した状況、またひどく暴れるさま) 
\\	会議は大荒れに荒れた。	
\\	会議[かいぎ]は 大[だい] 荒[あ]れに 荒[あ]れた。	大荒れ=おおあれ= (大嵐); (混乱した状況、またひどく暴れるさま) 
\\	彼の股間を膝で蹴りをあげた。	
\\	彼[かれ]の 股間[こかん]を 膝[ひざ]で 蹴[け]りをあげた。	股間=こかん= 
\\	(男性の) 
\\	彼女は彼の股間を蹴った。	
\\	彼女[かのじょ]は 彼[かれ]の 股間[こかん]を 蹴[け]った。	股間=こかん= 
\\	(男性の) 
\\	輸出部門では
\\	社がトップだった。	
\\	輸出[ゆしゅつ] 部門[ぶもん]では 
\\	社[しゃ]がトップだった。	部門=ぶもん= (部類) 
\\	霞がかかっている。	
\\	霞[かすみ]がかかっている。	霞・翳み=かすみ= (気象現象) [霞] 
\\	遠くの方が霞のためにぼうっとしていた。	
\\	遠[とお]くの 方[ほう]が 霞[かすみ]のためにぼうっとしていた。	霞・翳み=かすみ= (気象現象) [霞] 
\\	人間、霞を食べて生きて行くわけにはいかない。	
\\	人間[にんげん]、 霞[かすみ]を 食[た]べて 生[い]きて 行[い]くわけにはいかない。	霞・翳み=かすみ= (気象現象) [霞] 
\\	彼の仕事ぶりを見てあげてください。	
\\	彼[かれ]の 仕事[しごと]ぶりを 見[み]てあげてください。	仕事ぶり= 
\\	その美談は視聴率を上げるためにテレビ局が捏造したものであった。	
\\	その 美談[びだん]は 視聴[しちょう] 率[りつ]を 上[あ]げるために テレビ局[てれびきょく]が 捏造[ねつぞう]したものであった。	捏造=ねつぞう= 
\\	その話はマスコミによるまったくの捏造だ。	
\\	その 話[はなし]はマスコミによるまったくの 捏造[ねつぞう]だ。	捏造=ねつぞう= 
\\	警察は殺人事件の証拠を捏造した。	
\\	警察[けいさつ]は 殺人[さつじん] 事件[じけん]の 証拠[しょうこ]を 捏造[ねつぞう]した。	捏造=ねつぞう= 
\\	彼女は新しい本の題材を集めているところだ。	
\\	彼女[かのじょ]は 新[あたら]しい 本[ほん]の 題材[だいざい]を 集[あつ]めているところだ。	題材=だいざい= 
\\	爆撃でそのビルは跡形もなくなった。	
\\	爆撃[ばくげき]でそのビルは 跡形[あとかた]もなくなった。	跡形=あとかた= (形跡・痕跡) 
\\	(証跡) 
\\	跡形もなく=あとかたもなく= 
\\	大臣の発言は首相の昨日の声明を踏まえてのものだった。	
\\	大臣[だいじん]の 発言[はつげん]は 首相[しゅしょう]の 昨日[きのう]の 声明[せいめい]を 踏[ふ]まえてのものだった。	踏まえる=ふまえる= 
\\	奇跡でも起こらなくてはこの病人はとても助からない。	
\\	奇跡[きせき]でも 起[お]こらなくてはこの 病人[びょうにん]はとても 助[たす]からない。	奇跡・奇蹟=きせき= 
\\	奇跡が起こった。	
\\	奇跡[きせき]が 起[お]こった。	奇跡・奇蹟=きせき= 
\\	イエスはこの街で多くの奇跡を行った。	
\\	イエスはこの 街[まち]で 多[おお]くの 奇跡[きせき]を 行[おこな]った。	奇跡・奇蹟=きせき= 
\\	泥棒は裏口から入った形跡がある。	
\\	泥棒[どろぼう]は 裏口[うらぐち]から 入[はい]った 形跡[けいせき]がある。	形跡=けいせき= (痕跡) 
\\	(証跡) 
\\	ベッドには誰かが横になった形跡があった。	
\\	ベッドには 誰[だれ]かが 横[よこ]になった 形跡[けいせき]があった。	形跡=けいせき= (痕跡) 
\\	(証跡) 
\\	その島には人の住んでいる形跡がない。	
\\	その 島[しま]には 人[ひと]の 住[す]んでいる 形跡[けいせき]がない。	形跡=けいせき= (痕跡) 
\\	(証跡) 
\\	彼が居合わせたような形跡は少しも見えなかった。	
\\	彼[かれ]が 居合[いあ]わせたような 形跡[けいせき]は 少[すこ]しも 見[み]えなかった。	形跡=けいせき= (痕跡) 
\\	(証跡) 
\\	誰かがその家に入った形跡がある。	
\\	誰[だれ]かがその 家[いえ]に 入[はい]った 形跡[けいせき]がある。	形跡=けいせき= (痕跡) 
\\	(証跡) 
\\	彼はみけんに傷跡がある。	
\\	彼[かれ]はみけんに 傷跡[きずあと]がある。	傷跡=きずあと= 
\\	あごの傷跡がきれいに消えた。	
\\	あごの 傷跡[きずあと]がきれいに 消[き]えた。	傷跡=きずあと= 
\\	ほおの傷跡はもうほとんど見えない。	
\\	ほおの 傷跡[きずあと]はもうほとんど 見[み]えない。	傷跡=きずあと= 
\\	あの事件は彼の心に深い傷跡を残した。	
\\	あの 事件[じけん]は 彼[かれ]の 心[こころ]に 深[ふか]い 傷跡[きずあと]を 残[のこ]した。	傷跡=きずあと= 
\\	彼の腕には犬に噛まれた傷跡がある。	
\\	彼[かれ]の 腕[うで]には 犬[いぬ]に 噛[か]まれた 傷跡[きずあと]がある。	傷跡=きずあと= 
\\	この町にはまだ戦争の傷跡が残っている。	
\\	この 町[まち]にはまだ 戦争[せんそう]の 傷跡[きずあと]が 残[のこ]っている。	傷跡=きずあと= 
\\	景気は現在足踏み状態である。	
\\	景気[けいき]は 現在[げんざい] 足踏[あしぶ]み 状態[じょうたい]である。	足踏み=あしぶみ= 
\\	(停頓) 
\\	交渉は足踏みしている。	
\\	交渉[こうしょう]は 足踏[あしぶ]みしている。	足踏み=あしぶみ= 
\\	(停頓) 
\\	アルコールは脳を侵す。	
\\	アルコールは 脳[のう]を 侵[おか]す。	侵す=おかす= (侵害する) 
\\	(聖域などを) 
\\	(侵入する) 
\\	(害する) <病気が> 
\\	彼は肺を侵されている。	
\\	彼[かれ]は 肺[はい]を 侵[おか]されている。	侵す=おかす= (侵害する) 
\\	(聖域などを) 
\\	(侵入する) 
\\	(害する) <病気が> 
\\	国籍不明の航空機に領空が侵された。	
\\	国籍[こくせき] 不明[ふめい]の 航空機[こうくうき]に 領空[りょうくう]が 侵[おか]された。	侵す=おかす= (侵害する) 
\\	(聖域などを) 
\\	(侵入する) 
\\	(害する) <病気が> 
\\	マスコミによってプライバシーが侵された。	
\\	マスコミによってプライバシーが 侵[おか]された。	侵す=おかす= (侵害する) 
\\	(聖域などを) 
\\	(侵入する) 
\\	(害する) <病気が> 
\\	それは著作権を侵す行為だ。	
\\	それは 著作[ちょさく] 権[けん]を 侵[おか]す 行為[こうい]だ。	侵す=おかす= (侵害する) 
\\	(聖域などを) 
\\	(侵入する) 
\\	(害する) <病気が> 
\\	彼の私生活を侵す権利はだれにもない。	
\\	彼[かれ]の 私生活[しせいかつ]を 侵[おか]す 権利[けんり]はだれにもない。	侵す=おかす= (侵害する) 
\\	(聖域などを) 
\\	(侵入する) 
\\	(害する) <病気が> 
\\	戦車が首都に侵攻した。	
\\	戦車[せんしゃ]が 首都[しゅと]に 侵攻[しんこう]した。	侵攻=しんこう= 他国や他の領地に攻め込むこと。
\\	目の前のけが人を見捨てるわけにはいかない。	
\\	目[め]の 前[まえ]のけが 人[にん]を 見捨[みす]てるわけにはいかない。	見捨てる=みすてる= (放置する) 
\\	(関係を絶つ) 
\\	(窮地にある人を) 
\\	いつまでもお見捨てなくよろしくお願いします。	
\\	いつまでもお 見捨[みす]てなくよろしくお 願[ねが]いします。	見捨てる=みすてる= (放置する) 
\\	(関係を絶つ) 
\\	(窮地にある人を) 
\\	あの二人はまさに水と油の関係だ。	
\\	あの二 人[にん]はまさに 水[みず]と 油[あぶら]の 関係[かんけい]だ。	
\\	袋の中身が透けて見える。	
\\	袋[ふくろ]の 中身[なかみ]が 透[す]けて 見[み]える。	透ける=すける= 
\\	言葉のかげに彼の本音が透けて見えた。	
\\	言葉[ことば]のかげに 彼[かれ]の 本音[ほんね]が 透[す]けて 見[み]えた。	透ける=すける= 
\\	新設めかした言葉の裏に彼の悪意が透けて見えた。	
\\	新設[しんせつ]めかした 言葉[ことば]の 裏[うら]に 彼[かれ]の 悪意[あくい]が 透[す]けて 見[み]えた。	透けて見える=すけてみえる= 
\\	私たちの巡り会いはまったく奇遇だった。	
\\	私[わたし]たちの 巡り会[めぐりあ]いはまったく 奇遇[きぐう]だった。	奇遇=きぐう= 
\\	いやあ、こんな所で会うなんて、奇遇だね。	
\\	いやあ、こんな 所[ところ]で 会[あ]うなんて、 奇遇[きぐう]だね。	奇遇=きぐう= 
\\	ここで君に会うなんて奇遇だ。	
\\	ここで 君[きみ]に 会[あ]うなんて 奇遇[きぐう]だ。	奇遇=きぐう= 
\\	もうしゃばには用がない。	
\\	もうしゃばには 用[よう]がない。	娑婆=しゃば= (この世) 
\\	(刑務所・軍隊などからみた外の世界) 
\\	彼女の発言が党内で波紋を広げている。	
\\	彼女[かのじょ]の 発言[はつげん]が 党内[とうない]で 波紋[はもん]を 広[ひろ]げている。	波紋=はもん= 
\\	(あるきっかけが次々と他に影響を及ぼすこと) 
\\	現在この法律の意義が問われている。	
\\	現在[げんざい]この 法律[ほうりつ]の 意義[いぎ]が 問[と]われている。	意義=いぎ= (意味内容) 
\\	(他との関連における重要性・価値) 
\\	オリンピックは勝つことではなく参加することに意義がある。	
\\	オリンピックは 勝[か]つことではなく 参加[さんか]することに 意義[いぎ]がある。	意義=いぎ= (意味内容) 
\\	(他との関連における重要性・価値) 
\\	彼の調査は学問的な意義に乏しかった。	
\\	彼[かれ]の 調査[ちょうさ]は 学問[がくもん] 的[てき]な 意義[いぎ]に 乏[とぼ]しかった。	意義=いぎ= (意味内容) 
\\	(他との関連における重要性・価値) 
\\	彼女の仕事は身障者にとって意義のあるものだった。	
\\	彼女[かのじょ]の 仕事[しごと]は 身障者[しんしょうしゃ]にとって 意義[いぎ]のあるものだった。	意義=いぎ= (意味内容) 
\\	(他との関連における重要性・価値) 
\\	彼はこの著作で宇宙開発の意義を論じている。	
\\	彼[かれ]はこの 著作[ちょさく]で 宇宙[うちゅう] 開発[かいはつ]の 意義[いぎ]を 論[ろん]じている。	意義=いぎ= (意味内容) 
\\	(他との関連における重要性・価値) 
\\	会議は規定の時間より10分遅れて始まった。	
\\	会議[かいぎ]は 規定[きてい]の 時間[じかん]より10 分[ふん] 遅[おく]れて 始[はじ]まった。	規定=きてい= (定め) 
\\	(条項) 
\\	(規則) 
\\	労働者は規定の時間を超える残業を拒否する権利がある。	
\\	労働[ろうどう] 者[しゃ]は 規定[きてい]の 時間[じかん]を 超[こ]える 残業[ざんぎょう]を 拒否[きょひ]する 権利[けんり]がある。	規定=きてい= (定め) 
\\	(条項) 
\\	(規則) 
\\	試験は規定の時間通りに行われた。	
\\	試験[しけん]は 規定[きてい]の 時間[じかん] 通[どお]りに 行[おこな]われた。	規定=きてい= (定め) 
\\	(条項) 
\\	(規則) 
\\	思いがけない問題が派生した。	
\\	思[おも]いがけない 問題[もんだい]が 派生[はせい]した。	派生=はせい= 
\\	(〜する) 
\\	その語はサンスクリット語から派生したものだ。	
\\	その 語[ご]はサンスクリット 語[ご]から 派生[はせい]したものだ。	派生=はせい= 
\\	(〜する) 
\\	何から着手していいのかわからなかった。	
\\	何[なに]から 着手[ちゃくしゅ]していいのかわからなかった。	着手=ちゃくしゅ= 
\\	(〜する) 
\\	(新事業に) 
\\	工事はまだ着手できないでいる。	
\\	工事[こうじ]はまだ 着手[ちゃくしゅ]できないでいる。	着手=ちゃくしゅ= 
\\	(〜する) 
\\	(新事業に) 
\\	着手金が3万円いる。	
\\	着手[ちゃくしゅ] 金[きん]が3 万[まん] 円[えん]いる。	着手=ちゃくしゅ= 
\\	(〜する) 
\\	(新事業に) 
\\	ボスの逮捕で麻薬密輸組織は解体した。	
\\	ボスの 逮捕[たいほ]で 麻薬[まやく] 密輸[みつゆ] 組織[そしき]は 解体[かいたい]した。	解体=かいたい= (建物などの) 
\\	(機械・車などの) 
\\	(団体組織の) 
\\	(死体・肉牛・クジラなどの) 
\\	1987年国鉄は解体された。	
\\	年[ねん] 国鉄[こくてつ]は 解体[かいたい]された。	解体=かいたい= (建物などの) 
\\	(機械・車などの) 
\\	(団体組織の) 
\\	(死体・肉牛・クジラなどの) 
\\	あと一息というところで失敗することが多々ある。	
\\	あと 一息[ひといき]というところで 失敗[しっぱい]することが 多々[たた]ある。	一息=ひといき= (一回の呼吸) 
\\	(小休止) 
\\	(もう少しの努力) 
\\	もう一息だ。	
\\	もう 一息[ひといき]だ。	一息=ひといき= (一回の呼吸) 
\\	(小休止) 
\\	(もう少しの努力) 
\\	あと一息が足りなかった。	
\\	あと 一息[ひといき]が 足[た]りなかった。	一息=ひといき= (一回の呼吸) 
\\	(小休止) 
\\	(もう少しの努力) 
\\	一息入れてからまた始めましょう。	
\\	一息[ひといき] 入[い]れてからまた 始[はじ]めましょう。	一息=ひといき= (一回の呼吸) 
\\	(小休止) 
\\	(もう少しの努力) 
\\	彼はほっと一息ついた。	
\\	彼[かれ]はほっと 一息[ひといき]ついた。	一息=ひといき= (一回の呼吸) 
\\	(小休止) 
\\	(もう少しの努力) 
\\	この辺で一息入れよう。	
\\	この 辺[へん]で 一息[ひといき] 入[い]れよう。	一息=ひといき= (一回の呼吸) 
\\	(小休止) 
\\	(もう少しの努力) 
\\	それが用いられた痕跡はない。	
\\	それが 用[もち]いられた 痕跡[こんせき]はない。	痕跡=こんせき= 
\\	誰かが窓から出た痕跡はない。	
\\	誰[だれ]かが 窓[まど]から 出[で]た 痕跡[こんせき]はない。	痕跡=こんせき= 
\\	病気でまだ衰弱している。	
\\	病気[びょうき]でまだ 衰弱[すいじゃく]している。	衰弱=すいじゃく= 
\\	その患者はどんどん衰弱しつつある。	
\\	その 患者[かんじゃ]はどんどん 衰弱[すいじゃく]しつつある。	衰弱=すいじゃく= 
\\	彼女は衰弱がひどくて手術ができない。	
\\	彼女[かのじょ]は 衰弱[すいじゃく]がひどくて 手術[しゅじゅつ]ができない。	衰弱=すいじゃく= 
\\	発見された山岳遭難者は衰弱が激しかった。	
\\	発見[はっけん]された 山岳[さんがく] 遭難[そうなん] 者[しゃ]は 衰弱[すいじゃく]が 激[はげ]しかった。	衰弱=すいじゃく= 
\\	このリンゴはすかすかだ。	
\\	このリンゴはすかすかだ。	すかすかの= 
\\	彼のエッセイ集は内容がすかすかだ。	
\\	彼[かれ]のエッセイ 集[しゅう]は 内容[ないよう]がすかすかだ。	すかすかの= 
\\	どこへでもずかずか入ってくる。	
\\	どこへでもずかずか 入[はい]ってくる。	ずかずか= 
\\	(許可なく) 
\\	(不作法に) 
\\	(つかつかと)
\\	僕たちが食事をしているところへずかずかやってきた。	
\\	僕[ぼく]たちが 食事[しょくじ]をしているところへずかずかやってきた。	ずかずか= 
\\	(許可なく) 
\\	(不作法に) 
\\	(つかつかと)
\\	人の心にずかずかと踏み込んで来るような奴だ。	
\\	人[ひと]の 心[こころ]にずかずかと 踏み込[ふみこ]んで 来[く]るような 奴[やつ]だ。	ずかずか= 
\\	(許可なく) 
\\	(不作法に) 
\\	(つかつかと)
\\	彼は断りもなくずかずかと私の部屋に入ってきた。	
\\	彼[かれ]は 断[ことわ]りもなくずかずかと 私[わたし]の 部屋[へや]に 入[はい]ってきた。	ずかずか= 
\\	(許可なく) 
\\	(不作法に) 
\\	(つかつかと)
\\	彼女の詩には透明感がある。	
\\	彼女[かのじょ]の 詩[し]には 透明[とうめい] 感[かん]がある。	透明な=とうめいな= 
\\	(清い・純粋な) 
\\	彼に遺された財産は皆無に等しい。	
\\	彼[かれ]に 遺[のこ]された 財産[ざいさん]は 皆無[かいむ]に 等[ひと]しい。	皆無=かいむ= 
\\	生存者のいる可能性は皆無に近い。	
\\	生存[せいぞん] 者[しゃ]のいる 可能[かのう] 性[せい]は 皆無[かいむ]に 近[ちか]い。	皆無=かいむ= 
\\	彼は化学の知識が皆無だ。	
\\	彼[かれ]は 化学[かがく]の 知識[ちしき]が 皆無[かいむ]だ。	皆無=かいむ= 
\\	今日は取引が皆無だった。	
\\	今日[きょう]は 取引[とりひき]が 皆無[かいむ]だった。	皆無=かいむ= 
\\	成功の見込みは皆無だ。	
\\	成功[せいこう]の 見込[みこ]みは 皆無[かいむ]だ。	皆無=かいむ= 
\\	彼女には詩人の素質は皆無だ。	
\\	彼女[かのじょ]には 詩人[しじん]の 素質[そしつ]は 皆無[かいむ]だ。	皆無=かいむ= 
\\	彼の提案に賛成する人は皆無だった。	
\\	彼[かれ]の 提案[ていあん]に 賛成[さんせい]する 人[ひと]は 皆無[かいむ]だった。	皆無=かいむ= 
\\	彼が当選する可能性は皆無だ。	
\\	彼[かれ]が 当選[とうせん]する 可能[かのう] 性[せい]は 皆無[かいむ]だ。	皆無=かいむ= 
\\	法律に関する知識は皆無に等しい。	
\\	法律[ほうりつ]に 関[かん]する 知識[ちしき]は 皆無[かいむ]に 等[ひと]しい。	皆無=かいむ= 
\\	彼女は彼の話を聞くことに徹した。	
\\	彼女[かのじょ]は 彼[かれ]の 話[はなし]を 聞[き]くことに 徹[てっ]した。	徹する=てっする= (貫通する) 
\\	(深く染み通る) 
\\	(通す); (主義や態度を貫く) 
\\	私たちは夜を徹して語り明かした。	
\\	私[わたし]たちは 夜[よる]を 徹[てっ]して 語り明[かたりあ]かした。	徹する=てっする= (貫通する) 
\\	(深く染み通る) 
\\	(通す); (主義や態度を貫く) 
\\	徹夜してセーターを編み上げた。	
\\	徹夜[てつや]してセーターを 編み上[あみあ]げた。	徹夜=てつや= 
\\	前の晩徹夜で頑張って100点を取った。	
\\	前[まえ]の 晩[ばん] 徹夜[てつや]で 頑張[がんば]って100 点[てん]を 取[と]った。	徹夜=てつや= 
\\	彼女のあの顔は徹夜でもしたに違いない。	
\\	彼女[かのじょ]のあの 顔[かお]は 徹夜[てつや]でもしたに 違[ちが]いない。	徹夜=てつや= 
\\	彼は徹夜で疲れ切っていた。	
\\	彼[かれ]は 徹夜[てつや]で 疲[つか]れ 切[き]っていた。	徹夜=てつや= 
\\	その小説の中で彼は徹頭徹尾悪者に仕立て上げられている。	
\\	その 小説[しょうせつ]の 中[なか]で 彼[かれ]は 徹頭徹尾[てっとうてつび] 悪者[わるもの]に 仕立[した]て 上[あ]げられている。	徹頭徹尾=てっとうてつび= (どこまでも) 
\\	(どの点から見ても) 
\\	(終始) 
\\	彼が書くものは徹頭徹尾虚構の世界である。	
\\	彼[かれ]が 書[か]くものは 徹頭徹尾[てっとうてつび] 虚構[きょこう]の 世界[せかい]である。	徹頭徹尾=てっとうてつび= (どこまでも) 
\\	(どの点から見ても) 
\\	(終始) 
\\	日本の現代文明は徹頭徹尾西洋文明の模倣であるという人もいる。	
\\	日本[にほん]の 現代[げんだい] 文明[ぶんめい]は 徹頭徹尾[てっとうてつび] 西洋[せいよう] 文明[ぶんめい]の 模倣[もほう]であるという 人[ひと]もいる。	徹頭徹尾=てっとうてつび= (どこまでも) 
\\	(どの点から見ても) 
\\	(終始) 
\\	野党はその法案に徹頭徹尾反対した。	
\\	野党[やとう]はその 法案[ほうあん]に 徹頭徹尾[てっとうてつび] 反対[はんたい]した。	徹頭徹尾=てっとうてつび= (どこまでも) 
\\	(どの点から見ても) 
\\	(終始) 
\\	まあ、一杯干してくれ。	
\\	まあ、 一杯[いっぱい] 干[ほ]してくれ。	干す=ほす= (乾かすために) 
\\	(風を通すために) 
\\	(水をなくす); (飲み尽くす) 
\\	(仕事などを与えない)
\\	本にかびが生えていたのでしばらく日に干した。	
\\	本[ほん]にかびが 生[は]えていたのでしばらく 日[ひ]に 干[ほ]した。	干す=ほす= (乾かすために) 
\\	(風を通すために) 
\\	(水をなくす); (飲み尽くす) 
\\	(仕事などを与えない)
\\	たくさんの毛布が日なたに干してある。	
\\	たくさんの 毛布[もうふ]が 日[ひ]なたに 干[ほ]してある。	日なた=ひなた= 
\\	干す=ほす= (乾かすために) 
\\	(風を通すために) 
\\	(水をなくす); (飲み尽くす) 
\\	(仕事などを与えない)
\\	彼はテレビ業界で干されている。	
\\	彼[かれ]はテレビ 業界[ぎょうかい]で 干[ほ]されている。	干す=ほす= (乾かすために) 
\\	(風を通すために) 
\\	(水をなくす); (飲み尽くす) 
\\	(仕事などを与えない)
\\	雨続きなので洗濯物を外に干せない。	
\\	雨[あめ] 続[つづ]きなので 洗濯[せんたく] 物[もの]を 外[そと]に 干[ほ]せない。	干す=ほす= (乾かすために) 
\\	(風を通すために) 
\\	(水をなくす); (飲み尽くす) 
\\	(仕事などを与えない)
\\	雪のためダイヤに若干の乱れが生じた。	
\\	雪[ゆき]のためダイヤに 若干[じゃっかん]の 乱[みだ]れが 生[しょう]じた。	ダイヤ= 
\\	若干=じゃっかん= 
\\	まだ仕事に真剣になれない。	
\\	まだ 仕事[しごと]に 真剣[しんけん]になれない。	真剣=しんけん= (本物の剣) 
\\	(〜な) (まじめな) 
\\	興味の話になると彼はいつも真剣な顔つきになる。	
\\	興味[きょうみ]の 話[はなし]になると 彼[かれ]はいつも 真剣[しんけん]な 顔[かお]つきになる。	真剣=しんけん= (本物の剣) 
\\	(〜な) (まじめな) 
\\	冗談を言っているのかと思ったら、彼の顔は真剣そのものだった。	
\\	冗談[じょうだん]を 言[い]っているのかと 思[おも]ったら、 彼[かれ]の 顔[かお]は 真剣[しんけん]そのものだった。	真剣=しんけん= (本物の剣) 
\\	(〜な) (まじめな) 
\\	砂の感触が素足に気持ちいい。	
\\	砂[すな]の 感触[かんしょく]が 素足[すあし]に 気持[きも]ちいい。	素足=すあし= 
\\	素足に靴をはいている。	
\\	素足[すあし]に 靴[くつ]をはいている。	素足=すあし= 
\\	彼女は素足だった。	
\\	彼女[かのじょ]は 素足[すあし]だった。	素足=すあし= 
\\	彼女は素足にサンダルをつっかけた。	
\\	彼女[かのじょ]は 素足[すあし]にサンダルをつっかけた。	素足=すあし= 
\\	シルクは素肌に優しい。	
\\	シルクは 素肌[すはだ]に 優[やさ]しい。	
\\	父は入浴後よく真っ裸で家の中を歩き回る。	
\\	父[ちち]は 入浴[にゅうよく] 後[ご]よく 真っ裸[まっぱだか]で 家[いえ]の 中[なか]を 歩き回[あるきまわ]る。	真っ裸=まっぱだか= 
\\	彼は真昼間から酒を飲んでいた。	
\\	彼[かれ]は 真昼間[まっぴるま]から 酒[さけ]を 飲[の]んでいた。	真昼間=まっぴるま= 
\\	語学の上達に近道はない。	
\\	語学[ごがく]の 上達[じょうたつ]に 近道[ちかみち]はない。	近道=ちかみち= 
\\	この品物は小売りで一個300円だ。	
\\	この 品物[しなもの]は 小売[こう]りで 一個[いっこ]300 円[えん]だ。	小売り=こうり= 
\\	あの人の演説は板についている。	
\\	(演説を聞いて) 
\\	あの 人[ひと]の 演説[えんぜつ]は 板[いた]についている。	板につく=いたにつく= (演技が) 
\\	(態度・服装などが) 
\\	背広姿が板についていない。	
\\	背広[せびろ] 姿[すがた]が 板[いた]についていない。	背広=せびろ= 
\\	板につく=いたにつく= (演技が) 
\\	(態度・服装などが) 
\\	蛇口から水がぽたぽたたれている。	
\\	蛇口[じゃぐち]から 水[みず]がぽたぽたたれている。	ぽたぽた= 
\\	天井から水がぽたぽた落ちてきた。	
\\	天井[てんじょう]から 水[みず]がぽたぽた 落[お]ちてきた。	ぽたぽた= 
\\	ぽたぽたと音がする。	
\\	ぽたぽたと 音[おと]がする。	ぽたぽた= 
\\	急に冷え込んできてざわざわと背筋に悪寒を覚えた。	
\\	急[きゅう]に 冷え込[ひえこ]んできてざわざわと 背筋[せすじ]に 悪寒[おかん]を 覚[おぼ]えた。	ざわざわ= (おしゃべりなどでうるさく落ち着かないさま); (木の葉などが音を立てて揺れるさま); (悪寒がするさま・ぞくぞく)
\\	風に吹かれて木の葉がざわざわとなった。	
\\	風[かぜ]に 吹[ふ]かれて 木の葉[このは]がざわざわとなった。	木の葉=このは・きのはざわざわ= (おしゃべりなどでうるさく落ち着かないさま); (木の葉などが音を立てて揺れるさま); (悪寒がするさま・ぞくぞく)
\\	会場がざわざわしていて話がよく聞こえなかった。	
\\	会場[かいじょう]がざわざわしていて 話[はなし]がよく 聞[き]こえなかった。	ざわざわ= (おしゃべりなどでうるさく落ち着かないさま); (木の葉などが音を立てて揺れるさま); (悪寒がするさま・ぞくぞく)
\\	木の葉が風にざわざわと揺れている。	
\\	木の葉[このは]が 風[かぜ]にざわざわと 揺[ゆ]れている。	木の葉=このは・きのはざわざわ= (おしゃべりなどでうるさく落ち着かないさま); (木の葉などが音を立てて揺れるさま); (悪寒がするさま・ぞくぞく)
\\	ワイングラスは粉々に割れた。	
\\	ワイングラスは 粉々[こなごな]に 割[わ]れた。	粉々の=こなごなの= 
\\	私の希望は粉々に砕かれた。	
\\	私[わたし]の 希望[きぼう]は 粉々[こなごな]に 砕[くだ]かれた。	粉々の=こなごなの= 
\\	子ネコはぴちゃぴちゃとミルクを飲んだ。	
\\	子[こ]ネコはぴちゃぴちゃとミルクを 飲[の]んだ。	ぴちゃぴちゃと= 
\\	雨がびしょびしょ降っている。	
\\	雨[あめ]がびしょびしょ 降[ふ]っている。	びしょびしょの= 
\\	水が多すぎて炊けた飯はびしょびしょだった。	
\\	水[みず]が 多[おお]すぎて 炊[た]けた 飯[めし]はびしょびしょだった。	びしょびしょの= 
\\	汗でシャツがびしょびしょになった。	
\\	汗[あせ]でシャツがびしょびしょになった。	びしょびしょの= 
\\	お辞儀をしいしい部屋を出ていった。	
\\	お 辞儀[じぎ]をしいしい 部屋[へや]を 出[で]ていった。	しいしい= 
\\	(ながら、ながら)
\\	その部屋に入ると悪臭が鼻を突いた。	
\\	その 部屋[へや]に 入[はい]ると 悪臭[あくしゅう]が 鼻[はな]を 突[つ]いた。	悪臭=あくしゅう= 
\\	あたりに悪臭が漂っている。	
\\	あたりに 悪臭[あくしゅう]が 漂[ただよ]っている。	悪臭=あくしゅう= 
\\	その判決に対して抗議の嵐が起こった。	
\\	その 判決[はんけつ]に 対[たい]して 抗議[こうぎ]の 嵐[あらし]が 起[お]こった。	嵐=あらし= (激しい風) 
\\	(暴風雨) 
\\	嵐が迫ってきた。	
\\	嵐[あらし]が 迫[せま]ってきた。	嵐=あらし= (激しい風) 
\\	(暴風雨) 
\\	きのう一日嵐が吹き荒れた。	
\\	きのう 一日[いちにち] 嵐[あらし]が 吹き荒[ふきあ]れた。	嵐=あらし= (激しい風) 
\\	(暴風雨) 
\\	この地方は嵐が多い。	
\\	この 地方[ちほう]は 嵐[あらし]が 多[おお]い。	嵐=あらし= (激しい風) 
\\	(暴風雨) 
\\	それは彼が弁護士になって初めて手掛けた仕事だった。	
\\	それは 彼[かれ]が 弁護士[べんごし]になって 初[はじ]めて 手掛[てが]けた 仕事[しごと]だった。	手掛ける=てがける= (取り扱う) 
\\	(〜に経験がある) 
\\	この手の商品の宣伝はこれまで手掛けたことがない。	
\\	この 手[て]の 商品[しょうひん]の 宣伝[せんでん]はこれまで 手掛[てが]けたことがない。	手掛ける=てがける= (取り扱う) 
\\	(〜に経験がある) 
\\	彼はそのプロジェクトを手掛けた。	
\\	彼[かれ]はそのプロジェクトを 手掛[てが]けた。	手掛ける=てがける= (取り扱う) 
\\	(〜に経験がある) 
\\	つづりを正確に書くように心掛けなさい。	
\\	つづりを 正確[せいかく]に 書[か]くように 心掛[こころが]けなさい。	綴り=つづり= (スペル) 
\\	心掛ける=こころがける= (努める) 
\\	(常に念頭に置く) 
\\	不時の出費に平素から心掛けている。	
\\	不時[ふじ]の 出費[しゅっぴ]に 平素[へいそ]から 心掛[こころが]けている。	不時=ふじ= 予定外の時であること。思いがけない時であること。 平素=へいそ= ふだん。常日頃。 心掛ける=こころがける= (努める) 
\\	(常に念頭に置く) 
\\	もっと人のことを思いやるように心掛けなさい。	
\\	もっと 人[ひと]のことを 思[おも]いやるように 心掛[こころが]けなさい。	心掛ける=こころがける= (努める) 
\\	(常に念頭に置く) 
\\	私は常日頃から安全運動を心掛けている。	
\\	私[わたし]は 常[つね] 日頃[ひごろ]から 安全[あんぜん] 運動[うんどう]を 心掛[こころが]けている。	心掛ける=こころがける= (努める) 
\\	(常に念頭に置く) 
\\	私は早寝早起きを心掛けている。	
\\	私[わたし]は 早寝[はやね] 早起[はやお]きを 心掛[こころが]けている。	心掛ける=こころがける= (努める) 
\\	(常に念頭に置く) 
\\	金銭がからむと親しい人間関係もこじれてくる。	
\\	金銭[きんせん]がからむと 親[した]しい 人間[にんげん] 関係[かんけい]もこじれてくる。	金銭=きんせん= 
\\	(金)
\\	金銭の管理がルースだ。	
\\	金銭[きんせん]の 管理[かんり]がルースだ。	金銭=きんせん= 
\\	(金)
\\	彼は金銭に細かい人だ。	
\\	彼[かれ]は 金銭[きんせん]に 細[こま]かい 人[ひと]だ。	金銭=きんせん= 
\\	(金)
\\	彼は金銭感覚がない。	
\\	彼[かれ]は 金銭[きんせん] 感覚[かんかく]がない。	金銭=きんせん= 
\\	(金)
\\	今小銭がないんだ。	
\\	今[いま] 小銭[こぜに]がないんだ。	小銭=こぜに= 
\\	小銭で3000円ある。	
\\	小銭[こぜに]で 3000円[さんぜんえん]ある。	小銭=こぜに= 
\\	この1000円札を小銭に替えてください。	
\\	この 1000円[せんえん] 札[さつ]を 小銭[こぜに]に 替[か]えてください。	小銭=こぜに= 
\\	あいにく小銭がないんです。	
\\	あいにく 小銭[こぜに]がないんです。	小銭=こぜに= 
\\	(掲示)釣り銭のいらないようにお願いします。	
\\	掲示[けいじ]) 釣り銭[つりせん]のいらないようにお 願[ねが]いします。	釣り銭=つりせん= 
\\	(お釣り)
\\	彼は俳句の達人の域に到達した。	
\\	彼[かれ]は 俳句[はいく]の 達人[たつじん]の 域[いき]に 到達[とうたつ]した。	到達=とうたつ= (到着) 
\\	(達成) 
\\	横文字の氾濫は目に余る。	
\\	横文字[よこもじ]の 氾濫[はんらん]は 目[め]に 余[あま]る。	氾濫=はんらん= (水などがあふれること) 
\\	(洪水); (たくさん満ちあふれていること) 
\\	ポルノ雑誌が町に氾濫している。	
\\	ポルノ 雑誌[ざっし]が 町[まち]に 氾濫[はんらん]している。	氾濫=はんらん= (水などがあふれること) 
\\	(洪水); (たくさん満ちあふれていること) 
\\	その雨で川の水が村一面に氾濫した。	
\\	その 雨[あめ]で 川[かわ]の 水[みず]が 村[むら]一 面[めん]に 氾濫[はんらん]した。	氾濫=はんらん= (水などがあふれること) 
\\	(洪水); (たくさん満ちあふれていること) 
\\	市場には外国製品が氾濫している。	
\\	市場[しじょう]には 外国[がいこく] 製品[せいひん]が 氾濫[はんらん]している。	氾濫=はんらん= (水などがあふれること) 
\\	(洪水); (たくさん満ちあふれていること) 
\\	今の日本はカタカナ語が氾濫している。	
\\	今[いま]の 日本[にっぽん]はカタカナ 語[ご]が 氾濫[はんらん]している。	氾濫=はんらん= (水などがあふれること) 
\\	(洪水); (たくさん満ちあふれていること) 
\\	暴動が勃発した。	
\\	暴動[ぼうどう]が 勃発[ぼっぱつ]した。	勃発=ぼっぱつ= 事件などが突然に起こること。
\\	1941年に太平洋戦争が勃発した。	
\\	年[ねん]に 太平洋戦争[たいへいようせんそう]が 勃発[ぼっぱつ]した。	勃発=ぼっぱつ= 事件などが突然に起こること。
\\	彼がスーパーマーケット業界の繁栄の基礎を築いた男である。	
\\	彼[かれ]がスーパーマーケット 業界[ぎょうかい]の 繁栄[はんえい]の 基礎[きそ]を 築[きず]いた 男[おとこ]である。	繁栄=はんえい= 豊かに栄えること。栄えて発展すること。(〜する) 
\\	(手紙の結句)ますますのご繁栄をお祈りします。	
\\	手紙[てがみ]の 結句[けっく])ますますのご 繁栄[はんえい]をお 祈[いの]りします。	繁栄=はんえい= 豊かに栄えること。栄えて発展すること。(〜する) 
\\	最近 
\\	業界が繁栄している。	
\\	最近[さいきん] 
\\	業界[ぎょうかい]が 繁栄[はんえい]している。	繁栄=はんえい= 豊かに栄えること。栄えて発展すること。(〜する) 
\\	その運動は自然と消滅した。	
\\	その 運動[うんどう]は 自然[しぜん]と 消滅[しょうめつ]した。	消滅=しょうめつ= (絶滅) 
\\	(消失) 
\\	(失効) 
\\	(権利を行使しなかったために生じる) 
\\	この気高い行為で彼の罪はすべて消滅した。	
\\	この 気高[けだか]い 行為[こうい]で 彼[かれ]の 罪[つみ]はすべて 消滅[しょうめつ]した。	消滅=しょうめつ= (絶滅) 
\\	(消失) 
\\	(失効) 
\\	(権利を行使しなかったために生じる) 
\\	多くの言語が消滅の危機にある。	
\\	多[おお]くの 言語[げんご]が 消滅[しょうめつ]の 危機[きき]にある。	消滅=しょうめつ= (絶滅) 
\\	(消失) 
\\	(失効) 
\\	(権利を行使しなかったために生じる) 
\\	今月末にその協定の効力が消滅する。	
\\	今月[こんげつ] 末[まつ]にその 協定[きょうてい]の 効力[こうりょく]が 消滅[しょうめつ]する。	消滅=しょうめつ= (絶滅) 
\\	(消失) 
\\	(失効) 
\\	(権利を行使しなかったために生じる) 
\\	2千円札が流通しているはずだが、とんと見かけない。	
\\	千[せん] 円[えん] 札[さつ]が 流通[りゅうつう]しているはずだが、とんと 見[み]かけない。	流通=りゅうつう= (貨幣の) 
\\	(商品の) 
\\	(空気などの) 
\\	今日では多額の電子マネーが流通している。	
\\	今日[きょう]では 多額[たがく]の 電子[でんし]マネーが 流通[りゅうつう]している。	流通=りゅうつう= (貨幣の) 
\\	(商品の) 
\\	(空気などの) 
\\	彼女が当面の責任者と考えられている。	
\\	彼女[かのじょ]が 当面[とうめん]の 責任[せきにん] 者[しゃ]と 考[かんが]えられている。	当面=とうめん= (面と向かうこと) 
\\	(直面する); (さしあたり); (〜の) 
\\	当面このプロジェクトは延期しよう。	
\\	当面[とうめん]このプロジェクトは 延期[えんき]しよう。	当面=とうめん= (面と向かうこと) 
\\	(直面する); (さしあたり); (〜の) 
\\	その条約は来年失効する。	
\\	その 条約[じょうやく]は 来年[らいねん] 失効[しっこう]する。	失効=しっこう= 
\\	あなたのパスポートはもう失効していますよ。	
\\	あなたのパスポートはもう 失効[しっこう]していますよ。	失効=しっこう= 
\\	その風景には私の魂を揺さぶる何かがあった。	
\\	その 風景[ふうけい]には 私[わたし]の 魂[たましい]を 揺[ゆ]さぶる 何[なに]かがあった。	揺さぶる=ゆさぶる= (揺する) 
\\	(動揺させる) 
\\	私は彼の肩を揺さぶった。	
\\	私[わたし]は 彼[かれ]の 肩[かた]を 揺[ゆ]さぶった。	揺さぶる=ゆさぶる= (揺する) 
\\	(動揺させる) 
\\	波がボートを激しく揺さぶった。	
\\	波[なみ]がボートを 激[はげ]しく 揺[ゆ]さぶった。	揺さぶる=ゆさぶる= (揺する) 
\\	(動揺させる) 
\\	彼の演説は私の心を揺さぶった。	
\\	彼[かれ]の 演説[えんぜつ]は 私[わたし]の 心[こころ]を 揺[ゆ]さぶった。	揺さぶる=ゆさぶる= (揺する) 
\\	(動揺させる) 
\\	11のテロ事件はアメリカ社会を根底から揺さぶった。	
\\	11のテロ 事件[じけん]はアメリカ 社会[しゃかい]を 根底[こんてい]から 揺[ゆ]さぶった。	揺さぶる=ゆさぶる= (揺する) 
\\	(動揺させる) 
\\	新型車両は以前のものより揺れなくなった。	
\\	新型[しんがた] 車両[しゃりょう]は 以前[いぜん]のものより 揺[ゆ]れなくなった。	揺れる=ゆれる= 
\\	上下・前後・左右などに動く。 
\\	不安定な状態になる。
\\	家が揺れるのを感じた。	
\\	家[いえ]が 揺[ゆ]れるのを 感[かん]じた。	揺れる=ゆれる= 
\\	上下・前後・左右などに動く。 
\\	不安定な状態になる。
\\	船から降りてもまだ体が揺れているような気がする。	
\\	船[ふね]から 降[お]りてもまだ 体[からだ]が 揺[ゆ]れているような 気[き]がする。	揺れる=ゆれる= 
\\	上下・前後・左右などに動く。 
\\	不安定な状態になる。
\\	突然足元の大地が揺れた。	
\\	突然[とつぜん] 足元[あしもと]の 大地[だいち]が 揺[ゆ]れた。	揺れる=ゆれる= 
\\	上下・前後・左右などに動く。 
\\	不安定な状態になる。
\\	少子化で年金制度が揺れている。	
\\	少子化[しょうしか]で 年金[ねんきん] 制度[せいど]が 揺[ゆ]れている。	揺れる=ゆれる= 
\\	上下・前後・左右などに動く。 
\\	不安定な状態になる。
\\	3時間のフライトの間、飛行機はほとんど揺れなかった。	
\\	時間[じかん]のフライトの 間[あいだ]、 飛行機[ひこうき]はほとんど 揺[ゆ]れなかった。	揺れる=ゆれる= 
\\	上下・前後・左右などに動く。 
\\	不安定な状態になる。
\\	汚職事件で政界は大揺れに揺れた。	
\\	汚職[おしょく] 事件[じけん]で 政界[せいかい]は 大[おお] 揺[ゆ]れに 揺[ゆ]れた。	揺れる=ゆれる= 
\\	上下・前後・左右などに動く。 
\\	不安定な状態になる。
\\	大学進学か就職かで彼女の心は揺れた。	
\\	大学[だいがく] 進学[しんがく]か 就職[しゅうしょく]かで 彼女[かのじょ]の 心[こころ]は 揺[ゆ]れた。	揺れる=ゆれる= 
\\	上下・前後・左右などに動く。 
\\	不安定な状態になる。
\\	その教会では日曜日には3回礼拝を行う。	
\\	その 教会[きょうかい]では 日曜日[にちようび]には3 回[かい] 礼拝[れいはい]を 行[おこな]う。	礼拝=れいはい= 
\\	(教会の) 
\\	こんな大金はめったに拝めるものじゃない。	
\\	こんな 大金[たいきん]はめったに 拝[おが]めるものじゃない。	拝む=おがむ= (合掌する) 
\\	(祈願する) 
\\	(懇願する) 
\\	(うやうやしく見る) 
\\	君に拝まれては私も承知しないわけにはいかない。	
\\	君[きみ]に 拝[おが]まれては 私[わたし]も 承知[しょうち]しないわけにはいかない。	拝む=おがむ= (合掌する) 
\\	(祈願する) 
\\	(懇願する) 
\\	(うやうやしく見る) 
\\	富士山に登って初日の出を拝んだ。	
\\	富士山[ふじさん]に 登[のぼ]って 初日の出[はつひので]を 拝[おが]んだ。	拝む=おがむ= (合掌する) 
\\	(祈願する) 
\\	(懇願する) 
\\	(うやうやしく見る) 
\\	神社に行って子供の病気が早く治りますようにと拝んできた。	
\\	神社[じんじゃ]に 行[い]って 子供[こども]の 病気[びょうき]が 早[はや]く 治[なお]りますようにと 拝[おが]んできた。	拝む=おがむ= (合掌する) 
\\	(祈願する) 
\\	(懇願する) 
\\	(うやうやしく見る) 
\\	彼女は内心の動揺を悟られまいとした。	
\\	彼女[かのじょ]は 内心[ないしん]の 動揺[どうよう]を 悟[さと]られまいとした。	動揺=どうよう= (揺れること) 
\\	(船の) (横揺れ) 
\\	(振動) 
\\	(平静を失うこと) (興奮) 
\\	(不安) 
\\	政界の動揺の兆がある。	
\\	政界[せいかい]の 動揺[どうよう]の 兆[ちょう]がある。	動揺=どうよう= (揺れること) 
\\	(船の) (横揺れ) 
\\	(振動) 
\\	(平静を失うこと) (興奮) 
\\	(不安) 
\\	彼女を動揺させることは言うな。	
\\	彼女[かのじょ]を 動揺[どうよう]させることは 言[い]うな。	動揺=どうよう= (揺れること) 
\\	(船の) (横揺れ) 
\\	(振動) 
\\	(平静を失うこと) (興奮) 
\\	(不安) 
\\	彼は内心の動揺を隠せなかった。	
\\	彼[かれ]は 内心[ないしん]の 動揺[どうよう]を 隠[かく]せなかった。	動揺=どうよう= (揺れること) 
\\	(船の) (横揺れ) 
\\	(振動) 
\\	(平静を失うこと) (興奮) 
\\	(不安) 
\\	彼女は息子の逮捕でかなり動揺している。	
\\	彼女[かのじょ]は 息子[むすこ]の 逮捕[たいほ]でかなり 動揺[どうよう]している。	動揺=どうよう= (揺れること) 
\\	(船の) (横揺れ) 
\\	(振動) 
\\	(平静を失うこと) (興奮) 
\\	(不安) 
\\	彼女はその知らせを聞いてとても動揺した。	
\\	彼女[かのじょ]はその 知[し]らせを 聞[き]いてとても 動揺[どうよう]した。	動揺=どうよう= (揺れること) 
\\	(船の) (横揺れ) 
\\	(振動) 
\\	(平静を失うこと) (興奮) 
\\	(不安) 
\\	私は彼を揺すって起こした。	
\\	私[わたし]は 彼[かれ]を 揺[ゆ]すって 起[お]こした。	揺する=ゆする= 
\\	それは今なお存在している。	
\\	それは 今[いま]なお 存在[そんざい]している。	存在=そんざい= 
\\	(生存) 
\\	彼はその小説で存在を認められるようになった。	
\\	彼[かれ]はその 小説[しょうせつ]で 存在[そんざい]を 認[みと]められるようになった。	存在=そんざい= 
\\	(生存) 
\\	彼は私にとって父親のような存在だ。	
\\	彼[かれ]は 私[わたし]にとって 父親[ちちおや]のような 存在[そんざい]だ。	存在=そんざい= 
\\	(生存) 
\\	君は僕にとってなくてはならない存在だ。	
\\	君[きみ]は 僕[ぼく]にとってなくてはならない 存在[そんざい]だ。	存在=そんざい= 
\\	(生存) 
\\	火星に水は存在するだろうか。	
\\	火星[かせい]に 水[みず]は 存在[そんざい]するだろうか。	存在=そんざい= 
\\	(生存) 
\\	お寺の中を拝観することができますか。	
\\	お 寺[てら]の 中[なか]を 拝観[はいかん]することができますか。	拝観=はいかん= 
\\	(〜する) 
\\	砂浜で貝殻を拾った。	
\\	砂浜[すなはま]で 貝殻[かいがら]を 拾[ひろ]った。	
\\	これで頼みの綱も切れた。	
\\	これで 頼[たの]みの 綱[つな]も 切[き]れた。	綱=つな= (太い) 
\\	(細い) 
\\	(たよりとするもの); (横綱の地位) 
\\	頼みの綱=たのみのつな= 
\\	強く引っ張りすぎて綱が切れた。	
\\	強[つよ]く 引っ張[ひっぱ]りすぎて 綱[つな]が 切[き]れた。	綱=つな= (太い) 
\\	(細い) 
\\	(たよりとするもの); (横綱の地位) 
\\	その会社の経営は毎日が綱渡りだ。	
\\	その 会社[かいしゃ]の 経営[けいえい]は 毎日[まいにち]が 綱渡[つなわた]りだ。	綱渡り=つなわたり= 
\\	(危険を冒すこと) 
\\	彼女は解雇されて命綱を断たされた。	
\\	彼女[かのじょ]は 解雇[かいこ]されて 命綱[いのちづな]を 断[た]たされた。	命綱=いのちづな= 
\\	党は昨日新しい綱領を発表した。	
\\	党[とう]は 昨日[きのう] 新[あたら]しい 綱領[こうりょう]を 発表[はっぴょう]した。	綱領=こうりょう= (政治団体などが掲げる活動の基本方針) 
\\	(要点) 
\\	米作は昨年に比して約3%の増収である。	
\\	米作[べいさく]は 昨年[さくねん]に 比[ひ]して 約[やく]3 
\\	[ぱーせんと]の 増収[ぞうしゅう]である。	増収=ぞうしゅう= 収入・収穫の増えること。
\\	自殺者の数は10年前に比して約2倍に増えている。	
\\	自殺[じさつ] 者[しゃ]の 数[かず]は10 年[ねん] 前[まえ]に 比[ひ]して 約[やく]2 倍[ばい]に 増[ふ]えている。	比する=ひする= 比べる。比較する。
\\	経済成長の負の側面にも目を向ける必要がある。	
\\	経済[けいざい] 成長[せいちょう]の 負[ふ]の 側面[そくめん]にも 目[め]を 向[む]ける 必要[ひつよう]がある。	負=ふ= ある数が零より小さいこと。マイナス。
\\	どちらも負けず劣らずの勢いです。	
\\	どちらも 負[ま]けず 劣[おと]らずの 勢[いきお]いです。	負けず劣らず=まけずおとらず= 互いに優劣がつけにくいさま。
\\	負けず劣らずのよい取り組みだった。	
\\	負[ま]けず 劣[おと]らずのよい 取り組[とりく]みだった。	負けず劣らず=まけずおとらず= 互いに優劣がつけにくいさま。
\\	彼女は姉に負けず劣らず美しい。	
\\	彼女[かのじょ]は 姉[あね]に 負[ま]けず 劣[おと]らず 美[うつく]しい。	負けず劣らず=まけずおとらず= 互いに優劣がつけにくいさま。
\\	その飢饉は干ばつによって起こった。	
\\	その 飢饉[ききん]は 干[かん]ばつによって 起[お]こった。	飢饉=ききん= 
\\	干ばつ=かんばつ= 
\\	いくつものビルが地震で瓦礫の山と化した。	
\\	いくつものビルが 地震[じしん]で 瓦礫[がれき]の 山[やま]と 化[か]した。	瓦礫=がれき= (瓦と小石) 
\\	(つまらぬもの) 
\\	我々は飢餓にひんした。	
\\	我々[われわれ]は 飢餓[きが]にひんした。	飢餓=きが= 
\\	難民の多くは飢餓状態にある。	
\\	難民[なんみん]の 多[おお]くは 飢餓[きが] 状態[じょうたい]にある。	飢餓=きが= 
\\	問題が顕在化した。	
\\	問題[もんだい]が 顕在[けんざい] 化[か]した。	顕在化=けんざいか= はっきりと形に現れて存在すること。
\\	彼はイエローカードが累積3枚となり、次の試合に出られなくなった。	
\\	彼[かれ]はイエローカードが 累積[るいせき]3 枚[まい]となり、 次[つぎ]の 試合[しあい]に 出[で]られなくなった。	累積=るいせき= 物事が次から次へ重なり積もること。また、重ね積もること。
\\	ただ惰性で働いている。	
\\	ただ 惰性[だせい]で 働[はたら]いている。	惰性=だせい= 
\\	(これまでの練習や勢い); 
\\	(慣性)
\\	単調な仕事は惰性に陥りがちだ。	
\\	単調[たんちょう]な 仕事[しごと]は 惰性[だせい]に 陥[おちい]りがちだ。	惰性=だせい= 
\\	(これまでの練習や勢い); 
\\	(慣性)
\\	日々惰性に流されないように生きたい。	
\\	日々[ひび] 惰性[だせい]に 流[なが]されないように 生[い]きたい。	惰性=だせい= 
\\	(これまでの練習や勢い); 
\\	(慣性)
\\	これまでの経験の蓄積がものをいう。	
\\	これまでの 経験[けいけん]の 蓄積[ちくせき]がものをいう。	蓄積=ちくせき= 
\\	ものをいう= 
\\	疲労が蓄積すると病気にかかりやすくなる。	
\\	疲労[ひろう]が 蓄積[ちくせき]すると 病気[びょうき]にかかりやすくなる。	蓄積=ちくせき= 
\\	金銭欲や名誉欲が選手たちを蝕みやすい。	
\\	金銭[きんせん] 欲[よく]や 名誉[めいよ] 欲[よく]が 選手[せんしゅ]たちを 蝕[むしば]みやすい。	蝕む=むしばむ= (虫に食われる) 
\\	(食い入る) 
\\	(悪くする) 
\\	彼女の体はがんに蝕まれた。	
\\	彼女[かのじょ]の 体[からだ]はがんに 蝕[むしば]まれた。	蝕む=むしばむ= (虫に食われる) 
\\	(食い入る) 
\\	(悪くする) 
\\	重労働で彼の健康は蝕まれていた。	
\\	重労働[じゅうろうどう]で 彼[かれ]の 健康[けんこう]は 蝕[むしば]まれていた。	蝕む=むしばむ= (虫に食われる) 
\\	(食い入る) 
\\	(悪くする) 
\\	彼女はいつもただ傍観しているだけだ。	
\\	彼女[かのじょ]はいつもただ 傍観[ぼうかん]しているだけだ。	傍観する=ぼうかんする= 手を出さずに、ただそばで見る。そのものごとに関係のない立場で見る。
\\	我々の目指すところはスポーツの大衆化である。	
\\	我々[われわれ]の 目指[めざ]すところはスポーツの 大衆[たいしゅう] 化[か]である。	
\\	目指す町は眼下に見えた。	
\\	目指[めざ]す 町[まち]は 眼下[がんか]に 見[み]えた。	
\\	私たちはゴールを目指してひた走りに走った。	
\\	私[わたし]たちはゴールを 目指[めざ]してひた 走[はし]りに 走[はし]った。	
\\	私たちは平和的解決を目指して交渉に望んだ。	
\\	私[わたし]たちは 平和[へいわ] 的[てき] 解決[かいけつ]を 目指[めざ]して 交渉[こうしょう]に 望[のぞ]んだ。	
\\	かすかな不安を覚えた。	
\\	かすかな 不安[ふあん]を 覚[おぼ]えた。	微かな・幽かな=かすかな= (今にも消えそうな) 
\\	(かろうじて知覚できる程度の) 
\\	かすかな希望が見えてきた。	
\\	かすかな 希望[きぼう]が 見[み]えてきた。	微かな・幽かな=かすかな= (今にも消えそうな) 
\\	(かろうじて知覚できる程度の) 
\\	春のかすかな気配が感じられる。	
\\	春[はる]のかすかな 気配[けはい]が 感[かん]じられる。	微かな・幽かな=かすかな= (今にも消えそうな) 
\\	(かろうじて知覚できる程度の) 
\\	鐘の音が微かに聞こえる。	
\\	鐘[かね]の 音[ね]が 微[かす]かに 聞[き]こえる。	鐘の音=かねのね微かな・幽かな=かすかな= (今にも消えそうな) 
\\	(かろうじて知覚できる程度の) 
\\	窓に光がかすかにさした。	
\\	窓[まど]に 光[ひかり]がかすかにさした。	微かな・幽かな=かすかな= (今にも消えそうな) 
\\	(かろうじて知覚できる程度の) 
\\	そのことはかすかに記憶している。	
\\	そのことはかすかに 記憶[きおく]している。	微かな・幽かな=かすかな= (今にも消えそうな) 
\\	(かろうじて知覚できる程度の) 
\\	一度消した文字が微かに見える。	
\\	一度[いちど] 消[け]した 文字[もじ]が 微[かす]かに 見[み]える。	微かな・幽かな=かすかな= (今にも消えそうな) 
\\	(かろうじて知覚できる程度の) 
\\	私は彼が生きているのではないかというかすかな希望を抱いている。	
\\	私[わたし]は 彼[かれ]が 生[い]きているのではないかというかすかな 希望[きぼう]を 抱[だ]いている。	微かな・幽かな=かすかな= (今にも消えそうな) 
\\	(かろうじて知覚できる程度の) 
\\	祖母の顔はかすかに覚えている。	
\\	祖母[そぼ]の 顔[かお]はかすかに 覚[おぼ]えている。	微かな・幽かな=かすかな= (今にも消えそうな) 
\\	(かろうじて知覚できる程度の) 
\\	微量の亜鉛が健康維持に役立っている。	
\\	微量[びりょう]の 亜鉛[あえん]が 健康[けんこう] 維持[いじ]に 役立[やくだ]っている。	微量=びりょう=ごくわずかな量。 亜鉛=あえん= 
\\	その牛乳から微量の農薬が検出された。	
\\	その 牛乳[ぎゅうにゅう]から 微量[びりょう]の 農薬[のうやく]が 検出[けんしゅつ]された。	微量=びりょう=ごくわずかな量。
\\	輸入食品から微量の放射性物質が検出された。	
\\	輸入[ゆにゅう] 食品[しょくひん]から 微量[びりょう]の 放射[ほうしゃ] 性[せい] 物質[ぶっしつ]が 検出[けんしゅつ]された。	微量=びりょう=ごくわずかな量。
\\	口元にかすかに微笑が浮かんだ。	
\\	口元[くちもと]にかすかに 微笑[びしょう]が 浮[う]かんだ。	微かな・幽かな=かすかな= (今にも消えそうな) 
\\	(かろうじて知覚できる程度の) 
\\	微笑=びしょう=ほほえむこと。ほほえみ。
\\	満面に微笑を浮かべた。	
\\	満面[まんめん]に 微笑[びしょう]を 浮[う]かべた。	微笑=びしょう=ほほえむこと。ほほえみ。
\\	それには微笑を禁じ得なかった。	
\\	それには 微笑[びしょう]を 禁[きん]じ 得[え]なかった。	微笑=びしょう=ほほえむこと。ほほえみ。
\\	どんな時でも彼女は微笑を絶やさない。	
\\	どんな 時[とき]でも 彼女[かのじょ]は 微笑[びしょう]を 絶[た]やさない。	微笑=びしょう=ほほえむこと。ほほえみ。
\\	彼女は私を見て優しく微笑した。	
\\	彼女[かのじょ]は 私[わたし]を 見[み]て 優[やさ]しく 微笑[びしょう]した。	微笑=びしょう=ほほえむこと。ほほえみ。
\\	勝利の女神はだれにほほえむだろうか。	
\\	勝利[しょうり]の 女神[めがみ]はだれにほほえむだろうか。	微笑む=ほほえむ=声を立てずににっこり笑う。微笑する。ほおえむ。
\\	彼女は小さな女の子に優しく微笑みかけた。	
\\	彼女[かのじょ]は 小[ちい]さな 女の子[おんなのこ]に 優[やさ]しく 微笑[ほほえ]みかけた。	微笑む=ほほえむ=声を立てずににっこり笑う。微笑する。ほおえむ。
\\	彼は光栄にもキュリー夫人と面会した。	
\\	彼[かれ]は 光栄[こうえい]にもキュリー 夫人[ふじん]と 面会[めんかい]した。	夫人=ふじん=貴人の妻。また、他人の妻を敬っていう語。
\\	毎日かろうじて生きているというありさまだ。	
\\	毎日[まいにち]かろうじて 生[い]きているというありさまだ。	辛うじて=かろうじて=やっとのことで。どうにか。(例:〜終電に間に合う)
\\	かろうじて勝利を得た。	
\\	かろうじて 勝利[しょうり]を 得[え]た。	辛うじて=かろうじて=やっとのことで。どうにか。(例:〜終電に間に合う)
\\	発車時刻にかろうじて間に合った。	
\\	発車[はっしゃ] 時刻[じこく]にかろうじて 間に合[まにあ]った。	辛うじて=かろうじて=やっとのことで。どうにか。(例:〜終電に間に合う)
\\	かろうじて最終バスに間に合った。	
\\	かろうじて 最終[さいしゅう]バスに 間に合[まにあ]った。	辛うじて=かろうじて=やっとのことで。どうにか。(例:〜終電に間に合う)
\\	かろうじて食べていけるだけの収入はある。	
\\	かろうじて 食[た]べていけるだけの 収入[しゅうにゅう]はある。	辛うじて=かろうじて=やっとのことで。どうにか。(例:〜終電に間に合う)
\\	彼はかろうじて死を免れた。	
\\	彼[かれ]はかろうじて 死[し]を	辛うじて=かろうじて=やっとのことで。どうにか。(例:〜終電に間に合う)免れる=まぬかれる・まぬがれる= 
\\	機械が一斉に作動し始めた。	
\\	機械[きかい]が 一斉[いっせい]に 作動[さどう]し 始[はじ]めた。	作動=さどう= 
\\	機械や装置の運動部分が働くこと。
\\	機械の作動不良があった。	
\\	機械[きかい]の 作動[さどう] 不良[ふりょう]があった。	作動=さどう= 
\\	機械や装置の運動部分が働くこと。
\\	タバコの煙で火災報知器が作動した。	
\\	タバコの 煙[けむり]で 火災報知器[かさいほうちき]が 作動[さどう]した。	作動=さどう= 
\\	機械や装置の運動部分が働くこと。
\\	仕事の状況次第では帰りが遅くなるかもしれない。	
\\	仕事[しごと]の 状況[じょうきょう] 次第[しだい]では 帰[かえ]りが 遅[おそ]くなるかもしれない。	状況・情況=じょうきょう=移り変わる物事の、その時々のありさま。
\\	彼は絶望的な状況に立たされている。	
\\	彼[かれ]は 絶望[ぜつぼう] 的[てき]な 状況[じょうきょう]に 立[た]たされている。	状況・情況=じょうきょう=移り変わる物事の、その時々のありさま。
\\	我々を取り巻く状況は変化しつつある。	
\\	我々[われわれ]を 取り巻[とりま]く 状況[じょうきょう]は 変化[へんか]しつつある。	状況・情況=じょうきょう=移り変わる物事の、その時々のありさま。
\\	彼女は自分の置かれている状況を説明した。	
\\	彼女[かのじょ]は 自分[じぶん]の 置[お]かれている 状況[じょうきょう]を 説明[せつめい]した。	状況・情況=じょうきょう=移り変わる物事の、その時々のありさま。
\\	首相の一言で状況が一変した。	
\\	首相[しゅしょう]の 一言[ひとこと]で 状況[じょうきょう]が 一変[いっぺん]した。	状況・情況=じょうきょう=移り変わる物事の、その時々のありさま。
\\	その話にはかなり修飾がある。	
\\	その 話[はなし]にはかなり 修飾[しゅうしょく]がある。	修飾=しゅうしょく=美しく飾ること。見せるために上辺を飾ること。
\\	彼女はパーティーのために着飾っていた。	
\\	彼女[かのじょ]はパーティーのために 着飾[きかざ]っていた。	着飾る=きかざる=美しい衣服を着て身を飾ること。盛装する。
\\	彼は晩年は愛犬と共に鎌倉で静かに暮らそうと思っていた。	
\\	彼[かれ]は 晩年[ばんねん]は 愛犬[あいけん]と 共[とも]に 鎌倉[かまくら]で 静[しず]かに 暮[く]らそうと 思[おも]っていた。	晩年=ばんねん=一生の終わりに近い時期。年老いてからの時期。
\\	朝晩の冷え込みがだいぶ厳しくなってきた。	
\\	朝晩[あさばん]の 冷え込[ひえこ]みがだいぶ 厳[きび]しくなってきた。	朝晩=あさばん=朝と晩。いつも。明け暮れ。
\\	朝晩飲んでる味噌汁が私の健康の秘訣です。	
\\	朝晩[あさばん] 飲[の]んでる 味噌汁[みそしる]が 私[わたし]の 健康[けんこう]の 秘訣[ひけつ]です。	朝晩=あさばん=朝と晩。いつも。明け暮れ。
\\	彼は大学受験に備え、朝晩晩学に励んだ。	
\\	彼[かれ]は 大学[だいがく] 受験[じゅけん]に 備[そな]え、 朝晩[あさばん] 晩学[ばんがく]に 励[はげ]んだ。	朝晩=あさばん=朝と晩。いつも。明け暮れ。
\\	防災体制を整備しておくことが重要だ。	
\\	防災[ぼうさい] 体制[たいせい]を 整備[せいび]しておくことが 重要[じゅうよう]だ。	整備=せいび= 
\\	その研究所では整備された環境のもとで最先端の研究が進められている。	
\\	その 研究所[けんきゅうじょ]では 整備[せいび]された 環境[かんきょう]のもとで 最先端[さいせんたん]の 研究[けんきゅう]が 進[すす]められている。	整備=せいび= 
\\	その工場の防火設備はよく整備されている。	
\\	その 工場[こうじょう]の 防火[ぼうか] 設備[せつび]はよく 整備[せいび]されている。	整備=せいび= 
\\	顧みてわが国はどうだろう。	
\\	顧[かえり]みてわが 国[くに]はどうだろう。	顧みる・省みる=かえりみる= {顧みる} (振り向く) 
\\	(回想する) 
\\	(顧慮する) 
\\	{省みる} (反省する) 
\\	彼は旧友を顧みようともしない。	
\\	彼[かれ]は 旧友[きゅうゆう]を 顧[かえり]みようともしない。	顧みる・省みる=かえりみる= {顧みる} (振り向く) 
\\	(回想する) 
\\	(顧慮する) 
\\	{省みる} (反省する) 
\\	これは、当時はだれも顧みなかった事件である。	
\\	これは、 当時[とうじ]はだれも 顧[かえり]みなかった 事件[じけん]である。	顧みる・省みる=かえりみる= {顧みる} (振り向く) 
\\	(回想する) 
\\	(顧慮する) 
\\	{省みる} (反省する) 
\\	彼の意見は識者には顧みられなかった。	
\\	彼[かれ]の 意見[いけん]は 識者[しきしゃ]には 顧[かえり]みられなかった。	顧みる・省みる=かえりみる= {顧みる} (振り向く) 
\\	(回想する) 
\\	(顧慮する) 
\\	{省みる} (反省する) 
\\	顧みますと、私はいつも誰かに支えられてきました。	
\\	顧[かえり]みますと、 私[わたし]はいつも 誰[だれ]かに 支[ささ]えられてきました。	顧みる・省みる=かえりみる= {顧みる} (振り向く) 
\\	(回想する) 
\\	(顧慮する) 
\\	{省みる} (反省する) 
\\	他人への迷惑も顧みずに彼らは電車の中で大声で話していた。	
\\	他人[たにん]への 迷惑[めいわく]も 顧[かえり]みずに 彼[かれ]らは 電車[でんしゃ]の 中[なか]で 大声[おおごえ]で 話[はな]していた。	顧みる・省みる=かえりみる= {顧みる} (振り向く) 
\\	(回想する) 
\\	(顧慮する) 
\\	{省みる} (反省する) 
\\	彼は仕事一途で家庭を顧みなかった。	
\\	彼[かれ]は 仕事[しごと] 一途[いちず]で 家庭[かてい]を 顧[かえり]みなかった。	顧みる・省みる=かえりみる= {顧みる} (振り向く) 
\\	(回想する) 
\\	(顧慮する) 
\\	{省みる} (反省する) 
\\	一晩山中で過ごさなくてはならなかった。	
\\	一晩[ひとばん] 山中[さんちゅう]で 過[す]ごさなくてはならなかった。	一晩=ひとばん=1)日が暮れてから夜が明けるまでの間。         
\\	ある晩。いつかの夜。
\\	我々は一晩語り明かした。	
\\	我々[われわれ]は 一晩[ひとばん] 語り明[かたりあ]かした。	一晩=ひとばん=1)日が暮れてから夜が明けるまでの間。         
\\	ある晩。いつかの夜。
\\	私は昨夜一晩中そのことを考えた。	
\\	私[わたし]は 昨夜[さくや] 一晩[ひとばん] 中[ちゅう]そのことを 考[かんが]えた。	一晩=ひとばん=1)日が暮れてから夜が明けるまでの間。         
\\	ある晩。いつかの夜。
\\	一晩泊めてあげよう。	
\\	一晩[ひとばん] 泊[と]めてあげよう。	一晩=ひとばん=1)日が暮れてから夜が明けるまでの間。         
\\	ある晩。いつかの夜。
\\	それについては一晩考えさせてください。	
\\	それについては 一晩[ひとばん] 考[かんが]えさせてください。	一晩=ひとばん=1)日が暮れてから夜が明けるまでの間。         
\\	ある晩。いつかの夜。
\\	この地は昼夜の寒暖の差が激しい。	
\\	この 地[ち]は 昼夜[ちゅうや]の 寒暖[かんだん]の 差[さ]が 激[はげ]しい。	寒暖=かんだん= 
\\	(温度) 
\\	私の顧問になってくれる人がほしい。	
\\	私[わたし]の 顧問[こもん]になってくれる 人[ひと]がほしい。	顧問=こもん=1)会社、団体などで、相談を受けて意見を述べる役。また、その人。        
\\	意見を問うこと。相談すること。
\\	彼女はこの随筆で幼年時代を回顧している。	
\\	彼女[かのじょ]はこの 随筆[ずいひつ]で 幼年[ようねん] 時代[じだい]を 回顧[かいこ]している。	回顧=かいこ=1)過ぎ去ったことを思い起こすこと。「学生時代を〜する」        
\\	後ろを振り返ること。
\\	貴様の面などにドと見たくない。	
\\	貴様[きさま]の 面[つら]などにドと 見[み]たくない。	貴様=きさま= 
\\	面=つら
\\	これは貴様の仕業か。	
\\	これは 貴様[きさま]の 仕業[しわざ]か。	貴様=きさま= 
\\	森林浴には癒しの効果がある。	
\\	森林浴[しんりんよく]には 癒[いや]しの 効果[こうか]がある。	森林浴=しんりんよく= 
\\	癒し=いやし=肉体の疲れ、精神の悩み、苦しみを何かに頼って解消したり和らげたりすること。
\\	老後はこの子に頼るつもりだ。	
\\	老後[ろうご]はこの 子[こ]に 頼[たよ]るつもりだ。	頼る=たよる= (頼みにする;依存する) 
\\	(助力を仰ぐ) 
\\	(手づるにする) 
\\	彼女は人には頼らない。	
\\	彼女[かのじょ]は 人[ひと]には 頼[たよ]らない。	頼る=たよる= (頼みにする;依存する) 
\\	(助力を仰ぐ) 
\\	(手づるにする) 
\\	彼女は頼るべき肉親がいない。	
\\	彼女[かのじょ]は 頼[たよ]るべき 肉親[にくしん]がいない。	頼る=たよる= (頼みにする;依存する) 
\\	(助力を仰ぐ) 
\\	(手づるにする) 
\\	(演説社が)原稿に頼り過ぎだ。	
\\	演説[えんぜつ] 社[しゃ]が) 原稿[げんこう]に 頼[たよ]り 過[す]ぎだ。	頼る=たよる= (頼みにする;依存する) 
\\	(助力を仰ぐ) 
\\	(手づるにする) 
\\	人に頼ってばかりいないで、自分でやりなさい。	
\\	人[ひと]に 頼[たよ]ってばかりいないで、 自分[じぶん]でやりなさい。	頼る=たよる= (頼みにする;依存する) 
\\	(助力を仰ぐ) 
\\	(手づるにする) 
\\	彼は頼れる人だ。	
\\	彼[かれ]は 頼[たよ]れる 人[ひと]だ。	頼る=たよる= (頼みにする;依存する) 
\\	(助力を仰ぐ) 
\\	(手づるにする) 
\\	車が故障したので、バスに頼らざるをえなかった。	
\\	車[くるま]が 故障[こしょう]したので、バスに 頼[たよ]らざるをえなかった。	頼る=たよる= (頼みにする;依存する) 
\\	(助力を仰ぐ) 
\\	(手づるにする) 
\\	その言葉は下品だ。	
\\	その 言葉[ことば]は 下品[げひん]だ。	下品な=げひんな= 
\\	彼は食べ方が下品だ。	
\\	彼[かれ]は 食[た]べ 方[かた]が 下品[げひん]だ。	下品な=げひんな= 
\\	うわべは紳士を気取っているが、話すことは下品だ。	
\\	うわべは 紳士[しんし]を 気取[きど]っているが、 話[はな]すことは 下品[げひん]だ。	下品な=げひんな= 
\\	音を立てて食べるのは下品だ。	
\\	音[おと]を 立[た]てて 食[た]べるのは 下品[げひん]だ。	下品な=げひんな= 
\\	じゃんけんして勝った方が座ることにしよう。	
\\	じゃんけんして 勝[か]った 方[ほう]が 座[すわ]ることにしよう。	
\\	じゃんけんで決めよう。	
\\	じゃんけんで 決[き]めよう。	
\\	ぱあはぐうに勝つ。	
\\	ぱあはぐうに 勝[か]つ。	
\\	チョキはぐうに負ける。	
\\	チョキはぐうに 負[ま]ける。	
\\	あなたの年頃になればもう少しものがわかってもいいはずよ。	
\\	あなたの 年頃[としごろ]になればもう 少[すこ]しものがわかってもいいはずよ。	年頃=としごろ=1)外見から判断した、だいたいの年齢。年の頃。         
\\	一人前の年齢。特に、女性の結婚するのにふさわしい年齢。妙齢。         
\\	そのことにふさわしい年齢。また、一般的にある傾向になりやすい年齢。         
\\	ここ数年の間。年来。
\\	だれでも君たちの年頃にはそのように考えるものだ。	
\\	だれでも 君[きみ]たちの 年頃[としごろ]にはそのように 考[かんが]えるものだ。	年頃=としごろ=1)外見から判断した、だいたいの年齢。年の頃。         
\\	一人前の年齢。特に、女性の結婚するのにふさわしい年齢。妙齢。         
\\	そのことにふさわしい年齢。また、一般的にある傾向になりやすい年齢。         
\\	ここ数年の間。年来。
\\	彼はそのとき君と同じ年頃でした。	
\\	彼[かれ]はそのとき 君[きみ]と 同[おな]じ 年頃[としごろ]でした。	年頃=としごろ=1)外見から判断した、だいたいの年齢。年の頃。         
\\	一人前の年齢。特に、女性の結婚するのにふさわしい年齢。妙齢。         
\\	そのことにふさわしい年齢。また、一般的にある傾向になりやすい年齢。         
\\	ここ数年の間。年来。
\\	今はちょうど難しい年頃だ。	
\\	今[いま]はちょうど 難[むずか]しい 年頃[としごろ]だ。	年頃=としごろ=1)外見から判断した、だいたいの年齢。年の頃。         
\\	一人前の年齢。特に、女性の結婚するのにふさわしい年齢。妙齢。         
\\	そのことにふさわしい年齢。また、一般的にある傾向になりやすい年齢。         
\\	ここ数年の間。年来。
\\	彼女は今が遊びたい年頃だ。	
\\	彼女[かのじょ]は 今[いま]が 遊[あそ]びたい 年頃[としごろ]だ。	年頃=としごろ=1)外見から判断した、だいたいの年齢。年の頃。         
\\	一人前の年齢。特に、女性の結婚するのにふさわしい年齢。妙齢。         
\\	そのことにふさわしい年齢。また、一般的にある傾向になりやすい年齢。         
\\	ここ数年の間。年来。
\\	今度の選挙は与党に対する逆風が強いものと予想される。	
\\	今度[こんど]の 選挙[せんきょ]は 与党[よとう]に 対[たい]する 逆風[ぎゃくふう]が 強[つよ]いものと 予想[よそう]される。	逆風=ぎゃくふう=1)進行方向から吹いてくる風。向かい風。          
\\	不利な状況。進行を妨げる出来事。
\\	不祥事が続いて与党への逆風がますます強まっている。	
\\	不祥事[ふしょうじ]が 続[つづ]いて 与党[よとう]への 逆風[ぎゃくふう]がますます 強[つよ]まっている。	不祥事=ふしょうじ=関係者にとって不都合な事件、事柄。 逆風=ぎゃくふう=1)進行方向から吹いてくる風。向かい風。          
\\	不利な状況。進行を妨げる出来事。
\\	英語の綴りは覚えにくい。	
\\	英語[えいご]の 綴[つづ]りは 覚[おぼ]えにくい。	綴り=つづり= (スペル) 
\\	この綴りはめちゃめちゃだ。	
\\	この 綴[つづ]りはめちゃめちゃだ。	綴り=つづり= (スペル) 
\\	お名前の綴りを言ってください。	
\\	お 名前[なまえ]の 綴[つづ]りを 言[い]ってください。	綴り=つづり= (スペル) 
\\	それでは彼の名の綴りが違う。	
\\	それでは 彼[かれ]の 名[な]の 綴[つづ]りが 違[ちが]う。	綴り=つづり= (スペル) 
\\	この単語は綴り通りには発音しない。	
\\	この 単語[たんご]は 綴[つづ]り 通[どお]りには 発音[はつおん]しない。	綴り=つづり= (スペル) 
\\	彼女の手紙にはいくつかつづりの間違いがあった。	
\\	彼女[かのじょ]の 手紙[てがみ]にはいくつかつづりの 間違[まちが]いがあった。	綴り=つづり= (スペル) 
\\	この報告の真偽は未確認だ。	
\\	この 報告[ほうこく]の 真偽[しんぎ]は 未[み] 確認[かくにん]だ。	真偽=しんぎ=真実と、偽り。まことかうそか。
\\	真偽はともかくとして、その話は彼の性格を鮮やかに描き出している。	
\\	真偽[しんぎ]はともかくとして、その 話[はなし]は 彼[かれ]の 性格[せいかく]を 鮮[あざ]やかに 描き出[えがきだ]している。	真偽=しんぎ=真実と、偽り。まことかうそか。
\\	その報告の真偽のほどはわからない。	
\\	その 報告[ほうこく]の 真偽[しんぎ]のほどはわからない。	真偽=しんぎ=真実と、偽り。まことかうそか。
\\	モーツァルトには心の傷を癒す力がある。	
\\	モーツァルトには 心[こころ]の 傷[きず]を 癒[いや]す 力[ちから]がある。	癒す=いやす=病気や傷を治す。苦痛や飢えなどを直したり和らげたりする。
\\	時が悲しみを癒してくれた。	
\\	時[とき]が 悲[かな]しみを 癒[いや]してくれた。	癒す=いやす=病気や傷を治す。苦痛や飢えなどを直したり和らげたりする。
\\	温泉に入って疲れを癒した。	
\\	温泉[おんせん]に 入[はい]って 疲[つか]れを 癒[いや]した。	癒す=いやす=病気や傷を治す。苦痛や飢えなどを直したり和らげたりする。
\\	離婚で負った心の傷は癒えることがなかった。	
\\	離婚[りこん]で 負[お]った 心[こころ]の 傷[きず]は 癒[い]えることがなかった。	癒える=いえる=1)病気・傷などが治る。治癒する。         
\\	悲しみ・苦しみ・悩みなどが消える。
\\	けがが癒えるまで2週間かかった。	
\\	けがが 癒[い]えるまで 2週間[にしゅうかん]かかった。	癒える=いえる=1)病気・傷などが治る。治癒する。         
\\	悲しみ・苦しみ・悩みなどが消える。
\\	時が経てば傷も癒えますよ。	
\\	時[とき]が 経[た]てば 傷[きず]も 癒[い]えますよ。	癒える=いえる=1)病気・傷などが治る。治癒する。         
\\	悲しみ・苦しみ・悩みなどが消える。
\\	彼女の心の傷はまだ癒えていない。	
\\	彼女[かのじょ]の 心[こころ]の 傷[きず]はまだ 癒[い]えていない。	癒える=いえる=1)病気・傷などが治る。治癒する。         
\\	悲しみ・苦しみ・悩みなどが消える。
\\	それぞれの部品は何の秩序もなくただばらばらに置かれていた。	
\\	それぞれの 部品[ぶひん]は 何[なに]の 秩序[ちつじょ]もなくただばらばらに 置[お]かれていた。	秩序=ちつじょ=1)物事を行う場合の正しい順序・筋道。         
\\	その社会・団体などが、望ましい状態を保つための順序や決まり。
\\	寒さがずいぶん緩んできた。	
\\	寒[さむ]さがずいぶん 緩[ゆる]んできた。	緩む=ゆるむ= (締まっていたものが) 
\\	(気持ち・表情などが) 
\\	彼らの警戒が緩むまで待つことにした。	
\\	彼[かれ]らの 警戒[けいかい]が 緩[ゆる]むまで 待[ま]つことにした。	緩む=ゆるむ= (締まっていたものが) 
\\	(気持ち・表情などが) 
\\	気が緩んだら急に傷が痛み出した。	
\\	気[き]が 緩[ゆる]んだら 急[きゅう]に 傷[きず]が 痛[いた]み 出[だ]した。	緩む=ゆるむ= (締まっていたものが) 
\\	(気持ち・表情などが) 
\\	ねじが緩んだ。	
\\	ねじが 緩[ゆる]んだ。	緩む=ゆるむ= (締まっていたものが) 
\\	(気持ち・表情などが) 
\\	靴のひもが緩んだ。	
\\	靴[くつ]のひもが 緩[ゆる]んだ。	緩む=ゆるむ= (締まっていたものが) 
\\	(気持ち・表情などが) 
\\	彼の姿を見て思わず口元が緩んだ。	
\\	彼[かれ]の 姿[すがた]を 見[み]て 思[おも]わず 口元[くちもと]が 緩[ゆる]んだ。	緩む=ゆるむ= (締まっていたものが) 
\\	(気持ち・表情などが) 
\\	期末試験が終わって気が緩んだ。	
\\	期末[きまつ] 試験[しけん]が 終[お]わって 気[き]が 緩[ゆる]んだ。	緩む=ゆるむ= (締まっていたものが) 
\\	(気持ち・表情などが) 
\\	国中に鉄道網が張り巡らされている。	
\\	国[くに] 中[じゅう]に 鉄道[てつどう] 網[もう]が 張り巡[はりめぐ]らされている。	鉄道網=てつどうもう= 
\\	金網の破れたところからニワトリが逃げ出した。	
\\	金網[かなあみ]の 破[やぶ]れたところからニワトリが 逃げ出[にげだ]した。	金網=かなあみ= 
\\	彼は網にかかった魚のように身動きが取れなくなった。	
\\	彼[かれ]は 網[あみ]にかかった 魚[さかな]のように 身動[みうご]きが 取[と]れなくなった。	網=あみ= 
\\	(投網) 
\\	珍しい魚が網に迷い込んだ。	
\\	珍[めずら]しい 魚[さかな]が 網[あみ]に 迷い込[まよいこ]んだ。	網=あみ= 
\\	(投網) 
\\	クモは空中に網を張って虫を捕る。	
\\	クモは 空中[くうちゅう]に 網[あみ]を 張[は]って 虫[むし]を 捕[と]る。	網=あみ= 
\\	(投網) 
\\	網でセミを捕まえた。	
\\	網[あみ]でセミを 捕[つか]まえた。	網=あみ= 
\\	(投網) 
\\	サルの社会にも序列がある。	
\\	サルの 社会[しゃかい]にも 序列[じょれつ]がある。	序列=じょれつ=1)順序をつけて並べること。また、順に並ぶこと。         
\\	一定の基準に従って並べた順序。決まった順序。
\\	序列では彼は北村さんの次だ。	
\\	序列[じょれつ]では 彼[かれ]は 北村[きたむら]さんの 次[つぎ]だ。	序列=じょれつ=1)順序をつけて並べること。また、順に並ぶこと。         
\\	一定の基準に従って並べた順序。決まった順序。
\\	それらの学校間には学力上の序列がある。	
\\	それらの 学校[がっこう] 間[かん]には 学力[がくりょく] 上[じょう]の 序列[じょれつ]がある。	序列=じょれつ=1)順序をつけて並べること。また、順に並ぶこと。         
\\	一定の基準に従って並べた順序。決まった順序。
\\	これで驚いちゃだめだよ、まだほんの序の口なんだから。	
\\	これで 驚[おどろ]いちゃだめだよ、まだほんの 序の口[じょのくち]なんだから。	序の口=じょのくち=物事の始まったばかりのところ。
\\	この寒さはまだ序の口だ。	
\\	この 寒[さむ]さはまだ 序の口[じょのくち]だ。	序の口=じょのくち=物事の始まったばかりのところ。
\\	戦いに勝っても気を緩めるな。	
\\	戦[たたか]いに 勝[か]っても 気[き]を 緩[ゆる]めるな。	気を緩める=きをゆるめる= 
\\	緩める=ゆるめる=1)(締まっているものを) 
\\	(結び目などを) 
\\	(気持ち・表情などを) 
\\	(取り締まりや規則などを) 
\\	(力などを) 
\\	(速度) 
\\	彼は交差点で車の速度を緩めた。	
\\	彼[かれ]は 交差点[こうさてん]で 車[くるま]の 速度[そくど]を 緩[ゆる]めた。	緩める=ゆるめる=1)(締まっているものを) 
\\	(結び目などを) 
\\	(気持ち・表情などを) 
\\	(取り締まりや規則などを) 
\\	(力などを) 
\\	(速度) 
\\	規則が緩くなった。	
\\	規則[きそく]が 緩[ゆる]くなった。	緩い=ゆるい=1)『締まっていない』
\\	『厳しくない』
\\	『遅い』
\\	『固くない』
\\	痩せて指輪が緩くなった。	
\\	痩[や]せて 指輪[ゆびわ]が 緩[ゆる]くなった。	緩い=ゆるい=1)『締まっていない』
\\	『厳しくない』
\\	『遅い』
\\	『固くない』
\\	近ごろスカートのウエストが緩くなった。	
\\	近[ちか]ごろスカートのウエストが 緩[ゆる]くなった。	緩い=ゆるい=1)『締まっていない』
\\	『厳しくない』
\\	『遅い』
\\	『固くない』
\\	今日は便が緩い。	
\\	今日[きょう]は 便[べん]が 緩[ゆる]い。	便=べん緩い=ゆるい=1)『締まっていない』
\\	『厳しくない』
\\	『遅い』
\\	『固くない』
\\	帯(ベルト)が緩い。	
\\	帯[おび]が 緩[ゆる]い。	緩い=ゆるい=1)『締まっていない』
\\	『厳しくない』
\\	『遅い』
\\	『固くない』
\\	このスカートはウエストが少し緩いようだ。	
\\	このスカートはウエストが 少[すこ]し 緩[ゆる]いようだ。	緩い=ゆるい=1)『締まっていない』
\\	『厳しくない』
\\	『遅い』
\\	『固くない』
\\	取り締まりが緩い。	
\\	取り締[とりし]まりが 緩[ゆる]い。	緩い=ゆるい=1)『締まっていない』
\\	『厳しくない』
\\	『遅い』
\\	『固くない』
\\	道はゆるく右に曲がっている。	
\\	道[みち]はゆるく 右[みぎ]に 曲[ま]がっている。	緩い=ゆるい=1)『締まっていない』
\\	『厳しくない』
\\	『遅い』
\\	『固くない』
\\	ここは監視がゆるい。	
\\	ここは 監視[かんし]がゆるい。	緩い=ゆるい=1)『締まっていない』
\\	『厳しくない』
\\	『遅い』
\\	『固くない』
\\	輸出の伸び率がこのところ緩やかになった。	
\\	輸出[ゆしゅつ]の 伸[の]び 率[りつ]がこのところ 緩[ゆる]やかになった。	緩やかな=ゆるやかな=1)『緩慢な』
\\	『なだらかな』
\\	『寛大な』
\\	『きつくない』
\\	寺までは緩やかな登りだ。	
\\	寺[てら]までは 緩[ゆる]やかな 登[のぼ]りだ。	緩やかな=ゆるやかな=1)『緩慢な』
\\	『なだらかな』
\\	『寛大な』
\\	『きつくない』
\\	このスカートはゆるゆるです。	
\\	このスカートはゆるゆるです。	ゆるゆると=1)『急がないようす』『ゆっくり』       
\\	『緩んでいるようす・大きすぎるようす』
\\	私たちはゆるゆると旅を続けた。	
\\	私[わたし]たちはゆるゆると 旅[たび]を 続[つづ]けた。	ゆるゆると=1)『急がないようす』『ゆっくり』       
\\	『緩んでいるようす・大きすぎるようす』
\\	部屋にはイタリアの旗が飾ってあった。	
\\	部屋[へや]にはイタリアの 旗[はた]が 飾[かざ]ってあった。	飾る=かざる=1)『装飾する』
\\	『華やかさをそえる』        
\\	『陳列する・飾りとして置く』
\\	『きどる』
\\	食卓に花が一輪飾ってあった。	
\\	食卓[しょくたく]に 花[はな]が 一輪[いちりん] 飾[かざ]ってあった。	飾る=かざる=1)『装飾する』
\\	『華やかさをそえる』        
\\	『陳列する・飾りとして置く』
\\	『きどる』
\\	ショーウインドーにはいろいろな品が飾ってある。	
\\	ショーウインドーにはいろいろな 品[しな]が 飾[かざ]ってある。	飾る=かざる=1)『装飾する』
\\	『華やかさをそえる』        
\\	『陳列する・飾りとして置く』
\\	『きどる』
\\	なんとか先生に巻頭を飾っていただきたいのですが。	
\\	なんとか 先生[せんせい]に 巻頭[かんとう]を 飾[かざ]っていただきたいのですが。	飾る=かざる=1)『装飾する』
\\	『華やかさをそえる』        
\\	『陳列する・飾りとして置く』
\\	『きどる』
\\	盛大な花火が閉会式のフィナールを飾った。	
\\	盛大[せいだい]な 花火[はなび]が 閉会[へいかい] 式[しき]のフィナールを 飾[かざ]った。	飾る=かざる=1)『装飾する』
\\	『華やかさをそえる』        
\\	『陳列する・飾りとして置く』
\\	『きどる』
\\	卒業式の会場はたくさんの折り鶴で飾られていた。	
\\	卒業[そつぎょう] 式[しき]の 会場[かいじょう]はたくさんの 折り鶴[おりづる]で 飾[かざ]られていた。	飾る=かざる=1)『装飾する』
\\	『華やかさをそえる』        
\\	『陳列する・飾りとして置く』
\\	『きどる』
\\	通りは色とりどりの電飾で飾られていた。	
\\	通[とお]りは 色[いろ]とりどりの 電飾[でんしょく]で 飾[かざ]られていた。	飾る=かざる=1)『装飾する』
\\	『華やかさをそえる』        
\\	『陳列する・飾りとして置く』
\\	『きどる』
\\	彼の部屋には大きな絵が飾ってあった。	
\\	彼[かれ]の 部屋[へや]には 大[おお]きな 絵[え]が 飾[かざ]ってあった。	飾る=かざる=1)『装飾する』
\\	『華やかさをそえる』        
\\	『陳列する・飾りとして置く』
\\	『きどる』
\\	彼の名前が新聞の第一面を飾った。	
\\	彼[かれ]の 名前[なまえ]が 新聞[しんぶん]の 第[だい] 一面[いちめん]を 飾[かざ]った。	飾る=かざる=1)『装飾する』
\\	『華やかさをそえる』        
\\	『陳列する・飾りとして置く』
\\	『きどる』
\\	規則正しい生活を送ることを信条としている。	
\\	規則正[きそくただ]しい 生活[せいかつ]を 送[おく]ることを 信条[しんじょう]としている。	信条=しんじょう= (信念) 
\\	(教条) 
\\	彼は弱者を助けることを信条とした。	
\\	彼[かれ]は 弱者[じゃくしゃ]を 助[たす]けることを 信条[しんじょう]とした。	信条=しんじょう= (信念) 
\\	(教条) 
\\	彼は自分の信条を曲げようとしない。	
\\	彼[かれ]は 自分[じぶん]の 信条[しんじょう]を 曲[ま]げようとしない。	信条=しんじょう= (信念) 
\\	(教条) 
\\	この辞書は現行のアメリカの俗語を網羅している。	
\\	この 辞書[じしょ]は 現行[げんこう]のアメリカの 俗語[ぞくご]を 網羅[もうら]している。	網羅=もうら= (〜する) 『全てを包含する』
\\	『残らず集める・取り入れる』
\\	百科事典はあらゆる分野にわたる知識を網羅している。	
\\	百科[ひゃっか] 事典[じてん]はあらゆる 分野[ぶんや]にわたる 知識[ちしき]を 網羅[もうら]している。	網羅=もうら= (〜する) 『全てを包含する』
\\	『残らず集める・取り入れる』
\\	この団体には米国社会のあらゆる階級の人が網羅されている。	
\\	この 団体[だんたい]には 米国[べいこく] 社会[しゃかい]のあらゆる 階級[かいきゅう]の 人[ひと]が 網羅[もうら]されている。	網羅=もうら= (〜する) 『全てを包含する』
\\	『残らず集める・取り入れる』
\\	このデータベースは津波に関する過去80年間のデータを網羅している。	
\\	このデータベースは 津波[つなみ]に 関[かん]する 過去[かこ]80 年間[ねんかん]のデータを 網羅[もうら]している。	網羅=もうら= (〜する) 『全てを包含する』
\\	『残らず集める・取り入れる』
\\	警察は集団スリのグループを一網打尽にした。	
\\	警察[けいさつ]は 集団[しゅうだん]スリのグループを 一網打尽[いちもうだじん]にした。	一網打尽=いちもうだじん=一度打った網でそこにいる魚を全部捕らえること。転じて、一味の者を一度に全部とらえること。
\\	警察は銀行強盗一味を一網打尽にした。	
\\	警察[けいさつ]は 銀行[ぎんこう] 強盗[ごうとう] 一味[いちみ]を 一網打尽[いちもうだじん]にした。	一網打尽=いちもうだじん=一度打った網でそこにいる魚を全部捕らえること。転じて、一味の者を一度に全部とらえること。
\\	その受験塾の教育内容は一風変わっている。	
\\	その 受験[じゅけん] 塾[じゅく]の 教育[きょういく] 内容[ないよう]は 一風[いっぷう] 変[か]わっている。	一風変わった=いっぷうかわった= (奇異な) 
\\	(月並みでない) 
\\	(奇抜な) 
\\	彼は一風変わった人だ。	
\\	彼[かれ]は 一風[いっぷう] 変[か]わった 人[ひと]だ。	一風変わった=いっぷうかわった= (奇異な) 
\\	(月並みでない) 
\\	(奇抜な) 
\\	彼女の数学の教え方は一風変わっている。	
\\	彼女[かのじょ]の 数学[すうがく]の 教[おし]え 方[かた]は 一風[いっぷう] 変[か]わっている。	一風変わった=いっぷうかわった= (奇異な) 
\\	(月並みでない) 
\\	(奇抜な) 
\\	彼はせっぱ詰まってやったのだ。	
\\	彼[かれ]はせっぱ 詰[つ]まってやったのだ。	切羽詰まる=せっぱつまる=ある事態などが間近に迫ってどうにもならなくなる。身動きがとれなくなる。
\\	問いつめられ、私はせっぱ詰まってうそをついた。	
\\	問[と]いつめられ、 私[わたし]はせっぱ 詰[つ]まってうそをついた。	切羽詰まる=せっぱつまる=ある事態などが間近に迫ってどうにもならなくなる。身動きがとれなくなる。
\\	人はせっぱ詰まらないとなかなか一生懸命にならないものだ。	
\\	人[ひと]はせっぱ 詰[つ]まらないとなかなか 一生懸命[いっしょうけんめい]にならないものだ。	切羽詰まる=せっぱつまる=ある事態などが間近に迫ってどうにもならなくなる。身動きがとれなくなる。
\\	人間せっぱ詰まれば何でもできるものさ。	
\\	人間[にんげん]せっぱ 詰[つ]まれば 何[なに]でもできるものさ。	切羽詰まる=せっぱつまる=ある事態などが間近に迫ってどうにもならなくなる。身動きがとれなくなる。
\\	この話はチャラにしよう。	
\\	この 話[はなし]はチャラにしよう。	ちゃら=1)口から出まかせを言うこと。でたらめ。ちゃらくら。     
\\	貸し借りをなしにすること。差し引きゼロ。帳消し。     
\\	なかったことにすること。『話を〜にする』
\\	出まかせを言っているのではないよ。	
\\	出[で]まかせを 言[い]っているのではないよ。	出まかせ=でまかせ= 
\\	彼の改革案は絵に描いた餅にすぎない。	
\\	彼[かれ]の 改革[かいかく] 案[あん]は 絵[え]に 描[えが]いた 餅[もち]にすぎない。	絵に描いた餅=えにかいたもち=『どんなに巧みに描いてあっても食べられないところから』何の役も立たないもの。また、実物・本物でなければ何の値打ちもないこと。
\\	運否天賦だ、とにかくやってみよう。	
\\	運否天賦[うんぷてんぷ]だ、とにかくやってみよう。	運否天賦=うんぷてんぷ=運の善し悪しは天が決めるということ。運を天に任せること。
\\	あのチームはここ数年リーグの最下位に沈みっぱなしだ。	
\\	あのチームはここ 数[すう] 年[ねん]リーグの 最下位[さいかい]に 沈[しず]みっぱなしだ。	沈む=しずむ= (下降する) 
\\	(水面などから) 
\\	(底に着く) 
\\	(めり込む) 
\\	(水没する) 
\\	(見えなくなる) 
\\	(気分が落ち込む) 
\\	(地味になる) 
\\	(下位に転落する) 
\\	沈む瀬あれば浮かぶ瀬もある。	
\\	沈[しず]む 瀬[せ]あれば 浮[う]かぶ 瀬[せ]もある。	沈む=しずむ= (下降する) 
\\	(水面などから) 
\\	(底に着く) 
\\	(めり込む) 
\\	(水没する) 
\\	(見えなくなる) 
\\	(気分が落ち込む) 
\\	(地味になる) 
\\	(下位に転落する) 
\\	夫を失って彼女は悲しみに沈んでいた。	
\\	夫[おっと]を 失[うしな]って 彼女[かのじょ]は 悲[かな]しみに 沈[しず]んでいた。	沈む=しずむ= (下降する) 
\\	(水面などから) 
\\	(底に着く) 
\\	(めり込む) 
\\	(水没する) 
\\	(見えなくなる) 
\\	(気分が落ち込む) 
\\	(地味になる) 
\\	(下位に転落する) 
\\	沈んだ顔をしてるけど何かあったのか。	
\\	沈[しず]んだ 顔[かお]をしてるけど 何[なに]かあったのか。	沈む=しずむ= (下降する) 
\\	(水面などから) 
\\	(底に着く) 
\\	(めり込む) 
\\	(水没する) 
\\	(見えなくなる) 
\\	(気分が落ち込む) 
\\	(地味になる) 
\\	(下位に転落する) 
\\	最近彼女は沈みがちだ。	
\\	最近[さいきん] 彼女[かのじょ]は 沈[しず]みがちだ。	沈む=しずむ= (下降する) 
\\	(水面などから) 
\\	(底に着く) 
\\	(めり込む) 
\\	(水没する) 
\\	(見えなくなる) 
\\	(気分が落ち込む) 
\\	(地味になる) 
\\	(下位に転落する) 
\\	あまりの疲労で体が沈むような気がする。	
\\	あまりの 疲労[ひろう]で 体[からだ]が 沈[しず]むような 気[き]がする。	沈む=しずむ= (下降する) 
\\	(水面などから) 
\\	(底に着く) 
\\	(めり込む) 
\\	(水没する) 
\\	(見えなくなる) 
\\	(気分が落ち込む) 
\\	(地味になる) 
\\	(下位に転落する) 
\\	船は沈みかけている。	
\\	船[ふね]は 沈[しず]みかけている。	沈む=しずむ= (下降する) 
\\	(水面などから) 
\\	(底に着く) 
\\	(めり込む) 
\\	(水没する) 
\\	(見えなくなる) 
\\	(気分が落ち込む) 
\\	(地味になる) 
\\	(下位に転落する) 
\\	彼の体は波にのまれて沈んでいった。	
\\	彼[かれ]の 体[からだ]は 波[なみ]にのまれて 沈[しず]んでいった。	沈む=しずむ= (下降する) 
\\	(水面などから) 
\\	(底に着く) 
\\	(めり込む) 
\\	(水没する) 
\\	(見えなくなる) 
\\	(気分が落ち込む) 
\\	(地味になる) 
\\	(下位に転落する) 
\\	最近気分が沈みがちだ。	
\\	最近[さいきん] 気分[きぶん]が 沈[しず]みがちだ。	沈む=しずむ= (下降する) 
\\	(水面などから) 
\\	(底に着く) 
\\	(めり込む) 
\\	(水没する) 
\\	(見えなくなる) 
\\	(気分が落ち込む) 
\\	(地味になる) 
\\	(下位に転落する) 
\\	重苦しい沈黙が1、2分続いた。	
\\	重苦[おもくる]しい 沈黙[ちんもく]が1、2 分[ふん] 続[つづ]いた。	沈黙=ちんもく=1)だまりこむこと。口をきかないこと。         
\\	音を出さないこと。物音もなく静かなこと。         
\\	活動をせずにじっとしていること。
\\	沈黙に耐えられない。	
\\	沈黙[ちんもく]に 耐[た]えられない。	沈黙=ちんもく=1)だまりこむこと。口をきかないこと。         
\\	音を出さないこと。物音もなく静かなこと。         
\\	活動をせずにじっとしていること。
\\	彼は5年間の沈黙ののち長編小説を出版した。	
\\	彼[かれ]は 5年間[ごねんかん]の 沈黙[ちんもく]ののち 長編[ちょうへん] 小説[しょうせつ]を 出版[しゅっぱん]した。	沈黙=ちんもく=1)だまりこむこと。口をきかないこと。         
\\	音を出さないこと。物音もなく静かなこと。         
\\	活動をせずにじっとしていること。
\\	彼は今までの沈黙を破って事件のことを話し出した。	
\\	彼[かれ]は 今[いま]までの 沈黙[ちんもく]を 破[やぶ]って 事件[じけん]のことを 話し出[はなしだ]した。	沈黙=ちんもく=1)だまりこむこと。口をきかないこと。         
\\	音を出さないこと。物音もなく静かなこと。         
\\	活動をせずにじっとしていること。
\\	彼女は長い沈黙を破って大作を世に問うた。	
\\	彼女[かのじょ]は 長[なが]い 沈黙[ちんもく]を 破[やぶ]って 大作[たいさく]を 世[よ]に 問[と]うた。	沈黙=ちんもく=1)だまりこむこと。口をきかないこと。         
\\	音を出さないこと。物音もなく静かなこと。         
\\	活動をせずにじっとしていること。
\\	そのことについてかれは沈黙を守っている。	
\\	そのことについてかれは 沈黙[ちんもく]を 守[まも]っている。	沈黙=ちんもく=1)だまりこむこと。口をきかないこと。         
\\	音を出さないこと。物音もなく静かなこと。         
\\	活動をせずにじっとしていること。
\\	2人の間に気まずい沈黙が続いた。	
\\	2人[ふたり]の 間[あいだ]に 気[き]まずい 沈黙[ちんもく]が 続[つづ]いた。	沈黙=ちんもく=1)だまりこむこと。口をきかないこと。         
\\	音を出さないこと。物音もなく静かなこと。         
\\	活動をせずにじっとしていること。
\\	しばらく沈黙したあと、彼は再び話し始めた。	
\\	しばらく 沈黙[ちんもく]したあと、 彼[かれ]は 再[ふたた]び 話[はな]し 始[はじ]めた。	沈黙=ちんもく=1)だまりこむこと。口をきかないこと。         
\\	音を出さないこと。物音もなく静かなこと。         
\\	活動をせずにじっとしていること。
\\	その船は台風に遭って沈没した。	
\\	その 船[ふね]は 台風[たいふう]に 遭[あ]って 沈没[ちんぼつ]した。	沈没=ちんぼつ=1)船などが水中に沈むこと。         
\\	酒に酔いつぶれること。         
\\	遊びに夢中になって仕事や用事を忘れてしまうこと。特に歓楽街などに入り込んでしまうこと。
\\	貧乏には慣れている。	
\\	貧乏[びんぼう]には 慣[な]れている。	貧乏=びんぼう=財産や収入が少なくて生活が苦しいこと。また、そのさま。
\\	あいつが貧乏になったのは自業自得。	
\\	あいつが 貧乏[びんぼう]になったのは 自業自得[じごうじとく]。	貧乏=びんぼう=財産や収入が少なくて生活が苦しいこと。また、そのさま。 自業自得=じごうじとく= 
\\	貧乏の味を知っている。	
\\	貧乏[びんぼう]の 味[あじ]を 知[し]っている。	貧乏=びんぼう=財産や収入が少なくて生活が苦しいこと。また、そのさま。
\\	貧乏はしていても心は豊かです。	
\\	貧乏[びんぼう]はしていても 心[こころ]は 豊[ゆた]かです。	貧乏=びんぼう=財産や収入が少なくて生活が苦しいこと。また、そのさま。
\\	お人好しのせいか、彼はいつも貧乏くじを引いている。	
\\	お 人好[ひとよ]しのせいか、 彼[かれ]はいつも 貧乏[びんぼう]くじを 引[ひ]いている。	お人好し=おひとよし=何事も善意にとらえる傾向があり、他人に利用されたりだまされたりしやすいこと。また、そのさまや、そういう人物。 貧乏くじを引く=びんぼうくじをひく= 
\\	(お人好しで) 
\\	(損をする) 
\\	便に血が混じることはありませんか。	
\\	便[べん]に 血[ち]が 混[ま]じることはありませんか。	便=べん=1)都合が良いこと。      
\\	大便と小便。特に大便。
\\	3日ぶりに便が出た。	
\\	3日[みっか]ぶりに 便[べん]が 出[で]た。	便=べん=1)都合が良いこと。      
\\	大便と小便。特に大便。
\\	私はもう4日間も便が出ない。	
\\	私[わたし]はもう4 日間[にちかん]も 便[べん]が 出[で]ない。	便=べん=1)都合が良いこと。      
\\	大便と小便。特に大便。
\\	この国は鉄道の便はあまり良くない。	
\\	この 国[くに]は 鉄道[てつどう]の 便[べん]はあまり 良[よ]くない。	便=べん=1)都合が良いこと。      
\\	大便と小便。特に大便。
\\	夜間は輸送の便はない。	
\\	夜間[やかん]は 輸送[ゆそう]の 便[べん]はない。	輸送=ゆそう=車・船・航空機などで人や物資を運ぶこと。便=べん=1)都合が良いこと。      
\\	大便と小便。特に大便。
\\	この島は日に二度、本土まで船の便がある。	
\\	この 島[しま]は 日[ひ]に 二度[にど]、 本土[ほんど]まで 船[ふね]の 便[べん]がある。	便=べん=1)都合が良いこと。      
\\	大便と小便。特に大便。
\\	この住宅地は買い物の便がいい。	
\\	この 住宅[じゅうたく] 地[ち]は 買い物[かいもの]の 便[べん]がいい。	便=べん=1)都合が良いこと。      
\\	大便と小便。特に大便。
\\	寄付集めは一人当たり3000円を目安にした。	
\\	寄付[きふ] 集[あつ]めは 一人[ひとり] 当[あ]たり3000 円[えん]を 目安[めやす]にした。	目安=めやす=目当て。目標。おおよその基準。また、おおよその見当。
\\	彼女は将来の身の振り方を考えあぐねて占いに走った。	
\\	彼女[かのじょ]は 将来[しょうらい]の 身[み]の 振[ふ]り 方[かた]を 考[かんが]えあぐねて 占[うらな]いに 走[はし]った。	考え倦ねる=かんがえあぐねる= 
\\	デタラメを言うな。	
\\	デタラメを 言[い]うな。	でたらめな=根拠がないこと。首尾一貫しないこと。いいかげんなこと。また、そのさまや、そのような言動。
\\	驚いたことにその報道はデタラメだった。	
\\	驚[おどろ]いたことにその 報道[ほうどう]はデタラメだった。	でたらめな=根拠がないこと。首尾一貫しないこと。いいかげんなこと。また、そのさまや、そのような言動。
\\	あの先生の教え方はでたらめだ。	
\\	あの 先生[せんせい]の 教[おし]え 方[かた]はでたらめだ。	でたらめな=根拠がないこと。首尾一貫しないこと。いいかげんなこと。また、そのさまや、そのような言動。
\\	私がでたらめに選んだ番号が正解だった。	
\\	私[わたし]がでたらめに 選[えら]んだ 番号[ばんごう]が 正解[せいかい]だった。	でたらめな=根拠がないこと。首尾一貫しないこと。いいかげんなこと。また、そのさまや、そのような言動。
\\	彼女の言っていることは全部でたらめだ。	
\\	彼女[かのじょ]の 言[い]っていることは 全部[ぜんぶ]でたらめだ。	でたらめな=根拠がないこと。首尾一貫しないこと。いいかげんなこと。また、そのさまや、そのような言動。
\\	この報告書はまったくのでたらめだ。	
\\	この 報告[ほうこく] 書[しょ]はまったくのでたらめだ。	でたらめな=根拠がないこと。首尾一貫しないこと。いいかげんなこと。また、そのさまや、そのような言動。
\\	あいつは本当にでたらめなやつだ。	
\\	あいつは 本当[ほんとう]にでたらめなやつだ。	でたらめな=根拠がないこと。首尾一貫しないこと。いいかげんなこと。また、そのさまや、そのような言動。
\\	彼は泥沼から這い上がろうと必死であがいた。	
\\	彼[かれ]は 泥沼[どろぬま]から 這[は]い 上[あ]がろうと 必死[ひっし]であがいた。	足掻く=あがく= (もがく) 
\\	どうあがいても君があの名人に勝てるはずがない。	
\\	どうあがいても 君[きみ]があの 名人[めいじん]に 勝[か]てるはずがない。	足掻く=あがく= (もがく) 
\\	今さらあがいても無駄だ。	
\\	今[いま]さらあがいても 無駄[むだ]だ。	足掻く=あがく= (もがく) 
\\	毎度ご面倒をかけてすみません。	
\\	毎度[まいど]ご 面倒[めんどう]をかけてすみません。	毎度=まいど=1)同じことが繰り返されること。そのたびごと。        
\\	副詞的に用いて
\\	いつも。
\\	パスワードを毎度毎度入力するのは面倒だ。	
\\	パスワードを 毎度[まいど] 毎度[まいど] 入力[にゅうりょく]するのは 面倒[めんどう]だ。	毎度=まいど=1)同じことが繰り返されること。そのたびごと。        
\\	副詞的に用いて
\\	いつも。
\\	毎度のことながら彼の話は面白い。	
\\	毎度[まいど]のことながら 彼[かれ]の 話[はなし]は 面白[おもしろ]い。	毎度=まいど=1)同じことが繰り返されること。そのたびごと。        
\\	副詞的に用いて
\\	いつも。
\\	昨日は入れ替わり立ち替わり来客が絶えなかった。	
\\	昨日[きのう]は 入れ替[いれか]わり 立[た]ち 替[が]わり 来客[らいきゃく]が 絶[た]えなかった。	入れ替わり立ち替わり=いれかわりたちかわり= 
\\	店には2、3人連れの女子高生が入れ替わり立ち替わり冷やかしにきたものだ。	
\\	店[みせ]には2、3 人[にん] 連[づ]れの 女子高[じょしこう] 生[せい]が 入れ替[いれか]わり 立[た]ち 替[が]わり 冷[ひ]やかしにきたものだ。	入れ替わり立ち替わり=いれかわりたちかわり= 
\\	冷やかし・素見し=ひやかし=1)冗談などを言ってからかうこと。               
\\	買う気がないのに商品を見て回ること。また、その人。
\\	学生相手の賃貸マンションは住人の入れ替わりが激しい。	
\\	学生[がくせい] 相手[あいて]の 賃貸[ちんたい]マンションは 住人[じゅうにん]の 入れ替[いれか]わりが 激[はげ]しい。	入れ替わり=いれかわり=入れ替わること。交代。いりかわり。
\\	私たちの合唱団は一人の入れ替わりもなくこれまで20年やってまいりました。	
\\	私[わたし]たちの 合唱[がっしょう] 団[だん]は 一人[ひとり]の 入れ替[いれか]わりもなくこれまで20 年[ねん]やってまいりました。	入れ替わり=いれかわり=入れ替わること。交代。いりかわり。
\\	ボブと入れ替わりにアレックスがシャワー室へ入った。	
\\	ボブと 入れ替[いれか]わりにアレックスがシャワー 室[しつ]へ 入[はい]った。	入れ替わり=いれかわり=入れ替わること。交代。いりかわり。
\\	1日のうちに3回も出会うなんて、今日はあなたとご縁があるんですね。	
\\	1日[いちにち]のうちに3 回[かい]も 出会[であ]うなんて、 今日[きょう]はあなたとご 縁[えん]があるんですね。	縁=えん= 
\\	巡り合わせ
\\	きっかけ
\\	かかわり
\\	家族などとのつながり
\\	縁があったらまた会いましょう。	
\\	縁[えん]があったらまた 会[あ]いましょう。	縁=えん= 
\\	巡り合わせ
\\	きっかけ
\\	かかわり
\\	家族などとのつながり
\\	小学校からずっと一緒とは君たちとはよほど縁が深いんだな。	
\\	小学校[しょうがっこう]からずっと 一緒[いっしょ]とは 君[きみ]たちとはよほど 縁[えん]が 深[ふか]いんだな。	縁=えん= 
\\	巡り合わせ
\\	きっかけ
\\	かかわり
\\	家族などとのつながり
\\	卒業してようやく学校と縁が切れた。	
\\	卒業[そつぎょう]してようやく 学校[がっこう]と 縁[えん]が 切[き]れた。	縁=えん= 
\\	巡り合わせ
\\	きっかけ
\\	かかわり
\\	家族などとのつながり
\\	この病気とはなかなか縁が切れない。	
\\	この 病気[びょうき]とはなかなか 縁[えん]が 切[き]れない。	縁=えん= 
\\	巡り合わせ
\\	きっかけ
\\	かかわり
\\	家族などとのつながり
\\	彼とは飛行機で隣り合わせたのが縁で親しくなった。	
\\	彼[かれ]とは 飛行機[ひこうき]で 隣り合[となりあ]わせたのが 縁[えん]で 親[した]しくなった。	縁=えん= 
\\	巡り合わせ
\\	きっかけ
\\	かかわり
\\	家族などとのつながり
\\	彼女とは母校につらなる縁で今も交流がある。	
\\	彼女[かのじょ]とは 母校[ぼこう]につらなる 縁[えん]で 今[いま]も 交流[こうりゅう]がある。	縁=えん= 
\\	巡り合わせ
\\	きっかけ
\\	かかわり
\\	家族などとのつながり
\\	私たちは切っても切れない縁で結ばれている。	
\\	私[わたし]たちは 切[き]っても 切[き]れない 縁[えん]で 結[むす]ばれている。	縁=えん= 
\\	巡り合わせ
\\	きっかけ
\\	かかわり
\\	家族などとのつながり
\\	これをご縁にこれからもお目にかかりたいものです。	
\\	これをご 縁[えん]にこれからもお 目[め]にかかりたいものです。	縁=えん= 
\\	巡り合わせ
\\	きっかけ
\\	かかわり
\\	家族などとのつながり
\\	近ごろは読書とは縁のない生活をしている。	
\\	近[ちか]ごろは 読書[どくしょ]とは 縁[えん]のない 生活[せいかつ]をしている。	縁=えん= 
\\	巡り合わせ
\\	きっかけ
\\	かかわり
\\	家族などとのつながり
\\	私はこの町と浅からぬ縁を持っています。	
\\	私[わたし]はこの 町[まち]と 浅[あさ]からぬ 縁[えん]を 持[も]っています。	縁=えん= 
\\	巡り合わせ
\\	きっかけ
\\	かかわり
\\	家族などとのつながり
\\	縁あって二人は夫婦になった。	
\\	縁[えん]あって 二人[ふたり]は 夫婦[ふうふ]になった。	縁=えん= 
\\	巡り合わせ
\\	きっかけ
\\	かかわり
\\	家族などとのつながり
\\	彼女は私には縁もゆかりもない人物だ。	
\\	彼女[かのじょ]は 私[わたし]には 縁[えん]もゆかりもない 人物[じんぶつ]だ。	縁=えん= 
\\	巡り合わせ
\\	きっかけ
\\	かかわり
\\	家族などとのつながり
\\	これも何かの縁でしょう。	
\\	これも 何[なに]かの 縁[えん]でしょう。	縁=えん= 
\\	巡り合わせ
\\	きっかけ
\\	かかわり
\\	家族などとのつながり
\\	彼は縁もゆかりもない土地に転勤になった。	
\\	彼[かれ]は 縁[えん]もゆかりもない 土地[とち]に 転勤[てんきん]になった。	縁=えん= 
\\	巡り合わせ
\\	きっかけ
\\	かかわり
\\	家族などとのつながり
\\	事故の瞬間をまともに見てしまったショックで、しばらくは毎晩うなされた。	
\\	事故[じこ]の 瞬間[しゅんかん]をまともに 見[み]てしまったショックで、しばらくは 毎晩[まいばん]うなされた。	まとも=1)まっすぐに向かい合うこと。正しく向かい合うこと。また、そのさま。真正面。     
\\	策略や駆け引きをしないこと。また、そのさま。     
\\	まじめなこと。正当であること。また、そのさま。
\\	なんだか後ろめたくて彼の顔をまともに見られなかった。	
\\	なんだか 後[うし]ろめたくて 彼[かれ]の 顔[かお]をまともに 見[み]られなかった。	まとも=1)まっすぐに向かい合うこと。正しく向かい合うこと。また、そのさま。真正面。     
\\	策略や駆け引きをしないこと。また、そのさま。     
\\	まじめなこと。正当であること。また、そのさま。
\\	車はまともにダンプにぶつかって大破した。	
\\	車[くるま]はまともにダンプにぶつかって 大破[たいは]した。	まとも=1)まっすぐに向かい合うこと。正しく向かい合うこと。また、そのさま。真正面。     
\\	策略や駆け引きをしないこと。また、そのさま。     
\\	まじめなこと。正当であること。また、そのさま。
\\	人前でまともにほめられて少しきまりが悪かった。	
\\	人前[ひとまえ]でまともにほめられて 少[すこ]しきまりが 悪[わる]かった。	まとも=1)まっすぐに向かい合うこと。正しく向かい合うこと。また、そのさま。真正面。     
\\	策略や駆け引きをしないこと。また、そのさま。     
\\	まじめなこと。正当であること。また、そのさま。
\\	まともな人間ならそんなことはすまい。	
\\	まともな 人間[にんげん]ならそんなことはすまい。	まとも=1)まっすぐに向かい合うこと。正しく向かい合うこと。また、そのさま。真正面。     
\\	策略や駆け引きをしないこと。また、そのさま。     
\\	まじめなこと。正当であること。また、そのさま。
\\	まともな人間ならこんな危ない仕事に手は出さないだろう。	
\\	まともな 人間[にんげん]ならこんな 危[あぶ]ない 仕事[しごと]に 手[て]は 出[だ]さないだろう。	まとも=1)まっすぐに向かい合うこと。正しく向かい合うこと。また、そのさま。真正面。     
\\	策略や駆け引きをしないこと。また、そのさま。     
\\	まじめなこと。正当であること。また、そのさま。
\\	三日間まともなものを食べていない。	
\\	三日間[みっかかん]まともなものを 食[た]べていない。	まとも=1)まっすぐに向かい合うこと。正しく向かい合うこと。また、そのさま。真正面。     
\\	策略や駆け引きをしないこと。また、そのさま。     
\\	まじめなこと。正当であること。また、そのさま。
\\	顔面に彼のパンチをまともに食らった。	
\\	顔面[がんめん]に 彼[かれ]のパンチをまともに 食[く]らった。	まとも=1)まっすぐに向かい合うこと。正しく向かい合うこと。また、そのさま。真正面。     
\\	策略や駆け引きをしないこと。また、そのさま。     
\\	まじめなこと。正当であること。また、そのさま。
\\	この記事は内容が貧弱だ。	
\\	この 記事[きじ]は 内容[ないよう]が 貧弱[ひんじゃく]だ。	貧弱=ひんじゃく=1)みすぼらしく、見劣りのすること。また、そのさま。          
\\	乏しく、必要を満たすに十分でないこと。また、そのさま。
\\	あの男は発想も理解力も極めて貧弱だ。	
\\	あの 男[おとこ]は 発想[はっそう]も 理解[りかい] 力[りょく]も 極[きわ]めて 貧弱[ひんじゃく]だ。	貧弱=ひんじゃく=1)みすぼらしく、見劣りのすること。また、そのさま。          
\\	乏しく、必要を満たすに十分でないこと。また、そのさま。
\\	贈り物も包装が悪いと貧弱に見える。	
\\	贈り物[おくりもの]も 包装[ほうそう]が 悪[わる]いと 貧弱[ひんじゃく]に 見[み]える。	貧弱=ひんじゃく=1)みすぼらしく、見劣りのすること。また、そのさま。          
\\	乏しく、必要を満たすに十分でないこと。また、そのさま。
\\	貧乏揺すりをやめてくれ。	
\\	貧乏揺[びんぼうゆ]すりをやめてくれ。	貧乏揺すり=びんぼうゆすり=座っているとき、絶えず膝を細かく揺り動かすこと。
\\	私は彼女の意見に全面的に共鳴している。	
\\	私[わたし]は 彼女[かのじょ]の 意見[いけん]に 全面[ぜんめん] 的[てき]に 共鳴[きょうめい]している。	共鳴=きょうめい=他人の考えや行動などに心から同感すること。
\\	二人は趣味の点で大いに共鳴するところがあった。	
\\	二人[ふたり]は 趣味[しゅみ]の 点[てん]で 大[おお]いに 共鳴[きょうめい]するところがあった。	共鳴=きょうめい=他人の考えや行動などに心から同感すること。
\\	日本経済の主要な担い手の一つは自動車産業だ。	
\\	日本[にほん] 経済[けいざい]の 主要[しゅよう]な 担い手[にないて]の 一[ひと]つは 自動車[じどうしゃ] 産業[さんぎょう]だ。	担い手=にないて=1)荷物をかつぐ人。          
\\	中心となってある事柄を支え、推し進めていく人。
\\	彼は一家の担い手だ。	
\\	彼[かれ]は 一家[いっか]の 担い手[にないて]だ。	担い手=にないて=1)荷物をかつぐ人。          
\\	中心となってある事柄を支え、推し進めていく人。
\\	お母さんにとってはがらくたでも、私にとっては宝物なの。	
\\	お 母[かあ]さんにとってはがらくたでも、 私[わたし]にとっては 宝物[たからもの]なの。	がらくた= 
\\	そんながらくたを集めてどうするの。	
\\	そんながらくたを 集[あつ]めてどうするの。	がらくた= 
\\	引き出しはがらくたでいっぱいだ。	
\\	引き出[ひきだ]しはがらくたでいっぱいだ。	がらくた= 
\\	見てみろよ、あのおっさん居眠りしてるぜ。	
\\	見[み]てみろよ、あのおっさん 居眠[いねむ]りしてるぜ。	おっさん= 
\\	いちいち口うるさいおっさんだなあ。	
\\	いちいち 口[くち]うるさいおっさんだなあ。	おっさん= 
\\	騒動が持ち上がった。	
\\	騒動[そうどう]が 持ち上[もちあ]がった。	騒動=そうどう=1)多人数が騒ぎ立てて秩序が乱れること。また、そのような事件や事態。         
\\	もめごと。争い。
\\	騒動が収まった。	
\\	騒動[そうどう]が 収[おさ]まった。	騒動=そうどう=1)多人数が騒ぎ立てて秩序が乱れること。また、そのような事件や事態。         
\\	もめごと。争い。
\\	あの学校は騒動が絶えない。	
\\	あの 学校[がっこう]は 騒動[そうどう]が 絶[た]えない。	騒動=そうどう=1)多人数が騒ぎ立てて秩序が乱れること。また、そのような事件や事態。         
\\	もめごと。争い。
\\	ひと騒動持ち上がりそうだ。	
\\	ひと 騒動[そうどう] 持ち上[もちあ]がりそうだ。	騒動=そうどう=1)多人数が騒ぎ立てて秩序が乱れること。また、そのような事件や事態。         
\\	もめごと。争い。
\\	そうなったらひと騒動だぞ。	
\\	そうなったらひと 騒動[そうどう]だぞ。	騒動=そうどう=1)多人数が騒ぎ立てて秩序が乱れること。また、そのような事件や事態。         
\\	もめごと。争い。
\\	スタジアムは大騒動になった。	
\\	スタジアムは 大[だい] 騒動[そうどう]になった。	騒動=そうどう=1)多人数が騒ぎ立てて秩序が乱れること。また、そのような事件や事態。         
\\	もめごと。争い。
\\	その客のもてなしで大騒動だった。	
\\	その 客[きゃく]のもてなしで 大[だい] 騒動[そうどう]だった。	騒動=そうどう=1)多人数が騒ぎ立てて秩序が乱れること。また、そのような事件や事態。         
\\	もめごと。争い。
\\	ブログ上での発言が大きな騒動に発展した。	
\\	ブログ 上[じょう]での 発言[はつげん]が 大[おお]きな 騒動[そうどう]に 発展[はってん]した。	騒動=そうどう=1)多人数が騒ぎ立てて秩序が乱れること。また、そのような事件や事態。         
\\	もめごと。争い。
\\	いやはや物騒の世の中だ。	
\\	いやはや 物騒[ぶっそう]の 世の中[よのなか]だ。	物騒=ぶっそう=1)よくないことが起きたり起こしたりしそうな危険な感じがすること。また、そのさま。         
\\	ざわざわとして落ち着かないこと。また、そのさま。         
\\	落ち着きがなく、そそっかしいこと。また、そのさま。
\\	気持ちはわかるけど、学校なんて燃えてしまえなんて、そんな物騒なことを言うものじゃないよ。	
\\	気持[きも]ちはわかるけど、 学校[がっこう]なんて 燃[も]えてしまえなんて、そんな 物騒[ぶっそう]なことを 言[い]うものじゃないよ。	物騒=ぶっそう=1)よくないことが起きたり起こしたりしそうな危険な感じがすること。また、そのさま。         
\\	ざわざわとして落ち着かないこと。また、そのさま。         
\\	落ち着きがなく、そそっかしいこと。また、そのさま。
\\	(刃物を持っている人に)そんな物騒なものを持って、どこへ行く気だい。	
\\	刃物[はもの]を 持[も]っている 人[ひと]に)そんな 物騒[ぶっそう]なものを 持[も]って、どこへ 行[い]く 気[き]だい。	物騒=ぶっそう=1)よくないことが起きたり起こしたりしそうな危険な感じがすること。また、そのさま。         
\\	ざわざわとして落ち着かないこと。また、そのさま。         
\\	落ち着きがなく、そそっかしいこと。また、そのさま。
\\	女性の夜の一人歩きは物騒だ。	
\\	女性[じょせい]の 夜[よる]の 一人[ひとり] 歩[ある]きは 物騒[ぶっそう]だ。	物騒=ぶっそう=1)よくないことが起きたり起こしたりしそうな危険な感じがすること。また、そのさま。         
\\	ざわざわとして落ち着かないこと。また、そのさま。         
\\	落ち着きがなく、そそっかしいこと。また、そのさま。
\\	この辺は物騒だ。	
\\	この 辺[へん]は 物騒[ぶっそう]だ。	物騒=ぶっそう=1)よくないことが起きたり起こしたりしそうな危険な感じがすること。また、そのさま。         
\\	ざわざわとして落ち着かないこと。また、そのさま。         
\\	落ち着きがなく、そそっかしいこと。また、そのさま。
\\	ますます物騒な世の中になってきた。	
\\	ますます 物騒[ぶっそう]な 世の中[よのなか]になってきた。	物騒=ぶっそう=1)よくないことが起きたり起こしたりしそうな危険な感じがすること。また、そのさま。         
\\	ざわざわとして落ち着かないこと。また、そのさま。         
\\	落ち着きがなく、そそっかしいこと。また、そのさま。
\\	なんだか表が騒々しいな。	
\\	なんだか 表[おもて]が 騒々[そうぞう]しいな。	騒々しい=そうぞうしい=1)物音や人声が多くてうるさい。騒がしい。             
\\	大きな事件が続いて起こるなどして落ち着かない。
\\	あのテロ事件以来世の中が騒々しくなってきた。	
\\	あのテロ 事件[じけん] 以来[いらい] 世の中[よのなか]が 騒々[そうぞう]しくなってきた。	騒々しい=そうぞうしい=1)物音や人声が多くてうるさい。騒がしい。             
\\	大きな事件が続いて起こるなどして落ち着かない。
\\	突然の社長の辞任に社内は騒然となった。	
\\	突然[とつぜん]の 社長[しゃちょう]の 辞任[じにん]に 社内[しゃない]は 騒然[そうぜん]となった。	騒然=そうぜん=ざわざわと騒がしいさま。また、不穏で落ち着かないさま。
\\	驚異的な記録が生まれました。	
\\	驚異[きょうい] 的[てき]な 記録[きろく]が 生[う]まれました。	驚異的な=きょういてきな=驚くほど素晴らしいさま。
\\	彼女の英語があんな短期間であれだけ上達したのは驚異的だ。	
\\	彼女[かのじょ]の 英語[えいご]があんな 短期間[たんきかん]であれだけ 上達[じょうたつ]したのは 驚異[きょうい] 的[てき]だ。	驚異的な=きょういてきな=驚くほど素晴らしいさま。
\\	彼女は容姿がすぐれている。	
\\	彼女[かのじょ]は 容姿[ようし]がすぐれている。	容姿=ようし=顔だちと体つき。すがたかたち。
\\	彼は容姿が立派だ。	
\\	彼[かれ]は 容姿[ようし]が 立派[りっぱ]だ。	容姿=ようし=顔だちと体つき。すがたかたち。
\\	結婚相手を選ぶなら容姿は二の次で人間性が問題だと思う。	
\\	結婚[けっこん] 相手[あいて]を 選[えら]ぶなら 容姿[ようし]は 二の次[にのつぎ]で 人間[にんげん] 性[せい]が 問題[もんだい]だと 思[おも]う。	容姿=ようし=顔だちと体つき。すがたかたち。
\\	彼女は容姿は抜群だが、演技力はまだ未熟だ。	
\\	彼女[かのじょ]は 容姿[ようし]は 抜群[ばつぐん]だが、 演技[えんぎ] 力[りょく]はまだ 未熟[みじゅく]だ。	容姿=ようし=顔だちと体つき。すがたかたち。
\\	彼女は容姿に恵まれている。	
\\	彼女[かのじょ]は 容姿[ようし]に 恵[めぐ]まれている。	容姿=ようし=顔だちと体つき。すがたかたち。
\\	表が騒がしい。何事だろう。	
\\	表[おもて]が 騒[さわ]がしい。 何事[なにごと]だろう。	騒がしい=さわがしい=1)盛んに声や物音がしてうるさい。騒々しい。            
\\	事件などが起こって世情が落ち着かない。平静。平穏でない。            
\\	ことが多く忙しい。あわただしい。            
\\	ごたごたしている。乱雑である。
\\	近ごろは何となく世の中が騒がしい。	
\\	近[ちか]ごろは 何[なん]となく 世の中[よのなか]が 騒[さわ]がしい。	騒がしい=さわがしい=1)盛んに声や物音がしてうるさい。騒々しい。            
\\	事件などが起こって世情が落ち着かない。平静。平穏でない。            
\\	ことが多く忙しい。あわただしい。            
\\	ごたごたしている。乱雑である。
\\	隣の犬がいつになく騒がしく鳴いている。	
\\	隣[となり]の 犬[いぬ]がいつになく 騒[さわ]がしく 鳴[な]いている。	騒がしい=さわがしい=1)盛んに声や物音がしてうるさい。騒々しい。            
\\	事件などが起こって世情が落ち着かない。平静。平穏でない。            
\\	ことが多く忙しい。あわただしい。            
\\	ごたごたしている。乱雑である。
\\	何という騒がしさだ。	
\\	何[なに]という 騒[さわ]がしさだ。	騒がしい=さわがしい=1)盛んに声や物音がしてうるさい。騒々しい。            
\\	事件などが起こって世情が落ち着かない。平静。平穏でない。            
\\	ことが多く忙しい。あわただしい。            
\\	ごたごたしている。乱雑である。
\\	こんなにまわりが騒がしくては勉強に集中できない。	
\\	こんなにまわりが 騒[さわ]がしくては 勉強[べんきょう]に 集中[しゅうちゅう]できない。	騒がしい=さわがしい=1)盛んに声や物音がしてうるさい。騒々しい。            
\\	事件などが起こって世情が落ち着かない。平静。平穏でない。            
\\	ことが多く忙しい。あわただしい。            
\\	ごたごたしている。乱雑である。
\\	来日したその女優は片言の日本語で愛嬌を振りまいた。	
\\	来日[らいにち]したその 女優[じょゆう]は 片言[かたこと]の 日本語[にほんご]で 愛嬌[あいきょう]を 振[ふ]りまいた。	愛嬌=あいきょう=1)にこやかで、かわいらしいこと。          
\\	ひょうきんで、憎めない表情・しぐさ。          
\\	相手を喜ばせるような言葉・降るまい。
\\	こんなふうにミスが重なるとご愛嬌ですまなくなる。	
\\	こんなふうにミスが 重[かさ]なるとご 愛嬌[あいきょう]ですまなくなる。	愛嬌=あいきょう=1)にこやかで、かわいらしいこと。          
\\	ひょうきんで、憎めない表情・しぐさ。          
\\	相手を喜ばせるような言葉・降るまい。
\\	選挙直前とあって、その町に行くと候補者は盛んに愛嬌を振りまいていた。	
\\	選挙[せんきょ] 直前[ちょくぜん]とあって、その 町[まち]に 行[い]くと 候補[こうほ] 者[しゃ]は 盛[さか]んに 愛嬌[あいきょう]を 振[ふ]りまいていた。	愛嬌=あいきょう=1)にこやかで、かわいらしいこと。          
\\	ひょうきんで、憎めない表情・しぐさ。          
\\	相手を喜ばせるような言葉・降るまい。
\\	彼女はいつも愛嬌を振りまいている。	
\\	彼女[かのじょ]はいつも 愛嬌[あいきょう]を 振[ふ]りまいている。	愛嬌=あいきょう=1)にこやかで、かわいらしいこと。          
\\	ひょうきんで、憎めない表情・しぐさ。          
\\	相手を喜ばせるような言葉・降るまい。
\\	彼女は愛嬌がある娘だ。	
\\	彼女[かのじょ]は 愛嬌[あいきょう]がある 娘[むすめ]だ。	愛嬌=あいきょう=1)にこやかで、かわいらしいこと。          
\\	ひょうきんで、憎めない表情・しぐさ。          
\\	相手を喜ばせるような言葉・降るまい。
\\	ちまたにはカタカナ語が氾濫している。	
\\	ちまたにはカタカナ 語[ご]が 氾濫[はんらん]している。	巷=ちまた= (分かれ道) 
\\	(街路) 
\\	(場所) 
\\	氾濫=はんらん= (水などがあふれること) 
\\	(洪水); (たくさん満ちあふれていること) 
\\	巷では彼は近く政界を引退するとうわさされている。	
\\	巷[ちまた]では 彼[かれ]は 近[ちか]く 政界[せいかい]を 引退[いんたい]するとうわさされている。	巷=ちまた= (分かれ道) 
\\	(街路) 
\\	(場所) 
\\	政治家は巷の声にもっと耳を傾けるべきだ。	
\\	政治[せいじ] 家[か]は 巷[ちまた]の 声[こえ]にもっと 耳[みみ]を 傾[かたむ]けるべきだ。	巷=ちまた= (分かれ道) 
\\	(街路) 
\\	(場所) 
\\	彼のにこやかな顔が急に険しくなった。	
\\	彼[かれ]のにこやかな 顔[かお]が 急[きゅう]に 険[けわ]しくなった。	にこやかな=1)微笑みを浮かべるさま。にこにこしているさま。       
\\	物腰・筆跡などの、柔らかいさま。
\\	彼女はにこやかに私を迎えてくれた。	
\\	彼女[かのじょ]はにこやかに 私[わたし]を 迎[むか]えてくれた。	にこやかな=1)微笑みを浮かべるさま。にこにこしているさま。       
\\	物腰・筆跡などの、柔らかいさま。
\\	その少女の仕種がかわいらしい。	
\\	その 少女[しょうじょ]の 仕種[しぐさ]がかわいらしい。	仕種=しぐさ=1)何かをするときのちょっとした動作や身のこなし。        
\\	舞台上の俳優の動作や表情。所作。
\\	女の子は男の子と仕種が違う。	
\\	女の子[おんなのこ]は 男の子[おとこのこ]と 仕種[しぐさ]が 違[ちが]う。	仕種=しぐさ=1)何かをするときのちょっとした動作や身のこなし。        
\\	舞台上の俳優の動作や表情。所作。
\\	子供は大人の仕種を真似したがる。	
\\	子供[こども]は 大人[おとな]の 仕種[しぐさ]を 真似[まね]したがる。	仕種=しぐさ=1)何かをするときのちょっとした動作や身のこなし。        
\\	舞台上の俳優の動作や表情。所作。
\\	彼は私たちに入室するようしぐさで促した。	
\\	彼[かれ]は 私[わたし]たちに 入室[にゅうしつ]するようしぐさで 促[うなが]した。	仕種=しぐさ=1)何かをするときのちょっとした動作や身のこなし。        
\\	舞台上の俳優の動作や表情。所作。
\\	この一週間、政界の動きがあわただしい。	
\\	この 一週間[いっしゅうかん]、 政界[せいかい]の 動[うご]きがあわただしい。	慌ただしい=あわただしい=1)物事をしようとしてしきりにせきたてられるさま。落ち着かなくせわしいさま。              
\\	状況の移り変わりが急で、一定しないさま。
\\	今年も一年間慌ただしかった。	
\\	今年[ことし]も 一年間[いちねんかん] 慌[あわ]ただしかった。	慌ただしい=あわただしい=1)物事をしようとしてしきりにせきたてられるさま。落ち着かなくせわしいさま。              
\\	状況の移り変わりが急で、一定しないさま。
\\	よく閉店間際のデパートに慌ただしく駆け込んで夕食の買い物をする。	
\\	よく 閉店[へいてん] 間際[まぎわ]のデパートに 慌[あわ]ただしく 駆け込[かけこ]んで 夕食[ゆうしょく]の 買い物[かいもの]をする。	慌ただしい=あわただしい=1)物事をしようとしてしきりにせきたてられるさま。落ち着かなくせわしいさま。              
\\	状況の移り変わりが急で、一定しないさま。
\\	結婚式の準備で毎日が慌ただしい。	
\\	結婚式[けっこんしき]の 準備[じゅんび]で 毎日[まいにち]が 慌[あわ]ただしい。	慌ただしい=あわただしい=1)物事をしようとしてしきりにせきたてられるさま。落ち着かなくせわしいさま。              
\\	状況の移り変わりが急で、一定しないさま。
\\	急ぎ立ててすみません。	
\\	急[いそ]ぎ 立[た]ててすみません。	急き立てる=せきたてる=物事を早く行うように強く催促する。急がせる。
\\	彼は早く学校ヘ行けと息子をせきたてた。	
\\	彼[かれ]は 早[はや]く 学校[がっこう]ヘ 行[い]けと 息子[むすこ]をせきたてた。	急き立てる=せきたてる=物事を早く行うように強く催促する。急がせる。
\\	彼は借金の返金をせきたてられた。	
\\	彼[かれ]は 借金[しゃっきん]の 返金[へんきん]をせきたてられた。	急き立てる=せきたてる=物事を早く行うように強く催促する。急がせる。
\\	この焼酎の度数は30度だ。	
\\	この 焼酎[しょうちゅう]の 度数[どすう]は30 度[ど]だ。	度数=どすう=1)回数。頻度。        
\\	度合いを表す数値。        
\\	資料の整理をする場合、全体をいくつかの範囲に分けたとき、それぞれの範囲内の個数のこと。
\\	その品はもうおしまいになりました。	
\\	その 品[しな]はもうおしまいになりました。	お終い=おしまい=1)終わること。『夏休みも今日で〜だ』          
\\	物事がだめになること。また、非常に悪い状態になること。
\\	仕事をお終いにしよう。	
\\	仕事[しごと]をお 終[おわり]いにしよう。	お終い=おしまい=1)終わること。『夏休みも今日で〜だ』          
\\	物事がだめになること。また、非常に悪い状態になること。
\\	「私たち、もうおしまいにしない?」と別れ話を持ちかけてきたのは彼女だった。	
\\	私[わたし]たち、もうおしまいにしない?」と 別れ話[わかればなし]を 持[も]ちかけてきたのは 彼女[かのじょ]だった。	お終い=おしまい=1)終わること。『夏休みも今日で〜だ』          
\\	物事がだめになること。また、非常に悪い状態になること。
\\	これでおしまい。	
\\	これでおしまい。	お終い=おしまい=1)終わること。『夏休みも今日で〜だ』          
\\	物事がだめになること。また、非常に悪い状態になること。
\\	まだおしまいではない。	
\\	まだおしまいではない。	お終い=おしまい=1)終わること。『夏休みも今日で〜だ』          
\\	物事がだめになること。また、非常に悪い状態になること。
\\	話はそれでおしまい。	
\\	話[はなし]はそれでおしまい。	お終い=おしまい=1)終わること。『夏休みも今日で〜だ』          
\\	物事がだめになること。また、非常に悪い状態になること。
\\	今日の仕事はもうこれでおしまいだ。	
\\	今日[きょう]の 仕事[しごと]はもうこれでおしまいだ。	お終い=おしまい=1)終わること。『夏休みも今日で〜だ』          
\\	物事がだめになること。また、非常に悪い状態になること。
\\	あの男ももうおしまいだ。	
\\	あの 男[おとこ]ももうおしまいだ。	お終い=おしまい=1)終わること。『夏休みも今日で〜だ』          
\\	物事がだめになること。また、非常に悪い状態になること。
\\	黙って私の話をおしまいまで聞いてください。	
\\	黙[だま]って 私[わたし]の 話[はなし]をおしまいまで 聞[き]いてください。	お終い=おしまい=1)終わること。『夏休みも今日で〜だ』          
\\	物事がだめになること。また、非常に悪い状態になること。
\\	彼女はその難問を解いて全ての人々を驚嘆させた。	
\\	彼女[かのじょ]はその 難問[なんもん]を 解[と]いて 全[すべ]ての 人々[ひとびと]を 驚嘆[きょうたん]させた。	驚嘆=きょうたん=素晴らしい出来事や、思いも及ばない物事に接して、驚き感心すること。<英:
\\	彼の日本古典の造詣の深さは驚嘆に値する。	
\\	彼[かれ]の 日本[にっぽん] 古典[こてん]の 造詣[ぞうけい]の 深[ふか]さは 驚嘆[きょうたん]に 値[あたい]する。	驚嘆=きょうたん=素晴らしい出来事や、思いも及ばない物事に接して、驚き感心すること。<英:
\\	最先端の情報技術には驚嘆するばかりだ。	
\\	最先端[さいせんたん]の 情報[じょうほう] 技術[ぎじゅつ]には 驚嘆[きょうたん]するばかりだ。	驚嘆=きょうたん=素晴らしい出来事や、思いも及ばない物事に接して、驚き感心すること。<英:
\\	ご父君の訃報に接し驚愕に耐えません。	
\\	ご 父君[ふくん]の 訃報[ふほう]に 接[せっ]し 驚愕[きょうがく]に 耐[た]えません。	父君=ちちぎみ=父を敬っていう語。父上。 訃報=ふほう=死去したという知らせ。悲報。驚愕=きょうがく=非常に驚くこと。<例:市中を〜させた事件
\\	その大学に在籍する学生は5000名だ。	
\\	その 大学[だいがく]に 在籍[ざいせき]する 学生[がくせい]は5000 名[めい]だ。	在籍=ざいせき=団体・学校などに属するものとして登録されていること。
\\	その大学には外国人学生が多数在籍している。	
\\	その 大学[だいがく]には 外国[がいこく] 人[じん] 学生[がくせい]が 多数[たすう] 在籍[ざいせき]している。	在籍=ざいせき=団体・学校などに属するものとして登録されていること。
\\	旧友が土産を携えてふらりと現れた。	
\\	旧友[きゅうゆう]が 土産[みやげ]を 携[たずさ]えてふらりと 現[あらわ]れた。	携える=たずさえる=1)手にさげて、また、身につけて持つ。           
\\	連れ立っていく。伴う。           
\\	(「手を携える」の形で)手を取り合う。また、協力する。
\\	彼は銃を携えていた。	
\\	彼[かれ]は 銃[じゅう]を 携[たずさ]えていた。	携える=たずさえる=1)手にさげて、また、身につけて持つ。           
\\	連れ立っていく。伴う。           
\\	(「手を携える」の形で)手を取り合う。また、協力する。
\\	彼はインドとの貿易に携わっていた。	
\\	彼[かれ]はインドとの 貿易[ぼうえき]に 携[たずさ]わっていた。	携わる=たずさわる=1)ある物事に関係する。従事する。           
\\	手を取り合う。連れ立つ。
\\	国政交流イベントにボランティアとして携わった。	
\\	国政[こくせい] 交流[こうりゅう]イベントにボランティアとして 携[たずさ]わった。	携わる=たずさわる=1)ある物事に関係する。従事する。           
\\	手を取り合う。連れ立つ。
\\	民間企業が連携して新しいシステムの開発を進めている。	
\\	民間[みんかん] 企業[きぎょう]が 連携[れんけい]して 新[あたら]しいシステムの 開発[かいはつ]を 進[すす]めている。	連携=れんけい=互いに連絡を取り協力して物事を行うこと。
\\	新党の連携は考えていない。	
\\	新党[しんとう]の 連携[れんけい]は 考[かんが]えていない。	連携=れんけい=互いに連絡を取り協力して物事を行うこと。
\\	そのホテルは地元のタクシー会社と提携している。	
\\	そのホテルは 地元[じもと]のタクシー 会社[かいしゃ]と 提携[ていけい]している。	提携=ていけい=1)互いに助け合うこと。共同で物事を行うこと。タイアップ。         
\\	手に持つこと。たずさえること。
\\	我が社はドイツの企業と提携している。	
\\	我[わ]が 社[しゃ]はドイツの 企業[きぎょう]と 提携[ていけい]している。	提携=ていけい=1)互いに助け合うこと。共同で物事を行うこと。タイアップ。         
\\	手に持つこと。たずさえること。
\\	彼はひいきチームの勝ち負けに一喜一憂している。	
\\	彼[かれ]はひいきチームの 勝ち負[かちま]けに 一喜一憂[いっきいちゆう]している。	一喜一憂=いっきいちゆう=状況の変化に応じて、喜んだり心配したりすること。
\\	候補者たちは開票速報に一喜一憂の有様だ。	
\\	候補[こうほ] 者[しゃ]たちは 開票[かいひょう] 速報[そくほう]に 一喜一憂[いっきいちゆう]の 有様[ありさま]だ。	一喜一憂=いっきいちゆう=状況の変化に応じて、喜んだり心配したりすること。
\\	息子の病状に一喜一憂の状態です。	
\\	息子[むすこ]の 病状[びょうじょう]に 一喜一憂[いっきいちゆう]の 状態[じょうたい]です。	一喜一憂=いっきいちゆう=状況の変化に応じて、喜んだり心配したりすること。
\\	試合の経過に一喜一憂した。	
\\	試合[しあい]の 経過[けいか]に 一喜一憂[いっきいちゆう]した。	一喜一憂=いっきいちゆう=状況の変化に応じて、喜んだり心配したりすること。
\\	白昼堂々彼らは銀行強盗を働いた。	
\\	白昼[はくちゅう] 堂々[どうどう] 彼[かれ]らは 銀行[ぎんこう] 強盗[ごうとう]を 働[はたら]いた。	白昼=はくちゅう= 
\\	白昼にこんな痛ましい事件が起きるなんて!	
\\	白昼[はくちゅう]にこんな 痛[いた]ましい 事件[じけん]が 起[お]きるなんて!	白昼=はくちゅう= 
\\	彼らは白昼堂々と美術館から絵を盗んだ。	
\\	彼[かれ]らは 白昼[はくちゅう] 堂々[どうどう]と 美術館[びじゅつかん]から 絵[え]を 盗[ぬす]んだ。	白昼=はくちゅう= 
\\	(硬貨の)表が出たら僕の勝ちだ。	
\\	硬貨[こうか]の) 表[おもて]が 出[で]たら 僕[ぼく]の 勝[か]ちだ。	表=おもて=1)物の二つの面のうち、主だったほう。表面。また、外側。       
\\	他のものより前に位置すること。前面。       
\\	畳・草履・下駄などの表面につけるござ。       
\\	うわべ。外見。       
\\	表向きのこと。おおやけ。公式。正式。       
\\	正面。家の入り口。       
\\	家の外。戸外。また、家の前の通り。
\\	このコートはどちら側を表にしても着られる。	
\\	このコートはどちら 側[がわ]を 表[おもて]にしても 着[き]られる。	表=おもて=1)物の二つの面のうち、主だったほう。表面。また、外側。       
\\	他のものより前に位置すること。前面。       
\\	畳・草履・下駄などの表面につけるござ。       
\\	うわべ。外見。       
\\	表向きのこと。おおやけ。公式。正式。       
\\	正面。家の入り口。       
\\	家の外。戸外。また、家の前の通り。
\\	そのカードの表を見せてください。	
\\	そのカードの 表[おもて]を 見[み]せてください。	表=おもて=1)物の二つの面のうち、主だったほう。表面。また、外側。       
\\	他のものより前に位置すること。前面。       
\\	畳・草履・下駄などの表面につけるござ。       
\\	うわべ。外見。       
\\	表向きのこと。おおやけ。公式。正式。       
\\	正面。家の入り口。       
\\	家の外。戸外。また、家の前の通り。
\\	表に車を待たせてあるんだ。	
\\	表[おもて]に 車[くるま]を 待[ま]たせてあるんだ。	表=おもて=1)物の二つの面のうち、主だったほう。表面。また、外側。       
\\	他のものより前に位置すること。前面。       
\\	畳・草履・下駄などの表面につけるござ。       
\\	うわべ。外見。       
\\	表向きのこと。おおやけ。公式。正式。       
\\	正面。家の入り口。       
\\	家の外。戸外。また、家の前の通り。
\\	子供たちは表で遊んでいる。	
\\	子供[こども]たちは 表[おもて]で 遊[あそ]んでいる。	表=おもて=1)物の二つの面のうち、主だったほう。表面。また、外側。       
\\	他のものより前に位置すること。前面。       
\\	畳・草履・下駄などの表面につけるござ。       
\\	うわべ。外見。       
\\	表向きのこと。おおやけ。公式。正式。       
\\	正面。家の入り口。       
\\	家の外。戸外。また、家の前の通り。
\\	散歩しようと表に出た。	
\\	散歩[さんぽ]しようと 表[おもて]に 出[で]た。	表=おもて=1)物の二つの面のうち、主だったほう。表面。また、外側。       
\\	他のものより前に位置すること。前面。       
\\	畳・草履・下駄などの表面につけるござ。       
\\	うわべ。外見。       
\\	表向きのこと。おおやけ。公式。正式。       
\\	正面。家の入り口。       
\\	家の外。戸外。また、家の前の通り。
\\	掃除している間表に出ていてと妻は言った。	
\\	掃除[そうじ]している 間[あいだ] 表[おもて]に 出[で]ていてと 妻[つま]は 言[い]った。	表=おもて=1)物の二つの面のうち、主だったほう。表面。また、外側。       
\\	他のものより前に位置すること。前面。       
\\	畳・草履・下駄などの表面につけるござ。       
\\	うわべ。外見。       
\\	表向きのこと。おおやけ。公式。正式。       
\\	正面。家の入り口。       
\\	家の外。戸外。また、家の前の通り。
\\	夕べは表の音がうるさくて眠れなかった。	
\\	夕[ゆう]べは 表[おもて]の 音[おと]がうるさくて 眠[ねむ]れなかった。	表=おもて=1)物の二つの面のうち、主だったほう。表面。また、外側。       
\\	他のものより前に位置すること。前面。       
\\	畳・草履・下駄などの表面につけるござ。       
\\	うわべ。外見。       
\\	表向きのこと。おおやけ。公式。正式。       
\\	正面。家の入り口。       
\\	家の外。戸外。また、家の前の通り。
\\	まだ熱があるから表へ出てはいけない。	
\\	まだ 熱[ねつ]があるから 表[おもて]へ 出[で]てはいけない。	表=おもて=1)物の二つの面のうち、主だったほう。表面。また、外側。       
\\	他のものより前に位置すること。前面。       
\\	畳・草履・下駄などの表面につけるござ。       
\\	うわべ。外見。       
\\	表向きのこと。おおやけ。公式。正式。       
\\	正面。家の入り口。       
\\	家の外。戸外。また、家の前の通り。
\\	天気がいいから表へ出ようよ。	
\\	天気[てんき]がいいから 表[おもて]へ 出[で]ようよ。	表=おもて=1)物の二つの面のうち、主だったほう。表面。また、外側。       
\\	他のものより前に位置すること。前面。       
\\	畳・草履・下駄などの表面につけるござ。       
\\	うわべ。外見。       
\\	表向きのこと。おおやけ。公式。正式。       
\\	正面。家の入り口。       
\\	家の外。戸外。また、家の前の通り。
\\	娘は人見知りするたちでほとんど表へ出ない。	
\\	娘[むすめ]は 人見知[ひとみし]りするたちでほとんど 表[おもて]へ 出[で]ない。	表=おもて=1)物の二つの面のうち、主だったほう。表面。また、外側。       
\\	他のものより前に位置すること。前面。       
\\	畳・草履・下駄などの表面につけるござ。       
\\	うわべ。外見。       
\\	表向きのこと。おおやけ。公式。正式。       
\\	正面。家の入り口。       
\\	家の外。戸外。また、家の前の通り。
\\	パジャマ姿で表をうろうろするな。	
\\	パジャマ 姿[すがた]で 表[おもて]をうろうろするな。	表=おもて=1)物の二つの面のうち、主だったほう。表面。また、外側。       
\\	他のものより前に位置すること。前面。       
\\	畳・草履・下駄などの表面につけるござ。       
\\	うわべ。外見。       
\\	表向きのこと。おおやけ。公式。正式。       
\\	正面。家の入り口。       
\\	家の外。戸外。また、家の前の通り。
\\	表はすっかり暗くなっていた。	
\\	表[おもて]はすっかり 暗[くら]くなっていた。	表=おもて=1)物の二つの面のうち、主だったほう。表面。また、外側。       
\\	他のものより前に位置すること。前面。       
\\	畳・草履・下駄などの表面につけるござ。       
\\	うわべ。外見。       
\\	表向きのこと。おおやけ。公式。正式。       
\\	正面。家の入り口。       
\\	家の外。戸外。また、家の前の通り。
\\	彼女は感情を表に出さない。	
\\	彼女[かのじょ]は 感情[かんじょう]を 表[おもて]に 出[だ]さない。	表=おもて=1)物の二つの面のうち、主だったほう。表面。また、外側。       
\\	他のものより前に位置すること。前面。       
\\	畳・草履・下駄などの表面につけるござ。       
\\	うわべ。外見。       
\\	表向きのこと。おおやけ。公式。正式。       
\\	正面。家の入り口。       
\\	家の外。戸外。また、家の前の通り。
\\	表だけ見ている分にはあの会社は景気が良さそうだ。	
\\	表[おもて]だけ 見[み]ている 分[ぶん]にはあの 会社[かいしゃ]は 景気[けいき]が 良[よ]さそうだ。	表=おもて=1)物の二つの面のうち、主だったほう。表面。また、外側。       
\\	他のものより前に位置すること。前面。       
\\	畳・草履・下駄などの表面につけるござ。       
\\	うわべ。外見。       
\\	表向きのこと。おおやけ。公式。正式。       
\\	正面。家の入り口。       
\\	家の外。戸外。また、家の前の通り。
\\	私の言うことには表も裏もない。	
\\	私[わたし]の 言[い]うことには 表[おもて]も 裏[うら]もない。	表=おもて=1)物の二つの面のうち、主だったほう。表面。また、外側。       
\\	他のものより前に位置すること。前面。       
\\	畳・草履・下駄などの表面につけるござ。       
\\	うわべ。外見。       
\\	表向きのこと。おおやけ。公式。正式。       
\\	正面。家の入り口。       
\\	家の外。戸外。また、家の前の通り。
\\	僕は音痴だ。	
\\	僕[ぼく]は 音痴[おんち]だ。	音痴=おんち=1)生理的欠陥によって正しい音の認識と発声のできないこと。        
\\	音程や調子が外れて歌を正確に歌えないこと。        
\\	あることに関して感覚が鈍いこと。また、その人。
\\	彼はものすごい方向音痴だ。	
\\	彼[かれ]はものすごい 方向[ほうこう] 音痴[おんち]だ。	方向音痴=ほうこうおんち=方向に関する感覚の鈍いこと。また、その人。
\\	彼女は方向音痴だ。	
\\	彼女[かのじょ]は 方向[ほうこう] 音痴[おんち]だ。	方向音痴=ほうこうおんち=方向に関する感覚の鈍いこと。また、その人。
\\	私は幼いころから伯母を母のように慕っていた。	
\\	私[わたし]は 幼[おさな]いころから 伯母[おば]を 母[はは]のように 慕[した]っていた。	慕う=したう=1)離れている人などを恋しく思う。懐かしく思う。        
\\	離れがたく思ってあとを追う。        
\\	目上の人の人格・識見などに引かれる。憧れる。
\\	この子は後を慕って困ります。	
\\	この 子[こ]は 後[ご]を 慕[した]って 困[こま]ります。	慕う=したう=1)離れている人などを恋しく思う。懐かしく思う。        
\\	離れがたく思ってあとを追う。        
\\	目上の人の人格・識見などに引かれる。憧れる。
\\	彼女は彼を慕ってフランスまで行った。	
\\	彼女[かのじょ]は 彼[かれ]を 慕[した]ってフランスまで 行[い]った。	慕う=したう=1)離れている人などを恋しく思う。懐かしく思う。        
\\	離れがたく思ってあとを追う。        
\\	目上の人の人格・識見などに引かれる。憧れる。
\\	その先生は生徒にとても慕われている。	
\\	その 先生[せんせい]は 生徒[せいと]にとても 慕[した]われている。	慕う=したう=1)離れている人などを恋しく思う。懐かしく思う。        
\\	離れがたく思ってあとを追う。        
\\	目上の人の人格・識見などに引かれる。憧れる。
\\	心が優しいので彼女は皆から慕われた。	
\\	心[こころ]が 優[やさ]しいので 彼女[かのじょ]は 皆[みな]から 慕[した]われた。	慕う=したう=1)離れている人などを恋しく思う。懐かしく思う。        
\\	離れがたく思ってあとを追う。        
\\	目上の人の人格・識見などに引かれる。憧れる。
\\	彼女は密かに担任の先生を慕っている。	
\\	彼女[かのじょ]は 密[ひそ]かに 担任[たんにん]の 先生[せんせい]を 慕[した]っている。	慕う=したう=1)離れている人などを恋しく思う。懐かしく思う。        
\\	離れがたく思ってあとを追う。        
\\	目上の人の人格・識見などに引かれる。憧れる。
\\	姉の荷物も自分の荷物も一緒くたに詰め込んだものだから、後の整理が大変だった。	
\\	姉[あね]の 荷物[にもつ]も 自分[じぶん]の 荷物[にもつ]も 一緒[いっしょ]くたに 詰め込[つめこ]んだものだから、 後[ご]の 整理[せいり]が 大変[たいへん]だった。	一緒くた=いっしょくた=雑多な物事が秩序なく一つになっていること。ごちゃまぜ。
\\	それとこれとは別問題だ。ここで一緒くたに論じてはいけない。	
\\	それとこれとは 別[べつ] 問題[もんだい]だ。ここで 一緒[いっしょ]くたに 論[ろん]じてはいけない。	一緒くた=いっしょくた=雑多な物事が秩序なく一つになっていること。ごちゃまぜ。
\\	彼女は歌うというよりほとんど喚いていた。	
\\	彼女[かのじょ]は 歌[うた]うというよりほとんど 喚[わめ]いていた。	喚く=わめく=大声で叫ぶ。大声をあげて騒ぐ。
\\	今となっては泣いても喚いても取り返しがつかない。	
\\	今[いま]となっては 泣[な]いても 喚[わめ]いても 取り返[とりかえ]しがつかない。	喚く=わめく=大声で叫ぶ。大声をあげて騒ぐ。
\\	酔っ払いが何かわけのわからないことを喚いている。	
\\	酔っ払[よっぱら]いが 何[なに]かわけのわからないことを 喚[わめ]いている。	喚く=わめく=大声で叫ぶ。大声をあげて騒ぐ。
\\	お客さん、冷やかしかい。	
\\	お 客[きゃく]さん、 冷[ひ]やかしかい。	冷やかし=ひやかし=1)冗談などを言ってからかうこと。           
\\	買う気がないのに商品を見て回ること。また、その人。
\\	ほとんどが冷やかし客です。	
\\	ほとんどが 冷[ひ]やかし 客[きゃく]です。	冷やかし=ひやかし=1)冗談などを言ってからかうこと。           
\\	買う気がないのに商品を見て回ること。また、その人。
\\	冷やかしのつもりだったのにその店で買い物をしてしまった。	
\\	冷[ひ]やかしのつもりだったのにその 店[みせ]で 買い物[かいもの]をしてしまった。	冷やかし=ひやかし=1)冗談などを言ってからかうこと。           
\\	買う気がないのに商品を見て回ること。また、その人。
\\	おやじくさいことを言うなよ。	
\\	おやじくさいことを 言[い]うなよ。	親父・親爺=おやじ= 『親父』
\\	『親爺』(店の主人)
\\	(中高年の男)
\\	変なおやじが私のあとについてきて怖かったわ。	
\\	変[へん]なおやじが 私[わたし]のあとについてきて 怖[こわ]かったわ。	親父・親爺=おやじ= 『親父』
\\	『親爺』(店の主人)
\\	(中高年の男)
\\	(飲み屋で)おやじさん、ビールもう一本!	
\\	飲み屋[のみや]で)おやじさん、ビールもう一 本[ほん]!	親父・親爺=おやじ= 『親父』
\\	『親爺』(店の主人)
\\	(中高年の男)
\\	あの男がおやじぶった口調で説教を垂れたので腹が立った。	
\\	あの 男[おとこ]がおやじぶった 口調[くちょう]で 説教[せっきょう]を 垂[た]れたので 腹[はら]が 立[た]った。	親父・親爺=おやじ= 『親父』
\\	『親爺』(店の主人)
\\	(中高年の男)
\\	君のおやじさんはいくつだ。	
\\	君[きみ]のおやじさんはいくつだ。	親父・親爺=おやじ= 『親父』
\\	『親爺』(店の主人)
\\	(中高年の男)
\\	おやじもおふくろも田舎で健在です。	
\\	おやじもおふくろも 田舎[いなか]で 健在[けんざい]です。	親父・親爺=おやじ= 『親父』
\\	『親爺』(店の主人)
\\	(中高年の男)
\\	君の親父さんはどんな人だい。	
\\	君[きみ]の 親父[おやじ]さんはどんな 人[ひと]だい。	親父・親爺=おやじ= 『親父』
\\	『親爺』(店の主人)
\\	(中高年の男)
\\	防寒具必携のこと。	
\\	防寒[ぼうかん] 具[ぐ] 必携[ひっけい]のこと。	必携=ひっけい=1)必ず持っていなければならないこと。また、そのもの。         
\\	便利な案内所。手引書。ハンドブック。
\\	身分証明書はいつも携行することになっている。	
\\	身分[みぶん] 証明[しょうめい] 書[しょ]はいつも 携行[けいこう]することになっている。	携行=けいこう=身につけて持って行くこと。
\\	この絵は夕景の微妙な色合いがよく表現されている。	
\\	この 絵[え]は 夕景[ゆうけい]の 微妙[びみょう]な 色合[いろあ]いがよく 表現[ひょうげん]されている。	微妙な=びみょうな= (簡単に表せない) 
\\	(きわどい) 
\\	(わずかな) 
\\	審判員は微妙な判定を下した。	
\\	審判[しんぱん] 員[いん]は 微妙[びみょう]な 判定[はんてい]を 下[くだ]した。	微妙な=びみょうな= (簡単に表せない) 
\\	(きわどい) 
\\	(わずかな) 
\\	双方の思惑には微妙な違いがあった。	
\\	双方[そうほう]の 思惑[おもわく]には 微妙[びみょう]な 違[ちが]いがあった。	微妙な=びみょうな= (簡単に表せない) 
\\	(きわどい) 
\\	(わずかな) 
\\	彼女が次のオリンピックに出場できるか否かは微妙だ。	
\\	彼女[かのじょ]が 次[つぎ]のオリンピックに 出場[しゅつじょう]できるか 否[いな]かは 微妙[びみょう]だ。	微妙な=びみょうな= (簡単に表せない) 
\\	(きわどい) 
\\	(わずかな) 
\\	偽札は手触りが本物とは微妙に異なっていた。	
\\	偽札[にせさつ]は 手触[てざわ]りが 本物[ほんもの]とは 微妙[びみょう]に 異[こと]なっていた。	微妙な=びみょうな= (簡単に表せない) 
\\	(きわどい) 
\\	(わずかな) 
\\	目標を達成できるかどうかは微妙だ。	
\\	目標[もくひょう]を 達成[たっせい]できるかどうかは 微妙[びみょう]だ。	微妙な=びみょうな= (簡単に表せない) 
\\	(きわどい) 
\\	(わずかな) 
\\	あの
\\	の判定は微妙だった。	
\\	あの 
\\	の 判定[はんてい]は 微妙[びみょう]だった。	微妙な=びみょうな= (簡単に表せない) 
\\	(きわどい) 
\\	(わずかな) 
\\	その2語は意味が微妙に違う。	
\\	その2 語[ご]は 意味[いみ]が 微妙[びみょう]に 違[ちが]う。	微妙な=びみょうな= (簡単に表せない) 
\\	(きわどい) 
\\	(わずかな) 
\\	彼の絶妙な語り口は観客を魅了した。	
\\	彼[かれ]の 絶妙[ぜつみょう]な 語り口[かたりくち]は 観客[かんきゃく]を 魅了[みりょう]した。	絶妙=ぜつみょう=この上なく巧みですぐれていること。また、そのさま。
\\	妙案が浮かんだ。	
\\	妙案[みょうあん]が 浮[う]かんだ。	妙案=みょうあん=非常によい考え。素晴らしい思いつき。名案。
\\	妙案を思いついたよ。	
\\	妙案[みょうあん]を 思[おも]いついたよ。	妙案=みょうあん=非常によい考え。素晴らしい思いつき。名案。
\\	大陸同士を突き合わせるとそのへりが奇妙に一致する。	
\\	大陸[たいりく] 同士[どうし]を 突き合[つきあ]わせるとそのへりが 奇妙[きみょう]に 一致[いっち]する。	奇妙=きみょう=1)珍しく、不思議なこと。また、そのさま。         
\\	風変わりなこと。また、そのさま。         
\\	非常に趣・おもしろみ・うまみなどがあること。また、そのさま。
\\	私が外出するときは奇妙に雨が降る。	
\\	私[わたし]が 外出[がいしゅつ]するときは 奇妙[きみょう]に 雨[あめ]が 降[ふ]る。	奇妙=きみょう=1)珍しく、不思議なこと。また、そのさま。         
\\	風変わりなこと。また、そのさま。         
\\	非常に趣・おもしろみ・うまみなどがあること。また、そのさま。
\\	二人は奇妙にうまが合う。	
\\	二人[ふたり]は 奇妙[きみょう]にうまが 合[あ]う。	奇妙=きみょう=1)珍しく、不思議なこと。また、そのさま。         
\\	風変わりなこと。また、そのさま。         
\\	非常に趣・おもしろみ・うまみなどがあること。また、そのさま。
\\	昨夜遅く奇妙な電話があった。	
\\	昨夜[さくや] 遅[おそ]く 奇妙[きみょう]な 電話[でんわ]があった。	奇妙=きみょう=1)珍しく、不思議なこと。また、そのさま。         
\\	風変わりなこと。また、そのさま。         
\\	非常に趣・おもしろみ・うまみなどがあること。また、そのさま。
\\	奇妙なことに彼女はその日に限って来なかった。	
\\	奇妙[きみょう]なことに 彼女[かのじょ]はその 日[ひ]に 限[かぎ]って 来[こ]なかった。	奇妙=きみょう=1)珍しく、不思議なこと。また、そのさま。         
\\	風変わりなこと。また、そのさま。         
\\	非常に趣・おもしろみ・うまみなどがあること。また、そのさま。
\\	会場は奇妙に静まり返っていた。	
\\	会場[かいじょう]は 奇妙[きみょう]に 静まり返[しずまりかえ]っていた。	奇妙=きみょう=1)珍しく、不思議なこと。また、そのさま。         
\\	風変わりなこと。また、そのさま。         
\\	非常に趣・おもしろみ・うまみなどがあること。また、そのさま。
\\	彼は大男にしては奇妙に甲高い声をしていた。	
\\	彼[かれ]は 大男[おおおとこ]にしては 奇妙[きみょう]に 甲高[かんだか]い 声[こえ]をしていた。	奇妙=きみょう=1)珍しく、不思議なこと。また、そのさま。         
\\	風変わりなこと。また、そのさま。         
\\	非常に趣・おもしろみ・うまみなどがあること。また、そのさま。
\\	彼の巧妙な計略に引っかかった。	
\\	彼[かれ]の 巧妙[こうみょう]な 計略[けいりゃく]に 引[ひ]っかかった。	巧妙=こうみょう=非常に巧みであること。また、そのさま。
\\	麻薬密輸の手口はますます巧妙になっている。	
\\	麻薬[まやく] 密輸[みつゆ]の 手口[てぐち]はますます 巧妙[こうみょう]になっている。	巧妙=こうみょう=非常に巧みであること。また、そのさま。
\\	なかなか軽妙な筆致だ。	
\\	なかなか 軽妙[けいみょう]な 筆致[ひっち]だ。	軽妙な=けいみょうな=文章・話・技などが、軽快でうまみがあること。気が利いていて面白いこと。また、そのさま。 筆致=ひっち=書画や文章の書きぶり。筆のおもむき。筆付き。
\\	二人のお笑い芸人の軽妙なやりとりに観客はどっと笑った。	
\\	二人[ふたり]のお 笑[わら]い 芸人[げいにん]の 軽妙[けいみょう]なやりとりに 観客[かんきゃく]はどっと 笑[わら]った。	軽妙な=けいみょうな=文章・話・技などが、軽快でうまみがあること。気が利いていて面白いこと。また、そのさま。
\\	その絵があると室内が優雅な感じになる。	
\\	その 絵[え]があると 室内[しつない]が 優雅[ゆうが]な 感[かん]じになる。	優雅な=ゆうがな= (上品で美しい) 
\\	彼女は優雅に踊った。	
\\	彼女[かのじょ]は 優雅[ゆうが]に 踊[おど]った。	優雅な=ゆうがな= (上品で美しい) 
\\	増加する定年退職者たちの受け皿づくりが急務になってきている。	
\\	増加[ぞうか]する 定年[ていねん] 退職[たいしょく] 者[しゃ]たちの 受け皿[うけざら]づくりが 急務[きゅうむ]になってきている。	受け皿=うけざら=1)垂れるしすぐなどを受ける皿。          
\\	物事を引き受ける態勢。
\\	失業者の受け皿が必要だ。	
\\	失業[しつぎょう] 者[しゃ]の 受け皿[うけざら]が 必要[ひつよう]だ。	受け皿=うけざら=1)垂れるしすぐなどを受ける皿。          
\\	物事を引き受ける態勢。
\\	そこは見渡す限り草木が茂っていた。	
\\	そこは 見渡[みわた]す 限[かぎ]り 草木[くさき]が 茂[しげ]っていた。	茂る=しげる=草木が生長して、枝葉がたくさん生え出る。盛んに生える。
\\	潮の変わり目だ。	
\\	潮[しお]の 変わり目[かわりめ]だ。	潮=しお= 
\\	(潮流) 
\\	(海水) 
\\	(ちょうど良い時) 
\\	このあたりは潮が速いから気をつけて。	
\\	このあたりは 潮[しお]が 速[はや]いから 気[き]をつけて。	潮=しお= 
\\	(潮流) 
\\	(海水) 
\\	(ちょうど良い時) 
\\	潮が引いた後にはいくつもの貝殻が散らばっていた。	
\\	潮[しお]が 引[ひ]いた 後[のち]にはいくつもの 貝殻[かいがら]が 散[ち]らばっていた。	潮=しお= 
\\	(潮流) 
\\	(海水) 
\\	(ちょうど良い時) 
\\	物には全て潮時というものがある。	
\\	物[もの]には 全[すべ]て 潮時[しおどき]というものがある。	潮時=しおどき=1)潮の満ちる時、また、引く時。         
\\	物事を始めたり終えたりするのに、適当な時機。好機。
\\	それをするのにちょうどいい潮時ではないか。	
\\	それをするのにちょうどいい 潮時[しおどき]ではないか。	潮時=しおどき=1)潮の満ちる時、また、引く時。         
\\	物事を始めたり終えたりするのに、適当な時機。好機。
\\	引退するなら今が潮時だよ。	
\\	引退[いんたい]するなら 今[いま]が 潮時[しおどき]だよ。	潮時=しおどき=1)潮の満ちる時、また、引く時。         
\\	物事を始めたり終えたりするのに、適当な時機。好機。
\\	何にでも病名をつけるのが最近の風潮だ。	
\\	何[なに]にでも 病名[びょうめい]をつけるのが 最近[さいきん]の 風潮[ふうちょう]だ。	風潮=ふうちょう=1)風と潮。また、風によって起こる潮の流れ。          
\\	時代の推移に伴って変わる世の中の有様。
\\	興奮は最高潮に達していた。	
\\	興奮[こうふん]は 最高潮[さいこうちょう]に 達[たっ]していた。	最高潮=さいこうちょう=ある雰囲気や感情などが最も高まった状態。また、その場面や時代。クライマックス。
\\	これで彼女の人気が最高潮に達した。	
\\	これで 彼女[かのじょ]の 人気[にんき]が 最高潮[さいこうちょう]に 達[たっ]した。	最高潮=さいこうちょう=ある雰囲気や感情などが最も高まった状態。また、その場面や時代。クライマックス。
\\	彼女は疲労が重なって倒れた。	
\\	彼女[かのじょ]は 疲労[ひろう]が 重[かさ]なって 倒[たお]れた。	疲労=ひろう= 
\\	彼らの表情に疲労の色が濃くなった。	
\\	彼[かれ]らの 表情[ひょうじょう]に 疲労[ひろう]の 色[いろ]が 濃[こ]くなった。	疲労=ひろう= 
\\	延長戦になって選手の間に疲労の色が見える。	
\\	延長[えんちょう] 戦[せん]になって 選手[せんしゅ]の 間[あいだ]に 疲労[ひろう]の 色[いろ]が 見[み]える。	疲労=ひろう= 
\\	彼は慣れない環境で精神が疲弊した。	
\\	彼[かれ]は 慣[な]れない 環境[かんきょう]で 精神[せいしん]が 疲弊[ひへい]した。	疲弊=ひへい=1)心身が疲れて弱ること。        
\\	経済状態などが悪化して活力をなくしてしまうこと。
\\	長い戦争で国力が疲弊した。	
\\	長[なが]い 戦争[せんそう]で 国力[こくりょく]が 疲弊[ひへい]した。	疲弊=ひへい=1)心身が疲れて弱ること。        
\\	経済状態などが悪化して活力をなくしてしまうこと。
\\	彼は引退か移籍かの二者択一を迫られている。	
\\	彼[かれ]は 引退[いんたい]か 移籍[いせき]かの 二者択一[にしゃたくいつ]を 迫[せま]られている。	二者択一=にしゃたくいつ=二つのうち、どちらか一つを選ぶこと。
\\	コンピューターの故障で事務が停滞している。	
\\	コンピューターの 故障[こしょう]で 事務[じむ]が 停滞[ていたい]している。	停滞=ていたい=一カ所にとどまって動かないこと。物事が順調に進まないこと。
\\	景気の停滞はしばらく続くだろう。	
\\	景気[けいき]の 停滞[ていたい]はしばらく 続[つづ]くだろう。	停滞=ていたい=一カ所にとどまって動かないこと。物事が順調に進まないこと。
\\	米国の経済は停滞期に入った。	
\\	米国[べいこく]の 経済[けいざい]は 停滞[ていたい] 期[き]に 入[はい]った。	停滞=ていたい=一カ所にとどまって動かないこと。物事が順調に進まないこと。
\\	それぞれの選択しについて慎重に検討せよ。	
\\	それぞれの 選択[せんたく]しについて 慎重[しんちょう]に 検討[けんとう]せよ。	選択肢=せんたくし=質問に対して、そこから選択して答えるように用意されている二つ以上の答え。
\\	選択肢は行くか行かないかの二つしかない。	
\\	選択肢[せんたくし]は 行[い]くか 行[い]かないかの 二[ふた]つしかない。	選択肢=せんたくし=質問に対して、そこから選択して答えるように用意されている二つ以上の答え。
\\	この中から自由に選択しなさい。	
\\	この 中[なか]から 自由[じゆう]に 選択[せんたく]しなさい。	選択=せんたく= 
\\	選択した番号の解答欄を黒く塗りつぶしなさい。	
\\	選択[せんたく]した 番号[ばんごう]の 解答[かいとう] 欄[らん]を 黒[くろ]く 塗[ぬ]りつぶしなさい。	選択=せんたく= 
\\	和室と洋室のどちらかの部屋の選択ができる。	
\\	和室[わしつ]と 洋室[ようしつ]のどちらかの 部屋[へや]の 選択[せんたく]ができる。	選択=せんたく= 
\\	選択の範囲が狭い。	
\\	選択[せんたく]の 範囲[はんい]が 狭[せま]い。	選択=せんたく= 
\\	家を売るしか選択の余地はない。	
\\	家[いえ]を 売[う]るしか 選択[せんたく]の 余地[よち]はない。	選択=せんたく= 
\\	あなたはいい選択をした。	
\\	あなたはいい 選択[せんたく]をした。	選択=せんたく= 
\\	辞職するしか選択の余地はなかった。	
\\	辞職[じしょく]するしか 選択[せんたく]の 余地[よち]はなかった。	選択=せんたく= 
\\	進路選択の際には先生が相談に乗ってくれた。	
\\	進路[しんろ] 選択[せんたく]の 際[さい]には 先生[せんせい]が 相談[そうだん]に 乗[の]ってくれた。	選択=せんたく= 
\\	家賃の支払いが延滞しています。	
\\	家賃[やちん]の 支払[しはら]いが 延滞[えんたい]しています。	延滞=えんたい=1)物事が順調に進まないで、滞ること。遅滞。         
\\	収入や支払いが滞ること。
\\	電気料金の支払いが延滞している。	
\\	電気[でんき] 料金[りょうきん]の 支払[しはら]いが 延滞[えんたい]している。	延滞=えんたい=1)物事が順調に進まないで、滞ること。遅滞。         
\\	収入や支払いが滞ること。
\\	景気の沈滞が続いている。	
\\	景気[けいき]の 沈滞[ちんたい]が 続[つづ]いている。	沈滞=ちんたい=1)いつまでも一つ所に滞っていること。         
\\	活気がなく、進歩・発展の動きが見られないこと。意気が上がらずに停滞していること。
\\	チーム内には沈滞ムードが漂っている。	
\\	チーム 内[ない]には 沈滞[ちんたい]ムードが 漂[ただよ]っている。	沈滞=ちんたい=1)いつまでも一つ所に滞っていること。         
\\	活気がなく、進歩・発展の動きが見られないこと。意気が上がらずに停滞していること。
\\	小学生からの鋭い質問に首相は一瞬絶句した。	
\\	小学生[しょうがくせい]からの 鋭[するど]い 質問[しつもん]に 首相[しゅしょう]は 一瞬[いっしゅん] 絶句[ぜっく]した。	絶句=ぜっく=話や演説の途中で言葉に詰まること。また、役者が台詞を忘れてつかえること。
\\	そのニュースを聞いて絶句した。	
\\	そのニュースを 聞[き]いて 絶句[ぜっく]した。	絶句=ぜっく=話や演説の途中で言葉に詰まること。また、役者が台詞を忘れてつかえること。
\\	彼は鋭い目付きをしている。	
\\	彼[かれ]は 鋭[するど]い 目付[めつ]きをしている。	鋭い=するどい=1)物の先が細くてとがっている。また、刃物の切れ味がよい。         
\\	感覚が鋭敏である。反応が速い。また、判断力がすぐれている。         3ア)物に向かっていく勢いが激しくて強い。          イ)勢いが激しくて、人の心を突き刺すようである。          ウ)人の感覚を刺激する力が強い。
\\	君の観察はなかなか鋭い。	
\\	君[きみ]の 観察[かんさつ]はなかなか 鋭[するど]い。	鋭い=するどい=1)物の先が細くてとがっている。また、刃物の切れ味がよい。         
\\	感覚が鋭敏である。反応が速い。また、判断力がすぐれている。         3ア)物に向かっていく勢いが激しくて強い。          イ)勢いが激しくて、人の心を突き刺すようである。          ウ)人の感覚を刺激する力が強い。
\\	意見が鋭く対立した。	
\\	意見[いけん]が 鋭[するど]く 対立[たいりつ]した。	鋭い=するどい=1)物の先が細くてとがっている。また、刃物の切れ味がよい。         
\\	感覚が鋭敏である。反応が速い。また、判断力がすぐれている。         3ア)物に向かっていく勢いが激しくて強い。          イ)勢いが激しくて、人の心を突き刺すようである。          ウ)人の感覚を刺激する力が強い。
\\	彼の勘の鋭さには脱帽だ。	
\\	彼[かれ]の 勘[かん]の 鋭[するど]さには 脱帽[だつぼう]だ。	鋭い=するどい=1)物の先が細くてとがっている。また、刃物の切れ味がよい。         
\\	感覚が鋭敏である。反応が速い。また、判断力がすぐれている。         3ア)物に向かっていく勢いが激しくて強い。          イ)勢いが激しくて、人の心を突き刺すようである。          ウ)人の感覚を刺激する力が強い。
\\	猫の爪は鋭い。	
\\	猫[ねこ]の 爪[つめ]は 鋭[するど]い。	鋭い=するどい=1)物の先が細くてとがっている。また、刃物の切れ味がよい。         
\\	感覚が鋭敏である。反応が速い。また、判断力がすぐれている。         3ア)物に向かっていく勢いが激しくて強い。          イ)勢いが激しくて、人の心を突き刺すようである。          ウ)人の感覚を刺激する力が強い。
\\	胸部に鋭い痛みが走った。	
\\	胸部[きょうぶ]に 鋭[するど]い 痛[いた]みが 走[はし]った。	鋭い=するどい=1)物の先が細くてとがっている。また、刃物の切れ味がよい。         
\\	感覚が鋭敏である。反応が速い。また、判断力がすぐれている。         3ア)物に向かっていく勢いが激しくて強い。          イ)勢いが激しくて、人の心を突き刺すようである。          ウ)人の感覚を刺激する力が強い。
\\	記者たちは首相に鋭い質問を浴びせた。	
\\	記者[きしゃ]たちは 首相[しゅしょう]に 鋭[するど]い 質問[しつもん]を 浴[あ]びせた。	鋭い=するどい=1)物の先が細くてとがっている。また、刃物の切れ味がよい。         
\\	感覚が鋭敏である。反応が速い。また、判断力がすぐれている。         3ア)物に向かっていく勢いが激しくて強い。          イ)勢いが激しくて、人の心を突き刺すようである。          ウ)人の感覚を刺激する力が強い。
\\	この包丁は切れが鈍い。	
\\	この 包丁[ほうちょう]は 切[き]れが 鈍[にぶ]い。	鈍い=にぶい=1)刃物の切れ味が悪い。        2ア)動きがのろい。動作が機敏でない。         イ)感覚が鋭敏でない。反応が遅い。        
\\	人の感覚を刺激する力が弱い。
\\	ドスンという人が倒れるような鈍い音がした。	
\\	ドスンという 人[ひと]が 倒[たお]れるような 鈍[にぶ]い 音[おと]がした。	鈍い=にぶい=1)刃物の切れ味が悪い。        2ア)動きがのろい。動作が機敏でない。         イ)感覚が鋭敏でない。反応が遅い。        
\\	人の感覚を刺激する力が弱い。
\\	彼は運動神経が鈍い。	
\\	彼[かれ]は 運動[うんどう] 神経[しんけい]が 鈍[にぶ]い。	鈍い=にぶい=1)刃物の切れ味が悪い。        2ア)動きがのろい。動作が機敏でない。         イ)感覚が鋭敏でない。反応が遅い。        
\\	人の感覚を刺激する力が弱い。
\\	鈍いな、君は。彼女の気持ちがわからないのか。	
\\	鈍[にぶ]いな、 君[きみ]は。 彼女[かのじょ]の 気持[きも]ちがわからないのか。	鈍い=にぶい=1)刃物の切れ味が悪い。        2ア)動きがのろい。動作が機敏でない。         イ)感覚が鋭敏でない。反応が遅い。        
\\	人の感覚を刺激する力が弱い。
\\	彼の新作に対する客席からの反応は鈍い。	
\\	彼[かれ]の 新作[しんさく]に 対[たい]する 客席[きゃくせき]からの 反応[はんのう]は 鈍[にぶ]い。	鈍い=にぶい=1)刃物の切れ味が悪い。        2ア)動きがのろい。動作が機敏でない。         イ)感覚が鋭敏でない。反応が遅い。        
\\	人の感覚を刺激する力が弱い。
\\	雨のせいで有権者の出足が鈍い。	
\\	雨[あめ]のせいで 有権者[ゆうけんしゃ]の 出足[であし]が 鈍[にぶ]い。	鈍い=にぶい=1)刃物の切れ味が悪い。        2ア)動きがのろい。動作が機敏でない。         イ)感覚が鋭敏でない。反応が遅い。        
\\	人の感覚を刺激する力が弱い。
\\	酔うとたいていの人は頭の回転が鈍くなる。	
\\	酔[よ]うとたいていの 人[ひと]は 頭[あたま]の 回転[かいてん]が 鈍[にぶ]くなる。	鈍い=にぶい=1)刃物の切れ味が悪い。        2ア)動きがのろい。動作が機敏でない。         イ)感覚が鋭敏でない。反応が遅い。        
\\	人の感覚を刺激する力が弱い。
\\	指先の感覚が鈍い。	
\\	指先[ゆびさき]の 感覚[かんかく]が 鈍[にぶ]い。	鈍い=にぶい=1)刃物の切れ味が悪い。        2ア)動きがのろい。動作が機敏でない。         イ)感覚が鋭敏でない。反応が遅い。        
\\	人の感覚を刺激する力が弱い。
\\	何をするにものろい男だ。	
\\	何[なに]をするにものろい 男[おとこ]だ。	鈍い=のろい=1)進み方がゆっくりしている。遅い。        
\\	動作や頭の動きなどが悪い。にぶい。        
\\	女性に対して甘い。色事に溺れやすい。
\\	彼は頭の回転がのろい。	
\\	彼[かれ]は 頭[あたま]の 回転[かいてん]がのろい。	鈍い=のろい=1)進み方がゆっくりしている。遅い。        
\\	動作や頭の動きなどが悪い。にぶい。        
\\	女性に対して甘い。色事に溺れやすい。
\\	彼は走るのがのろい。	
\\	彼[かれ]は 走[はし]るのがのろい。	鈍い=のろい=1)進み方がゆっくりしている。遅い。        
\\	動作や頭の動きなどが悪い。にぶい。        
\\	女性に対して甘い。色事に溺れやすい。
\\	あいつはまったく動作がカメのようにのろいなあ。	
\\	あいつはまったく 動作[どうさ]がカメのようにのろいなあ。	鈍い=のろい=1)進み方がゆっくりしている。遅い。        
\\	動作や頭の動きなどが悪い。にぶい。        
\\	女性に対して甘い。色事に溺れやすい。
\\	その話は彼の前では禁句だ。	
\\	その 話[はなし]は 彼[かれ]の 前[まえ]では 禁句[きんく]だ。	禁句=きんく=1)使ってはならない語句。止め句。        
\\	聞き手の感情を害したり刺激したりするのをはばかって避けるべき言葉や話。
\\	その話題は禁句だ。	
\\	その 話題[わだい]は 禁句[きんく]だ。	禁句=きんく=1)使ってはならない語句。止め句。        
\\	聞き手の感情を害したり刺激したりするのをはばかって避けるべき言葉や話。
\\	最近、胃腸の具合が悪い。	
\\	最近[さいきん]、 胃腸[いちょう]の 具合[ぐあい]が 悪[わる]い。	胃腸=いちょう= 
\\	彼らの行動は克明に記録された。	
\\	彼[かれ]らの 行動[こうどう]は 克明[こくめい]に 記録[きろく]された。	克明=こくめい=1)細かいところまで念を入れて手落ちのないこと。また、そのさま。丹念。         
\\	まじめで正直なこと。また、そのさま。実直。
\\	講義の内容を克明にノートに取った。	
\\	講義[こうぎ]の 内容[ないよう]を 克明[こくめい]にノートに 取[と]った。	克明=こくめい=1)細かいところまで念を入れて手落ちのないこと。また、そのさま。丹念。         
\\	まじめで正直なこと。また、そのさま。実直。
\\	彼は国家の誉れである。	
\\	彼[かれ]は 国家[こっか]の 誉[ほま]れである。	誉れ=ほまれ=誇りとするにたる事柄。また、よいという評判をえること。
\\	彼女は我が校の誉れである。	
\\	彼女[かのじょ]は 我[わ]が 校[こう]の 誉[ほま]れである。	誉れ=ほまれ=誇りとするにたる事柄。また、よいという評判をえること。
\\	このような生徒は学校の名誉になる。	
\\	このような 生徒[せいと]は 学校[がっこう]の 名誉[めいよ]になる。	名誉=めいよ=1)能力や行為について、すぐれた評価を得ていること。また、そのさま。        
\\	社会的に認められている、その個人または集団の人格的価値。体面。面目。        
\\	身分や職名を表す語につけて、その人の功労をたたえて贈る称号とするもの。        
\\	有名であること。評判が高いこと。また、そのさま。よいことにも悪いことにもいう。
\\	ノーベル賞を得ることは作家としての最高の名誉である。	
\\	ノーベル 賞[しょう]を 得[え]ることは 作家[さっか]としての 最高[さいこう]の 名誉[めいよ]である。	名誉=めいよ=1)能力や行為について、すぐれた評価を得ていること。また、そのさま。        
\\	社会的に認められている、その個人または集団の人格的価値。体面。面目。        
\\	身分や職名を表す語につけて、その人の功労をたたえて贈る称号とするもの。        
\\	有名であること。評判が高いこと。また、そのさま。よいことにも悪いことにもいう。
\\	名誉にかけてそれは間違いないと断言する。	
\\	名誉[めいよ]にかけてそれは 間違[まちが]いないと 断言[だんげん]する。	名誉=めいよ=1)能力や行為について、すぐれた評価を得ていること。また、そのさま。        
\\	社会的に認められている、その個人または集団の人格的価値。体面。面目。        
\\	身分や職名を表す語につけて、その人の功労をたたえて贈る称号とするもの。        
\\	有名であること。評判が高いこと。また、そのさま。よいことにも悪いことにもいう。
\\	名誉にかけて、公約は果たします。	
\\	名誉[めいよ]にかけて、 公約[こうやく]は 果[は]たします。	名誉=めいよ=1)能力や行為について、すぐれた評価を得ていること。また、そのさま。        
\\	社会的に認められている、その個人または集団の人格的価値。体面。面目。        
\\	身分や職名を表す語につけて、その人の功労をたたえて贈る称号とするもの。        
\\	有名であること。評判が高いこと。また、そのさま。よいことにも悪いことにもいう。
\\	君は我が社の名誉を汚した。	
\\	君[きみ]は 我[わ]が 社[しゃ]の 名誉[めいよ]を 汚[よご]した。	名誉=めいよ=1)能力や行為について、すぐれた評価を得ていること。また、そのさま。        
\\	社会的に認められている、その個人または集団の人格的価値。体面。面目。        
\\	身分や職名を表す語につけて、その人の功労をたたえて贈る称号とするもの。        
\\	有名であること。評判が高いこと。また、そのさま。よいことにも悪いことにもいう。
\\	この試合の勝者には世界の覇者としての栄誉が授けられる。	
\\	この 試合[しあい]の 勝者[しょうしゃ]には 世界[せかい]の 覇者[はしゃ]としての 栄誉[えいよ]が 授[さづ]けられる。	栄誉=えいよ=輝かしい誉れ。栄名。
\\	彼はノーベル賞の栄誉に輝いた。	
\\	彼[かれ]はノーベル 賞[しょう]の 栄誉[えいよ]に 輝[かがや]いた。	栄誉=えいよ=輝かしい誉れ。栄名。
\\	彼の勝利が一家に栄誉をもたらした。	
\\	彼[かれ]の 勝利[しょうり]が 一家[いっか]に 栄誉[えいよ]をもたらした。	栄誉=えいよ=輝かしい誉れ。栄名。
\\	明日は卒業記念写真の撮影があります。	
\\	明日[あした]は 卒業[そつぎょう] 記念[きねん] 写真[しゃしん]の 撮影[さつえい]があります。	撮影=さつえい=写真や映画をとること。
\\	彼の撮影テクニックはすばらしい。	
\\	彼[かれ]の 撮影[さつえい]テクニックはすばらしい。	撮影=さつえい=写真や映画をとること。
\\	彼はアラスカでドキュメンタリー映画を撮影している。	
\\	彼[かれ]はアラスカでドキュメンタリー 映画[えいが]を 撮影[さつえい]している。	撮影=さつえい=写真や映画をとること。
\\	この映画は全編中国で撮影された。	
\\	この 映画[えいが]は 全編[ぜんぺん] 中国[ちゅうごく]で 撮影[さつえい]された。	撮影=さつえい=写真や映画をとること。
\\	これらの点を是正する必要がある。	
\\	これらの 点[てん]を 是正[ぜせい]する 必要[ひつよう]がある。	是正=ぜせい=悪い点や不都合な点を改め正すこと。
\\	そんな薄着をしていると風邪を引くぜ。	
\\	そんな 薄着[うすぎ]をしていると 風邪[かぜ]を 引[ひ]くぜ。	是正=ぜせい=悪い点や不都合な点を改め正すこと。
\\	町は地元チームの優勝に沸き立っている。	
\\	町[まち]は 地元[じもと]チームの 優勝[ゆうしょう]に 沸き立[わきた]っている。	沸き立つ・湧き立つ=わきたつ= 
\\	盛んに沸く。煮え立つ。 
\\	興奮して騒然とした状態になる。 
\\	感情が高ぶる。
\\	観衆はわき立っていた。	
\\	観衆[かんしゅう]はわき 立[た]っていた。	沸き立つ・湧き立つ=わきたつ= 
\\	盛んに沸く。煮え立つ。 
\\	興奮して騒然とした状態になる。 
\\	感情が高ぶる。
\\	勝利の瞬間、スタンドから大歓声が沸き上がった。	
\\	勝利[しょうり]の 瞬間[しゅんかん]、スタンドから 大[だい] 歓声[かんせい]が 沸き上[わきあ]がった。	沸き上がる・湧き上がる=わきあがる= 
\\	盛んに煮え立つ。沸騰する。 
\\	興奮した雰囲気が高まる。ある感情が激しく起こる。 
\\	水面などが激しく波立つ。
\\	静かに喜びが胸にわき上がってきた。	
\\	静[しず]かに 喜[よろこ]びが 胸[むね]にわき 上[あ]がってきた。	沸き上がる・湧き上がる=わきあがる= 
\\	盛んに煮え立つ。沸騰する。 
\\	興奮した雰囲気が高まる。ある感情が激しく起こる。 
\\	水面などが激しく波立つ。
\\	感謝の気持ちがわき上がってきた。	
\\	感謝[かんしゃ]の 気持[きも]ちがわき 上[あ]がってきた。	沸き上がる・湧き上がる=わきあがる= 
\\	盛んに煮え立つ。沸騰する。 
\\	興奮した雰囲気が高まる。ある感情が激しく起こる。 
\\	水面などが激しく波立つ。
\\	泉が湧き出ている。	
\\	泉[いずみ]が 湧き出[わきで]ている。	湧き出る=わきでる= 
\\	水が地中から湧いて出る。 
\\	涙などが流れ出る。 
\\	物がわいたように次々と現れ出る。 
\\	虫などが自然に発生する。 
\\	考え、感情などが、あふれるように生れ出る。
\\	彼女の目に涙が湧き出た。	
\\	彼女[かのじょ]の 目[め]に 涙[なみだ]が 湧き出[わきで]た。	湧き出る=わきでる= 
\\	水が地中から湧いて出る。 
\\	涙などが流れ出る。 
\\	物がわいたように次々と現れ出る。 
\\	虫などが自然に発生する。 
\\	考え、感情などが、あふれるように生れ出る。
\\	今日の彼は名案が次々に湧き出てくる。	
\\	今日[きょう]の 彼[かれ]は 名案[めいあん]が 次々[つぎつぎ]に 湧き出[わきで]てくる。	湧き出る=わきでる= 
\\	水が地中から湧いて出る。 
\\	涙などが流れ出る。 
\\	物がわいたように次々と現れ出る。 
\\	虫などが自然に発生する。 
\\	考え、感情などが、あふれるように生れ出る。
\\	収賄政治家に国中から非難の声が沸き起こった。	
\\	収賄[しゅうわい] 政治[せいじ] 家[か]に 国[くに] 中[ちゅう]から 非難[ひなん]の 声[こえ]が 沸き起[わきお]こった。	沸き起こる=わきおこる= 
\\	感情などがこみ上げてくる。 
\\	歓声などが盛んに起こる。
\\	場内から割れるような拍手が沸き起こった。	
\\	場内[じょうない]から 割[わ]れるような 拍手[はくしゅ]が 沸き起[わきお]こった。	沸き起こる=わきおこる= 
\\	感情などがこみ上げてくる。 
\\	歓声などが盛んに起こる。
\\	私の心にある疑問が沸き起こった。	
\\	私[わたし]の 心[こころ]にある 疑問[ぎもん]が 沸き起[わきお]こった。	沸き起こる=わきおこる= 
\\	感情などがこみ上げてくる。 
\\	歓声などが盛んに起こる。
\\	後悔の念が沸き起こった。	
\\	後悔[こうかい]の 念[ねん]が 沸き起[わきお]こった。	沸き起こる=わきおこる= 
\\	感情などがこみ上げてくる。 
\\	歓声などが盛んに起こる。
\\	あの俳優、ちょっといかすね。	
\\	あの 俳優[はいゆう]、ちょっといかすね。	いかす= 
\\	その髪型、あんまりいかさないな。	
\\	その 髪型[かみがた]、あんまりいかさないな。	いかす= 
\\	わが国は核兵器の不所持を国是としている。	
\\	わが 国[くに]は 核兵器[かくへいき]の 不[ふ] 所持[しょじ]を 国是[こくぜ]としている。	国是=こくぜ=国民が認めた、一国の政治の基本的な方針。
\\	市中至るところに国旗が掲げてあった。	
\\	市中[しちゅう] 至[いた]るところに 国旗[こっき]が 掲[かか]げてあった。	掲げる=かかげる= 
\\	人目につく高い所へあげる。また、手に持って高く差し上げる。 
\\	新聞・雑誌などの、目立つ場所に載せる。目立つように工夫して載せる。 
\\	主義・方針などを、人目につくように示す。広く、示して知らせる。 
\\	垂れ下がっているものを、上の方へ持ち上げる。まくり上げる。 
\\	灯火をかき立てて明るくする。
\\	やるからには目標は高く掲げるべきだ。	
\\	やるからには 目標[もくひょう]は 高[たか]く 掲[かか]げるべきだ。	掲げる=かかげる= 
\\	人目につく高い所へあげる。また、手に持って高く差し上げる。 
\\	新聞・雑誌などの、目立つ場所に載せる。目立つように工夫して載せる。 
\\	主義・方針などを、人目につくように示す。広く、示して知らせる。 
\\	垂れ下がっているものを、上の方へ持ち上げる。まくり上げる。 
\\	灯火をかき立てて明るくする。
\\	その政党は「変革」をスローガンに掲げている。	
\\	その 政党[せいとう]は
\\	変革[へんかく]」をスローガンに 掲[かか]げている。	掲げる=かかげる= 
\\	人目につく高い所へあげる。また、手に持って高く差し上げる。 
\\	新聞・雑誌などの、目立つ場所に載せる。目立つように工夫して載せる。 
\\	主義・方針などを、人目につくように示す。広く、示して知らせる。 
\\	垂れ下がっているものを、上の方へ持ち上げる。まくり上げる。 
\\	灯火をかき立てて明るくする。
\\	我々は乳幼児死亡率ゼロを目標に掲げた。	
\\	我々[われわれ]は 乳幼児[にゅうようじ] 死亡[しぼう] 率[りつ]ゼロを 目標[もくひょう]に 掲[かか]げた。	掲げる=かかげる= 
\\	人目につく高い所へあげる。また、手に持って高く差し上げる。 
\\	新聞・雑誌などの、目立つ場所に載せる。目立つように工夫して載せる。 
\\	主義・方針などを、人目につくように示す。広く、示して知らせる。 
\\	垂れ下がっているものを、上の方へ持ち上げる。まくり上げる。 
\\	灯火をかき立てて明るくする。
\\	壁に掲示が出ている。	
\\	壁[かべ]に 掲示[けいじ]が 出[で]ている。	掲示=けいじ=人に伝えるべき事柄を、紙に書くなどして掲げ示すこと。また、その文書など。
\\	学校は明日は休みだという掲示が出ている。	
\\	学校[がっこう]は 明日[あした]は 休[やす]みだという 掲示[けいじ]が 出[で]ている。	掲示=けいじ=人に伝えるべき事柄を、紙に書くなどして掲げ示すこと。また、その文書など。
\\	底に何か沈殿している。	
\\	底[そこ]に 何[なに]か 沈殿[ちんでん]している。	沈殿=ちんでん= 
\\	硬さの点でダイヤモンドに匹敵するものはない。	
\\	硬[かた]さの 点[てん]でダイヤモンドに 匹敵[ひってき]するものはない。	匹敵=ひってき= 比べてみて能力や価値などが同じ程度であること。肩を並べること。
\\	彼の酒量は我々3人分に匹敵する。	
\\	彼[かれ]の 酒量[しゅりょう]は 我々[われわれ] 3人[さんにん] 分[ぶん]に 匹敵[ひってき]する。	匹敵=ひってき= 比べてみて能力や価値などが同じ程度であること。肩を並べること。
\\	この分野で彼に匹敵する者はいない。	
\\	この 分野[ぶんや]で 彼[かれ]に 匹敵[ひってき]する 者[もの]はいない。	匹敵=ひってき= 比べてみて能力や価値などが同じ程度であること。肩を並べること。
\\	彼のマンションの家賃は私の一ヶ月の収入に匹敵する。	
\\	彼[かれ]のマンションの 家賃[やちん]は 私[わたし]の 一ヶ月[いっかげつ]の 収入[しゅうにゅう]に 匹敵[ひってき]する。	匹敵=ひってき= 比べてみて能力や価値などが同じ程度であること。肩を並べること。
\\	機は滑走路に出た。	
\\	機[き]は 滑走[かっそう] 路[ろ]に 出[で]た。	滑走路=かっそうろ= 
\\	目玉焼きが焦げていますよ!	
\\	目玉焼[めだまや]きが 焦[こ]げていますよ!	焦げる=こげる= 
\\	魚が少し焦げてしまった。	
\\	魚[さかな]が 少[すこ]し 焦[こ]げてしまった。	焦げる=こげる= 
\\	アイロンでシャツが焦げた。	
\\	アイロンでシャツが 焦[こ]げた。	焦げる=こげる= 
\\	どこかで何か焦げてるぞ。	
\\	どこかで 何[なに]か 焦[こ]げてるぞ。	焦げる=こげる= 
\\	髪の毛の焦げる匂いがした。	
\\	髪の毛[かみのけ]の 焦[こ]げる 匂[にお]いがした。	焦げる=こげる= 
\\	その肉は外は焦げていたが中は生だった。	
\\	その 肉[にく]は 外[そと]は 焦[こ]げていたが 中[なか]は 生[なま]だった。	焦げる=こげる= 
\\	彼が演説をやったとはこっけいだね。	
\\	彼[かれ]が 演説[えんぜつ]をやったとはこっけいだね。	滑稽=こっけい= 
\\	笑いの対象となる、面白いこと。おどけたこと。また、そのさま。 
\\	あまりにもばかばかしいこと。また、そのさま。
\\	彼の立候補はいささかこっけいだ。	
\\	彼[かれ]の 立候補[りっこうほ]はいささかこっけいだ。	滑稽=こっけい= 
\\	笑いの対象となる、面白いこと。おどけたこと。また、そのさま。 
\\	あまりにもばかばかしいこと。また、そのさま。
\\	彼の身なりったら実にこっけいだった。	
\\	彼[かれ]の 身[み]なりったら 実[じつ]にこっけいだった。	滑稽=こっけい= 
\\	笑いの対象となる、面白いこと。おどけたこと。また、そのさま。 
\\	あまりにもばかばかしいこと。また、そのさま。
\\	彼の文体には滑らかさがない。	
\\	彼[かれ]の 文体[ぶんたい]には 滑[なめ]らかさがない。	滑らか=なめらか= 
\\	物の表面にでこぼこがなくて、すべすべ、また、つるつるしているさま。 
\\	物事が、すらすらと滞りなく進むさま。
\\	折衝はさしたる障害もなく滑らかに進んだ。	
\\	折衝[せっしょう]はさしたる 障害[しょうがい]もなく 滑[なめ]らかに 進[すす]んだ。	滑らか=なめらか= 
\\	物の表面にでこぼこがなくて、すべすべ、また、つるつるしているさま。 
\\	物事が、すらすらと滞りなく進むさま。
\\	酒が入ると彼の口もいくぶん滑らかになった。	
\\	酒[さけ]が 入[はい]ると 彼[かれ]の 口[くち]もいくぶん 滑[なめ]らかになった。	滑らか=なめらか= 
\\	物の表面にでこぼこがなくて、すべすべ、また、つるつるしているさま。 
\\	物事が、すらすらと滞りなく進むさま。
\\	彼の口調は滑らかだった。	
\\	彼[かれ]の 口調[くちょう]は 滑[なめ]らかだった。	滑らか=なめらか= 
\\	物の表面にでこぼこがなくて、すべすべ、また、つるつるしているさま。 
\\	物事が、すらすらと滞りなく進むさま。
\\	彼女は滑らかに中国語を話した。	
\\	彼女[かのじょ]は 滑[なめ]らかに 中国語[ちゅうごくご]を 話[はな]した。	滑らか=なめらか= 
\\	物の表面にでこぼこがなくて、すべすべ、また、つるつるしているさま。 
\\	物事が、すらすらと滞りなく進むさま。
\\	滑るから気をつけなさい。	
\\	滑[すべ]るから 気[き]をつけなさい。	滑る=すべる= 
\\	物の表面をなめらかに移動する。 
\\	表面がなめらかで地面に接する物が安定を失って自然に動いてしまう。スリップする。 
\\	つかもうとした物が、支えられないで手をすり抜ける。 
\\	ある地位を保てなくなる。 
\\	調子に乗ったまま、事がの望ましくないところにまで至る。余計なことを言ったり書いたりしてしまう。 
\\	試験に失敗する。落第する。不合格になる。 
\\	俗に、面白いことをしゃべろうとして失敗する。冗談・ギャグが受けない状態をいう。
\\	つい口が滑ったのです。	
\\	つい 口[くち]が 滑[すべ]ったのです。	滑る=すべる= 
\\	物の表面をなめらかに移動する。 
\\	表面がなめらかで地面に接する物が安定を失って自然に動いてしまう。スリップする。 
\\	つかもうとした物が、支えられないで手をすり抜ける。 
\\	ある地位を保てなくなる。 
\\	調子に乗ったまま、事がの望ましくないところにまで至る。余計なことを言ったり書いたりしてしまう。 
\\	試験に失敗する。落第する。不合格になる。 
\\	俗に、面白いことをしゃべろうとして失敗する。冗談・ギャグが受けない状態をいう。
\\	手が滑って茶碗を落とした。	
\\	手[て]が 滑[すべ]って 茶碗[ちゃわん]を 落[お]とした。	滑る=すべる= 
\\	物の表面をなめらかに移動する。 
\\	表面がなめらかで地面に接する物が安定を失って自然に動いてしまう。スリップする。 
\\	つかもうとした物が、支えられないで手をすり抜ける。 
\\	ある地位を保てなくなる。 
\\	調子に乗ったまま、事がの望ましくないところにまで至る。余計なことを言ったり書いたりしてしまう。 
\\	試験に失敗する。落第する。不合格になる。 
\\	俗に、面白いことをしゃべろうとして失敗する。冗談・ギャグが受けない状態をいう。
\\	手元が滑ってコップを割ってしまった。	
\\	手元[てもと]が 滑[すべ]ってコップを 割[わ]ってしまった。	滑る=すべる= 
\\	物の表面をなめらかに移動する。 
\\	表面がなめらかで地面に接する物が安定を失って自然に動いてしまう。スリップする。 
\\	つかもうとした物が、支えられないで手をすり抜ける。 
\\	ある地位を保てなくなる。 
\\	調子に乗ったまま、事がの望ましくないところにまで至る。余計なことを言ったり書いたりしてしまう。 
\\	試験に失敗する。落第する。不合格になる。 
\\	俗に、面白いことをしゃべろうとして失敗する。冗談・ギャグが受けない状態をいう。
\\	汗で手が滑ってボールをうまくつかめない。	
\\	汗[あせ]で 手[て]が 滑[すべ]ってボールをうまくつかめない。	滑る=すべる= 
\\	物の表面をなめらかに移動する。 
\\	表面がなめらかで地面に接する物が安定を失って自然に動いてしまう。スリップする。 
\\	つかもうとした物が、支えられないで手をすり抜ける。 
\\	ある地位を保てなくなる。 
\\	調子に乗ったまま、事がの望ましくないところにまで至る。余計なことを言ったり書いたりしてしまう。 
\\	試験に失敗する。落第する。不合格になる。 
\\	俗に、面白いことをしゃべろうとして失敗する。冗談・ギャグが受けない状態をいう。
\\	ギャグが滑った。	
\\	ギャグが 滑[すべ]った。	滑る=すべる= 
\\	物の表面をなめらかに移動する。 
\\	表面がなめらかで地面に接する物が安定を失って自然に動いてしまう。スリップする。 
\\	つかもうとした物が、支えられないで手をすり抜ける。 
\\	ある地位を保てなくなる。 
\\	調子に乗ったまま、事がの望ましくないところにまで至る。余計なことを言ったり書いたりしてしまう。 
\\	試験に失敗する。落第する。不合格になる。 
\\	俗に、面白いことをしゃべろうとして失敗する。冗談・ギャグが受けない状態をいう。
\\	ついうっかり口を滑らせてそう言ってしまった。	
\\	ついうっかり 口[ぐち]を 滑[すべ]らせてそう 言[い]ってしまった。	滑る=すべる= 
\\	物の表面をなめらかに移動する。 
\\	表面がなめらかで地面に接する物が安定を失って自然に動いてしまう。スリップする。 
\\	つかもうとした物が、支えられないで手をすり抜ける。 
\\	ある地位を保てなくなる。 
\\	調子に乗ったまま、事がの望ましくないところにまで至る。余計なことを言ったり書いたりしてしまう。 
\\	試験に失敗する。落第する。不合格になる。 
\\	俗に、面白いことをしゃべろうとして失敗する。冗談・ギャグが受けない状態をいう。
\\	焦らなくていい。	
\\	焦[あせ]らなくていい。	焦る=あせる= 
\\	早くしなければならないと思っていらだつ。気をもむ。落ち着きを失う。気がせく。 
\\	いらだって暴れる。手足をバタバタさせる。
\\	焦らないで体が治るまでじっくり休みなさい。	
\\	焦[あせ]らないで 体[からだ]が 治[なお]るまでじっくり 休[やす]みなさい。	焦る=あせる= 
\\	早くしなければならないと思っていらだつ。気をもむ。落ち着きを失う。気がせく。 
\\	いらだって暴れる。手足をバタバタさせる。
\\	試験が迫ってから焦って勉強してもだめだよ。	
\\	試験[しけん]が 迫[せま]ってから 焦[あせ]って 勉強[べんきょう]してもだめだよ。	焦る=あせる= 
\\	早くしなければならないと思っていらだつ。気をもむ。落ち着きを失う。気がせく。 
\\	いらだって暴れる。手足をバタバタさせる。
\\	彼は損を取り返そうと焦った。	
\\	彼[かれ]は 損[そん]を 取り返[とりかえ]そうと 焦[あせ]った。	焦る=あせる= 
\\	早くしなければならないと思っていらだつ。気をもむ。落ち着きを失う。気がせく。 
\\	いらだって暴れる。手足をバタバタさせる。
\\	焦って結婚するな。	
\\	焦[あせ]って 結婚[けっこん]するな。	焦る=あせる= 
\\	早くしなければならないと思っていらだつ。気をもむ。落ち着きを失う。気がせく。 
\\	いらだって暴れる。手足をバタバタさせる。
\\	あせって結論を出してはいけません。	
\\	あせって 結論[けつろん]を 出[だ]してはいけません。	焦る=あせる= 
\\	早くしなければならないと思っていらだつ。気をもむ。落ち着きを失う。気がせく。 
\\	いらだって暴れる。手足をバタバタさせる。
\\	茶室は簡素なしつらえが特徴だ。	
\\	茶室[ちゃしつ]は 簡素[かんそ]なしつらえが 特徴[とくちょう]だ。	簡素=かんそ=飾り気がなく、質素なこと。また、そのさま。
\\	私は彼女の生活の簡素さに驚いた。	
\\	私[わたし]は 彼女[かのじょ]の 生活[せいかつ]の 簡素[かんそ]さに 驚[おどろ]いた。	簡素=かんそ=飾り気がなく、質素なこと。また、そのさま。
\\	結婚式は身内だけで簡素に済ませました。	
\\	結婚式[けっこんしき]は 身内[みうち]だけで 簡素[かんそ]に 済[す]ませました。	簡素=かんそ=飾り気がなく、質素なこと。また、そのさま。
\\	メールは郵便より簡便で安い。	
\\	メールは 郵便[ゆうびん]より 簡便[かんべん]で 安[やす]い。	簡便=かんべん=簡単で便利なこと。手軽なこと。また、そのさま。
\\	新社長の手腕に期待が寄せられている。	
\\	新[しん] 社長[しゃちょう]の 手腕[しゅわん]に 期待[きたい]が 寄[よ]せられている。	手腕=しゅわん= 
\\	彼は非常に腕力がある。	
\\	彼[かれ]は 非常[ひじょう]に 腕力[わんりょく]がある。	腕力=わんりょく=腕の力。また特に、相手を殴ったり、押さえつけたいする肉体的な力。
\\	腕力では彼にかなわない。	
\\	腕力[わんりょく]では 彼[かれ]にかなわない。	腕力=わんりょく=腕の力。また特に、相手を殴ったり、押さえつけたいする肉体的な力。
\\	(君は)大した腕前だ。	
\\	君[きみ]は) 大[たい]した 腕前[うでまえ]だ。	腕前=うでまえ=巧みに物事をなしうる能力や技術。手並み。技量。うで。
\\	彼の料理の腕前は一流だ。	
\\	彼[かれ]の 料理[りょうり]の 腕前[うでまえ]は 一流[いちりゅう]だ。	腕前=うでまえ=巧みに物事をなしうる能力や技術。手並み。技量。うで。
\\	山頂からの眺めは素晴らしい。	
\\	山頂[さんちょう]からの 眺[なが]めは 素晴[すば]らしい。	
\\	彼らが山頂に立ち、旗を振る姿が新聞に載った。	
\\	彼[かれ]らが 山頂[さんちょう]に 立[た]ち、 旗[はた]を 振[ふ]る 姿[すがた]が 新聞[しんぶん]に 載[の]った。	
\\	その時彼の人気は頂点に達した。	
\\	その 時[とき] 彼[かれ]の 人気[にんき]は 頂点[ちょうてん]に 達[たっ]した。	頂点=ちょうてん= 
\\	一番高い所
\\	(絶頂) 
\\	彼は今権力の頂点にいる。	
\\	彼[かれ]は 今[いま] 権力[けんりょく]の 頂点[ちょうてん]にいる。	頂点=ちょうてん= 
\\	一番高い所
\\	(絶頂) 
\\	人々の不満は頂点に達した。	
\\	人々[ひとびと]の 不満[ふまん]は 頂点[ちょうてん]に 達[たっ]した。	頂点=ちょうてん= 
\\	一番高い所
\\	(絶頂) 
\\	火の粉が舞い上がった。	
\\	火の粉[ひのこ]が 舞い上[まいあ]がった。	火の粉=ひのこ= 
\\	ベスはエリザベスの愛称である。	
\\	ベスはエリザベスの 愛称[あいしょう]である。	愛称=あいしょう= 
\\	あの2人は愛称で呼び合うほど仲がいい。	
\\	あの 2人[ふたり]は 愛称[あいしょう]で 呼[よ]び 合[あ]うほど 仲[なか]がいい。	愛称=あいしょう= 
\\	逮捕された犯人は山田と自称している。	
\\	逮捕[たいほ]された 犯人[はんにん]は 山田[やまだ]と 自称[じしょう]している。	自称=じしょう= 
\\	自分から名乗ること。真偽はともかく、名前・職業・肩書きなどを自分で称すること。 
\\	人称の一。一人称。 
\\	自分で自分を褒めること。
\\	彼女が大卒というのは自称にすぎない。	
\\	彼女[かのじょ]が 大卒[だいそつ]というのは 自称[じしょう]にすぎない。	自称=じしょう= 
\\	自分から名乗ること。真偽はともかく、名前・職業・肩書きなどを自分で称すること。 
\\	人称の一。一人称。 
\\	自分で自分を褒めること。
\\	彼は自称二枚目だ。	
\\	彼[かれ]は 自称[じしょう]二 枚[まい] 目[め]だ。	自称=じしょう= 
\\	自分から名乗ること。真偽はともかく、名前・職業・肩書きなどを自分で称すること。 
\\	人称の一。一人称。 
\\	自分で自分を褒めること。
\\	彼はジャーナリストを自称している。	
\\	彼[かれ]はジャーナリストを 自称[じしょう]している。	自称=じしょう= 
\\	自分から名乗ること。真偽はともかく、名前・職業・肩書きなどを自分で称すること。 
\\	人称の一。一人称。 
\\	自分で自分を褒めること。
\\	地球は赤道に関して対称ではない。	
\\	地球[ちきゅう]は 赤道[せきどう]に 関[かん]して 対称[たいしょう]ではない。	対称=たいしょう= 
\\	ヘビがカエルを狙っている。	
\\	ヘビがカエルを 狙[ねら]っている。	狙う=ねらう= 
\\	目標に命中させようとして、弓・鉄砲などを構える。照準を定める。また、矢・弾などを目標物に命中させようとする。 
\\	あるものを手に入れようとしたり遂行しようとしたりして、その機会をうかがう。 
\\	ある事柄を目標に置く。それを目標として目指す。
\\	鈴木氏は上の地位を狙って運動している。	
\\	鈴木[すずき] 氏[し]は 上[うえ]の 地位[ちい]を 狙[ねら]って 運動[うんどう]している。	狙う=ねらう= 
\\	目標に命中させようとして、弓・鉄砲などを構える。照準を定める。また、矢・弾などを目標物に命中させようとする。 
\\	あるものを手に入れようとしたり遂行しようとしたりして、その機会をうかがう。 
\\	ある事柄を目標に置く。それを目標として目指す。
\\	彼は次期社長のいすを狙っている。	
\\	彼[かれ]は 次期[じき] 社長[しゃちょう]のいすを 狙[ねら]っている。	狙う=ねらう= 
\\	目標に命中させようとして、弓・鉄砲などを構える。照準を定める。また、矢・弾などを目標物に命中させようとする。 
\\	あるものを手に入れようとしたり遂行しようとしたりして、その機会をうかがう。 
\\	ある事柄を目標に置く。それを目標として目指す。
\\	あいつは狙った獲物は逃さない男だ。	
\\	あいつは 狙[ねら]った 獲物[えもの]は 逃[のが]さない 男[おとこ]だ。	狙う=ねらう= 
\\	目標に命中させようとして、弓・鉄砲などを構える。照準を定める。また、矢・弾などを目標物に命中させようとする。 
\\	あるものを手に入れようとしたり遂行しようとしたりして、その機会をうかがう。 
\\	ある事柄を目標に置く。それを目標として目指す。
\\	テロリストたちが彼の命を狙っていた。	
\\	テロリストたちが 彼[かれ]の 命[いのち]を 狙[ねら]っていた。	狙う=ねらう= 
\\	目標に命中させようとして、弓・鉄砲などを構える。照準を定める。また、矢・弾などを目標物に命中させようとする。 
\\	あるものを手に入れようとしたり遂行しようとしたりして、その機会をうかがう。 
\\	ある事柄を目標に置く。それを目標として目指す。
\\	コンビニは夜も開いているので強盗に狙われやすい。	
\\	コンビニは 夜[よる]も 開[あ]いているので 強盗[ごうとう]に 狙[ねら]われやすい。	狙う=ねらう= 
\\	目標に命中させようとして、弓・鉄砲などを構える。照準を定める。また、矢・弾などを目標物に命中させようとする。 
\\	あるものを手に入れようとしたり遂行しようとしたりして、その機会をうかがう。 
\\	ある事柄を目標に置く。それを目標として目指す。
\\	僕は彼女に話しかけるチャンスを狙っていた。	
\\	僕[ぼく]は 彼女[かのじょ]に 話[はな]しかけるチャンスを 狙[ねら]っていた。	狙う=ねらう= 
\\	目標に命中させようとして、弓・鉄砲などを構える。照準を定める。また、矢・弾などを目標物に命中させようとする。 
\\	あるものを手に入れようとしたり遂行しようとしたりして、その機会をうかがう。 
\\	ある事柄を目標に置く。それを目標として目指す。
\\	憶測でものを言うな。	
\\	憶測[おくそく]でものを 言[い]うな。	臆測・憶測=おくそく= 自分でかってに推測すること。当て推量。
\\	これは私の憶測にすぎないのだが・・・	
\\	これは 私[わたし]の 憶測[おくそく]にすぎないのだが・・・	臆測・憶測=おくそく= 自分でかってに推測すること。当て推量。
\\	彼の憶測は見事に外れた。	
\\	彼[かれ]の 憶測[おくそく]は 見事[みごと]に 外[はず]れた。	臆測・憶測=おくそく= 自分でかってに推測すること。当て推量。
\\	それはただの憶測にすぎない。	
\\	それはただの 憶測[おくそく]にすぎない。	臆測・憶測=おくそく= 自分でかってに推測すること。当て推量。
\\	彼女の進退については様々な憶測が飛び交っている。	
\\	彼女[かのじょ]の 進退[しんたい]については 様々[さまざま]な 憶測[おくそく]が 飛び交[とびか]っている。	臆測・憶測=おくそく= 自分でかってに推測すること。当て推量。
\\	鼻先にぶらさがっている問題の方が先決だ。	
\\	鼻先[はなさき]にぶらさがっている 問題[もんだい]の 方[ほう]が 先決[せんけつ]だ。	鼻先=はなさき= 
\\	鼻の先端。鼻の頭。 
\\	目の前。
\\	彼は我々の案を鼻先でせせら笑った。	
\\	彼[かれ]は 我々[われわれ]の 案[あん]を 鼻先[はなさき]でせせら 笑[わら]った。	鼻先=はなさき= 
\\	鼻の先端。鼻の頭。 
\\	目の前。
\\	外国人労働者の移入がこの国の労働力不足を緩和した。	
\\	外国[がいこく] 人[じん] 労働[ろうどう] 者[しゃ]の 移入[いにゅう]がこの 国[くに]の 労働[ろうどう] 力不足[ちからぶそく]を 緩和[かんわ]した。	緩和=かんわ=厳しさや激しさの程度を和らげること。また、和らぐこと。
\\	この薬で鼻づまりが緩和され、楽に寝られます。	
\\	この 薬[くすり]で 鼻[はな]づまりが 緩和[かんわ]され、 楽[らく]に 寝[ね]られます。	緩和=かんわ=厳しさや激しさの程度を和らげること。また、和らぐこと。
\end{CJK}
\end{document}