\documentclass[8pt]{extreport} 
\usepackage{hyperref}
\usepackage{CJKutf8}
\begin{document}
\begin{CJK}{UTF8}{min}
\\	それ	それ	それ	
\\	それはとってもいい話だ。	それはとってもいい 話[はなし]だ。	それ は とっても いい はなし だ	
\\	はとってもいい 話[はなし]だ。			
\\	一つ	一[ひと]つ	ひとつ	
\\	それを一つください。	それを 一[ひと]つください。	それ を ひとつ ください	
\\	それを
\\	ください。			
\\	一	一[いち]	いち	
\\	一から始めましょう。	一[いち]から 始[はじ]めましょう。	いち から はじめましょう	
\\	から 始[はじ]めましょう。			
\\	二	二[に]	に	
\\	その人には二回会った。	その 人[ひと]には 二[に] 回会[かい あ]った。	その ひと に は に かい あった	
\\	その 人[ひと]には
\\	回会[かい あ]った。			
\\	二つ	二[ふた]つ	ふたつ	
\\	ソフトクリームを二つください。	ソフトクリームを 二[ふた]つください。	そふとくりーむ を ふたつ ください	
\\	ソフトクリームを
\\	ください。			
\\	三	三[さん]	さん	
\\	彼女は三人の子供の母親だ。	彼女[かのじょ]は 三[さん] 人[にん]の 子供[こども]の 母親[ははおや]だ。	かのじょ は さんにん の こども の ははおや だ	
\\	彼女[かのじょ]は
\\	人[にん]の 子供[こども]の 母親[ははおや]だ。			
\\	三つ	三[みっ]つ	みっつ	
\\	コップを三つ買いました。	コップを 三[みっ]つ 買[か]いました。	こっぷ を みっつ かいました	
\\	コップを
\\	買[か]いました。			
\\	四つ	四[よっ]つ	よっつ	
\\	私は腕時計を四つ持っています。	私[わたし]は 腕時計[うでどけい]を 四[よっ]つ 持[も]っています。	わたし は うでどけい を よっつ もって います	
\\	私[わたし]は 腕時計[うでどけい]を
\\	持[も]っています。			
\\	四	四[し]	し	
\\	四月に大学に入学しました。	四[し] 月[がつ]に 大学[だいがく]に 入学[にゅうがく]しました。	しがつ に だいがく に にゅうがく しました	
\\	月[がつ]に 大学[だいがく]に 入学[にゅうがく]しました。			
\\	これ	これ	これ	
\\	これをください。	これをください。	これをください。	
\\	をください。			
\\	四	四[よん]	よん	
\\	ハワイは四回目です。	ハワイは 四[よん] 回目[かいめ]です。	はわい は よんかいめ です	
\\	ハワイは
\\	回目[かいめ]です。			
\\	五つ	五[いつ]つ	いつつ	
\\	桃を五つください。	桃[もも]を 五[いつ]つください。	もも を いつつ ください	
\\	桃[もも]を
\\	ください。			
\\	五	五[ご]	ご	
\\	五人で旅行に行きました。	五[ご] 人[にん]で 旅行[りょこう]に 行[い]きました。	ごにん で りょこう に いきました	
\\	人[にん]で 旅行[りょこう]に 行[い]きました。			
\\	六つ	六[むっ]つ	むっつ	
\\	息子は六つになりました。	息子[むすこ]は 六[むっ]つになりました。	むすこ は むっつ に なりました	
\\	息子[むすこ]は
\\	になりました。			
\\	六	六[ろく]	ろく	
\\	彼には子供が六人います。	彼[かれ]には 子供[こども]が 六[ろく] 人[にん]います。	かれ に は こども が ろくにん います	
\\	彼[かれ]には 子供[こども]が
\\	人[にん]います。			
\\	七	七[なな]	なな	
\\	バナナが七本あります。	バナナが 七[なな] 本[ほん]あります。	ばなな が ななほん あります	
\\	バナナが
\\	本[ほん]あります。			
\\	七つ	七[なな]つ	ななつ	
\\	この子は今年七つになります。	この 子[こ]は 今年[ことし] 七[なな]つになります。	この こ は ことし ななつ に なります	
\\	この 子[こ]は 今年[ことし]
\\	になります。			
\\	八	八[はち]	はち	
\\	りんごを八個ください。	りんごを 八[はち] 個[こ]ください。	りんご を はちこ ください	
\\	りんごを
\\	個[こ]ください。			
\\	八つ	八[やっ]つ	やっつ	
\\	あの家には時計が八つあります。	あの 家[いえ]には 時計[とけい]が 八[やっ]つあります。	あの いえ に は とけい が やっつ あります	
\\	あの 家[いえ]には 時計[とけい]が
\\	あります。			
\\	なる	なる	なる	
\\	彼は医者になりました。	彼[かれ]は 医者[いしゃ]になりました。	かれ は いしゃ に なりました	
\\	彼[かれ]は 医者[いしゃ]に
\\	九	九[きゅう]	きゅう	
\\	野球は九人で1チームです。	野球[やきゅう]は 九[きゅう] 人[にん]で 1[ひと]チームです。	やきゅう は きゅうにん で ひとちーむ です	
\\	野球[やきゅう]は
\\	人[にん]で 1[ひと]チームです。			
\\	九つ	九[ここの]つ	ここのつ	
\\	娘は明日九つになります。	娘[むすめ]は 明日[あす] 九[ここの]つになります。	むすめ は あす ここのつ に なります	
\\	娘[むすめ]は 明日[あす]
\\	になります。			
\\	十	十[じゅう]	じゅう	
\\	その子は指で十数えました。	その 子[こ]は 指[ゆび]で 十[じゅう] 数[かぞ]えました。	その こ は ゆび で じゅう かぞえました	
\\	その 子[こ]は 指[ゆび]で
\\	数[かぞ]えました。			
\\	百	百[ひゃく]	ひゃく	
\\	私の祖母は百才です。	私[わたし]の 祖母[そぼ]は 百[ひゃく] 才[さい]です。	わたし の そぼ は ひゃくさい です	
\\	私[わたし]の 祖母[そぼ]は
\\	才[さい]です。			
\\	千	千[せん]	せん	
\\	千円貸してください。	千[せん] 円貸[えん か]してください。	せんえん かして ください	
\\	円貸[えん か]してください。			
\\	万	万[まん]	まん	
\\	この靴は1万円です。	この 靴[くつ]は 1[いち] 万[まん] 円[えん]です。	この くつ は いちまんえん です	
\\	この 靴[くつ]は 1[いち]
\\	円[えん]です。			
\\	円	円[えん]	えん	
\\	そこに大きな円を描いて。	そこに 大[おお]きな 円[えん]を 描[か]いて。	そこ に おおき な えん を かいて	
\\	そこに 大[おお]きな
\\	を 描[か]いて。			
\\	円	円[えん]	えん	
\\	カレーライスは700円です。	カレーライスは 700[ななひゃく] 円[えん]です。	かれーらいす は ななひゃくえん です	
\\	カレーライスは 700[ななひゃく]
\\	です。			
\\	時	時[とき]	とき	
\\	時の経つのは早い。	時[とき]の 経[た]つのは 早[はや]い。	とき の たつ の は はやい	
\\	の 経[た]つのは 早[はや]い。			
\\	する	する	する	
\\	友達と一緒に宿題をした。	友達[ともだち]と 一緒[いっしょ]に 宿題[しゅくだい]をした。	ともだち と いっしょ に しゅくだい を した	
\\	友達[ともだち]と 一緒[いっしょ]に 宿題[しゅくだい]を
\\	時々	時々[ときどき]	ときどき	
\\	彼は時々遅刻します。	彼[かれ]は 時々[ときどき] 遅刻[ちこく]します。	かれ は ときどき ちこく します	
\\	彼[かれ]は
\\	遅刻[ちこく]します。			
\\	日	日[にち]	にち	
\\	私たちは先月11日に結婚しました。	私[わたし]たちは 先月11[せんげつ じゅういち] 日[にち]に 結婚[けっこん]しました。	わたしたち は せんげつ じゅういち にち に けっこん しました	
\\	私[わたし]たちは 先月11[せんげつ じゅういち]
\\	に 結婚[けっこん]しました。			
\\	六日	六日[むいか]	むいか	
\\	六日前に日本に帰ってきました。	六日[むいか] 前[まえ]に 日本[にほん]に 帰[かえ]ってきました。	むいかまえ に にほん に かえって きました	
\\	前[まえ]に 日本[にほん]に 帰[かえ]ってきました。			
\\	三日	三日[みっか]	みっか	
\\	手紙が届くのに三日かかりました。	手紙[てがみ]が 届[とど]くのに 三日[みっか]かかりました。	てがみ が とどく の に みっか かかりました	
\\	手紙[てがみ]が 届[とど]くのに
\\	かかりました。			
\\	五日	五日[いつか]	いつか	
\\	五月五日は祝日です	五月[ごがつ] 五日[いつか]は 祝日[しゅくじつ]です	ごがつ いつか は しゅくじつ です	
\\	五月[ごがつ]
\\	は 祝日[しゅくじつ]です			
\\	八日	八日[ようか]	ようか	
\\	八日からイギリスに行きます。	八日[ようか]からイギリスに 行[い]きます。	ようか から いぎりす に いきます	
\\	からイギリスに 行[い]きます。			
\\	二十日	二十日[はつか]	はつか	
\\	来月の二十日は弟の誕生日です。	来月[らいげつ]の 二十日[はつか]は 弟[おとうと]の 誕生日[たんじょうび]です。	らいげつ の はつか は おとうと の たんじょうび です	
\\	来月[らいげつ]の
\\	は 弟[おとうと]の 誕生日[たんじょうび]です。			
\\	二日	二日[ふつか]	ふつか	
\\	私は二日待った。	私[わたし]は 二日[ふつか] 待[ま]った。	わたし は ふつか まった	
\\	私[わたし]は
\\	待[ま]った。			
\\	九日	九日[ここのか]	ここのか	
\\	九日に荷物が届きます。	九日[ここのか]に 荷物[にもつ]が 届[とど]きます。	ここのか に にもつ が とどきます	
\\	に 荷物[にもつ]が 届[とど]きます。			
\\	ところ	ところ	ところ	
\\	私は友達のところに泊まった。	私[わたし]は 友達[ともだち]のところに 泊[と]まった。	わたし は ともだち の ところ に とまった	
\\	私[わたし]は 友達[ともだち]の
\\	に 泊[と]まった。			
\\	一日	一日[ついたち]	ついたち	
\\	来月の一日は空いていますか。	来月[らいげつ]の 一日[ついたち]は 空[あ]いていますか。	らいげつ の ついたち は あいています か	
\\	来月[らいげつ]の
\\	は 空[あ]いていますか。			
\\	十日	十日[とおか]	とおか	
\\	十日後に帰ります。	十日[とおか] 後[ご]に 帰[かえ]ります。	とおかご に かえります	
\\	後[ご]に 帰[かえ]ります。			
\\	七日	七日[なのか]	なのか	
\\	先月の七日に孫が生まれました。	先月[せんげつ]の 七日[なのか]に 孫[まご]が 生[う]まれました。	せんげつ の なのか に まご が うまれました	
\\	先月[せんげつ]の
\\	に 孫[まご]が 生[う]まれました。			
\\	四日	四日[よっか]	よっか	
\\	新学期は来月の四日からです。	新学期[しんがっき]は 来月[らいげつ]の 四日[よっか]からです。	しんがっき は らいげつ の よっか から です	
\\	新学期[しんがっき]は 来月[らいげつ]の
\\	からです。			
\\	月	月[つき]	つき	
\\	今夜は月がとてもきれいです。	今夜[こんや]は 月[つき]がとてもきれいです。	こんや は つき が とても きれい です	
\\	今夜[こんや]は
\\	がとてもきれいです。			
\\	水	水[みず]	みず	
\\	水を一杯ください。	水[みず]を 一杯[いっぱい]ください。	みず を いっぱい ください	
\\	を 一杯[いっぱい]ください。			
\\	金	金[かね]	かね	
\\	これはかなり金がかかった。	これはかなり 金[かね]がかかった。	これ は かなり かね が かかった	
\\	これはかなり
\\	がかかった。			
\\	日曜日	日曜日[にちようび]	にちようび	
\\	日曜日は海に行きました。	日曜日[にちようび]は 海[うみ]に 行[い]きました。	にちようび は うみ に いきました	
\\	は 海[うみ]に 行[い]きました。			
\\	土曜日	土曜日[どようび]	どようび	
\\	土曜日の夜はクラブに行きます。	土曜日[どようび]の 夜[よる]はクラブに 行[い]きます。	どようび の よる は くらぶ に いきます	
\\	の 夜[よる]はクラブに 行[い]きます。			
\\	やる	やる	やる	
\\	一緒に宿題をやろう。	一緒[いっしょ]に 宿題[しゅくだい]をやろう。	いっしょ に しゅくだい を やろう	
\\	一緒[いっしょ]に 宿題[しゅくだい]を
\\	金曜日	金曜日[きんようび]	きんようび	
\\	金曜日の夜は友達と出かけます。	金曜日[きんようび]の 夜[よる]は 友達[ともだち]と 出[で]かけます。	きんようび の よる は ともだち と でかけます	
\\	の 夜[よる]は 友達[ともだち]と 出[で]かけます。			
\\	月曜日	月曜日[げつようび]	げつようび	
\\	月曜日に会いましょう。	月曜日[げつようび]に 会[あ]いましょう。	げつようび に あいましょう	
\\	に 会[あ]いましょう。			
\\	木曜日	木曜日[もくようび]	もくようび	
\\	木曜日は仕事が休みです。	木曜日[もくようび]は 仕事[しごと]が 休[やす]みです。	もくようび は しごと が やすみ です	
\\	は 仕事[しごと]が 休[やす]みです。			
\\	曜日	曜日[ようび]	ようび	
\\	曜日を間違えました。	曜日[ようび]を 間違[まちが]えました。	ようび を まちがえました	
\\	を 間違[まちが]えました。			
\\	火曜日	火曜日[かようび]	かようび	
\\	火曜日に会議があります。	火曜日[かようび]に 会議[かいぎ]があります。	かようび に かいぎ が あります	
\\	に 会議[かいぎ]があります。			
\\	水曜日	水曜日[すいようび]	すいようび	
\\	水曜日はバイトがあります。	水曜日[すいようび]はバイトがあります。	すいようび は ばいと が あります	
\\	はバイトがあります。			
\\	週	週[しゅう]	しゅう	
\\	その次の週は空いてますか。	その 次[つぎ]の 週[しゅう]は 空[あ]いてますか。	その つぎ の しゅう は あいてます か	
\\	その 次[つぎ]の
\\	は 空[あ]いてますか。			
\\	年	年[とし]	とし	
\\	新しい年が始まりました。	新[あたら]しい 年[とし]が 始[はじ]まりました。	あたらしい とし が はじまりました	
\\	新[あたら]しい
\\	が 始[はじ]まりました。			
\\	分かる	分[わ]かる	わかる	
\\	質問の意味は分かりましたか。	質問[しつもん]の 意味[いみ]は 分[わ]かりましたか。	しつもん の いみ は わかりました か	
\\	質問[しつもん]の 意味[いみ]は
\\	か。			
\\	そう	そう	そう	
\\	私もそう思います。	私[わたし]もそう 思[おも]います。	わたし も そう おもいます	
\\	私[わたし]も
\\	思[おも]います。			
\\	何	何[なに]	なに	
\\	夕食には何を食べたいですか。	夕食[ゆうしょく]には 何[なに]を 食[た]べたいですか。	ゆうしょく に は なに を たべたい です か	
\\	夕食[ゆうしょく]には
\\	を 食[た]べたいですか。			
\\	先	先[さき]	さき	
\\	お先にどうぞ。	お 先[さき]にどうぞ。	おさきに どうぞ	
\\	お
\\	にどうぞ。			
\\	今年	今年[ことし]	ことし	
\\	今年はイタリアに旅行したい。	今年[ことし]はイタリアに 旅行[りょこう]したい。	ことし は いたりあ に りょこう したい	
\\	はイタリアに 旅行[りょこう]したい。			
\\	今	今[いま]	いま	
\\	彼は今、勉強しています。	彼[かれ]は 今[いま]、 勉強[べんきょう]しています。	かれ は いま べんきょう して います	
\\	彼[かれ]は
\\	、 勉強[べんきょう]しています。			
\\	今日	今日[きょう]	きょう	
\\	今日は仕事がありません。	今日[きょう]は 仕事[しごと]がありません。	きょう は しごと が ありません	
\\	は 仕事[しごと]がありません。			
\\	今月	今月[こんげつ]	こんげつ	
\\	今月はとても忙しい。	今月[こんげつ]はとても 忙[いそが]しい。	こんげつ は とても いそがしい	
\\	はとても 忙[いそが]しい。			
\\	今週	今週[こんしゅう]	こんしゅう	
\\	今週は日本語のテストがあります。	今週[こんしゅう]は 日本語[にほんご]のテストがあります。	こんしゅう は にほんご の てすと が あります	
\\	は 日本語[にほんご]のテストがあります。			
\\	来る	来[く]る	くる	
\\	彼は昼過ぎに来ます。	彼[かれ]は 昼過[ひるす]ぎに 来[き]ます。	かれ は ひるすぎ に きます	
\\	彼[かれ]は 昼過[ひるす]ぎに
\\	来年	来年[らいねん]	らいねん	
\\	来年一緒に旅行しましょう。	来年[らいねん] 一緒[いっしょ]に 旅行[りょこう]しましょう。	らいねん いっしょ に りょこう しましょう	
\\	一緒[いっしょ]に 旅行[りょこう]しましょう。			
\\	もう	もう	もう	
\\	彼はもう帰りました。	彼[かれ]はもう 帰[かえ]りました。	かれ は もう かえりました 。	
\\	彼[かれ]は
\\	帰[かえ]りました。			
\\	行く	行[い]く	いく	
\\	日曜日は図書館に行きます。	日曜日[にちようび]は 図書館[としょかん]に 行[い]きます。	にちようび は としょかん に いきます	
\\	日曜日[にちようび]は 図書館[としょかん]に
\\	帰る	帰[かえ]る	かえる	
\\	家に帰ろう。	家[うち]に 帰[かえ]ろう。	うち に かえろう	
\\	家[うち]に
\\	大きい	大[おお]きい	おおきい	
\\	あの大きい建物は何ですか。	あの 大[おお]きい 建物[たてもの]は 何[なん]ですか。	あの おおきい たてもの は なん です か	
\\	あの
\\	建物[たてもの]は 何[なん]ですか。			
\\	小さい	小[ちい]さい	ちいさい	
\\	小さい花が咲いています。	小[ちい]さい 花[はな]が 咲[さ]いています。	ちいさい はな が さいて います	
\\	花[はな]が 咲[さ]いています。			
\\	少ない	少[すく]ない	すくない	
\\	今年は雨が少ないです。	今年[ことし]は 雨[あめ]が 少[すく]ないです。	ことし は あめ が すくない です 。	
\\	今年[ことし]は 雨[あめ]が
\\	です。			
\\	少し	少[すこ]し	すこし	
\\	少し疲れました。	少[すこ]し 疲[つか]れました。	すこし つかれました	
\\	疲[つか]れました。			
\\	多い	多[おお]い	おおい	
\\	京都にはお寺が多い。	京都[きょうと]にはお 寺[てら]が 多[おお]い。	きょうと に は おてら が おおい	
\\	京都[きょうと]にはお 寺[てら]が
\\	多分	多分[たぶん]	たぶん	
\\	彼女は多分家で寝ています。	彼女[かのじょ]は 多分[たぶん] 家[いえ]で 寝[ね]ています。	かのじょ は たぶん いえ で ねて います	
\\	彼女[かのじょ]は
\\	家[いえ]で 寝[ね]ています。			
\\	上げる	上[あ]げる	あげる	
\\	彼は荷物をあみだなに上げた。	彼[かれ]は 荷物[にもつ]をあみだなに 上[あ]げた。	かれ は にもつ を あみだな に あげた	
\\	彼[かれ]は 荷物[にもつ]をあみだなに
\\	よく	よく	よく	
\\	彼女はよく旅行に行きます。	彼女[かのじょ]はよく 旅行[りょこう]に 行[い]きます。	かのじょ は よく りょこう に いきます	
\\	彼女[かのじょ]は
\\	旅行[りょこう]に 行[い]きます。			
\\	上る	上[のぼ]る	のぼる	
\\	猫が屋根に上っている。	猫[ねこ]が 屋根[やね]に 上[のぼ]っている。	ねこ が やね に のぼって いる	
\\	猫[ねこ]が 屋根[やね]に
\\	上	上[うえ]	うえ	
\\	上を向いて。	上[うえ]を 向[む]いて。	うえ を むいて	
\\	を 向[む]いて。			
\\	下	下[した]	した	
\\	財布は机の下にあった。	財布[さいふ]は 机[つくえ]の 下[した]にあった。	さいふ は つくえ の した に あった	
\\	財布[さいふ]は 机[つくえ]の
\\	にあった。			
\\	右	右[みぎ]	みぎ	
\\	右のポケットにハンカチが入っています。	右[みぎ]のポケットにハンカチが 入[はい]っています。	みぎ の ぽけっと に はんかち が はいって います	
\\	のポケットにハンカチが 入[はい]っています。			
\\	左	左[ひだり]	ひだり	
\\	そこを左に曲がってください。	そこを 左[ひだり]に 曲[ま]がってください。	そこ を ひだり に まがって ください	
\\	そこを
\\	に 曲[ま]がってください。			
\\	方	方[ほう]	ほう	
\\	彼は私の方を見ました。	彼[かれ]は 私[わたし]の 方[ほう]を 見[み]ました。	かれ は わたし の ほう を みました	
\\	彼[かれ]は 私[わたし]の
\\	を 見[み]ました。			
\\	大人	大人[おとな]	おとな	
\\	お酒は大人になってから。	お 酒[さけ]は 大人[おとな]になってから。	おさけ は おとな に なって から	
\\	お 酒[さけ]は
\\	になってから。			
\\	人	人[ひと]	ひと	
\\	彼は優しい人です。	彼[かれ]は 優[やさ]しい 人[ひと]です。	かれ は やさしい ひと です	
\\	彼[かれ]は 優[やさ]しい
\\	です。			
\\	一人	一人[ひとり]	ひとり	
\\	そこには私一人しかいなかった。	そこには 私[わたし] 一人[ひとり]しかいなかった。	そこ に は わたし ひとり しか いなかった	
\\	そこには 私[わたし]
\\	しかいなかった。			
\\	どう	どう	どう	
\\	あなたはどう思いますか。	あなたはどう 思[おも]いますか。	あなた は どう おもいます か	
\\	あなたは
\\	思[おも]いますか。			
\\	入れる	入[い]れる	いれる	
\\	彼はかばんに手帳を入れた。	彼[かれ]はかばんに 手帳[てちょう]を 入[い]れた。	かれ は かばん に てちょう を いれた	
\\	彼[かれ]はかばんに 手帳[てちょう]を
\\	入る	入[はい]る	はいる	
\\	寒いので中に入ってください。	寒[さむ]いので 中[なか]に 入[はい]ってください。	さむい の で なか に はいって ください	
\\	寒[さむ]いので 中[なか]に
\\	ください。			
\\	出る	出[で]る	でる	
\\	今朝は早く家を出ました。	今朝[けさ]は 早[はや]く 家[いえ]を 出[で]ました。	けさ は はやく いえ を でました	
\\	今朝[けさ]は 早[はや]く 家[いえ]を
\\	出来る	出来[でき]る	できる	
\\	彼女はイタリア語が出来ます。	彼女[かのじょ]はイタリア 語[ご]が 出来[でき]ます。	かのじょ は いたりあご が できます	
\\	彼女[かのじょ]はイタリア 語[ご]が
\\	出す	出[だ]す	だす	
\\	彼はかばんから教科書を出した。	彼[かれ]はかばんから 教科書[きょうかしょ]を 出[だ]した。	かれ は かばん から きょうかしょ を だした	
\\	彼[かれ]はかばんから 教科書[きょうかしょ]を
\\	本	本[ほん]	ほん	
\\	本を1冊買いました。	本[ほん]を 1冊買[いっさつ か]いました。	ほん を いっさつ かいました	
\\	を 1冊買[いっさつ か]いました。			
\\	休む	休[やす]む	やすむ	
\\	明日、会社を休みます。	明日[あした]、 会社[かいしゃ]を 休[やす]みます。	あした かいしゃ を やすみます	
\\	明日[あした]、 会社[かいしゃ]を
\\	体	体[からだ]	からだ	
\\	私は体が丈夫だ。	私[わたし]は 体[からだ]が 丈夫[じょうぶ]だ。	わたし は からだ が じょうぶ だ	
\\	私[わたし]は
\\	が 丈夫[じょうぶ]だ。			
\\	目	目[め]	め	
\\	彼女は青い目をしています。	彼女[かのじょ]は 青[あお]い 目[め]をしています。	かのじょ は あおい め を して います	
\\	彼女[かのじょ]は 青[あお]い
\\	をしています。			
\\	どこ	どこ	どこ	
\\	あの本をどこに置きましたか。	あの 本[ほん]をどこに 置[お]きましたか。	あの ほん を どこ に おきました か	
\\	あの 本[ほん]を
\\	に 置[お]きましたか。			
\\	口	口[くち]	くち	
\\	口を大きく開けてください。	口[くち]を 大[おお]きく 開[あ]けてください。	くち を おおきく あけて ください	
\\	を 大[おお]きく 開[あ]けてください。			
\\	耳	耳[みみ]	みみ	
\\	彼女は耳がよく聞こえません。	彼女[かのじょ]は 耳[みみ]がよく 聞[き]こえません。	かのじょ は みみ が よく きこえません	
\\	彼女[かのじょ]は
\\	がよく 聞[き]こえません。			
\\	上手	上手[じょうず]	じょうず	
\\	妹は歌が上手です。	妹[いもうと]は 歌[うた]が 上手[じょうず]です。	いもうと は うた が じょうず です	
\\	妹[いもうと]は 歌[うた]が
\\	です。			
\\	手	手[て]	て	
\\	分かった人は手を上げてください。	分[わ]かった 人[ひと]は 手[て]を 上[あ]げてください。	わかった ひと は て を あげて ください	
\\	分[わ]かった 人[ひと]は
\\	を 上[あ]げてください。			
\\	足	足[あし]	あし	
\\	彼は足が長い。	彼[かれ]は 足[あし]が 長[なが]い。	かれ は あし が ながい	
\\	彼[かれ]は
\\	が 長[なが]い。			
\\	空く	空[す]く	すく	
\\	レストランは空いていました。	レストランは 空[す]いていました。	れすとらん は すいて いました	
\\	レストランは
\\	男	男[おとこ]	おとこ	
\\	男の人が私たちに話しかけた。	男[おとこ]の 人[ひと]が 私[わたし]たちに 話[はな]しかけた。	おとこ の ひと が わたしたち に はなしかけた	
\\	の 人[ひと]が 私[わたし]たちに 話[はな]しかけた。			
\\	女	女[おんな]	おんな	
\\	店員は若い女の人でした。	店員[てんいん]は 若[わか]い 女[おんな]の 人[ひと]でした。	てんいん は わかい おんな の ひと でした	
\\	店員[てんいん]は 若[わか]い
\\	の 人[ひと]でした。			
\\	子供	子供[こども]	こども	
\\	電車で子供が騒いでいた。	電車[でんしゃ]で 子供[こども]が 騒[さわ]いでいた。	でんしゃ で こども が さわいで いた	
\\	電車[でんしゃ]で
\\	が 騒[さわ]いでいた。			
\\	あげる	あげる	あげる	
\\	この本、あなたにあげます。	この 本[ほん]、あなたにあげます。	この ほん あなた に あげます	
\\	この 本[ほん]、あなたに
\\	好き	好[す]き	すき	
\\	私はワインが好きです。	私[わたし]はワインが 好[す]きです。	わたし は わいん が すき です	
\\	私[わたし]はワインが
\\	です。			
\\	大好き	大好[だいす]き	だいすき	
\\	私は犬が大好きだ。	私[わたし]は 犬[いぬ]が 大好[だいす]きだ。	わたし は いぬ が だいすき だ	
\\	私[わたし]は 犬[いぬ]が
\\	だ。			
\\	私	私[わたし]	わたし	
\\	私が行きましょう。	私[わたし]が 行[い]きましょう。	わたし が いきましょう	
\\	が 行[い]きましょう。			
\\	友達	友達[ともだち]	ともだち	
\\	私には友達がたくさんいます。	私[わたし]には 友達[ともだち]がたくさんいます。	わたし に は ともだち が たくさん います	
\\	私[わたし]には
\\	がたくさんいます。			
\\	家	家[うち]	うち	
\\	家に遊びに来てください。	家[うち]に 遊[あそ]びに 来[き]てください。	うち に あそび に きて ください	
\\	に 遊[あそ]びに 来[き]てください。			
\\	気	気[き]	き	
\\	彼は意外に気が小さい。	彼[かれ]は 意外[いがい]に 気[き]が 小[ちい]さい。	かれ は いがい に き が ちいさい	
\\	彼[かれ]は 意外[いがい]に
\\	が 小[ちい]さい。			
\\	元気	元気[げんき]	げんき	
\\	おかげさまで元気です。	おかげさまで 元気[げんき]です。	おかげさま で げんき です	
\\	おかげさまで
\\	です。			
\\	天気	天気[てんき]	てんき	
\\	今日はいい天気ですね。	今日[きょう]はいい 天気[てんき]ですね。	きょう は いい てんき です ね	
\\	今日[きょう]はいい
\\	ですね。			
\\	晴れる	晴[は]れる	はれる	
\\	明日は晴れるといいですね。	明日[あした]は 晴[は]れるといいですね。	あした は はれる と いい です ね	
\\	明日[あした]は
\\	といいですね。			
\\	こう	こう	こう	
\\	こう小さい字は読めない。	こう 小[ちい]さい 字[じ]は 読[よ]めない。	こう ちいさい じ は よめない	
\\	小[ちい]さい 字[じ]は 読[よ]めない。			
\\	昨日	昨日[きのう]	きのう	
\\	昨日、友達に会った。	昨日[きのう]、 友達[ともだち]に 会[あ]った。	きのう ともだち に あった	
\\	、 友達[ともだち]に 会[あ]った。			
\\	開ける	開[あ]ける	あける	
\\	窓を開けてください。	窓[まど]を 開[あ]けてください。	まど を あけて ください	
\\	窓[まど]を
\\	ください。			
\\	開く	開[ひら]く	ひらく	
\\	32ページを開いてください。	32[さんじゅうに]ページを 開[ひら]いてください。	さんじゅうにぺーじ を ひらいて ください	
\\	32[さんじゅうに]ページを
\\	ください。			
\\	閉じる	閉[と]じる	とじる	
\\	教科書を閉じてください。	教科書[きょうかしょ]を 閉[と]じてください。	きょうかしょ を とじて ください	
\\	教科書[きょうかしょ]を
\\	ください。			
\\	閉める	閉[し]める	しめる	
\\	ちゃんとドアを閉めてよ。	ちゃんとドアを 閉[し]めてよ。	ちゃんと どあ を しめてよ	
\\	ちゃんとドアを
\\	よ。			
\\	閉まる	閉[し]まる	しまる	
\\	お店はもう閉まっていました。	お 店[みせ]はもう 閉[し]まっていました。	おみせ は もう しまって いました	
\\	お 店[みせ]はもう
\\	聞く	聞[き]く	きく	
\\	彼女はラジオを聞いています。	彼女[かのじょ]はラジオを 聞[き]いています。	かのじょ は らじお を きいて います	
\\	彼女[かのじょ]はラジオを
\\	時間	時間[じかん]	じかん	
\\	今は時間がありません。	今[いま]は 時間[じかん]がありません。	いま は じかん が ありません	
\\	今[いま]は
\\	がありません。			
\\	高い	高[たか]い	たかい	
\\	これがこの町で一番高いビルです。	これがこの 町[まち]で 一番[いちばん] 高[たか]いビルです。	これ が この まち で いちばん たかい びる です	
\\	これがこの 町[まち]で 一番[いちばん]
\\	ビルです。			
\\	くれる	くれる	くれる	
\\	友達が誕生日プレゼントをくれた。	友達[ともだち]が 誕生日[たんじょうび]プレゼントをくれた。	ともだち が たんじょうび ぷれぜんと を くれた	
\\	友達[ともだち]が 誕生日[たんじょうび]プレゼントを
\\	安い	安[やす]い	やすい	
\\	この服はとても安かった。	この 服[ふく]はとても 安[やす]かった。	この ふく は とても やすかった	
\\	この 服[ふく]はとても
\\	低い	低[ひく]い	ひくい	
\\	彼は背が低い。	彼[かれ]は 背[せ]が 低[ひく]い。	かれ は せ が ひくい	
\\	彼[かれ]は 背[せ]が
\\	前	前[まえ]	まえ	
\\	その店の前で会いましょう。	その 店[みせ]の 前[まえ]で 会[あ]いましょう。	その みせ の まえ で あいましょう	
\\	その 店[みせ]の
\\	で 会[あ]いましょう。			
\\	後	後[あと]	あと	
\\	仕事の後、映画を見た。	仕事[しごと]の 後[あと]、 映画[えいが]を 見[み]た。	しごと の あと えいが を みた	
\\	仕事[しごと]の
\\	、 映画[えいが]を 見[み]た。			
\\	後ろ	後[うし]ろ	うしろ	
\\	後ろを向いて。	後[うし]ろを 向[む]いて。	うしろ を むいて	
\\	を 向[む]いて。			
\\	午後	午後[ごご]	ごご	
\\	明日の午後、お客様が来る。	明日[あす]の 午後[ごご]、お 客様[きゃくさま]が 来[く]る。	あす の ごご おきゃくさま が くる	
\\	明日[あす]の
\\	、お 客様[きゃくさま]が 来[く]る。			
\\	午前	午前[ごぜん]	ごぜん	
\\	午前9時のニュースです。	午前[ごぜん] 9時[くじ]のニュースです。	ごぜん くじ の にゅーす です	
\\	9時[くじ]のニュースです。			
\\	朝	朝[あさ]	あさ	
\\	気持ちのいい朝です。	気持[きも]ちのいい 朝[あさ]です。	きもち の いい あさ です	
\\	気持[きも]ちのいい
\\	です。			
\\	昼	昼[ひる]	ひる	
\\	私は昼のドラマを毎日見ます。	私[わたし]は 昼[ひる]のドラマを 毎日見[まいにち み]ます。	わたし は ひる の どらま を まいにち みます	
\\	私[わたし]は
\\	のドラマを 毎日見[まいにち み]ます。			
\\	かなり	かなり	かなり	
\\	彼はかなり英語が上手です。	彼[かれ]はかなり 英語[えいご]が 上手[じょうず]です。	かれ は かなり えいご が じょうず です	
\\	彼[かれ]は
\\	英語[えいご]が 上手[じょうず]です。			
\\	晩	晩[ばん]	ばん	
\\	晩ご飯は食べましたか。	晩[ばん]ご 飯[はん]は 食[た]べましたか。	ばんごはん は たべました か	
\\	ご 飯[はん]は 食[た]べましたか。			
\\	今晩	今晩[こんばん]	こんばん	
\\	今晩のパーティーは何時からですか。	今晩[こんばん]のパーティーは 何時[なんじ]からですか。	こんばん の ぱーてぃー は なんじ から です か	
\\	のパーティーは 何時[なんじ]からですか。			
\\	夜	夜[よる]	よる	
\\	きのうの夜は家にいました。	きのうの 夜[よる]は 家[いえ]にいました。	きのう の よる は いえ に いました	
\\	きのうの
\\	は 家[いえ]にいました。			
\\	食べる	食[た]べる	たべる	
\\	昨日タイカレーを食べました。	昨日[きのう]タイカレーを 食[た]べました。	きのう たいかれー を たべました	
\\	昨日[きのう]タイカレーを
\\	飲む	飲[の]む	のむ	
\\	友達とお酒を飲んでいます。	友達[ともだち]とお 酒[さけ]を 飲[の]んでいます。	ともだち と おさけ を のんで います	
\\	友達[ともだち]とお 酒[さけ]を
\\	ご飯	ご 飯[はん]	ごはん	
\\	私はパンよりご飯が好きだ。	私[わたし]はパンよりご 飯[はん]が 好[す]きだ。	わたし は ぱん より ごはん が すき だ	
\\	私[わたし]はパンより
\\	が 好[す]きだ。			
\\	買う	買[か]う	かう	
\\	郵便局で切手を買いました。	郵便局[ゆうびんきょく]で 切手[きって]を 買[か]いました。	ゆうびんきょく で きって を かいました	
\\	郵便局[ゆうびんきょく]で 切手[きって]を
\\	見る	見[み]る	みる	
\\	私は絵を見るのが好きです。	私[わたし]は 絵[え]を 見[み]るのが 好[す]きです。	わたし は え を みる の が すき です	
\\	私[わたし]は 絵[え]を
\\	のが 好[す]きです。			
\\	見せる	見[み]せる	みせる	
\\	その写真を見せてください。	その 写真[しゃしん]を 見[み]せてください。	その しゃしん を みせて ください	
\\	その 写真[しゃしん]を
\\	ください。			
\\	もっと	もっと	もっと	
\\	もっと近くに来てください。	もっと 近[ちか]くに 来[き]てください。	もっと ちかく に きて ください	
\\	近[ちか]くに 来[き]てください。			
\\	見つける	見[み]つける	みつける	
\\	新しい仕事を見つけました。	新[あたら]しい 仕事[しごと]を 見[み]つけました。	あたらしい しごと を みつけました。	
\\	新[あたら]しい 仕事[しごと]を
\\	見える	見[み]える	みえる	
\\	ここから富士山がよく見えます。	ここから 富士山[ふじさん]がよく 見[み]えます。	ここ から ふじさん が よく みえます	
\\	ここから 富士山[ふじさん]がよく
\\	見つかる	見[み]つかる	みつかる	
\\	メガネが見つかりません。	メガネが 見[み]つかりません。	めがね が みつかりません	
\\	メガネが
\\	言う	言[い]う	いう	
\\	上司が「一杯、飲もう。」と言った。	上司[じょうし]が
\\	一杯[いっぱい]、 飲[の]もう。」と 言[い]った。	じょうし が いっぱい のもう と いった	
\\	上司[じょうし]が
\\	一杯[いっぱい]、 飲[の]もう。」と
\\	話す	話[はな]す	はなす	
\\	母と電話で話しました。	母[はは]と 電話[でんわ]で 話[はな]しました。	はは と でんわ で はなしました	
\\	母[はは]と 電話[でんわ]で
\\	読む	読[よ]む	よむ	
\\	彼女は雑誌を読んでいます。	彼女[かのじょ]は 雑誌[ざっし]を 読[よ]んでいます。	かのじょ は ざっし を よんで います	
\\	彼女[かのじょ]は 雑誌[ざっし]を
\\	漢字	漢字[かんじ]	かんじ	
\\	漢字は中国から来ました。	漢字[かんじ]は 中国[ちゅうごく]から 来[き]ました。	かんじ は ちゅうごく から きました	
\\	は 中国[ちゅうごく]から 来[き]ました。			
\\	書く	書[か]く	かく	
\\	彼に手紙を書きました。	彼[かれ]に 手紙[てがみ]を 書[か]きました。	かれ に てがみ を かきました	
\\	彼[かれ]に 手紙[てがみ]を
\\	覚える	覚[おぼ]える	おぼえる	
\\	妹は平仮名を全部覚えました。	妹[いもうと]は 平仮名[ひらがな]を 全部[ぜんぶ] 覚[おぼ]えました。	いもうと は ひらがな を ぜんぶ おぼえました	
\\	妹[いもうと]は 平仮名[ひらがな]を 全部[ぜんぶ]
\\	そこ	そこ	そこ	
\\	そこに座ってください。	そこに 座[すわ]ってください。	そこ に すわって ください	
\\	に 座[すわ]ってください。			
\\	会う	会[あ]う	あう	
\\	また会いましょう。	また 会[あ]いましょう。	また あいましょう	
\\	また
\\	仕事	仕事[しごと]	しごと	
\\	3月は仕事が忙しい。	3月[さんがつ]は 仕事[しごと]が 忙[いそが]しい。	さんがつ は しごと が いそがしい	
\\	3月[さんがつ]は
\\	が 忙[いそが]しい。			
\\	場合	場合[ばあい]	ばあい	
\\	分からない場合は私に聞いてください。	分[わ]からない 場合[ばあい]は 私[わたし]に 聞[き]いてください。	わからない ばあい は わたし に きいて ください	
\\	分[わ]からない
\\	は 私[わたし]に 聞[き]いてください。			
\\	車	車[くるま]	くるま	
\\	弟が車を買った。	弟[おとうと]が 車[くるま]を 買[か]った。	おとうと が くるま を かった	
\\	弟[おとうと]が
\\	を 買[か]った。			
\\	電車	電車[でんしゃ]	でんしゃ	
\\	私は電車で通学しています。	私[わたし]は 電車[でんしゃ]で 通学[つうがく]しています。	わたし は でんしゃ で つうがく して います	
\\	私[わたし]は
\\	で 通学[つうがく]しています。			
\\	駅	駅[えき]	えき	
\\	駅はどこですか。	駅[えき]はどこですか。	えき は どこ です か	
\\	はどこですか。			
\\	道	道[みち]	みち	
\\	この道を真っ直ぐ行くと駅です。	この 道[みち]を 真[ま]っ 直[す]ぐ 行[い]くと 駅[えき]です。	この みち を まっすぐ いく と えき です	
\\	この
\\	を 真[ま]っ 直[す]ぐ 行[い]くと 駅[えき]です。			
\\	他	他[ほか]	ほか	
\\	他に方法がありません。	他[ほか]に 方法[ほうほう]がありません。	ほか に ほうほう が ありません	
\\	に 方法[ほうほう]がありません。			
\\	止める	止[や]める	やめる	
\\	話すのを止めてください。	話[はな]すのを 止[や]めてください。	はなす の を やめて ください	
\\	話[はな]すのを
\\	ください。			
\\	ここ	ここ	ここ	
\\	ここに本があります。	ここに 本[ほん]があります。	ここ に ほん が あります	
\\	に 本[ほん]があります。			
\\	歩く	歩[ある]く	あるく	
\\	駅まで歩きましょう。	駅[えき]まで 歩[ある]きましょう。	えき まで あるきましょう	
\\	駅[えき]まで
\\	走る	走[はし]る	はしる	
\\	彼は毎晩3キロ走っています。	彼[かれ]は 毎晩3[まいばん さん]キロ 走[はし]っています。	かれ は まいばん さんきろ はしって います	
\\	彼[かれ]は 毎晩3[まいばん さん]キロ
\\	近く	近[ちか]く	ちかく	
\\	駅の近くで食事をした。	駅[えき]の 近[ちか]くで 食事[しょくじ]をした。	えき の ちかく で しょくじ を した	
\\	駅[えき]の
\\	で 食事[しょくじ]をした。			
\\	近い	近[ちか]い	ちかい	
\\	駅はここから近いです。	駅[えき]はここから 近[ちか]いです。	えき は ここ から ちかい です	
\\	駅[えき]はここから
\\	です。			
\\	近く	近[ちか]く	ちかく	
\\	私の家は駅の近くです。	私[わたし]の 家[いえ]は 駅[えき]の 近[ちか]くです。	わたし の いえ は えき の ちかく です	
\\	私[わたし]の 家[いえ]は 駅[えき]の
\\	です。			
\\	遠い	遠[とお]い	とおい	
\\	家から学校までは遠いです。	家[いえ]から 学校[がっこう]までは 遠[とお]いです。	いえ から がっこう まで は とおい です	
\\	家[いえ]から 学校[がっこう]までは
\\	です。			
\\	長い	長[なが]い	ながい	
\\	彼女の髪はとても長い。	彼女[かのじょ]の 髪[かみ]はとても 長[なが]い。	かのじょ の かみ は とても ながい	
\\	彼女[かのじょ]の 髪[かみ]はとても
\\	短い	短[みじか]い	みじかい	
\\	彼は足が短い。	彼[かれ]は 足[あし]が 短[みじか]い。	かれ は あし が みじかい	
\\	彼[かれ]は 足[あし]が
\\	広い	広[ひろ]い	ひろい	
\\	彼の家はとても広い。	彼[かれ]の 家[いえ]はとても 広[ひろ]い。	かれ の いえ は とても ひろい	
\\	彼[かれ]の 家[いえ]はとても
\\	もらう	もらう	もらう	
\\	彼女からプレゼントをもらいました。	彼女[かのじょ]からプレゼントをもらいました。	かのじょ から ぷれぜんと を もらいました	
\\	彼女[かのじょ]からプレゼントを
\\	全部	全部[ぜんぶ]	ぜんぶ	
\\	それ、全部ください。	それ、 全部[ぜんぶ]ください。	それ ぜんぶ ください	
\\	それ、
\\	ください。			
\\	国	国[くに]	くに	
\\	私の国について少しお話しましょう。	私[わたし]の 国[くに]について 少[すこ]しお 話[はなし]しましょう。	わたし の くに に ついて すこし おはなし しましょう	
\\	私[わたし]の
\\	について 少[すこ]しお 話[はなし]しましょう。			
\\	白い	白[しろ]い	しろい	
\\	彼は歯が白い。	彼[かれ]は 歯[は]が 白[しろ]い。	かれ は は が しろい	
\\	彼[かれ]は 歯[は]が
\\	赤い	赤[あか]い	あかい	
\\	赤いバラを買いました。	赤[あか]いバラを 買[か]いました。	あかい ばら を かいました	
\\	バラを 買[か]いました。			
\\	部屋	部屋[へや]	へや	
\\	私の部屋は2階にあります。	私[わたし]の 部屋[へや]は 2階[にかい]にあります。	わたし の へや は にかい に あります	
\\	私[わたし]の
\\	は 2階[にかい]にあります。			
\\	米	米[こめ]	こめ	
\\	日本人はお米が大好きです。	日本人[にほんじん]はお 米[こめ]が 大好[だいす]きです。	にほんじん は おこめ が だいすき です	
\\	日本人[にほんじん]はお
\\	が 大好[だいす]きです。			
\\	未だ	未[ま]だ	まだ	
\\	宿題は未だ終わっていません。	宿題[しゅくだい]は 未[ま]だ 終[お]わっていません。	しゅくだい は まだ おわって いません	
\\	宿題[しゅくだい]は
\\	終[お]わっていません。			
\\	有る	有[あ]る	ある	
\\	私の机の上に書類がたくさん有ります。	私[わたし]の 机[つくえ]の 上[うえ]に 書類[しょるい]がたくさん 有[あ]ります。	わたし の つくえ の うえ に しょるい が たくさん あります	
\\	私[わたし]の 机[つくえ]の 上[うえ]に 書類[しょるい]がたくさん
\\	無い	無[な]い	ない	
\\	ここには何も無い。	ここには 何[なに]も 無[な]い。	ここ に は なに も ない	
\\	ここには 何[なに]も
\\	とても	とても	とても	
\\	この本はとてもおもしろい。	この 本[ほん]はとてもおもしろい。	この ほん は とても おもしろい	
\\	この 本[ほん]は
\\	おもしろい。			
\\	作る	作[つく]る	つくる	
\\	今、朝ご飯を作っています。	今[いま]、 朝[あさ]ご 飯[はん]を 作[つく]っています。	いま あさごはん を つくって います	
\\	今[いま]、 朝[あさ]ご 飯[はん]を
\\	使う	使[つか]う	つかう	
\\	このパソコンを使ってください。	このパソコンを 使[つか]ってください。	この ぱそこん を つかって ください	
\\	このパソコンを
\\	ください。			
\\	消す	消[け]す	けす	
\\	昼間は電気を消してください。	昼間[ひるま]は 電気[でんき]を 消[け]してください。	ひるま は でんき を けして ください	
\\	昼間[ひるま]は 電気[でんき]を
\\	ください。			
\\	売る	売[う]る	うる	
\\	彼は家を売った。	彼[かれ]は 家[いえ]を 売[う]った。	かれ は いえ を うった	
\\	彼[かれ]は 家[いえ]を
\\	店	店[みせ]	みせ	
\\	私はこの店によく来ます。	私[わたし]はこの 店[みせ]によく 来[き]ます。	わたし は この みせ に よく きます	
\\	私[わたし]はこの
\\	によく 来[き]ます。			
\\	春	春[はる]	はる	
\\	今年の春は暖かいね。	今年[ことし]の 春[はる]は 暖[あたた]かいね。	ことし の はる は あたたかい ね	
\\	今年[ことし]の
\\	は 暖[あたた]かいね。			
\\	夏	夏[なつ]	なつ	
\\	私は夏が大好き。	私[わたし]は 夏[なつ]が 大好[だいす]き。	わたし は なつ が だいすき	
\\	私[わたし]は
\\	が 大好[だいす]き。			
\\	秋	秋[あき]	あき	
\\	彼女は秋に結婚します。	彼女[かのじょ]は 秋[あき]に 結婚[けっこん]します。	かのじょ は あき に けっこん します	
\\	彼女[かのじょ]は
\\	に 結婚[けっこん]します。			
\\	冬	冬[ふゆ]	ふゆ	
\\	カナダの冬はとても寒いです。	カナダの 冬[ふゆ]はとても 寒[さむ]いです。	かなだ の ふゆ は とても さむい です	
\\	カナダの
\\	はとても 寒[さむ]いです。			
\\	みんな	みんな	みんな	
\\	みんなにお菓子をあげましょう。	みんなにお 菓子[かし]をあげましょう。	みんな に おかし を あげましょう	
\\	にお 菓子[かし]をあげましょう。			
\\	暑い	暑[あつ]い	あつい	
\\	今日はとても暑い。	今日[きょう]はとても 暑[あつ]い。	きょう は とても あつい	
\\	今日[きょう]はとても
\\	熱い	熱[あつ]い	あつい	
\\	このスープはとても熱い。	このスープはとても 熱[あつ]い。	この すーぷ は とても あつい	
\\	このスープはとても
\\	寒い	寒[さむ]い	さむい	
\\	この部屋は寒いです。	この 部屋[へや]は 寒[さむ]いです。	この へや は さむい です	
\\	この 部屋[へや]は
\\	です。			
\\	暖かい	暖[あたた]かい	あたたかい	
\\	このコートはとても暖かい。	このコートはとても 暖[あたた]かい。	この こーと は とても あたたかい。	
\\	このコートはとても
\\	新しい	新[あたら]しい	あたらしい	
\\	彼の車は新しい。	彼[かれ]の 車[くるま]は 新[あたら]しい。	かれ の くるま は あたらしい	
\\	彼[かれ]の 車[くるま]は
\\	古い	古[ふる]い	ふるい	
\\	私は古い車が好きです。	私[わたし]は 古[ふる]い 車[くるま]が 好[す]きです。	わたし は ふるい くるま が すき です	
\\	私[わたし]は
\\	車[くるま]が 好[す]きです。			
\\	良い	良[い]い	いい	
\\	彼女は良い友達です。	彼女[かのじょ]は 良[い]い 友達[ともだち]です。	かのじょ は いい ともだち です	
\\	彼女[かのじょ]は
\\	友達[ともだち]です。			
\\	悪い	悪[わる]い	わるい	
\\	たばこは体に悪い。	たばこは 体[からだ]に 悪[わる]い。	たばこ は からだ に わるい	
\\	たばこは 体[からだ]に
\\	思う	思[おも]う	おもう	
\\	私もそう思います。	私[わたし]もそう 思[おも]います。	わたし も そう おもいます	
\\	私[わたし]もそう
\\	いつも	いつも	いつも	
\\	彼女はいつも元気だ。	彼女[かのじょ]はいつも 元気[げんき]だ。	かのじょ は いつも げんき だ	
\\	彼女[かのじょ]は
\\	元気[げんき]だ。			
\\	忘れる	忘[わす]れる	わすれる	
\\	約束を忘れないでください。	約束[やくそく]を 忘[わす]れないでください。	やくそく を わすれない で ください	
\\	約束[やくそく]を
\\	ください。			
\\	考える	考[かんが]える	かんがえる	
\\	よく考えてください。	よく 考[かんが]えてください。	よく かんがえて ください	
\\	よく
\\	ください。			
\\	決める	決[き]める	きめる	
\\	帰国することに決めました。	帰国[きこく]することに 決[き]めました。	きこく する こと に きめました	
\\	帰国[きこく]することに
\\	決まる	決[き]まる	きまる	
\\	旅行の日程が決まりました。	旅行[りょこう]の 日程[にってい]が 決[き]まりました。	りょこう の にってい が きまりました	
\\	旅行[りょこう]の 日程[にってい]が
\\	知る	知[し]る	しる	
\\	誰もその話を知らない。	誰[だれ]もその 話[はなし]を 知[し]らない。	だれ も その はなし を しらない	
\\	誰[だれ]もその 話[はなし]を
\\	一番	一番[いちばん]	いちばん	
\\	彼はクラスで一番背が高い。	彼[かれ]はクラスで 一番[いちばん] 背[せ]が 高[たか]い。	かれ は くらす で いちばん せ が たかい	
\\	彼[かれ]はクラスで
\\	背[せ]が 高[たか]い。			
\\	住む	住[す]む	すむ	
\\	彼は会社の近くに住んでいる。	彼[かれ]は 会社[かいしゃ]の 近[ちか]くに 住[す]んでいる。	かれ は かいしゃ の ちかく に すんで いる	
\\	彼[かれ]は 会社[かいしゃ]の 近[ちか]くに
\\	名前	名前[なまえ]	なまえ	
\\	あなたの名前を教えてください。	あなたの 名前[なまえ]を 教[おし]えてください。	あなた の なまえ を おしえて ください	
\\	あなたの
\\	を 教[おし]えてください。			
\\	食べ物	食[た]べ 物[もの]	たべもの	
\\	日本の食べ物はとても美味しいです。	日本[にっぽん]の 食[た]べ 物[もの]はとても 美味[おい]しいです。	にっぽん の たべもの は とても おいしい です	
\\	日本[にっぽん]の
\\	はとても 美味[おい]しいです。			
\\	いつ	いつ	いつ	
\\	彼女はいつ来ますか。	彼女[かのじょ]はいつ 来[き]ますか。	かのじょ は いつ きます か	
\\	彼女[かのじょ]は
\\	来[き]ますか。			
\\	飲み物	飲[の]み 物[もの]	のみもの	
\\	何か飲み物が欲しいな。	何[なに]か 飲[の]み 物[もの]が 欲[ほ]しいな。	なに か のみもの が ほしい な 。	
\\	何[なに]か
\\	が 欲[ほ]しいな。			
\\	重い	重[おも]い	おもい	
\\	このかばんは重いです。	このかばんは 重[おも]いです。	この かばん は おもい です	
\\	このかばんは
\\	です。			
\\	軽い	軽[かる]い	かるい	
\\	この靴はとても軽い。	この 靴[くつ]はとても 軽[かる]い。	この くつ は とても かるい	
\\	この 靴[くつ]はとても
\\	送る	送[おく]る	おくる	
\\	彼の家に荷物を送りました。	彼[かれ]の 家[いえ]に 荷物[にもつ]を 送[おく]りました。	かれ の いえ に にもつ を おくりました	
\\	彼[かれ]の 家[いえ]に 荷物[にもつ]を
\\	取る	取[と]る	とる	
\\	テストでいい点を取った。	テストでいい 点[てん]を 取[と]った。	てすと で いい てん を とった	
\\	テストでいい 点[てん]を
\\	待つ	待[ま]つ	まつ	
\\	あなたが来るのを待っています。	あなたが 来[く]るのを 待[ま]っています。	あなた が くる の を まって います	
\\	あなたが 来[く]るのを
\\	持つ	持[も]つ	もつ	
\\	私は車を持っています。	私[わたし]は 車[くるま]を 持[も]っています。	わたし は くるま を もって います	
\\	私[わたし]は 車[くるま]を
\\	気持ち	気持[きも]ち	きもち	
\\	彼の気持ちが分からない。	彼[かれ]の 気持[きも]ちが 分[わ]からない。	かれ の きもち が わからない	
\\	彼[かれ]の
\\	が 分[わ]からない。			
\\	生きる	生[い]きる	いきる	
\\	皆一生懸命生きている。	皆一生懸命[みんな いっしょうけんめい] 生[い]きている。	みんな いっしょうけんめい いきて いる	
\\	皆一生懸命[みんな いっしょうけんめい]
\\	どちら	どちら	どちら	
\\	肉と魚とどちらが好きですか。	肉[にく]と 魚[さかな]とどちらが 好[す]きですか。	にく と さかな と どちら が すき です か	
\\	肉[にく]と 魚[さかな]と
\\	が 好[す]きですか。			
\\	先生	先生[せんせい]	せんせい	
\\	私は日本語の先生になりたいです。	私[わたし]は 日本語[にほんご]の 先生[せんせい]になりたいです。	わたし は にほんご の せんせい に なりたい です	
\\	私[わたし]は 日本語[にほんご]の
\\	になりたいです。			
\\	大学	大学[だいがく]	だいがく	
\\	大学に行ってもっと勉強したいです。	大学[だいがく]に 行[い]ってもっと 勉強[べんきょう]したいです。	だいがく に いって もっと べんきょう したい です	
\\	に 行[い]ってもっと 勉強[べんきょう]したいです。			
\\	学生	学生[がくせい]	がくせい	
\\	彼は真面目な学生です。	彼[かれ]は 真面目[まじめ]な 学生[がくせい]です。	かれ は まじめ な がくせい です	
\\	彼[かれ]は 真面目[まじめ]な
\\	です。			
\\	大学生	大学生[だいがくせい]	だいがくせい	
\\	姉は大学生です。	姉[あね]は 大学生[だいがくせい]です。	あね は だいがくせい です	
\\	姉[あね]は
\\	です。			
\\	学校	学校[がっこう]	がっこう	
\\	学校は8時半に始まります。	学校[がっこう]は 8時半[はちじはん]に 始[はじ]まります。	がっこう は はちじはん に はじまります	
\\	は 8時半[はちじはん]に 始[はじ]まります。			
\\	高校生	高校生[こうこうせい]	こうこうせい	
\\	私の弟は高校生です。	私[わたし]の 弟[おとうと]は 高校生[こうこうせい]です。	わたし の おとうと は こうこうせい です	
\\	私[わたし]の 弟[おとうと]は
\\	です。			
\\	教える	教[おし]える	おしえる	
\\	彼は数学を教えています。	彼[かれ]は 数学[すうがく]を 教[おし]えています。	かれ は すうがく を おしえて います	
\\	彼[かれ]は 数学[すうがく]を
\\	勉強	勉強[べんきょう]	べんきょう	
\\	私は日本語を勉強しています。	私[わたし]は 日本語[にほんご]を 勉強[べんきょう]しています。	わたし は にほんご を べんきょう して います	
\\	私[わたし]は 日本語[にほんご]を
\\	しています。			
\\	強い	強[つよ]い	つよい	
\\	今日は風が強い。	今日[きょう]は 風[かぜ]が 強[つよ]い。	きょう は かぜ が つよい	
\\	今日[きょう]は 風[かぜ]が
\\	どれ	どれ	どれ	
\\	この中でどれが好きですか。	この 中[なか]でどれが 好[す]きですか。	この なか で どれ が すき です か	
\\	この 中[なか]で
\\	が 好[す]きですか。			
\\	弱い	弱[よわ]い	よわい	
\\	その子は体が少し弱い。	その 子[こ]は 体[からだ]が 少[すこ]し 弱[よわ]い。	その こ は からだ が すこし よわい	
\\	その 子[こ]は 体[からだ]が 少[すこ]し
\\	引く	引[ひ]く	ひく	
\\	このドアは引いてください。	このドアは 引[ひ]いてください。	この どあ は ひいて ください	
\\	このドアは
\\	ください。			
\\	質問	質問[しつもん]	しつもん	
\\	質問のある方はどうぞ。	質問[しつもん]のある 方[かた]はどうぞ。	しつもん の ある かた は どうぞ	
\\	のある 方[かた]はどうぞ。			
\\	難しい	難[むずか]しい	むずかしい	
\\	この本は難しいですね。	この 本[ほん]は 難[むずか]しいですね。	この ほん は むずかしい です ね	
\\	この 本[ほん]は
\\	ですね。			
\\	数	数[かず]	かず	
\\	グラスの数が足りません。	グラスの 数[かず]が 足[た]りません。	ぐらす の かず が たりません	
\\	グラスの
\\	が 足[た]りません。			
\\	勝つ	勝[か]つ	かつ	
\\	今日はヤンキースが勝った。	今日[きょう]はヤンキースが 勝[か]った。	きょう は やんきーす が かった	
\\	今日[きょう]はヤンキースが
\\	負ける	負[ま]ける	まける	
\\	私たちのチームはその試合で負けた。	私[わたし]たちのチームはその 試合[しあい]で 負[ま]けた。	わたしたち の ちーむ は その しあい で まけた	
\\	私[わたし]たちのチームはその 試合[しあい]で
\\	本当に	本当[ほんとう]に	ほんとうに	
\\	あなたが本当に好きです。	あなたが 本当[ほんとう]に 好[す]きです。	あなた が ほんとうに すき です	
\\	あなたが
\\	好[す]きです。			
\\	要る	要[い]る	いる	
\\	予約は要りません。	予約[よやく]は 要[い]りません。	よやく は いりません	
\\	予約[よやく]は
\\	ドア	ドア	ドア	
\\	ドアを開けてください。	ドアを 開[あ]けてください。	どあ を あけて ください	
\\	を 開[あ]けてください。			
\\	時計	時計[とけい]	とけい	
\\	時計を見たらちょうど3時だった。	時計[とけい]を 見[み]たらちょうど 3時[さんじ]だった。	とけい を みたら ちょうど さんじ だった	
\\	を 見[み]たらちょうど 3時[さんじ]だった。			
\\	払う	払[はら]う	はらう	
\\	私が払いましょう。	私[わたし]が 払[はら]いましょう。	わたし が はらいましょう	
\\	私[わたし]が
\\	切る	切[き]る	きる	
\\	この紙を半分に切ってください。	この 紙[かみ]を 半分[はんぶん]に 切[き]ってください。	この かみ を はんぶん に きって ください	
\\	この 紙[かみ]を 半分[はんぶん]に
\\	ください。			
\\	変える	変[か]える	かえる	
\\	旅行の日程を変えました。	旅行[りょこう]の 日程[にってい]を 変[か]えました。	りょこう の にってい を かえました	
\\	旅行[りょこう]の 日程[にってい]を
\\	乗る	乗[の]る	のる	
\\	駅からはタクシーに乗ってください。	駅[えき]からはタクシーに 乗[の]ってください。	えき からは たくしー に のって ください	
\\	駅[えき]からはタクシーに
\\	ください。			
\\	着る	着[き]る	きる	
\\	今日はスーツを着ています。	今日[きょう]はスーツを 着[き]ています。	きょう は すーつ を きて います	
\\	今日[きょう]はスーツを
\\	立つ	立[た]つ	たつ	
\\	彼はステージに立った。	彼[かれ]はステージに 立[た]った。	かれ は すてーじ に たった	
\\	彼[かれ]はステージに
\\	座る	座[すわ]る	すわる	
\\	私は窓側の席に座った。	私[わたし]は 窓側[まどがわ]の 席[せき]に 座[すわ]った。	わたし は まどがわ の せき に すわった	
\\	私[わたし]は 窓側[まどがわ]の 席[せき]に
\\	次	次[つぎ]	つぎ	
\\	次はいつ会いましょうか。	次[つぎ]はいつ 会[あ]いましょうか。	つぎ は いつ あいましょう か	
\\	はいつ 会[あ]いましょうか。			
\\	しゃべる	しゃべる	しゃべる	
\\	彼女はよくしゃべるね。	彼女[かのじょ]はよくしゃべるね。	かのじょ は よく しゃべる ね	
\\	彼女[かのじょ]はよく
\\	ね。			
\\	動く	動[うご]く	うごく	
\\	動かないで。	動[うご]かないで 。	うごかない で	
\\	働く	働[はたら]く	はたらく	
\\	姉は銀行で働いています。	姉[あね]は 銀行[ぎんこう]で 働[はたら]いています。	あね は ぎんこう で はたらいています。	
\\	姉[あね]は 銀行[ぎんこう]で
\\	早い	早[はや]い	はやい	
\\	まだ学校へ行くには早い時間です。	まだ 学校[がっこう]へ 行[い]くには 早[はや]い 時間[じかん]です。	まだ がっこう へ いく に は はやい じかん です	
\\	まだ 学校[がっこう]へ 行[い]くには
\\	時間[じかん]です。			
\\	速い	速[はや]い	はやい	
\\	彼は走るのが速い。	彼[かれ]は 走[はし]るのが 速[はや]い。	かれ は はしる の が はやい	
\\	彼[かれ]は 走[はし]るのが
\\	遅い	遅[おそ]い	おそい	
\\	前の車はとても遅い。	前[まえ]の 車[くるま]はとても 遅[おそ]い。	まえ の くるま は とても おそい	
\\	前[まえ]の 車[くるま]はとても
\\	始める	始[はじ]める	はじめる	
\\	テストを始めてください。	テストを 始[はじ]めてください。	てすと を はじめて ください	
\\	テストを
\\	ください。			
\\	始まる	始[はじ]まる	はじまる	
\\	新しい仕事が始まりました。	新[あたら]しい 仕事[しごと]が 始[はじ]まりました。	あたらしい しごと が はじまりました	
\\	新[あたら]しい 仕事[しごと]が
\\	終わる	終[お]わる	おわる	
\\	会議は4時に終わります。	会議[かいぎ]は 4時[よじ]に 終[お]わります。	かいぎ は よじ に おわります	
\\	会議[かいぎ]は 4時[よじ]に
\\	終わり	終[お]わり	おわり	
\\	夏休みももう終わりだ。	夏休[なつやす]みももう 終[お]わりだ。	なつやすみ も もう おわり だ	
\\	夏休[なつやす]みももう
\\	だ。			
\\	テーブル	テーブル	テーブル	
\\	新しいテーブルを買いました。	新[あたら]しいテーブルを 買[か]いました。	あたらしい てーぶる を かいました	
\\	新[あたら]しい
\\	を 買[か]いました。			
\\	去年	去年[きょねん]	きょねん	
\\	私は去年フランスへ行った。	私[わたし]は 去年[きょねん]フランスへ 行[い]った。	わたし は きょねん ふらんす へ いった	
\\	私[わたし]は
\\	フランスへ 行[い]った。			
\\	紙	紙[かみ]	かみ	
\\	紙と鉛筆はありますか。	紙[かみ]と 鉛筆[えんぴつ]はありますか。	かみ と えんぴつ は あります か	
\\	と 鉛筆[えんぴつ]はありますか。			
\\	楽しむ	楽[たの]しむ	たのしむ	
\\	今日は一人の時間を楽しみたい。	今日[きょう]は 一人[ひとり]の 時間[じかん]を 楽[たの]しみたい。	きょう は ひとり の じかん を たのしみたい	
\\	今日[きょう]は 一人[ひとり]の 時間[じかん]を
\\	楽しい	楽[たの]しい	たのしい	
\\	彼はとても楽しい人です。	彼[かれ]はとても 楽[たの]しい 人[ひと]です。	かれ は とても たのしい ひと です	
\\	彼[かれ]はとても
\\	人[ひと]です。			
\\	歌う	歌[うた]う	うたう	
\\	私たちは大きな声で歌いました。	私[わたし]たちは 大[おお]きな 声[こえ]で 歌[うた]いました。	わたしたち は おおき な こえ で うたいました	
\\	私[わたし]たちは 大[おお]きな 声[こえ]で
\\	欲しい	欲[ほ]しい	ほしい	
\\	僕は新しい靴が欲しいです。	僕[ぼく]は 新[あたら]しい 靴[くつ]が 欲[ほ]しいです。	ぼく は あたらしい くつ が ほしい です	
\\	僕[ぼく]は 新[あたら]しい 靴[くつ]が
\\	です。			
\\	色	色[いろ]	いろ	
\\	すてきな色のセーターですね。	すてきな 色[いろ]のセーターですね。	すてき な いろ の せーたー です ね	
\\	すてきな
\\	のセーターですね。			
\\	茶色	茶色[ちゃいろ]	ちゃいろ	
\\	彼女は茶色の靴を履いています。	彼女[かのじょ]は 茶色[ちゃいろ]の 靴[くつ]を 履[は]いています。	かのじょ は ちゃいろ の くつ を はいて います	
\\	彼女[かのじょ]は
\\	の 靴[くつ]を 履[は]いています。			
\\	直ぐ	直[す]ぐ	すぐ	
\\	直ぐ行きます。	直[す]ぐ 行[い]きます。	すぐ いきます	
\\	行[い]きます。			
\\	ゲーム	ゲーム	ゲーム	
\\	私たちはビデオゲームをした。	私[わたし]たちはビデオゲームをした。	わたしたち は びでおげーむ を した	
\\	私[わたし]たちはビデオ
\\	をした。			
\\	書き直す	書[か]き 直[なお]す	かきなおす	
\\	この書類を書き直してください。	この 書類[しょるい]を 書[か]き 直[なお]してください。	この しょるい を かきなおして ください	
\\	この 書類[しょるい]を
\\	ください。			
\\	直る	直[なお]る	なおる	
\\	クーラーはまだ直りません。	クーラーはまだ 直[なお]りません。	くーらー は まだ なおりません	
\\	クーラーはまだ
\\	線	線[せん]	せん	
\\	赤い線を2本引いて下さい。	赤[あか]い 線[せん]を 2本引[に ほん ひ]いて 下[くだ]さい。	あかい せん を に ほん ひいて ください	
\\	赤[あか]い
\\	を 2本引[に ほん ひ]いて 下[くだ]さい。			
\\	曲がる	曲[ま]がる	まがる	
\\	そこを左に曲がってください。	そこを 左[ひだり]に 曲[ま]がってください。	そこ を ひだり に まがって ください	
\\	そこを 左[ひだり]に
\\	ください。			
\\	同じ	同[おな]じ	おなじ	
\\	彼の日本語のレベルは私と同じ位だ。	彼[かれ]の 日本語[にほんご]のレベルは 私[わたし]と 同[おな]じ 位[くらい]だ。	かれ の にほんご の れべる は わたし と おなじ くらい だ	
\\	彼[かれ]の 日本語[にほんご]のレベルは 私[わたし]と
\\	位[くらい]だ。			
\\	違う	違[ちが]う	ちがう	
\\	答えが違います。	答[こた]えが 違[ちが]います。	こたえ が ちがいます	
\\	答[こた]えが
\\	図書館	図書館[としょかん]	としょかん	
\\	図書館で料理の本を借りた。	図書館[としょかん]で 料理[りょうり]の 本[ほん]を 借[か]りた。	としょかん で りょうり の ほん を かりた	
\\	で 料理[りょうり]の 本[ほん]を 借[か]りた。			
\\	泊まる	泊[と]まる	とまる	
\\	今日はこのホテルに泊まります。	今日[きょう]はこのホテルに 泊[と]まります。	きょう は この ほてる に とまります	
\\	今日[きょう]はこのホテルに
\\	遊ぶ	遊[あそ]ぶ	あそぶ	
\\	子供たちが公園で遊んでいる。	子供[こども]たちが 公園[こうえん]で 遊[あそ]んでいる。	こどもたち が こうえん で あそんで いる	
\\	子供[こども]たちが 公園[こうえん]で
\\	どうして	どうして	どうして	
\\	どうして泣いているの。	どうして 泣[な]いているの。	どうして ないて いる の	
\\	泣[な]いているの。			
\\	服	服[ふく]	ふく	
\\	昨日、新しい服を買った。	昨日[きのう]、 新[あたら]しい 服[ふく]を 買[か]った。	きのう あたらしい ふく を かった	
\\	昨日[きのう]、 新[あたら]しい
\\	を 買[か]った。			
\\	お父さん	お 父[とう]さん	おとうさん	
\\	お父さんは会社員です。	お 父[とう]さんは 会社員[かいしゃいん]です。	おとうさん は かいしゃいん です	
\\	は 会社員[かいしゃいん]です。			
\\	父	父[ちち]	ちち	
\\	私は父が大好きです。	私[わたし]は 父[ちち]が 大好[だいす]きです。	わたし は ちち が だいすき です	
\\	私[わたし]は
\\	が 大好[だいす]きです。			
\\	お母さん	お 母[かあ]さん	おかあさん	
\\	お母さんによろしくお伝えください。	お 母[かあ]さんによろしくお 伝[つた]えください。	おかあさん に よろしく おつたえ ください	
\\	によろしくお 伝[つた]えください。			
\\	母	母[はは]	はは	
\\	昨日、母と話をしました。	昨日[きのう]、 母[はは]と 話[はなし]をしました。	きのう はは と はなし を しました	
\\	昨日[きのう]、
\\	と 話[はなし]をしました。			
\\	親	親[おや]	おや	
\\	親の愛は有り難い。	親[おや]の 愛[あい]は 有[あ]り 難[がた]い。	おや の あい は ありがたい	
\\	の 愛[あい]は 有[あ]り 難[がた]い。			
\\	姉	姉[あね]	あね	
\\	姉は大学生です。	姉[あね]は 大学生[だいがくせい]です。	あね は だいがくせい です	
\\	は 大学生[だいがくせい]です。			
\\	お姉さん	お 姉[ねえ]さん	おねえさん	
\\	昨日、あなたのお姉さんに会ったよ。	昨日[きのう]、あなたのお 姉[ねえ]さんに 会[あ]ったよ。	きのう あなた の おねえさん に あった よ	
\\	昨日[きのう]、あなたの
\\	に 会[あ]ったよ。			
\\	妹	妹[いもうと]	いもうと	
\\	私の妹は小学生です。	私[わたし]の 妹[いもうと]は 小学生[しょうがくせい]です。	わたし の いもうと は しょうがくせい です	
\\	私[わたし]の
\\	は 小学生[しょうがくせい]です。			
\\	おなか	おなか	おなか	
\\	おなかが空きました。	おなかが 空[す]きました。	おなか が すきました	
\\	が 空[す]きました。			
\\	兄	兄[あに]	あに	
\\	兄は水泳が得意です。	兄[あに]は 水泳[すいえい]が 得意[とくい]です。	あに は すいえい が とくい です	
\\	は 水泳[すいえい]が 得意[とくい]です。			
\\	お兄さん	お 兄[にい]さん	おにいさん	
\\	あなたのお兄さんは何歳?	あなたのお 兄[にい]さんは 何歳?[なんさい]	あなた の おにいさん は なんさい	
\\	あなたの
\\	は 何歳?[なんさい]			
\\	弟	弟[おとうと]	おとうと	
\\	弟は野球が好きです。	弟[おとうと]は 野球[やきゅう]が 好[す]きです。	おとうと は やきゅう が すき です	
\\	は 野球[やきゅう]が 好[す]きです。			
\\	娘	娘[むすめ]	むすめ	
\\	私の娘はアメリカにいます。	私[わたし]の 娘[むすめ]はアメリカにいます。	わたし の むすめ は あめりか に います	
\\	私[わたし]の
\\	はアメリカにいます。			
\\	息子	息子[むすこ]	むすこ	
\\	うちの息子は大学1年生です。	うちの 息子[むすこ]は 大学1年生[だいがく いちねんせい]です。	うち の むすこ は だいがく いちねんせい です	
\\	うちの
\\	は 大学1年生[だいがく いちねんせい]です。			
\\	若い	若[わか]い	わかい	
\\	彼はまだ若いです。	彼[かれ]はまだ 若[わか]いです。	かれ は まだ わかい です	
\\	彼[かれ]はまだ
\\	です。			
\\	彼女	彼女[かのじょ]	かのじょ	
\\	彼女は
\\	です。	彼女[かのじょ]は 
\\	[おーえる]です。	かのじょ は おーえる です	
\\	は 
\\	[おーえる]です。			
\\	彼	彼[かれ]	かれ	
\\	彼は私の上司です。	彼[かれ]は 私[わたし]の 上司[じょうし]です。	かれ は わたし の じょうし です	
\\	は 私[わたし]の 上司[じょうし]です。			
\\	結婚	結婚[けっこん]	けっこん	
\\	彼女は来月結婚します。	彼女[かのじょ]は 来月[らいげつ] 結婚[けっこん]します。	かのじょ は らいげつ けっこん します	
\\	彼女[かのじょ]は 来月[らいげつ]
\\	します。			
\\	ミーティング	ミーティング	ミーティング	
\\	朝9時からミーティングが始まった。	朝9時[あさ 
\\	じ]からミーティングが 始[はじ]まった。	あさ 
\\	じ から みーてぃんぐ が はじまった	
\\	朝9時[あさ 
\\	じ]から
\\	が 始[はじ]まった。			
\\	自転車	自転車[じてんしゃ]	じてんしゃ	
\\	毎日、駅まで自転車で行きます。	毎日[まいにち]、 駅[えき]まで 自転車[じてんしゃ]で 行[い]きます。	まいにち えき まで じてんしゃ で いきます	
\\	毎日[まいにち]、 駅[えき]まで
\\	で 行[い]きます。			
\\	自分	自分[じぶん]	じぶん	
\\	宿題は自分でやりなさい。	宿題[しゅくだい]は 自分[じぶん]でやりなさい。	しゅくだい は じぶん で やりなさい	
\\	宿題[しゅくだい]は
\\	でやりなさい。			
\\	ご主人	ご 主人[しゅじん]	ごしゅじん	
\\	ご主人はお元気ですか。	ご 主人[しゅじん]はお 元気[げんき]ですか。	ごしゅじん は おげんき です か	
\\	はお 元気[げんき]ですか。			
\\	答える	答[こた]える	こたえる	
\\	私の質問に答えてください。	私[わたし]の 質問[しつもん]に 答[こた]えてください。	わたし の しつもん に こたえて ください	
\\	私[わたし]の 質問[しつもん]に
\\	ください。			
\\	別	別[べつ]	べつ	
\\	別の本も見せてください。	別[べつ]の 本[ほん]も 見[み]せてください。	べつ の ほん も みせて ください	
\\	の 本[ほん]も 見[み]せてください。			
\\	病気	病気[びょうき]	びょうき	
\\	祖父が病気になった。	祖父[そふ]が 病気[びょうき]になった。	そふ が びょうき に なった	
\\	祖父[そふ]が
\\	になった。			
\\	死ぬ	死[し]ぬ	しぬ	
\\	犬が病気で死にました。	犬[いぬ]が 病気[びょうき]で 死[し]にました。	いぬ が びょうき で しにました。	
\\	犬[いぬ]が 病気[びょうき]で
\\	痛い	痛[いた]い	いたい	
\\	今日は頭が痛いです。	今日[きょう]は 頭[あたま]が 痛[いた]いです。	きょう は あたま が いたい です	
\\	今日[きょう]は 頭[あたま]が
\\	です。			
\\	酒	酒[さけ]	さけ	
\\	彼女は酒に強い。	彼女[かのじょ]は 酒[さけ]に 強[つよ]い。	かのじょ は さけ に つよい	
\\	彼女[かのじょ]は
\\	に 強[つよ]い。			
\\	つける	つける	つける	
\\	電気をつけてください。	電気[でんき]をつけてください。	でんき を つけて ください	
\\	電気[でんき]を
\\	ください。			
\\	一杯	一杯[いっぱい]	いっぱい	
\\	プールは人で一杯です。	プールは 人[ひと]で 一杯[いっぱい]です。	ぷーる は ひと で いっぱい です	
\\	プールは 人[ひと]で
\\	です。			
\\	飛ぶ	飛[と]ぶ	とぶ	
\\	鳥が飛んでいます。	鳥[とり]が 飛[と]んでいます。	とり が とんで います	
\\	鳥[とり]が
\\	飛行機	飛行機[ひこうき]	ひこうき	
\\	息子は飛行機のおもちゃが好きです。	息子[むすこ]は 飛行機[ひこうき]のおもちゃが 好[す]きです。	むすこ は ひこうき の おもちゃ が すき です	
\\	息子[むすこ]は
\\	のおもちゃが 好[す]きです。			
\\	お願い	お 願[ねが]い	おねがい	
\\	お願いがあります。	お 願[ねが]いがあります。	おねがい が あります	
\\	があります。			
\\	続く	続[つづ]く	つづく	
\\	工事は3月まで続きます。	工事[こうじ]は 3月[さんがつ]まで 続[つづ]きます。	こうじ は さんがつ まで つづきます	
\\	工事[こうじ]は 3月[さんがつ]まで
\\	日記	日記[にっき]	にっき	
\\	私は毎日、日記を付けています。	私[わたし]は 毎日[まいにち]、 日記[にっき]を 付[つ]けています。	わたし は まいにち にっき を つけて います	
\\	私[わたし]は 毎日[まいにち]、
\\	を 付[つ]けています。			
\\	首	首[くび]	くび	
\\	きりんの首は長い。	きりんの 首[くび]は 長[なが]い。	きりん の くび は ながい	
\\	きりんの
\\	は 長[なが]い。			
\\	頭	頭[あたま]	あたま	
\\	今朝から頭が痛い。	今朝[けさ]から 頭[あたま]が 痛[いた]い。	けさ から あたま が いたい	
\\	今朝[けさ]から
\\	が 痛[いた]い。			
\\	顔	顔[かお]	かお	
\\	彼はタオルで顔を拭きました。	彼[かれ]はタオルで 顔[かお]を 拭[ふ]きました。	かれ は たおる で かお を ふきました	
\\	彼[かれ]はタオルで
\\	を 拭[ふ]きました。			
\\	あれ	あれ	あれ	
\\	あれは何ですか。	あれは 何[なん]ですか。	あれ は なん です か	
\\	は 何[なん]ですか。			
\\	感じる	感[かん]じる	かんじる	
\\	膝に痛みを感じます。	膝[ひざ]に 痛[いた]みを 感[かん]じます。	ひざ に いたみ を かんじます	
\\	膝[ひざ]に 痛[いた]みを
\\	探す	探[さが]す	さがす	
\\	彼は郵便局を探していました。	彼[かれ]は 郵便局[ゆうびんきょく]を 探[さが]していました。	かれ は ゆうびんきょく を さがして いました	
\\	彼[かれ]は 郵便局[ゆうびんきょく]を
\\	落ちる	落[お]ちる	おちる	
\\	猿も木から落ちる。	猿[さる]も 木[き]から 落[お]ちる。	さる も き から おちる	
\\	猿[さる]も 木[き]から
\\	お手洗い	お 手洗[てあら]い	おてあらい	
\\	お手洗いはどこですか。	お 手洗[てあら]いはどこですか。	おてあらい は どこ です か	
\\	はどこですか。			
\\	冷たい	冷[つめ]たい	つめたい	
\\	冷たい飲み物をください。	冷[つめ]たい 飲[の]み 物[もの]をください。	つめたい のみもの を ください	
\\	飲[の]み 物[もの]をください。			
\\	汚い	汚[きたな]い	きたない	
\\	彼の部屋はとても汚い。	彼[かれ]の 部屋[へや]はとても 汚[きたな]い。	かれ の へや は とても きたない	
\\	彼[かれ]の 部屋[へや]はとても
\\	太い	太[ふと]い	ふとい	
\\	彼女は足が太い。	彼女[かのじょ]は 足[あし]が 太[ふと]い。	かのじょ は あし が ふとい	
\\	彼女[かのじょ]は 足[あし]が
\\	曇る	曇[くも]る	くもる	
\\	明日は昼頃から曇るでしょう。	明日[あす]は 昼頃[ひるごろ]から 曇[くも]るでしょう。	あす は ひるごろ から くもる でしょう	
\\	明日[あす]は 昼頃[ひるごろ]から
\\	でしょう。			
\\	神	神[かみ]	かみ	
\\	彼は神を信じている。	彼[かれ]は 神[かみ]を 信[しん]じている。	かれ は かみ を しんじて いる	
\\	彼[かれ]は
\\	を 信[しん]じている。			
\\	まずい	まずい	まずい	
\\	ここの料理はまずい。	ここの 料理[りょうり]はまずい。	ここ の りょうり は まずい	
\\	ここの 料理[りょうり]は
\\	建てる	建[た]てる	たてる	
\\	私たちは来年、家を建てます。	私[わたし]たちは 来年[らいねん]、 家[いえ]を 建[た]てます。	わたしたち は らいねん いえ を たてます	
\\	私[わたし]たちは 来年[らいねん]、 家[いえ]を
\\	置く	置[お]く	おく	
\\	彼はかばんをいすの上に置きました。	彼[かれ]はかばんをいすの 上[うえ]に 置[お]きました。	かれ は かばん を いす の うえ に おきました	
\\	彼[かれ]はかばんをいすの 上[うえ]に
\\	辺	辺[へん]	へん	
\\	彼はこの辺に住んでいます。	彼[かれ]はこの 辺[へん]に 住[す]んでいます。	かれ は このへん に すんで います	
\\	彼[かれ]はこの
\\	に 住[す]んでいます。			
\\	黄色い	黄色[きいろ]い	きいろい	
\\	彼女に黄色いバラを買いました。	彼女[かのじょ]に 黄色[きいろ]いバラを 買[か]いました。	かのじょ に きいろい ばら を かいました	
\\	彼女[かのじょ]に
\\	バラを 買[か]いました。			
\\	一緒に	一緒[いっしょ]に	いっしょに	
\\	一緒に帰りましょう。	一緒[いっしょ]に 帰[かえ]りましょう。	いっしょに かえりましょう	
\\	帰[かえ]りましょう。			
\\	緑	緑[みどり]	みどり	
\\	この町には緑がたくさんあります。	この 町[まち]には 緑[みどり]がたくさんあります。	この まち に は みどり が たくさん あります	
\\	この 町[まち]には
\\	がたくさんあります。			
\\	易しい	易[やさ]しい	やさしい	
\\	この問題はかなり易しいです。	この 問題[もんだい]はかなり 易[やさ]しいです。	この もんだい は かなり やさしい です	
\\	この 問題[もんだい]はかなり
\\	です。			
\\	留学生	留学生[りゅうがくせい]	りゅうがくせい	
\\	彼は留学生です。	彼[かれ]は 留学生[りゅうがくせい]です。	かれ は りゅうがくせい です	
\\	彼[かれ]は
\\	です。			
\\	戻る	戻[もど]る	もどる	
\\	今、会社に戻ります。	今[いま]、 会社[かいしゃ]に 戻[もど]ります。	いま かいしゃ に もどります	
\\	今[いま]、 会社[かいしゃ]に
\\	そば	そば	そば	
\\	そばにいて下さい。	そばにいて 下[くだ]さい。	そば に いて ください	
\\	にいて 下[くだ]さい。			
\\	起きる	起[お]きる	おきる	
\\	私は毎朝6時に起きます。	私[わたし]は 毎朝6時[まいあさ ろくじ]に 起[お]きます。	わたし は まいあさ ろくじ に おきます	
\\	私[わたし]は 毎朝6時[まいあさ ろくじ]に
\\	起こる	起[お]こる	おこる	
\\	町で大事件が起こりました。	町[まち]で 大事件[だいじけん]が 起[お]こりました。	まち で だいじけん が おこりました	
\\	町[まち]で 大事件[だいじけん]が
\\	起こす	起[お]こす	おこす	
\\	明日、6時に起こしてください。	明日[あす]、 6時[ろくじ]に 起[お]こしてください。	あす ろくじ に おこして ください	
\\	明日[あす]、 6時[ろくじ]に
\\	ください。			
\\	起きる	起[お]きる	おきる	
\\	近所で盗難事件が起きました。	近所[きんじょ]で 盗難事件[とうなん じけん]が 起[お]きました。	きんじょ で とうなん じけん が おきました	
\\	近所[きんじょ]で 盗難事件[とうなん じけん]が
\\	寝る	寝[ね]る	ねる	
\\	もう寝よう。	もう 寝[ね]よう。	もう ねよう	
\\	もう
\\	細い	細[ほそ]い	ほそい	
\\	彼女は指が細いですね。	彼女[かのじょ]は 指[ゆび]が 細[ほそ]いですね。	かのじょ は ゆび が ほそい です ね	
\\	彼女[かのじょ]は 指[ゆび]が
\\	ですね。			
\\	載せる	載[の]せる	のせる	
\\	皿にケーキを載せました。	皿[さら]にケーキを 載[の]せました。	さら に けーき を のせました。	
\\	皿[さら]にケーキを
\\	締める	締[し]める	しめる	
\\	彼はシートベルトを締めた。	彼[かれ]はシートベルトを 締[し]めた。	かれ は しーとべると を しめた	
\\	彼[かれ]はシートベルトを
\\	甘い	甘[あま]い	あまい	
\\	このイチゴは甘い。	このイチゴは 甘[あま]い。	この いちご は あまい	
\\	このイチゴは
\\	こっち	こっち	こっち	
\\	こっちに来て下さい。	こっちに 来[き]て 下[くだ]さい。	こっち に きて ください	
\\	に 来[き]て 下[くだ]さい。			
\\	辛い	辛[から]い	からい	
\\	彼は辛いものが好きです。	彼[かれ]は 辛[から]いものが 好[す]きです。	かれ は からい もの が すき です	
\\	彼[かれ]は
\\	ものが 好[す]きです。			
\\	優しい	優[やさ]しい	やさしい	
\\	彼はとても優しい人です。	彼[かれ]はとても 優[やさ]しい 人[ひと]です。	かれ は とても やさしい ひと です	
\\	彼[かれ]はとても
\\	人[ひと]です。			
\\	夫	夫[おっと]	おっと	
\\	私の夫はサラリーマンです。	私[わたし]の 夫[おっと]はサラリーマンです。	わたし の おっと は さらりーまん です	
\\	私[わたし]の
\\	はサラリーマンです。			
\\	妻	妻[つま]	つま	
\\	今日は妻の誕生日だ。	今日[きょう]は 妻[つま]の 誕生日[たんじょうび]だ。	きょう は つま の たんじょうび だ	
\\	今日[きょう]は
\\	の 誕生日[たんじょうび]だ。			
\\	誰	誰[だれ]	だれ	
\\	誰と会ってみたいですか。	誰[だれ]と 会[あ]ってみたいですか。	だれ と あって みたい です か	
\\	と 会[あ]ってみたいですか。			
\\	愛する	愛[あい]する	あいする	
\\	私は家族を愛しています。	私[わたし]は 家族[かぞく]を 愛[あい]しています。	わたし は かぞく を あいして います	
\\	私[わたし]は 家族[かぞく]を
\\	笑う	笑[わら]う	わらう	
\\	赤ん坊が笑っています。	赤[あか]ん 坊[ぼう]が 笑[わら]っています。	あかんぼう が わらって います	
\\	赤[あか]ん 坊[ぼう]が
\\	酸っぱい	酸[す]っぱい	すっぱい	
\\	このぶどうは酸っぱいです。	このぶどうは 酸[す]っぱいです。	この ぶどう は すっぱい です	
\\	このぶどうは
\\	です。			
\\	言葉	言葉[ことば]	ことば	
\\	この言葉の意味が分かりません。	この 言葉[ことば]の 意味[いみ]が 分[わ]かりません。	この ことば の いみ が わかりません	
\\	この
\\	の 意味[いみ]が 分[わ]かりません。			
\\	よう	よう	よう	
\\	彼女は眠いようです。	彼女[かのじょ]は 眠[ねむ]いようです。	かのじょ は ねむい よう です	
\\	彼女[かのじょ]は 眠[ねむ]い
\\	です。			
\\	呼ぶ	呼[よ]ぶ	よぶ	
\\	ウェイターを呼びましょう。	ウェイターを 呼[よ]びましょう。	うぇいたー を よびましょう	
\\	ウェイターを
\\	胸	胸[むね]	むね	
\\	胸に少し痛みがあります。	胸[むね]に 少[すこ]し 痛[いた]みがあります。	むね に すこし いたみ が あります	
\\	に 少[すこ]し 痛[いた]みがあります。			
\\	腰	腰[こし]	こし	
\\	昨日から腰が痛い。	昨日[きのう]から 腰[こし]が 痛[いた]い。	きのう から こし が いたい	
\\	昨日[きのう]から
\\	が 痛[いた]い。			
\\	背	背[せ]	せ	
\\	彼女は背が高い。	彼女[かのじょ]は 背[せ]が 高[たか]い。	かのじょ は せ が たかい	
\\	彼女[かのじょ]は
\\	が 高[たか]い。			
\\	片仮名	片仮名[かたかな]	かたかな	
\\	お名前を片仮名で書いてください。	お 名前[なまえ]を 片仮名[かたかな]で 書[か]いてください。	おなまえ を かたかな で かいて ください	
\\	お 名前[なまえ]を
\\	で 書[か]いてください。			
\\	平仮名	平仮名[ひらがな]	ひらがな	
\\	私は平仮名を全部読めます。	私[わたし]は 平仮名[ひらがな]を 全部読[ぜんぶ よ]めます。	わたし は ひらがな を ぜんぶ よめます	
\\	私[わたし]は
\\	を 全部読[ぜんぶ よ]めます。			
\\	悲しい	悲[かな]しい	かなしい	
\\	その映画はとても悲しかった。	その 映画[えいが]はとても 悲[かな]しかった。	その えいが は とても かなしかった	
\\	その 映画[えいが]はとても
\\	美しい	美[うつく]しい	うつくしい	
\\	この絵は美しいです。	この 絵[え]は 美[うつく]しいです。	この え は うつくしい です	
\\	この 絵[え]は
\\	です。			
\\	授業	授業[じゅぎょう]	じゅぎょう	
\\	今日は日本語の授業があります。	今日[きょう]は 日本語[にほんご]の 授業[じゅぎょう]があります。	きょう は にほんご の じゅぎょう が あります	
\\	今日[きょう]は 日本語[にほんご]の
\\	があります。			
\\	あそこ	あそこ	あそこ	
\\	あそこにバス停があります。	あそこにバス 停[てい]があります。	あそこ に ばすてい が あります	
\\	にバス 停[てい]があります。			
\\	手伝う	手伝[てつだ]う	てつだう	
\\	私が手伝いましょう。	私[わたし]が 手伝[てつだ]いましょう。	わたし が てつだいましょう	
\\	私[わたし]が
\\	狭い	狭[せま]い	せまい	
\\	私の部屋は狭いです。	私[わたし]の 部屋[へや]は 狭[せま]いです。	わたし の へや は せまい です	
\\	私[わたし]の 部屋[へや]は
\\	です。			
\\	触る	触[さわ]る	さわる	
\\	絵に触らないでください。	絵[え]に 触[さわ]らないでください。	え に さわらない で ください	
\\	絵[え]に
\\	ください。			
\\	嫌い	嫌[きら]い	きらい	
\\	私はタバコが嫌いです。	私[わたし]はタバコが 嫌[きら]いです。	わたし は たばこ が きらい です	
\\	私[わたし]はタバコが
\\	です。			
\\	浴びる	浴[あ]びる	あびる	
\\	私は朝、シャワーを浴びます。	私[わたし]は 朝[あさ]、シャワーを 浴[あ]びます。	わたし は あさ しゃわー を あびます	
\\	私[わたし]は 朝[あさ]、シャワーを
\\	渇く	渇[かわ]く	かわく	
\\	喉が渇きました。	喉[のど]が 渇[かわ]きました。	のど が かわきました	
\\	喉[のど]が
\\	髪の毛	髪[かみ]の 毛[け]	かみのけ	
\\	髪の毛が伸びたね。	髪[かみ]の 毛[け]が 伸[の]びたね。	かみのけ が のびた ね	
\\	が 伸[の]びたね。			
\\	肩	肩[かた]	かた	
\\	肩が凝りました。	肩[かた]が 凝[こ]りました。	かた が こりました	
\\	が 凝[こ]りました。			
\\	鼻	鼻[はな]	はな	
\\	鼻がかゆいです。	鼻[はな]がかゆいです。	はな が かゆい です	
\\	がかゆいです。			
\\	おいしい	おいしい	おいしい	
\\	このケーキはおいしいね。	このケーキはおいしいね。	この けーき は おいしい ね	
\\	このケーキは
\\	ね。			
\\	腕	腕[うで]	うで	
\\	彼の腕は太い。	彼[かれ]の 腕[うで]は 太[ふと]い。	かれ の うで は ふとい	
\\	彼[かれ]の
\\	は 太[ふと]い。			
\\	掛ける	掛[か]ける	かける	
\\	夫の服をハンガーに掛けた。	夫[おっと]の 服[ふく]をハンガーに 掛[か]けた。	おっと の ふく を はんがー に かけた	
\\	夫[おっと]の 服[ふく]をハンガーに
\\	僕	僕[ぼく]	ぼく	
\\	僕は学生です。	僕[ぼく]は 学生[がくせい]です。	ぼく は がくせい です	
\\	は 学生[がくせい]です。			
\\	駄目	駄目[だめ]	だめ	
\\	彼は駄目な男だ。	彼[かれ]は 駄目[だめ]な 男[おとこ]だ。	かれ は だめ な おとこ だ	
\\	彼[かれ]は
\\	な 男[おとこ]だ。			
\\	大丈夫	大丈夫[だいじょうぶ]	だいじょうぶ	
\\	大丈夫ですか。	大丈夫[だいじょうぶ]ですか。	だいじょうぶ です か	
\\	ですか。			
\\	風邪	風邪[かぜ]	かぜ	
\\	私は風邪を引きました。	私[わたし]は 風邪[かぜ]を 引[ひ]きました。	わたし は かぜ を ひきました 。	
\\	私[わたし]は
\\	を 引[ひ]きました。			
\\	奇麗	奇麗[きれい]	きれい	
\\	彼女はとても奇麗だ。	彼女[かのじょ]はとても 奇麗[きれい]だ。	かのじょ は とても きれい だ	
\\	彼女[かのじょ]はとても
\\	だ。			
\\	嬉しい	嬉[うれ]しい	うれしい	
\\	彼に会えて嬉しかった。	彼[かれ]に 会[あ]えて 嬉[うれ]しかった。	かれ に あえて うれしかった	
\\	彼[かれ]に 会[あ]えて
\\	なる	なる	なる	
\\	柿の木に実がたくさんなっています。	柿[かき]の 木[き]に 実[み]がたくさんなっています。	かきのき に み が たくさん なって います	
\\	柿[かき]の 木[き]に 実[み]がたくさん
\\	ため	ため	ため	
\\	これは医者のためのサイトです。	これは 医者[いしゃ]のためのサイトです。	これ は いしゃ の ため の さいと です	
\\	これは 医者[いしゃ]の
\\	のサイトです。			
\\	より	より	より	
\\	これから、より一層努力します。	これから、より 一層努力[いっそう どりょく]します。	これから より いっそう どりょく します	
\\	これから、
\\	一層努力[いっそう どりょく]します。			
\\	七	七[しち]	しち	
\\	そのグループのメンバーは全部で七人だ。	そのグループのメンバーは 全部[ぜんぶ]で 七[しち] 人[にん]だ。	その ぐるーぷ の めんばー は ぜんぶ で しちにん だ	
\\	そのグループのメンバーは 全部[ぜんぶ]で
\\	人[にん]だ。			
\\	九	九[く]	く	
\\	私は九月に行く予定です。	私[わたし]は 九[く] 月[がつ]に 行[い]く 予定[よてい]です。	わたし は くがつ に いく よてい です	
\\	私[わたし]は
\\	月[がつ]に 行[い]く 予定[よてい]です。			
\\	億	億[おく]	おく	
\\	世界人口は65億人だ。	世界人口[せかい じんこう]は 65[ろくじゅうご] 億[おく] 人[にん]だ。	せかい じんこう は ろくじゅうごおくにん だ	
\\	世界人口[せかい じんこう]は 65[ろくじゅうご]
\\	人[にん]だ。			
\\	寺	寺[てら]	てら	
\\	あそこに古いお寺があります。	あそこに 古[ふる]いお 寺[てら]があります。	あそこ に ふるい おてら が あります	
\\	あそこに 古[ふる]いお
\\	があります。			
\\	ドル	ドル	ドル	
\\	この服は300ドルしました。	この 服[ふく]は 300[さんびゃく]ドルしました。	この ふく は さんびゃくどる しました	
\\	この 服[ふく]は 300[さんびゃく]
\\	しました。			
\\	日	日[ひ]	ひ	
\\	夏は日が長い。	夏[なつ]は 日[ひ]が 長[なが]い。	なつ は ひ が ながい	
\\	夏[なつ]は
\\	が 長[なが]い。			
\\	火	火[ひ]	ひ	
\\	火を消して。	火[ひ]を 消[け]して。	ひ を けして	
\\	を 消[け]して。			
\\	木	木[き]	き	
\\	台風で木が倒れた。	台風[たいふう]で 木[き]が 倒[たお]れた。	たいふう で き が たおれた	
\\	台風[たいふう]で
\\	が 倒[たお]れた。			
\\	金	金[きん]	きん	
\\	彼女は金のネックレスをしています。	彼女[かのじょ]は 金[きん]のネックレスをしています。	かのじょ は きん の ねっくれす を して います	
\\	彼女[かのじょ]は
\\	のネックレスをしています。			
\\	システム	システム	システム	
\\	全てのシステムが停止した。	全[すべ]てのシステムが 停止[ていし]した。	すべて の しすてむ が ていし した	
\\	全[すべ]ての
\\	が 停止[ていし]した。			
\\	十分	十分[じゅうぶん]	じゅうぶん	
\\	お金はまだ十分あります。	お 金[かね]はまだ 十分[じゅうぶん]あります。	おかね は まだ じゅうぶん あります	
\\	お 金[かね]はまだ
\\	あります。			
\\	分	分[ぶん]	ぶん	
\\	このケーキはあなたの分です。	このケーキはあなたの 分[ぶん]です。	この けーき は あなた の ぶん です	
\\	このケーキはあなたの
\\	です。			
\\	分ける	分[わ]ける	わける	
\\	お菓子をみんなで分けました。	お 菓子[かし]をみんなで 分[わ]けました。	おかし を みんな で わけました	
\\	お 菓子[かし]をみんなで
\\	分かれる	分[わ]かれる	わかれる	
\\	グループの中で意見が分かれました。	グループの 中[なか]で 意見[いけん]が 分[わ]かれました。	ぐるーぷ の なか で いけん が わかれました	
\\	グループの 中[なか]で 意見[いけん]が
\\	コンピューター	コンピューター	コンピューター	
\\	新しいコンピューターを買った。	新[あたら]しいコンピューターを 買[か]った。	あたらしい こんぴゅーたー を かった	
\\	新[あたら]しい
\\	を 買[か]った。			
\\	何か	何[なに]か	なにか	
\\	道に何か落ちています。	道[みち]に 何[なに]か 落[お]ちています。	みち に なにか おちて います	
\\	道[みち]に
\\	落[お]ちています。			
\\	何	何[なん]	なん	
\\	質問は何ですか。	質問[しつもん]は 何[なん]ですか。	しつもん は なん です か	
\\	質問[しつもん]は
\\	ですか。			
\\	先ず	先[ま]ず	まず	
\\	帰ったら先ず手を洗いましょう。	帰[かえ]ったら 先[ま]ず 手[て]を 洗[あら]いましょう。	かえったら まず て を あらいましょう 。	
\\	帰[かえ]ったら
\\	手[て]を 洗[あら]いましょう。			
\\	先月	先月[せんげつ]	せんげつ	
\\	先月、友達の結婚式があった。	先月[せんげつ]、 友達[ともだち]の 結婚式[けっこんしき]があった。	せんげつ ともだち の けっこんしき が あった	
\\	、 友達[ともだち]の 結婚式[けっこんしき]があった。			
\\	やる	やる	やる	
\\	犬にえさをやった。	犬[いぬ]にえさをやった。	いぬ に えさ を やった	
\\	犬[いぬ]にえさを
\\	先週	先週[せんしゅう]	せんしゅう	
\\	先週は海に行った。	先週[せんしゅう]は 海[うみ]に 行[い]った。	せんしゅう は うみ に いった	
\\	は 海[うみ]に 行[い]った。			
\\	今まで	今[いま]まで	いままで	
\\	今までどこにいたのですか。	今[いま]までどこにいたのですか。	いままで どこ に いた の です か	
\\	どこにいたのですか。			
\\	来月	来月[らいげつ]	らいげつ	
\\	来月から大学生になります。	来月[らいげつ]から 大学生[だいがくせい]になります。	らいげつ から だいがくせい に なります	
\\	から 大学生[だいがくせい]になります。			
\\	来週	来週[らいしゅう]	らいしゅう	
\\	続きは来週やりましょう。	続[つづ]きは 来週[らいしゅう]やりましょう。	つづき は らいしゅう やりましょう	
\\	続[つづ]きは
\\	やりましょう。			
\\	ロボット	ロボット	ロボット	
\\	彼はロボットの研究をしています。	彼[かれ]はロボットの 研究[けんきゅう]をしています。	かれ は ろぼっと の けんきゅう を しています 。	
\\	彼[かれ]は
\\	の 研究[けんきゅう]をしています。			
\\	行う	行[おこな]う	おこなう	
\\	その会社は来月、キャンペーンを行う。	その 会社[かいしゃ]は 来月[らいげつ]、キャンペーンを 行[おこな]う。	その かいしゃ は らいげつ きゃんぺーん を おこなう	
\\	その 会社[かいしゃ]は 来月[らいげつ]、キャンペーンを
\\	行き	行[い]き	いき	
\\	行きは新幹線で行った。	行[い]きは 新幹線[しんかんせん]で 行[い]った。	いき は しんかんせん で いった	
\\	は 新幹線[しんかんせん]で 行[い]った。			
\\	行き	行[ゆ]き	ゆき	
\\	東京行きの列車に乗った。	東京[とうきょう] 行[ゆ]きの 列車[れっしゃ]に 乗[の]った。	とうきょうゆき の れっしゃ に のった	
\\	東京[とうきょう]
\\	の 列車[れっしゃ]に 乗[の]った。			
\\	帰り	帰[かえ]り	かえり	
\\	仕事の帰りにビールを飲んだ。	仕事[しごと]の 帰[かえ]りにビールを 飲[の]んだ。	しごと の かえり に びーる を のんだ	
\\	仕事[しごと]の
\\	にビールを 飲[の]んだ。			
\\	ほとんど	ほとんど	ほとんど	
\\	お金がほとんどありません。	お 金[かね]がほとんどありません。	おかね が ほとんど ありません	
\\	お 金[かね]が
\\	ありません。			
\\	大きさ	大[おお]きさ	おおきさ	
\\	この大きさの封筒が欲しいのですが。	この 大[おお]きさの 封筒[ふうとう]が 欲[ほ]しいのですが。	この おおきさ の ふうとう が ほしい の です が	
\\	この
\\	の 封筒[ふうとう]が 欲[ほ]しいのですが。			
\\	大分	大分[だいぶ]	だいぶ	
\\	大分ピアノが上手くなりました。	大分[だいぶ]ピアノが 上手[うま]くなりました。	だいぶ ぴあの が うまく なりました	
\\	ピアノが 上手[うま]くなりました。			
\\	中	中[なか]	なか	
\\	財布は引き出しの中にあります。	財布[さいふ]は 引[ひ]き 出[だ]しの 中[なか]にあります。	さいふ は ひきだし の なか に あります	
\\	財布[さいふ]は 引[ひ]き 出[だ]しの
\\	にあります。			
\\	少年	少年[しょうねん]	しょうねん	
\\	少年たちがサッカーをしている。	少年[しょうねん]たちがサッカーをしている。	しょうねんたち が さっかー を して いる	
\\	たちがサッカーをしている。			
\\	パソコン	パソコン	パソコン	
\\	彼はパソコンを2台持っています。	彼[かれ]はパソコンを 2台持[にだい も]っています。	かれ は ぱそこん を にだい もって います	
\\	彼[かれ]は
\\	を 2台持[にだい も]っています。			
\\	少しも	少[すこ]しも	すこしも	
\\	あなたは少しも悪くない。	あなたは 少[すこ]しも 悪[わる]くない。	あなた は すこしも わるく ない	
\\	あなたは
\\	悪[わる]くない。			
\\	少々	少々[しょうしょう]	しょうしょう	
\\	塩を少々入れてください。	塩[しお]を 少々[しょうしょう] 入[い]れてください。	しお を しょうしょう いれて ください	
\\	塩[しお]を
\\	入[い]れてください。			
\\	多く	多[おお]く	おおく	
\\	毎年多くの人が海外へ旅行する。	毎年[まいとし] 多[おお]くの 人[ひと]が 海外[かいがい]へ 旅行[りょこう]する。	まいとし おおく の ひと が かいがい へ りょこう する	
\\	毎年[まいとし]
\\	の 人[ひと]が 海外[かいがい]へ 旅行[りょこう]する。			
\\	上がる	上[あ]がる	あがる	
\\	私たちは2階に上がった。	私[わたし]たちは 2階[にかい]に 上[あ]がった。	わたしたち は にかい に あがった	
\\	私[わたし]たちは 2階[にかい]に
\\	もう	もう	もう	
\\	コーヒーをもう一杯ください。	コーヒーをもう 一杯[いっぱい]ください。	こーひー を もう いっぱい ください	
\\	コーヒーを
\\	一杯[いっぱい]ください。			
\\	上がる	上[あ]がる	あがる	
\\	彼は人前だと上がってしまう。	彼[かれ]は 人前[ひとまえ]だと 上[あ]がってしまう。	かれ は ひとまえ だ と あがって しまう	
\\	彼[かれ]は 人前[ひとまえ]だと
\\	年上	年上[としうえ]	としうえ	
\\	彼は私より年上です。	彼[かれ]は 私[わたし]より 年上[としうえ]です。	かれ は わたし より としうえ です	
\\	彼[かれ]は 私[わたし]より
\\	です。			
\\	上り	上[のぼ]り	のぼり	
\\	これは上り電車です。	これは 上[のぼ]り 電車[でんしゃ]です。	これ は のぼり でんしゃ です	
\\	これは
\\	電車[でんしゃ]です。			
\\	下げる	下[さ]げる	さげる	
\\	少し音量を下げてください。	少[すこ]し 音量[おんりょう]を 下[さ]げてください。	すこし おんりょう を さげて ください	
\\	少[すこ]し 音量[おんりょう]を
\\	ください。			
\\	プログラム	プログラム	プログラム	
\\	受付でプログラムを受け取った。	受付[うけつけ]でプログラムを 受[う]け 取[と]った。	うけつけ で ぷろぐらむ を うけとった	
\\	受付[うけつけ]で
\\	を 受[う]け 取[と]った。			
\\	下がる	下[さ]がる	さがる	
\\	やっと熱が下がった。	やっと 熱[ねつ]が 下[さ]がった。	やっと ねつ が さがった	
\\	やっと 熱[ねつ]が
\\	下りる	下[お]りる	おりる	
\\	そこの階段を下りてください。	そこの 階段[かいだん]を 下[お]りてください。	そこ の かいだん を おりて ください	
\\	そこの 階段[かいだん]を
\\	ください。			
\\	下ろす	下[お]ろす	おろす	
\\	棚からその箱を下ろしてください。	棚[たな]からその 箱[はこ]を 下[お]ろしてください。	たな から その はこ を おろして ください	
\\	棚[たな]からその 箱[はこ]を
\\	ください。			
\\	下り	下[くだ]り	くだり	
\\	もうすぐ下りの電車が発車します。	もうすぐ 下[くだ]りの 電車[でんしゃ]が 発車[はっしゃ]します。	もうすぐ くだり の でんしゃ が はっしゃ します	
\\	もうすぐ
\\	の 電車[でんしゃ]が 発車[はっしゃ]します。			
\\	よく	よく	よく	
\\	よく答えが分かりましたね。	よく 答[こた]えが 分[わ]かりましたね。	よく こたえ が わかりました ね	
\\	答[こた]えが 分[わ]かりましたね。			
\\	年下	年下[としした]	としした	
\\	彼は奥さんより年下です。	彼[かれ]は 奥[おく]さんより 年下[としした]です。	かれ は おくさん より としした です	
\\	彼[かれ]は 奥[おく]さんより
\\	です。			
\\	下る	下[くだ]る	くだる	
\\	小さな船が川を下っています。	小[ちい]さな 船[ふね]が 川[かわ]を 下[くだ]っています。	ちいさ な ふね が かわ を くだって います	
\\	小[ちい]さな 船[ふね]が 川[かわ]を
\\	一方	一方[いっぽう]	いっぽう	
\\	ここは一方通行です。	ここは 一方[いっぽう] 通行[つうこう]です。	ここ は いっぽう つうこう です	
\\	ここは
\\	通行[つうこう]です。			
\\	方	方[かた]	かた	
\\	次の方、どうぞ。	次[つぎ]の 方[かた]、どうぞ。	つぎ の かた どうぞ	
\\	次[つぎ]の
\\	、どうぞ。			
\\	まま	まま	まま	
\\	電気がついたままですよ。	電気[でんき]がついたままですよ。	でんき が ついた まま です よ	
\\	電気[でんき]がついた
\\	ですよ。			
\\	二人	二人[ふたり]	ふたり	
\\	今日は妻と二人で食事をします。	今日[きょう]は 妻[つま]と 二人[ふたり]で 食事[しょくじ]をします。	きょう は つま と ふたり で しょくじ を します	
\\	今日[きょう]は 妻[つま]と
\\	で 食事[しょくじ]をします。			
\\	大人しい	大人[おとな]しい	おとなしい	
\\	私の彼女はとても大人しいです。	私[わたし]の 彼女[かのじょ]はとても 大人[おとな]しいです。	わたし の かのじょ は とても おとなしい です	
\\	私[わたし]の 彼女[かのじょ]はとても
\\	です。			
\\	人々	人々[ひとびと]	ひとびと	
\\	あの村の人々はとても親切です。	あの 村[むら]の 人々[ひとびと]はとても 親切[しんせつ]です。	あの むら の ひとびと は とても しんせつ です	
\\	あの 村[むら]の
\\	はとても 親切[しんせつ]です。			
\\	一人で	一人[ひとり]で	ひとりで	
\\	今日は一人で映画を見ます。	今日[きょう]は 一人[ひとり]で 映画[えいが]を 見[み]ます。	きょう は ひとりで えいが を みます	
\\	今日[きょう]は
\\	映画[えいが]を 見[み]ます。			
\\	テレビ	テレビ	テレビ	
\\	私はテレビをあまり見ません。	私[わたし]はテレビをあまり 見[み]ません。	わたし は てれび を あまり みません	
\\	私[わたし]は
\\	をあまり 見[み]ません。			
\\	外人	外人[がいじん]	がいじん	
\\	この町には外人が少ない。	この 町[まち]には 外人[がいじん]が 少[すく]ない。	この まち に は がいじん が すくない	
\\	この 町[まち]には
\\	が 少[すく]ない。			
\\	外	外[そと]	そと	
\\	外は暑いよ。	外[そと]は 暑[あつ]いよ。	そと は あついよ	
\\	は 暑[あつ]いよ。			
\\	休日	休日[きゅうじつ]	きゅうじつ	
\\	休日は家でよくテレビを見ます。	休日[きゅうじつ]は 家[いえ]でよくテレビを 見[み]ます。	きゅうじつ は いえ で よく てれび を みます	
\\	は 家[いえ]でよくテレビを 見[み]ます。			
\\	休み	休[やす]み	やすみ	
\\	木曜日は仕事が休みです。	木曜日[もくようび]は 仕事[しごと]が 休[やす]みです。	もくようび は しごと が やすみ です	
\\	木曜日[もくようび]は 仕事[しごと]が
\\	です。			
\\	ソフト	ソフト	ソフト	
\\	このソフトで日本語を勉強することができます。	このソフトで 日本語[にほんご]を 勉強[べんきょう]することができます。	この そふと で にほんご を べんきょう する こと が できます	
\\	この
\\	で 日本語[にほんご]を 勉強[べんきょう]することができます。			
\\	力	力[ちから]	ちから	
\\	お相撲さんは、みんな力持ちだ。	お 相撲[すもう]さんは、みんな 力[ちから] 持[も]ちだ。	おすもうさん は みんな ちからもち だ	
\\	お 相撲[すもう]さんは、みんな
\\	持[も]ちだ。			
\\	協力	協力[きょうりょく]	きょうりょく	
\\	このプロジェクトにはみんなの協力が必要です。	このプロジェクトにはみんなの 協力[きょうりょく]が 必要[ひつよう]です。	この ぷろじぇくと に は みんな の きょうりょく が ひつよう です	
\\	このプロジェクトにはみんなの
\\	が 必要[ひつよう]です。			
\\	人口	人口[じんこう]	じんこう	
\\	その国の人口はどのくらいですか。	その 国[くに]の 人口[じんこう]はどのくらいですか。	その くに の じんこう は どの くらい です か	
\\	その 国[くに]の
\\	はどのくらいですか。			
\\	出口	出口[でぐち]	でぐち	
\\	出口はあそこです。	出口[でぐち]はあそこです。	でぐち は あそこ です	
\\	はあそこです。			
\\	ただ	ただ	ただ	
\\	この温泉はただです。	この 温泉[おんせん]はただです。	この おんせん は ただ です	
\\	この 温泉[おんせん]は
\\	です。			
\\	入り口	入[い]り 口[ぐち]	いりぐち	
\\	入り口は向こうです。	入[い]り 口[ぐち]は 向[む]こうです。	いりぐち は むこう です	
\\	は 向[む]こうです。			
\\	右手	右手[みぎて]	みぎて	
\\	私は右手で字を書きます。	私[わたし]は 右手[みぎて]で 字[じ]を 書[か]きます。	わたし は みぎて で じ を かきます	
\\	私[わたし]は
\\	で 字[じ]を 書[か]きます。			
\\	左手	左手[ひだりて]	ひだりて	
\\	彼女は左手で字を書く。	彼女[かのじょ]は 左手[ひだりて]で 字[じ]を 書[か]く。	かのじょ は ひだりて で じ を かく	
\\	彼女[かのじょ]は
\\	で 字[じ]を 書[か]く。			
\\	下手	下手[へた]	へた	
\\	私は歌が下手だ。	私[わたし]は 歌[うた]が 下手[へた]だ。	わたし は うた が へた だ	
\\	私[わたし]は 歌[うた]が
\\	だ。			
\\	これら	これら	これら	
\\	今日はこれらの問題について話し合います。	今日[きょう]はこれらの 問題[もんだい]について 話[はな]し 合[あ]います。	きょう は これら の もんだい に ついて はなしあいます	
\\	今日[きょう]は
\\	の 問題[もんだい]について 話[はな]し 合[あ]います。			
\\	足りる	足[た]りる	たりる	
\\	お金が足りなくて買えなかった。	お 金[かね]が 足[た]りなくて 買[か]えなかった。	おかね が たりなくて かえなかった	
\\	お 金[かね]が
\\	買[か]えなかった。			
\\	足す	足[た]す	たす	
\\	母は味噌汁に水を足した。	母[はは]は 味噌汁[みそしる]に 水[みず]を 足[た]した。	はは は みそしる に みず を たした	
\\	母[はは]は 味噌汁[みそしる]に 水[みず]を
\\	山	山[やま]	やま	
\\	山の空気はきれいだ。	山[やま]の 空気[くうき]はきれいだ。	やま の くうき は きれい だ	
\\	の 空気[くうき]はきれいだ。			
\\	川	川[かわ]	かわ	
\\	小さな川を渡りました。	小[ちい]さな 川[かわ]を 渡[わた]りました。	ちいさ な かわ を わたりました	
\\	小[ちい]さな
\\	を 渡[わた]りました。			
\\	いずれ	いずれ	いずれ	
\\	いずれまたお会いしましょう。	いずれまたお 会[あ]いしましょう。	いずれ また おあい しましょう	
\\	またお 会[あ]いしましょう。			
\\	空く	空[あ]く	あく	
\\	後ろの席が空いています。	後[うし]ろの 席[せき]が 空[あ]いています。	うしろ の せき が あいて います	
\\	後[うし]ろの 席[せき]が
\\	空手	空手[からて]	からて	
\\	彼は空手を習っています。	彼[かれ]は 空手[からて]を 習[なら]っています。	かれ は からて を ならって います	
\\	彼[かれ]は
\\	を 習[なら]っています。			
\\	空	空[そら]	そら	
\\	空が真っ青です。	空[そら]が 真[ま]っ 青[さお]です。	そら が まっさお です	
\\	が 真[ま]っ 青[さお]です。			
\\	海外	海外[かいがい]	かいがい	
\\	彼は海外での生活が長いです。	彼[かれ]は 海外[かいがい]での 生活[せいかつ]が 長[なが]いです。	かれ は かいがい で の せいかつ が ながい です	
\\	彼[かれ]は
\\	での 生活[せいかつ]が 長[なが]いです。			
\\	あまり	あまり	あまり	
\\	このビールはあまり美味しくありません。	このビールはあまり 美味[おい]しくありません。	この びーる は あまり おいしく ありません	
\\	このビールは
\\	美味[おい]しくありません。			
\\	海	海[うみ]	うみ	
\\	海は広くて大きい。	海[うみ]は 広[ひろ]くて 大[おお]きい。	うみ は ひろく て おおきい	
\\	は 広[ひろ]くて 大[おお]きい。			
\\	毎日	毎日[まいにち]	まいにち	
\\	私たちは毎日散歩をします。	私[わたし]たちは 毎日[まいにち] 散歩[さんぽ]をします。	わたしたち は まいにち さんぽ を します	
\\	私[わたし]たちは
\\	散歩[さんぽ]をします。			
\\	毎年	毎年[まいとし]	まいとし	
\\	私は毎年、海外旅行に行きます。	私[わたし]は 毎年[まいとし]、 海外旅行[かいがい りょこう]に 行[い]きます。	わたし は まいとし かいがい りょこう に いきます	
\\	私[わたし]は
\\	、 海外旅行[かいがい りょこう]に 行[い]きます。			
\\	毎年	毎年[まいねん]	まいねん	
\\	毎年給料が上がる。	毎年[まいねん] 給料[きゅうりょう]が 上[あ]がる。	まいねん きゅうりょう が あがる	
\\	給料[きゅうりょう]が 上[あ]がる。			
\\	なお	なお	なお	
\\	なお、雨の場合は中止です。	なお、 雨[あめ]の 場合[ばあい]は 中止[ちゅうし]です。	なお あめ の ばあい は ちゅうし です	
\\	、 雨[あめ]の 場合[ばあい]は 中止[ちゅうし]です。			
\\	毎週	毎週[まいしゅう]	まいしゅう	
\\	私は毎週母に電話をします。	私[わたし]は 毎週[まいしゅう] 母[はは]に 電話[でんわ]をします。	わたし は まいしゅう はは に でんわ を します	
\\	私[わたし]は
\\	母[はは]に 電話[でんわ]をします。			
\\	毎月	毎月[まいつき]	まいつき	
\\	私は毎月貯金をしています。	私[わたし]は 毎月[まいつき] 貯金[ちょきん]をしています。	わたし は まいつき ちょきん を して います	
\\	私[わたし]は
\\	貯金[ちょきん]をしています。			
\\	石	石[いし]	いし	
\\	私は石につまづいた。	私[わたし]は 石[いし]につまづいた。	わたし は いし に つまづいた	
\\	私[わたし]は
\\	につまづいた。			
\\	田んぼ	田[た]んぼ	たんぼ	
\\	この辺は田んぼがたくさんあります。	この 辺[へん]は 田[た]んぼがたくさんあります。	このへん は たんぼ が たくさん あります	
\\	この 辺[へん]は
\\	がたくさんあります。			
\\	ほぼ	ほぼ	ほぼ	
\\	仕事がほぼ終わりました。	仕事[しごと]がほぼ 終[お]わりました。	しごと が ほぼ おわりました	
\\	仕事[しごと]が
\\	終[お]わりました。			
\\	花	花[はな]	はな	
\\	きれいな花が咲きました。	きれいな 花[はな]が 咲[さ]きました。	きれい な はな が さきました	
\\	きれいな
\\	が 咲[さ]きました。			
\\	林	林[はやし]	はやし	
\\	私たちは林の中に入っていった。	私[わたし]たちは 林[はやし]の 中[なか]に 入[はい]っていった。	わたしたち は はやし の なか に はいって いった	
\\	私[わたし]たちは
\\	の 中[なか]に 入[はい]っていった。			
\\	森	森[もり]	もり	
\\	私は森を歩くのが好きです。	私[わたし]は 森[もり]を 歩[ある]くのが 好[す]きです。	わたし は もり を あるく の が すき です	
\\	私[わたし]は
\\	を 歩[ある]くのが 好[す]きです。			
\\	子	子[こ]	こ	
\\	その子は日本語が分からない。	その 子[こ]は 日本語[にほんご]が 分[わ]からない。	その こ は にほんご が わからない	
\\	その
\\	は 日本語[にほんご]が 分[わ]からない。			
\\	サービス	サービス	サービス	
\\	この店はサービスがいい。	この 店[みせ]はサービスがいい。	この みせ は さーびす が いい 。	
\\	この 店[みせ]は
\\	がいい。			
\\	女の子	女[おんな]の 子[こ]	おんなのこ	
\\	あの女の子を知っていますか。	あの 女[おんな]の 子[こ]を 知[し]っていますか。	あの おんなのこ を しって います か	
\\	あの
\\	を 知[し]っていますか。			
\\	男の子	男[おとこ]の 子[こ]	おとこのこ	
\\	男の子たちがサッカーをしている。	男[おとこ]の 子[こ]たちがサッカーをしている。	おとこのこたち が さっかー を して いる	
\\	たちがサッカーをしている。			
\\	私たち	私[わたし]たち	わたしたち	
\\	私たちは来月結婚します。	私[わたし]たちは 来月結婚[らいげつ けっこん]します。	わたしたち は らいげつ けっこん します 。	
\\	は 来月結婚[らいげつ けっこん]します。			
\\	達する	達[たっ]する	たっする	
\\	気温は35度に達した。	気温[きおん]は 35度[さんじゅうごど]に 達[たっ]した。	きおん は さんじゅうごど に たっした	
\\	気温[きおん]は 35度[さんじゅうごど]に
\\	グループ	グループ	グループ	
\\	店に学生のグループが来た。	店[みせ]に 学生[がくせい]のグループが 来[き]た。	みせ に がくせい の ぐるーぷ が きた	
\\	店[みせ]に 学生[がくせい]の
\\	が 来[き]た。			
\\	家	家[いえ]	いえ	
\\	ここが私の家です。	ここが 私[わたし]の 家[いえ]です。	ここ が わたし の いえ です	
\\	ここが 私[わたし]の
\\	です。			
\\	家内	家内[かない]	かない	
\\	家内は九州出身です。	家内[かない]は 九州出身[きゅうしゅう しゅっしん]です。	かない は きゅうしゅう しゅっしん です	
\\	は 九州出身[きゅうしゅう しゅっしん]です。			
\\	客	客[きゃく]	きゃく	
\\	その店は若い客が多いです。	その 店[みせ]は 若[わか]い 客[きゃく]が 多[おお]いです。	その みせ は わかい きゃく が おおい です	
\\	その 店[みせ]は 若[わか]い
\\	が 多[おお]いです。			
\\	空気	空気[くうき]	くうき	
\\	ここは空気がきれいです。	ここは 空気[くうき]がきれいです。	ここ は くうき が きれい です	
\\	ここは
\\	がきれいです。			
\\	ホテル	ホテル	ホテル	
\\	今ホテルに着きました。	今[いま]ホテルに 着[つ]きました。	いま ほてる に つきました	
\\	今[いま]
\\	に 着[つ]きました。			
\\	気に入る	気[き]に 入[い]る	きにいる	
\\	新しい靴がとても気に入りました。	新[あたら]しい 靴[くつ]がとても 気[き]に 入[い]りました。	あたらしい くつ が とても きにいりました	
\\	新[あたら]しい 靴[くつ]がとても
\\	人気	人気[にんき]	にんき	
\\	このバンドはとても人気があるよ。	このバンドはとても 人気[にんき]があるよ。	この ばんど は とても にんき が ある よ	
\\	このバンドはとても
\\	があるよ。			
\\	雨	雨[あめ]	あめ	
\\	雨が降っています。	雨[あめ]が 降[ふ]っています。	あめ が ふって います	
\\	が 降[ふ]っています。			
\\	雪	雪[ゆき]	ゆき	
\\	クリスマスに雪が降りました。	クリスマスに 雪[ゆき]が 降[ふ]りました。	くりすます に ゆき が ふりました	
\\	クリスマスに
\\	が 降[ふ]りました。			
\\	まとめる	まとめる	まとめる	
\\	彼女は荷物をまとめて出て行った。	彼女[かのじょ]は 荷物[にもつ]をまとめて 出[で]て 行[い]った。	かのじょ は にもつ を まとめて でて いった	
\\	彼女[かのじょ]は 荷物[にもつ]を
\\	出[で]て 行[い]った。			
\\	青い	青[あお]い	あおい	
\\	ここの海はとても青い。	ここの 海[うみ]はとても 青[あお]い。	ここ の うみ は とても あおい	
\\	ここの 海[うみ]はとても
\\	青	青[あお]	あお	
\\	私の好きな色は青です。	私[わたし]の 好[す]きな 色[いろ]は 青[あお]です。	わたし の すき な いろ は あお です	
\\	私[わたし]の 好[す]きな 色[いろ]は
\\	です。			
\\	晴れ	晴[は]れ	はれ	
\\	明日の天気は晴れです。	明日[あす]の 天気[てんき]は 晴[は]れです。	あす の てんき は はれ です	
\\	明日[あす]の 天気[てんき]は
\\	です。			
\\	明らか	明[あき]らか	あきらか	
\\	明らかに彼が悪い。	明[あき]らかに 彼[かれ]が 悪[わる]い。	あきらか に かれ が わるい	
\\	に 彼[かれ]が 悪[わる]い。			
\\	やはり	やはり	やはり	
\\	彼はやはり遅刻しました。	彼[かれ]はやはり 遅刻[ちこく]しました。	かれ は やはり ちこく しました	
\\	彼[かれ]は
\\	遅刻[ちこく]しました。			
\\	明るい	明[あか]るい	あかるい	
\\	彼女は明るい性格です。	彼女[かのじょ]は 明[あか]るい 性格[せいかく]です。	かのじょ は あかるい せいかく です	
\\	彼女[かのじょ]は
\\	性格[せいかく]です。			
\\	明日	明日[あした]	あした	
\\	明日、会社を休みます。	明日[あした]、 会社[かいしゃ]を 休[やす]みます。	あした かいしゃ を やすみます	
\\	、 会社[かいしゃ]を 休[やす]みます。			
\\	暗い	暗[くら]い	くらい	
\\	東の空が暗いです。	東[ひがし]の 空[そら]が 暗[くら]いです。	ひがし の そら が くらい です	
\\	東[ひがし]の 空[そら]が
\\	です。			
\\	昨年	昨年[さくねん]	さくねん	
\\	昨年は地震が多い年でした。	昨年[さくねん]は 地震[じしん]が 多[おお]い 年[とし]でした。	さくねん は じしん が おおい とし でした	
\\	は 地震[じしん]が 多[おお]い 年[とし]でした。			
\\	はっきり	はっきり	はっきり	
\\	今日は山がはっきり見える。	今日[きょう]は 山[やま]がはっきり 見[み]える。	きょう は やま が はっきり みえる	
\\	今日[きょう]は 山[やま]が
\\	見[み]える。			
\\	一昨年	一昨年[おととし]	おととし	
\\	一昨年初めて京都へ旅行しました。	一昨年[おととし] 初[はじ]めて 京都[きょうと]へ 旅行[りょこう]しました。	おととし はじめて きょうと へ りょこう しました	
\\	初[はじ]めて 京都[きょうと]へ 旅行[りょこう]しました。			
\\	一昨日	一昨日[おととい]	おととい	
\\	一昨日彼から電話がありました。	一昨日[おととい] 彼[かれ]から 電話[でんわ]がありました。	おととい かれ から でんわ が ありました	
\\	彼[かれ]から 電話[でんわ]がありました。			
\\	東	東[ひがし]	ひがし	
\\	東の空が暗いです。	東[ひがし]の 空[そら]が 暗[くら]いです。	ひがし の そら が くらい です	
\\	の 空[そら]が 暗[くら]いです。			
\\	西	西[にし]	にし	
\\	太陽は西に沈みます。	太陽[たいよう]は 西[にし]に 沈[しず]みます。	たいよう は にし に しずみます	
\\	太陽[たいよう]は
\\	に 沈[しず]みます。			
\\	つまり	つまり	つまり	
\\	つまり、あなたは何も知らないのですね。	つまり、あなたは 何[なに]も 知[し]らないのですね。	つまり あなた は なに も しらない の です ね	
\\	、あなたは 何[なに]も 知[し]らないのですね。			
\\	南	南[みなみ]	みなみ	
\\	私の家は町の南にあります。	私[わたし]の 家[いえ]は 町[まち]の 南[みなみ]にあります。	わたし の いえ は まち の みなみ に あります	
\\	私[わたし]の 家[いえ]は 町[まち]の
\\	にあります。			
\\	北	北[きた]	きた	
\\	ロシアは日本の北にあります。	ロシアは 日本[にっぽん]の 北[きた]にあります。	ろしあ は にっぽん の きた に あります	
\\	ロシアは 日本[にっぽん]の
\\	にあります。			
\\	方向	方向[ほうこう]	ほうこう	
\\	あの人たちは皆、同じ方向を見ている。	あの 人[ひと]たちは 皆[みな]、 同[おな]じ 方向[ほうこう]を 見[み]ている。	あの ひとたち は みな おなじ ほうこう を みて いる	
\\	あの 人[ひと]たちは 皆[みな]、 同[おな]じ
\\	を 見[み]ている。			
\\	向かう	向[む]かう	むかう	
\\	今、会社に向かっています。	今[いま]、 会社[かいしゃ]に 向[む]かっています。	いま かいしゃ に むかって います	
\\	今[いま]、 会社[かいしゃ]に
\\	ビル	ビル	ビル	
\\	私の会社はあのビルの8階です。	私[わたし]の 会社[かいしゃ]はあのビルの 8階[はちかい]です。	わたし の かいしゃ は あの びる の はちかい です	
\\	私[わたし]の 会社[かいしゃ]はあの
\\	の 8階[はちかい]です。			
\\	向こう	向[む]こう	むこう	
\\	友達は向こうにいます。	友達[ともだち]は 向[む]こうにいます。	ともだち は むこう に います	
\\	友達[ともだち]は
\\	にいます。			
\\	向く	向[む]く	むく	
\\	こっちを向いてください。	こっちを 向[む]いてください。	こっち を むいて ください	
\\	こっちを
\\	ください。			
\\	開く	開[あ]く	あく	
\\	電車のドアが開きました。	電車[でんしゃ]のドアが 開[あ]きました。	でんしゃ の どあ が あきました	
\\	電車[でんしゃ]のドアが
\\	聞こえる	聞[き]こえる	きこえる	
\\	隣の部屋からテレビの音が聞こえる。	隣[となり]の 部屋[へや]からテレビの 音[おと]が 聞[き]こえる。	となり の へや から てれび の おと が きこえる	
\\	隣[となり]の 部屋[へや]からテレビの 音[おと]が
\\	もちろん	もちろん	もちろん	
\\	もちろん一緒に行きます。	もちろん 一緒[いっしょ]に 行[い]きます。	もちろん いっしょ に いきます	
\\	一緒[いっしょ]に 行[い]きます。			
\\	年間	年間[ねんかん]	ねんかん	
\\	年間5万人がここを訪れます。	年間[ねんかん] 5万人[ごまんにん]がここを 訪[おとず]れます。	ねんかん ごまんにん が ここ を おとずれます	
\\	5万人[ごまんにん]がここを 訪[おとず]れます。			
\\	この間	この 間[あいだ]	このあいだ	
\\	この間彼女に会った。	この 間[あいだ] 彼女[かのじょ]に 会[あ]った。	このあいだ かのじょ に あった	
\\	彼女[かのじょ]に 会[あ]った。			
\\	間	間[あいだ]	あいだ	
\\	雲の間から月が出た。	雲[くも]の 間[あいだ]から 月[つき]が 出[で]た。	くも の あいだ から つき が でた	
\\	雲[くも]の
\\	から 月[つき]が 出[で]た。			
\\	人間	人間[にんげん]	にんげん	
\\	人間の心は複雑です。	人間[にんげん]の 心[こころ]は 複雑[ふくざつ]です。	にんげん の こころ は ふくざつ です	
\\	の 心[こころ]は 複雑[ふくざつ]です。			
\\	かつて	かつて	かつて	
\\	かつて私が学生だった頃のことです。	かつて 私[わたし]が 学生[がくせい]だった 頃[ころ]のことです。	かつて わたし が がくせい だった ころ の こと です	
\\	私[わたし]が 学生[がくせい]だった 頃[ころ]のことです。			
\\	高さ	高[たか]さ	たかさ	
\\	富士山の高さは3,776メートルです。	富士山[ふじさん]の 高[たか]さは 
\\	776[さんぜんななひゃく-ななじゅうろく]メートルです。	ふじさん の たかさ は さんぜんななひゃく-ななじゅうろくめーとる です	
\\	富士山[ふじさん]の
\\	は 
\\	776[さんぜんななひゃく-ななじゅうろく]メートルです。			
\\	最大	最大[さいだい]	さいだい	
\\	これは世界最大の船です。	これは 世界[せかい] 最大[さいだい]の 船[ふね]です。	これ は せかい さいだい の ふね です	
\\	これは 世界[せかい]
\\	の 船[ふね]です。			
\\	初めて	初[はじ]めて	はじめて	
\\	東京に来るのは初めてです。	東京[とうきょう]に 来[く]るのは 初[はじ]めてです。	とうきょう に くる の は はじめて です 。	
\\	東京[とうきょう]に 来[く]るのは
\\	です。			
\\	最初	最初[さいしょ]	さいしょ	
\\	5ページの最初を見てください。	5[ご]ページの 最初[さいしょ]を 見[み]てください。	ごぺーじ の さいしょ を みて ください	
\\	5[ご]ページの
\\	を 見[み]てください。			
\\	スポーツ	スポーツ	スポーツ	
\\	あなたは何かスポーツをしていますか。	あなたは 何[なに]かスポーツをしていますか。	あなた は なにか すぽーつ を して います か	
\\	あなたは 何[なに]か
\\	をしていますか。			
\\	初め	初[はじ]め	はじめ	
\\	初めは上手くできませんでした。	初[はじ]めは 上手[うま]くできませんでした。	はじめ は うまく できません でした 。	
\\	は 上手[うま]くできませんでした。			
\\	今後	今後[こんご]	こんご	
\\	今後ともよろしくお願いします。	今後[こんご]ともよろしくお 願[ねが]いします。	こんご とも よろしく おねがい します	
\\	ともよろしくお 願[ねが]いします。			
\\	後	後[のち]	のち	
\\	後に彼は総理大臣になりました。	後[のち]に 彼[かれ]は 総理大臣[そうり だいじん]になりました。	のち に かれ は そうり だいじん に なりました	
\\	に 彼[かれ]は 総理大臣[そうり だいじん]になりました。			
\\	最後	最後[さいご]	さいご	
\\	今日が夏休み最後の日だ。	今日[きょう]が 夏休[なつやす]み 最後[さいご]の 日[ひ]だ。	きょう が なつやすみ さいご の ひ だ	
\\	今日[きょう]が 夏休[なつやす]み
\\	の 日[ひ]だ。			
\\	なぜ	なぜ	なぜ	
\\	なぜ来なかったの。	なぜ 来[こ]なかったの。	なぜ こなかった の	
\\	来[こ]なかったの。			
\\	明後日	明後日[あさって]	あさって	
\\	明後日は休日です。	明後日[あさって]は 休日[きゅうじつ]です。	あさって は きゅうじつ です 。	
\\	は 休日[きゅうじつ]です。			
\\	牛	牛[うし]	うし	
\\	牛が草を食べています。	牛[うし]が 草[くさ]を 食[た]べています。	うし が くさ を たべて います	
\\	が 草[くさ]を 食[た]べています。			
\\	半分	半分[はんぶん]	はんぶん	
\\	お菓子を友達に半分あげた。	お 菓子[かし]を 友達[ともだち]に 半分[はんぶん]あげた。	おかし を ともだち に はんぶん あげた	
\\	お 菓子[かし]を 友達[ともだち]に
\\	あげた。			
\\	半年	半年[はんとし]	はんとし	
\\	日本に来て半年になります。	日本[にっぽん]に 来[き]て 半年[はんとし]になります。	にっぽん に きて はんとし に なります	
\\	日本[にっぽん]に 来[き]て
\\	になります。			
\\	そのまま	そのまま	そのまま	
\\	そのままお待ちください。	そのままお 待[ま]ちください。	そのまま おまち ください	
\\	お 待[ま]ちください。			
\\	半月	半月[はんつき]	はんつき	
\\	半月前に日本に来ました。	半月[はんつき] 前[まえ]に 日本[にほん]に 来[き]ました。	はんつきまえ に にほん に きました	
\\	前[まえ]に 日本[にほん]に 来[き]ました。			
\\	半日	半日[はんにち]	はんにち	
\\	今日は半日だけ仕事だ。	今日[きょう]は 半日[はんにち]だけ 仕事[しごと]だ。	きょう は はんにち だけ しごと だ	
\\	今日[きょう]は
\\	だけ 仕事[しごと]だ。			
\\	毎朝	毎朝[まいあさ]	まいあさ	
\\	私は毎朝ジョギングをします。	私[わたし]は 毎朝[まいあさ]ジョギングをします。	わたし は まいあさ じょぎんぐ を します	
\\	私[わたし]は
\\	ジョギングをします。			
\\	今朝	今朝[けさ]	けさ	
\\	今朝から頭が痛い。	今朝[けさ]から 頭[あたま]が 痛[いた]い。	けさ から あたま が いたい	
\\	から 頭[あたま]が 痛[いた]い。			
\\	もし	もし	もし	
\\	もし雨が降ったら、行きません。	もし 雨[あめ]が 降[ふ]ったら、 行[い]きません。	もし あめ が ふったら いきません	
\\	雨[あめ]が 降[ふ]ったら、 行[い]きません。			
\\	昼休み	昼休[ひるやす]み	ひるやすみ	
\\	昼休みに公園に行った。	昼休[ひるやす]みに 公園[こうえん]に 行[い]った。	ひるやすみ に こうえん に いった	
\\	に 公園[こうえん]に 行[い]った。			
\\	昼前	昼前[ひるまえ]	ひるまえ	
\\	昼前に会議があった。	昼前[ひるまえ]に 会議[かいぎ]があった。	ひるまえ に かいぎ が あった	
\\	に 会議[かいぎ]があった。			
\\	昼間	昼間[ひるま]	ひるま	
\\	昼間は仕事で忙しいです。	昼間[ひるま]は 仕事[しごと]で 忙[いそが]しいです。	ひるま は しごと で いそがしい です	
\\	は 仕事[しごと]で 忙[いそが]しいです。			
\\	毎晩	毎晩[まいばん]	まいばん	
\\	姉は毎晩日記を書いています。	姉[あね]は 毎晩[まいばん] 日記[にっき]を 書[か]いています。	あね は まいばん にっき を かいて います	
\\	姉[あね]は
\\	日記[にっき]を 書[か]いています。			
\\	つもり	つもり	つもり	
\\	明日からタバコを止めるつもりです。	明日[あした]からタバコを 止[や]めるつもりです。	あした から たばこ を やめる つもり です	
\\	明日[あした]からタバコを 止[や]める
\\	です。			
\\	今夜	今夜[こんや]	こんや	
\\	今夜は月がとてもきれいです。	今夜[こんや]は 月[つき]がとてもきれいです。	こんや は つき が とても きれい です	
\\	は 月[つき]がとてもきれいです。			
\\	昨夜	昨夜[ゆうべ]	ゆうべ	
\\	昨夜、流れ星を見ました。	昨夜[ゆうべ]、 流[なが]れ 星[ぼし]を 見[み]ました。	ゆうべ ながれぼし を みました	
\\	、 流[なが]れ 星[ぼし]を 見[み]ました。			
\\	夜中	夜中[よなか]	よなか	
\\	夜中に電話がありました。	夜中[よなか]に 電話[でんわ]がありました。	よなか に でんわ が ありました	
\\	に 電話[でんわ]がありました。			
\\	夕方	夕方[ゆうがた]	ゆうがた	
\\	夕方そちらに着きます。	夕方[ゆうがた]そちらに 着[つ]きます。	ゆうがた そちら に つきます	
\\	そちらに 着[つ]きます。			
\\	やっと	やっと	やっと	
\\	やっと仕事が終わりました。	やっと 仕事[しごと]が 終[お]わりました。	やっと しごと が おわりました	
\\	仕事[しごと]が 終[お]わりました。			
\\	昼食	昼食[ちゅうしょく]	ちゅうしょく	
\\	昼食に寿司を食べました。	昼食[ちゅうしょく]に 寿司[すし]を 食[た]べました。	ちゅうしょく に すし を たべました	
\\	に 寿司[すし]を 食[た]べました。			
\\	朝食	朝食[ちょうしょく]	ちょうしょく	
\\	朝食に納豆を食べました。	朝食[ちょうしょく]に 納豆[なっとう]を 食[た]べました。	ちょうしょく に なっとう を たべました	
\\	に 納豆[なっとう]を 食[た]べました。			
\\	夕食	夕食[ゆうしょく]	ゆうしょく	
\\	夕食は7時です。	夕食[ゆうしょく]は 7時[しちじ]です。	ゆうしょく は しちじ です	
\\	は 7時[しちじ]です。			
\\	夕飯	夕飯[ゆうはん]	ゆうはん	
\\	夕飯は寿司でした。	夕飯[ゆうはん]は 寿司[すし]でした。	ゆうはん は すし でした	
\\	は 寿司[すし]でした。			
\\	ニュース	ニュース	ニュース	
\\	夜のニュースを見ましたか。	夜[よる]のニュースを 見[み]ましたか。	よる の にゅーす を みました か	
\\	夜[よる]の
\\	を 見[み]ましたか。			
\\	見方	見方[みかた]	みかた	
\\	彼に対する見方が変わりました。	彼[かれ]に 対[たい]する 見方[みかた]が 変[か]わりました。	かれ に たいする みかた が かわりました	
\\	彼[かれ]に 対[たい]する
\\	が 変[か]わりました。			
\\	花見	花見[はなみ]	はなみ	
\\	友達と花見をしました。	友達[ともだち]と 花見[はなみ]をしました。	ともだち と はなみ を しました	
\\	友達[ともだち]と
\\	をしました。			
\\	言い方	言[い]い 方[かた]	いいかた	
\\	そんな言い方をしてはいけません。	そんな 言[い]い 方[かた]をしてはいけません。	そんな いいかた を して は いけません	
\\	そんな
\\	をしてはいけません。			
\\	話	話[はなし]	はなし	
\\	あなたの話は面白いね。	あなたの 話[はなし]は 面白[おもしろ]いね。	あなた の はなし は おもしろい ね	
\\	あなたの
\\	は 面白[おもしろ]いね。			
\\	ずっと	ずっと	ずっと	
\\	父は休みの日はずっとテレビを見ている。	父[ちち]は 休[やす]みの 日[ひ]はずっとテレビを 見[み]ている。	ちち は やすみ の ひ は ずっと てれび を みて いる	
\\	父[ちち]は 休[やす]みの 日[ひ]は
\\	テレビを 見[み]ている。			
\\	読み	読[よ]み	よみ	
\\	母に読み書きを習いました。	母[はは]に 読[よ]み 書[か]きを 習[なら]いました。	はは に よみかき を ならいました	
\\	母[はは]に
\\	書[か]きを 習[なら]いました。			
\\	読み方	読[よ]み 方[かた]	よみかた	
\\	この漢字の読み方を教えてください。	この 漢字[かんじ]の 読[よ]み 方[かた]を 教[おし]えてください。	この かんじ の よみかた を おしえて ください	
\\	この 漢字[かんじ]の
\\	を 教[おし]えてください。			
\\	言語	言語[げんご]	げんご	
\\	彼はアジアの言語を研究している。	彼[かれ]はアジアの 言語[げんご]を 研究[けんきゅう]している。	かれ は あじあ の げんご を けんきゅう して いる	
\\	彼[かれ]はアジアの
\\	を 研究[けんきゅう]している。			
\\	ビデオ	ビデオ	ビデオ	
\\	私はその番組をビデオに撮った。	私[わたし]はその 番組[ばんぐみ]をビデオに 撮[と]った。	わたし は その ばんぐみ を びでお に とった	
\\	私[わたし]はその 番組[ばんぐみ]を
\\	に 撮[と]った。			
\\	英語	英語[えいご]	えいご	
\\	あなたは英語が話せますか。	あなたは 英語[えいご]が 話[はな]せますか。	あなた は えいご が はなせます か	
\\	あなたは
\\	が 話[はな]せますか。			
\\	文字	文字[もじ]	もじ	
\\	壁に文字が書いてあった。	壁[かべ]に 文字[もじ]が 書[か]いてあった。	かべ に もじ が かいて あった	
\\	壁[かべ]に
\\	が 書[か]いてあった。			
\\	ローマ字	ローマ 字[じ]	ろーまじ	
\\	ローマ字で名前を書いてください。	ローマ 字[じ]で 名前[なまえ]を 書[か]いてください。	ろーまじ で なまえ を かいて ください	
\\	で 名前[なまえ]を 書[か]いてください。			
\\	字	字[じ]	じ	
\\	もっと大きく字を書いてください。	もっと 大[おお]きく 字[じ]を 書[か]いてください。	もっと おおきく じ を かいて ください	
\\	もっと 大[おお]きく
\\	を 書[か]いてください。			
\\	マンション	マンション	マンション	
\\	彼はマンションに住んでいます。	彼[かれ]はマンションに 住[す]んでいます。	かれ は まんしょん に すんで います	
\\	彼[かれ]は
\\	に 住[す]んでいます。			
\\	書き方	書[か]き 方[かた]	かきかた	
\\	彼はその漢字の書き方が分からない。	彼[かれ]はその 漢字[かんじ]の 書[か]き 方[かた]が 分[わ]からない。	かれ は その かんじ の かきかた が わからない	
\\	彼[かれ]はその 漢字[かんじ]の
\\	が 分[わ]からない。			
\\	覚める	覚[さ]める	さめる	
\\	今朝は6時に目が覚めた。	今朝[けさ]は 6時[ろくじ]に 目[め]が 覚[さ]めた。	けさ は ろくじ に め が さめた	
\\	今朝[けさ]は 6時[ろくじ]に 目[め]が
\\	覚ます	覚[さ]ます	さます	
\\	子供が目を覚ました。	子供[こども]が 目[め]を 覚[さ]ました。	こども が め を さました	
\\	子供[こども]が 目[め]を
\\	大会	大会[たいかい]	たいかい	
\\	夏には川辺で花火大会があります。	夏[なつ]には 川辺[かわべ]で 花火[はなび] 大会[たいかい]があります。	なつ に は かわべ で はなびたいかい が あります	
\\	夏[なつ]には 川辺[かわべ]で 花火[はなび]
\\	があります。			
\\	しばらく	しばらく	しばらく	
\\	そこでしばらく休んでいます。	そこでしばらく 休[やす]んでいます。	そこで しばらく やすんで います	
\\	そこで
\\	休[やす]んでいます。			
\\	会話	会話[かいわ]	かいわ	
\\	親子の会話は大切です。	親子[おやこ]の 会話[かいわ]は 大切[たいせつ]です。	おやこ の かいわ は たいせつ です	
\\	親子[おやこ]の
\\	は 大切[たいせつ]です。			
\\	話し合う	話[はな]し 合[あ]う	はなしあう	
\\	私たちはよく話し合いました。	私[わたし]たちはよく 話[はな]し 合[あ]いました。	わたしたち は よく はなしあいました	
\\	私[わたし]たちはよく
\\	合う	合[あ]う	あう	
\\	この靴は私の足に合っている。	この 靴[くつ]は 私[わたし]の 足[あし]に 合[あ]っている。	この くつ は わたし の あし に あって いる	
\\	この 靴[くつ]は 私[わたし]の 足[あし]に
\\	間に合う	間[ま]に 合[あ]う	まにあう	
\\	授業に間に合いました。	授業[じゅぎょう]に 間[ま]に 合[あ]いました。	じゅぎょう に まにあいました	
\\	授業[じゅぎょう]に
\\	ガス	ガス	ガス	
\\	地震でガスが止まった。	地震[じしん]でガスが 止[と]まった。	じしん で がす が とまった	
\\	地震[じしん]で
\\	が 止[と]まった。			
\\	会社	会社[かいしゃ]	かいしゃ	
\\	彼は小さな会社に勤めています。	彼[かれ]は 小[ちい]さな 会社[かいしゃ]に 勤[つと]めています。	かれ は ちいさ な かいしゃ に つとめて います	
\\	彼[かれ]は 小[ちい]さな
\\	に 勤[つと]めています。			
\\	社会	社会[しゃかい]	しゃかい	
\\	これは大きな社会問題になっている。	これは 大[おお]きな 社会[しゃかい] 問題[もんだい]になっている。	これ は おおき な しゃかい もんだい に なって いる	
\\	これは 大[おお]きな
\\	問題[もんだい]になっている。			
\\	社員	社員[しゃいん]	しゃいん	
\\	彼は優秀な社員です。	彼[かれ]は 優秀[ゆうしゅう]な 社員[しゃいん]です。	かれ は ゆうしゅう な しゃいん です	
\\	彼[かれ]は 優秀[ゆうしゅう]な
\\	です。			
\\	仕方	仕方[しかた]	しかた	
\\	ファイルのダウンロードの仕方が分かりません。	ファイルのダウンロードの 仕方[しかた]が 分[わ]かりません。	ふぁいる の だうんろーど の しかた が わかりません	
\\	ファイルのダウンロードの
\\	が 分[わ]かりません。			
\\	うまい	うまい	うまい	
\\	彼は野球がうまい。	彼[かれ]は 野球[やきゅう]がうまい。	かれ は やきゅう が うまい	
\\	彼[かれ]は 野球[やきゅう]が
\\	食事	食事[しょくじ]	しょくじ	
\\	今日は上司と食事をする。	今日[きょう]は 上司[じょうし]と 食事[しょくじ]をする。	きょう は じょうし と しょくじ を する	
\\	今日[きょう]は 上司[じょうし]と
\\	をする。			
\\	火事	火事[かじ]	かじ	
\\	火事です。119番に電話してください。	火事[かじ]です。 119番[ひゃくじゅうきゅうばん]に 電話[でんわ]してください。	かじ です ひゃくじゅうきゅうばん に でんわ して ください	
\\	です。 119番[ひゃくじゅうきゅうばん]に 電話[でんわ]してください。			
\\	大事	大事[だいじ]	だいじ	
\\	お体をお大事に。	お 体[からだ]をお 大事[だいじ]に。	おからだ を おだいじ に	
\\	お 体[からだ]をお
\\	に。			
\\	事故	事故[じこ]	じこ	
\\	彼は事故で怪我をしました。	彼[かれ]は 事故[じこ]で 怪我[けが]をしました。	かれ は じこ で けが を しました	
\\	彼[かれ]は
\\	で 怪我[けが]をしました。			
\\	サラリーマン	サラリーマン	サラリーマン	
\\	父はサラリーマンです。	父[ちち]はサラリーマンです。	ちち は さらりーまん です	
\\	父[ちち]は
\\	です。			
\\	工事	工事[こうじ]	こうじ	
\\	工事の音がうるさい。	工事[こうじ]の 音[おと]がうるさい。	こうじ の おと が うるさい	
\\	の 音[おと]がうるさい。			
\\	工場	工場[こうじょう]	こうじょう	
\\	彼は食品工場で働いています。	彼[かれ]は 食品[しょくひん] 工場[こうじょう]で 働[はたら]いています。	かれ は しょくひん こうじょう で はたらいて います	
\\	彼[かれ]は 食品[しょくひん]
\\	で 働[はたら]いています。			
\\	電話	電話[でんわ]	でんわ	
\\	あとで電話します。	あとで 電話[でんわ]します。	あとで でんわ します	
\\	あとで
\\	します。			
\\	電気	電気[でんき]	でんき	
\\	電気をつけてください。	電気[でんき]をつけてください。	でんき をつけてください	
\\	をつけてください。			
\\	バス	バス	バス	
\\	バスで行こう。	バスで 行[い]こう。	ばす で いこう	
\\	で 行[い]こう。			
\\	車	車[くるま]	くるま	
\\	この道は車が多い。	この 道[みち]は 車[くるま]が 多[おお]い。	この みち は くるま が おおい	
\\	この 道[みち]は
\\	が 多[おお]い。			
\\	駅員	駅員[えきいん]	えきいん	
\\	駅員に聞きましょう。	駅員[えきいん]に 聞[き]きましょう。	えきいん に ききましょう	
\\	に 聞[き]きましょう。			
\\	通り	通[とお]り	とおり	
\\	この通りはにぎやかですね。	この 通[とお]りはにぎやかですね。	この とおり は にぎやか です ね	
\\	この
\\	はにぎやかですね。			
\\	通る	通[とお]る	とおる	
\\	毎日、この道を通ります。	毎日[まいにち]、この 道[みち]を 通[とお]ります。	まいにち この みち を とおります	
\\	毎日[まいにち]、この 道[みち]を
\\	クラス	クラス	クラス	
\\	この学校は1クラス30人です。	この 学校[がっこう]は1クラス30 人[にん]です。	この がっこう は 
\\	くらす 
\\	にん です 。	
\\	この 学校[がっこう]は1
\\	人[にん]です。			
\\	通う	通[かよ]う	かよう	
\\	私はジムに通っています。	私[わたし]はジムに 通[かよ]っています。	わたし は じむ に かよって います	
\\	私[わたし]はジムに
\\	交通事故	交通事故[こうつうじこ]	こうつうじこ	
\\	彼は交通事故を起こした。	彼[かれ]は 交通事故[こうつうじこ]を 起[お]こした。	かれ は こうつうじこ を おこした	
\\	彼[かれ]は
\\	を 起[お]こした。			
\\	水道	水道[すいどう]	すいどう	
\\	東京は水道の水が不味い。	東京[とうきょう]は 水道[すいどう]の 水[みず]が 不味[まず]い。	とうきょう は すいどう の みず が まずい	
\\	東京[とうきょう]は
\\	の 水[みず]が 不味[まず]い。			
\\	車道	車道[しゃどう]	しゃどう	
\\	車道の工事が始まりました。	車道[しゃどう]の 工事[こうじ]が 始[はじ]まりました。	しゃどう の こうじ が はじまりました	
\\	の 工事[こうじ]が 始[はじ]まりました。			
\\	トラック	トラック	トラック	
\\	トラックを運転できますか。	トラックを 運転[うんてん]できますか。	とらっく を うんてん できます か	
\\	を 運転[うんてん]できますか。			
\\	道路	道路[どうろ]	どうろ	
\\	この道路は3年前にできました。	この 道路[どうろ]は 3年前[さんねんまえ]にできました。	この どうろ は さんねんまえ に できました	
\\	この
\\	は 3年前[さんねんまえ]にできました。			
\\	土地	土地[とち]	とち	
\\	ここは父の土地です。	ここは 父[ちち]の 土地[とち]です。	ここ は ちち の とち です	
\\	ここは 父[ちち]の
\\	です。			
\\	地図	地図[ちず]	ちず	
\\	地図を見て来てください。	地図[ちず]を 見[み]て 来[き]てください。	ちず を みて きて ください	
\\	を 見[み]て 来[き]てください。			
\\	他	他[た]	た	
\\	留学生はアジア人が多く、その他は3割です。	留学生[りゅうがくせい]はアジア 人[じん]が 多[おお]く、その 他[た]は 3割[さんわり]です。	りゅうがくせい は あじあじん が おおく そのた は さんわり です	
\\	留学生[りゅうがくせい]はアジア 人[じん]が 多[おお]く、その
\\	は 3割[さんわり]です。			
\\	パーティー	パーティー	パーティー	
\\	明日、うちでパーティーを開きます。	明日[あした]、うちでパーティーを 開[ひら]きます。	あした うち で ぱーてぃー を ひらきます	
\\	明日[あした]、うちで
\\	を 開[ひら]きます。			
\\	止める	止[と]める	とめる	
\\	車を止めて。	車[くるま]を 止[と]めて。	くるま を とめて	
\\	車[くるま]を
\\	止まる	止[と]まる	とまる	
\\	今朝、事故で電車が止まりました。	今朝[けさ]、 事故[じこ]で 電車[でんしゃ]が 止[と]まりました。	けさ じこ で でんしゃ が とまりました	
\\	今朝[けさ]、 事故[じこ]で 電車[でんしゃ]が
\\	止む	止[や]む	やむ	
\\	雨が止みました。	雨[あめ]が 止[や]みました。	あめ が やみました	
\\	雨[あめ]が
\\	歩道	歩道[ほどう]	ほどう	
\\	歩道を歩きましょう。	歩道[ほどう]を 歩[ある]きましょう。	ほどう を あるきましょう	
\\	を 歩[ある]きましょう。			
\\	うまい	うまい	うまい	
\\	うまい寿司屋を見つけたよ。	うまい 寿司屋[すしや]を 見[み]つけたよ。	うまい すしや を みつけた よ	
\\	寿司屋[すしや]を 見[み]つけたよ。			
\\	渡す	渡[わた]す	わたす	
\\	彼に手紙を渡しました。	彼[かれ]に 手紙[てがみ]を 渡[わた]しました。	かれ に てがみ を わたしました	
\\	彼[かれ]に 手紙[てがみ]を
\\	渡る	渡[わた]る	わたる	
\\	私たちは歩いて橋を渡った。	私[わたし]たちは 歩[ある]いて 橋[はし]を 渡[わた]った。	わたしたち は あるいて はし を わたった	
\\	私[わたし]たちは 歩[ある]いて 橋[はし]を
\\	年度	年度[ねんど]	ねんど	
\\	売り上げは年度によって違います。	売[う]り 上[あ]げは 年度[ねんど]によって 違[ちが]います。	うりあげ は ねんど に よって ちがいます	
\\	売[う]り 上[あ]げは
\\	によって 違[ちが]います。			
\\	今度	今度[こんど]	こんど	
\\	今度はどこに行きたいですか。	今度[こんど]はどこに 行[い]きたいですか。	こんど は どこ に いきたい です か	
\\	はどこに 行[い]きたいですか。			
\\	ガラス	ガラス	ガラス	
\\	クリスタルガラスの花瓶を買いました。	クリスタルガラスの 花瓶[かびん]を 買[か]いました。	くりすたるがらす の かびん を かいました	
\\	クリスタル
\\	の 花瓶[かびん]を 買[か]いました。			
\\	何度	何度[なんど]	なんど	
\\	あの店には何度も行きました。	あの 店[みせ]には 何度[なんど]も 行[い]きました。	あの みせ に は なんど も いきました	
\\	あの 店[みせ]には
\\	も 行[い]きました。			
\\	最近	最近[さいきん]	さいきん	
\\	それは最近話題の本ですね。	それは 最近[さいきん] 話題[わだい]の 本[ほん]ですね。	それ は さいきん わだい の ほん です ね	
\\	それは
\\	話題[わだい]の 本[ほん]ですね。			
\\	遠く	遠[とお]く	とおく	
\\	遠くに船が見えます。	遠[とお]くに 船[ふね]が 見[み]えます。	とおく に ふね が みえます	
\\	に 船[ふね]が 見[み]えます。			
\\	社長	社長[しゃちょう]	しゃちょう	
\\	おばは小さな会社の社長です。	おばは 小[ちい]さな 会社[かいしゃ]の 社長[しゃちょう]です。	おば は ちいさ な かいしゃ の しゃちょう です	
\\	おばは 小[ちい]さな 会社[かいしゃ]の
\\	です。			
\\	コース	コース	コース	
\\	私は日本語コースを取っています。	私[わたし]は 日本語[にほんご]コースを 取[と]っています。	わたし は にほんご こーす を とって います	
\\	私[わたし]は 日本語[にほんご]
\\	を 取[と]っています。			
\\	会長	会長[かいちょう]	かいちょう	
\\	初めに会長が挨拶した。	初[はじ]めに 会長[かいちょう]が 挨拶[あいさつ]した。	はじめ に かいちょう が あいさつ した	
\\	初[はじ]めに
\\	が 挨拶[あいさつ]した。			
\\	長さ	長[なが]さ	ながさ	
\\	このケーブルの長さは1メートルです。	このケーブルの 長[なが]さは 1[いち]メートルです。	この けーぶる の ながさ は いちめーとる です	
\\	このケーブルの
\\	は 1[いち]メートルです。			
\\	長男	長男[ちょうなん]	ちょうなん	
\\	彼の長男は8才です。	彼[かれ]の 長男[ちょうなん]は 8才[はっさい]です。	かれ の ちょうなん は はっさい です	
\\	彼[かれ]の
\\	は 8才[はっさい]です。			
\\	長女	長女[ちょうじょ]	ちょうじょ	
\\	うちの長女は10歳です。	うちの 長女[ちょうじょ]は 10歳[じゅっさい]です。	うち の ちょうじょ は じゅっさい です	
\\	うちの
\\	は 10歳[じゅっさい]です。			
\\	アパート	アパート	アパート	
\\	彼はアパートに住んでいます。	彼[かれ]はアパートに 住[す]んでいます。	かれ は あぱーと に すんで います	
\\	彼[かれ]は
\\	に 住[す]んでいます。			
\\	広がる	広[ひろ]がる	ひろがる	
\\	留学してから私の世界が広がった。	留学[りゅうがく]してから 私[わたし]の 世界[せかい]が 広[ひろ]がった。	りゅうがく して から わたし の せかい が ひろがった	
\\	留学[りゅうがく]してから 私[わたし]の 世界[せかい]が
\\	広さ	広[ひろ]さ	ひろさ	
\\	その家の広さはどれ位ですか。	その 家[いえ]の 広[ひろ]さはどれ 位[くらい]ですか。	その いえ の ひろさ は どれ くらい です か	
\\	その 家[いえ]の
\\	はどれ 位[くらい]ですか。			
\\	全体	全体[ぜんたい]	ぜんたい	
\\	全体の80パーセントが完成しました。	全体[ぜんたい]の 80[はちじゅっ]パーセントが 完成[かんせい]しました。	ぜんたい の はちじゅっぱーせんと が かんせい しました	
\\	の 80[はちじゅっ]パーセントが 完成[かんせい]しました。			
\\	全く	全[まった]く	まったく	
\\	私には全く分かりません。	私[わたし]には 全[まった]く 分[わ]かりません。	わたし に は まったく わかりません	
\\	私[わたし]には
\\	分[わ]かりません。			
\\	レコード	レコード	レコード	
\\	ジャズのレコードをかけました。	ジャズのレコードをかけました。	じゃず の れこーど を かけました	
\\	ジャズの
\\	をかけました。			
\\	安全	安全[あんぜん]	あんぜん	
\\	安全が第一です。	安全[あんぜん]が 第一[だいいち]です。	あんぜん が だいいち です	
\\	が 第一[だいいち]です。			
\\	一部	一部[いちぶ]	いちぶ	
\\	計画を一部変更しましょう。	計画[けいかく]を 一部[いちぶ] 変更[へんこう]しましょう。	けいかく を いちぶ へんこう しましょう	
\\	計画[けいかく]を
\\	変更[へんこう]しましょう。			
\\	部分	部分[ぶぶん]	ぶぶん	
\\	この部分は問題ないです。	この 部分[ぶぶん]は 問題[もんだい]ないです。	この ぶぶん は もんだい ない です	
\\	この
\\	は 問題[もんだい]ないです。			
\\	国内	国内[こくない]	こくない	
\\	この携帯電話が使えるのは国内だけです。	この 携帯電話[けいたい でんわ]が 使[つか]えるのは 国内[こくない]だけです。	この けいたい でんわ が つかえる の は こくない だけ です	
\\	この 携帯電話[けいたい でんわ]が 使[つか]えるのは
\\	だけです。			
\\	どんどん	どんどん	どんどん	
\\	ドアをどんどんとたたいた。	ドアをどんどんとたたいた。	どあ を どんどん と たたいた	
\\	ドアを
\\	とたたいた。			
\\	全国	全国[ぜんこく]	ぜんこく	
\\	次は全国のお天気です。	次[つぎ]は 全国[ぜんこく]のお 天気[てんき]です。	つぎ は ぜんこく の おてんき です	
\\	次[つぎ]は
\\	のお 天気[てんき]です。			
\\	外国	外国[がいこく]	がいこく	
\\	母はまだ外国に行ったことがありません。	母[はは]はまだ 外国[がいこく]に 行[い]ったことがありません。	はは は まだ がいこく に いった こと が ありません	
\\	母[はは]はまだ
\\	に 行[い]ったことがありません。			
\\	国会	国会[こっかい]	こっかい	
\\	国会が再開した。	国会[こっかい]が 再開[さいかい]した。	こっかい が さいかい した	
\\	が 再開[さいかい]した。			
\\	帰国	帰国[きこく]	きこく	
\\	彼は帰国しました。	彼[かれ]は 帰国[きこく]しました。	かれ は きこく しました	
\\	彼[かれ]は
\\	しました。			
\\	カメラ	カメラ	カメラ	
\\	カメラが壊れた。	カメラが 壊[こわ]れた。	かめら が こわれた	
\\	が 壊[こわ]れた。			
\\	外国人	外国人[がいこくじん]	がいこくじん	
\\	日本に住む外国人が増えています。	日本[にほん]に 住[す]む 外国人[がいこくじん]が 増[ふ]えています。	にほん に すむ がいこくじん が ふえて います	
\\	日本[にほん]に 住[す]む
\\	が 増[ふ]えています。			
\\	外国語	外国語[がいこくご]	がいこくご	
\\	外国語を習うのは難しい。	外国語[がいこくご]を 習[なら]うのは 難[むずか]しい。	がいこくご を ならう の は むずかしい	
\\	を 習[なら]うのは 難[むずか]しい。			
\\	世界	世界[せかい]	せかい	
\\	私は世界旅行をしたい。	私[わたし]は 世界[せかい] 旅行[りょこう]をしたい。	わたし は せかい りょこう を したい	
\\	私[わたし]は
\\	旅行[りょこう]をしたい。			
\\	白	白[しろ]	しろ	
\\	白は雪の色です。	白[しろ]は 雪[ゆき]の 色[いろ]です。	しろ は ゆき の いろ です	
\\	は 雪[ゆき]の 色[いろ]です。			
\\	テープ	テープ	テープ	
\\	彼女はその会話をテープに録音した。	彼女[かのじょ]はその 会話[かいわ]をテープに 録音[ろくおん]した。	かのじょ は その かいわ を てーぷ に ろくおん した	
\\	彼女[かのじょ]はその 会話[かいわ]を
\\	に 録音[ろくおん]した。			
\\	黒い	黒[くろ]い	くろい	
\\	彼女は黒いドレスを着ています。	彼女[かのじょ]は 黒[くろ]いドレスを 着[き]ています。	かのじょ は くろい どれす を きて います	
\\	彼女[かのじょ]は
\\	ドレスを 着[き]ています。			
\\	黒	黒[くろ]	くろ	
\\	黒のボールペンはありますか。	黒[くろ]のボールペンはありますか。	くろ の ぼーるぺん は あります か	
\\	のボールペンはありますか。			
\\	赤ちゃん	赤[あか]ちゃん	あかちゃん	
\\	ベッドで赤ちゃんが眠っています。	ベッドで 赤[あか]ちゃんが 眠[ねむ]っています。	べっど で あかちゃん が ねむって います	
\\	ベッドで
\\	が 眠[ねむ]っています。			
\\	赤	赤[あか]	あか	
\\	信号が赤に変わりました。	信号[しんごう]が 赤[あか]に 変[か]わりました。	しんごう が あか に かわりました	
\\	信号[しんごう]が
\\	に 変[か]わりました。			
\\	ビール	ビール	ビール	
\\	夏はビールがとても美味しい。	夏[なつ]はビールがとても 美味[おい]しい。	なつ は びーる が とても おいしい	
\\	夏[なつ]は
\\	がとても 美味[おい]しい。			
\\	銀行	銀行[ぎんこう]	ぎんこう	
\\	銀行は3時まで開いています。	銀行[ぎんこう]は 3時[さんじ]まで 開[あ]いています。	ぎんこう は さんじ まで あいて います	
\\	は 3時[さんじ]まで 開[あ]いています。			
\\	銀	銀[ぎん]	ぎん	
\\	彼は銀メダルを取った。	彼[かれ]は 銀[ぎん]メダルを 取[と]った。	かれ は ぎんめだる を とった	
\\	彼[かれ]は
\\	メダルを 取[と]った。			
\\	地下鉄	地下鉄[ちかてつ]	ちかてつ	
\\	私は地下鉄で通勤しています。	私[わたし]は 地下鉄[ちかてつ]で 通勤[つうきん]しています。	わたし は ちかてつ で つうきん して います	
\\	私[わたし]は
\\	で 通勤[つうきん]しています。			
\\	牛肉	牛肉[ぎゅうにく]	ぎゅうにく	
\\	夕食に牛肉を買った。	夕食[ゆうしょく]に 牛肉[ぎゅうにく]を 買[か]った。	ゆうしょく に ぎゅうにく を かった	
\\	夕食[ゆうしょく]に
\\	を 買[か]った。			
\\	ページ	ページ	ページ	
\\	32ページを開いてください。	32[さんじゅうに]ページを 開[ひら]いてください。	さんじゅうにぺーじ を ひらいて ください	
\\	32[さんじゅうに]
\\	を 開[ひら]いてください。			
\\	肉	肉[にく]	にく	
\\	肉が焼けました。	肉[にく]が 焼[や]けました。	にく が やけました	
\\	が 焼[や]けました。			
\\	魚	魚[さかな]	さかな	
\\	肉と魚とどちらが好きですか。	肉[にく]と 魚[さかな]とどちらが 好[す]きですか。	にく と さかな と どちら が すき です か	
\\	肉[にく]と
\\	とどちらが 好[す]きですか。			
\\	分野	分野[ぶんや]	ぶんや	
\\	音楽は彼の得意な分野です。	音楽[おんがく]は 彼[かれ]の 得意[とくい]な 分野[ぶんや]です。	おんがく は かれ の とくい な ぶんや です	
\\	音楽[おんがく]は 彼[かれ]の 得意[とくい]な
\\	です。			
\\	野菜	野菜[やさい]	やさい	
\\	私は毎日たくさん野菜を食べます。	私[わたし]は 毎日[まいにち]たくさん 野菜[やさい]を 食[た]べます。	わたし は まいにち たくさん やさい を たべます	
\\	私[わたし]は 毎日[まいにち]たくさん
\\	を 食[た]べます。			
\\	グラフ	グラフ	グラフ	
\\	彼は売上をグラフにした。	彼[かれ]は 売上[うりあげ]をグラフにした。	かれ は うりあげ を ぐらふ に した	
\\	彼[かれ]は 売上[うりあげ]を
\\	にした。			
\\	本屋	本屋[ほんや]	ほんや	
\\	駅前に本屋があります。	駅前[えきまえ]に 本屋[ほんや]があります。	えきまえ に ほんや が あります	
\\	駅前[えきまえ]に
\\	があります。			
\\	八百屋	八百屋[やおや]	やおや	
\\	八百屋でみかんを買いました。	八百屋[やおや]でみかんを 買[か]いました。	やおや で みかん を かいました	
\\	でみかんを 買[か]いました。			
\\	そば屋	そば 屋[や]	そばや	
\\	昼はそば屋に行きました。	昼[ひる]はそば 屋[や]に 行[い]きました。	ひる は そばや に いきました 。	
\\	昼[ひる]は
\\	に 行[い]きました。			
\\	たばこ屋	たばこ 屋[や]	たばこや	
\\	私はたばこ屋でライターを買った。	私[わたし]はたばこ 屋[や]でライターを 買[か]った。	わたし は たばこや で らいたー を かった	
\\	私[わたし]は
\\	でライターを 買[か]った。			
\\	ポスト	ポスト	ポスト	
\\	手紙をポストに入れました。	手紙[てがみ]をポストに 入[い]れました。	てがみ を ぽすと に いれました	
\\	手紙[てがみ]を
\\	に 入[い]れました。			
\\	茶	茶[ちゃ]	ちゃ	
\\	私たちは毎日お茶を飲みます。	私[わたし]たちは 毎日[まいにち]お 茶[ちゃ]を 飲[の]みます。	わたしたち は まいにち おちゃ を のみます	
\\	私[わたし]たちは 毎日[まいにち]お
\\	を 飲[の]みます。			
\\	お茶	お 茶[ちゃ]	おちゃ	
\\	お茶を入れましょうか。	お 茶[ちゃ]を 入[い]れましょうか。	おちゃ を いれましょう か	
\\	を 入[い]れましょうか。			
\\	茶わん	茶[ちゃ]わん	ちゃわん	
\\	茶わんにご飯をよそいました。	茶[ちゃ]わんにご 飯[はん]をよそいました。	ちゃわん に ごはん を よそいました	
\\	にご 飯[はん]をよそいました。			
\\	味	味[あじ]	あじ	
\\	この料理は味が薄い。	この 料理[りょうり]は 味[あじ]が 薄[うす]い。	この りょうり は あじ が うすい	
\\	この 料理[りょうり]は
\\	が 薄[うす]い。			
\\	テスト	テスト	テスト	
\\	テストを始めてください。	テストを 始[はじ]めてください。	てすと を はじめて ください	
\\	を 始[はじ]めてください。			
\\	未来	未来[みらい]	みらい	
\\	未来は誰にも分からない。	未来[みらい]は 誰[だれ]にも 分[わ]からない。	みらい は だれ に も わからない	
\\	は 誰[だれ]にも 分[わ]からない。			
\\	週末	週末[しゅうまつ]	しゅうまつ	
\\	週末は家でゆっくりします。	週末[しゅうまつ]は 家[うち]でゆっくりします。	しゅうまつ は うち で ゆっくり します	
\\	は 家[うち]でゆっくりします。			
\\	料理	料理[りょうり]	りょうり	
\\	母は料理が得意です。	母[はは]は 料理[りょうり]が 得意[とくい]です。	はは は りょうり が とくい です	
\\	母[はは]は
\\	が 得意[とくい]です。			
\\	無理	無理[むり]	むり	
\\	5時までに家に帰るのは無理です。	5時[ごじ]までに 家[いえ]に 帰[かえ]るのは 無理[むり]です。	ごじ まで に いえ に かえる の は むり です	
\\	5時[ごじ]までに 家[いえ]に 帰[かえ]るのは
\\	です。			
\\	あちこち	あちこち	あちこち	
\\	私たちは朝からあちこち散歩しました。	私[わたし]たちは 朝[あさ]からあちこち 散歩[さんぽ]しました。	わたしたち は あさ から あちこち さんぽ しました	
\\	私[わたし]たちは 朝[あさ]から
\\	散歩[さんぽ]しました。			
\\	無くす	無[な]くす	なくす	
\\	今日、鍵を無くしました。	今日[きょう]、 鍵[かぎ]を 無[な]くしました。	きょう かぎ を なくしました	
\\	今日[きょう]、 鍵[かぎ]を
\\	無くなる	無[な]くなる	なくなる	
\\	もうお金が無くなりました。	もうお 金[かね]が 無[な]くなりました。	もう おかね が なくなりました	
\\	もうお 金[かね]が
\\	作文	作文[さくぶん]	さくぶん	
\\	日本語で作文を書きました。	日本語[にほんご]で 作文[さくぶん]を 書[か]きました。	にほんご で さくぶん を かきました	
\\	日本語[にほんご]で
\\	を 書[か]きました。			
\\	用いる	用[もち]いる	もちいる	
\\	彼はその詩を用いて自分の気持ちを伝えた。	彼[かれ]はその 詩[し]を 用[もち]いて 自分[じぶん]の 気持[きも]ちを 伝[つた]えた。	かれ は その し を もちいて じぶん の きもち を つたえた	
\\	彼[かれ]はその 詩[し]を
\\	自分[じぶん]の 気持[きも]ちを 伝[つた]えた。			
\\	ゴルフ	ゴルフ	ゴルフ	
\\	兄はゴルフを始めました。	兄[あに]はゴルフを 始[はじ]めました。	あに は ごるふ を はじめました	
\\	兄[あに]は
\\	を 始[はじ]めました。			
\\	用事	用事[ようじ]	ようじ	
\\	父は用事で出掛けています。	父[ちち]は 用事[ようじ]で 出掛[でか]けています。	ちち は ようじ で でかけて います	
\\	父[ちち]は
\\	で 出掛[でか]けています。			
\\	交通費	交通費[こうつうひ]	こうつうひ	
\\	会社までの交通費は一ヶ月8,000円です。	会社[かいしゃ]までの 交通費[こうつうひ]は 一ヶ月8,000円[いっかげつ はっせんえん]です。	かいしゃ まで の こうつうひ は いっかげつ はっせんえん です	
\\	会社[かいしゃ]までの
\\	は 一ヶ月8,000円[いっかげつ はっせんえん]です。			
\\	消える	消[き]える	きえる	
\\	突然、電気が消えた。	突然[とつぜん]、 電気[でんき]が 消[き]えた。	とつぜん でんき が きえた	
\\	突然[とつぜん]、 電気[でんき]が
\\	消しゴム	消[け]しゴム	けしごむ	
\\	消しゴムを貸して下さい。	消[け]しゴムを 貸[か]して 下[くだ]さい。	けしごむ を かして ください	
\\	を 貸[か]して 下[くだ]さい。			
\\	ラジオ	ラジオ	ラジオ	
\\	彼女はラジオを聞いています。	彼女[かのじょ]はラジオを 聞[き]いています。	かのじょ は らじお を きいて います	
\\	彼女[かのじょ]は
\\	を 聞[き]いています。			
\\	売れる	売[う]れる	うれる	
\\	今年の夏はクーラーがよく売れた。	今年[ことし]の 夏[なつ]はクーラーがよく 売[う]れた。	ことし の なつ は くーらー が よく うれた	
\\	今年[ことし]の 夏[なつ]はクーラーがよく
\\	売り場	売[う]り 場[ば]	うりば	
\\	くつ売り場はどこですか。	くつ 売[う]り 場[ば]はどこですか。	くつうりば は どこ です か	
\\	くつ
\\	はどこですか。			
\\	店員	店員[てんいん]	てんいん	
\\	あの店員はとても親切です。	あの 店員[てんいん]はとても 親切[しんせつ]です。	あの てんいん は とても しんせつ です	
\\	あの
\\	はとても 親切[しんせつ]です。			
\\	売店	売店[ばいてん]	ばいてん	
\\	駅の売店で雑誌を買った。	駅[えき]の 売店[ばいてん]で 雑誌[ざっし]を 買[か]った。	えき の ばいてん で ざっし を かった	
\\	駅[えき]の
\\	で 雑誌[ざっし]を 買[か]った。			
\\	タクシー	タクシー	タクシー	
\\	タクシーを呼んでください。	タクシーを 呼[よ]んでください。	たくしー を よんで ください	
\\	を 呼[よ]んでください。			
\\	商品	商品[しょうひん]	しょうひん	
\\	この商品はよく売れている。	この 商品[しょうひん]はよく 売[う]れている。	この しょうひん は よく うれて いる	
\\	この
\\	はよく 売[う]れている。			
\\	作品	作品[さくひん]	さくひん	
\\	この絵はゴッホの作品だ。	この 絵[え]はゴッホの 作品[さくひん]だ。	この え は ごっほ の さくひん だ	
\\	この 絵[え]はゴッホの
\\	だ。			
\\	販売	販売[はんばい]	はんばい	
\\	前売券は窓口で販売しています。	前売券[まえうりけん]は 窓口[まどぐち]で 販売[はんばい]しています。	まえうりけん は まどぐち で はんばい して います	
\\	前売券[まえうりけん]は 窓口[まどぐち]で
\\	しています。			
\\	二階	二階[にかい]	にかい	
\\	兄は二階にいます。	兄[あに]は 二階[にかい]にいます。	あに は にかい に います	
\\	兄[あに]は
\\	にいます。			
\\	ゆっくり	ゆっくり	ゆっくり	
\\	もっとゆっくり話してください。	もっとゆっくり 話[はな]してください。	もっと ゆっくり はなして ください	
\\	もっと
\\	話[はな]してください。			
\\	段階	段階[だんかい]	だんかい	
\\	この段階では、決断するのはまだ早い。	この 段階[だんかい]では、 決断[けつだん]するのはまだ 早[はや]い。	この だんかい で は けつだん する の は まだ はやい	
\\	この
\\	では、 決断[けつだん]するのはまだ 早[はや]い。			
\\	階段	階段[かいだん]	かいだん	
\\	私たちは駅の階段をかけ上がった。	私[わたし]たちは 駅[えき]の 階段[かいだん]をかけ 上[あ]がった。	わたしたち は えき の かいだん を かけあがった	
\\	私[わたし]たちは 駅[えき]の
\\	をかけ 上[あ]がった。			
\\	段々	段々[だんだん]	だんだん	
\\	段々仕事が楽しくなってきました。	段々[だんだん] 仕事[しごと]が 楽[たの]しくなってきました。	だんだん しごと が たのしく なって きました	
\\	仕事[しごと]が 楽[たの]しくなってきました。			
\\	値段	値段[ねだん]	ねだん	
\\	このベルトの値段は2500円でした。	このベルトの 値段[ねだん]は 2500円[にせんごひゃくえん]でした。	この べると の ねだん は にせんごひゃくえん でした	
\\	このベルトの
\\	は 2500円[にせんごひゃくえん]でした。			
\\	レストラン	レストラン	レストラン	
\\	レストランでインド料理を食べました。	レストランでインド 料理[りょうり]を 食[た]べました。	れすとらん で いんど りょうり を たべました	
\\	でインド 料理[りょうり]を 食[た]べました。			
\\	価格	価格[かかく]	かかく	
\\	ガソリンの価格がどんどん上がっている。	ガソリンの 価格[かかく]がどんどん 上[あ]がっている。	がそりん の かかく が どんどん あがって いる	
\\	ガソリンの
\\	がどんどん 上[あ]がっている。			
\\	合格	合格[ごうかく]	ごうかく	
\\	娘が入学試験に合格しました。	娘[むすめ]が 入学試験[にゅうがく しけん]に 合格[ごうかく]しました。	むすめ が にゅうがく しけん に ごうかく しました	
\\	娘[むすめ]が 入学試験[にゅうがく しけん]に
\\	しました。			
\\	夏休み	夏休[なつやす]み	なつやすみ	
\\	今日が夏休み最後の日だ。	今日[きょう]が 夏休[なつやす]み 最後[さいご]の 日[ひ]だ。	きょう が なつやすみ さいご の ひ だ	
\\	今日[きょう]が
\\	最後[さいご]の 日[ひ]だ。			
\\	冬休み	冬休[ふゆやす]み	ふゆやすみ	
\\	冬休みにお祖父ちゃんの家に行きます。	冬休[ふゆやす]みにお 祖父[じい]ちゃんの 家[うち]に 行[い]きます。	ふゆやすみ に おじいちゃん の うち に いきます	
\\	にお 祖父[じい]ちゃんの 家[うち]に 行[い]きます。			
\\	カード	カード	カード	
\\	支払いはカードでお願いします。	支払[しはら]いはカードでお 願[ねが]いします。	しはらい は かーど で おねがい します	
\\	支払[しはら]いは
\\	でお 願[ねが]いします。			
\\	四季	四季[しき]	しき	
\\	日本には四季がある。	日本[にほん]には 四季[しき]がある。	にほん に は しき が ある	
\\	日本[にほん]には
\\	がある。			
\\	暑さ	暑[あつ]さ	あつさ	
\\	今年の夏は暑さが厳しい。	今年[ことし]の 夏[なつ]は 暑[あつ]さが 厳[きび]しい。	ことし の なつ は あつさ が きびしい	
\\	今年[ことし]の 夏[なつ]は
\\	が 厳[きび]しい。			
\\	熱	熱[ねつ]	ねつ	
\\	昨日の夜、熱が出ました。	昨日[きのう]の 夜[よる]、 熱[ねつ]が 出[で]ました。	きのう の よる ねつ が でました	
\\	昨日[きのう]の 夜[よる]、
\\	が 出[で]ました。			
\\	寒さ	寒[さむ]さ	さむさ	
\\	今日は厳しい寒さになるでしょう。	今日[きょう]は 厳[きび]しい 寒[さむ]さになるでしょう。	きょう は きびしい さむさ に なる でしょう	
\\	今日[きょう]は 厳[きび]しい
\\	になるでしょう。			
\\	アルバイト	アルバイト	アルバイト	
\\	兄はアルバイトをしています。	兄[あに]はアルバイトをしています。	あに は あるばいと を して います	
\\	兄[あに]は
\\	をしています。			
\\	暖める	暖[あたた]める	あたためる	
\\	今、車を暖めています。	今[いま]、 車[くるま]を 暖[あたた]めています。	いま くるま を あたためています	
\\	今[いま]、 車[くるま]を
\\	暖まる	暖[あたた]まる	あたたまる	
\\	まだ部屋が暖まらない。	まだ 部屋[へや]が 暖[あたた]まらない。	まだ へや が あたたまらない。	
\\	まだ 部屋[へや]が
\\	温度	温度[おんど]	おんど	
\\	今、部屋の温度は25度だ。	今[いま]、 部屋[へや]の 温度[おんど]は 25度[にじゅうごど]だ。	いま へや の おんど は にじゅうごど だ	
\\	今[いま]、 部屋[へや]の
\\	は 25度[にじゅうごど]だ。			
\\	気温	気温[きおん]	きおん	
\\	今日の気温は26度です。	今日[きょう]の 気温[きおん]は 26度[にじゅうろくど]です。	きょう の きおん は にじゅうろくど です	
\\	今日[きょう]の
\\	は 26度[にじゅうろくど]です。			
\\	コピー	コピー	コピー	
\\	会議で書類のコピーを配った。	会議[かいぎ]で 書類[しょるい]のコピーを 配[くば]った。	かいぎ で しょるい の こぴー を くばった	
\\	会議[かいぎ]で 書類[しょるい]の
\\	を 配[くば]った。			
\\	台	台[だい]	だい	
\\	そこにちょうど良い台がある。	そこにちょうど 良[い]い 台[だい]がある。	そこ に ちょうど いい だい が ある	
\\	そこにちょうど 良[い]い
\\	がある。			
\\	風	風[かぜ]	かぜ	
\\	今日は風が強いです。	今日[きょう]は 風[かぜ]が 強[つよ]いです。	きょう は かぜ が つよい です	
\\	今日[きょう]は
\\	が 強[つよ]いです。			
\\	台風	台風[たいふう]	たいふう	
\\	台風が近づいている。	台風[たいふう]が 近[ちか]づいている。	たいふう が ちかづいて いる	
\\	が 近[ちか]づいている。			
\\	事情	事情[じじょう]	じじょう	
\\	あなたの事情はよく分かりました。	あなたの 事情[じじょう]はよく 分[わ]かりました。	あなた の じじょう は よく わかりました	
\\	あなたの
\\	はよく 分[わ]かりました。			
\\	ぶつかる	ぶつかる	ぶつかる	
\\	車が電柱にぶつかった。	車[くるま]が 電柱[でんちゅう]にぶつかった。	くるま が でんちゅう に ぶつかった	
\\	車[くるま]が 電柱[でんちゅう]に
\\	情報	情報[じょうほう]	じょうほう	
\\	学生たちはインターネットでいろいろな情報を集めた。	学生[がくせい]たちはインターネットでいろいろな 情報[じょうほう]を 集[あつ]めた。	がくせいたち は いんたーねっと で いろいろな じょうほう を あつめた	
\\	学生[がくせい]たちはインターネットでいろいろな
\\	を 集[あつ]めた。			
\\	報告	報告[ほうこく]	ほうこく	
\\	昨日の会議について報告があります。	昨日[きのう]の 会議[かいぎ]について 報告[ほうこく]があります。	きのう の かいぎ に ついて ほうこく が あります	
\\	昨日[きのう]の 会議[かいぎ]について
\\	があります。			
\\	新聞	新聞[しんぶん]	しんぶん	
\\	今日の新聞、どこに置いた?	今日[きょう]の 新聞[しんぶん]、どこに 置[お]いた?	きょう の しんぶん どこ に おいた	
\\	今日[きょう]の
\\	、どこに 置[お]いた?			
\\	新年	新年[しんねん]	しんねん	
\\	新年明けましておめでとうございます。	新年[しんねん] 明[あ]けましておめでとうございます。	しんねん あけまして おめでとう ございます	
\\	明[あ]けましておめでとうございます。			
\\	フィルム	フィルム	フィルム	
\\	旅行のためにたくさんフィルムを買った。	旅行[りょこう]のためにたくさんフィルムを 買[か]った。	りょこう の ため に たくさん ふぃるむ を かった	
\\	旅行[りょこう]のためにたくさん
\\	を 買[か]った。			
\\	良い	良[よ]い	よい	
\\	彼は良いところだけを強調した。	彼[かれ]は 良[よ]いところだけを 強調[きょうちょう]した。	かれ は よい ところ だけ を きょうちょう した	
\\	彼[かれ]は
\\	ところだけを 強調[きょうちょう]した。			
\\	中心	中心[ちゅうしん]	ちゅうしん	
\\	ここが建物の中心です。	ここが 建物[たてもの]の 中心[ちゅうしん]です。	ここ が たてもの の ちゅうしん です	
\\	ここが 建物[たてもの]の
\\	です。			
\\	安心	安心[あんしん]	あんしん	
\\	それを聞いて安心しました。	それを 聞[き]いて 安心[あんしん]しました。	それ を きいて あんしん しました	
\\	それを 聞[き]いて
\\	しました。			
\\	思い出す	思[おも]い 出[だ]す	おもいだす	
\\	大切な用事を思い出しました。	大切[たいせつ]な 用事[ようじ]を 思[おも]い 出[だ]しました。	たいせつ な ようじ を おもいだしました	
\\	大切[たいせつ]な 用事[ようじ]を
\\	デパート	デパート	デパート	
\\	私はデパートで靴を買った。	私[わたし]はデパートで 靴[くつ]を 買[か]った。	わたし は でぱーと で くつ を かった	
\\	私[わたし]は
\\	で 靴[くつ]を 買[か]った。			
\\	思い出	思[おも]い 出[で]	おもいで	
\\	旅行で楽しい思い出ができました。	旅行[りょこう]で 楽[たの]しい 思[おも]い 出[で]ができました。	りょこう で たのしい おもいで が できました	
\\	旅行[りょこう]で 楽[たの]しい
\\	ができました。			
\\	考え	考[かんが]え	かんがえ	
\\	それは良い考えです。	それは 良[い]い 考[かんが]えです。	それ は いい かんがえ です	
\\	それは 良[い]い
\\	です。			
\\	解決	解決[かいけつ]	かいけつ	
\\	トラブルがやっと解決した。	トラブルがやっと 解決[かいけつ]した。	とらぶる が やっと かいけつ した	
\\	トラブルがやっと
\\	した。			
\\	知らせる	知[し]らせる	しらせる	
\\	皆に会議の日にちを知らせた。	皆[みんな]に 会議[かいぎ]の 日[ひ]にちを 知[し]らせた。	みんな に かいぎ の ひにち を しらせた	
\\	皆[みんな]に 会議[かいぎ]の 日[ひ]にちを
\\	ベッド	ベッド	ベッド	
\\	彼はベッドで寝ています。	彼[かれ]はベッドで 寝[ね]ています。	かれ は べっど で ねて います	
\\	彼[かれ]は
\\	で 寝[ね]ています。			
\\	能力	能力[のうりょく]	のうりょく	
\\	彼は能力のある社員です。	彼[かれ]は 能力[のうりょく]のある 社員[しゃいん]です。	かれ は のうりょく の ある しゃいん です	
\\	彼[かれ]は
\\	のある 社員[しゃいん]です。			
\\	可能	可能[かのう]	かのう	
\\	20キロのダイエットは可能だと思いますか。	20[にじゅっ]キロのダイエットは 可能[かのう]だと 思[おも]いますか。	にじゅっきろ の だいえっと は かのう だ と おもいます か	
\\	20[にじゅっ]キロのダイエットは
\\	だと 思[おも]いますか。			
\\	可	可[か]	か	
\\	このアルバイトは「学生可」ですね。	このアルバイトは
\\	学生[がくせい] 可[か]」ですね。	この あるばいと は がくせい か です ね	
\\	このアルバイトは
\\	学生[がくせい]
\\	ですね。			
\\	郵便	郵便[ゆうびん]	ゆうびん	
\\	さっき郵便が届きました。	さっき 郵便[ゆうびん]が 届[とど]きました。	さっき ゆうびん が とどきました	
\\	さっき
\\	が 届[とど]きました。			
\\	コート	コート	コート	
\\	寒かったのでコートを着た。	寒[さむ]かったのでコートを 着[き]た。	さむかった の で こーと を きた	
\\	寒[さむ]かったので
\\	を 着[き]た。			
\\	不便	不便[ふべん]	ふべん	
\\	私の家は駅から遠くて不便です。	私[わたし]の 家[いえ]は 駅[えき]から 遠[とお]くて 不便[ふべん]です。	わたし の いえ は えき から とおく て ふべん です	
\\	私[わたし]の 家[いえ]は 駅[えき]から 遠[とお]くて
\\	です。			
\\	郵便屋さん	郵便屋[ゆうびんや]さん	ゆうびんやさん	
\\	郵便屋さんはもう来ましたか。	郵便屋[ゆうびんや]さんはもう 来[き]ましたか。	ゆうびんやさん は もう きました か	
\\	はもう 来[き]ましたか。			
\\	郵便局	郵便局[ゆうびんきょく]	ゆうびんきょく	
\\	郵便局で切手を買いました。	郵便局[ゆうびんきょく]で 切手[きって]を 買[か]いました。	ゆうびんきょくで きって を かいました	
\\	で 切手[きって]を 買[か]いました。			
\\	交番	交番[こうばん]	こうばん	
\\	あそこの交番で道を聞きましょう。	あそこの 交番[こうばん]で 道[みち]を 聞[き]きましょう。	あそこ の こうばん で みち を ききましょう	
\\	あそこの
\\	で 道[みち]を 聞[き]きましょう。			
\\	ノート	ノート	ノート	
\\	ノートを開いてください。	ノートを 開[ひら]いてください。	のーと を ひらいて ください	
\\	を 開[ひら]いてください。			
\\	番地	番地[ばんち]	ばんち	
\\	その建物の番地は分かりますか。	その 建物[たてもの]の 番地[ばんち]は 分[わ]かりますか。	その たてもの の ばんち は わかります か	
\\	その 建物[たてもの]の
\\	は 分[わ]かりますか。			
\\	番号	番号[ばんごう]	ばんごう	
\\	この番号に電話してください。	この 番号[ばんごう]に 電話[でんわ]してください。	この ばんごう に でんわ して ください	
\\	この
\\	に 電話[でんわ]してください。			
\\	場所	場所[ばしょ]	ばしょ	
\\	会社の場所を教えてください。	会社[かいしゃ]の 場所[ばしょ]を 教[おし]えてください。	かいしゃ の ばしょ を おしえて ください	
\\	会社[かいしゃ]の
\\	を 教[おし]えてください。			
\\	近所	近所[きんじょ]	きんじょ	
\\	近所にカナダ人が住んでいる。	近所[きんじょ]にカナダ 人[じん]が 住[す]んでいる。	きんじょ に かなだじん が すんで いる	
\\	にカナダ 人[じん]が 住[す]んでいる。			
\\	ワイン	ワイン	ワイン	
\\	ワインを少し飲みました。	ワインを 少[すこ]し 飲[の]みました。	わいん を すこし のみました	
\\	を 少[すこ]し 飲[の]みました。			
\\	台所	台所[だいどころ]	だいどころ	
\\	お母さんは台所にいます。	お 母[かあ]さんは 台所[だいどころ]にいます。	おかあさん は だいどころ に います	
\\	お 母[かあ]さんは
\\	にいます。			
\\	住所	住所[じゅうしょ]	じゅうしょ	
\\	この住所に行ってください。	この 住所[じゅうしょ]に 行[い]ってください。	この じゅうしょ に いって ください	
\\	この
\\	に 行[い]ってください。			
\\	便所	便所[べんじょ]	べんじょ	
\\	便所はそこです。	便所[べんじょ]はそこです。	べんじょ は そこ です	
\\	はそこです。			
\\	有名	有名[ゆうめい]	ゆうめい	
\\	ボルドーはワインの生産で有名だ。	ボルドーはワインの 生産[せいさん]で 有名[ゆうめい]だ。	ぼるどー は わいん の せいさん で ゆうめい だ	
\\	ボルドーはワインの 生産[せいさん]で
\\	だ。			
\\	おかしい	おかしい	おかしい	
\\	彼の様子がおかしい。	彼[かれ]の 様子[ようす]がおかしい。	かれ の ようす が おかしい	
\\	彼[かれ]の 様子[ようす]が
\\	名字	名字[みょうじ]	みょうじ	
\\	あなたの名字は何ですか。	あなたの 名字[みょうじ]は 何[なん]ですか。	あなた の みょうじ は なん です か	
\\	あなたの
\\	は 何[なん]ですか。			
\\	氏名	氏名[しめい]	しめい	
\\	ここに住所と氏名を書いてください。	ここに 住所[じゅうしょ]と 氏名[しめい]を 書[か]いてください。	ここ に じゅうしょ と しめい を かいて ください	
\\	ここに 住所[じゅうしょ]と
\\	を 書[か]いてください。			
\\	各国	各国[かっこく]	かっこく	
\\	各国の代表がニューヨークに集まった。	各国[かっこく]の 代表[だいひょう]がニューヨークに 集[あつ]まった。	かっこく の だいひょう が にゅーよーく に あつまった	
\\	の 代表[だいひょう]がニューヨークに 集[あつ]まった。			
\\	町	町[まち]	まち	
\\	あの町は緑を増やしています。	あの 町[まち]は 緑[みどり]を 増[ふ]やしています。	あの まち は みどり を ふやして います	
\\	あの
\\	は 緑[みどり]を 増[ふ]やしています。			
\\	トイレ	トイレ	トイレ	
\\	トイレを掃除しましたか。	トイレを 掃除[そうじ]しましたか。	といれ を そうじ しました か	
\\	を 掃除[そうじ]しましたか。			
\\	都市	都市[とし]	とし	
\\	東京は日本一大きな都市です。	東京[とうきょう]は 日本一大[にほんいち おお]きな 都市[とし]です。	とうきょう は にほんいち おおき な とし です	
\\	東京[とうきょう]は 日本一大[にほんいち おお]きな
\\	です。			
\\	都合	都合[つごう]	つごう	
\\	今日は都合が悪くて行けません。	今日[きょう]は 都合[つごう]が 悪[わる]くて 行[い]けません。	きょう は つごう が わるくて いけません	
\\	今日[きょう]は
\\	が 悪[わる]くて 行[い]けません。			
\\	朝御飯	朝御飯[あさごはん]	あさごはん	
\\	七時に朝御飯を食べました。	七時[しちじ]に 朝御飯[あさごはん]を 食[た]べました。	しちじ に あさごはん を たべました	
\\	七時[しちじ]に
\\	を 食[た]べました。			
\\	買い物	買[か]い 物[もの]	かいもの	
\\	母は買い物に出かけています。	母[はは]は 買[か]い 物[もの]に 出[で]かけています。	はは は かいもの に でかけて います	
\\	母[はは]は
\\	に 出[で]かけています。			
\\	キャンプ	キャンプ	キャンプ	
\\	友達とキャンプに行った。	友達[ともだち]とキャンプに 行[い]った。	ともだち と きゃんぷ に いった	
\\	友達[ともだち]と
\\	に 行[い]った。			
\\	荷物	荷物[にもつ]	にもつ	
\\	彼の家に荷物を送りました。	彼[かれ]の 家[いえ]に 荷物[にもつ]を 送[おく]りました。	かれ の いえ に にもつ を おくりました	
\\	彼[かれ]の 家[いえ]に
\\	を 送[おく]りました。			
\\	品物	品物[しなもの]	しなもの	
\\	その店は色々な品物を売っている。	その 店[みせ]は 色々[いろいろ]な 品物[しなもの]を 売[う]っている。	その みせ は いろいろ な しなもの を うって いる	
\\	その 店[みせ]は 色々[いろいろ]な
\\	を 売[う]っている。			
\\	見物	見物[けんぶつ]	けんぶつ	
\\	皆で東京見物をした。	皆[みんな]で 東京[とうきょう] 見物[けんぶつ]をした。	みんな で とうきょう けんぶつ を した	
\\	皆[みんな]で 東京[とうきょう]
\\	をした。			
\\	物	物[もの]	もの	
\\	彼女の家には物がたくさんあります。	彼女[かのじょ]の 家[いえ]には 物[もの]がたくさんあります。	かのじょ の いえ に は もの が たくさん あります	
\\	彼女[かのじょ]の 家[いえ]には
\\	がたくさんあります。			
\\	プラスチック	プラスチック	プラスチック	
\\	このカップはプラスチックです。	このカップはプラスチックです。	この かっぷ は ぷらすちっく です	
\\	このカップは
\\	です。			
\\	忘れ物	忘[わす]れ 物[もの]	わすれもの	
\\	学校に忘れ物をしました。	学校[がっこう]に 忘[わす]れ 物[もの]をしました。	がっこう に わすれもの を しました	
\\	学校[がっこう]に
\\	をしました。			
\\	重さ	重[おも]さ	おもさ	
\\	この荷物の重さを計ってください。	この 荷物[にもつ]の 重[おも]さを 計[はか]ってください。	この にもつ の おもさ を はかって ください	
\\	この 荷物[にもつ]の
\\	を 計[はか]ってください。			
\\	配る	配[くば]る	くばる	
\\	会議で書類を配った。	会議[かいぎ]で 書類[しょるい]を 配[くば]った。	かいぎ で しょるい を くばった	
\\	会議[かいぎ]で 書類[しょるい]を
\\	配達	配達[はいたつ]	はいたつ	
\\	彼は新聞配達をしている。	彼[かれ]は 新聞[しんぶん] 配達[はいたつ]をしている。	かれ は しんぶん はいたつ を して いる	
\\	彼[かれ]は 新聞[しんぶん]
\\	をしている。			
\\	カラー	カラー	カラー	
\\	カラーコピーは一枚幾らですか。	カラーコピーは 一枚幾[いちまい いく]らですか。	からーこぴー は いちまい いくら です か	
\\	コピーは 一枚幾[いちまい いく]らですか。			
\\	心配	心配[しんぱい]	しんぱい	
\\	明日のプレゼンテーションが心配だ。	明日[あす]のプレゼンテーションが 心配[しんぱい]だ。	あす の ぷれぜんてーしょん が しんぱい だ	
\\	明日[あす]のプレゼンテーションが
\\	だ。			
\\	見送る	見送[みおく]る	みおくる	
\\	彼が外国に行くのを見送りました。	彼[かれ]が 外国[がいこく]に 行[い]くのを 見送[みおく]りました。	かれ が がいこく に いく の を みおくりました	
\\	彼[かれ]が 外国[がいこく]に 行[い]くのを
\\	見送り	見送[みおく]り	みおくり	
\\	駅に友人の見送りに行きました。	駅[えき]に 友人[ゆうじん]の 見送[みおく]りに 行[い]きました。	えき に ゆうじん の みおくり に いきました	
\\	駅[えき]に 友人[ゆうじん]の
\\	に 行[い]きました。			
\\	受ける	受[う]ける	うける	
\\	彼は就職試験を受けた。	彼[かれ]は 就職試験[しゅうしょく しけん]を 受[う]けた。	かれ は しゅうしょく しけん を うけた	
\\	彼[かれ]は 就職試験[しゅうしょく しけん]を
\\	ピアノ	ピアノ	ピアノ	
\\	昔、ピアノを習っていました。	昔[むかし]、ピアノを 習[なら]っていました。	むかし ぴあの を ならって いました	
\\	昔[むかし]、
\\	を 習[なら]っていました。			
\\	受け取る	受[う]け 取[と]る	うけとる	
\\	彼からメールを受け取りました。	彼[かれ]からメールを 受[う]け 取[と]りました。	かれ から めーる を うけとりました	
\\	彼[かれ]からメールを
\\	取れる	取[と]れる	とれる	
\\	シャツのボタンが取れた。	シャツのボタンが 取[と]れた。	しゃつ の ぼたん が とれた	
\\	シャツのボタンが
\\	書き取る	書[か]き 取[と]る	かきとる	
\\	話しのポイントを書き取った。	話[はな]しのポイントを 書[か]き 取[と]った。	はなし の ぽいんと を かきとった	
\\	話[はな]しのポイントを
\\	届く	届[とど]く	とどく	
\\	昨日、父から手紙が届いた。	昨日[きのう]、 父[ちち]から 手紙[てがみ]が 届[とど]いた。	きのう ちち から てがみ が とどいた	
\\	昨日[きのう]、 父[ちち]から 手紙[てがみ]が
\\	スキー	スキー	スキー	
\\	冬はよくスキーに行きます。	冬[ふゆ]はよくスキーに 行[い]きます。	ふゆ は よく すきー に いきます	
\\	冬[ふゆ]はよく
\\	に 行[い]きます。			
\\	届ける	届[とど]ける	とどける	
\\	これを彼に届けてください。	これを 彼[かれ]に 届[とど]けてください。	これ を かれ に とどけて ください	
\\	これを 彼[かれ]に
\\	ください。			
\\	持つ	持[も]つ	もつ	
\\	この車はよく持っているね。	この 車[くるま]はよく 持[も]っているね。	この くるま は よく もって いる ね	
\\	この 車[くるま]はよく
\\	ね。			
\\	金持ち	金持[かねも]ち	かねもち	
\\	彼は金持ちです。	彼[かれ]は 金持[かねも]ちです。	かれ は かねもち です	
\\	彼[かれ]は
\\	です。			
\\	持って行く	持[も]って 行[い]く	もっていく	
\\	水を持って行きましょう。	水[みず]を 持[も]って 行[い]きましょう。	みず を もっていきましょう	
\\	水[みず]を
\\	なかなか	なかなか	なかなか	
\\	荷物がなかなか届きません。	荷物[にもつ]がなかなか 届[とど]きません。	にもつ が なかなか とどきません	
\\	荷物[にもつ]が
\\	届[とど]きません。			
\\	持って来る	持[も]って 来[く]る	もってくる	
\\	そのいすを持って来てください。	そのいすを 持[も]って 来[き]てください。	その いす を もって きて ください	
\\	そのいすを
\\	ください。			
\\	打つ	打[う]つ	うつ	
\\	転んでひざを打ちました。	転[ころ]んでひざを 打[う]ちました。	ころんで ひざ を うちました	
\\	転[ころ]んでひざを
\\	投げる	投[な]げる	なげる	
\\	ボールをこっちに投げてください。	ボールをこっちに 投[な]げてください。	ぼーる を こっち に なげて ください	
\\	ボールをこっちに
\\	ください。			
\\	生まれる	生[う]まれる	うまれる	
\\	姉夫婦に男の子が生まれました。	姉夫婦[あね ふうふ]に 男[おとこ]の 子[こ]が 生[う]まれました。	あね ふうふ に おとこのこ が うまれました	
\\	姉夫婦[あね ふうふ]に 男[おとこ]の 子[こ]が
\\	プール	プール	プール	
\\	私は夏休みにプールに行った。	私[わたし]は 夏休[なつやす]みにプールに 行[い]った。	わたし は なつやすみ に ぷーる に いった	
\\	私[わたし]は 夏休[なつやす]みに
\\	に 行[い]った。			
\\	生む	生[う]む	うむ	
\\	うちのネコが子猫を生みました。	うちのネコが 子猫[こねこ]を 生[う]みました。	うち の ねこ が こねこ を うみました	
\\	うちのネコが 子猫[こねこ]を
\\	女性	女性[じょせい]	じょせい	
\\	そのパーティーに女性は何人来ますか。	そのパーティーに 女性[じょせい]は 何人来[なんにん き]ますか。	その ぱーてぃー に じょせい は なんにん きます か	
\\	そのパーティーに
\\	は 何人来[なんにん き]ますか。			
\\	生産	生産[せいさん]	せいさん	
\\	ボルドーはワインの生産で有名だ。	ボルドーはワインの 生産[せいさん]で 有名[ゆうめい]だ。	ぼるどー は わいん の せいさん で ゆうめい だ	
\\	ボルドーはワインの
\\	で 有名[ゆうめい]だ。			
\\	お土産	お 土産[みやげ]	おみやげ	
\\	父はお土産にお菓子を買ってきた。	父[ちち]はお 土産[みやげ]にお 菓子[かし]を 買[か]ってきた。	ちち は おみやげ に おかし を かって きた	
\\	父[ちち]は
\\	にお 菓子[かし]を 買[か]ってきた。			
\\	ホーム	ホーム	ホーム	
\\	もうすぐこのホームに電車が来ます。	もうすぐこのホームに 電車[でんしゃ]が 来[き]ます。	もうすぐ この ほーむ に でんしゃ が きます	
\\	もうすぐこの
\\	に 電車[でんしゃ]が 来[き]ます。			
\\	生活	生活[せいかつ]	せいかつ	
\\	日本での生活は楽しいです。	日本[にほん]での 生活[せいかつ]は 楽[たの]しいです。	にほん で の せいかつ は たのしい です	
\\	日本[にほん]での
\\	は 楽[たの]しいです。			
\\	生徒	生徒[せいと]	せいと	
\\	このクラスの生徒は30人です。	このクラスの 生徒[せいと]は 30人[さんじゅうにん]です。	この くらす の せいと は さんじゅうにん です	
\\	このクラスの
\\	は 30人[さんじゅうにん]です。			
\\	中学	中学[ちゅうがく]	ちゅうがく	
\\	息子は中学に通っています。	息子[むすこ]は 中学[ちゅうがく]に 通[かよ]っています。	むすこ は ちゅうがく に かよって います	
\\	息子[むすこ]は
\\	に 通[かよ]っています。			
\\	入学	入学[にゅうがく]	にゅうがく	
\\	妹は九月にアメリカの大学に入学します。	妹[いもうと]は 九月[くがつ]にアメリカの 大学[だいがく]に 入学[にゅうがく]します。	いもうと は くがつ に あめりか の だいがく に にゅうがく します	
\\	妹[いもうと]は 九月[くがつ]にアメリカの 大学[だいがく]に
\\	します。			
\\	エレベーター	エレベーター	エレベーター	
\\	エレベーターで下に降りましょう。	エレベーターで 下[した]に 降[お]りましょう。	えれべーたー で した に おりましょう	
\\	で 下[した]に 降[お]りましょう。			
\\	中学生	中学生[ちゅうがくせい]	ちゅうがくせい	
\\	息子は中学生です。	息子[むすこ]は 中学生[ちゅうがくせい]です。	むすこ は ちゅうがくせい です	
\\	息子[むすこ]は
\\	です。			
\\	小学生	小学生[しょうがくせい]	しょうがくせい	
\\	うちの息子は来年、小学生になります。	うちの 息子[むすこ]は 来年[らいねん]、 小学生[しょうがくせい]になります。	うち の むすこ は らいねん しょうがくせい に なります	
\\	うちの 息子[むすこ]は 来年[らいねん]、
\\	になります。			
\\	見学	見学[けんがく]	けんがく	
\\	今日、工場の見学に行きました。	今日[きょう]、 工場[こうじょう]の 見学[けんがく]に 行[い]きました。	きょう こうじょう の けんがく に いきました	
\\	今日[きょう]、 工場[こうじょう]の
\\	に 行[い]きました。			
\\	通学	通学[つうがく]	つうがく	
\\	毎朝、通学に1時間かかる。	毎朝[まいあさ]、 通学[つうがく]に 1時間[いちじかん]かかる。	まいあさ つうがく に いちじかん かかる	
\\	毎朝[まいあさ]、
\\	に 1時間[いちじかん]かかる。			
\\	メモ	メモ	メモ	
\\	メモを取ってください。	メモを 取[と]ってください。	めも を とって ください	
\\	を 取[と]ってください。			
\\	高校	高校[こうこう]	こうこう	
\\	妹は高校に通っています。	妹[いもうと]は 高校[こうこう]に 通[かよ]っています。	いもうと は こうこう に かよって います	
\\	妹[いもうと]は
\\	に 通[かよ]っています。			
\\	小学校	小学校[しょうがっこう]	しょうがっこう	
\\	家の近くに小学校があります。	家[いえ]の 近[ちか]くに 小学校[しょうがっこう]があります。	いえ の ちかく に しょうがっこう が あります	
\\	家[いえ]の 近[ちか]くに
\\	があります。			
\\	中学校	中学校[ちゅうがっこう]	ちゅうがっこう	
\\	息子の中学校は家から5分です。	息子[むすこ]の 中学校[ちゅうがっこう]は 家[いえ]から 5分[ごふん]です。	むすこ の ちゅうがっこう は いえ から ごふん です	
\\	息子[むすこ]の
\\	は 家[いえ]から 5分[ごふん]です。			
\\	校長	校長[こうちょう]	こうちょう	
\\	あの人は高校の校長だ。	あの 人[ひと]は 高校[こうこう]の 校長[こうちょう]だ。	あの ひと は こうこう の こうちょう だ	
\\	あの 人[ひと]は 高校[こうこう]の
\\	だ。			
\\	パン	パン	パン	
\\	朝ご飯にはいつもパンを食べる。	朝[あさ]ご 飯[はん]にはいつもパンを 食[た]べる。	あさごはん に は いつも ぱん を たべる	
\\	朝[あさ]ご 飯[はん]にはいつも
\\	を 食[た]べる。			
\\	休校	休校[きゅうこう]	きゅうこう	
\\	学校は今週は休校です。	学校[がっこう]は 今週[こんしゅう]は 休校[きゅうこう]です。	がっこう は こんしゅう は きゅうこう です	
\\	学校[がっこう]は 今週[こんしゅう]は
\\	です。			
\\	教会	教会[きょうかい]	きょうかい	
\\	私たちは教会で結婚式をしました。	私[わたし]たちは 教会[きょうかい]で 結婚式[けっこんしき]をしました。	わたしたち は きょうかい で けっこんしき を しました	
\\	私[わたし]たちは
\\	で 結婚式[けっこんしき]をしました。			
\\	教育	教育[きょういく]	きょういく	
\\	彼は海外で教育を受けました。	彼[かれ]は 海外[かいがい]で 教育[きょういく]を 受[う]けました。	かれ は かいがい で きょういく を うけました	
\\	彼[かれ]は 海外[かいがい]で
\\	を 受[う]けました。			
\\	育てる	育[そだ]てる	そだてる	
\\	彼女は三人の子を育てました。	彼女[かのじょ]は 三人[さんにん]の 子[こ]を 育[そだ]てました。	かのじょ は さんにん の こ を そだてました	
\\	彼女[かのじょ]は 三人[さんにん]の 子[こ]を
\\	びっくりする	びっくりする	びっくりする	
\\	大きな音にびっくりしました。	大[おお]きな 音[おと]にびっくりしました。	おおきな おと に びっくり しました	
\\	大[おお]きな 音[おと]に
\\	育つ	育[そだ]つ	そだつ	
\\	野菜がよく育っている。	野菜[やさい]がよく 育[そだ]っている。	やさい が よく そだって いる	
\\	野菜[やさい]がよく
\\	制度	制度[せいど]	せいど	
\\	来年から新しい制度が始まります。	来年[らいねん]から 新[あたら]しい 制度[せいど]が 始[はじ]まります。	らいねん から あたらしい せいど が はじまります	
\\	来年[らいねん]から 新[あたら]しい
\\	が 始[はじ]まります。			
\\	強さ	強[つよ]さ	つよさ	
\\	風の強さに驚きました。	風[かぜ]の 強[つよ]さに 驚[おどろ]きました。	かぜ の つよさ に おどろきました	
\\	風[かぜ]の
\\	に 驚[おどろ]きました。			
\\	取引	取引[とりひき]	とりひき	
\\	私たちは中国の会社と取引しています。	私[わたし]たちは 中国[ちゅうごく]の 会社[かいしゃ]と 取引[とりひき]しています。	わたしたち は ちゅうごく の かいしゃ と とりひき して います	
\\	私[わたし]たちは 中国[ちゅうごく]の 会社[かいしゃ]と
\\	しています。			
\\	ズボン	ズボン	ズボン	
\\	ズボンが汚れた。	ズボンが 汚[よご]れた。	ずぼん が よごれた	
\\	が 汚[よご]れた。			
\\	引き出し	引[ひ]き 出[だ]し	ひきだし	
\\	財布は引き出しの中にあります。	財布[さいふ]は 引[ひ]き 出[だ]しの 中[なか]にあります。	さいふ は ひきだし の なか に あります	
\\	財布[さいふ]は
\\	の 中[なか]にあります。			
\\	押さえる	押[お]さえる	おさえる	
\\	ドアを押さえてください。	ドアを 押[お]さえてください。	どあ を おさえて ください	
\\	ドアを
\\	ください。			
\\	押し入れ	押[お]し 入[い]れ	おしいれ	
\\	布団を押し入れにしまいました。	布団[ふとん]を 押[お]し 入[い]れにしまいました。	ふとん を おしいれ に しまいました	
\\	布団[ふとん]を
\\	にしまいました。			
\\	練習	練習[れんしゅう]	れんしゅう	
\\	娘は今、バイオリンを練習しています。	娘[むすめ]は 今[いま]、バイオリンを 練習[れんしゅう]しています。	むすめ は いま ばいおりん を れんしゅう して います	
\\	娘[むすめ]は 今[いま]、バイオリンを
\\	しています。			
\\	おもちゃ	おもちゃ	おもちゃ	
\\	赤ちゃんが自動車のおもちゃで遊んでいる。	赤[あか]ちゃんが 自動車[じどうしゃ]のおもちゃで 遊[あそ]んでいる。	あかちゃん が じどうしゃ の おもちゃ で あそんで いる	
\\	赤[あか]ちゃんが 自動車[じどうしゃ]の
\\	で 遊[あそ]んでいる。			
\\	習う	習[なら]う	ならう	
\\	彼は空手を習っています。	彼[かれ]は 空手[からて]を 習[なら]っています。	かれ は からて を ならって います	
\\	彼[かれ]は 空手[からて]を
\\	慣れる	慣[な]れる	なれる	
\\	新しい家にはもう慣れましたか。	新[あたら]しい 家[いえ]にはもう 慣[な]れましたか。	あたらしい いえ に は もう なれました か	
\\	新[あたら]しい 家[いえ]にはもう
\\	か。			
\\	習慣	習慣[しゅうかん]	しゅうかん	
\\	毎朝コーヒーを飲むのが習慣です。	毎朝[まいあさ]コーヒーを 飲[の]むのが 習慣[しゅうかん]です。	まいあさ こーひー を のむ の が しゅうかん です	
\\	毎朝[まいあさ]コーヒーを 飲[の]むのが
\\	です。			
\\	研究	研究[けんきゅう]	けんきゅう	
\\	彼は何年も地震の研究をしている。	彼[かれ]は 何年[なんねん]も 地震[じしん]の 研究[けんきゅう]をしている。	かれ は なんねん も じしん の けんきゅう を して いる	
\\	彼[かれ]は 何年[なんねん]も 地震[じしん]の
\\	をしている。			
\\	グラム	グラム	グラム	
\\	ひき肉を200グラムください。	ひき 肉[にく]を 200[にひゃく]グラムください。	ひきにく を にひゃくぐらむ ください	
\\	ひき 肉[にく]を 200[にひゃく]
\\	ください。			
\\	試験	試験[しけん]	しけん	
\\	明日の試験、頑張ってね。	明日[あした]の 試験[しけん]、 頑張[がんば]ってね。	あした の しけん がんばって ね	
\\	明日[あした]の
\\	、 頑張[がんば]ってね。			
\\	問題	問題[もんだい]	もんだい	
\\	問題が一つあります。	問題[もんだい]が 一[ひと]つあります。	もんだい が ひとつ あります	
\\	が 一[ひと]つあります。			
\\	簡単	簡単[かんたん]	かんたん	
\\	この料理はとても簡単です。	この 料理[りょうり]はとても 簡単[かんたん]です。	この りょうり は とても かんたん です	
\\	この 料理[りょうり]はとても
\\	です。			
\\	複雑	複雑[ふくざつ]	ふくざつ	
\\	このプログラムはとても複雑です。	このプログラムはとても 複雑[ふくざつ]です。	この ぷろぐらむ は とても ふくざつ です	
\\	このプログラムはとても
\\	です。			
\\	コーヒー	コーヒー	コーヒー	
\\	私は毎朝コーヒーを飲みます。	私[わたし]は 毎朝[まいあさ]コーヒーを 飲[の]みます。	わたし は まいあさ こーひー を のみます	
\\	私[わたし]は 毎朝[まいあさ]
\\	を 飲[の]みます。			
\\	数字	数字[すうじ]	すうじ	
\\	数字は苦手です。	数字[すうじ]は 苦手[にがて]です。	すうじ は にがて です	
\\	は 苦手[にがて]です。			
\\	数学	数学[すうがく]	すうがく	
\\	兄は数学の先生です。	兄[あに]は 数学[すうがく]の 先生[せんせい]です。	あに は すうがく の せんせい です	
\\	兄[あに]は
\\	の 先生[せんせい]です。			
\\	数える	数[かぞ]える	かぞえる	
\\	いすの数を数えてください。	いすの 数[かず]を 数[かぞ]えてください。	いす の かず を かぞえて ください	
\\	いすの 数[かず]を
\\	ください。			
\\	今回	今回[こんかい]	こんかい	
\\	まあ今回は許してあげよう。	まあ 今回[こんかい]は 許[ゆる]してあげよう。	まあ こんかい は ゆるして あげよう	
\\	まあ
\\	は 許[ゆる]してあげよう。			
\\	テント	テント	テント	
\\	みんなでテントを張りました。	みんなでテントを 張[は]りました。	みんな で てんと を はりました	
\\	みんなで
\\	を 張[は]りました。			
\\	回る	回[まわ]る	まわる	
\\	月は地球のまわりを回っています。	月[つき]は 地球[ちきゅう]のまわりを 回[まわ]っています。	つき は ちきゅう の まわり を まわって います	
\\	月[つき]は 地球[ちきゅう]のまわりを
\\	回す	回[まわ]す	まわす	
\\	ねじは左に回すと外れます。	ねじは 左[ひだり]に 回[まわ]すと 外[はず]れます。	ねじ は ひだり に まわす と はずれます	
\\	ねじは 左[ひだり]に
\\	と 外[はず]れます。			
\\	個人	個人[こじん]	こじん	
\\	これは私個人の意見です。	これは 私[わたし] 個人[こじん]の 意見[いけん]です。	これ は わたし こじん の いけん です	
\\	これは 私[わたし]
\\	の 意見[いけん]です。			
\\	担当	担当[たんとう]	たんとう	
\\	私はセールスを担当しています。	私[わたし]はセールスを 担当[たんとう]しています。	わたし は せーるす を たんとう して います	
\\	私[わたし]はセールスを
\\	しています。			
\\	ボート	ボート	ボート	
\\	池でボートに乗りました。	池[いけ]でボートに 乗[の]りました。	いけ で ぼーと に のりました	
\\	池[いけ]で
\\	に 乗[の]りました。			
\\	当たる	当[あ]たる	あたる	
\\	ボールが彼の頭に当たった。	ボールが 彼[かれ]の 頭[あたま]に 当[あ]たった。	ぼーる が かれ の あたま に あたった	
\\	ボールが 彼[かれ]の 頭[あたま]に
\\	当時	当時[とうじ]	とうじ	
\\	彼女は当時、まだ3才だった。	彼女[かのじょ]は 当時[とうじ]、まだ 3才[さんさい]だった。	かのじょ は とうじ まだ さんさい だった	
\\	彼女[かのじょ]は
\\	、まだ 3才[さんさい]だった。			
\\	本当	本当[ほんとう]	ほんとう	
\\	その話は本当ですか。	その 話[はなし]は 本当[ほんとう]ですか。	その はなし は ほんとう です か	
\\	その 話[はなし]は
\\	ですか。			
\\	当然	当然[とうぜん]	とうぜん	
\\	彼女が怒るのも当然だ。	彼女[かのじょ]が 怒[おこ]るのも 当然[とうぜん]だ。	かのじょ が おこる の も とうぜん だ	
\\	彼女[かのじょ]が 怒[おこ]るのも
\\	だ。			
\\	ボール	ボール	ボール	
\\	彼はボールを投げた。	彼[かれ]はボールを 投[な]げた。	かれ は ぼーる を なげた	
\\	彼[かれ]は
\\	を 投[な]げた。			
\\	全然	全然[ぜんぜん]	ぜんぜん	
\\	この本は全然面白くなかった。	この 本[ほん]は 全然[ぜんぜん] 面白[おもしろ]くなかった。	この ほん は ぜんぜん おもしろく なかった	
\\	この 本[ほん]は
\\	面白[おもしろ]くなかった。			
\\	方法	方法[ほうほう]	ほうほう	
\\	いい方法を思いつきました。	いい 方法[ほうほう]を 思[おも]いつきました。	いい ほうほう を おもいつきました	
\\	いい
\\	を 思[おも]いつきました。			
\\	法律	法律[ほうりつ]	ほうりつ	
\\	新しい法律ができた。	新[あたら]しい 法律[ほうりつ]ができた。	あたらしい ほうりつ が できた	
\\	新[あたら]しい
\\	ができた。			
\\	規則	規則[きそく]	きそく	
\\	あの会社の規則は厳しいです。	あの 会社[かいしゃ]の 規則[きそく]は 厳[きび]しいです。	あの かいしゃ の きそく は きびしい です	
\\	あの 会社[かいしゃ]の
\\	は 厳[きび]しいです。			
\\	オートバイ	オートバイ	オートバイ	
\\	彼はオートバイに乗っている。	彼[かれ]はオートバイに 乗[の]っている。	かれ は おーとばい に のって いる	
\\	彼[かれ]は
\\	に 乗[の]っている。			
\\	経験	経験[けいけん]	けいけん	
\\	今日の試合はいい経験になりました。	今日[きょう]の 試合[しあい]はいい 経験[けいけん]になりました。	きょう の しあい は いい けいけん に なりました	
\\	今日[きょう]の 試合[しあい]はいい
\\	になりました。			
\\	経つ	経[た]つ	たつ	
\\	あれから14年が経ちました。	あれから 14年[じゅうよねん]が 経[た]ちました。	あれ から じゅうよねん が たちました	
\\	あれから 14年[じゅうよねん]が
\\	経済	経済[けいざい]	けいざい	
\\	大学で経済を勉強しました。	大学[だいがく]で 経済[けいざい]を 勉強[べんきょう]しました。	だいがく で けいざい を べんきょう しました	
\\	大学[だいがく]で
\\	を 勉強[べんきょう]しました。			
\\	経営	経営[けいえい]	けいえい	
\\	我が社の経営はうまくいっています。	我[わ]が 社[しゃ]の 経営[けいえい]はうまくいっています。	わがしゃ の けいえい は うまく いって います	
\\	我[わ]が 社[しゃ]の
\\	はうまくいっています。			
\\	ひどい	ひどい	ひどい	
\\	妹とひどい喧嘩をした。	妹[いもうと]とひどい 喧嘩[けんか]をした。	いもうと と ひどい けんか を した	
\\	妹[いもうと]と
\\	喧嘩[けんか]をした。			
\\	株	株[かぶ]	かぶ	
\\	最近株を始めました。	最近[さいきん] 株[かぶ]を 始[はじ]めました。	さいきん かぶ を はじめました	
\\	最近[さいきん]
\\	を 始[はじ]めました。			
\\	企業	企業[きぎょう]	きぎょう	
\\	彼女はアメリカの企業で働いています。	彼女[かのじょ]はアメリカの 企業[きぎょう]で 働[はたら]いています。	かのじょ は あめりか の きぎょう で はたらいて います	
\\	彼女[かのじょ]はアメリカの
\\	で 働[はたら]いています。			
\\	作業	作業[さぎょう]	さぎょう	
\\	作業するにはもっと広いスペースが必要だ。	作業[さぎょう]するにはもっと 広[ひろ]いスペースが 必要[ひつよう]だ。	さぎょう する に は もっと ひろい すぺーす が ひつよう だ	
\\	するにはもっと 広[ひろ]いスペースが 必要[ひつよう]だ。			
\\	産業	産業[さんぎょう]	さんぎょう	
\\	日本の自動車産業は世界的に有名だ。	日本[にほん]の 自動車[じどうしゃ] 産業[さんぎょう]は 世界的[せかいてき]に 有名[ゆうめい]だ。	にほん の じどうしゃ さんぎょう は せかいてき に ゆうめい だ	
\\	日本[にほん]の 自動車[じどうしゃ]
\\	は 世界的[せかいてき]に 有名[ゆうめい]だ。			
\\	あなた	あなた	あなた	
\\	この本、あなたにあげます。	この 本[ほん]、あなたにあげます。	この ほん あなた に あげます	
\\	この 本[ほん]、
\\	にあげます。			
\\	工業	工業[こうぎょう]	こうぎょう	
\\	そこは工業都市だ。	そこは 工業[こうぎょう] 都市[とし]だ。	そこ は こうぎょう とし だ	
\\	そこは
\\	都市[とし]だ。			
\\	商業	商業[しょうぎょう]	しょうぎょう	
\\	この町では商業が盛んだ。	この 町[まち]では 商業[しょうぎょう]が 盛[さか]んだ。	この まち で は しょうぎょう が さかん だ	
\\	この 町[まち]では
\\	が 盛[さか]んだ。			
\\	利用	利用[りよう]	りよう	
\\	私はよく図書館を利用します。	私[わたし]はよく 図書館[としょかん]を 利用[りよう]します。	わたし は よく としょかん を りよう します	
\\	私[わたし]はよく 図書館[としょかん]を
\\	します。			
\\	便利	便利[べんり]	べんり	
\\	インターネットはとても便利です。	インターネットはとても 便利[べんり]です。	いんたーねっと は とても べんり です	
\\	インターネットはとても
\\	です。			
\\	スイッチ	スイッチ	スイッチ	
\\	彼はカーラジオのスイッチを入れた。	彼[かれ]はカーラジオのスイッチを 入[い]れた。	かれ は かーらじお の すいっち を いれた	
\\	彼[かれ]はカーラジオの
\\	を 入[い]れた。			
\\	技術	技術[ぎじゅつ]	ぎじゅつ	
\\	彼は非常に高い技術を持っている。	彼[かれ]は 非常[ひじょう]に 高[たか]い 技術[ぎじゅつ]を 持[も]っている。	かれ は ひじょう に たかい ぎじゅつ を もって いる	
\\	彼[かれ]は 非常[ひじょう]に 高[たか]い
\\	を 持[も]っている。			
\\	手術	手術[しゅじゅつ]	しゅじゅつ	
\\	父は胸の手術をした。	父[ちち]は 胸[むね]の 手術[しゅじゅつ]をした。	ちち は むね の しゅじゅつ を した	
\\	父[ちち]は 胸[むね]の
\\	をした。			
\\	製造	製造[せいぞう]	せいぞう	
\\	この工場ではエアコンを製造しています。	この 工場[こうじょう]ではエアコンを 製造[せいぞう]しています。	この こうじょう で は えあこん を せいぞう して います	
\\	この 工場[こうじょう]ではエアコンを
\\	しています。			
\\	必ず	必[かなら]ず	かならず	
\\	必ずシートベルトを着けて下さい。	必[かなら]ずシートベルトを 着[つ]けて 下[くだ]さい。	かならず しーとべると を つけて ください	
\\	シートベルトを 着[つ]けて 下[くだ]さい。			
\\	プレゼント	プレゼント	プレゼント	
\\	誕生日にプレゼントをもらいました。	誕生日[たんじょうび]にプレゼントをもらいました。	たんじょうび に ぷれぜんと を もらいました	
\\	誕生日[たんじょうび]に
\\	をもらいました。			
\\	必要	必要[ひつよう]	ひつよう	
\\	私にはたくさんのお金が必要だ。	私[わたし]にはたくさんのお 金[かね]が 必要[ひつよう]だ。	わたし に は たくさん の おかね が ひつよう だ	
\\	私[わたし]にはたくさんのお 金[かね]が
\\	だ。			
\\	重要	重要[じゅうよう]	じゅうよう	
\\	これは重要な書類です。	これは 重要[じゅうよう]な 書類[しょるい]です。	これ は じゅうよう な しょるい です	
\\	これは
\\	な 書類[しょるい]です。			
\\	要求	要求[ようきゅう]	ようきゅう	
\\	彼は私の要求にこたえた。	彼[かれ]は 私[わたし]の 要求[ようきゅう]にこたえた。	かれ は わたし の ようきゅう に こたえた	
\\	彼[かれ]は 私[わたし]の
\\	にこたえた。			
\\	目覚まし時計	目覚[めざ]まし 時計[どけい]	めざましどけい	
\\	7時に目覚まし時計が鳴りました。	7時[しちじ]に 目覚[めざ]まし 時計[どけい]が 鳴[な]りました。	しちじ に めざましどけい が なりました	
\\	7時[しちじ]に
\\	が 鳴[な]りました。			
\\	いつでも	いつでも	いつでも	
\\	いつでもうちに来てください。	いつでもうちに 来[き]てください。	いつでも うち に きて ください	
\\	うちに 来[き]てください。			
\\	計算	計算[けいさん]	けいさん	
\\	その計算は間違っている。	その 計算[けいさん]は 間違[まちが]っている。	その けいさん は まちがって いる	
\\	その
\\	は 間違[まちが]っている。			
\\	引き算	引[ひ]き 算[ざん]	ひきざん	
\\	娘は学校で引き算を習っている。	娘[むすめ]は 学校[がっこう]で 引[ひ]き 算[ざん]を 習[なら]っている。	むすめ は がっこう で ひきざん を ならって いる	
\\	娘[むすめ]は 学校[がっこう]で
\\	を 習[なら]っている。			
\\	足し算	足[た]し 算[ざん]	たしざん	
\\	娘は学校で足し算を習った。	娘[むすめ]は 学校[がっこう]で 足[た]し 算[ざん]を 習[なら]った。	むすめ は がっこう で たしざん を ならった	
\\	娘[むすめ]は 学校[がっこう]で
\\	を 習[なら]った。			
\\	交差点	交差点[こうさてん]	こうさてん	
\\	次の交差点を左に曲がってください。	次[つぎ]の 交差点[こうさてん]を 左[ひだり]に 曲[ま]がってください。	つぎ の こうさてん を ひだり に まがって ください	
\\	次[つぎ]の
\\	を 左[ひだり]に 曲[ま]がってください。			
\\	テニス	テニス	テニス	
\\	彼らはよくテニスをしています。	彼[かれ]らはよくテニスをしています。	かれら は よく てにす を して います	
\\	彼[かれ]らはよく
\\	をしています。			
\\	割る	割[わ]る	わる	
\\	皿を落として割った。	皿[さら]を 落[お]として 割[わ]った。	さら を おとして わった	
\\	皿[さら]を 落[お]として
\\	割れる	割[わ]れる	われる	
\\	コップが落ちて割れた。	コップが 落[お]ちて 割[わ]れた。	こっぷ が おちて われた	
\\	コップが 落[お]ちて
\\	割り算	割[わ]り 算[ざん]	わりざん	
\\	割り算は小学校で習います。	割[わ]り 算[ざん]は 小学校[しょうがっこう]で 習[なら]います。	わりざん は しょうがっこう で ならいます	
\\	は 小学校[しょうがっこう]で 習[なら]います。			
\\	残る	残[のこ]る	のこる	
\\	料理がたくさん残りました。	料理[りょうり]がたくさん 残[のこ]りました。	りょうり が たくさん のこりました	
\\	料理[りょうり]がたくさん
\\	こちら	こちら	こちら	
\\	受付はこちらです。	受付[うけつけ]はこちらです。	うけつけ は こちら です	
\\	受付[うけつけ]は
\\	です。			
\\	残す	残[のこ]す	のこす	
\\	彼女はメッセージを残しました。	彼女[かのじょ]はメッセージを 残[のこ]しました。	かのじょ は めっせーじ を のこしました	
\\	彼女[かのじょ]はメッセージを
\\	返す	返[かえ]す	かえす	
\\	図書館に本を返した。	図書館[としょかん]に 本[ほん]を 返[かえ]した。	としょかん に ほん を かえした	
\\	図書館[としょかん]に 本[ほん]を
\\	返事	返事[へんじ]	へんじ	
\\	手紙の返事を出しました。	手紙[てがみ]の 返事[へんじ]を 出[だ]しました。	てがみ の へんじ を だしました	
\\	手紙[てがみ]の
\\	を 出[だ]しました。			
\\	借りる	借[か]りる	かりる	
\\	彼にビデオを借りました。	彼[かれ]にビデオを 借[か]りました。	かれ に びでお を かりました	
\\	彼[かれ]にビデオを
\\	ボタン	ボタン	ボタン	
\\	ボタンを押してください。	ボタンを 押[お]してください。	ぼたん を おして ください	
\\	を 押[お]してください。			
\\	貸す	貸[か]す	かす	
\\	私は彼に本を貸しています。	私[わたし]は 彼[かれ]に 本[ほん]を 貸[か]しています。	わたし は かれ に ほん を かして います	
\\	私[わたし]は 彼[かれ]に 本[ほん]を
\\	貸し出す	貸[か]し 出[だ]す	かしだす	
\\	その本は貸し出し中です。	その 本[ほん]は 貸[か]し 出[だ]し 中[ちゅう]です。	その ほん は かしだしちゅう です	
\\	その 本[ほん]は
\\	中[ちゅう]です。			
\\	申し込む	申[もう]し 込[こ]む	もうしこむ	
\\	彼女はそのセミナーに申し込んだ。	彼女[かのじょ]はそのセミナーに 申[もう]し 込[こ]んだ。	かのじょ は その せみなー に もうしこんだ	
\\	彼女[かのじょ]はそのセミナーに
\\	期待	期待[きたい]	きたい	
\\	みんな私たちに期待しています。	みんな 私[わたし]たちに 期待[きたい]しています。	みんな わたしたち に きたい して います	
\\	みんな 私[わたし]たちに
\\	しています。			
\\	おかしい	おかしい	おかしい	
\\	彼の話はおかしかった。	彼[かれ]の 話[はなし]はおかしかった。	かれ の はなし は おかしかった	
\\	彼[かれ]の 話[はなし]は
\\	期間	期間[きかん]	きかん	
\\	テスト期間は10日から15日までだ。	テスト 期間[きかん]は 10日[とおか]から 15日[じゅうごにち]までだ。	てすと きかん は とおか から じゅうごにち まで だ	
\\	テスト
\\	は 10日[とおか]から 15日[じゅうごにち]までだ。			
\\	時期	時期[じき]	じき	
\\	今はあなたにとって大事な時期です。	今[いま]はあなたにとって 大事[だいじ]な 時期[じき]です。	いま は あなた に とって だいじ な じき です	
\\	今[いま]はあなたにとって 大事[だいじ]な
\\	です。			
\\	限る	限[かぎ]る	かぎる	
\\	このサービスは週末に限ります	このサービスは 週末[しゅうまつ]に 限[かぎ]ります	この さーびす は しゅうまつ に かぎります	
\\	このサービスは 週末[しゅうまつ]に
\\	急ぐ	急[いそ]ぐ	いそぐ	
\\	私たちは駅へ急ぎました。	私[わたし]たちは 駅[えき]へ 急[いそ]ぎました。	わたしたち は えき へ いそぎました	
\\	私[わたし]たちは 駅[えき]へ
\\	コート	コート	コート	
\\	新しいコートでテニスをしました。	新[あたら]しいコートでテニスをしました。	あたらしい こーと で てにす を しました	
\\	新[あたら]しい
\\	でテニスをしました。			
\\	急に	急[きゅう]に	きゅうに	
\\	急に用事を思い出した。	急[きゅう]に 用事[ようじ]を 思[おも]い 出[だ]した。	きゅうに ようじ を おもいだした	
\\	用事[ようじ]を 思[おも]い 出[だ]した。			
\\	急	急[きゅう]	きゅう	
\\	急な坂道を上った。	急[きゅう]な 坂道[さかみち]を 上[のぼ]った。	きゅう な さかみち を のぼった	
\\	な 坂道[さかみち]を 上[のぼ]った。			
\\	急行	急行[きゅうこう]	きゅうこう	
\\	ちょうど急行電車が来た。	ちょうど 急行[きゅうこう] 電車[でんしゃ]が 来[き]た。	ちょうど きゅうこう でんしゃ が きた	
\\	ちょうど
\\	電車[でんしゃ]が 来[き]た。			
\\	切れる	切[き]れる	きれる	
\\	このはさみはよく切れますね。	このはさみはよく 切[き]れますね。	この はさみ は よく きれます ね	
\\	このはさみはよく
\\	ね。			
\\	ドラマ	ドラマ	ドラマ	
\\	私はドラマを見るのが好きです。	私[わたし]はドラマを 見[み]るのが 好[す]きです。	わたし は どらま を みる の が すき です	
\\	私[わたし]は
\\	を 見[み]るのが 好[す]きです。			
\\	切手	切手[きって]	きって	
\\	郵便局で切手を買いました。	郵便局[ゆうびんきょく]で 切手[きって]を 買[か]いました。	ゆうびんきょく で きって を かいました	
\\	郵便局[ゆうびんきょく]で
\\	を 買[か]いました。			
\\	売り切れる	売[う]り 切[き]れる	うりきれる	
\\	その本は直ぐ売り切れた。	その 本[ほん]は 直[す]ぐ 売[う]り 切[き]れた。	その ほん は すぐ うりきれた	
\\	その 本[ほん]は 直[す]ぐ
\\	売り切れ	売[う]り 切[き]れ	うりきれ	
\\	チケットはもう売り切れだって。	チケットはもう 売[う]り 切[き]れだって。	ちけっと は もう うりきれ だって	
\\	チケットはもう
\\	だって。			
\\	大切	大切[たいせつ]	たいせつ	
\\	これは母が大切にしていた指輪です。	これは 母[はは]が 大切[たいせつ]にしていた 指輪[ゆびわ]です。	これ は はは が たいせつ に して いた ゆびわ です	
\\	これは 母[はは]が
\\	にしていた 指輪[ゆびわ]です。			
\\	ビザ	ビザ	ビザ	
\\	学生ビザを持っています。	学生[がくせい]ビザを 持[も]っています。	がくせい びざ を もって います	
\\	学生[がくせい]
\\	を 持[も]っています。			
\\	切符	切符[きっぷ]	きっぷ	
\\	東京までの切符を買った。	東京[とうきょう]までの 切符[きっぷ]を 買[か]った。	とうきょう まで の きっぷ を かった	
\\	東京[とうきょう]までの
\\	を 買[か]った。			
\\	入場券	入場券[にゅうじょうけん]	にゅうじょうけん	
\\	入場券は3000円です。	入場券[にゅうじょうけん]は 3000円[さんぜんえん]です。	にゅうじょうけん は さんぜんえん です	
\\	は 3000円[さんぜんえん]です。			
\\	家賃	家賃[やちん]	やちん	
\\	ここの家賃は12万円です。	ここの 家賃[やちん]は 12万円[じゅうにまんえん]です。	ここ の やちん は じゅうにまんえん です	
\\	ここの
\\	は 12万円[じゅうにまんえん]です。			
\\	時代	時代[じだい]	じだい	
\\	今は便利さとスピードの時代だ。	今[いま]は 便利[べんり]さとスピードの 時代[じだい]だ。	いま は べんりさ と すぴーど の じだい だ	
\\	今[いま]は 便利[べんり]さとスピードの
\\	だ。			
\\	ポケット	ポケット	ポケット	
\\	財布をポケットにしまった。	財布[さいふ]をポケットにしまった。	さいふ を ぽけっと に しまった	
\\	財布[さいふ]を
\\	にしまった。			
\\	代わる	代[か]わる	かわる	
\\	上司に代わって会議に出た。	上司[じょうし]に 代[か]わって 会議[かいぎ]に 出[で]た。	じょうし に かわって かいぎ に でた	
\\	上司[じょうし]に
\\	会議[かいぎ]に 出[で]た。			
\\	代える	代[か]える	かえる	
\\	社長に代えて部長を出席させます。	社長[しゃちょう]に 代[か]えて 部長[ぶちょう]を 出席[しゅっせき]させます。	しゃちょう に かえて ぶちょう を しゅっせき させます	
\\	社長[しゃちょう]に
\\	部長[ぶちょう]を 出席[しゅっせき]させます。			
\\	指	指[ゆび]	ゆび	
\\	彼は指が太い。	彼[かれ]は 指[ゆび]が 太[ふと]い。	かれ は ゆび が ふとい	
\\	彼[かれ]は
\\	が 太[ふと]い。			
\\	決定	決定[けってい]	けってい	
\\	会議で重要な決定がありました。	会議[かいぎ]で 重要[じゅうよう]な 決定[けってい]がありました。	かいぎ で じゅうよう な けってい が ありました	
\\	会議[かいぎ]で 重要[じゅうよう]な
\\	がありました。			
\\	そろそろ	そろそろ	そろそろ	
\\	そろそろ始めましょうか。	そろそろ 始[はじ]めましょうか。	そろそろ はじめましょう か	
\\	始[はじ]めましょうか。			
\\	一定	一定[いってい]	いってい	
\\	申し込むには一定の資格が要る。	申[もう]し 込[こ]むには 一定[いってい]の 資格[しかく]が 要[い]る。	もうしこむ に は いってい の しかく が いる	
\\	申[もう]し 込[こ]むには
\\	の 資格[しかく]が 要[い]る。			
\\	定期券	定期券[ていきけん]	ていきけん	
\\	定期券は1万2千円でした。	定期券[ていきけん]は 1万2千円[いちまんにせんえん]でした。	ていきけん は いちまんにせんえん でした	
\\	は 1万2千円[いちまんにせんえん]でした。			
\\	予定	予定[よてい]	よてい	
\\	今日の予定を教えてください。	今日[きょう]の 予定[よてい]を 教[おし]えてください。	きょう の よてい を おしえて ください	
\\	今日[きょう]の
\\	を 教[おし]えてください。			
\\	天気予報	天気予報[てんきよほう]	てんきよほう	
\\	明日の天気予報は雨です。	明日[あした]の 天気予報[てんきよほう]は 雨[あめ]です。	あした の てんきよほう は あめ です	
\\	明日[あした]の
\\	は 雨[あめ]です。			
\\	ぶどう	ぶどう	ぶどう	
\\	私はぶどうが好きです。	私[わたし]はぶどうが 好[す]きです。	わたし は ぶどう が すき です	
\\	私[わたし]は
\\	が 好[す]きです。			
\\	予習	予習[よしゅう]	よしゅう	
\\	明日の予習をしましょう。	明日[あした]の 予習[よしゅう]をしましょう。	あした の よしゅう を しましょう	
\\	明日[あした]の
\\	をしましょう。			
\\	予約	予約[よやく]	よやく	
\\	レストランを予約しました。	レストランを 予約[よやく]しました。	れすとらん を よやく しました	
\\	レストランを
\\	しました。			
\\	約束	約束[やくそく]	やくそく	
\\	約束は守ります。	約束[やくそく]は 守[まも]ります。	やくそく は まもります	
\\	は 守[まも]ります。			
\\	変わる	変[か]わる	かわる	
\\	信号が青に変わりました。	信号[しんごう]が 青[あお]に 変[か]わりました。	しんごう が あお に かわりました。	
\\	信号[しんごう]が 青[あお]に
\\	めったに	めったに	めったに	
\\	彼女はめったに怒りません。	彼女[かのじょ]はめったに 怒[おこ]りません。	かのじょ は めったに おこりません 。	
\\	彼女[かのじょ]は
\\	怒[おこ]りません。			
\\	大変	大変[たいへん]	たいへん	
\\	大変なことが起こりました。	大変[たいへん]なことが 起[お]こりました。	たいへん な こと が おこりました	
\\	なことが 起[お]こりました。			
\\	変	変[へん]	へん	
\\	変な音が聞こえます。	変[へん]な 音[おと]が 聞[き]こえます。	へん な おと が きこえます	
\\	な 音[おと]が 聞[き]こえます。			
\\	変化	変化[へんか]	へんか	
\\	今年は変化の多い年でした。	今年[ことし]は 変化[へんか]の 多[おお]い 年[とし]でした。	ことし は へんか の おおい とし でした	
\\	今年[ことし]は
\\	の 多[おお]い 年[とし]でした。			
\\	強化	強化[きょうか]	きょうか	
\\	国は国語教育を強化しています。	国[くに]は 国語教育[こくご きょういく]を 強化[きょうか]しています。	くに は こくご きょういく を きょうか して います	
\\	国[くに]は 国語教育[こくご きょういく]を
\\	しています。			
\\	クリスマス	クリスマス	クリスマス	
\\	クリスマスにはケーキを食べます。	クリスマスにはケーキを 食[た]べます。	くりすます に は けーき を たべます	
\\	にはケーキを 食[た]べます。			
\\	文化	文化[ぶんか]	ぶんか	
\\	私はこの国の文化を勉強しています。	私[わたし]はこの 国[くに]の 文化[ぶんか]を 勉強[べんきょう]しています。	わたし は この くに の ぶんか を べんきょう して います	
\\	私[わたし]はこの 国[くに]の
\\	を 勉強[べんきょう]しています。			
\\	増える	増[ふ]える	ふえる	
\\	この町は人口が増えた。	この 町[まち]は 人口[じんこう]が 増[ふ]えた。	この まち は じんこう が ふえた	
\\	この 町[まち]は 人口[じんこう]が
\\	増やす	増[ふ]やす	ふやす	
\\	あの町は緑を増やしています。	あの 町[まち]は 緑[みどり]を 増[ふ]やしています。	あの まち は みどり を ふやして います	
\\	あの 町[まち]は 緑[みどり]を
\\	減る	減[へ]る	へる	
\\	体重がかなり減りました。	体重[たいじゅう]がかなり 減[へ]りました。	たいじゅう が かなり へりました	
\\	体重[たいじゅう]がかなり
\\	ネクタイ	ネクタイ	ネクタイ	
\\	父の日にネクタイをプレゼントした。	父[ちち]の 日[ひ]にネクタイをプレゼントした。	ちちのひ に ねくたい を ぷれぜんと した	
\\	父[ちち]の 日[ひ]に
\\	をプレゼントした。			
\\	減らす	減[へ]らす	へらす	
\\	最近、食事を減らしています。	最近[さいきん]、 食事[しょくじ]を 減[へ]らしています。	さいきん しょくじ を へらして います	
\\	最近[さいきん]、 食事[しょくじ]を
\\	乗り物	乗[の]り 物[もの]	のりもの	
\\	自転車は便利な乗り物です。	自転車[じてんしゃ]は 便利[べんり]な 乗[の]り 物[もの]です。	じてんしゃ は べんり な のりもの です	
\\	自転車[じてんしゃ]は 便利[べんり]な
\\	です。			
\\	降りる	降[お]りる	おりる	
\\	次の駅で降ります。	次[つぎ]の 駅[えき]で 降[お]ります。	つぎ の えき で おります。	
\\	次[つぎ]の 駅[えき]で
\\	降ろす	降[お]ろす	おろす	
\\	彼は車から荷物を降ろした。	彼[かれ]は 車[くるま]から 荷物[にもつ]を 降[お]ろした。	かれ は くるま から にもつ を おろした	
\\	彼[かれ]は 車[くるま]から 荷物[にもつ]を
\\	バイオリン	バイオリン	バイオリン	
\\	彼女はバイオリンを習っています。	彼女[かのじょ]はバイオリンを 習[なら]っています。	かのじょ は ばいおりん を ならって います	
\\	彼女[かのじょ]は
\\	を 習[なら]っています。			
\\	降る	降[ふ]る	ふる	
\\	激しい雨が降っています。	激[はげ]しい 雨[あめ]が 降[ふ]っています。	はげしい あめ が ふって います	
\\	激[はげ]しい 雨[あめ]が
\\	着く	着[つ]く	つく	
\\	午後8時に大阪に着きます。	午後8時[ごご はちじ]に 大阪[おおさか]に 着[つ]きます。	ごご はちじ に おおさか に つきます	
\\	午後8時[ごご はちじ]に 大阪[おおさか]に
\\	着物	着物[きもの]	きもの	
\\	彼女は着物がよく似合います。	彼女[かのじょ]は 着物[きもの]がよく 似合[にあ]います。	かのじょ は きもの が よく にあいます	
\\	彼女[かのじょ]は
\\	がよく 似合[にあ]います。			
\\	下着	下着[したぎ]	したぎ	
\\	私は下着を手で洗う。	私[わたし]は 下着[したぎ]を 手[て]で 洗[あら]う。	わたし は したぎ を て で あらう	
\\	私[わたし]は
\\	を 手[て]で 洗[あら]う。			
\\	パスポート	パスポート	パスポート	
\\	パスポートが見つからない。	パスポートが 見[み]つからない。	ぱすぽーと が みつからない	
\\	が 見[み]つからない。			
\\	上着	上着[うわぎ]	うわぎ	
\\	暑いので上着を脱ぎました。	暑[あつ]いので 上着[うわぎ]を 脱[ぬ]ぎました。	あつい の で うわぎ を ぬぎました	
\\	暑[あつ]いので
\\	を 脱[ぬ]ぎました。			
\\	着せる	着[き]せる	きせる	
\\	娘に可愛いドレスを着せた。	娘[むすめ]に 可愛[かわい]いドレスを 着[き]せた。	むすめ に かわいい どれす を きせた	
\\	娘[むすめ]に 可愛[かわい]いドレスを
\\	脱ぐ	脱[ぬ]ぐ	ぬぐ	
\\	靴を脱いでください。	靴[くつ]を 脱[ぬ]いでください。	くつ を ぬいで ください	
\\	靴[くつ]を
\\	ください。			
\\	立場	立場[たちば]	たちば	
\\	彼女は自分の立場を分かっていない。	彼女[かのじょ]は 自分[じぶん]の 立場[たちば]を 分[わ]かっていない。	かのじょ は じぶん の たちば を わかって いない	
\\	彼女[かのじょ]は 自分[じぶん]の
\\	を 分[わ]かっていない。			
\\	バッグ	バッグ	バッグ	
\\	私は黒いバッグを持っています。	私[わたし]は 黒[くろ]いバッグを 持[も]っています。	わたし は くろい ばっぐ を もって います	
\\	私[わたし]は 黒[くろ]い
\\	を 持[も]っています。			
\\	目立つ	目立[めだ]つ	めだつ	
\\	彼女の大きな帽子はとても目立つ。	彼女[かのじょ]の 大[おお]きな 帽子[ぼうし]はとても 目立[めだ]つ。	かのじょ の おおき な ぼうし は とても めだつ	
\\	彼女[かのじょ]の 大[おお]きな 帽子[ぼうし]はとても
\\	立てる	立[た]てる	たてる	
\\	彼はケーキにろうそくを立てた。	彼[かれ]はケーキにろうそくを 立[た]てた。	かれ は けーき に ろうそく を たてた	
\\	彼[かれ]はケーキにろうそくを
\\	役に立つ	役[やく]に 立[た]つ	やくにたつ	
\\	私は人々の役に立ちたいと思っています。	私[わたし]は 人々[ひとびと]の 役[やく]に 立[た]ちたいと 思[おも]っています。	わたし は ひとびと の やくにたちたい と おもって います	
\\	私[わたし]は 人々[ひとびと]の
\\	と 思[おも]っています。			
\\	出席	出席[しゅっせき]	しゅっせき	
\\	午後は会議に出席します。	午後[ごご]は 会議[かいぎ]に 出席[しゅっせき]します。	ごご は かいぎ に しゅっせき します	
\\	午後[ごご]は 会議[かいぎ]に
\\	します。			
\\	ゴム	ゴム	ゴム	
\\	ゴムが伸びてしまった。	ゴムが 伸[の]びてしまった。	ごむ が のびて しまった	
\\	が 伸[の]びてしまった。			
\\	席	席[せき]	せき	
\\	この席、空いてますか。	この 席[せき]、 空[あ]いてますか。	この せき あいてます か	
\\	この
\\	、 空[あ]いてますか。			
\\	欠席	欠席[けっせき]	けっせき	
\\	風邪のため今日は欠席します。	風邪[かぜ]のため 今日[きょう]は 欠席[けっせき]します。	かぜ の ため きょう は けっせき します	
\\	風邪[かぜ]のため 今日[きょう]は
\\	します。			
\\	次男	次男[じなん]	じなん	
\\	次男は今、海外にいます。	次男[じなん]は 今[いま]、 海外[かいがい]にいます。	じなん は いま かいがい に います	
\\	は 今[いま]、 海外[かいがい]にいます。			
\\	次女	次女[じじょ]	じじょ	
\\	うちの次女は春から中学生です。	うちの 次女[じじょ]は 春[はる]から 中学生[ちゅうがくせい]です。	うち の じじょ は はる から ちゅうがくせい です	
\\	うちの
\\	は 春[はる]から 中学生[ちゅうがくせい]です。			
\\	ギター	ギター	ギター	
\\	彼女はギターが得意です。	彼女[かのじょ]はギターが 得意[とくい]です。	かのじょ は ぎたー が とくい です	
\\	彼女[かのじょ]は
\\	が 得意[とくい]です。			
\\	運ぶ	運[はこ]ぶ	はこぶ	
\\	いすを二階に運んでください。	いすを 二階[にかい]に 運[はこ]んでください。	いす を にかい に はこんで ください	
\\	いすを 二階[にかい]に
\\	ください。			
\\	運転	運転[うんてん]	うんてん	
\\	父は安全運転だ。	父[ちち]は 安全[あんぜん] 運転[うんてん]だ。	ちち は あんぜん うんてん だ	
\\	父[ちち]は 安全[あんぜん]
\\	だ。			
\\	運転手	運転手[うんてんしゅ]	うんてんしゅ	
\\	彼はタクシーの運転手です。	彼[かれ]はタクシーの 運転手[うんてんしゅ]です。	かれ は たくしー の うんてんしゅ です	
\\	彼[かれ]はタクシーの
\\	です。			
\\	転ぶ	転[ころ]ぶ	ころぶ	
\\	お祖母ちゃんがお風呂場で転んだ。	お 祖母[ばあ]ちゃんがお 風呂場[ふろば]で 転[ころ]んだ。	おばあちゃん が おふろば で ころんだ	
\\	お 祖母[ばあ]ちゃんがお 風呂場[ふろば]で
\\	セーター	セーター	セーター	
\\	このセーターはウールだ。	このセーターはウールだ。	この せーたー は うーる だ	
\\	この
\\	はウールだ。			
\\	移る	移[うつ]る	うつる	
\\	彼女は新しい会社に移った。	彼女[かのじょ]は 新[あたら]しい 会社[かいしゃ]に 移[うつ]った。	かのじょ は あたらしい かいしゃ に うつった	
\\	彼女[かのじょ]は 新[あたら]しい 会社[かいしゃ]に
\\	移す	移[うつ]す	うつす	
\\	机を窓の傍に移しました。	机[つくえ]を 窓[まど]の 傍[そば]に 移[うつ]しました。	つくえ を まど の そば に うつしました	
\\	机[つくえ]を 窓[まど]の 傍[そば]に
\\	動き	動[うご]き	うごき	
\\	まず相手の動きを見ましょう。	まず 相手[あいて]の 動[うご]きを 見[み]ましょう。	まず あいて の うごき を みましょう	
\\	まず 相手[あいて]の
\\	を 見[み]ましょう。			
\\	運動	運動[うんどう]	うんどう	
\\	彼はもっと運動した方がいい。	彼[かれ]はもっと 運動[うんどう]した 方[ほう]がいい。	かれ は もっと うんどう した ほう が いい	
\\	彼[かれ]はもっと
\\	した 方[ほう]がいい。			
\\	やっぱり	やっぱり	やっぱり	
\\	やっぱり旅が大好きだ。	やっぱり 旅[たび]が 大好[だいす]きだ。	やっぱり たび が だいすき だ	
\\	旅[たび]が 大好[だいす]きだ。			
\\	活動	活動[かつどう]	かつどう	
\\	彼は地域の活動に参加した。	彼[かれ]は 地域[ちいき]の 活動[かつどう]に 参加[さんか]した。	かれ は ちいき の かつどう に さんか した	
\\	彼[かれ]は 地域[ちいき]の
\\	に 参加[さんか]した。			
\\	動物	動物[どうぶつ]	どうぶつ	
\\	私は動物が大好きです。	私[わたし]は 動物[どうぶつ]が 大好[だいす]きです。	わたし は どうぶつ が だいすき です	
\\	私[わたし]は
\\	が 大好[だいす]きです。			
\\	動かす	動[うご]かす	うごかす	
\\	部屋の家具を動かしました。	部屋[へや]の 家具[かぐ]を 動[うご]かしました。	へや の かぐ を うごかしました	
\\	部屋[へや]の 家具[かぐ]を
\\	不動産屋	不動産屋[ふどうさんや]	ふどうさんや	
\\	私の父は不動産屋です。	私[わたし]の 父[ちち]は 不動産屋[ふどうさんや]です。	わたし の ちち は ふどうさんや です	
\\	私[わたし]の 父[ちち]は
\\	です。			
\\	カーテン	カーテン	カーテン	
\\	朝、カーテンを開けた。	朝[あさ]、カーテンを 開[あ]けた。	あさ かーてん を あけた	
\\	朝[あさ]、
\\	を 開[あ]けた。			
\\	早く	早[はや]く	はやく	
\\	なるべく早く来て下さい。	なるべく 早[はや]く 来[き]て 下[くだ]さい。	なるべく はやく きて ください	
\\	なるべく
\\	来[き]て 下[くだ]さい。			
\\	早口	早口[はやくち]	はやくち	
\\	彼女は早口だ。	彼女[かのじょ]は 早口[はやくち]だ。	かのじょ は はやくち だ	
\\	彼女[かのじょ]は
\\	だ。			
\\	速さ	速[はや]さ	はやさ	
\\	新幹線の速さはどれ位ですか。	新幹線[しんかんせん]の 速[はや]さはどれ 位[くらい]ですか。	しんかんせん の はやさ は どれ くらい です か	
\\	新幹線[しんかんせん]の
\\	はどれ 位[くらい]ですか。			
\\	速達	速達[そくたつ]	そくたつ	
\\	これを速達で送りたいのですが。	これを 速達[そくたつ]で 送[おく]りたいのですが。	これ を そくたつ で おくりたい の です が	
\\	これを
\\	で 送[おく]りたいのですが。			
\\	シャツ	シャツ	シャツ	
\\	このシャツはアイロンが必要だ。	このシャツはアイロンが 必要[ひつよう]だ。	この しゃつ は あいろん が ひつよう だ	
\\	この
\\	はアイロンが 必要[ひつよう]だ。			
\\	遅れる	遅[おく]れる	おくれる	
\\	今朝彼女は学校に遅れました。	今朝彼女[けさ かのじょ]は 学校[がっこう]に 遅[おく]れました。	けさ かのじょ は がっこう に おくれました	
\\	今朝彼女[けさ かのじょ]は 学校[がっこう]に
\\	開始	開始[かいし]	かいし	
\\	運動会は9時開始です。	運動会[うんどうかい]は 9時[くじ] 開始[かいし]です。	うんどうかい は くじ かいし です	
\\	運動会[うんどうかい]は 9時[くじ]
\\	です。			
\\	始めに	始[はじ]めに	はじめに	
\\	始めにスープが出ます。	始[はじ]めにスープが 出[で]ます。	はじめに すーぷ が でます	
\\	スープが 出[で]ます。			
\\	現在	現在[げんざい]	げんざい	
\\	現在の気温は30度です。	現在[げんざい]の 気温[きおん]は 30度[さんじゅうど]です。	げんざい の きおん は さんじゅうど です	
\\	の 気温[きおん]は 30度[さんじゅうど]です。			
\\	ナイフ	ナイフ	ナイフ	
\\	ナイフで手を切った。	ナイフで 手[て]を 切[き]った。	ないふ で て を きった	
\\	で 手[て]を 切[き]った。			
\\	実現	実現[じつげん]	じつげん	
\\	夢を実現するには努力が必要です。	夢[ゆめ]を 実現[じつげん]するには 努力[どりょく]が 必要[ひつよう]です。	ゆめ を じつげん する に は どりょく が ひつよう です	
\\	夢[ゆめ]を
\\	するには 努力[どりょく]が 必要[ひつよう]です。			
\\	実施	実施[じっし]	じっし	
\\	現在、スペシャルキャンペーンを実施中です。	現在[げんざい]、スペシャルキャンペーンを 実施[じっし] 中[ちゅう]です。	げんざい すぺしゃるきゃんぺーん を じっしちゅう です	
\\	現在[げんざい]、スペシャルキャンペーンを
\\	中[ちゅう]です。			
\\	事実	事実[じじつ]	じじつ	
\\	それは全て事実ですか。	それは 全[すべ]て 事実[じじつ]ですか。	それ は すべて じじつ です か	
\\	それは 全[すべ]て
\\	ですか。			
\\	実行	実行[じっこう]	じっこう	
\\	彼はその計画を実行した。	彼[かれ]はその 計画[けいかく]を 実行[じっこう]した。	かれ は その けいかく を じっこう した	
\\	彼[かれ]はその 計画[けいかく]を
\\	した。			
\\	バイク	バイク	バイク	
\\	兄はバイクが大好きです。	兄[あに]はバイクが 大好[だいす]きです。	あに は ばいく が だいすき です	
\\	兄[あに]は
\\	が 大好[だいす]きです。			
\\	実験	実験[じっけん]	じっけん	
\\	科学の授業で実験をした。	科学[かがく]の 授業[じゅぎょう]で 実験[じっけん]をした。	かがく の じゅぎょう で じっけん を した	
\\	科学[かがく]の 授業[じゅぎょう]で
\\	をした。			
\\	通過	通過[つうか]	つうか	
\\	次の駅は通過します。	次[つぎ]の 駅[えき]は 通過[つうか]します。	つぎ の えき は つうか します	
\\	次[つぎ]の 駅[えき]は
\\	します。			
\\	過ぎる	過[す]ぎる	すぎる	
\\	時が過ぎるのは速い。	時[とき]が 過[す]ぎるのは 速[はや]い。	とき が すぎる の は はやい 。	
\\	時[とき]が
\\	のは 速[はや]い。			
\\	昼過ぎ	昼過[ひるす]ぎ	ひるすぎ	
\\	彼は昼過ぎに来ます。	彼[かれ]は 昼過[ひるす]ぎに 来[き]ます。	かれ は ひるすぎ に きます	
\\	彼[かれ]は
\\	に 来[き]ます。			
\\	いつか	いつか	いつか	
\\	僕はいつかアフリカに行きたい。	僕[ぼく]はいつかアフリカに 行[い]きたい。	ぼく は いつか あふりか に いきたい	
\\	僕[ぼく]は
\\	アフリカに 行[い]きたい。			
\\	過去	過去[かこ]	かこ	
\\	それは過去の話だ。	それは 過去[かこ]の 話[はなし]だ。	それ は かこ の はなし だ	
\\	それは
\\	の 話[はなし]だ。			
\\	開発	開発[かいはつ]	かいはつ	
\\	ダムの開発に住民は反対しています。	ダムの 開発[かいはつ]に 住民[じゅうみん]は 反対[はんたい]しています。	だむ の かいはつ に じゅうみん は はんたい して います	
\\	ダムの
\\	に 住民[じゅうみん]は 反対[はんたい]しています。			
\\	発生	発生[はっせい]	はっせい	
\\	交差点で事故が発生した。	交差点[こうさてん]で 事故[じこ]が 発生[はっせい]した。	こうさてん で じこ が はっせい した	
\\	交差点[こうさてん]で 事故[じこ]が
\\	した。			
\\	発言	発言[はつげん]	はつげん	
\\	会議で全員が発言した。	会議[かいぎ]で 全員[ぜんいん]が 発言[はつげん]した。	かいぎ で ぜんいん が はつげん した	
\\	会議[かいぎ]で 全員[ぜんいん]が
\\	した。			
\\	オレンジ	オレンジ	オレンジ	
\\	私はオレンジが好きです。	私[わたし]はオレンジが 好[す]きです。	わたし は おれんじ が すき です	
\\	私[わたし]は
\\	が 好[す]きです。			
\\	発見	発見[はっけん]	はっけん	
\\	新しい星が発見された。	新[あたら]しい 星[ほし]が 発見[はっけん]された。	あたらしい ほし が はっけん された	
\\	新[あたら]しい 星[ほし]が
\\	された。			
\\	出発	出発[しゅっぱつ]	しゅっぱつ	
\\	あと15分で出発です。	あと 15分[じゅうごふん]で 出発[しゅっぱつ]です。	あと じゅうごふん で しゅっぱつ です	
\\	あと 15分[じゅうごふん]で
\\	です。			
\\	発車	発車[はっしゃ]	はっしゃ	
\\	バスが発車します。	バスが 発車[はっしゃ]します。	ばす が はっしゃ します	
\\	バスが
\\	します。			
\\	発表	発表[はっぴょう]	はっぴょう	
\\	合格者が発表された。	合格者[ごうかくしゃ]が 発表[はっぴょう]された。	ごうかくしゃ が はっぴょう された	
\\	合格者[ごうかくしゃ]が
\\	された。			
\\	きつい	きつい	きつい	
\\	このシャツは少しきついです。	このシャツは 少[すこ]しきついです。	この しゃつ は すこし きつい です	
\\	このシャツは 少[すこ]し
\\	です。			
\\	表現	表現[ひょうげん]	ひょうげん	
\\	彼は歌で自分の気持ちを表現した。	彼[かれ]は 歌[うた]で 自分[じぶん]の 気持[きも]ちを 表現[ひょうげん]した。	かれ は うた で じぶん の きもち を ひょうげん した	
\\	彼[かれ]は 歌[うた]で 自分[じぶん]の 気持[きも]ちを
\\	した。			
\\	代表	代表[だいひょう]	だいひょう	
\\	彼がクラスの代表だ。	彼[かれ]がクラスの 代表[だいひょう]だ。	かれ が くらす の だいひょう だ	
\\	彼[かれ]がクラスの
\\	だ。			
\\	表	表[おもて]	おもて	
\\	表に人が来ています。	表[おもて]に 人[ひと]が 来[き]ています。	おもてに ひと が きて います	
\\	に 人[ひと]が 来[き]ています。			
\\	表	表[ひょう]	ひょう	
\\	この表を見てください。	この 表[ひょう]を 見[み]てください。	この ひょう を みて ください	
\\	この
\\	を 見[み]てください。			
\\	さっき	さっき	さっき	
\\	さっきの話を続けましょう。	さっきの 話[はなし]を 続[つづ]けましょう。	さっき の はなし を つづけましょう	
\\	の 話[はなし]を 続[つづ]けましょう。			
\\	手紙	手紙[てがみ]	てがみ	
\\	友人から手紙をもらいました。	友人[ゆうじん]から 手紙[てがみ]をもらいました。	ゆうじん から てがみ を もらいました	
\\	友人[ゆうじん]から
\\	をもらいました。			
\\	絵	絵[え]	え	
\\	これは有名な画家の絵です。	これは 有名[ゆうめい]な 画家[がか]の 絵[え]です。	これ は ゆうめい な がか の え です	
\\	これは 有名[ゆうめい]な 画家[がか]の
\\	です。			
\\	雑誌	雑誌[ざっし]	ざっし	
\\	この雑誌はよく売れています。	この 雑誌[ざっし]はよく 売[う]れています。	この ざっし は よく うれて います	
\\	この
\\	はよく 売[う]れています。			
\\	音	音[おと]	おと	
\\	雨の音が聞こえる。	雨[あめ]の 音[おと]が 聞[き]こえる。	あめ の おと が きこえる	
\\	雨[あめ]の
\\	が 聞[き]こえる。			
\\	テキスト	テキスト	テキスト	
\\	テキストを読んでください。	テキストを 読[よ]んでください。	てきすと を よんで ください	
\\	を 読[よ]んでください。			
\\	音	音[おと]	おと	
\\	このピアノは音があまり良くない。	このピアノは 音[おと]があまり 良[よ]くない。	この ぴあの は おと が あまり よく ない	
\\	このピアノは
\\	があまり 良[よ]くない。			
\\	音楽	音楽[おんがく]	おんがく	
\\	私は音楽を聞くのが好きだ。	私[わたし]は 音楽[おんがく]を 聞[き]くのが 好[す]きだ。	わたし は おんがく を きく の が すき だ	
\\	私[わたし]は
\\	を 聞[き]くのが 好[す]きだ。			
\\	薬	薬[くすり]	くすり	
\\	この薬を必ず飲んでください。	この 薬[くすり]を 必[かなら]ず 飲[の]んでください。	この くすり を かならず のんで ください	
\\	この
\\	を 必[かなら]ず 飲[の]んでください。			
\\	歌	歌[うた]	うた	
\\	私はその歌を知らなかった。	私[わたし]はその 歌[うた]を 知[し]らなかった。	わたし は その うた を しらなかった 。	
\\	私[わたし]はその
\\	を 知[し]らなかった。			
\\	ビニール	ビニール	ビニール	
\\	ゴミはそのビニール袋に入れてください。	ゴミはそのビニール 袋[ぶくろ]に 入[い]れてください。	ごみ は その びにーるぶくろ に いれて ください	
\\	ゴミはその
\\	袋[ぶくろ]に 入[い]れてください。			
\\	歌手	歌手[かしゅ]	かしゅ	
\\	その歌手は歌が下手だ。	その 歌手[かしゅ]は 歌[うた]が 下手[へた]だ。	その かしゅ は うた が へた だ	
\\	その
\\	は 歌[うた]が 下手[へた]だ。			
\\	欲しがる	欲[ほ]しがる	ほしがる	
\\	子供がジュースを欲しがっています。	子供[こども]がジュースを 欲[ほ]しがっています。	こども が じゅーす を ほしがって います	
\\	子供[こども]がジュースを
\\	計画	計画[けいかく]	けいかく	
\\	彼は一人旅の計画を立てた。	彼[かれ]は 一人旅[ひとりたび]の 計画[けいかく]を 立[た]てた。	かれ は ひとりたび の けいかく を たてた	
\\	彼[かれ]は 一人旅[ひとりたび]の
\\	を 立[た]てた。			
\\	映画	映画[えいが]	えいが	
\\	彼はよく映画を見ます。	彼[かれ]はよく 映画[えいが]を 見[み]ます。	かれ は よく えいが を みます	
\\	彼[かれ]はよく
\\	を 見[み]ます。			
\\	りんご	りんご	りんご	
\\	りんごを一つください。	りんごを 一[ひと]つください。	りんご を ひとつ ください	
\\	を 一[ひと]つください。			
\\	面白い	面白[おもしろ]い	おもしろい	
\\	この本は全然面白くなかった。	この 本[ほん]は全然[ぜんぜん] 面白[おもしろ]くなかった。	この ほん は ぜんぜん おもしろく なかった	
\\	この 本[ほん]は 全然[ぜんぜん]
\\	写る	写[うつ]る	うつる	
\\	このカメラはよく写りますよ。	このカメラはよく 写[うつ]りますよ。	この かめら は よく うつります よ	
\\	このカメラはよく
\\	よ。			
\\	写す	写[うつ]す	うつす	
\\	彼は友達の答えを写した。	彼[かれ]は 友達[ともだち]の 答[こた]えを 写[うつ]した。	かれ は ともだち の こたえ を うつした	
\\	彼[かれ]は 友達[ともだち]の 答[こた]えを
\\	写真	写真[しゃしん]	しゃしん	
\\	写真は良い思い出になります。	写真[しゃしん]は 良[い]い 思[おも]い 出[で]になります。	しゃしん は いい おもいで に なります	
\\	は 良[い]い 思[おも]い 出[で]になります。			
\\	アルバム	アルバム	アルバム	
\\	私は彼のアルバムを見た。	私[わたし]は 彼[かれ]のアルバムを 見[み]た。	わたし は かれ の あるばむ を みた	
\\	私[わたし]は 彼[かれ]の
\\	を 見[み]た。			
\\	真っ赤	真[ま]っ 赤[か]	まっか	
\\	彼の顔は真っ赤でした。	彼[かれ]の 顔[かお]は 真[ま]っ 赤[か]でした。	かれ の かお は まっか でした	
\\	彼[かれ]の 顔[かお]は
\\	でした。			
\\	真面目	真面目[まじめ]	まじめ	
\\	彼は真面目な人です。	彼[かれ]は 真面目[まじめ]な 人[ひと]です。	かれ は まじめ な ひと です	
\\	彼[かれ]は
\\	な 人[ひと]です。			
\\	真ん中	真[ま]ん 中[なか]	まんなか	
\\	道の真ん中に人が立っている。	道[みち]の 真[ま]ん 中[なか]に 人[ひと]が 立[た]っている。	みち の まんなか に ひと が たって いる	
\\	道[みち]の
\\	に 人[ひと]が 立[た]っている。			
\\	真っ白	真[ま]っ 白[しろ]	まっしろ	
\\	外は雪で真っ白だった。	外[そと]は 雪[ゆき]で 真[ま]っ 白[しろ]だった。	そと は ゆき で まっしろ だった	
\\	外[そと]は 雪[ゆき]で
\\	だった。			
\\	スカート	スカート	スカート	
\\	彼女はあまりスカートははかない。	彼女[かのじょ]はあまりスカートははかない。	かのじょ は あまり すかーと は はかない	
\\	彼女[かのじょ]はあまり
\\	ははかない。			
\\	真っ暗	真[ま]っ 暗[くら]	まっくら	
\\	外は真っ暗です。	外[そと]は 真[ま]っ 暗[くら]です。	そと は まっくら です	
\\	外[そと]は
\\	です。			
\\	真っ黒	真[ま]っ 黒[くろ]	まっくろ	
\\	インクで手が真っ黒になった	インクで 手[て]が 真[ま]っ 黒[くろ]になった	いんく で て が まっくろ に なった	
\\	インクで 手[て]が
\\	になった			
\\	真っ青	真[ま]っ 青[さお]	まっさお	
\\	空が真っ青です。	空[そら]が 真[ま]っ 青[さお]です。	そら が まっさお です	
\\	空[そら]が
\\	です。			
\\	色々	色々[いろいろ]	いろいろ	
\\	彼は色々なことを知っている。	彼[かれ]は 色々[いろいろ]なことを 知[し]っている。	かれ は いろいろ な こと を しって いる	
\\	彼[かれ]は
\\	なことを 知[し]っている。			
\\	ペン	ペン	ペン	
\\	ペンを貸してください。	ペンを 貸[か]してください。	ぺん を かして ください	
\\	を 貸[か]してください。			
\\	人形	人形[にんぎょう]	にんぎょう	
\\	彼女は人形をたくさん持っています。	彼女[かのじょ]は 人形[にんぎょう]をたくさん 持[も]っています。	かのじょ は にんぎょう を たくさん もって います	
\\	彼女[かのじょ]は
\\	をたくさん 持[も]っています。			
\\	形	形[かたち]	かたち	
\\	その椅子は変わった形をしている。	その 椅子[いす]は 変[か]わった 形[かたち]をしている。	その いす は かわった かたち を して いる	
\\	その 椅子[いす]は 変[か]わった
\\	をしている。			
\\	大型	大型[おおがた]	おおがた	
\\	大型のテレビを買った。	大型[おおがた]のテレビを 買[か]った。	おおがた の てれび を かった	
\\	のテレビを 買[か]った。			
\\	種類	種類[しゅるい]	しゅるい	
\\	バラには色々な種類があります。	バラには 色々[いろいろ]な 種類[しゅるい]があります。	ばら に は いろいろ な しゅるい が あります	
\\	バラには 色々[いろいろ]な
\\	があります。			
\\	おしゃべり	おしゃべり	おしゃべり	
\\	妹はとてもおしゃべりです。	妹[いもうと]はとてもおしゃべりです。	いもうと は とても おしゃべり です	
\\	妹[いもうと]はとても
\\	です。			
\\	直す	直[なお]す	なおす	
\\	私がそれを直しました。	私[わたし]がそれを 直[なお]しました。	わたし が それ を なおしました	
\\	私[わたし]がそれを
\\	真っ直ぐ	真[ま]っ 直[す]ぐ	まっすぐ	
\\	この道を真っ直ぐ行ってください。	この 道[みち]を 真[ま]っ 直[す]ぐ 行[い]ってください。	この みち を まっすぐ いって ください	
\\	この 道[みち]を
\\	行[い]ってください。			
\\	直接	直接[ちょくせつ]	ちょくせつ	
\\	彼に直接お願いしなさい。	彼[かれ]に 直接[ちょくせつ]お 願[ねが]いしなさい。	かれ に ちょくせつ おねがい しなさい	
\\	彼[かれ]に
\\	お 願[ねが]いしなさい。			
\\	角	角[かど]	かど	
\\	次の角で左に曲がってください。	次[つぎ]の 角[かど]で 左[ひだり]に 曲[ま]がってください。	つぎ の かど で ひだり に まがって ください	
\\	次[つぎ]の
\\	で 左[ひだり]に 曲[ま]がってください。			
\\	きっと	きっと	きっと	
\\	明日はきっと雨が降ります。	明日[あした]はきっと 雨[あめ]が 降[ふ]ります。	あした は きっと あめ が ふります	
\\	明日[あした]は
\\	雨[あめ]が 降[ふ]ります。			
\\	三角	三角[さんかく]	さんかく	
\\	紙を三角に切りました。	紙[かみ]を 三角[さんかく]に 切[き]りました。	かみ を さんかく に きりました	
\\	紙[かみ]を
\\	に 切[き]りました。			
\\	四角	四角[しかく]	しかく	
\\	紙を四角に切ってください。	紙[かみ]を 四角[しかく]に 切[き]ってください。	かみ を しかく に きって ください	
\\	紙[かみ]を
\\	に 切[き]ってください。			
\\	四角い	四角[しかく]い	しかくい	
\\	こっちの四角いテーブルを買おうよ。	こっちの 四角[しかく]いテーブルを 買[か]おうよ。	こっち の しかくい てーぶる を かおうよ	
\\	こっちの
\\	テーブルを 買[か]おうよ。			
\\	四つ角	四[よ]つ 角[かど]	よつかど	
\\	あそこの四つ角を左に曲がってください。	あそこの 四[よ]つ 角[かど]を 左[ひだり]に 曲[ま]がってください。	あそこ の よつかど を ひだり に まがって ください	
\\	あそこの
\\	を 左[ひだり]に 曲[ま]がってください。			
\\	メニュー	メニュー	メニュー	
\\	メニューをください。	メニューをください。	めにゅー を ください	
\\	をください。			
\\	曲	曲[きょく]	きょく	
\\	私はこの曲が大好きです。	私[わたし]はこの 曲[きょく]が 大好[だいす]きです。	わたし は この きょく が だいすき です	
\\	私[わたし]はこの
\\	が 大好[だいす]きです。			
\\	曲げる	曲[ま]げる	まげる	
\\	ひざを曲げてください。	ひざを 曲[ま]げてください。	ひざ を まげて ください	
\\	ひざを
\\	ください。			
\\	曲がり角	曲[ま]がり 角[かど]	まがりかど	
\\	ポストはそこの曲がり角にあります。	ポストはそこの 曲[ま]がり 角[かど]にあります。	ぽすと は そこ の まがりかど に あります	
\\	ポストはそこの
\\	にあります。			
\\	同様	同様[どうよう]	どうよう	
\\	私たちは彼を家族同様に思っている。	私[わたし]たちは 彼[かれ]を 家族[かぞく] 同様[どうよう]に 思[おも]っている。	わたしたち は かれ を かぞく どうよう に おもって いる	
\\	私[わたし]たちは 彼[かれ]を 家族[かぞく]
\\	に 思[おも]っている。			
\\	アイスクリーム	アイスクリーム	アイスクリーム	
\\	弟はアイスクリームが大好きです。	弟[おとうと]はアイスクリームが 大好[だいす]きです。	おとうと は あいすくりーむ が だいすき です	
\\	弟[おとうと]は
\\	が 大好[だいす]きです。			
\\	間違い	間違[まちが]い	まちがい	
\\	この文には間違いがあります。	この 文[ぶん]には 間違[まちが]いがあります。	この ぶん に は まちがい が あります	
\\	この 文[ぶん]には
\\	があります。			
\\	間違える	間違[まちが]える	まちがえる	
\\	電話番号を間違えました。	電話番号[でんわ ばんごう]を 間違[まちが]えました。	でんわ ばんごう を まちがえました	
\\	電話番号[でんわ ばんごう]を
\\	間違う	間違[まちが]う	まちがう	
\\	あなたは間違っている。	あなたは 間違[まちが]っている。	あなた は まちがって いる	
\\	あなたは
\\	似ている	似[に]ている	にている	
\\	私は母に似ています。	私[わたし]は 母[はは]に 似[に]ています。	わたし は はは に にて います	
\\	私[わたし]は 母[はは]に
\\	おしゃれ	おしゃれ	おしゃれ	
\\	あの子はおしゃれだね。	あの 子[こ]はおしゃれだね。	あの こ は おしゃれ だ ね	
\\	あの 子[こ]は
\\	だね。			
\\	以上	以上[いじょう]	いじょう	
\\	飛行機が1時間以上遅れた。	飛行機[ひこうき]が 1時間[いちじかん] 以上[いじょう] 遅[おく]れた。	ひこうき が いちじかん いじょう おくれた	
\\	飛行機[ひこうき]が 1時間[いちじかん]
\\	遅[おく]れた。			
\\	旅行	旅行[りょこう]	りょこう	
\\	彼女は旅行が好きです。	彼女[かのじょ]は 旅行[りょこう]が 好[す]きです。	かのじょ は りょこう が すき です	
\\	彼女[かのじょ]は
\\	が 好[す]きです。			
\\	大使館	大使館[たいしかん]	たいしかん	
\\	彼は大使館に勤めています。	彼[かれ]は 大使館[たいしかん]に 勤[つと]めています。	かれ は たいしかん に つとめて います	
\\	彼[かれ]は
\\	に 勤[つと]めています。			
\\	旅館	旅館[りょかん]	りょかん	
\\	京都では旅館に泊まりました。	京都[きょうと]では 旅館[りょかん]に 泊[と]まりました。	きょうと で は りょかん に とまりました	
\\	京都[きょうと]では
\\	に 泊[と]まりました。			
\\	ジュース	ジュース	ジュース	
\\	このジュースは甘すぎる。	このジュースは 甘[あま]すぎる。	この じゅーす は あま すぎる	
\\	この
\\	は 甘[あま]すぎる。			
\\	映画館	映画館[えいがかん]	えいがかん	
\\	彼と近くの映画館に行きました。	彼[かれ]と 近[ちか]くの 映画館[えいがかん]に 行[い]きました。	かれ と ちかく の えいがかん に いきました	
\\	彼[かれ]と 近[ちか]くの
\\	に 行[い]きました。			
\\	宿題	宿題[しゅくだい]	しゅくだい	
\\	友達と一緒に宿題をした。	友達[ともだち]と 一緒[いっしょ]に 宿題[しゅくだい]をした。	ともだち と いっしょ に しゅくだい を した	
\\	友達[ともだち]と 一緒[いっしょ]に
\\	をした。			
\\	泊める	泊[と]める	とめる	
\\	友達をうちに泊めてあげました。	友達[ともだち]をうちに 泊[と]めてあげました。	ともだち を うち に とめて あげました	
\\	友達[ともだち]をうちに
\\	遊び	遊[あそ]び	あそび	
\\	お正月には色々な遊びをします。	お 正月[しょうがつ]には 色々[いろいろ]な 遊[あそ]びをします。	おしょうがつ に は いろいろ な あそび を します	
\\	お 正月[しょうがつ]には 色々[いろいろ]な
\\	をします。			
\\	ゼロ	ゼロ	ゼロ	
\\	今日の交通事故はゼロです。	今日[きょう]の 交通事故[こうつう じこ]はゼロです。	きょう の こうつう じこ は ぜろ です	
\\	今日[きょう]の 交通事故[こうつう じこ]は
\\	です。			
\\	洋服	洋服[ようふく]	ようふく	
\\	今日は洋服を買いに行きます。	今日[きょう]は 洋服[ようふく]を 買[か]いに 行[い]きます。	きょう は ようふく を かい に いきます	
\\	今日[きょう]は
\\	を 買[か]いに 行[い]きます。			
\\	教室	教室[きょうしつ]	きょうしつ	
\\	私の教室は3階にあります。	私[わたし]の 教室[きょうしつ]は 3階[さんがい]にあります。	わたし の きょうしつ は さんがい に あります	
\\	私[わたし]の
\\	は 3階[さんがい]にあります。			
\\	図書室	図書室[としょしつ]	としょしつ	
\\	図書室で勉強した。	図書室[としょしつ]で 勉強[べんきょう]した。	としょしつ で べんきょう した	
\\	で 勉強[べんきょう]した。			
\\	窓	窓[まど]	まど	
\\	窓を開けてください。	窓[まど]を 開[あ]けてください。	まど を あけて ください	
\\	を 開[あ]けてください。			
\\	ピンク	ピンク	ピンク	
\\	娘がピンクのドレスを着ている。	娘[むすめ]がピンクのドレスを 着[き]ている。	むすめ が ぴんく の どれす を きて いる	
\\	娘[むすめ]が
\\	のドレスを 着[き]ている。			
\\	親しい	親[した]しい	したしい	
\\	週末、親しい友達を家に呼んだ。	週末[しゅうまつ]、 親[した]しい 友達[ともだち]を 家[いえ]に 呼[よ]んだ。	しゅうまつ したしい ともだち を いえ に よんだ	
\\	週末[しゅうまつ]、
\\	友達[ともだち]を 家[いえ]に 呼[よ]んだ。			
\\	親切	親切[しんせつ]	しんせつ	
\\	親切にしてくださってどうもありがとうございます。	親切[しんせつ]にしてくださってどうもありがとうございます。	しんせつ に して くださって どうも ありがとう ございます	
\\	にしてくださってどうもありがとうございます。			
\\	不親切	不親切[ふしんせつ]	ふしんせつ	
\\	その店員は不親切だった。	その 店員[てんいん]は 不親切[ふしんせつ]だった。	その てんいん は ふしんせつ だった	
\\	その 店員[てんいん]は
\\	だった。			
\\	家族	家族[かぞく]	かぞく	
\\	うちは五人家族です。	うちは 五人[ごにん] 家族[かぞく]です。	うち は ごにん かぞく です	
\\	うちは 五人[ごにん]
\\	です。			
\\	グラス	グラス	グラス	
\\	これはきれいなグラスですね。	これはきれいなグラスですね。	これ は きれい な ぐらす です ね	
\\	これはきれいな
\\	ですね。			
\\	万歳	万歳[ばんざい]	ばんざい	
\\	勝った、万歳!	勝[か]った、 万歳[ばんざい]!	かった ばんざい	
\\	勝[か]った、
\\	二十歳	二十歳[はたち]	はたち	
\\	妹は来年、二十歳になります。	妹[いもうと]は 来年[らいねん]、 二十歳[はたち]になります。	いもうと は らいねん はたち に なります	
\\	妹[いもうと]は 来年[らいねん]、
\\	になります。			
\\	姉さん	姉[ねえ]さん	ねえさん	
\\	姉さん、ごめんね。	姉[ねえ]さん、ごめんね。	ねえさん ごめん ね	
\\	、ごめんね。			
\\	兄さん	兄[にい]さん	にいさん	
\\	兄さん、おめでとう。	兄[にい]さん、おめでとう。	にいさん おめでとう	
\\	、おめでとう。			
\\	ダンス	ダンス	ダンス	
\\	彼女はダンスが大好きです。	彼女[かのじょ]はダンスが 大好[だいす]きです。	かのじょ は だんす が だいすき です	
\\	彼女[かのじょ]は
\\	が 大好[だいす]きです。			
\\	兄弟	兄弟[きょうだい]	きょうだい	
\\	彼は3人兄弟です。	彼[かれ]は 3人[さんにん] 兄弟[きょうだい]です。	かれ は さんにん きょうだい です	
\\	彼[かれ]は 3人[さんにん]
\\	です。			
\\	業者	業者[ぎょうしゃ]	ぎょうしゃ	
\\	引っ越しを業者に頼んだ。	引[ひ]っ 越[こ]しを 業者[ぎょうしゃ]に 頼[たの]んだ。	ひっこし を ぎょうしゃ に たのんだ	
\\	引[ひ]っ 越[こ]しを
\\	に 頼[たの]んだ。			
\\	若者	若者[わかもの]	わかもの	
\\	最近の若者は本を読まない。	最近[さいきん]の 若者[わかもの]は 本[ほん]を 読[よ]まない。	さいきん の わかもの は ほん を よまない	
\\	最近[さいきん]の
\\	は 本[ほん]を 読[よ]まない。			
\\	彼ら	彼[かれ]ら	かれら	
\\	彼らはバスケットの選手です。	彼[かれ]らはバスケットの 選手[せんしゅ]です。	かれら は ばすけっと の せんしゅ です	
\\	はバスケットの 選手[せんしゅ]です。			
\\	おじ	おじ	おじ	
\\	おじは銀行に勤めています。	おじは 銀行[ぎんこう]に 勤[つと]めています。	おじ は ぎんこう に つとめて います	
\\	は 銀行[ぎんこう]に 勤[つと]めています。			
\\	結合	結合[けつごう]	けつごう	
\\	ファイルを結合して1つにしました。	ファイルを 結合[けつごう]して 1[ひと]つにしました。	ふぁいる を けつごう して ひとつ に しました	
\\	ファイルを
\\	して 1[ひと]つにしました。			
\\	結果	結果[けっか]	けっか	
\\	試合の結果を早く知りたい。	試合[しあい]の 結果[けっか]を 早[はや]く 知[し]りたい。	しあい の けっか を はやく しりたい	
\\	試合[しあい]の
\\	を 早[はや]く 知[し]りたい。			
\\	果物	果物[くだもの]	くだもの	
\\	デザートに果物を食べましょう。	デザートに 果物[くだもの]を 食[た]べましょう。	でざーと に くだもの を たべましょう	
\\	デザートに
\\	を 食[た]べましょう。			
\\	課題	課題[かだい]	かだい	
\\	夏休みの課題は何ですか。	夏休[なつやす]みの 課題[かだい]は 何[なん]ですか。	なつやすみ の かだい は なん です か	
\\	夏休[なつやす]みの
\\	は 何[なん]ですか。			
\\	サッカー	サッカー	サッカー	
\\	彼はサッカーの選手です。	彼[かれ]はサッカーの 選手[せんしゅ]です。	かれ は さっかー の せんしゅ です	
\\	彼[かれ]は
\\	の 選手[せんしゅ]です。			
\\	課	課[か]	か	
\\	今から課のミーティングがある。	今[いま]から 課[か]のミーティングがある。	いま から か の みーてぃんぐ が ある	
\\	今[いま]から
\\	のミーティングがある。			
\\	効果	効果[こうか]	こうか	
\\	この薬には胃を守る効果がある。	この 薬[くすり]には 胃[い]を 守[まも]る 効果[こうか]がある。	この くすり に は い を まもる こうか が ある	
\\	この 薬[くすり]には 胃[い]を 守[まも]る
\\	がある。			
\\	自動車	自動車[じどうしゃ]	じどうしゃ	
\\	彼は自動車会社に就職した。	彼[かれ]は 自動車[じどうしゃ] 会社[がいしゃ]に 就職[しゅうしょく]した。	かれ は じどうしゃ がいしゃ に しゅうしょく した	
\\	彼[かれ]は
\\	会社[がいしゃ]に 就職[しゅうしょく]した。			
\\	自然	自然[しぜん]	しぜん	
\\	みんなで自然を守りましょう。	みんなで 自然[しぜん]を 守[まも]りましょう。	みんな で しぜん を まもりましょう	
\\	みんなで
\\	を 守[まも]りましょう。			
\\	スープ	スープ	スープ	
\\	母がコーンスープを作っている。	母[はは]がコーンスープを 作[つく]っている。	はは が こーんすーぷ を つくって いる	
\\	母[はは]がコーン
\\	を 作[つく]っている。			
\\	自然	自然[しぜん]	しぜん	
\\	親が子供を守るのは自然なことだ。	親[おや]が 子供[こども]を 守[まも]るのは 自然[しぜん]なことだ。	おや が こども を まもる の は しぜん な こと だ	
\\	親[おや]が 子供[こども]を 守[まも]るのは
\\	なことだ。			
\\	自習	自習[じしゅう]	じしゅう	
\\	明日は自習の時間がある。	明日[あす]は 自習[じしゅう]の 時間[じかん]がある。	あす は じしゅう の じかん が ある	
\\	明日[あす]は
\\	の 時間[じかん]がある。			
\\	理由	理由[りゆう]	りゆう	
\\	遅れた理由を教えてください。	遅[おく]れた 理由[りゆう]を 教[おし]えてください。	おくれた りゆう を おしえて ください	
\\	遅[おく]れた
\\	を 教[おし]えてください。			
\\	自由	自由[じゆう]	じゆう	
\\	今日は自由な時間が多い。	今日[きょう]は 自由[じゆう]な 時間[じかん]が 多[おお]い。	きょう は じゆう な じかん が おおい	
\\	今日[きょう]は
\\	な 時間[じかん]が 多[おお]い。			
\\	バナナ	バナナ	バナナ	
\\	私は毎朝バナナを食べます。	私[わたし]は 毎朝[まいあさ]バナナを 食[た]べます。	わたし は まいあさ ばなな を たべます	
\\	私[わたし]は 毎朝[まいあさ]
\\	を 食[た]べます。			
\\	信じる	信[しん]じる	しんじる	
\\	彼はキリストを信じている。	彼[かれ]はキリストを 信[しん]じている。	かれ は きりすと を しんじて いる	
\\	彼[かれ]はキリストを
\\	信号	信号[しんごう]	しんごう	
\\	信号が青になった。	信号[しんごう]が 青[あお]になった。	しんごう が あお に なった	
\\	が 青[あお]になった。			
\\	頼む	頼[たの]む	たのむ	
\\	私はハンバーガーを頼みました。	私[わたし]はハンバーガーを 頼[たの]みました。	わたし は はんばーがー を たのみました	
\\	私[わたし]はハンバーガーを
\\	市民	市民[しみん]	しみん	
\\	市民の安全は大切だ。	市民[しみん]の 安全[あんぜん]は 大切[たいせつ]だ。	しみん の あんぜん は たいせつ だ	
\\	の 安全[あんぜん]は 大切[たいせつ]だ。			
\\	うるさい	うるさい	うるさい	
\\	この通りは車の音がうるさい。	この 通[とお]りは 車[くるま]の 音[おと]がうるさい。	この とおり は くるま の おと が うるさい	
\\	この 通[とお]りは 車[くるま]の 音[おと]が
\\	住民	住民[じゅうみん]	じゅうみん	
\\	地域の住民が集まって話合いをした。	地域[ちいき]の 住民[じゅうみん]が 集[あつ]まって 話合[はなしあ]いをした。	ちいき の じゅうみん が あつまって はなしあい を した	
\\	地域[ちいき]の
\\	が 集[あつ]まって 話合[はなしあ]いをした。			
\\	主人	主人[しゅじん]	しゅじん	
\\	主人は今、留守です。	主人[しゅじん]は 今[いま]、 留守[るす]です。	しゅじん は いま るす です	
\\	は 今[いま]、 留守[るす]です。			
\\	議員	議員[ぎいん]	ぎいん	
\\	彼は国会議員です。	彼[かれ]は 国会[こっかい] 議員[ぎいん]です。	かれ は こっかい ぎいん です	
\\	彼[かれ]は 国会[こっかい]
\\	です。			
\\	会議	会議[かいぎ]	かいぎ	
\\	今日の午後、大事な会議があります。	今日[きょう]の 午後[ごご]、 大事[だいじ]な 会議[かいぎ]があります。	きょう の ごご だいじ な かいぎ が あります	
\\	今日[きょう]の 午後[ごご]、 大事[だいじ]な
\\	があります。			
\\	パトカー	パトカー	パトカー	
\\	あそこにパトカーがいる。	あそこにパトカーがいる。	あそこ に ぱとかー が いる	
\\	あそこに
\\	がいる。			
\\	対する	対[たい]する	たいする	
\\	その質問に対する答えが見つからなかった。	その 質問[しつもん]に 対[たい]する 答[こた]えが 見[み]つからなかった。	その しつもん に たいする こたえ が みつからなかった	
\\	その 質問[しつもん]に
\\	答[こた]えが 見[み]つからなかった。			
\\	対立	対立[たいりつ]	たいりつ	
\\	その2社は対立しています。	その 2社[にしゃ]は 対立[たいりつ]しています。	その にしゃ は たいりつ して います	
\\	その 2社[にしゃ]は
\\	しています。			
\\	反対	反対[はんたい]	はんたい	
\\	私は反対です。	私[わたし]は 反対[はんたい]です。	わたし は はんたい です	
\\	私[わたし]は
\\	です。			
\\	答え	答[こた]え	こたえ	
\\	彼はその問題の答えが分からない。	彼[かれ]はその 問題[もんだい]の 答[こた]えが 分[わ]からない。	かれ は その もんだい の こたえ が わからない	
\\	彼[かれ]はその 問題[もんだい]の
\\	が 分[わ]からない。			
\\	ハンバーガー	ハンバーガー	ハンバーガー	
\\	今日の昼ご飯はハンバーガーでした。	今日[きょう]の 昼[ひる]ご 飯[はん]はハンバーガーでした。	きょう の ひるごはん は はんばーがー でした	
\\	今日[きょう]の 昼[ひる]ご 飯[はん]は
\\	でした。			
\\	特に	特[とく]に	とくに	
\\	特に質問はありません。	特[とく]に 質問[しつもん]はありません。	とくに しつもん は ありません	
\\	質問[しつもん]はありません。			
\\	特急	特急[とっきゅう]	とっきゅう	
\\	東京まで特急で3時間かかります。	東京[とうきょう]まで 特急[とっきゅう]で 3時間[さんじかん]かかります。	とうきょう まで とっきゅう で さんじかん かかります	
\\	東京[とうきょう]まで
\\	で 3時間[さんじかん]かかります。			
\\	特別	特別[とくべつ]	とくべつ	
\\	あなたは私にとって特別な人です。	あなたは 私[わたし]にとって 特別[とくべつ]な 人[ひと]です。	あなた は わたし に とって とくべつ な ひと です	
\\	あなたは 私[わたし]にとって
\\	な 人[ひと]です。			
\\	別々	別々[べつべつ]	べつべつ	
\\	別々に払いましょう。	別々[べつべつ]に 払[はら]いましょう。	べつべつ に はらいましょう	
\\	に 払[はら]いましょう。			
\\	エスカレーター	エスカレーター	エスカレーター	
\\	3階までエスカレーターで行きましょう。	3階[さんかい]までエスカレーターで 行[い]きましょう。	さんかい まで えすかれーたー で いきましょう	
\\	3階[さんかい]まで
\\	で 行[い]きましょう。			
\\	別れる	別[わか]れる	わかれる	
\\	駅で友だちと別れました。	駅[えき]で 友[とも]だちと 別[わか]れました。	えき で ともだち と わかれました	
\\	駅[えき]で 友[とも]だちと
\\	一般	一般[いっぱん]	いっぱん	
\\	一般の方はこちらの席へどうぞ。	一般[いっぱん]の 方[かた]はこちらの 席[せき]へどうぞ。	いっぱん の かた は こちら の せき へ どうぞ	
\\	の 方[かた]はこちらの 席[せき]へどうぞ。			
\\	目的	目的[もくてき]	もくてき	
\\	彼が来た目的が分かりません。	彼[かれ]が 来[き]た 目的[もくてき]が 分[わ]かりません。	かれ が きた もくてき が わかりません	
\\	彼[かれ]が 来[き]た
\\	が 分[わ]かりません。			
\\	普通	普通[ふつう]	ふつう	
\\	彼女は普通の女の子だ。	彼女[かのじょ]は 普通[ふつう]の 女[おんな]の 子[こ]だ。	かのじょ は ふつう の おんな の こ だ	
\\	彼女[かのじょ]は
\\	の 女[おんな]の 子[こ]だ。			
\\	タオル	タオル	タオル	
\\	私はタオルで顔をふいた。	私[わたし]はタオルで 顔[かお]をふいた。	わたし は たおる で かお を ふいた	
\\	私[わたし]は
\\	で 顔[かお]をふいた。			
\\	並ぶ	並[なら]ぶ	ならぶ	
\\	ここに並んでください。	ここに 並[なら]んでください。	ここ に ならんで ください	
\\	ここに
\\	ください。			
\\	並べる	並[なら]べる	ならべる	
\\	私は料理をテーブルに並べた。	私[わたし]は 料理[りょうり]をテーブルに 並[なら]べた。	わたし は りょうり を てーぶる に ならべた	
\\	私[わたし]は 料理[りょうり]をテーブルに
\\	平和	平和[へいわ]	へいわ	
\\	この国は平和です。	この 国[くに]は 平和[へいわ]です。	この くに は へいわ です	
\\	この 国[くに]は
\\	です。			
\\	高等学校	高等学校[こうとうがっこう]	こうとうがっこう	
\\	弟が高等学校を卒業しました。	弟[おとうと]が 高等学校[こうとうがっこう]を 卒業[そつぎょう]しました。	おとうと が こうとうがっこう を そつぎょう しました	
\\	弟[おとうと]が
\\	を 卒業[そつぎょう]しました。			
\\	パチンコ	パチンコ	パチンコ	
\\	彼は毎日パチンコをしています。	彼[かれ]は 毎日[まいにち]パチンコをしています。	かれ は まいにち ぱちんこ を して います	
\\	彼[かれ]は 毎日[まいにち]
\\	をしています。			
\\	病院	病院[びょういん]	びょういん	
\\	病院はどこですか。	病院[びょういん]はどこですか。	びょういん は どこ です か	
\\	はどこですか。			
\\	入院	入院[にゅういん]	にゅういん	
\\	昨日、母が入院しました。	昨日[きのう]、 母[はは]が 入院[にゅういん]しました。	きのう はは が にゅういん しました	
\\	昨日[きのう]、 母[はは]が
\\	しました。			
\\	大学院	大学院[だいがくいん]	だいがくいん	
\\	彼は大学院に進みました。	彼[かれ]は 大学院[だいがくいん]に 進[すす]みました。	かれ は だいがくいん に すすみました	
\\	彼[かれ]は
\\	に 進[すす]みました。			
\\	医者	医者[いしゃ]	いしゃ	
\\	私は医者に相談した。	私[わたし]は 医者[いしゃ]に 相談[そうだん]した。	わたし は いしゃ に そうだん した	
\\	私[わたし]は
\\	に 相談[そうだん]した。			
\\	みかん	みかん	みかん	
\\	みかんを1つください。	みかんを 1[ひと]つください。	みかん を ひとつ ください	
\\	を 1[ひと]つください。			
\\	お医者さん	お 医者[いしゃ]さん	おいしゃさん	
\\	熱があるのでお医者さんに行った。	熱[ねつ]があるのでお 医者[いしゃ]さんに 行[い]った。	ねつ が ある の で おいしゃさん に いった	
\\	熱[ねつ]があるので
\\	に 行[い]った。			
\\	歯医者	歯医者[はいしゃ]	はいしゃ	
\\	私は歯医者が嫌いです。	私[わたし]は 歯医者[はいしゃ]が 嫌[きら]いです。	わたし は はいしゃ が きらい です	
\\	私[わたし]は
\\	が 嫌[きら]いです。			
\\	歯	歯[は]	は	
\\	私の歯は丈夫です。	私[わたし]の 歯[は]は 丈夫[じょうぶ]です。	わたし の は は じょうぶ です	
\\	私[わたし]の
\\	は 丈夫[じょうぶ]です。			
\\	歯ブラシ	歯[は]ブラシ	はぶらし	
\\	新しい歯ブラシが必要だ。	新[あたら]しい 歯[は]ブラシが 必要[ひつよう]だ。	あたらしい はぶらし が ひつよう だ	
\\	新[あたら]しい
\\	が 必要[ひつよう]だ。			
\\	ケーキ	ケーキ	ケーキ	
\\	誕生日にケーキを食べました。	誕生日[たんじょうび]にケーキを 食[た]べました。	たんじょうび に けーき を たべました	
\\	誕生日[たんじょうび]に
\\	を 食[た]べました。			
\\	科学	科学[かがく]	かがく	
\\	科学は常に進歩している。	科学[かがく]は 常[つね]に 進歩[しんぽ]している。	かがく は つねに しんぽ して いる	
\\	は 常[つね]に 進歩[しんぽ]している。			
\\	教科書	教科書[きょうかしょ]	きょうかしょ	
\\	日本語の教科書を忘れた。	日本語[にほんご]の 教科書[きょうかしょ]を 忘[わす]れた。	にほんご の きょうかしょ を わすれた	
\\	日本語[にほんご]の
\\	を 忘[わす]れた。			
\\	理科	理科[りか]	りか	
\\	私は理科が得意です。	私[わたし]は 理科[りか]が 得意[とくい]です。	わたし は りか が とくい です	
\\	私[わたし]は
\\	が 得意[とくい]です。			
\\	亡くなる	亡[な]くなる	なくなる	
\\	おととい、昔の友人が亡くなった。	おととい、 昔[むかし]の 友人[ゆうじん]が 亡[な]くなった。	おととい むかし の ゆうじん が なくなった	
\\	おととい、 昔[むかし]の 友人[ゆうじん]が
\\	コップ	コップ	コップ	
\\	私は毎朝コップ一杯の水を飲む。	私[わたし]は 毎朝[まいあさ]コップ 一杯[いっぱい]の 水[みず]を 飲[の]む。	わたし は まいあさ こっぷ いっぱい の みず を のむ	
\\	私[わたし]は 毎朝[まいあさ]
\\	一杯[いっぱい]の 水[みず]を 飲[の]む。			
\\	忙しい	忙[いそが]しい	いそがしい	
\\	忙しいので手伝ってください。	忙[いそが]しいので 手伝[てつだ]ってください。	いそがしい の で てつだって ください	
\\	ので 手伝[てつだ]ってください。			
\\	疲れる	疲[つか]れる	つかれる	
\\	私は疲れてふらふらです。	私[わたし]は 疲[つか]れてふらふらです。	わたし は つかれて ふらふら です	
\\	私[わたし]は
\\	ふらふらです。			
\\	禁煙	禁煙[きんえん]	きんえん	
\\	彼は今、禁煙しています。	彼[かれ]は 今[いま]、 禁煙[きんえん]しています。	かれ は いま きんえん して います	
\\	彼[かれ]は 今[いま]、
\\	しています。			
\\	酔っ払い	酔[よ]っ 払[ぱら]い	よっぱらい	
\\	彼はただの酔っ払いです。	彼[かれ]はただの 酔[よ]っ 払[ぱら]いです。	かれ は ただ の よっぱらい です	
\\	彼[かれ]はただの
\\	です。			
\\	ナイロン	ナイロン	ナイロン	
\\	このジャケットはナイロンでできています。	このジャケットはナイロンでできています。	この じゃけっと は ないろん で できて います	
\\	このジャケットは
\\	でできています。			
\\	酔っ払う	酔[よ]っ 払[ぱら]う	よっぱらう	
\\	昨夜は酔っ払いました。	昨夜[ゆうべ]は 酔[よ]っ 払[ぱら]いました。	ゆうべ は よっぱらいました	
\\	昨夜[ゆうべ]は
\\	危ない	危[あぶ]ない	あぶない	
\\	その道は車が多くて危ない。	その 道[みち]は 車[くるま]が 多[おお]くて 危[あぶ]ない。	その みち は くるま が おおくて あぶない	
\\	その 道[みち]は 車[くるま]が 多[おお]くて
\\	危険	危険[きけん]	きけん	
\\	その地域は今、危険だ。	その 地域[ちいき]は 今[いま]、 危険[きけん]だ。	その ちいき は いま きけん だ	
\\	その 地域[ちいき]は 今[いま]、
\\	だ。			
\\	存在	存在[そんざい]	そんざい	
\\	宇宙人は存在すると思いますか。	宇宙人[うちゅうじん]は 存在[そんざい]すると 思[おも]いますか。	うちゅうじん は そんざい する と おもいます か	
\\	宇宙人[うちゅうじん]は
\\	すると 思[おも]いますか。			
\\	スーツ	スーツ	スーツ	
\\	あのスーツはそんなに高くない。	あのスーツはそんなに 高[たか]くない。	あの すーつ は そんなに たかく ない	
\\	あの
\\	はそんなに 高[たか]くない。			
\\	注目	注目[ちゅうもく]	ちゅうもく	
\\	私たちはその会社に注目している。	私[わたし]たちはその 会社[かいしゃ]に 注目[ちゅうもく]している。	わたしたち は その かいしゃ に ちゅうもく して いる	
\\	私[わたし]たちはその 会社[かいしゃ]に
\\	している。			
\\	注文	注文[ちゅうもん]	ちゅうもん	
\\	レストランでピザを注文しました。	レストランでピザを 注文[ちゅうもん]しました。	れすとらん で ぴざ を ちゅうもん しました	
\\	レストランでピザを
\\	しました。			
\\	意味	意味[いみ]	いみ	
\\	それはどういう意味ですか。	それはどういう 意味[いみ]ですか。	それ は どういう いみ です か	
\\	それはどういう
\\	ですか。			
\\	意見	意見[いけん]	いけん	
\\	あなたの意見が聞きたいです。	あなたの 意見[いけん]が 聞[き]きたいです。	あなた の いけん が ききたい です	
\\	あなたの
\\	が 聞[き]きたいです。			
\\	チケット	チケット	チケット	
\\	この遊園地のチケットは3000円です。	この 遊園地[ゆうえんち]のチケットは 3000円[さんぜんえん]です。	この ゆうえんち の ちけっと は さんぜんえん です	
\\	この 遊園地[ゆうえんち]の
\\	は 3000円[さんぜんえん]です。			
\\	注意	注意[ちゅうい]	ちゅうい	
\\	車に注意してください。	車[くるま]に 注意[ちゅうい]してください。	くるま に ちゅうい して ください	
\\	車[くるま]に
\\	してください。			
\\	用意	用意[ようい]	ようい	
\\	食事の用意ができました。	食事[しょくじ]の 用意[ようい]ができました。	しょくじ の ようい が できました	
\\	食事[しょくじ]の
\\	ができました。			
\\	確か	確[たし]か	たしか	
\\	彼の昇進は確かだ。	彼[かれ]の 昇進[しょうしん]は 確[たし]かだ。	かれ の しょうしん は たしか だ	
\\	彼[かれ]の 昇進[しょうしん]は
\\	だ。			
\\	確認	確認[かくにん]	かくにん	
\\	もう一度、予約を確認した。	もう 一度[いちど]、 予約[よやく]を 確認[かくにん]した。	もう いちど よやく を かくにん した	
\\	もう 一度[いちど]、 予約[よやく]を
\\	した。			
\\	チョコレート	チョコレート	チョコレート	
\\	妹はチョコレートが大好きです。	妹[いもうと]はチョコレートが 大好[だいす]きです。	いもうと は ちょこれーと が だいすき です	
\\	妹[いもうと]は
\\	が 大好[だいす]きです。			
\\	機能	機能[きのう]	きのう	
\\	このソフトにはいろいろな機能があります。	このソフトにはいろいろな 機能[きのう]があります。	この そふと に は いろいろな きのう が あります	
\\	このソフトにはいろいろな
\\	があります。			
\\	ジェット機	ジェット 機[き]	ジェットき	
\\	ジェット機が飛んでいる。	ジェット 機[き]が 飛[と]んでいる。	じぇっとき が とんで いる	
\\	が 飛[と]んでいる。			
\\	機械	機械[きかい]	きかい	
\\	新しい機械が壊れた。	新[あたら]しい 機械[きかい]が 壊[こわ]れた。	あたらしい きかい が こわれた	
\\	新[あたら]しい
\\	が 壊[こわ]れた。			
\\	材料	材料[ざいりょう]	ざいりょう	
\\	サラダの材料をそろえました。	サラダの 材料[ざいりょう]をそろえました。	さらだ の ざいりょう を そろえました	
\\	サラダの
\\	をそろえました。			
\\	チャンネル	チャンネル	チャンネル	
\\	テレビのチャンネルを変えてください。	テレビのチャンネルを 変[か]えてください。	てれび の ちゃんねる を かえて ください	
\\	テレビの
\\	を 変[か]えてください。			
\\	具体的	具体的[ぐたいてき]	ぐたいてき	
\\	具体的な例をいくつか見せてください。	具体的[ぐたいてき]な 例[れい]をいくつか 見[み]せてください。	ぐたいてき な れい を いくつか みせて ください	
\\	な 例[れい]をいくつか 見[み]せてください。			
\\	基づく	基[もと]づく	もとづく	
\\	この話は真実に基づいています。	この 話[はなし]は 真実[しんじつ]に 基[もと]づいています。	この はなし は しんじつ に もとづいて います	
\\	この 話[はなし]は 真実[しんじつ]に
\\	基本	基本[きほん]	きほん	
\\	今、ジャズダンスの基本を習っています。	今[いま]、ジャズダンスの 基本[きほん]を 習[なら]っています。	いま じゃず だんす の きほん を ならって います	
\\	今[いま]、ジャズダンスの
\\	を 習[なら]っています。			
\\	基準	基準[きじゅん]	きじゅん	
\\	判断の基準が示された。	判断[はんだん]の 基準[きじゅん]が 示[しめ]された。	はんだん の きじゅん が しめされた	
\\	判断[はんだん]の
\\	が 示[しめ]された。			
\\	つまらない	つまらない	つまらない	
\\	彼の話はつまらないですね。	彼[かれ]の 話[はなし]はつまらないですね。	かれ の はなし は つまらない です ね	
\\	彼[かれ]の 話[はなし]は
\\	ですね。			
\\	備える	備[そな]える	そなえる	
\\	災害に備えて大量の水を買い込んだ。	災害[さいがい]に 備[そな]えて 大量[たいりょう]の 水[みず]を 買[か]い 込[こ]んだ。	さいがい に そなえて たいりょう の みず を かいこんだ	
\\	災害[さいがい]に
\\	大量[たいりょう]の 水[みず]を 買[か]い 込[こ]んだ。			
\\	準備	準備[じゅんび]	じゅんび	
\\	明日の会議の準備をした。	明日[あす]の 会議[かいぎ]の 準備[じゅんび]をした。	あす の かいぎ の じゅんび を した	
\\	明日[あす]の 会議[かいぎ]の
\\	をした。			
\\	設計	設計[せっけい]	せっけい	
\\	兄は船の設計をしています。	兄[あに]は 船[ふね]の 設計[せっけい]をしています。	あに は ふね の せっけい を して います	
\\	兄[あに]は 船[ふね]の
\\	をしています。			
\\	施設	施設[しせつ]	しせつ	
\\	そのホテルにはレジャー施設がたくさんある。	そのホテルにはレジャー 施設[しせつ]がたくさんある。	その ほてる に は れじゃー しせつ が たくさん ある	
\\	そのホテルにはレジャー
\\	がたくさんある。			
\\	マッチ	マッチ	マッチ	
\\	マッチを持っていますか。	マッチを 持[も]っていますか。	まっち を もって います か	
\\	を 持[も]っていますか。			
\\	設ける	設[もう]ける	もうける	
\\	授業の終わりに復習の時間を設けた。	授業[じゅぎょう]の 終[お]わりに 復習[ふくしゅう]の 時間[じかん]を 設[もう]けた。	じゅぎょう の おわり に ふくしゅう の じかん を もうけた	
\\	授業[じゅぎょう]の 終[お]わりに 復習[ふくしゅう]の 時間[じかん]を
\\	説明	説明[せつめい]	せつめい	
\\	この単語の意味を説明してください。	この 単語[たんご]の 意味[いみ]を 説明[せつめい]してください。	この たんご の いみ を せつめい して ください	
\\	この 単語[たんご]の 意味[いみ]を
\\	してください。			
\\	小説	小説[しょうせつ]	しょうせつ	
\\	私は月に3冊くらい小説を読みます。	私[わたし]は 月[つき]に 3冊[さんさつ]くらい 小説[しょうせつ]を 読[よ]みます。	わたし は つき に さんさつ くらい しょうせつ を よみます	
\\	私[わたし]は 月[つき]に 3冊[さんさつ]くらい
\\	を 読[よ]みます。			
\\	公開	公開[こうかい]	こうかい	
\\	その映画は今日、公開されます。	その 映画[えいが]は 今日[きょう]、 公開[こうかい]されます。	その えいが は きょう こうかい されます	
\\	その 映画[えいが]は 今日[きょう]、
\\	されます。			
\\	いとこ	いとこ	いとこ	
\\	従兄弟と私は同じ年です。	従兄弟[いとこ]と 私[わたし]は 同[おな]じ 年[とし]です。	いとこ と わたし は おなじ とし です	
\\	と 私[わたし]は 同[おな]じ 年[とし]です。			
\\	公園	公園[こうえん]	こうえん	
\\	子供たちが公園で遊んでいる。	子供[こども]たちが 公園[こうえん]で 遊[あそ]んでいる。	こどもたち が こうえん で あそんで いる	
\\	子供[こども]たちが
\\	で 遊[あそ]んでいる。			
\\	動物園	動物園[どうぶつえん]	どうぶつえん	
\\	昨日子供たちと動物園に行きました。	昨日子供[きのう こども]たちと 動物園[どうぶつえん]に 行[い]きました。	きのう こどもたち と どうぶつえん に いきました	
\\	昨日子供[きのう こども]たちと
\\	に 行[い]きました。			
\\	祭り	祭[まつ]り	まつり	
\\	彼女は祭りが大好きです。	彼女[かのじょ]は 祭[まつ]りが 大好[だいす]きです。	かのじょ は まつり が だいすき です 。	
\\	彼女[かのじょ]は
\\	が 大好[だいす]きです。			
\\	国際	国際[こくさい]	こくさい	
\\	ここで国際会議が開かれます。	ここで 国際[こくさい] 会議[かいぎ]が 開[ひら]かれます。	ここ で こくさい かいぎ が ひらかれます	
\\	ここで
\\	会議[かいぎ]が 開[ひら]かれます。			
\\	ストーブ	ストーブ	ストーブ	
\\	うちではまだストーブを使っています。	うちではまだストーブを 使[つか]っています。	うち で は まだ すとーぶ を つかって います	
\\	うちではまだ
\\	を 使[つか]っています。			
\\	実際	実際[じっさい]	じっさい	
\\	彼は実際にはあまり背が高くない。	彼[かれ]は 実際[じっさい]にはあまり 背[せ]が 高[たか]くない。	かれ は じっさい に は あまり せ が たかく ない	
\\	彼[かれ]は
\\	にはあまり 背[せ]が 高[たか]くない。			
\\	飛行場	飛行場[ひこうじょう]	ひこうじょう	
\\	バスが飛行場に着きました。	バスが 飛行場[ひこうじょう]に 着[つ]きました。	ばす が ひこうじょう に つきました	
\\	バスが
\\	に 着[つ]きました。			
\\	航空便	航空便[こうくうびん]	こうくうびん	
\\	航空便で書類が届きました。	航空便[こうくうびん]で 書類[しょるい]が 届[とど]きました。	こうくうびん で しょるい が とどきました	
\\	で 書類[しょるい]が 届[とど]きました。			
\\	船便	船便[ふなびん]	ふなびん	
\\	アメリカの友人から船便が届いた。	アメリカの 友人[ゆうじん]から 船便[ふなびん]が 届[とど]いた。	あめりか の ゆうじん から ふなびん が とどいた	
\\	アメリカの 友人[ゆうじん]から
\\	が 届[とど]いた。			
\\	ガソリンスタンド	ガソリンスタンド	ガソリンスタンド	
\\	この近くにガソリンスタンドはありますか。	この 近[ちか]くにガソリンスタンドはありますか。	この ちかく に がそりんすたんど は あります か	
\\	この 近[ちか]くに
\\	はありますか。			
\\	船	船[ふね]	ふね	
\\	私たちは船に乗った。	私[わたし]たちは 船[ふね]に 乗[の]った。	わたしたち は ふね に のった	
\\	私[わたし]たちは
\\	に 乗[の]った。			
\\	空港	空港[くうこう]	くうこう	
\\	空港までリムジンバスで行った。	空港[くうこう]までリムジンバスで 行[い]った。	くうこう まで りむじん ばす で いった	
\\	までリムジンバスで 行[い]った。			
\\	港	港[みなと]	みなと	
\\	港に船が着きました。	港[みなと]に 船[ふね]が 着[つ]きました。	みなと に ふね が つきました	
\\	に 船[ふね]が 着[つ]きました。			
\\	島	島[しま]	しま	
\\	日本は島国です。	日本[にっぽん]は 島[しま] 国[ぐに]です。	にっぽん は しまぐに です	
\\	日本[にっぽん]は
\\	国[ぐに]です。			
\\	デート	デート	デート	
\\	デートで遊園地に行きました。	デートで 遊園地[ゆうえんち]に 行[い]きました。	でーと で ゆうえんち に いきました	
\\	で 遊園地[ゆうえんち]に 行[い]きました。			
\\	完成	完成[かんせい]	かんせい	
\\	新しいホームページが完成した。	新[あたら]しいホームページが 完成[かんせい]した。	あたらしい ほーむぺーじ が かんせい した	
\\	新[あたら]しいホームページが
\\	した。			
\\	平成	平成[へいせい]	へいせい	
\\	彼女は平成3年生まれです。	彼女[かのじょ]は 平成[へいせい] 3年生[さんねんう]まれです。	かのじょ は へいせい さんねん うまれ です	
\\	彼女[かのじょ]は
\\	3年生[さんねんう]まれです。			
\\	成功	成功[せいこう]	せいこう	
\\	ついに実験が成功した。	ついに 実験[じっけん]が 成功[せいこう]した。	ついに じっけん が せいこう した	
\\	ついに 実験[じっけん]が
\\	した。			
\\	原因	原因[げんいん]	げんいん	
\\	この事故の原因は何ですか。	この 事故[じこ]の 原因[げんいん]は 何[なん]ですか。	このじこ の げんいん は なん です か	
\\	この 事故[じこ]の
\\	は 何[なん]ですか。			
\\	ふすま	ふすま	ふすま	
\\	ふすまを閉めてください。	ふすまを 閉[し]めてください。	ふすま を しめて ください	
\\	を 閉[し]めてください。			
\\	資金	資金[しきん]	しきん	
\\	私たちは今、結婚資金を貯めています。	私[わたし]たちは 今[いま]、 結婚[けっこん] 資金[しきん]を 貯[た]めています。	わたしたち は いま けっこん しきん を ためて います	
\\	私[わたし]たちは 今[いま]、 結婚[けっこん]
\\	を 貯[た]めています。			
\\	投資	投資[とうし]	とうし	
\\	私は4つの会社に投資しています。	私[わたし]は 4[よっ]つの 会社[かいしゃ]に 投資[とうし]しています。	わたし は よっつ の かいしゃ に とうし して います	
\\	私[わたし]は 4[よっ]つの 会社[かいしゃ]に
\\	しています。			
\\	願う	願[ねが]う	ねがう	
\\	彼が元気になるよう願っています。	彼[かれ]が 元気[げんき]になるよう 願[ねが]っています。	かれ が げんき に なる よう ねがって います	
\\	彼[かれ]が 元気[げんき]になるよう
\\	正確	正確[せいかく]	せいかく	
\\	彼の計算は正確です。	彼[かれ]の 計算[けいさん]は 正確[せいかく]です。	かれ の けいさん は せいかく です	
\\	彼[かれ]の 計算[けいさん]は
\\	です。			
\\	レモン	レモン	レモン	
\\	紅茶にレモンを入れて飲んだ。	紅茶[こうちゃ]にレモンを 入[い]れて 飲[の]んだ。	こうちゃ に れもん を いれて のんだ	
\\	紅茶[こうちゃ]に
\\	を 入[い]れて 飲[の]んだ。			
\\	正しい	正[ただ]しい	ただしい	
\\	それは正しい答えです。	それは 正[ただ]しい 答[こた]えです。	それ は ただしい こたえ です	
\\	それは
\\	答[こた]えです。			
\\	正月	正月[しょうがつ]	しょうがつ	
\\	お正月にはたいてい、家族が集まる。	お 正月[しょうがつ]にはたいてい、 家族[かぞく]が 集[あつ]まる。	おしょうがつ に は たいてい かぞく が あつまる	
\\	お
\\	にはたいてい、 家族[かぞく]が 集[あつ]まる。			
\\	正直	正直[しょうじき]	しょうじき	
\\	彼女はとても正直だ。	彼女[かのじょ]はとても 正直[しょうじき]だ。	かのじょ は とても しょうじき だ	
\\	彼女[かのじょ]はとても
\\	だ。			
\\	異なる	異[こと]なる	ことなる	
\\	彼と私はいつも意見が異なる。	彼[かれ]と 私[わたし]はいつも 意見[いけん]が 異[こと]なる。	かれ と わたし は いつも いけん が ことなる	
\\	彼[かれ]と 私[わたし]はいつも 意見[いけん]が
\\	チーズ	チーズ	チーズ	
\\	チーズを一切れ食べました。	チーズを 一切[ひとき]れ 食[た]べました。	ちーず を ひときれ たべました	
\\	を 一切[ひとき]れ 食[た]べました。			
\\	通常	通常[つうじょう]	つうじょう	
\\	通常は夜8時まで営業しています。	通常[つうじょう]は 夜8時[よる はちじ]まで 営業[えいぎょう]しています。	つうじょう は よる はちじ まで えいぎょう して います	
\\	は 夜8時[よる はちじ]まで 営業[えいぎょう]しています。			
\\	非常に	非常[ひじょう]に	ひじょうに	
\\	これは非常に重要です。	これは 非常[ひじょう]に 重要[じゅうよう]です。	これ は ひじょう に じゅうよう です	
\\	これは
\\	重要[じゅうよう]です。			
\\	調べる	調[しら]べる	しらべる	
\\	この単語の意味を辞書で調べましょう。	この 単語[たんご]の 意味[いみ]を 辞書[じしょ]で 調[しら]べましょう。	この たんご の いみ を じしょ で しらべましょう	
\\	この 単語[たんご]の 意味[いみ]を 辞書[じしょ]で
\\	強調	強調[きょうちょう]	きょうちょう	
\\	彼は良いところだけを強調した。	彼[かれ]は 良[よ]いところだけを 強調[きょうちょう]した。	かれ は よい ところ だけ を きょうちょう した	
\\	彼[かれ]は 良[よ]いところだけを
\\	した。			
\\	ドライブ	ドライブ	ドライブ	
\\	今日は群馬までドライブしました。	今日[きょう]は 群馬[ぐんま]までドライブしました。	きょう は ぐんま まで どらいぶ しました	
\\	今日[きょう]は 群馬[ぐんま]まで
\\	しました。			
\\	季節	季節[きせつ]	きせつ	
\\	私の一番好きな季節は春です。	私[わたし]の 一番好[いちばん す]きな 季節[きせつ]は 春[はる]です。	わたし の いちばん すき な きせつ は はる です	
\\	私[わたし]の 一番好[いちばん す]きな
\\	は 春[はる]です。			
\\	調査	調査[ちょうさ]	ちょうさ	
\\	私たちがその問題を調査しています。	私[わたし]たちがその 問題[もんだい]を 調査[ちょうさ]しています。	わたしたち が その もんだい を ちょうさ して います	
\\	私[わたし]たちがその 問題[もんだい]を
\\	しています。			
\\	提供	提供[ていきょう]	ていきょう	
\\	彼がパーティー会場を提供してくれました。	彼[かれ]がパーティー 会場[かいじょう]を 提供[ていきょう]してくれました。	かれ が ぱーてぃー かいじょう を ていきょう して くれました	
\\	彼[かれ]がパーティー 会場[かいじょう]を
\\	してくれました。			
\\	提案	提案[ていあん]	ていあん	
\\	そのアイデアは彼の提案です。	そのアイデアは 彼[かれ]の 提案[ていあん]です。	その あいであ は かれ の ていあん です	
\\	そのアイデアは 彼[かれ]の
\\	です。			
\\	ラケット	ラケット	ラケット	
\\	テニスのラケットを買いました。	テニスのラケットを 買[か]いました。	てにす の らけっと を かいました	
\\	テニスの
\\	を 買[か]いました。			
\\	案内	案内[あんない]	あんない	
\\	私が中をご案内します。	私[わたし]が 中[なか]をご 案内[あんない]します。	わたし が なか を ごあんない します	
\\	私[わたし]が 中[なか]をご
\\	します。			
\\	示す	示[しめ]す	しめす	
\\	彼は新製品に興味を示している。	彼[かれ]は 新製品[しんせいひん]に 興味[きょうみ]を 示[しめ]している。	かれ は しんせいひん に きょうみ を しめして いる	
\\	彼[かれ]は 新製品[しんせいひん]に 興味[きょうみ]を
\\	連れて行く	連[つ]れて 行[い]く	つれていく	
\\	私も連れて行ってください。	私[わたし]も 連[つ]れて 行[い]ってください。	わたし も つれて いって ください	
\\	私[わたし]も
\\	ください。			
\\	連れて来る	連[つ]れて 来[く]る	つれてくる	
\\	息子が友達を連れて来ました。	息子[むすこ]が 友達[ともだち]を 連[つ]れて 来[き]ました。	むすこ が ともだち を つれて きました	
\\	息子[むすこ]が 友達[ともだち]を
\\	スチュワーデス	スチュワーデス	スチュワーデス	
\\	彼女はスチュワーデスになった。	彼女[かのじょ]はスチュワーデスになった。	かのじょ は すちゅわーです に なった	
\\	彼女[かのじょ]は
\\	になった。			
\\	続ける	続[つづ]ける	つづける	
\\	仕事を続けてください。	仕事[しごと]を 続[つづ]けてください。	しごと を つづけて ください	
\\	仕事[しごと]を
\\	ください。			
\\	相手	相手[あいて]	あいて	
\\	試合の相手は誰ですか。	試合[しあい]の 相手[あいて]は 誰[だれ]ですか。	しあい の あいて は だれ です か	
\\	試合[しあい]の
\\	は 誰[だれ]ですか。			
\\	会談	会談[かいだん]	かいだん	
\\	会談の内容が発表されました。	会談[かいだん]の 内容[ないよう]が 発表[はっぴょう]されました。	かいだん の ないよう が はっぴょう されました	
\\	の 内容[ないよう]が 発表[はっぴょう]されました。			
\\	相談	相談[そうだん]	そうだん	
\\	相談したいことがあります。	相談[そうだん]したいことがあります。	そうだん したい こと が あります	
\\	したいことがあります。			
\\	テープレコーダー	テープレコーダー	テープレコーダー	
\\	テープレコーダーで自分の声を録音しました。	テープレコーダーで 自分[じぶん]の 声[こえ]を 録音[ろくおん]しました。	てーぷれこーだー で じぶん の こえ を ろくおん しました	
\\	で 自分[じぶん]の 声[こえ]を 録音[ろくおん]しました。			
\\	記者	記者[きしゃ]	きしゃ	
\\	彼は新聞記者です。	彼[かれ]は 新聞[しんぶん] 記者[きしゃ]です。	かれ は しんぶん きしゃ です	
\\	彼[かれ]は 新聞[しんぶん]
\\	です。			
\\	記録	記録[きろく]	きろく	
\\	マラソンで世界記録が出た。	マラソンで 世界[せかい] 記録[きろく]が 出[で]た。	まらそん で せかい きろく が でた	
\\	マラソンで 世界[せかい]
\\	が 出[で]た。			
\\	録音	録音[ろくおん]	ろくおん	
\\	そのラジオ番組はもう録音しました。	そのラジオ 番組[ばんぐみ]はもう 録音[ろくおん]しました。	その らじお ばんぐみ は もう ろくおん しました	
\\	そのラジオ 番組[ばんぐみ]はもう
\\	しました。			
\\	登る	登[のぼ]る	のぼる	
\\	私たちは昨年、富士山に登りました。	私[わたし]たちは 昨年[さくねん]、 富士山[ふじさん]に 登[のぼ]りました。	わたしたち は さくねん ふじさん に のぼりました	
\\	私[わたし]たちは 昨年[さくねん]、 富士山[ふじさん]に
\\	ひげ	ひげ	ひげ	
\\	父はひげをはやしています。	父[ちち]はひげをはやしています。	ちち は ひげ を はやして います	
\\	父[ちち]は
\\	をはやしています。			
\\	関する	関[かん]する	かんする	
\\	その問題に関する記事を読みました。	その 問題[もんだい]に 関[かん]する 記事[きじ]を 読[よ]みました。	その もんだい に かんする きじ を よみました	
\\	その 問題[もんだい]に
\\	記事[きじ]を 読[よ]みました。			
\\	関連	関連[かんれん]	かんれん	
\\	関連のセクションへ連絡した。	関連[かんれん]のセクションへ 連絡[れんらく]した。	かんれん の せくしょん へ れんらく した	
\\	のセクションへ 連絡[れんらく]した。			
\\	関係	関係[かんけい]	かんけい	
\\	彼はその事件に関係がない。	彼[かれ]はその 事件[じけん]に 関係[かんけい]がない。	かれ は その じけん に かんけい が ない	
\\	彼[かれ]はその 事件[じけん]に
\\	がない。			
\\	状況	状況[じょうきょう]	じょうきょう	
\\	この状況では出発は難しいです。	この 状況[じょうきょう]では 出発[しゅっぱつ]は 難[むずか]しいです。	この じょうきょう で は しゅっぱつ は むずかしい です	
\\	この
\\	では 出発[しゅっぱつ]は 難[むずか]しいです。			
\\	ぐっすり	ぐっすり	ぐっすり	
\\	赤ちゃんがぐっすり寝ている。	赤[あか]ちゃんがぐっすり 寝[ね]ている。	あかちゃん が ぐっすり ねて いる	
\\	赤[あか]ちゃんが
\\	寝[ね]ている。			
\\	状態	状態[じょうたい]	じょうたい	
\\	ここは道の状態がとても悪いです。	ここは 道[みち]の 状態[じょうたい]がとても 悪[わる]いです。	ここ は みち の じょうたい が とても わるい です	
\\	ここは 道[みち]の
\\	がとても 悪[わる]いです。			
\\	治る	治[なお]る	なおる	
\\	けがはもう治りましたか。	けがはもう 治[なお]りましたか。	けが は もう なおりました か	
\\	けがはもう
\\	か。			
\\	政治	政治[せいじ]	せいじ	
\\	私は政治に関心がある。	私[わたし]は 政治[せいじ]に 関心[かんしん]がある。	わたし は せいじ に かんしん が ある	
\\	私[わたし]は
\\	に 関心[かんしん]がある。			
\\	治す	治[なお]す	なおす	
\\	早く風邪を治してください。	早[はや]く 風邪[かぜ]を 治[なお]してください。	はやく かぜ を なおして ください	
\\	早[はや]く 風邪[かぜ]を
\\	ください。			
\\	ソース	ソース	ソース	
\\	ソースはどれですか。	ソースはどれですか。	そーす は どれ です か	
\\	はどれですか。			
\\	政府	政府[せいふ]	せいふ	
\\	そのデモについて政府は何もしなかった。	そのデモについて 政府[せいふ]は 何[なに]もしなかった。	その でも に ついて せいふ は なにも しなかった	
\\	そのデモについて
\\	は 何[なに]もしなかった。			
\\	党	党[とう]	とう	
\\	党の代表が質問に答えました。	党[とう]の 代表[だいひょう]が 質問[しつもん]に 答[こた]えました。	とう の だいひょう が しつもん に こたえました	
\\	の 代表[だいひょう]が 質問[しつもん]に 答[こた]えました。			
\\	対策	対策[たいさく]	たいさく	
\\	一緒に対策を考えましょう。	一緒[いっしょ]に 対策[たいさく]を 考[かんが]えましょう。	いっしょ に たいさく を かんがえましょう	
\\	一緒[いっしょ]に
\\	を 考[かんが]えましょう。			
\\	政策	政策[せいさく]	せいさく	
\\	新しい政策はあまり良いとは思えません。	新[あたら]しい 政策[せいさく]はあまり 良[い]いとは 思[おも]えません。	あたらしい せいさく は あまり いい と は おもえません	
\\	新[あたら]しい
\\	はあまり 良[い]いとは 思[おも]えません。			
\\	タイプライター	タイプライター	タイプライター	
\\	母は古いタイプライターを持っています。	母[はは]は 古[ふる]いタイプライターを 持[も]っています。	はは は ふるい たいぷらいたー を もって います	
\\	母[はは]は 古[ふる]い
\\	を 持[も]っています。			
\\	選ぶ	選[えら]ぶ	えらぶ	
\\	良い家を選ぶのは難しい。	良[よ]い 家[いえ]を 選[えら]ぶのは 難[むずか]しい。	よい いえ を えらぶ の は むずかしい	
\\	良[よ]い 家[いえ]を
\\	のは 難[むずか]しい。			
\\	選手	選手[せんしゅ]	せんしゅ	
\\	彼はプロのサッカー選手だ。	彼[かれ]はプロのサッカー 選手[せんしゅ]だ。	かれ は ぷろ の さっかー せんしゅ だ	
\\	彼[かれ]はプロのサッカー
\\	だ。			
\\	選挙	選挙[せんきょ]	せんきょ	
\\	彼は選挙に出るつもりだ。	彼[かれ]は 選挙[せんきょ]に 出[で]るつもりだ。	かれ は せんきょ に でる つもり だ	
\\	彼[かれ]は
\\	に 出[で]るつもりだ。			
\\	候補	候補[こうほ]	こうほ	
\\	会長の候補は3人います。	会長[かいちょう]の 候補[こうほ]は 3人[さんにん]います。	かいちょう の こうほ は さんにん います	
\\	会長[かいちょう]の
\\	は 3人[さんにん]います。			
\\	トマト	トマト	トマト	
\\	私はトマトが大好きです。	私[わたし]はトマトが 大好[だいす]きです。	わたし は とまと が だいすき です	
\\	私[わたし]は
\\	が 大好[だいす]きです。			
\\	首相	首相[しゅしょう]	しゅしょう	
\\	今の首相はあまり力がない。	今[いま]の 首相[しゅしょう]はあまり 力[ちから]がない。	いま の しゅしょう は あまり ちから が ない	
\\	今[いま]の
\\	はあまり 力[ちから]がない。			
\\	首都	首都[しゅと]	しゅと	
\\	東京は日本の首都です。	東京[とうきょう]は 日本[にっぽん]の 首都[しゅと]です。	とうきょう は にっぽん の しゅと です	
\\	東京[とうきょう]は 日本[にっぽん]の
\\	です。			
\\	改革	改革[かいかく]	かいかく	
\\	彼は行政を改革したいと思っている。	彼[かれ]は 行政[ぎょうせい]を 改革[かいかく]したいと 思[おも]っている。	かれ は ぎょうせい を かいかく したい と おもって いる	
\\	彼[かれ]は 行政[ぎょうせい]を
\\	したいと 思[おも]っている。			
\\	革	革[かわ]	かわ	
\\	革のベルトを買いました。	革[かわ]のベルトを 買[か]いました。	かわ の べると を かいました	
\\	のベルトを 買[か]いました。			
\\	バター	バター	バター	
\\	パンにバターをぬって食べました。	パンにバターをぬって 食[た]べました。	ぱん に ばたー を ぬって たべました	
\\	パンに
\\	をぬって 食[た]べました。			
\\	命令	命令[めいれい]	めいれい	
\\	彼女は命令に従わなかった。	彼女[かのじょ]は 命令[めいれい]に 従[したが]わなかった。	かのじょ は めいれい に したがわなかった	
\\	彼女[かのじょ]は
\\	に 従[したが]わなかった。			
\\	番組	番組[ばんぐみ]	ばんぐみ	
\\	私はこの番組が好きです。	私[わたし]はこの 番組[ばんぐみ]が 好[す]きです。	わたし は この ばんぐみ が すき です	
\\	私[わたし]はこの
\\	が 好[す]きです。			
\\	組み立てる	組[く]み 立[た]てる	くみたてる	
\\	日曜日に本棚を組み立てます。	日曜日[にちようび]に 本棚[ほんだな]を 組[く]み 立[た]てます。	にちようび に ほんだな を くみたてます	
\\	日曜日[にちようび]に 本棚[ほんだな]を
\\	組織	組織[そしき]	そしき	
\\	彼はある組織のリーダーだ。	彼[かれ]はある 組織[そしき]のリーダーだ。	かれ は ある そしき の りーだー だ	
\\	彼[かれ]はある
\\	のリーダーだ。			
\\	バレーボール	バレーボール	バレーボール	
\\	妹はバレーボールが得意です。	妹[いもうと]はバレーボールが 得意[とくい]です。	いもうと は ばれーぼーる が とくい です	
\\	妹[いもうと]は
\\	が 得意[とくい]です。			
\\	進める	進[すす]める	すすめる	
\\	早く授業を進めましょう。	早[はや]く 授業[じゅぎょう]を 進[すす]めましょう。	はやく じゅぎょう を すすめましょう	
\\	早[はや]く 授業[じゅぎょう]を
\\	進む	進[すす]む	すすむ	
\\	前に進んでください。	前[まえ]に 進[すす]んでください。	まえ に すすんで ください	
\\	前[まえ]に
\\	ください。			
\\	進学	進学[しんがく]	しんがく	
\\	彼女は来年、大学に進学する。	彼女[かのじょ]は 来年[らいねん]、 大学[だいがく]に 進学[しんがく]する。	かのじょ は らいねん だいがく に しんがく する	
\\	彼女[かのじょ]は 来年[らいねん]、 大学[だいがく]に
\\	する。			
\\	拡大	拡大[かくだい]	かくだい	
\\	この図を拡大コピーしてください。	この 図[ず]を 拡大[かくだい]コピーしてください。	この ず を かくだい こぴー して ください	
\\	この 図[ず]を
\\	コピーしてください。			
\\	おかず	おかず	おかず	
\\	晩ご飯のおかずは何?	晩[ばん]ご 飯[はん]のおかずは 何?[なに]	ばんごはん の おかず は なに	
\\	晩[ばん]ご 飯[はん]の
\\	は 何?[なに]			
\\	責任	責任[せきにん]	せきにん	
\\	彼は失敗の責任を取って、会社を辞めた。	彼[かれ]は 失敗[しっぱい]の 責任[せきにん]を 取[と]って、 会社[かいしゃ]を 辞[や]めた。	かれ は しっぱい の せきにん を とって かいしゃ を やめた	
\\	彼[かれ]は 失敗[しっぱい]の
\\	を 取[と]って、 会社[かいしゃ]を 辞[や]めた。			
\\	辞める	辞[や]める	やめる	
\\	彼は会社を辞めます。	彼[かれ]は 会社[かいしゃ]を 辞[や]めます。	かれ は かいしゃ を やめます	
\\	彼[かれ]は 会社[かいしゃ]を
\\	辞書	辞書[じしょ]	じしょ	
\\	彼女は辞書をよく使います。	彼女[かのじょ]は 辞書[じしょ]をよく 使[つか]います。	かのじょ は じしょ を よく つかいます	
\\	彼女[かのじょ]は
\\	をよく 使[つか]います。			
\\	通勤	通勤[つうきん]	つうきん	
\\	毎朝、通勤に30分かかります。	毎朝[まいあさ]、 通勤[つうきん]に 30分[さんじゅっぷん]かかります。	まいあさ つうきん に さんじゅっぷん かかります	
\\	毎朝[まいあさ]、
\\	に 30分[さんじゅっぷん]かかります。			
\\	カレンダー	カレンダー	カレンダー	
\\	カレンダーに予定を書いた。	カレンダーに 予定[よてい]を 書[か]いた。	かれんだー に よてい を かいた	
\\	に 予定[よてい]を 書[か]いた。			
\\	勤める	勤[つと]める	つとめる	
\\	私は銀行に勤めています。	私[わたし]は 銀行[ぎんこう]に 勤[つと]めています。	わたし は ぎんこう に つとめて います	
\\	私[わたし]は 銀行[ぎんこう]に
\\	事務所	事務所[じむしょ]	じむしょ	
\\	後で事務所に来てください。	後[あと]で 事務所[じむしょ]に 来[き]てください。	あとで じむしょ に きて ください	
\\	後[あと]で
\\	に 来[き]てください。			
\\	事務室	事務室[じむしつ]	じむしつ	
\\	事務室でコピーを取って来ます。	事務室[じむしつ]でコピーを 取[と]って 来[き]ます。	じむしつ で こぴー を とって きます	
\\	でコピーを 取[と]って 来[き]ます。			
\\	従来	従来[じゅうらい]	じゅうらい	
\\	このプリンターは従来のものより速い。交換することができますか?	このプリンターは 従来[じゅうらい]のものより 速[はや]い。	この ぷりんたー は じゅうらい の もの より はやい。こうかんすることができませんか	
\\	このプリンターは
\\	のものより 速[はや]い。			
\\	ハンドバッグ	ハンドバッグ	ハンドバッグ	
\\	ハンドバッグを忘れました。	ハンドバッグを 忘[わす]れました。	はんどばっぐ を わすれました	
\\	を 忘[わす]れました。			
\\	成績	成績[せいせき]	せいせき	
\\	成績が上がりました。	成績[せいせき]が 上[あ]がりました。	せいせき が あがりました	
\\	が 上[あ]がりました。			
\\	集める	集[あつ]める	あつめる	
\\	弟は切手を集めています。	弟[おとうと]は 切手[きって]を 集[あつ]めています。	おとうと は きって を あつめて います	
\\	弟[おとうと]は 切手[きって]を
\\	集まる	集[あつ]まる	あつまる	
\\	駅前に人が集まっています。	駅前[えきまえ]に 人[ひと]が 集[あつ]まっています。	えきまえ に ひと が あつまって います	
\\	駅前[えきまえ]に 人[ひと]が
\\	採用	採用[さいよう]	さいよう	
\\	その会社は女性を多く採用している。	その 会社[かいしゃ]は 女性[じょせい]を 多[おお]く 採用[さいよう]している。	その かいしゃ は じょせい を おおく さいよう して いる	
\\	その 会社[かいしゃ]は 女性[じょせい]を 多[おお]く
\\	している。			
\\	ベル	ベル	ベル	
\\	玄関のベルが鳴った。	玄関[げんかん]のベルが 鳴[な]った。	げんかん の べる が なった	
\\	玄関[げんかん]の
\\	が 鳴[な]った。			
\\	給料	給料[きゅうりょう]	きゅうりょう	
\\	来年から給料が上がります。	来年[らいねん]から 給料[きゅうりょう]が 上[あ]がります。	らいねん から きゅうりょう が あがります	
\\	来年[らいねん]から
\\	が 上[あ]がります。			
\\	卒業	卒業[そつぎょう]	そつぎょう	
\\	私は去年、大学を卒業した。	私[わたし]は 去年[きょねん]、 大学[だいがく]を 卒業[そつぎょう]した。	わたし は きょねん だいがく を そつぎょう した	
\\	私[わたし]は 去年[きょねん]、 大学[だいがく]を
\\	した。			
\\	就職	就職[しゅうしょく]	しゅうしょく	
\\	最近、若い人たちの就職が難しくなっています。	最近[さいきん]、 若[わか]い 人[ひと]たちの 就職[しゅうしょく]が 難[むずか]しくなっています。	さいきん わかい ひとたち の しゅうしょく が むずかしく なって います	
\\	最近[さいきん]、 若[わか]い 人[ひと]たちの
\\	が 難[むずか]しくなっています。			
\\	退院	退院[たいいん]	たいいん	
\\	母が今日退院します。	母[はは]が 今日[きょう] 退院[たいいん]します。	はは が きょう たいいん します	
\\	母[はは]が 今日[きょう]
\\	します。			
\\	インク	インク	インク	
\\	プリンターのインクを買った。	プリンターのインクを 買[か]った。	ぷりんたー の いんく を かった	
\\	プリンターの
\\	を 買[か]った。			
\\	契約	契約[けいやく]	けいやく	
\\	その選手は新しいチームと契約した。	その 選手[せんしゅ]は 新[あたら]しいチームと 契約[けいやく]した。	その せんしゅ は あたらしい ちーむ と けいやく した	
\\	その 選手[せんしゅ]は 新[あたら]しいチームと
\\	した。			
\\	交渉	交渉[こうしょう]	こうしょう	
\\	今、値段を交渉しています。	今[いま]、 値段[ねだん]を 交渉[こうしょう]しています。	いま ねだん を こうしょう して います	
\\	今[いま]、 値段[ねだん]を
\\	しています。			
\\	事件	事件[じけん]	じけん	
\\	その事件の犯人はまだ捕まっていない。	その 事件[じけん]の 犯人[はんにん]はまだ 捕[つか]まっていない。	その じけん の はんにん は まだ つかまって いない	
\\	その
\\	の 犯人[はんにん]はまだ 捕[つか]まっていない。			
\\	条件	条件[じょうけん]	じょうけん	
\\	この条件では厳し過ぎます。	この 条件[じょうけん]では 厳[きび]し 過[す]ぎます。	この じょうけん で は きびし すぎます	
\\	この
\\	では 厳[きび]し 過[す]ぎます。			
\\	クーラー	クーラー	クーラー	
\\	暑いのでクーラーをつけました。	暑[あつ]いのでクーラーをつけました。	あつい の で くーらー を つけました	
\\	暑[あつ]いので
\\	をつけました。			
\\	参加	参加[さんか]	さんか	
\\	明日は市民マラソンに参加します。	明日[あす]は 市民[しみん]マラソンに 参加[さんか]します。	あす は しみん まらそん に さんか します	
\\	明日[あす]は 市民[しみん]マラソンに
\\	します。			
\\	増加	増加[ぞうか]	ぞうか	
\\	島の人口は年々増加しています。	島[しま]の 人口[じんこう]は 年々[ねんねん] 増加[ぞうか]しています。	しま の じんこう は ねんねん ぞうか して います	
\\	島[しま]の 人口[じんこう]は 年々[ねんねん]
\\	しています。			
\\	加える	加[くわ]える	くわえる	
\\	塩、コショウを加えてください。	塩[しお]、コショウを 加[くわ]えてください。	しお こしょう を くわえて ください	
\\	塩[しお]、コショウを
\\	ください。			
\\	加工	加工[かこう]	かこう	
\\	この工場では魚を加工している。	この 工場[こうじょう]では 魚[さかな]を 加工[かこう]している。	この こうじょう で は さかな を かこう して いる	
\\	この 工場[こうじょう]では 魚[さかな]を
\\	している。			
\\	ミルク	ミルク	ミルク	
\\	赤ちゃんにミルクをあげた。	赤[あか]ちゃんにミルクをあげた。	あかちゃん に みるく を あげた	
\\	赤[あか]ちゃんに
\\	をあげた。			
\\	比べる	比[くら]べる	くらべる	
\\	今月と先月の売上を比べた。	今月[こんげつ]と 先月[せんげつ]の 売上[うりあげ]を 比[くら]べた。	こんげつ と せんげつ の うりあげ を くらべた	
\\	今月[こんげつ]と 先月[せんげつ]の 売上[うりあげ]を
\\	批判	批判[ひはん]	ひはん	
\\	、非難(ひなん)		
\\	彼は同僚を批判した。	彼[かれ]は 同僚[どうりょう]を 批判[ひはん]した。	かれ は どうりょう を ひはん した	
\\	彼[かれ]は 同僚[どうりょう]を
\\	した。			
\\	評価	評価[ひょうか]	ひょうか	
\\	最近、彼の評価が上がった。	最近[さいきん]、 彼[かれ]の 評価[ひょうか]が 上[あ]がった。	さいきん かれ の ひょうか が あがった	
\\	最近[さいきん]、 彼[かれ]の
\\	が 上[あ]がった。			
\\	対象	対象[たいしょう]	たいしょう	
\\	このアンケートは大学生が対象です。	このアンケートは 大学生[だいがくせい]が 対象[たいしょう]です。	この あんけーと は だいがくせい が たいしょう です	
\\	このアンケートは 大学生[だいがくせい]が
\\	です。			
\\	サラダ	サラダ	サラダ	
\\	サラダをたくさん食べました。	サラダをたくさん 食[た]べました。	さらだ を たくさん たべました	
\\	をたくさん 食[た]べました。			
\\	故障	故障[こしょう]	こしょう	
\\	冷蔵庫が故障しました。	冷蔵庫[れいぞうこ]が 故障[こしょう]しました。	れいぞうこ が こしょう しました	
\\	冷蔵庫[れいぞうこ]が
\\	しました。			
\\	修理	修理[しゅうり]	しゅうり	
\\	車を修理に出した。	車[くるま]を 修理[しゅうり]に 出[だ]した。	くるま を しゅうり に だした	
\\	車[くるま]を
\\	に 出[だ]した。			
\\	乗り換える	乗[の]り 換[か]える	のりかえる	
\\	次の駅で地下鉄に乗り換えます。	次[つぎ]の 駅[えき]で 地下鉄[ちかてつ]に 乗[の]り 換[か]えます。	つぎ の えき で ちかてつ に のりかえます	
\\	次[つぎ]の 駅[えき]で 地下鉄[ちかてつ]に
\\	乗り換え	乗[の]り 換[か]え	のりかえ	
\\	次の駅で乗り換えです。	次[つぎ]の 駅[えき]で 乗[の]り 換[か]えです。	つぎ の えき で のりかえ です	
\\	次[つぎ]の 駅[えき]で
\\	です。			
\\	おじさん	おじさん	おじさん	
\\	昨日おじさんに会いました。	昨日[きのう]おじさんに 会[あ]いました。	きのう おじさん に あいました	
\\	昨日[きのう]
\\	に 会[あ]いました。			
\\	換える	換[か]える	かえる	
\\	車のタイヤを換えた。	車[くるま]のタイヤを 換[か]えた。	くるま の たいや を かえた	
\\	車[くるま]のタイヤを
\\	着替える	着替[きが]える	きがえる	
\\	彼はきれいな服に着替えた。	彼[かれ]はきれいな 服[ふく]に 着替[きが]えた。	かれ は きれい な ふく に きがえた	
\\	彼[かれ]はきれいな 服[ふく]に
\\	被る	被[かぶ]る	かぶる	
\\	帽子を被って外出した。	帽子[ぼうし]を 被[かぶ]って 外出[がいしゅつ]した。	ぼうし を かぶって がいしゅつ した	
\\	帽子[ぼうし]を
\\	外出[がいしゅつ]した。			
\\	破る	破[やぶ]る	やぶる	
\\	彼は約束を破った。	彼[かれ]は 約束[やくそく]を 破[やぶ]った。	かれ は やくそく を やぶった	
\\	彼[かれ]は 約束[やくそく]を
\\	ベルト	ベルト	ベルト	
\\	ベルトがきつくなりました。	ベルトがきつくなりました。	べると が きつく なりました	
\\	がきつくなりました。			
\\	破れる	破[やぶ]れる	やぶれる	
\\	シャツが破れている。	シャツが 破[やぶ]れている。	しゃつ が やぶれて いる	
\\	シャツが
\\	壊す	壊[こわ]す	こわす	
\\	彼女が私のケータイを壊した。	彼女[かのじょ]が 私[わたし]のケータイを 壊[こわ]した。	かのじょ が わたし の けーたい を こわした	
\\	彼女[かのじょ]が 私[わたし]のケータイを
\\	壊れる	壊[こわ]れる	こわれる	
\\	会社のパソコンが壊れた。	会社[かいしゃ]のパソコンが 壊[こわ]れた。	かいしゃ の ぱそこん が こわれた	
\\	会社[かいしゃ]のパソコンが
\\	救急車	救急車[きゅうきゅうしゃ]	きゅうきゅうしゃ	
\\	誰か救急車を呼んでください。	誰[だれ]か 救急車[きゅうきゅうしゃ]を 呼[よ]んでください。	だれか きゅうきゅうしゃ を よんで ください	
\\	誰[だれ]か
\\	を 呼[よ]んでください。			
\\	ラーメン	ラーメン	ラーメン	
\\	夕食にラーメンを食べました。	夕食[ゆうしょく]にラーメンを 食[た]べました。	ゆうしょく に らーめん を たべました	
\\	夕食[ゆうしょく]に
\\	を 食[た]べました。			
\\	助ける	助[たす]ける	たすける	
\\	彼女は病気の犬を助けた。	彼女[かのじょ]は 病気[びょうき]の 犬[いぬ]を 助[たす]けた。	かのじょ は びょうき の いぬ を たすけた	
\\	彼女[かのじょ]は 病気[びょうき]の 犬[いぬ]を
\\	立派	立派[りっぱ]	りっぱ	
\\	彼は立派な人です。	彼[かれ]は 立派[りっぱ]な 人[ひと]です。	かれ は りっぱ な ひと です	
\\	彼[かれ]は
\\	な 人[ひと]です。			
\\	警察	警察[けいさつ]	けいさつ	
\\	警察を呼んでください。	警察[けいさつ]を 呼[よ]んでください。	けいさつ を よんで ください	
\\	を 呼[よ]んでください。			
\\	管理	管理[かんり]	かんり	
\\	彼女が私のスケジュールを管理している。	彼女[かのじょ]が 私[わたし]のスケジュールを 管理[かんり]している。	かのじょ が わたし の すけじゅーる を かんり して いる	
\\	彼女[かのじょ]が 私[わたし]のスケジュールを
\\	している。			
\\	ライター	ライター	ライター	
\\	店にライターを忘れました。	店[みせ]にライターを 忘[わす]れました。	みせ に らいたー を わすれました	
\\	店[みせ]に
\\	を 忘[わす]れました。			
\\	盗む	盗[ぬす]む	ぬすむ	
\\	誰かが私のカバンを盗みました。	誰[だれ]かが 私[わたし]のカバンを 盗[ぬす]みました。	だれか が わたし の かばん を ぬすみました	
\\	誰[だれ]かが 私[わたし]のカバンを
\\	殺す	殺[ころ]す	ころす	
\\	私は生き物を殺すのが嫌いだ。	私[わたし]は 生[い]き 物[もの]を 殺[ころ]すのが 嫌[きら]いだ。	わたし は いきもの を ころす の が きらい だ	
\\	私[わたし]は 生[い]き 物[もの]を
\\	のが 嫌[きら]いだ。			
\\	逮捕	逮捕[たいほ]	たいほ	
\\	逃げていた犯人が逮捕されました。	逃[に]げていた 犯人[はんにん]が 逮捕[たいほ]されました。	にげて いた はんにん が たいほ されました	
\\	逃[に]げていた 犯人[はんにん]が
\\	されました。			
\\	逃げる	逃[に]げる	にげる	
\\	鳥が窓から逃げたよ。	鳥[とり]が 窓[まど]から 逃[に]げたよ。	とり が まど から にげた よ	
\\	鳥[とり]が 窓[まど]から
\\	よ。			
\\	ワイシャツ	ワイシャツ	ワイシャツ	
\\	彼は白いワイシャツを着ている。	彼[かれ]は 白[しろ]いワイシャツを 着[き]ている。	かれ は しろい わいしゃつ を きて いる	
\\	彼[かれ]は 白[しろ]い
\\	を 着[き]ている。			
\\	戦争	戦争[せんそう]	せんそう	
\\	2003年にイラクで戦争があった。	2003年[にせんさんねん]にイラクで 戦争[せんそう]があった。	にせんさんねん に いらく で せんそう が あった	
\\	2003年[にせんさんねん]にイラクで
\\	があった。			
\\	競争	競争[きょうそう]	きょうそう	
\\	2社は互いに競争している。	2社[にしゃ]は 互[たが]いに 競争[きょうそう]している。	にしゃ は たがい に きょうそう して いる	
\\	2社[にしゃ]は 互[たが]いに
\\	している。			
\\	混む・込む	混[こ]む・ 込[こ]む	こむ・こむ	
\\	電車が込んでいる。	電車[でんしゃ]が 込[こ]んでいる。	でんしゃ が こんで いる	
\\	電車[でんしゃ]が
\\	大統領	大統領[だいとうりょう]	だいとうりょう	
\\	フランスの大統領は誰ですか。	フランスの 大統領[だいとうりょう]は 誰[だれ]ですか。	ふらんす の だいとうりょう は だれ です か	
\\	フランスの
\\	は 誰[だれ]ですか。			
\\	シャワー	シャワー	シャワー	
\\	朝、急いでシャワーを浴びました。	朝[あさ]、 急[いそ]いでシャワーを 浴[あ]びました。	あさ いそいで しゃわー を あびました	
\\	朝[あさ]、 急[いそ]いで
\\	を 浴[あ]びました。			
\\	捨てる	捨[す]てる	すてる	
\\	ゴミを捨ててください。	ゴミを 捨[す]ててください。	ごみ を すてて ください	
\\	ゴミを
\\	ください。			
\\	拾う	拾[ひろ]う	ひろう	
\\	道で財布を拾った。	道[みち]で 財布[さいふ]を 拾[ひろ]った。	みち で さいふ を ひろった	
\\	道[みち]で 財布[さいふ]を
\\	池	池[いけ]	いけ	
\\	池に鯉がいます。	池[いけ]に 鯉[こい]がいます。	いけ に こい が います	
\\	に 鯉[こい]がいます。			
\\	電池	電池[でんち]	でんち	
\\	新しい電池を入れましょう。	新[あたら]しい 電池[でんち]を 入[い]れましょう。	あたらしい でんち を いれましょう	
\\	新[あたら]しい
\\	を 入[い]れましょう。			
\\	うどん	うどん	うどん	
\\	私はお昼にうどんを食べました。	私[わたし]はお 昼[ひる]にうどんを 食[た]べました。	わたし は おひる に うどん を たべました	
\\	私[わたし]はお 昼[ひる]に
\\	を 食[た]べました。			
\\	深い	深[ふか]い	ふかい	
\\	あの池はとても深い。	あの 池[いけ]はとても 深[ふか]い。	あの いけ は とても ふかい	
\\	あの 池[いけ]はとても
\\	深さ	深[ふか]さ	ふかさ	
\\	このプールの深さは2メートルです。	このプールの 深[ふか]さは 2[に]メートルです。	この ぷーる の ふかさ は にめーとる です	
\\	このプールの
\\	は 2[に]メートルです。			
\\	浅い	浅[あさ]い	あさい	
\\	この川は浅いです。	この 川[かわ]は 浅[あさ]いです。	この かわ は あさい です	
\\	この 川[かわ]は
\\	です。			
\\	落とす	落[お]とす	おとす	
\\	途中で財布を落としました。	途中[とちゅう]で 財布[さいふ]を 落[お]としました。	とちゅう で さいふ を おとしました	
\\	途中[とちゅう]で 財布[さいふ]を
\\	ごみ	ごみ	ごみ	
\\	今日はごみの日だ。	今日[きょう]はごみの 日[ひ]だ。	きょう は ごみ の ひ だ	
\\	今日[きょう]は
\\	の 日[ひ]だ。			
\\	泳ぐ	泳[およ]ぐ	およぐ	
\\	彼女はダイエットのために泳いでいる。	彼女[かのじょ]はダイエットのために 泳[およ]いでいる。	かのじょ は だいえっと の ため に およいで いる	
\\	彼女[かのじょ]はダイエットのために
\\	水泳	水泳[すいえい]	すいえい	
\\	母は健康のために水泳をしている。	母[はは]は 健康[けんこう]のために 水泳[すいえい]をしている。	はは は けんこう の ため に すいえい を して いる	
\\	母[はは]は 健康[けんこう]のために
\\	をしている。			
\\	流れる	流[なが]れる	ながれる	
\\	ラジオから美しい音楽が流れています。	ラジオから 美[うつく]しい 音楽[おんがく]が 流[なが]れています。	らじお から うつくしい おんがく が ながれて います	
\\	ラジオから 美[うつく]しい 音楽[おんがく]が
\\	流行る	流行[はや]る	はやる	
\\	去年はスニーカーが流行りました。	去年[きょねん]はスニーカーが 流行[はや]りました。	きょねん は すにーかー が はやりました	
\\	去年[きょねん]はスニーカーが
\\	スーパーマーケット	スーパーマーケット	スーパーマーケット	
\\	スーパーマーケットで人参を買った。	スーパーマーケットで 人参[にんじん]を 買[か]った。	すーぱーまーけっと で にんじん を かった	
\\	で 人参[にんじん]を 買[か]った。			
\\	洗う	洗[あら]う	あらう	
\\	早く顔を洗いなさい。	早[はや]く 顔[かお]を 洗[あら]いなさい。	はやく かお を あらいなさい	
\\	早[はや]く 顔[かお]を
\\	洗面所	洗面所[せんめんじょ]	せんめんじょ	
\\	洗面所で顔を洗った。	洗面所[せんめんじょ]で 顔[かお]を 洗[あら]った。	せんめんじょ で かお を あらった	
\\	で 顔[かお]を 洗[あら]った。			
\\	油	油[あぶら]	あぶら	
\\	水と油は混ざらない。	水[みず]と 油[あぶら]は 混[ま]ざらない。	みず と あぶら は まざらない	
\\	水[みず]と
\\	は 混[ま]ざらない。			
\\	沈む	沈[しず]む	しずむ	
\\	ボートが川に沈んだ。	ボートが 川[かわ]に 沈[しず]んだ。	ぼーと が かわ に しずんだ	
\\	ボートが 川[かわ]に
\\	ハンカチ	ハンカチ	ハンカチ	
\\	ハンカチで手をふきました。	ハンカチで 手[て]をふきました。	はんかち で て を ふきました	
\\	で 手[て]をふきました。			
\\	久しぶり	久[ひさ]しぶり	ひさしぶり	
\\	明日、久しぶりに友達に会います。	明日[あした]、 久[ひさ]しぶりに 友達[ともだち]に 会[あ]います。	あした ひさしぶり に ともだち に あいます	
\\	明日[あした]、
\\	に 友達[ともだち]に 会[あ]います。			
\\	氷	氷[こおり]	こおり	
\\	グラスに氷を入れてください。	グラスに 氷[こおり]を 入[い]れてください。	ぐらす に こおり を いれて ください	
\\	グラスに
\\	を 入[い]れてください。			
\\	冷える	冷[ひ]える	ひえる	
\\	クーラーで体が冷えた。	クーラーで 体[からだ]が 冷[ひ]えた。	くーらー で からだ が ひえた	
\\	クーラーで 体[からだ]が
\\	冷やす	冷[ひ]やす	ひやす	
\\	頭を冷やしなさい。	頭[あたま]を 冷[ひ]やしなさい。	あたま を ひやしなさい	
\\	頭[あたま]を
\\	ビデオテープ	ビデオテープ	ビデオテープ	
\\	私はビデオテープを30本持っています。	私[わたし]はビデオテープを 30本持[さんじゅっぽん も]っています。	わたし は びでおてーぷ を さんじゅっぽん もって います	
\\	私[わたし]は
\\	を 30本持[さんじゅっぽん も]っています。			
\\	冷める	冷[さ]める	さめる	
\\	彼への気持ちが冷めた。	彼[かれ]への 気持[きも]ちが 冷[さ]めた。	かれ へ の きもち が さめた	
\\	彼[かれ]への 気持[きも]ちが
\\	凍る	凍[こお]る	こおる	
\\	寒い朝は道路が凍ります。	寒[さむ]い 朝[あさ]は 道路[どうろ]が 凍[こお]ります。	さむい あさ は どうろ が こおります	
\\	寒[さむ]い 朝[あさ]は 道路[どうろ]が
\\	冷蔵庫	冷蔵庫[れいぞうこ]	れいぞうこ	
\\	飲み物は冷蔵庫にあります。	飲[の]み 物[もの]は 冷蔵庫[れいぞうこ]にあります。	のみもの は れいぞうこ に あります	
\\	飲[の]み 物[もの]は
\\	にあります。			
\\	涼しい	涼[すず]しい	すずしい	
\\	夕方は涼しくなりますよ。	夕方[ゆうがた]は 涼[すず]しくなりますよ。	ゆうがた は すずしく なります よ	
\\	夕方[ゆうがた]は
\\	なりますよ。			
\\	ぶらぶら	ぶらぶら	ぶらぶら	
\\	朝、近所をぶらぶらした。	朝[あさ]、 近所[きんじょ]をぶらぶらした。	あさ きんじょ を ぶらぶら した	
\\	朝[あさ]、 近所[きんじょ]を
\\	した。			
\\	汚す	汚[よご]す	よごす	
\\	彼は服を汚した。	彼[かれ]は 服[ふく]を 汚[よご]した。	かれ は ふく を よごした	
\\	彼[かれ]は 服[ふく]を
\\	汚れ	汚[よご]れ	よごれ	
\\	靴の汚れを落としました。	靴[くつ]の 汚[よご]れを 落[お]としました。	くつ の よごれ を おとしました	
\\	靴[くつ]の
\\	を 落[お]としました。			
\\	汚れる	汚[よご]れる	よごれる	
\\	エプロンをしないと服が汚れます。	エプロンをしないと 服[ふく]が 汚[よご]れます。	えぷろん を しない と ふく が よごれます	
\\	エプロンをしないと 服[ふく]が
\\	景色	景色[けしき]	けしき	
\\	ここは景色がきれいですね。	ここは 景色[けしき]がきれいですね。	ここ は けしき が きれい です ね	
\\	ここは
\\	がきれいですね。			
\\	アナウンサー	アナウンサー	アナウンサー	
\\	私はアナウンサーになりたい。	私[わたし]はアナウンサーになりたい。	わたし は あなうんさー に なりたい	
\\	私[わたし]は
\\	になりたい。			
\\	影響	影響[えいきょう]	えいきょう	
\\	私は彼から大きな影響を受けました。	私[わたし]は 彼[かれ]から 大[おお]きな 影響[えいきょう]を 受[う]けました。	わたし は かれ から おおき な えいきょう を うけました	
\\	私[わたし]は 彼[かれ]から 大[おお]きな
\\	を 受[う]けました。			
\\	光る	光[ひか]る	ひかる	
\\	波がきらきら光っていました。	波[なみ]がきらきら 光[ひか]っていました。	なみ が きらきら ひかって いました	
\\	波[なみ]がきらきら
\\	太る	太[ふと]る	ふとる	
\\	私の姉はすぐ太ります。	私[わたし]の 姉[あね]はすぐ 太[ふと]ります。	わたし の あね は すぐ ふとります	
\\	私[わたし]の 姉[あね]はすぐ
\\	太陽	太陽[たいよう]	たいよう	
\\	太陽が雲に隠れた。	太陽[たいよう]が 雲[くも]に 隠[かく]れた。	たいよう が くも に かくれた	
\\	が 雲[くも]に 隠[かく]れた。			
\\	スプーン	スプーン	スプーン	
\\	カレーライスはスプーンで食べます。	カレーライスはスプーンで 食[た]べます。	かれーらいす は すぷーん で たべます	
\\	カレーライスは
\\	で 食[た]べます。			
\\	星	星[ほし]	ほし	
\\	今夜は星がよく見えます。	今夜[こんや]は 星[ほし]がよく 見[み]えます。	こんや は ほし が よく みえます	
\\	今夜[こんや]は
\\	がよく 見[み]えます。			
\\	地球	地球[ちきゅう]	ちきゅう	
\\	地球は丸い。	地球[ちきゅう]は 丸[まる]い。	ちきゅう は まるい	
\\	は 丸[まる]い。			
\\	野球	野球[やきゅう]	やきゅう	
\\	友達と野球をしました。	友達[ともだち]と 野球[やきゅう]をしました。	ともだち と やきゅう を しました	
\\	友達[ともだち]と
\\	をしました。			
\\	雲	雲[くも]	くも	
\\	今日は雲が多い。	今日[きょう]は 雲[くも]が 多[おお]い。	きょう は くも が おおい	
\\	今日[きょう]は
\\	が 多[おお]い。			
\\	ブラウス	ブラウス	ブラウス	
\\	彼女は白いブラウスを着ている。	彼女[かのじょ]は 白[しろ]いブラウスを 着[き]ている。	かのじょ は しろい ぶらうす を きて いる	
\\	彼女[かのじょ]は 白[しろ]い
\\	を 着[き]ている。			
\\	曇り	曇[くも]り	くもり	
\\	今日は一日曇りでした。	今日[きょう]は 一日[いちにち] 曇[くも]りでした。	きょう は いちにち くもり でした	
\\	今日[きょう]は 一日[いちにち]
\\	でした。			
\\	地震	地震[じしん]	じしん	
\\	日本は地震が多いです。	日本[にほん]は 地震[じしん]が 多[おお]いです。	にほん は じしん が おおい です	
\\	日本[にほん]は
\\	が 多[おお]いです。			
\\	震える	震[ふる]える	ふるえる	
\\	彼は寒くて震えていました。	彼[かれ]は 寒[さむ]くて 震[ふる]えていました。	かれ は さむくて ふるえて いました	
\\	彼[かれ]は 寒[さむ]くて
\\	振る	振[ふ]る	ふる	
\\	犬がしっぽを振っている。	犬[いぬ]がしっぽを 振[ふ]っている。	いぬ が しっぽ を ふって いる	
\\	犬[いぬ]がしっぽを
\\	リボン	リボン	リボン	
\\	プレゼントにリボンを付けた。	プレゼントにリボンを 付[つ]けた。	ぷれぜんと に りぼん を つけた	
\\	プレゼントに
\\	を 付[つ]けた。			
\\	揺れる	揺[ゆ]れる	ゆれる	
\\	風で木が揺れています。	風[かぜ]で 木[き]が 揺[ゆ]れています。	かぜ で き が ゆれて います	
\\	風[かぜ]で 木[き]が
\\	います。			
\\	神社	神社[じんじゃ]	じんじゃ	
\\	京都には神社がたくさんある。	京都[きょうと]には 神社[じんじゃ]がたくさんある。	きょうと に は じんじゃ が たくさん ある	
\\	京都[きょうと]には
\\	がたくさんある。			
\\	秘密	秘密[ひみつ]	ひみつ	
\\	これは秘密です。	これは 秘密[ひみつ]です。	これ は ひみつ です	
\\	これは
\\	です。			
\\	厳しい	厳[きび]しい	きびしい	
\\	私の上司はとても厳しい。	私[わたし]の 上司[じょうし]はとても 厳[きび]しい。	わたし の じょうし は とても きびしい	
\\	私[わたし]の 上司[じょうし]はとても
\\	いかが	いかが	いかが	
\\	お加減はいかがですか。	お 加減[かげん]はいかがですか。	おかげん は いかが です か	
\\	お 加減[かげん]は
\\	ですか。			
\\	年寄り	年寄[としよ]り	としより	
\\	あの村にはお年寄りがたくさん住んでいます。	あの 村[むら]にはお 年寄[としよ]りがたくさん 住[す]んでいます。	あの むら に は おとしより が たくさん すんで います	
\\	あの 村[むら]にはお
\\	がたくさん 住[す]んでいます。			
\\	歴史	歴史[れきし]	れきし	
\\	私は歴史に興味があります。	私[わたし]は 歴史[れきし]に 興味[きょうみ]があります。	わたし は れきし に きょうみ が あります	
\\	私[わたし]は
\\	に 興味[きょうみ]があります。			
\\	世紀	世紀[せいき]	せいき	
\\	新しい世紀の始まりです。	新[あたら]しい 世紀[せいき]の 始[はじ]まりです。	あたらしい せいき の はじまり です	
\\	新[あたら]しい
\\	の 始[はじ]まりです。			
\\	建設	建設[けんせつ]	けんせつ	
\\	新しいビルの建設が始まった。	新[あたら]しいビルの 建設[けんせつ]が 始[はじ]まった。	あたらしい びる の けんせつ が はじまった	
\\	新[あたら]しいビルの
\\	が 始[はじ]まった。			
\\	ステレオ	ステレオ	ステレオ	
\\	彼はステレオで音楽を聞いた。	彼[かれ]はステレオで 音楽[おんがく]を 聞[き]いた。	かれ は すてれお で おんがく を きいた	
\\	彼[かれ]は
\\	で 音楽[おんがく]を 聞[き]いた。			
\\	建物	建物[たてもの]	たてもの	
\\	これは日本一古い建物です。	これは 日本一古[にほんいち ふる]い 建物[たてもの]です。	これ は にほんいち ふるい たてもの です	
\\	これは 日本一古[にほんいち ふる]い
\\	です。			
\\	建つ	建[た]つ	たつ	
\\	ここに来年、家が建ちます。	ここに 来年[らいねん]、 家[いえ]が 建[た]ちます。	ここ に らいねん いえ が たちます	
\\	ここに 来年[らいねん]、 家[いえ]が
\\	構成	構成[こうせい]	こうせい	
\\	'で=
\\	システムの構成を変えてみました。	システムの 構成[こうせい]を 変[か]えてみました。	しすてむ の こうせい を かえて みました	
\\	システムの
\\	を 変[か]えてみました。			
\\	構造	構造[こうぞう]	こうぞう	
\\	この建物の構造は複雑です。	この 建物[たてもの]の 構造[こうぞう]は 複雑[ふくざつ]です。	この たてもの の こうぞう は ふくざつ です	
\\	この 建物[たてもの]の
\\	は 複雑[ふくざつ]です。			
\\	フォーク	フォーク	フォーク	
\\	フォークをもらえますか。	フォークをもらえますか。	ふぉーく を もらえます か	
\\	をもらえますか。			
\\	橋	橋[はし]	はし	
\\	あの橋は日本で一番長い。	あの 橋[はし]は 日本[にほん]で 一番長[いちばん なが]い。	あの はし は にほん で いちばん ながい	
\\	あの
\\	は 日本[にほん]で 一番長[いちばん なが]い。			
\\	柱	柱[はしら]	はしら	
\\	この家の柱は太い。	この 家[いえ]の 柱[はしら]は 太[ふと]い。	この いえ の はしら は ふとい	
\\	この 家[いえ]の
\\	は 太[ふと]い。			
\\	位置	位置[いち]	いち	
\\	私の町は東京の北に位置します。	私[わたし]の 町[まち]は 東京[とうきょう]の 北[きた]に 位置[いち]します。	わたし の まち は とうきょう の きた に いち します	
\\	私[わたし]の 町[まち]は 東京[とうきょう]の 北[きた]に
\\	します。			
\\	離婚	離婚[りこん]	りこん	
\\	友人が離婚しました。	友人[ゆうじん]が 離婚[りこん]しました。	ゆうじん が りこん しました	
\\	友人[ゆうじん]が
\\	しました。			
\\	おば	おば	おば	
\\	おばは大阪に住んでいます。	おばは 大阪[おおさか]に 住[す]んでいます。	おば は おおさか に すんで います	
\\	は 大阪[おおさか]に 住[す]んでいます。			
\\	停車	停車[ていしゃ]	ていしゃ	
\\	この電車は東京まで停車致しません。	この 電車[でんしゃ]は 東京[とうきょう]まで 停車[ていしゃ] 致[いた]しません。	この でんしゃ は とうきょう まで ていしゃ いたしません	
\\	この 電車[でんしゃ]は 東京[とうきょう]まで
\\	致[いた]しません。			
\\	バス停	バス 停[てい]	ばすてい	
\\	次のバス停で降ります。	次[つぎ]のバス 停[てい]で 降[お]ります。	つぎ の ばすてい で おります	
\\	次[つぎ]の
\\	で 降[お]ります。			
\\	周辺	周辺[しゅうへん]	しゅうへん	
\\	この周辺には大学が多い。	この 周辺[しゅうへん]には 大学[だいがく]が 多[おお]い。	この しゅうへん に は だいがく が おおい	
\\	この
\\	には 大学[だいがく]が 多[おお]い。			
\\	隣	隣[となり]	となり	
\\	隣の家には犬がいます。	隣[となり]の 家[いえ]には 犬[いぬ]がいます。	となり の いえ に は いぬ が います	
\\	の 家[いえ]には 犬[いぬ]がいます。			
\\	きらきら	きらきら	きらきら	
\\	星がきらきら光っている。	星[ほし]がきらきら 光[ひか]っている。	ほし が きらきら ひかって いる	
\\	星[ほし]が
\\	光[ひか]っている。			
\\	黄色	黄色[きいろ]	きいろ	
\\	信号は黄色でした。	信号[しんごう]は 黄色[きいろ]でした。	しんごう は きいろ でした	
\\	信号[しんごう]は
\\	でした。			
\\	横	横[よこ]	よこ	
\\	横の長さは1メートルです。	横[よこ]の 長[なが]さは 1[いち]メートルです。	よこ の ながさ は いちめーとる です	
\\	の 長[なが]さは 1[いち]メートルです。			
\\	横書き	横書[よこが]き	よこがき	
\\	この本は横書きです。	この 本[ほん]は 横書[よこが]きです。	この ほん は よこがき です	
\\	この 本[ほん]は
\\	です。			
\\	判断	判断[はんだん]	はんだん	
\\	彼の判断は正しい。	彼[かれ]の 判断[はんだん]は 正[ただ]しい。	かれ の はんだん は ただしい	
\\	彼[かれ]の
\\	は 正[ただ]しい。			
\\	どうぞ	どうぞ	どうぞ	
\\	こちらへどうぞ。	こちらへどうぞ。	こちらへどうぞ。	
\\	こちらへ
\\	断る	断[ことわ]る	ことわる	
\\	私は彼のプロポーズを断った。	私[わたし]は 彼[かれ]のプロポーズを 断[ことわ]った。	わたし は かれ の ぷろぽーず を ことわった	
\\	私[わたし]は 彼[かれ]のプロポーズを
\\	横断歩道	横断歩道[おうだんほどう]	おうだんほどう	
\\	あそこに横断歩道があります。	あそこに 横断歩道[おうだんほどう]があります。	あそこ に おうだんほどう が あります	
\\	あそこに
\\	があります。			
\\	大幅	大幅[おおはば]	おおはば	
\\	計画を大幅に変更した。	計画[けいかく]を 大幅[おおはば]に 変更[へんこう]した。	けいかく を おおはば に へんこう した	
\\	計画[けいかく]を
\\	に 変更[へんこう]した。			
\\	訪れる	訪[おとず]れる	おとずれる	
\\	私は夏に京都を訪れました。	私[わたし]は 夏[なつ]に 京都[きょうと]を 訪[おとず]れました。	わたし は なつ に きょうと を おとずれました	
\\	私[わたし]は 夏[なつ]に 京都[きょうと]を
\\	ネックレス	ネックレス	ネックレス	
\\	妻にネックレスをプレゼントしました。	妻[つま]にネックレスをプレゼントしました。	つま に ねっくれす を ぷれぜんと しました	
\\	妻[つま]に
\\	をプレゼントしました。			
\\	訪ねる	訪[たず]ねる	たずねる	
\\	彼はニューヨークの友達を訪ねた。	彼[かれ]はニューヨークの 友達[ともだち]を 訪[たず]ねた。	かれ は にゅーよーく の ともだち を たずねた	
\\	彼[かれ]はニューヨークの 友達[ともだち]を
\\	冷房	冷房[れいぼう]	れいぼう	
\\	冷房を入れてください。	冷房[れいぼう]を 入[い]れてください。	れいぼう を いれて ください	
\\	を 入[い]れてください。			
\\	暖房	暖房[だんぼう]	だんぼう	
\\	冬は暖房が必要です。	冬[ふゆ]は 暖房[だんぼう]が 必要[ひつよう]です。	ふゆ は だんぼう が ひつよう です	
\\	冬[ふゆ]は
\\	が 必要[ひつよう]です。			
\\	文房具	文房具[ぶんぼうぐ]	ぶんぼうぐ	
\\	新しい文房具を買いました。	新[あたら]しい 文房具[ぶんぼうぐ]を 買[か]いました。	あたらしい ぶんぼうぐ を かいました	
\\	新[あたら]しい
\\	を 買[か]いました。			
\\	パジャマ	パジャマ	パジャマ	
\\	このパジャマを着て。	このパジャマを 着[き]て。	この ぱじゃま を きて	
\\	この
\\	を 着[き]て。			
\\	諸国	諸国[しょこく]	しょこく	
\\	彼はヨーロッパ諸国を旅行した。	彼[かれ]はヨーロッパ 諸国[しょこく]を 旅行[りょこう]した。	かれ は よーろっぱ しょこく を りょこう した	
\\	彼[かれ]はヨーロッパ
\\	を 旅行[りょこう]した。			
\\	緑色	緑色[みどりいろ]	みどりいろ	
\\	彼は緑色のジャケットを着ています。	彼[かれ]は 緑色[みどりいろ]のジャケットを 着[き]ています。	かれ は みどりいろ の じゃけっと を きて います	
\\	彼[かれ]は
\\	のジャケットを 着[き]ています。			
\\	貿易	貿易[ぼうえき]	ぼうえき	
\\	父は貿易の仕事をしています。	父[ちち]は 貿易[ぼうえき]の 仕事[しごと]をしています。	ちち は ぼうえき の しごと を して います	
\\	父[ちち]は
\\	の 仕事[しごと]をしています。			
\\	輸入	輸入[ゆにゅう]	ゆにゅう	
\\	これはイタリアから輸入した服です。	これはイタリアから 輸入[ゆにゅう]した 服[ふく]です。	これ は いたりあ から ゆにゅう した ふく です	
\\	これはイタリアから
\\	した 服[ふく]です。			
\\	ボールペン	ボールペン	ボールペン	
\\	ボールペンで名前を書いてください。	ボールペンで 名前[なまえ]を 書[か]いてください。	ぼーるぺん で なまえ を かいて ください	
\\	で 名前[なまえ]を 書[か]いてください。			
\\	輸出	輸出[ゆしゅつ]	ゆしゅつ	
\\	彼の会社は車を輸出しています。	彼[かれ]の 会社[かいしゃ]は 車[くるま]を 輸出[ゆしゅつ]しています。	かれ の かいしゃ は くるま を ゆしゅつ して います	
\\	彼[かれ]の 会社[かいしゃ]は 車[くるま]を
\\	しています。			
\\	指輪	指輪[ゆびわ]	ゆびわ	
\\	彼女に指輪をプレゼントしました。	彼女[かのじょ]に 指輪[ゆびわ]をプレゼントしました。	かのじょ に ゆびわ を ぷれぜんと しました	
\\	彼女[かのじょ]に
\\	をプレゼントしました。			
\\	往復	往復[おうふく]	おうふく	
\\	往復切符をください。	往復[おうふく] 切符[きっぷ]をください。	おうふく きっぷ を ください	
\\	切符[きっぷ]をください。			
\\	復習	復習[ふくしゅう]	ふくしゅう	
\\	昨日の復習をしましたか。	昨日[きのう]の 復習[ふくしゅう]をしましたか。	きのう の ふくしゅう を しました か	
\\	昨日[きのう]の
\\	をしましたか。			
\\	すいか	すいか	すいか	
\\	夏はすいかが美味しい。	夏[なつ]はすいかが 美味[おい]しい。	なつ は すいか が おいしい	
\\	夏[なつ]は
\\	が 美味[おい]しい。			
\\	繰り返す	繰[く]り 返[かえ]す	くりかえす	
\\	彼女は同じ間違いを繰り返した。	彼女[かのじょ]は 同[おな]じ 間違[まちが]いを 繰[く]り 返[かえ]した。	かのじょ は おなじ まちがい を くりかえした	
\\	彼女[かのじょ]は 同[おな]じ 間違[まちが]いを
\\	留学	留学[りゅうがく]	りゅうがく	
\\	1年間、アメリカに留学しました。	1年間[いちねんかん]、アメリカに 留学[りゅうがく]しました。	いちねんかん あめりか に りゅうがく しました	
\\	1年間[いちねんかん]、アメリカに
\\	しました。			
\\	停留所	停留所[ていりゅうじょ]	ていりゅうじょ	
\\	バスの停留所で10分待ちました。	バスの 停留所[ていりゅうじょ]で 10分待[じゅっぷん ま]ちました。	ばす の ていりゅうじょ で じゅっぷん まちました	
\\	バスの
\\	で 10分待[じゅっぷん ま]ちました。			
\\	書留	書留[かきとめ]	かきとめ	
\\	これを書留で送りたいのですが。	これを 書留[かきとめ]で 送[おく]りたいのですが。	これ を かきとめ で おくりたい の です が	
\\	これを
\\	で 送[おく]りたいのですが。			
\\	そちら	そちら	そちら	
\\	夕方そちらに着きます。	夕方[ゆうがた]そちらに 着[つ]きます。	ゆうがた そちら に つきます	
\\	夕方[ゆうがた]
\\	に 着[つ]きます。			
\\	守る	守[まも]る	まもる	
\\	彼は約束を守る人です。	彼[かれ]は 約束[やくそく]を 守[まも]る 人[ひと]です。	かれ は やくそく を まもる ひと です	
\\	彼[かれ]は 約束[やくそく]を
\\	人[ひと]です。			
\\	留守	留守[るす]	るす	
\\	父は今、留守です。	父[ちち]は 今[いま]、 留守[るす]です。	ちち は いま るす です	
\\	父[ちち]は 今[いま]、
\\	です。			
\\	住宅	住宅[じゅうたく]	じゅうたく	
\\	ここは静かな住宅地だ。	ここは 静[しず]かな 住宅[じゅうたく] 地[ち]だ。	ここ は しずかな じゅうたくち だ	
\\	ここは 静[しず]かな
\\	地[ち]だ。			
\\	自宅	自宅[じたく]	じたく	
\\	自宅に電話を下さい。	自宅[じたく]に 電話[でんわ]を 下[くだ]さい。	じたく に でんわ を ください	
\\	に 電話[でんわ]を 下[くだ]さい。			
\\	そば	そば	そば	
\\	お昼にそばを食べた。	お 昼[ひる]にそばを 食[た]べた。	おひる に そば を たべた	
\\	お 昼[ひる]に
\\	を 食[た]べた。			
\\	お宅	お 宅[たく]	おたく	
\\	先生のお宅はどちらですか。	先生[せんせい]のお 宅[たく]はどちらですか。	せんせい の おたく は どちら です か	
\\	先生[せんせい]の
\\	はどちらですか。			
\\	早起き	早起[はやお]き	はやおき	
\\	祖父は早起きです。	祖父[そふ]は 早起[はやお]きです。	そふ は はやおき です	
\\	祖父[そふ]は
\\	です。			
\\	昼寝	昼寝[ひるね]	ひるね	
\\	私の子供は毎日昼寝をします。	私[わたし]の 子供[こども]は 毎日[まいにち] 昼寝[ひるね]をします。	わたし の こども は まいにち ひるね を します	
\\	私[わたし]の 子供[こども]は 毎日[まいにち]
\\	をします。			
\\	静か	静[しず]か	しずか	
\\	今年の夏休みには静かな所へ行きたい。	今年[ことし]の 夏休[なつやす]みには 静[しず]かな 所[ところ]へ 行[い]きたい。	ことし の なつやすみ に は しずか な ところ へ いきたい	
\\	今年[ことし]の 夏休[なつやす]みには
\\	な 所[ところ]へ 行[い]きたい。			
\\	どっち	どっち	どっち	
\\	どっちの色が好きですか。	どっちの 色[いろ]が 好[す]きですか。	どっち の いろ が すき です か 。	
\\	の 色[いろ]が 好[す]きですか。			
\\	暇	暇[ひま]	ひま	
\\	明日は暇ですか。	明日[あした]は 暇[ひま]ですか。	あした は ひま です か	
\\	明日[あした]は
\\	ですか。			
\\	趣味	趣味[しゅみ]	しゅみ	
\\	私の趣味は映画とテニスです。	私[わたし]の 趣味[しゅみ]は 映画[えいが]とテニスです。	わたし の しゅみ は えいが と てにす です	
\\	私[わたし]の
\\	は 映画[えいが]とテニスです。			
\\	両方	両方[りょうほう]	りょうほう	
\\	チョコレートとケーキを両方ください。	チョコレートとケーキを 両方[りょうほう]ください。	ちょこれーと と けーき を りょうほう ください	
\\	チョコレートとケーキを
\\	ください。			
\\	両替	両替[りょうがえ]	りょうがえ	
\\	1万円を両替してください。	1万円[いちまんえん]を 両替[りょうがえ]してください。	いちまんえん を りょうがえ して ください	
\\	1万円[いちまんえん]を
\\	してください。			
\\	バスケットボール	バスケットボール	バスケットボール	
\\	姉はバスケットボールの選手です。	姉[あね]はバスケットボールの 選手[せんしゅ]です。	あね は ばすけっとぼーる の せんしゅ です	
\\	姉[あね]は
\\	の 選手[せんしゅ]です。			
\\	両親	両親[りょうしん]	りょうしん	
\\	私の両親は大阪に住んでいます。	私[わたし]の 両親[りょうしん]は 大阪[おおさか]に 住[す]んでいます。	わたし の りょうしん は おおさか に すんで います	
\\	私[わたし]の
\\	は 大阪[おおさか]に 住[す]んでいます。			
\\	片道	片道[かたみち]	かたみち	
\\	東京まで片道切符を買った。	東京[とうきょう]まで 片道[かたみち] 切符[きっぷ]を 買[か]った。	とうきょう まで かたみち きっぷ を かった	
\\	東京[とうきょう]まで
\\	切符[きっぷ]を 買[か]った。			
\\	内側	内側[うちがわ]	うちがわ	
\\	白線の内側に下がってください。	白線[はくせん]の 内側[うちがわ]に 下[さ]がってください。	はくせん の うちがわ に さがって ください	
\\	白線[はくせん]の
\\	に 下[さ]がってください。			
\\	向こう側	向[む]こう 側[がわ]	むこうがわ	
\\	私の家は川の向こう側にあります。	私[わたし]の 家[いえ]は 川[かわ]の 向[む]こう 側[がわ]にあります。	わたし の いえ は かわ の むこうがわ に あります	
\\	私[わたし]の 家[いえ]は 川[かわ]の
\\	にあります。			
\\	サンドイッチ	サンドイッチ	サンドイッチ	
\\	今日のお昼はサンドイッチです。	今日[きょう]のお 昼[ひる]はサンドイッチです。	きょう の おひる は さんどいっち です	
\\	今日[きょう]のお 昼[ひる]は
\\	です。			
\\	外側	外側[そとがわ]	そとがわ	
\\	白線の外側を歩かないでください。	白線[はくせん]の 外側[そとがわ]を 歩[ある]かないでください。	はくせん の そとがわ を あるかない で ください	
\\	白線[はくせん]の
\\	を 歩[ある]かないでください。			
\\	左側	左側[ひだりがわ]	ひだりがわ	
\\	画面の左側を見てください。	画面[がめん]の 左側[ひだりがわ]を 見[み]てください。	がめん の ひだりがわ を みて ください	
\\	画面[がめん]の
\\	を 見[み]てください。			
\\	右側	右側[みぎがわ]	みぎがわ	
\\	彼女はいつも私の右側を歩きます。	彼女[かのじょ]はいつも 私[わたし]の 右側[みぎがわ]を 歩[ある]きます。	かのじょ は いつも わたし の みぎがわ を あるきます	
\\	彼女[かのじょ]はいつも 私[わたし]の
\\	を 歩[ある]きます。			
\\	裏	裏[うら]	うら	
\\	会社の裏に公園があります。	会社[かいしゃ]の 裏[うら]に 公園[こうえん]があります。	かいしゃ の うら に こうえん が あります	
\\	会社[かいしゃ]の
\\	に 公園[こうえん]があります。			
\\	にこにこ	にこにこ	にこにこ	
\\	彼女はいつもにこにこしています。	彼女[かのじょ]はいつもにこにこしています。	かのじょ は いつも にこにこ して います	
\\	彼女[かのじょ]はいつも
\\	しています。			
\\	裏返す	裏返[うらがえ]す	うらがえす	
\\	彼は紙を裏返しました。	彼[かれ]は 紙[かみ]を 裏返[うらがえ]しました。	かれ は かみ を うらがえしました	
\\	彼[かれ]は 紙[かみ]を
\\	週刊誌	週刊誌[しゅうかんし]	しゅうかんし	
\\	その週刊誌はゴシップばかりだ。	その 週刊誌[しゅうかんし]はゴシップばかりだ。	その しゅうかんし は ごしっぷ ばかり だ	
\\	その
\\	はゴシップばかりだ。			
\\	朝刊	朝刊[ちょうかん]	ちょうかん	
\\	今日の朝刊に面白い記事があった。	今日[きょう]の 朝刊[ちょうかん]に 面白[おもしろ]い 記事[きじ]があった。	きょう の ちょうかん に おもしろい きじ が あった	
\\	今日[きょう]の
\\	に 面白[おもしろ]い 記事[きじ]があった。			
\\	夕刊	夕刊[ゆうかん]	ゆうかん	
\\	そのニュースは夕刊で見ました。	そのニュースは 夕刊[ゆうかん]で 見[み]ました。	その にゅーす は ゆうかん で みました	
\\	そのニュースは
\\	で 見[み]ました。			
\\	ウール	ウール	ウール	
\\	このウールのセーターは暖かい。	このウールのセーターは 暖[あたた]かい。	この うーる の せーたー は あたたかい	
\\	この
\\	のセーターは 暖[あたた]かい。			
\\	詳しい	詳[くわ]しい	くわしい	
\\	もっと詳しく説明してください。	もっと 詳[くわ]しく 説明[せつめい]してください。	もっと くわしく せつめい して ください	
\\	もっと
\\	説明[せつめい]してください。			
\\	細かい	細[こま]かい	こまかい	
\\	彼女は細かいことにうるさい。	彼女[かのじょ]は 細[こま]かいことにうるさい。	かのじょ は こまかい こと に うるさい	
\\	彼女[かのじょ]は
\\	ことにうるさい。			
\\	細か	細[こま]か	こまか	
\\	それは細かな問題です。	それは 細[こま]かな 問題[もんだい]です。	それ は こまか な もんだい です	
\\	それは
\\	な 問題[もんだい]です。			
\\	積もる	積[つ]もる	つもる	
\\	雪が積もっています。	雪[ゆき]が 積[つ]もっています。	ゆき が つもって います	
\\	雪[ゆき]が
\\	コーラ	コーラ	コーラ	
\\	私の兄はコーラが好きだ。	私[わたし]の 兄[あに]はコーラが 好[す]きだ。	わたし の あに は こーら が すき だ	
\\	私[わたし]の 兄[あに]は
\\	が 好[す]きだ。			
\\	訳	訳[やく]	やく	
\\	その文の訳を読みました。	その 文[ぶん]の 訳[やく]を 読[よ]みました。	その ぶん の やく を よみました	
\\	その 文[ぶん]の
\\	を 読[よ]みました。			
\\	訳す	訳[やく]す	やくす	
\\	この文を訳してください。	この 文[ぶん]を 訳[やく]してください。	この ぶん を やくして ください	
\\	この 文[ぶん]を
\\	ください。			
\\	検討	検討[けんとう]	けんとう	
\\	今日中にこの問題を検討してください。	今日中[きょう じゅう]にこの 問題[もんだい]を 検討[けんとう]してください。	きょう じゅう に この もんだい を けんとう して ください	
\\	今日中[きょう じゅう]にこの 問題[もんだい]を
\\	してください。			
\\	塗る	塗[ぬ]る	ぬる	
\\	壁にペンキを塗っています。	壁[かべ]にペンキを 塗[ぬ]っています。	かべ に ぺんき を ぬって います	
\\	壁[かべ]にペンキを
\\	トイレットペーパー	トイレットペーパー	トイレットペーパー	
\\	トイレットペーパーを交換しました。	トイレットペーパーを 交換[こうかん]しました。	といれっとぺーぱー を こうかん しました	
\\	を 交換[こうかん]しました。			
\\	付く	付[つ]く	つく	
\\	靴に泥が付いています。	靴[くつ]に 泥[どろ]が 付[つ]いています。	くつ に どろ が ついて います	
\\	靴[くつ]に 泥[どろ]が
\\	付ける	付[つ]ける	つける	
\\	おまけを付けました。	おまけを 付[つ]けました。	おまけ を つけました	
\\	おまけを
\\	受け付ける	受[う]け 付[つ]ける	うけつける	
\\	郵便物は5時まで受け付けています。	郵便物[ゆうびんぶつ]は 5時[ごじ]まで 受[う]け 付[つ]けています。	ゆうびんぶつ は ごじ まで うけつけて います	
\\	郵便物[ゆうびんぶつ]は 5時[ごじ]まで
\\	片付ける	片付[かたづ]ける	かたづける	
\\	早く部屋を片付けなさい。	早[はや]く 部屋[へや]を 片付[かたづ]けなさい。	はやく へや を かたづけなさい	
\\	早[はや]く 部屋[へや]を
\\	どなた	どなた	どなた	
\\	失礼ですが、どなたですか。	失礼[しつれい]ですが、どなたですか。	しつれい です が 、 どなた です か 。	
\\	失礼[しつれい]ですが、
\\	ですか。			
\\	受付	受付[うけつけ]	うけつけ	
\\	受付は9時からです。	受付[うけつけ]は 9時[くじ]からです。	うけつけ は くじ から です	
\\	は 9時[くじ]からです。			
\\	近付く	近付[ちかづ]く	ちかづく	
\\	女性が近付いてきました。	女性[じょせい]が 近付[ちかづ]いてきました。	じょせい が ちかづいて きました	
\\	女性[じょせい]が
\\	きました。			
\\	片付く	片付[かたづ]く	かたづく	
\\	仕事が大分片付いた。	仕事[しごと]が 大分[だいぶ] 片付[かたづ]いた。	しごと が だいぶ かたづいた	
\\	仕事[しごと]が 大分[だいぶ]
\\	気を付ける	気[き]を 付[つ]ける	きをつける	
\\	体に気を付けてください。	体[からだ]に 気[き]を 付[つ]けてください。	からだ に き を つけて ください	
\\	体[からだ]に
\\	ください。			
\\	ハイキング	ハイキング	ハイキング	
\\	昨日は友人とハイキングに行きました。	昨日[きのう]は 友人[ゆうじん]とハイキングに 行[い]きました。	きのう は ゆうじん と はいきんぐ に いきました	
\\	昨日[きのう]は 友人[ゆうじん]と
\\	に 行[い]きました。			
\\	貼る	貼[は]る	はる	
\\	机にシールを貼った。	机[つくえ]にシールを 貼[は]った。	つくえ に しーる を はった	
\\	机[つくえ]にシールを
\\	記念日	記念日[きねんび]	きねんび	
\\	今日は両親の結婚記念日です。	今日[きょう]は 両親[りょうしん]の 結婚[けっこん] 記念日[きねんび]です。	きょう は りょうしん の けっこん きねんび です	
\\	今日[きょう]は 両親[りょうしん]の 結婚[けっこん]
\\	です。			
\\	残念	残念[ざんねん]	ざんねん	
\\	その試合は残念な結果になった。	その 試合[しあい]は 残念[ざんねん]な 結果[けっか]になった。	その しあい は ざんねん な けっか に なった	
\\	その 試合[しあい]は
\\	な 結果[けっか]になった。			
\\	例えば	例[たと]えば	たとえば	
\\	例えば、このソフトで日本語を勉強することができます。	例[たと]えば、このソフトで 日本語[にほんご]を 勉強[べんきょう]することができます。	たとえば この そふと で にほんご を べんきょう する こと が できます	
\\	、このソフトで 日本語[にほんご]を 勉強[べんきょう]することができます。			
\\	ハム	ハム	ハム	
\\	ハムサンドをください。	ハムサンドをください。	はむ さんど を ください	
\\	サンドをください。			
\\	例文	例文[れいぶん]	れいぶん	
\\	例文を3つ作ってください。	例文[れいぶん]を 3[みっ]つ 作[つく]ってください。	れいぶん を みっつ つくって ください	
\\	を 3[みっ]つ 作[つく]ってください。			
\\	例	例[れい]	れい	
\\	一つ例をあげてください。	一[ひと]つ 例[れい]をあげてください。	ひとつ れい を あげて ください	
\\	一[ひと]つ
\\	をあげてください。			
\\	余る	余[あま]る	あまる	
\\	夕食の料理がたくさん余った。	夕食[ゆうしょく]の 料理[りょうり]がたくさん 余[あま]った。	ゆうしょく の りょうり が たくさん あまった	
\\	夕食[ゆうしょく]の 料理[りょうり]がたくさん
\\	除く	除[のぞ]く	のぞく	
\\	部長を除く全員が土曜日も働いた。	部長[ぶちょう]を 除[のぞ]く 全員[ぜんいん]が 土曜日[どようび]も 働[はたら]いた。	ぶちょう を のぞく ぜんいん が どようび も はたらいた	
\\	部長[ぶちょう]を
\\	全員[ぜんいん]が 土曜日[どようび]も 働[はたら]いた。			
\\	ボーイフレンド	ボーイフレンド	ボーイフレンド	
\\	私のボーイフレンドはハンサムな方でした。	私[わたし]のボーイフレンドはハンサムな 方[かた]でした。	わたし の ぼーいふれんど は はんさむな かた でした	
\\	私[わたし]の
\\	はハンサムな 方[かた]でした。			
\\	削る	削[けず]る	けずる	
\\	彼は家族のために仕事の時間を削った。	彼[かれ]は 家族[かぞく]のために 仕事[しごと]の 時間[じかん]を 削[けず]った。	かれ は かぞく の ため に しごと の じかん を けずった	
\\	彼[かれ]は 家族[かぞく]のために 仕事[しごと]の 時間[じかん]を
\\	遅刻	遅刻[ちこく]	ちこく	
\\	遅刻しないでください。	遅刻[ちこく]しないでください。	ちこく しない で ください	
\\	しないでください。			
\\	緩い	緩[ゆる]い	ゆるい	
\\	このズボンは緩いです。	このズボンは 緩[ゆる]いです。	この ずぼん は ゆるい です	
\\	このズボンは
\\	です。			
\\	苦しい	苦[くる]しい	くるしい	
\\	食べ過ぎておなかが苦しい。	食[た]べ 過[す]ぎておなかが 苦[くる]しい。	たべすぎて おなか が くるしい	
\\	食[た]べ 過[す]ぎておなかが
\\	アイロン	アイロン	アイロン	
\\	彼女はシャツにアイロンをかけた。	彼女[かのじょ]はシャツにアイロンをかけた。	かのじょ は しゃつ に あいろん を かけた	
\\	彼女[かのじょ]はシャツに
\\	をかけた。			
\\	苦い	苦[にが]い	にがい	
\\	私は苦いコーヒーが好きです。	私[わたし]は 苦[にが]いコーヒーが 好[す]きです。	わたし は にがい こーひー が すき です	
\\	私[わたし]は
\\	コーヒーが 好[す]きです。			
\\	苦手	苦手[にがて]	にがて	
\\	私は料理が苦手です。	私[わたし]は 料理[りょうり]が 苦手[にがて]です。	わたし は りょうり が にがて です	
\\	私[わたし]は 料理[りょうり]が
\\	です。			
\\	困る	困[こま]る	こまる	
\\	ケータイをなくして困っています。	ケータイをなくして 困[こま]っています。	けーたい を なくして こまって います	
\\	ケータイをなくして
\\	貧乏	貧乏[びんぼう]	びんぼう	
\\	彼は昔は貧乏だった。	彼[かれ]は 昔[むかし]は 貧乏[びんぼう]だった。	かれ は むかし は びんぼう だった	
\\	彼[かれ]は 昔[むかし]は
\\	だった。			
\\	カレーライス	カレーライス	カレーライス	
\\	子供はカレーライスが好きです。	子供[こども]はカレーライスが 好[す]きです。	こども は かれーらいす が すき です	
\\	子供[こども]は
\\	が 好[す]きです。			
\\	不幸	不幸[ふこう]	ふこう	
\\	彼女の家族に不幸があった。	彼女[かのじょ]の 家族[かぞく]に 不幸[ふこう]があった。	かのじょ の かぞく に ふこう が あった	
\\	彼女[かのじょ]の 家族[かぞく]に
\\	があった。			
\\	幸せ	幸[しあわ]せ	しあわせ	
\\	良い友達がいて私は幸せだ。	良[い]い 友達[ともだち]がいて 私[わたし]は 幸[しあわ]せだ。	いい ともだち が いて わたし は しあわせ だ	
\\	良[い]い 友達[ともだち]がいて 私[わたし]は
\\	だ。			
\\	塩	塩[しお]	しお	
\\	もうちょっと塩を入れて。	もうちょっと 塩[しお]を 入[い]れて。	もう ちょっと しお を いれて	
\\	もうちょっと
\\	を 入[い]れて。			
\\	塩辛い	塩辛[しおから]い	しおからい	
\\	海の水は塩辛い。	海[うみ]の 水[みず]は 塩辛[しおから]い。	うみ の みず は しおからい	
\\	海[うみ]の 水[みず]は
\\	キャッシュカード	キャッシュカード	キャッシュカード	
\\	キャッシュカードでお金を下ろした。	キャッシュカードでお 金[かね]を 下[お]ろした。	きゃっしゅかーど で おかね を おろした	
\\	でお 金[かね]を 下[お]ろした。			
\\	砂糖	砂糖[さとう]	さとう	
\\	コーヒーに砂糖は入れますか。	コーヒーに 砂糖[さとう]は 入[い]れますか。	こーひー に さとう は いれます か	
\\	コーヒーに
\\	は 入[い]れますか。			
\\	規模	規模[きぼ]	きぼ	
\\	この動物園は日本一の規模です。	この 動物園[どうぶつえん]は 日本一[にっぽんいち]の 規模[きぼ]です。	この どうぶつえん は にっぽんいち の きぼ です	
\\	この 動物園[どうぶつえん]は 日本一[にっぽんいち]の
\\	です。			
\\	農業	農業[のうぎょう]	のうぎょう	
\\	私は農業を勉強しています。	私[わたし]は 農業[のうぎょう]を 勉強[べんきょう]しています。	わたし は のうぎょう を べんきょう して います	
\\	私[わたし]は
\\	を 勉強[べんきょう]しています。			
\\	濃い	濃[こ]い	こい	
\\	私は濃い味が好きだ。	私[わたし]は 濃[こ]い 味[あじ]が 好[す]きだ。	わたし は こい あじ が すき だ	
\\	私[わたし]は
\\	味[あじ]が 好[す]きだ。			
\\	イヤリング	イヤリング	イヤリング	
\\	彼女はすてきなイヤリングをしている。	彼女[かのじょ]はすてきなイヤリングをしている。	かのじょ は すてき な いやりんぐ を して いる	
\\	彼女[かのじょ]はすてきな
\\	をしている。			
\\	薄い	薄[うす]い	うすい	
\\	この電子辞書はとても薄い。	この 電子辞書[でんし じしょ]はとても 薄[うす]い。	この でんし じしょ は とても うすい	
\\	この 電子辞書[でんし じしょ]はとても
\\	厚い	厚[あつ]い	あつい	
\\	その辞書はとても厚い。	その 辞書[じしょ]はとても 厚[あつ]い。	その じしょ は とても あつい	
\\	その 辞書[じしょ]はとても
\\	厚さ	厚[あつ]さ	あつさ	
\\	私は板の厚さを測った。	私[わたし]は 板[いた]の 厚[あつ]さを 測[はか]った。	わたし は いた の あつさ を はかった	
\\	私[わたし]は 板[いた]の
\\	を 測[はか]った。			
\\	迫る	迫[せま]る	せまる	
\\	締め切りが迫っています。	締[し]め 切[き]りが 迫[せま]っています。	しめきり が せまって います	
\\	締[し]め 切[き]りが
\\	ガールフレンド	ガールフレンド	ガールフレンド	
\\	昔は彼女のことをガールフレンドと言っていました。	昔[むかし]は 彼女[かのじょ]のことをガールフレンドと 言[い]っていました。	むかし は かのじょ の こと を がーるふれんど と いって いました	
\\	昔[むかし]は 彼女[かのじょ]のことを
\\	と 言[い]っていました。			
\\	伸びる	伸[の]びる	のびる	
\\	髪がだいぶ伸びたね。	髪[かみ]がだいぶ 伸[の]びたね。	かみ が だいぶ のびた ね	
\\	髪[かみ]がだいぶ
\\	ね。			
\\	引っ越す	引[ひ]っ 越[こ]す	ひっこす	
\\	来月、大阪に引っ越します。	来月[らいげつ]、 大阪[おおさか]に 引[ひ]っ 越[こ]します。	らいげつ おおさか に ひっこします	
\\	来月[らいげつ]、 大阪[おおさか]に
\\	越える	越[こ]える	こえる	
\\	私たちは高い山を越えました。	私[わたし]たちは 高[たか]い 山[やま]を 越[こ]えました。	わたしたち は たかい やま を こえました	
\\	私[わたし]たちは 高[たか]い 山[やま]を
\\	引っ越し	引[ひ]っ 越[こ]し	ひっこし	
\\	去年、引っ越ししました。	去年[きょねん]、 引[ひ]っ 越[こ]ししました。	きょねん ひっこし しました	
\\	去年[きょねん]、
\\	しました。			
\\	カセットテープ	カセットテープ	カセットテープ	
\\	昔はカセットテープを使っていました。	昔[むかし]はカセットテープを 使[つか]っていました。	むかし は かせっとてーぷ を つかっていました 。	
\\	昔[むかし]は
\\	を 使[つか]っていました。			
\\	追い越す	追[お]い 越[こ]す	おいこす	
\\	大きなトラックが私たちを追い越した。	大[おお]きなトラックが 私[わたし]たちを 追[お]い 越[こ]した。	おおき な とらっく が わたしたち を おいこした	
\\	大[おお]きなトラックが 私[わたし]たちを
\\	上昇	上昇[じょうしょう]	じょうしょう	
\\	地球の気温は上昇している。	地球[ちきゅう]の 気温[きおん]は 上昇[じょうしょう]している。	ちきゅう の きおん は じょうしょう して いる	
\\	地球[ちきゅう]の 気温[きおん]は
\\	している。			
\\	改札口	改札口[かいさつぐち]	かいさつぐち	
\\	改札口で会いましょう。	改札口[かいさつぐち]で 会[あ]いましょう。	かいさつぐち で あいましょう	
\\	で 会[あ]いましょう。			
\\	失礼	失礼[しつれい]	しつれい	
\\	ではそろそろ失礼します。	ではそろそろ 失礼[しつれい]します。	では そろそろ しつれい します	
\\	ではそろそろ
\\	します。			
\\	かゆい	かゆい	かゆい	
\\	背中がかゆいです。	背中[せなか]がかゆいです。	せなか が かゆい です	
\\	背中[せなか]が
\\	です。			
\\	お礼	お 礼[れい]	おれい	
\\	彼女にお礼の手紙を書きました。	彼女[かのじょ]にお 礼[れい]の 手紙[てがみ]を 書[か]きました。	かのじょ に おれい の てがみ を かきました	
\\	彼女[かのじょ]に
\\	の 手紙[てがみ]を 書[か]きました。			
\\	謝る	謝[あやま]る	あやまる	
\\	彼は直ぐに謝りました。	彼[かれ]は 直[す]ぐに 謝[あやま]りました。	かれ は すぐ に あやまりました	
\\	彼[かれ]は 直[す]ぐに
\\	注射	注射[ちゅうしゃ]	ちゅうしゃ	
\\	彼は注射があまり好きではありません。	彼[かれ]は 注射[ちゅうしゃ]があまり 好[す]きではありません。	かれ は ちゅうしゃ が あまり すき で は ありません	
\\	彼[かれ]は
\\	があまり 好[す]きではありません。			
\\	程度	程度[ていど]	ていど	
\\	この程度の怪我なら大丈夫です。	この 程度[ていど]の 怪我[けが]なら 大丈夫[だいじょうぶ]です。	この ていど の けが なら だいじょうぶ です	
\\	この
\\	の 怪我[けが]なら 大丈夫[だいじょうぶ]です。			
\\	ぐらぐら	ぐらぐら	ぐらぐら	
\\	地震で家がぐらぐらと揺れた。	地震[じしん]で 家[いえ]がぐらぐらと 揺[ゆ]れた。	じしん で いえ が ぐらぐら と ゆれた	
\\	地震[じしん]で 家[いえ]が
\\	と 揺[ゆ]れた。			
\\	誘う	誘[さそ]う	さそう	
\\	彼女をデートに誘った。	彼女[かのじょ]をデートに 誘[さそ]った。	かのじょ を でーと に さそった	
\\	彼女[かのじょ]をデートに
\\	導入	導入[どうにゅう]	どうにゅう	
\\	会社で新しいシステムを導入した。	会社[かいしゃ]で 新[あたら]しいシステムを 導入[どうにゅう]した。	かいしゃ で あたらしい しすてむ を どうにゅう した	
\\	会社[かいしゃ]で 新[あたら]しいシステムを
\\	した。			
\\	努力	努力[どりょく]	どりょく	
\\	もっと努力しよう。	もっと 努力[どりょく]しよう。	もっと どりょく しよう	
\\	もっと
\\	しよう。			
\\	怒る	怒[おこ]る	おこる	
\\	彼女が嘘をついたので、彼は怒った。	彼女[かのじょ]が 嘘[うそ]をついたので、 彼[かれ]は 怒[おこ]った。	かのじょ が うそ を ついた の で かれ は おこった	
\\	彼女[かのじょ]が 嘘[うそ]をついたので、 彼[かれ]は
\\	ジャム	ジャム	ジャム	
\\	このイチゴでジャムを作りましょう。	このイチゴでジャムを 作[つく]りましょう。	この いちご で じゃむ を つくりましょう	
\\	このイチゴで
\\	を 作[つく]りましょう。			
\\	独身	独身[どくしん]	どくしん	
\\	私の兄はまだ独身です。	私[わたし]の 兄[あに]はまだ 独身[どくしん]です。	わたし の あに は まだ どくしん です	
\\	私[わたし]の 兄[あに]はまだ
\\	です。			
\\	占める	占[し]める	しめる	
\\	私に届くメールのうち、迷惑メールが7割を占めている。	私[わたし]に 届[とど]くメールのうち、 迷惑[めいわく]メールが 7割[ななわり]を 占[し]めている。	わたし に とどく めーる の うち めいわく めーる が ななわり を しめて いる	
\\	私[わたし]に 届[とど]くメールのうち、 迷惑[めいわく]メールが 7割[ななわり]を
\\	処理	処理[しょり]	しょり	
\\	事務的な処理に1週間かかります。	事務的[じむてき]な 処理[しょり]に 1週間[いっしゅうかん]かかります。	じむてき な しょり に いっしゅうかん かかります	
\\	事務的[じむてき]な
\\	に 1週間[いっしゅうかん]かかります。			
\\	紹介	紹介[しょうかい]	しょうかい	
\\	両親に彼女を紹介した。	両親[りょうしん]に 彼女[かのじょ]を 紹介[しょうかい]した。	りょうしん に かのじょ を しょうかい した	
\\	両親[りょうしん]に 彼女[かのじょ]を
\\	した。			
\\	スリッパ	スリッパ	スリッパ	
\\	スリッパをはいてください。	スリッパをはいてください。	すりっぱ を はいて ください	
\\	をはいてください。			
\\	招く	招[まね]く	まねく	
\\	両親を食事に招いた。	両親[りょうしん]を 食事[しょくじ]に 招[まね]いた。	りょうしん を しょくじ に まねいた	
\\	両親[りょうしん]を 食事[しょくじ]に
\\	招待	招待[しょうたい]	しょうたい	
\\	高校の時の先生を結婚式に招待した。	高校[こうこう]の 時[とき]の 先生[せんせい]を 結婚式[けっこんしき]に 招待[しょうたい]した。	こうこう の とき の せんせい を けっこんしき に しょうたい した	
\\	高校[こうこう]の 時[とき]の 先生[せんせい]を 結婚式[けっこんしき]に
\\	した。			
\\	夫婦	夫婦[ふうふ]	ふうふ	
\\	その夫婦はとても仲がいい。	その 夫婦[ふうふ]はとても 仲[なか]がいい。	その ふうふ は とても なか が いい	
\\	その
\\	はとても 仲[なか]がいい。			
\\	奥	奥[おく]	おく	
\\	はさみは机の奥にあった。	はさみは 机[つくえ]の 奥[おく]にあった。	はさみ は つくえ の おく に あった	
\\	はさみは 机[つくえ]の
\\	にあった。			
\\	トランプ	トランプ	トランプ	
\\	友達とトランプをして遊びました。	友達[ともだち]とトランプをして 遊[あそ]びました。	ともだち と とらんぷ を して あそびました	
\\	友達[ともだち]と
\\	をして 遊[あそ]びました。			
\\	奥さん	奥[おく]さん	おくさん	
\\	彼の奥さんはきれいな方です。	彼[かれ]の 奥[おく]さんはきれいな 方[かた]です。	かれ の おくさん は きれい な かた です	
\\	彼[かれ]の
\\	はきれいな 方[かた]です。			
\\	皆さん	皆[みな]さん	みなさん	
\\	皆さんにお話があります。	皆[みな]さんにお 話[はなし]があります。	みなさん に おはなし が あります	
\\	にお 話[はなし]があります。			
\\	皆様	皆様[みなさま]	みなさま	
\\	皆様、こんにちは。	皆様[みなさま]、こんにちは。	みなさま こんにちは	
\\	、こんにちは。			
\\	誰か	誰[だれ]か	だれか	
\\	誰かに聞いてみてください。	誰[だれ]かに 聞[き]いてみてください。	だれか に きいて みて ください	
\\	に 聞[き]いてみてください。			
\\	よろしい	よろしい	よろしい	
\\	レポートはこれでよろしいですか。	レポートはこれでよろしいですか。	れぽーと は これ で よろしい です か	
\\	レポートはこれで
\\	ですか。			
\\	国籍	国籍[こくせき]	こくせき	
\\	私は日本国籍です。	私[わたし]は 日本[にほん] 国籍[こくせき]です。	わたし は にほん こくせき です	
\\	私[わたし]は 日本[にほん]
\\	です。			
\\	愛	愛[あい]	あい	
\\	彼女は愛をこめて手紙を書いた。	彼女[かのじょ]は 愛[あい]をこめて 手紙[てがみ]を 書[か]いた。	かのじょ は あい を こめて てがみ を かいた	
\\	彼女[かのじょ]は
\\	をこめて 手紙[てがみ]を 書[か]いた。			
\\	可愛い	可愛[かわい]い	かわいい	
\\	彼女の赤ちゃんは可愛いです。	彼女[かのじょ]の 赤[あか]ちゃんは 可愛[かわい]いです。	かのじょ の あかちゃん は かわいい です	
\\	彼女[かのじょ]の 赤[あか]ちゃんは
\\	です。			
\\	恋人	恋人[こいびと]	こいびと	
\\	彼は恋人を失った。	彼[かれ]は 恋人[こいびと]を 失[うしな]った。	かれ は こいびと を うしなった	
\\	彼[かれ]は
\\	を 失[うしな]った。			
\\	ラッシュアワー	ラッシュアワー	ラッシュアワー	
\\	私はラッシュアワーの電車が嫌いです。	私[わたし]はラッシュアワーの 電車[でんしゃ]が 嫌[きら]いです。	わたし は らっしゅあわー の でんしゃ が きらい です	
\\	私[わたし]は
\\	の 電車[でんしゃ]が 嫌[きら]いです。			
\\	誕生日	誕生日[たんじょうび]	たんじょうび	
\\	彼女の誕生日は7月16日です。	彼女[かのじょ]の 誕生日[たんじょうび]は 7月16日[しちがつ じゅうろくにち]です。	かのじょ の たんじょうび は しちがつ じゅうろくにち です	
\\	彼女[かのじょ]の
\\	は 7月16日[しちがつ じゅうろくにち]です。			
\\	祝日	祝日[しゅくじつ]	しゅくじつ	
\\	5月3日は祝日だ。	5月3日[ごがつ みっか]は 祝日[しゅくじつ]だ。	ごがつ みっか は しゅくじつ だ	
\\	5月3日[ごがつ みっか]は
\\	だ。			
\\	お祝い	お 祝[いわ]い	おいわい	
\\	彼の就職のお祝いをしよう。	彼[かれ]の 就職[しゅうしょく]のお 祝[いわ]いをしよう。	かれ の しゅうしょく の おいわい を しよう	
\\	彼[かれ]の 就職[しゅうしょく]の
\\	をしよう。			
\\	夢	夢[ゆめ]	ゆめ	
\\	昨夜恐ろしい夢を見た。	昨夜[ゆうべ] 恐[おそ]ろしい 夢[ゆめ]を 見[み]た。	ゆうべ おそろしい ゆめ を みた	
\\	昨夜[ゆうべ] 恐[おそ]ろしい
\\	を 見[み]た。			
\\	カップ	カップ	カップ	
\\	水を1カップ入れてください。	水[みず]を 1[いち]カップ 入[い]れてください。	みず を いちかっぷ いれて ください	
\\	水[みず]を 1[いち]
\\	入[い]れてください。			
\\	泣く	泣[な]く	なく	
\\	妹はすぐに泣く。	妹[いもうと]はすぐに 泣[な]く。	いもうと は すぐ に なく	
\\	妹[いもうと]はすぐに
\\	涙	涙[なみだ]	なみだ	
\\	彼女の目から涙がこぼれた。	彼女[かのじょ]の 目[め]から 涙[なみだ]がこぼれた。	かのじょ の め から なみだ が こぼれた	
\\	彼女[かのじょ]の 目[め]から
\\	がこぼれた。			
\\	喜ぶ	喜[よろこ]ぶ	よろこぶ	
\\	彼女はとても喜びました。	彼女[かのじょ]はとても 喜[よろこ]びました。	かのじょ は とても よろこびました	
\\	彼女[かのじょ]はとても
\\	恥ずかしい	恥[は]ずかしい	はずかしい	
\\	とても恥ずかしかった。	とても 恥[は]ずかしかった。	とても はずかしかった	
\\	とても
\\	スケート	スケート	スケート	
\\	湖でスケートをした。	湖[みずうみ]でスケートをした。	みずうみ で すけーと を した	
\\	湖[みずうみ]で
\\	をした。			
\\	弁当	弁当[べんとう]	べんとう	
\\	今日は弁当を持ってきました。	今日[きょう]は 弁当[べんとう]を 持[も]ってきました。	きょう は べんとう を もって きました	
\\	今日[きょう]は
\\	を 持[も]ってきました。			
\\	看護婦	看護婦[かんごふ]	かんごふ	
\\	母は看護婦です。	母[はは]は 看護婦[かんごふ]です。	はは は かんごふ です	
\\	母[はは]は
\\	です。			
\\	患者	患者[かんじゃ]	かんじゃ	
\\	患者は眠っています。	患者[かんじゃ]は 眠[ねむ]っています。	かんじゃ は ねむって います	
\\	は 眠[ねむ]っています。			
\\	述べる	述[の]べる	のべる	
\\	上司が意見を述べた。	上司[じょうし]が 意見[いけん]を 述[の]べた。	じょうし が いけん を のべた	
\\	上司[じょうし]が 意見[いけん]を
\\	たくさん	たくさん	たくさん	
\\	昨日はたくさん泳ぎました。	昨日[きのう]はたくさん 泳[およ]ぎました。	きのう は たくさん およぎました	
\\	昨日[きのう]は
\\	泳[およ]ぎました。			
\\	訴える	訴[うった]える	うったえる	
\\	彼女は会社を訴えた。	彼女[かのじょ]は 会社[かいしゃ]を 訴[うった]えた。	かのじょ は かいしゃ を うったえた	
\\	彼女[かのじょ]は 会社[かいしゃ]を
\\	迷う	迷[まよ]う	まよう	
\\	道に迷いました。	道[みち]に 迷[まよ]いました。	みち に まよいました	
\\	道[みち]に
\\	迷惑	迷惑[めいわく]	めいわく	
\\	人に迷惑をかけてはいけません。	人[ひと]に 迷惑[めいわく]をかけてはいけません。	ひと に めいわく を かけて は いけません	
\\	人[ひと]に
\\	をかけてはいけません。			
\\	地域	地域[ちいき]	ちいき	
\\	この地域は雨が多い。	この 地域[ちいき]は 雨[あめ]が 多[おお]い。	この ちいき は あめ が おおい	
\\	この
\\	は 雨[あめ]が 多[おお]い。			
\\	ピクニック	ピクニック	ピクニック	
\\	今日はピクニックに行きましょう。	今日[きょう]はピクニックに 行[い]きましょう。	きょう は ぴくにっく に いきましょう	
\\	今日[きょう]は
\\	に 行[い]きましょう。			
\\	政権	政権[せいけん]	せいけん	
\\	政権が交代した。	政権[せいけん]が 交代[こうたい]した。	せいけん が こうたい した	
\\	が 交代[こうたい]した。			
\\	贈る	贈[おく]る	おくる	
\\	母に花を贈った。	母[はは]に 花[はな]を 贈[おく]った。	はは に はな を おくった	
\\	母[はは]に 花[はな]を
\\	贈り物	贈[おく]り 物[もの]	おくりもの	
\\	すてきな贈り物をどうもありがとう。	すてきな 贈[おく]り 物[もの]をどうもありがとう。	すてき な おくりもの を どうも ありがとう	
\\	すてきな
\\	をどうもありがとう。			
\\	与える	与[あた]える	あたえる	
\\	そのニュースは彼に大きなショックを与えた。	そのニュースは 彼[かれ]に 大[おお]きなショックを 与[あた]えた。	その にゅーす は かれ に おおき な しょっく を あたえた	
\\	そのニュースは 彼[かれ]に 大[おお]きなショックを
\\	あちら	あちら	あちら	
\\	あちらに行ってみよう。	あちらに 行[い]ってみよう。	あちら に いって みよう	
\\	に 行[い]ってみよう。			
\\	貯金	貯金[ちょきん]	ちょきん	
\\	貯金は十分にあります。	貯金[ちょきん]は 十分[じゅうぶん]にあります。	ちょきん は じゅうぶん に あります	
\\	は 十分[じゅうぶん]にあります。			
\\	預ける	預[あず]ける	あずける	
\\	鍵を彼に預けた。	鍵[かぎ]を 彼[かれ]に 預[あず]けた。	かぎ を かれ に あずけた	
\\	鍵[かぎ]を 彼[かれ]に
\\	預かる	預[あず]かる	あずかる	
\\	荷物を預かってください。	荷物[にもつ]を 預[あず]かってください。	にもつ を あずかって ください	
\\	荷物[にもつ]を
\\	ください。			
\\	得意	得意[とくい]	とくい	
\\	彼は歌が得意です。	彼[かれ]は 歌[うた]が 得意[とくい]です。	かれ は うた が とくい です	
\\	彼[かれ]は 歌[うた]が
\\	です。			
\\	どきどき	どきどき	どきどき	
\\	彼女に会うとどきどきします。	彼女[かのじょ]に 会[あ]うとどきどきします。	かのじょ に あう と どきどき します	
\\	彼女[かのじょ]に 会[あ]うと
\\	します。			
\\	燃える	燃[も]える	もえる	
\\	山が燃えています。	山[やま]が 燃[も]えています。	やま が もえて います	
\\	山[やま]が
\\	焼ける	焼[や]ける	やける	
\\	肉が焼けました。	肉[にく]が 焼[や]けました。	にく が やけました	
\\	肉[にく]が
\\	すき焼き	すき 焼[や]き	すきやき	
\\	昨日の夜はすき焼きを食べた。	昨日[きのう]の 夜[よる]はすき 焼[や]きを 食[た]べた。	きのう の よる は すきやき を たべた	
\\	昨日[きのう]の 夜[よる]は
\\	を 食[た]べた。			
\\	焼く	焼[や]く	やく	
\\	今、魚を焼いています。	今[いま]、 魚[さかな]を 焼[や]いています。	いま さかな を やいて います	
\\	今[いま]、 魚[さかな]を
\\	まあまあ	まあまあ	まあまあ	
\\	彼の成績はまあまあです。	彼[かれ]の 成績[せいせき]はまあまあです。	かれ の せいせき は まあまあ です	
\\	彼[かれ]の 成績[せいせき]は
\\	です。			
\\	乾く	乾[かわ]く	かわく	
\\	夏は洗濯物がすぐ乾く。	夏[なつ]は 洗濯物[せんたくもの]がすぐ 乾[かわ]く。	なつ は せんたくもの が すぐ かわく	
\\	夏[なつ]は 洗濯物[せんたくもの]がすぐ
\\	乾杯	乾杯[かんぱい]	かんぱい	
\\	お二人の未来に乾杯しましょう。	お 二人[ふたり]の 未来[みらい]に 乾杯[かんぱい]しましょう。	おふたり の みらい に かんぱい しましょう	
\\	お 二人[ふたり]の 未来[みらい]に
\\	しましょう。			
\\	乾かす	乾[かわ]かす	かわかす	
\\	ぬれた服を乾かした。	ぬれた 服[ふく]を 乾[かわ]かした。	ぬれた ふく を かわかした	
\\	ぬれた 服[ふく]を
\\	新幹線	新幹線[しんかんせん]	しんかんせん	
\\	新幹線で京都に行きました。	新幹線[しんかんせん]で 京都[きょうと]に 行[い]きました。	しんかんせん で きょうと に いきました	
\\	で 京都[きょうと]に 行[い]きました。			
\\	マフラー	マフラー	マフラー	
\\	彼女は首にマフラーをまいていた。	彼女[かのじょ]は 首[くび]にマフラーをまいていた。	かのじょ は くび に まふらー を まいて いた	
\\	彼女[かのじょ]は 首[くび]に
\\	をまいていた。			
\\	素晴らしい	素晴[すば]らしい	すばらしい	
\\	素晴らしい景色ですね。	素晴[すば]らしい 景色[けしき]ですね。	すばらしい けしき です ね	
\\	景色[けしき]ですね。			
\\	海岸	海岸[かいがん]	かいがん	
\\	海岸を散歩しましょう。	海岸[かいがん]を 散歩[さんぽ]しましょう。	かいがん を さんぽ しましょう	
\\	を 散歩[さんぽ]しましょう。			
\\	家庭	家庭[かてい]	かてい	
\\	彼は家庭を大切にしている。	彼[かれ]は 家庭[かてい]を 大切[たいせつ]にしている。	かれ は かてい を たいせつ に して いる	
\\	彼[かれ]は
\\	を 大切[たいせつ]にしている。			
\\	庭	庭[にわ]	にわ	
\\	庭にバラを植えました。	庭[にわ]にバラを 植[う]えました。	にわ に ばら を うえました	
\\	にバラを 植[う]えました。			
\\	アクセサリー	アクセサリー	アクセサリー	
\\	このアクセサリーは素敵ね。	このアクセサリーは 素敵[すてき]ね。	この あくせさりー は すてき ね	
\\	この
\\	は 素敵[すてき]ね。			
\\	桜	桜[さくら]	さくら	
\\	桜は三月か四月に咲きます。	桜[さくら]は 三月[さんがつ]か 四月[しがつ]に 咲[さ]きます。	さくら は さんがつ か しがつ に さきます	
\\	は 三月[さんがつ]か 四月[しがつ]に 咲[さ]きます。			
\\	咲く	咲[さ]く	さく	
\\	桜の花が咲きました。	桜[さくら]の 花[はな]が 咲[さ]きました。	さくら の はな が さきました	
\\	桜[さくら]の 花[はな]が
\\	吹く	吹[ふ]く	ふく	
\\	今日は北風が吹いている。	今日[きょう]は 北風[きたかぜ]が 吹[ふ]いている。	きょう は きたかぜ が ふいて いる	
\\	今日[きょう]は 北風[きたかぜ]が
\\	散歩	散歩[さんぽ]	さんぽ	
\\	私のお祖父さんは毎日散歩します。	私[わたし]のお 祖父[じい]さんは 毎日[まいにち] 散歩[さんぽ]します。	わたし の おじいさん は まいにち さんぽ します 。	
\\	私[わたし]のお 祖父[じい]さんは
\\	します。			
\\	あくび	あくび	あくび	
\\	父があくびをした。	父[ちち]があくびをした。	ちち が あくび を した	
\\	父[ちち]が
\\	をした。			
\\	植える	植[う]える	うえる	
\\	庭にバラを植えました。	庭[にわ]にバラを 植[う]えました。	にわ に ばら を うえました	
\\	庭[にわ]にバラを
\\	屋根	屋根[やね]	やね	
\\	屋根にカラスが止まっています。	屋根[やね]にカラスが 止[と]まっています。	やね に からす が とまって います	
\\	にカラスが 止[と]まっています。			
\\	黒板	黒板[こくばん]	こくばん	
\\	答えを黒板に書いてください。	答[こた]えを 黒板[こくばん]に 書[か]いてください。	こたえ を こくばん に かいて ください	
\\	答[こた]えを
\\	に 書[か]いてください。			
\\	掲示板	掲示板[けいじばん]	けいじばん	
\\	掲示板のお知らせを見ましたか。	掲示板[けいじばん]のお 知[し]らせを 見[み]ましたか。	けいじばん の おしらせ を みました か	
\\	のお 知[し]らせを 見[み]ましたか。			
\\	からから	からから	からから	
\\	おしゃべりして喉がからからになりました。	おしゃべりして 喉[のど]がからからになりました。	おしゃべり して のど が からから に なりました	
\\	おしゃべりして 喉[のど]が
\\	になりました。			
\\	草	草[くさ]	くさ	
\\	庭に草が生えている。	庭[にわ]に 草[くさ]が 生[は]えている。	にわ に くさ が はえて いる	
\\	庭[にわ]に
\\	が 生[は]えている。			
\\	葉	葉[は]	は	
\\	これは桜の葉です。	これは 桜[さくら]の 葉[は]です。	これ は さくら の は です	
\\	これは 桜[さくら]の
\\	です。			
\\	絵葉書	絵葉書[えはがき]	えはがき	
\\	友達から絵葉書が届きました。	友達[ともだち]から 絵葉書[えはがき]が 届[とど]きました。	ともだち から えはがき が とどきました	
\\	友達[ともだち]から
\\	が 届[とど]きました。			
\\	葉書	葉書[はがき]	はがき	
\\	家族に葉書を書いています。	家族[かぞく]に 葉書[はがき]を 書[か]いています。	かぞく に はがき を かいています 。	
\\	家族[かぞく]に
\\	を 書[か]いています。			
\\	ふらふら	ふらふら	ふらふら	
\\	私は疲れてふらふらです。	私[わたし]は 疲[つか]れてふらふらです。	わたし は つかれて ふらふら です	
\\	私[わたし]は 疲[つか]れて
\\	です。			
\\	吸う	吸[す]う	すう	
\\	彼は大きく息を吸った。	彼[かれ]は 大[おお]きく 息[いき]を 吸[す]った。	かれ は おおきく いき を すった	
\\	彼[かれ]は 大[おお]きく 息[いき]を
\\	普及	普及[ふきゅう]	ふきゅう	
\\	ゴミのリサイクルが普及している。	ゴミのリサイクルが 普及[ふきゅう]している。	ごみ の りさいくる が ふきゅう して いる	
\\	ゴミのリサイクルが
\\	している。			
\\	胃	胃[い]	い	
\\	昨日から胃が痛い。	昨日[きのう]から 胃[い]が 痛[いた]い。	きのう から い が いたい	
\\	昨日[きのう]から
\\	が 痛[いた]い。			
\\	皿	皿[さら]	さら	
\\	皿にケーキを載せました。	皿[さら]にケーキを 載[の]せました。	さら に けーき を のせました。	
\\	にケーキを 載[の]せました。			
\\	こぼす	こぼす	こぼす	
\\	ソファにワインをこぼした。	ソファにワインをこぼした。	そふぁ に わいん を こぼした	
\\	ソファにワインを
\\	血	血[ち]	ち	
\\	血が出ていますよ。	血[ち]が 出[で]ていますよ。	ち が でて います よ	
\\	が 出[で]ていますよ。			
\\	内容	内容[ないよう]	ないよう	
\\	この本の内容を説明してください。	この 本[ほん]の 内容[ないよう]を 説明[せつめい]してください。	この ほん の ないよう を せつめい して ください	
\\	この 本[ほん]の
\\	を 説明[せつめい]してください。			
\\	背景	背景[はいけい]	はいけい	
\\	事件の背景に何があったのだろう。	事件[じけん]の 背景[はいけい]に 何[なに]があったのだろう。	じけん の はいけい に なに が あった の だろう	
\\	事件[じけん]の
\\	に 何[なに]があったのだろう。			
\\	骨	骨[ほね]	ほね	
\\	彼は足の骨を折りました。	彼[かれ]は 足[あし]の 骨[ほね]を 折[お]りました。	かれ は あし の ほね を おりました	
\\	彼[かれ]は 足[あし]の
\\	を 折[お]りました。			
\\	ランチ	ランチ	ランチ	
\\	一緒にランチに行きませんか。	一緒[いっしょ]にランチに 行[い]きませんか。	いっしょ に らんち に いきません か	
\\	一緒[いっしょ]に
\\	に 行[い]きませんか。			
\\	滑る	滑[すべ]る	すべる	
\\	彼は雪の上を滑った。	彼[かれ]は 雪[ゆき]の 上[うえ]を 滑[すべ]った。	かれ は ゆき の うえ を すべった	
\\	彼[かれ]は 雪[ゆき]の 上[うえ]を
\\	折れる	折[お]れる	おれる	
\\	強風で木の枝が折れた。	強風[きょうふう]で 木[き]の 枝[えだ]が 折[お]れた。	きょうふう で き の えだ が おれた	
\\	強風[きょうふう]で 木[き]の 枝[えだ]が
\\	折る	折[お]る	おる	
\\	祖父は足の骨を折りました。	祖父[そふ]は 足[あし]の 骨[ほね]を 折[お]りました。	そふ は あし の ほね を おりました	
\\	祖父[そふ]は 足[あし]の 骨[ほね]を
\\	健康	健康[けんこう]	けんこう	
\\	私は健康な生活を送っています。	私[わたし]は 健康[けんこう]な 生活[せいかつ]を 送[おく]っています。	わたし は けんこう な せいかつ を おくって います	
\\	私[わたし]は
\\	な 生活[せいかつ]を 送[おく]っています。			
\\	レシート	レシート	レシート	
\\	レシートを財布に入れました。	レシートを 財布[さいふ]に 入[い]れました。	れしーと を さいふ に いれました	
\\	を 財布[さいふ]に 入[い]れました。			
\\	珍しい	珍[めずら]しい	めずらしい	
\\	昨日珍しい果物を食べました。	昨日[きのう] 珍[めずら]しい 果物[くだもの]を 食[た]べました。	きのう めずらしい くだもの を たべました	
\\	昨日[きのう]
\\	果物[くだもの]を 食[た]べました。			
\\	撮る	撮[と]る	とる	
\\	写真をたくさん撮りました。	写真[しゃしん]をたくさん 撮[と]りました。	しゃしん を たくさん とりました	
\\	写真[しゃしん]をたくさん
\\	再び	再[ふたた]び	ふたたび	
\\	彼は再びここに戻ってきた。	彼[かれ]は 再[ふたた]びここに 戻[もど]ってきた。	かれ は ふたたび ここ に もどって きた	
\\	彼[かれ]は
\\	ここに 戻[もど]ってきた。			
\\	再来年	再来年[さらいねん]	さらいねん	
\\	次のオリンピックは再来年です。	次[つぎ]のオリンピックは 再来年[さらいねん]です。	つぎ の おりんぴっく は さらいねん です	
\\	次[つぎ]のオリンピックは
\\	です。			
\\	チョーク	チョーク	チョーク	
\\	チョークで黒板に字を書きました。	チョークで 黒板[こくばん]に 字[じ]を 書[か]きました。	ちょーく で こくばん に じ を かきました	
\\	で 黒板[こくばん]に 字[じ]を 書[か]きました。			
\\	再来月	再来月[さらいげつ]	さらいげつ	
\\	再来月まで予約で一杯です。	再来月[さらいげつ]まで 予約[よやく]で 一杯[いっぱい]です。	さらいげつ まで よやく で いっぱい です	
\\	まで 予約[よやく]で 一杯[いっぱい]です。			
\\	再来週	再来週[さらいしゅう]	さらいしゅう	
\\	再来週は忙しいです。	再来週[さらいしゅう]は 忙[いそが]しいです。	さらいしゅう は いそがしい です	
\\	は 忙[いそが]しいです。			
\\	放送	放送[ほうそう]	ほうそう	
\\	その番組は来週放送されます。	その 番組[ばんぐみ]は 来週[らいしゅう] 放送[ほうそう]されます。	その ばんぐみ は らいしゅう ほうそう されます	
\\	その 番組[ばんぐみ]は 来週[らいしゅう]
\\	されます。			
\\	装置	装置[そうち]	そうち	
\\	実験にはこの装置を使います。	実験[じっけん]にはこの 装置[そうち]を 使[つか]います。	じっけん に は この そうち を つかいます	
\\	実験[じっけん]にはこの
\\	を 使[つか]います。			
\\	ティッシュ	ティッシュ	ティッシュ	
\\	ティッシュを取ってください。	ティッシュを 取[と]ってください。	てぃっしゅ を とって ください	
\\	を 取[と]ってください。			
\\	仮名	仮名[かな]	かな	
\\	お名前に仮名を振ってください。	お 名前[なまえ]に 仮名[かな]を 振[ふ]ってください。	おなまえ に かな を ふって ください	
\\	お 名前[なまえ]に
\\	を 振[ふ]ってください。			
\\	送り仮名	送[おく]り 仮名[がな]	おくりがな	
\\	その送り仮名は間違っています。	その 送[おく]り 仮名[がな]は 間違[まちが]っています。	その おくりがな は まちがって います	
\\	その
\\	は 間違[まちが]っています。			
\\	鏡	鏡[かがみ]	かがみ	
\\	彼女は鏡を見て髪を直した。	彼女[かのじょ]は 鏡[かがみ]を 見[み]て 髪[かみ]を 直[なお]した。	かのじょ は かがみ を みて かみ を なおした	
\\	彼女[かのじょ]は
\\	を 見[み]て 髪[かみ]を 直[なお]した。			
\\	悲しむ	悲[かな]しむ	かなしむ	
\\	父は友だちの死を悲しんでいます。	父[ちち]は 友[とも]だちの 死[し]を 悲[かな]しんでいます。	ちち は ともだち の し を かなしんで います	
\\	父[ちち]は 友[とも]だちの 死[し]を
\\	ノック	ノック	ノック	
\\	入る時はドアをノックしてください。	入[はい]る 時[とき]はドアをノックしてください。	はいる とき は どあ を のっく して ください	
\\	入[はい]る 時[とき]はドアを
\\	してください。			
\\	固い	固[かた]い	かたい	
\\	私の上司は頭が固い。	私[わたし]の 上司[じょうし]は 頭[あたま]が 固[かた]い。	わたし の じょうし は あたま が かたい	
\\	私[わたし]の 上司[じょうし]は 頭[あたま]が
\\	美術館	美術館[びじゅつかん]	びじゅつかん	
\\	昨日、美術館に行きました。	昨日[きのう]、 美術館[びじゅつかん]に 行[い]きました。	きのう びじゅつかん に いきました	
\\	昨日[きのう]、
\\	に 行[い]きました。			
\\	美人	美人[びじん]	びじん	
\\	彼のお母さんは美人です。	彼[かれ]のお 母[かあ]さんは 美人[びじん]です。	かれ の おかあさん は びじん です	
\\	彼[かれ]のお 母[かあ]さんは
\\	です。			
\\	美容院	美容院[びよういん]	びよういん	
\\	父は美容院で髪を切ります。	父[ちち]は 美容院[びよういん]で 髪[かみ]を 切[き]ります。	ちち は びよういん で かみ を きります	
\\	父[ちち]は
\\	で 髪[かみ]を 切[き]ります。			
\\	よろしく	よろしく	よろしく	
\\	よろしくお願いします。	よろしくお 願[ねが]いします。	よろしく おねがい します	
\\	お 願[ねが]いします。			
\\	博物館	博物館[はくぶつかん]	はくぶつかん	
\\	昨日、車の博物館に行った。	昨日[きのう]、 車[くるま]の 博物館[はくぶつかん]に 行[い]った。	きのう くるま の はくぶつかん に いった	
\\	昨日[きのう]、 車[くるま]の
\\	に 行[い]った。			
\\	開催	開催[かいさい]	かいさい	
\\	京都で国際会議が開催された。	京都[きょうと]で 国際会議[こくさいかいぎ]が 開催[かいさい]された。	きょうと で こくさいかいぎ が かいさい された	
\\	京都[きょうと]で 国際会議[こくさいかいぎ]が
\\	された。			
\\	特徴	特徴[とくちょう]	とくちょう	
\\	この曲線がこの車の特徴です。	この 曲線[きょくせん]がこの 車[くるま]の 特徴[とくちょう]です。	この きょくせん が この くるま の とくちょう です	
\\	この 曲線[きょくせん]がこの 車[くるま]の
\\	です。			
\\	許す	許[ゆる]す	ゆるす	
\\	私は彼を許しました。	私[わたし]は 彼[かれ]を 許[ゆる]しました。	わたし は かれ を ゆるしました	
\\	私[わたし]は 彼[かれ]を
\\	がらがら	がらがら	がらがら	
\\	映画館はがらがらでした。	映画館[えいがかん]はがらがらでした。	えいがかん は がらがら でした	
\\	映画館[えいがかん]は
\\	でした。			
\\	免許証	免許証[めんきょしょう]	めんきょしょう	
\\	免許証を見せてください。	免許証[めんきょしょう]を 見[み]せてください。	めんきょしょう を みせて ください	
\\	を 見[み]せてください。			
\\	教師	教師[きょうし]	きょうし	
\\	彼は高校教師だ。	彼[かれ]は 高校[こうこう] 教師[きょうし]だ。	かれ は こうこう きょうし だ	
\\	彼[かれ]は 高校[こうこう]
\\	だ。			
\\	教授	教授[きょうじゅ]	きょうじゅ	
\\	彼は化学の教授です。	彼[かれ]は 化学[かがく]の 教授[きょうじゅ]です。	かれ は かがく の きょうじゅ です	
\\	彼[かれ]は 化学[かがく]の
\\	です。			
\\	伝える	伝[つた]える	つたえる	
\\	みんなにこのことを伝えてください。	みんなにこのことを 伝[つた]えてください。	みんな に この こと を つたえて ください	
\\	みんなにこのことを
\\	ください。			
\\	ぎらぎら	ぎらぎら	ぎらぎら	
\\	太陽がぎらぎらしている。	太陽[たいよう]がぎらぎらしている。	たいよう が ぎらぎら して いる	
\\	太陽[たいよう]が
\\	している。			
\\	鳥	鳥[とり]	とり	
\\	鳥が飛んでいます。	鳥[とり]が 飛[と]んでいます。	とり が とんで います	
\\	が 飛[と]んでいます。			
\\	鳴る	鳴[な]る	なる	
\\	今朝、5時に電話が鳴った。	今朝[けさ]、 5時[ごじ]に 電話[でんわ]が 鳴[な]った。	けさ ごじ に でんわ が なった	
\\	今朝[けさ]、 5時[ごじ]に 電話[でんわ]が
\\	鳴く	鳴[な]く	なく	
\\	どこかでネコが鳴いている。	どこかでネコが 鳴[な]いている。	どこか で ねこ が ないて いる	
\\	どこかでネコが
\\	声	声[こえ]	こえ	
\\	彼は大きな声で話した。	彼[かれ]は 大[おお]きな 声[こえ]で 話[はな]した。	かれ は おおき な こえ で はなした	
\\	彼[かれ]は 大[おお]きな
\\	で 話[はな]した。			
\\	そっち	そっち	そっち	
\\	そっちが私の部屋です。	そっちが 私[わたし]の 部屋[へや]です。	そっち が わたし の へや です	
\\	が 私[わたし]の 部屋[へや]です。			
\\	卵	卵[たまご]	たまご	
\\	ニワトリは卵を産みます。	ニワトリは 卵[たまご]を 産[う]みます。	にわとり は たまご を うみます	
\\	ニワトリは
\\	を 産[う]みます。			
\\	犬	犬[いぬ]	いぬ	
\\	この犬はとても賢い。	この 犬[いぬ]はとても 賢[かしこ]い。	この いぬ は とても かしこい	
\\	この
\\	はとても 賢[かしこ]い。			
\\	馬	馬[うま]	うま	
\\	彼は牧場で馬に乗った。	彼[かれ]は 牧場[ぼくじょう]で 馬[うま]に 乗[の]った。	かれ は ぼくじょう で うま に のった	
\\	彼[かれ]は 牧場[ぼくじょう]で
\\	に 乗[の]った。			
\\	駐車場	駐車場[ちゅうしゃじょう]	ちゅうしゃじょう	
\\	車は駐車場に止めてください。	車[くるま]は 駐車場[ちゅうしゃじょう]に 止[と]めてください。	くるま は ちゅうしゃじょう に とめて ください	
\\	車[くるま]は
\\	に 止[と]めてください。			
\\	ピンポン	ピンポン	ピンポン	
\\	昨日の夕方、友達とピンポンをした。	昨日[きのう]の 夕方[ゆうがた]、 友達[ともだち]とピンポンをした。	きのう の ゆうがた ともだち と ぴんぽん を した	
\\	昨日[きのう]の 夕方[ゆうがた]、 友達[ともだち]と
\\	をした。			
\\	騒ぐ	騒[さわ]ぐ	さわぐ	
\\	電車の中で騒がないでください。	電車[でんしゃ]の 中[なか]で 騒[さわ]がないでください。	でんしゃ の なか で さわがない で ください	
\\	電車[でんしゃ]の 中[なか]で
\\	ください。			
\\	刺す	刺[さ]す	さす	
\\	虫に腕を刺されました。	虫[むし]に 腕[うで]を 刺[さ]されました。	むし に うで を さされました	
\\	虫[むし]に 腕[うで]を
\\	刺身	刺身[さしみ]	さしみ	
\\	私は刺身は食べません。	私[わたし]は 刺身[さしみ]は 食[た]べません。	わたし は さしみ は たべません	
\\	私[わたし]は
\\	は 食[た]べません。			
\\	激しい	激[はげ]しい	はげしい	
\\	激しい雨が降っています。	激[はげ]しい 雨[あめ]が 降[ふ]っています。	はげしい あめ が ふって います	
\\	雨[あめ]が 降[ふ]っています。			
\\	ぺこぺこ	ぺこぺこ	ぺこぺこ	
\\	私はおなかがぺこぺこです。	私[わたし]はおなかがぺこぺこです。	わたし は おなか が ぺこぺこ です	
\\	私[わたし]はおなかが
\\	です。			
\\	驚く	驚[おどろ]く	おどろく	
\\	彼は血を見て驚いた。	彼[かれ]は 血[ち]を 見[み]て 驚[おどろ]いた。	かれ は ち を みて おどろいた	
\\	彼[かれ]は 血[ち]を 見[み]て
\\	倒れる	倒[たお]れる	たおれる	
\\	強風で木が倒れた。	強風[きょうふう]で 木[き]が 倒[たお]れた。	きょうふう で き が たおれた	
\\	強風[きょうふう]で 木[き]が
\\	倒す	倒[たお]す	たおす	
\\	そのスキーヤーはポールを倒した。	そのスキーヤーはポールを 倒[たお]した。	その すきーやー は ぽーる を たおした	
\\	そのスキーヤーはポールを
\\	傾向	傾向[けいこう]	けいこう	
\\	最近の若者は難しい本を読まない傾向がある。	最近[さいきん]の 若者[わかもの]は 難[むずか]しい 本[ほん]を 読[よ]まない 傾向[けいこう]がある。	さいきん の わかもの は むずかしい ほん を よまない けいこう が ある	
\\	最近[さいきん]の 若者[わかもの]は 難[むずか]しい 本[ほん]を 読[よ]まない
\\	がある。			
\\	ぺらぺら	ぺらぺら	ぺらぺら	
\\	彼女はドイツ語がぺらぺらです。	彼女[かのじょ]はドイツ 語[ご]がぺらぺらです。	かのじょ は どいつご が ぺらぺら です	
\\	彼女[かのじょ]はドイツ 語[ご]が
\\	です。			
\\	柔道	柔道[じゅうどう]	じゅうどう	
\\	私は柔道を習っています。	私[わたし]は 柔道[じゅうどう]を 習[なら]っています。	わたし は じゅうどう を ならって います	
\\	私[わたし]は
\\	を 習[なら]っています。			
\\	柔らかい	柔[やわ]らかい	やわらかい	
\\	布団がとても柔らかい。	布団[ふとん]がとても 柔[やわ]らかい。	ふとん が とても やわらかい	
\\	布団[ふとん]がとても
\\	柔らか	柔[やわ]らか	やわらか	
\\	彼の声は柔らかだ。	彼[かれ]の 声[こえ]は 柔[やわ]らかだ。	かれ の こえ は やわらか だ	
\\	彼[かれ]の 声[こえ]は
\\	だ。			
\\	主張	主張[しゅちょう]	しゅちょう	
\\	上司は私の主張を受け入れた。	上司[じょうし]は 私[わたし]の 主張[しゅちょう]を 受[う]け 入[い]れた。	じょうし は わたし の しゅちょう を うけいれた	
\\	上司[じょうし]は 私[わたし]の
\\	を 受[う]け 入[い]れた。			
\\	レインコート	レインコート	レインコート	
\\	雨なのでレインコートを着ました。	雨[あめ]なのでレインコートを 着[き]ました。	あめ な の で れいんこーと を きました	
\\	雨[あめ]なので
\\	を 着[き]ました。			
\\	引っ張る	引[ひ]っ 張[ぱ]る	ひっぱる	
\\	娘が私の手を引っ張った。	娘[むすめ]が 私[わたし]の 手[て]を 引[ひ]っ 張[ぱ]った。	むすめ が わたし の て を ひっぱった	
\\	娘[むすめ]が 私[わたし]の 手[て]を
\\	突き当たる	突[つ]き 当[あ]たる	つきあたる	
\\	突き当たったら右に曲がってください。	突[つ]き 当[あ]たったら 右[みぎ]に 曲[ま]がってください。	つきあたったら みぎ に まがって ください	
\\	右[みぎ]に 曲[ま]がってください。			
\\	突き当たり	突[つ]き 当[あ]たり	つきあたり	
\\	この先は突き当たりです。	この 先[さき]は 突[つ]き 当[あ]たりです。	この さき は つきあたり です	
\\	この 先[さき]は
\\	です。			
\\	壁	壁[かべ]	かべ	
\\	壁に絵が掛かっている。	壁[かべ]に 絵[え]が 掛[か]かっている。	かべ に え が かかって いる	
\\	に 絵[え]が 掛[か]かっている。			
\\	アイス	アイス	アイス	
\\	暑いのでアイスを食べました。	暑[あつ]いのでアイスを 食[た]べました。	あつい の で あいす を たべました	
\\	暑[あつ]いので
\\	を 食[た]べました。			
\\	弾く	弾[ひ]く	ひく	
\\	彼はギターを弾きます。	彼[かれ]はギターを 弾[ひ]きます。	かれ は ぎたー を ひきます	
\\	彼[かれ]はギターを
\\	丸い	丸[まる]い	まるい	
\\	地球は丸い。	地球[ちきゅう]は 丸[まる]い。	ちきゅう は まるい。	
\\	地球[ちきゅう]は
\\	丸	丸[まる]	まる	
\\	紙に大きな丸を書きました。	紙[かみ]に 大[おお]きな 丸[まる]を 書[か]きました。	かみ に おおき な まる を かきました	
\\	紙[かみ]に 大[おお]きな
\\	を 書[か]きました。			
\\	攻撃	攻撃[こうげき]	こうげき	
\\	2003年にアメリカはイラクを攻撃した。	2003年[にせんさんねん]にアメリカはイラクを 攻撃[こうげき]した。	にせんさんねん に あめりか は いらく を こうげき した	
\\	2003年[にせんさんねん]にアメリカはイラクを
\\	した。			
\\	いつごろ	いつごろ	いつごろ	
\\	いつごろ夏休みを取りますか。	いつごろ 夏休[なつやす]みを 取[と]りますか。	いつごろ なつやすみ を とります か	
\\	夏休[なつやす]みを 取[と]りますか。			
\\	絶対に	絶対[ぜったい]に	ぜったいに	
\\	絶対に駄目です。	絶対[ぜったい]に 駄目[だめ]です。	ぜったいに だめ です	
\\	駄目[だめ]です。			
\\	消防車	消防車[しょうぼうしゃ]	しょうぼうしゃ	
\\	消防車が4台も来た。	消防車[しょうぼうしゃ]が 4台[よんだい]も 来[き]た。	しょうぼうしゃ が よんだい も きた	
\\	が 4台[よんだい]も 来[き]た。			
\\	嫌	嫌[いや]	いや	
\\	私は待つのが嫌だ。	私[わたし]は 待[ま]つのが 嫌[いや]だ。	わたし は まつ の が いや だ	
\\	私[わたし]は 待[ま]つのが
\\	だ。			
\\	大嫌い	大嫌[だいきら]い	だいきらい	
\\	私はテストが大嫌い。	私[わたし]はテストが 大嫌[だいきら]い。	わたし は てすと が だいきらい	
\\	私[わたし]はテストが
\\	ウィスキー	ウィスキー	ウィスキー	
\\	このウィスキーは強いよ。	このウィスキーは 強[つよ]いよ。	この うぃすきー は つよい よ	
\\	この
\\	は 強[つよ]いよ。			
\\	大抵	大抵[たいてい]	たいてい	
\\	朝食は大抵7時頃に食べます。	朝食[ちょうしょく]は 大抵[たいてい] 7時頃[しちじごろ]に 食[た]べます。	ちょうしょく は たいてい しちじごろ に たべます	
\\	朝食[ちょうしょく]は
\\	7時頃[しちじごろ]に 食[た]べます。			
\\	大勢	大勢[おおぜい]	おおぜい	
\\	大勢で食事に出かけました。	大勢[おおぜい]で 食事[しょくじ]に 出[で]かけました。	おおぜい で しょくじ に でかけました	
\\	で 食事[しょくじ]に 出[で]かけました。			
\\	姿	姿[すがた]	すがた	
\\	遠くに彼女の姿が見えた。	遠[とお]くに 彼女[かのじょ]の 姿[すがた]が 見[み]えた。	とおく に かのじょ の すがた が みえた	
\\	遠[とお]くに 彼女[かのじょ]の
\\	が 見[み]えた。			
\\	姿勢	姿勢[しせい]	しせい	
\\	あの子はいつも姿勢が悪い。	あの 子[こ]はいつも 姿勢[しせい]が 悪[わる]い。	あの こ は いつも しせい が わるい	
\\	あの 子[こ]はいつも
\\	が 悪[わる]い。			
\\	エアメール	エアメール	エアメール	
\\	彼にエアメールを送りました。	彼[かれ]にエアメールを 送[おく]りました。	かれ に えあめーる を おくりました	
\\	彼[かれ]に
\\	を 送[おく]りました。			
\\	恐ろしい	恐[おそ]ろしい	おそろしい	
\\	昨夜恐ろしい夢を見た。	昨夜[ゆうべ] 恐[おそ]ろしい 夢[ゆめ]を 見[み]た。	ゆうべ おそろしい ゆめ を みた	
\\	昨夜[ゆうべ]
\\	夢[ゆめ]を 見[み]た。			
\\	怖い	怖[こわ]い	こわい	
\\	私は犬が怖いです。	私[わたし]は 犬[いぬ]が 怖[こわ]いです。	わたし は いぬ が こわい です	
\\	私[わたし]は 犬[いぬ]が
\\	です。			
\\	糸	糸[いと]	いと	
\\	母は糸を針に通した。	母[はは]は 糸[いと]を 針[はり]に 通[とお]した。	はは は いと を はり に とおした	
\\	母[はは]は
\\	を 針[はり]に 通[とお]した。			
\\	孫	孫[まご]	まご	
\\	昨日、孫が生まれました。	昨日[きのう]、 孫[まご]が 生[う]まれました。	きのう まご が うまれました	
\\	昨日[きのう]、
\\	が 生[う]まれました。			
\\	シャープペンシル	シャープペンシル	シャープペンシル	
\\	私のシャープペンシルがありません。	私[わたし]のシャープペンシルがありません。	わたし の しゃーぷぺんしる が ありません	
\\	私[わたし]の
\\	がありません。			
\\	木綿	木綿[もめん]	もめん	
\\	彼女は木綿のシャツを着ています。	彼女[かのじょ]は 木綿[もめん]のシャツを 着[き]ています。	かのじょ は もめん の しゃつ を きて います	
\\	彼女[かのじょ]は
\\	のシャツを 着[き]ています。			
\\	偉い	偉[えら]い	えらい	
\\	彼は偉い学者です。	彼[かれ]は 偉[えら]い 学者[がくしゃ]です。	かれ は えらい がくしゃ です	
\\	彼[かれ]は
\\	学者[がくしゃ]です。			
\\	爪	爪[つめ]	つめ	
\\	爪が伸びています。	爪[つめ]が 伸[の]びています。	つめ が のびて います	
\\	が 伸[の]びています。			
\\	机	机[つくえ]	つくえ	
\\	新しい机を買ってもらいました。	新[あたら]しい 机[つくえ]を 買[か]ってもらいました。	あたらしい つくえ を かって もらいました	
\\	新[あたら]しい
\\	を 買[か]ってもらいました。			
\\	セロテープ	セロテープ	セロテープ	
\\	セロテープはありますか。	セロテープはありますか。	せろてーぷ は あります か	
\\	はありますか。			
\\	棚	棚[たな]	たな	
\\	大きな棚はとても便利です。	大[おお]きな 棚[たな]はとても 便利[べんり]です。	おおき な たな は とても べんり です	
\\	大[おお]きな
\\	はとても 便利[べんり]です。			
\\	本棚	本棚[ほんだな]	ほんだな	
\\	これはとても大きな本棚ですね。	これはとても 大[おお]きな 本棚[ほんだな]ですね。	これ は とても おおき な ほんだな です ね	
\\	これはとても 大[おお]きな
\\	ですね。			
\\	方針	方針[ほうしん]	ほうしん	
\\	今後の方針が決まった。	今後[こんご]の 方針[ほうしん]が 決[き]まった。	こんご の ほうしん が きまった	
\\	今後[こんご]の
\\	が 決[き]まった。			
\\	釣る	釣[つ]る	つる	
\\	昨日大きな魚を釣りました。	昨日大[きのう おお]きな 魚[さかな]を 釣[つ]りました。	きのう おおき な さかな を つりました	
\\	昨日大[きのう おお]きな 魚[さかな]を
\\	やけど	やけど	やけど	
\\	彼は手にやけどをしました。	彼[かれ]は 手[て]にやけどをしました。	かれ は て に やけど を しました	
\\	彼[かれ]は 手[て]に
\\	をしました。			
\\	お釣り	お 釣[つ]り	おつり	
\\	母はお釣りを確かめた。	母[はは]はお 釣[つ]りを 確[たし]かめた。	はは は おつり を たしかめた	
\\	母[はは]は
\\	を 確[たし]かめた。			
\\	鍵	鍵[かぎ]	かぎ	
\\	出かける時は鍵を掛けてください。	出[で]かける 時[とき]は 鍵[かぎ]を 掛[か]けてください。	でかける とき は かぎ を かけて ください	
\\	出[で]かける 時[とき]は
\\	を 掛[か]けてください。			
\\	鍋	鍋[なべ]	なべ	
\\	鍋にスープが入っているよ。	鍋[なべ]にスープが 入[はい]っているよ。	なべ に すーぷ が はいって いる よ	
\\	にスープが 入[はい]っているよ。			
\\	寿司	寿司[すし]	すし	
\\	彼女は寿司を初めて食べました。	彼女[かのじょ]は 寿司[すし]を 初[はじ]めて 食[た]べました。	かのじょ は すし を はじめて たべました 。	
\\	彼女[かのじょ]は
\\	を 初[はじ]めて 食[た]べました。			
\\	アイスコーヒー	アイスコーヒー	アイスコーヒー	
\\	アイスコーヒーをください。	アイスコーヒーをください。	あいすこーひー を ください	
\\	をください。			
\\	泥棒	泥棒[どろぼう]	どろぼう	
\\	近所に泥棒が入った。	近所[きんじょ]に 泥棒[どろぼう]が 入[はい]った。	きんじょ に どろぼう が はいった	
\\	近所[きんじょ]に
\\	が 入[はい]った。			
\\	湯	湯[ゆ]	ゆ	
\\	お風呂のお湯が一杯です。	お 風呂[ふろ]のお 湯[ゆ]が 一杯[いっぱい]です。	おふろ の おゆ が いっぱい です	
\\	お 風呂[ふろ]のお
\\	が 一杯[いっぱい]です。			
\\	沸く	沸[わ]く	わく	
\\	お風呂が沸きました。	お 風呂[ふろ]が 沸[わ]きました。	お ふろ が わきました。	
\\	お 風呂[ふろ]が
\\	沸かす	沸[わ]かす	わかす	
\\	お湯を沸かしてください。	お 湯[ゆ]を 沸[わ]かしてください。	おゆ を わかして ください	
\\	お 湯[ゆ]を
\\	ください。			
\\	あっち	あっち	あっち	
\\	郵便局はあっちです。	郵便局[ゆうびんきょく]はあっちです。	ゆうびんきょく は あっち です	
\\	郵便局[ゆうびんきょく]は
\\	です。			
\\	洗濯機	洗濯機[せんたくき]	せんたくき	
\\	新しい洗濯機を買いました。	新[あたら]しい 洗濯機[せんたっき]を 買[か]いました。	あたらしい せんたっき を かいました	
\\	新[あたら]しい
\\	を 買[か]いました。			
\\	洗濯	洗濯[せんたく]	せんたく	
\\	一週間、洗濯をしていない。	一週間[いっしゅうかん]、 洗濯[せんたく]をしていない。	いっしゅうかん せんたく を して いない	
\\	一週間[いっしゅうかん]、
\\	をしていない。			
\\	濡れる	濡[ぬ]れる	ぬれる	
\\	雨で濡れてしまった。	雨[あめ]で 濡[ぬ]れてしまった。	あめ で ぬれて しまった	
\\	雨[あめ]で
\\	喫茶店	喫茶店[きっさてん]	きっさてん	
\\	喫茶店でコーヒーを飲んだ。	喫茶店[きっさてん]でコーヒーを 飲[の]んだ。	きっさてん で こーひー を のんだ	
\\	でコーヒーを 飲[の]んだ。			
\\	あんなに	あんなに	あんなに	
\\	あんなにいい人はいません。	あんなにいい 人[ひと]はいません。	あんなに いい ひと は いません	
\\	いい 人[ひと]はいません。			
\\	怠ける	怠[なま]ける	なまける	
\\	怠けていないで、手伝って。	怠[なま]けていないで、 手伝[てつだ]って。	なまけて いない で てつだって	
\\	、 手伝[てつだ]って。			
\\	一生懸命	一生懸命[いっしょうけんめい]	いっしょうけんめい	
\\	彼は毎日一生懸命働いている。	彼[かれ]は 毎日[まいにち] 一生懸命[いっしょうけんめい] 働[はたら]いている。	かれ は まいにち いっしょうけんめい はたらいて いる	
\\	彼[かれ]は 毎日[まいにち]
\\	働[はたら]いている。			
\\	休憩	休憩[きゅうけい]	きゅうけい	
\\	少し休憩しましょう。	少[すこ]し 休憩[きゅうけい]しましょう。	すこし きゅうけい しましょう	
\\	少[すこ]し
\\	しましょう。			
\\	天井	天井[てんじょう]	てんじょう	
\\	この部屋は天井が高いですね。	この 部屋[へや]は 天井[てんじょう]が 高[たか]いですね。	この へや は てんじょう が たかい です ね	
\\	この 部屋[へや]は
\\	が 高[たか]いですね。			
\\	いたずら	いたずら	いたずら	
\\	その子はいたずらが大好きだ。	その 子[こ]はいたずらが 大好[だいす]きだ。	その こ は いたずら が だいすき だ	
\\	その 子[こ]は
\\	が 大好[だいす]きだ。			
\\	納豆	納豆[なっとう]	なっとう	
\\	私は納豆をよく食べます。	私[わたし]は 納豆[なっとう]をよく 食[た]べます。	わたし は なっとう を よく たべます	
\\	私[わたし]は
\\	をよく 食[た]べます。			
\\	化粧	化粧[けしょう]	けしょう	
\\	彼女は化粧が上手い。	彼女[かのじょ]は 化粧[けしょう]が 上手[うま]い。	かのじょ は けしょう が うまい	
\\	彼女[かのじょ]は
\\	が 上手[うま]い。			
\\	畑	畑[はたけ]	はたけ	
\\	母は畑仕事が好きだ。	母[はは]は 畑[はたけ] 仕事[しごと]が 好[す]きだ。	はは は はたけ しごと が すき だ	
\\	母[はは]は
\\	仕事[しごと]が 好[す]きだ。			
\\	炊く	炊[た]く	たく	
\\	母は毎朝ご飯を炊く。	母[はは]は 毎朝[まいあさ]ご 飯[はん]を 炊[た]く。	はは は まいあさ ごはん を たく	
\\	母[はは]は 毎朝[まいあさ]ご 飯[はん]を
\\	ウェートレス	ウェートレス	ウェートレス	
\\	ウェートレスを呼んだ。	ウェートレスを 呼[よ]んだ。	うぇーとれす を よんだ	
\\	を 呼[よ]んだ。			
\\	自炊	自炊[じすい]	じすい	
\\	私は毎日、自炊している。	私[わたし]は 毎日[まいにち]、 自炊[じすい]している。	わたし は まいにち じすい して いる	
\\	私[わたし]は 毎日[まいにち]、
\\	している。			
\\	灰皿	灰皿[はいざら]	はいざら	
\\	灰皿をください。	灰皿[はいざら]をください。	はいざら を ください	
\\	をください。			
\\	灰	灰[はい]	はい	
\\	火事ですべて灰になった。	火事[かじ]ですべて 灰[はい]になった。	かじ で すべて はい に なった	
\\	火事[かじ]ですべて
\\	になった。			
\\	灰色	灰色[はいいろ]	はいいろ	
\\	今日の空は灰色だ。	今日[きょう]の 空[そら]は 灰色[はいいろ]だ。	きょう の そら は はいいろ だ	
\\	今日[きょう]の 空[そら]は
\\	だ。			
\\	おばさん	おばさん	おばさん	
\\	おばさん一家が遊びに来ました。	おばさん 一家[いっか]が 遊[あそ]びに 来[き]ました。	おばさん いっか が あそび に きました	
\\	一家[いっか]が 遊[あそ]びに 来[き]ました。			
\\	西暦	西暦[せいれき]	せいれき	
\\	西暦1964年に東京でオリンピックがあった。	西暦[せいれき] 1964年[せんきゅうひゃくろくじゅうよねん]に 東京[とうきょう]でオリンピックがあった。	せいれき せんきゅうひゃくろくじゅうよねん に とうきょう で おりんぴっく が あった	
\\	1964年[せんきゅうひゃくろくじゅうよねん]に 東京[とうきょう]でオリンピックがあった。			
\\	毛	毛[け]	け	
\\	猫の毛がセーターに付いた。	猫[ねこ]の 毛[け]がセーターに 付[つ]いた。	ねこ の け が せーたー に ついた	
\\	猫[ねこ]の
\\	がセーターに 付[つ]いた。			
\\	髪	髪[かみ]	かみ	
\\	昨日、髪を切りました。	昨日[きのう]、 髪[かみ]を 切[き]りました。	きのう かみ を きりました	
\\	昨日[きのう]、
\\	を 切[き]りました。			
\\	舌	舌[した]	した	
\\	舌を出してください。	舌[した]を 出[だ]してください。	した を だして ください	
\\	を 出[だ]してください。			
\\	くし	くし	くし	
\\	彼はくしで髪をとかした。	彼[かれ]はくしで 髪[かみ]をとかした。	かれ は くし で かみ を とかした	
\\	彼[かれ]は
\\	で 髪[かみ]をとかした。			
\\	臭い	臭[くさ]い	くさい	
\\	納豆は臭い。	納豆[なっとう]は 臭[くさ]い。	なっとう は くさい	
\\	納豆[なっとう]は
\\	匂い	匂[にお]い	におい	
\\	いい匂いがする。	いい 匂[にお]いがする。	いい におい が する	
\\	いい
\\	がする。			
\\	居る	居[い]る	いる	
\\	今日は一日中家に居ました。	今日[きょう]は 一日中家[いちにちじゅう うち]に 居[い]ました。	きょう は いちにちじゅう うち に いました	
\\	今日[きょう]は 一日中家[いちにちじゅう うち]に
\\	履く	履[は]く	はく	
\\	彼女はブーツを履いています。	彼女[かのじょ]はブーツを 履[は]いています。	かのじょ は ぶーつ を はいて います	
\\	彼女[かのじょ]はブーツを
\\	クリーニング	クリーニング	クリーニング	
\\	スーツをクリーニングに出しました。	スーツをクリーニングに 出[だ]しました。	すーつ を くりーにんぐ に だしました	
\\	スーツを
\\	に 出[だ]しました。			
\\	戸	戸[と]	と	
\\	部屋の戸が開いています。	部屋[へや]の 戸[と]が 開[あ]いています。	へや の と が あいて います	
\\	部屋[へや]の
\\	が 開[あ]いています。			
\\	扇風機	扇風機[せんぷうき]	せんぷうき	
\\	暑いから扇風機をつけよう。	暑[あつ]いから 扇風機[せんぷうき]をつけよう。	あつい から せんぷうき を つけよう	
\\	暑[あつ]いから
\\	をつけよう。			
\\	豚	豚[ぶた]	ぶた	
\\	豚はすごく鼻が良い。	豚[ぶた]はすごく 鼻[はな]が 良[い]い。	ぶた は すごく はな が いい	
\\	はすごく 鼻[はな]が 良[い]い。			
\\	豚肉	豚肉[ぶたにく]	ぶたにく	
\\	豚肉は美味しいです。	豚肉[ぶたにく]は 美味[おい]しいです。	ぶたにく は おいしい です	
\\	は 美味[おい]しいです。			
\\	ごちそう	ごちそう	ごちそう	
\\	テーブルの上にごちそうが並んでいる。	テーブルの 上[うえ]にごちそうが 並[なら]んでいる。	てーぶる の うえ に ごちそう が ならんで いる	
\\	テーブルの 上[うえ]に
\\	が 並[なら]んでいる。			
\\	鶏	鶏[にわとり]	にわとり	
\\	その鶏は毎朝鳴きます。	その 鶏[にわとり]は 毎朝鳴[まいあさ な]きます。	その にわとり は まいあさ なきます	
\\	その
\\	は 毎朝鳴[まいあさ な]きます。			
\\	腕時計	腕時計[うでどけい]	うでどけい	
\\	私の腕時計は遅れている。	私[わたし]の 腕時計[うでどけい]は 遅[おく]れている。	わたし の うでどけい は おくれて いる	
\\	私[わたし]の
\\	は 遅[おく]れている。			
\\	お菓子	お 菓子[かし]	おかし	
\\	みんなにお菓子をあげましょう。	みんなにお 菓子[かし]をあげましょう。	みんな に おかし を あげましょう	
\\	みんなに
\\	をあげましょう。			
\\	お辞儀	お 辞儀[じぎ]	おじぎ	
\\	皆、社長にお辞儀をした。	皆[みんな]、 社長[しゃちょう]にお 辞儀[じぎ]をした。	みんな しゃちょう に おじぎ を した	
\\	皆[みんな]、 社長[しゃちょう]に
\\	をした。			
\\	ごちそうする	ごちそうする	ごちそうする	
\\	今日の夕食は私がごちそうします。	今日[きょう]の 夕食[ゆうしょく]は 私[わたし]がごちそうします。	きょう の ゆうしょく は わたし が ごちそう します	
\\	今日[きょう]の 夕食[ゆうしょく]は 私[わたし]が
\\	寝坊	寝坊[ねぼう]	ねぼう	
\\	今朝は寝坊しました。	今朝[けさ]は 寝坊[ねぼう]しました。	けさ は ねぼう しました	
\\	今朝[けさ]は
\\	しました。			
\\	旗	旗[はた]	はた	
\\	旗が風に揺れている。	旗[はた]が 風[かぜ]に 揺[ゆ]れている。	はた が かぜ に ゆれて いる	
\\	が 風[かぜ]に 揺[ゆ]れている。			
\\	鉛筆	鉛筆[えんぴつ]	えんぴつ	
\\	鉛筆を貸して下さい。	鉛筆[えんぴつ]を 貸[か]して 下[くだ]さい。	えんぴつ を かして ください	
\\	を 貸[か]して 下[くだ]さい。			
\\	万年筆	万年筆[まんねんひつ]	まんねんひつ	
\\	父に万年筆をプレゼントしました。	父[ちち]に 万年筆[まんねんひつ]をプレゼントしました。	ちち に まんねんひつ を ぷれぜんと しました	
\\	父[ちち]に
\\	をプレゼントしました。			
\\	こぼれる	こぼれる	こぼれる	
\\	彼女の目から涙がこぼれた。	彼女[かのじょ]の 目[め]から 涙[なみだ]がこぼれた。	かのじょ の め から なみだ が こぼれた	
\\	彼女[かのじょ]の 目[め]から 涙[なみだ]が
\\	箱	箱[はこ]	はこ	
\\	この箱は重い。	この 箱[はこ]は 重[おも]い。	この はこ は おもい	
\\	この
\\	は 重[おも]い。			
\\	本箱	本箱[ほんばこ]	ほんばこ	
\\	雑誌を本箱に入れました。	雑誌[ざっし]を 本箱[ほんばこ]に 入[い]れました。	ざっし を ほんばこ に いれました	
\\	雑誌[ざっし]を
\\	に 入[い]れました。			
\\	手袋	手袋[てぶくろ]	てぶくろ	
\\	寒いので手袋をしました。	寒[さむ]いので 手袋[てぶくろ]をしました。	さむい の で てぶくろ を しました	
\\	寒[さむ]いので
\\	をしました。			
\\	袋	袋[ふくろ]	ふくろ	
\\	袋はいりません。	袋[ふくろ]はいりません。	ふくろ は いりません	
\\	はいりません。			
\\	こんなに	こんなに	こんなに	
\\	こんなに面白い本は初めて読んだ。	こんなに 面白[おもしろ]い 本[ほん]は 初[はじ]めて 読[よ]んだ。	こんなに おもしろい ほん は はじめて よんだ	
\\	面白[おもしろ]い 本[ほん]は 初[はじ]めて 読[よ]んだ。			
\\	財布	財布[さいふ]	さいふ	
\\	会社に財布を忘れた。	会社[かいしゃ]に 財布[さいふ]を 忘[わす]れた。	かいしゃ に さいふ を わすれた	
\\	会社[かいしゃ]に
\\	を 忘[わす]れた。			
\\	毛布	毛布[もうふ]	もうふ	
\\	この毛布は暖かい。	この 毛布[もうふ]は 暖[あたた]かい。	この もうふ は あたたかい	
\\	この
\\	は 暖[あたた]かい。			
\\	布団	布団[ふとん]	ふとん	
\\	母が布団を干している。	母[はは]が 布団[ふとん]を 干[ほ]している。	はは が ふとん を ほして いる	
\\	母[はは]が
\\	を 干[ほ]している。			
\\	小包	小包[こづつみ]	こづつみ	
\\	フランスの友達から小包が届いた。	フランスの 友達[ともだち]から 小包[こづつみ]が 届[とど]いた。	ふらんす の ともだち から こづつみ が とどいた	
\\	フランスの 友達[ともだち]から
\\	が 届[とど]いた。			
\\	ざあざあ	ざあざあ	ざあざあ	
\\	雨がざあざあ降っている。	雨[あめ]がざあざあ 降[ふ]っている。	あめ が ざあざあ ふって いる	
\\	雨[あめ]が
\\	降[ふ]っている。			
\\	包む	包[つつ]む	つつむ	
\\	プレゼントをきれいな紙で包みました。	プレゼントをきれいな 紙[かみ]で 包[つつ]みました。	ぷれぜんと を きれい な かみ で つつみました	
\\	プレゼントをきれいな 紙[かみ]で
\\	飾る	飾[かざ]る	かざる	
\\	テーブルの上に花を飾りました。	テーブルの 上[うえ]に 花[はな]を 飾[かざ]りました。	てーぶる の うえ に はな を かざりました	
\\	テーブルの 上[うえ]に 花[はな]を
\\	手帳	手帳[てちょう]	てちょう	
\\	新しい手帳を買いました。	新[あたら]しい 手帳[てちょう]を 買[か]いました。	あたらしい てちょう を かいました	
\\	新[あたら]しい
\\	を 買[か]いました。			
\\	電話帳	電話帳[でんわちょう]	でんわちょう	
\\	電話帳を見せてください。	電話帳[でんわちょう]を 見[み]せてください。	でんわちょう を みせて ください	
\\	を 見[み]せてください。			
\\	ジャガイモ	ジャガイモ	ジャガイモ	
\\	ポテトサラダを作るから、ジャガイモを買ってきて。	ポテトサラダを 作[つく]るから、ジャガイモを 買[か]ってきて。	ぽてと さらだ を つくる から じゃがいも を かって きて	
\\	ポテトサラダを 作[つく]るから、
\\	を 買[か]ってきて。			
\\	牛乳	牛乳[ぎゅうにゅう]	ぎゅうにゅう	
\\	私は毎朝、牛乳を飲む。	私[わたし]は 毎朝[まいあさ]、 牛乳[ぎゅうにゅう]を 飲[の]む。	わたし は まいあさ ぎゅうにゅう を のむ	
\\	私[わたし]は 毎朝[まいあさ]、
\\	を 飲[の]む。			
\\	玄関	玄関[げんかん]	げんかん	
\\	玄関に花を飾りました。	玄関[げんかん]に 花[はな]を 飾[かざ]りました。	げんかん に はな を かざりました	
\\	に 花[はな]を 飾[かざ]りました。			
\\	眼鏡	眼鏡[めがね]	めがね	
\\	彼は眼鏡をかけています。	彼[かれ]は 眼鏡[めがね]をかけています。	かれ は めがね を かけて います	
\\	彼[かれ]は
\\	をかけています。			
\\	眠い	眠[ねむ]い	ねむい	
\\	今日はとても眠いです。	今日[きょう]はとても 眠[ねむ]いです。	きょう は とても ねむい です	
\\	今日[きょう]はとても
\\	です。			
\\	ソファー	ソファー	ソファー	
\\	このソファーは気持ちがいい。	このソファーは 気持[きも]ちがいい。	この そふぁー は きもち が いい	
\\	この
\\	は 気持[きも]ちがいい。			
\\	眠る	眠[ねむ]る	ねむる	
\\	昨日は8時間眠りました。	昨日[きのう]は 8時間[はちじかん] 眠[ねむ]りました。	きのう は はちじかん ねむりました	
\\	昨日[きのう]は 8時間[はちじかん]
\\	封筒	封筒[ふうとう]	ふうとう	
\\	その手紙を封筒に入れた。	その 手紙[てがみ]を 封筒[ふうとう]に 入[い]れた。	その てがみ を ふうとう に いれた	
\\	その 手紙[てがみ]を
\\	に 入[い]れた。			
\\	出掛ける	出掛[でか]ける	でかける	
\\	主人はもう出掛けました。	主人[しゅじん]はもう 出掛[でか]けました。	しゅじん は もう でかけました	
\\	主人[しゅじん]はもう
\\	掛かる	掛[か]かる	かかる	
\\	壁に大きな時計が掛かっています。	壁[かべ]に 大[おお]きな 時計[とけい]が 掛[か]かっています。	かべ に おおき な とけい が かかって います	
\\	壁[かべ]に 大[おお]きな 時計[とけい]が
\\	たばこ	たばこ	たばこ	
\\	彼はたばこを吸いません。	彼[かれ]はたばこを 吸[す]いません。	かれ は たばこ を すいません	
\\	彼[かれ]は
\\	を 吸[す]いません。			
\\	掛け算	掛[か]け 算[ざん]	かけざん	
\\	弟は掛け算を習っている。	弟[おとうと]は 掛[か]け 算[ざん]を 習[なら]っている。	おとうと は かけざん を ならって いる	
\\	弟[おとうと]は
\\	を 習[なら]っている。			
\\	拍手	拍手[はくしゅ]	はくしゅ	
\\	大きな拍手が上がった。	大[おお]きな 拍手[はくしゅ]が 上[あ]がった。	おおき な はくしゅ が あがった	
\\	大[おお]きな
\\	が 上[あ]がった。			
\\	掃除	掃除[そうじ]	そうじ	
\\	週末は部屋の掃除をしました。	週末[しゅうまつ]は 部屋[へや]の 掃除[そうじ]をしました。	しゅうまつ は へや の そうじ を しました	
\\	週末[しゅうまつ]は 部屋[へや]の
\\	をしました。			
\\	掃く	掃[は]く	はく	
\\	床をほうきで掃きました。	床[ゆか]をほうきで 掃[は]きました。	ゆか を ほうき で はきました	
\\	床[ゆか]をほうきで
\\	ちょうど	ちょうど	ちょうど	
\\	値段はちょうど3万円です。	値段[ねだん]はちょうど 3万円[さんまんえん]です。	ねだん は ちょうど さんまんえん です	
\\	値段[ねだん]は
\\	3万円[さんまんえん]です。			
\\	掃除機	掃除機[そうじき]	そうじき	
\\	掃除機が壊れた。	掃除機[そうじき]が 壊[こわ]れた。	そうじき が こわれた	
\\	が 壊[こわ]れた。			
\\	握る	握[にぎ]る	にぎる	
\\	少女は母親の手を握った。	少女[しょうじょ]は 母親[ははおや]の 手[て]を 握[にぎ]った。	しょうじょ は ははおや の て を にぎった	
\\	少女[しょうじょ]は 母親[ははおや]の 手[て]を
\\	握手	握手[あくしゅ]	あくしゅ	
\\	彼らは握手をした。	彼[かれ]らは 握手[あくしゅ]をした。	かれら は あくしゅ を した	
\\	彼[かれ]らは
\\	をした。			
\\	迎える	迎[むか]える	むかえる	
\\	姉が空港まで迎えに来てくれます。	姉[あね]が 空港[くうこう]まで 迎[むか]えに 来[き]てくれます。	あね が くうこう まで むかえに きて くれます	
\\	姉[あね]が 空港[くうこう]まで
\\	に 来[き]てくれます。			
\\	ちょっと	ちょっと	ちょっと	
\\	もうちょっと塩を入れて。	もうちょっと 塩[しお]を 入[い]れて。	もう ちょっと しお を いれて	
\\	もう
\\	塩[しお]を 入[い]れて。			
\\	お巡りさん	お 巡[まわ]りさん	おまわりさん	
\\	あのお巡りさんに道を聞きましょう。	あのお 巡[まわ]りさんに 道[みち]を 聞[き]きましょう。	あの おまわりさん に みち を ききましょう	
\\	あの
\\	に 道[みち]を 聞[き]きましょう。			
\\	幾つ	幾[いく]つ	いくつ	
\\	娘さんは幾つになりましたか。	娘[むすめ]さんは 幾[いく]つになりましたか。	むすめさん は いくつ に なりました か	
\\	娘[むすめ]さんは
\\	になりましたか。			
\\	幾ら	幾[いく]ら	いくら	
\\	この靴は幾らですか。	この 靴[くつ]は 幾[いく]らですか。	この くつ は いくら です か	
\\	この 靴[くつ]は
\\	ですか。			
\\	冗談	冗談[じょうだん]	じょうだん	
\\	冗談は止めてください。	冗談[じょうだん]は 止[や]めてください。	じょうだん は やめて ください	
\\	は 止[や]めてください。			
\\	つく	つく	つく	
\\	部屋の電気がついています。	部屋[へや]の 電気[でんき]がついています。	へや の でんき が ついて います	
\\	部屋[へや]の 電気[でんき]が
\\	穴	穴[あな]	あな	
\\	靴下に穴が開いている。	靴下[くつした]に 穴[あな]が 開[あ]いている。	くつした に あな が あいて いる	
\\	靴下[くつした]に
\\	が 開[あ]いている。			
\\	寂しい	寂[さび]しい	さびしい	
\\	これは寂しい曲ですね。	これは 寂[さび]しい 曲[きょく]ですね。	これ は さびしい きょく です ね	
\\	これは
\\	曲[きょく]ですね。			
\\	丁寧	丁寧[ていねい]	ていねい	
\\	彼女はいつも丁寧に仕事をします。	彼女[かのじょ]はいつも 丁寧[ていねい]に 仕事[しごと]をします。	かのじょ は いつも ていねい に しごと を します	
\\	彼女[かのじょ]はいつも
\\	に 仕事[しごと]をします。			
\\	可哀相	可哀相[かわいそう]	かわいそう	
\\	その可哀相な子供たちは食べるものがない。	その 可哀相[かわいそう]な 子供[こども]たちは 食[た]べるものがない。	その かわいそう な こどもたち は たべる もの が ない	
\\	その
\\	な 子供[こども]たちは 食[た]べるものがない。			
\\	てんぷら	てんぷら	てんぷら	
\\	私はえびのてんぷらが好きです。	私[わたし]はえびのてんぷらが 好[す]きです。	わたし は えび の てんぷら が すき です	
\\	私[わたし]はえびの
\\	が 好[す]きです。			
\\	怪我	怪我[けが]	けが	
\\	彼女は腕を怪我した。	彼女[かのじょ]は 腕[うで]を 怪我[けが]した。	かのじょ は うで を けがした	
\\	彼女[かのじょ]は 腕[うで]を
\\	した。			
\\	我慢	我慢[がまん]	がまん	
\\	彼のわがままには我慢できません。	彼[かれ]のわがままには 我慢[がまん]できません。	かれ の わがまま に は がまん できません	
\\	彼[かれ]のわがままには
\\	できません。			
\\	幼稚園	幼稚園[ようちえん]	ようちえん	
\\	娘は幼稚園に通っています。	娘[むすめ]は 幼稚園[ようちえん]に 通[かよ]っています。	むすめ は ようちえん に かよって います	
\\	娘[むすめ]は
\\	に 通[かよ]っています。			
\\	隠れる	隠[かく]れる	かくれる	
\\	太陽が雲に隠れた。	太陽[たいよう]が 雲[くも]に 隠[かく]れた。	たいよう が くも に かくれた	
\\	太陽[たいよう]が 雲[くも]に
\\	とんとん	とんとん	とんとん	
\\	母の肩をとんとんたたいた。	母[はは]の 肩[かた]をとんとんたたいた。	はは の かた を とんとん たたいた	
\\	母[はは]の 肩[かた]を
\\	たたいた。			
\\	日陰	日陰[ひかげ]	ひかげ	
\\	暑いので日陰で休みましょう。	暑[あつ]いので 日陰[ひかげ]で 休[やす]みましょう。	あつい の で ひかげ で やすみましょう	
\\	暑[あつ]いので
\\	で 休[やす]みましょう。			
\\	随分	随分[ずいぶん]	ずいぶん	
\\	随分早く終わったね。	随分[ずいぶん] 早[はや]く 終[お]わったね。	ずいぶん はやく おわった ね	
\\	早[はや]く 終[お]わったね。			
\\	頑張る	頑張[がんば]る	がんばる	
\\	明日の試験、頑張ってね。	明日[あした]の 試験[しけん]、 頑張[がんば]ってね。	あした の しけん がんばって ね	
\\	明日[あした]の 試験[しけん]、
\\	ね。			
\\	頃	頃[ころ]	ころ	
\\	子供の頃、よくその公園で遊んだ。	子供[こども]の 頃[ころ]、よくその 公園[こうえん]で 遊[あそ]んだ。	こども の ころ よく その こうえん で あそんだ	
\\	子供[こども]の
\\	、よくその 公園[こうえん]で 遊[あそ]んだ。			
\\	にんじん	にんじん	にんじん	
\\	私はにんじんが嫌いだ。	私[わたし]はにんじんが 嫌[きら]いだ。	わたし は にんじん が きらい だ	
\\	私[わたし]は
\\	が 嫌[きら]いだ。			
\\	この頃	この 頃[ごろ]	このごろ	
\\	この頃、よく肩が凝る。	この 頃[ごろ]、よく 肩[かた]が 凝[こ]る。	このごろ よく かた が こる	
\\	、よく 肩[かた]が 凝[こ]る。			
\\	消防署	消防署[しょうぼうしょ]	しょうぼうしょ	
\\	この町には消防署が1つしかありません。	この 町[まち]には 消防署[しょうぼうしょ]が 1[ひと]つしかありません。	この まち に は しょうぼうしょ が ひとつ しか ありません	
\\	この 町[まち]には
\\	が 1[ひと]つしかありません。			
\\	尋ねる	尋[たず]ねる	たずねる	
\\	彼女は道を尋ねた。	彼女[かのじょ]は 道[みち]を 尋[たず]ねた。	かのじょ は みち を たずねた	
\\	彼女[かのじょ]は 道[みち]を
\\	缶詰	缶詰[かんづめ]	かんづめ	
\\	缶詰のフルーツはとても甘い。	缶詰[かんづめ]のフルーツはとても 甘[あま]い。	かんづめ の ふるーつ は とても あまい	
\\	のフルーツはとても 甘[あま]い。			
\\	ぬるい	ぬるい	ぬるい	
\\	風呂がぬるかった。	風呂[ふろ]がぬるかった。	ふろ が ぬるかった	
\\	風呂[ふろ]が
\\	缶	缶[かん]	かん	
\\	コーヒーはその缶に入っています。	コーヒーはその 缶[かん]に 入[はい]っています。	こーひー は その かん に はいって います	
\\	コーヒーはその
\\	に 入[はい]っています。			
\\	腐る	腐[くさ]る	くさる	
\\	リンゴが腐っている。	リンゴが 腐[くさ]っている。	りんご が くさって いる	
\\	リンゴが
\\	豆腐	豆腐[とうふ]	とうふ	
\\	私は毎日豆腐を食べます。	私[わたし]は 毎日[まいにち] 豆腐[とうふ]を 食[た]べます。	わたし は まいにち とうふ を たべます	
\\	私[わたし]は 毎日[まいにち]
\\	を 食[た]べます。			
\\	床	床[ゆか]	ゆか	
\\	床がぬれている。	床[ゆか]がぬれている。	ゆか が ぬれて いる	
\\	がぬれている。			
\\	パーセント	パーセント	パーセント	
\\	この村の80パーセントは老人です。	この 村[むら]の 80[はちじゅっ]パーセントは 老人[ろうじん]です。	この むら の はちじゅっぱーせんと は ろうじん です	
\\	この 村[むら]の 80[はちじゅっ]
\\	は 老人[ろうじん]です。			
\\	床屋	床屋[とこや]	とこや	
\\	昨日床屋で髪を切った。	昨日[きのう] 床屋[とこや]で 髪[かみ]を 切[き]った。	きのう とこや で かみ を きった	
\\	昨日[きのう]
\\	で 髪[かみ]を 切[き]った。			
\\	畳	畳[たたみ]	たたみ	
\\	そのホテルには畳の部屋がある。	そのホテルには 畳[たたみ]の 部屋[へや]がある。	その ほてる に は たたみ の へや が ある	
\\	そのホテルには
\\	の 部屋[へや]がある。			
\\	畳む	畳[たた]む	たたむ	
\\	布団を畳んでください。	布団[ふとん]を 畳[たた]んでください。	ふとん を たたんで ください	
\\	布団[ふとん]を
\\	ください。			
\\	干す	干[ほ]す	ほす	
\\	母は洗濯物を干しています。	母[はは]は 洗濯物[せんたくもの]を 干[ほ]しています。	はは は せんたくもの を ほして います	
\\	母[はは]は 洗濯物[せんたくもの]を
\\	います。			
\\	ひざ	ひざ	ひざ	
\\	スキーでひざを怪我しました。	スキーでひざを 怪我[けが]しました。	すきー で ひざ を けがしました	
\\	スキーで
\\	を 怪我[けが]しました。			
\\	帽子	帽子[ぼうし]	ぼうし	
\\	暑いので帽子を被りましょう。	暑[あつ]いので 帽子[ぼうし]を 被[かぶ]りましょう。	あつい の で ぼうし を かぶりましょう	
\\	暑[あつ]いので
\\	を 被[かぶ]りましょう。			
\\	是非	是非[ぜひ]	ぜひ	
\\	是非、うちに来てください。	是非[ぜひ]、うちに 来[き]てください。	ぜひ うち に きて ください	
\\	、うちに 来[き]てください。			
\\	敬語	敬語[けいご]	けいご	
\\	お客様には敬語を使いなさい。	お 客様[きゃくさま]には 敬語[けいご]を 使[つか]いなさい。	おきゃくさま に は けいご を つかいなさい	
\\	お 客様[きゃくさま]には
\\	を 使[つか]いなさい。			
\\	尊敬	尊敬[そんけい]	そんけい	
\\	祖父は家族みんなに尊敬されています。	祖父[そふ]は 家族[かぞく]みんなに 尊敬[そんけい]されています。	そふ は かぞく みんな に そんけい されて います	
\\	祖父[そふ]は 家族[かぞく]みんなに
\\	されています。			
\\	ひじ	ひじ	ひじ	
\\	彼はひじに怪我をした。	彼[かれ]はひじに 怪我[けが]をした。	かれ は ひじ に けが を した	
\\	彼[かれ]は
\\	に 怪我[けが]をした。			
\\	敷く	敷[し]く	しく	
\\	生まれて初めて布団を敷いた。	生[う]まれて 初[はじ]めて 布団[ふとん]を 敷[し]いた。	うまれて はじめて ふとん を しいた	
\\	生[う]まれて 初[はじ]めて 布団[ふとん]を
\\	雷	雷[かみなり]	かみなり	
\\	雷が鳴っています。	雷[かみなり]が 鳴[な]っています。	かみなり が なって います	
\\	が 鳴[な]っています。			
\\	零	零[れい]	れい	
\\	今ちょうど零時です。	今[いま]ちょうど 零[れい] 時[じ]です。	いま ちょうど れいじ です	
\\	今[いま]ちょうど
\\	時[じ]です。			
\\	仕舞う	仕舞[しま]う	しまう	
\\	彼女は大切な書類を机に仕舞った。	彼女[かのじょ]は 大切[たいせつ]な 書類[しょるい]を 机[つくえ]に 仕舞[しま]った。	かのじょ は たいせつ な しょるい を つくえ に しまった。	
\\	彼女[かのじょ]は 大切[たいせつ]な 書類[しょるい]を 机[つくえ]に
\\	ひも	ひも	ひも	
\\	靴のひもが切れた。	靴[くつ]のひもが 切[き]れた。	くつ の ひも が きれた	
\\	靴[くつ]の
\\	が 切[き]れた。			
\\	踊る	踊[おど]る	おどる	
\\	彼女はクラブで踊るのが好きです。	彼女[かのじょ]はクラブで 踊[おど]るのが 好[す]きです。	かのじょ は くらぶ で おどる の が すき です	
\\	彼女[かのじょ]はクラブで
\\	のが 好[す]きです。			
\\	踊り	踊[おど]り	おどり	
\\	この踊りは易しいですよ。	この 踊[おど]りは 易[やさ]しいですよ。	この おどり は やさしい です よ	
\\	この
\\	は 易[やさ]しいですよ。			
\\	踏む	踏[ふ]む	ふむ	
\\	運転手がブレーキを踏んだ。	運転手[うんてんしゅ]がブレーキを 踏[ふ]んだ。	うんてんしゅ が ぶれーき を ふんだ	
\\	運転手[うんてんしゅ]がブレーキを
\\	踏切	踏切[ふみきり]	ふみきり	
\\	その踏切は長い。	その 踏切[ふみきり]は 長[なが]い。	その ふみきり は ながい	
\\	その
\\	は 長[なが]い。			
\\	まく	まく	まく	
\\	日本では、二月に豆をまく行事がある。	日本[にほん]では、 二月[にがつ]に 豆[まめ]をまく 行事[ぎょうじ]がある。	にほん で は にがつ に まめ を まく ぎょうじ が ある	
\\	日本[にほん]では、 二月[にがつ]に 豆[まめ]を
\\	行事[ぎょうじ]がある。			
\\	蹴る	蹴[け]る	ける	
\\	ゴールキーパーがボールを蹴った。	ゴールキーパーがボールを 蹴[け]った。	ごーるきーぱー が ぼーる を けった。	
\\	ゴールキーパーがボールを
\\	食堂	食堂[しょくどう]	しょくどう	
\\	大学の食堂は安い。	大学[だいがく]の 食堂[しょくどう]は 安[やす]い。	だいがく の しょくどう は やすい	
\\	大学[だいがく]の
\\	は 安[やす]い。			
\\	猫	猫[ねこ]	ねこ	
\\	私は猫が大好きです。	私[わたし]は 猫[ねこ]が 大好[だいす]きです。	わたし は ねこ が だいすき です	
\\	私[わたし]は
\\	が 大好[だいす]きです。			
\\	文章	文章[ぶんしょう]	ぶんしょう	
\\	彼は文章がとてもうまい。	彼[かれ]は 文章[ぶんしょう]がとてもうまい。	かれ は ぶんしょう が とても うまい	
\\	彼[かれ]は
\\	がとてもうまい。			
\\	やかん	やかん	やかん	
\\	やかんでお湯を沸かしました。	やかんでお 湯[ゆ]を 沸[わ]かしました。	やかん で おゆ を わかしました	
\\	でお 湯[ゆ]を 沸[わ]かしました。			
\\	丈夫	丈夫[じょうぶ]	じょうぶ	
\\	祖母は身体が丈夫だ。	祖母[そぼ]は 身体[からだ]が 丈夫[じょうぶ]だ。	そぼ は からだ が じょうぶ だ	
\\	祖母[そぼ]は 身体[からだ]が
\\	だ。			
\\	又	又[また]	また	
\\	明日、また来ます。	明日[あした]、また 来[き]ます。	あした また きます	
\\	明日[あした]、
\\	来[き]ます。			
\\	お祖父さん	お 祖父[じい]さん	おじいさん	
\\	私のお祖父さんは毎日散歩します。	私[わたし]のお 祖父[じい]さんは 毎日散歩[まいにち さんぽ]します。	わたし の おじいさん は まいにち さんぽ します 。	
\\	私[わたし]の
\\	は 毎日散歩[まいにち さんぽ]します。			
\\	お祖母さん	お 祖母[ばあ]さん	おばあさん	
\\	彼女はお祖母さんと住んでいる。	彼女[かのじょ]はお 祖母[ばあ]さんと 住[す]んでいる。	かのじょ は おばあさん と すん でいる 。	
\\	彼女[かのじょ]は
\\	と 住[す]んでいる。			
\\	ようこそ	ようこそ	ようこそ	
\\	日本へようこそ。	日本[にっぽん]へようこそ。	にっぽん へ ようこそ	
\\	日本[にっぽん]へ
\\	祖父	祖父[そふ]	そふ	
\\	祖父は元気です。	祖父[そふ]は 元気[げんき]です。	そふ は げんき です	
\\	は 元気[げんき]です。			
\\	祖母	祖母[そぼ]	そぼ	
\\	祖母は京都で生まれました。	祖母[そぼ]は 京都[きょうと]で 生[う]まれました。	そぼ は きょうと で うまれました	
\\	は 京都[きょうと]で 生[う]まれました。			
\\	邪魔	邪魔[じゃま]	じゃま	
\\	邪魔です、どいてください。	邪魔[じゃま]です、どいてください。	じゃま です どいて ください	
\\	です、どいてください。			
\\	風邪薬	風邪薬[かぜぐすり]	かぜぐすり	
\\	この風邪薬を飲みなさい。	この 風邪薬[かぜぐすり]を 飲[の]みなさい。	この かぜぐすり を のみなさい	
\\	この
\\	を 飲[の]みなさい。			
\\	ウェーター	ウェーター	ウェーター	
\\	ウェーターが水を運んできました。	ウェーターが 水[みず]を 運[はこ]んできました。	うぇーたー が みず を はこんで きました	
\\	が 水[みず]を 運[はこ]んできました。			
\\	風呂屋	風呂屋[ふろや]	ふろや	
\\	昨日、友達とお風呂屋さんに行った。	昨日[きのう]、 友達[ともだち]とお 風呂屋[ふろや]さんに 行[い]った。	きのう ともだち と おふろやさん に いった	
\\	昨日[きのう]、 友達[ともだち]とお
\\	さんに 行[い]った。			
\\	風呂	風呂[ふろ]	ふろ	
\\	父は今お風呂に入っています。	父[ちち]は 今[いま]お 風呂[ふろ]に 入[はい]っています。	ちち は いま お ふろ に はいって います	
\\	父[ちち]は 今[いま]お
\\	に 入[はい]っています。			
\\	昭和	昭和[しょうわ]	しょうわ	
\\	私の両親は昭和生まれです。	私[わたし]の 両親[りょうしん]は 昭和[しょうわ] 生[う]まれです。	わたし の りょうしん は しょうわうまれ です	
\\	私[わたし]の 両親[りょうしん]は
\\	生[う]まれです。			
\\	紫	紫[むらさき]	むらさき	
\\	彼女は紫のドレスを着ていた。	彼女[かのじょ]は 紫[むらさき]のドレスを 着[き]ていた。	かのじょ は むらさき の どれす を きて いた	
\\	彼女[かのじょ]は
\\	のドレスを 着[き]ていた。			
\\	そんなに	そんなに	そんなに	
\\	一度にそんなにたくさんはできない。	一度[いちど]にそんなにたくさんはできない。	いちど に そんなに たくさん は できない	
\\	一度[いちど]に
\\	たくさんはできない。			
\\	紅茶	紅茶[こうちゃ]	こうちゃ	
\\	温かい紅茶が飲みたい。	温[あたた]かい 紅茶[こうちゃ]が 飲[の]みたい。	あたたかい こうちゃ が のみたい	
\\	温[あたた]かい
\\	が 飲[の]みたい。			
\\	梅雨	梅雨[つゆ]	つゆ	
\\	梅雨は6月頃です。	梅雨[つゆ]は 6月頃[ろくがつごろ]です。	つゆ は ろくがつごろ です	
\\	は 6月頃[ろくがつごろ]です。			
\\	桃	桃[もも]	もも	
\\	私の一番好きな果物は桃です。	私[わたし]の 一番好[いちばん す]きな 果物[くだもの]は 桃[もも]です。	わたし の いちばん すき な くだもの は もも です	
\\	私[わたし]の 一番好[いちばん す]きな 果物[くだもの]は
\\	です。			
\\	遭う	遭[あ]う	あう	
\\	彼は交通事故に遭った。	彼[かれ]は 交通事故[こうつう じこ]に 遭[あ]った。	かれ は こうつう じこ に あった	
\\	彼[かれ]は 交通事故[こうつう じこ]に
\\	枕	枕[まくら]	まくら	
\\	私は低い枕が好きです。	私[わたし]は 低[ひく]い 枕[まくら]が 好[す]きです。	わたし は ひくい まくら が すき です	
\\	私[わたし]は 低[ひく]い
\\	が 好[す]きです。			
\\	嘘	嘘[うそ]	うそ	
\\	嘘をついてはいけません。	嘘[うそ]をついてはいけません。	うそ を ついて は いけません	
\\	をついてはいけません。			
\\	遠慮	遠慮[えんりょ]	えんりょ	
\\	私は遠慮します。	私[わたし]は 遠慮[えんりょ]します。	わたし は えんりょ します	
\\	私[わたし]は
\\	します。			
\\	叱る	叱[しか]る	しかる	
\\	父親が子供を叱っている。	父親[ちちおや]が 子供[こども]を 叱[しか]っている。	ちちおや が こども を しかって いる	
\\	父親[ちちおや]が 子供[こども]を
\\	傘	傘[かさ]	かさ	
\\	電車に傘を忘れた。	電車[でんしゃ]に 傘[かさ]を 忘[わす]れた。	でんしゃ に かさ を わすれた	
\\	電車[でんしゃ]に
\\	を 忘[わす]れた。			
\\	お嬢さん	お 嬢[じょう]さん	おじょうさん	
\\	お嬢さんはおいくつですか。	お 嬢[じょう]さんはおいくつですか。	おじょうさん は おいくつ です か	
\\	はおいくつですか。			
\\	年賀状	年賀状[ねんがじょう]	ねんがじょう	
\\	昨日、年賀状を出しました。	昨日[きのう]、 年賀状[ねんがじょう]を 出[だ]しました。	きのう ねんがじょう を だしました	
\\	昨日[きのう]、
\\	を 出[だ]しました。			
\\	賑やか	賑[にぎ]やか	にぎやか	
\\	浅草は賑やかな街です。	浅草[あさくさ]は 賑[にぎ]やかな 街[まち]です。	あさくさ は にぎやか な まち です	
\\	浅草[あさくさ]は
\\	な 街[まち]です。			
\\	蚊	蚊[か]	か	
\\	蚊に足を刺された。	蚊[か]に 足[あし]を 刺[さ]された。	か に あし を さされた	
\\	に 足[あし]を 刺[さ]された。			
\\	拭く	拭[ふ]く	ふく	
\\	タオルで体を拭きました。	タオルで 体[からだ]を 拭[ふ]きました。	たおる で からだ を ふきました	
\\	タオルで 体[からだ]を
\\	挨拶	挨拶[あいさつ]	あいさつ	
\\	彼女は笑顔で挨拶した。	彼女[かのじょ]は 笑顔[えがお]で 挨拶[あいさつ]した。	かのじょ は えがお で あいさつ した	
\\	彼女[かのじょ]は 笑顔[えがお]で
\\	した。			
\\	伴う	伴[ともな]う	ともなう	
\\	その仕事は危険を伴う。	その 仕事[しごと]は 危険[きけん]を 伴[ともな]う。	その しごと は きけん を ともなう	
\\	その 仕事[しごと]は 危険[きけん]を
\\	巻く	巻[ま]く	まく	
\\	彼は頭にタオルを巻いていた。	彼[かれ]は 頭[あたま]にタオルを 巻[ま]いていた。	かれ は あたま に たおる を まいて いた	
\\	彼[かれ]は 頭[あたま]にタオルを
\\	靴	靴[くつ]	くつ	
\\	靴が汚れた。	靴[くつ]が 汚[よご]れた。	くつ が よごれた	
\\	が 汚[よご]れた。			
\\	靴下	靴下[くつした]	くつした	
\\	この靴下は3足で1000円です。	この 靴下[くつした]は 3足[さんそく]で 1000円[せんえん]です。	この くつした は さんそく で せんえん です	
\\	この
\\	は 3足[さんそく]で 1000円[せんえん]です。			
\\	磨く	磨[みが]く	みがく	
\\	靴を磨いてください。	靴[くつ]を 磨[みが]いてください。	くつ を みがいて ください	
\\	靴[くつ]を
\\	ください。			
\\	歯磨き	歯磨[はみが]き	はみがき	
\\	歯磨きはしましたか。	歯磨[はみが]きはしましたか。	はみがき は しました か	
\\	はしましたか。			
\\	廊下	廊下[ろうか]	ろうか	
\\	廊下は走らないでください。	廊下[ろうか]は 走[はし]らないでください。	ろうか は はしらない で ください	
\\	は 走[はし]らないでください。			
\\	瓶	瓶[びん]	びん	
\\	瓶ビールを注文した。	瓶[びん]ビールを 注文[ちゅうもん]した。	びんびーる を ちゅうもん した	
\\	ビールを 注文[ちゅうもん]した。			
\\	褒める	褒[ほ]める	ほめる	
\\	頑張ったので褒められました。	頑張[がんば]ったので 褒[ほ]められました。	がんばった の で ほめられました	
\\	頑張[がんば]ったので
\\	元旦	元旦[がんたん]	がんたん	
\\	元旦に彼から年賀状が来た。	元旦[がんたん]に 彼[かれ]から 年賀状[ねんがじょう]が 来[き]た。	がんたん に かれ から ねんがじょう が きた 。	
\\	に 彼[かれ]から 年賀状[ねんがじょう]が 来[き]た。			
\\	袖	袖[そで]	そで	
\\	このシャツは袖が短い。	このシャツは 袖[そで]が 短[みじか]い。	この しゃつ は そで が みじかい	
\\	このシャツは
\\	が 短[みじか]い。			
\\	長袖	長袖[ながそで]	ながそで	
\\	今日は寒いので長袖を着ました。	今日[きょう]は 寒[さむ]いので 長袖[ながそで]を 着[き]ました。	きょう は さむい ので ながそで を きました 。	
\\	今日[きょう]は 寒[さむ]いので
\\	を 着[き]ました。			
\\	半袖	半袖[はんそで]	はんそで	
\\	今日は、半袖のシャツを着よう。	今日[きょう]は、 半袖[はんそで]のシャツを 着[き]よう。	きょう は 、 はんそで の しゃつ を きよう 。	
\\	今日[きょう]は、
\\	のシャツを 着[き]よう。			
\\	馬鹿	馬鹿[ばか]	ばか	
\\	私は馬鹿だった。	私[わたし]は 馬鹿[ばか]だった。	わたし は ばか だった	
\\	私[わたし]は
\\	だった。			
\\	凄い	凄[すご]い	すごい	
\\	凄い雨になった。	凄[すご]い 雨[あめ]になった。	すごい あめ に なった	
\\	雨[あめ]になった。			
\\	剃る	剃[そ]る	そる	
\\	父は毎日髭を剃ります。	父[ちち]は 毎日髭[まいにち ひげ]を 剃[そ]ります。	ちち は まいにち ひげ を そります	
\\	父[ちち]は 毎日髭[まいにち ひげ]を
\\	喧嘩	喧嘩[けんか]	けんか	
\\	喧嘩はやめて。	喧嘩[けんか]はやめて。	けんか は やめて	
\\	はやめて。			
\\	叩く	叩[たた]く	たたく	
\\	彼は子供のおしりを叩いた。	彼[かれ]は 子供[こども]のおしりを 叩[たた]いた。	かれ は こども の おしり を たたいた	
\\	彼[かれ]は 子供[こども]のおしりを
\\	噛む	噛[か]む	かむ	
\\	もっとよく噛みなさい。	もっとよく 噛[か]みなさい。	もっと よく かみなさい	
\\	もっとよく
\\	味噌汁	味噌汁[みそしる]	みそしる	
\\	私は毎日味噌汁を飲みます。	私[わたし]は 毎日[まいにち] 味噌汁[みそしる]を 飲[の]みます。	わたし は まいにち みそしる を のみます 。	
\\	私[わたし]は 毎日[まいにち]
\\	を 飲[の]みます。			
\\	姪	姪[めい]	めい	
\\	私の姪は3才です。	私[わたし]の 姪[めい]は 3才[さんさい]です。	わたし の めい は さんさい です	
\\	私[わたし]の
\\	は 3才[さんさい]です。			
\\	苺	苺[いちご]	いちご	
\\	この苺はとても甘い。	この 苺[いちご]はとても 甘[あま]い。	この いちご は とても あまい	
\\	この
\\	はとても 甘[あま]い。			
\\	茄子	茄子[なす]	なす	
\\	夕飯に茄子の天ぷらを食べました。	夕飯[ゆうはん]に 茄子[なす]の 天[てん]ぷらを 食[た]べました。	ゆうはん に なす の てんぷら を たべました 。	
\\	夕飯[ゆうはん]に
\\	の 天[てん]ぷらを 食[た]べました。			
\\	逢う	逢[あ]う	あう	
\\	ついに素晴らしい女性に逢えた。	ついに 素晴[すば]らしい 女性[じょせい]に 逢[あ]えた。	ついに すばらしい じょせい に あえた 。	
\\	ついに 素晴[すば]らしい 女性[じょせい]に
\\	椅子	椅子[いす]	いす	
\\	そのお年寄りは椅子に座った。	そのお 年寄[としよ]りは 椅子[いす]に 座[すわ]った。	その お としより は いす に すわった 。	
\\	そのお 年寄[としよ]りは
\\	に 座[すわ]った。			
\\	痩せる	痩[や]せる	やせる	
\\	私は少し痩せました。	私[わたし]は 少[すこ]し 痩[や]せました。	わたし は すこし やせました	
\\	私[わたし]は 少[すこ]し
\\	箸	箸[はし]	はし	
\\	箸を上手に使えるよ。	箸[はし]を 上手[じょうず]に 使[つか]えるよ。	はし を じょうず に つかえる よ	
\\	を 上手[じょうず]に 使[つか]えるよ。			
\\	糊	糊[のり]	のり	
\\	糊で2枚の紙を貼り合わせた。	糊[のり]で 2枚[にまい]の 紙[かみ]を 貼[は]り 合[あ]わせた。	のり で にまい の かみ を はり あわせた	
\\	で 2枚[にまい]の 紙[かみ]を 貼[は]り 合[あ]わせた。			
\\	醤油	醤油[しょうゆ]	しょうゆ	
\\	もう少し醤油を足してください。	もう 少[すこ]し 醤油[しょうゆ]を 足[た]してください。	もうすこし しょうゆ を たして ください 。	
\\	もう 少[すこ]し
\\	を 足[た]してください。			
\\	鋏	鋏[はさみ]	はさみ	
\\	この鋏はよく切れる。	この 鋏[はさみ]はよく 切[き]れる。	この はさみ は よく きれる	
\\	この
\\	はよく 切[き]れる。			
\\	鞄	鞄[かばん]	かばん	
\\	その黒い鞄は僕のです。	その 黒[くろ]い 鞄[かばん]は 僕[ぼく]のです。	その くろい かばん は ぼく の です	
\\	その 黒[くろ]い
\\	は 僕[ぼく]のです。			
\\	顎	顎[あご]	あご	
\\	顎が痛い。	顎[あご]が 痛[いた]い。	あご が いたい	
\\	が 痛[いた]い。			
\\	飴	飴[あめ]	あめ	
\\	缶に飴が入っています。	缶[かん]に 飴[あめ]が 入[はい]っています。	かん に あめ が はいって います	
\\	缶[かん]に
\\	が 入[はい]っています。			
\\	石鹸	石鹸[せっけん]	せっけん	
\\	石鹸で手を洗ってください。	石鹸[せっけん]で 手[て]を 洗[あら]ってください。	せっけん で て を あらって ください	
\\	で 手[て]を 洗[あら]ってください。			
\\	一時	一時[いちじ]	いちじ	
\\	お店は一時休業になったんだ。	お 店[みせ]は 一時[いちじ] 休業[きゅうぎょう]になったんだ。	おみせ は いちじ きゅうぎょう に なった ん だ	
\\	お 店[みせ]は
\\	休業[きゅうぎょう]になったんだ。			
\\	月日	月日[がっぴ]	がっぴ	
\\	ここに生年月日を記入してください。	ここに 生年[せいねん] 月日[がっぴ]を 記入[きにゅう]してください。	ここ に せいねんがっぴ を きにゅう して ください	
\\	ここに 生年[せいねん]
\\	を 記入[きにゅう]してください。			
\\	金もうけ	金[かね]もうけ	かねもうけ	
\\	金もうけが彼の趣味だ。	金[かね]もうけが 彼[かれ]の 趣味[しゅみ]だ。	かねもうけ が かれ の しゅみ だ	
\\	が 彼[かれ]の 趣味[しゅみ]だ。			
\\	先に	先[さき]に	さきに	
\\	では、私たちは先に出発します。	では、 私[わたし]たちは 先[さき]に 出発[しゅっぱつ]します。	では わたしたち は さきに しゅっぱつ します	
\\	では、 私[わたし]たちは
\\	出発[しゅっぱつ]します。			
\\	先日	先日[せんじつ]	せんじつ	
\\	先日の件はどうなりましたか。	先日[せんじつ]の 件[けん]はどうなりましたか。	せんじつ の けん は どう なりました か	
\\	の 件[けん]はどうなりましたか。			
\\	今日	今日[こんにち]	こんにち	
\\	今日の日本の若者は欧米化している。	今日[こんにち]の 日本[にほん]の 若者[わかもの]は 欧米化[おうべいか]している。	こんにち の にほん の わかもの は おうべいか して いる	
\\	の 日本[にほん]の 若者[わかもの]は 欧米化[おうべいか]している。			
\\	今ごろ	今[いま]ごろ	いまごろ	
\\	今ごろそんなこと言わないで。	今[いま]ごろそんなこと 言[い]わないで。	いまごろ そんな こと いわない で	
\\	そんなこと 言[い]わないで。			
\\	ケース	ケース	ケース	
\\	このケースには
\\	が50枚入ります。	このケースには
\\	が50 枚入[まい はい]ります。	この けーす に は 
\\	が 
\\	まい はいります	
\\	この
\\	には
\\	が50 枚入[まい はい]ります。			
\\	今にも	今[いま]にも	いまにも	
\\	今にも雨が降りそうですね。	今[いま]にも 雨[あめ]が 降[ふ]りそうですね。	いまにも あめ が ふりそう です ね	
\\	雨[あめ]が 降[ふ]りそうですね。			
\\	今に	今[いま]に	いまに	
\\	あなたも今に分かるでしょう。	あなたも 今[いま]に 分[わ]かるでしょう。	あなた も いまに わかる でしょう	
\\	あなたも
\\	分[わ]かるでしょう。			
\\	行き来	行[い]き 来[き]	いきき	
\\	最近彼のところに行き来してないの。	最近彼[さいきん かれ]のところに 行[い]き 来[き]してないの。	さいきん かれ の ところ に いきき して ない の	
\\	最近彼[さいきん かれ]のところに
\\	してないの。			
\\	行	行[ぎょう]	ぎょう	
\\	5行以内で答えを書いてください。	5[ご] 行[ぎょう] 以内[いない]で 答[こた]えを 書[か]いてください。	ごぎょう いない で こたえ を かいて ください	
\\	5[ご]
\\	以内[いない]で 答[こた]えを 書[か]いてください。			
\\	帰す	帰[かえ]す	かえす	
\\	学校は生徒たちを午前中に帰したね。	学校[がっこう]は 生徒[せいと]たちを 午前中[ごぜんちゅう]に 帰[かえ]したね。	がっこう は せいとたち を ごぜんちゅう に かえした ね	
\\	学校[がっこう]は 生徒[せいと]たちを 午前中[ごぜんちゅう]に
\\	ね。			
\\	大いに	大[おお]いに	おおいに	
\\	大いに学び、大いに遊びなさい。	大[おお]いに 学[まな]び、 大[おお]いに 遊[あそ]びなさい。	おおいに まなび おおいに あそびなさい	
\\	学[まな]び、 大[おお]いに 遊[あそ]びなさい。			
\\	大げさ	大[おお]げさ	おおげさ	
\\	彼の話は大げさだ。	彼[かれ]の 話[はなし]は 大[おお]げさだ。	かれ の はなし は おおげさ だ	
\\	彼[かれ]の 話[はなし]は
\\	だ。			
\\	イメージ	イメージ	イメージ	
\\	彼には清潔なイメージがあるね。	彼[かれ]には 清潔[せいけつ]なイメージがあるね。	かれ に は せいけつ な いめーじ が ある ね	
\\	彼[かれ]には 清潔[せいけつ]な
\\	があるね。			
\\	大水	大水[おおみず]	おおみず	
\\	その年、この地域では大水がありました。	その 年[とし]、この 地域[ちいき]では 大水[おおみず]がありました。	その とし この ちいき で は おおみず が ありました	
\\	その 年[とし]、この 地域[ちいき]では
\\	がありました。			
\\	水中	水中[すいちゅう]	すいちゅう	
\\	このカメラなら水中の写真が撮れますね。	このカメラなら 水中[すいちゅう]の 写真[しゃしん]が 撮[と]れますね。	この かめら なら すいちゅう の しゃしん が とれます ね	
\\	このカメラなら
\\	の 写真[しゃしん]が 撮[と]れますね。			
\\	小	小[しょう]	しょう	
\\	この箱の小をください。	この 箱[はこ]の 小[しょう]をください。	この はこ の しょう を ください	
\\	この 箱[はこ]の
\\	をください。			
\\	少なくとも	少[すく]なくとも	すくなくとも	
\\	この仕事には少なくとも2週間必要でしょう。	この 仕事[しごと]には 少[すく]なくとも 2週間必要[にしゅうかん ひつよう]でしょう。	この しごと に は すくなくとも にしゅうかん ひつよう でしょう	
\\	この 仕事[しごと]には
\\	2週間必要[にしゅうかん ひつよう]でしょう。			
\\	少々	少々[しょうしょう]	しょうしょう	
\\	少々のことは我慢します。	少々[しょうしょう]のことは 我慢[がまん]します。	しょうしょう の こと は がまん します	
\\	のことは 我慢[がまん]します。			
\\	上がる	上[あ]がる	あがる	
\\	今日は仕事が早く上がったんだ。	今日[きょう]は 仕事[しごと]が 早[はや]く 上[あ]がったんだ。	きょう は しごと が はやく あがった ん だ	
\\	今日[きょう]は 仕事[しごと]が 早[はや]く
\\	んだ。			
\\	上がる	上[あ]がる	あがる	
\\	冷めないうちにどうぞお上がり下さい。	冷[さ]めないうちにどうぞお 上[あ]がり 下[くだ]さい。	さめない うち に どうぞ おあがり ください	
\\	冷[さ]めないうちにどうぞ
\\	下[くだ]さい。			
\\	せい	せい	せい	
\\	自分の失敗を他人のせいにするな。	自分[じぶん]の 失敗[しっぱい]を 他人[たにん]のせいにするな。	じぶん の しっぱい を たにん の せい に する な	
\\	自分[じぶん]の 失敗[しっぱい]を 他人[たにん]の
\\	にするな。			
\\	上	上[じょう]	じょう	
\\	握り寿司の上を注文したよ。	握[にぎ]り 寿司[ずし]の 上[じょう]を 注文[ちゅうもん]したよ。	にぎりずし の じょう を ちゅうもん した よ	
\\	握[にぎ]り 寿司[ずし]の
\\	を 注文[ちゅうもん]したよ。			
\\	上	上[かみ]	かみ	
\\	お上は何を考えてるんだろうね。	お 上[かみ]は 何[なに]を 考[かんが]えてるんだろうね。	おかみ は なに を かんがえて る ん だろう ね	
\\	お
\\	は 何[なに]を 考[かんが]えてるんだろうね。			
\\	下さる	下[くだ]さる	くださる	
\\	先生が手紙を下さいました。	先生[せんせい]が 手紙[てがみ]を 下[くだ]さいました。	せんせい が てがみ を くださいました	
\\	先生[せんせい]が 手紙[てがみ]を
\\	下水	下水[げすい]	げすい	
\\	その道は今、下水の工事をしているよ。	その 道[みち]は 今[いま]、 下水[げすい]の 工事[こうじ]をしているよ。	その みち は いま げすい の こうじ を して いる よ	
\\	その 道[みち]は 今[いま]、
\\	の 工事[こうじ]をしているよ。			
\\	上下	上下[じょうげ]	じょうげ	
\\	この服は上下セットで買いました。	この 服[ふく]は 上下[じょうげ]セットで 買[か]いました。	この ふく は じょうげ せっと で かいました	
\\	この 服[ふく]は
\\	セットで 買[か]いました。			
\\	上下	上下[うえした]	うえした	
\\	彼は上下おそろいの服を着ていますね。	彼[かれ]は 上下[うえした]おそろいの 服[ふく]を 着[き]ていますね。	かれ は うえした おそろい の ふく を きています ね 。	
\\	彼[かれ]は
\\	おそろいの 服[ふく]を 着[き]ていますね。			
\\	下	下[げ]	げ	
\\	彼女の成績は上の下です。	彼女[かのじょ]の 成績[せいせき]は 上[じょう]の 下[げ]です。	かのじょ の せいせき は じょう の げ です	
\\	彼女[かのじょ]の 成績[せいせき]は 上[じょう]の
\\	です。			
\\	左右	左右[さゆう]	さゆう	
\\	左右を見てから横断歩道を渡りなさい。	左右[さゆう]を 見[み]てから 横断歩道[おうだん ほどう]を 渡[わた]りなさい。	さゆう を みてから おうだん ほどう を わたりなさい	
\\	を 見[み]てから 横断歩道[おうだん ほどう]を 渡[わた]りなさい。			
\\	いつも	いつも	いつも	
\\	彼女の様子がいつもと違う。	彼女[かのじょ]の 様子[ようす]がいつもと 違[ちが]う。	かのじょ の ようす が いつも と ちがう	
\\	彼女[かのじょ]の 様子[ようす]が
\\	と 違[ちが]う。			
\\	四方	四方[しほう]	しほう	
\\	火が四方に広がったんだよ。	火[ひ]が 四方[しほう]に 広[ひろ]がったんだよ。	ひ が しほう に ひろがった ん だ よ	
\\	火[ひ]が
\\	に 広[ひろ]がったんだよ。			
\\	方々	方々[かたがた]	かたがた	
\\	大勢の方々にご出席いただきました。	大勢[おおぜい]の 方々[かたがた]にご 出席[しゅっせき]いただきました。	おおぜい の かたがた に ごしゅっせき いただきました	
\\	大勢[おおぜい]の
\\	にご 出席[しゅっせき]いただきました。			
\\	大人	大人[おとな]	おとな	
\\	あなたもだいぶ大人になったね。	あなたもだいぶ 大人[おとな]になったね。	あなた も だいぶ おとな に なった ね	
\\	あなたもだいぶ
\\	になったね。			
\\	外出	外出[がいしゅつ]	がいしゅつ	
\\	午後は外出の予定です。	午後[ごご]は 外出[がいしゅつ]の 予定[よてい]です。	ごご は がいしゅつ の よてい です	
\\	午後[ごご]は
\\	の 予定[よてい]です。			
\\	内	内[うち]	うち	
\\	心の内をお話し下さい。	心[こころ]の 内[うち]をお 話[はな]し 下[くだ]さい。	こころ の うち を お はなし ください	
\\	心[こころ]の
\\	をお 話[はな]し 下[くだ]さい。			
\\	週休	週休[しゅうきゅう]	しゅうきゅう	
\\	うちの会社は週休2日です。	うちの 会社[かいしゃ]は 週休[しゅうきゅう] 2日[ふつか]です。	うち の かいしゃ は しゅうきゅう ふつか です	
\\	うちの 会社[かいしゃ]は
\\	2日[ふつか]です。			
\\	一体	一体[いったい]	いったい	
\\	一体何事ですか。	一体[いったい] 何事[なにごと]ですか。	いったい なにごと です か	
\\	何事[なにごと]ですか。			
\\	きっかけ	きっかけ	きっかけ	
\\	大学に通うのが上京のきっかけでした。	大学[だいがく]に 通[かよ]うのが 上京[じょうきょう]のきっかけでした。	だいがく に かよう の が じょうきょう の きっかけ でした	
\\	大学[だいがく]に 通[かよ]うのが 上京[じょうきょう]の
\\	でした。			
\\	出力	出力[しゅつりょく]	しゅつりょく	
\\	このプリンターで写真を出力できます。	このプリンターで 写真[しゃしん]を 出力[しゅつりょく]できます。	この ぷりんたー で しゃしん を しゅつりょく できます	
\\	このプリンターで 写真[しゃしん]を
\\	できます。			
\\	火力	火力[かりょく]	かりょく	
\\	火力を弱めないと肉がこげるよ。	火力[かりょく]を 弱[よわ]めないと 肉[にく]がこげるよ。	かりょく を よわめない と にく が こげる よ	
\\	を 弱[よわ]めないと 肉[にく]がこげるよ。			
\\	水力	水力[すいりょく]	すいりょく	
\\	この島は発電を水力に頼っているんだ。	この 島[しま]は 発電[はつでん]を 水力[すいりょく]に 頼[たよ]っているんだ。	この しま は はつでん を すいりょく に たよって いる ん だ	
\\	この 島[しま]は 発電[はつでん]を
\\	に 頼[たよ]っているんだ。			
\\	口げんか	口[くち]げんか	くちげんか	
\\	きのう、弟と口げんかしました。	きのう、 弟[おとうと]と 口[くち]げんかしました。	きのう おとうと と くちげんか しました	
\\	きのう、 弟[おとうと]と
\\	しました。			
\\	大手	大手[おおて]	おおて	
\\	彼は大手のメーカーに勤めています。	彼[かれ]は 大手[おおて]のメーカーに 勤[つと]めています。	かれ は おおて の めーかー に つとめて います	
\\	彼[かれ]は
\\	のメーカーに 勤[つと]めています。			
\\	足下	足下[あしもと]	あしもと	
\\	暗いので足下に気を付けてください。	暗[くら]いので 足下[あしもと]に 気[き]を 付[つ]けてください。	くらい の で あしもと に き を つけて ください	
\\	暗[くら]いので
\\	に 気[き]を 付[つ]けてください。			
\\	火山	火山[かざん]	かざん	
\\	島で火山が噴火したよ。	島[しま]で 火山[かざん]が 噴火[ふんか]したよ。	しま で かざん が ふんか した よ	
\\	島[しま]で
\\	が 噴火[ふんか]したよ。			
\\	すっかり	すっかり	すっかり	
\\	買い物をすっかり忘れていた。	買[か]い 物[もの]をすっかり 忘[わす]れていた。	かいもの を すっかり わすれて いた	
\\	買[か]い 物[もの]を
\\	忘[わす]れていた。			
\\	小川	小川[おがわ]	おがわ	
\\	小川がさらさら流れています。	小川[おがわ]がさらさら 流[なが]れています。	おがわ が さらさら ながれて います	
\\	がさらさら 流[なが]れています。			
\\	空中	空中[くうちゅう]	くうちゅう	
\\	蝶々が空中を舞っています。	蝶々[ちょうちょう]が 空中[くうちゅう]を 舞[ま]っています。	ちょうちょう が くうちゅう を まって います	
\\	蝶々[ちょうちょう]が
\\	を 舞[ま]っています。			
\\	空ける	空[あ]ける	あける	
\\	彼女はお年寄りのために席を空けたんだ。	彼女[かのじょ]はお 年寄[としよ]りのために 席[せき]を 空[あ]けたんだ。	かのじょ は おとしより の ため に せき を あけた ん だ	
\\	彼女[かのじょ]はお 年寄[としよ]りのために 席[せき]を
\\	んだ。			
\\	空っぽ	空[から]っぽ	からっぽ	
\\	僕の財布は空っぽだよ。	僕[ぼく]の 財布[さいふ]は 空[から]っぽだよ。	ぼく の さいふ は からっぽ だ よ	
\\	僕[ぼく]の 財布[さいふ]は
\\	だよ。			
\\	空き	空[あ]き	あき	
\\	部屋の空きはありますか。	部屋[へや]の 空[あ]きはありますか。	へや の あき は あります か	
\\	部屋[へや]の
\\	はありますか。			
\\	空	空[から]	から	
\\	この瓶はもう空ね。	この 瓶[びん]はもう 空[から]ね。	この びん は もう から ね	
\\	この 瓶[びん]はもう
\\	ね。			
\\	海水	海水[かいすい]	かいすい	
\\	海水から塩を作ります。	海水[かいすい]から 塩[しお]を 作[つく]ります。	かいすい から しお を つくります	
\\	から 塩[しお]を 作[つく]ります。			
\\	海上	海上[かいじょう]	かいじょう	
\\	海上で衝突事故が発生した。	海上[かいじょう]で 衝突事故[しょうとつ じこ]が 発生[はっせい]した。	かいじょう で しょうとつ じこ が はっせい した	
\\	で 衝突事故[しょうとつ じこ]が 発生[はっせい]した。			
\\	シリーズ	シリーズ	シリーズ	
\\	このシリーズは50話もあるそうよ。	このシリーズは 50話[ごじゅうわ]もあるそうよ。	この しりーず は ごじゅうわ も ある そう よ	
\\	この
\\	は 50話[ごじゅうわ]もあるそうよ。			
\\	水田	水田[すいでん]	すいでん	
\\	窓の外に水田が広がっていたよ。	窓[まど]の 外[そと]に 水田[すいでん]が 広[ひろ]がっていたよ。	まど の そと に すいでん が ひろがって いた よ	
\\	窓[まど]の 外[そと]に
\\	が 広[ひろ]がっていたよ。			
\\	森林	森林[しんりん]	しんりん	
\\	世界中で森林が失われています。	世界中[せかいじゅう]で 森林[しんりん]が 失[うしな]われています。	せかいじゅう で しんりん が うしなわれて います	
\\	世界中[せかいじゅう]で
\\	が 失[うしな]われています。			
\\	男らしい	男[おとこ]らしい	おとこらしい	
\\	彼は男らしさを心掛けているね。	彼[かれ]は 男[おとこ]らしさを 心掛[こころが]けているね。	かれ は おとこらしさ を こころがけて いる ね	
\\	彼[かれ]は
\\	を 心掛[こころが]けているね。			
\\	女らしい	女[おんな]らしい	おんならしい	
\\	彼女は女らしい。	彼女[かのじょ]は 女[おんな]らしい。	かのじょ は おんならしい	
\\	彼女[かのじょ]は
\\	少女	少女[しょうじょ]	しょうじょ	
\\	少女は母親の手を握った。	少女[しょうじょ]は 母親[ははおや]の 手[て]を握[にぎ]った。	しょうじょ は ははおや の て を にぎった	
\\	は 母親[ははおや]の 手[て]を 握[にぎ]った。			
\\	女子	女子[じょし]	じょし	
\\	このクラスの女子は18人です。	このクラスの 女子[じょし]は 18人[じゅうはちにん]です。	この くらす の じょし は じゅうはちにん です	
\\	このクラスの
\\	は 18人[じゅうはちにん]です。			
\\	好む	好[この]む	このむ	
\\	彼女は背の高い男性を好みますね。	彼女[かのじょ]は 背[せ]の 高[たか]い 男性[だんせい]を 好[この]みますね。	かのじょ は せ の たかい だんせい を このみます ね	
\\	彼女[かのじょ]は 背[せ]の 高[たか]い 男性[だんせい]を
\\	ね。			
\\	エイズ	エイズ	エイズ	
\\	その国ではエイズが大きな社会問題ね。	その 国[くに]ではエイズが 大[おお]きな 社会問題[しゃかい もんだい]ね。	その くに で は えいず が おおき な しゃかい もんだい ね	
\\	その 国[くに]では
\\	が 大[おお]きな 社会問題[しゃかい もんだい]ね。			
\\	好み	好[この]み	このみ	
\\	姉と私は服の好みが似ています。	姉[あね]と 私[わたし]は 服[ふく]の 好[この]みが 似[に]ています。	あね と わたし は ふく の このみ が にて います	
\\	姉[あね]と 私[わたし]は 服[ふく]の
\\	が 似[に]ています。			
\\	上達	上達[じょうたつ]	じょうたつ	
\\	短い時間にずいぶん上達しましたね。	短[みじか]い 時間[じかん]にずいぶん 上達[じょうたつ]しましたね。	みじかい じかん に ずいぶん じょうたつ しました ね	
\\	短[みじか]い 時間[じかん]にずいぶん
\\	しましたね。			
\\	一家	一家[いっか]	いっか	
\\	あの一家は仲がいい。	あの 一家[いっか]は 仲[なか]がいい。	あの いっか は なか が いい	
\\	あの
\\	は 仲[なか]がいい。			
\\	家出	家出[いえで]	いえで	
\\	彼の息子が家出したそうよ。	彼[かれ]の 息子[むすこ]が 家出[いえで]したそうよ。	かれ の むすこ が いえで した そう よ	
\\	彼[かれ]の 息子[むすこ]が
\\	したそうよ。			
\\	大家	大家[おおや]	おおや	
\\	ここの大家は近くに住んでいますよ。	ここの 大家[おおや]は 近[ちか]くに 住[す]んでいますよ。	ここ の おおや は ちかく に すんで います よ	
\\	ここの
\\	は 近[ちか]くに 住[す]んでいますよ。			
\\	元来	元来[がんらい]	がんらい	
\\	彼は元来、真面目な人です。	彼[かれ]は 元来[がんらい]、 真面目[まじめ]な 人[ひと]です。	かれ は がんらい まじめ な ひと です	
\\	彼[かれ]は
\\	、 真面目[まじめ]な 人[ひと]です。			
\\	元日	元日[がんじつ]	がんじつ	
\\	元日には初詣でに行きます。	元日[がんじつ]には 初詣[はつもう]でに 行[い]きます。	がんじつ に は はつもうで に いきます	
\\	には 初詣[はつもう]でに 行[い]きます。			
\\	エンジン	エンジン	エンジン	
\\	車のエンジンが調子悪い。	車[くるま]のエンジンが 調子悪[ちょうし わる]い。	くるま の えんじん が ちょうし わるい	
\\	車[くるま]の
\\	が 調子悪[ちょうし わる]い。			
\\	お中元	お 中元[ちゅうげん]	おちゅうげん	
\\	お中元にビールが届きました。	お 中元[ちゅうげん]にビールが 届[とど]きました。	おちゅうげん に びーる が とどきました	
\\	にビールが 届[とど]きました。			
\\	天の川	天[あま]の 川[がわ]	あまのがわ	
\\	今夜は天の川が見えますね。	今夜[こんや]は 天[あま]の 川[がわ]が 見[み]えますね。	こんや は あまのがわ が みえます ね	
\\	今夜[こんや]は
\\	が 見[み]えますね。			
\\	気分	気分[きぶん]	きぶん	
\\	今日は最高にいい気分だよ。	今日[きょう]は 最高[さいこう]にいい 気分[きぶん]だよ。	きょう は さいこう に いい きぶん だ よ	
\\	今日[きょう]は 最高[さいこう]にいい
\\	だよ。			
\\	元気	元気[げんき]	げんき	
\\	彼女は最近元気がないですね。	彼女[かのじょ]は 最近[さいきん] 元気[げんき]がないですね。	かのじょ は さいきん げんき が ない です ね	
\\	彼女[かのじょ]は 最近[さいきん]
\\	がないですね。			
\\	気体	気体[きたい]	きたい	
\\	水が沸騰して気体になったんだ。	水[みず]が 沸騰[ふっとう]して 気体[きたい]になったんだ。	みず が ふっとう して きたい に なった ん だ	
\\	水[みず]が 沸騰[ふっとう]して
\\	になったんだ。			
\\	気力	気力[きりょく]	きりょく	
\\	彼は気力にあふれていますね。	彼[かれ]は 気力[きりょく]にあふれていますね。	かれ は きりょく に あふれて います ね	
\\	彼[かれ]は
\\	にあふれていますね。			
\\	大雨	大雨[おおあめ]	おおあめ	
\\	各地で大雨が降っています。	各地[かくち]で 大雨[おおあめ]が 降[ふ]っています。	かくち で おおあめ が ふって います	
\\	各地[かくち]で
\\	が 降[ふ]っています。			
\\	クラス	クラス	クラス	
\\	彼はビジネスクラスに乗ったの。	彼[かれ]はビジネスクラスに 乗[の]ったの。	かれ は びじねすくらす に のった の	
\\	彼[かれ]はビジネス
\\	に 乗[の]ったの。			
\\	小雨	小雨[こさめ]	こさめ	
\\	小雨なので傘はいりません。	小雨[こさめ]なので 傘[かさ]はいりません。	こさめ な の で かさ は いりません	
\\	なので 傘[かさ]はいりません。			
\\	雨天	雨天[うてん]	うてん	
\\	試合は雨天中止です。	試合[しあい]は 雨天[うてん] 中止[ちゅうし]です。	しあい は うてん ちゅうし です	
\\	試合[しあい]は
\\	中止[ちゅうし]です。			
\\	大雪	大雪[おおゆき]	おおゆき	
\\	10年振りの大雪です。	10年振[じゅうねん ぶ]りの 大雪[おおゆき]です。	じゅうねん ぶり の おおゆき です	
\\	10年振[じゅうねん ぶ]りの
\\	です。			
\\	青年	青年[せいねん]	せいねん	
\\	町の青年たちはボランティア活動をしています。	町[まち]の 青年[せいねん]たちはボランティア 活動[かつどう]をしています。	まち の せいねんたち は ぼらんてぃあ かつどう を して います	
\\	町[まち]の
\\	たちはボランティア 活動[かつどう]をしています。			
\\	青空	青空[あおぞら]	あおぞら	
\\	雲一つない青空ですね。	雲一[くも ひと]つない 青空[あおぞら]ですね。	くも ひとつ ない あおぞら です ね	
\\	雲一[くも ひと]つない
\\	ですね。			
\\	明日	明日[あす]	あす	
\\	明日のプレゼンテーションが心配だ。	明日[あす]のプレゼンテーションが 心配[しんぱい]だ。	あす の ぷれぜんてーしょん が しんぱい だ	
\\	の プレゼンテーションが 心配[しんぱい]だ。			
\\	明ける	明[あ]ける	あける	
\\	もうすぐ夜が明けるね。	もうすぐ 夜[よ]が 明[あ]けるね。	もうすぐ よ が あける ね	
\\	もうすぐ 夜[よ]が
\\	ね。			
\\	明かり	明[あ]かり	あかり	
\\	部屋の明かりを点けましょう。	部屋[へや]の 明[あ]かりを 点[つ]けましょう。	へや の あかり を つけましょう	
\\	部屋[へや]の
\\	を 点[つ]けましょう。			
\\	スピード	スピード	スピード	
\\	カーブを曲がるときはスピードを落としましょう。	カーブを 曲[ま]がるときはスピードを 落[お]としましょう。	かーぶ を まがる とき は すぴーど を おとしましょう	
\\	カーブを 曲[ま]がるときは
\\	を 落[お]としましょう。			
\\	明け方	明[あ]け 方[がた]	あけがた	
\\	明け方に雨が降り始めましたね。	明[あ]け 方[がた]に 雨[あめ]が 降[ふ]り 始[はじ]めましたね。	あけがた に あめ が ふりはじめました ね	
\\	に 雨[あめ]が 降[ふ]り 始[はじ]めましたね。			
\\	一昨年	一昨年[いっさくねん]	いっさくねん	
\\	大学卒業は一昨年です。	大学卒業[だいがく そつぎょう]は 一昨年[いっさくねん]です。	だいがく そつぎょう は いっさくねん です	
\\	大学卒業[だいがく そつぎょう]は
\\	です。			
\\	一昨日	一昨日[いっさくじつ]	いっさくじつ	
\\	一昨日、お電話を差し上げました。	一昨日[いっさくじつ]、お 電話[でんわ]を 差[さ]し 上[あ]げました。	いっさくじつ おでんわ を さしあげました	
\\	、お 電話[でんわ]を 差[さ]し 上[あ]げました。			
\\	昨日	昨日[さくじつ]	さくじつ	
\\	昨日は雨でしたね。	昨日[さくじつ]は 雨[あめ]でしたね。	さくじつ は あめ でした ね	
\\	は 雨[あめ]でしたね。			
\\	向上	向上[こうじょう]	こうじょう	
\\	全員で技術の向上に努めています。	全員[ぜんいん]で 技術[ぎじゅつ]の 向上[こうじょう]に 努[つと]めています。	ぜんいん で ぎじゅつ の こうじょう に つとめて います	
\\	全員[ぜんいん]で 技術[ぎじゅつ]の
\\	に 努[つと]めています。			
\\	一向に	一向[いっこう]に	いっこうに	
\\	彼は一向に興味を示さないの。	彼[かれ]は 一向[いっこう]に 興味[きょうみ]を 示[しめ]さないの。	かれ は いっこうに きょうみ を しめさない の	
\\	彼[かれ]は
\\	興味[きょうみ]を 示[しめ]さないの。			
\\	開き	開[あ]き	あき	
\\	このブラウスは後ろ開きです。	このブラウスは 後[うし]ろ 開[あ]きです。	この ぶらうす は うしろあき です	
\\	このブラウスは 後[うし]ろ
\\	です。			
\\	いかに	いかに	いかに	
\\	彼に会えば、彼がいかに良い人か分かります。	彼[かれ]に 会[あ]えば、 彼[かれ]がいかに 良[い]い 人[ひと]か 分[わ]かります。	かれ に あえば かれ が いかに いい ひと か わかります	
\\	彼[かれ]に 会[あ]えば、 彼[かれ]が
\\	良[い]い 人[ひと]か 分[わ]かります。			
\\	聞かす	聞[き]かす	きかす	
\\	子供に昔話を聞かせてあげたの。	子供[こども]に 昔話[むかしばなし]を 聞[き]かせてあげたの。	こども に むかしばなし を きかせてあげた の 。	
\\	子供[こども]に 昔話[むかしばなし]を
\\	あげたの。			
\\	聞き手	聞[き]き 手[て]	ききて	
\\	彼女はいつも聞き手にまわるね。	彼女[かのじょ]はいつも 聞[き]き 手[て]にまわるね。	かのじょ は いつも ききて に まわる ね	
\\	彼女[かのじょ]はいつも
\\	にまわるね。			
\\	客間	客間[きゃくま]	きゃくま	
\\	お客さんを客間にお通ししたわよ。	お 客[きゃく]さんを 客間[きゃくま]にお 通[とお]ししたわよ。	おきゃくさん を きゃくま に おとおし した わ よ	
\\	お 客[きゃく]さんを
\\	にお 通[とお]ししたわよ。			
\\	間	間[かん]	かん	
\\	その間に彼は居なくなっていました。	その 間[かん]に 彼[かれ]は 居[い]なくなっていました。	その かん に かれ は いなく なって いました	
\\	その
\\	に 彼[かれ]は 居[い]なくなっていました。			
\\	空間	空間[くうかん]	くうかん	
\\	狭い空間に物がたくさん置いてあるね。	狭[せま]い 空間[くうかん]に 物[もの]がたくさん 置[お]いてあるね。	せまい くうかん に もの が たくさん おいて ある ね	
\\	狭[せま]い
\\	に 物[もの]がたくさん 置[お]いてあるね。			
\\	円高	円高[えんだか]	えんだか	
\\	円高の影響で海外製品が安く買えますよ。	円高[えんだか]の 影響[えいきょう]で 海外製品[かいがい せいひん]が 安[やす]く 買[か]えますよ。	えんだか の えいきょう で かいがい せいひん が やすく かえます よ	
\\	の 影響[えいきょう]で 海外製品[かいがい せいひん]が 安[やす]く 買[か]えますよ。			
\\	最高	最高[さいこう]	さいこう	
\\	これまでで最高の結果が出たよ。	これまでで 最高[さいこう]の 結果[けっか]が 出[で]たよ。	これまで で さいこう の けっか が でた よ	
\\	これまでで
\\	の 結果[けっか]が 出[で]たよ。			
\\	スタート	スタート	スタート	
\\	マラソンは雨の中でスタートしたの。	マラソンは 雨[あめ]の 中[なか]でスタートしたの。	まらそん は あめ の なか で すたーと した の	
\\	マラソンは 雨[あめ]の 中[なか]で
\\	したの。			
\\	最低	最低[さいてい]	さいてい	
\\	これは今までで最低の記録だ。	これは 今[いま]までで 最低[さいてい]の 記録[きろく]だ。	これ は いま まで で さいてい の きろく だ	
\\	これは 今[いま]までで
\\	の 記録[きろく]だ。			
\\	最小	最小[さいしょう]	さいしょう	
\\	これは世界で最小のコンピューターです。	これは 世界[せかい]で 最小[さいしょう]のコンピューターです。	これ は せかい で さいしょう の こんぴゅーたー です	
\\	これは 世界[せかい]で
\\	のコンピューターです。			
\\	最上	最上[さいじょう]	さいじょう	
\\	このホテルでは最上のサービスが受けられます。	このホテルでは 最上[さいじょう]のサービスが 受[う]けられます。	この ほてる で は さいじょう の さーびす が うけられます	
\\	このホテルでは
\\	のサービスが 受[う]けられます。			
\\	最中	最中[さいちゅう]	さいちゅう	
\\	夕食の最中に電話がかかってきたの。	夕食[ゆうしょく]の 最中[さいちゅう]に 電話[でんわ]がかかってきたの。	ゆうしょく の さいちゅう に でんわ が かかって きた の	
\\	夕食[ゆうしょく]の
\\	に 電話[でんわ]がかかってきたの。			
\\	初日	初日[しょにち]	しょにち	
\\	会議の初日に市長がスピーチをしたよ。	会議[かいぎ]の 初日[しょにち]に 市長[しちょう]がスピーチをしたよ。	かいぎ の しょにち に しちょう が すぴーち を した よ	
\\	会議[かいぎ]の
\\	に 市長[しちょう]がスピーチをしたよ。			
\\	お前	お 前[まえ]	おまえ	
\\	お前の言うことは信じられない。	お 前[まえ]の 言[い]うことは 信[しん]じられない。	おまえ の いう こと は しんじられ ない	
\\	の 言[い]うことは 信[しん]じられない。			
\\	前後	前後[ぜんご]	ぜんご	
\\	そちらに着くのは6時前後です。	そちらに 着[つ]くのは 6時[ろくじ] 前後[ぜんご]です。	そちら に つく の は ろくじ ぜんご です	
\\	そちらに 着[つ]くのは 6時[ろくじ]
\\	です。			
\\	後方	後方[こうほう]	こうほう	
\\	彼は後方の座席に着いたの。	彼[かれ]は 後方[こうほう]の 座席[ざせき]に 着[つ]いたの。	かれ は こうほう の ざせき に ついた の	
\\	彼[かれ]は
\\	の 座席[ざせき]に 着[つ]いたの。			
\\	セット	セット	セット	
\\	このメニューはサラダと飲み物がセットになっています。	このメニューはサラダと 飲[の]み 物[もの]がセットになっています。	この めにゅー は さらだ と のみもの が せっと に なって います	
\\	このメニューはサラダと 飲[の]み 物[もの]が
\\	になっています。			
\\	後ろ向き	後[うし]ろ 向[む]き	うしろむき	
\\	彼は車を後ろ向きに駐車したの。	彼[かれ]は 車[くるま]を 後[うし]ろ 向[む]きに 駐車[ちゅうしゃ]したの。	かれ は くるま を うしろむき に ちゅうしゃ した の	
\\	彼[かれ]は 車[くるま]を
\\	に 駐車[ちゅうしゃ]したの。			
\\	明々後日	明々後日[しあさって]	しあさって	
\\	会議は明々後日に延期された。	会議[かいぎ]は 明々後日[しあさって]に 延期[えんき]された。	かいぎ は しあさって に えんき された	
\\	会議[かいぎ]は
\\	に 延期[えんき]された。			
\\	後半	後半[こうはん]	こうはん	
\\	ドラマの後半は来週放送されるんだ。	ドラマの 後半[こうはん]は 来週放送[らいしゅう ほうそう]されるんだ。	どらま の こうはん は らいしゅう ほうそう される ん だ	
\\	ドラマの
\\	は 来週放送[らいしゅう ほうそう]されるんだ。			
\\	朝日	朝日[あさひ]	あさひ	
\\	朝日が昇りましたよ。	朝日[あさひ]が 昇[のぼ]りましたよ。	あさひ が のぼりました よ	
\\	が 昇[のぼ]りましたよ。			
\\	昨晩	昨晩[さくばん]	さくばん	
\\	昨晩の雪がまだ庭に残っている。	昨晩[さくばん]の 雪[ゆき]がまだ 庭[にわ]に 残[のこ]っている。	さくばん の ゆき が まだ にわ に のこって いる	
\\	の 雪[ゆき]がまだ 庭[にわ]に 残[のこ]っている。			
\\	昨夜	昨夜[さくや]	さくや	
\\	昨夜はテレビで喜劇を見たよ。	昨夜[さくや]はテレビで 喜劇[きげき]を 見[み]たよ。	さくや は てれび で きげき を みた よ	
\\	は テレビで 喜劇[きげき]を 見[み]たよ。			
\\	食う	食[く]う	くう	
\\	腹一杯食った。	腹一杯[はらいっぱい] 食[く]った。	はらいっぱい くった	
\\	腹一杯[はらいっぱい]
\\	アップ	アップ	アップ	
\\	写真をアップで撮ろう。	写真[しゃしん]をアップで 撮[と]ろう。	しゃしん を あっぷ で とろう	
\\	写真[しゃしん]を
\\	で 撮[と]ろう。			
\\	外食	外食[がいしょく]	がいしょく	
\\	たまには外食しましょう。	たまには 外食[がいしょく]しましょう。	たま に は がいしょく しましょう	
\\	たまには
\\	しましょう。			
\\	食パン	食[しょく]パン	しょくぱん	
\\	朝食に食パンを2枚食べました。	朝食[ちょうしょく]に 食[しょく]パンを 2枚食[にまい た]べました。	ちょうしょく に しょくぱん を にまい たべました	
\\	朝食[ちょうしょく]に
\\	を 2枚食[にまい た]べました。			
\\	一見	一見[いっけん]	いっけん	
\\	彼は一見サラリーマン風ですね。	彼[かれ]は 一見[いっけん]サラリーマン 風[ふう]ですね。	かれ は いっけん さらりーまんふう です ね	
\\	彼[かれ]は
\\	サラリーマン 風[ふう]ですね。			
\\	言い出す	言[い]い 出[だ]す	いいだす	
\\	突然何を言い出すのかと思った。	突然何[とつぜん なに]を 言[い]い 出[だ]すのかと 思[おも]った。	とつぜん なに を いいだす の か と おもった	
\\	突然何[とつぜん なに]を
\\	のかと 思[おも]った。			
\\	外来語	外来語[がいらいご]	がいらいご	
\\	外来語は一般にカタカナで書かれます。	外来語[がいらいご]は 一般[いっぱん]にカタカナで 書[か]かれます。	がいらいご は いっぱんに かたかな で かかれます	
\\	は 一般[いっぱん]にカタカナで 書[か]かれます。			
\\	英文	英文[えいぶん]	えいぶん	
\\	彼女は英文の手紙を書きました。	彼女[かのじょ]は 英文[えいぶん]の 手紙[てがみ]を 書[か]きました。	かのじょ は えいぶん の てがみ を かきました	
\\	彼女[かのじょ]は
\\	の 手紙[てがみ]を 書[か]きました。			
\\	漢語	漢語[かんご]	かんご	
\\	漢語はもともと外来語です。	漢語[かんご]はもともと 外来語[がいらいご]です。	かんご は もともと がいらいご です	
\\	はもともと 外来語[がいらいご]です。			
\\	エネルギー	エネルギー	エネルギー	
\\	若者たちはエネルギーにあふれていますね。	若者[わかもの]たちはエネルギーにあふれていますね。	わかものたち は えねるぎー に あふれて います ね	
\\	若者[わかもの]たちは
\\	にあふれていますね。			
\\	下書き	下書[したが]き	したがき	
\\	今、論文の下書きをしているところです。	今[いま]、 論文[ろんぶん]の 下書[したが]きをしているところです。	いま ろんぶん の したがき を して いる ところ です	
\\	今[いま]、 論文[ろんぶん]の
\\	をしているところです。			
\\	後書き	後書[あとが]き	あとがき	
\\	後書きをよく読んで下さい。	後書[あとが]きをよく 読[よ]んで 下[くだ]さい。	あとがき を よく よんで ください	
\\	をよく 読[よ]んで 下[くだ]さい。			
\\	覚え	覚[おぼ]え	おぼえ	
\\	この子は覚えが早いね。	この 子[こ]は 覚[おぼ]えが 早[はや]いね。	この こ は おぼえ が はやい ね	
\\	この 子[こ]は
\\	が 早[はや]いね。			
\\	開会	開会[かいかい]	かいかい	
\\	運動会は9時に開会します。	運動会[うんどうかい]は 9時[くじ]に 開会[かいかい]します。	うんどうかい は くじ に かいかい します	
\\	運動会[うんどうかい]は 9時[くじ]に
\\	します。			
\\	会	会[かい]	かい	
\\	会は午後9時に終わりました。	会[かい]は 午後9時[ごご くじ]に 終[お]わりました。	かい は ごご くじ に おわりました	
\\	は 午後9時[ごご くじ]に 終[お]わりました。			
\\	合わせる	合[あ]わせる	あわせる	
\\	日時はご都合に合わせます。	日時[にちじ]はご 都合[つごう]に 合[あ]わせます。	にちじ は ごつごう に あわせます	
\\	日時[にちじ]はご 都合[つごう]に
\\	合わす	合[あ]わす	あわす	
\\	赤に黄色を合わすと何色になりますか。	赤[あか]に 黄色[きいろ]を 合[あ]わすと 何色[なにいろ]になりますか。	あか に きいろ を あわすと なにいろ に なります か	
\\	赤[あか]に 黄色[きいろ]を
\\	と 何色[なにいろ]になりますか。			
\\	社会人	社会人[しゃかいじん]	しゃかいじん	
\\	この春に彼は社会人になったよ。	この 春[はる]に 彼[かれ]は 社会人[しゃかいじん]になったよ。	この はる に かれ は しゃかいじん に なった よ	
\\	この 春[はる]に 彼[かれ]は
\\	になったよ。			
\\	かえって	かえって	かえって	
\\	そんなことをしたら、かえってよくないよ。	そんなことをしたら、かえってよくないよ。	そんなことをしたら、かえってよくないよ。	
\\	そんなことをしたら、
\\	よくないよ。			
\\	会員	会員[かいいん]	かいいん	
\\	彼女はクラブの会員です。	彼女[かのじょ]はクラブの 会員[かいいん]です。	かのじょ は くらぶ の かいいん です	
\\	彼女[かのじょ]はクラブの
\\	です。			
\\	一員	一員[いちいん]	いちいん	
\\	彼は野球部の一員です。	彼[かれ]は 野球部[やきゅうぶ]の 一員[いちいん]です。	かれ は やきゅうぶ の いちいん です	
\\	彼[かれ]は 野球部[やきゅうぶ]の
\\	です。			
\\	仕上げ	仕上[しあ]げ	しあげ	
\\	彼は論文の仕上げに入ったの。	彼[かれ]は 論文[ろんぶん]の 仕上[しあ]げに 入[はい]ったの。	かれ は ろんぶん の しあげ に はいった の	
\\	彼[かれ]は 論文[ろんぶん]の
\\	に 入[はい]ったの。			
\\	仕上げる	仕上[しあ]げる	しあげる	
\\	彼はやっと報告書を仕上げたよ。	彼[かれ]はやっと 報告書[ほうこくしょ]を 仕上[しあ]げたよ。	かれ は やっと ほうこくしょ を しあげた よ	
\\	彼[かれ]はやっと 報告書[ほうこくしょ]を
\\	よ。			
\\	人事	人事[じんじ]	じんじ	
\\	彼は人事を担当しているんだ。	彼[かれ]は 人事[じんじ]を 担当[たんとう]しているんだ。	かれ は じんじ を たんとう して いる ん だ	
\\	彼[かれ]は
\\	を 担当[たんとう]しているんだ。			
\\	行事	行事[ぎょうじ]	ぎょうじ	
\\	今日は学校で行事がありました。	今日[きょう]は 学校[がっこう]で 行事[ぎょうじ]がありました。	きょう は がっこう で ぎょうじ が ありました	
\\	今日[きょう]は 学校[がっこう]で
\\	がありました。			
\\	事	事[こと]	こと	
\\	そんな事をしてはいけません。	そんな 事[こと]をしてはいけません。	そんな こと を して は いけません	
\\	そんな
\\	をしてはいけません。			
\\	コンクリート	コンクリート	コンクリート	
\\	コンクリートにひびが入っていますね。	コンクリートにひびが 入[はい]っていますね。	こんくりーと に ひび が はいって います ね	
\\	にひびが 入[はい]っていますね。			
\\	家事	家事[かじ]	かじ	
\\	母は毎日てきぱきと家事をしているよ。	母[はは]は 毎日[まいにち]てきぱきと 家事[かじ]をしているよ。	はは は まいにち てきぱき と かじ を して いる よ	
\\	母[はは]は 毎日[まいにち]てきぱきと
\\	をしているよ。			
\\	人工	人工[じんこう]	じんこう	
\\	これは人工の湖です。	これは 人工[じんこう]の 湖[みずうみ]です。	これ は じんこう の みずうみ です	
\\	これは
\\	の 湖[みずうみ]です。			
\\	会場	会場[かいじょう]	かいじょう	
\\	会場は人で一杯になりました。	会場[かいじょう]は 人[ひと]で 一杯[いっぱい]になりました。	かいじょう は ひと で いっぱい に なりました	
\\	は 人[ひと]で 一杯[いっぱい]になりました。			
\\	出場	出場[しゅつじょう]	しゅつじょう	
\\	今日は8チームが出場しました。	今日[きょう]は 8[はち]チームが 出場[しゅつじょう]しました。	きょう は はち ちーむ が しゅつじょう しました	
\\	今日[きょう]は 8[はち]チームが
\\	しました。			
\\	車内	車内[しゃない]	しゃない	
\\	車内に忘れ物があったよ。	車内[しゃない]に 忘[わす]れ 物[もの]があったよ。	しゃない に わすれもの が あった よ	
\\	に 忘[わす]れ 物[もの]があったよ。			
\\	下車	下車[げしゃ]	げしゃ	
\\	景色が良かったので途中下車したよ。	景色[けしき]が 良[よ]かったので 途中[とちゅう] 下車[げしゃ]したよ。	けしき が よかった の で とちゅう げしゃ した よ	
\\	景色[けしき]が 良[よ]かったので 途中[とちゅう]
\\	したよ。			
\\	駅前	駅前[えきまえ]	えきまえ	
\\	駅前に交番が有ります。	駅前[えきまえ]に 交番[こうばん]が 有[あ]ります。	えきまえ に こうばん が あります	
\\	に 交番[こうばん]が 有[あ]ります。			
\\	いよいよ	いよいよ	いよいよ	
\\	明日はいよいよ出発の日です。	明日[あす]はいよいよ 出発[しゅっぱつ]の 日[ひ]です。	あす は いよいよ しゅっぱつ の ひ です	
\\	明日[あす]は
\\	出発[しゅっぱつ]の 日[ひ]です。			
\\	外交	外交[がいこう]	がいこう	
\\	政府は外交に力を入れているの。	政府[せいふ]は 外交[がいこう]に 力[ちから]を 入[い]れているの。	せいふ は がいこう に ちから を いれて いる の	
\\	政府[せいふ]は
\\	に 力[ちから]を 入[い]れているの。			
\\	交わす	交[か]わす	かわす	
\\	彼は私と再会の約束を交わした。	彼[かれ]は 私[わたし]と 再会[さいかい]の 約束[やくそく]を 交[か]わした。	かれ は わたし と さいかい の やくそく を かわした	
\\	彼[かれ]は 私[わたし]と 再会[さいかい]の 約束[やくそく]を
\\	交通	交通[こうつう]	こうつう	
\\	ここは交通の便がよいですね。	ここは 交通[こうつう]の 便[べん]がよいですね。	ここ は こうつう の べん が よい です ね	
\\	ここは
\\	の 便[べん]がよいですね。			
\\	大通り	大通[おおどお]り	おおどおり	
\\	大通りでパレードが始まるよ。	大通[おおどお]りでパレードが 始[はじ]まるよ。	おおどおり で ぱれーど が はじまる よ	
\\	でパレードが 始[はじ]まるよ。			
\\	書道	書道[しょどう]	しょどう	
\\	書道をすると姿勢も良くなります。	書道[しょどう]をすると 姿勢[しせい]も 良[よ]くなります。	しょどう を する と しせい も よく なります	
\\	をすると 姿勢[しせい]も 良[よ]くなります。			
\\	十字路	十字路[じゅうじろ]	じゅうじろ	
\\	十字路で車とバイクが衝突したぞ。	十字路[じゅうじろ]で 車[くるま]とバイクが 衝突[しょうとつ]したぞ。	じゅうじろ で くるま と ばいく が しょうとつ した ぞ	
\\	で 車[くるま]とバイクが 衝突[しょうとつ]したぞ。			
\\	空き地	空[あ]き 地[ち]	あきち	
\\	空き地で工事が始まった。	空[あ]き 地[ち]で 工事[こうじ]が 始[はじ]まった。	あきち で こうじ が はじまった	
\\	で 工事[こうじ]が 始[はじ]まった。			
\\	オリンピック	オリンピック	オリンピック	
\\	オリンピックは4年に1度開催されます。	オリンピックは 4年[よねん]に 1度開催[いちど かいさい]されます。	おりんぴっく は よねん に いちど かいさい されます	
\\	は 4年[よねん]に 1度開催[いちど かいさい]されます。			
\\	図	図[ず]	ず	
\\	図を描いて説明しましょう。	図[ず]を 描[か]いて 説明[せつめい]しましょう。	ず を かいて せつめい しましょう	
\\	を 描[か]いて 説明[せつめい]しましょう。			
\\	合図	合図[あいず]	あいず	
\\	車掌が発車の合図をしたよ。	車掌[しゃしょう]が 発車[はっしゃ]の 合図[あいず]をしたよ。	しゃしょう が はっしゃ の あいず を した よ	
\\	車掌[しゃしょう]が 発車[はっしゃ]の
\\	をしたよ。			
\\	行き止まり	行[い]き 止[ど]まり	いきどまり	
\\	ここから先は行き止まりです。	ここから 先[さき]は 行[い]き 止[ど]まりです。	ここ から さき は いきどまり です	
\\	ここから 先[さき]は
\\	です。			
\\	初歩	初歩[しょほ]	しょほ	
\\	英語を初歩から勉強しています。	英語[えいご]を 初歩[しょほ]から 勉強[べんきょう]しています。	えいご を しょほ から べんきょう して います	
\\	英語[えいご]を
\\	から 勉強[べんきょう]しています。			
\\	高度	高度[こうど]	こうど	
\\	高度な技術を持つ人材を集めました。	高度[こうど]な 技術[ぎじゅつ]を 持[も]つ 人材[じんざい]を 集[あつ]めました。	こうど な ぎじゅつ を もつ じんざい を あつめました	
\\	な 技術[ぎじゅつ]を 持[も]つ 人材[じんざい]を 集[あつ]めました。			
\\	一度に	一度[いちど]に	いちどに	
\\	皆の顔と名前を一度には覚えられません。	皆[みんな]の 顔[かお]と 名前[なまえ]を 一度[いちど]には 覚[おぼ]えられません。	みんな の かお と なまえ を いちどに は おぼえられません	
\\	皆[みんな]の 顔[かお]と 名前[なまえ]を
\\	は 覚[おぼ]えられません。			
\\	遠足	遠足[えんそく]	えんそく	
\\	明日は遠足で動物園に行きます。	明日[あす]は 遠足[えんそく]で 動物園[どうぶつえん]に 行[い]きます。	あす は えんそく で どうぶつえん に いきます	
\\	明日[あす]は
\\	で 動物園[どうぶつえん]に 行[い]きます。			
\\	駅長	駅長[えきちょう]	えきちょう	
\\	彼は駅長です。	彼[かれ]は 駅長[えきちょう]です。	かれ は えきちょう です	
\\	彼[かれ]は
\\	です。			
\\	ストップ	ストップ	ストップ	
\\	ここでいったんストップしてください。	ここでいったんストップしてください。	ここ で いったん すとっぷ して ください	
\\	ここでいったん
\\	してください。			
\\	全員	全員[ぜんいん]	ぜんいん	
\\	チーム全員に名札が配られました。	チーム 全員[ぜんいん]に 名札[なふだ]が 配[くば]られました。	ちーむ ぜんいん に なふだ が くばられました	
\\	チーム
\\	に 名札[なふだ]が 配[くば]られました。			
\\	全て	全[すべ]て	すべて	
\\	彼に全てのことを伝えた。	彼[かれ]に 全[すべ]てのことを 伝[つた]えた。	かれ に すべて の こと を つたえた	
\\	彼[かれ]に
\\	のことを 伝[つた]えた。			
\\	外部	外部[がいぶ]	がいぶ	
\\	これは外部には秘密です。	これは 外部[がいぶ]には 秘密[ひみつ]です。	これ は がいぶ に は ひみつ です	
\\	これは
\\	には 秘密[ひみつ]です。			
\\	一部分	一部分[いちぶぶん]	いちぶぶん	
\\	僕が知っているのは一部分に過ぎない。	僕[ぼく]が 知[し]っているのは 一部分[いちぶぶん]に 過[す]ぎない。	ぼく が しって いる の は いちぶぶん に すぎない	
\\	僕[ぼく]が 知[し]っているのは
\\	に 過[す]ぎない。			
\\	国家	国家[こっか]	こっか	
\\	首相は国家のリーダーだ。	首相[しゅしょう]は 国家[こっか]のリーダーだ。	しゅしょう は こっか の りーだー だ	
\\	首相[しゅしょう]は
\\	のリーダーだ。			
\\	国々	国々[くにぐに]	くにぐに	
\\	そのマラソン大会にはたくさんの国々から選手が集まったよ。	そのマラソン 大会[たいかい]にはたくさんの 国々[くにぐに]から 選手[せんしゅ]が 集[あつ]まったよ。	その まらそん たいかい に は たくさん の くにぐに から せんしゅ が あつまった よ	
\\	そのマラソン 大会[たいかい]にはたくさんの
\\	から 選手[せんしゅ]が 集[あつ]まったよ。			
\\	国外	国外[こくがい]	こくがい	
\\	犯人は国外に逃げたようです。	犯人[はんにん]は 国外[こくがい]に 逃[に]げたようです。	はんにん は こくがい に にげた よう です	
\\	犯人[はんにん]は
\\	に 逃[に]げたようです。			
\\	きちんと	きちんと	きちんと	
\\	問題についてきちんと話し合ったよ。	問題[もんだい]についてきちんと 話[はな]し 合[あ]ったよ。	もんだい に ついて きちんと はなしあった よ	
\\	問題[もんだい]について
\\	話[はな]し 合[あ]ったよ。			
\\	出国	出国[しゅっこく]	しゅっこく	
\\	1週間後に出国します。	1週間後[いっしゅうかん ご]に 出国[しゅっこく]します。	いっしゅうかん ご に しゅっこく します	
\\	1週間後[いっしゅうかん ご]に
\\	します。			
\\	国土	国土[こくど]	こくど	
\\	わが国の国土は70
\\	が森林です。	わが 国[くに]の 国土[こくど]は 
\\	[ななじゅっぱーせんと]が 森林[しんりん]です。	わがくに の こくど は ななじゅっぱーせんと が しんりん です	
\\	わが 国[くに]の
\\	は 
\\	[ななじゅっぱーせんと]が 森林[しんりん]です。			
\\	国語	国語[こくご]	こくご	
\\	今日の1時間目は国語です。	今日[きょう]の 1時間目[いちじかんめ]は 国語[こくご]です。	きょう の いちじかんめ は こくご です	
\\	今日[きょう]の 1時間目[いちじかんめ]は
\\	です。			
\\	国交	国交[こっこう]	こっこう	
\\	あの国とは国交がない。	あの 国[くに]とは 国交[こっこう]がない。	あの くに と は こっこう が ない	
\\	あの 国[くに]とは
\\	がない。			
\\	国道	国道[こくどう]	こくどう	
\\	この道をまっすぐ進むと国道に出ます。	この 道[みち]をまっすぐ 進[すす]むと 国道[こくどう]に 出[で]ます。	この みち を まっすぐ すすむ と こくどう に でます	
\\	この 道[みち]をまっすぐ 進[すす]むと
\\	に 出[で]ます。			
\\	世間	世間[せけん]	せけん	
\\	世間の反応は冷たかったよ。	世間[せけん]の 反応[はんのう]は 冷[つめ]たかったよ。	せけん の はんのう は つめたかった よ	
\\	の 反応[はんのう]は 冷[つめ]たかったよ。			
\\	世話	世話[せわ]	せわ	
\\	旅行の間、犬の世話をしてください。	旅行[りょこう]の 間[あいだ]、 犬[いぬ]の 世話[せわ]をしてください。	りょこう の あいだ いぬ の せわ を して ください	
\\	旅行[りょこう]の 間[あいだ]、 犬[いぬ]の
\\	をしてください。			
\\	クラブ	クラブ	クラブ	
\\	夏休みにはクラブの合宿に参加します。	夏休[なつやす]みにはクラブの 合宿[がっしゅく]に 参加[さんか]します。	なつやすみ に は くらぶ の がっしゅく に さんか します	
\\	夏休[なつやす]みには
\\	の 合宿[がっしゅく]に 参加[さんか]します。			
\\	出世	出世[しゅっせ]	しゅっせ	
\\	彼は出世するタイプだな。	彼[かれ]は 出世[しゅっせ]するタイプだな。	かれ は しゅっせ する たいぷ だ な	
\\	彼[かれ]は
\\	するタイプだな。			
\\	青白い	青白[あおじろ]い	あおじろい	
\\	彼女は青白い顔をしているね。	彼女[かのじょ]は 青白[あおじろ]い 顔[かお]をしているね。	かのじょ は あおじろい かお を して いる ね	
\\	彼女[かのじょ]は
\\	顔[かお]をしているね。			
\\	黒字	黒字[くろじ]	くろじ	
\\	わが社は今年度、黒字となりました。	わが 社[しゃ]は 今年度[こんねんど]、 黒字[くろじ]となりました。	わがしゃ は こんねんど くろじ と なりました	
\\	わが 社[しゃ]は 今年度[こんねんど]、
\\	となりました。			
\\	赤道	赤道[せきどう]	せきどう	
\\	赤道に沿って旅をしました。	赤道[せきどう]に 沿[そ]って 旅[たび]をしました。	せきどう に そって たび を しました	
\\	に 沿[そ]って 旅[たび]をしました。			
\\	赤字	赤字[あかじ]	あかじ	
\\	私たちの会社は今月は赤字よ。	私[わたし]たちの 会社[かいしゃ]は 今月[こんげつ]は 赤字[あかじ]よ。	わたしたち の かいしゃ は こんげつ は あかじ よ	
\\	私[わたし]たちの 会社[かいしゃ]は 今月[こんげつ]は
\\	よ。			
\\	国鉄	国鉄[こくてつ]	こくてつ	
\\	父は以前、国鉄に勤めていました。	父[ちち]は 以前[いぜん]、 国鉄[こくてつ]に 勤[つと]めていました。	ちち は いぜん こくてつ に つとめて いました	
\\	父[ちち]は 以前[いぜん]、
\\	に 勤[つと]めていました。			
\\	私鉄	私鉄[してつ]	してつ	
\\	私は私鉄の職員です。	私[わたし]は 私鉄[してつ]の 職員[しょくいん]です。	わたし は してつ の しょくいん です	
\\	私[わたし]は
\\	の 職員[しょくいん]です。			
\\	家屋	家屋[かおく]	かおく	
\\	私は木造の家屋が好きです。	私[わたし]は 木造[もくぞう]の 家屋[かおく]が 好[す]きです。	わたし は もくぞう の かおく が すき です	
\\	私[わたし]は 木造[もくぞう]の
\\	が 好[す]きです。			
\\	じっと	じっと	じっと	
\\	あの生徒はじっと先生の話を聞いていたね。	あの 生徒[せいと]はじっと 先生[せんせい]の 話[はなし]を 聞[き]いていたね。	あの せいと は じっと せんせい の はなし を きいて いた ね	
\\	あの 生徒[せいと]は
\\	先生[せんせい]の 話[はなし]を 聞[き]いていたね。			
\\	屋上	屋上[おくじょう]	おくじょう	
\\	屋上から富士山が見えました。	屋上[おくじょう]から 富士山[ふじさん]が 見[み]えました。	おくじょう から ふじさん が みえました	
\\	から 富士山[ふじさん]が 見[み]えました。			
\\	味わう	味[あじ]わう	あじわう	
\\	母の手料理をゆっくり味わいました。	母[はは]の 手料理[てりょうり]をゆっくり 味[あじ]わいました。	はは の てりょうり を ゆっくり あじわいました	
\\	母[はは]の 手料理[てりょうり]をゆっくり
\\	地味	地味[じみ]	じみ	
\\	今日、彼女は地味な服装をしていますね。	今日[きょう]、 彼女[かのじょ]は 地味[じみ]な 服装[ふくそう]をしていますね。	きょう かのじょ は じみ な ふくそう を して います ね	
\\	今日[きょう]、 彼女[かのじょ]は
\\	な 服装[ふくそう]をしていますね。			
\\	月末	月末[げつまつ]	げつまつ	
\\	月末までに申込書を送ってください。	月末[げつまつ]までに 申込書[もうしこみしょ]を 送[おく]ってください。	げつまつ まで に もうしこみしょ を おくって ください	
\\	までに 申込書[もうしこみしょ]を 送[おく]ってください。			
\\	末っ子	末[すえ]っ 子[こ]	すえっこ	
\\	彼は5人兄弟の末っ子です。	彼[かれ]は 5人兄弟[ごにん きょうだい]の 末[すえ]っ 子[こ]です。	かれ は ごにん きょうだい の すえっこ です	
\\	彼[かれ]は 5人兄弟[ごにん きょうだい]の
\\	です。			
\\	末	末[すえ]	すえ	
\\	長い話合いの末、やっと同意に至った。	長[なが]い 話合[はなしあ]いの 末[すえ]、やっと 同意[どうい]に 至[いた]った。	ながい はなしあい の すえ やっと どうい に いたった	
\\	長[なが]い 話合[はなしあ]いの
\\	、やっと 同意[どうい]に 至[いた]った。			
\\	飲料水	飲料水[いんりょうすい]	いんりょうすい	
\\	被災地では飲料水が不足しているの。	被災地[ひさいち]では 飲料水[いんりょうすい]が 不足[ふそく]しているの。	ひさいち で は いんりょうすい が ふそく して いる の	
\\	被災地[ひさいち]では
\\	が 不足[ふそく]しているの。			
\\	コンサート	コンサート	コンサート	
\\	友達とジャズのコンサートに行きました。	友達[ともだち]とジャズのコンサートに 行[い]きました。	ともだち と じゃず の こんさーと に いきました 。	
\\	友達[ともだち]とジャズの
\\	に 行[い]きました。			
\\	食料	食料[しょくりょう]	しょくりょう	
\\	食料はこの箱に入っています。	食料[しょくりょう]はこの 箱[はこ]に 入[はい]っています。	しょくりょう は この はこ に はいって います	
\\	はこの 箱[はこ]に 入[はい]っています。			
\\	見解	見解[けんかい]	けんかい	
\\	あなたの見解を聞かせてください。	あなたの 見解[けんかい]を 聞[き]かせてください。	あなた の けんかい を きかせて ください	
\\	あなたの
\\	を 聞[き]かせてください。			
\\	有りのまま	有[あ]りのまま	ありのまま	
\\	有りのままを話して下さい。	有[あ]りのままを 話[はな]して 下[くだ]さい。	ありのまま を はなして ください	
\\	を 話[はな]して 下[くだ]さい。			
\\	作家	作家[さっか]	さっか	
\\	彼女は有名な作家です。	彼女[かのじょ]は 有名[ゆうめい]な 作家[さっか]です。	かのじょ は ゆうめい な さっか です	
\\	彼女[かのじょ]は 有名[ゆうめい]な
\\	です。			
\\	使用	使用[しよう]	しよう	
\\	この製品を使用する前に、説明書をお読みください。	この 製品[せいひん]を 使用[しよう]する 前[まえ]に、 説明書[せつめいしょ]をお 読[よ]みください。	この せいひん を しよう する まえ に せつめいしょ を およみ ください	
\\	この 製品[せいひん]を
\\	する 前[まえ]に、 説明書[せつめいしょ]をお 読[よ]みください。			
\\	作用	作用[さよう]	さよう	
\\	この薬は神経に作用します。	この 薬[くすり]は 神経[しんけい]に 作用[さよう]します。	この くすり は しんけい に さよう します	
\\	この 薬[くすり]は 神経[しんけい]に
\\	します。			
\\	使用人	使用人[しようにん]	しようにん	
\\	彼は使用人を首にしたよ。	彼[かれ]は 使用人[しようにん]を 首[くび]にしたよ。	かれ は しようにん を くび に した よ	
\\	彼[かれ]は
\\	を 首[くび]にしたよ。			
\\	いけない	いけない	いけない	
\\	勉強を怠けてはいけないよ。	勉強[べんきょう]を 怠[なま]けてはいけないよ。	べんきょう を なまけて は いけない よ	
\\	勉強[べんきょう]を 怠[なま]けては
\\	よ。			
\\	私用	私用[しよう]	しよう	
\\	私用で出かけなければなりません。	私用[しよう]で 出[で]かけなければなりません。	しよう で でかけなければ なりません	
\\	で 出[で]かけなければなりません。			
\\	会費	会費[かいひ]	かいひ	
\\	会費が少し高いね。	会費[かいひ]が 少[すこ]し 高[たか]いね。	かいひ が すこし たかい ね	
\\	が 少[すこ]し 高[たか]いね。			
\\	国費	国費[こくひ]	こくひ	
\\	彼は国費で留学しています。	彼[かれ]は 国費[こくひ]で 留学[りゅうがく]しています。	かれ は こくひ で りゅうがく して います	
\\	彼[かれ]は
\\	で 留学[りゅうがく]しています。			
\\	食費	食費[しょくひ]	しょくひ	
\\	男の子が3人もいるので食費がかさみます。	男[おとこ]の 子[こ]が 3人[さんにん]もいるので 食費[しょくひ]がかさみます。	おとこ の こ が さんにん も いる の で しょくひ が かさみます	
\\	男[おとこ]の 子[こ]が 3人[さんにん]もいるので
\\	がかさみます。			
\\	私費	私費[しひ]	しひ	
\\	彼は私費で留学したんだ。	彼[かれ]は 私費[しひ]で 留学[りゅうがく]したんだ。	かれ は しひ で りゅうがく した ん だ	
\\	彼[かれ]は
\\	で 留学[りゅうがく]したんだ。			
\\	消費	消費[しょうひ]	しょうひ	
\\	日本は消費大国といわれています。	日本[にっぽん]は 消費[しょうひ] 大国[たいこく]といわれています。	にっぽん は しょうひ たいこく と いわれて います	
\\	日本[にっぽん]は
\\	大国[たいこく]といわれています。			
\\	消火	消火[しょうか]	しょうか	
\\	火事は無事消火されました。	火事[かじ]は 無事[ぶじ] 消火[しょうか]されました。	かじ は ぶじ しょうか されました	
\\	火事[かじ]は 無事[ぶじ]
\\	されました。			
\\	売り上げ	売[う]り 上[あ]げ	うりあげ	
\\	この会社の売り上げは昨年の2倍ね。	この 会社[かいしゃ]の 売[う]り 上[あ]げは 昨年[さくねん]の 2倍[に ばい]ね。	この かいしゃ の うりあげ は さくねん の に ばい ね	
\\	この 会社[かいしゃ]の
\\	は 昨年[さくねん]の 2倍[に ばい]ね。			
\\	あらかじめ	あらかじめ	あらかじめ	
\\	あらかじめ必要な物を書き出して下さい。	あらかじめ 必要[ひつよう]な 物[もの]を 書[か]き 出[だ]して 下[くだ]さい。	あらかじめ ひつよう な もの を かきだして ください	
\\	必要[ひつよう]な 物[もの]を 書[か]き 出[だ]して 下[くだ]さい。			
\\	売り出す	売[う]り 出[だ]す	うりだす	
\\	新しい車が売り出された。	新[あたら]しい 車[くるま]が 売[う]り 出[だ]された。	あたらしい くるま が うりだされた	
\\	新[あたら]しい 車[くるま]が
\\	書店	書店[しょてん]	しょてん	
\\	駅前に新しい書店ができました。	駅前[えきまえ]に 新[あたら]しい 書店[しょてん]ができました。	えきまえ に あたらしい しょてん が できました	
\\	駅前[えきまえ]に 新[あたら]しい
\\	ができました。			
\\	開店	開店[かいてん]	かいてん	
\\	デパートは10時に開店しますよ。	デパートは 10時[じゅうじ]に 開店[かいてん]しますよ。	でぱーと は じゅうじ に かいてん します よ	
\\	デパートは 10時[じゅうじ]に
\\	しますよ。			
\\	小売店	小売店[こうりてん]	こうりてん	
\\	この商品は小売店でも買えます。	この 商品[しょうひん]は 小売店[こうりてん]でも 買[か]えます。	この しょうひん は こうりてん で も かえます	
\\	この 商品[しょうひん]は
\\	でも 買[か]えます。			
\\	商社	商社[しょうしゃ]	しょうしゃ	
\\	兄は商社に勤めています。	兄[あに]は 商社[しょうしゃ]に 勤[つと]めています。	あに は しょうしゃ に つとめて います	
\\	兄[あに]は
\\	に 勤[つと]めています。			
\\	商店	商店[しょうてん]	しょうてん	
\\	この通りには商店が多いね。	この 通[とお]りには 商店[しょうてん]が 多[おお]いね。	この とおり に は しょうてん が おおい ね	
\\	この 通[とお]りには
\\	が 多[おお]いね。			
\\	商売	商売[しょうばい]	しょうばい	
\\	彼の商売は儲かっているな。	彼[かれ]の 商売[しょうばい]は 儲[もう]かっているな。	かれ の しょうばい は もうかって いる な	
\\	彼[かれ]の
\\	は 儲[もう]かっているな。			
\\	しばしば	しばしば	しばしば	
\\	課長はしばしば出張します。	課長[かちょう]はしばしば 出張[しゅっちょう]します。	かちょう は しばしば しゅっちょう します	
\\	課長[かちょう]は
\\	出張[しゅっちょう]します。			
\\	商人	商人[しょうにん]	しょうにん	
\\	商人は数字に強いね。	商人[しょうにん]は 数字[すうじ]に 強[つよ]いね。	しょうにん は すうじ に つよい ね	
\\	は 数字[すうじ]に 強[つよ]いね。			
\\	食品	食品[しょくひん]	しょくひん	
\\	食品は日曜日にまとめて買います。	食品[しょくひん]は 日曜日[にちようび]にまとめて 買[か]います。	しょくひん は にちようび に まとめて かいます	
\\	は 日曜日[にちようび]にまとめて 買[か]います。			
\\	品	品[しな]	しな	
\\	そちらの品は半額になっています。	そちらの 品[しな]は 半額[はんがく]になっています。	そちら の しな は はんがく に なって います	
\\	そちらの
\\	は 半額[はんがく]になっています。			
\\	上品	上品[じょうひん]	じょうひん	
\\	このドレスはとても上品なデザインですね。	このドレスはとても 上品[じょうひん]なデザインですね。	この どれす は とても じょうひん な でざいん です ね	
\\	このドレスはとても
\\	なデザインですね。			
\\	下品	下品[げひん]	げひん	
\\	彼女の下品なふるまいには我慢できない。	彼女[かのじょ]の 下品[げひん]なふるまいには 我慢[がまん]できない。	かのじょ の げひん な ふるまい に は がまん できない	
\\	彼女[かのじょ]の
\\	なふるまいには 我慢[がまん]できない。			
\\	手段	手段[しゅだん]	しゅだん	
\\	彼は目的のためには手段を選ばなかったわね。	彼[かれ]は 目的[もくてき]のためには 手段[しゅだん]を 選[えら]ばなかったわね。	かれ は もくてき の ため に は しゅだん を えらばなかった わ ね	
\\	彼[かれ]は 目的[もくてき]のためには
\\	を 選[えら]ばなかったわね。			
\\	一段と	一段[いちだん]と	いちだんと	
\\	物価は一段と上昇しそうだね。	物価[ぶっか]は 一段[いちだん]と 上昇[じょうしょう]しそうだね。	ぶっか は いちだんと じょうしょう しそう だ ね	
\\	物価[ぶっか]は
\\	上昇[じょうしょう]しそうだね。			
\\	ショック	ショック	ショック	
\\	彼女はショックで口もきけなかったわ。	彼女[かのじょ]はショックで 口[くち]もきけなかったわ。	かのじょ は しょっく で くち も きけなかった わ	
\\	彼女[かのじょ]は
\\	で 口[くち]もきけなかったわ。			
\\	値	値[あたい]	あたい	
\\	の値を求めなさい。	
\\	[えっくす]の 値[あたい]を 求[もと]めなさい。	えっくす の あたい を もとめなさい	
\\	[えっくす]の
\\	を 求[もと]めなさい。			
\\	値する	値[あたい]する	あたいする	
\\	その絵は一見に値しますよ。	その 絵[え]は 一見[いっけん]に 値[あたい]しますよ。	その え は いっけん に あたい します よ	
\\	その 絵[え]は 一見[いっけん]に
\\	よ。			
\\	価値	価値[かち]	かち	
\\	とても価値のある話を聞いたよ。	とても 価値[かち]のある 話[はなし]を 聞[き]いたよ。	とても かち の ある はなし を きいた よ	
\\	とても
\\	のある 話[はなし]を 聞[き]いたよ。			
\\	高価	高価[こうか]	こうか	
\\	彼女は高価な宝石を持っているわ。	彼女[かのじょ]は 高価[こうか]な 宝石[ほうせき]を 持[も]っているわ。	かのじょ は こうか な ほうせき を もって いる わ	
\\	彼女[かのじょ]は
\\	な 宝石[ほうせき]を 持[も]っているわ。			
\\	人格	人格[じんかく]	じんかく	
\\	私は彼の人格を尊敬しています。	私[わたし]は 彼[かれ]の 人格[じんかく]を 尊敬[そんけい]しています。	わたし は かれ の じんかく を そんけい して います	
\\	私[わたし]は 彼[かれ]の
\\	を 尊敬[そんけい]しています。			
\\	格好	格好[かっこう]	かっこう	
\\	格好のいい青年に会ったよ。	格好[かっこう]のいい 青年[せいねん]に 会[あ]ったよ。	かっこう の いい せいねん に あった よ	
\\	のいい 青年[せいねん]に 会[あ]ったよ。			
\\	青春	青春[せいしゅん]	せいしゅん	
\\	この曲を聞くと青春の頃を思い出します。	この 曲[きょく]を 聞[き]くと 青春[せいしゅん]の 頃[ころ]を 思[おも]い 出[だ]します。	この きょく を きく と せいしゅん の ころ を おもいだします	
\\	この 曲[きょく]を 聞[き]くと
\\	の 頃[ころ]を 思[おも]い 出[だ]します。			
\\	アンテナ	アンテナ	アンテナ	
\\	屋上にアンテナが立っていますね。	屋上[おくじょう]にアンテナが 立[た]っていますね。	おくじょう に あんてな が たって います ね	
\\	屋上[おくじょう]に
\\	が 立[た]っていますね。			
\\	春分	春分[しゅんぶん]	しゅんぶん	
\\	春分の日は毎年3月20日頃です。	春分[しゅんぶん]の 日[ひ]は 毎年3月20日頃[まいとし さんがつ はつかごろ]です。	しゅんぶん の ひ は まいとし さんがつ はつかごろ です	
\\	の 日[ひ]は 毎年3月20日頃[まいとし さんがつ はつかごろ]です。			
\\	初夏	初夏[しょか]	しょか	
\\	初夏の高原は気持ちがいいですね。	初夏[しょか]の 高原[こうげん]は 気持[きも]ちがいいですね。	しょか の こうげん は きもち が いい です ね	
\\	の 高原[こうげん]は 気持[きも]ちがいいですね。			
\\	秋分	秋分[しゅうぶん]	しゅうぶん	
\\	秋分の日は毎年9月23日頃です。	秋分[しゅうぶん]の 日[ひ]は 毎年9月23日頃[まいとし くがつ にじゅうさんにちごろ]です。	しゅうぶん の ひ は まいとし くがつ にじゅうさんにちごろ です	
\\	の 日[ひ]は 毎年9月23日頃[まいとし くがつ にじゅうさんにちごろ]です。			
\\	春夏秋冬	春夏秋冬[しゅんかしゅうとう]	しゅんかしゅうとう	
\\	春夏秋冬の移り変わりを見るのが大好きです。	春夏秋冬[しゅんかしゅうとう]の 移[うつ]り 変[か]わりを 見[み]るのが 大好[だいす]きです。	しゅんかしゅうとう の うつりかわり を みる の が だいすき です	
\\	の 移[うつ]り 変[か]わりを 見[み]るのが 大好[だいす]きです。			
\\	夏季	夏季[かき]	かき	
\\	夏季講習に申し込みした?	夏季[かき] 講習[こうしゅう]に 申[もう]し 込[こ]みした?	かき こうしゅう に もうしこみ した 
\\	講習[こうしゅう]に 申[もう]し 込[こ]みした?			
\\	寒気	寒気[さむけ]	さむけ	
\\	何だか寒気がします。	何[なん]だか 寒気[さむけ]がします。	なんだか さむけ が します	
\\	何[なん]だか
\\	がします。			
\\	暖か	暖[あたた]か	あたたか	
\\	最近は暖かです。	最近[さいきん]は 暖[あたた]かです。	さいきん は あたたか です	
\\	最近[さいきん]は
\\	です。			
\\	高温	高温[こうおん]	こうおん	
\\	金属は高温で溶かします。	金属[きんぞく]は 高温[こうおん]で 溶[と]かします。	きんぞく は こうおん で とかします	
\\	金属[きんぞく]は
\\	で 溶[と]かします。			
\\	あえて	あえて	あえて	
\\	彼はあえて危険を冒したの。	彼[かれ]はあえて 危険[きけん]を 冒[おか]したの。	かれ は あえて きけん を おかした	
\\	彼[かれ]は
\\	危険[きけん]を 冒[おか]したの。			
\\	温暖	温暖[おんだん]	おんだん	
\\	この地方は温暖で暮らしやすいな。	この 地方[ちほう]は 温暖[おんだん]で 暮[く]らしやすいな。	この ちほう は おんだん で くらし やすい な	
\\	この 地方[ちほう]は
\\	で 暮[く]らしやすいな。			
\\	北風	北風[きたかぜ]	きたかぜ	
\\	北風が冷たいです。	北風[きたかぜ]が 冷[つめ]たいです。	きたかぜ が つめたい です	
\\	が 冷[つめ]たいです。			
\\	秋風	秋風[あきかぜ]	あきかぜ	
\\	秋風が気持ちいいね。	秋風[あきかぜ]が 気持[きも]ちいいね。	あきかぜ が きもち いい ね	
\\	が 気持[きも]ちいいね。			
\\	情熱	情熱[じょうねつ]	じょうねつ	
\\	父は情熱を持って仕事に打ち込んでいます。	父[ちち]は 情熱[じょうねつ]を 持[も]って 仕事[しごと]に 打[う]ち 込[こ]んでいます。	ちち は じょうねつ を もって しごと に うちこんで います	
\\	父[ちち]は
\\	を 持[も]って 仕事[しごと]に 打[う]ち 込[こ]んでいます。			
\\	広告	広告[こうこく]	こうこく	
\\	その広告を新聞で見ました。	その 広告[こうこく]を 新聞[しんぶん]で 見[み]ました。	その こうこく を しんぶん で みました	
\\	その
\\	を 新聞[しんぶん]で 見[み]ました。			
\\	新た	新[あら]た	あらた	
\\	新たな計画が進んでいます。	新[あら]たな 計画[けいかく]が 進[すす]んでいます。	あらた な けいかく が すすんで います	
\\	な 計画[けいかく]が 進[すす]んでいます。			
\\	新聞社	新聞社[しんぶんしゃ]	しんぶんしゃ	
\\	このビルは新聞社です。	このビルは 新聞社[しんぶんしゃ]です。	この びる は しんぶんしゃ です	
\\	このビルは
\\	です。			
\\	カバー	カバー	カバー	
\\	本にカバーを掛けました。	本[ほん]にカバーを 掛[か]けました。	ほん に かばー を かけました	
\\	本[ほん]に
\\	を 掛[か]けました。			
\\	新人	新人[しんじん]	しんじん	
\\	彼は今日入ったばかりの新人です。	彼[かれ]は 今日入[きょう はい]ったばかりの 新人[しんじん]です。	かれ は きょう はいった ばかり の しんじん です	
\\	彼[かれ]は 今日入[きょう はい]ったばかりの
\\	です。			
\\	最悪	最悪[さいあく]	さいあく	
\\	何とか最悪の事態を避けることができました。	何[なん]とか 最悪[さいあく]の 事態[じたい]を 避[さ]けることができました。	なんとか さいあく の じたい を さける こと が できました	
\\	何[なん]とか
\\	の 事態[じたい]を 避[さ]けることができました。			
\\	悪用	悪用[あくよう]	あくよう	
\\	彼は地位を悪用しています。	彼[かれ]は 地位[ちい]を 悪用[あくよう]しています。	かれ は ちい を あくよう して います	
\\	彼[かれ]は 地位[ちい]を
\\	しています。			
\\	悪	悪[あく]	あく	
\\	彼は悪を憎んでいます。	彼[かれ]は 悪[あく]を 憎[にく]んでいます。	かれ は あく を にくんで います	
\\	彼[かれ]は
\\	を 憎[にく]んでいます。			
\\	悪女	悪女[あくじょ]	あくじょ	
\\	彼は悪女に騙されたんだ。	彼[かれ]は 悪女[あくじょ]に 騙[だま]されたんだ。	かれ は あくじょ に だまされた ん だ	
\\	彼[かれ]は
\\	に 騙[だま]されたんだ。			
\\	心理	心理[しんり]	しんり	
\\	顧客心理を理解することは重要です。	顧客[こきゃく] 心理[しんり]を 理解[りかい]することは 重要[じゅうよう]です。	こきゃく しんり を りかい する こと は じゅうよう です	
\\	顧客[こきゃく]
\\	を 理解[りかい]することは 重要[じゅうよう]です。			
\\	心	心[こころ]	こころ	
\\	彼は素直な心を持っている。	彼[かれ]は 素直[すなお]な 心[こころ]を 持[も]っている。	かれ は すなお な こころ を もって いる	
\\	彼[かれ]は 素直[すなお]な
\\	を 持[も]っている。			
\\	ジャーナリスト	ジャーナリスト	ジャーナリスト	
\\	彼女は有能なジャーナリストだ。	彼女[かのじょ]は 有能[ゆうのう]なジャーナリストだ。	かのじょ は ゆうのう な じゃーなりすと だ	
\\	彼女[かのじょ]は 有能[ゆうのう]な
\\	だ。			
\\	思い	思[おも]い	おもい	
\\	必死の思いで彼に頼んだよ。	必死[ひっし]の 思[おも]いで 彼[かれ]に 頼[たの]んだよ。	ひっし の おもい で かれ に たのんだ よ	
\\	必死[ひっし]の
\\	で 彼[かれ]に 頼[たの]んだよ。			
\\	思わず	思[おも]わず	おもわず	
\\	嬉しくて思わず涙が出ました。	嬉[うれ]しくて 思[おも]わず 涙[なみだ]が 出[で]ました。	うれしくて おもわず なみだ が でました	
\\	嬉[うれ]しくて
\\	涙[なみだ]が 出[で]ました。			
\\	思いがけない	思[おも]いがけない	おもいがけない	
\\	彼から思いがけないことを聞いた。	彼[かれ]から 思[おも]いがけないことを 聞[き]いた。	かれ から おもいがけない こと を きいた	
\\	彼[かれ]から
\\	ことを 聞[き]いた。			
\\	思いやり	思[おも]いやり	おもいやり	
\\	彼女の思いやりが嬉しかった。	彼女[かのじょ]の 思[おも]いやりが 嬉[うれ]しかった。	かのじょ の おもいやり が うれしかった	
\\	彼女[かのじょ]の
\\	が 嬉[うれ]しかった。			
\\	決して	決[けっ]して	けっして	
\\	このことを決して忘れないでください。	このことを 決[けっ]して 忘[わす]れないでください。	この こと を けっして わすれない で ください	
\\	このことを
\\	忘[わす]れないでください。			
\\	決心	決心[けっしん]	けっしん	
\\	今度こそタバコを止める決心をしました。	今度[こんど]こそタバコを 止[や]める 決心[けっしん]をしました。	こんど こそ たばこ を やめる けっしん を しました	
\\	今度[こんど]こそタバコを 止[や]める
\\	をしました。			
\\	決まり	決[き]まり	きまり	
\\	決まりを守ることは大切です。	決[き]まりを 守[まも]ることは 大切[たいせつ]です。	きまり を まもる こと は たいせつ です	
\\	を 守[まも]ることは 大切[たいせつ]です。			
\\	知り合う	知[し]り 合[あ]う	しりあう	
\\	お二人はどこで知り合ったのですか。	お 二人[ふたり]はどこで 知[し]り 合[あ]ったのですか。	おふたり は どこ で しりあった の です か	
\\	お 二人[ふたり]はどこで
\\	のですか。			
\\	スタイル	スタイル	スタイル	
\\	彼女はモデルのようにスタイルがいいね。	彼女[かのじょ]はモデルのようにスタイルがいいね。	かのじょ は もでる の よう に すたいる が いい ね	
\\	彼女[かのじょ]はモデルのように
\\	がいいね。			
\\	知れる	知[し]れる	しれる	
\\	他人に知れるとまずいことになるわね。	他人[たにん]に 知[し]れるとまずいことになるわね。	たにん に しれる と まずい こと に なる わ ね	
\\	他人[たにん]に
\\	とまずいことになるわね。			
\\	知り合い	知[し]り 合[あ]い	しりあい	
\\	街で知り合いを見かけたよ。	街[まち]で 知[し]り 合[あ]いを 見[み]かけたよ。	まち で しりあい を みかけた よ	
\\	街[まち]で
\\	を 見[み]かけたよ。			
\\	知らせ	知[し]らせ	しらせ	
\\	今日、合格の知らせをもらいました。	今日[きょう]、 合格[ごうかく]の 知[し]らせをもらいました。	きょう ごうかく の しらせ を もらいました	
\\	今日[きょう]、 合格[ごうかく]の
\\	をもらいました。			
\\	知らず知らず	知[し]らず 知[し]らず	しらずしらず	
\\	知らず知らずのうちに疲れがたまっていたよ。	知[し]らず 知[し]らずのうちに 疲[つか]れがたまっていたよ。	しらずしらず の うち に つかれ が たまって いた よ	
\\	のうちに 疲[つか]れがたまっていたよ。			
\\	才能	才能[さいのう]	さいのう	
\\	彼は芸術的な才能にあふれているね。	彼[かれ]は 芸術的[げいじゅつてき]な 才能[さいのう]にあふれているね。	かれ は げいじゅつてき な さいのう に あふれて いる ね	
\\	彼[かれ]は 芸術的[げいじゅつてき]な
\\	にあふれているね。			
\\	小便	小便[しょうべん]	しょうべん	
\\	ちょっと小便しに行って来る。	ちょっと 小便[しょうべん]しに 行[い]って 来[く]る。	ちょっと しょうべん し に いって くる	
\\	ちょっと
\\	しに 行[い]って 来[く]る。			
\\	局	局[きょく]	きょく	
\\	彼女はラジオ局で働いています。	彼女[かのじょ]はラジオ 局[きょく]で 働[はたら]いています。	かのじょ は らじおきょく で はたらいて います	
\\	彼女[かのじょ]はラジオ
\\	で 働[はたら]いています。			
\\	インタビュー	インタビュー	インタビュー	
\\	彼はインタビューに、はきはきと答えてたよ。	彼[かれ]はインタビューに、はきはきと 答[こた]えてたよ。	かれ は いんたびゅー に はきはき と こたえて た よ	
\\	彼[かれ]は
\\	に、はきはきと 答[こた]えてたよ。			
\\	住まい	住[す]まい	すまい	
\\	私の住まいは東京にあります。	私[わたし]の 住[す]まいは 東京[とうきょう]にあります。	わたし の すまい は とうきょう に あります	
\\	私[わたし]の
\\	は 東京[とうきょう]にあります。			
\\	氏	氏[し]	し	
\\	会長は田中氏に決定。	会長[かいちょう]は 田中[たなか] 氏[し]に 決定。[けってい]	かいちょう は たなかし に けってい	
\\	会長[かいちょう]は 田中[たなか]
\\	に 決定。[けってい]			
\\	人名	人名[じんめい]	じんめい	
\\	これは日本の人名ですか。	これは 日本[にほん]の 人名[じんめい]ですか。	これ は にほん の じんめい です か	
\\	これは 日本[にほん]の
\\	ですか。			
\\	国名	国名[こくめい]	こくめい	
\\	アジアの国名をいくつ知っていますか。	アジアの 国名[こくめい]をいくつ 知[し]っていますか。	あじあ の こくめい を いくつ しって います か	
\\	アジアの
\\	をいくつ 知[し]っていますか。			
\\	各地	各地[かくち]	かくち	
\\	各地で大雨が降っています。	各地[かくち]で 大雨[おおあめ]が 降[ふ]っています。	かくち で おおあめ が ふって います	
\\	で 大雨[おおあめ]が 降[ふ]っています。			
\\	県	県[けん]	けん	
\\	県の代表は2名です。	県[けん]の 代表[だいひょう]は 2名[にめい]です。	けん の だいひょう は にめい です	
\\	の 代表[だいひょう]は 2名[にめい]です。			
\\	市内	市内[しない]	しない	
\\	明日は市内を観光する予定です。	明日[あした]は 市内[しない]を 観光[かんこう]する 予定[よてい]です。	あした は しない を かんこう する よてい です 。	
\\	明日[あした]は
\\	を 観光[かんこう]する 予定[よてい]です。			
\\	シーズン	シーズン	シーズン	
\\	その選手は今シーズンも好調だね。	その 選手[せんしゅ]は 今[こん]シーズンも 好調[こうちょう]だね。	その せんしゅ は こん しーずん も こうちょう だ ね 。	
\\	その 選手[せんしゅ]は 今[こん]
\\	も 好調[こうちょう]だね。			
\\	市長	市長[しちょう]	しちょう	
\\	新しい市長が選ばれました。	新[あたら]しい 市長[しちょう]が 選[えら]ばれました。	あたらしい しちょう が えらばれました	
\\	新[あたら]しい
\\	が 選[えら]ばれました。			
\\	市場	市場[いちば]	いちば	
\\	市場で新鮮な魚を買ってきました。	市場[いちば]で 新鮮[しんせん]な 魚[さかな]を 買[か]ってきました。	いちば で しんせん な さかな を かって きました	
\\	で 新鮮[しんせん]な 魚[さかな]を 買[か]ってきました。			
\\	市場	市場[しじょう]	しじょう	
\\	デジカメ市場は急速に拡大している。	デジカメ 市場[しじょう]は 急速[きゅうそく]に 拡大[かくだい]している。	でじかめ しじょう は きゅうそく に かくだい して いる	
\\	デジカメ
\\	は 急速[きゅうそく]に 拡大[かくだい]している。			
\\	市外	市外[しがい]	しがい	
\\	祖父は市外の病院に通っているの。	祖父[そふ]は 市外[しがい]の 病院[びょういん]に 通[かよ]っているの。	そふ は しがい の びょういん に かよって いる の	
\\	祖父[そふ]は
\\	の 病院[びょういん]に 通[かよ]っているの。			
\\	市	市[し]	し	
\\	その市の人口は減り続けているの。	その 市[し]の 人口[じんこう]は 減[へ]り 続[つづ]けているの。	その し の じんこう は へりつづけて いる の	
\\	その
\\	の 人口[じんこう]は 減[へ]り 続[つづ]けているの。			
\\	下町	下町[したまち]	したまち	
\\	あの子は下町育ちだ。	あの 子[こ]は 下町[したまち] 育[そだ]ちだ。	あの こ は したまち そだち だ	
\\	あの 子[こ]は
\\	育[そだ]ちだ。			
\\	区分	区分[くぶん]	くぶん	
\\	この表は年齢区分ごとの人口を表しています。	この 表[ひょう]は 年齢[ねんれい] 区分[くぶん]ごとの 人口[じんこう]を 表[あらわ]しています。	この ひょう は ねんれい くぶん ごと の じんこう を あらわして います	
\\	この 表[ひょう]は 年齢[ねんれい]
\\	ごとの 人口[じんこう]を 表[あらわ]しています。			
\\	区	区[く]	く	
\\	東京には23の区がある。	東京[とうきょう]には23の 区[く]がある。	とうきょう に は 
\\	の く が ある 。	
\\	東京[とうきょう]には23の
\\	がある。			
\\	アイデア	アイデア	アイデア	
\\	彼がいいアイデアを出したね。	彼[かれ]がいいアイデアを 出[だ]したね。	かれ が いい あいであ を だした ね	
\\	彼[かれ]がいい
\\	を 出[だ]したね。			
\\	様々	様々[さまざま]	さまざま	
\\	その都市には様々な人種が集まっているわ。	その 都市[とし]には 様々[さまざま]な 人種[じんしゅ]が 集[あつ]まっているわ。	その とし に は さまざま な じんしゅ が あつまって いる わ	
\\	その 都市[とし]には
\\	な 人種[じんしゅ]が 集[あつ]まっているわ。			
\\	出荷	出荷[しゅっか]	しゅっか	
\\	ご注文の品は明日出荷致します。	ご 注文[ちゅうもん]の 品[しな]は 明日[あす] 出荷[しゅっか] 致[いた]します。	ごちゅうもん の しな は あす しゅっか いたします	
\\	ご 注文[ちゅうもん]の 品[しな]は 明日[あす]
\\	致[いた]します。			
\\	人物	人物[じんぶつ]	じんぶつ	
\\	彼は会社の重要な人物です。	彼[かれ]は 会社[かいしゃ]の 重要[じゅうよう]な 人物[じんぶつ]です。	かれ は かいしゃ の じゅうよう な じんぶつ です	
\\	彼[かれ]は 会社[かいしゃ]の 重要[じゅうよう]な
\\	です。			
\\	見物人	見物人[けんぶつにん]	けんぶつにん	
\\	見物人が大勢集まっているね。	見物人[けんぶつにん]が 大勢集[おおぜい あつ]まっているね。	けんぶつにん が おおぜい あつまって いる ね	
\\	が 大勢集[おおぜい あつ]まっているね。			
\\	入れ物	入[い]れ 物[もの]	いれもの	
\\	荷物が多いので大きな入れ物が必要です。	荷物[にもつ]が 多[おお]いので 大[おお]きな 入[い]れ 物[もの]が 必要[ひつよう]です。	にもつ が おおい の で おおき な いれもの が ひつよう です	
\\	荷物[にもつ]が 多[おお]いので 大[おお]きな
\\	が 必要[ひつよう]です。			
\\	作物	作物[さくもつ]	さくもつ	
\\	米はアジアでは大切な作物だ。	米[こめ]はアジアでは 大切[たいせつ]な 作物[さくもつ]だ。	こめ は あじあ で は たいせつ な さくもつ だ	
\\	米[こめ]はアジアでは 大切[たいせつ]な
\\	だ。			
\\	食物	食物[しょくもつ]	しょくもつ	
\\	人間にとって水は食物より大切なの。	人間[にんげん]にとって 水[みず]は 食物[しょくもつ]より 大切[たいせつ]なの。	にんげん に とって みず は しょくもつ より たいせつ なの	
\\	人間[にんげん]にとって 水[みず]は
\\	より 大切[たいせつ]なの。			
\\	いかにも	いかにも	いかにも	
\\	彼はいかにもスポーツマンらしいですね。	彼[かれ]はいかにもスポーツマンらしいですね。	かれ は いかにも すぽーつまん らしい です ね	
\\	彼[かれ]は
\\	スポーツマンらしいですね。			
\\	書物	書物[しょもつ]	しょもつ	
\\	彼は書物に囲まれて生活しているの。	彼[かれ]は 書物[しょもつ]に 囲[かこ]まれて 生活[せいかつ]しているの。	かれ は しょもつ に かこまれて せいかつ して いる の	
\\	彼[かれ]は
\\	に 囲[かこ]まれて 生活[せいかつ]しているの。			
\\	重ねる	重[かさ]ねる	かさねる	
\\	荷物はここに重ねてください。	荷物[にもつ]はここに 重[かさ]ねてください。	にもつ は ここ に かさねて ください	
\\	荷物[にもつ]はここに
\\	ください。			
\\	重なる	重[かさ]なる	かさなる	
\\	高速道路で事故が重なった。	高速道路[こうそく どうろ]で 事故[じこ]が 重[かさ]なった。	こうそく どうろ で じこ が かさなった	
\\	高速道路[こうそく どうろ]で 事故[じこ]が
\\	重大	重大[じゅうだい]	じゅうだい	
\\	重大な発表があります。	重大[じゅうだい]な 発表[はっぴょう]があります。	じゅうだい な はっぴょう が あります	
\\	な 発表[はっぴょう]があります。			
\\	重み	重[おも]み	おもみ	
\\	雪の重みで枝が折れそうだ。	雪[ゆき]の 重[おも]みで 枝[えだ]が 折[お]れそうだ。	ゆき の おもみ で えだ が おれ そう だ	
\\	雪[ゆき]の
\\	で 枝[えだ]が 折[お]れそうだ。			
\\	重たい	重[おも]たい	おもたい	
\\	この鞄は重たいです。	この 鞄[かばん]は 重[おも]たいです。	この かばん は おもたい です	
\\	この 鞄[かばん]は
\\	です。			
\\	気軽	気軽[きがる]	きがる	
\\	いつでも気軽に遊びに来て下さい。	いつでも 気軽[きがる]に 遊[あそ]びに 来[き]て 下[くだ]さい。	いつでも きがる に あそび に きて ください	
\\	いつでも
\\	に 遊[あそ]びに 来[き]て 下[くだ]さい。			
\\	いきなり	いきなり	いきなり	
\\	後ろからいきなり肩をたたかれた。	後[うし]ろからいきなり 肩[かた]をたたかれた。	うしろ から いきなり かた を たたかれた	
\\	後[うし]ろから
\\	肩[かた]をたたかれた。			
\\	重量	重量[じゅうりょう]	じゅうりょう	
\\	この荷物はかなりの重量ですね。	この 荷物[にもつ]はかなりの 重量[じゅうりょう]ですね。	この にもつ は かなり の じゅうりょう です ね	
\\	この 荷物[にもつ]はかなりの
\\	ですね。			
\\	少量	少量[しょうりょう]	しょうりょう	
\\	泡立てたクリームに少量のブランデーを加えます。	泡立[あわだ]てたクリームに 少量[しょうりょう]のブランデーを 加[くわ]えます。	あわだてた くりーむ に しょうりょう の ぶらんでー を くわえます	
\\	泡立[あわだ]てたクリームに
\\	のブランデーを 加[くわ]えます。			
\\	小量	小量[しょうりょう]	しょうりょう	
\\	私はコーヒー豆を小量で買うようにしています。	私[わたし]はコーヒー 豆[まめ]を 小量[しょうりょう]で 買[か]うようにしています。	わたし は こーひー まめ を しょうりょう で かうよう に しています 。	
\\	私[わたし]はコーヒー 豆[まめ]を
\\	で 買[か]うようにしています。			
\\	受け入れる	受[う]け 入[い]れる	うけいれる	
\\	私は彼の意見を受け入れました。	私[わたし]は 彼[かれ]の 意見[いけん]を 受[う]け 入[い]れました。	わたし は かれ の いけん を うけいれました	
\\	私[わたし]は 彼[かれ]の 意見[いけん]を
\\	受け止める	受[う]け 止[と]める	うけとめる	
\\	ボールが速過ぎて受け止められなかったの。	ボールが 速過[はや す]ぎて 受[う]け 止[と]められなかったの。	ぼーる が はや すぎて うけとめられ なかった の	
\\	ボールが 速過[はや す]ぎて
\\	の。			
\\	受かる	受[う]かる	うかる	
\\	第一志望の大学に受かりました。	第一志望[だいいち しぼう]の 大学[だいがく]に 受[う]かりました。	だいいち しぼう の だいがく に うかりました	
\\	第一志望[だいいち しぼう]の 大学[だいがく]に
\\	受け取り	受[う]け 取[と]り	うけとり	
\\	受け取りに判子をお願いします。	受[う]け 取[と]りに 判子[はんこ]をお 願[ねが]いします。	うけとり に はんこ を おねがい します	
\\	に 判子[はんこ]をお 願[ねが]いします。			
\\	いつまでも	いつまでも	いつまでも	
\\	いつまでもあなたを忘れません。	いつまでもあなたを 忘[わす]れません。	いつまでも あなた を わすれません	
\\	あなたを 忘[わす]れません。			
\\	聞き取り	聞[き]き 取[と]り	ききとり	
\\	英語の聞き取り試験を受けたよ。	英語[えいご]の 聞[き]き 取[と]り 試験[しけん]を 受[う]けたよ。	えいご の ききとり しけん を うけた よ	
\\	英語[えいご]の
\\	試験[しけん]を 受[う]けたよ。			
\\	書き取り	書[か]き 取[と]り	かきとり	
\\	僕たちは毎朝漢字の書き取りをします。	僕[ぼく]たちは 毎朝漢字[まいあさ かんじ]の 書[か]き 取[と]りをします。	ぼくたち は まいあさ かんじ の かきとり を します	
\\	僕[ぼく]たちは 毎朝漢字[まいあさ かんじ]の
\\	をします。			
\\	受け持つ	受[う]け 持[も]つ	うけもつ	
\\	1年生を受け持っています。	1年生[いちねんせい]を 受[う]け 持[も]っています。	いちねんせい を うけもって います	
\\	1年生[いちねんせい]を
\\	打ち上げる	打[う]ち 上[あ]げる	うちあげる	
\\	夏祭りで花火を打ち上げます。	夏祭[なつまつ]りで 花火[はなび]を 打[う]ち 上[あ]げます。	なつまつり で はなび を うちあげます	
\\	夏祭[なつまつ]りで 花火[はなび]を
\\	打ち合わせ	打[う]ち 合[あ]わせ	うちあわせ	
\\	午後に打ち合わせをしましょう。	午後[ごご]に 打[う]ち 合[あ]わせをしましょう。	ごご に うちあわせ を しましょう	
\\	午後[ごご]に
\\	をしましょう。			
\\	打ち明ける	打[う]ち 明[あ]ける	うちあける	
\\	親友に悩みを打ち明けたの。	親友[しんゆう]に 悩[なや]みを 打[う]ち 明[あ]けたの。	しんゆう に なやみ を うちあけた の	
\\	親友[しんゆう]に 悩[なや]みを
\\	の。			
\\	打ち合わせる	打[う]ち 合[あ]わせる	うちあわせる	
\\	来週の予定を打ち合わせましょう。	来週[らいしゅう]の 予定[よてい]を 打[う]ち 合[あ]わせましょう。	らいしゅう の よてい を うちあわせましょう	
\\	来週[らいしゅう]の 予定[よてい]を
\\	打ち消し	打[う]ち 消[け]し	うちけし	
\\	彼はうわさを打ち消したわよ。	彼[かれ]はうわさを 打[う]ち 消[け]したわよ。	かれ は うわさ を うちけした わ よ	
\\	彼[かれ]はうわさを
\\	たわよ。			
\\	しっかり	しっかり	しっかり	
\\	彼は若いのにしっかりしてるね。	彼[かれ]は 若[わか]いのにしっかりしてるね。	かれ は わかい の に しっかり してる ね	
\\	彼[かれ]は 若[わか]いのに
\\	してるね。			
\\	市役所	市役所[しやくしょ]	しやくしょ	
\\	市役所で書類をもらって来たの。	市役所[しやくしょ]で 書類[しょるい]をもらって 来[き]たの。	しやくしょ で しょるい を もらって きた の	
\\	で 書類[しょるい]をもらって 来[き]たの。			
\\	区役所	区役所[くやくしょ]	くやくしょ	
\\	区役所に書類を届けてください。	区役所[くやくしょ]に 書類[しょるい]を 届[とど]けてください。	くやくしょ に しょるい を とどけて ください	
\\	に 書類[しょるい]を 届[とど]けてください。			
\\	重役	重役[じゅうやく]	じゅうやく	
\\	明日、重役会議が開かれます。	明日[あす]、 重役[じゅうやく] 会議[かいぎ]が 開[ひら]かれます。	あす じゅうやくかいぎ が ひらかれます	
\\	明日[あす]、
\\	会議[かいぎ]が 開[ひら]かれます。			
\\	生じる	生[しょう]じる	しょうじる	
\\	両者の間に摩擦が生じています。	両者[りょうしゃ]の 間[あいだ]に 摩擦[まさつ]が 生[しょう]じています。	りょうしゃ の あいだ に まさつ が しょうじて います	
\\	両者[りょうしゃ]の 間[あいだ]に 摩擦[まさつ]が
\\	人生	人生[じんせい]	じんせい	
\\	彼は自分の人生を振り返ったわ。	彼[かれ]は 自分[じぶん]の 人生[じんせい]を 振[ふ]り 返[かえ]ったわ。	かれ は じぶん の じんせい を ふりかえった わ	
\\	彼[かれ]は 自分[じぶん]の
\\	を 振[ふ]り 返[かえ]ったわ。			
\\	生まれ	生[う]まれ	うまれ	
\\	彼女は京都の生まれです。	彼女[かのじょ]は 京都[きょうと]の 生[う]まれです。	かのじょ は きょうと の うまれ です	
\\	彼女[かのじょ]は 京都[きょうと]の
\\	です。			
\\	一生	一生[いっしょう]	いっしょう	
\\	一生のお願いがあります。	一生[いっしょう]のお 願[ねが]いがあります。	いっしょう の おねがい が あります	
\\	のお 願[ねが]いがあります。			
\\	アンケート	アンケート	アンケート	
\\	彼女はアンケートに答えたよ。	彼女[かのじょ]はアンケートに 答[こた]えたよ。	かのじょ は あんけーと に こたえた よ	
\\	彼女[かのじょ]は
\\	に 答[こた]えたよ。			
\\	生み出す	生[う]み 出[だ]す	うみだす	
\\	彼は数々の名作を生み出した。	彼[かれ]は 数々[かずかず]の 名作[めいさく]を 生[う]み 出[だ]した。	かれ は かずかず の めいさく を うみだした	
\\	彼[かれ]は 数々[かずかず]の 名作[めいさく]を
\\	生き方	生[い]き 方[かた]	いきかた	
\\	自分らしい生き方をしなさい。	自分[じぶん]らしい 生[い]き 方[かた]をしなさい。	じぶん らしい いきかた を しなさい	
\\	自分[じぶん]らしい
\\	をしなさい。			
\\	生かす	生[い]かす	いかす	
\\	彼女は語学力を仕事に生かしているね。	彼女[かのじょ]は 語学力[ごがくりょく]を 仕事[しごと]に 生[い]かしているね。	かのじょ は ごがくりょく を しごと に いかして いる ね	
\\	彼女[かのじょ]は 語学力[ごがくりょく]を 仕事[しごと]に
\\	ね。			
\\	生き物	生[い]き 物[もの]	いきもの	
\\	生き物を大切にしましょう。	生[い]き 物[もの]を 大切[たいせつ]にしましょう。	いきもの を たいせつ に しましょう	
\\	を 大切[たいせつ]にしましょう。			
\\	生け花	生[い]け 花[ばな]	いけばな	
\\	彼女は生け花の先生です。	彼女[かのじょ]は 生[い]け 花[ばな]の 先生[せんせい]です。	かのじょ は いけばな の せんせい です	
\\	彼女[かのじょ]は
\\	の 先生[せんせい]です。			
\\	生年月日	生年月日[せいねんがっぴ]	せいねんがっぴ	
\\	ここに生年月日を記入してください。	ここに 生年[せいねん] 月日[がっぴ]を 記入[きにゅう]してください。	ここ に せいねんがっぴ を きにゅう して ください	
\\	ここに
\\	を 記入[きにゅう]してください。			
\\	生理	生理[せいり]	せいり	
\\	昨日、生理が始まったの。	昨日[きのう]、 生理[せいり]が 始[はじ]まったの。	きのう せいり が はじまった の	
\\	昨日[きのう]、
\\	が 始[はじ]まったの。			
\\	スケジュール	スケジュール	スケジュール	
\\	スケジュールの調整は君に任せるわ。	スケジュールの 調整[ちょうせい]は 君[きみ]に 任[まか]せるわ。	すけじゅーる の ちょうせい は きみ に まかせる わ	
\\	の 調整[ちょうせい]は 君[きみ]に 任[まか]せるわ。			
\\	生まれつき	生[う]まれつき	うまれつき	
\\	彼女は生まれつき丈夫です。	彼女[かのじょ]は 生[う]まれつき 丈夫[じょうぶ]です。	かのじょ は うまれつき じょうぶ です	
\\	彼女[かのじょ]は
\\	丈夫[じょうぶ]です。			
\\	生	生[せい]	せい	
\\	私がこの世に生を受けて80年が過ぎたわ。	私[わたし]がこの 世[よ]に 生[せい]を 受[う]けて 80年[はちじゅうねん]が 過[す]ぎたわ。	わたし が このよ に せい を うけて はちじゅうねん が すぎた わ	
\\	私[わたし]がこの 世[よ]に
\\	を 受[う]けて 80年[はちじゅうねん]が 過[す]ぎたわ。			
\\	生物	生物[せいぶつ]	せいぶつ	
\\	海の底には不思議な生物がたくさんいるよ。	海[うみ]の 底[そこ]には 不思議[ふしぎ]な 生物[せいぶつ]がたくさんいるよ。	うみ の そこ に は ふしぎ な せいぶつ が たくさん いる よ	
\\	海[うみ]の 底[そこ]には 不思議[ふしぎ]な
\\	がたくさんいるよ。			
\\	生き生きと	生[い]き 生[い]きと	いきいきと	
\\	子供たちが生き生きと遊んでいるね。	子供[こども]たちが 生[い]き 生[い]きと 遊[あそ]んでいるね。	こどもたち が いきいきと あそんで いる ね	
\\	子供[こども]たちが
\\	遊[あそ]んでいるね。			
\\	性能	性能[せいのう]	せいのう	
\\	今度のパソコンは性能がすごく良い。	今度[こんど]のパソコンは 性能[せいのう]がすごく 良[い]い。	こんど の ぱそこん は せいのう が すごく いい	
\\	今度[こんど]のパソコンは
\\	がすごく 良[い]い。			
\\	性格	性格[せいかく]	せいかく	
\\	僕と姉の性格は正反対です。	僕[ぼく]と 姉[あね]の 性格[せいかく]は 正反対[せいはんたい]です。	ぼく と あね の せいかく は せいはんたい です	
\\	僕[ぼく]と 姉[あね]の
\\	は 正反対[せいはんたい]です。			
\\	性	性[せい]	せい	
\\	この会社では性による差別はありません。	この 会社[かいしゃ]では 性[せい]による 差別[さべつ]はありません。	この かいしゃ で は せい に よる さべつ は ありません	
\\	この 会社[かいしゃ]では
\\	による 差別[さべつ]はありません。			
\\	国産	国産[こくさん]	こくさん	
\\	このワインは国産です。	このワインは 国産[こくさん]です。	この わいん は こくさん です	
\\	このワインは
\\	です。			
\\	ガソリン	ガソリン	ガソリン	
\\	車にガソリンを入れました。	車[くるま]にガソリンを 入[い]れました。	くるま に がそりん を いれました	
\\	車[くるま]に
\\	を 入[い]れました。			
\\	産地	産地[さんち]	さんち	
\\	その地方はお茶の産地です。	その 地方[ちほう]はお 茶[ちゃ]の 産地[さんち]です。	その ちほう は おちゃ の さんち です	
\\	その 地方[ちほう]はお 茶[ちゃ]の
\\	です。			
\\	出産	出産[しゅっさん]	しゅっさん	
\\	彼女は女の子を出産したんだ。	彼女[かのじょ]は 女[おんな]の 子[こ]を 出産[しゅっさん]したんだ。	かのじょ は おんな の こ を しゅっさん した ん だ	
\\	彼女[かのじょ]は 女[おんな]の 子[こ]を
\\	したんだ。			
\\	活用	活用[かつよう]	かつよう	
\\	彼女はインターネットを活用しているの。	彼女[かのじょ]はインターネットを 活用[かつよう]しているの。	かのじょ は いんたーねっと を かつよう して いる の	
\\	彼女[かのじょ]はインターネットを
\\	しているの。			
\\	活字	活字[かつじ]	かつじ	
\\	新聞の活字が読みやすくなったね。	新聞[しんぶん]の 活字[かつじ]が 読[よ]みやすくなったね。	しんぶん の かつじ が よみ やすく なった ね	
\\	新聞[しんぶん]の
\\	が 読[よ]みやすくなったね。			
\\	学会	学会[がっかい]	がっかい	
\\	彼は学会で論文を発表したよ。	彼[かれ]は 学会[がっかい]で 論文[ろんぶん]を 発表[はっぴょう]したよ。	かれ は がっかい で ろんぶん を はっぴょう した よ	
\\	彼[かれ]は
\\	で 論文[ろんぶん]を 発表[はっぴょう]したよ。			
\\	学年	学年[がくねん]	がくねん	
\\	彼は私より一学年上です。	彼[かれ]は 私[わたし]より 一[ひと] 学年[がくねん] 上[うえ]です。	かれ は わたし より ひと がくねん うえ です	
\\	彼[かれ]は 私[わたし]より 一[ひと]
\\	上[うえ]です。			
\\	工学	工学[こうがく]	こうがく	
\\	彼は大学で工学を勉強しました。	彼[かれ]は 大学[だいがく]で 工学[こうがく]を 勉強[べんきょう]しました。	かれ は だいがく で こうがく を べんきょう しました	
\\	彼[かれ]は 大学[だいがく]で
\\	を 勉強[べんきょう]しました。			
\\	カット	カット	かっと	
\\	市長がテープをカットした。	市長[しちょう]がテープをカットした。	しちょう が てーぷ を かっと した	
\\	市長[しちょう]がテープを
\\	した。			
\\	学長	学長[がくちょう]	がくちょう	
\\	入学式で学長の挨拶がありました。	入学式[にゅうがくしき]で 学長[がくちょう]の 挨拶[あいさつ]がありました。	にゅうがくしき で がくちょう の あいさつ が ありました	
\\	入学式[にゅうがくしき]で
\\	の 挨拶[あいさつ]がありました。			
\\	語学	語学[ごがく]	ごがく	
\\	海外で語学の勉強をします。	海外[かいがい]で 語学[ごがく]の 勉強[べんきょう]をします。	かいがい で ごがく の べんきょう を します	
\\	海外[かいがい]で
\\	の 勉強[べんきょう]をします。			
\\	工学部	工学部[こうがくぶ]	こうがくぶ	
\\	彼は工学部の教授です。	彼[かれ]は 工学部[こうがくぶ]の 教授[きょうじゅ]です。	かれ は こうがくぶ の きょうじゅ です	
\\	彼[かれ]は
\\	の 教授[きょうじゅ]です。			
\\	学費	学費[がくひ]	がくひ	
\\	彼はアルバイトをして学費を稼いだんだ。	彼[かれ]はアルバイトをして 学費[がくひ]を 稼[かせ]いだんだ。	かれ は あるばいと を して がくひ を かせいだ ん だ 。	
\\	彼[かれ]はアルバイトをして
\\	を 稼[かせ]いだんだ。			
\\	学部	学部[がくぶ]	がくぶ	
\\	彼は経済学部の学生です。	彼[かれ]は 経済[けいざい] 学部[がくぶ]の 学生[がくせい]です。	かれ は けいざい がくぶ の がくせい です	
\\	彼[かれ]は 経済[けいざい]
\\	の 学生[がくせい]です。			
\\	学力	学力[がくりょく]	がくりょく	
\\	学力を付けてその大学に進みたい。	学力[がくりょく]を 付[つ]けてその 大学[だいがく]に 進[すす]みたい。	がくりょく を つけて その だいがく に すすみたい	
\\	を 付[つ]けてその 大学[だいがく]に 進[すす]みたい。			
\\	休学	休学[きゅうがく]	きゅうがく	
\\	1年休学することにしました。	1年[いちねん] 休学[きゅうがく]することにしました。	いちねん きゅうがく する こと に しました	
\\	1年[いちねん]
\\	することにしました。			
\\	スター	スター	スター	
\\	彼はその時スターだったよ。	彼[かれ]はその 時[とき]スターだったよ。	かれ は その とき すたー だった よ 。	
\\	彼[かれ]はその 時[とき]
\\	だったよ。			
\\	教員	教員[きょういん]	きょういん	
\\	彼は高校の教員です。	彼[かれ]は 高校[こうこう]の 教員[きょういん]です。	かれ は こうこう の きょういん です	
\\	彼[かれ]は 高校[こうこう]の
\\	です。			
\\	キリスト教	キリスト 教[きょう]	きりすときょう	
\\	この先にキリスト教の教会があります。	この 先[さき]にキリスト 教[きょう]の 教会[きょうかい]があります。	この さき に きりすときょう の きょうかい が あります	
\\	この 先[さき]に
\\	の 教会[きょうかい]があります。			
\\	イスラム教	イスラム 教[きょう]	イスラムきょう	
\\	これはイスラム教の寺院です。	これはイスラム 教[きょう]の 寺院[じいん]です。	これ は いすらむきょう の じいん です	
\\	これは
\\	の 寺院[じいん]です。			
\\	教わる	教[おそ]わる	おそわる	
\\	私は両親から多くを教わりました。	私[わたし]は 両親[りょうしん]から 多[おお]くを 教[おそ]わりました。	わたし は りょうしん から おおく を おそわりました	
\\	私[わたし]は 両親[りょうしん]から 多[おお]くを
\\	教え	教[おし]え	おしえ	
\\	父の教えは「自分に厳しく」です。	父[ちち]の 教[おし]えは
\\	自分[じぶん]に 厳[きび]しく」です。	ちち の おしえ は じぶん に きびしく です	
\\	父[ちち]の
\\	は
\\	自分[じぶん]に 厳[きび]しく」です。			
\\	制御	制御[せいぎょ]	せいぎょ	
\\	ここのパネルで機械全体を制御できます。	ここのパネルで 機械全体[きかい ぜんたい]を 制御[せいぎょ]できます。	ここ の ぱねる で きかい ぜんたい を せいぎょ できます	
\\	ここのパネルで 機械全体[きかい ぜんたい]を
\\	できます。			
\\	強力	強力[きょうりょく]	きょうりょく	
\\	これは強力な接着剤ね。	これは 強力[きょうりょく]な 接着剤[せっちゃくざい]ね。	これ は きょうりょく な せっちゃくざい ね	
\\	これは
\\	な 接着剤[せっちゃくざい]ね。			
\\	強制	強制[きょうせい]	きょうせい	
\\	彼らは労働を強制されたんだ。	彼[かれ]らは 労働[ろうどう]を 強制[きょうせい]されたんだ。	かれら は ろうどう を きょうせい された ん だ	
\\	彼[かれ]らは 労働[ろうどう]を
\\	されたんだ。			
\\	カメラマン	カメラマン	カメラマン	
\\	将来はプロのカメラマンになりたいです。	将来[しょうらい]はプロのカメラマンになりたいです。	しょうらい は ぷろ の かめらまん に なりたい です	
\\	将来[しょうらい]はプロの
\\	になりたいです。			
\\	最強	最強[さいきょう]	さいきょう	
\\	彼は最強チームの一員です。	彼[かれ]は 最強[さいきょう]チームの 一員[いちいん]です。	かれ は さいきょう ちーむ の いちいん です	
\\	彼[かれ]は
\\	チームの 一員[いちいん]です。			
\\	心強い	心強[こころづよ]い	こころづよい	
\\	あなたが一緒にいてくれると心強い。	あなたが 一緒[いっしょ]にいてくれると 心強[こころづよ]い。	あなた が いっしょ に いて くれる と こころづよい	
\\	あなたが 一緒[いっしょ]にいてくれると
\\	強引	強引[ごういん]	ごういん	
\\	友人の強引な誘いを断れませんでした。	友人[ゆうじん]の 強引[ごういん]な 誘[さそ]いを 断[ことわ]れませんでした。	ゆうじん の ごういん な さそい を ことわれません でした	
\\	友人[ゆうじん]の
\\	な 誘[さそ]いを 断[ことわ]れませんでした。			
\\	引用	引用[いんよう]	いんよう	
\\	論文にその本を引用したの。	論文[ろんぶん]にその 本[ほん]を 引用[いんよう]したの。	ろんぶん に その ほん を いんよう した の	
\\	論文[ろんぶん]にその 本[ほん]を
\\	したの。			
\\	字引	字引[じびき]	じびき	
\\	この漢字を字引で引いてみて。	この 漢字[かんじ]を 字引[じびき]で 引[ひ]いてみて。	この かんじ を じびき で ひいて みて	
\\	この 漢字[かんじ]を
\\	で 引[ひ]いてみて。			
\\	学習	学習[がくしゅう]	がくしゅう	
\\	今日は野外で学習した。	今日[きょう]は 野外[やがい]で 学習[がくしゅう]した。	きょう は やがい で がくしゅう した	
\\	今日[きょう]は 野外[やがい]で
\\	した。			
\\	試合	試合[しあい]	しあい	
\\	試合の結果を早く知りたい。	試合[しあい]の 結果[けっか]を 早[はや]く 知[し]りたい。	しあい の けっか を はやく しりたい	
\\	の 結果[けっか]を 早[はや]く 知[し]りたい。			
\\	アルコール	アルコール	アルコール	
\\	彼はアルコールに強い体質です。	彼[かれ]はアルコールに 強[つよ]い 体質[たいしつ]です。	かれ は あるこーる に つよい たいしつ です	
\\	彼[かれ]は
\\	に 強[つよ]い 体質[たいしつ]です。			
\\	試みる	試[こころ]みる	こころみる	
\\	彼は実験を試みたのよ。	彼[かれ]は 実験[じっけん]を 試[こころ]みたのよ。	かれ は じっけん を こころみた の よ	
\\	彼[かれ]は 実験[じっけん]を
\\	のよ。			
\\	受験	受験[じゅけん]	じゅけん	
\\	日本語能力試験を受験したんだ。	日本語能力試験[にほんご のうりょく しけん]を 受験[じゅけん]したんだ。	にほんご のうりょく しけん を じゅけん した ん だ	
\\	日本語能力試験[にほんご のうりょく しけん]を
\\	したんだ。			
\\	性質	性質[せいしつ]	せいしつ	
\\	この犬は穏やかな性質だよ。	この 犬[いぬ]は 穏[おだ]やかな 性質[せいしつ]だよ。	この いぬ は おだやか な せいしつ だ よ	
\\	この 犬[いぬ]は 穏[おだ]やかな
\\	だよ。			
\\	悪質	悪質[あくしつ]	あくしつ	
\\	最近は悪質な事件が多いですね。	最近[さいきん]は 悪質[あくしつ]な 事件[じけん]が 多[おお]いですね。	さいきん は あくしつ な じけん が おおい です ね	
\\	最近[さいきん]は
\\	な 事件[じけん]が 多[おお]いですね。			
\\	質	質[しつ]	しつ	
\\	量より質の方が大事です。	量[りょう]より 質[しつ]の 方[ほう]が 大事[だいじ]です。	りょう より しつ の ほう が だいじ です	
\\	量[りょう]より
\\	の 方[ほう]が 大事[だいじ]です。			
\\	学問	学問[がくもん]	がくもん	
\\	彼は少年の頃から学問が好きでした。	彼[かれ]は 少年[しょうねん]の 頃[ころ]から 学問[がくもん]が 好[す]きでした。	かれ は しょうねん の ころ から がくもん が すき でした	
\\	彼[かれ]は 少年[しょうねん]の 頃[ころ]から
\\	が 好[す]きでした。			
\\	有り難い	有[あ]り 難[がた]い	ありがたい	
\\	彼の助けは本当に有り難いな。	彼[かれ]の 助[たす]けは 本当[ほんとう]に 有[あ]り 難[がた]いな。	かれ の たすけ は ほんとう に ありがたい な	
\\	彼[かれ]の 助[たす]けは 本当[ほんとう]に
\\	な。			
\\	せめて	せめて	せめて	
\\	せめてこれだけは約束してください。	せめてこれだけは 約束[やくそく]してください。	せめて これ だけ は やくそく して ください 。	
\\	これだけは 約束[やくそく]してください。			
\\	重点	重点[じゅうてん]	じゅうてん	
\\	子供の自主性に重点を置いています。	子供[こども]の 自主性[じしゅせい]に 重点[じゅうてん]を 置[お]いています。	こども の じしゅせい に じゅうてん を おいて います	
\\	子供[こども]の 自主性[じしゅせい]に
\\	を 置[お]いています。			
\\	弱点	弱点[じゃくてん]	じゃくてん	
\\	彼の弱点はスタミナが足りないところです。	彼[かれ]の 弱点[じゃくてん]はスタミナが 足[た]りないところです。	かれ の じゃくてん は すたみな が たりない ところ です	
\\	彼[かれ]の
\\	はスタミナが 足[た]りないところです。			
\\	少数	少数[しょうすう]	しょうすう	
\\	その計画に反対の人はほんの少数だったよ。	その 計画[けいかく]に 反対[はんたい]の 人[ひと]はほんの 少数[しょうすう]だったよ。	その けいかく に はんたい の ひと は ほんの しょうすう だった よ	
\\	その 計画[けいかく]に 反対[はんたい]の 人[ひと]はほんの
\\	だったよ。			
\\	小数	小数[しょうすう]	しょうすう	
\\	小数は切り捨てて計算して下さい。	小数[しょうすう]は 切[き]り 捨[す]てて 計算[けいさん]して 下[くだ]さい。	しょうすう は きりすてて けいさん して ください	
\\	は 切[き]り 捨[す]てて 計算[けいさん]して 下[くだ]さい。			
\\	回路	回路[かいろ]	かいろ	
\\	コンピュータの電子回路が故障した。	コンピュータの 電子[でんし] 回路[かいろ]が 故障[こしょう]した。	こんぴゅーた の でんし かいろ が こしょう した	
\\	コンピュータの 電子[でんし]
\\	が 故障[こしょう]した。			
\\	回数	回数[かいすう]	かいすう	
\\	最近はテレビを見る回数が減りました。	最近[さいきん]はテレビを 見[み]る 回数[かいすう]が 減[へ]りました。	さいきん は てれび を みる かいすう が へりました	
\\	最近[さいきん]はテレビを 見[み]る
\\	が 減[へ]りました。			
\\	前回	前回[ぜんかい]	ぜんかい	
\\	前回の続きから始めます。	前回[ぜんかい]の 続[つづ]きから 始[はじ]めます。	ぜんかい の つづき から はじめます	
\\	の 続[つづ]きから 始[はじ]めます。			
\\	ストレス	ストレス	ストレス	
\\	ストレスがいろいろな病気の元になっているの。	ストレスがいろいろな 病気[びょうき]の 元[もと]になっているの。	すとれす が いろいろ な びょうき の もと に なっている の 。	
\\	がいろいろな 病気[びょうき]の 元[もと]になっているの。			
\\	後回し	後回[あとまわ]し	あとまわし	
\\	おしゃべりは後回しにしましょう。	おしゃべりは 後回[あとまわ]しにしましょう。	おしゃべり は あとまわし に しましょう	
\\	おしゃべりは
\\	にしましょう。			
\\	個性	個性[こせい]	こせい	
\\	彼女たちはそれぞれ個性が強いですね。	彼女[かのじょ]たちはそれぞれ 個性[こせい]が 強[つよ]いですね。	かのじょたち は それぞれ こせい が つよい です ね	
\\	彼女[かのじょ]たちはそれぞれ
\\	が 強[つよ]いですね。			
\\	個々	個々[ここ]	ここ	
\\	個々の問題を解決しましょう。	個々[ここ]の 問題[もんだい]を 解決[かいけつ]しましょう。	ここ の もんだい を かいけつ しましょう	
\\	の 問題[もんだい]を 解決[かいけつ]しましょう。			
\\	勝手	勝手[かって]	かって	
\\	勝手なことばかり言わないでくれ。	勝手[かって]なことばかり 言[い]わないでくれ。	かって な こと ばかり いわないで くれ	
\\	なことばかり 言[い]わないでくれ。			
\\	決勝	決勝[けっしょう]	けっしょう	
\\	僕たちは頑張って決勝まで進んだよ。	僕[ぼく]たちは 頑張[がんば]って 決勝[けっしょう]まで 進[すす]んだよ。	ぼくたち は がんばって けっしょう まで すすんだ よ	
\\	僕[ぼく]たちは 頑張[がんば]って
\\	まで 進[すす]んだよ。			
\\	勝ち	勝[か]ち	かち	
\\	歌合戦は赤組の勝ちでしたね。	歌合戦[うたがっせん]は 赤組[あか ぐみ]の 勝[か]ちでしたね。	うたがっせん は あか ぐみ の かち でした ね	
\\	歌合戦[うたがっせん]は 赤組[あか ぐみ]の
\\	でしたね。			
\\	勝負	勝負[しょうぶ]	しょうぶ	
\\	勝負はまだ始まったばかりよ。	勝負[しょうぶ]はまだ 始[はじ]まったばかりよ。	しょうぶ は まだ はじまった ばかり よ	
\\	はまだ 始[はじ]まったばかりよ。			
\\	担ぐ	担[かつ]ぐ	かつぐ	
\\	彼は大きな荷物を担いで来たの。	彼[かれ]は 大[おお]きな 荷物[にもつ]を 担[かつ]いで 来[き]たの。	かれ は おおき な にもつ を かついで きた の	
\\	彼[かれ]は 大[おお]きな 荷物[にもつ]を
\\	来[き]たの。			
\\	コード	コード	コード	
\\	コードが短くてコンセントに届きません。	コードが 短[みじか]くてコンセントに 届[とど]きません。	こーど が みじかく て こんせんと に とどきません	
\\	が 短[みじか]くてコンセントに 届[とど]きません。			
\\	当てる	当[あ]てる	あてる	
\\	彼はくじ引きで一等賞を当てたよ。	彼[かれ]はくじ 引[び]きで 一等賞[いっとう しょう]を 当[あ]てたよ。	かれ は くじびき で いっとう しょう を あてた よ	
\\	彼[かれ]はくじ 引[び]きで 一等賞[いっとう しょう]を
\\	よ。			
\\	当たり前	当[あ]たり 前[まえ]	あたりまえ	
\\	あなたの成績が下がったのは当たり前です。	あなたの 成績[せいせき]が 下[さ]がったのは 当[あ]たり 前[まえ]です。	あなた の せいせき が さがった の は あたりまえ です	
\\	あなたの 成績[せいせき]が 下[さ]がったのは
\\	です。			
\\	見当	見当[けんとう]	けんとう	
\\	この仕事には何日必要か見当もつかないね。	この 仕事[しごと]には 何日必要[なんにち ひつよう]か 見当[けんとう]もつかないね。	この しごと に は なんにち ひつよう か けんとう も つかない ね	
\\	この 仕事[しごと]には 何日必要[なんにち ひつよう]か
\\	もつかないね。			
\\	当たり	当[あ]たり	あたり	
\\	彼の予想は大当たりでした。	彼[かれ]の 予想[よそう]は 大[おお] 当[あ]たりでした。	かれ の よそう は おおあたり でした	
\\	彼[かれ]の 予想[よそう]は 大[おお]
\\	でした。			
\\	規制	規制[きせい]	きせい	
\\	牛肉の輸入が厳しく規制されているわね。	牛肉[ぎゅうにく]の 輸入[ゆにゅう]が 厳[きび]しく 規制[きせい]されているわね。	ぎゅうにく の ゆにゅう が きびしく きせい されて いる わ ね	
\\	牛肉[ぎゅうにく]の 輸入[ゆにゅう]が 厳[きび]しく
\\	されているわね。			
\\	経費	経費[けいひ]	けいひ	
\\	これからは経費を節約しましょう。	これからは 経費[けいひ]を 節約[せつやく]しましょう。	これ から は けいひ を せつやく しましょう	
\\	これからは
\\	を 節約[せつやく]しましょう。			
\\	済む	済[す]む	すむ	
\\	父の手術が無事に済みました。	父[ちち]の 手術[しゅじゅつ]が 無事[ぶじ]に 済[す]みました。	ちち の しゅじゅつ が ぶじ に すみました	
\\	父[ちち]の 手術[しゅじゅつ]が 無事[ぶじ]に
\\	コーチ	コーチ	コーチ	
\\	新しいコーチの指導は厳しかった。	新[あたら]しいコーチの 指導[しどう]は 厳[きび]しかった。	あたらしい こーち の しどう は きびしかった	
\\	新[あたら]しい
\\	の 指導[しどう]は 厳[きび]しかった。			
\\	済ませる	済[す]ませる	すませる	
\\	昼ごはんは簡単に済ませましょう。	昼[ひる]ごはんは 簡単[かんたん]に 済[す]ませましょう。	ひるごはん は かんたん に すませましょう	
\\	昼[ひる]ごはんは 簡単[かんたん]に
\\	済ます	済[す]ます	すます	
\\	宿題を済ませてから遊びなさい。	宿題[しゅくだい]を 済[す]ませてから 遊[あそ]びなさい。	しゅくだい を すませて から あそびなさい 。	
\\	宿題[しゅくだい]を
\\	から 遊[あそ]びなさい。			
\\	株式	株式[かぶしき]	かぶしき	
\\	彼は株式の売買で多額の利益を得たのさ。	彼[かれ]は 株式[かぶしき]の 売買[ばいばい]で 多額[たがく]の 利益[りえき]を 得[え]たのさ。	かれ は かぶしき の ばいばい で たがく の りえき を えた の さ	
\\	彼[かれ]は
\\	の 売買[ばいばい]で 多額[たがく]の 利益[りえき]を 得[え]たのさ。			
\\	式	式[しき]	しき	
\\	彼らは教会で式を挙げました。	彼[かれ]らは 教会[きょうかい]で 式[しき]を 挙[あ]げました。	かれら は きょうかい で しき を あげました	
\\	彼[かれ]らは 教会[きょうかい]で
\\	を 挙[あ]げました。			
\\	業界	業界[ぎょうかい]	ぎょうかい	
\\	私は
\\	業界で働いています。	私[わたし]は
\\	業界[ぎょうかい]で 働[はたら]いています。	わたし は 
\\	ぎょうかい で はたらいて います	
\\	私[わたし]は
\\	で 働[はたら]いています。			
\\	営業	営業[えいぎょう]	えいぎょう	
\\	彼は営業を担当しています。	彼[かれ]は 営業[えいぎょう]を 担当[たんとう]しています。	かれ は えいぎょう を たんとう しています 。	
\\	彼[かれ]は
\\	を 担当[たんとう]しています。			
\\	休業	休業[きゅうぎょう]	きゅうぎょう	
\\	明日は臨時に休業します。	明日[あした]は 臨時[りんじ]に 休業[きゅうぎょう]します。	あした は りんじ に きゅうぎょう します	
\\	明日[あした]は 臨時[りんじ]に
\\	します。			
\\	ステージ	ステージ	ステージ	
\\	彼はステージに立ったよ。	彼[かれ]はステージに 立[た]ったよ。	かれ は すてーじ に たった よ	
\\	彼[かれ]は
\\	に 立[た]ったよ。			
\\	事業	事業[じぎょう]	じぎょう	
\\	彼は事業に失敗したの。	彼[かれ]は 事業[じぎょう]に 失敗[しっぱい]したの。	かれ は じぎょう に しっぱい した の	
\\	彼[かれ]は
\\	に 失敗[しっぱい]したの。			
\\	金利	金利[きんり]	きんり	
\\	最近、銀行の金利が少し上がった。	最近[さいきん]、 銀行[ぎんこう]の 金利[きんり]が 少[すこ]し 上[あ]がった。	さいきん ぎんこう の きんり が すこし あがった	
\\	最近[さいきん]、 銀行[ぎんこう]の
\\	が 少[すこ]し 上[あ]がった。			
\\	利く	利[き]く	きく	
\\	彼女は気が利いている。	彼女[かのじょ]は 気[き]が 利[き]いている。	かのじょ は き が きいて いる	
\\	彼女[かのじょ]は 気[き]が
\\	収入	収入[しゅうにゅう]	しゅうにゅう	
\\	日本人の平均収入はどの位ですか。	日本人[にほんじん]の 平均[へいきん] 収入[しゅうにゅう]はどの 位[くらい]ですか。	にほんじん の へいきん しゅうにゅう は どの くらい です か	
\\	日本人[にほんじん]の 平均[へいきん]
\\	はどの 位[くらい]ですか。			
\\	芸術	芸術[げいじゅつ]	げいじゅつ	
\\	ここは芸術を愛する国です。	ここは 芸術[げいじゅつ]を 愛[あい]する 国[くに]です。	ここ は げいじゅつ を あいする くに です	
\\	ここは
\\	を 愛[あい]する 国[くに]です。			
\\	製品	製品[せいひん]	せいひん	
\\	当社の製品は3年間の保障つきです。	当社[とうしゃ]の 製品[せいひん]は 3年間[さんねんかん]の 保障[ほしょう]つきです。	とうしゃ の せいひん は さんねんかん の ほしょうつき です	
\\	当社[とうしゃ]の
\\	は 3年間[さんねんかん]の 保障[ほしょう]つきです。			
\\	製作	製作[せいさく]	せいさく	
\\	この映画は香港で製作されました。	この 映画[えいが]は 香港[ほんこん]で 製作[せいさく]されました。	この えいが は ほんこん で せいさく されました	
\\	この 映画[えいが]は 香港[ほんこん]で
\\	されました。			
\\	作製	作製[さくせい]	さくせい	
\\	合鍵の作製には2日ほどかかります。	合鍵[あいかぎ]の 作製[さくせい]には 2日[ふつか]ほどかかります。	あいかぎ の さくせい に は ふつか ほど かかります	
\\	合鍵[あいかぎ]の
\\	には 2日[ふつか]ほどかかります。			
\\	アドバイス	アドバイス	アドバイス	
\\	彼のアドバイスはいつも有り難いわね。	彼[かれ]のアドバイスはいつも 有[あ]り 難[がた]いわね。	かれ の あどばいす は いつも ありがたい わ ね	
\\	彼[かれ]の
\\	はいつも 有[あ]り 難[がた]いわね。			
\\	必ずしも	必[かなら]ずしも	かならずしも	
\\	親切は必ずしも喜ばれるわけではない。	親切[しんせつ]は 必[かなら]ずしも 喜[よろこ]ばれるわけではない。	しんせつ は かならずしも よろこばれる わけ で は ない	
\\	親切[しんせつ]は
\\	喜[よろこ]ばれるわけではない。			
\\	求人	求人[きゅうじん]	きゅうじん	
\\	彼は求人広告で仕事を見つけたんだ。	彼[かれ]は 求人[きゅうじん] 広告[こうこく]で 仕事[しごと]を 見[み]つけたんだ。	かれ は きゅうじん こうこく で しごと を みつけた ん だ	
\\	彼[かれ]は
\\	広告[こうこく]で 仕事[しごと]を 見[み]つけたんだ。			
\\	額	額[がく]	がく	
\\	写真を額に入れて飾ったんだ。	写真[しゃしん]を 額[がく]に 入[い]れて 飾[かざ]ったんだ。	しゃしん を がく に いれて かざった ん だ	
\\	写真[しゃしん]を
\\	に 入[い]れて 飾[かざ]ったんだ。			
\\	金額	金額[きんがく]	きんがく	
\\	レシートで買い物の金額を確かめたよ。	レシートで 買[か]い 物[もの]の 金額[きんがく]を 確[たし]かめたよ。	れしーと で かいもの の きんがく を たしかめた よ	
\\	レシートで 買[か]い 物[もの]の
\\	を 確[たし]かめたよ。			
\\	計	計[けい]	けい	
\\	3人分の代金は計6000円です。	3人分[さんにんぶん]の 代金[だいきん]は 計[けい] 6000円[ろくせんえん]です。	さんにんぶん の だいきん は けい ろくせんえん です	
\\	3人分[さんにんぶん]の 代金[だいきん]は
\\	6000円[ろくせんえん]です。			
\\	合計	合計[ごうけい]	ごうけい	
\\	合計金額を計算してください。	合計[ごうけい] 金額[きんがく]を 計算[けいさん]してください。	ごうけい きんがく を けいさん して ください	
\\	金額[きんがく]を 計算[けいさん]してください。			
\\	家計	家計[かけい]	かけい	
\\	彼女は家計を任されているの。	彼女[かのじょ]は 家計[かけい]を 任[まか]されているの。	かのじょ は かけい を まかされて いる の	
\\	彼女[かのじょ]は
\\	を 任[まか]されているの。			
\\	アルミニウム	アルミニウム	アルミニウム	
\\	このお鍋はアルミニウム製です。	このお 鍋[なべ]はアルミニウム 製[せい]です。	この おなべ は あるみにうむ せい です	
\\	このお 鍋[なべ]は
\\	製[せい]です。			
\\	会計	会計[かいけい]	かいけい	
\\	会計を済ませて店を出たんだ。	会計[かいけい]を 済[す]ませて 店[みせ]を 出[で]たんだ。	かいけい を すませて みせ を でた ん だ	
\\	を 済[す]ませて 店[みせ]を 出[で]たんだ。			
\\	寒暖計	寒暖計[かんだんけい]	かんだんけい	
\\	壁に寒暖計が掛かっていました。	壁[かべ]に 寒暖計[かんだんけい]が 掛[か]かっていました。	かべ に かんだんけい が かかって いました	
\\	壁[かべ]に
\\	が 掛[か]かっていました。			
\\	差	差[さ]	さ	
\\	都心と地方では収入に大きな差があるね。	都心[としん]と 地方[ちほう]では 収入[しゅうにゅう]に 大[おお]きな 差[さ]があるね。	としん と ちほう で は しゅうにゅう に おおき な さ が ある ね	
\\	都心[としん]と 地方[ちほう]では 収入[しゅうにゅう]に 大[おお]きな
\\	があるね。			
\\	格差	格差[かくさ]	かくさ	
\\	貧富の格差が大きくなっているな。	貧富[ひんぷ]の 格差[かくさ]が 大[おお]きくなっているな。	ひんぷ の かくさ が おおきく なって いる な	
\\	貧富[ひんぷ]の
\\	が 大[おお]きくなっているな。			
\\	差し出す	差[さ]し 出[だ]す	さしだす	
\\	彼は握手をしようと手を差し出したの。	彼[かれ]は 握手[あくしゅ]をしようと 手[て]を 差[さ]し 出[だ]したの。	かれ は あくしゅ を しようと て を さしだした の	
\\	彼[かれ]は 握手[あくしゅ]をしようと 手[て]を
\\	の。			
\\	時差	時差[じさ]	じさ	
\\	日本とフランスの時差は8時間です。	日本[にほん]とフランスの 時差[じさ]は 8時間[はちじかん]です。	にほん と ふらんす の じさ は はちじかん です	
\\	日本[にほん]とフランスの
\\	は 8時間[はちじかん]です。			
\\	差す	差[さ]す	さす	
\\	雨が降ってきたので傘を差しました。	雨[あめ]が 降[ふ]ってきたので 傘[かさ]を 差[さ]しました。	あめ が ふって きた の で かさ を さしました	
\\	雨[あめ]が 降[ふ]ってきたので 傘[かさ]を
\\	オーケストラ	オーケストラ	オーケストラ	
\\	彼はオーケストラを指揮しているの。	彼[かれ]はオーケストラを 指揮[しき]しているの。	かれ は おーけすとら を しき して いる の	
\\	彼[かれ]は
\\	を 指揮[しき]しているの。			
\\	差し上げる	差[さ]し 上[あ]げる	さしあげる	
\\	こちらを差し上げます。	こちらを 差[さ]し 上[あ]げます。	こちら を さしあげます	
\\	こちらを
\\	学割	学割[がくわり]	がくわり	
\\	学割だとだいぶ安いな。	学割[がくわり]だとだいぶ 安[やす]いな。	がくわり だ と だいぶ やすい な	
\\	だとだいぶ 安[やす]いな。			
\\	時間割り	時間割[じかんわ]り	じかんわり	
\\	明日の授業は時間割り通りです。	明日[あした]の 授業[じゅぎょう]は 時間割[じかんわ]り 通[どお]りです。	あした の じゅぎょう は じかんわり どおり です	
\\	明日[あした]の 授業[じゅぎょう]は
\\	通[どお]りです。			
\\	残業	残業[ざんぎょう]	ざんぎょう	
\\	昨日は遅くまで残業しました。	昨日[きのう]は 遅[おそ]くまで 残業[ざんぎょう]しました。	きのう は おそく まで ざんぎょう しました	
\\	昨日[きのう]は 遅[おそ]くまで
\\	しました。			
\\	残暑	残暑[ざんしょ]	ざんしょ	
\\	今年も残暑が厳しかった。	今年[ことし]も 残暑[ざんしょ]が 厳[きび]しかった。	ことし も ざんしょ が きびしかった	
\\	今年[ことし]も
\\	が 厳[きび]しかった。			
\\	支店	支店[してん]	してん	
\\	彼は支店に転勤したよ。	彼[かれ]は 支店[してん]に 転勤[てんきん]したよ。	かれ は してん に てんきん した よ	
\\	彼[かれ]は
\\	に 転勤[てんきん]したよ。			
\\	支持	支持[しじ]	しじ	
\\	彼は国民の支持を得たのよ。	彼[かれ]は 国民[こくみん]の 支持[しじ]を 得[え]たのよ。	かれ は こくみん の しじ を えた の よ	
\\	彼[かれ]は 国民[こくみん]の
\\	を 得[え]たのよ。			
\\	支出	支出[ししゅつ]	ししゅつ	
\\	今月のわが家の支出は15万円です。	今月[こんげつ]のわが 家[や]の 支出[ししゅつ]は 15万円[じゅうごまんえん]です。	こんげつ の わがや の ししゅつ は じゅうごまんえん です	
\\	今月[こんげつ]のわが 家[や]の
\\	は 15万円[じゅうごまんえん]です。			
\\	おじさん	おじさん	おじさん	
\\	今そこで隣のおじさんに会ったよ。	今[いま]そこで 隣[となり]のおじさんに 会[あ]ったよ。	いま そこで となり の おじさん に あった よ	
\\	今[いま]そこで 隣[となり]の
\\	に 会[あ]ったよ。			
\\	支配	支配[しはい]	しはい	
\\	その権力者による支配は50年以上続いたんです。	その 権力者[けんりょくしゃ]による 支配[しはい]は 50年以上続[ごじゅうねん いじょう つづ]いたんです。	その けんりょくしゃ に よる しはい は ごじゅうねん いじょう つづいた ん です	
\\	その 権力者[けんりょくしゃ]による
\\	は 50年以上続[ごじゅうねん いじょう つづ]いたんです。			
\\	収支	収支[しゅうし]	しゅうし	
\\	家計の収支が合わないの。	家計[かけい]の 収支[しゅうし]が 合[あ]わないの。	かけい の しゅうし が あわない の	
\\	家計[かけい]の
\\	が 合[あ]わないの。			
\\	支度	支度[したく]	したく	
\\	支度ができたら出かけましょう。	支度[したく]ができたら 出[で]かけましょう。	したく が できたら でかけましょう	
\\	ができたら 出[で]かけましょう。			
\\	支える	支[ささ]える	ささえる	
\\	父親には一家を支える責任がある。	父親[ちちおや]には 一家[いっか]を 支[ささ]える 責任[せきにん]がある。	ちちおや に は いっか を ささえる せきにん が ある	
\\	父親[ちちおや]には 一家[いっか]を
\\	責任[せきにん]がある。			
\\	支社	支社[ししゃ]	ししゃ	
\\	来月大阪に支社を開設します。	来月大阪[らいげつ おおさか]に 支社[ししゃ]を 開設[かいせつ]します。	らいげつ おおさか に ししゃ を かいせつ します	
\\	来月大阪[らいげつ おおさか]に
\\	を 開設[かいせつ]します。			
\\	支払う	支払[しはら]う	しはらう	
\\	カウンターで料金を支払った。	カウンターで 料金[りょうきん]を 支払[しはら]った。	かうんたー で りょうきん を しはらった	
\\	カウンターで 料金[りょうきん]を
\\	支払い	支払[しはら]い	しはらい	
\\	お支払いはカードもお使いいただけます。	お 支払[しはら]いはカードもお 使[つか]いいただけます。	おしはらい は かーど も お つかい いただけます	
\\	お
\\	はカードもお 使[つか]いいただけます。			
\\	コンクール	コンクール	コンクール	
\\	来年のコンクールに出場するつもりです。	来年[らいねん]のコンクールに 出場[しゅつじょう]するつもりです。	らいねん の こんくーる に しゅつじょう する つもり です	
\\	来年[らいねん]の
\\	に 出場[しゅつじょう]するつもりです。			
\\	言い返す	言[い]い 返[かえ]す	いいかえす	
\\	彼女は負けずに言い返したの。	彼女[かのじょ]は 負[ま]けずに 言[い]い 返[かえ]したの。	かのじょ は まけず に いいかえした の	
\\	彼女[かのじょ]は 負[ま]けずに
\\	の。			
\\	返る	返[かえ]る	かえる	
\\	もう一度原点に返って考えましょう。	もう 一度原点[いちど げんてん]に 返[かえ]って 考[かんが]えましょう。	もういちど げんてん に かえって かんがえましょう	
\\	もう 一度原点[いちど げんてん]に
\\	考[かんが]えましょう。			
\\	借り	借[か]り	かり	
\\	この借りは必ず返します。	この 借[か]りは 必[かなら]ず 返[かえ]します。	この かり は かならず かえします	
\\	この
\\	は 必[かなら]ず 返[かえ]します。			
\\	借金	借金[しゃっきん]	しゃっきん	
\\	彼は友達に借金をしたんだ。	彼[かれ]は 友達[ともだち]に 借金[しゃっきん]をしたんだ。	かれ は ともだち に しゃっきん を した ん だ	
\\	彼[かれ]は 友達[ともだち]に
\\	をしたんだ。			
\\	貸し出し	貸[か]し 出[だ]し	かしだし	
\\	その本は貸し出ししていません。	その 本[ほん]は 貸[か]し 出[だ]ししていません。	その ほん は かしだし して いません	
\\	その 本[ほん]は
\\	していません。			
\\	貸し	貸[か]し	かし	
\\	彼には貸しがあるの。	彼[かれ]には 貸[か]しがあるの。	かれ に は かし が ある の	
\\	彼[かれ]には
\\	があるの。			
\\	申請	申請[しんせい]	しんせい	
\\	これからパスポートの申請に行きます。	これからパスポートの 申請[しんせい]に 行[い]きます。	これから ぱすぽーと の しんせい に いきます	
\\	これからパスポートの
\\	に 行[い]きます。			
\\	アマチュア	アマチュア	アマチュア	
\\	彼はアマチュア音楽家です。	彼[かれ]はアマチュア 音楽家[おんがくか]です。	かれ は あまちゅあ おんがくか です	
\\	彼[かれ]は
\\	音楽家[おんがくか]です。			
\\	込める	込[こ]める	こめる	
\\	彼は感情を込めてその歌を歌ったの。	彼[かれ]は 感情[かんじょう]を 込[こ]めてその 歌[うた]を 歌[うた]ったの。	かれ は かんじょう を こめて その うた を うたった の	
\\	彼[かれ]は 感情[かんじょう]を
\\	その 歌[うた]を 歌[うた]ったの。			
\\	打ち込む	打[う]ち 込[こ]む	うちこむ	
\\	彼は研究に打ち込んでいます。	彼[かれ]は 研究[けんきゅう]に 打[う]ち 込[こ]んでいます。	かれ は けんきゅう に うちこんで います	
\\	彼[かれ]は 研究[けんきゅう]に
\\	思い込む	思[おも]い 込[こ]む	おもいこむ	
\\	彼は騙されたと思い込んでいるようです。	彼[かれ]は 騙[だま]されたと 思[おも]い 込[こ]んでいるようです。	かれ は だまされた と おもいこんで いる よう です	
\\	彼[かれ]は 騙[だま]されたと
\\	ようです。			
\\	初期	初期[しょき]	しょき	
\\	鼻水は風邪の初期症状のひとつです。	鼻水[はなみず]は 風邪[かぜ]の 初期[しょき] 症状[しょうじょう]のひとつです。	はなみず は かぜ の しょき しょうじょう の ひとつ です	
\\	鼻水[はなみず]は 風邪[かぜ]の
\\	症状[しょうじょう]のひとつです。			
\\	学期	学期[がっき]	がっき	
\\	新学期が始まったね。	新[しん] 学期[がっき]が 始[はじ]まったね。	しんがっき が はじまった ね	
\\	新[しん]
\\	が 始[はじ]まったね。			
\\	後期	後期[こうき]	こうき	
\\	後期の授業が始まりました。	後期[こうき]の 授業[じゅぎょう]が 始[はじ]まりました。	こうき の じゅぎょう が はじまりました	
\\	の 授業[じゅぎょう]が 始[はじ]まりました。			
\\	前期	前期[ぜんき]	ぜんき	
\\	前期の売上はとても良かったわ。	前期[ぜんき]の 売上[うりあげ]はとても 良[よ]かったわ。	ぜんき の うりあげ は とても よかった わ	
\\	の 売上[うりあげ]はとても 良[よ]かったわ。			
\\	きつい	きつい	きつい	
\\	彼女は性格がきついよね。	彼女[かのじょ]は 性格[せいかく]がきついよね。	かのじょ は せいかく が きつい よ ね	
\\	彼女[かのじょ]は 性格[せいかく]が
\\	よね。			
\\	期日	期日[きじつ]	きじつ	
\\	代金を期日までにお支払いください。	代金[だいきん]を 期日[きじつ]までにお 支払[しはら]いください。	だいきん を きじつ まで に おしはらい ください	
\\	代金[だいきん]を
\\	までにお 支払[しはら]いください。			
\\	新学期	新学期[しんがっき]	しんがっき	
\\	今日から新学期が始まります。	今日[きょう]から 新学期[しんがっき]が 始[はじ]まります。	きょう から しんがっき が はじまります	
\\	今日[きょう]から
\\	が 始[はじ]まります。			
\\	限り	限[かぎ]り	かぎり	
\\	限りある資源を大切にしよう。	限[かぎ]りある 資源[しげん]を 大切[たいせつ]にしよう。	かぎり ある しげん を たいせつ に しよう	
\\	ある 資源[しげん]を 大切[たいせつ]にしよう。			
\\	制限	制限[せいげん]	せいげん	
\\	彼女は食事を制限しています。	彼女[かのじょ]は 食事[しょくじ]を 制限[せいげん]しています。	かのじょ は しょくじ を せいげん して います	
\\	彼女[かのじょ]は 食事[しょくじ]を
\\	しています。			
\\	限界	限界[げんかい]	げんかい	
\\	もう我慢の限界です。	もう 我慢[がまん]の 限界[げんかい]です。	もう がまん の げんかい です	
\\	もう 我慢[がまん]の
\\	です。			
\\	期限	期限[きげん]	きげん	
\\	期限までに申し込みました。	期限[きげん]までに 申[もう]し 込[こ]みました。	きげん まで に もうしこみました	
\\	までに 申[もう]し 込[こ]みました。			
\\	限度	限度[げんど]	げんど	
\\	物事には限度があります。	物事[ものごと]には 限度[げんど]があります。	ものごと に は げんど が あります	
\\	物事[ものごと]には
\\	があります。			
\\	大急ぎ	大急[おおいそ]ぎ	おおいそぎ	
\\	大急ぎでその仕事を仕上げたよ。	大急[おおいそ]ぎでその 仕事[しごと]を 仕上[しあ]げたよ。	おおいそぎ で その しごと を しあげた よ	
\\	でその 仕事[しごと]を 仕上[しあ]げたよ。			
\\	コーナー	コーナー	コーナー	
\\	バーゲンコーナーで
\\	を買いました。	バーゲンコーナーで 
\\	[でぃーぶいでぃー]を 買[か]いました。	ばーげんこーなー で でぃーぶいでぃー を かいました	
\\	バーゲン
\\	で 
\\	[でぃーぶいでぃー]を 買[か]いました。			
\\	急用	急用[きゅうよう]	きゅうよう	
\\	彼は急用で帰りました。	彼[かれ]は 急用[きゅうよう]で 帰[かえ]りました。	かれ は きゅうよう で かえりました	
\\	彼[かれ]は
\\	で 帰[かえ]りました。			
\\	一切	一切[いっさい]	いっさい	
\\	私は一切その問題とは関係がありません。	私[わたし]は 一切[いっさい]その 問題[もんだい]とは 関係[かんけい]がありません。	わたし は いっさい その もんだい と は かんけい が ありません	
\\	私[わたし]は
\\	その 問題[もんだい]とは 関係[かんけい]がありません。			
\\	区切る	区切[くぎ]る	くぎる	
\\	フロアはパーティションで区切られています。	フロアはパーティションで 区切[くぎ]られています。	ふろあ は ぱーてぃしょん で くぎられて います	
\\	フロアはパーティションで
\\	思い切って	思[おも]い 切[き]って	おもいきって	
\\	思い切って彼に相談します。	思[おも]い 切[き]って 彼[かれ]に 相談[そうだん]します。	おもいきって かれ に そうだん します	
\\	彼[かれ]に 相談[そうだん]します。			
\\	品切れ	品切[しなぎ]れ	しなぎれ	
\\	牛乳は品切れだったよ。	牛乳[ぎゅうにゅう]は 品切[しなぎ]れだったよ。	ぎゅうにゅう は しなぎれ だった よ	
\\	牛乳[ぎゅうにゅう]は
\\	だったよ。			
\\	思い切り	思[おも]い 切[き]り	おもいきり	
\\	カラオケで思い切り歌ったの。	カラオケで 思[おも]い 切[き]り 歌[うた]ったの。	からおけ で おもいきり うたった の	
\\	カラオケで
\\	歌[うた]ったの。			
\\	切れ	切[き]れ	きれ	
\\	この包丁は切れが良いな。	この 包丁[ほうちょう]は 切[き]れが 良[い]いな。	この ほうちょう は きれ が いい な	
\\	この 包丁[ほうちょう]は
\\	が 良[い]いな。			
\\	ぎりぎり	ぎりぎり	ぎりぎり	
\\	会社の始業時間にぎりぎりで間に合ったよ。	会社[かいしゃ]の 始業時間[しぎょうじかん]にぎりぎりで 間[ま]に 合[あ]ったよ。	かいしゃ の しぎょうじかん に ぎりぎり で まにあった よ	
\\	会社[かいしゃ]の 始業時間[しぎょうじかん]に
\\	で 間[ま]に 合[あ]ったよ。			
\\	券	券[けん]	けん	
\\	入場券をお持ちですか。	入場[にゅうじょう] 券[けん]をお 持[も]ちですか。	にゅうじょうけん を お もち です か	
\\	入場[にゅうじょう]
\\	をお 持[も]ちですか。			
\\	回数券	回数券[かいすうけん]	かいすうけん	
\\	バスの回数券を買いました。	バスの 回数券[かいすうけん]を 買[か]いました。	ばす の かいすうけん を かいました	
\\	バスの
\\	を 買[か]いました。			
\\	世代	世代[せだい]	せだい	
\\	私は彼と同じ世代です。	私[わたし]は 彼[かれ]と 同[おな]じ 世代[せだい]です。	わたし は かれ と おなじ せだい です	
\\	私[わたし]は 彼[かれ]と 同[おな]じ
\\	です。			
\\	代わり	代[か]わり	かわり	
\\	ごま油の代わりにオリーブ油を使いましょう。	ごま 油[あぶら]の 代[か]わりにオリーブ 油[ゆ]を 使[つか]いましょう。	ごまあぶら の かわり に おりーぶゆ を つかいましょう	
\\	ごま 油[あぶら]の
\\	にオリーブ 油[ゆ]を 使[つか]いましょう。			
\\	近代	近代[きんだい]	きんだい	
\\	近代の技術の発展はものすごいです。	近代[きんだい]の 技術[ぎじゅつ]の 発展[はってん]はものすごいです。	きんだい の ぎじゅつ の はってん は ものすごい です	
\\	の 技術[ぎじゅつ]の 発展[はってん]はものすごいです。			
\\	古代	古代[こだい]	こだい	
\\	古代の歴史について勉強しました。	古代[こだい]の 歴史[れきし]について 勉強[べんきょう]しました。	こだい の れきし に ついて べんきょう しました	
\\	の 歴史[れきし]について 勉強[べんきょう]しました。			
\\	代わる代わる	代[か]わる 代[が]わる	かわるがわる	
\\	皆が代わる代わる彼らを祝福したよ。	皆[みな]が 代[か]わる 代[が]わる 彼[かれ]らを 祝福[しゅくふく]したよ。	みな が かわるがわる かれら を しゅくふく した よ	
\\	皆[みな]が
\\	彼[かれ]らを 祝福[しゅくふく]したよ。			
\\	ウサギ	ウサギ	ウサギ	
\\	ウサギとカメの話を知っていますか。	ウサギとカメの 話[はなし]を 知[し]っていますか。	うさぎ と かめ の はなし を しって います か	
\\	とカメの 話[はなし]を 知[し]っていますか。			
\\	指す	指[さ]す	さす	
\\	時計が12時を指してる。	時計[とけい]が 12時[じゅうにじ]を 指[さ]してる。	とけい が じゅうにじ を さして る	
\\	時計[とけい]が 12時[じゅうにじ]を
\\	小指	小指[こゆび]	こゆび	
\\	小指を切ってしまいました。	小指[こゆび]を 切[き]ってしまいました。	こゆび を きって しまいました	
\\	を 切[き]ってしまいました。			
\\	安定	安定[あんてい]	あんてい	
\\	彼は精神の安定が必要よ。	彼[かれ]は 精神[せいしん]の 安定[あんてい]が 必要[ひつよう]よ。	かれ は せいしん の あんてい が ひつよう よ	
\\	彼[かれ]は 精神[せいしん]の
\\	が 必要[ひつよう]よ。			
\\	規定	規定[きてい]	きてい	
\\	代金には規定の手数料が含まれます。	代金[だいきん]には 規定[きてい]の 手数料[てすうりょう]が 含[ふく]まれます。	だいきん に は きてい の てすうりょう が ふくまれます	
\\	代金[だいきん]には
\\	の 手数料[てすうりょう]が 含[ふく]まれます。			
\\	定める	定[さだ]める	さだめる	
\\	税に関する新しい法律が定められたぞ。	税[ぜい]に 関[かん]する 新[あたら]しい 法律[ほうりつ]が 定[さだ]められたぞ。	ぜい に かんする あたらしい ほうりつ が さだめられた ぞ	
\\	税[ぜい]に 関[かん]する 新[あたら]しい 法律[ほうりつ]が
\\	ぞ。			
\\	指定	指定[してい]	してい	
\\	指定された席にお座りください。	指定[してい]された 席[せき]にお 座[すわ]りください。	してい された せき に お すわり ください	
\\	された 席[せき]にお 座[すわ]りください。			
\\	悪化	悪化[あっか]	あっか	
\\	手の傷が悪化した。	手[て]の 傷[きず]が 悪化[あっか]した。	て の きず が あっか した	
\\	手[て]の 傷[きず]が
\\	した。			
\\	化学	化学[かがく]	かがく	
\\	彼は化学の教授です。	彼[かれ]は 化学[かがく]の 教授[きょうじゅ]です。	かれ は かがく の きょうじゅ です	
\\	彼[かれ]は
\\	の 教授[きょうじゅ]です。			
\\	グラウンド	グラウンド	グラウンド	
\\	野球部はグラウンドで練習しています。	野球部[やきゅうぶ]はグラウンドで 練習[れんしゅう]しています。	やきゅうぶ は ぐらうんど で れんしゅう して います	
\\	野球部[やきゅうぶ]は
\\	で 練習[れんしゅう]しています。			
\\	消化	消化[しょうか]	しょうか	
\\	彼は消化不良を起こしたんだ。	彼[かれ]は 消化[しょうか] 不良[ふりょう]を 起[お]こしたんだ。	かれ は しょうか ふりょう を おこした ん だ	
\\	彼[かれ]は
\\	不良[ふりょう]を 起[お]こしたんだ。			
\\	更に	更[さら]に	さらに	
\\	彼は更に質問を続けたの。	彼[かれ]は 更[さら]に 質問[しつもん]を 続[つづ]けたの。	かれ は さらに しつもん を つづけた の	
\\	彼[かれ]は
\\	質問[しつもん]を 続[つづ]けたの。			
\\	今更	今更[いまさら]	いまさら	
\\	今更後悔しても、もう遅いよ。	今更[いまさら] 後悔[こうかい]しても、もう 遅[おそ]いよ。	いまさら こうかい して も もう おそい よ	
\\	後悔[こうかい]しても、もう 遅[おそ]いよ。			
\\	急増	急増[きゅうぞう]	きゅうぞう	
\\	最近ヨガをやる人が急増しています。	最近[さいきん]ヨガをやる 人[ひと]が 急増[きゅうぞう]しています。	さいきん よが を やる ひと が きゅうぞう して います	
\\	最近[さいきん]ヨガをやる 人[ひと]が
\\	しています。			
\\	減少	減少[げんしょう]	げんしょう	
\\	この国は人口が減少しているわね。	この 国[くに]は 人口[じんこう]が 減少[げんしょう]しているわね。	この くに は じんこう が げんしょう して いる わ ね	
\\	この 国[くに]は 人口[じんこう]が
\\	しているわね。			
\\	乗車券	乗車券[じょうしゃけん]	じょうしゃけん	
\\	乗車券は無くさないように。	乗車券[じょうしゃけん]は 無[な]くさないように。	じょうしゃけん は なくさない よう に	
\\	は 無[な]くさないように。			
\\	乗車	乗車[じょうしゃ]	じょうしゃ	
\\	このバスの運賃は乗車するときに払います。	このバスの 運賃[うんちん]は 乗車[じょうしゃ]するときに 払[はら]います。	この ばす の うんちん は じょうしゃ する とき に はらいます	
\\	このバスの 運賃[うんちん]は
\\	するときに 払[はら]います。			
\\	さっぱり	さっぱり	さっぱり	
\\	彼女の言っていることがさっぱり分からないの。	彼女[かのじょ]の 言[い]っていることがさっぱり 分[わ]からないの。	かのじょ の いって いる こと が さっぱり わからない の	
\\	彼女[かのじょ]の 言[い]っていることが
\\	分[わ]からないの。			
\\	乗客	乗客[じょうきゃく]	じょうきゃく	
\\	乗客の一人の具合が悪くなったの。	乗客[じょうきゃく]の 一人[いち にん]の 具合[ぐあい]が 悪[わる]くなったの。	じょうきゃく の いち にん の ぐあい が わるく なった の 。	
\\	の 一人[いち にん]の 具合[ぐあい]が 悪[わる]くなったの。			
\\	雨降り	雨降[あめふ]り	あめふり	
\\	雨降りで月が見えなかったね。	雨降[あめふ]りで 月[つき]が 見[み]えなかったね。	あめふり で つき が みえなかった ね	
\\	で 月[つき]が 見[み]えなかったね。			
\\	税金	税金[ぜいきん]	ぜいきん	
\\	これは税金の無駄遣いだね。	これは 税金[ぜいきん]の 無駄遣[むだづか]いだね。	これ は ぜいきん の むだづかい だ ね	
\\	これは
\\	の 無駄遣[むだづか]いだね。			
\\	税	税[ぜい]	ぜい	
\\	これ以上税が上がると生活できないね。	これ 以上[いじょう] 税[ぜい]が 上[あ]がると 生活[せいかつ]できないね。	これ いじょう ぜい が あがる と せいかつ できない ね	
\\	これ 以上[いじょう]
\\	が 上[あ]がると 生活[せいかつ]できないね。			
\\	私立	私立[しりつ]	しりつ	
\\	私立の大学は学費が高い。	私立[しりつ]の 大学[だいがく]は 学費[がくひ]が 高[たか]い。	しりつ の だいがく は がくひ が たかい	
\\	の 大学[だいがく]は 学費[がくひ]が 高[たか]い。			
\\	市立	市立[しりつ]	しりつ	
\\	娘は市立の学校に通っています。	娘[むすめ]は 市立[しりつ]の 学校[がっこう]に 通[かよ]っています。	むすめ は しりつ の がっこう に かよって います	
\\	娘[むすめ]は
\\	の 学校[がっこう]に 通[かよ]っています。			
\\	県立	県立[けんりつ]	けんりつ	
\\	彼は県立の高校に通っているよ。	彼[かれ]は 県立[けんりつ]の 高校[こうこう]に 通[かよ]っているよ。	かれ は けんりつ の こうこう に かよって いる よ	
\\	彼[かれ]は
\\	の 高校[こうこう]に 通[かよ]っているよ。			
\\	シャッター	シャッター	シャッター	
\\	丸いボタンを押すとシャッターが閉まります。	丸[まる]いボタンを 押[お]すとシャッターが 閉[し]まります。	まるい ぼたん を おすと しゃったー が しまります	
\\	丸[まる]いボタンを 押[お]すと
\\	が 閉[し]まります。			
\\	国立	国立[こくりつ]	こくりつ	
\\	新しい国立劇場が完成しました。	新[あたら]しい 国立[こくりつ] 劇場[げきじょう]が 完成[かんせい]しました。	あたらしい こくりつ げきじょう が かんせい しました	
\\	新[あたら]しい
\\	劇場[げきじょう]が 完成[かんせい]しました。			
\\	座席	座席[ざせき]	ざせき	
\\	飛行機の座席はゆったりしていたよ。	飛行機[ひこうき]の 座席[ざせき]はゆったりしていたよ。	ひこうき の ざせき は ゆったり して いた よ	
\\	飛行機[ひこうき]の
\\	はゆったりしていたよ。			
\\	客席	客席[きゃくせき]	きゃくせき	
\\	私たちは客席に座ったんだ。	私[わたし]たちは 客席[きゃくせき]に 座[すわ]ったんだ。	わたしたち は きゃくせき に すわった ん だ	
\\	私[わたし]たちは
\\	に 座[すわ]ったんだ。			
\\	欠点	欠点[けってん]	けってん	
\\	欠点のない人間はいません。	欠点[けってん]のない 人間[にんげん]はいません。	けってん の ない にんげん は いません	
\\	のない 人間[にんげん]はいません。			
\\	欠ける	欠[か]ける	かける	
\\	お気に入りのカップが欠けてしまいました。	お 気[き]に 入[い]りのカップが 欠[か]けてしまいました。	おきにいり の かっぷ が かけて しまいました	
\\	お 気[き]に 入[い]りのカップが
\\	欠く	欠[か]く	かく	
\\	彼の態度は誠意を欠いています。	彼[かれ]の 態度[たいど]は 誠意[せいい]を 欠[か]いています。	かれ の たいど は せいい を かいて います	
\\	彼[かれ]の 態度[たいど]は 誠意[せいい]を
\\	次回	次回[じかい]	じかい	
\\	次回の会議は2週間後に行います。	次回[じかい]の 会議[かいぎ]は 2週間後[に しゅうかん ご]に 行[おこな]います。	じかい の かいぎ は に しゅうかん ご に おこないます	
\\	の 会議[かいぎ]は 2週間後[に しゅうかん ご]に 行[おこな]います。			
\\	運用	運用[うんよう]	うんよう	
\\	会計士に資金運用について相談したよ。	会計士[かいけいし]に 資金[しきん] 運用[うんよう]について 相談[そうだん]したよ。	かいけいし に しきん うんよう に ついて そうだん した よ	
\\	会計士[かいけいし]に 資金[しきん]
\\	について 相談[そうだん]したよ。			
\\	スピーカー	スピーカー	スピーカー	
\\	このスピーカーは音がいいですね。	このスピーカーは 音[おと]がいいですね。	この すぴーかー は おと が いい です ね	
\\	この
\\	は 音[おと]がいいですね。			
\\	運営	運営[うんえい]	うんえい	
\\	その事業は国が運営しています。	その 事業[じぎょう]は 国[くに]が 運営[うんえい]しています。	その じぎょう は くに が うんえい して います	
\\	その 事業[じぎょう]は 国[くに]が
\\	しています。			
\\	運賃	運賃[うんちん]	うんちん	
\\	鉄道の運賃が値上げされたね。	鉄道[てつどう]の 運賃[うんちん]が 値上[ねあ]げされたね。	てつどう の うんちん が ねあげ された ね	
\\	鉄道[てつどう]の
\\	が 値上[ねあ]げされたね。			
\\	運	運[うん]	うん	
\\	彼は運のいい男です。	彼[かれ]は 運[うん]のいい 男[おとこ]です。	かれ は うん の いい おとこ です	
\\	彼[かれ]は
\\	のいい 男[おとこ]です。			
\\	運送	運送[うんそう]	うんそう	
\\	彼は運送会社に勤めているの。	彼[かれ]は 運送[うんそう] 会社[がいしゃ]に 勤[つと]めているの。	かれ は うんそう がいしゃ に つとめて いる の	
\\	彼[かれ]は
\\	会社[がいしゃ]に 勤[つと]めているの。			
\\	回転	回転[かいてん]	かいてん	
\\	彼はボールに回転を掛けたんだ。	彼[かれ]はボールに 回転[かいてん]を 掛[か]けたんだ。	かれ は ぼーる に かいてん を かけた ん だ	
\\	彼[かれ]はボールに
\\	を 掛[か]けたんだ。			
\\	転がる	転[ころ]がる	ころがる	
\\	猫は砂の上で転がったんだ。	猫[ねこ]は 砂[すな]の 上[うえ]で 転[ころ]がったんだ。	ねこ は すな の うえ で ころがった ん だ	
\\	猫[ねこ]は 砂[すな]の 上[うえ]で
\\	んだ。			
\\	転がす	転[ころ]がす	ころがす	
\\	まず、肉をパン粉の上で転がします。	まず、 肉[にく]をパン 粉[こ]の 上[うえ]で 転[ころ]がします。	まず にく を ぱんこ の うえ で ころがします	
\\	まず、 肉[にく]をパン 粉[こ]の 上[うえ]で
\\	セールスマン	セールスマン	セールスマン	
\\	以前、セールスマンをしていたことがあります。	以前[いぜん]、セールスマンをしていたことがあります。	いぜん せーるすまん を して いた こと が あります	
\\	以前[いぜん]、
\\	をしていたことがあります。			
\\	移転	移転[いてん]	いてん	
\\	彼の会社は移転したよ。	彼[かれ]の 会社[かいしゃ]は 移転[いてん]したよ。	かれ の かいしゃ は いてん した よ	
\\	彼[かれ]の 会社[かいしゃ]は
\\	したよ。			
\\	移動	移動[いどう]	いどう	
\\	私たちはレンタカーで移動しました。	私[わたし]たちはレンタカーで 移動[いどう]しました。	わたしたち は れんたかー で いどう しました	
\\	私[わたし]たちはレンタカーで
\\	しました。			
\\	行動	行動[こうどう]	こうどう	
\\	彼の行動は理解できない。	彼[かれ]の 行動[こうどう]は 理解[りかい]できない。	かれ の こうどう は りかい できない	
\\	彼[かれ]の
\\	は 理解[りかい]できない。			
\\	運動会	運動会[うんどうかい]	うんどうかい	
\\	運動会で一位になったよ。	運動会[うんどうかい]で 一位[いちい]になったよ。	うんどうかい で いちい に なった よ	
\\	で 一位[いちい]になったよ。			
\\	運動場	運動場[うんどうじょう]	うんどうじょう	
\\	運動場でサッカーをしよう。	運動場[うんどうじょう]でサッカーをしよう。	うんどうじょう で さっかー を しよう	
\\	でサッカーをしよう。			
\\	運動場	運動場[うんどうじょう]	うんどうじょう	
\\	運動場でサッカーをしよう。	運動場[うんどうじょう]でサッカーをしよう。	うんどうじょう で さっかー を しよう	
\\	でサッカーをしよう。			
\\	高速	高速[こうそく]	こうそく	
\\	高速バスで東京に行ったんだ。	高速[こうそく]バスで 東京[とうきょう]に 行[い]ったんだ。	こうそくばす で とうきょう に いった ん だ	
\\	バスで 東京[とうきょう]に 行[い]ったんだ。			
\\	おしゃべり	おしゃべり	おしゃべり	
\\	彼女たちはおしゃべりに夢中です。	彼女[かのじょ]たちはおしゃべりに 夢中[むちゅう]です。	かのじょたち は おしゃべり に むちゅう です	
\\	彼女[かのじょ]たちは
\\	に 夢中[むちゅう]です。			
\\	急速	急速[きゅうそく]	きゅうそく	
\\	あの国の経済は急速に発展しているのね。	あの 国[くに]の 経済[けいざい]は 急速[きゅうそく]に 発展[はってん]しているのね。	あの くに の けいざい は きゅうそく に はってん して いる の ね	
\\	あの 国[くに]の 経済[けいざい]は
\\	に 発展[はってん]しているのね。			
\\	高速道路	高速道路[こうそくどうろ]	こうそくどうろ	
\\	高速道路は混んでたよ。	高速道路[こうそく どうろ]は 混[こ]んでたよ。	こうそく どうろ は こん でた よ 。	
\\	は 混[こ]んでたよ。			
\\	早速	早速[さっそく]	さっそく	
\\	では早速書類をお送りします。	では 早速[さっそく] 書類[しょるい]をお 送[おく]りします。	では さっそく しょるい を おおくり します	
\\	では
\\	書類[しょるい]をお 送[おく]りします。			
\\	時速	時速[じそく]	じそく	
\\	新幹線の最高時速は300キロです。	新幹線[しんかんせん]の 最高[さいこう] 時速[じそく]は 300[さんびゃく]キロです。	しんかんせん の さいこう じそく は さんびゃくきろ です	
\\	新幹線[しんかんせん]の 最高[さいこう]
\\	は 300[さんびゃく]キロです。			
\\	全速力	全速力[ぜんそくりょく]	ぜんそくりょく	
\\	久しぶりに全速力で走りました。	久[ひさ]しぶりに 全速力[ぜんそくりょく]で 走[はし]りました。	ひさしぶり に ぜんそくりょく で はしりました	
\\	久[ひさ]しぶりに
\\	で 走[はし]りました。			
\\	遅れ	遅[おく]れ	おくれ	
\\	電車が10分遅れで到着したんだ。	電車[でんしゃ]が 10分[じゅっ ぷん] 遅[おく]れで 到着[とうちゃく]したんだ。	でんしゃ が じゅっ ぷん おくれ で とうちゃく した ん だ	
\\	電車[でんしゃ]が 10分[じゅっ ぷん]
\\	で 到着[とうちゃく]したんだ。			
\\	遅らす	遅[おく]らす	おくらす	
\\	私たちは出発を1日遅らしたの。	私[わたし]たちは 出発[しゅっぱつ]を 1日[いちにち] 遅[おく]らしたの。	わたしたち は しゅっぱつ を いちにち おくらした の	
\\	私[わたし]たちは 出発[しゅっぱつ]を 1日[いちにち]
\\	の。			
\\	カラオケ	カラオケ	カラオケ	
\\	カラオケで思い切り歌ったの。	カラオケで 思[おも]い 切[き]り 歌[うた]ったの。	からおけ で おもいきり うたった の	
\\	で 思[おも]い 切[き]り 歌[うた]ったの。			
\\	終える	終[お]える	おえる	
\\	やっと宿題を終えた。	やっと 宿題[しゅくだい]を 終[お]えた。	やっと しゅくだい を おえた	
\\	やっと 宿題[しゅくだい]を
\\	最終	最終[さいしゅう]	さいしゅう	
\\	東京行きの最終電車は何時ですか。	東京行[とうきょうゆ]きの 最終[さいしゅう] 電車[でんしゃ]は 何時[なんじ]ですか。	とうきょうゆき の さいしゅう でんしゃ は なんじ です か	
\\	東京行[とうきょうゆ]きの
\\	電車[でんしゃ]は 何時[なんじ]ですか。			
\\	終点	終点[しゅうてん]	しゅうてん	
\\	電車の終点で降りました。	電車[でんしゃ]の 終点[しゅうてん]で 降[お]りました。	でんしゃ の しゅうてん で おりました	
\\	電車[でんしゃ]の
\\	で 降[お]りました。			
\\	始終	始終[しじゅう]	しじゅう	
\\	その部屋は始終、人が出入りしているね。	その 部屋[へや]は 始終[しじゅう]、 人[ひと]が 出入[でい]りしているね。	その へや は しじゅう ひと が でいり して いる ね	
\\	その 部屋[へや]は
\\	、 人[ひと]が 出入[でい]りしているね。			
\\	終電	終電[しゅうでん]	しゅうでん	
\\	終電にやっと間に合った。	終電[しゅうでん]にやっと 間[ま]に 合[あ]った。	しゅうでん に やっと まにあった	
\\	にやっと 間[ま]に 合[あ]った。			
\\	現地	現地[げんち]	げんち	
\\	現地の天気は雨です。	現地[げんち]の 天気[てんき]は 雨[あめ]です。	げんち の てんき は あめ です	
\\	の 天気[てんき]は 雨[あめ]です。			
\\	現代	現代[げんだい]	げんだい	
\\	現代の科学の進歩には驚くよ。	現代[げんだい]の 科学[かがく]の 進歩[しんぽ]には 驚[おどろ]くよ。	げんだい の かがく の しんぽ に は おどろく よ	
\\	の 科学[かがく]の 進歩[しんぽ]には 驚[おどろ]くよ。			
\\	現金	現金[げんきん]	げんきん	
\\	代金は現金でお願いします。	代金[だいきん]は 現金[げんきん]でお 願[ねが]いします。	だいきん は げんきん で おねがい します	
\\	代金[だいきん]は
\\	でお 願[ねが]いします。			
\\	こっそり	こっそり	こっそり	
\\	彼女はこっそりダイエットを始めたの。	彼女[かのじょ]はこっそりダイエットを 始[はじ]めたの。	かのじょ は こっそり だいえっと を はじめた の	
\\	彼女[かのじょ]は
\\	ダイエットを 始[はじ]めたの。			
\\	現れる	現[あらわ]れる	あらわれる	
\\	雲の間から太陽が現れました。	雲[くも]の 間[あいだ]から 太陽[たいよう]が 現[あらわ]れました。	くも の あいだ から たいよう が あらわれました	
\\	雲[くも]の 間[あいだ]から 太陽[たいよう]が
\\	現場	現場[げんば]	げんば	
\\	事故現場には入れません。	事故[じこ] 現場[げんば]には 入[はい]れません。	じこ げんば に は はいれません	
\\	事故[じこ]
\\	には 入[はい]れません。			
\\	現住所	現住所[げんじゅうしょ]	げんじゅうしょ	
\\	ここには現住所を書いてください。	ここには 現住所[げんじゅうしょ]を 書[か]いてください。	ここ に は げんじゅうしょ を かいて ください	
\\	ここには
\\	を 書[か]いてください。			
\\	現す	現[あらわ]す	あらわす	
\\	彼はようやく姿を現しましたね。	彼[かれ]はようやく 姿[すがた]を 現[あらわ]しましたね。	かれ は ようやく すがた を あらわしました ね	
\\	彼[かれ]はようやく 姿[すがた]を
\\	ね。			
\\	現れ	現[あらわ]れ	あらわれ	
\\	それは彼女の期待の現れです。	それは 彼女[かのじょ]の 期待[きたい]の 現[あらわ]れです。	それ は かのじょ の きたい の あらわれ です	
\\	それは 彼女[かのじょ]の 期待[きたい]の
\\	です。			
\\	在学	在学[ざいがく]	ざいがく	
\\	姉は大学に在学しています。	姉[あね]は 大学[だいがく]に 在学[ざいがく]しています。	あね は だいがく に ざいがく して います	
\\	姉[あね]は 大学[だいがく]に
\\	しています。			
\\	現実	現実[げんじつ]	げんじつ	
\\	現実は予想より厳しかった。	現実[げんじつ]は 予想[よそう]より 厳[きび]しかった。	げんじつ は よそう より きびしかった	
\\	は 予想[よそう]より 厳[きび]しかった。			
\\	ジョギング	ジョギング	ジョギング	
\\	私は毎日ジョギングをしています。	私[わたし]は 毎日[まいにち]ジョギングをしています。	わたし は まいにち じょぎんぐ を して います	
\\	私[わたし]は 毎日[まいにち]
\\	をしています。			
\\	実は	実[じつ]は	じつは	
\\	あれは実は私の勘違いでした。	あれは 実[じつ]は 私[わたし]の 勘違[かんちが]いでした。	あれ は じつは わたし の かんちがい でした	
\\	あれは
\\	私[わたし]の 勘違[かんちが]いでした。			
\\	実用	実用[じつよう]	じつよう	
\\	これはとても実用的なサイトだね。	これはとても 実用[じつよう] 的[てき]なサイトだね。	これ は とても じつようてき な さいと だ ね	
\\	これはとても
\\	的[てき]なサイトだね。			
\\	実力	実力[じつりょく]	じつりょく	
\\	二人の実力は互角です。	二人[ふたり]の 実力[じつりょく]は 互角[ごかく]です。	ふたり の じつりょく は ごかく です	
\\	二人[ふたり]の
\\	は 互角[ごかく]です。			
\\	実習	実習[じっしゅう]	じっしゅう	
\\	今日は料理の実習があった。	今日[きょう]は 料理[りょうり]の 実習[じっしゅう]があった。	きょう は りょうり の じっしゅう が あった	
\\	今日[きょう]は 料理[りょうり]の
\\	があった。			
\\	実物	実物[じつぶつ]	じつぶつ	
\\	何かの説明をする時は実物を使うとわかりやすいの。	何[なに]かの 説明[せつめい]をする 時[とき]は 実物[じつぶつ]を 使[つか]うとわかりやすいの。	なにか の せつめい を する とき は じつぶつ を つかう と わかり やすい の	
\\	何[なに]かの 説明[せつめい]をする 時[とき]は
\\	を 使[つか]うとわかりやすいの。			
\\	実に	実[じつ]に	じつに	
\\	昨日のコンサートは実に素晴らしかったよ。	昨日[さくじつ]のコンサートは 実[じつ]に 素晴[すば]らしかったよ。	さくじつ の こんさーと は じつに すばらしかった よ	
\\	昨日[さくじつ]のコンサートは
\\	素晴[すば]らしかったよ。			
\\	過ごす	過[す]ごす	すごす	
\\	夏休みを高原で過ごしました。	夏休[なつやす]みを 高原[こうげん]で 過[す]ごしました。	なつやすみ を こうげん で すごしました	
\\	夏休[なつやす]みを 高原[こうげん]で
\\	あれこれ	あれこれ	あれこれ	
\\	服を買うのにあれこれ迷いました。	服[ふく]を 買[か]うのにあれこれ 迷[まよ]いました。	ふく を かう の に あれこれ まよいました	
\\	服[ふく]を 買[か]うのに
\\	迷[まよ]いました。			
\\	経過	経過[けいか]	けいか	
\\	手術後の経過は順調です。	手術後[しゅじゅつ ご]の 経過[けいか]は 順調[じゅんちょう]です。	しゅじゅつ ご の けいか は じゅんちょう です	
\\	手術後[しゅじゅつ ご]の
\\	は 順調[じゅんちょう]です。			
\\	過ち	過[あやま]ち	あやまち	
\\	同じ過ちを繰り返さないことだ。	同[おな]じ 過[あやま]ちを 繰[く]り 返[かえ]さないことだ。	おなじ あやまち を くりかえさない こと だ	
\\	同[おな]じ
\\	を 繰[く]り 返[かえ]さないことだ。			
\\	去る	去[さ]る	さる	
\\	去る者は追わず。	去[さ]る 者[もの]は 追[お]わず。	さる もの は おわず	
\\	者[もの]は 追[お]わず。			
\\	活発	活発[かっぱつ]	かっぱつ	
\\	活発な意見が交されたの。	活発[かっぱつ]な 意見[いけん]が 交[かわ]されたの。	かっぱつ な いけん が かわされた の	
\\	な 意見[いけん]が 交[かわ]されたの。			
\\	始発	始発[しはつ]	しはつ	
\\	始発の電車に乗りました。	始発[しはつ]の 電車[でんしゃ]に 乗[の]りました。	しはつ の でんしゃ に のりました	
\\	の 電車[でんしゃ]に 乗[の]りました。			
\\	表す	表[あらわ]す	あらわす	
\\	私たちは万歳をして喜びを表しました。	私[わたし]たちは 万歳[ばんざい]をして 喜[よろこ]びを 表[あらわ]しました。	わたしたち は ばんざい を して よろこび を あらわしました	
\\	私[わたし]たちは 万歳[ばんざい]をして 喜[よろこ]びを
\\	言い表わす	言[い]い 表[あら]わす	いいあらわす	
\\	この気持ちをうまく言い表わせません。	この 気持[きも]ちをうまく 言[い]い 表[あら]わせません。	この きもち を うまく いいあらわせません	
\\	この 気持[きも]ちをうまく
\\	絵本	絵本[えほん]	えほん	
\\	その子は絵本が大好きです。	その 子[こ]は 絵本[えほん]が 大好[だいす]きです。	その こ は えほん が だいすき です	
\\	その 子[こ]は
\\	が 大好[だいす]きです。			
\\	およそ	およそ	およそ	
\\	家から駅までおよそ1キロあります。	家[いえ]から 駅[えき]までおよそ 1[いち]キロあります。	いえ から えき まで およそ いちきろ あります	
\\	家[いえ]から 駅[えき]まで
\\	1[いち]キロあります。			
\\	雑音	雑音[ざつおん]	ざつおん	
\\	雑音がひどくて声が聞こえません。	雑音[ざつおん]がひどくて 声[こえ]が 聞[き]こえません。	ざつおん が ひどくて こえ が きこえません	
\\	がひどくて 声[こえ]が 聞[き]こえません。			
\\	足音	足音[あしおと]	あしおと	
\\	廊下から大きな足音が聞こえた。	廊下[ろうか]から 大[おお]きな 足音[あしおと]が 聞[き]こえた。	ろうか から おおき な あしおと が きこえた	
\\	廊下[ろうか]から 大[おお]きな
\\	が 聞[き]こえた。			
\\	音読み	音読[おんよ]み	おんよみ	
\\	この漢字の音読みは何ですか。	この 漢字[かんじ]の 音読[おんよ]みは 何[なん]ですか。	この かんじ の おんよみ は なん です か	
\\	この 漢字[かんじ]の
\\	は 何[なん]ですか。			
\\	五十音	五十音[ごじゅうおん]	ごじゅうおん	
\\	名前は五十音順に並んでいます。	名前[なまえ]は 五十音[ごじゅうおん] 順[じゅん]に 並[なら]んでいます。	なまえ は ごじゅうおんじゅん に ならんで います	
\\	名前[なまえ]は
\\	順[じゅん]に 並[なら]んでいます。			
\\	気楽	気楽[きらく]	きらく	
\\	将来は気楽な生活がしたいです。	将来[しょうらい]は 気楽[きらく]な 生活[せいかつ]がしたいです。	しょうらい は きらく な せいかつ が したい です	
\\	将来[しょうらい]は
\\	な 生活[せいかつ]がしたいです。			
\\	行楽	行楽[こうらく]	こうらく	
\\	秋は行楽にいい季節ですね。	秋[あき]は 行楽[こうらく]にいい 季節[きせつ]ですね。	あき は こうらく に いい きせつ です ね	
\\	秋[あき]は
\\	にいい 季節[きせつ]ですね。			
\\	薬指	薬指[くすりゆび]	くすりゆび	
\\	彼女は薬指に指輪をはめています。	彼女[かのじょ]は 薬指[くすりゆび]に 指輪[ゆびわ]をはめています。	かのじょ は くすりゆび に ゆびわ を はめて います	
\\	彼女[かのじょ]は
\\	に 指輪[ゆびわ]をはめています。			
\\	サークル	サークル	サークル	
\\	どのサークルに入るか迷っちゃった。	どのサークルに 入[はい]るか 迷[まよ]っちゃった。	どの さーくる に はいる か まよっちゃった	
\\	どの
\\	に 入[はい]るか 迷[まよ]っちゃった。			
\\	食欲	食欲[しょくよく]	しょくよく	
\\	今日は食欲がありません。	今日[きょう]は 食欲[しょくよく]がありません。	きょう は しょくよく が ありません	
\\	今日[きょう]は
\\	がありません。			
\\	映る	映[うつ]る	うつる	
\\	水面に月が映っているね。	水面[みなも]に 月[つき]が 映[うつ]っているね。	みなも に つき が うつって いる ね	
\\	水面[みなも]に 月[つき]が
\\	ね。			
\\	映す	映[うつ]す	うつす	
\\	彼女は自分の姿を鏡に映したの。	彼女[かのじょ]は 自分[じぶん]の 姿[すがた]を 鏡[かがみ]に 映[うつ]したの。	かのじょ は じぶん の すがた を かがみ に うつした の	
\\	彼女[かのじょ]は 自分[じぶん]の 姿[すがた]を 鏡[かがみ]に
\\	の。			
\\	企画	企画[きかく]	きかく	
\\	独身者パーティーを企画した。	独身者[どくしんしゃ]パーティーを 企画[きかく]した。	どくしんしゃ ぱーてぃー を きかく した	
\\	独身者[どくしんしゃ]パーティーを
\\	した。			
\\	画家	画家[がか]	がか	
\\	この絵はスペインの画家が描きました。	この 絵[え]はスペインの 画家[がか]が 描[か]きました。	この え は すぺいん の がか が かきました	
\\	この 絵[え]はスペインの
\\	が 描[か]きました。			
\\	区画	区画[くかく]	くかく	
\\	あそこの3区画は売り出し中です。	あそこの 3[さん] 区画[くかく]は 売[う]り 出[だ]し 中[ちゅう]です。	あそこ の さんくかく は うりだしちゅう です	
\\	あそこの 3[さん]
\\	は 売[う]り 出[だ]し 中[ちゅう]です。			
\\	画面	画面[がめん]	がめん	
\\	テレビの画面が明るすぎる。	テレビの 画面[がめん]が 明[あか]るすぎる。	てれび の がめん が あかる すぎる	
\\	テレビの
\\	が 明[あか]るすぎる。			
\\	キリスト	キリスト	キリスト	
\\	キリストの母親の名前はマリアだ。	キリストの 母親[ははおや]の 名前[なまえ]はマリアだ。	きりすと の ははおや の なまえ は まりあ だ	
\\	の 母親[ははおや]の 名前[なまえ]はマリアだ。			
\\	地面	地面[じめん]	じめん	
\\	地面に何か絵が描いてあるぞ。	地面[じめん]に 何[なに]か 絵[え]が 描[か]いてあるぞ。	じめん に なにか え が かいて ある ぞ	
\\	に 何[なに]か 絵[え]が 描[か]いてあるぞ。			
\\	水面	水面[すいめん]	すいめん	
\\	湖の水面に小さく波が立っているな。	湖[みずうみ]の 水面[すいめん]に 小[ちい]さく 波[なみ]が 立[た]っているな。	みずうみ の すいめん に ちいさく なみ が たって いる な	
\\	湖[みずうみ]の
\\	に 小[ちい]さく 波[なみ]が 立[た]っているな。			
\\	一面	一面[いちめん]	いちめん	
\\	外は一面の雪景色だったね。	外[そと]は 一面[いちめん]の 雪景色[ゆきげしき]だったね。	そと は いちめん の ゆきげしき だった ね	
\\	外[そと]は
\\	の 雪景色[ゆきげしき]だったね。			
\\	真実	真実[しんじつ]	しんじつ	
\\	真実は一つです。	真実[しんじつ]は 一[ひと]つです。	しんじつ は ひとつ です	
\\	は 一[ひと]つです。			
\\	真理	真理[しんり]	しんり	
\\	彼は人生の真理を求めて旅に出た。	彼[かれ]は 人生[じんせい]の 真理[しんり]を 求[もと]めて 旅[たび]に 出[で]た。	かれ は じんせい の しんり を もとめて たび に でた	
\\	彼[かれ]は 人生[じんせい]の
\\	を 求[もと]めて 旅[たび]に 出[で]た。			
\\	金色	金色[きんいろ]	きんいろ	
\\	あの寺の屋根は金色ですね。	あの 寺[てら]の 屋根[やね]は 金色[きんいろ]ですね。	あの てら の やね は きんいろ です ね	
\\	あの 寺[てら]の 屋根[やね]は
\\	ですね。			
\\	形式	形式[けいしき]	けいしき	
\\	書類は形式を守って作ってください。	書類[しょるい]は 形式[けいしき]を 守[まも]って 作[つく]ってください。	しょるい は けいしき を まもって つくって ください	
\\	書類[しょるい]は
\\	を 守[まも]って 作[つく]ってください。			
\\	小型	小型[こがた]	こがた	
\\	小型のスーツケースを買いました。	小型[こがた]のスーツケースを 買[か]いました。	こがた の すーつけーす を かいました	
\\	のスーツケースを 買[か]いました。			
\\	あっさり	あっさり	あっさり	
\\	そうあっさり言わないで。	そうあっさり 言[い]わないで。	そう あっさり いわない で	
\\	そう
\\	言[い]わないで。			
\\	型	型[かた]	かた	
\\	新しい型のカメラを買いました。	新[あたら]しい 型[かた]のカメラを 買[か]いました。	あたらしい かた の かめら を かいました	
\\	新[あたら]しい
\\	のカメラを 買[か]いました。			
\\	一種	一種[いっしゅ]	いっしゅ	
\\	それは果物の一種です。	それは 果物[くだもの]の 一種[いっしゅ]です。	それ は くだもの の いっしゅ です	
\\	それは 果物[くだもの]の
\\	です。			
\\	各種	各種[かくしゅ]	かくしゅ	
\\	図書館には各種の雑誌が揃っています。	図書館[としょかん]には 各種[かくしゅ]の 雑誌[ざっし]が 揃[そろ]っています。	としょかん に は かくしゅ の ざっし が そろって います	
\\	図書館[としょかん]には
\\	の 雑誌[ざっし]が 揃[そろ]っています。			
\\	一種	一種[いっしゅ]	いっしゅ	
\\	これは一種独特の味がするね。	これは 一種[いっしゅ] 独特[どくとく]の 味[あじ]がするね。	これ は いっしゅ どくとく の あじ が する ね	
\\	これは
\\	独特[どくとく]の 味[あじ]がするね。			
\\	人種	人種[じんしゅ]	じんしゅ	
\\	この国には様々な人種がいます。	この 国[くに]には 様々[さまざま]な 人種[じんしゅ]がいます。	この くに に は さまざま な じんしゅ が います	
\\	この 国[くに]には 様々[さまざま]な
\\	がいます。			
\\	種目	種目[しゅもく]	しゅもく	
\\	その選手は2種目でメダルを取ったわ。	その 選手[せんしゅ]は 2[に] 種目[しゅもく]でメダルを 取[と]ったわ。	その せんしゅ は にしゅもく で めだる を とった わ	
\\	その 選手[せんしゅ]は 2[に]
\\	でメダルを 取[と]ったわ。			
\\	人類	人類[じんるい]	じんるい	
\\	人類は2本の足で歩きます。	人類[じんるい]は 2本[にほん]の 足[あし]で 歩[ある]きます。	じんるい は にほん の あし で あるきます	
\\	は 2本[にほん]の 足[あし]で 歩[ある]きます。			
\\	せっせと	せっせと	せっせと	
\\	彼は朝から晩までせっせと働いたの。	彼[かれ]は 朝[あさ]から 晩[ばん]までせっせと 働[はたら]いたの。	かれ は あさ から ばん まで せっせと はたらいた の	
\\	彼[かれ]は 朝[あさ]から 晩[ばん]まで
\\	働[はたら]いたの。			
\\	書類	書類[しょるい]	しょるい	
\\	書類を10枚コピーしました。	書類[しょるい]を 10枚[じゅうまい]コピーしました。	しょるい を じゅうまい こぴー しました	
\\	を 10枚[じゅうまい]コピーしました。			
\\	接近	接近[せっきん]	せっきん	
\\	台風が接近していますね。	台風[たいふう]が 接近[せっきん]していますね。	たいふう が せっきん して います ね	
\\	台風[たいふう]が
\\	していますね。			
\\	間接	間接[かんせつ]	かんせつ	
\\	それは間接的な原因の一つです。	それは 間接[かんせつ] 的[てき]な 原因[げんいん]の 一[ひと]つです。	それ は かんせつてき な げんいん の ひとつ です	
\\	それは
\\	的[てき]な 原因[げんいん]の 一[ひと]つです。			
\\	接する	接[せっ]する	せっする	
\\	子供が動物に接するのは良いことです。	子供[こども]が 動物[どうぶつ]に 接[せっ]するのは 良[よ]いことです。	こども が どうぶつ に せっする の は よい こと です	
\\	子供[こども]が 動物[どうぶつ]に
\\	のは 良[よ]いことです。			
\\	角度	角度[かくど]	かくど	
\\	この角度からは画面が見えにくいですね。	この 角度[かくど]からは 画面[がめん]が 見[み]えにくいですね。	この かくど からは がめん が みえ にくい です ね	
\\	この
\\	からは 画面[がめん]が 見[み]えにくいですね。			
\\	三角形	三角形[さんかくけい]	さんかくけい	
\\	この三角形の面積を出しなさい。	この 三角形[さんかくけい]の 面積[めんせき]を 出[だ]しなさい。	この さんかくけい の めんせき を だしなさい	
\\	この
\\	の 面積[めんせき]を 出[だ]しなさい。			
\\	曲線	曲線[きょくせん]	きょくせん	
\\	この道は、ゆるい曲線を描いているよ。	この 道[みち]は、ゆるい 曲線[きょくせん]を 描[えが]いているよ。	この みち は ゆるい きょくせん を えがいて いる よ	
\\	この 道[みち]は、ゆるい
\\	を 描[えが]いているよ。			
\\	カタログ	カタログ	カタログ	
\\	カタログを見て注文したんだ。	カタログを 見[み]て 注文[ちゅうもん]したんだ。	かたろぐ を みて ちゅうもん した ん だ	
\\	を 見[み]て 注文[ちゅうもん]したんだ。			
\\	作曲	作曲[さっきょく]	さっきょく	
\\	この曲は誰が作曲したのですか。	この 曲[きょく]は 誰[だれ]が 作曲[さっきょく]したのですか。	この きょく は だれ が さっきょく した の です か	
\\	この 曲[きょく]は 誰[だれ]が
\\	したのですか。			
\\	共通	共通[きょうつう]	きょうつう	
\\	私たちは共通の趣味を持っています。	私[わたし]たちは 共通[きょうつう]の 趣味[しゅみ]を 持[も]っています。	わたしたち は きょうつう の しゅみ を もって います	
\\	私[わたし]たちは
\\	の 趣味[しゅみ]を 持[も]っています。			
\\	共通語	共通語[きょうつうご]	きょうつうご	
\\	インドでは英語は共通語だよ。	インドでは 英語[えいご]は 共通語[きょうつうご]だよ。	いんど で は えいご は きょうつうご だ よ	
\\	インドでは 英語[えいご]は
\\	だよ。			
\\	共同	共同[きょうどう]	きょうどう	
\\	2社が共同で新製品を開発したんだ。	2社[にしゃ]が 共同[きょうどう]で 新製品[しんせいひん]を 開発[かいはつ]したんだ。	にしゃ が きょうどう で しんせいひん を かいはつ した ん だ	
\\	2社[にしゃ]が
\\	で 新製品[しんせいひん]を 開発[かいはつ]したんだ。			
\\	合同	合同[ごうどう]	ごうどう	
\\	3社合同で新作の発表会を開きました。	3社[さんしゃ] 合同[ごうどう]で 新作[しんさく]の 発表会[はっぴょうかい]を 開[ひら]きました。	さんしゃ ごうどう で しんさく の はっぴょうかい を ひらきました	
\\	3社[さんしゃ]
\\	で 新作[しんさく]の 発表会[はっぴょうかい]を 開[ひら]きました。			
\\	一同	一同[いちどう]	いちどう	
\\	一同顔を見合わせました。	一同[いちどう] 顔[かお]を 見合[みあ]わせました。	いちどう かお を みあわせました	
\\	顔[かお]を 見合[みあ]わせました。			
\\	以前	以前[いぜん]	いぜん	
\\	以前彼はこの町に住んでいたんだ。	以前[いぜん] 彼[かれ]はこの 町[まち]に 住[す]んでいたんだ。	いぜん かれ は この まち に すんで いた ん だ	
\\	彼[かれ]はこの 町[まち]に 住[す]んでいたんだ。			
\\	ジャズ	ジャズ	ジャズ	
\\	昨日の夜、ジャズを聴きに行きました。	昨日[きのう]の 夜[よる]、ジャズを 聴[き]きに 行[い]きました。	きのう の よる じゃず を きき に いきました	
\\	昨日[きのう]の 夜[よる]、
\\	を 聴[き]きに 行[い]きました。			
\\	以下	以下[いか]	いか	
\\	数学が平均点以下だった。	数学[すうがく]が 平均点[へいきんてん] 以下[いか]だった。	すうがく が へいきんてん いか だった	
\\	数学[すうがく]が 平均点[へいきんてん]
\\	だった。			
\\	以降	以降[いこう]	いこう	
\\	夕方以降にお電話を下さい。	夕方[ゆうがた] 以降[いこう]にお 電話[でんわ]を 下[くだ]さい。	ゆうがた いこう に お でんわ を ください	
\\	夕方[ゆうがた]
\\	にお 電話[でんわ]を 下[くだ]さい。			
\\	以来	以来[いらい]	いらい	
\\	それ以来彼女に会ってない。	それ 以来[いらい] 彼女[かのじょ]に 会[あ]ってない。	それ いらい かのじょ に あってない	
\\	それ
\\	彼女[かのじょ]に 会[あ]ってない。			
\\	以後	以後[いご]	いご	
\\	すみません、以後気を付けます。	すみません、 以後[いご] 気[き]を 付[つ]けます。	すみません いご き を つけます	
\\	すみません、
\\	気[き]を 付[つ]けます。			
\\	下宿	下宿[げしゅく]	げしゅく	
\\	親類の家に4年間下宿しました。	親類[しんるい]の 家[いえ]に 4年間[よねんかん] 下宿[げしゅく]しました。	しんるい の いえ に よねんかん げしゅく しました	
\\	親類[しんるい]の 家[いえ]に 4年間[よねんかん]
\\	しました。			
\\	漢和	漢和[かんわ]	かんわ	
\\	漢和辞典で漢字の意味を調べたの。	漢和[かんわ] 辞典[じてん]で 漢字[かんじ]の 意味[いみ]を 調[しら]べたの。	かんわ じてん で かんじ の いみ を しらべた の	
\\	辞典[じてん]で 漢字[かんじ]の 意味[いみ]を 調[しら]べたの。			
\\	英和	英和[えいわ]	えいわ	
\\	英和辞書をよく使います。	英和[えいわ] 辞書[じしょ]をよく 使[つか]います。	えいわ じしょ を よく つかいます	
\\	辞書[じしょ]をよく 使[つか]います。			
\\	西洋	西洋[せいよう]	せいよう	
\\	私は西洋の建築に興味があります。	私[わたし]は 西洋[せいよう]の 建築[けんちく]に 興味[きょうみ]があります。	わたし は せいよう の けんちく に きょうみ が あります	
\\	私[わたし]は
\\	の 建築[けんちく]に 興味[きょうみ]があります。			
\\	サイン	サイン	サイン	
\\	ここにサインしてください。	ここにサインしてください。	ここ に さいん して ください	
\\	ここに
\\	してください。			
\\	海洋	海洋[かいよう]	かいよう	
\\	その昔航海士達は未踏の地を求め海洋に乗り出しました。	その 昔航海士達[むかし こうかいしたち]は 未踏[みとう]の 地[ち]を 求[もと]め 海洋[かいよう]に 乗[の]り 出[だ]しました。	その むかし こうかいしたち は みとう の ち を もとめ かいよう に のりだしました	
\\	その 昔航海士達[むかし こうかいしたち]は 未踏[みとう]の 地[ち]を 求[もと]め
\\	に 乗[の]り 出[だ]しました。			
\\	西洋人	西洋人[せいようじん]	せいようじん	
\\	西洋人は正座が苦手よ。	西洋人[せいようじん]は 正座[せいざ]が 苦手[にがて]よ。	せいようじん は せいざ が にがて よ	
\\	は 正座[せいざ]が 苦手[にがて]よ。			
\\	制服	制服[せいふく]	せいふく	
\\	私の学校には制服がありません。	私[わたし]の 学校[がっこう]には 制服[せいふく]がありません。	わたし の がっこう に は せいふく が ありません	
\\	私[わたし]の 学校[がっこう]には
\\	がありません。			
\\	室内	室内[しつない]	しつない	
\\	雨の日は子供を室内で遊ばせます。	雨[あめ]の 日[ひ]は 子供[こども]を 室内[しつない]で 遊[あそ]ばせます。	あめ の ひ は こども を しつない で あそばせます	
\\	雨[あめ]の 日[ひ]は 子供[こども]を
\\	で 遊[あそ]ばせます。			
\\	親子	親子[おやこ]	おやこ	
\\	池に鴨の親子がいます。	池[いけ]に 鴨[かも]の 親子[おやこ]がいます。	いけ に かも の おやこ が います	
\\	池[いけ]に 鴨[かも]の
\\	がいます。			
\\	親類	親類[しんるい]	しんるい	
\\	彼は幼い時に親類に預けられたんだ。	彼[かれ]は 幼[おさな]い 時[とき]に 親類[しんるい]に 預[あず]けられたんだ。	かれ は おさない とき に しんるい に あずけられた ん だ	
\\	彼[かれ]は 幼[おさな]い 時[とき]に
\\	に 預[あず]けられたんだ。			
\\	親しむ	親[した]しむ	したしむ	
\\	ハイキングは自然に親しむ良い機会ですよ。	ハイキングは 自然[しぜん]に 親[した]しむ 良[い]い 機会[きかい]ですよ。	はいきんぐ は しぜん に したしむ いい きかい です よ	
\\	ハイキングは 自然[しぜん]に
\\	良[い]い 機会[きかい]ですよ。			
\\	ごまかす	ごまかす	ごまかす	
\\	彼は質問の答えをごまかしたね。	彼[かれ]は 質問[しつもん]の 答[こた]えをごまかしたね。	かれ は しつもん の こたえ を ごまかした ね	
\\	彼[かれ]は 質問[しつもん]の 答[こた]えを
\\	ね。			
\\	親切	親切[しんせつ]	しんせつ	
\\	ご親切は決して忘れません。	ご 親切[しんせつ]は 決[けっ]して 忘[わす]れません。	ごしんせつ は けっして わすれません	
\\	ご
\\	は 決[けっ]して 忘[わす]れません。			
\\	親友	親友[しんゆう]	しんゆう	
\\	彼は私の親友です。	彼[かれ]は 私[わたし]の 親友[しんゆう]です。	かれ は わたし の しんゆう です	
\\	彼[かれ]は 私[わたし]の
\\	です。			
\\	親指	親指[おやゆび]	おやゆび	
\\	親指を怪我しました。	親指[おやゆび]を 怪我[けが]しました。	おやゆび を けがしました	
\\	を 怪我[けが]しました。			
\\	親しみ	親[した]しみ	したしみ	
\\	彼には誰もが親しみを感じます。	彼[かれ]には 誰[だれ]もが 親[した]しみを 感[かん]じます。	かれ に は だれ も が したしみ を かんじます	
\\	彼[かれ]には 誰[だれ]もが
\\	を 感[かん]じます。			
\\	水族館	水族館[すいぞくかん]	すいぞくかん	
\\	ここの水族館にはイルカがいます。	ここの 水族館[すいぞくかん]にはイルカがいます。	ここ の すいぞくかん に は いるか が います	
\\	ここの
\\	にはイルカがいます。			
\\	歳末	歳末[さいまつ]	さいまつ	
\\	デパートの歳末大売出しが始まったよ。	デパートの 歳末[さいまつ] 大売出[おおうりだ]しが 始[はじ]まったよ。	でぱーと の さいまつ おおうりだし が はじまった よ	
\\	デパートの
\\	大売出[おおうりだ]しが 始[はじ]まったよ。			
\\	姉妹	姉妹[しまい]	しまい	
\\	うちは3人姉妹です。	うちは 3人[さんにん] 姉妹[しまい]です。	うち は さんにん しまい です	
\\	うちは 3人[さんにん]
\\	です。			
\\	インテリ	インテリ	インテリ	
\\	彼女はインテリで、しかも美人ですね。	彼女[かのじょ]はインテリで、しかも 美人[びじん]ですね。	かのじょ は いんてり で しかも びじん です ね	
\\	彼女[かのじょ]は
\\	で、しかも 美人[びじん]ですね。			
\\	次第に	次第[しだい]に	しだいに	
\\	その事件は次第に忘れられていったのさ。	その 事件[じけん]は 次第[しだい]に 忘[わす]れられていったのさ。	その じけん は しだいに わすれられて いった の さ	
\\	その 事件[じけん]は
\\	忘[わす]れられていったのさ。			
\\	次第	次第[しだい]	しだい	
\\	連絡があり次第出発します。	連絡[れんらく]があり 次第[しだい] 出発[しゅっぱつ]します。	れんらく が あり しだい しゅっぱつ します	
\\	連絡[れんらく]があり
\\	出発[しゅっぱつ]します。			
\\	息	息[いき]	いき	
\\	大きく息を吸ってください。	大[おお]きく 息[いき]を 吸[す]ってください。	おおきく いき を すって ください	
\\	大[おお]きく
\\	を 吸[す]ってください。			
\\	休息	休息[きゅうそく]	きゅうそく	
\\	休息をとることは大切です。	休息[きゅうそく]をとることは 大切[たいせつ]です。	きゅうそく を とる こと は たいせつ です	
\\	をとることは 大切[たいせつ]です。			
\\	消費者	消費者[しょうひしゃ]	しょうひしゃ	
\\	この法律は消費者を守るためのものです。	この 法律[ほうりつ]は 消費者[しょうひしゃ]を 守[まも]るためのものです。	この ほうりつ は しょうひしゃ を まもる ため の もの です	
\\	この 法律[ほうりつ]は
\\	を 守[まも]るためのものです。			
\\	学者	学者[がくしゃ]	がくしゃ	
\\	彼は作家であり学者です。	彼[かれ]は 作家[さっか]であり 学者[がくしゃ]です。	かれ は さっか で あり がくしゃ です	
\\	彼[かれ]は 作家[さっか]であり
\\	です。			
\\	後者	後者[こうしゃ]	こうしゃ	
\\	正しい答は後者です。	正[ただ]しい 答[こたえ]は 後者[こうしゃ]です。	ただしい こたえ は こうしゃ です	
\\	正[ただ]しい 答[こたえ]は
\\	です。			
\\	作者	作者[さくしゃ]	さくしゃ	
\\	この本の作者は誰ですか。	この 本[ほん]の 作者[さくしゃ]は 誰[だれ]ですか。	この ほん の さくしゃ は だれ です か	
\\	この 本[ほん]の
\\	は 誰[だれ]ですか。			
\\	いちいち	いちいち	いちいち	
\\	彼はいちいち私に指図する。	彼[かれ]はいちいち 私[わたし]に 指図[さしず]する。	かれ は いちいち わたし に さしず する	
\\	彼[かれ]は
\\	私[わたし]に 指図[さしず]する。			
\\	前者	前者[ぜんしゃ]	ぜんしゃ	
\\	私は前者の方が優れていると思います。	私[わたし]は 前者[ぜんしゃ]の 方[ほう]が 優[すぐ]れていると 思[おも]います。	わたし は ぜんしゃ の ほう が すぐれて いる と おもいます	
\\	私[わたし]は
\\	の 方[ほう]が 優[すぐ]れていると 思[おも]います。			
\\	結局	結局[けっきょく]	けっきょく	
\\	彼は結局何を言いたかったのだろう。	彼[かれ]は 結局[けっきょく] 何[なに]を 言[い]いたかったのだろう。	かれ は けっきょく なに を いいたかった の だろう	
\\	彼[かれ]は
\\	何[なに]を 言[い]いたかったのだろう。			
\\	新婚	新婚[しんこん]	しんこん	
\\	妹夫婦は新婚です。	妹夫婦[いもうとふうふ]は 新婚[しんこん]です。	いもうとふうふ は しんこん です	
\\	妹夫婦[いもうとふうふ]は
\\	です。			
\\	婚約	婚約[こんやく]	こんやく	
\\	二人は婚約しています。	二人[ふたり]は 婚約[こんやく]しています。	ふたり は こんやく して います	
\\	二人[ふたり]は
\\	しています。			
\\	課長	課長[かちょう]	かちょう	
\\	彼は課長に昇進しました。	彼[かれ]は 課長[かちょう]に 昇進[しょうしん]しました。	かれ は かちょう に しょうしん しました	
\\	彼[かれ]は
\\	に 昇進[しょうしん]しました。			
\\	効く	効[き]く	きく	
\\	この薬はあまりよく効かないよ。	この 薬[くすり]はあまりよく 効[き]かないよ。	この くすり は あまり よく きかない よ	
\\	この 薬[くすり]はあまりよく
\\	よ。			
\\	効力	効力[こうりょく]	こうりょく	
\\	その薬は効力が強いよ。	その 薬[くすり]は 効力[こうりょく]が 強[つよ]いよ。	その くすり は こうりょく が つよい よ	
\\	その 薬[くすり]は
\\	が 強[つよ]いよ。			
\\	ストライキ	ストライキ	ストライキ	
\\	従業員は明日からストライキです。	従業員[じゅうぎょういん]は 明日[あす]からストライキです。	じゅうぎょういん は あす から すとらいき です	
\\	従業員[じゅうぎょういん]は 明日[あす]から
\\	です。			
\\	効き目	効[き]き 目[め]	ききめ	
\\	この薬の効き目は素晴らしいの。	この 薬[くすり]の 効[き]き 目[め]は 素晴[すば]らしいの。	この くすり の ききめ は すばらしい の	
\\	この 薬[くすり]の
\\	は 素晴[すば]らしいの。			
\\	自動	自動[じどう]	じどう	
\\	このドアは自動よ。	このドアは 自動[じどう]よ。	この どあ は じどう よ	
\\	このドアは
\\	よ。			
\\	自国	自国[じこく]	じこく	
\\	自国の文化を大切にしましょう。	自国[じこく]の 文化[ぶんか]を 大切[たいせつ]にしましょう。	じこく の ぶんか を たいせつ に しましょう	
\\	の 文化[ぶんか]を 大切[たいせつ]にしましょう。			
\\	各自	各自[かくじ]	かくじ	
\\	ごみは各自で持ち帰ってください。	ごみは 各自[かくじ]で 持[も]ち 帰[かえ]ってください。	ごみ は かくじ で もちかえって ください	
\\	ごみは
\\	で 持[も]ち 帰[かえ]ってください。			
\\	自然に	自然[しぜん]に	しぜんに	
\\	硬くならないで、自然に話して下さい。	硬[かた]くならないで、 自然[しぜん]に 話[はな]して 下[くだ]さい。	かたく ならない で しぜんに はなして ください	
\\	硬[かた]くならないで、
\\	話[はな]して 下[くだ]さい。			
\\	経由	経由[けいゆ]	けいゆ	
\\	メールを経由して広がるウィルスもあります。	メールを 経由[けいゆ]して 広[ひろ]がるウィルスもあります。	めーる を けいゆ して ひろがる うぃるす も あります	
\\	メールを
\\	して 広[ひろ]がるウィルスもあります。			
\\	自信	自信[じしん]	じしん	
\\	あの人の顔には自信があふれているね。	あの 人[ひと]の 顔[かお]には 自信[じしん]があふれているね。	あの ひと の かお に は じしん が あふれて いる ね	
\\	あの 人[ひと]の 顔[かお]には
\\	があふれているね。			
\\	あんまり	あんまり	あんまり	
\\	これはあんまり好きじゃないな。	これはあんまり 好[す]きじゃないな。	これ は あんまり すき じゃ ない な	
\\	これは
\\	好[す]きじゃないな。			
\\	信用	信用[しんよう]	しんよう	
\\	彼の言うことは信用できない。	彼[かれ]の 言[い]うことは 信用[しんよう]できない。	かれ の いう こと は しんよう できない	
\\	彼[かれ]の 言[い]うことは
\\	できない。			
\\	信頼	信頼[しんらい]	しんらい	
\\	私は部下を信頼しています。	私[わたし]は 部下[ぶか]を 信頼[しんらい]しています。	わたし は ぶか を しんらい して います	
\\	私[わたし]は 部下[ぶか]を
\\	しています。			
\\	依頼	依頼[いらい]	いらい	
\\	彼に協力を依頼しました。	彼[かれ]に 協力[きょうりょく]を 依頼[いらい]しました。	かれ に きょうりょく を いらい しました	
\\	彼[かれ]に 協力[きょうりょく]を
\\	しました。			
\\	人民	人民[じんみん]	じんみん	
\\	彼は人民のための政治を行った。	彼[かれ]は 人民[じんみん]のための 政治[せいじ]を 行[おこな]った。	かれ は じんみん の ため の せいじ を おこなった	
\\	彼[かれ]は
\\	のための 政治[せいじ]を 行[おこな]った。			
\\	国民	国民[こくみん]	こくみん	
\\	国民の安全が最も大切です。	国民[こくみん]の 安全[あんぜん]が 最[もっと]も 大切[たいせつ]です。	こくみん の あんぜん が もっとも たいせつ です	
\\	の 安全[あんぜん]が 最[もっと]も 大切[たいせつ]です。			
\\	主要	主要[しゅよう]	しゅよう	
\\	産業はインドの主要産業よね。	
\\	産業[あいてぃー さんぎょう]はインドの 主要[しゅよう] 産業[さんぎょう]よね。	あいてぃー さんぎょう は いんど の しゅよう さんぎょう よ ね	
\\	産業[あいてぃー さんぎょう]はインドの
\\	産業[さんぎょう]よね。			
\\	主に	主[おも]に	おもに	
\\	この商品は主に女性に人気がありまして。	この 商品[しょうひん]は 主[おも]に 女性[じょせい]に 人気[にんき]がありまして。	この しょうひん は おもに じょせい に にんき が ありまし て	
\\	この 商品[しょうひん]は
\\	女性[じょせい]に 人気[にんき]がありまして。			
\\	主役	主役[しゅやく]	しゅやく	
\\	彼は学校の劇で主役に選ばれたよ。	彼[かれ]は 学校[がっこう]の 劇[げき]で 主役[しゅやく]に 選[えら]ばれたよ。	かれ は がっこう の げき で しゅやく に えらばれた よ	
\\	彼[かれ]は 学校[がっこう]の 劇[げき]で
\\	に 選[えら]ばれたよ。			
\\	うるさい	うるさい	うるさい	
\\	あまりうるさく言わないで下さい。	あまりうるさく 言[い]わないで 下[くだ]さい。	あまり うるさく いわない で ください	
\\	あまり
\\	言[い]わないで 下[くだ]さい。			
\\	主体	主体[しゅたい]	しゅたい	
\\	政治は国民主体であるべきです。	政治[せいじ]は 国民[こくみん] 主体[しゅたい]であるべきです。	せいじ は こくみん しゅたい で ある べき です	
\\	政治[せいじ]は 国民[こくみん]
\\	であるべきです。			
\\	主題	主題[しゅだい]	しゅだい	
\\	この小説の主題は反戦です。	この 小説[しょうせつ]の 主題[しゅだい]は 反戦[はんせん]です。	この しょうせつ の しゅだい は はんせん です	
\\	この 小説[しょうせつ]の
\\	は 反戦[はんせん]です。			
\\	自主	自主[じしゅ]	じしゅ	
\\	その選手は自主トレーニングを始めたの。	その 選手[せんしゅ]は 自主[じしゅ]トレーニングを 始[はじ]めたの。	その せんしゅ は じしゅとれーにんぐ を はじめた の	
\\	その 選手[せんしゅ]は
\\	トレーニングを 始[はじ]めたの。			
\\	主食	主食[しゅしょく]	しゅしょく	
\\	アジアの主食は米です。	アジアの 主食[しゅしょく]は 米[こめ]です。	あじあ の しゅしょく は こめ です	
\\	アジアの
\\	は 米[こめ]です。			
\\	主	主[おも]	おも	
\\	彼女の主な仕事は接客だよ。	彼女[かのじょ]の 主[おも]な 仕事[しごと]は 接客[せっきゃく]だよ。	かのじょ の おも な しごと は せっきゃく だ よ	
\\	彼女[かのじょ]の
\\	な 仕事[しごと]は 接客[せっきゃく]だよ。			
\\	主義	主義[しゅぎ]	しゅぎ	
\\	私は車は持たない主義です。	私[わたし]は 車[くるま]は 持[も]たない 主義[しゅぎ]です。	わたし は くるま は もたない しゅぎ です	
\\	私[わたし]は 車[くるま]は 持[も]たない
\\	です。			
\\	共産主義	共産主義[きょうさんしゅぎ]	きょうさんしゅぎ	
\\	その国は共産主義の国だよ。	その 国[くに]は 共産主義[きょうさんしゅぎ]の 国[くに]だよ。	その くに は きょうさんしゅぎ の くに だ よ	
\\	その 国[くに]は
\\	の 国[くに]だよ。			
\\	カーブ	カーブ	カーブ	
\\	このカーブは気をつけて。	このカーブは 気[き]をつけて。	この かーぶ は き を つけて	
\\	この
\\	は 気[き]をつけて。			
\\	義理	義理[ぎり]	ぎり	
\\	彼は義理を大切にしています。	彼[かれ]は 義理[ぎり]を 大切[たいせつ]にしています。	かれ は ぎり を たいせつ に して います	
\\	彼[かれ]は
\\	を 大切[たいせつ]にしています。			
\\	議会	議会[ぎかい]	ぎかい	
\\	議会で区の来年の予定が話し合われたんだ。	議会[ぎかい]で 区[く]の 来年[らいねん]の 予定[よてい]が 話[はな]し 合[あ]われたんだ。	ぎかい で く の らいねん の よてい が はなしあわれた ん だ 。	
\\	で 区[く]の 来年[らいねん]の 予定[よてい]が 話[はな]し 合[あ]われたんだ。			
\\	協議	協議[きょうぎ]	きょうぎ	
\\	委員会はその問題について協議しましたわ。	委員会[いいんかい]はその 問題[もんだい]について 協議[きょうぎ]しましたわ。	いいんかい は その もんだい に ついて きょうぎ しました わ	
\\	委員会[いいんかい]はその 問題[もんだい]について
\\	しましたわ。			
\\	議長	議長[ぎちょう]	ぎちょう	
\\	議長が開会の挨拶をしました。	議長[ぎちょう]が 開会[かいかい]の 挨拶[あいさつ]をしました。	ぎちょう が かいかい の あいさつ を しました	
\\	が 開会[かいかい]の 挨拶[あいさつ]をしました。			
\\	議題	議題[ぎだい]	ぎだい	
\\	今日の議題は面白そうです。	今日[きょう]の 議題[ぎだい]は 面白[おもしろ]そうです。	きょう の ぎだい は おもしろ そう です	
\\	今日[きょう]の
\\	は 面白[おもしろ]そうです。			
\\	議論	議論[ぎろん]	ぎろん	
\\	父は議論好きです。	父[ちち]は 議論[ぎろん] 好[ず]きです。	ちち は ぎろんずき です	
\\	父[ちち]は
\\	好[ず]きです。			
\\	結論	結論[けつろん]	けつろん	
\\	今日の話し合いでは結論が出なかった。	今日[きょう]の 話[はな]し 合[あ]いでは 結論[けつろん]が 出[で]なかった。	きょう の はなしあい で は けつろん が でなかった	
\\	今日[きょう]の 話[はな]し 合[あ]いでは
\\	が 出[で]なかった。			
\\	ガイド	ガイド	ガイド	
\\	ガイドつきの旅行を申し込んだの。	ガイドつきの 旅行[りょこう]を 申[もう]し 込[こ]んだの。	がいど つき の りょこう を もうしこんだ の	
\\	つきの 旅行[りょこう]を 申[もう]し 込[こ]んだの。			
\\	言論	言論[げんろん]	げんろん	
\\	言論の自由は国民の権利です。	言論[げんろん]の 自由[じゆう]は 国民[こくみん]の 権利[けんり]です。	げんろん の じゆう は こくみん の けんり です	
\\	の 自由[じゆう]は 国民[こくみん]の 権利[けんり]です。			
\\	世論	世論[せろん]	せろん	
\\	世論はあなたの味方です。	世論[せろん]はあなたの 味方[みかた]です。	せろん は あなた の みかた です	
\\	はあなたの 味方[みかた]です。			
\\	違反	違反[いはん]	いはん	
\\	彼は校則に違反しました。	彼[かれ]は 校則[こうそく]に 違反[いはん]しました。	かれ は こうそく に いはん しました	
\\	彼[かれ]は 校則[こうそく]に
\\	しました。			
\\	応じる	応[おう]じる	おうじる	
\\	私は彼の要望に応じました。	私[わたし]は 彼[かれ]の 要望[ようぼう]に 応[おう]じました。	わたし は かれ の ようぼう に おうじました	
\\	私[わたし]は 彼[かれ]の 要望[ようぼう]に
\\	応用	応用[おうよう]	おうよう	
\\	このレシピはいろいろ応用できます。	このレシピはいろいろ 応用[おうよう]できます。	この れしぴ は いろいろ おうよう できます	
\\	このレシピはいろいろ
\\	できます。			
\\	一応	一応[いちおう]	いちおう	
\\	彼にも一応知らせておいたほうがいいだろう。	彼[かれ]にも 一応[いちおう] 知[し]らせておいたほうがいいだろう。	かれ に も いちおう しらせて おいた ほう が いいだろう	
\\	彼[かれ]にも
\\	知[し]らせておいたほうがいいだろう。			
\\	回答	回答[かいとう]	かいとう	
\\	アンケートに回答しました。	アンケートに 回答[かいとう]しました。	あんけーと に かいとう しました	
\\	アンケートに
\\	しました。			
\\	かえる	かえる	かえる	
\\	田んぼの中でかえるが合唱している。	田[た]んぼの 中[なか]でかえるが 合唱[がっしょう]している。	たんぼ の なか で かえる が がっしょう している 。	
\\	田[た]んぼの 中[なか]で
\\	が 合唱[がっしょう]している。			
\\	解答	解答[かいとう]	かいとう	
\\	インターネットで試験の解答を確認したんだ。	インターネットで 試験[しけん]の 解答[かいとう]を 確認[かくにん]したんだ。	いんたーねっと で しけん の かいとう を かくにん した ん だ	
\\	インターネットで 試験[しけん]の
\\	を 確認[かくにん]したんだ。			
\\	区別	区別[くべつ]	くべつ	
\\	あの二人はそっくりで区別できない。	あの 二人[ふたり]はそっくりで 区別[くべつ]できない。	あの ふたり は そっくり で くべつ できない	
\\	あの 二人[ふたり]はそっくりで
\\	できない。			
\\	差別	差別[さべつ]	さべつ	
\\	彼は差別をなくす運動をしています。	彼[かれ]は 差別[さべつ]をなくす 運動[うんどう]をしています。	かれ は さべつ を なくす うんどう を して います	
\\	彼[かれ]は
\\	をなくす 運動[うんどう]をしています。			
\\	性別	性別[せいべつ]	せいべつ	
\\	出席者を性別で分けてください。	出席者[しゅっせきしゃ]を 性別[せいべつ]で 分[わ]けてください。	しゅっせきしゃ を せいべつ で わけて ください	
\\	出席者[しゅっせきしゃ]を
\\	で 分[わ]けてください。			
\\	一般に	一般[いっぱん]に	いっぱんに	
\\	一般に老人は早起きです。	一般[いっぱん]に 老人[ろうじん]は 早起[はやお]きです。	いっぱんに ろうじん は はやおき です	
\\	老人[ろうじん]は 早起[はやお]きです。			
\\	経済的	経済的[けいざいてき]	けいざいてき	
\\	経済的な効果は計り知れません。	経済的[けいざいてき]な 効果[こうか]は 計[はか]り 知[し]れません。	けいざいてき な こうか は はかり しれません	
\\	な 効果[こうか]は 計[はか]り 知[し]れません。			
\\	自主的	自主的[じしゅてき]	じしゅてき	
\\	学生たちは自主的に勉強会を開いたわよ。	学生[がくせい]たちは 自主的[じしゅてき]に 勉強会[べんきょうかい]を 開[ひら]いたわよ。	がくせいたち は じしゅてき に べんきょうかい を ひらいた わ よ	
\\	学生[がくせい]たちは
\\	に 勉強会[べんきょうかい]を 開[ひら]いたわよ。			
\\	合理的	合理的[ごうりてき]	ごうりてき	
\\	彼女は合理的な考え方をする人です。	彼女[かのじょ]は 合理的[ごうりてき]な 考[かんが]え 方[かた]をする 人[ひと]です。	かのじょ は ごうりてき な かんがえかた を する ひと です	
\\	彼女[かのじょ]は
\\	な 考[かんが]え 方[かた]をする 人[ひと]です。			
\\	ずれる	ずれる	ずれる	
\\	ポスターの位置が少しずれているよ。	ポスターの 位置[いち]が 少[すこ]しずれているよ。	ぽすたー の いち が すこし ずれて いる よ	
\\	ポスターの 位置[いち]が 少[すこ]し
\\	よ。			
\\	形式的	形式的[けいしきてき]	けいしきてき	
\\	彼は形式的な質問をしただけだった。	彼[かれ]は 形式的[けいしきてき]な 質問[しつもん]をしただけだった。	かれ は けいしきてき な しつもん を した だけ だった	
\\	彼[かれ]は
\\	な 質問[しつもん]をしただけだった。			
\\	実用的	実用的[じつようてき]	じつようてき	
\\	彼の発明品はみな実用的だよ。	彼[かれ]の 発明品[はつめいひん]はみな 実用的[じつようてき]だよ。	かれ の はつめいひん は みな じつようてき だ よ	
\\	彼[かれ]の 発明品[はつめいひん]はみな
\\	だよ。			
\\	女性的	女性的[じょせいてき]	じょせいてき	
\\	彼は言葉遣いが少し女性的だね。	彼[かれ]は 言葉遣[ことばづか]いが 少[すこ]し 女性的[じょせいてき]だね。	かれ は ことばづかい が すこし じょせいてき だ ね	
\\	彼[かれ]は 言葉遣[ことばづか]いが 少[すこ]し
\\	だね。			
\\	水平	水平[すいへい]	すいへい	
\\	この棚は水平になっていませんね。	この 棚[たな]は 水平[すいへい]になっていませんね。	この たな は すいへい に なって いません ね	
\\	この 棚[たな]は
\\	になっていませんね。			
\\	水平線	水平線[すいへいせん]	すいへいせん	
\\	水平線に太陽が沈んでいった。	水平線[すいへいせん]に 太陽[たいよう]が 沈[しず]んでいった。	すいへいせん に たいよう が しずんで いった	
\\	に 太陽[たいよう]が 沈[しず]んでいった。			
\\	高等	高等[こうとう]	こうとう	
\\	人間は高等な生物と言われている。	人間[にんげん]は 高等[こうとう]な 生物[せいぶつ]と 言[い]われている。	にんげん は こうとう な せいぶつ と いわれて いる	
\\	人間[にんげん]は
\\	な 生物[せいぶつ]と 言[い]われている。			
\\	上等	上等[じょうとう]	じょうとう	
\\	上等なワインを飲んだの。	上等[じょうとう]なワインを 飲[の]んだの。	じょうとう な わいん を のんだ の	
\\	なワインを 飲[の]んだの。			
\\	オーバー	オーバー	オーバー	
\\	彼の話はいつもオーバーです。	彼[かれ]の 話[はなし]はいつもオーバーです。	かれ の はなし は いつも おーばー です	
\\	彼[かれ]の 話[はなし]はいつも
\\	です。			
\\	急病	急病[きゅうびょう]	きゅうびょう	
\\	知り合いが急病で倒れた。	知[し]り 合[あ]いが 急病[きゅうびょう]で 倒[たお]れた。	しりあい が きゅうびょう で たおれた	
\\	知[し]り 合[あ]いが
\\	で 倒[たお]れた。			
\\	医学	医学[いがく]	いがく	
\\	彼は医学を学んでいます。	彼[かれ]は 医学[いがく]を 学[まな]んでいます。	かれ は いがく を まなんで います	
\\	彼[かれ]は
\\	を 学[まな]んでいます。			
\\	科学者	科学者[かがくしゃ]	かがくしゃ	
\\	彼はとても有名な科学者だ。	彼[かれ]はとても 有名[ゆうめい]な 科学者[かがくしゃ]だ。	かれ は とても ゆうめい な かがくしゃ だ	
\\	彼[かれ]はとても 有名[ゆうめい]な
\\	だ。			
\\	科目	科目[かもく]	かもく	
\\	今日は3科目のテストを受けたよ。	今日[きょう]は 3[さん] 科目[かもく]のテストを 受[う]けたよ。	きょう は さんかもく の てすと を うけた よ	
\\	今日[きょう]は 3[さん]
\\	のテストを 受[う]けたよ。			
\\	学科	学科[がっか]	がっか	
\\	彼の得意な学科は数学です。	彼[かれ]の 得意[とくい]な 学科[がっか]は 数学[すうがく]です。	かれ の とくい な がっか は すうがく です	
\\	彼[かれ]の 得意[とくい]な
\\	は 数学[すうがく]です。			
\\	外科	外科[げか]	げか	
\\	友人が骨折して外科に入院しました。	友人[ゆうじん]が 骨折[こっせつ]して 外科[げか]に 入院[にゅういん]しました。	ゆうじん が こっせつ して げか に にゅういん しました	
\\	友人[ゆうじん]が 骨折[こっせつ]して
\\	に 入院[にゅういん]しました。			
\\	歯科	歯科[しか]	しか	
\\	彼は歯科医師です。	彼[かれ]は 歯科[しか] 医師[いし]です。	かれ は しか いし です	
\\	彼[かれ]は
\\	医師[いし]です。			
\\	エアコン	エアコン	エアコン	
\\	部屋にエアコンを取り付けたんだ。	部屋[へや]にエアコンを 取[と]り 付[つ]けたんだ。	へや に えあこん を とりつけた ん だ	
\\	部屋[へや]に
\\	を 取[と]り 付[つ]けたんだ。			
\\	自然科学	自然科学[しぜんかがく]	しぜんかがく	
\\	姉は大学で自然科学を学んでいます。	姉[あね]は 大学[だいがく]で 自然科学[しぜんかがく]を 学[まな]んでいます。	あね は だいがく で しぜんかがく を まなんで います	
\\	姉[あね]は 大学[だいがく]で
\\	を 学[まな]んでいます。			
\\	人文科学	人文科学[じんぶんかがく]	じんぶんかがく	
\\	大学で人文科学を専攻しました。	大学[だいがく]で 人文科学[じんぶんかがく]を 専攻[せんこう]しました。	だいがく で じんぶんかがく を せんこう しました	
\\	大学[だいがく]で
\\	を 専攻[せんこう]しました。			
\\	社会科学	社会科学[しゃかいかがく]	しゃかいかがく	
\\	社会科学の分野に関心があります。	社会科学[しゃかいかがく]の 分野[ぶんや]に 関心[かんしん]があります。	しゃかいかがく の ぶんや に かんしん が あります	
\\	の 分野[ぶんや]に 関心[かんしん]があります。			
\\	死	死[し]	し	
\\	死を恐れるのは自然なことです。	死[し]を 恐[おそ]れるのは 自然[しぜん]なことです。	し を おそれる の は しぜん な こと です	
\\	を 恐[おそ]れるのは 自然[しぜん]なことです。			
\\	死者	死者[ししゃ]	ししゃ	
\\	その事故で30人の死者が出たの。	その 事故[じこ]で 30人[さんじゅうにん]の 死者[ししゃ]が 出[で]たの。	その じこ で さんじゅうにん の ししゃ が でた の	
\\	その 事故[じこ]で 30人[さんじゅうにん]の
\\	が 出[で]たの。			
\\	死体	死体[したい]	したい	
\\	公園で死体が見つかったの。	公園[こうえん]で 死体[したい]が 見[み]つかったの。	こうえん で したい が みつかった の	
\\	公園[こうえん]で
\\	が 見[み]つかったの。			
\\	生死	生死[せいし]	せいし	
\\	これは人の生死にかかわる問題です。	これは 人[ひと]の 生死[せいし]にかかわる 問題[もんだい]です。	これ は ひと の せいし に かかわる もんだい で す	
\\	これは 人[ひと]の
\\	にかかわる 問題[もんだい]です。			
\\	死亡	死亡[しぼう]	しぼう	
\\	その事故では2人死亡したの。	その 事故[じこ]では 2人[ふたり] 死亡[しぼう]したの。	その じこ で は ふたり しぼう した の	
\\	その 事故[じこ]では 2人[ふたり]
\\	したの。			
\\	しょっちゅう	しょっちゅう	しょっちゅう	
\\	彼はしょっちゅう遅刻しているな。	彼[かれ]はしょっちゅう 遅刻[ちこく]しているな。	かれ は しょっちゅう ちこく して いる な	
\\	彼[かれ]は
\\	遅刻[ちこく]しているな。			
\\	痛み	痛[いた]み	いたみ	
\\	背中に痛みがあります。	背中[せなか]に 痛[いた]みがあります。	せなか に いたみ が あります	
\\	背中[せなか]に
\\	があります。			
\\	痛む	痛[いた]む	いたむ	
\\	虫歯がずきずき痛みます。	虫歯[むしば]がずきずき 痛[いた]みます。	むしば が ずきずき いたみます	
\\	虫歯[むしば]がずきずき
\\	禁止	禁止[きんし]	きんし	
\\	交差点付近は駐車禁止です。	交差点付近[こうさてん ふきん]は 駐車[ちゅうしゃ] 禁止[きんし]です。	こうさてん ふきん は ちゅうしゃ きんし です	
\\	交差点付近[こうさてん ふきん]は 駐車[ちゅうしゃ]
\\	です。			
\\	禁じる	禁[きん]じる	きんじる	
\\	市は昨年から歩きタバコを禁じているの。	市[し]は 昨年[さくねん]から 歩[ある]きタバコを 禁[きん]じているの。	し は さくねん から あるきたばこ を きんじて いる の	
\\	市[し]は 昨年[さくねん]から 歩[ある]きタバコを
\\	の。			
\\	煙	煙[けむり]	けむり	
\\	煙突から煙がまっすぐ上っていたの。	煙突[えんとつ]から 煙[けむり]がまっすぐ 上[あが]っていたの。	えんとつ から けむり が まっすぐ あがって いた の	
\\	煙突[えんとつ]から
\\	がまっすぐ 上[あが]っていたの。			
\\	酒屋	酒屋[さかや]	さかや	
\\	そこの酒屋さんでビールを買って来て。	そこの 酒屋[さかや]さんでビールを 買[か]って 来[き]て。	そこ の さかやさん で びーる を かって きて	
\\	そこの
\\	さんでビールを 買[か]って 来[き]て。			
\\	禁酒	禁酒[きんしゅ]	きんしゅ	
\\	彼は禁酒しています。	彼[かれ]は 禁酒[きんしゅ]しています。	かれ は きんしゅ して います	
\\	彼[かれ]は
\\	しています。			
\\	スタンド	スタンド	スタンド	
\\	姉はスタンドの下で本を読んでいたんだ。	姉[あね]はスタンドの 下[した]で 本[ほん]を 読[よ]んでいたんだ。	あね は すたんど の した で ほん を よんで いた ん だ	
\\	姉[あね]は
\\	の 下[した]で 本[ほん]を 読[よ]んでいたんだ。			
\\	険しい	険[けわ]しい	けわしい	
\\	父の表情が険しくなったの。	父[ちち]の 表情[ひょうじょう]が 険[けわ]しくなったの。	ちち の ひょうじょう が けわしく なった の	
\\	父[ちち]の 表情[ひょうじょう]が
\\	の。			
\\	証券	証券[しょうけん]	しょうけん	
\\	証券を売って資金にしようと思うの。	証券[しょうけん]を 売[う]って 資金[しきん]にしようと 思[おも]うの。	しょうけん を うって しきん に しよう と おもう の	
\\	を 売[う]って 資金[しきん]にしようと 思[おも]うの。			
\\	証明	証明[しょうめい]	しょうめい	
\\	容疑者のアリバイが証明されました。	容疑者[ようぎしゃ]のアリバイが 証明[しょうめい]されました。	ようぎしゃ の ありばい が しょうめい されました	
\\	容疑者[ようぎしゃ]のアリバイが
\\	されました。			
\\	生存	生存[せいぞん]	せいぞん	
\\	乗客は全員生存しています。	乗客[じょうきゃく]は 全員[ぜんいん] 生存[せいぞん]しています。	じょうきゃく は ぜんいん せいぞん して います	
\\	乗客[じょうきゃく]は 全員[ぜんいん]
\\	しています。			
\\	ご存じ	ご 存[ぞん]じ	ごぞんじ	
\\	彼の名前をご存じですか。	彼[かれ]の 名前[なまえ]をご 存[ぞん]じですか。	かれ の なまえ を ごぞんじ です か	
\\	彼[かれ]の 名前[なまえ]を
\\	ですか。			
\\	合意	合意[ごうい]	ごうい	
\\	両社が合併に合意しました。	両社[りょうしゃ]が 合併[がっぺい]に 合意[ごうい]しました。	りょうしゃ が がっぺい に ごうい しました	
\\	両社[りょうしゃ]が 合併[がっぺい]に
\\	しました。			
\\	意向	意向[いこう]	いこう	
\\	彼の意向を聞いてみましょう。	彼[かれ]の 意向[いこう]を 聞[き]いてみましょう。	かれ の いこう を きいて みましょう	
\\	彼[かれ]の
\\	を 聞[き]いてみましょう。			
\\	がっかり	がっかり	がっかり	
\\	成績が落ちてがっかりした。	成績[せいせき]が 落[お]ちてがっかりした。	せいせき が おちて がっかり した	
\\	成績[せいせき]が 落[お]ちて
\\	した。			
\\	意欲	意欲[いよく]	いよく	
\\	彼は仕事に意欲を燃やしています。	彼[かれ]は 仕事[しごと]に 意欲[いよく]を 燃[も]やしています。	かれ は しごと に いよく を もやして います	
\\	彼[かれ]は 仕事[しごと]に
\\	を 燃[も]やしています。			
\\	決意	決意[けつい]	けつい	
\\	彼の決意は堅いな。	彼[かれ]の 決意[けつい]は 堅[かた]いな。	かれ の けつい は かたい な	
\\	彼[かれ]の
\\	は 堅[かた]いな。			
\\	意図	意図[いと]	いと	
\\	あなたの意図はよく分かりました。	あなたの 意図[いと]はよく 分[わ]かりました。	あなた の いと は よく わかりました	
\\	あなたの
\\	はよく 分[わ]かりました。			
\\	意外	意外[いがい]	いがい	
\\	意外にも彼は独身です。	意外[いがい]にも 彼[かれ]は 独身[どくしん]です。	いがい に も かれ は どくしん です	
\\	にも 彼[かれ]は 独身[どくしん]です。			
\\	意義	意義[いぎ]	いぎ	
\\	この事業には大きな意義があります。	この 事業[じぎょう]には 大[おお]きな 意義[いぎ]があります。	この じぎょう に は おおき な いぎ が あります	
\\	この 事業[じぎょう]には 大[おお]きな
\\	があります。			
\\	好意	好意[こうい]	こうい	
\\	彼は彼女に好意をもっています。	彼[かれ]は 彼女[かのじょ]に 好意[こうい]をもっています。	かれ は かのじょ に こうい を もって います	
\\	彼[かれ]は 彼女[かのじょ]に
\\	をもっています。			
\\	意地悪	意地悪[いじわる]	いじわる	
\\	彼は時々意地悪な質問をする。	彼[かれ]は 時々[ときどき] 意地悪[いじわる]な 質問[しつもん]をする。	かれ は ときどき いじわる な しつもん を する 。	
\\	彼[かれ]は 時々[ときどき]
\\	な 質問[しつもん]をする。			
\\	確保	確保[かくほ]	かくほ	
\\	優れた人材の確保は重要です。	優[すぐ]れた 人材[じんざい]の 確保[かくほ]は 重要[じゅうよう]です。	すぐれた じんざい の かくほ は じゅうよう です	
\\	優[すぐ]れた 人材[じんざい]の
\\	は 重要[じゅうよう]です。			
\\	スライド	スライド	スライド	
\\	このふたはスライドします。	このふたはスライドします。	この ふた は すらいど します	
\\	このふたは
\\	します。			
\\	確立	確立[かくりつ]	かくりつ	
\\	北欧では社会保障が確立されています。	北欧[ほくおう]では 社会保障[しゃかい ほしょう]が 確立[かくりつ]されています。	ほくおう で は しゃかい ほしょう が かくりつ されて います	
\\	北欧[ほくおう]では 社会保障[しゃかい ほしょう]が
\\	されています。			
\\	確定	確定[かくてい]	かくてい	
\\	試合の代表メンバーが確定したの。	試合[しあい]の 代表[だいひょう]メンバーが 確定[かくてい]したの。	しあい の だいひょう めんばー が かくてい した の	
\\	試合[しあい]の 代表[だいひょう]メンバーが
\\	したの。			
\\	確実	確実[かくじつ]	かくじつ	
\\	彼女が将来、大統領になることは確実です。	彼女[かのじょ]が 将来[しょうらい]、 大統領[だいとうりょう]になることは 確実[かくじつ]です。	かのじょ が しょうらい だいとうりょう に なる こと は かくじつ です	
\\	彼女[かのじょ]が 将来[しょうらい]、 大統領[だいとうりょう]になることは
\\	です。			
\\	確信	確信[かくしん]	かくしん	
\\	私は彼の成功を確信しています。	私[わたし]は 彼[かれ]の 成功[せいこう]を 確信[かくしん]しています。	わたし は かれ の せいこう を かくしん して います	
\\	私[わたし]は 彼[かれ]の 成功[せいこう]を
\\	しています。			
\\	効率	効率[こうりつ]	こうりつ	
\\	作業の効率を上げるにはどうしたらいいですか。	作業[さぎょう]の 効率[こうりつ]を 上[あ]げるにはどうしたらいいですか。	さぎょう の こうりつ を あげる に は どう したら いい です か	
\\	作業[さぎょう]の
\\	を 上[あ]げるにはどうしたらいいですか。			
\\	確率	確率[かくりつ]	かくりつ	
\\	天気予報は当たる確率が高くなりましたね。	天気予報[てんき よほう]は 当[あ]たる 確率[かくりつ]が 高[たか]くなりましたね。	てんき よほう は あたる かくりつ が たかく なりました ね	
\\	天気予報[てんき よほう]は 当[あ]たる
\\	が 高[たか]くなりましたね。			
\\	軽率	軽率[けいそつ]	けいそつ	
\\	軽率な発言はしないよう気をつけなさい。	軽率[けいそつ]な 発言[はつげん]はしないよう 気[き]をつけなさい。	けいそつ な はつげん は しない よう き を つけなさい	
\\	な 発言[はつげん]はしないよう 気[き]をつけなさい。			
\\	おっしゃる	おっしゃる	おっしゃる	
\\	あなたのおっしゃる通りです。	あなたのおっしゃる 通[とお]りです。	あなた の おっしゃる とおり です	
\\	あなたの
\\	通[とお]りです。			
\\	機会	機会[きかい]	きかい	
\\	家族で話し合う機会を持ちました。	家族[かぞく]で 話[はな]し 合[あ]う 機会[きかい]を 持[も]ちました。	かぞく で はなしあう きかい を もちました	
\\	家族[かぞく]で 話[はな]し 合[あ]う
\\	を 持[も]ちました。			
\\	危機	危機[きき]	きき	
\\	地球環境の危機が叫ばれているのよ。	地球環境[ちきゅう かんきょう]の 危機[きき]が 叫[さけ]ばれているのよ。	ちきゅう かんきょう の きき が さけばれて いる の よ	
\\	地球環境[ちきゅう かんきょう]の
\\	が 叫[さけ]ばれているのよ。			
\\	機長	機長[きちょう]	きちょう	
\\	機長の放送があったの。	機長[きちょう]の 放送[ほうそう]があったの。	きちょう の ほうそう が あった の	
\\	の 放送[ほうそう]があったの。			
\\	時機	時機[じき]	じき	
\\	あせらずに時機を待つべきです。	あせらずに 時機[じき]を 待[ま]つべきです。	あせらず に じき を まつ べき です	
\\	あせらずに
\\	を 待[ま]つべきです。			
\\	楽器	楽器[がっき]	がっき	
\\	私が好きな楽器はギターです。	私[わたし]が 好[す]きな 楽器[がっき]はギターです。	わたし が すき な がっき は ぎたー です	
\\	私[わたし]が 好[す]きな
\\	はギターです。			
\\	食器	食器[しょっき]	しょっき	
\\	食器を全部新しくしました。	食器[しょっき]を 全部新[ぜんぶ あたら]しくしました。	しょっき を ぜんぶ あたらしく しました	
\\	を 全部新[ぜんぶ あたら]しくしました。			
\\	器用	器用[きよう]	きよう	
\\	彼はかなり器用な人です。	彼[かれ]はかなり 器用[きよう]な 人[ひと]です。	かれ は かなり きよう な ひと です	
\\	彼[かれ]はかなり
\\	な 人[ひと]です。			
\\	クラシック	クラシック	クラシック	
\\	彼女はクラシック音楽が好きです。	彼女[かのじょ]はクラシック 音楽[おんがく]が 好[す]きです。	かのじょ は くらしっく おんがく が すき です	
\\	彼女[かのじょ]は
\\	音楽[おんがく]が 好[す]きです。			
\\	受話器	受話器[じゅわき]	じゅわき	
\\	受話器を取ってもらえますか。	受話器[じゅわき]を 取[と]ってもらえますか。	じゅわき を とって もらえます か	
\\	を 取[と]ってもらえますか。			
\\	器	器[うつわ]	うつわ	
\\	この器は上等ね。	この 器[うつわ]は 上等[じょうとう]ね。	この うつわ は じょうとう ね	
\\	この
\\	は 上等[じょうとう]ね。			
\\	消火器	消火器[しょうかき]	しょうかき	
\\	消火器を交換しました。	消火器[しょうかき]を 交換[こうかん]しました。	しょうかき を こうかん しました	
\\	を 交換[こうかん]しました。			
\\	器械	器械[きかい]	きかい	
\\	体育館で器械を使って運動したの。	体育館[たいいくかん]で 器械[きかい]を 使[つか]って 運動[うんどう]したの。	たいいくかん で きかい を つかって うんどう した の	
\\	体育館[たいいくかん]で
\\	を 使[つか]って 運動[うんどう]したの。			
\\	取材	取材[しゅざい]	しゅざい	
\\	テレビも取材に来たほど有名なレストランです。	テレビも 取材[しゅざい]に 来[き]たほど 有名[ゆうめい]なレストランです。	てれび も しゅざい に きた ほど ゆうめい な れすとらん です	
\\	テレビも
\\	に 来[き]たほど 有名[ゆうめい]なレストランです。			
\\	材木	材木[ざいもく]	ざいもく	
\\	船から材木が降ろされていますね。	船[ふね]から 材木[ざいもく]が 降[お]ろされていますね。	ふね から ざいもく が おろされて います ね	
\\	船[ふね]から
\\	が 降[お]ろされていますね。			
\\	具合	具合[ぐあい]	ぐあい	
\\	今日は体の具合が悪いです。	今日[きょう]は 体[からだ]の 具合[ぐあい]が 悪[わる]いです。	きょう は からだ の ぐあい が わるい です	
\\	今日[きょう]は 体[からだ]の
\\	が 悪[わる]いです。			
\\	さっと	さっと	さっと	
\\	こぼれた牛乳をさっと拭き取ったの。	こぼれた 牛乳[ぎゅうにゅう]をさっと 拭[ふ]き 取[と]ったの。	こぼれた ぎゅうにゅう を さっと ふきとった の	
\\	こぼれた 牛乳[ぎゅうにゅう]を
\\	拭[ふ]き 取[と]ったの。			
\\	器具	器具[きぐ]	きぐ	
\\	これはスポーツ施設用の器具です。	これはスポーツ 施設用[しせつよう]の 器具[きぐ]です。	これ は すぽーつ しせつよう の きぐ です	
\\	これはスポーツ 施設用[しせつよう]の
\\	です。			
\\	家具	家具[かぐ]	かぐ	
\\	部屋の家具を動かしました。	部屋[へや]の 家具[かぐ]を 動[うご]かしました。	へや の かぐ を うごかしました	
\\	部屋[へや]の
\\	を 動[うご]かしました。			
\\	雨具	雨具[あまぐ]	あまぐ	
\\	雨具の用意を忘れないように。	雨具[あまぐ]の 用意[ようい]を 忘[わす]れないように。	あまぐ の ようい を わすれない よう に	
\\	の 用意[ようい]を 忘[わす]れないように。			
\\	基地	基地[きち]	きち	
\\	ここは昔、軍事基地でした。	ここは 昔[むかし]、 軍事[ぐんじ] 基地[きち]でした。	ここ は むかし ぐんじ きち でした	
\\	ここは 昔[むかし]、 軍事[ぐんじ]
\\	でした。			
\\	水準	水準[すいじゅん]	すいじゅん	
\\	今年の応募作品は水準が高かったね。	今年[ことし]の 応募作品[おうぼ さくひん]は 水準[すいじゅん]が 高[たか]かったね。	ことし の おうぼ さくひん は すいじゅん が たかかった ね	
\\	今年[ことし]の 応募作品[おうぼ さくひん]は
\\	が 高[たか]かったね。			
\\	基準	基準[きじゅん]	きじゅん	
\\	この建物は建築の基準に達していないよ。	この 建物[たてもの]は 建築[けんちく]の 基準[きじゅん]に 達[たっ]していないよ。	この たてもの は けんちく の きじゅん に たっしていない よ 。	
\\	この 建物[たてもの]は 建築[けんちく]の
\\	に 達[たっ]していないよ。			
\\	準急	準急[じゅんきゅう]	じゅんきゅう	
\\	新宿駅まで準急で行ったのよ。	新宿駅[しんじゅくえき]まで 準急[じゅんきゅう]で 行[い]ったのよ。	しんじゅくえき まで じゅんきゅう で いった の よ	
\\	新宿駅[しんじゅくえき]まで
\\	で 行[い]ったのよ。			
\\	設立	設立[せつりつ]	せつりつ	
\\	当社は10年前に設立されたのよ。	当社[とうしゃ]は 10年前[じゅうねんまえ]に 設立[せつりつ]されたのよ。	とうしゃ は じゅうねんまえ に せつりつ された の よ	
\\	当社[とうしゃ]は 10年前[じゅうねんまえ]に
\\	されたのよ。			
\\	キャンセル	キャンセル	キャンセル	
\\	予約をキャンセルしたよ。	予約[よやく]をキャンセルしたよ。	よやく を きゃんせる した よ	
\\	予約[よやく]を
\\	したよ。			
\\	設備	設備[せつび]	せつび	
\\	この研究所には最新の設備が揃っています。	この 研究所[けんきゅうじょ]には 最新[さいしん]の 設備[せつび]が 揃[そろ]っています。	この けんきゅうじょ に は さいしん の せつび が そろって います	
\\	この 研究所[けんきゅうじょ]には 最新[さいしん]の
\\	が 揃[そろ]っています。			
\\	説	説[せつ]	せつ	
\\	私は彼の説が正しいと思う。	私[わたし]は 彼[かれ]の 説[せつ]が 正[ただ]しいと 思[おも]う。	わたし は かれ の せつ が ただしい と おもう	
\\	私[わたし]は 彼[かれ]の
\\	が 正[ただ]しいと 思[おも]う。			
\\	解説	解説[かいせつ]	かいせつ	
\\	新聞の解説欄を読みました。	新聞[しんぶん]の 解説[かいせつ] 欄[らん]を 読[よ]みました。	しんぶん の かいせつ らん を よみました	
\\	新聞[しんぶん]の
\\	欄[らん]を 読[よ]みました。			
\\	社説	社説[しゃせつ]	しゃせつ	
\\	この新聞の社説は面白いね。	この 新聞[しんぶん]の 社説[しゃせつ]は 面白[おもしろ]いね。	この しんぶん の しゃせつ は おもしろい ね	
\\	この 新聞[しんぶん]の
\\	は 面白[おもしろ]いね。			
\\	学説	学説[がくせつ]	がくせつ	
\\	それは最新の学説ね。	それは 最新[さいしん]の 学説[がくせつ]ね。	それ は さいしん の がくせつ ね	
\\	それは 最新[さいしん]の
\\	ね。			
\\	公式	公式[こうしき]	こうしき	
\\	今後の方針が公式に発表されました。	今後[こんご]の 方針[ほうしん]が 公式[こうしき]に 発表[はっぴょう]されました。	こんご の ほうしん が こうしき に はっぴょう されました	
\\	今後[こんご]の 方針[ほうしん]が
\\	に 発表[はっぴょう]されました。			
\\	公共	公共[こうきょう]	こうきょう	
\\	公共の乗り物は誰でも利用できます。	公共[こうきょう]の 乗[の]り 物[もの]は 誰[だれ]でも 利用[りよう]できます。	こうきょう の のりもの は だれ で も りよう できます	
\\	の 乗[の]り 物[もの]は 誰[だれ]でも 利用[りよう]できます。			
\\	シーツ	シーツ	シーツ	
\\	メイドがシーツを交換してくれたね。	メイドがシーツを 交換[こうかん]してくれたね。	めいど が しーつ を こうかん して くれた ね	
\\	メイドが
\\	を 交換[こうかん]してくれたね。			
\\	公平	公平[こうへい]	こうへい	
\\	賞金をもらったら公平に分けましょう。	賞金[しょうきん]をもらったら 公平[こうへい]に 分[わ]けましょう。	しょうきん を もらったら こうへい に わけましょう	
\\	賞金[しょうきん]をもらったら
\\	に 分[わ]けましょう。			
\\	公立	公立[こうりつ]	こうりつ	
\\	駅前に公立の図書館があるよ。	駅前[えきまえ]に 公立[こうりつ]の 図書館[としょかん]があるよ。	えきまえ に こうりつ の としょかん が ある よ	
\\	駅前[えきまえ]に
\\	の 図書館[としょかん]があるよ。			
\\	学園	学園[がくえん]	がくえん	
\\	彼女は郊外の学園に通っているね。	彼女[かのじょ]は 郊外[こうがい]の 学園[がくえん]に 通[かよ]っているね。	かのじょ は こうがい の がくえん に かよって いる ね	
\\	彼女[かのじょ]は 郊外[こうがい]の
\\	に 通[かよ]っているね。			
\\	祭日	祭日[さいじつ]	さいじつ	
\\	うちの会社は祭日は休みです。	うちの 会社[かいしゃ]は 祭日[さいじつ]は 休[やす]みです。	うち の かいしゃ は さいじつ は やすみ です	
\\	うちの 会社[かいしゃ]は
\\	は 休[やす]みです。			
\\	国際的	国際的[こくさいてき]	こくさいてき	
\\	彼女は国際的に有名な歌手です。	彼女[かのじょ]は 国際的[こくさいてき]に 有名[ゆうめい]な 歌手[かしゅ]です。	かのじょ は こくさいてき に ゆうめい な かしゅ です	
\\	彼女[かのじょ]は
\\	に 有名[ゆうめい]な 歌手[かしゅ]です。			
\\	国際化	国際化[こくさいか]	こくさいか	
\\	この大学も国際化してきたな。	この 大学[だいがく]も 国際化[こくさいか]してきたな。	この だいがく も こくさいか して きた な	
\\	この 大学[だいがく]も
\\	してきたな。			
\\	交際	交際[こうさい]	こうさい	
\\	彼はモデルと交際していたんだ。	彼[かれ]はモデルと 交際[こうさい]していたんだ。	かれ は もでる と こうさい して いた ん だ	
\\	彼[かれ]はモデルと
\\	していたんだ。			
\\	かかる	かかる	かかる	
\\	彼女はインフルエンザにかかっていますね。	彼女[かのじょ]はインフルエンザにかかっていますね。	かのじょ は いんふるえんざ に かかって います ね	
\\	彼女[かのじょ]はインフルエンザに
\\	ね。			
\\	際	際[さい]	さい	
\\	この際はっきり言っておきます。	この 際[さい]はっきり 言[い]っておきます。	この さい はっきり いって おきます	
\\	この
\\	はっきり 言[い]っておきます。			
\\	航空	航空[こうくう]	こうくう	
\\	会議で航空の安全について話し合ったの。	会議[かいぎ]で 航空[こうくう]の 安全[あんぜん]について 話[はな]し 合[あ]ったの。	かいぎ で こうくう の あんぜん に ついて はなしあった の	
\\	会議[かいぎ]で
\\	の 安全[あんぜん]について 話[はな]し 合[あ]ったの。			
\\	航空機	航空機[こうくうき]	こうくうき	
\\	去年は航空機の事故が多かった。	去年[きょねん]は 航空機[こうくうき]の 事故[じこ]が 多[おお]かった。	きょねん は こうくうき の じこ が おおかった	
\\	去年[きょねん]は
\\	の 事故[じこ]が 多[おお]かった。			
\\	航空券	航空券[こうくうけん]	こうくうけん	
\\	電話で航空券を予約しました。	電話[でんわ]で 航空券[こうくうけん]を 予約[よやく]しました。	でんわ で こうくうけん を よやく しました	
\\	電話[でんわ]で
\\	を 予約[よやく]しました。			
\\	完全	完全[かんぜん]	かんぜん	
\\	この古い寺院は今でも完全な形を保っているんだ。	この 古[ふる]い 寺院[じいん]は 今[いま]でも 完全[かんぜん]な 形[かたち]を 保[たも]っているんだ。	この ふるい じいん は いま で も かんぜん な かたち を たもって いる ん だ	
\\	この 古[ふる]い 寺院[じいん]は 今[いま]でも
\\	な 形[かたち]を 保[たも]っているんだ。			
\\	成長	成長[せいちょう]	せいちょう	
\\	庭の木、大きく成長したわね。	庭[にわ]の 木[き]、 大[おお]きく 成長[せいちょう]したわね。	にわ の き、 おおきく せいちょう した わ ね	
\\	庭[にわ]の 木[き]、 大[おお]きく
\\	したわね。			
\\	成立	成立[せいりつ]	せいりつ	
\\	あの会社との契約が成立しました。	あの 会社[かいしゃ]との 契約[けいやく]が 成立[せいりつ]しました。	あの かいしゃ と の けいやく が せいりつ しました	
\\	あの 会社[かいしゃ]との 契約[けいやく]が
\\	しました。			
\\	形成	形成[けいせい]	けいせい	
\\	今は骨が形成される大切な時期です。	今[いま]は 骨[ほね]が 形成[けいせい]される 大切[たいせつ]な 時期[じき]です。	いま は ほね が けいせい される たいせつ な じき です	
\\	今[いま]は 骨[ほね]が
\\	される 大切[たいせつ]な 時期[じき]です。			
\\	コック	コック	コック	
\\	彼は腕のいいコックです。	彼[かれ]は 腕[うで]のいいコックです。	かれ は うで の いい こっく です	
\\	彼[かれ]は 腕[うで]のいい
\\	です。			
\\	結成	結成[けっせい]	けっせい	
\\	新しい代表チームが結成されました。	新[あたら]しい 代表[だいひょう]チームが 結成[けっせい]されました。	あたらしい だいひょう ちーむ が けっせい されました	
\\	新[あたら]しい 代表[だいひょう]チームが
\\	されました。			
\\	成果	成果[せいか]	せいか	
\\	厳しい練習が良い成果に結び付きました。	厳[きび]しい 練習[れんしゅう]が 良[よ]い 成果[せいか]に 結[むす]び 付[つ]きました。	きびしい れんしゅう が よい せいか に むすびつきました	
\\	厳[きび]しい 練習[れんしゅう]が 良[よ]い
\\	に 結[むす]び 付[つ]きました。			
\\	賛成	賛成[さんせい]	さんせい	
\\	私は彼の意見に賛成だ。	私[わたし]は 彼[かれ]の 意見[いけん]に 賛成[さんせい]だ。	わたし は かれ の いけん に さんせい だ	
\\	私[わたし]は 彼[かれ]の 意見[いけん]に
\\	だ。			
\\	合成	合成[ごうせい]	ごうせい	
\\	このソフトで画像を合成できます。	このソフトで 画像[がぞう]を 合成[ごうせい]できます。	この そふと で がぞう を ごうせい できます	
\\	このソフトで 画像[がぞう]を
\\	できます。			
\\	成人	成人[せいじん]	せいじん	
\\	娘が今年成人します。	娘[むすめ]が 今年[ことし] 成人[せいじん]します。	むすめ が ことし せいじん します	
\\	娘[むすめ]が 今年[ことし]
\\	します。			
\\	成年	成年[せいねん]	せいねん	
\\	成年になると独立した戸籍を作れます。	成年[せいねん]になると 独立[どくりつ]した 戸籍[こせき]を 作[つく]れます。	せいねん に なる と どくりつ した こせき を つくれます	
\\	になると 独立[どくりつ]した 戸籍[こせき]を 作[つく]れます。			
\\	失う	失[うしな]う	うしなう	
\\	彼は地震で親を失いました。	彼[かれ]は 地震[じしん]で 親[おや]を 失[うしな]いました。	かれ は じしん で おや を うしないました	
\\	彼[かれ]は 地震[じしん]で 親[おや]を
\\	えび	えび	えび	
\\	私はえびのてんぷらが好きです。	私[わたし]はえびのてんぷらが 好[す]きです。	わたし は えび の てんぷら が すき です	
\\	私[わたし]は
\\	のてんぷらが 好[す]きです。			
\\	失業	失業[しつぎょう]	しつぎょう	
\\	友達のお父さんが突然失業したの。	友達[ともだち]のお 父[とう]さんが 突然[とつぜん] 失業[しつぎょう]したの。	ともだち の おとうさん が とつぜん しつぎょう した の	
\\	友達[ともだち]のお 父[とう]さんが 突然[とつぜん]
\\	したの。			
\\	失敗	失敗[しっぱい]	しっぱい	
\\	一度の失敗であきらめてはいけないよ。	一度[いちど]の 失敗[しっぱい]であきらめてはいけないよ。	いちど の しっぱい で あきらめては いけない よ	
\\	一度[いちど]の
\\	であきらめてはいけないよ。			
\\	勝敗	勝敗[しょうはい]	しょうはい	
\\	このセットで勝敗が決まります。	このセットで 勝敗[しょうはい]が 決[き]まります。	この せっと で しょうはい が きまります	
\\	このセットで
\\	が 決[き]まります。			
\\	原則	原則[げんそく]	げんそく	
\\	原則としてキャンセルできません。	原則[げんそく]としてキャンセルできません。	げんそく と して きゃんせる できません	
\\	としてキャンセルできません。			
\\	原料	原料[げんりょう]	げんりょう	
\\	石油はいろいろな製品の原料になります。	石油[せきゆ]はいろいろな 製品[せいひん]の 原料[げんりょう]になります。	せきゆ は いろいろ な せいひん の げんりょう に なります	
\\	石油[せきゆ]はいろいろな 製品[せいひん]の
\\	になります。			
\\	原理	原理[げんり]	げんり	
\\	見学者に機械の動く原理を説明しました。	見学者[けんがくしゃ]に 機械[きかい]の 動[うご]く 原理[げんり]を 説明[せつめい]しました。	けんがくしゃ に きかい の うごく げんり を せつめい しました	
\\	見学者[けんがくしゃ]に 機械[きかい]の 動[うご]く
\\	を 説明[せつめい]しました。			
\\	原子力	原子力[げんしりょく]	げんしりょく	
\\	原子力の安全な利用について考えましょう。	原子力[げんしりょく]の 安全[あんぜん]な 利用[りよう]について 考[かんが]えましょう。	げんしりょく の あんぜん な りよう に ついて かんがえましょう	
\\	の 安全[あんぜん]な 利用[りよう]について 考[かんが]えましょう。			
\\	グリーン	グリーン	グリーン	
\\	そのグリーンのスカーフ、素敵ですね。	そのグリーンのスカーフ、 素敵[すてき]ですね。	その ぐりーん の すかーふ すてき です ね	
\\	その
\\	のスカーフ、 素敵[すてき]ですね。			
\\	高原	高原[こうげん]	こうげん	
\\	高原には気持ちのいい風が吹いていたよ。	高原[こうげん]には 気持[きも]ちのいい 風[かぜ]が 吹[ふ]いていたよ。	こうげん に は きもち の いい かぜ が ふいていた よ 。	
\\	には 気持[きも]ちのいい 風[かぜ]が 吹[ふ]いていたよ。			
\\	資料	資料[しりょう]	しりょう	
\\	図書館で資料を借りて来ました。	図書館[としょかん]で 資料[しりょう]を 借[か]りて 来[き]ました。	としょかん で しりょう を かりて きました	
\\	図書館[としょかん]で
\\	を 借[か]りて 来[き]ました。			
\\	資産	資産[しさん]	しさん	
\\	大臣の資産が公表されたね。	大臣[だいじん]の 資産[しさん]が 公表[こうひょう]されたね。	だいじん の しさん が こうひょう された ね	
\\	大臣[だいじん]の
\\	が 公表[こうひょう]されたね。			
\\	資格	資格[しかく]	しかく	
\\	日本語を教える資格を取りたいです。	日本語[にほんご]を 教[おし]える 資格[しかく]を 取[と]りたいです。	にほんご を おしえる しかく を とりたい です	
\\	日本語[にほんご]を 教[おし]える
\\	を 取[と]りたいです。			
\\	資本主義	資本主義[しほんしゅぎ]	しほんしゅぎ	
\\	その国は次第に資本主義になったわね。	その 国[くに]は 次第[しだい]に 資本主義[しほんしゅぎ]になったわね。	その くに は しだいに しほんしゅぎ に なった わ ね	
\\	その 国[くに]は 次第[しだい]に
\\	になったわね。			
\\	資源	資源[しげん]	しげん	
\\	地球の資源を守りましょう。	地球[ちきゅう]の 資源[しげん]を 守[まも]りましょう。	ちきゅう の しげん を まもりましょう	
\\	地球[ちきゅう]の
\\	を 守[まも]りましょう。			
\\	願書	願書[がんしょ]	がんしょ	
\\	今日大学に願書を送りました。	今日大学[きょう だいがく]に 願書[がんしょ]を 送[おく]りました。	きょう だいがく に がんしょ を おくりました	
\\	今日大学[きょう だいがく]に
\\	を 送[おく]りました。			
\\	正式	正式[せいしき]	せいしき	
\\	正式な招待状を受け取りました。	正式[せいしき]な 招待状[しょうたいじょう]を 受[う]け 取[と]りました。	せいしき な しょうたいじょう を うけとりました	
\\	な 招待状[しょうたいじょう]を 受[う]け 取[と]りました。			
\\	ジーパン	ジーパン	ジーパン	
\\	彼はいつもジーパンをはいているね。	彼[かれ]はいつもジーパンをはいているね。	かれ は いつも じーぱん を はいて いる ね	
\\	彼[かれ]はいつも
\\	をはいているね。			
\\	正面	正面[しょうめん]	しょうめん	
\\	その家の正面には大きなバルコニーがあるの。	その 家[いえ]の 正面[しょうめん]には 大[おお]きなバルコニーがあるの。	その いえ の しょうめん に は おおき な ばるこにー が ある の	
\\	その 家[いえ]の
\\	には 大[おお]きなバルコニーがあるの。			
\\	正午	正午[しょうご]	しょうご	
\\	昼休みは正午からです。	昼休[ひるやす]みは 正午[しょうご]からです。	ひるやすみ は しょうご から です	
\\	昼休[ひるやす]みは
\\	からです。			
\\	正義	正義[せいぎ]	せいぎ	
\\	この世に正義はないのだろうか。	この 世[よ]に 正義[せいぎ]はないのだろうか。	この よ に せいぎ は ない の だろう か	
\\	この 世[よ]に
\\	はないのだろうか。			
\\	正門	正門[せいもん]	せいもん	
\\	受験生は正門から入って下さい。	受験生[じゅけんせい]は 正門[せいもん]から 入[はい]って 下[くだ]さい。	じゅけんせい は せいもん から はいって ください	
\\	受験生[じゅけんせい]は
\\	から 入[はい]って 下[くだ]さい。			
\\	正解	正解[せいかい]	せいかい	
\\	10問中9問正解しました。	10問中9問[じゅうもんちゅう きゅうもん] 正解[せいかい]しました。	じゅうもんちゅう きゅうもん せいかい しました	
\\	10問中9問[じゅうもんちゅう きゅうもん]
\\	しました。			
\\	正方形	正方形[せいほうけい]	せいほうけい	
\\	正方形の紙を用意しましょう。	正方形[せいほうけい]の 紙[かみ]を 用意[ようい]しましょう。	せいほうけい の かみ を ようい しましょう	
\\	の 紙[かみ]を 用意[ようい]しましょう。			
\\	正	正[せい]	せい	
\\	書類は正と副の2通あります。	書類[しょるい]は 正[せい]と 副[ふく]の 2通[につう]あります。	しょるい は せい と ふく の につう あります	
\\	書類[しょるい]は
\\	と 副[ふく]の 2通[につう]あります。			
\\	ジャーナリズム	ジャーナリズム	ジャーナリズム	
\\	彼はジャーナリズムを専攻している。	彼[かれ]はジャーナリズムを 専攻[せんこう]している。	かれ は じゃーなりずむ を せんこう して いる	
\\	彼[かれ]は
\\	を 専攻[せんこう]している。			
\\	正座	正座[せいざ]	せいざ	
\\	彼はきちんと正座して待っていたね。	彼[かれ]はきちんと 正座[せいざ]して 待[ま]っていたね。	かれ は きちんと せいざ して まって いた ね	
\\	彼[かれ]はきちんと
\\	して 待[ま]っていたね。			
\\	正当	正当[せいとう]	せいとう	
\\	これは正当な要求です。	これは 正当[せいとう]な 要求[ようきゅう]です。	これ は せいとう な ようきゅう です	
\\	これは
\\	な 要求[ようきゅう]です。			
\\	異性	異性[いせい]	いせい	
\\	息子はもう異性を意識している。	息子[むすこ]はもう 異性[いせい]を 意識[いしき]している。	むすこ は もう いせい を いしき して いる	
\\	息子[むすこ]はもう
\\	を 意識[いしき]している。			
\\	異常	異常[いじょう]	いじょう	
\\	今年の夏は異常な暑さですね。	今年[ことし]の 夏[なつ]は 異常[いじょう]な 暑[あつ]さですね。	ことし の なつ は いじょう な あつさ です ね	
\\	今年[ことし]の 夏[なつ]は
\\	な 暑[あつ]さですね。			
\\	正常	正常[せいじょう]	せいじょう	
\\	患者の呼吸は正常です。	患者[かんじゃ]の 呼吸[こきゅう]は 正常[せいじょう]です。	かんじゃ の こきゅう は せいじょう です	
\\	患者[かんじゃ]の 呼吸[こきゅう]は
\\	です。			
\\	意識	意識[いしき]	いしき	
\\	彼は意識を失いました。	彼[かれ]は 意識[いしき]を 失[うしな]いました。	かれ は いしき を うしないました	
\\	彼[かれ]は
\\	を 失[うしな]いました。			
\\	常識	常識[じょうしき]	じょうしき	
\\	そんなの常識だよ。	そんなの 常識[じょうしき]だよ。	そんな の じょうしき だ よ	
\\	そんなの
\\	だよ。			
\\	スピーチ	スピーチ	スピーチ	
\\	彼のスピーチは素晴らしかった。	彼[かれ]のスピーチは 素晴[すば]らしかった。	かれ の すぴーち は すばらしかった 。	
\\	彼[かれ]の
\\	は 素晴[すば]らしかった。			
\\	調べ	調[しら]べ	しらべ	
\\	警察の調べで女性の身元がわかりました。	警察[けいさつ]の 調[しら]べで 女性[じょせい]の 身元[みもと]がわかりました。	けいさつ の しらべ で じょせい の みもと が わかりました	
\\	警察[けいさつ]の
\\	で 女性[じょせい]の 身元[みもと]がわかりました。			
\\	好調	好調[こうちょう]	こうちょう	
\\	今月はエアコンの売り上げが好調です。	今月[こんげつ]はエアコンの 売[う]り 上[あ]げが 好調[こうちょう]です。	こんげつ は えあこん の うりあげ が こうちょう です	
\\	今月[こんげつ]はエアコンの 売[う]り 上[あ]げが
\\	です。			
\\	下調べ	下調[したしら]べ	したしらべ	
\\	まず第一に、しっかり下調べをしなさい。	まず 第一[だいいち]に、しっかり 下調[したしら]べをしなさい。	まず だいいち に しっかり したしらべ を しなさい	
\\	まず 第一[だいいち]に、しっかり
\\	をしなさい。			
\\	整備	整備[せいび]	せいび	
\\	車は整備に出しています。	車[くるま]は 整備[せいび]に 出[だ]しています。	くるま は せいび に だして います	
\\	車[くるま]は
\\	に 出[だ]しています。			
\\	整理	整理[せいり]	せいり	
\\	古い服を整理しました。	古[ふる]い 服[ふく]を 整理[せいり]しました。	ふるい ふく を せいり しました	
\\	古[ふる]い 服[ふく]を
\\	しました。			
\\	節約	節約[せつやく]	せつやく	
\\	電気や水を節約しましょう。	電気[でんき]や 水[みず]を 節約[せつやく]しましょう。	でんき や みず を せつやく しましょう	
\\	電気[でんき]や 水[みず]を
\\	しましょう。			
\\	検査	検査[けんさ]	けんさ	
\\	私は今日、目の検査を受けます。	私[わたし]は 今日[きょう]、 目[め]の 検査[けんさ]を 受[う]けます。	わたし は きょう め の けんさ を うけます	
\\	私[わたし]は 今日[きょう]、 目[め]の
\\	を 受[う]けます。			
\\	ああ	ああ	ああ	
\\	ああうるさい人は苦手です。	ああうるさい 人[ひと]は 苦手[にがて]です。	ああ うるさい ひと は にがて です	
\\	うるさい 人[ひと]は 苦手[にがて]です。			
\\	案	案[あん]	あん	
\\	もっと案を出し合いましょう。	もっと 案[あん]を 出[だ]し 合[あ]いましょう。	もっと あん を だしあいましょう	
\\	もっと
\\	を 出[だ]し 合[あ]いましょう。			
\\	案外	案外[あんがい]	あんがい	
\\	彼は案外いい人かも知れない。	彼[かれ]は 案外[あんがい]いい 人[ひと]かも 知[し]れない。	かれ は あんがい いい ひと かも しれない	
\\	彼[かれ]は
\\	いい 人[ひと]かも 知[し]れない。			
\\	案の定	案[あん]の 定[じょう]	あんのじょう	
\\	案の定、彼は遅刻したな。	案[あん]の 定[じょう]、 彼[かれ]は 遅刻[ちこく]したな。	あんのじょう かれ は ちこく した な	
\\	、 彼[かれ]は 遅刻[ちこく]したな。			
\\	国連	国連[こくれん]	こくれん	
\\	国連の本部はニューヨークにあります。	国連[こくれん]の 本部[ほんぶ]はニューヨークにあります。	こくれん の ほんぶ は にゅーよーく に あります	
\\	の 本部[ほんぶ]はニューヨークにあります。			
\\	接続	接続[せつぞく]	せつぞく	
\\	コンピューターをネットワークに接続しました。	コンピューターをネットワークに 接続[せつぞく]しました。	こんぴゅーたー を ねっとわーく に せつぞく しました	
\\	コンピューターをネットワークに
\\	しました。			
\\	外相	外相[がいしょう]	がいしょう	
\\	外相は来週訪米の予定です。	外相[がいしょう]は 来週訪米[らいしゅう ほうべい]の 予定[よてい]です。	がいしょう は らいしゅう ほうべい の よてい です	
\\	は 来週訪米[らいしゅう ほうべい]の 予定[よてい]です。			
\\	真相	真相[しんそう]	しんそう	
\\	最近、事件の真相が明らかになったよ。	最近[さいきん]、 事件[じけん]の 真相[しんそう]が 明[あき]らかになったよ。	さいきん じけん の しんそう が あきらか に なった よ	
\\	最近[さいきん]、 事件[じけん]の
\\	が 明[あき]らかになったよ。			
\\	相変わらず	相変[あいか]わらず	あいかわらず	
\\	彼は相変わらず忙しいですね。	彼[かれ]は 相変[あいか]わらず 忙[いそが]しいですね。	かれ は あいかわらず いそがしい です ね	
\\	彼[かれ]は
\\	忙[いそが]しいですね。			
\\	アクセント	アクセント	アクセント	
\\	アメリカ英語とイギリス英語ではアクセントが違うことがあるね。	アメリカ 英語[えいご]とイギリス 英語[えいご]ではアクセントが 違[ちが]うことがあるね。	あめりか えいご と いぎりす えいご で は あくせんと が ちがう こと が ある ね	
\\	アメリカ 英語[えいご]とイギリス 英語[えいご]では
\\	が 違[ちが]うことがあるね。			
\\	雑談	雑談[ざつだん]	ざつだん	
\\	その先生はいつも授業の前に雑談をするの。	その 先生[せんせい]はいつも 授業[じゅぎょう]の 前[まえ]に 雑談[ざつだん]をするの。	その せんせい は いつも じゅぎょう の まえ に ざつだん を する の	
\\	その 先生[せんせい]はいつも 授業[じゅぎょう]の 前[まえ]に
\\	をするの。			
\\	記事	記事[きじ]	きじ	
\\	環境問題に関する記事を読んだの。	環境問題[かんきょう もんだい]に 関[かん]する 記事[きじ]を 読[よ]んだの。	かんきょう もんだい に かんする きじ を よんだ の	
\\	環境問題[かんきょう もんだい]に 関[かん]する
\\	を 読[よ]んだの。			
\\	記号	記号[きごう]	きごう	
\\	地図にはいろいろな記号が使われているのね。	地図[ちず]にはいろいろな 記号[きごう]が 使[つか]われているのね。	ちず に は いろいろ な きごう が つかわれて いる の ね	
\\	地図[ちず]にはいろいろな
\\	が 使[つか]われているのね。			
\\	記入	記入[きにゅう]	きにゅう	
\\	こちらにお名前をご記入ください。	こちらにお 名前[なまえ]をご 記入[きにゅう]ください。	こちら に おなまえ を ご きにゅう ください	
\\	こちらにお 名前[なまえ]をご
\\	ください。			
\\	暗記	暗記[あんき]	あんき	
\\	試験の前に英文を暗記したんだ。	試験[しけん]の 前[まえ]に 英文[えいぶん]を 暗記[あんき]したんだ。	しけん の まえ に えいぶん を あんき した ん だ	
\\	試験[しけん]の 前[まえ]に 英文[えいぶん]を
\\	したんだ。			
\\	記憶	記憶[きおく]	きおく	
\\	当時のことはしっかり記憶しているよ。	当時[とうじ]のことはしっかり 記憶[きおく]しているよ。	とうじ の こと は しっかり きおく して いる よ	
\\	当時[とうじ]のことはしっかり
\\	しているよ。			
\\	関心	関心[かんしん]	かんしん	
\\	彼は政治に関心が強いね。	彼[かれ]は 政治[せいじ]に 関心[かんしん]が 強[つよ]いね。	かれ は せいじ に かんしん が つよい ね	
\\	彼[かれ]は 政治[せいじ]に
\\	が 強[つよ]いね。			
\\	コマーシャル	コマーシャル	コマーシャル	
\\	彼女はテレビのコマーシャルに出ているわ。	彼女[かのじょ]はテレビのコマーシャルに 出[で]ているわ。	かのじょ は てれび の こまーしゃる に でて いる わ	
\\	彼女[かのじょ]はテレビの
\\	に 出[で]ているわ。			
\\	税関	税関[ぜいかん]	ぜいかん	
\\	毛皮を税関で没収されたんだ。	毛皮[けがわ]を 税関[ぜいかん]で 没収[ぼっしゅう]されたんだ。	けがわ を ぜいかん で ぼっしゅう された ん だ	
\\	毛皮[けがわ]を
\\	で 没収[ぼっしゅう]されたんだ。			
\\	関節	関節[かんせつ]	かんせつ	
\\	手首の関節をひねっちゃった。	手首[てくび]の 関節[かんせつ]をひねっちゃった。	てくび の かんせつ を ひねっちゃった	
\\	手首[てくび]の
\\	をひねっちゃった。			
\\	関わる	関[かか]わる	かかわる	
\\	医師は人の命に関わる大切な職業だよ。	医師[いし]は 人[ひと]の 命[いのち]に 関[かか]わる 大切[たいせつ]な 職業[しょくぎょう]だよ。	いし は ひと の いのち に かかわる たいせつ な しょくぎょう だ よ	
\\	医師[いし]は 人[ひと]の 命[いのち]に
\\	大切[たいせつ]な 職業[しょくぎょう]だよ。			
\\	機関	機関[きかん]	きかん	
\\	台風で交通機関がストップしている。	台風[たいふう]で 交通[こうつう] 機関[きかん]がストップしている。	たいふう で こうつう きかん が すとっぷ して いる	
\\	台風[たいふう]で 交通[こうつう]
\\	がストップしている。			
\\	係	係[かかり]	かかり	
\\	彼女は会場整理の係だったの。	彼女[かのじょ]は 会場整理[かいじょう せいり]の 係[かかり]だったの。	かのじょ は かいじょう せいり の かかり だった の	
\\	彼女[かのじょ]は 会場整理[かいじょう せいり]の
\\	だったの。			
\\	現状	現状[げんじょう]	げんじょう	
\\	問題を解決できないのが現状です。	問題[もんだい]を 解決[かいけつ]できないのが 現状[げんじょう]です。	もんだい を かいけつ できない の が げんじょう です	
\\	問題[もんだい]を 解決[かいけつ]できないのが
\\	です。			
\\	事態	事態[じたい]	じたい	
\\	事態は深刻です。	事態[じたい]は 深刻[しんこく]です。	じたい は しんこく です	
\\	は 深刻[しんこく]です。			
\\	しつこい	しつこい	しつこい	
\\	しつこい迷惑メールに困っている。	しつこい 迷惑[めいわく]メールに 困[こま]っている。	しつこい めいわく めーる に こまって いる	
\\	迷惑[めいわく]メールに 困[こま]っている。			
\\	実態	実態[じったい]	じったい	
\\	その会社の経営の実態を調査中だ。	その 会社[かいしゃ]の 経営[けいえい]の 実態[じったい]を 調査中[ちょうさちゅう]だ。	その かいしゃ の けいえい の じったい を ちょうさちゅう だ	
\\	その 会社[かいしゃ]の 経営[けいえい]の
\\	を 調査中[ちょうさちゅう]だ。			
\\	行政	行政[ぎょうせい]	ぎょうせい	
\\	年金問題は行政の最大の課題のひとつよ。	年金問題[ねんきん もんだい]は 行政[ぎょうせい]の 最大[さいだい]の 課題[かだい]のひとつよ。	ねんきん もんだい は ぎょうせい の さいだい の かだい の ひとつ よ	
\\	年金問題[ねんきん もんだい]は
\\	の 最大[さいだい]の 課題[かだい]のひとつよ。			
\\	政治家	政治家[せいじか]	せいじか	
\\	大きくなったら政治家になりたいです。	大[おお]きくなったら 政治家[せいじか]になりたいです。	おおきく なったら せいじか に なりたい です	
\\	大[おお]きくなったら
\\	になりたいです。			
\\	治まる	治[おさ]まる	おさまる	
\\	咳が少し治まりました。	咳[せき]が 少[すこ]し 治[おさ]まりました。	せき が すこし おさまりました	
\\	咳[せき]が 少[すこ]し
\\	政党	政党[せいとう]	せいとう	
\\	選挙では3つの政党が争っています。	選挙[せんきょ]では 3[みっ]つの 政党[せいとう]が 争[あらそ]っています。	せんきょ で は みっつ の せいとう が あらそって います	
\\	選挙[せんきょ]では 3[みっ]つの
\\	が 争[あらそ]っています。			
\\	策	策[さく]	さく	
\\	その問題に対する策を皆で考えたの。	その 問題[もんだい]に 対[たい]する 策[さく]を 皆[みんな]で 考[かんが]えたの。	その もんだい に たいする さく を みんな で かんがえた の	
\\	その 問題[もんだい]に 対[たい]する
\\	を 皆[みんな]で 考[かんが]えたの。			
\\	挙げる	挙[あ]げる	あげる	
\\	例を幾つか挙げてみましょう。	例[れい]を 幾[いく]つか 挙[あ]げてみましょう。	れい を いくつ か あげて みましょう	
\\	例[れい]を 幾[いく]つか
\\	気候	気候[きこう]	きこう	
\\	日本の気候は温暖です。	日本[にほん]の 気候[きこう]は 温暖[おんだん]です。	にほん の きこう は おんだん です	
\\	日本[にほん]の
\\	は 温暖[おんだん]です。			
\\	しゃがむ	しゃがむ	しゃがむ	
\\	お年寄りが道端にしゃがんでいますね。	お 年寄[としよ]りが 道端[みちばた]にしゃがんでいますね。	おとしより が みちばた に しゃがんで います ね	
\\	お 年寄[としよ]りが 道端[みちばた]に
\\	ね。			
\\	補う	補[おぎな]う	おぎなう	
\\	夏は水分を十分に補いましょう。	夏[なつ]は 水分[すいぶん]を 十分[じゅうぶん]に 補[おぎな]いましょう。	なつ は すいぶん を じゅうぶん に おぎないましょう	
\\	夏[なつ]は 水分[すいぶん]を 十分[じゅうぶん]に
\\	足首	足首[あしくび]	あしくび	
\\	彼は足首を痛めています。	彼[かれ]は 足首[あしくび]を 痛[いた]めています。	かれ は あしくび を いためて います	
\\	彼[かれ]は
\\	を 痛[いた]めています。			
\\	首脳	首脳[しゅのう]	しゅのう	
\\	東京で五カ国の首脳会談が開かれています。	東京[とうきょう]で 五[ご]カ 国[こく]の 首脳[しゅのう] 会談[かいだん]が 開[ひら]かれています。	とうきょう で ごかこく の しゅのうかいだん が ひらかれて います	
\\	東京[とうきょう]で 五[ご]カ 国[こく]の
\\	会談[かいだん]が 開[ひら]かれています。			
\\	頭	頭[かしら]	かしら	
\\	彼は一家のお頭だったの。	彼[かれ]は 一家[いっか]のお 頭[かしら]だったの。	かれ は いっか の おかしら だった の	
\\	彼[かれ]は 一家[いっか]のお
\\	だったの。			
\\	頭痛	頭痛[ずつう]	ずつう	
\\	今日は頭痛がします。	今日[きょう]は 頭痛[ずつう]がします。	きょう は ずつう が します	
\\	今日[きょう]は
\\	がします。			
\\	顔色	顔色[かおいろ]	かおいろ	
\\	彼は顔色がよくありませんね。	彼[かれ]は 顔色[かおいろ]がよくありませんね。	かれ は かおいろ が よく ありません ね 。	
\\	彼[かれ]は
\\	がよくありませんね。			
\\	朝顔	朝顔[あさがお]	あさがお	
\\	紫の朝顔が咲きました。	紫[むらさき]の 朝顔[あさがお]が 咲[さ]きました。	むらさき の あさがお が さきました	
\\	紫[むらさき]の
\\	が 咲[さ]きました。			
\\	コンテスト	コンテスト	コンテスト	
\\	このコンテストに優勝すると車がもらえるんだ。	このコンテストに 優勝[ゆうしょう]すると 車[くるま]がもらえるんだ。	この こんてすと に ゆうしょう する と くるま が もらえる ん だ	
\\	この
\\	に 優勝[ゆうしょう]すると 車[くるま]がもらえるんだ。			
\\	改正	改正[かいせい]	かいせい	
\\	近く交通法が改正されます。	近[ちか]く 交通法[こうつう ほう]が 改正[かいせい]されます。	ちかく こうつう ほう が かいせい されます	
\\	近[ちか]く 交通法[こうつう ほう]が
\\	されます。			
\\	改良	改良[かいりょう]	かいりょう	
\\	日本では絶えず米の品種を改良しているの。	日本[にほん]では 絶[た]えず 米[こめ]の 品種[ひんしゅ]を 改良[かいりょう]しているの。	にほん で は たえず こめ の ひんしゅ を かいりょう して いる の	
\\	日本[にほん]では 絶[た]えず 米[こめ]の 品種[ひんしゅ]を
\\	しているの。			
\\	改める	改[あらた]める	あらためる	
\\	彼は悪い習慣を改めようとしているわね。	彼[かれ]は 悪[わる]い 習慣[しゅうかん]を 改[あらた]めようとしているわね。	かれ は わるい しゅうかん を あらためよう と して いる わ ね	
\\	彼[かれ]は 悪[わる]い 習慣[しゅうかん]を
\\	としているわね。			
\\	改造	改造[かいぞう]	かいぞう	
\\	首相は内閣の改造を行いました。	首相[しゅしょう]は 内閣[ないかく]の 改造[かいぞう]を 行[おこな]いました。	しゅしょう は ないかく の かいぞう を おこないました	
\\	首相[しゅしょう]は 内閣[ないかく]の
\\	を 行[おこな]いました。			
\\	改めて	改[あらた]めて	あらためて	
\\	改めてあなたのご意見を聞かせて下さい。	改[あらた]めてあなたのご 意見[いけん]を 聞[き]かせて 下[くだ]さい。	あらためて あなた の ごいけん を きかせ て ください	
\\	あなたのご 意見[いけん]を 聞[き]かせて 下[くだ]さい。			
\\	改まる	改[あらた]まる	あらたまる	
\\	年号が改まりました。	年号[ねんごう]が 改[あらた]まりました。	ねんごう が あらたまりました	
\\	年号[ねんごう]が
\\	革命	革命[かくめい]	かくめい	
\\	それは歴史上の大きな革命です。	それは 歴史上[れきしじょう]の 大[おお]きな 革命[かくめい]です。	それ は れきしじょう の おおき な かくめい です	
\\	それは 歴史上[れきしじょう]の 大[おお]きな
\\	です。			
\\	キャベツ	キャベツ	キャベツ	
\\	キャベツの千切りを添えたよ。	キャベツの 千切[せんぎ]りを 添[そ]えたよ。	きゃべつ の せんぎり を そえた よ	
\\	の 千切[せんぎ]りを 添[そ]えたよ。			
\\	生命	生命[せいめい]	せいめい	
\\	生命は海から始まったと言われている。	生命[せいめい]は 海[うみ]から 始[はじ]まったと 言[い]われている。	せいめい は うみ から はじまった と いわれて いる	
\\	は 海[うみ]から 始[はじ]まったと 言[い]われている。			
\\	命	命[いのち]	いのち	
\\	命より大切なものは無いよ。	命[いのち]より 大切[たいせつ]なものは 無[な]いよ。	いのち より たいせつ な もの は ない よ	
\\	より 大切[たいせつ]なものは 無[な]いよ。			
\\	運命	運命[うんめい]	うんめい	
\\	運命には逆らえないよ。	運命[うんめい]には 逆[さか]らえないよ。	うんめい に は さからえない よ	
\\	には 逆[さか]らえないよ。			
\\	組合	組合[くみあい]	くみあい	
\\	今日は組合の集まりがあるわ。	今日[きょう]は 組合[くみあい]の 集[あつ]まりがあるわ。	きょう は くみあい の あつまり が ある わ	
\\	今日[きょう]は
\\	の 集[あつ]まりがあるわ。			
\\	仕組み	仕組[しく]み	しくみ	
\\	この機械の仕組みは複雑だ。	この 機械[きかい]の 仕組[しく]みは 複雑[ふくざつ]だ。	この きかい の しくみ は ふくざつ だ	
\\	この 機械[きかい]の
\\	は 複雑[ふくざつ]だ。			
\\	組	組[くみ]	くみ	
\\	彼は1年2組の生徒です。	彼[かれ]は 1年2[いちねん に] 組[くみ]の 生徒[せいと]です。	かれ は いちねん にくみ の せいと です	
\\	彼[かれ]は 1年2[いちねん に]
\\	の 生徒[せいと]です。			
\\	組む	組[く]む	くむ	
\\	このプロジェクトで私は彼と組んでいるんだ。	このプロジェクトで 私[わたし]は 彼[かれ]と 組[く]んでいるんだ。	この ぷろじぇくと で わたし は かれ と くんで いる ん だ	
\\	このプロジェクトで 私[わたし]は 彼[かれ]と
\\	んだ。			
\\	組み合わせる	組[く]み 合[あ]わせる	くみあわせる	
\\	いろいろな花を組み合わせ花束を作りました。	いろいろな 花[はな]を 組[く]み 合[あ]わせ 花束[はなたば]を 作[つく]りました。	いろいろ な はな を くみあわせ はなたば を つくりました	
\\	いろいろな 花[はな]を
\\	花束[はなたば]を 作[つく]りました。			
\\	アイドル	アイドル	アイドル	
\\	彼女は若者のアイドルです。	彼女[かのじょ]は 若者[わかもの]のアイドルです。	かのじょ は わかもの の あいどる です	
\\	彼女[かのじょ]は 若者[わかもの]の
\\	です。			
\\	組み込む	組[く]み 込[こ]む	くみこむ	
\\	キャンペーンに新しいイベントを組み込みました。	キャンペーンに 新[あたら]しいイベントを 組[く]み 込[こ]みました。	きゃんぺーん に あたらしい いべんと を くみこみました	
\\	キャンペーンに 新[あたら]しいイベントを
\\	組み合わせ	組[く]み 合[あ]わせ	くみあわせ	
\\	色の組み合わせで印象が変わりますよ。	色[いろ]の 組[く]み 合[あ]わせで 印象[いんしょう]が 変[か]わりますよ。	いろ の くみあわせ で いんしょう が かわります よ	
\\	色[いろ]の
\\	で 印象[いんしょう]が 変[か]わりますよ。			
\\	織物	織物[おりもの]	おりもの	
\\	その町は織物業で有名です。	その 町[まち]は 織物[おりもの] 業[ぎょう]で 有名[ゆうめい]です。	その まち は おりものぎょう で ゆうめい です	
\\	その 町[まち]は
\\	業[ぎょう]で 有名[ゆうめい]です。			
\\	進出	進出[しんしゅつ]	しんしゅつ	
\\	大手チェーン店が進出している。	大手[おおて]チェーン 店[てん]が 進出[しんしゅつ]している。	おおて ちぇーんてん が しんしゅつ して いる	
\\	大手[おおて]チェーン 店[てん]が
\\	している。			
\\	進行	進行[しんこう]	しんこう	
\\	学会は予定通りに進行しています。	学会[がっかい]は 予定通[よてい どお]りに 進行[しんこう]しています。	がっかい は よてい どおり に しんこう して います	
\\	学会[がっかい]は 予定通[よてい どお]りに
\\	しています。			
\\	進歩	進歩[しんぽ]	しんぽ	
\\	科学技術は目覚しく進歩しているの。	科学技術[かがく ぎじゅつ]は 目覚[めざま]しく 進歩[しんぽ]しているの。	かがく ぎじゅつ は めざましく しんぽ して いる の	
\\	科学技術[かがく ぎじゅつ]は 目覚[めざま]しく
\\	しているの。			
\\	前進	前進[ぜんしん]	ぜんしん	
\\	一列に並んで前進して下さい。	一列[いちれつ]に 並[なら]んで 前進[ぜんしん]して 下[くだ]さい。	いちれつ に ならんで ぜんしん して ください	
\\	一列[いちれつ]に 並[なら]んで
\\	して 下[くだ]さい。			
\\	からかう	からかう	からかう	
\\	彼は時々妹をからかいます。	彼[かれ]は 時々妹[ときどき いもうと]をからかいます。	かれ は ときどき いもうと を からかいます	
\\	彼[かれ]は 時々妹[ときどき いもうと]を
\\	進路	進路[しんろ]	しんろ	
\\	卒業後の進路を迷っています。	卒業後[そつぎょう ご]の 進路[しんろ]を 迷[まよ]っています。	そつぎょう ご の しんろ を まよって います	
\\	卒業後[そつぎょう ご]の
\\	を 迷[まよ]っています。			
\\	行進	行進[こうしん]	こうしん	
\\	開会式で選手たちが行進しました。	開会式[かいかいしき]で 選手[せんしゅ]たちが 行進[こうしん]しました。	かいかいしき で せんしゅたち が こうしん しました	
\\	開会式[かいかいしき]で 選手[せんしゅ]たちが
\\	しました。			
\\	推進	推進[すいしん]	すいしん	
\\	その会社はリサイクルを推進しているね。	その 会社[かいしゃ]はリサイクルを 推進[すいしん]しているね。	その かいしゃ は りさいくる を すいしん して いる ね	
\\	その 会社[かいしゃ]はリサイクルを
\\	しているね。			
\\	主任	主任[しゅにん]	しゅにん	
\\	彼女は会計主任です。	彼女[かのじょ]は 会計[かいけい] 主任[しゅにん]です。	かのじょ は かいけい しゅにん です	
\\	彼女[かのじょ]は 会計[かいけい]
\\	です。			
\\	辞任	辞任[じにん]	じにん	
\\	社長の辞任が決まりました。	社長[しゃちょう]の 辞任[じにん]が 決[き]まりました。	しゃちょう の じにん が きまりました	
\\	社長[しゃちょう]の
\\	が 決[き]まりました。			
\\	お世辞	お 世辞[せじ]	おせじ	
\\	彼はお世辞を言うのが上手い。	彼[かれ]はお 世辞[せじ]を 言[い]うのが 上手[うま]い。	かれ は おせじ を いう の が うまい	
\\	彼[かれ]は
\\	を 言[い]うのが 上手[うま]い。			
\\	委員会	委員会[いいんかい]	いいんかい	
\\	明日、委員会が開かれます。	明日[あした]、 委員会[いいんかい]が 開[ひら]かれます。	あした いいんかい が ひらかれます	
\\	明日[あした]、
\\	が 開[ひら]かれます。			
\\	すっきり	すっきり	すっきり	
\\	よく眠ったら気分がすっきりした。	よく 眠[ねむ]ったら 気分[きぶん]がすっきりした。	よく ねむったら きぶん が すっきり した	
\\	よく 眠[ねむ]ったら 気分[きぶん]が
\\	した。			
\\	委員	委員[いいん]	いいん	
\\	彼は委員に選ばれました。	彼[かれ]は 委員[いいん]に 選[えら]ばれました。	かれ は いいん に えらばれました	
\\	彼[かれ]は
\\	に 選[えら]ばれました。			
\\	出勤	出勤[しゅっきん]	しゅっきん	
\\	毎朝7時5分に出勤します。	毎朝7時5分[まいあさ しち じ ご ふん]に 出勤[しゅっきん]します。	まいあさ しち じ ご ふん に しゅっきん します	
\\	毎朝7時5分[まいあさ しち じ ご ふん]に
\\	します。			
\\	勤勉	勤勉[きんべん]	きんべん	
\\	彼は勤勉な人です。	彼[かれ]は 勤勉[きんべん]な 人[ひと]です。	かれ は きんべん な ひと です	
\\	彼[かれ]は
\\	な 人[ひと]です。			
\\	勤務	勤務[きんむ]	きんむ	
\\	私の一日の勤務時間は8時間です。	私[わたし]の 一日[いちにち]の 勤務[きんむ] 時間[じかん]は 8時間[はちじかん]です。	わたし の いちにち の きんむじかん は はちじかん です	
\\	私[わたし]の 一日[いちにち]の
\\	時間[じかん]は 8時間[はちじかん]です。			
\\	義務	義務[ぎむ]	ぎむ	
\\	労働は国民の義務です。	労働[ろうどう]は 国民[こくみん]の 義務[ぎむ]です。	ろうどう は こくみん の ぎむ です	
\\	労働[ろうどう]は 国民[こくみん]の
\\	です。			
\\	事務	事務[じむ]	じむ	
\\	事務の経験が3年あります。	事務[じむ]の 経験[けいけん]が 3年[さんねん]あります。	じむ の けいけん が さんねん あります	
\\	の 経験[けいけん]が 3年[さんねん]あります。			
\\	公務員	公務員[こうむいん]	こうむいん	
\\	私の父は公務員です。	私[わたし]の 父[ちち]は 公務員[こうむいん]です。	わたし の ちち は こうむいん です	
\\	私[わたし]の 父[ちち]は
\\	です。			
\\	あいにく	あいにく	あいにく	
\\	途中であいにく雨が降り出したの。	途中[とちゅう]であいにく 雨[あめ]が 降[ふ]り 出[だ]したの。	とちゅう で あいにく あめ が ふりだした の	
\\	途中[とちゅう]で
\\	雨[あめ]が 降[ふ]り 出[だ]したの。			
\\	事務員	事務員[じむいん]	じむいん	
\\	新しい事務員が入りました。	新[あたら]しい 事務員[じむいん]が 入[はい]りました。	あたらしい じむいん が はいりました	
\\	新[あたら]しい
\\	が 入[はい]りました。			
\\	従う	従[したが]う	したがう	
\\	上司の指示に従った。	上司[じょうし]の 指示[しじ]に 従[したが]った。	じょうし の しじ に したがった	
\\	上司[じょうし]の 指示[しじ]に
\\	従業員	従業員[じゅうぎょういん]	じゅうぎょういん	
\\	会社は従業員の数を増やす予定だ。	会社[かいしゃ]は 従業員[じゅうぎょういん]の 数[かず]を 増[ふ]やす 予定[よてい]だ。	かいしゃ は じゅうぎょういん の かず を ふやす よてい だ	
\\	会社[かいしゃ]は
\\	の 数[かず]を 増[ふ]やす 予定[よてい]だ。			
\\	雇用	雇用[こよう]	こよう	
\\	彼はその会社と雇用契約を結んだの。	彼[かれ]はその 会社[かいしゃ]と 雇用[こよう] 契約[けいやく]を 結[むす]んだの。	かれ は その かいしゃ と こよう けいやく を むすんだ の	
\\	彼[かれ]はその 会社[かいしゃ]と
\\	契約[けいやく]を 結[むす]んだの。			
\\	実績	実績[じっせき]	じっせき	
\\	彼は営業で実績を上げたんだ。	彼[かれ]は 営業[えいぎょう]で 実績[じっせき]を 上[あ]げたんだ。	かれ は えいぎょう で じっせき を あげた ん だ	
\\	彼[かれ]は 営業[えいぎょう]で
\\	を 上[あ]げたんだ。			
\\	業績	業績[ぎょうせき]	ぎょうせき	
\\	彼の今月の業績は素晴らしいです。	彼[かれ]の 今月[こんげつ]の 業績[ぎょうせき]は 素晴[すば]らしいです。	かれ の こんげつ の ぎょうせき は すばらしい です	
\\	彼[かれ]の 今月[こんげつ]の
\\	は 素晴[すば]らしいです。			
\\	応募	応募[おうぼ]	おうぼ	
\\	求人に多数の応募があった。	求人[きゅうじん]に 多数[たすう]の 応募[おうぼ]があった。	きゅうじん に たすう の おうぼ が あった	
\\	求人[きゅうじん]に 多数[たすう]の
\\	があった。			
\\	集中	集中[しゅうちゅう]	しゅうちゅう	
\\	勉強に集中しなさい。	勉強[べんきょう]に 集中[しゅうちゅう]しなさい。	べんきょう に しゅうちゅう しなさい	
\\	勉強[べんきょう]に
\\	しなさい。			
\\	おばさん	おばさん	おばさん	
\\	おばさん、こんにちは。	おばさん、こんにちは。	おばさん、こんにちは。	
\\	、こんにちは。			
\\	集合	集合[しゅうごう]	しゅうごう	
\\	7時5分に駅で集合しましょう。	7時5分[しち じ ご ふん]に 駅[えき]で 集合[しゅうごう]しましょう。	しち じ ご ふん に えき で しゅうごう しましょう	
\\	7時5分[しち じ ご ふん]に 駅[えき]で
\\	しましょう。			
\\	集まり	集[あつ]まり	あつまり	
\\	雨で集まりが悪いですね。	雨[あめ]で 集[あつ]まりが 悪[わる]いですね。	あめ で あつまり が わるい です ね	
\\	雨[あめ]で
\\	が 悪[わる]いですね。			
\\	収集	収集[しゅうしゅう]	しゅうしゅう	
\\	彼の趣味は切手収集です。	彼[かれ]の 趣味[しゅみ]は 切手[きって] 収集[しゅうしゅう]です。	かれ の しゅみ は きって しゅうしゅう です	
\\	彼[かれ]の 趣味[しゅみ]は 切手[きって]
\\	です。			
\\	採算	採算[さいさん]	さいさん	
\\	コストがこんなに高くては採算が取れません。	コストがこんなに 高[たか]くては 採算[さいさん]が 取[と]れません。	こすと が こんなに たかくて は さいさん が とれません	
\\	コストがこんなに 高[たか]くては
\\	が 取[と]れません。			
\\	採点	採点[さいてん]	さいてん	
\\	先生は試験の採点が終わったようね。	先生[せんせい]は 試験[しけん]の 採点[さいてん]が 終[お]わったようね。	せんせい は しけん の さいてん が おわった よう ね	
\\	先生[せんせい]は 試験[しけん]の
\\	が 終[お]わったようね。			
\\	供給	供給[きょうきゅう]	きょうきゅう	
\\	彼の会社は電力を供給しています。	彼[かれ]の 会社[かいしゃ]は 電力[でんりょく]を 供給[きょうきゅう]しています。	かれ の かいしゃ は でんりょく を きょうきゅう して います	
\\	彼[かれ]の 会社[かいしゃ]は 電力[でんりょく]を
\\	しています。			
\\	月給	月給[げっきゅう]	げっきゅう	
\\	月給は毎月25日に支給されます。	月給[げっきゅう]は 毎月25日[まいつき にじゅうごにち]に 支給[しきゅう]されます。	げっきゅう は まいつき にじゅうごにち に しきゅう されます	
\\	は 毎月25日[まいつき にじゅうごにち]に 支給[しきゅう]されます。			
\\	ジャンプ	ジャンプ	ジャンプ	
\\	猿が高い木にジャンプした。	猿[さる]が 高[たか]い 木[き]にジャンプした。	さる が たかい き に じゃんぷ した	
\\	猿[さる]が 高[たか]い 木[き]に
\\	した。			
\\	時給	時給[じきゅう]	じきゅう	
\\	この仕事は時給1000円です。	この 仕事[しごと]は 時給[じきゅう] 1000円[せんえん]です。	この しごと は じきゅう せんえん です	
\\	この 仕事[しごと]は
\\	1000円[せんえん]です。			
\\	需要	需要[じゅよう]	じゅよう	
\\	需要が多過ぎて生産が追い付きません。	需要[じゅよう]が 多過[おおす]ぎて 生産[せいさん]が 追[お]い 付[つ]きません。	じゅよう が おおすぎ て せいさん が おいつきません	
\\	が 多過[おおす]ぎて 生産[せいさん]が 追[お]い 付[つ]きません。			
\\	就任	就任[しゅうにん]	しゅうにん	
\\	彼は新首相に就任しましたね。	彼[かれ]は 新首相[しんしゅしょう]に 就任[しゅうにん]しましたね。	かれ は しんしゅしょう に しゅうにん しました ね	
\\	彼[かれ]は 新首相[しんしゅしょう]に
\\	しましたね。			
\\	職員	職員[しょくいん]	しょくいん	
\\	ここは職員専用の出入り口です。	ここは 職員[しょくいん] 専用[せんよう]の 出入[でい]り 口[ぐち]です。	ここ は しょくいん せんよう の でいりぐち です	
\\	ここは
\\	専用[せんよう]の 出入[でい]り 口[ぐち]です。			
\\	職場	職場[しょくば]	しょくば	
\\	自宅から職場まで1時間かかります。	自宅[じたく]から 職場[しょくば]まで 1時間[いちじかん]かかります。	じたく から しょくば まで いちじかん かかります	
\\	自宅[じたく]から
\\	まで 1時間[いちじかん]かかります。			
\\	職業	職業[しょくぎょう]	しょくぎょう	
\\	あなたの職業を教えてください。	あなたの 職業[しょくぎょう]を 教[おし]えてください。	あなた の しょくぎょう を おしえて ください	
\\	あなたの
\\	を 教[おし]えてください。			
\\	条約	条約[じょうやく]	じょうやく	
\\	2国間で条約が結ばれました。	2国間[にこくかん]で 条約[じょうやく]が 結[むす]ばれました。	にこくかん で じょうやく が むすばれました	
\\	2国間[にこくかん]で
\\	が 結[むす]ばれました。			
\\	あちらこちら	あちらこちら	あちらこちら	
\\	あちらこちらで紅葉がきれいですね。	あちらこちらで 紅葉[こうよう]がきれいですね。	あちらこちら で こうよう が きれい です ね	
\\	で 紅葉[こうよう]がきれいですね。			
\\	参考	参考[さんこう]	さんこう	
\\	この本を参考にして下さい。	この 本[ほん]を 参考[さんこう]にして 下[くだ]さい。	この ほん を さんこう に して ください	
\\	この 本[ほん]を
\\	にして 下[くだ]さい。			
\\	参議院	参議院[さんぎいん]	さんぎいん	
\\	彼は参議院議員です。	彼[かれ]は 参議院[さんぎいん] 議員[ぎいん]です。	かれ は さんぎいん ぎいん です	
\\	彼[かれ]は
\\	議員[ぎいん]です。			
\\	お参り	お 参[まい]り	おまいり	
\\	家族でお寺にお参りに行きました。	家族[かぞく]でお 寺[てら]にお 参[まい]りに 行[い]きました。	かぞく で おてら に おまいり に いきました	
\\	家族[かぞく]でお 寺[てら]に
\\	に 行[い]きました。			
\\	参考書	参考書[さんこうしょ]	さんこうしょ	
\\	この参考書はとても役に立つよ。	この 参考書[さんこうしょ]はとても 役[やく]に 立[た]つよ。	この さんこうしょ は とても やく に たつ よ	
\\	この
\\	はとても 役[やく]に 立[た]つよ。			
\\	加わる	加[くわ]わる	くわわる	
\\	私たちのチームに彼が加わった。	私[わたし]たちのチームに 彼[かれ]が 加[くわ]わった。	わたしたち の ちーむ に かれ が くわわった	
\\	私[わたし]たちのチームに 彼[かれ]が
\\	いい加減	いい 加減[かげん]	いいかげん	
\\	いい加減なことを言ってはいけません。	いい 加減[かげん]なことを 言[い]ってはいけません。	いいかげん な こと を いって は いけません	
\\	なことを 言[い]ってはいけません。			
\\	追う	追[お]う	おう	
\\	警官は怪しい男の後を追ったよ。	警官[けいかん]は 怪[あや]しい 男[おとこ]の 後[あと]を 追[お]ったよ。	けいかん は あやしい おとこ の あと を おった よ	
\\	警官[けいかん]は 怪[あや]しい 男[おとこ]の 後[あと]を
\\	よ。			
\\	追い出す	追[お]い 出[だ]す	おいだす	
\\	彼は家から追い出された。	彼[かれ]は 家[いえ]から 追[お]い 出[だ]された。	かれ は いえ から おいだされた	
\\	彼[かれ]は 家[いえ]から
\\	いつのまにか	いつのまにか	いつのまにか	
\\	いつのまにか夜が明けていたね。	いつのまにか 夜[よ]が 明[あ]けていたね。	いつのまにか よ が あけて いた ね	
\\	夜[よ]が 明[あ]けていたね。			
\\	感じ	感[かん]じ	かんじ	
\\	あの子は感じの良い子です。	あの 子[こ]は 感[かん]じの 良[い]い 子[こ]です。	あの こ は かんじ の いい こ です	
\\	あの 子[こ]は
\\	の 良[い]い 子[こ]です。			
\\	感情	感情[かんじょう]	かんじょう	
\\	感情とは複雑なものです。	感情[かんじょう]とは 複雑[ふくざつ]なものです。	かんじょう と は ふくざつ な もの です	
\\	とは 複雑[ふくざつ]なものです。			
\\	感覚	感覚[かんかく]	かんかく	
\\	冷えて指の感覚がない。	冷[ひ]えて 指[ゆび]の 感覚[かんかく]がない。	ひえて ゆび の かんかく が ない	
\\	冷[ひ]えて 指[ゆび]の
\\	がない。			
\\	感動	感動[かんどう]	かんどう	
\\	感動する映画でした。	感動[かんどう]する 映画[えいが]でした。	かんどう する えいが でした	
\\	する 映画[えいが]でした。			
\\	実感	実感[じっかん]	じっかん	
\\	子供が歩き始めたとき、子供の成長を実感した。	子供[こども]が 歩[ある]き 始[はじ]めたとき、 子供[こども]の 成長[せいちょう]を 実感[じっかん]した。	こども が あるきはじめた とき こども の せいちょう を じっかん した	
\\	子供[こども]が 歩[ある]き 始[はじ]めたとき、 子供[こども]の 成長[せいちょう]を
\\	した。			
\\	感心	感心[かんしん]	かんしん	
\\	彼の我慢強さには感心しました。	彼[かれ]の 我慢強[がまんづよ]さには 感心[かんしん]しました。	かれ の がまんづよさ に は かんしん しました	
\\	彼[かれ]の 我慢強[がまんづよ]さには
\\	しました。			
\\	思想	思想[しそう]	しそう	
\\	人には思想の自由がある。	人[ひと]には 思想[しそう]の 自由[じゆう]がある。	ひと に は しそう の じゆう が ある	
\\	人[ひと]には
\\	の 自由[じゆう]がある。			
\\	おごる	おごる	おごる	
\\	彼に食事をおごってもらった。	彼[かれ]に 食事[しょくじ]をおごってもらった。	かれ に しょくじ を おごって もらった	
\\	彼[かれ]に 食事[しょくじ]を
\\	感想	感想[かんそう]	かんそう	
\\	ご感想をお聞かせ下さい。	ご 感想[かんそう]をお 聞[き]かせ 下[くだ]さい。	ごかんそう を おきかせ ください	
\\	ご
\\	をお 聞[き]かせ 下[くだ]さい。			
\\	空想	空想[くうそう]	くうそう	
\\	彼は空想にふけっているの。	彼[かれ]は 空想[くうそう]にふけっているの。	かれ は くうそう に ふけって いる の	
\\	彼[かれ]は
\\	にふけっているの。			
\\	現像	現像[げんぞう]	げんぞう	
\\	このフィルムを現像してください。	このフィルムを 現像[げんぞう]してください。	この ふぃるむ を げんぞう して ください	
\\	このフィルムを
\\	してください。			
\\	現象	現象[げんしょう]	げんしょう	
\\	村では最近、不思議な現象が起きています。	村[むら]では 最近[さいきん]、 不思議[ふしぎ]な 現象[げんしょう]が 起[お]きています。	むら で は さいきん ふしぎ な げんしょう が おきて います	
\\	村[むら]では 最近[さいきん]、 不思議[ふしぎ]な
\\	が 起[お]きています。			
\\	気象	気象[きしょう]	きしょう	
\\	テレビで明日の気象情報を確認したよ。	テレビで 明日[あす]の 気象[きしょう] 情報[じょうほう]を 確認[かくにん]したよ。	てれび で あす の きしょう じょうほう を かくにん した よ	
\\	テレビで 明日[あす]の
\\	情報[じょうほう]を 確認[かくにん]したよ。			
\\	障子	障子[しょうじ]	しょうじ	
\\	猫が障子を破った。	猫[ねこ]が 障子[しょうじ]を 破[やぶ]った。	ねこ が しょうじ を やぶった	
\\	猫[ねこ]が
\\	を 破[やぶ]った。			
\\	修正	修正[しゅうせい]	しゅうせい	
\\	検討の結果、案を修正したよ。	検討[けんとう]の 結果[けっか]、 案[あん]を 修正[しゅうせい]したよ。	けんとう の けっか あん を しゅうせい した よ	
\\	検討[けんとう]の 結果[けっか]、 案[あん]を
\\	したよ。			
\\	おやつ	おやつ	おやつ	
\\	今日のおやつはプリンだった。	今日[きょう]のおやつはプリンだった。	きょう の おやつ は ぷりん だった	
\\	今日[きょう]の
\\	はプリンだった。			
\\	傷	傷[きず]	きず	
\\	足の傷が痛みます。	足[あし]の 傷[きず]が 痛[いた]みます。	あし の きず が いたみます	
\\	足[あし]の
\\	が 痛[いた]みます。			
\\	傷める	傷[いた]める	いためる	
\\	彼は柔道で腰を傷めたんだ。	彼[かれ]は 柔道[じゅうどう]で 腰[こし]を 傷[いた]めたんだ。	かれ は じゅうどう で こし を いためた ん だ	
\\	彼[かれ]は 柔道[じゅうどう]で 腰[こし]を
\\	んだ。			
\\	交換	交換[こうかん]	こうかん	
\\	試合の相手とユニフォームを交換したよ。	試合[しあい]の 相手[あいて]とユニフォームを 交換[こうかん]したよ。	しあい の あいて と ゆにふぉーむ を こうかん した よ	
\\	試合[しあい]の 相手[あいて]とユニフォームを
\\	したよ。			
\\	言い換える	言[い]い 換[か]える	いいかえる	
\\	彼は易しい言葉に言い換えたんだ。	彼[かれ]は 易[やさ]しい 言葉[ことば]に 言[い]い 換[か]えたんだ。	かれ は やさしい ことば に いいかえた ん だ	
\\	彼[かれ]は 易[やさ]しい 言葉[ことば]に
\\	んだ。			
\\	着替え	着替[きが]え	きがえ	
\\	着替えを旅行カバンに詰めたよ。	着替[きが]えを 旅行[りょこう]カバンに 詰[つ]めたよ。	きがえ を りょこう かばん に つめた よ	
\\	を 旅行[りょこう]カバンに 詰[つ]めたよ。			
\\	交替	交替[こうたい]	こうたい	
\\	家まで交替で荷物を持ちました。	家[いえ]まで 交替[こうたい]で 荷物[にもつ]を 持[も]ちました。	いえ まで こうたい で にもつ を もちました	
\\	家[いえ]まで
\\	で 荷物[にもつ]を 持[も]ちました。			
\\	火災	火災[かさい]	かさい	
\\	火災の原因は放火だそうです。	火災[かさい]の 原因[げんいん]は 放火[ほうか]だそうです。	かさい の げんいん は ほうか だ そう です	
\\	の 原因[げんいん]は 放火[ほうか]だそうです。			
\\	災難	災難[さいなん]	さいなん	
\\	旅先で思いがけない災難にあいました。	旅先[たびさき]で 思[おも]いがけない 災難[さいなん]にあいました。	たびさき で おもいがけない さいなん に あいました	
\\	旅先[たびさき]で 思[おも]いがけない
\\	にあいました。			
\\	ごろごろ	ごろごろ	ごろごろ	
\\	雷がごろごろ鳴っている。	雷[かみなり]がごろごろ 鳴[な]っている。	かみなり が ごろごろ なって いる	
\\	雷[かみなり]が
\\	鳴[な]っている。			
\\	障害	障害[しょうがい]	しょうがい	
\\	まずは障害を取り除いてから計画を進めよう。	まずは 障害[しょうがい]を 取[と]り 除[のぞ]いてから 計画[けいかく]を 進[すす]めよう。	まず は しょうがい を とりのぞいて から けいかく を すすめよう	
\\	まずは
\\	を 取[と]り 除[のぞ]いてから 計画[けいかく]を 進[すす]めよう。			
\\	災害	災害[さいがい]	さいがい	
\\	地震は自然災害のひとつです。	地震[じしん]は 自然[しぜん] 災害[さいがい]のひとつです。	じしん は しぜん さいがい の ひとつ です	
\\	地震[じしん]は 自然[しぜん]
\\	のひとつです。			
\\	公害	公害[こうがい]	こうがい	
\\	私たちは公害を減らすよう努力しています。	私[わたし]たちは 公害[こうがい]を 減[へ]らすよう 努力[どりょく]しています。	わたしたち は こうがい を へらす よう どりょく して います	
\\	私[わたし]たちは
\\	を 減[へ]らすよう 努力[どりょく]しています。			
\\	水害	水害[すいがい]	すいがい	
\\	水害でたくさんの人が家を失ったの。	水害[すいがい]でたくさんの 人[ひと]が 家[いえ]を 失[うしな]ったの。	すいがい で たくさん の ひと が いえ を うしなった の	
\\	でたくさんの 人[ひと]が 家[いえ]を 失[うしな]ったの。			
\\	害	害[がい]	がい	
\\	お酒の飲み過ぎは健康に害があります。	お 酒[さけ]の 飲[の]み 過[す]ぎは 健康[けんこう]に 害[がい]があります。	お さけ の のみすぎ は けんこう に がい が あります 。	
\\	お 酒[さけ]の 飲[の]み 過[す]ぎは 健康[けんこう]に
\\	があります。			
\\	被せる	被[かぶ]せる	かぶせる	
\\	本にカバーを被せたの。	本[ほん]にカバーを 被[かぶ]せたの。	ほん に かばー を かぶせた の 。	
\\	本[ほん]にカバーを
\\	の。			
\\	救う	救[すく]う	すくう	
\\	彼女は通りがかりの人に救われたよ。	彼女[かのじょ]は 通[とお]りがかりの 人[ひと]に 救[すく]われたよ。	かのじょ は とおりがかり の ひと に すくわれた よ	
\\	彼女[かのじょ]は 通[とお]りがかりの 人[ひと]に
\\	よ。			
\\	シューズ	シューズ	シューズ	
\\	ジョギング用のシューズは安くないな。	ジョギング 用[よう]のシューズは 安[やす]くないな。	じょぎんぐ よう の しゅーず は やすくない な	
\\	ジョギング 用[よう]の
\\	は 安[やす]くないな。			
\\	救い	救[すく]い	すくい	
\\	娘の存在が私の救いでした。	娘[むすめ]の 存在[そんざい]が 私[わたし]の 救[すく]いでした。	むすめ の そんざい が わたし の すくい でした	
\\	娘[むすめ]の 存在[そんざい]が 私[わたし]の
\\	でした。			
\\	助手	助手[じょしゅ]	じょしゅ	
\\	資料は助手に預けておいてください。	資料[しりょう]は 助手[じょしゅ]に 預[あず]けておいてください。	しりょう は じょしゅ に あずけて おいて ください	
\\	資料[しりょう]は
\\	に 預[あず]けておいてください。			
\\	救助	救助[きゅうじょ]	きゅうじょ	
\\	プールで男の子が救助されました。	プールで 男[おとこ]の 子[こ]が 救助[きゅうじょ]されました。	ぷーる で おとこ の こ が きゅうじょ されました	
\\	プールで 男[おとこ]の 子[こ]が
\\	されました。			
\\	支援	支援[しえん]	しえん	
\\	彼の支援がなかったらどうなっていたか。	彼[かれ]の 支援[しえん]がなかったらどうなっていたか。	かれ の しえん が なかったら どう なって いた か	
\\	彼[かれ]の
\\	がなかったらどうなっていたか。			
\\	援助	援助[えんじょ]	えんじょ	
\\	その国には物資の援助が必要です。	その 国[くに]には 物資[ぶっし]の 援助[えんじょ]が 必要[ひつよう]です。	その くに に は ぶっし の えんじょ が ひつよう です	
\\	その 国[くに]には 物資[ぶっし]の
\\	が 必要[ひつよう]です。			
\\	応援	応援[おうえん]	おうえん	
\\	大勢が応援に駆けつけてくれたよ。	大勢[おおぜい]が 応援[おうえん]に 駆[か]けつけてくれたよ。	おおぜい が おうえん に かけつけて くれた よ	
\\	大勢[おおぜい]が
\\	に 駆[か]けつけてくれたよ。			
\\	小遣い	小遣[こづか]い	こづかい	
\\	おじいちゃんにお小遣いをもらったよ。	おじいちゃんにお 小遣[こづか]いをもらったよ。	おじいちゃん に おこづかい を もらった よ	
\\	おじいちゃんにお
\\	をもらったよ。			
\\	キス	キス	キス	
\\	初めてのキスは海岸でだったの。	初[はじ]めてのキスは 海岸[かいがん]でだったの。	はじめて の きす は かいがん で だった の	
\\	初[はじ]めての
\\	は 海岸[かいがん]でだったの。			
\\	警告	警告[けいこく]	けいこく	
\\	車に駐車違反の警告を貼られました。	車[くるま]に 駐車違反[ちゅうしゃ いはん]の 警告[けいこく]を 貼[は]られました。	くるま に ちゅうしゃ いはん の けいこく を はられました	
\\	車[くるま]に 駐車違反[ちゅうしゃ いはん]の
\\	を 貼[は]られました。			
\\	警官	警官[けいかん]	けいかん	
\\	道で警官に呼び止められた。	道[みち]で 警官[けいかん]に 呼[よ]び 止[と]められた。	みち で けいかん に よびとめられた	
\\	道[みち]で
\\	に 呼[よ]び 止[と]められた。			
\\	管	管[かん]	かん	
\\	ガス管が爆発しました。	ガス 管[かん]が 爆発[ばくはつ]しました。	がすかん が ばくはつ しました	
\\	ガス
\\	が 爆発[ばくはつ]しました。			
\\	犯す	犯[おか]す	おかす	
\\	彼は大きな過ちを犯している。	彼[かれ]は 大[おお]きな 過[あやま]ちを 犯[おか]している。	かれ は おおき な あやまち を おかして いる	
\\	彼[かれ]は 大[おお]きな 過[あやま]ちを
\\	強盗	強盗[ごうとう]	ごうとう	
\\	強盗がカメラに写っていました。	強盗[ごうとう]がカメラに 写[うつ]っていました。	ごうとう が かめら に うつって いました	
\\	がカメラに 写[うつ]っていました。			
\\	自殺	自殺[じさつ]	じさつ	
\\	犯人は警察に捕まる前に自殺しました。	犯人[はんにん]は 警察[けいさつ]に 捕[つか]まる 前[まえ]に 自殺[じさつ]しました。	はんにん は けいさつ に つかまる まえ に じさつ しました	
\\	犯人[はんにん]は 警察[けいさつ]に 捕[つか]まる 前[まえ]に
\\	しました。			
\\	殺人	殺人[さつじん]	さつじん	
\\	その殺人事件は白昼に起こったんだ。	その 殺人[さつじん] 事件[じけん]は 白昼[はくちゅう]に 起[お]こったんだ。	その さつじん じけん は はくちゅう に おこった ん だ	
\\	その
\\	事件[じけん]は 白昼[はくちゅう]に 起[お]こったんだ。			
\\	くるくる	くるくる	くるくる	
\\	猫は私の足元をくるくると回ったの。	猫[ねこ]は 私[わたし]の 足元[あしもと]をくるくると 回[まわ]ったの。	ねこ は わたし の あしもと を くるくる と まわった の	
\\	猫[ねこ]は 私[わたし]の 足元[あしもと]を
\\	と 回[まわ]ったの。			
\\	奪う	奪[うば]う	うばう	
\\	その男は彼女のバッグを奪ったぞ。	その 男[おとこ]は 彼女[かのじょ]のバッグを 奪[うば]ったぞ。	その おとこ は かのじょ の ばっぐ を うばった ぞ	
\\	その 男[おとこ]は 彼女[かのじょ]のバッグを
\\	ぞ。			
\\	戦後	戦後[せんご]	せんご	
\\	戦後の日本は混乱していました。	戦後[せんご]の 日本[にほん]は 混乱[こんらん]していました。	せんご の にほん は こんらん して いました	
\\	の 日本[にほん]は 混乱[こんらん]していました。			
\\	作戦	作戦[さくせん]	さくせん	
\\	彼らは次の試合のために作戦を立てたのよ。	彼[かれ]らは 次[つぎ]の 試合[しあい]のために 作戦[さくせん]を 立[た]てたのよ。	かれら は つぎ の しあい の ため に さくせん を たてた の よ	
\\	彼[かれ]らは 次[つぎ]の 試合[しあい]のために
\\	を 立[た]てたのよ。			
\\	戦場	戦場[せんじょう]	せんじょう	
\\	祖父は戦場に行ったことがあるそうです。	祖父[そふ]は 戦場[せんじょう]に 行[い]ったことがあるそうです。	そふ は せんじょう に いった こと が ある そう です	
\\	祖父[そふ]は
\\	に 行[い]ったことがあるそうです。			
\\	戦前	戦前[せんぜん]	せんぜん	
\\	戦前の生活は今と全く違いました。	戦前[せんぜん]の 生活[せいかつ]は 今[いま]と 全[まった]く 違[ちが]いました。	せんぜん の せいかつ は いま と まったく ちがいました	
\\	の 生活[せいかつ]は 今[いま]と 全[まった]く 違[ちが]いました。			
\\	戦死	戦死[せんし]	せんし	
\\	祖父は戦死しました。	祖父[そふ]は 戦死[せんし]しました。	そふ は せんし しました	
\\	祖父[そふ]は
\\	しました。			
\\	争い	争[あらそ]い	あらそい	
\\	その地域では争いが絶えません。	その 地域[ちいき]では 争[あらそ]いが 絶[た]えません。	その ちいき で は あらそい が たえません	
\\	その 地域[ちいき]では
\\	が 絶[た]えません。			
\\	争う	争[あらそ]う	あらそう	
\\	その2国は資源をめぐって争っています。	その 2国[にこく]は 資源[しげん]をめぐって 争[あらそ]っています。	その にこく は しげん を めぐって あらそって います	
\\	その 2国[にこく]は 資源[しげん]をめぐって
\\	サボる	サボる	サボる	
\\	また仕事をサボっていますね。	また 仕事[しごと]をサボっていますね。	また しごと を さぼって います ね	
\\	また 仕事[しごと]を
\\	ね。			
\\	競技	競技[きょうぎ]	きょうぎ	
\\	彼は個人競技のスポーツが好きです。	彼[かれ]は 個人[こじん] 競技[きょうぎ]のスポーツが 好[す]きです。	かれ は こじん きょうぎ の すぽーつ が すき です	
\\	彼[かれ]は 個人[こじん]
\\	のスポーツが 好[す]きです。			
\\	混雑	混雑[こんざつ]	こんざつ	
\\	今日はデパートが混雑していたよ。	今日[きょう]はデパートが 混雑[こんざつ]していたよ。	きょう は でぱーと が こんざつ して いた よ	
\\	今日[きょう]はデパートが
\\	していたよ。			
\\	混乱	混乱[こんらん]	こんらん	
\\	地震の後、町は大混乱だったよ。	地震[じしん]の 後[あと]、 町[まち]は 大[だい] 混乱[こんらん]だったよ。	じしん の あと まち は だいこんらん だった よ	
\\	地震[じしん]の 後[あと]、 町[まち]は 大[だい]
\\	だったよ。			
\\	捜す	捜[さが]す	さがす	
\\	警察がその男を捜しているの。	警察[けいさつ]がその 男[おとこ]を 捜[さが]しているの。	けいさつ が その おとこ を さがして いる の	
\\	警察[けいさつ]がその 男[おとこ]を
\\	の。			
\\	索引	索引[さくいん]	さくいん	
\\	索引はたいてい本の後ろについていますよ。	索引[さくいん]はたいてい 本[ほん]の 後[うし]ろについていますよ。	さくいん は たいてい ほん の うしろ に ついて います よ	
\\	はたいてい 本[ほん]の 後[うし]ろについていますよ。			
\\	落ち着く	落[お]ち 着[つ]く	おちつく	
\\	私の話を落ち着いて聞いてください。	私[わたし]の 話[はなし]を 落[お]ち 着[つ]いて 聞[き]いてください。	わたし の はなし を おちついて きいて ください	
\\	私[わたし]の 話[はなし]を
\\	聞[き]いてください。			
\\	落ち着き	落[お]ち 着[つ]き	おちつき	
\\	弟は落ち着きが足りません。	弟[おとうと]は 落[お]ち 着[つ]きが 足[た]りません。	おとうと は おちつき が たりません	
\\	弟[おとうと]は
\\	が 足[た]りません。			
\\	ショッピング	ショッピング	ショッピング	
\\	彼女のいちばんの楽しみはショッピングです。	彼女[かのじょ]のいちばんの 楽[たの]しみはショッピングです。	かのじょ の いちばん の たのしみ は しょっぴんぐ です	
\\	彼女[かのじょ]のいちばんの 楽[たの]しみは
\\	です。			
\\	落とし物	落[お]とし 物[もの]	おとしもの	
\\	財布の落とし物を拾いました。	財布[さいふ]の 落[お]とし 物[もの]を 拾[ひろ]いました。	さいふ の おとしもの を ひろいました	
\\	財布[さいふ]の
\\	を 拾[ひろ]いました。			
\\	交流	交流[こうりゅう]	こうりゅう	
\\	もっと他の町との交流を深めましょう。	もっと 他[ほか]の 町[まち]との 交流[こうりゅう]を 深[ふか]めましょう。	もっと ほか の まち と の こうりゅう を ふかめましょう	
\\	もっと 他[ほか]の 町[まち]との
\\	を 深[ふか]めましょう。			
\\	一流	一流[いちりゅう]	いちりゅう	
\\	彼は一流の選手です。	彼[かれ]は 一流[いちりゅう]の 選手[せんしゅ]です。	かれ は いちりゅう の せんしゅ です	
\\	彼[かれ]は
\\	の 選手[せんしゅ]です。			
\\	上流	上流[じょうりゅう]	じょうりゅう	
\\	上流には滝があります。	上流[じょうりゅう]には 滝[たき]があります。	じょうりゅう に は たき が あります	
\\	には 滝[たき]があります。			
\\	下流	下流[かりゅう]	かりゅう	
\\	下流に小さな滝があるよ。	下流[かりゅう]に 小[ちい]さな 滝[たき]があるよ。	かりゅう に ちいさ な たき が ある よ	
\\	に 小[ちい]さな 滝[たき]があるよ。			
\\	海流	海流[かいりゅう]	かいりゅう	
\\	ここで2つの海流が出合っている。	ここで 2[ふた]つの 海流[かいりゅう]が 出合[であ]っている。	ここ で ふたつ の かいりゅう が であって いる	
\\	ここで 2[ふた]つの
\\	が 出合[であ]っている。			
\\	三流	三流[さんりゅう]	さんりゅう	
\\	彼は三流大学を出たが、今は社長だよ。	彼[かれ]は 三流[さんりゅう] 大学[だいがく]を 出[で]たが、 今[いま]は 社長[しゃちょう]だよ。	かれ は さんりゅう だいがく を でた が いま は しゃちょう だ よ	
\\	彼[かれ]は
\\	大学[だいがく]を 出[で]たが、 今[いま]は 社長[しゃちょう]だよ。			
\\	スマート	スマート	スマート	
\\	彼はいつもスマートに行動するね。	彼[かれ]はいつもスマートに 行動[こうどう]するね。	かれ は いつも すまーと に こうどう する ね	
\\	彼[かれ]はいつも
\\	に 行動[こうどう]するね。			
\\	洪水	洪水[こうずい]	こうずい	
\\	洪水でたくさんの家が流されたの。	洪水[こうずい]でたくさんの 家[いえ]が 流[なが]されたの。	こうずい で たくさん の いえ が ながされた の	
\\	でたくさんの 家[いえ]が 流[なが]されたの。			
\\	崩れる	崩[くず]れる	くずれる	
\\	大雨で崖が崩れたね。	大雨[おおあめ]で 崖[がけ]が 崩[くず]れたね。	おおあめ で がけ が くずれた ね	
\\	大雨[おおあめ]で 崖[がけ]が
\\	ね。			
\\	崩す	崩[くず]す	くずす	
\\	彼女は体調を崩しています。	彼女[かのじょ]は 体調[たいちょう]を 崩[くず]しています。	かのじょ は たいちょう を くずして います	
\\	彼女[かのじょ]は 体調[たいちょう]を
\\	水洗	水洗[すいせん]	すいせん	
\\	今はほとんどのトイレが水洗ですよ。	今[いま]はほとんどのトイレが 水洗[すいせん]ですよ。	いま は ほとんど の といれ が すいせん です よ	
\\	今[いま]はほとんどのトイレが
\\	ですよ。			
\\	洗い物	洗[あら]い 物[もの]	あらいもの	
\\	母は台所で洗い物をしています。	母[はは]は 台所[だいどころ]で 洗[あら]い 物[もの]をしています。	はは は だいどころ で あらいもの を して います	
\\	母[はは]は 台所[だいどころ]で
\\	をしています。			
\\	石油	石油[せきゆ]	せきゆ	
\\	日本は石油のほとんどを輸入しています。	日本[にっぽん]は 石油[せきゆ]のほとんどを 輸入[ゆにゅう]しています。	にっぽん は せきゆ の ほとんど を ゆにゅう して います	
\\	日本[にっぽん]は
\\	のほとんどを 輸入[ゆにゅう]しています。			
\\	油絵	油絵[あぶらえ]	あぶらえ	
\\	趣味で油絵を描いています。	趣味[しゅみ]で 油絵[あぶらえ]を 描[か]いています。	しゅみ で あぶらえ を かいています	
\\	趣味[しゅみ]で
\\	を 描[か]いています。			
\\	浮かぶ	浮[う]かぶ	うかぶ	
\\	沖にボートが浮かんでいます。	沖[おき]にボートが 浮[う]かんでいます。	おき に ぼーと が うかんで います	
\\	沖[おき]にボートが
\\	ずらり	ずらり	ずらり	
\\	玄関に靴がずらりと並んでいたな。	玄関[げんかん]に 靴[くつ]がずらりと 並[なら]んでいたな。	げんかん に くつ が ずらり と ならんで いた な	
\\	玄関[げんかん]に 靴[くつ]が
\\	と 並[なら]んでいたな。			
\\	浮かべる	浮[う]かべる	うかべる	
\\	お風呂に花を浮かべて入ったの。	お 風呂[ふろ]に 花[はな]を 浮[う]かべて 入[はい]ったの。	おふろ に はな を うかべて はいった の	
\\	お 風呂[ふろ]に 花[はな]を
\\	入[はい]ったの。			
\\	浮く	浮[う]く	うく	
\\	氷は水に浮きます。	氷[こおり]は 水[みず]に 浮[う]きます。	こおり は みず に うきます	
\\	氷[こおり]は 水[みず]に
\\	沈める	沈[しず]める	しずめる	
\\	彼女はソファーに体を沈めたんだ。	彼女[かのじょ]はソファーに 体[からだ]を 沈[しず]めたんだ。	かのじょ は そふぁー に からだ を しずめた ん だ	
\\	彼女[かのじょ]はソファーに 体[からだ]を
\\	んだ。			
\\	将来	将来[しょうらい]	しょうらい	
\\	将来はパイロットになりたいです。	将来[しょうらい]はパイロットになりたいです。	しょうらい は ぱいろっと に なりたい です 。	
\\	はパイロットになりたいです。			
\\	永遠	永遠[えいえん]	えいえん	
\\	平和は人類の永遠のテーマです。	平和[へいわ]は 人類[じんるい]の 永遠[えいえん]のテーマです。	へいわ は じんるい の えいえん の てーま です	
\\	平和[へいわ]は 人類[じんるい]の
\\	のテーマです。			
\\	永久	永久[えいきゅう]	えいきゅう	
\\	彼は永久に帰らぬ人となったのよ。	彼[かれ]は 永久[えいきゅう]に 帰[かえ]らぬ 人[ひと]となったのよ。	かれ は えいきゅう に かえらぬ ひと と なった の よ	
\\	彼[かれ]は
\\	に 帰[かえ]らぬ 人[ひと]となったのよ。			
\\	河口	河口[かこう]	かこう	
\\	この川の河口は太平洋に注いでいます。	この 川[かわ]の 河口[かこう]は 太平洋[たいへいよう]に 注[そそ]いでいます。	この かわ の かこう は たいへいよう に そそいで います	
\\	この 川[かわ]の
\\	は 太平洋[たいへいよう]に 注[そそ]いでいます。			
\\	いらっしゃる	いらっしゃる	いらっしゃる	
\\	お客様がいらっしゃいました。	お 客様[きゃくさま]がいらっしゃいました。	おきゃくさま が いらっしゃいました	
\\	お 客様[きゃくさま]が
\\	冷ます	冷[さ]ます	さます	
\\	お茶がとても熱かったので冷ましてから飲んだの。	お 茶[ちゃ]がとても 熱[あつ]かったので 冷[さ]ましてから 飲[の]んだの。	おちゃ が とても あつかった の で さまして から のんだ の	
\\	お 茶[ちゃ]がとても 熱[あつ]かったので
\\	から 飲[の]んだの。			
\\	車庫	車庫[しゃこ]	しゃこ	
\\	車を車庫に入れておきました。	車[くるま]を 車庫[しゃこ]に 入[い]れておきました。	くるま を しゃこ に いれて おきました	
\\	車[くるま]を
\\	に 入[い]れておきました。			
\\	金庫	金庫[きんこ]	きんこ	
\\	ホテルの金庫に貴重品を入れたよ。	ホテルの 金庫[きんこ]に 貴重品[きちょうひん]を 入[い]れたよ。	ほてる の きんこ に きちょうひん を いれた よ	
\\	ホテルの
\\	に 貴重品[きちょうひん]を 入[い]れたよ。			
\\	心臓	心臓[しんぞう]	しんぞう	
\\	祖母は心臓が悪いんだ。	祖母[そぼ]は 心臓[しんぞう]が 悪[わる]いんだ。	そぼ は しんぞう が わるい ん だ	
\\	祖母[そぼ]は
\\	が 悪[わる]いんだ。			
\\	軽快	軽快[けいかい]	けいかい	
\\	彼らは軽快なステップで踊り出したの。	彼[かれ]らは 軽快[けいかい]なステップで 踊[おど]り 出[だ]したの。	かれら は けいかい な すてっぷ で おどりだした の	
\\	彼[かれ]らは
\\	なステップで 踊[おど]り 出[だ]したの。			
\\	快い	快[こころよ]い	こころよい	
\\	彼女は快い眠りについています。	彼女[かのじょ]は 快[こころよ]い 眠[ねむ]りについています。	かのじょ は こころよい ねむり に ついて います	
\\	彼女[かのじょ]は
\\	眠[ねむ]りについています。			
\\	快晴	快晴[かいせい]	かいせい	
\\	今日は快晴ですね。	今日[きょう]は 快晴[かいせい]ですね。	きょう は かいせい です ね	
\\	今日[きょう]は
\\	ですね。			
\\	ウナギ	ウナギ	ウナギ	
\\	ウナギの蒲焼きは美味しいね。	ウナギの 蒲焼[かばや]きは 美味[おい]しいね。	うなぎ の かばやき は おいしい ね	
\\	の 蒲焼[かばや]きは 美味[おい]しいね。			
\\	最適	最適[さいてき]	さいてき	
\\	ここは子育てに最適な環境です。	ここは 子育[こそだ]てに 最適[さいてき]な 環境[かんきょう]です。	ここ は こそだて に さいてき な かんきょう です	
\\	ここは 子育[こそだ]てに
\\	な 環境[かんきょう]です。			
\\	指摘	指摘[してき]	してき	
\\	ご指摘いただきありがとうございます。	ご 指摘[してき]いただきありがとうございます。	ごしてき いただき ありがとう ございます	
\\	ご
\\	いただきありがとうございます。			
\\	湿度	湿度[しつど]	しつど	
\\	日本の夏は湿度が高いです。	日本[にほん]の 夏[なつ]は 湿度[しつど]が 高[たか]いです。	にほん の なつ は しつど が たかい です	
\\	日本[にほん]の 夏[なつ]は
\\	が 高[たか]いです。			
\\	湿っぽい	湿[しめ]っぽい	しめっぽい	
\\	雨の日が続いて家の中が湿っぽいな。	雨[あめ]の 日[ひ]が 続[つづ]いて 家[いえ]の 中[なか]が 湿[しめ]っぽいな。	あめ の ひ が つづいて いえ の なか が しめっぽい な	
\\	雨[あめ]の 日[ひ]が 続[つづ]いて 家[いえ]の 中[なか]が
\\	な。			
\\	湿気	湿気[しっけ]	しっけ	
\\	この部屋は暗くて湿気が多いね。	この 部屋[へや]は 暗[くら]くて 湿気[しっけ]が 多[おお]いね。	この へや は くらくて しっけ が おおい ね	
\\	この 部屋[へや]は 暗[くら]くて
\\	が 多[おお]いね。			
\\	湿る	湿[しめ]る	しめる	
\\	洗濯物がまだ湿っています。	洗濯物[せんたくもの]がまだ 湿[しめ]っています。	せんたくもの が まだ しめって います	
\\	洗濯物[せんたくもの]がまだ
\\	汚染	汚染[おせん]	おせん	
\\	その川の水は汚染されています。	その 川[かわ]の 水[みず]は 汚染[おせん]されています。	その かわ の みず は おせん されて います	
\\	その 川[かわ]の 水[みず]は
\\	されています。			
\\	景気	景気[けいき]	けいき	
\\	景気が回復してきたね。	景気[けいき]が 回復[かいふく]してきたね。	けいき が かいふく して きた ね	
\\	が 回復[かいふく]してきたね。			
\\	ガード	ガード	ガード	
\\	歩道に新しくガードが付きましたね。	歩道[ほどう]に 新[あたら]しくガードが 付[つ]きましたね。	ほどう に あたらしく がーど が つきました ね	
\\	歩道[ほどう]に 新[あたら]しく
\\	が 付[つ]きましたね。			
\\	影	影[かげ]	かげ	
\\	窓に男性の影が映っています。	窓[まど]に 男性[だんせい]の 影[かげ]が 映[うつ]っています。	まど に だんせい の かげ が うつって います	
\\	窓[まど]に 男性[だんせい]の
\\	が 映[うつ]っています。			
\\	境界	境界[きょうかい]	きょうかい	
\\	ここは隣の市との境界です。	ここは 隣[となり]の 市[し]との 境界[きょうかい]です。	ここ は となり の し と の きょうかい です	
\\	ここは 隣[となり]の 市[し]との
\\	です。			
\\	環境	環境[かんきょう]	かんきょう	
\\	引っ越して環境が変わりました。	引[ひ]っ 越[こ]して 環境[かんきょう]が 変[か]わりました。	ひっこして かんきょう が かわりました	
\\	引[ひ]っ 越[こ]して
\\	が 変[か]わりました。			
\\	国境	国境[こっきょう]	こっきょう	
\\	あの山のすぐ近くが国境です。	あの 山[やま]のすぐ 近[ちか]くが 国境[こっきょう]です。	あの やま の すぐ ちかく が こっきょう です	
\\	あの 山[やま]のすぐ 近[ちか]くが
\\	です。			
\\	境	境[さかい]	さかい	
\\	2つの市の境に川が流れているの。	2[ふた]つの 市[し]の 境[さかい]に 川[かわ]が 流[なが]れているの。	ふたつ の し の さかい に かわ が ながれて いる の	
\\	2[ふた]つの 市[し]の
\\	に 川[かわ]が 流[なが]れているの。			
\\	観察	観察[かんさつ]	かんさつ	
\\	花の成長を観察して日記を書きなさい。	花[はな]の 成長[せいちょう]を 観察[かんさつ]して 日記[にっき]を 書[か]きなさい。	はな の せいちょう を かんさつ して にっき を かきなさい	
\\	花[はな]の 成長[せいちょう]を
\\	して 日記[にっき]を 書[か]きなさい。			
\\	外観	外観[がいかん]	がいかん	
\\	そのモダンな外観の建物が大使館です。	そのモダンな 外観[がいかん]の 建物[たてもの]が 大使館[たいしかん]です。	その もだん な がいかん の たてもの が たいしかん です	
\\	そのモダンな
\\	の 建物[たてもの]が 大使館[たいしかん]です。			
\\	かぼちゃ	かぼちゃ	かぼちゃ	
\\	かぼちゃのスープは美味しいです。	かぼちゃのスープは 美味[おい]しいです。	かぼちゃの すーぷ は おいしい です	
\\	のスープは 美味[おい]しいです。			
\\	客観的	客観的[きゃっかんてき]	きゃっかんてき	
\\	彼は自分の状況を客観的に見てみたのね。	彼[かれ]は 自分[じぶん]の 状況[じょうきょう]を 客観的[きゃっかんてき]に 見[み]てみたのね。	かれ は じぶん の じょうきょう を きゃっかんてき に みて みた の ね	
\\	彼[かれ]は 自分[じぶん]の 状況[じょうきょう]を
\\	に 見[み]てみたのね。			
\\	主観	主観[しゅかん]	しゅかん	
\\	主観だけで物事を見てはいけない。	主観[しゅかん]だけで 物事[ものごと]を 見[み]てはいけない。	しゅかん だけ で ものごと を みて は いけない	
\\	だけで 物事[ものごと]を 見[み]てはいけない。			
\\	主観的	主観的[しゅかんてき]	しゅかんてき	
\\	それは主観的な意見だ。	それは 主観的[しゅかんてき]な 意見[いけん]だ。	それ は しゅかんてき な いけん だ	
\\	それは
\\	な 意見[いけん]だ。			
\\	観客	観客[かんきゃく]	かんきゃく	
\\	観客は興奮していたよ。	観客[かんきゃく]は 興奮[こうふん]していたよ。	かんきゃく は こうふん して いた よ	
\\	は 興奮[こうふん]していたよ。			
\\	観光	観光[かんこう]	かんこう	
\\	京都で3日間観光しました。	京都[きょうと]で 3日間[みっかかん] 観光[かんこう]しました。	きょうと で みっかかん かんこう しました	
\\	京都[きょうと]で 3日間[みっかかん]
\\	しました。			
\\	観測	観測[かんそく]	かんそく	
\\	先生が星の観測に連れて行ってくれました。	先生[せんせい]が 星[ほし]の 観測[かんそく]に 連[つ]れて 行[い]ってくれました。	せんせい が ほし の かんそく に つれて いって くれました	
\\	先生[せんせい]が 星[ほし]の
\\	に 連[つ]れて 行[い]ってくれました。			
\\	推測	推測[すいそく]	すいそく	
\\	それは彼の推測にすぎない。	それは 彼[かれ]の 推測[すいそく]にすぎない。	それ は かれ の すいそく に すぎない	
\\	それは 彼[かれ]の
\\	にすぎない。			
\\	クイズ	クイズ	クイズ	
\\	クイズ番組を見るのが好きです。	クイズ 番組[ばんぐみ]を 見[み]るのが 好[す]きです。	くいず ばんぐみ を みる の が すき です	
\\	番組[ばんぐみ]を 見[み]るのが 好[す]きです。			
\\	宇宙	宇宙[うちゅう]	うちゅう	
\\	宇宙の謎は限りなく大きいの。	宇宙[うちゅう]の 謎[なぞ]は 限[かぎ]りなく 大[おお]きいの。	うちゅう の なぞ は かぎり なく おおきい の	
\\	の 謎[なぞ]は 限[かぎ]りなく 大[おお]きいの。			
\\	衛星	衛星[えいせい]	えいせい	
\\	衛星が打ち上げられましたね。	衛星[えいせい]が 打[う]ち 上[あ]げられましたね。	えいせい が うちあげられました ね	
\\	が 打[う]ち 上[あ]げられましたね。			
\\	衛生	衛生[えいせい]	えいせい	
\\	衛生には十分気を付けましょう。	衛生[えいせい]には 十分気[じゅうぶん き]を 付[つ]けましょう。	えいせい に は じゅうぶん き を つけましょう	
\\	には 十分気[じゅうぶん き]を 付[つ]けましょう。			
\\	衛生的	衛生的[えいせいてき]	えいせいてき	
\\	このレストランは衛生的で安心です。	このレストランは 衛生的[えいせいてき]で 安心[あんしん]です。	この れすとらん は えいせいてき で あんしん です	
\\	このレストランは
\\	で 安心[あんしん]です。			
\\	球	球[きゅう]	きゅう	
\\	彼は球技が大好きです。	彼[かれ]は 球[きゅう] 技[ぎ]が 大好[だいす]きです。	かれ は きゅうぎ が だいすき です	
\\	彼[かれ]は
\\	技[ぎ]が 大好[だいす]きです。			
\\	震度	震度[しんど]	しんど	
\\	今朝の地震は震度3でしたよ。	今朝[けさ]の 地震[じしん]は 震度[しんど] 3[さん]でしたよ。	けさ の じしん は しんど さん でした よ	
\\	今朝[けさ]の 地震[じしん]は
\\	3[さん]でしたよ。			
\\	振動	振動[しんどう]	しんどう	
\\	車の振動で棚の荷物が落ちた。	車[くるま]の 振動[しんどう]で 棚[たな]の 荷物[にもつ]が 落[お]ちた。	くるま の しんどう で たな の にもつ が おちた 。	
\\	車[くるま]の
\\	で 棚[たな]の 荷物[にもつ]が 落[お]ちた。			
\\	クリーム	クリーム	クリーム	
\\	私は洗い物をしたあと、手にクリームをぬります。	私[わたし]は 洗[あら]い 物[もの]をしたあと、 手[て]にクリームをぬります。	わたし は あらいもの を した あと て に くりーむ を ぬります	
\\	私[わたし]は 洗[あら]い 物[もの]をしたあと、 手[て]に
\\	をぬります。			
\\	神経	神経[しんけい]	しんけい	
\\	彼は神経が細かいですね。	彼[かれ]は 神経[しんけい]が 細[こま]かいですね。	かれ は しんけい が こまかい です ね	
\\	彼[かれ]は
\\	が 細[こま]かいですね。			
\\	神様	神様[かみさま]	かみさま	
\\	神様にお願いしました。	神様[かみさま]にお 願[ねが]いしました。	かみさま に おねがい しました	
\\	にお 願[ねが]いしました。			
\\	神道	神道[しんとう]	しんとう	
\\	彼女の家は神道です。	彼女[かのじょ]の 家[いえ]は 神道[しんとう]です。	かのじょ の いえ は しんとう です	
\\	彼女[かのじょ]の 家[いえ]は
\\	です。			
\\	精神	精神[せいしん]	せいしん	
\\	彼女は今、精神が不安定だ。	彼女[かのじょ]は 今[いま]、 精神[せいしん]が 不安定[ふあんてい]だ。	かのじょ は いま せいしん が ふあんてい だ	
\\	彼女[かのじょ]は 今[いま]、
\\	が 不安定[ふあんてい]だ。			
\\	精算	精算[せいさん]	せいさん	
\\	降りる駅で料金を精算して下さい。	降[お]りる 駅[えき]で 料金[りょうきん]を 精算[せいさん]して 下[くだ]さい。	おりる えき で りょうきん を せいさん して ください	
\\	降[お]りる 駅[えき]で 料金[りょうきん]を
\\	して 下[くだ]さい。			
\\	厳重	厳重[げんじゅう]	げんじゅう	
\\	上司から厳重に注意されました。	上司[じょうし]から 厳重[げんじゅう]に 注意[ちゅうい]されました。	じょうし から げんじゅう に ちゅうい されました	
\\	上司[じょうし]から
\\	に 注意[ちゅうい]されました。			
\\	貴重	貴重[きちょう]	きちょう	
\\	彼は貴重な体験をしたわね。	彼[かれ]は 貴重[きちょう]な 体験[たいけん]をしたわね。	かれ は きちょう な たいけん を した わ ね	
\\	彼[かれ]は
\\	な 体験[たいけん]をしたわね。			
\\	貴重品	貴重品[きちょうひん]	きちょうひん	
\\	貴重品は自分で持っていてください。	貴重品[きちょうひん]は 自分[じぶん]で 持[も]っていてください。	きちょうひん は じぶん で もって いて ください	
\\	は 自分[じぶん]で 持[も]っていてください。			
\\	ぐるぐる	ぐるぐる	ぐるぐる	
\\	犬が自分のしっぽを追いかけてぐるぐる回っている。	犬[いぬ]が 自分[じぶん]のしっぽを 追[お]いかけてぐるぐる 回[まわ]っている。	いぬ が じぶん の しっぽ を おいかけて ぐるぐる まわって いる	
\\	犬[いぬ]が 自分[じぶん]のしっぽを 追[お]いかけて
\\	回[まわ]っている。			
\\	跡	跡[あと]	あと	
\\	ここにタイヤの跡があるわ。	ここにタイヤの 跡[あと]があるわ。	ここ に たいや の あと が ある わ	
\\	ここにタイヤの
\\	があるわ。			
\\	足跡	足跡[あしあと]	あしあと	
\\	雪の上にうさぎの足跡があった。	雪[ゆき]の 上[うえ]にうさぎの 足跡[あしあと]があった。	ゆき の うえ に うさぎ の あしあと が あった	
\\	雪[ゆき]の 上[うえ]にうさぎの
\\	があった。			
\\	好奇心	好奇心[こうきしん]	こうきしん	
\\	子供は好奇心でいっぱいだね。	子供[こども]は 好奇心[こうきしん]でいっぱいだね。	こども は こうきしん で いっぱい だ ね	
\\	子供[こども]は
\\	でいっぱいだね。			
\\	奇跡	奇跡[きせき]	きせき	
\\	彼のマジックはまるで奇跡です。	彼[かれ]のマジックはまるで 奇跡[きせき]です。	かれ の まじっく は まるで きせき です	
\\	彼[かれ]のマジックはまるで
\\	です。			
\\	奇数	奇数[きすう]	きすう	
\\	3は奇数です。	3[さん]は 奇数[きすう]です。	さん は きすう です	
\\	3[さん]は
\\	です。			
\\	経歴	経歴[けいれき]	けいれき	
\\	あなたの経歴をメールで送ってください。	あなたの 経歴[けいれき]をメールで 送[おく]ってください。	あなた の けいれき を めーる で おくって ください	
\\	あなたの
\\	をメールで 送[おく]ってください。			
\\	学歴	学歴[がくれき]	がくれき	
\\	その職は大卒の学歴が必要だ。	その 職[しょく]は 大卒[だいそつ]の 学歴[がくれき]が 必要[ひつよう]だ。	その しょく は だいそつ の がくれき が ひつよう だ	
\\	その 職[しょく]は 大卒[だいそつ]の
\\	が 必要[ひつよう]だ。			
\\	さぞ	さぞ	さぞ	
\\	それはさぞがっかりしたことでしょう。	それはさぞがっかりしたことでしょう。	それはさぞがっかりしたことでしょう。	
\\	それは
\\	がっかりしたことでしょう。			
\\	王	王[おう]	おう	
\\	王の墓を見学したよ。	王[おう]の 墓[はか]を 見学[けんがく]したよ。	おう の はか を けんがく した よ	
\\	の 墓[はか]を 見学[けんがく]したよ。			
\\	王様	王様[おうさま]	おうさま	
\\	その国の王様はとても賢い。	その 国[くに]の 王様[おうさま]はとても 賢[かしこ]い。	その くに の おうさま は とても かしこい	
\\	その 国[くに]の
\\	はとても 賢[かしこ]い。			
\\	建築	建築[けんちく]	けんちく	
\\	彼らは家を建築中です。	彼[かれ]らは 家[いえ]を 建築[けんちく] 中[ちゅう]です。	かれら は いえ を けんちくちゅう です	
\\	彼[かれ]らは 家[いえ]を
\\	中[ちゅう]です。			
\\	新築	新築[しんちく]	しんちく	
\\	彼は去年、家を新築しました。	彼[かれ]は 去年[きょねん]、 家[いえ]を 新築[しんちく]しました。	かれ は きょねん いえ を しんちく しました	
\\	彼[かれ]は 去年[きょねん]、 家[いえ]を
\\	しました。			
\\	構想	構想[こうそう]	こうそう	
\\	彼は新しい小説の構想を練っているの。	彼[かれ]は 新[あたら]しい 小説[しょうせつ]の 構想[こうそう]を 練[ね]っているの。	かれ は あたらしい しょうせつ の こうそう を ねって いる の	
\\	彼[かれ]は 新[あたら]しい 小説[しょうせつ]の
\\	を 練[ね]っているの。			
\\	構える	構[かま]える	かまえる	
\\	彼はあの通りに店を構えているの。	彼[かれ]はあの 通[とお]りに 店[みせ]を 構[かま]えているの。	かれ は あの とおり に みせ を かまえて いる の	
\\	彼[かれ]はあの 通[とお]りに 店[みせ]を
\\	の。			
\\	結構	結構[けっこう]	けっこう	
\\	お腹がいっぱいなのでお代わりは結構です。	お 腹[なか]がいっぱいなのでお 代[か]わりは 結構[けっこう]です。	おなか が いっぱい な の で おかわり は けっこう です	
\\	お 腹[なか]がいっぱいなのでお 代[か]わりは
\\	です。			
\\	サングラス	サングラス	サングラス	
\\	彼はサングラスが似合うね。	彼[かれ]はサングラスが 似合[にあ]うね。	かれ は さんぐらす が にあう ね	
\\	彼[かれ]は
\\	が 似合[にあ]うね。			
\\	結構	結構[けっこう]	けっこう	
\\	彼女は結構めがねが似合うね。	彼女[かのじょ]は 結構[けっこう]めがねが 似合[にあ]うね。	かのじょ は けっこう めがね が にあう ね	
\\	彼女[かのじょ]は
\\	めがねが 似合[にあ]うね。			
\\	構う	構[かま]う	かまう	
\\	子供に構い過ぎてはいけない。	子供[こども]に 構[かま]い 過[す]ぎてはいけない。	こども に かまいすぎて は いけない	
\\	子供[こども]に
\\	過[す]ぎてはいけない。			
\\	位	位[くらい]	くらい	
\\	位が上がれば責任も増えるものだ。	位[くらい]が 上[あ]がれば 責任[せきにん]も 増[ふ]えるものだ。	くらい が あがれ ば せきにん も ふえる もの だ	
\\	が 上[あ]がれば 責任[せきにん]も 増[ふ]えるものだ。			
\\	設置	設置[せっち]	せっち	
\\	お店に防犯カメラが設置されたわね。	お 店[みせ]に 防犯[ぼうはん]カメラが 設置[せっち]されたわね。	お みせ に ぼうはん かめら が せっち された わ ね	
\\	お 店[みせ]に 防犯[ぼうはん]カメラが
\\	されたわね。			
\\	距離	距離[きょり]	きょり	
\\	彼らは長い距離を歩き続けたんだ。	彼[かれ]らは 長[なが]い 距離[きょり]を 歩[ある]き 続[つづ]けたんだ。	かれら は ながい きょり を あるきつづけた ん だ	
\\	彼[かれ]らは 長[なが]い
\\	を 歩[ある]き 続[つづ]けたんだ。			
\\	一周	一周[いっしゅう]	いっしゅう	
\\	私たちは庭園を一周しました。	私[わたし]たちは 庭園[ていえん]を 一周[いっしゅう]しました。	わたしたち は ていえん を いっしゅう しました	
\\	私[わたし]たちは 庭園[ていえん]を
\\	しました。			
\\	辺り	辺[あた]り	あたり	
\\	辺りを見回したの。	辺[あた]りを 見回[みまわ]したの。	あたり を みまわした の	
\\	を 見回[みまわ]したの。			
\\	周囲	周囲[しゅうい]	しゅうい	
\\	大声で話すと周囲の人に迷惑ですよ。	大声[おおごえ]で 話[はな]すと 周囲[しゅうい]の 人[ひと]に 迷惑[めいわく]ですよ。	おおごえ で はなす と しゅうい の ひと に めいわく です よ	
\\	大声[おおごえ]で 話[はな]すと
\\	の 人[ひと]に 迷惑[めいわく]ですよ。			
\\	きっちり	きっちり	きっちり	
\\	彼女はきっちり3時に来たわ。	彼女[かのじょ]はきっちり 3時[さんじ]に 来[き]たわ。	かのじょ は きっちり さんじ に きた わ	
\\	彼女[かのじょ]は
\\	3時[さんじ]に 来[き]たわ。			
\\	囲む	囲[かこ]む	かこむ	
\\	久しぶりに家族全員で食卓を囲みました。	久[ひさ]しぶりに 家族全員[かぞく ぜんいん]で 食卓[しょくたく]を 囲[かこ]みました。	ひさしぶり に かぞく ぜんいん で しょくたく を かこみました	
\\	久[ひさ]しぶりに 家族全員[かぞく ぜんいん]で 食卓[しょくたく]を
\\	横断	横断[おうだん]	おうだん	
\\	道路を横断するときは注意して。	道路[どうろ]を 横断[おうだん]するときは 注意[ちゅうい]して。	どうろ を おうだん する とき は ちゅうい して	
\\	道路[どうろ]を
\\	するときは 注意[ちゅうい]して。			
\\	継続	継続[けいぞく]	けいぞく	
\\	ものごとは根気よく継続することが大切です。	ものごとは 根気[こんき]よく 継続[けいぞく]することが 大切[たいせつ]です。	もの ごと は こんき よく けいぞく する こと が たいせつ です	
\\	ものごとは 根気[こんき]よく
\\	することが 大切[たいせつ]です。			
\\	欧米	欧米[おうべい]	おうべい	
\\	その会社は欧米に進出しているよね。	その 会社[かいしゃ]は 欧米[おうべい]に 進出[しんしゅつ]しているよね。	その かいしゃ は おうべい に しんしゅつ して いる よ ね	
\\	その 会社[かいしゃ]は
\\	に 進出[しんしゅつ]しているよね。			
\\	州	州[しゅう]	しゅう	
\\	来月、隣の州に引っ越します。	来月[らいげつ]、 隣[となり]の 州[しゅう]に 引[ひ]っ 越[こ]します。	らいげつ となり の しゅう に ひっこします	
\\	来月[らいげつ]、 隣[となり]の
\\	に 引[ひ]っ 越[こ]します。			
\\	陸	陸[おか]	おか	
\\	私たちは舟を降りて、陸に上がったんだ。	私[わたし]たちは 舟[ふね]を 降[お]りて、 陸[おか]に 上[あ]がったんだ。	わたしたち は ふね を おりて おか に あがった ん だ	
\\	私[わたし]たちは 舟[ふね]を 降[お]りて、
\\	に 上[あ]がったんだ。			
\\	極めて	極[きわ]めて	きわめて	
\\	これは極めて重要な問題です。	これは 極[きわ]めて 重要[じゅうよう]な 問題[もんだい]です。	これ は きわめて じゅうよう な もんだい です	
\\	これは
\\	重要[じゅうよう]な 問題[もんだい]です。			
\\	きゅうり	きゅうり	きゅうり	
\\	きゅうりに味噌をつけて食べたの。	きゅうりに 味噌[みそ]をつけて 食[た]べたの。	きゅうり に みそ を つけて たべた の	
\\	に 味噌[みそ]をつけて 食[た]べたの。			
\\	消極的	消極的[しょうきょくてき]	しょうきょくてき	
\\	消極的な人は成功しないよ。	消極的[しょうきょくてき]な 人[ひと]は 成功[せいこう]しないよ。	しょうきょくてき な ひと は せいこう しない よ	
\\	な 人[ひと]は 成功[せいこう]しないよ。			
\\	極端	極端[きょくたん]	きょくたん	
\\	それは極端な意見だね。	それは 極端[きょくたん]な 意見[いけん]だね。	それ は きょくたん な いけん だ ね	
\\	それは
\\	な 意見[いけん]だね。			
\\	最先端	最先端[さいせんたん]	さいせんたん	
\\	そのカメラには最先端の技術が使われています。	そのカメラには 最先端[さいせんたん]の 技術[ぎじゅつ]が 使[つか]われています。	その かめら に は さいせんたん の ぎじゅつ が つかわれて います	
\\	そのカメラには
\\	の 技術[ぎじゅつ]が 使[つか]われています。			
\\	一緒	一緒[いっしょ]	いっしょ	
\\	一緒に食事しようか。	一緒[いっしょ]に 食事[しょくじ]しようか。	いっしょ に しょくじ しよう か	
\\	に 食事[しょくじ]しようか。			
\\	外貨	外貨[がいか]	がいか	
\\	外貨を両替しました。	外貨[がいか]を 両替[りょうがえ]しました。	がいか を りょうがえ しました	
\\	を 両替[りょうがえ]しました。			
\\	貨物	貨物[かもつ]	かもつ	
\\	このトランクは貨物で送ろう。	このトランクは 貨物[かもつ]で 送[おく]ろう。	この とらんく は かもつ で おくろう	
\\	このトランクは
\\	で 送[おく]ろう。			
\\	車輪	車輪[しゃりん]	しゃりん	
\\	このトラックの車輪は頑丈そうですね。	このトラックの 車輪[しゃりん]は 頑丈[がんじょう]そうですね。	この とらっく の しゃりん は がんじょう そう です ね	
\\	このトラックの
\\	は 頑丈[がんじょう]そうですね。			
\\	すっと	すっと	すっと	
\\	言いたいことを言ったら胸がすっとした。	言[い]いたいことを 言[い]ったら 胸[むね]がすっとした。	いいたい こと を いったら むね が すっと した	
\\	言[い]いたいことを 言[い]ったら 胸[むね]が
\\	した。			
\\	回復	回復[かいふく]	かいふく	
\\	体がすっかり回復した。	体[からだ]がすっかり 回復[かいふく]した。	からだ が すっかり かいふく した	
\\	体[からだ]がすっかり
\\	した。			
\\	渋い	渋[しぶ]い	しぶい	
\\	この柿は渋いね。	この 柿[かき]は 渋[しぶ]いね。	この かき は しぶい ね	
\\	この 柿[かき]は
\\	ね。			
\\	渋滞	渋滞[じゅうたい]	じゅうたい	
\\	この国道はよく渋滞します。	この 国道[こくどう]はよく 渋滞[じゅうたい]します。	この こくどう は よく じゅうたい します	
\\	この 国道[こくどう]はよく
\\	します。			
\\	一帯	一帯[いったい]	いったい	
\\	この辺一帯はリンゴ畑です。	この 辺[へん] 一帯[いったい]はリンゴ 畑[ばたけ]です。	この へん いったい は りんごばたけ です	
\\	この 辺[へん]
\\	はリンゴ 畑[ばたけ]です。			
\\	帯	帯[おび]	おび	
\\	この帯は長過ぎます。	この 帯[おび]は 長過[なが す]ぎます。	この おび は なが すぎます	
\\	この
\\	は 長過[なが す]ぎます。			
\\	温帯	温帯[おんたい]	おんたい	
\\	日本は温帯にあります。	日本[にっぽん]は 温帯[おんたい]にあります。	にっぽん は おんたい に あります	
\\	日本[にっぽん]は
\\	にあります。			
\\	守備	守備[しゅび]	しゅび	
\\	そのチームは守備が甘いね。	そのチームは 守備[しゅび]が 甘[あま]いね。	その ちーむ は しゅび が あまい ね	
\\	そのチームは
\\	が 甘[あま]いね。			
\\	帰宅	帰宅[きたく]	きたく	
\\	夜の11時に帰宅しました。	夜[よる]の 11時[じゅういちじ]に 帰宅[きたく]しました。	よる の じゅういちじ に きたく しました	
\\	夜[よる]の 11時[じゅういちじ]に
\\	しました。			
\\	すらすら	すらすら	すらすら	
\\	その小学生は難しい本をすらすら読んだの。	その 小学生[しょうがくせい]は 難[むずか]しい 本[ほん]をすらすら 読[よ]んだの。	その しょうがくせい は むずかしい ほん を すらすら よんだ の	
\\	その 小学生[しょうがくせい]は 難[むずか]しい 本[ほん]を
\\	読[よ]んだの。			
\\	住宅地	住宅地[じゅうたくち]	じゅうたくち	
\\	その住宅地は便利な場所にあるね。	その 住宅地[じゅうたくち]は 便利[べんり]な 場所[ばしょ]にあるね。	その じゅうたくち は べんり な ばしょ に ある ね	
\\	その
\\	は 便利[べんり]な 場所[ばしょ]にあるね。			
\\	宛先	宛先[あてさき]	あてさき	
\\	宛先不明で手紙が戻ってきたの。	宛先[あてさき] 不明[ふめい]で 手紙[てがみ]が 戻[もど]ってきたの。	あてさき ふめい で てがみ が もどって きた の	
\\	不明[ふめい]で 手紙[てがみ]が 戻[もど]ってきたの。			
\\	宛名	宛名[あてな]	あてな	
\\	手紙に宛名を書き込んだよ。	手紙[てがみ]に 宛名[あてな]を 書[か]き 込[こ]んだよ。	てがみ に あてな を かきこんだ よ	
\\	手紙[てがみ]に
\\	を 書[か]き 込[こ]んだよ。			
\\	後戻り	後戻[あともど]り	あともどり	
\\	彼女は途中で後戻りしました。	彼女[かのじょ]は 途中[とちゅう]で 後戻[あともど]りしました。	かのじょ は とちゅう で あともどり しました	
\\	彼女[かのじょ]は 途中[とちゅう]で
\\	しました。			
\\	起こす	起[お]こす	おこす	
\\	彼は暴力事件を起こしたんだよ。	彼[かれ]は 暴力事件[ぼうりょく じけん]を 起[お]こしたんだよ。	かれ は ぼうりょく じけん を おこした ん だ よ	
\\	彼[かれ]は 暴力事件[ぼうりょく じけん]を
\\	んだよ。			
\\	起源	起源[きげん]	きげん	
\\	言葉の起源に大変興味があります。	言葉[ことば]の 起源[きげん]に 大変興味[たいへん きょうみ]があります。	ことば の きげん に たいへん きょうみ が あります	
\\	言葉[ことば]の
\\	に 大変興味[たいへん きょうみ]があります。			
\\	起き上がる	起[お]き 上[あ]がる	おきあがる	
\\	弟はようやくベッドから起き上がった。	弟[おとうと]はようやくベッドから 起[お]き 上[あ]がった。	おとうと は ようやく べっど から おきあがった。	
\\	弟[おとうと]はようやくベッドから
\\	うろうろ	うろうろ	うろうろ	
\\	その男はロビーをしばらくうろうろしていたよ。	その 男[おとこ]はロビーをしばらくうろうろしていたよ。	その おとこ は ろびー を しばらく うろうろ して いた よ	
\\	その 男[おとこ]はロビーをしばらく
\\	していたよ。			
\\	寝室	寝室[しんしつ]	しんしつ	
\\	寝室の壁紙を張り替えました。	寝室[しんしつ]の 壁紙[かべがみ]を 張[は]り 替[か]えました。	しんしつ の かべがみ を はりかえました	
\\	の 壁紙[かべがみ]を 張[は]り 替[か]えました。			
\\	静まる	静[しず]まる	しずまる	
\\	台風が去って風が静まったね。	台風[たいふう]が 去[さ]って 風[かぜ]が 静[しず]まったね。	たいふう が さって かぜ が しずまった ね	
\\	台風[たいふう]が 去[さ]って 風[かぜ]が
\\	ね。			
\\	休暇	休暇[きゅうか]	きゅうか	
\\	今度の休暇にフィリピンに行きます。	今度[こんど]の 休暇[きゅうか]にフィリピンに 行[い]きます。	こんど の きゅうか に ふぃりぴん に いきます	
\\	今度[こんど]の
\\	にフィリピンに 行[い]きます。			
\\	片手	片手[かたて]	かたて	
\\	片手運転は危ないよ。	片手[かたて] 運転[うんてん]は 危[あぶ]ないよ。	かたて うんてん は あぶないよ	
\\	運転[うんてん]は 危[あぶ]ないよ。			
\\	片方	片方[かたほう]	かたほう	
\\	片方の目がかゆいです。	片方[かたほう]の 目[め]がかゆいです。	かたほう の め が かゆい です	
\\	の 目[め]がかゆいです。			
\\	片側	片側[かたがわ]	かたがわ	
\\	この道は片側通行です。	この 道[みち]は 片側[かたがわ] 通行[つうこう]です。	この みち は かたがわ つうこう です	
\\	この 道[みち]は
\\	通行[つうこう]です。			
\\	裏切る	裏切[うらぎ]る	うらぎる	
\\	彼は仲間を裏切ったの。	彼[かれ]は 仲間[なかま]を 裏切[うらぎ]ったの。	かれ は なかま を うらぎった の	
\\	彼[かれ]は 仲間[なかま]を
\\	の。			
\\	おまけ	おまけ	おまけ	
\\	八百屋さんでたくさんおまけしてもらったよ。	八百屋[やおや]さんでたくさんおまけしてもらったよ。	やおや さん で たくさん おまけ して もらった よ	
\\	八百屋[やおや]さんでたくさん
\\	してもらったよ。			
\\	裏口	裏口[うらぐち]	うらぐち	
\\	裏口へお回りください。	裏口[うらぐち]へお 回[まわ]りください。	うらぐち へ おまわり ください	
\\	へお 回[まわ]りください。			
\\	裏返し	裏返[うらがえ]し	うらがえし	
\\	シャツを裏返しに着ているよ。	シャツを 裏返[うらがえ]しに 着[き]ているよ。	しゃつ を うらがえし に きて いる よ	
\\	シャツを
\\	に 着[き]ているよ。			
\\	裏門	裏門[うらもん]	うらもん	
\\	奴は裏門から出てきたぜ。	奴[やつ]は 裏門[うらもん]から 出[で]てきたぜ。	やつ は うらもん から でて きた ぜ	
\\	奴[やつ]は
\\	から 出[で]てきたぜ。			
\\	裏表	裏表[うらおもて]	うらおもて	
\\	シャツを裏表に着ていますよ。	シャツを 裏表[うらおもて]に 着[き]ていますよ。	しゃつ を うらおもて に きて います よ	
\\	シャツを
\\	に 着[き]ていますよ。			
\\	項目	項目[こうもく]	こうもく	
\\	論文の項目を分かりやすく整理したわ。	論文[ろんぶん]の 項目[こうもく]を 分[わ]かりやすく 整理[せいり]したわ。	ろんぶん の こうもく を わかりやすく せいり した わ	
\\	論文[ろんぶん]の
\\	を 分[わ]かりやすく 整理[せいり]したわ。			
\\	印象	印象[いんしょう]	いんしょう	
\\	彼から良い印象を受けたわ。	彼[かれ]から 良[い]い 印象[いんしょう]を 受[う]けたわ。	かれ から いい いんしょう を うけた わ	
\\	彼[かれ]から 良[い]い
\\	を 受[う]けたわ。			
\\	印	印[しるし]	しるし	
\\	間違いに印を付けておきました。	間違[まちが]いに 印[しるし]を 付[つ]けておきました。	まちがい に しるし を つけて おきました	
\\	間違[まちが]いに
\\	を 付[つ]けておきました。			
\\	クッキー	クッキー	クッキー	
\\	クッキーと紅茶をいただきました。	クッキーと 紅茶[こうちゃ]をいただきました。	くっきー と こうちゃ を いただきました	
\\	と 紅茶[こうちゃ]をいただきました。			
\\	印刷	印刷[いんさつ]	いんさつ	
\\	年賀状を印刷したの。	年賀状[ねんがじょう]を 印刷[いんさつ]したの。	ねんがじょう を いんさつ した の	
\\	年賀状[ねんがじょう]を
\\	したの。			
\\	週刊	週刊[しゅうかん]	しゅうかん	
\\	この週刊誌を毎週買っているの。	この 週刊[しゅうかん] 誌[し]を 毎週買[まいしゅう か]っているの。	この しゅうかんし を まいしゅう かって いる の	
\\	この
\\	誌[し]を 毎週買[まいしゅう か]っているの。			
\\	月刊	月刊[げっかん]	げっかん	
\\	この雑誌は月刊ですか。	この 雑誌[ざっし]は 月刊[げっかん]ですか。	この ざっし は げっかん です か	
\\	この 雑誌[ざっし]は
\\	ですか。			
\\	出版	出版[しゅっぱん]	しゅっぱん	
\\	この本は15年前に出版された。	この 本[ほん]は 15年前[じゅうごねんまえ]に 出版[しゅっぱん]された。	この ほん は じゅうごねんまえ に しゅっぱん された	
\\	この 本[ほん]は 15年前[じゅうごねんまえ]に
\\	された。			
\\	出版社	出版社[しゅっぱんしゃ]	しゅっぱんしゃ	
\\	彼女は出版社で働いています。	彼女[かのじょ]は 出版社[しゅっぱんしゃ]で 働[はたら]いています。	かのじょ は しゅっぱんしゃ で はたらいて います	
\\	彼女[かのじょ]は
\\	で 働[はたら]いています。			
\\	詳細	詳細[しょうさい]	しょうさい	
\\	詳細はお気軽にお問い合わせ下さい。	詳細[しょうさい]はお 気軽[きがる]にお 問[と]い 合[あ]わせ 下[くだ]さい。	しょうさい は おきがる に おといあわせ ください	
\\	はお 気軽[きがる]にお 問[と]い 合[あ]わせ 下[くだ]さい。			
\\	心細い	心細[こころぼそ]い	こころぼそい	
\\	夜道の一人歩きは心細いね。	夜道[よみち]の 一人歩[ひとりある]きは 心細[こころぼそ]いね。	よみち の ひとりあるき は こころぼそい ね	
\\	夜道[よみち]の 一人歩[ひとりある]きは
\\	ね。			
\\	掲示	掲示[けいじ]	けいじ	
\\	大会のスローガンを掲示したよ。	大会[たいかい]のスローガンを 掲示[けいじ]したよ。	たいかい の すろーがん を けいじ した よ	
\\	大会[たいかい]のスローガンを
\\	したよ。			
\\	コンセント	コンセント	コンセント	
\\	プラグをコンセントに挿し込んだよ。	プラグをコンセントに 挿[さ]し 込[こ]んだよ。	ぷらぐ を こんせんと に さしこんだ よ	
\\	プラグを
\\	に 挿[さ]し 込[こ]んだよ。			
\\	積極的	積極的[せっきょくてき]	せっきょくてき	
\\	姉は何に対しても積極的です。	姉[あね]は 何[なに]に 対[たい]しても 積極的[せっきょくてき]です。	あね は なに に たいして も せっきょくてき です	
\\	姉[あね]は 何[なに]に 対[たい]しても
\\	です。			
\\	言い訳	言[い]い 訳[わけ]	いいわけ	
\\	あなたは言い訳が多すぎます。	あなたは 言[い]い 訳[わけ]が 多[おお]すぎます。	あなた は いいわけ が おおすぎます	
\\	あなたは
\\	が 多[おお]すぎます。			
\\	誤り	誤[あやま]り	あやまり	
\\	解答に誤りが3つ有りますよ。	解答[かいとう]に 誤[あやま]りが 3[みっ]つ 有[あ]りますよ。	かいとう に あやまり が みっつ あります よ	
\\	解答[かいとう]に
\\	が 3[みっ]つ 有[あ]りますよ。			
\\	誤る	誤[あやま]る	あやまる	
\\	彼は機械の操作を誤った。	彼[かれ]は 機械[きかい]の 操作[そうさ]を 誤[あやま]った。	かれ は きかい の そうさ を あやまった	
\\	彼[かれ]は 機械[きかい]の 操作[そうさ]を
\\	誤解	誤解[ごかい]	ごかい	
\\	私の気持ちを誤解しているんじゃない!	私[わたし]の 気持[きも]ちを 誤解[ごかい]しているんじゃない!	わたし の きもち を ごかい して いる ん じゃ ない	
\\	私[わたし]の 気持[きも]ちを
\\	しているんじゃない!			
\\	気付く	気付[きづ]く	きづく	
\\	彼はやっと問題点に気付きました。	彼[かれ]はやっと 問題点[もんだいてん]に 気付[きづ]きました。	かれ は やっと もんだいてん に きづきました	
\\	彼[かれ]はやっと 問題点[もんだいてん]に
\\	くっ付く	くっ 付[つ]く	くっつく	
\\	靴の底にガムがくっ付いてしまった。	靴[くつ]の 底[そこ]にガムがくっ 付[つ]いてしまった。	くつ の そこ に がむ が くっついて しまった	
\\	靴[くつ]の 底[そこ]にガムが
\\	ずるい	ずるい	ずるい	
\\	あなたのやり方はずるい。	あなたのやり 方[かた]はずるい。	あなた の やりかた は ずるい	
\\	あなたのやり 方[かた]は
\\	追い付く	追[お]い 付[つ]く	おいつく	
\\	駅でようやく彼に追い付きました。	駅[えき]でようやく 彼[かれ]に 追[お]い 付[つ]きました。	えき で ようやく かれ に おいつきました	
\\	駅[えき]でようやく 彼[かれ]に
\\	顔付き	顔付[かおつ]き	かおつき	
\\	彼は恐い顔付きで話したの。	彼[かれ]は 恐[こわ]い 顔付[かおつ]きで 話[はな]したの。	かれ は こわい かおつき で はなした の	
\\	彼[かれ]は 恐[こわ]い
\\	で 話[はな]したの。			
\\	思い付く	思[おも]い 付[つ]く	おもいつく	
\\	彼は名案を思い付いたの。	彼[かれ]は 名案[めいあん]を 思[おも]い 付[つ]いたの。	かれ は めいあん を おもいついた の	
\\	彼[かれ]は 名案[めいあん]を
\\	の。			
\\	後片付け	後片付[あとかたづ]け	あとかたづけ	
\\	食事の後片付けを手伝ったの。	食事[しょくじ]の 後片付[あとかたづ]けを 手伝[てつだ]ったの。	しょくじ の あとかたづけ を てつだった の	
\\	食事[しょくじ]の
\\	を 手伝[てつだ]ったの。			
\\	傷付く	傷付[きずつ]く	きずつく	
\\	私は彼女の言葉に傷付きました。	私[わたし]は 彼女[かのじょ]の 言葉[ことば]に 傷付[きずつ]きました。	わたし は かのじょ の ことば に きずつきました	
\\	私[わたし]は 彼女[かのじょ]の 言葉[ことば]に
\\	傷付ける	傷付[きずつ]ける	きずつける	
\\	あなたを傷付けるつもりはありませんでした。	あなたを 傷付[きずつ]けるつもりはありませんでした。	あなた を きずつける つもり は ありませんでした	
\\	あなたを
\\	つもりはありませんでした。			
\\	くっ付ける	くっ 付[つ]ける	くっつける	
\\	彼は壁に耳をくっ付けて隣の話を聞いていたの。	彼[かれ]は 壁[かべ]に 耳[みみ]をくっ 付[つ]けて 隣[となり]の 話[はなし]を 聞[き]いていたの。	かれ は かべ に みみ を くっつけて となり の はなし を きいて いた の	
\\	彼[かれ]は 壁[かべ]に 耳[みみ]を
\\	隣[となり]の 話[はなし]を 聞[き]いていたの。			
\\	せっかち	せっかち	せっかち	
\\	彼はせっかちで困ります。	彼[かれ]はせっかちで 困[こま]ります。	かれ は せっかち で こまります	
\\	彼[かれ]は
\\	で 困[こま]ります。			
\\	言付ける	言付[ことづ]ける	ことづける	
\\	彼女への伝言を言付けたの。	彼女[かのじょ]への 伝言[でんごん]を 言付[ことづ]けたの。	かのじょ へ の でんごん を ことづけた の	
\\	彼女[かのじょ]への 伝言[でんごん]を
\\	の。			
\\	金属	金属[きんぞく]	きんぞく	
\\	アルミニウムは金属の一種です。	アルミニウムは 金属[きんぞく]の 一種[いっしゅ]です。	あるみにうむ は きんぞく の いっしゅ です	
\\	アルミニウムは
\\	の 一種[いっしゅ]です。			
\\	所属	所属[しょぞく]	しょぞく	
\\	学校では音楽部に所属していました。	学校[がっこう]では 音楽部[おんがくぶ]に 所属[しょぞく]していました。	がっこう で は おんがくぶ に しょぞく して いました	
\\	学校[がっこう]では 音楽部[おんがくぶ]に
\\	していました。			
\\	大蔵省	大蔵省[おおくらしょう]	おおくらしょう	
\\	彼は大蔵省に勤務しているんだよ。	彼[かれ]は 大蔵省[おおくらしょう]に 勤務[きんむ]しているんだよ。	かれ は おおくらしょう に きんむ して いる ん だ よ	
\\	彼[かれ]は
\\	に 勤務[きんむ]しているんだよ。			
\\	外務省	外務省[がいむしょう]	がいむしょう	
\\	ビザについて外務省に問い合わせた。	ビザについて 外務省[がいむしょう]に 問[と]い 合[あ]わせた。	びざ に ついて がいむしょう に といあわせた	
\\	ビザについて
\\	に 問[と]い 合[あ]わせた。			
\\	帰省	帰省[きせい]	きせい	
\\	来週、帰省します。	来週[らいしゅう]、 帰省[きせい]します。	らいしゅう きせい します	
\\	来週[らいしゅう]、
\\	します。			
\\	省略	省略[しょうりゃく]	しょうりゃく	
\\	地図内の小さい建物は省略してあります。	地図内[ちずない]の 小[ちい]さい 建物[たてもの]は 省略[しょうりゃく]してあります。	ちずない の ちいさい たてもの は しょうりゃく して あります	
\\	地図内[ちずない]の 小[ちい]さい 建物[たてもの]は
\\	してあります。			
\\	概念	概念[がいねん]	がいねん	
\\	インターネットは情報の概念を変えたよね。	インターネットは 情報[じょうほう]の 概念[がいねん]を 変[か]えたよね。	いんたーねっと は じょうほう の がいねん を かえた よ ね	
\\	インターネットは 情報[じょうほう]の
\\	を 変[か]えたよね。			
\\	うちわ	うちわ	うちわ	
\\	父はうちわで扇いでいるの。	父[ちち]はうちわで 扇[あお]いでいるの。	ちち は うちわ で あおいで いる の	
\\	父[ちち]は
\\	で 扇[あお]いでいるの。			
\\	記念	記念[きねん]	きねん	
\\	卒業の記念にみんなで旅行したよ。	卒業[そつぎょう]の 記念[きねん]にみんなで 旅行[りょこう]したよ。	そつぎょう の きねん に みんな で りょこう した よ	
\\	卒業[そつぎょう]の
\\	にみんなで 旅行[りょこう]したよ。			
\\	信念	信念[しんねん]	しんねん	
\\	自分の信念に従いたいと思います。	自分[じぶん]の 信念[しんねん]に 従[したが]いたいと 思[おも]います。	じぶん の しんねん に したがいたい と おもいます	
\\	自分[じぶん]の
\\	に 従[したが]いたいと 思[おも]います。			
\\	順調	順調[じゅんちょう]	じゅんちょう	
\\	進み具合は全て順調です。	進[すす]み 具合[ぐあい]は 全[すべ]て 順調[じゅんちょう]です。	すすみ ぐあい は すべて じゅんちょう です	
\\	進[すす]み 具合[ぐあい]は 全[すべ]て
\\	です。			
\\	順	順[じゅん]	じゅん	
\\	あいうえお順に並べて下さい。	あいうえお 順[じゅん]に 並[なら]べて 下[くだ]さい。	あいうえお じゅん に ならべて ください	
\\	あいうえお
\\	に 並[なら]べて 下[くだ]さい。			
\\	順番	順番[じゅんばん]	じゅんばん	
\\	これを順番通りに並べ替えて下さい。	これを 順番[じゅんばん] 通[どお]りに 並[なら]べ 替[か]えて 下[くだ]さい。	これ を じゅんばん どおり に ならべ かえて ください	
\\	これを
\\	通[どお]りに 並[なら]べ 替[か]えて 下[くだ]さい。			
\\	順位	順位[じゅんい]	じゅんい	
\\	今年は去年よりも順位が上がりました。	今年[ことし]は 去年[きょねん]よりも 順位[じゅんい]が 上[あ]がりました。	ことし は きょねん より も じゅんい が あがりました	
\\	今年[ことし]は 去年[きょねん]よりも
\\	が 上[あ]がりました。			
\\	順々に	順々[じゅんじゅん]に	じゅんじゅんに	
\\	子供たちは順々にお菓子を受け取ったよ。	子供[こども]たちは 順々[じゅんじゅん]にお 菓子[かし]を 受[う]け 取[と]ったよ。	こどもたち は じゅんじゅんに おかし を うけとった よ	
\\	子供[こども]たちは
\\	お 菓子[かし]を 受[う]け 取[と]ったよ。			
\\	うっかり	うっかり	うっかり	
\\	大事なメールをうっかり消してしまった。	大事[だいじ]なメールをうっかり 消[け]してしまった。	だいじ な めーる を うっかり けして しまった	
\\	大事[だいじ]なメールを
\\	消[け]してしまった。			
\\	順序	順序[じゅんじょ]	じゅんじょ	
\\	正しい順序で操作してください。	正[ただ]しい 順序[じゅんじょ]で 操作[そうさ]してください。	ただしい じゅんじょ で そうさ して ください	
\\	正[ただ]しい
\\	で 操作[そうさ]してください。			
\\	逆	逆[ぎゃく]	ぎゃく	
\\	駅は逆方向です。	駅[えき]は 逆[ぎゃく] 方向[ほうこう]です。	えき は ぎゃくほうこう です	
\\	駅[えき]は
\\	方向[ほうこう]です。			
\\	逆らう	逆[さか]らう	さからう	
\\	親に逆らうのは良くない事です。	親[おや]に 逆[さか]らうのは 良[よ]くない 事[こと]です。	おや に さからう の は よくない こと です	
\\	親[おや]に
\\	のは 良[よ]くない 事[こと]です。			
\\	逆さ	逆[さか]さ	さかさ	
\\	絵を逆さにしてみてごらん。	絵[え]を 逆[さか]さにしてみてごらん。	え を さかさ に して みて ごらん	
\\	絵[え]を
\\	にしてみてごらん。			
\\	逆さま	逆[さか]さま	さかさま	
\\	それじゃあ上と下が逆さまだよ。	それじゃあ 上[うえ]と 下[した]が 逆[さか]さまだよ。	それじゃあ うえ と した が さかさま だ よ	
\\	それじゃあ 上[うえ]と 下[した]が
\\	だよ。			
\\	行列	行列[ぎょうれつ]	ぎょうれつ	
\\	店の前に長い行列ができていますよ。	店[みせ]の 前[まえ]に 長[なが]い 行列[ぎょうれつ]ができていますよ。	みせ の まえ に ながい ぎょうれつ が できて います よ	
\\	店[みせ]の 前[まえ]に 長[なが]い
\\	ができていますよ。			
\\	整列	整列[せいれつ]	せいれつ	
\\	体育館に行って整列しなさい。	体育館[たいいくかん]に 行[い]って 整列[せいれつ]しなさい。	たいいくかん に いって せいれつ しなさい	
\\	体育館[たいいくかん]に 行[い]って
\\	しなさい。			
\\	くだらない	くだらない	くだらない	
\\	くだらないおしゃべりはやめなさい。	くだらないおしゃべりはやめなさい。	くだらないおしゃべりはやめなさい。	
\\	おしゃべりはやめなさい。			
\\	実例	実例[じつれい]	じつれい	
\\	実例を使って説明してください。	実例[じつれい]を 使[つか]って 説明[せつめい]してください。	じつれい を つかって せつめい して ください	
\\	を 使[つか]って 説明[せつめい]してください。			
\\	余り	余[あま]り	あまり	
\\	余りは次回使いましょう。	余[あま]りは 次回使[じかい つか]いましょう。	あまり は じかい つかいましょう	
\\	は 次回使[じかい つか]いましょう。			
\\	削減	削減[さくげん]	さくげん	
\\	福祉予算は削減されないことに決まったのよ。	福祉予算[ふくしよさん]は 削減[さくげん]されないことに 決[き]まったのよ。	ふくしよさん は さくげん されない こと に きまった の よ	
\\	福祉予算[ふくしよさん]は
\\	されないことに 決[き]まったのよ。			
\\	削除	削除[さくじょ]	さくじょ	
\\	要らないファイルは削除して下さい。	要[い]らないファイルは 削除[さくじょ]して 下[くだ]さい。	いらない ふぁいる は さくじょ して ください	
\\	要[い]らないファイルは
\\	して 下[くだ]さい。			
\\	既に	既[すで]に	すでに	
\\	そのことは既にみんな知っています。	そのことは 既[すで]にみんな 知[し]っています。	その こと は すでに みんな しって います	
\\	そのことは
\\	みんな 知[し]っています。			
\\	既製	既製[きせい]	きせい	
\\	彼の体型じゃ既製のサイズに合わないよ。	彼[かれ]の 体型[たいけい]じゃ 既製[きせい]のサイズに 合[あ]わないよ。	かれ の たいけい じゃ きせい の さいず に あわない よ	
\\	彼[かれ]の 体型[たいけい]じゃ
\\	のサイズに 合[あ]わないよ。			
\\	深刻	深刻[しんこく]	しんこく	
\\	これは深刻な問題です。	これは 深刻[しんこく]な 問題[もんだい]です。	これ は しんこく な もんだい で す	
\\	これは
\\	な 問題[もんだい]です。			
\\	刻む	刻[きざ]む	きざむ	
\\	玉ねぎを細かく刻んでください。	玉[たま]ねぎを 細[こま]かく 刻[きざ]んでください。	たまねぎ を こまかく きざんで ください	
\\	玉[たま]ねぎを 細[こま]かく
\\	ください。			
\\	ずうずうしい	ずうずうしい	ずうずうしい	
\\	何てずうずうしい人なんだろう。	何[なん]てずうずうしい 人[ひと]なんだろう。	なんて ずうずうしい ひと なん だろう	
\\	何[なん]て
\\	人[ひと]なんだろう。			
\\	時刻	時刻[じこく]	じこく	
\\	ただ今の時刻は6時35分です。	ただ 今[いま]の 時刻[じこく]は 6時35分[ろくじ さんじゅうごふん]です。	ただいま の じこく は ろくじ さんじゅうごふん です	
\\	ただ 今[いま]の
\\	は 6時35分[ろくじ さんじゅうごふん]です。			
\\	締め切り	締[し]め 切[き]り	しめきり	
\\	申し込みの締め切りはいつですか。	申[もう]し 込[こ]みの 締[し]め 切[き]りはいつですか。	もうしこみ の しめきり は いつ です か	
\\	申[もう]し 込[こ]みの
\\	はいつですか。			
\\	締め切る	締[し]め 切[き]る	しめきる	
\\	応募受付は締め切りました。	応募受付[おうぼ うけつけ]は 締[し]め 切[き]りました。	おうぼ うけつけ は しめきりました	
\\	応募受付[おうぼ うけつけ]は
\\	締まる	締[し]まる	しまる	
\\	レバーを右に回すと締まります。	レバーを 右[みぎ]に 回[まわ]すと 締[し]まります。	ればー を みぎ に まわす と しまります	
\\	レバーを 右[みぎ]に 回[まわ]すと
\\	栄える	栄[さか]える	さかえる	
\\	ここはかつてゴールドラッシュで栄えた町だよ。	ここはかつてゴールドラッシュで 栄[さか]えた 町[まち]だよ。	ここ は かつて ごーるど らっしゅ で さかえた まち だ よ	
\\	ここはかつてゴールドラッシュで
\\	町[まち]だよ。			
\\	栄養	栄養[えいよう]	えいよう	
\\	豆腐は栄養のある食べ物です。	豆腐[とうふ]は 栄養[えいよう]のある 食[た]べ 物[もの]です。	とうふ は えいよう の ある たべもの です	
\\	豆腐[とうふ]は
\\	のある 食[た]べ 物[もの]です。			
\\	教養	教養[きょうよう]	きょうよう	
\\	彼女はとても教養のある人ですね。	彼女[かのじょ]はとても 教養[きょうよう]のある 人[ひと]ですね。	かのじょ は とても きょうよう の ある ひと です ね	
\\	彼女[かのじょ]はとても
\\	のある 人[ひと]ですね。			
\\	すやすや	すやすや	すやすや	
\\	子供がすやすや眠っているね。	子供[こども]がすやすや 眠[ねむ]っているね。	こども が すやすや ねむって いる ね	
\\	子供[こども]が
\\	眠[ねむ]っているね。			
\\	休養	休養[きゅうよう]	きゅうよう	
\\	彼女は今、休養中です。	彼女[かのじょ]は 今[いま]、 休養[きゅうよう] 中[ちゅう]です。	かのじょ は いま きゅうようちゅう です	
\\	彼女[かのじょ]は 今[いま]、
\\	中[ちゅう]です。			
\\	苦しむ	苦[くる]しむ	くるしむ	
\\	彼女はアレルギーに苦しんでいます。	彼女[かのじょ]はアレルギーに 苦[くる]しんでいます。	かのじょ は あれるぎー に くるしんで います	
\\	彼女[かのじょ]はアレルギーに
\\	苦情	苦情[くじょう]	くじょう	
\\	店に苦情の電話をかけました。	店[みせ]に 苦情[くじょう]の 電話[でんわ]をかけました。	みせ に くじょう の でんわ を かけました	
\\	店[みせ]に
\\	の 電話[でんわ]をかけました。			
\\	苦心	苦心[くしん]	くしん	
\\	苦心して絵を描き上げました。	苦心[くしん]して 絵[え]を 描[か]き 上[あ]げました。	くしん して え を かきあげました	
\\	して 絵[え]を 描[か]き 上[あ]げました。			
\\	苦しみ	苦[くる]しみ	くるしみ	
\\	誰も彼の苦しみを理解していなかったな。	誰[だれ]も 彼[かれ]の 苦[くる]しみを 理解[りかい]していなかったな。	だれ も かれ の くるしみ を りかい して いなかった な	
\\	誰[だれ]も 彼[かれ]の
\\	を 理解[りかい]していなかったな。			
\\	苦痛	苦痛[くつう]	くつう	
\\	私は人前で話すことが苦痛です。	私[わたし]は 人前[ひとまえ]で 話[はな]すことが 苦痛[くつう]です。	わたし は ひとまえ で はなす こと が くつう です	
\\	私[わたし]は 人前[ひとまえ]で 話[はな]すことが
\\	です。			
\\	苦しめる	苦[くる]しめる	くるしめる	
\\	強い日差しが選手たちを苦しめたの。	強[つよ]い 日差[ひざ]しが 選手[せんしゅ]たちを 苦[くる]しめたの。	つよい ひざし が せんしゅたち を くるしめた の	
\\	強[つよ]い 日差[ひざ]しが 選手[せんしゅ]たちを
\\	の。			
\\	セルフサービス	セルフサービス	セルフサービス	
\\	この食堂はセルフサービスです。	この 食堂[しょくどう]はセルフサービスです。	この しょくどう は せるふさーびす です	
\\	この 食堂[しょくどう]は
\\	です。			
\\	苦労	苦労[くろう]	くろう	
\\	母は苦労して私たちを育てたの。	母[はは]は 苦労[くろう]して 私[わたし]たちを 育[そだ]てたの。	はは は くろう して わたしたち を そだてた の	
\\	母[はは]は
\\	して 私[わたし]たちを 育[そだ]てたの。			
\\	勤労	勤労[きんろう]	きんろう	
\\	父は長い勤労生活を送ったんです。	父[ちち]は 長[なが]い 勤労[きんろう] 生活[せいかつ]を 送[おく]ったんです。	ちち は ながい きんろう せいかつ を おくった ん です	
\\	父[ちち]は 長[なが]い
\\	生活[せいかつ]を 送[おく]ったんです。			
\\	過労	過労[かろう]	かろう	
\\	彼は過労のために倒れたの。	彼[かれ]は 過労[かろう]のために 倒[たお]れたの。	かれ は かろう の ため に たおれた の	
\\	彼[かれ]は
\\	のために 倒[たお]れたの。			
\\	困難	困難[こんなん]	こんなん	
\\	困難にあってもあきらめてはいけないよ。	困難[こんなん]にあってもあきらめてはいけないよ。	こんなん に あって も あきらめて は いけない よ	
\\	にあってもあきらめてはいけないよ。			
\\	甘える	甘[あま]える	あまえる	
\\	彼女の親切に甘えました。	彼女[かのじょ]の 親切[しんせつ]に 甘[あま]えました。	かのじょ の しんせつ に あまえました	
\\	彼女[かのじょ]の 親切[しんせつ]に
\\	甘やかす	甘[あま]やかす	あまやかす	
\\	子供を甘やかしてはいけないの。	子供[こども]を 甘[あま]やかしてはいけないの。	こども を あまやかしては いけない の	
\\	子供[こども]を
\\	はいけないの。			
\\	辛い	辛[から]い	からい	
\\	部長は僕の仕事に辛い評価を出したんだ。	部長[ぶちょう]は 僕[ぼく]の 仕事[しごと]に 辛[から]い 評価[ひょうか]を 出[だ]したんだ。	ぶちょう は ぼく の しごと に からい ひょうか を だした ん だ	
\\	部長[ぶちょう]は 僕[ぼく]の 仕事[しごと]に
\\	評価[ひょうか]を 出[だ]したんだ。			
\\	アルファベット	アルファベット	アルファベット	
\\	この子はアルファベットを全部言えます。	この 子[こ]はアルファベットを 全部言[ぜんぶ い]えます。	この こ は あるふぁべっと を ぜんぶ いえます	
\\	この 子[こ]は
\\	を 全部言[ぜんぶ い]えます。			
\\	幸い	幸[さいわ]い	さいわい	
\\	幸い、電車に嵐の影響はなかった。	幸[さいわ]い、 電車[でんしゃ]に 嵐[あらし]の 影響[えいきょう]はなかった。	さいわい でんしゃ に あらし の えいきょう は なかった	
\\	、 電車[でんしゃ]に 嵐[あらし]の 影響[えいきょう]はなかった。			
\\	幸運	幸運[こううん]	こううん	
\\	幸運にもチケットを手に入れました。	幸運[こううん]にもチケットを 手[て]に 入[い]れました。	こううん に も ちけっと を て に いれました	
\\	にもチケットを 手[て]に 入[い]れました。			
\\	幸福	幸福[こうふく]	こうふく	
\\	彼女は幸福な日々を過ごしているわ。	彼女[かのじょ]は 幸福[こうふく]な 日々[ひび]を 過[す]ごしているわ。	かのじょ は こうふく な ひび を すごして いる わ	
\\	彼女[かのじょ]は
\\	な 日々[ひび]を 過[す]ごしているわ。			
\\	砂	砂[すな]	すな	
\\	靴に砂が入ってしまった。	靴[くつ]に 砂[すな]が 入[はい]ってしまった。	くつ に すな が はいって しまった	
\\	靴[くつ]に
\\	が 入[はい]ってしまった。			
\\	砂漠	砂漠[さばく]	さばく	
\\	砂漠ではほとんど雨が降らないのよ。	砂漠[さばく]ではほとんど 雨[あめ]が 降[ふ]らないのよ。	さばく で は ほとんど あめ が ふらない の よ	
\\	ではほとんど 雨[あめ]が 降[ふ]らないのよ。			
\\	漁業	漁業[ぎょぎょう]	ぎょぎょう	
\\	彼らは漁業を営んでいるんだ。	彼[かれ]らは 漁業[ぎょぎょう]を 営[いとな]んでいるんだ。	かれら は ぎょぎょう を いとなん でいる ん だ 。	
\\	彼[かれ]らは
\\	を 営[いとな]んでいるんだ。			
\\	薄暗い	薄暗[うすぐら]い	うすぐらい	
\\	外はもう薄暗くなりましたよ。	外[そと]はもう 薄暗[うすぐら]くなりましたよ。	そと は もう うすぐらく なりました よ	
\\	外[そと]はもう
\\	よ。			
\\	薄める	薄[うす]める	うすめる	
\\	スープを少し薄めましょうか。	スープを 少[すこ]し 薄[うす]めましょうか。	すーぷ を すこし うすめましょう か	
\\	スープを 少[すこ]し
\\	か。			
\\	いびき	いびき	いびき	
\\	父のいびきは大きいんだ。	父[ちち]のいびきは 大[おお]きいんだ。	ちち の いびき は おおきい ん だ	
\\	父[ちち]の
\\	は 大[おお]きいんだ。			
\\	薄着	薄着[うすぎ]	うすぎ	
\\	子供たちは冬でも薄着で通します。	子供[こども]たちは 冬[ふゆ]でも 薄着[うすぎ]で 通[とお]します。	こどもたち は ふゆ で も うすぎ で とおします	
\\	子供[こども]たちは 冬[ふゆ]でも
\\	で 通[とお]します。			
\\	厚着	厚着[あつぎ]	あつぎ	
\\	普段から厚着しないようにしています。	普段[ふだん]から 厚着[あつぎ]しないようにしています。	ふだん から あつぎ しない よう に して います	
\\	普段[ふだん]から
\\	しないようにしています。			
\\	圧力	圧力[あつりょく]	あつりょく	
\\	相手会社から強い圧力が掛かった。	相手会社[あいてがいしゃ]から 強[つよ]い 圧力[あつりょく]が 掛[か]かった。	あいてがいしゃ から つよい あつりょく が かかった	
\\	相手会社[あいてがいしゃ]から 強[つよ]い
\\	が 掛[か]かった。			
\\	気圧	気圧[きあつ]	きあつ	
\\	高い山は気圧が低いね。	高[たか]い 山[やま]は 気圧[きあつ]が 低[ひく]いね。	たかい やま は きあつ が ひくい ね	
\\	高[たか]い 山[やま]は
\\	が 低[ひく]いね。			
\\	高気圧	高気圧[こうきあつ]	こうきあつ	
\\	この暑さは高気圧のせいです。	この 暑[あつ]さは 高気圧[こうきあつ]のせいです。	この あつさ は こうきあつ の せい です	
\\	この 暑[あつ]さは
\\	のせいです。			
\\	縮小	縮小[しゅくしょう]	しゅくしょう	
\\	これを縮小して印刷してください。	これを 縮小[しゅくしょう]して 印刷[いんさつ]してください。	これ を しゅくしょう して いんさつ して ください	
\\	これを
\\	して 印刷[いんさつ]してください。			
\\	越す	越[こ]す	こす	
\\	暑さは峠を越したね。	暑[あつ]さは 峠[とうげ]を 越[こ]したね。	あつさ は とうげ を こした ね	
\\	暑[あつ]さは 峠[とうげ]を
\\	ね。			
\\	いやらしい	いやらしい	いやらしい	
\\	彼はいやらしい男ね。	彼[かれ]はいやらしい 男[おとこ]ね。	かれ は いやらしい おとこ ね	
\\	彼[かれ]は
\\	男[おとこ]ね。			
\\	追い抜く	追[お]い 抜[ぬ]く	おいぬく	
\\	リレーで彼は2人追い抜いたね。	リレーで 彼[かれ]は 2人[ふたり] 追[お]い 抜[ぬ]いたね。	りれー で かれ は ふたり おいぬいた ね	
\\	リレーで 彼[かれ]は 2人[ふたり]
\\	ね。			
\\	札	札[さつ]	さつ	
\\	彼はカバンから札の束を取り出したんだ。	彼[かれ]はカバンから 札[さつ]の 束[たば]を 取[と]り 出[だ]したんだ。	かれ は かばん から さつ の たば を とりだした ん だ 。	
\\	彼[かれ]はカバンから
\\	の 束[たば]を 取[と]り 出[だ]したんだ。			
\\	改札	改札[かいさつ]	かいさつ	
\\	改札で3時に会おう。	改札[かいさつ]で 3時[さんじ]に 会[あ]おう。	かいさつ で さんじ に あおう	
\\	で 3時[さんじ]に 会[あ]おう。			
\\	感謝	感謝[かんしゃ]	かんしゃ	
\\	家族に感謝しています。	家族[かぞく]に 感謝[かんしゃ]しています。	かぞく に かんしゃ して います	
\\	家族[かぞく]に
\\	しています。			
\\	月謝	月謝[げっしゃ]	げっしゃ	
\\	先生に月謝を渡しましたか。	先生[せんせい]に 月謝[げっしゃ]を 渡[わた]しましたか。	せんせい に げっしゃ を わたしました か	
\\	先生[せんせい]に
\\	を 渡[わた]しましたか。			
\\	射す	射[さ]す	さす	
\\	今日は久しぶりに日が射してるね。	今日[きょう]は 久[ひさ]しぶりに 日[ひ]が 射[さ]してるね。	きょう は ひさしぶり に ひ が さして る ね	
\\	今日[きょう]は 久[ひさ]しぶりに 日[ひ]が
\\	ね。			
\\	過程	過程[かてい]	かてい	
\\	プログラムの過程に問題があるの。	プログラムの 過程[かてい]に 問題[もんだい]があるの。	ぷろぐらむ の かてい に もんだい が ある の	
\\	プログラムの
\\	に 問題[もんだい]があるの。			
\\	イントネーション	イントネーション	イントネーション	
\\	イントネーションの違いに気を付けて。	イントネーションの 違[ちが]いに 気[き]を 付[つ]けて。	いんとねーしょん の ちがい に き を つけて	
\\	の 違[ちが]いに 気[き]を 付[つ]けて。			
\\	課程	課程[かてい]	かてい	
\\	一年生の課程を修了したの。	一年生[いちねんせい]の 課程[かてい]を 修了[しゅうりょう]したの。	いちねんせい の かてい を しゅうりょう した の	
\\	一年生[いちねんせい]の
\\	を 修了[しゅうりょう]したの。			
\\	行程	行程[こうてい]	こうてい	
\\	全部で6時間の行程です。	全部[ぜんぶ]で 6時間[ろくじかん]の 行程[こうてい]です。	ぜんぶ で ろくじかん の こうてい です	
\\	全部[ぜんぶ]で 6時間[ろくじかん]の
\\	です。			
\\	先程	先程[さきほど]	さきほど	
\\	その件でしたら、先程上司から許可を得ております。	その 件[けん]でしたら、 先程[さきほど] 上司[じょうし]から 許可[きょか]を 得[え]ております。	その けん でしたら さきほど じょうし から きょか を えて おります	
\\	その 件[けん]でしたら、
\\	上司[じょうし]から 許可[きょか]を 得[え]ております。			
\\	優れる	優[すぐ]れる	すぐれる	
\\	彼は非常に優れた選手です。	彼[かれ]は 非常[ひじょう]に 優[すぐ]れた 選手[せんしゅ]です。	かれ は ひじょうに すぐれた せんしゅ です	
\\	彼[かれ]は 非常[ひじょう]に
\\	選手[せんしゅ]です。			
\\	女優	女優[じょゆう]	じょゆう	
\\	彼女はずっと女優になるのが夢でした。	彼女[かのじょ]はずっと 女優[じょゆう]になるのが 夢[ゆめ]でした。	かのじょ は ずっと じょゆう に なる の が ゆめ でした	
\\	彼女[かのじょ]はずっと
\\	になるのが 夢[ゆめ]でした。			
\\	透き通る	透[す]き 通[とお]る	すきとおる	
\\	湖の水は透き通っていたよ。	湖[みずうみ]の 水[みず]は 透[す]き 通[とお]っていたよ。	みずうみ の みず は すきとおって いた よ	
\\	湖[みずうみ]の 水[みず]は
\\	よ。			
\\	指導	指導[しどう]	しどう	
\\	彼は生徒の指導が上手ね。	彼[かれ]は 生徒[せいと]の 指導[しどう]が 上手[じょうず]ね。	かれ は せいと の しどう が じょうず ね	
\\	彼[かれ]は 生徒[せいと]の
\\	が 上手[じょうず]ね。			
\\	希望	希望[きぼう]	きぼう	
\\	彼は本社で働くことを希望しています。	彼[かれ]は 本社[ほんしゃ]で 働[はたら]くことを 希望[きぼう]しています。	かれ は ほんしゃ で はたらく こと を きぼう して います	
\\	彼[かれ]は 本社[ほんしゃ]で 働[はたら]くことを
\\	しています。			
\\	がたがた	がたがた	がたがた	
\\	この椅子はがたがたしているね。	この 椅子[いす]はがたがたしているね。	この いす は がたがた して いる ね	
\\	この 椅子[いす]は
\\	しているね。			
\\	失望	失望[しつぼう]	しつぼう	
\\	彼女は結婚生活に失望していたの。	彼女[かのじょ]は 結婚生活[けっこん せいかつ]に 失望[しつぼう]していたの。	かのじょ は けっこん せいかつ に しつぼう して いた の	
\\	彼女[かのじょ]は 結婚生活[けっこん せいかつ]に
\\	していたの。			
\\	意志	意志[いし]	いし	
\\	彼は意志の強い人です。	彼[かれ]は 意志[いし]の 強[つよ]い 人[ひと]です。	かれ は いし の つよい ひと です	
\\	彼[かれ]は
\\	の 強[つよ]い 人[ひと]です。			
\\	志す	志[こころざ]す	こころざす	
\\	私は医者を志しています。	私[わたし]は 医者[いしゃ]を 志[こころざ]しています。	わたし は いしゃ を こころざして います	
\\	私[わたし]は 医者[いしゃ]を
\\	怒り	怒[いか]り	いかり	
\\	彼ったら怒り爆発だったよ。	彼[かれ]ったら 怒[いか]り 爆発[ばくはつ]だったよ。	かれ ったら いかり ばくはつ だった よ	
\\	彼[かれ]ったら
\\	爆発[ばくはつ]だったよ。			
\\	自身	自身[じしん]	じしん	
\\	自分自身を疑ってはいけません。	自分[じぶん] 自身[じしん]を 疑[うたが]ってはいけません。	じぶん じしん を うたがって は いけません	
\\	自分[じぶん]
\\	を 疑[うたが]ってはいけません。			
\\	出身	出身[しゅっしん]	しゅっしん	
\\	家内は九州出身です。	家内[かない]は 九州[きゅうしゅう] 出身[しゅっしん]です。	かない は きゅうしゅう しゅっしん です	
\\	家内[かない]は 九州[きゅうしゅう]
\\	です。			
\\	全身	全身[ぜんしん]	ぜんしん	
\\	運動した翌日は全身の筋肉が痛い。	運動[うんどう]した 翌日[よくじつ]は 全身[ぜんしん]の 筋肉[きんにく]が 痛[いた]い。	うんどう した よくじつ は ぜんしん の きんにく が いたい	
\\	運動[うんどう]した 翌日[よくじつ]は
\\	の 筋肉[きんにく]が 痛[いた]い。			
\\	カンニング	カンニング	カンニング	
\\	カンニングをした人は0点です。	カンニングをした 人[ひと]は 0点[れいてん]です。	かんにんぐ を した ひと は れいてん です	
\\	をした 人[ひと]は 0点[れいてん]です。			
\\	身長	身長[しんちょう]	しんちょう	
\\	身長はどれくらいありますか。	身長[しんちょう]はどれくらいありますか。	しんちょう は どれ くらい あります か	
\\	はどれくらいありますか。			
\\	心身	心身[しんしん]	しんしん	
\\	私は心身共に疲れていました。	私[わたし]は 心身[しんしん] 共[とも]に 疲[つか]れていました。	わたし は しんしん ともに つかれて いました	
\\	私[わたし]は
\\	共[とも]に 疲[つか]れていました。			
\\	受け身	受[う]け 身[み]	うけみ	
\\	彼はいつも受け身の姿勢で、自分からは何もしないんだ。	彼[かれ]はいつも 受[う]け 身[み]の 姿勢[しせい]で、 自分[じぶん]からは 何[なに]もしないんだ。	かれ は いつも うけみ の しせい で じぶん からは なに も しない ん だ	
\\	彼[かれ]はいつも
\\	の 姿勢[しせい]で、 自分[じぶん]からは 何[なに]もしないんだ。			
\\	証拠	証拠[しょうこ]	しょうこ	
\\	証拠を見つけるまで逮捕はできません。	証拠[しょうこ]を 見[み]つけるまで 逮捕[たいほ]はできません。	しょうこ を みつける まで たいほ は できません	
\\	を 見[み]つけるまで 逮捕[たいほ]はできません。			
\\	照明	照明[しょうめい]	しょうめい	
\\	もっと照明を明るくしてください。	もっと 照明[しょうめい]を 明[あか]るくしてください。	もっと しょうめい を あかるく して ください	
\\	もっと
\\	を 明[あか]るくしてください。			
\\	工夫	工夫[くふう]	くふう	
\\	いろいろ工夫して仕事をやりとげたさ。	いろいろ 工夫[くふう]して 仕事[しごと]をやりとげたさ。	いろいろ くふう して しごと を やりとげた さ	
\\	いろいろ
\\	して 仕事[しごと]をやりとげたさ。			
\\	主婦	主婦[しゅふ]	しゅふ	
\\	主婦の主な仕事は家事よ。	主婦[しゅふ]の 主[おも]な 仕事[しごと]は 家事[かじ]よ。	しゅふ の おも な しごと は かじ よ	
\\	の 主[おも]な 仕事[しごと]は 家事[かじ]よ。			
\\	あぐら	あぐら	あぐら	
\\	彼はあぐらをかいてしゃべっていますね。	彼[かれ]はあぐらをかいてしゃべっていますね。	かれ は あぐら を かいて しゃべって います ね	
\\	彼[かれ]は
\\	をかいてしゃべっていますね。			
\\	産婦人科	産婦人科[さんふじんか]	さんふじんか	
\\	最近、産婦人科の不足が問題になっているね。	最近[さいきん]、 産婦人科[さんふじんか]の 不足[ふそく]が 問題[もんだい]になっているね。	さいきん さんふじんか の ふそく が もんだい に なって いる ね	
\\	最近[さいきん]、
\\	の 不足[ふそく]が 問題[もんだい]になっているね。			
\\	奥様	奥様[おくさま]	おくさま	
\\	社長の奥様はきれいな方です。	社長[しゃちょう]の 奥様[おくさま]はきれいな 方[かた]です。	しゃちょう の おくさま は きれい な かた です	
\\	社長[しゃちょう]の
\\	はきれいな 方[かた]です。			
\\	お互い	お 互[たが]い	おたがい	
\\	お互いを信頼することが大事です。	お 互[たが]いを 信頼[しんらい]することが 大事[だいじ]です。	おたがい を しんらい する こと が だいじ です	
\\	を 信頼[しんらい]することが 大事[だいじ]です。			
\\	交互	交互[こうご]	こうご	
\\	男女交互に並んでください。	男女[だんじょ] 交互[こうご]に 並[なら]んでください。	だんじょ こうご に ならんで ください	
\\	男女[だんじょ]
\\	に 並[なら]んでください。			
\\	高齢	高齢[こうれい]	こうれい	
\\	彼は高齢を理由に社長を辞めたよ。	彼[かれ]は 高齢[こうれい]を 理由[りゆう]に 社長[しゃちょう]を 辞[や]めたよ。	かれ は こうれい を りゆう に しゃちょう を やめた よ	
\\	彼[かれ]は
\\	を 理由[りゆう]に 社長[しゃちょう]を 辞[や]めたよ。			
\\	愛情	愛情[あいじょう]	あいじょう	
\\	子供はたくさんの愛情が必要です。	子供[こども]はたくさんの 愛情[あいじょう]が 必要[ひつよう]です。	こども は たくさん の あいじょう が ひつよう です	
\\	子供[こども]はたくさんの
\\	が 必要[ひつよう]です。			
\\	可愛らしい	可愛[かわい]らしい	かわいらしい	
\\	彼女は娘に可愛らしい服を作りましたね。	彼女[かのじょ]は 娘[むすめ]に 可愛[かわい]らしい 服[ふく]を 作[つく]りましたね。	かのじょ は むすめ に かわいらしい ふく を つくりました ね	
\\	彼女[かのじょ]は 娘[むすめ]に
\\	服[ふく]を 作[つく]りましたね。			
\\	可愛がる	可愛[かわい]がる	かわいがる	
\\	彼女は猫を可愛がっています。	彼女[かのじょ]は 猫[ねこ]を 可愛[かわい]がっています。	かのじょ は ねこ を かわいがって います	
\\	彼女[かのじょ]は 猫[ねこ]を
\\	いやいや	いやいや	いやいや	
\\	その子はいやいや学校へ行ったの。	その 子[こ]はいやいや 学校[がっこう]へ 行[い]ったの。	その こ は いやいや がっこう へ いった の	
\\	その 子[こ]は
\\	学校[がっこう]へ 行[い]ったの。			
\\	恋	恋[こい]	こい	
\\	彼女は恋をしてきれいになったね。	彼女[かのじょ]は 恋[こい]をしてきれいになったね。	かのじょ は こい を して きれい に なった ね	
\\	彼女[かのじょ]は
\\	をしてきれいになったね。			
\\	失恋	失恋[しつれん]	しつれん	
\\	彼は最近、失恋したらしいの。	彼[かれ]は 最近[さいきん]、 失恋[しつれん]したらしいの。	かれ は さいきん しつれん した らしい の	
\\	彼[かれ]は 最近[さいきん]、
\\	したらしいの。			
\\	恋する	恋[こい]する	こいする	
\\	恋する気持ちを歌にしました。	恋[こい]する 気持[きも]ちを 歌[うた]にしました。	こいする きもち を うた に しました	
\\	気持[きも]ちを 歌[うた]にしました。			
\\	延長	延長[えんちょう]	えんちょう	
\\	国会の会期が延長されたわね。	国会[こっかい]の 会期[かいき]が 延長[えんちょう]されたわね。	こっかい の かいき が えんちょう された わ ね	
\\	国会[こっかい]の 会期[かいき]が
\\	されたわね。			
\\	延期	延期[えんき]	えんき	
\\	運動会は雨で延期されました。	運動会[うんどうかい]は 雨[あめ]で 延期[えんき]されました。	うんどうかい は あめ で えんき されました	
\\	運動会[うんどうかい]は 雨[あめ]で
\\	されました。			
\\	祝う	祝[いわ]う	いわう	
\\	家族で彼の合格を祝ったよ。	家族[かぞく]で 彼[かれ]の 合格[ごうかく]を 祝[いわ]ったよ。	かぞく で かれ の ごうかく を いわった よ	
\\	家族[かぞく]で 彼[かれ]の 合格[ごうかく]を
\\	よ。			
\\	祝い	祝[いわ]い	いわい	
\\	結婚のお祝いに食器を頂きました。	結婚[けっこん]のお 祝[いわ]いに 食器[しょっき]を 頂[いただ]きました。	けっこん の おいわい に しょっき を いただきました	
\\	結婚[けっこん]のお
\\	に 食器[しょっき]を 頂[いただ]きました。			
\\	ウエスト	ウエスト	ウエスト	
\\	最近ウエストが太くなったな。	最近[さいきん]ウエストが 太[ふと]くなったな。	さいきん うえすと が ふとく なった な	
\\	最近[さいきん]
\\	が 太[ふと]くなったな。			
\\	純粋	純粋[じゅんすい]	じゅんすい	
\\	子供の純粋な心を傷付けてはいけません。	子供[こども]の 純粋[じゅんすい]な 心[こころ]を 傷付[きずつ]けてはいけません。	こども の じゅんすい な こころ を きずつけて は いけません	
\\	子供[こども]の
\\	な 心[こころ]を 傷付[きずつ]けてはいけません。			
\\	慎重	慎重[しんちょう]	しんちょう	
\\	もう一度慎重に見直しましょう。	もう 一度[いちど] 慎重[しんちょう]に 見直[みなお]しましょう。	もういちど しんちょう に みなおしましょう	
\\	もう 一度[いちど]
\\	に 見直[みなお]しましょう。			
\\	笑顔	笑顔[えがお]	えがお	
\\	赤ちゃんの笑顔が可愛いい。	赤[あか]ちゃんの 笑顔[えがお]が 可愛[かわ]いい。	あかちゃん の えがお が かわいい	
\\	赤[あか]ちゃんの
\\	が 可愛[かわ]いい。			
\\	大喜び	大喜[おおよろこ]び	おおよろこび	
\\	弟は新しい自転車に大喜びです。	弟[おとうと]は 新[あたら]しい 自転車[じてんしゃ]に 大喜[おおよろこ]びです。	おとうと は あたらしい じてんしゃ に おおよろこび です	
\\	弟[おとうと]は 新[あたら]しい 自転車[じてんしゃ]に
\\	です。			
\\	看護	看護[かんご]	かんご	
\\	この病院は24時間看護です。	この 病院[びょういん]は 24時間[にじゅうよじかん] 看護[かんご]です。	この びょういん は にじゅうよじかん かんご です	
\\	この 病院[びょういん]は 24時間[にじゅうよじかん]
\\	です。			
\\	看病	看病[かんびょう]	かんびょう	
\\	彼女は一晩中彼を看病したの。	彼女[かのじょ]は 一晩中彼[ひとばんじゅう かれ]を 看病[かんびょう]したの。	かのじょ は ひとばんじゅう かれ を かんびょう した の	
\\	彼女[かのじょ]は 一晩中彼[ひとばんじゅう かれ]を
\\	したの。			
\\	肯定	肯定[こうてい]	こうてい	
\\	僕は彼のしたことは肯定できないな。	僕[ぼく]は 彼[かれ]のしたことは 肯定[こうてい]できないな。	ぼく は かれ の した こと は こうてい できない な	
\\	僕[ぼく]は 彼[かれ]のしたことは
\\	できないな。			
\\	オルガン	オルガン	オルガン	
\\	僕たちはオルガンに合わせて賛美歌を歌ったんだ。	僕[ぼく]たちはオルガンに 合[あ]わせて 賛美歌[さんびか]を 歌[うた]ったんだ。	ぼくたち は おるがん に あわせて さんびか を うたった んだ	
\\	僕[ぼく]たちは
\\	に 合[あ]わせて 賛美歌[さんびか]を 歌[うた]ったんだ。			
\\	記述	記述[きじゅつ]	きじゅつ	
\\	日本の古い料理法についての記述を読んだんだ。	日本[にほん]の 古[ふる]い 料理法[りょうりほう]についての 記述[きじゅつ]を 読[よ]んだんだ。	にほん の ふるい りょうりほう に ついて の きじゅつ を よんだ ん だ	
\\	日本[にほん]の 古[ふる]い 料理法[りょうりほう]についての
\\	を 読[よ]んだんだ。			
\\	裁判	裁判[さいばん]	さいばん	
\\	裁判の様子はテレビで中継されたわよ。	裁判[さいばん]の 様子[ようす]はテレビで 中継[ちゅうけい]されたわよ。	さいばん の ようす は てれび で ちゅうけい された わ よ	
\\	の 様子[ようす]はテレビで 中継[ちゅうけい]されたわよ。			
\\	裁判所	裁判所[さいばんしょ]	さいばんしょ	
\\	彼らは裁判所の前で知らせを待っています。	彼[かれ]らは 裁判所[さいばんしょ]の 前[まえ]で 知[し]らせを 待[ま]っています。	かれら は さいばんしょ の まえ で しらせ を まって います	
\\	彼[かれ]らは
\\	の 前[まえ]で 知[し]らせを 待[ま]っています。			
\\	訴え	訴[うった]え	うったえ	
\\	彼女の訴えは認められたわ。	彼女[かのじょ]の 訴[うった]えは 認[みと]められたわ。	かのじょ の うったえ は みとめられた わ	
\\	彼女[かのじょ]の
\\	は 認[みと]められたわ。			
\\	区域	区域[くいき]	くいき	
\\	ここは危険区域よ。	ここは 危険[きけん] 区域[くいき]よ。	ここ は きけん くいき よ	
\\	ここは 危険[きけん]
\\	よ。			
\\	疑い	疑[うたが]い	うたがい	
\\	彼は盗みの疑いをかけられたの。	彼[かれ]は 盗[ぬす]みの 疑[うたが]いをかけられたの。	かれ は ぬすみ の うたがい を かけられた の	
\\	彼[かれ]は 盗[ぬす]みの
\\	をかけられたの。			
\\	疑問	疑問[ぎもん]	ぎもん	
\\	彼の言葉が本当かは疑問です。	彼[かれ]の 言葉[ことば]が 本当[ほんとう]かは 疑問[ぎもん]です。	かれ の ことば が ほんとう か は ぎもん です	
\\	彼[かれ]の 言葉[ことば]が 本当[ほんとう]かは
\\	です。			
\\	からから	からから	からから	
\\	彼はからからと笑いました。	彼[かれ]はからからと 笑[わら]いました。	かれ は からから と わらいました	
\\	彼[かれ]は
\\	と 笑[わら]いました。			
\\	疑う	疑[うたが]う	うたがう	
\\	なぜあなたは私を疑うのですか。	なぜあなたは 私[わたし]を 疑[うたが]うのですか。	なぜ あなた は わたし を うたがう の です か	
\\	なぜあなたは 私[わたし]を
\\	のですか。			
\\	疑わしい	疑[うたが]わしい	うたがわしい	
\\	疑わしい場所は全部調べよう。	疑[うたが]わしい 場所[ばしょ]は 全部調[ぜんぶ しら]べよう。	うたがわしい ばしょ は ぜんぶ しらべよう	
\\	場所[ばしょ]は 全部調[ぜんぶ しら]べよう。			
\\	著しい	著[いちじる]しい	いちじるしい	
\\	彼の成長は著しいです。	彼[かれ]の 成長[せいちょう]は 著[いちじる]しいです。	かれ の せいちょう は いちじるしい です	
\\	彼[かれ]の 成長[せいちょう]は
\\	です。			
\\	著す	著[あらわ]す	あらわす	
\\	この本では自然の大切さがよく著されているよ。	この 本[ほん]では 自然[しぜん]の 大切[たいせつ]さがよく 著[あらわ]されているよ。	この ほん で は しぜん の たいせつさ が よく あらわされて いる よ	
\\	この 本[ほん]では 自然[しぜん]の 大切[たいせつ]さがよく
\\	よ。			
\\	権利	権利[けんり]	けんり	
\\	私たちには知る権利があります。	私[わたし]たちには 知[し]る 権利[けんり]があります。	わたしたち に は しる けんり が あります	
\\	私[わたし]たちには 知[し]る
\\	があります。			
\\	権力	権力[けんりょく]	けんりょく	
\\	彼はこの国で大きな権力を持っているわ。	彼[かれ]はこの 国[くに]で 大[おお]きな 権力[けんりょく]を 持[も]っているわ。	かれ は この くに で おおき な けんりょく を もって いる わ	
\\	彼[かれ]はこの 国[くに]で 大[おお]きな
\\	を 持[も]っているわ。			
\\	人権	人権[じんけん]	じんけん	
\\	全ての人に人権がある。	全[すべ]ての 人[ひと]に 人権[じんけん]がある。	すべて の ひと に じんけん が ある	
\\	全[すべ]ての 人[ひと]に
\\	がある。			
\\	棄権	棄権[きけん]	きけん	
\\	彼は試合の途中で棄権したぞ。	彼[かれ]は 試合[しあい]の 途中[とちゅう]で 棄権[きけん]したぞ。	かれ は しあい の とちゅう で きけん した ぞ	
\\	彼[かれ]は 試合[しあい]の 途中[とちゅう]で
\\	したぞ。			
\\	きっちん	きっちん	きっちん	
\\	彼女はキッチンで夕食を作っています。	彼女[かのじょ]はキッチンで 夕食[ゆうしょく]を 作[つく]っています。	かのじょ は きっちん で ゆうしょく を つくって います	
\\	彼女[かのじょ]は
\\	で 夕食[ゆうしょく]を 作[つく]っています。			
\\	侵す	侵[おか]す	おかす	
\\	他人の権利を侵してはなりません。	他人[たにん]の 権利[けんり]を 侵[おか]してはなりません。	たにん の けんり を おかしては なりません	
\\	他人[たにん]の 権利[けんり]を
\\	はなりません。			
\\	行為	行為[こうい]	こうい	
\\	彼の行為はみんなの誤解を招いたわ。	彼[かれ]の 行為[こうい]はみんなの 誤解[ごかい]を 招[まね]いたわ。	かれ の こうい は みんな の ごかい を まねいた わ	
\\	彼[かれ]の
\\	はみんなの 誤解[ごかい]を 招[まね]いたわ。			
\\	賞	賞[しょう]	しょう	
\\	このデザインは数々の賞を受けています。	このデザインは 数々[かずかず]の 賞[しょう]を 受[う]けています。	この でざいん は かずかず の しょう を うけて います	
\\	このデザインは 数々[かずかず]の
\\	を 受[う]けています。			
\\	賞品	賞品[しょうひん]	しょうひん	
\\	パーティーのビンゴの賞品は何がいいだろう。	パーティーのビンゴの 賞品[しょうひん]は 何[なに]がいいだろう。	ぱーてぃー の びんご の しょうひん は なに が いいだろう	
\\	パーティーのビンゴの
\\	は 何[なに]がいいだろう。			
\\	観賞	観賞[かんしょう]	かんしょう	
\\	この鉢植えは観賞用です。	この 鉢植[はちう]えは 観賞[かんしょう] 用[よう]です。	この はちうえ は かんしょうよう です	
\\	この 鉢植[はちう]えは
\\	用[よう]です。			
\\	財政	財政[ざいせい]	ざいせい	
\\	国の財政はとても苦しい状態だな。	国[くに]の 財政[ざいせい]はとても 苦[くる]しい 状態[じょうたい]だな。	くに の ざいせい は とても くるしい じょうたい だ な	
\\	国[くに]の
\\	はとても 苦[くる]しい 状態[じょうたい]だな。			
\\	財産	財産[ざいさん]	ざいさん	
\\	彼は株で財産の半分を失ったの。	彼[かれ]は 株[かぶ]で 財産[ざいさん]の 半分[はんぶん]を 失[うしな]ったの。	かれ は かぶ で ざいさん の はんぶん を うしなった の	
\\	彼[かれ]は 株[かぶ]で
\\	の 半分[はんぶん]を 失[うしな]ったの。			
\\	くしゃくしゃ	くしゃくしゃ	くしゃくしゃ	
\\	紙をくしゃくしゃに丸めました。	紙[かみ]をくしゃくしゃに 丸[まる]めました。	かみ を くしゃくしゃ に まるめました	
\\	紙[かみ]を
\\	に 丸[まる]めました。			
\\	金融	金融[きんゆう]	きんゆう	
\\	彼は金融関係の会社で働いています。	彼[かれ]は 金融[きんゆう] 関係[かんけい]の 会社[かいしゃ]で 働[はたら]いています。	かれ は きんゆう かんけい の かいしゃ で はたらいて います	
\\	彼[かれ]は
\\	関係[かんけい]の 会社[かいしゃ]で 働[はたら]いています。			
\\	間隔	間隔[かんかく]	かんかく	
\\	電車は5分間隔で来ますよ。	電車[でんしゃ]は 5分[ごふん] 間隔[かんかく]で 来[き]ますよ。	でんしゃ は ごふん かんかく で きます よ	
\\	電車[でんしゃ]は 5分[ごふん]
\\	で 来[き]ますよ。			
\\	終了	終了[しゅうりょう]	しゅうりょう	
\\	コンサートは夜7時5分に終了しました。	コンサートは 夜7時5分[よる しち じ ご ふん]に 終了[しゅうりょう]しました。	こんさーと は よる しち じ ご ふん に しゅうりょう しました	
\\	コンサートは 夜7時5分[よる しち じ ご ふん]に
\\	しました。			
\\	完了	完了[かんりょう]	かんりょう	
\\	仕事は全て完了しました。	仕事[しごと]は 全[すべ]て 完了[かんりょう]しました。	しごと は すべて かんりょう しました	
\\	仕事[しごと]は 全[すべ]て
\\	しました。			
\\	修了	修了[しゅうりょう]	しゅうりょう	
\\	先月、そのコースを修了しました。	先月[せんげつ]、そのコースを 修了[しゅうりょう]しました。	せんげつ その こーす を しゅうりょう しました	
\\	先月[せんげつ]、そのコースを
\\	しました。			
\\	承認	承認[しょうにん]	しょうにん	
\\	これは政府の承認を受けた資格です。	これは 政府[せいふ]の 承認[しょうにん]を 受[う]けた 資格[しかく]です。	これ は せいふ の しょうにん を うけた しかく です	
\\	これは 政府[せいふ]の
\\	を 受[う]けた 資格[しかく]です。			
\\	承知	承知[しょうち]	しょうち	
\\	そのことは承知しております。	そのことは 承知[しょうち]しております。	その こと は しょうち して おります	
\\	そのことは
\\	しております。			
\\	けち	けち	けち	
\\	彼は金持ちだけど、けちね。	彼[かれ]は 金持[かねも]ちだけど、けちね。	かれ は かねもち だ けど けち ね	
\\	彼[かれ]は 金持[かねも]ちだけど、
\\	ね。			
\\	納める	納[おさ]める	おさめる	
\\	自動車税を納めたよ。	自動車税[じどうしゃぜい]を 納[おさ]めたよ。	じどうしゃぜい を おさめた よ	
\\	自動車税[じどうしゃぜい]を
\\	よ。			
\\	説得	説得[せっとく]	せっとく	
\\	なんとか親を説得してみるよ。	なんとか 親[おや]を 説得[せっとく]してみるよ。	なんとか おや を せっとく して みる よ	
\\	なんとか 親[おや]を
\\	してみるよ。			
\\	所得	所得[しょとく]	しょとく	
\\	ここに去年の所得をご記入ください。	ここに 去年[きょねん]の 所得[しょとく]をご 記入[きにゅう]ください。	ここ に きょねん の しょとく を ご きにゅう ください	
\\	ここに 去年[きょねん]の
\\	をご 記入[きにゅう]ください。			
\\	得る	得[え]る	える	
\\	彼は大金を得ましたよ。	彼[かれ]は 大金[たいきん]を 得[え]ましたよ。	かれ は たいきん を えました よ	
\\	彼[かれ]は 大金[たいきん]を
\\	よ。			
\\	乾燥	乾燥[かんそう]	かんそう	
\\	冬は空気が乾燥しますね。	冬[ふゆ]は 空気[くうき]が 乾燥[かんそう]しますね。	ふゆ は くうき が かんそう します ね	
\\	冬[ふゆ]は 空気[くうき]が
\\	しますね。			
\\	幹部	幹部[かんぶ]	かんぶ	
\\	あの会社の幹部は皆とても優秀だね。	あの 会社[かいしゃ]の 幹部[かんぶ]は 皆[みんな]とても 優秀[ゆうしゅう]だね。	あの かいしゃ の かんぶ は みんな とても ゆうしゅう だ ね	
\\	あの 会社[かいしゃ]の
\\	は 皆[みんな]とても 優秀[ゆうしゅう]だね。			
\\	酸素	酸素[さんそ]	さんそ	
\\	私たちは酸素無しでは生きていけない。	私[わたし]たちは 酸素[さんそ] 無[な]しでは 生[い]きていけない。	わたしたち は さんそ なし で は いきて いけない	
\\	私[わたし]たちは
\\	無[な]しでは 生[い]きていけない。			
\\	素直	素直[すなお]	すなお	
\\	彼女はとても素直で可愛いですね。	彼女[かのじょ]はとても 素直[すなお]で 可愛[かわい]いですね。	かのじょ は とても すなお で かわいい です ね	
\\	彼女[かのじょ]はとても
\\	で 可愛[かわい]いですね。			
\\	ざらざら	ざらざら	ざらざら	
\\	砂ぼこりで机がざらざらしているね。	砂[すな]ぼこりで 机[つくえ]がざらざらしているね。	すなぼこり で つくえ が ざらざら して いる ね	
\\	砂[すな]ぼこりで 机[つくえ]が
\\	しているね。			
\\	水素	水素[すいそ]	すいそ	
\\	水は水素と酸素でできています。	水[みず]は 水素[すいそ]と 酸素[さんそ]でできています。	みず は すいそ と さんそ で できて います	
\\	水[みず]は
\\	と 酸素[さんそ]でできています。			
\\	素早い	素早[すばや]い	すばやい	
\\	彼は素早くあたりを見回したの。	彼[かれ]は 素早[すばや]くあたりを 見回[みまわ]したの。	かれ は すばやく あたり を みまわした の	
\\	彼[かれ]は
\\	あたりを 見回[みまわ]したの。			
\\	石炭	石炭[せきたん]	せきたん	
\\	小屋に石炭の山があります。	小屋[こや]に 石炭[せきたん]の 山[やま]があります。	こや に せきたん の やま が あります	
\\	小屋[こや]に
\\	の 山[やま]があります。			
\\	岩	岩[いわ]	いわ	
\\	あの岩まで泳ごう。	あの 岩[いわ]まで 泳[およ]ごう。	あの いわ まで およごう	
\\	あの
\\	まで 泳[およ]ごう。			
\\	岸	岸[きし]	きし	
\\	船がやっと岸に着いたよ。	船[ふね]がやっと 岸[きし]に 着[つ]いたよ。	ふね が やっと きし に ついた よ	
\\	船[ふね]がやっと
\\	に 着[つ]いたよ。			
\\	校庭	校庭[こうてい]	こうてい	
\\	陸上部は校庭で練習しています。	陸上部[りくじょうぶ]は 校庭[こうてい]で 練習[れんしゅう]しています。	りくじょうぶ は こうてい で れんしゅう して います	
\\	陸上部[りくじょうぶ]は
\\	で 練習[れんしゅう]しています。			
\\	解散	解散[かいさん]	かいさん	
\\	来年、衆議院が解散されるだろう。	来年[らいねん]、 衆議院[しゅうぎいん]が 解散[かいさん]されるだろう。	らいねん しゅうぎいん が かいさん される だろう	
\\	来年[らいねん]、 衆議院[しゅうぎいん]が
\\	されるだろう。			
\\	スモッグ	スモッグ	スモッグ	
\\	都会の空はスモッグで灰色ね。	都会[とかい]の 空[そら]はスモッグで 灰色[はいいろ]ね。	とかい の そら は すもっぐ で はいいろ ね	
\\	都会[とかい]の 空[そら]は
\\	で 灰色[はいいろ]ね。			
\\	植物	植物[しょくぶつ]	しょくぶつ	
\\	休日は植物の世話をして過ごします。	休日[きゅうじつ]は 植物[しょくぶつ]の 世話[せわ]をして 過[す]ごします。	きゅうじつ は しょくぶつ の せわ を して すごします	
\\	休日[きゅうじつ]は
\\	の 世話[せわ]をして 過[す]ごします。			
\\	植民地	植民地[しょくみんち]	しょくみんち	
\\	この国はイギリスの植民地でした。	この 国[くに]はイギリスの 植民地[しょくみんち]でした。	この くに は いぎりす の しょくみんち でした	
\\	この 国[くに]はイギリスの
\\	でした。			
\\	植木	植木[うえき]	うえき	
\\	植木に水をやりました。	植木[うえき]に 水[みず]をやりました。	うえき に みず を やりました	
\\	に 水[みず]をやりました。			
\\	植物園	植物園[しょくぶつえん]	しょくぶつえん	
\\	植物園には珍しい花がたくさんありますね。	植物園[しょくぶつえん]には 珍[めずら]しい 花[はな]がたくさんありますね。	しょくぶつえん に は めずらしい はな が たくさん あります ね	
\\	には 珍[めずら]しい 花[はな]がたくさんありますね。			
\\	根拠	根拠[こんきょ]	こんきょ	
\\	何を根拠にそんな事を言うのですか。	何[なに]を 根拠[こんきょ]にそんな 事[こと]を 言[い]うのですか。	なに を こんきょ に そんな こと を いう の です か	
\\	何[なに]を
\\	にそんな 事[こと]を 言[い]うのですか。			
\\	根本	根本[こんぽん]	こんぽん	
\\	問題の根本を見直しましょう。	問題[もんだい]の 根本[こんぽん]を 見直[みなお]しましょう。	もんだい の こんぽん を みなおしましょう	
\\	問題[もんだい]の
\\	を 見直[みなお]しましょう。			
\\	板	板[いた]	いた	
\\	父は長い板を買って来たんだ。	父[ちち]は 長[なが]い 板[いた]を 買[か]って 来[き]たんだ。	ちち は ながい いた を かって きた ん だ	
\\	父[ちち]は 長[なが]い
\\	を 買[か]って 来[き]たんだ。			
\\	すり	すり	すり	
\\	すりに財布をとられた。	すりに 財布[さいふ]をとられた。	すり に さいふ を とられた	
\\	に 財布[さいふ]をとられた。			
\\	看板	看板[かんばん]	かんばん	
\\	店の看板を塗り替えました。	店[みせ]の 看板[かんばん]を 塗[ぬ]り 替[か]えました。	みせ の かんばん を ぬりかえました	
\\	店[みせ]の
\\	を 塗[ぬ]り 替[か]えました。			
\\	草花	草花[くさばな]	くさばな	
\\	草花を大切にしましょう。	草花[くさばな]を 大切[たいせつ]にしましょう。	くさばな を たいせつ に しましょう	
\\	を 大切[たいせつ]にしましょう。			
\\	草木	草木[くさき]	くさき	
\\	この庭は草木が枯れているね。	この 庭[にわ]は 草木[くさき]が 枯[か]れているね。	この にわ は くさき が かれて いる ね	
\\	この 庭[にわ]は
\\	が 枯[か]れているね。			
\\	言葉遣い	言葉遣[ことばづか]い	ことばづかい	
\\	先生にそんな言葉遣いをしてはいけません。	先生[せんせい]にそんな 言葉遣[ことばづか]いをしてはいけません。	せんせい に そんな ことばづかい を して は いけません	
\\	先生[せんせい]にそんな
\\	をしてはいけません。			
\\	落ち葉	落[お]ち 葉[ば]	おちば	
\\	落ち葉の季節になりましたね。	落[お]ち 葉[ば]の 季節[きせつ]になりましたね。	おちば の きせつ に なりました ね	
\\	の 季節[きせつ]になりましたね。			
\\	木の葉	木[こ]の 葉[は]	このは	
\\	秋には木の葉が赤くなります。	秋[あき]には 木[こ]の 葉[は]が 赤[あか]くなります。	あき に は このは が あかく なります	
\\	秋[あき]には
\\	が 赤[あか]くなります。			
\\	書き言葉	書[か]き 言葉[ことば]	かきことば	
\\	書き言葉と話し言葉はだいぶ違うことがあります。	書[か]き 言葉[ことば]と 話[はな]し 言葉[ことば]はだいぶ 違[ちが]うことがあります。	かきことば と はなしことば は だいぶ ちがう こと が あります	
\\	と 話[はな]し 言葉[ことば]はだいぶ 違[ちが]うことがあります。			
\\	吸収	吸収[きゅうしゅう]	きゅうしゅう	
\\	彼は知識の吸収が早いですね。	彼[かれ]は 知識[ちしき]の 吸収[きゅうしゅう]が 早[はや]いですね。	かれ は ちしき の きゅうしゅう が はやい です ね	
\\	彼[かれ]は 知識[ちしき]の
\\	が 早[はや]いですね。			
\\	おにぎり	おにぎり	おにぎり	
\\	昼食におにぎりを食べました。	昼食[ちゅうしょく]におにぎりを 食[た]べました。	ちゅうしょく に おにぎり を たべました	
\\	昼食[ちゅうしょく]に
\\	を 食[た]べました。			
\\	呼吸	呼吸[こきゅう]	こきゅう	
\\	ゆっくり呼吸してください。	ゆっくり 呼吸[こきゅう]してください。	ゆっくり こきゅう して ください	
\\	ゆっくり
\\	してください。			
\\	吸い込む	吸[す]い 込[こ]む	すいこむ	
\\	ほこりを吸い込んじゃった。	ほこりを 吸[す]い 込[こ]んじゃった。	ほこり を すいこんじゃった	
\\	ほこりを
\\	及ぶ	及[およ]ぶ	およぶ	
\\	あなたにまで迷惑が及んでごめんなさい。	あなたにまで 迷惑[めいわく]が 及[およ]んでごめんなさい。	あなた に まで めいわく が およんで ごめんなさい	
\\	あなたにまで 迷惑[めいわく]が
\\	ごめんなさい。			
\\	扱う	扱[あつか]う	あつかう	
\\	この荷物は丁寧に扱って下さい。	この 荷物[にもつ]は 丁寧[ていねい]に 扱[あつか]って 下[くだ]さい。	この にもつ は ていねい に あつかって ください	
\\	この 荷物[にもつ]は 丁寧[ていねい]に
\\	下[くだ]さい。			
\\	高級	高級[こうきゅう]	こうきゅう	
\\	私たちは高級ホテルに泊まったの。	私[わたし]たちは 高級[こうきゅう]ホテルに 泊[と]まったの。	わたしたち は こうきゅう ほてる に とまった の	
\\	私[わたし]たちは
\\	ホテルに 泊[と]まったの。			
\\	級	級[きゅう]	きゅう	
\\	彼は書道3級です。	彼[かれ]は 書道3[しょどう さん] 級[きゅう]です。	かれ は しょどう さんきゅう です	
\\	彼[かれ]は 書道3[しょどう さん]
\\	です。			
\\	上級	上級[じょうきゅう]	じょうきゅう	
\\	彼は上級のコースに上がったよ。	彼[かれ]は 上級[じょうきゅう]のコースに 上[あ]がったよ。	かれ は じょうきゅう の こーす に あがった よ	
\\	彼[かれ]は
\\	のコースに 上[あ]がったよ。			
\\	かみそり	かみそり	かみそり	
\\	かみそりの刃で指を切りました。	かみそりの 刃[は]で 指[ゆび]を 切[き]りました。	かみそり の は で ゆび を きりました	
\\	の 刃[は]で 指[ゆび]を 切[き]りました。			
\\	初級	初級[しょきゅう]	しょきゅう	
\\	これは初級の教科書です。	これは 初級[しょきゅう]の 教科書[きょうかしょ]です。	これ は しょきゅう の きょうかしょ です 。	
\\	これは
\\	の 教科書[きょうかしょ]です。			
\\	血管	血管[けっかん]	けっかん	
\\	年をとると血管が硬くなります。	年[とし]をとると 血管[けっかん]が 硬[かた]くなります。	とし を とる と けっかん が かたく なります	
\\	年[とし]をとると
\\	が 硬[かた]くなります。			
\\	出血	出血[しゅっけつ]	しゅっけつ	
\\	出血がひどいので、医者に行ったほうがいい。	出血[しゅっけつ]がひどいので、 医者[いしゃ]に 行[い]ったほうがいい。	しゅっけつ が ひどい の で いしゃ に いった ほう が いい	
\\	がひどいので、 医者[いしゃ]に 行[い]ったほうがいい。			
\\	液体	液体[えきたい]	えきたい	
\\	洗濯に液体の洗剤を使っています。	洗濯[せんたく]に 液体[えきたい]の 洗剤[せんざい]を 使[つか]っています。	せんたく に えきたい の せんざい を つかって います	
\\	洗濯[せんたく]に
\\	の 洗剤[せんざい]を 使[つか]っています。			
\\	血液	血液[けつえき]	けつえき	
\\	心臓は全身に血液を送り出している。	心臓[しんぞう]は 全身[ぜんしん]に 血液[けつえき]を 送[おく]り 出[だ]している。	しんぞう は ぜんしん に けつえき を おくりだして いる	
\\	心臓[しんぞう]は 全身[ぜんしん]に
\\	を 送[おく]り 出[だ]している。			
\\	液	液[えき]	えき	
\\	容器から液がこぼれていますよ。	容器[ようき]から 液[えき]がこぼれていますよ。	ようき から えき が こぼれて います よ	
\\	容器[ようき]から
\\	がこぼれていますよ。			
\\	背中	背中[せなか]	せなか	
\\	背中がかゆいです。	背中[せなか]がかゆいです。	せなか が かゆい です	
\\	がかゆいです。			
\\	がやがや	がやがや	がやがや	
\\	生徒たちががやがやと騒いでいますね。	生徒[せいと]たちががやがやと 騒[さわ]いでいますね。	せいとたち が がやがや と さわいで います ね	
\\	生徒[せいと]たちが
\\	と 騒[さわ]いでいますね。			
\\	背広	背広[せびろ]	せびろ	
\\	背広をクリーニングに出しました。	背広[せびろ]をクリーニングに 出[だ]しました。	せびろ を くりーにんぐ に だしました	
\\	をクリーニングに 出[だ]しました。			
\\	骨折	骨折[こっせつ]	こっせつ	
\\	彼女はスキーで足を骨折したの。	彼女[かのじょ]はスキーで 足[あし]を 骨折[こっせつ]したの。	かのじょ は すきー で あし を こっせつ した の	
\\	彼女[かのじょ]はスキーで 足[あし]を
\\	したの。			
\\	健全	健全[けんぜん]	けんぜん	
\\	子供の健全な心を育てましょう。	子供[こども]の 健全[けんぜん]な 心[こころ]を 育[そだ]てましょう。	こども の けんぜん な こころ を そだてましょう	
\\	子供[こども]の
\\	な 心[こころ]を 育[そだ]てましょう。			
\\	健康	健康[けんこう]	けんこう	
\\	健康が一番大切だ。	健康[けんこう]が 一番大切[いちばん たいせつ]だ。	けんこう が いちばん たいせつ だ 。	
\\	が 一番大切[いちばん たいせつ]だ。			
\\	診断	診断[しんだん]	しんだん	
\\	医師は異常なしと診断したんだ。	医師[いし]は 異常[いじょう]なしと 診断[しんだん]したんだ。	いし は いじょう なし と しんだん した ん だ	
\\	医師[いし]は 異常[いじょう]なしと
\\	したんだ。			
\\	診察	診察[しんさつ]	しんさつ	
\\	今日、病院で診察してもらったの。	今日[きょう]、 病院[びょういん]で 診察[しんさつ]してもらったの。	きょう びょういん で しんさつ して もらった の	
\\	今日[きょう]、 病院[びょういん]で
\\	してもらったの。			
\\	医療	医療[いりょう]	いりょう	
\\	医療関係の仕事をしています。	医療[いりょう] 関係[かんけい]の 仕事[しごと]をしています。	いりょう かんけい の しごと を して います	
\\	関係[かんけい]の 仕事[しごと]をしています。			
\\	きょろきょろ	きょろきょろ	きょろきょろ	
\\	会場で子供がきょろきょろしているわよ。	会場[かいじょう]で 子供[こども]がきょろきょろしているわよ。	かいじょう で こども が きょろきょろ して いる わ よ	
\\	会場[かいじょう]で 子供[こども]が
\\	しているわよ。			
\\	気の毒	気[き]の 毒[どく]	きのどく	
\\	彼らは気の毒な生活をしている。	彼[かれ]らは 気[き]の 毒[どく]な 生活[せいかつ]をしている。	かれら は きのどく な せいかつ を して いる	
\\	彼[かれ]らは
\\	な 生活[せいかつ]をしている。			
\\	消毒	消毒[しょうどく]	しょうどく	
\\	足の傷を消毒したわ。	足[あし]の 傷[きず]を 消毒[しょうどく]したわ。	あし の きず を しょうどく した わ	
\\	足[あし]の 傷[きず]を
\\	したわ。			
\\	症状	症状[しょうじょう]	しょうじょう	
\\	医者に症状を説明したよ。	医者[いしゃ]に 症状[しょうじょう]を 説明[せつめい]したよ。	いしゃ に しょうじょう を せつめい した よ	
\\	医者[いしゃ]に
\\	を 説明[せつめい]したよ。			
\\	清書	清書[せいしょ]	せいしょ	
\\	この手紙を清書してください。	この 手紙[てがみ]を 清書[せいしょ]してください。	この てがみ を せいしょ して ください	
\\	この 手紙[てがみ]を
\\	してください。			
\\	清潔	清潔[せいけつ]	せいけつ	
\\	レストランは清潔が第一です。	レストランは 清潔[せいけつ]が 第一[だいいち]です。	れすとらん は せいけつ が だいいち です	
\\	レストランは
\\	が 第一[だいいち]です。			
\\	監督	監督[かんとく]	かんとく	
\\	その監督はアカデミー賞を受賞したよな。	その 監督[かんとく]はアカデミー 賞[しょう]を 受賞[じゅしょう]したよな。	その かんとく は あかでみーしょう を じゅしょう した よ な	
\\	その
\\	はアカデミー 賞[しょう]を 受賞[じゅしょう]したよな。			
\\	撮影	撮影[さつえい]	さつえい	
\\	撮影は3ヶ月かけて行われました。	撮影[さつえい]は 3ヶ月[さんかげつ]かけて 行[おこな]われました。	さつえい は さんかげつ かけて おこなわれました	
\\	は 3ヶ月[さんかげつ]かけて 行[おこな]われました。			
\\	描く	描[えが]く	えがく	
\\	彼は人物を描くのがうまいな。	彼[かれ]は 人物[じんぶつ]を 描[えが]くのがうまいな。	かれ は じんぶつ を えがく の が うまい な	
\\	彼[かれ]は 人物[じんぶつ]を
\\	のがうまいな。			
\\	くしゃみ	くしゃみ	くしゃみ	
\\	風邪でくしゃみが止まりません。	風邪[かぜ]でくしゃみが 止[と]まりません。	かぜ で くしゃみ が とまりません	
\\	風邪[かぜ]で
\\	が 止[と]まりません。			
\\	活躍	活躍[かつやく]	かつやく	
\\	彼の活躍で優勝したよ。	彼[かれ]の 活躍[かつやく]で 優勝[ゆうしょう]したよ。	かれ の かつやく で ゆうしょう した よ	
\\	彼[かれ]の
\\	で 優勝[ゆうしょう]したよ。			
\\	再開	再開[さいかい]	さいかい	
\\	試合はすぐに再開されたの。	試合[しあい]はすぐに 再開[さいかい]されたの。	しあい は すぐ に さいかい された の	
\\	試合[しあい]はすぐに
\\	されたの。			
\\	再生	再生[さいせい]	さいせい	
\\	留守番電話のメッセージを再生したの。	留守番電話[るすばんでんわ]のメッセージを 再生[さいせい]したの。	るすばんでんわ の めっせーじ を さいせい した の	
\\	留守番電話[るすばんでんわ]のメッセージを
\\	したの。			
\\	再会	再会[さいかい]	さいかい	
\\	彼らは互いに再会を喜んだの。	彼[かれ]らは 互[たが]いに 再会[さいかい]を 喜[よろこ]んだの。	かれら は たがいに さいかい を よろこんだ の	
\\	彼[かれ]らは 互[たが]いに
\\	を 喜[よろこ]んだの。			
\\	編む	編[あ]む	あむ	
\\	妹はマフラーを編みました。	妹[いもうと]はマフラーを 編[あ]みました。	いもうと は まふらー を あみました	
\\	妹[いもうと]はマフラーを
\\	解放	解放[かいほう]	かいほう	
\\	人質が解放されてよかった。	人質[ひとじち]が 解放[かいほう]されてよかった。	ひとじち が かいほう されて よかった	
\\	人質[ひとじち]が
\\	されてよかった。			
\\	開放	開放[かいほう]	かいほう	
\\	その国は市場の開放を求められているの。	その 国[くに]は 市場[しじょう]の 開放[かいほう]を 求[もと]められているの。	その くに は しじょう の かいほう を もとめられて いる の	
\\	その 国[くに]は 市場[しじょう]の
\\	を 求[もと]められているの。			
\\	ぐずぐず	ぐずぐず	ぐずぐず	
\\	ぐずぐずしていると電車に乗り遅れますよ。	ぐずぐずしていると 電車[でんしゃ]に 乗[の]り 遅[おく]れますよ。	ぐずぐず して いる と でんしゃ に のりおくれます よ	
\\	していると 電車[でんしゃ]に 乗[の]り 遅[おく]れますよ。			
\\	重視	重視[じゅうし]	じゅうし	
\\	あの企業では学歴が重視されるよ。	あの 企業[きぎょう]では 学歴[がくれき]が 重視[じゅうし]されるよ。	あの きぎょう で は がくれき が じゅうし される よ	
\\	あの 企業[きぎょう]では 学歴[がくれき]が
\\	されるよ。			
\\	視点	視点[してん]	してん	
\\	視点を変えて見てみましょう。	視点[してん]を 変[か]えて 見[み]てみましょう。	してん を かえて みて みましょう	
\\	を 変[か]えて 見[み]てみましょう。			
\\	近視	近視[きんし]	きんし	
\\	彼は軽い近視です。	彼[かれ]は 軽[かる]い 近視[きんし]です。	かれ は かるい きんし です	
\\	彼[かれ]は 軽[かる]い
\\	です。			
\\	衣類	衣類[いるい]	いるい	
\\	衣類の整理をしたよ。	衣類[いるい]の 整理[せいり]をしたよ。	いるい の せいり を した よ	
\\	の 整理[せいり]をしたよ。			
\\	衣服	衣服[いふく]	いふく	
\\	彼の会社は衣服を扱っているよ。	彼[かれ]の 会社[かいしゃ]は 衣服[いふく]を 扱[あつか]っているよ。	かれ の かいしゃ は いふく を あつかって いる よ	
\\	彼[かれ]の 会社[かいしゃ]は
\\	を 扱[あつか]っているよ。			
\\	衣料	衣料[いりょう]	いりょう	
\\	その会社は衣料を扱っているんだ。	その 会社[かいしゃ]は 衣料[いりょう]を 扱[あつか]っているんだ。	その かいしゃ は いりょう を あつかって いる ん だ	
\\	その 会社[かいしゃ]は
\\	を 扱[あつか]っているんだ。			
\\	衣食住	衣食住[いしょくじゅう]	いしょくじゅう	
\\	土地によって衣食住は変化する。	土地[とち]によって 衣食住[いしょくじゅう]は 変化[へんか]する。	とち によって いしょくじゅう は へんか する	
\\	土地[とち]によって
\\	は 変化[へんか]する。			
\\	くすぐったい	くすぐったい	くすぐったい	
\\	犬に顔をなめられてくすぐったいよ。	犬[いぬ]に 顔[かお]をなめられてくすぐったいよ。	いぬ に かお を なめられて くすぐったい よ 。	
\\	犬[いぬ]に 顔[かお]をなめられて
\\	よ。			
\\	仮に	仮[かり]に	かりに	
\\	仮にそれが事実だとしてももう遅いよ。	仮[かり]にそれが 事実[じじつ]だとしてももう 遅[おそ]いよ。	かりに それ が じじつ だ と して も もう おそい よ	
\\	それが 事実[じじつ]だとしてももう 遅[おそ]いよ。			
\\	仮定	仮定[かてい]	かてい	
\\	仮定の話だけでは結論は出ません。	仮定[かてい]の 話[はなし]だけでは 結論[けつろん]は 出[で]ません。	かてい の はなし だけ で は けつろん は でません	
\\	の 話[はなし]だけでは 結論[けつろん]は 出[で]ません。			
\\	仮	仮[かり]	かり	
\\	仮の申し込みをしました。	仮[かり]の 申[もう]し 込[こ]みをしました。	かり の もうしこみ を しました	
\\	の 申[もう]し 込[こ]みをしました。			
\\	仮名遣い	仮名遣[かなづか]い	かなづかい	
\\	祖父は古い仮名遣いで書くことがあるの。	祖父[そふ]は 古[ふる]い 仮名遣[かなづか]いで 書[か]くことがあるの。	そふ は ふるい かなづかい で かく こと が ある の	
\\	祖父[そふ]は 古[ふる]い
\\	で 書[か]くことがあるの。			
\\	演説	演説[えんぜつ]	えんぜつ	
\\	彼は地方で演説したのよ。	彼[かれ]は 地方[ちほう]で 演説[えんぜつ]したのよ。	かれ は ちほう で えんぜつ した の よ	
\\	彼[かれ]は 地方[ちほう]で
\\	したのよ。			
\\	公演	公演[こうえん]	こうえん	
\\	彼らの海外公演が発表されたよ。	彼[かれ]らの 海外[かいがい] 公演[こうえん]が 発表[はっぴょう]されたよ。	かれら の かいがいこうえん が はっぴょう された よ	
\\	彼[かれ]らの 海外[かいがい]
\\	が 発表[はっぴょう]されたよ。			
\\	演じる	演[えん]じる	えんじる	
\\	彼はよく刑事の役を演じるね。	彼[かれ]はよく 刑事[けいじ]の 役[やく]を 演[えん]じるね。	かれ は よく けいじ の やく を えんじる ね	
\\	彼[かれ]はよく 刑事[けいじ]の 役[やく]を
\\	ね。			
\\	出演	出演[しゅつえん]	しゅつえん	
\\	彼女はその映画に出演しているよ。	彼女[かのじょ]はその 映画[えいが]に 出演[しゅつえん]しているよ。	かのじょ は その えいが に しゅつえん して いる よ	
\\	彼女[かのじょ]はその 映画[えいが]に
\\	しているよ。			
\\	くたびれる	くたびれる	くたびれる	
\\	たくさん歩いてくたびれました。	たくさん 歩[ある]いてくたびれました。	たくさん あるいて くたびれました	
\\	たくさん 歩[ある]いて
\\	劇	劇[げき]	げき	
\\	小学校で子供たちの劇を見ました。	小学校[しょうがっこう]で 子供[こども]たちの 劇[げき]を 見[み]ました。	しょうがっこう で こどもたち の げき を みました	
\\	小学校[しょうがっこう]で 子供[こども]たちの
\\	を 見[み]ました。			
\\	劇場	劇場[げきじょう]	げきじょう	
\\	午後8時に劇場の前で会いましょう。	午後8時[ごご はちじ]に 劇場[げきじょう]の 前[まえ]で 会[あ]いましょう。	ごご はちじ に げきじょう の まえ で あいましょう	
\\	午後8時[ごご はちじ]に
\\	の 前[まえ]で 会[あ]いましょう。			
\\	演劇	演劇[えんげき]	えんげき	
\\	彼女は演劇を学んでいます。	彼女[かのじょ]は 演劇[えんげき]を 学[まな]んでいます。	かのじょ は えんげき を まなんで います	
\\	彼女[かのじょ]は
\\	を 学[まな]んでいます。			
\\	喜劇	喜劇[きげき]	きげき	
\\	昨夜はテレビで喜劇を見たよ。	昨夜[さくや]はテレビで 喜劇[きげき]を 見[み]たよ。	さくや は てれび で きげき を みた よ	
\\	昨夜[さくや]はテレビで
\\	を 見[み]たよ。			
\\	悲しみ	悲[かな]しみ	かなしみ	
\\	突然の悲しみが一家を襲いました。	突然[とつぜん]の 悲[かな]しみが 一家[いっか]を 襲[おそ]いました。	とつぜん の かなしみ が いっか を おそいました	
\\	突然[とつぜん]の
\\	が 一家[いっか]を 襲[おそ]いました。			
\\	集団	集団[しゅうだん]	しゅうだん	
\\	テロ集団が警察に捕まりました。	テロ 集団[しゅうだん]が 警察[けいさつ]に 捕[つか]まりました。	てろ しゅうだん が けいさつ に つかまりました	
\\	テロ
\\	が 警察[けいさつ]に 捕[つか]まりました。			
\\	固定	固定[こてい]	こてい	
\\	棒をテープで固定しなさい。	棒[ぼう]をテープで 固定[こてい]しなさい。	ぼう を てーぷ で こてい しなさい	
\\	棒[ぼう]をテープで
\\	しなさい。			
\\	いらいら	いらいら	いらいら	
\\	彼女は朝からいらいらしている。	彼女[かのじょ]は 朝[あさ]からいらいらしている。	かのじょ は あさ から いらいら して いる	
\\	彼女[かのじょ]は 朝[あさ]から
\\	している。			
\\	固める	固[かた]める	かためる	
\\	私はもう決心を固めたの。	私[わたし]はもう 決心[けっしん]を 固[かた]めたの。	わたし は もう けっしん を かためた の	
\\	私[わたし]はもう 決心[けっしん]を
\\	の。			
\\	固まる	固[かた]まる	かたまる	
\\	もうプリンは固まったかな。	もうプリンは 固[かた]まったかな。	もう ぷりん は かたまった か な	
\\	もうプリンは
\\	かな。			
\\	固体	固体[こたい]	こたい	
\\	氷は固体です。	氷[こおり]は 固体[こたい]です。	こおり は こたい です	
\\	氷[こおり]は
\\	です。			
\\	固有	固有[こゆう]	こゆう	
\\	これは日本固有の鳥です。	これは 日本[にほん] 固有[こゆう]の 鳥[とり]です。	これ は にほん こゆう の とり です	
\\	これは 日本[にほん]
\\	の 鳥[とり]です。			
\\	主催	主催[しゅさい]	しゅさい	
\\	その芸術祭は市が主催しています。	その 芸術祭[げいじゅつさい]は 市[し]が 主催[しゅさい]しています。	その げいじゅつさい は し が しゅさい して います	
\\	その 芸術祭[げいじゅつさい]は 市[し]が
\\	しています。			
\\	促す	促[うなが]す	うながす	
\\	彼に集中するよう注意を促しといたよ。	彼[かれ]に 集中[しゅうちゅう]するよう 注意[ちゅうい]を 促[うなが]しといたよ。	かれ に しゅうちゅう する よう ちゅうい を うながし とい た よ	
\\	彼[かれ]に 集中[しゅうちゅう]するよう 注意[ちゅうい]を
\\	よ。			
\\	催促	催促[さいそく]	さいそく	
\\	彼女に本を返すように催促したの。	彼女[かのじょ]に 本[ほん]を 返[かえ]すように 催促[さいそく]したの。	かのじょ に ほん を かえす よう に さいそく した の	
\\	彼女[かのじょ]に 本[ほん]を 返[かえ]すように
\\	したの。			
\\	エチケット	エチケット	エチケット	
\\	エチケットを守ることは大切です。	エチケットを 守[まも]ることは 大切[たいせつ]です。	えちけっと を まもる こと は たいせつ です	
\\	を 守[まも]ることは 大切[たいせつ]です。			
\\	古典	古典[こてん]	こてん	
\\	私は古典を読むのが好きです。	私[わたし]は 古典[こてん]を 読[よ]むのが 好[す]きです。	わたし は こてん を よむ の が すき です	
\\	私[わたし]は
\\	を 読[よ]むのが 好[す]きです。			
\\	辞典	辞典[じてん]	じてん	
\\	知らない言葉を辞典で調べました。	知[し]らない 言葉[ことば]を 辞典[じてん]で 調[しら]べました。	しらない ことば を じてん で しらべました	
\\	知[し]らない 言葉[ことば]を
\\	で 調[しら]べました。			
\\	殊に	殊[こと]に	ことに	
\\	ロックは殊に若者に人気だ。	ロックは 殊[こと]に 若者[わかもの]に 人気[にんき]だ。	ろっく は ことに わかもの に にんき だ	
\\	ロックは
\\	若者[わかもの]に 人気[にんき]だ。			
\\	象徴	象徴[しょうちょう]	しょうちょう	
\\	天皇は日本国の象徴です。	天皇[てんのう]は 日本国[にほんこく]の 象徴[しょうちょう]です。	てんのう は にほんこく の しょうちょう です	
\\	天皇[てんのう]は 日本国[にほんこく]の
\\	です。			
\\	微か	微[かす]か	かすか	
\\	階下から微かな音が聞こえた。	階下[かいか]から 微[かす]かな 音[おと]が 聞[き]こえた。	かいか から かすか な おと が きこえた	
\\	階下[かいか]から
\\	な 音[おと]が 聞[き]こえた。			
\\	ご免	ご 免[めん]	ごめん	
\\	遅れてご免。	遅[おく]れてご 免[めん]。	おくれて ごめん	
\\	遅[おく]れて
\\	許可	許可[きょか]	きょか	
\\	先生に許可をもらって早退しました。	先生[せんせい]に 許可[きょか]をもらって 早退[そうたい]しました。	せんせい に きょか を もらって そうたい しました	
\\	先生[せんせい]に
\\	をもらって 早退[そうたい]しました。			
\\	訓練	訓練[くんれん]	くんれん	
\\	学校で避難訓練がありました。	学校[がっこう]で 避難[ひなん] 訓練[くんれん]がありました。	がっこう で ひなん くんれん が ありました	
\\	学校[がっこう]で 避難[ひなん]
\\	がありました。			
\\	かっと	かっと	かっと	
\\	かっとなって余計なことを言ってしまったな。	かっとなって 余計[よけい]なことを 言[い]ってしまったな。	かっと なって よけい な こと を いって しまった な	
\\	なって 余計[よけい]なことを 言[い]ってしまったな。			
\\	訓読み	訓読[くんよ]み	くんよみ	
\\	漢字には音読みと訓読みがあります。	漢字[かんじ]には 音読[おんよ]みと 訓読[くんよ]みがあります。	かんじ に は おんよみ と くんよみ が あります	
\\	漢字[かんじ]には 音読[おんよ]みと
\\	があります。			
\\	訓	訓[くん]	くん	
\\	漢字の読み方には、音と訓の2通りがあるの。	漢字[かんじ]の 読[よ]み 方[かた]には、 音[おん]と 訓[くん]の 2通[ふたとお]りがあるの。	かんじ の よみかた に は おん と くん の ふたとおり が ある の	
\\	漢字[かんじ]の 読[よ]み 方[かた]には、 音[おん]と
\\	の 2通[ふたとお]りがあるの。			
\\	助詞	助詞[じょし]	じょし	
\\	助詞を変えると文の意味が変わります。	助詞[じょし]を 変[か]えると 文[ぶん]の 意味[いみ]が 変[か]わります。	じょし を かえる と ぶん の いみ が かわります	
\\	を 変[か]えると 文[ぶん]の 意味[いみ]が 変[か]わります。			
\\	形容詞	形容詞[けいようし]	けいようし	
\\	「大きい」は形容詞です。	
\\	大[おお]きい」は 形容詞[けいようし]です。	おおきい は けいようし です	
\\	大[おお]きい」は
\\	です。			
\\	上司	上司[じょうし]	じょうし	
\\	上司に相談してみます。	上司[じょうし]に 相談[そうだん]してみます。	じょうし に そうだん して みます	
\\	に 相談[そうだん]してみます。			
\\	司会	司会[しかい]	しかい	
\\	彼は司会が上手ですね。	彼[かれ]は 司会[しかい]が 上手[じょうず]ですね。	かれ は しかい が じょうず です ね	
\\	彼[かれ]は
\\	が 上手[じょうず]ですね。			
\\	購入	購入[こうにゅう]	こうにゅう	
\\	入学式の後、教科書を購入してください。	入学式[にゅうがくしき]の 後[あと]、 教科書[きょうかしょ]を 購入[こうにゅう]してください。	にゅうがくしき の あと きょうかしょ を こうにゅう して ください	
\\	入学式[にゅうがくしき]の 後[あと]、 教科書[きょうかしょ]を
\\	してください。			
\\	がらがら	がらがら	がらがら	
\\	誰かがガラガラとうがいをしているね。	誰[だれ]かがガラガラとうがいをしているね。	だれか が がらがら と うがい を して いる ね	
\\	誰[だれ]かが
\\	とうがいをしているね。			
\\	講演	講演[こうえん]	こうえん	
\\	彼の講演は評判がいい。	彼[かれ]の 講演[こうえん]は 評判[ひょうばん]がいい。	かれ の こうえん は ひょうばん が いい	
\\	彼[かれ]の
\\	は 評判[ひょうばん]がいい。			
\\	講義	講義[こうぎ]	こうぎ	
\\	彼の講義はとても分かりやすいですね。	彼[かれ]の 講義[こうぎ]はとても 分[わ]かりやすいですね。	かれ の こうぎ は とても わかり やすい です ね	
\\	彼[かれ]の
\\	はとても 分[わ]かりやすいですね。			
\\	休講	休講[きゅうこう]	きゅうこう	
\\	今日のフランス語の講座は休講です。	今日[きょう]のフランス 語[ご]の 講座[こうざ]は 休講[きゅうこう]です。	きょう の ふらんすご の こうざ は きゅうこう です	
\\	今日[きょう]のフランス 語[ご]の 講座[こうざ]は
\\	です。			
\\	医師	医師[いし]	いし	
\\	医師に入院を勧められたの。	医師[いし]に 入院[にゅういん]を 勧[すす]められたの。	いし に にゅういん を すすめられた の	
\\	に 入院[にゅういん]を 勧[すす]められたの。			
\\	講師	講師[こうし]	こうし	
\\	専門家を講師に招いた。	専門家[せんもんか]を 講師[こうし]に 招[まね]いた。	せんもんか を こうし に まねいた	
\\	専門家[せんもんか]を
\\	に 招[まね]いた。			
\\	技師	技師[ぎし]	ぎし	
\\	彼はレントゲン技師です。	彼[かれ]はレントゲン 技師[ぎし]です。	かれ は れんとげん ぎし です	
\\	彼[かれ]はレントゲン
\\	です。			
\\	師走	師走[しわす]	しわす	
\\	師走に入ると忙しくなります。	師走[しわす]に 入[はい]ると 忙[いそが]しくなります。	しわす に はいる と いそがしく なります	
\\	に 入[はい]ると 忙[いそが]しくなります。			
\\	かんかん	かんかん	かんかん	
\\	彼女はかんかんに怒っています。	彼女[かのじょ]はかんかんに 怒[おこ]っています。	かのじょ は かんかん に おこって います	
\\	彼女[かのじょ]は
\\	に 怒[おこ]っています。			
\\	お手伝い	お 手伝[てつだ]い	おてつだい	
\\	何かお手伝いしましょうか。	何[なに]かお 手伝[てつだ]いしましょうか。	なに か おてつだい しましょう か	
\\	何[なに]か
\\	しましょうか。			
\\	小鳥	小鳥[ことり]	ことり	
\\	誕生日に小鳥を買ってもらいました。	誕生日[たんじょうび]に 小鳥[ことり]を 買[か]ってもらいました。	たんじょうび に ことり を かって もらいました	
\\	誕生日[たんじょうび]に
\\	を 買[か]ってもらいました。			
\\	大声	大声[おおごえ]	おおごえ	
\\	私たちは大声で歌を歌ったの。	私[わたし]たちは 大声[おおごえ]で 歌[うた]を 歌[うた]ったの。	わたしたち は おおごえ で うた を うたった の	
\\	私[わたし]たちは
\\	で 歌[うた]を 歌[うた]ったの。			
\\	歌声	歌声[うたごえ]	うたごえ	
\\	校舎から歌声が聞こえて来たよ。	校舎[こうしゃ]から 歌声[うたごえ]が 聞[き]こえて 来[き]たよ。	こうしゃ から うたごえ が きこえて きた よ	
\\	校舎[こうしゃ]から
\\	が 聞[き]こえて 来[き]たよ。			
\\	騒ぎ	騒[さわ]ぎ	さわぎ	
\\	一体何の騒ぎですか。	一体何[いったい なん]の 騒[さわ]ぎですか。	いったい なん の さわぎ です か	
\\	一体何[いったい なん]の
\\	ですか。			
\\	騒がしい	騒[さわ]がしい	さわがしい	
\\	何だか表が騒がしい。	何[なん]だか 表[おもて]が 騒[さわ]がしい。	なんだか おもて が さわがしい	
\\	何[なん]だか 表[おもて]が
\\	飼う	飼[か]う	かう	
\\	ペットを飼ったことはありますか。	ペットを 飼[か]ったことはありますか。	ぺっと を かった こと は あります か	
\\	ペットを
\\	ことはありますか。			
\\	これまで	これまで	これまで	
\\	これまでの私とは違うんです。	これまでの 私[わたし]とは 違[ちが]うんです。	これまで の わたし と は ちがうん です	
\\	の 私[わたし]とは 違[ちが]うんです。			
\\	刺さる	刺[さ]さる	ささる	
\\	靴の底に釘が刺さってしまったんだ。	靴[くつ]の 底[そこ]に 釘[くぎ]が 刺[さ]さってしまったんだ。	くつ の そこ に くぎ が ささって しまった ん だ	
\\	靴[くつ]の 底[そこ]に 釘[くぎ]が
\\	んだ。			
\\	急激	急激[きゅうげき]	きゅうげき	
\\	これから高齢化が急激に進みます。	これから 高齢化[こうれいか]が 急激[きゅうげき]に 進[すす]みます。	これから こうれいか が きゅうげき に すすみます	
\\	これから 高齢化[こうれいか]が
\\	に 進[すす]みます。			
\\	刺激	刺激[しげき]	しげき	
\\	そのクイズ番組は脳を刺激するね。	そのクイズ 番組[ばんぐみ]は 脳[のう]を 刺激[しげき]するね。	その くいずばんぐみ は のう を しげき する ね	
\\	そのクイズ 番組[ばんぐみ]は 脳[のう]を
\\	するね。			
\\	感激	感激[かんげき]	かんげき	
\\	感激して泣いてしまいました。	感激[かんげき]して 泣[な]いてしまいました。	かんげき して ないて しまいました	
\\	して 泣[な]いてしまいました。			
\\	興味	興味[きょうみ]	きょうみ	
\\	私は歴史に興味があります。	私[わたし]は 歴史[れきし]に 興味[きょうみ]があります。	わたし は れきし に きょうみ が あります	
\\	私[わたし]は 歴史[れきし]に
\\	があります。			
\\	興奮	興奮[こうふん]	こうふん	
\\	彼女の優勝に感激し興奮しました。	彼女[かのじょ]の 優勝[ゆうしょう]に 感激[かんげき]し 興奮[こうふん]しました。	かのじょ の ゆうしょう に かんげき し こうふん しました	
\\	彼女[かのじょ]の 優勝[ゆうしょう]に 感激[かんげき]し
\\	しました。			
\\	驚き	驚[おどろ]き	おどろき	
\\	彼女は驚きを隠せませんでしたよ。	彼女[かのじょ]は 驚[おどろ]きを 隠[かく]せませんでしたよ。	かのじょ は おどろき を かくせませんでした よ	
\\	彼女[かのじょ]は
\\	を 隠[かく]せませんでしたよ。			
\\	驚かす	驚[おどろ]かす	おどろかす	
\\	あなたを驚かす話があります。	あなたを 驚[おどろ]かす 話[はなし]があります。	あなた を おどろかす はなし が あります	
\\	あなたを
\\	話[はなし]があります。			
\\	じめじめ	じめじめ	じめじめ	
\\	梅雨時はじめじめする。	梅雨時[つゆどき]はじめじめする。	つゆどき は じめじめ する	
\\	梅雨時[つゆどき]は
\\	する。			
\\	至る	至[いた]る	いたる	
\\	ようやく結論に至ったようね	ようやく 結論[けつろん]に 至[いた]ったようね	ようやく けつろん に いたった よう ね	
\\	ようやく 結論[けつろん]に
\\	ようね			
\\	至る所	至[いた]る 所[ところ]	いたるところ	
\\	コンビニは至る所にあります。	コンビニは 至[いた]る 所[ところ]にあります。	こんびに は いたるところ に あります	
\\	コンビニは
\\	にあります。			
\\	至急	至急[しきゅう]	しきゅう	
\\	会社から「至急」との連絡があったんだ。	会社[かいしゃ]から
\\	至急[しきゅう]」との 連絡[れんらく]があったんだ。	かいしゃ から しきゅう と の れんらく が あった ん だ	
\\	会社[かいしゃ]から
\\	との 連絡[れんらく]があったんだ。			
\\	一致	一致[いっち]	いっち	
\\	皆の意見が一致しました。	皆[みんな]の 意見[いけん]が 一致[いっち]しました。	みんな の いけん が いっち しました	
\\	皆[みんな]の 意見[いけん]が
\\	しました。			
\\	致す	致[いた]す	いたす	
\\	私からご連絡致します。	私[わたし]からご 連絡[れんらく] 致[いた]します。	わたし から ごれんらく いたします	
\\	私[わたし]からご 連絡[れんらく]
\\	傾く	傾[かたむ]く	かたむく	
\\	お日様が西に傾きましたね。	お 日様[ひさま]が 西[にし]に 傾[かたむ]きましたね。	おひさま が にし に かたむきました ね	
\\	お 日様[ひさま]が 西[にし]に
\\	ね。			
\\	傾ける	傾[かたむ]ける	かたむける	
\\	彼女は首を少し傾けて笑うの。	彼女[かのじょ]は 首[くび]を 少[すこ]し 傾[かたむ]けて 笑[わら]うの。	かのじょ は くび を すこし かたむけて わらう の	
\\	彼女[かのじょ]は 首[くび]を 少[すこ]し
\\	笑[わら]うの。			
\\	じろじろ	じろじろ	じろじろ	
\\	人をじろじろ見るものではありません。	人[ひと]をじろじろ 見[み]るものではありません。	ひと を じろじろ みる もの で は ありません	
\\	人[ひと]を
\\	見[み]るものではありません。			
\\	坂	坂[さか]	さか	
\\	この坂を上るのはすごくきついね。	この 坂[さか]を 上[のぼ]るのはすごくきついね。	この さか を のぼる の は すごく きつい ね	
\\	この
\\	を 上[のぼ]るのはすごくきついね。			
\\	狭まる	狭[せば]まる	せばまる	
\\	ここから道の幅が狭まっています。	ここから 道[みち]の 幅[はば]が 狭[せば]まっています。	ここ から みち の はば が せばまって います	
\\	ここから 道[みち]の 幅[はば]が
\\	狭める	狭[せば]める	せばめる	
\\	もっと範囲を狭めて探しましょう	もっと 範囲[はんい]を 狭[せば]めて 探[さが]しましょう	もっと はんい を せばめて さがしましょう	
\\	もっと 範囲[はんい]を
\\	探[さが]しましょう			
\\	徐行	徐行[じょこう]	じょこう	
\\	この先は徐行して下さい。	この 先[さき]は 徐行[じょこう]して 下[くだ]さい。	この さき は じょこう して ください	
\\	この 先[さき]は
\\	して 下[くだ]さい。			
\\	硬さ	硬[かた]さ	かたさ	
\\	理科の時間に石の硬さを比べたよ。	理科[りか]の 時間[じかん]に 石[いし]の 硬[かた]さを 比[くら]べたよ。	りか の じかん に いし の かたさ を くらべた よ	
\\	理科[りか]の 時間[じかん]に 石[いし]の
\\	を 比[くら]べたよ。			
\\	柔軟	柔軟[じゅうなん]	じゅうなん	
\\	彼は柔軟に対応をした。	彼[かれ]は 柔軟[じゅうなん]に 対応[たいおう]をした。	かれ は じゅうなん に たいおう を した	
\\	彼[かれ]は
\\	に 対応[たいおう]をした。			
\\	緊急	緊急[きんきゅう]	きんきゅう	
\\	緊急事態が発生した。	緊急[きんきゅう] 事態[じたい]が 発生[はっせい]した。	きんきゅう じたい が はっせい した	
\\	事態[じたい]が 発生[はっせい]した。			
\\	あり	あり	あり	
\\	ありの群れが砂糖にたかっている。	ありの 群[む]れが 砂糖[さとう]にたかっている。	あり の むれ が さとう に たかっている 。	
\\	の 群[む]れが 砂糖[さとう]にたかっている。			
\\	緊張	緊張[きんちょう]	きんちょう	
\\	彼は緊張していたな。	彼[かれ]は 緊張[きんちょう]していたな。	かれ は きんちょう して いた な	
\\	彼[かれ]は
\\	していたな。			
\\	拡張	拡張[かくちょう]	かくちょう	
\\	その会社は店舗を拡張していますね。	その 会社[かいしゃ]は 店舗[てんぽ]を 拡張[かくちょう]していますね。	その かいしゃ は てんぽ を かくちょう して います ね	
\\	その 会社[かいしゃ]は 店舗[てんぽ]を
\\	していますね。			
\\	出張	出張[しゅっちょう]	しゅっちょう	
\\	部長は京都に出張中です。	部長[ぶちょう]は 京都[きょうと]に 出張[しゅっちょう] 中[ちゅう]です。	ぶちょう は きょうと に しゅっちょうちゅう です	
\\	部長[ぶちょう]は 京都[きょうと]に
\\	中[ちゅう]です。			
\\	衝突	衝突[しょうとつ]	しょうとつ	
\\	そこの角で車同士が衝突したのよ。	そこの 角[かど]で 車同士[くるま どうし]が 衝突[しょうとつ]したのよ。	そこ の かど で くるま どうし が しょうとつ した の よ	
\\	そこの 角[かど]で 車同士[くるま どうし]が
\\	したのよ。			
\\	煙突	煙突[えんとつ]	えんとつ	
\\	工場の煙突から煙が上がっている。	工場[こうじょう]の 煙突[えんとつ]から 煙[けむり]が 上[あ]がっている。	こうじょう の えんとつ から けむり が あがって いる	
\\	工場[こうじょう]の
\\	から 煙[けむり]が 上[あ]がっている。			
\\	避ける	避[さ]ける	さける	
\\	彼女は昨日から私のことを避けているようなんだ。	彼女[かのじょ]は 昨日[きのう]から 私[わたし]のことを 避[さ]けているようなんだ。	かのじょ は きのう から わたし の こと を さけて いる よう なん だ	
\\	彼女[かのじょ]は 昨日[きのう]から 私[わたし]のことを
\\	ようなんだ。			
\\	軍事	軍事[ぐんじ]	ぐんじ	
\\	彼は軍事に関わる仕事を続けてきたんだ。	彼[かれ]は 軍事[ぐんじ]に 関[かか]わる 仕事[しごと]を 続[つづ]けてきたんだ。	かれ は ぐんじ に かかわる しごと を つづけて きた ん だ	
\\	彼[かれ]は
\\	に 関[かか]わる 仕事[しごと]を 続[つづ]けてきたんだ。			
\\	軍	軍[ぐん]	ぐん	
\\	怪我人は軍の病院に運ばれたよ。	怪我人[けがにん]は 軍[ぐん]の 病院[びょういん]に 運[はこ]ばれたよ。	けがにん は ぐん の びょういん に はこばれた よ	
\\	怪我人[けがにん]は
\\	の 病院[びょういん]に 運[はこ]ばれたよ。			
\\	あいづち	あいづち	あいづち	
\\	彼の話に私はあいづちを打ったの。	彼[かれ]の 話[はなし]に 私[わたし]はあいづちを 打[う]ったの。	かれ の はなし に わたし は あいづち を うった の	
\\	彼[かれ]の 話[はなし]に 私[わたし]は
\\	を 打[う]ったの。			
\\	自衛隊	自衛隊[じえいたい]	じえいたい	
\\	自衛隊がイラクに派遣されたわ。	自衛隊[じえいたい]がイラクに 派遣[はけん]されたわ。	じえいたい が いらく に はけん された わ	
\\	がイラクに 派遣[はけん]されたわ。			
\\	軍隊	軍隊[ぐんたい]	ぐんたい	
\\	彼は軍隊に入ったよ。	彼[かれ]は 軍隊[ぐんたい]に 入[はい]ったよ。	かれ は ぐんたい に はいった よ	
\\	彼[かれ]は
\\	に 入[はい]ったよ。			
\\	核	核[かく]	かく	
\\	核戦争は絶対に防ぐべきよ。	核[かく] 戦争[せんそう]は 絶対[ぜったい]に 防[ふせ]ぐべきよ。	かくせんそう は ぜったい に ふせぐ べき よ	
\\	戦争[せんそう]は 絶対[ぜったい]に 防[ふせ]ぐべきよ。			
\\	結核	結核[けっかく]	けっかく	
\\	昔は結核でたくさんの人が亡くなった。	昔[むかし]は 結核[けっかく]でたくさんの 人[ひと]が 亡[な]くなった。	むかし は けっかく で たくさん の ひと が なくなった	
\\	昔[むかし]は
\\	でたくさんの 人[ひと]が 亡[な]くなった。			
\\	専攻	専攻[せんこう]	せんこう	
\\	大学では物理を専攻していました。	大学[だいがく]では 物理[ぶつり]を 専攻[せんこう]していました。	だいがく で は ぶつり を せんこう して いました	
\\	大学[だいがく]では 物理[ぶつり]を
\\	していました。			
\\	攻める	攻[せ]める	せめる	
\\	彼は積極的に攻めたが勝てなかったな。	彼[かれ]は 積極的[せっきょくてき]に 攻[せ]めたが 勝[か]てなかったな。	かれ は せっきょくてき に せめた が かてなかった な	
\\	彼[かれ]は 積極的[せっきょくてき]に
\\	が 勝[か]てなかったな。			
\\	撃つ	撃[う]つ	うつ	
\\	彼は拳銃で撃たれたわ。	彼[かれ]は 拳銃[けんじゅう]で 撃[う]たれたわ。	かれ は けんじゅう で うたれた わ	
\\	彼[かれ]は 拳銃[けんじゅう]で
\\	わ。			
\\	かちかち	かちかち	かちかち	
\\	池がかちかちに凍っています。	池[いけ]がかちかちに 凍[こお]っています。	いけ が かちかち に こおって います	
\\	池[いけ]が
\\	に 凍[こお]っています。			
\\	暴れる	暴[あば]れる	あばれる	
\\	彼は悪酔いして暴れたんだ。	彼[かれ]は 悪酔[わるよ]いして 暴[あば]れたんだ。	かれ は わるよい して あばれた ん だ	
\\	彼[かれ]は 悪酔[わるよ]いして
\\	んだ。			
\\	襲う	襲[おそ]う	おそう	
\\	山で登山客が熊に襲われたよ。	山[やま]で 登山客[とざん きゃく]が 熊[くま]に 襲[おそ]われたよ。	やま で とざん きゃく が くま に おそわれた よ	
\\	山[やま]で 登山客[とざん きゃく]が 熊[くま]に
\\	よ。			
\\	絶対	絶対[ぜったい]	ぜったい	
\\	そんなことは絶対できないよ。	そんなことは 絶対[ぜったい]できないよ。	そんな こと は ぜったい できない よ	
\\	そんなことは
\\	できないよ。			
\\	気絶	気絶[きぜつ]	きぜつ	
\\	彼女は驚いて気絶してしまったの。	彼女[かのじょ]は 驚[おどろ]いて 気絶[きぜつ]してしまったの。	かのじょ は おどろいて きぜつ して しまった の	
\\	彼女[かのじょ]は 驚[おどろ]いて
\\	してしまったの。			
\\	消防	消防[しょうぼう]	しょうぼう	
\\	消防士は勇敢でなければならない。	消防[しょうぼう] 士[し]は 勇敢[ゆうかん]でなければならない。	しょうぼうし は ゆうかん で なけれ ば ならない	
\\	士[し]は 勇敢[ゆうかん]でなければならない。			
\\	嫌う	嫌[きら]う	きらう	
\\	父はラッシュアワーを嫌っています。	父[ちち]はラッシュアワーを 嫌[きら]っています。	ちち は らっしゅあわー を きらって います	
\\	父[ちち]はラッシュアワーを
\\	嫌がる	嫌[いや]がる	いやがる	
\\	彼はタバコの煙を嫌がるの。	彼[かれ]はタバコの 煙[けむり]を 嫌[いや]がるの。	かれ は たばこ の けむり を いやがる の	
\\	彼[かれ]はタバコの 煙[けむり]を
\\	の。			
\\	くたくた	くたくた	くたくた	
\\	今日はたくさん歩いてくたくたです。	今日[きょう]はたくさん 歩[ある]いてくたくたです。	きょう は たくさん あるいて くたくた です	
\\	今日[きょう]はたくさん 歩[ある]いて
\\	です。			
\\	機嫌	機嫌[きげん]	きげん	
\\	彼女は大変機嫌がいいね。	彼女[かのじょ]は 大変[たいへん] 機嫌[きげん]がいいね。	かのじょ は たいへん きげん が いい ね	
\\	彼女[かのじょ]は 大変[たいへん]
\\	がいいね。			
\\	好き嫌い	好[す]き 嫌[きら]い	すききらい	
\\	食べ物の好き嫌いは特にありません。	食[た]べ 物[もの]の 好[す]き 嫌[きら]いは 特[とく]にありません。	たべもの の すききらい は とくに ありません	
\\	食[た]べ 物[もの]の
\\	は 特[とく]にありません。			
\\	抗議	抗議[こうぎ]	こうぎ	
\\	彼の発言に対してたくさんの抗議があったよ。	彼[かれ]の 発言[はつげん]に 対[たい]してたくさんの 抗議[こうぎ]があったよ。	かれ の はつげん に たいして たくさん の こうぎ が あった よ	
\\	彼[かれ]の 発言[はつげん]に 対[たい]してたくさんの
\\	があったよ。			
\\	素敵	素敵[すてき]	すてき	
\\	素敵なプレゼントをありがとう。	素敵[すてき]なプレゼントをありがとう。	すてき な ぷれぜんと を ありがとう	
\\	なプレゼントをありがとう。			
\\	脅かす	脅[おど]かす	おどかす	
\\	脅かさないでよ。	脅[おど]かさないでよ。	おどかさない で よ	
\\	よ。			
\\	権威	権威[けんい]	けんい	
\\	博士はその道の権威です。	博士[はかせ]はその 道[みち]の 権威[けんい]です。	はかせ は そのみち の けんい です	
\\	博士[はかせ]はその 道[みち]の
\\	です。			
\\	威張る	威張[いば]る	いばる	
\\	彼は威張ってなんかいません。	彼[かれ]は 威張[いば]ってなんかいません。	かれ は いばって なんか いません	
\\	彼[かれ]は
\\	なんかいません。			
\\	情勢	情勢[じょうせい]	じょうせい	
\\	私は世界情勢を知るために毎日ニュースを見るわ。	私[わたし]は 世界[せかい] 情勢[じょうせい]を 知[し]るために 毎日[まいにち]ニュースを 見[み]るわ。	わたし は せかい じょうせい を しる ため に まいにち にゅーす を みる わ	
\\	私[わたし]は 世界[せかい]
\\	を 知[し]るために 毎日[まいにち]ニュースを 見[み]るわ。			
\\	こしょう	こしょう	こしょう	
\\	こしょうを入れ過ぎてスープが辛い。	こしょうを 入[い]れ 過[す]ぎてスープが 辛[から]い。	こしょう を いれすぎて すーぷ が からい	
\\	を 入[い]れ 過[す]ぎてスープが 辛[から]い。			
\\	勢力	勢力[せいりょく]	せいりょく	
\\	この頃は、ローマが勢力を伸ばしていました。	この 頃[ころ]は、ローマが 勢力[せいりょく]を 伸[の]ばしていました。	この ころ は ろーま が せいりょく を のばして いました	
\\	この 頃[ころ]は、ローマが
\\	を 伸[の]ばしていました。			
\\	勢い	勢[いきお]い	いきおい	
\\	そのチームには勢いがありますね。	そのチームには 勢[いきお]いがありますね。	その ちーむ に は いきおい が あります ね	
\\	そのチームには
\\	がありますね。			
\\	恐れ	恐[おそ]れ	おそれ	
\\	叔父には心臓病の恐れがあります。	叔父[おじ]には 心臓病[しんぞうびょう]の 恐[おそ]れがあります。	おじ に は しんぞうびょう の おそれ が あります	
\\	叔父[おじ]には 心臓病[しんぞうびょう]の
\\	があります。			
\\	恐れる	恐[おそ]れる	おそれる	
\\	彼は受験の失敗を恐れています。	彼[かれ]は 受験[じゅけん]の 失敗[しっぱい]を 恐[おそ]れています。	かれ は じゅけん の しっぱい を おそれて います	
\\	彼[かれ]は 受験[じゅけん]の 失敗[しっぱい]を
\\	恐らく	恐[おそ]らく	おそらく	
\\	明日は恐らく晴れるでしょう。	明日[あした]は 恐[おそ]らく 晴[は]れるでしょう。	あした は おそらく はれる でしょう	
\\	明日[あした]は
\\	晴[は]れるでしょう。			
\\	怖がる	怖[こわ]がる	こわがる	
\\	彼女はクモを怖がります。	彼女[かのじょ]はクモを 怖[こわ]がります。	かのじょ は くも を こわがります	
\\	彼女[かのじょ]はクモを
\\	巨大	巨大[きょだい]	きょだい	
\\	あの巨大な建物は博物館です。	あの 巨大[きょだい]な 建物[たてもの]は 博物館[はくぶつかん]です。	あの きょだい な たてもの は はくぶつかん です	
\\	あの
\\	な 建物[たてもの]は 博物館[はくぶつかん]です。			
\\	すべすべ	すべすべ	すべすべ	
\\	彼女の肌はすべすべしているね。	彼女[かのじょ]の 肌[はだ]はすべすべしているね。	かのじょ の はだ は すべすべ して いる ね	
\\	彼女[かのじょ]の 肌[はだ]は
\\	しているね。			
\\	拒否	拒否[きょひ]	きょひ	
\\	彼女は出席を拒否した。	彼女[かのじょ]は 出席[しゅっせき]を 拒否[きょひ]した。	かのじょ は しゅっせき を きょひ した	
\\	彼女[かのじょ]は 出席[しゅっせき]を
\\	した。			
\\	系列	系列[けいれつ]	けいれつ	
\\	高校卒業後、系列の大学に進みました。	高校卒業後[こうこう そつぎょう ご]、 系列[けいれつ]の 大学[だいがく]に 進[すす]みました。	こうこう そつぎょう ご けいれつ の だいがく に すすみました	
\\	高校卒業後[こうこう そつぎょう ご]、
\\	の 大学[だいがく]に 進[すす]みました。			
\\	系統	系統[けいとう]	けいとう	
\\	電気系統を図面で確かめました。	電気[でんき] 系統[けいとう]を 図面[ずめん]で 確[たし]かめました。	でんき けいとう を ずめん で たしかめました	
\\	電気[でんき]
\\	を 図面[ずめん]で 確[たし]かめました。			
\\	子孫	子孫[しそん]	しそん	
\\	彼は織田信長の子孫だよ。	彼[かれ]は 織田信長[おだのぶなが]の 子孫[しそん]だよ。	かれ は おだのぶなが の しそん だ よ	
\\	彼[かれ]は 織田信長[おだのぶなが]の
\\	だよ。			
\\	絹	絹[きぬ]	きぬ	
\\	このシャツは絹でできています。	このシャツは 絹[きぬ]でできています。	この しゃつ は きぬ で できて います	
\\	このシャツは
\\	でできています。			
\\	維持	維持[いじ]	いじ	
\\	彼女は何とか健康を維持しているわね。	彼女[かのじょ]は 何[なん]とか 健康[けんこう]を 維持[いじ]しているわね。	かのじょ は なんとか けんこう を いじ して いる わ ね	
\\	彼女[かのじょ]は 何[なん]とか 健康[けんこう]を
\\	しているわね。			
\\	紳士	紳士[しんし]	しんし	
\\	身なりのいい紳士が話しかけてきたの。	身[み]なりのいい 紳士[しんし]が 話[はな]しかけてきたの。	みなり の いい しんし が はなしかけて きた の	
\\	身[み]なりのいい
\\	が 話[はな]しかけてきたの。			
\\	せっかく	せっかく	せっかく	
\\	せっかく来たんだからゆっくりして行きなさい。	せっかく 来[き]たんだからゆっくりして 行[い]きなさい。	せっかく きたん だから ゆっくり して いきなさい	
\\	来[き]たんだからゆっくりして 行[い]きなさい。			
\\	偉大	偉大[いだい]	いだい	
\\	彼は偉大な事業をなしとげました。	彼[かれ]は 偉大[いだい]な 事業[じぎょう]をなしとげました。	かれ は いだい な じぎょう を なしとげ ました	
\\	彼[かれ]は
\\	な 事業[じぎょう]をなしとげました。			
\\	刀	刀[かたな]	かたな	
\\	日本映画で刀を見た。	日本映画[にほん えいが]で 刀[かたな]を 見[み]た。	にほん えいが で かたな を みた	
\\	日本映画[にほん えいが]で
\\	を 見[み]た。			
\\	孤独	孤独[こどく]	こどく	
\\	彼は孤独な人生を送っていたんだ。	彼[かれ]は 孤独[こどく]な 人生[じんせい]を 送[おく]っていたんだ。	かれ は こどく な じんせい を おくって いた ん だ	
\\	彼[かれ]は
\\	な 人生[じんせい]を 送[おく]っていたんだ。			
\\	事柄	事柄[ことがら]	ことがら	
\\	これまで話し合った事柄をまとめてください。	これまで 話[はな]し 合[あ]った 事柄[ことがら]をまとめてください。	これ まで はなしあった ことがら を まとめて ください	
\\	これまで 話[はな]し 合[あ]った
\\	をまとめてください。			
\\	柄	柄[がら]	がら	
\\	彼は犬の柄の
\\	シャツを着ているよ。	彼[かれ]は 犬[いぬ]の 柄[がら]の
\\	シャツを 着[き]ているよ。	かれ は いぬ の がら の 
\\	しゃつ を きて いる よ	
\\	彼[かれ]は 犬[いぬ]の
\\	の
\\	シャツを 着[き]ているよ。			
\\	柄	柄[え]	え	
\\	この傘は柄が丈夫だな。	この 傘[かさ]は 柄[え]が 丈夫[じょうぶ]だな。	この かさ は え が じょうぶ だ な	
\\	この 傘[かさ]は
\\	が 丈夫[じょうぶ]だな。			
\\	枝	枝[えだ]	えだ	
\\	強風で木の枝が折れた。	強風[きょうふう]で 木[き]の 枝[えだ]が 折[お]れた。	きょうふう で き の えだ が おれた	
\\	強風[きょうふう]で 木[き]の
\\	が 折[お]れた。			
\\	あきれる	あきれる	あきれる	
\\	彼の頑固さにはあきれました。	彼[かれ]の 頑固[がんこ]さにはあきれました。	かれ の がんこさ に は あきれました	
\\	彼[かれ]の 頑固[がんこ]さには
\\	枯れる	枯[か]れる	かれる	
\\	花瓶の花が枯れました。	花瓶[かびん]の 花[はな]が 枯[か]れました。	かびん の はな が かれました	
\\	花瓶[かびん]の 花[はな]が
\\	木枯らし	木枯[こが]らし	こがらし	
\\	外は木枯らしが吹いているよ。	外[そと]は 木枯[こが]らしが 吹[ふ]いているよ。	そと は こがらし が ふいて いる よ	
\\	外[そと]は
\\	が 吹[ふ]いているよ。			
\\	詩	詩[し]	し	
\\	彼女の詩は世界中で有名になったんだ。	彼女[かのじょ]の 詩[し]は 世界中[せかいじゅう]で 有名[ゆうめい]になったんだ。	かのじょ の し は せかいじゅう で ゆうめい に なった ん だ	
\\	彼女[かのじょ]の
\\	は 世界中[せかいじゅう]で 有名[ゆうめい]になったんだ。			
\\	詩人	詩人[しじん]	しじん	
\\	彼は優れた詩人です。	彼[かれ]は 優[すぐ]れた 詩人[しじん]です。	かれ は すぐれた しじん です	
\\	彼[かれ]は 優[すぐ]れた
\\	です。			
\\	誠実	誠実[せいじつ]	せいじつ	
\\	彼はとても誠実な人です。	彼[かれ]はとても 誠実[せいじつ]な 人[ひと]です。	かれ は とても せいじつ な ひと です	
\\	彼[かれ]はとても
\\	な 人[ひと]です。			
\\	盛ん	盛[さか]ん	さかん	
\\	その都市は商業が盛んよ。	その 都市[とし]は 商業[しょうぎょう]が 盛[さか]んよ。	その とし は しょうぎょう が さかん よ	
\\	その 都市[とし]は 商業[しょうぎょう]が
\\	よ。			
\\	謙遜	謙遜[けんそん]	けんそん	
\\	そんなに謙遜しなくてもいい。	そんなに 謙遜[けんそん]しなくてもいい。	そんな に けんそん しなくて も いい	
\\	そんなに
\\	しなくてもいい。			
\\	兼ねる	兼[か]ねる	かねる	
\\	彼女は秘書と事務の担当を兼ねています。	彼女[かのじょ]は 秘書[ひしょ]と 事務[じむ]の 担当[たんとう]を 兼[か]ねています。	かのじょ は ひしょ と じむ の たんとう を かねて います	
\\	彼女[かのじょ]は 秘書[ひしょ]と 事務[じむ]の 担当[たんとう]を
\\	あだな	あだな	あだな	
\\	彼のあだなは「りき」です。	彼[かれ]のあだなは「りき」です。	かれ の あだな は 
\\	りき 
\\	です 。	
\\	彼[かれ]の
\\	は「りき」です。			
\\	鋭い	鋭[するど]い	するどい	
\\	鋭い刃物で指を怪我した。	鋭[するど]い 刃物[はもの]で 指[ゆび]を 怪我[けが]した。	するどい はもの で ゆび を けが した	
\\	刃物[はもの]で 指[ゆび]を 怪我[けが]した。			
\\	鎖	鎖[くさり]	くさり	
\\	犬を鎖でつなぎました。	犬[いぬ]を 鎖[くさり]でつなぎました。	いぬ を くさり で つなぎました	
\\	犬[いぬ]を
\\	でつなぎました。			
\\	鑑賞	鑑賞[かんしょう]	かんしょう	
\\	今夜はオペラ鑑賞に行きます。	今夜[こんや]はオペラ 鑑賞[かんしょう]に 行[い]きます。	こんや は おぺら かんしょう に いきます	
\\	今夜[こんや]はオペラ
\\	に 行[い]きます。			
\\	印鑑	印鑑[いんかん]	いんかん	
\\	ここに受け取りの印鑑をお願いします。	ここに 受[う]け 取[と]りの 印鑑[いんかん]をお 願[ねが]いします。	ここ に うけとり の いんかん を おねがい します	
\\	ここに 受[う]け 取[と]りの
\\	をお 願[ねが]いします。			
\\	鐘	鐘[かね]	かね	
\\	この鐘の音はとてもきれいですね。	この 鐘[かね]の 音[ね]はとてもきれいですね。	この かね の ね は とても きれい です ね	
\\	この
\\	の 音[ね]はとてもきれいですね。			
\\	寿命	寿命[じゅみょう]	じゅみょう	
\\	亀の寿命は長いんだ。	亀[かめ]の 寿命[じゅみょう]は 長[なが]いんだ。	かめ の じゅみょう は ながい ん だ	
\\	亀[かめ]の
\\	は 長[なが]いんだ。			
\\	海水浴	海水浴[かいすいよく]	かいすいよく	
\\	夏休みには海水浴に行きます。	夏休[なつやす]みには 海水浴[かいすいよく]に 行[い]きます。	なつやすみ に は かいすいよく に いきます	
\\	夏休[なつやす]みには
\\	に 行[い]きます。			
\\	あふれる	あふれる	あふれる	
\\	バスタブからお湯があふれました。	バスタブからお 湯[ゆ]があふれました。	ばすたぶ から おゆ が あふれました	
\\	バスタブからお 湯[ゆ]が
\\	沿岸	沿岸[えんがん]	えんがん	
\\	今日は沿岸の波が荒いでしょう。	今日[きょう]は 沿岸[えんがん]の 波[なみ]が 荒[あら]いでしょう。	きょう は えんがん の なみ が あらい でしょう	
\\	今日[きょう]は
\\	の 波[なみ]が 荒[あら]いでしょう。			
\\	砂浜	砂浜[すなはま]	すなはま	
\\	朝早く砂浜を散歩しました。	朝早[あさ はや]く 砂浜[すなはま]を 散歩[さんぽ]しました。	あさ はやく すなはま を さんぽ しました	
\\	朝早[あさ はや]く
\\	を 散歩[さんぽ]しました。			
\\	沖	沖[おき]	おき	
\\	沖に小島が見えます。	沖[おき]に 小島[こじま]が 見[み]えます。	おき に こじま が みえます	
\\	に 小島[こじま]が 見[み]えます。			
\\	泉	泉[いずみ]	いずみ	
\\	森の中にきれいな泉があるの。	森[もり]の 中[なか]にきれいな 泉[いずみ]があるの。	もり の なか に きれい な いずみ が ある の	
\\	森[もり]の 中[なか]にきれいな
\\	があるの。			
\\	温泉	温泉[おんせん]	おんせん	
\\	ここの温泉はよく効くそうです。	ここの 温泉[おんせん]はよく 効[き]くそうです。	ここ の おんせん は よく きく そう です	
\\	ここの
\\	はよく 効[き]くそうです。			
\\	澄む	澄[す]む	すむ	
\\	山の空気は澄んでいますね。	山[やま]の 空気[くうき]は 澄[す]んでいますね。	やま の くうき は すんで います ね	
\\	山[やま]の 空気[くうき]は
\\	ね。			
\\	叫び	叫[さけ]び	さけび	
\\	彼女の心の叫びに誰も気付かなかったよ。	彼女[かのじょ]の 心[こころ]の 叫[さけ]びに 誰[だれ]も 気付[きづ]かなかったよ。	かのじょ の こころ の さけび に だれ も きづかなかった よ	
\\	彼女[かのじょ]の 心[こころ]の
\\	に 誰[だれ]も 気付[きづ]かなかったよ。			
\\	いか	いか	いか	
\\	私はイカの刺し身が大好きです。	私[わたし]はイカの 刺[さ]し 身[み]が 大好[だいす]きです。	わたし は いか の さしみ が だいすき です	
\\	私[わたし]は
\\	の 刺[さ]し 身[み]が 大好[だいす]きです。			
\\	叫ぶ	叫[さけ]ぶ	さけぶ	
\\	彼女は助けを求めて大声で叫んだの。	彼女[かのじょ]は 助[たす]けを 求[もと]めて 大声[おおごえ]で 叫[さけ]んだの。	かのじょ は たすけ を もとめて おおごえ で さけんだ の	
\\	彼女[かのじょ]は 助[たす]けを 求[もと]めて 大声[おおごえ]で
\\	の。			
\\	喫煙	喫煙[きつえん]	きつえん	
\\	ここでは喫煙できません。	ここでは 喫煙[きつえん]できません。	ここ で は きつえん できません 。	
\\	ここでは
\\	できません。			
\\	懸ける	懸[か]ける	かける	
\\	彼は仕事に命を懸けているの。	彼[かれ]は 仕事[しごと]に 命[いのち]を 懸[か]けているの。	かれ は しごと に いのち を かけている の 。	
\\	彼[かれ]は 仕事[しごと]に 命[いのち]を
\\	の。			
\\	恩	恩[おん]	おん	
\\	このご恩は一生忘れません。	このご 恩[おん]は 一生忘[いっしょう わす]れません。	この ごおん は いっしょう わすれません	
\\	このご
\\	は 一生忘[いっしょう わす]れません。			
\\	市街	市街[しがい]	しがい	
\\	夕方の市街は車が渋滞するよ。	夕方[ゆうがた]の 市街[しがい]は 車[くるま]が 渋滞[じゅうたい]するよ。	ゆうがた の しがい は くるま が じゅうたい する よ	
\\	夕方[ゆうがた]の
\\	は 車[くるま]が 渋滞[じゅうたい]するよ。			
\\	粉	粉[こな]	こな	
\\	その白い粉は赤ちゃんのミルクです。	その 白[しろ]い 粉[こな]は 赤[あか]ちゃんのミルクです。	その しろい こな は あかちゃん の みるく です	
\\	その 白[しろ]い
\\	は 赤[あか]ちゃんのミルクです。			
\\	小麦	小麦[こむぎ]	こむぎ	
\\	小麦は色々な食べ物に使われている。	小麦[こむぎ]は 色々[いろいろ]な 食[た]べ 物[もの]に 使[つか]われている。	こむぎ は いろいろ な たべもの に つかわれて いる	
\\	は 色々[いろいろ]な 食[た]べ 物[もの]に 使[つか]われている。			
\\	小麦粉	小麦粉[こむぎこ]	こむぎこ	
\\	うどんは小麦粉から作られます。	うどんは 小麦粉[こむぎこ]から 作[つく]られます。	うどん は こむぎこ から つくられます	
\\	うどんは
\\	から 作[つく]られます。			
\\	いつのまに	いつのまに	いつのまに	
\\	彼はいつのまにいなくなったの。	彼[かれ]はいつのまにいなくなったの。	かれ は いつのまに いなく なった の	
\\	彼[かれ]は
\\	いなくなったの。			
\\	大麦	大麦[おおむぎ]	おおむぎ	
\\	大麦はビールの原料になります。	大麦[おおむぎ]はビールの 原料[げんりょう]になります。	おおむぎ は びーる の げんりょう に なります	
\\	はビールの 原料[げんりょう]になります。			
\\	炊事	炊事[すいじ]	すいじ	
\\	うちでは夫も炊事をします。	うちでは 夫[おっと]も 炊事[すいじ]をします。	うち で は おっと も すいじ を します	
\\	うちでは 夫[おっと]も
\\	をします。			
\\	暦	暦[こよみ]	こよみ	
\\	暦の上では今日から冬ですね。	暦[こよみ]の 上[うえ]では 今日[きょう]から 冬[ふゆ]ですね。	こよみ の うえ で は きょう から ふゆ です ね	
\\	の 上[うえ]では 今日[きょう]から 冬[ふゆ]ですね。			
\\	皮	皮[かわ]	かわ	
\\	りんごの皮をむきましたよ。	りんごの 皮[かわ]をむきましたよ。	りんご の かわ を むきました よ	
\\	りんごの
\\	をむきましたよ。			
\\	毛皮	毛皮[けがわ]	けがわ	
\\	彼女は毛皮のコートを着ていたの。	彼女[かのじょ]は 毛皮[けがわ]のコートを 着[き]ていたの。	かのじょ は けがわ の こーと を きて いた の	
\\	彼女[かのじょ]は
\\	のコートを 着[き]ていたの。			
\\	毛糸	毛糸[けいと]	けいと	
\\	彼女は毛糸のセーターを編みました。	彼女[かのじょ]は 毛糸[けいと]のセーターを 編[あ]みました。	かのじょ は けいと の せーたー を あみました	
\\	彼女[かのじょ]は
\\	のセーターを 編[あ]みました。			
\\	尾	尾[お]	お	
\\	尾の長い鳥が飛んでいますね。	尾[お]の 長[なが]い 鳥[とり]が 飛[と]んでいますね。	お の ながい とり が とんで います ね	
\\	の 長[なが]い 鳥[とり]が 飛[と]んでいますね。			
\\	インフレーション	インフレーション	インフレーション	
\\	市場にはインフレーションの影響が出ているね。	市場[しじょう]にはインフレーションの 影響[えいきょう]が 出[で]ているね。	しじょう に は いんふれーしょん の えいきょう が でて いる ね	
\\	市場[しじょう]には
\\	の 影響[えいきょう]が 出[で]ているね。			
\\	白髪	白髪[しらが]	しらが	
\\	祖父は白髪が少ないほうです。	祖父[そふ]は 白髪[しらが]が 少[すく]ないほうです。	そふ は しらが が すくない ほう です	
\\	祖父[そふ]は
\\	が 少[すく]ないほうです。			
\\	耳鼻科	耳鼻科[じびか]	じびか	
\\	今、耳鼻科にかかっています。	今[いま]、 耳鼻科[じびか]にかかっています。	いま じびか に かかって います	
\\	今[いま]、
\\	にかかっています。			
\\	唇	唇[くちびる]	くちびる	
\\	寒くて唇が青くなってしまった。	寒[さむ]くて 唇[くちびる]が 青[あお]くなってしまった。	さむく て くちびる が あおく なって しまった	
\\	寒[さむ]くて
\\	が 青[あお]くなってしまった。			
\\	解釈	解釈[かいしゃく]	かいしゃく	
\\	この詩を解釈してみましょう。	この 詩[し]を 解釈[かいしゃく]してみましょう。	この し を かいしゃく して みましょう	
\\	この 詩[し]を
\\	してみましょう。			
\\	居間	居間[いま]	いま	
\\	父は居間でテレビを見ている。	父[ちち]は 居間[いま]でテレビを 見[み]ている。	ちち は いま で てれび を みて いる	
\\	父[ちち]は
\\	でテレビを 見[み]ている。			
\\	居る	居[お]る	おる	
\\	母は今、うちに居りません。	母[はは]は 今[いま]、うちに 居[お]りません。	はは は いま うち に おりません	
\\	母[はは]は 今[いま]、うちに
\\	一層	一層[いっそう]	いっそう	
\\	雨が一層激しくなったね。	雨[あめ]が 一層[いっそう] 激[はげ]しくなったね。	あめ が いっそう はげしく なった ね	
\\	雨[あめ]が
\\	激[はげ]しくなったね。			
\\	ウィークエンド	ウィークエンド	ウィークエンド	
\\	今度のウィークエンドは映画を見ます。	今度[こんど]のウィークエンドは 映画[えいが]を 見[み]ます。	こんど の うぃーくえんど は えいが を みます	
\\	今度[こんど]の
\\	は 映画[えいが]を 見[み]ます。			
\\	高層	高層[こうそう]	こうそう	
\\	このあたりは高層ビルが増えましたね。	このあたりは 高層[こうそう]ビルが 増[ふ]えましたね。	この あたり は こうそうびる が ふえました ね	
\\	このあたりは
\\	ビルが 増[ふ]えましたね。			
\\	覆う	覆[おお]う	おおう	
\\	雲が空を覆っているね。	雲[くも]が 空[そら]を 覆[おお]っているね。	くも が そら を おおって いる ね	
\\	雲[くも]が 空[そら]を
\\	ね。			
\\	戸籍	戸籍[こせき]	こせき	
\\	結婚すると新しい戸籍が作られます。	結婚[けっこん]すると 新[あたら]しい 戸籍[こせき]が 作[つく]られます。	けっこん する と あたらしい こせき が つくられます	
\\	結婚[けっこん]すると 新[あたら]しい
\\	が 作[つく]られます。			
\\	ガラス戸	ガラス 戸[ど]	がらすど	
\\	お店のガラス戸が割られたんだ。	お 店[みせ]のガラス 戸[ど]が 割[わ]られたんだ。	おみせ の がらすど が わられた ん だ	
\\	お 店[みせ]の
\\	が 割[わ]られたんだ。			
\\	新鮮	新鮮[しんせん]	しんせん	
\\	この店では新鮮な野菜が買えますよ。	この 店[みせ]では 新鮮[しんせん]な 野菜[やさい]が 買[か]えますよ。	この みせ で は しんせん な やさい が かえます よ	
\\	この 店[みせ]では
\\	な 野菜[やさい]が 買[か]えますよ。			
\\	鮮やか	鮮[あざ]やか	あざやか	
\\	山は鮮やかな緑でした。	山[やま]は 鮮[あざ]やかな 緑[みどり]でした。	やま は あざやか な みどり でした	
\\	山[やま]は
\\	な 緑[みどり]でした。			
\\	群集	群集[ぐんしゅう]	ぐんしゅう	
\\	数百人の群集が集まっているの。	数百人[すうひゃくにん]の 群集[ぐんしゅう]が 集[あつ]まっているの。	すうひゃくにん の ぐんしゅう が あつまって いる の	
\\	数百人[すうひゃくにん]の
\\	が 集[あつ]まっているの。			
\\	郡	郡[ぐん]	ぐん	
\\	この郡は来年、市になりますよ。	この 郡[ぐん]は 来年[らいねん]、 市[し]になりますよ。	この ぐん は らいねん し に なります よ	
\\	この
\\	は 来年[らいねん]、 市[し]になりますよ。			
\\	うがい	うがい	うがい	
\\	冬はまめにうがいをします。	冬[ふゆ]はまめにうがいをします。	ふゆ は まめ に うがい を します	
\\	冬[ふゆ]はまめに
\\	をします。			
\\	君	君[きみ]	きみ	
\\	この本を君にあげます。	この 本[ほん]を 君[きみ]にあげます。	この ほん を きみ に あげます	
\\	この 本[ほん]を
\\	にあげます。			
\\	脂	脂[あぶら]	あぶら	
\\	この肉は脂が多い。	この 肉[にく]は 脂[あぶら]が 多[おお]い。	この にく は あぶら が おおい	
\\	この 肉[にく]は
\\	が 多[おお]い。			
\\	脂肪	脂肪[しぼう]	しぼう	
\\	お腹の脂肪を取りたい。	お 腹[なか]の 脂肪[しぼう]を 取[と]りたい。	おなか の しぼう を とりたい	
\\	お 腹[なか]の
\\	を 取[と]りたい。			
\\	胃腸	胃腸[いちょう]	いちょう	
\\	彼は胃腸が強くありません。	彼[かれ]は 胃腸[いちょう]が 強[つよ]くありません。	かれ は いちょう が つよく ありません	
\\	彼[かれ]は
\\	が 強[つよ]くありません。			
\\	肝心	肝心[かんじん]	かんじん	
\\	何事も最初が肝心だ。	何事[なにごと]も 最初[さいしょ]が 肝心[かんじん]だ。	なにごと も さいしょ が かんじん だ	
\\	何事[なにごと]も 最初[さいしょ]が
\\	だ。			
\\	山脈	山脈[さんみゃく]	さんみゃく	
\\	列車の窓から雄大な山脈が見えたんだよ。	列車[れっしゃ]の 窓[まど]から 雄大[ゆうだい]な 山脈[さんみゃく]が 見[み]えたんだよ。	れっしゃ の まど から ゆうだい な さんみゃく が みえた ん だ よ	
\\	列車[れっしゃ]の 窓[まど]から 雄大[ゆうだい]な
\\	が 見[み]えたんだよ。			
\\	暮らす	暮[く]らす	くらす	
\\	将来は海の近くで暮らしたいな。	将来[しょうらい]は 海[うみ]の 近[ちか]くで 暮[く]らしたいな。	しょうらい は うみ の ちかく で くらしたい な	
\\	将来[しょうらい]は 海[うみ]の 近[ちか]くで
\\	な。			
\\	うきうき	うきうき	うきうき	
\\	もうすぐ夏休みなのでうきうきしています。	もうすぐ 夏休[なつやす]みなのでうきうきしています。	もうすぐ なつやすみ な の で うきうき して います	
\\	もうすぐ 夏休[なつやす]みなので
\\	しています。			
\\	暮らし	暮[く]らし	くらし	
\\	彼女は毎日の暮らしを楽しんでいますね。	彼女[かのじょ]は 毎日[まいにち]の 暮[く]らしを 楽[たの]しんでいますね。	かのじょ は まいにち の くらし を たのしんで います ね	
\\	彼女[かのじょ]は 毎日[まいにち]の
\\	を 楽[たの]しんでいますね。			
\\	暮れ	暮[く]れ	くれ	
\\	暮れは用事が多くて忙しいです。	暮[く]れは 用事[ようじ]が 多[おお]くて 忙[いそが]しいです。	くれ は ようじ が おおくて いそがしい です	
\\	は 用事[ようじ]が 多[おお]くて 忙[いそが]しいです。			
\\	暮れる	暮[く]れる	くれる	
\\	日が暮れる前に帰りましょう。	日[ひ]が 暮[く]れる 前[まえ]に 帰[かえ]りましょう。	ひ が くれる まえ に かえりましょう	
\\	日[ひ]が
\\	前[まえ]に 帰[かえ]りましょう。			
\\	お歳暮	お 歳暮[せいぼ]	おせいぼ	
\\	デパートでお歳暮を送りました。	デパートでお 歳暮[せいぼ]を 送[おく]りました。	でぱーと で おせいぼ を おくりました	
\\	デパートで
\\	を 送[おく]りました。			
\\	芝生	芝生[しばふ]	しばふ	
\\	芝生がよく手入れされていますね。	芝生[しばふ]がよく 手入[てい]れされていますね。	しばふ が よく ていれ されて います ね	
\\	がよく 手入[てい]れされていますね。			
\\	茎	茎[くき]	くき	
\\	チューリップは茎が長いです。	チューリップは 茎[くき]が 長[なが]いです。	ちゅーりっぷ は くき が ながい です	
\\	チューリップは
\\	が 長[なが]いです。			
\\	推薦	推薦[すいせん]	すいせん	
\\	彼は会長に推薦されたよ。	彼[かれ]は 会長[かいちょう]に 推薦[すいせん]されたよ。	かれ は かいちょう に すいせん された よ	
\\	彼[かれ]は 会長[かいちょう]に
\\	されたよ。			
\\	うつむく	うつむく	うつむく	
\\	彼はうつむいて何かを考えているね。	彼[かれ]はうつむいて 何[なに]かを 考[かんが]えているね。	かれ は うつむいて なにか を かんがえて いる ね	
\\	彼[かれ]は
\\	何[なに]かを 考[かんが]えているね。			
\\	荒れる	荒[あ]れる	あれる	
\\	台風で山も海も荒れています。	台風[たいふう]で 山[やま]も 海[うみ]も 荒[あ]れています。	たいふう で やま も うみ も あれて います	
\\	台風[たいふう]で 山[やま]も 海[うみ]も
\\	荒い	荒[あら]い	あらい	
\\	彼は金遣いが荒いな。	彼[かれ]は 金遣[かねづか]いが 荒[あら]いな。	かれ は かねづかい が あらい な	
\\	彼[かれ]は 金遣[かねづか]いが
\\	な。			
\\	荒い	荒[あら]い	あらい	
\\	うちの犬は気が荒い。	うちの 犬[いぬ]は 気[き]が 荒[あら]い。	うち の いぬ は き が あらい	
\\	うちの 犬[いぬ]は 気[き]が
\\	荒す	荒[あら]す	あらす	
\\	彼は強盗に家の中を荒されたの。	彼[かれ]は 強盗[ごうとう]に 家[いえ]の 中[なか]を 荒[あら]されたの。	かれ は ごうとう に いえ の なか を あらされた の	
\\	彼[かれ]は 強盗[ごうとう]に 家[いえ]の 中[なか]を
\\	の。			
\\	慌ただしい	慌[あわ]ただしい	あわただしい	
\\	今日は慌ただしい一日でした。	今日[きょう]は 慌[あわ]ただしい 一日[いちにち]でした。	きょう は あわただしい いちにち でした	
\\	今日[きょう]は
\\	一日[いちにち]でした。			
\\	慌てる	慌[あわ]てる	あわてる	
\\	そんなに慌ててどこに行くの。	そんなに 慌[あわ]ててどこに 行[い]くの。	そんなに あわてて どこ に いく の	
\\	そんなに
\\	どこに 行[い]くの。			
\\	儀式	儀式[ぎしき]	ぎしき	
\\	儀式には作法があります。	儀式[ぎしき]には 作法[さほう]があります。	ぎしき に は さほう が あります	
\\	には 作法[さほう]があります。			
\\	うなずく	うなずく	うなずく	
\\	彼女はうなずいて同意を示したの。	彼女[かのじょ]はうなずいて 同意[どうい]を 示[しめ]したの。	かのじょ は うなずいて どうい を しめした の	
\\	彼女[かのじょ]は
\\	同意[どうい]を 示[しめ]したの。			
\\	行儀	行儀[ぎょうぎ]	ぎょうぎ	
\\	この子は本当に行儀の良い子です。	この 子[こ]は 本当[ほんとう]に 行儀[ぎょうぎ]の 良[い]い 子[こ]です。	この こ は ほんとう に ぎょうぎ の いい こ です	
\\	この 子[こ]は 本当[ほんとう]に
\\	の 良[い]い 子[こ]です。			
\\	犠牲	犠牲[ぎせい]	ぎせい	
\\	多くの人が災害の犠牲になったの。	多[おお]くの 人[ひと]が 災害[さいがい]の 犠牲[ぎせい]になったの。	おおく の ひと が さいがい の ぎせい に なった の	
\\	多[おお]くの 人[ひと]が 災害[さいがい]の
\\	になったの。			
\\	赤ん坊	赤[あか]ん 坊[ぼう]	あかんぼう	
\\	その頃彼はまだ赤ん坊だったよね。	その 頃彼[ころ かれ]はまだ 赤[あか]ん 坊[ぼう]だったよね。	その ころ かれ は まだ あかんぼう だった よ ね	
\\	その 頃彼[ころ かれ]はまだ
\\	だったよね。			
\\	お坊さん	お 坊[ぼう]さん	おぼうさん	
\\	お寺からお坊さんが出て来ましたよ。	お 寺[てら]からお 坊[ぼう]さんが 出[で]て 来[き]ましたよ。	おてら から おぼうさん が でて きました よ	
\\	お 寺[てら]から
\\	が 出[で]て 来[き]ましたよ。			
\\	国旗	国旗[こっき]	こっき	
\\	日本の国旗は描くのが簡単です。	日本[にほん]の 国旗[こっき]は 描[か]くのが 簡単[かんたん]です。	にほん の こっき は かく の が かんたん です	
\\	日本[にほん]の
\\	は 描[か]くのが 簡単[かんたん]です。			
\\	筋肉	筋肉[きんにく]	きんにく	
\\	彼は最近筋肉を鍛えているよ。	彼[かれ]は 最近[さいきん] 筋肉[きんにく]を 鍛[きた]えているよ。	かれ は さいきん きんにく を きたえて いる よ	
\\	彼[かれ]は 最近[さいきん]
\\	を 鍛[きた]えているよ。			
\\	筋道	筋道[すじみち]	すじみち	
\\	筋道を立てて考えなさい。	筋道[すじみち]を 立[た]てて 考[かんが]えなさい。	すじみち を たてて かんがえなさい	
\\	を 立[た]てて 考[かんが]えなさい。			
\\	筋	筋[すじ]	すじ	
\\	転んで筋を痛めてしまいました。	転[ころ]んで 筋[すじ]を 痛[いた]めてしまいました。	ころんで すじ を いためて しまいました	
\\	転[ころ]んで
\\	を 痛[いた]めてしまいました。			
\\	おだてる	おだてる	おだてる	
\\	彼は人をおだてるのが上手です。	彼[かれ]は 人[ひと]をおだてるのが 上手[じょうず]です。	かれ は ひと を おだてる の が じょうず です	
\\	彼[かれ]は 人[ひと]を
\\	のが 上手[じょうず]です。			
\\	裂ける	裂[さ]ける	さける	
\\	釘に引っかけて袖が裂けてしまったな。	釘[くぎ]に 引[ひ]っかけて 袖[そで]が 裂[さ]けてしまったな。	くぎ に ひっかけて そで が さけて しまった な	
\\	釘[くぎ]に 引[ひ]っかけて 袖[そで]が
\\	な。			
\\	座布団	座布団[ざぶとん]	ざぶとん	
\\	この座布団は座り心地がいいね。	この 座布団[ざぶとん]は 座[すわ]り 心地[ごこち]がいいね。	この ざぶとん は すわり ごこち が いい ね	
\\	この
\\	は 座[すわ]り 心地[ごこち]がいいね。			
\\	自己	自己[じこ]	じこ	
\\	これからは自己の判断で行動してください。	これからは 自己[じこ]の 判断[はんだん]で 行動[こうどう]してください。	これ から は じこ の はんだん で こうどう して ください	
\\	これからは
\\	の 判断[はんだん]で 行動[こうどう]してください。			
\\	抱える	抱[かか]える	かかえる	
\\	彼は大きな荷物を抱えているわ。	彼[かれ]は 大[おお]きな 荷物[にもつ]を 抱[かか]えているわ。	かれ は おおき な にもつ を かかえて いる わ	
\\	彼[かれ]は 大[おお]きな 荷物[にもつ]を
\\	わ。			
\\	抱く	抱[いだ]く	いだく	
\\	少年よ大志を抱け。	少年[しょうねん]よ 大志[たいし]を 抱[いだ]け。	しょうねん よ たいし を いだけ	
\\	少年[しょうねん]よ 大志[たいし]を
\\	句	句[く]	く	
\\	この句はどんな意味でしょうか。	この 句[く]はどんな 意味[いみ]でしょうか。	この く は どんな いみ でしょう か	
\\	この
\\	はどんな 意味[いみ]でしょうか。			
\\	慣用句	慣用句[かんようく]	かんようく	
\\	「手が空く」は慣用句です。	
\\	手[て]が 空[す]く」は 慣用句[かんようく]です。	て が すく は かんようく です	
\\	手[て]が 空[す]く」は
\\	です。			
\\	かかと	かかと	かかと	
\\	靴のかかとで彼の足を踏んでしまったの。	靴[くつ]のかかとで 彼[かれ]の 足[あし]を 踏[ふ]んでしまったの。	くつ の かかと で かれ の あし を ふん でしまった の 。	
\\	靴[くつ]の
\\	で 彼[かれ]の 足[あし]を 踏[ふ]んでしまったの。			
\\	下旬	下旬[げじゅん]	げじゅん	
\\	この仕事は来月の下旬には終わります。	この 仕事[しごと]は 来月[らいげつ]の 下旬[げじゅん]には 終[お]わります。	この しごと は らいげつ の げじゅん に は おわります	
\\	この 仕事[しごと]は 来月[らいげつ]の
\\	には 終[お]わります。			
\\	上旬	上旬[じょうじゅん]	じょうじゅん	
\\	7月上旬に夏祭りがありますよ。	7月[しちがつ] 上旬[じょうじゅん]に 夏祭[なつまつ]りがありますよ。	しちがつ じょうじゅん に なつまつり が あります よ	
\\	7月[しちがつ]
\\	に 夏祭[なつまつ]りがありますよ。			
\\	飽きる	飽[あ]きる	あきる	
\\	その子はおもちゃに飽きたようだね。	その 子[こ]はおもちゃに 飽[あ]きたようだね。	その こ は おもちゃ に あきた よう だ ね	
\\	その 子[こ]はおもちゃに
\\	ようだね。			
\\	飾り	飾[かざ]り	かざり	
\\	この部屋は飾りが多すぎるな。	この 部屋[へや]は 飾[かざ]りが 多[おお]すぎるな。	この へや は かざり が おおすぎる な	
\\	この 部屋[へや]は
\\	が 多[おお]すぎるな。			
\\	餌	餌[えさ]	えさ	
\\	毎日金魚に餌をやります。	毎日金魚[まいにち きんぎょ]に 餌[えさ]をやります。	まいにち きんぎょ に えさ を やります	
\\	毎日金魚[まいにち きんぎょ]に
\\	をやります。			
\\	旧	旧[きゅう]	きゅう	
\\	旧ソビエトは今はロシアと呼ばれている。	旧[きゅう]ソビエトは 今[いま]はロシアと 呼[よ]ばれている。	きゅうそびえと は いま は ろしあ と よばれて いる	
\\	ソビエトは 今[いま]はロシアと 呼[よ]ばれている。			
\\	育児	育児[いくじ]	いくじ	
\\	彼女は育児で忙しいよ。	彼女[かのじょ]は 育児[いくじ]で 忙[いそが]しいよ。	かのじょ は いくじ で いそがしい よ	
\\	彼女[かのじょ]は
\\	で 忙[いそが]しいよ。			
\\	がさがさ	がさがさ	がさがさ	
\\	玄関でがさがさと音がしましたよ。	玄関[げんかん]でがさがさと 音[おと]がしましたよ。	げんかん で がさがさ と おと が しました よ	
\\	玄関[げんかん]で
\\	と 音[おと]がしましたよ。			
\\	小児科	小児科[しょうにか]	しょうにか	
\\	子供を小児科に連れて行くところです。	子供[こども]を 小児科[しょうにか]に 連[つ]れて 行[い]くところです。	こども を しょうにか に つれて いく ところ です	
\\	子供[こども]を
\\	に 連[つ]れて 行[い]くところです。			
\\	姓名	姓名[せいめい]	せいめい	
\\	あなたの姓名を教えてください。	あなたの 姓名[せいめい]を 教[おし]えてください。	あなた の せいめい を おしえて ください	
\\	あなたの
\\	を 教[おし]えてください。			
\\	姓	姓[せい]	せい	
\\	結婚して姓が変わりました。	結婚[けっこん]して 姓[せい]が 変[か]わりました。	けっこん して せい が かわりました	
\\	結婚[けっこん]して
\\	が 変[か]わりました。			
\\	幼い	幼[おさな]い	おさない	
\\	彼女には幼い息子がいます。	彼女[かのじょ]には 幼[おさな]い 息子[むすこ]がいます。	かのじょ に は おさない むすこ が います	
\\	彼女[かのじょ]には
\\	息子[むすこ]がいます。			
\\	眼科	眼科[がんか]	がんか	
\\	眼科で視力検査をしました。	眼科[がんか]で 視力検査[しりょく けんさ]をしました。	がんか で しりょく けんさ を しました	
\\	で 視力検査[しりょく けんさ]をしました。			
\\	近眼	近眼[きんがん]	きんがん	
\\	彼女は近眼です。	彼女[かのじょ]は 近眼[きんがん]です。	かのじょ は きんがん です	
\\	彼女[かのじょ]は
\\	です。			
\\	居眠り	居眠[いねむ]り	いねむり	
\\	彼はソファーで居眠りをしているよ。	彼[かれ]はソファーで 居眠[いねむ]りをしているよ。	かれ は そふぁー で いねむり を して いる よ	
\\	彼[かれ]はソファーで
\\	をしているよ。			
\\	一瞬	一瞬[いっしゅん]	いっしゅん	
\\	一瞬自分の耳を疑ったよ。	一瞬[いっしゅん] 自分[じぶん]の 耳[みみ]を 疑[うたが]ったよ。	いっしゅん じぶん の みみ を うたがった よ	
\\	自分[じぶん]の 耳[みみ]を 疑[うたが]ったよ。			
\\	かじる	かじる	かじる	
\\	弟は美味しそうにりんごをかじっていたよ。	弟[おとうと]は 美味[おい]しそうにりんごをかじっていたよ。	おとうと は おいし そう に りんご を かじって いた よ	
\\	弟[おとうと]は 美味[おい]しそうにりんごを
\\	よ。			
\\	瞬間	瞬間[しゅんかん]	しゅんかん	
\\	その瞬間、猫が跳び出しました。	その 瞬間[しゅんかん]、 猫[ねこ]が 跳[と]び 出[だ]しました。	その しゅんかん ねこ が とびだしました	
\\	その
\\	、 猫[ねこ]が 跳[と]び 出[だ]しました。			
\\	睡眠	睡眠[すいみん]	すいみん	
\\	1日7時間は睡眠を取るようにしています。	1日7時間[いちにち しちじかん]は 睡眠[すいみん]を 取[と]るようにしています。	いちにち しちじかん は すいみん を とる よう に して います	
\\	1日7時間[いちにち しちじかん]は
\\	を 取[と]るようにしています。			
\\	垂直	垂直[すいちょく]	すいちょく	
\\	彼らは垂直のがけを登り始めたの。	彼[かれ]らは 垂直[すいちょく]のがけを 登[のぼ]り 始[はじ]めたの。	かれら は すいちょく の がけ を のぼりはじめた の	
\\	彼[かれ]らは
\\	のがけを 登[のぼ]り 始[はじ]めたの。			
\\	心掛ける	心掛[こころが]ける	こころがける	
\\	安全運転を心掛けてください。	安全運転[あんぜん うんてん]を 心掛[こころが]けてください。	あんぜん うんてん を こころがけて ください	
\\	安全運転[あんぜん うんてん]を
\\	ください。			
\\	腰掛ける	腰掛[こしか]ける	こしかける	
\\	私はベンチに腰掛けたの。	私[わたし]はベンチに 腰掛[こしか]けたの。	わたし は べんち に こしかけた の	
\\	私[わたし]はベンチに
\\	の。			
\\	追い掛ける	追[お]い 掛[か]ける	おいかける	
\\	パトカーがバイクを追いかけてるぞ。	パトカーがバイクを 追[お]いかけてるぞ。	ぱとかー が ばいく を おいかけて る ぞ	
\\	パトカーがバイクを
\\	ぞ。			
\\	腰掛け	腰掛[こしか]け	こしかけ	
\\	腰掛けはありませんか。	腰掛[こしか]けはありませんか。	こしかけ は ありません か	
\\	はありませんか。			
\\	がぶがぶ	がぶがぶ	がぶがぶ	
\\	彼は水をがぶがぶ飲んだんだ。	彼[かれ]は 水[みず]をがぶがぶ 飲[の]んだんだ。	かれ は みず を がぶがぶ のんだ ん だ	
\\	彼[かれ]は 水[みず]を
\\	飲[の]んだんだ。			
\\	お目に掛かる	お 目[め]に 掛[か]かる	おめにかかる	
\\	お目に掛かれて嬉しいです。	お 目[め]に 掛[か]かれて 嬉[うれ]しいです。	おめにかかれて うれしい です	
\\	嬉[うれ]しいです。			
\\	拝む	拝[おが]む	おがむ	
\\	仏像に手を合わせて拝みました。	仏像[ぶつぞう]に 手[て]を 合[あ]わせて 拝[おが]みました。	ぶつぞう に て を あわせて おがみました	
\\	仏像[ぶつぞう]に 手[て]を 合[あ]わせて
\\	括弧	括弧[かっこ]	かっこ	
\\	括弧の部分は省略できます。	括弧[かっこ]の 部分[ぶぶん]は 省略[しょうりゃく]できます。	かっこ の ぶぶん は しょうりゃく できます	
\\	の 部分[ぶぶん]は 省略[しょうりゃく]できます。			
\\	指揮	指揮[しき]	しき	
\\	彼がそのプロジェクトの指揮を取ったの。	彼[かれ]がそのプロジェクトの 指揮[しき]を 取[と]ったの。	かれ が その ぷろじぇくと の しき を とった の	
\\	彼[かれ]がそのプロジェクトの
\\	を 取[と]ったの。			
\\	輝く	輝[かがや]く	かがやく	
\\	彼女の瞳は喜びで輝いていますね。	彼女[かのじょ]の 瞳[ひとみ]は 喜[よろこ]びで 輝[かがや]いていますね。	かのじょ の ひとみ は よろこび で かがやいて います ね	
\\	彼女[かのじょ]の 瞳[ひとみ]は 喜[よろこ]びで
\\	ね。			
\\	抑える	抑[おさ]える	おさえる	
\\	彼は怒りを抑えていたの。	彼[かれ]は 怒[いか]りを 抑[おさ]えていたの。	かれ は いかり を おさえて いた の	
\\	彼[かれ]は 怒[いか]りを
\\	の。			
\\	信仰	信仰[しんこう]	しんこう	
\\	信仰は心の支えです。	信仰[しんこう]は 心[こころ]の 支[ささ]えです。	しんこう は こころ の ささえ です	
\\	は 心[こころ]の 支[ささ]えです。			
\\	からす	からす	からす	
\\	木の上でからすが鳴いています。	木[き]の 上[うえ]でからすが 鳴[な]いています。	き の うえ で からす が ないて います	
\\	木[き]の 上[うえ]で
\\	が 鳴[な]いています。			
\\	偶然	偶然[ぐうぜん]	ぐうぜん	
\\	街で偶然友人に会いました。	街[まち]で 偶然[ぐうぜん] 友人[ゆうじん]に 会[あ]いました。	まち で ぐうぜん ゆうじん に あいました	
\\	街[まち]で
\\	友人[ゆうじん]に 会[あ]いました。			
\\	偶数	偶数[ぐうすう]	ぐうすう	
\\	偶数は2で割り切れます。	偶数[ぐうすう]は 2[に]で 割[わ]り 切[き]れます。	ぐうすう は に で わりきれます	
\\	は 2[に]で 割[わ]り 切[き]れます。			
\\	隅	隅[すみ]	すみ	
\\	部屋の隅にいすが積んであったの。	部屋[へや]の 隅[すみ]にいすが 積[つ]んであったの。	へや の すみ に いす が つんで あった の	
\\	部屋[へや]の
\\	にいすが 積[つ]んであったの。			
\\	城	城[しろ]	しろ	
\\	今回の旅行ではヨーロッパの城を見て回ります。	今回[こんかい]の 旅行[りょこう]ではヨーロッパの 城[しろ]を 見[み]て 回[まわ]ります。	こんかい の りょこう で は よーろっぱ の しろ を みて まわります	
\\	今回[こんかい]の 旅行[りょこう]ではヨーロッパの
\\	を 見[み]て 回[まわ]ります。			
\\	栽培	栽培[さいばい]	さいばい	
\\	この地方ではみかんの栽培が盛んです。	この 地方[ちほう]ではみかんの 栽培[さいばい]が 盛[さか]んです。	この ちほう で は みかん の さいばい が さかん です	
\\	この 地方[ちほう]ではみかんの
\\	が 盛[さか]んです。			
\\	警戒	警戒[けいかい]	けいかい	
\\	地震のあとは津波に警戒してください。	地震[じしん]のあとは 津波[つなみ]に 警戒[けいかい]してください。	じしん の あと は つなみ に けいかい して ください	
\\	地震[じしん]のあとは 津波[つなみ]に
\\	してください。			
\\	幾ら	幾[いく]ら	いくら	
\\	幾ら呼んでも彼は返事をしなかったわ。	幾[いく]ら 呼[よ]んでも 彼[かれ]は 返事[へんじ]をしなかったわ。	いくら よんで も かれ は へんじ を しなかった わ	
\\	呼[よ]んでも 彼[かれ]は 返事[へんじ]をしなかったわ。			
\\	伺う	伺[うかが]う	うかがう	
\\	お話しを伺いたいのですが。	お 話[はな]しを 伺[うかが]いたいのですが。	おはなし を うかがいたい の です が	
\\	お 話[はな]しを
\\	のですが。			
\\	がん	がん	がん	
\\	祖父はがんで亡くなりました。	祖父[そふ]はがんで 亡[な]くなりました。	そふ は がん で なくなりました	
\\	祖父[そふ]は
\\	で 亡[な]くなりました。			
\\	後悔	後悔[こうかい]	こうかい	
\\	後悔しても、しょうがない。	後悔[こうかい]しても、しょうがない。	こうかい して も しょうがない	
\\	しても、しょうがない。			
\\	悔しい	悔[くや]しい	くやしい	
\\	試合に負けてとても悔しい。	試合[しあい]に 負[ま]けてとても 悔[くや]しい。	しあい に まけて とても くやしい	
\\	試合[しあい]に 負[ま]けてとても
\\	俺	俺[おれ]	おれ	
\\	俺の頼みを聞いてくれ。	俺[おれ]の 頼[たの]みを 聞[き]いてくれ。	おれ の たのみ を きいて くれ	
\\	の 頼[たの]みを 聞[き]いてくれ。			
\\	相撲	相撲[すもう]	すもう	
\\	お相撲さんは、みんな力持ちだ。	お 相撲[すもう]さんは、みんな 力[ちから] 持[も]ちだ。	おすもうさん は みんな ちからもち だ	
\\	お
\\	さんは、みんな 力[ちから] 持[も]ちだ。			
\\	偏る	偏[かたよ]る	かたよる	
\\	栄養が偏らないように食事に気を付けています。	栄養[えいよう]が 偏[かたよ]らないように 食事[しょくじ]に 気[き]を 付[つ]けています。	えいよう が かたよらない よう に しょくじ に き を つけて います	
\\	栄養[えいよう]が
\\	ように 食事[しょくじ]に 気[き]を 付[つ]けています。			
\\	一遍に	一遍[いっぺん]に	いっぺんに	
\\	春は一遍に花が咲く。	春[はる]は 一遍[いっぺん]に 花[はな]が 咲[さ]く。	はる は いっぺんに はな が さく	
\\	春[はる]は
\\	花[はな]が 咲[さ]く。			
\\	宗教	宗教[しゅうきょう]	しゅうきょう	
\\	宗教を持たない人もたくさんいるわ。	宗教[しゅうきょう]を 持[も]たない 人[ひと]もたくさんいるわ。	しゅうきょう を もたない ひと も たくさん いる わ	
\\	を 持[も]たない 人[ひと]もたくさんいるわ。			
\\	ぎっしり	ぎっしり	ぎっしり	
\\	この箱にはぎっしり物が詰まっています。	この 箱[はこ]にはぎっしり 物[もの]が 詰[つ]まっています。	この はこ に は ぎっしり もの が つまって います	
\\	この 箱[はこ]には
\\	物[もの]が 詰[つ]まっています。			
\\	審議	審議[しんぎ]	しんぎ	
\\	その問題は審議中です。	その 問題[もんだい]は 審議[しんぎ] 中[ちゅう]です。	その もんだい は しんぎちゅう です	
\\	その 問題[もんだい]は
\\	中[ちゅう]です。			
\\	憲法	憲法[けんぽう]	けんぽう	
\\	憲法を改正することは難しいわね。	憲法[けんぽう]を 改正[かいせい]することは 難[むずか]しいわね。	けんぽう を かいせい する こと は むずかしい わ ね	
\\	を 改正[かいせい]することは 難[むずか]しいわね。			
\\	衰える	衰[おとろ]える	おとろえる	
\\	年と共に体力が衰えています。	年[とし]と 共[とも]に 体力[たいりょく]が 衰[おとろ]えています。	とし と とも に たいりょく が おとろえて います	
\\	年[とし]と 共[とも]に 体力[たいりょく]が
\\	惜しむ	惜[お]しむ	おしむ	
\\	私たちはみな彼の死を惜しんだの。	私[わたし]たちはみな 彼[かれ]の 死[し]を 惜[お]しんだの。	わたしたち は みな かれ の し を おしんだ の	
\\	私[わたし]たちはみな 彼[かれ]の 死[し]を
\\	の。			
\\	惜しい	惜[お]しい	おしい	
\\	惜しい、もう少しで優勝だった。	惜[お]しい、もう 少[すこ]しで 優勝[ゆうしょう]だった。	おしい もうすこし で ゆうしょう だった	
\\	、もう 少[すこ]しで 優勝[ゆうしょう]だった。			
\\	恨み	恨[うら]み	うらみ	
\\	彼女は長年の恨みを晴らした。	彼女[かのじょ]は 長年[ながねん]の 恨[うら]みを 晴[は]らした。	かのじょ は ながねん の うらみ を はらした	
\\	彼女[かのじょ]は 長年[ながねん]の
\\	を 晴[は]らした。			
\\	恨む	恨[うら]む	うらむ	
\\	彼を恨んではいけません。	彼[かれ]を 恨[うら]んではいけません。	かれ を うらんで は いけません	
\\	彼[かれ]を
\\	はいけません。			
\\	ぎゅうぎゅう	ぎゅうぎゅう	ぎゅうぎゅう	
\\	電車がぎゅうぎゅうに込んでいますね。	電車[でんしゃ]がぎゅうぎゅうに 込[こ]んでいますね。	でんしゃ が ぎゅうぎゅう に こんで います ね	
\\	電車[でんしゃ]が
\\	に 込[こ]んでいますね。			
\\	覚悟	覚悟[かくご]	かくご	
\\	覚悟はできています。	覚悟[かくご]はできています。	かくご は できて います	
\\	はできています。			
\\	怪しい	怪[あや]しい	あやしい	
\\	その男の行動は怪しかったわ。	その 男[おとこ]の 行動[こうどう]は 怪[あや]しかったわ。	その おとこ の こうどう は あやしかった わ	
\\	その 男[おとこ]の 行動[こうどう]は
\\	わ。			
\\	怪しむ	怪[あや]しむ	あやしむ	
\\	警察はそのグループを怪しんでいます。	警察[けいさつ]はそのグループを 怪[あや]しんでいます。	けいさつ は その ぐるーぷ を あやしんで います	
\\	警察[けいさつ]はそのグループを
\\	自慢	自慢[じまん]	じまん	
\\	彼女はよく自分の成績を自慢するよね。	彼女[かのじょ]はよく 自分[じぶん]の 成績[せいせき]を 自慢[じまん]するよね。	かのじょ は よく じぶん の せいせき を じまん する よ ね	
\\	彼女[かのじょ]はよく 自分[じぶん]の 成績[せいせき]を
\\	するよね。			
\\	食卓	食卓[しょくたく]	しょくたく	
\\	食卓に花を飾りましょう。	食卓[しょくたく]に 花[はな]を 飾[かざ]りましょう。	しょくたく に はな を かざりましょう	
\\	に 花[はな]を 飾[かざ]りましょう。			
\\	歓迎	歓迎[かんげい]	かんげい	
\\	温かい歓迎を受けました。	温[あたた]かい 歓迎[かんげい]を 受[う]けました。	あたたかい かんげい を うけました	
\\	温[あたた]かい
\\	を 受[う]けました。			
\\	勧める	勧[すす]める	すすめる	
\\	勧められて欲しくもないものを買っちゃったよ。	勧[すす]められて 欲[ほ]しくもないものを 買[か]っちゃったよ。	すすめられて ほしく も ない もの を かっちゃった よ	
\\	欲[ほ]しくもないものを 買[か]っちゃったよ。			
\\	ぎょうざ	ぎょうざ	ぎょうざ	
\\	ここのぎょうざは美味しいよ。	ここのぎょうざは 美味[おい]しいよ。	ここ の ぎょうざ は おいしい よ	
\\	ここの
\\	は 美味[おい]しいよ。			
\\	焦点	焦点[しょうてん]	しょうてん	
\\	焦点を絞って話しましょう。	焦点[しょうてん]を 絞[しぼ]って 話[はな]しましょう。	しょうてん を しぼって はなしましょう	
\\	を 絞[しぼ]って 話[はな]しましょう。			
\\	焦る	焦[あせ]る	あせる	
\\	何をそんなに焦っているのですか。	何[なに]をそんなに 焦[あせ]っているのですか。	なに を そんなに あせって いる の です か	
\\	何[なに]をそんなに
\\	のですか。			
\\	焦げる	焦[こ]げる	こげる	
\\	シチューが焦げたよー。	シチューが 焦[こ]げたよー。	しちゅー が こげた よー	
\\	シチューが
\\	よー。			
\\	英雄	英雄[えいゆう]	えいゆう	
\\	彼は国の英雄ですね。	彼[かれ]は 国[くに]の 英雄[えいゆう]ですね。	かれ は くに の えいゆう です ね	
\\	彼[かれ]は 国[くに]の
\\	ですね。			
\\	雄	雄[おす]	おす	
\\	うちの猫は雄です。	うちの 猫[ねこ]は 雄[おす]です。	うち の ねこ は おす です	
\\	うちの 猫[ねこ]は
\\	です。			
\\	獲得	獲得[かくとく]	かくとく	
\\	その選手の獲得賞金は2億円だって。	その 選手[せんしゅ]の 獲得[かくとく] 賞金[しょうきん]は 2億円[におくえん]だって。	その せんしゅ の かくとく しょうきん は におくえん だって	
\\	その 選手[せんしゅ]の
\\	賞金[しょうきん]は 2億円[におくえん]だって。			
\\	収穫	収穫[しゅうかく]	しゅうかく	
\\	お米の収穫が始まったね。	お 米[こめ]の 収穫[しゅうかく]が 始[はじ]まったね。	おこめ の しゅうかく が はじまった ね	
\\	お 米[こめ]の
\\	が 始[はじ]まったね。			
\\	稲	稲[いね]	いね	
\\	日本は稲の品種が多いよ。	日本[にっぽん]は 稲[いね]の 品種[ひんしゅ]が 多[おお]いよ。	にっぽん は いね の ひんしゅ が おおい よ	
\\	日本[にっぽん]は
\\	の 品種[ひんしゅ]が 多[おお]いよ。			
\\	きれ	きれ	きれ	
\\	余ったきれで人形を作りましょう。	余[あま]ったきれで 人形[にんぎょう]を 作[つく]りましょう。	あまった きれ で にんぎょう を つくりましょう	
\\	余[あま]った
\\	で 人形[にんぎょう]を 作[つく]りましょう。			
\\	稼ぐ	稼[かせ]ぐ	かせぐ	
\\	彼はお金をだいぶ稼いだらしいよ。	彼[かれ]はお 金[かね]をだいぶ 稼[かせ]いだらしいよ。	かれ は おかね を だいぶ かせいだ らしい よ	
\\	彼[かれ]はお 金[かね]をだいぶ
\\	らしいよ。			
\\	原稿用紙	原稿用紙[げんこうようし]	げんこうようし	
\\	作文を書くのに原稿用紙を使ったの。	作文[さくぶん]を 書[か]くのに 原稿用紙[げんこうようし]を 使[つか]ったの。	さくぶん を かく の に げんこうようし を つかった の	
\\	作文[さくぶん]を 書[か]くのに
\\	を 使[つか]ったの。			
\\	穏やか	穏[おだ]やか	おだやか	
\\	彼らは穏やかな暮らしをしているの。	彼[かれ]らは 穏[おだ]やかな 暮[く]らしをしているの。	かれら は おだやか な くらし を して い の	
\\	彼[かれ]らは
\\	な 暮[く]らしをしているの。			
\\	隠す	隠[かく]す	かくす	
\\	僕に隠していることは無いですか。	僕[ぼく]に 隠[かく]していることは 無[な]いですか。	ぼく に かくして いる こと は ない です か	
\\	僕[ぼく]に
\\	ことは 無[な]いですか。			
\\	お陰	お 陰[かげ]	おかげ	
\\	先生のお陰で大学に合格できました。	先生[せんせい]のお 陰[かげ]で 大学[だいがく]に 合格[ごうかく]できました。	せんせい の おかげ で だいがく に ごうかく できました	
\\	先生[せんせい]の
\\	で 大学[だいがく]に 合格[ごうかく]できました。			
\\	陰	陰[かげ]	かげ	
\\	陰で少し休憩しましょう。	陰[かげ]で 少[すこ]し 休憩[きゅうけい]しましょう。	かげ で すこし きゅうけい しましょう	
\\	で 少[すこ]し 休憩[きゅうけい]しましょう。			
\\	塊	塊[かたまり]	かたまり	
\\	道に土の塊ができてたよ。	道[みち]に 土[つち]の 塊[かたまり]ができてたよ。	みち に つち の かたまり が できてた よ	
\\	道[みち]に 土[つち]の
\\	ができてたよ。			
\\	くすぐる	くすぐる	くすぐる	
\\	弟に足の裏をくすぐられた。	弟[おとうと]に 足[あし]の 裏[うら]をくすぐられた。	おとうと に あし の うら を くすぐられた	
\\	弟[おとうと]に 足[あし]の 裏[うら]を
\\	卑しい	卑[いや]しい	いやしい	
\\	彼は金に卑しいね。	彼[かれ]は 金[かね]に 卑[いや]しいね。	かれ は かね に いやしい ね	
\\	彼[かれ]は 金[かね]に
\\	ね。			
\\	砕く	砕[くだ]く	くだく	
\\	氷を細かく砕いてください。	氷[こおり]を 細[こま]かく 砕[くだ]いてください。	こおり を こまかく くだいて ください	
\\	氷[こおり]を 細[こま]かく
\\	ください。			
\\	砕ける	砕[くだ]ける	くだける	
\\	石が粉々に砕けましたね。	石[いし]が 粉々[こなごな]に 砕[くだ]けましたね。	いし が こなごな に くだけました ね	
\\	石[いし]が 粉々[こなごな]に
\\	ね。			
\\	基礎	基礎[きそ]	きそ	
\\	今ドイツ語の基礎を学んでいます。	今[いま]ドイツ 語[ご]の 基礎[きそ]を 学[まな]んでいます。	いま どいつご の きそ を まなんで います	
\\	今[いま]ドイツ 語[ご]の
\\	を 学[まな]んでいます。			
\\	貝	貝[かい]	かい	
\\	浜辺できれいな貝を拾いました。	浜辺[はまべ]できれいな 貝[かい]を 拾[ひろ]いました。	はまべ で きれい な かい を ひろいました	
\\	浜辺[はまべ]できれいな
\\	を 拾[ひろ]いました。			
\\	賢い	賢[かしこ]い	かしこい	
\\	彼は冷静で賢い男だね。	彼[かれ]は 冷静[れいせい]で 賢[かしこ]い 男[おとこ]だね。	かれ は れいせい で かしこい おとこ だ ね	
\\	彼[かれ]は 冷静[れいせい]で
\\	男[おとこ]だね。			
\\	頻りに	頻[しき]りに	しきりに	
\\	彼は頻りに時計を気にしていた。	彼[かれ]は 頻[しき]りに 時計[とけい]を 気[き]にしていた。	かれ は しきりに とけい を き に して いた	
\\	彼[かれ]は
\\	時計[とけい]を 気[き]にしていた。			
\\	げらげら	げらげら	げらげら	
\\	彼女はいつまでもげらげら笑っていたな。	彼女[かのじょ]はいつまでもげらげら 笑[わら]っていたな。	かのじょ は いつ まで も げらげら わらって いた な	
\\	彼女[かのじょ]はいつまでも
\\	笑[わら]っていたな。			
\\	頑固	頑固[がんこ]	がんこ	
\\	なんて頑固な子だ。	なんて 頑固[がんこ]な 子[こ]だ。	なんて がんこ な こ だ	
\\	なんて
\\	な 子[こ]だ。			
\\	頂く	頂[いただ]く	いただく	
\\	お客様にお菓子を頂いた。	お 客様[きゃくさま]にお 菓子[かし]を 頂[いただ]いた。	おきゃくさま に おかし を いただいた	
\\	お 客様[きゃくさま]にお 菓子[かし]を
\\	一斉に	一斉[いっせい]に	いっせいに	
\\	彼らは一斉に拍手したね。	彼[かれ]らは 一斉[いっせい]に 拍手[はくしゅ]したね。	かれら は いっせいに はくしゅ した ね	
\\	彼[かれ]らは
\\	拍手[はくしゅ]したね。			
\\	洗剤	洗剤[せんざい]	せんざい	
\\	床に洗剤をこぼしてしまいました。	床[ゆか]に 洗剤[せんざい]をこぼしてしまいました。	ゆか に せんざい を こぼして しまいました	
\\	床[ゆか]に
\\	をこぼしてしまいました。			
\\	真剣	真剣[しんけん]	しんけん	
\\	彼は真剣に話を聞いていました。	彼[かれ]は 真剣[しんけん]に 話[はなし]を 聞[き]いていました。	かれ は しんけん に はなし を きいて いました	
\\	彼[かれ]は
\\	に 話[はなし]を 聞[き]いていました。			
\\	刑務所	刑務所[けいむしょ]	けいむしょ	
\\	彼は二度と刑務所から出ることができないの。	彼[かれ]は 二度[にど]と 刑務所[けいむしょ]から 出[で]ることができないの。	かれ は にどと けいむしょ から でる こと が できない の	
\\	彼[かれ]は 二度[にど]と
\\	から 出[で]ることができないの。			
\\	刑事	刑事[けいじ]	けいじ	
\\	刑事が現場を調べているわ。	刑事[けいじ]が 現場[げんば]を 調[しら]べているわ。	けいじ が げんば を しらべて いる わ	
\\	が 現場[げんば]を 調[しら]べているわ。			
\\	強烈	強烈[きょうれつ]	きょうれつ	
\\	彼女は強烈な個性の持ち主ですよ。	彼女[かのじょ]は 強烈[きょうれつ]な 個性[こせい]の 持[も]ち 主[ぬし]ですよ。	かのじょ は きょうれつ な こせい の もちぬし です よ	
\\	彼女[かのじょ]は
\\	な 個性[こせい]の 持[も]ち 主[ぬし]ですよ。			
\\	ゴールデンウィーク	ゴールデンウィーク	ゴールデンウィーク	
\\	ゴールデンウィークに海外旅行をします。	ゴールデンウィークに 海外旅行[かいがい りょこう]をします。	ごーるでんうぃーく に かいがい りょこう を します	
\\	に 海外旅行[かいがい りょこう]をします。			
\\	獣	獣[けもの]	けもの	
\\	彼は獣のような目をしていたな。	彼[かれ]は 獣[けもの]のような 目[め]をしていたな。	かれ は けもの の よう な め を して いた な	
\\	彼[かれ]は
\\	のような 目[め]をしていたな。			
\\	駆ける	駆[か]ける	かける	
\\	子供たちは広場に駆けて行ったわよ。	子供[こども]たちは 広場[ひろば]に 駆[か]けて 行[い]ったわよ。	こどもたち は ひろば に かけて いった わ よ	
\\	子供[こども]たちは 広場[ひろば]に
\\	行[い]ったわよ。			
\\	駆け足	駆[か]け 足[あし]	かけあし	
\\	駅まで駆け足で行ったよ。	駅[えき]まで 駆[か]け 足[あし]で 行[い]ったよ。	えき まで かけあし で いった よ	
\\	駅[えき]まで
\\	で 行[い]ったよ。			
\\	丘	丘[おか]	おか	
\\	丘の上にホテルが建ちましたね。	丘[おか]の 上[うえ]にホテルが 建[た]ちましたね。	おか の うえ に ほてる が たちました ね	
\\	の 上[うえ]にホテルが 建[た]ちましたね。			
\\	官庁	官庁[かんちょう]	かんちょう	
\\	その古い建物は官庁です。	その 古[ふる]い 建物[たてもの]は 官庁[かんちょう]です。	その ふるい たてもの は かんちょう です	
\\	その 古[ふる]い 建物[たてもの]は
\\	です。			
\\	県庁	県庁[けんちょう]	けんちょう	
\\	あの白い建物が県庁です。	あの 白[しろ]い 建物[たてもの]が 県庁[けんちょう]です。	あの しろい たてもの が けんちょう です	
\\	あの 白[しろ]い 建物[たてもの]が
\\	です。			
\\	擦る	擦[こす]る	こする	
\\	冷えた手を擦って温めた。	冷[ひ]えた 手[て]を 擦[こす]って 温[あたた]めた。	ひえた て を こすって あたためた	
\\	冷[ひ]えた 手[て]を
\\	温[あたた]めた。			
\\	こたつ	こたつ	こたつ	
\\	寒いのでこたつを出しました。	寒[さむ]いのでこたつを 出[だ]しました。	さむい の で こたつ を だしました	
\\	寒[さむ]いので
\\	を 出[だ]しました。			
\\	汗	汗[あせ]	あせ	
\\	彼は額に汗をかいていたの。	彼[かれ]は 額[ひたい]に 汗[あせ]をかいていたの。	かれ は ひたい に あせ を かいていた の	
\\	彼[かれ]は 額[ひたい]に
\\	をかいていたの。			
\\	軸	軸[じく]	じく	
\\	この線を軸にして図形を回転してください。	この 線[せん]を 軸[じく]にして 図形[ずけい]を 回転[かいてん]してください。	この せん を じく に して ずけい を かいてん して ください	
\\	この 線[せん]を
\\	にして 図形[ずけい]を 回転[かいてん]してください。			
\\	後輩	後輩[こうはい]	こうはい	
\\	彼は大学の後輩です。	彼[かれ]は 大学[だいがく]の 後輩[こうはい]です。	かれ は だいがく の こうはい です	
\\	彼[かれ]は 大学[だいがく]の
\\	です。			
\\	香り	香[かお]り	かおり	
\\	この花はいい香りがしますね。	この 花[はな]はいい 香[かお]りがしますね。	この はな は いい かおり が します ね	
\\	この 花[はな]はいい
\\	がしますね。			
\\	香水	香水[こうすい]	こうすい	
\\	香水のいい香りがした。	香水[こうすい]のいい 香[かお]りがした。	こうすい の いい かおり が した 。	
\\	のいい 香[かお]りがした。			
\\	合唱	合唱[がっしょう]	がっしょう	
\\	私たちは校歌を合唱したの。	私[わたし]たちは 校歌[こうか]を 合唱[がっしょう]したの。	わたしたち は こうか を がっしょう した の	
\\	私[わたし]たちは 校歌[こうか]を
\\	したの。			
\\	結晶	結晶[けっしょう]	けっしょう	
\\	雪の結晶にはいろいろな形があります。	雪[ゆき]の 結晶[けっしょう]にはいろいろな 形[かたち]があります。	ゆき の けっしょう に は いろいろ な かたち が あります	
\\	雪[ゆき]の
\\	にはいろいろな 形[かたち]があります。			
\\	ことわざ	ことわざ	ことわざ	
\\	日本のことわざをいくつくらい知っていますか。	日本[にほん]のことわざをいくつくらい 知[し]っていますか。	にほん の ことわざ を いくつ くらい しって います か	
\\	日本[にほん]の
\\	をいくつくらい 知[し]っていますか。			
\\	敬う	敬[うやま]う	うやまう	
\\	両親を敬うことは大切です。	両親[りょうしん]を 敬[うやま]うことは 大切[たいせつ]です。	りょうしん を うやまう こと は たいせつ です	
\\	両親[りょうしん]を
\\	ことは 大切[たいせつ]です。			
\\	座敷	座敷[ざしき]	ざしき	
\\	明日はお座敷での宴会になります。	明日[あした]はお 座敷[ざしき]での 宴会[えんかい]になります。	あした は おざしき で の えんかい に なります	
\\	明日[あした]はお
\\	での 宴会[えんかい]になります。			
\\	敷金	敷金[しききん]	しききん	
\\	マンションの敷金を払いました。	マンションの 敷金[しききん]を 払[はら]いました。	まんしょん の しききん を はらいました	
\\	マンションの
\\	を 払[はら]いました。			
\\	劣る	劣[おと]る	おとる	
\\	私は体力では誰にも劣りません。	私[わたし]は 体力[たいりょく]では 誰[だれ]にも 劣[おと]りません。	わたし は たいりょく で は だれ に も おとりません	
\\	私[わたし]は 体力[たいりょく]では 誰[だれ]にも
\\	勘定	勘定[かんじょう]	かんじょう	
\\	お勘定をして下さい。	お 勘定[かんじょう]をして 下[くだ]さい。	おかんじょう を して ください	
\\	お
\\	をして 下[くだ]さい。			
\\	勘	勘[かん]	かん	
\\	女の勘を甘く見てはいけません。	女[おんな]の 勘[かん]を 甘[あま]く 見[み]てはいけません。	おんな の かん を あまく みて は いけません	
\\	女[おんな]の
\\	を 甘[あま]く 見[み]てはいけません。			
\\	勘違い	勘違[かんちが]い	かんちがい	
\\	待ち合わせは2時だと勘違いしていました。	待[ま]ち 合[あ]わせは 2時[にじ]だと 勘違[かんちが]いしていました。	まちあわせ は にじ だ と かんちがい して いました	
\\	待[ま]ち 合[あ]わせは 2時[にじ]だと
\\	していました。			
\\	霧	霧[きり]	きり	
\\	霧の深い夜のことでした。	霧[きり]の 深[ふか]い 夜[よる]のことでした。	きり の ふかい よる の こと でした	
\\	の 深[ふか]い 夜[よる]のことでした。			
\\	これから	これから	これから	
\\	これからはもっと気を付けます。	これからはもっと 気[き]を 付[つ]けます。	これから は もっと き を つけます	
\\	はもっと 気[き]を 付[つ]けます。			
\\	霜	霜[しも]	しも	
\\	今朝は庭の草に霜が降りていたの。	今朝[けさ]は 庭[にわ]の 草[くさ]に 霜[しも]が 降[お]りていたの。	けさ は にわ の くさ に しも が おりて いた の	
\\	今朝[けさ]は 庭[にわ]の 草[くさ]に
\\	が 降[お]りていたの。			
\\	公衆	公衆[こうしゅう]	こうしゅう	
\\	彼は公衆電話を探したの。	彼[かれ]は 公衆[こうしゅう] 電話[でんわ]を 探[さが]したの。	かれ は こうしゅう でんわ を さがした の	
\\	彼[かれ]は
\\	電話[でんわ]を 探[さが]したの。			
\\	衆議院	衆議院[しゅうぎいん]	しゅうぎいん	
\\	予算案が衆議院を通過したな。	予算案[よさんあん]が 衆議院[しゅうぎいん]を 通過[つうか]したな。	よさんあん が しゅうぎいん を つうか した な	
\\	予算案[よさんあん]が
\\	を 通過[つうか]したな。			
\\	観衆	観衆[かんしゅう]	かんしゅう	
\\	スタジアムは観衆で満員だったよ。	スタジアムは 観衆[かんしゅう]で 満員[まんいん]だったよ。	すたじあむ は かんしゅう で まんいん だった よ	
\\	スタジアムは
\\	で 満員[まんいん]だったよ。			
\\	群衆	群衆[ぐんしゅう]	ぐんしゅう	
\\	彼はマイクで群衆に話しかけたんだ。	彼[かれ]はマイクで 群衆[ぐんしゅう]に 話[はな]しかけたんだ。	かれ は まいく で ぐんしゅう に はなしかけた ん だ	
\\	彼[かれ]はマイクで
\\	に 話[はな]しかけたんだ。			
\\	暑中見舞い	暑中見舞[しょちゅうみま]い	しょちゅうみまい	
\\	先生に暑中見舞いを出しました。	先生[せんせい]に 暑中見舞[しょちゅうみま]いを 出[だ]しました。	せんせい に しょちゅうみまい を だしました	
\\	先生[せんせい]に
\\	を 出[だ]しました。			
\\	お仕舞い	お 仕舞[しま]い	おしまい	
\\	話はこれでお仕舞いです。	話[はなし]はこれでお 仕舞[しま]いです。	はなし は これ で おしまい です	
\\	話[はなし]はこれで
\\	です。			
\\	さす	さす	さす	
\\	自転車に油をさしたの。	自転車[じてんしゃ]に 油[あぶら]をさしたの。	じてんしゃ に あぶら を さした の 。	
\\	自転車[じてんしゃ]に 油[あぶら]を
\\	の。			
\\	金銭	金銭[きんせん]	きんせん	
\\	金銭のトラブルには関わりたくありません。	金銭[きんせん]のトラブルには 関[かか]わりたくありません。	きんせん の とらぶる に は かかわりたく ありません	
\\	のトラブルには 関[かか]わりたくありません。			
\\	児童	児童[じどう]	じどう	
\\	ここは児童の通学路です。	ここは 児童[じどう]の 通学路[つうがくろ]です。	ここ は じどう の つうがくろ です	
\\	ここは
\\	の 通学路[つうがくろ]です。			
\\	埋める	埋[う]める	うめる	
\\	庭に穴を掘ってそれを埋めました。	庭[にわ]に 穴[あな]を 掘[ほ]ってそれを 埋[う]めました。	にわ に あな を ほって それ を うめました	
\\	庭[にわ]に 穴[あな]を 掘[ほ]ってそれを
\\	埋める	埋[うず]める	うずめる	
\\	パレードと観衆が道を埋めていたよ。	パレードと 観衆[かんしゅう]が 道[みち]を 埋[うず]めていたよ。	ぱれーど と かんしゅう が みち を うずめて いた よ	
\\	パレードと 観衆[かんしゅう]が 道[みち]を
\\	よ。			
\\	墨	墨[すみ]	すみ	
\\	服に墨がついちゃった。	服[ふく]に 墨[すみ]がついちゃった。	ふく に すみ が ついちゃった	
\\	服[ふく]に
\\	がついちゃった。			
\\	講堂	講堂[こうどう]	こうどう	
\\	全員、講堂に集まってください。	全員[ぜんいん]、 講堂[こうどう]に 集[あつ]まってください。	ぜんいん こうどう に あつまって ください	
\\	全員[ぜんいん]、
\\	に 集[あつ]まってください。			
\\	奨学金	奨学金[しょうがくきん]	しょうがくきん	
\\	彼女は奨学金で大学に行きました。	彼女[かのじょ]は 奨学金[しょうがくきん]で 大学[だいがく]に 行[い]きました。	かのじょ は しょうがくきん で だいがく に いきました	
\\	彼女[かのじょ]は
\\	で 大学[だいがく]に 行[い]きました。			
\\	[じぇいあーる]	じぇいあーる	
\\	私は通勤に
\\	を使うんだ。	私[わたし]は 通勤[つうきん]に 
\\	[じぇいあーる]を 使[つか]うんだ。	わたし は つうきん に じぇいあーる を つかう ん だ	
\\	私[わたし]は 通勤[つうきん]に
\\	を 使[つか]うんだ。			
\\	狂う	狂[くる]う	くるう	
\\	この時計はすぐ狂うの。	この 時計[とけい]はすぐ 狂[くる]うの。	この とけい は すぐ くるう の	
\\	この 時計[とけい]はすぐ
\\	の。			
\\	猿	猿[さる]	さる	
\\	山で猿の親子を見ました。	山[やま]で 猿[さる]の 親子[おやこ]を 見[み]ました。	やま で さる の おやこ を みました	
\\	山[やま]で
\\	の 親子[おやこ]を 見[み]ました。			
\\	地獄	地獄[じごく]	じごく	
\\	地震の後、街は地獄のようだったよ。	地震[じしん]の 後[あと]、 街[まち]は 地獄[じごく]のようだったよ。	じしん の あと まち は じごく の よう だった よ	
\\	地震[じしん]の 後[あと]、 街[まち]は
\\	のようだったよ。			
\\	章	章[しょう]	しょう	
\\	この本の第6章が特に好きです。	この 本[ほん]の 第6[だいろく] 章[しょう]が 特[とく]に 好[す]きです。	この ほん の だいろくしょう が とくに すき です	
\\	この 本[ほん]の 第6[だいろく]
\\	が 特[とく]に 好[す]きです。			
\\	頑丈	頑丈[がんじょう]	がんじょう	
\\	錠を頑丈なものに替えました。	錠[じょう]を 頑丈[がんじょう]なものに 替[か]えました。	じょう を がんじょう な もの に かえました	
\\	錠[じょう]を
\\	なものに 替[か]えました。			
\\	親戚	親戚[しんせき]	しんせき	
\\	祖父の家に親戚が集まったんだ。	祖父[そふ]の 家[いえ]に 親戚[しんせき]が 集[あつ]まったんだ。	そふ の いえ に しんせき が あつまった ん だ	
\\	祖父[そふ]の 家[いえ]に
\\	が 集[あつ]まったんだ。			
\\	祈る	祈[いの]る	いのる	
\\	皆が人質の無事を祈っているわよ。	皆[みんな]が 人質[ひとじち]の 無事[ぶじ]を 祈[いの]っているわよ。	みんな が ひとじち の ぶじ を いのって いる わ よ	
\\	皆[みんな]が 人質[ひとじち]の 無事[ぶじ]を
\\	わよ。			
\\	しびれる	しびれる	しびれる	
\\	足がしびれた。	足[あし]がしびれた。	あし が しびれた	
\\	足[あし]が
\\	祈り	祈[いの]り	いのり	
\\	その日、国民は平和への祈りを捧げるの。	その 日[ひ]、 国民[こくみん]は 平和[へいわ]への 祈[いの]りを 捧[ささ]げるの。	その ひ こくみん は へいわ へ の いのり を ささげる の	
\\	その 日[ひ]、 国民[こくみん]は 平和[へいわ]への
\\	を 捧[ささ]げるの。			
\\	襟	襟[えり]	えり	
\\	襟の大きいコートを買いました。	襟[えり]の 大[おお]きいコートを 買[か]いました。	えり の おおきい こーと を かいました	
\\	の 大[おお]きいコートを 買[か]いました。			
\\	先祖	先祖[せんぞ]	せんぞ	
\\	神道では先祖を大切にします。	神道[しんとう]では 先祖[せんぞ]を 大切[たいせつ]にします。	しんとう で は せんぞ を たいせつ に します	
\\	神道[しんとう]では
\\	を 大切[たいせつ]にします。			
\\	粗筋	粗筋[あらすじ]	あらすじ	
\\	その映画はどんな粗筋ですか。	その 映画[えいが]はどんな 粗筋[あらすじ]ですか。	その えいが は どんな あらすじ です か	
\\	その 映画[えいが]はどんな
\\	ですか。			
\\	酢	酢[す]	す	
\\	お酢を入れすぎて酸っぱい。	お 酢[す]を 入[い]れすぎて 酸[す]っぱい。	お す を いれすぎて すっぱい	
\\	お
\\	を 入[い]れすぎて 酸[す]っぱい。			
\\	郊外	郊外[こうがい]	こうがい	
\\	私は郊外に家を買いました。	私[わたし]は 郊外[こうがい]に 家[いえ]を 買[か]いました。	わたし は こうがい に いえ を かいました	
\\	私[わたし]は
\\	に 家[いえ]を 買[か]いました。			
\\	近郊	近郊[きんこう]	きんこう	
\\	彼らは東京近郊に住んでいる。	彼[かれ]らは 東京[とうきょう] 近郊[きんこう]に 住[す]んでいる。	かれら は とうきょう きんこう に すんで いる	
\\	彼[かれ]らは 東京[とうきょう]
\\	に 住[す]んでいる。			
\\	邪魔	邪魔[じゃま]	じゃま	
\\	邪魔です、どいてください。	邪魔[じゃま]です、どいてください。	じゃま です どいて ください	
\\	です、どいてください。			
\\	じゃぶじゃぶ	じゃぶじゃぶ	じゃぶじゃぶ	
\\	彼は顔をじゃぶじゃぶ洗ったの。	彼[かれ]は 顔[かお]をじゃぶじゃぶ 洗[あら]ったの。	かれ は かお を じゃぶじゃぶ あらった の	
\\	彼[かれ]は 顔[かお]を
\\	洗[あら]ったの。			
\\	お年玉	お 年玉[としだま]	おとしだま	
\\	甥と姪にお年玉をあげたの。	甥[おい]と 姪[めい]にお 年玉[としだま]をあげたの。	おい と めい に おとしだま を あげた の	
\\	甥[おい]と 姪[めい]に
\\	をあげたの。			
\\	改善	改善[かいぜん]	かいぜん	
\\	彼は食生活を改善しました。	彼[かれ]は 食生活[しょくせいかつ]を 改善[かいぜん]しました。	かれ は しょくせいかつ を かいぜん しました	
\\	彼[かれ]は 食生活[しょくせいかつ]を
\\	しました。			
\\	親善	親善[しんぜん]	しんぜん	
\\	これからも両国の親善を深めましょう。	これからも 両国[りょうこく]の 親善[しんぜん]を 深[ふか]めましょう。	これから も りょうこく の しんぜん を ふかめましょう	
\\	これからも 両国[りょうこく]の
\\	を 深[ふか]めましょう。			
\\	網	網[あみ]	あみ	
\\	少年は網でその蝶を捕まえた。	少年[しょうねん]は 網[あみ]でその 蝶[ちょう]を 捕[つか]まえた。	しょうねん は あみ で その ちょう を つかまえた	
\\	少年[しょうねん]は
\\	でその 蝶[ちょう]を 捕[つか]まえた。			
\\	縛る	縛[しば]る	しばる	
\\	古新聞をひもで縛ったよ。	古新聞[ふるしんぶん]をひもで 縛[しば]ったよ。	ふるしんぶん を ひも で しばった よ 。	
\\	古新聞[ふるしんぶん]をひもで
\\	よ。			
\\	絞る	絞[しぼ]る	しぼる	
\\	布をもっと固く絞りなさい。	布[ぬの]をもっと 固[かた]く 絞[しぼ]りなさい。	ぬの を もっと かたく しぼりなさい	
\\	布[ぬの]をもっと 固[かた]く
\\	紺	紺[こん]	こん	
\\	日本の制服は紺が多いね。	日本[にほん]の 制服[せいふく]は 紺[こん]が 多[おお]いね。	にほん の せいふく は こん が おおい ね	
\\	日本[にほん]の 制服[せいふく]は
\\	が 多[おお]いね。			
\\	じゃんじゃん	じゃんじゃん	じゃんじゃん	
\\	電話がじゃんじゃん掛かってきたな。	電話[でんわ]がじゃんじゃん 掛[か]かってきたな。	でんわ が じゃんじゃん かかって きた な	
\\	電話[でんわ]が
\\	掛[か]かってきたな。			
\\	紅葉	紅葉[こうよう]	こうよう	
\\	この山は紅葉がとても美しい。	この 山[やま]は 紅葉[こうよう]がとても 美[うつく]しい。	この やま は こうよう が とても うつくしい	
\\	この 山[やま]は
\\	がとても 美[うつく]しい。			
\\	口紅	口紅[くちべに]	くちべに	
\\	赤い口紅を買いました。	赤[あか]い 口紅[くちべに]を 買[か]いました。	あかい くちべに を かいました	
\\	赤[あか]い
\\	を 買[か]いました。			
\\	梅	梅[うめ]	うめ	
\\	梅の花が咲きました。	梅[うめ]の 花[はな]が 咲[さ]きました。	うめ の はな が さきました	
\\	の 花[はな]が 咲[さ]きました。			
\\	梅干	梅干[うめぼし]	うめぼし	
\\	うちでは、朝食には必ず梅干が出ます。	うちでは、 朝食[ちょうしょく]には 必[かなら]ず 梅干[うめぼし]が 出[で]ます。	うち で は ちょうしょく に は かならず うめぼし が でます	
\\	うちでは、 朝食[ちょうしょく]には 必[かなら]ず
\\	が 出[で]ます。			
\\	巣	巣[す]	す	
\\	アリは土の中に巣を作ります。	アリは 土[つち]の 中[なか]に 巣[す]を 作[つく]ります。	あり は つち の なか に す を つくります	
\\	アリは 土[つち]の 中[なか]に
\\	を 作[つく]ります。			
\\	囲碁	囲碁[いご]	いご	
\\	彼の趣味は囲碁です。	彼[かれ]の 趣味[しゅみ]は 囲碁[いご]です。	かれ の しゅみ は いご です	
\\	彼[かれ]の 趣味[しゅみ]は
\\	です。			
\\	嘘つき	嘘[うそ]つき	うそつき	
\\	嘘つきは泥棒の始まりよ。	嘘[うそ]つきは 泥棒[どろぼう]の 始[はじ]まりよ。	うそつき は どろぼう の はじまり よ 。	
\\	は 泥棒[どろぼう]の 始[はじ]まりよ。			
\\	しわ	しわ	しわ	
\\	スカートにしわがよっているよ。	スカートにしわがよっているよ。	すかーと に しわ が よって いる よ	
\\	スカートに
\\	がよっているよ。			
\\	考慮	考慮[こうりょ]	こうりょ	
\\	あなたの事情を考慮して予定をたてました。	あなたの 事情[じじょう]を 考慮[こうりょ]して 予定[よてい]をたてました。	あなた の じじょう を こうりょ して よてい を たてました	
\\	あなたの 事情[じじょう]を
\\	して 予定[よてい]をたてました。			
\\	癖	癖[くせ]	くせ	
\\	爪をかむ癖は直した方がいい。	爪[つめ]をかむ 癖[くせ]は 直[なお]した 方[ほう]がいい。	つめ を かむ くせ は なおした ほう が いい	
\\	爪[つめ]をかむ
\\	は 直[なお]した 方[ほう]がいい。			
\\	下痢	下痢[げり]	げり	
\\	古いお寿司を食べて下痢をしてしまったんだ。	古[ふる]いお 寿司[すし]を 食[た]べて 下痢[げり]をしてしまったんだ。	ふるい おすし を たべて げり を して しまった ん だ	
\\	古[ふる]いお 寿司[すし]を 食[た]べて
\\	をしてしまったんだ。			
\\	崖	崖[がけ]	がけ	
\\	大雨で崖が崩れたんだ。	大雨[おおあめ]で 崖[がけ]が 崩[くず]れたんだ。	おおあめ で がけ が くずれた ん だ	
\\	大雨[おおあめ]で
\\	が 崩[くず]れたんだ。			
\\	嵐	嵐[あらし]	あらし	
\\	嵐で庭の木が折れたよ。	嵐[あらし]で 庭[にわ]の 木[き]が 折[お]れたよ。	あらし で にわ の き が おれた よ 。	
\\	で 庭[にわ]の 木[き]が 折[お]れたよ。			
\\	海峡	海峡[かいきょう]	かいきょう	
\\	その海峡に橋が掛けられました。	その 海峡[かいきょう]に 橋[はし]が 掛[か]けられました。	その かいきょう に はし が かけられました	
\\	その
\\	に 橋[はし]が 掛[か]けられました。			
\\	噂	噂[うわさ]	うわさ	
\\	その噂は本当ですか。	その 噂[うわさ]は 本当[ほんとう]ですか。	その うわさ は ほんとう です か	
\\	その
\\	は 本当[ほんとう]ですか。			
\\	田舎	田舎[いなか]	いなか	
\\	私は毎年夏に田舎に帰ります。	私[わたし]は 毎年夏[まいとし なつ]に 田舎[いなか]に 帰[かえ]ります。	わたし は まいとし なつ に いなか に かえります	
\\	私[わたし]は 毎年夏[まいとし なつ]に
\\	に 帰[かえ]ります。			
\\	ずるがしこい	ずるがしこい	ずるがしこい	
\\	あいつはずるがしこい顔をしているね。	あいつはずるがしこい 顔[かお]をしているね。	あいつ は ずるがしこい かお を して いる ね	
\\	あいつは
\\	顔[かお]をしているね。			
\\	校舎	校舎[こうしゃ]	こうしゃ	
\\	古い校舎の修理が必要です。	古[ふる]い 校舎[こうしゃ]の 修理[しゅうり]が 必要[ひつよう]です。	ふるい こうしゃ の しゅうり が ひつよう です	
\\	古[ふる]い
\\	の 修理[しゅうり]が 必要[ひつよう]です。			
\\	お嬢さん	お 嬢[じょう]さん	おじょうさん	
\\	彼はお嬢さんと一緒でした。	彼[かれ]はお 嬢[じょう]さんと 一緒[いっしょ]でした。	かれ は おじょうさん と いっしょ でした	
\\	彼[かれ]は
\\	と 一緒[いっしょ]でした。			
\\	娯楽	娯楽[ごらく]	ごらく	
\\	テレビは彼のいちばんの娯楽です。	テレビは 彼[かれ]のいちばんの 娯楽[ごらく]です。	てれび は かれ の いちばん の ごらく です	
\\	テレビは 彼[かれ]のいちばんの
\\	です。			
\\	生涯	生涯[しょうがい]	しょうがい	
\\	彼は80年の生涯を閉じました。	彼[かれ]は 80年[はちじゅうねん]の 生涯[しょうがい]を 閉[と]じました。	かれ は はちじゅうねん の しょうがい を とじました	
\\	彼[かれ]は 80年[はちじゅうねん]の
\\	を 閉[と]じました。			
\\	汽車	汽車[きしゃ]	きしゃ	
\\	汽車で街まで行った。	汽車[きしゃ]で 街[まち]まで 行[い]った。	きしゃ で まち まで いった	
\\	で 街[まち]まで 行[い]った。			
\\	賭ける	賭[か]ける	かける	
\\	彼は新しい仕事に人生を賭けている。	彼[かれ]は 新[あたら]しい 仕事[しごと]に 人生[じんせい]を 賭[か]けている。	かれ は あたらしい しごと に じんせい を かけて いる	
\\	彼[かれ]は 新[あたら]しい 仕事[しごと]に 人生[じんせい]を
\\	蛍光灯	蛍光灯[けいこうとう]	けいこうとう	
\\	古い蛍光灯を取り替えてください。	古[ふる]い 蛍光灯[けいこうとう]を 取[と]り 替[か]えてください。	ふるい けいこうとう を とりかえて ください	
\\	古[ふる]い
\\	を 取[と]り 替[か]えてください。			
\\	せいぜい	せいぜい	せいぜい	
\\	どんなに頑張っても、せいぜい3位くらいにしかなれないだろうな。	どんなに 頑張[がんば]っても、せいぜい 3位[さんい]くらいにしかなれないだろうな。	どんな に がんばって も せいぜい さんい くらい に しか なれない だろう な	
\\	どんなに 頑張[がんば]っても、
\\	3位[さんい]くらいにしかなれないだろうな。			
\\	車掌	車掌[しゃしょう]	しゃしょう	
\\	車掌さんが車内を回って来たよ。	車掌[しゃしょう]さんが 車内[しゃない]を 回[まわ]って 来[き]たよ。	しゃしょう さん が しゃない を まわって きた よ	
\\	さんが 車内[しゃない]を 回[まわ]って 来[き]たよ。			
\\	芋	芋[いも]	いも	
\\	お芋の料理はお好きですか。	お 芋[いも]の 料理[りょうり]はお 好[す]きですか。	お いも の りょうり は お すき です か	
\\	お
\\	の 料理[りょうり]はお 好[す]きですか。			
\\	菊	菊[きく]	きく	
\\	菊の花を買ってきました。	菊[きく]の 花[はな]を 買[か]ってきました。	きく の はな を かって きました	
\\	の 花[はな]を 買[か]ってきました。			
\\	軽蔑	軽蔑[けいべつ]	けいべつ	
\\	彼女は彼を軽蔑していたの。	彼女[かのじょ]は 彼[かれ]を 軽蔑[けいべつ]していたの。	かのじょ は かれ を けいべつ して いた の	
\\	彼女[かのじょ]は 彼[かれ]を
\\	していたの。			
\\	揚げる	揚[あ]げる	あげる	
\\	彼女は夕食に天ぷらを揚げました。	彼女[かのじょ]は 夕食[ゆうしょく]に 天[てん]ぷらを 揚[あ]げました。	かのじょ は ゆうしょく に てんぷら を あげました	
\\	彼女[かのじょ]は 夕食[ゆうしょく]に 天[てん]ぷらを
\\	諦める	諦[あきら]める	あきらめる	
\\	彼は留学の夢を諦めていないよ。	彼[かれ]は 留学[りゅうがく]の 夢[ゆめ]を 諦[あきら]めていないよ。	かれ は りゅうがく の ゆめ を あきらめて いない よ	
\\	彼[かれ]は 留学[りゅうがく]の 夢[ゆめ]を
\\	よ。			
\\	演奏	演奏[えんそう]	えんそう	
\\	彼女のピアノの演奏は素晴らしいね。	彼女[かのじょ]のピアノの 演奏[えんそう]は 素晴[すば]らしいね。	かのじょ の ぴあの の えんそう は すばらしい ね	
\\	彼女[かのじょ]のピアノの
\\	は 素晴[すば]らしいね。			
\\	ぜいたく	ぜいたく	ぜいたく	
\\	彼女は一生ぜいたくに暮らしたんだ。	彼女[かのじょ]は 一生[いっしょう]ぜいたくに 暮[く]らしたんだ。	かのじょ は いっしょう ぜいたく に くらした ん だ	
\\	彼女[かのじょ]は 一生[いっしょう]
\\	に 暮[く]らしたんだ。			
\\	窮屈	窮屈[きゅうくつ]	きゅうくつ	
\\	この服は窮屈になったな。	この 服[ふく]は 窮屈[きゅうくつ]になったな。	この ふく は きゅうくつ に なった な	
\\	この 服[ふく]は
\\	になったな。			
\\	貨幣	貨幣[かへい]	かへい	
\\	博物館で昔の貨幣を見ました。	博物館[はくぶつかん]で 昔[むかし]の 貨幣[かへい]を 見[み]ました。	はくぶつかん で むかし の かへい を みました	
\\	博物館[はくぶつかん]で 昔[むかし]の
\\	を 見[み]ました。			
\\	真珠	真珠[しんじゅ]	しんじゅ	
\\	母に真珠のネックレスをもらいました。	母[はは]に 真珠[しんじゅ]のネックレスをもらいました。	はは に しんじゅ の ねっくれす を もらいました	
\\	母[はは]に
\\	のネックレスをもらいました。			
\\	故郷	故郷[こきょう]	こきょう	
\\	彼女は久しぶりに故郷に帰りました。	彼女[かのじょ]は 久[ひさ]しぶりに 故郷[こきょう]に 帰[かえ]りました。	かのじょ は ひさしぶり に こきょう に かえりました	
\\	彼女[かのじょ]は 久[ひさ]しぶりに
\\	に 帰[かえ]りました。			
\\	花瓶	花瓶[かびん]	かびん	
\\	クリスタルガラスの花瓶を買いました。	クリスタルガラスの 花瓶[かびん]を 買[か]いました。	くりすたるがらす の かびん を かいました	
\\	クリスタルガラスの
\\	を 買[か]いました。			
\\	一旦	一旦[いったん]	いったん	
\\	疲れたでしょう、一旦休みましょう。	疲[つか]れたでしょう、 一旦[いったん] 休[やす]みましょう。	つかれた でしょう いったん やすみましょう	
\\	疲[つか]れたでしょう、
\\	休[やす]みましょう。			
\\	乞食	乞食[こじき]	こじき	
\\	乞食が公園のベンチで寝ている。	乞食[こじき]が 公園[こうえん]のベンチで 寝[ね]ている。	こじき が こうえん の べんち で ねて いる	
\\	が 公園[こうえん]のベンチで 寝[ね]ている。			
\\	亀	亀[かめ]	かめ	
\\	この池には亀がいますね。	この 池[いけ]には 亀[かめ]がいますね。	この いけ に は かめ が います ね	
\\	この 池[いけ]には
\\	がいますね。			
\\	がくんと	がくんと	がくんと	
\\	今学期は成績ががくんと落ちてしまった。	今学期[こんがっき]は 成績[せいせき]ががくんと 落[お]ちてしまった。	こんがっき は せいせき が がくんと おちて しまった	
\\	今学期[こんがっき]は 成績[せいせき]が
\\	落[お]ちてしまった。			
\\	羨ましい	羨[うらや]ましい	うらやましい	
\\	彼の才能は羨ましい程だね。	彼[かれ]の 才能[さいのう]は 羨[うらや]ましい 程[ほど]だね。	かれ の さいのう は うらやましい ほど だ ね	
\\	彼[かれ]の 才能[さいのう]は
\\	程[ほど]だね。			
\\	漕ぐ	漕[こ]ぐ	こぐ	
\\	彼らは交代でボートを漕いだんだ。	彼[かれ]らは 交代[こうたい]でボートを 漕[こ]いだんだ。	かれら は こうたい で ぼーと を こいだ ん だ	
\\	彼[かれ]らは 交代[こうたい]でボートを
\\	んだ。			
\\	稽古	稽古[けいこ]	けいこ	
\\	相撲の稽古はとても厳しいよ。	相撲[すもう]の 稽古[けいこ]はとても 厳[きび]しいよ。	すもう の けいこ は とても きびしい よ	
\\	相撲[すもう]の
\\	はとても 厳[きび]しいよ。			
\\	歌舞伎	歌舞伎[かぶき]	かぶき	
\\	歌舞伎の芝居を見に行きました。	歌舞伎[かぶき]の 芝居[しばい]を 見[み]に 行[い]きました。	かぶき の しばい を み に いきました 。	
\\	の 芝居[しばい]を 見[み]に 行[い]きました。			
\\	囁く	囁[ささや]く	ささやく	
\\	「この会議は退屈だ」と同僚が私に囁いたの。	「この 会議[かいぎ]は 退屈[たいくつ]だ」と 同僚[どうりょう]が 私[わたし]に 囁[ささや]いたの。	
\\	この かいぎ は たいくつ だ 
\\	と どうりょう が わたし に ささやいた の 。	
\\	「この 会議[かいぎ]は 退屈[たいくつ]だ」と 同僚[どうりょう]が 私[わたし]に
\\	の。			
\\	咳	咳[せき]	せき	
\\	咳が止まらないので病院に行ってきたの。	咳[せき]が 止[と]まらないので 病院[びょういん]に 行[い]ってきたの。	せき が とまらない ので びょういん に いってきた の 。	
\\	が 止[と]まらないので 病院[びょういん]に 行[い]ってきたの。			
\\	噛み付く	噛[か]み 付[つ]く	かみつく	
\\	犬が手に噛み付きました。	犬[いぬ]が 手[て]に 噛[か]み 付[つ]きました。	いぬ が て に かみつきました。	
\\	犬[いぬ]が 手[て]に
\\	きらりと	きらりと	きらりと	
\\	ダイヤモンドがきらりと光った。	ダイヤモンドがきらりと 光[ひか]った。	だいやもんど が きらりと ひかった	
\\	ダイヤモンドが
\\	光[ひか]った。			
\\	屑	屑[くず]	くず	
\\	彼の背広に糸屑がついているわ。	彼[かれ]の 背広[せびろ]に 糸[いと] 屑[くず]がついているわ。	かれ の せびろ に いとくず が ついて いる わ	
\\	彼[かれ]の 背広[せびろ]に 糸[いと]
\\	がついているわ。			
\\	紙屑	紙屑[かみくず]	かみくず	
\\	紙屑は屑入れに入れなさい。	紙屑[かみくず]は 屑入[くず い]れに 入[い]れなさい。	かみくず は くず いれ に いれなさい 。	
\\	は 屑入[くず い]れに 入[い]れなさい。			
\\	掻く	掻[か]く	かく	
\\	背中をお母さんに掻いてもらったの。	背中[せなか]をお 母[かあ]さんに 掻[か]いてもらったの。	せなか を おかあさん に かいて もらった の	
\\	背中[せなか]をお 母[かあ]さんに
\\	の。			
\\	掻き回す	掻[か]き 回[まわ]す	かきまわす	
\\	母は鍋のシチューを掻き回しているよ。	母[はは]は 鍋[なべ]のシチューを 掻[か]き 回[まわ]しているよ。	はは は なべ の しちゅー を かきまわしている よ 。	
\\	母[はは]は 鍋[なべ]のシチューを
\\	よ。			
\\	憧れ	憧[あこが]れ	あこがれ	
\\	海外に住むのは私の憧れです。	海外[かいがい]に 住[す]むのは 私[わたし]の 憧[あこが]れです。	かいがい に すむ の は わたし の あこがれ です	
\\	海外[かいがい]に 住[す]むのは 私[わたし]の
\\	です。			
\\	憧れる	憧[あこが]れる	あこがれる	
\\	彼はパイロットの職に憧れているんだ。	彼[かれ]はパイロットの 職[しょく]に 憧[あこが]れているんだ。	かれ は ぱいろっと の しょく に あこがれて いる ん だ	
\\	彼[かれ]はパイロットの 職[しょく]に
\\	んだ。			
\\	溺れる	溺[おぼ]れる	おぼれる	
\\	彼は溺れている子を助けたんだ。	彼[かれ]は 溺[おぼ]れている 子[こ]を 助[たす]けたんだ。	かれ は おぼれて いる こ を たすけた ん だ	
\\	彼[かれ]は
\\	子[こ]を 助[たす]けたんだ。			
\\	ごくんと	ごくんと	ごくんと	
\\	薬をごくんと飲み込んだんだ。	薬[くすり]をごくんと 飲[の]み 込[こ]んだんだ。	くすり を ごくんと のみこんだ ん だ	
\\	薬[くすり]を
\\	飲[の]み 込[こ]んだんだ。			
\\	御無沙汰	御無沙汰[ごぶさた]	ごぶさた	
\\	長いこと御無沙汰いたしました。	長[なが]いこと 御無沙汰[ごぶさた]いたしました。	ながい こと ごぶさた いたしました	
\\	長[なが]いこと
\\	いたしました。			
\\	汲む	汲[く]む	くむ	
\\	小さなバケツで水を汲んだの。	小[ちい]さなバケツで 水[みず]を 汲[く]んだの。	ちいさ な ばけつ で みず を くんだ の	
\\	小[ちい]さなバケツで 水[みず]を
\\	の。			
\\	苛め	苛[いじ]め	いじめ	
\\	学校での苛めはなかなか減らない。	学校[がっこう]での 苛[いじ]めはなかなか 減[へ]らない。	がっこう で の いじめ は なかなか へらない 。	
\\	学校[がっこう]での
\\	はなかなか 減[へ]らない。			
\\	苛める	苛[いじ]める	いじめる	
\\	動物を苛めてはいけません。	動物[どうぶつ]を 苛[いじ]めてはいけません。	どうぶつ を いじめて は いけません	
\\	動物[どうぶつ]を
\\	はいけません。			
\\	曖昧	曖昧[あいまい]	あいまい	
\\	彼女は曖昧な返事をしたね。	彼女[かのじょ]は 曖昧[あいまい]な 返事[へんじ]をしたね。	かのじょ は あいまい な へんじ を した ね	
\\	彼女[かのじょ]は
\\	な 返事[へんじ]をしたね。			
\\	大晦日	大晦日[おおみそか]	おおみそか	
\\	日本では、大晦日にそばを食べます。	日本[にっぽん]では、 大晦日[おおみそか]にそばを 食[た]べます。	にっぽん で は おおみそか に そば を たべます	
\\	日本[にっぽん]では、
\\	にそばを 食[た]べます。			
\\	臆病	臆病[おくびょう]	おくびょう	
\\	弟は小さい頃は臆病だったんだ。	弟[おとうと]は 小[ちい]さい 頃[ころ]は 臆病[おくびょう]だったんだ。	おとうと は ちいさい ころ は おくびょう だった ん だ	
\\	弟[おとうと]は 小[ちい]さい 頃[ころ]は
\\	だったんだ。			
\\	車椅子	車椅子[くるまいす]	くるまいす	
\\	彼女は車椅子に乗っています。	彼女[かのじょ]は 車椅子[くるまいす]に 乗[の]っています。	かのじょ は くるまいす に のっています 。	
\\	彼女[かのじょ]は
\\	に 乗[の]っています。			
\\	柿	柿[かき]	かき	
\\	庭に柿の実がなりました。	庭[にわ]に 柿[かき]の 実[み]がなりました。	にわ に かき の み が なりました	
\\	庭[にわ]に
\\	の 実[み]がなりました。			
\\	お詫び	お 詫[わ]び	おわび	
\\	誤解があったことをお詫びします。	誤解[ごかい]があったことをお 詫[わ]びします。	ごかい が あった こと を おわび します	
\\	誤解[ごかい]があったことを
\\	します。			
\\	釘	釘[くぎ]	くぎ	
\\	釘を踏んで怪我をしました。	釘[くぎ]を 踏[ふ]んで 怪我[けが]をしました。	くぎ を ふんで けが を しました	
\\	を 踏[ふ]んで 怪我[けが]をしました。			
\\	錆びる	錆[さ]びる	さびる	
\\	包丁が錆びてしまった。	包丁[ほうちょう]が 錆[さ]びてしまった。	ほうちょう が さびて しまった	
\\	包丁[ほうちょう]が
\\	爽やか	爽[さわ]やか	さわやか	
\\	レモンの爽やかな香りがした。	レモンの 爽[さわ]やかな 香[かお]りがした。	れもん の さわやか な かおり が した	
\\	レモンの
\\	な 香[かお]りがした。			
\\	雀	雀[すずめ]	すずめ	
\\	朝は雀の声で目が覚めます。	朝[あさ]は 雀[すずめ]の 声[こえ]で 目[め]が 覚[さ]めます。	あさ は すずめ の こえ で め が さめます	
\\	朝[あさ]は
\\	の 声[こえ]で 目[め]が 覚[さ]めます。			
\\	甥	甥[おい]	おい	
\\	私の甥は3才です。	私[わたし]の 甥[おい]は 3才[さんさい]です。	わたし の おい は さんさい です	
\\	私[わたし]の
\\	は 3才[さんさい]です。			
\\	嗅ぐ	嗅[か]ぐ	かぐ	
\\	犬がお皿の匂いをクンクン嗅いでいるね。	犬[いぬ]がお 皿[さら]の 匂[にお]いをクンクン 嗅[か]いでいるね。	いぬ が おさら の におい を くんくん かいで いる ね	
\\	犬[いぬ]がお 皿[さら]の 匂[にお]いをクンクン
\\	ね。			
\\	炒める	炒[いた]める	いためる	
\\	次に、野菜を炒めてください。	次[つぎ]に、 野菜[やさい]を 炒[いた]めてください。	つぎ に やさい を いためて ください	
\\	次[つぎ]に、 野菜[やさい]を
\\	ください。			
\\	一まず	一[ひと]まず	ひとまず	
\\	一まず休憩しましょう。	一[ひと]まず 休憩[きゅうけい]しましょう。	ひとまず きゅうけい しましょう	
\\	休憩[きゅうけい]しましょう。			
\\	万一	万一[まんいち]	まんいち	
\\	万一のために保険に入ったの。	万一[まんいち]のために 保険[ほけん]に 入[はい]ったの。	まんいち の ため に ほけん に はいった の	
\\	のために 保険[ほけん]に 入[はい]ったの。			
\\	日ごろ	日[ひ]ごろ	ひごろ	
\\	彼女には日ごろからお世話になっています。	彼女[かのじょ]には 日[ひ]ごろからお 世話[せわ]になっています。	かのじょ に は ひごろ から おせわ に なって います	
\\	彼女[かのじょ]には
\\	からお 世話[せわ]になっています。			
\\	日ソ	日[にっ]ソ	にっそ	
\\	当時、日ソ会談が開かれた。	当時[とうじ]、 日[にっ]ソ 会談[かいだん]が 開[ひら]かれた。	とうじ にっそ かいだん が ひらかれた	
\\	当時[とうじ]、
\\	会談[かいだん]が 開[ひら]かれた。			
\\	日	日[にち]	にち	
\\	日仏の共同研究が始まったな。	日[にち] 仏[ふつ]の 共同研究[きょうどう けんきゅう]が 始[はじ]まったな。	にちふつ の きょうどう けんきゅう が はじまった な	
\\	仏[ふつ]の 共同研究[きょうどう けんきゅう]が 始[はじ]まったな。			
\\	データ	データ	データ	
\\	去年のデータを見せてください。	去年[きょねん]のデータを 見[み]せてください。	きょねん の でーた を みせて ください 。	
\\	去年[きょねん]の
\\	を 見[み]せてください。			
\\	日時	日時[にちじ]	にちじ	
\\	試写会の日時を教えてください。	試写会[ししゃかい]の 日時[にちじ]を 教[おし]えてください。	ししゃかい の にちじ を おしえて ください	
\\	試写会[ししゃかい]の
\\	を 教[おし]えてください。			
\\	日日	日日[ひにち]	ひにち	
\\	ミーティングの日日を間違えました。	ミーティングの 日日[ひにち]を 間違[まちが]えました。	みーてぃんぐ の ひにち を まちがえました	
\\	ミーティングの
\\	を 間違[まちが]えました。			
\\	日々	日々[ひび]	ひび	
\\	日々の努力が大切です。	日々[ひび]の 努力[どりょく]が 大切[たいせつ]です。	ひび の どりょく が たいせつ です	
\\	の 努力[どりょく]が 大切[たいせつ]です。			
\\	三日月	三日月[みかづき]	みかづき	
\\	空に三日月が見えました。	空[そら]に 三日月[みかづき]が 見[み]えました。	そら に みかづき が みえました	
\\	空[そら]に
\\	が 見[み]えました。			
\\	月日	月日[つきひ]	つきひ	
\\	月日が経つのは早いものです。	月日[つきひ]が 経[た]つのは 早[はや]いものです。	つきひ が たつ の は はやい もの です	
\\	が 経[た]つのは 早[はや]いものです。			
\\	ただ	ただ	ただ	
\\	私はただ彼女と話したかっただけです。	私[わたし]はただ 彼女[かのじょ]と 話[はな]したかっただけです。	わたし は ただ かのじょ と はなしたかった だけ です	
\\	私[わたし]は
\\	彼女[かのじょ]と 話[はな]したかっただけです。			
\\	土	土[つち]	つち	
\\	土を掘って木を植えました。	土[つち]を 掘[ほ]って 木[き]を 植[う]えました。	つち を ほって き を うえました	
\\	を 掘[ほ]って 木[き]を 植[う]えました。			
\\	年月	年月[としつき]	としつき	
\\	あれから長い年月が経ちました。	あれから 長[なが]い 年月[としつき]が 経[た]ちました。	あれ から ながい としつき が たちました	
\\	あれから 長[なが]い
\\	が 経[た]ちました。			
\\	年月日	年月日[ねんがっぴ]	ねんがっぴ	
\\	申請の年月日を西暦で書いてください。	申請[しんせい]の 年月日[ねんがっぴ]を 西暦[せいれき]で 書[か]いてください。	しんせい の ねんがっぴ を せいれき で かいて ください	
\\	申請[しんせい]の
\\	を 西暦[せいれき]で 書[か]いてください。			
\\	年月	年月[ねんげつ]	ねんげつ	
\\	そのお寺は長い年月をかけて建てられた。	そのお 寺[てら]は 長[なが]い 年月[ねんげつ]をかけて 建[た]てられた。	その おてら は ながい ねんげつ を かけて たてられた	
\\	そのお 寺[てら]は 長[なが]い
\\	をかけて 建[た]てられた。			
\\	年々	年々[ねんねん]	ねんねん	
\\	東京の人口は年々増えています。	東京[とうきょう]の 人口[じんこう]は 年々[ねんねん] 増[ふ]えています。	とうきょう の じんこう は ねんねん ふえて います	
\\	東京[とうきょう]の 人口[じんこう]は
\\	増[ふ]えています。			
\\	もっとも	もっとも	もっとも	
\\	彼の意見はもっともです。	彼[かれ]の 意見[いけん]はもっともです。	かれ の いけん は もっとも です	
\\	彼[かれ]の 意見[いけん]は
\\	です。			
\\	何か	何[なに]か	なにか	
\\	何か質問はありますか。	何[なに]か 質問[しつもん]はありますか。	なにか しつもん は あります か	
\\	質問[しつもん]はありますか。			
\\	何とか	何[なん]とか	なんとか	
\\	何とかお願いします。	何[なん]とかお 願[ねが]いします。	なんとか おねがい します	
\\	お 願[ねが]いします。			
\\	何より	何[なに]より	なにより	
\\	ご無事で何よりです。	ご 無事[ぶじ]で 何[なに]よりです。	ごぶじ で なにより です	
\\	ご 無事[ぶじ]で
\\	です。			
\\	何と	何[なん]と	なんと	
\\	何と彼らは結婚したそうです。	何[なん]と 彼[かれ]らは 結婚[けっこん]したそうです。	なんと かれら は けっこん した そう です	
\\	彼[かれ]らは 結婚[けっこん]したそうです。			
\\	何となく	何[なん]となく	なんとなく	
\\	パーティーに行くのは何となく気が進まないな。	パーティーに 行[い]くのは 何[なん]となく 気[き]が 進[すす]まないな。	ぱーてぃー に いく の は なんとなく き が すすまない な	
\\	パーティーに 行[い]くのは
\\	気[き]が 進[すす]まないな。			
\\	何だか	何[なん]だか	なんだか	
\\	茶柱が立って何だか少し幸せな気分です。	茶柱[ちゃばしら]が 立[た]って 何[なん]だか 少[すこ]し 幸[しあわ]せな 気分[きぶん]です。	ちゃばしら が たって なんだか すこし しあわせ な きぶん です	
\\	茶柱[ちゃばしら]が 立[た]って
\\	少[すこ]し 幸[しあわ]せな 気分[きぶん]です。			
\\	メーカー	メーカー	メーカー	
\\	彼は靴のメーカーで働いています。	彼[かれ]は 靴[くつ]のメーカーで 働[はたら]いています。	かれ は くつ の めーかー で はたらいて います	
\\	彼[かれ]は 靴[くつ]の
\\	で 働[はたら]いています。			
\\	何で	何[なん]で	なんで	
\\	彼女は何で来ないんだろう。	彼女[かのじょ]は 何[なん]で 来[こ]ないんだろう。	かのじょ は なんで こないん だろう	
\\	彼女[かのじょ]は
\\	来[こ]ないんだろう。			
\\	何十	何十[なんじゅう]	なんじゅう	
\\	その店に何十人も行列していたよ。	その 店[みせ]に 何十[なんじゅう] 人[にん]も 行列[ぎょうれつ]していたよ。	その みせ に なんじゅうにん も ぎょうれつ して いた よ	
\\	その 店[みせ]に
\\	人[にん]も 行列[ぎょうれつ]していたよ。			
\\	何て	何[なん]て	なんて	
\\	何てきれいな人なんだ。	何[なん]てきれいな 人[ひと]なんだ。	なんて きれい な ひと なんだ	
\\	きれいな 人[ひと]なんだ。			
\\	何しろ	何[なに]しろ	なにしろ	
\\	彼は何しろよくしゃべります。	彼[かれ]は 何[なに]しろよくしゃべります。	かれ は なにしろ よく しゃべります	
\\	彼[かれ]は
\\	よくしゃべります。			
\\	何千	何千[なんぜん]	なんぜん	
\\	新聞社に何千ものメールが寄せられたよ。	新聞社[しんぶんしゃ]に 何千[なんぜん]ものメールが 寄[よ]せられたよ。	しんぶんしゃ に なんぜん も の めーる が よせられた よ	
\\	新聞社[しんぶんしゃ]に
\\	ものメールが 寄[よ]せられたよ。			
\\	レベル	レベル	レベル	
\\	自分のレベルに合った授業を選んでください。	自分[じぶん]のレベルに 合[あ]った 授業[じゅぎょう]を 選[えら]んでください。	じぶん の れべる に あった じゅぎょう を えらんで ください	
\\	自分[じぶん]の
\\	に 合[あ]った 授業[じゅぎょう]を 選[えら]んでください。			
\\	何百	何百[なんびゃく]	なんびゃく	
\\	海で何百もの美しい魚を見ました。	海[うみ]で 何百[なんびゃく]もの 美[うつく]しい 魚[さかな]を 見[み]ました。	うみ で なんびゃく も の うつくしい さかな を みました	
\\	海[うみ]で
\\	もの 美[うつく]しい 魚[さかな]を 見[み]ました。			
\\	やって来る	やって 来[く]る	やってくる	
\\	フランスからサーカスがやって来ます。	フランスからサーカスがやって 来[き]ます。	ふらんす から さーかす が やってきます	
\\	フランスからサーカスが
\\	来日	来日[らいにち]	らいにち	
\\	有名なバンドが来日していますね。	有名[ゆうめい]なバンドが 来日[らいにち]していますね。	ゆうめい な ばんど が らいにち して います ね	
\\	有名[ゆうめい]なバンドが
\\	していますね。			
\\	日帰り	日帰[ひがえ]り	ひがえり	
\\	私たちは日帰りで京都に行きました。	私[わたし]たちは 日帰[ひがえ]りで 京都[きょうと]に 行[い]きました。	わたしたち は ひがえり で きょうと に いきました	
\\	私[わたし]たちは
\\	で 京都[きょうと]に 行[い]きました。			
\\	大して	大[たい]して	たいして	
\\	彼は大して嬉しそうには見えなかったよね。	彼[かれ]は 大[たい]して 嬉[うれ]しそうには 見[み]えなかったよね。	かれ は たいして うれし そう に は みえなかった よ ね	
\\	彼[かれ]は
\\	嬉[うれ]しそうには 見[み]えなかったよね。			
\\	むしろ	むしろ	むしろ	
\\	彼は建築家というよりむしろ芸術家ね。	彼[かれ]は 建築家[けんちくか]というよりむしろ 芸術家[げいじゅつか]ね。	かれ は けんちくか と いう より むしろ げいじゅつか ね	
\\	彼[かれ]は 建築家[けんちくか]というより
\\	芸術家[げいじゅつか]ね。			
\\	大金	大金[たいきん]	たいきん	
\\	このかばんには大金が入っています。	このかばんには 大金[たいきん]が 入[はい]っています。	この かばん に は たいきん が はいって います	
\\	このかばんには
\\	が 入[はい]っています。			
\\	大	大[だい]	だい	
\\	チーズケーキの大を一つ下さい。	チーズケーキの 大[だい]を 一[ひと]つ 下[くだ]さい。	ちーずけーき の だい を ひとつ ください	
\\	チーズケーキの
\\	を 一[ひと]つ 下[くだ]さい。			
\\	日中	日中[にっちゅう]	にっちゅう	
\\	日中はずっと海で泳いでいました。	日中[にっちゅう]はずっと 海[うみ]で 泳[およ]いでいました。	にっちゅう は ずっと うみ で およいで いました	
\\	はずっと 海[うみ]で 泳[およ]いでいました。			
\\	日中	日中[にっちゅう]	にっちゅう	
\\	日中貿易は急激に伸びているわね。	日中[にっちゅう] 貿易[ぼうえき]は 急激[きゅうげき]に 伸[の]びているわね。	にっちゅう ぼうえき は きゅうげき に のびて いる わ ね	
\\	貿易[ぼうえき]は 急激[きゅうげき]に 伸[の]びているわね。			
\\	中年	中年[ちゅうねん]	ちゅうねん	
\\	これは、中年の男性によく見られる症状です。	これは、 中年[ちゅうねん]の 男性[だんせい]によく 見[み]られる 症状[しょうじょう]です。	これ は ちゅうねん の だんせい に よく みられる しょうじょう です	
\\	これは、
\\	の 男性[だんせい]によく 見[み]られる 症状[しょうじょう]です。			
\\	中	中[ちゅう]	ちゅう	
\\	うちの子の身長はクラスで中くらいです。	うちの 子[こ]の 身長[しんちょう]はクラスで 中[ちゅう]くらいです。	うち の こ の しんちょう は くらす で ちゅう くらい です	
\\	うちの 子[こ]の 身長[しんちょう]はクラスで
\\	くらいです。			
\\	ますます	ますます	ますます	
\\	状況はますます悪くなったね。	状況[じょうきょう]はますます 悪[わる]くなったね。	じょうきょう は ますます わるく なった ね	
\\	状況[じょうきょう]は
\\	悪[わる]くなったね。			
\\	年中	年中[ねんじゅう]	ねんじゅう	
\\	叔母は年中旅行しています。	叔母[おば]は 年中[ねんじゅう] 旅行[りょこう]しています。	おば は ねんじゅう りょこう して います	
\\	叔母[おば]は
\\	旅行[りょこう]しています。			
\\	大小	大小[だいしょう]	だいしょう	
\\	応募作品の大小は問いません。	応募作品[おうぼ さくひん]の 大小[だいしょう]は 問[と]いません。	おうぼ さくひん の だいしょう は といません	
\\	応募作品[おうぼ さくひん]の
\\	は 問[と]いません。			
\\	多少	多少[たしょう]	たしょう	
\\	このソフトには多少問題がある。	このソフトには 多少[たしょう] 問題[もんだい]がある。	この そふと に は たしょう もんだい が ある	
\\	このソフトには
\\	問題[もんだい]がある。			
\\	ぶら下がる	ぶら 下[さ]がる	ぶらさがる	
\\	猿が木にぶら下がってるな。	猿[さる]が 木[き]にぶら 下[さ]がってるな。	さる が き に ぶらさがって る な	
\\	猿[さる]が 木[き]に
\\	な。			
\\	ぶら下げる	ぶら 下[さ]げる	ぶらさげる	
\\	彼は首にカメラをぶら下げているね。	彼[かれ]は 首[くび]にカメラをぶら 下[さ]げているね。	かれ は くび に かめら を ぶらさげて いる ね	
\\	彼[かれ]は 首[くび]にカメラを
\\	ね。			
\\	モデル	モデル	モデル	
\\	彼女はこの小説の主人公のモデルです。	彼女[かのじょ]はこの 小説[しょうせつ]の 主人公[しゅじんこう]のモデルです。	かのじょ は この しょうせつ の しゅじんこう の もでる です	
\\	彼女[かのじょ]はこの 小説[しょうせつ]の 主人公[しゅじんこう]の
\\	です。			
\\	下	下[もと]	もと	
\\	私はその教授の下で論文を書いたの。	私[わたし]はその 教授[きょうじゅ]の 下[もと]で 論文[ろんぶん]を 書[か]いたの。	わたし は その きょうじゅ の もと で ろんぶん を かいた の	
\\	私[わたし]はその 教授[きょうじゅ]の
\\	で 論文[ろんぶん]を 書[か]いたの。			
\\	やり方	やり 方[かた]	やりかた	
\\	仕事のやり方を教えてください。	仕事[しごと]のやり 方[かた]を 教[おし]えてください。	しごと の やりかた を おしえて ください	
\\	仕事[しごと]の
\\	を 教[おし]えてください。			
\\	方々	方々[ほうぼう]	ほうぼう	
\\	彼の連絡先を方々に問い合わせたんだ。	彼[かれ]の 連絡先[れんらくさき]を 方々[ほうぼう]に 問[と]い 合[あ]わせたんだ。	かれ の れんらくさき を ほうぼう に といあわせた ん だ	
\\	彼[かれ]の 連絡先[れんらくさき]を
\\	に 問[と]い 合[あ]わせたんだ。			
\\	行方	行方[ゆくえ]	ゆくえ	
\\	彼の行方が分かりません。	彼[かれ]の 行方[ゆくえ]が 分[わ]かりません。	かれ の ゆくえ が わかりません	
\\	彼[かれ]の
\\	が 分[わ]かりません。			
\\	一人一人	一人一人[ひとりひとり]	ひとりひとり	
\\	この学校では一人一人の生徒を大切にしているの。	この 学校[がっこう]では 一人一人[ひとりひとり]の 生徒[せいと]を 大切[たいせつ]にしているの。	この がっこう で は ひとりひとり の せいと を たいせつ に して いる の	
\\	この 学校[がっこう]では
\\	の 生徒[せいと]を 大切[たいせつ]にしているの。			
\\	もたらす	もたらす	もたらす	
\\	彼はこの国に平和をもたらしたのよ。	彼[かれ]はこの 国[くに]に 平和[へいわ]をもたらしたのよ。	かれ は この くに に へいわ を もたらした の よ	
\\	彼[かれ]はこの 国[くに]に 平和[へいわ]を
\\	のよ。			
\\	日の入り	日[ひ]の 入[い]り	ひのいり	
\\	今日の日の入りは午後6時でした。	今日[きょう]の 日[ひ]の 入[い]りは 午後6時[ごご 
\\	じ]でした。	きょう の ひのいり は ごご 
\\	じ でした	
\\	今日[きょう]の
\\	は 午後6時[ごご 
\\	じ]でした。			
\\	出入り	出入[でい]り	でいり	
\\	彼の家は人の出入りが多いね。	彼[かれ]の 家[いえ]は 人[ひと]の 出入[でい]りが 多[おお]いね。	かれ の いえ は ひと の でいり が おおい ね	
\\	彼[かれ]の 家[いえ]は 人[ひと]の
\\	が 多[おお]いね。			
\\	出来上がる	出来上[できあ]がる	できあがる	
\\	この家は来年出来上がります。	この 家[いえ]は 来年[らいねん] 出来上[できあ]がります。	この いえ は らいねん できあがります	
\\	この 家[いえ]は 来年[らいねん]
\\	人出	人出[ひとで]	ひとで	
\\	今日も遊園地は大変な人出だったよ。	今日[きょう]も 遊園地[ゆうえんち]は 大変[たいへん]な 人出[ひとで]だったよ。	きょう も ゆうえんち は たいへん な ひとで だった よ	
\\	今日[きょう]も 遊園地[ゆうえんち]は 大変[たいへん]な
\\	だったよ。			
\\	日の出	日[ひ]の 出[で]	ひので	
\\	日の出がとてもきれいですね。	日[ひ]の 出[で]がとてもきれいですね。	ひので が とても きれい です ね	
\\	がとてもきれいですね。			
\\	それぞれ	それぞれ	それぞれ	
\\	人はそれぞれ考え方が違います。	人[ひと]はそれぞれ 考[かんが]え 方[かた]が 違[ちが]います。	ひと は それぞれ かんがえかた が ちがいます	
\\	人[ひと]は
\\	考[かんが]え 方[かた]が 違[ちが]います。			
\\	出来上がり	出来上[できあ]がり	できあがり	
\\	私は作品の出来上がりに満足しています。	私[わたし]は 作品[さくひん]の 出来上[できあ]がりに 満足[まんぞく]しています。	わたし は さくひん の できあがり に まんぞく して います	
\\	私[わたし]は 作品[さくひん]の
\\	に 満足[まんぞく]しています。			
\\	出来るだけ	出来[でき]るだけ	できるだけ	
\\	出来るだけ早く来てください。	出来[でき]るだけ 早[はや]く 来[き]てください。	できるだけ はやく きて ください	
\\	早[はや]く 来[き]てください。			
\\	外す	外[はず]す	はずす	
\\	彼はメガネを外しました。	彼[かれ]はメガネを 外[はず]しました。	かれ は めがね を はずしました	
\\	彼[かれ]はメガネを
\\	外れる	外[はず]れる	はずれる	
\\	びんのふたが外れません。	びんのふたが 外[はず]れません。	びん の ふた が はずれません	
\\	びんのふたが
\\	外れ	外[はず]れ	はずれ	
\\	このくじは外れです。	このくじは 外[はず]れです。	この くじ は はずれ です	
\\	このくじは
\\	です。			
\\	その内	その 内[うち]	そのうち	
\\	彼女もその内、親の気持ちが分かるだろう。	彼女[かのじょ]もその 内[うち]、 親[おや]の 気持[きも]ちが 分[わ]かるだろう。	かのじょ も そのうち おや の きもち が わかる だろう	
\\	彼女[かのじょ]も
\\	、 親[おや]の 気持[きも]ちが 分[わ]かるだろう。			
\\	トップ	トップ	トップ	
\\	彼はトップの成績で合格しました。	彼[かれ]はトップの 成績[せいせき]で 合格[ごうかく]しました。	かれ は とっぷ の せいせき で ごうかく しました	
\\	彼[かれ]は
\\	の 成績[せいせき]で 合格[ごうかく]しました。			
\\	本来	本来[ほんらい]	ほんらい	
\\	彼女はプレッシャーから解放されて本来の自分に戻ったな。	彼女[かのじょ]はプレッシャーから 解放[かいほう]されて 本来[ほんらい]の 自分[じぶん]に 戻[もど]ったな。	かのじょ は ぷれっしゃー から かいほう されて ほんらい の じぶん に もどった な	
\\	彼女[かのじょ]はプレッシャーから 解放[かいほう]されて
\\	の 自分[じぶん]に 戻[もど]ったな。			
\\	本人	本人[ほんにん]	ほんにん	
\\	それは本人に聞いてください。	それは 本人[ほんにん]に 聞[き]いてください。	それ は ほんにん に きいて ください	
\\	それは
\\	に 聞[き]いてください。			
\\	本年	本年[ほんねん]	ほんねん	
\\	会社の本年の目標が発表されたよ。	会社[かいしゃ]の 本年[ほんねん]の 目標[もくひょう]が 発表[はっぴょう]されたよ。	かいしゃ の ほんねん の もくひょう が はっぴょう された よ	
\\	会社[かいしゃ]の
\\	の 目標[もくひょう]が 発表[はっぴょう]されたよ。			
\\	本日	本日[ほんじつ]	ほんじつ	
\\	本日のランチはハンバーグでございます。	本日[ほんじつ]のランチはハンバーグでございます。	ほんじつ の らんち は はんばーぐ で ございます	
\\	のランチはハンバーグでございます。			
\\	休める	休[やす]める	やすめる	
\\	疲れた体を休めてください。	疲[つか]れた 体[からだ]を 休[やす]めてください。	つかれた からだ を やすめて ください	
\\	疲[つか]れた 体[からだ]を
\\	ください。			
\\	ブーム	ブーム	ブーム	
\\	最近、日本は健康ブームです。	最近[さいきん]、 日本[にほん]は 健康[けんこう]ブームです。	さいきん にほん は けんこう ぶーむ です	
\\	最近[さいきん]、 日本[にほん]は 健康[けんこう]
\\	です。			
\\	一休み	一休[ひとやす]み	ひとやすみ	
\\	ここで一休みしましょう。	ここで 一休[ひとやす]みしましょう。	ここ で ひとやすみ しましょう	
\\	ここで
\\	しましょう。			
\\	大体	大体[だいたい]	だいたい	
\\	大体、初めから無理な計画だったのだ。	大体[だいたい]、 初[はじ]めから 無理[むり]な 計画[けいかく]だったのだ。	だいたい はじめ から むり な けいかく だった の だ	
\\	、 初[はじ]めから 無理[むり]な 計画[けいかく]だったのだ。			
\\	入力	入力[にゅうりょく]	にゅうりょく	
\\	彼女の仕事はデータの入力です。	彼女[かのじょ]の 仕事[しごと]はデータの 入力[にゅうりょく]です。	かのじょ の しごと は でーた の にゅうりょく です	
\\	彼女[かのじょ]の 仕事[しごと]はデータの
\\	です。			
\\	体力	体力[たいりょく]	たいりょく	
\\	若者は体力がありますね。	若者[わかもの]は 体力[たいりょく]がありますね。	わかもの は たいりょく が あります ね	
\\	若者[わかもの]は
\\	がありますね。			
\\	目上	目上[めうえ]	めうえ	
\\	彼は目上の人にとても気を使います。	彼[かれ]は 目上[めうえ]の 人[ひと]にとても 気[き]を 使[つか]います。	かれ は めうえ の ひと に とても き を つかいます	
\\	彼[かれ]は
\\	の 人[ひと]にとても 気[き]を 使[つか]います。			
\\	ようやく	ようやく	ようやく	
\\	ようやくゴールが見えてきました。	ようやくゴールが 見[み]えてきました。	ようやく ごーる が みえて きました	
\\	ゴールが 見[み]えてきました。			
\\	目方	目方[めかた]	めかた	
\\	この荷の目方は70キロってとこかね。	この 荷[に]の 目方[めかた]は70キロってとこかね。	この に の めかた は 
\\	きろ って とこ か ね	
\\	この 荷[に]の
\\	は70キロってとこかね。			
\\	目下	目下[めした]	めした	
\\	彼は目下の者にも優しいんだ。	彼[かれ]は 目下[めした]の 者[もの]にも 優[やさ]しいんだ。	かれ は めした の もの に も やさしい ん だ	
\\	彼[かれ]は
\\	の 者[もの]にも 優[やさ]しいんだ。			
\\	目下	目下[もっか]	もっか	
\\	息子は目下勉強中です。	息子[むすこ]は 目下[もっか] 勉強中[べんきょうちゅう]です。	むすこ は もっか べんきょうちゅう です	
\\	息子[むすこ]は
\\	勉強中[べんきょうちゅう]です。			
\\	出入り口	出入[でい]り 口[ぐち]	でいりぐち	
\\	出入り口に車を停めないでください。	出入[でい]り 口[ぐち]に 車[くるま]を 停[と]めないでください。	でいりぐち に くるま を とめない で ください	
\\	に 車[くるま]を 停[と]めないでください。			
\\	一口	一口[ひとくち]	ひとくち	
\\	彼はおまんじゅうを一口で食べたの。	彼[かれ]はおまんじゅうを 一口[ひとくち]で 食[た]べたの。	かれ は おまんじゅう を ひとくち で たべた の	
\\	彼[かれ]はおまんじゅうを
\\	で 食[た]べたの。			
\\	人手	人手[ひとで]	ひとで	
\\	人手が足りなくて忙しい。	人手[ひとで]が 足[た]りなくて 忙[いそが]しい。	ひとで が たりなくて いそがしい	
\\	が 足[た]りなくて 忙[いそが]しい。			
\\	チーム	チーム	チーム	
\\	彼はチームの一員です。	彼[かれ]はチームの 一員[いちいん]です。	かれ は ちーむ の いちいん です	
\\	彼[かれ]は
\\	の 一員[いちいん]です。			
\\	手入れ	手入[てい]れ	ていれ	
\\	母は庭の手入れをしています。	母[はは]は 庭[にわ]の 手入[てい]れをしています。	はは は にわ の ていれ を して います	
\\	母[はは]は 庭[にわ]の
\\	をしています。			
\\	手本	手本[てほん]	てほん	
\\	手本を見ながら習字をしました。	手本[てほん]を 見[み]ながら 習字[しゅうじ]をしました。	てほん を みながら しゅうじ を しました	
\\	を 見[み]ながら 習字[しゅうじ]をしました。			
\\	手足	手足[てあし]	てあし	
\\	あの人は手足が長い。	あの 人[ひと]は 手足[てあし]が 長[なが]い。	あの ひと は てあし が ながい	
\\	あの 人[ひと]は
\\	が 長[なが]い。			
\\	田	田[た]	た	
\\	今は田に水を入れる季節です。	今[いま]は 田[た]に 水[みず]を 入[い]れる 季節[きせつ]です。	いま は た に みず を いれる きせつ です	
\\	今[いま]は
\\	に 水[みず]を 入[い]れる 季節[きせつ]です。			
\\	花火	花火[はなび]	はなび	
\\	昨日、花火を見に行ったよ。	昨日[きのう]、 花火[はなび]を 見[み]に 行[い]ったよ。	きのう はなび を み に いった よ	
\\	昨日[きのう]、
\\	を 見[み]に 行[い]ったよ。			
\\	タイプ	タイプ	タイプ	
\\	同じタイプで色の違うものはありますか。	同[おな]じタイプで 色[いろ]の 違[ちが]うものはありますか。	おなじ たいぷ で いろ の ちがう もの は あります か	
\\	同[おな]じ
\\	で 色[いろ]の 違[ちが]うものはありますか。			
\\	男女	男女[だんじょ]	だんじょ	
\\	そのスポーツは男女一緒にします。	そのスポーツは 男女[だんじょ] 一緒[いっしょ]にします。	その すぽーつ は だんじょ いっしょ に します	
\\	そのスポーツは
\\	一緒[いっしょ]にします。			
\\	男子	男子[だんし]	だんし	
\\	男子はこっちに並んで。	男子[だんし]はこっちに 並[なら]んで。	だんし は こっち に ならんで	
\\	はこっちに 並[なら]んで。			
\\	私	私[わたくし]	わたくし	
\\	私は北海道の出身です。	私[わたくし]は 北海道[ほっかいどう]の 出身[しゅっしん]です。	わたくし は ほっかいどう の しゅっしん です	
\\	は 北海道[ほっかいどう]の 出身[しゅっしん]です。			
\\	友人	友人[ゆうじん]	ゆうじん	
\\	彼は高校時代からの友人です。	彼[かれ]は 高校時代[こうこう じだい]からの 友人[ゆうじん]です。	かれ は こうこう じだい から の ゆうじん です	
\\	彼[かれ]は 高校時代[こうこう じだい]からの
\\	です。			
\\	友	友[とも]	とも	
\\	持つべきものは友よね。	持[も]つべきものは 友[とも]よね。	もつべき もの は とも よ ね	
\\	持[も]つべきものは
\\	よね。			
\\	ビジネス	ビジネス	ビジネス	
\\	彼はビジネスクラスに乗ったの。	彼[かれ]はビジネスクラスに 乗[の]ったの。	かれ は びじねすくらす に のった の	
\\	彼[かれ]は
\\	クラスに 乗[の]ったの。			
\\	元	元[もと]	もと	
\\	あの人は卓球の元選手です。	あの 人[ひと]は 卓球[たっきゅう]の 元[もと] 選手[せんしゅ]です。	あの ひと は たっきゅう の もとせんしゅ です	
\\	あの 人[ひと]は 卓球[たっきゅう]の
\\	選手[せんしゅ]です。			
\\	手元	手元[てもと]	てもと	
\\	説明書は手元にありますか。	説明書[せつめいしょ]は 手元[てもと]にありますか。	せつめいしょ は てもと に あります か	
\\	説明書[せつめいしょ]は
\\	にありますか。			
\\	元々	元々[もともと]	もともと	
\\	彼女は元々フランスに行くつもりだったの。	彼女[かのじょ]は 元々[もともと]フランスに 行[い]くつもりだったの。	かのじょ は もともと ふらんす に いく つもり だった の	
\\	彼女[かのじょ]は
\\	フランスに 行[い]くつもりだったの。			
\\	天	天[てん]	てん	
\\	天から恵みの雨が降ったね。	天[てん]から 恵[めぐ]みの 雨[あめ]が 降[ふ]ったね。	てん から めぐみ の あめ が ふった ね	
\\	から 恵[めぐ]みの 雨[あめ]が 降[ふ]ったね。			
\\	本気	本気[ほんき]	ほんき	
\\	いや、僕は本気なんだ。	いや、 僕[ぼく]は 本気[ほんき]なんだ。	いや ぼく は ほんき な ん だ	
\\	いや、 僕[ぼく]は
\\	なんだ。			
\\	明日	明日[みょうにち]	みょうにち	
\\	明日、会議を開きます。	明日[みょうにち]、 会議[かいぎ]を 開[ひら]きます。	みょうにち かいぎ を ひらきます	
\\	、 会議[かいぎ]を 開[ひら]きます。			
\\	ネットワーク	ネットワーク	ネットワーク	
\\	最近、社内ネットワークの調子が悪い。	最近[さいきん]、 社内[しゃない]ネットワークの 調子[ちょうし]が 悪[わる]い。	さいきん しゃない ねっとわーく の ちょうし が わるい	
\\	最近[さいきん]、 社内[しゃない]
\\	の 調子[ちょうし]が 悪[わる]い。			
\\	東西	東西[とうざい]	とうざい	
\\	東西に大きな道路が通っています。	東西[とうざい]に 大[おお]きな 道路[どうろ]が 通[とお]っています。	とうざい に おおき な どうろ が とおって います	
\\	に 大[おお]きな 道路[どうろ]が 通[とお]っています。			
\\	南西	南西[なんせい]	なんせい	
\\	ここから南西に進むと村があります。	ここから 南西[なんせい]に 進[すす]むと 村[むら]があります。	ここ から なんせい に すすむ と むら が あります	
\\	ここから
\\	に 進[すす]むと 村[むら]があります。			
\\	南東	南東[なんとう]	なんとう	
\\	南東の方角に月が見えるよ。	南東[なんとう]の 方角[ほうがく]に 月[つき]が 見[み]えるよ。	なんとう の ほうがく に つき が みえる よ	
\\	の 方角[ほうがく]に 月[つき]が 見[み]えるよ。			
\\	南北	南北[なんぼく]	なんぼく	
\\	南北に山が広がっています。	南北[なんぼく]に 山[やま]が 広[ひろ]がっています。	なんぼく に やま が ひろがって います	
\\	に 山[やま]が 広[ひろ]がっています。			
\\	北西	北西[ほくせい]	ほくせい	
\\	台風は北西に進んでいます。	台風[たいふう]は 北西[ほくせい]に 進[すす]んでいます。	たいふう は ほくせい に すすんで います	
\\	台風[たいふう]は
\\	に 進[すす]んでいます。			
\\	やや	やや	やや	
\\	この服、私にはやや小さいみたい。	この 服[ふく]、 私[わたし]にはやや 小[ちい]さいみたい。	この ふく わたし に は やや ちいさい みたい	
\\	この 服[ふく]、 私[わたし]には
\\	小[ちい]さいみたい。			
\\	北東	北東[ほくとう]	ほくとう	
\\	町の北東に山があります。	町[まち]の 北東[ほくとう]に 山[やま]があります。	まち の ほくとう に やま が あります	
\\	町[まち]の
\\	に 山[やま]があります。			
\\	向ける	向[む]ける	むける	
\\	彼は上司に怒りの目を向けたんだよ。	彼[かれ]は 上司[じょうし]に 怒[いか]りの 目[め]を 向[む]けたんだよ。	かれ は じょうし に いかり の め を むけた ん だ よ	
\\	彼[かれ]は 上司[じょうし]に 怒[いか]りの 目[め]を
\\	んだよ。			
\\	向き	向[む]き	むき	
\\	花瓶の向きを変えたの。	花瓶[かびん]の 向[む]きを 変[か]えたの。	かびん の むき を かえた の	
\\	花瓶[かびん]の
\\	を 変[か]えたの。			
\\	向かい	向[む]かい	むかい	
\\	向かいの席が空いていますよ。	向[む]かいの 席[せき]が 空[あ]いていますよ。	むかい の せき が あいて います よ	
\\	の 席[せき]が 空[あ]いていますよ。			
\\	入門	入門[にゅうもん]	にゅうもん	
\\	私は相撲部屋に入門しました。	私[わたし]は 相撲部屋[すもう べや]に 入門[にゅうもん]しました。	わたし は すもう べや に にゅうもん しました	
\\	私[わたし]は 相撲部屋[すもう べや]に
\\	しました。			
\\	デモ	デモ	デモ	
\\	彼女はデモに参加したよ。	彼女[かのじょ]はデモに 参加[さんか]したよ。	かのじょ は でも に さんか した よ	
\\	彼女[かのじょ]は
\\	に 参加[さんか]したよ。			
\\	門	門[もん]	もん	
\\	8時に学校の門が開きます。	
\\	時[じ]に 学校[がっこう]の 門[もん]が 開[ひら]きます。	
\\	じ に がっこう の もん が ひらきます	
\\	時[じ]に 学校[がっこう]の
\\	が 開[ひら]きます。			
\\	開ける	開[ひら]ける	ひらける	
\\	霧が晴れて視界が開けたね。	霧[きり]が 晴[は]れて 視界[しかい]が 開[ひら]けたね。	きり が はれて しかい が ひらけた ね	
\\	霧[きり]が 晴[は]れて 視界[しかい]が
\\	ね。			
\\	中間	中間[ちゅうかん]	ちゅうかん	
\\	そのふたつの町の中間には川が流れているの。	そのふたつの 町[まち]の 中間[ちゅうかん]には 川[かわ]が 流[なが]れているの。	その ふたつ の まち の ちゅうかん に は かわ が ながれて いる の	
\\	そのふたつの 町[まち]の
\\	には 川[かわ]が 流[なが]れているの。			
\\	手間	手間[てま]	てま	
\\	これはとても手間のかかる料理です。	これはとても 手間[てま]のかかる 料理[りょうり]です。	これ は とても てま の かかる りょうり です	
\\	これはとても
\\	のかかる 料理[りょうり]です。			
\\	間	間[ま]	ま	
\\	彼は少し間を置いてから返事した。	彼[かれ]は 少[すこ]し 間[ま]を 置[お]いてから 返事[へんじ]した。	かれ は すこし ま を おいて から へんじ した	
\\	彼[かれ]は 少[すこ]し
\\	を 置[お]いてから 返事[へんじ]した。			
\\	ワープロ	ワープロ	ワープロ	
\\	ワープロで報告書を打ったよ。	ワープロで 報告書[ほうこくしょ]を 打[う]ったよ。	わーぷろ で ほうこくしょ を うった よ	
\\	で 報告書[ほうこくしょ]を 打[う]ったよ。			
\\	高まる	高[たか]まる	たかまる	
\\	その計画に対する反対の声が高まっているわ。	その 計画[けいかく]に 対[たい]する 反対[はんたい]の 声[こえ]が 高[たか]まっているわ。	その けいかく に たいする はんたい の こえ が たかまって いる わ	
\\	その 計画[けいかく]に 対[たい]する 反対[はんたい]の 声[こえ]が
\\	わ。			
\\	高める	高[たか]める	たかめる	
\\	自分を高めるのは大切なことです。	自分[じぶん]を 高[たか]めるのは 大切[たいせつ]なことです。	じぶん を たかめる の は たいせつ な こと です	
\\	自分[じぶん]を
\\	のは 大切[たいせつ]なことです。			
\\	安っぽい	安[やす]っぽい	やすっぽい	
\\	そのシャツは安っぽいね。	そのシャツは 安[やす]っぽいね。	その しゃつ は やすっぽい ね	
\\	そのシャツは
\\	ね。			
\\	低下	低下[ていか]	ていか	
\\	最近、教育レベルが低下しています。	最近[さいきん]、 教育[きょういく]レベルが 低下[ていか]しています。	さいきん きょういく れべる が ていか して います	
\\	最近[さいきん]、 教育[きょういく]レベルが
\\	しています。			
\\	低	低[てい]	てい	
\\	彼は低賃金で働いているんだ。	彼[かれ]は 低[てい] 賃金[ちんぎん]で 働[はたら]いているんだ。	かれ は ていちんぎん で はたらいて いる ん だ	
\\	彼[かれ]は
\\	賃金[ちんぎん]で 働[はたら]いているんだ。			
\\	最も	最[もっと]も	もっとも	
\\	彼は世界で最も早い男です。	彼[かれ]は 世界[せかい]で 最[もっと]も 早[はや]い 男[おとこ]です。	かれ は せかい で もっとも はやい おとこ です	
\\	彼[かれ]は 世界[せかい]で
\\	早[はや]い 男[おとこ]です。			
\\	メンバー	メンバー	メンバー	
\\	グループのメンバーは10人です。	グループのメンバーは10 人[にん]です。	ぐるーぷ の めんばー は 
\\	にん です 。	
\\	グループの
\\	は10 人[にん]です。			
\\	月初め	月初[つきはじ]め	つきはじめ	
\\	いつも月初めに彼と会います。	いつも 月初[つきはじ]めに 彼[かれ]と 会[あ]います。	いつも つきはじめ に かれ と あいます	
\\	いつも
\\	に 彼[かれ]と 会[あ]います。			
\\	前年	前年[ぜんねん]	ぜんねん	
\\	前年に比べて利益が上がりました。	前年[ぜんねん]に 比[くら]べて 利益[りえき]が 上[あ]がりました。	ぜんねん に くらべて りえき が あがりました	
\\	に 比[くら]べて 利益[りえき]が 上[あ]がりました。			
\\	前方	前方[ぜんぽう]	ぜんぽう	
\\	前方に山が見えますね。	前方[ぜんぽう]に 山[やま]が 見[み]えますね。	ぜんぽう に やま が みえます ね	
\\	に 山[やま]が 見[み]えますね。			
\\	手前	手前[てまえ]	てまえ	
\\	駅の手前に郵便局があります。	駅[えき]の 手前[てまえ]に 郵便局[ゆうびんきょく]があります。	えき の てまえ に ゆうびんきょく が あります	
\\	駅[えき]の
\\	に 郵便局[ゆうびんきょく]があります。			
\\	前向き	前向[まえむ]き	まえむき	
\\	前向きに検討します。	前向[まえむ]きに 検討[けんとう]します。	まえむき に けんとう します	
\\	に 検討[けんとう]します。			
\\	ハードウェア	ハードウェア	ハードウェア	
\\	この問題はハードウェアの故障が原因です。	この 問題[もんだい]はハードウェアの 故障[こしょう]が 原因[げんいん]です。	この もんだい は はーどうぇあ の こしょう が げんいん です	
\\	この 問題[もんだい]は
\\	の 故障[こしょう]が 原因[げんいん]です。			
\\	人前	人前[ひとまえ]	ひとまえ	
\\	彼女は人前に出ると緊張します。	彼女[かのじょ]は 人前[ひとまえ]に 出[で]ると 緊張[きんちょう]します。	かのじょ は ひとまえ に でる と きんちょう します	
\\	彼女[かのじょ]は
\\	に 出[で]ると 緊張[きんちょう]します。			
\\	前もって	前[まえ]もって	まえもって	
\\	休む時は前もって連絡ください。	休[やす]む 時[とき]は 前[まえ]もって 連絡[れんらく]ください。	やすむ とき は まえもって れんらく ください	
\\	休[やす]む 時[とき]は
\\	連絡[れんらく]ください。			
\\	出前	出前[でまえ]	でまえ	
\\	寿司の出前を頼んだよ。	寿司[すし]の 出前[でまえ]を 頼[たの]んだよ。	すし の でまえ を たのんだ よ	
\\	寿司[すし]の
\\	を 頼[たの]んだよ。			
\\	前半	前半[ぜんはん]	ぜんはん	
\\	相手チームのリードで前半が終わったよ。	相手[あいて]チームのリードで 前半[ぜんはん]が 終[お]わったよ。	あいて ちーむ の りーど で ぜんはん が おわった よ	
\\	相手[あいて]チームのリードで
\\	が 終[お]わったよ。			
\\	半ば	半[なか]ば	なかば	
\\	彼は30代の半ばです。	彼[かれ]は30 代[だい]の 半[なか]ばです。	かれ は 
\\	だい の なかば です	
\\	彼[かれ]は30 代[だい]の
\\	です。			
\\	マスコミ	マスコミ	マスコミ	
\\	彼はよくマスコミにも登場するね。	彼[かれ]はよくマスコミにも 登場[とうじょう]するね。	かれ は よく ますこみ に も とうじょう する ね	
\\	彼[かれ]はよく
\\	にも 登場[とうじょう]するね。			
\\	半年	半年[はんとし]	はんとし	
\\	日本に来て半年になります。	日本[にっぽん]に 来[き]て 半年[はんとし]になります。	にっぽん に きて はんとし に なります	
\\	日本[にっぽん]に 来[き]て
\\	になります。			
\\	明朝	明朝[みょうちょう]	みょうちょう	
\\	明朝10時からまた会議です。	明朝[みょうちょう] 
\\	時[じ]からまた 会議[かいぎ]です。	みょうちょう 
\\	じ から また かいぎ です	
\\	時[じ]からまた 会議[かいぎ]です。			
\\	晩年	晩年[ばんねん]	ばんねん	
\\	彼は晩年を故郷で過ごしたんだ。	彼[かれ]は 晩年[ばんねん]を 故郷[こきょう]で 過[す]ごしたんだ。	かれ は ばんねん を こきょう で すごした ん だ	
\\	彼[かれ]は
\\	を 故郷[こきょう]で 過[す]ごしたんだ。			
\\	夜間	夜間[やかん]	やかん	
\\	夜間は裏口から入ってください。	夜間[やかん]は 裏口[うらぐち]から 入[はい]ってください。	やかん は うらぐち から はいって ください	
\\	は 裏口[うらぐち]から 入[はい]ってください。			
\\	夜空	夜空[よぞら]	よぞら	
\\	ふたりで夜空を見上げたの。	ふたりで 夜空[よぞら]を 見上[みあ]げたの。	ふたり で よぞら を みあげた の	
\\	ふたりで
\\	を 見上[みあ]げたの。			
\\	夜明け	夜明[よあ]け	よあけ	
\\	夜明けと共に目が覚めたんだ。	夜明[よあ]けと 共[とも]に 目[め]が 覚[さ]めたんだ。	よあけ と とも に め が さめた ん だ	
\\	と 共[とも]に 目[め]が 覚[さ]めたんだ。			
\\	やがて	やがて	やがて	
\\	やがて雪も止むでしょう。	やがて 雪[ゆき]も 止[や]むでしょう。	やがて ゆき も やむ でしょう	
\\	雪[ゆき]も 止[や]むでしょう。			
\\	月夜	月夜[つきよ]	つきよ	
\\	散歩にいい月夜ですね。	散歩[さんぽ]にいい 月夜[つきよ]ですね。	さんぽ に いい つきよ です ね	
\\	散歩[さんぽ]にいい
\\	ですね。			
\\	夜	夜[よ]	よ	
\\	あと1時間で夜が明けますね。	あと1 時間[じかん]で 夜[よ]が 明[あ]けますね。	あと 
\\	じかん で よ が あけます ね	
\\	あと1 時間[じかん]で
\\	が 明[あ]けますね。			
\\	七夕	七夕[たなばた]	たなばた	
\\	日本では7月7日は七夕の日です。	日本[にほん]では7 月7日[がつ 
\\	か]は 七夕[たなばた]の 日[ひ]です。	にほん で は 
\\	がつ 
\\	か は たなばた の ひ です	
\\	日本[にほん]では7 月7日[がつ 
\\	か]は
\\	の 日[ひ]です。			
\\	夕日	夕日[ゆうひ]	ゆうひ	
\\	夕日が西の空に沈んだね。	夕日[ゆうひ]が 西[にし]の 空[そら]に 沈[しず]んだね。	ゆうひ が にし の そら に しずんだ ね	
\\	が 西[にし]の 空[そら]に 沈[しず]んだね。			
\\	飲み水	飲[の]み 水[みず]	のみみず	
\\	昔は川の水が飲み水でした。	昔[むかし]は 川[かわ]の 水[みず]が 飲[の]み 水[みず]でした。	むかし は かわ の みず が のみみず でした	
\\	昔[むかし]は 川[かわ]の 水[みず]が
\\	でした。			
\\	わずか	わずか	わずか	
\\	彼はわずかな貯金で暮らしている。	彼[かれ]はわずかな 貯金[ちょきん]で 暮[く]らしている。	かれ は わずか な ちょきん で くらして いる	
\\	彼[かれ]は
\\	な 貯金[ちょきん]で 暮[く]らしている。			
\\	飯	飯[めし]	めし	
\\	今朝は飯を食べたかい。	今朝[けさ]は 飯[めし]を 食[た]べたかい。	けさ は めし を たべた かい	
\\	今朝[けさ]は
\\	を 食[た]べたかい。			
\\	見上げる	見上[みあ]げる	みあげる	
\\	私は空を見上げたの。	私[わたし]は 空[そら]を 見上[みあ]げたの。	わたし は そら を みあげた の	
\\	私[わたし]は 空[そら]を
\\	の。			
\\	見下ろす	見下[みお]ろす	みおろす	
\\	山の頂上からふもとを見下ろしたの。	山[やま]の 頂上[ちょうじょう]からふもとを 見下[みお]ろしたの。	やま の ちょうじょう から ふもと を みおろした の	
\\	山[やま]の 頂上[ちょうじょう]からふもとを
\\	の。			
\\	見出し	見出[みだ]し	みだし	
\\	新聞の見出しが彼の目を引いたんだ。	新聞[しんぶん]の 見出[みだ]しが 彼[かれ]の 目[め]を 引[ひ]いたんだ。	しんぶん の みだし が かれ の め を ひいた ん だ	
\\	新聞[しんぶん]の
\\	が 彼[かれ]の 目[め]を 引[ひ]いたんだ。			
\\	見本	見本[みほん]	みほん	
\\	見本はこちらにございます。	見本[みほん]はこちらにございます。	みほん は こちら に ございます	
\\	はこちらにございます。			
\\	パターン	パターン	パターン	
\\	彼女の行動パターンは興味深いな。	彼女[かのじょ]の 行動[こうどう]パターンは 興味深[きょうみぶか]いな。	かのじょ の こうどう ぱたーん は きょうみぶかい な	
\\	彼女[かのじょ]の 行動[こうどう]
\\	は 興味深[きょうみぶか]いな。			
\\	見つめる	見[み]つめる	みつめる	
\\	彼はその絵をずっと見つめていたの。	彼[かれ]はその 絵[え]をずっと 見[み]つめていたの。	かれ は その え を ずっと みつめていた の 。	
\\	彼[かれ]はその 絵[え]をずっと
\\	の。			
\\	見晴らし	見晴[みは]らし	みはらし	
\\	このベランダは見晴らしがいい。	このベランダは 見晴[みは]らしがいい。	この べらんだ は みはらし が いい	
\\	このベランダは
\\	がいい。			
\\	月見	月見[つきみ]	つきみ	
\\	9月には月見を楽しみます。	
\\	月[がつ]には 月見[つきみ]を 楽[たの]しみます。	
\\	がつ に は つきみ を たのしみます	
\\	月[がつ]には
\\	を 楽[たの]しみます。			
\\	よそ見	よそ 見[み]	よそみ	
\\	運転中はよそ見をしてはいけません。	運転中[うんてんちゅう]はよそ 見[み]をしてはいけません。	うんてんちゅう は よそみ を して は いけません	
\\	運転中[うんてんちゅう]は
\\	をしてはいけません。			
\\	見かけ	見[み]かけ	みかけ	
\\	彼は見かけによらず優しいね。	彼[かれ]は 見[み]かけによらず 優[やさ]しいね。	かれ は みかけ に よらず やさしい ね	
\\	彼[かれ]は
\\	によらず 優[やさ]しいね。			
\\	方言	方言[ほうげん]	ほうげん	
\\	彼は方言で話します。	彼[かれ]は 方言[ほうげん]で 話[はな]します。	かれ は ほうげん で はなします	
\\	彼[かれ]は
\\	で 話[はな]します。			
\\	ついに	ついに	ついに	
\\	彼はついに弁護士の資格を取ったよ。	彼[かれ]はついに 弁護士[べんごし]の 資格[しかく]を 取[と]ったよ。	かれ は ついに べんごし の しかく を とった よ	
\\	彼[かれ]は
\\	弁護士[べんごし]の 資格[しかく]を 取[と]ったよ。			
\\	一言	一言[ひとこと]	ひとこと	
\\	社長に一言お願いしたの。	社長[しゃちょう]に 一言[ひとこと]お 願[ねが]いしたの。	しゃちょう に ひとこと おねがい した の	
\\	社長[しゃちょう]に
\\	お 願[ねが]いしたの。			
\\	文明	文明[ぶんめい]	ぶんめい	
\\	多くの文明は川の近くで始まった。	多[おお]くの 文明[ぶんめい]は 川[かわ]の 近[ちか]くで 始[はじ]まった。	おおく の ぶんめい は かわ の ちかく で はじまった	
\\	多[おお]くの
\\	は 川[かわ]の 近[ちか]くで 始[はじ]まった。			
\\	本文	本文[ほんぶん]	ほんぶん	
\\	本文をよく読んで答えてください。	本文[ほんぶん]をよく 読[よ]んで 答[こた]えてください。	ほんぶん を よく よんで こたえて ください	
\\	をよく 読[よ]んで 答[こた]えてください。			
\\	文	文[ぶん]	ぶん	
\\	この文は意味が分かりません。	この 文[ぶん]は 意味[いみ]が 分[わ]かりません。	この ぶん は いみ が わかりません	
\\	この
\\	は 意味[いみ]が 分[わ]かりません。			
\\	読書	読書[どくしょ]	どくしょ	
\\	私の趣味は読書です。	私[わたし]の 趣味[しゅみ]は 読書[どくしょ]です。	わたし の しゅみ は どくしょ です	
\\	私[わたし]の 趣味[しゅみ]は
\\	です。			
\\	まるで	まるで	まるで	
\\	彼はまるで子供のようにはしゃいだよ。	彼[かれ]はまるで 子供[こども]のようにはしゃいだよ。	かれ は まるで こども の よう に はしゃいだ よ	
\\	彼[かれ]は
\\	子供[こども]のようにはしゃいだよ。			
\\	読み書き	読[よ]み 書[か]き	よみかき	
\\	息子は学校で読み書きを勉強しています。	息子[むすこ]は 学校[がっこう]で 読[よ]み 書[か]きを 勉強[べんきょう]しています。	むすこ は がっこう で よみかき を べんきょう して います	
\\	息子[むすこ]は 学校[がっこう]で
\\	を 勉強[べんきょう]しています。			
\\	目覚ましい	目覚[めざ]ましい	めざましい	
\\	あの国は目覚ましい発展を遂げたの。	あの 国[くに]は 目覚[めざ]ましい 発展[はってん]を 遂[と]げたの。	あの くに は めざましい はってん を とげた の	
\\	あの 国[くに]は
\\	発展[はってん]を 遂[と]げたの。			
\\	見覚え	見覚[みおぼ]え	みおぼえ	
\\	この人に見覚えがありますか。	この 人[ひと]に 見覚[みおぼ]えがありますか。	この ひと に みおぼえ が あります か	
\\	この 人[ひと]に
\\	がありますか。			
\\	目覚まし	目覚[めざ]まし	めざまし	
\\	目覚ましにコーヒーを飲んだの。	目覚[めざ]ましにコーヒーを 飲[の]んだの。	めざまし に こーひー を のんだ の	
\\	にコーヒーを 飲[の]んだの。			
\\	閉会	閉会[へいかい]	へいかい	
\\	会長が閉会の挨拶をしました。	会長[かいちょう]が 閉会[へいかい]の 挨拶[あいさつ]をしました。	かいちょう が へいかい の あいさつ を しました	
\\	会長[かいちょう]が
\\	の 挨拶[あいさつ]をしました。			
\\	どんどん	どんどん	どんどん	
\\	彼は山道をどんどん進んで行ったの。	彼[かれ]は 山道[やまみち]をどんどん 進[すす]んで 行[い]ったの。	かれ は やまみち を どんどん すすんで いった の	
\\	彼[かれ]は 山道[やまみち]を
\\	進[すす]んで 行[い]ったの。			
\\	出会う	出会[であ]う	であう	
\\	アメリカで彼女と出会いました。	アメリカで 彼女[かのじょ]と 出会[であ]いました。	あめりか で かのじょ と であいました。	
\\	アメリカで 彼女[かのじょ]と
\\	話し合い	話[はな]し 合[あ]い	はなしあい	
\\	プロジェクトメンバーと話し合いをしたの。	プロジェクトメンバーと 話[はな]し 合[あ]いをしたの。	ぷろじぇくと めんばー と はなしあい を した の	
\\	プロジェクトメンバーと
\\	をしたの。			
\\	見合い	見合[みあ]い	みあい	
\\	うちの両親はお見合い結婚でした。	うちの 両親[りょうしん]はお 見合[みあ]い 結婚[けっこん]でした。	うち の りょうしん は おみあい けっこん でした	
\\	うちの 両親[りょうしん]はお
\\	結婚[けっこん]でした。			
\\	間に合わせる	間[ま]に 合[あ]わせる	まにあわせる	
\\	昼食はクッキーで間に合わせたの。	昼食[ちゅうしょく]はクッキーで 間[ま]に 合[あ]わせたの。	ちゅうしょく は くっきー で まにあわせた の	
\\	昼食[ちゅうしょく]はクッキーで
\\	の。			
\\	本社	本社[ほんしゃ]	ほんしゃ	
\\	今日は本社で会議があります。	今日[きょう]は 本社[ほんしゃ]で 会議[かいぎ]があります。	きょう は ほんしゃ で かいぎ が あります	
\\	今日[きょう]は
\\	で 会議[かいぎ]があります。			
\\	それほど	それほど	それほど	
\\	彼がそれほど悩んでいたとは知らなかった。	彼[かれ]がそれほど 悩[なや]んでいたとは 知[し]らなかった。	かれ が それほど なやんで いた と は しらなかった	
\\	彼[かれ]が
\\	悩[なや]んでいたとは 知[し]らなかった。			
\\	入社	入社[にゅうしゃ]	にゅうしゃ	
\\	彼は昨年入社したの。	彼[かれ]は 昨年[さくねん] 入社[にゅうしゃ]したの。	かれ は さくねん にゅうしゃ した の	
\\	彼[かれ]は 昨年[さくねん]
\\	したの。			
\\	満足	満足[まんぞく]	まんぞく	
\\	彼は結果に満足したようです。	彼[かれ]は 結果[けっか]に 満足[まんぞく]したようです。	かれ は けっか に まんぞく した よう です	
\\	彼[かれ]は 結果[けっか]に
\\	したようです。			
\\	満たす	満[み]たす	みたす	
\\	彼は応募の条件を満たしていない。	彼[かれ]は 応募[おうぼ]の 条件[じょうけん]を 満[み]たしていない。	かれ は おうぼ の じょうけん を みたして いない	
\\	彼[かれ]は 応募[おうぼ]の 条件[じょうけん]を
\\	いない。			
\\	満ちる	満[み]ちる	みちる	
\\	月が満ちてきましたね。	月[つき]が 満[み]ちてきましたね。	つき が みちて きました ね	
\\	月[つき]が
\\	ね。			
\\	満員	満員[まんいん]	まんいん	
\\	このバスは満員です。	このバスは 満員[まんいん]です。	この ばす は まんいん です	
\\	このバスは
\\	です。			
\\	満月	満月[まんげつ]	まんげつ	
\\	今日は満月ですね。	今日[きょう]は 満月[まんげつ]ですね。	きょう は まんげつ です ね	
\\	今日[きょう]は
\\	ですね。			
\\	もはや	もはや	もはや	
\\	もはや彼の助けは必要じゃないの。	もはや 彼[かれ]の 助[たす]けは 必要[ひつよう]じゃないの。	もはや かれ の たすけ は ひつよう じゃ ない の	
\\	彼[かれ]の 助[たす]けは 必要[ひつよう]じゃないの。			
\\	出来事	出来事[できごと]	できごと	
\\	面白い出来事がありました。	面白[おもしろ]い 出来事[できごと]がありました。	おもしろい できごと が ありました	
\\	面白[おもしろ]い
\\	がありました。			
\\	見事	見事[みごと]	みごと	
\\	彼は見事なジャンプを見せたね。	彼[かれ]は 見事[みごと]なジャンプを 見[み]せたね。	かれ は みごと な じゃんぷ を みせた ね	
\\	彼[かれ]は
\\	なジャンプを 見[み]せたね。			
\\	大工	大工[だいく]	だいく	
\\	私の父は大工です。	私[わたし]の 父[ちち]は 大工[だいく]です。	わたし の ちち は だいく です	
\\	私[わたし]の 父[ちち]は
\\	です。			
\\	場	場[ば]	ば	
\\	この場でお礼を言わせてください。	この 場[ば]でお 礼[れい]を 言[い]わせてください。	この ば で おれい を いわせて ください	
\\	この
\\	でお 礼[れい]を 言[い]わせてください。			
\\	入場	入場[にゅうじょう]	にゅうじょう	
\\	選手の入場です。	選手[せんしゅ]の 入場[にゅうじょう]です。	せんしゅ の にゅうじょう です	
\\	選手[せんしゅ]の
\\	です。			
\\	デザイン	デザイン	デザイン	
\\	この服のデザインは素敵ですね。	この 服[ふく]のデザインは 素敵[すてき]ですね。	この ふく の でざいん は すてき です ね 。	
\\	この 服[ふく]の
\\	は 素敵[すてき]ですね。			
\\	電力	電力[でんりょく]	でんりょく	
\\	このエアコンはあまり電力を使いません。	このエアコンはあまり 電力[でんりょく]を 使[つか]いません。	この えあこん は あまり でんりょく を つかいません	
\\	このエアコンはあまり
\\	を 使[つか]いません。			
\\	電子	電子[でんし]	でんし	
\\	電子辞書はとても便利です。	電子[でんし] 辞書[じしょ]はとても 便利[べんり]です。	でんし じしょ は とても べんり です	
\\	辞書[じしょ]はとても 便利[べんり]です。			
\\	交じる	交[ま]じる	まじる	
\\	彼女は外国人に交じってダンスをしたの。	彼女[かのじょ]は 外国人[がいこくじん]に 交[ま]じってダンスをしたの。	かのじょ は がいこくじん に まじって だんす を した の	
\\	彼女[かのじょ]は 外国人[がいこくじん]に
\\	ダンスをしたの。			
\\	交わる	交[まじ]わる	まじわる	
\\	人と交わって、多くのことを学びました。	人[ひと]と 交[まじ]わって、 多[おお]くのことを 学[まな]びました。	ひと と まじわって おおく の こと を まなびました	
\\	人[ひと]と
\\	、 多[おお]くのことを 学[まな]びました。			
\\	交ぜる	交[ま]ぜる	まぜる	
\\	私も交ぜてください。	私[わたし]も 交[ま]ぜてください。	わたし も まぜて ください	
\\	私[わたし]も
\\	ください。			
\\	ファン	ファン	ファン	
\\	彼は大のサッカーファンです。	彼[かれ]は 大[だい]のサッカーファンです。	かれ は だい の さっかーふぁん です	
\\	彼[かれ]は 大[だい]のサッカー
\\	です。			
\\	交ざる	交[ま]ざる	まざる	
\\	大人も子供も交ざって遊んだの。	大人[おとな]も 子供[こども]も 交[ま]ざって 遊[あそ]んだの。	おとな も こども も まざって あそんだ の	
\\	大人[おとな]も 子供[こども]も
\\	遊[あそ]んだの。			
\\	交わる	交[まじ]わる	まじわる	
\\	二つの国道はここで交わります。	二[ふた]つの 国道[こくどう]はここで 交[まじ]わります。	ふたつ の こくどう は ここ で まじわります	
\\	二[ふた]つの 国道[こくどう]はここで
\\	見通し	見通[みとお]し	みとおし	
\\	仕事の見通しがたたない。	仕事[しごと]の 見通[みとお]しがたたない。	しごと の みとおし が たたない	
\\	仕事[しごと]の
\\	がたたない。			
\\	通じる	通[つう]じる	つうじる	
\\	その国では英語は通じますか。	その 国[くに]では 英語[えいご]は 通[つう]じますか。	その くに で は えいご は つうじます か	
\\	その 国[くに]では 英語[えいご]は
\\	か。			
\\	通す	通[とお]す	とおす	
\\	針に糸を通してください。	針[はり]に 糸[いと]を 通[とお]してください。	はり に いと を とおして ください	
\\	針[はり]に 糸[いと]を
\\	ください。			
\\	通行	通行[つうこう]	つうこう	
\\	この道は通行できません。	この 道[みち]は 通行[つうこう]できません。	この みち は つうこう できません	
\\	この 道[みち]は
\\	できません。			
\\	なさる	なさる	なさる	
\\	お申し込みなさるのでしたら、こちらにお並びください。	お 申[もう]し 込[こ]みなさるのでしたら、こちらにお 並[なら]びください。	おもうしこみ なさる の でしたら こちら に お ならび ください	
\\	お 申[もう]し 込[こ]み
\\	のでしたら、こちらにお 並[なら]びください。			
\\	人通り	人通[ひとどお]り	ひとどおり	
\\	ここは人通りが激しいね。	ここは 人通[ひとどお]りが 激[はげ]しいね。	ここ は ひとどおり が はげしい ね	
\\	ここは
\\	が 激[はげ]しいね。			
\\	一通り	一通[ひととお]り	ひととおり	
\\	説明書を一通り読んだの。	説明書[せつめいしょ]を 一通[ひととお]り 読[よ]んだの。	せつめいしょ を ひととおり よんだ の	
\\	説明書[せつめいしょ]を
\\	読[よ]んだの。			
\\	通路	通路[つうろ]	つうろ	
\\	通路の右側にトイレがありますよ。	通路[つうろ]の 右側[みぎがわ]にトイレがありますよ。	つうろ の みぎがわ に といれ が あります よ	
\\	の 右側[みぎがわ]にトイレがありますよ。			
\\	地上	地上[ちじょう]	ちじょう	
\\	この電車は地上を走ります。	この 電車[でんしゃ]は 地上[ちじょう]を 走[はし]ります。	この でんしゃ は ちじょう を はしります	
\\	この 電車[でんしゃ]は
\\	を 走[はし]ります。			
\\	地	地[ち]	ち	
\\	彼はその地で残りの生涯を過ごしたんだ。	彼[かれ]はその 地[ち]で 残[のこ]りの 生涯[しょうがい]を 過[す]ごしたんだ。	かれ は その ち で のこり の しょうがい を すごした ん だ	
\\	彼[かれ]はその
\\	で 残[のこ]りの 生涯[しょうがい]を 過[す]ごしたんだ。			
\\	まとまる	まとまる	まとまる	
\\	みんなの意見がまとまりました。	みんなの 意見[いけん]がまとまりました。	みんな の いけん が まとまりました	
\\	みんなの 意見[いけん]が
\\	地下道	地下道[ちかどう]	ちかどう	
\\	地下道を通って行きましょう。	地下道[ちかどう]を 通[とお]って 行[い]きましょう。	ちかどう を とおって いきましょう	
\\	を 通[とお]って 行[い]きましょう。			
\\	地下	地下[ちか]	ちか	
\\	スタジオは地下にあります。	スタジオは 地下[ちか]にあります。	すたじお は ちか に あります	
\\	スタジオは
\\	にあります。			
\\	地方	地方[ちほう]	ちほう	
\\	この地方は漁業が盛んです。	この 地方[ちほう]は 漁業[ぎょぎょう]が 盛[さか]んです。	この ちほう は ぎょぎょう が さかん です	
\\	この
\\	は 漁業[ぎょぎょう]が 盛[さか]んです。			
\\	図る	図[はか]る	はかる	
\\	これからは経営の合理化を図りたいと思います。	これからは 経営[けいえい]の 合理化[ごうりか]を 図[はか]りたいと 思[おも]います。	これから は けいえい の ごうりか を はかりたい と おもいます	
\\	これからは 経営[けいえい]の 合理化[ごうりか]を
\\	と 思[おも]います。			
\\	図書	図書[としょ]	としょ	
\\	これは児童図書です。	これは 児童[じどう] 図書[としょ]です。	これ は じどうとしょ です	
\\	これは 児童[じどう]
\\	です。			
\\	ルール	ルール	ルール	
\\	このゲームのルールは簡単です。	このゲームのルールは 簡単[かんたん]です。	この げーむ の るーる は かんたん です	
\\	このゲームの
\\	は 簡単[かんたん]です。			
\\	他方	他方[たほう]	たほう	
\\	他方の視点からも見てみましょう。	他方[たほう]の 視点[してん]からも 見[み]てみましょう。	たほう の してん から も みて みましょう	
\\	の 視点[してん]からも 見[み]てみましょう。			
\\	他人	他人[たにん]	たにん	
\\	私は他人に住所を教えたくない。	私[わたし]は 他人[たにん]に 住所[じゅうしょ]を 教[おし]えたくない。	わたし は たにん に じゅうしょ を おしえたく ない	
\\	私[わたし]は
\\	に 住所[じゅうしょ]を 教[おし]えたくない。			
\\	中止	中止[ちゅうし]	ちゅうし	
\\	雨で運動会が中止になったの。	雨[あめ]で 運動会[うんどうかい]が 中止[ちゅうし]になったの。	あめ で うんどうかい が ちゅうし に なった の	
\\	雨[あめ]で 運動会[うんどうかい]が
\\	になったの。			
\\	通行止め	通行止[つうこうど]め	つうこうどめ	
\\	あの道路は通行止めだそうです。	あの 道路[どうろ]は 通行止[つうこうど]めだそうです。	あの どうろ は つうこうどめ だ そう です	
\\	あの 道路[どうろ]は
\\	だそうです。			
\\	見渡す	見渡[みわた]す	みわたす	
\\	丘の上から草原を見渡したの。	丘[おか]の 上[うえ]から 草原[そうげん]を 見渡[みわた]したの。	おか の うえ から そうげん を みわたした の	
\\	丘[おか]の 上[うえ]から 草原[そうげん]を
\\	の。			
\\	二度と	二度[にど]と	にどと	
\\	もうここには二度と来ません。	もうここには 二度[にど]と 来[き]ません。	もう ここ に は にどと きません	
\\	もうここには
\\	来[き]ません。			
\\	ヘリコプター	ヘリコプター	ヘリコプター	
\\	事故現場の上空をヘリコプターが飛んでいます。	事故現場[じこ げんば]の 上空[じょうくう]をヘリコプターが 飛[と]んでいます。	じこ げんば の じょうくう を へりこぷたー が とんで います	
\\	事故現場[じこ げんば]の 上空[じょうくう]を
\\	が 飛[と]んでいます。			
\\	度々	度々[たびたび]	たびたび	
\\	彼から度々メールが来ます。	彼[かれ]から 度々[たびたび]メールが 来[き]ます。	かれ から たびたび めーる が きます	
\\	彼[かれ]から
\\	メールが 来[き]ます。			
\\	この度	この 度[たび]	このたび	
\\	この度はご結婚おめでとうございます。	この 度[たび]はご 結婚[けっこん]おめでとうございます。	このたび は ごけっこん おめでとう ございます	
\\	はご 結婚[けっこん]おめでとうございます。			
\\	毎度	毎度[まいど]	まいど	
\\	毎度ありがとうございます。	毎度[まいど]ありがとうございます。	まいど ありがとう ございます	
\\	ありがとうございます。			
\\	間近	間近[まぢか]	まぢか	
\\	有名人を間近で見たよ。	有名人[ゆうめいじん]を 間近[まぢか]で 見[み]たよ。	ゆうめいじん を まぢか で みた よ	
\\	有名人[ゆうめいじん]を
\\	で 見[み]たよ。			
\\	近道	近道[ちかみち]	ちかみち	
\\	こっちが近道です。	こっちが 近道[ちかみち]です。	こっち が ちかみち です	
\\	こっちが
\\	です。			
\\	ともかく	ともかく	ともかく	
\\	ともかく一度考え直しましょう。	ともかく 一度考[いちど かんが]え 直[なお]しましょう。	ともかく いちど かんがえなおしましょう	
\\	一度考[いちど かんが]え 直[なお]しましょう。			
\\	長年	長年[ながねん]	ながねん	
\\	彼は長年の友人です。	彼[かれ]は 長年[ながねん]の 友人[ゆうじん]です。	かれ は ながねん の ゆうじん です	
\\	彼[かれ]は
\\	の 友人[ゆうじん]です。			
\\	年長	年長[ねんちょう]	ねんちょう	
\\	彼がこのグループで一番年長です。	彼[かれ]がこのグループで 一番[いちばん] 年長[ねんちょう]です。	かれ が この ぐるーぷ で いちばん ねんちょう です	
\\	彼[かれ]がこのグループで 一番[いちばん]
\\	です。			
\\	長らく	長[なが]らく	ながらく	
\\	長らくお待たせしました。	長[なが]らくお 待[ま]たせしました。	ながらく お またせ しました 。	
\\	お 待[ま]たせしました。			
\\	長話	長話[ながばなし]	ながばなし	
\\	母が電話で長話をしているんだ。	母[はは]が 電話[でんわ]で 長話[ながばなし]をしているんだ。	はは が でんわ で ながばなし を して いる ん だ	
\\	母[はは]が 電話[でんわ]で
\\	をしているんだ。			
\\	短大	短大[たんだい]	たんだい	
\\	妹は短大を卒業しました。	妹[いもうと]は 短大[たんだい]を 卒業[そつぎょう]しました。	いもうと は たんだい を そつぎょう しました	
\\	妹[いもうと]は
\\	を 卒業[そつぎょう]しました。			
\\	つい	つい	つい	
\\	会議中、ついあくびをしてしまいました。	会議中[かいぎちゅう]、ついあくびをしてしまいました。	かいぎちゅう つい あくび を して しまいました	
\\	会議中[かいぎちゅう]、
\\	あくびをしてしまいました。			
\\	短気	短気[たんき]	たんき	
\\	彼は短気な人です。	彼[かれ]は 短気[たんき]な 人[ひと]です。	かれ は たんき な ひと です	
\\	彼[かれ]は
\\	な 人[ひと]です。			
\\	広げる	広[ひろ]げる	ひろげる	
\\	電車の中では新聞を広げないで。	電車[でんしゃ]の 中[なか]では 新聞[しんぶん]を 広[ひろ]げないで。	でんしゃ の なか で は しんぶん を ひろげない で	
\\	電車[でんしゃ]の 中[なか]では 新聞[しんぶん]を
\\	広場	広場[ひろば]	ひろば	
\\	広場に子供が沢山集まっていたよ。	広場[ひろば]に 子供[こども]が 沢山集[たくさん あつ]まっていたよ。	ひろば に こども が たくさん あつまって いた よ	
\\	に 子供[こども]が 沢山集[たくさん あつ]まっていたよ。			
\\	広まる	広[ひろ]まる	ひろまる	
\\	その噂はすぐに広まったよ。	その 噂[うわさ]はすぐに 広[ひろ]まったよ。	その うわさ は すぐ に ひろまった よ	
\\	その 噂[うわさ]はすぐに
\\	よ。			
\\	広める	広[ひろ]める	ひろめる	
\\	誰が噂を広めたんだろう。	誰[だれ]が 噂[うわさ]を 広[ひろ]めたんだろう。	だれ が うわさ を ひろめた ん だろう	
\\	誰[だれ]が 噂[うわさ]を
\\	んだろう。			
\\	トンネル	トンネル	トンネル	
\\	トンネルを抜けると海が見えたよ。	トンネルを 抜[ぬ]けると 海[うみ]が 見[み]えたよ。	とんねる を ぬける と うみ が みえた よ	
\\	を 抜[ぬ]けると 海[うみ]が 見[み]えたよ。			
\\	全力	全力[ぜんりょく]	ぜんりょく	
\\	全力で走れ。	全力[ぜんりょく]で 走[はし]れ。	ぜんりょく で はしれ	
\\	で 走[はし]れ。			
\\	部門	部門[ぶもん]	ぶもん	
\\	あの歌手は3部門で賞を取ったの。	あの 歌手[かしゅ]は3 部門[ぶもん]で 賞[しょう]を 取[と]ったの。	あの かしゅ は 
\\	ぶもん で しょう を とった の	
\\	あの 歌手[かしゅ]は3
\\	で 賞[しょう]を 取[と]ったの。			
\\	本部	本部[ほんぶ]	ほんぶ	
\\	その事件の直後、捜査本部が設置された。	その 事件[じけん]の 直後[ちょくご]、 捜査[そうさ] 本部[ほんぶ]が 設置[せっち]された。	その じけん の ちょくご そうさ ほんぶ が せっち された	
\\	その 事件[じけん]の 直後[ちょくご]、 捜査[そうさ]
\\	が 設置[せっち]された。			
\\	部長	部長[ぶちょう]	ぶちょう	
\\	部長に仕事の相談をしたんだ。	部長[ぶちょう]に 仕事[しごと]の 相談[そうだん]をしたんだ。	ぶちょう に しごと の そうだん を した ん だ	
\\	に 仕事[しごと]の 相談[そうだん]をしたんだ。			
\\	内部	内部[ないぶ]	ないぶ	
\\	これは機械の内部の問題です。	これは 機械[きかい]の 内部[ないぶ]の 問題[もんだい]です。	これ は きかい の ないぶ の もんだい です	
\\	これは 機械[きかい]の
\\	の 問題[もんだい]です。			
\\	部下	部下[ぶか]	ぶか	
\\	彼は優秀な部下を持っているわね。	彼[かれ]は 優秀[ゆうしゅう]な 部下[ぶか]を 持[も]っているわね。	かれ は ゆうしゅう な ぶか を もって いる わ ね	
\\	彼[かれ]は 優秀[ゆうしゅう]な
\\	を 持[も]っているわね。			
\\	バランス	バランス	バランス	
\\	栄養バランスの良い食事をしよう。	栄養[えいよう]バランスの 良[い]い 食事[しょくじ]をしよう。	えいよう ばらんす の いい しょくじ を しよう	
\\	栄養[えいよう]
\\	の 良[い]い 食事[しょくじ]をしよう。			
\\	大部分	大部分[だいぶぶん]	だいぶぶん	
\\	絵の大部分が水に濡れてしまったな。	絵[え]の 大部分[だいぶぶん]が 水[みず]に 濡[ぬ]れてしまったな。	え の だいぶぶん が みず に ぬれて しまった な	
\\	絵[え]の
\\	が 水[みず]に 濡[ぬ]れてしまったな。			
\\	大国	大国[たいこく]	たいこく	
\\	その国は経済大国よ。	その 国[くに]は 経済[けいざい] 大国[たいこく]よ。	その くに は けいざい たいこく よ	
\\	その 国[くに]は 経済[けいざい]
\\	よ。			
\\	入国	入国[にゅうこく]	にゅうこく	
\\	私は留学生として日本に入国しました。	私[わたし]は 留学生[りゅうがくせい]として 日本[にっぽん]に 入国[にゅうこく]しました。	わたし は りゅうがくせい として にっぽん に にゅうこく しました	
\\	私[わたし]は 留学生[りゅうがくせい]として 日本[にっぽん]に
\\	しました。			
\\	本国	本国[ほんごく]	ほんごく	
\\	彼女は本国に帰りました。	彼女[かのじょ]は 本国[ほんごく]に 帰[かえ]りました。	かのじょ は ほんごく に かえりました	
\\	彼女[かのじょ]は
\\	に 帰[かえ]りました。			
\\	天国	天国[てんごく]	てんごく	
\\	死んだら天国に行きたいです。	死[し]んだら 天国[てんごく]に 行[い]きたいです。	しんだら てんごく に いきたい です	
\\	死[し]んだら
\\	に 行[い]きたいです。			
\\	どうしても	どうしても	どうしても	
\\	どうしてもその訳を知りたい。	どうしてもその 訳[わけ]を 知[し]りたい。	どうしても その わけ を しりたい	
\\	その 訳[わけ]を 知[し]りたい。			
\\	世の中	世[よ]の 中[なか]	よのなか	
\\	世の中にはいろいろな人がいます。	世[よ]の 中[なか]にはいろいろな 人[ひと]がいます。	よのなか に は いろいろ な ひと が います	
\\	にはいろいろな 人[ひと]がいます。			
\\	中世	中世[ちゅうせい]	ちゅうせい	
\\	彼女は中世の音楽が好きです。	彼女[かのじょ]は 中世[ちゅうせい]の 音楽[おんがく]が 好[す]きです。	かのじょ は ちゅうせい の おんがく が すき です	
\\	彼女[かのじょ]は
\\	の 音楽[おんがく]が 好[す]きです。			
\\	世	世[よ]	よ	
\\	やっと私の作品が世に出たの。	やっと 私[わたし]の 作品[さくひん]が 世[よ]に 出[で]たの。	やっと わたし の さくひん が よ に でた の	
\\	やっと 私[わたし]の 作品[さくひん]が
\\	に 出[で]たの。			
\\	明白	明白[めいはく]	めいはく	
\\	彼が犯人なのは明白です。	彼[かれ]が 犯人[はんにん]なのは 明白[めいはく]です。	かれ が はんにん な の は めいはく です	
\\	彼[かれ]が 犯人[はんにん]なのは
\\	です。			
\\	鉄道	鉄道[てつどう]	てつどう	
\\	日本は鉄道がとても発達しています。	日本[にっぽん]は 鉄道[てつどう]がとても 発達[はったつ]しています。	にっぽん は てつどう が とても はったつ して います	
\\	日本[にっぽん]は
\\	がとても 発達[はったつ]しています。			
\\	どうか	どうか	どうか	
\\	どうかお許しください。	どうかお 許[ゆる]しください。	どうか おゆるし ください	
\\	お 許[ゆる]しください。			
\\	鉄	鉄[てつ]	てつ	
\\	この鍋は鉄でできています。	この 鍋[なべ]は 鉄[てつ]でできています。	この なべ は てつ で できて います	
\\	この 鍋[なべ]は
\\	でできています。			
\\	肉体	肉体[にくたい]	にくたい	
\\	肉体はいつか滅びます。	肉体[にくたい]はいつか 滅[ほろ]びます。	にくたい は いつか ほろびます	
\\	はいつか 滅[ほろ]びます。			
\\	白菜	白菜[はくさい]	はくさい	
\\	彼女は白菜の漬物が好きです。	彼女[かのじょ]は 白菜[はくさい]の 漬物[つけもの]が 好[す]きです。	かのじょ は はくさい の つけもの が すき です	
\\	彼女[かのじょ]は
\\	の 漬物[つけもの]が 好[す]きです。			
\\	日米	日米[にちべい]	にちべい	
\\	テレビで日米野球をやっていますよ。	テレビで 日米[にちべい] 野球[やきゅう]をやっていますよ。	てれび で にちべい やきゅう を やって います よ	
\\	テレビで
\\	野球[やきゅう]をやっていますよ。			
\\	味方	味方[みかた]	みかた	
\\	母はいつも私の味方です。	母[はは]はいつも 私[わたし]の 味方[みかた]です。	はは は いつも わたし の みかた です	
\\	母[はは]はいつも 私[わたし]の
\\	です。			
\\	年末	年末[ねんまつ]	ねんまつ	
\\	年末のセールはいつも込んでいるね。	年末[ねんまつ]のセールはいつも 込[こ]んでいるね。	ねんまつ の せーる は いつも こんで いる ね	
\\	のセールはいつも 込[こ]んでいるね。			
\\	プロ	プロ	プロ	
\\	彼はスケートのプロです。	彼[かれ]はスケートのプロです。	かれ は すけーと の ぷろ です	
\\	彼[かれ]はスケートの
\\	です。			
\\	末	末[まつ]	まつ	
\\	今月末にカナダに行きます。	今月[こんげつ] 末[まつ]にカナダに 行[い]きます。	こんげつ まつ に かなだ に いきます	
\\	今月[こんげつ]
\\	にカナダに 行[い]きます。			
\\	料金	料金[りょうきん]	りょうきん	
\\	まだ料金は払っていないけど。	まだ 料金[りょうきん]は 払[はら]っていないけど。	まだ りょうきん は はらって いない けど	
\\	まだ
\\	は 払[はら]っていないけど。			
\\	地理	地理[ちり]	ちり	
\\	彼は地理に詳しいの。	彼[かれ]は 地理[ちり]に 詳[くわ]しいの。	かれ は ちり に くわしい の	
\\	彼[かれ]は
\\	に 詳[くわ]しいの。			
\\	理解し合う	理解し合う[りかいしあう]	りかいしあう	
\\	日本語を勉強したらお互いに理解し合うことができる	にほんごを勉強したらお互いに理解し合う[りかいしあう]ことができる。	にほんごをべんきょうしたらおたがい理解し合うことができる。	
\\	してくれてありがとう。			
\\	解く	解[と]く	とく	
\\	この問題を解けますか。	この 問題[もんだい]を 解[と]けますか。	この もんだい を とけます か	
\\	この 問題[もんだい]を
\\	か。			
\\	チャンス	チャンス	チャンス	
\\	これは素晴らしいチャンスだ。	これは 素晴[すば]らしいチャンスだ。	これ は すばらしい ちゃんす だ	
\\	これは 素晴[すば]らしい
\\	だ。			
\\	分解	分解[ぶんかい]	ぶんかい	
\\	機械を分解してみたの。	機械[きかい]を 分解[ぶんかい]してみたの。	きかい を ぶんかい して みた の	
\\	機械[きかい]を
\\	してみたの。			
\\	解ける	解[と]ける	とける	
\\	やっと難しい問題が解けました。	やっと 難[むずか]しい 問題[もんだい]が 解[と]けました。	やっと むずかしい もんだい が とけました	
\\	やっと 難[むずか]しい 問題[もんだい]が
\\	有力	有力[ゆうりょく]	ゆうりょく	
\\	あの都市はオリンピックの有力な候補地です。	あの 都市[とし]はオリンピックの 有力[ゆうりょく]な 候補地[こうほち]です。	あの とし は おりんぴっく の ゆうりょく な こうほち です	
\\	あの 都市[とし]はオリンピックの
\\	な 候補地[こうほち]です。			
\\	有する	有[ゆう]する	ゆうする	
\\	資格を有する人のみ応募できます。	資格[しかく]を 有[ゆう]する 人[ひと]のみ 応募[おうぼ]できます。	しかく を ゆうする ひと のみ おうぼ できます	
\\	資格[しかく]を
\\	人[ひと]のみ 応募[おうぼ]できます。			
\\	有料	有料[ゆうりょう]	ゆうりょう	
\\	このトイレは有料です。	このトイレは 有料[ゆうりょう]です。	この といれ は ゆうりょう です	
\\	このトイレは
\\	です。			
\\	どうも	どうも	どうも	
\\	今日はどうも体の調子が悪い。	今日[きょう]はどうも 体[からだ]の 調子[ちょうし]が 悪[わる]い。	きょう は どうも からだ の ちょうし が わるい	
\\	今日[きょう]は
\\	体[からだ]の 調子[ちょうし]が 悪[わる]い。			
\\	無理	無理[むり]	むり	
\\	無理はしないでください。	無理[むり]はしないでください。	むり は しない で ください	
\\	はしないでください。			
\\	無料	無料[むりょう]	むりょう	
\\	お飲み物は無料でございます。	お 飲[の]み 物[もの]は 無料[むりょう]でございます。	お のみもの は むりょう で ございます	
\\	お 飲[の]み 物[もの]は
\\	でございます。			
\\	無事	無事[ぶじ]	ぶじ	
\\	無事、家に着きました。	無事[ぶじ]、 家[いえ]に 着[つ]きました。	ぶじ いえ に つきました	
\\	、 家[いえ]に 着[つ]きました。			
\\	無口	無口[むくち]	むくち	
\\	彼女は無口な人ですね。	彼女[かのじょ]は 無口[むくち]な 人[ひと]ですね。	かのじょ は むくち な ひと です ね	
\\	彼女[かのじょ]は
\\	な 人[ひと]ですね。			
\\	間も無く	間[ま]も 無[な]く	まもなく	
\\	あの飛行機は間も無く着陸しますね。	あの 飛行機[ひこうき]は 間[ま]も 無[な]く 着陸[ちゃくりく]しますね。	あの ひこうき は まもなく ちゃくりく します ね	
\\	あの 飛行機[ひこうき]は
\\	着陸[ちゃくりく]しますね。			
\\	無言	無言[むごん]	むごん	
\\	彼は一日中無言だったな。	彼[かれ]は 一日中[いちにちじゅう] 無言[むごん]だったな。	かれ は いちにちじゅう むごん だった な	
\\	彼[かれ]は 一日中[いちにちじゅう]
\\	だったな。			
\\	ハード	ハード	ハード	
\\	最近、仕事がかなりハードです。	最近[さいきん]、 仕事[しごと]がかなりハードです。	さいきん しごと が かなり はーど です	
\\	最近[さいきん]、 仕事[しごと]がかなり
\\	です。			
\\	無茶	無茶[むちゃ]	むちゃ	
\\	無茶をしないでくださいね。	無茶[むちゃ]をしないでくださいね。	むちゃ を しない で ください ね	
\\	をしないでくださいね。			
\\	無理やり	無理[むり]やり	むりやり	
\\	荷物を無理やりかばんに詰めたよ。	荷物[にもつ]を 無理[むり]やりかばんに 詰[つ]めたよ。	にもつ を むりやり かばん に つめた よ	
\\	荷物[にもつ]を
\\	かばんに 詰[つ]めたよ。			
\\	作り上げる	作[つく]り 上[あ]げる	つくりあげる	
\\	それは彼女が作り上げた話です。	それは 彼女[かのじょ]が 作[つく]り 上[あ]げた 話[はなし]です。	それ は かのじょ が つくりあげた はなし です	
\\	それは 彼女[かのじょ]が
\\	話[はなし]です。			
\\	作り話	作[つく]り 話[ばなし]	つくりばなし	
\\	その子が言っていることは作り話です。	その 子[こ]が 言[い]っていることは 作[つく]り 話[ばなし]です。	その こ が いって いる こと は つくりばなし です	
\\	その 子[こ]が 言[い]っていることは
\\	です。			
\\	大使	大使[たいし]	たいし	
\\	彼は昔、ドイツの大使でした。	彼[かれ]は 昔[むかし]、ドイツの 大使[たいし]でした。	かれ は むかし どいつ の たいし でした	
\\	彼[かれ]は 昔[むかし]、ドイツの
\\	でした。			
\\	マイナス	マイナス	マイナス	
\\	外の温度はマイナス3度です。	外[そと]の 温度[おんど]はマイナス3 度[ど]です。	そと の おんど は まいなす 
\\	ど です	
\\	外[そと]の 温度[おんど]は
\\	度[ど]です。			
\\	使い道	使[つか]い 道[みち]	つかいみち	
\\	ボーナスの使い道を考えているところです。	ボーナスの 使[つか]い 道[みち]を 考[かんが]えているところです。	ぼーなす の つかいみち を かんがえて いる ところ です	
\\	ボーナスの
\\	を 考[かんが]えているところです。			
\\	使い	使[つか]い	つかい	
\\	主人の使いで市役所に行くところです。	主人[しゅじん]の 使[つか]いで 市役所[しやくしょ]に 行[い]くところです。	しゅじん の つかい で しやくしょ に いく ところ です	
\\	主人[しゅじん]の
\\	で 市役所[しやくしょ]に 行[い]くところです。			
\\	用	用[よう]	よう	
\\	私に何か用ですか。	私[わたし]に 何[なに]か 用[よう]ですか。	わたし に なにか よう です か	
\\	私[わたし]に 何[なに]か
\\	ですか。			
\\	無用	無用[むよう]	むよう	
\\	心配は無用です。	心配[しんぱい]は 無用[むよう]です。	しんぱい は むよう です	
\\	心配[しんぱい]は
\\	です。			
\\	費用	費用[ひよう]	ひよう	
\\	イタリア旅行の費用は30万円です。	イタリア 旅行[りょこう]の 費用[ひよう]は30 万円[まんえん]です。	いたりあ りょこう の ひよう は 
\\	まんえん です	
\\	イタリア 旅行[りょこう]の
\\	は30 万円[まんえん]です。			
\\	とにかく	とにかく	とにかく	
\\	とにかく現場へ行ってみましょう。	とにかく 現場[げんば]へ 行[い]ってみましょう。	とにかく げんば へ いって みましょう	
\\	現場[げんば]へ 行[い]ってみましょう。			
\\	費やす	費[つい]やす	ついやす	
\\	私は語学の勉強にかなりの時間を費やしています。	私[わたし]は 語学[ごがく]の 勉強[べんきょう]にかなりの 時間[じかん]を 費[つい]やしています。	わたし は ごがく の べんきょう に かなり の じかん を ついやして います	
\\	私[わたし]は 語学[ごがく]の 勉強[べんきょう]にかなりの 時間[じかん]を
\\	安売り	安売[やすう]り	やすうり	
\\	あの店で野菜の安売りをしていましたよ。	あの 店[みせ]で 野菜[やさい]の 安売[やすう]りをしていましたよ。	あの みせ で やさい の やすうり を して いました よ	
\\	あの 店[みせ]で 野菜[やさい]の
\\	をしていましたよ。			
\\	売買	売買[ばいばい]	ばいばい	
\\	彼は不動産の売買をしています。	彼[かれ]は 不動産[ふどうさん]の 売買[ばいばい]をしています。	かれ は ふどうさん の ばいばい を して います	
\\	彼[かれ]は 不動産[ふどうさん]の
\\	をしています。			
\\	閉店	閉店[へいてん]	へいてん	
\\	この店は8時に閉店します。	この 店[みせ]は8 時[じ]に 閉店[へいてん]します。	この みせ は 
\\	じ に へいてん します	
\\	この 店[みせ]は8 時[じ]に
\\	します。			
\\	本店	本店[ほんてん]	ほんてん	
\\	ここはチェーン店の本店です。	ここはチェーン 店[てん]の 本店[ほんてん]です。	ここ は ちぇーんてん の ほんてん です	
\\	ここはチェーン 店[てん]の
\\	です。			
\\	パイプ	パイプ	パイプ	
\\	このパイプは詰まっていますよ。	このパイプは 詰[つ]まっていますよ。	この ぱいぷ は つまって います よ	
\\	この
\\	は 詰[つ]まっていますよ。			
\\	部品	部品[ぶひん]	ぶひん	
\\	車の部品を取り替えたんだ。	車[くるま]の 部品[ぶひん]を 取[と]り 替[か]えたんだ。	くるま の ぶひん を とりかえた ん だ	
\\	車[くるま]の
\\	を 取[と]り 替[か]えたんだ。			
\\	日用品	日用品[にちようひん]	にちようひん	
\\	今日は日用品の買い物をした。	今日[きょう]は 日用品[にちようひん]の 買[か]い 物[もの]をした。	きょう は にちようひん の かいもの を した	
\\	今日[きょう]は
\\	の 買[か]い 物[もの]をした。			
\\	段	段[だん]	だん	
\\	この階段は18段ありますね。	この 階[かい] 段[だん]は18 段[だん]ありますね。	この かいだん は 
\\	だん あります ね	
\\	この 階[かい]
\\	は18 段[だん]ありますね。			
\\	値上がり	値上[ねあ]がり	ねあがり	
\\	野菜が値上がりしていますね。	野菜[やさい]が 値上[ねあ]がりしていますね。	やさい が ねあがり して います ね	
\\	野菜[やさい]が
\\	していますね。			
\\	値上げ	値上[ねあ]げ	ねあげ	
\\	バス代が値上げされました。	バス 代[だい]が 値上[ねあ]げされました。	ばすだい が ねあげ されました	
\\	バス 代[だい]が
\\	されました。			
\\	値下がり	値下[ねさ]がり	ねさがり	
\\	ガソリンが値下がりしました。	ガソリンが 値下[ねさ]がりしました。	がそりん が ねさがり しました	
\\	ガソリンが
\\	しました。			
\\	たちまち	たちまち	たちまち	
\\	空がたちまち曇ってきたね。	空[そら]がたちまち 曇[くも]ってきたね。	そら が たちまち くもって きた ね	
\\	空[そら]が
\\	曇[くも]ってきたね。			
\\	値下げ	値下[ねさ]げ	ねさげ	
\\	電話料金が値下げされた。	電話料金[でんわ りょうきん]が 値下[ねさ]げされた。	でんわ りょうきん が ねさげ された	
\\	電話料金[でんわ りょうきん]が
\\	された。			
\\	地価	地価[ちか]	ちか	
\\	東京の地価は上がり続けているんだ。	東京[とうきょう]の 地価[ちか]は 上[あ]がり 続[つづ]けているんだ。	とうきょう の ちか は あがりつづけて いる ん だ	
\\	東京[とうきょう]の
\\	は 上[あ]がり 続[つづ]けているんだ。			
\\	体格	体格[たいかく]	たいかく	
\\	彼はとても体格がいいわ。	彼[かれ]はとても 体格[たいかく]がいいわ。	かれ は とても たいかく が いい わ	
\\	彼[かれ]はとても
\\	がいいわ。			
\\	冬季	冬季[とうき]	とうき	
\\	次の冬季オリンピックは2年後です。	次[つぎ]の 冬季[とうき]オリンピックは2 年後[ねんご]です。	つぎ の とうき おりんぴっく は 
\\	ねんご です	
\\	次[つぎ]の
\\	オリンピックは2 年後[ねんご]です。			
\\	熱する	熱[ねっ]する	ねっする	
\\	鉄は熱すると曲がるんだ。	鉄[てつ]は 熱[ねっ]すると 曲[ま]がるんだ。	てつ は ねっする と まがる ん だ	
\\	鉄[てつ]は
\\	と 曲[ま]がるんだ。			
\\	プラス	プラス	プラス	
\\	この経験はあなたにとってプラスになるでしょう。	この 経験[けいけん]はあなたにとってプラスになるでしょう。	この けいけん は あなた に とって ぷらす に なる でしょう	
\\	この 経験[けいけん]はあなたにとって
\\	になるでしょう。			
\\	低温	低温[ていおん]	ていおん	
\\	この製品は低温で保存してください。	この 製品[せいひん]は 低温[ていおん]で 保存[ほぞん]してください。	この せいひん は ていおん で ほぞん して ください	
\\	この 製品[せいひん]は
\\	で 保存[ほぞん]してください。			
\\	体温	体温[たいおん]	たいおん	
\\	今朝の体温は36度でした。	今朝[けさ]の 体温[たいおん]は36 度[ど]でした。	けさ の たいおん は 
\\	ど でした	
\\	今朝[けさ]の
\\	は36 度[ど]でした。			
\\	友情	友情[ゆうじょう]	ゆうじょう	
\\	彼らは強い友情で結ばれているな。	彼[かれ]らは 強[つよ]い 友情[ゆうじょう]で 結[むす]ばれているな。	かれら は つよい ゆうじょう で むすばれて いる な	
\\	彼[かれ]らは 強[つよ]い
\\	で 結[むす]ばれているな。			
\\	人情	人情[にんじょう]	にんじょう	
\\	この町の人たちには人情があるね。	この 町[まち]の 人[ひと]たちには 人情[にんじょう]があるね。	この まち の ひとたち に は にんじょう が ある ね	
\\	この 町[まち]の 人[ひと]たちには
\\	があるね。			
\\	情けない	情[なさ]けない	なさけない	
\\	こんなことも知らないとは情けない。	こんなことも 知[し]らないとは 情[なさ]けない。	こんな こと も しらない と は なさけない	
\\	こんなことも 知[し]らないとは
\\	リード	リード	リード	
\\	彼は彼女をリードしながら踊ったね。	彼[かれ]は 彼女[かのじょ]をリードしながら 踊[おど]ったね。	かれ は かのじょ を りーど しながら おどった ね	
\\	彼[かれ]は 彼女[かのじょ]を
\\	しながら 踊[おど]ったね。			
\\	報道	報道[ほうどう]	ほうどう	
\\	夜中もテレビで台風の報道をしていた。	夜中[よなか]もテレビで 台風[たいふう]の 報道[ほうどう]をしていた。	よなか も てれび で たいふう の ほうどう を して いた	
\\	夜中[よなか]もテレビで 台風[たいふう]の
\\	をしていた。			
\\	電報	電報[でんぽう]	でんぽう	
\\	実家の母から電報が来ました。	実家[じっか]の 母[はは]から 電報[でんぽう]が 来[き]ました。	じっか の はは から でんぽう が きました	
\\	実家[じっか]の 母[はは]から
\\	が 来[き]ました。			
\\	中古	中古[ちゅうこ]	ちゅうこ	
\\	その車は中古で買ったんだ。	その 車[くるま]は 中古[ちゅうこ]で 買[か]ったんだ。	その くるま は ちゅうこ で かった ん だ	
\\	その 車[くるま]は
\\	で 買[か]ったんだ。			
\\	古本	古本[ふるほん]	ふるほん	
\\	おととい古本を3冊買いました。	おととい 古本[ふるほん]を3 冊買[さつ か]いました。	おととい ふるほん を 
\\	さつ かいました	
\\	おととい
\\	を3 冊買[さつ か]いました。			
\\	昔	昔[むかし]	むかし	
\\	彼は昔は貧乏だった。	彼[かれ]は 昔[むかし]は 貧乏[びんぼう]だった。	かれ は むかし は びんぼう だった	
\\	彼[かれ]は
\\	は 貧乏[びんぼう]だった。			
\\	悪口	悪口[わるくち]	わるくち	
\\	彼は決して人の悪口を言わないの。	彼[かれ]は 決[けっ]して 人[ひと]の 悪口[わるくち]を 言[い]わないの。	かれ は けっして ひと の わるくち を いわない の	
\\	彼[かれ]は 決[けっ]して 人[ひと]の
\\	を 言[い]わないの。			
\\	たとえ	たとえ	たとえ	
\\	たとえ、嵐になっても絶対に行く。	たとえ、 嵐[あらし]になっても 絶対[ぜったい]に 行[い]く。	たとえ あらし に なって も ぜったい に いく	
\\	、 嵐[あらし]になっても 絶対[ぜったい]に 行[い]く。			
\\	熱心	熱心[ねっしん]	ねっしん	
\\	彼女は4年間熱心に勉強したわ。	彼女[かのじょ]は4 年間[ねんかん] 熱心[ねっしん]に 勉強[べんきょう]したわ。	かのじょ は 
\\	ねんかん ねっしん に べんきょう した わ	
\\	彼女[かのじょ]は4 年間[ねんかん]
\\	に 勉強[べんきょう]したわ。			
\\	良心	良心[りょうしん]	りょうしん	
\\	私は良心に従って行動します。	私[わたし]は 良心[りょうしん]に 従[したが]って 行動[こうどう]します。	わたし は りょうしん に したがって こうどう します	
\\	私[わたし]は
\\	に 従[したが]って 行動[こうどう]します。			
\\	内心	内心[ないしん]	ないしん	
\\	彼女は内心どきどきしていましたよ。	彼女[かのじょ]は 内心[ないしん]どきどきしていましたよ。	かのじょ は ないしん どきどき して いました よ	
\\	彼女[かのじょ]は
\\	どきどきしていましたよ。			
\\	用心	用心[ようじん]	ようじん	
\\	夜道は用心して歩きましょう。	夜道[よみち]は 用心[ようじん]して 歩[ある]きましょう。	よみち は ようじん して あるきましょう	
\\	夜道[よみち]は
\\	して 歩[ある]きましょう。			
\\	忘年会	忘年会[ぼうねんかい]	ぼうねんかい	
\\	明日は会社の忘年会があります。	明日[あした]は 会社[かいしゃ]の 忘年会[ぼうねんかい]があります。	あした は かいしゃ の ぼうねんかい が あります	
\\	明日[あした]は 会社[かいしゃ]の
\\	があります。			
\\	なるべく	なるべく	なるべく	
\\	なるべく早く仕事を終わらせてください。	なるべく 早[はや]く 仕事[しごと]を 終[お]わらせてください。	なるべく はやく しごと を おわらせて ください	
\\	早[はや]く 仕事[しごと]を 終[お]わらせてください。			
\\	度忘れ	度忘[どわす]れ	どわすれ	
\\	彼の名前を度忘れしたぞ。	彼[かれ]の 名前[なまえ]を 度忘[どわす]れしたぞ。	かれ の なまえ を どわすれ した ぞ	
\\	彼[かれ]の 名前[なまえ]を
\\	したぞ。			
\\	知事	知事[ちじ]	ちじ	
\\	彼は有能な知事ね。	彼[かれ]は 有能[ゆうのう]な 知事[ちじ]ね。	かれ は ゆうのう な ちじ ね	
\\	彼[かれ]は 有能[ゆうのう]な
\\	ね。			
\\	未知	未知[みち]	みち	
\\	ここからは未知の領域です。	ここからは 未知[みち]の 領域[りょういき]です。	ここ から は みち の りょういき です	
\\	ここからは
\\	の 領域[りょういき]です。			
\\	通知	通知[つうち]	つうち	
\\	明日、詳細を通知します。	明日[あす]、 詳細[しょうさい]を 通知[つうち]します。	あす、 しょうさい を つうち します	
\\	明日[あす]、 詳細[しょうさい]を
\\	します。			
\\	無知	無知[むち]	むち	
\\	彼女は政治について無知でした。	彼女[かのじょ]は 政治[せいじ]について 無知[むち]でした。	かのじょ は せいじ に ついて むち でした	
\\	彼女[かのじょ]は 政治[せいじ]について
\\	でした。			
\\	テーマ	テーマ	テーマ	
\\	講演のテーマは何ですか。	講演[こうえん]のテーマは 何[なん]ですか。	こうえん の てーま は なん です か	
\\	講演[こうえん]の
\\	は 何[なん]ですか。			
\\	知人	知人[ちじん]	ちじん	
\\	彼は昔からの知人です。	彼[かれ]は 昔[むかし]からの 知人[ちじん]です。	かれ は むかし から の ちじん です	
\\	彼[かれ]は 昔[むかし]からの
\\	です。			
\\	天才	天才[てんさい]	てんさい	
\\	彼は笑いの天才だね。	彼[かれ]は 笑[わら]いの 天才[てんさい]だね。	かれ は わらい の てんさい だ ね	
\\	彼[かれ]は 笑[わら]いの
\\	だね。			
\\	本能	本能[ほんのう]	ほんのう	
\\	動物は本能のまま動くね。	動物[どうぶつ]は 本能[ほんのう]のまま 動[うご]くね。	どうぶつ は ほんのう の まま うごく ね	
\\	動物[どうぶつ]は
\\	のまま 動[うご]くね。			
\\	有能	有能[ゆうのう]	ゆうのう	
\\	彼女はとても有能な部下です。	彼女[かのじょ]はとても 有能[ゆうのう]な 部下[ぶか]です。	かのじょ は とても ゆうのう な ぶか です	
\\	彼女[かのじょ]はとても
\\	な 部下[ぶか]です。			
\\	知能	知能[ちのう]	ちのう	
\\	あの子の知能はとても高いそうです。	あの 子[こ]の 知能[ちのう]はとても 高[たか]いそうです。	あの こ の ちのう は とても たかい そう です	
\\	あの 子[こ]の
\\	はとても 高[たか]いそうです。			
\\	無能	無能[むのう]	むのう	
\\	彼は無能だ。	彼[かれ]は 無能[むのう]だ。	かれ は むのう だ	
\\	彼[かれ]は
\\	だ。			
\\	ムード	ムード	ムード	
\\	部屋を暗くしてムードを出してみたよ。	部屋[へや]を 暗[くら]くしてムードを 出[だ]してみたよ。	へや を くらく して むーど を だして みた よ	
\\	部屋[へや]を 暗[くら]くして
\\	を 出[だ]してみたよ。			
\\	能	能[のう]	のう	
\\	先日、初めて能を見に行きました。	先日[せんじつ]、 初[はじ]めて 能[のう]を 見[み]に 行[い]きました。	せんじつ はじめて のう を み に いきました	
\\	先日[せんじつ]、 初[はじ]めて
\\	を 見[み]に 行[い]きました。			
\\	能	能[のう]	のう	
\\	彼は勉強するしか能のない人間だね。	彼[かれ]は 勉強[べんきょう]するしか 能[のう]のない 人間[にんげん]だね。	かれ は べんきょう する しか のう の ない にんげん だ ね	
\\	彼[かれ]は 勉強[べんきょう]するしか
\\	のない 人間[にんげん]だね。			
\\	不安	不安[ふあん]	ふあん	
\\	明日、病院に検査に行くので少し不安です。	明日[あす]、 病院[びょういん]に 検査[けんさ]に 行[い]くので 少[すこ]し 不安[ふあん]です。	あす びょういん に けんさ に いく の で すこし ふあん です	
\\	明日[あす]、 病院[びょういん]に 検査[けんさ]に 行[い]くので 少[すこ]し
\\	です。			
\\	不満	不満[ふまん]	ふまん	
\\	私は彼のやり方には不満です。	私[わたし]は 彼[かれ]のやり 方[かた]には 不満[ふまん]です。	わたし は かれ の やりかた に は ふまん です	
\\	私[わたし]は 彼[かれ]のやり 方[かた]には
\\	です。			
\\	不足	不足[ふそく]	ふそく	
\\	今年はひどい水不足よ。	今年[ことし]はひどい 水[みず] 不足[ぶそく]よ。	ことし は ひどい みずぶそく よ	
\\	今年[ことし]はひどい 水[みず]
\\	よ。			
\\	どく	どく	どく	
\\	そこをどいてください。	そこをどいてください。	そこをどいてください。	
\\	そこを
\\	ください。			
\\	不可能	不可能[ふかのう]	ふかのう	
\\	レポートを1日で仕上げるのは不可能です。	レポートを1 日[にち]で 仕上[しあ]げるのは 不可能[ふかのう]です。	れぽーと を 
\\	にち で しあげる の は ふかのう です	
\\	レポートを1 日[にち]で 仕上[しあ]げるのは
\\	です。			
\\	不十分	不十分[ふじゅうぶん]	ふじゅうぶん	
\\	その程度の努力では不十分です。	その 程度[ていど]の 努力[どりょく]では 不十分[ふじゅうぶん]です。	その ていど の どりょく で は ふじゅうぶん です	
\\	その 程度[ていど]の 努力[どりょく]では
\\	です。			
\\	不明	不明[ふめい]	ふめい	
\\	その病気は原因不明と言われているんだよ。	その 病気[びょうき]は 原因[げんいん] 不明[ふめい]と 言[い]われているんだよ。	その びょうき は げんいん ふめい と いわれている ん だ よ	
\\	その 病気[びょうき]は 原因[げんいん]
\\	と 言[い]われているんだよ。			
\\	不良	不良[ふりょう]	ふりょう	
\\	不良品を返品したよ。	不良[ふりょう] 品[ひん]を 返品[へんぴん]したよ。	ふりょうひん を へんぴん した よ	
\\	品[ひん]を 返品[へんぴん]したよ。			
\\	不合格	不合格[ふごうかく]	ふごうかく	
\\	残念ながら試験は不合格でした。	残念[ざんねん]ながら 試験[しけん]は 不合格[ふごうかく]でした。	ざんねん ながら しけん は ふごうかく でした	
\\	残念[ざんねん]ながら 試験[しけん]は
\\	でした。			
\\	マーケット	マーケット	マーケット	
\\	彼はイタリアのマーケットを開拓しました。	彼[かれ]はイタリアのマーケットを 開拓[かいたく]しました。	かれ は いたりあ の まーけっと を かいたく しました	
\\	彼[かれ]はイタリアの
\\	を 開拓[かいたく]しました。			
\\	不通	不通[ふつう]	ふつう	
\\	今朝、停電で電車が不通になったよ。	今朝[けさ]、 停電[ていでん]で 電車[でんしゃ]が 不通[ふつう]になったよ。	けさ ていでん で でんしゃ が ふつう に なった よ	
\\	今朝[けさ]、 停電[ていでん]で 電車[でんしゃ]が
\\	になったよ。			
\\	便り	便[たよ]り	たより	
\\	月に一度母から便りが来ます。	月[つき]に 一度母[いちど はは]から 便[たよ]りが 来[き]ます。	つき に いちど はは から たより が きます	
\\	月[つき]に 一度母[いちど はは]から
\\	が 来[き]ます。			
\\	大便	大便[だいべん]	だいべん	
\\	病院で大便の検査をした。	病院[びょういん]で 大便[だいべん]の 検査[けんさ]をした。	びょういん で だいべん の けんさ を した	
\\	病院[びょういん]で
\\	の 検査[けんさ]をした。			
\\	番	番[ばん]	ばん	
\\	今日は私が皿を洗う番ですね。	今日[きょう]は 私[わたし]が 皿[さら]を 洗[あら]う 番[ばん]ですね。	きょう は わたし が さら を あらう ばん です ね	
\\	今日[きょう]は 私[わたし]が 皿[さら]を 洗[あら]う
\\	ですね。			
\\	長所	長所[ちょうしょ]	ちょうしょ	
\\	君の長所は明るいところだね。	君[きみ]の 長所[ちょうしょ]は 明[あか]るいところだね。	きみ の ちょうしょ は あかるい ところ だ ね	
\\	君[きみ]の
\\	は 明[あか]るいところだね。			
\\	ファッション	ファッション	ファッション	
\\	彼女はファッションの専門家だ。	彼女[かのじょ]はファッションの 専門家[せんもんか]だ。	かのじょ は ふぁっしょん の せんもんか だ	
\\	彼女[かのじょ]は
\\	の 専門家[せんもんか]だ。			
\\	短所	短所[たんしょ]	たんしょ	
\\	すぐにあきらめてしまうのが彼の短所です。	すぐにあきらめてしまうのが 彼[かれ]の 短所[たんしょ]です。	すぐ に あきらめて しまう の が かれ の たんしょ です	
\\	すぐにあきらめてしまうのが 彼[かれ]の
\\	です。			
\\	所々	所々[ところどころ]	ところどころ	
\\	この本はページが所々破れているね。	この 本[ほん]はページが 所々[ところどころ] 破[やぶ]れているね。	この ほん は ぺーじ が ところどころ やぶれて いる ね	
\\	この 本[ほん]はページが
\\	破[やぶ]れているね。			
\\	名	名[な]	な	
\\	彼は名の通った会社に就職しました。	彼[かれ]は 名[な]の 通[とお]った 会社[かいしゃ]に 就職[しゅうしょく]しました。	かれ は な の とおった かいしゃ に しゅうしょく しました	
\\	彼[かれ]は
\\	の 通[とお]った 会社[かいしゃ]に 就職[しゅうしょく]しました。			
\\	名人	名人[めいじん]	めいじん	
\\	彼は釣りの名人です。	彼[かれ]は 釣[つ]りの 名人[めいじん]です。	かれ は つり の めいじん です	
\\	彼[かれ]は 釣[つ]りの
\\	です。			
\\	地名	地名[ちめい]	ちめい	
\\	その地名は聞いたことがないなあ。	その 地名[ちめい]は 聞[き]いたことがないなあ。	その ちめい は きいた こと が ない なあ	
\\	その
\\	は 聞[き]いたことがないなあ。			
\\	名所	名所[めいしょ]	めいしょ	
\\	ここは桜の名所です。	ここは 桜[さくら]の 名所[めいしょ]です。	ここ は さくら の めいしょ です	
\\	ここは 桜[さくら]の
\\	です。			
\\	ブレーキ	ブレーキ	ブレーキ	
\\	赤信号でブレーキを踏んだの。	赤信号[あかしんごう]でブレーキを 踏[ふ]んだの。	あかしんごう で ぶれーき を ふんだ の	
\\	赤信号[あかしんごう]で
\\	を 踏[ふ]んだの。			
\\	町外れ	町外[まちはず]れ	まちはずれ	
\\	彼女は町外れの工場で働いているよ。	彼女[かのじょ]は 町外[まちはず]れの 工場[こうじょう]で 働[はたら]いているよ。	かのじょ は まちはずれ の こうじょう で はたらいて いる よ	
\\	彼女[かのじょ]は
\\	の 工場[こうじょう]で 働[はたら]いているよ。			
\\	町中	町中[まちなか]	まちなか	
\\	町中で彼女に出会ったよ。	町中[まちなか]で 彼女[かのじょ]に 出会[であ]ったよ。	まちなか で かのじょ に であった よ	
\\	で 彼女[かのじょ]に 出会[であ]ったよ。			
\\	村	村[むら]	むら	
\\	私は隣の村から来ました。	私[わたし]は 隣[となり]の 村[むら]から 来[き]ました。	わたし は となり の むら から きました	
\\	私[わたし]は 隣[となり]の
\\	から 来[き]ました。			
\\	都内	都内[とない]	とない	
\\	彼の職場は都内にあります。	彼[かれ]の 職場[しょくば]は 都内[とない]にあります。	かれ の しょくば は とない に あります	
\\	彼[かれ]の 職場[しょくば]は
\\	にあります。			
\\	都心	都心[としん]	としん	
\\	彼は都心に住んでいます。	彼[かれ]は 都心[としん]に 住[す]んでいます。	かれ は としん に すんで います	
\\	彼[かれ]は
\\	に 住[す]んでいます。			
\\	ミス	ミス	ミス	
\\	ミスは誰にでもあります。	ミスは 誰[だれ]にでもあります。	みす は だれ に で も あります	
\\	は 誰[だれ]にでもあります。			
\\	都	都[と]	と	
\\	彼女は都の職員です。	彼女[かのじょ]は 都[と]の 職員[しょくいん]です。	かのじょ は と の しょくいん です	
\\	彼女[かのじょ]は
\\	の 職員[しょくいん]です。			
\\	都会	都会[とかい]	とかい	
\\	彼は都会での生活を楽しんでいるの。	彼[かれ]は 都会[とかい]での 生活[せいかつ]を 楽[たの]しんでいるの。	かれ は とかい で の せいかつ を たのしんで いる の	
\\	彼[かれ]は
\\	での 生活[せいかつ]を 楽[たの]しんでいるの。			
\\	都	都[みやこ]	みやこ	
\\	ミラノはファッションの都です。	ミラノはファッションの 都[みやこ]です。	みらの は ふぁっしょん の みやこ です	
\\	ミラノはファッションの
\\	です。			
\\	様子	様子[ようす]	ようす	
\\	彼女の様子を見てきます。	彼女[かのじょ]の 様子[ようす]を 見[み]てきます。	かのじょ の ようす を みて きます	
\\	彼女[かのじょ]の
\\	を 見[み]てきます。			
\\	物価	物価[ぶっか]	ぶっか	
\\	この国の物価はここ10年でだいぶ上がったね。	この 国[くに]の 物価[ぶっか]はここ10 年[ねん]でだいぶ 上[あ]がったね。	この くに の ぶっか は ここ 
\\	ねん で だいぶ あがった ね	
\\	この 国[くに]の
\\	はここ10 年[ねん]でだいぶ 上[あ]がったね。			
\\	ヨット	ヨット	ヨット	
\\	ヨットでクルージングを楽しみましたわ。	ヨットでクルージングを 楽[たの]しみましたわ。	よっと で くるーじんぐ を たのしみました わ	
\\	でクルージングを 楽[たの]しみましたわ。			
\\	物体	物体[ぶったい]	ぶったい	
\\	謎の物体が空を飛んでいます。	謎[なぞ]の 物体[ぶったい]が 空[そら]を 飛[と]んでいます。	なぞ の ぶったい が そら を とんで います	
\\	謎[なぞ]の
\\	が 空[そら]を 飛[と]んでいます。			
\\	本物	本物[ほんもの]	ほんもの	
\\	これは本物のダイヤモンドです。	これは 本物[ほんもの]のダイヤモンドです。	これ は ほんもの の だいやもんど です	
\\	これは
\\	のダイヤモンドです。			
\\	物理	物理[ぶつり]	ぶつり	
\\	彼は物理を専攻している。	彼[かれ]は 物理[ぶつり]を 専攻[せんこう]している。	かれ は ぶつり を せんこう して いる	
\\	彼[かれ]は
\\	を 専攻[せんこう]している。			
\\	物事	物事[ものごと]	ものごと	
\\	彼はいつも物事を深く考えるの。	彼[かれ]はいつも 物事[ものごと]を 深[ふか]く 考[かんが]えるの。	かれ は いつも ものごと を ふかく かんがえる の	
\\	彼[かれ]はいつも
\\	を 深[ふか]く 考[かんが]えるの。			
\\	名物	名物[めいぶつ]	めいぶつ	
\\	この町の名物はぶどうです。	この 町[まち]の 名物[めいぶつ]はぶどうです。	この まち の めいぶつ は ぶどう です	
\\	この 町[まち]の
\\	はぶどうです。			
\\	物知り	物知[ものし]り	ものしり	
\\	あの老人はとても物知りだね。	あの 老人[ろうじん]はとても 物知[ものし]りだね。	あの ろうじん は とても ものしり だ ね	
\\	あの 老人[ろうじん]はとても
\\	だね。			
\\	なかなか	なかなか	なかなか	
\\	彼女は絵がなかなか上手です。	彼女[かのじょ]は 絵[え]がなかなか 上手[じょうず]です。	かのじょ は え が なかなか じょうず です	
\\	彼女[かのじょ]は 絵[え]が
\\	上手[じょうず]です。			
\\	物覚え	物覚[ものおぼ]え	ものおぼえ	
\\	彼女は物覚えがいい。	彼女[かのじょ]は 物覚[ものおぼ]えがいい。	かのじょ は ものおぼえ が いい	
\\	彼女[かのじょ]は
\\	がいい。			
\\	体重	体重[たいじゅう]	たいじゅう	
\\	私は毎朝体重を測っています。	私[わたし]は 毎朝[まいあさ] 体重[たいじゅう]を 測[はか]っています。	わたし は まいあさ たいじゅう を はかって います	
\\	私[わたし]は 毎朝[まいあさ]
\\	を 測[はか]っています。			
\\	手軽	手軽[てがる]	てがる	
\\	手軽に作れる料理を教えてください。	手軽[てがる]に 作[つく]れる 料理[りょうり]を 教[おし]えてください。	てがる に つくれる りょうり を おしえて ください	
\\	に 作[つく]れる 料理[りょうり]を 教[おし]えてください。			
\\	多量	多量[たりょう]	たりょう	
\\	その事故で多量のガス漏れがあったね。	その 事故[じこ]で 多量[たりょう]のガス 漏[も]れがあったね。	その じこ で たりょう の がすもれ が あった ね	
\\	その 事故[じこ]で
\\	のガス 漏[も]れがあったね。			
\\	量る	量[はか]る	はかる	
\\	はかりで小麦粉の分量を量ったよ。	はかりで 小麦粉[こむぎこ]の 分量[ぶんりょう]を 量[はか]ったよ。	はかり で こむぎこ の ぶんりょう を はかった よ	
\\	はかりで 小麦粉[こむぎこ]の 分量[ぶんりょう]を
\\	よ。			
\\	ロック	ロック	ロック	
\\	俺はロックを聞くのが好きだ。	俺[おれ]はロックを 聞[き]くのが 好[す]きだ。	おれ は ろっく を きく の が すき だ	
\\	俺[おれ]は
\\	を 聞[き]くのが 好[す]きだ。			
\\	量	量[りょう]	りょう	
\\	最近、食事の量を減らしています。	最近[さいきん]、 食事[しょくじ]の 量[りょう]を 減[へ]らしています。	さいきん しょくじ の りょう を へらして います	
\\	最近[さいきん]、 食事[しょくじ]の
\\	を 減[へ]らしています。			
\\	大量	大量[たいりょう]	たいりょう	
\\	昨日大量のゴミが出たの。	昨日[きのう] 大量[たいりょう]のゴミが 出[で]たの。	きのう たいりょう の ごみ が でた の	
\\	昨日[きのう]
\\	のゴミが 出[で]たの。			
\\	年配	年配[ねんぱい]	ねんぱい	
\\	年配の人に席を譲りました。	年配[ねんぱい]の 人[ひと]に 席[せき]を 譲[ゆず]りました。	ねんぱい の ひと に せき を ゆずりました	
\\	の 人[ひと]に 席[せき]を 譲[ゆず]りました。			
\\	送金	送金[そうきん]	そうきん	
\\	取引先への送金を済ませました。	取引先[とりひきさき]への 送金[そうきん]を 済[す]ませました。	とりひきさき へ の そうきん を すませました	
\\	取引先[とりひきさき]への
\\	を 済[す]ませました。			
\\	郵送	郵送[ゆうそう]	ゆうそう	
\\	書類を郵送してください。	書類[しょるい]を 郵送[ゆうそう]してください。	しょるい を ゆうそう して ください	
\\	書類[しょるい]を
\\	してください。			
\\	わざわざ	わざわざ	わざわざ	
\\	わざわざ来てくれてありがとう。	わざわざ 来[き]てくれてありがとう。	わざわざ きて くれて ありがとう	
\\	来[き]てくれてありがとう。			
\\	取り上げる	取[と]り 上[あ]げる	とりあげる	
\\	危ないので子供からライターを取り上げました。	危[あぶ]ないので 子供[こども]からライターを 取[と]り 上[あ]げました。	あぶない の で こども から らいたー を とりあげました	
\\	危[あぶ]ないので 子供[こども]からライターを
\\	取り入れる	取[と]り 入[い]れる	とりいれる	
\\	彼は新しいアイデアをすぐ取り入れます。	彼[かれ]は 新[あたら]しいアイデアをすぐ 取[と]り 入[い]れます。	かれ は あたらしい あいであ を すぐ とりいれます	
\\	彼[かれ]は 新[あたら]しいアイデアをすぐ
\\	取り決め	取[と]り 決[き]め	とりきめ	
\\	これは会社間の取り決めです。	これは 会社間[かいしゃかん]の 取[と]り 決[き]めです。	これ は かいしゃかん の とりきめ です	
\\	これは 会社間[かいしゃかん]の
\\	です。			
\\	取れる	取[と]れる	とれる	
\\	このナスは畑で取れたばかりです。	このナスは 畑[はたけ]で 取[と]れたばかりです。	この なす は はたけ で とれた ばかり です	
\\	このナスは 畑[はたけ]で
\\	ばかりです。			
\\	取り消す	取[と]り 消[け]す	とりけす	
\\	ホテルの予約を取り消したよ。	ホテルの 予約[よやく]を 取[と]り 消[け]したよ。	ほてる の よやく を とりけした よ	
\\	ホテルの 予約[よやく]を
\\	よ。			
\\	取消し	取消[とりけ]し	とりけし	
\\	彼は免許取消しの処分を受けたよ。	彼[かれ]は 免許[めんきょ] 取消[とりけ]しの 処分[しょぶん]を 受[う]けたよ。	かれ は めんきょ とりけし の しょぶん を うけた よ	
\\	彼[かれ]は 免許[めんきょ]
\\	の 処分[しょぶん]を 受[う]けたよ。			
\\	パイロット	パイロット	パイロット	
\\	彼はパイロットです。	彼[かれ]はパイロットです。	かれ は ぱいろっと です	
\\	彼[かれ]は
\\	です。			
\\	取り出す	取[と]り 出[だ]す	とりだす	
\\	彼はポケットから財布を取り出したの。	彼[かれ]はポケットから 財布[さいふ]を 取[と]り 出[だ]したの。	かれ は ぽけっと から さいふ を とりだした の	
\\	彼[かれ]はポケットから 財布[さいふ]を
\\	の。			
\\	届け	届[とど]け	とどけ	
\\	郵便局に引っ越しの届けを出したよ。	郵便局[ゆうびんきょく]に 引[ひ]っ 越[こ]しの 届[とど]けを 出[だ]したよ。	ゆうびんきょく に ひっこし の とどけ を だした よ	
\\	郵便局[ゆうびんきょく]に 引[ひ]っ 越[こ]しの
\\	を 出[だ]したよ。			
\\	待ち合わせ	待[ま]ち 合[あ]わせ	まちあわせ	
\\	明日の待ち合わせは11時です。	明日[あす]の 待[ま]ち 合[あ]わせは11 時[じ]です。	あす の まちあわせ は 
\\	じ です	
\\	明日[あす]の
\\	は11 時[じ]です。			
\\	待ち遠しい	待[ま]ち 遠[どお]しい	まちどおしい	
\\	入学式が待ち遠しいです。	入学式[にゅうがくしき]が 待[ま]ち 遠[どお]しいです。	にゅうがくしき が まちどおしい です	
\\	入学式[にゅうがくしき]が
\\	です。			
\\	待ち合わせる	待[ま]ち 合[あ]わせる	まちあわせる	
\\	彼と新宿で待ち合わせました。	彼[かれ]と 新宿[しんじゅく]で 待[ま]ち 合[あ]わせました。	かれ と しんじゅく で まちあわせました	
\\	彼[かれ]と 新宿[しんじゅく]で
\\	どうやら	どうやら	どうやら	
\\	明日はどうやら雨らしいよ。	明日[あした]はどうやら 雨[あめ]らしいよ。	あした は どうやら あめ らしい よ	
\\	明日[あした]は
\\	雨[あめ]らしいよ。			
\\	持ち物	持[も]ち 物[もの]	もちもの	
\\	持ち物には名前を書いてください。	持[も]ち 物[もの]には 名前[なまえ]を 書[か]いてください。	もちもの に は なまえ を かいて ください	
\\	には 名前[なまえ]を 書[か]いてください。			
\\	長持ち	長持[ながも]ち	ながもち	
\\	このコートは長持ちしている。	このコートは 長持[ながも]ちしている。	この こーと は ながもち して いる	
\\	このコートは
\\	している。			
\\	値打ち	値打[ねう]ち	ねうち	
\\	この絵はとても値打ちがあります。	この 絵[え]はとても 値打[ねう]ちがあります。	この え は とても ねうち が あります	
\\	この 絵[え]はとても
\\	があります。			
\\	投書	投書[とうしょ]	とうしょ	
\\	その事件について新聞に投書したんだ。	その 事件[じけん]について 新聞[しんぶん]に 投書[とうしょ]したんだ。	その じけん に ついて しんぶん に とうしょ した ん だ	
\\	その 事件[じけん]について 新聞[しんぶん]に
\\	したんだ。			
\\	役所	役所[やくしょ]	やくしょ	
\\	役所で住民票をもらってきた。	役所[やくしょ]で 住民票[じゅうみんひょう]をもらってきた。	やくしょ で じゅうみんひょう を もらって きた	
\\	で 住民票[じゅうみんひょう]をもらってきた。			
\\	ふと	ふと	ふと	
\\	ふと昔の事を思い出したんだ。	ふと 昔[むかし]の 事[こと]を 思[おも]い 出[だ]したんだ。	ふと むかし の こと を おもいだした ん だ	
\\	昔[むかし]の 事[こと]を 思[おも]い 出[だ]したんだ。			
\\	役人	役人[やくにん]	やくにん	
\\	叔父は役人として30年働きました。	叔父[おじ]は 役人[やくにん]として30 年働[ねん はたら]きました。	おじ は やくにん と して 
\\	ねん はたらきました	
\\	叔父[おじ]は
\\	として30 年働[ねん はたら]きました。			
\\	役目	役目[やくめ]	やくめ	
\\	私は無事に役目を終えたよ。	私[わたし]は 無事[ぶじ]に 役目[やくめ]を 終[お]えたよ。	わたし は ぶじ に やくめ を おえた よ	
\\	私[わたし]は 無事[ぶじ]に
\\	を 終[お]えたよ。			
\\	役	役[やく]	やく	
\\	彼女は弁護士の役を演じているんだ。	彼女[かのじょ]は 弁護士[べんごし]の 役[やく]を 演[えん]じているんだ。	かのじょ は べんごし の やく を えんじて いる ん だ	
\\	彼女[かのじょ]は 弁護士[べんごし]の
\\	を 演[えん]じているんだ。			
\\	生える	生[は]える	はえる	
\\	息子に新しい歯が生えました。	息子[むすこ]に 新[あたら]しい 歯[は]が 生[は]えました。	むすこ に あたらしい は が はえました	
\\	息子[むすこ]に 新[あたら]しい 歯[は]が
\\	長生き	長生[ながい]き	ながいき	
\\	この村の人たちは長生きです。	この 村[むら]の 人[ひと]たちは 長生[ながい]きです。	この むら の ひとたち は ながいき です	
\\	この 村[むら]の 人[ひと]たちは
\\	です。			
\\	ベテラン	ベテラン	ベテラン	
\\	彼はベテランの運転手です。	彼[かれ]はベテランの 運転手[うんてんしゅ]です。	かれ は べてらん の うんてんしゅ です	
\\	彼[かれ]は
\\	の 運転手[うんてんしゅ]です。			
\\	生やす	生[は]やす	はやす	
\\	彼はヒゲを生やしています。	彼[かれ]はヒゲを 生[は]やしています。	かれ は ひげ を はやして います	
\\	彼[かれ]はヒゲを
\\	生	生[なま]	なま	
\\	彼は生の魚が食べられません。	彼[かれ]は 生[なま]の 魚[さかな]が 食[た]べられません。	かれ は なま の さかな が たべられません	
\\	彼[かれ]は
\\	の 魚[さかな]が 食[た]べられません。			
\\	無生物	無生物[むせいぶつ]	むせいぶつ	
\\	岩は無生物に分類される。	岩[いわ]は 無生物[むせいぶつ]に 分類[ぶんるい]される。	いわ は むせいぶつ に ぶんるい される	
\\	岩[いわ]は
\\	に 分類[ぶんるい]される。			
\\	男性	男性[だんせい]	だんせい	
\\	彼は素敵な男性です。	彼[かれ]は 素敵[すてき]な 男性[だんせい]です。	かれ は すてき な だんせい です	
\\	彼[かれ]は 素敵[すてき]な
\\	です。			
\\	理性	理性[りせい]	りせい	
\\	人間は理性を持つ動物です。	人間[にんげん]は 理性[りせい]を 持[も]つ 動物[どうぶつ]です。	にんげん は りせい を もつ どうぶつ です	
\\	人間[にんげん]は
\\	を 持[も]つ 動物[どうぶつ]です。			
\\	ボーナス	ボーナス	ボーナス	
\\	今年はボーナスがたくさん出ました。	今年[ことし]はボーナスがたくさん 出[で]ました。	ことし は ぼーなす が たくさん でました 。	
\\	今年[ことし]は
\\	がたくさん 出[で]ました。			
\\	土産	土産[みやげ]	みやげ	
\\	土産に日本酒をもらった。	土産[みやげ]に 日本酒[にほんしゅ]をもらった。	みやげ に にほんしゅ を もらった	
\\	に 日本酒[にほんしゅ]をもらった。			
\\	徒歩	徒歩[とほ]	とほ	
\\	家から駅まで徒歩3分です。	家[いえ]から 駅[えき]まで 徒歩[とほ]3 分[ぷん]です。	いえ から えき まで とほ 
\\	ぷん です	
\\	家[いえ]から 駅[えき]まで
\\	分[ぷん]です。			
\\	学ぶ	学[まな]ぶ	まなぶ	
\\	私は哲学を学んでいます。	私[わたし]は 哲学[てつがく]を 学[まな]んでいます。	わたし は てつがく を まなんで います	
\\	私[わたし]は 哲学[てつがく]を
\\	文学	文学[ぶんがく]	ぶんがく	
\\	彼女は文学に興味を持っているのよ。	彼女[かのじょ]は 文学[ぶんがく]に 興味[きょうみ]を 持[も]っているのよ。	かのじょ は ぶんがく に きょうみ を もっている の よ 。	
\\	彼女[かのじょ]は
\\	に 興味[きょうみ]を 持[も]っているのよ。			
\\	体育	体育[たいいく]	たいいく	
\\	今日は体育の授業があります。	今日[きょう]は 体育[たいいく]の 授業[じゅぎょう]があります。	きょう は たいいく の じゅぎょう が あります	
\\	今日[きょう]は
\\	の 授業[じゅぎょう]があります。			
\\	そっくり	そっくり	そっくり	
\\	あの親子はそっくりだね。	あの 親子[おやこ]はそっくりだね。	あの おやこ は そっくり だ ね	
\\	あの 親子[おやこ]は
\\	だね。			
\\	体制	体制[たいせい]	たいせい	
\\	政治の体制を変える必要があります。	政治[せいじ]の 体制[たいせい]を 変[か]える 必要[ひつよう]があります。	せいじ の たいせい を かえる ひつよう が あります	
\\	政治[せいじ]の
\\	を 変[か]える 必要[ひつよう]があります。			
\\	強まる	強[つよ]まる	つよまる	
\\	雨はだんだん強まります。	雨[あめ]はだんだん 強[つよ]まります。	あめ は だんだん つよまります	
\\	雨[あめ]はだんだん
\\	強める	強[つよ]める	つよめる	
\\	火を強めてください。	火[ひ]を 強[つよ]めてください。	ひ を つよめて ください	
\\	火[ひ]を
\\	ください。			
\\	力強い	力強[ちからづよ]い	ちからづよい	
\\	彼は力強い演技をするね。	彼[かれ]は 力強[ちからづよ]い 演技[えんぎ]をするね。	かれ は ちからづよい えんぎ を する ね	
\\	彼[かれ]は
\\	演技[えんぎ]をするね。			
\\	強気	強気[つよき]	つよき	
\\	彼女は強気な女性ですね。	彼女[かのじょ]は 強気[つよき]な 女性[じょせい]ですね。	かのじょ は つよき な じょせい です ね	
\\	彼女[かのじょ]は
\\	な 女性[じょせい]ですね。			
\\	レンズ	レンズ	レンズ	
\\	眼鏡のレンズを替えました。	眼鏡[めがね]のレンズを 替[か]えました。	めがね の れんず を かえました	
\\	眼鏡[めがね]の
\\	を 替[か]えました。			
\\	弱まる	弱[よわ]まる	よわまる	
\\	夜になって風が弱まったね。	夜[よる]になって 風[かぜ]が 弱[よわ]まったね。	よる に なって かぜ が よわまった ね	
\\	夜[よる]になって 風[かぜ]が
\\	ね。			
\\	弱める	弱[よわ]める	よわめる	
\\	火を弱めてください。	火[ひ]を 弱[よわ]めてください。	ひ を よわめて ください	
\\	火[ひ]を
\\	ください。			
\\	弱る	弱[よわ]る	よわる	
\\	彼は病気で弱っているんだ。	彼[かれ]は 病気[びょうき]で 弱[よわ]っているんだ。	かれ は びょうき で よわって いる ん だ	
\\	彼[かれ]は 病気[びょうき]で
\\	んだ。			
\\	弱み	弱[よわ]み	よわみ	
\\	彼は私の弱みを握っているんだ。	彼[かれ]は 私[わたし]の 弱[よわ]みを 握[にぎ]っているんだ。	かれ は わたし の よわみ を にぎって いる ん だ	
\\	彼[かれ]は 私[わたし]の
\\	を 握[にぎ]っているんだ。			
\\	弱気	弱気[よわき]	よわき	
\\	彼は少し弱気になっています。	彼[かれ]は 少[すこ]し 弱気[よわき]になっています。	かれ は すこし よわき に なって います	
\\	彼[かれ]は 少[すこ]し
\\	になっています。			
\\	引き受ける	引[ひ]き 受[う]ける	ひきうける	
\\	新しい仕事を引き受けたよ。	新[あたら]しい 仕事[しごと]を 引[ひ]き 受[う]けたよ。	あたらしい しごと を ひきうけた よ	
\\	新[あたら]しい 仕事[しごと]を
\\	よ。			
\\	ペース	ペース	ペース	
\\	彼はとても早いペースで走っているね。	彼[かれ]はとても 早[はや]いペースで 走[はし]っているね。	かれ は とても はやい ぺーす で はしって いる ね	
\\	彼[かれ]はとても 早[はや]い
\\	で 走[はし]っているね。			
\\	引き上げる	引[ひ]き 上[あ]げる	ひきあげる	
\\	沈んだ船を引き上げたんだ。	沈[しず]んだ 船[ふね]を 引[ひ]き 上[あ]げたんだ。	しずんだ ふね を ひきあげた ん だ	
\\	沈[しず]んだ 船[ふね]を
\\	んだ。			
\\	引き出す	引[ひ]き 出[だ]す	ひきだす	
\\	先生が私の能力を引き出してくれました。	先生[せんせい]が 私[わたし]の 能力[のうりょく]を 引[ひ]き 出[だ]してくれました。	せんせい が わたし の のうりょく を ひきだしてくれました	
\\	先生[せんせい]が 私[わたし]の 能力[のうりょく]を
\\	引き取る	引[ひ]き 取[と]る	ひきとる	
\\	彼女は息子を引き取ったの。	彼女[かのじょ]は 息子[むすこ]を 引[ひ]き 取[と]ったの。	かのじょ は むすこ を ひきとった の	
\\	彼女[かのじょ]は 息子[むすこ]を
\\	の。			
\\	値引き	値引[ねび]き	ねびき	
\\	あと1000円値引きしましょう。	あと1000 円[えん] 値引[ねび]きしましょう。	あと 
\\	えん ねびき しましょう	
\\	あと1000 円[えん]
\\	しましょう。			
\\	長引く	長引[ながび]く	ながびく	
\\	会議が長引いているようです。	会議[かいぎ]が 長引[ながび]いているようです。	かいぎ が ながびいて いる よう です	
\\	会議[かいぎ]が
\\	いるようです。			
\\	マラソン	マラソン	マラソン	
\\	彼はマラソンに出場したよ。	彼[かれ]はマラソンに 出場[しゅつじょう]したよ。	かれ は まらそん に しゅつじょう した よ	
\\	彼[かれ]は
\\	に 出場[しゅつじょう]したよ。			
\\	引きずる	引[ひ]きずる	ひきずる	
\\	彼はまだ失恋を引きずっています。	彼[かれ]はまだ 失恋[しつれん]を 引[ひ]きずっています。	かれ は まだ しつれん を ひきずって います	
\\	彼[かれ]はまだ 失恋[しつれん]を
\\	引き分け	引[ひ]き 分[わ]け	ひきわけ	
\\	この勝負は引き分けです。	この 勝負[しょうぶ]は 引[ひ]き 分[わ]けです。	この しょうぶ は ひきわけ です	
\\	この 勝負[しょうぶ]は
\\	です。			
\\	引き止める	引[ひ]き 止[と]める	ひきとめる	
\\	帰ろうとする友達を引き止めたんだ。	帰[かえ]ろうとする 友達[ともだち]を 引[ひ]き 止[と]めたんだ。	かえろう と する ともだち を ひきとめた ん だ	
\\	帰[かえ]ろうとする 友達[ともだち]を
\\	んだ。			
\\	見習う	見習[みなら]う	みならう	
\\	彼を見習ってもっと勉強します。	彼[かれ]を 見習[みなら]ってもっと 勉強[べんきょう]します。	かれ を みならって もっと べんきょう します	
\\	彼[かれ]を
\\	もっと 勉強[べんきょう]します。			
\\	慣れ	慣[な]れ	なれ	
\\	仕事には慣れも必要です。	仕事[しごと]には 慣[な]れも 必要[ひつよう]です。	しごと に は なれ も ひつよう です	
\\	仕事[しごと]には
\\	も 必要[ひつよう]です。			
\\	よほど	よほど	よほど	
\\	今日の遠足はよほど楽しかったらしい。	今日[きょう]の 遠足[えんそく]はよほど 楽[たの]しかったらしい。	きょう の えんそく は よほど たのしかった らしい	
\\	今日[きょう]の 遠足[えんそく]は
\\	楽[たの]しかったらしい。			
\\	慣らす	慣[な]らす	ならす	
\\	水の温度に体を慣らしてから、潜ったほうがいいぞ。	水[みず]の 温度[おんど]に 体[からだ]を 慣[な]らしてから、 潜[もぐ]ったほうがいいぞ。	みず の おんど に からだ を ならして から もぐった ほう が いい ぞ	
\\	水[みず]の 温度[おんど]に 体[からだ]を
\\	から、 潜[もぐ]ったほうがいいぞ。			
\\	入試	入試[にゅうし]	にゅうし	
\\	明日は高校の入試です。	明日[あした]は 高校[こうこう]の 入試[にゅうし]です。	あした は こうこう の にゅうし です	
\\	明日[あした]は 高校[こうこう]の
\\	です。			
\\	試す	試[ため]す	ためす	
\\	彼はそのソフトウェアを試したの。	彼[かれ]はそのソフトウェアを 試[ため]したの。	かれ は その そふとうぇあ を ためした の	
\\	彼[かれ]はそのソフトウェアを
\\	の。			
\\	試し	試[ため]し	ためし	
\\	試しにこの曲を弾いてみてください。	試[ため]しにこの 曲[きょく]を 弾[ひ]いてみてください。	ためし に この きょく を ひいて みて ください	
\\	にこの 曲[きょく]を 弾[ひ]いてみてください。			
\\	体験	体験[たいけん]	たいけん	
\\	今日、珍しい体験をしました。	今日[きょう]、 珍[めずら]しい 体験[たいけん]をしました。	きょう めずらしい たいけん を しました	
\\	今日[きょう]、 珍[めずら]しい
\\	をしました。			
\\	物質	物質[ぶっしつ]	ぶっしつ	
\\	この製品は有害な化学物質を含んでいるわよ。	この 製品[せいひん]は 有害[ゆうがい]な 化学[かがく] 物質[ぶっしつ]を 含[ふく]んでいるわよ。	この せいひん は ゆうがい な かがくぶっしつ を ふくんで いる わ よ	
\\	この 製品[せいひん]は 有害[ゆうがい]な 化学[かがく]
\\	を 含[ふく]んでいるわよ。			
\\	ラグビー	ラグビー	ラグビー	
\\	父は若いころラグビーの選手でした。	父[ちち]は 若[わか]いころラグビーの 選手[せんしゅ]でした。	ちち は わかい ころ らぐびー の せんしゅ でした	
\\	父[ちち]は 若[わか]いころ
\\	の 選手[せんしゅ]でした。			
\\	品質	品質[ひんしつ]	ひんしつ	
\\	このメーカーの製品は高品質だわね。	このメーカーの 製品[せいひん]は 高[こう] 品質[ひんしつ]だわね。	この めーかー の せいひん は こうひんしつ だ わ ね	
\\	このメーカーの 製品[せいひん]は 高[こう]
\\	だわね。			
\\	本質	本質[ほんしつ]	ほんしつ	
\\	彼は仕事の本質をよく理解しているわね。	彼[かれ]は 仕事[しごと]の 本質[ほんしつ]をよく 理解[りかい]しているわね。	かれ は しごと の ほんしつ を よく りかい して いる わ ね	
\\	彼[かれ]は 仕事[しごと]の
\\	をよく 理解[りかい]しているわね。			
\\	問う	問[と]う	とう	
\\	応募者の年齢は問いません。	応募者[おうぼしゃ]の 年齢[ねんれい]は 問[と]いません。	おうぼしゃ の ねんれい は といません	
\\	応募者[おうぼしゃ]の 年齢[ねんれい]は
\\	問い合わせる	問[と]い 合[あ]わせる	といあわせる	
\\	保険会社に問い合わせます。	保険会社[ほけん がいしゃ]に 問[と]い 合[あ]わせます。	ほけん がいしゃ に といあわせます	
\\	保険会社[ほけん がいしゃ]に
\\	問い	問[と]い	とい	
\\	この問いに答えられますか。	この 問[と]いに 答[こた]えられますか。	この とい に こたえられます か	
\\	この
\\	に 答[こた]えられますか。			
\\	ラッシュ	ラッシュ	ラッシュ	
\\	私は毎朝ラッシュの電車に乗っています。	私[わたし]は 毎朝[まいあさ]ラッシュの 電車[でんしゃ]に 乗[の]っています。	わたし は まいあさ らっしゅ の でんしゃ に のって います	
\\	私[わたし]は 毎朝[まいあさ]
\\	の 電車[でんしゃ]に 乗[の]っています。			
\\	問屋	問屋[とんや]	とんや	
\\	この街には家具の問屋がたくさんあります。	この 街[まち]には 家具[かぐ]の 問屋[とんや]がたくさんあります。	この まち に は かぐ の とんや が たくさん あります	
\\	この 街[まち]には 家具[かぐ]の
\\	がたくさんあります。			
\\	話題	話題[わだい]	わだい	
\\	ここが話題のレストランです。	ここが 話題[わだい]のレストランです。	ここ が わだい の れすとらん です	
\\	ここが
\\	のレストランです。			
\\	題名	題名[だいめい]	だいめい	
\\	この本の題名は「坊ちゃん」です。	この 本[ほん]の 題名[だいめい]は
\\	坊[ぼっ]ちゃん」です。	この ほん の だいめい は ぼっちゃん です	
\\	この 本[ほん]の
\\	は
\\	坊[ぼっ]ちゃん」です。			
\\	題	題[だい]	だい	
\\	その曲の題を思い出せません。	その 曲[きょく]の 題[だい]を 思[おも]い 出[だ]せません。	その きょく の だい を おもいだせません	
\\	その 曲[きょく]の
\\	を 思[おも]い 出[だ]せません。			
\\	単語	単語[たんご]	たんご	
\\	日本語の単語をいくつ知っていますか。	日本語[にほんご]の 単語[たんご]をいくつ 知[し]っていますか。	にほんご の たんご を いくつ しって います か	
\\	日本語[にほんご]の
\\	をいくつ 知[し]っていますか。			
\\	ロープ	ロープ	ロープ	
\\	ロープを使って崖を下りたんだ。	ロープを 使[つか]って 崖[がけ]を 下[お]りたんだ。	ろーぷ を つかって がけ を おりた ん だ	
\\	を 使[つか]って 崖[がけ]を 下[お]りたんだ。			
\\	単に	単[たん]に	たんに	
\\	心配しないで、単に眠いだけです。	心配[しんぱい]しないで、 単[たん]に 眠[ねむ]いだけです。	しんぱい しない で たんに ねむい だけ です	
\\	心配[しんぱい]しないで、
\\	眠[ねむ]いだけです。			
\\	地点	地点[ちてん]	ちてん	
\\	もうすぐ目標の地点に到達します。	もうすぐ 目標[もくひょう]の 地点[ちてん]に 到達[とうたつ]します。	もうすぐ もくひょう の ちてん に とうたつ します	
\\	もうすぐ 目標[もくひょう]の
\\	に 到達[とうたつ]します。			
\\	満点	満点[まんてん]	まんてん	
\\	国語のテストは満点でした。	国語[こくご]のテストは 満点[まんてん]でした。	こくご の てすと は まんてん でした	
\\	国語[こくご]のテストは
\\	でした。			
\\	点	点[てん]	てん	
\\	その点は心配ありません。	その 点[てん]は 心配[しんぱい]ありません。	その てん は しんぱい ありません	
\\	その
\\	は 心配[しんぱい]ありません。			
\\	多数	多数[たすう]	たすう	
\\	その仕事に多数の応募があったよ。	その 仕事[しごと]に 多数[たすう]の 応募[おうぼ]があったよ。	その しごと に たすう の おうぼ が あった よ	
\\	その 仕事[しごと]に
\\	の 応募[おうぼ]があったよ。			
\\	ダム	ダム	ダム	
\\	今、ダムの水が少ないね。	今[いま]、ダムの 水[みず]が 少[すく]ないね。	いま だむ の みず が すくない ね	
\\	今[いま]、
\\	の 水[みず]が 少[すく]ないね。			
\\	複数	複数[ふくすう]	ふくすう	
\\	複数のレポートをひとつにまとめています。	複数[ふくすう]のレポートをひとつにまとめています。	ふくすう の れぽーと を ひとつ に まとめて います	
\\	のレポートをひとつにまとめています。			
\\	無数	無数[むすう]	むすう	
\\	宇宙には無数の星があります。	宇宙[うちゅう]には 無数[むすう]の 星[ほし]があります。	うちゅう に は むすう の ほし が あります	
\\	宇宙[うちゅう]には
\\	の 星[ほし]があります。			
\\	日数	日数[にっすう]	にっすう	
\\	今月は出勤日数が多いです。	今月[こんげつ]は 出勤[しゅっきん] 日数[にっすう]が 多[おお]いです。	こんげつ は しゅっきん にっすう が おおい です	
\\	今月[こんげつ]は 出勤[しゅっきん]
\\	が 多[おお]いです。			
\\	点数	点数[てんすう]	てんすう	
\\	今回はテストの点数が悪かった。	今回[こんかい]はテストの 点数[てんすう]が 悪[わる]かった。	こんかい は てすと の てんすう が わるかった	
\\	今回[こんかい]はテストの
\\	が 悪[わる]かった。			
\\	単数	単数[たんすう]	たんすう	
\\	は単数で
\\	は複数です。	
\\	[ぺん]は 単数[たんすう]で 
\\	[ぺんず]は 複数[ふくすう]です。	ぺんは たんすう で ぺんず は ふくすう です	
\\	[ぺん]は
\\	で 
\\	[ぺんず]は 複数[ふくすう]です。			
\\	手数	手数[てすう]	てすう	
\\	お手数ですがよろしくお願いします。	お 手数[てすう]ですがよろしくお 願[ねが]いします。	おてすう です が よろしく おねがい します	
\\	お
\\	ですがよろしくお 願[ねが]いします。			
\\	ダウン	ダウン	ダウン	
\\	サーバーがダウンしています。	サーバーがダウンしています。	さーばー が だうん して います	
\\	サーバーが
\\	しています。			
\\	人数	人数[にんずう]	にんずう	
\\	参加者の人数を教えてください。	参加者[さんかしゃ]の 人数[にんずう]を 教[おし]えてください。	さんかしゃ の にんずう を おしえて ください	
\\	参加者[さんかしゃ]の
\\	を 教[おし]えてください。			
\\	回り	回[まわ]り	まわり	
\\	先生の周りに集まってください。	先生[せんせい]の 周[まわ]りに 集[あつ]まってください。	せんせい の まわり に あつまって ください 。	
\\	先生[せんせい]の
\\	に 集[あつ]まってください。			
\\	回り道	回[まわ]り 道[みち]	まわりみち	
\\	今日は回り道して帰ろう。	今日[きょう]は 回[まわ]り 道[みち]して 帰[かえ]ろう。	きょう は まわりみち して かえろう	
\\	今日[きょう]は
\\	して 帰[かえ]ろう。			
\\	枚数	枚数[まいすう]	まいすう	
\\	コピーの枚数を数えてください。	コピーの 枚数[まいすう]を 数[かぞ]えてください。	こぴー の まいすう を かぞえて ください	
\\	コピーの
\\	を 数[かぞ]えてください。			
\\	負け	負[ま]け	まけ	
\\	この勝負はあなたの負けです。	この 勝負[しょうぶ]はあなたの 負[ま]けです。	この しょうぶ は あなた の まけ です	
\\	この 勝負[しょうぶ]はあなたの
\\	です。			
\\	トレーニング	トレーニング	トレーニング	
\\	彼は毎日トレーニングをするの。	彼[かれ]は 毎日[まいにち]トレーニングをするの。	かれ は まいにち とれーにんぐ を する の	
\\	彼[かれ]は 毎日[まいにち]
\\	をするの。			
\\	負担	負担[ふたん]	ふたん	
\\	交通費は個人負担です。	交通費[こうつうひ]は 個人[こじん] 負担[ふたん]です。	こうつうひ は こじん ふたん です	
\\	交通費[こうつうひ]は 個人[こじん]
\\	です。			
\\	分担	分担[ぶんたん]	ぶんたん	
\\	私たち夫婦は家事を分担しています。	私[わたし]たち 夫婦[ふうふ]は 家事[かじ]を 分担[ぶんたん]しています。	わたしたち ふうふ は かじ を ぶんたん して います	
\\	私[わたし]たち 夫婦[ふうふ]は 家事[かじ]を
\\	しています。			
\\	当初	当初[とうしょ]	とうしょ	
\\	当初の計画ではもっと早く終わるはずでした。	当初[とうしょ]の 計画[けいかく]ではもっと 早[はや]く 終[お]わるはずでした。	とうしょ の けいかく で は もっと はやく おわる はず でした	
\\	の 計画[けいかく]ではもっと 早[はや]く 終[お]わるはずでした。			
\\	当局	当局[とうきょく]	とうきょく	
\\	その事件については当局が調査しています。	その 事件[じけん]については 当局[とうきょく]が 調査[ちょうさ]しています。	その じけん に ついて は とうきょく が ちょうさ して います	
\\	その 事件[じけん]については
\\	が 調査[ちょうさ]しています。			
\\	当日	当日[とうじつ]	とうじつ	
\\	入場券は当日でも買えますよ。	入場券[にゅうじょうけん]は 当日[とうじつ]でも 買[か]えますよ。	にゅうじょうけん は とうじつ で も かえます よ	
\\	入場券[にゅうじょうけん]は
\\	でも 買[か]えますよ。			
\\	そっと	そっと	そっと	
\\	母親は娘の髪をそっとなでたの。	母親[ははおや]は 娘[むすめ]の 髪[かみ]をそっとなでたの。	ははおや は むすめ の かみ を そっと なでた の	
\\	母親[ははおや]は 娘[むすめ]の 髪[かみ]を
\\	なでたの。			
\\	手当て	手当[てあ]て	てあて	
\\	彼女は急いで怪我の手当てをしたよ。	彼女[かのじょ]は 急[いそ]いで 怪我[けが]の 手当[てあ]てをしたよ。	かのじょ は いそいで けが の てあて を した よ	
\\	彼女[かのじょ]は 急[いそ]いで 怪我[けが]の
\\	をしたよ。			
\\	当分	当分[とうぶん]	とうぶん	
\\	彼女は当分学校を休むそうです。	彼女[かのじょ]は 当分[とうぶん] 学校[がっこう]を 休[やす]むそうです。	かのじょ は とうぶん がっこう を やすむ そう です	
\\	彼女[かのじょ]は
\\	学校[がっこう]を 休[やす]むそうです。			
\\	当人	当人[とうにん]	とうにん	
\\	当人は意外に平気なようね。	当人[とうにん]は 意外[いがい]に 平気[へいき]なようね。	とうにん は いがい に へいき なよう ね 。	
\\	は 意外[いがい]に 平気[へいき]なようね。			
\\	当番	当番[とうばん]	とうばん	
\\	今日は私が掃除の当番です。	今日[きょう]は 私[わたし]が 掃除[そうじ]の 当番[とうばん]です。	きょう は わたし が そうじ の とうばん です	
\\	今日[きょう]は 私[わたし]が 掃除[そうじ]の
\\	です。			
\\	日当たり	日当[ひあ]たり	ひあたり	
\\	この部屋は日当たりがいい。	この 部屋[へや]は 日当[ひあ]たりがいい。	この へや は ひあたり が いい	
\\	この 部屋[へや]は
\\	がいい。			
\\	天然	天然[てんねん]	てんねん	
\\	ここは天然の温泉です。	ここは 天然[てんねん]の 温泉[おんせん]です。	ここ は てんねん の おんせん です	
\\	ここは
\\	の 温泉[おんせん]です。			
\\	マーク	マーク	マーク	
\\	このブランドのマークは可愛いですね。	このブランドのマークは 可愛[かわい]いですね。	この ぶらんど の まーく は かわいい です ね	
\\	このブランドの
\\	は 可愛[かわい]いですね。			
\\	文法	文法[ぶんぽう]	ぶんぽう	
\\	今日は英語の文法を勉強します。	今日[きょう]は 英語[えいご]の 文法[ぶんぽう]を 勉強[べんきょう]します。	きょう は えいご の ぶんぽう を べんきょう します	
\\	今日[きょう]は 英語[えいご]の
\\	を 勉強[べんきょう]します。			
\\	法学部	法学部[ほうがくぶ]	ほうがくぶ	
\\	彼女は法学部の学生です。	彼女[かのじょ]は 法学部[ほうがくぶ]の 学生[がくせい]です。	かのじょ は ほうがくぶ の がくせい です	
\\	彼女[かのじょ]は
\\	の 学生[がくせい]です。			
\\	法	法[ほう]	ほう	
\\	国民は法に従わなければならないよ。	国民[こくみん]は 法[ほう]に 従[したが]わなければならないよ。	こくみん は ほう に したがわなければ ならない よ	
\\	国民[こくみん]は
\\	に 従[したが]わなければならないよ。			
\\	法則	法則[ほうそく]	ほうそく	
\\	勝利するには法則があるそうだ。	勝利[しょうり]するには 法則[ほうそく]があるそうだ。	しょうり する に は ほうそく が ある そう だ	
\\	勝利[しょうり]するには
\\	があるそうだ。			
\\	不規則	不規則[ふきそく]	ふきそく	
\\	最近、不規則な生活をしている。	最近[さいきん]、 不規則[ふきそく]な 生活[せいかつ]をしている。	さいきん ふきそく な せいかつ を して いる	
\\	最近[さいきん]、
\\	な 生活[せいかつ]をしている。			
\\	たった	たった	たった	
\\	財布の中にたった1000円しかないよ。	財布[さいふ]の 中[なか]にたった 1000円[せんえん]しかないよ。	さいふ の なか に たった せんえん しか ない よ	
\\	財布[さいふ]の 中[なか]に
\\	1000円[せんえん]しかないよ。			
\\	経る	経[へ]る	へる	
\\	彼は新聞記者を経て作家になりました。	彼[かれ]は 新聞記者[しんぶん きしゃ]を 経[へ]て 作家[さっか]になりました。	かれ は しんぶん きしゃ を へて さっか に なりました	
\\	彼[かれ]は 新聞記者[しんぶん きしゃ]を
\\	作家[さっか]になりました。			
\\	有利	有利[ゆうり]	ゆうり	
\\	資格があると就職に有利です。	資格[しかく]があると 就職[しゅうしょく]に 有利[ゆうり]です。	しかく が ある と しゅうしょく に ゆうり です	
\\	資格[しかく]があると 就職[しゅうしょく]に
\\	です。			
\\	不利	不利[ふり]	ふり	
\\	彼は今、不利な立場にいます。	彼[かれ]は 今[いま]、 不利[ふり]な 立場[たちば]にいます。	かれ は いま ふり な たちば に います	
\\	彼[かれ]は 今[いま]、
\\	な 立場[たちば]にいます。			
\\	利口	利口[りこう]	りこう	
\\	あの犬はとても利口ですね。	あの 犬[いぬ]はとても 利口[りこう]ですね。	あの いぬ は とても りこう です ね	
\\	あの 犬[いぬ]はとても
\\	ですね。			
\\	利子	利子[りし]	りし	
\\	借金に利子をつけて返したの。	借金[しゃっきん]に 利子[りし]をつけて 返[かえ]したの。	しゃっきん に りし を つけて かえした の	
\\	借金[しゃっきん]に
\\	をつけて 返[かえ]したの。			
\\	たまたま	たまたま	たまたま	
\\	道でたまたま友達に会った。	道[みち]でたまたま 友達[ともだち]に 会[あ]った。	みち で たまたま ともだち に あった	
\\	道[みち]で
\\	友達[ともだち]に 会[あ]った。			
\\	左利き	左利[ひだりき]き	ひだりきき	
\\	私の息子は左利きです。	私[わたし]の 息子[むすこ]は 左利[ひだりき]きです。	わたし の むすこ は ひだりきき です	
\\	私[わたし]の 息子[むすこ]は
\\	です。			
\\	利益	利益[りえき]	りえき	
\\	先月の利益は200万円でした。	先月[せんげつ]の 利益[りえき]は200 万円[まんえん]でした。	せんげつ の りえき は 
\\	まんえん でした	
\\	先月[せんげつ]の
\\	は200 万円[まんえん]でした。			
\\	有益	有益[ゆうえき]	ゆうえき	
\\	昨日の話し合いは有益でした。	昨日[きのう]の 話[はな]し 合[あ]いは 有益[ゆうえき]でした。	きのう の はなしあい は ゆうえき でした	
\\	昨日[きのう]の 話[はな]し 合[あ]いは
\\	でした。			
\\	買収	買収[ばいしゅう]	ばいしゅう	
\\	彼は買収されたらしいわ。	彼[かれ]は 買収[ばいしゅう]されたらしいわ。	かれ は ばいしゅう された らしい わ	
\\	彼[かれ]は
\\	されたらしいわ。			
\\	年収	年収[ねんしゅう]	ねんしゅう	
\\	税金の額は年収によって変わります。	税金[ぜいきん]の 額[がく]は 年収[ねんしゅう]によって 変[か]わります。	ぜいきん の がく は ねんしゅう に よって かわります	
\\	税金[ぜいきん]の 額[がく]は
\\	によって 変[か]わります。			
\\	木造	木造[もくぞう]	もくぞう	
\\	隣に木造の家が建ったね。	隣[となり]に 木造[もくぞう]の 家[いえ]が 建[た]ったね。	となり に もくぞう の いえ が たった ね	
\\	隣[となり]に
\\	の 家[いえ]が 建[た]ったね。			
\\	テンポ	テンポ	テンポ	
\\	この曲はテンポが速いですね。	この 曲[きょく]はテンポが 速[はや]いですね。	この きょく は てんぽ が はやい です ね	
\\	この 曲[きょく]は
\\	が 速[はや]いですね。			
\\	必然	必然[ひつぜん]	ひつぜん	
\\	私と彼が出会ったのは必然だったの。	私[わたし]と 彼[かれ]が 出会[であ]ったのは 必然[ひつぜん]だったの。	わたし と かれ が であった の は ひつぜん だった の	
\\	私[わたし]と 彼[かれ]が 出会[であ]ったのは
\\	だったの。			
\\	要する	要[よう]する	ようする	
\\	このビルは完成までに2年を要した。	このビルは 完成[かんせい]までに2 年[ねん]を 要[よう]した。	この びる は かんせい まで に 
\\	ねん を ようした	
\\	このビルは 完成[かんせい]までに2 年[ねん]を
\\	不要	不要[ふよう]	ふよう	
\\	不要になったパソコンを処分したんだ。	不要[ふよう]になったパソコンを 処分[しょぶん]したんだ。	ふよう に なった ぱそこん を しょぶん した ん だ	
\\	になったパソコンを 処分[しょぶん]したんだ。			
\\	要するに	要[よう]するに	ようするに	
\\	要するに時機を待つべきだ。	要[よう]するに 時機[じき]を 待[ま]つべきだ。	ようするに じき を まつべき だ	
\\	時機[じき]を 待[ま]つべきだ。			
\\	不必要	不必要[ふひつよう]	ふひつよう	
\\	不必要なファイルは削除してください。	不必要[ふひつよう]なファイルは 削除[さくじょ]してください。	ふひつよう な ふぁいる は さくじょ して ください	
\\	なファイルは 削除[さくじょ]してください。			
\\	プラン	プラン	プラン	
\\	彼女と旅行のプランを考えました。	彼女[かのじょ]と 旅行[りょこう]のプランを 考[かんが]えました。	かのじょ と りょこう の ぷらん を かんがえました	
\\	彼女[かのじょ]と 旅行[りょこう]の
\\	を 考[かんが]えました。			
\\	要点	要点[ようてん]	ようてん	
\\	話の要点だけ教えてください。	話[はなし]の 要点[ようてん]だけ 教[おし]えてください。	はなし の ようてん だけ おしえて ください	
\\	話[はなし]の
\\	だけ 教[おし]えてください。			
\\	求める	求[もと]める	もとめる	
\\	子供は親の愛を求めます。	子供[こども]は 親[おや]の 愛[あい]を 求[もと]めます。	こども は おや の あい を もとめます	
\\	子供[こども]は 親[おや]の 愛[あい]を
\\	要請	要請[ようせい]	ようせい	
\\	その国の政府は各国に支援を要請したのよ。	その 国[くに]の 政府[せいふ]は 各国[かっこく]に 支援[しえん]を 要請[ようせい]したのよ。	その くに の せいふ は かっこく に しえん を ようせい した の よ	
\\	その 国[くに]の 政府[せいふ]は 各国[かっこく]に 支援[しえん]を
\\	したのよ。			
\\	額	額[ひたい]	ひたい	
\\	額に汗をかいたよ。	額[ひたい]に 汗[あせ]をかいたよ。	ひたい に あせ を かいた よ	
\\	に 汗[あせ]をかいたよ。			
\\	計る	計[はか]る	はかる	
\\	100メートル走のタイムを計ったんだ。	100メートル 走[そう]のタイムを 計[はか]ったんだ。	
\\	めーとるそう の たいむ を はかった ん だ	
\\	100メートル 走[そう]のタイムを
\\	んだ。			
\\	ほっと	ほっと	ほっと	
\\	家に着いてほっとしたよ。	家[いえ]に 着[つ]いてほっとしたよ。	いえ に ついて ほっと した よ	
\\	家[いえ]に 着[つ]いて
\\	したよ。			
\\	体温計	体温計[たいおんけい]	たいおんけい	
\\	体温計が壊れてしまった。	体温計[たいおんけい]が 壊[こわ]れてしまった。	たいおんけい が こわれて しまった	
\\	が 壊[こわ]れてしまった。			
\\	日差し	日差[ひざ]し	ひざし	
\\	今日は日差しが強いですね。	今日[きょう]は 日差[ひざ]しが 強[つよ]いですね。	きょう は ひざし が つよい です ね	
\\	今日[きょう]は
\\	が 強[つよ]いですね。			
\\	物差し	物差[ものさ]し	ものさし	
\\	30センチの物差しをください。	30センチの 物差[ものさ]しをください。	
\\	せんち の ものさし を ください	
\\	30センチの
\\	をください。			
\\	役割	役割[やくわり]	やくわり	
\\	みんなで役割を分担しましょう。	みんなで 役割[やくわり]を 分担[ぶんたん]しましょう。	みんな で やくわり を ぶんたん しましょう	
\\	みんなで
\\	を 分担[ぶんたん]しましょう。			
\\	割合	割合[わりあい]	わりあい	
\\	二つの薬品を1対3の割合で混ぜたんだ。	二[ふた]つの 薬品[やくひん]を1 対3[たい 
\\	の 割合[わりあい]で 混[ま]ぜたんだ。	ふたつ の やくひん を 
\\	たい 
\\	の わりあい で まぜた ん だ	
\\	二[ふた]つの 薬品[やくひん]を1 対3[たい 
\\	の
\\	で 混[ま]ぜたんだ。			
\\	なるほど	なるほど	なるほど	
\\	なるほど、よく分かりました。	なるほど、よく 分[わ]かりました。	なるほど よく わかりました	
\\	、よく 分[わ]かりました。			
\\	割り当て	割[わ]り 当[あ]て	わりあて	
\\	チケットの割り当ては1人10枚です。	チケットの 割[わ]り 当[あ]ては 1人10枚[ひとり じゅうまい]です。	ちけっと の わりあて は ひとり じゅうまい です	
\\	チケットの
\\	は 1人10枚[ひとり じゅうまい]です。			
\\	割り当てる	割[わ]り 当[あ]てる	わりあてる	
\\	全員に作業が割り当てられました。	全員[ぜんいん]に 作業[さぎょう]が 割[わ]り 当[あ]てられました。	ぜんいん に さぎょう が わりあてられました	
\\	全員[ぜんいん]に 作業[さぎょう]が
\\	割り引く	割[わ]り 引[び]く	わりびく	
\\	定価から2000円割り引きますよ。	定価[ていか]から2 000円[せんえん] 割[わ]り 引[び]きますよ。	ていか から 
\\	せんえん わりびきます よ	
\\	定価[ていか]から2 000円[せんえん]
\\	よ。			
\\	割に	割[わり]に	わりに	
\\	この映画は割に面白いわ。	この 映画[えいが]は 割[わり]に 面白[おもしろ]いわ。	この えいが は わりに おもしろい わ	
\\	この 映画[えいが]は
\\	面白[おもしろ]いわ。			
\\	割り	割[わ]り	わり	
\\	3日に1度の割りで彼からメールが来るの。	3日[みっか]に1 度[ど]の 割[わ]りで 彼[かれ]からメールが 来[く]るの。	みっか に 
\\	ど の わり で かれ から めーる が くる の	
\\	3日[みっか]に1 度[ど]の
\\	で 彼[かれ]からメールが 来[く]るの。			
\\	割り引き	割[わ]り 引[び]き	わりびき	
\\	今日は全品10
\\	割り引きです。	今日[きょう]は 全品10[ぜんぴん 
\\	割[わ]り 引[び]きです。	きょう は ぜんぴん 
\\	わりびき です	
\\	今日[きょう]は 全品10[ぜんぴん 
\\	です。			
\\	バイト	バイト	バイト	
\\	今日は6時からバイトです。	今日[きょう]は6 時[じ]からバイトです。	きょう は 
\\	じ から ばいと です	
\\	今日[きょう]は6 時[じ]から
\\	です。			
\\	割合に	割合[わりあい]に	わりあいに	
\\	今回のテストは割合に簡単でした。	今回[こんかい]のテストは 割合[わりあい]に 簡単[かんたん]でした。	こんかい の てすと は わりあいに かんたん でした	
\\	今回[こんかい]のテストは
\\	簡単[かんたん]でした。			
\\	残り	残[のこ]り	のこり	
\\	仕事の残りは家でします。	仕事[しごと]の 残[のこ]りは 家[いえ]でします。	しごと の のこり は いえ で します	
\\	仕事[しごと]の
\\	は 家[いえ]でします。			
\\	残らず	残[のこ]らず	のこらず	
\\	ゴミを残らず拾ったよ。	ゴミを 残[のこ]らず 拾[ひろ]ったよ。	ごみ を のこらず ひろった よ	
\\	ゴミを
\\	拾[ひろ]ったよ。			
\\	払い	払[はら]い	はらい	
\\	飲み屋の払いがたまっているんだ。	飲[の]み 屋[や]の 払[はら]いがたまっているんだ。	のみや の はらい が たまって いる ん だ	
\\	飲[の]み 屋[や]の
\\	がたまっているんだ。			
\\	引き返す	引[ひ]き 返[かえ]す	ひきかえす	
\\	雨が強かったので引き返したよ。	雨[あめ]が 強[つよ]かったので 引[ひ]き 返[かえ]したよ。	あめ が つよかった の で ひきかえした よ	
\\	雨[あめ]が 強[つよ]かったので
\\	よ。			
\\	ぶつける	ぶつける	ぶつける	
\\	車を壁にぶつけてしまいました。	車[くるま]を 壁[かべ]にぶつけてしまいました。	くるま を かべ に ぶつけて しまいました	
\\	車[くるま]を 壁[かべ]に
\\	取り返す	取[と]り 返[かえ]す	とりかえす	
\\	彼はチャンピオンのタイトルを取り返したね。	彼[かれ]はチャンピオンのタイトルを 取[と]り 返[かえ]したね。	かれ は ちゃんぴおん の たいとる を とりかえした ね	
\\	彼[かれ]はチャンピオンのタイトルを
\\	ね。			
\\	申し出る	申[もう]し 出[で]る	もうしでる	
\\	彼はプロジェクトへの参加を申し出たよ。	彼[かれ]はプロジェクトへの 参加[さんか]を 申[もう]し 出[で]たよ。	かれ は ぷろじぇくと へ の さんか を もうしでた よ	
\\	彼[かれ]はプロジェクトへの 参加[さんか]を
\\	よ。			
\\	申し上げる	申[もう]し 上[あ]げる	もうしあげる	
\\	結果を申し上げます。	結果[けっか]を 申[もう]し 上[あ]げます。	けっか を もうしあげます	
\\	結果[けっか]を
\\	申す	申[もう]す	もうす	
\\	私は鈴木と申します。	私[わたくし]は 鈴木[すずき]と 申[もう]します。	わたくし は すずき と もうします	
\\	私[わたくし]は 鈴木[すずき]と
\\	持ち込む	持[も]ち 込[こ]む	もちこむ	
\\	機内に荷物を持ち込んだの。	機内[きない]に 荷物[にもつ]を 持[も]ち 込[こ]んだの。	きない に にもつ を もちこんだ の	
\\	機内[きない]に 荷物[にもつ]を
\\	の。			
\\	タイヤ	タイヤ	タイヤ	
\\	車のタイヤを換えた。	車[くるま]のタイヤを 換[か]えた。	くるま の たいや を かえた	
\\	車[くるま]の
\\	を 換[か]えた。			
\\	見込み	見込[みこ]み	みこみ	
\\	3月に大学を卒業の見込みです。	
\\	月[がつ]に 大学[だいがく]を 卒業[そつぎょう]の 見込[みこ]みです。	
\\	がつ に だいがく を そつぎょう の みこみ です	
\\	月[がつ]に 大学[だいがく]を 卒業[そつぎょう]の
\\	です。			
\\	申し込み	申[もう]し 込[こ]み	もうしこみ	
\\	今日、スポーツジムの申し込みをしました。	今日[きょう]、スポーツジムの 申[もう]し 込[こ]みをしました。	きょう すぽーつじむ の もうしこみ を しました	
\\	今日[きょう]、スポーツジムの
\\	をしました。			
\\	払い込む	払[はら]い 込[こ]む	はらいこむ	
\\	授業料を学校に払い込みました。	授業料[じゅぎょうりょう]を 学校[がっこう]に 払[はら]い 込[こ]みました。	じゅぎょうりょう を がっこう に はらいこみました	
\\	授業料[じゅぎょうりょう]を 学校[がっこう]に
\\	飲み込む	飲[の]み 込[こ]む	のみこむ	
\\	彼は薬を一気に飲み込んだ。	彼[かれ]は 薬[くすり]を 一気[いっき]に 飲[の]み 込[こ]んだ。	かれ は くすり を いっきに のみこんだ	
\\	彼[かれ]は 薬[くすり]を 一気[いっき]に
\\	割り込む	割[わ]り 込[こ]む	わりこむ	
\\	車が前に割り込んできた。	車[くるま]が 前[まえ]に 割[わ]り 込[こ]んできた。	くるま が まえ に わりこんで きた	
\\	車[くるま]が 前[まえ]に
\\	引っ込む	引[ひ]っ 込[こ]む	ひっこむ	
\\	ダイエットをしてお腹が引っ込みました。	ダイエットをしてお 腹[なか]が 引[ひ]っ 込[こ]みました。	だいえっと を して おなか が ひっこみました	
\\	ダイエットをしてお 腹[なか]が
\\	マイク	マイク	マイク	
\\	彼女はマイクを持って話し始めたの。	彼女[かのじょ]はマイクを 持[も]って 話[はな]し 始[はじ]めたの。	かのじょ は まいく を もって はなしはじめた の	
\\	彼女[かのじょ]は
\\	を 持[も]って 話[はな]し 始[はじ]めたの。			
\\	人込み	人込[ひとご]み	ひとごみ	
\\	私は人込みが好きではありません。	私[わたし]は 人込[ひとご]みが 好[す]きではありません。	わたし は ひとごみ が すき で は ありません	
\\	私[わたし]は
\\	が 好[す]きではありません。			
\\	長期	長期[ちょうき]	ちょうき	
\\	今回は長期の滞在です。	今回[こんかい]は 長期[ちょうき]の 滞在[たいざい]です。	こんかい は ちょうき の たいざい です	
\\	今回[こんかい]は
\\	の 滞在[たいざい]です。			
\\	短期	短期[たんき]	たんき	
\\	明日から2週間、短期のアルバイトをします。	明日[あした]から2 週間[しゅうかん]、 短期[たんき]のアルバイトをします。	あした から 
\\	しゅうかん たんき の あるばいと を します	
\\	明日[あした]から2 週間[しゅうかん]、
\\	のアルバイトをします。			
\\	短期大学	短期大学[たんきだいがく]	たんきだいがく	
\\	彼女は短期大学で日本文学を勉強しました。	彼女[かのじょ]は 短期大学[たんきだいがく]で 日本文学[にほん ぶんがく]を 勉強[べんきょう]しました。	かのじょ は たんきだいがく で にほん ぶんがく を べんきょう しました	
\\	彼女[かのじょ]は
\\	で 日本文学[にほん ぶんがく]を 勉強[べんきょう]しました。			
\\	有限	有限[ゆうげん]	ゆうげん	
\\	宇宙は有限だと思いますか。	宇宙[うちゅう]は 有限[ゆうげん]だと 思[おも]いますか。	うちゅう は ゆうげん だ と おもいます か	
\\	宇宙[うちゅう]は
\\	だと 思[おも]いますか。			
\\	ゆとり	ゆとり	ゆとり	
\\	最近、生活にゆとりがでてきました。	最近[さいきん]、 生活[せいかつ]にゆとりがでてきました。	さいきん せいかつ に ゆとり が でて きました	
\\	最近[さいきん]、 生活[せいかつ]に
\\	がでてきました。			
\\	無限	無限[むげん]	むげん	
\\	資源は無限ではありません。	資源[しげん]は 無限[むげん]ではありません。	しげん は むげん で は ありません	
\\	資源[しげん]は
\\	ではありません。			
\\	無制限	無制限[むせいげん]	むせいげん	
\\	このサイトでは音楽を無制限でダウンロードできる。	このサイトでは 音楽[おんがく]を 無制限[むせいげん]でダウンロードできる。	この さいと で は おんがく を むせいげん で だうんろーど できる	
\\	このサイトでは 音楽[おんがく]を
\\	でダウンロードできる。			
\\	値切る	値切[ねぎ]る	ねぎる	
\\	彼は値切るのが上手です。	彼[かれ]は 値切[ねぎ]るのが 上手[じょうず]です。	かれ は ねぎる の が じょうず です	
\\	彼[かれ]は
\\	のが 上手[じょうず]です。			
\\	前売り券	前売[まえう]り 券[けん]	まえうりけん	
\\	コンサートの前売り券を手に入れたんだ。	コンサートの 前売[まえう]り 券[けん]を 手[て]に 入[い]れたんだ。	こんさーと の まえうりけん を て に いれた ん だ	
\\	コンサートの
\\	を 手[て]に 入[い]れたんだ。			
\\	年代	年代[ねんだい]	ねんだい	
\\	私と彼は同じ年代です。	私[わたし]と 彼[かれ]は 同[おな]じ 年代[ねんだい]です。	わたし と かれ は おなじ ねんだい です	
\\	私[わたし]と 彼[かれ]は 同[おな]じ
\\	です。			
\\	ぼんやり	ぼんやり	ぼんやり	
\\	彼は遠くをぼんやり見ていたの。	彼[かれ]は 遠[とお]くをぼんやり 見[み]ていたの。	かれ は とおく を ぼんやり みて いた の	
\\	彼[かれ]は 遠[とお]くを
\\	見[み]ていたの。			
\\	代金	代金[だいきん]	だいきん	
\\	ここで代金をお支払いください。	ここで 代金[だいきん]をお 支払[しはら]いください。	ここ で だいきん を お しはらい ください	
\\	ここで
\\	をお 支払[しはら]いください。			
\\	目指す	目指[めざ]す	めざす	
\\	私は料理人を目指しています。	私[わたし]は 料理人[りょうりにん]を 目指[めざ]しています。	わたし は りょうりにん を めざして います	
\\	私[わたし]は 料理人[りょうりにん]を
\\	指差す	指差[ゆびさ]す	ゆびさす	
\\	みんなが彼の指差す方を見たんだ。	みんなが 彼[かれ]の 指差[ゆびさ]す 方[ほう]を 見[み]たんだ。	みんな が かれ の ゆびさす ほう を みたんだ	
\\	みんなが 彼[かれ]の
\\	方[ほう]を 見[み]たんだ。			
\\	人差し指	人差[ひとさ]し 指[ゆび]	ひとさしゆび	
\\	彼女は人差し指を怪我したの。	彼女[かのじょ]は 人差[ひとさ]し 指[ゆび]を 怪我[けが]したの。	かのじょ は ひとさしゆび を けが した の 。	
\\	彼女[かのじょ]は
\\	を 怪我[けが]したの。			
\\	中指	中指[なかゆび]	なかゆび	
\\	中指をドアに挟んでしまった。	中指[なかゆび]をドアに 挟[はさ]んでしまった。	なかゆび を どあ に はさんで しまった	
\\	をドアに 挟[はさ]んでしまった。			
\\	定年	定年[ていねん]	ていねん	
\\	彼は来年定年を迎える。	彼[かれ]は 来年[らいねん] 定年[ていねん]を 迎[むか]える。	かれ は らいねん ていねん を むかえる	
\\	彼[かれ]は 来年[らいねん]
\\	を 迎[むか]える。			
\\	たまらない	たまらない	たまらない	
\\	頭が痛くてたまらない。	頭[あたま]が 痛[いた]くてたまらない。	あたま が いたくて たまらない	
\\	頭[あたま]が 痛[いた]くて
\\	不安定	不安定[ふあんてい]	ふあんてい	
\\	最近、体調が少し不安定です。	最近[さいきん]、 体調[たいちょう]が 少[すこ]し 不安定[ふあんてい]です。	さいきん たいちょう が すこし ふあんてい です	
\\	最近[さいきん]、 体調[たいちょう]が 少[すこ]し
\\	です。			
\\	定員	定員[ていいん]	ていいん	
\\	降りてください、定員オーバーです。	降[お]りてください、 定員[ていいん]オーバーです。	おりて ください ていいん おーばー です	
\\	降[お]りてください、
\\	オーバーです。			
\\	定期	定期[ていき]	ていき	
\\	定期演奏会は年に4回あります。	定期[ていき] 演奏会[えんそうかい]は 年[ねん]に4 回[かい]あります。	ていきえんそうかい は ねん に 
\\	かい あります	
\\	演奏会[えんそうかい]は 年[ねん]に4 回[かい]あります。			
\\	定価	定価[ていか]	ていか	
\\	この本の定価は525円です。	この 本[ほん]の 定価[ていか]は525 円[えん]です。	この ほん の ていか は 
\\	えん です	
\\	この 本[ほん]の
\\	は525 円[えん]です。			
\\	未定	未定[みてい]	みてい	
\\	この件の担当者は未定です。	この 件[けん]の 担当者[たんとうしゃ]は 未定[みてい]です。	この けん の たんとうしゃ は みてい です	
\\	この 件[けん]の 担当者[たんとうしゃ]は
\\	です。			
\\	モーター	モーター	モーター	
\\	車のモーターを修理したぜ!	車[くるま]のモーターを 修理[しゅうり]したぜ!	くるま の もーたー を しゅうり した ぜ	
\\	車[くるま]の
\\	を 修理[しゅうり]したぜ!			
\\	定食	定食[ていしょく]	ていしょく	
\\	昼の定食は3種類あります。	昼[ひる]の 定食[ていしょく]は3 種類[しゅるい]あります。	ひる の ていしょく は 
\\	しゅるい あります	
\\	昼[ひる]の
\\	は3 種類[しゅるい]あります。			
\\	定休日	定休日[ていきゅうび]	ていきゅうび	
\\	この店は水曜が定休日です。	この 店[みせ]は 水曜[すいよう]が 定休日[ていきゅうび]です。	この みせ は すいよう が ていきゅうび です	
\\	この 店[みせ]は 水曜[すいよう]が
\\	です。			
\\	予算	予算[よさん]	よさん	
\\	車の費用、予算オーバーだ。	車[くるま]の 費用[ひよう]、 予算[よさん]オーバーだ。	くるま の ひよう、 よさん おーばー だ	
\\	車[くるま]の 費用[ひよう]、
\\	オーバーだ。			
\\	予報	予報[よほう]	よほう	
\\	予報では明日は雨ですね。	予報[よほう]では 明日[あした]は 雨[あめ]ですね。	よほう で は あした は あめ です ね	
\\	では 明日[あした]は 雨[あめ]ですね。			
\\	予言	予言[よげん]	よげん	
\\	彼の予言は当たったことがないね。	彼[かれ]の 予言[よげん]は 当[あ]たったことがないね。	かれ の よげん は あたった こと が ない ね	
\\	彼[かれ]の
\\	は 当[あ]たったことがないね。			
\\	まして	まして	まして	
\\	他人でも悲しいのだから、まして本人はどれほどでしょう。	他人[たにん]でも 悲[かな]しいのだから、まして 本人[ほんにん]はどれほどでしょう。	たにん で も かなしい の だ から まして ほんにん は どれほど でしょう	
\\	他人[たにん]でも 悲[かな]しいのだから、
\\	本人[ほんにん]はどれほどでしょう。			
\\	束	束[たば]	たば	
\\	これは一束で300円です。	これは 一[ひと] 束[たば]で300 円[えん]です。	これ は ひとたば で 
\\	えん です	
\\	これは 一[ひと]
\\	で300 円[えん]です。			
\\	花束	花束[はなたば]	はなたば	
\\	卒業式に花束をもらいました。	卒業式[そつぎょうしき]に 花束[はなたば]をもらいました。	そつぎょうしき に はなたば を もらいました	
\\	卒業式[そつぎょうしき]に
\\	をもらいました。			
\\	変更	変更[へんこう]	へんこう	
\\	予定が変更になりました。	予定[よてい]が 変更[へんこう]になりました。	よてい が へんこう に なりました	
\\	予定[よてい]が
\\	になりました。			
\\	夜更かし	夜更[よふ]かし	よふかし	
\\	最近の子供たちは夜更かしです。	最近[さいきん]の 子供[こども]たちは 夜更[よふ]かしです。	さいきん の こども たち は よふかし です	
\\	最近[さいきん]の 子供[こども]たちは
\\	です。			
\\	増す	増[ま]す	ます	
\\	大雨で川の水かさが増しているな。	大雨[おおあめ]で 川[かわ]の 水[みず]かさが 増[ま]しているな。	おおあめ で かわ の みずかさ が まして いる な	
\\	大雨[おおあめ]で 川[かわ]の 水[みず]かさが
\\	な。			
\\	ピストル	ピストル	ピストル	
\\	犯人はピストルを持っているわ。	犯人[はんにん]はピストルを 持[も]っているわ。	はんにん は ぴすとる を もって いる わ	
\\	犯人[はんにん]は
\\	を 持[も]っているわ。			
\\	増大	増大[ぞうだい]	ぞうだい	
\\	生産コスト増大のため、値上げします。	生産[せいさん]コスト 増大[ぞうだい]のため、 値上[ねあ]げします。	せいさん こすと ぞうだい の ため ねあげ します	
\\	生産[せいさん]コスト
\\	のため、 値上[ねあ]げします。			
\\	増減	増減[ぞうげん]	ぞうげん	
\\	この数年、体重は増減していません。	この 数年[すうねん]、 体重[たいじゅう]は 増減[ぞうげん]していません。	この すうねん たいじゅう は ぞうげん して いません	
\\	この 数年[すうねん]、 体重[たいじゅう]は
\\	していません。			
\\	乗り出す	乗[の]り 出[だ]す	のりだす	
\\	船が長い航海に乗り出したの。	船[ふね]が 長[なが]い 航海[こうかい]に 乗[の]り 出[だ]したの。	ふね が ながい こうかい に のりだした の	
\\	船[ふね]が 長[なが]い 航海[こうかい]に
\\	の。			
\\	乗り込む	乗[の]り 込[こ]む	のりこむ	
\\	あの駅で学生がたくさん乗り込んだね。	あの 駅[えき]で 学生[がくせい]がたくさん 乗[の]り 込[こ]んだね。	あの えき で がくせい が たくさん のりこんだ ね	
\\	あの 駅[えき]で 学生[がくせい]がたくさん
\\	ね。			
\\	乗り降り	乗[の]り 降[お]り	のりおり	
\\	この駅でたくさんの人が乗り降りしますね。	この 駅[えき]でたくさんの 人[ひと]が 乗[の]り 降[お]りしますね。	この えき で たくさん の ひと が のりおり します ね	
\\	この 駅[えき]でたくさんの 人[ひと]が
\\	しますね。			
\\	着ける	着[つ]ける	つける	
\\	玄関に車を着けます。	玄関[げんかん]に 車[くるま]を 着[つ]けます。	げんかん に くるま を つけます	
\\	玄関[げんかん]に 車[くるま]を
\\	たっぷり	たっぷり	たっぷり	
\\	たっぷりとマッサージしてもらいました。	たっぷりとマッサージしてもらいました。	たっぷり と まっさーじ して もらいました	
\\	とマッサージしてもらいました。			
\\	水着	水着[みずぎ]	みずぎ	
\\	水着に着替えました。	水着[みずぎ]に 着替[きが]えました。	みずぎ に きがえました	
\\	に 着替[きが]えました。			
\\	役立つ	役立[やくだ]つ	やくだつ	
\\	学校で勉強したことが役立った。	学校[がっこう]で 勉強[べんきょう]したことが 役立[やくだ]った。	がっこう で べんきょう した こと が やくだった	
\\	学校[がっこう]で 勉強[べんきょう]したことが
\\	立ち上がる	立[た]ち 上[あ]がる	たちあがる	
\\	彼は急に立ち上がったの。	彼[かれ]は 急[きゅう]に 立[た]ち 上[あ]がったの。	かれ は きゅう に たちあがった の	
\\	彼[かれ]は 急[きゅう]に
\\	の。			
\\	中立	中立[ちゅうりつ]	ちゅうりつ	
\\	私は中立の立場を取っています。	私[わたし]は 中立[ちゅうりつ]の 立場[たちば]を 取[と]っています。	わたし は ちゅうりつ の たちば を とって います	
\\	私[わたし]は
\\	の 立場[たちば]を 取[と]っています。			
\\	都立	都立[とりつ]	とりつ	
\\	ここは都立の病院です。	ここは 都立[とりつ]の 病院[びょういん]です。	ここ は とりつ の びょういん です	
\\	ここは
\\	の 病院[びょういん]です。			
\\	つくづく	つくづく	つくづく	
\\	来てよかったとつくづく思います。	来[き]てよかったとつくづく 思[おも]います。	きて よかったと つくづく おもいます	
\\	来[き]てよかったと
\\	思[おも]います。			
\\	立て込む	立[た]て 込[こ]む	たてこむ	
\\	この辺は住宅が立て込んで います。	この 辺[へん]は 住宅[じゅうたく]が 立[た]て 込[こ]んでいます。	このへん は じゅうたく が たてこんで います	
\\	この 辺[へん]は 住宅[じゅうたく]が
\\	います。			
\\	夕立	夕立[ゆうだち]	ゆうだち	
\\	帰宅中、夕立にあったの。	帰宅中[きたくちゅう]、 夕立[ゆうだち]にあったの。	きたくちゅう ゆうだち に あった の	
\\	帰宅中[きたくちゅう]、
\\	にあったの。			
\\	立ち止まる	立[た]ち 止[ど]まる	たちどまる	
\\	人々は立ち止まって上を見上げたの。	人々[ひとびと]は 立[た]ち 止[ど]まって 上[うえ]を 見上[みあ]げたの。	ひとびと は たちどまって うえ を みあげた の	
\\	人々[ひとびと]は
\\	上[うえ]を 見上[みあ]げたの。			
\\	着席	着席[ちゃくせき]	ちゃくせき	
\\	みなさん、着席してください。	みなさん、 着席[ちゃくせき]してください。	みなさん ちゃくせき して ください	
\\	みなさん、
\\	してください。			
\\	次々に	次々[つぎつぎ]に	つぎつぎに	
\\	走者が次々にゴールしました。	走者[そうしゃ]が 次々[つぎつぎ]にゴールしました。	そうしゃ が つぎつぎに ごーる しました	
\\	走者[そうしゃ]が
\\	ゴールしました。			
\\	どうせ	どうせ	どうせ	
\\	どうせ間に合わないならゆっくり行こう。	どうせ 間[ま]に 合[あ]わないならゆっくり 行[い]こう。	どうせ まにあわない なら ゆっくり いこう	
\\	間[ま]に 合[あ]わないならゆっくり 行[い]こう。			
\\	次ぐ	次[つ]ぐ	つぐ	
\\	彼はトップに次ぐ好成績でした。	彼[かれ]はトップに 次[つ]ぐ 好成績[こうせいせき]でした。	かれ は とっぷ に つぐ こうせいせき でした	
\\	彼[かれ]はトップに
\\	好成績[こうせいせき]でした。			
\\	取り次ぐ	取[と]り 次[つ]ぐ	とりつぐ	
\\	電話があったら取り次いでください。	電話[でんわ]があったら 取[と]り 次[つ]いでください。	でんわ が あったら とりついで ください	
\\	電話[でんわ]があったら
\\	ください。			
\\	目次	目次[もくじ]	もくじ	
\\	読みたい章を目次で探しました。	読[よ]みたい 章[しょう]を 目次[もくじ]で 探[さが]しました。	よみたい しょう を もくじ で さがしました	
\\	読[よ]みたい 章[しょう]を
\\	で 探[さが]しました。			
\\	不運	不運[ふうん]	ふうん	
\\	彼に不運な出来事が起こったの。	彼[かれ]に 不運[ふうん]な 出来事[できごと]が 起[お]こったの。	かれ に ふうん な できごと が おこった の	
\\	彼[かれ]に
\\	な 出来事[できごと]が 起[お]こったの。			
\\	動向	動向[どうこう]	どうこう	
\\	今、経済の動向は読みにくいわ。	今[いま]、 経済[けいざい]の 動向[どうこう]は 読[よ]みにくいわ。	いま けいざい の どうこう は よみ にくい わ	
\\	今[いま]、 経済[けいざい]の
\\	は 読[よ]みにくいわ。			
\\	動作	動作[どうさ]	どうさ	
\\	彼は動作が機敏です。	彼[かれ]は 動作[どうさ]が 機敏[きびん]です。	かれ は どうさ が きびん です	
\\	彼[かれ]は
\\	が 機敏[きびん]です。			
\\	リズム	リズム	リズム	
\\	最近彼は生活のリズムが乱れています。	最近彼[さいきん かれ]は 生活[せいかつ]のリズムが 乱[みだ]れています。	さいきん かれ は せいかつ の りずむ が みだれて います	
\\	最近彼[さいきん かれ]は 生活[せいかつ]の
\\	が 乱[みだ]れています。			
\\	動力	動力[どうりょく]	どうりょく	
\\	この車の動力は電気です。	この 車[くるま]の 動力[どうりょく]は 電気[でんき]です。	この くるま の どうりょく は でんき です	
\\	この 車[くるま]の
\\	は 電気[でんき]です。			
\\	働き	働[はたら]き	はたらき	
\\	部下が素晴らしい働きをしたな。	部下[ぶか]が 素晴[すば]らしい 働[はたら]きをしたな。	ぶか が すばらしい はたらき を した な	
\\	部下[ぶか]が 素晴[すば]らしい
\\	をしたな。			
\\	早朝	早朝[そうちょう]	そうちょう	
\\	私は早朝のジョギングを日課にしています。	私[わたし]は 早朝[そうちょう]のジョギングを 日課[にっか]にしています。	わたし は そうちょう の じょぎんぐ を にっか に して います	
\\	私[わたし]は
\\	のジョギングを 日課[にっか]にしています。			
\\	早める	早[はや]める	はやめる	
\\	集合時間を30分早めました。	集合時間[しゅうごうじかん]を30 分[ぷん] 早[はや]めました。	しゅうごうじかん を 
\\	ぷん はやめました	
\\	集合時間[しゅうごうじかん]を30 分[ぷん]
\\	早まる	早[はや]まる	はやまる	
\\	早まらないでよく考えましょう。	早[はや]まらないでよく 考[かんが]えましょう。	はやまらない で よく かんがえましょう	
\\	よく 考[かんが]えましょう。			
\\	ハンドル	ハンドル	ハンドル	
\\	この車は左ハンドルです。	この 車[くるま]は 左[ひだり]ハンドルです。	この くるま は ひだり はんどる です	
\\	この 車[くるま]は 左[ひだり]
\\	です。			
\\	速度	速度[そくど]	そくど	
\\	新幹線の速度はどれくらいですか。	新幹線[しんかんせん]の 速度[そくど]はどれくらいですか。	しんかんせん の そくど は どれ くらい です か	
\\	新幹線[しんかんせん]の
\\	はどれくらいですか。			
\\	速達便	速達便[そくたつびん]	そくたつびん	
\\	速達便なら明日の午前中に届きます。	速達便[そくたつびん]なら 明日[あした]の 午前中[ごぜんちゅう]に 届[とど]きます。	そくたつびん なら あした の ごぜんちゅう に とどきます	
\\	なら 明日[あした]の 午前中[ごぜんちゅう]に 届[とど]きます。			
\\	乗り遅れる	乗[の]り 遅[おく]れる	のりおくれる	
\\	寝坊して新幹線に乗り遅れたよ。	寝坊[ねぼう]して 新幹線[しんかんせん]に 乗[の]り 遅[おく]れたよ。	ねぼう して しんかんせん に のりおくれた よ	
\\	寝坊[ねぼう]して 新幹線[しんかんせん]に
\\	よ。			
\\	始め	始[はじ]め	はじめ	
\\	私たちの旅は始めはよかったんだ。	私[わたし]たちの 旅[たび]は 始[はじ]めはよかったんだ。	わたしたち の たび は はじめ は よかった ん だ	
\\	私[わたし]たちの 旅[たび]は
\\	はよかったんだ。			
\\	始まり	始[はじ]まり	はじまり	
\\	いよいよ劇の始まりですね。	いよいよ 劇[げき]の 始[はじ]まりですね。	いよいよ げき の はじまり です ね	
\\	いよいよ 劇[げき]の
\\	ですね。			
\\	プリント	プリント	プリント	
\\	デジカメで撮った写真をプリントしたんだ。	デジカメで 撮[と]った 写真[しゃしん]をプリントしたんだ。	でじかめ で とった しゃしん を ぷりんと した ん だ	
\\	デジカメで 撮[と]った 写真[しゃしん]を
\\	したんだ。			
\\	年始	年始[ねんし]	ねんし	
\\	部下の方が年始の挨拶に見えましたよ。	部下[ぶか]の 方[かた]が 年始[ねんし]の 挨拶[あいさつ]に 見[み]えましたよ。	ぶか の かた が ねんし の あいさつ に みえました よ	
\\	部下[ぶか]の 方[かた]が
\\	の 挨拶[あいさつ]に 見[み]えましたよ。			
\\	不在	不在[ふざい]	ふざい	
\\	妻の不在中、夫は毎日外食したんだ。	妻[つま]の 不在[ふざい] 中[ちゅう]、 夫[おっと]は 毎日外食[まいにち がいしょく]したんだ。	つま の ふざいちゅう おっと は まいにち がいしょく した ん だ	
\\	妻[つま]の
\\	中[ちゅう]、 夫[おっと]は 毎日外食[まいにち がいしょく]したんだ。			
\\	実る	実[みの]る	みのる	
\\	やっと努力が実りました。	やっと 努力[どりょく]が 実[みの]りました。	やっと どりょく が みのりました	
\\	やっと 努力[どりょく]が
\\	実	実[み]	み	
\\	庭の木が赤い実をつけた。	庭[にわ]の 木[き]が 赤[あか]い 実[み]をつけた。	にわ の き が あかい み を つけた	
\\	庭[にわ]の 木[き]が 赤[あか]い
\\	をつけた。			
\\	通り過ぎる	通[とお]り 過[す]ぎる	とおりすぎる	
\\	うっかり目的地を通り過ぎた。	うっかり 目的地[もくてきち]を 通[とお]り 過[す]ぎた。	うっかり もくてきち を とおりすぎた	
\\	うっかり 目的地[もくてきち]を
\\	発売	発売[はつばい]	はつばい	
\\	新しい車が発売された。	新[あたら]しい 車[くるま]が 発売[はつばい]された。	あたらしい くるま が はつばい された	
\\	新[あたら]しい 車[くるま]が
\\	された。			
\\	まぶしい	まぶしい	まぶしい	
\\	夏の日差しがまぶしかったよ。	夏[なつ]の 日差[ひざ]しがまぶしかったよ。	なつ の ひざし が まぶしかった よ	
\\	夏[なつ]の 日差[ひざ]しが
\\	よ。			
\\	発つ	発[た]つ	たつ	
\\	彼は明日メキシコへ発ちます。	彼[かれ]は 明日[あした]メキシコへ 発[た]ちます。	かれ は あした めきしこ へ たちます	
\\	彼[かれ]は 明日[あした]メキシコへ
\\	発行	発行[はっこう]	はっこう	
\\	その雑誌は年に4回発行されているんだ。	その 雑誌[ざっし]は 年[ねん]に4 回[かい] 発行[はっこう]されているんだ。	その ざっし は ねん に 
\\	かい はっこう されて いる ん だ	
\\	その 雑誌[ざっし]は 年[ねん]に4 回[かい]
\\	されているんだ。			
\\	発達	発達[はったつ]	はったつ	
\\	通信技術の発達は目覚ましいな。	通信技術[つうしん ぎじゅつ]の 発達[はったつ]は 目覚[めざ]ましいな。	つうしん ぎじゅつ の はったつ は めざましい な	
\\	通信技術[つうしん ぎじゅつ]の
\\	は 目覚[めざ]ましいな。			
\\	発明	発明[はつめい]	はつめい	
\\	彼は偉大な発明王です。	彼[かれ]は 偉大[いだい]な 発明[はつめい] 王[おう]です。	かれ は いだい な はつめいおう です	
\\	彼[かれ]は 偉大[いだい]な
\\	王[おう]です。			
\\	発電	発電[はつでん]	はつでん	
\\	ここでは太陽エネルギーを使って発電しています。	ここでは 太陽[たいよう]エネルギーを 使[つか]って 発電[はつでん]しています。	ここ で は たいよう えねるぎー を つかって はつでん して います	
\\	ここでは 太陽[たいよう]エネルギーを 使[つか]って
\\	しています。			
\\	よそ	よそ	よそ	
\\	夕食はよそでごちそうになりました。	夕食[ゆうしょく]はよそでごちそうになりました。	ゆうしょく は よそ で ごちそう に なりました	
\\	夕食[ゆうしょく]は
\\	でごちそうになりました。			
\\	発電所	発電所[はつでんしょ]	はつでんしょ	
\\	すぐそこに発電所があります。	すぐそこに 発電所[はつでんしょ]があります。	すぐ そこ に はつでんしょ が あります	
\\	すぐそこに
\\	があります。			
\\	発熱	発熱[はつねつ]	はつねつ	
\\	娘が突然、発熱したんだ。	娘[むすめ]が 突然[とつぜん]、 発熱[はつねつ]したんだ。	むすめ が とつぜん はつねつ した ん だ	
\\	娘[むすめ]が 突然[とつぜん]、
\\	したんだ。			
\\	表情	表情[ひょうじょう]	ひょうじょう	
\\	彼はとても表情が豊かですね。	彼[かれ]はとても 表情[ひょうじょう]が 豊[ゆた]かですね。	かれ は とても ひょうじょう が ゆたか です ね	
\\	彼[かれ]はとても
\\	が 豊[ゆた]かですね。			
\\	用紙	用紙[ようし]	ようし	
\\	この用紙に名前を書いてください。	この 用紙[ようし]に 名前[なまえ]を 書[か]いてください。	この ようし に なまえ を かいて ください	
\\	この
\\	に 名前[なまえ]を 書[か]いてください。			
\\	表紙	表紙[ひょうし]	ひょうし	
\\	その人気アイドルが今月の表紙だよ。	その 人気[にんき]アイドルが 今月[こんげつ]の 表紙[ひょうし]だよ。	その にんき あいどる が こんげつ の ひょうし だ よ	
\\	その 人気[にんき]アイドルが 今月[こんげつ]の
\\	だよ。			
\\	つぐ	つぐ	つぐ	
\\	お酒をおつぎしましょう。	お 酒[さけ]をおつぎしましょう。	おさけ を お つぎ しましょう	
\\	お 酒[さけ]をお
\\	白紙	白紙[はくし]	はくし	
\\	テストを白紙で出したの。	テストを 白紙[はくし]で 出[だ]したの。	てすと を はくし で だした の	
\\	テストを
\\	で 出[だ]したの。			
\\	音	音[ね]	ね	
\\	秋は虫の音が心地良いです。	秋[あき]は 虫[むし]の 音[ね]が 心地良[ここちい]いです。	あき は むし の ね が ここちいい です	
\\	秋[あき]は 虫[むし]の
\\	が 心地良[ここちい]いです。			
\\	発音	発音[はつおん]	はつおん	
\\	この単語を発音してください。	この 単語[たんご]を 発音[はつおん]してください。	この たんご を はつおん して ください	
\\	この 単語[たんご]を
\\	してください。			
\\	物音	物音[ものおと]	ものおと	
\\	物音がしたので見に行った。	物音[ものおと]がしたので 見[み]に 行[い]った。	ものおと が した の で み に いった	
\\	がしたので 見[み]に 行[い]った。			
\\	楽しみ	楽[たの]しみ	たのしみ	
\\	旅行は父の老後の楽しみです。	旅行[りょこう]は 父[ちち]の 老後[ろうご]の 楽[たの]しみです。	りょこう は ちち の ろうご の たのしみ です	
\\	旅行[りょこう]は 父[ちち]の 老後[ろうご]の
\\	です。			
\\	ディスコ	ディスコ	ディスコ	
\\	昔はよくディスコに行ったな。	昔[むかし]はよくディスコに 行[い]ったな。	むかし は よく でぃすこ に いった な	
\\	昔[むかし]はよく
\\	に 行[い]ったな。			
\\	楽	楽[らく]	らく	
\\	彼には楽な仕事が与えられたよ。	彼[かれ]には 楽[らく]な 仕事[しごと]が 与[あた]えられたよ。	かれ に は らく な しごと が あたえられた よ	
\\	彼[かれ]には
\\	な 仕事[しごと]が 与[あた]えられたよ。			
\\	薬品	薬品[やくひん]	やくひん	
\\	彼女は薬品を戸棚から出したんだ。	彼女[かのじょ]は 薬品[やくひん]を 戸棚[とだな]から 出[だ]したんだ。	かのじょ は やくひん を とだな から だした ん だ	
\\	彼女[かのじょ]は
\\	を 戸棚[とだな]から 出[だ]したんだ。			
\\	薬局	薬局[やっきょく]	やっきょく	
\\	薬局で目薬を買いました。	薬局[やっきょく]で 目薬[めぐすり]を 買[か]いました。	やっきょく で めぐすり を かいました	
\\	で 目薬[めぐすり]を 買[か]いました。			
\\	目薬	目薬[めぐすり]	めぐすり	
\\	目が疲れたので目薬をさしたよ。	目[め]が 疲[つか]れたので 目薬[めぐすり]をさしたよ。	め が つかれた の で めぐすり を さした よ	
\\	目[め]が 疲[つか]れたので
\\	をさしたよ。			
\\	欲	欲[よく]	よく	
\\	あまり欲を出しちゃだめだよ。	あまり 欲[よく]を 出[だ]しちゃだめだよ。	あまり よく を だし ちゃ だめ だ よ	
\\	あまり
\\	を 出[だ]しちゃだめだよ。			
\\	欲求	欲求[よっきゅう]	よっきゅう	
\\	時には自分の欲求を抑えることも必要です。	時[とき]には 自分[じぶん]の 欲求[よっきゅう]を 抑[おさ]えることも 必要[ひつよう]です。	ときには じぶん の よっきゅう を おさえる こと も ひつよう です	
\\	時[とき]には 自分[じぶん]の
\\	を 抑[おさ]えることも 必要[ひつよう]です。			
\\	ぴったり	ぴったり	ぴったり	
\\	この服はあなたにぴったりですね。	この 服[ふく]はあなたにぴったりですね。	この ふく は あなた に ぴったり です ね	
\\	この 服[ふく]はあなたに
\\	ですね。			
\\	表面	表面[ひょうめん]	ひょうめん	
\\	月の表面にはクレーターがたくさんあるね。	月[つき]の 表面[ひょうめん]にはクレーターがたくさんあるね。	つき の ひょうめん に は くれーたー が たくさん ある ね	
\\	月[つき]の
\\	にはクレーターがたくさんあるね。			
\\	場面	場面[ばめん]	ばめん	
\\	ここがいちばん面白い場面です。	ここがいちばん 面白[おもしろ]い 場面[ばめん]です。	ここ が いちばん おもしろい ばめん です	
\\	ここがいちばん 面白[おもしろ]い
\\	です。			
\\	面する	面[めん]する	めんする	
\\	私の家は川に面しています。	私[わたし]の 家[いえ]は 川[かわ]に 面[めん]しています。	わたし の いえ は かわ に めんして います	
\\	私[わたし]の 家[いえ]は 川[かわ]に
\\	方面	方面[ほうめん]	ほうめん	
\\	沖縄方面にお出かけの方は台風にご注意ください。	沖縄[おきなわ] 方面[ほうめん]にお 出[で]かけの 方[かた]は 台風[たいふう]にご 注意[ちゅうい]ください。	おきなわ ほうめん に おでかけ の かた は たいふう に ごちゅうい ください	
\\	沖縄[おきなわ]
\\	にお 出[で]かけの 方[かた]は 台風[たいふう]にご 注意[ちゅうい]ください。			
\\	半面	半面[はんめん]	はんめん	
\\	テニスコートの半面を使って試合をしたの。	テニスコートの 半面[はんめん]を 使[つか]って 試合[しあい]をしたの。	てにすこーと の はんめん を つかって しあい を した の	
\\	テニスコートの
\\	を 使[つか]って 試合[しあい]をしたの。			
\\	まさか	まさか	まさか	
\\	まさか車が当たるとは思わなかった。	まさか 車[くるま]が 当[あ]たるとは 思[おも]わなかった。	まさか くるま が あたる と は おもわなかった	
\\	車[くるま]が 当[あ]たるとは 思[おも]わなかった。			
\\	面	面[めん]	めん	
\\	彼の意外な面を見た。	彼[かれ]の 意外[いがい]な 面[めん]を 見[み]た。	かれ の いがい な めん を みた	
\\	彼[かれ]の 意外[いがい]な
\\	を 見[み]た。			
\\	真っ先	真[ま]っ 先[さき]	まっさき	
\\	合格を真っ先に両親に伝えたの。	合格[ごうかく]を 真[ま]っ 先[さき]に 両親[りょうしん]に 伝[つた]えたの。	ごうかく を まっさき に りょうしん に つたえた の	
\\	合格[ごうかく]を
\\	に 両親[りょうしん]に 伝[つた]えたの。			
\\	真夏	真夏[まなつ]	まなつ	
\\	真夏のビールはおいしい。	真夏[まなつ]のビールはおいしい。	まなつ の びーる は おいしい	
\\	のビールはおいしい。			
\\	真夜中	真夜中[まよなか]	まよなか	
\\	彼は真夜中に帰ってきたの。	彼[かれ]は 真夜中[まよなか]に 帰[かえ]ってきたの。	かれ は まよなか に かえって きた の	
\\	彼[かれ]は
\\	に 帰[かえ]ってきたの。			
\\	真っ最中	真[ま]っ 最中[さいちゅう]	まっさいちゅう	
\\	今は試験の真っ最中です。	今[いま]は 試験[しけん]の 真[ま]っ 最中[さいちゅう]です。	いま は しけん の まっさいちゅう です	
\\	今[いま]は 試験[しけん]の
\\	です。			
\\	どっと	どっと	どっと	
\\	観衆がどっと笑ったんだ。	観衆[かんしゅう]がどっと 笑[わら]ったんだ。	かんしゅう が どっと わらった ん だ	
\\	観衆[かんしゅう]が
\\	笑[わら]ったんだ。			
\\	真心	真心[まごころ]	まごころ	
\\	彼女は真心をこめて彼にケーキを作ったよ。	彼女[かのじょ]は 真心[まごころ]をこめて 彼[かれ]にケーキを 作[つく]ったよ。	かのじょ は まごころ を こめて かれ に けーき を つくった よ	
\\	彼女[かのじょ]は
\\	をこめて 彼[かれ]にケーキを 作[つく]ったよ。			
\\	無色	無色[むしょく]	むしょく	
\\	アルコールは無色です。	アルコールは 無色[むしょく]です。	あるこーる は むしょく です	
\\	アルコールは
\\	です。			
\\	ばら色	ばら 色[いろ]	ばらいろ	
\\	彼女の人生はばら色だったの。	彼女[かのじょ]の 人生[じんせい]はばら 色[いろ]だったの。	かのじょ の じんせい は ばらいろ だった の 。	
\\	彼女[かのじょ]の 人生[じんせい]は
\\	だったの。			
\\	水色	水色[みずいろ]	みずいろ	
\\	箱に水色のリボンがかかっていたの。	箱[はこ]に 水色[みずいろ]のリボンがかかっていたの。	はこ に みずいろ の りぼん が かかって いた の	
\\	箱[はこ]に
\\	のリボンがかかっていたの。			
\\	長方形	長方形[ちょうほうけい]	ちょうほうけい	
\\	私の部屋は長方形です。	私[わたし]の 部屋[へや]は 長方形[ちょうほうけい]です。	わたし の へや は ちょうほうけい です	
\\	私[わたし]の 部屋[へや]は
\\	です。			
\\	地形	地形[ちけい]	ちけい	
\\	ここはなだらかな地形です。	ここはなだらかな 地形[ちけい]です。	ここ は なだらか な ちけい です	
\\	ここはなだらかな
\\	です。			
\\	もうける	もうける	もうける	
\\	先月は株で100万円もうけたの。	先月[せんげつ]は 株[かぶ]で100 万円[まん えん]もうけたの。	せんげつ は かぶ で 
\\	まん えん もうけた の	
\\	先月[せんげつ]は 株[かぶ]で100 万円[まん えん]
\\	の。			
\\	種	種[たね]	たね	
\\	プランターにトマトの種をまきました。	プランターにトマトの 種[たね]をまきました。	ぷらんたーに とまと の たね を まきました 。	
\\	プランターにトマトの
\\	をまきました。			
\\	分類	分類[ぶんるい]	ぶんるい	
\\	これらの本を分類してください。	これらの 本[ほん]を 分類[ぶんるい]してください。	これら の ほん を ぶんるい して ください	
\\	これらの 本[ほん]を
\\	してください。			
\\	直前	直前[ちょくぜん]	ちょくぜん	
\\	彼は旅行の直前に熱が出たの。	彼[かれ]は 旅行[りょこう]の 直前[ちょくぜん]に 熱[ねつ]が 出[で]たの。	かれ は りょこう の ちょくぜん に ねつ が でた の	
\\	彼[かれ]は 旅行[りょこう]の
\\	に 熱[ねつ]が 出[で]たの。			
\\	直ちに	直[ただ]ちに	ただちに	
\\	直ちに会社に戻ってください。	直[ただ]ちに 会社[かいしゃ]に 戻[もど]ってください。	ただちに かいしゃ に もどって ください	
\\	会社[かいしゃ]に 戻[もど]ってください。			
\\	見直す	見直[みなお]す	みなおす	
\\	彼のことを見直しました。	彼[かれ]のことを 見直[みなお]しました。	かれ の こと を みなおしました	
\\	彼[かれ]のことを
\\	ゆったり	ゆったり	ゆったり	
\\	彼女はゆったりした服を着ていますね。	彼女[かのじょ]はゆったりした 服[ふく]を 着[き]ていますね。	かのじょ は ゆったり した ふく を きて います ね	
\\	彼女[かのじょ]は
\\	した 服[ふく]を 着[き]ていますね。			
\\	直後	直後[ちょくご]	ちょくご	
\\	食べた直後に寝ないほうがいいよ。	食[た]べた 直後[ちょくご]に 寝[ね]ないほうがいいよ。	たべた ちょくご に ねない ほう が いい よ	
\\	食[た]べた
\\	に 寝[ね]ないほうがいいよ。			
\\	立ち直る	立[た]ち 直[なお]る	たちなおる	
\\	彼女は失敗から立ち直ったよ。	彼女[かのじょ]は 失敗[しっぱい]から 立[た]ち 直[なお]ったよ。	かのじょ は しっぱい から たちなおった よ	
\\	彼女[かのじょ]は 失敗[しっぱい]から
\\	よ。			
\\	やり直し	やり 直[なお]し	やりなおし	
\\	仕事がやり直しになった。	仕事[しごと]がやり 直[なお]しになった。	しごと が やりなおし に なった	
\\	仕事[しごと]が
\\	になった。			
\\	やり直す	やり 直[なお]す	やりなおす	
\\	急いでやり直します。	急[いそ]いでやり 直[なお]します。	いそい で やりなおします。	
\\	急[いそ]いで
\\	直通	直通[ちょくつう]	ちょくつう	
\\	これは私の直通の電話番号です。	これは 私[わたし]の 直通[ちょくつう]の 電話番号[でんわ ばんごう]です。	これ は わたし の ちょくつう の でんわ ばんごう です 。	
\\	これは 私[わたし]の
\\	の 電話番号[でんわ ばんごう]です。			
\\	ちっとも	ちっとも	ちっとも	
\\	彼女が結婚したなんて、ちっとも知らなかった。	彼女[かのじょ]が 結婚[けっこん]したなんて、ちっとも 知[し]らなかった。	かのじょ が けっこん した なんて ちっとも しらなかった	
\\	彼女[かのじょ]が 結婚[けっこん]したなんて、
\\	知[し]らなかった。			
\\	面接	面接[めんせつ]	めんせつ	
\\	新しい仕事の面接に行ってきました。	新[あたら]しい 仕事[しごと]の 面接[めんせつ]に 行[い]ってきました。	あたらしい しごと の めんせつ に いって きました	
\\	新[あたら]しい 仕事[しごと]の
\\	に 行[い]ってきました。			
\\	無線	無線[むせん]	むせん	
\\	運転手が無線で話していたの。	運転手[うんてんしゅ]が 無線[むせん]で 話[はな]していたの。	うんてんしゅ が むせん で はなして いた の	
\\	運転手[うんてんしゅ]が
\\	で 話[はな]していたの。			
\\	線路	線路[せんろ]	せんろ	
\\	子猫が線路に迷い込んだぞ。	子猫[こねこ]が 線路[せんろ]に 迷[まよ]い 込[こ]んだぞ。	こねこ が せんろ に まよいこんだ ぞ	
\\	子猫[こねこ]が
\\	に 迷[まよ]い 込[こ]んだぞ。			
\\	直線	直線[ちょくせん]	ちょくせん	
\\	直線を描いてください。	直線[ちょくせん]を 描[か]いてください。	ちょくせん を かいて ください	
\\	を 描[か]いてください。			
\\	電線	電線[でんせん]	でんせん	
\\	風で電線が揺れていますね。	風[かぜ]で 電線[でんせん]が 揺[ゆ]れていますね。	かぜ で でんせん が ゆれて います ね	
\\	風[かぜ]で
\\	が 揺[ゆ]れていますね。			
\\	脱線	脱線[だっせん]	だっせん	
\\	地震で電車が脱線したよ。	地震[じしん]で 電車[でんしゃ]が 脱線[だっせん]したよ。	じしん で でんしゃ が だっせん した よ	
\\	地震[じしん]で 電車[でんしゃ]が
\\	したよ。			
\\	ドレス	ドレス	ドレス	
\\	昨日素敵なドレスを買いました。	昨日素敵[きのう すてき]なドレスを 買[か]いました。	きのう すてき な どれす を かいました	
\\	昨日素敵[きのう すてき]な
\\	を 買[か]いました。			
\\	白線	白線[はくせん]	はくせん	
\\	白線の内側に下がってお待ちください。	白線[はくせん]の 内側[うちがわ]に 下[さ]がってお 待[ま]ちください。	はくせん の うちがわ に さがって おまち ください	
\\	の 内側[うちがわ]に 下[さ]がってお 待[ま]ちください。			
\\	方角	方角[ほうがく]	ほうがく	
\\	私と彼は帰る方角が同じです。	私[わたし]と 彼[かれ]は 帰[かえ]る 方角[ほうがく]が 同[おな]じです。	わたし と かれ は かえる ほうがく が おなじ です	
\\	私[わたし]と 彼[かれ]は 帰[かえ]る
\\	が 同[おな]じです。			
\\	直角	直角[ちょっかく]	ちょっかく	
\\	ここで直角に曲がってください。	ここで 直角[ちょっかく]に 曲[ま]がってください。	ここ で ちょっかく に まがって ください	
\\	ここで
\\	に 曲[ま]がってください。			
\\	町角	町角[まちかど]	まちかど	
\\	町角にカフェがありました。	町角[まちかど]にカフェがありました。	まちかど に かふぇ が ありました	
\\	にカフェがありました。			
\\	真四角	真四角[ましかく]	ましかく	
\\	彼の部屋には真四角な窓があったんだ。	彼[かれ]の 部屋[へや]には 真四角[ましかく]な 窓[まど]があったんだ。	かれ の へや に は ましかく な まど が あった ん だ	
\\	彼[かれ]の 部屋[へや]には
\\	な 窓[まど]があったんだ。			
\\	パンク	パンク	パンク	
\\	自転車のタイヤがパンクしました。	自転車[じてんしゃ]のタイヤがパンクしました。	じてんしゃ の たいや が ぱんく しました	
\\	自転車[じてんしゃ]のタイヤが
\\	しました。			
\\	共に	共[とも]に	ともに	
\\	最後まで共にがんばりましょう。	最後[さいご]まで 共[とも]にがんばりましょう。	さいご まで ともに がんばりましょう	
\\	最後[さいご]まで
\\	がんばりましょう。			
\\	同時	同時[どうじ]	どうじ	
\\	二人の走者は同時にゴールしたよ。	二人[ふたり]の 走者[そうしゃ]は 同時[どうじ]にゴールしたよ。	ふたり の そうしゃ は どうじ に ごーる した よ	
\\	二人[ふたり]の 走者[そうしゃ]は
\\	にゴールしたよ。			
\\	同一	同一[どういつ]	どういつ	
\\	この人とその人は、同一人物ですか。	この 人[ひと]とその 人[ひと]は、 同一[どういつ] 人物[じんぶつ]ですか。	この ひと と その ひと は どういつ じんぶつ です か	
\\	この 人[ひと]とその 人[ひと]は、
\\	人物[じんぶつ]ですか。			
\\	同情	同情[どうじょう]	どうじょう	
\\	友人は私に同情してくれたよ。	友人[ゆうじん]は 私[わたし]に 同情[どうじょう]してくれたよ。	ゆうじん は わたし に どうじょう して くれた よ	
\\	友人[ゆうじん]は 私[わたし]に
\\	してくれたよ。			
\\	同性	同性[どうせい]	どうせい	
\\	同性の友達より異性の友達のほうが多いよ。	同性[どうせい]の 友達[ともだち]より 異性[いせい]の 友達[ともだち]のほうが 多[おお]いよ。	どうせい の ともだち より いせい の ともだち の ほう が おおい よ	
\\	の 友達[ともだち]より 異性[いせい]の 友達[ともだち]のほうが 多[おお]いよ。			
\\	のんびり	のんびり	のんびり	
\\	休みの日は家でのんびり過ごします。	休[やす]みの 日[ひ]は 家[うち]でのんびり 過[す]ごします。	やすみ の ひ は うち で のんびり すごします	
\\	休[やす]みの 日[ひ]は 家[うち]で
\\	過[す]ごします。			
\\	違いない	違[ちが]いない	ちがいない	
\\	彼は成功するに違いないわ。	彼[かれ]は 成功[せいこう]するに 違[ちが]いないわ。	かれ は せいこう する に ちがいない わ	
\\	彼[かれ]は 成功[せいこう]するに
\\	わ。			
\\	人違い	人違[ひとちが]い	ひとちがい	
\\	すみません、人違いでした。	すみません、 人違[ひとちが]いでした。	すみません ひとちがい でした	
\\	すみません、
\\	でした。			
\\	見違える	見違[みちが]える	みちがえる	
\\	掃除をしたら部屋が見違えるようになったの。	掃除[そうじ]をしたら 部屋[へや]が 見違[みちが]えるようになったの。	そうじ を したら へや が みちがえる よう に なった の	
\\	掃除[そうじ]をしたら 部屋[へや]が
\\	ようになったの。			
\\	違い	違[ちが]い	ちがい	
\\	この二つには大きな違いがある。	この 二[ふた]つには 大[おお]きな 違[ちが]いがある。	この ふたつ に は おおき な ちがい が ある	
\\	この 二[ふた]つには 大[おお]きな
\\	がある。			
\\	似る	似[に]る	にる	
\\	女の子は話し方まで母親に似るわね。	女[おんな]の 子[こ]は 話[はな]し 方[かた]まで 母親[ははおや]に 似[に]るわね。	おんな の こ は はなしかた まで ははおや に にる わ ね	
\\	女[おんな]の 子[こ]は 話[はな]し 方[かた]まで 母親[ははおや]に
\\	わね。			
\\	はめる	はめる	はめる	
\\	彼女は指輪をたくさんはめていますね。	彼女[かのじょ]は 指輪[ゆびわ]をたくさんはめていますね。	かのじょ は ゆびわ を たくさん はめて います ね	
\\	彼女[かのじょ]は 指輪[ゆびわ]をたくさん
\\	いますね。			
\\	似合う	似合[にあ]う	にあう	
\\	彼女は着物がよく似合います。	彼女[かのじょ]は 着物[きもの]がよく 似合[にあ]います。	かのじょ は きもの が よく にあいます	
\\	彼女[かのじょ]は 着物[きもの]がよく
\\	似せる	似[に]せる	にせる	
\\	彼の字に似せて書きました。	彼[かれ]の 字[じ]に 似[に]せて 書[か]きました。	かれ の じ に にせて かきました	
\\	彼[かれ]の 字[じ]に
\\	書[か]きました。			
\\	旅	旅[たび]	たび	
\\	姉はよく旅をします。	姉[あね]はよく 旅[たび]をします。	あね は よく たび を します	
\\	姉[あね]はよく
\\	をします。			
\\	旅客	旅客[りょかく]	りょかく	
\\	その便は外国人の旅客が多かったよ。	その 便[びん]は 外国人[がいこくじん]の 旅客[りょかく]が 多[おお]かったよ。	その びん は がいこくじん の りょかく が おおかった よ	
\\	その 便[びん]は 外国人[がいこくじん]の
\\	が 多[おお]かったよ。			
\\	旅費	旅費[りょひ]	りょひ	
\\	父が旅費を出してくれました。	父[ちち]が 旅費[りょひ]を 出[だ]してくれました。	ちち が りょひ を だして くれました	
\\	父[ちち]が
\\	を 出[だ]してくれました。			
\\	和らげる	和[やわ]らげる	やわらげる	
\\	ユーモアは場の雰囲気を和らげるわね。	ユーモアは 場[ば]の 雰囲気[ふんいき]を 和[やわ]らげるわね。	ゆーもあ は ば の ふんいき を やわらげる わ ね	
\\	ユーモアは 場[ば]の 雰囲気[ふんいき]を
\\	わね。			
\\	ふた	ふた	ふた	
\\	箱にふたをしたよ。	箱[はこ]にふたをしたよ。	はこ に ふた を した よ	
\\	箱[はこ]に
\\	をしたよ。			
\\	和語	和語[わご]	わご	
\\	日本で生まれた言葉を和語といいます。	日本[にっぽん]で 生[う]まれた 言葉[ことば]を 和語[わご]といいます。	にっぽん で うまれた ことば を わご と いいます	
\\	日本[にっぽん]で 生[う]まれた 言葉[ことば]を
\\	といいます。			
\\	和風	和風[わふう]	わふう	
\\	夕食に和風パスタを作ったよ。	夕食[ゆうしょく]に 和風[わふう]パスタを 作[つく]ったよ。	ゆうしょく に わふう ぱすた を つくった よ	
\\	夕食[ゆうしょく]に
\\	パスタを 作[つく]ったよ。			
\\	和食	和食[わしょく]	わしょく	
\\	私は和食が好きです。	私[わたし]は 和食[わしょく]が 好[す]きです。	わたし は わしょく が すき です	
\\	私[わたし]は
\\	が 好[す]きです。			
\\	和やか	和[なご]やか	なごやか	
\\	彼らは和やかに食事をした。	彼[かれ]らは 和[なご]やかに 食事[しょくじ]をした。	かれら は なごやか に しょくじ を した	
\\	彼[かれ]らは
\\	に 食事[しょくじ]をした。			
\\	和らぐ	和[やわ]らぐ	やわらぐ	
\\	この曲を聞くと気持ちが和らぎます。	この 曲[きょく]を 聞[き]くと 気持[きも]ちが 和[やわ]らぎます。	この きょく を きく と きもち が やわらぎます	
\\	この 曲[きょく]を 聞[き]くと 気持[きも]ちが
\\	レポート	レポート	レポート	
\\	教授にレポートを提出しました。	教授[きょうじゅ]にレポートを 提出[ていしゅつ]しました。	きょうじゅ に れぽーと を ていしゅつ しました	
\\	教授[きょうじゅ]に
\\	を 提出[ていしゅつ]しました。			
\\	和式	和式[わしき]	わしき	
\\	あの家のトイレは和式です。	あの 家[うち]のトイレは 和式[わしき]です。	あの うち の といれ は わしき です	
\\	あの 家[うち]のトイレは
\\	です。			
\\	和英	和英[わえい]	わえい	
\\	私は和英辞書をよく使います。	私[わたし]は 和英[わえい] 辞書[じしょ]をよく 使[つか]います。	わたし は わえい じしょ を よく つかいます	
\\	私[わたし]は
\\	辞書[じしょ]をよく 使[つか]います。			
\\	東洋	東洋[とうよう]	とうよう	
\\	彼は東洋文化を研究しているよ。	彼[かれ]は 東洋[とうよう] 文化[ぶんか]を 研究[けんきゅう]しているよ。	かれ は とうようぶんか を けんきゅう して いる よ	
\\	彼[かれ]は
\\	文化[ぶんか]を 研究[けんきゅう]しているよ。			
\\	洋風	洋風[ようふう]	ようふう	
\\	私は洋風の家に住んでいます。	私[わたし]は 洋風[ようふう]の 家[いえ]に 住[す]んでいます。	わたし は ようふう の いえ に すんで います	
\\	私[わたし]は
\\	の 家[いえ]に 住[す]んでいます。			
\\	洋画	洋画[ようが]	ようが	
\\	私は週に3本洋画を見ます。	私[わたし]は 週[しゅう]に3 本[ぼん] 洋画[ようが]を 見[み]ます。	わたし は しゅう に 
\\	ぼん ようが を みます	
\\	私[わたし]は 週[しゅう]に3 本[ぼん]
\\	を 見[み]ます。			
\\	マネー	マネー	マネー	
\\	インターネットでマネー情報をチェックしたよ。	インターネットでマネー 情報[じょうほう]をチェックしたよ。	いんたーねっと で まねー じょうほう を ちぇっく した よ	
\\	インターネットで
\\	情報[じょうほう]をチェックしたよ。			
\\	洋式	洋式[ようしき]	ようしき	
\\	彼の家のトイレは洋式です。	彼[かれ]の 家[いえ]のトイレは 洋式[ようしき]です。	かれ の いえ の といれ は ようしき です	
\\	彼[かれ]の 家[いえ]のトイレは
\\	です。			
\\	洋食	洋食[ようしょく]	ようしょく	
\\	昨日の晩御飯は洋食でした。	昨日[きのう]の 晩御飯[ばんごはん]は 洋食[ようしょく]でした。	きのう の ばんごはん は ようしょく でした	
\\	昨日[きのう]の 晩御飯[ばんごはん]は
\\	でした。			
\\	洋間	洋間[ようま]	ようま	
\\	彼の家には洋間があります。	彼[かれ]の 家[いえ]には 洋間[ようま]があります。	かれ の いえ に は ようま が あります	
\\	彼[かれ]の 家[いえ]には
\\	があります。			
\\	和服	和服[わふく]	わふく	
\\	彼女は和服がよく似合う。	彼女[かのじょ]は 和服[わふく]がよく 似合[にあ]う。	かのじょ は わふく が よく にあう	
\\	彼女[かのじょ]は
\\	がよく 似合[にあ]う。			
\\	待合室	待合室[まちあいしつ]	まちあいしつ	
\\	待合室はとても込んでいたよ。	待合室[まちあいしつ]はとても 込[こ]んでいたよ。	まちあいしつ は とても こんで いた よ	
\\	はとても 込[こ]んでいたよ。			
\\	和室	和室[わしつ]	わしつ	
\\	この和室の天井は低いね。	この 和室[わしつ]の 天井[てんじょう]は 低[ひく]いね。	この わしつ の てんじょう は ひくい ね	
\\	この
\\	の 天井[てんじょう]は 低[ひく]いね。			
\\	もろい	もろい	もろい	
\\	この岩は意外ともろい。	この 岩[いわ]は 意外[いがい]ともろい。	この いわ は いがい と もろい	
\\	この 岩[いわ]は 意外[いがい]と
\\	洋室	洋室[ようしつ]	ようしつ	
\\	このテーブルは洋室に合わない。	このテーブルは 洋室[ようしつ]に 合[あ]わない。	この てーぶる は ようしつ に あわない	
\\	このテーブルは
\\	に 合[あ]わない。			
\\	窓口	窓口[まどぐち]	まどぐち	
\\	お振込みは3番の窓口です。	お 振込[ふりこ]みは3 番[ばん]の 窓口[まどぐち]です。	おふりこみ は 
\\	ばん の まどぐち です	
\\	お 振込[ふりこ]みは3 番[ばん]の
\\	です。			
\\	父母	父母[ふぼ]	ふぼ	
\\	学校から父母に連絡があったんだ。	学校[がっこう]から 父母[ふぼ]に 連絡[れんらく]があったんだ。	がっこう から ふぼ に れんらく が あった ん だ	
\\	学校[がっこう]から
\\	に 連絡[れんらく]があったんだ。			
\\	母親	母親[ははおや]	ははおや	
\\	彼女は2才の子の母親です。	彼女[かのじょ]は2 才[さい]の 子[こ]の 母親[ははおや]です。	かのじょ は 
\\	さい の こ の ははおや です	
\\	彼女[かのじょ]は2 才[さい]の 子[こ]の
\\	です。			
\\	父親	父親[ちちおや]	ちちおや	
\\	彼の父親は先生です。	彼[かれ]の 父親[ちちおや]は 先生[せんせい]です。	かれ の ちちおや は せんせい です 。	
\\	彼[かれ]の
\\	は 先生[せんせい]です。			
\\	ユーモア	ユーモア	ユーモア	
\\	彼女はユーモアのセンスがあります。	彼女[かのじょ]はユーモアのセンスがあります。	かのじょ は ゆーもあ の せんす が あります 。	
\\	彼女[かのじょ]は
\\	のセンスがあります。			
\\	父兄	父兄[ふけい]	ふけい	
\\	父兄の方々はこちらへどうぞ。	父兄[ふけい]の 方々[かたがた]はこちらへどうぞ。	ふけい の かたがた は こちら へ どうぞ	
\\	の 方々[かたがた]はこちらへどうぞ。			
\\	第一	第一[だいいち]	だいいち	
\\	私には仕事が第一です。	私[わたし]には 仕事[しごと]が 第一[だいいち]です。	わたし に は しごと が だいいち です	
\\	私[わたし]には 仕事[しごと]が
\\	です。			
\\	利息	利息[りそく]	りそく	
\\	預金には利息がつきます。	預金[よきん]には 利息[りそく]がつきます。	よきん に は りそく が つきます	
\\	預金[よきん]には
\\	がつきます。			
\\	若々しい	若々[わかわか]しい	わかわかしい	
\\	彼女はとても若々しい人です。	彼女[かのじょ]はとても 若々[わかわか]しい 人[ひと]です。	かのじょ は とても わかわかしい ひと です	
\\	彼女[かのじょ]はとても
\\	人[ひと]です。			
\\	読者	読者[どくしゃ]	どくしゃ	
\\	新聞の読者の投書欄は好きです。	新聞[しんぶん]の 読者[どくしゃ]の 投書欄[とうしょらん]は 好[す]きです。	しんぶん の どくしゃ の とうしょらん は すき です	
\\	新聞[しんぶん]の
\\	の 投書欄[とうしょらん]は 好[す]きです。			
\\	バック	バック	バック	
\\	車のギアをバックに入れたよ。	車[くるま]のギアをバックに 入[い]れたよ。	くるま の ぎあ を ばっく に いれた よ	
\\	車[くるま]のギアを
\\	に 入[い]れたよ。			
\\	歩行者	歩行者[ほこうしゃ]	ほこうしゃ	
\\	歩行者は道の右側を歩いてください。	歩行者[ほこうしゃ]は 道[みち]の 右側[みぎがわ]を 歩[ある]いてください。	ほこうしゃ は みち の みぎがわ を あるいて ください	
\\	は 道[みち]の 右側[みぎがわ]を 歩[ある]いてください。			
\\	文学者	文学者[ぶんがくしゃ]	ぶんがくしゃ	
\\	彼は有名な文学者です。	彼[かれ]は 有名[ゆうめい]な 文学者[ぶんがくしゃ]です。	かれ は ゆうめい な ぶんがくしゃ です	
\\	彼[かれ]は 有名[ゆうめい]な
\\	です。			
\\	歩行者天国	歩行者天国[ほこうしゃてんごく]	ほこうしゃてんごく	
\\	日曜日はこの通りが歩行者天国になります。	日曜日[にちようび]はこの 通[とお]りが 歩行者天国[ほこうしゃてんごく]になります。	にちようび は この とおり が ほこうしゃてんごく に なります	
\\	日曜日[にちようび]はこの 通[とお]りが
\\	になります。			
\\	者	者[もの]	もの	
\\	私は先ほど電話した者です。	私[わたし]は 先[さき]ほど 電話[でんわ]した 者[もの]です。	わたし は さきほど でんわ した もの です	
\\	私[わたし]は 先[さき]ほど 電話[でんわ]した
\\	です。			
\\	結ぶ	結[むす]ぶ	むすぶ	
\\	彼女は髪にリボンを結んだね。	彼女[かのじょ]は 髪[かみ]にリボンを 結[むす]んだね。	かのじょ は かみ に りぼん を むすんだ ね	
\\	彼女[かのじょ]は 髪[かみ]にリボンを
\\	ね。			
\\	結び	結[むす]び	むすび	
\\	彼は会の結びとしてスピーチをしたよ。	彼[かれ]は 会[かい]の 結[むす]びとしてスピーチをしたよ。	かれ は かい の むすび と して すぴーち を した よ	
\\	彼[かれ]は 会[かい]の
\\	としてスピーチをしたよ。			
\\	そびえる	そびえる	そびえる	
\\	いくつもの高層ビルがそびえていたんだ。	いくつもの 高層[こうそう]ビルがそびえていたんだ。	いくつ も の こうそう びる が そびえていた ん だ 。	
\\	いくつもの 高層[こうそう]ビルが
\\	んだ。			
\\	未婚	未婚[みこん]	みこん	
\\	彼はまだ未婚です。	彼[かれ]はまだ 未婚[みこん]です。	かれ は まだ みこん です	
\\	彼[かれ]はまだ
\\	です。			
\\	果たす	果[は]たす	はたす	
\\	彼はしっかりと責任を果たしました。	彼[かれ]はしっかりと 責任[せきにん]を 果[は]たしました。	かれ は しっかり と せきにん を はたしました	
\\	彼[かれ]はしっかりと 責任[せきにん]を
\\	果たして	果[は]たして	はたして	
\\	果たして彼は現れるだろうか。	果[は]たして 彼[かれ]は 現[あらわ]れるだろうか。	はたして かれ は あらわれるだろう か	
\\	彼[かれ]は 現[あらわ]れるだろうか。			
\\	日課	日課[にっか]	にっか	
\\	犬の散歩は私の日課です。	犬[いぬ]の 散歩[さんぽ]は 私[わたし]の 日課[にっか]です。	いぬ の さんぽ は わたし の にっか です	
\\	犬[いぬ]の 散歩[さんぽ]は 私[わたし]の
\\	です。			
\\	有効	有効[ゆうこう]	ゆうこう	
\\	私の免許は来年まで有効です。	私[わたし]の 免許[めんきょ]は 来年[らいねん]まで 有効[ゆうこう]です。	わたし の めんきょ は らいねん まで ゆうこう です	
\\	私[わたし]の 免許[めんきょ]は 来年[らいねん]まで
\\	です。			
\\	タレント	タレント	タレント	
\\	彼はタレントなのでよくテレビに出ています。	彼[かれ]はタレントなのでよくテレビに 出[で]ています。	かれ は たれんと な の で よく てれび に でて います	
\\	彼[かれ]は
\\	なのでよくテレビに 出[で]ています。			
\\	無効	無効[むこう]	むこう	
\\	このクーポンはもう無効です。	このクーポンはもう 無効[むこう]です。	この くーぽん は もう むこう です	
\\	このクーポンはもう
\\	です。			
\\	自ら	自[みずか]ら	みずから	
\\	社長自らがその会社と交渉したよ。	社長[しゃちょう] 自[みずか]らがその 会社[かいしゃ]と 交渉[こうしょう]したよ。	しゃちょう みずから が その かいしゃ と こうしょう した よ	
\\	社長[しゃちょう]
\\	がその 会社[かいしゃ]と 交渉[こうしょう]したよ。			
\\	不自然	不自然[ふしぜん]	ふしぜん	
\\	彼の態度はどこか不自然だったな。	彼[かれ]の 態度[たいど]はどこか 不自然[ふしぜん]だったな。	かれ の たいど は どこか ふしぜん だった な	
\\	彼[かれ]の 態度[たいど]はどこか
\\	だったな。			
\\	不自由	不自由[ふじゆう]	ふじゆう	
\\	彼は不自由な暮らしに慣れたようだね。	彼[かれ]は 不自由[ふじゆう]な 暮[く]らしに 慣[な]れたようだね。	かれ は ふじゆう な くらし に なれた よう だ ね	
\\	彼[かれ]は
\\	な 暮[く]らしに 慣[な]れたようだね。			
\\	通信	通信[つうしん]	つうしん	
\\	通信教育はとても便利です。	通信[つうしん] 教育[きょういく]はとても 便利[べんり]です。	つうしん きょういく は とても べんり です	
\\	教育[きょういく]はとても 便利[べんり]です。			
\\	たまに	たまに	たまに	
\\	彼はたまに料理をします。	彼[かれ]はたまに 料理[りょうり]をします。	かれ は たまに りょうり を します	
\\	彼[かれ]は
\\	料理[りょうり]をします。			
\\	頼る	頼[たよ]る	たよる	
\\	人に頼らないで、自分でやってごらん。	人[ひと]に 頼[たよ]らないで、 自分[じぶん]でやってごらん。	ひと に たよらない で じぶん で やって ごらん	
\\	人[ひと]に
\\	、 自分[じぶん]でやってごらん。			
\\	頼み	頼[たの]み	たのみ	
\\	あなたに頼みがあるんですけど。	あなたに 頼[たの]みがあるんですけど。	あなた に たのみ が ある ん です けど	
\\	あなたに
\\	があるんですけど。			
\\	頼もしい	頼[たの]もしい	たのもしい	
\\	彼は頼もしい人です。	彼[かれ]は 頼[たの]もしい 人[ひと]です。	かれ は たのもしい ひと です	
\\	彼[かれ]は
\\	人[ひと]です。			
\\	民間	民間[みんかん]	みんかん	
\\	その土地は民間企業に売却されたんだ。	その 土地[とち]は 民間[みんかん] 企業[きぎょう]に 売却[ばいきゃく]されたんだ。	その とち は みんかん きぎょう に ばいきゃく された ん だ	
\\	その 土地[とち]は
\\	企業[きぎょう]に 売却[ばいきゃく]されたんだ。			
\\	民族	民族[みんぞく]	みんぞく	
\\	私は民族の歴史に興味があります。	私[わたし]は 民族[みんぞく]の 歴史[れきし]に 興味[きょうみ]があります。	わたし は みんぞく の れきし に きょうみ が あります	
\\	私[わたし]は
\\	の 歴史[れきし]に 興味[きょうみ]があります。			
\\	ペンキ	ペンキ	ペンキ	
\\	床にペンキをこぼしてしまいました。	床[ゆか]にペンキをこぼしてしまいました。	ゆか に ぺんき を こぼして しまいました	
\\	床[ゆか]に
\\	をこぼしてしまいました。			
\\	持ち主	持[も]ち 主[ぬし]	もちぬし	
\\	この自転車の持ち主は誰ですか。	この 自転車[じてんしゃ]の 持[も]ち 主[ぬし]は 誰[だれ]ですか。	この じてんしゃ の もちぬし は だれ です か	
\\	この 自転車[じてんしゃ]の
\\	は 誰[だれ]ですか。			
\\	民主	民主[みんしゅ]	みんしゅ	
\\	民主主義について勉強しました。	民主[みんしゅ] 主義[しゅぎ]について 勉強[べんきょう]しました。	みんしゅしゅぎ に ついて べんきょう しました	
\\	主義[しゅぎ]について 勉強[べんきょう]しました。			
\\	家主	家主[やぬし]	やぬし	
\\	家主は1階に住んでいます。	家主[やぬし]は1 階[かい]に 住[す]んでいます。	やぬし は 
\\	かい に すんで います	
\\	は1 階[かい]に 住[す]んでいます。			
\\	定義	定義[ていぎ]	ていぎ	
\\	美しさを定義してください。	美[うつく]しさを 定義[ていぎ]してください。	うつくしさ を ていぎ して ください	
\\	美[うつく]しさを
\\	してください。			
\\	不思議	不思議[ふしぎ]	ふしぎ	
\\	それはとても不思議な話だね。	それはとても 不思議[ふしぎ]な 話[はなし]だね。	それ は とても ふしぎ な はなし だ ね	
\\	それはとても
\\	な 話[はなし]だね。			
\\	理論	理論[りろん]	りろん	
\\	彼は大学で音楽の理論を勉強したんだよ。	彼[かれ]は 大学[だいがく]で 音楽[おんがく]の 理論[りろん]を 勉強[べんきょう]したんだよ。	かれ は だいがく で おんがく の りろん を べんきょう した ん だ よ	
\\	彼[かれ]は 大学[だいがく]で 音楽[おんがく]の
\\	を 勉強[べんきょう]したんだよ。			
\\	ベンチ	ベンチ	ベンチ	
\\	公園のベンチで本を読んでいます。	公園[こうえん]のベンチで 本[ほん]を 読[よ]んでいます。	こうえん の べんち で ほん を よんで います	
\\	公園[こうえん]の
\\	で 本[ほん]を 読[よ]んでいます。			
\\	論理	論理[ろんり]	ろんり	
\\	数学は論理に基づく学問です。	数学[すうがく]は 論理[ろんり]に 基[もと]づく 学問[がくもん]です。	すうがく は ろんり に もとづく がくもん です	
\\	数学[すうがく]は
\\	に 基[もと]づく 学問[がくもん]です。			
\\	論文	論文[ろんぶん]	ろんぶん	
\\	論文を仕上げないと卒業できませんよ。	論文[ろんぶん]を 仕上[しあ]げないと 卒業[そつぎょう]できませんよ。	ろんぶん を しあげない と そつぎょう できません よ	
\\	を 仕上[しあ]げないと 卒業[そつぎょう]できませんよ。			
\\	論じる	論[ろん]じる	ろんじる	
\\	仲間と地球の未来について論じたの。	仲間[なかま]と 地球[ちきゅう]の 未来[みらい]について 論[ろん]じたの。	なかま と ちきゅう の みらい に ついて ろんじた の	
\\	仲間[なかま]と 地球[ちきゅう]の 未来[みらい]について
\\	の。			
\\	無論	無論[むろん]	むろん	
\\	ぼくは無論きみに賛成だ。	ぼくは 無論[むろん]きみに 賛成[さんせい]だ。	ぼく は むろん きみ に さんせい だ	
\\	ぼくは
\\	きみに 賛成[さんせい]だ。			
\\	世論	世論[よろん]	よろん	
\\	世論は新しい法律に反対です。	世論[よろん]は 新[あたら]しい 法律[ほうりつ]に 反対[はんたい]です。	よろん は あたらしい ほうりつ に はんたい です	
\\	は 新[あたら]しい 法律[ほうりつ]に 反対[はんたい]です。			
\\	めいめい	めいめい	めいめい	
\\	交通費はめいめい支払ってください。	交通費[こうつうひ]はめいめい 支払[しはら]ってください。	こうつうひ は めいめい しはらって ください	
\\	交通費[こうつうひ]は
\\	支払[しはら]ってください。			
\\	論	論[ろん]	ろん	
\\	彼の人生論は独特だな。	彼[かれ]の 人生[じんせい] 論[ろん]は 独特[どくとく]だな。	かれ の じんせいろん は どくとく だ な	
\\	彼[かれ]の 人生[じんせい]
\\	は 独特[どくとく]だな。			
\\	反発	反発[はんぱつ]	はんぱつ	
\\	彼は先生に反発していたよ。	彼[かれ]は 先生[せんせい]に 反発[はんぱつ]していたよ。	かれ は せんせい に はんぱつ して いた よ	
\\	彼[かれ]は 先生[せんせい]に
\\	していたよ。			
\\	反映	反映[はんえい]	はんえい	
\\	新型車に研究の成果が反映されています。	新型車[しんがたしゃ]に 研究[けんきゅう]の 成果[せいか]が 反映[はんえい]されています。	しんがたしゃ に けんきゅう の せいか が はんえい されて います	
\\	新型車[しんがたしゃ]に 研究[けんきゅう]の 成果[せいか]が
\\	されています。			
\\	反論	反論[はんろん]	はんろん	
\\	反論がある人はどうぞ。	反論[はんろん]がある 人[ひと]はどうぞ。	はんろん が ある ひと は どうぞ	
\\	がある 人[ひと]はどうぞ。			
\\	反する	反[はん]する	はんする	
\\	彼の行いはモラルに反しています。	彼[かれ]の 行[おこな]いはモラルに 反[はん]しています。	かれ の おこない は もらる に はんして います	
\\	彼[かれ]の 行[おこな]いはモラルに
\\	います。			
\\	どうにか	どうにか	どうにか	
\\	どうにか締め切りに間に合った。	どうにか 締[し]め 切[き]りに 間[ま]に 合[あ]った。	どうにか しめきり に まにあった	
\\	締[し]め 切[き]りに 間[ま]に 合[あ]った。			
\\	反則	反則[はんそく]	はんそく	
\\	その選手は反則で負けました。	その 選手[せんしゅ]は 反則[はんそく]で 負[ま]けました。	その せんしゅ は はんそく で まけました	
\\	その 選手[せんしゅ]は
\\	で 負[ま]けました。			
\\	対話	対話[たいわ]	たいわ	
\\	親子の対話は大切だよ。	親子[おやこ]の 対話[たいわ]は 大切[たいせつ]だよ。	おやこ の たいわ は たいせつ だ よ	
\\	親子[おやこ]の
\\	は 大切[たいせつ]だよ。			
\\	対	対[つい]	つい	
\\	このズボンは上着と対になっています。	このズボンは 上着[うわぎ]と 対[つい]になっています。	この ずぼん は うわぎ と つい に なって います	
\\	このズボンは 上着[うわぎ]と
\\	になっています。			
\\	対応	対応[たいおう]	たいおう	
\\	彼はいつも素早い対応をするね。	彼[かれ]はいつも 素早[すばや]い 対応[たいおう]をするね。	かれ は いつも すばやい たいおう を する ね	
\\	彼[かれ]はいつも 素早[すばや]い
\\	をするね。			
\\	反応	反応[はんのう]	はんのう	
\\	彼女は良い反応を示したわ。	彼女[かのじょ]は 良[よ]い 反応[はんのう]を 示[しめ]したわ。	かのじょ は よい はんのう を しめした わ	
\\	彼女[かのじょ]は 良[よ]い
\\	を 示[しめ]したわ。			
\\	問答	問答[もんどう]	もんどう	
\\	あなたと問答している暇はないの。	あなたと 問答[もんどう]している 暇[ひま]はないの。	あなた と もんどう して いる ひま は ない の	
\\	あなたと
\\	している 暇[ひま]はないの。			
\\	ちゃんと	ちゃんと	ちゃんと	
\\	朝食はちゃんと食べましたか。	朝食[ちょうしょく]はちゃんと 食[た]べましたか。	ちょうしょく は ちゃんと たべました か	
\\	朝食[ちょうしょく]は
\\	食[た]べましたか。			
\\	特定	特定[とくてい]	とくてい	
\\	警察は犯人を特定したらしいよ。	警察[けいさつ]は 犯人[はんにん]を 特定[とくてい]したらしいよ。	けいさつ は はんにん を とくてい した らしい よ	
\\	警察[けいさつ]は 犯人[はんにん]を
\\	したらしいよ。			
\\	特色	特色[とくしょく]	とくしょく	
\\	その学校の教育は特色がありますね。	その 学校[がっこう]の 教育[きょういく]は 特色[とくしょく]がありますね。	その がっこう の きょういく は とくしょく が あります ね	
\\	その 学校[がっこう]の 教育[きょういく]は
\\	がありますね。			
\\	特有	特有[とくゆう]	とくゆう	
\\	これは子供に特有の病気です。	これは 子供[こども]に 特有[とくゆう]の 病気[びょうき]です。	これ は こども に とくゆう の びょうき です	
\\	これは 子供[こども]に
\\	の 病気[びょうき]です。			
\\	別に	別[べつ]に	べつに	
\\	私は別に気になりません。	私[わたし]は 別[べつ]に 気[き]になりません。	わたし は べつに き に なりません	
\\	私[わたし]は
\\	気[き]になりません。			
\\	別れ	別[わか]れ	わかれ	
\\	別れはいつでも悲しいものです。	別[わか]れはいつでも 悲[かな]しいものです。	わかれ は いつでも かなしい もの です	
\\	はいつでも 悲[かな]しいものです。			
\\	のどか	のどか	のどか	
\\	私の田舎はのどかな所です。	私[わたし]の 田舎[いなか]はのどかな 所[ところ]です。	わたし の いなか は のどか な ところ です	
\\	私[わたし]の 田舎[いなか]は
\\	な 所[ところ]です。			
\\	送別	送別[そうべつ]	そうべつ	
\\	送別の辞は誰に頼みましょうか。	送別[そうべつ]の 辞[じ]は 誰[だれ]に 頼[たの]みましょうか。	そうべつ の じ は だれ に たのみましょう か	
\\	の 辞[じ]は 誰[だれ]に 頼[たの]みましょうか。			
\\	送別会	送別会[そうべつかい]	そうべつかい	
\\	来週、課長の送別会を開きます。	来週[らいしゅう]、 課長[かちょう]の 送別会[そうべつかい]を 開[ひら]きます。	らいしゅう かちょう の そうべつかい を ひらきます	
\\	来週[らいしゅう]、 課長[かちょう]の
\\	を 開[ひら]きます。			
\\	専門家	専門家[せんもんか]	せんもんか	
\\	教授はフランス文学の専門家。	教授[きょうじゅ]はフランス 文学[ぶんがく]の 専門家[せんもんか]。	きょうじゅ は ふらんす ぶんがく の せんもんか	
\\	教授[きょうじゅ]はフランス 文学[ぶんがく]の
\\	専門	専門[せんもん]	せんもん	
\\	法律は私の専門です。	法律[ほうりつ]は 私[わたし]の 専門[せんもん]です。	ほうりつ は わたし の せんもん です	
\\	法律[ほうりつ]は 私[わたし]の
\\	です。			
\\	専用	専用[せんよう]	せんよう	
\\	これは女性専用の車両です。	これは 女性[じょせい] 専用[せんよう]の 車両[しゃりょう]です。	これ は じょせい せんよう の しゃりょう です	
\\	これは 女性[じょせい]
\\	の 車両[しゃりょう]です。			
\\	メロディー	メロディー	メロディー	
\\	このメロディーは聞いたことがある。	このメロディーは 聞[き]いたことがある。	この めろでぃー は きいた こと が ある	
\\	この
\\	は 聞[き]いたことがある。			
\\	全般	全般[ぜんぱん]	ぜんぱん	
\\	彼は植物全般に詳しいね。	彼[かれ]は 植物[しょくぶつ] 全般[ぜんぱん]に 詳[くわ]しいね。	かれ は しょくぶつ ぜんぱん に くわしい ね	
\\	彼[かれ]は 植物[しょくぶつ]
\\	に 詳[くわ]しいね。			
\\	本格的	本格的[ほんかくてき]	ほんかくてき	
\\	彼は絵を本格的に勉強しているんだ。	彼[かれ]は 絵[え]を 本格的[ほんかくてき]に 勉強[べんきょう]しているんだ。	かれ は え を ほんかくてき に べんきょう して いる ん だ	
\\	彼[かれ]は 絵[え]を
\\	に 勉強[べんきょう]しているんだ。			
\\	全面的	全面的[ぜんめんてき]	ぜんめんてき	
\\	彼が全面的に協力してくれるそうです。	彼[かれ]が 全面的[ぜんめんてき]に 協力[きょうりょく]してくれるそうです。	かれ が ぜんめんてき に きょうりょく して くれる そう です	
\\	彼[かれ]が
\\	に 協力[きょうりょく]してくれるそうです。			
\\	論理的	論理的[ろんりてき]	ろんりてき	
\\	彼は論理的な人です。	彼[かれ]は 論理的[ろんりてき]な 人[ひと]です。	かれ は ろんりてき な ひと です	
\\	彼[かれ]は
\\	な 人[ひと]です。			
\\	知的	知的[ちてき]	ちてき	
\\	彼はすごく知的な人です。	彼[かれ]はすごく 知的[ちてき]な 人[ひと]です。	かれ は すごく ちてき な ひと です	
\\	彼[かれ]はすごく
\\	な 人[ひと]です。			
\\	目的地	目的地[もくてきち]	もくてきち	
\\	やっと目的地に着いたよ。	やっと 目的地[もくてきち]に 着[つ]いたよ。	やっと もくてきち に ついた よ	
\\	やっと
\\	に 着[つ]いたよ。			
\\	リットル	リットル	リットル	
\\	今日は水を2リットル以上飲んだわ。	今日[きょう]は 水[みず]を2リットル 以上飲[いじょう の]んだわ。	きょう は みず を 
\\	りっとる いじょう のんだ わ	
\\	今日[きょう]は 水[みず]を2
\\	以上飲[いじょう の]んだわ。			
\\	文化的	文化的[ぶんかてき]	ぶんかてき	
\\	この国は文化的な事業に力を入れています。	この 国[くに]は 文化的[ぶんかてき]な 事業[じぎょう]に 力[ちから]を 入[い]れています。	この くに は ぶんかてき な じぎょう に ちから を いれて います	
\\	この 国[くに]は
\\	な 事業[じぎょう]に 力[ちから]を 入[い]れています。			
\\	男性的	男性的[だんせいてき]	だんせいてき	
\\	彼はとても男性的な人です。	彼[かれ]はとても 男性的[だんせいてき]な 人[ひと]です。	かれ は とても だんせいてき な ひと です	
\\	彼[かれ]はとても
\\	な 人[ひと]です。			
\\	普段	普段[ふだん]	ふだん	
\\	私は普段は
\\	シャツとジーンズを着ています。	私[わたし]は 普段[ふだん]は 
\\	[てぃー]シャツとジーンズを 着[き]ています。	わたし は ふだん は てぃーしゃつ と じーんず を きて います	
\\	私[わたし]は
\\	は 
\\	[てぃー]シャツとジーンズを 着[き]ています。			
\\	並み	並[な]み	なみ	
\\	彼は並みの人間ではありません。	彼[かれ]は 並[な]みの 人間[にんげん]ではありません。	かれ は なみ の にんげん で は ありません	
\\	彼[かれ]は
\\	の 人間[にんげん]ではありません。			
\\	並木	並木[なみき]	なみき	
\\	駅前の並木が台風で倒れたらしい。	駅前[えきまえ]の 並木[なみき]が 台風[たいふう]で 倒[たお]れたらしい。	えきまえ の なみき が たいふう で たおれた らしい	
\\	駅前[えきまえ]の
\\	が 台風[たいふう]で 倒[たお]れたらしい。			
\\	わざと	わざと	わざと	
\\	彼はわざと負けたように見えたな。	彼[かれ]はわざと 負[ま]けたように 見[み]えたな。	かれ は わざと まけた よう に みえた な	
\\	彼[かれ]は
\\	負[ま]けたように 見[み]えたな。			
\\	平面	平面[へいめん]	へいめん	
\\	このメガネをかけると平面が立体に見えます。	このメガネをかけると 平面[へいめん]が 立体[りったい]に 見[み]えます。	この めがね を かける と へいめん が りったい に みえます	
\\	このメガネをかけると
\\	が 立体[りったい]に 見[み]えます。			
\\	平気	平気[へいき]	へいき	
\\	彼女は平気な顔をしていた。	彼女[かのじょ]は 平気[へいき]な 顔[かお]をしていた。	かのじょ は へいき な かお を して いた	
\\	彼女[かのじょ]は
\\	な 顔[かお]をしていた。			
\\	平ら	平[たい]ら	たいら	
\\	その建物の屋根は平らだね。	その 建物[たてもの]の 屋根[やね]は 平[たい]らだね。	その たてもの の やね は たいら だ ね	
\\	その 建物[たてもの]の 屋根[やね]は
\\	だね。			
\\	平行	平行[へいこう]	へいこう	
\\	平行に線を引いてください。	平行[へいこう]に 線[せん]を 引[ひ]いてください。	へいこう に せん を ひいて ください	
\\	に 線[せん]を 引[ひ]いてください。			
\\	不平	不平[ふへい]	ふへい	
\\	私はいつも同僚の不平を聞いているんだ。	私[わたし]はいつも 同僚[どうりょう]の 不平[ふへい]を 聞[き]いているんだ。	わたし は いつも どうりょう の ふへい を きいて いる ん だ	
\\	私[わたし]はいつも 同僚[どうりょう]の
\\	を 聞[き]いているんだ。			
\\	とっくに	とっくに	とっくに	
\\	彼ならとっくに帰りましたよ。	彼[かれ]ならとっくに 帰[かえ]りましたよ。	かれ なら とっくに かえりました よ	
\\	彼[かれ]なら
\\	帰[かえ]りましたよ。			
\\	地平線	地平線[ちへいせん]	ちへいせん	
\\	地平線に夕日が沈むところだったの。	地平線[ちへいせん]に 夕日[ゆうひ]が 沈[しず]むところだったの。	ちへいせん に ゆうひ が しずむ ところ だった の	
\\	に 夕日[ゆうひ]が 沈[しず]むところだったの。			
\\	平野	平野[へいや]	へいや	
\\	広い平野が一面雪で真っ白でした。	広[ひろ]い 平野[へいや]が 一面雪[いちめん ゆき]で 真[ま]っ 白[しろ]でした。	ひろい へいや が いちめん ゆき で まっしろ でした	
\\	広[ひろ]い
\\	が 一面雪[いちめん ゆき]で 真[ま]っ 白[しろ]でした。			
\\	平たい	平[ひら]たい	ひらたい	
\\	平たいお皿を一枚取って。	平[ひら]たいお 皿[さら]を 一枚取[いちまい と]って。	ひらたい おさら を いちまい とって	
\\	お 皿[さら]を 一枚取[いちまい と]って。			
\\	平方	平方[へいほう]	へいほう	
\\	この土地の面積は約100平方メートルです。	この 土地[とち]の 面積[めんせき]は 約100[やく 
\\	平方[へいほう]メートルです。	この とち の めんせき は やく 
\\	へいほうめーとる です	
\\	この 土地[とち]の 面積[めんせき]は 約100[やく 
\\	メートルです。			
\\	平日	平日[へいじつ]	へいじつ	
\\	彼は平日がお休みです。	彼[かれ]は 平日[へいじつ]がお 休[やす]みです。	かれ は へいじつ が おやすみ です	
\\	彼[かれ]は
\\	がお 休[やす]みです。			
\\	ベスト	ベスト	ベスト	
\\	私はベストを尽くしました。	私[わたし]はベストを 尽[つ]くしました。	わたし は べすと を つくしました	
\\	私[わたし]は
\\	を 尽[つ]くしました。			
\\	平均	平均[へいきん]	へいきん	
\\	平均で一日に8時間ぐらい働いています。	平均[へいきん]で 一日[いちにち]に8 時間[じかん]ぐらい 働[はたら]いています。	へいきん で いちにち に 
\\	じかん ぐらい はたらいて います	
\\	で 一日[いちにち]に8 時間[じかん]ぐらい 働[はたら]いています。			
\\	等しい	等[ひと]しい	ひとしい	
\\	私は株の知識がないに等しいです。	私[わたし]は 株[かぶ]の 知識[ちしき]がないに 等[ひと]しいです。	わたし は かぶ の ちしき が ない に ひとしい です	
\\	私[わたし]は 株[かぶ]の 知識[ちしき]がないに
\\	です。			
\\	平等	平等[びょうどう]	びょうどう	
\\	あの先生は生徒をみな平等に扱います。	あの 先生[せんせい]は 生徒[せいと]をみな 平等[びょうどう]に 扱[あつか]います。	あの せんせい は せいと を みな びょうどう に あつかいます	
\\	あの 先生[せんせい]は 生徒[せいと]をみな
\\	に 扱[あつか]います。			
\\	同等	同等[どうとう]	どうとう	
\\	彼には大学生と同等の学力があります。	彼[かれ]には 大学生[だいがくせい]と 同等[どうとう]の 学力[がくりょく]があります。	かれ に は だいがくせい と どうとう の がくりょく が あります	
\\	彼[かれ]には 大学生[だいがくせい]と
\\	の 学力[がくりょく]があります。			
\\	対等	対等[たいとう]	たいとう	
\\	その子供は大人と対等に話していたよ。	その 子供[こども]は 大人[おとな]と 対等[たいとう]に 話[はな]していたよ。	その こども は おとな と たいとう に はなして いた よ	
\\	その 子供[こども]は 大人[おとな]と
\\	に 話[はな]していたよ。			
\\	不平等	不平等[ふびょうどう]	ふびょうどう	
\\	職場での男女不平等はよく見られるね。	職場[しょくば]での 男女[だんじょ] 不平等[ふびょうどう]はよく 見[み]られるね。	しょくば で の だんじょ ふびょうどう は よく みられる ね	
\\	職場[しょくば]での 男女[だんじょ]
\\	はよく 見[み]られるね。			
\\	ワンピース	ワンピース	ワンピース	
\\	彼女は白いワンピースを着ていたよ。	彼女[かのじょ]は 白[しろ]いワンピースを 着[き]ていたよ。	かのじょ は しろい わんぴーす を きて いた よ	
\\	彼女[かのじょ]は 白[しろ]い
\\	を 着[き]ていたよ。			
\\	病室	病室[びょうしつ]	びょうしつ	
\\	その病室はとても広くてきれいだった。	その 病室[びょうしつ]はとても 広[ひろ]くてきれいだった。	その びょうしつ は とても ひろくて きれい だった	
\\	その
\\	はとても 広[ひろ]くてきれいだった。			
\\	病人	病人[びょうにん]	びょうにん	
\\	病人が出たので電車が少し止まったんだ。	病人[びょうにん]が 出[で]たので 電車[でんしゃ]が 少[すこ]し 止[と]まったんだ。	びょうにん が でた の で でんしゃ が すこし とまった ん だ	
\\	が 出[で]たので 電車[でんしゃ]が 少[すこ]し 止[と]まったんだ。			
\\	内科	内科[ないか]	ないか	
\\	内科で胃の調子を診てもらったよ。	内科[ないか]で 胃[い]の 調子[ちょうし]を 診[み]てもらったよ。	ないか で い の ちょうし を みて もらった よ	
\\	で 胃[い]の 調子[ちょうし]を 診[み]てもらったよ。			
\\	必死	必死[ひっし]	ひっし	
\\	学生たちは授業についていくのに必死です。	学生[がくせい]たちは 授業[じゅぎょう]についていくのに 必死[ひっし]です。	がくせいたち は じゅぎょう に ついて いく の に ひっし です	
\\	学生[がくせい]たちは 授業[じゅぎょう]についていくのに
\\	です。			
\\	病死	病死[びょうし]	びょうし	
\\	その作家は35歳という若さで病死した。	その 作家[さっか]は35 歳[さい]という 若[わか]さで 病死[びょうし]した。	その さっか は 
\\	さい と いう わかさ で びょうし した	
\\	その 作家[さっか]は35 歳[さい]という 若[わか]さで
\\	した。			
\\	ダイヤモンド	ダイヤモンド	ダイヤモンド	
\\	ダイヤモンドはとても高価ですね。	ダイヤモンドはとても 高価[こうか]ですね。	だいやもんど は とても こうか です ね	
\\	はとても 高価[こうか]ですね。			
\\	必死に	必死[ひっし]に	ひっしに	
\\	必死に単語を暗記したよ。	必死[ひっし]に 単語[たんご]を 暗記[あんき]したよ。	ひっしに たんご を あんき した よ	
\\	単語[たんご]を 暗記[あんき]したよ。			
\\	亡くす	亡[な]くす	なくす	
\\	昨年、友人を亡くしました。	昨年[さくねん]、 友人[ゆうじん]を 亡[な]くしました。	さくねん ゆうじん を なくしました	
\\	昨年[さくねん]、 友人[ゆうじん]を
\\	多忙	多忙[たぼう]	たぼう	
\\	彼女は多忙な人です。	彼女[かのじょ]は 多忙[たぼう]な 人[ひと]です。	かのじょ は たぼう な ひと です	
\\	彼女[かのじょ]は
\\	な 人[ひと]です。			
\\	疲れ	疲[つか]れ	つかれ	
\\	最近疲れがたまっています。	最近[さいきん] 疲[つか]れがたまっています。	さいきん つかれ が たまって います	
\\	最近[さいきん]
\\	がたまっています。			
\\	立ち入り禁止	立[た]ち 入[い]り 禁止[きんし]	たちいりきんし	
\\	ここは立ち入り禁止です。	ここは 立[た]ち 入[い]り 禁止[きんし]です。	ここ は たちいりきんし です	
\\	ここは
\\	です。			
\\	ナンバー	ナンバー	ナンバー	
\\	その車のナンバーを覚えていますか。	その 車[くるま]のナンバーを 覚[おぼ]えていますか。	その くるま の なんばー を おぼえて います か	
\\	その 車[くるま]の
\\	を 覚[おぼ]えていますか。			
\\	日本酒	日本酒[にほんしゅ]	にほんしゅ	
\\	珍しい日本酒が手に入りました。	珍[めずら]しい 日本酒[にほんしゅ]が 手[て]に 入[はい]りました。	めずらしい にほんしゅ が て に はいりました	
\\	珍[めずら]しい
\\	が 手[て]に 入[はい]りました。			
\\	酔う	酔[よ]う	よう	
\\	みんなかなり酔っていたの。	みんなかなり 酔[よ]っていたの。	みんな かなり よって いた の	
\\	みんなかなり
\\	の。			
\\	保つ	保[たも]つ	たもつ	
\\	彼女は若さを保とうと必死だ。	彼女[かのじょ]は 若[わか]さを 保[たも]とうと 必死[ひっし]だ。	かのじょ は わかさ を たもとう と ひっし だ	
\\	彼女[かのじょ]は 若[わか]さを
\\	と 必死[ひっし]だ。			
\\	保険	保険[ほけん]	ほけん	
\\	あなたは保険に加入していますか。	あなたは 保険[ほけん]に 加入[かにゅう]していますか。	あなた は ほけん に かにゅう して います か	
\\	あなたは
\\	に 加入[かにゅう]していますか。			
\\	保証	保証[ほしょう]	ほしょう	
\\	この製品の保証期間は5年間となっております。	この 製品[せいひん]の 保証[ほしょう] 期間[きかん]は5 年間[ねん かん]となっております。	この せいひん の ほしょう きかん は 
\\	ねん かん と なって おります	
\\	この 製品[せいひん]の
\\	期間[きかん]は5 年間[ねん かん]となっております。			
\\	保証人	保証人[ほしょうにん]	ほしょうにん	
\\	彼が私の保証人になってくれました。	彼[かれ]が 私[わたし]の 保証人[ほしょうにん]になってくれました。	かれ が わたし の ほしょうにん に なって くれました	
\\	彼[かれ]が 私[わたし]の
\\	になってくれました。			
\\	とうとう	とうとう	とうとう	
\\	とうとう引っ越しの日がきました。	とうとう 引[ひ]っ 越[こ]しの 日[ひ]がきました。	とうとう ひっこし の ひ が きました	
\\	引[ひ]っ 越[こ]しの 日[ひ]がきました。			
\\	保存	保存[ほぞん]	ほぞん	
\\	データを30分おきに保存してください。	データを30 分[ぷん]おきに 保存[ほぞん]してください。	でーた を 
\\	ぷん おき に ほぞん して ください	
\\	データを30 分[ぷん]おきに
\\	してください。			
\\	存じる	存[ぞん]じる	ぞんじる	
\\	郵便局はどこかご存じですか。	郵便局[ゆうびんきょく]はどこかご 存[ぞん]じですか。	ゆうびんきょく は どこ か ごぞんじ です か	
\\	郵便局[ゆうびんきょく]はどこかご
\\	ですか。			
\\	注ぐ	注[そそ]ぐ	そそぐ	
\\	みんなのグラスにジュースを注いだよ。	みんなのグラスにジュースを 注[そそ]いだよ。	みんな の ぐらす に じゅーす を そそいだ よ	
\\	みんなのグラスにジュースを
\\	よ。			
\\	注	注[ちゅう]	ちゅう	
\\	詳しくは注を読んでください。	詳[くわ]しくは 注[ちゅう]を 読[よ]んでください。	くわしくは ちゅう を よんで ください	
\\	詳[くわ]しくは
\\	を 読[よ]んでください。			
\\	同意	同意[どうい]	どうい	
\\	彼の意見には同意できません。	彼[かれ]の 意見[いけん]には 同意[どうい]できません。	かれ の いけん に は どうい できません	
\\	彼[かれ]の 意見[いけん]には
\\	できません。			
\\	とんでもない	とんでもない	とんでもない	
\\	仕事中に帰るなんてとんでもない。	仕事中[しごとちゅう]に 帰[かえ]るなんてとんでもない。	しごとちゅう に かえる なんて とんでもない	
\\	仕事中[しごとちゅう]に 帰[かえ]るなんて
\\	無意味	無意味[むいみ]	むいみ	
\\	そんなことをしても無意味よ。	そんなことをしても 無意味[むいみ]よ。	そんな こと を して も むいみ よ	
\\	そんなことをしても
\\	よ。			
\\	不注意	不注意[ふちゅうい]	ふちゅうい	
\\	その事故は運転手の不注意が原因で起きたの。	その 事故[じこ]は 運転手[うんてんしゅ]の 不注意[ふちゅうい]が 原因[げんいん]で 起[お]きたの。	その じこ は うんてんしゅ の ふちゅうい が げんいん で おきた の	
\\	その 事故[じこ]は 運転手[うんてんしゅ]の
\\	が 原因[げんいん]で 起[お]きたの。			
\\	生意気	生意気[なまいき]	なまいき	
\\	あの子は生意気だと思います。	あの 子[こ]は 生意気[なまいき]だと 思[おも]います。	あの こ は なまいき だ と おもいます	
\\	あの 子[こ]は
\\	だと 思[おも]います。			
\\	確かめる	確[たし]かめる	たしかめる	
\\	母はお釣りを確かめたの。	母[はは]はお 釣[つ]りを 確[たし]かめたの。	はは は おつり を たしかめた の	
\\	母[はは]はお 釣[つ]りを
\\	の。			
\\	明確	明確[めいかく]	めいかく	
\\	彼女には明確な目標があるね。	彼女[かのじょ]には 明確[めいかく]な 目標[もくひょう]があるね。	かのじょ に は めいかく な もくひょう が ある ね	
\\	彼女[かのじょ]には
\\	な 目標[もくひょう]があるね。			
\\	にこやか	にこやか	にこやか	
\\	彼らはにこやかに挨拶を交わしたけどね。	彼[かれ]らはにこやかに 挨拶[あいさつ]を 交[か]わしたけどね。	かれら は にこやか に あいさつ を かわした けど ね	
\\	彼[かれ]らは
\\	に 挨拶[あいさつ]を 交[か]わしたけどね。			
\\	不確か	不確[ふたし]か	ふたしか	
\\	人の記憶は不確かよ。	人[ひと]の 記憶[きおく]は 不確[ふたし]かよ。	ひと の きおく は ふたしか よ	
\\	人[ひと]の 記憶[きおく]は
\\	よ。			
\\	認可	認可[にんか]	にんか	
\\	この薬はまだ認可されていません。	この 薬[くすり]はまだ 認可[にんか]されていません。	この くすり は まだ にんか されて いません	
\\	この 薬[くすり]はまだ
\\	されていません。			
\\	認める	認[みと]める	みとめる	
\\	父が彼女との結婚を認めてくれました。	父[ちち]が 彼女[かのじょ]との 結婚[けっこん]を 認[みと]めてくれました。	ちち が かのじょ と の けっこん を みとめて くれました	
\\	父[ちち]が 彼女[かのじょ]との 結婚[けっこん]を
\\	率	率[りつ]	りつ	
\\	その手術の成功率は90
\\	だそうです。	その 手術[しゅじゅつ]の 成功[せいこう] 率[りつ]は90
\\	だそうです。	その しゅじゅつ の せいこうりつ は 
\\	だ そう です	
\\	その 手術[しゅじゅつ]の 成功[せいこう]
\\	は90
\\	だそうです。			
\\	率直	率直[そっちょく]	そっちょく	
\\	率直なご意見ありがとうございました。	率直[そっちょく]なご 意見[いけん]ありがとうございました。	そっちょく な ごいけん ありがとう ございました	
\\	なご 意見[いけん]ありがとうございました。			
\\	能率	能率[のうりつ]	のうりつ	
\\	この方法だと能率がいいですね。	この 方法[ほうほう]だと 能率[のうりつ]がいいですね。	この ほうほう だ と のうりつ が いい です ね	
\\	この 方法[ほうほう]だと
\\	がいいですね。			
\\	ロッカー	ロッカー	ロッカー	
\\	荷物を駅のロッカーに入れた。	荷物[にもつ]を 駅[えき]のロッカーに 入[い]れた。	にもつ を えき の ろっかー に いれた	
\\	荷物[にもつ]を 駅[えき]の
\\	に 入[い]れた。			
\\	旅客機	旅客機[りょかくき]	りょかくき	
\\	旅客機が墜落したよ。	旅客機[りょかくき]が 墜落[ついらく]したよ。	りょかくき が ついらく した よ	
\\	が 墜落[ついらく]したよ。			
\\	不器用	不器用[ぶきよう]	ぶきよう	
\\	妹は不器用で、料理も苦手なの。	妹[いもうと]は 不器用[ぶきよう]で、 料理[りょうり]も 苦手[にがて]なの。	いもうと は ぶきよう で りょうり も にがて な の	
\\	妹[いもうと]は
\\	で、 料理[りょうり]も 苦手[にがて]なの。			
\\	道具	道具[どうぐ]	どうぐ	
\\	道具は全部揃っていますか。	道具[どうぐ]は 全部揃[ぜんぶ そろ]っていますか。	どうぐ は ぜんぶ そろって います か	
\\	は 全部揃[ぜんぶ そろ]っていますか。			
\\	予備	予備[よび]	よび	
\\	旅行には予備の靴を持って行きます。	旅行[りょこう]には 予備[よび]の 靴[くつ]を 持[も]って 行[い]きます。	りょこう に は よび の くつ を もって いきます	
\\	旅行[りょこう]には
\\	の 靴[くつ]を 持[も]って 行[い]きます。			
\\	説く	説[と]く	とく	
\\	彼は非暴力を説きました。	彼[かれ]は 非暴力[ひぼうりょく]を 説[と]きました。	かれ は ひぼうりょく を ときました	
\\	彼[かれ]は 非暴力[ひぼうりょく]を
\\	プライド	プライド	プライド	
\\	彼はプライドが高い人です。	彼[かれ]はプライドが 高[たか]い 人[ひと]です。	かれ は ぷらいど が たかい ひと です	
\\	彼[かれ]は
\\	が 高[たか]い 人[ひと]です。			
\\	不公平	不公平[ふこうへい]	ふこうへい	
\\	彼のやり方は不公平です。	彼[かれ]のやり 方[かた]は 不公平[ふこうへい]です。	かれ の やりかた は ふこうへい です	
\\	彼[かれ]のやり 方[かた]は
\\	です。			
\\	遊園地	遊園地[ゆうえんち]	ゆうえんち	
\\	この遊園地のチケットは3000円です。	この 遊園地[ゆうえんち]のチケットは 3000円[さんぜんえん]です。	この ゆうえんち の ちけっと は さんぜんえん です	
\\	この
\\	のチケットは 3000円[さんぜんえん]です。			
\\	保育園	保育園[ほいくえん]	ほいくえん	
\\	娘を保育園に迎えに行きます。	娘[むすめ]を 保育園[ほいくえん]に 迎[むか]えに 行[い]きます。	むすめ を ほいくえん に むかえ に いきます	
\\	娘[むすめ]を
\\	に 迎[むか]えに 行[い]きます。			
\\	飛び出す	飛[と]び 出[だ]す	とびだす	
\\	彼は道路に飛び出したの。	彼[かれ]は 道路[どうろ]に 飛[と]び 出[だ]したの。	かれ は どうろ に とびだした の	
\\	彼[かれ]は 道路[どうろ]に
\\	の。			
\\	飛ばす	飛[と]ばす	とばす	
\\	子供が紙飛行機を飛ばしています。	子供[こども]が 紙飛行機[かみひこうき]を 飛[と]ばしています。	こども が かみひこうき を とばして います	
\\	子供[こども]が 紙飛行機[かみひこうき]を
\\	ママ	ママ	ママ	
\\	ママに聞いてみよう。	ママに 聞[き]いてみよう。	まま に きいて みよう	
\\	に 聞[き]いてみよう。			
\\	飛び上がる	飛[と]び 上[あ]がる	とびあがる	
\\	大きな音にびっくりして飛び上がりました。	大[おお]きな 音[おと]にびっくりして 飛[と]び 上[あ]がりました。	おおき な おと に びっくり して とびあがりました	
\\	大[おお]きな 音[おと]にびっくりして
\\	飛び下りる	飛[と]び 下[お]りる	とびおりる	
\\	猫が屋根から飛び下りました。	猫[ねこ]が 屋根[やね]から 飛[と]び 下[お]りました。	ねこ が やね から とびおりました	
\\	猫[ねこ]が 屋根[やね]から
\\	飛び込む	飛[と]び 込[こ]む	とびこむ	
\\	カエルが池に飛び込んだね。	カエルが 池[いけ]に 飛[と]び 込[こ]んだね。	かえる が いけ に とびこんだ ね	
\\	カエルが 池[いけ]に
\\	ね。			
\\	飛行	飛行[ひこう]	ひこう	
\\	このフライトの飛行時間は約3時間です。	このフライトの 飛行[ひこう] 時間[じかん]は 約3時間[やく 
\\	じかん]です。	この ふらいと の ひこう じかん は やく 
\\	じかん です	
\\	このフライトの
\\	時間[じかん]は 約3時間[やく 
\\	じかん]です。			
\\	船長	船長[せんちょう]	せんちょう	
\\	私がこの船の船長です。	私[わたし]がこの 船[ふね]の 船長[せんちょう]です。	わたし が この ふね の せんちょう です	
\\	私[わたし]がこの 船[ふね]の
\\	です。			
\\	よける	よける	よける	
\\	椅子の荷物をよけて座ったけど。	椅子[いす]の 荷物[にもつ]をよけて 座[すわ]ったけど。	いす の にもつ を よけて すわった けど	
\\	椅子[いす]の 荷物[にもつ]を
\\	座[すわ]ったけど。			
\\	風船	風船[ふうせん]	ふうせん	
\\	子供が風船を膨らませているね。	子供[こども]が 風船[ふうせん]を 膨[ふく]らませているね。	こども が ふうせん を ふくらませて いる ね	
\\	子供[こども]が
\\	を 膨[ふく]らませているね。			
\\	半島	半島[はんとう]	はんとう	
\\	台風がその半島を通過したの。	台風[たいふう]がその 半島[はんとう]を 通過[つうか]したの。	たいふう が その はんとう を つうか した の	
\\	台風[たいふう]がその
\\	を 通過[つうか]したの。			
\\	島	島[とう]	とう	
\\	私たちはハワイのマウイ島に旅行したの。	私[わたし]たちはハワイのマウイ 島[とう]に 旅行[りょこう]したの。	わたしたち は はわい の まういとう に りょこう した の	
\\	私[わたし]たちはハワイのマウイ
\\	に 旅行[りょこう]したの。			
\\	不完全	不完全[ふかんぜん]	ふかんぜん	
\\	このデータはまだ不完全ね。	このデータはまだ 不完全[ふかんぜん]ね。	この でーた は まだ ふかんぜん ね	
\\	このデータはまだ
\\	ね。			
\\	達成	達成[たっせい]	たっせい	
\\	彼は今月の売上目標を達成した。	彼[かれ]は 今月[こんげつ]の 売上目標[うりあげ もくひょう]を 達成[たっせい]した。	かれ は こんげつ の うりあげ もくひょう を たっせい した	
\\	彼[かれ]は 今月[こんげつ]の 売上目標[うりあげ もくひょう]を
\\	した。			
\\	成り立つ	成[な]り 立[た]つ	なりたつ	
\\	この島は観光で成り立っています。	この 島[しま]は 観光[かんこう]で 成[な]り 立[た]っています。	この しま は かんこう で なりたって います	
\\	この 島[しま]は 観光[かんこう]で
\\	います。			
\\	バッジ	バッジ	バッジ	
\\	彼は弁護士バッジを付けていたよ。	彼[かれ]は 弁護士[べんごし]バッジを 付[つ]けていたよ。	かれ は べんごし ばっじ を つけて いた よ	
\\	彼[かれ]は 弁護士[べんごし]
\\	を 付[つ]けていたよ。			
\\	未成年	未成年[みせいねん]	みせいねん	
\\	未成年はお酒を飲めません。	未成年[みせいねん]はお 酒[さけ]を 飲[の]めません。	みせいねん は おさけ を のめません	
\\	はお 酒[さけ]を 飲[の]めません。			
\\	敗れる	敗[やぶ]れる	やぶれる	
\\	私のチームは1回戦で敗れたよ。	私[わたし]のチームは1 回戦[かいせん]で 敗[やぶ]れたよ。	わたし の ちーむ は 
\\	かいせん で やぶれた よ	
\\	私[わたし]のチームは1 回戦[かいせん]で
\\	よ。			
\\	野原	野原[のはら]	のはら	
\\	私たちは野原で花をつんだの。	私[わたし]たちは 野原[のはら]で 花[はな]をつんだの。	わたしたち は のはら で はな を つんだ の	
\\	私[わたし]たちは
\\	で 花[はな]をつんだの。			
\\	原	原[はら]	はら	
\\	クローバーの原でピクニックをしたよ。	クローバーの 原[はら]でピクニックをしたよ。	くろーばー の はら で ぴくにっく を した よ	
\\	クローバーの
\\	でピクニックをしたよ。			
\\	要因	要因[よういん]	よういん	
\\	私たちはがんの要因を研究しています。	私[わたし]たちはがんの 要因[よういん]を 研究[けんきゅう]しています。	わたしたち は がん の よういん を けんきゅう して います	
\\	私[わたし]たちはがんの
\\	を 研究[けんきゅう]しています。			
\\	ピン	ピン	ピン	
\\	彼女はいつも髪をピンで留めているね。	彼女[かのじょ]はいつも 髪[かみ]をピンで 留[と]めているね。	かのじょ は いつも かみ を ぴん で とめて いる ね	
\\	彼女[かのじょ]はいつも 髪[かみ]を
\\	で 留[と]めているね。			
\\	因る	因[よ]る	よる	
\\	彼の病気は過労に因るものです。	彼[かれ]の 病気[びょうき]は 過労[かろう]に 因[よ]るものです。	かれ の びょうき は かろう に よる もの です	
\\	彼[かれ]の 病気[びょうき]は 過労[かろう]に
\\	ものです。			
\\	物資	物資[ぶっし]	ぶっし	
\\	被災地に物資を送ったよ。	被災地[ひさいち]に 物資[ぶっし]を 送[おく]ったよ。	ひさいち に ぶっし を おくった よ	
\\	被災地[ひさいち]に
\\	を 送[おく]ったよ。			
\\	願い	願[ねが]い	ねがい	
\\	世界の平和が私たちの願いです。	世界[せかい]の 平和[へいわ]が 私[わたし]たちの 願[ねが]いです。	せかい の へいわ が わたし たち の ねがい です	
\\	世界[せかい]の 平和[へいわ]が 私[わたし]たちの
\\	です。			
\\	不正	不正[ふせい]	ふせい	
\\	試験で不正が見つかったよ。	試験[しけん]で 不正[ふせい]が 見[み]つかったよ。	しけん で ふせい が みつかった よ	
\\	試験[しけん]で
\\	が 見[み]つかったよ。			
\\	大正	大正[たいしょう]	たいしょう	
\\	祖母は大正生まれです。	祖母[そぼ]は 大正[たいしょう] 生[う]まれです。	そぼ は たいしょう うまれ です	
\\	祖母[そぼ]は
\\	生[う]まれです。			
\\	まあ	まあ	まあ	
\\	まあこれでいいだろう。	まあこれでいいだろう。	まあこれでいいだろう。	
\\	これでいいだろう。			
\\	正に	正[まさ]に	まさに	
\\	彼は正に英雄ね。	彼[かれ]は 正[まさ]に 英雄[えいゆう]ね。	かれ は まさに えいゆう ね	
\\	彼[かれ]は
\\	英雄[えいゆう]ね。			
\\	常に	常[つね]に	つねに	
\\	彼は常に姿勢がいい。	彼[かれ]は 常[つね]に 姿勢[しせい]がいい。	かれ は つねに しせい が いい	
\\	彼[かれ]は
\\	姿勢[しせい]がいい。			
\\	日常	日常[にちじょう]	にちじょう	
\\	音楽は私の日常の一部です。	音楽[おんがく]は 私[わたし]の 日常[にちじょう]の 一部[いちぶ]です。	おんがく は わたし の にちじょう の いちぶ です	
\\	音楽[おんがく]は 私[わたし]の
\\	の 一部[いちぶ]です。			
\\	知識	知識[ちしき]	ちしき	
\\	私は旅行から多くの知識を得た。	私[わたし]は 旅行[りょこう]から 多[おお]くの 知識[ちしき]を 得[え]た。	わたし は りょこう から おおく の ちしき を えた	
\\	私[わたし]は 旅行[りょこう]から 多[おお]くの
\\	を 得[え]た。			
\\	認識	認識[にんしき]	にんしき	
\\	その件は終わったと認識しています。	その 件[けん]は 終[お]わったと 認識[にんしき]しています。	その けん は おわった と にんしき して います	
\\	その 件[けん]は 終[お]わったと
\\	しています。			
\\	無意識	無意識[むいしき]	むいしき	
\\	私は無意識に彼を傷つけてしまった。	私[わたし]は 無意識[むいしき]に 彼[かれ]を 傷[きず]つけてしまった。	わたし は むいしき に かれ を きずつけて しまった	
\\	私[わたし]は
\\	に 彼[かれ]を 傷[きず]つけてしまった。			
\\	むなしい	むなしい	むなしい	
\\	愛がなければ人生はむなしいよ。	愛[あい]がなければ 人生[じんせい]はむなしいよ。	あい が なければ じんせい は むなしい よ	
\\	愛[あい]がなければ 人生[じんせい]は
\\	よ。			
\\	非難	非難[ひなん]	ひなん	
\\	国民は総理大臣を非難しているね。	国民[こくみん]は 総理大臣[そうり だいじん]を 非難[ひなん]しているね。	こくみん は そうり だいじん を ひなん して いる ね	
\\	国民[こくみん]は 総理大臣[そうり だいじん]を
\\	しているね。			
\\	非常	非常[ひじょう]	ひじょう	
\\	非常事態です。	非常[ひじょう] 事態[じたい]です。	ひじょう じたい です	
\\	事態[じたい]です。			
\\	非常識	非常識[ひじょうしき]	ひじょうしき	
\\	彼は非常識な時間に電話してきたの。	彼[かれ]は 非常識[ひじょうしき]な 時間[じかん]に 電話[でんわ]してきたの。	かれ は ひじょうしき な じかん に でんわ して きた の	
\\	彼[かれ]は
\\	な 時間[じかん]に 電話[でんわ]してきたの。			
\\	非常口	非常口[ひじょうぐち]	ひじょうぐち	
\\	ビルの非常口を確認したの。	ビルの 非常口[ひじょうぐち]を 確認[かくにん]したの。	びる の ひじょうぐち を かくにん した の	
\\	ビルの
\\	を 確認[かくにん]したの。			
\\	調子	調子[ちょうし]	ちょうし	
\\	体の調子がとても良いです。	体[からだ]の 調子[ちょうし]がとても 良[い]いです。	からだ の ちょうし が とても いい です	
\\	体[からだ]の
\\	がとても 良[い]いです。			
\\	ついで	ついで	ついで	
\\	ついでだから彼も呼ぼうよ。	ついでだから 彼[かれ]も 呼[よ]ぼうよ。	ついで だから かれ も よぼうよ	
\\	だから 彼[かれ]も 呼[よ]ぼうよ。			
\\	調和	調和[ちょうわ]	ちょうわ	
\\	彼女は自然と調和した暮らし方をしているの。	彼女[かのじょ]は 自然[しぜん]と 調和[ちょうわ]した 暮[く]らし 方[かた]をしているの。	かのじょ は しぜん と ちょうわ した くらし かた を して いる の	
\\	彼女[かのじょ]は 自然[しぜん]と
\\	した 暮[く]らし 方[かた]をしているの。			
\\	体調	体調[たいちょう]	たいちょう	
\\	今日は体調が悪いです。	今日[きょう]は 体調[たいちょう]が 悪[わる]いです。	きょう は たいちょう が わるい です	
\\	今日[きょう]は
\\	が 悪[わる]いです。			
\\	単調	単調[たんちょう]	たんちょう	
\\	この曲は単調でつまらないな。	この 曲[きょく]は 単調[たんちょう]でつまらないな。	この きょく は たんちょう で つまらない な	
\\	この 曲[きょく]は
\\	でつまらないな。			
\\	調味料	調味料[ちょうみりょう]	ちょうみりょう	
\\	塩、コショウはよく使われる調味料です。	塩[しお]、コショウはよく 使[つか]われる 調味料[ちょうみりょう]です。	しお こしょう は よく つかわれる ちょうみりょう です	
\\	塩[しお]、コショウはよく 使[つか]われる
\\	です。			
\\	調整	調整[ちょうせい]	ちょうせい	
\\	今、スケジュールの調整をしています。	今[いま]、スケジュールの 調整[ちょうせい]をしています。	いま すけじゅーる の ちょうせい を して います	
\\	今[いま]、スケジュールの
\\	をしています。			
\\	バケツ	バケツ	バケツ	
\\	バケツに水を汲んできてください。	バケツに 水[みず]を 汲[く]んできてください。	ばけつ に みず を くんで きて ください	
\\	に 水[みず]を 汲[く]んできてください。			
\\	整える	整[ととの]える	ととのえる	
\\	彼はスピーチの前に服装を整えた。	彼[かれ]はスピーチの 前[まえ]に 服装[ふくそう]を 整[ととの]えた。	かれ は すぴーち の まえ に ふくそう を ととのえた	
\\	彼[かれ]はスピーチの 前[まえ]に 服装[ふくそう]を
\\	整う	整[ととの]う	ととのう	
\\	パーティーの準備が整いました。	パーティーの 準備[じゅんび]が 整[ととの]いました。	ぱーてぃー の じゅんび が ととのいました	
\\	パーティーの 準備[じゅんび]が
\\	調節	調節[ちょうせつ]	ちょうせつ	
\\	ここで部屋の温度が調節できます。	ここで 部屋[へや]の 温度[おんど]が 調節[ちょうせつ]できます。	ここ で へや の おんど が ちょうせつ できます	
\\	ここで 部屋[へや]の 温度[おんど]が
\\	できます。			
\\	提出	提出[ていしゅつ]	ていしゅつ	
\\	課題は7月5日までに提出してください。	課題[かだい]は7 月5日[がつ 
\\	か]までに 提出[ていしゅつ]してください。	かだい は 
\\	がつ 
\\	か まで に ていしゅつ して ください	
\\	課題[かだい]は7 月5日[がつ 
\\	か]までに
\\	してください。			
\\	前提	前提[ぜんてい]	ぜんてい	
\\	交渉が成立するという前提で話を進めます。	交渉[こうしょう]が 成立[せいりつ]するという 前提[ぜんてい]で 話[はなし]を 進[すす]めます。	こうしょう が せいりつ する と いう ぜんてい で はなし を すすめます	
\\	交渉[こうしょう]が 成立[せいりつ]するという
\\	で 話[はなし]を 進[すす]めます。			
\\	答案	答案[とうあん]	とうあん	
\\	今から答案を集めます。	今[いま]から 答案[とうあん]を 集[あつ]めます。	いま から とうあん を あつめます	
\\	今[いま]から
\\	を 集[あつ]めます。			
\\	はっと	はっと	はっと	
\\	彼女の美しさにはっとしたよ。	彼女[かのじょ]の 美[うつく]しさにはっとしたよ。	かのじょ の うつくしさ に はっと した よ	
\\	彼女[かのじょ]の 美[うつく]しさに
\\	したよ。			
\\	投票	投票[とうひょう]	とうひょう	
\\	私は朝早く投票を済ませました。	私[わたし]は 朝早[あさ はや]く 投票[とうひょう]を 済[す]ませました。	わたし は あさ はやく とうひょう を すませました	
\\	私[わたし]は 朝早[あさ はや]く
\\	を 済[す]ませました。			
\\	目標	目標[もくひょう]	もくひょう	
\\	私は父を目標にしています。	私[わたし]は 父[ちち]を 目標[もくひょう]にしています。	わたし は ちち を もくひょう に して います	
\\	私[わたし]は 父[ちち]を
\\	にしています。			
\\	標準	標準[ひょうじゅん]	ひょうじゅん	
\\	ニュースでは標準語が使われるの。	ニュースでは 標準[ひょうじゅん] 語[ご]が 使[つか]われるの。	にゅーす で は ひょうじゅんご が つかわれる の	
\\	ニュースでは
\\	語[ご]が 使[つか]われるの。			
\\	連れる	連[つ]れる	つれる	
\\	親が君を一度連れて来いって言うんだ。	親[おや]が 君[きみ]を 一度[いちど] 連[つ]れて 来[こ]いって 言[い]うんだ。	おや が きみ を いちど つれて こい って いう ん だ	
\\	親[おや]が 君[きみ]を 一度[いちど]
\\	来[こ]いって 言[い]うんだ。			
\\	連日	連日[れんじつ]	れんじつ	
\\	展覧会は連日賑わいました。	展覧会[てんらんかい]は 連日[れんじつ] 賑[にぎ]わいました。	てんらんかい は れんじつ にぎわいました	
\\	展覧会[てんらんかい]は
\\	賑[にぎ]わいました。			
\\	まずい	まずい	まずい	
\\	彼女の運転がまずくて、はらはらした。	彼女[かのじょ]の 運転[うんてん]がまずくて、はらはらした。	かのじょ の うんてん が まずくて はらはら した	
\\	彼女[かのじょ]の 運転[うんてん]が
\\	、はらはらした。			
\\	連休	連休[れんきゅう]	れんきゅう	
\\	今度の連休は実家に帰ります。	今度[こんど]の 連休[れんきゅう]は 実家[じっか]に 帰[かえ]ります。	こんど の れんきゅう は じっか に かえります	
\\	今度[こんど]の
\\	は 実家[じっか]に 帰[かえ]ります。			
\\	連れ	連[つ]れ	つれ	
\\	彼女は私の連れです。	彼女[かのじょ]は 私[わたし]の 連[つ]れです。	かのじょ は わたし の つれ です	
\\	彼女[かのじょ]は 私[わたし]の
\\	です。			
\\	連絡	連絡[れんらく]	れんらく	
\\	仕事が終わったら連絡します。	仕事[しごと]が 終[お]わったら 連絡[れんらく]します。	しごと が おわったら れんらく します	
\\	仕事[しごと]が 終[お]わったら
\\	します。			
\\	手続き	手続[てつづ]き	てつづき	
\\	入国手続きが終わりました。	入国[にゅうこく] 手続[てつづ]きが 終[お]わりました。	にゅうこく てつづき が おわりました	
\\	入国[にゅうこく]
\\	が 終[お]わりました。			
\\	連続	連続[れんぞく]	れんぞく	
\\	彼女の人生は苦労の連続でした。	彼女[かのじょ]の 人生[じんせい]は 苦労[くろう]の 連続[れんぞく]でした。	かのじょ の じんせい は くろう の れんぞく でした	
\\	彼女[かのじょ]の 人生[じんせい]は 苦労[くろう]の
\\	でした。			
\\	もしも	もしも	もしも	
\\	もしも彼女と結婚できたらどんなに嬉しいだろう。	もしも 彼女[かのじょ]と 結婚[けっこん]できたらどんなに 嬉[うれ]しいだろう。	もしも かのじょ と けっこん できたら どんなに うれしい だろう	
\\	彼女[かのじょ]と 結婚[けっこん]できたらどんなに 嬉[うれ]しいだろう。			
\\	続々	続々[ぞくぞく]	ぞくぞく	
\\	お客さんが続々とやって来ましたよ。	お 客[きゃく]さんが 続々[ぞくぞく]とやって 来[き]ましたよ。	おきゃくさん が ぞくぞく と やってきました よ	
\\	お 客[きゃく]さんが
\\	とやって 来[き]ましたよ。			
\\	続き	続[つづ]き	つづき	
\\	話の続きは電話でしましょう。	話[はなし]の 続[つづ]きは 電話[でんわ]でしましょう。	はなし の つづき は でんわ で しましょう	
\\	話[はなし]の
\\	は 電話[でんわ]でしましょう。			
\\	長続き	長続[ながつづ]き	ながつづき	
\\	彼は仕事が長続きしません。	彼[かれ]は 仕事[しごと]が 長続[ながつづ]きしません。	かれ は しごと が ながつづき しません	
\\	彼[かれ]は 仕事[しごと]が
\\	しません。			
\\	相場	相場[そうば]	そうば	
\\	この辺りの家賃の相場はいくらですか。	この 辺[あた]りの 家賃[やちん]の 相場[そうば]はいくらですか。	この あたり の やちん の そうば は いくら です か	
\\	この 辺[あた]りの 家賃[やちん]の
\\	はいくらですか。			
\\	相当	相当[そうとう]	そうとう	
\\	1ポンドは454グラムに相当します。	1[いち]ポンドは454グラムに 相当[そうとう]します。	いちぽんど は 
\\	ぐらむ に そうとう します	
\\	1[いち]ポンドは454グラムに
\\	します。			
\\	やかましい	やかましい	やかましい	
\\	スピーカーの音がやかましいな。	スピーカーの 音[おと]がやかましいな。	すぴーかー の おと が やかましい な	
\\	スピーカーの 音[おと]が
\\	な。			
\\	相応しい	相応[ふさわ]しい	ふさわしい	
\\	その場に相応しい服装で来てください。	その 場[ば]に 相応[ふさわ]しい 服装[ふくそう]で 来[き]てください。	その ば に ふさわしい ふくそう で きて ください	
\\	その 場[ば]に
\\	服装[ふくそう]で 来[き]てください。			
\\	対談	対談[たいだん]	たいだん	
\\	雑誌にその女優の対談が載っていたよ。	雑誌[ざっし]にその 女優[じょゆう]の 対談[たいだん]が 載[の]っていたよ。	ざっし に その じょゆう の たいだん が のって いた よ	
\\	雑誌[ざっし]にその 女優[じょゆう]の
\\	が 載[の]っていたよ。			
\\	録画	録画[ろくが]	ろくが	
\\	好きな番組を録画したの。	好[す]きな 番組[ばんぐみ]を 録画[ろくが]したの。	すき な ばんぐみ を ろくが した の	
\\	好[す]きな 番組[ばんぐみ]を
\\	したの。			
\\	録音テープ	録音[ろくおん]テープ	ろくおんテープ	
\\	インタビューはこの録音テープに入っています。	インタビューはこの 録音[ろくおん]テープに 入[はい]っています。	いんたびゅー は この ろくおんてーぷ に はいって います	
\\	インタビューはこの
\\	に 入[はい]っています。			
\\	登場	登場[とうじょう]	とうじょう	
\\	彼女の登場で会場は盛り上がったね。	彼女[かのじょ]の 登場[とうじょう]で 会場[かいじょう]は 盛[も]り 上[あ]がったね。	かのじょ の とうじょう で かいじょう は もりあがった ね	
\\	彼女[かのじょ]の
\\	で 会場[かいじょう]は 盛[も]り 上[あ]がったね。			
\\	登録	登録[とうろく]	とうろく	
\\	心理学の授業に登録しましたか。	心理学[しんりがく]の 授業[じゅぎょう]に 登録[とうろく]しましたか。	しんりがく の じゅぎょう に とうろく しました か	
\\	心理学[しんりがく]の 授業[じゅぎょう]に
\\	しましたか。			
\\	バット	バット	バット	
\\	このバットは金属で出来ています。	このバットは 金属[きんぞく]で 出来[でき]ています。	この ばっと は きんぞく で できて います	
\\	この
\\	は 金属[きんぞく]で 出来[でき]ています。			
\\	登山	登山[とざん]	とざん	
\\	夏休みには家族で登山をします。	夏休[なつやす]みには 家族[かぞく]で 登山[とざん]をします。	なつやすみ に は かぞく で とざん を します	
\\	夏休[なつやす]みには 家族[かぞく]で
\\	をします。			
\\	登校	登校[とうこう]	とうこう	
\\	生徒たちは朝8時ごろ登校します。	生徒[せいと]たちは 朝8時[あさ 
\\	じ]ごろ 登校[とうこう]します。	せいとたち は あさ 
\\	じごろ とうこう します	
\\	生徒[せいと]たちは 朝8時[あさ 
\\	じ]ごろ
\\	します。			
\\	山登り	山登[やまのぼ]り	やまのぼり	
\\	明日は友達と山登りに行きます。	明日[あした]は 友達[ともだち]と 山登[やまのぼ]りに 行[い]きます。	あした は ともだち と やまのぼり に いきます	
\\	明日[あした]は 友達[ともだち]と
\\	に 行[い]きます。			
\\	無関心	無関心[むかんしん]	むかんしん	
\\	彼女は政治に無関心です。	彼女[かのじょ]は 政治[せいじ]に 無関心[むかんしん]です。	かのじょ は せいじ に むかんしん です	
\\	彼女[かのじょ]は 政治[せいじ]に
\\	です。			
\\	無関係	無関係[むかんけい]	むかんけい	
\\	彼はこの事件と無関係です。	彼[かれ]はこの 事件[じけん]と 無関係[むかんけい]です。	かれ は この じけん と むかんけい です	
\\	彼[かれ]はこの 事件[じけん]と
\\	です。			
\\	マナー	マナー	マナー	
\\	食事のマナーを守りましょう。	食事[しょくじ]のマナーを 守[まも]りましょう。	しょくじ の まなー を まもりましょう	
\\	食事[しょくじ]の
\\	を 守[まも]りましょう。			
\\	不況	不況[ふきょう]	ふきょう	
\\	不況の影響で仕事が少ないね。	不況[ふきょう]の 影響[えいきょう]で 仕事[しごと]が 少[すく]ないね。	ふきょう の えいきょう で しごと が すくない ね	
\\	の 影響[えいきょう]で 仕事[しごと]が 少[すく]ないね。			
\\	態度	態度[たいど]	たいど	
\\	あの男の態度にみんなあきれてたよ。	あの 男[おとこ]の 態度[たいど]にみんなあきれてたよ。	あの おとこ の たいど に みんな あきれて た よ	
\\	あの 男[おとこ]の
\\	にみんなあきれてたよ。			
\\	明治	明治[めいじ]	めいじ	
\\	祖父は明治の生まれです。	祖父[そふ]は 明治[めいじ]の 生[う]まれです。	そふ は めいじ の うまれ です	
\\	祖父[そふ]は
\\	の 生[う]まれです。			
\\	府	府[ふ]	ふ	
\\	彼は大阪府に住んでいます。	彼[かれ]は 大阪[おおさか] 府[ふ]に 住[す]んでいます。	かれ は おおさかふ に すんで います	
\\	彼[かれ]は 大阪[おおさか]
\\	に 住[す]んでいます。			
\\	府立	府立[ふりつ]	ふりつ	
\\	彼女は府立大学に通っています。	彼女[かのじょ]は 府立[ふりつ] 大学[だいがく]に 通[かよ]っています。	かのじょ は ふりつ だいがく に かよって います	
\\	彼女[かのじょ]は
\\	大学[だいがく]に 通[かよ]っています。			
\\	ばらばら	ばらばら	ばらばら	
\\	生徒たちはばらばらに帰宅したね。	生徒[せいと]たちはばらばらに 帰宅[きたく]したね。	せいとたち は ばらばら に きたく した ね	
\\	生徒[せいと]たちは
\\	に 帰宅[きたく]したね。			
\\	野党	野党[やとう]	やとう	
\\	野党が与党を厳しく非難していましたね。	野党[やとう]が 与党[よとう]を 厳[きび]しく 非難[ひなん]していましたね。	やとう が よとう を きびしく ひなん して いました ね	
\\	が 与党[よとう]を 厳[きび]しく 非難[ひなん]していましたね。			
\\	当選	当選[とうせん]	とうせん	
\\	彼は選挙に当選しました。	彼[かれ]は 選挙[せんきょ]に 当選[とうせん]しました。	かれ は せんきょ に とうせん しました	
\\	彼[かれ]は 選挙[せんきょ]に
\\	しました。			
\\	入選	入選[にゅうせん]	にゅうせん	
\\	彼の絵がコンクールに入選した。	彼[かれ]の 絵[え]がコンクールに 入選[にゅうせん]した。	かれ の え が こんくーる に にゅうせん した	
\\	彼[かれ]の 絵[え]がコンクールに
\\	した。			
\\	選択	選択[せんたく]	せんたく	
\\	この5種類から選択できます。	この5 種類[しゅるい]から 選択[せんたく]できます。	この 
\\	しゅるい から せんたく できます	
\\	この5 種類[しゅるい]から
\\	できます。			
\\	天候	天候[てんこう]	てんこう	
\\	ここは天候の変化が激しいですね。	ここは 天候[てんこう]の 変化[へんか]が 激[はげ]しいですね。	ここ は てんこう の へんか が はげしい です ね	
\\	ここは
\\	の 変化[へんか]が 激[はげ]しいですね。			
\\	手首	手首[てくび]	てくび	
\\	手首の関節をひねっちゃった。	手首[てくび]の 関節[かんせつ]をひねっちゃった。	てくび の かんせつ を ひねっちゃった	
\\	の 関節[かんせつ]をひねっちゃった。			
\\	プラットホーム	プラットホーム	プラットホーム	
\\	プラットホームで電車が来るのを待ったの。	プラットホームで 電車[でんしゃ]が 来[く]るのを 待[ま]ったの。	ぷらっとほーむ で でんしゃ が くる の を まった の	
\\	で 電車[でんしゃ]が 来[く]るのを 待[ま]ったの。			
\\	脳	脳[のう]	のう	
\\	そのクイズ番組は脳を刺激するね。	そのクイズ 番組[ばんぐみ]は 脳[のう]を 刺激[しげき]するね。	その くいずばんぐみ は のう を しげき する ね	
\\	そのクイズ 番組[ばんぐみ]は
\\	を 刺激[しげき]するね。			
\\	悩む	悩[なや]む	なやむ	
\\	彼は受験のことで悩んでいます。	彼[かれ]は 受験[じゅけん]のことで 悩[なや]んでいます。	かれ は じゅけん の こと で なやんで います	
\\	彼[かれ]は 受験[じゅけん]のことで
\\	います。			
\\	悩み	悩[なや]み	なやみ	
\\	彼は大きな悩みを抱えていました。	彼[かれ]は 大[おお]きな 悩[なや]みを 抱[かか]えていました。	かれ は おおき な なやみ を かかえて いました	
\\	彼[かれ]は 大[おお]きな
\\	を 抱[かか]えていました。			
\\	先頭	先頭[せんとう]	せんとう	
\\	先頭の人はプラカードを持ってください。	先頭[せんとう]の 人[ひと]はプラカードを 持[も]ってください。	せんとう の ひと は ぷらかーど を もって ください	
\\	の 人[ひと]はプラカードを 持[も]ってください。			
\\	命じる	命[めい]じる	めいじる	
\\	急に出張を命じられました。	急[きゅう]に 出張[しゅっちょう]を 命[めい]じられました。	きゅう に しゅっちょう を めいじられました	
\\	急[きゅう]に 出張[しゅっちょう]を
\\	まぐろ	まぐろ	まぐろ	
\\	まぐろの刺身を食べました。	まぐろの 刺身[さしみ]を 食[た]べました。	まぐろ の さしみ を たべました	
\\	の 刺身[さしみ]を 食[た]べました。			
\\	領土	領土[りょうど]	りょうど	
\\	ここから先は隣の国の領土です。	ここから 先[さき]は 隣[となり]の 国[くに]の 領土[りょうど]です。	ここ から さき は となり の くに の りょうど です	
\\	ここから 先[さき]は 隣[となり]の 国[くに]の
\\	です。			
\\	領事館	領事館[りょうじかん]	りょうじかん	
\\	彼はアメリカ領事館に出かけたよ。	彼[かれ]はアメリカ 領事館[りょうじかん]に 出[で]かけたよ。	かれ は あめりか りょうじかん に でかけた よ	
\\	彼[かれ]はアメリカ
\\	に 出[で]かけたよ。			
\\	領収書	領収書[りょうしゅうしょ]	りょうしゅうしょ	
\\	領収書をください。	領収書[りょうしゅうしょ]をください。	りょうしゅうしょ を ください	
\\	をください。			
\\	取り組む	取[と]り 組[く]む	とりくむ	
\\	彼女はスペイン語に取り組んでいます。	彼女[かのじょ]はスペイン 語[ご]に 取[と]り 組[く]んでいます。	かのじょ は すぺいんご に とりくんで います	
\\	彼女[かのじょ]はスペイン 語[ご]に
\\	任せる	任[まか]せる	まかせる	
\\	この仕事は君に任せる。	この 仕事[しごと]は 君[きみ]に 任[まか]せる。	この しごと は きみ に まかせる	
\\	この 仕事[しごと]は 君[きみ]に
\\	ろくに	ろくに	ろくに	
\\	ゆうべはろくに寝ていない。	ゆうべはろくに 寝[ね]ていない。	ゆうべ は ろくに ねて いない	
\\	ゆうべは
\\	寝[ね]ていない。			
\\	担任	担任[たんにん]	たんにん	
\\	私は3年生のクラスを担任しています。	私[わたし]は3 年生[ねんせい]のクラスを 担任[たんにん]しています。	わたし は 
\\	ねんせい の くらす を たんにん して います	
\\	私[わたし]は3 年生[ねんせい]のクラスを
\\	しています。			
\\	無責任	無責任[むせきにん]	むせきにん	
\\	無責任な行動は許されません。	無責任[むせきにん]な 行動[こうどう]は 許[ゆる]されません。	むせきにん な こうどう は ゆるされません	
\\	な 行動[こうどう]は 許[ゆる]されません。			
\\	転勤	転勤[てんきん]	てんきん	
\\	彼は大阪に転勤しました。	彼[かれ]は 大阪[おおさか]に 転勤[てんきん]しました。	かれ は おおさか に てんきん しました	
\\	彼[かれ]は 大阪[おおさか]に
\\	しました。			
\\	勤め先	勤[つと]め 先[さき]	つとめさき	
\\	私の勤め先にご連絡ください。	私[わたし]の 勤[つと]め 先[さき]にご 連絡[れんらく]ください。	わたし の つとめさき に ごれんらく ください	
\\	私[わたし]の
\\	にご 連絡[れんらく]ください。			
\\	勤め	勤[つと]め	つとめ	
\\	来月で勤めをやめます。	来月[らいげつ]で 勤[つと]めをやめます。	らいげつ で つとめ を やめます	
\\	来月[らいげつ]で
\\	をやめます。			
\\	務める	務[つと]める	つとめる	
\\	私が司会を務めます。	私[わたし]が 司会[しかい]を 務[つと]めます。	わたし が しかい を つとめます	
\\	私[わたし]が 司会[しかい]を
\\	つまむ	つまむ	つまむ	
\\	お菓子でもつまんでください。	お 菓子[かし]でもつまんでください。	おかし で も つまんで ください	
\\	お 菓子[かし]でも
\\	ください。			
\\	任務	任務[にんむ]	にんむ	
\\	彼は最後まで自分の任務を果たした。	彼[かれ]は 最後[さいご]まで 自分[じぶん]の 任務[にんむ]を 果[は]たした。	かれ は さいご まで じぶん の にんむ を はたした	
\\	彼[かれ]は 最後[さいご]まで 自分[じぶん]の
\\	を 果[は]たした。			
\\	務め	務[つと]め	つとめ	
\\	人々を守るのが私の務めです。	人々[ひとびと]を 守[まも]るのが 私[わたし]の 務[つと]めです。	ひとびと を まもる の が わたし の つとめ です	
\\	人々[ひとびと]を 守[まも]るのが 私[わたし]の
\\	です。			
\\	雇う	雇[やと]う	やとう	
\\	新しい社員を雇いました。	新[あたら]しい 社員[しゃいん]を 雇[やと]いました。	あたらしい しゃいん を やといました	
\\	新[あたら]しい 社員[しゃいん]を
\\	募集	募集[ぼしゅう]	ぼしゅう	
\\	私の会社で社員を募集しています。	私[わたし]の 会社[かいしゃ]で 社員[しゃいん]を 募集[ぼしゅう]しています。	わたし の かいしゃ で しゃいん を ぼしゅう して います	
\\	私[わたし]の 会社[かいしゃ]で 社員[しゃいん]を
\\	しています。			
\\	採る	採[と]る	とる	
\\	この山ではきのこが採れますよ	この 山[やま]ではきのこが 採[と]れますよ	この やま で は きのこ が とれます よ	
\\	この 山[やま]ではきのこが
\\	よ			
\\	ボーイ	ボーイ	ボーイ	
\\	ボーイに荷物を運んでもらった。	ボーイに 荷物[にもつ]を 運[はこ]んでもらった。	ぼーい に にもつ を はこんで もらった	
\\	に 荷物[にもつ]を 運[はこ]んでもらった。			
\\	就く	就[つ]く	つく	
\\	今年から新しい仕事に就きます。	今年[ことし]から 新[あたら]しい 仕事[しごと]に 就[つ]きます。	ことし から あたらしい しごと に つきます	
\\	今年[ことし]から 新[あたら]しい 仕事[しごと]に
\\	退職	退職[たいしょく]	たいしょく	
\\	私は今月一杯で退職します。	私[わたし]は 今月一杯[こんげつ いっぱい]で 退職[たいしょく]します。	わたし は こんげつ いっぱい で たいしょく します	
\\	私[わたし]は 今月一杯[こんげつ いっぱい]で
\\	します。			
\\	退学	退学[たいがく]	たいがく	
\\	弟は高校を退学しました。	弟[おとうと]は 高校[こうこう]を 退学[たいがく]しました。	おとうと は こうこう を たいがく しました	
\\	弟[おとうと]は 高校[こうこう]を
\\	しました。			
\\	早退	早退[そうたい]	そうたい	
\\	具合が悪かったので仕事を早退しました。	具合[ぐあい]が 悪[わる]かったので 仕事[しごと]を 早退[そうたい]しました。	ぐあい が わるかった の で しごと を そうたい しました	
\\	具合[ぐあい]が 悪[わる]かったので 仕事[しごと]を
\\	しました。			
\\	理屈	理屈[りくつ]	りくつ	
\\	彼には理屈が通じない。	彼[かれ]には 理屈[りくつ]が 通[つう]じない。	かれ に は りくつ が つうじない	
\\	彼[かれ]には
\\	が 通[つう]じない。			
\\	ホット	ホット	ホット	
\\	コーヒーをホットでください。	コーヒーをホットでください。	こーひー を ほっと で ください	
\\	コーヒーを
\\	でください。			
\\	退屈	退屈[たいくつ]	たいくつ	
\\	校長の退屈な話が続いたんだよ。	校長[こうちょう]の 退屈[たいくつ]な 話[はなし]が 続[つづ]いたんだよ。	こうちょう の たいくつ な はなし が つづいた ん だ よ	
\\	校長[こうちょう]の
\\	な 話[はなし]が 続[つづ]いたんだよ。			
\\	用件	用件[ようけん]	ようけん	
\\	用件をメモしておきました。	用件[ようけん]をメモしておきました。	ようけん を めも して おきました	
\\	をメモしておきました。			
\\	参る	参[まい]る	まいる	
\\	さあ、参りましょうか。	さあ、 参[まい]りましょうか。	さあ まいりましょう か	
\\	さあ、
\\	か。			
\\	惨め	惨[みじ]め	みじめ	
\\	彼は惨めな気持ちになったの。	彼[かれ]は 惨[みじ]めな 気持[きも]ちになったの。	かれ は みじめ な きもち に なった の	
\\	彼[かれ]は
\\	な 気持[きも]ちになったの。			
\\	追加	追加[ついか]	ついか	
\\	オーダーを追加しました。	オーダーを 追加[ついか]しました。	おーだー を ついか しました	
\\	オーダーを
\\	しました。			
\\	まし	まし	まし	
\\	これでもないよりましだ。	これでもないよりましだ。	これでもないよりましだ。	
\\	これでもないより
\\	だ。			
\\	比率	比率[ひりつ]	ひりつ	
\\	業界は女性の比率が低い。	
\\	業界[あいてぃーぎょうかい]は 女性[じょせい]の 比率[ひりつ]が 低[ひく]い。	あいてぃーぎょうかい は じょせい の ひりつ が ひくい	
\\	業界[あいてぃーぎょうかい]は 女性[じょせい]の
\\	が 低[ひく]い。			
\\	対比	対比[たいひ]	たいひ	
\\	この絵は赤と黒の対比が美しいですね。	この 絵[え]は 赤[あか]と 黒[くろ]の 対比[たいひ]が 美[うつく]しいですね。	この え は あか と くろ の たいひ が うつくしい です ね	
\\	この 絵[え]は 赤[あか]と 黒[くろ]の
\\	が 美[うつく]しいですね。			
\\	見比べる	見比[みくら]べる	みくらべる	
\\	彼女は二人の顔を見比べたの。	彼女[かのじょ]は 二人[ふたり]の 顔[かお]を 見比[みくら]べたの。	かのじょ は ふたり の かお を みくらべた の	
\\	彼女[かのじょ]は 二人[ふたり]の 顔[かお]を
\\	の。			
\\	比較的	比較的[ひかくてき]	ひかくてき	
\\	今年は比較的景気がいい。	今年[ことし]は 比較的[ひかくてき] 景気[けいき]がいい。	ことし は ひかくてき けいき が いい	
\\	今年[ことし]は
\\	景気[けいき]がいい。			
\\	比較	比較[ひかく]	ひかく	
\\	去年の売り上げと比較しましょう。	去年[きょねん]の 売[う]り 上[あ]げと 比較[ひかく]しましょう。	きょねん の うりあげ と ひかく しましょう	
\\	去年[きょねん]の 売[う]り 上[あ]げと
\\	しましょう。			
\\	判決	判決[はんけつ]	はんけつ	
\\	判決が下された。	判決[はんけつ]が 下[くだ]された。	はんけつ が くだされた	
\\	が 下[くだ]された。			
\\	ぱっと	ぱっと	ぱっと	
\\	彼女はぱっと目を開いた。	彼女[かのじょ]はぱっと 目[め]を 開[ひら]いた。	かのじょ は ぱっと め を ひらいた	
\\	彼女[かのじょ]は
\\	目[め]を 開[ひら]いた。			
\\	判子	判子[はんこ]	はんこ	
\\	ここに判子を押してください。	ここに 判子[はんこ]を 押[お]してください。	ここ に はんこ を おして ください	
\\	ここに
\\	を 押[お]してください。			
\\	評判	評判[ひょうばん]	ひょうばん	
\\	評判の良いレストランに行きました。	評判[ひょうばん]の 良[い]いレストランに 行[い]きました。	ひょうばん の いい れすとらん に いきました	
\\	の 良[い]いレストランに 行[い]きました。			
\\	批評	批評[ひひょう]	ひひょう	
\\	その映画はよい批評を得ているんだ。	その 映画[えいが]はよい 批評[ひひょう]を 得[え]ているんだ。	その えいが は よい ひひょう を えて いる ん だ	
\\	その 映画[えいが]はよい
\\	を 得[え]ているんだ。			
\\	反感	反感[はんかん]	はんかん	
\\	俺たちは彼の言動に反感を持ったね。	俺[おれ]たちは 彼[かれ]の 言動[げんどう]に 反感[はんかん]を 持[も]ったね。	おれたち は かれ の げんどう に はんかん を もった ね	
\\	俺[おれ]たちは 彼[かれ]の 言動[げんどう]に
\\	を 持[も]ったね。			
\\	予想	予想[よそう]	よそう	
\\	私の予想が当たった。	私[わたし]の 予想[よそう]が 当[あ]たった。	わたし の よそう が あたった	
\\	私[わたし]の
\\	が 当[あ]たった。			
\\	つなぐ	つなぐ	つなぐ	
\\	インターネットは世界の人々をつなぎますね。	インターネットは 世界[せかい]の 人々[ひとびと]をつなぎますね。	いんたーねっと は せかい の ひとびと を つなぎます ね	
\\	インターネットは 世界[せかい]の 人々[ひとびと]を
\\	ね。			
\\	理想	理想[りそう]	りそう	
\\	彼は高い理想を持っているの。	彼[かれ]は 高[たか]い 理想[りそう]を 持[も]っているの。	かれ は たかい りそう を もって いる の	
\\	彼[かれ]は 高[たか]い
\\	を 持[も]っているの。			
\\	想像	想像[そうぞう]	そうぞう	
\\	そんなことは想像できないよ。	そんなことは 想像[そうぞう]できないよ。	そんな こと は そうぞう できない よ	
\\	そんなことは
\\	できないよ。			
\\	象	象[ぞう]	ぞう	
\\	象に乗ってみたいです。	象[ぞう]に 乗[の]ってみたいです。	ぞう に のって みたい です	
\\	に 乗[の]ってみたいです。			
\\	抽象的	抽象的[ちゅうしょうてき]	ちゅうしょうてき	
\\	彼は抽象的な絵が好きだね。	彼[かれ]は 抽象的[ちゅうしょうてき]な 絵[え]が 好[す]きだね。	かれ は ちゅうしょうてき な え が すき だ ね	
\\	彼[かれ]は
\\	な 絵[え]が 好[す]きだね。			
\\	保障	保障[ほしょう]	ほしょう	
\\	私があなたの安全を保障します。	私[わたし]があなたの 安全[あんぜん]を 保障[ほしょう]します。	わたし が あなた の あんぜん を ほしょう します	
\\	私[わたし]があなたの 安全[あんぜん]を
\\	します。			
\\	ひょっとしたら	ひょっとしたら	ひょっとしたら	
\\	ひょっとしたら彼はそのことを知らないのかもしれない。	ひょっとしたら 彼[かれ]はそのことを 知[し]らないのかもしれない。	ひょっとしたら かれ は その こと を しらない の かも しれない	
\\	彼[かれ]はそのことを 知[し]らないのかもしれない。			
\\	負傷	負傷[ふしょう]	ふしょう	
\\	その事故で多くの人が負傷したの。	その 事故[じこ]で 多[おお]くの 人[ひと]が 負傷[ふしょう]したの。	その じこ で おおく の ひと が ふしょう した の	
\\	その 事故[じこ]で 多[おお]くの 人[ひと]が
\\	したの。			
\\	変換	変換[へんかん]	へんかん	
\\	ひらがなをカタカナに変換しました。	ひらがなをカタカナに 変換[へんかん]しました。	ひらがな を かたかな に へんかん しました	
\\	ひらがなをカタカナに
\\	しました。			
\\	取り替える	取[と]り 替[か]える	とりかえる	
\\	シーツを取り替えました。	シーツを 取[と]り 替[か]えました。	しーつ を とりかえました	
\\	シーツを
\\	立て替える	立[た]て 替[か]える	たてかえる	
\\	お金がないなら私が立て替えておきます。	お 金[かね]がないなら 私[わたし]が 立[た]て 替[か]えておきます。	おかね が ない なら わたし が たてかえて おきます	
\\	お 金[かね]がないなら 私[わたし]が
\\	天災	天災[てんさい]	てんさい	
\\	天災を防ぐことはできません。	天災[てんさい]を 防[ふせ]ぐことはできません。	てんさい を ふせぐ こと は できません	
\\	を 防[ふせ]ぐことはできません。			
\\	有害	有害[ゆうがい]	ゆうがい	
\\	この材料は有害だよ。	この 材料[ざいりょう]は 有害[ゆうがい]だよ。	この ざいりょう は ゆうがい だ よ	
\\	この 材料[ざいりょう]は
\\	だよ。			
\\	ひとりでに	ひとりでに	ひとりでに	
\\	ドアがひとりでに閉まったな。	ドアがひとりでに 閉[し]まったな。	どあ が ひとりでに しまった な	
\\	ドアが
\\	閉[し]まったな。			
\\	無害	無害[むがい]	むがい	
\\	この農薬は人には無害です。	この 農薬[のうやく]は 人[ひと]には 無害[むがい]です。	この のうやく は ひと に は むがい です	
\\	この 農薬[のうやく]は 人[ひと]には
\\	です。			
\\	被害	被害[ひがい]	ひがい	
\\	彼の家は台風の被害にあったの。	彼[かれ]の 家[いえ]は 台風[たいふう]の 被害[ひがい]にあったの。	かれ の いえ は たいふう の ひがい に あった の	
\\	彼[かれ]の 家[いえ]は 台風[たいふう]の
\\	にあったの。			
\\	破る	破[やぶ]る	やぶる	
\\	彼は強敵を見事に破ったね。	彼[かれ]は 強敵[きょうてき]を 見事[みごと]に 破[やぶ]ったね。	かれ は きょうてき を みごと に やぶった ね	
\\	彼[かれ]は 強敵[きょうてき]を 見事[みごと]に
\\	ね。			
\\	破壊	破壊[はかい]	はかい	
\\	自然の破壊が進んでいるのよ。	自然[しぜん]の 破壊[はかい]が 進[すす]んでいるのよ。	しぜん の はかい が すすんで いる の よ	
\\	自然[しぜん]の
\\	が 進[すす]んでいるのよ。			
\\	助かる	助[たす]かる	たすかる	
\\	彼はシートベルトをしていたので助かったんだ。	彼[かれ]はシートベルトをしていたので 助[たす]かったんだ。	かれ は しーとべると を して いた の で たすかった ん だ	
\\	彼[かれ]はシートベルトをしていたので
\\	んだ。			
\\	ひねる	ひねる	ひねる	
\\	彼は水道の蛇口をひねったの。	彼[かれ]は 水道[すいどう]の 蛇口[じゃぐち]をひねったの。	かれ は すいどう の じゃぐち を ひねった の	
\\	彼[かれ]は 水道[すいどう]の 蛇口[じゃぐち]を
\\	の。			
\\	派手	派手[はで]	はで	
\\	雪道で派手に転んでしまったの。	雪道[ゆきみち]で 派手[はで]に 転[ころ]んでしまったの。	ゆきみち で はで に ころんで しまった の	
\\	雪道[ゆきみち]で
\\	に 転[ころ]んでしまったの。			
\\	派出所	派出所[はしゅつじょ]	はしゅつじょ	
\\	派出所にだれもいないな。	派出所[はしゅつじょ]にだれもいないな。	はしゅつじょ に だれ も いない な	
\\	にだれもいないな。			
\\	派遣	派遣[はけん]	はけん	
\\	彼はイギリスに派遣されました。	彼[かれ]はイギリスに 派遣[はけん]されました。	かれ は いぎりす に はけん されました	
\\	彼[かれ]はイギリスに
\\	されました。			
\\	犯人	犯人[はんにん]	はんにん	
\\	あいつが犯人です。	あいつが 犯人[はんにん]です。	あいつ が はんにん です	
\\	あいつが
\\	です。			
\\	犯罪	犯罪[はんざい]	はんざい	
\\	最近犯罪が減っています。	最近[さいきん] 犯罪[はんざい]が 減[へ]っています。	さいきん はんざい が へって います	
\\	最近[さいきん]
\\	が 減[へ]っています。			
\\	ひび	ひび	ひび	
\\	窓ガラスにひびが入ったよ。	窓[まど]ガラスにひびが 入[はい]ったよ。	まどがらす に ひび が はいった よ	
\\	窓[まど]ガラスに
\\	が 入[はい]ったよ。			
\\	罪	罪[つみ]	つみ	
\\	彼の罪は重いな。	彼[かれ]の 罪[つみ]は 重[おも]いな。	かれ の つみ は おもい な	
\\	彼[かれ]の
\\	は 重[おも]いな。			
\\	盗難	盗難[とうなん]	とうなん	
\\	昨夜、自転車の盗難にあったよ。	昨夜[さくや]、 自転車[じてんしゃ]の 盗難[とうなん]にあったよ。	さくや じてんしゃ の とうなん に あった よ	
\\	昨夜[さくや]、 自転車[じてんしゃ]の
\\	にあったよ。			
\\	捕まる	捕[つか]まる	つかまる	
\\	彼女はついに捕まりました。	彼女[かのじょ]はついに 捕[つか]まりました。	かのじょ は ついに つかまりました	
\\	彼女[かのじょ]はついに
\\	捕まえる	捕[つか]まえる	つかまえる	
\\	少年は網でその蝶を捕まえた。	少年[しょうねん]は 網[あみ]でその 蝶[ちょう]を 捕[つか]まえた。	しょうねん は あみ で その ちょう を つかまえた	
\\	少年[しょうねん]は 網[あみ]でその 蝶[ちょう]を
\\	逃げ出す	逃[に]げ 出[だ]す	にげだす	
\\	トラがおりから逃げ出しました。	トラがおりから 逃[に]げ 出[だ]しました。	とら が おり から にげだしました	
\\	トラがおりから
\\	逃がす	逃[に]がす	にがす	
\\	釣った魚を逃がしました。	釣[つ]った 魚[さかな]を 逃[に]がしました。	つった さかな を にがしました	
\\	釣[つ]った 魚[さかな]を
\\	やっつける	やっつける	やっつける	
\\	主人公が悪者をやっつけた。	主人公[しゅじんこう]が 悪者[わるもの]をやっつけた。	しゅじんこう が わるもの を やっつけた	
\\	主人公[しゅじんこう]が 悪者[わるもの]を
\\	逃げ道	逃[に]げ 道[みち]	にげみち	
\\	失敗した時の逃げ道を考えたの。	失敗[しっぱい]した 時[とき]の 逃[に]げ 道[みち]を 考[かんが]えたの。	しっぱい した とき の にげみち を かんがえた の	
\\	失敗[しっぱい]した 時[とき]の
\\	を 考[かんが]えたの。			
\\	戦う	戦[たたか]う	たたかう	
\\	彼は最後まで戦ったよ。	彼[かれ]は 最後[さいご]まで 戦[たたか]ったよ。	かれ は さいご まで たたかった よ	
\\	彼[かれ]は 最後[さいご]まで
\\	よ。			
\\	挑戦	挑戦[ちょうせん]	ちょうせん	
\\	彼は新しいことに挑戦している。	彼[かれ]は 新[あたら]しいことに 挑戦[ちょうせん]している。	かれ は あたらしい こと に ちょうせん して いる	
\\	彼[かれ]は 新[あたら]しいことに
\\	している。			
\\	敗戦	敗戦[はいせん]	はいせん	
\\	敗戦の原因は何だろう。	敗戦[はいせん]の 原因[げんいん]は 何[なん]だろう。	はいせん の げんいん は なん だろう	
\\	の 原因[げんいん]は 何[なん]だろう。			
\\	戦い	戦[たたか]い	たたかい	
\\	長い戦いが終わった。	長[なが]い 戦[たたか]いが 終[お]わった。	ながい たたかい が おわった	
\\	長[なが]い
\\	が 終[お]わった。			
\\	レントゲン	レントゲン	レントゲン	
\\	病院でレントゲンを撮ったよ。	病院[びょういん]でレントゲンを 撮[と]ったよ。	びょういん で れんとげん を とった よ	
\\	病院[びょういん]で
\\	を 撮[と]ったよ。			
\\	大戦	大戦[たいせん]	たいせん	
\\	大戦で多くの人が亡くなりました。	大戦[たいせん]で 多[おお]くの 人[ひと]が 亡[な]くなりました。	たいせん で おおく の ひと が なくなりました	
\\	で 多[おお]くの 人[ひと]が 亡[な]くなりました。			
\\	混ぜる	混[ま]ぜる	まぜる	
\\	カレーにヨーグルトを入れて、よく混ぜてください。	カレーにヨーグルトを 入[い]れて、よく 混[ま]ぜてください。	かれー に よーぐると を いれて よく まぜて ください	
\\	カレーにヨーグルトを 入[い]れて、よく
\\	ください。			
\\	混ざる	混[ま]ざる	まざる	
\\	水と油は混ざりません。	水[みず]と 油[あぶら]は 混[ま]ざりません。	みず と あぶら は まざりません	
\\	水[みず]と 油[あぶら]は
\\	混じる	混[ま]じる	まじる	
\\	電話の声に雑音が混じっていたの。	電話[でんわ]の 声[こえ]に 雑音[ざつおん]が 混[ま]じっていたの。	でんわ の こえ に ざつおん が まじって いた の	
\\	電話[でんわ]の 声[こえ]に 雑音[ざつおん]が
\\	の。			
\\	乱れる	乱[みだ]れる	みだれる	
\\	風で髪が乱れたね。	風[かぜ]で 髪[かみ]が 乱[みだ]れたね。	かぜ で かみ が みだれた ね	
\\	風[かぜ]で 髪[かみ]が
\\	ね。			
\\	パンツ	パンツ	パンツ	
\\	彼女はスカートよりパンツが似合うね。	彼女[かのじょ]はスカートよりパンツが 似合[にあ]うね。	かのじょ は すかーと より ぱんつ が にあう ね	
\\	彼女[かのじょ]はスカートより
\\	が 似合[にあ]うね。			
\\	乱す	乱[みだ]す	みだす	
\\	彼は風紀を乱している。	彼[かれ]は 風紀[ふうき]を 乱[みだ]している。	かれ は ふうき を みだして いる	
\\	彼[かれ]は 風紀[ふうき]を
\\	統合	統合[とうごう]	とうごう	
\\	三つの町が統合されて新しい市が誕生したのよ。	三[みっ]つの 町[まち]が 統合[とうごう]されて 新[あたら]しい 市[し]が 誕生[たんじょう]したのよ。	みっつ の まち が とうごう されて あたらしい し が たんじょう した の よ	
\\	三[みっ]つの 町[まち]が
\\	されて 新[あたら]しい 市[し]が 誕生[たんじょう]したのよ。			
\\	統一	統一[とういつ]	とういつ	
\\	文字の大きさは統一してください。	文字[もじ]の 大[おお]きさは 統一[とういつ]してください。	もじ の おおきさ は とういつ して ください	
\\	文字[もじ]の 大[おお]きさは
\\	してください。			
\\	統計	統計[とうけい]	とうけい	
\\	これは昨年の売り上げの統計です。	これは 昨年[さくねん]の 売[う]り 上[あ]げの 統計[とうけい]です。	これ は さくねん の うりあげ の とうけい です	
\\	これは 昨年[さくねん]の 売[う]り 上[あ]げの
\\	です。			
\\	総会	総会[そうかい]	そうかい	
\\	来月の総会は東京で行われます。	来月[らいげつ]の 総会[そうかい]は 東京[とうきょう]で 行[おこな]われます。	らいげつ の そうかい は とうきょう で おこなわれます	
\\	来月[らいげつ]の
\\	は 東京[とうきょう]で 行[おこな]われます。			
\\	むける	むける	むける	
\\	日焼けで皮がむけた。	日焼[ひや]けで 皮[かわ]がむけた。	ひやけ で かわ が むけた	
\\	日焼[ひや]けで 皮[かわ]が
\\	総合	総合[そうごう]	そうごう	
\\	全員の意見を総合してみましょう。	全員[ぜんいん]の 意見[いけん]を 総合[そうごう]してみましょう。	ぜんいん の いけん を そうごう して みましょう	
\\	全員[ぜんいん]の 意見[いけん]を
\\	してみましょう。			
\\	総理	総理[そうり]	そうり	
\\	総理はヨーロッパを訪問中です。	総理[そうり]はヨーロッパを 訪問中[ほうもんちゅう]です。	そうり は よーろっぱ を ほうもんちゅう です	
\\	はヨーロッパを 訪問中[ほうもんちゅう]です。			
\\	総数	総数[そうすう]	そうすう	
\\	参加者の総数は705人でした。	参加者[さんかしゃ]の 総数[そうすう]は705 人[にん]でした。	さんかしゃ の そうすう は 
\\	にん でした	
\\	参加者[さんかしゃ]の
\\	は705 人[にん]でした。			
\\	捜査	捜査[そうさ]	そうさ	
\\	その殺人事件の捜査は2年間続きました。	その 殺人事件[さつじん じけん]の 捜査[そうさ]は2 年間続[ねんかん つづ]きました。	その さつじん じけん の そうさ は 
\\	ねんかん つづきました	
\\	その 殺人事件[さつじん じけん]の
\\	は2 年間続[ねんかん つづ]きました。			
\\	湖	湖[みずうみ]	みずうみ	
\\	湖でスケートをした。	湖[みずうみ]でスケートをした。	みずうみ で すけーと を した	
\\	でスケートをした。			
\\	深める	深[ふか]める	ふかめる	
\\	彼女は異文化に対する理解を深めたね。	彼女[かのじょ]は 異文化[いぶんか]に 対[たい]する 理解[りかい]を 深[ふか]めたね。	かのじょ は いぶんか に たいする りかい を ふかめた ね	
\\	彼女[かのじょ]は 異文化[いぶんか]に 対[たい]する 理解[りかい]を
\\	ね。			
\\	ぶつぶつ	ぶつぶつ	ぶつぶつ	
\\	彼はぶつぶつと独り言を言ったんだ。	彼[かれ]はぶつぶつと 独[ひと]り 言[ごと]を 言[い]ったんだ。	かれ は ぶつぶつ と ひとりごと を いった ん だ	
\\	彼[かれ]は
\\	と 独[ひと]り 言[ごと]を 言[い]ったんだ。			
\\	深まる	深[ふか]まる	ふかまる	
\\	二人の愛はますます深まっているね。	二人[ふたり]の 愛[あい]はますます 深[ふか]まっているね。	ふたり の あい は ますます ふかまっている ね	
\\	二人[ふたり]の 愛[あい]はますます
\\	ね。			
\\	注意深い	注意深[ちゅういぶか]い	ちゅういぶかい	
\\	彼は注意深い人です。	彼[かれ]は 注意深[ちゅういぶか]い 人[ひと]です。	かれ は ちゅういぶかい ひと です	
\\	彼[かれ]は
\\	人[ひと]です。			
\\	段落	段落[だんらく]	だんらく	
\\	次の段落を読んでください。	次[つぎ]の 段落[だんらく]を 読[よ]んでください。	つぎ の だんらく を よんで ください	
\\	次[つぎ]の
\\	を 読[よ]んでください。			
\\	見落とす	見落[みお]とす	みおとす	
\\	間違いを見落としたの。	間違[まちが]いを 見落[みお]としたの。	まちがい を みおとした の	
\\	間違[まちが]いを
\\	の。			
\\	落第	落第[らくだい]	らくだい	
\\	このテストに失敗したら落第です。	このテストに 失敗[しっぱい]したら 落第[らくだい]です。	この てすと に しっぱい したら らくだい です	
\\	このテストに 失敗[しっぱい]したら
\\	です。			
\\	ぶるぶる	ぶるぶる	ぶるぶる	
\\	彼はぶるぶる震えていたよ。	彼[かれ]はぶるぶる 震[ふる]えていたよ。	かれ は ぶるぶる ふるえて いた よ	
\\	彼[かれ]は
\\	震[ふる]えていたよ。			
\\	波	波[なみ]	なみ	
\\	今日の海は波が穏やかです。	今日[きょう]の 海[うみ]は 波[なみ]が 穏[おだ]やかです。	きょう の うみ は なみ が おだやか です	
\\	今日[きょう]の 海[うみ]は
\\	が 穏[おだ]やかです。			
\\	流れ	流[なが]れ	ながれ	
\\	川の上流は流れが速いよ。	川[かわ]の 上流[じょうりゅう]は 流[なが]れが 速[はや]いよ。	かわ の じょうりゅう は ながれ が はやい よ	
\\	川[かわ]の 上流[じょうりゅう]は
\\	が 速[はや]いよ。			
\\	流通	流通[りゅうつう]	りゅうつう	
\\	今日は流通の仕組みを勉強しましょう。	今日[きょう]は 流通[りゅうつう]の 仕組[しく]みを 勉強[べんきょう]しましょう。	きょう は りゅうつう の しくみ を べんきょう しましょう	
\\	今日[きょう]は
\\	の 仕組[しく]みを 勉強[べんきょう]しましょう。			
\\	流す	流[なが]す	ながす	
\\	彼女は涙を流したんだ。	彼女[かのじょ]は 涙[なみだ]を 流[なが]したんだ。	かのじょ は なみだ を ながした ん だ	
\\	彼女[かのじょ]は 涙[なみだ]を
\\	んだ。			
\\	中流	中流[ちゅうりゅう]	ちゅうりゅう	
\\	川の中流あたりにその村はあるよ。	川[かわ]の 中流[ちゅうりゅう]あたりにその 村[むら]はあるよ。	かわ の ちゅうりゅう あたり に その むら は ある よ	
\\	川[かわ]の
\\	あたりにその 村[むら]はあるよ。			
\\	ボリューム	ボリューム	ボリューム	
\\	ラジオのボリュームを上げてください。	ラジオのボリュームを 上[あ]げてください。	らじお の ぼりゅーむ を あげて ください	
\\	ラジオの
\\	を 上[あ]げてください。			
\\	二流	二流[にりゅう]	にりゅう	
\\	彼はまだまだ二流の芸人だね。	彼[かれ]はまだまだ 二流[にりゅう]の 芸人[げいにん]だね。	かれ は まだまだ にりゅう の げいにん だ ね	
\\	彼[かれ]はまだまだ
\\	の 芸人[げいにん]だね。			
\\	流行	流行[はやり]	はやり	
\\	この服は今の流行です。	この 服[ふく]は 今[いま]の 流行[はやり]です。	この ふく は いま の はやり です	
\\	この 服[ふく]は 今[いま]の
\\	です。			
\\	流行	流行[りゅうこう]	りゅうこう	
\\	このスタイルは今年の流行です。	このスタイルは 今年[ことし]の 流行[りゅうこう]です。	この すたいる は ことし の りゅうこう です	
\\	このスタイルは 今年[ことし]の
\\	です。			
\\	洗面	洗面[せんめん]	せんめん	
\\	洗面用具を忘れた。	洗面[せんめん] 用具[ようぐ]を 忘[わす]れた。	せんめん ようぐ を わすれた	
\\	用具[ようぐ]を 忘[わす]れた。			
\\	手洗い	手洗[てあら]い	てあらい	
\\	風邪をひかないように手洗いとうがいをしましょう。	風邪[かぜ]をひかないように 手洗[てあら]いとうがいをしましょう。	かぜ を ひかない よう に てあらい と うがい を しましょう	
\\	風邪[かぜ]をひかないように
\\	とうがいをしましょう。			
\\	洗面器	洗面器[せんめんき]	せんめんき	
\\	洗面器でハンカチを洗ったの。	洗面器[せんめんき]でハンカチを 洗[あら]ったの。	せんめんき で はんかち を あらった の	
\\	でハンカチを 洗[あら]ったの。			
\\	ちり	ちり	ちり	
\\	ちりも積もれば山となる。	ちりも 積[つ]もれば 山[やま]となる。	ちり も つもれば やま と なる 。	
\\	も 積[つ]もれば 山[やま]となる。			
\\	沈没	沈没[ちんぼつ]	ちんぼつ	
\\	船は沈没しました。	船[ふね]は 沈没[ちんぼつ]しました。	ふね は ちんぼつ しました	
\\	船[ふね]は
\\	しました。			
\\	冷え込む	冷[ひ]え 込[こ]む	ひえこむ	
\\	明日から急に冷え込むそうです。	明日[あす]から 急[きゅう]に 冷[ひ]え 込[こ]むそうです。	あす から きゅう に ひえこむ そう です	
\\	明日[あす]から 急[きゅう]に
\\	そうです。			
\\	冷やかす	冷[ひ]やかす	ひやかす	
\\	友達はそのカップルを冷やかしたんだ。	友達[ともだち]はそのカップルを 冷[ひ]やかしたんだ。	ともだち は その かっぷる を ひやかした ん だ	
\\	友達[ともだち]はそのカップルを
\\	んだ。			
\\	冷凍	冷凍[れいとう]	れいとう	
\\	残ったカレーを冷凍しました。	残[のこ]ったカレーを 冷凍[れいとう]しました。	のこった かれー を れいとう しました	
\\	残[のこ]ったカレーを
\\	しました。			
\\	内臓	内臓[ないぞう]	ないぞう	
\\	来週、内臓を検査します。	来週[らいしゅう]、 内臓[ないぞう]を 検査[けんさ]します。	らいしゅう ないぞう を けんさ します	
\\	来週[らいしゅう]、
\\	を 検査[けんさ]します。			
\\	はげる	はげる	はげる	
\\	壁のペンキがはげてきたな。	壁[かべ]のペンキがはげてきたな。	かべ の ぺんき が はげて きた な	
\\	壁[かべ]のペンキが
\\	きたな。			
\\	適用	適用[てきよう]	てきよう	
\\	この場合、保険が適用されますか。	この 場合[ばあい]、 保険[ほけん]が 適用[てきよう]されますか。	この ばあい ほけん が てきよう されます か	
\\	この 場合[ばあい]、 保険[ほけん]が
\\	されますか。			
\\	適切	適切[てきせつ]	てきせつ	
\\	彼は適切な言葉で説明してくれました。	彼[かれ]は 適切[てきせつ]な 言葉[ことば]で 説明[せつめい]してくれました。	かれ は てきせつ な ことば で せつめい して くれました	
\\	彼[かれ]は
\\	な 言葉[ことば]で 説明[せつめい]してくれました。			
\\	適応	適応[てきおう]	てきおう	
\\	彼はどんな環境にも適応できます。	彼[かれ]はどんな 環境[かんきょう]にも 適応[てきおう]できます。	かれ は どんな かんきょう に も てきおう できます	
\\	彼[かれ]はどんな 環境[かんきょう]にも
\\	できます。			
\\	適する	適[てき]する	てきする	
\\	彼はこの仕事に適しています。	彼[かれ]はこの 仕事[しごと]に 適[てき]しています。	かれ は この しごと に てきしています	
\\	彼[かれ]はこの 仕事[しごと]に
\\	適当	適当[てきとう]	てきとう	
\\	その質問の適当な答えが見つかりません。	その 質問[しつもん]の 適当[てきとう]な 答[こた]えが 見[み]つかりません。	その しつもん の てきとう な こたえ が みつかりません	
\\	その 質問[しつもん]の
\\	な 答[こた]えが 見[み]つかりません。			
\\	パパ	パパ	パパ	
\\	私のパパは36歳です。	私[わたし]のパパは36 歳[さい]です。	わたし の ぱぱ は 
\\	さい です	
\\	私[わたし]の
\\	は36 歳[さい]です。			
\\	適度	適度[てきど]	てきど	
\\	健康のため、適度な運動が必要です。	健康[けんこう]のため、 適度[てきど]な 運動[うんどう]が 必要[ひつよう]です。	けんこう の ため てきど な うんどう が ひつよう です	
\\	健康[けんこう]のため、
\\	な 運動[うんどう]が 必要[ひつよう]です。			
\\	染める	染[そ]める	そめる	
\\	髪を赤に染めてみた。	髪[かみ]を 赤[あか]に 染[そ]めてみた。	かみ を あか に そめて みた	
\\	髪[かみ]を 赤[あか]に
\\	廃止	廃止[はいし]	はいし	
\\	その制度は廃止されました。	その 制度[せいど]は 廃止[はいし]されました。	その せいど は はいし されました	
\\	その 制度[せいど]は
\\	されました。			
\\	風景	風景[ふうけい]	ふうけい	
\\	私は山の風景が好きです。	私[わたし]は 山[やま]の 風景[ふうけい]が 好[す]きです。	わたし は やま の ふうけい が すき です	
\\	私[わたし]は 山[やま]の
\\	が 好[す]きです。			
\\	不景気	不景気[ふけいき]	ふけいき	
\\	今、あの国は不景気らしいよ。	今[いま]、あの 国[くに]は 不景気[ふけいき]らしいよ。	いま あの くに は ふけいき らしい よ	
\\	今[いま]、あの 国[くに]は
\\	らしいよ。			
\\	響く	響[ひび]く	ひびく	
\\	彼の声はよく響きます。	彼[かれ]の 声[こえ]はよく 響[ひび]きます。	かれ の こえ は よく ひびきます	
\\	彼[かれ]の 声[こえ]はよく
\\	つかむ	つかむ	つかむ	
\\	彼女が僕の手をつかみました。	彼女[かのじょ]が 僕[ぼく]の 手[て]をつかみました。	かのじょ が ぼく の て を つかみました	
\\	彼女[かのじょ]が 僕[ぼく]の 手[て]を
\\	日光	日光[にっこう]	にっこう	
\\	この部屋は日光がよく当たるね。	この 部屋[へや]は 日光[にっこう]がよく 当[あ]たるね。	この へや は にっこう が よく あたる ね	
\\	この 部屋[へや]は
\\	がよく 当[あ]たるね。			
\\	光	光[ひかり]	ひかり	
\\	一筋の光が窓から差し込んだの。	一筋[ひとすじ]の 光[ひかり]が 窓[まど]から 差[さ]し 込[こ]んだの。	ひとすじ の ひかり が まど から さしこんだ の	
\\	一筋[ひとすじ]の
\\	が 窓[まど]から 差[さ]し 込[こ]んだの。			
\\	測定	測定[そくてい]	そくてい	
\\	これから身長と体重を測定します。	これから 身長[しんちょう]と 体重[たいじゅう]を 測定[そくてい]します。	これから しんちょう と たいじゅう を そくてい します	
\\	これから 身長[しんちょう]と 体重[たいじゅう]を
\\	します。			
\\	測る	測[はか]る	はかる	
\\	このドアの高さを測ってください。	このドアの 高[たか]さを 測[はか]ってください。	この どあ の たかさ を はかって ください	
\\	このドアの 高[たか]さを
\\	ください。			
\\	陽気	陽気[ようき]	ようき	
\\	彼女はとても陽気です。	彼女[かのじょ]はとても 陽気[ようき]です。	かのじょ は とても ようき です	
\\	彼女[かのじょ]はとても
\\	です。			
\\	もったいない	もったいない	もったいない	
\\	食べ物を残してはもったいないわよ。	食[た]べ 物[もの]を 残[のこ]してはもったいないわよ。	たべもの を のこして は もったいない わ よ	
\\	食[た]べ 物[もの]を 残[のこ]しては
\\	わよ。			
\\	プロ野球	プロ 野球[やきゅう]	プロやきゅう	
\\	最近のプロ野球は面白くなってきたね。	最近[さいきん]のプロ 野球[やきゅう]は 面白[おもしろ]くなってきたね。	さいきん の ぷろやきゅう は おもしろく なって きた ね	
\\	最近[さいきん]の
\\	は 面白[おもしろ]くなってきたね。			
\\	電球	電球[でんきゅう]	でんきゅう	
\\	電球が切れたので交換しましょう。	電球[でんきゅう]が 切[き]れたので 交換[こうかん]しましょう。	でんきゅう が きれた の で こうかん しましょう	
\\	が 切[き]れたので 交換[こうかん]しましょう。			
\\	振り返る	振[ふ]り 返[かえ]る	ふりかえる	
\\	学生時代を懐かしく振り返ったんだ。	学生時代[がくせい じだい]を 懐[なつ]かしく 振[ふ]り 返[かえ]ったんだ。	がくせい じだい を なつかしく ふりかえった ん だ	
\\	学生時代[がくせい じだい]を 懐[なつ]かしく
\\	んだ。			
\\	振り向く	振[ふ]り 向[む]く	ふりむく	
\\	彼女は振り向いて俺に微笑んだんだ。	彼女[かのじょ]は 振[ふ]り 向[む]いて 俺[おれ]に 微笑[ほほえ]んだんだ。	かのじょ は ふりむいて おれ に ほほえんだ ん だ	
\\	彼女[かのじょ]は
\\	俺[おれ]に 微笑[ほほえ]んだんだ。			
\\	振り	振[ふ]り	ふり	
\\	彼はバットの振りが大きすぎる。	彼[かれ]はバットの 振[ふ]りが 大[おお]きすぎる。	かれ は ばっと の ふり が おおき すぎる	
\\	彼[かれ]はバットの
\\	が 大[おお]きすぎる。			
\\	ふざける	ふざける	ふざける	
\\	ふざけるのは止めて。	ふざけるのは 止[や]めて。	ふざける の は やめて	
\\	のは 止[や]めて。			
\\	秘書	秘書[ひしょ]	ひしょ	
\\	私の秘書はとても優秀です。	私[わたし]の 秘書[ひしょ]はとても 優秀[ゆうしゅう]です。	わたし の ひしょ は とても ゆうしゅう です	
\\	私[わたし]の
\\	はとても 優秀[ゆうしゅう]です。			
\\	密か	密[ひそ]か	ひそか	
\\	彼女の誕生日パーティーを密かに計画しています。	彼女[かのじょ]の 誕生日[たんじょうび]パーティーを 密[ひそ]かに 計画[けいかく]しています。	かのじょ の たんじょうび ぱーてぃー を ひそか に けいかく して います	
\\	彼女[かのじょ]の 誕生日[たんじょうび]パーティーを
\\	に 計画[けいかく]しています。			
\\	貴い	貴[とうと]い	とうとい	
\\	その事故で貴い命が失われました。	その 事故[じこ]で 貴[とうと]い 命[いのち]が 失[うしな]われました。	その じこ で とうとい いのち が うしなわれました	
\\	その 事故[じこ]で
\\	命[いのち]が 失[うしな]われました。			
\\	追跡	追跡[ついせき]	ついせき	
\\	パトカーが車を追跡しているわ。	パトカーが 車[くるま]を 追跡[ついせき]しているわ。	ぱとかー が くるま を ついせき して いる わ	
\\	パトカーが 車[くるま]を
\\	しているわ。			
\\	寄せる	寄[よ]せる	よせる	
\\	車を塀に寄せたよ。	車[くるま]を 塀[へい]に 寄[よ]せたよ。	くるま を へい に よせた よ	
\\	車[くるま]を 塀[へい]に
\\	よ。			
\\	だるい	だるい	だるい	
\\	昨日は風邪で少しだるかったんだ。	昨日[きのう]は 風邪[かぜ]で 少[すこ]しだるかったんだ。	きのう は かぜ で すこし だるかった ん だ	
\\	昨日[きのう]は 風邪[かぜ]で 少[すこ]し
\\	んだ。			
\\	寄る	寄[よ]る	よる	
\\	帰りに叔母の家に寄ります。	帰[かえ]りに 叔母[おば]の 家[いえ]に 寄[よ]ります。	かえり に おば の いえ に よります	
\\	帰[かえ]りに 叔母[おば]の 家[いえ]に
\\	近寄る	近寄[ちかよ]る	ちかよる	
\\	彼に近寄らないで。	彼[かれ]に 近寄[ちかよ]らないで。	かれ に ちかよらない で	
\\	彼[かれ]に
\\	寄り道	寄[よ]り 道[みち]	よりみち	
\\	今日は寄り道してから帰ります。	今日[きょう]は 寄[よ]り 道[みち]してから 帰[かえ]ります。	きょう は よりみち して から かえります	
\\	今日[きょう]は
\\	してから 帰[かえ]ります。			
\\	寄り集まる	寄[よ]り 集[あつ]まる	よりあつまる	
\\	ニューヨークには芸術家が寄り集まっているの。	ニューヨークには 芸術家[げいじゅつか]が 寄[よ]り 集[あつ]まっているの。	にゅーよーく に は げいじゅつか が よりあつまって いる の	
\\	ニューヨークには 芸術家[げいじゅつか]が
\\	いるの。			
\\	歴史的	歴史的[れきしてき]	れきしてき	
\\	今日は歴史的な日です。	今日[きょう]は 歴史的[れきしてき]な 日[ひ]です。	きょう は れきしてき な ひ です	
\\	今日[きょう]は
\\	な 日[ひ]です。			
\\	宝石	宝石[ほうせき]	ほうせき	
\\	私が一番好きな宝石はダイヤモンドなの。	私[わたし]が 一番好[いちばん す]きな 宝石[ほうせき]はダイヤモンドなの。	わたし が いちばん すき な ほうせき は だいやもんど なの	
\\	私[わたし]が 一番好[いちばん す]きな
\\	はダイヤモンドなの。			
\\	ちぎる	ちぎる	ちぎる	
\\	紙を細かくちぎってください。	紙[かみ]を 細[こま]かくちぎってください。	かみ を こまかく ちぎって ください	
\\	紙[かみ]を 細[こま]かく
\\	ください。			
\\	建て前	建[た]て 前[まえ]	たてまえ	
\\	本音と建て前は違うことが多いよ。	本音[ほんね]と 建[た]て 前[まえ]は 違[ちが]うことが 多[おお]いよ。	ほんね と たてまえ は ちがう こと が おおい よ	
\\	本音[ほんね]と
\\	は 違[ちが]うことが 多[おお]いよ。			
\\	単位	単位[たんい]	たんい	
\\	メートルは長さの単位です。	メートルは 長[なが]さの 単位[たんい]です。	めーとる は ながさ の たんい です	
\\	メートルは 長[なが]さの
\\	です。			
\\	地位	地位[ちい]	ちい	
\\	彼女は会社で高い地位に就いているよ。	彼女[かのじょ]は 会社[かいしゃ]で 高[たか]い 地位[ちい]に 就[つ]いているよ。	かのじょ は かいしゃ で たかい ちい に ついて いる よ	
\\	彼女[かのじょ]は 会社[かいしゃ]で 高[たか]い
\\	に 就[つ]いているよ。			
\\	分離	分離[ぶんり]	ぶんり	
\\	自民党から新しい党が分離したね。	自民党[じみんとう]から 新[あたら]しい 党[とう]が 分離[ぶんり]したね。	じみんとう から あたらしい とう が ぶんり した ね	
\\	自民党[じみんとう]から 新[あたら]しい 党[とう]が
\\	したね。			
\\	離れる	離[はな]れる	はなれる	
\\	実家を4年間離れていました。	実家[じっか]を4 年間[ねんかん] 離[はな]れていました。	じっか を 
\\	ねんかん はなれて いました	
\\	実家[じっか]を4 年間[ねんかん]
\\	ねぎ	ねぎ	ねぎ	
\\	みそ汁にねぎを入れました。	みそ 汁[しる]にねぎを 入[い]れました。	みそしる に ねぎ を いれました	
\\	みそ 汁[しる]に
\\	を 入[い]れました。			
\\	離す	離[はな]す	はなす	
\\	子供の手を離さないでください。	子供[こども]の 手[て]を 離[はな]さないでください。	こども の て を はなさないで ください	
\\	子供[こども]の 手[て]を
\\	ください。			
\\	停電	停電[ていでん]	ていでん	
\\	台風で停電したよ。	台風[たいふう]で 停電[ていでん]したよ。	たいふう で ていでん した よ	
\\	台風[たいふう]で
\\	したよ。			
\\	停止	停止[ていし]	ていし	
\\	突然、機械が停止してしまったの。	突然[とつぜん]、 機械[きかい]が 停止[ていし]してしまったの。	とつぜん きかい が ていし して しまった の	
\\	突然[とつぜん]、 機械[きかい]が
\\	してしまったの。			
\\	範囲	範囲[はんい]	はんい	
\\	知っている範囲で教えてください。	知[し]っている 範囲[はんい]で 教[おし]えてください。	しって いる はんい で おしえて ください	
\\	知[し]っている
\\	で 教[おし]えてください。			
\\	隣り合う	隣[とな]り 合[あ]う	となりあう	
\\	この町は山と海が隣り合っています。	この 町[まち]は 山[やま]と 海[うみ]が 隣[とな]り 合[あ]っています。	この まち は やま と うみ が となりあって います	
\\	この 町[まち]は 山[やま]と 海[うみ]が
\\	びしょびしょ	びしょびしょ	びしょびしょ	
\\	服が雨でびしょびしょになったよ。	服[ふく]が 雨[あめ]でびしょびしょになったよ。	ふく が あめ で びしょびしょ に なった よ	
\\	服[ふく]が 雨[あめ]で
\\	になったよ。			
\\	横切る	横切[よこぎ]る	よこぎる	
\\	目の前を猫が横切ったんだ。	目[め]の 前[まえ]を 猫[ねこ]が 横切[よこぎ]ったんだ。	め の まえ を ねこ が よこぎった ん だ	
\\	目[め]の 前[まえ]を 猫[ねこ]が
\\	んだ。			
\\	横顔	横顔[よこがお]	よこがお	
\\	彼女の横顔は素敵だ。	彼女[かのじょ]の 横顔[よこがお]は 素敵[すてき]だ。	かのじょ の よこがお は すてき だ	
\\	彼女[かのじょ]の
\\	は 素敵[すてき]だ。			
\\	中断	中断[ちゅうだん]	ちゅうだん	
\\	停電のため仕事を中断しました。	停電[ていでん]のため 仕事[しごと]を 中断[ちゅうだん]しました。	ていでん の ため しごと を ちゅうだん しました	
\\	停電[ていでん]のため 仕事[しごと]を
\\	しました。			
\\	断水	断水[だんすい]	だんすい	
\\	地震のために1週間、断水したの。	地震[じしん]のために1 週間[しゅうかん]、 断水[だんすい]したの。	じしん の ため に 
\\	しゅうかん だんすい した の	
\\	地震[じしん]のために1 週間[しゅうかん]、
\\	したの。			
\\	油断	油断[ゆだん]	ゆだん	
\\	少しの油断が大きな事故につながります。	少[すこ]しの 油断[ゆだん]が 大[おお]きな 事故[じこ]につながります。	すこし の ゆだん が おおき な じこ に つながります	
\\	少[すこ]しの
\\	が 大[おお]きな 事故[じこ]につながります。			
\\	断定	断定[だんてい]	だんてい	
\\	まだ原因は断定できません。	まだ 原因[げんいん]は 断定[だんてい]できません。	まだ げんいん は だんてい できません	
\\	まだ 原因[げんいん]は
\\	できません。			
\\	バーゲン	バーゲン	バーゲン	
\\	昨日バーゲンでスーツを買いました。	昨日[きのう]バーゲンでスーツを 買[か]いました。	きのう ばーげん で すーつ を かいました	
\\	昨日[きのう]
\\	でスーツを 買[か]いました。			
\\	継ぐ	継[つ]ぐ	つぐ	
\\	彼は父親の店を継いだの。	彼[かれ]は 父親[ちちおや]の 店[みせ]を 継[つ]いだの。	かれ は ちちおや の みせ を ついだ の	
\\	彼[かれ]は 父親[ちちおや]の 店[みせ]を
\\	の。			
\\	中継	中継[ちゅうけい]	ちゅうけい	
\\	京都から中継で放送しています。	京都[きょうと]から 中継[ちゅうけい]で 放送[ほうそう]しています。	きょうと から ちゅうけい で ほうそう して います	
\\	京都[きょうと]から
\\	で 放送[ほうそう]しています。			
\\	幅広い	幅広[はばひろ]い	はばひろい	
\\	彼は幅広い知識を持っています。	彼[かれ]は 幅広[はばひろ]い 知識[ちしき]を 持[も]っています。	かれ は はばひろい ちしき を もって います	
\\	彼[かれ]は
\\	知識[ちしき]を 持[も]っています。			
\\	幅	幅[はば]	はば	
\\	この道は幅が狭いので気をつけて運転してください。	この 道[みち]は 幅[はば]が 狭[せま]いので 気[き]をつけて 運転[うんてん]してください。	この みち は はば が せまい の で き を つけて うんてん して ください	
\\	この 道[みち]は
\\	が 狭[せま]いので 気[き]をつけて 運転[うんてん]してください。			
\\	大陸	大陸[たいりく]	たいりく	
\\	ユーラシアは世界で最も大きい大陸です。	ユーラシアは 世界[せかい]で 最[もっと]も 大[おお]きい 大陸[たいりく]です。	ゆーらしあ は せかい で もっとも おおきい たいりく です	
\\	ユーラシアは 世界[せかい]で 最[もっと]も 大[おお]きい
\\	です。			
\\	メーター	メーター	メーター	
\\	ガス会社がメーターを調べにきたぞ。	ガス 会社[がいしゃ]がメーターを 調[しら]べにきたぞ。	がす がいしゃ が めーたー を しらべ に きた ぞ	
\\	ガス 会社[がいしゃ]が
\\	を 調[しら]べにきたぞ。			
\\	着陸	着陸[ちゃくりく]	ちゃくりく	
\\	飛行機が無事着陸したわね。	飛行機[ひこうき]が 無事[ぶじ] 着陸[ちゃくりく]したわね。	ひこうき が ぶじ ちゃくりく した わ ね	
\\	飛行機[ひこうき]が 無事[ぶじ]
\\	したわね。			
\\	離陸	離陸[りりく]	りりく	
\\	まもなく飛行機が離陸します。	まもなく 飛行機[ひこうき]が 離陸[りりく]します。	まもなく ひこうき が りりく します	
\\	まもなく 飛行機[ひこうき]が
\\	します。			
\\	内陸	内陸[ないりく]	ないりく	
\\	彼は中国の内陸を旅行しました。	彼[かれ]は 中国[ちゅうごく]の 内陸[ないりく]を 旅行[りょこう]しました。	かれ は ちゅうごく の ないりく を りょこう しました	
\\	彼[かれ]は 中国[ちゅうごく]の
\\	を 旅行[りょこう]しました。			
\\	陸上	陸上[りくじょう]	りくじょう	
\\	あの動物は陸上で生活しています。	あの 動物[どうぶつ]は 陸上[りくじょう]で 生活[せいかつ]しています。	あの どうぶつ は りくじょう で せいかつ して います	
\\	あの 動物[どうぶつ]は
\\	で 生活[せいかつ]しています。			
\\	陸	陸[りく]	りく	
\\	ウミガメが陸に上がってきたんだ。	ウミガメが 陸[りく]に 上[あ]がってきたんだ。	うみがめ が りく に あがって きた ん だ	
\\	ウミガメが
\\	に 上[あ]がってきたんだ。			
\\	もしかすると	もしかすると	もしかすると	
\\	もしかするとあなたに一度お会いしてますか。	もしかするとあなたに 一度[いちど]お 会[あ]いしてますか。	もしかすると あなた に いちど おあい してます か	
\\	あなたに 一度[いちど]お 会[あ]いしてますか。			
\\	南極	南極[なんきょく]	なんきょく	
\\	南極でペンギンを見ました。	南極[なんきょく]でペンギンを 見[み]ました。	なんきょく で ぺんぎん を みました	
\\	でペンギンを 見[み]ました。			
\\	北極	北極[ほっきょく]	ほっきょく	
\\	北極にペンギンはいません。	北極[ほっきょく]にペンギンはいません。	ほっきょく に ぺんぎん は いません	
\\	にペンギンはいません。			
\\	先端	先端[せんたん]	せんたん	
\\	棒の先端を持って下さい。	棒[ぼう]の 先端[せんたん]を 持[も]って 下[くだ]さい。	ぼう の せんたん を もって ください	
\\	棒[ぼう]の
\\	を 持[も]って 下[くだ]さい。			
\\	端	端[はし]	はし	
\\	辞書は本棚の端にあります。	辞書[じしょ]は 本棚[ほんだな]の 端[はし]にあります。	じしょ は ほんだな の はし に あります	
\\	辞書[じしょ]は 本棚[ほんだな]の
\\	にあります。			
\\	訪問	訪問[ほうもん]	ほうもん	
\\	来週、妻の実家を訪問します。	来週[らいしゅう]、 妻[つま]の 実家[じっか]を 訪問[ほうもん]します。	らいしゅう つま の じっか を ほうもん します	
\\	来週[らいしゅう]、 妻[つま]の 実家[じっか]を
\\	します。			
\\	女房	女房[にょうぼう]	にょうぼう	
\\	女房は実家に帰っています。	女房[にょうぼう]は 実家[じっか]に 帰[かえ]っています。	にょうぼう は じっか に かえって います	
\\	は 実家[じっか]に 帰[かえ]っています。			
\\	ちぎれる	ちぎれる	ちぎれる	
\\	荷物が重くて手がちぎれそうだ。	荷物[にもつ]が 重[おも]くて 手[て]がちぎれそうだ。	にもつ が おもくて て が ちぎれ そう だ	
\\	荷物[にもつ]が 重[おも]くて 手[て]が
\\	だ。			
\\	内緒	内緒[ないしょ]	ないしょ	
\\	この話は課長には内緒ですよ。	この 話[はなし]は 課長[かちょう]には 内緒[ないしょ]ですよ。	この はなし は かちょう に は ないしょ です よ	
\\	この 話[はなし]は 課長[かちょう]には
\\	ですよ。			
\\	百貨店	百貨店[ひゃっかてん]	ひゃっかてん	
\\	友達と百貨店で買い物をしました。	友達[ともだち]と 百貨店[ひゃっかてん]で 買[か]い 物[もの]をしました。	ともだち と ひゃっかてん で かいもの を しました	
\\	友達[ともだち]と
\\	で 買[か]い 物[もの]をしました。			
\\	輸送	輸送[ゆそう]	ゆそう	
\\	この商品はトラックで輸送します。	この 商品[しょうひん]はトラックで 輸送[ゆそう]します。	この しょうひん は とらっく で ゆそう します	
\\	この 商品[しょうひん]はトラックで
\\	します。			
\\	復活	復活[ふっかつ]	ふっかつ	
\\	その選手は怪我を乗り越えて復活したわね。	その 選手[せんしゅ]は 怪我[けが]を 乗[の]り 越[こ]えて 復活[ふっかつ]したわね。	その せんしゅ は けが を のりこえて ふっかつ した わ ね	
\\	その 選手[せんしゅ]は 怪我[けが]を 乗[の]り 越[こ]えて
\\	したわね。			
\\	操作	操作[そうさ]	そうさ	
\\	この携帯電話は操作が簡単です。	この 携帯電話[けいたい でんわ]は 操作[そうさ]が 簡単[かんたん]です。	この けいたい でんわ は そうさ が かんたん です	
\\	この 携帯電話[けいたい でんわ]は
\\	が 簡単[かんたん]です。			
\\	にっこり	にっこり	にっこり	
\\	彼女はにっこりほほえんだ。	彼女[かのじょ]はにっこりほほえんだ。	かのじょ は にっこり ほほえんだ	
\\	彼女[かのじょ]は
\\	ほほえんだ。			
\\	体操	体操[たいそう]	たいそう	
\\	兄は体操の選手です。	兄[あに]は 体操[たいそう]の 選手[せんしゅ]です。	あに は たいそう の せんしゅ です	
\\	兄[あに]は
\\	の 選手[せんしゅ]です。			
\\	縦	縦[たて]	たて	
\\	縦2列に並んでください。	縦[たて] 
\\	列[れつ]に 並[なら]んでください。	たて 
\\	れつ に ならんで ください	
\\	列[れつ]に 並[なら]んでください。			
\\	操縦	操縦[そうじゅう]	そうじゅう	
\\	このボートは操縦が簡単です。	このボートは 操縦[そうじゅう]が 簡単[かんたん]です。	この ぼーと は そうじゅう が かんたん です	
\\	このボートは
\\	が 簡単[かんたん]です。			
\\	縦書き	縦書[たてが]き	たてがき	
\\	作文は縦書きで書いてください。	作文[さくぶん]は 縦書[たてが]きで 書[か]いてください。	さくぶん は たてがき で かいてください	
\\	作文[さくぶん]は
\\	で 書[か]いてください。			
\\	滞在	滞在[たいざい]	たいざい	
\\	日本では北海道に滞在しました。	日本[にっぽん]では 北海道[ほっかいどう]に 滞在[たいざい]しました。	にっぽん で は ほっかいどう に たいざい しました	
\\	日本[にっぽん]では 北海道[ほっかいどう]に
\\	しました。			
\\	ぱいなっぷる	ぱいなっぷる	ぱいなっぷる	
\\	このパイナップルを切ってください。	このパイナップルを 切[き]ってください。	この ぱいなっぷる を きって ください	
\\	この
\\	を 切[き]ってください。			
\\	地帯	地帯[ちたい]	ちたい	
\\	この都市は工業地帯です。	この 都市[とし]は 工業[こうぎょう] 地帯[ちたい]です。	この とし は こうぎょう ちたい です	
\\	この 都市[とし]は 工業[こうぎょう]
\\	です。			
\\	熱帯	熱帯[ねったい]	ねったい	
\\	店に熱帯の果物が並んでいますよ。	店[みせ]に 熱帯[ねったい]の 果物[くだもの]が 並[なら]んでいますよ。	みせ に ねったい の くだもの が ならんで います よ	
\\	店[みせ]に
\\	の 果物[くだもの]が 並[なら]んでいますよ。			
\\	保守	保守[ほしゅ]	ほしゅ	
\\	あの村は伝統を保守している。	あの 村[むら]は 伝統[でんとう]を 保守[ほしゅ]している。	あの むら は でんとう を ほしゅ して いる	
\\	あの 村[むら]は 伝統[でんとう]を
\\	している。			
\\	見守る	見守[みまも]る	みまもる	
\\	皆で暖かく見守りましょう。	皆[みんな]で 暖[あたた]かく 見守[みまも]りましょう。	みんな で あたたかく みまもりましょう	
\\	皆[みんな]で 暖[あたた]かく
\\	留守番	留守番[るすばん]	るすばん	
\\	私は留守番の者です。	私[わたし]は 留守番[るすばん]の 者[もの]です。	わたし は るすばん の もの です	
\\	私[わたし]は
\\	の 者[もの]です。			
\\	ぱんだ	ぱんだ	ぱんだ	
\\	パンダは笹を食べます。	パンダは 笹[ささ]を 食[た]べます。	ぱんだ は ささ を たべます	
\\	は 笹[ささ]を 食[た]べます。			
\\	戻す	戻[もど]す	もどす	
\\	話を戻しましょう。	話[はなし]を 戻[もど]しましょう。	はなし を もどしましょう	
\\	話[はなし]を
\\	取り戻す	取[と]り 戻[もど]す	とりもどす	
\\	緑を取り戻す必要があります。	緑[みどり]を 取[と]り 戻[もど]す 必要[ひつよう]があります。	みどり を とりもどす ひつよう が あります	
\\	緑[みどり]を
\\	必要[ひつよう]があります。			
\\	払い戻す	払[はら]い 戻[もど]す	はらいもどす	
\\	飛行機の運賃が払い戻されたの。	飛行機[ひこうき]の 運賃[うんちん]が 払[はら]い 戻[もど]されたの。	ひこうき の うんちん が はらいもどされた の	
\\	飛行機[ひこうき]の 運賃[うんちん]が
\\	の。			
\\	寝かす	寝[ね]かす	ねかす	
\\	いつも9時に子供を寝かします。	いつも9 時[じ]に 子供[こども]を 寝[ね]かします。	いつも 
\\	じ に こども を ねかします	
\\	いつも9 時[じ]に 子供[こども]を
\\	寝過ごす	寝過[ねす]ごす	ねすごす	
\\	うっかり寝過ごしてしまったんだ。	うっかり 寝過[ねす]ごしてしまったんだ。	うっかり ねすごして しまった ん だ	
\\	うっかり
\\	んだ。			
\\	早寝	早寝[はやね]	はやね	
\\	早寝は健康のためによいことです。	早寝[はやね]は 健康[けんこう]のためによいことです。	はやね は けんこう の ため に よい こと です	
\\	は 健康[けんこう]のためによいことです。			
\\	びっしょり	びっしょり	びっしょり	
\\	運動して汗びっしょりです。	運動[うんどう]して 汗[あせ]びっしょりです。	うんどう して あせ びっしょり です	
\\	運動[うんどう]して 汗[あせ]
\\	です。			
\\	寝かせる	寝[ね]かせる	ねかせる	
\\	赤ちゃんをベッドに寝かせた。	赤[あか]ちゃんをベッドに 寝[ね]かせた。	あかちゃん を べっど に ねかせた	
\\	赤[あか]ちゃんをベッドに
\\	寝転ぶ	寝転[ねころ]ぶ	ねころぶ	
\\	土手に寝転んで空をながめました。	土手[どて]に 寝転[ねころ]んで 空[そら]をながめました。	どて に ねころんで そら を ながめました	
\\	土手[どて]に
\\	空[そら]をながめました。			
\\	冷静	冷静[れいせい]	れいせい	
\\	冷静に話し合いましょう。	冷静[れいせい]に 話[はな]し 合[あ]いましょう。	れいせい に はなしあいましょう	
\\	に 話[はな]し 合[あ]いましょう。			
\\	両者	両者[りょうしゃ]	りょうしゃ	
\\	両者の意見を聞きましょう。	両者[りょうしゃ]の 意見[いけん]を 聞[き]きましょう。	りょうしゃ の いけん を ききましょう	
\\	の 意見[いけん]を 聞[き]きましょう。			
\\	両立	両立[りょうりつ]	りょうりつ	
\\	勉強と仕事の両立は難しいよ。	勉強[べんきょう]と 仕事[しごと]の 両立[りょうりつ]は 難[むずか]しいよ。	べんきょう と しごと の りょうりつ は むずかしい よ	
\\	勉強[べんきょう]と 仕事[しごと]の
\\	は 難[むずか]しいよ。			
\\	ペンチ	ペンチ	ペンチ	
\\	ペンチで針金を切ります。	ペンチで 針金[はりがね]を 切[き]ります。	ぺんち で はりがね を きります	
\\	で 針金[はりがね]を 切[き]ります。			
\\	両手	両手[りょうて]	りょうて	
\\	彼は両手を握り締めたの。	彼[かれ]は 両手[りょうて]を 握[にぎ]り 締[し]めたの。	かれ は りょうて を にぎりしめた の	
\\	彼[かれ]は
\\	を 握[にぎ]り 締[し]めたの。			
\\	両面	両面[りょうめん]	りょうめん	
\\	紙の両面に字が書かれていたよ。	紙[かみ]の 両面[りょうめん]に 字[じ]が 書[か]かれていたよ。	かみ の りょうめん に じ が かかれて いた よ	
\\	紙[かみ]の
\\	に 字[じ]が 書[か]かれていたよ。			
\\	側面	側面[そくめん]	そくめん	
\\	これが建物の側面の写真です。	これが 建物[たてもの]の 側面[そくめん]の 写真[しゃしん]です。	これ が たてもの の そくめん の しゃしん です	
\\	これが 建物[たてもの]の
\\	の 写真[しゃしん]です。			
\\	両側	両側[りょうがわ]	りょうがわ	
\\	道の両側にたくさんお店がありますよ。	道[みち]の 両側[りょうがわ]にたくさんお 店[みせ]がありますよ。	みち の りょうがわ に たくさん おみせ が あります よ	
\\	道[みち]の
\\	にたくさんお 店[みせ]がありますよ。			
\\	要項	要項[ようこう]	ようこう	
\\	募集要項をよくお読みください。	募集[ぼしゅう] 要項[ようこう]をよくお 読[よ]みください。	ぼしゅう ようこう を よく お よみ ください	
\\	募集[ぼしゅう]
\\	をよくお 読[よ]みください。			
\\	モダン	モダン	モダン	
\\	あの建物はモダンだね。	あの 建物[たてもの]はモダンだね。	あの たてもの は もだん だ ね	
\\	あの 建物[たてもの]は
\\	だね。			
\\	目印	目印[めじるし]	めじるし	
\\	私の家は赤い屋根が目印です。	私[わたし]の 家[いえ]は 赤[あか]い 屋根[やね]が 目印[めじるし]です。	わたし の いえ は あかい やね が めじるし です	
\\	私[わたし]の 家[いえ]は 赤[あか]い 屋根[やね]が
\\	です。			
\\	細長い	細長[ほそなが]い	ほそながい	
\\	その細長い棒を取ってください。	その 細長[ほそなが]い 棒[ぼう]を 取[と]ってください。	その ほそながい ぼう を とって ください	
\\	その
\\	棒[ぼう]を 取[と]ってください。			
\\	載る	載[の]る	のる	
\\	先生の論文が雑誌に載りましたよ。	先生[せんせい]の 論文[ろんぶん]が 雑誌[ざっし]に 載[の]りましたよ。	せんせい の ろんぶん が ざっし に のりました よ	
\\	先生[せんせい]の 論文[ろんぶん]が 雑誌[ざっし]に
\\	よ。			
\\	積む	積[つ]む	つむ	
\\	車に荷物を積んでください。	車[くるま]に 荷物[にもつ]を 積[つ]んでください。	くるま に にもつ を つんで ください	
\\	車[くるま]に 荷物[にもつ]を
\\	ください。			
\\	面積	面積[めんせき]	めんせき	
\\	この土地の面積はどれ位ですか。	この 土地[とち]の 面積[めんせき]はどれ 位[くらい]ですか。	この とち の めんせき は どれ くらい です か	
\\	この 土地[とち]の
\\	はどれ 位[くらい]ですか。			
\\	翻訳	翻訳[ほんやく]	ほんやく	
\\	彼の翻訳は分かりやすいですよ。	彼[かれ]の 翻訳[ほんやく]は 分[わ]かりやすいですよ。	かれ の ほんやく は わかりやすい です よ	
\\	彼[かれ]の
\\	は 分[わ]かりやすいですよ。			
\\	リクエスト	リクエスト	リクエスト	
\\	ラジオ番組にリクエストを送ったの。	ラジオ 番組[ばんぐみ]にリクエストを 送[おく]ったの。	らじお ばんぐみ に りくえすと を おくった の	
\\	ラジオ 番組[ばんぐみ]に
\\	を 送[おく]ったの。			
\\	通訳	通訳[つうやく]	つうやく	
\\	私は通訳です。	私[わたし]は 通訳[つうやく]です。	わたし は つうやく です	
\\	私[わたし]は
\\	です。			
\\	申し訳ない	申[もう]し 訳[わけ]ない	もうしわけない	
\\	彼には申し訳ないことをした。	彼[かれ]には 申[もう]し 訳[わけ]ないことをした。	かれ に は もうしわけない こと を した	
\\	彼[かれ]には
\\	ことをした。			
\\	訳	訳[わけ]	わけ	
\\	そんな訳で、私は仕事を辞めました。	そんな 訳[わけ]で、 私[わたし]は 仕事[しごと]を 辞[や]めました。	そんな わけ で わたし は しごと を やめました	
\\	そんな
\\	で、 私[わたし]は 仕事[しごと]を 辞[や]めました。			
\\	申し訳	申[もう]し 訳[わけ]	もうしわけ	
\\	申し訳ございません。	申[もう]し 訳[わけ]ございません。	もうしわけ ございません	
\\	ございません。			
\\	訂正	訂正[ていせい]	ていせい	
\\	間違いを訂正しました。	間違[まちが]いを 訂正[ていせい]しました。	まちがい を ていせい しました	
\\	間違[まちが]いを
\\	しました。			
\\	ビーチ	ビーチ	ビーチ	
\\	ビーチで友達とサーフィンしたよ。	ビーチで 友達[ともだち]とサーフィンしたよ。	びーち で ともだち と さーふぃん した よ	
\\	で 友達[ともだち]とサーフィンしたよ。			
\\	討論	討論[とうろん]	とうろん	
\\	その問題についてクラスで討論しました。	その 問題[もんだい]についてクラスで 討論[とうろん]しました。	その もんだい に ついて くらす で とうろん しました	
\\	その 問題[もんだい]についてクラスで
\\	しました。			
\\	添える	添[そ]える	そえる	
\\	贈り物に手書きのカードを添えました。	贈[おく]り 物[もの]に 手書[てが]きのカードを 添[そ]えました。	おくりもの に てがき の かーど を そえました	
\\	贈[おく]り 物[もの]に 手書[てが]きのカードを
\\	添う	添[そ]う	そう	
\\	あなたの期待に添えなくてすみません。	あなたの 期待[きたい]に 添[そ]えなくてすみません。	あなた の きたい に そえなくて すみません	
\\	あなたの 期待[きたい]に
\\	すみません。			
\\	付近	付近[ふきん]	ふきん	
\\	事件現場付近で怪しい人物を見た。	事件現場[じけん げんば] 付近[ふきん]で 怪[あや]しい 人物[じんぶつ]を 見[み]た。	じけん げんば ふきん で あやしい じんぶつ を みた	
\\	事件現場[じけん げんば]
\\	で 怪[あや]しい 人物[じんぶつ]を 見[み]た。			
\\	付き合い	付[つ]き 合[あ]い	つきあい	
\\	今日は付き合いで帰りが遅くなります。	今日[きょう]は 付[つ]き 合[あ]いで 帰[かえ]りが 遅[おそ]くなります。	きょう は つきあい で かえり が おそく なります	
\\	今日[きょう]は
\\	で 帰[かえ]りが 遅[おそ]くなります。			
\\	ビタミン	ビタミン	ビタミン	
\\	ビタミンを毎日とりましょう。	ビタミンを 毎日[まいにち]とりましょう。	びたみん を まいにち とりましょう	
\\	を 毎日[まいにち]とりましょう。			
\\	付け加える	付[つ]け 加[くわ]える	つけくわえる	
\\	自分の考えを付け加えました。	自分[じぶん]の 考[かんが]えを 付[つ]け 加[くわ]えました。	じぶん の かんがえ を つけくわえました	
\\	自分[じぶん]の 考[かんが]えを
\\	付き合う	付[つ]き 合[あ]う	つきあう	
\\	彼女と4年ほど付き合っています。	彼女[かのじょ]と4 年[ねん]ほど 付[つ]き 合[あ]っています。	かのじょ と 
\\	ねん ほど つきあって います	
\\	彼女[かのじょ]と4 年[ねん]ほど
\\	います。			
\\	名付ける	名付[なづ]ける	なづける	
\\	子猫にトラと名付けました。	子猫[こねこ]にトラと 名付[なづ]けました。	こねこ に とら と なづけました	
\\	子猫[こねこ]にトラと
\\	付録	付録[ふろく]	ふろく	
\\	今月の付録はアニメの
\\	です。	今月[こんげつ]の 付録[ふろく]はアニメの 
\\	[でぃーぶぃーでぃー]です。	こんげつ の ふろく は あにめ の でぃーぶぃーでぃー です	
\\	今月[こんげつ]の
\\	はアニメの 
\\	[でぃーぶぃーでぃー]です。			
\\	近付ける	近付[ちかづ]ける	ちかづける	
\\	私は顔を窓に近付けたんだ。	私[わたし]は 顔[かお]を 窓[まど]に 近付[ちかづ]けたんだ。	わたし は かお を まど に ちかづけた ん だ	
\\	私[わたし]は 顔[かお]を 窓[まど]に
\\	んだ。			
\\	付け足す	付[つ]け 足[た]す	つけたす	
\\	何か付け足すことはありますか。	何[なに]か 付[つ]け 足[た]すことはありますか。	なにか つけたす こと は あります か	
\\	何[なに]か
\\	ことはありますか。			
\\	フライパン	フライパン	フライパン	
\\	フライパンで目玉焼きを作ったよ。	フライパンで 目玉焼[めだまや]きを 作[つく]ったよ。	ふらいぱん で めだまやき を つくった よ	
\\	で 目玉焼[めだまや]きを 作[つく]ったよ。			
\\	日付け	日付[ひづ]け	ひづけ	
\\	今日の日付けは6月19日です。	今日[きょう]の 日付[ひづ]けは6 月19日[がつ 
\\	にち]です。	きょう の ひづけ は 
\\	がつ 
\\	にち です	
\\	今日[きょう]の
\\	は6 月19日[がつ 
\\	にち]です。			
\\	付属	付属[ふぞく]	ふぞく	
\\	このソフトには色々なツールが付属しています。	このソフトには 色々[いろいろ]なツールが 付属[ふぞく]しています。	この そふと に は いろいろ な つーる が ふぞく して います	
\\	このソフトには 色々[いろいろ]なツールが
\\	しています。			
\\	属する	属[ぞく]する	ぞくする	
\\	私は市民オーケストラに属しています。	私[わたし]は 市民[しみん]オーケストラに 属[ぞく]しています。	わたし は しみん おーけすとら に ぞくして います	
\\	私[わたし]は 市民[しみん]オーケストラに
\\	反省	反省[はんせい]	はんせい	
\\	彼は十分反省しています。	彼[かれ]は 十分[じゅうぶん] 反省[はんせい]しています。	かれ は じゅうぶん はんせい して います	
\\	彼[かれ]は 十分[じゅうぶん]
\\	しています。			
\\	省く	省[はぶ]く	はぶく	
\\	時間がないので詳細は省きます。	時間[じかん]がないので 詳細[しょうさい]は 省[はぶ]きます。	じかん が ない の で しょうさい は はぶきます	
\\	時間[じかん]がないので 詳細[しょうさい]は
\\	まあまあ	まあまあ	まあまあ	
\\	まあまあ、落ち着いてください。	まあまあ、 落[お]ち 着[つ]いてください。	まあまあ おちついて ください	
\\	、 落[お]ち 着[つ]いてください。			
\\	略す	略[りゃく]す	りゃくす	
\\	国際連合を略して国連といいます。	国際連合[こくさいれんごう]を 略[りゃく]して 国連[こくれん]といいます。	こくさいれんごう を りゃく して こくれん と いいます	
\\	国際連合[こくさいれんごう]を
\\	国連[こくれん]といいます。			
\\	略	略[りゃく]	りゃく	
\\	は何の略か知っていますか。	
\\	[あいてぃー]は 何[なん]の 略[りゃく]か 知[し]っていますか。	あいてぃー は なん の りゃく か しって います か	
\\	[あいてぃー]は 何[なん]の
\\	か 知[し]っていますか。			
\\	大概	大概[たいがい]	たいがい	
\\	大概、風邪は寝ていれば治ります。	大概[たいがい]、 風邪[かぜ]は 寝[ね]ていれば 治[なお]ります。	たいがい かぜ は ねて いれば なおります	
\\	、 風邪[かぜ]は 寝[ね]ていれば 治[なお]ります。			
\\	含める	含[ふく]める	ふくめる	
\\	私を含めて10人が参加しました。	私[わたし]を 含[ふく]めて10 人[にん]が 参加[さんか]しました。	わたし を ふくめて 
\\	にん が さんか しました	
\\	私[わたし]を
\\	人[にん]が 参加[さんか]しました。			
\\	含む	含[ふく]む	ふくむ	
\\	その食品は有害物質を含んでいるぞ。	その 食品[しょくひん]は 有害物質[ゆうがい ぶっしつ]を 含[ふく]んでいるぞ。	その しょくひん は ゆうがい ぶっしつ を ふくんで いる ぞ	
\\	その 食品[しょくひん]は 有害物質[ゆうがい ぶっしつ]を
\\	ぞ。			
\\	たこ	たこ	たこ	
\\	タコには足が8本ある。	タコには 足[あし]が8 本[ぽん]ある。	たこ に は あし が 
\\	ぽん ある	
\\	には 足[あし]が8 本[ぽん]ある。			
\\	道順	道順[みちじゅん]	みちじゅん	
\\	家から学校までの道順を教えてください。	家[いえ]から 学校[がっこう]までの 道順[みちじゅん]を 教[おし]えてください。	いえ から がっこう まで の みちじゅん を おしえて ください	
\\	家[いえ]から 学校[がっこう]までの
\\	を 教[おし]えてください。			
\\	列車	列車[れっしゃ]	れっしゃ	
\\	列車がホームに入ってきた。	列車[れっしゃ]がホームに 入[はい]ってきた。	れっしゃ が ほーむ に はいって きた	
\\	がホームに 入[はい]ってきた。			
\\	列島	列島[れっとう]	れっとう	
\\	日本は列島です。	日本[にっぽん]は 列島[れっとう]です。	にっぽん は れっとう です	
\\	日本[にっぽん]は
\\	です。			
\\	列	列[れつ]	れつ	
\\	店の前に長い列ができていたね。	店[みせ]の 前[まえ]に 長[なが]い 列[れつ]ができていたね。	みせ の まえ に ながい れつ が できて いた ね	
\\	店[みせ]の 前[まえ]に 長[なが]い
\\	ができていたね。			
\\	例外	例外[れいがい]	れいがい	
\\	例外は認めません。	例外[れいがい]は 認[みと]めません。	れいがい は みとめません	
\\	は 認[みと]めません。			
\\	デザート	デザート	デザート	
\\	デザートにケーキを食べました。	デザートにケーキを 食[た]べました。	でざーと に けーき を たべました	
\\	にケーキを 食[た]べました。			
\\	比例	比例[ひれい]	ひれい	
\\	努力と結果が比例していないの。	努力[どりょく]と 結果[けっか]が 比例[ひれい]していないの。	どりょく と けっか が ひれい して いない の	
\\	努力[どりょく]と 結果[けっか]が
\\	していないの。			
\\	用例	用例[ようれい]	ようれい	
\\	用例を使って説明してください。	用例[ようれい]を 使[つか]って 説明[せつめい]してください。	ようれい を つかって せつめい して ください	
\\	を 使[つか]って 説明[せつめい]してください。			
\\	例える	例[たと]える	たとえる	
\\	彼は彼女をバラに例えたんだ。	彼[かれ]は 彼女[かのじょ]をバラに 例[たと]えたんだ。	かれ は かのじょ を ばら に たとえた ん だ	
\\	彼[かれ]は 彼女[かのじょ]をバラに
\\	んだ。			
\\	例題	例題[れいだい]	れいだい	
\\	みんなで例題を解いてみましょう。	みんなで 例題[れいだい]を 解[と]いてみましょう。	みんな で れいだい を といて みましょう	
\\	みんなで
\\	を 解[と]いてみましょう。			
\\	例年	例年[れいねん]	れいねん	
\\	例年8月は雨が少ないね。	例年[れいねん] 
\\	月[がつ]は 雨[あめ]が 少[すく]ないね。	れいねん 
\\	がつ は あめ が すくない ね	
\\	月[がつ]は 雨[あめ]が 少[すく]ないね。			
\\	例え	例[たと]え	たとえ	
\\	例えを上げると話が分かり易くなる。	例[たと]えを 上[あ]げると 話[はなし]が 分[わ]かり 易[やす]くなる。	たとえ を あげる と はなし が わかり やすく なる	
\\	を 上[あ]げると 話[はなし]が 分[わ]かり 易[やす]くなる。			
\\	とっさに	とっさに	とっさに	
\\	とっさに彼の名前が出てこなかった。	とっさに 彼[かれ]の 名前[なまえ]が 出[で]てこなかった。	とっさ に かれ の なまえ が でてこなかった 。	
\\	に 彼[かれ]の 名前[なまえ]が 出[で]てこなかった。			
\\	余地	余地[よち]	よち	
\\	彼に言い訳の余地はありません。	彼[かれ]に 言[い]い 訳[わけ]の 余地[よち]はありません。	かれ に いいわけ の よち は ありません	
\\	彼[かれ]に 言[い]い 訳[わけ]の
\\	はありません。			
\\	余計	余計[よけい]	よけい	
\\	余計なことを言ってはだめよ。	余計[よけい]なことを 言[い]ってはだめよ。	よけい な こと を いって は だめ よ	
\\	なことを 言[い]ってはだめよ。			
\\	余分	余分[よぶん]	よぶん	
\\	食事は余分に用意してあります。	食事[しょくじ]は 余分[よぶん]に 用意[ようい]してあります。	しょくじ は よぶん に ようい して あります	
\\	食事[しょくじ]は
\\	に 用意[ようい]してあります。			
\\	途中	途中[とちゅう]	とちゅう	
\\	途中で30分ほど休みましょう。	途中[とちゅう]で30 分[ぷん]ほど 休[やす]みましょう。	とちゅう で 
\\	ぷん ほど やすみましょう	
\\	で30 分[ぷん]ほど 休[やす]みましょう。			
\\	用途	用途[ようと]	ようと	
\\	この道具の用途を説明します。	この 道具[どうぐ]の 用途[ようと]を 説明[せつめい]します。	この どうぐ の ようと を せつめい します	
\\	この 道具[どうぐ]の
\\	を 説明[せつめい]します。			
\\	ぬるぬる	ぬるぬる	ぬるぬる	
\\	うなぎはぬるぬるしていますね。	うなぎはぬるぬるしていますね。	うなぎはぬるぬるしていますね。	
\\	うなぎは
\\	していますね。			
\\	中途	中途[ちゅうと]	ちゅうと	
\\	私は中途採用で入社しました。	私[わたし]は 中途[ちゅうと] 採用[さいよう]で 入社[にゅうしゃ]しました。	わたし は ちゅうと さいよう で にゅうしゃ しました	
\\	私[わたし]は
\\	採用[さいよう]で 入社[にゅうしゃ]しました。			
\\	取り締まり	取[と]り 締[し]まり	とりしまり	
\\	違法駐車の取り締まりが厳しくなったの。	違法駐車[いほう ちゅうしゃ]の 取[と]り 締[し]まりが 厳[きび]しくなったの。	いほう ちゅうしゃ の とりしまり が きびしく なった の	
\\	違法駐車[いほう ちゅうしゃ]の
\\	が 厳[きび]しくなったの。			
\\	取り締まる	取[と]り 締[し]まる	とりしまる	
\\	警察が飲酒運転を取り締まっていますよ。	警察[けいさつ]が 飲酒運転[いんしゅ うんてん]を 取[と]り 締[し]まっていますよ。	けいさつ が いんしゅ うんてん を とりしまって います よ	
\\	警察[けいさつ]が 飲酒運転[いんしゅ うんてん]を
\\	よ。			
\\	緩める	緩[ゆる]める	ゆるめる	
\\	食べ過ぎたのでベルトを緩めたの。	食[た]べ 過[す]ぎたのでベルトを 緩[ゆる]めたの。	たべすぎた の で べると を ゆるめた の	
\\	食[た]べ 過[す]ぎたのでベルトを
\\	の。			
\\	緩やか	緩[ゆる]やか	ゆるやか	
\\	ここから先は緩やかな坂になっています。	ここから 先[さき]は 緩[ゆる]やかな 坂[さか]になっています。	ここ から さき は ゆるやか な さか に なって います	
\\	ここから 先[さき]は
\\	な 坂[さか]になっています。			
\\	のろい	のろい	のろい	
\\	この電車は本当にのろいですね。	この 電車[でんしゃ]は 本当[ほんとう]にのろいですね。	この でんしゃ は ほんとうに のろい です ね	
\\	この 電車[でんしゃ]は 本当[ほんとう]に
\\	ですね。			
\\	緩む	緩[ゆる]む	ゆるむ	
\\	彼は気が緩んでいます。	彼[かれ]は 気[き]が 緩[ゆる]んでいます。	かれ は き が ゆるんで います	
\\	彼[かれ]は 気[き]が
\\	養う	養[やしな]う	やしなう	
\\	私は3人の子供を養っています。	私[わたし]は3 人[にん]の 子供[こども]を 養[やしな]っています。	わたし は 
\\	にん の こども を やしなって います	
\\	私[わたし]は3 人[にん]の 子供[こども]を
\\	います。			
\\	豊か	豊[ゆた]か	ゆたか	
\\	彼は心が豊かな人です。	彼[かれ]は 心[こころ]が 豊[ゆた]かな 人[ひと]です。	かれ は こころ が ゆたか な ひと です	
\\	彼[かれ]は 心[こころ]が
\\	な 人[ひと]です。			
\\	富む	富[と]む	とむ	
\\	彼の人生は変化に富んでいるな。	彼[かれ]の 人生[じんせい]は 変化[へんか]に 富[と]んでいるな。	かれ の じんせい は へんか に とんで いる な	
\\	彼[かれ]の 人生[じんせい]は 変化[へんか]に
\\	な。			
\\	豊富	豊富[ほうふ]	ほうふ	
\\	この果物はビタミンが豊富です。	この 果物[くだもの]はビタミンが 豊富[ほうふ]です。	この くだもの は びたみん が ほうふ です	
\\	この 果物[くだもの]はビタミンが
\\	です。			
\\	目茶苦茶	目茶苦茶[めちゃくちゃ]	めちゃくちゃ	
\\	彼の運転は目茶苦茶です。	彼[かれ]の 運転[うんてん]は 目茶苦茶[めちゃくちゃ]です。	かれ の うんてん は めちゃくちゃ です	
\\	彼[かれ]の 運転[うんてん]は
\\	です。			
\\	ぴょんぴょん	ぴょんぴょん	ぴょんぴょん	
\\	カエルがぴょんぴょんはねているね。	カエルがぴょんぴょんはねているね。	かえる が ぴょんぴょん はねて いる ね	
\\	カエルが
\\	はねているね。			
\\	労働者	労働者[ろうどうしゃ]	ろうどうしゃ	
\\	労働者がストをしたそうだよ。	労働者[ろうどうしゃ]がストをしたそうだよ。	ろうどうしゃ が すと を した そう だ よ	
\\	がストをしたそうだよ。			
\\	労働	労働[ろうどう]	ろうどう	
\\	彼らは1日7時間労働している。	彼[かれ]らは1 日7時間[にち 
\\	じかん] 労働[ろうどう]している。	かれら は 
\\	にち 
\\	じかん ろうどう して いる	
\\	彼[かれ]らは1 日7時間[にち 
\\	じかん]
\\	している。			
\\	疲労	疲労[ひろう]	ひろう	
\\	部下が疲労で倒れたの。	部下[ぶか]が 疲労[ひろう]で 倒[たお]れたの。	ぶか が ひろう で たおれた の	
\\	部下[ぶか]が
\\	で 倒[たお]れたの。			
\\	労力	労力[ろうりょく]	ろうりょく	
\\	仕事には時間と労力が必要です。	仕事[しごと]には 時間[じかん]と 労力[ろうりょく]が 必要[ひつよう]です。	しごと に は じかん と ろうりょく が ひつよう です	
\\	仕事[しごと]には 時間[じかん]と
\\	が 必要[ひつよう]です。			
\\	貧しい	貧[まず]しい	まずしい	
\\	彼は貧しい家庭で育ったんだ。	彼[かれ]は 貧[まず]しい 家庭[かてい]で 育[そだ]ったんだ。	かれ は まずしい かてい で そだった ん だ	
\\	彼[かれ]は
\\	家庭[かてい]で 育[そだ]ったんだ。			
\\	ブラシ	ブラシ	ブラシ	
\\	犬の毛をブラシでとかしてやったの。	犬[いぬ]の 毛[け]をブラシでとかしてやったの。	いぬ の け を ぶらし で とかして やった の	
\\	犬[いぬ]の 毛[け]を
\\	でとかしてやったの。			
\\	乏しい	乏[とぼ]しい	とぼしい	
\\	彼女はまだ仕事の経験が乏しい。	彼女[かのじょ]はまだ 仕事[しごと]の 経験[けいけん]が 乏[とぼ]しい。	かのじょ は まだ しごと の けいけん が とぼしい。	
\\	彼女[かのじょ]はまだ 仕事[しごと]の 経験[けいけん]が
\\	貧乏人	貧乏人[びんぼうにん]	びんぼうにん	
\\	貧乏人が大金持ちになることもあるさ。	貧乏人[びんぼうにん]が 大金持[おおがねも]ちになることもあるさ。	びんぼうにん が おおがねもち に なる こと も ある さ	
\\	が 大金持[おおがねも]ちになることもあるさ。			
\\	辛い	辛[つら]い	つらい	
\\	この仕事は辛いです。	この 仕事[しごと]は 辛[つら]いです。	この しごと は つらい です	
\\	この 仕事[しごと]は
\\	です。			
\\	不幸せ	不幸[ふしあわ]せ	ふしあわせ	
\\	彼ほど不幸せな人はいないな。	彼[かれ]ほど 不幸[ふしあわ]せな 人[ひと]はいないな。	かれ ほど ふしあわせ な ひと は いない な	
\\	彼[かれ]ほど
\\	な 人[ひと]はいないな。			
\\	福祉	福祉[ふくし]	ふくし	
\\	姉は福祉の勉強をしています。	姉[あね]は 福祉[ふくし]の 勉強[べんきょう]をしています。	あね は ふくし の べんきょう を して います	
\\	姉[あね]は
\\	の 勉強[べんきょう]をしています。			
\\	マヨネーズ	マヨネーズ	マヨネーズ	
\\	サラダにマヨネーズをかけたの。	サラダにマヨネーズをかけたの。	さらだ に まよねーず を かけた の	
\\	サラダに
\\	をかけたの。			
\\	模様	模様[もよう]	もよう	
\\	彼女は水玉模様のスカートをはいているね。	彼女[かのじょ]は 水玉[みずたま] 模様[もよう]のスカートをはいているね。	かのじょ は みずたま もよう の すかーと を はいて いる ね	
\\	彼女[かのじょ]は 水玉[みずたま]
\\	のスカートをはいているね。			
\\	模範	模範[もはん]	もはん	
\\	彼は全校生徒の模範です。	彼[かれ]は 全校生徒[ぜんこう せいと]の 模範[もはん]です。	かれ は ぜんこう せいと の もはん です	
\\	彼[かれ]は 全校生徒[ぜんこう せいと]の
\\	です。			
\\	農民	農民[のうみん]	のうみん	
\\	中世の農民は貧しかった。	中世[ちゅうせい]の 農民[のうみん]は 貧[まず]しかった。	ちゅうせい の のうみん は まずしかった	
\\	中世[ちゅうせい]の
\\	は 貧[まず]しかった。			
\\	農家	農家[のうか]	のうか	
\\	彼は農家に生まれました。	彼[かれ]は 農家[のうか]に 生[う]まれました。	かれ は のうか に うまれました	
\\	彼[かれ]は
\\	に 生[う]まれました。			
\\	農村	農村[のうそん]	のうそん	
\\	私は農村で育ちました。	私[わたし]は 農村[のうそん]で 育[そだ]ちました。	わたし は のうそん で そだちました	
\\	私[わたし]は
\\	で 育[そだ]ちました。			
\\	低気圧	低気圧[ていきあつ]	ていきあつ	
\\	低気圧が近づいています。	低気圧[ていきあつ]が 近[ちか]づいています。	ていきあつ が ちかづいて います	
\\	が 近[ちか]づいています。			
\\	だらしない	だらしない	だらしない	
\\	だらしない格好をしないでください。	だらしない 格好[かっこう]をしないでください。	だらしない かっこう を しない で ください	
\\	格好[かっこう]をしないでください。			
\\	短縮	短縮[たんしゅく]	たんしゅく	
\\	今日は授業を1時間に短縮します。	今日[きょう]は 授業[じゅぎょう]を1 時間[じかん]に 短縮[たんしゅく]します。	きょう は じゅぎょう を 
\\	じかん に たんしゅく します	
\\	今日[きょう]は 授業[じゅぎょう]を1 時間[じかん]に
\\	します。			
\\	縮める	縮[ちぢ]める	ちぢめる	
\\	彼はタイムを1秒縮めたの。	彼[かれ]はタイムを1 秒[びょう] 縮[ちぢ]めたの。	かれ は たいむ を 
\\	びょう ちぢめた の	
\\	彼[かれ]はタイムを1 秒[びょう]
\\	の。			
\\	縮む	縮[ちぢ]む	ちぢむ	
\\	この服の縮み具合はひどい。	この 服[ふく]の 縮[ちぢ]み 具合[ぐあい]はひどい。	この ふく の ちぢみ ぐあい は ひどい	
\\	この 服[ふく]の
\\	具合[ぐあい]はひどい。			
\\	縮み	縮[ちぢ]み	ちぢみ	
\\	この服の縮み具合はひどい。	この 服[ふく]の 縮[ちぢ]み 具合[ぐあい]はひどい。	この ふく の ちぢみ ぐあい は ひどい	
\\	この 服[ふく]の
\\	具合[ぐあい]はひどい。			
\\	縮まる	縮[ちぢ]まる	ちぢまる	
\\	兄との身長の差が縮まった。	兄[あに]との 身長[しんちょう]の 差[さ]が 縮[ちぢ]まった。	あに と の しんちょう の さ が ちぢまった	
\\	兄[あに]との 身長[しんちょう]の 差[さ]が
\\	とうもろこし	とうもろこし	とうもろこし	
\\	このとうもろこしは甘くておいしい。	このとうもろこしは 甘[あま]くておいしい。	この とうもろこし は あまく て おいしい	
\\	この
\\	は 甘[あま]くておいしい。			
\\	伸ばす	伸[の]ばす	のばす	
\\	ストレッチで筋肉を伸ばしましょう。	ストレッチで 筋肉[きんにく]を 伸[の]ばしましょう。	すとれっち で きんにく を のばしましょう	
\\	ストレッチで 筋肉[きんにく]を
\\	引き伸ばす	引[ひ]き 伸[の]ばす	ひきのばす	
\\	この写真を引き伸ばしてください。	この 写真[しゃしん]を 引[ひ]き 伸[の]ばしてください。	この しゃしん を ひきのばして ください	
\\	この 写真[しゃしん]を
\\	ください。			
\\	追伸	追伸[ついしん]	ついしん	
\\	追伸、お兄さんはお元気ですか。	追伸[ついしん]、お 兄[にい]さんはお 元気[げんき]ですか。	ついしん おにいさん は おげんき です か	
\\	、お 兄[にい]さんはお 元気[げんき]ですか。			
\\	倍	倍[ばい]	ばい	
\\	以前の収入は今の倍はあった。	以前[いぜん]の 収入[しゅうにゅう]は 今[いま]の 倍[ばい]はあった。	いぜん の しゅうにゅう は いま の ばい は あった	
\\	以前[いぜん]の 収入[しゅうにゅう]は 今[いま]の
\\	はあった。			
\\	超過	超過[ちょうか]	ちょうか	
\\	この荷物は重量超過です。	この 荷物[にもつ]は 重量[じゅうりょう] 超過[ちょうか]です。	この にもつ は じゅうりょう ちょうか です	
\\	この 荷物[にもつ]は 重量[じゅうりょう]
\\	です。			
\\	ひらひら	ひらひら	ひらひら	
\\	木の葉がひらひらと落ちたの。	木[こ]の 葉[は]がひらひらと 落[お]ちたの。	このは が ひらひら と おちた の	
\\	木[こ]の 葉[は]が
\\	と 落[お]ちたの。			
\\	乗り越える	乗[の]り 越[こ]える	のりこえる	
\\	彼は悲しみを乗り越えて強く生きた。	彼[かれ]は 悲[かな]しみを 乗[の]り 越[こ]えて 強[つよ]く 生[い]きた。	かれ は かなしみ を のりこえて つよく いきた	
\\	彼[かれ]は 悲[かな]しみを
\\	強[つよ]く 生[い]きた。			
\\	乗り越し	乗[の]り 越[こ]し	のりこし	
\\	乗り越しを機械で精算したの。	乗[の]り 越[こ]しを 機械[きかい]で 精算[せいさん]したの。	のりこし を きかい で せいさん した の	
\\	を 機械[きかい]で 精算[せいさん]したの。			
\\	乗り越す	乗[の]り 越[こ]す	のりこす	
\\	居眠りして降りる駅を乗り越した。	居眠[いねむ]りして 降[お]りる 駅[えき]を 乗[の]り 越[こ]した。	いねむり して おりる えき を のりこした	
\\	居眠[いねむ]りして 降[お]りる 駅[えき]を
\\	抜ける	抜[ぬ]ける	ぬける	
\\	彼はグループから抜けました。	彼[かれ]はグループから 抜[ぬ]けました。	かれ は ぐるーぷ から ぬけました	
\\	彼[かれ]はグループから
\\	抜く	抜[ぬ]く	ぬく	
\\	ワインのコルクを抜きました。	ワインのコルクを 抜[ぬ]きました。	わいん の こるく を ぬきました	
\\	ワインのコルクを
\\	ぽかぽか	ぽかぽか	ぽかぽか	
\\	今日はぽかぽか暖かい日です。	今日[きょう]はぽかぽか 暖[あたた]かい 日[ひ]です。	きょう は ぽかぽか あたたかい ひ です	
\\	今日[きょう]は
\\	暖[あたた]かい 日[ひ]です。			
\\	昇る	昇[のぼ]る	のぼる	
\\	太陽は東から昇ります。	太陽[たいよう]は 東[ひがし]から 昇[のぼ]ります。	たいよう は ひがし から のぼります	
\\	太陽[たいよう]は 東[ひがし]から
\\	膨らむ	膨[ふく]らむ	ふくらむ	
\\	桜のつぼみが膨らんだのね。	桜[さくら]のつぼみが 膨[ふく]らんだのね。	さくら の つぼみ が ふくらんだ の ね	
\\	桜[さくら]のつぼみが
\\	のね。			
\\	膨れる	膨[ふく]れる	ふくれる	
\\	お腹が膨れたら眠くなった。	お 腹[なか]が 膨[ふく]れたら 眠[ねむ]くなった。	おなか が ふくれたら ねむく なった	
\\	お 腹[なか]が
\\	眠[ねむ]くなった。			
\\	札	札[ふだ]	ふだ	
\\	店の外にまだ営業中の札がでているよ。	店[みせ]の 外[そと]にまだ 営業中[えいぎょう ちゅう]の 札[ふだ]がでているよ。	みせ の そと に まだ えいぎょう ちゅう の ふだ が でている よ	
\\	店[みせ]の 外[そと]にまだ 営業中[えいぎょう ちゅう]の
\\	がでているよ。			
\\	名札	名札[なふだ]	なふだ	
\\	生徒たちは校内では名札をつけます。	生徒[せいと]たちは 校内[こうない]では 名札[なふだ]をつけます。	せいとたち は こうない で は なふだ を つけます	
\\	生徒[せいと]たちは 校内[こうない]では
\\	をつけます。			
\\	礼	礼[れい]	れい	
\\	先生にお礼の手紙を書きました。	先生[せんせい]にお 礼[れい]の 手紙[てがみ]を 書[か]きました。	せんせい に おれい の てがみ を かきました	
\\	先生[せんせい]にお
\\	の 手紙[てがみ]を 書[か]きました。			
\\	よる	よる	よる	
\\	人は見かけによらないな。	人[ひと]は 見[み]かけによらないな。	ひと は みかけ に よらない な	
\\	人[ひと]は 見[み]かけに
\\	な。			
\\	反射	反射[はんしゃ]	はんしゃ	
\\	車のライトが反射していますね。	車[くるま]のライトが 反射[はんしゃ]していますね。	くるま の らいと が はんしゃ して います ね	
\\	車[くるま]のライトが
\\	していますね。			
\\	日程	日程[にってい]	にってい	
\\	試験の日程が発表されました。	試験[しけん]の 日程[にってい]が 発表[はっぴょう]されました。	しけん の にってい が はっぴょう されました	
\\	試験[しけん]の
\\	が 発表[はっぴょう]されました。			
\\	程	程[ほど]	ほど	
\\	冗談にも程がある。	冗談[じょうだん]にも 程[ほど]がある。	じょうだん に も ほど が ある	
\\	冗談[じょうだん]にも
\\	がある。			
\\	優勝	優勝[ゆうしょう]	ゆうしょう	
\\	私たちのチームが優勝しました。	私[わたし]たちのチームが 優勝[ゆうしょう]しました。	わたしたち の ちーむ が ゆうしょう しました	
\\	私[わたし]たちのチームが
\\	しました。			
\\	優秀	優秀[ゆうしゅう]	ゆうしゅう	
\\	彼はとても優秀な生徒です。	彼[かれ]はとても 優秀[ゆうしゅう]な 生徒[せいと]です。	かれ は とても ゆうしゅう な せいと です	
\\	彼[かれ]はとても
\\	な 生徒[せいと]です。			
\\	だぶだぶ	だぶだぶ	だぶだぶ	
\\	この服は大き過ぎてだぶだぶです。	この 服[ふく]は 大[おお]き 過[す]ぎてだぶだぶです。	この ふく は おおき すぎて だぶだぶ です	
\\	この 服[ふく]は 大[おお]き 過[す]ぎて
\\	です。			
\\	透明	透明[とうめい]	とうめい	
\\	ゴミは透明な袋に入れて出してください。	ゴミは 透明[とうめい]な 袋[ふくろ]に 入[い]れて 出[だ]してください。	ごみ は とうめい な ふくろ に いれて だして ください	
\\	ゴミは
\\	な 袋[ふくろ]に 入[い]れて 出[だ]してください。			
\\	半導体	半導体[はんどうたい]	はんどうたい	
\\	半導体は様々な製品に使われているよ。	半導体[はんどうたい]は 様々[さまざま]な 製品[せいひん]に 使[つか]われているよ。	はんどうたい は さまざま な せいひん に つかわれて いる よ	
\\	は 様々[さまざま]な 製品[せいひん]に 使[つか]われているよ。			
\\	導く	導[みちび]く	みちびく	
\\	先生は私たちを導いてくれます。	先生[せんせい]は 私[わたし]たちを 導[みちび]いてくれます。	せんせい は わたしたち を みちびいて くれます	
\\	先生[せんせい]は 私[わたし]たちを
\\	くれます。			
\\	望む	望[のぞ]む	のぞむ	
\\	彼は私との結婚を望んでいます。	彼[かれ]は 私[わたし]との 結婚[けっこん]を 望[のぞ]んでいます。	かれ は わたし と の けっこん を のぞんで います	
\\	彼[かれ]は 私[わたし]との 結婚[けっこん]を
\\	望ましい	望[のぞ]ましい	のぞましい	
\\	夜は10時までに寝るのが望ましいの。	夜[よる]は10 時[じ]までに 寝[ね]るのが 望[のぞ]ましいの。	よる は 
\\	じ まで に ねる の が のぞましい の	
\\	夜[よる]は10 時[じ]までに 寝[ね]るのが
\\	の。			
\\	つるつる	つるつる	つるつる	
\\	床がつるつるすべります。	床[ゆか]がつるつるすべります。	ゆか が つるつる すべります	
\\	床[ゆか]が
\\	すべります。			
\\	要望	要望[ようぼう]	ようぼう	
\\	お客様の要望を聞かせてください。	お 客様[きゃくさま]の 要望[ようぼう]を 聞[き]かせてください。	おきゃくさま の ようぼう を きかせて ください	
\\	お 客様[きゃくさま]の
\\	を 聞[き]かせてください。			
\\	望み	望[のぞ]み	のぞみ	
\\	私の望みは海外で暮らすことです。	私[わたし]の 望[のぞ]みは 海外[かいがい]で 暮[く]らすことです。	わたし の のぞみ は かいがい で くらす こと です	
\\	私[わたし]の
\\	は 海外[かいがい]で 暮[く]らすことです。			
\\	有望	有望[ゆうぼう]	ゆうぼう	
\\	彼は有望な社員です。	彼[かれ]は 有望[ゆうぼう]な 社員[しゃいん]です。	かれ は ゆうぼう な しゃいん です	
\\	彼[かれ]は
\\	な 社員[しゃいん]です。			
\\	欲望	欲望[よくぼう]	よくぼう	
\\	彼は欲望が強い人です。	彼[かれ]は 欲望[よくぼう]が 強[つよ]い 人[ひと]です。	かれ は よくぼう が つよい ひと です	
\\	彼[かれ]は
\\	が 強[つよ]い 人[ひと]です。			
\\	待ち望む	待[ま]ち 望[のぞ]む	まちのぞむ	
\\	その国の人々は平和を待ち望んでいるの。	その 国[くに]の 人々[ひとびと]は 平和[へいわ]を 待[ま]ち 望[のぞ]んでいるの。	その くに の ひとびと は へいわ を まちのぞんで いる の	
\\	その 国[くに]の 人々[ひとびと]は 平和[へいわ]を
\\	の。			
\\	努める	努[つと]める	つとめる	
\\	良い成績が取れるように努めます。	良[い]い 成績[せいせき]が 取[と]れるように 努[つと]めます。	いい せいせき が とれる よう に つとめます	
\\	良[い]い 成績[せいせき]が 取[と]れるように
\\	トースト	トースト	トースト	
\\	私は毎朝トーストを2枚食べます。	私[わたし]は 毎朝[まいあさ]トーストを2 枚食[まい た]べます。	わたし は まいあさ とーすと を 
\\	まい たべます	
\\	私[わたし]は 毎朝[まいあさ]
\\	を2 枚食[まい た]べます。			
\\	独立	独立[どくりつ]	どくりつ	
\\	先日会社から独立しました。	先日会社[せんじつ かいしゃ]から 独立[どくりつ]しました。	せんじつ かいしゃ から どくりつ しました	
\\	先日会社[せんじつ かいしゃ]から
\\	しました。			
\\	独特	独特[どくとく]	どくとく	
\\	彼の服のセンスは独特よね。	彼[かれ]の 服[ふく]のセンスは 独特[どくとく]よね。	かれ の ふく の せんす は どくとく よ ね	
\\	彼[かれ]の 服[ふく]のセンスは
\\	よね。			
\\	独り言	独[ひと]り 言[ごと]	ひとりごと	
\\	彼女はいつも独り言を言うんだ。	彼女[かのじょ]はいつも 独[ひと]り 言[ごと]を 言[い]うんだ。	かのじょ は いつも ひとりごと を いう ん だ	
\\	彼女[かのじょ]はいつも
\\	を 言[い]うんだ。			
\\	身	身[み]	み	
\\	旅行中は身の安全が第一です。	旅行中[りょこうちゅう]は 身[み]の 安全[あんぜん]が 第一[だいいち]です。	りょこうちゅう は み の あんぜん が だいいち です	
\\	旅行中[りょこうちゅう]は
\\	の 安全[あんぜん]が 第一[だいいち]です。			
\\	中身	中身[なかみ]	なかみ	
\\	かばんの中身を見せてください。	かばんの 中身[なかみ]を 見[み]せてください。	かばん の なかみ を みせて ください	
\\	かばんの
\\	を 見[み]せてください。			
\\	どきっと	どきっと	どきっと	
\\	突然の質問にどきっとしたよ。	突然[とつぜん]の 質問[しつもん]にどきっとしたよ。	とつぜん の しつもん に どきっと した よ	
\\	突然[とつぜん]の 質問[しつもん]に
\\	したよ。			
\\	身近	身近[みぢか]	みぢか	
\\	私にとって動物は身近な存在です。	私[わたし]にとって 動物[どうぶつ]は 身近[みぢか]な 存在[そんざい]です。	わたし に とって どうぶつ は みぢか な そんざい です	
\\	私[わたし]にとって 動物[どうぶつ]は
\\	な 存在[そんざい]です。			
\\	身の回り	身[み]の 回[まわ]り	みのまわり	
\\	身の回りのお世話は私がします。	身[み]の 回[まわ]りのお 世話[せわ]は 私[わたし]がします。	みのまわり の おせわ は わたし が します 。	
\\	のお 世話[せわ]は 私[わたし]がします。			
\\	身なり	身[み]なり	みなり	
\\	身なりのいい人が入ってきた。	身[み]なりのいい 人[ひと]が 入[はい]ってきた。	みなり の いい ひと が はいって きた	
\\	のいい 人[ひと]が 入[はい]ってきた。			
\\	身振り	身振[みぶ]り	みぶり	
\\	彼は身振りを交えて説明してくれたの。	彼[かれ]は 身振[みぶ]りを 交[まじ]えて 説明[せつめい]してくれたの。	かれ は みぶり を まじえて せつめい して くれた の	
\\	彼[かれ]は
\\	を 交[まじ]えて 説明[せつめい]してくれたの。			
\\	独占	独占[どくせん]	どくせん	
\\	この2社が市場を独占しています。	この2 社[しゃ]が 市場[しじょう]を 独占[どくせん]しています。	この 
\\	しゃ が しじょう を どくせん して います	
\\	この2 社[しゃ]が 市場[しじょう]を
\\	しています。			
\\	のろのろ	のろのろ	のろのろ	
\\	車は雪道をのろのろと走ったの。	車[くるま]は 雪道[ゆきみち]をのろのろと 走[はし]ったの。	くるま は ゆきみち を のろのろ と はしった の	
\\	車[くるま]は 雪道[ゆきみち]を
\\	と 走[はし]ったの。			
\\	仲良く	仲良[なかよ]く	なかよく	
\\	あの夫婦は仲良く暮らしているよ。	あの 夫婦[ふうふ]は 仲良[なかよ]く 暮[く]らしているよ。	あの ふうふ は なかよく くらして いる よ	
\\	あの 夫婦[ふうふ]は
\\	暮[く]らしているよ。			
\\	仲良し	仲良[なかよ]し	なかよし	
\\	あの3人組は仲良しですね。	あの3 人組[にんぐみ]は 仲良[なかよ]しですね。	あの 
\\	にんぐみ は なかよし です ね	
\\	あの3 人組[にんぐみ]は
\\	ですね。			
\\	仲直り	仲直[なかなお]り	なかなおり	
\\	けんかした友達と仲直りしました。	けんかした 友達[ともだち]と 仲直[なかなお]りしました。	けんか した ともだち と なかなおり しました	
\\	けんかした 友達[ともだち]と
\\	しました。			
\\	仲	仲[なか]	なか	
\\	あなたたちは仲がいいですね。	あなたたちは 仲[なか]がいいですね。	あなたたち は なか が いい です ね	
\\	あなたたちは
\\	がいいですね。			
\\	仲間	仲間[なかま]	なかま	
\\	彼には仲間がたくさんいる。	彼[かれ]には 仲間[なかま]がたくさんいる。	かれ に は なかま が たくさん いる	
\\	彼[かれ]には
\\	がたくさんいる。			
\\	仲人	仲人[なこうど]	なこうど	
\\	あの夫婦は私たちの仲人です。	あの 夫婦[ふうふ]は 私[わたし]たちの 仲人[なこうど]です。	あの ふうふ は わたしたち の なこうど です	
\\	あの 夫婦[ふうふ]は 私[わたし]たちの
\\	です。			
\\	ぴかぴか	ぴかぴか	ぴかぴか	
\\	新車はぴかぴかですね。	新車[しんしゃ]はぴかぴかですね。	しんしゃ は ぴかぴか です ね	
\\	新車[しんしゃ]は
\\	ですね。			
\\	照らす	照[て]らす	てらす	
\\	月が庭を照らしている。	月[つき]が 庭[にわ]を 照[て]らしている。	つき が にわ を てらして いる	
\\	月[つき]が 庭[にわ]を
\\	照る	照[て]る	てる	
\\	日差しが強く照りつけますね。	日差[ひざ]しが 強[つよ]く 照[て]りつけますね。	ひざし が つよく てりつけます ね	
\\	日差[ひざ]しが 強[つよ]く
\\	ね。			
\\	夫人	夫人[ふじん]	ふじん	
\\	スミス夫人がいらっしゃいました。	スミス 夫人[ふじん]がいらっしゃいました。	すみすふじん が いらっしゃいました	
\\	スミス
\\	がいらっしゃいました。			
\\	婦人	婦人[ふじん]	ふじん	
\\	婦人服売り場は5階でございます。	婦人[ふじん] 服売[ふく う]り 場[ば]は5 階[かい]でございます。	ふじんふく うりば は 
\\	かい で ございます	
\\	服売[ふく う]り 場[ば]は5 階[かい]でございます。			
\\	夫妻	夫妻[ふさい]	ふさい	
\\	昨日の夜、社長ご夫妻と食事をしました。	昨日[きのう]の 夜[よる]、 社長[しゃちょう]ご 夫妻[ふさい]と 食事[しょくじ]をしました。	きのう の よる しゃちょうごふさい と しょくじ を しました	
\\	昨日[きのう]の 夜[よる]、 社長[しゃちょう]ご
\\	と 食事[しょくじ]をしました。			
\\	ぺこぺこ	ぺこぺこ	ぺこぺこ	
\\	彼は上司にぺこぺこしている。	彼[かれ]は 上司[じょうし]にぺこぺこしている。	かれ は じょうし に ぺこぺこ して いる	
\\	彼[かれ]は 上司[じょうし]に
\\	している。			
\\	相互	相互[そうご]	そうご	
\\	チームの中では相互の信頼が大切です。	チームの 中[なか]では 相互[そうご]の 信頼[しんらい]が 大切[たいせつ]です。	ちーむ の なか で は そうご の しんらい が たいせつ です	
\\	チームの 中[なか]では
\\	の 信頼[しんらい]が 大切[たいせつ]です。			
\\	互い	互[たが]い	たがい	
\\	互いの話をよく聞きなさい。	互[たが]いの 話[はなし]をよく 聞[き]きなさい。	たがい の はなし を よく ききなさい	
\\	の 話[はなし]をよく 聞[き]きなさい。			
\\	皆	皆[みな]	みな	
\\	私の昇進を皆が喜んでくれたの。	私[わたし]の 昇進[しょうしん]を 皆[みな]が 喜[よろこ]んでくれたの。	わたし の しょうしん を みな が よろこんで くれた の	
\\	私[わたし]の 昇進[しょうしん]を
\\	が 喜[よろこ]んでくれたの。			
\\	我々	我々[われわれ]	われわれ	
\\	我々の決意は固いです。	我々[われわれ]の 決意[けつい]は 固[かた]いです。	われわれ の けつい は かたい です	
\\	の 決意[けつい]は 固[かた]いです。			
\\	我が国	我[わ]が 国[くに]	わがくに	
\\	彼は我が国を代表する作家です。	彼[かれ]は 我[わ]が 国[くに]を 代表[だいひょう]する 作家[さっか]です。	かれ は わがくに を だいひょう する さっか です	
\\	彼[かれ]は
\\	を 代表[だいひょう]する 作家[さっか]です。			
\\	ぺらぺら	ぺらぺら	ぺらぺら	
\\	この本は薄くてぺらぺらですね。	この 本[ほん]は 薄[うす]くてぺらぺらですね。	この ほん は うすくて ぺらぺら です ね	
\\	この 本[ほん]は 薄[うす]くて
\\	ですね。			
\\	我が家	我[わ]が 家[や]	わがや	
\\	ぜひ我が家に遊びに来てください。	ぜひ 我[わ]が 家[や]に 遊[あそ]びに 来[き]てください。	ぜひ わがや に あそび に きて ください 。	
\\	ぜひ
\\	に 遊[あそ]びに 来[き]てください。			
\\	年齢	年齢[ねんれい]	ねんれい	
\\	彼女の年齢は27です。	彼女[かのじょ]の 年齢[ねんれい]は27です。	かのじょ の ねんれい は 
\\	です	
\\	彼女[かのじょ]の
\\	は27です。			
\\	恋愛	恋愛[れんあい]	れんあい	
\\	彼女は恋愛にあこがれる年ごろです。	彼女[かのじょ]は 恋愛[れんあい]にあこがれる 年[とし]ごろです。	かのじょ は れんあい に あこがれる とし ごろ です	
\\	彼女[かのじょ]は
\\	にあこがれる 年[とし]ごろです。			
\\	初恋	初恋[はつこい]	はつこい	
\\	私の初恋は小学生の時です。	私[わたし]の 初恋[はつこい]は 小学生[しょうがくせい]の 時[とき]です。	わたし の はつこい は しょうがくせい の とき です	
\\	私[わたし]の
\\	は 小学生[しょうがくせい]の 時[とき]です。			
\\	誕生	誕生[たんじょう]	たんじょう	
\\	先月、娘が誕生しました。	先月[せんげつ]、 娘[むすめ]が 誕生[たんじょう]しました。	せんげつ むすめ が たんじょう しました	
\\	先月[せんげつ]、 娘[むすめ]が
\\	しました。			
\\	ほこり	ほこり	ほこり	
\\	この部屋はほこりだらけですね。	この 部屋[へや]はほこりだらけですね。	この へや は ほこり だらけ です ね 。	
\\	この 部屋[へや]は
\\	だらけですね。			
\\	延びる	延[の]びる	のびる	
\\	工事の予定が1ヶ月延びてしまった。	工事[こうじ]の 予定[よてい]が1 ヶ月[かげつ] 延[の]びてしまった。	こうじ の よてい が 
\\	かげつ のびて しまった	
\\	工事[こうじ]の 予定[よてい]が1 ヶ月[かげつ]
\\	しまった。			
\\	延ばす	延[の]ばす	のばす	
\\	出発を一週間延ばしたの。	出発[しゅっぱつ]を 一週間[いっしゅうかん] 延[の]ばしたの。	しゅっぱつ を いっしゅうかん のばした の	
\\	出発[しゅっぱつ]を 一週間[いっしゅうかん]
\\	の。			
\\	引き延ばす	引[ひ]き 延[の]ばす	ひきのばす	
\\	司会者は話を引き延ばしたわ。	司会者[しかいしゃ]は 話[はなし]を 引[ひ]き 延[の]ばしたわ。	しかいしゃ は はなし を ひきのばした わ	
\\	司会者[しかいしゃ]は 話[はなし]を
\\	わ。			
\\	単純	単純[たんじゅん]	たんじゅん	
\\	彼は単純な人です。	彼[かれ]は 単純[たんじゅん]な 人[ひと]です。	かれ は たんじゅん な ひと です	
\\	彼[かれ]は
\\	な 人[ひと]です。			
\\	夢中	夢中[むちゅう]	むちゅう	
\\	うちの子はその本に夢中です。	うちの 子[こ]はその 本[ほん]に 夢中[むちゅう]です。	うち の こ は その ほん に むちゅう です	
\\	うちの 子[こ]はその 本[ほん]に
\\	です。			
\\	泣き顔	泣[な]き 顔[がお]	なきがお	
\\	彼女は泣き顔になったの。	彼女[かのじょ]は 泣[な]き 顔[がお]になったの。	かのじょ は なきがお に なった の	
\\	彼女[かのじょ]は
\\	になったの。			
\\	ポット	ポット	ポット	
\\	お湯はポットに入っています。	お 湯[ゆ]はポットに 入[はい]っています。	おゆ は ぽっと に はいって います	
\\	お 湯[ゆ]は
\\	に 入[はい]っています。			
\\	笑い	笑[わら]い	わらい	
\\	あの家は笑いが絶えないね。	あの 家[いえ]は 笑[わら]いが 絶[た]えないね。	あの いえ は わらい が たえない ね	
\\	あの 家[いえ]は
\\	が 絶[た]えないね。			
\\	喜び	喜[よろこ]び	よろこび	
\\	人々は喜びに沸いた。	人々[ひとびと]は 喜[よろこ]びに 沸[わ]いた。	ひとびと は よろこび に わいた	
\\	人々[ひとびと]は
\\	に 沸[わ]いた。			
\\	喜ばす	喜[よろこ]ばす	よろこばす	
\\	私は人を喜ばすのが大好きです。	私[わたし]は 人[ひと]を 喜[よろこ]ばすのが 大好[だいす]きです。	わたし は ひと を よろこばす の が だいすき です	
\\	私[わたし]は 人[ひと]を
\\	のが 大好[だいす]きです。			
\\	恥	恥[はじ]	はじ	
\\	間違えることは恥ではありません。	間違[まちが]えることは 恥[はじ]ではありません。	まちがえる こと は はじ で は ありません	
\\	間違[まちが]えることは
\\	ではありません。			
\\	弁論	弁論[べんろん]	べんろん	
\\	弁論大会で優勝したことがあります。	弁論[べんろん] 大会[たいかい]で 優勝[ゆうしょう]したことがあります。	べんろん たいかい で ゆうしょう した こと が あります	
\\	大会[たいかい]で 優勝[ゆうしょう]したことがあります。			
\\	ミュージック	ミュージック	ミュージック	
\\	彼はソウルミュージックをよく聞くの。	彼[かれ]はソウルミュージックをよく 聞[き]くの。	かれ は そうる みゅーじっく を よく きく の	
\\	彼[かれ]はソウル
\\	をよく 聞[き]くの。			
\\	保護	保護[ほご]	ほご	
\\	みんなで環境を保護しましょう。	みんなで 環境[かんきょう]を 保護[ほご]しましょう。	みんな で かんきょう を ほご しましょう	
\\	みんなで 環境[かんきょう]を
\\	しましょう。			
\\	弁護	弁護[べんご]	べんご	
\\	友人が私を弁護してくれました。	友人[ゆうじん]が 私[わたし]を 弁護[べんご]してくれました。	ゆうじん が わたし を べんご して くれました	
\\	友人[ゆうじん]が 私[わたし]を
\\	してくれました。			
\\	保護者	保護者[ほごしゃ]	ほごしゃ	
\\	私はこの子の保護者です。	私[わたし]はこの 子[こ]の 保護者[ほごしゃ]です。	わたし は この こ の ほごしゃ です	
\\	私[わたし]はこの 子[こ]の
\\	です。			
\\	同士	同士[どうし]	どうし	
\\	彼と私はいとこ同士です。	彼[かれ]と 私[わたし]はいとこ 同士[どうし]です。	かれ と わたし は いとこ どうし です	
\\	彼[かれ]と 私[わたし]はいとこ
\\	です。			
\\	弁護士	弁護士[べんごし]	べんごし	
\\	父は弁護士です。	父[ちち]は 弁護士[べんごし]です。	ちち は べんごし です	
\\	父[ちち]は
\\	です。			
\\	ロマンチック	ロマンチック	ロマンチック	
\\	彼女はロマンチックな人です。	彼女[かのじょ]はロマンチックな 人[ひと]です。	かのじょ は ろまんちっく な ひと です	
\\	彼女[かのじょ]は
\\	な 人[ひと]です。			
\\	否定	否定[ひてい]	ひてい	
\\	彼、友達の意見を否定した。	彼[かれ]、 友達[ともだち]の 意見[いけん]を 否定[ひてい]した。	かれ ともだち の いけん を ひてい した	
\\	彼[かれ]、 友達[ともだち]の 意見[いけん]を
\\	した。			
\\	総裁	総裁[そうさい]	そうさい	
\\	あの人は日本銀行の総裁です。	あの 人[ひと]は 日本銀行[にっぽんぎんこう]の 総裁[そうさい]です。	あの ひと は にっぽんぎんこう の そうさい です	
\\	あの 人[ひと]は 日本銀行[にっぽんぎんこう]の
\\	です。			
\\	迷信	迷信[めいしん]	めいしん	
\\	村の人たちは迷信を信じています。	村[むら]の 人[ひと]たちは 迷信[めいしん]を 信[しん]じています。	むら の ひとたち は めいしん を しんじて います	
\\	村[むら]の 人[ひと]たちは
\\	を 信[しん]じています。			
\\	迷子	迷子[まいご]	まいご	
\\	うちの子が迷子になりました。	うちの 子[こ]が 迷子[まいご]になりました。	うち の こ が まいご に なりました	
\\	うちの 子[こ]が
\\	になりました。			
\\	領域	領域[りょういき]	りょういき	
\\	彼女は料理の腕が素人の領域を超えているわ。	彼女[かのじょ]は 料理[りょうり]の 腕[うで]が 素人[しろうと]の 領域[りょういき]を 超[こ]えているわ。	かのじょ は りょうり の うで が しろうと の りょういき を こえて いる わ	
\\	彼女[かのじょ]は 料理[りょうり]の 腕[うで]が 素人[しろうと]の
\\	を 超[こ]えているわ。			
\\	謎	謎[なぞ]	なぞ	
\\	ピラミッドには謎が多いんだ。	ピラミッドには 謎[なぞ]が 多[おお]いんだ。	ぴらみっど に は なぞ が おおい ん だ	
\\	ピラミッドには
\\	が 多[おお]いんだ。			
\\	わくわく	わくわく	わくわく	
\\	遠足が楽しみでわくわくしています。	遠足[えんそく]が 楽[たの]しみでわくわくしています。	えんそく が たのしみ で わくわく して います	
\\	遠足[えんそく]が 楽[たの]しみで
\\	しています。			
\\	著者	著者[ちょしゃ]	ちょしゃ	
\\	この本の著者はイギリス人です。	この 本[ほん]の 著者[ちょしゃ]はイギリス 人[じん]です。	この ほん の ちょしゃ は いぎりすじん です	
\\	この 本[ほん]の
\\	はイギリス 人[じん]です。			
\\	著書	著書[ちょしょ]	ちょしょ	
\\	彼の新しい著書が出版されたね。	彼[かれ]の 新[あたら]しい 著書[ちょしょ]が 出版[しゅっぱん]されたね。	かれ の あたらしい ちょしょ が しゅっぱん された ね	
\\	彼[かれ]の 新[あたら]しい
\\	が 出版[しゅっぱん]されたね。			
\\	偽物	偽物[にせもの]	にせもの	
\\	彼らが売っていたのは偽物だ。	彼[かれ]らが 売[う]っていたのは 偽物[にせもの]だ。	かれら が うって いた の は にせもの だ	
\\	彼[かれ]らが 売[う]っていたのは
\\	だ。			
\\	ノーベル賞	ノーベル 賞[しょう]	ノーベルしょう	
\\	日本人がノーベル賞を取ったよ。	日本人[にほんじん]がノーベル 賞[しょう]を 取[と]ったよ。	にほんじん が のーべるしょう を とった よ	
\\	日本人[にほんじん]が
\\	を 取[と]ったよ。			
\\	文化財	文化財[ぶんかざい]	ぶんかざい	
\\	この建物は国の文化財です。	この 建物[たてもの]は 国[くに]の 文化財[ぶんかざい]です。	この たてもの は くに の ぶんかざい です	
\\	この 建物[たてもの]は 国[くに]の
\\	です。			
\\	わさび	わさび	わさび	
\\	刺身にわさびは欠かせませんね。	刺身[さしみ]にわさびは 欠[か]かせませんね。	さしみ に わさび は かかせません ね	
\\	刺身[さしみ]に
\\	は 欠[か]かせませんね。			
\\	預金	預金[よきん]	よきん	
\\	私はこの銀行に預金しています。	私[わたし]はこの 銀行[ぎんこう]に 預金[よきん]しています。	わたし は この ぎんこう に よきん して います	
\\	私[わたし]はこの 銀行[ぎんこう]に
\\	しています。			
\\	納得	納得[なっとく]	なっとく	
\\	彼の説明で納得できました。	彼[かれ]の 説明[せつめい]で 納得[なっとく]できました。	かれ の せつめい で なっとく できました	
\\	彼[かれ]の 説明[せつめい]で
\\	できました。			
\\	不得意	不得意[ふとくい]	ふとくい	
\\	私は数学が不得意です。	私[わたし]は 数学[すうがく]が 不得意[ふとくい]です。	わたし は すうがく が ふとくい です	
\\	私[わたし]は 数学[すうがく]が
\\	です。			
\\	得する	得[とく]する	とくする	
\\	ネットで得する情報を見つけたよ。	ネットで 得[とく]する 情報[じょうほう]を 見[み]つけたよ。	ねっと で とく する じょうほう を みつけた よ	
\\	ネットで
\\	情報[じょうほう]を 見[み]つけたよ。			
\\	得	得[とく]	とく	
\\	この車を今買うとお得ですよ。	この 車[くるま]を 今買[いま か]うとお 得[とく]ですよ。	この くるま を いま かう と お とく です よ	
\\	この 車[くるま]を 今買[いま か]うとお
\\	ですよ。			
\\	チキン	チキン	チキン	
\\	チキンソテーは私の大好物です。	チキンソテーは 私[わたし]の 大好物[だいこうぶつ]です。	ちきんそてー は わたし の だいこうぶつ です	
\\	ソテーは 私[わたし]の 大好物[だいこうぶつ]です。			
\\	損	損[そん]	そん	
\\	パチンコで5000円損しました。	パチンコで5000 円[えん] 損[そん]しました。	ぱちんこ で 
\\	えん そん しました	
\\	パチンコで5000 円[えん]
\\	しました。			
\\	損害	損害[そんがい]	そんがい	
\\	町は台風で大きな損害を受けたんだ。	町[まち]は 台風[たいふう]で 大[おお]きな 損害[そんがい]を 受[う]けたんだ。	まち は たいふう で おおきな そんがい を うけた ん だ	
\\	町[まち]は 台風[たいふう]で 大[おお]きな
\\	を 受[う]けたんだ。			
\\	損する	損[そん]する	そんする	
\\	わざわざ行って損した。	わざわざ 行[い]って 損[そん]した。	わざわざ いって そん した	
\\	わざわざ 行[い]って
\\	燃料	燃料[ねんりょう]	ねんりょう	
\\	車から燃料がもれていますよ。	車[くるま]から 燃料[ねんりょう]がもれていますよ。	くるま から ねんりょう が もれて います よ	
\\	車[くるま]から
\\	がもれていますよ。			
\\	燃やす	燃[も]やす	もやす	
\\	古い手紙を燃やしたんだ。	古[ふる]い 手紙[てがみ]を 燃[も]やしたんだ。	ふるい てがみ を もやした ん だ	
\\	古[ふる]い 手紙[てがみ]を
\\	んだ。			
\\	日焼け	日焼[ひや]け	ひやけ	
\\	海で日焼けしたんだ。	海[うみ]で 日焼[ひや]けしたんだ。	うみ で ひやけ した ん だ	
\\	海[うみ]で
\\	したんだ。			
\\	とじる	とじる	とじる	
\\	書類はこのファイルにとじてください。	書類[しょるい]はこのファイルにとじてください。	しょるい は この ふぁいる に とじて ください	
\\	書類[しょるい]はこのファイルに
\\	ください。			
\\	焼き肉	焼[や]き 肉[にく]	やきにく	
\\	彼は焼き肉が大好きです。	彼[かれ]は 焼[や]き 肉[にく]が 大好[だいす]きです。	かれ は やきにく が だいすき です	
\\	彼[かれ]は
\\	が 大好[だいす]きです。			
\\	夕焼け	夕焼[ゆうや]け	ゆうやけ	
\\	今日は夕焼けがきれいです。	今日[きょう]は 夕焼[ゆうや]けがきれいです。	きょう は ゆうやけ が きれい です	
\\	今日[きょう]は
\\	がきれいです。			
\\	焼きそば	焼[や]きそば	やきそば	
\\	お祭りで焼きそばを食べたよ。	お 祭[まつ]りで 焼[や]きそばを 食[た]べたよ。	おまつり で やきそば を たべた よ	
\\	お 祭[まつ]りで
\\	を 食[た]べたよ。			
\\	幹	幹[みき]	みき	
\\	この木の幹はとても太いよ。	この 木[き]の 幹[みき]はとても 太[ふと]いよ。	この き の みき は とても ふとい よ	
\\	この 木[き]の
\\	はとても 太[ふと]いよ。			
\\	分散	分散[ぶんさん]	ぶんさん	
\\	その会社はいろいろな国に投資を分散しているね。	その 会社[かいしゃ]はいろいろな 国[くに]に 投資[とうし]を 分散[ぶんさん]しているね。	その かいしゃ は いろいろ な くに に とうし を ぶんさん して いる ね	
\\	その 会社[かいしゃ]はいろいろな 国[くに]に 投資[とうし]を
\\	しているね。			
\\	にやにや	にやにや	にやにや	
\\	なぜか彼はにやにやしています。	なぜか 彼[かれ]はにやにやしています。	なぜ か かれ は にやにや して います	
\\	なぜか 彼[かれ]は
\\	しています。			
\\	散らばる	散[ち]らばる	ちらばる	
\\	机の上に書類が散らばっている。	机[つくえ]の 上[うえ]に 書類[しょるい]が 散[ち]らばっている。	つくえ の うえ に しょるい が ちらばって いる	
\\	机[つくえ]の 上[うえ]に 書類[しょるい]が
\\	散る	散[ち]る	ちる	
\\	風で桜の花が散ってるね。	風[かぜ]で 桜[さくら]の 花[はな]が 散[ち]ってるね。	かぜ で さくら の はな が ちってる ね	
\\	風[かぜ]で 桜[さくら]の 花[はな]が
\\	ね。			
\\	散らかる	散[ち]らかる	ちらかる	
\\	弟の部屋はいつも散らかっているんだ。	弟[おとうと]の 部屋[へや]はいつも 散[ち]らかっているんだ。	おとうと の へや は いつも ちらかって いる ん だ	
\\	弟[おとうと]の 部屋[へや]はいつも
\\	んだ。			
\\	散らかす	散[ち]らかす	ちらかす	
\\	部屋を散らかさないでください。	部屋[へや]を 散[ち]らかさないでください。	へや を ちらかさない で ください	
\\	部屋[へや]を
\\	ください。			
\\	田植え	田植[たう]え	たうえ	
\\	5月は田植えの季節です。	
\\	月[がつ]は 田植[たう]えの 季節[きせつ]です。	
\\	がつ は たうえ の きせつ です	
\\	月[がつ]は
\\	の 季節[きせつ]です。			
\\	ばら	ばら	ばら	
\\	お祝いにばらの花束を贈りました。	お 祝[いわ]いにばらの 花束[はなたば]を 贈[おく]りました。	おいわい に ばら の はなたば を おくりました	
\\	お 祝[いわ]いに
\\	の 花束[はなたば]を 贈[おく]りました。			
\\	根	根[ね]	ね	
\\	この木の根はとても太いな。	この 木[き]の 根[ね]はとても 太[ふと]いな。	この き の ね は とても ふとい な	
\\	この 木[き]の
\\	はとても 太[ふと]いな。			
\\	大根	大根[だいこん]	だいこん	
\\	大根は白くて長い野菜です。	大根[だいこん]は 白[しろ]くて 長[なが]い 野菜[やさい]です。	だいこん は しろくて ながい やさい です	
\\	は 白[しろ]くて 長[なが]い 野菜[やさい]です。			
\\	まな板	まな 板[いた]	まないた	
\\	魚を切った後、まな板を洗ったの。	魚[さかな]を 切[き]った 後[あと]、まな 板[いた]を 洗[あら]ったの。	さかな を きった あと、 まないた を あらった の	
\\	魚[さかな]を 切[き]った 後[あと]、
\\	を 洗[あら]ったの。			
\\	ほうれん草	ほうれん 草[そう]	ほうれんそう	
\\	ほうれん草はビタミンが豊富です。	ほうれん 草[そう]はビタミンが 豊富[ほうふ]です。	ほうれんそう は びたみん が ほうふ です 。	
\\	はビタミンが 豊富[ほうふ]です。			
\\	話し言葉	話[はな]し 言葉[ことば]	はなしことば	
\\	話し言葉と書き言葉は少し違いますね。	話[はな]し 言葉[ことば]と 書[か]き 言葉[ことば]は 少[すこ]し 違[ちが]いますね。	はなしことば と かきことば は すこし ちがいます ね	
\\	と 書[か]き 言葉[ことば]は 少[すこ]し 違[ちが]いますね。			
\\	パンティー	パンティー	パンティー	
\\	白いパンティーを買いました。	白[しろ]いパンティーを 買[か]いました。	しろい ぱんてぃー を かいました	
\\	白[しろ]い
\\	を 買[か]いました。			
\\	葉っぱ	葉[は]っぱ	はっぱ	
\\	もみじの葉っぱが赤くなりましたね。	もみじの 葉[は]っぱが 赤[あか]くなりましたね。	もみじ の はっぱ が あかく なりました ね	
\\	もみじの
\\	が 赤[あか]くなりましたね。			
\\	呼び出す	呼[よ]び 出[だ]す	よびだす	
\\	親が学校に呼び出されたんだ。	親[おや]が 学校[がっこう]に 呼[よ]び 出[だ]されたんだ。	おや が がっこう に よびだされた ん だ	
\\	親[おや]が 学校[がっこう]に
\\	んだ。			
\\	取り扱う	取[と]り 扱[あつか]う	とりあつかう	
\\	当店ではお酒を取り扱っておりません。	当店[とうてん]ではお 酒[さけ]を 取[と]り 扱[あつか]っておりません。	とうてん で は お さけ を とりあつかって おりません	
\\	当店[とうてん]ではお 酒[さけ]を
\\	おりません。			
\\	取り扱い	取[と]り 扱[あつか]い	とりあつかい	
\\	この機械は取り扱いに注意してください。	この 機械[きかい]は 取[と]り 扱[あつか]いに 注意[ちゅうい]してください。	この きかい は とりあつかい に ちゅうい して ください	
\\	この 機械[きかい]は
\\	に 注意[ちゅうい]してください。			
\\	同級生	同級生[どうきゅうせい]	どうきゅうせい	
\\	私たちは同級生です。	私[わたし]たちは 同級生[どうきゅうせい]です。	わたしたち は どうきゅうせい です	
\\	私[わたし]たちは
\\	です。			
\\	中級	中級[ちゅうきゅう]	ちゅうきゅう	
\\	彼は中級レベルの日本語を習っています。	彼[かれ]は 中級[ちゅうきゅう]レベルの 日本語[にほんご]を 習[なら]っています。	かれ は ちゅうきゅう れべる の にほんご を ならって います	
\\	彼[かれ]は
\\	レベルの 日本語[にほんご]を 習[なら]っています。			
\\	ふわふわ	ふわふわ	ふわふわ	
\\	ふわふわの布団に寝たよ。	ふわふわの 布団[ふとん]に 寝[ね]たよ。	ふわふわ の ふとん に ねた よ	
\\	の 布団[ふとん]に 寝[ね]たよ。			
\\	腹	腹[はら]	はら	
\\	腹が減って動けない。	腹[はら]が 減[へ]って 動[うご]けない。	はら が へって うごけない	
\\	が 減[へ]って 動[うご]けない。			
\\	腹一杯	腹一杯[はらいっぱい]	はらいっぱい	
\\	夕食を腹一杯食べた。	夕食[ゆうしょく]を 腹一杯[はらいっぱい] 食[た]べた。	ゆうしょく を はらいっぱい たべた	
\\	夕食[ゆうしょく]を
\\	食[た]べた。			
\\	肺	肺[はい]	はい	
\\	彼は肺の病気にかかったのよ。	彼[かれ]は 肺[はい]の 病気[びょうき]にかかったのよ。	かれ は はい の びょうき に かかった の よ	
\\	彼[かれ]は
\\	の 病気[びょうき]にかかったのよ。			
\\	溶ける	溶[と]ける	とける	
\\	暑さでアイスクリームが溶けてしまった。	暑[あつ]さでアイスクリームが 溶[と]けてしまった。	あつさ で あいすくりーむ が とけて しまった	
\\	暑[あつ]さでアイスクリームが
\\	溶かす	溶[と]かす	とかす	
\\	春の太陽が雪を溶かしました。	春[はる]の 太陽[たいよう]が 雪[ゆき]を 溶[と]かしました。	はる の たいよう が ゆき を とかしました	
\\	春[はる]の 太陽[たいよう]が 雪[ゆき]を
\\	まずい	まずい	まずい	
\\	今日中に返答しなければまずい。	今日中[きょうじゅう]に 返答[へんとう]しなければまずい。	きょうじゅう に へんとう しなければ まずい	
\\	今日中[きょうじゅう]に 返答[へんとう]しなければ
\\	容易	容易[ようい]	ようい	
\\	彼はその問題を容易に解決したわ。	彼[かれ]はその 問題[もんだい]を 容易[ようい]に 解決[かいけつ]したわ。	かれ は その もんだい を ようい に かいけつ した わ	
\\	彼[かれ]はその 問題[もんだい]を
\\	に 解決[かいけつ]したわ。			
\\	容器	容器[ようき]	ようき	
\\	容器のふたはきちんと閉めましょう。	容器[ようき]のふたはきちんと 閉[し]めましょう。	ようき の ふた は きちんと しめましょう	
\\	のふたはきちんと 閉[し]めましょう。			
\\	滑らか	滑[なめ]らか	なめらか	
\\	この生地は滑らかな手触りが特長です。	この 生地[きじ]は 滑[なめ]らかな 手触[てざわ]りが 特長[とくちょう]です。	この きじ は なめらか な てざわり が とくちょう です	
\\	この 生地[きじ]は
\\	な 手触[てざわ]りが 特長[とくちょう]です。			
\\	分析	分析[ぶんせき]	ぶんせき	
\\	今の経済の動きを分析しています。	今[いま]の 経済[けいざい]の 動[うご]きを 分析[ぶんせき]しています。	いま の けいざい の うごき を ぶんせき して います	
\\	今[いま]の 経済[けいざい]の 動[うご]きを
\\	しています。			
\\	保健	保健[ほけん]	ほけん	
\\	保健室で少し休んだわ。	保健[ほけん] 室[しつ]で 少[すこ]し 休[やす]んだわ。	ほけんしつ で すこし やすんだ わ	
\\	室[しつ]で 少[すこ]し 休[やす]んだわ。			
\\	もむ	もむ	もむ	
\\	肩をもんでください。	肩[かた]をもんでください。	かた を もんで ください	
\\	肩[かた]を
\\	ください。			
\\	診る	診[み]る	みる	
\\	今日、医者に診てもらいました。	今日[きょう]、 医者[いしゃ]に 診[み]てもらいました。	きょう 、 いしゃ に みてもらいました。	
\\	今日[きょう]、 医者[いしゃ]に
\\	治療	治療[ちりょう]	ちりょう	
\\	今、歯を治療しています。	今[いま]、 歯[は]を 治療[ちりょう]しています。	いま は を ちりょう して います	
\\	今[いま]、 歯[は]を
\\	しています。			
\\	毒	毒[どく]	どく	
\\	飲み過ぎは体に毒ですよ。	飲[の]み 過[す]ぎは 体[からだ]に 毒[どく]ですよ。	のみ すぎ は からだ に どく です よ	
\\	飲[の]み 過[す]ぎは 体[からだ]に
\\	ですよ。			
\\	中毒	中毒[ちゅうどく]	ちゅうどく	
\\	そのホテルで食中毒が発生したんだ。	そのホテルで 食[しょく] 中毒[ちゅうどく]が 発生[はっせい]したんだ。	その ほてる で しょくちゅうどく が はっせい した ん だ	
\\	そのホテルで 食[しょく]
\\	が 発生[はっせい]したんだ。			
\\	有毒	有毒[ゆうどく]	ゆうどく	
\\	その工場は有毒ガスを出しているのね。	その 工場[こうじょう]は 有毒[ゆうどく]ガスを 出[だ]しているのね。	その こうじょう は ゆうどくがす を だして いる の ね	
\\	その 工場[こうじょう]は
\\	ガスを 出[だ]しているのね。			
\\	ばい菌	ばい 菌[きん]	ばいきん	
\\	傷口にばい菌が入った。	傷口[きずぐち]にばい 菌[きん]が 入[はい]った。	きずぐち に ばいきん が はいった	
\\	傷口[きずぐち]に
\\	が 入[はい]った。			
\\	センター	センター	センター	
\\	このセンターで工場全体を管理しています。	このセンターで 工場全体[こうじょう ぜんたい]を 管理[かんり]しています。	この せんたー で こうじょう ぜんたい を かんり して います	
\\	この
\\	で 工場全体[こうじょう ぜんたい]を 管理[かんり]しています。			
\\	不潔	不潔[ふけつ]	ふけつ	
\\	体を不潔にしているとかゆくなりますよ。	体[からだ]を 不潔[ふけつ]にしているとかゆくなりますよ。	からだ を ふけつ に して いる と かゆく なります よ	
\\	体[からだ]を
\\	にしているとかゆくなりますよ。			
\\	編集	編集[へんしゅう]	へんしゅう	
\\	私は雑誌の編集の仕事をしています。	私[わたし]は 雑誌[ざっし]の 編集[へんしゅう]の 仕事[しごと]をしています。	わたし は ざっし の へんしゅう の しごと を して います	
\\	私[わたし]は 雑誌[ざっし]の
\\	の 仕事[しごと]をしています。			
\\	追放	追放[ついほう]	ついほう	
\\	彼は国外に追放されました。	彼[かれ]は 国外[こくがい]に 追放[ついほう]されました。	かれ は こくがい に ついほう されました	
\\	彼[かれ]は 国外[こくがい]に
\\	されました。			
\\	放射能	放射能[ほうしゃのう]	ほうしゃのう	
\\	この地区は放射能に汚染されたんだ。	この 地区[ちく]は 放射能[ほうしゃのう]に 汚染[おせん]されたんだ。	この ちく は ほうしゃのう に おせん された ん だ	
\\	この 地区[ちく]は
\\	に 汚染[おせん]されたんだ。			
\\	放す	放[はな]す	はなす	
\\	公園で犬を放したの。	公園[こうえん]で 犬[いぬ]を 放[はな]したの。	こうえん で いぬ を はなした の	
\\	公園[こうえん]で 犬[いぬ]を
\\	の。			
\\	ぞんざい	ぞんざい	ぞんざい	
\\	そんなぞんざいな口のききかたはよくないよ。	そんなぞんざいな 口[くち]のききかたはよくないよ。	そんな ぞんざい な くち の ききかた は よくない よ	
\\	そんな
\\	な 口[くち]のききかたはよくないよ。			
\\	放る	放[ほう]る	ほうる	
\\	ボールを空中に放ったの。	ボールを 空中[くうちゅう]に 放[ほう]ったの。	ぼーる を くうちゅう に ほうった の	
\\	ボールを 空中[くうちゅう]に
\\	の。			
\\	無視	無視[むし]	むし	
\\	彼の意見は無視されたよ。	彼[かれ]の 意見[いけん]は 無視[むし]されたよ。	かれ の いけん は むし された よ	
\\	彼[かれ]の 意見[いけん]は
\\	されたよ。			
\\	服装	服装[ふくそう]	ふくそう	
\\	そのパーティーはカジュアルな服装で大丈夫です。	そのパーティーはカジュアルな 服装[ふくそう]で 大丈夫[だいじょうぶ]です。	その ぱーてぃー は かじゅある な ふくそう で だいじょうぶ です	
\\	そのパーティーはカジュアルな
\\	で 大丈夫[だいじょうぶ]です。			
\\	振り仮名	振[ふ]り 仮名[がな]	ふりがな	
\\	名前に振り仮名をつけてください。	名前[なまえ]に 振[ふ]り 仮名[がな]をつけてください。	なまえ に ふりがな を つけて ください	
\\	名前[なまえ]に
\\	をつけてください。			
\\	銅	銅[どう]	どう	
\\	10円玉は銅でできています。	
\\	円玉[えんだま]は 銅[どう]でできています。	
\\	えんだま は どう で できて います	
\\	円玉[えんだま]は
\\	でできています。			
\\	たんす	たんす	たんす	
\\	昨日、たんすを買いました。	昨日[きのう]、たんすを 買[か]いました。	きのう たんす を かいました	
\\	昨日[きのう]、
\\	を 買[か]いました。			
\\	悲劇	悲劇[ひげき]	ひげき	
\\	あの悲劇を繰り返してはいけない。	あの 悲劇[ひげき]を 繰[く]り 返[かえ]してはいけない。	あの ひげき を くりかえして は いけない	
\\	あの
\\	を 繰[く]り 返[かえ]してはいけない。			
\\	団体	団体[だんたい]	だんたい	
\\	サッカーは団体競技です。	サッカーは 団体[だんたい] 競技[きょうぎ]です。	さっかー は だんたい きょうぎ です	
\\	サッカーは
\\	競技[きょうぎ]です。			
\\	団地	団地[だんち]	だんち	
\\	私の弟は団地に住んでいます。	私[わたし]の 弟[おとうと]は 団地[だんち]に 住[す]んでいます。	わたし の おとうと は だんち に すんで います	
\\	私[わたし]の 弟[おとうと]は
\\	に 住[す]んでいます。			
\\	美術	美術[びじゅつ]	びじゅつ	
\\	弟は美術を専攻しています。	弟[おとうと]は 美術[びじゅつ]を 専攻[せんこう]しています。	おとうと は びじゅつ を せんこう して います	
\\	弟[おとうと]は
\\	を 専攻[せんこう]しています。			
\\	博士	博士[はくし]	はくし	
\\	彼は数学の博士だそうです。	彼[かれ]は 数学[すうがく]の 博士[はくし]だそうです。	かれ は すうがく の はくし だ そう です	
\\	彼[かれ]は 数学[すうがく]の
\\	だそうです。			
\\	博士	博士[はかせ]	はかせ	
\\	彼は物理学の博士です。	彼[かれ]は 物理学[ぶつりがく]の 博士[はかせ]です。	かれ は ぶつりがく の はかせ です	
\\	彼[かれ]は 物理学[ぶつりがく]の
\\	です。			
\\	チェンジ	チェンジ	チェンジ	
\\	坂道でギアーをチェンジしたよ。	坂道[さかみち]でギアーをチェンジしたよ。	さかみち で ぎあー を ちぇんじ した よ	
\\	坂道[さかみち]でギアーを
\\	したよ。			
\\	展開	展開[てんかい]	てんかい	
\\	話の展開についていけない。	話[はなし]の 展開[てんかい]についていけない。	はなし の てんかい に ついていけない	
\\	話[はなし]の
\\	についていけない。			
\\	発展	発展[はってん]	はってん	
\\	あの会社は目覚しく発展しているね。	あの 会社[かいしゃ]は 目覚[めざま]しく 発展[はってん]しているね。	あの かいしゃ は めざましく はってん して いる ね	
\\	あの 会社[かいしゃ]は 目覚[めざま]しく
\\	しているね。			
\\	催す	催[もよお]す	もよおす	
\\	静かな音楽で眠気を催したよ。	静[しず]かな 音楽[おんがく]で 眠気[ねむけ]を 催[もよお]したよ。	しずか な おんがく で ねむけ を もよおした よ	
\\	静[しず]かな 音楽[おんがく]で 眠気[ねむけ]を
\\	よ。			
\\	催し	催[もよお]し	もよおし	
\\	学校の催しに参加しました。	学校[がっこう]の 催[もよお]しに 参加[さんか]しました。	がっこう の もよおし に さんか しました	
\\	学校[がっこう]の
\\	に 参加[さんか]しました。			
\\	典型的	典型的[てんけいてき]	てんけいてき	
\\	彼女は典型的なイタリア人です。	彼女[かのじょ]は 典型的[てんけいてき]なイタリア 人[じん]です。	かのじょ は てんけいてき な いたりあじん です	
\\	彼女[かのじょ]は
\\	なイタリア 人[じん]です。			
\\	つかまる	つかまる	つかまる	
\\	しっかりとつかまっていてください。	しっかりとつかまっていてください。	しっかりとつかまっていてください。	
\\	しっかりと
\\	いてください。			
\\	典型	典型[てんけい]	てんけい	
\\	彼は職人の典型です。	彼[かれ]は 職人[しょくにん]の 典型[てんけい]です。	かれ は しょくにん の てんけい です	
\\	彼[かれ]は 職人[しょくにん]の
\\	です。			
\\	百科事典	百科事典[ひゃっかじてん]	ひゃっかじてん	
\\	めずらしい猫について百科事典で調べたの。	めずらしい 猫[ねこ]について 百科事典[ひゃっかじてん]で 調[しら]べたの。	めずらしい ねこ に ついて ひゃっかじてん で しらべた の	
\\	めずらしい 猫[ねこ]について
\\	で 調[しら]べたの。			
\\	特殊	特殊[とくしゅ]	とくしゅ	
\\	彼は特殊な能力を持っています。	彼[かれ]は 特殊[とくしゅ]な 能力[のうりょく]を 持[も]っています。	かれ は とくしゅ な のうりょく を もって います	
\\	彼[かれ]は
\\	な 能力[のうりょく]を 持[も]っています。			
\\	微妙	微妙[びみょう]	びみょう	
\\	彼は会社で微妙な立場にあります。	彼[かれ]は 会社[かいしゃ]で 微妙[びみょう]な 立場[たちば]にあります。	かれ は かいしゃ で びみょう な たちば に あります	
\\	彼[かれ]は 会社[かいしゃ]で
\\	な 立場[たちば]にあります。			
\\	免税	免税[めんぜい]	めんぜい	
\\	この商品は免税です。	この 商品[しょうひん]は 免税[めんぜい]です。	この しょうひん は めんぜい です	
\\	この 商品[しょうひん]は
\\	です。			
\\	つながり	つながり	つながり	
\\	この文のつながりはおかしいです。	この 文[ぶん]のつながりはおかしいです。	この ぶん の つながり は おかしい です	
\\	この 文[ぶん]の
\\	はおかしいです。			
\\	免許	免許[めんきょ]	めんきょ	
\\	やっと運転免許を手に入れたよ。	やっと 運転[うんてん] 免許[めんきょ]を 手[て]に 入[い]れたよ。	やっと うんてん めんきょ を て に いれた よ	
\\	やっと 運転[うんてん]
\\	を 手[て]に 入[い]れたよ。			
\\	動詞	動詞[どうし]	どうし	
\\	「食べる」は動詞です。	
\\	食[た]べる」は 動詞[どうし]です。	たべる は どうし です	
\\	食[た]べる」は
\\	です。			
\\	名詞	名詞[めいし]	めいし	
\\	「学校」は名詞です。	
\\	学校」[がっこう]は 名詞[めいし]です。	がっこう は めいし です	
\\	学校」[がっこう]は
\\	です。			
\\	副詞	副詞[ふくし]	ふくし	
\\	「ゆっくり歩く」の「ゆっくり」は副詞です。	「ゆっくり 歩[ある]く」の「ゆっくり」は 副詞[ふくし]です。	ゆっくり あるく の ゆっくり は ふくし です	
\\	「ゆっくり 歩[ある]く」の「ゆっくり」は
\\	です。			
\\	代名詞	代名詞[だいめいし]	だいめいし	
\\	「彼」は代名詞の一つです。	
\\	彼」[かれ]は 代名詞[だいめいし]の 一[ひと]つです。	かれ は だいめいし の ひとつ です	
\\	彼」[かれ]は
\\	の 一[ひと]つです。			
\\	つながる	つながる	つながる	
\\	電話がつながりません。	電話[でんわ]がつながりません。	でんわ が つながりません	
\\	電話[でんわ]が
\\	伝統	伝統[でんとう]	でんとう	
\\	私は日本の伝統を大切にします。	私[わたし]は 日本[にっぽん]の 伝統[でんとう]を 大切[たいせつ]にします。	わたし は にっぽん の でんとう を たいせつ に します	
\\	私[わたし]は 日本[にっぽん]の
\\	を 大切[たいせつ]にします。			
\\	伝わる	伝[つた]わる	つたわる	
\\	漢字は中国から伝わった。	漢字[かんじ]は 中国[ちゅうごく]から 伝[つた]わった。	かんじ は ちゅうごく から つたわった	
\\	漢字[かんじ]は 中国[ちゅうごく]から
\\	伝説	伝説[でんせつ]	でんせつ	
\\	彼は数々の伝説を残しました。	彼[かれ]は 数々[かずかず]の 伝説[でんせつ]を 残[のこ]しました。	かれ は かずかず の でんせつ を のこしました	
\\	彼[かれ]は 数々[かずかず]の
\\	を 残[のこ]しました。			
\\	手伝い	手伝[てつだ]い	てつだい	
\\	会議の準備に手伝いが必要です。	会議[かいぎ]の 準備[じゅんび]に 手伝[てつだ]いが 必要[ひつよう]です。	かいぎ の じゅんび に てつだい が ひつよう です	
\\	会議[かいぎ]の 準備[じゅんび]に
\\	が 必要[ひつよう]です。			
\\	伝言	伝言[でんごん]	でんごん	
\\	課長から伝言があります。	課長[かちょう]から 伝言[でんごん]があります。	かちょう から でんごん が あります	
\\	課長[かちょう]から
\\	があります。			
\\	伝染	伝染[でんせん]	でんせん	
\\	その国ではコレラの伝染が蔓延しています。	その 国[くに]ではコレラの 伝染[でんせん]が 蔓延[まんえん]しています。	その くに で は これら の でんせん が まんえん して います	
\\	その 国[くに]ではコレラの
\\	が 蔓延[まんえん]しています。			
\\	つばめ	つばめ	つばめ	
\\	つばめが飛んでいますね。	つばめが 飛[と]んでいますね。	つばめ が とんで います ね	
\\	が 飛[と]んでいますね。			
\\	焼き鳥	焼[や]き 鳥[とり]	やきとり	
\\	焼き鳥はビールに合うな。	焼[や]き 鳥[とり]はビールに 合[あ]うな。	やきとり は びーる に あう な	
\\	はビールに 合[あ]うな。			
\\	鳴らす	鳴[な]らす	ならす	
\\	お坊さんが鐘を鳴らしていますね。	お 坊[ぼう]さんが 鐘[かね]を 鳴[な]らしていますね。	おぼうさん が かね を ならして います ね	
\\	お 坊[ぼう]さんが 鐘[かね]を
\\	ね。			
\\	悲鳴	悲鳴[ひめい]	ひめい	
\\	外から悲鳴が聞こえたな。	外[そと]から 悲鳴[ひめい]が 聞[き]こえたな。	そと から ひめい が きこえた な	
\\	外[そと]から
\\	が 聞[き]こえたな。			
\\	笑い声	笑[わら]い 声[ごえ]	わらいごえ	
\\	部屋の中から笑い声が聞こえた。	部屋[へや]の 中[なか]から 笑[わら]い 声[ごえ]が 聞[き]こえた。	へや の なか から わらいごえ が きこえた	
\\	部屋[へや]の 中[なか]から
\\	が 聞[き]こえた。			
\\	泣き声	泣[な]き 声[ごえ]	なきごえ	
\\	赤ちゃんの泣き声が聞こえますね。	赤[あか]ちゃんの 泣[な]き 声[ごえ]が 聞[き]こえますね。	あかちゃん の なきごえ が きこえます ね	
\\	赤[あか]ちゃんの
\\	が 聞[き]こえますね。			
\\	つぶやく	つぶやく	つぶやく	
\\	彼は一人で何かつぶやいていたの。	彼[かれ]は 一人[ひとり]で 何[なに]かつぶやいていたの。	かれ は ひとり で なにか つぶやいて いた の	
\\	彼[かれ]は 一人[ひとり]で 何[なに]かつぶやいていたの。			
\\	話し声	話[はな]し 声[ごえ]	はなしごえ	
\\	隣の部屋から話し声が聞こえます。	隣[となり]の 部屋[へや]から 話[はな]し 声[ごえ]が 聞[き]こえます。	となり の へや から はなしごえ が きこえます	
\\	隣[となり]の 部屋[へや]から
\\	が 聞[き]こえます。			
\\	駐車	駐車[ちゅうしゃ]	ちゅうしゃ	
\\	車はここに駐車してください。	車[くるま]はここに 駐車[ちゅうしゃ]してください。	くるま は ここ に ちゅうしゃ して ください	
\\	車[くるま]はここに
\\	してください。			
\\	騒音	騒音[そうおん]	そうおん	
\\	窓から車の騒音が入って来ます。	窓[まど]から 車[くるま]の 騒音[そうおん]が 入[はい]って 来[き]ます。	まど から くるま の そうおん が はいって きます	
\\	窓[まど]から 車[くるま]の
\\	が 入[はい]って 来[き]ます。			
\\	騒動	騒動[そうどう]	そうどう	
\\	警察がやって来て騒動を静めたんだ。	警察[けいさつ]がやって 来[き]て 騒動[そうどう]を 静[しず]めたんだ。	けいさつ が やってきて そうどう を しずめた ん だ	
\\	警察[けいさつ]がやって 来[き]て
\\	を 静[しず]めたんだ。			
\\	騒々しい	騒々[そうぞう]しい	そうぞうしい	
\\	隣の家はいつも騒々しい。	隣[となり]の 家[いえ]はいつも 騒々[そうぞう]しい。	となり の いえ は いつも そうぞうしい	
\\	隣[となり]の 家[いえ]はいつも
\\	つぶる	つぶる	つぶる	
\\	目をつぶってください。	目[め]をつぶってください。	め を つぶって ください	
\\	目[め]を
\\	ください。			
\\	名刺	名刺[めいし]	めいし	
\\	私たちは名刺を交換しました。	私[わたし]たちは 名刺[めいし]を 交換[こうかん]しました。	わたしたち は めいし を こうかん しました	
\\	私[わたし]たちは
\\	を 交換[こうかん]しました。			
\\	到着	到着[とうちゃく]	とうちゃく	
\\	夜9時に東京に到着しました。	夜9時[よる 
\\	じ]に 東京[とうきょう]に 到着[とうちゃく]しました。	よる 
\\	じ に とうきょう に とうちゃく しました	
\\	夜9時[よる 
\\	じ]に 東京[とうきょう]に
\\	しました。			
\\	到達	到達[とうたつ]	とうたつ	
\\	ついに山頂に到達しました。	ついに 山頂[さんちょう]に 到達[とうたつ]しました。	ついに さんちょう に とうたつ しました	
\\	ついに 山頂[さんちょう]に
\\	しました。			
\\	倒産	倒産[とうさん]	とうさん	
\\	会社が倒産しました。	会社[かいしゃ]が 倒産[とうさん]しました。	かいしゃ が とうさん しました	
\\	会社[かいしゃ]が
\\	しました。			
\\	面倒を見る	面倒[めんどう]を見る「みる」	めんどうをみる	
\\	会社でこはいの面倒を見ています	会社[かいしゃ]でこはいの 面倒を見ています[めんどうを見ています]。	かいしゃ で こはいのめんどうをみています 。	
\\	会社[かいしゃ]で
\\	なことが 起[お]こったの。			
\\	斜め	斜[なな]め	ななめ	
\\	ここに斜めに線を引いてください。	ここに 斜[なな]めに 線[せん]を 引[ひ]いてください。	ここ に ななめ に せん を ひいて ください	
\\	ここに
\\	に 線[せん]を 引[ひ]いてください。			
\\	つぼみ	つぼみ	つぼみ	
\\	花のつぼみがたくさんついていますね。	花[はな]のつぼみがたくさんついていますね。	はな の つぼみ が たくさん ついています ね 。	
\\	花[はな]の
\\	がたくさんついていますね。			
\\	柔らかい	柔[やわ]らかい	やわらかい	
\\	柔らかい日差しが気持ちいいね。	柔[やわ]らかい 日差[ひざ]しが 気持[きも]ちいいね。	やわらかい ひざし が きもち いい ね	
\\	日差[ひざ]しが 気持[きも]ちいいね。			
\\	張る	張[は]る	はる	
\\	疲れて肩が張っています。	疲[つか]れて 肩[かた]が 張[は]っています。	つかれて かた が はって います	
\\	疲[つか]れて 肩[かた]が
\\	張り切る	張[は]り 切[き]る	はりきる	
\\	母は張り切ってお弁当を用意したの。	母[はは]は 張[は]り 切[き]ってお 弁当[べんとう]を 用意[ようい]したの。	はは は はりきって おべんとう を ようい した の	
\\	母[はは]は
\\	お 弁当[べんとう]を 用意[ようい]したの。			
\\	欲張り	欲張[よくば]り	よくばり	
\\	彼女は欲張りです。	彼女[かのじょ]は 欲張[よくば]りです。	かのじょ は よくばり です	
\\	彼女[かのじょ]は
\\	です。			
\\	突然	突然[とつぜん]	とつぜん	
\\	彼は突然走り出したのよ。	彼[かれ]は 突然[とつぜん] 走[はし]り 出[だ]したのよ。	かれ は とつぜん はしりだした の よ	
\\	彼[かれ]は
\\	走[はし]り 出[だ]したのよ。			
\\	つまずく	つまずく	つまずく	
\\	石につまずきました。	石[いし]につまずきました。	いし に つまずきました 。	
\\	石[いし]に
\\	突っ込む	突[つ]っ 込[こ]む	つっこむ	
\\	ポケットに財布を突っ込んじゃった。	ポケットに 財布[さいふ]を 突[つ]っ 込[こ]んじゃった。	ぽけっと に さいふ を つっこん じゃった	
\\	ポケットに 財布[さいふ]を
\\	突く	突[つ]く	つく	
\\	彼はビリヤードの球を上手に突くね。	彼[かれ]はビリヤードの 球[たま]を 上手[じょうず]に 突[つ]くね。	かれ は びりやーど の たま を じょうず に つく ね	
\\	彼[かれ]はビリヤードの 球[たま]を 上手[じょうず]に
\\	ね。			
\\	追突	追突[ついとつ]	ついとつ	
\\	バスが乗用車に追突したよ。	バスが 乗用車[じょうようしゃ]に 追突[ついとつ]したよ。	ばす が じょうようしゃ に ついとつ した よ	
\\	バスが 乗用車[じょうようしゃ]に
\\	したよ。			
\\	触れる	触[ふ]れる	ふれる	
\\	手と手が触れてどきどきしたよ。	手[て]と 手[て]が 触[ふ]れてどきどきしたよ。	て と て が ふれて どきどき した よ	
\\	手[て]と 手[て]が
\\	どきどきしたよ。			
\\	避難	避難[ひなん]	ひなん	
\\	危ないので避難してください。	危[あぶ]ないので 避難[ひなん]してください。	あぶない の で ひなん して ください	
\\	危[あぶ]ないので
\\	してください。			
\\	ツル	ツル	ツル	
\\	湖にツルがいますよ。	湖[みずうみ]にツルがいますよ。	みずうみ に つる が います よ	
\\	湖[みずうみ]に
\\	がいますよ。			
\\	墜落	墜落[ついらく]	ついらく	
\\	飛行機の墜落事故があったんだ。	飛行機[ひこうき]の 墜落[ついらく] 事故[じこ]があったんだ。	ひこうき の ついらく じこ が あった ん だ	
\\	飛行機[ひこうき]の
\\	事故[じこ]があったんだ。			
\\	兵士	兵士[へいし]	へいし	
\\	その爆撃で兵士が3人負傷しました。	その 爆撃[ばくげき]で 兵士[へいし]が 3人負傷[さんにん ふしょう]しました。	その ばくげき で へいし が さんにん ふしょう しました	
\\	その 爆撃[ばくげき]で
\\	が 3人負傷[さんにん ふしょう]しました。			
\\	兵器	兵器[へいき]	へいき	
\\	あの国は強力な兵器を持っている。	あの 国[くに]は 強力[きょうりょく]な 兵器[へいき]を 持[も]っている。	あの くに は きょうりょく な へいき を もって いる	
\\	あの 国[くに]は 強力[きょうりょく]な
\\	を 持[も]っている。			
\\	兵隊	兵隊[へいたい]	へいたい	
\\	兵隊が銃をかまえていたよ。	兵隊[へいたい]が 銃[じゅう]をかまえていたよ。	へいたい が じゅう を かまえて いた よ	
\\	が 銃[じゅう]をかまえていたよ。			
\\	武器	武器[ぶき]	ぶき	
\\	彼らは武器を取り、立ち上がった。	彼[かれ]らは 武器[ぶき]を 取[と]り、 立[た]ち 上[あ]がった。	かれら は ぶき を とり たちあがった	
\\	彼[かれ]らは
\\	を 取[と]り、 立[た]ち 上[あ]がった。			
\\	武士	武士[ぶし]	ぶし	
\\	彼の家柄は武士でした。	彼[かれ]の 家柄[いえがら]は 武士[ぶし]でした。	かれ の いえがら は ぶし でした	
\\	彼[かれ]の 家柄[いえがら]は
\\	でした。			
\\	でこぼこ	でこぼこ	でこぼこ	
\\	この道はでこぼこしていますね。	この 道[みち]はでこぼこしていますね。	この みち は でこぼこ して います ね	
\\	この 道[みち]は
\\	していますね。			
\\	弾	弾[たま]	たま	
\\	彼はピストルに弾を込めたんだ。	彼[かれ]はピストルに 弾[たま]を 込[こ]めたんだ。	かれ は ぴすとる に たま を こめた ん だ	
\\	彼[かれ]はピストルに
\\	を 込[こ]めたんだ。			
\\	弾む	弾[はず]む	はずむ	
\\	このボールはよく弾みますね。	このボールはよく 弾[はず]みますね。	この ぼーる は よく はずみます ね	
\\	このボールはよく
\\	ね。			
\\	日の丸	日[ひ]の 丸[まる]	ひのまる	
\\	日本の国旗は日の丸と呼ばれています。	日本[にっぽん]の 国旗[こっき]は 日[ひ]の 丸[まる]と 呼[よ]ばれています。	にっぽん の こっき は ひのまる と よばれて います	
\\	日本[にっぽん]の 国旗[こっき]は
\\	と 呼[よ]ばれています。			
\\	真ん丸	真[ま]ん 丸[まる]	まんまる	
\\	今日は月が真ん丸です。	今日[きょう]は 月[つき]が 真[ま]ん 丸[まる]です。	きょう は つき が まんまる です	
\\	今日[きょう]は 月[つき]が
\\	です。			
\\	爆発	爆発[ばくはつ]	ばくはつ	
\\	ダイナマイトが爆発したんだ。	ダイナマイトが 爆発[ばくはつ]したんだ。	だいなまいと が ばくはつ した ん だ	
\\	ダイナマイトが
\\	したんだ。			
\\	てのひら	てのひら	てのひら	
\\	てのひらに汗をかきました。	てのひらに 汗[あせ]をかきました。	てのひら に あせ を かきました	
\\	に 汗[あせ]をかきました。			
\\	爆弾	爆弾[ばくだん]	ばくだん	
\\	その町に爆弾が落とされたの。	その 町[まち]に 爆弾[ばくだん]が 落[お]とされたの。	その まち に ばくだん が おとされた の	
\\	その 町[まち]に
\\	が 落[お]とされたの。			
\\	暴落	暴落[ぼうらく]	ぼうらく	
\\	昨日株価が暴落しました。	昨日株価[きのう かぶか]が 暴落[ぼうらく]しました。	きのう かぶか が ぼうらく しました	
\\	昨日株価[きのう かぶか]が
\\	しました。			
\\	暴力	暴力[ぼうりょく]	ぼうりょく	
\\	暴力はいけません。	暴力[ぼうりょく]はいけません。	ぼうりょく は いけません	
\\	はいけません。			
\\	乱暴	乱暴[らんぼう]	らんぼう	
\\	人に乱暴してはいけません。	人[ひと]に 乱暴[らんぼう]してはいけません。	ひと に らんぼう して は いけません	
\\	人[ひと]に
\\	してはいけません。			
\\	絶えず	絶[た]えず	たえず	
\\	今日は絶えず電話が鳴りました。	今日[きょう]は 絶[た]えず 電話[でんわ]が 鳴[な]りました。	きょう は たえず でんわ が なりました	
\\	今日[きょう]は
\\	電話[でんわ]が 鳴[な]りました。			
\\	とがる	とがる	とがる	
\\	この鉛筆はとがっていますね。	この 鉛筆[えんぴつ]はとがっていますね。	この えんぴつ は とがって います ね	
\\	この 鉛筆[えんぴつ]は
\\	ね。			
\\	滅びる	滅[ほろ]びる	ほろびる	
\\	その文明は何千年も前に滅びてしまったの。	その 文明[ぶんめい]は 何千年[なんぜんねん]も 前[まえ]に 滅[ほろ]びてしまったの。	その ぶんめい は なんぜんねん も まえ に ほろびて しまった の	
\\	その 文明[ぶんめい]は 何千年[なんぜんねん]も 前[まえ]に
\\	しまったの。			
\\	防ぐ	防[ふせ]ぐ	ふせぐ	
\\	怪我を防ぐためによくストレッチをしてください。	怪我[けが]を 防[ふせ]ぐためによくストレッチをしてください。	けが を ふせぐ ため に よく すとれっち を して ください	
\\	怪我[けが]を
\\	ためによくストレッチをしてください。			
\\	防衛	防衛[ぼうえい]	ぼうえい	
\\	チャンピオンがタイトルを防衛した。	チャンピオンがタイトルを 防衛[ぼうえい]した。	ちゃんぴおん が たいとる を ぼうえい した	
\\	チャンピオンがタイトルを
\\	した。			
\\	防止	防止[ぼうし]	ぼうし	
\\	警察は犯罪の防止に努めています。	警察[けいさつ]は 犯罪[はんざい]の 防止[ぼうし]に 努[つと]めています。	けいさつ は はんざい の ぼうし に つとめて います	
\\	警察[けいさつ]は 犯罪[はんざい]の
\\	に 努[つと]めています。			
\\	予防	予防[よぼう]	よぼう	
\\	虫歯は予防できます。	虫歯[むしば]は 予防[よぼう]できます。	むしば は よぼう できます	
\\	虫歯[むしば]は
\\	できます。			
\\	とんかつ	とんかつ	とんかつ	
\\	昼食にとんかつを食べました。	昼食[ちゅうしょく]にとんかつを 食[た]べました。	ちゅうしょく に とん かつ を たべました 。	
\\	昼食[ちゅうしょく]に
\\	を 食[た]べました。			
\\	防火	防火[ぼうか]	ぼうか	
\\	学校で防火訓練がありました。	学校[がっこう]で 防火[ぼうか] 訓練[くんれん]がありました。	がっこう で ぼうかくんれん が ありました	
\\	学校[がっこう]で
\\	訓練[くんれん]がありました。			
\\	妨害	妨害[ぼうがい]	ぼうがい	
\\	彼に営業を妨害されました。	彼[かれ]に 営業[えいぎょう]を 妨害[ぼうがい]されました。	かれ に えいぎょう を ぼうがい されました	
\\	彼[かれ]に 営業[えいぎょう]を
\\	されました。			
\\	不機嫌	不機嫌[ふきげん]	ふきげん	
\\	この頃、あの人は不機嫌ですね。	この 頃[ごろ]、あの 人[ひと]は 不機嫌[ふきげん]ですね。	このごろ あの ひと は ふきげん です ね	
\\	この 頃[ごろ]、あの 人[ひと]は
\\	ですね。			
\\	徹夜	徹夜[てつや]	てつや	
\\	ゆうべは徹夜しました。	ゆうべは 徹夜[てつや]しました。	ゆうべ は てつや しました	
\\	ゆうべは
\\	しました。			
\\	底	底[そこ]	そこ	
\\	コップの底が濡れていますよ。	コップの 底[そこ]が 濡[ぬ]れていますよ。	こっぷ の そこ が ぬれて います よ	
\\	コップの
\\	が 濡[ぬ]れていますよ。			
\\	徹底的	徹底的[てっていてき]	てっていてき	
\\	部屋の中を徹底的に探しました。	部屋[へや]の 中[なか]を 徹底的[てっていてき]に 探[さが]しました。	へや の なか を てっていてき に さがしました	
\\	部屋[へや]の 中[なか]を
\\	に 探[さが]しました。			
\\	どんなに	どんなに	どんなに	
\\	彼女はどんなに嬉しかったろう。	彼女[かのじょ]はどんなに 嬉[うれ]しかったろう。	かのじょ は どんなに うれしかったろう	
\\	彼女[かのじょ]は
\\	嬉[うれ]しかったろう。			
\\	到底	到底[とうてい]	とうてい	
\\	今夜中に東京に着くのは到底無理でしょう。	今夜中[こんやじゅう]に 東京[とうきょう]に 着[つ]くのは 到底[とうてい] 無理[むり]でしょう。	こんやじゅう に とうきょう に つく の は とうてい むり でしょう	
\\	今夜中[こんやじゅう]に 東京[とうきょう]に 着[つ]くのは
\\	無理[むり]でしょう。			
\\	抵抗	抵抗[ていこう]	ていこう	
\\	犯人は警察に抵抗したの。	犯人[はんにん]は 警察[けいさつ]に 抵抗[ていこう]したの。	はんにん は けいさつ に ていこう した の	
\\	犯人[はんにん]は 警察[けいさつ]に
\\	したの。			
\\	敵	敵[てき]	てき	
\\	あそこに敵がひそんでいる。	あそこに 敵[てき]がひそんでいる。	あそこ に てき が ひそんで いる	
\\	あそこに
\\	がひそんでいる。			
\\	態勢	態勢[たいせい]	たいせい	
\\	作業を始める態勢は整っています。	作業[さぎょう]を 始[はじ]める 態勢[たいせい]は 整[ととの]っています。	さぎょう を はじめる たいせい は ととのって います	
\\	作業[さぎょう]を 始[はじ]める
\\	は 整[ととの]っています。			
\\	体系	体系[たいけい]	たいけい	
\\	勤務の体系を見直すことにしました。	勤務[きんむ]の 体系[たいけい]を 見直[みなお]すことにしました。	きんむ の たいけい を みなおす こと に しました	
\\	勤務[きんむ]の
\\	を 見直[みなお]すことにしました。			
\\	なめる	なめる	なめる	
\\	犬に顔をなめられました。	犬[いぬ]に 顔[かお]をなめられました。	いぬ に かお を なめられました	
\\	犬[いぬ]に 顔[かお]を
\\	理系	理系[りけい]	りけい	
\\	彼は理系の学生です。	彼[かれ]は 理系[りけい]の 学生[がくせい]です。	かれ は りけい の がくせい です	
\\	彼[かれ]は
\\	の 学生[がくせい]です。			
\\	文系	文系[ぶんけい]	ぶんけい	
\\	彼女は文系です。	彼女[かのじょ]は 文系[ぶんけい]です。	かのじょ は ぶんけい です	
\\	彼女[かのじょ]は
\\	です。			
\\	綿	綿[めん]	めん	
\\	私は綿のシャツをよく着ます。	私[わたし]は 綿[めん]のシャツをよく 着[き]ます。	わたし は めん の しゃつ を よく きます	
\\	私[わたし]は
\\	のシャツをよく 着[き]ます。			
\\	繁栄	繁栄[はんえい]	はんえい	
\\	あの国は経済的に繁栄しているわね。	あの 国[くに]は 経済的[けいざいてき]に 繁栄[はんえい]しているわね。	あの くに は けいざいてき に はんえい して いる わ ね	
\\	あの 国[くに]は 経済的[けいざいてき]に
\\	しているわね。			
\\	日韓	日韓[にっかん]	にっかん	
\\	日韓合同のコンサートが開かれたよ。	日韓[にっかん] 合同[ごうどう]のコンサートが 開[ひら]かれたよ。	にっかん ごうどう の こんさーと が ひらかれた よ	
\\	合同[ごうどう]のコンサートが 開[ひら]かれたよ。			
\\	にらむ	にらむ	にらむ	
\\	彼女は私をにらんだの。	彼女[かのじょ]は 私[わたし]をにらんだの。	かのじょ は わたし を にらんだ の	
\\	彼女[かのじょ]は 私[わたし]を
\\	の。			
\\	耐える	耐[た]える	たえる	
\\	このビルは大地震に耐えられるでしょうか。	このビルは 大地震[おおじしん]に 耐[た]えられるでしょうか。	この びる は おおじしん に たえられる でしょう か	
\\	このビルは 大地震[おおじしん]に
\\	でしょうか。			
\\	矢印	矢印[やじるし]	やじるし	
\\	矢印にそって歩いてください。	矢印[やじるし]にそって 歩[ある]いてください。	やじるし に そって あるいて ください	
\\	にそって 歩[ある]いてください。			
\\	爪切り	爪切[つめき]り	つめきり	
\\	爪切りを取ってください。	爪切[つめき]りを 取[と]ってください。	つめきり を とって ください	
\\	を 取[と]ってください。			
\\	枠	枠[わく]	わく	
\\	枠の中に答えを書いてください。	枠[わく]の 中[なか]に 答[こた]えを 書[か]いてください。	わく の なか に こたえ を かいて ください	
\\	の 中[なか]に 答[こた]えを 書[か]いてください。			
\\	棒	棒[ぼう]	ぼう	
\\	この棒は何に使うのですか。	この 棒[ぼう]は 何[なに]に 使[つか]うのですか。	この ぼう は なに に つかう の です か	
\\	この
\\	は 何[なに]に 使[つか]うのですか。			
\\	人柄	人柄[ひとがら]	ひとがら	
\\	みんな彼の温かい人柄が大好きだよ。	みんな 彼[かれ]の 温[あたた]かい 人柄[ひとがら]が 大好[だいす]きだよ。	みんな かれ の あたたかい ひとがら が だいすき だ よ	
\\	みんな 彼[かれ]の 温[あたた]かい
\\	が 大好[だいす]きだよ。			
\\	ねずみ	ねずみ	ねずみ	
\\	台所にねずみが出た。	台所[だいどころ]にねずみが 出[で]た。	だいどころ に ねずみ が でた	
\\	台所[だいどころ]に
\\	が 出[で]た。			
\\	詰める	詰[つ]める	つめる	
\\	かばんに荷物を詰めました。	かばんに 荷物[にもつ]を 詰[つ]めました。	かばん に にもつ を つめました	
\\	かばんに 荷物[にもつ]を
\\	詰まる	詰[つ]まる	つまる	
\\	プリンタのインクが詰まってしまいました。	プリンタのインクが 詰[つ]まってしまいました。	ぷりんた の いんく が つまって しまいました	
\\	プリンタのインクが
\\	しまいました。			
\\	譲る	譲[ゆず]る	ゆずる	
\\	上司が車を譲ってくれたの。	上司[じょうし]が 車[くるま]を 譲[ゆず]ってくれたの。	じょうし が くるま を ゆずって くれた の	
\\	上司[じょうし]が 車[くるま]を
\\	の。			
\\	誠に	誠[まこと]に	まことに	
\\	誠にありがとうございます。	誠[まこと]にありがとうございます。	まことに ありがとう ございます	
\\	ありがとうございます。			
\\	盛り上げる	盛[も]り 上[あ]げる	もりあげる	
\\	応援団が試合を盛り上げたんだ。	応援団[おうえんだん]が 試合[しあい]を 盛[も]り 上[あ]げたんだ。	おうえんだん が しあい を もりあげた ん だ	
\\	応援団[おうえんだん]が 試合[しあい]を
\\	んだ。			
\\	のぞく	のぞく	のぞく	
\\	ドアの隙間から中をのぞいたんだ。	ドアの 隙間[すきま]から 中[なか]をのぞいたんだ。	どあ の すきま から なか を のぞいた ん だ	
\\	ドアの 隙間[すきま]から 中[なか]を
\\	んだ。			
\\	盛り上がる	盛[も]り 上[あ]がる	もりあがる	
\\	昨夜のパーティはとても盛り上がったな。	昨夜[さくや]のパーティはとても 盛[も]り 上[あ]がったな。	さくや の ぱーてぃ は とても もりあがった な	
\\	昨夜[さくや]のパーティはとても
\\	な。			
\\	盛る	盛[も]る	もる	
\\	ご飯を盛ってください。	ご 飯[はん]を 盛[も]ってください。	ごはん を もって ください	
\\	ご 飯[はん]を
\\	ください。			
\\	針	針[はり]	はり	
\\	母は糸を針に通した。	母[はは]は 糸[いと]を 針[はり]に 通[とお]した。	はは は いと を はり に とおした	
\\	母[はは]は 糸[いと]を
\\	に 通[とお]した。			
\\	釣り	釣[つ]り	つり	
\\	父は釣りが大好きです。	父[ちち]は 釣[つ]りが 大好[だいす]きです。	ちち は つり が だいすき です	
\\	父[ちち]は
\\	が 大好[だいす]きです。			
\\	釣り合い	釣[つ]り 合[あ]い	つりあい	
\\	この紙飛行機は左右の釣り合いがとれていないな。	この 紙飛行機[かみひこうき]は 左右[さゆう]の 釣[つ]り 合[あ]いがとれていないな。	この かみひこうき は さゆう の つりあい が とれて いない な	
\\	この 紙飛行機[かみひこうき]は 左右[さゆう]の
\\	がとれていないな。			
\\	のんき	のんき	のんき	
\\	彼はずいぶんのんきなやつだ。	彼[かれ]はずいぶんのんきなやつだ。	かれ は ずいぶん のんき な やつ だ	
\\	彼[かれ]はずいぶん
\\	なやつだ。			
\\	釣り合う	釣[つ]り 合[あ]う	つりあう	
\\	収入と支出が釣り合っていない。	収入[しゅうにゅう]と 支出[ししゅつ]が 釣[つ]り 合[あ]っていない。	しゅうにゅう と ししゅつ が つりあって いない	
\\	収入[しゅうにゅう]と 支出[ししゅつ]が
\\	鈍い	鈍[にぶ]い	にぶい	
\\	今日は頭の回転が鈍いです。	今日[きょう]は 頭[あたま]の 回転[かいてん]が 鈍[にぶ]いです。	きょう は あたま の かいてん が にぶい です	
\\	今日[きょう]は 頭[あたま]の 回転[かいてん]が
\\	です。			
\\	鈍感	鈍感[どんかん]	どんかん	
\\	あの人は少し鈍感だと思います。	あの 人[ひと]は 少[すこ]し 鈍感[どんかん]だと 思[おも]います。	あの ひと は すこし どんかん だ と おもいます	
\\	あの 人[ひと]は 少[すこ]し
\\	だと 思[おも]います。			
\\	谷	谷[たに]	たに	
\\	その村は深い谷にあるの。	その 村[むら]は 深[ふか]い 谷[たに]にあるの。	その むら は ふかい たに に ある の	
\\	その 村[むら]は 深[ふか]い
\\	にあるの。			
\\	余裕	余裕[よゆう]	よゆう	
\\	出発まで時間の余裕があります。	出発[しゅっぱつ]まで 時間[じかん]の 余裕[よゆう]があります。	しゅっぱつ まで じかん の よゆう が あります	
\\	出発[しゅっぱつ]まで 時間[じかん]の
\\	があります。			
\\	入浴	入浴[にゅうよく]	にゅうよく	
\\	私は毎日寝る前に入浴します。	私[わたし]は 毎日寝[まいにち ね]る 前[まえ]に 入浴[にゅうよく]します。	わたし は まいにち ねる まえ に にゅうよく します	
\\	私[わたし]は 毎日寝[まいにち ね]る 前[まえ]に
\\	します。			
\\	はえ	はえ	はえ	
\\	はえがうるさく飛んでいるね。	はえがうるさく 飛[と]んでいるね。	はえ が うるさく とん でいる ね 。	
\\	がうるさく 飛[と]んでいるね。			
\\	沿う	沿[そ]う	そう	
\\	川に沿って歩いたんだ。	川[かわ]に 沿[そ]って 歩[ある]いたんだ。	かわ に そって あるいた ん だ	
\\	川[かわ]に
\\	歩[ある]いたんだ。			
\\	浜	浜[はま]	はま	
\\	今晩、浜で花火大会がありますよ。	今晩[こんばん]、 浜[はま]で 花火大会[はなび たいかい]がありますよ。	こんばん はま で はなび たいかい が あります よ	
\\	今晩[こんばん]、
\\	で 花火大会[はなび たいかい]がありますよ。			
\\	浜辺	浜辺[はまべ]	はまべ	
\\	浜辺できれいな貝がらを拾いました。	浜辺[はまべ]できれいな 貝[かい]がらを 拾[ひろ]いました。	はまべ で きれい な かいがら を ひろいました	
\\	できれいな 貝[かい]がらを 拾[ひろ]いました。			
\\	湾	湾[わん]	わん	
\\	東京湾は重要な役割を果たしている。	東京[とうきょう] 湾[わん]は 重要[じゅうよう]な 役割[やくわり]を 果[は]たしている。	とうきょうわん は じゅうよう な やくわり を はたして いる	
\\	東京[とうきょう]
\\	は 重要[じゅうよう]な 役割[やくわり]を 果[は]たしている。			
\\	潜る	潜[もぐ]る	もぐる	
\\	彼は長い間海に潜っていたの。	彼[かれ]は 長[なが]い 間海[あいだ うみ]に 潜[もぐ]っていたの。	かれ は ながい あいだ うみ に もぐって いた の	
\\	彼[かれ]は 長[なが]い 間海[あいだ うみ]に
\\	の。			
\\	ばかばかしい	ばかばかしい	ばかばかしい	
\\	ばかばかしい映画だが面白かった。	ばかばかしい 映画[えいが]だが 面白[おもしろ]かった。	ばかばかしい えいが だ が おもしろかった	
\\	映画[えいが]だが 面白[おもしろ]かった。			
\\	冷淡	冷淡[れいたん]	れいたん	
\\	彼は彼女に冷淡な態度をとったね。	彼[かれ]は 彼女[かのじょ]に 冷淡[れいたん]な 態度[たいど]をとったね。	かれ は かのじょ に れいたん な たいど を とった ね	
\\	彼[かれ]は 彼女[かのじょ]に
\\	な 態度[たいど]をとったね。			
\\	泥	泥[どろ]	どろ	
\\	靴が泥だらけになったよ。	靴[くつ]が 泥[どろ]だらけになったよ。	くつ が どろだらけ に なった よ	
\\	靴[くつ]が
\\	だらけになったよ。			
\\	濁る	濁[にご]る	にごる	
\\	その池の水は濁っているね。	その 池[いけ]の 水[みず]は 濁[にご]っているね。	その いけ の みず は にごって いる ね	
\\	その 池[いけ]の 水[みず]は
\\	ね。			
\\	湯飲み	湯飲[ゆの]み	ゆのみ	
\\	湯飲みにはお茶を入れてお茶碗にはご飯をよそいます。	湯飲[ゆの]みにはお 茶[ちゃ]を 入[い]れてお 茶碗[ちゃわん]にはご 飯[はん]をよそいます。	ゆのみ に は おちゃ を いれて お ちゃわん に は ごはん を よそいます	
\\	にはお 茶[ちゃ]を 入[い]れてお 茶碗[ちゃわん]にはご 飯[はん]をよそいます。			
\\	湯気	湯気[ゆげ]	ゆげ	
\\	やかんから湯気が出ていますよ。	やかんから 湯気[ゆげ]が 出[で]ていますよ。	やかん から ゆげ が でて います よ	
\\	やかんから
\\	が 出[で]ていますよ。			
\\	はかり	はかり	はかり	
\\	ケーキを作る時ははかりを使います。	ケーキを 作[つく]る 時[とき]ははかりを 使[つか]います。	けーき を つくる とき は はかり を つかいます 。	
\\	ケーキを 作[つく]る 時[とき]は
\\	を 使[つか]います。			
\\	漏れる	漏[も]れる	もれる	
\\	彼のヘッドフォンから音が漏れているね。	彼[かれ]のヘッドフォンから 音[おと]が 漏[も]れているね。	かれ の へっどふぉん から おと が もれて いる ね	
\\	彼[かれ]のヘッドフォンから 音[おと]が
\\	いるね。			
\\	漏らす	漏[も]らす	もらす	
\\	秘密を漏らしたのは彼です。	秘密[ひみつ]を 漏[も]らしたのは 彼[かれ]です。	ひみつ を もらした の は かれ です	
\\	秘密[ひみつ]を
\\	のは 彼[かれ]です。			
\\	漏る	漏[も]る	もる	
\\	天井から雨が漏るの。	天井[てんじょう]から 雨[あめ]が 漏[も]るの。	てんじょう から あめ が もる の	
\\	天井[てんじょう]から 雨[あめ]が
\\	の。			
\\	濡らす	濡[ぬ]らす	ぬらす	
\\	携帯電話を水で濡らしてしまったんだ。	携帯電話[けいたい でんわ]を 水[みず]で 濡[ぬ]らしてしまったんだ。	けいたい でんわ を みず で ぬらして しまった ん だ	
\\	携帯電話[けいたい でんわ]を 水[みず]で
\\	んだ。			
\\	溝	溝[みぞ]	みぞ	
\\	車が溝にはまったの。	車[くるま]が 溝[みぞ]にはまったの。	くるま が みぞ に はまった の	
\\	車[くるま]が
\\	にはまったの。			
\\	はしご	はしご	はしご	
\\	はしごを使って屋根に上りました。	はしごを 使[つか]って 屋根[やね]に 上[のぼ]りました。	はしご を つかって やね に のぼりました	
\\	を 使[つか]って 屋根[やね]に 上[のぼ]りました。			
\\	吐く	吐[は]く	はく	
\\	彼は乱暴な言葉を吐いたぞ。	彼[かれ]は 乱暴[らんぼう]な 言葉[ことば]を 吐[は]いたぞ。	かれ は らんぼう な ことば を はいた ぞ	
\\	彼[かれ]は 乱暴[らんぼう]な 言葉[ことば]を
\\	ぞ。			
\\	嘆く	嘆[なげ]く	なげく	
\\	嘆いていても何も変わりません。	嘆[なげ]いていても 何[なに]も 変[か]わりません。	なげいていて も なに も かわりません	
\\	も 何[なに]も 変[か]わりません。			
\\	虫歯	虫歯[むしば]	むしば	
\\	虫歯は予防できます。	虫歯[むしば]は 予防[よぼう]できます。	むしば は よぼう できます	
\\	は 予防[よぼう]できます。			
\\	虫	虫[むし]	むし	
\\	庭で秋の虫が鳴いているね。	庭[にわ]で 秋[あき]の 虫[むし]が 鳴[な]いているね。	にわ で あき の むし が ないて いる ね	
\\	庭[にわ]で 秋[あき]の
\\	が 鳴[な]いているね。			
\\	忠実	忠実[ちゅうじつ]	ちゅうじつ	
\\	犬は飼い主に忠実です。	犬[いぬ]は 飼[か]い 主[ぬし]に 忠実[ちゅうじつ]です。	いぬ は かいぬし に ちゅうじつ です	
\\	犬[いぬ]は 飼[か]い 主[ぬし]に
\\	です。			
\\	忠告	忠告[ちゅうこく]	ちゅうこく	
\\	先生からの忠告を聞くべきだよ。	先生[せんせい]からの 忠告[ちゅうこく]を 聞[き]くべきだよ。	せんせい から の ちゅうこく を きく べき だ よ	
\\	先生[せんせい]からの
\\	を 聞[き]くべきだよ。			
\\	はだし	はだし	はだし	
\\	砂浜をはだしでかけ回ったんだ。	砂浜[すなはま]をはだしでかけ 回[まわ]ったんだ。	すなはま を はだし で かけまわった ん だ	
\\	砂浜[すなはま]を
\\	でかけ 回[まわ]ったんだ。			
\\	恵まれる	恵[めぐ]まれる	めぐまれる	
\\	当日は天気に恵まれました。	当日[とうじつ]は 天気[てんき]に 恵[めぐ]まれました。	とうじつ は てんき に めぐまれました	
\\	当日[とうじつ]は 天気[てんき]に
\\	道徳	道徳[どうとく]	どうとく	
\\	子供には道徳をきちんと教えていきたいと思います。	子供[こども]には 道徳[どうとく]をきちんと 教[おし]えていきたいと 思[おも]います。	こども に は どうとく を きちんと おしえて いきたいと おもいます	
\\	子供[こども]には
\\	をきちんと 教[おし]えていきたいと 思[おも]います。			
\\	直径	直径[ちょっけい]	ちょっけい	
\\	この円は直径6センチです。	この 円[えん]は 直径[ちょっけい]6センチです。	この えん は ちょっけい 
\\	せんち です	
\\	この 円[えん]は
\\	6センチです。			
\\	半径	半径[はんけい]	はんけい	
\\	この円の半径は5センチです。	この 円[えん]の 半径[はんけい]は5センチです。	この えん の はんけい は 
\\	せんち です	
\\	この 円[えん]の
\\	は5センチです。			
\\	丼	丼[どんぶり]	どんぶり	
\\	彼はご飯を丼で食べたよ。	彼[かれ]はご 飯[はん]を 丼[どんぶり]で 食[た]べたよ。	かれ は ごはん を どんぶり で たべた よ	
\\	彼[かれ]はご 飯[はん]を
\\	で 食[た]べたよ。			
\\	バツ	バツ	バツ	
\\	バツが三つあるので85点です。	バツが 三[みっ]つあるので85 点[てん]です。	ばつ が みっつ ある の で 
\\	てん です	
\\	が 三[みっ]つあるので85 点[てん]です。			
\\	豆	豆[まめ]	まめ	
\\	今、豆を煮ています。	今[いま]、 豆[まめ]を 煮[に]ています。	いま まめ を にて います	
\\	今[いま]、
\\	を 煮[に]ています。			
\\	粒	粒[つぶ]	つぶ	
\\	その子は大粒の涙を浮かべていたの。	その 子[こ]は 大[おお] 粒[つぶ]の 涙[なみだ]を 浮[う]かべていたの。	その こ は おおつぶ の なみだ を うかべていた の	
\\	その 子[こ]は 大[おお]
\\	の 涙[なみだ]を 浮[う]かべていたの。			
\\	麦	麦[むぎ]	むぎ	
\\	この畑では麦を作っています。	この 畑[はたけ]では 麦[むぎ]を 作[つく]っています。	この はたけ で は むぎ を つくって います	
\\	この 畑[はたけ]では
\\	を 作[つく]っています。			
\\	田畑	田畑[たはた]	たはた	
\\	この村には田畑がたくさんあります。	この 村[むら]には 田畑[たはた]がたくさんあります。	この むら に は たはた が たくさん あります	
\\	この 村[むら]には
\\	がたくさんあります。			
\\	灯油	灯油[とうゆ]	とうゆ	
\\	ストーブの灯油がなくなったわよ。	ストーブの 灯油[とうゆ]がなくなったわよ。	すとーぶ の とうゆ が なくなった わ よ	
\\	ストーブの
\\	がなくなったわよ。			
\\	はねる	はねる	はねる	
\\	彼は車にはねられたけど無事だったの。	彼[かれ]は 車[くるま]にはねられたけど 無事[ぶじ]だったの。	かれ は くるま に はねられた けど ぶじ だった の	
\\	彼[かれ]は 車[くるま]に
\\	けど 無事[ぶじ]だったの。			
\\	電灯	電灯[でんとう]	でんとう	
\\	午後5時には電灯がつきます。	午後5時[ごご 
\\	じ]には 電灯[でんとう]がつきます。	ごご 
\\	じ に は でんとう が つきます	
\\	午後5時[ごご 
\\	じ]には
\\	がつきます。			
\\	炎	炎[ほのお]	ほのお	
\\	ろうそくの炎が部屋を照らしたんだ。	ろうそくの 炎[ほのお]が 部屋[へや]を 照[て]らしたんだ。	ろうそく の ほのお が へや を てらした ん だ	
\\	ろうそくの
\\	が 部屋[へや]を 照[て]らしたんだ。			
\\	皮肉	皮肉[ひにく]	ひにく	
\\	彼は皮肉ばかり言います。	彼[かれ]は 皮肉[ひにく]ばかり 言[い]います。	かれ は ひにく ばかり いいます	
\\	彼[かれ]は
\\	ばかり 言[い]います。			
\\	皮膚	皮膚[ひふ]	ひふ	
\\	冬は皮膚が乾燥しますね。	冬[ふゆ]は 皮膚[ひふ]が 乾燥[かんそう]しますね。	ふゆ は ひふ が かんそう します ね	
\\	冬[ふゆ]は
\\	が 乾燥[かんそう]しますね。			
\\	臭い	臭[にお]い	におい	
\\	ここはいやな臭いがする。	ここはいやな 臭[にお]いがする。	ここ は いや な におい が する 。	
\\	ここはいやな
\\	がする。			
\\	生臭い	生臭[なまぐさ]い	なまぐさい	
\\	まな板が生臭いです。	まな 板[いた]が 生臭[なまぐさ]いです。	まないた が なまぐさい です	
\\	まな 板[いた]が
\\	です。			
\\	はれる	はれる	はれる	
\\	今朝は目がはれています。	今朝[けさ]は 目[め]がはれています。	けさ は め が はれて います	
\\	今朝[けさ]は 目[め]が
\\	臭う	臭[にお]う	におう	
\\	流しが臭うのできれいにしたよ。	流[なが]しが 臭[にお]うのできれいにしたよ。	ながし が におう の で きれい に した よ	
\\	流[なが]しが
\\	のできれいにしたよ。			
\\	面倒臭い	面倒臭[めんどうくさ]い	めんどうくさい	
\\	この計算は面倒臭いなあ。	この 計算[けいさん]は 面倒臭[めんどうくさ]いなあ。	この けいさん は めんどうくさい なあ	
\\	この 計算[けいさん]は
\\	なあ。			
\\	匂う	匂[にお]う	におう	
\\	バラの花が甘く匂っているね。	バラの 花[はな]が 甘[あま]く 匂[にお]っているね。	ばら の はな が あまく におって いる ね	
\\	バラの 花[はな]が 甘[あま]く
\\	ね。			
\\	同居	同居[どうきょ]	どうきょ	
\\	私はまだ両親と同居しています。	私[わたし]はまだ 両親[りょうしん]と 同居[どうきょ]しています。	わたし は まだ りょうしん と どうきょ して います	
\\	私[わたし]はまだ 両親[りょうしん]と
\\	しています。			
\\	掘る	掘[ほ]る	ほる	
\\	ここに穴を掘りましょう。	ここに 穴[あな]を 掘[ほ]りましょう。	ここ に あな を ほりましょう	
\\	ここに 穴[あな]を
\\	ひく	ひく	ひく	
\\	車にひかれないよう気をつけなさい。	車[くるま]にひかれないよう 気[き]をつけなさい。	くるま に ひかれない よう き を つけなさい 。	
\\	車[くるま]に
\\	気[き]をつけなさい。			
\\	塀	塀[へい]	へい	
\\	猫が塀の上で寝ているぞ。	猫[ねこ]が 塀[へい]の 上[うえ]で 寝[ね]ているぞ。	ねこ が へい の うえ で ねて いる ぞ	
\\	猫[ねこ]が
\\	の 上[うえ]で 寝[ね]ているぞ。			
\\	大層	大層[たいそう]	たいそう	
\\	彼は大層喜んでいました。	彼[かれ]は 大層[たいそう] 喜[よろこ]んでいました。	かれ は たいそう よろこんで いました	
\\	彼[かれ]は
\\	喜[よろこ]んでいました。			
\\	履歴	履歴[りれき]	りれき	
\\	最近使ったファイルは、履歴からすぐ開けます。	最近使[さいきん つか]ったファイルは、 履歴[りれき]からすぐ 開[ひら]けます。	さいきん つかった ふぁいる は、 りれき から すぐ ひらけます	
\\	最近使[さいきん つか]ったファイルは、
\\	からすぐ 開[ひら]けます。			
\\	履歴書	履歴書[りれきしょ]	りれきしょ	
\\	面接のために履歴書を書きました。	面接[めんせつ]のために 履歴書[りれきしょ]を 書[か]きました。	めんせつ の ため に りれきしょ を かきました	
\\	面接[めんせつ]のために
\\	を 書[か]きました。			
\\	履物	履物[はきもの]	はきもの	
\\	履物は靴箱に入れてください。	履物[はきもの]は 靴箱[くつばこ]に 入[い]れてください。	はきもの は くつばこ に いれて ください	
\\	は 靴箱[くつばこ]に 入[い]れてください。			
\\	ヒント	ヒント	ヒント	
\\	何かヒントをください。	何[なに]かヒントをください。	なにか ひんと を ください	
\\	何[なに]か
\\	をください。			
\\	戸棚	戸棚[とだな]	とだな	
\\	この皿を戸棚にしまってください。	この 皿[さら]を 戸棚[とだな]にしまってください。	この さら を とだな に しまって ください	
\\	この 皿[さら]を
\\	にしまってください。			
\\	扉	扉[とびら]	とびら	
\\	彼は扉を開けたんだ。	彼[かれ]は 扉[とびら]を 開[あ]けたんだ。	かれ は とびら を あけた ん だ	
\\	彼[かれ]は
\\	を 開[あ]けたんだ。			
\\	羽根	羽根[はね]	はね	
\\	このペンは鳥の羽根で作られています。	このペンは 鳥[とり]の 羽根[はね]で 作[つく]られています。	この ぺん は とり の はね で つくられて います	
\\	このペンは 鳥[とり]の
\\	で 作[つく]られています。			
\\	翌日	翌日[よくじつ]	よくじつ	
\\	翌日、彼に会いに行きました。	翌日[よくじつ]、 彼[かれ]に 会[あ]いに 行[い]きました。	よくじつ かれ に あい に いきました	
\\	、 彼[かれ]に 会[あ]いに 行[い]きました。			
\\	翌朝	翌朝[よくあさ]	よくあさ	
\\	翌朝は快晴だったね。	翌朝[よくあさ]は 快晴[かいせい]だったね。	よくあさ は かいせい だった ね	
\\	は 快晴[かいせい]だったね。			
\\	翌年	翌年[よくねん]	よくねん	
\\	彼女は大学卒業の翌年に結婚したの。	彼女[かのじょ]は 大学卒業[だいがく そつぎょう]の 翌年[よくねん]に 結婚[けっこん]したの。	かのじょ は だいがく そつぎょう の よくねん に けっこん した の	
\\	彼女[かのじょ]は 大学卒業[だいがく そつぎょう]の
\\	に 結婚[けっこん]したの。			
\\	ファックス	ファックス	ファックス	
\\	詳細はファックスで送ります。	詳細[しょうさい]はファックスで 送[おく]ります。	しょうさい は ふぁっくす で おくります	
\\	詳細[しょうさい]は
\\	で 送[おく]ります。			
\\	群れ	群[む]れ	むれ	
\\	湖に鳥の群れがいたよ。	湖[みずうみ]に 鳥[とり]の 群[む]れがいたよ。	みずうみ に とり の むれ が いた よ	
\\	湖[みずうみ]に 鳥[とり]の
\\	がいたよ。			
\\	殴る	殴[なぐ]る	なぐる	
\\	彼は思わず友人を殴ったの。	彼[かれ]は 思[おも]わず 友人[ゆうじん]を 殴[なぐ]ったの。	かれ は おもわず ゆうじん を なぐった の	
\\	彼[かれ]は 思[おも]わず 友人[ゆうじん]を
\\	の。			
\\	要旨	要旨[ようし]	ようし	
\\	会議の要旨は次の通りです。	会議[かいぎ]の 要旨[ようし]は 次[つぎ]の 通[とお]りです。	かいぎ の ようし は つぎ の とおり です	
\\	会議[かいぎ]の
\\	は 次[つぎ]の 通[とお]りです。			
\\	肌	肌[はだ]	はだ	
\\	彼女は肌がとても白いですね。	彼女[かのじょ]は 肌[はだ]がとても 白[しろ]いですね。	かのじょ は はだ が とても しろい です ね	
\\	彼女[かのじょ]は
\\	がとても 白[しろ]いですね。			
\\	肌着	肌着[はだぎ]	はだぎ	
\\	寒いので暖かい肌着を着ました。	寒[さむ]いので 暖[あたた]かい 肌着[はだぎ]を 着[き]ました。	さむい の で あたたかい はだぎ を きました	
\\	寒[さむ]いので 暖[あたた]かい
\\	を 着[き]ました。			
\\	ふもと	ふもと	ふもと	
\\	彼は山のふもとに住んでいるよ。	彼[かれ]は 山[やま]のふもとに 住[す]んでいるよ。	かれ は やま の ふもと に すんで いる よ	
\\	彼[かれ]は 山[やま]の
\\	に 住[す]んでいるよ。			
\\	肌色	肌色[はだいろ]	はだいろ	
\\	彼女は肌色のシャツを着ています。	彼女[かのじょ]は 肌色[はだいろ]のシャツを 着[き]ています。	かのじょ は はだいろ の しゃつ を きて います	
\\	彼女[かのじょ]は
\\	のシャツを 着[き]ています。			
\\	腸	腸[ちょう]	ちょう	
\\	彼は腸の手術を受けたの。	彼[かれ]は 腸[ちょう]の 手術[しゅじゅつ]を 受[う]けたの。	かれ は ちょう の しゅじゅつ を うけた の	
\\	彼[かれ]は
\\	の 手術[しゅじゅつ]を 受[う]けたの。			
\\	大胆	大胆[だいたん]	だいたん	
\\	彼女はずいぶん大胆なことを言うね。	彼女[かのじょ]はずいぶん 大胆[だいたん]なことを 言[い]うね。	かのじょ は ずいぶん だいたん な こと を いう ね	
\\	彼女[かのじょ]はずいぶん
\\	なことを 言[い]うね。			
\\	幕	幕[まく]	まく	
\\	ステージの幕が上がったんだ。	ステージの 幕[まく]が 上[あ]がったんだ。	すてーじ の まく が あがった ん だ	
\\	ステージの
\\	が 上[あ]がったんだ。			
\\	夕暮れ	夕暮[ゆうぐ]れ	ゆうぐれ	
\\	夕暮れの空がきれいですね。	夕暮[ゆうぐ]れの 空[そら]がきれいですね。	ゆうぐれ の そら が きれい です ね	
\\	の 空[そら]がきれいですね。			
\\	フルーツ	フルーツ	フルーツ	
\\	おいしそうなフルーツゼリーだね。	おいしそうなフルーツゼリーだね。	おいし そう な ふるーつぜりー だ ね	
\\	おいしそうな
\\	ゼリーだね。			
\\	墓	墓[はか]	はか	
\\	祖父の墓は近くにあります。	祖父[そふ]の 墓[はか]は 近[ちか]くにあります。	そふ の はか は ちかく に あります	
\\	祖父[そふ]の
\\	は 近[ちか]くにあります。			
\\	墓地	墓地[ぼち]	ぼち	
\\	寺のとなりに墓地があります。	寺[てら]のとなりに 墓地[ぼち]があります。	てら の となり に ぼち が あります	
\\	寺[てら]のとなりに
\\	があります。			
\\	墓参り	墓参[はかまい]り	はかまいり	
\\	明日は家族で墓参りに行きます。	明日[あした]は 家族[かぞく]で 墓参[はかまい]りに 行[い]きます。	あした は かぞく で はかまいり に いきます	
\\	明日[あした]は 家族[かぞく]で
\\	に 行[い]きます。			
\\	芽	芽[め]	め	
\\	チューリップの芽が出ました。	チューリップの 芽[め]が 出[で]ました。	ちゅーりっぷ の め が でました	
\\	チューリップの
\\	が 出[で]ました。			
\\	葬式	葬式[そうしき]	そうしき	
\\	彼の葬式は明日です。	彼[かれ]の 葬式[そうしき]は 明日[あした]です。	かれ の そうしき は あした です	
\\	彼[かれ]の
\\	は 明日[あした]です。			
\\	へそ	へそ	へそ	
\\	カエルにはへそがないんだ。	カエルにはへそがないんだ。	かえる に は へそ が ない ん だ	
\\	カエルには
\\	がないんだ。			
\\	礼儀	礼儀[れいぎ]	れいぎ	
\\	彼女は礼儀が身に付いています。	彼女[かのじょ]は 礼儀[れいぎ]が 身[み]に 付[つ]いています。	かのじょ は れいぎ が み に ついて います	
\\	彼女[かのじょ]は
\\	が 身[み]に 付[つ]いています。			
\\	憎む	憎[にく]む	にくむ	
\\	彼はもう彼女を憎んではいないよ。	彼[かれ]はもう 彼女[かのじょ]を 憎[にく]んではいないよ。	かれ は もう かのじょ を にくんで は いない よ	
\\	彼[かれ]はもう 彼女[かのじょ]を
\\	はいないよ。			
\\	憎しみ	憎[にく]しみ	にくしみ	
\\	彼の心は憎しみに満ちていたんだ。	彼[かれ]の 心[こころ]は 憎[にく]しみに 満[み]ちていたんだ。	かれ の こころ は にくしみ に みちて いた ん だ	
\\	彼[かれ]の 心[こころ]は
\\	に 満[み]ちていたんだ。			
\\	憎らしい	憎[にく]らしい	にくらしい	
\\	妹は時々憎らしいことを言うんだ。	妹[いもうと]は 時々[ときどき] 憎[にく]らしいことを 言[い]うんだ。	いもうと は ときどき にくらしい こと を いう ん だ	
\\	妹[いもうと]は 時々[ときどき]
\\	ことを 言[い]うんだ。			
\\	憎い	憎[にく]い	にくい	
\\	彼が憎いですか。	彼[かれ]が 憎[にく]いですか。	かれ が にくい です か	
\\	彼[かれ]が
\\	ですか。			
\\	仏	仏[ほとけ]	ほとけ	
\\	仏の教えを勉強しました。	仏[ほとけ]の 教[おし]えを 勉強[べんきょう]しました。	ほとけ の おしえ を べんきょう しました	
\\	の 教[おし]えを 勉強[べんきょう]しました。			
\\	べたべた	べたべた	べたべた	
\\	カップルがべたべたしているね。	カップルがべたべたしているね。	かっぷる が べたべた して いる ね	
\\	カップルが
\\	しているね。			
\\	仏教	仏教[ぶっきょう]	ぶっきょう	
\\	お葬式は仏教で行うことが多いです。	お 葬式[そうしき]は 仏教[ぶっきょう]で 行[おこな]うことが 多[おお]いです。	おそうしき は ぶっきょう で おこなう こと が おおい です	
\\	お 葬式[そうしき]は
\\	で 行[おこな]うことが 多[おお]いです。			
\\	坊さん	坊[ぼう]さん	ぼうさん	
\\	私の友人はお坊さんをしているの。	私[わたし]の 友人[ゆうじん]はお 坊[ぼう]さんをしているの。	わたし の ゆうじん は おぼうさん を して いる の	
\\	私[わたし]の 友人[ゆうじん]はお
\\	をしているの。			
\\	竹	竹[たけ]	たけ	
\\	この笛は竹でできている。	この 笛[ふえ]は 竹[たけ]でできている。	この ふえ は たけ で できて いる	
\\	この 笛[ふえ]は
\\	でできている。			
\\	筆者	筆者[ひっしゃ]	ひっしゃ	
\\	筆者の趣旨を考えてください。	筆者[ひっしゃ]の 趣旨[しゅし]を 考[かんが]えてください。	ひっしゃ の しゅし を かんがえて ください	
\\	の 趣旨[しゅし]を 考[かんが]えてください。			
\\	筆記試験	筆記試験[ひっきしけん]	ひっきしけん	
\\	その会社の筆記試験は難しかったよ。	その 会社[かいしゃ]の 筆記試験[ひっきしけん]は 難[むずか]しかったよ。	その かいしゃ の ひっきしけん は むずかしかった よ	
\\	その 会社[かいしゃ]の
\\	は 難[むずか]しかったよ。			
\\	ほうき	ほうき	ほうき	
\\	ほうきで庭を掃除しました。	ほうきで 庭[にわ]を 掃除[そうじ]しました。	ほうき で にわ を そうじ しました 。	
\\	で 庭[にわ]を 掃除[そうじ]しました。			
\\	筆記用具	筆記用具[ひっきようぐ]	ひっきようぐ	
\\	今日は筆記用具を忘れました。	今日[きょう]は 筆記用具[ひっきようぐ]を 忘[わす]れました。	きょう は ひっきようぐ を わすれました	
\\	今日[きょう]は
\\	を 忘[わす]れました。			
\\	筆	筆[ふで]	ふで	
\\	彼は筆で手紙を書いたの。	彼[かれ]は 筆[ふで]で 手紙[てがみ]を 書[か]いたの。	かれ は ふで で てがみ を かいた の	
\\	彼[かれ]は
\\	で 手紙[てがみ]を 書[か]いたの。			
\\	笛	笛[ふえ]	ふえ	
\\	彼は笛を吹くのが上手いね。	彼[かれ]は 笛[ふえ]を 吹[ふ]くのが 上手[うま]いね。	かれ は ふえ を ふく の が うまい ね	
\\	彼[かれ]は
\\	を 吹[ふ]くのが 上手[うま]いね。			
\\	名簿	名簿[めいぼ]	めいぼ	
\\	これが参加者の名簿です。	これが 参加者[さんかしゃ]の 名簿[めいぼ]です。	これ が さんかしゃ の めいぼ です	
\\	これが 参加者[さんかしゃ]の
\\	です。			
\\	分裂	分裂[ぶんれつ]	ぶんれつ	
\\	会社が二つの派閥に分裂しているんだ。	会社[かいしゃ]が 二[ふた]つの 派閥[はばつ]に 分裂[ぶんれつ]しているんだ。	かいしゃ が ふたつ の はばつ に ぶんれつ して いる ん だ	
\\	会社[かいしゃ]が 二[ふた]つの 派閥[はばつ]に
\\	しているんだ。			
\\	ポスター	ポスター	ポスター	
\\	電柱にポスターが貼ってあった。	電柱[でんちゅう]にポスターが 貼[は]ってあった。	でんちゅう に ぽすたー が はって あった	
\\	電柱[でんちゅう]に
\\	が 貼[は]ってあった。			
\\	布	布[ぬの]	ぬの	
\\	この布はカーテンに使えます。	この 布[ぬの]はカーテンに 使[つか]えます。	この ぬの は かーてん に つかえます	
\\	この
\\	はカーテンに 使[つか]えます。			
\\	包丁	包丁[ほうちょう]	ほうちょう	
\\	包丁で指を切った。	包丁[ほうちょう]で 指[ゆび]を 切[き]った。	ほうちょう で ゆび を きった	
\\	で 指[ゆび]を 切[き]った。			
\\	包帯	包帯[ほうたい]	ほうたい	
\\	彼女は手に包帯を巻いていたんだ。	彼女[かのじょ]は 手[て]に 包帯[ほうたい]を 巻[ま]いていたんだ。	かのじょ は て に ほうたい を まいて いた ん だ	
\\	彼女[かのじょ]は 手[て]に
\\	を 巻[ま]いていたんだ。			
\\	包み	包[つつ]み	つつみ	
\\	この包みは誰のですか。	この 包[つつ]みは 誰[だれ]のですか。	この つつみ は だれ の です か	
\\	この
\\	は 誰[だれ]のですか。			
\\	包み紙	包[つつ]み 紙[がみ]	つつみがみ	
\\	包み紙はたたんでください。	包[つつ]み 紙[がみ]はたたんでください。	つつみがみ は たたんで ください	
\\	はたたんでください。			
\\	抱く	抱[だ]く	だく	
\\	祖母がうちの猫を抱いているわよ。	祖母[そぼ]がうちの 猫[ねこ]を 抱[だ]いているわよ。	そぼ が うち の ねこ を だいて いる わ よ	
\\	祖母[そぼ]がうちの 猫[ねこ]を
\\	いるわよ。			
\\	ほどく	ほどく	ほどく	
\\	靴のひもをほどいたの。	靴[くつ]のひもをほどいたの。	くつ の ひも を ほどいた の	
\\	靴[くつ]のひもを
\\	の。			
\\	文句	文句[もんく]	もんく	
\\	彼女はいつも文句ばかり言う。	彼女[かのじょ]はいつも 文句[もんく]ばかり 言[い]う。	かのじょ は いつも もんく ばかり いう	
\\	彼女[かのじょ]はいつも
\\	ばかり 言[い]う。			
\\	中旬	中旬[ちゅうじゅん]	ちゅうじゅん	
\\	来月中旬にフランスへ行きます。	来月[らいげつ] 中旬[ちゅうじゅん]にフランスへ 行[い]きます。	らいげつ ちゅうじゅん に ふらんす へ いきます 。	
\\	来月[らいげつ]
\\	にフランスへ 行[い]きます。			
\\	平凡	平凡[へいぼん]	へいぼん	
\\	彼はごく平凡な人です。	彼[かれ]はごく 平凡[へいぼん]な 人[ひと]です。	かれ は ごく へいぼん な ひと です	
\\	彼[かれ]はごく
\\	な 人[ひと]です。			
\\	通帳	通帳[つうちょう]	つうちょう	
\\	銀行の通帳を持ってきてください。	銀行[ぎんこう]の 通帳[つうちょう]を 持[も]ってきてください。	ぎんこう の つうちょう を もって きて ください	
\\	銀行[ぎんこう]の
\\	を 持[も]ってきてください。			
\\	妊娠	妊娠[にんしん]	にんしん	
\\	彼女が妊娠したそうです。	彼女[かのじょ]が 妊娠[にんしん]したそうです。	かのじょ が にんしん した そう です	
\\	彼女[かのじょ]が
\\	したそうです。			
\\	ほほ	ほほ	ほほ	
\\	彼女は真っ赤なほほをしているね。	彼女[かのじょ]は 真[ま]っ 赤[か]なほほをしているね。	かのじょ は まっか な ほほ を している ね 。	
\\	彼女[かのじょ]は 真[ま]っ 赤[か]な
\\	をしているね。			
\\	同姓	同姓[どうせい]	どうせい	
\\	日本では夫婦同姓が一般的よ。	日本[にっぽん]では 夫婦[ふうふ] 同姓[どうせい]が 一般的[いっぱんてき]よ。	にっぽん で は ふうふ どうせい が いっぱんてき よ	
\\	日本[にっぽん]では 夫婦[ふうふ]
\\	が 一般的[いっぱんてき]よ。			
\\	嫁	嫁[よめ]	よめ	
\\	姉が嫁に行ったの。	姉[あね]が 嫁[よめ]に 行[い]ったの。	あね が よめ に いった の	
\\	姉[あね]が
\\	に 行[い]ったの。			
\\	花嫁	花嫁[はなよめ]	はなよめ	
\\	花嫁が父親と一緒に入ってきましたね。	花嫁[はなよめ]が 父親[ちちおや]と 一緒[いっしょ]に 入[はい]ってきましたね。	はなよめ が ちちおや と いっしょ に はいって きました ね	
\\	が 父親[ちちおや]と 一緒[いっしょ]に 入[はい]ってきましたね。			
\\	花婿	花婿[はなむこ]	はなむこ	
\\	花婿と花嫁が並んで登場したんだ。	花婿[はなむこ]と 花嫁[はなよめ]が 並[なら]んで 登場[とうじょう]したんだ。	はなむこ と はなよめ が ならんで とうじょう した ん だ	
\\	と 花嫁[はなよめ]が 並[なら]んで 登場[とうじょう]したんだ。			
\\	幼児	幼児[ようじ]	ようじ	
\\	幼児は入場無料です。	幼児[ようじ]は 入場無料[にゅうじょう むりょう]です。	ようじ は にゅうじょう むりょう です	
\\	は 入場無料[にゅうじょう むりょう]です。			
\\	ほほえむ	ほほえむ	ほほえむ	
\\	彼女は私にほほえんだよ。	彼女[かのじょ]は 私[わたし]にほほえんだよ。	かのじょ は わたし に ほほえんだ よ	
\\	彼女[かのじょ]は 私[わたし]に
\\	よ。			
\\	眠たい	眠[ねむ]たい	ねむたい	
\\	まだ8時なのに、もう眠たいです。	まだ8 時[じ]なのに、もう 眠[ねむ]たいです。	まだ 
\\	じ な の に もう ねむたい です	
\\	まだ8 時[じ]なのに、もう
\\	です。			
\\	眠り	眠[ねむ]り	ねむり	
\\	彼女はいつもより早く眠りに着いたよ。	彼女[かのじょ]はいつもより 早[はや]く 眠[ねむ]りに 着[つ]いたよ。	かのじょ は いつも より はやく ねむり に ついた よ	
\\	彼女[かのじょ]はいつもより 早[はや]く
\\	に 着[つ]いたよ。			
\\	眺める	眺[なが]める	ながめる	
\\	猫が窓から外を眺めているよ。	猫[ねこ]が 窓[まど]から 外[そと]を 眺[なが]めているよ。	ねこ が まど から そと を ながめて いる よ	
\\	猫[ねこ]が 窓[まど]から 外[そと]を
\\	いるよ。			
\\	眺め	眺[なが]め	ながめ	
\\	ここからの眺めは最高です。	ここからの 眺[なが]めは 最高[さいこう]です。	ここ からの ながめ は さいこう です	
\\	ここからの
\\	は 最高[さいこう]です。			
\\	同封	同封[どうふう]	どうふう	
\\	手紙に写真が同封されていたよ。	手紙[てがみ]に 写真[しゃしん]が 同封[どうふう]されていたよ。	てがみ に しゃしん が どうふう されて いた よ	
\\	手紙[てがみ]に 写真[しゃしん]が
\\	されていたよ。			
\\	封	封[ふう]	ふう	
\\	手紙に封をしたよ。	手紙[てがみ]に 封[ふう]をしたよ。	てがみ に ふう を した よ	
\\	手紙[てがみ]に
\\	をしたよ。			
\\	まく	まく	まく	
\\	庭に花の種をまきました。	庭[にわ]に 花[はな]の 種[たね]をまきました。	にわ に はな の たね を まきました	
\\	庭[にわ]に 花[はな]の 種[たね]を
\\	呼び掛ける	呼[よ]び 掛[か]ける	よびかける	
\\	友人達に協力を呼び掛けたの。	友人達[ゆうじんたち]に 協力[きょうりょく]を 呼[よ]び 掛[か]けたの。	ゆうじんたち に きょうりょく を よびかけた の	
\\	友人達[ゆうじんたち]に 協力[きょうりょく]を
\\	の。			
\\	引っ掛かる	引[ひ]っ 掛[か]かる	ひっかかる	
\\	魚の骨がのどに引っ掛かった。	魚[さかな]の 骨[ほね]がのどに 引[ひ]っ 掛[か]かった。	さかな の ほね が のど に ひっかかった	
\\	魚[さかな]の 骨[ほね]がのどに
\\	話し掛ける	話[はな]し 掛[か]ける	はなしかける	
\\	知らない人が話し掛けてきた。	知[し]らない 人[ひと]が 話[はな]し 掛[か]けてきた。	しらない ひと が はなしかけて きた	
\\	知[し]らない 人[ひと]が
\\	通り掛かる	通[とお]り 掛[か]かる	とおりかかる	
\\	その店を通り掛かったらバーゲンをやっていた。	その 店[みせ]を 通[とお]り 掛[か]かったらバーゲンをやっていた。	その みせ を とおりかかったら ばーげん を やっていた 。	
\\	その 店[みせ]を
\\	バーゲンをやっていた。			
\\	引っ掛ける	引[ひ]っ 掛[か]ける	ひっかける	
\\	その選手はハードルに足を引っ掛けたんだ。	その 選手[せんしゅ]はハードルに 足[あし]を 引[ひ]っ 掛[か]けたんだ。	その せんしゅ は はーどる に あし を ひっかけた ん だ	
\\	その 選手[せんしゅ]はハードルに 足[あし]を
\\	んだ。			
\\	まとめ	まとめ	まとめ	
\\	論文のまとめを書いています。	論文[ろんぶん]のまとめを 書[か]いています。	ろんぶん の まとめ を かいています 。	
\\	論文[ろんぶん]の
\\	を 書[か]いています。			
\\	寄り掛かる	寄[よ]り 掛[か]かる	よりかかる	
\\	彼はフェンスに寄り掛かったの。	彼[かれ]はフェンスに 寄[よ]り 掛[か]かったの。	かれ は ふぇんす に よりかかった の 。	
\\	彼[かれ]はフェンスに
\\	の。			
\\	措置	措置[そち]	そち	
\\	被災者を救うための特別な措置が取られました。	被災者[ひさいしゃ]を 救[すく]うための 特別[とくべつ]な 措置[そち]が 取[と]られました。	ひさいしゃ を すくう ため の とくべつ な そち が とられました	
\\	被災者[ひさいしゃ]を 救[すく]うための 特別[とくべつ]な
\\	が 取[と]られました。			
\\	拝見	拝見[はいけん]	はいけん	
\\	あなたの著書を拝見しました。	あなたの 著書[ちょしょ]を 拝見[はいけん]しました。	あなた の ちょしょ を はいけん しました	
\\	あなたの 著書[ちょしょ]を
\\	しました。			
\\	控える	控[ひか]える	ひかえる	
\\	最近甘いものを控えています。	最近[さいきん] 甘[あま]いものを 控[ひか]えています。	さいきん あまい もの を ひかえて います	
\\	最近[さいきん] 甘[あま]いものを
\\	控え室	控[ひか]え 室[しつ]	ひかえしつ	
\\	ここはお客様用の控え室です。	ここはお 客様用[きゃくさま よう]の 控[ひか]え 室[しつ]です。	ここ は おきゃくさま よう の ひかえしつ です 。	
\\	ここはお 客様用[きゃくさま よう]の
\\	です。			
\\	まね	まね	まね	
\\	娘はよく私のまねをします。	娘[むすめ]はよく 私[わたし]のまねをします。	むすめ は よく わたし の まね を します	
\\	娘[むすめ]はよく 私[わたし]の
\\	をします。			
\\	握り締める	握[にぎ]り 締[し]める	にぎりしめる	
\\	彼は両手を握り締めたの。	彼[かれ]は 両手[りょうて]を 握[にぎ]り 締[し]めたの。	かれ は りょうて を にぎりしめた の	
\\	彼[かれ]は 両手[りょうて]を
\\	の。			
\\	揃う	揃[そろ]う	そろう	
\\	朝は家族全員が揃って食事します。	朝[あさ]は 家族全員[かぞく ぜんいん]が 揃[そろ]って 食事[しょくじ]します。	あさ は かぞく ぜんいん が そろって しょくじ します	
\\	朝[あさ]は 家族全員[かぞく ぜんいん]が
\\	食事[しょくじ]します。			
\\	揃える	揃[そろ]える	そろえる	
\\	花は長さを揃えて花瓶にさしましょう。	花[はな]は 長[なが]さを 揃[そろ]えて 花瓶[かびん]にさしましょう。	はな は ながさ を そろえて かびん に さしましょう	
\\	花[はな]は 長[なが]さを
\\	花瓶[かびん]にさしましょう。			
\\	出迎え	出迎[でむか]え	でむかえ	
\\	空港へ彼女を出迎えに行きました。	空港[くうこう]へ 彼女[かのじょ]を 出迎[でむか]えに 行[い]きました。	くうこう へ かのじょ を でむかえ に いきました	
\\	空港[くうこう]へ 彼女[かのじょ]を
\\	に 行[い]きました。			
\\	迎え	迎[むか]え	むかえ	
\\	駅まで迎えに来て下さい。	駅[えき]まで 迎[むか]えに 来[き]て 下[くだ]さい。	えき まで むかえ に きて ください	
\\	駅[えき]まで
\\	に 来[き]て 下[くだ]さい。			
\\	まねる	まねる	まねる	
\\	私の動きをまねてください。	私[わたし]の 動[うご]きをまねてください。	わたし の うごき を まねて ください	
\\	私[わたし]の 動[うご]きを
\\	ください。			
\\	遂げる	遂[と]げる	とげる	
\\	この会社は急成長を遂げました。	この 会社[かいしゃ]は 急成長[きゅうせいちょう]を 遂[と]げました。	この かいしゃ は きゅうせいちょう を とげました	
\\	この 会社[かいしゃ]は 急成長[きゅうせいちょう]を
\\	巡る	巡[めぐ]る	めぐる	
\\	明日から温泉を巡る旅に出ます。	明日[あした]から 温泉[おんせん]を 巡[めぐ]る 旅[たび]に 出[で]ます。	あした から おんせん を めぐる たび に でます	
\\	明日[あした]から 温泉[おんせん]を
\\	旅[たび]に 出[で]ます。			
\\	待遇	待遇[たいぐう]	たいぐう	
\\	社員たちは待遇の改善を要求している。	社員[しゃいん]たちは 待遇[たいぐう]の 改善[かいぜん]を 要求[ようきゅう]している。	しゃいんたち は たいぐう の かいぜん を ようきゅう して いる	
\\	社員[しゃいん]たちは
\\	の 改善[かいぜん]を 要求[ようきゅう]している。			
\\	俳句	俳句[はいく]	はいく	
\\	「さくら」という言葉を使って俳句を書いたよ。	「さくら」という 言葉[ことば]を 使[つか]って 俳句[はいく]を 書[か]いたよ。	「さくら」 という ことば を つかって はいく を かいた よ	
\\	「さくら」という 言葉[ことば]を 使[つか]って
\\	を 書[か]いたよ。			
\\	俳優	俳優[はいゆう]	はいゆう	
\\	彼は俳優です。	彼[かれ]は 俳優[はいゆう]です。	かれ は はいゆう です	
\\	彼[かれ]は
\\	です。			
\\	敏感	敏感[びんかん]	びんかん	
\\	彼女は流行に敏感だね。	彼女[かのじょ]は 流行[りゅうこう]に 敏感[びんかん]だね。	かのじょ は りゅうこう に びんかん だ ね	
\\	彼女[かのじょ]は 流行[りゅうこう]に
\\	だね。			
\\	まぶた	まぶた	まぶた	
\\	眠くてまぶたが重くなってきた。	眠[ねむ]くてまぶたが 重[おも]くなってきた。	ねむくて まぶた が おもく なってきた 。	
\\	眠[ねむ]くて
\\	が 重[おも]くなってきた。			
\\	倣う	倣[なら]う	ならう	
\\	この街はロンドンに倣って作られました。	この 街[まち]はロンドンに 倣[なら]って 作[つく]られました。	この まち は ろんどん に ならって つくられました	
\\	この 街[まち]はロンドンに
\\	作[つく]られました。			
\\	寮	寮[りょう]	りょう	
\\	彼は会社の寮に住んでいます。	彼[かれ]は 会社[かいしゃ]の 寮[りょう]に 住[す]んでいます。	かれ は かいしゃ の りょう に すんで います	
\\	彼[かれ]は 会社[かいしゃ]の
\\	に 住[す]んでいます。			
\\	偏見	偏見[へんけん]	へんけん	
\\	彼は世の中の偏見と戦ったんだ。	彼[かれ]は 世[よ]の 中[なか]の 偏見[へんけん]と 戦[たたか]ったんだ。	かれ は よのなか の へんけん と たたかった ん だ	
\\	彼[かれ]は 世[よ]の 中[なか]の
\\	と 戦[たたか]ったんだ。			
\\	何遍	何遍[なんべん]	なんべん	
\\	この映画は何遍も見ました。	この 映画[えいが]は 何遍[なんべん]も 見[み]ました。	この えいが は なんべん も みました	
\\	この 映画[えいが]は
\\	も 見[み]ました。			
\\	宣伝	宣伝[せんでん]	せんでん	
\\	その商品の宣伝をよく見かけます。	その 商品[しょうひん]の 宣伝[せんでん]をよく 見[み]かけます。	その しょうひん の せんでん を よく みかけます	
\\	その 商品[しょうひん]の
\\	をよく 見[み]かけます。			
\\	まれ	まれ	まれ	
\\	彼が仕事を休むのはまれです。	彼[かれ]が 仕事[しごと]を 休[やす]むのはまれです。	かれ が しごと を やすむ の は まれ です	
\\	彼[かれ]が 仕事[しごと]を 休[やす]むのは
\\	です。			
\\	展覧会	展覧会[てんらんかい]	てんらんかい	
\\	昨日、絵の展覧会に行ってきました。	昨日[きのう]、 絵[え]の 展覧会[てんらんかい]に 行[い]ってきました。	きのう え の てんらんかい に いって きました	
\\	昨日[きのう]、 絵[え]の
\\	に 行[い]ってきました。			
\\	総理大臣	総理大臣[そうりだいじん]	そうりだいじん	
\\	国民の6割が総理大臣を支持しています。	国民[こくみん]の6 割[わり]が 総理大臣[そうりだいじん]を 支持[しじ]しています。	こくみん の 
\\	わり が そうりだいじん を しじ して います	
\\	国民[こくみん]の6 割[わり]が
\\	を 支持[しじ]しています。			
\\	大臣	大臣[だいじん]	だいじん	
\\	大臣が汚職で逮捕されました。	大臣[だいじん]が 汚職[おしょく]で 逮捕[たいほ]されました。	だいじん が おしょく で たいほ されました	
\\	が 汚職[おしょく]で 逮捕[たいほ]されました。			
\\	臨む	臨[のぞ]む	のぞむ	
\\	選手は最高の状態で大会に臨んだわ。	選手[せんしゅ]は 最高[さいこう]の 状態[じょうたい]で 大会[たいかい]に 臨[のぞ]んだわ。	せんしゅ は さいこう の じょうたい で たいかい に のぞんだ わ	
\\	選手[せんしゅ]は 最高[さいこう]の 状態[じょうたい]で 大会[たいかい]に
\\	わ。			
\\	臨時	臨時[りんじ]	りんじ	
\\	彼女は臨時の店員です。	彼女[かのじょ]は 臨時[りんじ]の 店員[てんいん]です。	かのじょ は りんじ の てんいん です	
\\	彼女[かのじょ]は
\\	の 店員[てんいん]です。			
\\	ミスプリント	ミスプリント	ミスプリント	
\\	資料にミスプリントがありました。	資料[しりょう]にミスプリントがありました。	しりょう に みすぷりんと が ありました	
\\	資料[しりょう]に
\\	がありました。			
\\	内閣	内閣[ないかく]	ないかく	
\\	新しい内閣が誕生したね。	新[あたら]しい 内閣[ないかく]が 誕生[たんじょう]したね。	あたらしい ないかく が たんじょう した ね	
\\	新[あたら]しい
\\	が 誕生[たんじょう]したね。			
\\	闘う	闘[たたか]う	たたかう	
\\	彼は病気と闘ったの。	彼[かれ]は 病気[びょうき]と 闘[たたか]ったの。	かれ は びょうき と たたかった の	
\\	彼[かれ]は 病気[びょうき]と
\\	の。			
\\	派閥	派閥[はばつ]	はばつ	
\\	あの大学には派閥がたくさんあります。	あの 大学[だいがく]には 派閥[はばつ]がたくさんあります。	あの だいがく に は はばつ が たくさん あります	
\\	あの 大学[だいがく]には
\\	がたくさんあります。			
\\	懐かしい	懐[なつ]かしい	なつかしい	
\\	ここは私にとって懐かしい場所です。	ここは 私[わたし]にとって 懐[なつ]かしい 場所[ばしょ]です。	ここ は わたし に とって なつかしい ばしょ です	
\\	ここは 私[わたし]にとって
\\	場所[ばしょ]です。			
\\	噴火	噴火[ふんか]	ふんか	
\\	島で火山が噴火したよ。	島[しま]で 火山[かざん]が 噴火[ふんか]したよ。	しま で かざん が ふんか した よ	
\\	島[しま]で 火山[かざん]が
\\	したよ。			
\\	噴水	噴水[ふんすい]	ふんすい	
\\	公園の噴水の前で会いましょう。	公園[こうえん]の 噴水[ふんすい]の 前[まえ]で 会[あ]いましょう。	こうえん の ふんすい の まえ で あいましょう	
\\	公園[こうえん]の
\\	の 前[まえ]で 会[あ]いましょう。			
\\	むく	むく	むく	
\\	人参の皮をむきました。	人参[にんじん]の 皮[かわ]をむきました。	にんじん の かわ を むきました	
\\	人参[にんじん]の 皮[かわ]を
\\	不愉快	不愉快[ふゆかい]	ふゆかい	
\\	彼の話を聞いて不愉快になった。	彼[かれ]の 話[はなし]を 聞[き]いて 不愉快[ふゆかい]になった。	かれ の はなし を きいて ふゆかい に なった	
\\	彼[かれ]の 話[はなし]を 聞[き]いて
\\	になった。			
\\	愉快	愉快[ゆかい]	ゆかい	
\\	彼らはとても愉快な人たちです。	彼[かれ]らはとても 愉快[ゆかい]な 人[ひと]たちです。	かれら は とても ゆかい な ひとたち です	
\\	彼[かれ]らはとても
\\	な 人[ひと]たちです。			
\\	漫画	漫画[まんが]	まんが	
\\	妹は漫画が好きです。	妹[いもうと]は 漫画[まんが]が 好[す]きです。	いもうと は まんが が すき です	
\\	妹[いもうと]は
\\	が 好[す]きです。			
\\	電卓	電卓[でんたく]	でんたく	
\\	この電卓はポケットに入ります。	この 電卓[でんたく]はポケットに 入[はい]ります。	この でんたく は ぽけっと に はいります	
\\	この
\\	はポケットに 入[はい]ります。			
\\	雄大	雄大[ゆうだい]	ゆうだい	
\\	雄大な景色に感動したの。	雄大[ゆうだい]な 景色[けしき]に 感動[かんどう]したの。	ゆうだい な けしき に かんどう した の	
\\	な 景色[けしき]に 感動[かんどう]したの。			
\\	めくる	めくる	めくる	
\\	彼はページをめくったの。	彼[かれ]はページをめくったの。	かれ は ぺーじ を めくった の	
\\	彼[かれ]はページを
\\	の。			
\\	雌	雌[めす]	めす	
\\	この猫は雌です。	この 猫[ねこ]は 雌[めす]です。	この ねこ は めす です	
\\	この 猫[ねこ]は
\\	です。			
\\	幼稚	幼稚[ようち]	ようち	
\\	その考えは少し幼稚だと思う。	その 考[かんが]えは 少[すこ]し 幼稚[ようち]だと 思[おも]う。	その かんがえ は すこし ようち だ と おもう	
\\	その 考[かんが]えは 少[すこ]し
\\	だと 思[おも]う。			
\\	名称	名称[めいしょう]	めいしょう	
\\	みんなで製品の名称を考えました。	みんなで 製品[せいひん]の 名称[めいしょう]を 考[かんが]えました。	みんな で せいひん の めいしょう を かんがえました	
\\	みんなで 製品[せいひん]の
\\	を 考[かんが]えました。			
\\	秩序	秩序[ちつじょ]	ちつじょ	
\\	その国の社会秩序は乱れているわ。	その 国[くに]の 社会[しゃかい] 秩序[ちつじょ]は 乱[みだ]れているわ。	その くに の しゃかい ちつじょ は みだれて いる わ	
\\	その 国[くに]の 社会[しゃかい]
\\	は 乱[みだ]れているわ。			
\\	物陰	物陰[ものかげ]	ものかげ	
\\	私たちは物陰に隠れたんだ。	私[わたし]たちは 物陰[ものかげ]に 隠[かく]れたんだ。	わたしたち は ものかげ に かくれた ん だ	
\\	私[わたし]たちは
\\	に 隠[かく]れたんだ。			
\\	めでたい	めでたい	めでたい	
\\	それはめでたいことだ。	それはめでたいことだ。	それはめでたいことだ。	
\\	それは
\\	ことだ。			
\\	魅力	魅力[みりょく]	みりょく	
\\	彼女の明るさに魅力を感じました。	彼女[かのじょ]の 明[あか]るさに 魅力[みりょく]を 感[かん]じました。	かのじょ の あかるさ に みりょく を かんじました	
\\	彼女[かのじょ]の 明[あか]るさに
\\	を 感[かん]じました。			
\\	醜い	醜[みにく]い	みにくい	
\\	醜い争いはやめましょう。	醜[みにく]い 争[あらそ]いはやめましょう。	みにくい あらそい は やめましょう	
\\	争[あらそ]いはやめましょう。			
\\	悪賢い	悪賢[わるがしこ]い	わるがしこい	
\\	彼は悪賢い男だな。	彼[かれ]は 悪賢[わるがしこ]い 男[おとこ]だな。	かれ は わるがしこい おとこ だ な	
\\	彼[かれ]は
\\	男[おとこ]だな。			
\\	頂上	頂上[ちょうじょう]	ちょうじょう	
\\	あと少しで山の頂上です。	あと 少[すこ]しで 山[やま]の 頂上[ちょうじょう]です。	あと すこし で やま の ちょうじょう です	
\\	あと 少[すこ]しで 山[やま]の
\\	です。			
\\	近頃	近頃[ちかごろ]	ちかごろ	
\\	近頃、彼女の様子がおかしい。	近頃[ちかごろ]、 彼女[かのじょ]の 様子[ようす]がおかしい。	ちかごろ かのじょ の ようす が おかしい	
\\	、 彼女[かのじょ]の 様子[ようす]がおかしい。			
\\	矛盾	矛盾[むじゅん]	むじゅん	
\\	彼の言っていることは矛盾しています。	彼[かれ]の 言[い]っていることは 矛盾[むじゅん]しています。	かれ の いって いる こと は むじゅん して います	
\\	彼[かれ]の 言[い]っていることは
\\	しています。			
\\	もうかる	もうかる	もうかる	
\\	彼の商売はもうかっているらしいね。	彼[かれ]の 商売[しょうばい]はもうかっているらしいね。	かれ の しょうばい は もうかって いる らしい ね	
\\	彼[かれ]の 商売[しょうばい]は
\\	らしいね。			
\\	倉庫	倉庫[そうこ]	そうこ	
\\	この荷物を倉庫に運びましょう。	この 荷物[にもつ]を 倉庫[そうこ]に 運[はこ]びましょう。	この にもつ を そうこ に はこびましょう	
\\	この 荷物[にもつ]を
\\	に 運[はこ]びましょう。			
\\	創立	創立[そうりつ]	そうりつ	
\\	この学校は1962年に創立されました。	この 学校[がっこう]は1962 年[ねん]に 創立[そうりつ]されました。	この がっこう は 
\\	ねん に そうりつ されました	
\\	この 学校[がっこう]は1962 年[ねん]に
\\	されました。			
\\	罰金	罰金[ばっきん]	ばっきん	
\\	罰金を3万円もとられたよ。	罰金[ばっきん]を3 万円[まんえん]もとられたよ。	ばっきん を 
\\	まんえん も とられた よ	
\\	を3 万円[まんえん]もとられたよ。			
\\	罰する	罰[ばっ]する	ばっする	
\\	彼は違法駐車で罰せられたんだ。	彼[かれ]は 違法駐車[いほう ちゅうしゃ]で 罰[ばっ]せられたんだ。	かれ は いほう ちゅうしゃ で ばっせられた ん だ	
\\	彼[かれ]は 違法駐車[いほう ちゅうしゃ]で
\\	んだ。			
\\	罰	罰[ばつ]	ばつ	
\\	彼は悪いことをしたので罰を受けたんだ。	彼[かれ]は 悪[わる]いことをしたので 罰[ばつ]を 受[う]けたんだ。	かれ は わるい こと を した ので ばつ を うけた ん だ	
\\	彼[かれ]は 悪[わる]いことをしたので
\\	を 受[う]けたんだ。			
\\	もうけ	もうけ	もうけ	
\\	今日のもうけは2万円でした。	今日[きょう]のもうけは2 万円[まんえん]でした。	きょう の もうけ は 
\\	まんえん でした	
\\	今日[きょう]の
\\	は2 万円[まんえん]でした。			
\\	老人	老人[ろうじん]	ろうじん	
\\	子供が老人の手を引いて歩いていたの。	子供[こども]が 老人[ろうじん]の 手[て]を 引[ひ]いて 歩[ある]いていたの。	こども が ろうじん の て を ひいて あるいて いた の	
\\	子供[こども]が
\\	の 手[て]を 引[ひ]いて 歩[ある]いていたの。			
\\	煮る	煮[に]る	にる	
\\	今、豆を煮ています。	今[いま]、 豆[まめ]を 煮[に]ています。	いま まめ を にて います	
\\	今[いま]、 豆[まめ]を
\\	煮える	煮[に]える	にえる	
\\	もうすぐじゃがいもが煮えます。	もうすぐじゃがいもが 煮[に]えます。	もうすぐ じゃがいも が にえます	
\\	もうすぐじゃがいもが
\\	蒸し暑い	蒸[む]し 暑[あつ]い	むしあつい	
\\	今日は蒸し暑いね。	今日[きょう]は 蒸[む]し 暑[あつ]いね。	きょう は むしあつい ね	
\\	今日[きょう]は
\\	ね。			
\\	黙る	黙[だま]る	だまる	
\\	彼はいつも黙って仕事をします。	彼[かれ]はいつも 黙[だま]って 仕事[しごと]をします。	かれ は いつも だまって しごと を します	
\\	彼[かれ]はいつも
\\	仕事[しごと]をします。			
\\	もしかしたら	もしかしたら	もしかしたら	
\\	もしかしたら来年転勤になるかも知れない。	もしかしたら 来年転勤[らいねん てんきん]になるかも 知[し]れない。	もしかしたら らいねん てんきん に なる かも しれない	
\\	来年転勤[らいねん てんきん]になるかも 知[し]れない。			
\\	無駄	無駄[むだ]	むだ	
\\	それは時間の無駄です。	それは 時間[じかん]の 無駄[むだ]です。	それ は じかん の むだ です	
\\	それは 時間[じかん]の
\\	です。			
\\	無駄遣い	無駄遣[むだづか]い	むだづかい	
\\	これは税金の無駄遣いだね。	これは 税金[ぜいきん]の 無駄遣[むだづか]いだね。	これ は ぜいきん の むだづかい だ ね	
\\	これは 税金[ぜいきん]の
\\	だね。			
\\	沸騰	沸騰[ふっとう]	ふっとう	
\\	水が沸騰したよ。	水[みず]が 沸騰[ふっとう]したよ。	みず が ふっとう した よ	
\\	水[みず]が
\\	したよ。			
\\	返却	返却[へんきゃく]	へんきゃく	
\\	図書館に本を返却したよ。	図書館[としょかん]に 本[ほん]を 返却[へんきゃく]したよ。	としょかん に ほん を へんきゃく した よ	
\\	図書館[としょかん]に 本[ほん]を
\\	したよ。			
\\	庁	庁[ちょう]	ちょう	
\\	彼は県庁で働いています。	彼[かれ]は 県[けん] 庁[ちょう]で 働[はたら]いています。	かれ は けんちょう で はたらいて います	
\\	彼[かれ]は 県[けん]
\\	で 働[はたら]いています。			
\\	もてる	もてる	もてる	
\\	彼は非常にもてるね。	彼[かれ]は 非常[ひじょう]にもてるね。	かれ は ひじょう に もてる ね	
\\	彼[かれ]は 非常[ひじょう]に
\\	ね。			
\\	都庁	都庁[とちょう]	とちょう	
\\	あの高い建物が都庁です。	あの 高[たか]い 建物[たてもの]が 都庁[とちょう]です。	あの たかい たてもの が とちょう です	
\\	あの 高[たか]い 建物[たてもの]が
\\	です。			
\\	府庁	府庁[ふちょう]	ふちょう	
\\	彼は府庁で働いています。	彼[かれ]は 府庁[ふちょう]で 働[はたら]いています。	かれ は ふちょう で はたらいて います	
\\	彼[かれ]は
\\	で 働[はたら]いています。			
\\	摩擦	摩擦[まさつ]	まさつ	
\\	その二つの国に摩擦が生じたんだ。	その 二[ふた]つの 国[くに]に 摩擦[まさつ]が 生[しょう]じたんだ。	その ふたつ の くに に まさつ が しょうじた ん だ	
\\	その 二[ふた]つの 国[くに]に
\\	が 生[しょう]じたんだ。			
\\	先輩	先輩[せんぱい]	せんぱい	
\\	日本人は先輩、後輩の関係をとても大切に考えます。	日本人[にっぽんじん]は 先輩[せんぱい]、 後輩[こうはい]の 関係[かんけい]をとても 大切[たいせつ]に 考[かんが]えます。	にっぽんじん は せんぱい こうはい の かんけい を とても たいせつ に かんがえます	
\\	日本人[にっぽんじん]は
\\	、 後輩[こうはい]の 関係[かんけい]をとても 大切[たいせつ]に 考[かんが]えます。			
\\	冒険	冒険[ぼうけん]	ぼうけん	
\\	昔の子供たちは冒険小説をよく読みました。	昔[むかし]の 子供[こども]たちは 冒険[ぼうけん] 小説[しょうせつ]をよく 読[よ]みました。	むかし の こどもたち は ぼうけん しょうせつ を よく よみました	
\\	昔[むかし]の 子供[こども]たちは
\\	小説[しょうせつ]をよく 読[よ]みました。			
\\	牧場	牧場[ぼくじょう]	ぼくじょう	
\\	叔父は牧場を持っているんだ。	叔父[おじ]は 牧場[ぼくじょう]を 持[も]っているんだ。	おじ は ぼくじょう を もって いる ん だ	
\\	叔父[おじ]は
\\	を 持[も]っているんだ。			
\\	ゆでる	ゆでる	ゆでる	
\\	今、野菜をゆでています。	今[いま]、 野菜[やさい]をゆでています。	いま やさい を ゆでて います	
\\	今[いま]、 野菜[やさい]を
\\	取り敢えず	取[と]り 敢[あ]えず	とりあえず	
\\	取り敢えずお知らせしておきます。	取[と]り 敢[あ]えずお 知[し]らせしておきます。	とりあえず おしらせ して おきます	
\\	お 知[し]らせしておきます。			
\\	勇気	勇気[ゆうき]	ゆうき	
\\	彼は勇気があるね。	彼[かれ]は 勇気[ゆうき]があるね。	かれ は ゆうき が ある ね	
\\	彼[かれ]は
\\	があるね。			
\\	勇敢	勇敢[ゆうかん]	ゆうかん	
\\	彼女はとても勇敢でした。	彼女[かのじょ]はとても 勇敢[ゆうかん]でした。	かのじょ は とても ゆうかん でした	
\\	彼女[かのじょ]はとても
\\	でした。			
\\	励ます	励[はげ]ます	はげます	
\\	父が息子を励ましたの。	父[ちち]が 息子[むすこ]を 励[はげ]ましたの。	ちち が むすこ を はげました の	
\\	父[ちち]が 息子[むすこ]を
\\	の。			
\\	(ように)	
\\	露	露[つゆ]	つゆ	
\\	花に露がついていました。	花[はな]に 露[つゆ]がついていました。	はな に つゆ が ついて いました	
\\	花[はな]に
\\	がついていました。			
\\	よける	よける	よける	
\\	彼は飛んできたボールをよけたの。	彼[かれ]は 飛[と]んできたボールをよけたの。	かれ は とんで きた ぼーる を よけた の	
\\	彼[かれ]は 飛[と]んできたボールを
\\	の。			
\\	零下	零下[れいか]	れいか	
\\	今日は零下の寒さでしたね。	今日[きょう]は 零下[れいか]の 寒[さむ]さでしたね。	きょう は れいか の さむさ でした ね	
\\	今日[きょう]は
\\	の 寒[さむ]さでしたね。			
\\	雰囲気	雰囲気[ふんいき]	ふんいき	
\\	とても雰囲気のいいお店ですね。	とても 雰囲気[ふんいき]のいいお 店[みせ]ですね。	とても ふんいき の いい おみせ です ね	
\\	とても
\\	のいいお 店[みせ]ですね。			
\\	盆地	盆地[ぼんち]	ぼんち	
\\	盆地は夏、とても暑いんだ。	盆地[ぼんち]は 夏[なつ]、とても 暑[あつ]いんだ。	ぼんち は なつ、とても あつい ん だ	
\\	は 夏[なつ]、とても 暑[あつ]いんだ。			
\\	盆	盆[ぼん]	ぼん	
\\	お盆は実家に帰りました。	お 盆[ぼん]は 実家[じっか]に 帰[かえ]りました。	お ぼん は じっか に かえりました	
\\	お
\\	は 実家[じっか]に 帰[かえ]りました。			
\\	盆	盆[ぼん]	ぼん	
\\	母が料理をお盆に乗せたの。	母[はは]が 料理[りょうり]をお 盆[ぼん]に 乗[の]せたの。	はは が りょうり を おぼん に のせた の	
\\	母[はは]が 料理[りょうり]をお
\\	に 乗[の]せたの。			
\\	よこす	よこす	よこす	
\\	父が長い手紙をよこしたよ。	父[ちち]が 長[なが]い 手紙[てがみ]をよこしたよ。	ちち が ながい てがみ を よこした よ	
\\	父[ちち]が 長[なが]い 手紙[てがみ]を
\\	よ。			
\\	大衆	大衆[たいしゅう]	たいしゅう	
\\	能は元々大衆の文化でした。	能[のう]は 元々[もともと] 大衆[たいしゅう]の 文化[ぶんか]でした。	のう は もともと たいしゅう の ぶんか でした	
\\	能[のう]は 元々[もともと]
\\	の 文化[ぶんか]でした。			
\\	舞台	舞台[ぶたい]	ぶたい	
\\	その物語は京都が舞台だ。	その 物語[ものがたり]は 京都[きょうと]が 舞台[ぶたい]だ。	その ものがたり は きょうと が ぶたい だ	
\\	その 物語[ものがたり]は 京都[きょうと]が
\\	だ。			
\\	見舞う	見舞[みま]う	みまう	
\\	昨日友人を見舞ったの。	昨日友人[きのう ゆうじん]を 見舞[みま]ったの。	きのう ゆうじん を みまった の	
\\	昨日友人[きのう ゆうじん]を
\\	の。			
\\	見舞い	見舞[みま]い	みまい	
\\	友達が見舞いに来てくれたよ。	友達[ともだち]が 見舞[みま]いに 来[き]てくれたよ。	ともだち が みまい に きて くれた よ	
\\	友達[ともだち]が
\\	に 来[き]てくれたよ。			
\\	盆踊り	盆踊[ぼんおど]り	ぼんおどり	
\\	みんなで盆踊りに行きました。	みんなで 盆踊[ぼんおど]りに 行[い]きました。	みんな で ぼんおどり に いきました	
\\	みんなで
\\	に 行[い]きました。			
\\	跳ねる	跳[は]ねる	はねる	
\\	ウサギが雪の上を跳ねていったよ。	ウサギが 雪[ゆき]の 上[うえ]を 跳[は]ねていったよ。	うさぎ が ゆき の うえ を はねて いった よ	
\\	ウサギが 雪[ゆき]の 上[うえ]を
\\	よ。			
\\	よす	よす	よす	
\\	人をからかうのはよしなさい。	人[ひと]をからかうのはよしなさい。	ひと を からかう の は よしなさい	
\\	人[ひと]をからかうのは
\\	銭湯	銭湯[せんとう]	せんとう	
\\	銭湯は昔より少なくなったわね。	銭湯[せんとう]は 昔[むかし]より 少[すく]なくなったわね。	せんとう は むかし より すくなく なった わ ね	
\\	は 昔[むかし]より 少[すく]なくなったわね。			
\\	古里	古里[ふるさと]	ふるさと	
\\	いつも古里を懐かしく思い出すの。	いつも 古里[ふるさと]を 懐[なつ]かしく 思[おも]い 出[だ]すの。	いつも ふるさと を なつかしく おもいだす の	
\\	いつも
\\	を 懐[なつ]かしく 思[おも]い 出[だ]すの。			
\\	狙う	狙[ねら]う	ねらう	
\\	来年は優勝を狙います。	来年[らいねん]は 優勝[ゆうしょう]を 狙[ねら]います。	らいねん は ゆうしょう を ねらいます	
\\	来年[らいねん]は 優勝[ゆうしょう]を
\\	彫る	彫[ほ]る	ほる	
\\	彼は木の像を彫ったの。	彼[かれ]は 木[き]の 像[ぞう]を 彫[ほ]ったの。	かれ は き の ぞう を ほった の	
\\	彼[かれ]は 木[き]の 像[ぞう]を
\\	の。			
\\	中央	中央[ちゅうおう]	ちゅうおう	
\\	その公園は町のほぼ中央に位置するんだ。	その 公園[こうえん]は 町[まち]のほぼ 中央[ちゅうおう]に 位置[いち]するんだ。	その こうえん は まち の ほぼ ちゅうおう に いち する ん だ	
\\	その 公園[こうえん]は 町[まち]のほぼ
\\	に 位置[いち]するんだ。			
\\	レクリエーション	レクリエーション	レクリエーション	
\\	町内会のレクリエーションに参加したの。	町内会[ちょうないかい]のレクリエーションに 参加[さんか]したの。	ちょうないかい の れくりえーしょん に さんか した の	
\\	町内会[ちょうないかい]の
\\	に 参加[さんか]したの。			
\\	双子	双子[ふたご]	ふたご	
\\	友達に双子が生まれたよ。	友達[ともだち]に 双子[ふたご]が 生[う]まれたよ。	ともだち に ふたご が うまれた よ	
\\	友達[ともだち]に
\\	が 生[う]まれたよ。			
\\	裸	裸[はだか]	はだか	
\\	彼らは裸のつきあいをしているよ。	彼[かれ]らは 裸[はだか]のつきあいをしているよ。	かれら は はだか の つきあい を して いる よ	
\\	彼[かれ]らは
\\	のつきあいをしているよ。			
\\	祖先	祖先[そせん]	そせん	
\\	犬も猫も祖先は同じ動物らしいよ。	犬[いぬ]も 猫[ねこ]も 祖先[そせん]は 同[おな]じ 動物[どうぶつ]らしいよ。	いぬ も ねこ も そせん は おなじ どうぶつ らしい よ	
\\	犬[いぬ]も 猫[ねこ]も
\\	は 同[おな]じ 動物[どうぶつ]らしいよ。			
\\	報酬	報酬[ほうしゅう]	ほうしゅう	
\\	その仕事の報酬として20万円もらいました。	その 仕事[しごと]の 報酬[ほうしゅう]として20 万円[まんえん]もらいました。	その しごと の ほうしゅう と して 
\\	まんえん もらいました	
\\	その 仕事[しごと]の
\\	として20 万円[まんえん]もらいました。			
\\	無邪気	無邪気[むじゃき]	むじゃき	
\\	子供たちが無邪気に遊んでいるね。	子供[こども]たちが 無邪気[むじゃき]に 遊[あそ]んでいるね。	こどもたち が むじゃき に あそんで いる ね	
\\	子供[こども]たちが
\\	に 遊[あそ]んでいるね。			
\\	ろうそく	ろうそく	ろうそく	
\\	バースデーケーキにろうそくを立てました。	バースデーケーキにろうそくを 立[た]てました。	ばーすでーけーき に ろうそく を たてました	
\\	バースデーケーキに
\\	を 立[た]てました。			
\\	玉ねぎ	玉[たま]ねぎ	たまねぎ	
\\	私は玉ねぎが嫌いです。	私[わたし]は 玉[たま]ねぎが 嫌[きら]いです。	わたし は たまねぎ が きらい です 。	
\\	私[わたし]は
\\	が 嫌[きら]いです。			
\\	玉	玉[たま]	たま	
\\	彼の顔に玉のような汗が流れていたの。	彼[かれ]の 顔[かお]に 玉[たま]のような 汗[あせ]が 流[なが]れていたの。	かれ の かお に たま の よう な あせ が ながれて いた の	
\\	彼[かれ]の 顔[かお]に
\\	のような 汗[あせ]が 流[なが]れていたの。			
\\	天皇	天皇[てんのう]	てんのう	
\\	日本には天皇がいます。	日本[にっぽん]には 天皇[てんのう]がいます。	にっぽん に は てんのう が います	
\\	日本[にっぽん]には
\\	がいます。			
\\	風呂場	風呂場[ふろば]	ふろば	
\\	私は風呂場で歯を磨きます。	私[わたし]は 風呂場[ふろば]で 歯[は]を 磨[みが]きます。	わたし は ふろば で は を みがきます	
\\	私[わたし]は
\\	で 歯[は]を 磨[みが]きます。			
\\	班	班[はん]	はん	
\\	班のメンバーは5人です。	班[はん]のメンバーは5 人[にん]です。	はん の めんばー は 
\\	にん です	
\\	のメンバーは5 人[にん]です。			
\\	召し上がる	召[め]し 上[あ]がる	めしあがる	
\\	どうぞ召し上がってください。	どうぞ 召[め]し 上[あ]がってください。	どうぞ めしあがって ください	
\\	どうぞ
\\	ください。			
\\	わがまま	わがまま	わがまま	
\\	子供のわがままを叱ったの。	子供[こども]のわがままを 叱[しか]ったの。	こども の わがまま を しかった の	
\\	子供[こども]の
\\	を 叱[しか]ったの。			
\\	哲学	哲学[てつがく]	てつがく	
\\	哲学は興味深い学問です。	哲学[てつがく]は 興味深[きょうみぶか]い 学問[がくもん]です。	てつがく は きょうみぶかい がくもん です	
\\	は 興味深[きょうみぶか]い 学問[がくもん]です。			
\\	綱	綱[つな]	つな	
\\	この綱は直径20
\\	あるそうです。	この 綱[つな]は 直径20
\\	[ちょっけい 
\\	せんち]あるそうです。	この つな は ちょっけい 
\\	せんち ある そう です	
\\	この
\\	は 直径20
\\	[ちょっけい 
\\	せんち]あるそうです。			
\\	縄	縄[なわ]	なわ	
\\	枝を縄でしばったよ。	枝[えだ]を 縄[なわ]でしばったよ。	えだ を なわ で しばった よ	
\\	枝[えだ]を
\\	でしばったよ。			
\\	縫う	縫[ぬ]う	ぬう	
\\	彼女は子供のスカートを縫ったんだ。	彼女[かのじょ]は 子供[こども]のスカートを 縫[ぬ]ったんだ。	かのじょ は こども の すかーと を ぬった ん だ	
\\	彼女[かのじょ]は 子供[こども]のスカートを
\\	んだ。			
\\	虹	虹[にじ]	にじ	
\\	雨が上がって美しい虹が出たね。	雨[あめ]が 上[あ]がって 美[うつく]しい 虹[にじ]が 出[で]たね。	あめ が あがって うつくしい にじ が でた ね 。	
\\	雨[あめ]が 上[あ]がって 美[うつく]しい
\\	が 出[で]たね。			
\\	わく	わく	わく	
\\	友達に励まされて勇気がわいたよ。	友達[ともだち]に 励[はげ]まされて 勇気[ゆうき]がわいたよ。	ともだち に はげまされて ゆうき が わいた よ	
\\	友達[ともだち]に 励[はげ]まされて 勇気[ゆうき]が
\\	よ。			
\\	梅雨明け	梅雨明[つゆあ]け	つゆあけ	
\\	梅雨明けは来週だそうです。	梅雨明[つゆあ]けは 来週[らいしゅう]だそうです。	つゆあけ は らいしゅう だ そう です	
\\	は 来週[らいしゅう]だそうです。			
\\	梅雨入り	梅雨入[つゆい]り	つゆいり	
\\	梅雨入りは6月17日でした。	梅雨入[つゆい]りは6 月17日[がつ 
\\	にち]でした。	つゆいり は 
\\	がつ 
\\	にち でした	
\\	は6 月17日[がつ 
\\	にち]でした。			
\\	欄	欄[らん]	らん	
\\	お名前は上の欄にお書きください。	お 名前[なまえ]は 上[うえ]の 欄[らん]にお 書[か]きください。	お なまえ は うえ の らん に お かき ください	
\\	お 名前[なまえ]は 上[うえ]の
\\	にお 書[か]きください。			
\\	栓抜き	栓抜[せんぬ]き	せんぬき	
\\	栓抜きはどこですか。	栓抜[せんぬ]きはどこですか。	せんぬき は どこ です か	
\\	はどこですか。			
\\	峠	峠[とうげ]	とうげ	
\\	この峠を越えると村があります。	この 峠[とうげ]を 越[こ]えると 村[むら]があります。	この とうげ を こえる と むら が あります	
\\	この
\\	を 越[こ]えると 村[むら]があります。			
\\	ばからしい	ばからしい	ばからしい	
\\	小さなことにくよくよするのはばからしいよ。	小[ちい]さなことにくよくよするのはばからしいよ。	ちいさ な こと に くよくよ する の は ばからしい よ	
\\	小[ちい]さなことにくよくよするのは
\\	よ。			
\\	婆さん	婆[ばあ]さん	ばあさん	
\\	彼女は元気なお婆さんね。	彼女[かのじょ]は 元気[げんき]なお 婆[ばあ]さんね。	かのじょ は げんき な おばあさん ね	
\\	彼女[かのじょ]は 元気[げんき]なお
\\	ね。			
\\	漬ける	漬[つ]ける	つける	
\\	魚を味噌に漬けました。	魚[さかな]を 味噌[みそ]に 漬[つ]けました。	さかな を みそ に つけました	
\\	魚[さかな]を 味噌[みそ]に
\\	漬け物	漬[つ]け 物[もの]	つけもの	
\\	祖母は漬け物を自分で作ります。	祖母[そぼ]は 漬[つ]け 物[もの]を 自分[じぶん]で 作[つく]ります。	そぼ は つけもの を じぶん で つくります	
\\	祖母[そぼ]は
\\	を 自分[じぶん]で 作[つく]ります。			
\\	潰す	潰[つぶ]す	つぶす	
\\	ペットボトルは潰して捨てましょう。	ペットボトルは 潰[つぶ]して 捨[す]てましょう。	ぺっとぼとる は つぶして すてましょう	
\\	ペットボトルは
\\	捨[す]てましょう。			
\\	潰れる	潰[つぶ]れる	つぶれる	
\\	箱の角が潰れていますよ。	箱[はこ]の 角[かど]が 潰[つぶ]れていますよ。	はこ の かど が つぶれて います よ	
\\	箱[はこ]の 角[かど]が
\\	いますよ。			
\\	なぜか	なぜか	なぜか	
\\	今日はなぜか体がだるい。	今日[きょう]はなぜか 体[からだ]がだるい。	きょう は なぜか からだ が だるい	
\\	今日[きょう]は
\\	体[からだ]がだるい。			
\\	年賀	年賀[ねんが]	ねんが	
\\	年賀葉書が売り出されたね。	年賀[ねんが] 葉書[はがき]が 売[う]り 出[だ]されたね。	ねんがはがき が うりだされた ね	
\\	葉書[はがき]が 売[う]り 出[だ]されたね。			
\\	賑わう	賑[にぎ]わう	にぎわう	
\\	ここは正月はたくさんの人で賑わいます。	ここは 正月[しょうがつ]はたくさんの 人[ひと]で 賑[にぎ]わいます。	ここ は しょうがつ は たくさん の ひと で にぎわいます	
\\	ここは 正月[しょうがつ]はたくさんの 人[ひと]で
\\	蛇	蛇[へび]	へび	
\\	蛇がカエルを捕まえたの。	蛇[へび]がカエルを 捕[つか]まえたの。	へび が かえる を つかまえた の	
\\	がカエルを 捕[つか]まえたの。			
\\	華やか	華[はな]やか	はなやか	
\\	彼女は華やかな女性ですね。	彼女[かのじょ]は 華[はな]やかな 女性[じょせい]ですね。	かのじょ は はなやか な じょせい です ね	
\\	彼女[かのじょ]は
\\	な 女性[じょせい]ですね。			
\\	挟む	挟[はさ]む	はさむ	
\\	ドアに指を挟んだ。	ドアに 指[ゆび]を 挟[はさ]んだ。	どあ に ゆび を はさんだ	
\\	ドアに 指[ゆび]を
\\	塔	塔[とう]	とう	
\\	あの塔の高さはどれくらいですか。	あの 塔[とう]の 高[たか]さはどれくらいですか。	あの とう の たかさ は どれ くらい です か	
\\	あの
\\	の 高[たか]さはどれくらいですか。			
\\	にわかに	にわかに	にわかに	
\\	空がにわかに暗くなったな。	空[そら]がにわかに 暗[くら]くなったな。	そら が にわかに くらく なった な	
\\	空[そら]が
\\	暗[くら]くなったな。			
\\	誓う	誓[ちか]う	ちかう	
\\	二人は一生を共にすることを誓ったの。	二人[ふたり]は 一生[いっしょう]を 共[とも]にすることを 誓[ちか]ったの。	ふたり は いっしょう を ともに する こと を ちかった の	
\\	二人[ふたり]は 一生[いっしょう]を 共[とも]にすることを
\\	の。			
\\	慰める	慰[なぐさ]める	なぐさめる	
\\	友人が慰めてくれました。	友人[ゆうじん]が 慰[なぐさ]めてくれました。	ゆうじん が なぐさめて くれました	
\\	友人[ゆうじん]が
\\	朗らか	朗[ほが]らか	ほがらか	
\\	彼はとても朗らかな人です。	彼[かれ]はとても 朗[ほが]らかな 人[ひと]です。	かれ は とても ほがらか な ひと です	
\\	彼[かれ]はとても
\\	な 人[ひと]です。			
\\	蝶蝶	蝶蝶[ちょうちょう]	ちょうちょう	
\\	見て、きれいな蝶蝶が飛んでいる。	見[み]て、きれいな 蝶蝶[ちょうちょう]が 飛[と]んでいる。	みて 、 きれい な ちょうちょう が とん でいる 。	
\\	見[み]て、きれいな
\\	が 飛[と]んでいる。			
\\	鳩	鳩[はと]	はと	
\\	公園の鳩にえさをやったの。	公園[こうえん]の 鳩[はと]にえさをやったの。	こうえん の はと に えさ を やった の	
\\	公園[こうえん]の
\\	にえさをやったの。			
\\	ぴたりと	ぴたりと	ぴたりと	
\\	風がぴたりと止んだね。	風[かぜ]がぴたりと 止[や]んだね。	かぜ が ぴたりと やんだ ね	
\\	風[かぜ]が
\\	止[や]んだね。			
\\	雛祭	雛祭[ひなまつり]	ひなまつり	
\\	3月3日は雛祭りです。	
\\	月[がつ]3 日[か]は 雛祭[ひなまつ]りです。	
\\	がつ 
\\	か は ひなまつり です	
\\	月[がつ]3 日[か]は
\\	です。			
\\	物凄い	物凄[ものすご]い	ものすごい	
\\	夕方、物凄い雨が降ったね。	夕方[ゆうがた]、 物凄[ものすご]い 雨[あめ]が 降[ふ]ったね。	ゆうがた、 ものすごい あめ が ふった ね	
\\	夕方[ゆうがた]、
\\	雨[あめ]が 降[ふ]ったね。			
\\	喉	喉[のど]	のど	
\\	風邪をひいて喉が痛いです。	風邪[かぜ]をひいて 喉[のど]が 痛[いた]いです。	かぜ を ひいて のど が いたい です	
\\	風邪[かぜ]をひいて
\\	が 痛[いた]いです。			
\\	唾	唾[つば]	つば	
\\	彼は地面に唾を吐いたよ。	彼[かれ]は 地面[じめん]に 唾[つば]を 吐[は]いたよ。	かれ は じめん に つば を はいた よ	
\\	彼[かれ]は 地面[じめん]に
\\	を 吐[は]いたよ。			
\\	吠える	吠[ほ]える	ほえる	
\\	どこかで犬が吠えていますね。	どこかで 犬[いぬ]が 吠[ほ]えていますね。	どこか で いぬ が ほえて います ね	
\\	どこかで 犬[いぬ]が
\\	ね。			
\\	ぴょんと	ぴょんと	ぴょんと	
\\	子供が水たまりをぴょんと飛びこえたね。	子供[こども]が 水[みず]たまりをぴょんと 飛[と]びこえたね。	こども が みずたまり を ぴょんと とびこえた ね	
\\	子供[こども]が 水[みず]たまりを
\\	飛[と]びこえたね。			
\\	吊るす	吊[つ]るす	つるす	
\\	木にブランコを吊るしました。	木[き]にブランコを 吊[つ]るしました。	き に ぶらんこ を つるしました	
\\	木[き]にブランコを
\\	味噌	味噌[みそ]	みそ	
\\	きゅうりに味噌をつけて食べたんだ。	きゅうりに 味噌[みそ]をつけて 食[た]べたんだ。	きゅうり に みそ を つけて たべた ん だ	
\\	きゅうりに
\\	をつけて 食[た]べたんだ。			
\\	塞がる	塞[ふさ]がる	ふさがる	
\\	傷口はもう塞がりました。	傷口[きずぐち]はもう 塞[ふさ]がりました。	きずぐち は もう ふさがりました	
\\	傷口[きずぐち]はもう
\\	塞ぐ	塞[ふさ]ぐ	ふさぐ	
\\	あまりにうるさかったので耳を塞いだわ。	あまりにうるさかったので 耳[みみ]を 塞[ふさ]いだわ。	あまりに うるさかった の で みみ を ふさいだ わ	
\\	あまりにうるさかったので 耳[みみ]を
\\	わ。			
\\	撫でる	撫[な]でる	なでる	
\\	猫の頭を撫でてあげた。	猫[ねこ]の 頭[あたま]を 撫[な]でてあげた。	ねこ の あたま を なでて あげた	
\\	猫[ねこ]の 頭[あたま]を
\\	卑怯	卑怯[ひきょう]	ひきょう	
\\	彼らは卑怯な方法で勝利したの。	彼[かれ]らは 卑怯[ひきょう]な 方法[ほうほう]で 勝利[しょうり]したの。	かれら は ひきょう な ほうほう で しょうり した の	
\\	彼[かれ]らは
\\	な 方法[ほうほう]で 勝利[しょうり]したの。			
\\	ひらりと	ひらりと	ひらりと	
\\	桜の花びらがひらりと舞い落ちたな。	桜[さくら]の 花[はな]びらがひらりと 舞[ま]い 落[お]ちたな。	さくら の はなびら が ひらりと まいおちた な	
\\	桜[さくら]の 花[はな]びらが
\\	舞[ま]い 落[お]ちたな。			
\\	溜め息	溜[た]め 息[いき]	ためいき	
\\	母は溜め息をついたの。	母[はは]は 溜[た]め 息[いき]をついたの。	はは は ためいき を ついた の	
\\	母[はは]は
\\	をついたの。			
\\	溜まる	溜[た]まる	たまる	
\\	彼はストレスが溜まっているの。	彼[かれ]はストレスが 溜[た]まっているの。	かれ は すとれす が たまって いる の	
\\	彼[かれ]はストレスが
\\	の。			
\\	溜める	溜[た]める	ためる	
\\	お風呂に水を溜めておいてください。	お 風呂[ふろ]に 水[みず]を 溜[た]めておいてください。	おふろ に みず を ためて おいて ください	
\\	お 風呂[ふろ]に 水[みず]を
\\	ください。			
\\	海苔	海苔[のり]	のり	
\\	旅館の朝食にのりが出ました。	旅館[りょかん]の 朝食[ちょうしょく]にのりが 出[で]ました。	りょかん の ちょうしょく に のり が でました 。	
\\	旅館[りょかん]の 朝食[ちょうしょく]に
\\	が 出[で]ました。			
\\	遥か	遥[はる]か	はるか	
\\	遥か向こうに目的地が見えてきた。	遥[はる]か 向[む]こうに 目的地[もくてきち]が 見[み]えてきた。	はるか むこう に もくてきち が みえて きた	
\\	向[む]こうに 目的地[もくてきち]が 見[み]えてきた。			
\\	もうじき	もうじき	もうじき	
\\	彼はもうじき父親になります。	彼[かれ]はもうじき 父親[ちちおや]になります。	かれ は もうじき ちちおや に なります	
\\	彼[かれ]は
\\	父親[ちちおや]になります。			
\\	股	股[また]	また	
\\	自転車の乗り過ぎで股が痛いです。	自転車[じてんしゃ]の 乗[の]り 過[す]ぎで 股[また]が 痛[いた]いです。	じてんしゃ の のりすぎ で また が いたい です	
\\	自転車[じてんしゃ]の 乗[の]り 過[す]ぎで
\\	が 痛[いた]いです。			
\\	焚く	焚[た]く	たく	
\\	薪を集めて火を焚いたよ。	薪[まき]を 集[あつ]めて 火[ひ]を 焚[た]いたよ。	まき を あつめて ひ を たいた よ	
\\	薪[まき]を 集[あつ]めて 火[ひ]を
\\	よ。			
\\	明瞭	明瞭[めいりょう]	めいりょう	
\\	彼は話し方が明瞭ね。	彼[かれ]は 話[はな]し 方[かた]が 明瞭[めいりょう]ね。	かれ は はなしかた が めいりょう ね	
\\	彼[かれ]は 話[はな]し 方[かた]が
\\	ね。			
\\	眉	眉[まゆ]	まゆ	
\\	彼は眉が濃いね。	彼[かれ]は 眉[まゆ]が 濃[こ]いね。	かれ は まゆ が こい ね	
\\	彼[かれ]は
\\	が 濃[こ]いね。			
\\	眉毛	眉毛[まゆげ]	まゆげ	
\\	彼女の眉毛は太いな。	彼女[かのじょ]の 眉毛[まゆげ]は 太[ふと]いな。	かのじょ の まゆげ は ふとい な	
\\	彼女[かのじょ]の
\\	は 太[ふと]いな。			
\\	やたらに	やたらに	やたらに	
\\	彼はやたらに話しかけてきた。	彼[かれ]はやたらに 話[はな]しかけてきた。	かれ は やたらに はなしかけて きた	
\\	彼[かれ]は
\\	話[はな]しかけてきた。			
\\	割り箸	割[わ]り 箸[ばし]	わりばし	
\\	彼女は割り箸を使わず、自分の箸を使うの。	彼女[かのじょ]は 割[わ]り 箸[ばし]を 使[つか]わず、 自分[じぶん]の 箸[はし]を 使[つか]うの。	かのじょ は わりばし を つかわず じぶん の はし を つかう の	
\\	彼女[かのじょ]は
\\	を 使[つか]わず、 自分[じぶん]の 箸[はし]を 使[つか]うの。			
\\	蛋白質	蛋白質[たんぱくしつ]	たんぱくしつ	
\\	蛋白質は大切な栄養です。	蛋白質[たんぱくしつ]は 大切[たいせつ]な 栄養[えいよう]です。	たんぱくしつ は たいせつ な えいよう です	
\\	は 大切[たいせつ]な 栄養[えいよう]です。			
\\	蜂	蜂[はち]	はち	
\\	友達が蜂に刺されたんだ。	友達[ともだち]が 蜂[はち]に 刺[さ]されたんだ。	ともだち が はち に さされた ん だ	
\\	友達[ともだち]が
\\	に 刺[さ]されたんだ。			
\\	頂戴	頂戴[ちょうだい]	ちょうだい	
\\	それ、ひとつ頂戴。	それ、ひとつ 頂戴[ちょうだい]。	それ ひとつ ちょうだい	
\\	それ、ひとつ
\\	初詣で	初詣[はつもう]で	はつもうで	
\\	近くの神社に初詣でに行きました。	近[ちか]くの 神社[じんじゃ]に 初詣[はつもう]でに 行[い]きました。	ちかく の じんじゃ に はつもうで に いきました	
\\	近[ちか]くの 神社[じんじゃ]に
\\	に 行[い]きました。			
\\	詫びる	詫[わ]びる	わびる	
\\	彼は彼女に心から詫びたんだ。	彼[かれ]は 彼女[かのじょ]に 心[こころ]から 詫[わ]びたんだ。	かれ は かのじょ に こころ から わびた ん だ	
\\	彼[かれ]は 彼女[かのじょ]に 心[こころ]から
\\	んだ。			
\\	頬	頬[ほお]	ほお	
\\	彼女はほおを赤く染めたわ。	彼女[かのじょ]はほおを 赤[あか]く 染[そ]めたわ。	かのじょ は ほお を あかく そめた わ	
\\	彼女[かのじょ]は
\\	を 赤[あか]く 染[そ]めたわ。			
\\	餅	餅[もち]	もち	
\\	正月には餅を食べますよ。	正月[しょうがつ]には 餅[もち]を 食[た]べますよ。	しょうがつ に は もち を たべます よ	
\\	正月[しょうがつ]には
\\	を 食[た]べますよ。			
\\	騙す	騙[だま]す	だます	
\\	人を騙してはいけません。	人[ひと]を 騙[だま]してはいけません。	ひと を だまして は いけません	
\\	人[ひと]を
\\	はいけません。			
\\	便箋	便箋[びんせん]	びんせん	
\\	友達に手紙を書こうと便箋を買ったんだ。	友達[ともだち]に 手紙[てがみ]を 書[か]こうと 便箋[びんせん]を 買[か]ったんだ。	ともだち に てがみ を かこう と びんせん を かった ん だ	
\\	友達[ともだち]に 手紙[てがみ]を 書[か]こうと
\\	を 買[か]ったんだ。			
\\	請求	請求[せいきゅう]	せいきゅう	
\\	英語で話したいなら一時間当たり五千円料金を請求します	英語[えいご]で 話[はな]したいなら一 時間[じかん] 当[あ]たり五 千[せん] 円[えん] 料金[りょうきん]を 請求[せいきゅう]します	えいごではなしたかったらいちじかんあたりごせんえんりょうきんをせいきゅうします。	
\\	対処	対処「たいしょ」	たいしょ	
\\	この問題に対処しています	この 問題[もんだい]に 対処[たいしょ]しています	そのもんだいにたしょしています	
\\	妥協	妥協[だきょう]	だきょう	
\\	私は妥協することはしない	わたしはだきょうすることはしない	わたしはだきょうすることはしない	
\\	扱い(される)	扱い「あつかい」される	あつかいされる	
\\	彼は子供扱いされた	彼[かれ]は 子供[こども] 扱[あつか]いされた	かれはこどもあつかいされた	
\\	ふりをする	ふりをする	ふりをする	
\\	英語が分からないふりをするな	英語[えいご]が 分[わ]からないふりをするな	えいごがわからないふりをするな	
\\	庇う	庇う「かばう」	かばう	
\\	何があっても俺は君を庇うんだ	何[なに]があっても 俺[おれ]は 君[きみ]を 庇[かば]うんだ	なにがあってもおれはきみをくばうんだ	
\\	抱え込む	抱「かか」え込「こ」む	かかえこむ	
\\	その罪を独りで抱え込まないでください	その 罪[つみ]を一 人[にん]で 抱え込[かかえこ]まないでください	そのつみをひとりでかかえこまないでください	
\\	意地を張る	意地[いじ]を[はる]	いじをはる	
\\	意地を張らないで	意地[いじ]を 張[は]らないで	いじをはらないで	
\\	取り除く	取り除く[とりのぞく]	とりのぞく	
\\	この気持ちを取り除きたいんです。	この 気持[きも]ちを 取り除[とりのぞ]きたいんです。	このきもちをとりのぞきたいんです。	
\\	欠陥	欠陥「けっかん」	けっかん	
\\	欠陥のある部品は弊社までお送りください。	欠陥[けっかん]のある 部品[ぶひん]は 弊社[へいしゃ]までお 送[おく]りください。	けっかんのあるぶひんはへいしゃまでおくりください	
\\	選択肢	選択肢「せんたくし」	せんたくし	
\\	逃げるのは選択肢ではない	逃[に]げるのは 選択肢[せんたくし]ではない	にげるのはせんたくしではない	
\\	設計する	設計する[せっけいする]	せっけいする	
\\	(される)		
\\	あれは複製するように設計されたウイルス	あれは複製[ふくせい]するように 設計[せっけい]されたウイルス	あれはふくせいするようにせっけいされたウイルス	
\\	ためらう	ためらう	ためらう	
\\	彼は彼女に直接連絡するのをためらった	彼[かれ]は 彼女[かのじょ]に 直接[ちょくせつ] 連絡[れんらく]するのをためらった	彼は彼女にちょくせつれんらくするのをためらった	
\\	見なす	見なす[みなす]	みなす	
\\	彼女が彼のやさしさを弱みと見なした。	彼女が彼のやさしさを弱みと見なした。	かのじょがかれのやさしさをよわみとみなした	
\\	~を何かと見なす			
\\	共有	共[きょう]有[ゆう]	きょうゆう	
\\	会議で情報を共有しました	会議で情報を共有しました	かいぎでじょうほうをきょうゆうしました	
\\	共有			
\\	上級	上級[じょうきゅう]	上級	
\\	彼女はダンスの上級クラスを受けている	彼女[かのじょ]は ダンスの上級[じょうきゅう]クラスを 受[う]けている	彼女はダンスのじょうきゅうくらすをうけている。	
\\	こだわる	こだわる	こだわる	
\\	自分を熊本にこだわるな。	自分を熊本にこだわるな	じぶんをくまもとにこだわるな	
\\	探検	探検[たんけん]	たんけん	
\\	広島市内を探検しましょうか?	広島市内を探検しましょうか?	ひろしましないをたんけんしましょうか	
\end{CJK}
\end{document}