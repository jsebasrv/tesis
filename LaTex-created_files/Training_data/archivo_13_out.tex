\documentclass[8pt]{extreport} 
\usepackage{hyperref}
\usepackage{CJKutf8}
\begin{document}
\begin{CJK}{UTF8}{min}
\\	一			
\\	イチ、イツ	ひと-、ひと.つ	一々(いちいち): 
\\	一概に(いちがいに): 
\\	一言(いちげん): 
\\	一見(いちげん): 
\\	一日(いちじつ): 
\\	一定(いちじょう): 
\\	一同(いちどう): 
\\	一人(いちにん): 
\\	一部(いちぶ): 
\\	一部分(いちぶぶん): 
\\	一別(いちべつ): 
\\	一面(いちめん): 
\\	一目(いちもく): 
\\	一様(いちよう): 
\\	一律(いちりつ): 
\\	一連(いちれん): 
\\	一括(いっかつ): 
\\	一気(いっき): 
\\	一挙に(いっきょに): 
\\	一切(いっさい): 
\\	一心(いっしん): 
\\	一帯(いったい): 
\\	一敗(いっぱい): 
\\	一変(いっぺん): 
\\	一昨日(おととい): 
\\	一昨年(おととし): 
\\	一昨昨日(さきおととい): 
\\	第一(だいいち): 
\\	単一(たんいつ): 
\\	一寸(ちょっと): 
\\	一向(ひたすら): 
\\	一息(ひといき): 
\\	一頃(ひところ): 
\\	一筋(ひとすき): 
\\	一まず(ひとまず): 
\\	一人でに(ひとりでに): 
\\	一応(いちおう): 
\\	一時(いちじ): 
\\	一段と(いちだんと): 
\\	一度に(いちどに): 
\\	一流(いちりゅう): 
\\	一家(いっか): 
\\	一昨日(いっさくじつ): 
\\	一昨年(いっさくねん): 
\\	一種(いっしゅ): 
\\	一瞬(いっしゅん): 
\\	一生(いっしょう): 
\\	一斉(いっせい): 
\\	一層(いっそう): 
\\	一体(いったい): 
\\	一旦(いったん): 
\\	一致(いっち): 
\\	一定(いってい): 
\\	一般(いっぱん): 
\\	一方(いっぽう): 
\\	統一(とういつ): 
\\	同一(どういつ): 
\\	一言(ひとこと): 
\\	一通り(ひととおり): 
\\	一休み(ひとやすみ): 
\\	一人一人(ひとりひとり): 
\\	万一(まんいち): 
\\	唯一(ゆいいつ): 
\\	一度(いちど): 
\\	一生懸命(いっしょうけんめい): 
\\	一(いち): 
\\	一日(いちにち): 
\\	一番(いちばん): 
\\	一緒(いっしょ): 
\\	一日(ついたち): 
\\	一つ(ひとつ): 
\\	一月(ひとつき): 
\\	一人(ひとり): 
\\	一度 (いちど), 一座 (いちざ), 第一 (だいいち), 一 (ひと), 一つ (ひと.つ)
\\	二			
\\	ニ、ジ	ふた、ふた.つ、ふたた.び	二人(ににん): 
\\	真っ二つ(まっぷたつ): 
\\	二(に): 
\\	二十歳(はたち): 
\\	二十日(はつか): 
\\	二つ(ふたつ): 
\\	二人(ふたり): 
\\	二日(ふつか): 
\\	二 (ふた), 二つ (ふた.つ)
\\	三			
\\	(一 
\\	二 
\\	三) 
\\	サン、ゾウ	み、み.つ、みっ.つ	三味線(さみせん): 
\\	三(み): 
\\	三日月(みかずき): 
\\	再三(さいさん): 
\\	三角(さんかく): 
\\	三日月(みかづき): 
\\	三日(みっか): 
\\	三(さん): 
\\	三つ(みっつ): 
\\	三 (み), 三つ (み.つ), 三つ (みっ.つ)
\\	四			
\\	シ	よ、よ.つ、よっ.つ、よん	四角(しかく): 
\\	四角い(しかくい): 
\\	四季(しき): 
\\	四捨五入(ししゃごにゅう): 
\\	四つ角(よつかど): 
\\	四(よん): 
\\	四(し): 
\\	四日(よっか): 
\\	四つ(よっつ): 
\\	四 (よ), 四つ (よ.つ), 四つ (よっ.つ), 四 (よん)
\\	五			
\\	三 
\\	ゴ	いつ、いつ.つ	五月蝿い(うるさい): 
\\	五十音(ごじゅうおん): 
\\	四捨五入(ししゃごにゅう): 
\\	五日(いつか): 
\\	五つ(いつつ): 
\\	五(ご): 
\\	五 (いつ), 五つ (いつ.つ)
\\	六			
\\	ロク、リク	む、む.つ、むっ.つ、むい	六(む): 
\\	六日(むいか): 
\\	六つ(むっつ): 
\\	六(ろく): 
\\	六 (む), 六つ (む.つ), 六つ (むっ.つ)
\\	七			
\\	シチ	なな、なな.つ、なの	七日(なのか): 
\\	七(しち): 
\\	七つ(ななつ): 
\\	七日(なのか): 
\\	七 (なな), 七つ (なな.つ)
\\	八			
\\	ハチ	や、や.つ、やっ.つ、よう	お八(おやつ): 
\\	八(はち): 
\\	八百屋(やおや): 
\\	八つ(やっつ): 
\\	八日(ようか): 
\\	八 (や), 八つ (や.つ), 八つ (やっ.つ)
\\	九			
\\	一 
\\	八
\\	一+八=九.	
\\	キュウ、ク	ここの、ここの.つ	九(く): 
\\	九(きゅう): 
\\	九日(ここのか): 
\\	九つ(ここのつ): 
\\	九百 (きゅうひゃく), 三拝九拝 (さんはいきゅうはい), 九 (ここの), 九つ (ここの.つ)
\\	十			
\\	ジュウ、ジッ、ジュッ	とお、と	十分(じっぷん): 
\\	十字路(じゅうじろ): 
\\	五十音(ごじゅうおん): 
\\	十分(じゅうぶん): 
\\	十(じゅう): 
\\	十(とお): 
\\	十日(とおか): 
\\	二十歳(はたち): 
\\	二十日(はつか): 
\\	十字架 (じゅうじか), 十文字 (じゅうもんじ), 十 (と), 十 (とお)
\\	口			
\\	コウ、ク	くち	悪口(あっこう): 
\\	甘口(あまくち): 
\\	入口(いりくち): 
\\	口ずさむ(くちずさむ): 
\\	口述(こうじゅつ): 
\\	口頭(こうとう): 
\\	出入り口(でいりぐち): 
\\	閉口(へいこう): 
\\	無口(むくち): 
\\	裏口(うらぐち): 
\\	火口(かこう): 
\\	口紅(くちべに): 
\\	口実(こうじつ): 
\\	蛇口(じゃぐち): 
\\	早口(はやくち): 
\\	窓口(まどぐち): 
\\	利口(りこう): 
\\	悪口(わるくち): 
\\	人口(じんこう): 
\\	入口(いりぐち): 
\\	口(くち): 
\\	出口(でぐち): 
\\	口述 (こうじゅつ), 人口 (じんこう), 開口 (かいこう), 口 (くち)
\\	日			
\\	ニチ、ジツ	ひ、-び、-か	悪日(あくび): 
\\	明後日(あさって): 
\\	明日(あした): 
\\	一日(いちじつ): 
\\	一昨日(おととい): 
\\	日付(かづけ): 
\\	月日(がっぴ): 
\\	期日(きじつ): 
\\	今日は(こんにちは): 
\\	一昨昨日(さきおととい): 
\\	明々後日(しあさって): 
\\	終日(しゅうじつ): 
\\	七日(なのか): 
\\	西日(にしび): 
\\	日夜(にちや): 
\\	日当(にっとう): 
\\	日頃(ひごろ): 
\\	日取り(ひどり): 
\\	日向(ひなた): 
\\	日の丸(ひのまる): 
\\	日々(ひび): 
\\	日焼け(ひやけ): 
\\	三日月(みかずき): 
\\	連日(れんじつ): 
\\	一昨日(いっさくじつ): 
\\	今日(こんにち): 
\\	祭日(さいじつ): 
\\	祝日(しゅくじつ): 
\\	生年月日(せいねんがっぴ): 
\\	先日(せんじつ): 
\\	月日(つきひ): 
\\	定休日(ていきゅうび): 
\\	当日(とうじつ): 
\\	日(にち): 
\\	日時(にちじ): 
\\	日常(にちじょう): 
\\	日用品(にちようひん): 
\\	日課(にっか): 
\\	日光(にっこう): 
\\	日中(にっちゅう): 
\\	日程(にってい): 
\\	日当たり(ひあたり): 
\\	日帰り(ひがえり): 
\\	日陰(ひかげ): 
\\	日付(ひづけ): 
\\	日日(ひにち): 
\\	日の入り(ひのいり): 
\\	日の出(ひので): 
\\	平日(へいじつ): 
\\	三日月(みかづき): 
\\	三日(みっか): 
\\	明後日(みょうごにち): 
\\	夕日(ゆうひ): 
\\	曜日(ようび): 
\\	来日(らいにち): 
\\	明日(あす): 
\\	日記(にっき): 
\\	日(ひ): 
\\	一日(いちにち): 
\\	五日(いつか): 
\\	火曜日(かようび): 
\\	金曜日(きんようび): 
\\	昨日(きのう): 
\\	今日(きょう): 
\\	月曜日(げつようび): 
\\	九日(ここのか): 
\\	水曜日(すいようび): 
\\	誕生日(たんじょうび): 
\\	一日(ついたち): 
\\	十日(とおか): 
\\	土曜日(どようび): 
\\	七日(なのか): 
\\	日曜日(にちようび): 
\\	二十日(はつか): 
\\	二日(ふつか): 
\\	毎日(まいにち): 
\\	六日(むいか): 
\\	木曜日(もくようび): 
\\	八日(ようか): 
\\	四日(よっか): 
\\	日時 (にちじ), 日光 (にっこう), 毎日 (まいにち), 日 (か), 日 (ひ)
\\	月			
\\	ゲツ、ガツ	つき	五月蝿い(うるさい): 
\\	ヶ月(かげつ): 
\\	月日(がっぴ): 
\\	月謝(げっしゃ): 
\\	月賦(げっぷ): 
\\	月曜(げつよう): 
\\	月並み(つきなみ): 
\\	満月(まんげつ): 
\\	三日月(みかずき): 
\\	月給(げっきゅう): 
\\	月末(げつまつ): 
\\	再来月(さらいげつ): 
\\	生年月日(せいねんがっぴ): 
\\	先々月(せんせんげつ): 
\\	月(つき): 
\\	月日(つきひ): 
\\	年月(としつき): 
\\	年月(ねんげつ): 
\\	三日月(みかづき): 
\\	さ来月(さらいげつ): 
\\	月曜日(げつようび): 
\\	今月(こんげつ): 
\\	先月(せんげつ): 
\\	一月(ひとつき): 
\\	毎月(まいげつ): 
\\	来月(らいげつ): 
\\	月曜 (げつよう), 明月 (めいげつ), 歳月 (さいげつ), 月 (つき)
\\	田			
\\	デン	た	水田(すいでん): 
\\	田園(でんえん): 
\\	田(た): 
\\	田植え(たうえ): 
\\	田ぼ(たんぼ): 
\\	田舎(いなか): 
\\	田 (た)
\\	目			
\\	目_目.	
\\	モク、ボク	め、-め、ま-	一目(いちもく): 
\\	お目出度う(おめでとう): 
\\	効き目(ききめ): 
\\	生真面目(きまじめ): 
\\	切れ目(きれめ): 
\\	人目(じんもく): 
\\	着目(ちゃくもく): 
\\	丁目(ちょうめ): 
\\	継ぎ目(つぎめ): 
\\	出鱈目(でたらめ): 
\\	番目(ばんめ): 
\\	目蓋(まぶた): 
\\	目方(めかた): 
\\	目覚しい(めざましい): 
\\	目覚める(めざめる): 
\\	目付き(めつき): 
\\	目眩(めまい): 
\\	目盛(めもり): 
\\	面目(めんぼく): 
\\	目録(もくろく): 
\\	目論見(もくろみ): 
\\	お目に掛かる(おめにかかる): 
\\	科目(かもく): 
\\	項目(こうもく): 
\\	駄目(だめ): 
\\	注目(ちゅうもく): 
\\	真面目(まじめ): 
\\	目上(めうえ): 
\\	目指す(めざす): 
\\	目覚し(めざまし): 
\\	目下(めした): 
\\	目印(めじるし): 
\\	目立つ(めだつ): 
\\	目安(めやす): 
\\	目次(もくじ): 
\\	目的(もくてき): 
\\	目標(もくひょう): 
\\	役目(やくめ): 
\\	目(め): 
\\	目的 (もくてき), 目前 (もくぜん), 項目 (こうもく), 目 (め)
\\	古			
\\	コ	ふる.い、ふる-、-ふる.す	古(いにしえ): 
\\	考古学(こうこがく): 
\\	古代(こだい): 
\\	稽古(けいこ): 
\\	古典(こてん): 
\\	中古(ちゅうこ): 
\\	古い(ふるい): 
\\	古い (ふる.い), 古す (ふる.す)
\\	吾			
\\	ゴ	われ、わが-、あ-		
\\	冒			
\\	ボウ	おか.す	冒頭(ぼうとう): 
\\	冒険(ぼうけん): 
\\	冒す (おか.す)
\\	朋			
\\	ホウ	とも		
\\	明			
\\	メイ、ミョウ、ミン	あ.かり、あか.るい、あか.るむ、あか.らむ、あき.らか、あ.ける、-あ.け、あ.く、あ.くる、あ.かす	明かす(あかす): 
\\	明白(あからさま): 
\\	明るい(あかるい): 
\\	明き(あき): 
\\	明くる(あくる): 
\\	明後日(あさって): 
\\	明日(あした): 
\\	賢明(けんめい): 
\\	明々後日(しあさって): 
\\	照明(しょうめい): 
\\	声明(せいめい): 
\\	不明(ふめい): 
\\	明瞭(めいりょう): 
\\	明朗(めいろう): 
\\	明かり(あかり): 
\\	明らか(あきらか): 
\\	明け方(あけがた): 
\\	明ける(あける): 
\\	証明(しょうめい): 
\\	透明(とうめい): 
\\	発明(はつめい): 
\\	文明(ぶんめい): 
\\	明後日(みょうごにち): 
\\	明確(めいかく): 
\\	夜明け(よあけ): 
\\	明日(あす): 
\\	説明(せつめい): 
\\	明い(あかるい): 
\\	明暗 (めいあん), 説明 (せつめい), 鮮明 (せんめい), 明かす (あ.かす), 明かり (あ.かり), 明く (あ.く), 明くる (あ.くる), 明ける (あ.ける), 明らむ (あか.らむ), 明るい (あか.るい), 明るむ (あか.るむ), 明らか (あき.らか)
\\	唱			
\\	ショウ	とな.える	合唱(がっしょう): 
\\	唱える(となえる): 
\\	唱える (とな.える)
\\	晶			
\\	ショウ		結晶(けっしょう): 
\\	品			
\\	ヒン、ホン	しな	下品(かひん): 
\\	気品(きひん): 
\\	骨董品(こっとうひん): 
\\	出品(しゅっぴん): 
\\	品質(ひんしつ): 
\\	品種(ひんしゅ): 
\\	用品(ようひん): 
\\	下品(げひん): 
\\	作品(さくひん): 
\\	品(しな): 
\\	商品(しょうひん): 
\\	賞品(しょうひん): 
\\	食品(しょくひん): 
\\	製品(せいひん): 
\\	手品(てじな): 
\\	日用品(にちようひん): 
\\	必需品(ひつじゅひん): 
\\	品(ひん): 
\\	部品(ぶひん): 
\\	薬品(やくひん): 
\\	洋品店(ようひんてん): 
\\	品物(しなもの): 
\\	食料品(しょくりょうひん): 
\\	品 (しな)
\\	呂			
\\	ロ、リョ	せぼね	風呂(ふろ): 
\\	風呂敷(ふろしき): 
\\	昌			
\\	ショウ	さかん		
\\	早			
\\	ソウ、サッ	はや.い、はや、はや-、はや.まる、はや.める、さ-	お早う(おはよう): 
\\	早急(さっきゅう): 
\\	素早い(すばやい): 
\\	早める(はやめる): 
\\	最早(もはや): 
\\	早速(さっそく): 
\\	早口(はやくち): 
\\	早い(はやい): 
\\	早期 (そうき), 早晩 (そうばん), 早々に (そうそうに), 早い (はや.い), 早まる (はや.まる), 早める (はや.める)
\\	旭			
\\	キョク	あさひ		
\\	世			
\\	セイ、セ、ソウ	よ、さんじゅう	お世辞(おせじ): 
\\	出世(しゅっせ): 
\\	世辞(せじ): 
\\	世帯(せたい): 
\\	世代(せだい): 
\\	世論(せろん): 
\\	世(よ): 
\\	世紀(せいき): 
\\	世間(せけん): 
\\	中世(ちゅうせい): 
\\	世の中(よのなか): 
\\	世界(せかい): 
\\	世話(せわ): 
\\	世紀 (せいき), 時世 (じせい), 処世 (しょせい), 世 (よ)
\\	胃			
\\	イ		胃(い): 
\\	旦			
\\	旦夕 (たんせき) 
\\	元旦 (がんたん) 
\\	タン、ダン	あき.らか、あきら、ただし、あさ、あした	旦那(だんな): 
\\	一旦(いったん): 
\\	胆			
\\	タン	きも	大胆(だいたん): 
\\	亘			
\\	コウ、カン	わた.る、もと.める		
\\	凹			
\\	オウ	くぼ.む、へこ.む、ぼこ	凸凹(でこぼこ): 
\\	凹む(へこむ): 
\\	凸			
\\	トツ	でこ	凸凹(でこぼこ): 
\\	旧			
\\	キュウ	ふる.い、もと	旧知(きゅうち): 
\\	旧事(くじ): 
\\	復旧(ふくきゅう): 
\\	旧(きゅう): 
\\	自			
\\	ジ、シ	みずか.ら、おの.ずから、おの.ずと	自惚れ(うぬぼれ): 
\\	自ずから(おのずから): 
\\	自我(じが): 
\\	自覚(じかく): 
\\	自己(じこ): 
\\	自在(じざい): 
\\	自主(じしゅ): 
\\	自首(じしゅ): 
\\	自信(じしん): 
\\	自尊心(じそんしん): 
\\	自転(じてん): 
\\	自動詞(じどうし): 
\\	自立(じりつ): 
\\	独自(どくじ): 
\\	各自(かくじ): 
\\	自衛(じえい): 
\\	自殺(じさつ): 
\\	自習(じしゅう): 
\\	自身(じしん): 
\\	自然(しぜん): 
\\	自然科学(しぜんかがく): 
\\	自治(じち): 
\\	自動(じどう): 
\\	自慢(じまん): 
\\	不自由(ふじゆう): 
\\	自由(じゆう): 
\\	自転車(じてんしゃ): 
\\	自動車(じどうしゃ): 
\\	自分(じぶん): 
\\	自分 (じぶん), 自由 (じゆう), 各自 (かくじ), 自ら (みずか.ら)
\\	白			
\\	(日) 
\\	ハク、ビャク	しろ、しら-、しろ.い	青白い(あおじろい): 
\\	明白(あからさま): 
\\	面白い(おもしろい): 
\\	告白(こくはく): 
\\	蛋白質(たんぱくしつ): 
\\	白状(はくじょう): 
\\	白髪(しらが): 
\\	白(しろ): 
\\	真っ白(まっしろ): 
\\	白い(しろい): 
\\	白髪 (しらが), 紅白 (こうはく), 明白 (めいはく), 白 (しろ), 白い (しろ.い)
\\	百			
\\	ヒャク、ビャク	もも	百科事典(ひゃっかじてん): 
\\	百科辞典(ひゃっかじてん): 
\\	百(ひゃく): 
\\	八百屋(やおや): 
\\	中			
\\	チュウ	なか、うち、あた.る	心中(しんじゅう): 
\\	中継(ちゅうけい): 
\\	中指(ちゅうし): 
\\	中傷(ちゅうしょう): 
\\	中枢(ちゅうすう): 
\\	中断(ちゅうだん): 
\\	中腹(ちゅうっぱら): 
\\	中毒(ちゅうどく): 
\\	中立(ちゅうりつ): 
\\	中和(ちゅうわ): 
\\	途中(つちゅう): 
\\	中々(なかなか): 
\\	中程(なかほど): 
\\	中味(なかみ): 
\\	中身(なかみ): 
\\	命中(めいちゅう): 
\\	夜中(やちゅう): 
\\	連中(れんじゅう): 
\\	御中(おんちゅう): 
\\	空中(くうちゅう): 
\\	最中(さいちゅう): 
\\	集中(しゅうちゅう): 
\\	中(ちゅう): 
\\	中央(ちゅうおう): 
\\	中学(ちゅうがく): 
\\	中間(ちゅうかん): 
\\	中古(ちゅうこ): 
\\	中止(ちゅうし): 
\\	中旬(ちゅうじゅん): 
\\	中心(ちゅうしん): 
\\	中性(ちゅうせい): 
\\	中世(ちゅうせい): 
\\	中途(ちゅうと): 
\\	中年(ちゅうねん): 
\\	中指(なかゆび): 
\\	日中(にっちゅう): 
\\	熱中(ねっちゅう): 
\\	年中(ねんじゅう): 
\\	話中(はなしちゅう): 
\\	夢中(むちゅう): 
\\	夜中(よなか): 
\\	世の中(よのなか): 
\\	背中(せなか): 
\\	中学校(ちゅうがっこう): 
\\	途中(とちゅう): 
\\	真ん中(まんなか): 
\\	中(なか): 
\\	中 (なか)
\\	千			
\\	セン	ち	千(せん): 
\\	千 (ち)
\\	舌			
\\	ゼツ	した	舌(した): 
\\	舌 (した)
\\	升			
\\	ショウ	ます		升 (ます)
\\	昇			
\\	ショウ	のぼ.る	上昇(じょうしょう): 
\\	昇進(しょうしん): 
\\	昇る(のぼる): 
\\	昇る (のぼ.る)
\\	丸			
\\	ガン	まる、まる.める、まる.い	日の丸(ひのまる): 
\\	丸ごと(まるごと): 
\\	丸っきり(まるっきり): 
\\	丸々(まるまる): 
\\	丸める(まるめる): 
\\	真ん丸い(まんまるい): 
\\	丸(まる): 
\\	丸い(まるい): 
\\	丸 (まる), 丸い (まる.い), 丸める (まる.める)
\\	寸			
\\	スン		寸法(すんぽう): 
\\	一寸(ちょっと): 
\\	専			
\\	セン	もっぱ.ら	専用(せんよう): 
\\	専修(せんしゅう): 
\\	専ら(もっぱら): 
\\	専攻(せんこう): 
\\	専制(せんせい): 
\\	専ら (もっぱ.ら)
\\	博			
\\	医 
\\	ハク、バク		博士(はかせ): 
\\	博物館(はくぶつかん): 
\\	博識 (はくしき), 博覧 (はくらん), 博士号 (はくしごう)
\\	占			
\\	セン	し.める、うらな.う	占領(せんりょう): 
\\	独占(どくせん): 
\\	占う(うらなう): 
\\	占める(しめる): 
\\	占う (うらな.う), 占める (し.める)
\\	上			
\\	卜. 
\\	ト
\\	ジョウ 以上 いじょう
\\	ショウ 上人 しょうにん
\\	ジョウ、ショウ、シャン	うえ、-うえ、うわ-、かみ、あ.げる、-あ.げる、あ.がる、-あ.がる、あ.がり、-あ.がり、のぼ.る、のぼ.り、のぼ.せる、のぼ.す、よ.す	上がり(あがり): 
\\	上下(うえした): 
\\	上手(うわて): 
\\	上回る(うわまわる): 
\\	お手上げ(おてあげ): 
\\	向上(こうじょう): 
\\	逆上る(さかのぼる): 
\\	参上(さんじょう): 
\\	仕上がり(しあがり): 
\\	仕上げ(しあげ): 
\\	仕上げる(しあげる): 
\\	上位(じょうい): 
\\	上演(じょうえん): 
\\	上空(じょうくう): 
\\	上司(じょうし): 
\\	上昇(じょうしょう): 
\\	上陸(じょうりく): 
\\	その上(そのうえ): 
\\	途上(とじょう): 
\\	上る(のぼる): 
\\	引き上げる(ひきあげる): 
\\	真上(まうえ): 
\\	盛り上がる(もりあがる): 
\\	読み上げる(よみあげる): 
\\	売上(うりあげ): 
\\	上(かみ): 
\\	仕上がる(しあがる): 
\\	上(じょう): 
\\	上級(じょうきゅう): 
\\	上京(じょうきょう): 
\\	上下(じょうげ): 
\\	上旬(じょうじゅん): 
\\	上達(じょうたつ): 
\\	上等(じょうとう): 
\\	立ち上がる(たちあがる): 
\\	頂上(ちょうじょう): 
\\	出来上がり(できあがり): 
\\	出来上がる(できあがる): 
\\	取り上げる(とりあげる): 
\\	上り(のぼり): 
\\	目上(めうえ): 
\\	持ち上げる(もちあげる): 
\\	上がる(あがる): 
\\	以上(いじょう): 
\\	屋上(おくじょう): 
\\	差し上げる(さしあげる): 
\\	召し上がる(めしあがる): 
\\	申し上げる(もうしあげる): 
\\	上げる(あげる): 
\\	上(うえ): 
\\	上着(うわぎ): 
\\	上手(じょうず): 
\\	上旬 (じょうじゅん), 上昇 (じょうしょう), 地上 (ちじょう), 上がる (あ.がる), 上げる (あ.げる), 上 (うえ), 上 (かみ), 上す (のぼ.す), 上せる (のぼ.せる), 上る (のぼ.る)
\\	下			
\\	カ、ゲ	した、しも、もと、さ.げる、さ.がる、くだ.る、くだ.り、くだ.す、-くだ.す、くだ.さる、お.ろす、お.りる	上下(うえした): 
\\	下位(かい): 
\\	下番(かばん): 
\\	下品(かひん): 
\\	下吏(かり): 
\\	下さる(くださる): 
\\	下痢(げり): 
\\	下心(したごころ): 
\\	下地(したじ): 
\\	下調べ(したしらべ): 
\\	下取り(したどり): 
\\	下火(したび): 
\\	城下(じょうか): 
\\	引き下げる(ひきさげる): 
\\	部下(ぶか): 
\\	ぶら下げる(ぶらさげる): 
\\	真下(ました): 
\\	落下(らっか): 
\\	下す(おろす): 
\\	下降(かこう): 
\\	下線(かせん): 
\\	下り(くだり): 
\\	下る(くだる): 
\\	下車(げしゃ): 
\\	下旬(げじゅん): 
\\	下水(げすい): 
\\	下駄(げた): 
\\	下品(げひん): 
\\	下る(さがる): 
\\	下書き(したがき): 
\\	下町(したまち): 
\\	下(しも): 
\\	上下(じょうげ): 
\\	地下(ちか): 
\\	地下水(ちかすい): 
\\	低下(ていか): 
\\	見下ろす(みおろす): 
\\	目下(めした): 
\\	廊下(ろうか): 
\\	以下(いか): 
\\	下りる(おりる): 
\\	下宿(げしゅく): 
\\	下がる(さがる): 
\\	下げる(さげる): 
\\	下着(したぎ): 
\\	靴下(くつした): 
\\	下(した): 
\\	地下鉄(ちかてつ): 
\\	下手(へた): 
\\	下流 (かりゅう), 下降 (かこう), 落下 (らっか), 下りる (お.りる), 下ろす (お.ろす), 下さる (くだ.さる), 下す (くだ.す), 下る (くだ.る), 下がる (さ.がる), 下げる (さ.げる), 下 (した), 下 (しも), 下 (もと)
\\	卓			
\\	タク		食卓(しょくたく): 
\\	朝			
\\	チョウ	あさ	朝寝坊(あさねぼう): 
\\	朝(あさ): 
\\	朝御飯(あさごはん): 
\\	今朝(けさ): 
\\	毎朝(まいあさ): 
\\	朝 (あさ)
\\	只			
\\	シ	ただ	只(ただ): 
\\	貝			
\\	バイ	かい	貝殻(かいがら): 
\\	貝(かい): 
\\	貝 (かい)
\\	貞			
\\	テイ	さだ		
\\	員			
\\	イン		員(いん): 
\\	教員(きょういん): 
\\	行員(こういん): 
\\	従業員(じゅうぎょういん): 
\\	職員(しょくいん): 
\\	動員(どういん): 
\\	委員(いいん): 
\\	会員(かいいん): 
\\	議員(ぎいん): 
\\	工員(こういん): 
\\	全員(ぜんいん): 
\\	定員(ていいん): 
\\	満員(まんいん): 
\\	公務員(こうむいん): 
\\	店員(てんいん): 
\\	見			
\\	ケン	み.る、み.える、み.せる	異見(いけん): 
\\	一見(いちげん): 
\\	会見(かいけん): 
\\	見地(けんち): 
\\	偏見(へんけん): 
\\	見合い(みあい): 
\\	見合わせる(みあわせる): 
\\	見落とす(みおとす): 
\\	見掛ける(みかける): 
\\	見方(みかた): 
\\	見苦しい(みぐるしい): 
\\	見込み(みこみ): 
\\	見すぼらしい(みすぼらしい): 
\\	見せびらかす(みせびらかす): 
\\	見せ物(みせもの): 
\\	見積り(みつもり): 
\\	見通し(みとおし): 
\\	見逃す(みのがす): 
\\	見晴らし(みはらし): 
\\	見舞(みまい): 
\\	見渡す(みわたす): 
\\	目論見(もくろみ): 
\\	余所見(よそみ): 
\\	見解(けんかい): 
\\	見学(けんがく): 
\\	見当(けんとう): 
\\	発見(はっけん): 
\\	花見(はなみ): 
\\	見送り(みおくり): 
\\	見送る(みおくる): 
\\	見下ろす(みおろす): 
\\	見掛け(みかけ): 
\\	見事(みごと): 
\\	見出し(みだし): 
\\	見付かる(みつかる): 
\\	見付ける(みつける): 
\\	見直す(みなおす): 
\\	見慣れる(みなれる): 
\\	見本(みほん): 
\\	見舞い(みまい): 
\\	見舞う(みまう): 
\\	意見(いけん): 
\\	お見舞い(おみまい): 
\\	見物(けんぶつ): 
\\	拝見(はいけん): 
\\	見える(みえる): 
\\	見つかる(みつかる): 
\\	見つける(みつける): 
\\	見せる(みせる): 
\\	見る(みる): 
\\	見える (み.える), 見せる (み.せる), 見る (み.る)
\\	児			
\\	ジ 児童 じどう 
\\	ニ 小児 しょうに
\\	ジ、ニ、ゲイ	こ、-こ、-っこ	児(こ): 
\\	孤児(こじ): 
\\	小児科(しょうにか): 
\\	育児(いくじ): 
\\	児童(じどう): 
\\	幼児(ようじ): 
\\	児童 (じどう), 幼児 (ようじ), 優良児 (ゆうりょうじ)
\\	元			
\\	ゲン、ガン	もと	還元(かんげん): 
\\	元年(がんねん): 
\\	元来(がんらい): 
\\	元首(げんしゅ): 
\\	元素(げんそ): 
\\	地元(じもと): 
\\	手元(てもと): 
\\	元(もと): 
\\	元々(もともと): 
\\	元気(げんき): 
\\	元素 (げんそ), 元気 (げんき), 多元 (たげん), 元 (もと)
\\	頁			
\\	ケツ	ぺえじ、おおがい、かしら		
\\	頑			
\\	ガン	かたく	頑固(がんこ): 
\\	頑丈(がんじょう): 
\\	凡			
\\	ボン、ハン	およ.そ、おうよ.そ、すべ.て	凡そ(およそ): 
\\	大凡(おおよそ): 
\\	平凡(へいぼん): 
\\	凡人 (ぼんじん), 凡百 (ぼんひゃく), 平凡 (へいぼん)
\\	負			
\\	フ	ま.ける、ま.かす、お.う	負う(おう): 
\\	御負け(おまけ): 
\\	背負う(しょう): 
\\	勝負(しょうぶ): 
\\	負債(ふさい): 
\\	負傷(ふしょう): 
\\	負担(ふたん): 
\\	負かす(まかす): 
\\	背負う(せおう): 
\\	負け(まけ): 
\\	負ける(まける): 
\\	負う (お.う), 負かす (ま.かす), 負ける (ま.ける)
\\	万			
\\	マン、バン	よろず	万(ばん): 
\\	万人(ばんじん): 
\\	万能(ばんのう): 
\\	万歳(ばんざい): 
\\	万一(まんいち): 
\\	万(まん): 
\\	万年筆(まんねんひつ): 
\\	万一 (まんいち), 万年筆 (まんねんひつ), 巨万 (きょまん)
\\	句			
\\	ク		佳句(かく): 
\\	語句(ごく): 
\\	句(く): 
\\	句読点(くとうてん): 
\\	俳句(はいく): 
\\	文句(もんく): 
\\	句 (く)
\\	肌			
\\	キ	はだ	肌(はだ): 
\\	肌着(はだぎ): 
\\	肌 (はだ)
\\	旬			
\\	ジュン、シュン		下旬(げじゅん): 
\\	上旬(じょうじゅん): 
\\	初旬(しょじゅん): 
\\	中旬(ちゅうじゅん): 
\\	勺			
\\	シャク			
\\	的			
\\	テキ	まと	静的(せいてき): 
\\	先天的(せんてんてき): 
\\	知的(ちてき): 
\\	的(てき): 
\\	的確(てきかく): 
\\	動的(どうてき): 
\\	的(まと): 
\\	消極的(しょうきょくてき): 
\\	積極的(せっきょくてき): 
\\	比較的(ひかくてき): 
\\	目的(もくてき): 
\\	的 (まと)
\\	首			
\\	シュ	くび	首飾り(くびかざり): 
\\	首輪(くびわ): 
\\	元首(げんしゅ): 
\\	自首(じしゅ): 
\\	首脳(しゅのう): 
\\	首相(しゅしょう): 
\\	首都(しゅと): 
\\	手首(てくび): 
\\	部首(ぶしゅ): 
\\	首(くび): 
\\	首 (くび)
\\	乙			
\\	オツ、イツ	おと-、きのと	乙(おつ): 
\\	乱			
\\	ラン、ロン	みだ.れる、みだ.る、みだ.す、みだ、おさ.める、わた.る	内乱(ないらん): 
\\	反乱(はんらん): 
\\	乱す(みだす): 
\\	乱れる(みだれる): 
\\	混乱(こんらん): 
\\	乱暴(らんぼう): 
\\	乱す (みだ.す), 乱れる (みだ.れる)
\\	直			
\\	チョク、ジキ、ジカ	ただ.ちに、なお.す、-なお.す、なお.る、なお.き、す.ぐ	直ぐ(すぐ): 
\\	率直(そっちょく): 
\\	直面(ちょくめん): 
\\	直感(ちょっかん): 
\\	出直し(でなおし): 
\\	直に(じかに): 
\\	直(じき): 
\\	正直(しょうじき): 
\\	垂直(すいちょく): 
\\	素直(すなお): 
\\	卒直(そっちょく): 
\\	直ちに(ただちに): 
\\	直後(ちょくご): 
\\	直接(ちょくせつ): 
\\	直線(ちょくせん): 
\\	直前(ちょくぜん): 
\\	直通(ちょくつう): 
\\	直流(ちょくりゅう): 
\\	直角(ちょっかく): 
\\	直径(ちょっけい): 
\\	仲直り(なかなおり): 
\\	真っ直ぐ(まっすぐ): 
\\	見直す(みなおす): 
\\	直す(なおす): 
\\	直る(なおる): 
\\	直立 (ちょくりつ), 直接 (ちょくせつ), 実直 (じっちょく), 直ちに (ただ.ちに), 直す (なお.す), 直る (なお.る)
\\	具			
\\	グ	そな.える、つぶさ.に	雨具(あまぐ): 
\\	敬具(けいぐ): 
\\	夜具(やぐ): 
\\	絵の具(えのぐ): 
\\	家具(かぐ): 
\\	器具(きぐ): 
\\	具体(ぐたい): 
\\	具える(そなえる): 
\\	文房具(ぶんぼうぐ): 
\\	具合(ぐあい): 
\\	道具(どうぐ): 
\\	真			
\\	シン	ま、ま-、まこと	生真面目(きまじめ): 
\\	真実(さな): 
\\	真珠(しんじゅ): 
\\	真相(しんそう): 
\\	真理(しんり): 
\\	真上(まうえ): 
\\	真心(まこころ): 
\\	真に(まことに): 
\\	真下(ました): 
\\	真っ二つ(まっぷたつ): 
\\	真ん前(まんまえ): 
\\	真ん丸い(まんまるい): 
\\	真空(しんくう): 
\\	真剣(しんけん): 
\\	真面目(まじめ): 
\\	真っ赤(まっか): 
\\	真っ暗(まっくら): 
\\	真っ黒(まっくろ): 
\\	真っ青(まっさお): 
\\	真っ先(まっさき): 
\\	真っ白(まっしろ): 
\\	真っ直ぐ(まっすぐ): 
\\	真似(まね): 
\\	真似る(まねる): 
\\	真ん中(まんなか): 
\\	写真(しゃしん): 
\\	真 (ま)
\\	工			
\\	コウ、ク、グ		加工(かこう): 
\\	工学(こうがく): 
\\	工作(こうさく): 
\\	細工(さいく): 
\\	着工(ちゃっこう): 
\\	工夫(くふう): 
\\	工員(こういん): 
\\	工芸(こうげい): 
\\	工事(こうじ): 
\\	工場(こうば): 
\\	人工(じんこう): 
\\	大工(だいく): 
\\	工業(こうぎょう): 
\\	工場(こうじょう): 
\\	工場 (こうじょう), 加工 (かこう), 人工 (じんこう)
\\	左			
\\	サ、シャ	ひだり	左程(さほど): 
\\	左様なら(さようなら): 
\\	左利き(ひだりきき): 
\\	左右(さゆう): 
\\	左(ひだり): 
\\	左 (ひだり)
\\	右			
\\	ウ、ユウ	みぎ	左右(さゆう): 
\\	右(みぎ): 
\\	右岸 (みぎぎし), 右折 (うせつ), 右派 (うは), 右 (みぎ)
\\	有			
\\	ユウ、ウ	あ.る	有難う(ありがとう): 
\\	有様(ありさま): 
\\	有りのまま(ありのまま): 
\\	有る(ある): 
\\	国有(こくゆう): 
\\	固有(こゆう): 
\\	私有(しゆう): 
\\	特有(とくゆう): 
\\	有益(ゆうえき): 
\\	有機(ゆうき): 
\\	有する(ゆうする): 
\\	有望(ゆうぼう): 
\\	有力(ゆうりょく): 
\\	有難い(ありがたい): 
\\	有無(うむ): 
\\	有効(ゆうこう): 
\\	有能(ゆうのう): 
\\	有利(ゆうり): 
\\	有料(ゆうりょう): 
\\	有名(ゆうめい): 
\\	有益 (ゆうえき), 所有 (しょゆう), 特有 (とくゆう), 有 (ある)
\\	賄			
\\	ワイ	まかな.う	賄う(まかなう): 
\\	賄う (まかな.う)
\\	貢			
\\	コウ、ク	みつ.ぐ	貢献(こうけん): 
\\	貢献 (こうけん), 貢 (みつぐ)
\\	項			
\\	コウ	うなじ	事項(じこう): 
\\	項目(こうもく): 
\\	刀			
\\	カ 
\\	カタナ.	
\\	トウ	かたな、そり	刀(かたな): 
\\	剃刀(かみそり): 
\\	刀 (かたな)
\\	刃			
\\	ジン、ニン	は、やいば、き.る	刃(は): 
\\	刃 (は)
\\	切			
\\	セツ、サイ	き.る、-き.る、き.り、-き.り、-ぎ.り、き.れる、-き.れる、き.れ、-き.れ、-ぎ.れ	一切(いっさい): 
\\	打ち切る(うちきる): 
\\	噛み切る(かみきる): 
\\	切っ掛け(きっかけ): 
\\	切り(きり): 
\\	切り替える(きりかえる): 
\\	切れ目(きれめ): 
\\	区切り(くぎり): 
\\	小切手(こぎって): 
\\	仕切る(しきる): 
\\	締め切り(しめきり): 
\\	切開(せっかい): 
\\	切実(せつじつ): 
\\	切ない(せつない): 
\\	痛切(つうせつ): 
\\	出切る(できる): 
\\	跡切れる(とぎれる): 
\\	持ち切り(もちきり): 
\\	裏切る(うらぎる): 
\\	売り切れ(うりきれ): 
\\	売り切れる(うりきれる): 
\\	切れ(きれ): 
\\	切れる(きれる): 
\\	区切る(くぎる): 
\\	締切(しめきり): 
\\	締め切る(しめきる): 
\\	適切(てきせつ): 
\\	張り切る(はりきる): 
\\	踏切(ふみきり): 
\\	横切る(よこぎる): 
\\	親切(しんせつ): 
\\	切手(きって): 
\\	切符(きっぷ): 
\\	切る(きる): 
\\	大切(たいせつ): 
\\	切断 (せつだん), 親切 (しんせつ), 切に (せつに), 切る (き.る), 切れる (き.れる)
\\	召			
\\	ショウ	め.す	召す(めす): 
\\	召し上がる(めしあがる): 
\\	召す (め.す)
\\	昭			
\\	ショウ			
\\	則			
\\	ソク	のっと.る	原則(げんそく): 
\\	不規則(ふきそく): 
\\	法則(ほうそく): 
\\	規則(きそく): 
\\	副			
\\	フク		副詞(ふくし): 
\\	別			
\\	ベツ	わか.れる、わ.ける	一別(いちべつ): 
\\	個別(こべつ): 
\\	別(べつ): 
\\	格別(かくべつ): 
\\	区別(くべつ): 
\\	差別(さべつ): 
\\	性別(せいべつ): 
\\	送別(そうべつ): 
\\	別荘(べっそう): 
\\	別に(べつに): 
\\	別々(べつべつ): 
\\	別れ(わかれ): 
\\	特別(とくべつ): 
\\	別れる(わかれる): 
\\	別れる (わか.れる)
\\	丁			
\\	チョウ、テイ、チン、トウ、チ	ひのと	丁目(ちょうめ): 
\\	丁々(とうとう): 
\\	庖丁(ほうちょう): 
\\	丁寧(ていねい): 
\\	丁数 (ちょうかず), 落丁 (らくちょう), 二丁目 (にちょうめ)
\\	町			
\\	チョウ	まち	下町(したまち): 
\\	町(まち): 
\\	町 (まち)
\\	可			
\\	カ、コク	-べ.き	可笑しい(おかしい): 
\\	可成(かなり): 
\\	可愛い(かわいい): 
\\	可愛がる(かわいがる): 
\\	可哀想(かわいそう): 
\\	可愛らしい(かわいらしい): 
\\	不可欠(ふかけつ): 
\\	可(か): 
\\	可決(かけつ): 
\\	可能(かのう): 
\\	許可(きょか): 
\\	不可(ふか): 
\\	頂			
\\	チョウ	いただ.く、いただき	頂(いただき): 
\\	頂く(いただく): 
\\	頂上(ちょうじょう): 
\\	頂点(ちょうてん): 
\\	頂く (いただ.く), 頂 (いただき)
\\	子			
\\	シ、ス、ツ	こ、-こ、ね	女子(おなご): 
\\	原子(げんし): 
\\	子息(しそく): 
\\	扇子(せんす): 
\\	梯子(ていし): 
\\	捻子(ねじ): 
\\	分子(ぶんし): 
\\	利子(りし): 
\\	椅子(いす): 
\\	王子(おうじ): 
\\	菓子(かし): 
\\	子孫(しそん): 
\\	障子(しょうじ): 
\\	女子(じょし): 
\\	末っ子(すえっこ): 
\\	男子(だんし): 
\\	調子(ちょうし): 
\\	電子(でんし): 
\\	梯子(はしご): 
\\	判子(はんこ): 
\\	双子(ふたご): 
\\	迷子(まいご): 
\\	様子(ようす): 
\\	子(こ): 
\\	息子(むすこ): 
\\	お菓子(おかし): 
\\	男の子(おとこのこ): 
\\	女の子(おんなのこ): 
\\	子供(こども): 
\\	帽子(ぼうし): 
\\	子孫 (しそん), 女子 (じょし), 帽子 (ぼうし), 子 (こ)
\\	孔			
\\	コウ	あな		
\\	了			
\\	リョウ		"修了(しゅうりょう): 
\\	了(りょう): 
\\	了解(りょうかい): 
\\	了承(りょうしょう): 
\\	完了(かんりょう): 
\\	終了(しゅうりょう): 
\\	女			
\\	く
\\	ノ
\\	一
\\	ジョ、ニョ、ニョウ	おんな、め	貴女(あなた): 
\\	少女(おとめ): 
\\	女子(おなご): 
\\	女史(じょし): 
\\	王女(おうじょ): 
\\	女の人(おんなのひと): 
\\	少女(しょうじょ): 
\\	女王(じょおう): 
\\	女子(じょし): 
\\	女優(じょゆう): 
\\	長女(ちょうじょ): 
\\	女房(にょうぼう): 
\\	彼女(かのじょ): 
\\	女性(じょせい): 
\\	女(おんな): 
\\	女の子(おんなのこ): 
\\	女子 (じょし), 女流 (じょりゅう), 少女 (しょうじょ), 女人 (にょにん), 天女 (てんにょ), 善男善女 (ぜんなんぜんにょ), 女 (おんな), 女 (め)
\\	好			
\\	コウ 好評 こうひょう
\\	コウ	この.む、す.く、よ.い、い.い	格好(かっこう): 
\\	好意(こうい): 
\\	好況(こうきょう): 
\\	好調(こうちょう): 
\\	好評(こうひょう): 
\\	好ましい(このましい): 
\\	嗜好(しこう): 
\\	物好き(ものずき): 
\\	好い(よい): 
\\	良好(りょうこう): 
\\	好み(このみ): 
\\	好む(このむ): 
\\	好き嫌い(すききらい): 
\\	好き好き(すきずき): 
\\	友好(ゆうこう): 
\\	好き(すき): 
\\	大好き(だいすき): 
\\	好む (この.む), 好く (す.く)
\\	如			
\\	口
\\	ジョ、ニョ	ごと.し	如何(いかが): 
\\	如何に(いかに): 
\\	如何にも(いかにも): 
\\	如何しても(どうしても): 
\\	突如(とつじょ): 
\\	欠如 (けつじょ), 突如 (とつじょ), 躍如 (やくじょ)
\\	母			
\\	ボ	はは、も	お祖母さん(おばあさん): 
\\	伯母さん(おばさん): 
\\	父母(ちちはは): 
\\	分母(ぶんぼ): 
\\	母校(ぼこう): 
\\	保母(ほぼ): 
\\	母親(ははおや): 
\\	父母(ふぼ): 
\\	祖母(そぼ): 
\\	お母さん(おかあさん): 
\\	母 (はは)
\\	貫			
\\	カン	つらぬ.く、ぬ.く、ぬき	貫禄(かんろく): 
\\	貫く(つらぬく): 
\\	貫く (つらぬ.く)
\\	兄			
\\	ケイ 父兄 ふけい 
\\	キョウ 兄弟 きょうだい
\\	(コク): 克104 (キョウ): 況145 競434 (シュク): 祝1089.	
\\	ケイ、キョウ	あに	"従兄弟(いとこ): 
\\	兄(あに): 
\\	お兄さん(おにいさん): 
\\	兄弟(きょうだい): 
\\	兄事 (あにこと), 父兄 (ふけい), 義兄 (ぎあに), 兄 (あに)
\\	克			
\\	コク	か.つ	克服(こくふく): 
\\	小			
\\	ショウ	ちい.さい、こ-、お-、さ-	小売(こうり): 
\\	小柄(こがら): 
\\	小切手(こぎって): 
\\	小銭(こぜに): 
\\	小児科(しょうにか): 
\\	小父さん(おじさん): 
\\	小遣い(こづかい): 
\\	小包(こづつみ): 
\\	小麦(こむぎ): 
\\	小屋(こや): 
\\	小指(こゆび): 
\\	縮小(しゅくしょう): 
\\	小(しょう): 
\\	小学生(しょうがくせい): 
\\	小数(しょうすう): 
\\	小便(しょうべん): 
\\	大小(だいしょう): 
\\	小鳥(ことり): 
\\	小学校(しょうがっこう): 
\\	小説(しょうせつ): 
\\	小さな(ちいさな): 
\\	小さい(ちいさい): 
\\	小 (お), 小 (こ), 小さい (ちい.さい)
\\	少			
\\	ショウ	すく.ない、すこ.し	少女(おとめ): 
\\	減少(げんしょう): 
\\	少数(しょうすう): 
\\	少ない(すくない): 
\\	少なくとも(すくなくとも): 
\\	少女(しょうじょ): 
\\	少々(しょうしょう): 
\\	少年(しょうねん): 
\\	少しも(すこしも): 
\\	青少年(せいしょうねん): 
\\	多少(たしょう): 
\\	少し(すこし): 
\\	少ない (すく.ない), 少し (すこ.し)
\\	大			
\\	ダイ、タイ	おお-、おお.きい、-おお.いに	大方(おおかた): 
\\	大柄(おおがら): 
\\	大げさ(おおげさ): 
\\	大事(おおごと): 
\\	大ざっぱ(おおざっぱ): 
\\	大筋(おおすじ): 
\\	大空(おおぞら): 
\\	大幅(おおはば): 
\\	大水(おおみず): 
\\	お大事に(おだいじに): 
\\	盛大(せいだい): 
\\	壮大(そうだい): 
\\	大概(たいがい): 
\\	大金(たいきん): 
\\	大衆(たいしゅう): 
\\	大胆(だいたん): 
\\	大抵(たいてい): 
\\	大部(たいぶ): 
\\	大便(だいべん): 
\\	短大(たんだい): 
\\	長大(ちょうだい): 
\\	偉大(いだい): 
\\	大いに(おおいに): 
\\	大通り(おおどおり): 
\\	大家(おおや): 
\\	大凡(おおよそ): 
\\	拡大(かくだい): 
\\	巨大(きょだい): 
\\	重大(じゅうだい): 
\\	増大(ぞうだい): 
\\	総理大臣(そうりだいじん): 
\\	大(だい): 
\\	大会(たいかい): 
\\	大学院(だいがくいん): 
\\	大気(たいき): 
\\	大工(だいく): 
\\	大使(たいし): 
\\	大した(たいした): 
\\	大して(たいして): 
\\	大小(だいしょう): 
\\	大臣(だいじん): 
\\	大戦(たいせん): 
\\	大層(たいそう): 
\\	大体(だいたい): 
\\	大統領(だいとうりょう): 
\\	大半(たいはん): 
\\	大分(だいぶ): 
\\	大部分(だいぶぶん): 
\\	大分(だいぶん): 
\\	大木(たいぼく): 
\\	大陸(たいりく): 
\\	莫大(ばくだい): 
\\	膨大(ぼうだい): 
\\	大きな(おおきな): 
\\	大学生(だいがくせい): 
\\	大事(だいじ): 
\\	大きい(おおきい): 
\\	大勢(おおぜい): 
\\	大人(おとな): 
\\	大学(だいがく): 
\\	大使館(たいしかん): 
\\	大丈夫(だいじょうぶ): 
\\	大好き(だいすき): 
\\	大切(たいせつ): 
\\	大小 (だいしょう), 大胆 (だいたん), 拡大 (かくだい), 大 (おお), 大いに (おお.いに), 大きい (おお.きい)
\\	多			
\\	タ	おお.い、まさ.に、まさ.る	幾多(いくた): 
\\	過多(かた): 
\\	加留多(かるた): 
\\	多数決(たすうけつ): 
\\	多分(たぶん): 
\\	多忙(たぼう): 
\\	多様(たよう): 
\\	多少(たしょう): 
\\	滅多に(めったに): 
\\	多い(おおい): 
\\	多い (おお.い)
\\	夕			
\\	セキ	ゆう	夕暮れ(ゆうぐれ): 
\\	夕焼け(ゆうやけ): 
\\	夕刊(ゆうかん): 
\\	夕立(ゆうだち): 
\\	夕日(ゆうひ): 
\\	夕べ(ゆうべ): 
\\	夕飯(ゆうはん): 
\\	夕方(ゆうがた): 
\\	夕 (ゆう)
\\	汐			
\\	セキ	しお、うしお、せい		
\\	外			
\\	ガイ、ゲ	そと、ほか、はず.す、はず.れる、と-	域外(いきがい): 
\\	外貨(がいか): 
\\	外観(がいかん): 
\\	外相(がいしょう): 
\\	外来(がいらい): 
\\	課外(かがい): 
\\	除外(じょがい): 
\\	外方(そっぽ): 
\\	その外(そのほか): 
\\	野外(やがい): 
\\	案外(あんがい): 
\\	以外(いがい): 
\\	意外(いがい): 
\\	屋外(おくがい): 
\\	海外(かいがい): 
\\	外交(がいこう): 
\\	外出(がいしゅつ): 
\\	外部(がいぶ): 
\\	外科(げか): 
\\	外す(はずす): 
\\	外れる(はずれる): 
\\	例外(れいがい): 
\\	郊外(こうがい): 
\\	外国(がいこく): 
\\	外国人(がいこくじん): 
\\	外(そと): 
\\	外(ほか): 
\\	外出 (がいしゅつ), 海外 (かいがい), 除外 (じょがい), 外 (そと), 外す (はず.す), 外れる (はず.れる), 外 (ほか)
\\	名			
\\	メイ、ミョウ	な、-な	記名(きめい): 
\\	姓名(せいめい): 
\\	著名(ちょめい): 
\\	名残(なごり): 
\\	名高い(なだかい): 
\\	名付ける(なづける): 
\\	名札(なふだ): 
\\	本名(ほんみょう): 
\\	名産(めいさん): 
\\	名称(めいしょう): 
\\	名簿(めいぼ): 
\\	名誉(めいよ): 
\\	宛名(あてな): 
\\	送り仮名(おくりがな): 
\\	仮名(かな): 
\\	仮名遣い(かなづかい): 
\\	氏名(しめい): 
\\	署名(しょめい): 
\\	題名(だいめい): 
\\	代名詞(だいめいし): 
\\	地名(ちめい): 
\\	名(な): 
\\	振り仮名(ふりがな): 
\\	名字(みょうじ): 
\\	名作(めいさく): 
\\	名刺(めいし): 
\\	名詞(めいし): 
\\	名所(めいしょ): 
\\	名人(めいじん): 
\\	名物(めいぶつ): 
\\	片仮名(かたかな): 
\\	名前(なまえ): 
\\	平仮名(ひらがな): 
\\	有名(ゆうめい): 
\\	名誉 (めいよ), 氏名 (しめい), 有名 (ゆうめい), 名 (な)
\\	石			
\\	[いし] 
\\	[ほうせき] 
\\	セキ、シャク、コク	いし	化石(かせき): 
\\	岩石(がんせき): 
\\	石鹸(せっけん): 
\\	流石(さすが): 
\\	磁石(じしゃく): 
\\	石炭(せきたん): 
\\	石油(せきゆ): 
\\	宝石(ほうせき): 
\\	石(いし): 
\\	石材 (せきざい), 岩石 (がんせき), 宝石 (ほうせき), 磁石 (じしゃく), 石 (いし)
\\	肖			
\\	ショウ	あやか.る		
\\	硝			
\\	ショウ			
\\	砕			
\\	サイ	くだ.く、くだ.ける	砕く(くだく): 
\\	砕ける(くだける): 
\\	砕く (くだ.く), 砕ける (くだ.ける)
\\	砂			
\\	サ、シャ	すな	砂利(じゃり): 
\\	砂漠(さばく): 
\\	砂(すな): 
\\	砂糖(さとう): 
\\	砂丘 (さきゅう), 砂糖 (さとう), 砂 (すな)
\\	削			
\\	サク	けず.る、はつ.る、そ.ぐ	削減(さくげん): 
\\	削る(けずる): 
\\	削除(さくじょ): 
\\	削る (けず.る)
\\	光			
\\	コウ	ひか.る、ひかり	稲光(いなびかり): 
\\	光沢(こうたく): 
\\	光熱費(こうねつひ): 
\\	観光(かんこう): 
\\	蛍光灯(けいこうとう): 
\\	光景(こうけい): 
\\	光線(こうせん): 
\\	日光(にっこう): 
\\	光(ひかり): 
\\	光る(ひかる): 
\\	光る (ひか.る), 光 (ひかり)
\\	太			
\\	タイ、タ	ふと.い、ふと.る	太鼓(たいこ): 
\\	太陽(たいよう): 
\\	太る(ふとる): 
\\	太い(ふとい): 
\\	太陽 (たいよう), 太鼓 (たいこ), 皇太子 (こうたいし), 太い (ふと.い), 太る (ふと.る)
\\	器			
\\	器 
\\	食器 
\\	キ	うつわ	器(うつわ): 
\\	器官(きかん): 
\\	計器(けいき): 
\\	磁器(じき): 
\\	聴診器(ちょうしんき): 
\\	陶器(とうき): 
\\	兵器(へいき): 
\\	宝器(ほうき): 
\\	楽器(がっき): 
\\	器械(きかい): 
\\	器具(きぐ): 
\\	器用(きよう): 
\\	受話器(じゅわき): 
\\	食器(しょっき): 
\\	武器(ぶき): 
\\	容器(ようき): 
\\	器 (うつわ)
\\	臭			
\\	シュウ	くさ.い、-くさ.い、にお.う、にお.い	生臭い(なまぐさい): 
\\	臭い(くさい): 
\\	面倒臭い(めんどうくさい): 
\\	臭い (くさ.い)
\\	妙			
\\	ミョウ、ビョウ	たえ	巧妙(こうみょう): 
\\	奇妙(きみょう): 
\\	微妙(びみょう): 
\\	妙(みょう): 
\\	省			
\\	セイ、ショウ	かえり.みる、はぶ.く	省みる(かえりみる): 
\\	省略(しょうりゃく): 
\\	省く(はぶく): 
\\	反省(はんせい): 
\\	反省 (はんせい), 内省 (ないせい), 帰省 (きせい), 省みる (かえり.みる), 省く (はぶ.く)
\\	厚			
\\	コウ	あつ.い、あか	厚かましい(あつかましい): 
\\	厚い(あつい): 
\\	厚い (あつ.い)
\\	奇			
\\	キ	く.しき、あや.しい、くし、めずら.しい	奇数(きすう): 
\\	奇麗(きれい): 
\\	奇妙(きみょう): 
\\	川			
\\	セン	かわ	河川(かせん): 
\\	川(かわ): 
\\	川 (かわ)
\\	州			
\\	"シュウ 
\\	シュウ、ス	す	州(しゅう): 
\\	州 (す)
\\	順			
\\	ジュン		手順(てじゅん): 
\\	不順(ふじゅん): 
\\	順(じゅん): 
\\	順々(じゅんじゅん): 
\\	順序(じゅんじょ): 
\\	順調(じゅんちょう): 
\\	順番(じゅんばん): 
\\	道順(みちじゅん): 
\\	水			
\\	スイ	みず、みず-	大水(おおみず): 
\\	降水(こうすい): 
\\	洪水(こうずい): 
\\	水気(すいき): 
\\	水源(すいげん): 
\\	水洗(すいせん): 
\\	水田(すいでん): 
\\	水曜(すいよう): 
\\	潜水(せんすい): 
\\	排水(はいすい): 
\\	海水浴(かいすいよく): 
\\	下水(げすい): 
\\	香水(こうすい): 
\\	水産(すいさん): 
\\	水準(すいじゅん): 
\\	水蒸気(すいじょうき): 
\\	水素(すいそ): 
\\	水滴(すいてき): 
\\	水筒(すいとう): 
\\	水分(すいぶん): 
\\	水平(すいへい): 
\\	水平線(すいへいせん): 
\\	淡水(たんすい): 
\\	断水(だんすい): 
\\	地下水(ちかすい): 
\\	噴水(ふんすい): 
\\	水泳(すいえい): 
\\	水道(すいどう): 
\\	水曜日(すいようび): 
\\	水(みず): 
\\	水 (みず)
\\	氷			
\\	ヒョウ	こおり、ひ、こお.る	氷(こおり): 
\\	氷 (こおり), 氷 (ひ)
\\	永			
\\	エイ	なが.い	永遠(えいえん): 
\\	永久(えいきゅう): 
\\	永い(ながい): 
\\	永い (なが.い)
\\	泉			
\\	温泉 (おんせん, 
\\	セン	いずみ	泉(いずみ): 
\\	温泉(おんせん): 
\\	泉 (いずみ)
\\	原			
\\	ゲン	はら	原油(げんゆ): 
\\	原(げん): 
\\	原形(げんけい): 
\\	原作(げんさく): 
\\	原子(げんし): 
\\	原書(げんしょ): 
\\	原則(げんそく): 
\\	原典(げんてん): 
\\	原点(げんてん): 
\\	原爆(げんばく): 
\\	原文(げんぶん): 
\\	高原(こうげん): 
\\	原っぱ(はらっぱ): 
\\	原稿(げんこう): 
\\	原産(げんさん): 
\\	原始(げんし): 
\\	原理(げんり): 
\\	原料(げんりょう): 
\\	原(はら): 
\\	原因(げんいん): 
\\	原 (はら)
\\	願			
\\	""お願いします、おねがいします。”.
\\	ガン	ねが.う、-ねがい	お願いします(おねがいします): 
\\	願書(がんしょ): 
\\	願い(ねがい): 
\\	願う(ねがう): 
\\	願う (ねが.う)
\\	泳			
\\	エイ	およ.ぐ	泳ぎ(およぎ): 
\\	水泳(すいえい): 
\\	泳ぐ(およぐ): 
\\	泳ぐ (およ.ぐ)
\\	沼			
\\	ショウ	ぬま	沼(ぬま): 
\\	沼 (ぬま)
\\	沖			
\\	チュウ	おき、おきつ、ちゅう.する、わく	沖(おき): 
\\	沖 (おき)
\\	江			
\\	コウ	え		江 (え)
\\	汁			
\\	ジュウ	しる、-しる、つゆ	汁(しる): 
\\	汁 (しる)
\\	潮			
\\	チョウ	しお、うしお	潮(うしお): 
\\	潮 (しお)
\\	源			
\\	ゲン	みなもと	起源(きげん): 
\\	語源(ごげん): 
\\	財源(ざいげん): 
\\	水源(すいげん): 
\\	電源(でんげん): 
\\	源(みなもと): 
\\	資源(しげん): 
\\	源 (みなもと)
\\	活			
\\	カツ	い.きる、い.かす、い.ける	活ける(いける): 
\\	活発(かっぱつ): 
\\	生活(せいかつ): 
\\	復活(ふっかつ): 
\\	活気(かっき): 
\\	活字(かつじ): 
\\	活動(かつどう): 
\\	活躍(かつやく): 
\\	活用(かつよう): 
\\	活力(かつりょく): 
\\	消			
\\	ショウ	き.える、け.す	打ち消し(うちけし): 
\\	消去(しょうきょ): 
\\	消耗(しょうこう): 
\\	消息(しょうそく): 
\\	打ち消す(うちけす): 
\\	消しゴム(けしゴム): 
\\	消化(しょうか): 
\\	消極的(しょうきょくてき): 
\\	消毒(しょうどく): 
\\	消費(しょうひ): 
\\	消防(しょうぼう): 
\\	消防署(しょうぼうしょ): 
\\	消耗(しょうもう): 
\\	取り消す(とりけす): 
\\	消える(きえる): 
\\	消す(けす): 
\\	消える (き.える), 消す (け.す)
\\	況			
\\	"キョウ 
\\	状況 (じょうきょう) 
\\	キョウ	まし.て、いわ.んや、おもむき	好況(こうきょう): 
\\	不況(ふきょう): 
\\	況して(まして): 
\\	状況(じょうきょう): 
\\	河			
\\	(黄河), 
\\	(銀河). 
\\	カ	かわ	河川(かせん): 
\\	運河(うんが): 
\\	河(かわ): 
\\	河 (かわ)
\\	泊			
\\	ハク	と.まる、と.める	泊(はく): 
\\	宿泊(しゅくはく): 
\\	泊まる(とまる): 
\\	泊める(とめる): 
\\	泊まる (と.まる), 泊める (と.める)
\\	湖			
\\	コ	みずうみ	湖(みずうみ): 
\\	湖 (みずうみ)
\\	測			
\\	ソク	はか.る	推測(すいそく): 
\\	測る(はかる): 
\\	観測(かんそく): 
\\	測定(そくてい): 
\\	測量(そくりょう): 
\\	予測(よそく): 
\\	測る (はか.る)
\\	土			
\\	ド、ト	つち	国土(こくど): 
\\	土産(どさん): 
\\	土台(どだい): 
\\	土手(どて): 
\\	土俵(どひょう): 
\\	土木(どぼく): 
\\	土曜(どよう): 
\\	風土(ふうど): 
\\	領土(りょうど): 
\\	土(つち): 
\\	土地(とち): 
\\	土産(みやげ): 
\\	お土産(おみやげ): 
\\	土曜日(どようび): 
\\	土木 (どぼく), 国土 (こくど), 粘土 (ねんど), 土 (つち)
\\	吐			
\\	ト	は.く、つ.く	吐く(つく): 
\\	吐き気(はきけ): 
\\	吐く(はく): 
\\	吐く (は.く)
\\	圧			
\\	アツ、エン、オウ	お.す、へ.す、おさ.える、お.さえる	圧迫(あっぱく): 
\\	圧力(あつりょく): 
\\	抑圧(よくあつ): 
\\	圧縮(あっしゅく): 
\\	気圧(きあつ): 
\\	血圧(けつあつ): 
\\	埼			
\\	キ	さき、さい、みさき		埼 (さい)
\\	垣			
\\	エン	かき	垣根(かきね): 
\\	垣 (かき)
\\	圭			
\\	ケイ、ケ			
\\	封			
\\	フウ、ホウ		同封(どうふう): 
\\	封(ふう): 
\\	封鎖(ふうさ): 
\\	封建(ほうけん): 
\\	封筒(ふうとう): 
\\	封鎖 (ふうさ), 封書 (ふうしょ), 密封 (みっぷう)
\\	涯		
\\	ガイ	はて		
\\	寺			
\\	ジ	てら	寺院(じいん): 
\\	寺(てら): 
\\	寺 (てら)
\\	時			
\\	ジ	とき、-どき	何時(いつ): 
\\	何時か(いつか): 
\\	何時でも(いつでも): 
\\	何時の間にか(いつのまにか): 
\\	何時も(いつも): 
\\	時刻表(じこくひょう): 
\\	時差(じさ): 
\\	時折(ときおり): 
\\	一時(いちじ): 
\\	時間割(じかんわり): 
\\	時期(じき): 
\\	時刻(じこく): 
\\	時速(じそく): 
\\	当時(とうじ): 
\\	同時(どうじ): 
\\	時(とき): 
\\	日時(にちじ): 
\\	臨時(りんじ): 
\\	時代(じだい): 
\\	時間(じかん): 
\\	時々(ときどき): 
\\	時計(とけい): 
\\	時 (とき)
\\	均			
\\	キン	なら.す	均衡(きんこう): 
\\	平均(ならし): 
\\	平均(へいきん): 
\\	火			
\\	カ	ひ、-び、ほ-	火傷(かしょう): 
\\	火星(かせい): 
\\	火曜(かよう): 
\\	火燵(こたつ): 
\\	下火(したび): 
\\	焚火(たきび): 
\\	点火(てんか): 
\\	火花(ひばな): 
\\	防火(ぼうか): 
\\	火口(かこう): 
\\	火災(かさい): 
\\	火山(かざん): 
\\	花火(はなび): 
\\	噴火(ふんか): 
\\	火傷(やけど): 
\\	火事(かじ): 
\\	火(ひ): 
\\	火曜日(かようび): 
\\	火 (ひ)
\\	炎			
\\	エン	ほのお	炎(ほのお): 
\\	炎 (ほのお)
\\	煩			
\\	ハン、ボン	わずら.う、わずら.わす	煩わしい(わずらわしい): 
\\	煩雑 (はんざつ), 煩う (わずら.う), 煩わす (わずら.わす)
\\	淡			
\\	タン	あわ.い	冷淡(れいたん): 
\\	淡水(たんすい): 
\\	淡い (あわ.い)
\\	灯			
\\	トウ	ひ、ほ-、ともしび、とも.す、あかり	灯(ともしび): 
\\	蛍光灯(けいこうとう): 
\\	灯台(とうだい): 
\\	灯油(とうゆ): 
\\	灯(ひ): 
\\	電灯(でんとう): 
\\	灯 (ひ)
\\	畑			
\\	はた、はたけ、-ばたけ	畑(はたけ): 
\\	畑 (はた), 畑 (はたけ)
\\	災			
\\	サイ	わざわ.い	災害(さいがい): 
\\	戦災(せんさい): 
\\	天災(てんさい): 
\\	火災(かさい): 
\\	災難(さいなん): 
\\	災い (わざわ.い)
\\	灰			
\\	カイ	はい	灰(あく): 
\\	灰皿(はいさら): 
\\	灰(はい): 
\\	灰色(はいいろ): 
\\	灰皿(はいざら): 
\\	灰 (はい)
\\	点			
\\	テン	つ.ける、つ.く、た.てる、さ.す、とぼ.す、とも.す、ぼち	観点(かんてん): 
\\	起点(きてん): 
\\	原点(げんてん): 
\\	減点(げんてん): 
\\	視点(してん): 
\\	点く(つく): 
\\	点火(てんか): 
\\	点検(てんけん): 
\\	点線(てんせん): 
\\	得点(とくてん): 
\\	盲点(もうてん): 
\\	利点(りてん): 
\\	句読点(くとうてん): 
\\	欠点(けってん): 
\\	交差点(こうさてん): 
\\	採点(さいてん): 
\\	弱点(じゃくてん): 
\\	終点(しゅうてん): 
\\	重点(じゅうてん): 
\\	焦点(しょうてん): 
\\	地点(ちてん): 
\\	頂点(ちょうてん): 
\\	点ける(つける): 
\\	点数(てんすう): 
\\	点々(てんてん): 
\\	満点(まんてん): 
\\	要点(ようてん): 
\\	零点(れいてん): 
\\	点(てん): 
\\	照			
\\	ショウ	て.る、て.らす、て.れる	参照(さんしょう): 
\\	照合(しょうごう): 
\\	照明(しょうめい): 
\\	照り返す(てりかえす): 
\\	対照(たいしょう): 
\\	照らす(てらす): 
\\	照る(てる): 
\\	照らす (て.らす), 照る (て.る), 照れる (て.れる)
\\	魚			
\\	ギョ	うお、さかな、-ざかな	魚(うお): 
\\	金魚(きんぎょ): 
\\	魚(さかな): 
\\	魚 (うお), 魚 (さかな)
\\	漁			
\\	ギョ、リョウ	あさ.る	漁船(ぎょせん): 
\\	漁村(ぎょそん): 
\\	漁業(ぎょぎょう): 
\\	漁師(りょうし): 
\\	漁業 (ぎょぎょう), 漁船 (ぎょせん), 漁村 (ぎょそん)
\\	里			
\\	リ	さと	郷里(きょうり): 
\\	里 (さと)
\\	黒			
\\	コク	くろ、くろ.ずむ、くろ.い	黒字(くろじ): 
\\	黒(くろ): 
\\	黒板(こくばん): 
\\	真っ黒(まっくろ): 
\\	黒い(くろい): 
\\	黒 (くろ), 黒い (くろ.い)
\\	墨			
\\	ボク	すみ	墨(すみ): 
\\	墨 (すみ)
\\	鯉			
\\	リ	こい		
\\	量			
\\	リョウ	はか.る	感無量(かんむりょう): 
\\	熱量(ねつりょう): 
\\	量る(はかる): 
\\	微量(びりょう): 
\\	重量(じゅうりょう): 
\\	測量(そくりょう): 
\\	分量(ぶんりょう): 
\\	量(りょう): 
\\	量る (はか.る)
\\	厘			
\\	リン			
\\	埋			
\\	マイ	う.める、う.まる、う.もれる、うず.める、うず.まる、い.ける	埋まる(うずまる): 
\\	埋め込む(うめこむ): 
\\	埋蔵(まいぞう): 
\\	埋める(うめる): 
\\	埋まる (う.まる), 埋める (う.める), 埋もれる (う.もれる)
\\	同			
\\	ドウ	おな.じ	一同(いちどう): 
\\	同い年(おないどし): 
\\	混同(こんどう): 
\\	同(どう): 
\\	同意(どうい): 
\\	同感(どうかん): 
\\	同級(どうきゅう): 
\\	同居(どうきょ): 
\\	同志(どうし): 
\\	同士(どうし): 
\\	同情(どうじょう): 
\\	同調(どうちょう): 
\\	同等(どうとう): 
\\	同封(どうふう): 
\\	同盟(どうめい): 
\\	共同(きょうどう): 
\\	合同(ごうどう): 
\\	同一(どういつ): 
\\	同格(どうかく): 
\\	同時(どうじ): 
\\	同様(どうよう): 
\\	同僚(どうりょう): 
\\	同じ(おなじ): 
\\	同じ (おな.じ)
\\	洞			
\\	ドウ	ほら		洞 (ほら)
\\	胴			
\\	ドウ		胴(どう): 
\\	向			
\\	くち// 
\\	コウ	む.く、む.い、-む.き、む.ける、-む.け、む.かう、む.かい、む.こう、む.こう-、むこ、むか.い	意向(いこう): 
\\	向上(こうじょう): 
\\	志向(しこう): 
\\	動向(どうこう): 
\\	一向(ひたすら): 
\\	日向(ひなた): 
\\	向かう(むかう): 
\\	向き(むき): 
\\	向け(むけ): 
\\	傾向(けいこう): 
\\	方向(ほうこう): 
\\	向かい(むかい): 
\\	向う(むかう): 
\\	向く(むく): 
\\	向ける(むける): 
\\	向こう(むこう): 
\\	向かう (む.かう), 向く (む.く), 向ける (む.ける), 向こう (む.こう)
\\	尚			
\\	ショウ	なお	高尚(こうしょう): 
\\	尚(なお): 
\\	尚更(なおさら): 
\\	字			
\\	ジ	あざ、あざな、-な	赤字(あかじ): 
\\	字(あざ): 
\\	当て字(あてじ): 
\\	英字(えいじ): 
\\	黒字(くろじ): 
\\	字体(じたい): 
\\	十字路(じゅうじろ): 
\\	文字(もじ): 
\\	ローマ字(ローマじ): 
\\	活字(かつじ): 
\\	習字(しゅうじ): 
\\	数字(すうじ): 
\\	名字(みょうじ): 
\\	ローマ字(ローマじ): 
\\	字(じ): 
\\	漢字(かんじ): 
\\	字引(じびき): 
\\	字 (あざ)
\\	守			
\\	シュ、ス	まも.る、まも.り、もり、-もり、かみ	守衛(しゅえい): 
\\	守備(しゅび): 
\\	保守(ほしゅ): 
\\	守る(まもる): 
\\	留守番(るすばん): 
\\	留守(るす): 
\\	守備 (しゅび), 保守 (ほしゅ), 攻守 (こうしゅ), 守る (まも.る), 守 (もり)
\\	完			
\\	カン		完璧(かんぺき): 
\\	完成(かんせい): 
\\	完全(かんぜん): 
\\	完了(かんりょう): 
\\	宣			
\\	セン	のたま.う、のたま.わく	宣教(せんきょう): 
\\	宣言(せんげん): 
\\	宣伝(せんでん): 
\\	宵			
\\	宵の口(よいのくち)
\\	今宵(こよい)
\\	ショウ	よい		宵 (よい)
\\	安			
\\	家で休暇を取るは安いですね。.	
\\	アン	やす.い、やす.まる、やす、やす.らか	安静(あんせい): 
\\	治安(ちあん): 
\\	安っぽい(やすっぽい): 
\\	安易(あんい): 
\\	安定(あんてい): 
\\	不安(ふあん): 
\\	目安(めやす): 
\\	安心(あんしん): 
\\	安全(あんぜん): 
\\	安い(やすい): 
\\	安い (やす.い)
\\	宴			
\\	エン	うたげ	宴会(えんかい): 
\\	寄			
\\	キ	よ.る、-よ.り、よ.せる	押し寄せる(おしよせる): 
\\	寄贈(きそう): 
\\	寄与(きよ): 
\\	立ち寄る(たちよる): 
\\	年寄り(としより): 
\\	取り寄せる(とりよせる): 
\\	寄こす(よこす): 
\\	寄り掛かる(よりかかる): 
\\	片寄る(かたよる): 
\\	寄付(きふ): 
\\	近寄る(ちかよる): 
\\	寄せる(よせる): 
\\	寄る(よる): 
\\	寄せる (よ.せる), 寄る (よ.る)
\\	富			
\\	フ、フウ	と.む、とみ	富(とみ): 
\\	富む(とむ): 
\\	富豪(ふごう): 
\\	豊富(ほうふ): 
\\	富強 (ふきょう), 富裕 (ふゆう), 貧富 (ひんぷ), 富む (と.む), 富 (とみ)
\\	貯			
\\	チョ	た.める、たくわ.える	貯蓄(ちょちく): 
\\	貯金(ちょきん): 
\\	貯蔵(ちょぞう): 
\\	木			
\\	ボク、モク	き、こ-	木綿(きわた): 
\\	雑木(ざつぼく): 
\\	樹木(じゅもく): 
\\	土木(どぼく): 
\\	木曜(もくよう): 
\\	植木(うえき): 
\\	材木(ざいもく): 
\\	大木(たいぼく): 
\\	並木(なみき): 
\\	木材(もくざい): 
\\	木綿(もめん): 
\\	木(き): 
\\	木曜日(もくようび): 
\\	木石 (ぼくせき), 大木 (たいぼく), 土木 (どぼく), 木 (き)
\\	林			
\\	リン 森林 しんりん
\\	リン	はやし	林業(りんぎょう): 
\\	山林(さんりん): 
\\	森林(しんりん): 
\\	林(はやし): 
\\	林 (はやし)
\\	森			
\\	シン	もり	森林(しんりん): 
\\	森(もり): 
\\	森 (もり)
\\	桂			
\\	ケイ	かつら		
\\	柏			
\\	ハク、ヒャク、ビャク	かしわ		
\\	枠			
\\	わく	枠(わく): 
\\	枠 (わく)
\\	梢			
\\	ショウ	こずえ、くすのき	梢(こずえ): 
\\	棚			
\\	ホウ	たな、-だな	戸棚(とだな): 
\\	棚(たな): 
\\	本棚(ほんだな): 
\\	棚 (たな)
\\	杏			
\\	キョウ、アン、コウ	あんず		
\\	桐			
\\	(きり)
\\	トウ、ドウ	きり	桐(きり): 
\\	植			
\\	ショク	う.える、う.わる	植わる(うわる): 
\\	植民地(しょくみんち): 
\\	植木(うえき): 
\\	植物(しょくぶつ): 
\\	田植え(たうえ): 
\\	植える(うえる): 
\\	植える (う.える), 植わる (う.わる)
\\	枯			
\\	コ	か.れる、か.らす	枯れる(かれる): 
\\	枯らす (か.らす), 枯れる (か.れる)
\\	朴			
\\	ボク	ほう、ほお、えのき	素朴(そぼく): 
\\	村			
\\	ソン	むら	漁村(ぎょそん): 
\\	農村(のうそん): 
\\	村(むら): 
\\	村 (むら)
\\	相			
\\	"ソウ/ショウ/あ 
\\	(首, 
\\	「首相 (しゅしょう)
\\	ソウ、ショウ	あい-	相(あい): 
\\	相変わらず(あいかわらず): 
\\	相対(あいたい): 
\\	外相(がいしょう): 
\\	真相(しんそう): 
\\	相応(そうおう): 
\\	蔵相(ぞうしょう): 
\\	相談(そうだん): 
\\	相場(そうば): 
\\	相応しい(ふさわしい): 
\\	様相(ようそう): 
\\	相手(あいて): 
\\	首相(しゅしょう): 
\\	相撲(すもう): 
\\	相違(そうい): 
\\	相互(そうご): 
\\	相続(そうぞく): 
\\	相当(そうとう): 
\\	相当 (そうとう), 相談 (そうだん), 真相 (しんそう), 相 (あい)
\\	机			
\\	キ	つくえ	机(つくえ): 
\\	机 (つくえ)
\\	本			
\\	ホン	もと	脚本(きゃくほん): 
\\	根本(こんぽん): 
\\	台本(だいほん): 
\\	手本(てほん): 
\\	本格(ほんかく): 
\\	本館(ほんかん): 
\\	本気(ほんき): 
\\	本質(ほんしつ): 
\\	本体(ほんたい): 
\\	本当(ほんとう): 
\\	本音(ほんね): 
\\	本の(ほんの): 
\\	本能(ほんのう): 
\\	本場(ほんば): 
\\	本文(ほんぶん): 
\\	本名(ほんみょう): 
\\	基本(きほん): 
\\	資本(しほん): 
\\	標本(ひょうほん): 
\\	本人(ほんにん): 
\\	本部(ほんぶ): 
\\	本物(ほんもの): 
\\	本来(ほんらい): 
\\	見本(みほん): 
\\	本(ほん): 
\\	本棚(ほんだな): 
\\	本 (もと)
\\	札			
\\	さつ (千円札=1000 
\\	ふだ 
\\	花札 
\\	トランプの札 
\\	サツ	ふだ	名札(なふだ): 
\\	改札(かいさつ): 
\\	札(さつ): 
\\	札 (ふだ)
\\	暦			
\\	レキ	こよみ、りゃく	還暦(かんれき): 
\\	暦(こよみ): 
\\	西暦(せいれき): 
\\	暦 (こよみ)
\\	案			
\\	あん!	
\\	アン	つくえ	案じる(あんじる): 
\\	案の定(あんのじょう): 
\\	議案(ぎあん): 
\\	法案(ほうあん): 
\\	案(あん): 
\\	案外(あんがい): 
\\	案内(あんない): 
\\	提案(ていあん): 
\\	答案(とうあん): 
\\	燥			
\\	ソウ	はしゃ.ぐ	乾燥(かんそう): 
\\	未			
\\	(木) 
\\	(未) 
\\	(末) 
\\	(朱).	
\\	(木) 
\\	(末) 
\\	(朱).	
\\	ミ、ビ	いま.だ、ま.だ、ひつじ	未だ(いまだ): 
\\	未(ひつじ): 
\\	未開(みかい): 
\\	未婚(みこん): 
\\	未熟(みじゅく): 
\\	未知(みち): 
\\	未定(みてい): 
\\	未練(みれん): 
\\	未だ(まだ): 
\\	未満(みまん): 
\\	未来(みらい): 
\\	末			
\\	マツ、バツ	すえ	末(うら): 
\\	期末(きまつ): 
\\	始末(しまつ): 
\\	粉末(ふんまつ): 
\\	末期(まっき): 
\\	月末(げつまつ): 
\\	末(すえ): 
\\	末っ子(すえっこ): 
\\	粗末(そまつ): 
\\	末代 (まつだい), 本末 (ほんまつ), 粉末 (ふんまつ), 末 (すえ)
\\	沫			
\\	マツ、バツ	あわ、しぶき、つばき		
\\	味			
\\	ミ	あじ、あじ.わう	味わい(あじわい): 
\\	加味(かみ): 
\\	吟味(ぎんみ): 
\\	三味線(さみせん): 
\\	中味(なかみ): 
\\	不味い(まずい): 
\\	味覚(みかく): 
\\	無意味(むいみ): 
\\	味わう(あじわう): 
\\	気味(きみ): 
\\	地味(じみ): 
\\	正味(しょうみ): 
\\	調味料(ちょうみりょう): 
\\	味方(みかた): 
\\	味噌(みそ): 
\\	味(あじ): 
\\	興味(きょうみ): 
\\	趣味(しゅみ): 
\\	意味(いみ): 
\\	味 (あじ), 味わう (あじ.わう)
\\	妹			
\\	マイ	いもうと	姉妹(きょうだい): 
\\	従姉妹(いとこ): 
\\	姉妹(しまい): 
\\	妹(いもうと): 
\\	妹 (いもうと)
\\	朱			
\\	(木) 
\\	(未)
\\	(朱).	
\\	シュ	あけ		
\\	株			
\\	シュ	かぶ	株式(かぶしき): 
\\	株(かぶ): 
\\	株 (かぶ)
\\	若			
\\	ニャク 老若 ろうにゃく 
\\	ジャク 若干 じゃっかん
\\	ジャク、ニャク、ニャ	わか.い、わか-、も.しくわ、も.し、も.しくは	若干(じゃっかん): 
\\	若し(もし): 
\\	若しかして(もしかして): 
\\	若しくは(もしくは): 
\\	若しも(もしも): 
\\	若々しい(わかわかしい): 
\\	若い(わかい): 
\\	若年 (じゃくねん), 若干 (じゃっかん), 自若 (じじゃく), 若しくは (も.しくは), 若い (わか.い)
\\	草			
\\	ソウ	くさ、くさ-、-ぐさ	草臥れる(くたびれる): 
\\	煙草(たばこ): 
\\	草履(ぞうり): 
\\	草(くさ): 
\\	草 (くさ)
\\	苦			
\\	ク	くる.しい、-ぐる.しい、くる.しむ、くる.しめる、にが.い、にが.る	苦しめる(くるしめる): 
\\	ご苦労様(ごくろうさま): 
\\	見苦しい(みぐるしい): 
\\	無茶苦茶(むちゃくちゃ): 
\\	滅茶苦茶(めちゃくちゃ): 
\\	苦情(くじょう): 
\\	苦心(くしん): 
\\	苦痛(くつう): 
\\	苦しい(くるしい): 
\\	苦しむ(くるしむ): 
\\	苦労(くろう): 
\\	苦手(にがて): 
\\	苦い(にがい): 
\\	苦しい (くる.しい), 苦しむ (くる.しむ), 苦しめる (くる.しめる), 苦い (にが.い), 苦る (にが.る)
\\	寛			
\\	カン	くつろ.ぐ、ひろ.い、ゆる.やか	寛容(かんよう): 
\\	薄			
\\	(落 薄 藩 藻) 
\\	満 
\\	満, 
\\	ハク	うす.い、うす-、-うす、うす.める、うす.まる、うす.らぐ、うす.ら-、うす.れる、すすき	薄弱(はくじゃく): 
\\	薄暗い(うすぐらい): 
\\	薄める(うすめる): 
\\	薄い(うすい): 
\\	薄い (うす.い), 薄まる (うす.まる), 薄める (うす.める), 薄らく (うす.らく), 薄れる (うす.れる)
\\	葉			
\\	ヨウ	は	落ち葉(おちば): 
\\	紅葉(こうよう): 
\\	言葉遣い(ことばづかい): 
\\	紅葉(もみじ): 
\\	葉(は): 
\\	言葉(ことば): 
\\	葉書(はがき): 
\\	葉 (は)
\\	模			
\\	モ、ボ		規模(きぼ): 
\\	模型(もけい): 
\\	模索(もさく): 
\\	模範(もはん): 
\\	模倣(もほう): 
\\	模様(もよう): 
\\	模範 (もはん), 模型 (もけい), 模倣 (もほう)
\\	漠			
\\	バク		漠然(ばくぜん): 
\\	砂漠(さばく): 
\\	墓			
\\	ボ	はか	墓地(はかち): 
\\	墓(はか): 
\\	墓 (はか)
\\	暮			
\\	ボ	く.れる、く.らす	夕暮れ(ゆうぐれ): 
\\	暮らし(くらし): 
\\	暮らす(くらす): 
\\	暮れ(くれ): 
\\	暮れる(くれる): 
\\	暮らす (く.らす), 暮れる (く.れる)
\\	膜			
\\	マク		膜(まく): 
\\	苗			
\\	ビョウ、ミョウ	なえ、なわ-	苗(なえ): 
\\	苗 (なえ)
\\	兆			
\\	チョウ	きざ.す、きざ.し	兆し(きざし): 
\\	兆し (きざ.し), 兆す (きざ.す)
\\	桃			
\\	トウ	もも		桃 (もも)
\\	眺			
\\	チョウ	なが.める	眺め(ながめ): 
\\	眺める(ながめる): 
\\	眺める (なが.める)
\\	犬			
\\	犭 
\\	ケン	いぬ、いぬ-	犬(いぬ): 
\\	犬 (いぬ)
\\	状			
\\	ジョウ		状(じょう): 
\\	白状(はくじょう): 
\\	現状(げんじょう): 
\\	状況(じょうきょう): 
\\	症状(しょうじょう): 
\\	状態(じょうたい): 
\\	黙			
\\	モク、ボク	だま.る、もだ.す	沈黙(ちんもく): 
\\	黙る(だまる): 
\\	黙る (だま.る)
\\	然			
\\	月 
\\	ゼン、ネン	しか、しか.り、しか.し、さ	依然(いぜん): 
\\	公然(こうぜん): 
\\	然も(さも): 
\\	然しながら(しかしながら): 
\\	整然(せいぜん): 
\\	全然(ぜんぜん): 
\\	然して(そして): 
\\	断然(だんぜん): 
\\	漠然(ばくぜん): 
\\	必然(ひつぜん): 
\\	呆然(ぼうぜん): 
\\	偶然(ぐうぜん): 
\\	自然(しぜん): 
\\	自然科学(しぜんかがく): 
\\	天然(てんねん): 
\\	突然(とつぜん): 
\\	当然 (とうぜん), 自然 (しぜん), 必然 (ひつぜん)
\\	荻			
\\	テキ	おぎ		
\\	狩			
\\	シュ	か.る、か.り、-が.り		狩り (か.り), 狩る (か.る)
\\	猫			
\\	ビョウ	ねこ	猫(ねこ): 
\\	猫 (ねこ)
\\	牛			
\\	ギュウ	うし	牛(うし): 
\\	牛乳(ぎゅうにゅう): 
\\	牛肉(ぎゅうにく): 
\\	牛 (うし)
\\	特			
\\	トク		特技(とくぎ): 
\\	特産(とくさん): 
\\	特集(とくしゅう): 
\\	特派(とくは): 
\\	特有(とくゆう): 
\\	特許(とっきょ): 
\\	特権(とっけん): 
\\	特殊(とくしゅ): 
\\	特色(とくしょく): 
\\	特長(とくちょう): 
\\	特徴(とくちょう): 
\\	特定(とくてい): 
\\	独特(どくとく): 
\\	特売(とくばい): 
\\	特急(とっきゅう): 
\\	特に(とくに): 
\\	特別(とくべつ): 
\\	告			
\\	コク	つ.げる	勧告(かんこく): 
\\	告白(こくはく): 
\\	申告(しんこく): 
\\	忠告(ちゅうこく): 
\\	告げる(つげる): 
\\	布告(ふこく): 
\\	警告(けいこく): 
\\	広告(こうこく): 
\\	報告(ほうこく): 
\\	告げる (つ.げる)
\\	先			
\\	セン	さき、ま.ず	お先に(おさきに): 
\\	先に(さきに): 
\\	先行(せんこう): 
\\	先代(せんだい): 
\\	先だって(せんだって): 
\\	先着(せんちゃく): 
\\	先天的(せんてんてき): 
\\	勤め先(つとめさき): 
\\	優先(ゆうせん): 
\\	先程(さきほど): 
\\	先日(せんじつ): 
\\	先々月(せんせんげつ): 
\\	先々週(せんせんしゅう): 
\\	先祖(せんぞ): 
\\	先端(せんたん): 
\\	先頭(せんとう): 
\\	祖先(そせん): 
\\	先ず(まず): 
\\	真っ先(まっさき): 
\\	先輩(せんぱい): 
\\	先(さき): 
\\	先月(せんげつ): 
\\	先週(せんしゅう): 
\\	先生(せんせい): 
\\	先 (さき)
\\	洗			
\\	セン	あら.う	水洗(すいせん): 
\\	洗剤(せんざい): 
\\	手洗い(てあらい): 
\\	洗う(あらう): 
\\	お手洗い(おてあらい): 
\\	洗濯(せんたく): 
\\	洗う (あら.う)
\\	介			
\\	カイ		介護(かいご): 
\\	介入(かいにゅう): 
\\	介抱(かいほう): 
\\	厄介(やっかい): 
\\	紹介(しょうかい): 
\\	界			
\\	カイ		境界(きょうかい): 
\\	限界(げんかい): 
\\	世界(せかい): 
\\	茶			
\\	ホ).
\\	チャ、サ		喫茶(きっさ): 
\\	焦げ茶(こげちゃ): 
\\	茶の間(ちゃのま): 
\\	茶の湯(ちゃのゆ): 
\\	茶碗(ちゃわん): 
\\	無茶(むちゃ): 
\\	無茶苦茶(むちゃくちゃ): 
\\	滅茶苦茶(めちゃくちゃ): 
\\	紅茶(こうちゃ): 
\\	茶(ちゃ): 
\\	茶色い(ちゃいろい): 
\\	お茶(おちゃ): 
\\	喫茶店(きっさてん): 
\\	茶色(ちゃいろ): 
\\	茶色 (ちゃいろ), 茶番劇 (ちゃばんげき), 番茶 (ばんちゃ)
\\	合			
\\	ゴウ、ガッ、カッ	あ.う、-あ.う、あ.い、あい-、-あ.い、-あい、あ.わす、あ.わせる、-あ.わせる	合間(あいま): 
\\	合わす(あわす): 
\\	合わせ(あわせ): 
\\	打ち合わせ(うちあわせ): 
\\	打ち合わせる(うちあわせる): 
\\	化合(かごう): 
\\	合唱(がっしょう): 
\\	合致(がっち): 
\\	合併(がっぺい): 
\\	組み合わせ(くみあわせ): 
\\	組み合わせる(くみあわせる): 
\\	結合(けつごう): 
\\	合意(ごうい): 
\\	合議(ごうぎ): 
\\	合成(ごうせい): 
\\	照合(しょうごう): 
\\	知り合い(しりあい): 
\\	総合(そうごう): 
\\	付き合う(つきあう): 
\\	出合う(であう): 
\\	問い合わせる(といあわせる): 
\\	統合(とうごう): 
\\	話し合い(はなしあい): 
\\	複合(ふくごう): 
\\	待ち合わせ(まちあわせ): 
\\	間に合う(まにあう): 
\\	見合い(みあい): 
\\	見合わせる(みあわせる): 
\\	割合(わりあい): 
\\	合図(あいず): 
\\	合わせる(あわせる): 
\\	打合せ(うちあわせ): 
\\	会合(かいごう): 
\\	組合(くみあい): 
\\	合格(ごうかく): 
\\	合計(ごうけい): 
\\	合同(ごうどう): 
\\	合理(ごうり): 
\\	合流(ごうりゅう): 
\\	混合(こんごう): 
\\	集合(しゅうごう): 
\\	付き合い(つきあい): 
\\	付合う(つきあう): 
\\	釣り合う(つりあう): 
\\	出合い(であい): 
\\	問い合わせ(といあわせ): 
\\	似合う(にあう): 
\\	話し合う(はなしあう): 
\\	待合室(まちあいしつ): 
\\	待ち合わせる(まちあわせる): 
\\	連合(れんごう): 
\\	合う(あう): 
\\	具合(ぐあい): 
\\	試合(しあい): 
\\	都合(つごう): 
\\	場合(ばあい): 
\\	割合に(わりあいに): 
\\	合同 (ごうどう), 合計 (ごうけい), 結合 (けつごう), 合併 (がっぺい), 合宿 (がっしゅく), 合点 (がてん), 合う (あ.う), 合わす (あ.わす), 合わせる (あ.わせる)
\\	塔			
\\	トウ		塔(とう): 
\\	王			
\\	オウ、ノウ		王(おう): 
\\	王様(おうさま): 
\\	王子(おうじ): 
\\	王女(おうじょ): 
\\	国王(こくおう): 
\\	女王(じょおう): 
\\	玉			
\\	ギョク	たま、たま-、-だま	玉(ぎょく): 
\\	玉(たま): 
\\	玉 (たま)
\\	宝			
\\	ホウ	たから	重宝(じゅうほう): 
\\	宝器(ほうき): 
\\	宝(たから): 
\\	宝石(ほうせき): 
\\	宝 (たから)
\\	珠			
\\	シュ	たま	真珠(しんじゅ): 
\\	現			
\\	ゲン	あらわ.れる、あらわ.す、うつつ、うつ.つ	現われ(あらわれ): 
\\	現われる(あらわれる): 
\\	現行(げんこう): 
\\	現場(げんじょう): 
\\	現像(げんぞう): 
\\	現地(げんち): 
\\	再現(さいげん): 
\\	現す(あらわす): 
\\	現れ(あらわれ): 
\\	現れる(あらわれる): 
\\	現金(げんきん): 
\\	現在(げんざい): 
\\	現実(げんじつ): 
\\	現象(げんしょう): 
\\	現状(げんじょう): 
\\	現代(げんだい): 
\\	現に(げんに): 
\\	現場(げんば): 
\\	実現(じつげん): 
\\	表現(ひょうげん): 
\\	現す (あらわ.す), 現れる (あらわ.れる)
\\	狂			
\\	キョウ	くる.う、くる.おしい、くるお.しい	狂う(くるう): 
\\	狂う (くる.う), 狂おしい (くる.おしい)
\\	皇			
\\	コウ、オウ		皇居(こうきょ): 
\\	天皇(すめらぎ): 
\\	天皇(てんのう): 
\\	皇帝 (こうてい), 皇室 (こうしつ), 皇后 (こうごう)
\\	呈			
\\	テイ		進呈(しんてい): 
\\	全			
\\	ゼン	まった.く、すべ.て	健全(けんぜん): 
\\	全快(ぜんかい): 
\\	全盛(ぜんせい): 
\\	全然(ぜんぜん): 
\\	全滅(ぜんめつ): 
\\	完全(かんぜん): 
\\	全て(すべて): 
\\	全員(ぜんいん): 
\\	全(ぜん): 
\\	全国(ぜんこく): 
\\	全集(ぜんしゅう): 
\\	全身(ぜんしん): 
\\	全体(ぜんたい): 
\\	全般(ぜんぱん): 
\\	全く(まったく): 
\\	安全(あんぜん): 
\\	全部(ぜんぶ): 
\\	全く (まった.く)
\\	栓			
\\	セン		栓(せん): 
\\	理			
\\	リ	ことわり	義理(ぎり): 
\\	真理(しんり): 
\\	推理(すいり): 
\\	生理(せいり): 
\\	調理(ちょうり): 
\\	理屈(りくつ): 
\\	理性(りせい): 
\\	理論(りろん): 
\\	論理(ろんり): 
\\	管理(かんり): 
\\	原理(げんり): 
\\	合理(ごうり): 
\\	修理(しゅうり): 
\\	処理(しょり): 
\\	心理(しんり): 
\\	整理(せいり): 
\\	総理大臣(そうりだいじん): 
\\	代理(だいり): 
\\	物理(ぶつり): 
\\	理科(りか): 
\\	理解(りかい): 
\\	理想(りそう): 
\\	地理(ちり): 
\\	無理(むり): 
\\	理由(りゆう): 
\\	料理(りょうり): 
\\	主			
\\	シュ、ス、シュウ	ぬし、おも、あるじ	家主(いえぬし): 
\\	君主(くんしゅ): 
\\	自主(じしゅ): 
\\	地主(じぬし): 
\\	主演(しゅえん): 
\\	主観(しゅかん): 
\\	主義(しゅぎ): 
\\	主権(しゅけん): 
\\	主催(しゅさい): 
\\	主食(しゅしょく): 
\\	主人公(しゅじんこう): 
\\	主体(しゅたい): 
\\	主題(しゅだい): 
\\	主導(しゅどう): 
\\	主任(しゅにん): 
\\	民主(みんしゅ): 
\\	主に(おもに): 
\\	主語(しゅご): 
\\	主張(しゅちょう): 
\\	主婦(しゅふ): 
\\	主役(しゅやく): 
\\	主要(しゅよう): 
\\	家主(やぬし): 
\\	御主人(ごしゅじん): 
\\	主人 (しゅじん), 主権 (しゅけん), 施主 (せしゅ), 主 (おも), 主 (ぬし)
\\	注			
\\	チュウ	そそ.ぐ、さ.す、つ.ぐ	注す(さす): 
\\	注文(ちゅうもん): 
\\	注ぐ(そそぐ): 
\\	注(ちゅう): 
\\	注目(ちゅうもく): 
\\	注ぐ(つぐ): 
\\	注意(ちゅうい): 
\\	注射(ちゅうしゃ): 
\\	注ぐ (そそ.ぐ)
\\	柱			
\\	チュウ	はしら	柱(はしら): 
\\	電柱(でんちゅう): 
\\	柱 (はしら)
\\	金			
\\	キン、コン、ゴン	かね、かな-、-がね	黄金(おうごん): 
\\	金槌(かなづち): 
\\	金庫(かねぐら): 
\\	金持ち(かねもち): 
\\	基金(ききん): 
\\	金曜(きんよう): 
\\	残金(ざんきん): 
\\	資金(しきん): 
\\	送金(そうきん): 
\\	大金(たいきん): 
\\	賃金(ちんぎん): 
\\	募金(ぼきん): 
\\	預金(よきん): 
\\	金(かね): 
\\	金融(きんゆう): 
\\	金(きん): 
\\	金額(きんがく): 
\\	金魚(きんぎょ): 
\\	金庫(きんこ): 
\\	金銭(きんせん): 
\\	金属(きんぞく): 
\\	現金(げんきん): 
\\	借金(しゃっきん): 
\\	集金(しゅうきん): 
\\	奨学金(しょうがくきん): 
\\	賞金(しょうきん): 
\\	税金(ぜいきん): 
\\	代金(だいきん): 
\\	貯金(ちょきん): 
\\	針金(はりがね): 
\\	料金(りょうきん): 
\\	お金(おかね): 
\\	金曜日(きんようび): 
\\	金属 (きんぞく), 金銭 (きんせん), 純金 (じゅんきん), 金 (かね)
\\	銑			
\\	(金) 
\\	(先) 
\\	セン			
\\	鉢			
\\	ハチ、ハツ		鉢(はち): 
\\	鉢 (はち), 植木鉢 (うえきばち)
\\	銅			
\\	ドウ	あかがね	銅(あかがね): 
\\	銅(どう): 
\\	釣			
\\	チョウ	つ.る、つ.り、つ.り-	釣り(つり): 
\\	釣鐘(つりがね): 
\\	釣(つり): 
\\	釣(つり): 
\\	釣り合う(つりあう): 
\\	釣る(つる): 
\\	釣る (つ.る)
\\	針			
\\	シン	はり	針路(しんろ): 
\\	針(はり): 
\\	針金(はりがね): 
\\	方針(ほうしん): 
\\	針 (はり)
\\	銘			
\\	メイ		銘々(めいめい): 
\\	鎮			
\\	チン	しず.める、しず.まる、おさえ		鎮まる (しず.まる), 鎮める (しず.める)
\\	道			
\\	ドウ、トウ	みち	街道(かいどう): 
\\	軌道(きどう): 
\\	使い道(つかいみち): 
\\	道場(どうじょう): 
\\	報道(ほうどう): 
\\	片道(かたみち): 
\\	書道(しょどう): 
\\	赤道(せきどう): 
\\	鉄道(てつどう): 
\\	道徳(どうとく): 
\\	道路(どうろ): 
\\	歩道(ほどう): 
\\	回り道(まわりみち): 
\\	道順(みちじゅん): 
\\	柔道(じゅうどう): 
\\	水道(すいどう): 
\\	道具(どうぐ): 
\\	道(みち): 
\\	道路 (どうろ), 道徳 (どうとく), 報道 (ほうどう), 道 (みち)
\\	導			
\\	ドウ	みちび.く	主導(しゅどう): 
\\	導入(どうにゅう): 
\\	導く(みちびく): 
\\	誘導(ゆうどう): 
\\	指導(しどう): 
\\	導く (みちび.く)
\\	辻			
\\	辻斬り 【つじぎり】 
\\	つじ	辻褄(つじつま): 
\\	迅			
\\	ジン		迅速(じんそく): 
\\	造			
\\	ゾウ	つく.る、つく.り、-づく.り	偽造(ぎぞう): 
\\	創造(そうぞう): 
\\	造り(つくり): 
\\	荷造り(にづくり): 
\\	改造(かいぞう): 
\\	構造(こうぞう): 
\\	人造(じんぞう): 
\\	製造(せいぞう): 
\\	造船(ぞうせん): 
\\	造る(つくる): 
\\	造る (つく.る)
\\	迫			
\\	ハク	せま.る	圧迫(あっぱく): 
\\	脅迫(きょうはく): 
\\	迫害(はくがい): 
\\	迫る(せまる): 
\\	迫る (せま.る)
\\	逃			
\\	トウ 逃走 とうそう
\\	桃236 (チョウ): 兆235 眺237 挑658 跳1284.	
\\	トウ	に.げる、に.がす、のが.す、のが.れる	逃走(とうそう): 
\\	逃亡(とうぼう): 
\\	逃げ出す(にげだす): 
\\	逃す(のがす): 
\\	逃れる(のがれる): 
\\	見逃す(みのがす): 
\\	逃がす(にがす): 
\\	逃げる(にげる): 
\\	逃がす (に.がす), 逃げる (に.げる), 逃す (のが.す), 逃れる (のが.れる)
\\	辺			
\\	ヘン	あた.り、ほと.り、-べ	対辺(たいへん): 
\\	浜辺(はまべ): 
\\	辺り(ほとり): 
\\	辺り(あたり): 
\\	周辺(しゅうへん): 
\\	辺(へん): 
\\	辺境 (へんきょう), 周辺 (しゅうへん), その辺 (そのへん), 辺り (あた.り), 辺 (べ)
\\	巡			
\\	ジュン	めぐ.る、めぐ.り	お巡りさん(おまわりさん): 
\\	巡査(じゅんさ): 
\\	巡る(めぐる): 
\\	巡る (めぐ.る)
\\	車			
\\	シャ	くるま	風車(かざぐるま): 
\\	機関車(きかんしゃ): 
\\	下車(げしゃ): 
\\	車庫(しゃこ): 
\\	車掌(しゃしょう): 
\\	車輪(しゃりん): 
\\	乗車(じょうしゃ): 
\\	駐車(ちゅうしゃ): 
\\	停車(ていしゃ): 
\\	歯車(はぐるま): 
\\	発車(はっしゃ): 
\\	列車(れっしゃ): 
\\	汽車(きしゃ): 
\\	駐車場(ちゅうしゃじょう): 
\\	車(くるま): 
\\	自転車(じてんしゃ): 
\\	自動車(じどうしゃ): 
\\	電車(でんしゃ): 
\\	車 (くるま)
\\	連		
\\	レン	つら.なる、つら.ねる、つ.れる、-づ.れ	一連(いちれん): 
\\	国連(こくれん): 
\\	連なる(つらなる): 
\\	連ねる(つらねる): 
\\	連休(れんきゅう): 
\\	連日(れんじつ): 
\\	連中(れんじゅう): 
\\	連帯(れんたい): 
\\	連邦(れんぽう): 
\\	連盟(れんめい): 
\\	連絡(れんらく): 
\\	関連(かんれん): 
\\	連れ(つれ): 
\\	連合(れんごう): 
\\	連想(れんそう): 
\\	連続(れんぞく): 
\\	連れる(つれる): 
\\	連れる (つ.れる), 連なる (つら.なる), 連ねる (つら.ねる)
\\	軌			
\\	キ		軌道(きどう): 
\\	輸			
\\	ユ、シュ		運輸(うんゆ): 
\\	輸血(ゆけつ): 
\\	輸送(ゆそう): 
\\	輸入(ゆにゅう): 
\\	輸出(ゆしゅつ): 
\\	前			
\\	ゼン	まえ、-まえ	当たり前(あたりまえ): 
\\	腕前(うでまえ): 
\\	事前(じぜん): 
\\	前(せん): 
\\	前提(ぜんてい): 
\\	前途(ぜんと): 
\\	前例(ぜんれい): 
\\	建前(たてまえ): 
\\	前売り(まえうり): 
\\	前置き(まえおき): 
\\	前もって(まえもって): 
\\	真ん前(まんまえ): 
\\	以前(いぜん): 
\\	前後(ぜんご): 
\\	前者(ぜんしゃ): 
\\	前進(ぜんしん): 
\\	直前(ちょくぜん): 
\\	手前(てまえ): 
\\	午前(ごぜん): 
\\	名前(なまえ): 
\\	前(まえ): 
\\	前 (まえ)
\\	各			
\\	カク	おのおの	各種(かくしゅ): 
\\	各々(それぞれ): 
\\	各々(おのおの): 
\\	各自(かくじ): 
\\	各地(かくち): 
\\	各 (おのおの)
\\	格			
\\	カク、コウ、キャク、ゴウ		格(かく): 
\\	格差(かくさ): 
\\	格好(かっこう): 
\\	規格(きかく): 
\\	資格(しかく): 
\\	失格(しっかく): 
\\	人格(じんかく): 
\\	体格(たいかく): 
\\	本格(ほんかく): 
\\	価格(かかく): 
\\	格別(かくべつ): 
\\	合格(ごうかく): 
\\	性格(せいかく): 
\\	同格(どうかく): 
\\	格式 (かくしき), 規格 (きかく), 性格 (せいかく)
\\	略			
\\	リャク	ほぼ、おか.す、おさ.める、はかりごと、はか.る、はぶ.く、りゃく.す、りゃく.する	概略(がいりゃく): 
\\	侵略(しんりゃく): 
\\	略語(りゃくご): 
\\	略奪(りゃくだつ): 
\\	省略(しょうりゃく): 
\\	略す(りゃくす): 
\\	客			
\\	キャク、カク		客(きゃく): 
\\	客観(きゃっかん): 
\\	乗客(じょうかく): 
\\	旅客(りょかく): 
\\	観客(かんきゃく): 
\\	客席(きゃくせき): 
\\	客間(きゃくま): 
\\	乗客(じょうきゃく): 
\\	客(きゃく): 
\\	客間 (きゃくま), 客車 (きゃくしゃ), 乗客 (じょうきゃく)
\\	額			
\\	ガク	ひたい	差額(さがく): 
\\	額(がく): 
\\	金額(きんがく): 
\\	額(ひたい): 
\\	額 (ひたい)
\\	夏			
\\	カ、ガ、ゲ	なつ	夏(なつ): 
\\	夏休み(なつやすみ): 
\\	夏季 (かき), 初夏 (しょか), 盛夏 (せいか), 夏 (なつ)
\\	処			
\\	ショ	ところ、-こ、お.る	彼処(あそこ): 
\\	処置(しょち): 
\\	処罰(しょばつ): 
\\	処分(しょぶん): 
\\	其処(そこ): 
\\	其処で(そこで): 
\\	其処ら(そこら): 
\\	対処(たいしょ): 
\\	処理(しょり): 
\\	条			
\\	ジョウ、チョウ、デキ	えだ、すじ	箇条書き(かじょうがき): 
\\	条約(じょうやく): 
\\	発条(ばね): 
\\	条件(じょうけん): 
\\	落			
\\	ラク	お.ちる、お.ち、お.とす	お洒落(おしゃれ): 
\\	落ち込む(おちこむ): 
\\	落ち着き(おちつき): 
\\	落ち葉(おちば): 
\\	洒落(しゃらく): 
\\	洒落る(しゃれる): 
\\	墜落(ついらく): 
\\	転落(てんらく): 
\\	没落(ぼつらく): 
\\	見落とす(みおとす): 
\\	落下(らっか): 
\\	落着く(おちつく): 
\\	落し物(おとしもの): 
\\	洒落(しゃれ): 
\\	落第(らくだい): 
\\	落ちる(おちる): 
\\	落とす(おとす): 
\\	落ちる (お.ちる), 落とす (お.とす)
\\	冗			
\\	冗談 
\\	冗語 
\\	冗員 
\\	員-
\\	ジョウ		冗談(じょうだん): 
\\	軍			
\\	グン		軍(いくさ): 
\\	軍艦(ぐんかん): 
\\	軍事(ぐんじ): 
\\	軍備(ぐんび): 
\\	軍服(ぐんぷく): 
\\	軍(ぐん): 
\\	軍隊(ぐんたい): 
\\	輝			
\\	キ	かがや.く	輝く(かがやく): 
\\	輝く (かがや.く)
\\	運			
\\	ウン	はこ.ぶ	運輸(うんゆ): 
\\	運用(うんよう): 
\\	運営(うんえい): 
\\	運送(うんそう): 
\\	運賃(うんちん): 
\\	運搬(うんぱん): 
\\	運命(うんめい): 
\\	海運(かいうん): 
\\	運(うん): 
\\	運河(うんが): 
\\	運転(うんてん): 
\\	運動(うんどう): 
\\	幸運(こううん): 
\\	不運(ふうん): 
\\	運転手(うんてんしゅ): 
\\	運ぶ(はこぶ): 
\\	運ぶ (はこ.ぶ)
\\	冠			
\\	カン	かんむり	冠(かん): 
\\	冠(かんむり): 
\\	冠 (かんむり)
\\	夢			
\\	ム、ボウ	ゆめ、ゆめ.みる、くら.い	夢中(むちゅう): 
\\	夢(ゆめ): 
\\	夢 (ゆめ)
\\	坑			
\\	コウ			
\\	高		
\\	コウ	たか.い、たか、-だか、たか.まる、たか.める	高原(こうげん): 
\\	高尚(こうしょう): 
\\	高等学校(こうとうがっこう): 
\\	残高(ざんだか): 
\\	高(たか): 
\\	高まる(たかまる): 
\\	名高い(なだかい): 
\\	高価(こうか): 
\\	高級(こうきゅう): 
\\	高層(こうそう): 
\\	高速(こうそく): 
\\	高度(こうど): 
\\	高等(こうとう): 
\\	最高(さいこう): 
\\	高める(たかめる): 
\\	高校(こうこう): 
\\	高校生(こうこうせい): 
\\	高い(たかい): 
\\	高 (たか), 高い (たか.い), 高まる (たか.まる), 高める (たか.める)
\\	享			
\\	キョウ、コウ	う.ける	享受(きょうじゅ): 
\\	塾			
\\	ジュク		塾(じゅく): 
\\	熟			
\\	ジュク	う.れる	成熟(せいじゅく): 
\\	未熟(みじゅく): 
\\	熟語(じゅくご): 
\\	熟れる (う.れる)
\\	亭			
\\	テイ、チン			
\\	京			
\\	キョウ、ケイ、キン	みやこ	帰京(ききょう): 
\\	上京(じょうきょう): 
\\	涼			
\\	リョウ	すず.しい、すず.む、すず.やか、うす.い、ひや.す、まことに	涼む(すずむ): 
\\	涼しい(すずしい): 
\\	涼しい (すず.しい), 涼む (すず.む)
\\	景			
\\	ケイ		背景(はいけい): 
\\	不景気(ふけいき): 
\\	景気(けいき): 
\\	光景(こうけい): 
\\	風景(ふうけい): 
\\	景色(けしき): 
\\	鯨			
\\	ゲイ	くじら	捕鯨(ほげい): 
\\	鯨 (くじら)
\\	舎			
\\	シャ、セキ	やど.る	校舎(こうしゃ): 
\\	田舎(いなか): 
\\	周			
\\	シュウ	まわ.り	周(しゅう): 
\\	周期(しゅうき): 
\\	円周(えんしゅう): 
\\	周囲(しゅうい): 
\\	周辺(しゅうへん): 
\\	周り(まわり): 
\\	周り (まわ.り)
\\	週			
\\	シュウ		隔週(かくしゅう): 
\\	週間(しゅうかん): 
\\	再来週(さらいしゅう): 
\\	週(しゅう): 
\\	先々週(せんせんしゅう): 
\\	さ来週(さらいしゅう): 
\\	今週(こんしゅう): 
\\	先週(せんしゅう): 
\\	毎週(まいしゅう): 
\\	来週(らいしゅう): 
\\	士			
\\	シ		学士(がくし): 
\\	修士(しゅうし): 
\\	紳士(しんし): 
\\	同士(どうし): 
\\	兵士(へいし): 
\\	博士(はかせ): 
\\	武士(ぶし): 
\\	吉			
\\	キチ、キツ	よし	不吉(ふきつ): 
\\	吉日 (きちじつ), 吉例 (きちれい), 大吉 (だいきち)
\\	壮			
\\	ソウ	さかん	壮大(そうだい): 
\\	荘			
\\	ソウ、ショウ、チャン	ほうき、おごそ.か	別荘(べっそう): 
\\	売			
\\	バイ	う.る、う.れる	売り出し(うりだし): 
\\	売り出す(うりだす): 
\\	売れ行き(うれゆき): 
\\	小売(こうり): 
\\	前売り(まえうり): 
\\	売上(うりあげ): 
\\	売り切れ(うりきれ): 
\\	売り切れる(うりきれる): 
\\	売れる(うれる): 
\\	商売(しょうばい): 
\\	特売(とくばい): 
\\	売店(ばいてん): 
\\	売買(ばいばい): 
\\	発売(はつばい): 
\\	販売(はんばい): 
\\	売り場(うりば): 
\\	売る(うる): 
\\	売る (う.る), 売れる (う.れる)
\\	学			
\\	字).	
\\	ガク	まな.ぶ	学芸(がくげい): 
\\	学士(がくし): 
\\	学説(がくせつ): 
\\	学歴(がくれき): 
\\	休学(きゅうがく): 
\\	共学(きょうがく): 
\\	工学(こうがく): 
\\	考古学(こうこがく): 
\\	高等学校(こうとうがっこう): 
\\	修学(しゅうがく): 
\\	退学(たいがく): 
\\	法学(ほうがく): 
\\	化学(かがく): 
\\	学(がく): 
\\	学者(がくしゃ): 
\\	学習(がくしゅう): 
\\	学術(がくじゅつ): 
\\	学年(がくねん): 
\\	学部(がくぶ): 
\\	学問(がくもん): 
\\	学力(がくりょく): 
\\	学科(がっか): 
\\	学会(がっかい): 
\\	学期(がっき): 
\\	学級(がっきゅう): 
\\	見学(けんがく): 
\\	語学(ごがく): 
\\	在学(ざいがく): 
\\	自然科学(しぜんかがく): 
\\	社会科学(しゃかいかがく): 
\\	奨学金(しょうがくきん): 
\\	小学生(しょうがくせい): 
\\	進学(しんがく): 
\\	人文科学(じんぶんかがく): 
\\	大学院(だいがくいん): 
\\	中学(ちゅうがく): 
\\	通学(つうがく): 
\\	哲学(てつがく): 
\\	学ぶ(まなぶ): 
\\	留学(りゅうがく): 
\\	医学(いがく): 
\\	科学(かがく): 
\\	小学校(しょうがっこう): 
\\	数学(すうがく): 
\\	大学生(だいがくせい): 
\\	中学校(ちゅうがっこう): 
\\	入学(にゅうがく): 
\\	文学(ぶんがく): 
\\	学生(がくせい): 
\\	学校(がっこう): 
\\	大学(だいがく): 
\\	留学生(りゅうがくせい): 
\\	学ぶ (まな.ぶ)
\\	覚			
\\	カク	おぼ.える、さ.ます、さ.める、さと.る	覚え(おぼえ): 
\\	錯覚(さっかく): 
\\	視覚(しかく): 
\\	自覚(じかく): 
\\	聴覚(ちょうかく): 
\\	味覚(みかく): 
\\	目覚しい(めざましい): 
\\	目覚める(めざめる): 
\\	覚悟(かくご): 
\\	感覚(かんかく): 
\\	覚ます(さます): 
\\	覚める(さめる): 
\\	目覚し(めざまし): 
\\	覚える(おぼえる): 
\\	覚える (おぼ.える), 覚ます (さ.ます), 覚める (さ.める)
\\	栄			
\\	エイ、ヨウ	さか.える、は.え、-ば.え、は.える、え	栄える(さかえる): 
\\	繁栄(はんえい): 
\\	栄養(えいよう): 
\\	栄える (さか.える), 栄え (は.え), 栄える (は.える)
\\	書			
\\	ショ	か.く、-が.き、-がき	書き取り(かきとり): 
\\	書き取る(かきとる): 
\\	箇条書き(かじょうがき): 
\\	願書(がんしょ): 
\\	原書(げんしょ): 
\\	書評(しょひょう): 
\\	聖書(せいしょ): 
\\	著書(ちょしょ): 
\\	秘書(ひしょ): 
\\	文書(ぶんしょ): 
\\	書留(かきとめ): 
\\	教科書(きょうかしょ): 
\\	下書き(したがき): 
\\	書斎(しょさい): 
\\	書籍(しょせき): 
\\	書店(しょてん): 
\\	書道(しょどう): 
\\	書物(しょもつ): 
\\	書類(しょるい): 
\\	清書(せいしょ): 
\\	投書(とうしょ): 
\\	読書(どくしょ): 
\\	図書(としょ): 
\\	書く(かく): 
\\	辞書(じしょ): 
\\	図書館(としょかん): 
\\	葉書(はがき): 
\\	書く (か.く)
\\	津			
\\	シン	つ	津波(つなみ): 
\\	津 (つ)
\\	牧			
\\	ボク	まき	牧師(ぼくし): 
\\	遊牧(ゆうぼく): 
\\	牧場(ぼくじょう): 
\\	牧畜(ぼくちく): 
\\	牧 (まき)
\\	攻			
\\	コウ	せ.める	攻め(せめ): 
\\	攻撃(こうげき): 
\\	攻める(せめる): 
\\	専攻(せんこう): 
\\	攻める (せ.める)
\\	敗			
\\	ハイ	やぶ.れる	一敗(いっぱい): 
\\	敗戦(はいせん): 
\\	腐敗(ふはい): 
\\	勝敗(しょうはい): 
\\	失敗(しっぱい): 
\\	敗れる (やぶ.れる)
\\	枚			
\\	マイ、バイ		枚(まい): 
\\	枚数(まいすう): 
\\	故			
\\	コ	ゆえ、ふる.い、もと	故(こ): 
\\	故人(こじん): 
\\	其れ故(それゆえ): 
\\	何故(なぜ): 
\\	故郷(こきょう): 
\\	故障(こしょう): 
\\	事故(じこ): 
\\	故 (ゆえ)
\\	敬			
\\	敬 【けい】 
\\	敬語 【けいご】 
\\	ケイ、キョウ	うやま.う	敬具(けいぐ): 
\\	敬う(うやまう): 
\\	敬意(けいい): 
\\	敬語(けいご): 
\\	尊敬(そんけい): 
\\	敬う (うやま.う)
\\	言			
\\	ゲン、ゴン	い.う、こと	言い訳(いいわけ): 
\\	一言(いちげん): 
\\	片言(かたこと): 
\\	予言(かねごと): 
\\	言論(げんろん): 
\\	言伝(ことづて): 
\\	証言(しょうげん): 
\\	助言(じょげん): 
\\	宣言(せんげん): 
\\	断言(だんげん): 
\\	伝言(つてごと): 
\\	発言(はつげん): 
\\	無言(むごん): 
\\	言い出す(いいだす): 
\\	言い付ける(いいつける): 
\\	言わば(いわば): 
\\	言語(げんご): 
\\	言付ける(ことづける): 
\\	言葉遣い(ことばづかい): 
\\	一言(ひとこと): 
\\	独り言(ひとりごと): 
\\	方言(ほうげん): 
\\	言う(いう): 
\\	言葉(ことば): 
\\	言行 (げんこう), 言論 (げんろん), 宣言 (せんげん), 言う (い.う), 言 (こと)
\\	警			
\\	ケイ	いまし.める	警戒(けいかい): 
\\	警部(けいぶ): 
\\	警告(けいこく): 
\\	警備(けいび): 
\\	警官(けいかん): 
\\	警察(けいさつ): 
\\	計			
\\	ケイ	はか.る、はか.らう	家計(かけい): 
\\	計器(けいき): 
\\	集計(しゅうけい): 
\\	生計(せいけい): 
\\	計る(はかる): 
\\	会計(かいけい): 
\\	計(けい): 
\\	計算(けいさん): 
\\	合計(ごうけい): 
\\	設計(せっけい): 
\\	統計(とうけい): 
\\	余計(よけい): 
\\	計画(けいかく): 
\\	時計(とけい): 
\\	計らう (はか.らう), 計る (はか.る)
\\	獄			
\\	ゴク		地獄(じごく): 
\\	訂		
\\	テイ		改訂(かいてい): 
\\	訂正(ていせい): 
\\	討			
\\	トウ	う.つ	討議(とうぎ): 
\\	討論(とうろん): 
\\	討つ(うつ): 
\\	検討(けんとう): 
\\	討つ (う.つ)
\\	訓			
\\	クン、キン	おし.える、よ.む、くん.ずる	教訓(きょうくん): 
\\	訓(くん): 
\\	訓練(くんれん): 
\\	詔		
\\	ショウ	みことのり		詔 (みことのり)
\\	詰			
\\	キツ、キチ	つ.める、つ.め、-づ.め、つ.まる、つ.む	詰らない(つまらない): 
\\	詰まり(つまり): 
\\	詰る(なじる): 
\\	缶詰(かんづめ): 
\\	詰まる(つまる): 
\\	詰める(つめる): 
\\	瓶詰(びんづめ): 
\\	詰まる (つ.まる), 詰む (つ.む), 詰める (つ.める)
\\	話			
\\	ワ	はな.す、はなし	対話(たいわ): 
\\	話し合い(はなしあい): 
\\	受話器(じゅわき): 
\\	神話(しんわ): 
\\	童話(どうわ): 
\\	話し合う(はなしあう): 
\\	話し掛ける(はなしかける): 
\\	話中(はなしちゅう): 
\\	話題(わだい): 
\\	会話(かいわ): 
\\	世話(せわ): 
\\	電話(でんわ): 
\\	話(はなし): 
\\	話す(はなす): 
\\	話す (はな.す), 話 (はなし)
\\	詠			
\\	エイ	よ.む、うた.う		詠む (よ.む)
\\	詩			
\\	シ	うた	詩(し): 
\\	詩人(しじん): 
\\	語			
\\	ゴ	かた.る、かた.らう	漢語(かんご): 
\\	語彙(ごい): 
\\	語句(ごく): 
\\	語源(ごげん): 
\\	標語(ひょうご): 
\\	文語(ぶんご): 
\\	略語(りゃくご): 
\\	語る(かたる): 
\\	敬語(けいご): 
\\	言語(げんご): 
\\	語(ご): 
\\	語学(ごがく): 
\\	国語(こくご): 
\\	熟語(じゅくご): 
\\	主語(しゅご): 
\\	述語(じゅつご): 
\\	単語(たんご): 
\\	物語(ものがたり): 
\\	物語る(ものがたる): 
\\	用語(ようご): 
\\	英語(えいご): 
\\	語らう (かた.らう), 語る (かた.る)
\\	読			
\\	(言) 
\\	(売); 
\\	(読).	
\\	ドク、トク、トウ	よ.む、-よ.み	購読(こうどく): 
\\	講読(こうどく): 
\\	読者(どくしゃ): 
\\	読み上げる(よみあげる): 
\\	朗読(ろうどく): 
\\	句読点(くとうてん): 
\\	読書(どくしょ): 
\\	読み(よみ): 
\\	読む(よむ): 
\\	読書 (どくしょ), 音読 (おんどく), 購読 (こうどく), 読本 (どくほん), 読む (よ.む)
\\	調			
\\	チョウ	しら.べる、しら.べ、ととの.う、ととの.える	協調(きょうちょう): 
\\	好調(こうちょう): 
\\	下調べ(したしらべ): 
\\	失調(しっちょう): 
\\	調べ(しらべ): 
\\	単調(たんちょう): 
\\	調印(ちょういん): 
\\	調停(ちょうてい): 
\\	調理(ちょうり): 
\\	調和(ちょうわ): 
\\	同調(どうちょう): 
\\	取り調べる(とりしらべる): 
\\	不調(ふちょう): 
\\	強調(きょうちょう): 
\\	順調(じゅんちょう): 
\\	調査(ちょうさ): 
\\	調子(ちょうし): 
\\	調整(ちょうせい): 
\\	調節(ちょうせつ): 
\\	調味料(ちょうみりょう): 
\\	調べる(しらべる): 
\\	調べる (しら.べる), 調う (ととの.う), 調える (ととの.える)
\\	談			
\\	ダン		縁談(えんだん): 
\\	会談(かいだん): 
\\	座談会(ざだんかい): 
\\	雑談(ざつだん): 
\\	相談(そうだん): 
\\	対談(たいだん): 
\\	冗談(じょうだん): 
\\	諾			
\\	ダク		承諾(しょうだく): 
\\	諭			
\\	ユ	さと.す		諭す (さと.す)
\\	式			
\\	シキ		株式(かぶしき): 
\\	式場(しきじょう): 
\\	方式(ほうしき): 
\\	様式(ようしき): 
\\	儀式(ぎしき): 
\\	形式(けいしき): 
\\	公式(こうしき): 
\\	式(しき): 
\\	正式(せいしき): 
\\	葬式(そうしき): 
\\	方程式(ほうていしき): 
\\	試			
\\	シ	こころ.みる、ため.す	試み(こころみ): 
\\	試みる(こころみる): 
\\	試し(ためし): 
\\	試す(ためす): 
\\	試合(しあい): 
\\	試験(しけん): 
\\	試みる (こころ.みる), 試す (ため.す)
\\	弐			
\\	ニ、ジ	ふた.つ、そえ		
\\	域			
\\	域, 
\\	國 
\\	イキ		域外(いきがい): 
\\	領域(りょういき): 
\\	区域(くいき): 
\\	地域(ちいき): 
\\	流域(りゅういき): 
\\	賊			
\\	ゾク			
\\	栽			
\\	サイ		栽培(さいばい): 
\\	載			
\\	サイ	の.せる、の.る	記載(きさい): 
\\	掲載(けいさい): 
\\	載せる(のせる): 
\\	載る(のる): 
\\	載せる (の.せる), 載る (の.る)
\\	茂		
\\	モ	しげ.る	茂る(しげる): 
\\	茂る (しげ.る)
\\	成			
\\	セイ、ジョウ	な.る、な.す、-な.す	行き成り(いきなり): 
\\	育成(いくせい): 
\\	可成(かなり): 
\\	形成(けいせい): 
\\	結成(けっせい): 
\\	合成(ごうせい): 
\\	成果(せいか): 
\\	成熟(せいじゅく): 
\\	成年(せいねん): 
\\	達成(たっせい): 
\\	成り立つ(なりたつ): 
\\	成る(なる): 
\\	成る丈(なるたけ): 
\\	成るべく(なるべく): 
\\	持て成す(もてなす): 
\\	養成(ようせい): 
\\	完成(かんせい): 
\\	構成(こうせい): 
\\	作成(さくせい): 
\\	賛成(さんせい): 
\\	成功(せいこう): 
\\	成人(せいじん): 
\\	成績(せいせき): 
\\	成長(せいちょう): 
\\	成分(せいぶん): 
\\	成立(せいりつ): 
\\	成功 (せいこう), 完成 (かんせい), 賛成 (さんせい), 成す (な.す), 成る (な.る)
\\	城		
\\	ジョウ	しろ	城下(じょうか): 
\\	城(しろ): 
\\	城 (しろ)
\\	誠			
\\	セイ	まこと	誠実(せいじつ): 
\\	誠(まこと): 
\\	誠 (まこと)
\\	威			
\\	イ	おど.す、おど.し、おど.かす	威力(いりょく): 
\\	権威(けんい): 
\\	威張る(いばる): 
\\	滅			
\\	メツ	ほろ.びる、ほろ.ぶ、ほろ.ぼす	全滅(ぜんめつ): 
\\	滅びる(ほろびる): 
\\	滅ぼす(ほろぼす): 
\\	滅茶苦茶(めちゃくちゃ): 
\\	滅亡(めつぼう): 
\\	絶滅(ぜつめつ): 
\\	滅多に(めったに): 
\\	滅びる (ほろ.びる), 滅ぼす (ほろ.ぼす)
\\	減			
\\	ゲン	へ.る、へ.らす	いい加減(いいかげん): 
\\	軽減(けいげん): 
\\	減少(げんしょう): 
\\	減点(げんてん): 
\\	削減(さくげん): 
\\	加減(かげん): 
\\	増減(ぞうげん): 
\\	減らす(へらす): 
\\	減る(へる): 
\\	減らす (へ.らす), 減る (へ.る)
\\	桟			
\\	サン、セン	かけはし	桟橋(さんきょう): 
\\	銭			
\\	セン、ゼン	ぜに、すき	小銭(こぜに): 
\\	金銭(きんせん): 
\\	銭 (ぜに)
\\	浅			
\\	セン	あさ.い	浅ましい(あさましい): 
\\	浅い(あさい): 
\\	浅い (あさ.い)
\\	止			
\\	シ	と.まる、-ど.まり、と.める、-と.める、-ど.め、とど.める、とど.め、とど.まる、や.める、や.む、-や.む、よ.す、-さ.す、-さ.し	受け止める(うけとめる): 
\\	静止(せいし): 
\\	阻止(そし): 
\\	止まる(とどまる): 
\\	止める(とどめる): 
\\	廃止(はいし): 
\\	止むを得ない(やむをえない): 
\\	呼び止める(よびとめる): 
\\	禁止(きんし): 
\\	立ち止まる(たちどまる): 
\\	中止(ちゅうし): 
\\	停止(ていし): 
\\	引き止める(ひきとめる): 
\\	防止(ぼうし): 
\\	止む(やむ): 
\\	止める(やめる): 
\\	止す(よす): 
\\	止まる(とまる): 
\\	止まる (と.まる), 止める (と.める)
\\	歩			
\\	ホ、ブ、フ	ある.く、あゆ.む	歩み(あゆみ): 
\\	歩む(あゆむ): 
\\	譲歩(じょうほ): 
\\	徒歩(とほ): 
\\	歩(ふ): 
\\	進歩(しんぽ): 
\\	歩道(ほどう): 
\\	歩く(あるく): 
\\	散歩(さんぽ): 
\\	歩道 (ほどう), 徒歩 (とほ), 進歩 (しんぽ), 歩合 (ぶあい), 日歩 (ひぶ), 歩む (あゆ.む), 歩く (ある.く)
\\	渉			
\\	ショウ	わた.る	干渉(かんしょう): 
\\	交渉(こうしょう): 
\\	頻			
\\	ヒン	しき.りに	頻繁(ひんぱん): 
\\	肯		
\\	コウ	がえんじ.る	肯定(こうてい): 
\\	企		
\\	キ	くわだ.てる、たくら.む	企画(きかく): 
\\	企業(きぎょう): 
\\	企てる (くわだ.てる)
\\	歴			
\\	レキ、レッキ		学歴(がくれき): 
\\	経歴(けいれき): 
\\	履歴(りれき): 
\\	歴史(れきし): 
\\	武			
\\	ブ、ム	たけ.し	武装(ぶそう): 
\\	武力(ぶりょく): 
\\	武器(ぶき): 
\\	武士(ぶし): 
\\	武力 (ぶりょく), 武士 (ぶし), 文武 (ぶんぶ)
\\	賦			
\\	フ、ブ		月賦(げっぷ): 
\\	正			
\\	セイ、ショウ	ただ.しい、ただ.す、まさ、まさ.に	正解(せいかい): 
\\	正規(せいき): 
\\	正義(せいぎ): 
\\	正常(せいじょう): 
\\	正当(せいとう): 
\\	正門(せいもん): 
\\	是正(ぜせい): 
\\	訂正(ていせい): 
\\	正しく(まさしく): 
\\	正に(まさに): 
\\	改正(かいせい): 
\\	公正(こうせい): 
\\	修正(しゅうせい): 
\\	正午(しょうご): 
\\	正直(しょうじき): 
\\	正味(しょうみ): 
\\	正面(しょうめん): 
\\	正(せい): 
\\	正確(せいかく): 
\\	正式(せいしき): 
\\	正方形(せいほうけい): 
\\	不正(ふせい): 
\\	正しい(ただしい): 
\\	正義 (せいぎ), 正誤 (せいご), 訂正 (ていせい), 正しい (ただ.しい), 正す (ただ.す), 正 (まさ)
\\	証			
\\	ショウ	あかし	証(あかし): 
\\	証言(しょうげん): 
\\	証拠(しょうこ): 
\\	証人(しょうにん): 
\\	証明(しょうめい): 
\\	保証(ほしょう): 
\\	証 (あかし)
\\	政			
\\	セイ、ショウ	まつりごと、まん	行政(ぎょうせい): 
\\	財政(ざいせい): 
\\	政権(せいけん): 
\\	政策(せいさく): 
\\	政党(せいとう): 
\\	政府(せいふ): 
\\	政治(せいじ): 
\\	政治 (せいじ), 行政 (ぎょうせい), 家政 (かせい), 政 (まつりごと)
\\	定			
\\	テイ、ジョウ	さだ.める、さだ.まる、さだ.か	案の定(あんのじょう): 
\\	一定(いちじょう): 
\\	改定(かいてい): 
\\	確定(かくてい): 
\\	規定(きてい): 
\\	協定(きょうてい): 
\\	限定(げんてい): 
\\	国定(こくてい): 
\\	固定(こてい): 
\\	定まる(さだまる): 
\\	定める(さだめる): 
\\	所定(しょてい): 
\\	制定(せいてい): 
\\	設定(せってい): 
\\	定義(ていぎ): 
\\	定食(ていしょく): 
\\	定年(ていねん): 
\\	判定(はんてい): 
\\	未定(みてい): 
\\	安定(あんてい): 
\\	一定(いってい): 
\\	仮定(かてい): 
\\	勘定(かんじょう): 
\\	決定(けってい): 
\\	肯定(こうてい): 
\\	指定(してい): 
\\	定規(じょうぎ): 
\\	推定(すいてい): 
\\	測定(そくてい): 
\\	断定(だんてい): 
\\	定員(ていいん): 
\\	定価(ていか): 
\\	定期(ていき): 
\\	定期券(ていきけん): 
\\	定休日(ていきゅうび): 
\\	特定(とくてい): 
\\	否定(ひてい): 
\\	予定(よてい): 
\\	定価 (ていか), 安定 (あんてい), 決定 (けってい), 定か (さだ.か), 定まる (さだ.まる), 定める (さだ.める)
\\	錠			
\\	ジョウ		手錠(てじょう): 
\\	走			
\\	ソウ	はし.る	御馳走(ごちそう): 
\\	ご馳走さま(ごちそうさま): 
\\	走行(そうこう): 
\\	逃走(とうそう): 
\\	走る(はしる): 
\\	走る (はし.る)
\\	超		
\\	チョウ	こ.える、こ.す	超える(こえる): 
\\	超(ちょう): 
\\	超す(こす): 
\\	超過(ちょうか): 
\\	超える (こ.える), 超す (こ.す)
\\	赴		
\\	フ	おもむ.く	赴く(おもむく): 
\\	赴任(ふにん): 
\\	赴く (おもむ.く)
\\	越			
\\	エツ、オツ	こ.す、-こ.す、-ご.し、こ.える、-ご.え	越える(こえる): 
\\	優越(ゆうえつ): 
\\	追い越す(おいこす): 
\\	越す(こす): 
\\	引越し(ひっこし): 
\\	引っ越す(ひっこす): 
\\	越える (こ.える), 越す (こ.す)
\\	是			
\\	ゼ、シ	これ、この、ここ	"是正(ぜせい): 
\\	是非とも(ぜひとも): 
\\	是非(ぜひ): 
\\	題		
\\	ダイ		課題(かだい): 
\\	議題(ぎだい): 
\\	主題(しゅだい): 
\\	出題(しゅつだい): 
\\	題(だい): 
\\	題名(だいめい): 
\\	話題(わだい): 
\\	宿題(しゅくだい): 
\\	問題(もんだい): 
\\	堤		
\\	テイ	つつみ	堤防(ていぼう): 
\\	堤 (つつみ)
\\	建		
\\	ケン、コン	た.てる、た.て、-だ.て、た.つ	再建(さいけん): 
\\	建前(たてまえ): 
\\	封建(ほうけん): 
\\	建設(けんせつ): 
\\	建築(けんちく): 
\\	建つ(たつ): 
\\	建てる(たてる): 
\\	建物(たてもの): 
\\	建築 (けんちく), 建議 (けんぎ), 封建的 (ほうけんてき), 建つ (た.つ), 建てる (た.てる)
\\	延		
\\	正 
\\	エン	の.びる、の.べる、の.べ、の.ばす	延ばす(のばす): 
\\	延びる(のびる): 
\\	延べ(のべ): 
\\	延いては(ひいては): 
\\	延期(えんき): 
\\	延長(えんちょう): 
\\	延ばす (の.ばす), 延びる (の.びる), 延べる (の.べる)
\\	誕		
\\	タン		誕生(たんじょう): 
\\	誕生日(たんじょうび): 
\\	礎			
\\	ソ	いしずえ	基礎(きそ): 
\\	礎 (いしずえ)
\\	婿			
\\	セイ	むこ	婿(むこ): 
\\	婿 (むこ)
\\	衣			
\\	(ネ) 
\\	イ、エ	ころも、きぬ、-ぎ	衣装(いしょう): 
\\	衣料(いりょう): 
\\	衣類(いるい): 
\\	衣食住(いしょくじゅう): 
\\	衣服(いふく): 
\\	浴衣(ゆかた): 
\\	衣 (ころも)
\\	裁			
\\	サイ	た.つ、さば.く	裁く(さばく): 
\\	制裁(せいさい): 
\\	体裁(ていさい): 
\\	独裁(どくさい): 
\\	裁判(さいばん): 
\\	裁縫(さいほう): 
\\	裁く (さば.く), 裁つ (た.つ)
\\	装		
\\	ソウ、ショウ	よそお.う、よそお.い	衣装(いしょう): 
\\	盛装(せいそう): 
\\	装飾(そうしょく): 
\\	装備(そうび): 
\\	武装(ぶそう): 
\\	舗装(ほそう): 
\\	装置(そうち): 
\\	服装(ふくそう): 
\\	包装(ほうそう): 
\\	装置 (そうち), 服装 (ふくそう), 変装 (へんそう), 装う (よそお.う)
\\	裏			
\\	リ	うら	裏返し(うらがえし): 
\\	裏返す(うらがえす): 
\\	裏切る(うらぎる): 
\\	裏口(うらぐち): 
\\	裏(うら): 
\\	裏 (うら)
\\	壊		
\\	カイ、エ	こわ.す、こわ.れる、やぶ.る	破壊(はかい): 
\\	崩壊(ほうかい): 
\\	壊す(こわす): 
\\	壊れる(こわれる): 
\\	壊す (こわ.す), 壊れる (こわ.れる)
\\	哀			
\\	アイ	あわ.れ、あわ.れむ、かな.しい	可哀想(かわいそう): 
\\	哀れ(あわれ): 
\\	哀れ (あわ.れ), 哀れむ (あわ.れむ)
\\	遠			
\\	エン、オン	とお.い	遠方(えんぽう): 
\\	遠ざかる(とおざかる): 
\\	遠回り(とおまわり): 
\\	待ち遠しい(まちどおしい): 
\\	永遠(えいえん): 
\\	遠足(えんそく): 
\\	遠慮(えんりょ): 
\\	望遠鏡(ぼうえんきょう): 
\\	遠く(とおく): 
\\	遠い(とおい): 
\\	遠近 (えんきん), 永遠 (えいえん), 敬遠 (けいえん), 遠い (とお.い)
\\	猿			
\\	エン	さる	猿(さる): 
\\	猿 (さる)
\\	初		
\\	ショ	はじ.め、はじ.めて、はつ、はつ-、うい-、-そ.める、-ぞ.め	初版(しょはん): 
\\	初(はつ): 
\\	初耳(はつみみ): 
\\	初級(しょきゅう): 
\\	初旬(しょじゅん): 
\\	最初(さいしょ): 
\\	初めて(はじめて): 
\\	初めに(はじめに): 
\\	初 (うい), 初める (そ.める), 初め (はじ.め), 初めて (はじ.めて), 初 (はつ)
\\	布		
\\	フ	ぬの	配布(はいふ): 
\\	布巾(ふきん): 
\\	布告(ふこく): 
\\	座布団(ざぶとん): 
\\	布(ぬの): 
\\	分布(ぶんぷ): 
\\	毛布(もうふ): 
\\	財布(さいふ): 
\\	布団(ふとん): 
\\	布 (ぬの)
\\	帆			
\\	ハン	ほ		帆 (ほ)
\\	幅			
\\	フク	はば	大幅(おおはば): 
\\	幅(はば): 
\\	幅 (はば)
\\	帽		
\\	ボウ、モウ	ずきん、おお.う	帽子(ぼうし): 
\\	幕			
\\	マク、バク	とばり	幕(とばり): 
\\	幕(まく): 
\\	幕切れ (まくぎれ), 天幕 (てんまく), 暗幕 (あんまく)
\\	幌			
\\	コウ	ほろ、とばり		
\\	錦			
\\	キン	にしき		錦 (にしき)
\\	市			
\\	シ	いち	市(いち): 
\\	市街(しがい): 
\\	市場(いちば): 
\\	市(し): 
\\	市民(しみん): 
\\	都市(とし): 
\\	市 (いち)
\\	姉			
\\	(妹), 
\\	シ	あね、はは	"姉妹(きょうだい): 
\\	従姉妹(いとこ): 
\\	姉妹(しまい): 
\\	姉(あね): 
\\	お姉さん(おねえさん): 
\\	姉 (あね)
\\	肺			
\\	ハイ		肺(はい): 
\\	帯			
\\	タイ	お.びる、おび	一帯(いったい): 
\\	帯びる(おびる): 
\\	携帯(けいたい): 
\\	世帯(せたい): 
\\	連帯(れんたい): 
\\	帯(おび): 
\\	温帯(おんたい): 
\\	寒帯(かんたい): 
\\	地帯(ちたい): 
\\	熱帯(ねったい): 
\\	包帯(ほうたい): 
\\	帯びる (お.びる), 帯 (おび)
\\	滞			
\\	タイ、テイ	とどこお.る	滞納(たいのう): 
\\	停滞(ていたい): 
\\	滞る(とどこおる): 
\\	渋滞(じゅうたい): 
\\	滞在(たいざい): 
\\	滞る (とどこお.る)
\\	刺			
\\	シ	さ.す、さ.さる、さ.し、さし、とげ	刺繍(ししゅう): 
\\	刺(とげ): 
\\	刺さる(ささる): 
\\	刺身(さしみ): 
\\	刺す(さす): 
\\	刺激(しげき): 
\\	名刺(めいし): 
\\	刺さる (さ.さる), 刺す (さ.す)
\\	制			
\\	セイ		規制(きせい): 
\\	強制(きょうせい): 
\\	制(せい): 
\\	制裁(せいさい): 
\\	制する(せいする): 
\\	制定(せいてい): 
\\	制服(せいふく): 
\\	制約(せいやく): 
\\	統制(とうせい): 
\\	抑制(よくせい): 
\\	制限(せいげん): 
\\	制作(せいさく): 
\\	制度(せいど): 
\\	専制(せんせい): 
\\	体制(たいせい): 
\\	製		
\\	セイ		製(せい): 
\\	製鉄(せいてつ): 
\\	製法(せいほう): 
\\	作製(さくせい): 
\\	製作(せいさく): 
\\	製造(せいぞう): 
\\	製品(せいひん): 
\\	転			
\\	{ニム}バス, 
\\	テン	ころ.がる、ころ.げる、ころ.がす、ころ.ぶ、まろ.ぶ、うたた、うつ.る、くる.めく	逆転(ぎゃくてん): 
\\	自転(じてん): 
\\	転回(てんかい): 
\\	転換(てんかん): 
\\	転居(てんきょ): 
\\	転勤(てんきん): 
\\	転校(てんこう): 
\\	転じる(てんじる): 
\\	転任(てんにん): 
\\	転落(てんらく): 
\\	移転(いてん): 
\\	運転(うんてん): 
\\	回転(かいてん): 
\\	転がす(ころがす): 
\\	転がる(ころがる): 
\\	転ぶ(ころぶ): 
\\	転々(てんてん): 
\\	運転手(うんてんしゅ): 
\\	自転車(じてんしゃ): 
\\	転がす (ころ.がす), 転がる (ころ.がる), 転げる (ころ.げる), 転ぶ (ころ.ぶ)
\\	芸			
\\	ゲイ、ウン	う.える、のり、わざ	学芸(がくげい): 
\\	手芸(しゅげい): 
\\	園芸(えんげい): 
\\	芸術(げいじゅつ): 
\\	芸能(げいのう): 
\\	工芸(こうげい): 
\\	文芸(ぶんげい): 
\\	雨			
\\	ウ	あめ、あま-、-さめ	雨具(あまぐ): 
\\	雨天(うてん): 
\\	梅雨(つゆ): 
\\	雨戸(あまど): 
\\	雨(あめ): 
\\	雨 (あま), 雨 (あめ)
\\	雲			
\\	ウン	くも、-ぐも	雲(くも): 
\\	雲 (くも)
\\	曇			
\\	ドン 曇天 どんてん
\\	ドン	くも.る	曇り(くもり): 
\\	曇る(くもる): 
\\	曇る (くも.る)
\\	雷			
\\	ライ	かみなり	雷(いかずち): 
\\	雷(かみなり): 
\\	雷 (かみなり)
\\	霜			
\\	ソウ	しも	霜(しも): 
\\	霜 (しも)
\\	冬			
\\	トウ	ふゆ	冬眠(とうみん): 
\\	冬(ふゆ): 
\\	冬 (ふゆ)
\\	天			
\\	テン	あまつ、あめ、あま-	天(あまつ): 
\\	天地(あめつち): 
\\	雨天(うてん): 
\\	天皇(すめらぎ): 
\\	晴天(せいてん): 
\\	先天的(せんてんてき): 
\\	天国(てんごく): 
\\	天才(てんさい): 
\\	天災(てんさい): 
\\	天井(てんじょう): 
\\	天体(てんたい): 
\\	天候(てんこう): 
\\	天然(てんねん): 
\\	天皇(てんのう): 
\\	天気予報(てんきよほう): 
\\	天気(てんき): 
\\	天 (あめ)
\\	橋			
\\	キョウ	はし	桟橋(さんきょう): 
\\	橋渡し(はしわたし): 
\\	鉄橋(てっきょう): 
\\	橋(はし): 
\\	橋 (はし)
\\	嬌		
\\	キョウ	なまめか.しい		
\\	立			
\\	リツ、リュウ、リットル	た.つ、-た.つ、た.ち-、た.てる、-た.てる、た.て-、たて-、-た.て、-だ.て、-だ.てる	確立(かくりつ): 
\\	気立て(きだて): 
\\	公立(こうりつ): 
\\	孤立(こりつ): 
\\	逆立ち(さかだち): 
\\	仕立てる(したてる): 
\\	樹立(じゅりつ): 
\\	自立(じりつ): 
\\	設立(せつりつ): 
\\	創立(そうりつ): 
\\	立方(たちかた): 
\\	立ち寄る(たちよる): 
\\	中立(ちゅうりつ): 
\\	取り立てる(とりたてる): 
\\	成り立つ(なりたつ): 
\\	腹立ち(はらだち): 
\\	役立つ(やくだつ): 
\\	立体(りったい): 
\\	立法(りっぽう): 
\\	両立(りょうりつ): 
\\	組み立てる(くみたてる): 
\\	国立(こくりつ): 
\\	献立(こんだて): 
\\	私立(しりつ): 
\\	成立(せいりつ): 
\\	対立(たいりつ): 
\\	立ち上がる(たちあがる): 
\\	立ち止まる(たちどまる): 
\\	立場(たちば): 
\\	独立(どくりつ): 
\\	目立つ(めだつ): 
\\	夕立(ゆうだち): 
\\	立派(りっぱ): 
\\	立てる(たてる): 
\\	役に立つ(やくにたつ): 
\\	立つ(たつ): 
\\	立案 (りつあん), 起立 (きりつ), 独立 (どくりつ), 立つ (た.つ), 立てる (た.てる)
\\	泣			
\\	キュウ	な.く	泣く(なく): 
\\	泣く (な.く)
\\	章			
\\	ショウ		章(しょう): 
\\	文章(ぶんしょう): 
\\	競			
\\	キョウ、ケイ	きそ.う、せ.る	競技(きょうぎ): 
\\	競馬(けいば): 
\\	競争(きょうそう): 
\\	競争 (きょうそう), 競技 (きょうぎ), 競泳 (きょうえい), 競う (きそ.う), 競る (せ.る)
\\	帝			
\\	テイ	みかど		
\\	童			
\\	ドウ	わらべ	児童(じどう): 
\\	童話(どうわ): 
\\	童 (わらべ)
\\	瞳		
\\	ドウ、トウ	ひとみ	瞳(ひとみ): 
\\	瞳 (ひとみ)
\\	鐘		
\\	ショウ	かね	釣鐘(つりがね): 
\\	鐘(かね): 
\\	鐘 (かね)
\\	商			
\\	ショウ	あきな.う	商人(あきうど): 
\\	商(しょう): 
\\	商業(しょうぎょう): 
\\	商社(しょうしゃ): 
\\	商店(しょうてん): 
\\	商人(しょうにん): 
\\	商売(しょうばい): 
\\	商品(しょうひん): 
\\	商う (あきな.う)
\\	嫡			
\\	チャク、テキ			
\\	適			
\\	テキ	かな.う	適応(てきおう): 
\\	適確(てきかく): 
\\	適宜(てきぎ): 
\\	適性(てきせい): 
\\	快適(かいてき): 
\\	適する(てきする): 
\\	適切(てきせつ): 
\\	適度(てきど): 
\\	適用(てきよう): 
\\	適当(てきとう): 
\\	滴			
\\	テキ	しずく、したた.る	水滴(すいてき): 
\\	滴 (しずく), 滴る (したた.る)
\\	敵			
\\	テキ 敵地 てきち
\\	嫡440 適441 滴442  摘657.	
\\	テキ	かたき、あだ、かな.う	敵(かたき): 
\\	素敵(すてき): 
\\	匹敵(ひってき): 
\\	敵(てき): 
\\	敵 (かたき)
\\	匕			
\\	ヒ	さじ、さじのひ		
\\	北			
\\	ホク	きた	南北(なんぼく): 
\\	北極(ほっきょく): 
\\	北(きた): 
\\	北 (きた)
\\	背			
\\	ハイ	せ、せい、そむ.く、そむ.ける	背負う(しょう): 
\\	背く(そむく): 
\\	背景(はいけい): 
\\	背後(はいご): 
\\	背(せ): 
\\	背負う(せおう): 
\\	背中(せなか): 
\\	背(せい): 
\\	背広(せびろ): 
\\	背 (せ), 背 (せい), 背く (そむ.く), 背ける (そむ.ける)
\\	比			
\\	ヒ	くら.べる	対比(たいひ): 
\\	比重(ひじゅう): 
\\	比率(ひりつ): 
\\	比例(ひれい): 
\\	比較(ひかく): 
\\	比較的(ひかくてき): 
\\	比べる(くらべる): 
\\	比べる (くら.べる)
\\	昆			
\\	コン		昆虫(こんちゅう): 
\\	皆			
\\	カイ	みな、みんな	皆(みな): 
\\	皆さん(みなさん): 
\\	皆 (みな)
\\	混			
\\	コン 混乱 こんらん
\\	コン	ま.じる、-ま.じり、ま.ざる、ま.ぜる、こ.む	混む(こむ): 
\\	混血(こんけつ): 
\\	混同(こんどう): 
\\	取り混ぜる(とりまぜる): 
\\	混ざる(まざる): 
\\	混じる(まじる): 
\\	混合(こんごう): 
\\	混雑(こんざつ): 
\\	混乱(こんらん): 
\\	混ぜる(まぜる): 
\\	混ざる (ま.ざる), 混じる (ま.じる), 混ぜる (ま.ぜる)
\\	渇			
\\	カツ	かわ.く	渇く(かわく): 
\\	渇く (かわ.く)
\\	謁			
\\	エツ			
\\	褐			
\\	カツ			
\\	喝			
\\	カツ			
\\	旨			
\\	シ	むね、うま.い	趣旨(しゅし): 
\\	要旨(ようし): 
\\	旨 (むね)
\\	脂			
\\	シ	あぶら	脂肪(しぼう): 
\\	脂(あぶら): 
\\	脂 (あぶら)
\\	壱			
\\	イチ、イツ	ひとつ		
\\	毎			
\\	マイ	ごと、-ごと.に	毎(ごと): 
\\	毎度(まいど): 
\\	毎朝(まいあさ): 
\\	毎月(まいげつ): 
\\	毎週(まいしゅう): 
\\	毎日(まいにち): 
\\	毎晩(まいばん): 
\\	敏			
\\	ビン	さとい	敏感(びんかん): 
\\	梅		
\\	バイ	うめ	梅干(うめぼし): 
\\	梅雨(つゆ): 
\\	梅(うめ): 
\\	梅 (うめ)
\\	海			
\\	カイ	うみ	海路(うみじ): 
\\	海運(かいうん): 
\\	海峡(かいきょう): 
\\	海抜(かいばつ): 
\\	海流(かいりゅう): 
\\	航海(こうかい): 
\\	領海(りょうかい): 
\\	海外(かいがい): 
\\	海水浴(かいすいよく): 
\\	海洋(かいよう): 
\\	海岸(かいがん): 
\\	海(うみ): 
\\	海 (うみ)
\\	乞		
\\	コツ、キツ、キ、キケ、コチ	こ.う		乞う (こ.う)
\\	乾			
\\	カン、ケン	かわ.く、かわ.かす、ほ.す、ひ.る、いぬい	乾(かん): 
\\	乾(ほし): 
\\	乾かす(かわかす): 
\\	乾燥(かんそう): 
\\	乾電池(かんでんち): 
\\	乾杯(かんぱい): 
\\	乾く(かわく): 
\\	乾かす (かわ.かす), 乾く (かわ.く)
\\	腹			
\\	フク	はら	お腹(おなか): 
\\	空腹(くうふく): 
\\	山腹(さんぷく): 
\\	中腹(ちゅうっぱら): 
\\	腹立ち(はらだち): 
\\	腹(はら): 
\\	腹 (はら)
\\	複			
\\	フク		重複(じゅうふく): 
\\	複合(ふくごう): 
\\	複写(ふくしゃ): 
\\	複数(ふくすう): 
\\	複雑(ふくざつ): 
\\	欠		
\\	ケツ、ケン	か.ける、か.く、あくび	欠く(かく): 
\\	欠乏(けつぼう): 
\\	不可欠(ふかけつ): 
\\	欠ける(かける): 
\\	欠陥(けっかん): 
\\	欠席(けっせき): 
\\	欠点(けってん): 
\\	欠く (か.く), 欠ける (か.ける)
\\	吹		
\\	スイ	ふ.く	吹奏(すいそう): 
\\	吹雪(ふぶき): 
\\	吹く(ふく): 
\\	吹く (ふ.く)
\\	炊		
\\	スイ	た.く、-だ.き	炊事(すいじ): 
\\	炊く(たく): 
\\	炊く (た.く)
\\	歌		
\\	カ	うた、うた.う	短歌(たんか): 
\\	歌手(かしゅ): 
\\	歌謡(かよう): 
\\	歌(うた): 
\\	歌う(うたう): 
\\	歌 (うた), 歌う (うた.う)
\\	軟		
\\	ナン	やわ.らか、やわ.らかい	柔軟(じゅうなん): 
\\	軟らかい(やわらかい): 
\\	軟らか (やわ.らか), 軟らかい (やわ.らかい)
\\	次		
\\	ジ、シ	つ.ぐ、つぎ	次(し): 
\\	次いで(ついで): 
\\	取り次ぐ(とりつぐ): 
\\	次第(しだい): 
\\	次々(つぎつぎ): 
\\	次ぐ(つぐ): 
\\	目次(もくじ): 
\\	次(つぎ): 
\\	次回 (じかい), 次元 (じげん), 目次 (もくじ), 次ぐ (つ.ぐ), 次 (つぎ)
\\	茨		
\\	シ、ジ	いばら、かや、くさぶき		茨 (いばら)
\\	資		
\\	シ		資格(しかく): 
\\	資金(しきん): 
\\	資産(しさん): 
\\	投資(とうし): 
\\	物資(ぶっし): 
\\	融資(ゆうし): 
\\	資源(しげん): 
\\	資本(しほん): 
\\	資料(しりょう): 
\\	姿		
\\	シ	すがた	姿勢(しせい): 
\\	姿(すがた): 
\\	姿 (すがた)
\\	諮		
\\	シ	はか.る	諮る(はかる): 
\\	諮る (はか.る)
\\	賠			
\\	バイ		賠償(ばいしょう): 
\\	培			
\\	バイ	つちか.う	栽培(さいばい): 
\\	培う (つちか.う)
\\	剖			
\\	ボウ		解剖(かいぼう): 
\\	音		
\\	オン、イン、ノン	おと、ね	音色(おんいろ): 
\\	音(ね): 
\\	本音(ほんね): 
\\	音(おん): 
\\	五十音(ごじゅうおん): 
\\	雑音(ざつおん): 
\\	騒音(そうおん): 
\\	物音(ものおと): 
\\	録音(ろくおん): 
\\	音(おと): 
\\	発音(はつおん): 
\\	音楽(おんがく): 
\\	音楽 (おんがく), 発音 (はつおん), 騒音 (そうおん), 音 (おと), 音 (ね)
\\	暗			
\\	アン	くら.い、くら.む、くれ.る	暗殺(あんさつ): 
\\	暗算(あんざん): 
\\	暗示(あんじ): 
\\	暗記(あんき): 
\\	薄暗い(うすぐらい): 
\\	真っ暗(まっくら): 
\\	暗い(くらい): 
\\	暗い (くら.い)
\\	韻		
\\	イン			
\\	識			
\\	シキ	し.る、しる.す	認識(にんしき): 
\\	良識(りょうしき): 
\\	意識(いしき): 
\\	常識(じょうしき): 
\\	知識(ちしき): 
\\	標識(ひょうしき): 
\\	鏡			
\\	キョウ、ケイ	かがみ	眼鏡(がんきょう): 
\\	顕微鏡(けんびきょう): 
\\	望遠鏡(ぼうえんきょう): 
\\	鏡(かがみ): 
\\	眼鏡(めがね): 
\\	鏡 (かがみ)
\\	境			
\\	キョウ、ケイ	さかい	境遇(きょうぐう): 
\\	国境(くにざかい): 
\\	環境(かんきょう): 
\\	境界(きょうかい): 
\\	国境(こっきょう): 
\\	境(さかい): 
\\	境界 (きょうかい), 境地 (きょうち), 逆境 (ぎゃっきょう), 境 (さかい)
\\	亡			
\\	ボウ、モウ	な.い、な.き-、ほろ.びる、ほろ.ぶ、ほろ.ぼす	逃亡(とうぼう): 
\\	滅亡(めつぼう): 
\\	死亡(しぼう): 
\\	亡くす(なくす): 
\\	亡くなる(なくなる): 
\\	亡父 (ぼうふ), 亡命 (ぼうめい), 存亡 (そんぼう), 亡い (な.い)
\\	盲			
\\	モウ	めくら	盲点(もうてん): 
\\	妄			
\\	モウ、ボウ	みだ.りに		妄信 (もうしん), 妄想 (もうそう), 迷妄 (めいもう)
\\	荒			
\\	コウ	あら.い、あら-、あ.れる、あ.らす、-あ.らし、すさ.む	荒らす(あらす): 
\\	荒っぽい(あらっぽい): 
\\	荒廃(こうはい): 
\\	荒い(あらい): 
\\	荒らす (あ.らす), 荒れる (あ.れる), 荒い (あら.い)
\\	望		
\\	ボウ、モウ	のぞ.む、もち	志望(しぼう): 
\\	絶望(ぜつぼう): 
\\	待望(たいぼう): 
\\	展望(てんぼう): 
\\	望ましい(のぞましい): 
\\	待ち望む(まちのぞむ): 
\\	有望(ゆうぼう): 
\\	要望(ようぼう): 
\\	欲望(よくぼう): 
\\	希望(きぼう): 
\\	失望(しつぼう): 
\\	望み(のぞみ): 
\\	望む(のぞむ): 
\\	望遠鏡(ぼうえんきょう): 
\\	望郷 (ぼうきょう), 希望 (きぼう), 人望 (じんぼう), 望む (のぞ.む)
\\	方		
\\	ホウ	かた、-かた、-がた	彼方此方(あちこち): 
\\	彼方(あちら): 
\\	彼方此方(あちらこちら): 
\\	遠方(えんぽう): 
\\	大方(おおかた): 
\\	方(かた): 
\\	地方(じかた): 
\\	其方(そちら): 
\\	外方(そっぽ): 
\\	立方(たちかた): 
\\	他方(たほう): 
\\	何方(どちら): 
\\	平方(へいほう): 
\\	方(ほう): 
\\	方策(ほうさく): 
\\	方式(ほうしき): 
\\	見方(みかた): 
\\	目方(めかた): 
\\	明け方(あけがた): 
\\	一方(いっぽう): 
\\	方々(かたがた): 
\\	正方形(せいほうけい): 
\\	地方(ちほう): 
\\	長方形(ちょうほうけい): 
\\	方角(ほうがく): 
\\	方言(ほうげん): 
\\	方向(ほうこう): 
\\	方針(ほうしん): 
\\	方程式(ほうていしき): 
\\	方法(ほうほう): 
\\	方々(ほうぼう): 
\\	方面(ほうめん): 
\\	味方(みかた): 
\\	仕方(しかた): 
\\	両方(りょうほう): 
\\	夕方(ゆうがた): 
\\	方 (かた)
\\	妨			
\\	ボウ	さまた.げる	妨害(ぼうがい): 
\\	妨げる(さまたげる): 
\\	妨げる (さまた.げる)
\\	坊			
\\	ボウ、ボッ		坊ちゃん(ぼっちゃん): 
\\	寝坊(ねぼう): 
\\	坊さん(ぼうさん): 
\\	坊や(ぼうや): 
\\	赤ん坊(あかんぼう): 
\\	朝寝坊(あさねぼう): 
\\	坊主 (ぼうず), 朝寝坊 (あさねぼう), 赤ん坊 (あかんぼう)
\\	芳			
\\	ホウ	かんば.しい		芳しい (かんば.しい)
\\	肪			
\\	ボウ		脂肪(しぼう): 
\\	訪			
\\	ホウ	おとず.れる、たず.ねる、と.う	訪れる(おとずれる): 
\\	訪問(ほうもん): 
\\	訪ねる(たずねる): 
\\	訪れる (おとず.れる), 訪ねる (たず.ねる)
\\	放		
\\	ホウ 放射 ほうしゃ
\\	倣979.	
\\	ホウ	はな.す、-っぱな.し、はな.つ、はな.れる、こ.く、ほう.る	追放(ついほう): 
\\	放棄(ほうき): 
\\	放射(ほうしゃ): 
\\	放射能(ほうしゃのう): 
\\	放出(ほうしゅつ): 
\\	放置(ほうち): 
\\	放り込む(ほうりこむ): 
\\	放り出す(ほうりだす): 
\\	解放(かいほう): 
\\	開放(かいほう): 
\\	放す(はなす): 
\\	放れる(はなれる): 
\\	放る(ほうる): 
\\	放送(ほうそう): 
\\	放す (はな.す), 放つ (はな.つ), 放れる (はな.れる)
\\	激		
\\	ゲキ	はげ.しい	激励(げきれい): 
\\	感激(かんげき): 
\\	急激(きゅうげき): 
\\	激増(げきぞう): 
\\	刺激(しげき): 
\\	激しい(はげしい): 
\\	激しい (はげ.しい)
\\	脱			
\\	ダツ	ぬ.ぐ、ぬ.げる	脱出(だっしゅつ): 
\\	脱する(だっする): 
\\	脱退(だったい): 
\\	脱線(だっせん): 
\\	脱ぐ(ぬぐ): 
\\	脱ぐ (ぬ.ぐ), 脱げる (ぬ.げる)
\\	説			
\\	セツ、ゼイ	と.く	概説(がいせつ): 
\\	学説(がくせつ): 
\\	説得(せっとく): 
\\	伝説(でんせつ): 
\\	説く(とく): 
\\	演説(えんぜつ): 
\\	解説(かいせつ): 
\\	社説(しゃせつ): 
\\	説(せつ): 
\\	小説(しょうせつ): 
\\	説明(せつめい): 
\\	説明 (せつめい), 小説 (しょうせつ), 演説 (えんぜつ), 説く (と.く)
\\	鋭		
\\	エイ	するど.い	鋭い(するどい): 
\\	鋭い (するど.い)
\\	曽		
\\	ソウ、ソ、ゾウ	かつ、かつて、すなわち		
\\	増			
\\	ゾウ	ま.す、ま.し、ふ.える、ふ.やす	増強(ぞうきょう): 
\\	増進(ぞうしん): 
\\	増やす(ふやす): 
\\	増し(まし): 
\\	激増(げきぞう): 
\\	増加(ぞうか): 
\\	増減(ぞうげん): 
\\	増大(ぞうだい): 
\\	増す(ます): 
\\	増える(ふえる): 
\\	増える (ふ.える), 増やす (ふ.やす), 増す (ま.す)
\\	贈			
\\	ゾウ、ソウ	おく.る	寄贈(きそう): 
\\	贈る(おくる): 
\\	贈り物(おくりもの): 
\\	贈与 (ぞうよ), 贈呈 (ぞうてい), 贈答 (ぞうとう), 贈る (おく.る)
\\	東		
\\	トウ 東北 とうほく 
\\	東京 とうきょう
\\	凍506 (チン):陳1301 (レン): 練1343 錬2030.	
\\	トウ	ひがし	東(あずま): 
\\	関東(かんとう): 
\\	東西(とうざい): 
\\	東洋(とうよう): 
\\	東(ひがし): 
\\	東 (ひがし)
\\	棟			
\\	トウ	むね、むな-	棟(とう): 
\\	棟 (むね)
\\	凍			
\\	トウ	こお.る、こご.える、こご.る、い.てる、し.みる	凍る(こおる): 
\\	凍える(こごえる): 
\\	冷凍(れいとう): 
\\	凍る (こお.る), 凍える (こご.える)
\\	妊			
\\	ニン、ジン	はら.む、みごも.る	妊娠(にんしん): 
\\	廷			
\\	テイ		法廷(ほうてい): 
\\	染		
\\	セン	そ.める、-ぞ.め、-ぞめ、そ.まる、し.みる、-じ.みる、し.み、-し.める	感染(かんせん): 
\\	染みる(しみる): 
\\	染まる(そまる): 
\\	染める(そめる): 
\\	汚染(おせん): 
\\	伝染(でんせん): 
\\	染み (し.み), 染みる (し.みる), 染まる (そ.まる), 染める (そ.める)
\\	燃		
\\	ネン	も.える、も.やす、も.す	燃焼(ねんしょう): 
\\	燃料(ねんりょう): 
\\	燃える(もえる): 
\\	燃やす(もやす): 
\\	燃える (も.える), 燃す (も.す), 燃やす (も.やす)
\\	賓		
\\	ヒン			
\\	歳		
\\	サイ、セイ	とし、とせ、よわい	歳(さい): 
\\	万歳(ばんざい): 
\\	二十歳(はたち): 
\\	歳末 (さいまつ), 歳月 (さいげつ), 二十歳 (はたち)
\\	県		
\\	ケン	か.ける	県(けん): 
\\	県庁(けんちょう): 
\\	栃		
\\	とち		栃 (とち)
\\	地		
\\	チ、ジ		天地(あめつち): 
\\	意地(いじ): 
\\	位地(いち): 
\\	見地(けんち): 
\\	現地(げんち): 
\\	心地(ここち): 
\\	地方(じかた): 
\\	地形(じぎょう): 
\\	地獄(じごく): 
\\	下地(したじ): 
\\	地主(じぬし): 
\\	地元(じもと): 
\\	植民地(しょくみんち): 
\\	農地(のうち): 
\\	墓地(はかち): 
\\	余地(よち): 
\\	領地(りょうち): 
\\	意地悪(いじわる): 
\\	各地(かくち): 
\\	生地(きじ): 
\\	基地(きち): 
\\	耕地(こうち): 
\\	産地(さんち): 
\\	敷地(しきち): 
\\	地盤(じばん): 
\\	地味(じみ): 
\\	団地(だんち): 
\\	地(ち): 
\\	地位(ちい): 
\\	地域(ちいき): 
\\	地下(ちか): 
\\	地下水(ちかすい): 
\\	地球(ちきゅう): 
\\	地区(ちく): 
\\	地質(ちしつ): 
\\	地帯(ちたい): 
\\	地点(ちてん): 
\\	地平線(ちへいせん): 
\\	地方(ちほう): 
\\	地名(ちめい): 
\\	土地(とち): 
\\	番地(ばんち): 
\\	盆地(ぼんち): 
\\	無地(むじ): 
\\	遊園地(ゆうえんち): 
\\	地震(じしん): 
\\	地理(ちり): 
\\	地下鉄(ちかてつ): 
\\	地図(ちず): 
\\	地下 (ちか), 天地 (てんち), 境地 (きょうち)
\\	池		
\\	チ	いけ	乾電池(かんでんち): 
\\	電池(でんち): 
\\	池(いけ): 
\\	池 (いけ)
\\	虫		
\\	チュウ、キ	むし	昆虫(こんちゅう): 
\\	虫歯(むしば): 
\\	虫(むし): 
\\	虫 (むし)
\\	蛍		
\\	ケイ	ほたる	蛍光灯(けいこうとう): 
\\	蛍 (ほたる)
\\	蛇		
\\	ジャ、ダ、イ、ヤ	へび	蛇口(じゃぐち): 
\\	蛇の目 (へびのめ), 蛇腹 (じゃばら), 大蛇 (おろち), 蛇 (へび)
\\	虹		
\\	コウ	にじ	虹(にじ): 
\\	虹 (にじ)
\\	蝶		
\\	チョウ		蝶(ちょう): 
\\	独		
\\	ドク、トク	ひと.り	孤独(こどく): 
\\	単独(たんどく): 
\\	独裁(どくさい): 
\\	独自(どくじ): 
\\	独占(どくせん): 
\\	独創(どくそう): 
\\	独身(どくしん): 
\\	独特(どくとく): 
\\	独立(どくりつ): 
\\	独り(ひとり): 
\\	独り言(ひとりごと): 
\\	独り (ひと.り)
\\	蚕		
\\	サン、テン	かいこ、こ		蚕 (かいこ)
\\	風		
\\	フウ、フ	かぜ、かざ-、-かぜ	風車(かざぐるま): 
\\	気風(きふう): 
\\	風習(ふうしゅう): 
\\	風俗(ふうぞく): 
\\	風土(ふうど): 
\\	暴風(ぼうふう): 
\\	洋風(ようふう): 
\\	和風(わふう): 
\\	風邪(かぜ): 
\\	扇風機(せんぷうき): 
\\	風景(ふうけい): 
\\	風船(ふうせん): 
\\	風呂(ふろ): 
\\	風呂敷(ふろしき): 
\\	台風(たいふう): 
\\	風(かぜ): 
\\	風力 (ふうりょく), 風俗 (ふうぞく), 強風 (きょうふう), 風 (かぜ)
\\	己		
\\	コ、キ	おのれ、つちのと、な	自己(じこ): 
\\	自己 (じこ), 利己 (りこ), 己 (おのれ)
\\	起		
\\	キ	お.きる、お.こる、お.こす、おこ.す、た.つ	起源(きげん): 
\\	起点(きてん): 
\\	起伏(きふく): 
\\	引き起こす(ひきおこす): 
\\	起す(おこす): 
\\	起こる(おこる): 
\\	起床(きしょう): 
\\	起こす(おこす): 
\\	起きる(おきる): 
\\	起きる (お.きる), 起こす (お.こす), 起こる (お.こる)
\\	妃		
\\	ヒ	きさき		
\\	改		
\\	こわいですね!.	
\\	カイ	あらた.める、あらた.まる	改まる(あらたまる): 
\\	改悪(かいあく): 
\\	改革(かいかく): 
\\	改修(かいしゅう): 
\\	改定(かいてい): 
\\	改訂(かいてい): 
\\	改良(かいりょう): 
\\	改めて(あらためて): 
\\	改める(あらためる): 
\\	改札(かいさつ): 
\\	改正(かいせい): 
\\	改善(かいぜん): 
\\	改造(かいぞう): 
\\	改まる (あらた.まる), 改める (あらた.める)
\\	記		
\\	キ	しる.す	記載(きさい): 
\\	記述(きじゅつ): 
\\	記名(きめい): 
\\	記す(しるす): 
\\	暗記(あんき): 
\\	記憶(きおく): 
\\	記号(きごう): 
\\	記事(きじ): 
\\	記者(きしゃ): 
\\	記入(きにゅう): 
\\	記念(きねん): 
\\	記録(きろく): 
\\	伝記(でんき): 
\\	筆記(ひっき): 
\\	日記(にっき): 
\\	記す (しる.す)
\\	包		
\\	ホウ	つつ.む、くる.む	包む(くるむ): 
\\	小包(こづつみ): 
\\	包み(つつみ): 
\\	包装(ほうそう): 
\\	包帯(ほうたい): 
\\	包む(つつむ): 
\\	包む (つつ.む)
\\	胞		
\\	ホウ		細胞(さいぼう): 
\\	砲		
\\	ホウ		鉄砲(てっぽう): 
\\	泡		
\\	ホウ	あわ	泡(あわ): 
\\	泡 (あわ)
\\	亀		
\\	⻲, 
\\	キ、キュウ、キン	かめ		亀 (かめ)
\\	電		
\\	電話 でんわ 
\\	デン		電源(でんげん): 
\\	電線(でんせん): 
\\	乾電池(かんでんち): 
\\	停電(ていでん): 
\\	電球(でんきゅう): 
\\	電子(でんし): 
\\	電池(でんち): 
\\	電柱(でんちゅう): 
\\	電波(でんぱ): 
\\	電流(でんりゅう): 
\\	電力(でんりょく): 
\\	発電(はつでん): 
\\	電灯(でんとう): 
\\	電報(でんぽう): 
\\	電気(でんき): 
\\	電車(でんしゃ): 
\\	電話(でんわ): 
\\	竜		
\\	リュウ、リョウ、ロウ	たつ、いせ		
\\	滝		
\\	ロウ、ソウ	たき	滝(たき): 
\\	滝 (たき)
\\	豚		
\\	トン	ぶた	豚肉(ぶたにく): 
\\	豚 (ぶた)
\\	逐		
\\	チク			
\\	遂		
\\	スイ	と.げる、つい.に	遂げる(とげる): 
\\	やり遂げる(やりとげる): 
\\	遂に(ついに): 
\\	遂げる (と.げる)
\\	家		
\\	カ、ケ	いえ、や、うち	家出(いえで): 
\\	家主(いえぬし): 
\\	家計(かけい): 
\\	家畜(かちく): 
\\	家来(けらい): 
\\	実業家(じつぎょうか): 
\\	一家(いっか): 
\\	大家(おおや): 
\\	家屋(かおく): 
\\	画家(がか): 
\\	家具(かぐ): 
\\	家事(かじ): 
\\	貸家(かしや): 
\\	家庭(かてい): 
\\	国家(こっか): 
\\	作家(さっか): 
\\	農家(のうか): 
\\	家賃(やちん): 
\\	家主(やぬし): 
\\	家(いえ): 
\\	家族(かぞく): 
\\	家内(かない): 
\\	家屋 (かおく), 家庭 (かてい), 作家 (さっか), 家 (いえ), 家 (や)
\\	嫁			
\\	カ	よめ、とつ.ぐ、い.く、ゆ.く	花嫁(はなよめ): 
\\	嫁(よめ): 
\\	嫁ぐ (とつ.ぐ), 嫁 (よめ)
\\	豪		
\\	ゴウ	えら.い	富豪(ふごう): 
\\	豪華(ごうか): 
\\	腸		
\\	チョウ	はらわた	腸(ちょう): 
\\	場		
\\	ジョウ、チョウ	ば	現場(げんじょう): 
\\	式場(しきじょう): 
\\	相場(そうば): 
\\	道場(どうじょう): 
\\	農場(のうじょう): 
\\	本場(ほんば): 
\\	満場(まんじょう): 
\\	役場(やくば): 
\\	来場(らいじょう): 
\\	市場(いちば): 
\\	劇場(げきじょう): 
\\	現場(げんば): 
\\	工場(こうば): 
\\	酒場(さかば): 
\\	立場(たちば): 
\\	登場(とうじょう): 
\\	入場(にゅうじょう): 
\\	場(ば): 
\\	場所(ばしょ): 
\\	場面(ばめん): 
\\	広場(ひろば): 
\\	牧場(ぼくじょう): 
\\	売り場(うりば): 
\\	会場(かいじょう): 
\\	工場(こうじょう): 
\\	駐車場(ちゅうしゃじょう): 
\\	場合(ばあい): 
\\	飛行場(ひこうじょう): 
\\	場 (ば)
\\	湯		
\\	トウ	ゆ	茶の湯(ちゃのゆ): 
\\	熱湯(ねっとう): 
\\	湯気(ゆげ): 
\\	湯飲み(ゆのみ): 
\\	湯(ゆ): 
\\	湯 (ゆ)
\\	羊		
\\	⺷. 
\\	ヨウ	ひつじ	羊毛(ようもう): 
\\	羊 (ひつじ)
\\	美			
\\	ビ、ミ	うつく.しい	美しい(うつくしい): 
\\	華美(かび): 
\\	賛美(さんび): 
\\	美(び): 
\\	美術(びじゅつ): 
\\	褒美(ほうび): 
\\	優美(ゆうび): 
\\	美人(びじん): 
\\	美容(びよう): 
\\	美しい(うつくしい): 
\\	美術館(びじゅつかん): 
\\	美しい (うつく.しい)
\\	洋			
\\	ヨウ		洋風(ようふう): 
\\	海洋(かいよう): 
\\	東洋(とうよう): 
\\	洋品店(ようひんてん): 
\\	西洋(せいよう): 
\\	洋服(ようふく): 
\\	詳			
\\	ショウ	くわ.しい、つまび.らか	詳細(しょうさい): 
\\	詳しい(くわしい): 
\\	詳しい (くわ.しい)
\\	鮮			
\\	セン	あざ.やか	鮮やか(あざやか): 
\\	新鮮(しんせん): 
\\	鮮やか (あざ.やか)
\\	達			
\\	タツ、ダ	-たち	達者(たっしゃ): 
\\	達成(たっせい): 
\\	伝達(でんたつ): 
\\	到達(とうたつ): 
\\	上達(じょうたつ): 
\\	速達(そくたつ): 
\\	達する(たっする): 
\\	配達(はいたつ): 
\\	発達(はったつ): 
\\	友達(ともだち): 
\\	羨			
\\	セン、エン	うらや.む、あまり	羨ましい(うらやましい): 
\\	羨む(うらやむ): 
\\	羨ましい (うらや.ましい), 羨む (うらや.む)
\\	差			
\\	サ	さ.す、さ.し	格差(かくさ): 
\\	誤差(ごさ): 
\\	差異(さい): 
\\	差額(さがく): 
\\	差し掛かる(さしかかる): 
\\	差し出す(さしだす): 
\\	差し支える(さしつかえる): 
\\	差し引く(さしひく): 
\\	時差(じさ): 
\\	指差す(ゆびさす): 
\\	交差(こうさ): 
\\	交差点(こうさてん): 
\\	差(さ): 
\\	差し支え(さしつかえ): 
\\	差し引き(さしひき): 
\\	差す(さす): 
\\	差別(さべつ): 
\\	人差指(ひとさしゆび): 
\\	物差し(ものさし): 
\\	差し上げる(さしあげる): 
\\	差す (さ.す)
\\	着			
\\	チャク、ジャク	き.る、-ぎ、き.せる、-き.せ、つ.く、つ.ける	落ち着き(おちつき): 
\\	着飾る(きかざる): 
\\	執着(しゅうじゃく): 
\\	先着(せんちゃく): 
\\	辿り着く(たどりつく): 
\\	着(ちゃく): 
\\	着手(ちゃくしゅ): 
\\	着色(ちゃくしょく): 
\\	着席(ちゃくせき): 
\\	着目(ちゃくもく): 
\\	着陸(ちゃくりく): 
\\	着工(ちゃっこう): 
\\	着ける(つける): 
\\	落着く(おちつく): 
\\	着替え(きがえ): 
\\	着せる(きせる): 
\\	着々(ちゃくちゃく): 
\\	到着(とうちゃく): 
\\	肌着(はだぎ): 
\\	着物(きもの): 
\\	下着(したぎ): 
\\	上着(うわぎ): 
\\	着る(きる): 
\\	着く(つく): 
\\	着用 (ちゃくよう), 着手 (ちゃくしゅ), 土着 (どちゃく), 着せる (き.せる), 着る (き.る), 着く (つ.く), 着ける (つ.ける)
\\	唯			
\\	ユイ、イ	ただ	唯(ただ): 
\\	唯(たった): 
\\	唯一(ゆいいつ): 
\\	唯一 (ゆいいつ), 唯物論 (ゆいぶつろん), 唯美主義 (ただびしゅぎ)
\\	焦			
\\	ショウ	こ.げる、こ.がす、こ.がれる、あせ.る	焦る(あせる): 
\\	焦げ茶(こげちゃ): 
\\	焦がす(こがす): 
\\	焦げる(こげる): 
\\	焦点(しょうてん): 
\\	焦る (あせ.る), 焦がす (こ.がす), 焦がれる (こ.がれる), 焦げる (こ.げる)
\\	礁			
\\	ショウ			
\\	集			
\\	シュウ	あつ.まる、あつ.める、つど.う	集まり(あつまり): 
\\	群集(ぐんしゅう): 
\\	採集(さいしゅう): 
\\	集計(しゅうけい): 
\\	収集(しゅうしゅう): 
\\	特集(とくしゅう): 
\\	密集(みっしゅう): 
\\	集会(しゅうかい): 
\\	集金(しゅうきん): 
\\	集合(しゅうごう): 
\\	集団(しゅうだん): 
\\	集中(しゅうちゅう): 
\\	全集(ぜんしゅう): 
\\	編集(へんしゅう): 
\\	募集(ぼしゅう): 
\\	集まる(あつまる): 
\\	集める(あつめる): 
\\	集まる (あつ.まる), 集める (あつ.める), 集う (つど.う)
\\	准			
\\	ジュン			
\\	進			
\\	シン	すす.む、すす.める	行進(こうしん): 
\\	昇進(しょうしん): 
\\	進化(しんか): 
\\	進行(しんこう): 
\\	進出(しんしゅつ): 
\\	進呈(しんてい): 
\\	進展(しんてん): 
\\	進度(しんど): 
\\	進路(しんろ): 
\\	推進(すいしん): 
\\	進み(すすみ): 
\\	増進(ぞうしん): 
\\	促進(そくしん): 
\\	進学(しんがく): 
\\	進歩(しんぽ): 
\\	進む(すすむ): 
\\	進める(すすめる): 
\\	前進(ぜんしん): 
\\	進む (すす.む), 進める (すす.める)
\\	雑			
\\	ザツ、ゾウ	まじ.える、まじ.る	雑(ざつ): 
\\	雑貨(ざっか): 
\\	雑談(ざつだん): 
\\	雑木(ざつぼく): 
\\	混雑(こんざつ): 
\\	雑音(ざつおん): 
\\	雑巾(ぞうきん): 
\\	複雑(ふくざつ): 
\\	雑誌(ざっし): 
\\	雑談 (ざつだん), 雑音 (ざつおん), 混雑 (こんざつ)
\\	雌			
\\	シ	め-、めす、めん	雌(めす): 
\\	雌 (め), 雌 (めす)
\\	準			
\\	ジュン	じゅん.じる、じゅん.ずる、なぞら.える、のり、ひと.しい、みずもり	基準(きじゅん): 
\\	準急(じゅんきゅう): 
\\	準じる(じゅんじる): 
\\	準ずる(じゅんずる): 
\\	規準(きじゅん): 
\\	水準(すいじゅん): 
\\	標準(ひょうじゅん): 
\\	準備(じゅんび): 
\\	奮			
\\	フン	ふる.う	興奮(こうふん): 
\\	奮闘(ふんとう): 
\\	奮う (ふる.う)
\\	奪			
\\	ダツ	うば.う	略奪(りゃくだつ): 
\\	奪う(うばう): 
\\	奪う (うば.う)
\\	確			
\\	(一石二鳥 
\\	カク、コウ	たし.か、たし.かめる	確信(かくしん): 
\\	確定(かくてい): 
\\	確保(かくほ): 
\\	確立(かくりつ): 
\\	確り(しっかり): 
\\	的確(てきかく): 
\\	適確(てきかく): 
\\	確実(かくじつ): 
\\	確認(かくにん): 
\\	確率(かくりつ): 
\\	正確(せいかく): 
\\	確かめる(たしかめる): 
\\	明確(めいかく): 
\\	確か(たしか): 
\\	確か (たし.か), 確かめる (たし.かめる)
\\	午			
\\	ゴ	うま	正午(しょうご): 
\\	午後(ごご): 
\\	午前(ごぜん): 
\\	許			
\\	キョ	ゆる.す、もと	許容(きょよう): 
\\	特許(とっきょ): 
\\	許可(きょか): 
\\	免許(めんきょ): 
\\	許す(ゆるす): 
\\	許す (ゆる.す)
\\	歓			
\\	カン	よろこ.ぶ	歓声(かんせい): 
\\	歓迎(かんげい): 
\\	権			
\\	ケン、ゴン	おもり、かり、はか.る	棄権(きけん): 
\\	権威(けんい): 
\\	権(けん): 
\\	権限(けんげん): 
\\	権力(けんりょく): 
\\	主権(しゅけん): 
\\	政権(せいけん): 
\\	特権(とっけん): 
\\	権利(けんり): 
\\	権利 (けんり), 権威 (けんい), 人権 (じんけん)
\\	観			
\\	カン	み.る、しめ.す	外観(がいかん): 
\\	観(かん): 
\\	観衆(かんしゅう): 
\\	観点(かんてん): 
\\	観覧(かんらん): 
\\	客観(きゃっかん): 
\\	主観(しゅかん): 
\\	悲観(ひかん): 
\\	楽観(らっかん): 
\\	観客(かんきゃく): 
\\	観光(かんこう): 
\\	観察(かんさつ): 
\\	観測(かんそく): 
\\	観念(かんねん): 
\\	羽			
\\	ョョ 
\\	ウ	は、わ、はね	羽(はね): 
\\	羽根(はね): 
\\	羽 (は), 羽 (はね)
\\	習			
\\	シュウ、ジュ	なら.う、なら.い	演習(えんしゅう): 
\\	慣習(かんしゅう): 
\\	教習(きょうしゅう): 
\\	講習(こうしゅう): 
\\	風習(ふうしゅう): 
\\	学習(がくしゅう): 
\\	自習(じしゅう): 
\\	実習(じっしゅう): 
\\	習字(しゅうじ): 
\\	習慣(しゅうかん): 
\\	復習(ふくしゅう): 
\\	予習(よしゅう): 
\\	習う(ならう): 
\\	練習(れんしゅう): 
\\	習う (なら.う)
\\	翌			
\\	ヨク			
\\	曜			
\\	ヨウ		火曜(かよう): 
\\	金曜(きんよう): 
\\	月曜(げつよう): 
\\	水曜(すいよう): 
\\	土曜(どよう): 
\\	木曜(もくよう): 
\\	曜日(ようび): 
\\	火曜日(かようび): 
\\	金曜日(きんようび): 
\\	月曜日(げつようび): 
\\	水曜日(すいようび): 
\\	土曜日(どようび): 
\\	日曜日(にちようび): 
\\	木曜日(もくようび): 
\\	濯			
\\	タク	すす.ぐ、ゆす.ぐ	濯ぐ(すすぐ): 
\\	洗濯(せんたく): 
\\	曰			
\\	エツ	いわ.く、のたま.う、のたま.わく、ここに、ひらび		
\\	困			
\\	コン	こま.る	貧困(ひんこん): 
\\	困難(こんなん): 
\\	困る(こまる): 
\\	困る (こま.る)
\\	固			
\\	コ	かた.める、かた.まる、かた.まり、かた.い	固める(かためる): 
\\	頑固(がんこ): 
\\	固体(こたい): 
\\	固定(こてい): 
\\	固有(こゆう): 
\\	固い(かたい): 
\\	固まる(かたまる): 
\\	固い (かた.い), 固まる (かた.まる), 固める (かた.める)
\\	国			
\\	コク	くに	国境(くにざかい): 
\\	国産(こくさん): 
\\	国定(こくてい): 
\\	国土(こくど): 
\\	国防(こくぼう): 
\\	国有(こくゆう): 
\\	国連(こくれん): 
\\	国交(こっこう): 
\\	天国(てんごく): 
\\	国王(こくおう): 
\\	国語(こくご): 
\\	国籍(こくせき): 
\\	国民(こくみん): 
\\	国立(こくりつ): 
\\	国家(こっか): 
\\	国会(こっかい): 
\\	国境(こっきょう): 
\\	全国(ぜんこく): 
\\	国際(こくさい): 
\\	外国(がいこく): 
\\	外国人(がいこくじん): 
\\	国(くに): 
\\	国 (くに)
\\	団			
\\	(だん) 
\\	ダン、トン	かたまり、まる.い	団扇(うちわ): 
\\	劇団(げきだん): 
\\	公団(こうだん): 
\\	団結(だんけつ): 
\\	座布団(ざぶとん): 
\\	集団(しゅうだん): 
\\	団体(だんたい): 
\\	団地(だんち): 
\\	布団(ふとん): 
\\	団結 (だんけつ), 団地 (だんち), 集団 (しゅうだん)
\\	因			
\\	イン	よ.る、ちな.む	要因(よういん): 
\\	因る(よる): 
\\	原因(げんいん): 
\\	因る (よ.る)
\\	姻			
\\	イン			
\\	園			
\\	エン	その	園(えん): 
\\	園(その): 
\\	田園(でんえん): 
\\	園芸(えんげい): 
\\	遊園地(ゆうえんち): 
\\	幼稚園(ようちえん): 
\\	動物園(どうぶつえん): 
\\	公園(こうえん): 
\\	園 (その)
\\	回			
\\	次回 
\\	毎回 
\\	回る 
\\	カイ、エ	まわ.る、-まわ.る、-まわ.り、まわ.す、-まわ.す、まわ.し-、-まわ.し、もとお.る、か.える	後回し(あとまわし): 
\\	上回る(うわまわる): 
\\	回収(かいしゅう): 
\\	回送(かいそう): 
\\	回覧(かいらん): 
\\	回路(かいろ): 
\\	掻き回す(かきまわす): 
\\	手回し(てまわし): 
\\	転回(てんかい): 
\\	遠回り(とおまわり): 
\\	ねじ回し(ねじまわし): 
\\	根回し(ねまわし): 
\\	回(かい): 
\\	回数(かいすう): 
\\	回数券(かいすうけん): 
\\	回転(かいてん): 
\\	回答(かいとう): 
\\	回復(かいふく): 
\\	今回(こんかい): 
\\	回す(まわす): 
\\	回り(まわり): 
\\	回り道(まわりみち): 
\\	回る(まわる): 
\\	回答 (かいとう), 転回 (てんかい), 次回 (じかい), 回す (まわ.す), 回る (まわ.る)
\\	壇			
\\	ダン、タン		花壇(かだん): 
\\	壇(だん): 
\\	壇上 (だんじょう), 花壇 (かだん), 文壇 (ぶんだん)
\\	店			
\\	テン	みせ、たな	店(てん): 
\\	支店(してん): 
\\	商店(しょうてん): 
\\	書店(しょてん): 
\\	売店(ばいてん): 
\\	店屋(みせや): 
\\	洋品店(ようひんてん): 
\\	店員(てんいん): 
\\	喫茶店(きっさてん): 
\\	店(みせ): 
\\	店 (みせ)
\\	庫			
\\	コ、ク	くら	金庫(かねぐら): 
\\	在庫(ざいこ): 
\\	金庫(きんこ): 
\\	車庫(しゃこ): 
\\	倉庫(そうこ): 
\\	冷蔵庫(れいぞうこ): 
\\	倉庫 (そうこ), 文庫 (ぶんこ), 車庫 (しゃこ)
\\	庭			
\\	庭 (にわ): 
\\	庭園 (ていえん): 
\\	校庭 (こうてい): 
\\	テイ	にわ	家庭(かてい): 
\\	校庭(こうてい): 
\\	庭(にわ): 
\\	庭 (にわ)
\\	庁			
\\	チョウ、テイ	やくしょ	庁(ちょう): 
\\	官庁(かんちょう): 
\\	県庁(けんちょう): 
\\	床			
\\	ショウ	とこ、ゆか	床(とこ): 
\\	床屋(とこや): 
\\	起床(きしょう): 
\\	床の間(とこのま): 
\\	床(ゆか): 
\\	床 (とこ), 床 (ゆか)
\\	麻			
\\	マ、マア	あさ	麻(あさ): 
\\	麻酔(ますい): 
\\	麻痺(まひ): 
\\	麻 (あさ)
\\	磨			
\\	マ	みが.く、す.る	歯磨き(はみがき): 
\\	磨く(みがく): 
\\	磨く (みが.く)
\\	心			
\\	⺖, 
\\	⺗, 
\\	シン 安心 あんしん
\\	シン	こころ、-ごころ、りっしんべん	一心(いっしん): 
\\	肝心(かんじん): 
\\	心地(ここち): 
\\	心得(こころえ): 
\\	心掛け(こころがけ): 
\\	心掛ける(こころがける): 
\\	心強い(こころづよい): 
\\	心細い(こころぼそい): 
\\	自尊心(じそんしん): 
\\	下心(したごころ): 
\\	心中(しんじゅう): 
\\	心情(しんじょう): 
\\	心配(しんぱい): 
\\	真心(まこころ): 
\\	野心(やしん): 
\\	良心(りょうしん): 
\\	感心(かんしん): 
\\	関心(かんしん): 
\\	苦心(くしん): 
\\	決心(けっしん): 
\\	心当たり(こころあたり): 
\\	心得る(こころえる): 
\\	心身(しんしん): 
\\	心臓(しんぞう): 
\\	心理(しんり): 
\\	中心(ちゅうしん): 
\\	都心(としん): 
\\	用心(ようじん): 
\\	安心(あんしん): 
\\	心(こころ): 
\\	熱心(ねっしん): 
\\	心 (こころ)
\\	忘			
\\	ボウ	わす.れる	度忘れ(どわすれ): 
\\	忘れ物(わすれもの): 
\\	忘れる(わすれる): 
\\	忘れる (わす.れる)
\\	忍			
\\	忍者 
\\	ニン	しの.ぶ、しの.ばせる		忍ばせる (しの.ばせる), 忍ぶ (しの.ぶ)
\\	認			
\\	ニン	みと.める、したた.める	公認(こうにん): 
\\	認める(したためる): 
\\	認識(にんしき): 
\\	確認(かくにん): 
\\	承認(しょうにん): 
\\	認める(みとめる): 
\\	認める (みと.める)
\\	忌			
\\	キ	い.む、い.み、い.まわしい		忌まわしい (い.まわしい), 忌む (い.む)
\\	志			
\\	シ、シリング	こころざ.す、こころざし	志(こころざし): 
\\	志す(こころざす): 
\\	志向(しこう): 
\\	志望(しぼう): 
\\	同志(どうし): 
\\	意志(いし): 
\\	志す (こころざ.す), 志 (こころざし)
\\	誌			
\\	シ		雑誌(ざっし): 
\\	忠			
\\	チュウ		忠告(ちゅうこく): 
\\	忠実(ちゅうじつ): 
\\	串			
\\	カン、ケン、セン	くし、つらぬ.く		串 (くし)
\\	患			
\\	カン	わずら.う	患者(かんじゃ): 
\\	患う (わずら.う)
\\	思			
\\	シ	おも.う、おもえら.く、おぼ.す	思い付き(おもいつき): 
\\	片思い(かたおもい): 
\\	思考(しこう): 
\\	意思(いし): 
\\	思い掛けない(おもいがけない): 
\\	思い込む(おもいこむ): 
\\	思い付く(おもいつく): 
\\	思い出(おもいで): 
\\	思わず(おもわず): 
\\	思想(しそう): 
\\	不思議(ふしぎ): 
\\	思い出す(おもいだす): 
\\	思う(おもう): 
\\	思う (おも.う)
\\	恩			
\\	オン		恩(おん): 
\\	恩恵(おんけい): 
\\	応			
\\	オウ、ヨウ、ノウ	あた.る、まさに、こた.える	応急(おうきゅう): 
\\	応募(おうぼ): 
\\	相応(そうおう): 
\\	対応(たいおう): 
\\	適応(てきおう): 
\\	反応(はんのう): 
\\	相応しい(ふさわしい): 
\\	一応(いちおう): 
\\	応援(おうえん): 
\\	応じる(おうじる): 
\\	応ずる(おうずる): 
\\	応接(おうせつ): 
\\	応対(おうたい): 
\\	応用(おうよう): 
\\	意			
\\	イ		意気込む(いきごむ): 
\\	意向(いこう): 
\\	意地(いじ): 
\\	意図(いと): 
\\	意欲(いよく): 
\\	決意(けつい): 
\\	好意(こうい): 
\\	合意(ごうい): 
\\	他意(たい): 
\\	同意(どうい): 
\\	熱意(ねつい): 
\\	不意(ふい): 
\\	無意味(むいみ): 
\\	意外(いがい): 
\\	意義(いぎ): 
\\	意思(いし): 
\\	意志(いし): 
\\	意識(いしき): 
\\	意地悪(いじわる): 
\\	敬意(けいい): 
\\	得意(とくい): 
\\	生意気(なまいき): 
\\	意見(いけん): 
\\	注意(ちゅうい): 
\\	用意(ようい): 
\\	意味(いみ): 
\\	想			
\\	ソウ、ソ	おも.う	愛想(あいそ): 
\\	可哀想(かわいそう): 
\\	構想(こうそう): 
\\	予想(よそう): 
\\	感想(かんそう): 
\\	空想(くうそう): 
\\	思想(しそう): 
\\	想像(そうぞう): 
\\	発想(はっそう): 
\\	理想(りそう): 
\\	連想(れんそう): 
\\	想像 (そうぞう), 感想 (かんそう), 予想 (よそう)
\\	息			
\\	ソク	いき	子息(しそく): 
\\	消息(しょうそく): 
\\	窒息(ちっそく): 
\\	一息(ひといき): 
\\	利息(りそく): 
\\	息(いき): 
\\	休息(きゅうそく): 
\\	溜息(ためいき): 
\\	息子(むすこ): 
\\	息 (いき)
\\	憩			
\\	ケイ	いこ.い、いこ.う	休憩(きゅうけい): 
\\	憩い (いこ.い), 憩う (いこ.う)
\\	恵			
\\	エ 知恵 ちえ
\\	ゴ 互恵 ごけい
\\	ケイ、エ	めぐ.む、めぐ.み	恵み(めぐみ): 
\\	恵む(めぐむ): 
\\	恩恵(おんけい): 
\\	知恵(ちえ): 
\\	恵まれる(めぐまれる): 
\\	恵贈 (けいぞう), 恵与 (けいよ), 恩恵 (おんけい), 恵む (めぐ.む)
\\	恐			
\\	キョウ	おそ.れる、おそ.る、おそ.ろしい、こわ.い、こわ.がる	恐らく(おそらく): 
\\	恐れ(おそれ): 
\\	恐れ入る(おそれいる): 
\\	恐れる(おそれる): 
\\	恐ろしい(おそろしい): 
\\	恐縮(きょうしゅく): 
\\	恐怖(きょうふ): 
\\	恐れる (おそ.れる), 恐ろしい (おそ.ろしい)
\\	惑			
\\	ワク	まど.う	疑惑(ぎわく): 
\\	誘惑(ゆうわく): 
\\	惑星(わくせい): 
\\	迷惑(めいわく): 
\\	惑う (まど.う)
\\	感			
\\	カン		感慨(かんがい): 
\\	感触(かんしょく): 
\\	感染(かんせん): 
\\	感度(かんど): 
\\	感無量(かんむりょう): 
\\	共感(きょうかん): 
\\	直感(ちょっかん): 
\\	痛感(つうかん): 
\\	同感(どうかん): 
\\	鈍感(どんかん): 
\\	反感(はんかん): 
\\	敏感(びんかん): 
\\	予感(よかん): 
\\	感覚(かんかく): 
\\	感激(かんげき): 
\\	感じ(かんじ): 
\\	感謝(かんしゃ): 
\\	感情(かんじょう): 
\\	感じる(かんじる): 
\\	感心(かんしん): 
\\	感ずる(かんずる): 
\\	感想(かんそう): 
\\	感動(かんどう): 
\\	実感(じっかん): 
\\	憂			
\\	ユウ	うれ.える、うれ.い、う.い、う.き	憂鬱(ゆううつ): 
\\	憂い (う.い), 憂い (うれ.い), 憂える (うれ.える)
\\	寡			
\\	カ			
\\	忙			
\\	ボウ、モウ	いそが.しい、せわ.しい、おそ.れる、うれえるさま	多忙(たぼう): 
\\	忙しい(いそがしい): 
\\	忙しい (いそが.しい)
\\	悦			
\\	エツ	よろこ.ぶ、よろこ.ばす		
\\	恒			
\\	コウ	つね、つねに		
\\	悼			
\\	トウ	いた.む		悼む (いた.む)
\\	悟			
\\	ゴ	さと.る	悟る(さとる): 
\\	覚悟(かくご): 
\\	悟る (さと.る)
\\	怖			
\\	フ、ホ	こわ.い、こわ.がる、お.じる、おそ.れる	恐怖(きょうふ): 
\\	怖い(こわい): 
\\	怖い (こわ.い)
\\	慌			
\\	コウ	あわ.てる、あわ.ただしい	慌ただしい(あわただしい): 
\\	慌てる(あわてる): 
\\	慌ただしい (あわ.ただしい), 慌てる (あわ.てる)
\\	悔			
\\	カイ	く.いる、く.やむ、くや.しい	後悔(こうかい): 
\\	悔しい(くやしい): 
\\	悔やむ(くやむ): 
\\	悔いる (く.いる), 悔やむ (く.やむ), 悔しい (くや.しい)
\\	憎			
\\	ゾウ	にく.む、にく.い、にく.らしい、にく.しみ	愛憎(あいにく): 
\\	憎しみ(にくしみ): 
\\	憎い(にくい): 
\\	憎む(にくむ): 
\\	憎らしい(にくらしい): 
\\	憎い (にく.い), 憎しみ (にく.しみ), 憎む (にく.む), 憎らしい (にく.らしい)
\\	慣			
\\	カン	な.れる、な.らす	慣用(かんよう): 
\\	慣行(かんこう): 
\\	慣習(かんしゅう): 
\\	慣例(かんれい): 
\\	慣らす(ならす): 
\\	慣れ(なれ): 
\\	見慣れる(みなれる): 
\\	習慣(しゅうかん): 
\\	慣れる(なれる): 
\\	慣らす (な.らす), 慣れる (な.れる)
\\	愉			
\\	ユ	たの.しい、たの.しむ	愉快(ゆかい): 
\\	惰			
\\	ダ			
\\	慎			
\\	シン	つつし.む、つつし、つつし.み	慎む(つつしむ): 
\\	慎重(しんちょう): 
\\	慎む (つつし.む)
\\	憾			
\\	カン	うら.む		
\\	憶			
\\	オク		記憶(きおく): 
\\	慕			
\\	ボ	した.う	慕う(したう): 
\\	慕う (した.う)
\\	添			
\\	テン 添付 てんぷ
\\	テン	そ.える、そ.う、も.える、も.う	添う(そう): 
\\	添える(そえる): 
\\	添う (そ.う), 添える (そ.える)
\\	必			
\\	ソ 
\\	ヒツ	かなら.ず	必修(ひっしゅう): 
\\	必然(ひつぜん): 
\\	必ずしも(かならずしも): 
\\	必死(ひっし): 
\\	必需品(ひつじゅひん): 
\\	必ず(かならず): 
\\	必要(ひつよう): 
\\	必ず (かなら.ず)
\\	泌			
\\	ヒツ、ヒ		泌み泌み(しみじみ): 
\\	分泌 (ぶんぴつ)
\\	手			
\\	千一
\\	シュ、ズ	て、て-、-て、た-	上手(うわて): 
\\	お手上げ(おてあげ): 
\\	小切手(こぎって): 
\\	手芸(しゅげい): 
\\	手法(しゅほう): 
\\	着手(ちゃくしゅ): 
\\	手当て(てあて): 
\\	手遅れ(ておくれ): 
\\	手掛かり(てがかり): 
\\	手掛ける(てがける): 
\\	手数(てかず): 
\\	手軽(てがる): 
\\	手際(てぎわ): 
\\	手順(てじゅん): 
\\	手錠(てじょう): 
\\	手近(てぢか): 
\\	手配(てはい): 
\\	手筈(てはず): 
\\	手引き(てびき): 
\\	手本(てほん): 
\\	手回し(てまわし): 
\\	手元(てもと): 
\\	手分け(てわけ): 
\\	取っ手(とって): 
\\	土手(どて): 
\\	入手(にゅうしゅ): 
\\	相手(あいて): 
\\	握手(あくしゅ): 
\\	お手伝いさん(おてつだいさん): 
\\	歌手(かしゅ): 
\\	手術(しゅじゅつ): 
\\	手段(しゅだん): 
\\	助手(じょしゅ): 
\\	選手(せんしゅ): 
\\	手洗い(てあらい): 
\\	手入れ(ていれ): 
\\	手首(てくび): 
\\	手頃(てごろ): 
\\	手品(てじな): 
\\	手帳(てちょう): 
\\	手伝い(てつだい): 
\\	手続き(てつづき): 
\\	手拭い(てぬぐい): 
\\	手間(てま): 
\\	手前(てまえ): 
\\	苦手(にがて): 
\\	拍手(はくしゅ): 
\\	派手(はで): 
\\	運転手(うんてんしゅ): 
\\	手伝う(てつだう): 
\\	手袋(てぶくろ): 
\\	お手洗い(おてあらい): 
\\	切手(きって): 
\\	上手(じょうず): 
\\	手(て): 
\\	手紙(てがみ): 
\\	下手(へた): 
\\	手 (て)
\\	看			
\\	カン	み.る	看護(かんご): 
\\	看板(かんばん): 
\\	看病(かんびょう): 
\\	看護婦(かんごふ): 
\\	摩			
\\	マ	ま.する、さ.する、す.る	摩擦(まさつ): 
\\	我			
\\	ガ	われ、わ、わ.が-、わが-	怪我(けが): 
\\	自我(じが): 
\\	我がまま(わがまま): 
\\	我慢(がまん): 
\\	我々(われわれ): 
\\	我 (わ), 我 (われ)
\\	義			
\\	ギ		義理(ぎり): 
\\	主義(しゅぎ): 
\\	正義(せいぎ): 
\\	定義(ていぎ): 
\\	意義(いぎ): 
\\	義務(ぎむ): 
\\	講義(こうぎ): 
\\	議			
\\	ギ		異議(いぎ): 
\\	議案(ぎあん): 
\\	議決(ぎけつ): 
\\	議事堂(ぎじどう): 
\\	議題(ぎだい): 
\\	協議(きょうぎ): 
\\	決議(けつぎ): 
\\	抗議(こうぎ): 
\\	合議(ごうぎ): 
\\	参議院(さんぎいん): 
\\	衆議院(しゅうぎいん): 
\\	審議(しんぎ): 
\\	討議(とうぎ): 
\\	物議(ぶつぎ): 
\\	論議(ろんぎ): 
\\	議員(ぎいん): 
\\	議会(ぎかい): 
\\	議長(ぎちょう): 
\\	議論(ぎろん): 
\\	不思議(ふしぎ): 
\\	会議(かいぎ): 
\\	犠			
\\	ギ、キ	いけにえ	犠牲(ぎせい): 
\\	抹			
\\	マツ			
\\	抱			
\\	ホウ	だ.く、いだ.く、かか.える	介抱(かいほう): 
\\	辛抱(しんぼう): 
\\	抱く(いだく): 
\\	抱える(かかえる): 
\\	抱く (いだ.く), 抱える (かか.える), 抱く (だ.く)
\\	搭			
\\	トウ			
\\	抄			
\\	ショウ			
\\	抗			
\\	コウ		抗議(こうぎ): 
\\	抗争(こうそう): 
\\	対抗(たいこう): 
\\	抵抗(ていこう): 
\\	反抗(はんこう): 
\\	批			
\\	ヒ		批判(ひはん): 
\\	批評(ひひょう): 
\\	招			
\\	ショウ	まね.く	招き(まねき): 
\\	招く(まねく): 
\\	招待(しょうたい): 
\\	招く (まね.く)
\\	拓			
\\	タク	ひら.く	開拓(かいたく): 
\\	拍			
\\	ハク、ヒョウ		拍手(はくしゅ): 
\\	拍手 (はくしゅ), 拍車 (はくしゃ), 一拍 (いっぱく)
\\	打			
\\	ダ、ダアス	う.つ、う.ち-、ぶ.つ	打ち合わせ(うちあわせ): 
\\	打ち合わせる(うちあわせる): 
\\	打ち切る(うちきる): 
\\	打ち消し(うちけし): 
\\	打ち込む(うちこむ): 
\\	打開(だかい): 
\\	打撃(だげき): 
\\	値打ち(ねうち): 
\\	打付ける(ぶつける): 
\\	打合せ(うちあわせ): 
\\	打ち消す(うちけす): 
\\	打つ(ぶつ): 
\\	打つ(うつ): 
\\	打つ (う.つ)
\\	拘			
\\	コウ	かか.わる	拘束(こうそく): 
\\	にも拘らず(にもかかわらず): 
\\	捨			
\\	シャ	す.てる	四捨五入(ししゃごにゅう): 
\\	捨てる(すてる): 
\\	捨てる (す.てる)
\\	拐			
\\	カイ		拐う(さらう): 
\\	摘			
\\	テキ	つ.む	指摘(してき): 
\\	摘む(つまむ): 
\\	摘む (つ.む)
\\	挑			
\\	チョウ	いど.む	挑む(いどむ): 
\\	挑戦(ちょうせん): 
\\	挑む (いど.む)
\\	指			
\\	シ	ゆび、さ.す、-さ.し	指図(さしず): 
\\	指す(さす): 
\\	指揮(しき): 
\\	指示(しじ): 
\\	指摘(してき): 
\\	指令(しれい): 
\\	中指(ちゅうし): 
\\	指差す(ゆびさす): 
\\	親指(おやゆび): 
\\	薬指(くすりゆび): 
\\	小指(こゆび): 
\\	指定(してい): 
\\	指導(しどう): 
\\	中指(なかゆび): 
\\	人差指(ひとさしゆび): 
\\	目指す(めざす): 
\\	指(ゆび): 
\\	指輪(ゆびわ): 
\\	指す (さ.す), 指 (ゆび)
\\	持			
\\	ジ	も.つ、-も.ち、も.てる	金持ち(かねもち): 
\\	支持(しじ): 
\\	持続(じぞく): 
\\	所持(しょじ): 
\\	持ち(もち): 
\\	持ち切り(もちきり): 
\\	持て成す(もてなす): 
\\	持てる(もてる): 
\\	維持(いじ): 
\\	受け持つ(うけもつ): 
\\	持参(じさん): 
\\	持ち上げる(もちあげる): 
\\	気持ち(きもち): 
\\	持つ(もつ): 
\\	持つ (も.つ)
\\	括			
\\	カツ	くく.る	一括(いっかつ): 
\\	括弧(かっこ): 
\\	揮			
\\	キ	ふる.う	指揮(しき): 
\\	発揮(はっき): 
\\	推			
\\	スイ	お.す	推進(すいしん): 
\\	推測(すいそく): 
\\	推理(すいり): 
\\	類推(るいすい): 
\\	推薦(すいせん): 
\\	推定(すいてい): 
\\	推す (お.す)
\\	揚			
\\	ヨウ	あ.げる、-あ.げ、あ.がる		揚がる (あ.がる), 揚げる (あ.げる)
\\	提			
\\	テイ、チョウ、ダイ	さ.げる	前提(ぜんてい): 
\\	提供(ていきょう): 
\\	提携(ていけい): 
\\	提示(ていじ): 
\\	提案(ていあん): 
\\	提出(ていしゅつ): 
\\	提げる (さ.げる)
\\	損			
\\	ソン	そこ.なう、そこな.う、-そこ.なう、そこ.ねる、-そこ.ねる	損なう(そこなう): 
\\	損失(そんしつ): 
\\	破損(はそん): 
\\	損(そん): 
\\	損害(そんがい): 
\\	損得(そんとく): 
\\	損なう (そこ.なう), 損ねる (そこ.ねる)
\\	拾			
\\	シュウ、ジュウ	ひろ.う	拾う(ひろう): 
\\	拾得 (しゅうとく), 収拾 (しゅうしゅう), 拾う (ひろ.う)
\\	担			
\\	タン	かつ.ぐ、にな.う	担架(たんか): 
\\	担う(になう): 
\\	負担(ふたん): 
\\	分担(ぶんたん): 
\\	担ぐ(かつぐ): 
\\	担当(たんとう): 
\\	担ぐ (かつ.ぐ), 担う (にな.う)
\\	拠			
\\	キョ、コ	よ.る	根拠(こんきょ): 
\\	証拠(しょうこ): 
\\	拠点 (きょてん), 占拠 (せんきょ), 根拠 (こんきょ)
\\	描			
\\	ビョウ	えが.く、か.く	描写(びょうしゃ): 
\\	描く(えがく): 
\\	描く (えが.く)
\\	操			
\\	ソウ、サン	みさお、あやつ.る	操る(あやつる): 
\\	操縦(そうじゅう): 
\\	操作(そうさ): 
\\	体操(たいそう): 
\\	操る (あやつ.る), 操 (みさお)
\\	接			
\\	セツ、ショウ	つ.ぐ	接触(せっしょく): 
\\	接続詞(せつぞくし): 
\\	接ぐ(つぐ): 
\\	密接(みっせつ): 
\\	応接(おうせつ): 
\\	間接(かんせつ): 
\\	接近(せっきん): 
\\	接する(せっする): 
\\	接続(せつぞく): 
\\	直接(ちょくせつ): 
\\	面接(めんせつ): 
\\	接ぐ (つ.ぐ)
\\	掲			
\\	ケイ	かか.げる	掲げる(かかげる): 
\\	掲載(けいさい): 
\\	掲示(けいじ): 
\\	掲げる (かか.げる)
\\	掛			
\\	カイ、ケイ	か.ける、-か.ける、か.け、-か.け、-が.け、か.かる、-か.かる、-が.かる、か.かり、-が.かり、かかり、-がかり	掛かる(かかる): 
\\	掛け(かけ): 
\\	切っ掛け(きっかけ): 
\\	心掛け(こころがけ): 
\\	心掛ける(こころがける): 
\\	差し掛かる(さしかかる): 
\\	仕掛け(しかけ): 
\\	仕掛ける(しかける): 
\\	手掛かり(てがかり): 
\\	手掛ける(てがける): 
\\	引っ掛ける(ひっかける): 
\\	見掛ける(みかける): 
\\	寄り掛かる(よりかかる): 
\\	追い掛ける(おいかける): 
\\	お出掛け(おでかけ): 
\\	お目に掛かる(おめにかかる): 
\\	思い掛けない(おもいがけない): 
\\	掛け算(かけざん): 
\\	掛ける(かける): 
\\	腰掛け(こしかけ): 
\\	腰掛ける(こしかける): 
\\	出掛ける(でかける): 
\\	通り掛かる(とおりかかる): 
\\	話し掛ける(はなしかける): 
\\	引っ掛かる(ひっかかる): 
\\	見掛け(みかけ): 
\\	呼び掛ける(よびかける): 
\\	掛かる (か.かる), 掛ける (か.ける), 掛 (かかり)
\\	研			
\\	ケン	と.ぐ	研ぐ(とぐ): 
\\	研修(けんしゅう): 
\\	研究(けんきゅう): 
\\	研究室(けんきゅうしつ): 
\\	研ぐ (と.ぐ)
\\	戒			
\\	カイ	いまし.める	警戒(けいかい): 
\\	戒める (いまし.める)
\\	械			
\\	カイ	かせ	器械(きかい): 
\\	機械(きかい): 
\\	鼻			
\\	ビ	はな	耳鼻科(じびか): 
\\	鼻(はな): 
\\	鼻 (はな)
\\	刑			
\\	ケイ		刑(けい): 
\\	刑罰(けいばつ): 
\\	死刑(しけい): 
\\	刑事(けいじ): 
\\	型			
\\	ケイ	かた、-がた	模型(もけい): 
\\	型(かた): 
\\	典型(てんけい): 
\\	型 (かた)
\\	才			
\\	何歳ですか?   。。。天才です!.	
\\	サイ		天才(てんさい): 
\\	才能(さいのう): 
\\	財			
\\	ザイ、サイ、ゾク	たから	財(ざい): 
\\	財源(ざいげん): 
\\	財政(ざいせい): 
\\	文化財(ぶんかざい): 
\\	財産(ざいさん): 
\\	財布(さいふ): 
\\	財産 (ざいさん), 私財 (しざい), 文化財 (ぶんかざい)
\\	材			
\\	ザイ		教材(きょうざい): 
\\	取材(しゅざい): 
\\	人材(じんざい): 
\\	素材(そざい): 
\\	材木(ざいもく): 
\\	材料(ざいりょう): 
\\	木材(もくざい): 
\\	存			
\\	ソン、ゾン		依存(いそん): 
\\	共存(きょうそん): 
\\	存続(そんぞく): 
\\	生存(せいぞん): 
\\	存在(そんざい): 
\\	存じる(ぞんじる): 
\\	存ずる(ぞんずる): 
\\	保存(ほぞん): 
\\	ご存じ(ごぞんじ): 
\\	存在 (そんざい), 存続 (そんぞく), 既存 (きぞん)
\\	在			
\\	ザイ	あ.る	健在(けんざい): 
\\	在庫(ざいこ): 
\\	自在(じざい): 
\\	所在(しょざい): 
\\	不在(ふざい): 
\\	在る(ある): 
\\	現在(げんざい): 
\\	在学(ざいがく): 
\\	存在(そんざい): 
\\	滞在(たいざい): 
\\	在る (あ.る)
\\	乃			
\\	ナイ、ダイ、ノ、アイ	の、すなわ.ち、なんじ	乃至(ないし): 
\\	携			
\\	ケイ	たずさ.える、たずさ.わる	携帯(けいたい): 
\\	携わる(たずさわる): 
\\	提携(ていけい): 
\\	携える (たずさ.える), 携わる (たずさ.わる)
\\	及			
\\	キュウ	およ.ぶ、およ.び、および、およ.ぼす	及び(および): 
\\	及ぶ(およぶ): 
\\	追及(ついきゅう): 
\\	及ぼす(およぼす): 
\\	普及(ふきゅう): 
\\	及び (およ.び), 及ぶ (およ.ぶ), 及ぼす (およ.ぼす)
\\	吸			
\\	キュウ	す.う	吸収(きゅうしゅう): 
\\	呼吸(こきゅう): 
\\	吸う(すう): 
\\	吸う (す.う)
\\	扱			
\\	ソウ、キュウ	あつか.い、あつか.う、あつか.る、こ.く	扱い(あつかい): 
\\	取り扱い(とりあつかい): 
\\	取り扱う(とりあつかう): 
\\	扱う(あつかう): 
\\	扱う (あつか.う)
\\	丈			
\\	ジョウ	たけ、だけ	頑丈(がんじょう): 
\\	丈夫(じょうふ): 
\\	丈(たけ): 
\\	丈(だけ): 
\\	成る丈(なるたけ): 
\\	丈夫(じょうぶ): 
\\	大丈夫(だいじょうぶ): 
\\	丈 (たけ)
\\	史			
\\	シ		女史(じょし): 
\\	歴史(れきし): 
\\	吏			
\\	リ		下吏(かり): 
\\	捕吏(ほり): 
\\	更			
\\	コウ	さら、さら.に、ふ.ける、ふ.かす	今更(いまさら): 
\\	尚更(なおさら): 
\\	夜更かし(よふかし): 
\\	夜更け(よふけ): 
\\	更に(さらに): 
\\	更ける(ふける): 
\\	変更(へんこう): 
\\	更 (さら), 更かす (ふ.かす), 更ける (ふ.ける)
\\	硬			
\\	コウ	かた.い	強硬(きょうこう): 
\\	硬い(かたい): 
\\	硬貨(こうか): 
\\	硬い (かた.い)
\\	又			
\\	""じゃ又
\\	""又ね
\\	""又明日""。 
\\	又明日.
\\	ユウ	また、また-、また.の-	又(また): 
\\	又は(または): 
\\	又 (また)
\\	双			
\\	ソウ	ふた、たぐい、ならぶ、ふたつ	双子(ふたご): 
\\	双 (ふた)
\\	桑			
\\	ソウ	くわ		桑 (くわ)
\\	隻			
\\	セキ			
\\	護			
\\	ゴ	まも.る	介護(かいご): 
\\	看護(かんご): 
\\	護衛(ごえい): 
\\	弁護(べんご): 
\\	保護(ほご): 
\\	養護(ようご): 
\\	看護婦(かんごふ): 
\\	獲			
\\	カク	え.る	獲物(えもの): 
\\	獲得(かくとく): 
\\	捕獲(ほかく): 
\\	獲る (え.る)
\\	奴			
\\	ド	やつ、やっこ	奴(やっこ): 
\\	怒			
\\	ド、ヌ	いか.る、おこ.る	怒り(いかり): 
\\	怒る(いかる): 
\\	怒鳴る(どなる): 
\\	怒る(おこる): 
\\	怒る (いか.る), 怒る (おこ.る)
\\	友			
\\	セフレ, 
\\	ユウ	とも	友(とも): 
\\	親友(しんゆう): 
\\	友好(ゆうこう): 
\\	友情(ゆうじょう): 
\\	友人(ゆうじん): 
\\	友達(ともだち): 
\\	友 (とも)
\\	抜			
\\	バツ、ハツ、ハイ	ぬ.く、-ぬ.く、ぬ.き、ぬ.ける、ぬ.かす、ぬ.かる	海抜(かいばつ): 
\\	抜かす(ぬかす): 
\\	抜け出す(ぬけだす): 
\\	抜く(ぬく): 
\\	抜ける(ぬける): 
\\	抜かす (ぬ.かす), 抜かる (ぬ.かる), 抜く (ぬ.く), 抜ける (ぬ.ける)
\\	投			
\\	トウ	な.げる、-な.げ	投資(とうし): 
\\	投入(とうにゅう): 
\\	投げ出す(なげだす): 
\\	投書(とうしょ): 
\\	投票(とうひょう): 
\\	投げる(なげる): 
\\	投げる (な.げる)
\\	没			
\\	ボツ、モツ	おぼ.れる、しず.む、ない	沈没(ちんぼつ): 
\\	没収(ぼっしゅう): 
\\	没落(ぼつらく): 
\\	設			
\\	セツ	もう.ける	施設(しせつ): 
\\	設置(せっち): 
\\	設定(せってい): 
\\	設立(せつりつ): 
\\	設ける(もうける): 
\\	建設(けんせつ): 
\\	設計(せっけい): 
\\	設備(せつび): 
\\	設ける (もう.ける)
\\	撃			
\\	ゲキ	う.つ	襲撃(しゅうげき): 
\\	衝撃(しょうげき): 
\\	打撃(だげき): 
\\	反撃(はんげき): 
\\	撃つ(うつ): 
\\	攻撃(こうげき): 
\\	撃つ (う.つ)
\\	殻			
\\	甲殻 卵殻 貝殻.	
\\	カク、コク、バイ	から、がら	貝殻(かいがら): 
\\	殻(から): 
\\	殻 (から)
\\	支			
\\	シ	ささ.える、つか.える、か.う、しんよう、じゅうまた	差し支える(さしつかえる): 
\\	支持(しじ): 
\\	収支(しゅうし): 
\\	支える(ささえる): 
\\	差し支え(さしつかえ): 
\\	支給(しきゅう): 
\\	支出(ししゅつ): 
\\	支度(したく): 
\\	支店(してん): 
\\	支配(しはい): 
\\	支払(しはらい): 
\\	支払う(しはらう): 
\\	支える (ささ.える)
\\	技			
\\	ギ	わざ	技能(ぎのう): 
\\	特技(とくぎ): 
\\	技(わざ): 
\\	演技(えんぎ): 
\\	技師(ぎし): 
\\	競技(きょうぎ): 
\\	技術(ぎじゅつ): 
\\	技 (わざ)
\\	枝			
\\	(支店), 
\\	木 
\\	シ	えだ	枝(えだ): 
\\	枝 (えだ)
\\	肢			
\\	シ			
\\	茎			
\\	ケイ、キョウ	くき	茎(くき): 
\\	茎 (くき)
\\	怪			
\\	カイ、ケ	あや.しい、あや.しむ	怪獣(かいじゅう): 
\\	怪我(けが): 
\\	怪しい(あやしい): 
\\	怪しい (あや.しい), 怪しむ (あや.しむ)
\\	軽			
\\	ケイ	かる.い、かろ.やか、かろ.んじる	気軽(きがる): 
\\	軽快(けいかい): 
\\	軽減(けいげん): 
\\	軽率(けいそつ): 
\\	軽蔑(けいべつ): 
\\	手軽(てがる): 
\\	軽い(かるい): 
\\	軽い (かる.い), 軽やか (かろ.やか)
\\	叔			
\\	叔 
\\	(寂), 
\\	(淑) 
\\	(督). 
\\	(叔) 
\\	シュク		叔父(おじ): 
\\	督			
\\	叔 
\\	(寂), 
\\	(淑) 
\\	(督). 
\\	(叔) 
\\	トク		監督(かんとく): 
\\	寂			
\\	叔 
\\	(寂), 
\\	(淑) 
\\	(督). 
\\	(叔) 
\\	ジャク、セキ	さび、さび.しい、さび.れる、さみ.しい	寂しい(さびしい): 
\\	寂滅 (じゃくめつ), 静寂 (せいじゃく), 閑寂 (かんじゃく), 寂 (さび), 寂しい (さび.しい), 寂れる (さび.れる)
\\	淑			
\\	叔 
\\	(寂), 
\\	(淑) 
\\	(督). 
\\	(叔) 
\\	シュク	しと.やか		
\\	反			
\\	ハン、ホン、タン、ホ	そ.る、そ.らす、かえ.す、かえ.る、-かえ.る	反り(そり): 
\\	反(たん): 
\\	反感(はんかん): 
\\	反響(はんきょう): 
\\	反撃(はんげき): 
\\	反射(はんしゃ): 
\\	反する(はんする): 
\\	反応(はんのう): 
\\	反発(はんぱつ): 
\\	反乱(はんらん): 
\\	違反(いはん): 
\\	反る(かえる): 
\\	反映(はんえい): 
\\	反抗(はんこう): 
\\	反省(はんせい): 
\\	反対(はんたい): 
\\	反映 (はんえい), 反対 (はんたい), 違反 (いはん), 謀反 (むほん), 反らす (そ.らす), 反る (そ.る)
\\	坂			
\\	ハン	さか	坂(さか): 
\\	坂 (さか)
\\	板			
\\	ハン、バン	いた	板(いた): 
\\	看板(かんばん): 
\\	黒板(こくばん): 
\\	乾板 (かんばん), 鉄板 (てっぱん), 板 (いた)
\\	返			
\\	ヘン	かえ.す、-かえ.す、かえ.る、-かえ.る	裏返し(うらがえし): 
\\	折り返す(おりかえす): 
\\	返る(かえる): 
\\	宙返り(ちゅうがえり): 
\\	照り返す(てりかえす): 
\\	返還(へんかん): 
\\	返済(へんさい): 
\\	返事(へんじ): 
\\	返答(へんとう): 
\\	裏返す(うらがえす): 
\\	繰り返す(くりかえす): 
\\	引返す(ひきかえす): 
\\	引っ繰り返す(ひっくりかえす): 
\\	引っ繰り返る(ひっくりかえる): 
\\	返す(かえす): 
\\	返す (かえ.す), 返る (かえ.る)
\\	販			
\\	ハン		販売(はんばい): 
\\	爪			
\\	⺥.	
\\	ソウ	つめ、つま-	爪(つめ): 
\\	爪 (つめつま)
\\	妥			
\\	ダ		妥協(だきょう): 
\\	妥結(だけつ): 
\\	妥当(だとう): 
\\	乳			
\\	ニュウ	ちち、ち	乳(ちち): 
\\	牛乳(ぎゅうにゅう): 
\\	乳 (ち), 乳 (ちち)
\\	浮			
\\	フ	う.く、う.かれる、う.かぶ、む、う.かべる	浮かぶ(うかぶ): 
\\	浮気(うわき): 
\\	浮力(ふりょく): 
\\	浮ぶ(うかぶ): 
\\	浮かべる(うかべる): 
\\	浮く(うく): 
\\	浮かぶ (う.かぶ), 浮かべる (う.かべる), 浮かれる (う.かれる), 浮く (う.く)
\\	将			
\\	ショウ、ソウ	まさ.に、はた、まさ、ひきい.る、もって	将棋(しょうぎ): 
\\	将来(しょうらい): 
\\	奨			
\\	ショウ、ソウ	すす.める	奨励(しょうれい): 
\\	奨学金(しょうがくきん): 
\\	採			
\\	采 
\\	サイ	と.る	採掘(さいくつ): 
\\	採決(さいけつ): 
\\	採算(さいさん): 
\\	採集(さいしゅう): 
\\	採択(さいたく): 
\\	採用(さいよう): 
\\	採点(さいてん): 
\\	採る(とる): 
\\	採る (と.る)
\\	菜			
\\	サイ	な	お菜(おかず): 
\\	野菜(やさい): 
\\	菜 (な)
\\	受			
\\	ジュ	う.ける、-う.け、う.かる	受かる(うかる): 
\\	受け入れ(うけいれ): 
\\	受け入れる(うけいれる): 
\\	受け継ぐ(うけつぐ): 
\\	受け付ける(うけつける): 
\\	受け止める(うけとめる): 
\\	受け取り(うけとり): 
\\	受身(うけみ): 
\\	享受(きょうじゅ): 
\\	引き受ける(ひきうける): 
\\	受け取る(うけとる): 
\\	受け持つ(うけもつ): 
\\	受験(じゅけん): 
\\	受話器(じゅわき): 
\\	引受る(ひきうける): 
\\	受付(うけつけ): 
\\	受ける(うける): 
\\	受かる (う.かる), 受ける (う.ける)
\\	授			
\\	ジュ	さず.ける、さず.かる	授ける(さずける): 
\\	教授(きょうじゅ): 
\\	助教授(じょきょうじゅ): 
\\	授業(じゅぎょう): 
\\	授かる (さず.かる), 授ける (さず.ける)
\\	愛			
\\	(あい) 
\\	アイ	いと.しい	愛想(あいそ): 
\\	愛憎(あいにく): 
\\	可愛い(かわいい): 
\\	可愛がる(かわいがる): 
\\	可愛らしい(かわいらしい): 
\\	愛でたい(めでたい): 
\\	恋愛(れんあい): 
\\	愛(あい): 
\\	愛情(あいじょう): 
\\	愛する(あいする): 
\\	払			
\\	フツ、ヒツ、ホツ	はら.う、-はら.い、-ばら.い	支払(しはらい): 
\\	支払う(しはらう): 
\\	払い込む(はらいこむ): 
\\	払い戻す(はらいもどす): 
\\	酔っ払い(よっぱらい): 
\\	払う(はらう): 
\\	払う (はら.う)
\\	広			
\\	コウ	ひろ.い、ひろ.まる、ひろ.める、ひろ.がる、ひろ.げる	広まる(ひろまる): 
\\	広告(こうこく): 
\\	広がる(ひろがる): 
\\	広げる(ひろげる): 
\\	広さ(ひろさ): 
\\	広場(ひろば): 
\\	広々(ひろびろ): 
\\	広める(ひろめる): 
\\	背広(せびろ): 
\\	広い(ひろい): 
\\	広い (ひろ.い), 広がる (ひろ.がる), 広げる (ひろ.げる), 広まる (ひろ.まる), 広める (ひろ.める)
\\	拡			
\\	カク、コウ	ひろ.がる、ひろ.げる、ひろ.める	拡散(かくさん): 
\\	拡充(かくじゅう): 
\\	拡大(かくだい): 
\\	拡張(かくちょう): 
\\	鉱			
\\	コウ	あらがね	鉱業(こうぎょう): 
\\	鉱山(こうざん): 
\\	鉱物(こうぶつ): 
\\	炭鉱(たんこう): 
\\	弁			
\\	ベン、ヘン	わきま.える、わ.ける、はなびら、あらそ.う	勘弁(かんべん): 
\\	代弁(だいべん): 
\\	弁解(べんかい): 
\\	弁護(べんご): 
\\	弁償(べんしょう): 
\\	弁論(べんろん): 
\\	弁当(べんとう): 
\\	お弁当(おべんとう): 
\\	雄			
\\	ユウ	お-、おす、おん	英雄(えいゆう): 
\\	雄(おす): 
\\	雄 (お), 雄 (おす)
\\	台			
\\	ダイ、タイ	うてな、われ、つかさ	台無し(だいなし): 
\\	台本(だいほん): 
\\	土台(どだい): 
\\	寝台(しんだい): 
\\	台(だい): 
\\	灯台(とうだい): 
\\	舞台(ぶたい): 
\\	台風(たいふう): 
\\	台所(だいどころ): 
\\	台地 (だいち), 灯台 (とうだい), 一台 (いちだい)
\\	怠			
\\	タイ	おこた.る、なま.ける	怠慢(たいまん): 
\\	怠い(だるい): 
\\	怠る(おこたる): 
\\	怠ける(なまける): 
\\	怠る (おこた.る), 怠ける (なま.ける)
\\	治			
\\	ジ、チ	おさ.める、おさ.まる、なお.る、なお.す	治まる(おさまる): 
\\	退治(たいじ): 
\\	治安(ちあん): 
\\	治療(ちりょう): 
\\	統治(とうじ): 
\\	治める(おさめる): 
\\	自治(じち): 
\\	治す(なおす): 
\\	政治(せいじ): 
\\	治る(なおる): 
\\	政治 (せいじ), 療治 (りょうじ), 治まる (おさ.まる), 治める (おさ.める), 治す (なお.す), 治る (なお.る)
\\	始			
\\	シ	はじ.める、-はじ.める、はじ.まる	始発(しはつ): 
\\	始末(しまつ): 
\\	終始(しゅうし): 
\\	始め(はじめ): 
\\	始めまして(はじめまして): 
\\	開始(かいし): 
\\	原始(げんし): 
\\	始終(しじゅう): 
\\	始まり(はじまり): 
\\	始めに(はじめに): 
\\	始める(はじめる): 
\\	始まる(はじまる): 
\\	始まる (はじ.まる), 始める (はじ.める)
\\	胎			
\\	タイ			
\\	窓			
\\	宀 
\\	儿 
\\	厶 
\\	心. 
\\	穴、あな, 
\\	窓、まど, 
\\	窓口、まどぐち, 
\\	車窓、しゃそう, 
\\	ソウ、ス	まど、てんまど、けむだし	窓口(まどぐち): 
\\	窓(まど): 
\\	窓 (まど)
\\	去			
\\	キョ、コ	さ.る、-さ.る	消去(しょうきょ): 
\\	過去(かこ): 
\\	去る(さる): 
\\	去年(きょねん): 
\\	去年 (きょねん), 去就 (きょしゅう), 除去 (じょきょ), 去る (さ.る)
\\	法			
\\	ホウ、ハッ、ホッ、フラン	のり	司法(しほう): 
\\	手法(しゅほう): 
\\	寸法(すんぽう): 
\\	製法(せいほう): 
\\	法(ほう): 
\\	法案(ほうあん): 
\\	法学(ほうがく): 
\\	法廷(ほうてい): 
\\	用法(ようほう): 
\\	立法(りっぽう): 
\\	憲法(けんぽう): 
\\	作法(さほう): 
\\	法則(ほうそく): 
\\	方法(ほうほう): 
\\	文法(ぶんぽう): 
\\	法律(ほうりつ): 
\\	法律 (ほうりつ), 文法 (ぶんぽう), 方法 (ほうほう), 法度 (ほうど)
\\	会			
\\	カイ、エ	あ.う、あ.わせる、あつ.まる	会(え): 
\\	会(かい): 
\\	会見(かいけん): 
\\	会談(かいだん): 
\\	協会(きょうかい): 
\\	再会(さいかい): 
\\	座談会(ざだんかい): 
\\	総会(そうかい): 
\\	出会う(であう): 
\\	面会(めんかい): 
\\	宴会(えんかい): 
\\	会員(かいいん): 
\\	開会(かいかい): 
\\	会館(かいかん): 
\\	会計(かいけい): 
\\	会合(かいごう): 
\\	学会(がっかい): 
\\	議会(ぎかい): 
\\	国会(こっかい): 
\\	司会(しかい): 
\\	社会科学(しゃかいかがく): 
\\	集会(しゅうかい): 
\\	大会(たいかい): 
\\	出会い(であい): 
\\	都会(とかい): 
\\	閉会(へいかい): 
\\	会議(かいぎ): 
\\	会場(かいじょう): 
\\	会話(かいわ): 
\\	機会(きかい): 
\\	教会(きょうかい): 
\\	社会(しゃかい): 
\\	展覧会(てんらんかい): 
\\	会う(あう): 
\\	会社(かいしゃ): 
\\	会話 (かいわ), 会計 (かいけい), 社会 (しゃかい), 会う (あ.う)
\\	至			
\\	シ	いた.る	至って(いたって): 
\\	乃至(ないし): 
\\	至る(いたる): 
\\	至急(しきゅう): 
\\	至る (いた.る)
\\	室			
\\	シツ	むろ	室(しつ): 
\\	控室(ひかえしつ): 
\\	浴室(よくしつ): 
\\	温室(おんしつ): 
\\	待合室(まちあいしつ): 
\\	研究室(けんきゅうしつ): 
\\	教室(きょうしつ): 
\\	室 (むろ)
\\	到			
\\	トウ	いた.る	到達(とうたつ): 
\\	到底(とうてい): 
\\	到着(とうちゃく): 
\\	致			
\\	(どう致しまして) 
\\	(お願い致します; おねがいいたします). 
\\	""お願い致します
\\	""どう致しまして
\\	チ	いた.す	雅致(がち): 
\\	合致(がっち): 
\\	一致(いっち): 
\\	致す(いたす): 
\\	致す (いた.す)
\\	互			
\\	ゴ	たが.い、かたみ.に	交互(こうご): 
\\	お互い(おたがい): 
\\	相互(そうご): 
\\	互い(たがい): 
\\	互い (たが.い)
\\	棄			
\\	キ	す.てる	棄権(きけん): 
\\	廃棄(はいき): 
\\	破棄(はき): 
\\	放棄(ほうき): 
\\	棄てる(すてる): 
\\	育			
\\	イク	そだ.つ、そだ.ち、そだ.てる、はぐく.む	育成(いくせい): 
\\	飼育(しいく): 
\\	生育(せいいく): 
\\	育ち(そだち): 
\\	発育(はついく): 
\\	保育(ほいく): 
\\	育児(いくじ): 
\\	育つ(そだつ): 
\\	体育(たいいく): 
\\	教育(きょういく): 
\\	育てる(そだてる): 
\\	育つ (そだ.つ), 育てる (そだ.てる)
\\	撤			
\\	テツ			
\\	充			
\\	ジュウ 補充 ほじゅう
\\	銃762 (トウ): 統1347.	
\\	ジュウ	あ.てる、み.たす	充実(じゅうじつ): 
\\	補充(ほじゅう): 
\\	拡充(かくじゅう): 
\\	充てる (あ.てる)
\\	銃			
\\	ジュウ	つつ	銃(つつ): 
\\	銃(じゅう): 
\\	硫			
\\	リュウ			
\\	流			
\\	リュウ、ル	なが.れる、なが.れ、なが.す、-なが.す	海流(かいりゅう): 
\\	気流(きりゅう): 
\\	流し(ながし): 
\\	流れる(ながれる): 
\\	流行(はやり): 
\\	流(りゅう): 
\\	流通(りゅうつう): 
\\	一流(いちりゅう): 
\\	交流(こうりゅう): 
\\	合流(ごうりゅう): 
\\	流石(さすが): 
\\	直流(ちょくりゅう): 
\\	電流(でんりゅう): 
\\	流す(ながす): 
\\	流れ(ながれ): 
\\	流行る(はやる): 
\\	流域(りゅういき): 
\\	流行(りゅうこう): 
\\	流行 (りゅうこう), 流動 (りゅうどう), 電流 (でんりゅう), 流す (なが.す), 流れる (なが.れる)
\\	允			
\\	イン	じょう、まこと.に、ゆるす		
\\	唆			
\\	サ	そそのか.す		唆す (そそのか.す)
\\	出			
\\	シュツ、スイ	で.る、-で、だ.す、-だ.す、い.でる、い.だす	家出(いえで): 
\\	売り出し(うりだし): 
\\	売り出す(うりだす): 
\\	演出(えんしゅつ): 
\\	追い出す(おいだす): 
\\	お出でになる(おいでになる): 
\\	お目出度う(おめでとう): 
\\	差し出す(さしだす): 
\\	産出(さんしゅつ): 
\\	出演(しゅつえん): 
\\	出血(しゅっけつ): 
\\	出産(しゅっさん): 
\\	出社(しゅっしゃ): 
\\	出生(しゅっしょう): 
\\	出世(しゅっせ): 
\\	出題(しゅつだい): 
\\	出動(しゅつどう): 
\\	出費(しゅっぴ): 
\\	出品(しゅっぴん): 
\\	進出(しんしゅつ): 
\\	脱出(だっしゅつ): 
\\	出会う(であう): 
\\	出合う(であう): 
\\	出入り口(でいりぐち): 
\\	出来物(できもの): 
\\	出切る(できる): 
\\	出くわす(でくわす): 
\\	出鱈目(でたらめ): 
\\	出直し(でなおし): 
\\	投げ出す(なげだす): 
\\	逃げ出す(にげだす): 
\\	抜け出す(ぬけだす): 
\\	引き出す(ひきだす): 
\\	振り出し(ふりだし): 
\\	噴出(ふんしゅつ): 
\\	放出(ほうしゅつ): 
\\	放り出す(ほうりだす): 
\\	申出(もうしで): 
\\	申し出る(もうしでる): 
\\	言い出す(いいだす): 
\\	お出掛け(おでかけ): 
\\	思い出(おもいで): 
\\	外出(がいしゅつ): 
\\	貸し出し(かしだし): 
\\	支出(ししゅつ): 
\\	出勤(しゅっきん): 
\\	出身(しゅっしん): 
\\	出張(しゅっちょう): 
\\	出版(しゅっぱん): 
\\	出合い(であい): 
\\	出会い(であい): 
\\	提出(ていしゅつ): 
\\	出入り(でいり): 
\\	出掛ける(でかける): 
\\	出来上がり(できあがり): 
\\	出来上がる(できあがる): 
\\	出来事(できごと): 
\\	出来るだけ(できるだけ): 
\\	出迎え(でむかえ): 
\\	出迎える(でむかえる): 
\\	飛び出す(とびだす): 
\\	取り出す(とりだす): 
\\	引出す(ひきだす): 
\\	日の出(ひので): 
\\	見出し(みだし): 
\\	呼び出す(よびだす): 
\\	思い出す(おもいだす): 
\\	出席(しゅっせき): 
\\	出発(しゅっぱつ): 
\\	引き出し(ひきだし): 
\\	輸出(ゆしゅつ): 
\\	出す(だす): 
\\	出かける(でかける): 
\\	出口(でぐち): 
\\	出る(でる): 
\\	出入 (しゅつにゅう), 出現 (しゅつげん), 提出 (ていしゅつ), 出す (だ.す), 出る (で.る)
\\	山			
\\	サン、セン	やま	鉱山(こうざん): 
\\	山岳(さんがく): 
\\	山腹(さんぷく): 
\\	山脈(さんみゃく): 
\\	不山戯る(ふざける): 
\\	火山(かざん): 
\\	山林(さんりん): 
\\	登山(とざん): 
\\	山(やま): 
\\	山 (やま)
\\	拙			
\\	セツ	つたな.い		
\\	岩			
\\	ガン	いわ	岩石(がんせき): 
\\	岩(いわ): 
\\	溶岩(ようがん): 
\\	岩 (いわ)
\\	炭			
\\	火山) 
\\	タン	すみ	炭素(たんそ): 
\\	石炭(せきたん): 
\\	炭鉱(たんこう): 
\\	炭 (すみ)
\\	岐			
\\	キ、ギ			
\\	峠			
\\	とうげ	峠(とうげ): 
\\	峠 (とうげ)
\\	崩			
\\	ホウ	くず.れる、-くず.れ、くず.す	雪崩(なだれ): 
\\	崩壊(ほうかい): 
\\	崩す(くずす): 
\\	崩れる(くずれる): 
\\	崩す (くず.す), 崩れる (くず.れる)
\\	密			
\\	ミツ	ひそ.か	過密(かみつ): 
\\	厳密(げんみつ): 
\\	精密(せいみつ): 
\\	密か(ひそか): 
\\	密集(みっしゅう): 
\\	密接(みっせつ): 
\\	密度(みつど): 
\\	秘密(ひみつ): 
\\	密(みつ): 
\\	蜜			
\\	ミツ、ビツ		蜂蜜(はちみつ): 
\\	蜜(みつ): 
\\	嵐			
\\	ラン	あらし	嵐(あらし): 
\\	嵐 (あらし)
\\	崎			
\\	碕 
\\	碕, 埼 
\\	崎 
\\	キ	さき、さい、みさき		崎 (さき)
\\	入			
\\	ニュウ 入院 にゅういん
\\	ニュウ、ジュ	い.る、-い.る、-い.り、い.れる、-い.れ、はい.る	入口(いりくち): 
\\	入る(いる): 
\\	受け入れ(うけいれ): 
\\	受け入れる(うけいれる): 
\\	恐れ入る(おそれいる): 
\\	介入(かいにゅう): 
\\	加入(かにゅう): 
\\	購入(こうにゅう): 
\\	仕入れる(しいれる): 
\\	新入生(しんにゅうせい): 
\\	潜入(せんにゅう): 
\\	出入り口(でいりぐち): 
\\	投入(とうにゅう): 
\\	導入(どうにゅう): 
\\	入手(にゅうしゅ): 
\\	入賞(にゅうしょう): 
\\	入浴(にゅうよく): 
\\	納入(のうにゅう): 
\\	申し入れる(もうしいれる): 
\\	入れ物(いれもの): 
\\	気に入る(きにいる): 
\\	記入(きにゅう): 
\\	四捨五入(ししゃごにゅう): 
\\	収入(しゅうにゅう): 
\\	侵入(しんにゅう): 
\\	出入り(でいり): 
\\	手入れ(ていれ): 
\\	取り入れる(とりいれる): 
\\	入社(にゅうしゃ): 
\\	入場(にゅうじょう): 
\\	日の入り(ひのいり): 
\\	輸入(ゆにゅう): 
\\	押し入れ(おしいれ): 
\\	入院(にゅういん): 
\\	入学(にゅうがく): 
\\	入口(いりぐち): 
\\	入れる(いれる): 
\\	入る(はいる): 
\\	入る (い.る), 入れる (い.れる), 入る (はい.る)
\\	込			
\\	こ.み、 -こ.み、 -こ.む、 こ.む、 こ.める	意気込む(いきごむ): 
\\	打ち込む(うちこむ): 
\\	埋め込む(うめこむ): 
\\	追い込む(おいこむ): 
\\	押し込む(おしこむ): 
\\	落ち込む(おちこむ): 
\\	組み込む(くみこむ): 
\\	込める(こめる): 
\\	飲み込む(のみこむ): 
\\	乗り込む(のりこむ): 
\\	放り込む(ほうりこむ): 
\\	見込み(みこみ): 
\\	申し込み(もうしこみ): 
\\	割り込む(わりこむ): 
\\	思い込む(おもいこむ): 
\\	突っ込む(つっこむ): 
\\	溶け込む(とけこむ): 
\\	飛び込む(とびこむ): 
\\	払い込む(はらいこむ): 
\\	引っ込む(ひっこむ): 
\\	人込み(ひとごみ): 
\\	申し込む(もうしこむ): 
\\	込む(こむ): 
\\	込む (こ.む), 込める (こ.める)
\\	分			
\\	ブン、フン、ブ	わ.ける、わ.け、わ.かれる、わ.かる、わ.かつ	一部分(いちぶぶん): 
\\	十分(じっぷん): 
\\	処分(しょぶん): 
\\	随分(ずいぶん): 
\\	多分(たぶん): 
\\	手分け(てわけ): 
\\	取り分(とりわけ): 
\\	配分(はいぶん): 
\\	引き分け(ひきわけ): 
\\	分(ふん): 
\\	分業(ぶんぎょう): 
\\	分散(ぶんさん): 
\\	分子(ぶんし): 
\\	分担(ぶんたん): 
\\	分配(ぶんぱい): 
\\	分母(ぶんぼ): 
\\	分離(ぶんり): 
\\	分裂(ぶんれつ): 
\\	申し分(もうしぶん): 
\\	分かる(わかる): 
\\	幾分(いくぶん): 
\\	区分(くぶん): 
\\	水分(すいぶん): 
\\	成分(せいぶん): 
\\	大分(だいぶ): 
\\	大部分(だいぶぶん): 
\\	大分(だいぶん): 
\\	等分(とうぶん): 
\\	何分(なにぶん): 
\\	引分け(ひきわけ): 
\\	分(ぶ): 
\\	分野(ぶんや): 
\\	部分(ぶぶん): 
\\	分(ぶん): 
\\	分解(ぶんかい): 
\\	分数(ぶんすう): 
\\	分析(ぶんせき): 
\\	分布(ぶんぷ): 
\\	分量(ぶんりょう): 
\\	分類(ぶんるい): 
\\	身分(みぶん): 
\\	養分(ようぶん): 
\\	余分(よぶん): 
\\	分かれる(わかれる): 
\\	分ける(わける): 
\\	気分(きぶん): 
\\	十分(じゅうぶん): 
\\	自分(じぶん): 
\\	半分(はんぶん): 
\\	分る(わかる): 
\\	分解 (ぶんかい), 自分 (じぶん), 水分 (すいぶん), 分別 (ふんべつ), 分銅 (ふんどう), 三十分 (さんじゅうふん), 分かつ (わ.かつ), 分かる (わ.かる), 分かれる (わ.かれる), 分ける (わ.ける)
\\	貧			
\\	ヒン、ビン	まず.しい	貧困(ひんこん): 
\\	貧弱(ひんじゃく): 
\\	貧乏(びんぼう): 
\\	貧しい(まずしい): 
\\	貧富 (ひんぷ), 貧弱 (ひんじゃく), 清貧 (せいひん), 貧しい (まず.しい)
\\	頒			
\\	ハン	わか.つ		
\\	公			
\\	コウ、ク	おおやけ	公(おおやけ): 
\\	公演(こうえん): 
\\	公開(こうかい): 
\\	公然(こうぜん): 
\\	公団(こうだん): 
\\	公認(こうにん): 
\\	公募(こうぼ): 
\\	公務(こうむ): 
\\	公用(こうよう): 
\\	公立(こうりつ): 
\\	主人公(しゅじんこう): 
\\	公害(こうがい): 
\\	公共(こうきょう): 
\\	公式(こうしき): 
\\	公衆(こうしゅう): 
\\	公正(こうせい): 
\\	公表(こうひょう): 
\\	公平(こうへい): 
\\	公務員(こうむいん): 
\\	公園(こうえん): 
\\	公 (おおやけ)
\\	松			
\\	ショウ	まつ	松(まつ): 
\\	松 (まつ)
\\	翁			
\\	オウ	おきな		
\\	訟			
\\	ショウ		訴訟(そしょう): 
\\	谷			
\\	風の谷のナウシカ 
\\	コク	たに、きわ.まる	谷(たに): 
\\	谷 (たに)
\\	浴			
\\	混浴 (こんよく) 
\\	(温泉), 
\\	谷間.	
\\	ヨク	あ.びる、あ.びせる	浴びる(あびる): 
\\	入浴(にゅうよく): 
\\	浴室(よくしつ): 
\\	海水浴(かいすいよく): 
\\	浴衣(ゆかた): 
\\	浴びせる (あ.びせる), 浴びる (あ.びる)
\\	容			
\\	ヨウ	い.れる	"寛容(かんよう): 
\\	許容(きょよう): 
\\	収容(しゅうよう): 
\\	容易い(たやすい): 
\\	形容詞(けいようし): 
\\	形容動詞(けいようどうし): 
\\	内容(ないよう): 
\\	美容(びよう): 
\\	容易(ようい): 
\\	容器(ようき): 
\\	容積(ようせき): 
\\	溶			
\\	ヨウ	と.ける、と.かす、と.く	溶液(ようえき): 
\\	溶かす(とかす): 
\\	溶く(とく): 
\\	溶け込む(とけこむ): 
\\	溶ける(とける): 
\\	溶岩(ようがん): 
\\	溶かす (と.かす), 溶く (と.く), 溶ける (と.ける)
\\	欲			
\\	性欲; 
\\	ヨク	ほっ.する、ほ.しい	意欲(いよく): 
\\	欲深い(よくふかい): 
\\	欲望(よくぼう): 
\\	食欲(しょくよく): 
\\	欲張り(よくばり): 
\\	欲しい(ほしい): 
\\	欲しい (ほ.しい), 欲する (ほっ.する)
\\	裕			
\\	ユウ		余裕(よゆう): 
\\	鉛			
\\	エン	なまり	鉛(なまり): 
\\	鉛筆(えんぴつ): 
\\	鉛 (なまり)
\\	沿			
\\	エン	そ.う、-ぞ.い	沿岸(えんがん): 
\\	沿線(えんせん): 
\\	沿い(ぞい): 
\\	沿う(そう): 
\\	沿う (そ.う)
\\	賞			
\\	ショウ	ほ.める	懸賞(けんしょう): 
\\	入賞(にゅうしょう): 
\\	鑑賞(かんしょう): 
\\	賞(しょう): 
\\	賞金(しょうきん): 
\\	賞品(しょうひん): 
\\	党			
\\	トウ	なかま、むら	野党(やとう): 
\\	与党(よとう): 
\\	政党(せいとう): 
\\	党(とう): 
\\	堂			
\\	ドウ		議事堂(ぎじどう): 
\\	堂々(どうどう): 
\\	講堂(こうどう): 
\\	食堂(しょくどう): 
\\	常			
\\	ジョウ	つね、とこ-	正常(せいじょう): 
\\	通常(つうじょう): 
\\	平常(へいじょう): 
\\	異常(いじょう): 
\\	常識(じょうしき): 
\\	常に(つねに): 
\\	日常(にちじょう): 
\\	非常(ひじょう): 
\\	非常に(ひじょうに): 
\\	常 (つね), 常 (とこ)
\\	裳			
\\	ショウ	も、もすそ		
\\	掌			
\\	ショウ	てのひら、たなごころ	掌(たなごころ): 
\\	車掌(しゃしょう): 
\\	皮			
\\	彼(かれ 
\\	波(なみ 
\\	ヒ	かわ、けがわ	皮(かわ): 
\\	毛皮(けがわ): 
\\	皮肉(ひにく): 
\\	皮膚(ひふ): 
\\	皮 (かわ)
\\	波			
\\	ハ	なみ	短波(たんぱ): 
\\	津波(つなみ): 
\\	電波(でんぱ): 
\\	波(なみ): 
\\	波 (なみ)
\\	婆			
\\	バ	ばば、ばあ		
\\	披			
\\	ヒ			
\\	破			
\\	ハ	やぶ.る、やぶ.れる	突破(とっぱ): 
\\	破壊(はかい): 
\\	破棄(はき): 
\\	爆破(ばくは): 
\\	破損(はそん): 
\\	破裂(はれつ): 
\\	破産(はさん): 
\\	破片(はへん): 
\\	破く(やぶく): 
\\	破る(やぶる): 
\\	破れる(やぶれる): 
\\	破る (やぶ.る), 破れる (やぶ.れる)
\\	被			
\\	ヒ	こうむ.る、おお.う、かぶ.る、かぶ.せる	被る(かぶる): 
\\	被せる(かぶせる): 
\\	被害(ひがい): 
\\	被る (こうむ.る)
\\	残			
\\	ザン、サン	のこ.る、のこ.す、そこな.う、のこ.り	残金(ざんきん): 
\\	残酷(ざんこく): 
\\	残高(ざんだか): 
\\	名残(なごり): 
\\	残り(のこり): 
\\	残す(のこす): 
\\	残らず(のこらず): 
\\	残念(ざんねん): 
\\	残る(のこる): 
\\	残す (のこ.す), 残る (のこ.る)
\\	殉			
\\	ジュン			
\\	殊			
\\	シュ	こと	殊に(ことに): 
\\	特殊(とくしゅ): 
\\	殊 (こと)
\\	殖			
\\	ショク	ふ.える、ふ.やす	繁殖(はんしょく): 
\\	殖やす(ふやす): 
\\	殖える(ふえる): 
\\	殖える (ふ.える), 殖やす (ふ.やす)
\\	列			
\\	レツ、レ		整列(せいれつ): 
\\	陳列(ちんれつ): 
\\	配列(はいれつ): 
\\	並列(へいれつ): 
\\	行列(ぎょうれつ): 
\\	列(れつ): 
\\	列車(れっしゃ): 
\\	列島(れっとう): 
\\	裂			
\\	レツ	さ.く、さ.ける、-ぎ.れ	裂ける(さける): 
\\	破裂(はれつ): 
\\	分裂(ぶんれつ): 
\\	裂く(さく): 
\\	裂く (さ.く), 裂ける (さ.ける)
\\	烈			
\\	レツ	はげ.しい	強烈(きょうれつ): 
\\	猛烈(もうれつ): 
\\	死			
\\	シ	し.ぬ、し.に-	死(し): 
\\	死刑(しけい): 
\\	生死(せいし): 
\\	死体(したい): 
\\	死亡(しぼう): 
\\	必死(ひっし): 
\\	死ぬ(しぬ): 
\\	死ぬ (し.ぬ)
\\	葬			
\\	ソウ	ほうむ.る	葬る(ほうむる): 
\\	葬式(そうしき): 
\\	葬る (ほうむ.る)
\\	瞬			
\\	シュン	またた.く、まじろ.ぐ	瞬き(またたき): 
\\	一瞬(いっしゅん): 
\\	瞬間(しゅんかん): 
\\	瞬く (またた.く)
\\	耳			
\\	ジ	みみ	耳鼻科(じびか): 
\\	初耳(はつみみ): 
\\	耳(みみ): 
\\	耳 (みみ)
\\	取			
\\	シュ	と.る、と.り、と.り-、とり、-ど.り	受け取り(うけとり): 
\\	書き取り(かきとり): 
\\	書き取る(かきとる): 
\\	聞き取り(ききとり): 
\\	下取り(したどり): 
\\	取材(しゅざい): 
\\	塵取り(ちりとり): 
\\	取っ手(とって): 
\\	取りあえず(とりあえず): 
\\	取り扱い(とりあつかい): 
\\	取り扱う(とりあつかう): 
\\	取り替え(とりかえ): 
\\	取り組む(とりくむ): 
\\	取り締まり(とりしまり): 
\\	取り締まる(とりしまる): 
\\	取り調べる(とりしらべる): 
\\	取り立てる(とりたてる): 
\\	取り次ぐ(とりつぐ): 
\\	取り除く(とりのぞく): 
\\	取り引き(とりひき): 
\\	取り巻く(とりまく): 
\\	取り混ぜる(とりまぜる): 
\\	取り戻す(とりもどす): 
\\	取り寄せる(とりよせる): 
\\	取り分(とりわけ): 
\\	乗っ取る(のっとる): 
\\	引き取る(ひきとる): 
\\	日取り(ひどり): 
\\	受け取る(うけとる): 
\\	取り上げる(とりあげる): 
\\	取り入れる(とりいれる): 
\\	取り消す(とりけす): 
\\	取り出す(とりだす): 
\\	取れる(とれる): 
\\	取り替える(とりかえる): 
\\	取る(とる): 
\\	取る (と.る)
\\	趣			
\\	シュ	おもむき、おもむ.く	趣(おもむき): 
\\	趣旨(しゅし): 
\\	趣味(しゅみ): 
\\	趣 (おもむき)
\\	最			
\\	サイ、シュ	もっと.も、つま	最善(さいぜん): 
\\	最早(もはや): 
\\	最高(さいこう): 
\\	最終(さいしゅう): 
\\	最中(さいちゅう): 
\\	最低(さいてい): 
\\	最も(もっとも): 
\\	最近(さいきん): 
\\	最後(さいご): 
\\	最初(さいしょ): 
\\	最も (もっと.も)
\\	撮			
\\	サツ	と.る、つま.む、-ど.り	撮影(さつえい): 
\\	撮る(とる): 
\\	撮る (と.る)
\\	恥			
\\	チ	は.じる、はじ、は.じらう、は.ずかしい	恥(はじ): 
\\	恥じらう(はじらう): 
\\	恥じる(はじる): 
\\	恥ずかしい(はずかしい): 
\\	恥じらう (は.じらう), 恥じる (は.じる), 恥ずかしい (は.ずかしい), 恥 (はじ)
\\	職			
\\	ショク、ソク		教職(きょうしょく): 
\\	辞職(じしょく): 
\\	職員(しょくいん): 
\\	職務(しょくむ): 
\\	退職(たいしょく): 
\\	役職(やくしょく): 
\\	就職(しゅうしょく): 
\\	職(しょく): 
\\	職業(しょくぎょう): 
\\	職人(しょくにん): 
\\	聖			
\\	耳 
\\	口. 
\\	セイ、ショウ	ひじり	神聖(しんせい): 
\\	聖書(せいしょ): 
\\	敢			
\\	カン	あ.えて、あ.えない、あ.えず	敢えて(あえて): 
\\	勇敢(ゆうかん): 
\\	聴			
\\	チョウ、テイ	き.く、ゆる.す	聴覚(ちょうかく): 
\\	聴講(ちょうこう): 
\\	聴診器(ちょうしんき): 
\\	聴く (き.く)
\\	懐			
\\	カイ、エ	ふところ、なつ.かしい、なつ.かしむ、なつ.く、なつ.ける、なず.ける、いだ.く、おも.う	懐く(なつく): 
\\	懐かしい(なつかしい): 
\\	懐かしい (なつ.かしい), 懐かしむ (なつ.かしむ), 懐く (なつ.く), 懐ける (なつ.ける), 懐 (ふところ)
\\	慢			
\\	(曼荼羅). 
\\	マン		怠慢(たいまん): 
\\	我慢(がまん): 
\\	自慢(じまん): 
\\	漫			
\\	(漫画). 
\\	マン	みがりに	漫画(まんが): 
\\	買			
\\	バイ	か.う	購買(こうばい): 
\\	売買(ばいばい): 
\\	買い物(かいもの): 
\\	買う(かう): 
\\	買う (か.う)
\\	置			
\\	チ	お.く、-お.き	位置(いち): 
\\	処置(しょち): 
\\	設置(せっち): 
\\	措置(そち): 
\\	配置(はいち): 
\\	放置(ほうち): 
\\	前置き(まえおき): 
\\	物置き(ものおき): 
\\	装置(そうち): 
\\	物置(ものおき): 
\\	置く(おく): 
\\	置く (お.く)
\\	罰			
\\	バチ, バツ).	
\\	バツ、バチ、ハツ	ばっ.する	刑罰(けいばつ): 
\\	処罰(しょばつ): 
\\	罰(ばち): 
\\	罰する(ばっする): 
\\	罰金 (ばっきん), 処罰 (しょばつ), 天罰 (てんばつ)
\\	寧			
\\	ネイ	むし.ろ	寧ろ(むしろ): 
\\	丁寧(ていねい): 
\\	濁			
\\	し 
\\	じ 
\\	ダク、ジョク	にご.る、にご.す	清濁(せいだく): 
\\	濁る(にごる): 
\\	濁す (にご.す), 濁る (にご.る)
\\	環			
\\	カン	わ	環境(かんきょう): 
\\	循環(じゅんかん): 
\\	還			
\\	カン	かえ.る	還元(かんげん): 
\\	還暦(かんれき): 
\\	返還(へんかん): 
\\	夫			
\\	フ、フウ、ブ	おっと、そ.れ	丈夫(じょうふ): 
\\	夫(おっと): 
\\	工夫(くふう): 
\\	夫婦(ふうふ): 
\\	夫妻(ふさい): 
\\	夫人(ふじん): 
\\	丈夫(じょうぶ): 
\\	大丈夫(だいじょうぶ): 
\\	夫妻 (ふさい), 農夫 (のうふ), 凡夫 (ぼんおっと), 夫 (おっと)
\\	扶			
\\	フ	たす.ける	扶養(ふよう): 
\\	渓			
\\	ケイ	たに、たにがわ		
\\	規			
\\	キ		規格(きかく): 
\\	規制(きせい): 
\\	規定(きてい): 
\\	規範(きはん): 
\\	規模(きぼ): 
\\	規約(きやく): 
\\	正規(せいき): 
\\	規準(きじゅん): 
\\	規律(きりつ): 
\\	定規(じょうぎ): 
\\	不規則(ふきそく): 
\\	規則(きそく): 
\\	替			
\\	タイ	か.える、か.え-、か.わる	替える(かえる): 
\\	切り替える(きりかえる): 
\\	取り替え(とりかえ): 
\\	為替(かわせ): 
\\	着替え(きがえ): 
\\	交替(こうたい): 
\\	両替(りょうがえ): 
\\	取り替える(とりかえる): 
\\	替える (か.える), 替わる (か.わる)
\\	賛			
\\	サン	たす.ける、たた.える	賛美(さんび): 
\\	賛成(さんせい): 
\\	潜			
\\	セン	ひそ.む、もぐ.る、かく.れる、くぐ.る、ひそ.める	潜る(くぐる): 
\\	潜水(せんすい): 
\\	潜入(せんにゅう): 
\\	潜る(もぐる): 
\\	潜む (ひそ.む), 潜る (もぐ.る)
\\	失			
\\	シツ	うしな.う、う.せる	失格(しっかく): 
\\	失調(しっちょう): 
\\	損失(そんしつ): 
\\	紛失(ふんしつ): 
\\	失う(うしなう): 
\\	過失(かしつ): 
\\	失業(しつぎょう): 
\\	失望(しつぼう): 
\\	失恋(しつれん): 
\\	失敗(しっぱい): 
\\	失う (うしな.う)
\\	鉄			
\\	テツ	くろがね	鉄棒(かなぼう): 
\\	製鉄(せいてつ): 
\\	鉄鋼(てっこう): 
\\	鉄片(てっぺん): 
\\	私鉄(してつ): 
\\	鉄(てつ): 
\\	鉄橋(てっきょう): 
\\	鉄道(てつどう): 
\\	鉄砲(てっぽう): 
\\	地下鉄(ちかてつ): 
\\	迭			
\\	テツ			
\\	臣			
\\	シン、ジン		総理大臣(そうりだいじん): 
\\	大臣(だいじん): 
\\	臣下 (しんか), 君臣 (くんしん)
\\	姫			
\\	キ	ひめ、ひめ-		姫 (ひめ)
\\	蔵			
\\	ゾウ、ソウ	くら、おさ.める、かく.れる	蔵(くら): 
\\	蔵相(ぞうしょう): 
\\	埋蔵(まいぞう): 
\\	冷蔵(れいぞう): 
\\	貯蔵(ちょぞう): 
\\	冷蔵庫(れいぞうこ): 
\\	蔵 (くら)
\\	臓			
\\	ゾウ	はらわた	内臓(ないぞう): 
\\	心臓(しんぞう): 
\\	賢			
\\	ケン	かしこ.い	賢明(けんめい): 
\\	賢い(かしこい): 
\\	賢い (かしこ.い)
\\	堅			
\\	=ケン).	
\\	ケン	かた.い、-がた.い	堅い(かたい): 
\\	堅い (かた.い)
\\	臨			
\\	リン	のぞ.む	臨む(のぞむ): 
\\	臨時(りんじ): 
\\	臨む (のぞ.む)
\\	覧			
\\	ラン	み.る	閲覧(えつらん): 
\\	回覧(かいらん): 
\\	観覧(かんらん): 
\\	御覧なさい(ごらんなさい): 
\\	御覧(ごらん): 
\\	展覧会(てんらんかい): 
\\	巨			
\\	キョ		巨(こ): 
\\	巨大(きょだい): 
\\	拒			
\\	キョ、ゴ	こば.む	拒絶(きょぜつ): 
\\	拒否(きょひ): 
\\	拒む (こば.む)
\\	力			
\\	カ 
\\	リョク、リキ、リイ	ちから	圧力(あつりょく): 
\\	威力(いりょく): 
\\	権力(けんりょく): 
\\	勢力(せいりょく): 
\\	戦力(せんりょく): 
\\	速力(そくりょく): 
\\	体力(たいりょく): 
\\	弾力(だんりょく): 
\\	動力(どうりょく): 
\\	浮力(ふりょく): 
\\	武力(ぶりょく): 
\\	暴力(ぼうりょく): 
\\	有力(ゆうりょく): 
\\	労力(ろうりょく): 
\\	引力(いんりょく): 
\\	学力(がくりょく): 
\\	活力(かつりょく): 
\\	強力(きょうりょく): 
\\	協力(きょうりょく): 
\\	効力(こうりょく): 
\\	実力(じつりょく): 
\\	重力(じゅうりょく): 
\\	力強い(ちからづよい): 
\\	電力(でんりょく): 
\\	努力(どりょく): 
\\	能力(のうりょく): 
\\	魅力(みりょく): 
\\	力(ちから): 
\\	権力 (けんりょく), 努力 (どりょく), 能力 (のうりょく), 力 (ちから)
\\	男			
\\	ダン、ナン	おとこ、お	男の人(おとこのひと): 
\\	男子(だんし): 
\\	長男(ちょうなん): 
\\	男性(だんせい): 
\\	男(おとこ): 
\\	男の子(おとこのこ): 
\\	男子 (だんし), 男女 (だんじょ), 男性 (だんせい), 男 (おとこ)
\\	労			
\\	ロウ	ろう.する、いたわ.る、いた.ずき、ねぎら、つか.れる、ねぎら.う	労る(いたわる): 
\\	過労(かろう): 
\\	勤労(きんろう): 
\\	ご苦労様(ごくろうさま): 
\\	疲労(ひろう): 
\\	労力(ろうりょく): 
\\	苦労(くろう): 
\\	労働(ろうどう): 
\\	募			
\\	ボ	つの.る	応募(おうぼ): 
\\	公募(こうぼ): 
\\	募る(つのる): 
\\	募金(ぼきん): 
\\	募集(ぼしゅう): 
\\	募る (つの.る)
\\	劣			
\\	レツ	おと.る	劣る(おとる): 
\\	劣る (おと.る)
\\	功			
\\	コウ、ク	いさお	功績(こうせき): 
\\	成功(せいこう): 
\\	功名 (こうみょう), 功績 (こうせき), 成功 (せいこう)
\\	勧			
\\	カン 勧奨 かんしょう
\\	歓570 観572  (ケン): 権571.	
\\	カン、ケン	すす.める	勧誘(かんゆう): 
\\	勧告(かんこく): 
\\	勧め(すすめ): 
\\	勧める(すすめる): 
\\	勧める (すす.める)
\\	努			
\\	ド 努力 どりょく 
\\	奴702  怒703.	
\\	ド	つと.める	努めて(つとめて): 
\\	努める(つとめる): 
\\	努力(どりょく): 
\\	努める (つと.める)
\\	励			
\\	レイ	はげ.む、はげ.ます	激励(げきれい): 
\\	奨励(しょうれい): 
\\	励ます(はげます): 
\\	励む(はげむ): 
\\	励ます (はげ.ます), 励む (はげ.む)
\\	加			
\\	カロ
\\	カ	くわ.える、くわ.わる	いい加減(いいかげん): 
\\	加工(かこう): 
\\	加入(かにゅう): 
\\	加味(かみ): 
\\	加留多(かるた): 
\\	付け加える(つけくわえる): 
\\	加減(かげん): 
\\	加速(かそく): 
\\	加速度(かそくど): 
\\	加熱(かねつ): 
\\	加える(くわえる): 
\\	加わる(くわわる): 
\\	参加(さんか): 
\\	増加(ぞうか): 
\\	追加(ついか): 
\\	加える (くわ.える), 加わる (くわ.わる)
\\	賀			
\\	ガ		祝賀(しゅくが): 
\\	架			
\\	カ	か.ける、か.かる	担架(たんか): 
\\	架空(かくう): 
\\	架かる (か.かる), 架ける (か.ける)
\\	脇			
\\	キョウ	わき、わけ	脇(わき): 
\\	脇 (わき)
\\	脅			
\\	キョウ 脅迫 きょうはく 
\\	協872.	
\\	キョウ	おびや.かす、おど.す、おど.かす	脅かす(おどかす): 
\\	脅す(おどす): 
\\	脅迫(きょうはく): 
\\	脅かす (おど.かす), 脅す (おど.す), 脅かす (おびや.かす)
\\	協			
\\	キョウ		協会(きょうかい): 
\\	協議(きょうぎ): 
\\	協調(きょうちょう): 
\\	協定(きょうてい): 
\\	妥協(だきょう): 
\\	協力(きょうりょく): 
\\	行			
\\	コウ、ギョウ、アン	い.く、ゆ.く、-ゆ.き、-ゆき、-い.き、-いき、おこな.う、おこ.なう	行き(いき): 
\\	行き違い(いきちがい): 
\\	行き成り(いきなり): 
\\	行く(いく): 
\\	移行(いこう): 
\\	売れ行き(うれゆき): 
\\	行い(おこない): 
\\	行う(おこなう): 
\\	慣行(かんこう): 
\\	刊行(かんこう): 
\\	行(ぎょう): 
\\	強行(きょうこう): 
\\	行政(ぎょうせい): 
\\	決行(けっこう): 
\\	現行(げんこう): 
\\	行為(こうい): 
\\	行員(こういん): 
\\	行進(こうしん): 
\\	施行(しぎょう): 
\\	修行(しゅうぎょう): 
\\	徐行(じょこう): 
\\	進行(しんこう): 
\\	先行(せんこう): 
\\	走行(そうこう): 
\\	流行(はやり): 
\\	非行(ひこう): 
\\	平行(へいこう): 
\\	夜行(やぎょう): 
\\	行儀(ぎょうぎ): 
\\	行事(ぎょうじ): 
\\	行列(ぎょうれつ): 
\\	孝行(こうこう): 
\\	行動(こうどう): 
\\	実行(じっこう): 
\\	通行(つうこう): 
\\	発行(はっこう): 
\\	流行る(はやる): 
\\	飛行(ひこう): 
\\	並行(へいこう): 
\\	夜行(やこう): 
\\	流行(りゅうこう): 
\\	行う(おこなう): 
\\	急行(きゅうこう): 
\\	飛行場(ひこうじょう): 
\\	銀行(ぎんこう): 
\\	飛行機(ひこうき): 
\\	旅行(りょこう): 
\\	行進 (こうしん), 行為 (こうい), 旅行 (りょこう), 行列 (ぎょうれつ), 行政 (ぎょうせい), 修行 (しゅぎょう), 行く (い.く), 行う (おこな.う), 行く (ゆ.く)
\\	律			
\\	リツ、リチ、レツ		一律(いちりつ): 
\\	規律(きりつ): 
\\	法律(ほうりつ): 
\\	律動 (りつどう), 規律 (きりつ), 法律 (ほうりつ)
\\	復			
\\	フク	また	復旧(ふくきゅう): 
\\	復活(ふっかつ): 
\\	復興(ふっこう): 
\\	往復(おうふく): 
\\	回復(かいふく): 
\\	復習(ふくしゅう): 
\\	得			
\\	トク	え.る、う.る	獲得(かくとく): 
\\	心得(こころえ): 
\\	所得(しょとく): 
\\	説得(せっとく): 
\\	得点(とくてん): 
\\	止むを得ない(やむをえない): 
\\	得る(うる): 
\\	得る(える): 
\\	心得る(こころえる): 
\\	損得(そんとく): 
\\	得意(とくい): 
\\	納得(なっとく): 
\\	得る (う.る), 得る (え.る)
\\	従			
\\	ジュウ、ショウ、ジュ	したが.う、したが.える、より	従業員(じゅうぎょういん): 
\\	従事(じゅうじ): 
\\	従来(じゅうらい): 
\\	従兄弟(いとこ): 
\\	従姉妹(いとこ): 
\\	従う(したがう): 
\\	従事 (じゅうじ), 従順 (じゅうじゅん), 服従 (ふくじゅう), 従容 (しょうよう), 従う (したが.う), 従える (したが.える)
\\	徒			
\\	ト	いたずら、あだ	徒歩(とほ): 
\\	生徒(せいと): 
\\	待			
\\	タイ	ま.つ、-ま.ち	待遇(たいぐう): 
\\	待望(たいぼう): 
\\	待ち合わせ(まちあわせ): 
\\	待ち遠しい(まちどおしい): 
\\	待ち望む(まちのぞむ): 
\\	期待(きたい): 
\\	待合室(まちあいしつ): 
\\	待ち合わせる(まちあわせる): 
\\	招待(しょうたい): 
\\	待つ(まつ): 
\\	待つ (ま.つ)
\\	往			
\\	オウ	い.く、いにしえ、さき.に、ゆ.く	往復(おうふく): 
\\	征			
\\	セイ		征服(せいふく): 
\\	径			
\\	ケイ	みち、こみち、さしわたし、ただちに	直径(ちょっけい): 
\\	半径(はんけい): 
\\	彼			
\\	ヒ	かれ、かの、か.の	彼処(あそこ): 
\\	彼方此方(あちこち): 
\\	彼方(あちら): 
\\	彼方此方(あちらこちら): 
\\	彼此(あれこれ): 
\\	彼等(かれら): 
\\	彼女(かのじょ): 
\\	彼(かれ): 
\\	彼ら(かれら): 
\\	彼 (かれ)
\\	役			
\\	ヤク、エキ		役(えき): 
\\	重役(おもやく): 
\\	役職(やくしょく): 
\\	役立つ(やくだつ): 
\\	役場(やくば): 
\\	重役(じゅうやく): 
\\	主役(しゅやく): 
\\	役(やく): 
\\	役者(やくしゃ): 
\\	役所(やくしょ): 
\\	役人(やくにん): 
\\	役目(やくめ): 
\\	役割(やくわり): 
\\	役に立つ(やくにたつ): 
\\	役所 (やくしょ), 役目 (やくめ), 荷役 (かやく)
\\	徳			
\\	トク		道徳(どうとく): 
\\	徹			
\\	テツ		徹する(てっする): 
\\	徹底(てってい): 
\\	徹夜(てつや): 
\\	徴			
\\	チョウ 象徴 しょうちょう
\\	懲888.	
\\	チョウ、チ	しるし	象徴(しょうちょう): 
\\	徴収(ちょうしゅう): 
\\	特徴(とくちょう): 
\\	懲			
\\	懲りる 
\\	チョウ	こ.りる、こ.らす、こ.らしめる	懲りる(こりる): 
\\	懲らしめる (こ.らしめる), 懲らす (こ.らす), 懲りる (こ.りる)
\\	微			
\\	ビ 微笑 びしょう
\\	ビ	かす.か	微か(かすか): 
\\	微笑(びしょう): 
\\	微量(びりょう): 
\\	微塵(みじん): 
\\	顕微鏡(けんびきょう): 
\\	微妙(びみょう): 
\\	微笑む(ほほえむ): 
\\	街			
\\	ガイ、カイ	まち	街(がい): 
\\	街道(かいどう): 
\\	街頭(がいとう): 
\\	市街(しがい): 
\\	街(まち): 
\\	街角(まちかど): 
\\	街頭 (がいとう), 市街 (しがい), 商店街 (しょうてんがい), 街 (まち)
\\	衡			
\\	コウ		均衡(きんこう): 
\\	稿			
\\	高校.
\\	コウ	わら、したがき	原稿(げんこう): 
\\	稼			
\\	カ	かせ.ぐ	共稼ぎ(ともかせぎ): 
\\	稼ぐ(かせぐ): 
\\	稼ぐ (かせ.ぐ)
\\	程			
\\	テイ	ほど、-ほど	左程(さほど): 
\\	其れ程(それほど): 
\\	程度(ていど): 
\\	中程(なかほど): 
\\	程(ほど): 
\\	余程(よっぽど): 
\\	課程(かてい): 
\\	過程(かてい): 
\\	先程(さきほど): 
\\	日程(にってい): 
\\	方程式(ほうていしき): 
\\	程 (ほど)
\\	税			
\\	ゼイ		関税(かんぜい): 
\\	税務署(ぜいむしょ): 
\\	課税(かぜい): 
\\	税(ぜい): 
\\	税関(ぜいかん): 
\\	税金(ぜいきん): 
\\	免税(めんぜい): 
\\	稚			
\\	チ、ジ	いとけない、おさない、おくて、おでる	幼稚(ようち): 
\\	幼稚園(ようちえん): 
\\	和			
\\	ワ、オ、カ	やわ.らぐ、やわ.らげる、なご.む、なご.やか	温和(おんわ): 
\\	緩和(かんわ): 
\\	共和(きょうわ): 
\\	中和(ちゅうわ): 
\\	調和(ちょうわ): 
\\	和やか(なごやか): 
\\	飽和(ほうわ): 
\\	和らげる(やわらげる): 
\\	和(わ): 
\\	和風(わふう): 
\\	和文(わぶん): 
\\	英和(えいわ): 
\\	漢和(かんわ): 
\\	平和(へいわ): 
\\	和英(わえい): 
\\	和服(わふく): 
\\	和解 (わかい), 和服 (わふく), 柔和 (にゅうわ), 和む (なご.む), 和やか (なご.やか), 和らぐ (やわ.らぐ), 和らげる (やわ.らげる)
\\	移			
\\	イ	うつ.る、うつ.す	移行(いこう): 
\\	移住(いじゅう): 
\\	移民(いみん): 
\\	移転(いてん): 
\\	移動(いどう): 
\\	移す(うつす): 
\\	移る(うつる): 
\\	移す (うつ.す), 移る (うつ.る)
\\	秒			
\\	ビョウ		秒(びょう): 
\\	秋			
\\	シュウ	あき、とき	秋(あき): 
\\	秋 (あき)
\\	愁			
\\	シュウ	うれ.える、うれ.い	郷愁(きょうしゅう): 
\\	愁い (うれ.い), 愁える (うれ.える)
\\	私			
\\	シ	わたくし、わたし	私(あたし): 
\\	私物(しぶつ): 
\\	私有(しゆう): 
\\	私用(しよう): 
\\	私鉄(してつ): 
\\	私立(しりつ): 
\\	私(わたくし): 
\\	私 (わたくし)
\\	秩			
\\	チツ		秩序(ちつじょ): 
\\	秘			
\\	ヒ	ひ.める、ひそ.か、かく.す	神秘(しんぴ): 
\\	秘書(ひしょ): 
\\	秘密(ひみつ): 
\\	秘める (ひ.める)
\\	称			
\\	ショウ	たた.える、とな.える、あ.げる、かな.う、はか.り、はか.る、ほめ.る	称する(しょうする): 
\\	名称(めいしょう): 
\\	利			
\\	リ	き.く	砂利(じゃり): 
\\	勝利(しょうり): 
\\	左利き(ひだりきき): 
\\	利根(りこん): 
\\	利子(りし): 
\\	利潤(りじゅん): 
\\	利息(りそく): 
\\	利点(りてん): 
\\	権利(けんり): 
\\	不利(ふり): 
\\	有利(ゆうり): 
\\	利益(りえき): 
\\	利害(りがい): 
\\	利口(りこう): 
\\	利用(りよう): 
\\	便利(べんり): 
\\	利く (き.く)
\\	梨			
\\	リ	なし		梨 (なし)
\\	穫			
\\	カク		収穫(しゅうかく): 
\\	穂			
\\	スイ	ほ	穂(ほ): 
\\	穂 (ほ)
\\	稲			
\\	トウ、テ	いね、いな-	稲光(いなびかり): 
\\	稲(いね): 
\\	稲 (いね)
\\	香			
\\	コウ、キョウ	か、かお.り、かお.る	香辛料(こうしんりょう): 
\\	香り(かおり): 
\\	香水(こうすい): 
\\	香水 (こうすい), 香気 (こうき), 線香 (せんこう), 香 (か), 香り (かお.り), 香る (かお.る)
\\	季			
\\	キ		季刊(きかん): 
\\	四季(しき): 
\\	季節(きせつ): 
\\	委			
\\	イ	ゆだ.ねる	委託(いたく): 
\\	委員(いいん): 
\\	秀			
\\	シュウ	ひい.でる	優秀(ゆうしゅう): 
\\	秀でる (ひい.でる)
\\	透			
\\	トウ	す.く、す.かす、す.ける、とう.る、とう.す	透き通る(すきとおる): 
\\	透明(とうめい): 
\\	透かす (す.かす), 透く (す.く), 透ける (す.ける)
\\	誘			
\\	ユウ、イウ	さそ.う、いざな.う	勧誘(かんゆう): 
\\	誘導(ゆうどう): 
\\	誘惑(ゆうわく): 
\\	誘う(さそう): 
\\	誘う (さそ.う)
\\	穀			
\\	コク		穀物(こくもつ): 
\\	菌			
\\	キン		細菌(さいきん): 
\\	黴菌(ばいきん): 
\\	米			
\\	ベイ、マイ、メエトル	こめ、よね	欧米(おうべい): 
\\	南米(なんべい): 
\\	米(こめ): 
\\	米作 (べいさく), 米価 (べいか), 米食 (べいしょく), 米 (こめ)
\\	粉			
\\	フン、デシメートル	こ、こな	花粉(かふん): 
\\	粉々(こなごな): 
\\	粉末(ふんまつ): 
\\	粉(こな): 
\\	粉 (こ), 粉 (こな)
\\	粘			
\\	ネン	ねば.る	粘り(ねばり): 
\\	粘る(ねばる): 
\\	粘る (ねば.る)
\\	粒			
\\	リュウ	つぶ	粒(つぶ): 
\\	粒 (つぶ)
\\	粧			
\\	ショウ		化粧(けしょう): 
\\	迷			
\\	メイ	まよ.う	迷子(まいご): 
\\	迷信(めいしん): 
\\	迷惑(めいわく): 
\\	迷う (まよ.う)
\\	粋			
\\	スイ	いき	粋(いき): 
\\	純粋(じゅんすい): 
\\	糧			
\\	リョウ、ロウ	かて	食糧(しょくりょう): 
\\	糧食 (かてしょく), 糧道 (りょうどう), 糧 (かて)
\\	菊			
\\	キク			
\\	奥			
\\	"奥さん (おくさん) 
\\	オウ、オク	おく.まる、くま	奥(おく): 
\\	奥さん(おくさん): 
\\	奥 (おく)
\\	数			
\\	スウ、ス、サク、ソク、シュ	かず、かぞ.える、しばしば、せ.める、わずらわ.しい	奇数(きすう): 
\\	少数(しょうすう): 
\\	数詞(すうし): 
\\	多数決(たすうけつ): 
\\	手数(てかず): 
\\	回数(かいすう): 
\\	回数券(かいすうけん): 
\\	数(かず): 
\\	数える(かぞえる): 
\\	過半数(かはんすう): 
\\	偶数(ぐうすう): 
\\	算数(さんすう): 
\\	小数(しょうすう): 
\\	数(すう): 
\\	数字(すうじ): 
\\	整数(せいすう): 
\\	単数(たんすう): 
\\	点数(てんすう): 
\\	複数(ふくすう): 
\\	分数(ぶんすう): 
\\	枚数(まいすう): 
\\	無数(むすう): 
\\	数学(すうがく): 
\\	数字 (すうじ), 数量 (すうりょう), 年数 (ねんすう), 数 (かず), 数える (かぞ.える)
\\	楼			
\\	ロウ	たかどの		
\\	類			
\\	ルイ	たぐ.い	衣類(いるい): 
\\	類(たぐい): 
\\	類似(るいじ): 
\\	類推(るいすい): 
\\	種類(しゅるい): 
\\	書類(しょるい): 
\\	親類(しんるい): 
\\	人類(じんるい): 
\\	分類(ぶんるい): 
\\	漆			
\\	シツ	うるし		漆 (うるし)
\\	様			
\\	ヨウ、ショウ	さま、さん	有様(ありさま): 
\\	一様(いちよう): 
\\	お蔭様で(おかげさまで): 
\\	ご苦労様(ごくろうさま): 
\\	様(さま): 
\\	左様なら(さようなら): 
\\	仕様(しよう): 
\\	多様(たよう): 
\\	殿様(とのさま): 
\\	様式(ようしき): 
\\	様相(ようそう): 
\\	王様(おうさま): 
\\	神様(かみさま): 
\\	逆様(さかさま): 
\\	様々(さまざま): 
\\	同様(どうよう): 
\\	模様(もよう): 
\\	様(よう): 
\\	様子(ようす): 
\\	様 (さま)
\\	求			
\\	キュウ、グ	もと.める	請求(せいきゅう): 
\\	求める(もとめる): 
\\	要求(ようきゅう): 
\\	求める (もと.める)
\\	球			
\\	キュウ	たま	眼球(がんきゅう): 
\\	球根(きゅうこん): 
\\	球(きゅう): 
\\	球(たま): 
\\	地球(ちきゅう): 
\\	電球(でんきゅう): 
\\	球 (たま)
\\	救			
\\	キュウ	すく.う	救援(きゅうえん): 
\\	救済(きゅうさい): 
\\	救い(すくい): 
\\	救助(きゅうじょ): 
\\	救う(すくう): 
\\	救う (すく.う)
\\	竹			
\\	チク 松竹梅 しょうちくばい
\\	チク	たけ	竹(たけ): 
\\	竹 (たけ)
\\	笑			
\\	ショウ	わら.う、え.む	あざ笑う(あざわらう): 
\\	可笑しい(おかしい): 
\\	微笑(びしょう): 
\\	笑顔(えがお): 
\\	微笑む(ほほえむ): 
\\	笑い(わらい): 
\\	笑う(わらう): 
\\	笑む (え.む), 笑う (わら.う)
\\	笠			
\\	リュウ	かさ		
\\	笹			
\\	ささ		
\\	筋			
\\	キン	すじ	粗筋(あらすじ): 
\\	大筋(おおすじ): 
\\	一筋(ひとすき): 
\\	筋肉(きんにく): 
\\	筋(すじ): 
\\	筋 (すじ)
\\	箱			
\\	ソウ	はこ	箱(はこ): 
\\	箱 (はこ)
\\	筆			
\\	ヒツ 鉛筆 えんぴつ
\\	ヒツ	ふで	執筆(しっぴつ): 
\\	随筆(ずいひつ): 
\\	筆記(ひっき): 
\\	筆者(ひっしゃ): 
\\	筆(ふで): 
\\	鉛筆(えんぴつ): 
\\	万年筆(まんねんひつ): 
\\	筆 (ふで)
\\	筒			
\\	トウ	つつ	筒(つつ): 
\\	水筒(すいとう): 
\\	封筒(ふうとう): 
\\	筒 (つつ)
\\	等			
\\	トウ	ひと.しい、など、-ら	高等学校(こうとうがっこう): 
\\	対等(たいとう): 
\\	等(とう): 
\\	等級(とうきゅう): 
\\	同等(どうとう): 
\\	彼等(かれら): 
\\	高等(こうとう): 
\\	上等(じょうとう): 
\\	等分(とうぶん): 
\\	等(など): 
\\	等しい(ひとしい): 
\\	平等(びょうどう): 
\\	等しい (ひと.しい)
\\	算			
\\	サン	そろ	暗算(あんざん): 
\\	換算(かんさん): 
\\	決算(けっさん): 
\\	採算(さいさん): 
\\	清算(せいさん): 
\\	足し算(たしざん): 
\\	割り算(わりざん): 
\\	掛け算(かけざん): 
\\	計算(けいさん): 
\\	算数(さんすう): 
\\	算盤(そろばん): 
\\	引算(ひきざん): 
\\	予算(よさん): 
\\	割算(わりざん): 
\\	答			
\\	トウ	こた.える、こた.え	返答(へんとう): 
\\	解答(かいとう): 
\\	回答(かいとう): 
\\	答(こたえ): 
\\	答案(とうあん): 
\\	問答(もんどう): 
\\	答え(こたえ): 
\\	答える(こたえる): 
\\	答え (こた.え), 答える (こた.える)
\\	策			
\\	-----------対策  たいさく  
\\	サク		策(さく): 
\\	政策(せいさく): 
\\	方策(ほうさく): 
\\	対策(たいさく): 
\\	簿			
\\	ボ		名簿(めいぼ): 
\\	築			
\\	(ちく) 
\\	(きず) 
\\	チク	きず.く	築く(きずく): 
\\	新築(しんちく): 
\\	建築(けんちく): 
\\	築く (きず.く)
\\	人			
\\	⺅ 
\\	イ). 
\\	ジン、ニン	ひと、-り、-と	商人(あきうど): 
\\	他人(あだびと): 
\\	一人(いちにん): 
\\	玄人(くろうと): 
\\	故人(こじん): 
\\	殺人(さつじん): 
\\	産婦人科(さんふじんか): 
\\	主人公(しゅじんこう): 
\\	証人(しょうにん): 
\\	使用人(しようにん): 
\\	人(じん): 
\\	人格(じんかく): 
\\	人材(じんざい): 
\\	新人(しんじん): 
\\	人体(じんたい): 
\\	人民(じんみん): 
\\	人目(じんもく): 
\\	知人(ちじん): 
\\	仲人(ちゅうにん): 
\\	当人(とうにん): 
\\	二人(ににん): 
\\	人情(にんじょう): 
\\	万人(ばんじん): 
\\	人柄(ひとがら): 
\\	人質(ひとじち): 
\\	一人でに(ひとりでに): 
\\	男の人(おとこのひと): 
\\	女の人(おんなのひと): 
\\	恋人(こいびと): 
\\	個人(こじん): 
\\	詩人(しじん): 
\\	商人(しょうにん): 
\\	職人(しょくにん): 
\\	素人(しろうと): 
\\	人工(じんこう): 
\\	人種(じんしゅ): 
\\	人生(じんせい): 
\\	人造(じんぞう): 
\\	人物(じんぶつ): 
\\	人文科学(じんぶんかがく): 
\\	人命(じんめい): 
\\	人類(じんるい): 
\\	成人(せいじん): 
\\	他人(たにん): 
\\	人気(にんき): 
\\	人間(にんげん): 
\\	犯人(はんにん): 
\\	美人(びじん): 
\\	人込み(ひとごみ): 
\\	人差指(ひとさしゆび): 
\\	人通り(ひとどおり): 
\\	一人一人(ひとりひとり): 
\\	夫人(ふじん): 
\\	婦人(ふじん): 
\\	本人(ほんにん): 
\\	名人(めいじん): 
\\	役人(やくにん): 
\\	友人(ゆうじん): 
\\	老人(ろうじん): 
\\	人口(じんこう): 
\\	人形(にんぎょう): 
\\	大人(おとな): 
\\	外国人(がいこくじん): 
\\	御主人(ごしゅじん): 
\\	人(ひと): 
\\	一人(ひとり): 
\\	二人(ふたり): 
\\	人道 (じんどう), 人員 (じんいん), 成人 (せいじん), 人 (ひと)
\\	佐			
\\	サ		佐(さ): 
\\	但			
\\	タン	ただ.し	但し(ただし): 
\\	但し (ただ.し)
\\	住			
\\	ジュウ、ヂュウ、チュウ	す.む、す.まう、-ず.まい	移住(いじゅう): 
\\	居住(きょじゅう): 
\\	住(じゅう): 
\\	衣食住(いしょくじゅう): 
\\	住居(じゅうきょ): 
\\	住宅(じゅうたく): 
\\	住民(じゅうみん): 
\\	住まい(すまい): 
\\	住所(じゅうしょ): 
\\	住む(すむ): 
\\	住まう (す.まう), 住む (す.む)
\\	位			
\\	イ	くらい、ぐらい	位地(いち): 
\\	位置(いち): 
\\	下位(かい): 
\\	上位(じょうい): 
\\	優位(ゆうい): 
\\	位(くらい): 
\\	単位(たんい): 
\\	地位(ちい): 
\\	位 (くらい)
\\	仲			
\\	チュウ	なか	仲人(ちゅうにん): 
\\	仲(なか): 
\\	仲直り(なかなおり): 
\\	仲間(なかま): 
\\	仲良し(なかよし): 
\\	仲 (なか)
\\	体			
\\	タイ、テイ	からだ、かたち	身体(からだ): 
\\	体付き(からだつき): 
\\	固体(こたい): 
\\	字体(じたい): 
\\	主体(しゅたい): 
\\	人体(じんたい): 
\\	体格(たいかく): 
\\	体験(たいけん): 
\\	体力(たいりょく): 
\\	体(てい): 
\\	体裁(ていさい): 
\\	天体(てんたい): 
\\	肉体(にくたい): 
\\	物体(ぶったい): 
\\	本体(ほんたい): 
\\	立体(りったい): 
\\	一体(いったい): 
\\	液体(えきたい): 
\\	気体(きたい): 
\\	具体(ぐたい): 
\\	個体(こたい): 
\\	死体(したい): 
\\	重体(じゅうたい): 
\\	身体(しんたい): 
\\	全体(ぜんたい): 
\\	体育(たいいく): 
\\	体温(たいおん): 
\\	体系(たいけい): 
\\	体制(たいせい): 
\\	体積(たいせき): 
\\	体操(たいそう): 
\\	大体(だいたい): 
\\	団体(だんたい): 
\\	文体(ぶんたい): 
\\	体(からだ): 
\\	体格 (たいかく), 人体 (じんたい), 主体 (しゅたい), 体 (からだ)
\\	悠			
\\	ユウ		悠々(ゆうゆう): 
\\	件			
\\	ケン	くだん	件(くだん): 
\\	用件(ようけん): 
\\	事件(じけん): 
\\	条件(じょうけん): 
\\	仕			
\\	シ、ジ	つか.える	給仕(きゅうじ): 
\\	仕上がり(しあがり): 
\\	仕上げ(しあげ): 
\\	仕上げる(しあげる): 
\\	仕入れる(しいれる): 
\\	仕掛け(しかけ): 
\\	仕掛ける(しかける): 
\\	仕切る(しきる): 
\\	仕組み(しくみ): 
\\	仕立てる(したてる): 
\\	仕付ける(しつける): 
\\	仕舞(しまい): 
\\	仕舞う(しまう): 
\\	仕様(しよう): 
\\	仕える(つかえる): 
\\	奉仕(ほうし): 
\\	仕上がる(しあがる): 
\\	仕方(しかた): 
\\	仕事(しごと): 
\\	仕事 (しごと), 出仕 (しゅつし), 仕える (つか.える)
\\	他			
\\	タ	ほか	他人(あだびと): 
\\	他意(たい): 
\\	他動詞(たどうし): 
\\	他方(たほう): 
\\	他(た): 
\\	他人(たにん): 
\\	他(ほか): 
\\	伏			
\\	フク	ふ.せる、ふ.す	起伏(きふく): 
\\	降伏(こうふく): 
\\	伏す (ふ.す), 伏せる (ふ.せる)
\\	伝			
\\	デン、テン	つた.わる、つた.える、つた.う、つだ.う、-づた.い、つて	言伝(ことづて): 
\\	伝言(つてごと): 
\\	伝説(でんせつ): 
\\	伝達(でんたつ): 
\\	伝来(でんらい): 
\\	お手伝いさん(おてつだいさん): 
\\	宣伝(せんでん): 
\\	伝わる(つたわる): 
\\	手伝い(てつだい): 
\\	伝記(でんき): 
\\	伝染(でんせん): 
\\	伝統(でんとう): 
\\	伝える(つたえる): 
\\	手伝う(てつだう): 
\\	伝う (つた.う), 伝える (つた.える), 伝わる (つた.わる)
\\	仏			
\\	ブツ、フツ	ほとけ	仏(ふつ): 
\\	仏像(ぶつぞう): 
\\	仏(ほとけ): 
\\	仏 (ほとけ)
\\	休			
\\	キュウ	やす.む、やす.まる、やす.める	お休み(おやすみ): 
\\	休学(きゅうがく): 
\\	休戦(きゅうせん): 
\\	産休(さんきゅう): 
\\	休める(やすめる): 
\\	連休(れんきゅう): 
\\	休暇(きゅうか): 
\\	休業(きゅうぎょう): 
\\	休憩(きゅうけい): 
\\	休講(きゅうこう): 
\\	休息(きゅうそく): 
\\	休養(きゅうよう): 
\\	定休日(ていきゅうび): 
\\	一休み(ひとやすみ): 
\\	昼休み(ひるやすみ): 
\\	夏休み(なつやすみ): 
\\	休み(やすみ): 
\\	休む(やすむ): 
\\	休まる (やす.まる), 休む (やす.む), 休める (やす.める)
\\	仮			
\\	カ、ケ	かり、かり-	仮(か): 
\\	仮令(たとえ): 
\\	送り仮名(おくりがな): 
\\	仮定(かてい): 
\\	仮名(かな): 
\\	仮名遣い(かなづかい): 
\\	振り仮名(ふりがな): 
\\	片仮名(かたかな): 
\\	平仮名(ひらがな): 
\\	仮面 (かめん), 仮定 (かてい), 仮装 (かそう), 仮 (かり)
\\	伯			
\\	ハク		伯父さん(おじさん): 
\\	伯母さん(おばさん): 
\\	俗			
\\	ゾク		風俗(ふうぞく): 
\\	民俗(みんぞく): 
\\	信			
\\	シン		確信(かくしん): 
\\	自信(じしん): 
\\	信者(しんじゃ): 
\\	信任(しんにん): 
\\	信用(しんよう): 
\\	信仰(しんこう): 
\\	信号(しんごう): 
\\	信じる(しんじる): 
\\	信ずる(しんずる): 
\\	信頼(しんらい): 
\\	通信(つうしん): 
\\	迷信(めいしん): 
\\	佳			
\\	カ		佳句(かく): 
\\	依			
\\	イ、エ	よ.る	依(い): 
\\	依然(いぜん): 
\\	依存(いそん): 
\\	依って(よって): 
\\	依頼(いらい): 
\\	依頼 (いらい), 依拠 (いきょ), 依然 (いぜん)
\\	例			
\\	レイ	たと.える	慣例(かんれい): 
\\	前例(ぜんれい): 
\\	例え(たとえ): 
\\	例(ためし): 
\\	比例(ひれい): 
\\	実例(じつれい): 
\\	例える(たとえる): 
\\	例(れい): 
\\	例外(れいがい): 
\\	例えば(たとえば): 
\\	例える (たと.える)
\\	個			
\\	コ、カ		個(か): 
\\	個々(ここ): 
\\	個性(こせい): 
\\	個別(こべつ): 
\\	個所(かしょ): 
\\	個人(こじん): 
\\	個体(こたい): 
\\	健			
\\	ケン	すこ.やか	健在(けんざい): 
\\	健全(けんぜん): 
\\	健やか(すこやか): 
\\	健康(けんこう): 
\\	保健(ほけん): 
\\	健やか (すこ.やか)
\\	側			
\\	ソク	かわ、がわ、そば	縁側(えんがわ): 
\\	側(かわ): 
\\	側(がわ): 
\\	側面(そくめん): 
\\	側(そば): 
\\	両側(りょうがわ): 
\\	側 (かわ)
\\	侍			
\\	待. 
\\	侍! 待 
\\	侍.	
\\	ジ、シ	さむらい、はべ.る	侍(さむらい): 
\\	侍 (さむらい)
\\	停			
\\	テイ	と.める、と.まる	調停(ちょうてい): 
\\	停滞(ていたい): 
\\	停止(ていし): 
\\	停車(ていしゃ): 
\\	停電(ていでん): 
\\	停留所(ていりゅうじょ): 
\\	値			
\\	チ	ね、あたい	値(あたい): 
\\	値する(あたいする): 
\\	値打ち(ねうち): 
\\	値段(ねだん): 
\\	値引き(ねびき): 
\\	価値(かち): 
\\	値(ね): 
\\	値 (あたい), 値 (ね)
\\	倣			
\\	ホウ	なら.う	倣(ほう): 
\\	模倣(もほう): 
\\	倣う(ならう): 
\\	倣う (なら.う)
\\	倒			
\\	トウ	たお.れる、-だお.れ、たお.す	倒産(とうさん): 
\\	倒す(たおす): 
\\	面倒(めんどう): 
\\	面倒臭い(めんどうくさい): 
\\	倒れる(たおれる): 
\\	倒す (たお.す), 倒れる (たお.れる)
\\	偵			
\\	テイ			
\\	僧			
\\	ソウ		僧(そう): 
\\	億			
\\	オク		億(おく): 
\\	儀			
\\	ギ		御辞儀(おじぎ): 
\\	儀式(ぎしき): 
\\	行儀(ぎょうぎ): 
\\	礼儀(れいぎ): 
\\	償			
\\	賠. 
\\	賠 
\\	賠償... 
\\	償 
\\	ショウ	つぐな.う	賠償(ばいしょう): 
\\	弁償(べんしょう): 
\\	補償(ほしょう): 
\\	償う (つぐな.う)
\\	仙			
\\	セン、セント		仙(せん): 
\\	催			
\\	サイ	もよう.す、もよお.す	開催(かいさい): 
\\	主催(しゅさい): 
\\	催す(もよおす): 
\\	催促(さいそく): 
\\	催し(もよおし): 
\\	催す (もよお.す)
\\	仁			
\\	ジン、ニ、ニン			仁義 (じんぎ), 仁術 (ひとしじゅつ)
\\	侮			
\\	ブ	あなど.る、あなず.る	侮辱(ぶじょく): 
\\	侮る (あなど.る)
\\	使			
\\	シ	つか.う、つか.い、-つか.い、-づか.い	お使い(おつかい): 
\\	使命(しめい): 
\\	使用人(しようにん): 
\\	使い道(つかいみち): 
\\	使用(しよう): 
\\	大使(たいし): 
\\	大使館(たいしかん): 
\\	使う(つかう): 
\\	使う (つか.う)
\\	便			
\\	ベン、ビン	たよ.り	大便(だいべん): 
\\	不便(ふびん): 
\\	便宜(べんぎ): 
\\	小便(しょうべん): 
\\	便り(たより): 
\\	便(びん): 
\\	便箋(びんせん): 
\\	船便(ふなびん): 
\\	便所(べんじょ): 
\\	郵便(ゆうびん): 
\\	不便(ふべん): 
\\	便利(べんり): 
\\	郵便局(ゆうびんきょく): 
\\	便利 (べんり), 便法 (べんぽう), 簡便 (かんべん), 便り (たよ.り)
\\	倍			
\\	バイ		倍率(ばいりつ): 
\\	倍(ばい): 
\\	優			
\\	ユウ、ウ	やさ.しい、すぐ.れる、まさ.る	俳優(はいゆう): 
\\	優(やさ): 
\\	優位(ゆうい): 
\\	優越(ゆうえつ): 
\\	優勢(ゆうせい): 
\\	優先(ゆうせん): 
\\	優美(ゆうび): 
\\	女優(じょゆう): 
\\	優れる(すぐれる): 
\\	優秀(ゆうしゅう): 
\\	優勝(ゆうしょう): 
\\	優しい(やさしい): 
\\	優れる (すぐ.れる), 優しい (やさ.しい)
\\	伐			
\\	バツ、ハツ、カ、ボチ	き.る、そむ.く、う.つ	伐(ばつ): 
\\	宿			
\\	シュク	やど、やど.る、やど.す	宿命(しゅくめい): 
\\	民宿(みんしゅく): 
\\	宿泊(しゅくはく): 
\\	宿(やど): 
\\	下宿(げしゅく): 
\\	宿題(しゅくだい): 
\\	宿 (やど), 宿す (やど.す), 宿る (やど.る)
\\	傷			
\\	ショウ	きず、いた.む、いた.める	火傷(かしょう): 
\\	傷付く(きずつく): 
\\	傷(しょう): 
\\	中傷(ちゅうしょう): 
\\	負傷(ふしょう): 
\\	傷(きず): 
\\	火傷(やけど): 
\\	傷む (いた.む), 傷める (いた.める), 傷 (きず)
\\	保			
\\	呆 
\\	ホ、ホウ	たも.つ	確保(かくほ): 
\\	保つ(たもつ): 
\\	保育(ほいく): 
\\	保温(ほおん): 
\\	保管(ほかん): 
\\	保険(ほけん): 
\\	保護(ほご): 
\\	保守(ほしゅ): 
\\	保障(ほしょう): 
\\	保母(ほぼ): 
\\	保養(ほよう): 
\\	保健(ほけん): 
\\	保証(ほしょう): 
\\	保存(ほぞん): 
\\	保つ (たも.つ)
\\	褒			
\\	ホウ 過褒 かほう
\\	ホウ	ほ.める	褒美(ほうび): 
\\	褒める(ほめる): 
\\	褒める (ほ.める)
\\	傑			
\\	ケツ	すぐ.れる	傑(けつ): 
\\	傑作(けっさく): 
\\	付			
\\	賀 
\\	千 
\\	読
\\	吉 
\\	往. 
\\	フ	つ.ける、-つ.ける、-づ.ける、つ.け、つ.け-、-つ.け、-づ.け、-づけ、つ.く、-づ.く、つ.き、-つ.き、-つき、-づ.き、-づき	受け付ける(うけつける): 
\\	思い付き(おもいつき): 
\\	顔付き(かおつき): 
\\	片付け(かたづけ): 
\\	日付(かづけ): 
\\	体付き(からだつき): 
\\	傷付く(きずつく): 
\\	くっ付く(くっつく): 
\\	くっ付ける(くっつける): 
\\	交付(こうふ): 
\\	仕付ける(しつける): 
\\	据え付ける(すえつける): 
\\	備え付ける(そなえつける): 
\\	近付く(ちかづく): 
\\	付き(つき): 
\\	付き合う(つきあう): 
\\	付け加える(つけくわえる): 
\\	付ける(つける): 
\\	名付ける(なづける): 
\\	付属(ふぞく): 
\\	打付ける(ぶつける): 
\\	付録(ふろく): 
\\	結び付き(むすびつき): 
\\	結び付く(むすびつく): 
\\	結び付ける(むすびつける): 
\\	目付き(めつき): 
\\	やっ付ける(やっつける): 
\\	言い付ける(いいつける): 
\\	追い付く(おいつく): 
\\	思い付く(おもいつく): 
\\	片付く(かたづく): 
\\	気付く(きづく): 
\\	寄付(きふ): 
\\	気を付ける(きをつける): 
\\	言付ける(ことづける): 
\\	近付ける(ちかづける): 
\\	付き合い(つきあい): 
\\	付合う(つきあう): 
\\	付く(つく): 
\\	日付(ひづけ): 
\\	付近(ふきん): 
\\	見付かる(みつかる): 
\\	見付ける(みつける): 
\\	受付(うけつけ): 
\\	片付ける(かたづける): 
\\	付く (つ.く), 付ける (つ.ける)
\\	符			
\\	フ		符号(ふごう): 
\\	切符(きっぷ): 
\\	府			
\\	フ		政府(せいふ): 
\\	任			
\\	ニン	まか.せる、まか.す	就任(しゅうにん): 
\\	主任(しゅにん): 
\\	信任(しんにん): 
\\	転任(てんにん): 
\\	任務(にんむ): 
\\	任命(にんめい): 
\\	赴任(ふにん): 
\\	任す(まかす): 
\\	責任(せきにん): 
\\	任せる(まかせる): 
\\	任す (まか.す), 任せる (まか.せる)
\\	賃			
\\	チン		運賃(うんちん): 
\\	賃金(ちんぎん): 
\\	家賃(やちん): 
\\	代			
\\	ダイ、タイ	か.わる、かわ.る、かわ.り、-がわ.り、か.える、よ、しろ	代える(かえる): 
\\	代わる(かわる): 
\\	代わる代わる(かわるがわる): 
\\	古代(こだい): 
\\	代(しろ): 
\\	世代(せだい): 
\\	先代(せんだい): 
\\	代弁(だいべん): 
\\	代用(だいよう): 
\\	お代わり(おかわり): 
\\	代る(かわる): 
\\	近代(きんだい): 
\\	現代(げんだい): 
\\	代金(だいきん): 
\\	代表(だいひょう): 
\\	代名詞(だいめいし): 
\\	代理(だいり): 
\\	年代(ねんだい): 
\\	代わり(かわり): 
\\	時代(じだい): 
\\	代理 (だいり), 世代 (せだい), 現代 (げんだい), 代える (か.える), 代わる (か.わる), 代 (しろ), 代 (よ)
\\	袋			
\\	タイ、ダイ	ふくろ	足袋(たび): 
\\	袋(ふくろ): 
\\	手袋(てぶくろ): 
\\	袋 (ふくろ)
\\	貸			
\\	タイ	か.す、か.し-、かし-	貸し(かし): 
\\	貸し出し(かしだし): 
\\	貸間(かしま): 
\\	貸家(かしや): 
\\	貸す(かす): 
\\	貸す (か.す)
\\	化			
\\	カ、ケ	ば.ける、ば.かす、ふ.ける、け.する	悪化(あっか): 
\\	化合(かごう): 
\\	化する(かする): 
\\	化石(かせき): 
\\	化繊(かせん): 
\\	誤魔化す(ごまかす): 
\\	酸化(さんか): 
\\	進化(しんか): 
\\	退化(たいか): 
\\	化ける(ばける): 
\\	文化財(ぶんかざい): 
\\	化学(かがく): 
\\	強化(きょうか): 
\\	化粧(けしょう): 
\\	消化(しょうか): 
\\	変化(へんか): 
\\	文化(ぶんか): 
\\	化石 (かせき), 化学 (かがく), 文化 (ぶんか), 化かす (ば.かす), 化ける (ば.ける)
\\	花			
\\	カ、ケ	はな	花壇(かだん): 
\\	花粉(かふん): 
\\	花びら(はなびら): 
\\	火花(ひばな): 
\\	生け花(いけばな): 
\\	花火(はなび): 
\\	花見(はなみ): 
\\	花嫁(はなよめ): 
\\	花瓶(かびん): 
\\	花(はな): 
\\	花 (はな)
\\	貨			
\\	賃 
\\	カ	たから	外貨(がいか): 
\\	貨幣(かへい): 
\\	雑貨(ざっか): 
\\	貨物(かもつ): 
\\	硬貨(こうか): 
\\	通貨(つうか): 
\\	傾			
\\	ケイ 傾向 けいこう
\\	ケイ	かたむ.く、かたむ.ける、かたぶ.く、かた.げる、かし.げる	傾く(かたぶく): 
\\	傾ける(かたむける): 
\\	傾(けい): 
\\	傾斜(けいしゃ): 
\\	傾く(かたむく): 
\\	傾向(けいこう): 
\\	傾らか(なだらか): 
\\	傾く (かたむ.く), 傾ける (かたむ.ける)
\\	何			
\\	カ	なに、なん、なに-、なん-	如何(いかが): 
\\	如何に(いかに): 
\\	如何にも(いかにも): 
\\	何れ(いずれ): 
\\	何時(いつ): 
\\	何時か(いつか): 
\\	何時でも(いつでも): 
\\	何時の間にか(いつのまにか): 
\\	何時も(いつも): 
\\	如何しても(どうしても): 
\\	何方(どちら): 
\\	何の(どの): 
\\	何れ(どれ): 
\\	何々(どれどれ): 
\\	何故(なぜ): 
\\	何気ない(なにげない): 
\\	何しろ(なにしろ): 
\\	何卒(なにとぞ): 
\\	何も(なにも): 
\\	何より(なにより): 
\\	何だか(なんだか): 
\\	何て(なんて): 
\\	何と(なんと): 
\\	何となく(なんとなく): 
\\	何とも(なんとも): 
\\	何なり(なんなり): 
\\	何か(なにか): 
\\	何々(なになに): 
\\	何分(なにぶん): 
\\	何で(なんで): 
\\	何でも(なんでも): 
\\	何とか(なんとか): 
\\	何(なに): 
\\	何 (なに)
\\	荷			
\\	カ	に	荷(に): 
\\	荷造り(にづくり): 
\\	荷物(にもつ): 
\\	荷 (に)
\\	俊			
\\	シュン			
\\	傍			
\\	ボウ	かたわ.ら、わき、おか-、はた、そば	傍ら(かたわら): 
\\	傍ら (かたわ.ら)
\\	久			
\\	ひさ 
\\	ひさしぶり 
\\	キュウ、ク	ひさ.しい	久しい(ひさしい): 
\\	久し振り(ひさしぶり): 
\\	永久(えいきゅう): 
\\	久しぶり(ひさしぶり): 
\\	永久 (えいきゅう), 持久 (じきゅう), 耐久 (たいきゅう), 久しい (ひさ.しい)
\\	畝			
\\	ボウ、ホ、モ、ム	せ、うね		畝 (うね), 畝 (せ)
\\	囚			
\\	シュウ	とら.われる		
\\	内			
\\	ナイ、ダイ	うち	内(うち): 
\\	内訳(うちわけ): 
\\	内閣(ないかく): 
\\	内臓(ないぞう): 
\\	内部(ないぶ): 
\\	内乱(ないらん): 
\\	内陸(ないりく): 
\\	案内(あんない): 
\\	内科(ないか): 
\\	内線(ないせん): 
\\	内容(ないよう): 
\\	以内(いない): 
\\	家内(かない): 
\\	内外 (ないがい), 内容 (ないよう), 家内 (かない), 内 (うち)
\\	丙			
\\	ヘイ	ひのえ		
\\	柄			
\\	ヘイ	がら、え、つか	間柄(あいだがら): 
\\	柄(え): 
\\	大柄(おおがら): 
\\	小柄(こがら): 
\\	事柄(ことがら): 
\\	人柄(ひとがら): 
\\	柄(がら): 
\\	柄 (え), 柄 (がら)
\\	肉			
\\	月.
\\	ニク	しし	肉親(にくしん): 
\\	肉体(にくたい): 
\\	筋肉(きんにく): 
\\	皮肉(ひにく): 
\\	牛肉(ぎゅうにく): 
\\	肉(にく): 
\\	豚肉(ぶたにく): 
\\	腐			
\\	フ	くさ.る、-くさ.る、くさ.れる、くさ.れ、くさ.らす、くさ.す	腐敗(ふはい): 
\\	腐る(くさる): 
\\	腐らす (くさ.らす), 腐る (くさ.る), 腐れる (くさ.れる)
\\	座			
\\	ザ	すわ.る	ご座います(ございます): 
\\	座談会(ざだんかい): 
\\	座標(ざひょう): 
\\	星座(せいざ): 
\\	即座に(そくざに): 
\\	座敷(ざしき): 
\\	座席(ざせき): 
\\	座布団(ざぶとん): 
\\	座る(すわる): 
\\	座る (すわ.る)
\\	卒			
\\	ソツ、シュツ	そっ.する、お.える、お.わる、ついに、にわか	何卒(なにとぞ): 
\\	卒直(そっちょく): 
\\	卒業(そつぎょう): 
\\	傘			
\\	サン	かさ	傘(かさ): 
\\	傘 (かさ)
\\	匁			
\\	もんめ、め		匁 (もんめ)
\\	以			
\\	イ	もっ.て	以て(もって): 
\\	以外(いがい): 
\\	以後(いご): 
\\	以降(いこう): 
\\	以前(いぜん): 
\\	以来(いらい): 
\\	以下(いか): 
\\	以上(いじょう): 
\\	以内(いない): 
\\	似			
\\	ジ	に.る、ひ.る	似通う(にかよう): 
\\	類似(るいじ): 
\\	似合う(にあう): 
\\	真似(まね): 
\\	真似る(まねる): 
\\	似る(にる): 
\\	似る (に.る)
\\	併			
\\	ヘイ	あわ.せる	合併(がっぺい): 
\\	併せる (あわ.せる)
\\	瓦			
\\	//ガ 
\\	ガ	かわら、ぐらむ	瓦(かわら): 
\\	煉瓦(れんが): 
\\	瓦 (かわら)
\\	瓶			
\\	ビン	かめ	瓶(かめ): 
\\	瓶(びん): 
\\	瓶詰(びんづめ): 
\\	花瓶(かびん): 
\\	宮			
\\	キュウ、グウ、ク、クウ	みや	お宮(おみや): 
\\	宮殿(きゅうでん): 
\\	宮殿 (きゅうでん), 宮廷 (きゅうてい), 離宮 (りきゅう), 宮 (みや)
\\	営			
\\	営倉 
\\	エイ	いとな.む、いとな.み	営む(いとなむ): 
\\	運営(うんえい): 
\\	営業(えいぎょう): 
\\	経営(けいえい): 
\\	営む (いとな.む)
\\	善			
\\	ゼン	よ.い、い.い、よ.く、よし.とする	最善(さいぜん): 
\\	親善(しんぜん): 
\\	善良(ぜんりょう): 
\\	善し悪し(よしあし): 
\\	改善(かいぜん): 
\\	善(ぜん): 
\\	善い (よ.い)
\\	年			
\\	ネン	とし	一昨年(おととし): 
\\	同い年(おないどし): 
\\	元年(がんねん): 
\\	成年(せいねん): 
\\	定年(ていねん): 
\\	年頃(としごろ): 
\\	年寄り(としより): 
\\	年鑑(ねんかん): 
\\	年号(ねんごう): 
\\	年生(ねんせい): 
\\	年長(ねんちょう): 
\\	年輪(ねんりん): 
\\	一昨年(いっさくねん): 
\\	学年(がくねん): 
\\	少年(しょうねん): 
\\	青少年(せいしょうねん): 
\\	青年(せいねん): 
\\	生年月日(せいねんがっぴ): 
\\	中年(ちゅうねん): 
\\	年月(としつき): 
\\	年間(ねんかん): 
\\	年月(ねんげつ): 
\\	年中(ねんじゅう): 
\\	年代(ねんだい): 
\\	年度(ねんど): 
\\	年齢(ねんれい): 
\\	年(とし): 
\\	去年(きょねん): 
\\	今年(ことし): 
\\	再来年(さらいねん): 
\\	万年筆(まんねんひつ): 
\\	来年(らいねん): 
\\	年 (とし)
\\	夜			
\\	ヤ	よ、よる	日夜(にちや): 
\\	夜行(やぎょう): 
\\	夜具(やぐ): 
\\	夜中(やちゅう): 
\\	夜更かし(よふかし): 
\\	夜更け(よふけ): 
\\	深夜(しんや): 
\\	徹夜(てつや): 
\\	夜間(やかん): 
\\	夜行(やこう): 
\\	夜(よ): 
\\	夜明け(よあけ): 
\\	夜中(よなか): 
\\	今夜(こんや): 
\\	夜(よる): 
\\	夜 (よ), 夜 (よる)
\\	液			
\\	エキ		液(えき): 
\\	溶液(ようえき): 
\\	液体(えきたい): 
\\	血液(けつえき): 
\\	塚			
\\	チョウ	つか、-づか		塚 (つか)
\\	幣			
\\	ヘイ	ぬさ	貨幣(かへい): 
\\	紙幣(しへい): 
\\	弊			
\\	ヘイ 弊害 へいがい
\\	幣1041.	
\\	ヘイ			
\\	喚			
\\	カン	わめ.く		
\\	換			
\\	カン	か.える、-か.える、か.わる	換える(かえる): 
\\	換算(かんさん): 
\\	転換(てんかん): 
\\	乗り換え(のりかえ): 
\\	換気(かんき): 
\\	交換(こうかん): 
\\	乗換(のりかえ): 
\\	乗り換える(のりかえる): 
\\	換える (か.える), 換わる (か.わる)
\\	融			
\\	ユウ	と.ける、と.かす	融資(ゆうし): 
\\	融通(ゆうずう): 
\\	金融(きんゆう): 
\\	施			
\\	シ、セ	ほどこ.す	施行(しぎょう): 
\\	施設(しせつ): 
\\	施す(ほどこす): 
\\	実施(じっし): 
\\	施設 (しせつ), 施政 (しせい), 実施 (じっし), 施す (ほどこ.す)
\\	旋			
\\	セン			
\\	遊			
\\	ユウ、ユ	あそ.ぶ、あそ.ばす	遊牧(ゆうぼく): 
\\	遊園地(ゆうえんち): 
\\	遊び(あそび): 
\\	遊ぶ(あそぶ): 
\\	遊戯 (ゆうぎ), 遊離 (ゆうり), 交遊 (こうゆう), 遊ぶ (あそ.ぶ)
\\	旅			
\\	リョ	たび	旅客(りょかく): 
\\	旅券(りょけん): 
\\	旅(たび): 
\\	旅館(りょかん): 
\\	旅行(りょこう): 
\\	旅 (たび)
\\	勿			
\\	モチ、ブツ、ボツ	なか.れ、なし	勿論(もちろん): 
\\	物			
\\	ブツ、モツ	もの、もの-	獲物(えもの): 
\\	織物(おりもの): 
\\	贋物(がんぶつ): 
\\	禁物(きんもつ): 
\\	作物(さくぶつ): 
\\	産物(さんぶつ): 
\\	私物(しぶつ): 
\\	出来物(できもの): 
\\	物議(ぶつぎ): 
\\	物資(ぶっし): 
\\	物体(ぶったい): 
\\	干し物(ほしもの): 
\\	見せ物(みせもの): 
\\	物置き(ものおき): 
\\	物好き(ものずき): 
\\	物足りない(ものたりない): 
\\	編物(あみもの): 
\\	生き物(いきもの): 
\\	入れ物(いれもの): 
\\	落し物(おとしもの): 
\\	貨物(かもつ): 
\\	鉱物(こうぶつ): 
\\	穀物(こくもつ): 
\\	作物(さくもつ): 
\\	実物(じつぶつ): 
\\	植物(しょくぶつ): 
\\	食物(しょくもつ): 
\\	書物(しょもつ): 
\\	人物(じんぶつ): 
\\	生物(せいぶつ): 
\\	瀬戸物(せともの): 
\\	農産物(のうさんぶつ): 
\\	博物館(はくぶつかん): 
\\	物価(ぶっか): 
\\	物質(ぶっしつ): 
\\	物騒(ぶっそう): 
\\	物理(ぶつり): 
\\	本物(ほんもの): 
\\	名物(めいぶつ): 
\\	物置(ものおき): 
\\	物音(ものおと): 
\\	物語(ものがたり): 
\\	物語る(ものがたる): 
\\	物事(ものごと): 
\\	物差し(ものさし): 
\\	物凄い(ものすごい): 
\\	贈り物(おくりもの): 
\\	着物(きもの): 
\\	見物(けんぶつ): 
\\	品物(しなもの): 
\\	動物園(どうぶつえん): 
\\	乗り物(のりもの): 
\\	忘れ物(わすれもの): 
\\	買い物(かいもの): 
\\	果物(くだもの): 
\\	建物(たてもの): 
\\	食べ物(たべもの): 
\\	動物(どうぶつ): 
\\	荷物(にもつ): 
\\	飲み物(のみもの): 
\\	物(もの): 
\\	物質 (ぶっしつ), 人物 (じんぶつ), 動物 (どうぶつ), 物 (もの)
\\	易			
\\	エキ、イ	やさ.しい、やす.い	簡易(かんい): 
\\	交易(こうえき): 
\\	容易い(たやすい): 
\\	辟易(へきえき): 
\\	易い(やすい): 
\\	安易(あんい): 
\\	容易(ようい): 
\\	貿易(ぼうえき): 
\\	易しい(やさしい): 
\\	易者 (えきしゃ), 貿易 (ぼうえき), 不易 (ふえき), 易しい (やさ.しい)
\\	賜			
\\	シ	たまわ.る、たま.う、たも.う	賜る(たまわる): 
\\	賜る (たまわ.る)
\\	尿			
\\	ニョウ		屎尿(しにょう): 
\\	尿(にょう): 
\\	尼			
\\	ニ	あま		尼 (あま)
\\	泥			
\\	デイ、ナイ、デ、ニ	どろ	泥(どろ): 
\\	泥棒(どろぼう): 
\\	泥 (どろ)
\\	塀			
\\	ヘイ、ベイ		塀(へい): 
\\	履			
\\	はく, 
\\	リ	は.く	履く(はく): 
\\	履歴(りれき): 
\\	草履(ぞうり): 
\\	履く (は.く)
\\	屋			
\\	オク	や	問屋(といや): 
\\	床屋(とこや): 
\\	屋敷(やしき): 
\\	屋外(おくがい): 
\\	家屋(かおく): 
\\	小屋(こや): 
\\	店屋(みせや): 
\\	屋根(やね): 
\\	屋上(おくじょう): 
\\	部屋(へや): 
\\	八百屋(やおや): 
\\	屋 (や)
\\	握			
\\	にぎ 
\\	アク	にぎ.る	把握(はあく): 
\\	握手(あくしゅ): 
\\	握る(にぎる): 
\\	握る (にぎ.る)
\\	屈			
\\	クツ	かが.む、かが.める	窮屈(きゅうくつ): 
\\	屈折(くっせつ): 
\\	理屈(りくつ): 
\\	退屈(たいくつ): 
\\	掘			
\\	クツ	ほ.る	採掘(さいくつ): 
\\	発掘(はっくつ): 
\\	掘る(ほる): 
\\	掘る (ほ.る)
\\	堀			
\\	クツ	ほり	堀(ほり): 
\\	堀 (ほり)
\\	居			
\\	キョ、コ	い.る、-い、お.る	隠居(いんきょ): 
\\	居る(おる): 
\\	居住(きょじゅう): 
\\	皇居(こうきょ): 
\\	転居(てんきょ): 
\\	同居(どうきょ): 
\\	鳥居(とりい): 
\\	居眠り(いねむり): 
\\	居間(いま): 
\\	芝居(しばい): 
\\	住居(じゅうきょ): 
\\	居る(いる): 
\\	居る (い.る)
\\	据			
\\	キョ	す.える、す.わる	据え付ける(すえつける): 
\\	据える(すえる): 
\\	据える (す.える), 据わる (す.わる)
\\	層			
\\	ソウ		階層(かいそう): 
\\	一層(いっそう): 
\\	高層(こうそう): 
\\	大層(たいそう): 
\\	局			
\\	キョク	つぼね	局限(きょくげん): 
\\	局(きょく): 
\\	結局(けっきょく): 
\\	薬局(やっきょく): 
\\	郵便局(ゆうびんきょく): 
\\	遅			
\\	チ	おく.れる、おく.らす、おそ.い	遅らす(おくらす): 
\\	遅れ(おくれ): 
\\	遅くとも(おそくとも): 
\\	手遅れ(ておくれ): 
\\	遅刻(ちこく): 
\\	遅れる(おくれる): 
\\	遅い(おそい): 
\\	遅らす (おく.らす), 遅れる (おく.れる), 遅い (おそ.い)
\\	漏			
\\	ロウ	も.る、も.れる、も.らす	漏らす(もらす): 
\\	漏る(もる): 
\\	漏れる(もれる): 
\\	漏らす (も.らす), 漏る (も.る), 漏れる (も.れる)
\\	刷			
\\	サツ	す.る、-ず.り、-ずり、は.く	刷り(すり): 
\\	印刷(いんさつ): 
\\	刷る(する): 
\\	刷る (す.る)
\\	尺			
\\	シャク			
\\	尽			
\\	ジン、サン	つ.くす、-つ.くす、-づ.くし、-つ.く、-づ.く、-ず.く、つ.きる、つ.かす、さかづき、ことごと.く、つか、つき	尽きる(つきる): 
\\	尽くす(つくす): 
\\	尽かす (つ.かす), 尽きる (つ.きる), 尽くす (つ.くす)
\\	沢			
\\	タク	さわ、うるお.い、うるお.す、つや	光沢(こうたく): 
\\	贅沢(ぜいたく): 
\\	沢 (さわ)
\\	訳			
\\	ヤク	わけ	言い訳(いいわけ): 
\\	内訳(うちわけ): 
\\	通訳(つうやく): 
\\	申し訳(もうしわけ): 
\\	申し訳ない(もうしわけない): 
\\	訳(やく): 
\\	訳す(やくす): 
\\	翻訳(ほんやく): 
\\	訳(わけ): 
\\	訳 (わけ)
\\	択			
\\	タク	えら.ぶ	採択(さいたく): 
\\	選択(せんたく): 
\\	昼			
\\	チュウ	ひる	昼間(ちゅうかん): 
\\	昼飯(ちゅうはん): 
\\	お昼(おひる): 
\\	昼食(ちゅうしょく): 
\\	昼寝(ひるね): 
\\	昼間(ひるま): 
\\	昼休み(ひるやすみ): 
\\	昼(ひる): 
\\	昼御飯(ひるごはん): 
\\	昼 (ひる)
\\	戸			
\\	コ	と	戸(こ): 
\\	戸籍(こせき): 
\\	戸締り(とじまり): 
\\	雨戸(あまど): 
\\	井戸(いど): 
\\	瀬戸物(せともの): 
\\	戸棚(とだな): 
\\	戸(と): 
\\	戸 (と)
\\	肩			
\\	ケン	かた	肩(かた): 
\\	肩 (かた)
\\	房			
\\	(ボウ).
\\	ボウ	ふさ	女房(にょうぼう): 
\\	文房具(ぶんぼうぐ): 
\\	冷房(れいぼう): 
\\	暖房(だんぼう): 
\\	房 (ふさ)
\\	扇			
\\	セン	おうぎ	団扇(うちわ): 
\\	扇子(せんす): 
\\	扇ぐ(あおぐ): 
\\	扇風機(せんぷうき): 
\\	扇 (おうぎ)
\\	炉			
\\	ロ	いろり		
\\	戻			
\\	レイ	もど.す、もど.る	取り戻す(とりもどす): 
\\	払い戻す(はらいもどす): 
\\	戻す(もどす): 
\\	戻る(もどる): 
\\	戻す (もど.す), 戻る (もど.る)
\\	涙			
\\	ルイ、レイ	なみだ	涙(なみだ): 
\\	涙 (なみだ)
\\	雇			
\\	コ	やと.う	雇用(こよう): 
\\	雇う(やとう): 
\\	雇う (やと.う)
\\	顧			
\\	コ	かえり.みる	顧みる(かえりみる): 
\\	顧みる (かえり.みる)
\\	啓			
\\	ケイ	ひら.く、さと.す	拝啓(はいけい): 
\\	示			
\\	ネ, 
\\	ジ、シ	しめ.す	暗示(あんじ): 
\\	指示(しじ): 
\\	提示(ていじ): 
\\	展示(てんじ): 
\\	掲示(けいじ): 
\\	示す(しめす): 
\\	示威 (しい), 示談 (じだん), 指示 (しじ), 示す (しめ.す)
\\	礼			
\\	レイ、ライ		無礼(ぶれい): 
\\	礼(れい): 
\\	礼儀(れいぎ): 
\\	お礼(おれい): 
\\	礼儀 (れいぎ), 謝礼 (しゃれい), 無礼 (ぶれい)
\\	祥			
\\	ショウ	さいわ.い、きざ.し、よ.い、つまび.らか		
\\	祝			
\\	シュク、シュウ	いわ.う	祝賀(しゅくが): 
\\	祝い(いわい): 
\\	祝う(いわう): 
\\	祝日(しゅくじつ): 
\\	お祝い(おいわい): 
\\	祝賀 (しゅくが), 祝日 (しゅくじつ), 慶祝 (けいしゅく), 祝う (いわ.う)
\\	福			
\\	フク		福(ふく): 
\\	福祉(ふくし): 
\\	幸福(こうふく): 
\\	祉			
\\	シ		福祉(ふくし): 
\\	社			
\\	シャ	やしろ	社交(しゃこう): 
\\	社宅(しゃたく): 
\\	出社(しゅっしゃ): 
\\	社(やしろ): 
\\	社会科学(しゃかいかがく): 
\\	社説(しゃせつ): 
\\	商社(しょうしゃ): 
\\	入社(にゅうしゃ): 
\\	社会(しゃかい): 
\\	社長(しゃちょう): 
\\	神社(じんじゃ): 
\\	新聞社(しんぶんしゃ): 
\\	会社(かいしゃ): 
\\	社 (やしろ)
\\	視			
\\	示 
\\	見 
\\	シ	み.る	監視(かんし): 
\\	近視(きんし): 
\\	視覚(しかく): 
\\	視察(しさつ): 
\\	視点(してん): 
\\	視野(しや): 
\\	重視(じゅうし): 
\\	無視(むし): 
\\	奈			
\\	ナ、ナイ、ダイ	いかん、からなし		
\\	尉			
\\	イ、ジョウ		尉(じょう): 
\\	慰			
\\	慰安婦 いあんふ, 
\\	イ	なぐさ.める、なぐさ.む	慰める(なぐさめる): 
\\	慰む (なぐさ.む), 慰める (なぐさ.める)
\\	款			
\\	カン			
\\	禁			
\\	(きん).
\\	キン		禁じる(きんじる): 
\\	禁ずる(きんずる): 
\\	禁物(きんもつ): 
\\	禁煙(きんえん): 
\\	禁止(きんし): 
\\	襟			
\\	キン	えり	襟(えり): 
\\	襟 (えり)
\\	宗			
\\	シュウ、ソウ	むね	宗教(しゅうきょう): 
\\	宗教 (しゅうきょう), 宗派 (しゅうは), 改宗 (かいしゅう)
\\	崇			
\\	スウ	あが.める	崇拝(すうはい): 
\\	祭			
\\	サイ	まつ.る、まつ.り、まつり	祭日(さいじつ): 
\\	祭(まつり): 
\\	祭る(まつる): 
\\	お祭り(おまつり): 
\\	祭り (まつ.り), 祭る (まつ.る)
\\	察			
\\	(祭-
\\	サツ 警察 けいさつ
\\	擦1104.	
\\	サツ		察する(さっする): 
\\	視察(しさつ): 
\\	観察(かんさつ): 
\\	診察(しんさつ): 
\\	警察(けいさつ): 
\\	擦			
\\	サツ	す.る、す.れる、-ず.れ、こす.る、こす.れる	擦る(かする): 
\\	擦れ違い(すれちがい): 
\\	擦れる(すれる): 
\\	擦る(こする): 
\\	摩擦(まさつ): 
\\	擦る (す.る), 擦れる (す.れる)
\\	由			
\\	ユ、ユウ、ユイ	よし、よ.る	経由(けいゆ): 
\\	不自由(ふじゆう): 
\\	自由(じゆう): 
\\	理由(りゆう): 
\\	由来 (ゆらい), 経由 (けいゆ), 自由 (じゆう), 理由 (りゆう), 事由 (じゆう), 由 (よし)
\\	抽			
\\	チュウ	ひき-	抽選(ちゅうせん): 
\\	抽象(ちゅうしょう): 
\\	油			
\\	ユ、ユウ	あぶら	油絵(あぶらえ): 
\\	原油(げんゆ): 
\\	油(あぶら): 
\\	醤油(しょうゆ): 
\\	石油(せきゆ): 
\\	灯油(とうゆ): 
\\	油断(ゆだん): 
\\	油 (あぶら)
\\	袖			
\\	シュウ	そで	袖(そで): 
\\	袖 (そで)
\\	宙			
\\	チュウ		宙返り(ちゅうがえり): 
\\	宇宙(うちゅう): 
\\	届			
\\	カイ	とど.ける、-とど.け、とど.く	届け(とどけ): 
\\	届く(とどく): 
\\	届ける(とどける): 
\\	届く (とど.く), 届ける (とど.ける)
\\	笛			
\\	テキ	ふえ	笛(ふえ): 
\\	笛 (ふえ)
\\	軸			
\\	ジク		軸(じく): 
\\	甲			
\\	コウ、カン、カ	きのえ	甲(きのえ): 
\\	甲乙 (こうおつ), 装甲車 (そうこうしゃ)
\\	押			
\\	オウ	お.す、お.し-、お.っ-、お.さえる、おさ.える	押さえる(おさえる): 
\\	押し込む(おしこむ): 
\\	押し寄せる(おしよせる): 
\\	押える(おさえる): 
\\	押し入れ(おしいれ): 
\\	押す(おす): 
\\	押さえる (お.さえる), 押す (お.す)
\\	岬			
\\	コウ	みさき	岬(みさき): 
\\	岬 (みさき)
\\	挿			
\\	ソウ	さ.す、はさ.む	挿す(さす): 
\\	挿す (さ.す)
\\	申			
\\	曰 
\\	シン	もう.す、もう.し-、さる	申告(しんこく): 
\\	申し入れる(もうしいれる): 
\\	申し込み(もうしこみ): 
\\	申出(もうしで): 
\\	申し出る(もうしでる): 
\\	申し分(もうしぶん): 
\\	申請(しんせい): 
\\	申し込む(もうしこむ): 
\\	申し訳(もうしわけ): 
\\	申し訳ない(もうしわけない): 
\\	申し上げる(もうしあげる): 
\\	申す(もうす): 
\\	申す (もう.す)
\\	伸			
\\	シン	の.びる、の.ばす、の.べる、の.す	伸ばす(のばす): 
\\	伸びる(のびる): 
\\	伸ばす (の.ばす), 伸びる (の.びる)
\\	神			
\\	シン、ジン	かみ、かん-、こう-	神聖(しんせい): 
\\	神殿(しんでん): 
\\	神秘(しんぴ): 
\\	神(かみ): 
\\	神様(かみさま): 
\\	神経(しんけい): 
\\	神話(しんわ): 
\\	精神(せいしん): 
\\	神社(じんじゃ): 
\\	神聖 (しんせい), 神経 (しんけい), 精神 (せいしん), 神 (かみ)
\\	捜			
\\	(西遊記 
\\	ソウ、シュ、シュウ	さが.す	捜査(そうさ): 
\\	捜索(そうさく): 
\\	捜す(さがす): 
\\	捜す (さが.す)
\\	果			
\\	カ 果汁 かじゅう 
\\	菓1122 課1123 (ラ): 裸1124.	
\\	カ	は.たす、はた.す、-は.たす、は.てる、-は.てる、は.て	成果(せいか): 
\\	果ない(はかない): 
\\	果たして(はたして): 
\\	果たす(はたす): 
\\	果て(はて): 
\\	果てる(はてる): 
\\	果実(かじつ): 
\\	結果(けっか): 
\\	効果(こうか): 
\\	果物(くだもの): 
\\	果たす (は.たす), 果て (は.て), 果てる (は.てる)
\\	菓			
\\	カ		菓子(かし): 
\\	お菓子(おかし): 
\\	課			
\\	カ		課外(かがい): 
\\	課題(かだい): 
\\	課(か): 
\\	課税(かぜい): 
\\	課程(かてい): 
\\	日課(にっか): 
\\	裸			
\\	ラ	はだか	裸足(はだし): 
\\	裸(はだか): 
\\	裸 (はだか)
\\	斤			
\\	キン	おの		
\\	析			
\\	セキ		分析(ぶんせき): 
\\	所			
\\	ショ	ところ、-ところ、どころ、とこ	所謂(いわゆる): 
\\	箇所(かしょ): 
\\	所在(しょざい): 
\\	所持(しょじ): 
\\	所々(しょしょ): 
\\	所属(しょぞく): 
\\	所定(しょてい): 
\\	所得(しょとく): 
\\	所が(ところが): 
\\	所で(ところで): 
\\	余所見(よそみ): 
\\	個所(かしょ): 
\\	所為(せい): 
\\	短所(たんしょ): 
\\	長所(ちょうしょ): 
\\	停留所(ていりゅうじょ): 
\\	所々(ところどころ): 
\\	場所(ばしょ): 
\\	便所(べんじょ): 
\\	名所(めいしょ): 
\\	役所(やくしょ): 
\\	余所(よそ): 
\\	近所(きんじょ): 
\\	事務所(じむしょ): 
\\	住所(じゅうしょ): 
\\	台所(だいどころ): 
\\	所(ところ): 
\\	所 (ところ)
\\	祈			
\\	キ	いの.る	祈り(いのり): 
\\	祈る(いのる): 
\\	祈る (いの.る)
\\	近			
\\	キン、コン	ちか.い	近眼(きんがん): 
\\	近々(きんきん): 
\\	近郊(きんこう): 
\\	近視(きんし): 
\\	近付く(ちかづく): 
\\	手近(てぢか): 
\\	身近(みぢか): 
\\	近代(きんだい): 
\\	接近(せっきん): 
\\	近頃(ちかごろ): 
\\	近々(ちかぢか): 
\\	近付ける(ちかづける): 
\\	近寄る(ちかよる): 
\\	付近(ふきん): 
\\	近所(きんじょ): 
\\	最近(さいきん): 
\\	近い(ちかい): 
\\	近く(ちかく): 
\\	近い (ちか.い)
\\	折			
\\	セツ	お.る、おり、お.り、-お.り、お.れる	折り返す(おりかえす): 
\\	屈折(くっせつ): 
\\	折衷(せっちゅう): 
\\	時折(ときおり): 
\\	折る(おる): 
\\	骨折(こっせつ): 
\\	折角(せっかく): 
\\	折れる(おれる): 
\\	折る (お.る), 折れる (お.れる), 折 (おり)
\\	哲			
\\	テツ	さとい、あき.らか、し.る、さば.く	哲学(てつがく): 
\\	逝			
\\	セイ	ゆ.く、い.く		逝く (ゆ.く)
\\	誓			
\\	セイ	ちか.う	誓う(ちかう): 
\\	誓う (ちか.う)
\\	暫			
\\	ザン	しばら.く	暫く(しばらく): 
\\	漸			
\\	ゼン	やや	漸く(ようやく): 
\\	断			
\\	ダン	た.つ、ことわ.る、さだ.める	決断(けつだん): 
\\	断言(だんげん): 
\\	断然(だんぜん): 
\\	断面(だんめん): 
\\	中断(ちゅうだん): 
\\	無断(むだん): 
\\	横断(おうだん): 
\\	断る(ことわる): 
\\	診断(しんだん): 
\\	断水(だんすい): 
\\	断定(だんてい): 
\\	判断(はんだん): 
\\	油断(ゆだん): 
\\	断る (ことわ.る), 断つ (た.つ)
\\	質			
\\	質問).	
\\	質【しつ】 
\\	シツ、シチ、チ	たち、ただ.す、もと、わりふ	気質(かたぎ): 
\\	質疑(しつぎ): 
\\	実質(じっしつ): 
\\	質素(しっそ): 
\\	蛋白質(たんぱくしつ): 
\\	人質(ひとじち): 
\\	品質(ひんしつ): 
\\	本質(ほんしつ): 
\\	良質(りょうしつ): 
\\	質(しつ): 
\\	性質(せいしつ): 
\\	素質(そしつ): 
\\	地質(ちしつ): 
\\	物質(ぶっしつ): 
\\	質問(しつもん): 
\\	質問 (しつもん), 質実 (しつじつ), 本質 (ほんしつ), 質屋 (しちや), 人質 (ひとじち)
\\	斥			
\\	セキ	しりぞ.ける		
\\	訴			
\\	ソ	うった.える	訴え(うったえ): 
\\	訴訟(そしょう): 
\\	訴える(うったえる): 
\\	訴える (うった.える)
\\	昨			
\\	サク		一昨日(おととい): 
\\	一昨年(おととし): 
\\	一昨昨日(さきおととい): 
\\	一昨日(いっさくじつ): 
\\	一昨年(いっさくねん): 
\\	昨(さく): 
\\	昨日(きのう): 
\\	詐			
\\	サ	いつわ.る	詐欺(さぎ): 
\\	作			
\\	サク、サ	つく.る、つく.り、-づく.り	凶作(きょうさく): 
\\	原作(げんさく): 
\\	工作(こうさく): 
\\	耕作(こうさく): 
\\	作(さく): 
\\	作戦(さくせん): 
\\	作物(さくぶつ): 
\\	作用(さよう): 
\\	創作(そうさく): 
\\	駄作(ださく): 
\\	作り(つくり): 
\\	豊作(ほうさく): 
\\	発作(ほっさ): 
\\	傑作(けっさく): 
\\	作業(さぎょう): 
\\	作者(さくしゃ): 
\\	作成(さくせい): 
\\	作製(さくせい): 
\\	作品(さくひん): 
\\	作物(さくもつ): 
\\	作家(さっか): 
\\	作曲(さっきょく): 
\\	作法(さほう): 
\\	制作(せいさく): 
\\	製作(せいさく): 
\\	操作(そうさ): 
\\	動作(どうさ): 
\\	名作(めいさく): 
\\	作文(さくぶん): 
\\	作る(つくる): 
\\	作為 (さくい), 著作 (ちょさく), 豊作 (ほうさく), 作る (つく.る)
\\	雪			
\\	セツ 積雪 せきせつ
\\	セツ	ゆき	雪崩(なだれ): 
\\	吹雪(ふぶき): 
\\	雪(ゆき): 
\\	雪 (ゆき)
\\	録			
\\	ロク		登録(とうろく): 
\\	付録(ふろく): 
\\	目録(もくろく): 
\\	記録(きろく): 
\\	録音(ろくおん): 
\\	尋			
\\	エロ 
\\	ジン	たず.ねる、ひろ	尋ねる(たずねる): 
\\	尋ねる (たず.ねる)
\\	急			
\\	キュウ	いそ.ぐ、いそ.ぎ	応急(おうきゅう): 
\\	緊急(きんきゅう): 
\\	早急(さっきゅう): 
\\	準急(じゅんきゅう): 
\\	急かす(せかす): 
\\	急激(きゅうげき): 
\\	急速(きゅうそく): 
\\	急に(きゅうに): 
\\	至急(しきゅう): 
\\	特急(とっきゅう): 
\\	急ぐ(いそぐ): 
\\	急(きゅう): 
\\	急行(きゅうこう): 
\\	急ぐ (いそ.ぐ)
\\	穏			
\\	オン	おだ.やか	穏やか(おだやか): 
\\	穏やか (おだ.やか)
\\	侵			
\\	シン	おか.す	侵す(おかす): 
\\	侵略(しんりゃく): 
\\	侵入(しんにゅう): 
\\	侵す (おか.す)
\\	浸			
\\	シン	ひた.す、ひた.る	浸ける(つける): 
\\	浸す(ひたす): 
\\	浸す (ひた.す), 浸る (ひた.る)
\\	寝			
\\	シン	ね.る、ね.かす、い.ぬ、みたまや、や.める	寝かせる(ねかせる): 
\\	寝台(しんだい): 
\\	寝坊(ねぼう): 
\\	寝巻(ねまき): 
\\	昼寝(ひるね): 
\\	朝寝坊(あさねぼう): 
\\	寝る(ねる): 
\\	寝かす (ね.かす), 寝る (ね.る)
\\	婦			
\\	フ	よめ	産婦人科(さんふじんか): 
\\	主婦(しゅふ): 
\\	夫婦(ふうふ): 
\\	婦人(ふじん): 
\\	看護婦(かんごふ): 
\\	掃			
\\	ソウ、シュ	は.く	清掃(せいそう): 
\\	掃く(はく): 
\\	掃除(そうじ): 
\\	掃く (は.く)
\\	当			
\\	(本当?!)
\\	トウ	あ.たる、あ.たり、あ.てる、あ.て、まさ.に、まさ.にべし	当たり(あたり): 
\\	当たり前(あたりまえ): 
\\	当て(あて): 
\\	当て字(あてじ): 
\\	当てはまる(あてはまる): 
\\	当てはめる(あてはめる): 
\\	該当(がいとう): 
\\	正当(せいとう): 
\\	手当て(てあて): 
\\	当選(とうせん): 
\\	当人(とうにん): 
\\	日当(にっとう): 
\\	不当(ふとう): 
\\	本当(ほんとう): 
\\	割り当て(わりあて): 
\\	当たる(あたる): 
\\	当てる(あてる): 
\\	見当(けんとう): 
\\	心当たり(こころあたり): 
\\	相当(そうとう): 
\\	妥当(だとう): 
\\	担当(たんとう): 
\\	突き当たり(つきあたり): 
\\	突き当たる(つきあたる): 
\\	当時(とうじ): 
\\	当日(とうじつ): 
\\	当番(とうばん): 
\\	日当たり(ひあたり): 
\\	弁当(べんとう): 
\\	適当(てきとう): 
\\	お弁当(おべんとう): 
\\	当たる (あ.たる), 当てる (あ.てる)
\\	争			
\\	ソウ	あらそ.う、いか.でか	争い(あらそい): 
\\	抗争(こうそう): 
\\	紛争(ふんそう): 
\\	争う(あらそう): 
\\	論争(ろんそう): 
\\	競争(きょうそう): 
\\	戦争(せんそう): 
\\	争う (あらそ.う)
\\	浄			
\\	ジョウ、セイ	きよ.める、きよ.い		
\\	事			
\\	ジ、ズ	こと、つか.う、つか.える	大事(おおごと): 
\\	お大事に(おだいじに): 
\\	議事堂(ぎじどう): 
\\	旧事(くじ): 
\\	軍事(ぐんじ): 
\\	検事(けんじ): 
\\	事(こと): 
\\	事柄(ことがら): 
\\	事によると(ことによると): 
\\	些事(さじ): 
\\	事業(じぎょう): 
\\	事項(じこう): 
\\	事前(じぜん): 
\\	従事(じゅうじ): 
\\	食事(しょくじ): 
\\	百科事典(ひゃっかじてん): 
\\	返事(へんじ): 
\\	家事(かじ): 
\\	記事(きじ): 
\\	行事(ぎょうじ): 
\\	刑事(けいじ): 
\\	工事(こうじ): 
\\	事件(じけん): 
\\	事実(じじつ): 
\\	事情(じじょう): 
\\	事態(じたい): 
\\	事務(じむ): 
\\	炊事(すいじ): 
\\	知事(ちじ): 
\\	出来事(できごと): 
\\	判事(はんじ): 
\\	無事(ぶじ): 
\\	見事(みごと): 
\\	物事(ものごと): 
\\	領事(りょうじ): 
\\	火事(かじ): 
\\	事故(じこ): 
\\	事務所(じむしょ): 
\\	大事(だいじ): 
\\	用事(ようじ): 
\\	仕事(しごと): 
\\	事物 (じぶつ), 無事 (ぶじ), 師事 (しじ), 事 (こと)
\\	唐			
\\	トウ	から		唐 (から)
\\	糖			
\\	トウ		砂糖(さとう): 
\\	康			
\\	健康, 
\\	コウ		健康(けんこう): 
\\	逮			
\\	タイ		逮捕(たいほ): 
\\	伊			
\\	イ	かれ	伊井(いい): 
\\	君			
\\	クン	きみ、-ぎみ	君(くん): 
\\	君主(くんしゅ): 
\\	諸君(しょくん): 
\\	君(きみ): 
\\	君 (きみ)
\\	群			
\\	""羊君、きて!
\\	グン	む.れる、む.れ、むら、むら.がる	群(ぐん): 
\\	群集(ぐんしゅう): 
\\	群がる(むらがる): 
\\	群れ(むれ): 
\\	群れ (む.れ), 群れる (む.れる)
\\	耐			
\\	タイ	た.える	耐える(たえる): 
\\	耐える (た.える)
\\	需			
\\	ジュ		需要(じゅよう): 
\\	必需品(ひつじゅひん): 
\\	儒			
\\	ジュ			
\\	端			
\\	タン	はし、は、はた、-ばた、はな	極端(きょくたん): 
\\	半端(はんぱ): 
\\	先端(せんたん): 
\\	途端(とたん): 
\\	端(はし): 
\\	端 (は), 端 (はし), 端 (はた)
\\	両			
\\	リョウ 両親 りょうしん
\\	リョウ	てる、ふたつ	両極(りょうきょく): 
\\	両立(りょうりつ): 
\\	両替(りょうがえ): 
\\	両側(りょうがわ): 
\\	両方(りょうほう): 
\\	両親(りょうしん): 
\\	満			
\\	マン、バン	み.ちる、み.つ、み.たす	円満(えんまん): 
\\	満月(まんげつ): 
\\	満場(まんじょう): 
\\	満たす(みたす): 
\\	不満(ふまん): 
\\	満員(まんいん): 
\\	満足(まんぞく): 
\\	満点(まんてん): 
\\	満ちる(みちる): 
\\	未満(みまん): 
\\	満たす (み.たす), 満ちる (み.ちる)
\\	画			
\\	ガ、カク、エ、カイ	えが.く、かく.する、かぎ.る、はかりごと、はか.る	画(かく): 
\\	画期(かっき): 
\\	企画(きかく): 
\\	区画(くかく): 
\\	版画(はんが): 
\\	絵画(かいが): 
\\	画家(がか): 
\\	計画(けいかく): 
\\	漫画(まんが): 
\\	映画(えいが): 
\\	映画館(えいがかん): 
\\	画家 (がか), 図画 (ずが), 映画 (えいが)
\\	歯			
\\	シ	よわい、は、よわ.い、よわい.する	歯科(しか): 
\\	歯車(はぐるま): 
\\	歯磨き(はみがき): 
\\	虫歯(むしば): 
\\	歯医者(はいしゃ): 
\\	歯(は): 
\\	歯 (は)
\\	曲			
\\	キョク	ま.がる、ま.げる	婉曲(えんきょく): 
\\	戯曲(ぎきょく): 
\\	曲(きょく): 
\\	曲がる(まがる): 
\\	曲線(きょくせん): 
\\	作曲(さっきょく): 
\\	曲げる(まげる): 
\\	曲る(まがる): 
\\	曲がる (ま.がる), 曲げる (ま.げる)
\\	曹			
\\	ソウ、ゾウ	つかさ、ともがら、へや		
\\	遭			
\\	ソウ	あ.う、あ.わせる	遭難(そうなん): 
\\	遭う(あう): 
\\	遭う (あ.う)
\\	漕			
\\	ソウ	こ.ぐ、はこ.ぶ	漕ぐ(こぐ): 
\\	槽			
\\	ソウ	ふね		
\\	斗			
\\	ト、トウ			
\\	料			
\\	リョウ		衣料(いりょう): 
\\	香辛料(こうしんりょう): 
\\	燃料(ねんりょう): 
\\	肥料(ひりょう): 
\\	料(りょう): 
\\	給料(きゅうりょう): 
\\	原料(げんりょう): 
\\	材料(ざいりょう): 
\\	食料(しょくりょう): 
\\	資料(しりょう): 
\\	送料(そうりょう): 
\\	調味料(ちょうみりょう): 
\\	無料(むりょう): 
\\	有料(ゆうりょう): 
\\	料金(りょうきん): 
\\	食料品(しょくりょうひん): 
\\	料理(りょうり): 
\\	科			
\\	カ 科学 かがく 
\\	カ		亜科(あか): 
\\	科(か): 
\\	眼科(がんか): 
\\	教科(きょうか): 
\\	産婦人科(さんふじんか): 
\\	歯科(しか): 
\\	耳鼻科(じびか): 
\\	小児科(しょうにか): 
\\	百科事典(ひゃっかじてん): 
\\	百科辞典(ひゃっかじてん): 
\\	学科(がっか): 
\\	科目(かもく): 
\\	教科書(きょうかしょ): 
\\	外科(げか): 
\\	自然科学(しぜんかがく): 
\\	社会科学(しゃかいかがく): 
\\	人文科学(じんぶんかがく): 
\\	内科(ないか): 
\\	理科(りか): 
\\	科学(かがく): 
\\	図			
\\	ツ 
\\	メ. 
\\	図る はかる 
\\	地図 ちず 
\\	意図する いとする 
\\	ズ、ト	え、はか.る	意図(いと): 
\\	指図(さしず): 
\\	図々しい(ずうずうしい): 
\\	図る(はかる): 
\\	合図(あいず): 
\\	図(ず): 
\\	図鑑(ずかん): 
\\	図形(ずけい): 
\\	図表(ずひょう): 
\\	図書(としょ): 
\\	地図(ちず): 
\\	図書館(としょかん): 
\\	図画 (ずが), 図表 (ずひょう), 地図 (ちず), 図る (はか.る)
\\	用			
\\	ヨウ	もち.いる	運用(うんよう): 
\\	慣用(かんよう): 
\\	兼用(けんよう): 
\\	公用(こうよう): 
\\	雇用(こよう): 
\\	採用(さいよう): 
\\	作用(さよう): 
\\	私用(しよう): 
\\	使用人(しようにん): 
\\	専用(せんよう): 
\\	代用(だいよう): 
\\	無用(むよう): 
\\	用件(ようけん): 
\\	用紙(ようし): 
\\	用品(ようひん): 
\\	用法(ようほう): 
\\	濫用(らんよう): 
\\	引用(いんよう): 
\\	応用(おうよう): 
\\	活用(かつよう): 
\\	器用(きよう): 
\\	実用(じつよう): 
\\	信用(しんよう): 
\\	使用(しよう): 
\\	通用(つうよう): 
\\	適用(てきよう): 
\\	日用品(にちようひん): 
\\	費用(ひよう): 
\\	用いる(もちいる): 
\\	用語(ようご): 
\\	用心(ようじん): 
\\	用途(ようと): 
\\	用(よう): 
\\	用意(ようい): 
\\	用事(ようじ): 
\\	利用(りよう): 
\\	用いる (もち.いる)
\\	庸			
\\	ヨウ			
\\	備			
\\	ビ	そな.える、そな.わる、つぶさ.に	軍備(ぐんび): 
\\	守備(しゅび): 
\\	装備(そうび): 
\\	備え付ける(そなえつける): 
\\	備える(そなえる): 
\\	備わる(そなわる): 
\\	警備(けいび): 
\\	整備(せいび): 
\\	設備(せつび): 
\\	予備(よび): 
\\	準備(じゅんび): 
\\	備える (そな.える), 備わる (そな.わる)
\\	昔			
\\	セキ、シャク	むかし	昔(むかし): 
\\	昔日 (むかしにち), 昔年 (むかしねん), 昔時 (むかしとき), 昔 (むかし)
\\	錯			
\\	サク、シャク		錯誤(さくご): 
\\	錯覚(さっかく): 
\\	借			
\\	シャク	か.りる	借り(かり): 
\\	拝借(はいしゃく): 
\\	借金(しゃっきん): 
\\	借りる(かりる): 
\\	借りる (か.りる)
\\	惜			
\\	セキ	お.しい、お.しむ	惜しむ(おしむ): 
\\	惜しい(おしい): 
\\	惜しい (お.しい), 惜しむ (お.しむ)
\\	措			
\\	昔 
\\	旧 
\\	ソ	お.く	措置(そち): 
\\	散			
\\	サン	ち.る、ち.らす、-ち.らす、ち.らかす、ち.らかる、ばら	拡散(かくさん): 
\\	散蒔く(ばらまく): 
\\	分散(ぶんさん): 
\\	解散(かいさん): 
\\	散らかす(ちらかす): 
\\	散らかる(ちらかる): 
\\	散らす(ちらす): 
\\	散る(ちる): 
\\	散歩(さんぽ): 
\\	散らかす (ち.らかす), 散らかる (ち.らかる), 散らす (ち.らす), 散る (ち.る)
\\	廿			
\\	ジュウ、ニュウ	にじゅう		
\\	庶			
\\	ショ		庶民(しょみん): 
\\	庶務(しょむ): 
\\	遮			
\\	シャ	さえぎ.る	遮る(さえぎる): 
\\	遮る (さえぎ.る)
\\	席			
\\	セキ	むしろ	着席(ちゃくせき): 
\\	客席(きゃくせき): 
\\	欠席(けっせき): 
\\	座席(ざせき): 
\\	出席(しゅっせき): 
\\	席(せき): 
\\	度			
\\	ド、ト、タク	たび、-た.い	お目出度う(おめでとう): 
\\	感度(かんど): 
\\	屹度(きっと): 
\\	進度(しんど): 
\\	度々(たびたび): 
\\	恰度(ちょうど): 
\\	程度(ていど): 
\\	度忘れ(どわすれ): 
\\	密度(みつど): 
\\	一度に(いちどに): 
\\	緯度(いど): 
\\	温度(おんど): 
\\	角度(かくど): 
\\	加速度(かそくど): 
\\	経度(けいど): 
\\	限度(げんど): 
\\	高度(こうど): 
\\	支度(したく): 
\\	湿度(しつど): 
\\	制度(せいど): 
\\	速度(そくど): 
\\	態度(たいど): 
\\	度(たび): 
\\	適度(てきど): 
\\	度(ど): 
\\	年度(ねんど): 
\\	濃度(のうど): 
\\	毎度(まいど): 
\\	一度(いちど): 
\\	今度(こんど): 
\\	度胸 (どきょう), 制度 (せいど), 限度 (げんど), 法度 (ほうど), 度 (たび)
\\	渡			
\\	ト	わた.る、-わた.る、わた.す	橋渡し(はしわたし): 
\\	見渡す(みわたす): 
\\	渡り鳥(わたりどり): 
\\	渡す(わたす): 
\\	渡る(わたる): 
\\	渡す (わた.す), 渡る (わた.る)
\\	奔			
\\	ホン	はし.る		
\\	噴			
\\	フン	ふ.く	噴出(ふんしゅつ): 
\\	噴火(ふんか): 
\\	噴水(ふんすい): 
\\	噴く (ふ.く)
\\	墳			
\\	フン			
\\	憤			
\\	フン	いきどお.る	憤慨(ふんがい): 
\\	憤る (いきどお.る)
\\	焼			
\\	尭 
\\	ショウ 全焼 ぜんしょう
\\	ショウ	や.く、や.き、や.き-、-や.き、や.ける	燃焼(ねんしょう): 
\\	日焼け(ひやけ): 
\\	夕焼け(ゆうやけ): 
\\	焼く(やく): 
\\	焼ける(やける): 
\\	焼く (や.く), 焼ける (や.ける)
\\	暁			
\\	ギョウ、キョウ	あかつき、さと.る		暁 (あかつき)
\\	半			
\\	ハン	なか.ば	半端(はんぱ): 
\\	過半数(かはんすう): 
\\	大半(たいはん): 
\\	半ば(なかば): 
\\	半径(はんけい): 
\\	半島(はんとう): 
\\	半(はん): 
\\	半分(はんぶん): 
\\	半ば (なか.ば)
\\	伴			
\\	ハン、バン	ともな.う	伴う(ともなう): 
\\	同伴 (どうはん), 随伴 (ずいはん), 伴う (ともな.う)
\\	畔			
\\	ハン	あぜ、くろ、ほとり		
\\	判			
\\	ハン、バン	わか.る	審判(しんばん): 
\\	判(はん): 
\\	判(ばん): 
\\	判決(はんけつ): 
\\	判定(はんてい): 
\\	裁判(さいばん): 
\\	審判(しんぱん): 
\\	判子(はんこ): 
\\	判事(はんじ): 
\\	判断(はんだん): 
\\	批判(ひはん): 
\\	評判(ひょうばん): 
\\	判定 (はんてい), 判明 (はんめい), 裁判 (さいばん)
\\	券			
\\	ケン		旅券(りょけん): 
\\	回数券(かいすうけん): 
\\	券(けん): 
\\	定期券(ていきけん): 
\\	巻			
\\	カン、ケン	ま.く、まき、ま.き	取り巻く(とりまく): 
\\	巻(まき): 
\\	寝巻(ねまき): 
\\	巻く(まく): 
\\	巻き (ま.き), 巻く (ま.く)
\\	圏			
\\	ケン	かこ.い	圏(けん): 
\\	勝			
\\	ショウ	か.つ、-が.ち、まさ.る、すぐ.れる、かつ	決勝(けっしょう): 
\\	勝負(しょうぶ): 
\\	勝利(しょうり): 
\\	勝る(まさる): 
\\	勝ち(かち): 
\\	勝敗(しょうはい): 
\\	優勝(ゆうしょう): 
\\	勝つ(かつ): 
\\	勝つ (か.つ), 勝る (まさ.る)
\\	藤			
\\	トウ、ドウ	ふじ		藤 (ふじ)
\\	謄			
\\	トウ			
\\	片			
\\	ヘン	かた-、かた	片思い(かたおもい): 
\\	片言(かたこと): 
\\	片付け(かたづけ): 
\\	鉄片(てっぺん): 
\\	片付く(かたづく): 
\\	片道(かたみち): 
\\	片寄る(かたよる): 
\\	破片(はへん): 
\\	片付ける(かたづける): 
\\	片仮名(かたかな): 
\\	片 (かた)
\\	版			
\\	ハン		初版(しょはん): 
\\	絶版(ぜっぱん): 
\\	版(はん): 
\\	版画(はんが): 
\\	出版(しゅっぱん): 
\\	之			
\\	え, 
\\	之
\\	シ	の、これ、おいて、ゆく、この		
\\	乏			
\\	之 
\\	ボウ	とぼ.しい、とも.しい	窮乏(きゅうぼう): 
\\	欠乏(けつぼう): 
\\	乏しい(とぼしい): 
\\	貧乏(びんぼう): 
\\	乏しい (とぼ.しい)
\\	芝			
\\	シ	しば	芝(しば): 
\\	芝居(しばい): 
\\	芝生(しばふ): 
\\	芝 (しば)
\\	不			
\\	フ、ブ		不意(ふい): 
\\	不可欠(ふかけつ): 
\\	不吉(ふきつ): 
\\	不況(ふきょう): 
\\	不景気(ふけいき): 
\\	不在(ふざい): 
\\	不山戯る(ふざける): 
\\	不順(ふじゅん): 
\\	不振(ふしん): 
\\	不審(ふしん): 
\\	不調(ふちょう): 
\\	不当(ふとう): 
\\	不動産(ふどうさん): 
\\	不評(ふひょう): 
\\	不便(ふびん): 
\\	不服(ふふく): 
\\	不明(ふめい): 
\\	不良(ふりょう): 
\\	不味い(まずい): 
\\	不安(ふあん): 
\\	不運(ふうん): 
\\	不可(ふか): 
\\	不規則(ふきそく): 
\\	不潔(ふけつ): 
\\	不幸(ふこう): 
\\	不思議(ふしぎ): 
\\	不自由(ふじゆう): 
\\	不正(ふせい): 
\\	不足(ふそく): 
\\	不通(ふつう): 
\\	不平(ふへい): 
\\	不満(ふまん): 
\\	不利(ふり): 
\\	不便(ふべん): 
\\	不当 (ふとう), 不利 (ふり), 不賛成 (ふさんせい)
\\	否			
\\	否 
\\	不 
\\	口, 
\\	ヒ	いな、いや	拒否(きょひ): 
\\	否決(ひけつ): 
\\	否(いや): 
\\	否定(ひてい): 
\\	否 (いな)
\\	杯			
\\	ハイ 一杯 いっぱい
\\	ハイ	さかずき	杯(さかずき): 
\\	乾杯(かんぱい): 
\\	杯 (さかずき)
\\	矢			
\\	失.	
\\	シ	や	矢(や): 
\\	矢っ張り(やっぱり): 
\\	矢印(やじるし): 
\\	矢 (や)
\\	矯			
\\	キョウ	た.める		矯める (た.める)
\\	族			
\\	ゾク		貴族(きぞく): 
\\	民族(みんぞく): 
\\	家族(かぞく): 
\\	知			
\\	矢 
\\	口] 
\\	チ	し.る、し.らせる	旧知(きゅうち): 
\\	知り合い(しりあい): 
\\	知人(ちじん): 
\\	知性(ちせい): 
\\	知的(ちてき): 
\\	未知(みち): 
\\	無知(むち): 
\\	知らせ(しらせ): 
\\	知恵(ちえ): 
\\	知事(ちじ): 
\\	知識(ちしき): 
\\	知能(ちのう): 
\\	通知(つうち): 
\\	承知(しょうち): 
\\	知らせる(しらせる): 
\\	知る(しる): 
\\	知 (しる)
\\	智			
\\	チ			
\\	矛			
\\	マオ 
\\	ム、ボウ	ほこ	矛盾(むじゅん): 
\\	矛 (ほこ)
\\	柔			
\\	ジュウ、ニュウ	やわ.らか、やわ.らかい、やわ、やわ.ら	柔軟(じゅうなん): 
\\	柔道(じゅうどう): 
\\	柔らかい(やわらかい): 
\\	柔軟 (じゅうなん), 柔道 (じゅうどう), 懐柔 (かいじゅう), 柔らか (やわ.らか), 柔らかい (やわ.らかい)
\\	務			
\\	ム	つと.める	業務(ぎょうむ): 
\\	勤務(きんむ): 
\\	公務(こうむ): 
\\	職務(しょくむ): 
\\	庶務(しょむ): 
\\	税務署(ぜいむしょ): 
\\	責務(せきむ): 
\\	務める(つとめる): 
\\	任務(にんむ): 
\\	義務(ぎむ): 
\\	事務(じむ): 
\\	務め(つとめ): 
\\	公務員(こうむいん): 
\\	事務所(じむしょ): 
\\	務める (つと.める)
\\	霧			
\\	ム、ボウ、ブ	きり	霧(きり): 
\\	霧 (きり)
\\	班			
\\	ハン		班(はん): 
\\	帰			
\\	キ	かえ.る、かえ.す、おく.る、とつ.ぐ	帰す(かえす): 
\\	帰京(ききょう): 
\\	帰宅(きたく): 
\\	日帰り(ひがえり): 
\\	帰り(かえり): 
\\	帰る(かえる): 
\\	帰す (かえ.す), 帰る (かえ.る)
\\	弓			
\\	キュウ	ゆみ	弓(ゆみ): 
\\	弓 (ゆみ)
\\	引			
\\	イン	ひ.く、ひ.き、ひ.き-、-び.き、ひ.ける	籤引(くじびき): 
\\	差し引く(さしひく): 
\\	手引き(てびき): 
\\	取り引き(とりひき): 
\\	値引き(ねびき): 
\\	引き上げる(ひきあげる): 
\\	引き受ける(ひきうける): 
\\	引き起こす(ひきおこす): 
\\	引き下げる(ひきさげる): 
\\	引きずる(ひきずる): 
\\	引き出す(ひきだす): 
\\	引き取る(ひきとる): 
\\	引き分け(ひきわけ): 
\\	引っ掻く(ひっかく): 
\\	引っ掛ける(ひっかける): 
\\	引用(いんよう): 
\\	引退(いんたい): 
\\	引力(いんりょく): 
\\	強引(ごういん): 
\\	索引(さくいん): 
\\	差し引き(さしひき): 
\\	長引く(ながびく): 
\\	引受る(ひきうける): 
\\	引返す(ひきかえす): 
\\	引算(ひきざん): 
\\	引出す(ひきだす): 
\\	引き止める(ひきとめる): 
\\	引分け(ひきわけ): 
\\	引っ掛かる(ひっかかる): 
\\	引っ繰り返す(ひっくりかえす): 
\\	引っ繰り返る(ひっくりかえる): 
\\	引越し(ひっこし): 
\\	引っ込む(ひっこむ): 
\\	引っ張る(ひっぱる): 
\\	割引(わりびき): 
\\	引き出し(ひきだし): 
\\	引っ越す(ひっこす): 
\\	字引(じびき): 
\\	引く(ひく): 
\\	引く (ひ.く), 引ける (ひ.ける)
\\	弔			
\\	チョウ	とむら.う、とぶら.う		弔う (とむら.う)
\\	弘			
\\	コウ、グ	ひろ.い		
\\	強			
\\	キョウ、ゴウ	つよ.い、つよ.まる、つよ.める、し.いる、こわ.い	強行(きょうこう): 
\\	強硬(きょうこう): 
\\	強制(きょうせい): 
\\	強烈(きょうれつ): 
\\	強気(ごうぎ): 
\\	心強い(こころづよい): 
\\	強いて(しいて): 
\\	強いる(しいる): 
\\	増強(ぞうきょう): 
\\	強まる(つよまる): 
\\	強める(つよめる): 
\\	強請る(ねだる): 
\\	補強(ほきょう): 
\\	強化(きょうか): 
\\	強調(きょうちょう): 
\\	強力(きょうりょく): 
\\	強引(ごういん): 
\\	強盗(ごうとう): 
\\	力強い(ちからづよい): 
\\	強気(つよき): 
\\	強い(つよい): 
\\	勉強(べんきょう): 
\\	強弱 (きょうじゃく), 強要 (きょうよう), 勉強 (べんきょう), 強いる (し.いる), 強い (つよ.い), 強まる (つよ.まる), 強める (つよ.める)
\\	弱			
\\	ジャク	よわ.い、よわ.る、よわ.まる、よわ.める	弱(じゃく): 
\\	薄弱(はくじゃく): 
\\	貧弱(ひんじゃく): 
\\	弱まる(よわまる): 
\\	弱める(よわめる): 
\\	弱る(よわる): 
\\	弱点(じゃくてん): 
\\	弱い(よわい): 
\\	弱い (よわ.い), 弱まる (よわ.まる), 弱める (よわ.める), 弱る (よわ.る)
\\	沸			
\\	フツ	わ.く、わ.かす	沸騰(ふっとう): 
\\	沸かす(わかす): 
\\	沸く(わく): 
\\	沸かす (わ.かす), 沸く (わ.く)
\\	費			
\\	ヒ	つい.やす、つい.える	経費(けいひ): 
\\	光熱費(こうねつひ): 
\\	実費(じっぴ): 
\\	出費(しゅっぴ): 
\\	費やす(ついやす): 
\\	費(ひ): 
\\	浪費(ろうひ): 
\\	消費(しょうひ): 
\\	費用(ひよう): 
\\	費える (つい.える), 費やす (つい.やす)
\\	第			
\\	ダイ、テイ		第(だい): 
\\	第一(だいいち): 
\\	次第(しだい): 
\\	落第(らくだい): 
\\	弟			
\\	テイ、ダイ、デ	おとうと	弟(おと): 
\\	従兄弟(いとこ): 
\\	弟(おとうと): 
\\	兄弟(きょうだい): 
\\	弟妹 (ていまい), 義弟 (ぎてい), 子弟 (してい), 兄弟 (きょうだい), 弟 (おとうと)
\\	巧			
\\	コウ	たく.み、たく.む、うま.い	巧妙(こうみょう): 
\\	精巧(せいこう): 
\\	巧み(たくみ): 
\\	巧み (たく.み)
\\	号			
\\	ゴウ	さけ.ぶ、よびな	号(ごう): 
\\	年号(ねんごう): 
\\	記号(きごう): 
\\	信号(しんごう): 
\\	符号(ふごう): 
\\	番号(ばんごう): 
\\	朽			
\\	枯 
\\	キュウ	く.ちる	朽ちる(くちる): 
\\	朽ちる (く.ちる)
\\	誇			
\\	コ	ほこ.る	誇張(こちょう): 
\\	誇る(ほこる): 
\\	誇り(ほこり): 
\\	誇る (ほこ.る)
\\	汚			
\\	オ	けが.す、けが.れる、けが.らわしい、よご.す、よご.れる、きたな.い	汚す(けがす): 
\\	汚らわしい(けがらわしい): 
\\	汚れ(けがれ): 
\\	汚れる(けがれる): 
\\	汚染(おせん): 
\\	汚す(よごす): 
\\	汚れる(よごれる): 
\\	汚い(きたない): 
\\	汚い (きたな.い), 汚す (けが.す), 汚らわしい (けが.らわしい), 汚れる (けが.れる), 汚す (よご.す), 汚れる (よご.れる)
\\	与			
\\	ヨ	あた.える、あずか.る、くみ.する、ともに	関与(かんよ): 
\\	寄与(きよ): 
\\	与党(よとう): 
\\	与える(あたえる): 
\\	給与(きゅうよ): 
\\	与える (あた.える)
\\	写			
\\	シャ、ジャ	うつ.す、うつ.る、うつ-、うつ.し	写し(うつし): 
\\	映写(えいしゃ): 
\\	描写(びょうしゃ): 
\\	写る(うつる): 
\\	写生(しゃせい): 
\\	複写(ふくしゃ): 
\\	写す(うつす): 
\\	写真(しゃしん): 
\\	写す (うつ.す), 写る (うつ.る)
\\	身			
\\	シン	み	受身(うけみ): 
\\	身体(からだ): 
\\	中身(なかみ): 
\\	生身(なまみ): 
\\	身近(みぢか): 
\\	身なり(みなり): 
\\	身振り(みぶり): 
\\	刺身(さしみ): 
\\	自身(じしん): 
\\	出身(しゅっしん): 
\\	心身(しんしん): 
\\	身体(しんたい): 
\\	身長(しんちょう): 
\\	全身(ぜんしん): 
\\	独身(どくしん): 
\\	身(み): 
\\	身分(みぶん): 
\\	身 (み)
\\	射			
\\	シャ	い.る、さ.す	反射(はんしゃ): 
\\	放射(ほうしゃ): 
\\	放射能(ほうしゃのう): 
\\	射す(さす): 
\\	発射(はっしゃ): 
\\	陽射(ひざし): 
\\	注射(ちゅうしゃ): 
\\	射る (い.る)
\\	謝			
\\	シャ	あやま.る	月謝(げっしゃ): 
\\	謝絶(しゃぜつ): 
\\	感謝(かんしゃ): 
\\	謝る(あやまる): 
\\	謝る (あやま.る)
\\	老			
\\	ロウ	お.いる、ふ.ける	老いる(おいる): 
\\	老ける(ふける): 
\\	老衰(ろうすい): 
\\	老人(ろうじん): 
\\	老いる (お.いる), 老ける (ふ.ける)
\\	考			
\\	コウ	かんが.える、かんが.え	考古学(こうこがく): 
\\	思考(しこう): 
\\	選考(せんこう): 
\\	考え(かんがえ): 
\\	考慮(こうりょ): 
\\	参考(さんこう): 
\\	考える(かんがえる): 
\\	考える (かんが.える)
\\	孝			
\\	コウ、キョウ		孝行(こうこう): 
\\	教			
\\	キョウ	おし.える、おそ.わる	教え(おしえ): 
\\	教員(きょういん): 
\\	教科(きょうか): 
\\	教訓(きょうくん): 
\\	教材(きょうざい): 
\\	教習(きょうしゅう): 
\\	教職(きょうしょく): 
\\	宣教(せんきょう): 
\\	教わる(おそわる): 
\\	教科書(きょうかしょ): 
\\	教師(きょうし): 
\\	教授(きょうじゅ): 
\\	教養(きょうよう): 
\\	宗教(しゅうきょう): 
\\	助教授(じょきょうじゅ): 
\\	教育(きょういく): 
\\	教会(きょうかい): 
\\	教える(おしえる): 
\\	教室(きょうしつ): 
\\	教える (おし.える), 教わる (おそ.わる)
\\	拷			
\\	ゴウ			
\\	者			
\\	忍者 
\\	忍). 
\\	忍者 
\\	身 
\\	者 
\\	忍 
\\	拷!	
\\	シャ	もの	業者(ぎょうしゃ): 
\\	信者(しんじゃ): 
\\	達者(たっしゃ): 
\\	読者(どくしゃ): 
\\	配偶者(はいぐうしゃ): 
\\	悪者(わるもの): 
\\	学者(がくしゃ): 
\\	患者(かんじゃ): 
\\	記者(きしゃ): 
\\	後者(こうしゃ): 
\\	作者(さくしゃ): 
\\	前者(ぜんしゃ): 
\\	著者(ちょしゃ): 
\\	筆者(ひっしゃ): 
\\	者(もの): 
\\	役者(やくしゃ): 
\\	歯医者(はいしゃ): 
\\	医者(いしゃ): 
\\	者 (もの)
\\	煮			
\\	シャ	に.る、-に、に.える、に.やす	煮える(にえる): 
\\	煮る(にる): 
\\	煮える (に.える), 煮やす (に.やす), 煮る (に.る)
\\	著			
\\	チョ、チャク	あらわ.す、いちじる.しい	著しい(いちじるしい): 
\\	著書(ちょしょ): 
\\	著名(ちょめい): 
\\	著す(あらわす): 
\\	著者(ちょしゃ): 
\\	著す (あらわ.す), 著しい (いちじる.しい)
\\	署			
\\	ショ		税務署(ぜいむしょ): 
\\	消防署(しょうぼうしょ): 
\\	署名(しょめい): 
\\	暑			
\\	(暑い) 
\\	ショ	あつ.い	蒸し暑い(むしあつい): 
\\	暑い(あつい): 
\\	暑い (あつ.い)
\\	諸			
\\	ショ	もろ	諸(しょ): 
\\	諸君(しょくん): 
\\	猪			
\\	チョ	い、いのしし		
\\	渚			
\\	ショ	なぎさ		
\\	賭			
\\	ト	か.ける、かけ	賭け(かけ): 
\\	賭ける(かける): 
\\	賭ける (か.ける)
\\	峡			
\\	キョウ、コウ	はざま	海峡(かいきょう): 
\\	狭			
\\	キョウ、コウ	せま.い、せば.める、せば.まる、さ	狭い(せまい): 
\\	狭まる (せば.まる), 狭める (せば.める), 狭い (せま.い)
\\	挟			
\\	挟まる 【はさまる】 
\\	キョウ、ショウ	はさ.む、はさ.まる、わきばさ.む、さしはさ.む	挟まる(はさまる): 
\\	挟む(はさむ): 
\\	挟まる (はさ.まる), 挟む (はさ.む)
\\	追			
\\	ツイ	お.う	追い込む(おいこむ): 
\\	追い出す(おいだす): 
\\	追及(ついきゅう): 
\\	追跡(ついせき): 
\\	追放(ついほう): 
\\	追い掛ける(おいかける): 
\\	追い越す(おいこす): 
\\	追い付く(おいつく): 
\\	追う(おう): 
\\	追加(ついか): 
\\	追う (お.う)
\\	師			
\\	シ、ス	もろもろ、なら.う	牧師(ぼくし): 
\\	医師(いし): 
\\	技師(ぎし): 
\\	教師(きょうし): 
\\	講師(こうし): 
\\	漁師(りょうし): 
\\	帥			
\\	スイ			
\\	官			
\\	カン		官僚(かんりょう): 
\\	器官(きかん): 
\\	長官(ちょうかん): 
\\	官庁(かんちょう): 
\\	警官(けいかん): 
\\	棺			
\\	カン			
\\	管			
\\	カン	くだ	管(かん): 
\\	血管(けっかん): 
\\	保管(ほかん): 
\\	管理(かんり): 
\\	管(くだ): 
\\	管 (くだ)
\\	父			
\\	フ	ちち	お祖父さん(おじいさん): 
\\	伯父さん(おじさん): 
\\	父母(ちちはは): 
\\	小父さん(おじさん): 
\\	父親(ちちおや): 
\\	父母(ふぼ): 
\\	祖父(そふ): 
\\	叔父(おじ): 
\\	お父さん(おとうさん): 
\\	父 (ちち)
\\	交			
\\	コウ	まじ.わる、まじ.える、ま.じる、まじ.る、ま.ざる、ま.ぜる、-か.う、か.わす、かわ.す、こもごも	交わす(かわす): 
\\	交易(こうえき): 
\\	交互(こうご): 
\\	交渉(こうしょう): 
\\	交付(こうふ): 
\\	国交(こっこう): 
\\	社交(しゃこう): 
\\	交ざる(まざる): 
\\	交える(まじえる): 
\\	交じる(まじる): 
\\	交わる(まじわる): 
\\	外交(がいこう): 
\\	交換(こうかん): 
\\	交差(こうさ): 
\\	交際(こうさい): 
\\	交差点(こうさてん): 
\\	交替(こうたい): 
\\	交通機関(こうつうきかん): 
\\	交流(こうりゅう): 
\\	交ぜる(まぜる): 
\\	交通(こうつう): 
\\	交番(こうばん): 
\\	交う (か.う), 交わす (か.わす), 交ざる (ま.ざる), 交じる (ま.じる), 交ぜる (ま.ぜる), 交える (まじ.える), 交わる (まじ.わる)
\\	効			
\\	コウ	き.く、ききめ、なら.う	効き目(ききめ): 
\\	効率(こうりつ): 
\\	無効(むこう): 
\\	効く(きく): 
\\	効果(こうか): 
\\	効力(こうりょく): 
\\	有効(ゆうこう): 
\\	効く (き.く)
\\	較			
\\	カク、コウ	くら.べる	比較(ひかく): 
\\	比較的(ひかくてき): 
\\	校			
\\	試.	
\\	校長 
\\	中学校 
\\	コウ、キョウ		校(こう): 
\\	高等学校(こうとうがっこう): 
\\	転校(てんこう): 
\\	登校(とうこう): 
\\	母校(ぼこう): 
\\	校舎(こうしゃ): 
\\	校庭(こうてい): 
\\	高校(こうこう): 
\\	高校生(こうこうせい): 
\\	校長(こうちょう): 
\\	小学校(しょうがっこう): 
\\	中学校(ちゅうがっこう): 
\\	学校(がっこう): 
\\	足			
\\	ソク	あし、た.りる、た.る、た.す	駆け足(かけあし): 
\\	足し算(たしざん): 
\\	裸足(はだし): 
\\	発足(はっそく): 
\\	補足(ほそく): 
\\	物足りない(ものたりない): 
\\	足跡(あしあと): 
\\	遠足(えんそく): 
\\	足袋(たび): 
\\	足りる(たりる): 
\\	足る(たる): 
\\	不足(ふそく): 
\\	満足(まんぞく): 
\\	足す(たす): 
\\	足(あし): 
\\	足 (あし), 足す (た.す), 足りる (た.りる), 足る (た.る)
\\	促			
\\	ソク	うなが.す	促す(うながす): 
\\	促進(そくしん): 
\\	催促(さいそく): 
\\	促す (うなが.す)
\\	距			
\\	キョ	へだ.たる、けづめ	距離(きょり): 
\\	路			
\\	ロ、ル	-じ、 みち	海路(うみじ): 
\\	回路(かいろ): 
\\	経路(けいろ): 
\\	十字路(じゅうじろ): 
\\	進路(しんろ): 
\\	針路(しんろ): 
\\	線路(せんろ): 
\\	通路(つうろ): 
\\	道路(どうろ): 
\\	路 (じ)
\\	露			
\\	ロ、ロウ	つゆ	露(つゆ): 
\\	暴露(ばくろ): 
\\	露骨(ろこつ): 
\\	露出 (ろしゅつ), 露店 (ろてん), 雨露 (あめつゆ), 露 (つゆ)
\\	跳			
\\	チョウ	は.ねる、と.ぶ、-と.び	跳ぶ(とぶ): 
\\	跳ねる(はねる): 
\\	跳ぶ (と.ぶ), 跳ねる (は.ねる)
\\	躍			
\\	ヤク	おど.る	活躍(かつやく): 
\\	躍る (おど.る)
\\	践			
\\	セン	ふ.む	実践(じっせん): 
\\	踏			
\\	トウ	ふ.む、ふ.まえる	踏まえる(ふまえる): 
\\	踏切(ふみきり): 
\\	踏む(ふむ): 
\\	踏まえる (ふ.まえる), 踏む (ふ.む)
\\	骨			
\\	口 
\\	田 
\\	月 
\\	コツ	ほね	骨(こつ): 
\\	骨董品(こっとうひん): 
\\	露骨(ろこつ): 
\\	骨折(こっせつ): 
\\	骨(ほね): 
\\	骨 (ほね)
\\	滑			
\\	カツ、コツ	すべ.る、なめ.らか	円滑(えんかつ): 
\\	滑稽(こっけい): 
\\	滑る(すべる): 
\\	滑らか(なめらか): 
\\	滑れる(ずれる): 
\\	滑る (すべ.る), 滑らか (なめ.らか)
\\	髄			
\\	ズイ			
\\	禍			
\\	か 禍福 かふく 
\\	渦1292 過1293.	
\\	カ	わざわい		
\\	渦			
\\	カ	うず	渦(うず): 
\\	渦 (うず)
\\	過			
\\	カ	す.ぎる、-す.ぎる、-す.ぎ、す.ごす、あやま.つ、あやま.ち、よ.ぎる	過ち(あやまち): 
\\	過疎(かそ): 
\\	過多(かた): 
\\	過密(かみつ): 
\\	過労(かろう): 
\\	経過(けいか): 
\\	過ぎ(すぎ): 
\\	過去(かこ): 
\\	過失(かしつ): 
\\	過剰(かじょう): 
\\	過程(かてい): 
\\	過半数(かはんすう): 
\\	過ごす(すごす): 
\\	超過(ちょうか): 
\\	通過(つうか): 
\\	通り過ぎる(とおりすぎる): 
\\	過ぎる(すぎる): 
\\	過ち (あやま.ち), 過つ (あやま.つ), 過ぎる (す.ぎる), 過ごす (す.ごす)
\\	阪			
\\	大阪 
\\	ハン	さか		阪 (さか)
\\	阿			
\\	ア、オ	おもね.る		
\\	際			
\\	サイ	きわ、-ぎわ	際(きわ): 
\\	手際(てぎわ): 
\\	交際(こうさい): 
\\	際(さい): 
\\	実際(じっさい): 
\\	国際(こくさい): 
\\	際 (きわ)
\\	障			
\\	ショウ	さわ.る	気障(きざ): 
\\	障る(さわる): 
\\	保障(ほしょう): 
\\	障害(しょうがい): 
\\	障子(しょうじ): 
\\	故障(こしょう): 
\\	障る (さわ.る)
\\	随			
\\	ズイ	まにまに、したが.う	随分(ずいぶん): 
\\	随筆(ずいひつ): 
\\	陪			
\\	バイ			
\\	陽			
\\	ヨウ	ひ	太陽(たいよう): 
\\	陽射(ひざし): 
\\	陽気(ようき): 
\\	陳			
\\	チン	ひ.ねる	陳列(ちんれつ): 
\\	防			
\\	ボウ	ふせ.ぐ	国防(こくぼう): 
\\	堤防(ていぼう): 
\\	防衛(ぼうえい): 
\\	防火(ぼうか): 
\\	消防(しょうぼう): 
\\	消防署(しょうぼうしょ): 
\\	防ぐ(ふせぐ): 
\\	防止(ぼうし): 
\\	防犯(ぼうはん): 
\\	予防(よぼう): 
\\	防ぐ (ふせ.ぐ)
\\	附			
\\	フ	つ.ける、つ.く	附属(ふぞく): 
\\	院			
\\	イン		参議院(さんぎいん): 
\\	衆議院(しゅうぎいん): 
\\	寺院(じいん): 
\\	大学院(だいがくいん): 
\\	退院(たいいん): 
\\	入院(にゅういん): 
\\	病院(びょういん): 
\\	陣			
\\	ジン			
\\	隊			
\\	タイ		軍隊(ぐんたい): 
\\	兵隊(へいたい): 
\\	墜			
\\	ツイ	お.ちる、お.つ	墜落(ついらく): 
\\	降			
\\	コウ、ゴ	お.りる、お.ろす、ふ.る、ふ.り、くだ.る、くだ.す	降ろす(おろす): 
\\	降水(こうすい): 
\\	降伏(こうふく): 
\\	以降(いこう): 
\\	下降(かこう): 
\\	降りる(おりる): 
\\	降る(ふる): 
\\	降りる (お.りる), 降ろす (お.ろす), 降る (ふ.る)
\\	階			
\\	カイ	きざはし	階(かい): 
\\	階級(かいきゅう): 
\\	階層(かいそう): 
\\	段階(だんかい): 
\\	階段(かいだん): 
\\	陛			
\\	ヘイ			
\\	隣			
\\	リン	とな.る、となり	隣(となり): 
\\	隣る (とな.る), 隣 (となり)
\\	隔			
\\	カク	へだ.てる、へだ.たる	隔週(かくしゅう): 
\\	隔たる(へだたる): 
\\	間隔(かんかく): 
\\	隔てる(へだてる): 
\\	隔たる (ひだ.たる), 隔てる (へだ.てる)
\\	隠			
\\	イン、オン	かく.す、かく.し、かく.れる、かか.す、よ.る	隠居(いんきょ): 
\\	隠す(かくす): 
\\	隠れる(かくれる): 
\\	隠す (かく.す), 隠れる (かく.れる)
\\	堕			
\\	ダ	お.ちる、くず.す、くず.れる		
\\	陥			
\\	カン	おちい.る、おとしい.れる	欠陥(けっかん): 
\\	陥る (おちい.る), 陥れる (おとしい.れる)
\\	穴			
\\	ケツ	あな	穴(あな): 
\\	穴 (あな)
\\	空			
\\	クウ	そら、あ.く、あ.き、あ.ける、から、す.く、す.かす、むな.しい	空き(あき): 
\\	空間(あきま): 
\\	空ろ(うつろ): 
\\	大空(おおぞら): 
\\	空腹(くうふく): 
\\	上空(じょうくう): 
\\	空く(すく): 
\\	空しい(むなしい): 
\\	架空(かくう): 
\\	空(から): 
\\	空っぽ(からっぽ): 
\\	空想(くうそう): 
\\	空中(くうちゅう): 
\\	航空(こうくう): 
\\	真空(しんくう): 
\\	空く(あく): 
\\	空気(くうき): 
\\	空港(くうこう): 
\\	空(そら): 
\\	空く (あ.く), 空ける (あ.ける), 空 (から), 空 (そら)
\\	控			
\\	空手 
\\	控 
\\	コウ	ひか.える、ひか.え	控除(こうじょ): 
\\	控室(ひかえしつ): 
\\	控える(ひかえる): 
\\	控える (ひか.える)
\\	突			
\\	トツ、カ	つ.く	突く(つつく): 
\\	突っ張る(つっぱる): 
\\	突如(とつじょ): 
\\	突破(とっぱ): 
\\	煙突(えんとつ): 
\\	衝突(しょうとつ): 
\\	突き当たり(つきあたり): 
\\	突き当たる(つきあたる): 
\\	突く(つく): 
\\	突っ込む(つっこむ): 
\\	突然(とつぜん): 
\\	突く (つ.く)
\\	究			
\\	キュウ、ク	きわ.める	究極(きゅうきょく): 
\\	研究(けんきゅう): 
\\	研究室(けんきゅうしつ): 
\\	究める (きわ.める)
\\	窒			
\\	チツ		窒息(ちっそく): 
\\	窃			
\\	セツ	ぬす.む、ひそ.か		
\\	窪			
\\	ワ、ア	くぼ.む、くぼ.み、くぼ.まる、くぼ		
\\	搾			
\\	サク	しぼ.る		搾る (しぼ.る)
\\	窯			
\\	ヨウ	かま		窯 (かま)
\\	窮			
\\	キュウ、キョウ	きわ.める、きわ.まる、きわ.まり、きわ.み	窮屈(きゅうくつ): 
\\	窮乏(きゅうぼう): 
\\	窮まる (きわ.まる), 窮める (きわ.める)
\\	探			
\\	タン	さぐ.る、さが.す	探検(たんけん): 
\\	探る(さぐる): 
\\	探す(さがす): 
\\	探す (さが.す), 探る (さぐ.る)
\\	深			
\\	シン	ふか.い、-ぶか.い、ふか.まる、ふか.める、み-	情け深い(なさけぶかい): 
\\	深める(ふかめる): 
\\	欲深い(よくふかい): 
\\	深夜(しんや): 
\\	深刻(しんこく): 
\\	深まる(ふかまる): 
\\	深い(ふかい): 
\\	深い (ふか.い), 深まる (ふか.まる), 深める (ふか.める)
\\	丘			
\\	キュウ	おか	丘陵(きゅうりょう): 
\\	丘(おか): 
\\	丘 (おか)
\\	岳			
\\	ガク	たけ	山岳(さんがく): 
\\	岳 (たけ)
\\	兵			
\\	ヘイ、ヒョウ	つわもの	兵器(へいき): 
\\	兵士(へいし): 
\\	兵隊(へいたい): 
\\	兵器 (へいき), 兵隊 (へいたい), 撤兵 (てっぺい)
\\	浜			
\\	ヒン	はま	浜(はま): 
\\	浜辺(はまべ): 
\\	浜 (はま)
\\	糸			
\\	シ	いと	毛糸(けいと): 
\\	糸(いと): 
\\	糸 (いと)
\\	織			
\\	ショク、シキ	お.る、お.り、おり、-おり、-お.り	織(おり): 
\\	織物(おりもの): 
\\	織る(おる): 
\\	組織(そしき): 
\\	織機 (しょっき), 染織 (せんしょく), 紡織 (ぼうしょく), 織る (お.る)
\\	繕			
\\	ゼン	つくろ.う	繕う(つくろう): 
\\	修繕(しゅうぜん): 
\\	繕う (つくろ.う)
\\	縮			
\\	シュク	ちぢ.む、ちぢ.まる、ちぢ.める、ちぢ.れる、ちぢ.らす	短縮(たんしゅく): 
\\	縮まる(ちぢまる): 
\\	圧縮(あっしゅく): 
\\	恐縮(きょうしゅく): 
\\	縮小(しゅくしょう): 
\\	縮む(ちぢむ): 
\\	縮める(ちぢめる): 
\\	縮れる(ちぢれる): 
\\	縮まる (ちぢ.まる), 縮む (ちぢ.む), 縮める (ちぢ.める), 縮らす (ちぢ.らす), 縮れる (ちぢ.れる)
\\	繁			
\\	ハン	しげ.る、しげ.く	繁栄(はんえい): 
\\	繁盛(はんじょう): 
\\	繁殖(はんしょく): 
\\	頻繁(ひんぱん): 
\\	縦			
\\	ジュウ	たて	操縦(そうじゅう): 
\\	縦(たて): 
\\	縦 (たて)
\\	線		
\\	セン	すじ	沿線(えんせん): 
\\	幹線(かんせん): 
\\	三味線(さみせん): 
\\	点線(てんせん): 
\\	電線(でんせん): 
\\	無線(むせん): 
\\	下線(かせん): 
\\	曲線(きょくせん): 
\\	光線(こうせん): 
\\	新幹線(しんかんせん): 
\\	水平線(すいへいせん): 
\\	線路(せんろ): 
\\	脱線(だっせん): 
\\	地平線(ちへいせん): 
\\	直線(ちょくせん): 
\\	内線(ないせん): 
\\	線(せん): 
\\	締			
\\	テイ	し.まる、し.まり、し.める、-し.め、-じ.め	締め切り(しめきり): 
\\	戸締り(とじまり): 
\\	取り締まり(とりしまり): 
\\	取り締まる(とりしまる): 
\\	締切(しめきり): 
\\	締め切る(しめきる): 
\\	締める(しめる): 
\\	締まる (し.まる), 締める (し.める)
\\	維			
\\	イ		繊維(せんい): 
\\	維持(いじ): 
\\	羅			
\\	ラ	うすもの		
\\	練			
\\	レン	ね.る、ね.り	練る(ねる): 
\\	未練(みれん): 
\\	訓練(くんれん): 
\\	練習(れんしゅう): 
\\	練る (ね.る)
\\	緒			
\\	ショ、チョ	お、いとぐち	情緒(じょうしょ): 
\\	一緒(いっしょ): 
\\	緒戦 (しょせん), 由緒 (ゆいしょ), 端緒 (たんしょ), 緒 (お)
\\	続			
\\	ゾク、ショク、コウ、キョウ	つづ.く、つづ.ける、つぐ.ない	持続(じぞく): 
\\	接続詞(せつぞくし): 
\\	存続(そんぞく): 
\\	継続(けいぞく): 
\\	接続(せつぞく): 
\\	相続(そうぞく): 
\\	続々(ぞくぞく): 
\\	続き(つづき): 
\\	手続き(てつづき): 
\\	連続(れんぞく): 
\\	続く(つづく): 
\\	続ける(つづける): 
\\	続く (つづ.く), 続ける (つづ.ける)
\\	絵			
\\	カイ、エ		油絵(あぶらえ): 
\\	絵の具(えのぐ): 
\\	絵画(かいが): 
\\	絵(え): 
\\	絵画 (かいが)
\\	統			
\\	トウ	す.べる、ほび.る	統合(とうごう): 
\\	統治(とうじ): 
\\	統制(とうせい): 
\\	統率(とうそつ): 
\\	系統(けいとう): 
\\	大統領(だいとうりょう): 
\\	伝統(でんとう): 
\\	統一(とういつ): 
\\	統計(とうけい): 
\\	統べる (す.べる)
\\	絞			
\\	コウ	しぼ.る、し.める、し.まる	絞る(しぼる): 
\\	絞まる (し.まる), 絞める (し.める), 絞る (しぼ.る)
\\	給			
\\	糸 
\\	合 
\\	キュウ	たま.う、たも.う、-たま.え	給仕(きゅうじ): 
\\	給食(きゅうしょく): 
\\	給う(たまう): 
\\	配給(はいきゅう): 
\\	補給(ほきゅう): 
\\	給与(きゅうよ): 
\\	給料(きゅうりょう): 
\\	供給(きょうきゅう): 
\\	月給(げっきゅう): 
\\	支給(しきゅう): 
\\	絡			
\\	ラク	から.む、から.まる	絡む(からむ): 
\\	連絡(れんらく): 
\\	絡まる (から.まる), 絡む (から.む)
\\	結			
\\	結婚 けっこん 
\\	ケツ、ケチ	むす.ぶ、ゆ.う、ゆ.わえる	結核(けっかく): 
\\	結合(けつごう): 
\\	結晶(けっしょう): 
\\	結成(けっせい): 
\\	結束(けっそく): 
\\	妥結(だけつ): 
\\	団結(だんけつ): 
\\	結び(むすび): 
\\	結び付き(むすびつき): 
\\	結び付く(むすびつく): 
\\	結び付ける(むすびつける): 
\\	結果(けっか): 
\\	結局(けっきょく): 
\\	結論(けつろん): 
\\	結ぶ(むすぶ): 
\\	結構(けっこう): 
\\	結婚(けっこん): 
\\	結ぶ (むす.ぶ), 結う (ゆ.う), 結わえる (ゆ.わえる)
\\	終			
\\	シュウ	お.わる、-お.わる、おわ.る、お.える、つい、つい.に	終わる(おわる): 
\\	終始(しゅうし): 
\\	終日(しゅうじつ): 
\\	終える(おえる): 
\\	終る(おわる): 
\\	最終(さいしゅう): 
\\	始終(しじゅう): 
\\	終点(しゅうてん): 
\\	終了(しゅうりょう): 
\\	終わり(おわり): 
\\	終える (お.える), 終わる (お.わる)
\\	級			
\\	(一級) 
\\	キュウ		階級(かいきゅう): 
\\	等級(とうきゅう): 
\\	同級(どうきゅう): 
\\	学級(がっきゅう): 
\\	級(きゅう): 
\\	高級(こうきゅう): 
\\	上級(じょうきゅう): 
\\	初級(しょきゅう): 
\\	紀			
\\	キ		世紀(せいき): 
\\	紅			
\\	コウ、ク	べに、くれない、あか.い	口紅(くちべに): 
\\	紅茶(こうちゃ): 
\\	紅葉(こうよう): 
\\	紅葉(もみじ): 
\\	紅白 (こうはく), 紅茶 (こうちゃ), 紅葉 (こうよう), 紅 (くれない), 紅 (べに)
\\	納			
\\	ノウ、ナッ、ナ、ナン、トウ	おさ.める、-おさ.める、おさ.まる	納まる(おさまる): 
\\	納める(おさめる): 
\\	滞納(たいのう): 
\\	納入(のうにゅう): 
\\	納得(なっとく): 
\\	納入 (のうにゅう), 納涼 (のうりょう), 収納 (しゅうのう), 納得 (なっとく), 納豆 (なっとう), 納屋 (なや), 納戸 (なんど), 納まる (おさ.まる), 納める (おさ.める)
\\	紡			
\\	ボウ	つむ.ぐ	紡績(ぼうせき): 
\\	紡ぐ (つむ.ぐ)
\\	紛			
\\	フン	まぎ.れる、-まぎ.れ、まぎ.らす、まぎ.らわす、まぎ.らわしい	紛失(ふんしつ): 
\\	紛争(ふんそう): 
\\	紛らわしい(まぎらわしい): 
\\	紛れる(まぎれる): 
\\	紛らす (まぎ.らす), 紛らわしい (まぎ.らわしい), 紛らわす (まぎ.らわす), 紛れる (まぎ.れる)
\\	紹			
\\	ショウ		紹介(しょうかい): 
\\	経			
\\	ケイ、キョウ	へ.る、た.つ、たていと、はか.る、のり	経緯(いきさつ): 
\\	経過(けいか): 
\\	経費(けいひ): 
\\	経歴(けいれき): 
\\	経路(けいろ): 
\\	経る(へる): 
\\	経営(けいえい): 
\\	経度(けいど): 
\\	経由(けいゆ): 
\\	神経(しんけい): 
\\	経つ(たつ): 
\\	経験(けいけん): 
\\	経済(けいざい): 
\\	経費 (けいひ), 経済 (けいざい), 経験 (けいけん), 経る (へ.る)
\\	紳			
\\	シン		紳士(しんし): 
\\	約			
\\	ヤク		規約(きやく): 
\\	倹約(けんやく): 
\\	条約(じょうやく): 
\\	制約(せいやく): 
\\	契約(けいやく): 
\\	婚約(こんやく): 
\\	節約(せつやく): 
\\	約(やく): 
\\	約束(やくそく): 
\\	予約(よやく): 
\\	細			
\\	サイ	ほそ.い、ほそ.る、こま.か、こま.かい	心細い(こころぼそい): 
\\	細やか(こまやか): 
\\	細菌(さいきん): 
\\	細工(さいく): 
\\	細胞(さいぼう): 
\\	詳細(しょうさい): 
\\	細かい(こまかい): 
\\	細い(ほそい): 
\\	細か (こま.か), 細かい (こま.かい), 細い (ほそ.い), 細る (ほそ.る)
\\	累			
\\	ルイ			
\\	索			
\\	サク		捜索(そうさく): 
\\	模索(もさく): 
\\	索引(さくいん): 
\\	総			
\\	ソウ	す.べて、すべ.て、ふさ	総(そう): 
\\	総会(そうかい): 
\\	総合(そうごう): 
\\	総理大臣(そうりだいじん): 
\\	綿			
\\	メン	わた	木綿(きわた): 
\\	綿(めん): 
\\	綿(わた): 
\\	木綿(もめん): 
\\	綿 (わた)
\\	絹			
\\	ケン	きぬ	絹(きぬ): 
\\	絹 (きぬ)
\\	繰			
\\	ソウ	く.る	繰り返す(くりかえす): 
\\	引っ繰り返す(ひっくりかえす): 
\\	引っ繰り返る(ひっくりかえる): 
\\	繰る (く.る)
\\	継			
\\	ケイ	つ.ぐ、まま-	跡継ぎ(あとつぎ): 
\\	受け継ぐ(うけつぐ): 
\\	中継(ちゅうけい): 
\\	継ぎ目(つぎめ): 
\\	継ぐ(つぐ): 
\\	継続(けいぞく): 
\\	継ぐ (つ.ぐ)
\\	緑			
\\	(緑茶=りょくちゃ) 
\\	(茶筌 =ちゃせん) 
\\	リョク、ロク	みどり	緑(みどり): 
\\	緑茶 (りょくちゃ), 緑陰 (みどりいん), 新緑 (しんりょく), 緑 (みどり)
\\	縁			
\\	エン、ネン	ふち、ふちど.る、ゆかり、よすが、へり、えにし	縁(えん): 
\\	縁側(えんがわ): 
\\	縁談(えんだん): 
\\	縁(ふち): 
\\	縁 (ふち)
\\	網			
\\	糸 
\\	モウ	あみ	網(あみ): 
\\	網 (あみ)
\\	緊			
\\	キン		緊急(きんきゅう): 
\\	緊張(きんちょう): 
\\	紫			
\\	シ	むらさき	紫(むらさき): 
\\	紫 (むらさき)
\\	縛			
\\	バク	しば.る	束縛(そくばく): 
\\	縛る(しばる): 
\\	縛る (しば.る)
\\	縄			
\\	ジョウ	なわ、ただ.す	縄(なわ): 
\\	縄 (なわ)
\\	幼			
\\	ヨウ	おさな.い	幼い(おさない): 
\\	幼児(ようじ): 
\\	幼稚(ようち): 
\\	幼稚園(ようちえん): 
\\	幼い (おさな.い)
\\	後			
\\	ゴ、コウ	のち、うし.ろ、うしろ、あと、おく.れる	明後日(あさって): 
\\	後回し(あとまわし): 
\\	後悔(こうかい): 
\\	後退(こうたい): 
\\	産後(さんご): 
\\	明々後日(しあさって): 
\\	背後(はいご): 
\\	以後(いご): 
\\	後(ご): 
\\	後者(こうしゃ): 
\\	後輩(こうはい): 
\\	今後(こんご): 
\\	前後(ぜんご): 
\\	直後(ちょくご): 
\\	後(のち): 
\\	明後日(みょうごにち): 
\\	最後(さいご): 
\\	後(あと): 
\\	後(うしろ): 
\\	午後(ごご): 
\\	後刻 (ごこく), 前後 (ぜんご), 午後 (ごご), 後 (あと), 後ろ (うし.ろ), 後れる (おく.れる), 後 (のち)
\\	幽			
\\	ユウ、ヨウ	ふか.い、かす.か、くら.い、しろ.い	幽霊(ゆうれい): 
\\	幾			
\\	キ	いく-、いく.つ、いく.ら	幾多(いくた): 
\\	幾つ(いくつ): 
\\	幾分(いくぶん): 
\\	幾ら(いくら): 
\\	幾 (いく)
\\	機			
\\	キ	はた	危機(きき): 
\\	機構(きこう): 
\\	契機(けいき): 
\\	動機(どうき): 
\\	機(はた): 
\\	有機(ゆうき): 
\\	機械(きかい): 
\\	機関(きかん): 
\\	機関車(きかんしゃ): 
\\	機嫌(きげん): 
\\	機能(きのう): 
\\	交通機関(こうつうきかん): 
\\	ジェット機(ジェットき): 
\\	扇風機(せんぷうき): 
\\	機会(きかい): 
\\	飛行機(ひこうき): 
\\	機 (はた)
\\	玄			
\\	ゲン		玄人(くろうと): 
\\	玄関(げんかん): 
\\	畜			
\\	チク		家畜(かちく): 
\\	畜産(ちくさん): 
\\	畜生(ちくしょう): 
\\	牧畜(ぼくちく): 
\\	蓄			
\\	チク 蓄電池 ちくでんち
\\	畜1384.	
\\	チク	たくわ.える	蓄積(ちくせき): 
\\	貯蓄(ちょちく): 
\\	蓄える(たくわえる): 
\\	蓄える (たくわ.える)
\\	弦			
\\	ゲン	つる		弦 (つる)
\\	擁			
\\	ヨウ			
\\	滋			
\\	ジ			
\\	慈			
\\	ジ	いつく.しむ		慈しむ (いつく.しむ)
\\	磁			
\\	ジ		磁気(じき): 
\\	磁器(じき): 
\\	磁石(じしゃく): 
\\	系			
\\	ケイ		系(けい): 
\\	系統(けいとう): 
\\	体系(たいけい): 
\\	係			
\\	ケイ	かか.る、かかり、-がかり、かか.わる	係り(かかり): 
\\	係わる(かかわる): 
\\	関係(かんけい): 
\\	係る (かか.る), 係 (かかり)
\\	孫			
\\	ソン	まご	子孫(しそん): 
\\	孫(まご): 
\\	孫 (まご)
\\	懸			
\\	ケン、ケ	か.ける、か.かる	懸賞(けんしょう): 
\\	一生懸命(いっしょうけんめい): 
\\	懸垂 (けんすい), 懸賞 (けんしょう), 懸命 (けんめい), 懸かる (か.かる), 懸ける (か.ける)
\\	却			
\\	キャク	かえ.って、しりぞ.く、しりぞ.ける	却って(かえって): 
\\	脚			
\\	キャク、キャ、カク	あし	脚色(きゃくしょく): 
\\	脚本(きゃくほん): 
\\	脚部 (きゃくぶ), 脚本 (きゃくほん), 三脚 (さんきゃく), 脚 (あし)
\\	卸			
\\	シャ	おろ.す、おろし、おろ.し	卸す(おろす): 
\\	卸す (おろ.す), 卸 (おろし)
\\	御			
\\	ギョ、ゴ	おん-、お-、み-	御負け(おまけ): 
\\	御(ご): 
\\	御馳走(ごちそう): 
\\	御免ください(ごめんください): 
\\	御免なさい(ごめんなさい): 
\\	御覧なさい(ごらんなさい): 
\\	御(お): 
\\	御辞儀(おじぎ): 
\\	御中(おんちゅう): 
\\	御免(ごめん): 
\\	御覧(ごらん): 
\\	朝御飯(あさごはん): 
\\	御主人(ごしゅじん): 
\\	御飯(ごはん): 
\\	晩御飯(ばんごはん): 
\\	昼御飯(ひるごはん): 
\\	御者 (ぎょしゃ), 制御 (せいぎょ), 御 (おん)
\\	服			
\\	フク		軍服(ぐんぷく): 
\\	征服(せいふく): 
\\	制服(せいふく): 
\\	不服(ふふく): 
\\	衣服(いふく): 
\\	克服(こくふく): 
\\	服装(ふくそう): 
\\	和服(わふく): 
\\	服(ふく): 
\\	洋服(ようふく): 
\\	命			
\\	メイ、ミョウ	いのち	運命(うんめい): 
\\	革命(かくめい): 
\\	使命(しめい): 
\\	宿命(しゅくめい): 
\\	任命(にんめい): 
\\	命中(めいちゅう): 
\\	命(いのち): 
\\	寿命(じゅみょう): 
\\	人命(じんめい): 
\\	生命(せいめい): 
\\	命じる(めいじる): 
\\	命ずる(めいずる): 
\\	命令(めいれい): 
\\	一生懸命(いっしょうけんめい): 
\\	命令 (めいれい), 運命 (うんめい), 生命 (せいめい), 命 (いのち)
\\	令			
\\	レイ		指令(しれい): 
\\	仮令(たとえ): 
\\	命令(めいれい): 
\\	零			
\\	レイ	ぜろ、こぼ.す、こぼ.れる	零す(こぼす): 
\\	零れる(こぼれる): 
\\	零点(れいてん): 
\\	零(れい): 
\\	齢			
\\	レイ	よわ.い、とし	年齢(ねんれい): 
\\	冷			
\\	レイ	つめ.たい、ひ.える、ひ.や、ひ.ややか、ひ.やす、ひ.やかす、さ.める、さ.ます	冷やかす(ひやかす): 
\\	冷酷(れいこく): 
\\	冷蔵(れいぞう): 
\\	冷淡(れいたん): 
\\	冷ます(さます): 
\\	冷める(さめる): 
\\	冷やす(ひやす): 
\\	冷静(れいせい): 
\\	冷凍(れいとう): 
\\	冷房(れいぼう): 
\\	冷える(ひえる): 
\\	冷たい(つめたい): 
\\	冷蔵庫(れいぞうこ): 
\\	冷ます (さ.ます), 冷める (さ.める), 冷たい (つめ.たい), 冷える (ひ.える), 冷や (ひ.や), 冷やかす (ひ.やかす), 冷やす (ひ.やす)
\\	領			
\\	リョウ	えり	占領(せんりょう): 
\\	領域(りょういき): 
\\	領海(りょうかい): 
\\	領地(りょうち): 
\\	領土(りょうど): 
\\	大統領(だいとうりょう): 
\\	要領(ようりょう): 
\\	領事(りょうじ): 
\\	領収(りょうしゅう): 
\\	鈴			
\\	レイ、リン	すず	鈴(すず): 
\\	電鈴 (でんれい), 振鈴 (しんすず), 予鈴 (よれい), 鈴 (すず)
\\	勇			
\\	ユウ	いさ.む	勇敢(ゆうかん): 
\\	勇ましい(いさましい): 
\\	勇気(ゆうき): 
\\	勇む (いさ.む)
\\	通			
\\	ツウ、ツ	とお.る、とお.り、-とお.り、-どお.り、とお.す、とお.し、-どお.し、かよ.う	通(つう): 
\\	通常(つうじょう): 
\\	通りかかる(とおりかかる): 
\\	似通う(にかよう): 
\\	見通し(みとおし): 
\\	遣り通す(やりとおす): 
\\	融通(ゆうずう): 
\\	流通(りゅうつう): 
\\	大通り(おおどおり): 
\\	開通(かいつう): 
\\	共通(きょうつう): 
\\	交通機関(こうつうきかん): 
\\	透き通る(すきとおる): 
\\	直通(ちょくつう): 
\\	通過(つうか): 
\\	通貨(つうか): 
\\	通学(つうがく): 
\\	通勤(つうきん): 
\\	通行(つうこう): 
\\	通じる(つうじる): 
\\	通信(つうしん): 
\\	通ずる(つうずる): 
\\	通知(つうち): 
\\	通帳(つうちょう): 
\\	通訳(つうやく): 
\\	通用(つうよう): 
\\	通路(つうろ): 
\\	通す(とおす): 
\\	通り掛かる(とおりかかる): 
\\	通り過ぎる(とおりすぎる): 
\\	一通り(ひととおり): 
\\	人通り(ひとどおり): 
\\	不通(ふつう): 
\\	通う(かよう): 
\\	交通(こうつう): 
\\	通り(とおり): 
\\	通る(とおる): 
\\	普通(ふつう): 
\\	通行 (つうこう), 通読 (つうどく), 普通 (ふつう), 通う (かよ.う), 通す (とお.す), 通る (とお.る)
\\	踊			
\\	ヨウ	おど.る	踊り(おどり): 
\\	踊る(おどる): 
\\	踊り (おど.り), 踊る (おど.る)
\\	疑			
\\	ギ	うたが.う	疑惑(ぎわく): 
\\	質疑(しつぎ): 
\\	疑う(うたがう): 
\\	疑問(ぎもん): 
\\	疑う (うたが.う)
\\	擬			
\\	ギ	まが.い、もど.き		
\\	凝			
\\	ギョウ	こ.る、こ.らす、こご.らす、こご.らせる、こご.る	凝らす(こごらす): 
\\	凝る(こごる): 
\\	凝らす (こ.らす), 凝る (こ.る)
\\	範			
\\	ハン		規範(きはん): 
\\	模範(もはん): 
\\	範囲(はんい): 
\\	犯			
\\	ハン、ボン	おか.す	犯す(おかす): 
\\	犯罪(はんざい): 
\\	犯人(はんにん): 
\\	防犯(ぼうはん): 
\\	犯す (おか.す)
\\	厄			
\\	ヤク		厄介(やっかい): 
\\	危			
\\	キ	あぶ.ない、あや.うい、あや.ぶむ	危ぶむ(あやぶむ): 
\\	危害(きがい): 
\\	危機(きき): 
\\	危うい(あやうい): 
\\	危険(きけん): 
\\	危ない(あぶない): 
\\	危ない (あぶ.ない), 危うい (あや.うい), 危ぶむ (あや.ぶむ)
\\	宛			
\\	エン	あ.てる、-あて、-づつ、あたか.も	宛(あて): 
\\	宛てる(あてる): 
\\	宛名(あてな): 
\\	宛てる (あ.てる)
\\	腕			
\\	ワン	うで	腕前(うでまえ): 
\\	腕(うで): 
\\	腕 (うで)
\\	苑			
\\	エン、オン	その、う.つ		
\\	怨			
\\	エン、オン、ウン	うら.む、うらみ、うら.めしい		
\\	柳			
\\	リュウ	やなぎ		柳 (やなぎ)
\\	卵			
\\	卯
\\	ラン 卵巣 らんそう
\\	ラン	たまご	卵(たまご): 
\\	卵 (たまご)
\\	留			
\\	リュウ、ル	と.める、と.まる、とど.める、とど.まる、るうぶる	加留多(かるた): 
\\	蒸留(じょうりゅう): 
\\	留める(とどめる): 
\\	書留(かきとめ): 
\\	停留所(ていりゅうじょ): 
\\	留まる(とどまる): 
\\	留まる(とまる): 
\\	留学(りゅうがく): 
\\	留守番(るすばん): 
\\	留守(るす): 
\\	留学生(りゅうがくせい): 
\\	留意 (りゅうい), 留学 (りゅうがく), 保留 (ほりゅう), 留まる (と.まる), 留める (と.める)
\\	貿			
\\	ボウ		貿易(ぼうえき): 
\\	印			
\\	イン	しるし、-じるし、しる.す	印(いん): 
\\	印鑑(いんかん): 
\\	調印(ちょういん): 
\\	印刷(いんさつ): 
\\	印象(いんしょう): 
\\	印(しるし): 
\\	目印(めじるし): 
\\	矢印(やじるし): 
\\	印 (しるし)
\\	興			
\\	コウ、キョウ	おこ.る、おこ.す	興じる(きょうじる): 
\\	興業(こうぎょう): 
\\	興奮(こうふん): 
\\	新興(しんこう): 
\\	振興(しんこう): 
\\	復興(ふっこう): 
\\	余興(よきょう): 
\\	興味(きょうみ): 
\\	興行 (こうぎょう), 復興 (ふっこう), 振興 (しんこう), 興す (おこ.す), 興る (おこ.る)
\\	酉			
\\	ユウ	とり		
\\	酒			
\\	シュ	さけ、さか-	酒場(さかば): 
\\	酒(さけ): 
\\	お酒(おさけ): 
\\	酒 (さけ)
\\	酌			
\\	シャク	く.む	酌む(くむ): 
\\	酌む (く.む)
\\	酵			
\\	コウ			
\\	酷			
\\	コク	ひど.い	残酷(ざんこく): 
\\	酷い(ひどい): 
\\	冷酷(れいこく): 
\\	酬			
\\	シュウ、シュ、トウ	むく.いる	報酬(ほうしゅう): 
\\	酪			
\\	ラク		酪農(らくのう): 
\\	酢			
\\	サク	す	酢(す): 
\\	酢 (す)
\\	酔			
\\	スイ	よ.う、よ.い、よ	麻酔(ますい): 
\\	酔う(よう): 
\\	酔っ払い(よっぱらい): 
\\	酔う (よ.う)
\\	配			
\\	ハイ	くば.る	気配(きはい): 
\\	心配(しんぱい): 
\\	手配(てはい): 
\\	配給(はいきゅう): 
\\	配偶者(はいぐうしゃ): 
\\	配置(はいち): 
\\	配布(はいふ): 
\\	配分(はいぶん): 
\\	配慮(はいりょ): 
\\	配列(はいれつ): 
\\	分配(ぶんぱい): 
\\	配る(くばる): 
\\	気配(けはい): 
\\	支配(しはい): 
\\	配達(はいたつ): 
\\	配る (くば.る)
\\	酸			
\\	サン	す.い	酸(さん): 
\\	酸化(さんか): 
\\	酸素(さんそ): 
\\	酸っぱい(すっぱい): 
\\	酸性(さんせい): 
\\	酸い (す.い)
\\	猶			
\\	ユウ、ユ	なお		
\\	尊			
\\	ソン	たっと.い、とうと.い、たっと.ぶ、とうと.ぶ	自尊心(じそんしん): 
\\	尊い(たっとい): 
\\	尊ぶ(たっとぶ): 
\\	尊敬(そんけい): 
\\	尊重(そんちょう): 
\\	尊い (たっと.い), 尊ぶ (たっと.ぶ), 尊い (とうと.い), 尊ぶ (とうと.ぶ)
\\	豆			
\\	トウ、ズ	まめ、まめ-	豆(まめ): 
\\	豆腐 (とうふ), 納豆 (なっとう), 豆 (まめ)
\\	頭			
\\	トウ、ズ、ト	あたま、かしら、-がしら、かぶり	街頭(がいとう): 
\\	口頭(こうとう): 
\\	冒頭(ぼうとう): 
\\	頭痛(ずつう): 
\\	頭脳(ずのう): 
\\	先頭(せんとう): 
\\	頭(あたま): 
\\	頭部 (とうぶ), 年頭 (ねんとう), 船頭 (せんどう), 頭脳 (ずのう), 頭上 (ずじょう), 頭痛 (ずつう), 頭 (あたま), 頭 (かしら)
\\	短			
\\	タン	みじか.い	短歌(たんか): 
\\	短気(たんき): 
\\	短縮(たんしゅく): 
\\	短大(たんだい): 
\\	短波(たんぱ): 
\\	短期(たんき): 
\\	短所(たんしょ): 
\\	短編(たんぺん): 
\\	長短(ちょうたん): 
\\	短い(みじかい): 
\\	短い (みじか.い)
\\	豊			
\\	いただきま〜す!.	
\\	ホウ、ブ	ゆた.か、とよ	豊作(ほうさく): 
\\	豊富(ほうふ): 
\\	豊か(ゆたか): 
\\	豊 (ゆたか)
\\	鼓			
\\	コ	つづみ	太鼓(たいこ): 
\\	鼓 (つづみ)
\\	喜			
\\	キ	よろこ.ぶ、よろこ.ばす	喜劇(きげき): 
\\	喜び(よろこび): 
\\	喜ぶ(よろこぶ): 
\\	喜ぶ (よろこ.ぶ)
\\	樹			
\\	ジュ	き	樹木(じゅもく): 
\\	樹立(じゅりつ): 
\\	皿			
\\	ベイ	さら	灰皿(はいさら): 
\\	皿(さら): 
\\	お皿(おさら): 
\\	灰皿(はいざら): 
\\	皿 (さら)
\\	血			
\\	ケツ	ち	血管(けっかん): 
\\	混血(こんけつ): 
\\	出血(しゅっけつ): 
\\	血圧(けつあつ): 
\\	血液(けつえき): 
\\	輸血(ゆけつ): 
\\	血(ち): 
\\	血 (ち)
\\	盆			
\\	ボン		盆(ぼん): 
\\	盆地(ぼんち): 
\\	盟			
\\	メイ		同盟(どうめい): 
\\	連盟(れんめい): 
\\	盗			
\\	トウ	ぬす.む、ぬす.み	盗み(ぬすみ): 
\\	強盗(ごうとう): 
\\	盗難(とうなん): 
\\	盗む(ぬすむ): 
\\	盗む (ぬす.む)
\\	温			
\\	オン	あたた.か、あたた.かい、あたた.まる、あたた.める、ぬく	温和(おんわ): 
\\	生温い(なまぬるい): 
\\	保温(ほおん): 
\\	温室(おんしつ): 
\\	温泉(おんせん): 
\\	温帯(おんたい): 
\\	温暖(おんだん): 
\\	温度(おんど): 
\\	気温(きおん): 
\\	体温(たいおん): 
\\	温い(ぬるい): 
\\	温か (あたた.か), 温かい (あたた.かい), 温まる (あたた.まる), 温める (あたた.める)
\\	監			
\\	カン		監視(かんし): 
\\	監督(かんとく): 
\\	濫			
\\	ラン	みだ.りに、みだ.りがましい	氾濫(はんらん): 
\\	濫用(らんよう): 
\\	鑑			
\\	カン	かんが.みる、かがみ	印鑑(いんかん): 
\\	年鑑(ねんかん): 
\\	鑑賞(かんしょう): 
\\	図鑑(ずかん): 
\\	猛			
\\	モウ		猛烈(もうれつ): 
\\	盛			
\\	セイ、ジョウ	も.る、さか.る、さか.ん	盛る(さかる): 
\\	盛装(せいそう): 
\\	盛大(せいだい): 
\\	全盛(ぜんせい): 
\\	繁盛(はんじょう): 
\\	目盛(めもり): 
\\	盛り上がる(もりあがる): 
\\	盛り(さかり): 
\\	盛る(もる): 
\\	盛ん(さかん): 
\\	盛大 (せいだい), 隆盛 (りゅうせい), 全盛 (ぜんせい), 盛る (さか.る), 盛ん (さか.ん), 盛る (も.る)
\\	塩			
\\	土, 
\\	口, 
\\	皿 
\\	エン	しお	塩(えん): 
\\	塩辛い(しおからい): 
\\	食塩(しょくえん): 
\\	塩(しお): 
\\	塩 (しお)
\\	銀			
\\	ギン	しろがね	銀(ぎん): 
\\	銀行(ぎんこう): 
\\	恨			
\\	コン	うら.む、うら.めしい	恨み(うらみ): 
\\	恨む(うらむ): 
\\	恨む (うら.む), 恨めしい (うら.めしい)
\\	根			
\\	コン	ね、-ね	球根(きゅうこん): 
\\	根気(こんき): 
\\	根拠(こんきょ): 
\\	根底(こんてい): 
\\	根本(こんぽん): 
\\	根回し(ねまわし): 
\\	利根(りこん): 
\\	垣根(かきね): 
\\	根(ね): 
\\	羽根(はね): 
\\	屋根(やね): 
\\	根 (ね)
\\	即			
\\	ソク	つ.く、つ.ける、すなわ.ち	即ち(すなわち): 
\\	即座に(そくざに): 
\\	即する(そくする): 
\\	爵			
\\	シャク			
\\	節			
\\	セツ、セチ	ふし、-ぶし、のっと	節(せつ): 
\\	節約(せつやく): 
\\	調節(ちょうせつ): 
\\	節(ふし): 
\\	季節(きせつ): 
\\	節約 (せつやく), 季節 (きせつ), 関節 (かんせつ), 節 (ふし)
\\	退			
\\	タイ	しりぞ.く、しりぞ.ける、ひ.く、の.く、の.ける、ど.く	後退(こうたい): 
\\	辞退(じたい): 
\\	退く(しりぞく): 
\\	退ける(しりぞける): 
\\	退化(たいか): 
\\	退学(たいがく): 
\\	退治(たいじ): 
\\	退職(たいしょく): 
\\	脱退(だったい): 
\\	引退(いんたい): 
\\	退屈(たいくつ): 
\\	退く(どく): 
\\	退ける(どける): 
\\	退院(たいいん): 
\\	退く (しりぞ.く), 退ける (しりぞ.ける)
\\	限			
\\	ゲン	かぎ.る、かぎ.り、-かぎ.り	局限(きょくげん): 
\\	権限(けんげん): 
\\	限定(げんてい): 
\\	限る(かぎる): 
\\	期限(きげん): 
\\	限界(げんかい): 
\\	限度(げんど): 
\\	制限(せいげん): 
\\	無限(むげん): 
\\	限る (かぎ.る)
\\	眼			
\\	ガン、ゲン	まなこ、め	眼科(がんか): 
\\	眼球(がんきゅう): 
\\	眼鏡(がんきょう): 
\\	近眼(きんがん): 
\\	眼鏡(めがね): 
\\	眼球 (がんきゅう), 眼力 (がんりき), 主眼 (しゅがん), 眼 (まなこ)
\\	良			
\\	リョウ	よ.い、-よ.い、い.い、-い.い	良い(いい): 
\\	改良(かいりょう): 
\\	善良(ぜんりょう): 
\\	不良(ふりょう): 
\\	良好(りょうこう): 
\\	良識(りょうしき): 
\\	良質(りょうしつ): 
\\	良心(りょうしん): 
\\	仲良し(なかよし): 
\\	良い (よ.い)
\\	朗			
\\	ロウ	ほが.らか、あき.らか	明朗(めいろう): 
\\	朗読(ろうどく): 
\\	朗らか(ほがらか): 
\\	朗らか (ほが.らか)
\\	浪			
\\	ロウ		浪費(ろうひ): 
\\	娘			
\\	ジョウ	むすめ、こ	娘(むすめ): 
\\	娘 (むすめ)
\\	食			
\\	ショク、ジキ	く.う、く.らう、た.べる、は.む	給食(きゅうしょく): 
\\	食い違う(くいちがう): 
\\	主食(しゅしょく): 
\\	食事(しょくじ): 
\\	定食(ていしょく): 
\\	衣食住(いしょくじゅう): 
\\	食う(くう): 
\\	食塩(しょくえん): 
\\	食卓(しょくたく): 
\\	食品(しょくひん): 
\\	食物(しょくもつ): 
\\	食欲(しょくよく): 
\\	食料(しょくりょう): 
\\	食糧(しょくりょう): 
\\	食器(しょっき): 
\\	昼食(ちゅうしょく): 
\\	食料品(しょくりょうひん): 
\\	食堂(しょくどう): 
\\	食べ物(たべもの): 
\\	食べる(たべる): 
\\	食事 (しょくじ), 食料 (しょくりょう), 会食 (かいしょく), 食う (く.う), 食らう (く.らう), 食べる (た.べる)
\\	飯			
\\	ハン	めし	昼飯(ちゅうはん): 
\\	飯(めし): 
\\	夕飯(ゆうはん): 
\\	朝御飯(あさごはん): 
\\	御飯(ごはん): 
\\	晩御飯(ばんごはん): 
\\	昼御飯(ひるごはん): 
\\	飯 (めし)
\\	飲			
\\	イン、オン	の.む、-の.み	飲み込む(のみこむ): 
\\	湯飲み(ゆのみ): 
\\	飲み物(のみもの): 
\\	飲む(のむ): 
\\	飲む (の.む)
\\	飢			
\\	キ	う.える	飢える(うえる): 
\\	飢饉(ききん): 
\\	飢える (う.える)
\\	餓			
\\	ガ	う.える		
\\	飾			
\\	ショク	かざ.る、かざ.り	着飾る(きかざる): 
\\	首飾り(くびかざり): 
\\	修飾(しゅうしょく): 
\\	装飾(そうしょく): 
\\	飾り(かざり): 
\\	飾る(かざる): 
\\	飾る (かざ.る)
\\	館			
\\	カン	やかた、たて	館(かん): 
\\	館(たち): 
\\	本館(ほんかん): 
\\	会館(かいかん): 
\\	博物館(はくぶつかん): 
\\	美術館(びじゅつかん): 
\\	旅館(りょかん): 
\\	映画館(えいがかん): 
\\	大使館(たいしかん): 
\\	図書館(としょかん): 
\\	養			
\\	ヨウ、リョウ	やしな.う	扶養(ふよう): 
\\	保養(ほよう): 
\\	養う(やしなう): 
\\	養護(ようご): 
\\	養成(ようせい): 
\\	栄養(えいよう): 
\\	休養(きゅうよう): 
\\	教養(きょうよう): 
\\	養分(ようぶん): 
\\	養う (やしな.う)
\\	飽			
\\	ホウ	あ.きる、あ.かす、あ.く	飽和(ほうわ): 
\\	飽きる(あきる): 
\\	飽くまで(あくまで): 
\\	飽かす (あ.かす), 飽きる (あ.きる)
\\	既			
\\	キ	すで.に	既婚(きこん): 
\\	既に(すでに): 
\\	既に (すで.に)
\\	概			
\\	ガイ	おうむ.ね	一概に(いちがいに): 
\\	概説(がいせつ): 
\\	概念(がいねん): 
\\	概略(がいりゃく): 
\\	大概(たいがい): 
\\	概論(がいろん): 
\\	慨			
\\	ガイ		感慨(かんがい): 
\\	憤慨(ふんがい): 
\\	平			
\\	金閣寺(きんかくじ) 
\\	干 
\\	ヘイ  平均 へいきん
\\	ビョウ 平等 びょうどう
\\	ヘイ、ビョウ、ヒョウ	たい.ら、-だいら、ひら、ひら-	平均(ならし): 
\\	平たい(ひらたい): 
\\	平行(へいこう): 
\\	平常(へいじょう): 
\\	平方(へいほう): 
\\	公平(こうへい): 
\\	水平(すいへい): 
\\	水平線(すいへいせん): 
\\	平ら(たいら): 
\\	地平線(ちへいせん): 
\\	平等(びょうどう): 
\\	不平(ふへい): 
\\	平気(へいき): 
\\	平均(へいきん): 
\\	平日(へいじつ): 
\\	平凡(へいぼん): 
\\	平野(へいや): 
\\	平和(へいわ): 
\\	平仮名(ひらがな): 
\\	平面 (へいめん), 平和 (へいわ), 公平 (こうへい), 平ら (たい.ら), 平 (ひら)
\\	呼			
\\	コ	よ.ぶ	嗚呼(ああ): 
\\	呼び止める(よびとめる): 
\\	呼吸(こきゅう): 
\\	呼び掛ける(よびかける): 
\\	呼び出す(よびだす): 
\\	呼ぶ(よぶ): 
\\	呼ぶ (よ.ぶ)
\\	坪			
\\	ヘイ	つぼ		坪 (つぼ)
\\	評			
\\	ヒョウ		好評(こうひょう): 
\\	書評(しょひょう): 
\\	不評(ふひょう): 
\\	批評(ひひょう): 
\\	評価(ひょうか): 
\\	評判(ひょうばん): 
\\	評論(ひょうろん): 
\\	刈			
\\	メリ
\\	ガイ、カイ	か.る	刈る(かる): 
\\	刈る (か.る)
\\	希			
\\	キ、ケ	まれ	希望(きぼう): 
\\	凶			
\\	キョウ		凶作(きょうさく): 
\\	胸			
\\	キョウ	むね、むな-	胸(むね): 
\\	胸 (むね)
\\	離			
\\	リ	はな.れる、はな.す	距離(きょり): 
\\	分離(ぶんり): 
\\	離す(はなす): 
\\	離れる(はなれる): 
\\	離婚(りこん): 
\\	離す (はな.す), 離れる (はな.れる)
\\	殺			
\\	サツ、サイ、セツ	ころ.す、-ごろ.し、そ.ぐ	暗殺(あんさつ): 
\\	殺人(さつじん): 
\\	殺す(ころす): 
\\	自殺(じさつ): 
\\	殺人 (さつじん), 殺到 (さっとう), 黙殺 (もくさつ), 相殺 (そうさい), 殺す (ころ.す)
\\	純			
\\	ジュン		清純(せいじゅん): 
\\	純情(じゅんじょう): 
\\	純粋(じゅんすい): 
\\	単純(たんじゅん): 
\\	鈍			
\\	ドン	にぶ.い、にぶ.る、にぶ-、なま.る、なまく.ら	鈍感(どんかん): 
\\	鈍る(なまる): 
\\	鈍い(にぶい): 
\\	鈍い(のろい): 
\\	鈍い (にぶ.い), 鈍る (にぶ.る)
\\	辛			
\\	シン	から.い、つら.い、-づら.い、かのと	辛うじて(かろうじて): 
\\	香辛料(こうしんりょう): 
\\	辛抱(しんぼう): 
\\	塩辛い(しおからい): 
\\	辛い(つらい): 
\\	辛い(からい): 
\\	辛い (から.い)
\\	辞			
\\	ジ	や.める、いな.む	お世辞(おせじ): 
\\	辞職(じしょく): 
\\	辞退(じたい): 
\\	世辞(せじ): 
\\	百科辞典(ひゃっかじてん): 
\\	御辞儀(おじぎ): 
\\	辞める(やめる): 
\\	辞典(じてん): 
\\	辞書(じしょ): 
\\	辞める (や.める)
\\	梓			
\\	シ	あずさ		
\\	宰			
\\	サイ			
\\	壁			
\\	ヘキ	かべ	壁(かべ): 
\\	壁 (かべ)
\\	避			
\\	ヒ	さ.ける、よ.ける	避難(ひなん): 
\\	避ける(さける): 
\\	避ける (さ.ける)
\\	新			
\\	シン	あたら.しい、あら.た、あら-、にい-	革新(かくしん): 
\\	新(しん): 
\\	新興(しんこう): 
\\	新婚(しんこん): 
\\	新人(しんじん): 
\\	新築(しんちく): 
\\	新入生(しんにゅうせい): 
\\	新た(あらた): 
\\	新幹線(しんかんせん): 
\\	新鮮(しんせん): 
\\	新聞社(しんぶんしゃ): 
\\	新しい(あたらしい): 
\\	新聞(しんぶん): 
\\	新しい (あたら.しい), 新た (あら.た), 新 (にい)
\\	薪			
\\	シン	たきぎ、まき		薪 (たきぎ)
\\	親			
\\	シン	おや、おや-、した.しい、した.しむ	親しむ(したしむ): 
\\	親善(しんぜん): 
\\	肉親(にくしん): 
\\	親(おや): 
\\	親指(おやゆび): 
\\	親しい(したしい): 
\\	親友(しんゆう): 
\\	親戚(しんせき): 
\\	親類(しんるい): 
\\	父親(ちちおや): 
\\	母親(ははおや): 
\\	親切(しんせつ): 
\\	両親(りょうしん): 
\\	親 (おや), 親しい (した.しい), 親しむ (した.しむ)
\\	幸			
\\	「幸せから一本取ると辛い」 
\\	百).	
\\	コウ	さいわ.い、さち、しあわ.せ	幸運(こううん): 
\\	幸福(こうふく): 
\\	幸い(さいわい): 
\\	幸せ(しあわせ): 
\\	不幸(ふこう): 
\\	幸い (さいわ.い), 幸 (さち), 幸せ (しあわ.せ)
\\	執			
\\	シツ、シュウ	と.る	執着(しゅうじゃく): 
\\	執筆(しっぴつ): 
\\	執務 (しつむ), 執筆 (しっぴつ), 確執 (かくしつ), 執る (と.る)
\\	報			
\\	ホウ	むく.いる	報酬(ほうしゅう): 
\\	報じる(ほうじる): 
\\	報ずる(ほうずる): 
\\	報道(ほうどう): 
\\	情報(じょうほう): 
\\	報告(ほうこく): 
\\	予報(よほう): 
\\	天気予報(てんきよほう): 
\\	電報(でんぽう): 
\\	報いる (むく.いる)
\\	叫			
\\	キョウ	さけ.ぶ	叫び(さけび): 
\\	叫ぶ(さけぶ): 
\\	叫ぶ (さけ.ぶ)
\\	糾			
\\	キュウ	ただ.す		
\\	収			
\\	シュウ	おさ.める、おさ.まる	収まる(おさまる): 
\\	収める(おさめる): 
\\	回収(かいしゅう): 
\\	収益(しゅうえき): 
\\	収支(しゅうし): 
\\	収集(しゅうしゅう): 
\\	収容(しゅうよう): 
\\	徴収(ちょうしゅう): 
\\	没収(ぼっしゅう): 
\\	吸収(きゅうしゅう): 
\\	収穫(しゅうかく): 
\\	収入(しゅうにゅう): 
\\	領収(りょうしゅう): 
\\	収まる (おさ.まる), 収める (おさ.める)
\\	卑			
\\	ヒ	いや.しい、いや.しむ、いや.しめる	卑しい(いやしい): 
\\	卑怯(ひきょう): 
\\	卑しい (いや.しい), 卑しむ (いや.しむ), 卑しめる (いや.しめる)
\\	碑			
\\	ヒ	いしぶみ	碑(いしぶみ): 
\\	陸			
\\	リク、ロク	おか	上陸(じょうりく): 
\\	着陸(ちゃくりく): 
\\	内陸(ないりく): 
\\	大陸(たいりく): 
\\	陸(りく): 
\\	睦			
\\	ボク、モク	むつ.まじい、むつ.む、むつ.ぶ		
\\	勢			
\\	セイ、ゼイ	いきお.い、はずみ	形勢(けいせい): 
\\	情勢(じょうせい): 
\\	勢力(せいりょく): 
\\	態勢(たいせい): 
\\	優勢(ゆうせい): 
\\	勢い(いきおい): 
\\	姿勢(しせい): 
\\	大勢(おおぜい): 
\\	勢い (いきお.い)
\\	熱			
\\	暑). 
\\	ネツ	あつ.い	光熱費(こうねつひ): 
\\	情熱(じょうねつ): 
\\	熱意(ねつい): 
\\	熱湯(ねっとう): 
\\	熱量(ねつりょう): 
\\	加熱(かねつ): 
\\	熱する(ねっする): 
\\	熱帯(ねったい): 
\\	熱中(ねっちゅう): 
\\	熱(ねつ): 
\\	熱心(ねっしん): 
\\	熱い(あつい): 
\\	熱い (あつ.い)
\\	菱			
\\	リョウ	ひし		
\\	陵			
\\	リョウ	みささぎ	丘陵(きゅうりょう): 
\\	陵 (みささぎ)
\\	亥			
\\	ガイ、カイ	い		
\\	核			
\\	カク		核(かく): 
\\	結核(けっかく): 
\\	刻			
\\	コク	きざ.む、きざ.み	時刻表(じこくひょう): 
\\	刻む(きざむ): 
\\	時刻(じこく): 
\\	深刻(しんこく): 
\\	遅刻(ちこく): 
\\	彫刻(ちょうこく): 
\\	刻む (きざ.む)
\\	該			
\\	ガイ		該当(がいとう): 
\\	劾			
\\	ガイ			
\\	述			
\\	ジュツ	の.べる	記述(きじゅつ): 
\\	口述(こうじゅつ): 
\\	述語(じゅつご): 
\\	述べる(のべる): 
\\	述べる (の.べる)
\\	術			
\\	ジュツ	すべ	戦術(せんじゅつ): 
\\	美術(びじゅつ): 
\\	学術(がくじゅつ): 
\\	芸術(げいじゅつ): 
\\	手術(しゅじゅつ): 
\\	技術(ぎじゅつ): 
\\	美術館(びじゅつかん): 
\\	寒			
\\	寒い 
\\	さむい 
\\	カン	さむ.い	寒気(かんき): 
\\	寒帯(かんたい): 
\\	寒い(さむい): 
\\	寒い (さむ.い)
\\	醸			
\\	ジョウ	かも.す		醸す (かも.す)
\\	譲			
\\	ジョウ	ゆず.る	譲歩(じょうほ): 
\\	譲る(ゆずる): 
\\	譲る (ゆず.る)
\\	壌			
\\	ジョウ	つち		
\\	嬢			
\\	ジョウ	むすめ	嬢(じょう): 
\\	お嬢さん(おじょうさん): 
\\	毒			
\\	ドク		中毒(ちゅうどく): 
\\	気の毒(きのどく): 
\\	消毒(しょうどく): 
\\	毒(どく): 
\\	素			
\\	ソ、ス	もと	簡素(かんそ): 
\\	元素(げんそ): 
\\	酸素(さんそ): 
\\	質素(しっそ): 
\\	素敵(すてき): 
\\	素早い(すばやい): 
\\	素晴らしい(すばらしい): 
\\	素材(そざい): 
\\	素っ気ない(そっけない): 
\\	素朴(そぼく): 
\\	炭素(たんそ): 
\\	素人(しろうと): 
\\	水素(すいそ): 
\\	素直(すなお): 
\\	素質(そしつ): 
\\	要素(ようそ): 
\\	素材 (そざい), 元素 (げんそ), 平素 (へいそ)
\\	麦			
\\	バク	むぎ	小麦(こむぎ): 
\\	蕎麦(そば): 
\\	麦 (むぎ)
\\	青			
\\	セイ、ショウ	あお、あお-、あお.い	青白い(あおじろい): 
\\	青春(せいしゅん): 
\\	青(あお): 
\\	青少年(せいしょうねん): 
\\	青年(せいねん): 
\\	真っ青(まっさお): 
\\	青い(あおい): 
\\	青天 (せいてん), 青銅 (せいどう), 青年 (せいねん), 青 (あお), 青い (あお.い)
\\	精			
\\	セイ、ショウ、シヤウ		精巧(せいこう): 
\\	精密(せいみつ): 
\\	精神(せいしん): 
\\	精々(せいぜい): 
\\	精米 (せいまい), 精密 (せいみつ), 精力 (せいりょく)
\\	請			
\\	セイ、シン、ショウ	こ.う、う.ける	強請る(ねだる): 
\\	要請(ようせい): 
\\	申請(しんせい): 
\\	請求(せいきゅう): 
\\	請求 (せいきゅう), 請願 (せいがん), 申請 (しんせい), 請ける (う.ける), 請う (こ.う)
\\	情			
\\	ジョウ、セイ	なさ.け	実情(じつじょう): 
\\	情(じょう): 
\\	情緒(じょうしょ): 
\\	情勢(じょうせい): 
\\	情熱(じょうねつ): 
\\	心情(しんじょう): 
\\	同情(どうじょう): 
\\	情け(なさけ): 
\\	情け深い(なさけぶかい): 
\\	人情(にんじょう): 
\\	愛情(あいじょう): 
\\	感情(かんじょう): 
\\	苦情(くじょう): 
\\	事情(じじょう): 
\\	純情(じゅんじょう): 
\\	情報(じょうほう): 
\\	表情(ひょうじょう): 
\\	友情(ゆうじょう): 
\\	情報 (じょうほう), 情熱 (じょうねつ), 人情 (にんじょう), 情け (なさ.け)
\\	晴			
\\	セイ	は.れる、は.れ、は.れ-、-ば.れ、は.らす	素晴らしい(すばらしい): 
\\	晴天(せいてん): 
\\	見晴らし(みはらし): 
\\	快晴(かいせい): 
\\	晴れ(はれ): 
\\	晴れる(はれる): 
\\	晴らす (は.らす), 晴れる (は.れる)
\\	清			
\\	セイ、ショウ、シン	きよ.い、きよ.まる、きよ.める	清らか(きよらか): 
\\	清々しい(すがすがしい): 
\\	清算(せいさん): 
\\	清純(せいじゅん): 
\\	清濁(せいだく): 
\\	清い(きよい): 
\\	清む(すむ): 
\\	清潔(せいけつ): 
\\	清書(せいしょ): 
\\	清掃(せいそう): 
\\	清潔 (せいけつ), 清算 (せいさん), 粛清 (しゅくせい), 清い (きよ.い), 清まる (きよ.まる), 清める (きよ.める)
\\	静			
\\	セイ、ジョウ	しず-、しず.か、しず.まる、しず.める	安静(あんせい): 
\\	静止(せいし): 
\\	静的(せいてき): 
\\	静まる(しずまる): 
\\	冷静(れいせい): 
\\	静か(しずか): 
\\	静止 (せいし), 静穏 (せいおん), 安静 (あんせい), 静 (しず), 静か (しず.か), 静まる (しず.まる), 静める (しず.める)
\\	責			
\\	セキ	せ.める	責務(せきむ): 
\\	責任(せきにん): 
\\	責める(せめる): 
\\	責める (せ.める)
\\	績			
\\	セキ		業績(ぎょうせき): 
\\	紡績(ぼうせき): 
\\	功績(こうせき): 
\\	実績(じっせき): 
\\	成績(せいせき): 
\\	積			
\\	セキ	つ.む、-づ.み、つ.もる、つ.もり	蓄積(ちくせき): 
\\	積もり(つもり): 
\\	見積り(みつもり): 
\\	積極的(せっきょくてき): 
\\	体積(たいせき): 
\\	積む(つむ): 
\\	積もる(つもる): 
\\	面積(めんせき): 
\\	容積(ようせき): 
\\	積む (つ.む), 積もる (つ.もる)
\\	債			
\\	サイ		負債(ふさい): 
\\	漬			
\\	シ	つ.ける、つ.かる、-づ.け、-づけ	漬ける(つける): 
\\	漬かる (つ.かる), 漬ける (つ.ける)
\\	表			
\\	ヒョウ	おもて、-おもて、あらわ.す、あらわ.れる、あら.わす	時刻表(じこくひょう): 
\\	表す(あらわす): 
\\	公表(こうひょう): 
\\	図表(ずひょう): 
\\	代表(だいひょう): 
\\	発表(はっぴょう): 
\\	表(ひょう): 
\\	表現(ひょうげん): 
\\	表紙(ひょうし): 
\\	表情(ひょうじょう): 
\\	表面(ひょうめん): 
\\	表(おもて): 
\\	表す (あらわ.す), 表れる (あらわ.れる), 表 (おもて)
\\	俵			
\\	ヒョウ	たわら	土俵(どひょう): 
\\	俵 (たわら)
\\	潔			
\\	ケツ	いさぎよ.い	簡潔(かんけつ): 
\\	清潔(せいけつ): 
\\	不潔(ふけつ): 
\\	潔い (いさぎよ.い)
\\	契			
\\	ケイ	ちぎ.る	契機(けいき): 
\\	契る(ちぎる): 
\\	契約(けいやく): 
\\	契る (ちぎ.る)
\\	喫			
\\	キツ		喫茶(きっさ): 
\\	喫茶店(きっさてん): 
\\	害			
\\	ガイ		害する(がいする): 
\\	危害(きがい): 
\\	災害(さいがい): 
\\	迫害(はくがい): 
\\	妨害(ぼうがい): 
\\	害(がい): 
\\	公害(こうがい): 
\\	障害(しょうがい): 
\\	損害(そんがい): 
\\	被害(ひがい): 
\\	利害(りがい): 
\\	轄			
\\	カツ	くさび		
\\	割			
\\	カツ	わ.る、わり、わ.り、わ.れる、さ.く	割(かつ): 
\\	割合(わりあい): 
\\	割り当て(わりあて): 
\\	割り込む(わりこむ): 
\\	割り算(わりざん): 
\\	時間割(じかんわり): 
\\	役割(やくわり): 
\\	割算(わりざん): 
\\	割と(わりと): 
\\	割引(わりびき): 
\\	割る(わる): 
\\	割合に(わりあいに): 
\\	割れる(われる): 
\\	割く (さ.く), 割る (わ.る), 割れる (わ.れる), 割 (わり)
\\	憲			
\\	ケン		憲法(けんぽう): 
\\	生			
\\	セイ、ショウ	い.きる、い.かす、い.ける、う.まれる、う.まれ、うまれ、う.む、お.う、は.える、は.やす、き、なま、なま-、な.る、な.す、む.す、-う	生かす(いかす): 
\\	生まれつき(うまれつき): 
\\	衛生(えいせい): 
\\	生(き): 
\\	生真面目(きまじめ): 
\\	再生(さいせい): 
\\	出生(しゅっしょう): 
\\	新入生(しんにゅうせい): 
\\	生育(せいいく): 
\\	生活(せいかつ): 
\\	生計(せいけい): 
\\	生死(せいし): 
\\	生長(せいちょう): 
\\	生理(せいり): 
\\	畜生(ちくしょう): 
\\	生臭い(なまぐさい): 
\\	生温い(なまぬるい): 
\\	生身(なまみ): 
\\	年生(ねんせい): 
\\	発生(はっせい): 
\\	生やす(はやす): 
\\	野生(やせい): 
\\	生き生き(いきいき): 
\\	生き物(いきもの): 
\\	生け花(いけばな): 
\\	一生(いっしょう): 
\\	生まれ(うまれ): 
\\	生地(きじ): 
\\	芝生(しばふ): 
\\	写生(しゃせい): 
\\	小学生(しょうがくせい): 
\\	生じる(しょうじる): 
\\	生ずる(しょうずる): 
\\	人生(じんせい): 
\\	生産(せいさん): 
\\	生存(せいぞん): 
\\	生年月日(せいねんがっぴ): 
\\	生物(せいぶつ): 
\\	生命(せいめい): 
\\	誕生(たんじょう): 
\\	生(なま): 
\\	生意気(なまいき): 
\\	生る(なる): 
\\	生える(はえる): 
\\	生きる(いきる): 
\\	一生懸命(いっしょうけんめい): 
\\	高校生(こうこうせい): 
\\	大学生(だいがくせい): 
\\	生まれる(うまれる): 
\\	学生(がくせい): 
\\	生徒(せいと): 
\\	先生(せんせい): 
\\	誕生日(たんじょうび): 
\\	留学生(りゅうがくせい): 
\\	生活 (せいかつ), 発生 (はっせい), 先生 (せんせい), 生かす (い.かす), 生きる (い.きる), 生ける (い.ける), 生まれる (う.まれる), 生む (う.む), 生う (お.う), 生 (き), 生 (なま), 生える (は.える), 生やす (は.やす)
\\	星			
\\	セイ、ショウ	ほし、-ぼし	火星(かせい): 
\\	星座(せいざ): 
\\	惑星(わくせい): 
\\	衛星(えいせい): 
\\	星(ほし): 
\\	星座 (せいざ), 流星 (りゅうせい), 衛星 (えいせい), 星 (ほし)
\\	姓			
\\	セイ、ショウ		姓名(せいめい): 
\\	姓(せい): 
\\	姓名 (せいめい), 改姓 (かいせい), 同姓 (どうせい)
\\	性			
\\	セイ、ショウ	さが	異性(いせい): 
\\	個性(こせい): 
\\	知性(ちせい): 
\\	適性(てきせい): 
\\	理性(りせい): 
\\	酸性(さんせい): 
\\	性(せい): 
\\	性格(せいかく): 
\\	性質(せいしつ): 
\\	性能(せいのう): 
\\	性別(せいべつ): 
\\	中性(ちゅうせい): 
\\	女性(じょせい): 
\\	男性(だんせい): 
\\	性質 (せいしつ), 理性 (りせい), 男性 (だんせい)
\\	牲			
\\	セイ		犠牲(ぎせい): 
\\	産			
\\	サン	う.む、う.まれる、うぶ-、む.す	産む(うむ): 
\\	お産(おさん): 
\\	国産(こくさん): 
\\	産休(さんきゅう): 
\\	産後(さんご): 
\\	産出(さんしゅつ): 
\\	産婦人科(さんふじんか): 
\\	産物(さんぶつ): 
\\	資産(しさん): 
\\	出産(しゅっさん): 
\\	畜産(ちくさん): 
\\	倒産(とうさん): 
\\	特産(とくさん): 
\\	土産(どさん): 
\\	不動産(ふどうさん): 
\\	名産(めいさん): 
\\	共産(きょうさん): 
\\	原産(げんさん): 
\\	財産(ざいさん): 
\\	産業(さんぎょう): 
\\	産地(さんち): 
\\	水産(すいさん): 
\\	生産(せいさん): 
\\	農産物(のうさんぶつ): 
\\	破産(はさん): 
\\	土産(みやげ): 
\\	お土産(おみやげ): 
\\	産まれる (う.まれる), 産む (う.む), 産 (うぶ)
\\	隆			
\\	リュウ			
\\	峰			
\\	縫 
\\	逢 
\\	蜂 
\\	鋒 
\\	ホウ	みね、ね	峰(みね): 
\\	峰 (みね)
\\	縫			
\\	ホウ	ぬ.う	裁縫(さいほう): 
\\	縫う(ぬう): 
\\	縫う (ぬ.う)
\\	拝			
\\	ハイ	おが.む、おろが.む	崇拝(すうはい): 
\\	拝啓(はいけい): 
\\	拝借(はいしゃく): 
\\	拝む(おがむ): 
\\	拝見(はいけん): 
\\	拝む (おが.む)
\\	寿			
\\	ジュ、ス、シュウ	ことぶき、ことぶ.く、ことほ.ぐ	寿命(じゅみょう): 
\\	寿 (ことぶき)
\\	鋳			
\\	チュウ、イ、シュ、シュウ	い.る		鋳る (い.る)
\\	籍			
\\	セキ		戸籍(こせき): 
\\	国籍(こくせき): 
\\	書籍(しょせき): 
\\	春			
\\	シュン	はる	青春(せいしゅん): 
\\	春(はる): 
\\	春 (はる)
\\	椿			
\\	チン、チュン	つばき		
\\	泰			
\\	タイ 泰国 たいこく 
\\	タイ			
\\	奏			
\\	ソウ	かな.でる	吹奏(すいそう): 
\\	演奏(えんそう): 
\\	奏でる (かな.でる)
\\	実			
\\	ジツ、シツ	み、みの.る、まこと、まことに、みの、みち.る	真実(さな): 
\\	実(じつ): 
\\	実業家(じつぎょうか): 
\\	実質(じっしつ): 
\\	実情(じつじょう): 
\\	実践(じっせん): 
\\	実態(じったい): 
\\	実は(じつは): 
\\	実費(じっぴ): 
\\	充実(じゅうじつ): 
\\	誠実(せいじつ): 
\\	切実(せつじつ): 
\\	忠実(ちゅうじつ): 
\\	確実(かくじつ): 
\\	果実(かじつ): 
\\	現実(げんじつ): 
\\	口実(こうじつ): 
\\	事実(じじつ): 
\\	実感(じっかん): 
\\	実験(じっけん): 
\\	実現(じつげん): 
\\	実行(じっこう): 
\\	実際(じっさい): 
\\	実施(じっし): 
\\	実習(じっしゅう): 
\\	実績(じっせき): 
\\	実に(じつに): 
\\	実物(じつぶつ): 
\\	実用(じつよう): 
\\	実力(じつりょく): 
\\	実例(じつれい): 
\\	実(み): 
\\	実る(みのる): 
\\	実 (み), 実る (みの.る)
\\	奉			
\\	ホウ、ブ	たてまつ.る、まつ.る、ほう.ずる	奉る(たてまつる): 
\\	奉仕(ほうし): 
\\	奉納 (ほうのう), 奉仕 (ほうし), 信奉 (しんぽう), 奉る (たてまつ.る)
\\	俸			
\\	ホウ			
\\	棒			
\\	ボウ		鉄棒(かなぼう): 
\\	棒(ぼう): 
\\	泥棒(どろぼう): 
\\	謹			
\\	キン	つつし.む		謹む (つつし.む)
\\	勤			
\\	キン、ゴン	つと.める、-づと.め、つと.まる、いそ.しむ	勤勉(きんべん): 
\\	勤務(きんむ): 
\\	勤労(きんろう): 
\\	勤まる(つとまる): 
\\	勤め(つとめ): 
\\	勤め先(つとめさき): 
\\	転勤(てんきん): 
\\	出勤(しゅっきん): 
\\	通勤(つうきん): 
\\	勤める(つとめる): 
\\	勤務 (きんむ), 勤勉 (きんべん), 出勤 (しゅっきん), 勤まる (つと.まる), 勤める (つと.める)
\\	漢			
\\	カン		漢語(かんご): 
\\	漢和(かんわ): 
\\	漢字(かんじ): 
\\	嘆			
\\	タン	なげ.く、なげ.かわしい	嘆く(なげく): 
\\	嘆かわしい (なげ.かわしい), 嘆く (なげ.く)
\\	難			
\\	ナン	かた.い、-がた.い、むずか.しい、むづか.しい、むつか.しい、-にく.い	有難う(ありがとう): 
\\	難い(かたい): 
\\	遭難(そうなん): 
\\	難(なん): 
\\	避難(ひなん): 
\\	非難(ひなん): 
\\	無難(ぶなん): 
\\	有難い(ありがたい): 
\\	困難(こんなん): 
\\	災難(さいなん): 
\\	盗難(とうなん): 
\\	難しい(むずかしい): 
\\	難い (かた.い), 難しい (むずか.しい)
\\	華			
\\	カ、ケ	はな	華奢(かしゃ): 
\\	華美(かび): 
\\	華々しい(はなばなしい): 
\\	華やか(はなやか): 
\\	豪華(ごうか): 
\\	華美 (かび), 繁華 (はんか), 栄華 (えいが), 華 (はな)
\\	垂			
\\	スイ	た.れる、た.らす、た.れ、-た.れ、なんなんと.す	垂れる(たれる): 
\\	垂直(すいちょく): 
\\	垂らす (た.らす), 垂れる (た.れる)
\\	睡			
\\	スイ		睡眠(すいみん): 
\\	錘			
\\	スイ	つむ、おもり		錘 (つむ)
\\	乗			
\\	ジョウ、ショウ	の.る、-の.り、の.せる	乗客(じょうかく): 
\\	乗せる(のせる): 
\\	乗っ取る(のっとる): 
\\	乗り換え(のりかえ): 
\\	乗り込む(のりこむ): 
\\	乗客(じょうきゃく): 
\\	乗車(じょうしゃ): 
\\	乗換(のりかえ): 
\\	乗り換える(のりかえる): 
\\	乗り物(のりもの): 
\\	乗る(のる): 
\\	乗せる (の.せる), 乗る (の.る)
\\	剰			
\\	過剰 (かじょう) 
\\	ジョウ	あまつさえ、あま.り、あま.る	過剰(かじょう): 
\\	今			
\\	コン、キン	いま	今更(いまさら): 
\\	今日は(こんにちは): 
\\	今に(いまに): 
\\	今にも(いまにも): 
\\	今回(こんかい): 
\\	今後(こんご): 
\\	今日(こんにち): 
\\	今夜(こんや): 
\\	今度(こんど): 
\\	今(いま): 
\\	今日(きょう): 
\\	今朝(けさ): 
\\	今年(ことし): 
\\	今月(こんげつ): 
\\	今週(こんしゅう): 
\\	今晩(こんばん): 
\\	今後 (こんご), 今日 (きょう), 今朝 (けさ), 今年 (こんねん), 昨今 (さっこん), 今 (いま)
\\	含			
\\	ガン	ふく.む、ふく.める	含む(ふくむ): 
\\	含める(ふくめる): 
\\	含む (ふく.む), 含める (ふく.める)
\\	吟			
\\	ギン		吟味(ぎんみ): 
\\	念			
\\	ネン		概念(がいねん): 
\\	念(ねん): 
\\	無念(むねん): 
\\	観念(かんねん): 
\\	記念(きねん): 
\\	残念(ざんねん): 
\\	琴			
\\	キン 琴線 きんせん
\\	キン	こと	琴(こと): 
\\	琴 (こと)
\\	陰			
\\	イン	かげ、かげ.る	陰気(いんき): 
\\	日陰(ひかげ): 
\\	陰(かげ): 
\\	陰 (かげ), 陰る (かげ.る)
\\	予			
\\	ヨ、シャ	あらかじ.め	予め(あらかじめ): 
\\	予言(かねごと): 
\\	予感(よかん): 
\\	予想(よそう): 
\\	予期(よき): 
\\	予算(よさん): 
\\	予測(よそく): 
\\	予備(よび): 
\\	予報(よほう): 
\\	予防(よぼう): 
\\	天気予報(てんきよほう): 
\\	予習(よしゅう): 
\\	予定(よてい): 
\\	予約(よやく): 
\\	序			
\\	ジョ	つい.で、ついで	秩序(ちつじょ): 
\\	順序(じゅんじょ): 
\\	預			
\\	預金機(よきんき)
\\	ヨ	あず.ける、あず.かる	預金(よきん): 
\\	預かる(あずかる): 
\\	預ける(あずける): 
\\	預かる (あず.かる), 預ける (あず.ける)
\\	野			
\\	ヤ、ショ	の、の-	視野(しや): 
\\	野外(やがい): 
\\	野心(やしん): 
\\	野生(やせい): 
\\	野党(やとう): 
\\	野(の): 
\\	分野(ぶんや): 
\\	平野(へいや): 
\\	野菜(やさい): 
\\	野 (の)
\\	兼			
\\	ケン	か.ねる、-か.ねる	兼ねて(かねて): 
\\	気兼ね(きがね): 
\\	兼用(けんよう): 
\\	兼業(けんぎょう): 
\\	兼ねる(かねる): 
\\	兼ねる (か.ねる)
\\	嫌			
\\	ケン、ゲン	きら.う、きら.い、いや	嫌がる(いやがる): 
\\	機嫌(きげん): 
\\	嫌う(きらう): 
\\	好き嫌い(すききらい): 
\\	嫌(いや): 
\\	嫌い(きらい): 
\\	嫌悪 (けんお), 嫌疑 (けんぎ), 嫌 (いや), 嫌う (きら.う)
\\	鎌			
\\	レン、ケン	かま		鎌 (かま)
\\	謙			
\\	ケン		謙る(へりくだる): 
\\	謙虚(けんきょ): 
\\	謙遜(けんそん): 
\\	廉			
\\	レン 廉売 れんばい
\\	鎌1599 (ケン): 兼1597 嫌1598 謙1600.	
\\	レン			
\\	西			
\\	⻃. 
\\	セイ、サイ、ス	にし	西日(にしび): 
\\	関西(かんさい): 
\\	西暦(せいれき): 
\\	東西(とうざい): 
\\	西洋(せいよう): 
\\	西(にし): 
\\	西暦 (せいれき), 西部 (せいぶ), 北西 (ほくせい), 西 (にし)
\\	価			
\\	カ、ケ	あたい	価格(かかく): 
\\	価値(かち): 
\\	高価(こうか): 
\\	定価(ていか): 
\\	評価(ひょうか): 
\\	物価(ぶっか): 
\\	価 (あたい)
\\	要			
\\	(ヨウ) 
\\	ヨウ	い.る	要因(よういん): 
\\	要する(ようする): 
\\	要請(ようせい): 
\\	要望(ようぼう): 
\\	重要(じゅうよう): 
\\	主要(しゅよう): 
\\	需要(じゅよう): 
\\	要求(ようきゅう): 
\\	要旨(ようし): 
\\	要するに(ようするに): 
\\	要素(ようそ): 
\\	要点(ようてん): 
\\	要領(ようりょう): 
\\	必要(ひつよう): 
\\	要る(いる): 
\\	要る (い.る)
\\	腰			
\\	腰(こし) 
\\	腰がいたい!
\\	ヨウ	こし	腰(こし): 
\\	腰掛け(こしかけ): 
\\	腰掛ける(こしかける): 
\\	腰 (こし)
\\	票			
\\	ヒョウ		票(ひょう): 
\\	投票(とうひょう): 
\\	漂			
\\	ヒョウ	ただよ.う	漂う(ただよう): 
\\	漂う (ただよ.う)
\\	標			
\\	ヒョウ	しるべ、しるし	座標(ざひょう): 
\\	標語(ひょうご): 
\\	標識(ひょうしき): 
\\	標準(ひょうじゅん): 
\\	標本(ひょうほん): 
\\	目標(もくひょう): 
\\	栗			
\\	リツ、リ	くり、おののく		
\\	遷			
\\	セン	うつ.る、うつ.す、みやこがえ	変遷(へんせん): 
\\	覆			
\\	フク	おお.う、くつがえ.す、くつがえ.る	覆す(くつがえす): 
\\	覆面(ふくめん): 
\\	覆う(おおう): 
\\	覆う (おお.う), 覆す (くつがえ.す), 覆る (くつがえ.る)
\\	煙			
\\	エン	けむ.る、けむり、けむ.い	煙たい(けむたい): 
\\	煙る(けむる): 
\\	煙草(たばこ): 
\\	煙突(えんとつ): 
\\	禁煙(きんえん): 
\\	煙い(けむい): 
\\	煙(けむり): 
\\	煙い (けむ.い), 煙る (けむ.る), 煙 (けむり)
\\	南			
\\	ナン、ナ	みなみ	南(なん): 
\\	南極(なんきょく): 
\\	南米(なんべい): 
\\	南北(なんぼく): 
\\	南(みなみ): 
\\	南北 (なんぼく), 南端 (なんたん), 指南 (しなん), 南 (みなみ)
\\	楠			
\\	ナン、ダン、ゼン、ネン	くす、くすのき		
\\	献			
\\	ケン、コン	たてまつ.る	貢献(こうけん): 
\\	献立(こんだて): 
\\	文献(ぶんけん): 
\\	献上 (けんじょう), 献身的 (けんしんてき), 文献 (ぶんけん)
\\	門			
\\	モン	かど、と	門(かど): 
\\	正門(せいもん): 
\\	部門(ぶもん): 
\\	門(もん): 
\\	門 (かど)
\\	問			
\\	モン 質問 しつもん
\\	モン	と.う、と.い、とん	問い合わせる(といあわせる): 
\\	問屋(といや): 
\\	問う(とう): 
\\	問(もん): 
\\	学問(がくもん): 
\\	疑問(ぎもん): 
\\	問い(とい): 
\\	問い合わせ(といあわせ): 
\\	訪問(ほうもん): 
\\	問答(もんどう): 
\\	質問(しつもん): 
\\	問題(もんだい): 
\\	問い (と.い), 問う (と.う)
\\	閲			
\\	エツ	けみ.する	閲覧(えつらん): 
\\	閥			
\\	バツ			
\\	間			
\\	カン、ケン	あいだ、ま、あい	間柄(あいだがら): 
\\	合間(あいま): 
\\	空間(あきま): 
\\	何時の間にか(いつのまにか): 
\\	区間(くかん): 
\\	この間(このあいだ): 
\\	週間(しゅうかん): 
\\	茶の間(ちゃのま): 
\\	昼間(ちゅうかん): 
\\	束の間(つかのま): 
\\	間違う(まちがう): 
\\	間に合う(まにあう): 
\\	間々(まま): 
\\	間もなく(まもなく): 
\\	間(あいだ): 
\\	居間(いま): 
\\	貸間(かしま): 
\\	間隔(かんかく): 
\\	間接(かんせつ): 
\\	期間(きかん): 
\\	客間(きゃくま): 
\\	時間割(じかんわり): 
\\	瞬間(しゅんかん): 
\\	隙間(すきま): 
\\	世間(せけん): 
\\	中間(ちゅうかん): 
\\	手間(てま): 
\\	床の間(とこのま): 
\\	仲間(なかま): 
\\	人間(にんげん): 
\\	年間(ねんかん): 
\\	間(ま): 
\\	間違い(まちがい): 
\\	民間(みんかん): 
\\	夜間(やかん): 
\\	昼間(ひるま): 
\\	間違える(まちがえる): 
\\	時間(じかん): 
\\	間隔 (かんかく), 中間 (ちゅうかん), 時間 (じかん), 間 (あいだ), 間 (ま)
\\	簡			
\\	ししおどし 
\\	カン		簡易(かんい): 
\\	簡潔(かんけつ): 
\\	簡素(かんそ): 
\\	簡単(かんたん): 
\\	開			
\\	カイ	ひら.く、ひら.き、-びら.き、ひら.ける、あ.く、あ.ける	開催(かいさい): 
\\	開拓(かいたく): 
\\	開発(かいはつ): 
\\	公開(こうかい): 
\\	切開(せっかい): 
\\	打開(だかい): 
\\	未開(みかい): 
\\	開会(かいかい): 
\\	開始(かいし): 
\\	開通(かいつう): 
\\	開放(かいほう): 
\\	展開(てんかい): 
\\	開く(ひらく): 
\\	開く(あく): 
\\	開ける(あける): 
\\	開く (あ.く), 開ける (あ.ける), 開く (ひら.く), 開ける (ひら.ける)
\\	閉			
\\	ヘイ	と.じる、と.ざす、し.める、し.まる、た.てる	閉口(へいこう): 
\\	閉鎖(へいさ): 
\\	閉じる(とじる): 
\\	閉会(へいかい): 
\\	閉まる(しまる): 
\\	閉める(しめる): 
\\	閉まる (し.まる), 閉める (し.める), 閉ざす (と.ざす), 閉じる (と.じる)
\\	閣			
\\	カク		内閣(ないかく): 
\\	閑			
\\	カン		長閑(のどか): 
\\	聞			
\\	ブン、モン	き.く、き.こえる	聞き取り(ききとり): 
\\	聞こえる(きこえる): 
\\	新聞社(しんぶんしゃ): 
\\	聞く(きく): 
\\	新聞(しんぶん): 
\\	新聞 (しんぶん), 風聞 (ふうぶん), 見聞 (けんぶん), 聞く (き.く), 聞こえる (き.こえる)
\\	潤			
\\	ジュン	うるお.う、うるお.す、うる.む	潤う(うるおう): 
\\	利潤(りじゅん): 
\\	潤む (うる.む), 潤う (うるお.う), 潤す (うるお.す)
\\	欄			
\\	ラン	てすり	欄(らん): 
\\	闘			
\\	トウ	たたか.う、あらそ.う	戦闘(せんとう): 
\\	奮闘(ふんとう): 
\\	闘う (たたか.う)
\\	倉			
\\	ソウ	くら	倉庫(そうこ): 
\\	倉 (くら)
\\	創			
\\	創造 そうぞう 
\\	ソウ、ショウ	つく.る、はじ.める、きず、けず.しける	創刊(そうかん): 
\\	創作(そうさく): 
\\	創造(そうぞう): 
\\	創立(そうりつ): 
\\	独創(どくそう): 
\\	非			
\\	ヒ	あら.ず	"是非とも(ぜひとも): 
\\	非(ひ): 
\\	非行(ひこう): 
\\	非難(ひなん): 
\\	是非(ぜひ): 
\\	非常(ひじょう): 
\\	非常に(ひじょうに): 
\\	俳			
\\	ハイ		俳優(はいゆう): 
\\	俳句(はいく): 
\\	排			
\\	ハイ		排除(はいじょ): 
\\	排水(はいすい): 
\\	悲			
\\	ヒ	かな.しい、かな.しむ	悲観(ひかん): 
\\	悲惨(ひさん): 
\\	悲鳴(ひめい): 
\\	悲しむ(かなしむ): 
\\	悲劇(ひげき): 
\\	悲しい(かなしい): 
\\	悲しい (かな.しい), 悲しむ (かな.しむ)
\\	罪			
\\	ザイ	つみ	罪(つみ): 
\\	犯罪(はんざい): 
\\	罪 (つみ)
\\	輩			
\\	ハイ	ともがら、 -ばら、 やかい、 やから	後輩(こうはい): 
\\	先輩(せんぱい): 
\\	扉			
\\	ヒ	とびら	扉(とびら): 
\\	扉 (とびら)
\\	侯			
\\	コウ			
\\	候			
\\	コウ	そうろう	気候(きこう): 
\\	候補(こうほ): 
\\	天候(てんこう): 
\\	候 (そうろう)
\\	決			
\\	ケツ	き.める、-ぎ.め、き.まる、さ.く	議決(ぎけつ): 
\\	決まり悪い(きまりわるい): 
\\	決意(けつい): 
\\	決議(けつぎ): 
\\	決行(けっこう): 
\\	決算(けっさん): 
\\	決勝(けっしょう): 
\\	決断(けつだん): 
\\	採決(さいけつ): 
\\	対決(たいけつ): 
\\	多数決(たすうけつ): 
\\	判決(はんけつ): 
\\	否決(ひけつ): 
\\	解決(かいけつ): 
\\	可決(かけつ): 
\\	決まり(きまり): 
\\	決心(けっしん): 
\\	決定(けってい): 
\\	決まる(きまる): 
\\	決める(きめる): 
\\	決して(けっして): 
\\	決まる (き.まる), 決める (き.める)
\\	快			
\\	カイ	こころよ.い	軽快(けいかい): 
\\	快い(こころよい): 
\\	全快(ぜんかい): 
\\	快晴(かいせい): 
\\	快適(かいてき): 
\\	愉快(ゆかい): 
\\	快い (こころよ.い)
\\	偉			
\\	イ	えら.い	偉大(いだい): 
\\	偉い(えらい): 
\\	偉い (えら.い)
\\	違			
\\	イ	ちが.う、ちが.い、ちが.える、-ちが.える、たが.う、たが.える	行き違い(いきちがい): 
\\	食い違う(くいちがう): 
\\	擦れ違い(すれちがい): 
\\	すれ違う(すれちがう): 
\\	違える(ちがえる): 
\\	間違う(まちがう): 
\\	違反(いはん): 
\\	勘違い(かんちがい): 
\\	相違(そうい): 
\\	違い(ちがい): 
\\	違いない(ちがいない): 
\\	間違い(まちがい): 
\\	間違える(まちがえる): 
\\	違う(ちがう): 
\\	違う (ちが.う), 違える (ちが.える)
\\	緯			
\\	イ	よこいと、ぬき	経緯(いきさつ): 
\\	緯度(いど): 
\\	衛			
\\	エイ、エ		衛生(えいせい): 
\\	護衛(ごえい): 
\\	守衛(しゅえい): 
\\	防衛(ぼうえい): 
\\	衛星(えいせい): 
\\	自衛(じえい): 
\\	韓			
\\	カン	から、いげた		
\\	干			
\\	カン	ほ.す、ほ.し-、-ぼ.し、ひ.る	梅干(うめぼし): 
\\	干渉(かんしょう): 
\\	若干(じゃっかん): 
\\	干し物(ほしもの): 
\\	干す(ほす): 
\\	干る (ひ.る), 干す (ほ.す)
\\	肝			
\\	カン	きも	肝心(かんじん): 
\\	肝 (きも)
\\	刊			
\\	カン		刊行(かんこう): 
\\	季刊(きかん): 
\\	創刊(そうかん): 
\\	夕刊(ゆうかん): 
\\	汗			
\\	カン	あせ	汗(あせ): 
\\	汗 (あせ)
\\	軒			
\\	ケン	のき	軒並み(のきなみ): 
\\	軒(のき): 
\\	軒 (のき)
\\	岸			
\\	ガン	きし	沿岸(えんがん): 
\\	岸(きし): 
\\	海岸(かいがん): 
\\	岸 (きし)
\\	幹			
\\	みき.	
\\	カン	みき	幹(かん): 
\\	幹線(かんせん): 
\\	幹部(かんぶ): 
\\	新幹線(しんかんせん): 
\\	幹 (みき)
\\	芋			
\\	ウ	いも		芋 (いも)
\\	宇			
\\	ウ		宇宙(うちゅう): 
\\	余			
\\	「あまり 
\\	ヨ	あま.る、あま.り、あま.す、あんま.り	余り(あんまり): 
\\	余暇(よか): 
\\	余興(よきょう): 
\\	余所見(よそみ): 
\\	余地(よち): 
\\	余程(よっぽど): 
\\	余り(あまり): 
\\	余り(あまり): 
\\	余る(あまる): 
\\	余計(よけい): 
\\	余所(よそ): 
\\	余分(よぶん): 
\\	余裕(よゆう): 
\\	余す (あま.す), 余る (あま.る)
\\	除			
\\	ジョ、ジ	のぞ.く、-よ.け	解除(かいじょ): 
\\	控除(こうじょ): 
\\	除外(じょがい): 
\\	取り除く(とりのぞく): 
\\	排除(はいじょ): 
\\	免除(めんじょ): 
\\	削除(さくじょ): 
\\	除く(のぞく): 
\\	掃除(そうじ): 
\\	除外 (じょがい), 除数 (じょすう), 解除 (かいじょ), 除く (のぞ.く)
\\	徐			
\\	ジョ	おもむ.ろに	徐行(じょこう): 
\\	徐々(そろそろ): 
\\	徐々に(じょじょに): 
\\	叙			
\\	ジョ	つい.ず、ついで		
\\	途			
\\	ト	みち	前途(ぜんと): 
\\	途中(つちゅう): 
\\	途上(とじょう): 
\\	途絶える(とだえる): 
\\	中途(ちゅうと): 
\\	途端(とたん): 
\\	用途(ようと): 
\\	途中(とちゅう): 
\\	斜			
\\	シャ	なな.め、はす	傾斜(けいしゃ): 
\\	斜面(しゃめん): 
\\	斜(はす): 
\\	斜め(ななめ): 
\\	斜め (なな.め)
\\	塗			
\\	ト	ぬ.る、ぬ.り、まみ.れる	塗る(ぬる): 
\\	塗る (ぬ.る)
\\	束			
\\	杏, 
\\	ソク	たば、たば.ねる、つか、つか.ねる	結束(けっそく): 
\\	拘束(こうそく): 
\\	束縛(そくばく): 
\\	束の間(つかのま): 
\\	束(たば): 
\\	約束(やくそく): 
\\	束 (たば)
\\	頼			
\\	頁 
\\	束 
\\	ライ	たの.む、たの.もしい、たよ.る	頼み(たのみ): 
\\	依頼(いらい): 
\\	信頼(しんらい): 
\\	頼もしい(たのもしい): 
\\	頼る(たよる): 
\\	頼む(たのむ): 
\\	頼む (たの.む), 頼もしい (たの.もしい), 頼る (たよ.る)
\\	瀬			
\\	ライ	せ	瀬戸物(せともの): 
\\	瀬 (せ)
\\	勅			
\\	チョク	いまし.める、みことのり		
\\	疎			
\\	ソ、ショ	うと.い、うと.む、まば.ら	疎か(おろそか): 
\\	過疎(かそ): 
\\	疎い (うと.い), 疎む (うと.む)
\\	速			
\\	ソク	はや.い、はや-、はや.める、すみ.やか	迅速(じんそく): 
\\	速力(そくりょく): 
\\	加速(かそく): 
\\	加速度(かそくど): 
\\	急速(きゅうそく): 
\\	高速(こうそく): 
\\	早速(さっそく): 
\\	時速(じそく): 
\\	速達(そくたつ): 
\\	速度(そくど): 
\\	速い(はやい): 
\\	速やか (すみ.やか), 速い (はや.い), 速める (はや.める)
\\	整			
\\	セイ	ととの.える、ととの.う	整然(せいぜん): 
\\	整列(せいれつ): 
\\	整える(ととのえる): 
\\	整数(せいすう): 
\\	整備(せいび): 
\\	整理(せいり): 
\\	調整(ちょうせい): 
\\	整う(ととのう): 
\\	整う (ととの.う), 整える (ととの.える)
\\	剣			
\\	ケン	つるぎ	真剣(しんけん): 
\\	剣 (つるぎ)
\\	険			
\\	ケン	けわ.しい	保険(ほけん): 
\\	険しい(けわしい): 
\\	冒険(ぼうけん): 
\\	危険(きけん): 
\\	険しい (けわ.しい)
\\	検			
\\	ケン	しら.べる	検事(けんじ): 
\\	探検(たんけん): 
\\	点検(てんけん): 
\\	検査(けんさ): 
\\	検討(けんとう): 
\\	倹			
\\	ケン	つま.しい、つづまやか	倹約(けんやく): 
\\	重			
\\	ジュウ、チョウ	え、おも.い、おも.り、おも.なう、かさ.ねる、かさ.なる、おも	重(え): 
\\	重なる(おもなる): 
\\	重役(おもやく): 
\\	重んじる(おもんじる): 
\\	重んずる(おもんずる): 
\\	重複(じゅうふく): 
\\	重宝(じゅうほう): 
\\	比重(ひじゅう): 
\\	重たい(おもたい): 
\\	重なる(かさなる): 
\\	重ねる(かさねる): 
\\	貴重(きちょう): 
\\	厳重(げんじゅう): 
\\	重視(じゅうし): 
\\	重体(じゅうたい): 
\\	重大(じゅうだい): 
\\	重点(じゅうてん): 
\\	重役(じゅうやく): 
\\	重要(じゅうよう): 
\\	重量(じゅうりょう): 
\\	重力(じゅうりょく): 
\\	慎重(しんちょう): 
\\	尊重(そんちょう): 
\\	重い(おもい): 
\\	重量 (じゅうりょう), 重大 (じゅうだい), 二重 (にじゅう), 重 (え), 重い (おも.い), 重なる (かさ.なる), 重ねる (かさ.ねる)
\\	動			
\\	ドウ	うご.く、うご.かす	異動(いどう): 
\\	動き(うごき): 
\\	動く(うごく): 
\\	自動詞(じどうし): 
\\	出動(しゅつどう): 
\\	助動詞(じょどうし): 
\\	振動(しんどう): 
\\	騒動(そうどう): 
\\	他動詞(たどうし): 
\\	動員(どういん): 
\\	動機(どうき): 
\\	動向(どうこう): 
\\	動的(どうてき): 
\\	動揺(どうよう): 
\\	動力(どうりょく): 
\\	不動産(ふどうさん): 
\\	変動(へんどう): 
\\	暴動(ぼうどう): 
\\	移動(いどう): 
\\	動かす(うごかす): 
\\	運動(うんどう): 
\\	活動(かつどう): 
\\	感動(かんどう): 
\\	形容動詞(けいようどうし): 
\\	行動(こうどう): 
\\	自動(じどう): 
\\	動作(どうさ): 
\\	動詞(どうし): 
\\	動く(うごく): 
\\	動物園(どうぶつえん): 
\\	自動車(じどうしゃ): 
\\	動物(どうぶつ): 
\\	動かす (うご.かす), 動く (うご.く)
\\	勲			
\\	クン	いさお		
\\	働			
\\	ドウ、リュク、リキ、ロク、リョク	はたら.く	働(どう): 
\\	共働き(ともばたらき): 
\\	働き(はたらき): 
\\	労働(ろうどう): 
\\	働く(はたらく): 
\\	働く (はたら.く)
\\	種			
\\	シュ	たね、-ぐさ	各種(かくしゅ): 
\\	種々(くさぐさ): 
\\	種(しゅ): 
\\	品種(ひんしゅ): 
\\	一種(いっしゅ): 
\\	種類(しゅるい): 
\\	人種(じんしゅ): 
\\	種(たね): 
\\	種 (たね)
\\	衝			
\\	衝 【しょう】 
\\	ショウ	つ.く	衝撃(しょうげき): 
\\	衝突(しょうとつ): 
\\	薫			
\\	クン	かお.る		薫る (かお.る)
\\	病			
\\	ビョウ、ヘイ	や.む、-や.み、やまい	臆病(おくびょう): 
\\	発病(はつびょう): 
\\	病(やまい): 
\\	病む(やむ): 
\\	看病(かんびょう): 
\\	病院(びょういん): 
\\	病気(びょうき): 
\\	病気 (びょうき), 病根 (びょうこん), 看病 (かんびょう), 病む (や.む), 病 (やまい)
\\	痴			
\\	チ	し.れる、おろか	愚痴(ぐち): 
\\	痘			
\\	トウ			
\\	症			
\\	ショウ		症(しょう): 
\\	症状(しょうじょう): 
\\	疾			
\\	シツ	はや.い		
\\	痢			
\\	リ		下痢(げり): 
\\	疲			
\\	ヒ	つか.れる、-づか.れ、つか.らす	疲労(ひろう): 
\\	疲れ(つかれ): 
\\	疲れる(つかれる): 
\\	疲らす (つか.らす), 疲れる (つか.れる)
\\	疫			
\\	エキ、ヤク			疫病 (えきびょう), 悪疫 (あくえき), 防疫 (ぼうえき)
\\	痛			
\\	ツウ	いた.い、いた.む、いた.ましい、いた.める	痛む(いたむ): 
\\	痛める(いためる): 
\\	痛感(つうかん): 
\\	痛切(つうせつ): 
\\	痛み(いたみ): 
\\	苦痛(くつう): 
\\	頭痛(ずつう): 
\\	痛い(いたい): 
\\	痛い (いた.い), 痛む (いた.む), 痛める (いた.める)
\\	癖			
\\	ヘキ	くせ、くせ.に	癖(くせ): 
\\	癖 (くせ)
\\	匿			
\\	トク	かくま.う		
\\	匠			
\\	ショウ	たくみ		
\\	医			
\\	イ	い.やす、い.する、くすし	医師(いし): 
\\	医療(いりょう): 
\\	医学(いがく): 
\\	歯医者(はいしゃ): 
\\	医者(いしゃ): 
\\	匹			
\\	ヒツ	ひき	匹(ひき): 
\\	匹敵(ひってき): 
\\	匹 (ひき)
\\	区			
\\	ク、オウ、コウ		区(く): 
\\	区画(くかく): 
\\	区間(くかん): 
\\	区切り(くぎり): 
\\	区々(まちまち): 
\\	区域(くいき): 
\\	区切る(くぎる): 
\\	区分(くぶん): 
\\	区別(くべつ): 
\\	地区(ちく): 
\\	枢			
\\	スウ、シュ	とぼそ、からくり	中枢(ちゅうすう): 
\\	殴			
\\	オウ	なぐ.る	殴る(なぐる): 
\\	殴る (なぐ.る)
\\	欧			
\\	オウ	うた.う、は.く	欧米(おうべい): 
\\	抑			
\\	ヨク	おさ.える	抑圧(よくあつ): 
\\	抑制(よくせい): 
\\	抑える (おさ.える)
\\	仰			
\\	ギョウ、コウ	あお.ぐ、おお.せ、お.っしゃる、おっしゃ.る	仰ぐ(あおぐ): 
\\	仰っしゃる(おっしゃる): 
\\	信仰(しんこう): 
\\	仰視 (ぎょうし), 仰天 (ぎょうてん), 仰角 (ぎょうかく), 仰ぐ (あお.ぐ), 仰せ (おお.せ)
\\	迎			
\\	ゲイ 歓迎 かんげい
\\	ゲイ	むか.える	歓迎(かんげい): 
\\	出迎え(でむかえ): 
\\	出迎える(でむかえる): 
\\	迎え(むかえ): 
\\	迎える(むかえる): 
\\	迎える (むか.える)
\\	登			
\\	トウ、ト、ドウ、ショウ、チョウ	のぼ.る、あ.がる	登校(とうこう): 
\\	登録(とうろく): 
\\	登場(とうじょう): 
\\	登山(とざん): 
\\	登る(のぼる): 
\\	登壇 (とうだん), 登校 (とうこう), 登記 (とうき), 登る (のぼ.る)
\\	澄			
\\	チョウ	す.む、す.ます、-す.ます	澄ます(すます): 
\\	澄む(すむ): 
\\	澄ます (す.ます), 澄む (す.む)
\\	発			
\\	出発 
\\	ハツ、ホツ	た.つ、あば.く、おこ.る、つか.わす、はな.つ	開発(かいはつ): 
\\	活発(かっぱつ): 
\\	再発(さいはつ): 
\\	始発(しはつ): 
\\	発(はつ): 
\\	発育(はついく): 
\\	発芽(はつが): 
\\	発掘(はっくつ): 
\\	発言(はつげん): 
\\	発生(はっせい): 
\\	発足(はっそく): 
\\	発病(はつびょう): 
\\	発条(ばね): 
\\	反発(はんぱつ): 
\\	発作(ほっさ): 
\\	蒸発(じょうはつ): 
\\	発つ(たつ): 
\\	爆発(ばくはつ): 
\\	発揮(はっき): 
\\	発見(はっけん): 
\\	発行(はっこう): 
\\	発車(はっしゃ): 
\\	発射(はっしゃ): 
\\	発想(はっそう): 
\\	発達(はったつ): 
\\	発展(はってん): 
\\	発電(はつでん): 
\\	発売(はつばい): 
\\	発表(はっぴょう): 
\\	発明(はつめい): 
\\	出発(しゅっぱつ): 
\\	発音(はつおん): 
\\	発明 (はつめい), 発射 (はっしゃ), 突発 (とっぱつ)
\\	廃			
\\	ハイ	すた.れる、すた.る	荒廃(こうはい): 
\\	廃れる(すたれる): 
\\	廃棄(はいき): 
\\	廃止(はいし): 
\\	廃る (すた.る), 廃れる (すた.れる)
\\	僚			
\\	リョウ		官僚(かんりょう): 
\\	同僚(どうりょう): 
\\	寮			
\\	リョウ		寮(りょう): 
\\	療			
\\	リョウ		診療(しんりょう): 
\\	治療(ちりょう): 
\\	医療(いりょう): 
\\	彫			
\\	チョウ	ほ.る、-ぼ.り	彫刻(ちょうこく): 
\\	彫る(ほる): 
\\	彫る (ほ.る)
\\	形			
\\	ケイ、ギョウ	かた、-がた、かたち、なり	"形成(けいせい): 
\\	形勢(けいせい): 
\\	形態(けいたい): 
\\	原形(げんけい): 
\\	地形(じぎょう): 
\\	形式(けいしき): 
\\	形容詞(けいようし): 
\\	形容動詞(けいようどうし): 
\\	図形(ずけい): 
\\	正方形(せいほうけい): 
\\	長方形(ちょうほうけい): 
\\	形(かたち): 
\\	人形(にんぎょう): 
\\	形態 (けいたい), 形成 (けいせい), 図形 (ずけい), 形 (かた), 形 (かたち)
\\	影			
\\	エイ	かげ	影響(えいきょう): 
\\	影(かげ): 
\\	撮影(さつえい): 
\\	影 (かげ)
\\	杉			
\\	サン	すぎ	杉(すぎ): 
\\	杉 (すぎ)
\\	彩			
\\	サイ	いろど.る	色彩(しきさい): 
\\	彩る (いろど.る)
\\	彰			
\\	ショウ			
\\	彦			
\\	ゲン	ひこ		
\\	顔			
\\	ガン	かお	顔付き(かおつき): 
\\	笑顔(えがお): 
\\	顔 (かお)
\\	須			
\\	ス、シュ	すべから.く、すべし、ひげ、まつ、もち.いる、もと.める		
\\	膨			
\\	(腫れる). 
\\	ボウ	ふく.らむ、ふく.れる	膨れる(ふくれる): 
\\	膨脹(ぼうちょう): 
\\	膨らます(ふくらます): 
\\	膨らむ(ふくらむ): 
\\	膨大(ぼうだい): 
\\	膨らむ (ふく.らむ), 膨れる (ふく.れる)
\\	参			
\\	サン、シン	まい.る、まい-、まじわる、みつ	参議院(さんぎいん): 
\\	参照(さんしょう): 
\\	参上(さんじょう): 
\\	お参り(おまいり): 
\\	参加(さんか): 
\\	参考(さんこう): 
\\	持参(じさん): 
\\	参る(まいる): 
\\	参る (まい.る)
\\	惨			
\\	サン、ザン	みじ.め、いた.む、むご.い	悲惨(ひさん): 
\\	惨め(みじめ): 
\\	惨劇 (さんげき), 悲惨 (ひさん), 陰惨 (いんさん), 惨め (みじ.め)
\\	修			
\\	シュウ、シュ	おさ.める、おさ.まる	改修(かいしゅう): 
\\	修学(しゅうがく): 
\\	修行(しゅうぎょう): 
\\	修士(しゅうし): 
\\	修飾(しゅうしょく): 
\\	修了(しゅうりょう): 
\\	専修(せんしゅう): 
\\	必修(ひっしゅう): 
\\	研修(けんしゅう): 
\\	修正(しゅうせい): 
\\	修繕(しゅうぜん): 
\\	修理(しゅうり): 
\\	修飾 (しゅうしょく), 修養 (しゅうよう), 改修 (かいしゅう), 修まる (おさ.まる), 修める (おさ.める)
\\	珍			
\\	チン	めずら.しい、たから	珍しい(めずらしい): 
\\	珍しい (めずら.しい)
\\	診			
\\	シン	み.る	診療(しんりょう): 
\\	聴診器(ちょうしんき): 
\\	診察(しんさつ): 
\\	診断(しんだん): 
\\	診る(みる): 
\\	診る (み.る)
\\	文			
\\	ブン、モン	ふみ、あや	原文(げんぶん): 
\\	注文(ちゅうもん): 
\\	文(ふみ): 
\\	文化財(ぶんかざい): 
\\	文語(ぶんご): 
\\	文書(ぶんしょ): 
\\	本文(ほんぶん): 
\\	文字(もじ): 
\\	和文(わぶん): 
\\	英文(えいぶん): 
\\	人文科学(じんぶんかがく): 
\\	文(ぶん): 
\\	文芸(ぶんげい): 
\\	文献(ぶんけん): 
\\	文章(ぶんしょう): 
\\	文体(ぶんたい): 
\\	文房具(ぶんぼうぐ): 
\\	文脈(ぶんみゃく): 
\\	文明(ぶんめい): 
\\	文句(もんく): 
\\	論文(ろんぶん): 
\\	文化(ぶんか): 
\\	文学(ぶんがく): 
\\	文法(ぶんぽう): 
\\	作文(さくぶん): 
\\	文学 (ぶんがく), 文化 (ぶんか), 作文 (さくぶん), 文 (ふみ)
\\	対			
\\	タイ、ツイ	あいて、こた.える、そろ.い、つれあ.い、なら.ぶ、むか.う	相対(あいたい): 
\\	対応(たいおう): 
\\	対決(たいけつ): 
\\	対抗(たいこう): 
\\	対して(たいして): 
\\	対処(たいしょ): 
\\	対談(たいだん): 
\\	対等(たいとう): 
\\	対比(たいひ): 
\\	対辺(たいへん): 
\\	対面(たいめん): 
\\	対話(たいわ): 
\\	応対(おうたい): 
\\	絶対(ぜったい): 
\\	対(たい): 
\\	対策(たいさく): 
\\	対象(たいしょう): 
\\	対照(たいしょう): 
\\	対する(たいする): 
\\	対立(たいりつ): 
\\	反対(はんたい): 
\\	対立 (たいりつ), 絶対 (ぜったい), 反対 (はんたい)
\\	紋			
\\	モン			
\\	蚊			
\\	ブン	か	蚊(か): 
\\	斉			
\\	セイ、サイ	そろ.う、ひと.しい、ひと.しく、あたる、はやい	一斉(いっせい): 
\\	剤			
\\	ザイ、スイ、セイ	かる、けず.る	洗剤(せんざい): 
\\	済			
\\	サイ、セイ	す.む、-ず.み、-ずみ、す.まない、す.ます、-す.ます、すく.う、な.す、わたし、わた.る	救済(きゅうさい): 
\\	済ます(すます): 
\\	済まない(すまない): 
\\	返済(へんさい): 
\\	済ませる(すませる): 
\\	経済(けいざい): 
\\	済む(すむ): 
\\	済ます (す.ます), 済む (す.む)
\\	斎			
\\	サイ	とき、つつし.む、ものいみ、い.む、いわ.う、いつ.く	書斎(しょさい): 
\\	粛			
\\	シュク、スク	つつし.む		
\\	塁			
\\	ルイ、ライ、スイ	とりで		
\\	楽			
\\	ガク、ラク、ゴウ	たの.しい、たの.しむ、この.む	楽譜(がくふ): 
\\	気楽(きらく): 
\\	極楽(ごくらく): 
\\	楽しむ(たのしむ): 
\\	楽観(らっかん): 
\\	楽器(がっき): 
\\	娯楽(ごらく): 
\\	楽む(たのしむ): 
\\	楽(らく): 
\\	楽しみ(たのしみ): 
\\	音楽(おんがく): 
\\	楽しい(たのしい): 
\\	楽隊 (がくたい), 楽器 (がっき), 音楽 (おんがく), 楽しい (たの.しい), 楽しむ (たの.しむ)
\\	薬			
\\	ヤク	くすり	薬指(くすりゆび): 
\\	農薬(のうやく): 
\\	薬缶(やかん): 
\\	薬品(やくひん): 
\\	薬局(やっきょく): 
\\	薬(くすり): 
\\	薬 (くすり)
\\	率			
\\	ソツ、リツ、シュツ	ひき.いる	軽率(けいそつ): 
\\	効率(こうりつ): 
\\	率直(そっちょく): 
\\	統率(とうそつ): 
\\	倍率(ばいりつ): 
\\	率いる(ひきいる): 
\\	比率(ひりつ): 
\\	確率(かくりつ): 
\\	能率(のうりつ): 
\\	率(りつ): 
\\	率先 (そっせん), 引率 (いんそつ), 軽率 (けいそつ), 率いる (ひき.いる)
\\	渋			
\\	(しぶい). 
\\	ジュウ、シュウ	しぶ、しぶ.い、しぶ.る	"渋い(しぶい): 
\\	渋滞(じゅうたい): 
\\	渋 (しぶ), 渋い (しぶ.い), 渋る (しぶ.る)
\\	摂			
\\	セツ、ショウ	おさ.める、かね.る、と.る		
\\	央			
\\	オウ 中央 ちゅうおう
\\	(エイ): 英1741  映1742.	
\\	オウ		中央(ちゅうおう): 
\\	英			
\\	中華).	
\\	エイ	はなぶさ	英字(えいじ): 
\\	英雄(えいゆう): 
\\	英文(えいぶん): 
\\	英和(えいわ): 
\\	和英(わえい): 
\\	英語(えいご): 
\\	映			
\\	エイ	うつ.る、うつ.す、は.える、-ば.え	映写(えいしゃ): 
\\	映像(えいぞう): 
\\	映える(はえる): 
\\	映す(うつす): 
\\	映る(うつる): 
\\	反映(はんえい): 
\\	映画(えいが): 
\\	映画館(えいがかん): 
\\	映す (うつ.す), 映る (うつ.る), 映える (は.える)
\\	赤			
\\	セキ、シャク	あか、あか-、あか.い、あか.らむ、あか.らめる	赤字(あかじ): 
\\	赤ちゃん(あかちゃん): 
\\	赤らむ(あからむ): 
\\	赤(あか): 
\\	赤道(せきどう): 
\\	真っ赤(まっか): 
\\	赤ん坊(あかんぼう): 
\\	赤い(あかい): 
\\	赤道 (せきどう), 赤貧 (せきひん), 発赤 (はつあか), 赤 (あか), 赤い (あか.い), 赤らむ (あか.らむ), 赤らめる (あか.らめる)
\\	赦			
\\	シャ 赦免 しゃめん
\\	シャ			
\\	変			
\\	ヘン	か.わる、か.わり、か.える	相変わらず(あいかわらず): 
\\	一変(いっぺん): 
\\	変革(へんかく): 
\\	変遷(へんせん): 
\\	変動(へんどう): 
\\	変化(へんか): 
\\	変更(へんこう): 
\\	変える(かえる): 
\\	変わる(かわる): 
\\	変(へん): 
\\	変える (か.える), 変わる (か.わる)
\\	跡			
\\	セキ	あと	跡継ぎ(あとつぎ): 
\\	遺跡(いせき): 
\\	追跡(ついせき): 
\\	跡切れる(とぎれる): 
\\	足跡(あしあと): 
\\	跡(あと): 
\\	跡 (あと)
\\	蛮			
\\	バン	えびす		
\\	恋			
\\	レン	こ.う、こい、こい.しい	恋する(こいする): 
\\	恋愛(れんあい): 
\\	恋(こい): 
\\	恋しい(こいしい): 
\\	恋人(こいびと): 
\\	失恋(しつれん): 
\\	恋う (こ.う), 恋 (こい), 恋しい (こい.しい)
\\	湾			
\\	ワン	いりえ	湾(わん): 
\\	黄			
\\	コウ、オウ	き、こ-	黄金(おうごん): 
\\	黄色(おうしょく): 
\\	黄色(きいろ): 
\\	黄色い(きいろい): 
\\	黄葉 (こうよう), 黄 (き)
\\	横			
\\	オウ	よこ	横綱(よこづな): 
\\	横断(おうだん): 
\\	横切る(よこぎる): 
\\	横(よこ): 
\\	横 (よこ)
\\	把			
\\	ハ、ワ		把握(はあく): 
\\	色			
\\	ショク、シキ	いろ	色々(いろいろ): 
\\	黄色(おうしょく): 
\\	音色(おんいろ): 
\\	脚色(きゃくしょく): 
\\	色彩(しきさい): 
\\	着色(ちゃくしょく): 
\\	黄色(きいろ): 
\\	茶色い(ちゃいろい): 
\\	特色(とくしょく): 
\\	灰色(はいいろ): 
\\	景色(けしき): 
\\	色(いろ): 
\\	黄色い(きいろい): 
\\	茶色(ちゃいろ): 
\\	原色 (げんしょく), 特色 (とくしょく), 物色 (ぶっしょく), 色 (いろ)
\\	絶			
\\	ゼツ	た.える、た.やす、た.つ	拒絶(きょぜつ): 
\\	謝絶(しゃぜつ): 
\\	絶版(ぜっぱん): 
\\	絶望(ぜつぼう): 
\\	絶える(たえる): 
\\	絶つ(たつ): 
\\	途絶える(とだえる): 
\\	絶対(ぜったい): 
\\	絶滅(ぜつめつ): 
\\	絶えず(たえず): 
\\	絶える (た.える), 絶つ (た.つ), 絶やす (た.やす)
\\	艶			
\\	エン 艶聞 えんぶん
\\	エン	つや、なま.めかしい、あで.やか、つや.めく、なま.めく	艶(えん): 
\\	艶(つや): 
\\	艶 (つや)
\\	肥			
\\	ヒ	こ.える、こえ、こ.やす、こ.やし、ふと.る	肥料(ひりょう): 
\\	肥える (こ.える), 肥やし (こ.やし), 肥やす (こ.やす), 肥 (こえ)
\\	甘			
\\	カン	あま.い、あま.える、あま.やかす、うま.い	甘える(あまえる): 
\\	甘口(あまくち): 
\\	甘い(うまい): 
\\	甘やかす(あまやかす): 
\\	甘い(あまい): 
\\	甘い (あま.い), 甘える (あま.える), 甘やかす (あま.やかす)
\\	紺			
\\	コン		紺(こん): 
\\	某			
\\	ボウ	それがし、なにがし		
\\	謀			
\\	ボウ、ム	はか.る、たばか.る、はかりごと		謀略 (ぼうりゃく), 無謀 (むぼう), 首謀者 (しゅぼうしゃ), 謀る (はか.る)
\\	媒			
\\	女 
\\	某
\\	バイ	なこうど		
\\	欺			
\\	ギ	あざむ.く	欺く(あざむく): 
\\	詐欺(さぎ): 
\\	欺く (あざむ.く)
\\	棋			
\\	キ	ご	将棋(しょうぎ): 
\\	旗			
\\	キ	はた	旗(はた): 
\\	旗 (はた)
\\	期			
\\	キ、ゴ		画期(かっき): 
\\	期(き): 
\\	期日(きじつ): 
\\	期末(きまつ): 
\\	周期(しゅうき): 
\\	末期(まっき): 
\\	延期(えんき): 
\\	学期(がっき): 
\\	期間(きかん): 
\\	期限(きげん): 
\\	期待(きたい): 
\\	時期(じき): 
\\	短期(たんき): 
\\	長期(ちょうき): 
\\	定期(ていき): 
\\	定期券(ていきけん): 
\\	予期(よき): 
\\	期間 (きかん), 期待 (きたい), 予期 (よき)
\\	碁			
\\	ゴ		碁盤(ごばん): 
\\	碁(ご): 
\\	基			
\\	キ	もと、もとい	基金(ききん): 
\\	基準(きじゅん): 
\\	基(もとい): 
\\	基礎(きそ): 
\\	基地(きち): 
\\	基盤(きばん): 
\\	基本(きほん): 
\\	基(もと): 
\\	基づく(もとづく): 
\\	基 (もと), 基 (もとい)
\\	甚			
\\	ジン	はなは.だ、はなは.だしい	甚だ(はなはだ): 
\\	甚だしい(はなはだしい): 
\\	甚だ (はなは.だ), 甚だしい (はなは.だしい)
\\	勘			
\\	カン		勘弁(かんべん): 
\\	勘(かん): 
\\	勘定(かんじょう): 
\\	勘違い(かんちがい): 
\\	堪			
\\	カン、タン	た.える、たま.る、こ.らえる、こた.える	堪える(こたえる): 
\\	堪える(こらえる): 
\\	堪える(たえる): 
\\	堪らない(たまらない): 
\\	堪える (た.える)
\\	貴			
\\	(き) 
\\	キ	たっと.い、とうと.い、たっと.ぶ、とうと.ぶ	貴女(あなた): 
\\	貴族(きぞく): 
\\	貴い(たっとい): 
\\	貴重(きちょう): 
\\	貴い (たっと.い), 貴ぶ (たっと.ぶ), 貴い (とうと.い), 貴ぶ (とうと.ぶ)
\\	遺			
\\	イ、ユイ	のこ.す	遺跡(いせき): 
\\	遺棄 (いき), 遺産 (いさん), 遺失 (いしつ)
\\	遣			
\\	ケン	つか.う、-つか.い、-づか.い、つか.わす、や.る	遣い(つかい): 
\\	派遣(はけん): 
\\	無駄遣い(むだづかい): 
\\	遣り通す(やりとおす): 
\\	遣る(やる): 
\\	仮名遣い(かなづかい): 
\\	小遣い(こづかい): 
\\	言葉遣い(ことばづかい): 
\\	遣う (つか.う), 遣わす (つか.わす)
\\	舞			
\\	ブ	ま.う、-ま.う、まい	仕舞(しまい): 
\\	仕舞う(しまう): 
\\	舞う(まう): 
\\	見舞(みまい): 
\\	舞台(ぶたい): 
\\	振舞う(ふるまう): 
\\	見舞い(みまい): 
\\	見舞う(みまう): 
\\	お見舞い(おみまい): 
\\	舞い (ま.い), 舞う (ま.う)
\\	無			
\\	ム、ブ	な.い	感無量(かんむりょう): 
\\	ご無沙汰(ごぶさた): 
\\	台無し(だいなし): 
\\	無難(ぶなん): 
\\	無礼(ぶれい): 
\\	無意味(むいみ): 
\\	無口(むくち): 
\\	無効(むこう): 
\\	無言(むごん): 
\\	無邪気(むじゃき): 
\\	無線(むせん): 
\\	無駄遣い(むだづかい): 
\\	無断(むだん): 
\\	無知(むち): 
\\	無茶(むちゃ): 
\\	無茶苦茶(むちゃくちゃ): 
\\	無念(むねん): 
\\	無能(むのう): 
\\	無闇に(むやみに): 
\\	無用(むよう): 
\\	無論(むろん): 
\\	有無(うむ): 
\\	無し(なし): 
\\	無沙汰(ぶさた): 
\\	無事(ぶじ): 
\\	無(む): 
\\	無限(むげん): 
\\	無視(むし): 
\\	無地(むじ): 
\\	無数(むすう): 
\\	無駄(むだ): 
\\	無料(むりょう): 
\\	無くす(なくす): 
\\	無くなる(なくなる): 
\\	無理(むり): 
\\	無名 (むめい), 無理 (むり), 皆無 (かいむ), 無い (な.い)
\\	組			
\\	ソ	く.む、くみ、-ぐみ	組み合わせ(くみあわせ): 
\\	組み合わせる(くみあわせる): 
\\	組み込む(くみこむ): 
\\	仕組み(しくみ): 
\\	取り組む(とりくむ): 
\\	組(くみ): 
\\	組合(くみあい): 
\\	組み立てる(くみたてる): 
\\	組む(くむ): 
\\	組織(そしき): 
\\	番組(ばんぐみ): 
\\	組む (く.む), 組 (くみ)
\\	粗			
\\	ソ	あら.い、あら-	粗筋(あらすじ): 
\\	粗い(あらい): 
\\	粗末(そまつ): 
\\	粗い (あら.い)
\\	租			
\\	ソ			
\\	祖			
\\	ソ		お祖父さん(おじいさん): 
\\	お祖母さん(おばあさん): 
\\	先祖(せんぞ): 
\\	祖先(そせん): 
\\	祖父(そふ): 
\\	祖母(そぼ): 
\\	阻			
\\	ソ	はば.む	阻止(そし): 
\\	阻む(はばむ): 
\\	阻む (はば.む)
\\	査			
\\	サ		審査(しんさ): 
\\	捜査(そうさ): 
\\	検査(けんさ): 
\\	巡査(じゅんさ): 
\\	調査(ちょうさ): 
\\	助			
\\	ジョ	たす.ける、たす.かる、す.ける、すけ	助(じょ): 
\\	助言(じょげん): 
\\	助詞(じょし): 
\\	助動詞(じょどうし): 
\\	助け(たすけ): 
\\	補助(ほじょ): 
\\	援助(えんじょ): 
\\	救助(きゅうじょ): 
\\	助教授(じょきょうじゅ): 
\\	助手(じょしゅ): 
\\	助かる(たすかる): 
\\	助ける(たすける): 
\\	助 (すけ), 助かる (たす.かる), 助ける (たす.ける)
\\	宜			
\\	ギ 便宜主義 べんぎしゅぎ
\\	ギ	よろ.しい、よろ.しく	適宜(てきぎ): 
\\	どうぞ宜しく(どうぞよろしく): 
\\	便宜(べんぎ): 
\\	宜しく(よろしく): 
\\	宜しい(よろしい): 
\\	畳			
\\	ジョウ、チョウ	たた.む、たたみ、かさ.なる	畳(じょう): 
\\	畳む(たたむ): 
\\	畳(たたみ): 
\\	畳む (たた.む), 畳 (たたみ)
\\	並			
\\	竝, 
\\	ヘイ、ホウ	な.み、なら.べる、なら.ぶ、なら.びに	月並み(つきなみ): 
\\	並み(なみ): 
\\	並びに(ならびに): 
\\	軒並み(のきなみ): 
\\	並列(へいれつ): 
\\	並木(なみき): 
\\	並行(へいこう): 
\\	並ぶ(ならぶ): 
\\	並べる(ならべる): 
\\	並 (なみ), 並びに (なら.びに), 並ぶ (なら.ぶ), 並べる (なら.べる)
\\	普			
\\	フ	あまね.く、あまねし	普遍(ふへん): 
\\	普及(ふきゅう): 
\\	普段(ふだん): 
\\	普通(ふつう): 
\\	譜			
\\	フ		楽譜(がくふ): 
\\	湿			
\\	シツ 湿気 しっけ
\\	シツ、シュウ	しめ.る、しめ.す、うるお.う、うるお.す	湿気る(しける): 
\\	湿気(しっき): 
\\	湿度(しつど): 
\\	湿る(しめる): 
\\	湿す (しめ.す), 湿る (しめ.る)
\\	顕			
\\	ケン	あきらか、あらわ.れる	顕微鏡(けんびきょう): 
\\	繊			
\\	セン		化繊(かせん): 
\\	繊維(せんい): 
\\	霊			
\\	レイ、リョウ	たま	幽霊(ゆうれい): 
\\	霊感 (れいかん), 霊魂 (れいこん), 霊長類 (れいちょうるい), 霊 (たま)
\\	業			
\\	ギョウ、ゴウ	わざ	業者(ぎょうしゃ): 
\\	業績(ぎょうせき): 
\\	業務(ぎょうむ): 
\\	兼業(けんぎょう): 
\\	業(ごう): 
\\	興業(こうぎょう): 
\\	鉱業(こうぎょう): 
\\	事業(じぎょう): 
\\	実業家(じつぎょうか): 
\\	就業(しゅうぎょう): 
\\	従業員(じゅうぎょういん): 
\\	分業(ぶんぎょう): 
\\	林業(りんぎょう): 
\\	営業(えいぎょう): 
\\	企業(きぎょう): 
\\	休業(きゅうぎょう): 
\\	漁業(ぎょぎょう): 
\\	作業(さぎょう): 
\\	産業(さんぎょう): 
\\	失業(しつぎょう): 
\\	商業(しょうぎょう): 
\\	職業(しょくぎょう): 
\\	農業(のうぎょう): 
\\	工業(こうぎょう): 
\\	卒業(そつぎょう): 
\\	授業(じゅぎょう): 
\\	業績 (ぎょうせき), 職業 (しょくぎょう), 卒業 (そつぎょう), 業 (わざ)
\\	撲			
\\	ボク		相撲(すもう): 
\\	僕			
\\	ボク	しもべ	僕(しもべ): 
\\	僕(ぼく): 
\\	共			
\\	キョウ	とも、とも.に、-ども	共(きょう): 
\\	共学(きょうがく): 
\\	共感(きょうかん): 
\\	共存(きょうそん): 
\\	共鳴(きょうめい): 
\\	共和(きょうわ): 
\\	共稼ぎ(ともかせぎ): 
\\	共働き(ともばたらき): 
\\	共産(きょうさん): 
\\	共通(きょうつう): 
\\	共同(きょうどう): 
\\	公共(こうきょう): 
\\	共に(ともに): 
\\	共 (とも)
\\	供			
\\	キョウ、ク、クウ、グ	そな.える、とも、-ども	お供(おとも): 
\\	供(きょう): 
\\	提供(ていきょう): 
\\	供給(きょうきゅう): 
\\	子供(こども): 
\\	供給 (きょうきゅう), 提供 (ていきょう), 自供 (じきょう), 供える (そな.える), 供 (とも)
\\	異			
\\	イ	こと、こと.なる、け	異議(いぎ): 
\\	異見(いけん): 
\\	異性(いせい): 
\\	異動(いどう): 
\\	異論(いろん): 
\\	驚異(きょうい): 
\\	差異(さい): 
\\	異常(いじょう): 
\\	異なる(ことなる): 
\\	異 (こと)
\\	翼			
\\	ヨク	つばさ	翼(つばさ): 
\\	翼 (つばさ)
\\	洪			
\\	コウ		洪水(こうずい): 
\\	港			
\\	コウ	みなと	空港(くうこう): 
\\	港(みなと): 
\\	港 (みなと)
\\	暴			
\\	ボウ、バク	あば.く、あば.れる	暴露(ばくろ): 
\\	暴動(ぼうどう): 
\\	暴風(ぼうふう): 
\\	暴力(ぼうりょく): 
\\	暴れる(あばれる): 
\\	乱暴(らんぼう): 
\\	暴言 (ぼうげん), 横暴 (おうぼう), 乱暴 (らんぼう), 暴く (あば.く), 暴れる (あば.れる)
\\	爆			
\\	バク	は.ぜる	原爆(げんばく): 
\\	爆弾(ばくだん): 
\\	爆破(ばくは): 
\\	爆発(ばくはつ): 
\\	恭			
\\	キョウ	うやうや.しい		恭しい (うやうや.しい)
\\	選			
\\	セン	えら.ぶ	選挙(せんきょ): 
\\	選考(せんこう): 
\\	抽選(ちゅうせん): 
\\	当選(とうせん): 
\\	選手(せんしゅ): 
\\	選択(せんたく): 
\\	選ぶ(えらぶ): 
\\	選ぶ (えら.ぶ)
\\	殿			
\\	デン、テン	との、-どの	宮殿(きゅうでん): 
\\	殿(しんがり): 
\\	神殿(しんでん): 
\\	沈殿(ちんでん): 
\\	殿様(とのさま): 
\\	殿堂 (でんどう), 宮殿 (きゅうでん), 貴殿 (きでん), 殿 (との), 殿 (どの)
\\	井			
\\	セイ、ショウ	い	伊井(いい): 
\\	天井(てんじょう): 
\\	井戸(いど): 
\\	油井 (ゆい), 市井 (しせい), 井 (い)
\\	囲			
\\	イ	かこ.む、かこ.う、かこ.い	囲む(かこむ): 
\\	周囲(しゅうい): 
\\	範囲(はんい): 
\\	雰囲気(ふんいき): 
\\	囲う (かこ.う), 囲む (かこ.む)
\\	耕			
\\	コウ	たがや.す	耕作(こうさく): 
\\	農耕(のうこう): 
\\	耕地(こうち): 
\\	耕す(たがやす): 
\\	耕す (たがや.す)
\\	亜			
\\	ア、アシア	つ.ぐ	亜科(あか): 
\\	悪			
\\	アク、オ	わる.い、わる-、あ.し、にく.い、-にく.い、ああ、いずくに、いずくんぞ、にく.む	悪(あく): 
\\	悪日(あくび): 
\\	悪化(あっか): 
\\	悪口(あっこう): 
\\	悪戯(いたずら): 
\\	改悪(かいあく): 
\\	決まり悪い(きまりわるい): 
\\	悪い(にくい): 
\\	善し悪し(よしあし): 
\\	悪者(わるもの): 
\\	悪魔(あくま): 
\\	意地悪(いじわる): 
\\	悪口(わるくち): 
\\	悪い(わるい): 
\\	悪事 (あくじ), 悪意 (あくい), 醜悪 (しゅうあく), 悪い (わる.い)
\\	円			
\\	エン	まる.い、まる、まど、まど.か、まろ.やか	円滑(えんかつ): 
\\	円満(えんまん): 
\\	円(えん): 
\\	円周(えんしゅう): 
\\	楕円(だえん): 
\\	円(まる): 
\\	円い(まるい): 
\\	円い (まる.い)
\\	角			
\\	カク 兎に角 とにかく
\\	カク	かど、つの	角(かく): 
\\	兎角(とかく): 
\\	角度(かくど): 
\\	三角(さんかく): 
\\	四角(しかく): 
\\	四角い(しかくい): 
\\	角(すみ): 
\\	折角(せっかく): 
\\	直角(ちょっかく): 
\\	方角(ほうがく): 
\\	街角(まちかど): 
\\	四つ角(よつかど): 
\\	角(かど): 
\\	角 (かど), 角 (つの)
\\	触			
\\	ショク	ふ.れる、さわ.る、さわ	気触れる(かぶれる): 
\\	感触(かんしょく): 
\\	接触(せっしょく): 
\\	触れる(ふれる): 
\\	触る(さわる): 
\\	触る (さわ.る), 触れる (ふ.れる)
\\	解			
\\	カイ、ゲ	と.く、と.かす、と.ける、ほど.く、ほど.ける、わか.る、さと.る	解除(かいじょ): 
\\	解剖(かいぼう): 
\\	正解(せいかい): 
\\	弁解(べんかい): 
\\	解く(ほどく): 
\\	了解(りょうかい): 
\\	解決(かいけつ): 
\\	解散(かいさん): 
\\	解釈(かいしゃく): 
\\	解説(かいせつ): 
\\	解答(かいとう): 
\\	解放(かいほう): 
\\	見解(けんかい): 
\\	誤解(ごかい): 
\\	解く(とく): 
\\	解ける(とける): 
\\	分解(ぶんかい): 
\\	理解(りかい): 
\\	解決 (かいけつ), 解禁 (かいきん), 理解 (りかい), 解かす (と.かす), 解く (と.く), 解ける (と.ける)
\\	再			
\\	サイ、サ	ふたた.び	再(さい): 
\\	再会(さいかい): 
\\	再建(さいけん): 
\\	再現(さいげん): 
\\	再生(さいせい): 
\\	再発(さいはつ): 
\\	再三(さいさん): 
\\	再来月(さらいげつ): 
\\	再来週(さらいしゅう): 
\\	再び(ふたたび): 
\\	再来年(さらいねん): 
\\	再度 (さいど), 再選 (さいせん), 再出発 (さいしゅっぱつ), 再び (ふたた.び)
\\	講			
\\	コウ		講習(こうしゅう): 
\\	講読(こうどく): 
\\	聴講(ちょうこう): 
\\	休講(きゅうこう): 
\\	講演(こうえん): 
\\	講師(こうし): 
\\	講義(こうぎ): 
\\	講堂(こうどう): 
\\	購			
\\	コウ		購読(こうどく): 
\\	購入(こうにゅう): 
\\	購買(こうばい): 
\\	構			
\\	コウ	かま.える、かま.う	構え(かまえ): 
\\	構える(かまえる): 
\\	機構(きこう): 
\\	構想(こうそう): 
\\	構う(かまう): 
\\	構成(こうせい): 
\\	構造(こうぞう): 
\\	結構(けっこう): 
\\	構う (かま.う), 構える (かま.える)
\\	溝			
\\	コウ	みぞ	溝(こう): 
\\	溝 (みぞ)
\\	論			
\\	ロン		異論(いろん): 
\\	言論(げんろん): 
\\	世論(せろん): 
\\	討論(とうろん): 
\\	弁論(べんろん): 
\\	無論(むろん): 
\\	目論見(もくろみ): 
\\	理論(りろん): 
\\	論議(ろんぎ): 
\\	論理(ろんり): 
\\	概論(がいろん): 
\\	議論(ぎろん): 
\\	結論(けつろん): 
\\	評論(ひょうろん): 
\\	勿論(もちろん): 
\\	論じる(ろんじる): 
\\	論ずる(ろんずる): 
\\	論争(ろんそう): 
\\	論文(ろんぶん): 
\\	倫			
\\	リン			
\\	輪			
\\	リン	わ	首輪(くびわ): 
\\	年輪(ねんりん): 
\\	輪(りん): 
\\	車輪(しゃりん): 
\\	輪(わ): 
\\	指輪(ゆびわ): 
\\	輪 (わ)
\\	偏			
\\	ヘン	かたよ.る	偏る(かたよる): 
\\	偏(へん): 
\\	偏見(へんけん): 
\\	偏る (かたよ.る)
\\	遍			
\\	ヘン	あまね.く	普遍(ふへん): 
\\	編			
\\	ヘン	あ.む、-あ.み	長編(ちょうへん): 
\\	編(へん): 
\\	編物(あみもの): 
\\	編む(あむ): 
\\	短編(たんぺん): 
\\	編集(へんしゅう): 
\\	編む (あ.む)
\\	冊			
\\	冊 さつ 
\\	サツ、サク	ふみ	冊(さつ): 
\\	冊子 (さっし), 別冊 (べっさつ)
\\	典			
\\	テン、デン		原典(げんてん): 
\\	百科事典(ひゃっかじてん): 
\\	百科辞典(ひゃっかじてん): 
\\	古典(こてん): 
\\	典型(てんけい): 
\\	辞典(じてん): 
\\	氏			
\\	シ	うじ、-うじ	氏(うじ): 
\\	氏(し): 
\\	氏名(しめい): 
\\	氏 (うじ)
\\	紙			
\\	シ	かみ	張り紙(はりがみ): 
\\	用紙(ようし): 
\\	紙屑(かみくず): 
\\	紙幣(しへい): 
\\	塵紙(ちりがみ): 
\\	表紙(ひょうし): 
\\	紙(かみ): 
\\	手紙(てがみ): 
\\	紙 (かみ)
\\	婚			
\\	昏 
\\	コン		既婚(きこん): 
\\	新婚(しんこん): 
\\	未婚(みこん): 
\\	婚約(こんやく): 
\\	離婚(りこん): 
\\	結婚(けっこん): 
\\	低			
\\	テイ	ひく.い、ひく.める、ひく.まる	最低(さいてい): 
\\	低下(ていか): 
\\	低い(ひくい): 
\\	低い (ひく.い), 低まる (ひく.まる), 低める (ひく.める)
\\	抵			
\\	テイ		大抵(たいてい): 
\\	抵抗(ていこう): 
\\	底			
\\	テイ	そこ	根底(こんてい): 
\\	底(そこ): 
\\	到底(とうてい): 
\\	徹底(てってい): 
\\	底 (そこ)
\\	民			
\\	ミン	たみ	移民(いみん): 
\\	植民地(しょくみんち): 
\\	庶民(しょみん): 
\\	人民(じんみん): 
\\	民主(みんしゅ): 
\\	民宿(みんしゅく): 
\\	民俗(みんぞく): 
\\	民族(みんぞく): 
\\	国民(こくみん): 
\\	市民(しみん): 
\\	住民(じゅうみん): 
\\	農民(のうみん): 
\\	民謡(みんよう): 
\\	民間(みんかん): 
\\	民 (たみ)
\\	眠			
\\	ミン	ねむ.る、ねむ.い	冬眠(とうみん): 
\\	眠たい(ねむたい): 
\\	居眠り(いねむり): 
\\	睡眠(すいみん): 
\\	眠い(ねむい): 
\\	眠る(ねむる): 
\\	眠い (ねむ.い), 眠る (ねむ.る)
\\	捕			
\\	ホ	と.らえる、と.らわれる、と.る、とら.える、とら.われる、つか.まえる、つか.まる	捕らえる(とらえる): 
\\	捕獲(ほかく): 
\\	捕鯨(ほげい): 
\\	捕吏(ほり): 
\\	捕虜(ほりょ): 
\\	逮捕(たいほ): 
\\	捕まる(つかまる): 
\\	捕える(とらえる): 
\\	捕る(とる): 
\\	捕まえる(つかまえる): 
\\	捕まえる (つか.まえる), 捕まる (つか.まる), 捕らえる (と.らえる), 捕らわれる (と.らわれる), 捕る (と.る)
\\	浦			
\\	ホ	うら		浦 (うら)
\\	蒲			
\\	ホ、ボ、フ、ブ	がま、かば、かま		
\\	舗			
\\	(舎 
\\	ホ		舗装(ほそう): 
\\	補			
\\	ホ	おぎな.う	補給(ほきゅう): 
\\	補強(ほきょう): 
\\	補充(ほじゅう): 
\\	補助(ほじょ): 
\\	補償(ほしょう): 
\\	補足(ほそく): 
\\	補う(おぎなう): 
\\	候補(こうほ): 
\\	補う (おぎな.う)
\\	邸			
\\	テイ	やしき	邸宅(ていたく): 
\\	郭			
\\	カク	くるわ		
\\	郡			
\\	グン	こうり	郡(ぐん): 
\\	郊			
\\	コウ		近郊(きんこう): 
\\	郊外(こうがい): 
\\	部			
\\	へ 部屋 へや
\\	ブ 全部 ぜんぶ
\\	ブ	-べ	一部(いちぶ): 
\\	一部分(いちぶぶん): 
\\	幹部(かんぶ): 
\\	警部(けいぶ): 
\\	大部(たいぶ): 
\\	内部(ないぶ): 
\\	部(ぶ): 
\\	部下(ぶか): 
\\	部門(ぶもん): 
\\	外部(がいぶ): 
\\	学部(がくぶ): 
\\	大部分(だいぶぶん): 
\\	部首(ぶしゅ): 
\\	部品(ぶひん): 
\\	部分(ぶぶん): 
\\	本部(ほんぶ): 
\\	全部(ぜんぶ): 
\\	部屋(へや): 
\\	都			
\\	ト、ツ	みやこ	首都(しゅと): 
\\	都(と): 
\\	都会(とかい): 
\\	都市(とし): 
\\	都心(としん): 
\\	都(みやこ): 
\\	都合(つごう): 
\\	都会 (とかい), 都心 (としん), 首都 (しゅと), 都 (みやこ)
\\	郵			
\\	ユウ		郵送(ゆうそう): 
\\	郵便(ゆうびん): 
\\	郵便局(ゆうびんきょく): 
\\	邦			
\\	ホウ	くに	連邦(れんぽう): 
\\	郷			
\\	キョウ、ゴウ	さと	郷愁(きょうしゅう): 
\\	郷里(きょうり): 
\\	故郷(こきょう): 
\\	郷里 (きょうり), 郷土 (きょうど), 異郷 (いきょう)
\\	響			
\\	キョウ	ひび.く	反響(はんきょう): 
\\	影響(えいきょう): 
\\	響き(ひびき): 
\\	響く(ひびく): 
\\	響く (ひび.く)
\\	郎			
\\	ロウ 新郎 しんろう
\\	廊1852.	
\\	ロウ、リョウ	おとこ		
\\	廊			
\\	(ロウ) 
\\	ロウ		廊下(ろうか): 
\\	盾			
\\	ジュン	たて	盾(たて): 
\\	矛盾(むじゅん): 
\\	盾 (たて)
\\	循			
\\	循環, 
\\	ジュン		循環(じゅんかん): 
\\	派			
\\	ハ		特派(とくは): 
\\	派(は): 
\\	派遣(はけん): 
\\	派手(はで): 
\\	立派(りっぱ): 
\\	脈			
\\	ミャク	すじ	山脈(さんみゃく): 
\\	脈(みゃく): 
\\	文脈(ぶんみゃく): 
\\	衆			
\\	シュウ、シュ	おお.い	観衆(かんしゅう): 
\\	衆(しゅう): 
\\	衆議院(しゅうぎいん): 
\\	大衆(たいしゅう): 
\\	公衆(こうしゅう): 
\\	衆寡 (しゅうか), 民衆 (みんしゅう), 聴衆 (ちょうしゅう)
\\	逓			
\\	テイ	かわ.る、たがいに		
\\	段			
\\	ダン、タン		段々(だんだん): 
\\	値段(ねだん): 
\\	一段と(いちだんと): 
\\	手段(しゅだん): 
\\	段(だん): 
\\	段階(だんかい): 
\\	普段(ふだん): 
\\	階段(かいだん): 
\\	鍛			
\\	タン	きた.える	鍛える(きたえる): 
\\	鍛える (きた.える)
\\	后			
\\	コウ、ゴ	きさき		
\\	幻			
\\	ゲン	まぼろし		幻 (まぼろし)
\\	司			
\\	シ	つかさど.る	司法(しほう): 
\\	上司(じょうし): 
\\	司る(つかさどる): 
\\	司会(しかい): 
\\	伺			
\\	シ	うかが.う	伺う(うかがう): 
\\	伺う (うかが.う)
\\	詞			
\\	シ		"自動詞(じどうし): 
\\	助詞(じょし): 
\\	助動詞(じょどうし): 
\\	数詞(すうし): 
\\	接続詞(せつぞくし): 
\\	他動詞(たどうし): 
\\	形容詞(けいようし): 
\\	形容動詞(けいようどうし): 
\\	代名詞(だいめいし): 
\\	動詞(どうし): 
\\	副詞(ふくし): 
\\	名詞(めいし): 
\\	飼			
\\	シ	か.う	飼育(しいく): 
\\	飼う(かう): 
\\	飼う (か.う)
\\	嗣			
\\	シ			
\\	舟			
\\	(丹) 
\\	シュウ	ふね、ふな-、-ぶね	舟(ふね): 
\\	舟 (ふね)
\\	舶			
\\	ハク		船舶(せんぱく): 
\\	航			
\\	コウ		航海(こうかい): 
\\	航空(こうくう): 
\\	般			
\\	ハン		一般(いっぱん): 
\\	全般(ぜんぱん): 
\\	盤			
\\	バン		碁盤(ごばん): 
\\	基盤(きばん): 
\\	地盤(じばん): 
\\	算盤(そろばん): 
\\	搬			
\\	搬送周波数 
\\	搬送波 
\\	ハン		運搬(うんぱん): 
\\	船			
\\	セン	ふね、ふな-	汽船(きせん): 
\\	漁船(ぎょせん): 
\\	船舶(せんぱく): 
\\	船(ふね): 
\\	造船(ぞうせん): 
\\	風船(ふうせん): 
\\	船便(ふなびん): 
\\	船 (ふね)
\\	艦			
\\	カン		軍艦(ぐんかん): 
\\	艇			
\\	テイ			
\\	瓜			
\\	何これ???? 
\\	カ、ケ	うり		
\\	弧			
\\	コ		括弧(かっこ): 
\\	孤			
\\	コ		孤児(こじ): 
\\	孤独(こどく): 
\\	孤立(こりつ): 
\\	繭			
\\	ケン	まゆ、きぬ		繭 (まゆ)
\\	益			
\\	エキ、ヤク	ま.す	収益(しゅうえき): 
\\	益々(ますます): 
\\	有益(ゆうえき): 
\\	利益(りえき): 
\\	有益 (ゆうえき), 利益 (りえき), 益する (まする), 益 (やく)
\\	暇			
\\	カ	ひま、いとま	暇(いとま): 
\\	余暇(よか): 
\\	休暇(きゅうか): 
\\	暇(ひま): 
\\	暇 (ひま)
\\	敷			
\\	専, 
\\	方, 
\\	フ	し.く、-し.き	屋敷(やしき): 
\\	座敷(ざしき): 
\\	敷地(しきち): 
\\	敷く(しく): 
\\	風呂敷(ふろしき): 
\\	敷く (し.く)
\\	来			
\\	ライ、タイ	く.る、きた.る、きた.す、き.たす、き.たる、き、こ	外来(がいらい): 
\\	元来(がんらい): 
\\	来る(きたる): 
\\	家来(けらい): 
\\	為来り(しきたり): 
\\	従来(じゅうらい): 
\\	出来物(できもの): 
\\	伝来(でんらい): 
\\	来場(らいじょう): 
\\	以来(いらい): 
\\	再来月(さらいげつ): 
\\	再来週(さらいしゅう): 
\\	出来上がり(できあがり): 
\\	出来上がる(できあがる): 
\\	出来事(できごと): 
\\	出来るだけ(できるだけ): 
\\	本来(ほんらい): 
\\	未来(みらい): 
\\	来(らい): 
\\	来日(らいにち): 
\\	さ来月(さらいげつ): 
\\	さ来週(さらいしゅう): 
\\	将来(しょうらい): 
\\	来る(くる): 
\\	再来年(さらいねん): 
\\	来月(らいげつ): 
\\	来週(らいしゅう): 
\\	来年(らいねん): 
\\	来す (きた.す), 来る (きた.る), 来る (く.る)
\\	気			
\\	キ、ケ	いき	呆気ない(あっけない): 
\\	意気込む(いきごむ): 
\\	一気(いっき): 
\\	陰気(いんき): 
\\	浮気(うわき): 
\\	気質(かたぎ): 
\\	気触れる(かぶれる): 
\\	寒気(かんき): 
\\	気兼ね(きがね): 
\\	気軽(きがる): 
\\	気障(きざ): 
\\	気象(きしょう): 
\\	気立て(きだて): 
\\	気配(きはい): 
\\	気品(きひん): 
\\	気風(きふう): 
\\	気まぐれ(きまぐれ): 
\\	気楽(きらく): 
\\	気流(きりゅう): 
\\	強気(ごうぎ): 
\\	根気(こんき): 
\\	磁気(じき): 
\\	湿気る(しける): 
\\	湿気(しっき): 
\\	水気(すいき): 
\\	素っ気ない(そっけない): 
\\	短気(たんき): 
\\	何気ない(なにげない): 
\\	不景気(ふけいき): 
\\	本気(ほんき): 
\\	無邪気(むじゃき): 
\\	活気(かっき): 
\\	換気(かんき): 
\\	気圧(きあつ): 
\\	気温(きおん): 
\\	気候(きこう): 
\\	気体(きたい): 
\\	気付く(きづく): 
\\	気に入る(きにいる): 
\\	気の毒(きのどく): 
\\	気味(きみ): 
\\	気を付ける(きをつける): 
\\	景気(けいき): 
\\	気配(けはい): 
\\	蒸気(じょうき): 
\\	水蒸気(すいじょうき): 
\\	大気(たいき): 
\\	強気(つよき): 
\\	生意気(なまいき): 
\\	人気(にんき): 
\\	呑気(のんき): 
\\	吐き気(はきけ): 
\\	雰囲気(ふんいき): 
\\	平気(へいき): 
\\	勇気(ゆうき): 
\\	湯気(ゆげ): 
\\	陽気(ようき): 
\\	気(き): 
\\	気分(きぶん): 
\\	気持ち(きもち): 
\\	空気(くうき): 
\\	天気予報(てんきよほう): 
\\	元気(げんき): 
\\	天気(てんき): 
\\	電気(でんき): 
\\	病気(びょうき): 
\\	気体 (きたい), 気候 (きこう), 元気 (げんき)
\\	汽			
\\	キ		汽船(きせん): 
\\	汽車(きしゃ): 
\\	飛			
\\	ヒ 飛行機 ひこうき
\\	ヒ	と.ぶ、と.ばす、-と.ばす	蹴飛ばす(けとばす): 
\\	飛ばす(とばす): 
\\	飛び込む(とびこむ): 
\\	飛び出す(とびだす): 
\\	飛行(ひこう): 
\\	飛行場(ひこうじょう): 
\\	飛ぶ(とぶ): 
\\	飛行機(ひこうき): 
\\	飛ばす (と.ばす), 飛ぶ (と.ぶ)
\\	沈			
\\	チン、ジン	しず.む、しず.める	沈める(しずめる): 
\\	沈殿(ちんでん): 
\\	沈没(ちんぼつ): 
\\	沈黙(ちんもく): 
\\	沈む(しずむ): 
\\	沈む (しず.む), 沈める (しず.める)
\\	妻			
\\	サイ	つま	夫妻(ふさい): 
\\	妻(つま): 
\\	妻 (つま)
\\	衰			
\\	哀 
\\	哀 
\\	スイ	おとろ.える	衰える(おとろえる): 
\\	老衰(ろうすい): 
\\	衰える (おとろ.える)
\\	衷			
\\	チュウ		折衷(せっちゅう): 
\\	面			
\\	メン、ベン	おも、おもて、つら	一面(いちめん): 
\\	面(おも): 
\\	面白い(おもしろい): 
\\	几帳面(きちょうめん): 
\\	生真面目(きまじめ): 
\\	斜面(しゃめん): 
\\	側面(そくめん): 
\\	対面(たいめん): 
\\	断面(だんめん): 
\\	直面(ちょくめん): 
\\	面皰(にきび): 
\\	覆面(ふくめん): 
\\	面会(めんかい): 
\\	面する(めんする): 
\\	面目(めんぼく): 
\\	正面(しょうめん): 
\\	場面(ばめん): 
\\	表面(ひょうめん): 
\\	方面(ほうめん): 
\\	真面目(まじめ): 
\\	面(めん): 
\\	面積(めんせき): 
\\	面接(めんせつ): 
\\	面倒(めんどう): 
\\	面倒臭い(めんどうくさい): 
\\	面 (おも), 面 (おもて), 面 (つら)
\\	革			
\\	カク	かわ	改革(かいかく): 
\\	革新(かくしん): 
\\	革命(かくめい): 
\\	吊り革(つりかわ): 
\\	変革(へんかく): 
\\	革(かわ): 
\\	革 (かわ)
\\	靴			
\\	カ	くつ	靴(くつ): 
\\	靴下(くつした): 
\\	靴 (くつ)
\\	覇			
\\	ハ、ハク	はたがしら		
\\	声			
\\	セイ、ショウ	こえ、こわ-	歓声(かんせい): 
\\	声明(せいめい): 
\\	声(こえ): 
\\	声楽 (せいがく), 声援 (せいえん), 名声 (めいせい), 声 (こえ)
\\	呉			
\\	口 
\\	一. 
\\	ハ 
\\	呉れる、くれる, 
\\	伯父がこの時計を呉れました, 
\\	-て 
\\	窓を開けて呉れませんか, 
\\	ゴ	く.れる、くれ	呉れる(くれる): 
\\	娯			
\\	ゴ		娯楽(ごらく): 
\\	誤			
\\	ゴ	あやま.る、-あやま.る	誤る(あやまる): 
\\	誤差(ごさ): 
\\	誤魔化す(ごまかす): 
\\	錯誤(さくご): 
\\	誤り(あやまり): 
\\	誤解(ごかい): 
\\	誤る (あやま.る)
\\	蒸			
\\	ジョウ、セイ	む.す、む.れる、む.らす	蒸留(じょうりゅう): 
\\	蒸気(じょうき): 
\\	蒸発(じょうはつ): 
\\	水蒸気(すいじょうき): 
\\	蒸し暑い(むしあつい): 
\\	蒸す(むす): 
\\	蒸す (む.す), 蒸らす (む.らす), 蒸れる (む.れる)
\\	承			
\\	ショウ、ジョウ	うけたまわ.る、う.ける、ささ.げる、とど.める、たす.ける、こ.らす、つい.で、すく.う	"承諾(しょうだく): 
\\	了承(りょうしょう): 
\\	承る(うけたまわる): 
\\	承認(しょうにん): 
\\	承知(しょうち): 
\\	承る (うけたまわ.る)
\\	函			
\\	カン	はこ、い.れる		
\\	極			
\\	キョク、ゴク	きわ.める、きわ.まる、きわ.まり、きわ.み、き.める、-ぎ.め、き.まる	究極(きゅうきょく): 
\\	極端(きょくたん): 
\\	極めて(きわめて): 
\\	極楽(ごくらく): 
\\	両極(りょうきょく): 
\\	極(ごく): 
\\	消極的(しょうきょくてき): 
\\	積極的(せっきょくてき): 
\\	南極(なんきょく): 
\\	北極(ほっきょく): 
\\	極限 (きょくげん), 終極 (おわりきょく), 積極的 (せっきょくてき), 極まる (きわ.まる), 極み (きわ.み), 極める (きわ.める)
\\	牙			
\\	ガ、ゲ	きば、は、きばへん		牙 (きば)
\\	芽			
\\	ガ	め	発芽(はつが): 
\\	芽(め): 
\\	芽 (め)
\\	邪			
\\	ジャ	よこし.ま	お邪魔します(おじゃまします): 
\\	邪魔(じゃま): 
\\	無邪気(むじゃき): 
\\	風邪(かぜ): 
\\	雅			
\\	ガ	みや.び	雅致(がち): 
\\	釈			
\\	シャク、セキ	とく、す.てる、ゆる.す	解釈(かいしゃく): 
\\	番			
\\	バン 電話番号 でんわばんごう
\\	バン	つが.い	下番(かばん): 
\\	番目(ばんめ): 
\\	順番(じゅんばん): 
\\	当番(とうばん): 
\\	番地(ばんち): 
\\	留守番(るすばん): 
\\	番組(ばんぐみ): 
\\	一番(いちばん): 
\\	交番(こうばん): 
\\	番号(ばんごう): 
\\	審			
\\	シン	つまび.らか、つぶさ.に	審議(しんぎ): 
\\	審査(しんさ): 
\\	審判(しんばん): 
\\	不審(ふしん): 
\\	審判(しんぱん): 
\\	翻			
\\	ホン、ハン	ひるがえ.る、ひるがえ.す	翻訳(ほんやく): 
\\	翻す (ひるがえ.す), 翻る (ひるがえ.る)
\\	藩			
\\	(廃藩置県), 
\\	落, 
\\	ハン			
\\	毛			
\\	モウ	け	毛(け): 
\\	髪の毛(かみのけ): 
\\	毛糸(けいと): 
\\	毛皮(けがわ): 
\\	毛布(もうふ): 
\\	羊毛(ようもう): 
\\	毛 (け)
\\	耗			
\\	モウ、コウ、カウ		消耗(しょうこう): 
\\	消耗(しょうもう): 
\\	消耗 (しょうもう)
\\	尾			
\\	ビ	お	尾(お): 
\\	尻尾(しっぽ): 
\\	尾 (お)
\\	宅			
\\	タク		社宅(しゃたく): 
\\	邸宅(ていたく): 
\\	帰宅(きたく): 
\\	住宅(じゅうたく): 
\\	宅(たく): 
\\	お宅(おたく): 
\\	託			
\\	タク	かこつ.ける、かこ.つ、かこ.つける	委託(いたく): 
\\	為			
\\	ソ, ユ, 
\\	ユ 
\\	(ソ) 
\\	(ユ) 
\\	(ユ) 
\\	イ	ため、な.る、な.す、す.る、たり、つく.る、なり	行為(こうい): 
\\	為来り(しきたり): 
\\	その為(そのため): 
\\	為さる(なさる): 
\\	為替(かわせ): 
\\	為る(する): 
\\	所為(せい): 
\\	為(ため): 
\\	為す(なす): 
\\	為る(なる): 
\\	偽			
\\	ギ、カ	いつわ.る、にせ、いつわ.り	偽造(ぎぞう): 
\\	偽る (いつわ.る), 偽 (にせ)
\\	長			
\\	チョウ	なが.い、おさ	長(おさ): 
\\	生長(せいちょう): 
\\	長官(ちょうかん): 
\\	長大(ちょうだい): 
\\	長編(ちょうへん): 
\\	長々(ながなが): 
\\	年長(ねんちょう): 
\\	長閑(のどか): 
\\	延長(えんちょう): 
\\	議長(ぎちょう): 
\\	身長(しんちょう): 
\\	成長(せいちょう): 
\\	長期(ちょうき): 
\\	長所(ちょうしょ): 
\\	長女(ちょうじょ): 
\\	長短(ちょうたん): 
\\	長男(ちょうなん): 
\\	長方形(ちょうほうけい): 
\\	特長(とくちょう): 
\\	長引く(ながびく): 
\\	校長(こうちょう): 
\\	社長(しゃちょう): 
\\	長い(ながい): 
\\	長い (なが.い)
\\	張			
\\	チョウ	は.る、-は.り、-ば.り	嵩張る(かさばる): 
\\	誇張(こちょう): 
\\	突っ張る(つっぱる): 
\\	張り紙(はりがみ): 
\\	矢っ張り(やっぱり): 
\\	威張る(いばる): 
\\	拡張(かくちょう): 
\\	緊張(きんちょう): 
\\	主張(しゅちょう): 
\\	出張(しゅっちょう): 
\\	張り切る(はりきる): 
\\	張る(はる): 
\\	引っ張る(ひっぱる): 
\\	欲張り(よくばり): 
\\	張る (は.る)
\\	帳			
\\	チョウ	とばり	几帳面(きちょうめん): 
\\	帳(とばり): 
\\	通帳(つうちょう): 
\\	手帳(てちょう): 
\\	脹			
\\	チョウ	は.れる、ふく.らむ、ふく.れる	膨脹(ぼうちょう): 
\\	髪			
\\	ハツ	かみ	髪(かみ): 
\\	髪の毛(かみのけ): 
\\	白髪(しらが): 
\\	髪 (かみ)
\\	展			
\\	テン		進展(しんてん): 
\\	展示(てんじ): 
\\	展望(てんぼう): 
\\	展開(てんかい): 
\\	発展(はってん): 
\\	展覧会(てんらんかい): 
\\	喪			
\\	ソウ	も		喪 (も)
\\	巣			
\\	ソウ	す、す.くう	巣(す): 
\\	巣 (す)
\\	単			
\\	タン 簡単 かんたん
\\	タン	ひとえ	単一(たんいつ): 
\\	単調(たんちょう): 
\\	単独(たんどく): 
\\	単(ひとえ): 
\\	単位(たんい): 
\\	単語(たんご): 
\\	単純(たんじゅん): 
\\	単数(たんすう): 
\\	単なる(たんなる): 
\\	単に(たんに): 
\\	簡単(かんたん): 
\\	戦			
\\	セン	いくさ、たたか.う、おのの.く、そよぐ、わなな.く	戦(いくさ): 
\\	休戦(きゅうせん): 
\\	作戦(さくせん): 
\\	戦災(せんさい): 
\\	戦術(せんじゅつ): 
\\	戦闘(せんとう): 
\\	戦力(せんりょく): 
\\	挑戦(ちょうせん): 
\\	敗戦(はいせん): 
\\	大戦(たいせん): 
\\	戦い(たたかい): 
\\	戦う(たたかう): 
\\	戦争(せんそう): 
\\	戦 (いくさ), 戦う (たたか.う)
\\	禅			
\\	ゼン、セン	しずか、ゆず.る	禅(ぜん): 
\\	弾			
\\	ダン、タン	ひ.く、-ひ.き、はず.む、たま、はじ.く、はじ.ける、ただ.す、はじ.きゆみ	弾力(だんりょく): 
\\	爆弾(ばくだん): 
\\	弾く(はじく): 
\\	弾む(はずむ): 
\\	弾(たま): 
\\	弾く(ひく): 
\\	弾 (たま), 弾む (はず.む), 弾く (ひ.く)
\\	桜			
\\	花見, 
\\	(さくら), 
\\	オウ、ヨウ	さくら	桜(さくら): 
\\	桜 (さくら)
\\	獣			
\\	ジュウ	けもの、けだもの	怪獣(かいじゅう): 
\\	獣(けだもの): 
\\	獣 (けもの)
\\	脳			
\\	ノウ、ドウ	のうずる	首脳(しゅのう): 
\\	脳(のう): 
\\	頭脳(ずのう): 
\\	悩			
\\	ノウ	なや.む、なや.ます、なや.ましい、なやみ	悩ましい(なやましい): 
\\	悩ます(なやます): 
\\	悩み(なやみ): 
\\	悩む(なやむ): 
\\	悩ます (なや.ます), 悩む (なや.む)
\\	厳			
\\	ゲン 厳禁  げんきん 
\\	ゴン 華厳経 けごんギョウ
\\	ゲン、ゴン	おごそ.か、きび.しい、いか.めしい、いつくし	厳か(おごそか): 
\\	厳密(げんみつ): 
\\	厳重(げんじゅう): 
\\	厳しい(きびしい): 
\\	厳格 (げんかく), 厳重 (げんじゅう), 威厳 (いげん), 厳か (おごそ.か), 厳しい (きび.しい)
\\	鎖			
\\	サ	くさり、とざ.す	封鎖(ふうさ): 
\\	閉鎖(へいさ): 
\\	鎖(くさり): 
\\	鎖 (くさり)
\\	挙			
\\	キョ	あ.げる、あ.がる、こぞ.る	挙げる(あげる): 
\\	一挙に(いっきょに): 
\\	選挙(せんきょ): 
\\	挙がる (あ.がる), 挙げる (あ.げる)
\\	誉			
\\	ヨ	ほま.れ、ほ.める	名誉(めいよ): 
\\	誉 (ほまれ)
\\	猟			
\\	鼡 
\\	鼠 
\\	鼡 
\\	猟 
\\	リョウ、レフ	かり、か.る		
\\	鳥			
\\	チョウ	とり	鳥居(とりい): 
\\	渡り鳥(わたりどり): 
\\	小鳥(ことり): 
\\	鳥(とり): 
\\	鳥 (とり)
\\	鳴			
\\	メイ	な.く、な.る、な.らす	共鳴(きょうめい): 
\\	怒鳴る(どなる): 
\\	悲鳴(ひめい): 
\\	鳴らす(ならす): 
\\	鳴る(なる): 
\\	鳴く(なく): 
\\	鳴く (な.く), 鳴らす (な.らす), 鳴る (な.る)
\\	鶴			
\\	カク	つる		鶴 (つる)
\\	烏			
\\	ウ、オ	からす、いずくんぞ、なんぞ		
\\	蔦			
\\	チョウ	つた		
\\	鳩			
\\	キュウ、ク	はと、あつ.める		
\\	鶏			
\\	鶏 【にわとり】 
\\	鶏肉 【けいにく】 
\\	ケイ	にわとり、とり		鶏 (にわとり)
\\	島			
\\	トウ 半島 はんとう
\\	トウ	しま	半島(はんとう): 
\\	列島(れっとう): 
\\	島(しま): 
\\	島 (しま)
\\	暖			
\\	ダン、ノン	あたた.か、あたた.かい、あたた.まる、あたた.める	暖まる(あたたまる): 
\\	暖める(あたためる): 
\\	温暖(おんだん): 
\\	暖房(だんぼう): 
\\	暖かい(あたたかい): 
\\	暖か (あたた.か), 暖かい (あたた.かい), 暖まる (あたた.まる), 暖める (あたた.める)
\\	媛			
\\	エン	ひめ		媛 (ひめ)
\\	援			
\\	エン		救援(きゅうえん): 
\\	援助(えんじょ): 
\\	応援(おうえん): 
\\	緩			
\\	カン 緩和 かんわ
\\	カン	ゆる.い、ゆる.やか、ゆる.む、ゆる.める	緩和(かんわ): 
\\	緩む(ゆるむ): 
\\	緩める(ゆるめる): 
\\	緩やか(ゆるやか): 
\\	緩い(ゆるい): 
\\	緩い (ゆる.い), 緩む (ゆる.む), 緩める (ゆる.める), 緩やか (ゆる.やか)
\\	属			
\\	ゾク、ショク	さかん、つく、やから	所属(しょぞく): 
\\	付属(ふぞく): 
\\	金属(きんぞく): 
\\	属する(ぞくする): 
\\	附属(ふぞく): 
\\	嘱			
\\	ショク	しょく.する、たの.む		
\\	偶			
\\	グウ	たま	偶(たま): 
\\	偶に(たまに): 
\\	配偶者(はいぐうしゃ): 
\\	偶数(ぐうすう): 
\\	偶然(ぐうぜん): 
\\	偶々(たまたま): 
\\	遇			
\\	グウ	あ.う	境遇(きょうぐう): 
\\	待遇(たいぐう): 
\\	愚			
\\	グ	おろ.か	愚か(おろか): 
\\	愚痴(ぐち): 
\\	愚か (おろ.か)
\\	隅			
\\	グウ	すみ	隅(すみ): 
\\	隅 (すみ)
\\	逆			
\\	ギャク、ゲキ	さか、さか.さ、さか.らう	逆転(ぎゃくてん): 
\\	逆立ち(さかだち): 
\\	逆上る(さかのぼる): 
\\	吃逆(しゃっくり): 
\\	逆(ぎゃく): 
\\	逆さ(さかさ): 
\\	逆様(さかさま): 
\\	逆らう(さからう): 
\\	逆 (さか), 逆らう (さか.らう)
\\	塑			
\\	ソ	でく		
\\	岡			
\\	丘, 
\\	コウ	おか		岡 (おか)
\\	鋼			
\\	コウ 鋼鉄 こうてつ
\\	綱1963 (ゴウ): 剛1964.	
\\	コウ	はがね	鉄鋼(てっこう): 
\\	鋼 (はがね)
\\	綱			
\\	コウ	つな	横綱(よこづな): 
\\	綱(つな): 
\\	綱 (つな)
\\	剛			
\\	ゴウ			
\\	缶			
\\	カン	かま	缶(かん): 
\\	缶詰(かんづめ): 
\\	薬缶(やかん): 
\\	陶			
\\	トウ		鬱陶しい(うっとうしい): 
\\	陶器(とうき): 
\\	揺			
\\	ヨウ	ゆ.れる、ゆ.る、ゆ.らぐ、ゆ.るぐ、ゆ.する、ゆ.さぶる、ゆ.すぶる、うご.く	動揺(どうよう): 
\\	揺さぶる(ゆさぶる): 
\\	揺らぐ(ゆらぐ): 
\\	揺れる(ゆれる): 
\\	揺さぶる (ゆ.さぶる), 揺すぶる (ゆ.すぶる), 揺する (ゆ.する), 揺らぐ (ゆ.らぐ), 揺る (ゆ.る), 揺るぐ (ゆ.るぐ), 揺れる (ゆ.れる)
\\	謡			
\\	ヨウ	うた.い、うた.う	歌謡(かよう): 
\\	民謡(みんよう): 
\\	謡う (うた.う), 謡 (うたい)
\\	就			
\\	シュウ、ジュ	つ.く、つ.ける	就業(しゅうぎょう): 
\\	就任(しゅうにん): 
\\	就職(しゅうしょく): 
\\	就く(つく): 
\\	就任 (しゅうにん), 就寝 (しゅうしん), 去就 (きょしゅう), 就く (つ.く), 就ける (つ.ける)
\\	懇			
\\	コン	ねんご.ろ		懇ろ (ねんご.ろ)
\\	墾			
\\	コン			
\\	免			
\\	兔.	
\\	メン	まぬか.れる、まぬが.れる	御免ください(ごめんください): 
\\	御免なさい(ごめんなさい): 
\\	免れる(まぬかれる): 
\\	免除(めんじょ): 
\\	御免(ごめん): 
\\	免許(めんきょ): 
\\	免税(めんぜい): 
\\	免れる (まぬか.れる)
\\	逸			
\\	イツ	そ.れる、そ.らす、はぐ.れる	逸らす(そらす): 
\\	逸れる(それる): 
\\	晩			
\\	バン		今晩(こんばん): 
\\	晩(ばん): 
\\	晩御飯(ばんごはん): 
\\	毎晩(まいばん): 
\\	勉			
\\	ベン	つと.める	勤勉(きんべん): 
\\	勉強(べんきょう): 
\\	象			
\\	ショウ、ゾウ	かたど.る	気象(きしょう): 
\\	象(しょう): 
\\	象徴(しょうちょう): 
\\	印象(いんしょう): 
\\	現象(げんしょう): 
\\	象(ぞう): 
\\	対象(たいしょう): 
\\	抽象(ちゅうしょう): 
\\	象徴 (しょうちょう), 対象 (たいしょう), 現象 (げんしょう)
\\	像			
\\	ゾウ		映像(えいぞう): 
\\	現像(げんぞう): 
\\	像(ぞう): 
\\	仏像(ぶつぞう): 
\\	想像(そうぞう): 
\\	馬			
\\	隹 
\\	バ	うま、うま-、ま	馬鹿馬鹿しい(ばかばかしい): 
\\	馬鹿らしい(ばからしい): 
\\	馬(うま): 
\\	競馬(けいば): 
\\	馬鹿(ばか): 
\\	馬 (うま)
\\	駒			
\\	ク	こま		駒 (こま)
\\	験			
\\	ケン、ゲン	あかし、しるし、ため.す、ためし	体験(たいけん): 
\\	実験(じっけん): 
\\	受験(じゅけん): 
\\	経験(けいけん): 
\\	試験(しけん): 
\\	試験 (しけん), 経験 (けいけん), 実験 (じっけん)
\\	騎			
\\	キ			
\\	駐			
\\	チュウ		駐車(ちゅうしゃ): 
\\	駐車場(ちゅうしゃじょう): 
\\	駆			
\\	ク	か.ける、か.る	駆け足(かけあし): 
\\	駆けっこ(かけっこ): 
\\	駆ける(かける): 
\\	駆ける (か.ける), 駆る (か.る)
\\	駅			
\\	エキ		駅(えき): 
\\	騒			
\\	ソウ	さわ.ぐ、うれい、さわ.がしい	騒動(そうどう): 
\\	騒がしい(さわがしい): 
\\	騒ぎ(さわぎ): 
\\	騒音(そうおん): 
\\	騒々しい(そうぞうしい): 
\\	物騒(ぶっそう): 
\\	騒ぐ(さわぐ): 
\\	騒ぐ (さわ.ぐ)
\\	駄			
\\	ダ、タ		駄作(ださく): 
\\	無駄遣い(むだづかい): 
\\	下駄(げた): 
\\	駄目(だめ): 
\\	無駄(むだ): 
\\	驚			
\\	驚く=おどろく= 
\\	敬語 
\\	""わたくしたちを殺さないでください!!お願い致します!!
\\	キョウ	おどろ.く、おどろ.かす	驚き(おどろき): 
\\	驚異(きょうい): 
\\	吃驚(びっくり): 
\\	驚かす(おどろかす): 
\\	驚く(おどろく): 
\\	驚かす (おどろ.かす), 驚く (おどろ.く)
\\	篤			
\\	トク	あつ.い		
\\	騰			
\\	トウ		沸騰(ふっとう): 
\\	虎			
\\	虍, 
\\	(皮) 
\\	コ	とら	虎(とら): 
\\	虎 (とら)
\\	虜			
\\	リョ、ロ	とりこ、とりく	捕虜(ほりょ): 
\\	膚			
\\	フ	はだ	皮膚(ひふ): 
\\	虚			
\\	キョ、コ	むな.しい、うつ.ろ	謙虚(けんきょ): 
\\	虚無 (きょむ), 虚偽 (きょぎ), 空虚 (くうきょ)
\\	戯			
\\	ギ、ゲ	たわむ.れる、ざ.れる、じゃ.れる	悪戯(いたずら): 
\\	戯曲(ぎきょく): 
\\	不山戯る(ふざける): 
\\	戯れる (たわむ.れる)
\\	虞			
\\	グ	おそれ、おもんぱか.る、はか.る、うれ.える、あざむ.く、あやま.る、のぞ.む、たの.しむ		虞 (おそれ)
\\	慮			
\\	虎.
\\	リョ	おもんぱく.る、おもんぱか.る	配慮(はいりょ): 
\\	遠慮(えんりょ): 
\\	考慮(こうりょ): 
\\	劇			
\\	ゲキ		喜劇(きげき): 
\\	劇団(げきだん): 
\\	演劇(えんげき): 
\\	劇(げき): 
\\	劇場(げきじょう): 
\\	悲劇(ひげき): 
\\	虐			
\\	二+
\\	直).
\\	ギャク	しいた.げる		虐げる (しいた.げる)
\\	鹿			
\\	ロク	しか、か	馬鹿馬鹿しい(ばかばかしい): 
\\	馬鹿らしい(ばからしい): 
\\	馬鹿(ばか): 
\\	鹿 (しかか)
\\	薦			
\\	(白) 
\\	セン	すす.める	推薦(すいせん): 
\\	薦める (すす.める)
\\	慶			
\\	ケイ	よろこ.び	慶び(よろこび): 
\\	慶ぶ(よろこぶ): 
\\	麗			
\\	レイ	うるわ.しい、うら.らか	奇麗(きれい): 
\\	麗しい (うるわ.しい)
\\	熊			
\\	ユウ	くま		熊 (くま)
\\	能			
\\	ノウ	よ.く	技能(ぎのう): 
\\	万能(ばんのう): 
\\	放射能(ほうしゃのう): 
\\	本能(ほんのう): 
\\	無能(むのう): 
\\	可能(かのう): 
\\	機能(きのう): 
\\	芸能(げいのう): 
\\	才能(さいのう): 
\\	性能(せいのう): 
\\	知能(ちのう): 
\\	能(のう): 
\\	能率(のうりつ): 
\\	能力(のうりょく): 
\\	有能(ゆうのう): 
\\	態			
\\	タイ	わざ.と	形態(けいたい): 
\\	実態(じったい): 
\\	態勢(たいせい): 
\\	態々(わざわざ): 
\\	事態(じたい): 
\\	状態(じょうたい): 
\\	態度(たいど): 
\\	寅			
\\	イン	とら		
\\	演			
\\	エン		演習(えんしゅう): 
\\	演出(えんしゅつ): 
\\	演じる(えんじる): 
\\	演ずる(えんずる): 
\\	公演(こうえん): 
\\	主演(しゅえん): 
\\	出演(しゅつえん): 
\\	上演(じょうえん): 
\\	演技(えんぎ): 
\\	演劇(えんげき): 
\\	演説(えんぜつ): 
\\	演奏(えんそう): 
\\	講演(こうえん): 
\\	辰			
\\	シン、ジン	たつ		
\\	辱			
\\	ジョク	はずかし.める	侮辱(ぶじょく): 
\\	辱める (はずかし.める)
\\	震			
\\	シン	ふる.う、ふる.える	震わせる(ふるわせる): 
\\	震える(ふるえる): 
\\	地震(じしん): 
\\	震う (ふる.う), 震える (ふる.える)
\\	振			
\\	シン	ふ.る、ぶ.る、ふ.り、-ぶ.り、ふ.るう	振興(しんこう): 
\\	振動(しんどう): 
\\	久し振り(ひさしぶり): 
\\	不振(ふしん): 
\\	振り(ふり): 
\\	振り出し(ふりだし): 
\\	身振り(みぶり): 
\\	振り仮名(ふりがな): 
\\	振る(ふる): 
\\	振舞う(ふるまう): 
\\	振る (ふ.る), 振るう (ふ.るう)
\\	娠			
\\	シン		妊娠(にんしん): 
\\	唇			
\\	シン	くちびる	唇(くちびる): 
\\	唇 (くちびる)
\\	農			
\\	ノウ		農耕(のうこう): 
\\	農場(のうじょう): 
\\	農村(のうそん): 
\\	農地(のうち): 
\\	酪農(らくのう): 
\\	農家(のうか): 
\\	農業(のうぎょう): 
\\	農産物(のうさんぶつ): 
\\	農民(のうみん): 
\\	農薬(のうやく): 
\\	濃			
\\	ノウ	こ.い	濃い(こい): 
\\	濃度(のうど): 
\\	濃い (こ.い)
\\	送			
\\	ソウ	おく.る	運送(うんそう): 
\\	回送(かいそう): 
\\	送金(そうきん): 
\\	送り仮名(おくりがな): 
\\	送る(おくる): 
\\	送別(そうべつ): 
\\	送料(そうりょう): 
\\	見送り(みおくり): 
\\	見送る(みおくる): 
\\	郵送(ゆうそう): 
\\	輸送(ゆそう): 
\\	放送(ほうそう): 
\\	送る (おく.る)
\\	関			
\\	カン	せき、-ぜき、かか.わる、からくり、かんぬき	関与(かんよ): 
\\	関税(かんぜい): 
\\	関西(かんさい): 
\\	関心(かんしん): 
\\	関する(かんする): 
\\	関東(かんとう): 
\\	関連(かんれん): 
\\	機関(きかん): 
\\	機関車(きかんしゃ): 
\\	交通機関(こうつうきかん): 
\\	税関(ぜいかん): 
\\	関係(かんけい): 
\\	玄関(げんかん): 
\\	関 (せき)
\\	咲			
\\	ショウ	さ.く、-ざき	咲く(さく): 
\\	咲く (さ.く)
\\	鬼			
\\	"鬼(おに) 
\\	キ	おに、おに-	鬼(おに): 
\\	鬼 (おに)
\\	醜			
\\	シュウ	みにく.い、しこ	醜い(みにくい): 
\\	醜い (みにく.い)
\\	魂			
\\	コン	たましい、たま	魂(こん): 
\\	魂 (たましい)
\\	魔			
\\	マ		お邪魔します(おじゃまします): 
\\	誤魔化す(ごまかす): 
\\	邪魔(じゃま): 
\\	悪魔(あくま): 
\\	魅			
\\	ミ		魅力(みりょく): 
\\	塊			
\\	カイ、ケ	かたまり、つちくれ	塊(かたまり): 
\\	塊 (かたまり)
\\	襲			
\\	龍 
\\	シュウ	おそ.う、かさ.ね	襲う(おそう): 
\\	襲撃(しゅうげき): 
\\	襲う (おそ.う)
\\	嚇			
\\	赫 
\\	カク	おど.かす		
\\	朕			
\\	チン			
\\	雰			
\\	フン		雰囲気(ふんいき): 
\\	箇			
\\	カ、コ		箇所(かしょ): 
\\	箇条書き(かじょうがき): 
\\	箇箇(ここ): 
\\	錬			
\\	レン	ね.る		
\\	遵			
\\	ジュン			
\\	罷			
\\	ヒ 罷業 ひぎょう
\\	ヒ	まか.り-、や.める		
\\	屯			
\\	トン			
\\	且			
\\	ショ、ソ、ショウ	か.つ	且つ(かつ): 
\\	且つ (か.つ)
\\	藻			
\\	ソウ	も	藻掻く(もがく): 
\\	藻 (も)
\\	隷			
\\	レイ	したが.う、しもべ		
\\	癒			
\\	愈 
\\	ユ	い.える、いや.す		
\\	丹			
\\	タン	に		
\\	潟			
\\	セキ	かた、-がた		潟 (かた)
\\	丑			
\\	チュウ	うし		
\\	卯			
\\	ボウ、モウ	う	卯(ぼう): 
\\	巳			
\\	巳
\\	シ	み		
\\	璃			
\\	リ			
\\	俺			
\\	エン	おれ、われ	俺(おれ): 
\\	俺 (おれ)
\\	臼			
\\	キュウ、グ	うす、うすづ.く		臼 (うす)
\\	毀			
\\	キ	こぼ.つ、こわ.す、こぼ.れる、こわ.れる、そし.る、やぶ.る		
\\	脊			
\\	セキ	せ、せい		
\\	璽			
\\	ジ			
\\	妖			
\\	ヨウ	あや.しい、なま.めく、わざわ.い		妖しい (あや.しい)
\\	沃			
\\	ヨウ、ヨク、オク	そそ.ぐ		
\\	稽			
\\	ケイ	かんが.える、とど.める	滑稽(こっけい): 
\\	稽古(けいこ): 
\\	采			
\\	サイ	と.る、いろどり		
\\	斬			
\\	ザン、サン、セン、ゼン	き.る	斬る(きる): 
\\	斬る (き.る)
\\	巾			
\\	キン、フク	おお.い、ちきり、きれ、はば	布巾(ふきん): 
\\	雑巾(ぞうきん): 
\\	僅			
\\	キン、ゴン	わずか	僅(きん): 
\\	僅か(わずか): 
\\	僅 (わずか)
\\	侶			
\\	リョ、ロ	とも		
\\	伎			
\\	ギ、キ	わざ、わざおぎ		
\\	凄			
\\	セイ、サイ	さむ.い、すご.い、すさ.まじい	凄い(すごい): 
\\	物凄い(ものすごい): 
\\	冶			
\\	ヤ	い.る		
\\	刹			
\\	セチ、セツ、サツ			
\\	剥・剝			
\\	ハク、ホク	へ.ぐ、へず.る、む.く、む.ける、は.がれる、は.ぐ、は.げる、は.がす	剥がす(はがす): 
\\	剥ぐ(はぐ): 
\\	剥げる(はげる): 
\\	剥す(はがす): 
\\	剥く(むく): 
\\	剥がす (は.がす), 剥ぐ (は.ぐ)
\\	匂			
\\	にお.う、にお.い、にお.わせる	匂い(におい): 
\\	匂う(におう): 
\\	匂う (にお.う)
\\	勾			
\\	コウ、ク	かぎ、ま.がる		
\\	嘲			
\\	チョウ、トウ	あざけ.る		嘲る (あざけ.る)
\\	咽			
\\	イン、エン、エツ	むせ.ぶ、むせ.る、のど、の.む		
\\	喉			
\\	コウ	のど	喉(のど): 
\\	喉 (のど)
\\	唾			
\\	つば.	
\\	ダ、タ	つば、つばき	唾(つば): 
\\	唾 (つば)
\\	呪			
\\	ジュ、シュ、シュウ、ズ	まじな.う、のろ.い、まじな.い、のろ.う		呪う (のろ.う)
\\	唄			
\\	バイ	うた、うた.う		唄 (うた)
\\	叱			
\\	シツ、シチ	しか.る	叱る(しかる): 
\\	叱る (しか.る)
\\	堆			
\\	タイ、ツイ	うずたか.い		
\\	填・塡			
\\	テン、チン	は.まる、は.める、うず.める、しず.める、ふさ.ぐ	填まる(はまる): 
\\	填める(はめる): 
\\	妬			
\\	ト、ツ	ねた.む、そね.む、つも.る、ふさ.ぐ	嫉妬(しっと): 
\\	妬む(ねたむ): 
\\	妬む (ねた.む)
\\	嫉			
\\	シツ	そね.む、ねた.む、にく.む	嫉妬(しっと): 
\\	塞			
\\	ソク、サイ	ふさ.ぐ、とりで、み.ちる	塞がる(ふさがる): 
\\	塞ぐ(ふさぐ): 
\\	塞がる (ふさ.がる), 塞ぐ (ふさ.ぐ)
\\	尻			
\\	コウ	しり	尻尾(しっぽ): 
\\	尻(しり): 
\\	尻 (しり)
\\	崖			
\\	ガイ、ゲ、ギ	がけ、きし、はて	崖(がけ): 
\\	崖 (がけ)
\\	弥			
\\	ミ、ビ	いや、や、あまねし、いよいよ、とおい、ひさし、ひさ.しい、わた.る		弥 (や)
\\	挨			
\\	アイ	ひら.く	挨拶(あいさつ): 
\\	捻			
\\	ネン、ジョウ	ね.じる、ねじ.る、ひね.くる、ひね.る	捻子(ねじ): 
\\	捻じれる(ねじれる): 
\\	捻る(ひねる): 
\\	拭			
\\	ショク、シキ	ぬぐ.う、ふ.く	手拭い(てぬぐい): 
\\	拭く(ふく): 
\\	拭う (ぬぐ.う), 拭く (ふ.く)
\\	捉			
\\	ソク、サク	とら.える		捉える (とら.える)
\\	拶			
\\	サツ	せま.る	挨拶(あいさつ): 
\\	捗			
\\	チョク、ホ	はかど.る	捗る(はかどる): 
\\	憧			
\\	ショウ、トウ、ドウ	あこが.れる	憧れ(あこがれ): 
\\	憧れる(あこがれる): 
\\	憧れる (あこが.れる)
\\	湧			
\\	ユウ、ヨウ、ユ	わ.く	湧く(わく): 
\\	湧く (わ.く)
\\	沙			
\\	サ、シャ	すな、よなげる	ご無沙汰(ごぶさた): 
\\	無沙汰(ぶさた): 
\\	淫			
\\	イン	ひた.す、ほしいまま、みだ.ら、みだ.れる、みだり		淫ら (みだ.ら)
\\	氾			
\\	ハン	ひろ.がる	氾濫(はんらん): 
\\	溺			
\\	デキ、ジョウ、ニョウ	いばり、おぼ.れる	溺れる(おぼれる): 
\\	溺れる (おぼ.れる)
\\	汰			
\\	沙汰(さた)
\\	タ、タイ	おご.る、にご.る、よな.げる	ご無沙汰(ごぶさた): 
\\	無沙汰(ぶさた): 
\\	潰			
\\	カイ、エ	つぶ.す、つぶ.れる、つい.える	潰す(つぶす): 
\\	潰れる(つぶれる): 
\\	汎			
\\	ハン、ブ、フウ、ホウ、ホン	ただよ.う、ひろ.い		
\\	釜			
\\	フ	かま	釜(かま): 
\\	釜 (かま)
\\	狙			
\\	ソ、ショ	ねら.う、ねら.い	狙い(ねらい): 
\\	狙う(ねらう): 
\\	狙う (ねら.う)
\\	萎			
\\	萎む(しぼむ)
\\	イ	な、しお.れる、しな.びる、しぼ.む、な.える	萎びる(しなびる): 
\\	萎む(しぼむ): 
\\	萎える (な.える)
\\	蔽			
\\	ヘイ、ヘツ、フツ	おお.う、おお.い		
\\	芯			
\\	シン		芯(しん): 
\\	藍			
\\	青 
\\	藍 
\\	緑. 
\\	ラン	あい	藍褸(ぼろ): 
\\	藍 (あい)
\\	苛			
\\	カ	いじ.める、さいな.む、いらだ.つ、からい、こまかい	苛める(いじめる): 
\\	蓋			
\\	ガイ、カイ、コウ	ふた、けだ.し、おお.う、かさ、かこう	蓋(がい): 
\\	目蓋(まぶた): 
\\	蓋(ふた): 
\\	蓋 (ふた)
\\	蔑			
\\	ベツ	ないがしろ、なみ.する、くらい、さげす.む	軽蔑(けいべつ): 
\\	蔑む (さげす.む)
\\	葛			
\\	カツ、カチ	つづら、くず		葛 (くず)
\\	遜			
\\	ソン	したが.う、へりくだ.る、ゆず.る	謙遜(けんそん): 
\\	隙			
\\	ゲキ、キャク、ケキ	すき、す.く、す.かす、ひま	隙間(すきま): 
\\	隙 (すき)
\\	曖			
\\	"曖昧。 
\\	アイ	くら.い	曖昧(あいまい): 
\\	昧			
\\	マイ、バイ	くら.い、むさぼ.る	曖昧(あいまい): 
\\	旺			
\\	オウ、キョウ、ゴウ	かがや.き、うつくし.い、さかん		
\\	腎			
\\	ジン			
\\	股			
\\	コ	また、もも	股(また): 
\\	股 (また)
\\	臆			
\\	オク、ヨク	むね、おくする	臆病(おくびょう): 
\\	膝			
\\	シツ	ひざ	膝(ひざ): 
\\	膝 (ひざ)
\\	肘			
\\	チュウ	ひじ	肘(ひじ): 
\\	肘 (ひじ)
\\	腺			
\\	セン			
\\	腫			
\\	シュ、ショウ	は.れる、は.れ、は.らす、く.む、はれもの	腫れる(はれる): 
\\	腫らす (は.らす), 腫れる (は.れる)
\\	膳			
\\	ゼン、セン	かしわ、すす.める、そな.える	膳(ぜん): 
\\	枕			
\\	チン、シン	まくら	枕(まくら): 
\\	枕 (まくら)
\\	椅			
\\	イ		椅子(いす): 
\\	柿			
\\	シ	かき、こけら		柿 (かき)
\\	桁			
\\	コウ	けた	桁(けた): 
\\	桁 (けた)
\\	梗			
\\	コウ、キョウ	ふさぐ、やまにれ、おおむね		
\\	椎			
\\	椎茸 
\\	ツイ、スイ	つち、う.つ		
\\	柵			
\\	サク、サン	しがら.む、しがらみ、とりで、やらい	柵(さく): 
\\	煎			
\\	セン	せん.じる、い.る、に.る	煎る(いる): 
\\	煎る (い.る)
\\	瑠			
\\	ル、リュウ			
\\	斑			
\\	ハン	ふ、まだら	斑(ぶち): 
\\	弄			
\\	ロウ、ル	いじく.る、ろう.する、いじ.る、ひねく.る、たわむ.れる、もてあそ.ぶ	弄る(いじる): 
\\	弄ぶ (もてあそ.ぶ)
\\	玩			
\\	ガン	もちあそ.ぶ、もてあそ.ぶ		
\\	畏			
\\	イ	おそ.れる、かしこ.まる、かしこ、かしこ.し	畏まりました(かしこまりました): 
\\	畏れる (おそ.れる)
\\	痩			
\\	ソウ、チュウ、シュウ、シュ	や.せる	痩せる(やせる): 
\\	痩せる (や.せる)
\\	痕			
\\	コン	あと		痕 (あと)
\\	瞭			
\\	リョウ	あきらか	明瞭(めいりょう): 
\\	眉			
\\	ビ ミ まゆ.	
\\	ビ、ミ	まゆ	眉(まゆ): 
\\	眉 (まゆ)
\\	窟			
\\	クツ、コツ	いわや、いはや、あな		
\\	裾			
\\	キョ コ すそ 
\\	キョ、コ	すそ	裾(すそ): 
\\	裾 (すそ)
\\	篭・籠			
\\	龍, 
\\	ロウ、ル	かご、こ.める、こも.る、こ.む	篭る(こもる): 
\\	箸			
\\	チョ、チャク	はし	箸(はし): 
\\	箸 (はし)
\\	綻			
\\	タン	ほころ.びる	綻びる(ほころびる): 
\\	舷			
\\	ゲン	ふなべい、ふなばた		
\\	蜂			
\\	ホウ	はち	蜂蜜(はちみつ): 
\\	蜂 (はち)
\\	罵			
\\	バ	ののし.る	罵る(ののしる): 
\\	罵る (ののし.る)
\\	戴			
\\	タイ	いただ.く		
\\	謎			
\\	メイ、ベイ	なぞ	謎(なぞ): 
\\	謎 (なぞ)
\\	誰			
\\	スイ	だれ、たれ、た	誰(たれ): 
\\	誰(だれ): 
\\	誰か(だれか): 
\\	誰 (だれ)
\\	詣			
\\	肉まん 
\\	ケイ、ゲイ	けい.する、まい.る、いた.る、もう.でる		詣でる (もう.でる)
\\	諦			
\\	あきら 
\\	諦める 
\\	あきらめる).	
\\	テイ、タイ	あき.らめる、あきら.める、つまびらか、まこと	諦め(あきらめ): 
\\	諦める(あきらめる): 
\\	諦める (あきら.める)
\\	詮			
\\	セン	せん.ずる、かい、あき.らか		
\\	貌			
\\	ボウ、バク	かたち、かたどる		
\\	貼			
\\	テン、チョウ	は.る、つ.く	貼る(はる): 
\\	貼る (は.る)
\\	賂			
\\	賄賂【わいろ】 
\\	ロ	まいな.い、まいな.う		
\\	蹴			
\\	シュク、シュウ	け.る	蹴飛ばす(けとばす): 
\\	蹴る(ける): 
\\	蹴る (け.る)
\\	酎			
\\	焼酎. 
\\	チュウ、チュ	かも.す		
\\	醒			
\\	セイ	さ.ます、さ.める		
\\	麺			
\\	メン、ベン	むぎこ		
\\	鍋			
\\	カ	なべ	鍋(なべ): 
\\	鍋 (なべ)
\\	鍵			
\\	鍵盤
\\	ケン, かぎ 
\\	金, 建 
\\	錠前 
\\	熟語: 
\\	錠前.	
\\	ケン	かぎ	鍵(かぎ): 
\\	鍵 (かぎ)
\\	闇			
\\	アン、オン	やみ、くら.い	無闇に(むやみに): 
\\	闇(やみ): 
\\	闇 (やみ)
\\	頓			
\\	トン トツ にわか.に とん.と つまず.く とみ.に ぬかずく 
\\	トン、トツ	にわか.に、とん.と、つまず.く、とみ.に、ぬかずく		
\\	頃			
\\	ケイ、キョウ	ころ、ごろ、しばら.く	頃(けい): 
\\	この頃(このごろ): 
\\	年頃(としごろ): 
\\	日頃(ひごろ): 
\\	一頃(ひところ): 
\\	頃(ころ): 
\\	近頃(ちかごろ): 
\\	手頃(てごろ): 
\\	頃 (ころ)
\\	頬・頰			
\\	キョウ	ほお、ほほ	頬(ほお): 
\\	頬っぺた(ほっぺた): 
\\	頬 (ほお)
\\	顎			
\\	ガク	あご、あぎと	顎(あご): 
\\	顎 (あご)
\\	餌			
\\	ジ、ニ	え、えば、えさ、もち	餌(えさ): 
\\	餌 (ええさ)
\\	餅			
\\	ヘイ、ヒョウ	もち、もちい	餅(もち): 
\\	餅 (もち)
\\	麓			
\\	ロク	ふもと	麓(ふもと): 
\\	麓 (ふもと)
\\	冥			
\\	メイ、ミョウ	くら.い		
\\	挫			
\\	ザ、サ	くじ.く、くじ.ける		
\\	遡			
\\	ソ、サク	さかのぼ.る	遡る(さかのぼる): 
\\	遡る (さかのぼ.る)
\\	爽			
\\	ソウ	あき.らか、さわ.やか、たがう	爽やか(さわやか): 
\\	爽やか (さわ.やか)
\\	勃			
\\	ボツ、ホツ	おこ.る、にわかに		
\\	骸			
\\	ガイ、カイ	むくろ		
\\	戚			
\\	上 
\\	小 
\\	叔, 
\\	ソク、セキ	いた.む、うれ.える、みうち	親戚(しんせき): 
\\	丼			
\\	トン、タン、ショウ、セイ	どんぶり	丼(どんぶり): 
\\	丼 (どんぶりどん)
\\	畿			
\\	キ	みやこ		
\\	拳			
\\	鉄拳 (てっけん) 
\\	ケン、ゲン	こぶし		拳 (こぶし)
\\	阜			
\\	岐阜県 ぎふけん 
\\	フ、フウ			
\\	那			
\\	ナ、ダ	なに、なんぞ、いかん	旦那(だんな): 
\\	箋・䇳			
\\	セン	ふだ	便箋(びんせん): 
\\	彙			
\\	イ はりねずみ 
\\	語彙.	
\\	イ	はりねずみ	語彙(ごい): 
\\	嗅				
\\	キュウ	か.ぐ		
\\	喩・喻				
\\	くく 
\\	(口+兪) 
\\	(口+俞)) 
\\	ユ	さと.す、 たと.える		
\\	訃				
\\	フ	しらせ		
\\	楷				
\\	カイ			
\\	諧				
\\	起立(きりつ)! 
\\	カイ	かな.う、 やわ.らぐ		
\\	錮				
\\	コ	ふさ.ぐ		
\\	恣				
\\	シ	ほしいまま		
\\	惧				
\\	ク	おそ.れる		
\\	憬				
\\	ケイ	あこが.れる		
\\	拉				
\\	ラ、 ラツ、 ロウ	くだ.く、 ひし.ぐ、 らっ.する		
\\	傲				
\\	ゴウ	あなど.る、 おご.る		
\\	踪				
\\	ショウ、 ソウ	あと		
\\	緻				
\\	チ	こまか.い		
\\	璧				
\\	(完璧) 
\\	たま 
\\	玉) 
\\	ヘキ	たま		
\\	摯				
\\	執 
\\	手, 
\\	真摯、しんし, 
\\	シ	いた.る		
\\	貪				
\\	タン、 トン	むさぼ.る		
\\	慄				
\\	リツ	おそ.れる、 おのの.く、 ふる.える		
\\	辣				
\\	ラツ	から.い		
\\	瘍				
\\	ヨウ	かさ		
\\	哺				
\\	ホ	はぐく.む、 ふく.む		
\\	鬱				
\\	ウツ	うっ.する、 しげ.る、 ふさ.ぐ		
\\	羞				
\\	シュウ	すすめ.る、 はじ.る、 は.ずかしい		
\end{CJK}
\end{document}