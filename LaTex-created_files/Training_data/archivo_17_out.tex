\documentclass[8pt]{extreport} 
\usepackage{hyperref}
\usepackage{CJKutf8}
\begin{document}
\begin{CJK}{UTF8}{min}
\\	一	イチ、イツ	ひと-、ひと.つ	かず, い, いっ, いる, かつ, かづ, てん, はじめ, ひ, ひとつ, まこと	
\\	"一人(ひとり): 
\\	一昨年(おととし): 
\\	一つ(ひとつ): 
\\	一番(いちばん): 
\\	一部(いちぶ): 
\\	一方(いっぽう): 
\\	一緒(いっしょ): 
\\	一層(いっそう): 
\\	一定(いってい): 
\\	一切(いっさい): 
\\	一生(いっしょう): 
\\	一言(ひとこと): 
\\	一帯(いったい): 
\\	一同(いちどう): 
\\	一度(いちど): 
\\	一斉(いっせい): 
\\	一昨日(おととい): 
\\	一日(いちにち): 
\\	一度に(いちどに): 
\\	一日(ついたち): 
\\	一月(ひとつき): 
\\	一: 
\\	一	
\\	いち	
\\	指事 
\\	二	ニ、ジ	ふた、ふた.つ、ふたたび	おと, つぐ, にい, は, ふ, ふたつ, ふだ, わ	
\\	二(に): 
\\	二人(ふたり): 
\\	二つ(ふたつ): 
\\	二十歳(はたち): 
\\	二階建て(にかいだて): 
\\	真っ二つ(まっぷたつ): 
\\	二日(ふつか): 
\\	二十日(はつか): 
\\	二: 
\\	二	
\\	に・ふた	
\\	指事 
\\	三	サン、ゾウ	み、み.つ、みっ.つ	か, さ, さい, さえ, さぶ, ざ, ざえ, ざぶ, そう, ぞ, ただ, みつ, みん, も, や	
\\	三(さん): 
\\	三つ(みっつ): 
\\	再三(さいさん): 
\\	三角(さんかく): 
\\	三味線(しゃみせん): 
\\	三日月(みかづき): 
\\	三日(みっか): 
\\	一: 
\\	一: 
\\	一: 
\\	一	
\\	いち	
\\	指事 
\\	四	シ	よ、よ.つ、よっ.つ、よん	あ, つ, よつ	
\\	四(し): 
\\	四つ(よっつ): 
\\	四季(しき): 
\\	四角い(しかくい): 
\\	四角(しかく): 
\\	四捨五入(ししゃごにゅう): 
\\	四つ角(よつかど): 
\\	四日(よっか): 
\\	囗: 
\\	八: 
\\	囗	
\\	くに・くにがまえ	
\\	指事 
\\	五	ゴ	いつ、いつ.つ	い, さ, さつ, ち, ふ, み, め	
\\	五(ご): 
\\	五つ(いつつ): 
\\	五十音(ごじゅうおん): 
\\	四捨五入(ししゃごにゅう): 
\\	五日(いつか): 
\\	五味(ごみ): 
\\	二: 
\\	二	
\\	に・ふた	
\\	仮借 
\\	六	ロク、リク	む、む.つ、むっ.つ、むい	く, むつ, ろっ, ろつ	
\\	六(ろく): 
\\	六日(むいか): 
\\	六つ(むっつ): 
\\	亠: 
\\	八: 
\\	八	
\\	はち・は・はちがしら	
\\	仮借 
\\	七	シチ	なな、なな.つ、なの	し, しっ, な, ひち	
\\	七(しち): 
\\	七つ(ななつ): 
\\	七日(なのか): 
\\	一: 
\\	乙: 
\\	一	
\\	いち	
\\	仮借 
\\	八	ハチ	や、や.つ、やっ.つ、よう	な, は, はっ, はつ, やち, やつ	
\\	八(はち): 
\\	八百屋(やおや): 
\\	八日(ようか): 
\\	八つ(やっつ): 
\\	お八つ(おやつ): 
\\	八: 
\\	八	
\\	はち・は・はちがしら	
\\	指事 
\\	九	キュウ、ク	ここの、ここの.つ	いちじく, いちのく, この, ひさし	
\\	九(きゅう): 
\\	九つ(ここのつ): 
\\	九日(ここのか): 
\\	丿: 
\\	乙: 
\\	乙	
\\	おつ・おつにょう・つりばり	
\\	象形 
\\	十	ジュウ、ジッ、ジュッ	とお、と	い, か, ぎ, さ, し, そ, そう, ち, とう, ね, ま, る, わ	
\\	十(じゅう): 
\\	十分(じゅうぶん): 
\\	十日(とおか): 
\\	二十歳(はたち): 
\\	五十音(ごじゅうおん): 
\\	十字路(じゅうじろ): 
\\	二十日(はつか): 
\\	十分(じっぷん): 
\\	十: 
\\	十	
\\	じゅう・じゅうへん	
\\	指事 
\\	百	ヒャク、ビャク	もも	お, ど, どう, なり, ひゃっ, ひゅく, も, もんど, ゆ	
\\	百科辞典(ひゃっかじてん): 
\\	百(ひゃく): 
\\	八百屋(やおや): 
\\	一: 
\\	白: 
\\	白	
\\	しろ・しろへん	
\\	指事 
\\	千	セン	ち	かず, ゆき	
\\	千(せん): 
\\	千切る(ちぎる): 
\\	丿: 
\\	十: 
\\	十	
\\	じゅう・じゅうへん	
\\	形声 
\\	上	ジョウ、ショウ、シャン	うえ、-うえ、うわ-、かみ、あ.げる、-あ.げる、あ.がる、-あ.がる、あ.がり、-あ.がり、のぼ.る、のぼ.り、のぼ.せる、のぼ.す、よ.す	あおい, あげ, い, か, かき, かず, かん, こう, のぼり, ほつ	
\\	"上がる(あがる): 
\\	売上げ(うりあげ): 
\\	上る(のぼる): 
\\	引上げる(ひきあげる): 
\\	上昇(じょうしょう): 
\\	向上(こうじょう): 
\\	上下(じょうげ): 
\\	上空(じょうくう): 
\\	上司(じょうし): 
\\	上演(じょうえん): 
\\	上がり(あがり): 
\\	上陸(じょうりく): 
\\	上旬(じょうじゅん): 
\\	屋上(おくじょう): 
\\	上り(のぼり): 
\\	上級(じょうきゅう): 
\\	以上(いじょう): 
\\	上手(じょうず): 
\\	仕上げ(しあげ): 
\\	頂上(ちょうじょう): 
\\	途上(とじょう): 
\\	申し上げる(もうしあげる): 
\\	上着(うわぎ): 
\\	出来上がり(できあがり): 
\\	上位(じょうい): 
\\	取上げる(とりあげる): 
\\	仕上がり(しあがり): 
\\	盛り上がる(もりあがる): 
\\	上達(じょうたつ): 
\\	身の上(みのうえ): 
\\	お手上げ(おてあげ): 
\\	差し上げる(さしあげる): 
\\	目上(めうえ): 
\\	真上(まうえ): 
\\	読み上げる(よみあげる): 
\\	見上げる(みあげる): 
\\	出来上がる(できあがる): 
\\	仕上げる(しあげる): 
\\	立ち上がる(たちあがる): 
\\	上等(じょうとう): 
\\	召し上がる(めしあがる): 
\\	上品(じょうひん): 
\\	その上(そのうえ): 
\\	上(うえ): 
\\	上(かみ): 
\\	上(じょう): 
\\	上下(うえした): 
\\	持ち上げる(もちあげる): 
\\	上回る(うわまわる): 
\\	仕上がる(しあがる): 
\\	参上(さんじょう): 
\\	上(うわ): 
\\	上手(うわて): 
\\	卜: 
\\	一: 
\\	一	
\\	いち	
\\	指事 
\\	下	カ、ゲ	した、しも、もと、さ.げる、さ.がる、くだ.る、くだ.り、くだ.す、-くだ.す、くだ.さる、お.ろす、お.りる	さか, しと	
\\	"足下(あしもと): 
\\	下りる(おりる): 
\\	引下げる(ひきさげる): 
\\	下車(げしゃ): 
\\	地下(ちか): 
\\	低下(ていか): 
\\	以下(いか): 
\\	下旬(げじゅん): 
\\	上下(じょうげ): 
\\	地下鉄(ちかてつ): 
\\	下げる(さげる): 
\\	下り(くだり): 
\\	部下(ぶか): 
\\	下町(したまち): 
\\	廊下(ろうか): 
\\	下着(したぎ): 
\\	目下(めした): 
\\	天下(てんか): 
\\	下降(かこう): 
\\	下手(へた): 
\\	下痢(げり): 
\\	下る(くだる): 
\\	下宿(げしゅく): 
\\	下水(げすい): 
\\	落下(らっか): 
\\	下さる(くださる): 
\\	靴下(くつした): 
\\	下がる(さがる): 
\\	下さい(ください): 
\\	地下水(ちかすい): 
\\	下地(したじ): 
\\	下駄(げた): 
\\	下火(したび): 
\\	下品(げひん): 
\\	下ろす(おろす): 
\\	下書き(したがき): 
\\	見下ろす(みおろす): 
\\	下調べ(したしらべ): 
\\	ぶら下げる(ぶらさげる): 
\\	下(した): 
\\	下らない(くだらない): 
\\	上下(うえした): 
\\	目下(もっか): 
\\	真下(ました): 
\\	下取り(したどり): 
\\	下心(したごころ): 
\\	城下(じょうか): 
\\	下線(かせん): 
\\	下品(かひん): 
\\	下(げ): 
\\	下(しも): 
\\	下(か): 
\\	お掛け下さい(おかけください): 
\\	一: 
\\	卜: 
\\	一	
\\	いち	
\\	指事 
\\	左	サ、シャ	ひだり	そ	
\\	左(ひだり): 
\\	左右(さゆう): 
\\	左利き(ひだりきき): 
\\	左様なら(さようなら): 
\\	工: 
\\	工	
\\	こう・たくみ・たくみへん・え	
\\	会意 
\\	右	ウ、ユウ	みぎ	あき, すけ	
\\	右(みぎ): 
\\	左右(さゆう): 
\\	口: 
\\	口	
\\	くち・くちへん	
\\	会意 
\\	中	チュウ	なか、うち、あた.る	あたる, かなえ	
\\	中央(ちゅうおう): 
\\	中心(ちゅうしん): 
\\	中間(ちゅうかん): 
\\	途中(とちゅう): 
\\	集中(しゅうちゅう): 
\\	中学(ちゅうがく): 
\\	中止(ちゅうし): 
\\	中学校(ちゅうがっこう): 
\\	日中(にっちゅう): 
\\	中身(なかみ): 
\\	中継(ちゅうけい): 
\\	中旬(ちゅうじゅん): 
\\	世の中(よのなか): 
\\	中立(ちゅうりつ): 
\\	空中(くうちゅう): 
\\	背中(せなか): 
\\	中年(ちゅうねん): 
\\	真ん中(まんなか): 
\\	中世(ちゅうせい): 
\\	中古(ちゅうこ): 
\\	中断(ちゅうだん): 
\\	中途(ちゅうと): 
\\	中毒(ちゅうどく): 
\\	連中(れんちゅう): 
\\	夜中(よなか): 
\\	心中(しんじゅう): 
\\	年中(ねんじゅう): 
\\	中傷(ちゅうしょう): 
\\	命中(めいちゅう): 
\\	中枢(ちゅうすう): 
\\	中性(ちゅうせい): 
\\	中和(ちゅうわ): 
\\	中腹(ちゅうふく): 
\\	中指(なかゆび): 
\\	夢中(むちゅう): 
\\	熱中(ねっちゅう): 
\\	最中(さいちゅう): 
\\	御中(おんちゅう): 
\\	中(なか): 
\\	中中(なかなか): 
\\	中(ちゅう): 
\\	話し中(はなしちゅう): 
\\	中程(なかほど): 
\\	中っ腹(ちゅうっぱら): 
\\	夜中(やちゅう): 
\\	中(じゅう): 
\\	口: 
\\	丨: 
\\	丨	
\\	ぼう・たてぼう	
\\	象形 
\\	大	ダイ、タイ	おお-、おお.きい、-おお.いに	うふ, お, おう, た, たかし, とも, はじめ, ひろ, ひろし, まさ, まさる, もと, わ	
\\	大分(だいぶ): 
\\	拡大(かくだい): 
\\	大会(たいかい): 
\\	大学(だいがく): 
\\	大統領(だいとうりょう): 
\\	大幅(おおはば): 
\\	大使(たいし): 
\\	大切(たいせつ): 
\\	大変(たいへん): 
\\	重大(じゅうだい): 
\\	大使館(たいしかん): 
\\	大臣(だいじん): 
\\	大人(おとな): 
\\	大半(たいはん): 
\\	巨大(きょだい): 
\\	大学院(だいがくいん): 
\\	大勢(おおぜい): 
\\	大戦(たいせん): 
\\	大気(たいき): 
\\	大衆(たいしゅう): 
\\	大陸(たいりく): 
\\	短大(たんだい): 
\\	大いに(おおいに): 
\\	大丈夫(だいじょうぶ): 
\\	大胆(だいたん): 
\\	膨大(ぼうだい): 
\\	大工(だいく): 
\\	大きい(おおきい): 
\\	増大(ぞうだい): 
\\	大好き(だいすき): 
\\	大事(だいじ): 
\\	大部分(だいぶぶん): 
\\	偉大(いだい): 
\\	総理大臣(そうりだいじん): 
\\	大小(だいしょう): 
\\	大方(おおかた): 
\\	大体(だいたい): 
\\	大金(たいきん): 
\\	盛大(せいだい): 
\\	大家(たいか): 
\\	大げさ(おおげさ): 
\\	大通り(おおどおり): 
\\	大抵(たいてい): 
\\	大きな(おおきな): 
\\	大筋(おおすじ): 
\\	壮大(そうだい): 
\\	大木(たいぼく): 
\\	大ざっぱ(おおざっぱ): 
\\	大空(おおぞら): 
\\	大水(おおみず): 
\\	大屋(おおや): 
\\	お大事に(おだいじに): 
\\	大した(たいした): 
\\	大して(たいして): 
\\	大概(たいがい): 
\\	大学生(だいがくせい): 
\\	大人しい(おとなしい): 
\\	大層(たいそう): 
\\	大便(だいべん): 
\\	莫大(ばくだい): 
\\	大部(たいぶ): 
\\	大柄(おおがら): 
\\	大よそ(おおよそ): 
\\	大(おお): 
\\	大(だい): 
\\	大: 
\\	大	
\\	だい・おおきい	
\\	象形 
\\	小	ショウ	ちい.さい、こ-、お-、さ-	いさら, こう, さざ, しゃお, ちいさ	
\\	小売り(こうり): 
\\	小包み(こづつみ): 
\\	小学校(しょうがっこう): 
\\	小説(しょうせつ): 
\\	小学生(しょうがくせい): 
\\	縮小(しゅくしょう): 
\\	小屋(こや): 
\\	小切手(こぎって): 
\\	小麦(こむぎ): 
\\	大小(だいしょう): 
\\	小遣い(こづかい): 
\\	小児科(しょうにか): 
\\	小鳥(ことり): 
\\	小指(こゆび): 
\\	小さい(ちいさい): 
\\	小柄(こがら): 
\\	小便(しょうべん): 
\\	小数(しょうすう): 
\\	小銭(こぜに): 
\\	小(しょう): 
\\	小さな(ちいさな): 
\\	小(こ): 
\\	小母さん(おばさん): 
\\	小: 
\\	小	
\\	しょう・ちいさい・なおがしら・つ	
\\	象形 
\\	月	ゲツ、ガツ	つき	おと, がっ, す, ずき, もり	
\\	年月(ねんげつ): 
\\	今月(こんげつ): 
\\	先月(せんげつ): 
\\	来月(らいげつ): 
\\	毎月(まいつき): 
\\	正月(しょうがつ): 
\\	月曜(げつよう): 
\\	月給(げっきゅう): 
\\	月末(げつまつ): 
\\	カ月(かげつ): 
\\	生年月日(せいねんがっぴ): 
\\	月並(つきなみ): 
\\	月日(つきひ): 
\\	満月(まんげつ): 
\\	月謝(げっしゃ): 
\\	三日月(みかづき): 
\\	月(つき): 
\\	月賦(げっぷ): 
\\	月曜日(げつようび): 
\\	再来月(さらいげつ): 
\\	月日(がっぴ): 
\\	一月(ひとつき): 
\\	先々月(せんせんげつ): 
\\	月(がつ): 
\\	月: 
\\	月	
\\	つき・つきへん	
\\	象形 
\\	日	ニチ、ジツ	ひ、-び、-か	あ, あき, いる, く, くさ, こう, す, たち, に, にっ, につ, へ	
\\	明日(あした): 
\\	今日(きょう): 
\\	日付け(ひづけ): 
\\	昨日(きのう): 
\\	毎日(まいにち): 
\\	日程(にってい): 
\\	日(ひ): 
\\	日常(にちじょう): 
\\	来日(らいにち): 
\\	日中(にっちゅう): 
\\	曜日(ようび): 
\\	日々(ひび): 
\\	当日(とうじつ): 
\\	日曜(にちよう): 
\\	連日(れんじつ): 
\\	先日(せんじつ): 
\\	日記(にっき): 
\\	元日(がんじつ): 
\\	日の丸(ひのまる): 
\\	日ごろ(ひごろ): 
\\	期日(きじつ): 
\\	平日(へいじつ): 
\\	祝日(しゅくじつ): 
\\	日時(にちじ): 
\\	日光(にっこう): 
\\	日差し(ひざし): 
\\	火曜日(かようび): 
\\	日の出(ひので): 
\\	日帰り(ひがえり): 
\\	日向(ひなた): 
\\	日課(にっか): 
\\	日用品(にちようひん): 
\\	生年月日(せいねんがっぴ): 
\\	日焼け(ひやけ): 
\\	日陰(ひかげ): 
\\	日日(ひにち): 
\\	夕日(ゆうひ): 
\\	十日(とおか): 
\\	日本語(にほんご): 
\\	日夜(にちや): 
\\	終日(しゅうじつ): 
\\	一昨日(おととい): 
\\	日取り(ひどり): 
\\	月日(つきひ): 
\\	日当たり(ひあたり): 
\\	日本式(にほんしき): 
\\	三日月(みかづき): 
\\	明後日(あさって): 
\\	定休日(ていきゅうび): 
\\	金曜日(きんようび): 
\\	九日(ここのか): 
\\	月曜日(げつようび): 
\\	五日(いつか): 
\\	今日は(こんにちは): 
\\	祭日(さいじつ): 
\\	三日(みっか): 
\\	四日(よっか): 
\\	水曜日(すいようび): 
\\	誕生日(たんじょうび): 
\\	土曜日(どようび): 
\\	二日(ふつか): 
\\	日の入り(ひのいり): 
\\	日曜日(にちようび): 
\\	八日(ようか): 
\\	木曜日(もくようび): 
\\	六日(むいか): 
\\	一日(いちにち): 
\\	七日(なのか): 
\\	明明後日(しあさって): 
\\	二十日(はつか): 
\\	月日(がっぴ): 
\\	一日(ついたち): 
\\	日本(にほん): 
\\	西日(にしび): 
\\	日当(にっとう): 
\\	悪日(あくにち): 
\\	一昨昨日(いっさくさくじつ): 
\\	日(にち): 
\\	日(か): 
\\	日: 
\\	日	
\\	ひ・ひへん・にち・にちへん	
\\	象形 
\\	年	ネン	とし	ね	
\\	一昨年(おととし): 
\\	毎年(まいねん): 
\\	年月(ねんげつ): 
\\	年間(ねんかん): 
\\	来年(らいねん): 
\\	今年(ことし): 
\\	少年(しょうねん): 
\\	年(とし): 
\\	年齢(ねんれい): 
\\	青年(せいねん): 
\\	年度(ねんど): 
\\	年寄り(としより): 
\\	定年(ていねん): 
\\	年代(ねんだい): 
\\	中年(ちゅうねん): 
\\	学年(がくねん): 
\\	去年(きょねん): 
\\	晩年(ばんねん): 
\\	年中(ねんじゅう): 
\\	成年(せいねん): 
\\	万年筆(まんねんひつ): 
\\	生年月日(せいねんがっぴ): 
\\	年賀(ねんが): 
\\	年生(ねんせい): 
\\	青少年(せいしょうねん): 
\\	元年(がんねん): 
\\	年号(ねんごう): 
\\	再来年(さらいねん): 
\\	年輪(ねんりん): 
\\	年長(ねんちょう): 
\\	年頃(としごろ): 
\\	年鑑(ねんかん): 
\\	同い年(おないどし): 
\\	年(ねん): 
\\	丿: 
\\	干: 
\\	干	
\\	かん・いちじゅう・ほす・ひる・たてかん	
\\	会意 
\\	早	ソウ、サッ	はや.い、はや、はや-、はや.まる、はや.める、さ-	さか, さわ, そ, わ	
\\	早い(はやい): 
\\	早口(はやくち): 
\\	早急(さっきゅう): 
\\	早速(さっそく): 
\\	素早い(すばやい): 
\\	お早うございます(おはようございます): 
\\	最早(もはや): 
\\	お早う(おはよう): 
\\	日: 
\\	十: 
\\	日	
\\	ひ・ひへん・にち・にちへん	
\\	仮借 
\\	木	ボク、モク	き、こ-	ぐ, も, もと	
\\	木(き): 
\\	木材(もくざい): 
\\	樹木(じゅもく): 
\\	木曜(もくよう): 
\\	並木(なみき): 
\\	植木(うえき): 
\\	材木(ざいもく): 
\\	木綿(もめん): 
\\	土木(どぼく): 
\\	大木(たいぼく): 
\\	木曜日(もくようび): 
\\	雑木(ざつぼく): 
\\	木: 
\\	木	
\\	き・きへん	
\\	象形 
\\	林	リン	はやし	し	
\\	林(はやし): 
\\	森林(しんりん): 
\\	山林(さんりん): 
\\	林業(りんぎょう): 
\\	木: 
\\	木: 
\\	木	
\\	き・きへん	
\\	会意 
\\	山	サン、セン	やま	さ, やの, やん	
\\	山(やま): 
\\	火山(かざん): 
\\	登山(とざん): 
\\	鉱山(こうざん): 
\\	山林(さんりん): 
\\	山脈(さんみゃく): 
\\	山岳(さんがく): 
\\	沢山(たくさん): 
\\	巫山戯る(ふざける): 
\\	山腹(さんぷく): 
\\	山(さん): 
\\	山: 
\\	山	
\\	やま・やまへん・やまかんむり	
\\	象形 
\\	川	セン	かわ	か, こ, さわ	
\\	川(かわ): 
\\	河川(かせん): 
\\	巛: 
\\	巛	
\\	かわ・まがりがわ・さんぼんがわ	
\\	象形 
\\	土	ド、ト	つち	つ, つく, は, ひじ	
\\	土地(とち): 
\\	領土(りょうど): 
\\	土(つち): 
\\	土曜(どよう): 
\\	国土(こくど): 
\\	土産(みやげ): 
\\	土台(どだい): 
\\	土手(どて): 
\\	土俵(どひょう): 
\\	土木(どぼく): 
\\	風土(ふうど): 
\\	郷土(きょうど): 
\\	お土産(おみやげ): 
\\	混凝土(コンクリート): 
\\	土曜日(どようび): 
\\	率土(そっと): 
\\	土産(どさん): 
\\	土(ど): 
\\	土: 
\\	土	
\\	つち・つちへん・どへん	
\\	象形 
\\	空	クウ	そら、あ.く、あ.き、あ.ける、から、す.く、す.かす、むな.しい	うつ, き, く	
\\	空港(くうこう): 
\\	航空(こうくう): 
\\	空(から): 
\\	空気(くうき): 
\\	空間(くうかん): 
\\	上空(じょうくう): 
\\	架空(かくう): 
\\	空中(くうちゅう): 
\\	空き(あき): 
\\	空想(くうそう): 
\\	空っぽ(からっぽ): 
\\	空腹(くうふく): 
\\	真空(しんくう): 
\\	大空(おおぞら): 
\\	空(そら): 
\\	空しい(むなしい): 
\\	空く(すく): 
\\	空き間(あきま): 
\\	空(くう): 
\\	穴: 
\\	工: 
\\	穴	
\\	あな・あなかんむり	
\\	形声 
\\	田	デン	た	いなか, おか, たん, で, とう, や	
\\	水田(すいでん): 
\\	田舎(いなか): 
\\	田園(でんえん): 
\\	田植え(たうえ): 
\\	田んぼ(たんぼ): 
\\	田(た): 
\\	田: 
\\	田	
\\	た・たへん	
\\	象形 
\\	天	テン	あまつ、あめ、あま-	あき, あも, た, たかし, て, なま	
\\	天皇(てんのう): 
\\	天(てん): 
\\	天気(てんき): 
\\	天井(てんじょう): 
\\	天然(てんねん): 
\\	天候(てんこう): 
\\	天国(てんごく): 
\\	天才(てんさい): 
\\	天下(てんか): 
\\	天体(てんたい): 
\\	雨天(うてん): 
\\	晴天(せいてん): 
\\	天災(てんさい): 
\\	天地(てんち): 
\\	先天的(せんてんてき): 
\\	天気予報(てんきよほう): 
\\	天辺(てっぺん): 
\\	天津(あまつ): 
\\	一: 
\\	大: 
\\	大	
\\	だい・おおきい	
\\	象形 
\\	生	セイ、ショウ	い.きる、い.かす、い.ける、う.まれる、うま.れる、う.まれ、うまれ、う.む、お.う、は.える、は.やす、き、なま、なま-、な.る、な.す、む.す、-う	あさ, いき, いく, いけ, うぶ, うまい, え, おい, ぎゅう, くるみ, ごせ, さ, じょう, すぎ, そ, そう, ちる, なば, にう, にゅう, ふ, み, もう, よい, りゅう	
\\	出生(しゅっせい): 
\\	学生(がくせい): 
\\	生活(せいかつ): 
\\	生産(せいさん): 
\\	人生(じんせい): 
\\	生徒(せいと): 
\\	先生(せんせい): 
\\	高校生(こうこうせい): 
\\	生命(せいめい): 
\\	誕生(たんじょう): 
\\	発生(はっせい): 
\\	再生(さいせい): 
\\	小学生(しょうがくせい): 
\\	生きる(いきる): 
\\	生物(せいぶつ): 
\\	生涯(しょうがい): 
\\	留学生(りゅうがくせい): 
\\	一生懸命(いっしょうけんめい): 
\\	衛生(えいせい): 
\\	一生(いっしょう): 
\\	生存(せいぞん): 
\\	生る(なる): 
\\	生き生き(いきいき): 
\\	生地(きじ): 
\\	生かす(いかす): 
\\	生き物(いきもの): 
\\	芝生(しばふ): 
\\	生死(せいし): 
\\	生理(せいり): 
\\	生計(せいけい): 
\\	生長(せいちょう): 
\\	生年月日(せいねんがっぴ): 
\\	生花(いけばな): 
\\	生む(うむ): 
\\	年生(ねんせい): 
\\	野生(やせい): 
\\	写生(しゃせい): 
\\	生育(せいいく): 
\\	生まれつき(うまれつき): 
\\	生じる(しょうじる): 
\\	生ぬるい(なまぬるい): 
\\	生臭い(なまぐさい): 
\\	生意気(なまいき): 
\\	生やす(はやす): 
\\	生(なま): 
\\	生える(はえる): 
\\	生き甲斐(いきがい): 
\\	生ずる(しょうずる): 
\\	生まれる(うまれる): 
\\	生憎(あいにく): 
\\	大学生(だいがくせい): 
\\	誕生日(たんじょうび): 
\\	畜生(ちくしょう): 
\\	生まれ(うまれ): 
\\	生身(なまみ): 
\\	生ける(いける): 
\\	新入生(しんにゅうせい): 
\\	生(き): 
\\	生真面目(きまじめ): 
\\	生(せい): 
\\	生: 
\\	生	
\\	うむ・うまれる・いきる	
\\	象形 
\\	花	カ、ケ	はな	わ	
\\	花(はな): 
\\	花火(はなび): 
\\	花見(はなみ): 
\\	花嫁(はなよめ): 
\\	火花(ひばな): 
\\	花弁(はなびら): 
\\	生花(いけばな): 
\\	花粉(かふん): 
\\	花壇(かだん): 
\\	花瓶(かびん): 
\\	艸: 
\\	化: 
\\	艸	
\\	くさ・くさかんむり	
\\	形声 
\\	草	ソウ	くさ、くさ-、-ぐさ	そ, や	
\\	草(くさ): 
\\	煙草(タバコ): 
\\	草履(ぞうり): 
\\	草臥れる(くたびれる): 
\\	艸: 
\\	早: 
\\	艸	
\\	くさ・くさかんむり	
\\	形声 
\\	虫	チュウ、キ	むし	む	
\\	虫(むし): 
\\	昆虫(こんちゅう): 
\\	虫歯(むしば): 
\\	中: 
\\	虫	
\\	むし・むしへん	蟲	会意 
\\	犬	ケン	いぬ、いぬ-		
\\	犬(いぬ): 
\\	大: 
\\	丶: 
\\	犬	
\\	いぬ・けものへん	
\\	象形 
\\	人	ジン、ニン	ひと、-り、-と	じ, と, ね, ひこ, ふみ	
\\	一人(ひとり): 
\\	万人(ばんじん): 
\\	外国人(がいこくじん): 
\\	個人(こじん): 
\\	人間(にんげん): 
\\	人気(にんき): 
\\	人口(じんこう): 
\\	二人(ふたり): 
\\	新人(しんじん): 
\\	人事(じんじ): 
\\	人生(じんせい): 
\\	人物(じんぶつ): 
\\	人民(じんみん): 
\\	夫人(ふじん): 
\\	本人(ほんにん): 
\\	友人(ゆうじん): 
\\	老人(ろうじん): 
\\	殺人(さつじん): 
\\	大人(おとな): 
\\	人材(じんざい): 
\\	人類(じんるい): 
\\	知人(ちじん): 
\\	婦人(ふじん): 
\\	名人(めいじん): 
\\	人形(にんぎょう): 
\\	人工(じんこう): 
\\	人種(じんしゅ): 
\\	犯人(はんにん): 
\\	他人(たにん): 
\\	詩人(しじん): 
\\	人質(ひとじち): 
\\	成人(せいじん): 
\\	恋人(こいびと): 
\\	役人(やくにん): 
\\	職人(しょくにん): 
\\	人柄(ひとがら): 
\\	素人(しろうと): 
\\	人格(じんかく): 
\\	人体(じんたい): 
\\	美人(びじん): 
\\	産婦人科(さんふじんか): 
\\	人情(にんじょう): 
\\	商人(しょうにん): 
\\	人目(ひとめ): 
\\	当人(とうにん): 
\\	仲人(なこうど): 
\\	人影(ひとかげ): 
\\	証人(しょうにん): 
\\	主人公(しゅじんこう): 
\\	故人(こじん): 
\\	人ごみ(ひとごみ): 
\\	人通り(ひとどおり): 
\\	人命(じんめい): 
\\	玄人(くろうと): 
\\	使用人(しようにん): 
\\	人文科学(じんぶんかがく): 
\\	人造(じんぞう): 
\\	一人一人(ひとりひとり): 
\\	ご主人(ごしゅじん): 
\\	人(じん): 
\\	大人しい(おとなしい): 
\\	人(ひと): 
\\	人差し指(ひとさしゆび): 
\\	人(にん): 
\\	一人子(ひとりご): 
\\	女の人(おんなのひと): 
\\	人気(ひとけ): 
\\	男の人(おとこのひと): 
\\	人: 
\\	人	
\\	ひと・にんべん・ひとやね	
\\	象形 
\\	名	メイ、ミョウ	な、-な	と	
\\	名(な): 
\\	名前(なまえ): 
\\	名誉(めいよ): 
\\	氏名(しめい): 
\\	有名(ゆうめい): 
\\	署名(しょめい): 
\\	本名(ほんみょう): 
\\	名人(めいじん): 
\\	名簿(めいぼ): 
\\	名称(めいしょう): 
\\	著名(ちょめい): 
\\	名作(めいさく): 
\\	名刺(めいし): 
\\	名所(めいしょ): 
\\	名物(めいぶつ): 
\\	題名(だいめい): 
\\	地名(ちめい): 
\\	代名詞(だいめいし): 
\\	名残(なごり): 
\\	名高い(なだかい): 
\\	片仮名(カタカナ): 
\\	名付ける(なづける): 
\\	名札(なふだ): 
\\	姓名(せいめい): 
\\	平仮名(ひらがな): 
\\	名詞(めいし): 
\\	仮名遣い(かなづかい): 
\\	振り仮名(ふりがな): 
\\	送り仮名(おくりがな): 
\\	宛名(あてな): 
\\	仮名(かな): 
\\	記名(きめい): 
\\	名産(めいさん): 
\\	御名(みな): 
\\	名(めい): 
\\	夕: 
\\	口: 
\\	口	
\\	くち・くちへん	
\\	会意 
\\	女	ジョ、ニョ、ニョウ	おんな、め	おな, た, つき, な	
\\	"女のコ(おんなのコ): 
\\	女子(じょし): 
\\	女性(じょせい): 
\\	女(おんな): 
\\	彼女(かのじょ): 
\\	女優(じょゆう): 
\\	長女(ちょうじょ): 
\\	少女(しょうじょ): 
\\	女王(じょおう): 
\\	女房(にょうぼう): 
\\	王女(おうじょ): 
\\	女史(じょし): 
\\	女(じょ): 
\\	女の人(おんなのひと): 
\\	女: 
\\	女	
\\	おんな・おんなへん	
\\	象形 
\\	男	ダン、ナン	おとこ、お	み	
\\	男のコ(おとこのコ): 
\\	男(おとこ): 
\\	男子(だんし): 
\\	男性(だんせい): 
\\	長男(ちょうなん): 
\\	男の人(おとこのひと): 
\\	田: 
\\	力: 
\\	田	
\\	た・たへん	
\\	会意 
\\	子	シ、ス、ツ	こ、-こ、ね	い, き, ぎ, く, け, ねっ	
\\	子供(こども): 
\\	子(こ): 
\\	女子(じょし): 
\\	男子(だんし): 
\\	様子(ようす): 
\\	息子(むすこ): 
\\	電子(でんし): 
\\	調子(ちょうし): 
\\	王子(おうじ): 
\\	分子(ぶんし): 
\\	帽子(ぼうし): 
\\	利子(りし): 
\\	弟子(でし): 
\\	子孫(しそん): 
\\	原子(げんし): 
\\	双子(ふたご): 
\\	椅子(いす): 
\\	菓子(かし): 
\\	障子(しょうじ): 
\\	末っ子(すえっこ): 
\\	扇子(せんす): 
\\	子息(しそく): 
\\	迷子(まいご): 
\\	お菓子(おかし): 
\\	硝子(ガラス): 
\\	梯子(はしご): 
\\	判子(はんこ): 
\\	一人子(ひとりご): 
\\	捻子(ねじ): 
\\	お子さん(おこさん): 
\\	稲子(いなご): 
\\	子: 
\\	子	
\\	こ・こへん	
\\	象形 
\\	目	モク、ボク	め、-め、ま-	さかん, さがん, さっか, さつか	
\\	課目(かもく): 
\\	面目(めんもく): 
\\	目(め): 
\\	利き目(ききめ): 
\\	目的(もくてき): 
\\	目標(もくひょう): 
\\	注目(ちゅうもく): 
\\	項目(こうもく): 
\\	目安(めやす): 
\\	役目(やくめ): 
\\	目下(めした): 
\\	目録(もくろく): 
\\	人目(ひとめ): 
\\	一目(ひとめ): 
\\	駄目(だめ): 
\\	着目(ちゃくもく): 
\\	目印(めじるし): 
\\	切れ目(きれめ): 
\\	目次(もくじ): 
\\	目覚める(めざめる): 
\\	目上(めうえ): 
\\	目盛り(めもり): 
\\	真面目(まじめ): 
\\	目覚まし(めざまし): 
\\	目立つ(めだつ): 
\\	丁目(ちょうめ): 
\\	目覚ましい(めざましい): 
\\	目指す(めざす): 
\\	目付き(めつき): 
\\	お目出度う(おめでとう): 
\\	出鱈目(でたらめ): 
\\	番目(ばんめ): 
\\	目茶苦茶(めちゃくちゃ): 
\\	目下(もっか): 
\\	目処(めど): 
\\	目方(めかた): 
\\	目眩(めまい): 
\\	目出度い(めでたい): 
\\	お目出度うご座います(おめでとうございます): 
\\	継ぎ目(つぎめ): 
\\	生真面目(きまじめ): 
\\	お目にかかる(おめにかかる): 
\\	目論見(もくろみ): 
\\	お目出度い(おめでたい): 
\\	時間目(じかんめ): 
\\	目: 
\\	目	
\\	め・めへん・よこめ	
\\	象形 
\\	耳	ジ	みみ	がみ	
\\	耳(みみ): 
\\	耳鼻科(じびか): 
\\	初耳(はつみみ): 
\\	耳: 
\\	耳	
\\	みみ・みみへん	
\\	象形 
\\	口	コウ、ク	くち		
\\	入口(いりぐち): 
\\	悪口(わるくち): 
\\	出入口(でいりぐち): 
\\	人口(じんこう): 
\\	口(くち): 
\\	窓口(まどぐち): 
\\	出口(でぐち): 
\\	口実(こうじつ): 
\\	口紅(くちべに): 
\\	早口(はやくち): 
\\	蛇口(じゃぐち): 
\\	裏口(うらぐち): 
\\	口頭(こうとう): 
\\	無口(むくち): 
\\	閉口(へいこう): 
\\	利口(りこう): 
\\	口述(こうじゅつ): 
\\	甘口(あまくち): 
\\	口ずさむ(くちずさむ): 
\\	火口(かこう): 
\\	口: 
\\	口	
\\	くち・くちへん	
\\	象形 
\\	手	シュ、ズ	て、て-、-て、た-		
\\	手続(てつづき): 
\\	手当て(てあて): 
\\	手引(てびき): 
\\	手(て): 
\\	選手(せんしゅ): 
\\	相手(あいて): 
\\	取っ手(とって): 
\\	歌手(かしゅ): 
\\	手紙(てがみ): 
\\	手術(しゅじゅつ): 
\\	手段(しゅだん): 
\\	運転手(うんてんしゅ): 
\\	助手(じょしゅ): 
\\	拍手(はくしゅ): 
\\	苦手(にがて): 
\\	手前(てまえ): 
\\	切手(きって): 
\\	派手(はで): 
\\	手軽(てがる): 
\\	手順(てじゅん): 
\\	手帳(てちょう): 
\\	手間(てま): 
\\	手元(てもと): 
\\	小切手(こぎって): 
\\	握手(あくしゅ): 
\\	手伝い(てつだい): 
\\	上手(じょうず): 
\\	入手(にゅうしゅ): 
\\	手掛かり(てがかり): 
\\	手入れ(ていれ): 
\\	手本(てほん): 
\\	下手(へた): 
\\	勝手(かって): 
\\	手袋(てぶくろ): 
\\	手配(てはい): 
\\	着手(ちゃくしゅ): 
\\	手際(てぎわ): 
\\	手首(てくび): 
\\	手錠(てじょう): 
\\	土手(どて): 
\\	手法(しゅほう): 
\\	手遅れ(ておくれ): 
\\	手ごろ(てごろ): 
\\	手洗い(てあらい): 
\\	手品(てじな): 
\\	手分け(てわけ): 
\\	手数(てすう): 
\\	お手上げ(おてあげ): 
\\	手拭い(てぬぐい): 
\\	手伝う(てつだう): 
\\	お手洗い(おてあらい): 
\\	お手伝いさん(おてつだいさん): 
\\	手掛ける(てがける): 
\\	手芸(しゅげい): 
\\	手近(てぢか): 
\\	手回し(てまわし): 
\\	手筈(てはず): 
\\	勝手に(かってに): 
\\	上手(うわて): 
\\	攻め手(せめて): 
\\	手: 
\\	手	
\\	て・てへん	
\\	象形 
\\	足	ソク	あし、た.りる、た.る、た.す	あ, あす, おす, たらし	
\\	"足(あし): 
\\	足あと(あしあと): 
\\	足下(あしもと): 
\\	不足(ふそく): 
\\	発足(ほっそく): 
\\	満足(まんぞく): 
\\	補足(ほそく): 
\\	遠足(えんそく): 
\\	足る(たる): 
\\	物足りない(ものたりない): 
\\	駆け足(かけあし): 
\\	足し算(たしざん): 
\\	裸足(はだし): 
\\	足す(たす): 
\\	足りる(たりる): 
\\	足袋(たび): 
\\	口: 
\\	足	
\\	あし・あしへん	
\\	象形 
\\	見	ケン	み.る、み.える、み.せる		
\\	見付かる(みつかる): 
\\	見積り(みつもり): 
\\	意見(いけん): 
\\	会見(かいけん): 
\\	見方(みかた): 
\\	見通し(みとおし): 
\\	見解(けんかい): 
\\	見込み(みこみ): 
\\	発見(はっけん): 
\\	見事(みごと): 
\\	見学(けんがく): 
\\	見出し(みだし): 
\\	見送り(みおくり): 
\\	見送る(みおくる): 
\\	見舞う(みまう): 
\\	偏見(へんけん): 
\\	見舞い(みまい): 
\\	花見(はなみ): 
\\	見直す(みなおす): 
\\	見詰める(みつめる): 
\\	見る(みる): 
\\	見地(けんち): 
\\	見物(けんぶつ): 
\\	見える(みえる): 
\\	見合い(みあい): 
\\	一見(いっけん): 
\\	見当(けんとう): 
\\	見本(みほん): 
\\	見付ける(みつける): 
\\	見かけ(みかけ): 
\\	見渡す(みわたす): 
\\	見合わせる(みあわせる): 
\\	拝見(はいけん): 
\\	見せびらかす(みせびらかす): 
\\	見せる(みせる): 
\\	見苦しい(みぐるしい): 
\\	見かける(みかける): 
\\	見上げる(みあげる): 
\\	見晴らし(みはらし): 
\\	見落とす(みおとす): 
\\	見下ろす(みおろす): 
\\	見逃す(みのがす): 
\\	見習う(みならう): 
\\	見窄らしい(みすぼらしい): 
\\	見っともない(みっともない): 
\\	よそ見(よそみ): 
\\	見なす(みなす): 
\\	見慣れる(みなれる): 
\\	見計らう(みはからう): 
\\	見世物(みせもの): 
\\	お見舞い(おみまい): 
\\	異見(いけん): 
\\	目論見(もくろみ): 
\\	一見(いちげん): 
\\	目: 
\\	見	
\\	みる	
\\	象形 
\\	音	オン、イン、-ノン	おと、ね	お, と	
\\	音(ね): 
\\	音楽(おんがく): 
\\	本音(ほんね): 
\\	騒音(そうおん): 
\\	録音(ろくおん): 
\\	音色(ねいろ): 
\\	発音(はつおん): 
\\	雑音(ざつおん): 
\\	五十音(ごじゅうおん): 
\\	物音(ものおと): 
\\	立: 
\\	日: 
\\	音	
\\	おと・おとへん	
\\	会意 
\\	力	リョク、リキ、リイ	ちから	じから, つとむ	
\\	勢力(せいりょく): 
\\	努力(どりょく): 
\\	協力(きょうりょく): 
\\	権力(けんりょく): 
\\	実力(じつりょく): 
\\	能力(のうりょく): 
\\	武力(ぶりょく): 
\\	有力(ゆうりょく): 
\\	圧力(あつりょく): 
\\	全力(ぜんりょく): 
\\	電力(でんりょく): 
\\	魅力(みりょく): 
\\	強力(きょうりょく): 
\\	暴力(ぼうりょく): 
\\	体力(たいりょく): 
\\	活力(かつりょく): 
\\	学力(がくりょく): 
\\	重力(じゅうりょく): 
\\	効力(こうりょく): 
\\	力強い(ちからづよい): 
\\	動力(どうりょく): 
\\	労力(ろうりょく): 
\\	戦力(せんりょく): 
\\	威力(いりょく): 
\\	弾力(だんりょく): 
\\	引力(いんりょく): 
\\	力(ちから): 
\\	力(りょく): 
\\	浮力(ふりょく): 
\\	速力(そくりょく): 
\\	力: 
\\	力	
\\	ちから・りきづくり	
\\	象形 
\\	気	キ、ケ	いき		
\\	気(き): 
\\	景気(けいき): 
\\	人気(にんき): 
\\	気持ち(きもち): 
\\	電気(でんき): 
\\	病気(びょうき): 
\\	気分(きぶん): 
\\	空気(くうき): 
\\	雰囲気(ふんいき): 
\\	気温(きおん): 
\\	大気(たいき): 
\\	気象(きしょう): 
\\	気配(けはい): 
\\	強気(つよき): 
\\	元気(げんき): 
\\	天気(てんき): 
\\	勇気(ゆうき): 
\\	気候(きこう): 
\\	気味(きみ): 
\\	本気(ほんき): 
\\	気圧(きあつ): 
\\	気軽(きがる): 
\\	活気(かっき): 
\\	気楽(きらく): 
\\	不景気(ふけいき): 
\\	気質(きしつ): 
\\	換気(かんき): 
\\	寒気(かんき): 
\\	湿気(しっけ): 
\\	根気(こんき): 
\\	陽気(ようき): 
\\	水蒸気(すいじょうき): 
\\	浮気(うわき): 
\\	一気(いっき): 
\\	蒸気(じょうき): 
\\	磁気(じき): 
\\	無邪気(むじゃき): 
\\	意気込む(いきごむ): 
\\	気体(きたい): 
\\	吐き気(はきけ): 
\\	気の毒(きのどく): 
\\	平気(へいき): 
\\	気付く(きづく): 
\\	気まぐれ(きまぐれ): 
\\	水気(みずけ): 
\\	湯気(ゆげ): 
\\	何気ない(なにげない): 
\\	気兼ね(きがね): 
\\	陰気(いんき): 
\\	気に入る(きにいる): 
\\	生意気(なまいき): 
\\	短気(たんき): 
\\	気障(きざ): 
\\	殺気(さっき): 
\\	天気予報(てんきよほう): 
\\	気を付ける(きをつける): 
\\	暢気(のんき): 
\\	気品(きひん): 
\\	気風(きふう): 
\\	気流(きりゅう): 
\\	素っ気ない(そっけない): 
\\	気触れる(かぶれる): 
\\	気立て(きだて): 
\\	湿気る(しける): 
\\	人気(ひとけ): 
\\	呆気ない(あっけない): 
\\	気味(ぎみ): 
\\	お気の毒に(おきのどくに): 
\\	お元気ですか(おげんきですか): 
\\	气: 
\\	乂: 
\\	气	
\\	きがまえ	氣	形声 
\\	円	エン	まる.い、まる、まど、まど.か、まろ.やか	つぶら, のぶ, まどか, みつ	
\\	円い(まるい): 
\\	円(えん): 
\\	円滑(えんかつ): 
\\	円満(えんまん): 
\\	円周(えんしゅう): 
\\	楕円(だえん): 
\\	真ん円い(まんまるい): 
\\	円ら(つぶら): 
\\	冂: 
\\	囗	
\\	くに・くにがまえ	圓	形声 
\\	入	ニュウ、ジュ	い.る、-い.る、-い.り、い.れる、-い.れ、はい.る	いり, いる, に, の, りり	
\\	入口(いりぐち): 
\\	出入口(でいりぐち): 
\\	輸入(ゆにゅう): 
\\	購入(こうにゅう): 
\\	収入(しゅうにゅう): 
\\	導入(どうにゅう): 
\\	加入(かにゅう): 
\\	入院(にゅういん): 
\\	入学(にゅうがく): 
\\	入社(にゅうしゃ): 
\\	入場(にゅうじょう): 
\\	入賞(にゅうしょう): 
\\	受け入れる(うけいれる): 
\\	侵入(しんにゅう): 
\\	入手(にゅうしゅ): 
\\	入浴(にゅうよく): 
\\	納入(のうにゅう): 
\\	手入れ(ていれ): 
\\	出入り(でいり): 
\\	記入(きにゅう): 
\\	押し入れ(おしいれ): 
\\	介入(かいにゅう): 
\\	受け入れ(うけいれ): 
\\	投入(とうにゅう): 
\\	入る(はいる): 
\\	仕入れる(しいれる): 
\\	申し入れる(もうしいれる): 
\\	入れ物(いれもの): 
\\	気に入る(きにいる): 
\\	四捨五入(ししゃごにゅう): 
\\	取り入れる(とりいれる): 
\\	日の入り(ひのいり): 
\\	入る(いる): 
\\	入れる(いれる): 
\\	潜入(せんにゅう): 
\\	恐れ入る(おそれいる): 
\\	新入生(しんにゅうせい): 
\\	入: 
\\	入	
\\	いる・にゅう・いりがしら・いりやね	
\\	象形 
\\	出	シュツ、スイ	で.る、-で、だ.す、-だ.す、い.でる、い.だす	いず, いづ, いで, じ, すっ, すつ, てん	
\\	出生(しゅっせい): 
\\	引出し(ひきだし): 
\\	出入口(でいりぐち): 
\\	出合う(であう): 
\\	引出す(ひきだす): 
\\	出身(しゅっしん): 
\\	輸出(ゆしゅつ): 
\\	出場(しゅつじょう): 
\\	進出(しんしゅつ): 
\\	演出(えんしゅつ): 
\\	出す(だす): 
\\	出演(しゅつえん): 
\\	出席(しゅっせき): 
\\	出版(しゅっぱん): 
\\	提出(ていしゅつ): 
\\	支出(ししゅつ): 
\\	出産(しゅっさん): 
\\	出発(しゅっぱつ): 
\\	出来事(できごと): 
\\	思い出(おもいで): 
\\	出張(しゅっちょう): 
\\	脱出(だっしゅつ): 
\\	見出し(みだし): 
\\	出口(でぐち): 
\\	貸し出し(かしだし): 
\\	出血(しゅっけつ): 
\\	外出(がいしゅつ): 
\\	出現(しゅつげん): 
\\	出品(しゅっぴん): 
\\	出る(でる): 
\\	出勤(しゅっきん): 
\\	飛び出す(とびだす): 
\\	出入り(でいり): 
\\	出世(しゅっせ): 
\\	思い出す(おもいだす): 
\\	出来る(できる): 
\\	売り出す(うりだす): 
\\	出来るだけ(できるだけ): 
\\	日の出(ひので): 
\\	出迎え(でむかえ): 
\\	出題(しゅつだい): 
\\	抜け出す(ぬけだす): 
\\	出来上がり(できあがり): 
\\	追い出す(おいだす): 
\\	家出(いえで): 
\\	呼び出す(よびだす): 
\\	出会い(であい): 
\\	出動(しゅつどう): 
\\	申し出(もうしで): 
\\	出(で): 
\\	出直し(でなおし): 
\\	出費(しゅっぴ): 
\\	放出(ほうしゅつ): 
\\	出迎える(でむかえる): 
\\	振り出し(ふりだし): 
\\	出合い(であい): 
\\	噴出(ふんしゅつ): 
\\	売り出し(うりだし): 
\\	出掛ける(でかける): 
\\	逃げ出す(にげだす): 
\\	投げ出す(なげだす): 
\\	差し出す(さしだす): 
\\	申し出る(もうしでる): 
\\	取り出す(とりだす): 
\\	出来上がる(できあがる): 
\\	言い出す(いいだす): 
\\	お目出度う(おめでとう): 
\\	出鱈目(でたらめ): 
\\	目出度い(めでたい): 
\\	産出(さんしゅつ): 
\\	出くわす(でくわす): 
\\	出社(しゅっしゃ): 
\\	お目出度うご座います(おめでとうございます): 
\\	降り出す(ふりだす): 
\\	お出でになる(おいでになる): 
\\	お出掛け(おでかけ): 
\\	出切る(できる): 
\\	放り出す(ほうりだす): 
\\	出端(では): 
\\	出来物(できもの): 
\\	お目出度い(おめでたい): 
\\	山: 
\\	凵: 
\\	凵	
\\	かんにょう・うけばこ・かんがまえ	
\\	象形 
\\	立	リツ、リュウ、リットル	た.つ、-た.つ、た.ち-、た.てる、-た.てる、た.て-、たて-、-た.て、-だ.て、-だ.てる	たち, たっ, たつ, だて, つい	
\\	立場(たちば): 
\\	国立(こくりつ): 
\\	成立(せいりつ): 
\\	対立(たいりつ): 
\\	独立(どくりつ): 
\\	私立(しりつ): 
\\	設立(せつりつ): 
\\	確立(かくりつ): 
\\	立法(りっぽう): 
\\	公立(こうりつ): 
\\	創立(そうりつ): 
\\	中立(ちゅうりつ): 
\\	立派(りっぱ): 
\\	孤立(こりつ): 
\\	立体(りったい): 
\\	成り立つ(なりたつ): 
\\	両立(りょうりつ): 
\\	献立(こんだて): 
\\	立つ(たつ): 
\\	役に立つ(やくにたつ): 
\\	立ち寄る(たちよる): 
\\	樹立(じゅりつ): 
\\	自立(じりつ): 
\\	目立つ(めだつ): 
\\	立ち止まる(たちどまる): 
\\	夕立(ゆうだち): 
\\	役立つ(やくだつ): 
\\	立ち上がる(たちあがる): 
\\	逆立ち(さかだち): 
\\	組み立てる(くみたてる): 
\\	立て替える(たてかえる): 
\\	仕立てる(したてる): 
\\	立ち去る(たちさる): 
\\	立てる(たてる): 
\\	立方(たちかた): 
\\	取り立てる(とりたてる): 
\\	気立て(きだて): 
\\	腹立ち(はらだち): 
\\	立方(りっぽう): 
\\	亠: 
\\	立	
\\	たつ・たつへん	
\\	会意 
\\	休	キュウ	やす.む、やす.まる、やす.める		
\\	夏休み(なつやすみ): 
\\	休み(やすみ): 
\\	休暇(きゅうか): 
\\	休業(きゅうぎょう): 
\\	連休(れんきゅう): 
\\	休憩(きゅうけい): 
\\	休養(きゅうよう): 
\\	昼休み(ひるやすみ): 
\\	休む(やすむ): 
\\	休息(きゅうそく): 
\\	休戦(きゅうせん): 
\\	休学(きゅうがく): 
\\	定休日(ていきゅうび): 
\\	休講(きゅうこう): 
\\	お休みなさい(おやすみなさい): 
\\	一休み(ひとやすみ): 
\\	休める(やすめる): 
\\	お休み(おやすみ): 
\\	産休(さんきゅう): 
\\	人: 
\\	木: 
\\	人	
\\	ひと・にんべん・ひとやね	
\\	会意 
\\	先	セン	さき、ま.ず	ぽん	
\\	先月(せんげつ): 
\\	先生(せんせい): 
\\	優先(ゆうせん): 
\\	先に(さきに): 
\\	先頭(せんとう): 
\\	先週(せんしゅう): 
\\	先輩(せんぱい): 
\\	先日(せんじつ): 
\\	先端(せんたん): 
\\	先祖(せんぞ): 
\\	先着(せんちゃく): 
\\	真っ先(まっさき): 
\\	祖先(そせん): 
\\	勤め先(つとめさき): 
\\	先ほど(さきほど): 
\\	先行(せんこう): 
\\	先だって(せんだって): 
\\	先代(せんだい): 
\\	先天的(せんてんてき): 
\\	先(さき): 
\\	先ず(まず): 
\\	お先に(おさきに): 
\\	先(せん): 
\\	先々月(せんせんげつ): 
\\	先先週(せんせんしゅう): 
\\	八: 
\\	儿	
\\	にんにょう・ひとあし	
\\	会意 
\\	夕	セキ	ゆう	ゆ	
\\	夕(ゆう): 
\\	夕方(ゆうがた): 
\\	夕刊(ゆうかん): 
\\	夕暮れ(ゆうぐれ): 
\\	夕日(ゆうひ): 
\\	夕飯(ゆうめし): 
\\	夕焼け(ゆうやけ): 
\\	夕立(ゆうだち): 
\\	夕: 
\\	夕	
\\	ゆう・ゆうべ・た	
\\	象形 
\\	本	ホン	もと	まと, ごう	
\\	本(もと): 
\\	本文(ほんもん): 
\\	基本(きほん): 
\\	本部(ほんぶ): 
\\	本人(ほんにん): 
\\	本当(ほんとう): 
\\	資本(しほん): 
\\	本来(ほんらい): 
\\	本名(ほんみょう): 
\\	脚本(きゃくほん): 
\\	本音(ほんね): 
\\	本体(ほんたい): 
\\	本質(ほんしつ): 
\\	本物(ほんもの): 
\\	本の(ほんの): 
\\	本国(ほんごく): 
\\	本気(ほんき): 
\\	本場(ほんば): 
\\	手本(てほん): 
\\	本能(ほんのう): 
\\	標本(ひょうほん): 
\\	見本(みほん): 
\\	本々(もともと): 
\\	日本語(にほんご): 
\\	本格(ほんかく): 
\\	台本(だいほん): 
\\	本館(ほんかん): 
\\	本棚(ほんだな): 
\\	日本式(にほんしき): 
\\	本(ほん): 
\\	日本(にほん): 
\\	木: 
\\	木	
\\	き・きへん	
\\	指事 
\\	文	ブン、モン	ふみ、あや	かざり, ふ, も	
\\	文字(もんじ): 
\\	本文(ほんもん): 
\\	文化(ぶんか): 
\\	文学(ぶんがく): 
\\	文書(ぶんしょ): 
\\	注文(ちゅうもん): 
\\	論文(ろんぶん): 
\\	文章(ぶんしょう): 
\\	文明(ぶんめい): 
\\	文芸(ぶんげい): 
\\	文化財(ぶんかざい): 
\\	文句(もんく): 
\\	作文(さくぶん): 
\\	英文(えいぶん): 
\\	文献(ぶんけん): 
\\	文体(ぶんたい): 
\\	文脈(ぶんみゃく): 
\\	文房具(ぶんぼうぐ): 
\\	文法(ぶんぽう): 
\\	文(ふみ): 
\\	人文科学(じんぶんかがく): 
\\	原文(げんぶん): 
\\	文語(ぶんご): 
\\	文(ぶん): 
\\	和文(わぶん): 
\\	亠: 
\\	乂: 
\\	文	
\\	ぶん・ぶんにょう	
\\	象形 
\\	字	ジ	あざ、あざな、-な		
\\	"文字(もんじ): 
\\	黒字(くろじ): 
\\	数字(すうじ): 
\\	赤字(あかじ): 
\\	漢字(かんじ): 
\\	活字(かつじ): 
\\	苗字(みょうじ): 
\\	字(あざ): 
\\	英字(えいじ): 
\\	十字路(じゅうじろ): 
\\	ローマ字(ローマじ): 
\\	字(じ): 
\\	字引(じびき): 
\\	習字(しゅうじ): 
\\	字体(じたい): 
\\	当て字(あてじ): 
\\	宀: 
\\	子: 
\\	子	
\\	こ・こへん	
\\	会意 
\\	学	ガク	まな.ぶ	たか, のり	
\\	学校(がっこう): 
\\	学生(がくせい): 
\\	大学(だいがく): 
\\	医学(いがく): 
\\	科学(かがく): 
\\	学部(がくぶ): 
\\	文学(ぶんがく): 
\\	学者(がくしゃ): 
\\	学習(がくしゅう): 
\\	小学校(しょうがっこう): 
\\	中学(ちゅうがく): 
\\	化学(かがく): 
\\	学会(がっかい): 
\\	小学生(しょうがくせい): 
\\	大学院(だいがくいん): 
\\	中学校(ちゅうがっこう): 
\\	入学(にゅうがく): 
\\	哲学(てつがく): 
\\	留学(りゅうがく): 
\\	留学生(りゅうがくせい): 
\\	学科(がっか): 
\\	工学(こうがく): 
\\	進学(しんがく): 
\\	法学(ほうがく): 
\\	見学(けんがく): 
\\	学歴(がくれき): 
\\	在学(ざいがく): 
\\	数学(すうがく): 
\\	学年(がくねん): 
\\	学問(がくもん): 
\\	語学(ごがく): 
\\	通学(つうがく): 
\\	学力(がくりょく): 
\\	奨学金(しょうがくきん): 
\\	高等学校(こうとうがっこう): 
\\	学期(がっき): 
\\	退学(たいがく): 
\\	社会科学(しゃかいかがく): 
\\	自然科学(しぜんかがく): 
\\	学説(がくせつ): 
\\	学芸(がくげい): 
\\	学術(がくじゅつ): 
\\	学級(がっきゅう): 
\\	考古学(こうこがく): 
\\	学(がく): 
\\	学ぶ(まなぶ): 
\\	学士(がくし): 
\\	人文科学(じんぶんかがく): 
\\	共学(きょうがく): 
\\	休学(きゅうがく): 
\\	大学生(だいがくせい): 
\\	修学(しゅうがく): 
\\	小: 
\\	冖: 
\\	子: 
\\	子	
\\	こ・こへん	學	会意 
\\	校	コウ、キョウ		めん	
\\	学校(がっこう): 
\\	高校(こうこう): 
\\	校長(こうちょう): 
\\	高校生(こうこうせい): 
\\	小学校(しょうがっこう): 
\\	中学校(ちゅうがっこう): 
\\	登校(とうこう): 
\\	校舎(こうしゃ): 
\\	高等学校(こうとうがっこう): 
\\	校庭(こうてい): 
\\	母校(ぼこう): 
\\	転校(てんこう): 
\\	校(こう): 
\\	木: 
\\	交: 
\\	木	
\\	き・きへん	
\\	形声 
\\	村	ソン	むら	え, むた, ら	
\\	村(むら): 
\\	農村(のうそん): 
\\	漁村(ぎょそん): 
\\	木: 
\\	寸: 
\\	木	
\\	き・きへん	
\\	形声 
\\	町	チョウ	まち		
\\	町角(まちかど): 
\\	町(まち): 
\\	下町(したまち): 
\\	田: 
\\	丁: 
\\	田	
\\	た・たへん	
\\	形声 
\\	森	シン	もり	もと	
\\	森(もり): 
\\	森林(しんりん): 
\\	木: 
\\	林: 
\\	木	
\\	き・きへん	
\\	会意 
\\	正	セイ、ショウ	ただ.しい、ただ.す、まさ、まさ.に	おお, くに, ま, まさし, ただし	
\\	改正(かいせい): 
\\	修正(しゅうせい): 
\\	正式(せいしき): 
\\	正常(せいじょう): 
\\	是正(ぜせい): 
\\	正義(せいぎ): 
\\	正午(しょうご): 
\\	正(せい): 
\\	正面(しょうめん): 
\\	不正(ふせい): 
\\	正月(しょうがつ): 
\\	正当(せいとう): 
\\	公正(こうせい): 
\\	正直(しょうじき): 
\\	正規(せいき): 
\\	訂正(ていせい): 
\\	正体(しょうたい): 
\\	正解(せいかい): 
\\	正門(せいもん): 
\\	正確(せいかく): 
\\	正しい(ただしい): 
\\	正方形(せいほうけい): 
\\	正に(まさに): 
\\	正味(しょうみ): 
\\	正しく(まさしく): 
\\	一: 
\\	止: 
\\	止	
\\	とまる・とめる・とめへん	
\\	会意 
\\	水	スイ	みず、みず-	うず, ずみ, つ, ど, み, みさ, みつ, みな, みん	
\\	水(みず): 
\\	水準(すいじゅん): 
\\	水道(すいどう): 
\\	水面(すいめん): 
\\	水泳(すいえい): 
\\	水田(すいでん): 
\\	水産(すいさん): 
\\	洪水(こうずい): 
\\	排水(はいすい): 
\\	水曜(すいよう): 
\\	水素(すいそ): 
\\	水分(すいぶん): 
\\	水着(みずぎ): 
\\	水平(すいへい): 
\\	香水(こうすい): 
\\	下水(げすい): 
\\	水蒸気(すいじょうき): 
\\	断水(だんすい): 
\\	水源(すいげん): 
\\	地下水(ちかすい): 
\\	海水浴(かいすいよく): 
\\	淡水(たんすい): 
\\	噴水(ふんすい): 
\\	水洗(すいせん): 
\\	潜水(せんすい): 
\\	水滴(すいてき): 
\\	水平線(すいへいせん): 
\\	水気(みずけ): 
\\	降水(こうすい): 
\\	水筒(すいとう): 
\\	大水(おおみず): 
\\	水曜日(すいようび): 
\\	水: 
\\	水	
\\	みず・したみず・さんずい	
\\	象形 
\\	火	カ	ひ、-び、ほ-		
\\	火(ひ): 
\\	火山(かざん): 
\\	火災(かさい): 
\\	噴火(ふんか): 
\\	火事(かじ): 
\\	花火(はなび): 
\\	火星(かせい): 
\\	火花(ひばな): 
\\	火曜日(かようび): 
\\	防火(ぼうか): 
\\	火曜(かよう): 
\\	下火(したび): 
\\	火傷(やけど): 
\\	火口(かこう): 
\\	焚き火(たきび): 
\\	火: 
\\	火	
\\	ひ・ひへん・れっか・れんが	
\\	象形 
\\	玉	ギョク	たま、たま-、-だま	おう, だん	
\\	玉(たま): 
\\	玉(ぎょく): 
\\	玉: 
\\	丶: 
\\	玉	
\\	たま・たまへん ・ おう・おうへん・あうさま	
\\	象形 
\\	王	オウ、-ノウ		おお, おおきみ, わ	
\\	女王(じょおう): 
\\	王子(おうじ): 
\\	王様(おうさま): 
\\	王女(おうじょ): 
\\	王(おう): 
\\	国王(こくおう): 
\\	玉: 
\\	玉	
\\	たま・たまへん ・ おう・おうへん・あうさま	
\\	象形 
\\	石	セキ、シャク、コク	いし	いさ, いす, いわ, し, せっく, と	
\\	"石油(せきゆ): 
\\	化石(かせき): 
\\	石炭(せきたん): 
\\	宝石(ほうせき): 
\\	磁石(じしゃく): 
\\	岩石(がんせき): 
\\	石鹸(せっけん): 
\\	石(いし): 
\\	流石(さすが): 
\\	口: 
\\	石	
\\	いし・いしへん	
\\	会意 
\\	竹	チク	たけ	たか	
\\	竹(たけ): 
\\	竹: 
\\	竹	
\\	たけ・たけかんむり	
\\	象形 
\\	糸	シ	いと		
\\	糸(いと): 
\\	毛糸(けいと): 
\\	糸: 
\\	糸	
\\	いと・いとへん	絲	会意 
\\	貝	バイ	かい		
\\	貝(かい): 
\\	貝殼(かいがら): 
\\	目: 
\\	貝	
\\	かい・かいへん・こがい	
\\	象形 
\\	車	シャ	くるま	くら, くろま	
\\	自動車(じどうしゃ): 
\\	車(くるま): 
\\	下車(げしゃ): 
\\	自転車(じてんしゃ): 
\\	電車(でんしゃ): 
\\	列車(れっしゃ): 
\\	駐車場(ちゅうしゃじょう): 
\\	車庫(しゃこ): 
\\	駐車(ちゅうしゃ): 
\\	乗車(じょうしゃ): 
\\	車掌(しゃしょう): 
\\	車輪(しゃりん): 
\\	停車(ていしゃ): 
\\	歯車(はぐるま): 
\\	汽車(きしゃ): 
\\	機関車(きかんしゃ): 
\\	発車(はっしゃ): 
\\	風車(かざぐるま): 
\\	車: 
\\	車	
\\	くるま・くるまへん	
\\	象形 
\\	金	キン、コン、ゴン	かね、かな-、-がね	かん, きむ, こ, この, ん	
\\	黄金(こがね): 
\\	金融(きんゆう): 
\\	資金(しきん): 
\\	料金(りょうきん): 
\\	金額(きんがく): 
\\	現金(げんきん): 
\\	預金(よきん): 
\\	お金(おかね): 
\\	金属(きんぞく): 
\\	賞金(しょうきん): 
\\	貯金(ちょきん): 
\\	賃金(ちんぎん): 
\\	借金(しゃっきん): 
\\	税金(ぜいきん): 
\\	代金(だいきん): 
\\	金銭(きんせん): 
\\	金曜(きんよう): 
\\	金庫(きんこ): 
\\	金持ち(かねもち): 
\\	送金(そうきん): 
\\	奨学金(しょうがくきん): 
\\	大金(たいきん): 
\\	金魚(きんぎょ): 
\\	基金(ききん): 
\\	募金(ぼきん): 
\\	針金(はりがね): 
\\	集金(しゅうきん): 
\\	金(かね): 
\\	金(きん): 
\\	金曜日(きんようび): 
\\	金槌(かなづち): 
\\	残金(ざんきん): 
\\	金蔵(かねぐら): 
\\	金: 
\\	金	
\\	かね・かねへん	
\\	象形 
\\	雨	ウ	あめ、あま-、-さめ		
\\	梅雨(つゆ): 
\\	雨(あめ): 
\\	雨天(うてん): 
\\	雨戸(あまど): 
\\	雨具(あまぐ): 
\\	雨: 
\\	雨	
\\	あめ・あめかんむり・あまかんむり	
\\	象形 
\\	赤	セキ、シャク	あか、あか-、あか.い、あか.らむ、あか.らめる	あ, あこ, あま	
\\	赤字(あかじ): 
\\	赤ちゃん(あかちゃん): 
\\	真っ赤(まっか): 
\\	赤ん坊(あかんぼう): 
\\	赤道(せきどう): 
\\	赤らむ(あからむ): 
\\	赤(あか): 
\\	土: 
\\	赤	
\\	あか・あかへん	
\\	会意 
\\	青	セイ、ショウ	あお、あお-、あお.い	お	
\\	青(あお): 
\\	青年(せいねん): 
\\	青春(せいしゅん): 
\\	青少年(せいしょうねん): 
\\	青い(あおい): 
\\	真っ青(まっさお): 
\\	青白い(あおじろい): 
\\	月: 
\\	青	
\\	あお・あおへん	
\\	形声 
\\	白	ハク、ビャク	しろ、しら-、しろ.い	あき, か, はっ	
\\	白髪(しらが): 
\\	白(しろ): 
\\	明白(めいはく): 
\\	真っ白(まっしろ): 
\\	白い(しろい): 
\\	告白(こくはく): 
\\	面白い(おもしろい): 
\\	白状(はくじょう): 
\\	青白い(あおじろい): 
\\	蛋白質(たんぱくしつ): 
\\	明白(あからさま): 
\\	日: 
\\	白	
\\	しろ・しろへん	
\\	象形 
\\	数	スウ、ス、サク、ソク、シュ	かず、かぞ.える、しばしば、せ.める、わずらわ.しい	じゅ	
\\	数字(すうじ): 
\\	数(かず): 
\\	少数(しょうすう): 
\\	複数(ふくすう): 
\\	過半数(かはんすう): 
\\	回数(かいすう): 
\\	数学(すうがく): 
\\	点数(てんすう): 
\\	無数(むすう): 
\\	枚数(まいすう): 
\\	算数(さんすう): 
\\	数える(かぞえる): 
\\	回数券(かいすうけん): 
\\	多数決(たすうけつ): 
\\	手数(てすう): 
\\	偶数(ぐうすう): 
\\	奇数(きすう): 
\\	小数(しょうすう): 
\\	整数(せいすう): 
\\	単数(たんすう): 
\\	分数(ぶんすう): 
\\	数(すう): 
\\	御数(おかず): 
\\	数詞(すうし): 
\\	米: 
\\	女: 
\\	攵: 
\\	攴	
\\	ぼくにょう・ぼくづくり・とまた・のぶん・しぶん	數	会意 
\\	多	タ	おお.い、まさ.に、まさ.る		
\\	多様(たよう): 
\\	多少(たしょう): 
\\	多分(たぶん): 
\\	多忙(たぼう): 
\\	多数決(たすうけつ): 
\\	多い(おおい): 
\\	滅多に(めったに): 
\\	幾多(いくた): 
\\	歌留多(かるた): 
\\	夕: 
\\	夕: 
\\	夕	
\\	ゆう・ゆうべ・た	
\\	会意 
\\	少	ショウ	すく.ない、すこ.し		
\\	少し(すこし): 
\\	少年(しょうねん): 
\\	少なくとも(すくなくとも): 
\\	減少(げんしょう): 
\\	少数(しょうすう): 
\\	少女(しょうじょ): 
\\	多少(たしょう): 
\\	少々(しょうしょう): 
\\	少ない(すくない): 
\\	青少年(せいしょうねん): 
\\	少しも(すこしも): 
\\	小: 
\\	丿: 
\\	小	
\\	しょう・ちいさい・なおがしら・つ	
\\	象形 
\\	万	マン、バン	よろず	かず, ま, ゆる	
\\	"万人(ばんじん): 
\\	万能(ばんのう): 
\\	万一(まんいち): 
\\	万歳(ばんざい): 
\\	万年筆(まんねんひつ): 
\\	万(まん): 
\\	万(ばん): 
\\	一: 
\\	艸	
\\	くさ・くさかんむり	萬	象形 
\\	半	ハン	なか.ば	は	
\\	半分(はんぶん): 
\\	大半(たいはん): 
\\	半(はん): 
\\	過半数(かはんすう): 
\\	半島(はんとう): 
\\	半ば(なかば): 
\\	半径(はんけい): 
\\	半端(はんぱ): 
\\	丶: 
\\	二: 
\\	十: 
\\	十	
\\	じゅう・じゅうへん	
\\	象形 
\\	形	ケイ、ギョウ	かた、-がた、かたち、なり	ち	
\\	形(かたち): 
\\	形式(けいしき): 
\\	形成(けいせい): 
\\	人形(にんぎょう): 
\\	地形(ちけい): 
\\	図形(ずけい): 
\\	形態(けいたい): 
\\	長方形(ちょうほうけい): 
\\	形勢(けいせい): 
\\	形容詞(けいようし): 
\\	原形(げんけい): 
\\	正方形(せいほうけい): 
\\	形容動詞(けいようどうし): 
\\	形(けい): 
\\	幵: 
\\	彡: 
\\	彡	
\\	さん・さんづくり	
\\	会意 
\\	太	タイ、タ	ふと.い、ふと.る	おお, たか, ひろ	
\\	太陽(たいよう): 
\\	太鼓(たいこ): 
\\	太い(ふとい): 
\\	太る(ふとる): 
\\	大: 
\\	丶: 
\\	大	
\\	だい・おおきい	
\\	形声 
\\	細	サイ	ほそ.い、ほそ.る、こま.か、こま.かい		
\\	"細胞(さいぼう): 
\\	細かい(こまかい): 
\\	詳細(しょうさい): 
\\	細菌(さいきん): 
\\	細工(さいく): 
\\	細い(ほそい): 
\\	心細い(こころぼそい): 
\\	亜細亜(アジア): 
\\	細やか(こまやか): 
\\	糸: 
\\	田: 
\\	糸	
\\	いと・いとへん	
\\	形声 
\\	広	コウ	ひろ.い、ひろ.まる、ひろ.める、ひろ.がる、ひろ.げる		
\\	広告(こうこく): 
\\	広い(ひろい): 
\\	広場(ひろば): 
\\	広げる(ひろげる): 
\\	広さ(ひろさ): 
\\	背広(せびろ): 
\\	広まる(ひろまる): 
\\	広める(ひろめる): 
\\	広がる(ひろがる): 
\\	広々(ひろびろ): 
\\	广: 
\\	厶: 
\\	广	
\\	まだれ	廣	形声 
\\	長	チョウ	なが.い、おさ	お, おしゃ, たかし, たけ, な, は, ひさ	
\\	議長(ぎちょう): 
\\	社長(しゃちょう): 
\\	成長(せいちょう): 
\\	長男(ちょうなん): 
\\	部長(ぶちょう): 
\\	延長(えんちょう): 
\\	課長(かちょう): 
\\	長期(ちょうき): 
\\	校長(こうちょう): 
\\	長女(ちょうじょ): 
\\	身長(しんちょう): 
\\	長い(ながい): 
\\	長所(ちょうしょ): 
\\	生長(せいちょう): 
\\	長官(ちょうかん): 
\\	長短(ちょうたん): 
\\	長(ちょう): 
\\	長編(ちょうへん): 
\\	長方形(ちょうほうけい): 
\\	年長(ねんちょう): 
\\	長引く(ながびく): 
\\	長閑(のどか): 
\\	特長(とくちょう): 
\\	長々(ながなが): 
\\	長(おさ): 
\\	長: 
\\	長	
\\	ながい	
\\	象形 
\\	点	テン	つ.ける、つ.く、た.てる、さ.す、とぼ.す、とも.す、ぼち		
\\	焦点(しょうてん): 
\\	視点(してん): 
\\	重点(じゅうてん): 
\\	原点(げんてん): 
\\	地点(ちてん): 
\\	点(てん): 
\\	点検(てんけん): 
\\	頂点(ちょうてん): 
\\	点数(てんすう): 
\\	弱点(じゃくてん): 
\\	交差点(こうさてん): 
\\	満点(まんてん): 
\\	欠点(けってん): 
\\	採点(さいてん): 
\\	終点(しゅうてん): 
\\	得点(とくてん): 
\\	観点(かんてん): 
\\	利点(りてん): 
\\	起点(きてん): 
\\	要点(ようてん): 
\\	盲点(もうてん): 
\\	句読点(くとうてん): 
\\	点く(つく): 
\\	点ける(つける): 
\\	零点(れいてん): 
\\	点線(てんせん): 
\\	点々(てんてん): 
\\	占: 
\\	火: 
\\	黑	
\\	くろ・くろへん	點	形声 
\\	丸	ガン	まる、まる.める、まる.い	ま, わ, わに	
\\	日の丸(ひのまる): 
\\	丸ごと(まるごと): 
\\	丸める(まるめる): 
\\	丸(まる): 
\\	丸っきり(まるっきり): 
\\	丸で(まるで): 
\\	丸々(まるまる): 
\\	九: 
\\	丶: 
\\	丶	
\\	てん・ちょぼ・ちゅ・ちゅてん	
\\	象形 
\\	交	コウ	まじ.わる、まじ.える、ま.じる、まじ.る、ま.ざる、ま.ぜる、-か.う、か.わす、かわ.す、こもごも	かた	
\\	交替(こうたい): 
\\	外交(がいこう): 
\\	交渉(こうしょう): 
\\	交流(こうりゅう): 
\\	交換(こうかん): 
\\	交通(こうつう): 
\\	国交(こっこう): 
\\	交付(こうふ): 
\\	交際(こうさい): 
\\	交差点(こうさてん): 
\\	交互(こうご): 
\\	交わす(かわす): 
\\	交番(こうばん): 
\\	社交(しゃこう): 
\\	交じる(まじる): 
\\	交ざる(まざる): 
\\	交える(まじえる): 
\\	交易(こうえき): 
\\	交通機関(こうつうきかん): 
\\	交わる(まじわる): 
\\	交差(こうさ): 
\\	亠: 
\\	父: 
\\	亠	
\\	けいさん・けいさんかんむり・なべぶた	
\\	象形 
\\	光	コウ	ひか.る、ひかり	あき, あきら, こお, てる, ひこ, み, みつ	
\\	光(ひかり): 
\\	観光(かんこう): 
\\	光景(こうけい): 
\\	日光(にっこう): 
\\	光線(こうせん): 
\\	蛍光灯(けいこうとう): 
\\	光る(ひかる): 
\\	光沢(こうたく): 
\\	光熱費(こうねつひ): 
\\	稲光(いなびかり): 
\\	小: 
\\	兀: 
\\	儿	
\\	にんにょう・ひとあし	
\\	会意 
\\	角	カク	かど、つの	い, す, すみ, ずみ, ふさ	
\\	"町角(まちかど): 
\\	角(かど): 
\\	角度(かくど): 
\\	三角(さんかく): 
\\	四角い(しかくい): 
\\	方角(ほうがく): 
\\	四角(しかく): 
\\	直角(ちょっかく): 
\\	角(かく): 
\\	角(つの): 
\\	四つ角(よつかど): 
\\	兎に角(とにかく): 
\\	兎も角(ともかく): 
\\	兎角(とかく): 
\\	折角(せっかく): 
\\	勹: 
\\	角	
\\	つの・かく・つのへん	
\\	象形 
\\	計	ケイ	はか.る、はか.らう	え, かず, け	
\\	計る(はかる): 
\\	計画(けいかく): 
\\	合計(ごうけい): 
\\	設計(せっけい): 
\\	計算(けいさん): 
\\	統計(とうけい): 
\\	会計(かいけい): 
\\	計(けい): 
\\	時計(とけい): 
\\	家計(かけい): 
\\	余計(よけい): 
\\	生計(せいけい): 
\\	集計(しゅうけい): 
\\	計器(けいき): 
\\	見計らう(みはからう): 
\\	言: 
\\	十: 
\\	言	
\\	こと・げん・ごんべん・ことば	
\\	会意 
\\	直	チョク、ジキ、ジカ	ただ.ちに、なお.す、-なお.す、なお.る、なお.き、す.ぐ	すぐ, なお, のう, のお	
\\	直後(ちょくご): 
\\	直接(ちょくせつ): 
\\	直前(ちょくぜん): 
\\	直(じき): 
\\	直ちに(ただちに): 
\\	率直(そっちょく): 
\\	正直(しょうじき): 
\\	素直(すなお): 
\\	直径(ちょっけい): 
\\	見直す(みなおす): 
\\	直線(ちょくせん): 
\\	垂直(すいちょく): 
\\	直通(ちょくつう): 
\\	直観(ちょっかん): 
\\	直る(なおる): 
\\	直面(ちょくめん): 
\\	出直し(でなおし): 
\\	直角(ちょっかく): 
\\	仲直り(なかなおり): 
\\	直ぐ(すぐ): 
\\	直に(じかに): 
\\	真っ直ぐ(まっすぐ): 
\\	直流(ちょくりゅう): 
\\	直(なお): 
\\	直ぐに(すぐに): 
\\	もう直ぐ(もうすぐ): 
\\	十: 
\\	目: 
\\	目	
\\	め・めへん・よこめ	
\\	会意 
\\	線	セン	すじ		
\\	新幹線(しんかんせん): 
\\	線(せん): 
\\	無線(むせん): 
\\	線路(せんろ): 
\\	直線(ちょくせん): 
\\	沿線(えんせん): 
\\	光線(こうせん): 
\\	電線(でんせん): 
\\	曲線(きょくせん): 
\\	内線(ないせん): 
\\	幹線(かんせん): 
\\	三味線(しゃみせん): 
\\	地平線(ちへいせん): 
\\	水平線(すいへいせん): 
\\	脱線(だっせん): 
\\	下線(かせん): 
\\	点線(てんせん): 
\\	糸: 
\\	泉: 
\\	糸	
\\	いと・いとへん	
\\	形声 
\\	矢	シ	や		
\\	矢(や): 
\\	矢っ張り(やっぱり): 
\\	矢印(やじるし): 
\\	矢鱈に(やたらに): 
\\	天: 
\\	矢	
\\	や・やへん	
\\	象形 
\\	弱	ジャク	よわ.い、よわ.る、よわ.まる、よわ.める		
\\	弱点(じゃくてん): 
\\	弱い(よわい): 
\\	貧弱(ひんじゃく): 
\\	弱まる(よわまる): 
\\	弱める(よわめる): 
\\	弱る(よわる): 
\\	薄弱(はくじゃく): 
\\	弱(じゃく): 
\\	弓: 
\\	冫: 
\\	弓: 
\\	冫: 
\\	弓	
\\	ゆみ・ゆみへん	
\\	会意 
\\	強	キョウ、ゴウ	つよ.い、つよ.まる、つよ.める、し.いる、こわ.い	すね	
\\	強化(きょうか): 
\\	強制(きょうせい): 
\\	勉強(べんきょう): 
\\	強力(きょうりょく): 
\\	強盗(ごうとう): 
\\	強気(つよき): 
\\	強烈(きょうれつ): 
\\	強引(ごういん): 
\\	強める(つよめる): 
\\	補強(ほきょう): 
\\	強行(きょうこう): 
\\	力強い(ちからづよい): 
\\	強い(つよい): 
\\	強いる(しいる): 
\\	強硬(きょうこう): 
\\	増強(ぞうきょう): 
\\	強調(きょうちょう): 
\\	心強い(こころづよい): 
\\	強まる(つよまる): 
\\	強いて(しいて): 
\\	強請る(ねだる): 
\\	強(きょう): 
\\	弓: 
\\	厶: 
\\	虫: 
\\	弓	
\\	ゆみ・ゆみへん	
\\	会意 
\\	高	コウ	たか.い、たか、-だか、たか.まる、たか.める	か, こ, じょい, た, はか	
\\	"高校(こうこう): 
\\	最高(さいこう): 
\\	高校生(こうこうせい): 
\\	高級(こうきゅう): 
\\	高層(こうそう): 
\\	高速(こうそく): 
\\	高度(こうど): 
\\	高等(こうとう): 
\\	残高(ざんだか): 
\\	高い(たかい): 
\\	高まる(たかまる): 
\\	高める(たかめる): 
\\	高価(こうか): 
\\	高原(こうげん): 
\\	高等学校(こうとうがっこう): 
\\	名高い(なだかい): 
\\	高(たか): 
\\	高尚(こうしょう): 
\\	高(こう): 
\\	亠: 
\\	口: 
\\	冂: 
\\	口: 
\\	高	
\\	たかい	
\\	会意 
\\	同	ドウ	おな.じ		
\\	共同(きょうどう): 
\\	同時(どうじ): 
\\	同様(どうよう): 
\\	同盟(どうめい): 
\\	合同(ごうどう): 
\\	同士(どうし): 
\\	同一(どういつ): 
\\	同意(どうい): 
\\	同志(どうし): 
\\	同情(どうじょう): 
\\	同居(どうきょ): 
\\	同感(どうかん): 
\\	同等(どうとう): 
\\	一同(いちどう): 
\\	同僚(どうりょう): 
\\	同調(どうちょう): 
\\	同級(どうきゅう): 
\\	同封(どうふう): 
\\	混同(こんどう): 
\\	同(どう): 
\\	同じ(おなじ): 
\\	同い年(おないどし): 
\\	同格(どうかく): 
\\	冂: 
\\	一: 
\\	口: 
\\	口	
\\	くち・くちへん	
\\	会意 
\\	親	シン	おや、おや-、した.しい、した.しむ	ぎ, ちか, のり	
\\	親(おや): 
\\	父親(ちちおや): 
\\	母親(ははおや): 
\\	両親(りょうしん): 
\\	親類(しんるい): 
\\	親善(しんぜん): 
\\	肉親(にくしん): 
\\	親しむ(したしむ): 
\\	親友(しんゆう): 
\\	親切(しんせつ): 
\\	親指(おやゆび): 
\\	親しい(したしい): 
\\	親戚(しんせき): 
\\	親父(おやじ): 
\\	立: 
\\	木: 
\\	見: 
\\	見	
\\	みる	
\\	会意 
\\	母	ボ	はは、も		
\\	叔母(おば): 
\\	母(はは): 
\\	母親(ははおや): 
\\	お母さん(おかあさん): 
\\	父母(ふぼ): 
\\	祖母(そぼ): 
\\	母国(ぼこく): 
\\	母校(ぼこう): 
\\	保母(ほぼ): 
\\	分母(ぶんぼ): 
\\	お祖母さん(おばあさん): 
\\	伯母さん(おばさん): 
\\	小母さん(おばさん): 
\\	毋: 
\\	毋	
\\	なし・なかれ ・ははのかん	
\\	象形 
\\	父	フ	ちち		
\\	叔父(おじ): 
\\	父(ちち): 
\\	父親(ちちおや): 
\\	父母(ふぼ): 
\\	お父さん(おとうさん): 
\\	祖父(そふ): 
\\	叔父(おじ): 
\\	お祖父さん(おじいさん): 
\\	親父(おやじ): 
\\	伯父さん(おじさん): 
\\	父: 
\\	父	
\\	ちち	
\\	会意 
\\	姉	シ	あね、はは		
\\	姉(あね): 
\\	姉妹(しまい): 
\\	お姉さん(おねえさん): 
\\	従姉妹(いとこ): 
\\	女: 
\\	市: 
\\	女	
\\	おんな・おんなへん	
\\	形声 
\\	兄	ケイ、キョウ	あに	え, せ, よし	
\\	兄(あに): 
\\	兄弟(きょうだい): 
\\	お兄さん(おにいさん): 
\\	従兄弟(いとこ): 
\\	口: 
\\	八: 
\\	儿	
\\	にんにょう・ひとあし	
\\	会意 
\\	弟	テイ、ダイ、デ	おとうと	て	
\\	弟(おとうと): 
\\	兄弟(きょうだい): 
\\	弟子(でし): 
\\	従兄弟(いとこ): 
\\	八: 
\\	弔: 
\\	丿: 
\\	弓	
\\	ゆみ・ゆみへん	
\\	象形 
\\	妹	マイ	いもうと	す, せ, も	
\\	妹(いもうと): 
\\	姉妹(しまい): 
\\	従姉妹(いとこ): 
\\	女: 
\\	未: 
\\	女	
\\	おんな・おんなへん	
\\	形声 
\\	自	ジ、シ	みずか.ら、おの.ずから、おの.ずと	より	
\\	自ら(みずから): 
\\	自身(じしん): 
\\	自然(しぜん): 
\\	自宅(じたく): 
\\	自動車(じどうしゃ): 
\\	自分(じぶん): 
\\	自由(じゆう): 
\\	自己(じこ): 
\\	自治(じち): 
\\	自信(じしん): 
\\	独自(どくじ): 
\\	自主(じしゅ): 
\\	自転車(じてんしゃ): 
\\	自動(じどう): 
\\	自殺(じさつ): 
\\	自衛(じえい): 
\\	不自由(ふじゆう): 
\\	自覚(じかく): 
\\	自慢(じまん): 
\\	各自(かくじ): 
\\	自然科学(しぜんかがく): 
\\	自我(じが): 
\\	自立(じりつ): 
\\	自在(じざい): 
\\	自首(じしゅ): 
\\	自習(じしゅう): 
\\	自ずから(おのずから): 
\\	自惚れ(うぬぼれ): 
\\	自転(じてん): 
\\	自尊心(じそんしん): 
\\	自動詞(じどうし): 
\\	目: 
\\	自	
\\	みずから	
\\	象形 
\\	友	ユウ	とも	う, ど, ゆ	
\\	友人(ゆうじん): 
\\	友好(ゆうこう): 
\\	友(とも): 
\\	友達(ともだち): 
\\	友情(ゆうじょう): 
\\	親友(しんゆう): 
\\	又: 
\\	又	
\\	また	
\\	会意 
\\	体	タイ、テイ	からだ、かたち	なり	
\\	"身体(しんたい): 
\\	全体(ぜんたい): 
\\	体制(たいせい): 
\\	団体(だんたい): 
\\	体験(たいけん): 
\\	体育(たいいく): 
\\	一体(いったい): 
\\	主体(しゅたい): 
\\	体重(たいじゅう): 
\\	体力(たいりょく): 
\\	本体(ほんたい): 
\\	死体(したい): 
\\	体系(たいけい): 
\\	体操(たいそう): 
\\	立体(りったい): 
\\	人体(じんたい): 
\\	液体(えきたい): 
\\	肉体(にくたい): 
\\	大体(だいたい): 
\\	体温(たいおん): 
\\	天体(てんたい): 
\\	正体(しょうたい): 
\\	物体(ぶったい): 
\\	体格(たいかく): 
\\	固体(こたい): 
\\	文体(ぶんたい): 
\\	体裁(ていさい): 
\\	具体(ぐたい): 
\\	気体(きたい): 
\\	個体(こたい): 
\\	体積(たいせき): 
\\	体付き(からだつき): 
\\	勿体ない(もったいない): 
\\	字体(じたい): 
\\	体(てい): 
\\	人: 
\\	本: 
\\	骨	
\\	ほね・ほねへん	體	形声 
\\	毛	モウ	け	めん, も	
\\	"毛(け): 
\\	毛布(もうふ): 
\\	毛皮(けがわ): 
\\	髪の毛(かみのけ): 
\\	毛糸(けいと): 
\\	羊毛(ようもう): 
\\	毛: 
\\	毛	
\\	け	
\\	象形 
\\	頭	トウ、ズ、ト	あたま、かしら、-がしら、かぶり	かみ, がみ, ちゃん, つむり, づ	
\\	冒頭(ぼうとう): 
\\	先頭(せんとう): 
\\	街頭(がいとう): 
\\	頭痛(ずつう): 
\\	頭脳(ずのう): 
\\	口頭(こうとう): 
\\	頭(あたま): 
\\	頭(とう): 
\\	豆: 
\\	頁: 
\\	頁	
\\	おおがい・いちのかい	
\\	形声 
\\	顔	ガン	かお		
\\	顔(かお): 
\\	笑顔(えがお): 
\\	顔付き(かおつき): 
\\	彦: 
\\	頁: 
\\	頁	
\\	おおがい・いちのかい	
\\	形声 
\\	首	シュ	くび	おびと, こべ, す	
\\	首相(しゅしょう): 
\\	首脳(しゅのう): 
\\	首都(しゅと): 
\\	首(くび): 
\\	元首(げんしゅ): 
\\	手首(てくび): 
\\	自首(じしゅ): 
\\	首飾り(くびかざり): 
\\	部首(ぶしゅ): 
\\	首輪(くびわ): 
\\	自: 
\\	首	
\\	くび	
\\	象形 
\\	心	シン	こころ、-ごころ		
\\	"関心(かんしん): 
\\	中心(ちゅうしん): 
\\	心臓(しんぞう): 
\\	心配(しんぱい): 
\\	心理(しんり): 
\\	安心(あんしん): 
\\	都心(としん): 
\\	熱心(ねっしん): 
\\	肝心(かんじん): 
\\	感心(かんしん): 
\\	心身(しんしん): 
\\	心地(ここち): 
\\	野心(やしん): 
\\	良心(りょうしん): 
\\	心中(しんじゅう): 
\\	内心(ないしん): 
\\	決心(けっしん): 
\\	心得る(こころえる): 
\\	苦心(くしん): 
\\	心得(こころえ): 
\\	心がける(こころがける): 
\\	心掛け(こころがけ): 
\\	心情(しんじょう): 
\\	用心(ようじん): 
\\	一心(いっしん): 
\\	心当たり(こころあたり): 
\\	真心(まごころ): 
\\	心強い(こころづよい): 
\\	心細い(こころぼそい): 
\\	心(こころ): 
\\	心(しん): 
\\	下心(したごころ): 
\\	自尊心(じそんしん): 
\\	心: 
\\	心	
\\	こころ・したごころ・りっしんべん	
\\	象形 
\\	時	ジ	とき、-どき	と, とぎ	
\\	時間(じかん): 
\\	時期(じき): 
\\	時代(じだい): 
\\	当時(とうじ): 
\\	同時(どうじ): 
\\	一時(いちじ): 
\\	臨時(りんじ): 
\\	時(とき): 
\\	時計(とけい): 
\\	時速(じそく): 
\\	時々(ときどき): 
\\	時刻(じこく): 
\\	日時(にちじ): 
\\	時折(ときおり): 
\\	時差(じさ): 
\\	時間割り(じかんわり): 
\\	何時(いつ): 
\\	何時までも(いつまでも): 
\\	何時も(いつも): 
\\	何時でも(いつでも): 
\\	時刻表(じこくひょう): 
\\	何時か(いつか): 
\\	時(じ): 
\\	時間目(じかんめ): 
\\	日: 
\\	寺: 
\\	日	
\\	ひ・ひへん・にち・にちへん	
\\	形声 
\\	曜	ヨウ		てる	
\\	土曜(どよう): 
\\	曜日(ようび): 
\\	日曜(にちよう): 
\\	金曜(きんよう): 
\\	月曜(げつよう): 
\\	水曜(すいよう): 
\\	木曜(もくよう): 
\\	火曜日(かようび): 
\\	火曜(かよう): 
\\	金曜日(きんようび): 
\\	月曜日(げつようび): 
\\	水曜日(すいようび): 
\\	土曜日(どようび): 
\\	日曜日(にちようび): 
\\	木曜日(もくようび): 
\\	日: 
\\	羽: 
\\	隹: 
\\	日	
\\	ひ・ひへん・にち・にちへん	
\\	形声 
\\	朝	チョウ	あさ	あ, あそ, ささ, ちか, とも	
\\	朝(あさ): 
\\	毎朝(まいあさ): 
\\	今朝(けさ): 
\\	朝ご飯(あさごはん): 
\\	朝寝坊(あさねぼう): 
\\	十: 
\\	早: 
\\	肉: 
\\	月	
\\	つき・つきへん	
\\	会意 
\\	昼	チュウ	ひる		
\\	昼(ひる): 
\\	昼食(ちゅうしょく): 
\\	昼間(ひるま): 
\\	昼休み(ひるやすみ): 
\\	昼寝(ひるね): 
\\	お昼(おひる): 
\\	昼飯(ひるめし): 
\\	昼ご飯(ひるごはん): 
\\	尺: 
\\	旦: 
\\	日	
\\	ひ・ひへん・にち・にちへん	晝	会意 
\\	夜	ヤ	よ、よる		
\\	夜(よる): 
\\	夜間(やかん): 
\\	徹夜(てつや): 
\\	夜中(よなか): 
\\	夜明け(よあけ): 
\\	今夜(こんや): 
\\	深夜(しんや): 
\\	日夜(にちや): 
\\	夜更かし(よふかし): 
\\	夜具(やぐ): 
\\	夜更け(よふけ): 
\\	夜行(やこう): 
\\	夜中(やちゅう): 
\\	亠: 
\\	人: 
\\	夕: 
\\	夕	
\\	ゆう・ゆうべ・た	
\\	会意 
\\	分	ブン、フン、ブ	わ.ける、わ.け、わ.かれる、わ.かる、わ.かつ	いた, わけ	
\\	大分(だいぶ): 
\\	引分け(ひきわけ): 
\\	自分(じぶん): 
\\	部分(ぶぶん): 
\\	分野(ぶんや): 
\\	十分(じゅうぶん): 
\\	処分(しょぶん): 
\\	半分(はんぶん): 
\\	気分(きぶん): 
\\	分析(ぶんせき): 
\\	配分(はいぶん): 
\\	分離(ぶんり): 
\\	分裂(ぶんれつ): 
\\	分ける(わける): 
\\	成分(せいぶん): 
\\	分担(ぶんたん): 
\\	大部分(だいぶぶん): 
\\	身分(みぶん): 
\\	分子(ぶんし): 
\\	分配(ぶんぱい): 
\\	分かれる(わかれる): 
\\	水分(すいぶん): 
\\	分散(ぶんさん): 
\\	区分(くぶん): 
\\	多分(たぶん): 
\\	分布(ぶんぷ): 
\\	分類(ぶんるい): 
\\	分解(ぶんかい): 
\\	分業(ぶんぎょう): 
\\	分量(ぶんりょう): 
\\	一部分(いちぶぶん): 
\\	幾分(いくぶん): 
\\	手分け(てわけ): 
\\	取り分け(とりわけ): 
\\	随分(ずいぶん): 
\\	等分(とうぶん): 
\\	何分(なにぶん): 
\\	分母(ぶんぼ): 
\\	余分(よぶん): 
\\	分(ふん): 
\\	分(ぶん): 
\\	分数(ぶんすう): 
\\	申し分(もうしぶん): 
\\	養分(ようぶん): 
\\	十分(じっぷん): 
\\	分(ぶ): 
\\	八: 
\\	刀: 
\\	刀	
\\	かたな・りっとう	
\\	会意 
\\	週	シュウ			
\\	週(しゅう): 
\\	今週(こんしゅう): 
\\	先週(せんしゅう): 
\\	毎週(まいしゅう): 
\\	来週(らいしゅう): 
\\	週間(しゅうかん): 
\\	隔週(かくしゅう): 
\\	再来週(さらいしゅう): 
\\	先先週(せんせんしゅう): 
\\	周: 
\\	辶: 
\\	辵	
\\	しんにょう・しんにゅう	
\\	形声 
\\	春	シュン	はる	あずま, かす, すの, ひ, わら	
\\	春(はる): 
\\	青春(せいしゅん): 
\\	三: 
\\	大: 
\\	日: 
\\	日	
\\	ひ・ひへん・にち・にちへん	
\\	形声 
\\	夏	カ、ガ、ゲ	なつ		
\\	夏(なつ): 
\\	夏休み(なつやすみ): 
\\	一: 
\\	自: 
\\	夂: 
\\	夊	
\\	なつあし・すいにょう	
\\	象形 
\\	秋	シュウ	あき、とき	あい	
\\	秋(あき): 
\\	禾: 
\\	火: 
\\	禾	
\\	のぎ・のぎへん	
\\	会意 
\\	冬	トウ	ふゆ		
\\	冬(ふゆ): 
\\	冬眠(とうみん): 
\\	夂: 
\\	冫: 
\\	冫	
\\	にすい	
\\	象形 
\\	今	コン、キン	いま	な	
\\	今日(きょう): 
\\	今回(こんかい): 
\\	今月(こんげつ): 
\\	今後(こんご): 
\\	今年(ことし): 
\\	今度(こんど): 
\\	今週(こんしゅう): 
\\	今(いま): 
\\	今夜(こんや): 
\\	今朝(けさ): 
\\	今に(いまに): 
\\	今晩(こんばん): 
\\	今にも(いまにも): 
\\	今更(いまさら): 
\\	今日は(こんにちは): 
\\	今晩は(こんばんは): 
\\	ただ今(ただいま): 
\\	今(こん): 
\\	人: 
\\	一: 
\\	人	
\\	ひと・にんべん・ひとやね	
\\	仮借 
\\	新	シン	あたら.しい、あら.た、あら-、にい-	あせ, あたらし, し, に, にっ, につ, よし	
\\	新た(あらた): 
\\	新聞(しんぶん): 
\\	新人(しんじん): 
\\	新幹線(しんかんせん): 
\\	革新(かくしん): 
\\	新聞社(しんぶんしゃ): 
\\	新鮮(しんせん): 
\\	新しい(あたらしい): 
\\	新興(しんこう): 
\\	新築(しんちく): 
\\	新婚(しんこん): 
\\	新(しん): 
\\	新入生(しんにゅうせい): 
\\	立: 
\\	木: 
\\	斤: 
\\	斤	
\\	おの・おのづくり・きん	
\\	会意 
\\	古	コ	ふる.い、ふる-、-ふる.す	ふゆ	
\\	古里(ふるさと): 
\\	古代(こだい): 
\\	古典(こてん): 
\\	中古(ちゅうこ): 
\\	古い(ふるい): 
\\	考古学(こうこがく): 
\\	古(いにしえ): 
\\	稽古(けいこ): 
\\	古(ふる): 
\\	十: 
\\	口: 
\\	口	
\\	くち・くちへん	
\\	会意 
\\	間	カン、ケン	あいだ、ま、あい	ちか, は, はざ, はし	
\\	期間(きかん): 
\\	時間(じかん): 
\\	人間(にんげん): 
\\	年間(ねんかん): 
\\	民間(みんかん): 
\\	中間(ちゅうかん): 
\\	仲間(なかま): 
\\	空間(くうかん): 
\\	瞬間(しゅんかん): 
\\	間(あいだ): 
\\	間も無く(まもなく): 
\\	世間(せけん): 
\\	夜間(やかん): 
\\	間違い(まちがい): 
\\	区間(くかん): 
\\	昼間(ひるま): 
\\	合間(あいま): 
\\	手間(てま): 
\\	間隔(かんかく): 
\\	間接(かんせつ): 
\\	居間(いま): 
\\	週間(しゅうかん): 
\\	茶の間(ちゃのま): 
\\	間に合う(まにあう): 
\\	床の間(とこのま): 
\\	時間割り(じかんわり): 
\\	いつの間にか(いつのまにか): 
\\	つかの間(つかのま): 
\\	間違える(まちがえる): 
\\	間違う(まちがう): 
\\	隙間(すきま): 
\\	この間(このあいだ): 
\\	間(ま): 
\\	間柄(あいだがら): 
\\	客間(きゃくま): 
\\	貸間(かしま): 
\\	間誤付く(まごつく): 
\\	此間(こないだ): 
\\	間々(まま): 
\\	空き間(あきま): 
\\	間(かん): 
\\	時間目(じかんめ): 
\\	門: 
\\	日: 
\\	門	
\\	もん・もんがまえ・かどがまえ	
\\	会意 
\\	方	ホウ	かた、-かた、-がた	から, な, なた, ふさ, まさ, みち, も, わ	
\\	"一方(いっぽう): 
\\	見方(みかた): 
\\	地方(ちほう): 
\\	方向(ほうこう): 
\\	方針(ほうしん): 
\\	方法(ほうほう): 
\\	行方(ゆくえ): 
\\	仕方(しかた): 
\\	方面(ほうめん): 
\\	夕方(ゆうがた): 
\\	両方(りょうほう): 
\\	方々(かたがた): 
\\	味方(みかた): 
\\	他方(たほう): 
\\	大方(おおかた): 
\\	方(ほう): 
\\	方言(ほうげん): 
\\	明け方(あけがた): 
\\	彼方此方(あちらこちら): 
\\	方式(ほうしき): 
\\	方策(ほうさく): 
\\	彼方(あちら): 
\\	長方形(ちょうほうけい): 
\\	方程式(ほうていしき): 
\\	遠方(えんぽう): 
\\	方角(ほうがく): 
\\	彼方(かなた): 
\\	正方形(せいほうけい): 
\\	平方(へいほう): 
\\	此方(こちら): 
\\	其方(そちら): 
\\	何方(どちら): 
\\	仕方がない(しかたがない): 
\\	何方(どなた): 
\\	貴方(あなた): 
\\	方(かた): 
\\	目方(めかた): 
\\	方々(ほうぼう): 
\\	立方(たちかた): 
\\	外方(そっぽ): 
\\	立方(りっぽう): 
\\	泳ぎ方(およぎかた): 
\\	亠: 
\\	方	
\\	ほう・ほうへん・かた・かたへん	
\\	象形 
\\	北	ホク	きた	きら, ほう, ほっ, ほつ	
\\	北(きた): 
\\	南北(なんぼく): 
\\	敗北(はいぼく): 
\\	北極(ほっきょく): 
\\	匕: 
\\	匕	
\\	さじ・さじのひ・さじづくり・ひ	
\\	会意 
\\	南	ナン、ナ	みなみ	なみ, は, みな, みまみ	
\\	南(みなみ): 
\\	南北(なんぼく): 
\\	南極(なんきょく): 
\\	南米(なんべい): 
\\	十: 
\\	冂: 
\\	干: 
\\	十	
\\	じゅう・じゅうへん	
\\	象形 
\\	東	トウ	ひがし	あい, あがり, あずま, あづま, こ, さき, しの, とお, はる, ひが, もと	
\\	東(ひがし): 
\\	東西(とうざい): 
\\	東洋(とうよう): 
\\	関東(かんとう): 
\\	東(あずま): 
\\	木: 
\\	日: 
\\	木	
\\	き・きへん	
\\	仮借 
\\	西	セイ、サイ、ス	にし	いり, ひし, むら	
\\	西(にし): 
\\	東西(とうざい): 
\\	西洋(せいよう): 
\\	西暦(せいれき): 
\\	関西(かんさい): 
\\	西日(にしび): 
\\	襾: 
\\	襾	
\\	にし・かなめのかしら・おおいかんむり	
\\	仮借 
\\	遠	エン、オン	とお.い	お, おに, ど, どお	
\\	永遠(えいえん): 
\\	望遠鏡(ぼうえんきょう): 
\\	遠慮(えんりょ): 
\\	遠足(えんそく): 
\\	遠方(えんぽう): 
\\	待ち遠しい(まちどおしい): 
\\	遠ざかる(とおざかる): 
\\	遠い(とおい): 
\\	ご遠慮ください(ごえんりょください): 
\\	遠く(とおく): 
\\	遠回り(とおまわり): 
\\	お待ち遠様(おまちどおさま): 
\\	袁: 
\\	辶: 
\\	辵	
\\	しんにょう・しんにゅう	
\\	形声 
\\	近	キン、コン	ちか.い	おう, おお, この	
\\	"最近(さいきん): 
\\	近代(きんだい): 
\\	付近(ふきん): 
\\	近所(きんじょ): 
\\	近郊(きんこう): 
\\	接近(せっきん): 
\\	近い(ちかい): 
\\	身近(みぢか): 
\\	近づける(ちかづける): 
\\	近づく(ちかづく): 
\\	近々(ちかぢか): 
\\	近寄る(ちかよる): 
\\	近視(きんし): 
\\	近く(ちかく): 
\\	近眼(きんがん): 
\\	近頃(ちかごろ): 
\\	手近(てぢか): 
\\	斤: 
\\	辶: 
\\	辵	
\\	しんにょう・しんにゅう	
\\	形声 
\\	前	ゼン	まえ、-まえ	さき, さと, まい	
\\	"当り前(あたりまえ): 
\\	建て前(たてまえ): 
\\	午前(ごぜん): 
\\	前提(ぜんてい): 
\\	直前(ちょくぜん): 
\\	名前(なまえ): 
\\	以前(いぜん): 
\\	前後(ぜんご): 
\\	前進(ぜんしん): 
\\	手前(てまえ): 
\\	前例(ぜんれい): 
\\	前途(ぜんと): 
\\	前者(ぜんしゃ): 
\\	前売り(まえうり): 
\\	腕前(うでまえ): 
\\	お前(おまえ): 
\\	事前(じぜん): 
\\	前もって(まえもって): 
\\	前置き(まえおき): 
\\	前(ぜん): 
\\	前(まえ): 
\\	真ん前(まんまえ): 
\\	八: 
\\	月: 
\\	刀: 
\\	刀	
\\	かたな・りっとう	
\\	会意 
\\	後	ゴ、コウ	のち、うし.ろ、うしろ、あと、おく.れる	こし, し, しい, しり	
\\	"午後(ごご): 
\\	後(あと): 
\\	今後(こんご): 
\\	最後(さいご): 
\\	直後(ちょくご): 
\\	以後(いご): 
\\	後ろ(うしろ): 
\\	前後(ぜんご): 
\\	後輩(こうはい): 
\\	後者(こうしゃ): 
\\	後回し(あとまわし): 
\\	後退(こうたい): 
\\	背後(はいご): 
\\	後悔(こうかい): 
\\	明後日(あさって): 
\\	後(のち): 
\\	明明後日(しあさって): 
\\	産後(さんご): 
\\	後(ご): 
\\	彳: 
\\	幺: 
\\	夂: 
\\	彳	
\\	ぎょうにんべん	
\\	会意 
\\	内	ナイ、ダイ	うち	いと, ただ, ち, のち	
\\	内閣(ないかく): 
\\	内容(ないよう): 
\\	以内(いない): 
\\	内部(ないぶ): 
\\	案内(あんない): 
\\	内(うち): 
\\	内科(ないか): 
\\	内陸(ないりく): 
\\	内心(ないしん): 
\\	家内(かない): 
\\	内訳(うちわけ): 
\\	内線(ないせん): 
\\	内蔵(ないぞう): 
\\	内乱(ないらん): 
\\	その内(そのうち): 
\\	内緒(ないしょ): 
\\	内(ない): 
\\	冂: 
\\	人: 
\\	入	
\\	いる・にゅう・いりがしら・いりやね	
\\	象形 
\\	外	ガイ、ゲ	そと、ほか、はず.す、はず.れる、と-	うい, け, ふか	
\\	海外(かいがい): 
\\	外交(がいこう): 
\\	外国(がいこく): 
\\	外国人(がいこくじん): 
\\	外相(がいしょう): 
\\	外(ほか): 
\\	意外(いがい): 
\\	郊外(こうがい): 
\\	例外(れいがい): 
\\	外貨(がいか): 
\\	外部(がいぶ): 
\\	外科(げか): 
\\	外出(がいしゅつ): 
\\	野外(やがい): 
\\	屋外(おくがい): 
\\	外す(はずす): 
\\	外観(がいかん): 
\\	案外(あんがい): 
\\	除外(じょがい): 
\\	外れる(はずれる): 
\\	外来(がいらい): 
\\	以外(いがい): 
\\	外(そと): 
\\	課外(かがい): 
\\	外方(そっぽ): 
\\	外(がい): 
\\	夕: 
\\	卜: 
\\	夕	
\\	ゆう・ゆうべ・た	
\\	会意 
\\	場	ジョウ、チョウ	ば		
\\	工場(こうじょう): 
\\	牧場(まきば): 
\\	現場(げんば): 
\\	市場(いちば): 
\\	場所(ばしょ): 
\\	立場(たちば): 
\\	会場(かいじょう): 
\\	劇場(げきじょう): 
\\	出場(しゅつじょう): 
\\	場面(ばめん): 
\\	職場(しょくば): 
\\	相場(そうば): 
\\	入場(にゅうじょう): 
\\	広場(ひろば): 
\\	駐車場(ちゅうしゃじょう): 
\\	登場(とうじょう): 
\\	売り場(うりば): 
\\	場合(ばあい): 
\\	農場(のうじょう): 
\\	本場(ほんば): 
\\	飛行場(ひこうじょう): 
\\	道場(どうじょう): 
\\	役場(やくば): 
\\	式場(しきじょう): 
\\	酒場(さかば): 
\\	満場(まんじょう): 
\\	市場(しじょう): 
\\	場(ば): 
\\	来場(らいじょう): 
\\	土: 
\\	昜: 
\\	土	
\\	つち・つちへん・どへん	
\\	形声 
\\	地	チ、ジ		どま	
\\	現地(げんち): 
\\	地域(ちいき): 
\\	地球(ちきゅう): 
\\	地元(じもと): 
\\	地方(ちほう): 
\\	土地(とち): 
\\	基地(きち): 
\\	地震(じしん): 
\\	地(ち): 
\\	地位(ちい): 
\\	地下(ちか): 
\\	地帯(ちたい): 
\\	地下鉄(ちかてつ): 
\\	地図(ちず): 
\\	地点(ちてん): 
\\	敷地(しきち): 
\\	余地(よち): 
\\	産地(さんち): 
\\	植民地(しょくみんち): 
\\	農地(のうち): 
\\	団地(だんち): 
\\	地獄(じごく): 
\\	墓地(ぼち): 
\\	地味(じみ): 
\\	地理(ちり): 
\\	意地(いじ): 
\\	地面(じめん): 
\\	生地(きじ): 
\\	見地(けんち): 
\\	心地(ここち): 
\\	地形(ちけい): 
\\	地主(じぬし): 
\\	地名(ちめい): 
\\	遊園地(ゆうえんち): 
\\	地質(ちしつ): 
\\	盆地(ぼんち): 
\\	地区(ちく): 
\\	地盤(じばん): 
\\	地下水(ちかすい): 
\\	地平線(ちへいせん): 
\\	下地(したじ): 
\\	天地(てんち): 
\\	無地(むじ): 
\\	意地悪(いじわる): 
\\	番地(ばんち): 
\\	各地(かくち): 
\\	耕地(こうち): 
\\	領地(りょうち): 
\\	畑地(はたち): 
\\	土: 
\\	也: 
\\	土	
\\	つち・つちへん・どへん	
\\	形声 
\\	国	コク	くに	くな, こ	
\\	外国(がいこく): 
\\	外国人(がいこくじん): 
\\	国家(こっか): 
\\	国会(こっかい): 
\\	国際(こくさい): 
\\	国民(こくみん): 
\\	国連(こくれん): 
\\	全国(ぜんこく): 
\\	国(くに): 
\\	国境(こっきょう): 
\\	国籍(こくせき): 
\\	国防(こくぼう): 
\\	国立(こくりつ): 
\\	国交(こっこう): 
\\	国産(こくさん): 
\\	国土(こくど): 
\\	本国(ほんごく): 
\\	国語(こくご): 
\\	天国(てんごく): 
\\	国有(こくゆう): 
\\	国王(こくおう): 
\\	母国(ぼこく): 
\\	国定(こくてい): 
\\	囗: 
\\	玉: 
\\	囗	
\\	くに・くにがまえ	國	会意 
\\	園	エン	その	おん, ぞの	
\\	園(えん): 
\\	公園(こうえん): 
\\	動物園(どうぶつえん): 
\\	幼稚園(ようちえん): 
\\	田園(でんえん): 
\\	園芸(えんげい): 
\\	遊園地(ゆうえんち): 
\\	囗: 
\\	袁: 
\\	囗	
\\	くに・くにがまえ	
\\	形声 
\\	谷	コク	たに、きわ.まる	がい, がえ, がや, せ, たり, たん, や	
\\	谷(たに): 
\\	口: 
\\	谷	
\\	たに・たにへん	
\\	象形 
\\	野	ヤ、ショ	の、の-	ずけ, つけ, ぬ	
\\	分野(ぶんや): 
\\	野党(やとう): 
\\	野菜(やさい): 
\\	野(の): 
\\	視野(しや): 
\\	平野(へいや): 
\\	野外(やがい): 
\\	野心(やしん): 
\\	野生(やせい): 
\\	里: 
\\	予: 
\\	里	
\\	さと・さとへん	
\\	形声 
\\	原	ゲン	はら	た, ばる, ら, わた, わら	
\\	原因(げんいん): 
\\	原則(げんそく): 
\\	原油(げんゆ): 
\\	原理(げんり): 
\\	原点(げんてん): 
\\	原爆(げんばく): 
\\	原稿(げんこう): 
\\	原作(げんさく): 
\\	原料(げんりょう): 
\\	高原(こうげん): 
\\	原子(げんし): 
\\	原始(げんし): 
\\	原型(げんけい): 
\\	原産(げんさん): 
\\	原形(げんけい): 
\\	原文(げんぶん): 
\\	原っぱ(はらっぱ): 
\\	原書(げんしょ): 
\\	原(はら): 
\\	原典(げんてん): 
\\	原(げん): 
\\	厂: 
\\	白: 
\\	小: 
\\	厂	
\\	がんだれ	
\\	象形 
\\	里	リ	さと	さ	
\\	古里(ふるさと): 
\\	郷里(きょうり): 
\\	里: 
\\	里	
\\	さと・さとへん	
\\	会意 
\\	市	シ	いち	い, ち	
\\	市場(いちば): 
\\	市民(しみん): 
\\	都市(とし): 
\\	市街(しがい): 
\\	市(いち): 
\\	市(し): 
\\	市場(しじょう): 
\\	亠: 
\\	巾: 
\\	巾	
\\	はば・はばへん・きんべん	
\\	象形 
\\	京	キョウ、ケイ、キン	みやこ	たか	
\\	帰京(ききょう): 
\\	亠: 
\\	口: 
\\	小: 
\\	亠	
\\	けいさん・けいさんかんむり・なべぶた	
\\	象形 
\\	風	フウ、フ	かぜ、かざ-、-かぜ	い, え	
\\	風景(ふうけい): 
\\	台風(たいふう): 
\\	風邪(かぜ): 
\\	風俗(ふうぞく): 
\\	風船(ふうせん): 
\\	和風(わふう): 
\\	風呂(ふろ): 
\\	風土(ふうど): 
\\	扇風機(せんぷうき): 
\\	洋風(ようふう): 
\\	暴風(ぼうふう): 
\\	風習(ふうしゅう): 
\\	風(かぜ): 
\\	風(ふう): 
\\	風呂敷(ふろしき): 
\\	風車(かざぐるま): 
\\	気風(きふう): 
\\	お風呂(おふろ): 
\\	几: 
\\	虫: 
\\	風	
\\	かぜ	
\\	形声 
\\	雪	セツ	ゆき	せっ, ぶき	
\\	雪(ゆき): 
\\	吹雪(ふぶき): 
\\	雪崩(なだれ): 
\\	雨: 
\\	彑: 
\\	雨	
\\	あめ・あめかんむり・あまかんむり	
\\	象形 
\\	雲	ウン	くも、-ぐも	き, ずも, のめ	
\\	雲(くも): 
\\	雨: 
\\	云: 
\\	雨	
\\	あめ・あめかんむり・あまかんむり	
\\	形声 
\\	池	チ	いけ		
\\	池(いけ): 
\\	電池(でんち): 
\\	乾電池(かんでんち): 
\\	水: 
\\	也: 
\\	水	
\\	みず・したみず・さんずい	
\\	形声 
\\	海	カイ	うみ	あ, あま, うな, うん, え, か, た, ひろ, ひろし, ぶ, まち, まま, み, め, わたる	
\\	海外(かいがい): 
\\	海(うみ): 
\\	海岸(かいがん): 
\\	海洋(かいよう): 
\\	海峡(かいきょう): 
\\	航海(こうかい): 
\\	領海(りょうかい): 
\\	海運(かいうん): 
\\	海水浴(かいすいよく): 
\\	海流(かいりゅう): 
\\	海抜(かいばつ): 
\\	海路(かいろ): 
\\	海鷂魚(えい): 
\\	水: 
\\	毎: 
\\	水	
\\	みず・したみず・さんずい	海	形声 
\\	岩	ガン	いわ		
\\	岩(いわ): 
\\	溶岩(ようがん): 
\\	岩石(がんせき): 
\\	山: 
\\	石: 
\\	山	
\\	やま・やまへん・やまかんむり	
\\	象形 
\\	星	セイ、ショウ	ほし、-ぼし		
\\	衛星(えいせい): 
\\	星(ほし): 
\\	火星(かせい): 
\\	星座(せいざ): 
\\	惑星(わくせい): 
\\	日: 
\\	生: 
\\	日	
\\	ひ・ひへん・にち・にちへん	
\\	形声 
\\	室	シツ	むろ		
\\	"教室(きょうしつ): 
\\	温室(おんしつ): 
\\	控え室(ひかえしつ): 
\\	浴室(よくしつ): 
\\	会議室(かいぎしつ): 
\\	研究室(けんきゅうしつ): 
\\	室(しつ): 
\\	待合室(まちあいしつ): 
\\	宀: 
\\	至: 
\\	宀	
\\	うかんむり	
\\	会意 
\\	戸	コ	と	え, へ	
\\	戸(と): 
\\	井戸(いど): 
\\	戸籍(こせき): 
\\	雨戸(あまど): 
\\	戸棚(とだな): 
\\	戸締まり(とじまり): 
\\	瀬戸物(せともの): 
\\	戸: 
\\	戸	
\\	と・とだれ・とかんむり	
\\	象形 
\\	家	カ、ケ	いえ、や、うち	あり, え, く, つか, べ	
\\	家族(かぞく): 
\\	家庭(かてい): 
\\	国家(こっか): 
\\	作家(さっか): 
\\	農家(のうか): 
\\	一家(いっか): 
\\	画家(がか): 
\\	家事(かじ): 
\\	家賃(やちん): 
\\	実家(じっか): 
\\	家具(かぐ): 
\\	家計(かけい): 
\\	家屋(かおく): 
\\	家畜(かちく): 
\\	大家(たいか): 
\\	家主(やぬし): 
\\	家内(かない): 
\\	家出(いえで): 
\\	実業家(じつぎょうか): 
\\	貸家(かしや): 
\\	家(いえ): 
\\	家(うち): 
\\	家(け): 
\\	家来(けらい): 
\\	家(か): 
\\	宀: 
\\	豕: 
\\	宀	
\\	うかんむり	
\\	会意 
\\	寺	ジ	てら		
\\	寺(てら): 
\\	寺院(じいん): 
\\	寺(じ): 
\\	土: 
\\	寸: 
\\	寸	
\\	すん・すんづくり	
\\	形声 
\\	通	ツウ、ツ	とお.る、とお.り、-とお.り、-どお.り、とお.す、とお.し、-どお.し、かよ.う	とん, どうし, どおり, みち	
\\	通信(つうしん): 
\\	見通し(みとおし): 
\\	共通(きょうつう): 
\\	交通(こうつう): 
\\	通貨(つうか): 
\\	通常(つうじょう): 
\\	普通(ふつう): 
\\	流通(りゅうつう): 
\\	通過(つうか): 
\\	通勤(つうきん): 
\\	通知(つうち): 
\\	通訳(つうやく): 
\\	通行(つうこう): 
\\	通学(つうがく): 
\\	通路(つうろ): 
\\	通る(とおる): 
\\	通帳(つうちょう): 
\\	融通(ゆうずう): 
\\	直通(ちょくつう): 
\\	不通(ふつう): 
\\	通う(かよう): 
\\	通す(とおす): 
\\	大通り(おおどおり): 
\\	似通う(にかよう): 
\\	開通(かいつう): 
\\	人通り(ひとどおり): 
\\	通り過ぎる(とおりすぎる): 
\\	交通機関(こうつうきかん): 
\\	通(つう): 
\\	通用(つうよう): 
\\	透き通る(すきとおる): 
\\	一通り(ひととおり): 
\\	通じる(つうじる): 
\\	通り(とおり): 
\\	通り掛かる(とおりかかる): 
\\	遣り通す(やりとおす): 
\\	通ずる(つうずる): 
\\	甬: 
\\	辶: 
\\	辵	
\\	しんにょう・しんにゅう	
\\	形声 
\\	門	モン	かど、と	じょう, も, と, ゆき	
\\	部門(ぶもん): 
\\	専門(せんもん): 
\\	門(もん): 
\\	正門(せいもん): 
\\	門: 
\\	門	
\\	もん・もんがまえ・かどがまえ	
\\	象形 
\\	道	ドウ、トウ	みち	さ, じ, ど, みつ	
\\	"道(みち): 
\\	報道(ほうどう): 
\\	道路(どうろ): 
\\	鉄道(てつどう): 
\\	軌道(きどう): 
\\	柔道(じゅうどう): 
\\	水道(すいどう): 
\\	書道(しょどう): 
\\	道具(どうぐ): 
\\	歩道(ほどう): 
\\	街道(かいどう): 
\\	道徳(どうとく): 
\\	道場(どうじょう): 
\\	赤道(せきどう): 
\\	使い道(つかいみち): 
\\	片道(かたみち): 
\\	回り道(まわりみち): 
\\	道順(みちじゅん): 
\\	道(どう): 
\\	首: 
\\	辶: 
\\	辵	
\\	しんにょう・しんにゅう	
\\	会意 
\\	話	ワ	はな.す、はなし		
\\	話合い(はなしあい): 
\\	電話(でんわ): 
\\	話(はなし): 
\\	対話(たいわ): 
\\	話題(わだい): 
\\	会話(かいわ): 
\\	世話(せわ): 
\\	神話(しんわ): 
\\	童話(どうわ): 
\\	話し合う(はなしあう): 
\\	受話器(じゅわき): 
\\	話す(はなす): 
\\	話し掛ける(はなしかける): 
\\	話し中(はなしちゅう): 
\\	お世話になる(おせわになる): 
\\	お世話(おせわ): 
\\	言: 
\\	舌: 
\\	言	
\\	こと・げん・ごんべん・ことば	
\\	形声 
\\	言	ゲン、ゴン	い.う、こと	とき	
\\	言う(いう): 
\\	予言(よげん): 
\\	言わば(いわば): 
\\	言葉(ことば): 
\\	宣言(せんげん): 
\\	発言(はつげん): 
\\	言語(げんご): 
\\	一言(ひとこと): 
\\	言論(げんろん): 
\\	無言(むごん): 
\\	助言(じょげん): 
\\	方言(ほうげん): 
\\	言い訳(いいわけ): 
\\	伝言(でんごん): 
\\	言葉遣い(ことばづかい): 
\\	証言(しょうげん): 
\\	断言(だんげん): 
\\	独り言(ひとりごと): 
\\	言い出す(いいだす): 
\\	言付ける(ことづける): 
\\	そう言えば(そういえば): 
\\	片言(かたこと): 
\\	言い付ける(いいつける): 
\\	言伝(ことづて): 
\\	言付け(ことづけ): 
\\	伝言(つてこと): 
\\	口: 
\\	言	
\\	こと・げん・ごんべん・ことば	
\\	象形 
\\	答	トウ	こた.える、こた.え	どう	
\\	答(こたえ): 
\\	回答(かいとう): 
\\	解答(かいとう): 
\\	答案(とうあん): 
\\	返答(へんとう): 
\\	問答(もんどう): 
\\	答える(こたえる): 
\\	竹: 
\\	合: 
\\	竹	
\\	たけ・たけかんむり	
\\	形声 
\\	声	セイ、ショウ	こえ、こわ-		
\\	声(こえ): 
\\	声明(せいめい): 
\\	歓声(かんせい): 
\\	土: 
\\	耳	
\\	みみ・みみへん	聲	会意 
\\	聞	ブン、モン	き.く、き.こえる		
\\	新聞(しんぶん): 
\\	新聞社(しんぶんしゃ): 
\\	聞き取り(ききとり): 
\\	聞こえる(きこえる): 
\\	聞く(きく): 
\\	門: 
\\	耳: 
\\	耳	
\\	みみ・みみへん	
\\	形声 
\\	語	ゴ	かた.る、かた.らう		
\\	英語(えいご): 
\\	物語(ものがたり): 
\\	言語(げんご): 
\\	用語(ようご): 
\\	国語(こくご): 
\\	語る(かたる): 
\\	語学(ごがく): 
\\	物語る(ものがたる): 
\\	単語(たんご): 
\\	語(ご): 
\\	標語(ひょうご): 
\\	語源(ごげん): 
\\	日本語(にほんご): 
\\	敬語(けいご): 
\\	熟語(じゅくご): 
\\	略語(りゃくご): 
\\	文語(ぶんご): 
\\	漢語(かんご): 
\\	語彙(ごい): 
\\	語句(ごく): 
\\	主語(しゅご): 
\\	述語(じゅつご): 
\\	言: 
\\	吾: 
\\	言	
\\	こと・げん・ごんべん・ことば	
\\	形声 
\\	読	ドク、トク、トウ	よ.む、-よ.み	よみ	
\\	読者(どくしゃ): 
\\	読書(どくしょ): 
\\	読み(よみ): 
\\	読む(よむ): 
\\	朗読(ろうどく): 
\\	購読(こうどく): 
\\	読み上げる(よみあげる): 
\\	句読点(くとうてん): 
\\	講読(こうどく): 
\\	言: 
\\	売: 
\\	言	
\\	こと・げん・ごんべん・ことば	讀	形声 
\\	書	ショ	か.く、-が.き、-がき	かき	
\\	秘書(ひしょ): 
\\	文書(ぶんしょ): 
\\	教科書(きょうかしょ): 
\\	書店(しょてん): 
\\	著書(ちょしょ): 
\\	書類(しょるい): 
\\	図書館(としょかん): 
\\	読書(どくしょ): 
\\	書道(しょどう): 
\\	書籍(しょせき): 
\\	図書(としょ): 
\\	辞書(じしょ): 
\\	投書(とうしょ): 
\\	書物(しょもつ): 
\\	聖書(せいしょ): 
\\	書留(かきとめ): 
\\	願書(がんしょ): 
\\	書斎(しょさい): 
\\	書評(しょひょう): 
\\	原書(げんしょ): 
\\	書き取り(かきとり): 
\\	下書き(したがき): 
\\	書き取る(かきとる): 
\\	書く(かく): 
\\	清書(せいしょ): 
\\	葉書(はがき): 
\\	箇条書き(かじょうがき): 
\\	書(しょ): 
\\	聿: 
\\	日: 
\\	曰	
\\	ひらび・いわく	
\\	会意 
\\	記	キ	しる.す	のり	
\\	記事(きじ): 
\\	記者(きしゃ): 
\\	記録(きろく): 
\\	記念(きねん): 
\\	記憶(きおく): 
\\	記述(きじゅつ): 
\\	日記(にっき): 
\\	記す(しるす): 
\\	記号(きごう): 
\\	記入(きにゅう): 
\\	筆記(ひっき): 
\\	記載(きさい): 
\\	伝記(でんき): 
\\	暗記(あんき): 
\\	記名(きめい): 
\\	言: 
\\	己: 
\\	言	
\\	こと・げん・ごんべん・ことば	
\\	形声 
\\	紙	シ	かみ		
\\	紙(かみ): 
\\	手紙(てがみ): 
\\	用紙(ようし): 
\\	紙幣(しへい): 
\\	表紙(ひょうし): 
\\	張り紙(はりがみ): 
\\	紙屑(かみくず): 
\\	ちり紙(ちりがみ): 
\\	糸: 
\\	氏: 
\\	糸	
\\	いと・いとへん	
\\	形声 
\\	画	ガ、カク、エ、カイ	えが.く、かく.する、かぎ.る、はかりごと、はか.る		
\\	映画(えいが): 
\\	計画(けいかく): 
\\	企画(きかく): 
\\	絵画(かいが): 
\\	画(かく): 
\\	画家(がか): 
\\	漫画(まんが): 
\\	映画館(えいがかん): 
\\	版画(はんが): 
\\	区画(くかく): 
\\	画期(かっき): 
\\	画(が): 
\\	一: 
\\	由: 
\\	凵: 
\\	田	
\\	た・たへん	畫	会意 
\\	絵	カイ、エ			
\\	絵画(かいが): 
\\	絵の具(えのぐ): 
\\	油絵(あぶらえ): 
\\	絵(え): 
\\	糸: 
\\	会: 
\\	糸	
\\	いと・いとへん	繪	形声 
\\	図	ズ、ト	え、はか.る	づ	
\\	意図(いと): 
\\	図書館(としょかん): 
\\	地図(ちず): 
\\	図(ず): 
\\	図書(としょ): 
\\	合図(あいず): 
\\	図形(ずけい): 
\\	指図(さしず): 
\\	図表(ずひょう): 
\\	図鑑(ずかん): 
\\	図々しい(ずうずうしい): 
\\	不図(ふと): 
\\	囗: 
\\	乂: 
\\	囗	
\\	くに・くにがまえ	圖	会意 
\\	工	コウ、ク、グ		もく	
\\	工場(こうじょう): 
\\	工業(こうぎょう): 
\\	工事(こうじ): 
\\	工作(こうさく): 
\\	加工(かこう): 
\\	工夫(くふう): 
\\	人工(じんこう): 
\\	工学(こうがく): 
\\	大工(だいく): 
\\	工芸(こうげい): 
\\	細工(さいく): 
\\	着工(ちゃっこう): 
\\	工員(こういん): 
\\	工: 
\\	工	
\\	こう・たくみ・たくみへん・え	
\\	象形 
\\	教	キョウ	おし.える、おそ.わる	のり, ひさ	
\\	教育(きょういく): 
\\	教授(きょうじゅ): 
\\	教師(きょうし): 
\\	宗教(しゅうきょう): 
\\	教室(きょうしつ): 
\\	教科書(きょうかしょ): 
\\	助教授(じょきょうじゅ): 
\\	教会(きょうかい): 
\\	教員(きょういん): 
\\	教訓(きょうくん): 
\\	教養(きょうよう): 
\\	教え(おしえ): 
\\	教材(きょうざい): 
\\	教科(きょうか): 
\\	教習(きょうしゅう): 
\\	教える(おしえる): 
\\	教わる(おそわる): 
\\	教職(きょうしょく): 
\\	宣教(せんきょう): 
\\	孝: 
\\	攵: 
\\	攴	
\\	ぼくにょう・ぼくづくり・とまた・のぶん・しぶん	
\\	会意 
\\	晴	セイ	は.れる、は.れ、は.れ-、-ば.れ、は.らす	はる, はれ	
\\	素晴しい(すばらしい): 
\\	晴れ(はれ): 
\\	快晴(かいせい): 
\\	晴天(せいてん): 
\\	晴れる(はれる): 
\\	見晴らし(みはらし): 
\\	日: 
\\	青: 
\\	日	
\\	ひ・ひへん・にち・にちへん	
\\	形声 
\\	思	シ	おも.う、おもえら.く、おぼ.す		
\\	思いっきり(おもいっきり): 
\\	思わず(おもわず): 
\\	思う(おもう): 
\\	思想(しそう): 
\\	思い出(おもいで): 
\\	思考(しこう): 
\\	不思議(ふしぎ): 
\\	思い出す(おもいだす): 
\\	思いつき(おもいつき): 
\\	意思(いし): 
\\	片思い(かたおもい): 
\\	思い付く(おもいつく): 
\\	思いがけない(おもいがけない): 
\\	思い込む(おもいこむ): 
\\	田: 
\\	心: 
\\	心	
\\	こころ・したごころ・りっしんべん	
\\	形声 
\\	考	コウ	かんが.える、かんが.え	たか	
\\	考え(かんがえ): 
\\	参考(さんこう): 
\\	思考(しこう): 
\\	考える(かんがえる): 
\\	考慮(こうりょ): 
\\	選考(せんこう): 
\\	考古学(こうこがく): 
\\	老: 
\\	丂: 
\\	老	
\\	おい・おいかんむり・おいがしら	
\\	形声 
\\	知	チ	し.る、し.らせる	さと, さとる, しり, しれ, とも, のり	
\\	知合い(しりあい): 
\\	知事(ちじ): 
\\	知る(しる): 
\\	知識(ちしき): 
\\	知人(ちじん): 
\\	知恵(ちえ): 
\\	知的(ちてき): 
\\	通知(つうち): 
\\	承知(しょうち): 
\\	未知(みち): 
\\	知性(ちせい): 
\\	知能(ちのう): 
\\	無知(むち): 
\\	かも知れない(かもしれない): 
\\	知らせ(しらせ): 
\\	知らせる(しらせる): 
\\	旧知(きゅうち): 
\\	矢: 
\\	口: 
\\	矢	
\\	や・やへん	
\\	会意 
\\	才	サイ			
\\	才(さい): 
\\	才能(さいのう): 
\\	天才(てんさい): 
\\	手: 
\\	手	
\\	て・てへん	
\\	象形 
\\	理	リ	ことわり	あや, おさむ, さと, さとる, ただ, ただし, とおる, に, のり, ひ, まこと, まさ, まさし, まろ, みち, よし	
\\	管理(かんり): 
\\	理由(りゆう): 
\\	処理(しょり): 
\\	理解(りかい): 
\\	料理(りょうり): 
\\	代理(だいり): 
\\	原理(げんり): 
\\	心理(しんり): 
\\	整理(せいり): 
\\	無理(むり): 
\\	理想(りそう): 
\\	理論(りろん): 
\\	論理(ろんり): 
\\	修理(しゅうり): 
\\	調理(ちょうり): 
\\	総理大臣(そうりだいじん): 
\\	物理(ぶつり): 
\\	理科(りか): 
\\	真理(しんり): 
\\	地理(ちり): 
\\	理屈(りくつ): 
\\	合理(ごうり): 
\\	義理(ぎり): 
\\	生理(せいり): 
\\	理性(りせい): 
\\	推理(すいり): 
\\	玉: 
\\	里: 
\\	玉	
\\	たま・たまへん ・ おう・おうへん・あうさま	
\\	形声 
\\	算	サン	そろ		
\\	予算(よさん): 
\\	決算(けっさん): 
\\	計算(けいさん): 
\\	採算(さいさん): 
\\	換算(かんさん): 
\\	算数(さんすう): 
\\	精算(せいさん): 
\\	足し算(たしざん): 
\\	引き算(ひきざん): 
\\	掛け算(かけざん): 
\\	算盤(そろばん): 
\\	割り算(わりざん): 
\\	暗算(あんざん): 
\\	竹: 
\\	目: 
\\	廾: 
\\	竹	
\\	たけ・たけかんむり	
\\	会意 
\\	作	サク、サ	つく.る、つく.り、-づく.り	くり, さか, さっ, づくり, とも, なお, はぎ, まさか	
\\	作業(さぎょう): 
\\	作品(さくひん): 
\\	作家(さっか): 
\\	作戦(さくせん): 
\\	制作(せいさく): 
\\	工作(こうさく): 
\\	作曲(さっきょく): 
\\	製作(せいさく): 
\\	操作(そうさ): 
\\	原作(げんさく): 
\\	作者(さくしゃ): 
\\	作(さく): 
\\	創作(そうさく): 
\\	作文(さくぶん): 
\\	作用(さよう): 
\\	発作(ほっさ): 
\\	名作(めいさく): 
\\	動作(どうさ): 
\\	傑作(けっさく): 
\\	作物(さくもつ): 
\\	豊作(ほうさく): 
\\	作成(さくせい): 
\\	作り(つくり): 
\\	荷作り(にづくり): 
\\	凶作(きょうさく): 
\\	耕作(こうさく): 
\\	作製(さくせい): 
\\	作法(さほう): 
\\	作物(さくぶつ): 
\\	駄作(ださく): 
\\	人: 
\\	乍: 
\\	人	
\\	ひと・にんべん・ひとやね	
\\	形声 
\\	元	ゲン、ガン	もと	ちか, はじめ, はる, ゆき, よし	
\\	根元(ねもと): 
\\	地元(じもと): 
\\	元首(げんしゅ): 
\\	元気(げんき): 
\\	元日(がんじつ): 
\\	還元(かんげん): 
\\	手元(てもと): 
\\	元来(がんらい): 
\\	元年(がんねん): 
\\	元素(げんそ): 
\\	元(もと): 
\\	お元気ですか(おげんきですか): 
\\	二: 
\\	八: 
\\	儿	
\\	にんにょう・ひとあし	
\\	象形 
\\	食	ショク、ジキ	く.う、く.らう、た.べる、は.む	ぐい	
\\	食事(しょくじ): 
\\	食品(しょくひん): 
\\	食料(しょくりょう): 
\\	昼食(ちゅうしょく): 
\\	給食(きゅうしょく): 
\\	食べ物(たべもの): 
\\	食堂(しょくどう): 
\\	食卓(しょくたく): 
\\	食物(しょくもつ): 
\\	食器(しょっき): 
\\	食欲(しょくよく): 
\\	主食(しゅしょく): 
\\	定食(ていしょく): 
\\	食違う(くいちがう): 
\\	食糧(しょくりょう): 
\\	食料品(しょくりょうひん): 
\\	食べる(たべる): 
\\	衣食住(いしょくじゅう): 
\\	食塩(しょくえん): 
\\	食う(くう): 
\\	食: 
\\	食	
\\	しょく・しょくへん	
\\	象形 
\\	肉	ニク	しし		
\\	肉(にく): 
\\	牛肉(ぎゅうにく): 
\\	筋肉(きんにく): 
\\	肉親(にくしん): 
\\	肉体(にくたい): 
\\	皮肉(ひにく): 
\\	豚肉(ぶたにく): 
\\	鶏肉(けいにく): 
\\	肉: 
\\	肉	
\\	にく・にくづき	
\\	象形 
\\	馬	バ	うま、うま-、ま	た, ばん, め, も	
\\	馬(うま): 
\\	競馬(けいば): 
\\	馬鹿(ばか): 
\\	馬鹿らしい(ばからしい): 
\\	馬鹿馬鹿しい(ばかばかしい): 
\\	火: 
\\	馬	
\\	うま・うまへん	
\\	象形 
\\	牛	ギュウ	うし	うじ, ご	
\\	"牛(うし): 
\\	牛肉(ぎゅうにく): 
\\	牛乳(ぎゅうにゅう): 
\\	牛: 
\\	牛	
\\	うし	
\\	象形 
\\	魚	ギョ	うお、さかな、-ざかな	い	
\\	魚(さかな): 
\\	金魚(きんぎょ): 
\\	海鷂魚(えい): 
\\	勹: 
\\	田: 
\\	火: 
\\	魚	
\\	うお・うおへん・さかな	
\\	象形 
\\	鳥	チョウ	とり	か, と, とっ	
\\	鳥(とり): 
\\	小鳥(ことり): 
\\	鳥居(とりい): 
\\	渡り鳥(わたりどり): 
\\	火: 
\\	鳥	
\\	とり・とりへん	
\\	象形 
\\	羽	ウ	は、わ、はね	しゅう, ば	
\\	羽(はね): 
\\	羽(わ): 
\\	羽: 
\\	羽	
\\	はね	
\\	象形 
\\	鳴	メイ	な.く、な.る、な.らす	なり, なる	
\\	悲鳴(ひめい): 
\\	怒鳴る(どなる): 
\\	共鳴(きょうめい): 
\\	鳴らす(ならす): 
\\	鳴る(なる): 
\\	鳴く(なく): 
\\	口: 
\\	鳥: 
\\	鳥	
\\	とり・とりへん	
\\	会意 
\\	麦	バク	むぎ		
\\	小麦(こむぎ): 
\\	麦酒(ビール): 
\\	蕎麦(そば): 
\\	夂: 
\\	麥	
\\	むぎ・むぎへん・ぼく・ばくにょう	麥	会意 
\\	米	ベイ、マイ、メエトル	こめ、よね	は, べ, まべ, め, よ, よな, よの, よま	
\\	南米(なんべい): 
\\	米(メートル): 
\\	米(こめ): 
\\	欧米(おうべい): 
\\	亜米利加(アメリカ): 
\\	米: 
\\	米	
\\	こめ・こめへん	
\\	象形 
\\	茶	チャ、サ		ちや	
\\	お茶(おちゃ): 
\\	茶(ちゃ): 
\\	喫茶店(きっさてん): 
\\	紅茶(こうちゃ): 
\\	茶色(ちゃいろ): 
\\	茶の間(ちゃのま): 
\\	茶わん(ちゃわん): 
\\	喫茶(きっさ): 
\\	無茶(むちゃ): 
\\	無茶苦茶(むちゃくちゃ): 
\\	目茶苦茶(めちゃくちゃ): 
\\	茶の湯(ちゃのゆ): 
\\	焦げ茶(こげちゃ): 
\\	茶色い(ちゃいろい): 
\\	艸: 
\\	人: 
\\	木: 
\\	艸	
\\	くさ・くさかんむり	
\\	形声 
\\	色	ショク、シキ	いろ	しか, しこ	
\\	色(いろ): 
\\	色彩(しきさい): 
\\	特色(とくしょく): 
\\	黄色(きいろ): 
\\	灰色(はいいろ): 
\\	景色(けしき): 
\\	茶色(ちゃいろ): 
\\	音色(ねいろ): 
\\	脚色(きゃくしょく): 
\\	着色(ちゃくしょく): 
\\	黄色い(きいろい): 
\\	色々(いろいろ): 
\\	茶色い(ちゃいろい): 
\\	色(しょく): 
\\	勹: 
\\	巴: 
\\	色	
\\	いろ	
\\	会意 
\\	黄	コウ、オウ	き、こ-	うい, れい	
\\	黄金(こがね): 
\\	黄色(きいろ): 
\\	黄色い(きいろい): 
\\	黄: 
\\	黃	
\\	き・きいろ	黃	象形 
\\	黒	コク	くろ、くろ.ずむ、くろ.い		
\\	黒字(くろじ): 
\\	黒(くろ): 
\\	黒い(くろい): 
\\	黒板(こくばん): 
\\	真っ黒(まっくろ): 
\\	里: 
\\	火: 
\\	黑	
\\	くろ・くろへん	黑	会意 
\\	来	ライ、タイ	く.る、きた.る、きた.す、き.たす、き.たる、き、こ	くり, くる, ごろ, さ	
\\	"以来(いらい): 
\\	従来(じゅうらい): 
\\	将来(しょうらい): 
\\	来年(らいねん): 
\\	本来(ほんらい): 
\\	来月(らいげつ): 
\\	来日(らいにち): 
\\	未来(みらい): 
\\	来(らい): 
\\	出来事(できごと): 
\\	来週(らいしゅう): 
\\	出来る(できる): 
\\	来る(くる): 
\\	元来(がんらい): 
\\	出来るだけ(できるだけ): 
\\	出来上がり(できあがり): 
\\	外来(がいらい): 
\\	再来年(さらいねん): 
\\	出来上がる(できあがる): 
\\	再来月(さらいげつ): 
\\	再来週(さらいしゅう): 
\\	来る(きたる): 
\\	伝来(でんらい): 
\\	来場(らいじょう): 
\\	家来(けらい): 
\\	出来物(できもの): 
\\	仕来たり(しきたり): 
\\	米: 
\\	人	
\\	ひと・にんべん・ひとやね	來	象形 
\\	行	コウ、ギョウ、アン	い.く、ゆ.く、-ゆ.き、-ゆき、-い.き、-いき、おこな.う、おこ.なう	いく, なみ, なめ, みち, ゆき, ゆく	
\\	"流行(はやり): 
\\	行き(いき): 
\\	平行(へいこう): 
\\	銀行(ぎんこう): 
\\	行政(ぎょうせい): 
\\	行動(こうどう): 
\\	行う(おこなう): 
\\	行為(こうい): 
\\	行方(ゆくえ): 
\\	発行(はっこう): 
\\	旅行(りょこう): 
\\	実行(じっこう): 
\\	飛行(ひこう): 
\\	行事(ぎょうじ): 
\\	飛行機(ひこうき): 
\\	施行(しこう): 
\\	進行(しんこう): 
\\	刊行(かんこう): 
\\	行(ぎょう): 
\\	通行(つうこう): 
\\	行進(こうしん): 
\\	行列(ぎょうれつ): 
\\	強行(きょうこう): 
\\	孝行(こうこう): 
\\	飛行場(ひこうじょう): 
\\	非行(ひこう): 
\\	急行(きゅうこう): 
\\	現行(げんこう): 
\\	移行(いこう): 
\\	先行(せんこう): 
\\	走行(そうこう): 
\\	売れ行き(うれゆき): 
\\	慣行(かんこう): 
\\	行員(こういん): 
\\	行い(おこない): 
\\	行儀(ぎょうぎ): 
\\	徐行(じょこう): 
\\	行き違い(いきちがい): 
\\	行く(いく): 
\\	行ってきます(いってきます): 
\\	行き成り(いきなり): 
\\	流行る(はやる): 
\\	行ってらっしゃい(いってらっしゃい): 
\\	夜行(やこう): 
\\	決行(けっこう): 
\\	行けない(いけない): 
\\	行けません(いけません): 
\\	行ってまいります(いってまいります): 
\\	彳: 
\\	行	
\\	ぎょう・ぎょうがまえ・いく・ゆきがまえ	
\\	象形 
\\	帰	キ	かえ.る、かえ.す、おく.る、とつ.ぐ		
\\	帰り(かえり): 
\\	帰宅(きたく): 
\\	日帰り(ひがえり): 
\\	帰京(ききょう): 
\\	帰る(かえる): 
\\	帰す(かえす): 
\\	お帰りなさい(おかえりなさい): 
\\	お帰り(おかえり): 
\\	刀: 
\\	帚: 
\\	止	
\\	とまる・とめる・とめへん	歸	会意 
\\	歩	ホ、ブ、フ	ある.く、あゆ.む	あ, ゆき, ゆみ	
\\	譲歩(じょうほ): 
\\	進歩(しんぽ): 
\\	徒歩(とほ): 
\\	歩(ほ): 
\\	散歩(さんぽ): 
\\	歩道(ほどう): 
\\	初歩(しょほ): 
\\	歩み(あゆみ): 
\\	歩む(あゆむ): 
\\	歩く(あるく): 
\\	止: 
\\	少: 
\\	止	
\\	とまる・とめる・とめへん	步	会意 
\\	走	ソウ	はし.る	はしり	
\\	走る(はしる): 
\\	逃走(とうそう): 
\\	走行(そうこう): 
\\	ご馳走さま(ごちそうさま): 
\\	ご馳走(ごちそう): 
\\	ご馳走様でした(ごちそうさまでした): 
\\	土: 
\\	走	
\\	はしる・そうにょう	
\\	象形 
\\	止	シ	と.まる、-ど.まり、と.める、-と.める、-ど.め、とど.める、とど.め、とど.まる、や.める、や.む、-や.む、よ.す、-さ.す、-さ.し	どめ	
\\	禁止(きんし): 
\\	停止(ていし): 
\\	防止(ぼうし): 
\\	中止(ちゅうし): 
\\	廃止(はいし): 
\\	受け止める(うけとめる): 
\\	静止(せいし): 
\\	阻止(そし): 
\\	引き止める(ひきとめる): 
\\	立ち止まる(たちどまる): 
\\	止す(よす): 
\\	止む(やむ): 
\\	止める(とめる): 
\\	止める(やめる): 
\\	止まる(とどまる): 
\\	呼び止める(よびとめる): 
\\	止: 
\\	止	
\\	とまる・とめる・とめへん	
\\	象形 
\\	活	カツ	い.きる、い.かす、い.ける		
\\	活動(かつどう): 
\\	生活(せいかつ): 
\\	活発(かっぱつ): 
\\	活躍(かつやく): 
\\	復活(ふっかつ): 
\\	活用(かつよう): 
\\	活力(かつりょく): 
\\	活字(かつじ): 
\\	活気(かっき): 
\\	水: 
\\	舌: 
\\	水	
\\	みず・したみず・さんずい	
\\	形声 
\\	店	テン	みせ、たな		
\\	支店(してん): 
\\	書店(しょてん): 
\\	商店(しょうてん): 
\\	店員(てんいん): 
\\	喫茶店(きっさてん): 
\\	売店(ばいてん): 
\\	店(みせ): 
\\	店(てん): 
\\	洋品店(ようひんてん): 
\\	店屋(みせや): 
\\	广: 
\\	占: 
\\	广	
\\	まだれ	
\\	形声 
\\	買	バイ	か.う		
\\	買物(かいもの): 
\\	売買(ばいばい): 
\\	買う(かう): 
\\	購買(こうばい): 
\\	网: 
\\	貝: 
\\	貝	
\\	かい・かいへん・こがい	
\\	会意 
\\	売	バイ	う.る、う.れる	うり, うる, め	
\\	売上げ(うりあげ): 
\\	小売り(こうり): 
\\	販売(はんばい): 
\\	売買(ばいばい): 
\\	発売(はつばい): 
\\	商売(しょうばい): 
\\	売り場(うりば): 
\\	売る(うる): 
\\	前売り(まえうり): 
\\	売り出す(うりだす): 
\\	売店(ばいてん): 
\\	売れ行き(うれゆき): 
\\	特売(とくばい): 
\\	売り出し(うりだし): 
\\	売り切れ(うりきれ): 
\\	売り切れる(うりきれる): 
\\	売れる(うれる): 
\\	土: 
\\	冖: 
\\	八: 
\\	貝	
\\	かい・かいへん・こがい	賣	会意 
\\	午	ゴ	うま		
\\	午後(ごご): 
\\	午前(ごぜん): 
\\	正午(しょうご): 
\\	丿: 
\\	干: 
\\	十	
\\	じゅう・じゅうへん	
\\	象形 
\\	汽	キ			
\\	汽車(きしゃ): 
\\	汽船(きせん): 
\\	水: 
\\	气: 
\\	水	
\\	みず・したみず・さんずい	
\\	形声 
\\	弓	キュウ	ゆみ	こ, ゆ	
\\	弓(ゆみ): 
\\	弓: 
\\	弓	
\\	ゆみ・ゆみへん	
\\	象形 
\\	回	カイ、エ	まわ.る、-まわ.る、-まわ.り、まわ.す、-まわ.す、まわ.し-、-まわ.し、もとお.る、か.える		
\\	回り(まわり): 
\\	回復(かいふく): 
\\	今回(こんかい): 
\\	回答(かいとう): 
\\	回収(かいしゅう): 
\\	回数(かいすう): 
\\	回転(かいてん): 
\\	回(かい): 
\\	回路(かいろ): 
\\	根回し(ねまわし): 
\\	身の回り(みのまわり): 
\\	回す(まわす): 
\\	後回し(あとまわし): 
\\	回数券(かいすうけん): 
\\	回り道(まわりみち): 
\\	回る(まわる): 
\\	掻き回す(かきまわす): 
\\	回送(かいそう): 
\\	回覧(かいらん): 
\\	上回る(うわまわる): 
\\	遠回り(とおまわり): 
\\	転回(てんかい): 
\\	ねじ回し(ねじまわし): 
\\	手回し(てまわし): 
\\	囗: 
\\	口: 
\\	囗	
\\	くに・くにがまえ	
\\	象形 
\\	会	カイ、エ	あ.う、あ.わせる、あつ.まる	あい, い	
\\	会員(かいいん): 
\\	会議(かいぎ): 
\\	会見(かいけん): 
\\	会社(かいしゃ): 
\\	会談(かいだん): 
\\	議会(ぎかい): 
\\	国会(こっかい): 
\\	社会(しゃかい): 
\\	総会(そうかい): 
\\	大会(たいかい): 
\\	会合(かいごう): 
\\	会場(かいじょう): 
\\	機会(きかい): 
\\	集会(しゅうかい): 
\\	会話(かいわ): 
\\	開会(かいかい): 
\\	学会(がっかい): 
\\	会計(かいけい): 
\\	教会(きょうかい): 
\\	都会(とかい): 
\\	司会(しかい): 
\\	展覧会(てんらんかい): 
\\	面会(めんかい): 
\\	閉会(へいかい): 
\\	宴会(えんかい): 
\\	再会(さいかい): 
\\	座談会(ざだんかい): 
\\	社会科学(しゃかいかがく): 
\\	協会(きょうかい): 
\\	会館(かいかん): 
\\	会議室(かいぎしつ): 
\\	出会い(であい): 
\\	会(かい): 
\\	会(え): 
\\	人: 
\\	云: 
\\	曰	
\\	ひらび・いわく	會	象形 
\\	組	ソ	く.む、くみ、-ぐみ		
\\	組(くみ): 
\\	組織(そしき): 
\\	番組(ばんぐみ): 
\\	組合(くみあい): 
\\	仕組み(しくみ): 
\\	組み込む(くみこむ): 
\\	組む(くむ): 
\\	組み合わせ(くみあわせ): 
\\	取り組む(とりくむ): 
\\	組み立てる(くみたてる): 
\\	組み合わせる(くみあわせる): 
\\	糸: 
\\	且: 
\\	糸	
\\	いと・いとへん	
\\	形声 
\\	船	セン	ふね、ふな-	ふ	
\\	船(ふね): 
\\	漁船(ぎょせん): 
\\	船舶(せんぱく): 
\\	造船(ぞうせん): 
\\	風船(ふうせん): 
\\	汽船(きせん): 
\\	船便(ふなびん): 
\\	舟: 
\\	㕣: 
\\	舟	
\\	ふね・ふねへん	
\\	形声 
\\	明	メイ、ミョウ、ミン	あ.かり、あか.るい、あか.るむ、あか.らむ、あき.らか、あ.ける、-あ.け、あ.く、あ.くる、あ.かす	あきら, あけ, あす, きら, け, さや, さやか, とし, はる, み, め	
\\	明日(あした): 
\\	明らか(あきらか): 
\\	声明(せいめい): 
\\	説明(せつめい): 
\\	不明(ふめい): 
\\	明確(めいかく): 
\\	透明(とうめい): 
\\	文明(ぶんめい): 
\\	証明(しょうめい): 
\\	照明(しょうめい): 
\\	明るい(あかるい): 
\\	明白(めいはく): 
\\	賢明(けんめい): 
\\	夜明け(よあけ): 
\\	発明(はつめい): 
\\	明朗(めいろう): 
\\	明け方(あけがた): 
\\	明後日(あさって): 
\\	明瞭(めいりょう): 
\\	打ち明ける(うちあける): 
\\	明かす(あかす): 
\\	明くる(あくる): 
\\	明明後日(しあさって): 
\\	明白(あからさま): 
\\	明(みょう): 
\\	日: 
\\	月: 
\\	日	
\\	ひ・ひへん・にち・にちへん	
\\	会意 
\\	社	シャ	やしろ	こそ	
\\	会社(かいしゃ): 
\\	社会(しゃかい): 
\\	社長(しゃちょう): 
\\	商社(しょうしゃ): 
\\	入社(にゅうしゃ): 
\\	新聞社(しんぶんしゃ): 
\\	神社(じんじゃ): 
\\	社説(しゃせつ): 
\\	社会科学(しゃかいかがく): 
\\	社交(しゃこう): 
\\	社宅(しゃたく): 
\\	出社(しゅっしゃ): 
\\	社(やしろ): 
\\	社(しゃ): 
\\	示: 
\\	土: 
\\	示	
\\	しめす・しめすへん	社	形声 
\\	切	セツ、サイ	き.る、-き.る、き.り、-き.り、-ぎ.り、き.れる、-き.れる、き.れ、-き.れ、-ぎ.れ	きつ, きり, ぎり	
\\	締切り(しめきり): 
\\	踏切り(ふみきり): 
\\	大切(たいせつ): 
\\	切掛(きっかけ): 
\\	一切(いっさい): 
\\	適切(てきせつ): 
\\	切手(きって): 
\\	小切手(こぎって): 
\\	切り(きり): 
\\	切符(きっぷ): 
\\	切る(きる): 
\\	切実(せつじつ): 
\\	親切(しんせつ): 
\\	裏切る(うらぎる): 
\\	締め切る(しめきる): 
\\	切り替え(きりかえ): 
\\	区切り(くぎり): 
\\	打ち切る(うちきる): 
\\	切れ目(きれめ): 
\\	押し切る(おしきる): 
\\	横切る(よこぎる): 
\\	張り切る(はりきる): 
\\	途切れる(とぎれる): 
\\	区切る(くぎる): 
\\	切ない(せつない): 
\\	切れ(きれ): 
\\	切れる(きれる): 
\\	千切る(ちぎる): 
\\	売り切れ(うりきれ): 
\\	売り切れる(うりきれる): 
\\	痛切(つうせつ): 
\\	仕切る(しきる): 
\\	切り替える(きりかえる): 
\\	切開(せっかい): 
\\	持ち切り(もちきり): 
\\	噛み切る(かみきる): 
\\	出切る(できる): 
\\	七: 
\\	刀: 
\\	刀	
\\	かたな・りっとう	
\\	会意 
\\	電	デン			
\\	電話(でんわ): 
\\	電気(でんき): 
\\	電力(でんりょく): 
\\	電子(でんし): 
\\	電車(でんしゃ): 
\\	電波(でんぱ): 
\\	発電(はつでん): 
\\	電源(でんげん): 
\\	電池(でんち): 
\\	停電(ていでん): 
\\	電報(でんぽう): 
\\	電流(でんりゅう): 
\\	乾電池(かんでんち): 
\\	電線(でんせん): 
\\	電柱(でんちゅう): 
\\	電球(でんきゅう): 
\\	電灯(でんとう): 
\\	雨: 
\\	日: 
\\	雨	
\\	あめ・あめかんむり・あまかんむり	
\\	会意 
\\	毎	マイ	ごと、-ごと.に	つね	
\\	毎年(まいねん): 
\\	毎日(まいにち): 
\\	毎週(まいしゅう): 
\\	毎月(まいつき): 
\\	毎朝(まいあさ): 
\\	毎晩(まいばん): 
\\	毎度(まいど): 
\\	毎(ごと): 
\\	毎(まい): 
\\	丿: 
\\	毋: 
\\	毋	
\\	なし・なかれ ・ははのかん	每	象形 
\\	合	ゴウ、ガッ、カッ	あ.う、-あ.う、あ.い、あい-、-あ.い、-あい、あ.わす、あ.わせる、-あ.わせる	あう, あん, い, か, こう, ごお, に, ね, や, り, わい	
\\	合わせる(あわせる): 
\\	知合い(しりあい): 
\\	出合う(であう): 
\\	合意(ごうい): 
\\	試合(しあい): 
\\	連合(れんごう): 
\\	会合(かいごう): 
\\	組合(くみあい): 
\\	統合(とうごう): 
\\	合計(ごうけい): 
\\	合同(ごうどう): 
\\	割合(わりあい): 
\\	合格(ごうかく): 
\\	付き合い(つきあい): 
\\	複合(ふくごう): 
\\	都合(つごう): 
\\	具合(ぐあい): 
\\	合う(あう): 
\\	集合(しゅうごう): 
\\	場合(ばあい): 
\\	合間(あいま): 
\\	合唱(がっしょう): 
\\	合理(ごうり): 
\\	話し合う(はなしあう): 
\\	合成(ごうせい): 
\\	問い合わせる(といあわせる): 
\\	結合(けつごう): 
\\	見合い(みあい): 
\\	間に合う(まにあう): 
\\	組み合わせ(くみあわせ): 
\\	出合い(であい): 
\\	待ち合わせ(まちあわせ): 
\\	見合わせる(みあわせる): 
\\	合議(ごうぎ): 
\\	合わす(あわす): 
\\	付き合う(つきあう): 
\\	化合(かごう): 
\\	組み合わせる(くみあわせる): 
\\	待合室(まちあいしつ): 
\\	人: 
\\	一: 
\\	口: 
\\	口	
\\	くち・くちへん	
\\	象形 
\\	当	トウ	あ.たる、あ.たり、あ.てる、あ.て、まさ.に、まさ.にべし	たい	
\\	当り前(あたりまえ): 
\\	当り(あたり): 
\\	手当て(てあて): 
\\	割当て(わりあて): 
\\	当時(とうじ): 
\\	担当(たんとう): 
\\	当選(とうせん): 
\\	本当(ほんとう): 
\\	相当(そうとう): 
\\	当然(とうぜん): 
\\	当日(とうじつ): 
\\	不当(ふとう): 
\\	正当(せいとう): 
\\	適当(てきとう): 
\\	弁当(べんとう): 
\\	当てる(あてる): 
\\	該当(がいとう): 
\\	見当(けんとう): 
\\	当人(とうにん): 
\\	当番(とうばん): 
\\	日当たり(ひあたり): 
\\	心当たり(こころあたり): 
\\	当たる(あたる): 
\\	突き当たる(つきあたる): 
\\	妥当(だとう): 
\\	当てはまる(あてはまる): 
\\	突き当たり(つきあたり): 
\\	当てはめる(あてはめる): 
\\	当て字(あてじ): 
\\	日当(にっとう): 
\\	お弁当(おべんとう): 
\\	当(とう): 
\\	小: 
\\	彑: 
\\	田	
\\	た・たへん	當	形声 
\\	台	ダイ、タイ	うてな、われ、つかさ		
\\	台(だい): 
\\	舞台(ぶたい): 
\\	台風(たいふう): 
\\	台所(だいどころ): 
\\	土台(どだい): 
\\	灯台(とうだい): 
\\	寝台(しんだい): 
\\	台本(だいほん): 
\\	台無し(だいなし): 
\\	台詞(せりふ): 
\\	厶: 
\\	口: 
\\	至	
\\	いたる	臺	会意 
\\	楽	ガク、ラク、ゴウ	たの.しい、たの.しむ、この.む	さ, た, やす, ら	
\\	音楽(おんがく): 
\\	楽しみ(たのしみ): 
\\	楽器(がっき): 
\\	娯楽(ごらく): 
\\	楽(らく): 
\\	気楽(きらく): 
\\	楽しい(たのしい): 
\\	楽しむ(たのしむ): 
\\	楽譜(がくふ): 
\\	極楽(ごくらく): 
\\	楽観(らっかん): 
\\	日: 
\\	冫: 
\\	木: 
\\	木	
\\	き・きへん	樂	象形 
\\	公	コウ、ク	おおやけ	あきら, き, きみ, きん, たか, ただし, とも, ひろ, まさ	
\\	公園(こうえん): 
\\	公演(こうえん): 
\\	公開(こうかい): 
\\	公共(こうきょう): 
\\	公式(こうしき): 
\\	公害(こうがい): 
\\	公認(こうにん): 
\\	公表(こうひょう): 
\\	公務員(こうむいん): 
\\	公立(こうりつ): 
\\	公正(こうせい): 
\\	公団(こうだん): 
\\	公平(こうへい): 
\\	公務(こうむ): 
\\	公然(こうぜん): 
\\	公衆(こうしゅう): 
\\	主人公(しゅじんこう): 
\\	公募(こうぼ): 
\\	公用(こうよう): 
\\	公(おおやけ): 
\\	八: 
\\	厶: 
\\	八	
\\	はち・は・はちがしら	
\\	象形 
\\	引	イン	ひ.く、ひ.き、ひ.き-、-び.き、ひ.ける	いな, ひき, ひけ, びき	
\\	割り引き(わりびき): 
\\	引出し(ひきだし): 
\\	引分け(ひきわけ): 
\\	引上げる(ひきあげる): 
\\	引下げる(ひきさげる): 
\\	引取る(ひきとる): 
\\	引出す(ひきだす): 
\\	手引(てびき): 
\\	取引(とりひき): 
\\	引退(いんたい): 
\\	強引(ごういん): 
\\	引っ越し(ひっこし): 
\\	引きずる(ひきずる): 
\\	値引き(ねびき): 
\\	引用(いんよう): 
\\	差し引き(さしひき): 
\\	引っ越す(ひっこす): 
\\	引っ張る(ひっぱる): 
\\	索引(さくいん): 
\\	引き起す(ひきおこす): 
\\	引っ掛る(ひっかかる): 
\\	引受ける(ひきうける): 
\\	引返す(ひきかえす): 
\\	長引く(ながびく): 
\\	籤引き(くじびき): 
\\	引き算(ひきざん): 
\\	引き止める(ひきとめる): 
\\	引力(いんりょく): 
\\	引っ掛ける(ひっかける): 
\\	引っ込む(ひっこむ): 
\\	引く(ひく): 
\\	引っ繰り返る(ひっくりかえる): 
\\	字引(じびき): 
\\	引っ掻く(ひっかく): 
\\	差し引く(さしひく): 
\\	弓: 
\\	丨: 
\\	弓	
\\	ゆみ・ゆみへん	
\\	会意 
\\	科	カ		しな	
\\	百科辞典(ひゃっかじてん): 
\\	科学(かがく): 
\\	教科書(きょうかしょ): 
\\	学科(がっか): 
\\	外科(げか): 
\\	理科(りか): 
\\	内科(ないか): 
\\	科(か): 
\\	歯科(しか): 
\\	産婦人科(さんふじんか): 
\\	小児科(しょうにか): 
\\	社会科学(しゃかいかがく): 
\\	自然科学(しぜんかがく): 
\\	眼科(がんか): 
\\	教科(きょうか): 
\\	人文科学(じんぶんかがく): 
\\	耳鼻科(じびか): 
\\	禾: 
\\	斗: 
\\	禾	
\\	のぎ・のぎへん	
\\	会意 
\\	歌	カ	うた、うた.う		
\\	歌(うた): 
\\	歌手(かしゅ): 
\\	歌謡(かよう): 
\\	歌留多(かるた): 
\\	哥: 
\\	欠: 
\\	欠	
\\	あくび・けんづくり・かける・けつ	
\\	形声 
\\	刀	トウ	かたな、そり	き, ち, と, わき	
\\	刀(かたな): 
\\	剃刀(かみそり): 
\\	刀: 
\\	刀	
\\	かたな・りっとう	
\\	象形 
\\	番	バン	つが.い	は, ま	
\\	一番(いちばん): 
\\	番組(ばんぐみ): 
\\	番号(ばんごう): 
\\	順番(じゅんばん): 
\\	留守番(るすばん): 
\\	交番(こうばん): 
\\	当番(とうばん): 
\\	番地(ばんち): 
\\	番目(ばんめ): 
\\	番(ばん): 
\\	釆: 
\\	田: 
\\	田	
\\	た・たへん	
\\	象形 
\\	用	ヨウ	もち.いる	たから	
\\	雇用(こよう): 
\\	採用(さいよう): 
\\	使用(しよう): 
\\	費用(ひよう): 
\\	利用(りよう): 
\\	運用(うんよう): 
\\	信用(しんよう): 
\\	適用(てきよう): 
\\	専用(せんよう): 
\\	用意(ようい): 
\\	活用(かつよう): 
\\	実用(じつよう): 
\\	用語(ようご): 
\\	用紙(ようし): 
\\	応用(おうよう): 
\\	作用(さよう): 
\\	用途(ようと): 
\\	乱用(らんよう): 
\\	無用(むよう): 
\\	引用(いんよう): 
\\	代用(だいよう): 
\\	日用品(にちようひん): 
\\	用いる(もちいる): 
\\	用品(ようひん): 
\\	用事(ようじ): 
\\	公用(こうよう): 
\\	用心(ようじん): 
\\	使用人(しようにん): 
\\	用件(ようけん): 
\\	通用(つうよう): 
\\	私用(しよう): 
\\	器用(きよう): 
\\	用(よう): 
\\	兼用(けんよう): 
\\	用法(ようほう): 
\\	用: 
\\	用	
\\	もちいる・よう	
\\	象形 
\\	何	カ	なに、なん、なに-、なん-	あが	
\\	"何(なん): 
\\	何か(なにか): 
\\	何とか(なんとか): 
\\	何より(なにより): 
\\	何でも(なんでも): 
\\	何と(なんと): 
\\	何とも(なんとも): 
\\	何で(なんで): 
\\	何だか(なんだか): 
\\	何しろ(なにしろ): 
\\	何故(なぜ): 
\\	何卒(なにとぞ): 
\\	何も(なにも): 
\\	何の(どの): 
\\	何時(いつ): 
\\	何て(なんて): 
\\	何気ない(なにげない): 
\\	如何(どう): 
\\	何方(どちら): 
\\	何分(なにぶん): 
\\	何処(どこ): 
\\	如何にも(いかにも): 
\\	何れ(どれ): 
\\	何故なら(なぜなら): 
\\	何だかんだ(なんだかんだ): 
\\	何れ何れ(どれどれ): 
\\	何時までも(いつまでも): 
\\	何時も(いつも): 
\\	何処か(どこか): 
\\	何方(どなた): 
\\	如何して(どうして): 
\\	如何しても(どうしても): 
\\	如何に(いかに): 
\\	何時でも(いつでも): 
\\	何となく(なんとなく): 
\\	何々(なになに): 
\\	何なり(なんなり): 
\\	何時か(いつか): 
\\	人: 
\\	可: 
\\	人	
\\	ひと・にんべん・ひとやね	
\\	形声 
\\	丁	チョウ、テイ、チン、トウ、チ	ひのと		
\\	丁寧(ていねい): 
\\	包丁(ほうちょう): 
\\	丁目(ちょうめ): 
\\	丁度(ちょうど): 
\\	丁々(ちょうちょう): 
\\	一: 
\\	亅: 
\\	一	
\\	いち	
\\	象形 
\\	世	セイ、セ、ソウ	よ、さんじゅう	とし, ゆ, ゆき	
\\	世論(よろん): 
\\	世界(せかい): 
\\	世代(せだい): 
\\	世紀(せいき): 
\\	世帯(せたい): 
\\	世話(せわ): 
\\	世の中(よのなか): 
\\	世間(せけん): 
\\	中世(ちゅうせい): 
\\	出世(しゅっせ): 
\\	お世辞(おせじ): 
\\	見世物(みせもの): 
\\	お世話になる(おせわになる): 
\\	世辞(せじ): 
\\	お世話(おせわ): 
\\	廿: 
\\	一	
\\	いち	
\\	象形 
\\	両	リョウ	てる、ふたつ	もろ	
\\	両親(りょうしん): 
\\	両方(りょうほう): 
\\	両側(りょうがわ): 
\\	両立(りょうりつ): 
\\	両替(りょうがえ): 
\\	両極(りょうきょく): 
\\	両(りょう): 
\\	一: 
\\	冂: 
\\	山: 
\\	入	
\\	いる・にゅう・いりがしら・いりやね	兩	象形 
\\	主	シュ、ス、シュウ	ぬし、おも、あるじ	かず, ず, もん	
\\	主人(しゅじん): 
\\	民主(みんしゅ): 
\\	主義(しゅぎ): 
\\	主催(しゅさい): 
\\	主張(しゅちょう): 
\\	主婦(しゅふ): 
\\	主要(しゅよう): 
\\	自主(じしゅ): 
\\	主権(しゅけん): 
\\	主体(しゅたい): 
\\	主任(しゅにん): 
\\	主役(しゅやく): 
\\	主に(おもに): 
\\	主演(しゅえん): 
\\	主題(しゅだい): 
\\	主食(しゅしょく): 
\\	地主(じぬし): 
\\	家主(やぬし): 
\\	主導(しゅどう): 
\\	主人公(しゅじんこう): 
\\	君主(くんしゅ): 
\\	主観(しゅかん): 
\\	ご主人(ごしゅじん): 
\\	主(しゅ): 
\\	主(ぬし): 
\\	主語(しゅご): 
\\	主なる(おもなる): 
\\	亠: 
\\	玉: 
\\	丶	
\\	てん・ちょぼ・ちゅ・ちゅてん	
\\	象形 
\\	乗	ジョウ、ショウ	の.る、-の.り、の.せる	のり	
\\	乗物(のりもの): 
\\	乗換え(のりかえ): 
\\	乗客(じょうきゃく): 
\\	乗せる(のせる): 
\\	乗車(じょうしゃ): 
\\	乗る(のる): 
\\	乗り換える(のりかえる): 
\\	乗っ取る(のっとる): 
\\	乗り越し(のりこし): 
\\	乗り込む(のりこむ): 
\\	千: 
\\	木: 
\\	丿	
\\	の・のかんむり・はらいぼう	乘	会意 
\\	予	ヨ、シャ	あらかじ.め		
\\	予言(よげん): 
\\	予算(よさん): 
\\	予定(よてい): 
\\	予想(よそう): 
\\	予備(よび): 
\\	予防(よぼう): 
\\	予約(よやく): 
\\	予報(よほう): 
\\	予感(よかん): 
\\	予期(よき): 
\\	予て(かねて): 
\\	予測(よそく): 
\\	予習(よしゅう): 
\\	天気予報(てんきよほう): 
\\	予め(あらかじめ): 
\\	マ: 
\\	了: 
\\	豕	
\\	いのこ・いのこへん・ぶた	豫	象形 
\\	事	ジ、ズ	こと、つか.う、つか.える	ろ	
\\	記事(きじ): 
\\	軍事(ぐんじ): 
\\	仕事(しごと): 
\\	事業(じぎょう): 
\\	事件(じけん): 
\\	事故(じこ): 
\\	事実(じじつ): 
\\	事情(じじょう): 
\\	事務(じむ): 
\\	事務所(じむしょ): 
\\	知事(ちじ): 
\\	事態(じたい): 
\\	工事(こうじ): 
\\	人事(じんじ): 
\\	食事(しょくじ): 
\\	刑事(けいじ): 
\\	検事(けんじ): 
\\	見事(みごと): 
\\	行事(ぎょうじ): 
\\	家事(かじ): 
\\	出来事(できごと): 
\\	判事(はんじ): 
\\	事(こと): 
\\	大事(だいじ): 
\\	返事(へんじ): 
\\	無事(ぶじ): 
\\	火事(かじ): 
\\	物事(ものごと): 
\\	従事(じゅうじ): 
\\	事柄(ことがら): 
\\	事前(じぜん): 
\\	事項(じこう): 
\\	用事(ようじ): 
\\	炊事(すいじ): 
\\	領事(りょうじ): 
\\	議事堂(ぎじどう): 
\\	お大事に(おだいじに): 
\\	事による(ことによる): 
\\	些事(さじ): 
\\	口: 
\\	彑: 
\\	亅: 
\\	亅	
\\	はねぼう	
\\	会意 
\\	仕	シ、ジ	つか.える		
\\	仕事(しごと): 
\\	仕組み(しくみ): 
\\	仕方(しかた): 
\\	奉仕(ほうし): 
\\	仕上げ(しあげ): 
\\	仕掛け(しかけ): 
\\	仕様(しよう): 
\\	仕上がり(しあがり): 
\\	仕入れる(しいれる): 
\\	仕掛ける(しかける): 
\\	仕方がない(しかたがない): 
\\	仕上げる(しあげる): 
\\	仕立てる(したてる): 
\\	仕舞う(しまう): 
\\	仕切る(しきる): 
\\	仕える(つかえる): 
\\	給仕(きゅうじ): 
\\	仕上がる(しあがる): 
\\	仕様が無い(しようがない): 
\\	仕来たり(しきたり): 
\\	仕舞い(しまい): 
\\	人: 
\\	士: 
\\	人	
\\	ひと・にんべん・ひとやね	
\\	形声 
\\	他	タ	ほか		
\\	他人(たにん): 
\\	他方(たほう): 
\\	その他(そのほか): 
\\	他(た): 
\\	他動詞(たどうし): 
\\	人: 
\\	也: 
\\	人	
\\	ひと・にんべん・ひとやね	
\\	形声 
\\	代	ダイ、タイ	か.わる、かわ.る、かわ.り、か.わり、-がわ.り、-が.わり、か.える、よ、しろ	す	
\\	代わり(かわり): 
\\	代(よ): 
\\	時代(じだい): 
\\	代表(だいひょう): 
\\	現代(げんだい): 
\\	世代(せだい): 
\\	近代(きんだい): 
\\	代理(だいり): 
\\	代金(だいきん): 
\\	古代(こだい): 
\\	年代(ねんだい): 
\\	代名詞(だいめいし): 
\\	代用(だいよう): 
\\	代わる(かわる): 
\\	代弁(だいべん): 
\\	先代(せんだい): 
\\	代わる代わる(かわるがわる): 
\\	お代わり(おかわり): 
\\	代(しろ): 
\\	代(だい): 
\\	人: 
\\	弋: 
\\	人	
\\	ひと・にんべん・ひとやね	
\\	形声 
\\	住	ジュウ、ヂュウ、チュウ	す.む、す.まう、-ず.まい	し, じゅ, すみ	
\\	住まい(すまい): 
\\	住宅(じゅうたく): 
\\	住民(じゅうみん): 
\\	住所(じゅうしょ): 
\\	居住(きょじゅう): 
\\	移住(いじゅう): 
\\	住(じゅう): 
\\	衣食住(いしょくじゅう): 
\\	住む(すむ): 
\\	人: 
\\	主: 
\\	人	
\\	ひと・にんべん・ひとやね	
\\	形声 
\\	使	シ	つか.う、つか.い、-つか.い、-づか.い		
\\	使う(つかう): 
\\	使用(しよう): 
\\	大使(たいし): 
\\	大使館(たいしかん): 
\\	使命(しめい): 
\\	使い(つかい): 
\\	使い道(つかいみち): 
\\	使用人(しようにん): 
\\	お使い(おつかい): 
\\	人: 
\\	吏: 
\\	人	
\\	ひと・にんべん・ひとやね	
\\	形声 
\\	係	ケイ	かか.る、かかり、-がかり、かか.わる		
\\	関係(かんけい): 
\\	係(かかり): 
\\	係わる(かかわる): 
\\	人: 
\\	系: 
\\	人	
\\	ひと・にんべん・ひとやね	
\\	会意 
\\	倍	バイ		べ, ます	
\\	倍(ばい): 
\\	倍率(ばいりつ): 
\\	人: 
\\	立: 
\\	口: 
\\	人	
\\	ひと・にんべん・ひとやね	
\\	形声 
\\	全	ゼン	まった.く、すべ.て	たけ, まさ	
\\	安全(あんぜん): 
\\	全体(ぜんたい): 
\\	全国(ぜんこく): 
\\	全員(ぜんいん): 
\\	全部(ぜんぶ): 
\\	全力(ぜんりょく): 
\\	完全(かんぜん): 
\\	健全(けんぜん): 
\\	全身(ぜんしん): 
\\	全般(ぜんぱん): 
\\	全集(ぜんしゅう): 
\\	全然(ぜんぜん): 
\\	全盛(ぜんせい): 
\\	全滅(ぜんめつ): 
\\	全(ぜん): 
\\	全く(まったく): 
\\	全快(ぜんかい): 
\\	全て(すべて): 
\\	人: 
\\	玉: 
\\	入	
\\	いる・にゅう・いりがしら・いりやね	
\\	象形 
\\	具	グ	そな.える、つぶさ.に	とも	
\\	具わる(そなわる): 
\\	道具(どうぐ): 
\\	具合(ぐあい): 
\\	家具(かぐ): 
\\	器具(きぐ): 
\\	絵の具(えのぐ): 
\\	文房具(ぶんぼうぐ): 
\\	玩具(おもちゃ): 
\\	具体(ぐたい): 
\\	雨具(あまぐ): 
\\	敬具(けいぐ): 
\\	夜具(やぐ): 
\\	目: 
\\	八: 
\\	八	
\\	はち・は・はちがしら	
\\	会意 
\\	写	シャ、ジャ	うつ.す、うつ.る、うつ-、うつ.し		
\\	写真(しゃしん): 
\\	描写(びょうしゃ): 
\\	写る(うつる): 
\\	写す(うつす): 
\\	写生(しゃせい): 
\\	複写(ふくしゃ): 
\\	写し(うつし): 
\\	映写(えいしゃ): 
\\	冖: 
\\	与: 
\\	宀	
\\	うかんむり	寫	会意 
\\	列	レツ、レ		れっ	
\\	列車(れっしゃ): 
\\	列島(れっとう): 
\\	列(れつ): 
\\	行列(ぎょうれつ): 
\\	配列(はいれつ): 
\\	陳列(ちんれつ): 
\\	並列(へいれつ): 
\\	整列(せいれつ): 
\\	歹: 
\\	刀: 
\\	刀	
\\	かたな・りっとう	
\\	会意 
\\	助	ジョ	たす.ける、たす.かる、す.ける、すけ		
\\	援助(えんじょ): 
\\	補助(ほじょ): 
\\	助教授(じょきょうじゅ): 
\\	助手(じょしゅ): 
\\	救助(きゅうじょ): 
\\	助言(じょげん): 
\\	助ける(たすける): 
\\	助け(たすけ): 
\\	助かる(たすかる): 
\\	助詞(じょし): 
\\	助(じょ): 
\\	助動詞(じょどうし): 
\\	且: 
\\	力: 
\\	力	
\\	ちから・りきづくり	
\\	会意 
\\	勉	ベン	つと.める	ひこ, やつ	
\\	勉強(べんきょう): 
\\	勤勉(きんべん): 
\\	免: 
\\	力: 
\\	力	
\\	ちから・りきづくり	勉	形声 
\\	動	ドウ	うご.く、うご.かす	るぎ	
\\	運動(うんどう): 
\\	活動(かつどう): 
\\	行動(こうどう): 
\\	自動車(じどうしゃ): 
\\	動き(うごき): 
\\	動物(どうぶつ): 
\\	不動産(ふどうさん): 
\\	動向(どうこう): 
\\	移動(いどう): 
\\	自動(じどう): 
\\	変動(へんどう): 
\\	感動(かんどう): 
\\	動機(どうき): 
\\	騒動(そうどう): 
\\	動物園(どうぶつえん): 
\\	暴動(ぼうどう): 
\\	動員(どういん): 
\\	異動(いどう): 
\\	振動(しんどう): 
\\	動揺(どうよう): 
\\	動作(どうさ): 
\\	動かす(うごかす): 
\\	動力(どうりょく): 
\\	出動(しゅつどう): 
\\	動詞(どうし): 
\\	動く(うごく): 
\\	動的(どうてき): 
\\	形容動詞(けいようどうし): 
\\	自動詞(じどうし): 
\\	助動詞(じょどうし): 
\\	他動詞(たどうし): 
\\	重: 
\\	力: 
\\	力	
\\	ちから・りきづくり	
\\	形声 
\\	勝	ショウ	か.つ、-が.ち、まさ.る、すぐ.れる、かつ	かち, と, よし	
\\	優勝(ゆうしょう): 
\\	決勝(けっしょう): 
\\	勝利(しょうり): 
\\	勝負(しょうぶ): 
\\	勝る(まさる): 
\\	勝ち(かち): 
\\	勝敗(しょうはい): 
\\	勝手(かって): 
\\	勝つ(かつ): 
\\	勝手に(かってに): 
\\	勝ち(がち): 
\\	肉: 
\\	劵: 
\\	力	
\\	ちから・りきづくり	
\\	形声 
\\	化	カ、ケ	ば.ける、ば.かす、ふ.ける、け.する		
\\	強化(きょうか): 
\\	文化(ぶんか): 
\\	変化(へんか): 
\\	悪化(あっか): 
\\	化学(かがく): 
\\	酸化(さんか): 
\\	化粧(けしょう): 
\\	化石(かせき): 
\\	文化財(ぶんかざい): 
\\	消化(しょうか): 
\\	進化(しんか): 
\\	化ける(ばける): 
\\	化合(かごう): 
\\	化繊(かせん): 
\\	誤魔化す(ごまかす): 
\\	退化(たいか): 
\\	化(か): 
\\	人: 
\\	匕: 
\\	匕	
\\	さじ・さじのひ・さじづくり・ひ	
\\	会意 
\\	区	ク、オウ、コウ			
\\	区域(くいき): 
\\	区間(くかん): 
\\	区(く): 
\\	区別(くべつ): 
\\	区分(くぶん): 
\\	区画(くかく): 
\\	地区(ちく): 
\\	区切り(くぎり): 
\\	区切る(くぎる): 
\\	区々(まちまち): 
\\	匚: 
\\	乂: 
\\	匸	
\\	かくしがまえ	區	会意 
\\	医	イ	い.やす、い.する、くすし		
\\	医療(いりょう): 
\\	医学(いがく): 
\\	医師(いし): 
\\	医者(いしゃ): 
\\	医院(いいん): 
\\	歯医者(はいしゃ): 
\\	匚: 
\\	矢: 
\\	酉	
\\	とり・とりへん・ひよみのとり・さけのとり	醫	会意 
\\	去	キョ、コ	さ.る、-さ.る	い	
\\	"過去(かこ): 
\\	去年(きょねん): 
\\	去る(さる): 
\\	立ち去る(たちさる): 
\\	消去(しょうきょ): 
\\	土: 
\\	厶: 
\\	厶	
\\	む	
\\	会意 
\\	反	ハン、ホン、タン、ホ	そ.る、そ.らす、かえ.す、かえ.る、-かえ.る	そり, た	
\\	反す(かえす): 
\\	違反(いはん): 
\\	反対(はんたい): 
\\	反発(はんぱつ): 
\\	反応(はんのう): 
\\	反省(はんせい): 
\\	反響(はんきょう): 
\\	反撃(はんげき): 
\\	反乱(はんらん): 
\\	反映(はんえい): 
\\	反感(はんかん): 
\\	反射(はんしゃ): 
\\	反する(はんする): 
\\	反る(かえる): 
\\	反抗(はんこう): 
\\	反(はん): 
\\	反る(そる): 
\\	厂: 
\\	又: 
\\	又	
\\	また	
\\	会意 
\\	取	シュ	と.る、と.り、と.り-、とり、-ど.り	どる	
\\	取扱い(とりあつかい): 
\\	取締まり(とりしまり): 
\\	受取り(うけとり): 
\\	引取る(ひきとる): 
\\	取引(とりひき): 
\\	取材(しゅざい): 
\\	取っ手(とって): 
\\	取る(とる): 
\\	受け取る(うけとる): 
\\	聞き取り(ききとり): 
\\	取り消す(とりけす): 
\\	取りあえず(とりあえず): 
\\	取上げる(とりあげる): 
\\	日取り(ひどり): 
\\	取り替える(とりかえる): 
\\	取り分け(とりわけ): 
\\	乗っ取る(のっとる): 
\\	取り扱う(とりあつかう): 
\\	取り寄せる(とりよせる): 
\\	取り出す(とりだす): 
\\	ちり取り(ちりとり): 
\\	書き取り(かきとり): 
\\	取り組む(とりくむ): 
\\	取り締まる(とりしまる): 
\\	取り戻す(とりもどす): 
\\	取り次ぐ(とりつぐ): 
\\	取り巻く(とりまく): 
\\	取り除く(とりのぞく): 
\\	取り入れる(とりいれる): 
\\	取れる(とれる): 
\\	書き取る(かきとる): 
\\	取り調べる(とりしらべる): 
\\	下取り(したどり): 
\\	取り立てる(とりたてる): 
\\	取り混ぜる(とりまぜる): 
\\	取り付ける(とりつける): 
\\	取り替え(とりかえ): 
\\	耳: 
\\	又: 
\\	又	
\\	また	
\\	会意 
\\	受	ジュ	う.ける、-う.け、う.かる	じょ	
\\	受取り(うけとり): 
\\	受ける(うける): 
\\	受身(うけみ): 
\\	受験(じゅけん): 
\\	受け止める(うけとめる): 
\\	受け入れる(うけいれる): 
\\	受付(うけつけ): 
\\	受け取る(うけとる): 
\\	受話器(じゅわき): 
\\	受け継ぐ(うけつぐ): 
\\	受け入れ(うけいれ): 
\\	引受ける(ひきうける): 
\\	享受(きょうじゅ): 
\\	受け付ける(うけつける): 
\\	受け持ち(うけもち): 
\\	受け持つ(うけもつ): 
\\	受かる(うかる): 
\\	爪: 
\\	冖: 
\\	又: 
\\	又	
\\	また	
\\	会意 
\\	号	ゴウ	さけ.ぶ、よびな		
\\	番号(ばんごう): 
\\	信号(しんごう): 
\\	号(ごう): 
\\	記号(きごう): 
\\	年号(ねんごう): 
\\	符号(ふごう): 
\\	口: 
\\	一: 
\\	虍	
\\	とら・とらがしら・とらかんむり	號	会意 
\\	向	コウ	む.く、む.い、-む.き、む.ける、-む.け、む.かう、む.かい、む.こう、む.こう-、むこ、むか.い	こお, た, な, むか, むかい, むこう	
\\	方向(ほうこう): 
\\	意向(いこう): 
\\	傾向(けいこう): 
\\	向上(こうじょう): 
\\	動向(どうこう): 
\\	向き(むき): 
\\	向かい(むかい): 
\\	向こう(むこう): 
\\	日向(ひなた): 
\\	向かう(むかう): 
\\	志向(しこう): 
\\	向く(むく): 
\\	向け(むけ): 
\\	向ける(むける): 
\\	振り向く(ふりむく): 
\\	丿: 
\\	冂: 
\\	口: 
\\	口	
\\	くち・くちへん	
\\	会意 
\\	君	クン	きみ、-ぎみ	み	
\\	諸君(しょくん): 
\\	君主(くんしゅ): 
\\	君(きみ): 
\\	君(くん): 
\\	尹: 
\\	口: 
\\	口	
\\	くち・くちへん	
\\	会意 
\\	味	ミ	あじ、あじ.わう		
\\	意味(いみ): 
\\	味(あじ): 
\\	興味(きょうみ): 
\\	趣味(しゅみ): 
\\	気味(きみ): 
\\	地味(じみ): 
\\	味方(みかた): 
\\	味覚(みかく): 
\\	調味料(ちょうみりょう): 
\\	味わい(あじわい): 
\\	三味線(しゃみせん): 
\\	加味(かみ): 
\\	味わう(あじわう): 
\\	味噌(みそ): 
\\	無意味(むいみ): 
\\	吟味(ぎんみ): 
\\	美味しい(おいしい): 
\\	不味い(まずい): 
\\	正味(しょうみ): 
\\	気味(ぎみ): 
\\	味(み): 
\\	五味(ごみ): 
\\	口: 
\\	未: 
\\	口	
\\	くち・くちへん	
\\	形声 
\\	命	メイ、ミョウ	いのち		
\\	"革命(かくめい): 
\\	生命(せいめい): 
\\	命(いのち): 
\\	命令(めいれい): 
\\	一生懸命(いっしょうけんめい): 
\\	運命(うんめい): 
\\	使命(しめい): 
\\	寿命(じゅみょう): 
\\	任命(にんめい): 
\\	宿命(しゅくめい): 
\\	命中(めいちゅう): 
\\	命じる(めいじる): 
\\	懸命(けんめい): 
\\	人命(じんめい): 
\\	命ずる(めいずる): 
\\	人: 
\\	一: 
\\	口: 
\\	卩: 
\\	口	
\\	くち・くちへん	
\\	会意 
\\	和	ワ、オ、カ	やわ.らぐ、やわ.らげる、なご.む、なご.やか	あい, いず, かず, かつ, かつり, かづ, たけ, ち, とも, な, にぎ, まさ, やす, よし, より, わだこ, わっ	
\\	平和(へいわ): 
\\	緩和(かんわ): 
\\	調和(ちょうわ): 
\\	和風(わふう): 
\\	和服(わふく): 
\\	英和(えいわ): 
\\	飽和(ほうわ): 
\\	穏和(おんわ): 
\\	共和(きょうわ): 
\\	和やか(なごやか): 
\\	中和(ちゅうわ): 
\\	和英(わえい): 
\\	和らげる(やわらげる): 
\\	漢和(かんわ): 
\\	和(わ): 
\\	和文(わぶん): 
\\	禾: 
\\	口: 
\\	口	
\\	くち・くちへん	
\\	会意 
\\	品	ヒン、ホン	しな		
\\	作品(さくひん): 
\\	商品(しょうひん): 
\\	製品(せいひん): 
\\	食品(しょくひん): 
\\	部品(ぶひん): 
\\	品質(ひんしつ): 
\\	品種(ひんしゅ): 
\\	薬品(やくひん): 
\\	出品(しゅっぴん): 
\\	品物(しなもの): 
\\	賞品(しょうひん): 
\\	日用品(にちようひん): 
\\	用品(ようひん): 
\\	食料品(しょくりょうひん): 
\\	手品(てじな): 
\\	下品(げひん): 
\\	上品(じょうひん): 
\\	必需品(ひつじゅひん): 
\\	品(しな): 
\\	品(ひん): 
\\	気品(きひん): 
\\	下品(かひん): 
\\	骨董品(こっとうひん): 
\\	洋品店(ようひんてん): 
\\	口: 
\\	口: 
\\	口: 
\\	口	
\\	くち・くちへん	
\\	会意 
\\	員	イン		いな, かず	
\\	委員(いいん): 
\\	会員(かいいん): 
\\	議員(ぎいん): 
\\	職員(しょくいん): 
\\	全員(ぜんいん): 
\\	従業員(じゅうぎょういん): 
\\	定員(ていいん): 
\\	公務員(こうむいん): 
\\	教員(きょういん): 
\\	店員(てんいん): 
\\	満員(まんいん): 
\\	動員(どういん): 
\\	行員(こういん): 
\\	工員(こういん): 
\\	員(いん): 
\\	口: 
\\	貝: 
\\	口	
\\	くち・くちへん	
\\	象形 
\\	商	ショウ	あきな.う	あき	
\\	商品(しょうひん): 
\\	商業(しょうぎょう): 
\\	商社(しょうしゃ): 
\\	商店(しょうてん): 
\\	商売(しょうばい): 
\\	商人(しょうにん): 
\\	商(しょう): 
\\	立: 
\\	冏: 
\\	口	
\\	くち・くちへん	
\\	会意 
\\	問	モン	と.う、と.い、とん	はる	
\\	質問(しつもん): 
\\	訪問(ほうもん): 
\\	問題(もんだい): 
\\	疑問(ぎもん): 
\\	問う(とう): 
\\	問い(とい): 
\\	学問(がくもん): 
\\	問屋(とんや): 
\\	問い合わせる(といあわせる): 
\\	問答(もんどう): 
\\	問い合わせ(といあわせ): 
\\	問(もん): 
\\	門: 
\\	口: 
\\	口	
\\	くち・くちへん	
\\	会意 
\\	坂	ハン	さか	か, ざ	
\\	坂(さか): 
\\	土: 
\\	反: 
\\	土	
\\	つち・つちへん・どへん	
\\	形声 
\\	央	オウ		あきら, え, お, さと, ちか, てる, なか, ひさ, ひろ, や	
\\	中央(ちゅうおう): 
\\	丨: 
\\	大: 
\\	大	
\\	だい・おおきい	
\\	象形 
\\	始	シ	はじ.める、-はじ.める、はじ.まる	もと	
\\	開始(かいし): 
\\	始まり(はじまり): 
\\	終始(しゅうし): 
\\	始末(しまつ): 
\\	始める(はじめる): 
\\	原始(げんし): 
\\	始発(しはつ): 
\\	始まる(はじまる): 
\\	始終(しじゅう): 
\\	始めに(はじめに): 
\\	女: 
\\	台: 
\\	女	
\\	おんな・おんなへん	
\\	形声 
\\	委	イ	ゆだ.ねる	まかせ	
\\	委員(いいん): 
\\	委託(いたく): 
\\	禾: 
\\	女: 
\\	女	
\\	おんな・おんなへん	
\\	会意 
\\	守	シュ、ス	まも.る、まも.り、もり、-もり、かみ	う, し, も, て	
\\	保守(ほしゅ): 
\\	守備(しゅび): 
\\	守る(まもる): 
\\	留守(るす): 
\\	留守番(るすばん): 
\\	守衛(しゅえい): 
\\	宀: 
\\	寸: 
\\	宀	
\\	うかんむり	
\\	会意 
\\	安	アン	やす.い、やす.まる、やす、やす.らか	あ, あす, あず, あっ, や	
\\	安全(あんぜん): 
\\	安定(あんてい): 
\\	不安(ふあん): 
\\	治安(ちあん): 
\\	安心(あんしん): 
\\	安易(あんい): 
\\	目安(めやす): 
\\	安静(あんせい): 
\\	安い(やすい): 
\\	安っぽい(やすっぽい): 
\\	宀: 
\\	女: 
\\	宀	
\\	うかんむり	
\\	会意 
\\	定	テイ、ジョウ	さだ.める、さだ.まる、さだ.か	さた	
\\	決定(けってい): 
\\	予定(よてい): 
\\	安定(あんてい): 
\\	一定(いってい): 
\\	規定(きてい): 
\\	指定(してい): 
\\	定期(ていき): 
\\	特定(とくてい): 
\\	制定(せいてい): 
\\	設定(せってい): 
\\	定員(ていいん): 
\\	確定(かくてい): 
\\	推定(すいてい): 
\\	定年(ていねん): 
\\	判定(はんてい): 
\\	否定(ひてい): 
\\	限定(げんてい): 
\\	未定(みてい): 
\\	固定(こてい): 
\\	測定(そくてい): 
\\	定義(ていぎ): 
\\	肯定(こうてい): 
\\	定価(ていか): 
\\	勘定(かんじょう): 
\\	定める(さだめる): 
\\	定期券(ていきけん): 
\\	断定(だんてい): 
\\	定食(ていしょく): 
\\	定まる(さだまる): 
\\	協定(きょうてい): 
\\	改定(かいてい): 
\\	所定(しょてい): 
\\	案の定(あんのじょう): 
\\	仮定(かてい): 
\\	定規(じょうぎ): 
\\	定休日(ていきゅうび): 
\\	国定(こくてい): 
\\	宀: 
\\	疋: 
\\	宀	
\\	うかんむり	
\\	会意 
\\	実	ジツ、シツ	み、みの.る、まこと、みの、みち.る	ぐみ, さね	
\\	実に(じつに): 
\\	忠実(まめ): 
\\	現実(げんじつ): 
\\	事実(じじつ): 
\\	実施(じっし): 
\\	確実(かくじつ): 
\\	実験(じっけん): 
\\	実現(じつげん): 
\\	実際(じっさい): 
\\	実績(じっせき): 
\\	実態(じったい): 
\\	実力(じつりょく): 
\\	実は(じつは): 
\\	実行(じっこう): 
\\	充実(じゅうじつ): 
\\	実情(じつじょう): 
\\	実用(じつよう): 
\\	実家(じっか): 
\\	実践(じっせん): 
\\	真実(しんじつ): 
\\	実感(じっかん): 
\\	誠実(せいじつ): 
\\	実習(じっしゅう): 
\\	切実(せつじつ): 
\\	果実(かじつ): 
\\	口実(こうじつ): 
\\	実例(じつれい): 
\\	実物(じつぶつ): 
\\	実費(じっぴ): 
\\	実質(じっしつ): 
\\	実(じつ): 
\\	実業家(じつぎょうか): 
\\	実る(みのる): 
\\	実(み): 
\\	宀: 
\\	三: 
\\	人: 
\\	宀	
\\	うかんむり	實	会意 
\\	客	キャク、カク			
\\	客(きゃく): 
\\	観客(かんきゃく): 
\\	乗客(じょうきゃく): 
\\	旅客(りょかく): 
\\	客席(きゃくせき): 
\\	客観(きゃっかん): 
\\	客間(きゃくま): 
\\	宀: 
\\	各: 
\\	宀	
\\	うかんむり	
\\	会意 
\\	宮	キュウ、グウ、ク、クウ	みや	ぐ, み	
\\	宮殿(きゅうでん): 
\\	お宮(おみや): 
\\	宀: 
\\	呂: 
\\	宀	
\\	うかんむり	
\\	会意 
\\	宿	シュク	やど、やど.る、やど.す	すく, ぶすき, やけ	
\\	宿泊(しゅくはく): 
\\	宿(やど): 
\\	宿題(しゅくだい): 
\\	宿命(しゅくめい): 
\\	下宿(げしゅく): 
\\	民宿(みんしゅく): 
\\	宀: 
\\	佰: 
\\	宀	
\\	うかんむり	
\\	会意 
\\	寒	カン	さむ.い	さ, さん	
\\	寒気(かんき): 
\\	寒い(さむい): 
\\	寒帯(かんたい): 
\\	宀: 
\\	三: 
\\	八: 
\\	冫: 
\\	宀	
\\	うかんむり	
\\	会意 
\\	対	タイ、ツイ	あいて、こた.える、そろ.い、つれあ.い、なら.ぶ、むか.う	つし	
\\	"対応(たいおう): 
\\	対策(たいさく): 
\\	対象(たいしょう): 
\\	反対(はんたい): 
\\	対立(たいりつ): 
\\	対話(たいわ): 
\\	絶対(ぜったい): 
\\	対決(たいけつ): 
\\	対抗(たいこう): 
\\	対処(たいしょ): 
\\	対等(たいとう): 
\\	対談(たいだん): 
\\	応対(おうたい): 
\\	対照(たいしょう): 
\\	対比(たいひ): 
\\	相対(あいたい): 
\\	対面(たいめん): 
\\	対する(たいする): 
\\	相対(そうたい): 
\\	対(たい): 
\\	対(つい): 
\\	対して(たいして): 
\\	文: 
\\	寸: 
\\	寸	
\\	すん・すんづくり	對	会意 
\\	局	キョク	つぼね		
\\	結局(けっきょく): 
\\	郵便局(ゆうびんきょく): 
\\	局(きょく): 
\\	薬局(やっきょく): 
\\	局限(きょくげん): 
\\	尸: 
\\	句: 
\\	尸	
\\	しかばね・かばね・かばねだれ	
\\	会意 
\\	屋	オク	や	た	
\\	部屋(へや): 
\\	屋根(やね): 
\\	小屋(こや): 
\\	屋上(おくじょう): 
\\	屋敷(やしき): 
\\	家屋(かおく): 
\\	屋外(おくがい): 
\\	問屋(とんや): 
\\	八百屋(やおや): 
\\	大屋(おおや): 
\\	床屋(とこや): 
\\	店屋(みせや): 
\\	屋(や): 
\\	尸: 
\\	至: 
\\	尸	
\\	しかばね・かばね・かばねだれ	
\\	会意 
\\	岸	ガン	きし	けし	
\\	海岸(かいがん): 
\\	沿岸(えんがん): 
\\	岸(きし): 
\\	山: 
\\	厂: 
\\	干: 
\\	山	
\\	やま・やまへん・やまかんむり	
\\	形声 
\\	島	トウ	しま		
\\	島(しま): 
\\	半島(はんとう): 
\\	列島(れっとう): 
\\	島(とう): 
\\	鳥: 
\\	山: 
\\	山	
\\	やま・やまへん・やまかんむり	
\\	会意 
\\	州	シュウ、ス	す	くに	
\\	州(しゅう): 
\\	丶: 
\\	巛: 
\\	丶: 
\\	丶: 
\\	巛	
\\	かわ・まがりがわ・さんぼんがわ	
\\	象形 
\\	帳	チョウ	とばり		
\\	手帳(てちょう): 
\\	通帳(つうちょう): 
\\	几帳面(きちょうめん): 
\\	帳(ちょう): 
\\	帳(とばり): 
\\	巾: 
\\	長: 
\\	巾	
\\	はば・はばへん・きんべん	
\\	形声 
\\	平	ヘイ、ビョウ、ヒョウ	たい.ら、-だいら、ひら、ひら-	たいら, たら, はち, ひ, ひとし, へ, へん	
\\	平行(へいこう): 
\\	平和(へいわ): 
\\	平均(へいきん): 
\\	平等(びょうどう): 
\\	平日(へいじつ): 
\\	平野(へいや): 
\\	公平(こうへい): 
\\	水平(すいへい): 
\\	平常(へいじょう): 
\\	地平線(ちへいせん): 
\\	不平(ふへい): 
\\	平凡(へいぼん): 
\\	平気(へいき): 
\\	水平線(すいへいせん): 
\\	平方(へいほう): 
\\	平仮名(ひらがな): 
\\	平たい(ひらたい): 
\\	平ら(たいら): 
\\	干: 
\\	干	
\\	かん・いちじゅう・ほす・ひる・たてかん	
\\	会意 
\\	幸	コウ	さいわ.い、さち、しあわ.せ	こ, さき, さし, さっ, とも, ひろ, みゆき, ゆ, ゆき, よし	
\\	不幸(ふこう): 
\\	幸せ(しあわせ): 
\\	幸福(こうふく): 
\\	幸い(さいわい): 
\\	幸運(こううん): 
\\	土: 
\\	干: 
\\	干	
\\	かん・いちじゅう・ほす・ひる・たてかん	
\\	象形 
\\	度	ド、ト、タク	たび、-た.い	のり	
\\	制度(せいど): 
\\	程度(ていど): 
\\	今度(こんど): 
\\	態度(たいど): 
\\	年度(ねんど): 
\\	温度(おんど): 
\\	高度(こうど): 
\\	限度(げんど): 
\\	速度(そくど): 
\\	角度(かくど): 
\\	密度(みつど): 
\\	支度(したく): 
\\	湿度(しつど): 
\\	適度(てきど): 
\\	度々(たびたび): 
\\	一度(いちど): 
\\	濃度(のうど): 
\\	緯度(いど): 
\\	感度(かんど): 
\\	毎度(まいど): 
\\	経度(けいど): 
\\	屹度(きっと): 
\\	お目出度う(おめでとう): 
\\	丁度(ちょうど): 
\\	度(たび): 
\\	度(ど): 
\\	目出度い(めでたい): 
\\	一度に(いちどに): 
\\	加速度(かそくど): 
\\	調度(ちょうど): 
\\	お目出度うご座います(おめでとうございます): 
\\	度忘れ(どわすれ): 
\\	進度(しんど): 
\\	お目出度い(おめでたい): 
\\	もう一度(もういちど): 
\\	广: 
\\	廿: 
\\	又: 
\\	广	
\\	まだれ	
\\	会意 
\\	庫	コ、ク	くら		
\\	倉庫(そうこ): 
\\	冷蔵庫(れいぞうこ): 
\\	金庫(きんこ): 
\\	車庫(しゃこ): 
\\	在庫(ざいこ): 
\\	广: 
\\	車: 
\\	广	
\\	まだれ	
\\	会意 
\\	庭	テイ	にわ	ば	
\\	家庭(かてい): 
\\	庭(にわ): 
\\	校庭(こうてい): 
\\	广: 
\\	廷: 
\\	广	
\\	まだれ	
\\	形声 
\\	式	シキ		のり	
\\	株式(かぶしき): 
\\	公式(こうしき): 
\\	正式(せいしき): 
\\	形式(けいしき): 
\\	儀式(ぎしき): 
\\	様式(ようしき): 
\\	葬式(そうしき): 
\\	方式(ほうしき): 
\\	方程式(ほうていしき): 
\\	式場(しきじょう): 
\\	日本式(にほんしき): 
\\	式(しき): 
\\	弋: 
\\	工: 
\\	弋	
\\	しきがまえ・いぐるみ	
\\	会意 
\\	役	ヤク、エキ		ちゃく	
\\	役割(やくわり): 
\\	役(やく): 
\\	主役(しゅやく): 
\\	役所(やくしょ): 
\\	役者(やくしゃ): 
\\	役人(やくにん): 
\\	役目(やくめ): 
\\	役場(やくば): 
\\	重役(じゅうやく): 
\\	役に立つ(やくにたつ): 
\\	役職(やくしょく): 
\\	役立つ(やくだつ): 
\\	彳: 
\\	殳: 
\\	彳	
\\	ぎょうにんべん	
\\	会意 
\\	待	タイ	ま.つ、-ま.ち	まち, まつ	
\\	期待(きたい): 
\\	招待(しょうたい): 
\\	待遇(たいぐう): 
\\	待望(たいぼう): 
\\	待つ(まつ): 
\\	待ち望む(まちのぞむ): 
\\	待ち遠しい(まちどおしい): 
\\	待ち合わせ(まちあわせ): 
\\	待ち合わせる(まちあわせる): 
\\	待合室(まちあいしつ): 
\\	お待ち遠様(おまちどおさま): 
\\	お待ち(おまち): 
\\	お待たせしました(おまたせしました): 
\\	彳: 
\\	寺: 
\\	彳	
\\	ぎょうにんべん	
\\	形声 
\\	急	キュウ	いそ.ぐ、いそ.ぎ		
\\	緊急(きんきゅう): 
\\	急(きゅう): 
\\	急速(きゅうそく): 
\\	急激(きゅうげき): 
\\	特急(とっきゅう): 
\\	急行(きゅうこう): 
\\	急ぐ(いそぐ): 
\\	早急(さっきゅう): 
\\	応急(おうきゅう): 
\\	至急(しきゅう): 
\\	急かす(せかす): 
\\	準急(じゅんきゅう): 
\\	急に(きゅうに): 
\\	勹: 
\\	彑: 
\\	心: 
\\	心	
\\	こころ・したごころ・りっしんべん	
\\	形声 
\\	息	ソク	いき		
\\	息子(むすこ): 
\\	息(いき): 
\\	利息(りそく): 
\\	消息(しょうそく): 
\\	窒息(ちっそく): 
\\	休息(きゅうそく): 
\\	ため息(ためいき): 
\\	子息(しそく): 
\\	一息(ひといき): 
\\	自: 
\\	心: 
\\	心	
\\	こころ・したごころ・りっしんべん	
\\	会意 
\\	悪	アク、オ	わる.い、わる-、あ.し、にく.い、-にく.い、ああ、いずくに、いずくんぞ、にく.む		
\\	悪口(わるくち): 
\\	悪化(あっか): 
\\	悪(あく): 
\\	悪魔(あくま): 
\\	悪者(わるもの): 
\\	意地悪(いじわる): 
\\	悪い(わるい): 
\\	悪しからず(あしからず): 
\\	悪戯(いたずら): 
\\	善し悪し(よしあし): 
\\	改悪(かいあく): 
\\	きまり悪い(きまりわるい): 
\\	悪どい(あくどい): 
\\	悪日(あくにち): 
\\	亜: 
\\	心: 
\\	心	
\\	こころ・したごころ・りっしんべん	惡	形声 
\\	悲	ヒ	かな.しい、かな.しむ		
\\	悲劇(ひげき): 
\\	悲観(ひかん): 
\\	悲惨(ひさん): 
\\	悲鳴(ひめい): 
\\	悲しむ(かなしむ): 
\\	悲しい(かなしい): 
\\	非: 
\\	心: 
\\	心	
\\	こころ・したごころ・りっしんべん	
\\	形声 
\\	想	ソウ、ソ	おも.う		
\\	愛想(あいそ): 
\\	構想(こうそう): 
\\	予想(よそう): 
\\	思想(しそう): 
\\	感想(かんそう): 
\\	発想(はっそう): 
\\	理想(りそう): 
\\	想像(そうぞう): 
\\	空想(くうそう): 
\\	連想(れんそう): 
\\	相: 
\\	心: 
\\	心	
\\	こころ・したごころ・りっしんべん	
\\	形声 
\\	意	イ		もと, よし	
\\	意見(いけん): 
\\	意味(いみ): 
\\	合意(ごうい): 
\\	意向(いこう): 
\\	意識(いしき): 
\\	意欲(いよく): 
\\	決意(けつい): 
\\	意外(いがい): 
\\	意義(いぎ): 
\\	注意(ちゅうい): 
\\	意図(いと): 
\\	得意(とくい): 
\\	用意(ようい): 
\\	同意(どうい): 
\\	熱意(ねつい): 
\\	好意(こうい): 
\\	意地(いじ): 
\\	敬意(けいい): 
\\	意志(いし): 
\\	意思(いし): 
\\	不意(ふい): 
\\	意気込む(いきごむ): 
\\	無意味(むいみ): 
\\	意地悪(いじわる): 
\\	生意気(なまいき): 
\\	意(い): 
\\	音: 
\\	心: 
\\	心	
\\	こころ・したごころ・りっしんべん	
\\	会意 
\\	感	カン			
\\	感じ(かんじ): 
\\	感覚(かんかく): 
\\	感染(かんせん): 
\\	感情(かんじょう): 
\\	感想(かんそう): 
\\	感動(かんどう): 
\\	感謝(かんしゃ): 
\\	実感(じっかん): 
\\	敏感(びんかん): 
\\	感じる(かんじる): 
\\	感心(かんしん): 
\\	共感(きょうかん): 
\\	感激(かんげき): 
\\	感触(かんしょく): 
\\	予感(よかん): 
\\	同感(どうかん): 
\\	反感(はんかん): 
\\	痛感(つうかん): 
\\	感ずる(かんずる): 
\\	鈍感(どんかん): 
\\	感度(かんど): 
\\	感無量(かんむりょう): 
\\	感(かん): 
\\	咸: 
\\	心: 
\\	心	
\\	こころ・したごころ・りっしんべん	
\\	形声 
\\	所	ショ	ところ、-ところ、どころ、とこ	せ	
\\	事務所(じむしょ): 
\\	場所(ばしょ): 
\\	所得(しょとく): 
\\	住所(じゅうしょ): 
\\	所属(しょぞく): 
\\	所有(しょゆう): 
\\	近所(きんじょ): 
\\	役所(やくしょ): 
\\	台所(だいどころ): 
\\	名所(めいしょ): 
\\	箇所(かしょ): 
\\	長所(ちょうしょ): 
\\	所在(しょざい): 
\\	所持(しょじ): 
\\	所定(しょてい): 
\\	便所(べんじょ): 
\\	短所(たんしょ): 
\\	所々(ところどころ): 
\\	停留所(ていりゅうじょ): 
\\	所が(ところが): 
\\	所(ところ): 
\\	所で(ところで): 
\\	所謂(いわゆる): 
\\	余所(よそ): 
\\	所為(せい): 
\\	所か(どころか): 
\\	所(しょ): 
\\	戸: 
\\	斤: 
\\	戸	
\\	と・とだれ・とかんむり	
\\	会意 
\\	打	ダ、ダアス	う.つ、う.ち-、ぶ.つ	うち	
\\	打撃(だげき): 
\\	打つ(うつ): 
\\	打ち込む(うちこむ): 
\\	打ち合わせ(うちあわせ): 
\\	打消す(うちけす): 
\\	打消し(うちけし): 
\\	打開(だかい): 
\\	値打ち(ねうち): 
\\	打ち切る(うちきる): 
\\	打ち明ける(うちあける): 
\\	打付ける(ぶつける): 
\\	打(ダース): 
\\	打つ(ぶつ): 
\\	打ち合わせる(うちあわせる): 
\\	打つかる(ぶつかる): 
\\	手: 
\\	丁: 
\\	手	
\\	て・てへん	
\\	会意 
\\	投	トウ	な.げる、-な.げ		
\\	投資(とうし): 
\\	投票(とうひょう): 
\\	投書(とうしょ): 
\\	投入(とうにゅう): 
\\	投げ出す(なげだす): 
\\	投げる(なげる): 
\\	手: 
\\	殳: 
\\	手	
\\	て・てへん	
\\	会意 
\\	拾	シュウ、ジュウ	ひろ.う		
\\	拾う(ひろう): 
\\	手: 
\\	合: 
\\	手	
\\	て・てへん	
\\	形声 
\\	持	ジ	も.つ、-も.ち、も.てる	もち, もつ	
\\	維持(いじ): 
\\	支持(しじ): 
\\	気持ち(きもち): 
\\	持ち(もち): 
\\	持つ(もつ): 
\\	持続(じぞく): 
\\	金持ち(かねもち): 
\\	持参(じさん): 
\\	所持(しょじ): 
\\	受け持ち(うけもち): 
\\	受け持つ(うけもつ): 
\\	持てる(もてる): 
\\	持て成す(もてなす): 
\\	持ち上げる(もちあげる): 
\\	持ち切り(もちきり): 
\\	手: 
\\	寺: 
\\	手	
\\	て・てへん	
\\	形声 
\\	指	シ	ゆび、さ.す、-さ.し	い, いぶ, さし, さす	
\\	物指(ものさし): 
\\	指導(しどう): 
\\	指揮(しき): 
\\	指定(してい): 
\\	指摘(してき): 
\\	指(ゆび): 
\\	指輪(ゆびわ): 
\\	親指(おやゆび): 
\\	小指(こゆび): 
\\	指令(しれい): 
\\	指図(さしず): 
\\	中指(なかゆび): 
\\	薬指(くすりゆび): 
\\	目指す(めざす): 
\\	指す(さす): 
\\	指差す(ゆびさす): 
\\	人差し指(ひとさしゆび): 
\\	手: 
\\	旨: 
\\	手	
\\	て・てへん	
\\	形声 
\\	放	ホウ	はな.す、-っぱな.し、はな.つ、はな.れる、こ.く、ほう.る	はなれ	
\\	放送(ほうそう): 
\\	開放(かいほう): 
\\	解放(かいほう): 
\\	追放(ついほう): 
\\	放る(ほうる): 
\\	放射能(ほうしゃのう): 
\\	放置(ほうち): 
\\	放出(ほうしゅつ): 
\\	放射(ほうしゃ): 
\\	放す(はなす): 
\\	放れる(はなれる): 
\\	放り込む(ほうりこむ): 
\\	放り出す(ほうりだす): 
\\	方: 
\\	攵: 
\\	攴	
\\	ぼくにょう・ぼくづくり・とまた・のぶん・しぶん	
\\	会意 
\\	整	セイ	ととの.える、ととの.う	ひとし	
\\	調整(ちょうせい): 
\\	整備(せいび): 
\\	整理(せいり): 
\\	整う(ととのう): 
\\	整列(せいれつ): 
\\	整える(ととのえる): 
\\	整数(せいすう): 
\\	整然(せいぜん): 
\\	敕: 
\\	正: 
\\	攴	
\\	ぼくにょう・ぼくづくり・とまた・のぶん・しぶん	
\\	形声 
\\	旅	リョ	たび		
\\	旅行(りょこう): 
\\	旅(たび): 
\\	旅客(りょかく): 
\\	旅館(りょかん): 
\\	旅券(りょけん): 
\\	方: 
\\	人: 
\\	方	
\\	ほう・ほうへん・かた・かたへん	
\\	会意 
\\	族	ゾク		つぎ	
\\	家族(かぞく): 
\\	民族(みんぞく): 
\\	貴族(きぞく): 
\\	方: 
\\	矢: 
\\	方	
\\	ほう・ほうへん・かた・かたへん	
\\	会意 
\\	昔	セキ、シャク	むかし		
\\	昔(むかし): 
\\	廾: 
\\	日: 
\\	日	
\\	ひ・ひへん・にち・にちへん	
\\	仮借 
\\	昭	ショウ		あき, あきら, かず, かずみ, てる	
\\	日: 
\\	召: 
\\	日	
\\	ひ・ひへん・にち・にちへん	
\\	形声 
\\	暑	ショ	あつ.い		
\\	暑い(あつい): 
\\	蒸し暑い(むしあつい): 
\\	日: 
\\	者: 
\\	日	
\\	ひ・ひへん・にち・にちへん	暑	形声 
\\	暗	アン	くら.い、くら.む、くれ.る		
\\	暗殺(あんさつ): 
\\	真っ暗(まっくら): 
\\	薄暗い(うすぐらい): 
\\	暗記(あんき): 
\\	暗示(あんじ): 
\\	暗い(くらい): 
\\	暗算(あんざん): 
\\	日: 
\\	音: 
\\	日	
\\	ひ・ひへん・にち・にちへん	
\\	形声 
\\	曲	キョク	ま.がる、ま.げる、くま	まがた	
\\	曲(きょく): 
\\	作曲(さっきょく): 
\\	戯曲(ぎきょく): 
\\	曲線(きょくせん): 
\\	曲がる(まがる): 
\\	曲げる(まげる): 
\\	婉曲(えんきょく): 
\\	曰: 
\\	廾: 
\\	曰	
\\	ひらび・いわく	
\\	象形 
\\	有	ユウ、ウ	あ.る	あ, あら, あり, ある, くに, なお, ゆ	
\\	有る(ある): 
\\	有様(ありさま): 
\\	有効(ゆうこう): 
\\	有力(ゆうりょく): 
\\	有名(ゆうめい): 
\\	有利(ゆうり): 
\\	所有(しょゆう): 
\\	有無(うむ): 
\\	有料(ゆうりょう): 
\\	固有(こゆう): 
\\	特有(とくゆう): 
\\	国有(こくゆう): 
\\	有望(ゆうぼう): 
\\	有益(ゆうえき): 
\\	有能(ゆうのう): 
\\	有りのまま(ありのまま): 
\\	有(ゆう): 
\\	有難い(ありがたい): 
\\	有機(ゆうき): 
\\	私有(しゆう): 
\\	有する(ゆうする): 
\\	有難うございます(ありがとうございます): 
\\	有り難う(ありがとう): 
\\	有らゆる(あらゆる): 
\\	有り触れる(ありふれる): 
\\	どうも有難うございます(どうもありがとうございます): 
\\	肉: 
\\	月	
\\	つき・つきへん	
\\	会意 
\\	服	フク		はっ, はつ, はら	
\\	服(ふく): 
\\	制服(せいふく): 
\\	服装(ふくそう): 
\\	不服(ふふく): 
\\	洋服(ようふく): 
\\	克服(こくふく): 
\\	衣服(いふく): 
\\	和服(わふく): 
\\	征服(せいふく): 
\\	軍服(ぐんぷく): 
\\	肉: 
\\	卩: 
\\	又: 
\\	月	
\\	つき・つきへん	
\\	形声 
\\	期	キ、ゴ			
\\	期間(きかん): 
\\	期待(きたい): 
\\	時期(じき): 
\\	長期(ちょうき): 
\\	期限(きげん): 
\\	定期(ていき): 
\\	短期(たんき): 
\\	延期(えんき): 
\\	末期(まっき): 
\\	期日(きじつ): 
\\	周期(しゅうき): 
\\	学期(がっき): 
\\	予期(よき): 
\\	定期券(ていきけん): 
\\	期末(きまつ): 
\\	画期(かっき): 
\\	期(き): 
\\	其: 
\\	肉: 
\\	月	
\\	つき・つきへん	
\\	形声 
\\	板	ハン、バン	いた		
\\	看板(かんばん): 
\\	板(いた): 
\\	黒板(こくばん): 
\\	木: 
\\	反: 
\\	木	
\\	き・きへん	
\\	形声 
\\	柱	チュウ	はしら		
\\	柱(はしら): 
\\	電柱(でんちゅう): 
\\	木: 
\\	主: 
\\	木	
\\	き・きへん	
\\	形声 
\\	根	コン	ね、-ね		
\\	根元(ねもと): 
\\	根拠(こんきょ): 
\\	根(ね): 
\\	屋根(やね): 
\\	根回し(ねまわし): 
\\	垣根(かきね): 
\\	根気(こんき): 
\\	根底(こんてい): 
\\	球根(きゅうこん): 
\\	利根(りこん): 
\\	木: 
\\	艮: 
\\	木	
\\	き・きへん	
\\	形声 
\\	植	ショク	う.える、う.わる	うえ, え, げ	
\\	植物(しょくぶつ): 
\\	植民地(しょくみんち): 
\\	田植え(たうえ): 
\\	植木(うえき): 
\\	植える(うえる): 
\\	植わる(うわる): 
\\	木: 
\\	直: 
\\	木	
\\	き・きへん	
\\	形声 
\\	業	ギョウ、ゴウ	わざ	なり, のぶ	
\\	企業(きぎょう): 
\\	業者(ぎょうしゃ): 
\\	作業(さぎょう): 
\\	産業(さんぎょう): 
\\	事業(じぎょう): 
\\	農業(のうぎょう): 
\\	営業(えいぎょう): 
\\	工業(こうぎょう): 
\\	業績(ぎょうせき): 
\\	失業(しつぎょう): 
\\	授業(じゅぎょう): 
\\	従業員(じゅうぎょういん): 
\\	職業(しょくぎょう): 
\\	卒業(そつぎょう): 
\\	商業(しょうぎょう): 
\\	漁業(ぎょぎょう): 
\\	休業(きゅうぎょう): 
\\	林業(りんぎょう): 
\\	修業(しゅうぎょう): 
\\	鉱業(こうぎょう): 
\\	分業(ぶんぎょう): 
\\	業務(ぎょうむ): 
\\	就業(しゅうぎょう): 
\\	実業家(じつぎょうか): 
\\	興業(こうぎょう): 
\\	兼業(けんぎょう): 
\\	業(ぎょう): 
\\	八: 
\\	羊: 
\\	木: 
\\	木	
\\	き・きへん	
\\	象形 
\\	様	ヨウ、ショウ	さま、さん		
\\	有様(ありさま): 
\\	同様(どうよう): 
\\	様子(ようす): 
\\	多様(たよう): 
\\	模様(もよう): 
\\	様相(ようそう): 
\\	様々(さまざま): 
\\	様式(ようしき): 
\\	神様(かみさま): 
\\	一様(いちよう): 
\\	王様(おうさま): 
\\	仕様(しよう): 
\\	殿様(とのさま): 
\\	お蔭様で(おかげさまで): 
\\	左様なら(さようなら): 
\\	様(さま): 
\\	逆様(さかさま): 
\\	様(よう): 
\\	お待ち遠様(おまちどおさま): 
\\	ご馳走様でした(ごちそうさまでした): 
\\	仕様が無い(しようがない): 
\\	木: 
\\	羊: 
\\	水: 
\\	木	
\\	き・きへん	樣	形声 
\\	横	オウ	よこ		
\\	横(よこ): 
\\	横綱(よこづな): 
\\	横断(おうだん): 
\\	横切る(よこぎる): 
\\	木: 
\\	黄: 
\\	木	
\\	き・きへん	橫	形声 
\\	橋	キョウ	はし	ばせ	
\\	橋(はし): 
\\	桟橋(さんばし): 
\\	橋渡し(はしわたし): 
\\	鉄橋(てっきょう): 
\\	橋(きょう): 
\\	木: 
\\	喬: 
\\	木	
\\	き・きへん	
\\	形声 
\\	次	ジ、シ	つ.ぐ、つぎ	き, すき, つぐ, よし	
\\	次ぐ(つぐ): 
\\	次第(しだい): 
\\	次々(つぎつぎ): 
\\	目次(もくじ): 
\\	次いで(ついで): 
\\	取り次ぐ(とりつぐ): 
\\	次(つぎ): 
\\	次(じ): 
\\	冫: 
\\	欠: 
\\	欠	
\\	あくび・けんづくり・かける・けつ	
\\	象形 
\\	歯	シ	よわい、は、よわ.い、よわい.する		
\\	歯(は): 
\\	歯科(しか): 
\\	虫歯(むしば): 
\\	歯車(はぐるま): 
\\	歯磨き(はみがき): 
\\	歯医者(はいしゃ): 
\\	止: 
\\	米: 
\\	凵: 
\\	齒	
\\	は・はへん	齒	形声 
\\	死	シ	し.ぬ、し.に-		
\\	死亡(しぼう): 
\\	死刑(しけい): 
\\	必死(ひっし): 
\\	死体(したい): 
\\	生死(せいし): 
\\	死(し): 
\\	死ぬ(しぬ): 
\\	歹: 
\\	匕: 
\\	歹	
\\	がつ・がってん・がつへん・かばね・かばねへん・いちたへん	
\\	会意 
\\	氷	ヒョウ	こおり、ひ、こお.る	すい	
\\	氷(こおり): 
\\	水: 
\\	丶: 
\\	水	
\\	みず・したみず・さんずい	
\\	象形 
\\	決	ケツ	き.める、-ぎ.め、き.まる、さ.く		
\\	解決(かいけつ): 
\\	決議(けつぎ): 
\\	決定(けってい): 
\\	判決(はんけつ): 
\\	決算(けっさん): 
\\	決勝(けっしょう): 
\\	決して(けっして): 
\\	決意(けつい): 
\\	決断(けつだん): 
\\	採決(さいけつ): 
\\	対決(たいけつ): 
\\	議決(ぎけつ): 
\\	可決(かけつ): 
\\	否決(ひけつ): 
\\	決まる(きまる): 
\\	決心(けっしん): 
\\	決める(きめる): 
\\	決まり(きまり): 
\\	決(けつ): 
\\	多数決(たすうけつ): 
\\	決行(けっこう): 
\\	水: 
\\	夬: 
\\	水	
\\	みず・したみず・さんずい	
\\	形声 
\\	油	ユ、ユウ	あぶら		
\\	石油(せきゆ): 
\\	原油(げんゆ): 
\\	油(あぶら): 
\\	灯油(とうゆ): 
\\	油絵(あぶらえ): 
\\	油断(ゆだん): 
\\	醤油(しょうゆ): 
\\	水: 
\\	由: 
\\	水	
\\	みず・したみず・さんずい	
\\	形声 
\\	波	ハ	なみ	ひら, みな, みなみ, わ	
\\	波(なみ): 
\\	電波(でんぱ): 
\\	津波(つなみ): 
\\	短波(たんぱ): 
\\	水: 
\\	皮: 
\\	水	
\\	みず・したみず・さんずい	
\\	形声 
\\	注	チュウ	そそ.ぐ、さ.す、つ.ぐ		
\\	注目(ちゅうもく): 
\\	注意(ちゅうい): 
\\	注文(ちゅうもん): 
\\	注射(ちゅうしゃ): 
\\	注(ちゅう): 
\\	注ぐ(そそぐ): 
\\	注す(さす): 
\\	注ぐ(つぐ): 
\\	水: 
\\	主: 
\\	水	
\\	みず・したみず・さんずい	
\\	形声 
\\	泳	エイ	およ.ぐ		
\\	水泳(すいえい): 
\\	泳ぐ(およぐ): 
\\	泳ぎ(およぎ): 
\\	泳ぎ方(およぎかた): 
\\	水: 
\\	永: 
\\	水	
\\	みず・したみず・さんずい	
\\	形声 
\\	洋	ヨウ		なだ, ひろ, ひろし, よ, よし	
\\	東洋(とうよう): 
\\	海洋(かいよう): 
\\	西洋(せいよう): 
\\	洋服(ようふく): 
\\	洋風(ようふう): 
\\	洋品店(ようひんてん): 
\\	水: 
\\	羊: 
\\	水	
\\	みず・したみず・さんずい	
\\	形声 
\\	流	リュウ、ル	なが.れる、なが.れ、なが.す、-なが.す	な, ながれ, めぐる	
\\	"流行(はやり): 
\\	交流(こうりゅう): 
\\	流れ(ながれ): 
\\	流通(りゅうつう): 
\\	一流(いちりゅう): 
\\	流す(ながす): 
\\	流し(ながし): 
\\	電流(でんりゅう): 
\\	流域(りゅういき): 
\\	海流(かいりゅう): 
\\	流れる(ながれる): 
\\	流(りゅう): 
\\	流行る(はやる): 
\\	流石(さすが): 
\\	合流(ごうりゅう): 
\\	気流(きりゅう): 
\\	直流(ちょくりゅう): 
\\	水: 
\\	㐬: 
\\	水	
\\	みず・したみず・さんずい	
\\	会意 
\\	消	ショウ	き.える、け.す		
\\	消費(しょうひ): 
\\	消極的(しょうきょくてき): 
\\	消防(しょうぼう): 
\\	消息(しょうそく): 
\\	消しゴム(けしゴム): 
\\	消化(しょうか): 
\\	消毒(しょうどく): 
\\	取り消す(とりけす): 
\\	消耗(しょうもう): 
\\	打消す(うちけす): 
\\	打消し(うちけし): 
\\	消す(けす): 
\\	消える(きえる): 
\\	消防署(しょうぼうしょ): 
\\	消去(しょうきょ): 
\\	水: 
\\	肖: 
\\	水	
\\	みず・したみず・さんずい	
\\	形声 
\\	深	シン	ふか.い、-ぶか.い、ふか.まる、ふか.める、み-		
\\	深刻(しんこく): 
\\	深める(ふかめる): 
\\	深まる(ふかまる): 
\\	深い(ふかい): 
\\	深夜(しんや): 
\\	情け深い(なさけぶかい): 
\\	欲深い(よくぶかい): 
\\	水: 
\\	㓁: 
\\	木: 
\\	水	
\\	みず・したみず・さんずい	
\\	形声 
\\	温	オン	あたた.か、あたた.かい、あたた.まる、あたた.める、ぬく	あつ, あつし, なお, はる, ゆ, ゆたか	
\\	気温(きおん): 
\\	温泉(おんせん): 
\\	温暖(おんだん): 
\\	温度(おんど): 
\\	温室(おんしつ): 
\\	体温(たいおん): 
\\	保温(ほおん): 
\\	温帯(おんたい): 
\\	温める(あたためる): 
\\	温い(ぬるい): 
\\	水: 
\\	日: 
\\	皿: 
\\	水	
\\	みず・したみず・さんずい	溫	形声 
\\	港	コウ	みなと		
\\	空港(くうこう): 
\\	港(みなと): 
\\	港(こう): 
\\	水: 
\\	巷: 
\\	水	
\\	みず・したみず・さんずい	
\\	形声 
\\	湖	コ	みずうみ	うみ, み	
\\	湖(みずうみ): 
\\	湖(こ): 
\\	水: 
\\	胡: 
\\	水	
\\	みず・したみず・さんずい	
\\	形声 
\\	湯	トウ	ゆ		
\\	湯(ゆ): 
\\	熱湯(ねっとう): 
\\	湯気(ゆげ): 
\\	湯飲み(ゆのみ): 
\\	茶の湯(ちゃのゆ): 
\\	水: 
\\	昜: 
\\	水	
\\	みず・したみず・さんずい	
\\	形声 
\\	漢	カン		はん	
\\	漢字(かんじ): 
\\	漢語(かんご): 
\\	漢和(かんわ): 
\\	水: 
\\	艸: 
\\	口: 
\\	夫: 
\\	水	
\\	みず・したみず・さんずい	漢	形声 
\\	炭	タン	すみ		
\\	石炭(せきたん): 
\\	炭素(たんそ): 
\\	炭鉱(たんこう): 
\\	山: 
\\	灰: 
\\	火	
\\	ひ・ひへん・れっか・れんが	
\\	会意 
\\	物	ブツ、モツ	もの、もの-		
\\	物指(ものさし): 
\\	買物(かいもの): 
\\	贈物(おくりもの): 
\\	乗物(のりもの): 
\\	飲物(のみもの): 
\\	人物(じんぶつ): 
\\	動物(どうぶつ): 
\\	建物(たてもの): 
\\	物価(ぶっか): 
\\	物語(ものがたり): 
\\	物質(ぶっしつ): 
\\	植物(しょくぶつ): 
\\	生物(せいぶつ): 
\\	博物館(はくぶつかん): 
\\	物資(ぶっし): 
\\	貨物(かもつ): 
\\	本物(ほんもの): 
\\	動物園(どうぶつえん): 
\\	食べ物(たべもの): 
\\	果物(くだもの): 
\\	荷物(にもつ): 
\\	着物(きもの): 
\\	物理(ぶつり): 
\\	物語る(ものがたる): 
\\	穀物(こくもつ): 
\\	食物(しょくもつ): 
\\	名物(めいぶつ): 
\\	産物(さんぶつ): 
\\	生き物(いきもの): 
\\	物事(ものごと): 
\\	見物(けんぶつ): 
\\	書物(しょもつ): 
\\	品物(しなもの): 
\\	作物(さくもつ): 
\\	物体(ぶったい): 
\\	実物(じつぶつ): 
\\	織物(おりもの): 
\\	鉱物(こうぶつ): 
\\	禁物(きんもつ): 
\\	物置(ものおき): 
\\	偽物(にせもの): 
\\	編物(あみもの): 
\\	農産物(のうさんぶつ): 
\\	物足りない(ものたりない): 
\\	物騒(ぶっそう): 
\\	物議(ぶつぎ): 
\\	物音(ものおと): 
\\	忘れ物(わすれもの): 
\\	獲物(えもの): 
\\	落とし物(おとしもの): 
\\	入れ物(いれもの): 
\\	物好き(ものずき): 
\\	瀬戸物(せともの): 
\\	物(もの): 
\\	物凄い(ものすごい): 
\\	私物(しぶつ): 
\\	見世物(みせもの): 
\\	干し物(ほしもの): 
\\	作物(さくぶつ): 
\\	出来物(できもの): 
\\	物(ぶつ): 
\\	牛: 
\\	勿: 
\\	牛	
\\	うし	
\\	形声 
\\	球	キュウ	たま	く	
\\	地球(ちきゅう): 
\\	球(きゅう): 
\\	電球(でんきゅう): 
\\	眼球(がんきゅう): 
\\	球根(きゅうこん): 
\\	玉: 
\\	求: 
\\	玉	
\\	たま・たまへん ・ おう・おうへん・あうさま	
\\	形声 
\\	由	ユ、ユウ、ユイ	よし、よ.る		
\\	自由(じゆう): 
\\	理由(りゆう): 
\\	由(よし): 
\\	経由(けいゆ): 
\\	不自由(ふじゆう): 
\\	田: 
\\	丨: 
\\	田	
\\	た・たへん	
\\	象形 
\\	申	シン	もう.す、もう.し-、さる		
\\	申込み(もうしこみ): 
\\	申請(しんせい): 
\\	申告(しんこく): 
\\	申し上げる(もうしあげる): 
\\	申し訳(もうしわけ): 
\\	申す(もうす): 
\\	申し出(もうしで): 
\\	申し入れる(もうしいれる): 
\\	申し出る(もうしでる): 
\\	申し訳ない(もうしわけない): 
\\	申し込む(もうしこむ): 
\\	申し分(もうしぶん): 
\\	田: 
\\	丨: 
\\	田	
\\	た・たへん	
\\	象形 
\\	界	カイ			
\\	世界(せかい): 
\\	限界(げんかい): 
\\	境界(きょうかい): 
\\	界(かい): 
\\	田: 
\\	介: 
\\	田	
\\	た・たへん	
\\	形声 
\\	畑		はた、はたけ、-ばたけ	かま, まま	
\\	畑(はたけ): 
\\	畑地(はたち): 
\\	火: 
\\	田: 
\\	田	
\\	た・たへん	
\\	国字 
\\	病	ビョウ、ヘイ	や.む、-や.み、やまい		
\\	病院(びょういん): 
\\	病気(びょうき): 
\\	発病(はつびょう): 
\\	看病(かんびょう): 
\\	病む(やむ): 
\\	臆病(おくびょう): 
\\	病(やまい): 
\\	疒: 
\\	丙: 
\\	疒	
\\	やまい・やまいだれ	
\\	形声 
\\	発	ハツ、ホツ	た.つ、あば.く、おこ.る、つか.わす、はな.つ	ば, わ	
\\	開発(かいはつ): 
\\	発言(はつげん): 
\\	発行(はっこう): 
\\	発展(はってん): 
\\	発表(はっぴょう): 
\\	反発(はんぱつ): 
\\	活発(かっぱつ): 
\\	発生(はっせい): 
\\	発売(はつばい): 
\\	発足(ほっそく): 
\\	爆発(ばくはつ): 
\\	発見(はっけん): 
\\	発想(はっそう): 
\\	出発(しゅっぱつ): 
\\	発電(はつでん): 
\\	発(はつ): 
\\	発掘(はっくつ): 
\\	発作(ほっさ): 
\\	発達(はったつ): 
\\	発病(はつびょう): 
\\	発明(はつめい): 
\\	発音(はつおん): 
\\	発揮(はっき): 
\\	始発(しはつ): 
\\	発育(はついく): 
\\	再発(さいはつ): 
\\	発射(はっしゃ): 
\\	発車(はっしゃ): 
\\	蒸発(じょうはつ): 
\\	発芽(はつが): 
\\	発条(ばね): 
\\	癶: 
\\	二: 
\\	八: 
\\	癶	
\\	はつがしら	發	会意 
\\	登	トウ、ト、ドウ、ショウ、チョウ	のぼ.る、あ.がる	たか, のぼし, のぼり, のり	
\\	登録(とうろく): 
\\	登場(とうじょう): 
\\	登校(とうこう): 
\\	登山(とざん): 
\\	癶: 
\\	豆: 
\\	癶	
\\	はつがしら	
\\	会意 
\\	皮	ヒ	かわ		
\\	"皮(かわ): 
\\	皮膚(ひふ): 
\\	皮肉(ひにく): 
\\	毛皮(けがわ): 
\\	皮: 
\\	皮	
\\	かわ・けがわ・ひのかわ	
\\	象形 
\\	皿	ベイ	さら		
\\	皿(さら): 
\\	灰皿(はいざら): 
\\	お皿(おさら): 
\\	皿: 
\\	皿	
\\	さら	
\\	象形 
\\	相	ソウ、ショウ	あい-	あ, い, おう, さ, さが, すけ	
\\	外相(がいしょう): 
\\	首相(しゅしょう): 
\\	相手(あいて): 
\\	相談(そうだん): 
\\	相互(そうご): 
\\	相場(そうば): 
\\	相当(そうとう): 
\\	相撲(すもう): 
\\	真相(しんそう): 
\\	相変わらず(あいかわらず): 
\\	様相(ようそう): 
\\	相違(そうい): 
\\	相続(そうぞく): 
\\	相応(そうおう): 
\\	可哀相(かわいそう): 
\\	蔵相(ぞうしょう): 
\\	相対(あいたい): 
\\	相応しい(ふさわしい): 
\\	相対(そうたい): 
\\	相(あい): 
\\	相(そう): 
\\	木: 
\\	目: 
\\	目	
\\	め・めへん・よこめ	
\\	会意 
\\	県	ケン	か.ける	あがた, がた	
\\	県庁(けんちょう): 
\\	県(けん): 
\\	目: 
\\	小: 
\\	糸	
\\	いと・いとへん	縣	会意 
\\	真	シン	ま、ま-、まこと	さな, さね, ただ, ただし, なお, のり, まあ, まこ, まさ, まっ, まど, まな, まゆ, みち, も	
\\	写真(しゃしん): 
\\	真相(しんそう): 
\\	真実(しんじつ): 
\\	真ん中(まんなか): 
\\	真理(しんり): 
\\	真っ暗(まっくら): 
\\	真っ赤(まっか): 
\\	真っ先(まっさき): 
\\	真っ白(まっしろ): 
\\	真珠(しんじゅ): 
\\	真剣(しんけん): 
\\	真っ青(まっさお): 
\\	真空(しんくう): 
\\	真っ二つ(まっぷたつ): 
\\	真心(まごころ): 
\\	真上(まうえ): 
\\	真面目(まじめ): 
\\	真似(まね): 
\\	真逆(まさか): 
\\	真似る(まねる): 
\\	真っ直ぐ(まっすぐ): 
\\	真っ黒(まっくろ): 
\\	真下(ました): 
\\	真ん円い(まんまるい): 
\\	真ん前(まんまえ): 
\\	生真面目(きまじめ): 
\\	十: 
\\	具: 
\\	目	
\\	め・めへん・よこめ	眞	会意 
\\	着	チャク、ジャク	き.る、-ぎ、き.せる、-き.せ、つ.く、つ.ける		
\\	着陸(ちゃくりく): 
\\	到着(とうちゃく): 
\\	着物(きもの): 
\\	下着(したぎ): 
\\	先着(せんちゃく): 
\\	着る(きる): 
\\	水着(みずぎ): 
\\	落ち着き(おちつき): 
\\	上着(うわぎ): 
\\	着手(ちゃくしゅ): 
\\	執着(しゅうちゃく): 
\\	着替える(きがえる): 
\\	着(ちゃく): 
\\	着工(ちゃっこう): 
\\	着目(ちゃくもく): 
\\	着替え(きがえ): 
\\	着色(ちゃくしょく): 
\\	着飾る(きかざる): 
\\	肌着(はだぎ): 
\\	着せる(きせる): 
\\	落ち着く(おちつく): 
\\	着々(ちゃくちゃく): 
\\	着席(ちゃくせき): 
\\	たどり着く(たどりつく): 
\\	着く(つく): 
\\	羊: 
\\	丿: 
\\	目: 
\\	羊	
\\	ひつじ・ひつじへん	
\\	形声 
\\	短	タン	みじか.い		
\\	短期(たんき): 
\\	短大(たんだい): 
\\	短縮(たんしゅく): 
\\	長短(ちょうたん): 
\\	短編(たんぺん): 
\\	短い(みじかい): 
\\	短所(たんしょ): 
\\	短気(たんき): 
\\	短波(たんぱ): 
\\	矢: 
\\	豆: 
\\	矢	
\\	や・やへん	
\\	形声 
\\	研	ケン	と.ぐ	のり	
\\	研究(けんきゅう): 
\\	研修(けんしゅう): 
\\	研究室(けんきゅうしつ): 
\\	研ぐ(とぐ): 
\\	石: 
\\	幵: 
\\	石	
\\	いし・いしへん	硏	形声 
\\	礼	レイ、ライ		あや, なり, のり, ひろし, れ	
\\	礼(れい): 
\\	礼儀(れいぎ): 
\\	失礼(しつれい): 
\\	無礼(ぶれい): 
\\	お礼(おれい): 
\\	失礼しました(しつれいしました): 
\\	失礼します(しつれいします): 
\\	示: 
\\	乙: 
\\	示	
\\	しめす・しめすへん	禮	形声 
\\	神	シン、ジン	かみ、かん-、こう-	か, かぐ, かな, かも, くま, こ, こは, だま, み	
\\	精神(せいしん): 
\\	神(かみ): 
\\	神経(しんけい): 
\\	神社(じんじゃ): 
\\	神話(しんわ): 
\\	神様(かみさま): 
\\	神秘(しんぴ): 
\\	神聖(しんせい): 
\\	神殿(しんでん): 
\\	示: 
\\	申: 
\\	示	
\\	しめす・しめすへん	神	形声 
\\	祭	サイ	まつ.る、まつ.り、まつり		
\\	祭(まつり): 
\\	祭る(まつる): 
\\	祭日(さいじつ): 
\\	雛祭(ひなまつり): 
\\	お祭り(おまつり): 
\\	月: 
\\	示: 
\\	示	
\\	しめす・しめすへん	
\\	会意 
\\	福	フク		とし, とみ, ふ, ふき, ふっ, ぼく, よし	
\\	福祉(ふくし): 
\\	幸福(こうふく): 
\\	福(ふく): 
\\	示: 
\\	畐: 
\\	示	
\\	しめす・しめすへん	福	形声 
\\	秒	ビョウ			
\\	秒(びょう): 
\\	禾: 
\\	少: 
\\	禾	
\\	のぎ・のぎへん	
\\	会意 
\\	究	キュウ、ク	きわ.める	きゅ	
\\	研究(けんきゅう): 
\\	究極(きゅうきょく): 
\\	研究室(けんきゅうしつ): 
\\	穴: 
\\	九: 
\\	穴	
\\	あな・あなかんむり	
\\	形声 
\\	章	ショウ		あき, あきら, あや, ふみ	
\\	章(しょう): 
\\	文章(ぶんしょう): 
\\	音: 
\\	早: 
\\	立	
\\	たつ・たつへん	
\\	象形 
\\	童	ドウ	わらべ	ぱ	
\\	児童(じどう): 
\\	童話(どうわ): 
\\	童謡(どうよう): 
\\	立: 
\\	里: 
\\	立	
\\	たつ・たつへん	
\\	形声 
\\	笛	テキ	ふえ	う	
\\	笛(ふえ): 
\\	竹: 
\\	田: 
\\	丨: 
\\	竹	
\\	たけ・たけかんむり	
\\	形声 
\\	第	ダイ、テイ			
\\	次第(しだい): 
\\	落第(らくだい): 
\\	第一(だいいち): 
\\	第(だい): 
\\	竹: 
\\	弔: 
\\	丿: 
\\	竹	
\\	たけ・たけかんむり	
\\	形声 
\\	筆	ヒツ	ふで	くし	
\\	執筆(しっぴつ): 
\\	筆(ふで): 
\\	筆者(ひっしゃ): 
\\	鉛筆(えんぴつ): 
\\	随筆(ずいひつ): 
\\	筆記(ひっき): 
\\	万年筆(まんねんひつ): 
\\	竹: 
\\	聿: 
\\	竹	
\\	たけ・たけかんむり	
\\	会意 
\\	等	トウ	ひと.しい、など、-ら	と, ひ	
\\	高等(こうとう): 
\\	平等(びょうどう): 
\\	対等(たいとう): 
\\	高等学校(こうとうがっこう): 
\\	同等(どうとう): 
\\	等々(とうとう): 
\\	等分(とうぶん): 
\\	等しい(ひとしい): 
\\	上等(じょうとう): 
\\	等(など): 
\\	等級(とうきゅう): 
\\	これ等(これら): 
\\	等(とう): 
\\	等(ら): 
\\	竹: 
\\	寺: 
\\	竹	
\\	たけ・たけかんむり	
\\	形声 
\\	箱	ソウ	はこ		
\\	箱(はこ): 
\\	竹: 
\\	相: 
\\	竹	
\\	たけ・たけかんむり	
\\	形声 
\\	級	キュウ		しな	
\\	高級(こうきゅう): 
\\	階級(かいきゅう): 
\\	上級(じょうきゅう): 
\\	初級(しょきゅう): 
\\	級(きゅう): 
\\	学級(がっきゅう): 
\\	同級(どうきゅう): 
\\	等級(とうきゅう): 
\\	糸: 
\\	及: 
\\	糸	
\\	いと・いとへん	
\\	形声 
\\	終	シュウ	お.わる、-お.わる、おわ.る、お.える、つい、つい.に	ばて	
\\	最終(さいしゅう): 
\\	終了(しゅうりょう): 
\\	終わり(おわり): 
\\	終える(おえる): 
\\	終始(しゅうし): 
\\	終わる(おわる): 
\\	終点(しゅうてん): 
\\	終日(しゅうじつ): 
\\	始終(しじゅう): 
\\	糸: 
\\	冬: 
\\	糸	
\\	いと・いとへん	
\\	形声 
\\	緑	リョク、ロク	みどり		
\\	緑(みどり): 
\\	糸: 
\\	彔: 
\\	糸	
\\	いと・いとへん	綠	形声 
\\	練	レン	ね.る、ね.り	ねり	
\\	練習(れんしゅう): 
\\	訓練(くんれん): 
\\	未練(みれん): 
\\	練る(ねる): 
\\	糸: 
\\	東: 
\\	糸	
\\	いと・いとへん	練	形声 
\\	羊	ヨウ	ひつじ	よ	
\\	羊毛(ようもう): 
\\	羊: 
\\	羊	
\\	ひつじ・ひつじへん	
\\	象形 
\\	美	ビ、ミ	うつく.しい	はる, よし, よしみ, り	
\\	美術(びじゅつ): 
\\	美術館(びじゅつかん): 
\\	美しい(うつくしい): 
\\	美(び): 
\\	美人(びじん): 
\\	美容(びよう): 
\\	優美(ゆうび): 
\\	賛美(さんび): 
\\	褒美(ほうび): 
\\	美味しい(おいしい): 
\\	華美(かび): 
\\	羊: 
\\	大: 
\\	羊	
\\	ひつじ・ひつじへん	
\\	象形 
\\	習	シュウ、ジュ	なら.う、なら.い		
\\	練習(れんしゅう): 
\\	学習(がくしゅう): 
\\	演習(えんしゅう): 
\\	習慣(しゅうかん): 
\\	実習(じっしゅう): 
\\	慣習(かんしゅう): 
\\	講習(こうしゅう): 
\\	教習(きょうしゅう): 
\\	風習(ふうしゅう): 
\\	予習(よしゅう): 
\\	自習(じしゅう): 
\\	復習(ふくしゅう): 
\\	習う(ならう): 
\\	見習う(みならう): 
\\	習字(しゅうじ): 
\\	羽: 
\\	白: 
\\	羽	
\\	はね	
\\	会意 
\\	者	シャ	もの		
\\	患者(かんじゃ): 
\\	記者(きしゃ): 
\\	業者(ぎょうしゃ): 
\\	学者(がくしゃ): 
\\	著者(ちょしゃ): 
\\	読者(どくしゃ): 
\\	医者(いしゃ): 
\\	作者(さくしゃ): 
\\	役者(やくしゃ): 
\\	後者(こうしゃ): 
\\	配偶者(はいぐうしゃ): 
\\	信者(しんじゃ): 
\\	筆者(ひっしゃ): 
\\	前者(ぜんしゃ): 
\\	悪者(わるもの): 
\\	歯医者(はいしゃ): 
\\	達者(たっしゃ): 
\\	者(もの): 
\\	者(しゃ): 
\\	老: 
\\	日: 
\\	老	
\\	おい・おいかんむり・おいがしら	者	会意 
\\	育	イク	そだ.つ、そだ.ち、そだ.てる、はぐく.む	やす	
\\	教育(きょういく): 
\\	体育(たいいく): 
\\	育児(いくじ): 
\\	飼育(しいく): 
\\	育ち(そだち): 
\\	育つ(そだつ): 
\\	発育(はついく): 
\\	保育(ほいく): 
\\	育成(いくせい): 
\\	生育(せいいく): 
\\	育てる(そだてる): 
\\	成育(せいいく): 
\\	亠: 
\\	厶: 
\\	月: 
\\	肉	
\\	にく・にくづき	
\\	会意 
\\	苦	ク	くる.しい、-ぐる.しい、くる.しむ、くる.しめる、にが.い、にが.る		
\\	苦労(くろう): 
\\	苦情(くじょう): 
\\	苦手(にがて): 
\\	苦い(にがい): 
\\	苦痛(くつう): 
\\	苦しめる(くるしめる): 
\\	苦しい(くるしい): 
\\	苦心(くしん): 
\\	ご苦労さま(ごくろうさま): 
\\	見苦しい(みぐるしい): 
\\	苦しむ(くるしむ): 
\\	無茶苦茶(むちゃくちゃ): 
\\	目茶苦茶(めちゃくちゃ): 
\\	苦(く): 
\\	艸: 
\\	古: 
\\	艸	
\\	くさ・くさかんむり	
\\	形声 
\\	荷	カ	に	はす, り	
\\	荷物(にもつ): 
\\	荷(に): 
\\	荷作り(にづくり): 
\\	艸: 
\\	何: 
\\	艸	
\\	くさ・くさかんむり	
\\	形声 
\\	落	ラク	お.ちる、お.ち、お.とす	おち	
\\	墜落(ついらく): 
\\	落ちる(おちる): 
\\	落ち込む(おちこむ): 
\\	落ち着き(おちつき): 
\\	落下(らっか): 
\\	落とす(おとす): 
\\	落ち葉(おちば): 
\\	転落(てんらく): 
\\	没落(ぼつらく): 
\\	落ち着く(おちつく): 
\\	落第(らくだい): 
\\	見落とす(みおとす): 
\\	落とし物(おとしもの): 
\\	洒落(しゃれ): 
\\	御洒落(おしゃれ): 
\\	洒落る(しゃれる): 
\\	洒落(しゃらく): 
\\	艸: 
\\	洛: 
\\	艸	
\\	くさ・くさかんむり	
\\	形声 
\\	葉	ヨウ	は	よ, わ	
\\	紅葉(もみじ): 
\\	言葉(ことば): 
\\	葉(は): 
\\	落ち葉(おちば): 
\\	言葉遣い(ことばづかい): 
\\	葉書(はがき): 
\\	艸: 
\\	世: 
\\	木: 
\\	艸	
\\	くさ・くさかんむり	
\\	形声 
\\	薬	ヤク	くすり	み	
\\	薬(くすり): 
\\	薬品(やくひん): 
\\	薬局(やっきょく): 
\\	農薬(のうやく): 
\\	薬指(くすりゆび): 
\\	薬缶(やかん): 
\\	薬(やく): 
\\	艸: 
\\	楽: 
\\	艸	
\\	くさ・くさかんむり	藥	形声 
\\	血	ケツ	ち		
\\	血(ち): 
\\	血液(けつえき): 
\\	出血(しゅっけつ): 
\\	血圧(けつあつ): 
\\	血管(けっかん): 
\\	混血(こんけつ): 
\\	輸血(ゆけつ): 
\\	皿: 
\\	血	
\\	ち・ちへん	
\\	会意 
\\	表	ヒョウ	おもて、-おもて、あらわ.す、あらわ.れる、あら.わす		
\\	"代表(だいひょう): 
\\	発表(はっぴょう): 
\\	表現(ひょうげん): 
\\	表情(ひょうじょう): 
\\	表面(ひょうめん): 
\\	公表(こうひょう): 
\\	表(おもて): 
\\	表紙(ひょうし): 
\\	図表(ずひょう): 
\\	表(ひょう): 
\\	時刻表(じこくひょう): 
\\	二: 
\\	丨: 
\\	衣: 
\\	衣	
\\	ころも・ころもへん	
\\	会意 
\\	詩	シ	うた		
\\	詩人(しじん): 
\\	詩(し): 
\\	言: 
\\	寺: 
\\	言	
\\	こと・げん・ごんべん・ことば	
\\	形声 
\\	調	チョウ	しら.べる、しら.べ、ととの.う、ととの.える	ぎ, つぎ	
\\	調べ(しらべ): 
\\	調査(ちょうさ): 
\\	調整(ちょうせい): 
\\	好調(こうちょう): 
\\	順調(じゅんちょう): 
\\	調印(ちょういん): 
\\	調子(ちょうし): 
\\	調停(ちょうてい): 
\\	調理(ちょうり): 
\\	調和(ちょうわ): 
\\	単調(たんちょう): 
\\	調節(ちょうせつ): 
\\	調味料(ちょうみりょう): 
\\	協調(きょうちょう): 
\\	不調(ふちょう): 
\\	同調(どうちょう): 
\\	強調(きょうちょう): 
\\	調べる(しらべる): 
\\	下調べ(したしらべ): 
\\	調度(ちょうど): 
\\	取り調べる(とりしらべる): 
\\	失調(しっちょう): 
\\	言: 
\\	周: 
\\	言	
\\	こと・げん・ごんべん・ことば	
\\	形声 
\\	談	ダン			
\\	会談(かいだん): 
\\	相談(そうだん): 
\\	冗談(じょうだん): 
\\	対談(たいだん): 
\\	座談会(ざだんかい): 
\\	雑談(ざつだん): 
\\	縁談(えんだん): 
\\	言: 
\\	炎: 
\\	言	
\\	こと・げん・ごんべん・ことば	
\\	形声 
\\	豆	トウ、ズ	まめ、まめ-	ど, ま	
\\	豆(まめ): 
\\	口: 
\\	豆	
\\	まめ・まめへん	
\\	象形 
\\	負	フ	ま.ける、ま.かす、お.う		
\\	負担(ふたん): 
\\	勝負(しょうぶ): 
\\	負け(まけ): 
\\	負債(ふさい): 
\\	負傷(ふしょう): 
\\	負う(おう): 
\\	負ける(まける): 
\\	背負う(せおう): 
\\	負かす(まかす): 
\\	負んぶ(おんぶ): 
\\	お負け(おまけ): 
\\	勹: 
\\	貝: 
\\	貝	
\\	かい・かいへん・こがい	
\\	会意 
\\	起	キ	お.きる、お.こる、お.こす、おこ.す、た.つ		
\\	起こす(おこす): 
\\	起源(きげん): 
\\	起こる(おこる): 
\\	起床(きしょう): 
\\	引き起す(ひきおこす): 
\\	起点(きてん): 
\\	起伏(きふく): 
\\	起きる(おきる): 
\\	走: 
\\	己: 
\\	走	
\\	はしる・そうにょう	
\\	形声 
\\	路	ロ、ル	みち, -じ		
\\	道路(どうろ): 
\\	進路(しんろ): 
\\	線路(せんろ): 
\\	回路(かいろ): 
\\	通路(つうろ): 
\\	経路(けいろ): 
\\	十字路(じゅうじろ): 
\\	海路(かいろ): 
\\	針路(しんろ): 
\\	足: 
\\	各: 
\\	足	
\\	あし・あしへん	
\\	形声 
\\	身	シン	み		
\\	身体(しんたい): 
\\	受身(うけみ): 
\\	自身(じしん): 
\\	出身(しゅっしん): 
\\	身(み): 
\\	中身(なかみ): 
\\	身長(しんちょう): 
\\	全身(ぜんしん): 
\\	独身(どくしん): 
\\	身分(みぶん): 
\\	心身(しんしん): 
\\	刺身(さしみ): 
\\	身の回り(みのまわり): 
\\	身近(みぢか): 
\\	身の上(みのうえ): 
\\	身なり(みなり): 
\\	身振り(みぶり): 
\\	生身(なまみ): 
\\	身: 
\\	身	
\\	み・みへん	
\\	象形 
\\	転	テン	ころ.がる、ころ.げる、ころ.がす、ころ.ぶ、まろ.ぶ、うたた、うつ.る、くる.めく		
\\	運転(うんてん): 
\\	移転(いてん): 
\\	運転手(うんてんしゅ): 
\\	自転車(じてんしゃ): 
\\	回転(かいてん): 
\\	転勤(てんきん): 
\\	転居(てんきょ): 
\\	転換(てんかん): 
\\	逆転(ぎゃくてん): 
\\	転じる(てんじる): 
\\	転落(てんらく): 
\\	転々(てんてん): 
\\	転がす(ころがす): 
\\	転がる(ころがる): 
\\	転ぶ(ころぶ): 
\\	転ずる(てんずる): 
\\	転校(てんこう): 
\\	自転(じてん): 
\\	転任(てんにん): 
\\	転回(てんかい): 
\\	転寝(うたたね): 
\\	車: 
\\	云: 
\\	車	
\\	くるま・くるまへん	轉	形声 
\\	軽	ケイ	かる.い、かろ.やか、かろ.んじる		
\\	手軽(てがる): 
\\	気軽(きがる): 
\\	軽快(けいかい): 
\\	軽率(けいそつ): 
\\	軽減(けいげん): 
\\	軽い(かるい): 
\\	軽蔑(けいべつ): 
\\	車: 
\\	圣: 
\\	車	
\\	くるま・くるまへん	輕	形声 
\\	農	ノウ		な, の, み	
\\	農業(のうぎょう): 
\\	農家(のうか): 
\\	農村(のうそん): 
\\	農民(のうみん): 
\\	農地(のうち): 
\\	農場(のうじょう): 
\\	農耕(のうこう): 
\\	農薬(のうやく): 
\\	農産物(のうさんぶつ): 
\\	酪農(らくのう): 
\\	曲: 
\\	辰: 
\\	辰	
\\	しんのたつ	
\\	会意 
\\	返	ヘン	かえ.す、-かえ.す、かえ.る、-かえ.る		
\\	返還(へんかん): 
\\	返済(へんさい): 
\\	返事(へんじ): 
\\	繰り返す(くりかえす): 
\\	振り返る(ふりかえる): 
\\	裏返し(うらがえし): 
\\	返答(へんとう): 
\\	引返す(ひきかえす): 
\\	折り返す(おりかえす): 
\\	裏返す(うらがえす): 
\\	引っ繰り返る(ひっくりかえる): 
\\	宙返り(ちゅうがえり): 
\\	照り返す(てりかえす): 
\\	ひっくり返す(ひっくりかえす): 
\\	反: 
\\	辶: 
\\	辵	
\\	しんにょう・しんにゅう	
\\	形声 
\\	追	ツイ	お.う	おい	
\\	追加(ついか): 
\\	追い込む(おいこむ): 
\\	追う(おう): 
\\	追及(ついきゅう): 
\\	追放(ついほう): 
\\	追跡(ついせき): 
\\	追い付く(おいつく): 
\\	追い出す(おいだす): 
\\	追いかける(おいかける): 
\\	追い越す(おいこす): 
\\	𠂤: 
\\	辶: 
\\	辵	
\\	しんにょう・しんにゅう	
\\	会意 
\\	送	ソウ	おく.る		
\\	放送(ほうそう): 
\\	輸送(ゆそう): 
\\	送る(おくる): 
\\	見送り(みおくり): 
\\	見送る(みおくる): 
\\	運送(うんそう): 
\\	送金(そうきん): 
\\	郵送(ゆうそう): 
\\	送料(そうりょう): 
\\	送別(そうべつ): 
\\	送り仮名(おくりがな): 
\\	回送(かいそう): 
\\	天: 
\\	辶: 
\\	辵	
\\	しんにょう・しんにゅう	
\\	形声 
\\	速	ソク	はや.い、はや-、はや.める、すみ.やか	わ	
\\	速める(はやめる): 
\\	急速(きゅうそく): 
\\	高速(こうそく): 
\\	時速(じそく): 
\\	速度(そくど): 
\\	速やか(すみやか): 
\\	加速(かそく): 
\\	迅速(じんそく): 
\\	早速(さっそく): 
\\	速達(そくたつ): 
\\	加速度(かそくど): 
\\	速力(そくりょく): 
\\	束: 
\\	辶: 
\\	辵	
\\	しんにょう・しんにゅう	
\\	形声 
\\	進	シン	すす.む、すす.める	のぶ	
\\	推進(すいしん): 
\\	進出(しんしゅつ): 
\\	促進(そくしん): 
\\	進学(しんがく): 
\\	進路(しんろ): 
\\	前進(ぜんしん): 
\\	昇進(しょうしん): 
\\	進行(しんこう): 
\\	進歩(しんぽ): 
\\	行進(こうしん): 
\\	進化(しんか): 
\\	進む(すすむ): 
\\	進展(しんてん): 
\\	増進(ぞうしん): 
\\	進み(すすみ): 
\\	進める(すすめる): 
\\	進呈(しんてい): 
\\	進度(しんど): 
\\	隹: 
\\	辶: 
\\	辵	
\\	しんにょう・しんにゅう	
\\	形声 
\\	遊	ユウ、ユ	あそ.ぶ、あそ.ばす	あす, う	
\\	遊び(あそび): 
\\	遊園地(ゆうえんち): 
\\	遊ぶ(あそぶ): 
\\	遊牧(ゆうぼく): 
\\	斿: 
\\	辶: 
\\	辵	
\\	しんにょう・しんにゅう	
\\	形声 
\\	運	ウン	はこ.ぶ		
\\	運動(うんどう): 
\\	運営(うんえい): 
\\	運転(うんてん): 
\\	運輸(うんゆ): 
\\	運用(うんよう): 
\\	運転手(うんてんしゅ): 
\\	運ぶ(はこぶ): 
\\	運賃(うんちん): 
\\	運命(うんめい): 
\\	運送(うんそう): 
\\	運(うん): 
\\	運搬(うんぱん): 
\\	運河(うんが): 
\\	幸運(こううん): 
\\	不運(ふうん): 
\\	海運(かいうん): 
\\	軍: 
\\	辶: 
\\	辵	
\\	しんにょう・しんにゅう	
\\	形声 
\\	部	ブ	へ、-べ	とり, ふ, ぺ, ま	
\\	一部(いちぶ): 
\\	幹部(かんぶ): 
\\	部長(ぶちょう): 
\\	部分(ぶぶん): 
\\	部門(ぶもん): 
\\	本部(ほんぶ): 
\\	学部(がくぶ): 
\\	内部(ないぶ): 
\\	部屋(へや): 
\\	全部(ぜんぶ): 
\\	部品(ぶひん): 
\\	外部(がいぶ): 
\\	大部分(だいぶぶん): 
\\	部下(ぶか): 
\\	一部分(いちぶぶん): 
\\	警部(けいぶ): 
\\	部(ぶ): 
\\	大部(たいぶ): 
\\	部首(ぶしゅ): 
\\	立: 
\\	口: 
\\	邑: 
\\	邑	
\\	むら・おおざと・おおざとづくり・ゆう	
\\	形声 
\\	都	ト、ツ	みやこ	くに, ず, ち, づめ, みや	
\\	都市(とし): 
\\	首都(しゅと): 
\\	都(みやこ): 
\\	都心(としん): 
\\	都会(とかい): 
\\	都合(つごう): 
\\	都(と): 
\\	者: 
\\	邑: 
\\	邑	
\\	むら・おおざと・おおざとづくり・ゆう	都	形声 
\\	配	ハイ	くば.る		
\\	支配(しはい): 
\\	心配(しんぱい): 
\\	配分(はいぶん): 
\\	配慮(はいりょ): 
\\	気配(けはい): 
\\	配置(はいち): 
\\	配偶者(はいぐうしゃ): 
\\	分配(ぶんぱい): 
\\	配達(はいたつ): 
\\	配布(はいふ): 
\\	配る(くばる): 
\\	手配(てはい): 
\\	配列(はいれつ): 
\\	配給(はいきゅう): 
\\	酉: 
\\	己: 
\\	酉	
\\	とり・とりへん・ひよみのとり・さけのとり	
\\	会意 
\\	酒	シュ	さけ、さか-	き, さ, し	
\\	酒(さけ): 
\\	酒場(さかば): 
\\	麦酒(ビール): 
\\	お酒(おさけ): 
\\	水: 
\\	酉: 
\\	酉	
\\	とり・とりへん・ひよみのとり・さけのとり	
\\	形声 
\\	重	ジュウ、チョウ	え、おも.い、おも.り、おも.なう、かさ.ねる、かさ.なる、おも	さね, しげ, しげる	
\\	重態(じゅうたい): 
\\	重複(ちょうふく): 
\\	重要(じゅうよう): 
\\	重大(じゅうだい): 
\\	貴重(きちょう): 
\\	重視(じゅうし): 
\\	重点(じゅうてん): 
\\	慎重(しんちょう): 
\\	体重(たいじゅう): 
\\	重量(じゅうりょう): 
\\	尊重(そんちょう): 
\\	一重(ひとえ): 
\\	厳重(げんじゅう): 
\\	重ねる(かさねる): 
\\	重力(じゅうりょく): 
\\	重役(じゅうやく): 
\\	比重(ひじゅう): 
\\	重い(おもい): 
\\	重んじる(おもんじる): 
\\	重宝(ちょうほう): 
\\	重なる(かさなる): 
\\	重たい(おもたい): 
\\	重んずる(おもんずる): 
\\	重(え): 
\\	重(じゅう): 
\\	千: 
\\	里: 
\\	里	
\\	さと・さとへん	
\\	会意 
\\	鉄	テツ	くろがね	けん, てっ	
\\	鉄道(てつどう): 
\\	鉄(てつ): 
\\	地下鉄(ちかてつ): 
\\	私鉄(してつ): 
\\	製鉄(せいてつ): 
\\	鉄棒(てつぼう): 
\\	鉄砲(てっぽう): 
\\	鉄鋼(てっこう): 
\\	鉄橋(てっきょう): 
\\	金: 
\\	失: 
\\	金	
\\	かね・かねへん	鐵	形声 
\\	銀	ギン	しろがね	うん, かな, かね	
\\	銀行(ぎんこう): 
\\	銀(ぎん): 
\\	金: 
\\	艮: 
\\	金	
\\	かね・かねへん	
\\	形声 
\\	開	カイ	ひら.く、ひら.き、-びら.き、ひら.ける、あ.く、あ.ける	はる, か, ひらき	
\\	開発(かいはつ): 
\\	開催(かいさい): 
\\	開始(かいし): 
\\	開放(かいほう): 
\\	公開(こうかい): 
\\	展開(てんかい): 
\\	開会(かいかい): 
\\	開拓(かいたく): 
\\	開ける(あける): 
\\	開く(あく): 
\\	打開(だかい): 
\\	開通(かいつう): 
\\	未開(みかい): 
\\	開く(ひらく): 
\\	切開(せっかい): 
\\	門: 
\\	幵: 
\\	門	
\\	もん・もんがまえ・かどがまえ	
\\	会意 
\\	院	イン			
\\	病院(びょういん): 
\\	大学院(だいがくいん): 
\\	入院(にゅういん): 
\\	参議院(さんぎいん): 
\\	寺院(じいん): 
\\	衆議院(しゅうぎいん): 
\\	退院(たいいん): 
\\	医院(いいん): 
\\	阜: 
\\	完: 
\\	阜	
\\	こざと・こざとへん・おか・ぎふのふ	
\\	形声 
\\	陽	ヨウ	ひ	あき, あきら, あけ, はる, ひろ, やん, よ	
\\	太陽(たいよう): 
\\	陽気(ようき): 
\\	阜: 
\\	昜: 
\\	阜	
\\	こざと・こざとへん・おか・ぎふのふ	
\\	形声 
\\	階	カイ	きざはし	しな, と, はし	
\\	段階(だんかい): 
\\	階級(かいきゅう): 
\\	階段(かいだん): 
\\	階(かい): 
\\	二階建て(にかいだて): 
\\	階層(かいそう): 
\\	阜: 
\\	皆: 
\\	阜	
\\	こざと・こざとへん・おか・ぎふのふ	
\\	形声 
\\	集	シュウ	あつ.まる、あつ.める、つど.う	あつまり, ず	
\\	集会(しゅうかい): 
\\	集団(しゅうだん): 
\\	編集(へんしゅう): 
\\	集中(しゅうちゅう): 
\\	募集(ぼしゅう): 
\\	収集(しゅうしゅう): 
\\	特集(とくしゅう): 
\\	集める(あつめる): 
\\	集合(しゅうごう): 
\\	集まり(あつまり): 
\\	全集(ぜんしゅう): 
\\	密集(みっしゅう): 
\\	集まる(あつまる): 
\\	採集(さいしゅう): 
\\	集計(しゅうけい): 
\\	集(しゅう): 
\\	集金(しゅうきん): 
\\	群集(ぐんしゅう): 
\\	隹: 
\\	木: 
\\	隹	
\\	ふるとり	
\\	会意 
\\	面	メン、ベン	おも、おもて、つら	お, ずら, ほおつき, も	
\\	面目(めんもく): 
\\	場面(ばめん): 
\\	正面(しょうめん): 
\\	表面(ひょうめん): 
\\	面積(めんせき): 
\\	側面(そくめん): 
\\	一面(いちめん): 
\\	水面(すいめん): 
\\	方面(ほうめん): 
\\	面接(めんせつ): 
\\	面倒(めんどう): 
\\	斜面(しゃめん): 
\\	面会(めんかい): 
\\	地面(じめん): 
\\	面白い(おもしろい): 
\\	覆面(ふくめん): 
\\	面倒くさい(めんどうくさい): 
\\	直面(ちょくめん): 
\\	対面(たいめん): 
\\	洗面(せんめん): 
\\	断面(だんめん): 
\\	面する(めんする): 
\\	真面目(まじめ): 
\\	面皰(にきび): 
\\	几帳面(きちょうめん): 
\\	面(おもて): 
\\	面(めん): 
\\	生真面目(きまじめ): 
\\	面: 
\\	面	
\\	めん	
\\	象形 
\\	題	ダイ			
\\	課題(かだい): 
\\	問題(もんだい): 
\\	話題(わだい): 
\\	議題(ぎだい): 
\\	主題(しゅだい): 
\\	宿題(しゅくだい): 
\\	題名(だいめい): 
\\	題(だい): 
\\	出題(しゅつだい): 
\\	題する(だいする): 
\\	是: 
\\	頁: 
\\	頁	
\\	おおがい・いちのかい	
\\	形声 
\\	飲	イン、オン	の.む、-の.み		
\\	飲物(のみもの): 
\\	飲む(のむ): 
\\	飲み込む(のみこむ): 
\\	湯飲み(ゆのみ): 
\\	食: 
\\	欠: 
\\	食	
\\	しょく・しょくへん	
\\	会意 
\\	館	カン	やかた、たて	たち	
\\	大使館(たいしかん): 
\\	美術館(びじゅつかん): 
\\	博物館(はくぶつかん): 
\\	図書館(としょかん): 
\\	映画館(えいがかん): 
\\	旅館(りょかん): 
\\	会館(かいかん): 
\\	本館(ほんかん): 
\\	館(やかた): 
\\	館(かん): 
\\	食: 
\\	官: 
\\	食	
\\	しょく・しょくへん	
\\	形声 
\\	駅	エキ			
\\	駅(えき): 
\\	馬: 
\\	尺: 
\\	馬	
\\	うま・うまへん	驛	形声 
\\	鼻	ビ	はな		
\\	鼻(はな): 
\\	耳鼻科(じびか): 
\\	自: 
\\	畀: 
\\	鼻	
\\	はな・はなへん	
\\	形声 
\\	不	フ、ブ			
\\	不況(ふきょう): 
\\	不安(ふあん): 
\\	不足(ふそく): 
\\	不動産(ふどうさん): 
\\	不満(ふまん): 
\\	不明(ふめい): 
\\	不振(ふしん): 
\\	不正(ふせい): 
\\	不当(ふとう): 
\\	不在(ふざい): 
\\	不良(ふりょう): 
\\	不可欠(ふかけつ): 
\\	不自由(ふじゆう): 
\\	不幸(ふこう): 
\\	不利(ふり): 
\\	不審(ふしん): 
\\	不服(ふふく): 
\\	不景気(ふけいき): 
\\	不思議(ふしぎ): 
\\	不可(ふか): 
\\	不規則(ふきそく): 
\\	不通(ふつう): 
\\	不運(ふうん): 
\\	不順(ふじゅん): 
\\	不便(ふべん): 
\\	不調(ふちょう): 
\\	不意(ふい): 
\\	不平(ふへい): 
\\	不潔(ふけつ): 
\\	不吉(ふきつ): 
\\	不評(ふひょう): 
\\	不味い(まずい): 
\\	不図(ふと): 
\\	不便(ふびん): 
\\	不(ふ): 
\\	一: 
\\	丿: 
\\	丨: 
\\	一	
\\	いち	
\\	仮借 
\\	争	ソウ	あらそ.う、いか.でか		
\\	戦争(せんそう): 
\\	競争(きょうそう): 
\\	紛争(ふんそう): 
\\	争い(あらそい): 
\\	論争(ろんそう): 
\\	争う(あらそう): 
\\	抗争(こうそう): 
\\	勹: 
\\	彑: 
\\	亅: 
\\	爪	
\\	つめ・つめがしら・つめかんむり・そうにょう・のつ	爭	会意 
\\	付	フ	つ.ける、-つ.ける、-づ.ける、つ.け、つ.け-、-つ.け、-づ.け、-づけ、つ.く、-づ.く、つ.き、-つ.き、-つき、-づ.き、-づき	つけ	
\\	日付け(ひづけ): 
\\	見付かる(みつかる): 
\\	付近(ふきん): 
\\	交付(こうふ): 
\\	付き合い(つきあい): 
\\	受付(うけつけ): 
\\	追い付く(おいつく): 
\\	見付ける(みつける): 
\\	付録(ふろく): 
\\	付き(つき): 
\\	名付ける(なづける): 
\\	気付く(きづく): 
\\	顔付き(かおつき): 
\\	傷付ける(きずつける): 
\\	受け付ける(うけつける): 
\\	付ける(つける): 
\\	片付ける(かたづける): 
\\	思い付く(おもいつく): 
\\	付け加える(つけくわえる): 
\\	片付く(かたづく): 
\\	目付き(めつき): 
\\	付き合う(つきあう): 
\\	くっ付く(くっつく): 
\\	くっ付ける(くっつける): 
\\	打付ける(ぶつける): 
\\	体付き(からだつき): 
\\	備え付ける(そなえつける): 
\\	気を付ける(きをつける): 
\\	言付ける(ことづける): 
\\	やっ付ける(やっつける): 
\\	片付け(かたづけ): 
\\	据え付ける(すえつける): 
\\	間誤付く(まごつく): 
\\	言い付ける(いいつける): 
\\	取り付ける(とりつける): 
\\	付く(つく): 
\\	言付け(ことづけ): 
\\	付(づけ): 
\\	人: 
\\	寸: 
\\	人	
\\	ひと・にんべん・ひとやね	
\\	会意 
\\	令	レイ		のり, りょう, れ	
\\	命令(めいれい): 
\\	指令(しれい): 
\\	仮令(たとえ): 
\\	人: 
\\	一: 
\\	マ: 
\\	人	
\\	ひと・にんべん・ひとやね	
\\	象形 
\\	以	イ	もっ.て	もち	
\\	以来(いらい): 
\\	以内(いない): 
\\	以前(いぜん): 
\\	以下(いか): 
\\	以後(いご): 
\\	以上(いじょう): 
\\	以降(いこう): 
\\	以外(いがい): 
\\	以て(もって): 
\\	丶: 
\\	人: 
\\	人	
\\	ひと・にんべん・ひとやね	
\\	象形 
\\	仲	チュウ	なか	ちゅん, つづき, なかつ	
\\	仲間(なかま): 
\\	仲(なか): 
\\	仲人(なこうど): 
\\	仲良し(なかよし): 
\\	仲直り(なかなおり): 
\\	人: 
\\	中: 
\\	人	
\\	ひと・にんべん・ひとやね	
\\	形声 
\\	伝	デン、テン	つた.わる、つた.える、つた.う、つだ.う、-づた.い、つて	つたえ	
\\	伝統(でんとう): 
\\	宣伝(せんでん): 
\\	伝説(でんせつ): 
\\	手伝い(てつだい): 
\\	伝達(でんたつ): 
\\	伝言(でんごん): 
\\	伝わる(つたわる): 
\\	伝記(でんき): 
\\	伝染(でんせん): 
\\	手伝う(てつだう): 
\\	お手伝いさん(おてつだいさん): 
\\	伝える(つたえる): 
\\	伝来(でんらい): 
\\	言伝(ことづて): 
\\	伝言(つてこと): 
\\	人: 
\\	云: 
\\	人	
\\	ひと・にんべん・ひとやね	傳	会意 
\\	位	イ	くらい、ぐらい	ぞき	
\\	単位(たんい): 
\\	地位(ちい): 
\\	位置(いち): 
\\	上位(じょうい): 
\\	優位(ゆうい): 
\\	位(くらい): 
\\	位(い): 
\\	人: 
\\	立: 
\\	人	
\\	ひと・にんべん・ひとやね	
\\	会意 
\\	低	テイ	ひく.い、ひく.める、ひく.まる		
\\	最低(さいてい): 
\\	低下(ていか): 
\\	低い(ひくい): 
\\	人: 
\\	氏: 
\\	一: 
\\	人	
\\	ひと・にんべん・ひとやね	
\\	形声 
\\	例	レイ	たと.える		
\\	比例(ひれい): 
\\	例えば(たとえば): 
\\	例(れい): 
\\	例外(れいがい): 
\\	前例(ぜんれい): 
\\	実例(じつれい): 
\\	慣例(かんれい): 
\\	例え(たとえ): 
\\	例える(たとえる): 
\\	人: 
\\	列: 
\\	人	
\\	ひと・にんべん・ひとやね	
\\	形声 
\\	便	ベン、ビン	たよ.り		
\\	郵便(ゆうびん): 
\\	便利(べんり): 
\\	郵便局(ゆうびんきょく): 
\\	便宜(べんぎ): 
\\	便り(たより): 
\\	不便(ふべん): 
\\	便(びん): 
\\	便所(べんじょ): 
\\	船便(ふなびん): 
\\	小便(しょうべん): 
\\	便箋(びんせん): 
\\	大便(だいべん): 
\\	不便(ふびん): 
\\	人: 
\\	更: 
\\	人	
\\	ひと・にんべん・ひとやね	
\\	会意 
\\	信	シン		し, しが, しな, しの, しぶ, とき, のび, のぶ, まこと	
\\	通信(つうしん): 
\\	自信(じしん): 
\\	信用(しんよう): 
\\	信頼(しんらい): 
\\	信号(しんごう): 
\\	信仰(しんこう): 
\\	信者(しんじゃ): 
\\	確信(かくしん): 
\\	信任(しんにん): 
\\	信ずる(しんずる): 
\\	迷信(めいしん): 
\\	信じる(しんじる): 
\\	人: 
\\	言: 
\\	人	
\\	ひと・にんべん・ひとやね	
\\	会意 
\\	倉	ソウ	くら		
\\	倉庫(そうこ): 
\\	人: 
\\	口: 
\\	人	
\\	ひと・にんべん・ひとやね	
\\	象形 
\\	候	コウ	そうろう		
\\	候補(こうほ): 
\\	気候(きこう): 
\\	天候(てんこう): 
\\	人: 
\\	丨: 
\\	矢: 
\\	人	
\\	ひと・にんべん・ひとやね	
\\	形声 
\\	借	シャク	か.りる		
\\	借金(しゃっきん): 
\\	借りる(かりる): 
\\	借り(かり): 
\\	拝借(はいしゃく): 
\\	人: 
\\	昔: 
\\	人	
\\	ひと・にんべん・ひとやね	
\\	形声 
\\	停	テイ	と.める、と.まる		
\\	停止(ていし): 
\\	調停(ちょうてい): 
\\	停滞(ていたい): 
\\	停電(ていでん): 
\\	停車(ていしゃ): 
\\	停留所(ていりゅうじょ): 
\\	人: 
\\	亭: 
\\	人	
\\	ひと・にんべん・ひとやね	
\\	形声 
\\	健	ケン	すこ.やか	かつ, たけ, たけし, たて, とし, やす, やすし	
\\	健康(けんこう): 
\\	保健(ほけん): 
\\	健全(けんぜん): 
\\	健在(けんざい): 
\\	健やか(すこやか): 
\\	人: 
\\	建: 
\\	人	
\\	ひと・にんべん・ひとやね	
\\	形声 
\\	側	ソク	かわ、がわ、そば		
\\	側(そば): 
\\	側面(そくめん): 
\\	両側(りょうがわ): 
\\	縁側(えんがわ): 
\\	側(がわ): 
\\	人: 
\\	則: 
\\	人	
\\	ひと・にんべん・ひとやね	
\\	形声 
\\	働	ドウ、リュク、リキ、ロク、リョク	はたら.く		
\\	労働(ろうどう): 
\\	働き(はたらき): 
\\	共働き(ともばたらき): 
\\	働く(はたらく): 
\\	人: 
\\	動: 
\\	人	
\\	ひと・にんべん・ひとやね	
\\	形声 
\\	億	オク		お	
\\	億(おく): 
\\	人: 
\\	意: 
\\	人	
\\	ひと・にんべん・ひとやね	
\\	形声 
\\	兆	チョウ	きざ.す、きざ.し		
\\	兆し(きざし): 
\\	兆(ちょう): 
\\	八: 
\\	儿	
\\	にんにょう・ひとあし	
\\	象形 
\\	児	ジ、ニ、ゲイ	こ、-こ、-っこ	る	
\\	児童(じどう): 
\\	育児(いくじ): 
\\	幼児(ようじ): 
\\	小児科(しょうにか): 
\\	孤児(こじ): 
\\	児(じ): 
\\	旧: 
\\	八: 
\\	儿	
\\	にんにょう・ひとあし	兒	象形 
\\	共	キョウ	とも、とも.に、-ども		
\\	共同(きょうどう): 
\\	共通(きょうつう): 
\\	公共(こうきょう): 
\\	共に(ともに): 
\\	共存(きょうぞん): 
\\	共感(きょうかん): 
\\	共働き(ともばたらき): 
\\	共和(きょうわ): 
\\	共産(きょうさん): 
\\	共学(きょうがく): 
\\	共鳴(きょうめい): 
\\	共稼ぎ(ともかせぎ): 
\\	其れ共(それとも): 
\\	八: 
\\	八	
\\	はち・は・はちがしら	
\\	会意 
\\	兵	ヘイ、ヒョウ	つわもの	へ	
\\	兵器(へいき): 
\\	兵士(へいし): 
\\	兵隊(へいたい): 
\\	丘: 
\\	八: 
\\	八	
\\	はち・は・はちがしら	
\\	会意 
\\	典	テン、デン		すけ, つね, の, のり	
\\	百科辞典(ひゃっかじてん): 
\\	古典(こてん): 
\\	典型(てんけい): 
\\	辞典(じてん): 
\\	原典(げんてん): 
\\	曲: 
\\	八: 
\\	八	
\\	はち・は・はちがしら	
\\	会意 
\\	冷	レイ	つめ.たい、ひ.える、ひ.や、ひ.ややか、ひ.やす、ひ.やかす、さ.める、さ.ます		
\\	冷静(れいせい): 
\\	冷蔵庫(れいぞうこ): 
\\	冷凍(れいとう): 
\\	冷房(れいぼう): 
\\	冷淡(れいたん): 
\\	冷える(ひえる): 
\\	冷やす(ひやす): 
\\	冷酷(れいこく): 
\\	冷たい(つめたい): 
\\	冷める(さめる): 
\\	冷蔵(れいぞう): 
\\	冷やかす(ひやかす): 
\\	冷ます(さます): 
\\	冫: 
\\	令: 
\\	冫	
\\	にすい	
\\	形声 
\\	初	ショ	はじ.め、はじ.めて、はつ、はつ-、うい-、-そ.める、-ぞ.め	し, はっ	
\\	初め(はじめ): 
\\	最初(さいしょ): 
\\	初(はつ): 
\\	初歩(しょほ): 
\\	初級(しょきゅう): 
\\	初旬(しょじゅん): 
\\	初版(しょはん): 
\\	初めて(はじめて): 
\\	初耳(はつみみ): 
\\	初めまして(はじめまして): 
\\	衣: 
\\	刀: 
\\	刀	
\\	かたな・りっとう	
\\	会意 
\\	別	ベツ	わか.れる、わ.ける	べっ	
\\	特別(とくべつ): 
\\	差別(さべつ): 
\\	別に(べつに): 
\\	別れ(わかれ): 
\\	別荘(べっそう): 
\\	性別(せいべつ): 
\\	区別(くべつ): 
\\	別々(べつべつ): 
\\	格別(かくべつ): 
\\	別れる(わかれる): 
\\	個別(こべつ): 
\\	送別(そうべつ): 
\\	別(べつ): 
\\	一別(いちべつ): 
\\	口: 
\\	勹: 
\\	刀: 
\\	刀	
\\	かたな・りっとう	
\\	会意 
\\	利	リ	き.く	かが, と, とし, のり, み, りい	
\\	利き目(ききめ): 
\\	権利(けんり): 
\\	勝利(しょうり): 
\\	利益(りえき): 
\\	利用(りよう): 
\\	有利(ゆうり): 
\\	便利(べんり): 
\\	不利(ふり): 
\\	利害(りがい): 
\\	利子(りし): 
\\	利息(りそく): 
\\	利点(りてん): 
\\	砂利(じゃり): 
\\	左利き(ひだりきき): 
\\	利潤(りじゅん): 
\\	利口(りこう): 
\\	利根(りこん): 
\\	亜米利加(アメリカ): 
\\	阿弗利加(アフリカ): 
\\	禾: 
\\	刀: 
\\	刀	
\\	かたな・りっとう	
\\	会意 
\\	刷	サツ	す.る、-ず.り、-ずり、は.く		
\\	印刷(いんさつ): 
\\	刷る(する): 
\\	刷り(すり): 
\\	尸: 
\\	巾: 
\\	刀: 
\\	刀	
\\	かたな・りっとう	
\\	会意 
\\	副	フク		そい, そえ	
\\	副(ふく): 
\\	副詞(ふくし): 
\\	畐: 
\\	刀: 
\\	刀	
\\	かたな・りっとう	
\\	形声 
\\	功	コウ、ク	いさお	いさ, かつ, くぬ, ぐう, こと, つとむ, とし, のり, よし	
\\	成功(せいこう): 
\\	功績(こうせき): 
\\	工: 
\\	力: 
\\	力	
\\	ちから・りきづくり	
\\	形声 
\\	加	カ	くわ.える、くわ.わる		
\\	参加(さんか): 
\\	増加(ぞうか): 
\\	追加(ついか): 
\\	加入(かにゅう): 
\\	加工(かこう): 
\\	加速(かそく): 
\\	加減(かげん): 
\\	加わる(くわわる): 
\\	加える(くわえる): 
\\	加味(かみ): 
\\	加熱(かねつ): 
\\	付け加える(つけくわえる): 
\\	いい加減(いいかげん): 
\\	加速度(かそくど): 
\\	亜米利加(アメリカ): 
\\	阿弗利加(アフリカ): 
\\	力: 
\\	口: 
\\	力	
\\	ちから・りきづくり	
\\	会意 
\\	努	ド	つと.める		
\\	努力(どりょく): 
\\	努める(つとめる): 
\\	努めて(つとめて): 
\\	奴: 
\\	力: 
\\	力	
\\	ちから・りきづくり	
\\	形声 
\\	労	ロウ	ろう.する、いたわ.る、いた.ずき、ねぎら、つか.れる、ねぎら.う		
\\	労働(ろうどう): 
\\	苦労(くろう): 
\\	疲労(ひろう): 
\\	過労(かろう): 
\\	勤労(きんろう): 
\\	労力(ろうりょく): 
\\	ご苦労さま(ごくろうさま): 
\\	労る(いたわる): 
\\	小: 
\\	冖: 
\\	力: 
\\	力	
\\	ちから・りきづくり	勞	会意 
\\	勇	ユウ	いさ.む	お, はや	
\\	勇気(ゆうき): 
\\	勇敢(ゆうかん): 
\\	勇ましい(いさましい): 
\\	マ: 
\\	男: 
\\	力	
\\	ちから・りきづくり	
\\	形声 
\\	包	ホウ	つつ.む、くる.む	お, かね	
\\	包む(つつむ): 
\\	小包み(こづつみ): 
\\	包丁(ほうちょう): 
\\	包装(ほうそう): 
\\	包み(つつみ): 
\\	包帯(ほうたい): 
\\	勹: 
\\	己: 
\\	勹	
\\	つつみがまえ・くがまえ	
\\	象形 
\\	卒	ソツ、シュツ	そっ.する、お.える、お.わる、ついに、にわか		
\\	卒業(そつぎょう): 
\\	何卒(なにとぞ): 
\\	亠: 
\\	从: 
\\	十: 
\\	十	
\\	じゅう・じゅうへん	
\\	象形 
\\	協	キョウ			
\\	協議(きょうぎ): 
\\	協力(きょうりょく): 
\\	妥協(だきょう): 
\\	協会(きょうかい): 
\\	協定(きょうてい): 
\\	協調(きょうちょう): 
\\	十: 
\\	力: 
\\	力: 
\\	力: 
\\	十	
\\	じゅう・じゅうへん	
\\	形声 
\\	単	タン	ひとえ		
\\	簡単(かんたん): 
\\	単位(たんい): 
\\	単純(たんじゅん): 
\\	単なる(たんなる): 
\\	単に(たんに): 
\\	単一(たんいつ): 
\\	単語(たんご): 
\\	単調(たんちょう): 
\\	単独(たんどく): 
\\	単数(たんすう): 
\\	小: 
\\	甲: 
\\	口	
\\	くち・くちへん	單	象形 
\\	博	ハク、バク		ぐれ, と, はか, ひろ	
\\	博士(はくし): 
\\	博物館(はくぶつかん): 
\\	十: 
\\	尃: 
\\	十	
\\	じゅう・じゅうへん	
\\	形声 
\\	印	イン	しるし、-じるし、しる.す		
\\	印象(いんしょう): 
\\	調印(ちょういん): 
\\	印刷(いんさつ): 
\\	印(しるし): 
\\	印鑑(いんかん): 
\\	目印(めじるし): 
\\	矢印(やじるし): 
\\	印(いん): 
\\	丿: 
\\	丨: 
\\	卩: 
\\	卩	
\\	ふし・ふしづくり・わりふ	
\\	会意 
\\	参	サン、シン	まい.る、まい-、まじわる、みつ	み	
\\	参加(さんか): 
\\	参考(さんこう): 
\\	参議院(さんぎいん): 
\\	参照(さんしょう): 
\\	持参(じさん): 
\\	参る(まいる): 
\\	お参り(おまいり): 
\\	参上(さんじょう): 
\\	厶: 
\\	大: 
\\	彡: 
\\	厶	
\\	む	參	会意 
\\	史	シ		あきら, あや, お, こ, ちか, とし, なか, のぶ, ひさ, ひと, ふみ	
\\	歴史(れきし): 
\\	女史(じょし): 
\\	史(し): 
\\	口: 
\\	乂: 
\\	口	
\\	くち・くちへん	
\\	会意 
\\	司	シ	つかさど.る	つ, つか, つかさ	
\\	司法(しほう): 
\\	上司(じょうし): 
\\	司会(しかい): 
\\	司る(つかさどる): 
\\	一: 
\\	口: 
\\	口	
\\	くち・くちへん	
\\	会意 
\\	各	カク	おのおの	かか, かが	
\\	各各(それぞれ): 
\\	各自(かくじ): 
\\	各(かく): 
\\	各種(かくしゅ): 
\\	各地(かくち): 
\\	夂: 
\\	口: 
\\	口	
\\	くち・くちへん	
\\	会意 
\\	告	コク	つ.げる	い	
\\	報告(ほうこく): 
\\	広告(こうこく): 
\\	申告(しんこく): 
\\	警告(けいこく): 
\\	告白(こくはく): 
\\	告げる(つげる): 
\\	勧告(かんこく): 
\\	忠告(ちゅうこく): 
\\	布告(ふこく): 
\\	牛: 
\\	口: 
\\	口	
\\	くち・くちへん	
\\	象形 
\\	周	シュウ	まわ.り	あまね, しゅ, す, ちか, のり, ひろ, びび, まこと	
\\	周辺(しゅうへん): 
\\	周囲(しゅうい): 
\\	周期(しゅうき): 
\\	円周(えんしゅう): 
\\	周(しゅう): 
\\	冂: 
\\	吉: 
\\	口	
\\	くち・くちへん	
\\	会意 
\\	唱	ショウ	とな.える		
\\	合唱(がっしょう): 
\\	唱える(となえる): 
\\	口: 
\\	昌: 
\\	口	
\\	くち・くちへん	
\\	形声 
\\	喜	キ	よろこ.ぶ、よろこ.ばす	あき, きゅ, のぶ, ゆき, よし	
\\	喜び(よろこび): 
\\	喜劇(きげき): 
\\	喜ぶ(よろこぶ): 
\\	吉: 
\\	口: 
\\	口	
\\	くち・くちへん	
\\	会意 
\\	器	キ	うつわ		
\\	器械(きかい): 
\\	武器(ぶき): 
\\	兵器(へいき): 
\\	楽器(がっき): 
\\	容器(ようき): 
\\	器(うつわ): 
\\	器具(きぐ): 
\\	食器(しょっき): 
\\	受話器(じゅわき): 
\\	陶器(とうき): 
\\	器官(きかん): 
\\	計器(けいき): 
\\	器用(きよう): 
\\	磁器(じき): 
\\	聴診器(ちょうしんき): 
\\	器(き): 
\\	口: 
\\	口: 
\\	大: 
\\	口: 
\\	口: 
\\	口	
\\	くち・くちへん	器	会意 
\\	囲	イ	かこ.む、かこ.う、かこ.い		
\\	周囲(しゅうい): 
\\	範囲(はんい): 
\\	雰囲気(ふんいき): 
\\	囲む(かこむ): 
\\	囗: 
\\	井: 
\\	囗	
\\	くに・くにがまえ	圍	形声 
\\	固	コ	かた.める、かた.まる、かた.まり、かた.い		
\\	固定(こてい): 
\\	固める(かためる): 
\\	固有(こゆう): 
\\	頑固(がんこ): 
\\	固体(こたい): 
\\	固まる(かたまる): 
\\	囗: 
\\	古: 
\\	囗	
\\	くに・くにがまえ	
\\	会意 
\\	型	ケイ	かた、-がた		
\\	型(かた): 
\\	典型(てんけい): 
\\	模型(もけい): 
\\	原型(げんけい): 
\\	刑: 
\\	土: 
\\	土	
\\	つち・つちへん・どへん	
\\	形声 
\\	堂	ドウ			
\\	食堂(しょくどう): 
\\	講堂(こうどう): 
\\	堂々(どうどう): 
\\	議事堂(ぎじどう): 
\\	尚: 
\\	土: 
\\	土	
\\	つち・つちへん・どへん	
\\	形声 
\\	塩	エン	しお		
\\	塩(しお): 
\\	食塩(しょくえん): 
\\	塩辛い(しおからい): 
\\	土: 
\\	口: 
\\	皿: 
\\	鹵	
\\	ろ・しお・しおへん	鹽	形声 
\\	士	シ	さむらい	お, ま	
\\	博士(はくし): 
\\	同士(どうし): 
\\	兵士(へいし): 
\\	紳士(しんし): 
\\	武士(ぶし): 
\\	修士(しゅうし): 
\\	学士(がくし): 
\\	士: 
\\	士	
\\	さむらい	
\\	象形 
\\	変	ヘン	か.わる、か.わり、か.える		
\\	変化(へんか): 
\\	大変(たいへん): 
\\	変わる(かわる): 
\\	変更(へんこう): 
\\	変動(へんどう): 
\\	相変わらず(あいかわらず): 
\\	変(へん): 
\\	変革(へんかく): 
\\	一変(いっぺん): 
\\	変遷(へんせん): 
\\	変える(かえる): 
\\	亦: 
\\	夂: 
\\	言	
\\	こと・げん・ごんべん・ことば	變	会意 
\\	夫	フ、フウ、ブ	おっと、それ	お, と, ゆう, よ	
\\	夫(おっと): 
\\	夫人(ふじん): 
\\	夫婦(ふうふ): 
\\	夫妻(ふさい): 
\\	工夫(くふう): 
\\	大丈夫(だいじょうぶ): 
\\	丈夫(じょうぶ): 
\\	丈夫(じょうふ): 
\\	大: 
\\	大	
\\	だい・おおきい	
\\	象形 
\\	失	シツ	うしな.う、う.せる		
\\	失業(しつぎょう): 
\\	損失(そんしつ): 
\\	失敗(しっぱい): 
\\	失う(うしなう): 
\\	過失(かしつ): 
\\	失望(しつぼう): 
\\	失格(しっかく): 
\\	紛失(ふんしつ): 
\\	失脚(しっきゃく): 
\\	失礼(しつれい): 
\\	失恋(しつれん): 
\\	失調(しっちょう): 
\\	失礼しました(しつれいしました): 
\\	失礼します(しつれいします): 
\\	丿: 
\\	夫: 
\\	大	
\\	だい・おおきい	
\\	象形 
\\	好	コウ	この.む、す.く、よ.い、い.い	こ, たか, とし, よし	
\\	好調(こうちょう): 
\\	友好(ゆうこう): 
\\	格好(かっこう): 
\\	好み(このみ): 
\\	大好き(だいすき): 
\\	好意(こうい): 
\\	良好(りょうこう): 
\\	好ましい(このましい): 
\\	好む(このむ): 
\\	好評(こうひょう): 
\\	好き嫌い(すききらい): 
\\	好況(こうきょう): 
\\	物好き(ものずき): 
\\	好き(すき): 
\\	好き好き(すきずき): 
\\	嗜好(しこう): 
\\	女: 
\\	子: 
\\	女	
\\	おんな・おんなへん	
\\	会意 
\\	季	キ		すえ, とし	
\\	季節(きせつ): 
\\	四季(しき): 
\\	季刊(きかん): 
\\	禾: 
\\	子: 
\\	子	
\\	こ・こへん	
\\	会意 
\\	孫	ソン	まご	ひ	
\\	孫(まご): 
\\	子孫(しそん): 
\\	子: 
\\	系: 
\\	子	
\\	こ・こへん	
\\	会意 
\\	完	カン			
\\	完成(かんせい): 
\\	完全(かんぜん): 
\\	完了(かんりょう): 
\\	完璧(かんぺき): 
\\	宀: 
\\	元: 
\\	宀	
\\	うかんむり	
\\	会意 
\\	官	カン			
\\	官僚(かんりょう): 
\\	官庁(かんちょう): 
\\	警官(けいかん): 
\\	長官(ちょうかん): 
\\	器官(きかん): 
\\	官(かん): 
\\	宀: 
\\	口: 
\\	宀	
\\	うかんむり	
\\	会意 
\\	害	ガイ			
\\	障害(しょうがい): 
\\	被害(ひがい): 
\\	災害(さいがい): 
\\	損害(そんがい): 
\\	公害(こうがい): 
\\	妨害(ぼうがい): 
\\	利害(りがい): 
\\	迫害(はくがい): 
\\	害(がい): 
\\	害する(がいする): 
\\	危害(きがい): 
\\	宀: 
\\	口: 
\\	宀	
\\	うかんむり	
\\	会意 
\\	察	サツ			
\\	警察(けいさつ): 
\\	視察(しさつ): 
\\	観察(かんさつ): 
\\	診察(しんさつ): 
\\	察する(さっする): 
\\	宀: 
\\	祭: 
\\	宀	
\\	うかんむり	
\\	会意 
\\	巣	ソウ	す、す.くう		
\\	巣(す): 
\\	小: 
\\	果: 
\\	巛	
\\	かわ・まがりがわ・さんぼんがわ	巢	象形 
\\	差	サ	さ.す、さ.し		
\\	差(さ): 
\\	格差(かくさ): 
\\	差別(さべつ): 
\\	交差点(こうさてん): 
\\	時差(じさ): 
\\	日差し(ひざし): 
\\	差し引き(さしひき): 
\\	誤差(ごさ): 
\\	差額(さがく): 
\\	差異(さい): 
\\	交差(こうさ): 
\\	差し上げる(さしあげる): 
\\	差し出す(さしだす): 
\\	差し支える(さしつかえる): 
\\	差し支え(さしつかえ): 
\\	差し掛かる(さしかかる): 
\\	差す(さす): 
\\	指差す(ゆびさす): 
\\	人差し指(ひとさしゆび): 
\\	差し引く(さしひく): 
\\	羊: 
\\	丿: 
\\	工: 
\\	工	
\\	こう・たくみ・たくみへん・え	
\\	形声 
\\	希	キ、ケ	まれ	のぞ, のぞみ	
\\	希望(きぼう): 
\\	乂: 
\\	布: 
\\	巾	
\\	はば・はばへん・きんべん	
\\	象形 
\\	席	セキ	むしろ		
\\	出席(しゅっせき): 
\\	席(せき): 
\\	座席(ざせき): 
\\	客席(きゃくせき): 
\\	欠席(けっせき): 
\\	着席(ちゃくせき): 
\\	广: 
\\	廿: 
\\	巾: 
\\	巾	
\\	はば・はばへん・きんべん	
\\	会意 
\\	帯	タイ	お.びる、おび	たて	
\\	地帯(ちたい): 
\\	世帯(せたい): 
\\	携帯(けいたい): 
\\	熱帯(ねったい): 
\\	連帯(れんたい): 
\\	一帯(いったい): 
\\	帯(たい): 
\\	温帯(おんたい): 
\\	包帯(ほうたい): 
\\	帯びる(おびる): 
\\	帯(おび): 
\\	寒帯(かんたい): 
\\	丗: 
\\	冖: 
\\	巾: 
\\	巾	
\\	はば・はばへん・きんべん	帶	象形 
\\	底	テイ	そこ		
\\	徹底(てってい): 
\\	底(そこ): 
\\	根底(こんてい): 
\\	到底(とうてい): 
\\	广: 
\\	氐: 
\\	广	
\\	まだれ	
\\	形声 
\\	府	フ		い, う, お, はん	
\\	政府(せいふ): 
\\	广: 
\\	付: 
\\	广	
\\	まだれ	
\\	形声 
\\	康	コウ		かん, こ, みち, やす, やすし	
\\	健康(けんこう): 
\\	广: 
\\	隶: 
\\	广	
\\	まだれ	
\\	会意 
\\	建	ケン、コン	た.てる、た.て、-だ.て、た.つ	たけ, たつ, たて	
\\	建て前(たてまえ): 
\\	建設(けんせつ): 
\\	建築(けんちく): 
\\	建物(たてもの): 
\\	再建(さいけん): 
\\	建つ(たつ): 
\\	二階建て(にかいだて): 
\\	建てる(たてる): 
\\	封建(ほうけん): 
\\	聿: 
\\	廴: 
\\	廴	
\\	えんにょう・いんにょう	
\\	会意 
\\	径	ケイ	みち、こみち、さしわたし、ただちに		
\\	直径(ちょっけい): 
\\	半径(はんけい): 
\\	彳: 
\\	圣: 
\\	彳	
\\	ぎょうにんべん	徑	形声 
\\	徒	ト	いたずら、あだ	かち	
\\	生徒(せいと): 
\\	徒歩(とほ): 
\\	彳: 
\\	走: 
\\	彳	
\\	ぎょうにんべん	
\\	形声 
\\	得	トク	え.る、う.る	あつ, てろ	
\\	所得(しょとく): 
\\	獲得(かくとく): 
\\	説得(せっとく): 
\\	得意(とくい): 
\\	納得(なっとく): 
\\	心得る(こころえる): 
\\	心得(こころえ): 
\\	得点(とくてん): 
\\	得る(うる): 
\\	損得(そんとく): 
\\	やむを得ない(やむをえない): 
\\	得る(える): 
\\	彳: 
\\	旦: 
\\	寸: 
\\	彳	
\\	ぎょうにんべん	
\\	会意 
\\	必	ヒツ	かなら.ず		
\\	必要(ひつよう): 
\\	必ず(かならず): 
\\	必ずしも(かならずしも): 
\\	必死(ひっし): 
\\	必然(ひつぜん): 
\\	必修(ひっしゅう): 
\\	必需品(ひつじゅひん): 
\\	心: 
\\	丿: 
\\	心	
\\	こころ・したごころ・りっしんべん	
\\	象形 
\\	念	ネン			
\\	記念(きねん): 
\\	残念(ざんねん): 
\\	概念(がいねん): 
\\	念願(ねんがん): 
\\	観念(かんねん): 
\\	念(ねん): 
\\	無念(むねん): 
\\	今: 
\\	心: 
\\	心	
\\	こころ・したごころ・りっしんべん	
\\	形声 
\\	愛	アイ	いと.しい、かな.しい、め.でる、お.しむ、まな	あ, あし, え, かな, なる, めぐ, めぐみ, よし, ちか	
\\	愛想(あいそ): 
\\	愛(あい): 
\\	愛情(あいじょう): 
\\	恋愛(れんあい): 
\\	愛する(あいする): 
\\	可愛い(かわいい): 
\\	可愛がる(かわいがる): 
\\	可愛らしい(かわいらしい): 
\\	爪: 
\\	冖: 
\\	心: 
\\	夂: 
\\	心	
\\	こころ・したごころ・りっしんべん	
\\	会意 
\\	成	セイ、ジョウ	な.る、な.す、-な.す	あき, あきら, しげ, そん, たえ, なお, なり, なる, のり, ひら, まさ, よし, り	
\\	成長(せいちょう): 
\\	賛成(さんせい): 
\\	成果(せいか): 
\\	成立(せいりつ): 
\\	結成(けっせい): 
\\	構成(こうせい): 
\\	成功(せいこう): 
\\	成績(せいせき): 
\\	完成(かんせい): 
\\	形成(けいせい): 
\\	達成(たっせい): 
\\	成人(せいじん): 
\\	成分(せいぶん): 
\\	合成(ごうせい): 
\\	成熟(せいじゅく): 
\\	成り立つ(なりたつ): 
\\	成年(せいねん): 
\\	作成(さくせい): 
\\	育成(いくせい): 
\\	成る(なる): 
\\	行き成り(いきなり): 
\\	持て成す(もてなす): 
\\	成るべく(なるべく): 
\\	成程(なるほど): 
\\	成育(せいいく): 
\\	成る丈(なるたけ): 
\\	勹: 
\\	戈: 
\\	戈	
\\	ほこ・ほこづくり・ほこがまえ・かのほこ	
\\	会意 
\\	戦	セン	いくさ、たたか.う、おのの.く、そよ.ぐ、わなな.く	せ	
\\	戦う(たたかう): 
\\	戦争(せんそう): 
\\	作戦(さくせん): 
\\	戦闘(せんとう): 
\\	挑戦(ちょうせん): 
\\	戦い(たたかい): 
\\	大戦(たいせん): 
\\	敗戦(はいせん): 
\\	戦災(せんさい): 
\\	戦術(せんじゅつ): 
\\	戦力(せんりょく): 
\\	休戦(きゅうせん): 
\\	戦(いくさ): 
\\	戦(せん): 
\\	単: 
\\	戈: 
\\	戈	
\\	ほこ・ほこづくり・ほこがまえ・かのほこ	戰	会意 
\\	折	セツ	お.る、おり、お.り、-お.り、お.れる	せき	
\\	折(おり): 
\\	時折(ときおり): 
\\	骨折(こっせつ): 
\\	折衷(せっちゅう): 
\\	折り返す(おりかえす): 
\\	屈折(くっせつ): 
\\	折る(おる): 
\\	折れる(おれる): 
\\	折角(せっかく): 
\\	手: 
\\	斤: 
\\	手	
\\	て・てへん	
\\	会意 
\\	挙	キョ	あ.げる、あ.がる、こぞ.る	たか	
\\	挙げる(あげる): 
\\	選挙(せんきょ): 
\\	一挙に(いっきょに): 
\\	小: 
\\	八: 
\\	手: 
\\	手	
\\	て・てへん	擧	会意 
\\	改	カイ	あらた.める、あらた.まる		
\\	改革(かいかく): 
\\	改正(かいせい): 
\\	改善(かいぜん): 
\\	改造(かいぞう): 
\\	改良(かいりょう): 
\\	改める(あらためる): 
\\	改札(かいさつ): 
\\	改定(かいてい): 
\\	改修(かいしゅう): 
\\	改めて(あらためて): 
\\	改訂(かいてい): 
\\	改まる(あらたまる): 
\\	改悪(かいあく): 
\\	己: 
\\	攵: 
\\	攴	
\\	ぼくにょう・ぼくづくり・とまた・のぶん・しぶん	
\\	形声 
\\	救	キュウ	すく.う		
\\	救援(きゅうえん): 
\\	救済(きゅうさい): 
\\	救助(きゅうじょ): 
\\	救い(すくい): 
\\	救う(すくう): 
\\	求: 
\\	攵: 
\\	攴	
\\	ぼくにょう・ぼくづくり・とまた・のぶん・しぶん	
\\	会意 
\\	敗	ハイ	やぶ.れる		
\\	腐敗(ふはい): 
\\	失敗(しっぱい): 
\\	敗戦(はいせん): 
\\	敗北(はいぼく): 
\\	勝敗(しょうはい): 
\\	一敗(いっぱい): 
\\	貝: 
\\	攵: 
\\	攴	
\\	ぼくにょう・ぼくづくり・とまた・のぶん・しぶん	
\\	会意 
\\	散	サン	ち.る、ち.らす、-ち.らす、ち.らかす、ち.らかる、ち.らばる、ばら、ばら.ける	ちる	
\\	解散(かいさん): 
\\	散歩(さんぽ): 
\\	分散(ぶんさん): 
\\	拡散(かくさん): 
\\	散る(ちる): 
\\	散らかる(ちらかる): 
\\	散らす(ちらす): 
\\	散らかす(ちらかす): 
\\	散蒔く(ばらまく): 
\\	月: 
\\	攵: 
\\	攴	
\\	ぼくにょう・ぼくづくり・とまた・のぶん・しぶん	
\\	会意 
\\	料	リョウ			
\\	資料(しりょう): 
\\	無料(むりょう): 
\\	料金(りょうきん): 
\\	料理(りょうり): 
\\	材料(ざいりょう): 
\\	燃料(ねんりょう): 
\\	食料(しょくりょう): 
\\	給料(きゅうりょう): 
\\	原料(げんりょう): 
\\	有料(ゆうりょう): 
\\	衣料(いりょう): 
\\	肥料(ひりょう): 
\\	調味料(ちょうみりょう): 
\\	食料品(しょくりょうひん): 
\\	送料(そうりょう): 
\\	香辛料(こうしんりょう): 
\\	料(りょう): 
\\	米: 
\\	斗: 
\\	斗	
\\	と・とます・ますづくり	
\\	会意 
\\	旗	キ	はた		
\\	旗(はた): 
\\	方: 
\\	其: 
\\	方	
\\	ほう・ほうへん・かた・かたへん	
\\	形声 
\\	昨	サク			
\\	一昨年(おととし): 
\\	昨日(きのう): 
\\	一昨日(おととい): 
\\	昨(さく): 
\\	一昨昨日(いっさくさくじつ): 
\\	日: 
\\	乍: 
\\	日	
\\	ひ・ひへん・にち・にちへん	
\\	形声 
\\	景	ケイ		かげ	
\\	景気(けいき): 
\\	背景(はいけい): 
\\	風景(ふうけい): 
\\	光景(こうけい): 
\\	不景気(ふけいき): 
\\	景色(けしき): 
\\	日: 
\\	京: 
\\	日	
\\	ひ・ひへん・にち・にちへん	
\\	形声 
\\	最	サイ、シュ	もっと.も、つま	も	
\\	"最も(もっとも): 
\\	最近(さいきん): 
\\	最後(さいご): 
\\	最高(さいこう): 
\\	最終(さいしゅう): 
\\	最初(さいしょ): 
\\	最低(さいてい): 
\\	最善(さいぜん): 
\\	最中(さいちゅう): 
\\	最早(もはや): 
\\	最(さい): 
\\	日: 
\\	取: 
\\	曰	
\\	ひらび・いわく	
\\	会意 
\\	望	ボウ、モウ	のぞ.む、もち	み, も	
\\	希望(きぼう): 
\\	要望(ようぼう): 
\\	展望(てんぼう): 
\\	絶望(ぜつぼう): 
\\	望み(のぞみ): 
\\	失望(しつぼう): 
\\	望む(のぞむ): 
\\	有望(ゆうぼう): 
\\	待望(たいぼう): 
\\	望遠鏡(ぼうえんきょう): 
\\	欲望(よくぼう): 
\\	望ましい(のぞましい): 
\\	待ち望む(まちのぞむ): 
\\	志望(しぼう): 
\\	亡: 
\\	月: 
\\	玉: 
\\	月	
\\	つき・つきへん	
\\	形声 
\\	未	ミ、ビ	いま.だ、ま.だ、ひつじ		
\\	未来(みらい): 
\\	未定(みてい): 
\\	未婚(みこん): 
\\	未知(みち): 
\\	未熟(みじゅく): 
\\	未練(みれん): 
\\	未満(みまん): 
\\	未開(みかい): 
\\	未(ひつじ): 
\\	未だ(まだ): 
\\	未(み): 
\\	木: 
\\	木	
\\	き・きへん	
\\	象形 
\\	末	マツ、バツ	すえ		
\\	末(すえ): 
\\	末期(まっき): 
\\	月末(げつまつ): 
\\	始末(しまつ): 
\\	粉末(ふんまつ): 
\\	粗末(そまつ): 
\\	期末(きまつ): 
\\	末っ子(すえっこ): 
\\	末(うら): 
\\	末(まつ): 
\\	木: 
\\	木	
\\	き・きへん	
\\	指事 
\\	札	サツ	ふだ	さっ	
\\	札(ふだ): 
\\	改札(かいさつ): 
\\	名札(なふだ): 
\\	札(さつ): 
\\	木: 
\\	乙: 
\\	木	
\\	き・きへん	
\\	会意 
\\	材	ザイ		き, さい	
\\	取材(しゅざい): 
\\	材料(ざいりょう): 
\\	人材(じんざい): 
\\	素材(そざい): 
\\	木材(もくざい): 
\\	材木(ざいもく): 
\\	教材(きょうざい): 
\\	木: 
\\	才: 
\\	木	
\\	き・きへん	
\\	形声 
\\	束	ソク	たば、たば.ねる、つか、つか.ねる		
\\	約束(やくそく): 
\\	拘束(こうそく): 
\\	束(たば): 
\\	結束(けっそく): 
\\	束縛(そくばく): 
\\	束ねる(たばねる): 
\\	木: 
\\	口: 
\\	木	
\\	き・きへん	
\\	象形 
\\	松	ショウ	まつ	おお, しょ, ま, まっ	
\\	松(まつ): 
\\	木: 
\\	公: 
\\	木	
\\	き・きへん	
\\	形声 
\\	果	カ	は.たす、はた.す、-は.たす、は.てる、-は.てる、は.て	み	
\\	結果(けっか): 
\\	効果(こうか): 
\\	成果(せいか): 
\\	果たして(はたして): 
\\	果物(くだもの): 
\\	果実(かじつ): 
\\	果て(はて): 
\\	果たす(はたす): 
\\	果てる(はてる): 
\\	田: 
\\	木: 
\\	木	
\\	き・きへん	
\\	象形 
\\	栄	エイ、ヨウ	さか.える、は.え、-ば.え、は.える、え	さかえ, しげ, てる, なが, ひで, よし	
\\	栄養(えいよう): 
\\	繁栄(はんえい): 
\\	栄える(さかえる): 
\\	小: 
\\	冖: 
\\	木: 
\\	木	
\\	き・きへん	榮	形声 
\\	案	アン	つくえ		
\\	提案(ていあん): 
\\	案内(あんない): 
\\	案外(あんがい): 
\\	答案(とうあん): 
\\	法案(ほうあん): 
\\	案の定(あんのじょう): 
\\	議案(ぎあん): 
\\	案(あん): 
\\	案じる(あんじる): 
\\	安: 
\\	木: 
\\	木	
\\	き・きへん	
\\	形声 
\\	梅	バイ	うめ		
\\	梅雨(つゆ): 
\\	梅(うめ): 
\\	梅干(うめぼし): 
\\	木: 
\\	毎: 
\\	木	
\\	き・きへん	梅	形声 
\\	械	カイ	かせ		
\\	器械(きかい): 
\\	木: 
\\	戒: 
\\	木	
\\	き・きへん	
\\	形声 
\\	極	キョク、ゴク	きわ.める、きわ.まる、きわ.まり、きわ.み、き.める、-ぎ.め、き.まる		
\\	積極的(せっきょくてき): 
\\	消極的(しょうきょくてき): 
\\	極端(きょくたん): 
\\	南極(なんきょく): 
\\	究極(きゅうきょく): 
\\	極(ごく): 
\\	北極(ほっきょく): 
\\	極楽(ごくらく): 
\\	極めて(きわめて): 
\\	両極(りょうきょく): 
\\	木: 
\\	亟: 
\\	木	
\\	き・きへん	
\\	形声 
\\	標	ヒョウ	しるべ、しるし	しべ	
\\	目標(もくひょう): 
\\	標準(ひょうじゅん): 
\\	標識(ひょうしき): 
\\	標本(ひょうほん): 
\\	標語(ひょうご): 
\\	座標(ざひょう): 
\\	木: 
\\	票: 
\\	木	
\\	き・きへん	
\\	形声 
\\	機	キ	はた		
\\	危機(きき): 
\\	機関(きかん): 
\\	機構(きこう): 
\\	機能(きのう): 
\\	機会(きかい): 
\\	飛行機(ひこうき): 
\\	動機(どうき): 
\\	機嫌(きげん): 
\\	機関車(きかんしゃ): 
\\	契機(けいき): 
\\	有機(ゆうき): 
\\	扇風機(せんぷうき): 
\\	交通機関(こうつうきかん): 
\\	ジェット機(ジェットき): 
\\	機(はた): 
\\	機(き): 
\\	木: 
\\	幾: 
\\	木	
\\	き・きへん	
\\	形声 
\\	欠	ケツ、ケン	か.ける、か.く		
\\	"欠陥(けっかん): 
\\	不可欠(ふかけつ): 
\\	欠席(けっせき): 
\\	欠点(けってん): 
\\	欠乏(けつぼう): 
\\	欠く(かく): 
\\	欠ける(かける): 
\\	欠伸(あくび): 
\\	欠: 
\\	缶	
\\	ほとぎ・ほとぎへん	缺	形声 
\\	歴	レキ、レッキ			
\\	歴史(れきし): 
\\	学歴(がくれき): 
\\	経歴(けいれき): 
\\	履歴(りれき): 
\\	厂: 
\\	林: 
\\	止: 
\\	止	
\\	とまる・とめる・とめへん	歷	形声 
\\	残	ザン、サン	のこ.る、のこ.す、そこな.う、のこ.り		
\\	残り(のこり): 
\\	残念(ざんねん): 
\\	残す(のこす): 
\\	残高(ざんだか): 
\\	残酷(ざんこく): 
\\	残る(のこる): 
\\	残らず(のこらず): 
\\	名残(なごり): 
\\	残金(ざんきん): 
\\	歹: 
\\	戔: 
\\	歹	
\\	がつ・がってん・がつへん・かばね・かばねへん・いちたへん	殘	形声 
\\	殺	サツ、サイ、セツ	ころ.す、-ごろ.し、そ.ぐ		
\\	殺人(さつじん): 
\\	殺す(ころす): 
\\	自殺(じさつ): 
\\	暗殺(あんさつ): 
\\	殺気(さっき): 
\\	乂: 
\\	木: 
\\	殳: 
\\	殳	
\\	るまた・ほこづくり	殺	会意 
\\	毒	ドク			
\\	毒(どく): 
\\	中毒(ちゅうどく): 
\\	消毒(しょうどく): 
\\	気の毒(きのどく): 
\\	お気の毒に(おきのどくに): 
\\	三: 
\\	母: 
\\	毋	
\\	なし・なかれ ・ははのかん	
\\	象形 
\\	氏	シ	うじ、-うじ		
\\	氏名(しめい): 
\\	氏(うじ): 
\\	氏(し): 
\\	氏: 
\\	氏	
\\	うじ	
\\	象形 
\\	民	ミン	たみ	ひと, み	
\\	国民(こくみん): 
\\	市民(しみん): 
\\	住民(じゅうみん): 
\\	民間(みんかん): 
\\	民主(みんしゅ): 
\\	民族(みんぞく): 
\\	人民(じんみん): 
\\	農民(のうみん): 
\\	庶民(しょみん): 
\\	移民(いみん): 
\\	植民地(しょくみんち): 
\\	民謡(みんよう): 
\\	民宿(みんしゅく): 
\\	民俗(みんぞく): 
\\	氏: 
\\	氏	
\\	うじ	
\\	象形 
\\	求	キュウ、グ	もと.める		
\\	要求(ようきゅう): 
\\	請求(せいきゅう): 
\\	求める(もとめる): 
\\	水: 
\\	丶: 
\\	水	
\\	みず・したみず・さんずい	
\\	象形 
\\	治	ジ、チ	おさ.める、おさ.まる、なお.る、なお.す	し, ぢ, はり, はる, みち	
\\	政治(せいじ): 
\\	治療(ちりょう): 
\\	自治(じち): 
\\	治安(ちあん): 
\\	統治(とうち): 
\\	退治(たいじ): 
\\	治す(なおす): 
\\	治まる(おさまる): 
\\	治める(おさめる): 
\\	水: 
\\	台: 
\\	水	
\\	みず・したみず・さんずい	
\\	形声 
\\	法	ホウ、ハッ、ホッ、フラン	のり	ほ	
\\	憲法(けんぽう): 
\\	方法(ほうほう): 
\\	法律(ほうりつ): 
\\	司法(しほう): 
\\	法廷(ほうてい): 
\\	立法(りっぽう): 
\\	法学(ほうがく): 
\\	法則(ほうそく): 
\\	寸法(すんぽう): 
\\	文法(ぶんぽう): 
\\	法案(ほうあん): 
\\	手法(しゅほう): 
\\	作法(さほう): 
\\	製法(せいほう): 
\\	法(ほう): 
\\	用法(ようほう): 
\\	水: 
\\	去: 
\\	水	
\\	みず・したみず・さんずい	
\\	会意 
\\	泣	キュウ	な.く		
\\	泣く(なく): 
\\	水: 
\\	立: 
\\	水	
\\	みず・したみず・さんずい	
\\	形声 
\\	浅	セン	あさ.い	あざ, さ	
\\	浅い(あさい): 
\\	浅ましい(あさましい): 
\\	水: 
\\	戔: 
\\	水	
\\	みず・したみず・さんずい	淺	形声 
\\	浴	ヨク	あ.びる、あ.びせる	えき, さこ	
\\	浴びる(あびる): 
\\	入浴(にゅうよく): 
\\	浴室(よくしつ): 
\\	浴衣(ゆかた): 
\\	海水浴(かいすいよく): 
\\	水: 
\\	谷: 
\\	水	
\\	みず・したみず・さんずい	
\\	形声 
\\	清	セイ、ショウ、シン	きよ.い、きよ.まる、きよ.める	あき, さや, し, すが, すみ, せ, ちん	
\\	清掃(せいそう): 
\\	清潔(せいけつ): 
\\	清らか(きよらか): 
\\	清い(きよい): 
\\	清書(せいしょ): 
\\	清々しい(すがすがしい): 
\\	清純(せいじゅん): 
\\	清濁(せいだく): 
\\	水: 
\\	青: 
\\	水	
\\	みず・したみず・さんずい	
\\	形声 
\\	満	マン、バン	み.ちる、み.つ、み.たす	ま, みち, みつ, みつる	
\\	不満(ふまん): 
\\	満足(まんぞく): 
\\	満員(まんいん): 
\\	満たす(みたす): 
\\	満点(まんてん): 
\\	円満(えんまん): 
\\	未満(みまん): 
\\	満月(まんげつ): 
\\	満ちる(みちる): 
\\	満場(まんじょう): 
\\	水: 
\\	廿: 
\\	水	
\\	みず・したみず・さんずい	滿	形声 
\\	漁	ギョ、リョウ	あさ.る		
\\	漁業(ぎょぎょう): 
\\	漁船(ぎょせん): 
\\	漁師(りょうし): 
\\	漁村(ぎょそん): 
\\	水: 
\\	魚: 
\\	水	
\\	みず・したみず・さんずい	
\\	形声 
\\	灯	トウ	ひ、ほ-、ともしび、とも.す、あかり		
\\	灯り(あかり): 
\\	灯(ひ): 
\\	灯油(とうゆ): 
\\	電灯(でんとう): 
\\	灯台(とうだい): 
\\	蛍光灯(けいこうとう): 
\\	火: 
\\	丁: 
\\	火	
\\	ひ・ひへん・れっか・れんが	燈	形声 
\\	無	ム、ブ	な.い		
\\	無料(むりょう): 
\\	間も無く(まもなく): 
\\	無理(むり): 
\\	無線(むせん): 
\\	無効(むこう): 
\\	有無(うむ): 
\\	無事(ぶじ): 
\\	無言(むごん): 
\\	無断(むだん): 
\\	無数(むすう): 
\\	無用(むよう): 
\\	無限(むげん): 
\\	無視(むし): 
\\	無駄(むだ): 
\\	無難(ぶなん): 
\\	無駄遣い(むだづかい): 
\\	無論(むろん): 
\\	ご無沙汰(ごぶさた): 
\\	無邪気(むじゃき): 
\\	無口(むくち): 
\\	無知(むち): 
\\	無念(むねん): 
\\	台無し(だいなし): 
\\	無地(むじ): 
\\	無能(むのう): 
\\	無意味(むいみ): 
\\	無礼(ぶれい): 
\\	無くす(なくす): 
\\	無し(なし): 
\\	無茶(むちゃ): 
\\	無い(ない): 
\\	無くなる(なくなる): 
\\	無闇に(むやみに): 
\\	無茶苦茶(むちゃくちゃ): 
\\	だらし無い(だらしない): 
\\	とんでも無い(とんでもない): 
\\	感無量(かんむりょう): 
\\	仕様が無い(しようがない): 
\\	無沙汰(ぶさた): 
\\	無(む): 
\\	無(ぶ): 
\\	丿: 
\\	一: 
\\	火: 
\\	火	
\\	ひ・ひへん・れっか・れんが	
\\	仮借 
\\	然	ゼン、ネン	しか、しか.り、しか.し、さ		
\\	自然(しぜん): 
\\	当然(とうぜん): 
\\	依然(いぜん): 
\\	天然(てんねん): 
\\	突然(とつぜん): 
\\	偶然(ぐうぜん): 
\\	全然(ぜんぜん): 
\\	公然(こうぜん): 
\\	自然科学(しぜんかがく): 
\\	断然(だんぜん): 
\\	必然(ひつぜん): 
\\	漠然(ばくぜん): 
\\	然し(しかし): 
\\	然して(そして): 
\\	然も(しかも): 
\\	然うして(そうして): 
\\	然う(そう): 
\\	整然(せいぜん): 
\\	然も(さも): 
\\	然程(さほど): 
\\	呆然(ぼうぜん): 
\\	月: 
\\	犬: 
\\	火: 
\\	火	
\\	ひ・ひへん・れっか・れんが	
\\	会意 
\\	焼	ショウ	や.く、や.き、や.き-、-や.き、や.ける	やい, やき	
\\	燃焼(ねんしょう): 
\\	日焼け(ひやけ): 
\\	夕焼け(ゆうやけ): 
\\	焼く(やく): 
\\	焼ける(やける): 
\\	火: 
\\	尭: 
\\	火	
\\	ひ・ひへん・れっか・れんが	燒	形声 
\\	照	ショウ	て.る、て.らす、て.れる	あき, てる	
\\	参照(さんしょう): 
\\	照明(しょうめい): 
\\	対照(たいしょう): 
\\	照らす(てらす): 
\\	照る(てる): 
\\	照合(しょうごう): 
\\	照り返す(てりかえす): 
\\	昭: 
\\	火: 
\\	火	
\\	ひ・ひへん・れっか・れんが	
\\	形声 
\\	熱	ネツ	あつ.い	あた	
\\	熱(ねつ): 
\\	熱心(ねっしん): 
\\	熱帯(ねったい): 
\\	情熱(じょうねつ): 
\\	熱意(ねつい): 
\\	熱い(あつい): 
\\	熱湯(ねっとう): 
\\	熱する(ねっする): 
\\	加熱(かねつ): 
\\	光熱費(こうねつひ): 
\\	熱中(ねっちゅう): 
\\	熱量(ねつりょう): 
\\	埶: 
\\	火: 
\\	火	
\\	ひ・ひへん・れっか・れんが	
\\	会意 
\\	牧	ボク	まき	ま, まい, もく	
\\	牧場(まきば): 
\\	牧師(ぼくし): 
\\	遊牧(ゆうぼく): 
\\	牧畜(ぼくちく): 
\\	牛: 
\\	攵: 
\\	牛	
\\	うし	
\\	会意 
\\	特	トク			
\\	特に(とくに): 
\\	特別(とくべつ): 
\\	特徴(とくちょう): 
\\	特定(とくてい): 
\\	特殊(とくしゅ): 
\\	独特(どくとく): 
\\	特集(とくしゅう): 
\\	特色(とくしょく): 
\\	特許(とっきょ): 
\\	特有(とくゆう): 
\\	特急(とっきゅう): 
\\	特権(とっけん): 
\\	特産(とくさん): 
\\	特技(とくぎ): 
\\	特売(とくばい): 
\\	特長(とくちょう): 
\\	特派(とくは): 
\\	牛: 
\\	寺: 
\\	牛	
\\	うし	
\\	形声 
\\	産	サン	う.む、う.まれる、うぶ-、む.す	もと	
\\	産業(さんぎょう): 
\\	生産(せいさん): 
\\	資産(しさん): 
\\	不動産(ふどうさん): 
\\	財産(ざいさん): 
\\	国産(こくさん): 
\\	出産(しゅっさん): 
\\	倒産(とうさん): 
\\	産地(さんち): 
\\	水産(すいさん): 
\\	破産(はさん): 
\\	土産(みやげ): 
\\	産婦人科(さんふじんか): 
\\	産物(さんぶつ): 
\\	特産(とくさん): 
\\	共産(きょうさん): 
\\	農産物(のうさんぶつ): 
\\	畜産(ちくさん): 
\\	原産(げんさん): 
\\	お産(おさん): 
\\	お土産(おみやげ): 
\\	産休(さんきゅう): 
\\	産後(さんご): 
\\	産出(さんしゅつ): 
\\	名産(めいさん): 
\\	土産(どさん): 
\\	産(さん): 
\\	立: 
\\	厂: 
\\	生: 
\\	生	
\\	うむ・うまれる・いきる	
\\	会意 
\\	的	テキ	まと	いくは, ゆくは	
\\	積極的(せっきょくてき): 
\\	目的(もくてき): 
\\	消極的(しょうきょくてき): 
\\	知的(ちてき): 
\\	比較的(ひかくてき): 
\\	先天的(せんてんてき): 
\\	的(てき): 
\\	的(まと): 
\\	動的(どうてき): 
\\	静的(せいてき): 
\\	日: 
\\	勺: 
\\	白	
\\	しろ・しろへん	
\\	形声 
\\	省	セイ、ショウ	かえり.みる、はぶ.く	さとし, み	
\\	反省(はんせい): 
\\	省略(しょうりゃく): 
\\	省く(はぶく): 
\\	省(しょう): 
\\	少: 
\\	目: 
\\	目	
\\	め・めへん・よこめ	
\\	形声 
\\	祝	シュク、シュウ	いわ.う	のり, ほぎ, ゆわい	
\\	祝い(いわい): 
\\	祝日(しゅくじつ): 
\\	祝う(いわう): 
\\	祝賀(しゅくが): 
\\	お祝い(おいわい): 
\\	示: 
\\	兄: 
\\	示	
\\	しめす・しめすへん	祝	会意 
\\	票	ヒョウ			
\\	投票(とうひょう): 
\\	票(ひょう): 
\\	襾: 
\\	示: 
\\	示	
\\	しめす・しめすへん	
\\	会意 
\\	種	シュ	たね、-ぐさ	おい, くさ, た, ほ	
\\	一種(いっしゅ): 
\\	種類(しゅるい): 
\\	人種(じんしゅ): 
\\	品種(ひんしゅ): 
\\	種々(しゅじゅ): 
\\	各種(かくしゅ): 
\\	種(たね): 
\\	種(しゅ): 
\\	禾: 
\\	重: 
\\	禾	
\\	のぎ・のぎへん	
\\	形声 
\\	積	セキ	つ.む、-づ.み、つ.もる、つ.もり	か, さか, しゃこ, ずみ, つみ	
\\	見積り(みつもり): 
\\	積極的(せっきょくてき): 
\\	面積(めんせき): 
\\	蓄積(ちくせき): 
\\	積む(つむ): 
\\	容積(ようせき): 
\\	体積(たいせき): 
\\	積もり(つもり): 
\\	積もる(つもる): 
\\	禾: 
\\	責: 
\\	禾	
\\	のぎ・のぎへん	
\\	形声 
\\	競	キョウ、ケイ	きそ.う、せ.る、くら.べる	かい, わたなべ	
\\	競技(きょうぎ): 
\\	競争(きょうそう): 
\\	競馬(けいば): 
\\	立: 
\\	兄: 
\\	立: 
\\	兄: 
\\	立	
\\	たつ・たつへん	
\\	会意 
\\	笑	ショウ	わら.う、え.む	えみ	
\\	笑い(わらい): 
\\	笑顔(えがお): 
\\	微笑(びしょう): 
\\	笑う(わらう): 
\\	可笑しい(おかしい): 
\\	微笑む(ほほえむ): 
\\	あざ笑う(あざわらう): 
\\	竹: 
\\	夭: 
\\	竹	
\\	たけ・たけかんむり	
\\	象形 
\\	管	カン	くだ	すが	
\\	管(くだ): 
\\	管理(かんり): 
\\	血管(けっかん): 
\\	保管(ほかん): 
\\	只管(ひたすら): 
\\	竹: 
\\	官: 
\\	竹	
\\	たけ・たけかんむり	
\\	形声 
\\	節	セツ、セチ	ふし、-ぶし、のっと	たかし	
\\	季節(きせつ): 
\\	節(せつ): 
\\	節約(せつやく): 
\\	調節(ちょうせつ): 
\\	節(ふし): 
\\	竹: 
\\	即: 
\\	竹	
\\	たけ・たけかんむり	節	形声 
\\	粉	フン	デシメートル、こ、こな		
\\	粉(こな): 
\\	粉末(ふんまつ): 
\\	花粉(かふん): 
\\	粉々(こなごな): 
\\	米: 
\\	分: 
\\	米	
\\	こめ・こめへん	
\\	形声 
\\	紀	キ		おさむ, ただす, とし, とも, のり, もと	
\\	世紀(せいき): 
\\	糸: 
\\	己: 
\\	糸	
\\	いと・いとへん	
\\	形声 
\\	約	ヤク	つづ.まる、つづ.める、つづま.やか		
\\	契約(けいやく): 
\\	条約(じょうやく): 
\\	約束(やくそく): 
\\	予約(よやく): 
\\	制約(せいやく): 
\\	婚約(こんやく): 
\\	節約(せつやく): 
\\	規約(きやく): 
\\	約(やく): 
\\	倹約(けんやく): 
\\	糸: 
\\	勺: 
\\	糸	
\\	いと・いとへん	
\\	形声 
\\	結	ケツ、ケチ	むす.ぶ、ゆ.う、ゆ.わえる	ゆい, ゆう	
\\	"結果(けっか): 
\\	結局(けっきょく): 
\\	結婚(けっこん): 
\\	結論(けつろん): 
\\	結成(けっせい): 
\\	結ぶ(むすぶ): 
\\	結構(けっこう): 
\\	結核(けっかく): 
\\	団結(だんけつ): 
\\	結び(むすび): 
\\	妥結(だけつ): 
\\	結晶(けっしょう): 
\\	結合(けつごう): 
\\	結びつき(むすびつき): 
\\	結束(けっそく): 
\\	結びつく(むすびつく): 
\\	結びつける(むすびつける): 
\\	糸: 
\\	吉: 
\\	糸	
\\	いと・いとへん	
\\	形声 
\\	給	キュウ	たま.う、たも.う、-たま.え	きい	
\\	供給(きょうきゅう): 
\\	給与(きゅうよ): 
\\	給食(きゅうしょく): 
\\	給料(きゅうりょう): 
\\	補給(ほきゅう): 
\\	月給(げっきゅう): 
\\	支給(しきゅう): 
\\	配給(はいきゅう): 
\\	給仕(きゅうじ): 
\\	給う(たまう): 
\\	糸: 
\\	合: 
\\	糸	
\\	いと・いとへん	
\\	形声 
\\	続	ゾク、ショク、コウ、キョウ	つづ.く、つづ.ける、つぐ.ない	つぐ	
\\	手続(てつづき): 
\\	連続(れんぞく): 
\\	継続(けいぞく): 
\\	持続(じぞく): 
\\	続々(ぞくぞく): 
\\	続き(つづき): 
\\	相続(そうぞく): 
\\	接続(せつぞく): 
\\	存続(そんぞく): 
\\	続く(つづく): 
\\	続ける(つづける): 
\\	接続詞(せつぞくし): 
\\	糸: 
\\	売: 
\\	糸	
\\	いと・いとへん	續	形声 
\\	置	チ	お.く、-お.き	おき, おけ, き	
\\	措置(そち): 
\\	設置(せっち): 
\\	装置(そうち): 
\\	位置(いち): 
\\	配置(はいち): 
\\	処置(しょち): 
\\	放置(ほうち): 
\\	物置(ものおき): 
\\	前置き(まえおき): 
\\	置く(おく): 
\\	网: 
\\	直: 
\\	网	
\\	あみ・あみがしら・あみめ・よこめ・よんかしら	
\\	形声 
\\	老	ロウ	お.いる、ふ.ける	えび, おい, び	
\\	老人(ろうじん): 
\\	老衰(ろうすい): 
\\	老い(おい): 
\\	老いる(おいる): 
\\	老ける(ふける): 
\\	老: 
\\	匕: 
\\	老	
\\	おい・おいかんむり・おいがしら	
\\	会意 
\\	胃	イ			
\\	"胃(い): 
\\	田: 
\\	月: 
\\	肉	
\\	にく・にくづき	
\\	会意 
\\	脈	ミャク	すじ		
\\	山脈(さんみゃく): 
\\	文脈(ぶんみゃく): 
\\	脈(みゃく): 
\\	肉: 
\\	𠂢: 
\\	肉	
\\	にく・にくづき	
\\	会意 
\\	腸	チョウ	はらわた、わた		
\\	腸(ちょう): 
\\	肉: 
\\	昜: 
\\	肉	
\\	にく・にくづき	
\\	形声 
\\	臣	シン、ジン		おみ, たか, と, とみ, み	
\\	大臣(だいじん): 
\\	総理大臣(そうりだいじん): 
\\	臣: 
\\	臣	
\\	しん・おみ	
\\	象形 
\\	航	コウ		わたる	
\\	航空(こうくう): 
\\	航海(こうかい): 
\\	舟: 
\\	亢: 
\\	舟	
\\	ふね・ふねへん	
\\	形声 
\\	良	リョウ	よ.い、-よ.い、い.い、-い.い	じ, なが, まこと, よし, ら, りょ, ろう	
\\	良い(いい): 
\\	不良(ふりょう): 
\\	改良(かいりょう): 
\\	良好(りょうこう): 
\\	良心(りょうしん): 
\\	仲良し(なかよし): 
\\	良識(りょうしき): 
\\	良質(りょうしつ): 
\\	良く(よく): 
\\	善良(ぜんりょう): 
\\	良し(よし): 
\\	艮: 
\\	艮	
\\	うしとら・こん・こんづくり・ねづくり	
\\	象形 
\\	芸	ゲイ、ウン	う.える、のり、わざ	き, げ, なり	
\\	芸術(げいじゅつ): 
\\	芸能(げいのう): 
\\	文芸(ぶんげい): 
\\	工芸(こうげい): 
\\	園芸(えんげい): 
\\	芸(げい): 
\\	学芸(がくげい): 
\\	手芸(しゅげい): 
\\	艸: 
\\	云: 
\\	艸	
\\	くさ・くさかんむり	藝	会意 
\\	芽	ガ	め	じ, めぐ	
\\	芽(め): 
\\	発芽(はつが): 
\\	艸: 
\\	牙: 
\\	艸	
\\	くさ・くさかんむり	
\\	形声 
\\	英	エイ	はなぶさ	あ, あい, え, すぐる, はな, ひ, ひで, よし, ら	
\\	英語(えいご): 
\\	英雄(えいゆう): 
\\	英文(えいぶん): 
\\	英和(えいわ): 
\\	英字(えいじ): 
\\	和英(わえい): 
\\	艸: 
\\	央: 
\\	艸	
\\	くさ・くさかんむり	
\\	形声 
\\	菜	サイ	な	よう	
\\	野菜(やさい): 
\\	艸: 
\\	采: 
\\	艸	
\\	くさ・くさかんむり	
\\	形声 
\\	街	ガイ、カイ	まち	また	
\\	街頭(がいとう): 
\\	市街(しがい): 
\\	街道(かいどう): 
\\	街(がい): 
\\	行: 
\\	圭: 
\\	行	
\\	ぎょう・ぎょうがまえ・いく・ゆきがまえ	
\\	形声 
\\	衣	イ、エ	ころも、きぬ、-ぎ		
\\	衣装(いしょう): 
\\	衣料(いりょう): 
\\	衣類(いるい): 
\\	衣服(いふく): 
\\	浴衣(ゆかた): 
\\	衣食住(いしょくじゅう): 
\\	粗衣(そい): 
\\	亠: 
\\	衣	
\\	ころも・ころもへん	
\\	象形 
\\	要	ヨウ	い.る、かなめ	とし	
\\	重要(じゅうよう): 
\\	必要(ひつよう): 
\\	要求(ようきゅう): 
\\	主要(しゅよう): 
\\	需要(じゅよう): 
\\	要請(ようせい): 
\\	要因(よういん): 
\\	要素(ようそ): 
\\	要望(ようぼう): 
\\	要旨(ようし): 
\\	要する(ようする): 
\\	要するに(ようするに): 
\\	要領(ようりょう): 
\\	要点(ようてん): 
\\	要る(いる): 
\\	襾: 
\\	女: 
\\	襾	
\\	にし・かなめのかしら・おおいかんむり	
\\	象形 
\\	覚	カク	おぼ.える、さ.ます、さ.める、さと.る		
\\	感覚(かんかく): 
\\	覚悟(かくご): 
\\	自覚(じかく): 
\\	覚え(おぼえ): 
\\	味覚(みかく): 
\\	聴覚(ちょうかく): 
\\	錯覚(さっかく): 
\\	視覚(しかく): 
\\	覚える(おぼえる): 
\\	目覚める(めざめる): 
\\	目覚まし(めざまし): 
\\	目覚ましい(めざましい): 
\\	覚める(さめる): 
\\	覚ます(さます): 
\\	小: 
\\	冖: 
\\	見: 
\\	見	
\\	みる	覺	形声 
\\	観	カン	み.る、しめ.す		
\\	観光(かんこう): 
\\	観測(かんそく): 
\\	観客(かんきゃく): 
\\	観察(かんさつ): 
\\	悲観(ひかん): 
\\	観衆(かんしゅう): 
\\	客観(きゃっかん): 
\\	外観(がいかん): 
\\	観念(かんねん): 
\\	直観(ちょっかん): 
\\	観点(かんてん): 
\\	楽観(らっかん): 
\\	主観(しゅかん): 
\\	観(かん): 
\\	観覧(かんらん): 
\\	隹: 
\\	見: 
\\	見	
\\	みる	觀	形声 
\\	訓	クン、キン	おし.える、よ.む、くん.ずる	く, くに, くの, さとし, のり, ふみ	
\\	訓練(くんれん): 
\\	教訓(きょうくん): 
\\	訓(くん): 
\\	言: 
\\	巛: 
\\	言	
\\	こと・げん・ごんべん・ことば	
\\	形声 
\\	試	シ	こころ.みる、ため.す		
\\	試合(しあい): 
\\	試験(しけん): 
\\	試み(こころみ): 
\\	試す(ためす): 
\\	試し(ためし): 
\\	試みる(こころみる): 
\\	模試(もし): 
\\	言: 
\\	式: 
\\	言	
\\	こと・げん・ごんべん・ことば	
\\	形声 
\\	説	セツ、ゼイ	と.く	さとし, とき	
\\	演説(えんぜつ): 
\\	説明(せつめい): 
\\	小説(しょうせつ): 
\\	解説(かいせつ): 
\\	説得(せっとく): 
\\	伝説(でんせつ): 
\\	社説(しゃせつ): 
\\	学説(がくせつ): 
\\	説(せつ): 
\\	説く(とく): 
\\	概説(がいせつ): 
\\	言: 
\\	兌: 
\\	言	
\\	こと・げん・ごんべん・ことば	
\\	形声 
\\	課	カ			
\\	課目(かもく): 
\\	課題(かだい): 
\\	課長(かちょう): 
\\	課税(かぜい): 
\\	課(か): 
\\	課程(かてい): 
\\	日課(にっか): 
\\	課外(かがい): 
\\	言: 
\\	果: 
\\	言	
\\	こと・げん・ごんべん・ことば	
\\	形声 
\\	議	ギ		かた, のり	
\\	会議(かいぎ): 
\\	議員(ぎいん): 
\\	議会(ぎかい): 
\\	議長(ぎちょう): 
\\	協議(きょうぎ): 
\\	決議(けつぎ): 
\\	審議(しんぎ): 
\\	議論(ぎろん): 
\\	抗議(こうぎ): 
\\	議題(ぎだい): 
\\	参議院(さんぎいん): 
\\	討議(とうぎ): 
\\	議決(ぎけつ): 
\\	衆議院(しゅうぎいん): 
\\	不思議(ふしぎ): 
\\	論議(ろんぎ): 
\\	会議室(かいぎしつ): 
\\	異議(いぎ): 
\\	議案(ぎあん): 
\\	議事堂(ぎじどう): 
\\	物議(ぶつぎ): 
\\	合議(ごうぎ): 
\\	言: 
\\	義: 
\\	言	
\\	こと・げん・ごんべん・ことば	
\\	形声 
\\	象	ショウ、ゾウ	かたど.る	きさ	
\\	対象(たいしょう): 
\\	印象(いんしょう): 
\\	現象(げんしょう): 
\\	気象(きしょう): 
\\	象徴(しょうちょう): 
\\	抽象(ちゅうしょう): 
\\	象(ぞう): 
\\	勹: 
\\	豕: 
\\	豕	
\\	いのこ・いのこへん・ぶた	
\\	象形 
\\	貨	カ	たから		
\\	通貨(つうか): 
\\	貨物(かもつ): 
\\	外貨(がいか): 
\\	硬貨(こうか): 
\\	貨幣(かへい): 
\\	雑貨(ざっか): 
\\	銅貨(どうか): 
\\	化: 
\\	貝: 
\\	貝	
\\	かい・かいへん・こがい	
\\	形声 
\\	貯	チョ	た.める、たくわ.える		
\\	貯金(ちょきん): 
\\	貯蓄(ちょちく): 
\\	貯蔵(ちょぞう): 
\\	貝: 
\\	宀: 
\\	丁: 
\\	貝	
\\	かい・かいへん・こがい	
\\	形声 
\\	費	ヒ	つい.やす、つい.える		
\\	消費(しょうひ): 
\\	費用(ひよう): 
\\	経費(けいひ): 
\\	実費(じっぴ): 
\\	浪費(ろうひ): 
\\	出費(しゅっぴ): 
\\	光熱費(こうねつひ): 
\\	費やす(ついやす): 
\\	費(ひ): 
\\	弗: 
\\	貝: 
\\	貝	
\\	かい・かいへん・こがい	
\\	形声 
\\	賞	ショウ	ほ.める		
\\	賞金(しょうきん): 
\\	入賞(にゅうしょう): 
\\	賞(しょう): 
\\	鑑賞(かんしょう): 
\\	賞品(しょうひん): 
\\	懸賞(けんしょう): 
\\	尚: 
\\	貝: 
\\	貝	
\\	かい・かいへん・こがい	
\\	形声 
\\	軍	グン	いくさ		
\\	軍事(ぐんじ): 
\\	軍(ぐん): 
\\	軍隊(ぐんたい): 
\\	軍備(ぐんび): 
\\	軍艦(ぐんかん): 
\\	軍服(ぐんぷく): 
\\	冖: 
\\	車: 
\\	車	
\\	くるま・くるまへん	
\\	象形 
\\	輪	リン	わ	なわ, も	
\\	輪(わ): 
\\	指輪(ゆびわ): 
\\	車輪(しゃりん): 
\\	年輪(ねんりん): 
\\	首輪(くびわ): 
\\	輪(りん): 
\\	車: 
\\	侖: 
\\	車	
\\	くるま・くるまへん	
\\	形声 
\\	辞	ジ	や.める、いな.む		
\\	百科辞典(ひゃっかじてん): 
\\	辞職(じしょく): 
\\	辞書(じしょ): 
\\	辞典(じてん): 
\\	辞退(じたい): 
\\	お世辞(おせじ): 
\\	辞める(やめる): 
\\	お辞儀(おじぎ): 
\\	世辞(せじ): 
\\	舌: 
\\	辛: 
\\	辛	
\\	しん・からい	辭	会意 
\\	辺	ヘン	あた.り、ほと.り、-べ	なべ	
\\	周辺(しゅうへん): 
\\	辺(へん): 
\\	辺り(あたり): 
\\	浜辺(はまべ): 
\\	天辺(てっぺん): 
\\	刀: 
\\	辶: 
\\	辵	
\\	しんにょう・しんにゅう	邊	形声 
\\	連	レン	つら.なる、つら.ねる、つ.れる、-づ.れ	ずれ, つれ, むらじ, れ	
\\	関連(かんれん): 
\\	国連(こくれん): 
\\	連合(れんごう): 
\\	連続(れんぞく): 
\\	連邦(れんぽう): 
\\	連絡(れんらく): 
\\	連日(れんじつ): 
\\	連休(れんきゅう): 
\\	連帯(れんたい): 
\\	連中(れんちゅう): 
\\	連れる(つれる): 
\\	連れ(つれ): 
\\	連盟(れんめい): 
\\	一連(いちれん): 
\\	連ねる(つらねる): 
\\	連想(れんそう): 
\\	連なる(つらなる): 
\\	車: 
\\	辶: 
\\	辵	
\\	しんにょう・しんにゅう	
\\	会意 
\\	達	タツ、ダ	たち	かつ, さと, て, てつ, とおる, みち	
\\	達成(たっせい): 
\\	友達(ともだち): 
\\	配達(はいたつ): 
\\	発達(はったつ): 
\\	伝達(でんたつ): 
\\	到達(とうたつ): 
\\	達する(たっする): 
\\	伊達(だて): 
\\	上達(じょうたつ): 
\\	達者(たっしゃ): 
\\	速達(そくたつ): 
\\	達(たち): 
\\	土: 
\\	羊: 
\\	辶: 
\\	辵	
\\	しんにょう・しんにゅう	
\\	形声 
\\	選	セン	えら.ぶ		
\\	選挙(せんきょ): 
\\	選手(せんしゅ): 
\\	当選(とうせん): 
\\	選ぶ(えらぶ): 
\\	選択(せんたく): 
\\	抽選(ちゅうせん): 
\\	選考(せんこう): 
\\	巽: 
\\	辶: 
\\	辵	
\\	しんにょう・しんにゅう	
\\	形声 
\\	郡	グン	こおり		
\\	郡(ぐん): 
\\	君: 
\\	邑: 
\\	邑	
\\	むら・おおざと・おおざとづくり・ゆう	
\\	形声 
\\	量	リョウ	はか.る	かず	
\\	重量(じゅうりょう): 
\\	測量(そくりょう): 
\\	分量(ぶんりょう): 
\\	微量(びりょう): 
\\	量(りょう): 
\\	熱量(ねつりょう): 
\\	感無量(かんむりょう): 
\\	旦: 
\\	里: 
\\	里	
\\	さと・さとへん	
\\	象形 
\\	録	ロク	しる.す、と.る		
\\	記録(きろく): 
\\	登録(とうろく): 
\\	録音(ろくおん): 
\\	目録(もくろく): 
\\	付録(ふろく): 
\\	金: 
\\	彔: 
\\	金	
\\	かね・かねへん	錄	形声 
\\	鏡	キョウ、ケイ	かがみ	あき, かが, かがん	
\\	鏡(かがみ): 
\\	望遠鏡(ぼうえんきょう): 
\\	眼鏡(めがね): 
\\	顕微鏡(けんびきょう): 
\\	金: 
\\	竟: 
\\	金	
\\	かね・かねへん	
\\	形声 
\\	関	カン	せき、-ぜき、かか.わる、からくり、かんぬき		
\\	関係(かんけい): 
\\	関心(かんしん): 
\\	関連(かんれん): 
\\	機関(きかん): 
\\	関税(かんぜい): 
\\	玄関(げんかん): 
\\	税関(ぜいかん): 
\\	機関車(きかんしゃ): 
\\	にも関わらず(にもかかわらず): 
\\	関西(かんさい): 
\\	関与(かんよ): 
\\	交通機関(こうつうきかん): 
\\	関する(かんする): 
\\	関東(かんとう): 
\\	門: 
\\	关: 
\\	門	
\\	もん・もんがまえ・かどがまえ	關	会意 
\\	陸	リク、ロク	おか	くが, たち, みち, む, むつ	
\\	大陸(たいりく): 
\\	陸(りく): 
\\	上陸(じょうりく): 
\\	着陸(ちゃくりく): 
\\	内陸(ないりく): 
\\	阜: 
\\	坴: 
\\	阜	
\\	こざと・こざとへん・おか・ぎふのふ	
\\	形声 
\\	隊	タイ			
\\	軍隊(ぐんたい): 
\\	兵隊(へいたい): 
\\	隊(たい): 
\\	阜: 
\\	豕: 
\\	阜	
\\	こざと・こざとへん・おか・ぎふのふ	
\\	会意 
\\	静	セイ、ジョウ	しず-、しず.か、しず.まる、しず.める	しづ	
\\	静か(しずか): 
\\	冷静(れいせい): 
\\	静止(せいし): 
\\	安静(あんせい): 
\\	静的(せいてき): 
\\	青: 
\\	争: 
\\	青	
\\	あお・あおへん	靜	会意 
\\	順	ジュン		あや, あり, おき, おさむ, しげ, したがう, とし, なお, のぶ, のり, まさ, むね, もと, ゆき, よし, より	
\\	順調(じゅんちょう): 
\\	順(じゅん): 
\\	手順(てじゅん): 
\\	順番(じゅんばん): 
\\	順序(じゅんじょ): 
\\	不順(ふじゅん): 
\\	道順(みちじゅん): 
\\	順々(じゅんじゅん): 
\\	巛: 
\\	頁: 
\\	頁	
\\	おおがい・いちのかい	
\\	形声 
\\	願	ガン	ねが.う、-ねがい	ら	
\\	願い(ねがい): 
\\	念願(ねんがん): 
\\	願書(がんしょ): 
\\	願う(ねがう): 
\\	お願いします(おねがいします): 
\\	原: 
\\	頁: 
\\	頁	
\\	おおがい・いちのかい	
\\	形声 
\\	類	ルイ	たぐ.い		
\\	人類(じんるい): 
\\	種類(しゅるい): 
\\	書類(しょるい): 
\\	親類(しんるい): 
\\	衣類(いるい): 
\\	分類(ぶんるい): 
\\	類似(るいじ): 
\\	類推(るいすい): 
\\	類(るい): 
\\	類い(たぐい): 
\\	米: 
\\	大: 
\\	頁: 
\\	頁	
\\	おおがい・いちのかい	類	会意 
\\	飛	ヒ	と.ぶ、と.ばす、-と.ばす	あす, とび	
\\	飛行(ひこう): 
\\	飛行機(ひこうき): 
\\	飛び出す(とびだす): 
\\	飛行場(ひこうじょう): 
\\	飛ぶ(とぶ): 
\\	飛ばす(とばす): 
\\	飛び込む(とびこむ): 
\\	蹴飛ばす(けとばす): 
\\	升: 
\\	飛	
\\	とぶ	
\\	象形 
\\	飯	ハン	めし	い, いい, いり, え	
\\	飯(めし): 
\\	朝ご飯(あさごはん): 
\\	夕飯(ゆうめし): 
\\	ご飯(ごはん): 
\\	昼飯(ひるめし): 
\\	晩御飯(ばんごはん): 
\\	昼ご飯(ひるごはん): 
\\	食: 
\\	反: 
\\	食	
\\	しょく・しょくへん	
\\	形声 
\\	養	ヨウ、リョウ	やしな.う	や	
\\	栄養(えいよう): 
\\	教養(きょうよう): 
\\	休養(きゅうよう): 
\\	扶養(ふよう): 
\\	保養(ほよう): 
\\	養う(やしなう): 
\\	養護(ようご): 
\\	養分(ようぶん): 
\\	羊: 
\\	食: 
\\	食	
\\	しょく・しょくへん	
\\	形声 
\\	験	ケン、ゲン	あかし、しるし、ため.す、ためし		
\\	経験(けいけん): 
\\	試験(しけん): 
\\	実験(じっけん): 
\\	受験(じゅけん): 
\\	体験(たいけん): 
\\	馬: 
\\	僉: 
\\	馬	
\\	うま・うまへん	驗	形声 
\\	久	キュウ、ク	ひさ.しい	きゅ, わ	
\\	久し振り(ひさしぶり): 
\\	永久(えいきゅう): 
\\	久しい(ひさしい): 
\\	勹: 
\\	丿	
\\	の・のかんむり・はらいぼう	
\\	象形 
\\	仏	ブツ、フツ	ほとけ		
\\	仏像(ぶつぞう): 
\\	仏(ほとけ): 
\\	仏(ふつ): 
\\	人: 
\\	厶: 
\\	人	
\\	ひと・にんべん・ひとやね	佛	形声 
\\	仮	カ、ケ	かり、かり-		
\\	仮に(かりに): 
\\	片仮名(カタカナ): 
\\	仮定(かてい): 
\\	平仮名(ひらがな): 
\\	仮令(たとえ): 
\\	仮名遣い(かなづかい): 
\\	振り仮名(ふりがな): 
\\	送り仮名(おくりがな): 
\\	仮名(かな): 
\\	人: 
\\	反: 
\\	人	
\\	ひと・にんべん・ひとやね	假	形声 
\\	件	ケン	くだん		
\\	事件(じけん): 
\\	条件(じょうけん): 
\\	件(けん): 
\\	用件(ようけん): 
\\	件(くだん): 
\\	人: 
\\	牛: 
\\	人	
\\	ひと・にんべん・ひとやね	
\\	会意 
\\	任	ニン	まか.せる、まか.す	さ, とう, ひで	
\\	責任(せきにん): 
\\	就任(しゅうにん): 
\\	主任(しゅにん): 
\\	任務(にんむ): 
\\	赴任(ふにん): 
\\	任命(にんめい): 
\\	信任(しんにん): 
\\	任す(まかす): 
\\	任せる(まかせる): 
\\	転任(てんにん): 
\\	人: 
\\	壬: 
\\	人	
\\	ひと・にんべん・ひとやね	
\\	形声 
\\	似	ジ	に.る、ひ.る	にた	
\\	類似(るいじ): 
\\	似通う(にかよう): 
\\	似合う(にあう): 
\\	似る(にる): 
\\	真似(まね): 
\\	真似る(まねる): 
\\	人: 
\\	以: 
\\	人	
\\	ひと・にんべん・ひとやね	
\\	形声 
\\	余	ヨ	あま.る、あま.り、あま.す、あんま.り	あまる	
\\	余り(あんまり): 
\\	余裕(よゆう): 
\\	余地(よち): 
\\	余暇(よか): 
\\	余計(よけい): 
\\	余る(あまる): 
\\	余程(よほど): 
\\	余分(よぶん): 
\\	余所(よそ): 
\\	余興(よきょう): 
\\	人: 
\\	小: 
\\	食	
\\	しょく・しょくへん	餘	形声 
\\	価	カ、ケ	あたい		
\\	価格(かかく): 
\\	評価(ひょうか): 
\\	価値(かち): 
\\	物価(ぶっか): 
\\	定価(ていか): 
\\	高価(こうか): 
\\	人: 
\\	襾: 
\\	人	
\\	ひと・にんべん・ひとやね	價	形声 
\\	保	ホ、ホウ	たも.つ	う, お, ぶ, もり, やす, やすし	
\\	保障(ほしょう): 
\\	保護(ほご): 
\\	確保(かくほ): 
\\	保険(ほけん): 
\\	保守(ほしゅ): 
\\	保健(ほけん): 
\\	保存(ほぞん): 
\\	保管(ほかん): 
\\	保つ(たもつ): 
\\	保温(ほおん): 
\\	保育(ほいく): 
\\	保養(ほよう): 
\\	保母(ほぼ): 
\\	人: 
\\	呆: 
\\	人	
\\	ひと・にんべん・ひとやね	
\\	会意 
\\	修	シュウ、シュ	おさ.める、おさ.まる	おき, なが, のぶ, おさむ	
\\	修正(しゅうせい): 
\\	修理(しゅうり): 
\\	修了(しゅうりょう): 
\\	修業(しゅうぎょう): 
\\	修士(しゅうし): 
\\	修繕(しゅうぜん): 
\\	必修(ひっしゅう): 
\\	研修(けんしゅう): 
\\	改修(かいしゅう): 
\\	専修(せんしゅう): 
\\	修学(しゅうがく): 
\\	修飾(しゅうしょく): 
\\	攸: 
\\	彡: 
\\	人	
\\	ひと・にんべん・ひとやね	
\\	会意 
\\	俵	ヒョウ	たわら		
\\	土俵(どひょう): 
\\	人: 
\\	表: 
\\	人	
\\	ひと・にんべん・ひとやね	
\\	形声 
\\	個	コ、カ			
\\	個人(こじん): 
\\	個性(こせい): 
\\	個々(ここ): 
\\	個別(こべつ): 
\\	個体(こたい): 
\\	個(こ): 
\\	人: 
\\	固: 
\\	人	
\\	ひと・にんべん・ひとやね	
\\	形声 
\\	備	ビ	そな.える、そな.わる、つぶさ.に	びっ, びん	
\\	備える(そなえる): 
\\	準備(じゅんび): 
\\	整備(せいび): 
\\	設備(せつび): 
\\	守備(しゅび): 
\\	予備(よび): 
\\	軍備(ぐんび): 
\\	警備(けいび): 
\\	装備(そうび): 
\\	備え付ける(そなえつける): 
\\	人: 
\\	艸: 
\\	厂: 
\\	用: 
\\	人	
\\	ひと・にんべん・ひとやね	
\\	形声 
\\	像	ゾウ		かた	
\\	映像(えいぞう): 
\\	想像(そうぞう): 
\\	仏像(ぶつぞう): 
\\	像(ぞう): 
\\	現像(げんぞう): 
\\	人: 
\\	象: 
\\	人	
\\	ひと・にんべん・ひとやね	
\\	形声 
\\	再	サイ、サ	ふたた.び	ふた	
\\	再び(ふたたび): 
\\	再建(さいけん): 
\\	再生(さいせい): 
\\	再三(さいさん): 
\\	再会(さいかい): 
\\	再発(さいはつ): 
\\	再現(さいげん): 
\\	再来年(さらいねん): 
\\	再(さい): 
\\	再来月(さらいげつ): 
\\	再来週(さらいしゅう): 
\\	一: 
\\	冉: 
\\	冂	
\\	けいがまえ・まきがまえ・どうがまえ・えんがまえ	
\\	象形 
\\	刊	カン			
\\	創刊(そうかん): 
\\	刊行(かんこう): 
\\	夕刊(ゆうかん): 
\\	季刊(きかん): 
\\	干: 
\\	刀: 
\\	刀	
\\	かたな・りっとう	
\\	形声 
\\	判	ハン、バン	わか.る		
\\	裁判(さいばん): 
\\	判決(はんけつ): 
\\	判断(はんだん): 
\\	批判(ひはん): 
\\	審判(しんぱん): 
\\	判定(はんてい): 
\\	評判(ひょうばん): 
\\	判事(はんじ): 
\\	判(はん): 
\\	判子(はんこ): 
\\	半: 
\\	刀: 
\\	刀	
\\	かたな・りっとう	
\\	形声 
\\	制	セイ			
\\	規制(きせい): 
\\	制度(せいど): 
\\	体制(たいせい): 
\\	制裁(せいさい): 
\\	強制(きょうせい): 
\\	制限(せいげん): 
\\	制作(せいさく): 
\\	制定(せいてい): 
\\	抑制(よくせい): 
\\	制約(せいやく): 
\\	制服(せいふく): 
\\	統制(とうせい): 
\\	制(せい): 
\\	制する(せいする): 
\\	専制(せんせい): 
\\	牛: 
\\	巾: 
\\	刀: 
\\	刀	
\\	かたな・りっとう	
\\	会意 
\\	券	ケン			
\\	旅券(りょけん): 
\\	券(けん): 
\\	定期券(ていきけん): 
\\	回数券(かいすうけん): 
\\	𠔉: 
\\	刀: 
\\	刀	
\\	かたな・りっとう	
\\	会意 
\\	則	ソク	のっと.る	のり	
\\	原則(げんそく): 
\\	規則(きそく): 
\\	法則(ほうそく): 
\\	不規則(ふきそく): 
\\	貝: 
\\	刀: 
\\	刀	
\\	かたな・りっとう	
\\	会意 
\\	効	コウ	き.く、ききめ、なら.う		
\\	効果(こうか): 
\\	有効(ゆうこう): 
\\	効率(こうりつ): 
\\	無効(むこう): 
\\	効力(こうりょく): 
\\	効く(きく): 
\\	交: 
\\	力: 
\\	力	
\\	ちから・りきづくり	效	会意 
\\	務	ム	つと.める	つかさ, み	
\\	務め(つとめ): 
\\	事務(じむ): 
\\	事務所(じむしょ): 
\\	義務(ぎむ): 
\\	勤務(きんむ): 
\\	公務員(こうむいん): 
\\	任務(にんむ): 
\\	職務(しょくむ): 
\\	公務(こうむ): 
\\	税務署(ぜいむしょ): 
\\	業務(ぎょうむ): 
\\	責務(せきむ): 
\\	庶務(しょむ): 
\\	矛: 
\\	攵: 
\\	力: 
\\	力	
\\	ちから・りきづくり	
\\	形声 
\\	勢	セイ、ゼイ	いきお.い、はずみ	せ	
\\	姿勢(しせい): 
\\	情勢(じょうせい): 
\\	勢力(せいりょく): 
\\	勢い(いきおい): 
\\	態勢(たいせい): 
\\	大勢(おおぜい): 
\\	優勢(ゆうせい): 
\\	形勢(けいせい): 
\\	土: 
\\	八: 
\\	土: 
\\	丸: 
\\	力: 
\\	力	
\\	ちから・りきづくり	
\\	会意 
\\	厚	コウ	あつ.い、あか	あ, あっ	
\\	厚い(あつい): 
\\	厚かましい(あつかましい): 
\\	厂: 
\\	日: 
\\	子: 
\\	厂	
\\	がんだれ	
\\	会意 
\\	句	ク		すく	
\\	俳句(はいく): 
\\	文句(もんく): 
\\	句(く): 
\\	句読点(くとうてん): 
\\	語句(ごく): 
\\	佳句(かく): 
\\	勹: 
\\	口: 
\\	口	
\\	くち・くちへん	
\\	会意 
\\	可	カ、コク	べ.き、-べ.し	よし	
\\	可能(かのう): 
\\	許可(きょか): 
\\	可(か): 
\\	不可欠(ふかけつ): 
\\	可決(かけつ): 
\\	不可(ふか): 
\\	可哀相(かわいそう): 
\\	可笑しい(おかしい): 
\\	可愛い(かわいい): 
\\	可愛がる(かわいがる): 
\\	可愛らしい(かわいらしい): 
\\	可也(かなり): 
\\	丁: 
\\	口: 
\\	口	
\\	くち・くちへん	
\\	会意 
\\	営	エイ	いとな.む、いとな.み		
\\	経営(けいえい): 
\\	運営(うんえい): 
\\	営業(えいぎょう): 
\\	営む(いとなむ): 
\\	小: 
\\	冖: 
\\	呂: 
\\	火	
\\	ひ・ひへん・れっか・れんが	營	形声 
\\	因	イン	よ.る、ちな.む	て	
\\	原因(げんいん): 
\\	要因(よういん): 
\\	因る(よる): 
\\	因って(よって): 
\\	囗: 
\\	大: 
\\	囗	
\\	くに・くにがまえ	
\\	会意 
\\	団	ダン、トン	かたまり、まる.い		
\\	団体(だんたい): 
\\	集団(しゅうだん): 
\\	公団(こうだん): 
\\	団地(だんち): 
\\	団結(だんけつ): 
\\	布団(ふとん): 
\\	劇団(げきだん): 
\\	座布団(ざぶとん): 
\\	団扇(うちわ): 
\\	団(だん): 
\\	囗: 
\\	寸: 
\\	囗	
\\	くに・くにがまえ	團	形声 
\\	圧	アツ、エン、オウ	お.す、へ.す、おさ.える、お.さえる		
\\	圧力(あつりょく): 
\\	気圧(きあつ): 
\\	血圧(けつあつ): 
\\	圧縮(あっしゅく): 
\\	圧倒(あっとう): 
\\	抑圧(よくあつ): 
\\	圧迫(あっぱく): 
\\	厂: 
\\	土: 
\\	土	
\\	つち・つちへん・どへん	壓	会意 
\\	在	ザイ	あ.る	あり	
\\	現在(げんざい): 
\\	存在(そんざい): 
\\	滞在(たいざい): 
\\	不在(ふざい): 
\\	在学(ざいがく): 
\\	在庫(ざいこ): 
\\	所在(しょざい): 
\\	自在(じざい): 
\\	健在(けんざい): 
\\	人: 
\\	土: 
\\	土	
\\	つち・つちへん・どへん	
\\	会意 
\\	均	キン	なら.す	ひとし	
\\	平均(へいきん): 
\\	均衡(きんこう): 
\\	土: 
\\	匀: 
\\	土	
\\	つち・つちへん・どへん	
\\	形声 
\\	基	キ	もと、もとい	きい, とも	
\\	基準(きじゅん): 
\\	基本(きほん): 
\\	基礎(きそ): 
\\	基地(きち): 
\\	基盤(きばん): 
\\	基金(ききん): 
\\	基づく(もとづく): 
\\	基(もとい): 
\\	其: 
\\	土: 
\\	土	
\\	つち・つちへん・どへん	
\\	形声 
\\	報	ホウ	むく.いる		
\\	情報(じょうほう): 
\\	報告(ほうこく): 
\\	報道(ほうどう): 
\\	報酬(ほうしゅう): 
\\	予報(よほう): 
\\	電報(でんぽう): 
\\	報じる(ほうじる): 
\\	天気予報(てんきよほう): 
\\	報ずる(ほうずる): 
\\	幸: 
\\	卩: 
\\	又: 
\\	土	
\\	つち・つちへん・どへん	
\\	会意 
\\	境	キョウ、ケイ	さかい	さか, じき	
\\	環境(かんきょう): 
\\	国境(こっきょう): 
\\	境(さかい): 
\\	境界(きょうかい): 
\\	境遇(きょうぐう): 
\\	土: 
\\	竟: 
\\	土	
\\	つち・つちへん・どへん	
\\	形声 
\\	墓	ボ	はか		
\\	墓(はか): 
\\	墓地(ぼち): 
\\	莫: 
\\	土: 
\\	土	
\\	つち・つちへん・どへん	
\\	形声 
\\	増	ゾウ	ま.す、ま.し、ふ.える、ふ.やす	まし, ます	
\\	増加(ぞうか): 
\\	増大(ぞうだい): 
\\	増し(まし): 
\\	増減(ぞうげん): 
\\	増強(ぞうきょう): 
\\	増進(ぞうしん): 
\\	増す(ます): 
\\	激増(げきぞう): 
\\	土: 
\\	曽: 
\\	土	
\\	つち・つちへん・どへん	增	形声 
\\	夢	ム、ボウ	ゆめ、ゆめ.みる、くら.い		
\\	夢(ゆめ): 
\\	夢中(むちゅう): 
\\	艸: 
\\	网: 
\\	冖: 
\\	夕: 
\\	夕	
\\	ゆう・ゆうべ・た	
\\	会意 
\\	妻	サイ	つま	ずま, め	
\\	妻(つま): 
\\	夫妻(ふさい): 
\\	彑: 
\\	女: 
\\	女	
\\	おんな・おんなへん	
\\	象形 
\\	婦	フ	よめ	ね	
\\	主婦(しゅふ): 
\\	夫婦(ふうふ): 
\\	看護婦(かんごふ): 
\\	婦人(ふじん): 
\\	産婦人科(さんふじんか): 
\\	女: 
\\	帚: 
\\	女	
\\	おんな・おんなへん	
\\	形声 
\\	容	ヨウ	い.れる	かた, ひろ, まさ	
\\	内容(ないよう): 
\\	容易(ようい): 
\\	容器(ようき): 
\\	収容(しゅうよう): 
\\	美容(びよう): 
\\	許容(きょよう): 
\\	容積(ようせき): 
\\	寛容(かんよう): 
\\	形容詞(けいようし): 
\\	容易い(たやすい): 
\\	形容動詞(けいようどうし): 
\\	宀: 
\\	谷: 
\\	宀	
\\	うかんむり	
\\	会意 
\\	寄	キ	よ.る、-よ.り、よ.せる	よせ, より, よろ	
\\	寄附(きふ): 
\\	年寄り(としより): 
\\	寄せる(よせる): 
\\	寄り掛かる(よりかかる): 
\\	寄贈(きぞう): 
\\	立ち寄る(たちよる): 
\\	片寄る(かたよる): 
\\	寄与(きよ): 
\\	取り寄せる(とりよせる): 
\\	近寄る(ちかよる): 
\\	寄る(よる): 
\\	寄越す(よこす): 
\\	押し寄せる(おしよせる): 
\\	寄り(より): 
\\	宀: 
\\	奇: 
\\	宀	
\\	うかんむり	
\\	形声 
\\	富	フ、フウ	と.む、とみ	と, とん, ふっ	
\\	豊富(ほうふ): 
\\	富(とみ): 
\\	富む(とむ): 
\\	富豪(ふごう): 
\\	宀: 
\\	畐: 
\\	宀	
\\	うかんむり	
\\	形声 
\\	導	ドウ	みちび.く	みち	
\\	指導(しどう): 
\\	導入(どうにゅう): 
\\	誘導(ゆうどう): 
\\	主導(しゅどう): 
\\	導く(みちびく): 
\\	道: 
\\	寸: 
\\	寸	
\\	すん・すんづくり	
\\	形声 
\\	居	キョ、コ	い.る、-い、お.る	おき, ぐ, すえ	
\\	芝居(しばい): 
\\	居住(きょじゅう): 
\\	居間(いま): 
\\	同居(どうきょ): 
\\	居る(おる): 
\\	鳥居(とりい): 
\\	転居(てんきょ): 
\\	皇居(こうきょ): 
\\	居眠り(いねむり): 
\\	居る(いる): 
\\	隠居(いんきょ): 
\\	尸: 
\\	古: 
\\	尸	
\\	しかばね・かばね・かばねだれ	
\\	会意 
\\	属	ゾク、ショク	さかん、つく、やから	さっか, つき	
\\	附属(ふぞく): 
\\	金属(きんぞく): 
\\	所属(しょぞく): 
\\	属する(ぞくする): 
\\	属(ぞく): 
\\	尸: 
\\	禹: 
\\	尸	
\\	しかばね・かばね・かばねだれ	屬	会意 
\\	布	フ	ぬの	う, の, ほ	
\\	布(ぬの): 
\\	財布(さいふ): 
\\	布団(ふとん): 
\\	配布(はいふ): 
\\	分布(ぶんぷ): 
\\	毛布(もうふ): 
\\	座布団(ざぶとん): 
\\	布告(ふこく): 
\\	布巾(ふきん): 
\\	巾: 
\\	巾	
\\	はば・はばへん・きんべん	
\\	形声 
\\	師	シ	いくさ	のし, のり, もろ, かず, つかさ, みつ	
\\	医師(いし): 
\\	教師(きょうし): 
\\	講師(こうし): 
\\	技師(ぎし): 
\\	漁師(りょうし): 
\\	牧師(ぼくし): 
\\	看護師(かんごし): 
\\	丿: 
\\	丨: 
\\	巾: 
\\	巾	
\\	はば・はばへん・きんべん	
\\	会意 
\\	常	ジョウ	つね、とこ-	とき, のぶ, ひ, ひた	
\\	正常(せいじょう): 
\\	通常(つうじょう): 
\\	非常(ひじょう): 
\\	日常(にちじょう): 
\\	常に(つねに): 
\\	異常(いじょう): 
\\	常識(じょうしき): 
\\	平常(へいじょう): 
\\	非常に(ひじょうに): 
\\	尚: 
\\	吊: 
\\	巾	
\\	はば・はばへん・きんべん	
\\	形声 
\\	幹	カン	みき	つよし, まさ, もと, えだ, き, くる, たかし, つね, とも, み, もとき, よし, より	
\\	幹部(かんぶ): 
\\	新幹線(しんかんせん): 
\\	幹(みき): 
\\	幹線(かんせん): 
\\	十: 
\\	早: 
\\	人: 
\\	干: 
\\	干	
\\	かん・いちじゅう・ほす・ひる・たてかん	
\\	形声 
\\	序	ジョ	つい.で、ついで	つぐ	
\\	秩序(ちつじょ): 
\\	序で(ついで): 
\\	順序(じゅんじょ): 
\\	广: 
\\	予: 
\\	广	
\\	まだれ	
\\	形声 
\\	弁	ベン、ヘン	かんむり、わきま.える、わ.ける、はなびら、あらそ.う	べ	
\\	弁護(べんご): 
\\	弁当(べんとう): 
\\	弁論(べんろん): 
\\	花弁(はなびら): 
\\	弁解(べんかい): 
\\	弁(べん): 
\\	代弁(だいべん): 
\\	弁償(べんしょう): 
\\	勘弁(かんべん): 
\\	お弁当(おべんとう): 
\\	厶: 
\\	廾: 
\\	廾	
\\	にじゅう・にじゅうあし・こまぬき	辨 瓣 辯	象形 
\\	張	チョウ	は.る、-は.り、-ば.り	はり, わり	
\\	主張(しゅちょう): 
\\	緊張(きんちょう): 
\\	出張(しゅっちょう): 
\\	拡張(かくちょう): 
\\	引っ張る(ひっぱる): 
\\	頑張る(がんばる): 
\\	矢っ張り(やっぱり): 
\\	誇張(こちょう): 
\\	張り紙(はりがみ): 
\\	突っ張る(つっぱる): 
\\	張り切る(はりきる): 
\\	欲張り(よくばり): 
\\	威張る(いばる): 
\\	嵩張る(かさばる): 
\\	弓: 
\\	長: 
\\	弓	
\\	ゆみ・ゆみへん	
\\	形声 
\\	往	オウ	い.く、いにしえ、さき.に、ゆ.く	みち	
\\	往復(おうふく): 
\\	往診(おうしん): 
\\	彳: 
\\	主: 
\\	彳	
\\	ぎょうにんべん	
\\	形声 
\\	復	フク	また		
\\	回復(かいふく): 
\\	復活(ふっかつ): 
\\	往復(おうふく): 
\\	復興(ふっこう): 
\\	復旧(ふっきゅう): 
\\	復習(ふくしゅう): 
\\	彳: 
\\	复: 
\\	彳	
\\	ぎょうにんべん	
\\	形声 
\\	徳	トク		あつ, なる, のり, ゆき, よし	
\\	道徳(どうとく): 
\\	彳: 
\\	十: 
\\	网: 
\\	心: 
\\	彳	
\\	ぎょうにんべん	德	会意 
\\	志	シ	シリング、こころざ.す、こころざし	じん, べ, べし, ゆき	
\\	志(こころざし): 
\\	同志(どうし): 
\\	意志(いし): 
\\	志向(しこう): 
\\	志望(しぼう): 
\\	志す(こころざす): 
\\	土: 
\\	心: 
\\	心	
\\	こころ・したごころ・りっしんべん	
\\	形声 
\\	応	オウ、ヨウ、-ノウ	あた.る、まさに、こた.える	お, たか, まさ	
\\	対応(たいおう): 
\\	応援(おうえん): 
\\	応募(おうぼ): 
\\	反応(はんのう): 
\\	応じる(おうじる): 
\\	一応(いちおう): 
\\	応用(おうよう): 
\\	適応(てきおう): 
\\	応接(おうせつ): 
\\	応対(おうたい): 
\\	相応(そうおう): 
\\	応急(おうきゅう): 
\\	応ずる(おうずる): 
\\	相応しい(ふさわしい): 
\\	广: 
\\	心: 
\\	心	
\\	こころ・したごころ・りっしんべん	應	形声 
\\	快	カイ	こころよ.い	よし	
\\	軽快(けいかい): 
\\	愉快(ゆかい): 
\\	快い(こころよい): 
\\	快晴(かいせい): 
\\	快適(かいてき): 
\\	全快(ぜんかい): 
\\	心: 
\\	夬: 
\\	心	
\\	こころ・したごころ・りっしんべん	
\\	形声 
\\	性	セイ、ショウ	さが		
\\	女性(じょせい): 
\\	男性(だんせい): 
\\	性格(せいかく): 
\\	個性(こせい): 
\\	酸性(さんせい): 
\\	性能(せいのう): 
\\	性別(せいべつ): 
\\	性質(せいしつ): 
\\	異性(いせい): 
\\	理性(りせい): 
\\	知性(ちせい): 
\\	適性(てきせい): 
\\	中性(ちゅうせい): 
\\	性(せい): 
\\	心: 
\\	生: 
\\	心	
\\	こころ・したごころ・りっしんべん	
\\	形声 
\\	恩	オン		めぐみ	
\\	恩恵(おんけい): 
\\	恩(おん): 
\\	因: 
\\	心: 
\\	心	
\\	こころ・したごころ・りっしんべん	
\\	形声 
\\	情	ジョウ、セイ	なさ.け		
\\	情緒(じょうしょ): 
\\	事情(じじょう): 
\\	情勢(じょうせい): 
\\	情報(じょうほう): 
\\	表情(ひょうじょう): 
\\	感情(かんじょう): 
\\	実情(じつじょう): 
\\	苦情(くじょう): 
\\	情熱(じょうねつ): 
\\	愛情(あいじょう): 
\\	情(じょう): 
\\	同情(どうじょう): 
\\	友情(ゆうじょう): 
\\	人情(にんじょう): 
\\	心情(しんじょう): 
\\	情け(なさけ): 
\\	純情(じゅんじょう): 
\\	情けない(なさけない): 
\\	情け深い(なさけぶかい): 
\\	心: 
\\	青: 
\\	心	
\\	こころ・したごころ・りっしんべん	
\\	形声 
\\	態	タイ	わざ.と		
\\	重態(じゅうたい): 
\\	状態(じょうたい): 
\\	事態(じたい): 
\\	実態(じったい): 
\\	態度(たいど): 
\\	態勢(たいせい): 
\\	形態(けいたい): 
\\	態と(わざと): 
\\	態々(わざわざ): 
\\	能: 
\\	心: 
\\	心	
\\	こころ・したごころ・りっしんべん	
\\	形声 
\\	慣	カン	な.れる、な.らす		
\\	習慣(しゅうかん): 
\\	慣習(かんしゅう): 
\\	慣れ(なれ): 
\\	慣行(かんこう): 
\\	慣例(かんれい): 
\\	慣らす(ならす): 
\\	慣れる(なれる): 
\\	見慣れる(みなれる): 
\\	心: 
\\	貫: 
\\	心	
\\	こころ・したごころ・りっしんべん	
\\	形声 
\\	承	ショウ、ジョウ	うけたまわ.る、う.ける、ささ.げる、とど.める、こ.らす、つい.で、すく.う	つぐ, こと, すけ, つぎ, よし	
\\	"承認(しょうにん): 
\\	了承(りょうしょう): 
\\	承知(しょうち): 
\\	承諾(しょうだく): 
\\	承る(うけたまわる): 
\\	了: 
\\	三: 
\\	水: 
\\	手	
\\	て・てへん	
\\	会意 
\\	技	ギ	わざ		
\\	技術(ぎじゅつ): 
\\	競技(きょうぎ): 
\\	演技(えんぎ): 
\\	技(わざ): 
\\	技能(ぎのう): 
\\	技師(ぎし): 
\\	特技(とくぎ): 
\\	手: 
\\	支: 
\\	手	
\\	て・てへん	
\\	形声 
\\	招	ショウ	まね.く		
\\	招待(しょうたい): 
\\	招き(まねき): 
\\	招く(まねく): 
\\	手: 
\\	召: 
\\	手	
\\	て・てへん	
\\	形声 
\\	授	ジュ	さず.ける、さず.かる		
\\	教授(きょうじゅ): 
\\	授業(じゅぎょう): 
\\	助教授(じょきょうじゅ): 
\\	授ける(さずける): 
\\	手: 
\\	受: 
\\	手	
\\	て・てへん	
\\	形声 
\\	採	サイ	と.る		
\\	採用(さいよう): 
\\	採決(さいけつ): 
\\	採算(さいさん): 
\\	採点(さいてん): 
\\	採集(さいしゅう): 
\\	採択(さいたく): 
\\	採掘(さいくつ): 
\\	採る(とる): 
\\	手: 
\\	采: 
\\	手	
\\	て・てへん	
\\	形声 
\\	接	セツ、ショウ	つ.ぐ		
\\	直接(ちょくせつ): 
\\	接触(せっしょく): 
\\	面接(めんせつ): 
\\	密接(みっせつ): 
\\	接近(せっきん): 
\\	接する(せっする): 
\\	間接(かんせつ): 
\\	応接(おうせつ): 
\\	接続(せつぞく): 
\\	接ぐ(つぐ): 
\\	接続詞(せつぞくし): 
\\	手: 
\\	妾: 
\\	手	
\\	て・てへん	
\\	形声 
\\	提	テイ、チョウ、ダイ	さ.げる		
\\	前提(ぜんてい): 
\\	提案(ていあん): 
\\	提供(ていきょう): 
\\	提出(ていしゅつ): 
\\	提携(ていけい): 
\\	提示(ていじ): 
\\	手: 
\\	是: 
\\	手	
\\	て・てへん	
\\	形声 
\\	損	ソン	そこ.なう、そこな.う、-そこ.なう、そこ.ねる、-そこ.ねる		
\\	損う(そこなう): 
\\	損害(そんがい): 
\\	損失(そんしつ): 
\\	損(そん): 
\\	破損(はそん): 
\\	損得(そんとく): 
\\	手: 
\\	員: 
\\	手	
\\	て・てへん	
\\	会意 
\\	支	シ	ささ.える、つか.える、か.う		
\\	支持(しじ): 
\\	支店(してん): 
\\	支配(しはい): 
\\	収支(しゅうし): 
\\	支払い(しはらい): 
\\	支出(ししゅつ): 
\\	支払う(しはらう): 
\\	支度(したく): 
\\	支給(しきゅう): 
\\	差し支える(さしつかえる): 
\\	差し支え(さしつかえ): 
\\	支える(ささえる): 
\\	十: 
\\	又: 
\\	支	
\\	し・しにょう・えだにょう・じゅうまた	
\\	会意 
\\	政	セイ、ショウ	まつりごと、まん	ただ, まさ	
\\	行政(ぎょうせい): 
\\	財政(ざいせい): 
\\	政権(せいけん): 
\\	政策(せいさく): 
\\	政治(せいじ): 
\\	政党(せいとう): 
\\	政府(せいふ): 
\\	正: 
\\	攵: 
\\	攴	
\\	ぼくにょう・ぼくづくり・とまた・のぶん・しぶん	
\\	形声 
\\	故	コ	ゆえ、ふる.い、もと		
\\	事故(じこ): 
\\	故障(こしょう): 
\\	何故(なぜ): 
\\	故人(こじん): 
\\	何故なら(なぜなら): 
\\	故(こ): 
\\	故(ゆえ): 
\\	其れ故(それゆえ): 
\\	古: 
\\	攵: 
\\	攴	
\\	ぼくにょう・ぼくづくり・とまた・のぶん・しぶん	
\\	会意 
\\	敵	テキ	かたき、あだ、かな.う		
\\	敵(てき): 
\\	匹敵(ひってき): 
\\	素敵(すてき): 
\\	敵わない(かなわない): 
\\	敵(かたき): 
\\	亠: 
\\	冂: 
\\	古: 
\\	攵: 
\\	攴	
\\	ぼくにょう・ぼくづくり・とまた・のぶん・しぶん	
\\	形声 
\\	断	ダン	た.つ、ことわ.る、さだ.める		
\\	判断(はんだん): 
\\	決断(けつだん): 
\\	診断(しんだん): 
\\	横断(おうだん): 
\\	断る(ことわる): 
\\	中断(ちゅうだん): 
\\	無断(むだん): 
\\	断然(だんぜん): 
\\	断水(だんすい): 
\\	断定(だんてい): 
\\	油断(ゆだん): 
\\	断言(だんげん): 
\\	断面(だんめん): 
\\	米: 
\\	斤: 
\\	斤	
\\	おの・おのづくり・きん	斷	会意 
\\	旧	キュウ	ふる.い、もと		
\\	復旧(ふっきゅう): 
\\	旧(きゅう): 
\\	旧知(きゅうち): 
\\	丨: 
\\	日: 
\\	臼	
\\	うす	舊	会意 
\\	易	エキ、イ	やさ.しい、やす.い		
\\	貿易(ぼうえき): 
\\	容易(ようい): 
\\	安易(あんい): 
\\	簡易(かんい): 
\\	交易(こうえき): 
\\	易しい(やさしい): 
\\	易い(やすい): 
\\	容易い(たやすい): 
\\	辟易(へきえき): 
\\	日: 
\\	勿: 
\\	日	
\\	ひ・ひへん・にち・にちへん	
\\	会意 
\\	暴	ボウ、バク	あば.く、あば.れる		
\\	暴力(ぼうりょく): 
\\	暴動(ぼうどう): 
\\	乱暴(らんぼう): 
\\	暴露(ばくろ): 
\\	暴れる(あばれる): 
\\	暴風(ぼうふう): 
\\	日: 
\\	共: 
\\	水: 
\\	日	
\\	ひ・ひへん・にち・にちへん	
\\	会意 
\\	条	ジョウ、チョウ、デキ	えだ、すじ		
\\	条件(じょうけん): 
\\	条約(じょうやく): 
\\	発条(ばね): 
\\	箇条書き(かじょうがき): 
\\	条(じょう): 
\\	夂: 
\\	木: 
\\	木	
\\	き・きへん	條	会意 
\\	枝	シ	えだ	え, ぐさ	
\\	枝(えだ): 
\\	木: 
\\	支: 
\\	木	
\\	き・きへん	
\\	形声 
\\	査	サ			
\\	捜査(そうさ): 
\\	調査(ちょうさ): 
\\	検査(けんさ): 
\\	巡査(じゅんさ): 
\\	審査(しんさ): 
\\	木: 
\\	且: 
\\	木	
\\	き・きへん	
\\	形声 
\\	格	カク、コウ、キャク、ゴウ		いたる, のり	
\\	価格(かかく): 
\\	格差(かくさ): 
\\	資格(しかく): 
\\	性格(せいかく): 
\\	格好(かっこう): 
\\	合格(ごうかく): 
\\	人格(じんかく): 
\\	体格(たいかく): 
\\	失格(しっかく): 
\\	格(かく): 
\\	格別(かくべつ): 
\\	本格(ほんかく): 
\\	規格(きかく): 
\\	同格(どうかく): 
\\	木: 
\\	各: 
\\	木	
\\	き・きへん	
\\	形声 
\\	桜	オウ、ヨウ	さくら	さ, ろう	
\\	桜(さくら): 
\\	木: 
\\	小: 
\\	女: 
\\	木	
\\	き・きへん	櫻	形声 
\\	検	ケン	しら.べる		
\\	検査(けんさ): 
\\	検討(けんとう): 
\\	検事(けんじ): 
\\	点検(てんけん): 
\\	木: 
\\	僉: 
\\	木	
\\	き・きへん	檢	形声 
\\	構	コウ	かま.える、かま.う	とち	
\\	"機構(きこう): 
\\	構造(こうぞう): 
\\	構想(こうそう): 
\\	構成(こうせい): 
\\	結構(けっこう): 
\\	構う(かまう): 
\\	構える(かまえる): 
\\	構いません(かまいません): 
\\	お構いなく(おかまいなく): 
\\	構え(かまえ): 
\\	木: 
\\	冓: 
\\	木	
\\	き・きへん	
\\	形声 
\\	武	ブ、ム	たけ.し	う, お, たけし, たけん, ん	
\\	武器(ぶき): 
\\	武装(ぶそう): 
\\	武力(ぶりょく): 
\\	武士(ぶし): 
\\	一: 
\\	弋: 
\\	止: 
\\	止	
\\	とまる・とめる・とめへん	
\\	会意 
\\	比	ヒ	くら.べる	い, ぴっ	
\\	比べる(くらべる): 
\\	比例(ひれい): 
\\	比率(ひりつ): 
\\	比較(ひかく): 
\\	対比(たいひ): 
\\	比重(ひじゅう): 
\\	比較的(ひかくてき): 
\\	匕: 
\\	比	
\\	ひ・くらべる・ならびひ	
\\	会意 
\\	永	エイ	なが.い	え, と, のり, ひさ, ひさし	
\\	永遠(えいえん): 
\\	永久(えいきゅう): 
\\	丶: 
\\	水: 
\\	水	
\\	みず・したみず・さんずい	
\\	象形 
\\	河	カ	かわ	かっ, こ, こう	
\\	河川(かせん): 
\\	運河(うんが): 
\\	水: 
\\	可: 
\\	水	
\\	みず・したみず・さんずい	
\\	形声 
\\	液	エキ			
\\	血液(けつえき): 
\\	液体(えきたい): 
\\	液(えき): 
\\	溶液(ようえき): 
\\	水: 
\\	夜: 
\\	水	
\\	みず・したみず・さんずい	
\\	形声 
\\	混	コン	ま.じる、-ま.じり、ま.ざる、ま.ぜる、こ.む		
\\	混乱(こんらん): 
\\	混雑(こんざつ): 
\\	混血(こんけつ): 
\\	混ぜる(まぜる): 
\\	混合(こんごう): 
\\	混む(こむ): 
\\	混同(こんどう): 
\\	混凝土(コンクリート): 
\\	取り混ぜる(とりまぜる): 
\\	水: 
\\	昆: 
\\	水	
\\	みず・したみず・さんずい	
\\	形声 
\\	減	ゲン	へ.る、へ.らす		
\\	削減(さくげん): 
\\	減少(げんしょう): 
\\	増減(ぞうげん): 
\\	加減(かげん): 
\\	減らす(へらす): 
\\	減る(へる): 
\\	軽減(けいげん): 
\\	いい加減(いいかげん): 
\\	水: 
\\	咸: 
\\	水	
\\	みず・したみず・さんずい	
\\	形声 
\\	測	ソク	はか.る		
\\	観測(かんそく): 
\\	測定(そくてい): 
\\	測量(そくりょう): 
\\	推測(すいそく): 
\\	予測(よそく): 
\\	水: 
\\	則: 
\\	水	
\\	みず・したみず・さんずい	
\\	形声 
\\	準	ジュン	じゅん.じる、じゅん.ずる、なぞら.える、のり、ひと.しい、みずもり		
\\	基準(きじゅん): 
\\	準備(じゅんび): 
\\	水準(すいじゅん): 
\\	標準(ひょうじゅん): 
\\	準じる(じゅんじる): 
\\	準ずる(じゅんずる): 
\\	準急(じゅんきゅう): 
\\	水: 
\\	隼: 
\\	水	
\\	みず・したみず・さんずい	
\\	形声 
\\	演	エン			
\\	演説(えんぜつ): 
\\	公演(こうえん): 
\\	講演(こうえん): 
\\	演出(えんしゅつ): 
\\	出演(しゅつえん): 
\\	演奏(えんそう): 
\\	演劇(えんげき): 
\\	演技(えんぎ): 
\\	演習(えんしゅう): 
\\	主演(しゅえん): 
\\	上演(じょうえん): 
\\	演ずる(えんずる): 
\\	演じる(えんじる): 
\\	水: 
\\	寅: 
\\	水	
\\	みず・したみず・さんずい	
\\	形声 
\\	潔	ケツ	いさぎよ.い	きよ, きよし	
\\	清潔(せいけつ): 
\\	不潔(ふけつ): 
\\	簡潔(かんけつ): 
\\	水: 
\\	絜: 
\\	水	
\\	みず・したみず・さんずい	
\\	形声 
\\	災	サイ	わざわ.い		
\\	災害(さいがい): 
\\	火災(かさい): 
\\	戦災(せんさい): 
\\	天災(てんさい): 
\\	災難(さいなん): 
\\	巛: 
\\	火: 
\\	火	
\\	ひ・ひへん・れっか・れんが	
\\	会意 
\\	燃	ネン	も.える、も.やす、も.す		
\\	燃料(ねんりょう): 
\\	燃焼(ねんしょう): 
\\	燃える(もえる): 
\\	燃す(もす): 
\\	火: 
\\	然: 
\\	火	
\\	ひ・ひへん・れっか・れんが	
\\	形声 
\\	版	ハン			
\\	出版(しゅっぱん): 
\\	版画(はんが): 
\\	版(はん): 
\\	初版(しょはん): 
\\	絶版(ぜっぱん): 
\\	片: 
\\	反: 
\\	片	
\\	かた・かたへん	
\\	形声 
\\	犯	ハン、ボン	おか.す		
\\	犯罪(はんざい): 
\\	犯人(はんにん): 
\\	防犯(ぼうはん): 
\\	犯す(おかす): 
\\	犬: 
\\	卩: 
\\	犬	
\\	いぬ・けものへん	
\\	形声 
\\	状	ジョウ			
\\	状況(じょうきょう): 
\\	状態(じょうたい): 
\\	現状(げんじょう): 
\\	症状(しょうじょう): 
\\	白状(はくじょう): 
\\	状(じょう): 
\\	爿: 
\\	犬: 
\\	犬	
\\	いぬ・けものへん	狀	形声 
\\	独	ドク、トク	ひと.り	どいつ, どっ	
\\	独自(どくじ): 
\\	独立(どくりつ): 
\\	独特(どくとく): 
\\	独裁(どくさい): 
\\	独占(どくせん): 
\\	独身(どくしん): 
\\	孤独(こどく): 
\\	単独(たんどく): 
\\	独創(どくそう): 
\\	独り言(ひとりごと): 
\\	独りでに(ひとりでに): 
\\	犬: 
\\	虫: 
\\	犬	
\\	いぬ・けものへん	獨	形声 
\\	率	ソツ、リツ、シュツ	ひき.いる		
\\	比率(ひりつ): 
\\	効率(こうりつ): 
\\	率直(そっちょく): 
\\	確率(かくりつ): 
\\	倍率(ばいりつ): 
\\	能率(のうりつ): 
\\	軽率(けいそつ): 
\\	率いる(ひきいる): 
\\	率(りつ): 
\\	統率(とうそつ): 
\\	率土(そっと): 
\\	玄: 
\\	冫: 
\\	十: 
\\	玄	
\\	げん	
\\	象形 
\\	現	ゲン	あらわ.れる、あらわ.す、うつつ、うつ.つ	あきら, きら	
\\	現われ(あらわれ): 
\\	現わす(あらわす): 
\\	現れる(あらわれる): 
\\	現在(げんざい): 
\\	現実(げんじつ): 
\\	現場(げんば): 
\\	現地(げんち): 
\\	現状(げんじょう): 
\\	現代(げんだい): 
\\	実現(じつげん): 
\\	表現(ひょうげん): 
\\	現金(げんきん): 
\\	現象(げんしょう): 
\\	出現(しゅつげん): 
\\	現に(げんに): 
\\	現行(げんこう): 
\\	再現(さいげん): 
\\	現像(げんぞう): 
\\	現(げん): 
\\	玉: 
\\	見: 
\\	玉	
\\	たま・たまへん ・ おう・おうへん・あうさま	
\\	形声 
\\	留	リュウ、ル	と.める、と.まる、とど.める、とど.まる、るうぶる	とめ	
\\	留まる(とまる): 
\\	留学(りゅうがく): 
\\	留学生(りゅうがくせい): 
\\	留守(るす): 
\\	書留(かきとめ): 
\\	留守番(るすばん): 
\\	停留所(ていりゅうじょ): 
\\	留める(とどめる): 
\\	歌留多(かるた): 
\\	蒸留(じょうりゅう): 
\\	卯: 
\\	田: 
\\	田	
\\	た・たへん	
\\	会意 
\\	略	リャク	ほぼ、おか.す、おさ.める、はかりごと、はか.る、はぶ.く、りゃく.す、りゃく.する		
\\	侵略(しんりゃく): 
\\	略(ほぼ): 
\\	略奪(りゃくだつ): 
\\	省略(しょうりゃく): 
\\	略語(りゃくご): 
\\	略す(りゃくす): 
\\	概略(がいりゃく): 
\\	田: 
\\	各: 
\\	田	
\\	た・たへん	
\\	形声 
\\	益	エキ、ヤク	ま.す	まし, ます	
\\	利益(りえき): 
\\	有益(ゆうえき): 
\\	収益(しゅうえき): 
\\	益々(ますます): 
\\	八: 
\\	皿: 
\\	皿	
\\	さら	
\\	会意 
\\	眼	ガン、ゲン	まなこ、め		
\\	眼鏡(めがね): 
\\	眼科(がんか): 
\\	近眼(きんがん): 
\\	眼球(がんきゅう): 
\\	目: 
\\	艮: 
\\	目	
\\	め・めへん・よこめ	
\\	形声 
\\	破	ハ	やぶ.る、やぶ.れる、わ.れる		
\\	破壊(はかい): 
\\	突破(とっぱ): 
\\	破産(はさん): 
\\	破片(はへん): 
\\	破る(やぶる): 
\\	破裂(はれつ): 
\\	破損(はそん): 
\\	破棄(はき): 
\\	爆破(ばくは): 
\\	破れる(やぶれる): 
\\	破く(やぶく): 
\\	石: 
\\	皮: 
\\	石	
\\	いし・いしへん	
\\	形声 
\\	確	カク、コウ	たし.か、たし.かめる		
\\	適確(てっかく): 
\\	確実(かくじつ): 
\\	確保(かくほ): 
\\	明確(めいかく): 
\\	確か(たしか): 
\\	確認(かくにん): 
\\	確立(かくりつ): 
\\	確定(かくてい): 
\\	確率(かくりつ): 
\\	確信(かくしん): 
\\	確かめる(たしかめる): 
\\	正確(せいかく): 
\\	確り(しっかり): 
\\	石: 
\\	隺: 
\\	石	
\\	いし・いしへん	
\\	形声 
\\	示	ジ、シ	しめ.す		
\\	示す(しめす): 
\\	展示(てんじ): 
\\	掲示(けいじ): 
\\	提示(ていじ): 
\\	暗示(あんじ): 
\\	示: 
\\	示	
\\	しめす・しめすへん	
\\	象形 
\\	祖	ソ		い	
\\	祖父(そふ): 
\\	祖母(そぼ): 
\\	先祖(せんぞ): 
\\	祖先(そせん): 
\\	お祖父さん(おじいさん): 
\\	お祖母さん(おばあさん): 
\\	示: 
\\	且: 
\\	示	
\\	しめす・しめすへん	祖	形声 
\\	禁	キン			
\\	禁止(きんし): 
\\	禁煙(きんえん): 
\\	禁じる(きんじる): 
\\	禁物(きんもつ): 
\\	禁ずる(きんずる): 
\\	林: 
\\	示: 
\\	示	
\\	しめす・しめすへん	
\\	会意 
\\	移	イ	うつ.る、うつ.す		
\\	移転(いてん): 
\\	移動(いどう): 
\\	移民(いみん): 
\\	移住(いじゅう): 
\\	移す(うつす): 
\\	移行(いこう): 
\\	移る(うつる): 
\\	禾: 
\\	多: 
\\	禾	
\\	のぎ・のぎへん	
\\	会意 
\\	程	テイ	ほど、-ほど	ほと	
\\	程度(ていど): 
\\	日程(にってい): 
\\	過程(かてい): 
\\	課程(かてい): 
\\	余程(よほど): 
\\	方程式(ほうていしき): 
\\	程(ほど): 
\\	其れ程(それほど): 
\\	成程(なるほど): 
\\	然程(さほど): 
\\	中程(なかほど): 
\\	禾: 
\\	呈: 
\\	禾	
\\	のぎ・のぎへん	
\\	形声 
\\	税	ゼイ		さい	
\\	関税(かんぜい): 
\\	課税(かぜい): 
\\	税金(ぜいきん): 
\\	税関(ぜいかん): 
\\	税務署(ぜいむしょ): 
\\	免税(めんぜい): 
\\	税(ぜい): 
\\	禾: 
\\	兌: 
\\	禾	
\\	のぎ・のぎへん	
\\	形声 
\\	築	チク	きず.く	つい, つき, つく, づき	
\\	建築(けんちく): 
\\	新築(しんちく): 
\\	築く(きずく): 
\\	筑: 
\\	木: 
\\	竹	
\\	たけ・たけかんむり	
\\	形声 
\\	精	セイ、ショウ、シヤウ		きよ	
\\	精神(せいしん): 
\\	精密(せいみつ): 
\\	精算(せいさん): 
\\	精巧(せいこう): 
\\	精々(せいぜい): 
\\	米: 
\\	青: 
\\	米	
\\	こめ・こめへん	
\\	形声 
\\	素	ソ、ス	もと		
\\	素晴しい(すばらしい): 
\\	素材(そざい): 
\\	要素(ようそ): 
\\	素直(すなお): 
\\	酸素(さんそ): 
\\	素人(しろうと): 
\\	水素(すいそ): 
\\	素朴(そぼく): 
\\	素敵(すてき): 
\\	素質(そしつ): 
\\	炭素(たんそ): 
\\	簡素(かんそ): 
\\	元素(げんそ): 
\\	質素(しっそ): 
\\	素早い(すばやい): 
\\	素っ気ない(そっけない): 
\\	素っ(すっ): 
\\	三: 
\\	糸: 
\\	糸	
\\	いと・いとへん	
\\	象形 
\\	経	ケイ、キョウ	へ.る、た.つ、たていと、はか.る、のり	つね, のぶ	
\\	経営(けいえい): 
\\	経済(けいざい): 
\\	経験(けいけん): 
\\	経費(けいひ): 
\\	経過(けいか): 
\\	神経(しんけい): 
\\	経由(けいゆ): 
\\	経歴(けいれき): 
\\	経る(へる): 
\\	経緯(いきさつ): 
\\	経路(けいろ): 
\\	経つ(たつ): 
\\	経度(けいど): 
\\	糸: 
\\	圣: 
\\	糸	
\\	いと・いとへん	經	形声 
\\	統	トウ	す.べる、ほび.る	のり, むね	
\\	大統領(だいとうりょう): 
\\	統一(とういつ): 
\\	伝統(でんとう): 
\\	統合(とうごう): 
\\	統計(とうけい): 
\\	統治(とうち): 
\\	統制(とうせい): 
\\	系統(けいとう): 
\\	統率(とうそつ): 
\\	糸: 
\\	充: 
\\	糸	
\\	いと・いとへん	
\\	形声 
\\	絶	ゼツ	た.える、た.やす、た.つ		
\\	絶つ(たつ): 
\\	絶対(ぜったい): 
\\	絶望(ぜつぼう): 
\\	絶滅(ぜつめつ): 
\\	途絶える(とだえる): 
\\	拒絶(きょぜつ): 
\\	絶える(たえる): 
\\	絶えず(たえず): 
\\	絶版(ぜっぱん): 
\\	謝絶(しゃぜつ): 
\\	糸: 
\\	色: 
\\	糸	
\\	いと・いとへん	
\\	形声 
\\	綿	メン	わた	う	
\\	綿(わた): 
\\	木綿(もめん): 
\\	綿(めん): 
\\	糸: 
\\	帛: 
\\	糸	
\\	いと・いとへん	
\\	会意 
\\	総	ソウ	す.べて、すべ.て、ふさ	うさ, ずさ	
\\	総会(そうかい): 
\\	総合(そうごう): 
\\	総理大臣(そうりだいじん): 
\\	総(そう): 
\\	糸: 
\\	忩: 
\\	糸	
\\	いと・いとへん	總	形声 
\\	編	ヘン	あ.む、-あ.み		
\\	編集(へんしゅう): 
\\	編物(あみもの): 
\\	短編(たんぺん): 
\\	長編(ちょうへん): 
\\	編(へん): 
\\	編む(あむ): 
\\	糸: 
\\	扁: 
\\	糸	
\\	いと・いとへん	
\\	形声 
\\	績	セキ		み	
\\	実績(じっせき): 
\\	業績(ぎょうせき): 
\\	成績(せいせき): 
\\	紡績(ぼうせき): 
\\	功績(こうせき): 
\\	糸: 
\\	責: 
\\	糸	
\\	いと・いとへん	
\\	形声 
\\	織	ショク、シキ	お.る、お.り、おり、-おり、-お.り	こおり, こり, のり	
\\	組織(そしき): 
\\	織物(おりもの): 
\\	織る(おる): 
\\	糸: 
\\	戠: 
\\	糸	
\\	いと・いとへん	
\\	形声 
\\	罪	ザイ	つみ		
\\	犯罪(はんざい): 
\\	罪(つみ): 
\\	网: 
\\	非: 
\\	网	
\\	あみ・あみがしら・あみめ・よこめ・よんかしら	
\\	会意 
\\	群	グン	む.れる、む.れ、むら、むら.がる	ぐり, ぐ, こお, こおり, ごうり	
\\	群(ぐん): 
\\	群れ(むれ): 
\\	群衆(ぐんしゅう): 
\\	群がる(むらがる): 
\\	群集(ぐんしゅう): 
\\	君: 
\\	羊: 
\\	羊	
\\	ひつじ・ひつじへん	
\\	形声 
\\	義	ギ		ただし, ちか, のり, よし	
\\	主義(しゅぎ): 
\\	義務(ぎむ): 
\\	正義(せいぎ): 
\\	意義(いぎ): 
\\	講義(こうぎ): 
\\	定義(ていぎ): 
\\	義理(ぎり): 
\\	羊: 
\\	我: 
\\	羊	
\\	ひつじ・ひつじへん	
\\	会意 
\\	耕	コウ	たがや.す	こお	
\\	農耕(のうこう): 
\\	耕作(こうさく): 
\\	耕す(たがやす): 
\\	耕地(こうち): 
\\	耒: 
\\	井: 
\\	耒	
\\	すき・すきへん・らいすき・らいへん	
\\	形声 
\\	職	ショク、ソク		もと	
\\	職員(しょくいん): 
\\	就職(しゅうしょく): 
\\	職業(しょくぎょう): 
\\	職場(しょくば): 
\\	辞職(じしょく): 
\\	退職(たいしょく): 
\\	職務(しょくむ): 
\\	職人(しょくにん): 
\\	職(しょく): 
\\	役職(やくしょく): 
\\	教職(きょうしょく): 
\\	耳: 
\\	戠: 
\\	耳	
\\	みみ・みみへん	
\\	形声 
\\	肥	ヒ	こ.える、こえ、こ.やす、こ.やし、ふと.る	い, こい, ひえ	
\\	肥料(ひりょう): 
\\	肉: 
\\	巴: 
\\	肉	
\\	にく・にくづき	
\\	会意 
\\	能	ノウ	よ.く	たか, の, のり, よし	
\\	可能(かのう): 
\\	機能(きのう): 
\\	能力(のうりょく): 
\\	芸能(げいのう): 
\\	能(のう): 
\\	放射能(ほうしゃのう): 
\\	才能(さいのう): 
\\	性能(せいのう): 
\\	技能(ぎのう): 
\\	万能(ばんのう): 
\\	本能(ほんのう): 
\\	能率(のうりつ): 
\\	知能(ちのう): 
\\	有能(ゆうのう): 
\\	無能(むのう): 
\\	厶: 
\\	月: 
\\	匕: 
\\	匕: 
\\	肉	
\\	にく・にくづき	
\\	象形 
\\	興	コウ、キョウ	おこ.る、おこ.す	おき, おこっ, とも	
\\	興味(きょうみ): 
\\	振興(しんこう): 
\\	新興(しんこう): 
\\	興奮(こうふん): 
\\	復興(ふっこう): 
\\	興業(こうぎょう): 
\\	興じる(きょうじる): 
\\	余興(よきょう): 
\\	臼: 
\\	同: 
\\	八: 
\\	臼	
\\	うす	
\\	会意 
\\	舌	ゼツ	した		
\\	舌(した): 
\\	口: 
\\	舌	
\\	した・したへん	
\\	象形 
\\	舎	シャ、セキ	やど.る	さ, とり	
\\	校舎(こうしゃ): 
\\	田舎(いなか): 
\\	人: 
\\	吉: 
\\	舌	
\\	した・したへん	
\\	会意 
\\	術	ジュツ	すべ		
\\	技術(ぎじゅつ): 
\\	芸術(げいじゅつ): 
\\	美術(びじゅつ): 
\\	手術(しゅじゅつ): 
\\	美術館(びじゅつかん): 
\\	戦術(せんじゅつ): 
\\	学術(がくじゅつ): 
\\	行: 
\\	朮: 
\\	行	
\\	ぎょう・ぎょうがまえ・いく・ゆきがまえ	
\\	会意 
\\	衛	エイ、エ		もり, い, まもる	
\\	防衛(ぼうえい): 
\\	衛星(えいせい): 
\\	衛生(えいせい): 
\\	自衛(じえい): 
\\	護衛(ごえい): 
\\	守衛(しゅえい): 
\\	行: 
\\	韋: 
\\	行	
\\	ぎょう・ぎょうがまえ・いく・ゆきがまえ	衞	形声 
\\	製	セイ			
\\	製造(せいぞう): 
\\	製品(せいひん): 
\\	製作(せいさく): 
\\	製鉄(せいてつ): 
\\	作製(さくせい): 
\\	製法(せいほう): 
\\	製(せい): 
\\	制: 
\\	衣: 
\\	衣	
\\	ころも・ころもへん	
\\	形声 
\\	複	フク			
\\	重複(ちょうふく): 
\\	複雑(ふくざつ): 
\\	複数(ふくすう): 
\\	複合(ふくごう): 
\\	複写(ふくしゃ): 
\\	衣: 
\\	复: 
\\	衣	
\\	ころも・ころもへん	
\\	形声 
\\	規	キ		すのり, ただし, のり, み	
\\	規制(きせい): 
\\	規模(きぼ): 
\\	規定(きてい): 
\\	規則(きそく): 
\\	正規(せいき): 
\\	規律(きりつ): 
\\	不規則(ふきそく): 
\\	規格(きかく): 
\\	規約(きやく): 
\\	定規(じょうぎ): 
\\	夫: 
\\	見: 
\\	見	
\\	みる	
\\	会意 
\\	解	カイ、ゲ	と.く、と.かす、と.ける、ほど.く、ほど.ける、わか.る、さと.る	さとる, とけ	
\\	解決(かいけつ): 
\\	解散(かいさん): 
\\	見解(けんかい): 
\\	理解(りかい): 
\\	解除(かいじょ): 
\\	解放(かいほう): 
\\	解説(かいせつ): 
\\	解る(わかる): 
\\	解釈(かいしゃく): 
\\	誤解(ごかい): 
\\	了解(りょうかい): 
\\	解剖(かいぼう): 
\\	分解(ぶんかい): 
\\	解答(かいとう): 
\\	正解(せいかい): 
\\	弁解(べんかい): 
\\	解く(とく): 
\\	解く(ほどく): 
\\	解ける(とける): 
\\	解ける(ほどける): 
\\	角: 
\\	刀: 
\\	牛: 
\\	角	
\\	つの・かく・つのへん	
\\	会意 
\\	設	セツ	もう.ける	した	
\\	建設(けんせつ): 
\\	施設(しせつ): 
\\	設置(せっち): 
\\	設備(せつび): 
\\	設計(せっけい): 
\\	設立(せつりつ): 
\\	設定(せってい): 
\\	設ける(もうける): 
\\	言: 
\\	殳: 
\\	言	
\\	こと・げん・ごんべん・ことば	
\\	会意 
\\	許	キョ	ゆる.す、もと		
\\	許可(きょか): 
\\	許す(ゆるす): 
\\	免許(めんきょ): 
\\	特許(とっきょ): 
\\	許容(きょよう): 
\\	許り(ばかり): 
\\	言: 
\\	午: 
\\	言	
\\	こと・げん・ごんべん・ことば	
\\	形声 
\\	証	ショウ	あかし		
\\	証拠(しょうこ): 
\\	証明(しょうめい): 
\\	証人(しょうにん): 
\\	証言(しょうげん): 
\\	証(あかし): 
\\	言: 
\\	正: 
\\	言	
\\	こと・げん・ごんべん・ことば	證	形声 
\\	評	ヒョウ			
\\	評価(ひょうか): 
\\	評判(ひょうばん): 
\\	評論(ひょうろん): 
\\	批評(ひひょう): 
\\	書評(しょひょう): 
\\	好評(こうひょう): 
\\	不評(ふひょう): 
\\	言: 
\\	平: 
\\	言	
\\	こと・げん・ごんべん・ことば	
\\	形声 
\\	講	コウ			
\\	講演(こうえん): 
\\	講師(こうし): 
\\	講義(こうぎ): 
\\	講堂(こうどう): 
\\	聴講(ちょうこう): 
\\	講習(こうしゅう): 
\\	休講(きゅうこう): 
\\	講読(こうどく): 
\\	言: 
\\	冓: 
\\	言	
\\	こと・げん・ごんべん・ことば	
\\	形声 
\\	謝	シャ	あやま.る	さ, ざ	
\\	感謝(かんしゃ): 
\\	月謝(げっしゃ): 
\\	謝る(あやまる): 
\\	謝絶(しゃぜつ): 
\\	言: 
\\	射: 
\\	言	
\\	こと・げん・ごんべん・ことば	
\\	形声 
\\	識	シキ	し.る、しる.す	さと, さとる	
\\	意識(いしき): 
\\	認識(にんしき): 
\\	知識(ちしき): 
\\	常識(じょうしき): 
\\	標識(ひょうしき): 
\\	良識(りょうしき): 
\\	言: 
\\	戠: 
\\	言	
\\	こと・げん・ごんべん・ことば	
\\	形声 
\\	護	ゴ	まも.る	もり	
\\	保護(ほご): 
\\	看護婦(かんごふ): 
\\	看護(かんご): 
\\	弁護(べんご): 
\\	介護(かいご): 
\\	護衛(ごえい): 
\\	護謨(ゴム): 
\\	養護(ようご): 
\\	看護師(かんごし): 
\\	言: 
\\	艸: 
\\	隻: 
\\	言	
\\	こと・げん・ごんべん・ことば	
\\	形声 
\\	豊	ホウ、ブ	ゆた.か、とよ	て, で, と, ひろし, ふう, ぶん, ほ, ゆたか	
\\	豊か(ゆたか): 
\\	豊富(ほうふ): 
\\	豊作(ほうさく): 
\\	曲: 
\\	豆: 
\\	豆	
\\	まめ・まめへん	豐	象形 
\\	財	ザイ、サイ、ゾク	たから		
\\	財政(ざいせい): 
\\	財産(ざいさん): 
\\	文化財(ぶんかざい): 
\\	財布(さいふ): 
\\	財源(ざいげん): 
\\	財(ざい): 
\\	貝: 
\\	才: 
\\	貝	
\\	かい・かいへん・こがい	
\\	形声 
\\	貧	ヒン、ビン	まず.しい		
\\	貧困(ひんこん): 
\\	貧しい(まずしい): 
\\	貧乏(びんぼう): 
\\	貧弱(ひんじゃく): 
\\	分: 
\\	貝: 
\\	貝	
\\	かい・かいへん・こがい	
\\	会意 
\\	責	セキ	せ.める		
\\	責任(せきにん): 
\\	責務(せきむ): 
\\	責める(せめる): 
\\	貝: 
\\	貝	
\\	かい・かいへん・こがい	
\\	形声 
\\	貸	タイ	か.す、か.し-、かし-		
\\	貸し出し(かしだし): 
\\	貸し(かし): 
\\	貸す(かす): 
\\	貸家(かしや): 
\\	貸間(かしま): 
\\	代: 
\\	貝: 
\\	貝	
\\	かい・かいへん・こがい	
\\	形声 
\\	貿	ボウ			
\\	貿易(ぼうえき): 
\\	卯: 
\\	貝: 
\\	貝	
\\	かい・かいへん・こがい	
\\	形声 
\\	賀	ガ		か, のり, よし, より	
\\	年賀(ねんが): 
\\	祝賀(しゅくが): 
\\	加: 
\\	貝: 
\\	貝	
\\	かい・かいへん・こがい	
\\	会意 
\\	資	シ		すけ, もと	
\\	資金(しきん): 
\\	投資(とうし): 
\\	資産(しさん): 
\\	資料(しりょう): 
\\	資格(しかく): 
\\	資源(しげん): 
\\	資本(しほん): 
\\	物資(ぶっし): 
\\	融資(ゆうし): 
\\	次: 
\\	貝: 
\\	貝	
\\	かい・かいへん・こがい	
\\	形声 
\\	賛	サン	たす.ける、たた.える		
\\	賛成(さんせい): 
\\	賛美(さんび): 
\\	夫: 
\\	夫: 
\\	貝: 
\\	貝	
\\	かい・かいへん・こがい	贊	会意 
\\	質	シツ、シチ、チ	たち、ただ.す、もと、わりふ		
\\	質問(しつもん): 
\\	物質(ぶっしつ): 
\\	質(しつ): 
\\	本質(ほんしつ): 
\\	人質(ひとじち): 
\\	品質(ひんしつ): 
\\	性質(せいしつ): 
\\	気質(きしつ): 
\\	素質(そしつ): 
\\	地質(ちしつ): 
\\	実質(じっしつ): 
\\	質疑(しつぎ): 
\\	良質(りょうしつ): 
\\	質素(しっそ): 
\\	蛋白質(たんぱくしつ): 
\\	斤: 
\\	斤: 
\\	貝: 
\\	貝	
\\	かい・かいへん・こがい	
\\	会意 
\\	輸	ユ、シュ			
\\	輸出(ゆしゅつ): 
\\	輸入(ゆにゅう): 
\\	輸送(ゆそう): 
\\	運輸(うんゆ): 
\\	輸血(ゆけつ): 
\\	車: 
\\	兪: 
\\	車	
\\	くるま・くるまへん	
\\	形声 
\\	述	ジュツ	の.べる		
\\	記述(きじゅつ): 
\\	述べる(のべる): 
\\	口述(こうじゅつ): 
\\	述語(じゅつご): 
\\	朮: 
\\	辶: 
\\	辵	
\\	しんにょう・しんにゅう	
\\	会意 
\\	迷	メイ	まよ.う		
\\	迷惑(めいわく): 
\\	迷う(まよう): 
\\	迷子(まいご): 
\\	迷信(めいしん): 
\\	米: 
\\	辶: 
\\	辵	
\\	しんにょう・しんにゅう	
\\	形声 
\\	退	タイ	しりぞ.く、しりぞ.ける、ひ.く、の.く、の.ける、ど.く		
\\	退職(たいしょく): 
\\	引退(いんたい): 
\\	脱退(だったい): 
\\	辞退(じたい): 
\\	退院(たいいん): 
\\	退学(たいがく): 
\\	退治(たいじ): 
\\	後退(こうたい): 
\\	退ける(のける): 
\\	退屈(たいくつ): 
\\	退く(のく): 
\\	退く(しりぞく): 
\\	退化(たいか): 
\\	退ける(しりぞける): 
\\	艮: 
\\	辶: 
\\	辵	
\\	しんにょう・しんにゅう	
\\	会意 
\\	逆	ギャク、ゲキ	さか、さか.さ、さか.らう		
\\	逆(ぎゃく): 
\\	逆転(ぎゃくてん): 
\\	逆らう(さからう): 
\\	逆立ち(さかだち): 
\\	吃逆(しゃっくり): 
\\	真逆(まさか): 
\\	逆様(さかさま): 
\\	屮: 
\\	辶: 
\\	辵	
\\	しんにょう・しんにゅう	
\\	形声 
\\	造	ゾウ	つく.る、つく.り、-づく.り	ずくり, づくり, み	
\\	造る(つくる): 
\\	構造(こうぞう): 
\\	製造(せいぞう): 
\\	改造(かいぞう): 
\\	偽造(ぎぞう): 
\\	創造(そうぞう): 
\\	造船(ぞうせん): 
\\	造り(つくり): 
\\	人造(じんぞう): 
\\	告: 
\\	辶: 
\\	辵	
\\	しんにょう・しんにゅう	
\\	会意 
\\	過	カ	す.ぎる、-す.ぎる、-す.ぎ、す.ごす、あやま.つ、あやま.ち、よ.ぎる		
\\	過去(かこ): 
\\	過程(かてい): 
\\	過半数(かはんすう): 
\\	経過(けいか): 
\\	過ぎ(すぎ): 
\\	通過(つうか): 
\\	過失(かしつ): 
\\	過剰(かじょう): 
\\	過疎(かそ): 
\\	過労(かろう): 
\\	超過(ちょうか): 
\\	過ち(あやまち): 
\\	通り過ぎる(とおりすぎる): 
\\	過密(かみつ): 
\\	過ごす(すごす): 
\\	過ぎる(すぎる): 
\\	咼: 
\\	辶: 
\\	辵	
\\	しんにょう・しんにゅう	
\\	形声 
\\	適	テキ	かな.う		
\\	適確(てっかく): 
\\	適切(てきせつ): 
\\	適用(てきよう): 
\\	適当(てきとう): 
\\	適応(てきおう): 
\\	適度(てきど): 
\\	適性(てきせい): 
\\	快適(かいてき): 
\\	適宜(てきぎ): 
\\	適する(てきする): 
\\	亠: 
\\	冂: 
\\	古: 
\\	辶: 
\\	辵	
\\	しんにょう・しんにゅう	
\\	形声 
\\	酸	サン	す.い		
\\	酸化(さんか): 
\\	酸性(さんせい): 
\\	酸素(さんそ): 
\\	酸(さん): 
\\	酸っぱい(すっぱい): 
\\	酉: 
\\	夋: 
\\	酉	
\\	とり・とりへん・ひよみのとり・さけのとり	
\\	形声 
\\	鉱	コウ	あらがね		
\\	鉱山(こうざん): 
\\	鉱業(こうぎょう): 
\\	鉱物(こうぶつ): 
\\	炭鉱(たんこう): 
\\	金: 
\\	広: 
\\	金	
\\	かね・かねへん	鑛	形声 
\\	銅	ドウ	あかがね		
\\	銅(どう): 
\\	銅貨(どうか): 
\\	金: 
\\	同: 
\\	金	
\\	かね・かねへん	
\\	形声 
\\	銭	セン、ゼン	ぜに、すき		
\\	金銭(きんせん): 
\\	小銭(こぜに): 
\\	金: 
\\	戔: 
\\	金	
\\	かね・かねへん	錢	形声 
\\	防	ボウ	ふせ.ぐ	あた, う, ほう	
\\	防衛(ぼうえい): 
\\	国防(こくぼう): 
\\	防止(ぼうし): 
\\	予防(よぼう): 
\\	消防(しょうぼう): 
\\	堤防(ていぼう): 
\\	防火(ぼうか): 
\\	防犯(ぼうはん): 
\\	防ぐ(ふせぐ): 
\\	消防署(しょうぼうしょ): 
\\	阜: 
\\	方: 
\\	阜	
\\	こざと・こざとへん・おか・ぎふのふ	
\\	形声 
\\	限	ゲン	かぎ.る、かぎ.り、-かぎ.り		
\\	期限(きげん): 
\\	権限(けんげん): 
\\	制限(せいげん): 
\\	限界(げんかい): 
\\	限る(かぎる): 
\\	限定(げんてい): 
\\	限度(げんど): 
\\	限り(かぎり): 
\\	無限(むげん): 
\\	局限(きょくげん): 
\\	阜: 
\\	艮: 
\\	阜	
\\	こざと・こざとへん・おか・ぎふのふ	
\\	会意 
\\	険	ケン	けわ.しい		
\\	探険(たんけん): 
\\	危険(きけん): 
\\	保険(ほけん): 
\\	冒険(ぼうけん): 
\\	険しい(けわしい): 
\\	阜: 
\\	僉: 
\\	阜	
\\	こざと・こざとへん・おか・ぎふのふ	險	形声 
\\	際	サイ	きわ、-ぎわ	わ	
\\	国際(こくさい): 
\\	実際(じっさい): 
\\	際(さい): 
\\	交際(こうさい): 
\\	手際(てぎわ): 
\\	際(きわ): 
\\	阜: 
\\	祭: 
\\	阜	
\\	こざと・こざとへん・おか・ぎふのふ	
\\	会意 
\\	雑	ザツ、ゾウ	まじ.える、まじ.る		
\\	雑誌(ざっし): 
\\	複雑(ふくざつ): 
\\	混雑(こんざつ): 
\\	雑音(ざつおん): 
\\	雑談(ざつだん): 
\\	雑貨(ざっか): 
\\	雑(ざつ): 
\\	雑巾(ぞうきん): 
\\	雑木(ざつぼく): 
\\	九: 
\\	木: 
\\	隹: 
\\	隹	
\\	ふるとり	雜	形声 
\\	非	ヒ	あら.ず		
\\	"非常(ひじょう): 
\\	是非(ぜひ): 
\\	非難(ひなん): 
\\	非行(ひこう): 
\\	非(ひ): 
\\	非常に(ひじょうに): 
\\	是非とも(ぜひとも): 
\\	非: 
\\	非	
\\	あらず	
\\	象形 
\\	預	ヨ	あず.ける、あず.かる		
\\	預金(よきん): 
\\	預ける(あずける): 
\\	預かる(あずかる): 
\\	予: 
\\	頁: 
\\	頁	
\\	おおがい・いちのかい	
\\	形声 
\\	領	リョウ	えり	よう, よし	
\\	大統領(だいとうりょう): 
\\	領土(りょうど): 
\\	占領(せんりょう): 
\\	領域(りょういき): 
\\	要領(ようりょう): 
\\	領海(りょうかい): 
\\	領事(りょうじ): 
\\	領収(りょうしゅう): 
\\	領地(りょうち): 
\\	領(りょう): 
\\	令: 
\\	頁: 
\\	頁	
\\	おおがい・いちのかい	
\\	形声 
\\	額	ガク	ひたい	ぬか	
\\	額(がく): 
\\	金額(きんがく): 
\\	差額(さがく): 
\\	額(ひたい): 
\\	客: 
\\	頁: 
\\	頁	
\\	おおがい・いちのかい	
\\	形声 
\\	飼	シ	か.う	かい	
\\	飼育(しいく): 
\\	飼う(かう): 
\\	食: 
\\	司: 
\\	食	
\\	しょく・しょくへん	
\\	形声 
\\	並	ヘイ、ホウ	な.み、なみ、なら.べる、なら.ぶ、なら.びに	なび	
\\	並(なみ): 
\\	並木(なみき): 
\\	並びに(ならびに): 
\\	軒並(のきなみ): 
\\	月並(つきなみ): 
\\	並列(へいれつ): 
\\	並ぶ(ならぶ): 
\\	並べる(ならべる): 
\\	八: 
\\	一: 
\\	立	
\\	たつ・たつへん	竝	会意 
\\	乱	ラン、ロン	みだ.れる、みだ.る、みだ.す、みだ、おさ.める、わた.る	ら	
\\	混乱(こんらん): 
\\	反乱(はんらん): 
\\	乱用(らんよう): 
\\	乱れる(みだれる): 
\\	乱暴(らんぼう): 
\\	内乱(ないらん): 
\\	乱す(みだす): 
\\	舌: 
\\	乙: 
\\	乙	
\\	おつ・おつにょう・つりばり	亂	会意 
\\	乳	ニュウ	ちち、ち		
\\	牛乳(ぎゅうにゅう): 
\\	乳(ちち): 
\\	孚: 
\\	乙: 
\\	乙	
\\	おつ・おつにょう・つりばり	
\\	会意 
\\	亡	ボウ、モウ	な.い、な.き-、ほろ.びる、ほろ.ぶ、ほろ.ぼす		
\\	死亡(しぼう): 
\\	逃亡(とうぼう): 
\\	滅亡(めつぼう): 
\\	亡くなる(なくなる): 
\\	亠: 
\\	匚: 
\\	亠	
\\	けいさん・けいさんかんむり・なべぶた	
\\	象形 
\\	仁	ジン、ニ、ニン		きみ, く, さと, しのぶ, じ, と, のり, ひと, ひとし, ひろ, まさ, まさし, やす, よし	
\\	人: 
\\	二: 
\\	人	
\\	ひと・にんべん・ひとやね	
\\	会意 
\\	供	キョウ、ク、クウ、グ	そな.える、とも、-ども		
\\	子供(こども): 
\\	供給(きょうきゅう): 
\\	提供(ていきょう): 
\\	お供(おとも): 
\\	人: 
\\	共: 
\\	人	
\\	ひと・にんべん・ひとやね	
\\	形声 
\\	俳	ハイ			
\\	俳優(はいゆう): 
\\	俳句(はいく): 
\\	人: 
\\	非: 
\\	人	
\\	ひと・にんべん・ひとやね	
\\	形声 
\\	値	チ	ね、あたい	じ	
\\	価値(かち): 
\\	値段(ねだん): 
\\	値(あたい): 
\\	値引き(ねびき): 
\\	値打ち(ねうち): 
\\	値する(あたいする): 
\\	人: 
\\	直: 
\\	人	
\\	ひと・にんべん・ひとやね	
\\	形声 
\\	傷	ショウ	きず、いた.む、いた.める		
\\	傷(きず): 
\\	負傷(ふしょう): 
\\	中傷(ちゅうしょう): 
\\	傷む(いたむ): 
\\	傷つく(きずつく): 
\\	傷付ける(きずつける): 
\\	火傷(やけど): 
\\	傷める(いためる): 
\\	人: 
\\	昜: 
\\	人	
\\	ひと・にんべん・ひとやね	
\\	形声 
\\	優	ユウ、ウ	やさ.しい、すぐ.れる、まさ.る	ゆ, よし	
\\	優勝(ゆうしょう): 
\\	俳優(はいゆう): 
\\	優先(ゆうせん): 
\\	女優(じょゆう): 
\\	優秀(ゆうしゅう): 
\\	優勢(ゆうせい): 
\\	優しい(やさしい): 
\\	優位(ゆうい): 
\\	優美(ゆうび): 
\\	優れる(すぐれる): 
\\	優越(ゆうえつ): 
\\	優(やさ): 
\\	優(ゆう): 
\\	人: 
\\	憂: 
\\	人	
\\	ひと・にんべん・ひとやね	
\\	形声 
\\	党	トウ	なかま、むら		
\\	政党(せいとう): 
\\	野党(やとう): 
\\	与党(よとう): 
\\	党(とう): 
\\	尚: 
\\	兄: 
\\	黑	
\\	くろ・くろへん	黨	形声 
\\	冊	サツ、サク	ふみ		
\\	冊(さつ): 
\\	冂: 
\\	廾: 
\\	冂	
\\	けいがまえ・まきがまえ・どうがまえ・えんがまえ	
\\	象形 
\\	処	ショ	ところ、-こ、お.る		
\\	処分(しょぶん): 
\\	処理(しょり): 
\\	処置(しょち): 
\\	対処(たいしょ): 
\\	処罰(しょばつ): 
\\	何処(どこ): 
\\	彼処(あそこ): 
\\	其処(そこ): 
\\	何処か(どこか): 
\\	此処(ここ): 
\\	其処で(そこで): 
\\	目処(めど): 
\\	其処ら(そこら): 
\\	夂: 
\\	几: 
\\	虍	
\\	とら・とらがしら・とらかんむり	處	会意 
\\	刻	コク	きざ.む、きざ.み		
\\	深刻(しんこく): 
\\	彫刻(ちょうこく): 
\\	時刻(じこく): 
\\	刻む(きざむ): 
\\	遅刻(ちこく): 
\\	時刻表(じこくひょう): 
\\	亥: 
\\	刀: 
\\	刀	
\\	かたな・りっとう	
\\	形声 
\\	割	カツ	わ.る、わり、わ.り、わ.れる、さ.く		
\\	割り引き(わりびき): 
\\	割(わり): 
\\	割当て(わりあて): 
\\	役割(やくわり): 
\\	割合(わりあい): 
\\	時間割り(じかんわり): 
\\	割る(わる): 
\\	割り込む(わりこむ): 
\\	割れる(われる): 
\\	割り算(わりざん): 
\\	割合に(わりあいに): 
\\	割りと(わりと): 
\\	害: 
\\	刀: 
\\	刀	
\\	かたな・りっとう	
\\	形声 
\\	創	ソウ、ショウ	つく.る、はじ.める、きず、けず.しける	はじめ	
\\	創立(そうりつ): 
\\	創刊(そうかん): 
\\	創造(そうぞう): 
\\	創作(そうさく): 
\\	独創(どくそう): 
\\	倉: 
\\	刀: 
\\	刀	
\\	かたな・りっとう	
\\	形声 
\\	劇	ゲキ			
\\	劇場(げきじょう): 
\\	劇(げき): 
\\	演劇(えんげき): 
\\	悲劇(ひげき): 
\\	喜劇(きげき): 
\\	劇団(げきだん): 
\\	虍: 
\\	豕: 
\\	刀: 
\\	刀	
\\	かたな・りっとう	
\\	会意 
\\	勤	キン、ゴン	つと.める、-づと.め、つと.まる、いそ.しむ		
\\	勤める(つとめる): 
\\	勤務(きんむ): 
\\	通勤(つうきん): 
\\	勤労(きんろう): 
\\	転勤(てんきん): 
\\	出勤(しゅっきん): 
\\	勤勉(きんべん): 
\\	勤め先(つとめさき): 
\\	勤まる(つとまる): 
\\	菫: 
\\	力: 
\\	力	
\\	ちから・りきづくり	勤	形声 
\\	危	キ	あぶ.ない、あや.うい、あや.ぶむ		
\\	危機(きき): 
\\	危険(きけん): 
\\	危うい(あやうい): 
\\	危ぶむ(あやぶむ): 
\\	危害(きがい): 
\\	危ない(あぶない): 
\\	勹: 
\\	厄: 
\\	卩	
\\	ふし・ふしづくり・わりふ	
\\	形声 
\\	卵	ラン	たまご		
\\	卵(たまご): 
\\	卯: 
\\	卩	
\\	ふし・ふしづくり・わりふ	
\\	象形 
\\	厳	ゲン、ゴン	おごそ.か、きび.しい、いか.めしい、いつくし	いつ, いづ, きゅうら, とし	
\\	厳重(げんじゅう): 
\\	厳密(げんみつ): 
\\	厳しい(きびしい): 
\\	厳か(おごそか): 
\\	小: 
\\	厂: 
\\	敢: 
\\	口	
\\	くち・くちへん	嚴	形声 
\\	収	シュウ	おさ.める、おさ.まる	のぶ	
\\	収入(しゅうにゅう): 
\\	収支(しゅうし): 
\\	回収(かいしゅう): 
\\	収集(しゅうしゅう): 
\\	収穫(しゅうかく): 
\\	収容(しゅうよう): 
\\	吸収(きゅうしゅう): 
\\	徴収(ちょうしゅう): 
\\	収益(しゅうえき): 
\\	領収(りょうしゅう): 
\\	没収(ぼっしゅう): 
\\	丨: 
\\	又: 
\\	攴	
\\	ぼくにょう・ぼくづくり・とまた・のぶん・しぶん	收	会意 
\\	后	コウ、ゴ	きさき		
\\	厂: 
\\	口: 
\\	口	
\\	くち・くちへん	
\\	会意 
\\	否	ヒ	いな、いや		
\\	否(いえ): 
\\	拒否(きょひ): 
\\	否定(ひてい): 
\\	否決(ひけつ): 
\\	否々(いやいや): 
\\	不: 
\\	口: 
\\	口	
\\	くち・くちへん	
\\	会意 
\\	吸	キュウ	す.う		
\\	呼吸(こきゅう): 
\\	吸収(きゅうしゅう): 
\\	吸う(すう): 
\\	口: 
\\	及: 
\\	口	
\\	くち・くちへん	
\\	形声 
\\	呼	コ	よ.ぶ	よぶ	
\\	呼びかける(よびかける): 
\\	呼ぶ(よぶ): 
\\	呼吸(こきゅう): 
\\	呼び出す(よびだす): 
\\	呼び止める(よびとめる): 
\\	嗚呼(ああ): 
\\	口: 
\\	乎: 
\\	口	
\\	くち・くちへん	
\\	形声 
\\	善	ゼン	よ.い、い.い、よ.く、よし.とする	たる, よし	
\\	改善(かいぜん): 
\\	善(ぜん): 
\\	親善(しんぜん): 
\\	最善(さいぜん): 
\\	善し悪し(よしあし): 
\\	善良(ぜんりょう): 
\\	羊: 
\\	口: 
\\	口	
\\	くち・くちへん	
\\	会意 
\\	困	コン	こま.る		
\\	困難(こんなん): 
\\	貧困(ひんこん): 
\\	困る(こまる): 
\\	囗: 
\\	木: 
\\	囗	
\\	くに・くにがまえ	
\\	象形 
\\	垂	スイ	た.れる、た.らす、た.れ、-た.れ、なんなんと.す	たる, だれ	
\\	垂直(すいちょく): 
\\	垂れる(たれる): 
\\	千: 
\\	土: 
\\	土	
\\	つち・つちへん・どへん	
\\	会意 
\\	城	ジョウ	しろ	いく, がき, き, くに, ぐしく, ぐすく, しき, すく, ぜい, たち, なり	
\\	城(しろ): 
\\	城下(じょうか): 
\\	土: 
\\	成: 
\\	土	
\\	つち・つちへん・どへん	
\\	形声 
\\	域	イキ			
\\	地域(ちいき): 
\\	区域(くいき): 
\\	領域(りょういき): 
\\	流域(りゅういき): 
\\	土: 
\\	或: 
\\	土	
\\	つち・つちへん・どへん	
\\	形声 
\\	奏	ソウ	かな.でる	すすむ	
\\	演奏(えんそう): 
\\	吹奏(すいそう): 
\\	三: 
\\	人: 
\\	天: 
\\	大	
\\	だい・おおきい	
\\	会意 
\\	奮	フン	ふる.う		
\\	興奮(こうふん): 
\\	奮闘(ふんとう): 
\\	大: 
\\	隹: 
\\	田: 
\\	大	
\\	だい・おおきい	
\\	会意 
\\	姿	シ	すがた	しな	
\\	姿(すがた): 
\\	姿勢(しせい): 
\\	次: 
\\	女: 
\\	女	
\\	おんな・おんなへん	
\\	形声 
\\	存	ソン、ゾン	ながら.える、あ.る、たも.つ と.う	まさ	
\\	存在(そんざい): 
\\	保存(ほぞん): 
\\	生存(せいぞん): 
\\	共存(きょうぞん): 
\\	ご存じ(ごぞんじ): 
\\	依存(いぞん): 
\\	存続(そんぞく): 
\\	存じる(ぞんじる): 
\\	存ずる(ぞんずる): 
\\	人: 
\\	子: 
\\	子	
\\	こ・こへん	
\\	会意 
\\	孝	コウ、キョウ		たか, たかし, のり, よし	
\\	孝行(こうこう): 
\\	老: 
\\	子: 
\\	子	
\\	こ・こへん	
\\	会意 
\\	宅	タク		け, たか, たけ, や, やけ	
\\	自宅(じたく): 
\\	住宅(じゅうたく): 
\\	帰宅(きたく): 
\\	邸宅(ていたく): 
\\	宅(たく): 
\\	社宅(しゃたく): 
\\	お宅(おたく): 
\\	宀: 
\\	乇: 
\\	宀	
\\	うかんむり	
\\	形声 
\\	宇	ウ		いえ, たか, ひろ, ひろし	
\\	宇宙(うちゅう): 
\\	宀: 
\\	于: 
\\	宀	
\\	うかんむり	
\\	形声 
\\	宗	シュウ、ソウ	むね	かず, し, そ, そお, たか, とし, のり, ひろ, むな, もと, よし	
\\	宗教(しゅうきょう): 
\\	宗(しゅう): 
\\	宀: 
\\	示: 
\\	宀	
\\	うかんむり	
\\	会意 
\\	宙	チュウ		ひろ, ゆ	
\\	宇宙(うちゅう): 
\\	宙返り(ちゅうがえり): 
\\	宀: 
\\	由: 
\\	宀	
\\	うかんむり	
\\	形声 
\\	宝	ホウ	たから	ほ	
\\	宝(たから): 
\\	宝石(ほうせき): 
\\	重宝(ちょうほう): 
\\	宀: 
\\	玉: 
\\	宀	
\\	うかんむり	寶	形声 
\\	宣	セン	のたま.う	とおる, のぶ, のぼる, のり, ひさ, よし	
\\	宣言(せんげん): 
\\	宣伝(せんでん): 
\\	宣教(せんきょう): 
\\	宀: 
\\	亘: 
\\	宀	
\\	うかんむり	
\\	会意 
\\	密	ミツ	ひそ.か		
\\	秘密(ひみつ): 
\\	密接(みっせつ): 
\\	精密(せいみつ): 
\\	厳密(げんみつ): 
\\	密度(みつど): 
\\	密集(みっしゅう): 
\\	過密(かみつ): 
\\	密か(ひそか): 
\\	密(みつ): 
\\	宀: 
\\	必: 
\\	山: 
\\	宀	
\\	うかんむり	
\\	会意 
\\	寸	スン		す, みき	
\\	寸法(すんぽう): 
\\	一寸(ちょっと): 
\\	燐寸(マッチ): 
\\	寸: 
\\	寸	
\\	すん・すんづくり	
\\	会意 
\\	専	セン	もっぱ.ら		
\\	専門(せんもん): 
\\	専攻(せんこう): 
\\	専用(せんよう): 
\\	専修(せんしゅう): 
\\	専ら(もっぱら): 
\\	専制(せんせい): 
\\	由: 
\\	寸: 
\\	寸	
\\	すん・すんづくり	專	会意 
\\	射	シャ	い.る、さ.す、う.つ		
\\	放射能(ほうしゃのう): 
\\	注射(ちゅうしゃ): 
\\	反射(はんしゃ): 
\\	発射(はっしゃ): 
\\	放射(ほうしゃ): 
\\	射す(さす): 
\\	身: 
\\	寸: 
\\	寸	
\\	すん・すんづくり	
\\	会意 
\\	将	ショウ、ソウ	まさ.に、はた、まさ、ひきい.る、もって	かつ, かつり, すすむ, たか, ゆき	
\\	将来(しょうらい): 
\\	将棋(しょうぎ): 
\\	爿: 
\\	爪: 
\\	寸: 
\\	寸	
\\	すん・すんづくり	將	会意 
\\	尊	ソン	たっと.い、とうと.い、たっと.ぶ、とうと.ぶ	さだ, たか, たけ, みこと	
\\	尊重(そんちょう): 
\\	尊敬(そんけい): 
\\	尊い(とうとい): 
\\	尊ぶ(とうとぶ): 
\\	自尊心(じそんしん): 
\\	酋: 
\\	寸: 
\\	寸	
\\	すん・すんづくり	
\\	会意 
\\	就	シュウ、ジュ	つ.く、つ.ける	たか, なり	
\\	就任(しゅうにん): 
\\	就職(しゅうしょく): 
\\	就業(しゅうぎょう): 
\\	就いて(ついて): 
\\	就く(つく): 
\\	京: 
\\	尤: 
\\	尢	
\\	まげあし・だいのまげあし・あう・おうにょう	
\\	会意 
\\	尺	シャク		せき	
\\	尸: 
\\	八: 
\\	尸	
\\	しかばね・かばね・かばねだれ	
\\	象形 
\\	届	カイ	とど.ける、-とど.け、とど.く		
\\	届け(とどけ): 
\\	届く(とどく): 
\\	届ける(とどける): 
\\	尸: 
\\	由: 
\\	尸	
\\	しかばね・かばね・かばねだれ	屆	会意 
\\	展	テン		のぶ, のり, ひろ, ゆき	
\\	展開(てんかい): 
\\	発展(はってん): 
\\	展示(てんじ): 
\\	展望(てんぼう): 
\\	展覧会(てんらんかい): 
\\	進展(しんてん): 
\\	尸: 
\\	廾: 
\\	尸	
\\	しかばね・かばね・かばねだれ	
\\	会意 
\\	層	ソウ			
\\	一層(いっそう): 
\\	高層(こうそう): 
\\	階層(かいそう): 
\\	大層(たいそう): 
\\	尸: 
\\	曽: 
\\	尸	
\\	しかばね・かばね・かばねだれ	層	形声 
\\	己	コ、キ	おのれ、つちのと、な	し, み	
\\	自己(じこ): 
\\	已: 
\\	己	
\\	おのれ・き・つちのと	
\\	象形 
\\	巻	カン、ケン	ま.く、まき、ま.き		
\\	寝巻き(ねまき): 
\\	巻(まき): 
\\	巻く(まく): 
\\	取り巻く(とりまく): 
\\	巻(かん): 
\\	𠔉: 
\\	己: 
\\	卩	
\\	ふし・ふしづくり・わりふ	卷	会意 
\\	幕	マク、バク	とばり		
\\	幕(まく): 
\\	莫: 
\\	巾: 
\\	巾	
\\	はば・はばへん・きんべん	
\\	形声 
\\	干	カン	ほ.す、ほ.し-、-ぼ.し、ひ.る	ほし	
\\	若干(じゃっかん): 
\\	干渉(かんしょう): 
\\	干す(ほす): 
\\	梅干(うめぼし): 
\\	干し(ほし): 
\\	干し物(ほしもの): 
\\	十: 
\\	干	
\\	かん・いちじゅう・ほす・ひる・たてかん	
\\	象形 
\\	幼	ヨウ	おさな.い	うぶ, わか	
\\	幼児(ようじ): 
\\	幼稚園(ようちえん): 
\\	幼稚(ようち): 
\\	幼い(おさない): 
\\	幺: 
\\	力: 
\\	幺	
\\	いとがしら・よう	
\\	象形 
\\	庁	チョウ、テイ	やくしょ		
\\	庁(ちょう): 
\\	官庁(かんちょう): 
\\	県庁(けんちょう): 
\\	广: 
\\	丁: 
\\	广	
\\	まだれ	廳	形声 
\\	座	ザ	すわ.る		
\\	座席(ざせき): 
\\	座談会(ざだんかい): 
\\	座敷(ざしき): 
\\	星座(せいざ): 
\\	座る(すわる): 
\\	座布団(ざぶとん): 
\\	御座います(ございます): 
\\	即座に(そくざに): 
\\	お目出度うご座います(おめでとうございます): 
\\	座標(ざひょう): 
\\	广: 
\\	坐: 
\\	广	
\\	まだれ	
\\	形声 
\\	延	エン	の.びる、の.べる、の.べ、の.ばす	たか, のぶ, のべ	
\\	延長(えんちょう): 
\\	延期(えんき): 
\\	延ばす(のばす): 
\\	延べ(のべ): 
\\	延いては(ひいては): 
\\	正: 
\\	廴: 
\\	廴	
\\	えんにょう・いんにょう	
\\	会意 
\\	律	リツ、リチ、レツ		たかし, のり	
\\	法律(ほうりつ): 
\\	規律(きりつ): 
\\	一律(いちりつ): 
\\	彳: 
\\	聿: 
\\	彳	
\\	ぎょうにんべん	
\\	形声 
\\	従	ジュウ、ショウ、ジュ	したが.う、したが.える、より		
\\	従来(じゅうらい): 
\\	従業員(じゅうぎょういん): 
\\	従って(したがって): 
\\	従事(じゅうじ): 
\\	従う(したがう): 
\\	従兄弟(いとこ): 
\\	従姉妹(いとこ): 
\\	彳: 
\\	疋: 
\\	彳	
\\	ぎょうにんべん	從	会意 
\\	忘	ボウ	わす.れる		
\\	忘れ物(わすれもの): 
\\	忘れる(わすれる): 
\\	度忘れ(どわすれ): 
\\	亡: 
\\	心: 
\\	心	
\\	こころ・したごころ・りっしんべん	
\\	形声 
\\	忠	チュウ		きよし, たた, ただ, ただし, なお	
\\	忠実(まめ): 
\\	忠告(ちゅうこく): 
\\	中: 
\\	心: 
\\	心	
\\	こころ・したごころ・りっしんべん	
\\	形声 
\\	憲	ケン		かず, のり, よし	
\\	憲法(けんぽう): 
\\	宀: 
\\	网: 
\\	心: 
\\	心	
\\	こころ・したごころ・りっしんべん	
\\	形声 
\\	我	ガ	われ、わ、わ.が-、わが-	あ, あが, か	
\\	我々(われわれ): 
\\	我が(わが): 
\\	我慢(がまん): 
\\	自我(じが): 
\\	我(われ): 
\\	我が儘(わがまま): 
\\	怪我(けが): 
\\	丿: 
\\	戈: 
\\	亅: 
\\	戈	
\\	ほこ・ほこづくり・ほこがまえ・かのほこ	
\\	仮借 
\\	批	ヒ			
\\	批判(ひはん): 
\\	批評(ひひょう): 
\\	手: 
\\	比: 
\\	手	
\\	て・てへん	
\\	形声 
\\	担	タン	かつ.ぐ、にな.う		
\\	担当(たんとう): 
\\	負担(ふたん): 
\\	分担(ぶんたん): 
\\	担ぐ(かつぐ): 
\\	担う(になう): 
\\	手: 
\\	旦: 
\\	手	
\\	て・てへん	擔	形声 
\\	拝	ハイ	おが.む、おろが.む		
\\	崇拝(すうはい): 
\\	拝見(はいけん): 
\\	拝啓(はいけい): 
\\	拝む(おがむ): 
\\	拝借(はいしゃく): 
\\	手: 
\\	一: 
\\	三: 
\\	丨: 
\\	手	
\\	て・てへん	拜	会意 
\\	拡	カク、コウ	ひろ.がる、ひろ.げる、ひろ.める		
\\	拡大(かくだい): 
\\	拡張(かくちょう): 
\\	拡散(かくさん): 
\\	拡充(かくじゅう): 
\\	手: 
\\	広: 
\\	手	
\\	て・てへん	擴	形声 
\\	捨	シャ	す.てる		
\\	四捨五入(ししゃごにゅう): 
\\	捨てる(すてる): 
\\	手: 
\\	舎: 
\\	手	
\\	て・てへん	
\\	形声 
\\	探	タン	さぐ.る、さが.す		
\\	探険(たんけん): 
\\	探す(さがす): 
\\	探る(さぐる): 
\\	手: 
\\	㓁: 
\\	木: 
\\	手	
\\	て・てへん	
\\	形声 
\\	推	スイ	お.す		
\\	推進(すいしん): 
\\	推薦(すいせん): 
\\	推定(すいてい): 
\\	推理(すいり): 
\\	推測(すいそく): 
\\	類推(るいすい): 
\\	手: 
\\	隹: 
\\	手	
\\	て・てへん	
\\	形声 
\\	揮	キ	ふる.う		
\\	指揮(しき): 
\\	発揮(はっき): 
\\	手: 
\\	軍: 
\\	手	
\\	て・てへん	
\\	形声 
\\	操	ソウ、サン	みさお、あやつ.る	さお, みさ	
\\	操作(そうさ): 
\\	操縦(そうじゅう): 
\\	体操(たいそう): 
\\	操る(あやつる): 
\\	手: 
\\	品: 
\\	木: 
\\	手	
\\	て・てへん	
\\	形声 
\\	敬	ケイ、キョウ	うやま.う	け, たか, たかし, たけ, とし, のり, ひろ, ゆき, よし	
\\	敬意(けいい): 
\\	尊敬(そんけい): 
\\	敬語(けいご): 
\\	敬う(うやまう): 
\\	敬具(けいぐ): 
\\	苟: 
\\	攵: 
\\	攴	
\\	ぼくにょう・ぼくづくり・とまた・のぶん・しぶん	
\\	会意 
\\	映	エイ	うつ.る、うつ.す、は.える、-ば.え	あきら, え, てる	
\\	映画(えいが): 
\\	映像(えいぞう): 
\\	映画館(えいがかん): 
\\	反映(はんえい): 
\\	映る(うつる): 
\\	映す(うつす): 
\\	映える(はえる): 
\\	映写(えいしゃ): 
\\	日: 
\\	央: 
\\	日	
\\	ひ・ひへん・にち・にちへん	
\\	形声 
\\	晩	バン			
\\	晩(ばん): 
\\	晩年(ばんねん): 
\\	毎晩(まいばん): 
\\	今晩(こんばん): 
\\	今晩は(こんばんは): 
\\	晩御飯(ばんごはん): 
\\	日: 
\\	免: 
\\	日	
\\	ひ・ひへん・にち・にちへん	晚	形声 
\\	暖	ダン、ノン	あたた.か、あたた.かい、あたた.まる、あたた.める		
\\	暖かい(あたたかい): 
\\	暖まる(あたたまる): 
\\	温暖(おんだん): 
\\	暖房(だんぼう): 
\\	暖か(あたたか): 
\\	日: 
\\	爰: 
\\	日	
\\	ひ・ひへん・にち・にちへん	
\\	形声 
\\	暮	ボ	く.れる、く.らす	ぐらし, ぐれ, ぽ	
\\	暮らし(くらし): 
\\	暮れ(くれ): 
\\	暮れる(くれる): 
\\	夕暮れ(ゆうぐれ): 
\\	暮らす(くらす): 
\\	莫: 
\\	日: 
\\	日	
\\	ひ・ひへん・にち・にちへん	
\\	形声 
\\	朗	ロウ	ほが.らか、あき.らか	あき, あきら, お, さえ, ろ	
\\	朗読(ろうどく): 
\\	明朗(めいろう): 
\\	朗らか(ほがらか): 
\\	良: 
\\	肉: 
\\	月	
\\	つき・つきへん	朗	形声 
\\	机	キ	つくえ		
\\	机(つくえ): 
\\	木: 
\\	几: 
\\	木	
\\	き・きへん	
\\	象形 
\\	枚	マイ、バイ		ひら	
\\	枚数(まいすう): 
\\	枚(まい): 
\\	木: 
\\	攵: 
\\	木	
\\	き・きへん	
\\	会意 
\\	染	セン	そ.める、-ぞ.め、-ぞめ、そ.まる、し.みる、-じ.みる、し.み、-し.める	そめ	
\\	汚染(おせん): 
\\	感染(かんせん): 
\\	染める(そめる): 
\\	伝染(でんせん): 
\\	染まる(そまる): 
\\	染みる(しみる): 
\\	染みる(じみる): 
\\	水: 
\\	九: 
\\	木: 
\\	木	
\\	き・きへん	
\\	会意 
\\	株	シュ	かぶ		
\\	株式(かぶしき): 
\\	株(かぶ): 
\\	木: 
\\	朱: 
\\	木	
\\	き・きへん	
\\	形声 
\\	棒	ボウ			
\\	棒(ぼう): 
\\	泥棒(どろぼう): 
\\	鉄棒(てつぼう): 
\\	木: 
\\	奉: 
\\	木	
\\	き・きへん	
\\	形声 
\\	模	モ、ボ		がみ	
\\	規模(きぼ): 
\\	模様(もよう): 
\\	模型(もけい): 
\\	模範(もはん): 
\\	模索(もさく): 
\\	模倣(もほう): 
\\	模試(もし): 
\\	木: 
\\	莫: 
\\	木	
\\	き・きへん	
\\	形声 
\\	権	ケン、ゴン	おもり、かり、はか.る		
\\	政権(せいけん): 
\\	権利(けんり): 
\\	権力(けんりょく): 
\\	権限(けんげん): 
\\	主権(しゅけん): 
\\	権威(けんい): 
\\	棄権(きけん): 
\\	特権(とっけん): 
\\	権(けん): 
\\	木: 
\\	隹: 
\\	木	
\\	き・きへん	權	形声 
\\	樹	ジュ	き	いつき, うえ, こ, しげ, じ, たちき, たつ, たつる, な	
\\	樹木(じゅもく): 
\\	樹立(じゅりつ): 
\\	木: 
\\	吉: 
\\	豆: 
\\	寸: 
\\	木	
\\	き・きへん	
\\	形声 
\\	欲	ヨク	ほっ.する、ほ.しい		
\\	意欲(いよく): 
\\	欲(よく): 
\\	食欲(しょくよく): 
\\	欲望(よくぼう): 
\\	欲しい(ほしい): 
\\	欲張り(よくばり): 
\\	欲深い(よくぶかい): 
\\	谷: 
\\	欠: 
\\	欠	
\\	あくび・けんづくり・かける・けつ	
\\	形声 
\\	段	ダン、タン			
\\	段階(だんかい): 
\\	手段(しゅだん): 
\\	値段(ねだん): 
\\	一段(いちだん): 
\\	階段(かいだん): 
\\	普段(ふだん): 
\\	段(だん): 
\\	段々(だんだん): 
\\	殳: 
\\	殳	
\\	るまた・ほこづくり	
\\	会意 
\\	沿	エン	そ.う、-ぞ.い		
\\	沿岸(えんがん): 
\\	沿い(ぞい): 
\\	沿線(えんせん): 
\\	沿う(そう): 
\\	水: 
\\	㕣: 
\\	水	
\\	みず・したみず・さんずい	
\\	形声 
\\	泉	セン	いずみ	いず, ずい, ずみ, ぜい, ぜん, の	
\\	温泉(おんせん): 
\\	泉(いずみ): 
\\	白: 
\\	水: 
\\	水	
\\	みず・したみず・さんずい	
\\	象形 
\\	洗	セン	あら.う	あらい, らい	
\\	洗濯(せんたく): 
\\	洗剤(せんざい): 
\\	洗う(あらう): 
\\	手洗い(てあらい): 
\\	水洗(すいせん): 
\\	洗面(せんめん): 
\\	お手洗い(おてあらい): 
\\	水: 
\\	先: 
\\	水	
\\	みず・したみず・さんずい	
\\	形声 
\\	派	ハ			
\\	派遣(はけん): 
\\	立派(りっぱ): 
\\	派手(はで): 
\\	派(は): 
\\	特派(とくは): 
\\	水: 
\\	𠂢: 
\\	水	
\\	みず・したみず・さんずい	
\\	形声 
\\	済	サイ、セイ	す.む、-ず.み、-ずみ、す.まない、す.ます、-す.ます、すく.う、な.す、わたし、わた.る	すむ, なり, わたる	
\\	経済(けいざい): 
\\	救済(きゅうさい): 
\\	返済(へんさい): 
\\	済ます(すます): 
\\	済む(すむ): 
\\	済ませる(すませる): 
\\	済みません(すみません): 
\\	済まない(すまない): 
\\	済み(ずみ): 
\\	水: 
\\	齊: 
\\	水	
\\	みず・したみず・さんずい	濟	形声 
\\	源	ゲン	みなもと	はら, みな, もと	
\\	資源(しげん): 
\\	源(みなもと): 
\\	電源(でんげん): 
\\	起源(きげん): 
\\	語源(ごげん): 
\\	財源(ざいげん): 
\\	水源(すいげん): 
\\	水: 
\\	原: 
\\	水	
\\	みず・したみず・さんずい	
\\	形声 
\\	潮	チョウ	しお、うしお	いた	
\\	潮(しお): 
\\	水: 
\\	朝: 
\\	水	
\\	みず・したみず・さんずい	
\\	形声 
\\	激	ゲキ	はげ.しい		
\\	急激(きゅうげき): 
\\	刺激(しげき): 
\\	感激(かんげき): 
\\	激励(げきれい): 
\\	激しい(はげしい): 
\\	激増(げきぞう): 
\\	水: 
\\	敫: 
\\	水	
\\	みず・したみず・さんずい	
\\	形声 
\\	灰	カイ	はい		
\\	灰色(はいいろ): 
\\	灰皿(はいざら): 
\\	灰(はい): 
\\	厂: 
\\	火: 
\\	火	
\\	ひ・ひへん・れっか・れんが	
\\	会意 
\\	熟	ジュク	う.れる		
\\	成熟(せいじゅく): 
\\	熟(つくづく): 
\\	未熟(みじゅく): 
\\	熟語(じゅくご): 
\\	孰: 
\\	火: 
\\	火	
\\	ひ・ひへん・れっか・れんが	
\\	形声 
\\	片	ヘン	かた-、かた		
\\	破片(はへん): 
\\	片道(かたみち): 
\\	片仮名(カタカナ): 
\\	片寄る(かたよる): 
\\	片付ける(かたづける): 
\\	片思い(かたおもい): 
\\	片付く(かたづく): 
\\	片言(かたこと): 
\\	片付け(かたづけ): 
\\	片(かた): 
\\	片: 
\\	片	
\\	かた・かたへん	
\\	象形 
\\	班	ハン			
\\	班(はん): 
\\	玉: 
\\	刀: 
\\	玉: 
\\	玉	
\\	たま・たまへん ・ おう・おうへん・あうさま	
\\	会意 
\\	異	イ	こと、こと.なる、け		
\\	異常(いじょう): 
\\	異動(いどう): 
\\	異性(いせい): 
\\	異論(いろん): 
\\	異議(いぎ): 
\\	驚異(きょうい): 
\\	差異(さい): 
\\	異なる(ことなる): 
\\	異見(いけん): 
\\	異(い): 
\\	田: 
\\	共: 
\\	田	
\\	た・たへん	
\\	象形 
\\	疑	ギ	うたが.う		
\\	疑問(ぎもん): 
\\	疑う(うたがう): 
\\	疑惑(ぎわく): 
\\	質疑(しつぎ): 
\\	匕: 
\\	矢: 
\\	マ: 
\\	疋: 
\\	疋	
\\	ひき・ひきへん	
\\	象形 
\\	痛	ツウ	いた.い、いた.む、いた.ましい、いた.める		
\\	痛み(いたみ): 
\\	苦痛(くつう): 
\\	頭痛(ずつう): 
\\	痛い(いたい): 
\\	痛感(つうかん): 
\\	痛切(つうせつ): 
\\	痛める(いためる): 
\\	疒: 
\\	甬: 
\\	疒	
\\	やまい・やまいだれ	
\\	形声 
\\	皇	コウ、オウ		おうじ, おお, み	
\\	天皇(てんのう): 
\\	皇居(こうきょ): 
\\	白: 
\\	玉: 
\\	白	
\\	しろ・しろへん	
\\	象形 
\\	盛	セイ、ジョウ	も.る、さか.る、さか.ん	もり	
\\	盛ん(さかん): 
\\	盛り(さかり): 
\\	盛る(もる): 
\\	盛大(せいだい): 
\\	全盛(ぜんせい): 
\\	繁盛(はんじょう): 
\\	盛り上がる(もりあがる): 
\\	目盛り(めもり): 
\\	盛る(さかる): 
\\	盛装(せいそう): 
\\	成: 
\\	皿: 
\\	皿	
\\	さら	
\\	形声 
\\	盟	メイ			
\\	同盟(どうめい): 
\\	連盟(れんめい): 
\\	明: 
\\	皿: 
\\	皿	
\\	さら	
\\	会意 
\\	看	カン	み.る		
\\	看護婦(かんごふ): 
\\	看護(かんご): 
\\	看板(かんばん): 
\\	看病(かんびょう): 
\\	看護師(かんごし): 
\\	手: 
\\	目: 
\\	目	
\\	め・めへん・よこめ	
\\	会意 
\\	砂	サ、シャ	すな	いさ, ご	
\\	砂(すな): 
\\	砂漠(さばく): 
\\	砂糖(さとう): 
\\	砂利(じゃり): 
\\	石: 
\\	少: 
\\	石	
\\	いし・いしへん	
\\	形声 
\\	磁	ジ			
\\	磁石(じしゃく): 
\\	磁気(じき): 
\\	磁器(じき): 
\\	石: 
\\	茲: 
\\	石	
\\	いし・いしへん	
\\	形声 
\\	私	シ	わたくし、わたし		
\\	私(わたし): 
\\	私立(しりつ): 
\\	私鉄(してつ): 
\\	私有(しゆう): 
\\	私用(しよう): 
\\	私物(しぶつ): 
\\	禾: 
\\	厶: 
\\	禾	
\\	のぎ・のぎへん	
\\	会意 
\\	秘	ヒ	ひ.める、ひそ.か、かく.す		
\\	秘書(ひしょ): 
\\	秘密(ひみつ): 
\\	神秘(しんぴ): 
\\	禾: 
\\	必: 
\\	示	
\\	しめす・しめすへん	祕	形声 
\\	穀	コク		たけ	
\\	穀物(こくもつ): 
\\	士: 
\\	冖: 
\\	禾: 
\\	殳: 
\\	禾	
\\	のぎ・のぎへん	穀	形声 
\\	穴	ケツ	あな	けな, しし, な	
\\	穴(あな): 
\\	宀: 
\\	八: 
\\	穴	
\\	あな・あなかんむり	
\\	象形 
\\	窓	ソウ、ス	まど、てんまど、けむだし		
\\	窓(まど): 
\\	窓口(まどぐち): 
\\	穴: 
\\	厶: 
\\	心: 
\\	穴	
\\	あな・あなかんむり	
\\	形声 
\\	筋	キン	すじ		
\\	筋(すじ): 
\\	筋肉(きんにく): 
\\	大筋(おおすじ): 
\\	一筋(ひとすじ): 
\\	粗筋(あらすじ): 
\\	竹: 
\\	肋: 
\\	竹	
\\	たけ・たけかんむり	
\\	象形 
\\	策	サク			
\\	政策(せいさく): 
\\	対策(たいさく): 
\\	策(さく): 
\\	方策(ほうさく): 
\\	竹: 
\\	朿: 
\\	竹	
\\	たけ・たけかんむり	
\\	形声 
\\	簡	カン、ケン	えら.ぶ、ふだ		
\\	簡単(かんたん): 
\\	簡易(かんい): 
\\	簡素(かんそ): 
\\	簡潔(かんけつ): 
\\	竹: 
\\	間: 
\\	竹	
\\	たけ・たけかんむり	
\\	形声 
\\	糖	トウ			
\\	砂糖(さとう): 
\\	米: 
\\	唐: 
\\	米	
\\	こめ・こめへん	
\\	形声 
\\	系	ケイ			
\\	体系(たいけい): 
\\	系統(けいとう): 
\\	系(けい): 
\\	丿: 
\\	糸: 
\\	糸	
\\	いと・いとへん	
\\	象形 
\\	紅	コウ、ク	べに、くれない、あか.い	くれ, もみ	
\\	紅い(あかい): 
\\	紅葉(もみじ): 
\\	紅茶(こうちゃ): 
\\	口紅(くちべに): 
\\	糸: 
\\	工: 
\\	糸	
\\	いと・いとへん	
\\	形声 
\\	納	ノウ、ナッ、ナ、ナン、トウ	おさ.める、-おさ.める、おさ.まる	の, ろ	
\\	納める(おさめる): 
\\	納まる(おさまる): 
\\	納得(なっとく): 
\\	納入(のうにゅう): 
\\	滞納(たいのう): 
\\	糸: 
\\	内: 
\\	糸	
\\	いと・いとへん	
\\	形声 
\\	純	ジュン		あつ, すみ, み, やすし, よし	
\\	単純(たんじゅん): 
\\	純粋(じゅんすい): 
\\	純情(じゅんじょう): 
\\	清純(せいじゅん): 
\\	糸: 
\\	屯: 
\\	糸	
\\	いと・いとへん	
\\	形声 
\\	絹	ケン	きぬ		
\\	絹(きぬ): 
\\	糸: 
\\	肙: 
\\	糸	
\\	いと・いとへん	
\\	形声 
\\	縦	ジュウ	たて		
\\	操縦(そうじゅう): 
\\	縦(たて): 
\\	糸: 
\\	従: 
\\	糸	
\\	いと・いとへん	縱	形声 
\\	縮	シュク	ちぢ.む、ちぢ.まる、ちぢ.める、ちぢ.れる、ちぢ.らす		
\\	縮小(しゅくしょう): 
\\	短縮(たんしゅく): 
\\	圧縮(あっしゅく): 
\\	恐縮(きょうしゅく): 
\\	縮む(ちぢむ): 
\\	縮まる(ちぢまる): 
\\	縮める(ちぢめる): 
\\	縮れる(ちぢれる): 
\\	糸: 
\\	宿: 
\\	糸	
\\	いと・いとへん	
\\	形声 
\\	署	ショ			
\\	署名(しょめい): 
\\	税務署(ぜいむしょ): 
\\	消防署(しょうぼうしょ): 
\\	网: 
\\	者: 
\\	网	
\\	あみ・あみがしら・あみめ・よこめ・よんかしら	署	形声 
\\	翌	ヨク			
\\	翌(よく): 
\\	羽: 
\\	立: 
\\	羽	
\\	はね	
\\	形声 
\\	聖	セイ、ショウ	ひじり	きよ, さと, さとし, せ, たか, ただ, ひろ, まさ, み	
\\	聖書(せいしょ): 
\\	神聖(しんせい): 
\\	耳: 
\\	口: 
\\	玉: 
\\	耳	
\\	みみ・みみへん	
\\	会意 
\\	肺	ハイ			
\\	肺(はい): 
\\	肉: 
\\	市: 
\\	肉	
\\	にく・にくづき	
\\	形声 
\\	背	ハイ	せ、せい、そむ.く、そむ.ける		
\\	背景(はいけい): 
\\	背中(せなか): 
\\	背(せい): 
\\	背広(せびろ): 
\\	背後(はいご): 
\\	背負う(せおう): 
\\	背く(そむく): 
\\	背(せ): 
\\	北: 
\\	肉: 
\\	肉	
\\	にく・にくづき	
\\	形声 
\\	胸	キョウ	むね、むな-		
\\	胸(むね): 
\\	肉: 
\\	匈: 
\\	肉	
\\	にく・にくづき	
\\	形声 
\\	脳	ノウ、ドウ	のうずる		
\\	首脳(しゅのう): 
\\	脳(のう): 
\\	頭脳(ずのう): 
\\	肉: 
\\	小: 
\\	凶: 
\\	肉	
\\	にく・にくづき	腦	会意 
\\	腹	フク	はら		
\\	腹(はら): 
\\	空腹(くうふく): 
\\	中腹(ちゅうふく): 
\\	お腹(おなか): 
\\	山腹(さんぷく): 
\\	中っ腹(ちゅうっぱら): 
\\	腹立ち(はらだち): 
\\	肉: 
\\	复: 
\\	肉	
\\	にく・にくづき	
\\	形声 
\\	臓	ゾウ	はらわた		
\\	心臓(しんぞう): 
\\	肉: 
\\	蔵: 
\\	肉	
\\	にく・にくづき	臟	形声 
\\	臨	リン	のぞ.む	み	
\\	臨時(りんじ): 
\\	臨む(のぞむ): 
\\	臣: 
\\	品: 
\\	臣	
\\	しん・おみ	
\\	会意 
\\	至	シ	いた.る	のぶ, のり, みち, よし	
\\	至って(いたって): 
\\	至急(しきゅう): 
\\	至る(いたる): 
\\	乃至(ないし): 
\\	厶: 
\\	土: 
\\	至	
\\	いたる	
\\	会意 
\\	若	ジャク、ニャク、ニャ	わか.い、わか-、も.しくわ、も.し、も.しくは、ごと.し	わく, わこ	
\\	若干(じゃっかん): 
\\	若い(わかい): 
\\	若々しい(わかわかしい): 
\\	若し(もし): 
\\	若しかしたら(もしかしたら): 
\\	若しかすると(もしかすると): 
\\	若しくは(もしくは): 
\\	若しも(もしも): 
\\	若しかして(もしかして): 
\\	艸: 
\\	右: 
\\	艸	
\\	くさ・くさかんむり	
\\	象形 
\\	著	チョ、チャク	あらわ.す、いちじる.しい		
\\	著者(ちょしゃ): 
\\	著書(ちょしょ): 
\\	著名(ちょめい): 
\\	著(ちょ): 
\\	著す(あらわす): 
\\	著しい(いちじるしい): 
\\	艸: 
\\	者: 
\\	艸	
\\	くさ・くさかんむり	著	形声 
\\	蒸	ジョウ、セイ	む.す、む.れる、む.らす		
\\	蒸す(むす): 
\\	水蒸気(すいじょうき): 
\\	蒸気(じょうき): 
\\	蒸発(じょうはつ): 
\\	蒸し暑い(むしあつい): 
\\	蒸留(じょうりゅう): 
\\	艸: 
\\	烝: 
\\	艸	
\\	くさ・くさかんむり	
\\	形声 
\\	蔵	ゾウ、ソウ	くら、おさ.める、かく.れる	くらし, くらん, くろう, さし, ざ, ろう	
\\	冷蔵庫(れいぞうこ): 
\\	蔵(くら): 
\\	蔵相(ぞうしょう): 
\\	貯蔵(ちょぞう): 
\\	内蔵(ないぞう): 
\\	埋蔵(まいぞう): 
\\	冷蔵(れいぞう): 
\\	金蔵(かねぐら): 
\\	艸: 
\\	厂: 
\\	戈: 
\\	艸	
\\	くさ・くさかんむり	藏	形声 
\\	蚕	サン、テン	かいこ、こ		
\\	天: 
\\	虫: 
\\	虫	
\\	むし・むしへん	蠶	形声 
\\	衆	シュウ、シュ	おお.い		
\\	大衆(たいしゅう): 
\\	衆議院(しゅうぎいん): 
\\	観衆(かんしゅう): 
\\	群衆(ぐんしゅう): 
\\	公衆(こうしゅう): 
\\	衆(しゅう): 
\\	血: 
\\	人: 
\\	血	
\\	ち・ちへん	
\\	会意 
\\	裁	サイ	た.つ、さば.く		
\\	裁判(さいばん): 
\\	制裁(せいさい): 
\\	独裁(どくさい): 
\\	体裁(ていさい): 
\\	裁縫(さいほう): 
\\	裁く(さばく): 
\\	土: 
\\	戈: 
\\	衣: 
\\	衣	
\\	ころも・ころもへん	
\\	形声 
\\	装	ソウ、ショウ	よそお.う、よそお.い		
\\	装置(そうち): 
\\	武装(ぶそう): 
\\	衣装(いしょう): 
\\	服装(ふくそう): 
\\	装飾(そうしょく): 
\\	包装(ほうそう): 
\\	舗装(ほそう): 
\\	装備(そうび): 
\\	盛装(せいそう): 
\\	壮: 
\\	衣: 
\\	衣	
\\	ころも・ころもへん	裝	形声 
\\	裏	リ	うら		
\\	裏切る(うらぎる): 
\\	裏返し(うらがえし): 
\\	裏口(うらぐち): 
\\	裏(うら): 
\\	裏返す(うらがえす): 
\\	衣: 
\\	里: 
\\	衣	
\\	ころも・ころもへん	
\\	形声 
\\	補	ホ	おぎな.う		
\\	候補(こうほ): 
\\	補助(ほじょ): 
\\	補償(ほしょう): 
\\	補給(ほきゅう): 
\\	補強(ほきょう): 
\\	補充(ほじゅう): 
\\	補足(ほそく): 
\\	補う(おぎなう): 
\\	衣: 
\\	甫: 
\\	衣	
\\	ころも・ころもへん	
\\	形声 
\\	視	シ	み.る	とも, み	
\\	監視(かんし): 
\\	視点(してん): 
\\	重視(じゅうし): 
\\	視野(しや): 
\\	視察(しさつ): 
\\	無視(むし): 
\\	視覚(しかく): 
\\	近視(きんし): 
\\	示: 
\\	見: 
\\	見	
\\	みる	視	形声 
\\	覧	ラン	み.る		
\\	展覧会(てんらんかい): 
\\	閲覧(えつらん): 
\\	ご覧になる(ごらんになる): 
\\	観覧(かんらん): 
\\	回覧(かいらん): 
\\	ご覧(ごらん): 
\\	御覧なさい(ごらんなさい): 
\\	臣: 
\\	見: 
\\	見	
\\	みる	覽	会意 
\\	討	トウ	う.つ		
\\	検討(けんとう): 
\\	討論(とうろん): 
\\	討議(とうぎ): 
\\	言: 
\\	寸: 
\\	言	
\\	こと・げん・ごんべん・ことば	
\\	形声 
\\	訪	ホウ	おとず.れる、たず.ねる、と.う	わ	
\\	訪問(ほうもん): 
\\	訪ねる(たずねる): 
\\	訪れる(おとずれる): 
\\	言: 
\\	方: 
\\	言	
\\	こと・げん・ごんべん・ことば	
\\	形声 
\\	訳	ヤク	わけ		
\\	"翻訳(ほんやく): 
\\	通訳(つうやく): 
\\	言い訳(いいわけ): 
\\	申し訳(もうしわけ): 
\\	訳(わけ): 
\\	内訳(うちわけ): 
\\	訳す(やくす): 
\\	申し訳ない(もうしわけない): 
\\	訳(やく): 
\\	言: 
\\	尺: 
\\	言	
\\	こと・げん・ごんべん・ことば	譯	形声 
\\	詞	シ			
\\	代名詞(だいめいし): 
\\	形容詞(けいようし): 
\\	動詞(どうし): 
\\	名詞(めいし): 
\\	助詞(じょし): 
\\	副詞(ふくし): 
\\	台詞(せりふ): 
\\	形容動詞(けいようどうし): 
\\	自動詞(じどうし): 
\\	助動詞(じょどうし): 
\\	数詞(すうし): 
\\	接続詞(せつぞくし): 
\\	他動詞(たどうし): 
\\	言: 
\\	司: 
\\	言	
\\	こと・げん・ごんべん・ことば	
\\	形声 
\\	誌	シ			
\\	雑誌(ざっし): 
\\	言: 
\\	志: 
\\	言	
\\	こと・げん・ごんべん・ことば	
\\	形声 
\\	認	ニン	みと.める、したた.める		
\\	承認(しょうにん): 
\\	認識(にんしき): 
\\	確認(かくにん): 
\\	公認(こうにん): 
\\	認める(みとめる): 
\\	認める(したためる): 
\\	言: 
\\	忍: 
\\	言	
\\	こと・げん・ごんべん・ことば	
\\	形声 
\\	誕	タン			
\\	誕生(たんじょう): 
\\	誕生日(たんじょうび): 
\\	言: 
\\	延: 
\\	言	
\\	こと・げん・ごんべん・ことば	
\\	形声 
\\	誠	セイ	まこと	きよ, さと, しげ, とも, のぶ, ま, まこ, まさ	
\\	誠実(せいじつ): 
\\	誠(まこと): 
\\	言: 
\\	成: 
\\	言	
\\	こと・げん・ごんべん・ことば	
\\	形声 
\\	誤	ゴ	あやま.る、-あやま.る		
\\	誤り(あやまり): 
\\	誤解(ごかい): 
\\	誤差(ごさ): 
\\	錯誤(さくご): 
\\	誤る(あやまる): 
\\	誤魔化す(ごまかす): 
\\	間誤付く(まごつく): 
\\	言: 
\\	呉: 
\\	言	
\\	こと・げん・ごんべん・ことば	
\\	形声 
\\	論	ロン			
\\	世論(よろん): 
\\	議論(ぎろん): 
\\	結論(けつろん): 
\\	論文(ろんぶん): 
\\	討論(とうろん): 
\\	理論(りろん): 
\\	論争(ろんそう): 
\\	論理(ろんり): 
\\	言論(げんろん): 
\\	評論(ひょうろん): 
\\	弁論(べんろん): 
\\	無論(むろん): 
\\	論議(ろんぎ): 
\\	異論(いろん): 
\\	論じる(ろんじる): 
\\	論ずる(ろんずる): 
\\	概論(がいろん): 
\\	勿論(もちろん): 
\\	目論見(もくろみ): 
\\	論(ろん): 
\\	言: 
\\	侖: 
\\	言	
\\	こと・げん・ごんべん・ことば	
\\	形声 
\\	諸	ショ	もろ		
\\	諸君(しょくん): 
\\	諸(しょ): 
\\	諸に(もろに): 
\\	言: 
\\	者: 
\\	言	
\\	こと・げん・ごんべん・ことば	諸	形声 
\\	警	ケイ	いまし.める		
\\	警察(けいさつ): 
\\	警戒(けいかい): 
\\	警官(けいかん): 
\\	警告(けいこく): 
\\	警備(けいび): 
\\	警部(けいぶ): 
\\	敬: 
\\	言: 
\\	言	
\\	こと・げん・ごんべん・ことば	
\\	形声 
\\	貴	キ	たっと.い、とうと.い、たっと.ぶ、とうと.ぶ	きよ, ぎ, たか, たかし, よし	
\\	貴重(きちょう): 
\\	貴族(きぞく): 
\\	貴方(あなた): 
\\	中: 
\\	貝: 
\\	貝	
\\	かい・かいへん・こがい	
\\	会意 
\\	賃	チン		すけ	
\\	賃金(ちんぎん): 
\\	運賃(うんちん): 
\\	家賃(やちん): 
\\	任: 
\\	貝: 
\\	貝	
\\	かい・かいへん・こがい	
\\	形声 
\\	遺	イ、ユイ	のこ.す		
\\	遺跡(いせき): 
\\	貴: 
\\	辶: 
\\	辵	
\\	しんにょう・しんにゅう	
\\	形声 
\\	郵	ユウ			
\\	郵便(ゆうびん): 
\\	郵便局(ゆうびんきょく): 
\\	郵送(ゆうそう): 
\\	垂: 
\\	邑: 
\\	邑	
\\	むら・おおざと・おおざとづくり・ゆう	
\\	会意 
\\	郷	キョウ、ゴウ	さと	くに	
\\	郷里(きょうり): 
\\	郷愁(きょうしゅう): 
\\	郷土(きょうど): 
\\	幺: 
\\	艮: 
\\	邑: 
\\	邑	
\\	むら・おおざと・おおざとづくり・ゆう	鄕	会意 
\\	針	シン	はり	は	
\\	方針(ほうしん): 
\\	針(はり): 
\\	針金(はりがね): 
\\	針路(しんろ): 
\\	金: 
\\	十: 
\\	金	
\\	かね・かねへん	
\\	形声 
\\	鋼	コウ	はがね		
\\	鉄鋼(てっこう): 
\\	金: 
\\	岡: 
\\	金	
\\	かね・かねへん	
\\	形声 
\\	閉	ヘイ	と.じる、と.ざす、し.める、し.まる、た.てる	へ	
\\	閉鎖(へいさ): 
\\	閉会(へいかい): 
\\	閉じる(とじる): 
\\	閉まる(しまる): 
\\	閉める(しめる): 
\\	閉口(へいこう): 
\\	門: 
\\	才: 
\\	門	
\\	もん・もんがまえ・かどがまえ	
\\	会意 
\\	閣	カク			
\\	内閣(ないかく): 
\\	門: 
\\	各: 
\\	門	
\\	もん・もんがまえ・かどがまえ	
\\	形声 
\\	降	コウ、ゴ	お.りる、お.ろす、ふ.る、ふ.り、くだ.る、くだ.す	ふり, ふる	
\\	以降(いこう): 
\\	下降(かこう): 
\\	降伏(こうふく): 
\\	降る(ふる): 
\\	降水(こうすい): 
\\	降り出す(ふりだす): 
\\	阜: 
\\	夅: 
\\	阜	
\\	こざと・こざとへん・おか・ぎふのふ	
\\	会意 
\\	陛	ヘイ			
\\	阜: 
\\	比: 
\\	土: 
\\	阜	
\\	こざと・こざとへん・おか・ぎふのふ	
\\	形声 
\\	除	ジョ、ジ	のぞ.く、-よ.け		
\\	解除(かいじょ): 
\\	排除(はいじょ): 
\\	掃除(そうじ): 
\\	除外(じょがい): 
\\	削除(さくじょ): 
\\	控除(こうじょ): 
\\	免除(めんじょ): 
\\	取り除く(とりのぞく): 
\\	除く(のぞく): 
\\	阜: 
\\	余: 
\\	阜	
\\	こざと・こざとへん・おか・ぎふのふ	
\\	形声 
\\	障	ショウ	さわ.る		
\\	保障(ほしょう): 
\\	障害(しょうがい): 
\\	故障(こしょう): 
\\	障子(しょうじ): 
\\	気障(きざ): 
\\	障る(さわる): 
\\	阜: 
\\	章: 
\\	阜	
\\	こざと・こざとへん・おか・ぎふのふ	
\\	形声 
\\	難	ナン	かた.い、-がた.い、むずか.しい、むづか.しい、むつか.しい、-にく.い	な, なに	
\\	困難(こんなん): 
\\	避難(ひなん): 
\\	非難(ひなん): 
\\	難しい(むずかしい): 
\\	盗難(とうなん): 
\\	遭難(そうなん): 
\\	無難(ぶなん): 
\\	有難い(ありがたい): 
\\	難(なん): 
\\	難い(にくい): 
\\	災難(さいなん): 
\\	有難うございます(ありがとうございます): 
\\	有り難う(ありがとう): 
\\	どうも有難うございます(どうもありがとうございます): 
\\	艸: 
\\	口: 
\\	夫: 
\\	隹: 
\\	隹	
\\	ふるとり	難	会意 
\\	革	カク	かわ		
\\	改革(かいかく): 
\\	革命(かくめい): 
\\	革新(かくしん): 
\\	革(かわ): 
\\	変革(へんかく): 
\\	つり革(つりかわ): 
\\	廿: 
\\	口: 
\\	革	
\\	かわ・かくのかわ・かわへん・つくりがわ	
\\	象形 
\\	頂	チョウ	いただ.く、いただき		
\\	頂点(ちょうてん): 
\\	頂上(ちょうじょう): 
\\	頂(いただき): 
\\	頂戴(ちょうだい): 
\\	頂きます(いただきます): 
\\	丁: 
\\	頁: 
\\	頁	
\\	おおがい・いちのかい	
\\	形声 
\\	骨	コツ	ほね		
\\	骨(ほね): 
\\	骨折(こっせつ): 
\\	露骨(ろこつ): 
\\	骨(こつ): 
\\	骨董品(こっとうひん): 
\\	月: 
\\	骨	
\\	ほね・ほねへん	
\\	会意 
\\	歳	サイ、セイ	とし、とせ、よわい		
\\	"万歳(ばんざい): 
\\	二十歳(はたち): 
\\	止: 
\\	戍: 
\\	止	
\\	とまる・とめる・とめへん	
\\	象形 
\\	与	ヨ	あた.える、あずか.る、くみ.する、ともに	とも, ゆ	
\\	与党(よとう): 
\\	給与(きゅうよ): 
\\	関与(かんよ): 
\\	与える(あたえる): 
\\	寄与(きよ): 
\\	一: 
\\	臼	
\\	うす	與	会意 
\\	違	イ	ちが.う、ちが.い、ちが.える、-ちが.える、たが.う、たが.える		
\\	違反(いはん): 
\\	違い(ちがい): 
\\	間違い(まちがい): 
\\	相違(そうい): 
\\	食違う(くいちがう): 
\\	違う(ちがう): 
\\	勘違い(かんちがい): 
\\	間違える(まちがえる): 
\\	行き違い(いきちがい): 
\\	間違う(まちがう): 
\\	違いない(ちがいない): 
\\	違える(ちがえる): 
\\	すれ違い(すれちがい): 
\\	すれ違う(すれちがう): 
\\	韋: 
\\	辶: 
\\	辵	
\\	しんにょう・しんにゅう	
\\	形声 
\\	渡	ト	わた.る、-わた.る、わた.す	お, たり, わたな, わたら, わたり	
\\	渡す(わたす): 
\\	橋渡し(はしわたし): 
\\	渡り鳥(わたりどり): 
\\	見渡す(みわたす): 
\\	渡る(わたる): 
\\	水: 
\\	度: 
\\	水	
\\	みず・したみず・さんずい	
\\	形声 
\\	突	トツ、カ	つ.く		
\\	衝突(しょうとつ): 
\\	突破(とっぱ): 
\\	突然(とつぜん): 
\\	煙突(えんとつ): 
\\	突如(とつじょ): 
\\	突っ込む(つっこむ): 
\\	突っ張る(つっぱる): 
\\	突き当たる(つきあたる): 
\\	突き当たり(つきあたり): 
\\	突く(つく): 
\\	突く(つつく): 
\\	突っ突く(つっつく): 
\\	穴: 
\\	大: 
\\	穴	
\\	あな・あなかんむり	突	会意 
\\	捕	ホ	と.らえる、と.らわれる、と.る、とら.える、とら.われる、つか.まえる、つか.まる		
\\	逮捕(たいほ): 
\\	捕虜(ほりょ): 
\\	捕える(とらえる): 
\\	捕鯨(ほげい): 
\\	捕獲(ほかく): 
\\	捕る(とる): 
\\	捕まえる(つかまえる): 
\\	捕吏(ほり): 
\\	手: 
\\	甫: 
\\	手	
\\	て・てへん	
\\	形声 
\\	迎	ゲイ	むか.える	むかえ	
\\	歓迎(かんげい): 
\\	迎え(むかえ): 
\\	出迎え(でむかえ): 
\\	迎える(むかえる): 
\\	出迎える(でむかえる): 
\\	卬: 
\\	辶: 
\\	辵	
\\	しんにょう・しんにゅう	
\\	形声 
\\	彼	ヒ	かれ、かの、か.の	その	
\\	彼ら(かれら): 
\\	彼女(かのじょ): 
\\	彼方此方(あちらこちら): 
\\	彼方(あちら): 
\\	彼方(かなた): 
\\	彼処(あそこ): 
\\	彼の(あの): 
\\	彼(あれ): 
\\	彼(かれ): 
\\	彼此(あれこれ): 
\\	彳: 
\\	皮: 
\\	彳	
\\	ぎょうにんべん	
\\	仮借 
\\	舞	ブ	ま.う、-ま.う、まい		
\\	舞台(ぶたい): 
\\	見舞う(みまう): 
\\	見舞い(みまい): 
\\	舞う(まう): 
\\	振る舞う(ふるまう): 
\\	仕舞う(しまう): 
\\	お見舞い(おみまい): 
\\	仕舞い(しまい): 
\\	丿: 
\\	一: 
\\	舛: 
\\	舛	
\\	ます・まいあし	
\\	会意 
\\	込		こ.む、-こ.む、こ.み、-こ.み、こ.める	こみ, ごめ	
\\	申込み(もうしこみ): 
\\	見込み(みこみ): 
\\	追い込む(おいこむ): 
\\	組み込む(くみこむ): 
\\	落ち込む(おちこむ): 
\\	打ち込む(うちこむ): 
\\	意気込む(いきごむ): 
\\	突っ込む(つっこむ): 
\\	押し込む(おしこむ): 
\\	申し込む(もうしこむ): 
\\	飛び込む(とびこむ): 
\\	飲み込む(のみこむ): 
\\	割り込む(わりこむ): 
\\	払い込む(はらいこむ): 
\\	引っ込む(ひっこむ): 
\\	乗り込む(のりこむ): 
\\	思い込む(おもいこむ): 
\\	溶け込む(とけこむ): 
\\	込める(こめる): 
\\	踏み込む(ふみこむ): 
\\	放り込む(ほうりこむ): 
\\	埋め込む(うめこむ): 
\\	入: 
\\	辶: 
\\	辵	
\\	しんにょう・しんにゅう	
\\	国字 
\\	頼	ライ	たの.む、たの.もしい、たよ.る	よち, より	
\\	信頼(しんらい): 
\\	依頼(いらい): 
\\	頼む(たのむ): 
\\	頼み(たのみ): 
\\	頼る(たよる): 
\\	頼もしい(たのもしい): 
\\	束: 
\\	頁: 
\\	貝	
\\	かい・かいへん・こがい	賴	形声 
\\	途	ト	みち		
\\	途中(とちゅう): 
\\	用途(ようと): 
\\	前途(ぜんと): 
\\	中途(ちゅうと): 
\\	途上(とじょう): 
\\	途絶える(とだえる): 
\\	途端(とたん): 
\\	途切れる(とぎれる): 
\\	余: 
\\	辶: 
\\	辵	
\\	しんにょう・しんにゅう	
\\	形声 
\\	抜	バツ、ハツ、ハイ	ぬ.く、-ぬ.く、ぬ.き、ぬ.ける、ぬ.かす、ぬ.かる	ぬき	
\\	抜け出す(ぬけだす): 
\\	海抜(かいばつ): 
\\	抜かす(ぬかす): 
\\	抜く(ぬく): 
\\	抜ける(ぬける): 
\\	手: 
\\	友: 
\\	手	
\\	て・てへん	拔	形声 
\\	婚	コン			
\\	結婚(けっこん): 
\\	離婚(りこん): 
\\	婚約(こんやく): 
\\	未婚(みこん): 
\\	新婚(しんこん): 
\\	既婚(きこん): 
\\	女: 
\\	昏: 
\\	女	
\\	おんな・おんなへん	
\\	形声 
\\	浮	フ	う.く、う.かれる、う.かぶ、む、う.かべる	うき	
\\	浮かべる(うかべる): 
\\	浮気(うわき): 
\\	浮かぶ(うかぶ): 
\\	浮く(うく): 
\\	浮力(ふりょく): 
\\	水: 
\\	孚: 
\\	水	
\\	みず・したみず・さんずい	
\\	形声 
\\	押	オウ	お.す、お.し-、お.っ-、お.さえる、おさ.える	おし, おす, おや	
\\	押す(おす): 
\\	押し入れ(おしいれ): 
\\	押し切る(おしきる): 
\\	押し込む(おしこむ): 
\\	押し寄せる(おしよせる): 
\\	手: 
\\	甲: 
\\	手	
\\	て・てへん	
\\	形声 
\\	倒	トウ	たお.れる、-だお.れ、たお.す、さかさま、さかさ、さかしま		
\\	倒さ(さかさ): 
\\	倒産(とうさん): 
\\	面倒(めんどう): 
\\	倒れる(たおれる): 
\\	圧倒(あっとう): 
\\	倒す(たおす): 
\\	面倒くさい(めんどうくさい): 
\\	人: 
\\	到: 
\\	人	
\\	ひと・にんべん・ひとやね	
\\	形声 
\\	払	フツ、ヒツ、ホツ	はら.う、-はら.い、-ばら.い	はらい	
\\	支払い(しはらい): 
\\	支払う(しはらう): 
\\	払う(はらう): 
\\	酔っ払い(よっぱらい): 
\\	払い込む(はらいこむ): 
\\	払い戻す(はらいもどす): 
\\	手: 
\\	厶: 
\\	手	
\\	て・てへん	拂	形声 
\\	遅	チ	おく.れる、おく.らす、おそ.い	じ	
\\	"遅れ(おくれ): 
\\	遅い(おそい): 
\\	遅刻(ちこく): 
\\	手遅れ(ておくれ): 
\\	遅れる(おくれる): 
\\	遅らす(おくらす): 
\\	遅くとも(おそくとも): 
\\	尸: 
\\	羊: 
\\	辶: 
\\	辵	
\\	しんにょう・しんにゅう	遲	形声 
\\	更	コウ	さら、さら.に、ふ.ける、ふ.かす		
\\	変更(へんこう): 
\\	更に(さらに): 
\\	夜更かし(よふかし): 
\\	更ける(ふける): 
\\	今更(いまさら): 
\\	尚更(なおさら): 
\\	夜更け(よふけ): 
\\	一: 
\\	日: 
\\	乂: 
\\	曰	
\\	ひらび・いわく	
\\	会意 
\\	抱	ホウ	だ.く、いだ.く、かか.える	たば	
\\	抱える(かかえる): 
\\	辛抱(しんぼう): 
\\	抱く(いだく): 
\\	抱っこ(だっこ): 
\\	介抱(かいほう): 
\\	手: 
\\	包: 
\\	手	
\\	て・てへん	
\\	形声 
\\	恐	キョウ	おそ.れる、おそ.る、おそ.ろしい、こわ.い、こわ.がる		
\\	恐れ(おそれ): 
\\	恐怖(きょうふ): 
\\	恐らく(おそらく): 
\\	恐れる(おそれる): 
\\	恐ろしい(おそろしい): 
\\	恐縮(きょうしゅく): 
\\	恐れ入る(おそれいる): 
\\	工: 
\\	凡: 
\\	心: 
\\	心	
\\	こころ・したごころ・りっしんべん	
\\	形声 
\\	戻	レイ	もど.す、もど.る		
\\	戻す(もどす): 
\\	取り戻す(とりもどす): 
\\	戻る(もどる): 
\\	払い戻す(はらいもどす): 
\\	戸: 
\\	大: 
\\	戸	
\\	と・とだれ・とかんむり	戾	会意 
\\	越	エツ、オツ	こ.す、-こ.す、-ご.し、こ.える、-ご.え	えち, えっ, お, こえ, こし, ごえ, ごし, ごや	
\\	引っ越し(ひっこし): 
\\	引っ越す(ひっこす): 
\\	越す(こす): 
\\	乗り越し(のりこし): 
\\	追い越す(おいこす): 
\\	寄越す(よこす): 
\\	優越(ゆうえつ): 
\\	走: 
\\	戉: 
\\	走	
\\	はしる・そうにょう	
\\	形声 
\\	互	ゴ	たが.い、かたみ.に		
\\	相互(そうご): 
\\	互い(たがい): 
\\	交互(こうご): 
\\	お互い(おたがい): 
\\	二: 
\\	彑: 
\\	二	
\\	に・ふた	
\\	象形 
\\	逃	トウ	に.げる、に.がす、のが.す、のが.れる		
\\	逃げる(にげる): 
\\	逃走(とうそう): 
\\	逃れる(のがれる): 
\\	逃亡(とうぼう): 
\\	逃げ出す(にげだす): 
\\	逃がす(にがす): 
\\	見逃す(みのがす): 
\\	逃す(のがす): 
\\	兆: 
\\	辶: 
\\	辵	
\\	しんにょう・しんにゅう	
\\	形声 
\\	緒	ショ、チョ	お、いとぐち	ほ	
\\	情緒(じょうしょ): 
\\	一緒(いっしょ): 
\\	内緒(ないしょ): 
\\	糸: 
\\	者: 
\\	糸	
\\	いと・いとへん	緖	形声 
\\	掛	カイ、ケイ	か.ける、-か.ける、か.け、-か.け、-が.け、か.かる、-か.かる、-が.かる、か.かり、-が.かり、かかり、-がかり	かけ	
\\	切掛(きっかけ): 
\\	手掛かり(てがかり): 
\\	寄り掛かる(よりかかる): 
\\	掛ける(かける): 
\\	掛け(かけ): 
\\	引っ掛る(ひっかかる): 
\\	腰掛(こしかけ): 
\\	心掛け(こころがけ): 
\\	仕掛け(しかけ): 
\\	出掛ける(でかける): 
\\	仕掛ける(しかける): 
\\	話し掛ける(はなしかける): 
\\	腰掛ける(こしかける): 
\\	引っ掛ける(ひっかける): 
\\	差し掛かる(さしかかる): 
\\	通り掛かる(とおりかかる): 
\\	掛け算(かけざん): 
\\	手掛ける(てがける): 
\\	お出掛け(おでかけ): 
\\	お掛け下さい(おかけください): 
\\	手: 
\\	卦: 
\\	手	
\\	て・てへん	
\\	形声 
\\	到	トウ	いた.る		
\\	到着(とうちゃく): 
\\	到達(とうたつ): 
\\	到底(とうてい): 
\\	至: 
\\	刀: 
\\	刀	
\\	かたな・りっとう	
\\	会意 
\\	寝	シン	ね.る、ね.かす、い.ぬ、みたまや、や.める		
\\	寝る(ねる): 
\\	昼寝(ひるね): 
\\	寝台(しんだい): 
\\	寝巻き(ねまき): 
\\	寝かせる(ねかせる): 
\\	寝坊(ねぼう): 
\\	朝寝坊(あさねぼう): 
\\	転寝(うたたね): 
\\	宀: 
\\	爿: 
\\	彑: 
\\	冖: 
\\	又: 
\\	宀	
\\	うかんむり	寢	会意 
\\	盗	トウ	ぬす.む、ぬす.み		
\\	強盗(ごうとう): 
\\	盗む(ぬすむ): 
\\	盗難(とうなん): 
\\	盗み(ぬすみ): 
\\	次: 
\\	皿: 
\\	皿	
\\	さら	盜	会意 
\\	御	ギョ、ゴ	おん-、お-、み-	う	
\\	御(ご): 
\\	御免(ごめん): 
\\	御襁褓(おむつ): 
\\	御座います(ございます): 
\\	御洒落(おしゃれ): 
\\	御中(おんちゅう): 
\\	御数(おかず): 
\\	御免なさい(ごめんなさい): 
\\	御覧なさい(ごらんなさい): 
\\	晩御飯(ばんごはん): 
\\	御名(みな): 
\\	御(おん): 
\\	彳: 
\\	卸: 
\\	彳	
\\	ぎょうにんべん	
\\	形声 
\\	吹	スイ	ふ.く	ふき	
\\	吹雪(ふぶき): 
\\	吹く(ふく): 
\\	吹奏(すいそう): 
\\	口: 
\\	欠: 
\\	口	
\\	くち・くちへん	
\\	会意 
\\	娘	ジョウ	むすめ、こ		
\\	娘(むすめ): 
\\	女: 
\\	良: 
\\	女	
\\	おんな・おんなへん	
\\	形声 
\\	怒	ド、ヌ	いか.る、おこ.る		
\\	怒り(いかり): 
\\	怒る(おこる): 
\\	怒鳴る(どなる): 
\\	奴: 
\\	心: 
\\	心	
\\	こころ・したごころ・りっしんべん	
\\	形声 
\\	杯	ハイ	さかずき		
\\	一杯(いっぱい): 
\\	乾杯(かんぱい): 
\\	杯(さかずき): 
\\	杯(はい): 
\\	木: 
\\	不: 
\\	木	
\\	き・きへん	
\\	形声 
\\	疲	ヒ	つか.れる、-づか.れ、つか.らす		
\\	疲れ(つかれ): 
\\	疲労(ひろう): 
\\	疲れる(つかれる): 
\\	疒: 
\\	皮: 
\\	疒	
\\	やまい・やまいだれ	
\\	形声 
\\	皆	カイ	みな、みんな	むな	
\\	皆(みな): 
\\	皆さん(みなさん): 
\\	比: 
\\	日: 
\\	白	
\\	しろ・しろへん	
\\	会意 
\\	煙	エン	けむ.る、けむり、けむ.い	たば	
\\	煙(けむり): 
\\	禁煙(きんえん): 
\\	煙突(えんとつ): 
\\	煙草(タバコ): 
\\	煙い(けむい): 
\\	煙たい(けむたい): 
\\	煙る(けむる): 
\\	火: 
\\	襾: 
\\	土: 
\\	火	
\\	ひ・ひへん・れっか・れんが	
\\	形声 
\\	眠	ミン	ねむ.る、ねむ.い	ね	
\\	睡眠(すいみん): 
\\	居眠り(いねむり): 
\\	眠い(ねむい): 
\\	冬眠(とうみん): 
\\	眠る(ねむる): 
\\	眠たい(ねむたい): 
\\	目: 
\\	民: 
\\	目	
\\	め・めへん・よこめ	
\\	形声 
\\	怖	フ、ホ	こわ.い、こわ.がる、お.じる、おそ.れる		
\\	恐怖(きょうふ): 
\\	怖い(こわい): 
\\	心: 
\\	布: 
\\	心	
\\	こころ・したごころ・りっしんべん	
\\	形声 
\\	髪	ハツ	かみ	がた, ひげ	
\\	白髪(しらが): 
\\	髪(かみ): 
\\	髪の毛(かみのけ): 
\\	髟: 
\\	友: 
\\	髟	
\\	かみがしら・かみかんむり	髮	形声 
\\	忙	ボウ、モウ	いそが.しい、せわ.しい、おそ.れる、うれえるさま		
\\	忙しい(いそがしい): 
\\	多忙(たぼう): 
\\	心: 
\\	亡: 
\\	心	
\\	こころ・したごころ・りっしんべん	
\\	形声 
\\	靴	カ	くつ		
\\	靴(くつ): 
\\	靴下(くつした): 
\\	革: 
\\	化: 
\\	革	
\\	かわ・かくのかわ・かわへん・つくりがわ	
\\	形声 
\\	恥	チ	は.じる、はじ、は.じらう、は.ずかしい		
\\	恥(はじ): 
\\	恥ずかしい(はずかしい): 
\\	恥じる(はじる): 
\\	恥じらう(はじらう): 
\\	耳: 
\\	心: 
\\	心	
\\	こころ・したごころ・りっしんべん	
\\	会意 
\\	偶	グウ	たま		
\\	配偶者(はいぐうしゃ): 
\\	偶然(ぐうぜん): 
\\	偶数(ぐうすう): 
\\	偶々(たまたま): 
\\	偶に(たまに): 
\\	偶(たま): 
\\	人: 
\\	禺: 
\\	人	
\\	ひと・にんべん・ひとやね	
\\	形声 
\\	偉	イ	えら.い	ひで	
\\	偉大(いだい): 
\\	偉い(えらい): 
\\	人: 
\\	韋: 
\\	人	
\\	ひと・にんべん・ひとやね	
\\	形声 
\\	猫	ビョウ	ねこ		
\\	猫(ねこ): 
\\	犬: 
\\	苗: 
\\	犬	
\\	いぬ・けものへん	
\\	形声 
\\	幾	キ	いく-、いく.つ、いく.ら	い, く	
\\	幾つ(いくつ): 
\\	幾分(いくぶん): 
\\	幾ら(いくら): 
\\	幾(いく): 
\\	幾多(いくた): 
\\	幾らでも(いくらでも): 
\\	幺: 
\\	幺: 
\\	戌: 
\\	幺	
\\	いとがしら・よう	
\\	会意 
\\	環	カン	わ	たま, たまき	
\\	環境(かんきょう): 
\\	循環(じゅんかん): 
\\	玉: 
\\	睘: 
\\	玉	
\\	たま・たまへん ・ おう・おうへん・あうさま	
\\	形声 
\\	欧	オウ	うた.う、は.く	おお, ひろ	
\\	欧米(おうべい): 
\\	欧羅巴(ヨーロッパ): 
\\	区: 
\\	欠: 
\\	欠	
\\	あくび・けんづくり・かける・けつ	歐	会意 
\\	被	ヒ	こうむ.る、おお.う、かぶ.る、かぶ.せる	ぎぬ	
\\	被害(ひがい): 
\\	被る(かぶる): 
\\	被せる(かぶせる): 
\\	被(ひ): 
\\	衣: 
\\	皮: 
\\	衣	
\\	ころも・ころもへん	
\\	形声 
\\	含	ガン	ふく.む、ふく.める		
\\	含む(ふくむ): 
\\	含める(ふくめる): 
\\	含嗽(うがい): 
\\	今: 
\\	口: 
\\	口	
\\	くち・くちへん	
\\	会意 
\\	況	キョウ	まし.て、いわ.んや、おもむき		
\\	状況(じょうきょう): 
\\	不況(ふきょう): 
\\	好況(こうきょう): 
\\	況して(まして): 
\\	水: 
\\	兄: 
\\	水	
\\	みず・したみず・さんずい	
\\	形声 
\\	湾	ワン	いりえ		
\\	湾(わん): 
\\	水: 
\\	弯: 
\\	水	
\\	みず・したみず・さんずい	灣	形声 
\\	捜	ソウ、シュ、シュウ	さが.す		
\\	捜査(そうさ): 
\\	捜索(そうさく): 
\\	手: 
\\	叟: 
\\	手	
\\	て・てへん	搜	形声 
\\	超	チョウ	こ.える、こ.す	まさる, わたる	
\\	超える(こえる): 
\\	超過(ちょうか): 
\\	超(ちょう): 
\\	走: 
\\	召: 
\\	走	
\\	はしる・そうにょう	
\\	形声 
\\	療	リョウ			
\\	医療(いりょう): 
\\	治療(ちりょう): 
\\	診療(しんりょう): 
\\	疒: 
\\	尞: 
\\	疒	
\\	やまい・やまいだれ	
\\	形声 
\\	介	カイ		すけ, ゆき	
\\	紹介(しょうかい): 
\\	厄介(やっかい): 
\\	介入(かいにゅう): 
\\	介護(かいご): 
\\	介抱(かいほう): 
\\	人: 
\\	丿: 
\\	丨: 
\\	人	
\\	ひと・にんべん・ひとやね	
\\	象形 
\\	販	ハン			
\\	販売(はんばい): 
\\	貝: 
\\	反: 
\\	貝	
\\	かい・かいへん・こがい	
\\	形声 
\\	幅	フク	はば		
\\	大幅(おおはば): 
\\	幅(はば): 
\\	巾: 
\\	畐: 
\\	巾	
\\	はば・はばへん・きんべん	
\\	形声 
\\	般	ハン			
\\	一般(いっぱん): 
\\	全般(ぜんぱん): 
\\	舟: 
\\	殳: 
\\	舟	
\\	ふね・ふねへん	
\\	会意 
\\	換	カン	か.える、-か.える、か.わる		
\\	換える(かえる): 
\\	乗換え(のりかえ): 
\\	交換(こうかん): 
\\	換算(かんさん): 
\\	換気(かんき): 
\\	乗り換える(のりかえる): 
\\	転換(てんかん): 
\\	手: 
\\	奐: 
\\	手	
\\	て・てへん	
\\	形声 
\\	占	セン	し.める、うらな.う	うら, しむ, じめ	
\\	占領(せんりょう): 
\\	独占(どくせん): 
\\	占う(うらなう): 
\\	占める(しめる): 
\\	占めた(しめた): 
\\	卜: 
\\	口: 
\\	卜	
\\	ぼく・ぼくづくり・ぼくのと・と・うらない	
\\	会意 
\\	伸	シン	の.びる、の.ばす、の.べる、の.す	のぶ, よぼる	
\\	"伸びる(のびる): 
\\	欠伸(あくび): 
\\	人: 
\\	申: 
\\	人	
\\	ひと・にんべん・ひとやね	
\\	形声 
\\	爆	バク	は.ぜる		
\\	爆発(ばくはつ): 
\\	原爆(げんばく): 
\\	爆弾(ばくだん): 
\\	爆破(ばくは): 
\\	火: 
\\	暴: 
\\	火	
\\	ひ・ひへん・れっか・れんが	
\\	形声 
\\	普	フ	あまね.く、あまねし	しん, ひろ	
\\	普通(ふつう): 
\\	普及(ふきゅう): 
\\	普段(ふだん): 
\\	普遍(ふへん): 
\\	並: 
\\	日: 
\\	日	
\\	ひ・ひへん・にち・にちへん	
\\	会意 
\\	齢	レイ	よわい、とし		
\\	年齢(ねんれい): 
\\	齒: 
\\	令: 
\\	齒	
\\	は・はへん	齡	形声 
\\	了	リョウ		さとる	
\\	"了承(りょうしょう): 
\\	終了(しゅうりょう): 
\\	了解(りょうかい): 
\\	修了(しゅうりょう): 
\\	完了(かんりょう): 
\\	了(りょう): 
\\	亅: 
\\	亅	
\\	はねぼう	
\\	象形 
\\	絡	ラク	から.む、から.まる		
\\	連絡(れんらく): 
\\	絡む(からむ): 
\\	糸: 
\\	各: 
\\	糸	
\\	いと・いとへん	
\\	形声 
\\	募	ボ	つの.る		
\\	応募(おうぼ): 
\\	募集(ぼしゅう): 
\\	募る(つのる): 
\\	募金(ぼきん): 
\\	公募(こうぼ): 
\\	莫: 
\\	力: 
\\	力	
\\	ちから・りきづくり	
\\	形声 
\\	昇	ショウ	のぼ.る	しゅう, のり	
\\	上昇(じょうしょう): 
\\	昇進(しょうしん): 
\\	日: 
\\	升: 
\\	日	
\\	ひ・ひへん・にち・にちへん	
\\	形声 
\\	香	コウ、キョウ	か、かお.り、かお.る	こ, こお, ひゃん, よし	
\\	香り(かおり): 
\\	香水(こうすい): 
\\	香辛料(こうしんりょう): 
\\	禾: 
\\	日: 
\\	香	
\\	か・こう・かおり・においこう	
\\	会意 
\\	巨	キョ		おお, か, こ, なお	
\\	巨大(きょだい): 
\\	匚: 
\\	二: 
\\	工	
\\	こう・たくみ・たくみへん・え	
\\	象形 
\\	震	シン	ふる.う、ふる.える		
\\	地震(じしん): 
\\	震える(ふるえる): 
\\	震わせる(ふるわせる): 
\\	雨: 
\\	辰: 
\\	雨	
\\	あめ・あめかんむり・あまかんむり	
\\	形声 
\\	触	ショク	ふ.れる、さわ.る、さわ		
\\	接触(せっしょく): 
\\	感触(かんしょく): 
\\	触る(さわる): 
\\	触れる(ふれる): 
\\	気触れる(かぶれる): 
\\	有り触れる(ありふれる): 
\\	角: 
\\	虫: 
\\	角	
\\	つの・かく・つのへん	觸	形声 
\\	依	イ、エ	よ.る	よ, より	
\\	依然(いぜん): 
\\	依頼(いらい): 
\\	依存(いぞん): 
\\	依ると(よると): 
\\	人: 
\\	衣: 
\\	人	
\\	ひと・にんべん・ひとやね	
\\	会意 
\\	汚	オ	けが.す、けが.れる、けが.らわしい、よご.す、よご.れる、きたな.い		
\\	汚染(おせん): 
\\	汚れ(よごれ): 
\\	汚れる(よごれる): 
\\	汚い(きたない): 
\\	汚す(よごす): 
\\	汚らわしい(けがらわしい): 
\\	汚れ(けがれ): 
\\	汚れる(けがれる): 
\\	水: 
\\	二: 
\\	水	
\\	みず・したみず・さんずい	
\\	形声 
\\	恵	ケイ、エ	めぐ.む、めぐ.み	あや, け, さと, さとし, しげ, へい, み, やす	
\\	知恵(ちえ): 
\\	恩恵(おんけい): 
\\	恵み(めぐみ): 
\\	恵む(めぐむ): 
\\	恵まれる(めぐまれる): 
\\	由: 
\\	心: 
\\	心	
\\	こころ・したごころ・りっしんべん	惠	形声 
\\	傾	ケイ	かたむ.く、かたむ.ける、かたぶ.く、かた.げる、かし.げる		
\\	傾向(けいこう): 
\\	傾斜(けいしゃ): 
\\	傾ける(かたむける): 
\\	傾く(かたむく): 
\\	傾らか(なだらか): 
\\	人: 
\\	頃: 
\\	人	
\\	ひと・にんべん・ひとやね	
\\	形声 
\\	駐	チュウ			
\\	駐車場(ちゅうしゃじょう): 
\\	駐車(ちゅうしゃ): 
\\	馬: 
\\	主: 
\\	馬	
\\	うま・うまへん	
\\	形声 
\\	紹	ショウ		つぐ	
\\	紹介(しょうかい): 
\\	糸: 
\\	召: 
\\	糸	
\\	いと・いとへん	
\\	形声 
\\	雇	コ	やと.う		
\\	雇用(こよう): 
\\	雇う(やとう): 
\\	戸: 
\\	隹: 
\\	隹	
\\	ふるとり	
\\	形声 
\\	替	タイ	か.える、か.え-、か.わる	かえ	
\\	交替(こうたい): 
\\	為替(かわせ): 
\\	両替(りょうがえ): 
\\	着替える(きがえる): 
\\	切り替え(きりかえ): 
\\	取り替える(とりかえる): 
\\	着替え(きがえ): 
\\	立て替える(たてかえる): 
\\	切り替える(きりかえる): 
\\	取り替え(とりかえ): 
\\	夫: 
\\	夫: 
\\	日: 
\\	曰	
\\	ひらび・いわく	
\\	会意 
\\	贈	ゾウ、ソウ	おく.る		
\\	贈物(おくりもの): 
\\	贈る(おくる): 
\\	寄贈(きぞう): 
\\	貝: 
\\	曽: 
\\	貝	
\\	かい・かいへん・こがい	贈	形声 
\\	薄	ハク	うす.い、うす-、-うす、うす.める、うす.まる、うす.らぐ、うす.ら-、うす.れる、すすき		
\\	薄い(うすい): 
\\	薄暗い(うすぐらい): 
\\	薄める(うすめる): 
\\	薄弱(はくじゃく): 
\\	艸: 
\\	溥: 
\\	艸	
\\	くさ・くさかんむり	
\\	形声 
\\	奥	オウ	おく、おく.まる、くま	お, おお, おん, つ, のく	
\\	奥(おく): 
\\	奥さん(おくさん): 
\\	丿: 
\\	冂: 
\\	米: 
\\	大: 
\\	大	
\\	だい・おおきい	奧	会意 
\\	詰	キツ、キチ	つ.める、つ.め、-づ.め、つ.まる、つ.む	ずめ, づめ	
\\	缶詰め(かんづめ): 
\\	見詰める(みつめる): 
\\	詰まる(つまる): 
\\	瓶詰め(びんづめ): 
\\	詰まり(つまり): 
\\	詰らない(つまらない): 
\\	詰める(つめる): 
\\	詰る(なじる): 
\\	言: 
\\	吉: 
\\	言	
\\	こと・げん・ごんべん・ことば	
\\	形声 
\\	双	ソウ	ふた、たぐい、ならぶ、ふたつ	ふ	
\\	双子(ふたご): 
\\	又: 
\\	又: 
\\	隹	
\\	ふるとり	雙	会意 
\\	刺	シ	さ.す、さ.さる、さ.し、さし、とげ		
\\	刺激(しげき): 
\\	名刺(めいし): 
\\	刺身(さしみ): 
\\	刺す(さす): 
\\	刺さる(ささる): 
\\	刺繍(ししゅう): 
\\	朿: 
\\	刀: 
\\	刀	
\\	かたな・りっとう	
\\	形声 
\\	壁	ヘキ	かべ		
\\	"壁(かべ): 
\\	辟: 
\\	土: 
\\	土	
\\	つち・つちへん・どへん	
\\	形声 
\\	悩	ノウ	なや.む、なや.ます、なや.ましい、なやみ		
\\	悩み(なやみ): 
\\	悩む(なやむ): 
\\	悩ます(なやます): 
\\	悩ましい(なやましい): 
\\	心: 
\\	小: 
\\	凶: 
\\	心	
\\	こころ・したごころ・りっしんべん	惱	形声 
\\	荒	コウ	あら.い、あら-、あ.れる、あ.らす、-あ.らし、すさ.む	ら	
\\	荒す(あらす): 
\\	荒れる(あれる): 
\\	荒廃(こうはい): 
\\	荒っぽい(あらっぽい): 
\\	荒い(あらい): 
\\	艸: 
\\	亡: 
\\	巛: 
\\	艸	
\\	くさ・くさかんむり	
\\	形声 
\\	硬	コウ	かた.い		
\\	硬貨(こうか): 
\\	強硬(きょうこう): 
\\	石: 
\\	更: 
\\	石	
\\	いし・いしへん	
\\	形声 
\\	埋	マイ	う.める、う.まる、う.もれる、うず.める、うず.まる、い.ける		
\\	埋める(うめる): 
\\	埋まる(うまる): 
\\	埋蔵(まいぞう): 
\\	埋める(うずめる): 
\\	埋め込む(うめこむ): 
\\	土: 
\\	里: 
\\	土	
\\	つち・つちへん・どへん	
\\	形声 
\\	袋	タイ、ダイ	ふくろ	てい, ない, ぶく	
\\	袋(ふくろ): 
\\	手袋(てぶくろ): 
\\	お袋(おふくろ): 
\\	足袋(たび): 
\\	代: 
\\	衣: 
\\	衣	
\\	ころも・ころもへん	
\\	形声 
\\	封	フウ、ホウ			
\\	封鎖(ふうさ): 
\\	封筒(ふうとう): 
\\	封(ふう): 
\\	同封(どうふう): 
\\	封建(ほうけん): 
\\	圭: 
\\	寸: 
\\	寸	
\\	すん・すんづくり	
\\	会意 
\\	賢	ケン	かしこ.い	かた, さか, さと, さとし, たか, たて, のり, まこと, まさ, まさる	
\\	賢明(けんめい): 
\\	賢い(かしこい): 
\\	臣: 
\\	又: 
\\	貝: 
\\	貝	
\\	かい・かいへん・こがい	
\\	形声 
\\	腕	ワン	うで		
\\	腕(うで): 
\\	腕前(うでまえ): 
\\	肉: 
\\	宛: 
\\	肉	
\\	にく・にくづき	
\\	形声 
\\	較	カク、コウ	くら.べる		
\\	比較(ひかく): 
\\	比較的(ひかくてき): 
\\	車: 
\\	交: 
\\	車	
\\	くるま・くるまへん	
\\	形声 
\\	床	ショウ	とこ、ゆか		
\\	"床(とこ): 
\\	床の間(とこのま): 
\\	起床(きしょう): 
\\	床(ゆか): 
\\	床屋(とこや): 
\\	广: 
\\	木: 
\\	广	
\\	まだれ	
\\	形声 
\\	柔	ジュウ、ニュウ	やわ.らか、やわ.らかい、やわ、やわ.ら	とお	
\\	柔軟(じゅうなん): 
\\	柔道(じゅうどう): 
\\	矛: 
\\	木: 
\\	木	
\\	き・きへん	
\\	会意 
\\	殿	デン、テン	との、-どの	て, どん	
\\	殿(どの): 
\\	宮殿(きゅうでん): 
\\	神殿(しんでん): 
\\	殿様(とのさま): 
\\	殿(しんがり): 
\\	尸: 
\\	共: 
\\	殳: 
\\	殳	
\\	るまた・ほこづくり	
\\	会意 
\\	濃	ノウ	こ.い	の	
\\	"濃度(のうど): 
\\	濃い(こい): 
\\	水: 
\\	農: 
\\	水	
\\	みず・したみず・さんずい	
\\	形声 
\\	肩	ケン	かた		
\\	肩(かた): 
\\	戸: 
\\	月: 
\\	肉	
\\	にく・にくづき	
\\	象形 
\\	零	レイ	ぜろ、こぼ.す、こぼ.れる		
\\	零(ゼロ): 
\\	零す(こぼす): 
\\	零れる(こぼれる): 
\\	零点(れいてん): 
\\	雨: 
\\	令: 
\\	雨	
\\	あめ・あめかんむり・あまかんむり	
\\	形声 
\\	泊	ハク	と.まる、と.める	とまり, はつ	
\\	宿泊(しゅくはく): 
\\	泊まる(とまる): 
\\	泊める(とめる): 
\\	泊(はく): 
\\	水: 
\\	白: 
\\	水	
\\	みず・したみず・さんずい	
\\	形声 
\\	甘	カン	あま.い、あま.える、あま.やかす、うま.い	かも	
\\	甘い(あまい): 
\\	甘える(あまえる): 
\\	甘やかす(あまやかす): 
\\	甘口(あまくち): 
\\	廿: 
\\	甘	
\\	あまい・かん	
\\	象形 
\\	掃	ソウ、シュ	は.く	か	
\\	清掃(せいそう): 
\\	掃除(そうじ): 
\\	掃く(はく): 
\\	手: 
\\	帚: 
\\	手	
\\	て・てへん	
\\	形声 
\\	掘	クツ	ほ.る	ぼり	
\\	発掘(はっくつ): 
\\	掘る(ほる): 
\\	採掘(さいくつ): 
\\	手: 
\\	屈: 
\\	手	
\\	て・てへん	
\\	形声 
\\	軟	ナン	やわ.らか、やわ.らかい		
\\	軟らかい(やわらかい): 
\\	柔軟(じゅうなん): 
\\	車: 
\\	欠: 
\\	車	
\\	くるま・くるまへん	
\\	形声 
\\	沈	チン、ジン	しず.む、しず.める	しん	
\\	沈黙(ちんもく): 
\\	沈没(ちんぼつ): 
\\	沈む(しずむ): 
\\	沈める(しずめる): 
\\	沈澱(ちんでん): 
\\	水: 
\\	冘: 
\\	水	
\\	みず・したみず・さんずい	
\\	形声 
\\	凍	トウ	こお.る、こご.える、こご.る、い.てる、し.みる	こおり	
\\	冷凍(れいとう): 
\\	凍る(こおる): 
\\	凍える(こごえる): 
\\	冫: 
\\	東: 
\\	冫	
\\	にすい	
\\	形声 
\\	恋	レン	こ.う、こい、こい.しい		
\\	恋(こい): 
\\	恋人(こいびと): 
\\	恋愛(れんあい): 
\\	失恋(しつれん): 
\\	恋しい(こいしい): 
\\	恋する(こいする): 
\\	亦: 
\\	心: 
\\	心	
\\	こころ・したごころ・りっしんべん	戀	形声 
\\	郊	コウ			
\\	郊外(こうがい): 
\\	近郊(きんこう): 
\\	交: 
\\	邑: 
\\	邑	
\\	むら・おおざと・おおざとづくり・ゆう	
\\	形声 
\\	腰	ヨウ	こし	うすぐ	
\\	腰(こし): 
\\	腰掛(こしかけ): 
\\	腰掛ける(こしかける): 
\\	肉: 
\\	要: 
\\	肉	
\\	にく・にくづき	
\\	形声 
\\	踊	ヨウ	おど.る		
\\	踊り(おどり): 
\\	踊る(おどる): 
\\	足: 
\\	甬: 
\\	足	
\\	あし・あしへん	
\\	形声 
\\	珍	チン	めずら.しい、たから	じん	
\\	珍しい(めずらしい): 
\\	王: 
\\	人: 
\\	彡: 
\\	玉	
\\	たま・たまへん ・ おう・おうへん・あうさま	
\\	形声 
\\	狭	キョウ、コウ	せま.い、せば.める、せば.まる、さ	はざ	
\\	狭い(せまい): 
\\	犬: 
\\	夾: 
\\	犬	
\\	いぬ・けものへん	狹	形声 
\\	喫	キツ	の.む		
\\	喫茶店(きっさてん): 
\\	喫茶(きっさ): 
\\	口: 
\\	契: 
\\	口	
\\	くち・くちへん	
\\	形声 
\\	溶	ヨウ	と.ける、と.かす、と.く		
\\	溶岩(ようがん): 
\\	溶かす(とかす): 
\\	溶液(ようえき): 
\\	溶け込む(とけこむ): 
\\	溶く(とく): 
\\	溶ける(とける): 
\\	水: 
\\	容: 
\\	水	
\\	みず・したみず・さんずい	
\\	形声 
\\	涙	ルイ、レイ	なみだ		
\\	涙(なみだ): 
\\	水: 
\\	戻: 
\\	水	
\\	みず・したみず・さんずい	淚	形声 
\\	匹	ヒツ	ひき		
\\	匹(ひき): 
\\	匹敵(ひってき): 
\\	匚: 
\\	儿: 
\\	匸	
\\	かくしがまえ	
\\	象形 
\\	鋭	エイ	するど.い	とし	
\\	鋭い(するどい): 
\\	金: 
\\	兌: 
\\	金	
\\	かね・かねへん	
\\	形声 
\\	塗	ト	ぬ.る、ぬ.り、まみ.れる		
\\	塗る(ぬる): 
\\	塗れ(まみれ): 
\\	水: 
\\	余: 
\\	土: 
\\	土	
\\	つち・つちへん・どへん	
\\	形声 
\\	軒	ケン	のき		
\\	軒(のき): 
\\	軒並(のきなみ): 
\\	軒(けん): 
\\	車: 
\\	干: 
\\	車	
\\	くるま・くるまへん	
\\	形声 
\\	叫	キョウ	さけ.ぶ		
\\	叫ぶ(さけぶ): 
\\	叫び(さけび): 
\\	口: 
\\	丩: 
\\	口	
\\	くち・くちへん	
\\	形声 
\\	乾	カン、ケン	かわ.く、かわ.かす、ほ.す、ひ.る、いぬい		
\\	乾燥(かんそう): 
\\	乾電池(かんでんち): 
\\	乾杯(かんぱい): 
\\	乾かす(かわかす): 
\\	乾く(かわく): 
\\	十: 
\\	早: 
\\	乞: 
\\	乙	
\\	おつ・おつにょう・つりばり	
\\	会意 
\\	祈	キ	いの.る	のり, れい	
\\	祈り(いのり): 
\\	祈る(いのる): 
\\	示: 
\\	斤: 
\\	示	
\\	しめす・しめすへん	祈	形声 
\\	汗	カン	あせ		
\\	汗(あせ): 
\\	水: 
\\	干: 
\\	水	
\\	みず・したみず・さんずい	
\\	形声 
\\	湿	シツ、シュウ	しめ.る、しめ.す、うるお.う、うるお.す		
\\	湿度(しつど): 
\\	湿気(しっけ): 
\\	湿る(しめる): 
\\	湿気る(しける): 
\\	水: 
\\	日: 
\\	水	
\\	みず・したみず・さんずい	濕	会意 
\\	咲	ショウ	さ.く、-ざき	さ, さき	
\\	咲く(さく): 
\\	口: 
\\	关: 
\\	口	
\\	くち・くちへん	
\\	形声 
\\	召	ショウ	め.す		
\\	召し上がる(めしあがる): 
\\	召す(めす): 
\\	刀: 
\\	口: 
\\	口	
\\	くち・くちへん	
\\	会意 
\\	脂	シ	あぶら		
\\	脂肪(しぼう): 
\\	脂(あぶら): 
\\	肉: 
\\	旨: 
\\	肉	
\\	にく・にくづき	
\\	形声 
\\	肌	キ	はだ		
\\	肌(はだ): 
\\	肌着(はだぎ): 
\\	肉: 
\\	几: 
\\	肉	
\\	にく・にくづき	
\\	形声 
\\	鈍	ドン	にぶ.い、にぶ.る、にぶ-、なま.る、なまく.ら		
\\	鈍い(のろい): 
\\	鈍感(どんかん): 
\\	鈍る(にぶる): 
\\	金: 
\\	屯: 
\\	金	
\\	かね・かねへん	
\\	形声 
\\	泥	デイ、ナイ、デ、ニ	どろ	ひじ	
\\	泥(どろ): 
\\	泥棒(どろぼう): 
\\	水: 
\\	尼: 
\\	水	
\\	みず・したみず・さんずい	
\\	形声 
\\	隅	グウ	すみ		
\\	隅(すみ): 
\\	阜: 
\\	禺: 
\\	阜	
\\	こざと・こざとへん・おか・ぎふのふ	
\\	形声 
\\	辛	シン	から.い、つら.い、-づら.い、かのと		
\\	辛い(からい): 
\\	香辛料(こうしんりょう): 
\\	辛うじて(かろうじて): 
\\	辛抱(しんぼう): 
\\	塩辛い(しおからい): 
\\	辛い(つらい): 
\\	辛い(づらい): 
\\	立: 
\\	十: 
\\	辛	
\\	しん・からい	
\\	象形 
\\	磨	マ	みが.く、す.る	おさむ	
\\	歯磨き(はみがき): 
\\	磨く(みがく): 
\\	麻: 
\\	石: 
\\	石	
\\	いし・いしへん	
\\	形声 
\\	姓	セイ、ショウ			
\\	姓(せい): 
\\	姓名(せいめい): 
\\	女: 
\\	生: 
\\	女	
\\	おんな・おんなへん	
\\	形声 
\\	筒	トウ	つつ		
\\	封筒(ふうとう): 
\\	筒(つつ): 
\\	水筒(すいとう): 
\\	竹: 
\\	同: 
\\	竹	
\\	たけ・たけかんむり	
\\	形声 
\\	粒	リュウ	つぶ		
\\	粒(つぶ): 
\\	米: 
\\	立: 
\\	米	
\\	こめ・こめへん	
\\	形声 
\\	畳	ジョウ、チョウ	たた.む、たたみ、かさ.なる		
\\	畳(たたみ): 
\\	畳(じょう): 
\\	畳む(たたむ): 
\\	田: 
\\	冖: 
\\	且: 
\\	田	
\\	た・たへん	疊	会意 
\\	膚	フ	はだ		
\\	皮膚(ひふ): 
\\	虍: 
\\	胃: 
\\	肉	
\\	にく・にくづき	
\\	形声 
\\	濯	タク	すす.ぐ、ゆす.ぐ		
\\	洗濯(せんたく): 
\\	濯ぐ(すすぐ): 
\\	水: 
\\	羽: 
\\	隹: 
\\	水	
\\	みず・したみず・さんずい	
\\	形声 
\\	塔	トウ			
\\	塔(とう): 
\\	土: 
\\	荅: 
\\	土	
\\	つち・つちへん・どへん	
\\	形声 
\\	沸	フツ	わ.く、わ.かす		
\\	沸騰(ふっとう): 
\\	沸かす(わかす): 
\\	沸く(わく): 
\\	水: 
\\	弗: 
\\	水	
\\	みず・したみず・さんずい	
\\	形声 
\\	菓	カ			
\\	菓子(かし): 
\\	お菓子(おかし): 
\\	艸: 
\\	果: 
\\	艸	
\\	くさ・くさかんむり	
\\	形声 
\\	帽	ボウ、モウ	ずきん、おお.う		
\\	帽子(ぼうし): 
\\	巾: 
\\	冒: 
\\	巾	
\\	はば・はばへん・きんべん	
\\	形声 
\\	枯	コ	か.れる、か.らす		
\\	枯れる(かれる): 
\\	木: 
\\	古: 
\\	木	
\\	き・きへん	
\\	形声 
\\	涼	リョウ	すず.しい、すず.む、すず.やか、うす.い、ひや.す、まことに	りょ	
\\	涼しい(すずしい): 
\\	涼む(すずむ): 
\\	水: 
\\	京: 
\\	水	
\\	みず・したみず・さんずい	
\\	形声 
\\	舟	シュウ	ふね、ふな-、-ぶね		
\\	舟: 
\\	舟	
\\	ふね・ふねへん	
\\	象形 
\\	符	フ			
\\	切符(きっぷ): 
\\	符号(ふごう): 
\\	竹: 
\\	付: 
\\	竹	
\\	たけ・たけかんむり	
\\	形声 
\\	憎	ゾウ	にく.む、にく.い、にく.らしい、にく.しみ		
\\	憎しみ(にくしみ): 
\\	憎む(にくむ): 
\\	生憎(あいにく): 
\\	憎らしい(にくらしい): 
\\	憎い(にくい): 
\\	心: 
\\	曽: 
\\	心	
\\	こころ・したごころ・りっしんべん	憎	形声 
\\	肯	コウ	がえんじ.る		
\\	肯定(こうてい): 
\\	止: 
\\	月: 
\\	肉	
\\	にく・にくづき	
\\	象形 
\\	燥	ソウ	はしゃ.ぐ		
\\	乾燥(かんそう): 
\\	火: 
\\	品: 
\\	木: 
\\	火	
\\	ひ・ひへん・れっか・れんが	
\\	形声 
\\	畜	チク			
\\	家畜(かちく): 
\\	畜産(ちくさん): 
\\	牧畜(ぼくちく): 
\\	畜生(ちくしょう): 
\\	玄: 
\\	田: 
\\	田	
\\	た・たへん	
\\	会意 
\\	坊	ボウ、ボッ			
\\	赤ん坊(あかんぼう): 
\\	坊さん(ぼうさん): 
\\	坊ちゃん(ぼっちゃん): 
\\	寝坊(ねぼう): 
\\	坊や(ぼうや): 
\\	朝寝坊(あさねぼう): 
\\	土: 
\\	方: 
\\	土	
\\	つち・つちへん・どへん	
\\	形声 
\\	挟	キョウ、ショウ	はさ.む、はさ.まる、わきばさ.む、さしはさ.む		
\\	挟まる(はさまる): 
\\	手: 
\\	夾: 
\\	手	
\\	て・てへん	挾	形声 
\\	曇	ドン	くも.る	ど, ずみ	
\\	曇り(くもり): 
\\	曇る(くもる): 
\\	日: 
\\	雲: 
\\	日	
\\	ひ・ひへん・にち・にちへん	
\\	形声 
\\	滴	テキ	しずく、したた.る		
\\	水滴(すいてき): 
\\	滴(しずく): 
\\	水: 
\\	亠: 
\\	冂: 
\\	古: 
\\	水	
\\	みず・したみず・さんずい	
\\	形声 
\\	伺	シ	うかが.う		
\\	伺う(うかがう): 
\\	人: 
\\	司: 
\\	人	
\\	ひと・にんべん・ひとやね	
\\	形声 
\\	企	キ	くわだ.てる、たくら.む		
\\	企業(きぎょう): 
\\	企画(きかく): 
\\	人: 
\\	止: 
\\	人	
\\	ひと・にんべん・ひとやね	
\\	象形 
\\	藤	トウ、ドウ	ふじ	ぞう, と, ふじゅ	
\\	艸: 
\\	滕: 
\\	艸	
\\	くさ・くさかんむり	
\\	形声 
\\	沢	タク	さわ、うるお.い、うるお.す、つや	おも, さ, さわん, わさ	
\\	光沢(こうたく): 
\\	沢山(たくさん): 
\\	贅沢(ぜいたく): 
\\	水: 
\\	尺: 
\\	水	
\\	みず・したみず・さんずい	澤	形声 
\\	援	エン			
\\	援助(えんじょ): 
\\	応援(おうえん): 
\\	救援(きゅうえん): 
\\	手: 
\\	爰: 
\\	手	
\\	て・てへん	
\\	形声 
\\	施	シ、セ	ほどこ.す		
\\	施設(しせつ): 
\\	実施(じっし): 
\\	施行(しこう): 
\\	施す(ほどこす): 
\\	方: 
\\	也: 
\\	方	
\\	ほう・ほうへん・かた・かたへん	
\\	形声 
\\	井	セイ、ショウ	い	いの, さい, ひ	
\\	天井(てんじょう): 
\\	井戸(いど): 
\\	二: 
\\	廾: 
\\	二	
\\	に・ふた	
\\	象形 
\\	鮮	セン	あざ.やか		
\\	新鮮(しんせん): 
\\	鮮やか(あざやか): 
\\	魚: 
\\	羊: 
\\	魚	
\\	うお・うおへん・さかな	
\\	形声 
\\	監	カン		けん	
\\	監督(かんとく): 
\\	監視(かんし): 
\\	臣: 
\\	皿: 
\\	皿	
\\	さら	
\\	会意 
\\	審	シン	つまび.らか、つぶさ.に	あきら	
\\	審議(しんぎ): 
\\	審判(しんぱん): 
\\	不審(ふしん): 
\\	審査(しんさ): 
\\	宀: 
\\	番: 
\\	宀	
\\	うかんむり	
\\	会意 
\\	訴	ソ	うった.える		
\\	訴訟(そしょう): 
\\	訴え(うったえ): 
\\	訴える(うったえる): 
\\	言: 
\\	斥: 
\\	言	
\\	こと・げん・ごんべん・ことば	
\\	形声 
\\	影	エイ	かげ		
\\	影響(えいきょう): 
\\	撮影(さつえい): 
\\	影(かげ): 
\\	人影(ひとかげ): 
\\	景: 
\\	彡: 
\\	彡	
\\	さん・さんづくり	
\\	会意 
\\	撃	ゲキ	う.つ		
\\	撃つ(うつ): 
\\	攻撃(こうげき): 
\\	打撃(だげき): 
\\	衝撃(しょうげき): 
\\	反撃(はんげき): 
\\	襲撃(しゅうげき): 
\\	車: 
\\	殳: 
\\	手: 
\\	手	
\\	て・てへん	擊	形声 
\\	佐	サ		すけ	
\\	人: 
\\	左: 
\\	人	
\\	ひと・にんべん・ひとやね	
\\	形声 
\\	核	カク			
\\	核(かく): 
\\	結核(けっかく): 
\\	木: 
\\	亥: 
\\	木	
\\	き・きへん	
\\	形声 
\\	融	ユウ	と.ける、と.かす	あきら	
\\	金融(きんゆう): 
\\	融通(ゆうずう): 
\\	融資(ゆうし): 
\\	鬲: 
\\	虫: 
\\	虫	
\\	むし・むしへん	
\\	会意 
\\	渉	ショウ	わた.る	えん	
\\	交渉(こうしょう): 
\\	干渉(かんしょう): 
\\	水: 
\\	歩: 
\\	水	
\\	みず・したみず・さんずい	涉	会意 
\\	響	キョウ	ひび.く		
\\	影響(えいきょう): 
\\	反響(はんきょう): 
\\	響き(ひびき): 
\\	響く(ひびく): 
\\	郷: 
\\	音: 
\\	音	
\\	おと・おとへん	響	形声 
\\	請	セイ、シン、ショウ	こ.う、う.ける	うけ	
\\	要請(ようせい): 
\\	申請(しんせい): 
\\	請求(せいきゅう): 
\\	強請る(ねだる): 
\\	請う(こう): 
\\	言: 
\\	青: 
\\	言	
\\	こと・げん・ごんべん・ことば	
\\	形声 
\\	攻	コウ	せ.める	おさむ	
\\	攻撃(こうげき): 
\\	専攻(せんこう): 
\\	攻める(せめる): 
\\	攻め(せめ): 
\\	攻め手(せめて): 
\\	工: 
\\	攵: 
\\	攴	
\\	ぼくにょう・ぼくづくり・とまた・のぶん・しぶん	
\\	会意 
\\	崎	キ	さき、さい、みさき		
\\	山: 
\\	奇: 
\\	山	
\\	やま・やまへん・やまかんむり	
\\	形声 
\\	督	トク		ただ, ただし	
\\	監督(かんとく): 
\\	叔: 
\\	目: 
\\	目	
\\	め・めへん・よこめ	
\\	形声 
\\	催	サイ	もよう.す、もよお.す		
\\	開催(かいさい): 
\\	主催(しゅさい): 
\\	催し(もよおし): 
\\	催す(もよおす): 
\\	催促(さいそく): 
\\	人: 
\\	崔: 
\\	人	
\\	ひと・にんべん・ひとやね	
\\	形声 
\\	及	キュウ	およ.ぶ、およ.び、および、およ.ぼす	おい, の	
\\	普及(ふきゅう): 
\\	追及(ついきゅう): 
\\	及ぶ(およぶ): 
\\	及ぼす(およぼす): 
\\	及び(および): 
\\	丿: 
\\	又: 
\\	又	
\\	また	
\\	会意 
\\	離	リ	はな.れる、はな.す		
\\	距離(きょり): 
\\	分離(ぶんり): 
\\	離れる(はなれる): 
\\	離婚(りこん): 
\\	離す(はなす): 
\\	离: 
\\	隹: 
\\	隹	
\\	ふるとり	
\\	形声 
\\	摘	テキ	つ.む	つむ	
\\	指摘(してき): 
\\	摘む(つまむ): 
\\	摘む(つむ): 
\\	手: 
\\	亠: 
\\	冂: 
\\	古: 
\\	手	
\\	て・てへん	
\\	形声 
\\	郎	ロウ、リョウ	おとこ	いら, お, とう, もん, ろ, ろお	
\\	良: 
\\	邑: 
\\	邑	
\\	むら・おおざと・おおざとづくり・ゆう	郞	形声 
\\	振	シン	ふ.る、ぶ.る、ふ.り、-ぶ.り、ふ.るう	ふり	
\\	久し振り(ひさしぶり): 
\\	振興(しんこう): 
\\	不振(ふしん): 
\\	振動(しんどう): 
\\	振り(ふり): 
\\	振り返る(ふりかえる): 
\\	振り出し(ふりだし): 
\\	身振り(みぶり): 
\\	振る(ふる): 
\\	振る舞う(ふるまう): 
\\	振り(ぶり): 
\\	振り仮名(ふりがな): 
\\	振り向く(ふりむく): 
\\	手: 
\\	辰: 
\\	手	
\\	て・てへん	
\\	形声 
\\	献	ケン、コン	たてまつ.る		
\\	文献(ぶんけん): 
\\	献立(こんだて): 
\\	貢献(こうけん): 
\\	南: 
\\	犬: 
\\	犬	
\\	いぬ・けものへん	獻	会意 
\\	維	イ		これ, たもつ, つぐ, ゆい, ゆき	
\\	維持(いじ): 
\\	繊維(せんい): 
\\	糸: 
\\	隹: 
\\	糸	
\\	いと・いとへん	
\\	形声 
\\	浜	ヒン	はま		
\\	浜(はま): 
\\	浜辺(はまべ): 
\\	水: 
\\	兵: 
\\	水	
\\	みず・したみず・さんずい	濱	形声 
\\	塁	ルイ、ライ、スイ	とりで	る	
\\	田: 
\\	冫: 
\\	土: 
\\	土	
\\	つち・つちへん・どへん	壘	形声 
\\	邦	ホウ	くに		
\\	連邦(れんぽう): 
\\	丰: 
\\	邑: 
\\	邑	
\\	むら・おおざと・おおざとづくり・ゆう	
\\	形声 
\\	遣	ケン	つか.う、-つか.い、-づか.い、つか.わす、や.る		
\\	派遣(はけん): 
\\	小遣い(こづかい): 
\\	無駄遣い(むだづかい): 
\\	言葉遣い(ことばづかい): 
\\	仮名遣い(かなづかい): 
\\	遣る(やる): 
\\	遣り遂げる(やりとげる): 
\\	遣り通す(やりとおす): 
\\	中: 
\\	辶: 
\\	辵	
\\	しんにょう・しんにゅう	
\\	形声 
\\	抗	コウ	あらが.う		
\\	抵抗(ていこう): 
\\	抗議(こうぎ): 
\\	対抗(たいこう): 
\\	抗争(こうそう): 
\\	反抗(はんこう): 
\\	手: 
\\	亢: 
\\	手	
\\	て・てへん	
\\	形声 
\\	雄	ユウ	お-、おす、おん	かつ, たけ, つよし, ゆ, よう	
\\	雄(おす): 
\\	英雄(えいゆう): 
\\	厷: 
\\	隹: 
\\	隹	
\\	ふるとり	
\\	形声 
\\	緊	キン	し.める、し.まる		
\\	緊急(きんきゅう): 
\\	緊張(きんちょう): 
\\	臣: 
\\	又: 
\\	糸: 
\\	糸	
\\	いと・いとへん	
\\	形声 
\\	廃	ハイ	すた.れる、すた.る		
\\	廃止(はいし): 
\\	廃棄(はいき): 
\\	荒廃(こうはい): 
\\	廃れる(すたれる): 
\\	广: 
\\	発: 
\\	广	
\\	まだれ	廢	形声 
\\	江	コウ	え	くん, とうみ, み, りえ	
\\	水: 
\\	工: 
\\	水	
\\	みず・したみず・さんずい	
\\	形声 
\\	僚	リョウ			
\\	官僚(かんりょう): 
\\	同僚(どうりょう): 
\\	人: 
\\	大: 
\\	日: 
\\	小: 
\\	人	
\\	ひと・にんべん・ひとやね	
\\	形声 
\\	吉	キチ、キツ	よし	え, き, きっ, きる, こし, と, よ	
\\	不吉(ふきつ): 
\\	土: 
\\	口: 
\\	口	
\\	くち・くちへん	
\\	会意 
\\	踏	トウ	ふ.む、ふ.まえる		
\\	踏切り(ふみきり): 
\\	踏む(ふむ): 
\\	踏まえる(ふまえる): 
\\	踏み込む(ふみこむ): 
\\	足: 
\\	沓: 
\\	足	
\\	あし・あしへん	
\\	形声 
\\	壊	カイ、エ	こわ.す、こわ.れる、やぶ.る		
\\	崩壊(ほうかい): 
\\	破壊(はかい): 
\\	壊す(こわす): 
\\	壊れる(こわれる): 
\\	土: 
\\	十: 
\\	网: 
\\	衣: 
\\	土	
\\	つち・つちへん・どへん	壞	形声 
\\	債	サイ			
\\	負債(ふさい): 
\\	人: 
\\	責: 
\\	人	
\\	ひと・にんべん・ひとやね	
\\	形声 
\\	儀	ギ		のり, よし	
\\	儀式(ぎしき): 
\\	礼儀(れいぎ): 
\\	行儀(ぎょうぎ): 
\\	お辞儀(おじぎ): 
\\	人: 
\\	義: 
\\	人	
\\	ひと・にんべん・ひとやね	
\\	形声 
\\	継	ケイ	つ.ぐ、まま-		
\\	継続(けいぞく): 
\\	中継(ちゅうけい): 
\\	受け継ぐ(うけつぐ): 
\\	継ぐ(つぐ): 
\\	跡継ぎ(あとつぎ): 
\\	継ぎ目(つぎめ): 
\\	糸: 
\\	米: 
\\	糸	
\\	いと・いとへん	繼	形声 
\\	闘	トウ	たたか.う、あらそ.う	と	
\\	戦闘(せんとう): 
\\	奮闘(ふんとう): 
\\	門: 
\\	豆: 
\\	寸: 
\\	鬥	
\\	たたかいがまえ・とうがまえ	鬭	形声 
\\	葬	ソウ	ほうむ.る	はふり	
\\	葬式(そうしき): 
\\	葬る(ほうむる): 
\\	艸: 
\\	死: 
\\	廾: 
\\	艸	
\\	くさ・くさかんむり	
\\	会意 
\\	避	ヒ	さ.ける、よ.ける		
\\	避難(ひなん): 
\\	避ける(さける): 
\\	辟: 
\\	辶: 
\\	辵	
\\	しんにょう・しんにゅう	
\\	形声 
\\	逮	タイ			
\\	逮捕(たいほ): 
\\	隶: 
\\	辶: 
\\	辵	
\\	しんにょう・しんにゅう	
\\	形声 
\\	迫	ハク	せま.る	さこ, せ, せこ, はさ, はさま, はざま	
\\	迫る(せまる): 
\\	脅迫(きょうはく): 
\\	迫害(はくがい): 
\\	圧迫(あっぱく): 
\\	白: 
\\	辶: 
\\	辵	
\\	しんにょう・しんにゅう	
\\	形声 
\\	惑	ワク	まど.う		
\\	迷惑(めいわく): 
\\	誘惑(ゆうわく): 
\\	疑惑(ぎわく): 
\\	惑星(わくせい): 
\\	或: 
\\	心: 
\\	心	
\\	こころ・したごころ・りっしんべん	
\\	形声 
\\	崩	ホウ	くず.れる、-くず.れ、くず.す		
\\	崩壊(ほうかい): 
\\	崩れる(くずれる): 
\\	雪崩(なだれ): 
\\	崩す(くずす): 
\\	山: 
\\	朋: 
\\	山	
\\	やま・やまへん・やまかんむり	
\\	形声 
\\	聴	チョウ、テイ	き.く、ゆる.す	きく	
\\	聴覚(ちょうかく): 
\\	聴講(ちょうこう): 
\\	聴診器(ちょうしんき): 
\\	耳: 
\\	十: 
\\	网: 
\\	心: 
\\	耳	
\\	みみ・みみへん	聽	会意 
\\	脱	ダツ	ぬ.ぐ、ぬ.げる		
\\	脱出(だっしゅつ): 
\\	脱退(だったい): 
\\	脱する(だっする): 
\\	脱線(だっせん): 
\\	脱ぐ(ぬぐ): 
\\	肉: 
\\	兌: 
\\	肉	
\\	にく・にくづき	
\\	形声 
\\	患	カン	わずら.う	くろ	
\\	患者(かんじゃ): 
\\	串: 
\\	心: 
\\	心	
\\	こころ・したごころ・りっしんべん	
\\	形声 
\\	締	テイ	し.まる、し.まり、し.める、-し.め、-じ.め		
\\	取締まり(とりしまり): 
\\	締切り(しめきり): 
\\	締め切る(しめきる): 
\\	締める(しめる): 
\\	取り締まる(とりしまる): 
\\	戸締まり(とじまり): 
\\	糸: 
\\	帝: 
\\	糸	
\\	いと・いとへん	
\\	形声 
\\	執	シツ、シュウ	と.る	し	
\\	執筆(しっぴつ): 
\\	執着(しゅうちゃく): 
\\	幸: 
\\	丸: 
\\	土	
\\	つち・つちへん・どへん	
\\	会意 
\\	房	ボウ	ふさ	お, のぶ, わ	
\\	女房(にょうぼう): 
\\	冷房(れいぼう): 
\\	暖房(だんぼう): 
\\	文房具(ぶんぼうぐ): 
\\	戸: 
\\	方: 
\\	戸	
\\	と・とだれ・とかんむり	
\\	形声 
\\	撤	テツ			
\\	手: 
\\	育: 
\\	攵: 
\\	手	
\\	て・てへん	
\\	形声 
\\	削	サク	けず.る、はつ.る、そ.ぐ	げ	
\\	削減(さくげん): 
\\	削除(さくじょ): 
\\	削る(けずる): 
\\	肖: 
\\	刀: 
\\	刀	
\\	かたな・りっとう	
\\	会意 
\\	措	ソ	お.く		
\\	措置(そち): 
\\	手: 
\\	昔: 
\\	手	
\\	て・てへん	
\\	形声 
\\	載	サイ	の.せる、の.る	とし, のり	
\\	掲載(けいさい): 
\\	記載(きさい): 
\\	載る(のる): 
\\	土: 
\\	戈: 
\\	車: 
\\	車	
\\	くるま・くるまへん	
\\	形声 
\\	陣	ジン			
\\	陣(じん): 
\\	阜: 
\\	車: 
\\	阜	
\\	こざと・こざとへん・おか・ぎふのふ	
\\	会意 
\\	為	イ	ため、な.る、な.す、す.る、たり、つく.る、なり	びい	
\\	"行為(こうい): 
\\	為替(かわせ): 
\\	為(ため): 
\\	為す(なす): 
\\	為さる(なさる): 
\\	為る(する): 
\\	所為(せい): 
\\	その為(そのため): 
\\	丶: 
\\	勹: 
\\	火: 
\\	爪	
\\	つめ・つめがしら・つめかんむり・そうにょう・のつ	爲	会意 
\\	抑	ヨク	おさ.える		
\\	抑える(おさえる): 
\\	抑制(よくせい): 
\\	抑圧(よくあつ): 
\\	手: 
\\	卩: 
\\	手	
\\	て・てへん	
\\	会意 
\\	奈	ナ、ナイ、ダイ	いかん、からなし		
\\	大: 
\\	示: 
\\	大	
\\	だい・おおきい	
\\	形声 
\\	択	タク	えら.ぶ		
\\	選択(せんたく): 
\\	採択(さいたく): 
\\	手: 
\\	尺: 
\\	手	
\\	て・てへん	擇	形声 
\\	秀	シュウ	ひい.でる	しゅ, しょう, ひで, ひでし, ほ	
\\	優秀(ゆうしゅう): 
\\	禾: 
\\	乃: 
\\	禾	
\\	のぎ・のぎへん	
\\	象形 
\\	徴	チョウ、チ	しるし		
\\	特徴(とくちょう): 
\\	象徴(しょうちょう): 
\\	徴収(ちょうしゅう): 
\\	彳: 
\\	山: 
\\	玉: 
\\	攵: 
\\	彳	
\\	ぎょうにんべん	徵	会意 
\\	弾	ダン、タン	ひ.く、-ひ.き、はず.む、たま、はじ.く、はじ.ける、ただ.す、はじ.きゆみ		
\\	爆弾(ばくだん): 
\\	弾力(だんりょく): 
\\	弾む(はずむ): 
\\	弾く(はじく): 
\\	弾く(ひく): 
\\	弓: 
\\	単: 
\\	弓	
\\	ゆみ・ゆみへん	彈	形声 
\\	償	ショウ	つぐな.う		
\\	賠償(ばいしょう): 
\\	補償(ほしょう): 
\\	弁償(べんしょう): 
\\	人: 
\\	賞: 
\\	人	
\\	ひと・にんべん・ひとやね	
\\	形声 
\\	拠	キョ、コ	よ.る		
\\	根拠(こんきょ): 
\\	証拠(しょうこ): 
\\	手: 
\\	処: 
\\	手	
\\	て・てへん	據	形声 
\\	拒	キョ、ゴ	こば.む		
\\	拒否(きょひ): 
\\	拒絶(きょぜつ): 
\\	手: 
\\	巨: 
\\	手	
\\	て・てへん	
\\	形声 
\\	刑	ケイ		おさか, ぎょう	
\\	刑事(けいじ): 
\\	死刑(しけい): 
\\	刑(けい): 
\\	刑罰(けいばつ): 
\\	幵: 
\\	刀: 
\\	刀	
\\	かたな・りっとう	
\\	形声 
\\	塚	チョウ	つか、-づか	ずか, つ	
\\	土: 
\\	冢: 
\\	土	
\\	つち・つちへん・どへん	塚	形声 
\\	致	チ	いた.す		
\\	一致(いっち): 
\\	致す(いたす): 
\\	合致(がっち): 
\\	どう致しまして(どういたしまして): 
\\	雅致(がち): 
\\	至: 
\\	攵: 
\\	至	
\\	いたる	
\\	会意 
\\	繰	ソウ	く.る	くり	
\\	繰り返す(くりかえす): 
\\	引っ繰り返る(ひっくりかえる): 
\\	糸: 
\\	品: 
\\	木: 
\\	糸	
\\	いと・いとへん	
\\	形声 
\\	尾	ビ	お		
\\	尾(お): 
\\	尻尾(しっぽ): 
\\	尸: 
\\	毛: 
\\	尸	
\\	しかばね・かばね・かばねだれ	
\\	象形 
\\	描	ビョウ	えが.く、か.く		
\\	描写(びょうしゃ): 
\\	描く(えがく): 
\\	手: 
\\	苗: 
\\	手	
\\	て・てへん	
\\	形声 
\\	鈴	レイ、リン	すず	ず	
\\	鈴(すず): 
\\	金: 
\\	令: 
\\	金	
\\	かね・かねへん	
\\	形声 
\\	盤	バン		ち, わ	
\\	基盤(きばん): 
\\	地盤(じばん): 
\\	算盤(そろばん): 
\\	碁盤(ごばん): 
\\	般: 
\\	皿: 
\\	皿	
\\	さら	
\\	形声 
\\	項	コウ	うなじ		
\\	項目(こうもく): 
\\	事項(じこう): 
\\	工: 
\\	頁: 
\\	頁	
\\	おおがい・いちのかい	
\\	形声 
\\	喪	ソウ	も		
\\	十: 
\\	口: 
\\	口: 
\\	衣: 
\\	口	
\\	くち・くちへん	
\\	会意 
\\	伴	ハン、バン	ともな.う	とも	
\\	伴(とも): 
\\	伴う(ともなう): 
\\	人: 
\\	半: 
\\	人	
\\	ひと・にんべん・ひとやね	
\\	形声 
\\	懸	ケン、ケ	か.ける、か.かる		
\\	一生懸命(いっしょうけんめい): 
\\	懸賞(けんしょう): 
\\	懸命(けんめい): 
\\	懸かる(かかる): 
\\	縣: 
\\	心: 
\\	心	
\\	こころ・したごころ・りっしんべん	
\\	形声 
\\	契	ケイ	ちぎ.る		
\\	契約(けいやく): 
\\	契機(けいき): 
\\	契る(ちぎる): 
\\	刀: 
\\	大: 
\\	大	
\\	だい・おおきい	
\\	会意 
\\	掲	ケイ	かか.げる		
\\	掲載(けいさい): 
\\	掲げる(かかげる): 
\\	掲示(けいじ): 
\\	手: 
\\	曷: 
\\	手	
\\	て・てへん	揭	形声 
\\	躍	ヤク	おど.る	おどり	
\\	活躍(かつやく): 
\\	足: 
\\	羽: 
\\	隹: 
\\	足	
\\	あし・あしへん	
\\	形声 
\\	棄	キ	す.てる		
\\	棄権(きけん): 
\\	破棄(はき): 
\\	廃棄(はいき): 
\\	㐬: 
\\	木: 
\\	木	
\\	き・きへん	
\\	会意 
\\	邸	テイ	やしき	むら	
\\	邸宅(ていたく): 
\\	氐: 
\\	邑: 
\\	邑	
\\	むら・おおざと・おおざとづくり・ゆう	
\\	形声 
\\	籍	セキ			
\\	国籍(こくせき): 
\\	書籍(しょせき): 
\\	戸籍(こせき): 
\\	竹: 
\\	耒: 
\\	昔: 
\\	竹	
\\	たけ・たけかんむり	
\\	形声 
\\	還	カン	かえ.る		
\\	返還(へんかん): 
\\	還元(かんげん): 
\\	還暦(かんれき): 
\\	睘: 
\\	辶: 
\\	辵	
\\	しんにょう・しんにゅう	
\\	形声 
\\	慮	リョ	おもんぱく.る、おもんぱか.る	ぜ	
\\	配慮(はいりょ): 
\\	考慮(こうりょ): 
\\	遠慮(えんりょ): 
\\	ご遠慮ください(ごえんりょください): 
\\	虍: 
\\	思: 
\\	心	
\\	こころ・したごころ・りっしんべん	
\\	形声 
\\	枠		わく		
\\	枠(わく): 
\\	木: 
\\	卆: 
\\	木	
\\	き・きへん	
\\	国字 
\\	露	ロ、ロウ	つゆ	や, ゆ	
\\	露骨(ろこつ): 
\\	暴露(ばくろ): 
\\	露(つゆ): 
\\	雨: 
\\	路: 
\\	雨	
\\	あめ・あめかんむり・あまかんむり	
\\	形声 
\\	沖	チュウ	おき、おきつ、ちゅう.する、わく		
\\	沖(おき): 
\\	水: 
\\	中: 
\\	水	
\\	みず・したみず・さんずい	
\\	形声 
\\	緩	カン	ゆる.い、ゆる.やか、ゆる.む、ゆる.める	ひろ	
\\	緩和(かんわ): 
\\	緩やか(ゆるやか): 
\\	緩む(ゆるむ): 
\\	緩い(ゆるい): 
\\	緩める(ゆるめる): 
\\	糸: 
\\	爰: 
\\	糸	
\\	いと・いとへん	
\\	形声 
\\	需	ジュ			
\\	需要(じゅよう): 
\\	必需品(ひつじゅひん): 
\\	雨: 
\\	而: 
\\	雨	
\\	あめ・あめかんむり・あまかんむり	
\\	会意 
\\	購	コウ			
\\	購入(こうにゅう): 
\\	購買(こうばい): 
\\	購読(こうどく): 
\\	貝: 
\\	冓: 
\\	貝	
\\	かい・かいへん・こがい	
\\	形声 
\\	充	ジュウ	あ.てる、み.たす	あつ, のぶ, まさ, みち, みつ, よし	
\\	充実(じゅうじつ): 
\\	補充(ほじゅう): 
\\	拡充(かくじゅう): 
\\	亠: 
\\	允: 
\\	儿	
\\	にんにょう・ひとあし	
\\	象形 
\\	跡	セキ	あと		
\\	跡(あと): 
\\	遺跡(いせき): 
\\	追跡(ついせき): 
\\	跡継ぎ(あとつぎ): 
\\	足: 
\\	亦: 
\\	足	
\\	あし・あしへん	
\\	形声 
\\	貢	コウ、ク	みつ.ぐ		
\\	貢献(こうけん): 
\\	工: 
\\	貝: 
\\	貝	
\\	かい・かいへん・こがい	
\\	形声 
\\	鹿	ロク	しか、か	しし	
\\	馬鹿(ばか): 
\\	馬鹿らしい(ばからしい): 
\\	馬鹿馬鹿しい(ばかばかしい): 
\\	广: 
\\	比: 
\\	鹿	
\\	しか・しかへん	
\\	象形 
\\	却	キャク	かえ.って、しりぞ.く、しりぞ.ける		
\\	却って(かえって): 
\\	去: 
\\	卩: 
\\	卩	
\\	ふし・ふしづくり・わりふ	
\\	会意 
\\	端	タン	はし、は、はた、-ばた、はな	ただし, みず	
\\	端(はし): 
\\	極端(きょくたん): 
\\	先端(せんたん): 
\\	半端(はんぱ): 
\\	途端(とたん): 
\\	出端(では): 
\\	立: 
\\	耑: 
\\	立	
\\	たつ・たつへん	
\\	形声 
\\	獲	カク	え.る		
\\	獲得(かくとく): 
\\	捕獲(ほかく): 
\\	獲物(えもの): 
\\	犬: 
\\	艸: 
\\	隻: 
\\	犬	
\\	いぬ・けものへん	
\\	形声 
\\	併	ヘイ	あわ.せる		
\\	合併(がっぺい): 
\\	併し乍ら(しかしながら): 
\\	人: 
\\	并: 
\\	人	
\\	ひと・にんべん・ひとやね	倂	形声 
\\	徹	テツ		あき, つ, てっ, とおる	
\\	徹底(てってい): 
\\	徹夜(てつや): 
\\	徹する(てっする): 
\\	彳: 
\\	育: 
\\	攵: 
\\	彳	
\\	ぎょうにんべん	
\\	形声 
\\	衝	ショウ	つ.く		
\\	衝突(しょうとつ): 
\\	衝撃(しょうげき): 
\\	行: 
\\	重: 
\\	行	
\\	ぎょう・ぎょうがまえ・いく・ゆきがまえ	
\\	形声 
\\	焦	ショウ	こ.げる、こ.がす、こ.がれる、あせ.る		
\\	焦点(しょうてん): 
\\	焦げる(こげる): 
\\	焦る(あせる): 
\\	焦がす(こがす): 
\\	焦げ茶(こげちゃ): 
\\	隹: 
\\	火: 
\\	火	
\\	ひ・ひへん・れっか・れんが	
\\	会意 
\\	奪	ダツ	うば.う		
\\	奪う(うばう): 
\\	略奪(りゃくだつ): 
\\	大: 
\\	隹: 
\\	寸: 
\\	大	
\\	だい・おおきい	
\\	会意 
\\	浦	ホ	うら		
\\	水: 
\\	甫: 
\\	水	
\\	みず・したみず・さんずい	
\\	形声 
\\	析	セキ			
\\	分析(ぶんせき): 
\\	木: 
\\	斤: 
\\	木	
\\	き・きへん	
\\	会意 
\\	譲	ジョウ	ゆず.る	ゆずり	
\\	譲歩(じょうほ): 
\\	譲る(ゆずる): 
\\	言: 
\\	襄: 
\\	言	
\\	こと・げん・ごんべん・ことば	讓	形声 
\\	称	ショウ	たた.える、とな.える、あ.げる、かな.う、はか.り、はか.る、ほめ.る	ね	
\\	名称(めいしょう): 
\\	称する(しょうする): 
\\	禾: 
\\	尓: 
\\	禾	
\\	のぎ・のぎへん	稱	形声 
\\	挑	チョウ	いど.む		
\\	挑戦(ちょうせん): 
\\	挑む(いどむ): 
\\	手: 
\\	兆: 
\\	手	
\\	て・てへん	
\\	形声 
\\	誘	ユウ、イウ	さそ.う、いざな.う		
\\	誘導(ゆうどう): 
\\	誘う(さそう): 
\\	誘惑(ゆうわく): 
\\	勧誘(かんゆう): 
\\	言: 
\\	秀: 
\\	言	
\\	こと・げん・ごんべん・ことば	
\\	形声 
\\	紛	フン	まぎ.れる、-まぎ.れ、まぎ.らす、まぎ.らわす、まぎ.らわしい		
\\	紛争(ふんそう): 
\\	紛失(ふんしつ): 
\\	紛れる(まぎれる): 
\\	紛らわしい(まぎらわしい): 
\\	糸: 
\\	分: 
\\	糸	
\\	いと・いとへん	
\\	形声 
\\	促	ソク	うなが.す		
\\	促進(そくしん): 
\\	促す(うながす): 
\\	催促(さいそく): 
\\	人: 
\\	足: 
\\	人	
\\	ひと・にんべん・ひとやね	
\\	形声 
\\	慎	シン	つつし.む、つつし、つつし.み	ちか, のり, まこと, みつ	
\\	慎重(しんちょう): 
\\	心: 
\\	真: 
\\	心	
\\	こころ・したごころ・りっしんべん	愼	形声 
\\	控	コウ	ひか.える、ひか.え		
\\	控える(ひかえる): 
\\	控え室(ひかえしつ): 
\\	控除(こうじょ): 
\\	手: 
\\	空: 
\\	手	
\\	て・てへん	
\\	形声 
\\	握	アク	にぎ.る		
\\	握手(あくしゅ): 
\\	握る(にぎる): 
\\	把握(はあく): 
\\	手: 
\\	屋: 
\\	手	
\\	て・てへん	
\\	形声 
\\	俊	シュン		すぐる, とし	
\\	人: 
\\	夋: 
\\	人	
\\	ひと・にんべん・ひとやね	
\\	形声 
\\	渋	ジュウ、シュウ	しぶ、しぶ.い、しぶ.る		
\\	渋滞(じゅうたい): 
\\	渋い(しぶい): 
\\	水: 
\\	止: 
\\	冫: 
\\	水	
\\	みず・したみず・さんずい	澁	会意 
\\	銃	ジュウ	つつ		
\\	銃(じゅう): 
\\	金: 
\\	充: 
\\	金	
\\	かね・かねへん	
\\	形声 
\\	携	ケイ	たずさ.える、たずさ.わる		
\\	携帯(けいたい): 
\\	提携(ていけい): 
\\	携わる(たずさわる): 
\\	手: 
\\	隽: 
\\	手	
\\	て・てへん	
\\	形声 
\\	診	シン	み.る		
\\	診断(しんだん): 
\\	診療(しんりょう): 
\\	診察(しんさつ): 
\\	診る(みる): 
\\	往診(おうしん): 
\\	聴診器(ちょうしんき): 
\\	言: 
\\	人: 
\\	彡: 
\\	言	
\\	こと・げん・ごんべん・ことば	
\\	形声 
\\	託	タク	かこつ.ける、かこ.つ、かこ.つける		
\\	委託(いたく): 
\\	言: 
\\	乇: 
\\	言	
\\	こと・げん・ごんべん・ことば	
\\	形声 
\\	撮	サツ	と.る、つま.む、-ど.り		
\\	撮影(さつえい): 
\\	撮る(とる): 
\\	手: 
\\	最: 
\\	手	
\\	て・てへん	
\\	形声 
\\	侵	シン	おか.す		
\\	侵略(しんりゃく): 
\\	侵入(しんにゅう): 
\\	侵す(おかす): 
\\	人: 
\\	彑: 
\\	冖: 
\\	又: 
\\	人	
\\	ひと・にんべん・ひとやね	
\\	形声 
\\	括	カツ	くく.る		
\\	一括(いっかつ): 
\\	括弧(かっこ): 
\\	手: 
\\	舌: 
\\	手	
\\	て・てへん	
\\	形声 
\\	駆	ク	か.ける、か.る		
\\	駆ける(かける): 
\\	駆け足(かけあし): 
\\	駆けっこ(かけっこ): 
\\	馬: 
\\	区: 
\\	馬	
\\	うま・うまへん	驅	形声 
\\	透	トウ	す.く、す.かす、す.ける、とう.る、とう.す	とおる	
\\	透明(とうめい): 
\\	透き通る(すきとおる): 
\\	秀: 
\\	辶: 
\\	辵	
\\	しんにょう・しんにゅう	
\\	形声 
\\	津	シン	つ	ず, ち, と	
\\	津波(つなみ): 
\\	天津(あまつ): 
\\	水: 
\\	聿: 
\\	水	
\\	みず・したみず・さんずい	
\\	形声 
\\	稲	トウ、テ	いね、いな-	いの, しね, せ, な	
\\	稲(いね): 
\\	稲光(いなびかり): 
\\	稲子(いなご): 
\\	禾: 
\\	爪: 
\\	旧: 
\\	禾	
\\	のぎ・のぎへん	稻	形声 
\\	裂	レツ	さ.く、さ.ける、-ぎ.れ		
\\	分裂(ぶんれつ): 
\\	破裂(はれつ): 
\\	裂く(さく): 
\\	裂ける(さける): 
\\	列: 
\\	衣: 
\\	衣	
\\	ころも・ころもへん	
\\	形声 
\\	敏	ビン	さとい	さとし, ちょう, とし, び	
\\	敏感(びんかん): 
\\	毎: 
\\	攵: 
\\	攴	
\\	ぼくにょう・ぼくづくり・とまた・のぶん・しぶん	敏	会意 
\\	是	ゼ、シ	これ、この、ここ	すなお, ただし, つな, ゆき, よし	
\\	"是正(ぜせい): 
\\	是非(ぜひ): 
\\	是非とも(ぜひとも): 
\\	日: 
\\	疋: 
\\	日	
\\	ひ・ひへん・にち・にちへん	
\\	象形 
\\	排	ハイ		おし	
\\	排除(はいじょ): 
\\	排水(はいすい): 
\\	手: 
\\	非: 
\\	手	
\\	て・てへん	
\\	形声 
\\	裕	ユウ		すけ, のり, ひろ, ひろし, やす, ゆ, ゆたか	
\\	余裕(よゆう): 
\\	衣: 
\\	谷: 
\\	衣	
\\	ころも・ころもへん	
\\	形声 
\\	堅	ケン	かた.い、-がた.い	きん	
\\	堅い(かたい): 
\\	臣: 
\\	又: 
\\	土: 
\\	土	
\\	つち・つちへん・どへん	
\\	形声 
\\	芝	シ	しば	こげ, しは	
\\	芝居(しばい): 
\\	芝生(しばふ): 
\\	芝(しば): 
\\	艸: 
\\	之: 
\\	艸	
\\	くさ・くさかんむり	
\\	形声 
\\	綱	コウ	つな		
\\	横綱(よこづな): 
\\	綱(つな): 
\\	糸: 
\\	岡: 
\\	糸	
\\	いと・いとへん	
\\	形声 
\\	扱	ソウ、キュウ	あつか.い、あつか.う、あつか.る、こ.く		
\\	取扱い(とりあつかい): 
\\	扱う(あつかう): 
\\	扱い(あつかい): 
\\	取り扱う(とりあつかう): 
\\	手: 
\\	及: 
\\	手	
\\	て・てへん	
\\	形声 
\\	顧	コ	かえり.みる	たか, み	
\\	顧みる(かえりみる): 
\\	雇: 
\\	頁: 
\\	頁	
\\	おおがい・いちのかい	
\\	会意 
\\	訟	ショウ			
\\	訴訟(そしょう): 
\\	言: 
\\	公: 
\\	言	
\\	こと・げん・ごんべん・ことば	
\\	形声 
\\	戒	カイ	いまし.める		
\\	警戒(けいかい): 
\\	戈: 
\\	廾: 
\\	戈	
\\	ほこ・ほこづくり・ほこがまえ・かのほこ	
\\	会意 
\\	祉	シ			
\\	福祉(ふくし): 
\\	示: 
\\	止: 
\\	示	
\\	しめす・しめすへん	祉	形声 
\\	誉	ヨ	ほま.れ、ほ.める	え, たか, たけ, ほまれ, ほめ, ほん	
\\	名誉(めいよ): 
\\	小: 
\\	八: 
\\	言: 
\\	言	
\\	こと・げん・ごんべん・ことば	譽	形声 
\\	歓	カン	よろこ.ぶ	ぶ	
\\	歓迎(かんげい): 
\\	歓声(かんせい): 
\\	隹: 
\\	欠: 
\\	欠	
\\	あくび・けんづくり・かける・けつ	歡	形声 
\\	勧	カン、ケン	すす.める		
\\	勧告(かんこく): 
\\	勧め(すすめ): 
\\	勧誘(かんゆう): 
\\	隹: 
\\	力: 
\\	力	
\\	ちから・りきづくり	勸	形声 
\\	騒	ソウ	さわ.ぐ、うれい、さわ.がしい		
\\	騒ぎ(さわぎ): 
\\	騒音(そうおん): 
\\	騒動(そうどう): 
\\	騒ぐ(さわぐ): 
\\	物騒(ぶっそう): 
\\	騒々しい(そうぞうしい): 
\\	騒がしい(さわがしい): 
\\	馬: 
\\	又: 
\\	虫: 
\\	馬	
\\	うま・うまへん	騷	形声 
\\	閥	バツ			
\\	門: 
\\	伐: 
\\	門	
\\	もん・もんがまえ・かどがまえ	
\\	形声 
\\	甲	コウ、カン	きのえ	かぶと, き, まさ, まさる	
\\	甲(こう): 
\\	生き甲斐(いきがい): 
\\	甲(きのえ): 
\\	田: 
\\	丨: 
\\	田	
\\	た・たへん	
\\	象形 
\\	縄	ジョウ	なわ、ただ.す		
\\	縄(なわ): 
\\	糸: 
\\	日: 
\\	日: 
\\	糸	
\\	いと・いとへん	繩	形声 
\\	揺	ヨウ	ゆ.れる、ゆ.る、ゆ.らぐ、ゆ.るぐ、ゆ.する、ゆ.さぶる、ゆ.すぶる、うご.く		
\\	動揺(どうよう): 
\\	揺らぐ(ゆらぐ): 
\\	揺さぶる(ゆさぶる): 
\\	揺れる(ゆれる): 
\\	手: 
\\	䍃: 
\\	手	
\\	て・てへん	搖	形声 
\\	免	メン	まぬか.れる、まぬが.れる	め	
\\	免許(めんきょ): 
\\	免税(めんぜい): 
\\	免除(めんじょ): 
\\	御免(ごめん): 
\\	ご免ください(ごめんください): 
\\	免れる(まぬかれる): 
\\	御免なさい(ごめんなさい): 
\\	勹: 
\\	八: 
\\	儿	
\\	にんにょう・ひとあし	免	象形 
\\	既	キ	すで.に		
\\	既に(すでに): 
\\	既婚(きこん): 
\\	艮: 
\\	无: 
\\	无	
\\	む・むにょう・すでのつくり	既	会意 
\\	薦	セン	すす.める	こも	
\\	推薦(すいせん): 
\\	薦める(すすめる): 
\\	艸: 
\\	广: 
\\	火: 
\\	艸	
\\	くさ・くさかんむり	
\\	会意 
\\	隣	リン	とな.る、となり	ちか	
\\	隣(となり): 
\\	阜: 
\\	米: 
\\	舛: 
\\	阜	
\\	こざと・こざとへん・おか・ぎふのふ	
\\	会意 
\\	華	カ、ケ	はな	わ	
\\	豪華(ごうか): 
\\	華やか(はなやか): 
\\	華々しい(はなばなしい): 
\\	華美(かび): 
\\	華奢(きゃしゃ): 
\\	艸: 
\\	丨: 
\\	艸	
\\	くさ・くさかんむり	
\\	象形 
\\	範	ハン		のり	
\\	軌範(きはん): 
\\	範囲(はんい): 
\\	模範(もはん): 
\\	竹: 
\\	車: 
\\	卩: 
\\	竹	
\\	たけ・たけかんむり	
\\	形声 
\\	隠	イン、オン	かく.す、かく.し、かく.れる、かか.す、よ.る	お, がくし	
\\	隠す(かくす): 
\\	隠れる(かくれる): 
\\	隠居(いんきょ): 
\\	阜: 
\\	爪: 
\\	彑: 
\\	心: 
\\	阜	
\\	こざと・こざとへん・おか・ぎふのふ	隱	形声 
\\	哲	テツ	さとい、あき.らか	あき, あきら, さと, さとし, さとる, てっ, てつん, のり, よし	
\\	哲学(てつがく): 
\\	折: 
\\	口: 
\\	口	
\\	くち・くちへん	
\\	形声 
\\	杉	サン	すぎ		
\\	杉(すぎ): 
\\	木: 
\\	彡: 
\\	木	
\\	き・きへん	
\\	形声 
\\	釈	シャク、セキ	とく、す.てる、ゆる.す	しゃ	
\\	解釈(かいしゃく): 
\\	釆: 
\\	尺: 
\\	釆	
\\	のごめ・のごめへん	釋	会意 
\\	妥	ダ		やす	
\\	妥協(だきょう): 
\\	妥結(だけつ): 
\\	妥当(だとう): 
\\	爪: 
\\	女: 
\\	女	
\\	おんな・おんなへん	
\\	会意 
\\	威	イ	おど.す、おど.し、おど.かす	いさ, たけ, たけし	
\\	権威(けんい): 
\\	威力(いりょく): 
\\	威張る(いばる): 
\\	戌: 
\\	女: 
\\	女	
\\	おんな・おんなへん	
\\	会意 
\\	豪	ゴウ	えら.い	こう, ご, すぐる, たけ, たけし, ひで, まさる	
\\	豪華(ごうか): 
\\	富豪(ふごう): 
\\	亠: 
\\	口: 
\\	冖: 
\\	豕: 
\\	豕	
\\	いのこ・いのこへん・ぶた	
\\	形声 
\\	熊	ユウ	くま		
\\	能: 
\\	火: 
\\	火	
\\	ひ・ひへん・れっか・れんが	
\\	会意 
\\	滞	タイ、テイ	とどこお.る		
\\	滞在(たいざい): 
\\	渋滞(じゅうたい): 
\\	停滞(ていたい): 
\\	滞納(たいのう): 
\\	滞る(とどこおる): 
\\	水: 
\\	帯: 
\\	水	
\\	みず・したみず・さんずい	滯	形声 
\\	微	ビ	かす.か	み	
\\	微妙(びみょう): 
\\	顕微鏡(けんびきょう): 
\\	微笑(びしょう): 
\\	微量(びりょう): 
\\	微笑む(ほほえむ): 
\\	微塵(みじん): 
\\	微か(かすか): 
\\	彳: 
\\	山: 
\\	兀: 
\\	攵: 
\\	彳	
\\	ぎょうにんべん	
\\	形声 
\\	隆	リュウ		お, たか, たかし	
\\	阜: 
\\	夂: 
\\	生: 
\\	阜	
\\	こざと・こざとへん・おか・ぎふのふ	隆	会意 
\\	症	ショウ			
\\	症状(しょうじょう): 
\\	症(しょう): 
\\	疒: 
\\	正: 
\\	疒	
\\	やまい・やまいだれ	
\\	形声 
\\	暫	ザン	しばら.く		
\\	暫く(しばらく): 
\\	斬: 
\\	日: 
\\	日	
\\	ひ・ひへん・にち・にちへん	
\\	形声 
\\	肝	カン	きも		
\\	肝心(かんじん): 
\\	肉: 
\\	干: 
\\	肉	
\\	にく・にくづき	
\\	形声 
\\	喚	カン	わめ.く		
\\	口: 
\\	奐: 
\\	口	
\\	くち・くちへん	
\\	形声 
\\	妙	ミョウ、ビョウ	たえ		
\\	微妙(びみょう): 
\\	妙(みょう): 
\\	奇妙(きみょう): 
\\	巧妙(こうみょう): 
\\	女: 
\\	少: 
\\	女	
\\	おんな・おんなへん	
\\	形声 
\\	索	サク			
\\	捜索(そうさく): 
\\	索引(さくいん): 
\\	模索(もさく): 
\\	十: 
\\	冖: 
\\	糸: 
\\	糸	
\\	いと・いとへん	
\\	象形 
\\	襲	シュウ	おそ.う、かさ.ね	そい	
\\	襲う(おそう): 
\\	襲撃(しゅうげき): 
\\	龍: 
\\	衣: 
\\	衣	
\\	ころも・ころもへん	
\\	会意 
\\	懇	コン	ねんご.ろ		
\\	豸: 
\\	艮: 
\\	心: 
\\	心	
\\	こころ・したごころ・りっしんべん	
\\	形声 
\\	柄	ヘイ	がら、え、つか	から, ら	
\\	柄(がら): 
\\	人柄(ひとがら): 
\\	事柄(ことがら): 
\\	小柄(こがら): 
\\	間柄(あいだがら): 
\\	柄(え): 
\\	大柄(おおがら): 
\\	木: 
\\	丙: 
\\	木	
\\	き・きへん	
\\	形声 
\\	驚	キョウ	おどろ.く、おどろ.かす		
\\	驚き(おどろき): 
\\	驚かす(おどろかす): 
\\	驚異(きょうい): 
\\	吃驚(びっくり): 
\\	驚く(おどろく): 
\\	敬: 
\\	馬: 
\\	馬	
\\	うま・うまへん	
\\	形声 
\\	麻	マ、マア	あさ	あ, あざ, お	
\\	麻酔(ますい): 
\\	麻(あさ): 
\\	麻痺(まひ): 
\\	麻姑(まこ): 
\\	广: 
\\	林: 
\\	麻	
\\	あさ・あさかんむり	
\\	会意 
\\	剤	ザイ、スイ、セイ	かる、けず.る		
\\	洗剤(せんざい): 
\\	齊: 
\\	刀: 
\\	刀	
\\	かたな・りっとう	劑	会意 
\\	瀬	ライ	せ	いわた, がせ, しげ, せい, せっ	
\\	瀬戸物(せともの): 
\\	水: 
\\	頼: 
\\	水	
\\	みず・したみず・さんずい	瀨	形声 
\\	趣	シュ	おもむき、おもむ.く		
\\	趣旨(しゅし): 
\\	趣味(しゅみ): 
\\	趣き(おもむき): 
\\	走: 
\\	取: 
\\	走	
\\	はしる・そうにょう	
\\	形声 
\\	陥	カン	おちい.る、おとしい.れる		
\\	欠陥(けっかん): 
\\	阜: 
\\	勹: 
\\	旧: 
\\	阜	
\\	こざと・こざとへん・おか・ぎふのふ	陷	形声 
\\	斎	サイ	とき、つつし.む、ものいみ、い.む、いわ.う、いつ.く	いつき, さえ, ひとし, いつ, きよ, ただ, とき, よし	
\\	書斎(しょさい): 
\\	齊: 
\\	齊	
\\	さい・せい・ととのえる	齋	会意 
\\	貫	カン	つらぬ.く、ぬ.く、ぬき	つら, ぬく	
\\	貫く(つらぬく): 
\\	貫禄(かんろく): 
\\	母: 
\\	貝: 
\\	貝	
\\	かい・かいへん・こがい	
\\	会意 
\\	仙	セン、セント		そま, のり	
\\	人: 
\\	山: 
\\	人	
\\	ひと・にんべん・ひとやね	
\\	形声 
\\	慰	イ	なぐさ.める、なぐさ.む		
\\	慰める(なぐさめる): 
\\	尉: 
\\	心: 
\\	心	
\\	こころ・したごころ・りっしんべん	
\\	形声 
\\	旬	ジュン、シュン			
\\	下旬(げじゅん): 
\\	中旬(ちゅうじゅん): 
\\	上旬(じょうじゅん): 
\\	初旬(しょじゅん): 
\\	勹: 
\\	日: 
\\	日	
\\	ひ・ひへん・にち・にちへん	
\\	象形 
\\	兼	ケン	か.ねる、-か.ねる	かね	
\\	兼ねる(かねる): 
\\	気兼ね(きがね): 
\\	兼業(けんぎょう): 
\\	兼用(けんよう): 
\\	八: 
\\	彑: 
\\	丨: 
\\	丨: 
\\	八	
\\	はち・は・はちがしら	
\\	会意 
\\	旨	シ	むね、うま.い		
\\	旨い(うまい): 
\\	趣旨(しゅし): 
\\	要旨(ようし): 
\\	匕: 
\\	日: 
\\	日	
\\	ひ・ひへん・にち・にちへん	
\\	会意 
\\	即	ソク	つ.く、つ.ける、すなわ.ち		
\\	即する(そくする): 
\\	即ち(すなわち): 
\\	即座に(そくざに): 
\\	即(そく): 
\\	艮: 
\\	卩: 
\\	卩	
\\	ふし・ふしづくり・わりふ	卽	会意 
\\	柳	リュウ	やなぎ	なぎ, や, やい, やぎ, やな, やない	
\\	木: 
\\	卯: 
\\	木	
\\	き・きへん	
\\	形声 
\\	偽	ギ、カ	いつわ.る、にせ、いつわ.り		
\\	偽造(ぎぞう): 
\\	偽物(にせもの): 
\\	人: 
\\	為: 
\\	人	
\\	ひと・にんべん・ひとやね	僞	形声 
\\	覇	ハ、ハク	はたがしら	はる	
\\	襾: 
\\	革: 
\\	月: 
\\	襾	
\\	にし・かなめのかしら・おおいかんむり	霸	会意 
\\	詳	ショウ	くわ.しい、つまび.らか	よし	
\\	詳しい(くわしい): 
\\	詳細(しょうさい): 
\\	言: 
\\	羊: 
\\	言	
\\	こと・げん・ごんべん・ことば	
\\	形声 
\\	抵	テイ			
\\	抵抗(ていこう): 
\\	大抵(たいてい): 
\\	手: 
\\	氐: 
\\	手	
\\	て・てへん	
\\	形声 
\\	脅	キョウ	おびや.かす、おど.す、おど.かす		
\\	脅迫(きょうはく): 
\\	脅す(おどす): 
\\	脅かす(おどかす): 
\\	力: 
\\	力: 
\\	力: 
\\	月: 
\\	肉	
\\	にく・にくづき	
\\	形声 
\\	茂	モ	しげ.る	うむさ, き, し, つとむ, む, もて	
\\	茂る(しげる): 
\\	艸: 
\\	戊: 
\\	艸	
\\	くさ・くさかんむり	
\\	形声 
\\	犠	ギ、キ	いけにえ		
\\	犠牲(ぎせい): 
\\	牛: 
\\	義: 
\\	牛	
\\	うし	犧	形声 
\\	距	キョ	へだ.たる、けづめ		
\\	距離(きょり): 
\\	足: 
\\	巨: 
\\	足	
\\	あし・あしへん	
\\	形声 
\\	雅	ガ	みや.び	う, お, か, ただし, のり, まさ, まさし, よし	
\\	雅致(がち): 
\\	牙: 
\\	隹: 
\\	隹	
\\	ふるとり	
\\	形声 
\\	飾	ショク	かざ.る、かざ.り	しか	
\\	飾り(かざり): 
\\	装飾(そうしょく): 
\\	飾る(かざる): 
\\	着飾る(きかざる): 
\\	首飾り(くびかざり): 
\\	修飾(しゅうしょく): 
\\	食: 
\\	巾: 
\\	食	
\\	しょく・しょくへん	
\\	会意 
\\	網	モウ	あみ	あ, ずな	
\\	網(あみ): 
\\	糸: 
\\	罔: 
\\	糸	
\\	いと・いとへん	
\\	形声 
\\	竜	リュウ、リョウ、ロウ	たつ、いせ	りう	
\\	龍: 
\\	龍	
\\	りゅう	龍	象形 
\\	繁	ハン	しげ.る、しげ.く		
\\	繁栄(はんえい): 
\\	頻繁(ひんぱん): 
\\	繁殖(はんしょく): 
\\	繁盛(はんじょう): 
\\	敏: 
\\	糸: 
\\	糸	
\\	いと・いとへん	繁	会意 
\\	翼	ヨク	つばさ		
\\	"翼(つばさ): 
\\	羽: 
\\	異: 
\\	羽	
\\	はね	
\\	形声 
\\	潟	セキ	かた、-がた	がら	
\\	水: 
\\	舄: 
\\	水	
\\	みず・したみず・さんずい	
\\	形声 
\\	魅	ミ			
\\	魅力(みりょく): 
\\	鬼: 
\\	未: 
\\	鬼	
\\	おに・きにょう	
\\	会意 
\\	嫌	ケン、ゲン	きら.う、きら.い、いや		
\\	嫌(いや): 
\\	嫌い(きらい): 
\\	機嫌(きげん): 
\\	嫌う(きらう): 
\\	嫌らしい(いやらしい): 
\\	好き嫌い(すききらい): 
\\	嫌がる(いやがる): 
\\	嫌に(いやに): 
\\	女: 
\\	兼: 
\\	女	
\\	おんな・おんなへん	
\\	形声 
\\	斉	セイ、サイ	そろ.う、ひと.しい、ひと.しく、あたる、はやい	ただ, なり, ひと, ひとし	
\\	一斉(いっせい): 
\\	文: 
\\	二: 
\\	齊	
\\	さい・せい・ととのえる	齊	象形 
\\	敷	フ	し.く、-し.き	しき, にゅう	
\\	敷地(しきち): 
\\	屋敷(やしき): 
\\	座敷(ざしき): 
\\	敷く(しく): 
\\	風呂敷(ふろしき): 
\\	旉: 
\\	攵: 
\\	攴	
\\	ぼくにょう・ぼくづくり・とまた・のぶん・しぶん	
\\	形声 
\\	擁	ヨウ			
\\	手: 
\\	雍: 
\\	手	
\\	て・てへん	
\\	形声 
\\	圏	ケン	かこ.い		
\\	"圏(けん): 
\\	囗: 
\\	巻: 
\\	囗	
\\	くに・くにがまえ	圈	形声 
\\	罰	バツ、バチ、ハツ	ばっ.する		
\\	罰(ばち): 
\\	処罰(しょばつ): 
\\	罰する(ばっする): 
\\	刑罰(けいばつ): 
\\	罰(ばつ): 
\\	网: 
\\	言: 
\\	刀: 
\\	网	
\\	あみ・あみがしら・あみめ・よこめ・よんかしら	
\\	会意 
\\	滅	メツ	ほろ.びる、ほろ.ぶ、ほろ.ぼす		
\\	絶滅(ぜつめつ): 
\\	滅亡(めつぼう): 
\\	滅びる(ほろびる): 
\\	全滅(ぜんめつ): 
\\	滅多に(めったに): 
\\	滅ぼす(ほろぼす): 
\\	水: 
\\	戍: 
\\	火: 
\\	水	
\\	みず・したみず・さんずい	
\\	形声 
\\	礎	ソ	いしずえ	もと	
\\	基礎(きそ): 
\\	石: 
\\	楚: 
\\	石	
\\	いし・いしへん	
\\	形声 
\\	腐	フ	くさ.る、-くさ.る、くさ.れる、くさ.れ、くさ.らす、くさ.す		
\\	腐敗(ふはい): 
\\	腐る(くさる): 
\\	府: 
\\	肉: 
\\	肉	
\\	にく・にくづき	
\\	形声 
\\	脚	キャク、キャ、カク	あし	し	
\\	脚本(きゃくほん): 
\\	脚色(きゃくしょく): 
\\	失脚(しっきゃく): 
\\	肉: 
\\	却: 
\\	肉	
\\	にく・にくづき	
\\	形声 
\\	尽	ジン、サン	つ.くす、-つ.くす、-づ.くし、-つ.く、-づ.く、-ず.く、つ.きる、つ.かす、さかづき、ことごと.く、つか、つき		
\\	尽くす(つくす): 
\\	尽きる(つきる): 
\\	尺: 
\\	冫: 
\\	皿	
\\	さら	盡	会意 
\\	僕	ボク	しもべ		
\\	僕(ぼく): 
\\	僕(しもべ): 
\\	人: 
\\	菐: 
\\	人	
\\	ひと・にんべん・ひとやね	
\\	形声 
\\	滑	カツ、コツ	すべ.る、なめ.らか	かり, なめり	
\\	円滑(えんかつ): 
\\	滑らか(なめらか): 
\\	滑る(すべる): 
\\	滑稽(こっけい): 
\\	滑らす(ずらす): 
\\	水: 
\\	骨: 
\\	水	
\\	みず・したみず・さんずい	
\\	形声 
\\	孤	コ			
\\	孤立(こりつ): 
\\	孤独(こどく): 
\\	孤児(こじ): 
\\	子: 
\\	瓜: 
\\	子	
\\	こ・こへん	
\\	形声 
\\	炎	エン	ほのお	ぬく	
\\	炎(ほのお): 
\\	火: 
\\	火: 
\\	火	
\\	ひ・ひへん・れっか・れんが	
\\	会意 
\\	賠	バイ			
\\	賠償(ばいしょう): 
\\	貝: 
\\	立: 
\\	口: 
\\	貝	
\\	かい・かいへん・こがい	
\\	形声 
\\	寿	ジュ、ス、シュウ	ことぶき、ことぶ.く、ことほ.ぐ	かず, じ, じゅう, すっ, すみ, とし, としかつ, なが, のぶ, のり, ひさ, ひさし, やす	
\\	寿命(じゅみょう): 
\\	三: 
\\	丿: 
\\	寸: 
\\	士	
\\	さむらい	壽	形声 
\\	頑	ガン	かたく		
\\	頑固(がんこ): 
\\	頑張る(がんばる): 
\\	頑丈(がんじょう): 
\\	元: 
\\	頁: 
\\	頁	
\\	おおがい・いちのかい	
\\	形声 
\\	鎖	サ	くさり、とざ.す		
\\	閉鎖(へいさ): 
\\	封鎖(ふうさ): 
\\	鎖(くさり): 
\\	金: 
\\	小: 
\\	貝: 
\\	金	
\\	かね・かねへん	
\\	形声 
\\	彩	サイ	いろど.る	あや, さ, さえ, つや	
\\	色彩(しきさい): 
\\	采: 
\\	彡: 
\\	彡	
\\	さん・さんづくり	
\\	形声 
\\	摩	マ	ま.する、さす.る、す.る		
\\	摩擦(まさつ): 
\\	摩る(さする): 
\\	麻: 
\\	手: 
\\	手	
\\	て・てへん	
\\	形声 
\\	励	レイ	はげ.む、はげ.ます		
\\	奨励(しょうれい): 
\\	激励(げきれい): 
\\	励ます(はげます): 
\\	励む(はげむ): 
\\	厂: 
\\	万: 
\\	力: 
\\	力	
\\	ちから・りきづくり	勵	形声 
\\	輝	キ	かがや.く	あき, あきら, さき, てる, ひ, ひかる	
\\	輝く(かがやく): 
\\	光: 
\\	軍: 
\\	車	
\\	くるま・くるまへん	
\\	形声 
\\	蓄	チク	たくわ.える		
\\	貯蓄(ちょちく): 
\\	蓄積(ちくせき): 
\\	蓄える(たくわえる): 
\\	艸: 
\\	畜: 
\\	艸	
\\	くさ・くさかんむり	
\\	形声 
\\	軸	ジク			
\\	軸(じく): 
\\	車: 
\\	由: 
\\	車	
\\	くるま・くるまへん	
\\	形声 
\\	巡	ジュン	めぐ.る、めぐ.り		
\\	巡査(じゅんさ): 
\\	巡る(めぐる): 
\\	お巡りさん(おまわりさん): 
\\	巛: 
\\	辶: 
\\	巛	
\\	かわ・まがりがわ・さんぼんがわ	
\\	形声 
\\	稼	カ	かせ.ぐ		
\\	共稼ぎ(ともかせぎ): 
\\	稼ぐ(かせぐ): 
\\	禾: 
\\	家: 
\\	禾	
\\	のぎ・のぎへん	
\\	形声 
\\	瞬	シュン	またた.く、まじろ.ぐ		
\\	瞬間(しゅんかん): 
\\	一瞬(いっしゅん): 
\\	瞬き(まばたき): 
\\	目: 
\\	舜: 
\\	目	
\\	め・めへん・よこめ	
\\	形声 
\\	砲	ホウ		づつ	
\\	鉄砲(てっぽう): 
\\	石: 
\\	包: 
\\	石	
\\	いし・いしへん	
\\	形声 
\\	噴	フン	ふ.く		
\\	噴火(ふんか): 
\\	噴出(ふんしゅつ): 
\\	噴水(ふんすい): 
\\	口: 
\\	賁: 
\\	口	
\\	くち・くちへん	
\\	形声 
\\	誇	コ	ほこ.る		
\\	誇り(ほこり): 
\\	誇る(ほこる): 
\\	誇張(こちょう): 
\\	言: 
\\	夸: 
\\	言	
\\	こと・げん・ごんべん・ことば	
\\	形声 
\\	祥	ショウ	さいわ.い、きざ.し、よ.い、つまび.らか	あき, さか, さち, ひろ, まさ, やす, ゆき, よし, あきら, さき, さむ, ただ, なか, なが	
\\	示: 
\\	羊: 
\\	示	
\\	しめす・しめすへん	祥	形声 
\\	牲	セイ			
\\	犠牲(ぎせい): 
\\	牛: 
\\	生: 
\\	牛	
\\	うし	
\\	形声 
\\	秩	チツ		ちち, ちっ	
\\	秩序(ちつじょ): 
\\	禾: 
\\	失: 
\\	禾	
\\	のぎ・のぎへん	
\\	形声 
\\	帝	テイ	みかど		
\\	亠: 
\\	冖: 
\\	巾: 
\\	巾	
\\	はば・はばへん・きんべん	
\\	象形 
\\	唆	サ	そそ.る、そそのか.す		
\\	口: 
\\	夋: 
\\	口	
\\	くち・くちへん	
\\	形声 
\\	阻	ソ	はば.む		
\\	阻む(はばむ): 
\\	阻止(そし): 
\\	阜: 
\\	且: 
\\	阜	
\\	こざと・こざとへん・おか・ぎふのふ	
\\	形声 
\\	泰	タイ		た, はす, ひろ, や, やす, やすし, ゆたか, よし	
\\	三: 
\\	大: 
\\	水: 
\\	水	
\\	みず・したみず・さんずい	
\\	会意 
\\	賄	ワイ	まかな.う		
\\	賄う(まかなう): 
\\	貝: 
\\	有: 
\\	貝	
\\	かい・かいへん・こがい	
\\	形声 
\\	撲	ボク			
\\	相撲(すもう): 
\\	手: 
\\	菐: 
\\	手	
\\	て・てへん	
\\	形声 
\\	堀	クツ	ほり	ほっ	
\\	堀(ほり): 
\\	土: 
\\	屈: 
\\	土	
\\	つち・つちへん・どへん	
\\	形声 
\\	菊	キク			
\\	艸: 
\\	勹: 
\\	米: 
\\	艸	
\\	くさ・くさかんむり	
\\	会意 
\\	絞	コウ	しぼ.る、し.める、し.まる		
\\	糸: 
\\	交: 
\\	糸	
\\	いと・いとへん	
\\	形声 
\\	縁	エン、-ネン	ふち、ふちど.る、ゆかり、よすが、へり、えにし		
\\	縁(ふち): 
\\	縁談(えんだん): 
\\	縁側(えんがわ): 
\\	縁(ゆかり): 
\\	縁(えん): 
\\	縁(へり): 
\\	糸: 
\\	彖: 
\\	糸	
\\	いと・いとへん	緣	形声 
\\	唯	ユイ、イ	ただ	ゆ	
\\	唯一(ゆいつ): 
\\	口: 
\\	隹: 
\\	口	
\\	くち・くちへん	
\\	会意 
\\	膨	ボウ	ふく.らむ、ふく.れる		
\\	膨脹(ぼうちょう): 
\\	膨大(ぼうだい): 
\\	膨れる(ふくれる): 
\\	膨らむ(ふくらむ): 
\\	膨らます(ふくらます): 
\\	肉: 
\\	彭: 
\\	肉	
\\	にく・にくづき	
\\	形声 
\\	耐	タイ	た.える	たえ	
\\	而: 
\\	寸: 
\\	而	
\\	しかし・しかして・しこうして	
\\	会意 
\\	塾	ジュク			
\\	塾(じゅく): 
\\	孰: 
\\	土: 
\\	土	
\\	つち・つちへん・どへん	
\\	形声 
\\	漏	ロウ	も.る、も.れる、も.らす		
\\	漏らす(もらす): 
\\	漏る(もる): 
\\	漏れる(もれる): 
\\	水: 
\\	尸: 
\\	雨: 
\\	水	
\\	みず・したみず・さんずい	
\\	形声 
\\	慶	ケイ	よろこ.び	き, きよん, け, みち, む, やす, よし	
\\	广: 
\\	心: 
\\	夂: 
\\	心	
\\	こころ・したごころ・りっしんべん	
\\	会意 
\\	猛	モウ		たけ, たけし, たける	
\\	猛烈(もうれつ): 
\\	犬: 
\\	孟: 
\\	犬	
\\	いぬ・けものへん	
\\	形声 
\\	芳	ホウ	かんば.しい	お, かおる, は, ほ, みち, やす, よし	
\\	艸: 
\\	方: 
\\	艸	
\\	くさ・くさかんむり	
\\	形声 
\\	懲	チョウ	こ.りる、こ.らす、こ.らしめる		
\\	懲りる(こりる): 
\\	徴: 
\\	心: 
\\	心	
\\	こころ・したごころ・りっしんべん	懲	形声 
\\	剣	ケン	つるぎ		
\\	真剣(しんけん): 
\\	僉: 
\\	刀: 
\\	刀	
\\	かたな・りっとう	劍	形声 
\\	彰	ショウ		あき, あきら	
\\	章: 
\\	彡: 
\\	彡	
\\	さん・さんづくり	
\\	会意 
\\	棋	キ	ご		
\\	将棋(しょうぎ): 
\\	木: 
\\	其: 
\\	木	
\\	き・きへん	
\\	形声 
\\	恒	コウ	つね、つねに	のぶ, ひさ, ひさし	
\\	心: 
\\	亘: 
\\	心	
\\	こころ・したごころ・りっしんべん	恆	形声 
\\	揚	ヨウ	あ.げる、-あ.げ、あ.がる	あがり	
\\	手: 
\\	昜: 
\\	手	
\\	て・てへん	
\\	形声 
\\	冒	ボウ	おか.す		
\\	冒頭(ぼうとう): 
\\	冒険(ぼうけん): 
\\	日: 
\\	目: 
\\	冂	
\\	けいがまえ・まきがまえ・どうがまえ・えんがまえ	
\\	会意 
\\	倫	リン		とも, のり, ひとし, ひろ, みち	
\\	人: 
\\	侖: 
\\	人	
\\	ひと・にんべん・ひとやね	
\\	形声 
\\	陳	チン	ひ.ねる	のぶ	
\\	陳列(ちんれつ): 
\\	阜: 
\\	東: 
\\	阜	
\\	こざと・こざとへん・おか・ぎふのふ	
\\	会意 
\\	憶	オク			
\\	記憶(きおく): 
\\	心: 
\\	意: 
\\	心	
\\	こころ・したごころ・りっしんべん	
\\	形声 
\\	潜	セン	ひそ.む、もぐ.る、かく.れる、くぐ.る、ひそ.める		
\\	潜水(せんすい): 
\\	潜る(くぐる): 
\\	潜入(せんにゅう): 
\\	潜る(もぐる): 
\\	水: 
\\	替: 
\\	水	
\\	みず・したみず・さんずい	潛	形声 
\\	梨	リ	なし	か	
\\	利: 
\\	木: 
\\	木	
\\	き・きへん	
\\	形声 
\\	克	コク	か.つ	かつ, まさる, よし	
\\	克服(こくふく): 
\\	古: 
\\	兄: 
\\	儿	
\\	にんにょう・ひとあし	
\\	象形 
\\	岳	ガク	たけ	おか, たか, たけん	
\\	山岳(さんがく): 
\\	丘: 
\\	山: 
\\	山	
\\	やま・やまへん・やまかんむり	嶽	象形 
\\	概	ガイ	おおむ.ね		
\\	概念(がいねん): 
\\	一概に(いちがいに): 
\\	概論(がいろん): 
\\	大概(たいがい): 
\\	概略(がいりゃく): 
\\	概説(がいせつ): 
\\	木: 
\\	既: 
\\	木	
\\	き・きへん	槪	形声 
\\	拘	コウ	かか.わる		
\\	拘束(こうそく): 
\\	拘わる(こだわる): 
\\	手: 
\\	句: 
\\	手	
\\	て・てへん	
\\	形声 
\\	黙	モク、ボク	だま.る、もだ.す		
\\	沈黙(ちんもく): 
\\	黙る(だまる): 
\\	黑: 
\\	犬: 
\\	黑	
\\	くろ・くろへん	默	形声 
\\	須	ス、シュ	すべから.く、すべし、ひげ、まつ、もち.いる、もと.める	もとむ, ぞ	
\\	彡: 
\\	頁: 
\\	頁	
\\	おおがい・いちのかい	
\\	会意 
\\	偏	ヘン	かたよ.る		
\\	偏見(へんけん): 
\\	偏(へん): 
\\	人: 
\\	扁: 
\\	人	
\\	ひと・にんべん・ひとやね	
\\	形声 
\\	雰	フン			
\\	雰囲気(ふんいき): 
\\	雨: 
\\	分: 
\\	雨	
\\	あめ・あめかんむり・あまかんむり	
\\	形声 
\\	遇	グウ	あ.う		
\\	待遇(たいぐう): 
\\	境遇(きょうぐう): 
\\	禺: 
\\	辶: 
\\	辵	
\\	しんにょう・しんにゅう	
\\	形声 
\\	諮	シ	はか.る		
\\	諮る(はかる): 
\\	言: 
\\	咨: 
\\	言	
\\	こと・げん・ごんべん・ことば	
\\	形声 
\\	卓	タク		すぐる, たか, たかし	
\\	食卓(しょくたく): 
\\	卜: 
\\	早: 
\\	十	
\\	じゅう・じゅうへん	
\\	象形 
\\	亀	キ、キュウ、キン	かめ	ひさ, ひさし	
\\	勹: 
\\	日: 
\\	日: 
\\	龜	
\\	かめ	龜	象形 
\\	糧	リョウ、ロウ	かて		
\\	食糧(しょくりょう): 
\\	米: 
\\	量: 
\\	米	
\\	こめ・こめへん	
\\	形声 
\\	簿	ボ			
\\	名簿(めいぼ): 
\\	竹: 
\\	溥: 
\\	竹	
\\	たけ・たけかんむり	
\\	形声 
\\	炉	ロ	いろり		
\\	火: 
\\	戸: 
\\	火	
\\	ひ・ひへん・れっか・れんが	爐	形声 
\\	殊	シュ	こと		
\\	特殊(とくしゅ): 
\\	殊に(ことに): 
\\	歹: 
\\	朱: 
\\	歹	
\\	がつ・がってん・がつへん・かばね・かばねへん・いちたへん	
\\	形声 
\\	殖	ショク	ふ.える、ふ.やす	うえ, え, げ	
\\	殖える(ふえる): 
\\	繁殖(はんしょく): 
\\	殖やす(ふやす): 
\\	歹: 
\\	直: 
\\	歹	
\\	がつ・がってん・がつへん・かばね・かばねへん・いちたへん	
\\	形声 
\\	艦	カン			
\\	軍艦(ぐんかん): 
\\	舟: 
\\	監: 
\\	舟	
\\	ふね・ふねへん	
\\	形声 
\\	輩	ハイ	やから、やかい、-ばら、ともがら		
\\	先輩(せんぱい): 
\\	後輩(こうはい): 
\\	非: 
\\	車: 
\\	車	
\\	くるま・くるまへん	
\\	形声 
\\	奇	キ	く.しき、あや.しい、くし、めずら.しい		
\\	奇妙(きみょう): 
\\	奇麗(きれい): 
\\	奇数(きすう): 
\\	大: 
\\	可: 
\\	大	
\\	だい・おおきい	
\\	会意 
\\	慢	マン			
\\	自慢(じまん): 
\\	我慢(がまん): 
\\	怠慢(たいまん): 
\\	心: 
\\	曼: 
\\	心	
\\	こころ・したごころ・りっしんべん	
\\	形声 
\\	鶴	カク	つる	たず, ず, か, つ, づ	
\\	寉: 
\\	鳥: 
\\	鳥	
\\	とり・とりへん	
\\	形声 
\\	謀	ボウ、ム	はか.る、たばか.る、はかりごと		
\\	謀る(はかる): 
\\	言: 
\\	某: 
\\	言	
\\	こと・げん・ごんべん・ことば	
\\	形声 
\\	拍	ハク、ヒョウ			
\\	拍手(はくしゅ): 
\\	手: 
\\	白: 
\\	手	
\\	て・てへん	
\\	形声 
\\	丈	ジョウ	たけ、だけ	とも, ますら	
\\	大丈夫(だいじょうぶ): 
\\	丈夫(じょうぶ): 
\\	丈(たけ): 
\\	頑丈(がんじょう): 
\\	丈(だけ): 
\\	成る丈(なるたけ): 
\\	丈夫(じょうふ): 
\\	丈(じょう): 
\\	一: 
\\	乂: 
\\	一	
\\	いち	
\\	会意 
\\	寛	カン	くつろ.ぐ、ひろ.い、ゆる.やか	とも, のぶ, のり, ひろし, ひろん, ゆた, ゆたか	
\\	寛容(かんよう): 
\\	宀: 
\\	莧: 
\\	宀	
\\	うかんむり	寬	会意 
\\	覆	フク	おお.う、くつがえ.す、くつがえ.る		
\\	覆う(おおう): 
\\	覆面(ふくめん): 
\\	覆す(くつがえす): 
\\	襾: 
\\	復: 
\\	襾	
\\	にし・かなめのかしら・おおいかんむり	
\\	形声 
\\	胞	ホウ			
\\	細胞(さいぼう): 
\\	肉: 
\\	包: 
\\	肉	
\\	にく・にくづき	
\\	形声 
\\	隔	カク	へだ.てる、へだ.たる		
\\	間隔(かんかく): 
\\	隔てる(へだてる): 
\\	隔週(かくしゅう): 
\\	隔たる(へだたる): 
\\	阜: 
\\	鬲: 
\\	阜	
\\	こざと・こざとへん・おか・ぎふのふ	
\\	形声 
\\	浄	ジョウ、セイ	きよ.める、きよ.い		
\\	水: 
\\	争: 
\\	水	
\\	みず・したみず・さんずい	淨	形声 
\\	没	ボツ、モツ	おぼ.れる、しず.む、ない		
\\	沈没(ちんぼつ): 
\\	没落(ぼつらく): 
\\	没収(ぼっしゅう): 
\\	水: 
\\	殳: 
\\	水	
\\	みず・したみず・さんずい	
\\	形声 
\\	暇	カ	ひま、いとま		
\\	休暇(きゅうか): 
\\	余暇(よか): 
\\	暇(ひま): 
\\	日: 
\\	又: 
\\	日	
\\	ひ・ひへん・にち・にちへん	
\\	形声 
\\	貞	テイ	さだ	さざ, ただし, りょう	
\\	卜: 
\\	貝: 
\\	貝	
\\	かい・かいへん・こがい	
\\	会意 
\\	鑑	カン	かんが.みる、かがみ	あき, あきら	
\\	鑑賞(かんしょう): 
\\	印鑑(いんかん): 
\\	図鑑(ずかん): 
\\	年鑑(ねんかん): 
\\	金: 
\\	監: 
\\	金	
\\	かね・かねへん	
\\	形声 
\\	陰	イン	かげ、かげ.る		
\\	日陰(ひかげ): 
\\	陰気(いんき): 
\\	陰(かげ): 
\\	阜: 
\\	今: 
\\	云: 
\\	阜	
\\	こざと・こざとへん・おか・ぎふのふ	
\\	形声 
\\	銘	メイ		たか, み, め	
\\	銘々(めいめい): 
\\	金: 
\\	名: 
\\	金	
\\	かね・かねへん	
\\	形声 
\\	随	ズイ	まにま.に、したが.う		
\\	随筆(ずいひつ): 
\\	随分(ずいぶん): 
\\	陏: 
\\	辶: 
\\	阜	
\\	こざと・こざとへん・おか・ぎふのふ	隨	形声 
\\	烈	レツ	はげ.しい	やす	
\\	強烈(きょうれつ): 
\\	猛烈(もうれつ): 
\\	列: 
\\	火: 
\\	火	
\\	ひ・ひへん・れっか・れんが	
\\	会意 
\\	尋	ジン	たず.ねる、ひろ	ず, つぐ	
\\	尋ねる(たずねる): 
\\	彑: 
\\	工: 
\\	口: 
\\	寸: 
\\	寸	
\\	すん・すんづくり	
\\	会意 
\\	稿	コウ	わら、したがき		
\\	原稿(げんこう): 
\\	禾: 
\\	高: 
\\	禾	
\\	のぎ・のぎへん	
\\	形声 
\\	丹	タン	に	た, たみ, まこと	
\\	冂: 
\\	丶: 
\\	丶	
\\	てん・ちょぼ・ちゅ・ちゅてん	
\\	象形 
\\	啓	ケイ	ひら.く、さと.す	あき, あきら, さとし, はじめ, ひろ, よし	
\\	拝啓(はいけい): 
\\	戸: 
\\	攵: 
\\	口: 
\\	口	
\\	くち・くちへん	
\\	会意 
\\	丘	キュウ	おか	たかし	
\\	丘(おか): 
\\	丘陵(きゅうりょう): 
\\	斤: 
\\	一: 
\\	一	
\\	いち	
\\	象形 
\\	棟	トウ	むね、むな-		
\\	棟(とう): 
\\	木: 
\\	東: 
\\	木	
\\	き・きへん	
\\	形声 
\\	壌	ジョウ	つち		
\\	土: 
\\	襄: 
\\	土	
\\	つち・つちへん・どへん	壤	形声 
\\	漫	マン	みだり.に、そぞ.ろ		
\\	漫画(まんが): 
\\	水: 
\\	曼: 
\\	水	
\\	みず・したみず・さんずい	
\\	形声 
\\	玄	ゲン		くろ, けん, はる, はるか	
\\	玄関(げんかん): 
\\	玄人(くろうと): 
\\	亠: 
\\	幺: 
\\	玄	
\\	げん	
\\	会意 
\\	粘	ネン	ねば.る		
\\	粘り(ねばり): 
\\	粘る(ねばる): 
\\	米: 
\\	占: 
\\	米	
\\	こめ・こめへん	
\\	形声 
\\	悟	ゴ	さと.る		
\\	覚悟(かくご): 
\\	悟る(さとる): 
\\	心: 
\\	吾: 
\\	心	
\\	こころ・したごころ・りっしんべん	
\\	形声 
\\	舗	ホ		き	
\\	舗装(ほそう): 
\\	人: 
\\	吉: 
\\	甫: 
\\	舌	
\\	した・したへん	
\\	形声 
\\	妊	ニン、ジン	はら.む、みごも.る		
\\	妊娠(にんしん): 
\\	女: 
\\	壬: 
\\	女	
\\	おんな・おんなへん	
\\	形声 
\\	騰	トウ	あが.る、のぼ.る		
\\	沸騰(ふっとう): 
\\	朕: 
\\	馬: 
\\	馬	
\\	うま・うまへん	
\\	形声 
\\	遂	スイ	と.げる、つい.に		
\\	遂げる(とげる): 
\\	遂に(ついに): 
\\	遣り遂げる(やりとげる): 
\\	豕: 
\\	辶: 
\\	辵	
\\	しんにょう・しんにゅう	
\\	形声 
\\	狂	キョウ	くる.う、くる.おしい、くるお.しい		
\\	狂う(くるう): 
\\	犬: 
\\	玉: 
\\	犬	
\\	いぬ・けものへん	
\\	形声 
\\	岐	キ、ギ		たかし, また	
\\	山: 
\\	支: 
\\	山	
\\	やま・やまへん・やまかんむり	
\\	形声 
\\	緯	イ	よこいと、ぬき		
\\	経緯(いきさつ): 
\\	緯度(いど): 
\\	糸: 
\\	韋: 
\\	糸	
\\	いと・いとへん	
\\	形声 
\\	培	バイ	つちか.う		
\\	栽培(さいばい): 
\\	土: 
\\	立: 
\\	口: 
\\	土	
\\	つち・つちへん・どへん	
\\	形声 
\\	衰	スイ	おとろ.える		
\\	老衰(ろうすい): 
\\	衰える(おとろえる): 
\\	衣: 
\\	口: 
\\	衣	
\\	ころも・ころもへん	
\\	会意 
\\	艇	テイ			
\\	舟: 
\\	廷: 
\\	舟	
\\	ふね・ふねへん	
\\	形声 
\\	屈	クツ	かが.む、かが.める	くっ	
\\	理屈(りくつ): 
\\	窮屈(きゅうくつ): 
\\	屈折(くっせつ): 
\\	退屈(たいくつ): 
\\	尸: 
\\	出: 
\\	尸	
\\	しかばね・かばね・かばねだれ	
\\	象形 
\\	淡	タン	あわ.い		
\\	冷淡(れいたん): 
\\	淡水(たんすい): 
\\	水: 
\\	炎: 
\\	水	
\\	みず・したみず・さんずい	
\\	形声 
\\	抽	チュウ	ひき-		
\\	抽選(ちゅうせん): 
\\	抽象(ちゅうしょう): 
\\	手: 
\\	由: 
\\	手	
\\	て・てへん	
\\	形声 
\\	披	ヒ			
\\	手: 
\\	皮: 
\\	手	
\\	て・てへん	
\\	形声 
\\	廷	テイ		たか	
\\	法廷(ほうてい): 
\\	壬: 
\\	廴: 
\\	廴	
\\	えんにょう・いんにょう	
\\	形声 
\\	錦	キン	にしき	かね, あや, にし	
\\	金: 
\\	帛: 
\\	金	
\\	かね・かねへん	
\\	形声 
\\	准	ジュン			
\\	冫: 
\\	隹: 
\\	冫	
\\	にすい	
\\	形声 
\\	奨	ショウ、ソウ	すす.める	まさし	
\\	奨励(しょうれい): 
\\	奨学金(しょうがくきん): 
\\	将: 
\\	大: 
\\	大	
\\	だい・おおきい	奬	形声 
\\	浸	シン	ひた.す、ひた.る		
\\	浸す(ひたす): 
\\	水: 
\\	彑: 
\\	冖: 
\\	又: 
\\	水	
\\	みず・したみず・さんずい	
\\	形声 
\\	剰	ジョウ	あまつさえ、あま.り、あま.る		
\\	過剰(かじょう): 
\\	乗: 
\\	刀: 
\\	刀	
\\	かたな・りっとう	剩	形声 
\\	胆	タン	きも	い, まこと	
\\	大胆(だいたん): 
\\	肉: 
\\	旦: 
\\	肉	
\\	にく・にくづき	膽	形声 
\\	繊	セン			
\\	繊維(せんい): 
\\	化繊(かせん): 
\\	糸: 
\\	土: 
\\	戈: 
\\	糸	
\\	いと・いとへん	纖	形声 
\\	駒	ク	こま		
\\	馬: 
\\	句: 
\\	馬	
\\	うま・うまへん	
\\	形声 
\\	虚	キョ、コ	むな.しい、うつ.ろ		
\\	謙虚(けんきょ): 
\\	虚ろ(うつろ): 
\\	虍: 
\\	一: 
\\	虍	
\\	とら・とらがしら・とらかんむり	虛	形声 
\\	霊	レイ、リョウ	たま		
\\	幽霊(ゆうれい): 
\\	雨: 
\\	二: 
\\	雨	
\\	あめ・あめかんむり・あまかんむり	靈	会意 
\\	悔	カイ	く.いる、く.やむ、くや.しい		
\\	後悔(こうかい): 
\\	悔しい(くやしい): 
\\	悔やむ(くやむ): 
\\	心: 
\\	毎: 
\\	心	
\\	こころ・したごころ・りっしんべん	悔	形声 
\\	諭	ユ	さと.す	ゆう	
\\	言: 
\\	兪: 
\\	言	
\\	こと・げん・ごんべん・ことば	
\\	形声 
\\	惨	サン、ザン	みじ.め、いた.む、むご.い		
\\	悲惨(ひさん): 
\\	惨め(みじめ): 
\\	心: 
\\	参: 
\\	心	
\\	こころ・したごころ・りっしんべん	慘	形声 
\\	虐	ギャク	しいた.げる		
\\	虍: 
\\	虍	
\\	とら・とらがしら・とらかんむり	
\\	象形 
\\	翻	ホン、ハン	ひるがえ.る、ひるがえ.す		
\\	翻訳(ほんやく): 
\\	番: 
\\	羽: 
\\	飛	
\\	とぶ	飜	形声 
\\	墜	ツイ	お.ちる、お.つ		
\\	墜落(ついらく): 
\\	隊: 
\\	土: 
\\	土	
\\	つち・つちへん・どへん	
\\	会意 
\\	沼	ショウ	ぬま		
\\	沼(ぬま): 
\\	水: 
\\	召: 
\\	水	
\\	みず・したみず・さんずい	
\\	形声 
\\	据	キョ	す.える、す.わる		
\\	据える(すえる): 
\\	据え付ける(すえつける): 
\\	手: 
\\	居: 
\\	手	
\\	て・てへん	
\\	形声 
\\	徐	ジョ	おもむ.ろに		
\\	徐行(じょこう): 
\\	徐々(じょじょ): 
\\	彳: 
\\	余: 
\\	彳	
\\	ぎょうにんべん	
\\	形声 
\\	搭	トウ			
\\	手: 
\\	荅: 
\\	手	
\\	て・てへん	
\\	形声 
\\	盾	ジュン	たて		
\\	矛盾(むじゅん): 
\\	盾(たて): 
\\	厂: 
\\	十: 
\\	目: 
\\	目	
\\	め・めへん・よこめ	
\\	象形 
\\	滝	ロウ、ソウ	たき	らき	
\\	滝(たき): 
\\	水: 
\\	竜: 
\\	水	
\\	みず・したみず・さんずい	瀧	形声 
\\	軌	キ			
\\	軌範(きはん): 
\\	軌道(きどう): 
\\	車: 
\\	九: 
\\	車	
\\	くるま・くるまへん	
\\	形声 
\\	妨	ボウ	さまた.げる		
\\	妨害(ぼうがい): 
\\	妨げる(さまたげる): 
\\	女: 
\\	方: 
\\	女	
\\	おんな・おんなへん	
\\	形声 
\\	擦	サツ	す.る、す.れる、-ず.れ、こす.る、こす.れる		
\\	摩擦(まさつ): 
\\	擦る(する): 
\\	擦る(こする): 
\\	擦れる(すれる): 
\\	手: 
\\	察: 
\\	手	
\\	て・てへん	
\\	形声 
\\	鯨	ゲイ	くじら		
\\	捕鯨(ほげい): 
\\	魚: 
\\	京: 
\\	魚	
\\	うお・うおへん・さかな	
\\	形声 
\\	荘	ソウ、ショウ、チャン	ほうき、おごそ.か	そ	
\\	別荘(べっそう): 
\\	艸: 
\\	壮: 
\\	艸	
\\	くさ・くさかんむり	莊	形声 
\\	諾	ダク			
\\	承諾(しょうだく): 
\\	言: 
\\	若: 
\\	言	
\\	こと・げん・ごんべん・ことば	
\\	形声 
\\	雷	ライ	かみなり、いかずち、いかづち		
\\	雷(かみなり): 
\\	雨: 
\\	田: 
\\	雨	
\\	あめ・あめかんむり・あまかんむり	
\\	形声 
\\	漂	ヒョウ	ただよ.う		
\\	漂う(ただよう): 
\\	水: 
\\	票: 
\\	水	
\\	みず・したみず・さんずい	
\\	形声 
\\	懐	カイ、エ	ふところ、なつ.かしい、なつ.かしむ、なつ.く、なつ.ける、なず.ける、いだ.く、おも.う	かね	
\\	懐かしい(なつかしい): 
\\	懐く(なつく): 
\\	心: 
\\	十: 
\\	网: 
\\	衣: 
\\	心	
\\	こころ・したごころ・りっしんべん	懷	形声 
\\	勘	カン		か, さとる	
\\	勘定(かんじょう): 
\\	勘(かん): 
\\	勘違い(かんちがい): 
\\	勘弁(かんべん): 
\\	甚: 
\\	力: 
\\	力	
\\	ちから・りきづくり	
\\	会意 
\\	栽	サイ			
\\	栽培(さいばい): 
\\	土: 
\\	戈: 
\\	木: 
\\	木	
\\	き・きへん	
\\	形声 
\\	拐	カイ			
\\	手: 
\\	口: 
\\	刀: 
\\	手	
\\	て・てへん	
\\	形声 
\\	駄	ダ、タ			
\\	無駄(むだ): 
\\	無駄遣い(むだづかい): 
\\	駄目(だめ): 
\\	下駄(げた): 
\\	駄作(ださく): 
\\	馬: 
\\	太: 
\\	馬	
\\	うま・うまへん	
\\	形声 
\\	添	テン	そ.える、そ.う、も.える、も.う	そえ, ぞい	
\\	添う(そう): 
\\	添える(そえる): 
\\	水: 
\\	忝: 
\\	心: 
\\	水	
\\	みず・したみず・さんずい	
\\	形声 
\\	冠	カン	かんむり	か, かっぷ, まさる	
\\	冠(かんむり): 
\\	冖: 
\\	元: 
\\	冖	
\\	わかんむり	
\\	会意 
\\	斜	シャ	なな.め、はす		
\\	斜面(しゃめん): 
\\	傾斜(けいしゃ): 
\\	斜め(ななめ): 
\\	斜(はす): 
\\	余: 
\\	斗: 
\\	斗	
\\	と・とます・ますづくり	
\\	形声 
\\	浪	ロウ		なに, なみ	
\\	浪費(ろうひ): 
\\	水: 
\\	良: 
\\	水	
\\	みず・したみず・さんずい	
\\	形声 
\\	亜	ア	つ.ぐ	や, つぎ, つぐ	
\\	"亜細亜(アジア): 
\\	亜米利加(アメリカ): 
\\	亜(あ): 
\\	二: 
\\	口: 
\\	二	
\\	に・ふた	亞	象形 
\\	詐	サ	いつわ.る		
\\	詐欺(さぎ): 
\\	言: 
\\	乍: 
\\	言	
\\	こと・げん・ごんべん・ことば	
\\	形声 
\\	壇	ダン、タン			
\\	壇(だん): 
\\	花壇(かだん): 
\\	土: 
\\	亶: 
\\	土	
\\	つち・つちへん・どへん	
\\	形声 
\\	勲	クン	いさお	いさむ	
\\	動: 
\\	火: 
\\	力	
\\	ちから・りきづくり	勳	形声 
\\	魔	マ			
\\	悪魔(あくま): 
\\	邪魔(じゃま): 
\\	誤魔化す(ごまかす): 
\\	お邪魔します(おじゃまします): 
\\	麻: 
\\	鬼: 
\\	鬼	
\\	おに・きにょう	
\\	形声 
\\	酬	シュウ、シュ、トウ	むく.いる		
\\	報酬(ほうしゅう): 
\\	酉: 
\\	州: 
\\	酉	
\\	とり・とりへん・ひよみのとり・さけのとり	
\\	形声 
\\	紫	シ	むらさき	さい, ゆかり	
\\	紫(むらさき): 
\\	此: 
\\	糸: 
\\	糸	
\\	いと・いとへん	
\\	形声 
\\	紋	モン		あや	
\\	糸: 
\\	文: 
\\	糸	
\\	いと・いとへん	
\\	形声 
\\	卸	シャ	おろ.す、おろし、おろ.し		
\\	卸す(おろす): 
\\	午: 
\\	正: 
\\	卩: 
\\	卩	
\\	ふし・ふしづくり・わりふ	
\\	会意 
\\	欄	ラン	てすり		
\\	欄(らん): 
\\	木: 
\\	闌: 
\\	木	
\\	き・きへん	欄	形声 
\\	逸	イツ	そ.れる、そ.らす、はぐ.れる	いっ, はや, へん	
\\	逸れる(それる): 
\\	逸らす(そらす): 
\\	免: 
\\	辶: 
\\	辵	
\\	しんにょう・しんにゅう	逸	会意 
\\	涯	ガイ	はて		
\\	生涯(しょうがい): 
\\	水: 
\\	厓: 
\\	水	
\\	みず・したみず・さんずい	
\\	形声 
\\	拓	タク	ひら.く	つ, ひろ	
\\	開拓(かいたく): 
\\	手: 
\\	石: 
\\	手	
\\	て・てへん	
\\	形声 
\\	瓶	ビン	かめ	へい, べ, ぺ	
\\	瓶(びん): 
\\	瓶詰め(びんづめ): 
\\	花瓶(かびん): 
\\	并: 
\\	瓦: 
\\	瓦	
\\	かわら	甁	形声 
\\	獄	ゴク			
\\	地獄(じごく): 
\\	犬: 
\\	言: 
\\	犬: 
\\	犬	
\\	いぬ・けものへん	
\\	会意 
\\	尚	ショウ	なお	たか, たかし, ないし, なり, ひさ, ひさし	
\\	尚(なお): 
\\	尚更(なおさら): 
\\	高尚(こうしょう): 
\\	小: 
\\	冂: 
\\	口: 
\\	小	
\\	しょう・ちいさい・なおがしら・つ	
\\	会意 
\\	彫	チョウ	ほ.る、-ぼ.り		
\\	彫刻(ちょうこく): 
\\	彫る(ほる): 
\\	周: 
\\	彡: 
\\	彡	
\\	さん・さんづくり	
\\	会意 
\\	穏	オン	おだ.やか		
\\	穏やか(おだやか): 
\\	穏和(おんわ): 
\\	禾: 
\\	爪: 
\\	彑: 
\\	心: 
\\	禾	
\\	のぎ・のぎへん	穩	形声 
\\	顕	ケン	あきらか、あらわ.れる	あき, あきら	
\\	顕微鏡(けんびきょう): 
\\	日: 
\\	頁: 
\\	頁	
\\	おおがい・いちのかい	顯	会意 
\\	巧	コウ	たく.み、たく.む、うま.い	かつ, たくみ, よし	
\\	巧み(たくみ): 
\\	巧妙(こうみょう): 
\\	精巧(せいこう): 
\\	工: 
\\	丂: 
\\	工	
\\	こう・たくみ・たくみへん・え	
\\	形声 
\\	矛	ム、ボウ	ほこ		
\\	矛盾(むじゅん): 
\\	予: 
\\	丿: 
\\	矛	
\\	ほこ・ほこへん・むのほこ	
\\	象形 
\\	垣	エン	かき	がい	
\\	垣根(かきね): 
\\	土: 
\\	亘: 
\\	土	
\\	つち・つちへん・どへん	
\\	形声 
\\	欺	ギ	あざむ.く		
\\	詐欺(さぎ): 
\\	欺く(あざむく): 
\\	其: 
\\	欠: 
\\	欠	
\\	あくび・けんづくり・かける・けつ	
\\	形声 
\\	釣	チョウ	つ.る、つ.り、つ.り-	つり	
\\	釣り(つり): 
\\	釣る(つる): 
\\	お釣り(おつり): 
\\	釣り合う(つりあう): 
\\	釣り鐘(つりがね): 
\\	金: 
\\	勺: 
\\	金	
\\	かね・かねへん	
\\	形声 
\\	缶	カン	かま	ふ, べ	
\\	缶詰め(かんづめ): 
\\	缶(かん): 
\\	薬缶(やかん): 
\\	缶: 
\\	缶	
\\	ほとぎ・ほとぎへん	罐	形声 
\\	粧	ショウ		さ	
\\	化粧(けしょう): 
\\	米: 
\\	庄: 
\\	米	
\\	こめ・こめへん	
\\	形声 
\\	隻	セキ			
\\	隻(せき): 
\\	隹: 
\\	又: 
\\	隹	
\\	ふるとり	
\\	会意 
\\	粛	シュク、スク	つつし.む		
\\	聿: 
\\	米: 
\\	聿	
\\	ふで・ふでづくり	肅	会意 
\\	愚	グ	おろ.か		
\\	愚痴(ぐち): 
\\	愚か(おろか): 
\\	禺: 
\\	心: 
\\	心	
\\	こころ・したごころ・りっしんべん	
\\	形声 
\\	遭	ソウ	あ.う、あ.わせる		
\\	遭難(そうなん): 
\\	曹: 
\\	辶: 
\\	辵	
\\	しんにょう・しんにゅう	
\\	形声 
\\	架	カ	か.ける、か.かる		
\\	架空(かくう): 
\\	加: 
\\	木: 
\\	木	
\\	き・きへん	
\\	形声 
\\	鬼	キ	おに、おに-		
\\	"鬼(おに): 
\\	丿: 
\\	田: 
\\	儿: 
\\	厶: 
\\	鬼	
\\	おに・きにょう	
\\	象形 
\\	庶	ショ			
\\	庶民(しょみん): 
\\	庶務(しょむ): 
\\	广: 
\\	廿: 
\\	火: 
\\	广	
\\	まだれ	
\\	会意 
\\	稚	チ、ジ	いとけない、おさない、おくて、おでる	まさ, わか, わく, わっか	
\\	幼稚園(ようちえん): 
\\	幼稚(ようち): 
\\	禾: 
\\	隹: 
\\	禾	
\\	のぎ・のぎへん	
\\	形声 
\\	滋	ジ		し, しげ, しげる	
\\	水: 
\\	茲: 
\\	水	
\\	みず・したみず・さんずい	
\\	形声 
\\	幻	ゲン	まぼろし		
\\	幺: 
\\	幺	
\\	いとがしら・よう	
\\	象形 
\\	煮	シャ	に.る、-に、に.える、に.やす		
\\	煮る(にる): 
\\	煮える(にえる): 
\\	者: 
\\	火: 
\\	火	
\\	ひ・ひへん・れっか・れんが	煮	形声 
\\	姫	キ	ひめ、ひめ-		
\\	女: 
\\	臣: 
\\	女	
\\	おんな・おんなへん	
\\	会意 
\\	誓	セイ	ちか.う		
\\	誓う(ちかう): 
\\	折: 
\\	言: 
\\	言	
\\	こと・げん・ごんべん・ことば	
\\	形声 
\\	把	ハ、ワ		たば	
\\	把握(はあく): 
\\	手: 
\\	巴: 
\\	手	
\\	て・てへん	
\\	形声 
\\	践	セン	ふ.む		
\\	実践(じっせん): 
\\	足: 
\\	戔: 
\\	足	
\\	あし・あしへん	踐	形声 
\\	呈	テイ			
\\	進呈(しんてい): 
\\	口: 
\\	玉: 
\\	口	
\\	くち・くちへん	
\\	会意 
\\	疎	ソ、ショ	うと.い、うと.む、まば.ら		
\\	過疎(かそ): 
\\	疎か(おろそか): 
\\	疋: 
\\	束: 
\\	疋	
\\	ひき・ひきへん	
\\	形声 
\\	仰	ギョウ、コウ	あお.ぐ、おお.せ、お.っしゃる、おっしゃ.る		
\\	信仰(しんこう): 
\\	仰ぐ(あおぐ): 
\\	仰っしゃる(おっしゃる): 
\\	人: 
\\	卬: 
\\	人	
\\	ひと・にんべん・ひとやね	
\\	形声 
\\	剛	ゴウ		かた, こう, ご, たか, たけ, たけし, つよ, つよし, ひさ, まさ, よし	
\\	岡: 
\\	刀: 
\\	刀	
\\	かたな・りっとう	
\\	会意 
\\	疾	シツ	はや.い		
\\	疾っくに(とっくに): 
\\	疒: 
\\	矢: 
\\	疒	
\\	やまい・やまいだれ	
\\	会意 
\\	征	セイ		いく, そ, ただ, まさ, ゆき	
\\	征服(せいふく): 
\\	彳: 
\\	正: 
\\	彳	
\\	ぎょうにんべん	
\\	形声 
\\	砕	サイ	くだ.く、くだ.ける		
\\	砕く(くだく): 
\\	砕ける(くだける): 
\\	石: 
\\	卆: 
\\	石	
\\	いし・いしへん	碎	形声 
\\	謡	ヨウ	うた.い、うた.う		
\\	民謡(みんよう): 
\\	童謡(どうよう): 
\\	歌謡(かよう): 
\\	言: 
\\	䍃: 
\\	言	
\\	こと・げん・ごんべん・ことば	謠	形声 
\\	嫁	カ	よめ、とつ.ぐ、い.く、ゆ.く		
\\	嫁(よめ): 
\\	花嫁(はなよめ): 
\\	女: 
\\	家: 
\\	女	
\\	おんな・おんなへん	
\\	形声 
\\	謙	ケン	へりくだ.る	かね	
\\	謙虚(けんきょ): 
\\	謙遜(けんそん): 
\\	謙る(へりくだる): 
\\	言: 
\\	兼: 
\\	言	
\\	こと・げん・ごんべん・ことば	
\\	形声 
\\	嘆	タン	なげ.く、なげ.かわしい		
\\	嘆く(なげく): 
\\	口: 
\\	艸: 
\\	口: 
\\	夫: 
\\	口	
\\	くち・くちへん	嘆	形声 
\\	菌	キン			
\\	細菌(さいきん): 
\\	菌(きん): 
\\	ばい菌(ばいきん): 
\\	艸: 
\\	囗: 
\\	禾: 
\\	艸	
\\	くさ・くさかんむり	
\\	形声 
\\	鎌	レン、ケン	かま	かた, かね	
\\	金: 
\\	兼: 
\\	金	
\\	かね・かねへん	
\\	形声 
\\	頻	ヒン	しき.りに		
\\	頻繁(ひんぱん): 
\\	頻りに(しきりに): 
\\	歩: 
\\	頁: 
\\	頁	
\\	おおがい・いちのかい	頻	会意 
\\	琴	キン	こと		
\\	琴(こと): 
\\	玉: 
\\	玉: 
\\	今: 
\\	玉	
\\	たま・たまへん ・ おう・おうへん・あうさま	
\\	形声 
\\	棚	ホウ	たな、-だな		
\\	棚(たな): 
\\	本棚(ほんだな): 
\\	戸棚(とだな): 
\\	木: 
\\	朋: 
\\	木	
\\	き・きへん	
\\	形声 
\\	酷	コク	ひど.い		
\\	残酷(ざんこく): 
\\	冷酷(れいこく): 
\\	酷い(ひどい): 
\\	酉: 
\\	告: 
\\	酉	
\\	とり・とりへん・ひよみのとり・さけのとり	
\\	形声 
\\	宰	サイ		ただ	
\\	宀: 
\\	辛: 
\\	宀	
\\	うかんむり	
\\	会意 
\\	廊	ロウ			
\\	廊下(ろうか): 
\\	广: 
\\	郎: 
\\	广	
\\	まだれ	廊	形声 
\\	寂	ジャク、セキ	さび、さび.しい、さび.れる、さみ.しい		
\\	寂しい(さびしい): 
\\	宀: 
\\	叔: 
\\	宀	
\\	うかんむり	
\\	形声 
\\	伏	フク	ふ.せる、ふ.す	ふし, ふせ	
\\	降伏(こうふく): 
\\	起伏(きふく): 
\\	人: 
\\	犬: 
\\	人	
\\	ひと・にんべん・ひとやね	
\\	会意 
\\	碁	ゴ			
\\	碁(ご): 
\\	碁盤(ごばん): 
\\	其: 
\\	石: 
\\	石	
\\	いし・いしへん	
\\	形声 
\\	俗	ゾク			
\\	風俗(ふうぞく): 
\\	民俗(みんぞく): 
\\	人: 
\\	谷: 
\\	人	
\\	ひと・にんべん・ひとやね	
\\	形声 
\\	漠	バク			
\\	砂漠(さばく): 
\\	漠然(ばくぜん): 
\\	水: 
\\	莫: 
\\	水	
\\	みず・したみず・さんずい	
\\	形声 
\\	邪	ジャ	よこし.ま		
\\	風邪(かぜ): 
\\	邪魔(じゃま): 
\\	無邪気(むじゃき): 
\\	お邪魔します(おじゃまします): 
\\	牙: 
\\	邑: 
\\	邑	
\\	むら・おおざと・おおざとづくり・ゆう	
\\	形声 
\\	晶	ショウ		あ, あき, あきら, ひかり, まさ	
\\	結晶(けっしょう): 
\\	日: 
\\	日: 
\\	日: 
\\	日	
\\	ひ・ひへん・にち・にちへん	
\\	象形 
\\	墨	ボク	すみ	すの	
\\	墨(すみ): 
\\	黑: 
\\	土: 
\\	土	
\\	つち・つちへん・どへん	墨	会意 
\\	鎮	チン	しず.める、しず.まる、おさえ	しげ, じん, ちか	
\\	鎮まる(しずまる): 
\\	金: 
\\	真: 
\\	金	
\\	かね・かねへん	鎭	形声 
\\	洞	ドウ	ほら	とう	
\\	水: 
\\	同: 
\\	水	
\\	みず・したみず・さんずい	
\\	形声 
\\	履	リ	は.く		
\\	履歴(りれき): 
\\	草履(ぞうり): 
\\	履く(はく): 
\\	尸: 
\\	復: 
\\	尸	
\\	しかばね・かばね・かばねだれ	
\\	会意 
\\	劣	レツ	おと.る		
\\	劣る(おとる): 
\\	少: 
\\	力: 
\\	力	
\\	ちから・りきづくり	
\\	会意 
\\	那	ナ、ダ	なに、なんぞ、いかん	とも, やす	
\\	旦那(だんな): 
\\	二: 
\\	邑: 
\\	邑	
\\	むら・おおざと・おおざとづくり・ゆう	
\\	形声 
\\	殴	オウ	なぐ.る		
\\	殴る(なぐる): 
\\	区: 
\\	殳: 
\\	殳	
\\	るまた・ほこづくり	毆	会意 
\\	娠	シン			
\\	妊娠(にんしん): 
\\	女: 
\\	辰: 
\\	女	
\\	おんな・おんなへん	
\\	形声 
\\	奉	ホウ、ブ	たてまつ.る、まつ.る、ほう.ずる	とも, やす	
\\	奉仕(ほうし): 
\\	奉る(たてまつる): 
\\	三: 
\\	大: 
\\	丨: 
\\	大	
\\	だい・おおきい	
\\	会意 
\\	憂	ユウ	うれ.える、うれ.い、う.い、う.き	ゆ	
\\	憂鬱(ゆううつ): 
\\	百: 
\\	冖: 
\\	心: 
\\	夂: 
\\	心	
\\	こころ・したごころ・りっしんべん	
\\	会意 
\\	朴	ボク	ほう、ほお、えのき		
\\	素朴(そぼく): 
\\	木: 
\\	卜: 
\\	木	
\\	き・きへん	
\\	形声 
\\	亭	テイ、チン			
\\	高: 
\\	高: 
\\	冖: 
\\	丁: 
\\	亠	
\\	けいさん・けいさんかんむり・なべぶた	
\\	象形 
\\	怪	カイ、ケ	あや.しい、あや.しむ		
\\	怪獣(かいじゅう): 
\\	怪しい(あやしい): 
\\	怪我(けが): 
\\	心: 
\\	圣: 
\\	心	
\\	こころ・したごころ・りっしんべん	
\\	形声 
\\	酔	スイ	よ.う、よ.い、よ	よい	
\\	麻酔(ますい): 
\\	酔っ払い(よっぱらい): 
\\	酔う(よう): 
\\	酉: 
\\	卆: 
\\	酉	
\\	とり・とりへん・ひよみのとり・さけのとり	醉	形声 
\\	惜	セキ	お.しい、お.しむ		
\\	惜しむ(おしむ): 
\\	惜しい(おしい): 
\\	心: 
\\	昔: 
\\	心	
\\	こころ・したごころ・りっしんべん	
\\	形声 
\\	穫	カク			
\\	収穫(しゅうかく): 
\\	禾: 
\\	艸: 
\\	隻: 
\\	禾	
\\	のぎ・のぎへん	
\\	形声 
\\	佳	カ		けい, よし	
\\	佳句(かく): 
\\	人: 
\\	圭: 
\\	人	
\\	ひと・にんべん・ひとやね	
\\	形声 
\\	潤	ジュン	うるお.う、うるお.す、うる.む	めぐみ	
\\	利潤(りじゅん): 
\\	潤う(うるおう): 
\\	水: 
\\	閏: 
\\	水	
\\	みず・したみず・さんずい	
\\	形声 
\\	悼	トウ	いた.む		
\\	心: 
\\	卓: 
\\	心	
\\	こころ・したごころ・りっしんべん	
\\	形声 
\\	乏	ボウ	とぼ.しい、とも.しい		
\\	貧乏(びんぼう): 
\\	欠乏(けつぼう): 
\\	乏しい(とぼしい): 
\\	窮乏(きゅうぼう): 
\\	丿: 
\\	之: 
\\	丿	
\\	の・のかんむり・はらいぼう	
\\	象形 
\\	該	ガイ			
\\	該当(がいとう): 
\\	言: 
\\	亥: 
\\	言	
\\	こと・げん・ごんべん・ことば	
\\	形声 
\\	赴	フ	おもむ.く		
\\	赴任(ふにん): 
\\	赴く(おもむく): 
\\	走: 
\\	卜: 
\\	走	
\\	はしる・そうにょう	
\\	形声 
\\	桑	ソウ	くわ	こ	
\\	又: 
\\	又: 
\\	又: 
\\	木: 
\\	木	
\\	き・きへん	
\\	象形 
\\	髄	ズイ			
\\	骨: 
\\	有: 
\\	辶: 
\\	骨	
\\	ほね・ほねへん	髓	形声 
\\	虎	コ	とら	たけ	
\\	虎(トラ): 
\\	虍: 
\\	八: 
\\	虍	
\\	とら・とらがしら・とらかんむり	
\\	象形 
\\	盆	ボン			
\\	盆(ぼん): 
\\	盆地(ぼんち): 
\\	分: 
\\	皿: 
\\	皿	
\\	さら	
\\	形声 
\\	穂	スイ	ほ	お, こう, のり, ほい	
\\	穂(ほ): 
\\	禾: 
\\	恵: 
\\	禾	
\\	のぎ・のぎへん	穗	会意 
\\	壮	ソウ	さかん	つよし	
\\	壮大(そうだい): 
\\	爿: 
\\	士: 
\\	士	
\\	さむらい	壯	形声 
\\	堤	テイ	つつみ		
\\	堤防(ていぼう): 
\\	土: 
\\	是: 
\\	土	
\\	つち・つちへん・どへん	
\\	形声 
\\	飢	キ	う.える		
\\	飢える(うえる): 
\\	飢饉(ききん): 
\\	食: 
\\	几: 
\\	食	
\\	しょく・しょくへん	
\\	形声 
\\	傍	ボウ	かたわ.ら、わき、おか-、はた、そば	び	
\\	傍ら(かたわら): 
\\	人: 
\\	旁: 
\\	人	
\\	ひと・にんべん・ひとやね	
\\	形声 
\\	疫	エキ、ヤク			
\\	疒: 
\\	殳: 
\\	疒	
\\	やまい・やまいだれ	
\\	形声 
\\	累	ルイ			
\\	田: 
\\	糸: 
\\	糸	
\\	いと・いとへん	
\\	形声 
\\	痴	チ	し.れる、おろか		
\\	愚痴(ぐち): 
\\	疒: 
\\	知: 
\\	疒	
\\	やまい・やまいだれ	癡	会意 
\\	搬	ハン			
\\	運搬(うんぱん): 
\\	手: 
\\	般: 
\\	手	
\\	て・てへん	
\\	形声 
\\	癒	ユ	い.える、いや.す、い.やす		
\\	疒: 
\\	愈: 
\\	疒	
\\	やまい・やまいだれ	
\\	形声 
\\	郭	カク	くるわ	ひろ	
\\	享: 
\\	邑: 
\\	邑	
\\	むら・おおざと・おおざとづくり・ゆう	
\\	形声 
\\	尿	ニョウ			
\\	尿(にょう): 
\\	屎尿(しにょう): 
\\	尸: 
\\	水: 
\\	尸	
\\	しかばね・かばね・かばねだれ	
\\	象形 
\\	凶	キョウ			
\\	凶作(きょうさく): 
\\	乂: 
\\	凵: 
\\	凵	
\\	かんにょう・うけばこ・かんがまえ	
\\	象形 
\\	吐	ト	は.く、つ.く		
\\	吐き気(はきけ): 
\\	吐く(つく): 
\\	吐く(はく): 
\\	口: 
\\	土: 
\\	口	
\\	くち・くちへん	
\\	形声 
\\	宴	エン	うたげ	うた, やす	
\\	宴会(えんかい): 
\\	宀: 
\\	日: 
\\	女: 
\\	宀	
\\	うかんむり	
\\	会意 
\\	賓	ヒン			
\\	宀: 
\\	少: 
\\	貝: 
\\	貝	
\\	かい・かいへん・こがい	賓	会意 
\\	虜	リョ、ロ	とりこ、とりく		
\\	捕虜(ほりょ): 
\\	虍: 
\\	男: 
\\	虍	
\\	とら・とらがしら・とらかんむり	虜	形声 
\\	陶	トウ		す, すえ	
\\	陶器(とうき): 
\\	鬱陶しい(うっとうしい): 
\\	阜: 
\\	勹: 
\\	缶: 
\\	阜	
\\	こざと・こざとへん・おか・ぎふのふ	
\\	形声 
\\	鐘	ショウ	かね		
\\	鐘(かね): 
\\	釣り鐘(つりがね): 
\\	金: 
\\	童: 
\\	金	
\\	かね・かねへん	
\\	形声 
\\	憾	カン	うら.む		
\\	心: 
\\	感: 
\\	心	
\\	こころ・したごころ・りっしんべん	
\\	形声 
\\	弥	ミ、ビ	いや、や、あまねし、いよいよ、とおい、ひさし、ひさ.しい、わた.る	わたる, みつ, ひろ, よ	
\\	弓: 
\\	尓: 
\\	弓	
\\	ゆみ・ゆみへん	彌	会意 
\\	昆	コン			
\\	昆虫(こんちゅう): 
\\	日: 
\\	比: 
\\	日	
\\	ひ・ひへん・にち・にちへん	
\\	象形 
\\	粗	ソ	あら.い、あら-		
\\	粗末(そまつ): 
\\	粗い(あらい): 
\\	粗筋(あらすじ): 
\\	粗(あら): 
\\	粗衣(そい): 
\\	米: 
\\	且: 
\\	米	
\\	こめ・こめへん	
\\	形声 
\\	訂	テイ			
\\	訂正(ていせい): 
\\	改訂(かいてい): 
\\	言: 
\\	丁: 
\\	言	
\\	こと・げん・ごんべん・ことば	
\\	形声 
\\	傘	サン	かさ		
\\	傘(かさ): 
\\	人: 
\\	从: 
\\	从: 
\\	十: 
\\	人	
\\	ひと・にんべん・ひとやね	
\\	象形 
\\	騎	キ			
\\	馬: 
\\	奇: 
\\	馬	
\\	うま・うまへん	
\\	形声 
\\	寧	ネイ	むし.ろ	あき, やす, やすし, よし	
\\	丁寧(ていねい): 
\\	寧ろ(むしろ): 
\\	寍: 
\\	丁: 
\\	宀	
\\	うかんむり	
\\	会意 
\\	循	ジュン			
\\	循環(じゅんかん): 
\\	彳: 
\\	盾: 
\\	彳	
\\	ぎょうにんべん	
\\	形声 
\\	忍	ニン	しの.ぶ、しの.ばせる	おし	
\\	刃: 
\\	心: 
\\	心	
\\	こころ・したごころ・りっしんべん	
\\	形声 
\\	怠	タイ	おこた.る、なま.ける		
\\	怠る(おこたる): 
\\	怠慢(たいまん): 
\\	怠い(だるい): 
\\	怠ける(なまける): 
\\	台: 
\\	心: 
\\	心	
\\	こころ・したごころ・りっしんべん	
\\	形声 
\\	如	ジョ、ニョ	ごと.し	き, ね, ゆき, よし	
\\	突如(とつじょ): 
\\	如何(どう): 
\\	如何にも(いかにも): 
\\	如何して(どうして): 
\\	如何しても(どうしても): 
\\	如何に(いかに): 
\\	女: 
\\	口: 
\\	女	
\\	おんな・おんなへん	
\\	会意 
\\	寮	リョウ			
\\	寮(りょう): 
\\	宀: 
\\	尞: 
\\	宀	
\\	うかんむり	
\\	形声 
\\	鉛	エン	なまり		
\\	鉛筆(えんぴつ): 
\\	鉛(なまり): 
\\	金: 
\\	㕣: 
\\	金	
\\	かね・かねへん	
\\	形声 
\\	珠	シュ	たま	す, とも, ま, み	
\\	真珠(しんじゅ): 
\\	玉: 
\\	朱: 
\\	玉	
\\	たま・たまへん ・ おう・おうへん・あうさま	
\\	形声 
\\	凝	ギョウ	こ.る、こ.らす、こご.らす、こご.らせる、こご.る		
\\	凝る(こる): 
\\	混凝土(コンクリート): 
\\	凝乎と(じっと): 
\\	凝らす(こらす): 
\\	凝らす(こごらす): 
\\	凝る(こごる): 
\\	冫: 
\\	疑: 
\\	冫	
\\	にすい	
\\	形声 
\\	苗	ビョウ、ミョウ	なえ、なわ-	ねい, のら, みう, みつ	
\\	苗(なえ): 
\\	苗字(みょうじ): 
\\	艸: 
\\	田: 
\\	艸	
\\	くさ・くさかんむり	
\\	会意 
\\	獣	ジュウ	けもの、けだもの		
\\	獣(けだもの): 
\\	怪獣(かいじゅう): 
\\	小: 
\\	田: 
\\	口: 
\\	犬: 
\\	犬	
\\	いぬ・けものへん	獸	会意 
\\	哀	アイ	あわ.れ、あわ.れむ、かな.しい		
\\	可哀相(かわいそう): 
\\	哀れ(あわれ): 
\\	衣: 
\\	口: 
\\	口	
\\	くち・くちへん	
\\	会意 
\\	跳	チョウ	は.ねる、と.ぶ、-と.び		
\\	跳ねる(はねる): 
\\	足: 
\\	兆: 
\\	足	
\\	あし・あしへん	
\\	形声 
\\	匠	ショウ	たくみ		
\\	匚: 
\\	斤: 
\\	匚	
\\	はこ・はこがまえ	
\\	会意 
\\	蛇	ジャ、ダ、イ、ヤ	へび	あぶ, み	
\\	蛇口(じゃぐち): 
\\	虫: 
\\	它: 
\\	虫	
\\	むし・むしへん	
\\	形声 
\\	澄	チョウ	す.む、す.ます、-す.ます	きよ, すす, すみ, み	
\\	澄む(すむ): 
\\	澄ます(すます): 
\\	水: 
\\	登: 
\\	水	
\\	みず・したみず・さんずい	
\\	形声 
\\	縫	ホウ	ぬ.う	ぬい	
\\	縫う(ぬう): 
\\	裁縫(さいほう): 
\\	糸: 
\\	逢: 
\\	糸	
\\	いと・いとへん	
\\	形声 
\\	僧	ソウ			
\\	僧(そう): 
\\	人: 
\\	曽: 
\\	人	
\\	ひと・にんべん・ひとやね	僧	形声 
\\	眺	チョウ	なが.める		
\\	眺める(ながめる): 
\\	眺め(ながめ): 
\\	目: 
\\	兆: 
\\	目	
\\	め・めへん・よこめ	
\\	形声 
\\	唐	トウ	から	かろ, たん	
\\	广: 
\\	彑: 
\\	口: 
\\	口	
\\	くち・くちへん	
\\	会意 
\\	呉	ゴ	く.れる、くれ	ぐ	
\\	呉れる(くれる): 
\\	呉れ呉れも(くれぐれも): 
\\	口: 
\\	八: 
\\	口	
\\	くち・くちへん	
\\	会意 
\\	凡	ボン、ハン	およ.そ、おうよ.そ、すべ.て	なみ, ひろ, みな	
\\	平凡(へいぼん): 
\\	凡そ(およそ): 
\\	几: 
\\	丶: 
\\	几	
\\	つくえ・きにょう・かぜかんむり	
\\	象形 
\\	憩	ケイ	いこ.い、いこ.う		
\\	休憩(きゅうけい): 
\\	舌: 
\\	息: 
\\	心	
\\	こころ・したごころ・りっしんべん	
\\	形声 
\\	媛	エン	ひめ		
\\	女: 
\\	爰: 
\\	女	
\\	おんな・おんなへん	
\\	形声 
\\	溝	コウ	みぞ	どぶ	
\\	溝(みぞ): 
\\	溝(こう): 
\\	水: 
\\	冓: 
\\	水	
\\	みず・したみず・さんずい	
\\	形声 
\\	恭	キョウ	うやうや.しい	きよ, やす, やすし, ゆき, よし	
\\	共: 
\\	心: 
\\	心	
\\	こころ・したごころ・りっしんべん	
\\	形声 
\\	刈	ガイ、カイ	か.る	かっ, かり	
\\	刈る(かる): 
\\	乂: 
\\	刀: 
\\	刀	
\\	かたな・りっとう	
\\	象形 
\\	睡	スイ	ねむ.る、ねむ.い		
\\	睡眠(すいみん): 
\\	目: 
\\	垂: 
\\	目	
\\	め・めへん・よこめ	
\\	形声 
\\	錯	サク、シャク			
\\	錯覚(さっかく): 
\\	錯誤(さくご): 
\\	金: 
\\	昔: 
\\	金	
\\	かね・かねへん	
\\	形声 
\\	伯	ハク		いき, えき, か, き, は, ひろ	
\\	伯父さん(おじさん): 
\\	伯母さん(おばさん): 
\\	人: 
\\	白: 
\\	人	
\\	ひと・にんべん・ひとやね	
\\	形声 
\\	陵	リョウ	みささぎ		
\\	丘陵(きゅうりょう): 
\\	阜: 
\\	坴: 
\\	阜	
\\	こざと・こざとへん・おか・ぎふのふ	
\\	形声 
\\	霧	ム、ボウ、ブ	きり		
\\	霧(きり): 
\\	雨: 
\\	務: 
\\	雨	
\\	あめ・あめかんむり・あまかんむり	
\\	形声 
\\	魂	コン	たましい、たま		
\\	魂(たましい): 
\\	魂(こん): 
\\	云: 
\\	鬼: 
\\	鬼	
\\	おに・きにょう	
\\	会意 
\\	弊	ヘイ			
\\	敝: 
\\	廾: 
\\	廾	
\\	にじゅう・にじゅうあし・こまぬき	
\\	形声 
\\	妃	ヒ	きさき	き, ぴ, み	
\\	女: 
\\	己: 
\\	女	
\\	おんな・おんなへん	
\\	会意 
\\	舶	ハク			
\\	船舶(せんぱく): 
\\	舟: 
\\	白: 
\\	舟	
\\	ふね・ふねへん	
\\	形声 
\\	餓	ガ	う.える		
\\	食: 
\\	我: 
\\	食	
\\	しょく・しょくへん	
\\	形声 
\\	窮	キュウ、キョウ	きわ.める、きわ.まる、きわ.まり、きわ.み		
\\	窮屈(きゅうくつ): 
\\	窮乏(きゅうぼう): 
\\	穴: 
\\	躬: 
\\	穴	
\\	あな・あなかんむり	
\\	会意 
\\	掌	ショウ	てのひら、たなごころ		
\\	掌(てのひら): 
\\	車掌(しゃしょう): 
\\	尚: 
\\	手: 
\\	手	
\\	て・てへん	
\\	形声 
\\	麗	レイ	うるわ.しい、うら.らか	ま, よし, り	
\\	奇麗(きれい): 
\\	一: 
\\	一: 
\\	鹿: 
\\	鹿	
\\	しか・しかへん	
\\	象形 
\\	臭	シュウ	くさ.い、-くさ.い、にお.う、にお.い		
\\	生臭い(なまぐさい): 
\\	臭い(くさい): 
\\	自: 
\\	大: 
\\	自	
\\	みずから	臭	会意 
\\	悦	エツ	よろこ.ぶ、よろこ.ばす	や, よし	
\\	心: 
\\	兌: 
\\	心	
\\	こころ・したごころ・りっしんべん	
\\	形声 
\\	刃	ジン、ニン	は、やいば、き.る	ち, と	
\\	刃(は): 
\\	刀: 
\\	丶: 
\\	刀	
\\	かたな・りっとう	
\\	象形 
\\	縛	バク	しば.る		
\\	縛る(しばる): 
\\	束縛(そくばく): 
\\	糸: 
\\	尃: 
\\	糸	
\\	いと・いとへん	
\\	形声 
\\	暦	レキ	こよみ、りゃく		
\\	暦(こよみ): 
\\	西暦(せいれき): 
\\	還暦(かんれき): 
\\	厂: 
\\	林: 
\\	日: 
\\	日	
\\	ひ・ひへん・にち・にちへん	曆	会意 
\\	宜	ギ	よろ.しい、よろ.しく	き, たか, のぶ, のり, よし	
\\	便宜(べんぎ): 
\\	適宜(てきぎ): 
\\	宜しい(よろしい): 
\\	宜しく(よろしく): 
\\	どうぞ宜しく(どうぞよろしく): 
\\	宀: 
\\	且: 
\\	宀	
\\	うかんむり	
\\	象形 
\\	盲	モウ	めくら		
\\	盲点(もうてん): 
\\	亡: 
\\	目: 
\\	目	
\\	め・めへん・よこめ	
\\	形声 
\\	粋	スイ	いき		
\\	純粋(じゅんすい): 
\\	粋(いき): 
\\	米: 
\\	卆: 
\\	米	
\\	こめ・こめへん	粹	形声 
\\	辱	ジョク	はずかし.める		
\\	侮辱(ぶじょく): 
\\	辰: 
\\	寸: 
\\	辰	
\\	しんのたつ	
\\	会意 
\\	轄	カツ	くさび		
\\	車: 
\\	害: 
\\	車	
\\	くるま・くるまへん	
\\	形声 
\\	猿	エン	さる	さ, さわ, ざる, まし	
\\	猿(さる): 
\\	犬: 
\\	袁: 
\\	犬	
\\	いぬ・けものへん	
\\	形声 
\\	弦	ゲン	つる		
\\	弓: 
\\	玄: 
\\	弓	
\\	ゆみ・ゆみへん	
\\	形声 
\\	窒	チツ			
\\	窒息(ちっそく): 
\\	穴: 
\\	至: 
\\	穴	
\\	あな・あなかんむり	
\\	形声 
\\	炊	スイ	た.く、-だ.き	い	
\\	炊事(すいじ): 
\\	炊く(たく): 
\\	火: 
\\	欠: 
\\	火	
\\	ひ・ひへん・れっか・れんが	
\\	会意 
\\	洪	コウ		ほん	
\\	洪水(こうずい): 
\\	水: 
\\	共: 
\\	水	
\\	みず・したみず・さんずい	
\\	形声 
\\	摂	セツ、ショウ	おさ.める、かね.る、と.る	せっ	
\\	手: 
\\	聶: 
\\	冫: 
\\	手	
\\	て・てへん	攝	形声 
\\	飽	ホウ	あ.きる、あ.かす、あ.く	あき, あく	
\\	飽和(ほうわ): 
\\	飽きる(あきる): 
\\	飽く迄(あくまで): 
\\	食: 
\\	包: 
\\	食	
\\	しょく・しょくへん	
\\	形声 
\\	冗	ジョウ			
\\	冗談(じょうだん): 
\\	冖: 
\\	几: 
\\	冖	
\\	わかんむり	
\\	会意 
\\	桃	トウ	もも	み, もの	
\\	木: 
\\	兆: 
\\	木	
\\	き・きへん	
\\	形声 
\\	狩	シュ	か.る、か.り、-が.り	かり	
\\	狩り(かり): 
\\	犬: 
\\	守: 
\\	犬	
\\	いぬ・けものへん	
\\	形声 
\\	朱	シュ	あけ	あ, あか, あき, す	
\\	丿: 
\\	未: 
\\	木	
\\	き・きへん	
\\	象形 
\\	渦	カ	うず		
\\	渦(うず): 
\\	水: 
\\	咼: 
\\	水	
\\	みず・したみず・さんずい	
\\	形声 
\\	紳	シン		まこと	
\\	紳士(しんし): 
\\	糸: 
\\	申: 
\\	糸	
\\	いと・いとへん	
\\	形声 
\\	枢	スウ、シュ	とぼそ、からくり		
\\	中枢(ちゅうすう): 
\\	木: 
\\	区: 
\\	木	
\\	き・きへん	樞	会意 
\\	碑	ヒ	いしぶみ		
\\	碑(いしぶみ): 
\\	石: 
\\	卑: 
\\	石	
\\	いし・いしへん	碑	形声 
\\	鍛	タン	きた.える	か	
\\	鍛える(きたえる): 
\\	金: 
\\	段: 
\\	金	
\\	かね・かねへん	
\\	形声 
\\	鼓	コ	つづみ		
\\	太鼓(たいこ): 
\\	吉: 
\\	豆: 
\\	支: 
\\	鼓	
\\	つづみ	
\\	会意 
\\	裸	ラ	はだか		
\\	裸(はだか): 
\\	裸足(はだし): 
\\	衣: 
\\	果: 
\\	衣	
\\	ころも・ころもへん	
\\	形声 
\\	猶	ユウ、ユ	なお		
\\	犬: 
\\	酋: 
\\	犬	
\\	いぬ・けものへん	
\\	形声 
\\	塊	カイ、ケ	かたまり、つちくれ		
\\	塊(かたまり): 
\\	土: 
\\	鬼: 
\\	土	
\\	つち・つちへん・どへん	
\\	形声 
\\	旋	セン			
\\	斡旋(あっせん): 
\\	方: 
\\	疋: 
\\	方	
\\	ほう・ほうへん・かた・かたへん	
\\	会意 
\\	幣	ヘイ	ぬさ	しで	
\\	紙幣(しへい): 
\\	貨幣(かへい): 
\\	敝: 
\\	巾: 
\\	巾	
\\	はば・はばへん・きんべん	
\\	形声 
\\	膜	マク			
\\	膜(まく): 
\\	肉: 
\\	莫: 
\\	肉	
\\	にく・にくづき	
\\	形声 
\\	扇	セン	おうぎ	おう, おおぎ	
\\	扇風機(せんぷうき): 
\\	扇子(せんす): 
\\	団扇(うちわ): 
\\	扇ぐ(あおぐ): 
\\	戸: 
\\	羽: 
\\	戸	
\\	と・とだれ・とかんむり	
\\	会意 
\\	槽	ソウ	ふね		
\\	木: 
\\	曹: 
\\	木	
\\	き・きへん	
\\	形声 
\\	慈	ジ	いつく.しむ	しげ, ちか, めぐみ, よし	
\\	茲: 
\\	心: 
\\	心	
\\	こころ・したごころ・りっしんべん	
\\	形声 
\\	伐	バツ、ハツ、カ、ボチ	き.る、そむ.く、う.つ		
\\	人: 
\\	戈: 
\\	人	
\\	ひと・にんべん・ひとやね	
\\	会意 
\\	漬	シ	つ.ける、つ.かる、-づ.け、-づけ		
\\	漬ける(つける): 
\\	水: 
\\	責: 
\\	水	
\\	みず・したみず・さんずい	
\\	形声 
\\	糾	キュウ	ただ.す		
\\	糸: 
\\	丩: 
\\	糸	
\\	いと・いとへん	
\\	形声 
\\	墳	フン			
\\	土: 
\\	賁: 
\\	土	
\\	つち・つちへん・どへん	
\\	形声 
\\	坪	ヘイ	つぼ		
\\	土: 
\\	平: 
\\	土	
\\	つち・つちへん・どへん	
\\	形声 
\\	紺	コン			
\\	紺(こん): 
\\	糸: 
\\	甘: 
\\	糸	
\\	いと・いとへん	
\\	形声 
\\	慌	コウ	あわ.てる、あわ.ただしい		
\\	慌てる(あわてる): 
\\	慌ただしい(あわただしい): 
\\	心: 
\\	荒: 
\\	心	
\\	こころ・したごころ・りっしんべん	
\\	形声 
\\	娯	ゴ			
\\	娯楽(ごらく): 
\\	女: 
\\	呉: 
\\	女	
\\	おんな・おんなへん	
\\	形声 
\\	羅	ラ	うすもの		
\\	欧羅巴(ヨーロッパ): 
\\	网: 
\\	維: 
\\	网	
\\	あみ・あみがしら・あみめ・よこめ・よんかしら	
\\	会意 
\\	峡	キョウ、コウ	はざま	き, ば	
\\	海峡(かいきょう): 
\\	山: 
\\	夾: 
\\	山	
\\	やま・やまへん・やまかんむり	峽	形声 
\\	俸	ホウ			
\\	人: 
\\	奉: 
\\	人	
\\	ひと・にんべん・ひとやね	
\\	形声 
\\	厘	リン			
\\	厂: 
\\	里: 
\\	厂	
\\	がんだれ	
\\	形声 
\\	峰	ホウ	みね、ね	ぶ, ほ	
\\	峰(みね): 
\\	山: 
\\	夂: 
\\	丰: 
\\	山	
\\	やま・やまへん・やまかんむり	
\\	形声 
\\	醸	ジョウ	かも.す		
\\	酉: 
\\	襄: 
\\	酉	
\\	とり・とりへん・ひよみのとり・さけのとり	釀	形声 
\\	弔	チョウ	とむら.う、とぶら.う		
\\	弓: 
\\	丨: 
\\	弓	
\\	ゆみ・ゆみへん	
\\	象形 
\\	乙	オツ、イツ	おと-、きのと		
\\	乙(おつ): 
\\	乙: 
\\	乙	
\\	おつ・おつにょう・つりばり	
\\	象形 
\\	汁	ジュウ	しる、-しる、つゆ		
\\	汁(しる): 
\\	水: 
\\	十: 
\\	水	
\\	みず・したみず・さんずい	
\\	形声 
\\	尼	ニ	あま		
\\	尸: 
\\	匕: 
\\	尸	
\\	しかばね・かばね・かばねだれ	
\\	会意 
\\	遍	ヘン	あまね.く		
\\	普遍(ふへん): 
\\	遍(へん): 
\\	扁: 
\\	辶: 
\\	辵	
\\	しんにょう・しんにゅう	
\\	形声 
\\	衡	コウ		ひら	
\\	均衡(きんこう): 
\\	行: 
\\	魚: 
\\	行	
\\	ぎょう・ぎょうがまえ・いく・ゆきがまえ	
\\	会意 
\\	薫	クン	かお.る	かおり, かおる, かほ, かほる, くに, しげ, にほ, のぶ, よし	
\\	艸: 
\\	重: 
\\	火: 
\\	艸	
\\	くさ・くさかんむり	薰	形声 
\\	猟	リョウ、レフ	かり、か.る		
\\	犬: 
\\	鼠: 
\\	犬	
\\	いぬ・けものへん	獵	形声 
\\	款	カン		まさ	
\\	士: 
\\	示: 
\\	欠: 
\\	欠	
\\	あくび・けんづくり・かける・けつ	
\\	会意 
\\	閲	エツ	けみ.する		
\\	閲覧(えつらん): 
\\	門: 
\\	兌: 
\\	門	
\\	もん・もんがまえ・かどがまえ	
\\	形声 
\\	偵	テイ			
\\	人: 
\\	貞: 
\\	人	
\\	ひと・にんべん・ひとやね	
\\	形声 
\\	喝	カツ			
\\	口: 
\\	曷: 
\\	口	
\\	くち・くちへん	喝	形声 
\\	敢	カン	あ.えて、あ.えない、あ.えず		
\\	勇敢(ゆうかん): 
\\	敢えて(あえて): 
\\	耳: 
\\	攴: 
\\	攴	
\\	ぼくにょう・ぼくづくり・とまた・のぶん・しぶん	
\\	象形 
\\	胎	タイ			
\\	肉: 
\\	台: 
\\	肉	
\\	にく・にくづき	
\\	形声 
\\	酵	コウ			
\\	酉: 
\\	孝: 
\\	酉	
\\	とり・とりへん・ひよみのとり・さけのとり	
\\	形声 
\\	憤	フン	いきどお.る		
\\	憤慨(ふんがい): 
\\	心: 
\\	賁: 
\\	心	
\\	こころ・したごころ・りっしんべん	
\\	形声 
\\	豚	トン	ぶた		
\\	豚肉(ぶたにく): 
\\	肉: 
\\	豕: 
\\	豕	
\\	いのこ・いのこへん・ぶた	
\\	会意 
\\	遮	シャ	さえぎ.る		
\\	遮る(さえぎる): 
\\	庶: 
\\	辶: 
\\	辵	
\\	しんにょう・しんにゅう	
\\	形声 
\\	扉	ヒ	とびら		
\\	扉(とびら): 
\\	戸: 
\\	非: 
\\	戸	
\\	と・とだれ・とかんむり	
\\	形声 
\\	硫	リュウ			
\\	石: 
\\	㐬: 
\\	石	
\\	いし・いしへん	
\\	形声 
\\	赦	シャ			
\\	赤: 
\\	攴: 
\\	赤	
\\	あか・あかへん	
\\	会意 
\\	窃	セツ	ぬす.む、ひそ.か		
\\	穴: 
\\	切: 
\\	穴	
\\	あな・あなかんむり	竊	会意 
\\	泡	ホウ	あわ		
\\	泡(あぶく): 
\\	水: 
\\	包: 
\\	水	
\\	みず・したみず・さんずい	
\\	形声 
\\	又	ユウ	また、また-、また.の-	やす	
\\	又は(または): 
\\	又(また): 
\\	又: 
\\	又	
\\	また	
\\	象形 
\\	慨	ガイ			
\\	憤慨(ふんがい): 
\\	心: 
\\	既: 
\\	心	
\\	こころ・したごころ・りっしんべん	慨	形声 
\\	紡	ボウ	つむ.ぐ		
\\	紡績(ぼうせき): 
\\	糸: 
\\	方: 
\\	糸	
\\	いと・いとへん	
\\	形声 
\\	恨	コン	うら.む、うら.めしい		
\\	恨み(うらみ): 
\\	恨む(うらむ): 
\\	心: 
\\	艮: 
\\	心	
\\	こころ・したごころ・りっしんべん	
\\	形声 
\\	肪	ボウ			
\\	脂肪(しぼう): 
\\	肉: 
\\	方: 
\\	肉	
\\	にく・にくづき	
\\	形声 
\\	扶	フ	たす.ける		
\\	扶養(ふよう): 
\\	手: 
\\	夫: 
\\	手	
\\	て・てへん	
\\	形声 
\\	戯	ギ、ゲ	たわむ.れる、ざ.れる、じゃ.れる		
\\	戯曲(ぎきょく): 
\\	悪戯(いたずら): 
\\	巫山戯る(ふざける): 
\\	虚: 
\\	戈: 
\\	戈	
\\	ほこ・ほこづくり・ほこがまえ・かのほこ	戲	会意 
\\	忌	キ	い.む、い.み、い.まわしい	いまわ	
\\	己: 
\\	心: 
\\	心	
\\	こころ・したごころ・りっしんべん	
\\	形声 
\\	濁	ダク、ジョク	にご.る、にご.す	にごり	
\\	濁る(にごる): 
\\	清濁(せいだく): 
\\	水: 
\\	蜀: 
\\	水	
\\	みず・したみず・さんずい	
\\	形声 
\\	奔	ホン	はし.る		
\\	大: 
\\	卉: 
\\	大	
\\	だい・おおきい	
\\	会意 
\\	斗	ト、トウ		ます	
\\	丶: 
\\	斗	
\\	と・とます・ますづくり	
\\	象形 
\\	迅	ジン			
\\	迅速(じんそく): 
\\	卂: 
\\	辶: 
\\	辵	
\\	しんにょう・しんにゅう	
\\	形声 
\\	肖	ショウ	あやか.る		
\\	小: 
\\	月: 
\\	肉	
\\	にく・にくづき	
\\	形声 
\\	鉢	ハチ、ハツ			
\\	鉢(はち): 
\\	金: 
\\	本: 
\\	金	
\\	かね・かねへん	
\\	形声 
\\	朽	キュウ	く.ちる	くつ	
\\	朽ちる(くちる): 
\\	木: 
\\	丂: 
\\	木	
\\	き・きへん	
\\	形声 
\\	殻	カク、コク、バイ	から、がら		
\\	殻(から): 
\\	壳: 
\\	殳: 
\\	殳	
\\	るまた・ほこづくり	殼	会意 
\\	享	キョウ、コウ	う.ける	たか, たかし, とおる, みち	
\\	享受(きょうじゅ): 
\\	亠: 
\\	口: 
\\	子: 
\\	亠	
\\	けいさん・けいさんかんむり・なべぶた	
\\	象形 
\\	藩	ハン			
\\	艸: 
\\	潘: 
\\	艸	
\\	くさ・くさかんむり	
\\	形声 
\\	沙	サ、シャ	すな、よなげる		
\\	ご無沙汰(ごぶさた): 
\\	無沙汰(ぶさた): 
\\	水: 
\\	少: 
\\	水	
\\	みず・したみず・さんずい	
\\	形声 
\\	媒	バイ	なこうど		
\\	女: 
\\	某: 
\\	女	
\\	おんな・おんなへん	
\\	形声 
\\	鶏	ケイ	にわとり、とり		
\\	鶏肉(けいにく): 
\\	奚: 
\\	鳥: 
\\	鳥	
\\	とり・とりへん	鷄	形声 
\\	禅	ゼン、セン	しずか、ゆず.る		
\\	禅(ぜん): 
\\	示: 
\\	単: 
\\	示	
\\	しめす・しめすへん	禪	形声 
\\	嘱	ショク	しょく.する、たの.む		
\\	口: 
\\	属: 
\\	口	
\\	くち・くちへん	囑	形声 
\\	胴	ドウ			
\\	胴(どう): 
\\	肉: 
\\	同: 
\\	肉	
\\	にく・にくづき	
\\	形声 
\\	迭	テツ			
\\	失: 
\\	辶: 
\\	辵	
\\	しんにょう・しんにゅう	
\\	形声 
\\	挿	ソウ	さ.す、はさ.む		
\\	挿す(さす): 
\\	手: 
\\	千: 
\\	十: 
\\	日: 
\\	手	
\\	て・てへん	插	形声 
\\	嵐	ラン	あらし	ぞれ	
\\	嵐(あらし): 
\\	山: 
\\	風: 
\\	山	
\\	やま・やまへん・やまかんむり	
\\	会意 
\\	椎	ツイ、スイ	つち、う.つ	しい	
\\	木: 
\\	隹: 
\\	木	
\\	き・きへん	
\\	形声 
\\	陪	バイ			
\\	阜: 
\\	立: 
\\	口: 
\\	阜	
\\	こざと・こざとへん・おか・ぎふのふ	
\\	形声 
\\	剖	ボウ			
\\	解剖(かいぼう): 
\\	立: 
\\	口: 
\\	刀: 
\\	刀	
\\	かたな・りっとう	
\\	形声 
\\	譜	フ			
\\	楽譜(がくふ): 
\\	言: 
\\	普: 
\\	言	
\\	こと・げん・ごんべん・ことば	
\\	形声 
\\	悠	ユウ		ゆ	
\\	悠々(ゆうゆう): 
\\	攸: 
\\	心: 
\\	心	
\\	こころ・したごころ・りっしんべん	
\\	形声 
\\	淑	シュク	しと.やか	すく, とし, よし	
\\	淑やか(しとやか): 
\\	水: 
\\	叔: 
\\	水	
\\	みず・したみず・さんずい	
\\	形声 
\\	帆	ハン	ほ		
\\	巾: 
\\	凡: 
\\	巾	
\\	はば・はばへん・きんべん	
\\	形声 
\\	暁	ギョウ、キョウ	あかつき、さと.る	あき, あきら, あけ, さと, さとる, てる	
\\	日: 
\\	尭: 
\\	日	
\\	ひ・ひへん・にち・にちへん	曉	形声 
\\	傑	ケツ	すぐ.れる	すぐ, たけ, まさ	
\\	傑作(けっさく): 
\\	人: 
\\	桀: 
\\	人	
\\	ひと・にんべん・ひとやね	
\\	形声 
\\	奴	ド	やつ、やっこ	ぬ	
\\	奴(やつ): 
\\	奴(やっこ): 
\\	女: 
\\	又: 
\\	女	
\\	おんな・おんなへん	
\\	会意 
\\	錠	ジョウ			
\\	手錠(てじょう): 
\\	金: 
\\	定: 
\\	金	
\\	かね・かねへん	
\\	形声 
\\	拳	ケン、ゲン	こぶし		
\\	じゃん拳(じゃんけん): 
\\	二: 
\\	人: 
\\	手: 
\\	手	
\\	て・てへん	
\\	形声 
\\	遷	セン	うつ.る、うつ.す、みやこがえ		
\\	変遷(へんせん): 
\\	襾: 
\\	大: 
\\	己: 
\\	辶: 
\\	辵	
\\	しんにょう・しんにゅう	
\\	形声 
\\	拙	セツ	つたな.い		
\\	拙い(つたない): 
\\	手: 
\\	出: 
\\	手	
\\	て・てへん	
\\	形声 
\\	侍	ジ、シ	さむらい、はべ.る	かみ	
\\	侍(さむらい): 
\\	人: 
\\	寺: 
\\	人	
\\	ひと・にんべん・ひとやね	
\\	形声 
\\	峠		とうげ		
\\	峠(とうげ): 
\\	山: 
\\	上: 
\\	下: 
\\	山	
\\	やま・やまへん・やまかんむり	
\\	国字 
\\	篤	トク	あつ.い		
\\	竹: 
\\	馬: 
\\	竹	
\\	たけ・たけかんむり	
\\	会意 
\\	渇	カツ	かわ.く		
\\	渇く(かわく): 
\\	水: 
\\	曷: 
\\	水	
\\	みず・したみず・さんずい	渴	形声 
\\	叔	シュク		よし	
\\	叔父(おじ): 
\\	叔母(おば): 
\\	叔父(おじ): 
\\	上: 
\\	小: 
\\	又: 
\\	又	
\\	また	
\\	会意 
\\	雌	シ	め-、めす、めん		
\\	雌(めす): 
\\	此: 
\\	隹: 
\\	隹	
\\	ふるとり	
\\	形声 
\\	堪	カン、タン	た.える、たま.る、こら.える、こた.える		
\\	堪える(こらえる): 
\\	堪らない(たまらない): 
\\	土: 
\\	甚: 
\\	土	
\\	つち・つちへん・どへん	
\\	形声 
\\	叙	ジョ	つい.ず、ついで		
\\	余: 
\\	又: 
\\	攴	
\\	ぼくにょう・ぼくづくり・とまた・のぶん・しぶん	敍	会意 
\\	酢	サク	す		
\\	酢(す): 
\\	酉: 
\\	乍: 
\\	酉	
\\	とり・とりへん・ひよみのとり・さけのとり	
\\	形声 
\\	吟	ギン			
\\	吟味(ぎんみ): 
\\	口: 
\\	今: 
\\	口	
\\	くち・くちへん	
\\	形声 
\\	逓	テイ	かわ.る、たがいに		
\\	乕: 
\\	辶: 
\\	辵	
\\	しんにょう・しんにゅう	遞	形声 
\\	甚	ジン	はなは.だ、はなは.だしい	じ	
\\	甚だ(はなはだ): 
\\	甚だしい(はなはだしい): 
\\	甘: 
\\	匹: 
\\	甘	
\\	あまい・かん	
\\	象形 
\\	崇	スウ	あが.める	す, たか, たかし, たかむ	
\\	崇拝(すうはい): 
\\	山: 
\\	宗: 
\\	山	
\\	やま・やまへん・やまかんむり	
\\	形声 
\\	漆	シツ	うるし	うる	
\\	水: 
\\	桼: 
\\	水	
\\	みず・したみず・さんずい	
\\	形声 
\\	岬	コウ	みさき	さき	
\\	岬(みさき): 
\\	山: 
\\	甲: 
\\	山	
\\	やま・やまへん・やまかんむり	
\\	形声 
\\	癖	ヘキ	くせ、くせ.に		
\\	癖(くせ): 
\\	疒: 
\\	辟: 
\\	疒	
\\	やまい・やまいだれ	
\\	形声 
\\	愉	ユ	たの.しい、たの.しむ	ゆう	
\\	愉快(ゆかい): 
\\	心: 
\\	兪: 
\\	心	
\\	こころ・したごころ・りっしんべん	
\\	形声 
\\	礁	ショウ			
\\	石: 
\\	焦: 
\\	石	
\\	いし・いしへん	
\\	形声 
\\	屯	トン	たむろ		
\\	屯(トン): 
\\	丿: 
\\	屮: 
\\	屮	
\\	くさのめ・めばえ・てつ	
\\	象形 
\\	姻	イン			
\\	女: 
\\	因: 
\\	女	
\\	おんな・おんなへん	
\\	形声 
\\	擬	ギ	まが.い、もど.き		
\\	手: 
\\	疑: 
\\	手	
\\	て・てへん	
\\	形声 
\\	塀	ヘイ、ベイ			
\\	塀(へい): 
\\	土: 
\\	屏: 
\\	土	
\\	つち・つちへん・どへん	塀	形声 
\\	唇	シン	くちびる		
\\	唇(くちびる): 
\\	辰: 
\\	口: 
\\	口	
\\	くち・くちへん	
\\	形声 
\\	睦	ボク、モク	むつ.まじい、むつ.む、むつ.ぶ	むつ, ちか, よし, あつ, む, むね	
\\	目: 
\\	土: 
\\	八: 
\\	土: 
\\	目	
\\	め・めへん・よこめ	
\\	形声 
\\	閑	カン		が, より	
\\	長閑(のどか): 
\\	門: 
\\	木: 
\\	門	
\\	もん・もんがまえ・かどがまえ	
\\	会意 
\\	幽	ユウ	ふか.い、かす.か、くら.い、しろ.い		
\\	幽霊(ゆうれい): 
\\	山: 
\\	幺: 
\\	幺: 
\\	幺	
\\	いとがしら・よう	
\\	会意 
\\	曹	ソウ、ゾウ			
\\	曲: 
\\	日: 
\\	曰	
\\	ひらび・いわく	
\\	会意 
\\	詠	エイ	よ.む、うた.う	うた, え, ええ	
\\	言: 
\\	永: 
\\	言	
\\	こと・げん・ごんべん・ことば	
\\	形声 
\\	卑	ヒ	いや.しい、いや.しむ、いや.しめる		
\\	卑怯(ひきょう): 
\\	卑しい(いやしい): 
\\	丿: 
\\	田: 
\\	丿: 
\\	十: 
\\	十	
\\	じゅう・じゅうへん	卑	会意 
\\	侮	ブ	あなど.る、あなず.る		
\\	侮辱(ぶじょく): 
\\	人: 
\\	毎: 
\\	人	
\\	ひと・にんべん・ひとやね	侮	形声 
\\	鋳	チュウ、イ、シュ、シュウ	い.る		
\\	金: 
\\	寿: 
\\	金	
\\	かね・かねへん	鑄	形声 
\\	抹	マツ			
\\	手: 
\\	末: 
\\	手	
\\	て・てへん	
\\	形声 
\\	尉	イ、ジョウ			
\\	尸: 
\\	示: 
\\	寸: 
\\	寸	
\\	すん・すんづくり	
\\	会意 
\\	隷	レイ	したが.う、しもべ		
\\	士: 
\\	示: 
\\	隶: 
\\	隶	
\\	たい・たいづくり・れいづくり・れいのつくり	
\\	会意 
\\	禍	カ	わざわい		
\\	示: 
\\	咼: 
\\	示	
\\	しめす・しめすへん	禍	形声 
\\	酪	ラク			
\\	酪農(らくのう): 
\\	酉: 
\\	各: 
\\	酉	
\\	とり・とりへん・ひよみのとり・さけのとり	
\\	形声 
\\	茎	ケイ、キョウ	くき		
\\	茎(くき): 
\\	艸: 
\\	圣: 
\\	艸	
\\	くさ・くさかんむり	莖	形声 
\\	帥	スイ			
\\	丿: 
\\	巾: 
\\	巾	
\\	はば・はばへん・きんべん	
\\	会意 
\\	逝	セイ	ゆ.く、い.く		
\\	折: 
\\	辶: 
\\	辵	
\\	しんにょう・しんにゅう	
\\	形声 
\\	匿	トク	かくま.う		
\\	匚: 
\\	若: 
\\	匸	
\\	かくしがまえ	
\\	会意 
\\	襟	キン	えり		
\\	襟(えり): 
\\	衣: 
\\	禁: 
\\	衣	
\\	ころも・ころもへん	
\\	形声 
\\	蛍	ケイ	ほたる		
\\	蛍光灯(けいこうとう): 
\\	小: 
\\	冖: 
\\	虫: 
\\	虫	
\\	むし・むしへん	螢	形声 
\\	寡	カ			
\\	宀: 
\\	自: 
\\	分: 
\\	宀	
\\	うかんむり	
\\	会意 
\\	痢	リ			
\\	下痢(げり): 
\\	疒: 
\\	利: 
\\	疒	
\\	やまい・やまいだれ	
\\	形声 
\\	庸	ヨウ		つね, のぶ, やす	
\\	广: 
\\	彑: 
\\	用: 
\\	广	
\\	まだれ	
\\	会意 
\\	坑	コウ			
\\	土: 
\\	亢: 
\\	土	
\\	つち・つちへん・どへん	
\\	形声 
\\	藍	ラン	あい		
\\	艸: 
\\	監: 
\\	艸	
\\	くさ・くさかんむり	
\\	形声 
\\	賊	ゾク			
\\	貝: 
\\	戎: 
\\	貝	
\\	かい・かいへん・こがい	
\\	会意 
\\	搾	サク	しぼ.る		
\\	搾る(しぼる): 
\\	手: 
\\	窄: 
\\	手	
\\	て・てへん	
\\	形声 
\\	畔	ハン	あぜ、くろ、ほとり	ぐろ	
\\	田: 
\\	半: 
\\	田	
\\	た・たへん	
\\	形声 
\\	唄	バイ	うた、うた.う		
\\	唄う(うたう): 
\\	口: 
\\	貝: 
\\	口	
\\	くち・くちへん	
\\	形声 
\\	孔	コウ	あな	のり	
\\	子: 
\\	乙: 
\\	子	
\\	こ・こへん	
\\	象形 
\\	呂	ロ、リョ	せぼね	とも, なが	
\\	風呂(ふろ): 
\\	風呂敷(ふろしき): 
\\	お風呂(おふろ): 
\\	口: 
\\	丿: 
\\	口: 
\\	口	
\\	くち・くちへん	
\\	象形 
\\	拷	ゴウ			
\\	手: 
\\	考: 
\\	手	
\\	て・てへん	
\\	形声 
\\	嬢	ジョウ	むすめ		
\\	お嬢さん(おじょうさん): 
\\	嬢(じょう): 
\\	女: 
\\	襄: 
\\	女	
\\	おんな・おんなへん	孃	形声 
\\	渓	ケイ	たに、たにがわ		
\\	水: 
\\	奚: 
\\	水	
\\	みず・したみず・さんずい	溪	形声 
\\	翁	オウ	おきな	お, おな	
\\	公: 
\\	羽: 
\\	羽	
\\	はね	
\\	形声 
\\	廉	レン		きよ	
\\	广: 
\\	兼: 
\\	广	
\\	まだれ	
\\	形声 
\\	謹	キン	つつし.む		
\\	謹む(つつしむ): 
\\	言: 
\\	言	
\\	こと・げん・ごんべん・ことば	謹	形声 
\\	瞳	ドウ、トウ	ひとみ	あきら	
\\	瞳(ひとみ): 
\\	目: 
\\	童: 
\\	目	
\\	め・めへん・よこめ	
\\	形声 
\\	湧	ユウ、ヨウ、ユ	わ.く	わき, わく	
\\	湧く(わく): 
\\	水: 
\\	勇: 
\\	水	
\\	みず・したみず・さんずい	
\\	形声 
\\	窯	ヨウ	かま		
\\	穴: 
\\	羔: 
\\	穴	
\\	あな・あなかんむり	
\\	形声 
\\	褒	ホウ	ほ.める		
\\	褒める(ほめる): 
\\	褒美(ほうび): 
\\	衣: 
\\	保: 
\\	衣	
\\	ころも・ころもへん	襃	形声 
\\	醜	シュウ	みにく.い、しこ		
\\	醜い(みにくい): 
\\	酉: 
\\	鬼: 
\\	酉	
\\	とり・とりへん・ひよみのとり・さけのとり	
\\	形声 
\\	升	ショウ	ます	のぼる	
\\	丿: 
\\	廾: 
\\	十	
\\	じゅう・じゅうへん	
\\	象形 
\\	殉	ジュン			
\\	歹: 
\\	旬: 
\\	歹	
\\	がつ・がってん・がつへん・かばね・かばねへん・いちたへん	
\\	形声 
\\	煩	ハン、ボン	わずら.う、わずら.わす、うるさ.がる、うるさ.い		
\\	煩わしい(わずらわしい): 
\\	煩い(うるさい): 
\\	火: 
\\	頁: 
\\	火	
\\	ひ・ひへん・れっか・れんが	
\\	会意 
\\	劾	ガイ			
\\	亥: 
\\	力: 
\\	力	
\\	ちから・りきづくり	
\\	形声 
\\	堕	ダ	お.ちる、くず.す、くず.れる		
\\	陏: 
\\	土: 
\\	土	
\\	つち・つちへん・どへん	墮	形声 
\\	租	ソ			
\\	禾: 
\\	且: 
\\	禾	
\\	のぎ・のぎへん	
\\	形声 
\\	桟	サン、セン	かけはし	さ	
\\	桟橋(さんばし): 
\\	木: 
\\	戔: 
\\	木	
\\	き・きへん	棧	形声 
\\	婿	セイ	むこ		
\\	婿(むこ): 
\\	女: 
\\	胥: 
\\	女	
\\	おんな・おんなへん	
\\	形声 
\\	慕	ボ	した.う		
\\	慕う(したう): 
\\	莫: 
\\	心: 
\\	心	
\\	こころ・したごころ・りっしんべん	
\\	形声 
\\	罷	ヒ	まか.り-、や.める		
\\	网: 
\\	能: 
\\	网	
\\	あみ・あみがしら・あみめ・よこめ・よんかしら	
\\	会意 
\\	矯	キョウ	た.める		
\\	矢: 
\\	喬: 
\\	矢	
\\	や・やへん	
\\	形声 
\\	某	ボウ	それがし、なにがし		
\\	甘: 
\\	木: 
\\	木	
\\	き・きへん	
\\	会意 
\\	囚	シュウ	とら.われる		
\\	囗: 
\\	人: 
\\	囗	
\\	くに・くにがまえ	
\\	会意 
\\	虹	コウ	にじ		
\\	虹(にじ): 
\\	虫: 
\\	工: 
\\	虫	
\\	むし・むしへん	
\\	形声 
\\	泌	ヒツ、ヒ			
\\	水: 
\\	必: 
\\	水	
\\	みず・したみず・さんずい	
\\	形声 
\\	漸	ゼン	ようや.く、やや、ようよ.う、すす.む		
\\	漸く(ようやく): 
\\	水: 
\\	斬: 
\\	水	
\\	みず・したみず・さんずい	
\\	形声 
\\	蚊	ブン	か		
\\	蚊(か): 
\\	虫: 
\\	文: 
\\	虫	
\\	むし・むしへん	
\\	形声 
\\	厄	ヤク			
\\	厄介(やっかい): 
\\	厂: 
\\	卩: 
\\	厂	
\\	がんだれ	
\\	形声 
\\	藻	ソウ	も		
\\	藻掻く(もがく): 
\\	艸: 
\\	澡: 
\\	艸	
\\	くさ・くさかんむり	
\\	形声 
\\	嫡	チャク、テキ			
\\	女: 
\\	亠: 
\\	冂: 
\\	古: 
\\	女	
\\	おんな・おんなへん	
\\	形声 
\\	嚇	カク	おど.す		
\\	口: 
\\	赫: 
\\	口	
\\	くち・くちへん	
\\	形声 
\\	凸	トツ	でこ		
\\	凸凹(でこぼこ): 
\\	凵: 
\\	凵	
\\	かんにょう・うけばこ・かんがまえ	
\\	象形 
\\	韻	イン			
\\	音: 
\\	員: 
\\	音	
\\	おと・おとへん	
\\	形声 
\\	霜	ソウ	しも		
\\	霜(しも): 
\\	雨: 
\\	相: 
\\	雨	
\\	あめ・あめかんむり・あまかんむり	
\\	形声 
\\	硝	ショウ			
\\	硝子(ガラス): 
\\	石: 
\\	肖: 
\\	石	
\\	いし・いしへん	
\\	形声 
\\	勅	チョク	いまし.める、みことのり	て, のり	
\\	束: 
\\	力: 
\\	力	
\\	ちから・りきづくり	敕	会意 
\\	棺	カン			
\\	木: 
\\	官: 
\\	木	
\\	き・きへん	
\\	形声 
\\	儒	ジュ			
\\	人: 
\\	需: 
\\	人	
\\	ひと・にんべん・ひとやね	
\\	形声 
\\	愁	シュウ	うれ.える、うれ.い		
\\	郷愁(きょうしゅう): 
\\	秋: 
\\	心: 
\\	心	
\\	こころ・したごころ・りっしんべん	
\\	形声 
\\	楼	ロウ	たかどの		
\\	木: 
\\	婁: 
\\	木	
\\	き・きへん	樓	形声 
\\	眉	ビ、ミ	まゆ		
\\	眉(まゆ): 
\\	丿: 
\\	目: 
\\	目	
\\	め・めへん・よこめ	
\\	象形 
\\	薪	シン	たきぎ、まき		
\\	艸: 
\\	新: 
\\	艸	
\\	くさ・くさかんむり	
\\	形声 
\\	褐	カツ			
\\	衣: 
\\	曷: 
\\	衣	
\\	ころも・ころもへん	褐	形声 
\\	賜	シ	たまわ.る、たま.う、たも.う		
\\	賜る(たまわる): 
\\	貝: 
\\	易: 
\\	貝	
\\	かい・かいへん・こがい	
\\	形声 
\\	繕	ゼン	つくろ.う		
\\	修繕(しゅうぜん): 
\\	繕う(つくろう): 
\\	糸: 
\\	善: 
\\	糸	
\\	いと・いとへん	
\\	形声 
\\	栓	セン			
\\	栓(せん): 
\\	木: 
\\	全: 
\\	木	
\\	き・きへん	
\\	形声 
\\	凹	オウ	くぼ.む、へこ.む、ぼこ		
\\	凸凹(でこぼこ): 
\\	凹む(へこむ): 
\\	凵: 
\\	凵	
\\	かんにょう・うけばこ・かんがまえ	
\\	象形 
\\	艶	エン	つや、なま.めかしい、あで.やか、つや.めく、なま.めく	もろ, よし	
\\	艶(つや): 
\\	艶(えん): 
\\	豊: 
\\	色: 
\\	色	
\\	いろ	艷	会意 
\\	錬	レン	ね.る		
\\	金: 
\\	東: 
\\	金	
\\	かね・かねへん	鍊	形声 
\\	衷	チュウ			
\\	折衷(せっちゅう): 
\\	衣: 
\\	口: 
\\	丨: 
\\	衣	
\\	ころも・ころもへん	
\\	形声 
\\	逐	チク			
\\	豕: 
\\	辶: 
\\	辵	
\\	しんにょう・しんにゅう	
\\	会意 
\\	斥	セキ	しりぞ.ける		
\\	斤: 
\\	丶: 
\\	斤	
\\	おの・おのづくり・きん	
\\	象形 
\\	詔	ショウ	みことのり	さとし, のり	
\\	言: 
\\	召: 
\\	言	
\\	こと・げん・ごんべん・ことば	
\\	形声 
\\	憧	ショウ、トウ、ドウ	あこが.れる		
\\	憧れ(あこがれ): 
\\	憧れる(あこがれる): 
\\	心: 
\\	童: 
\\	心	
\\	こころ・したごころ・りっしんべん	
\\	形声 
\\	宵	ショウ	よい		
\\	宀: 
\\	肖: 
\\	宀	
\\	うかんむり	
\\	形声 
\\	妄	モウ、ボウ	みだ.りに		
\\	亡: 
\\	女: 
\\	女	
\\	おんな・おんなへん	
\\	形声 
\\	酌	シャク	く.む		
\\	酌む(くむ): 
\\	酉: 
\\	勺: 
\\	酉	
\\	とり・とりへん・ひよみのとり・さけのとり	
\\	形声 
\\	頒	ハン	わか.つ		
\\	分: 
\\	頁: 
\\	頁	
\\	おおがい・いちのかい	
\\	形声 
\\	肢	シ			
\\	肉: 
\\	支: 
\\	肉	
\\	にく・にくづき	
\\	形声 
\\	謄	トウ			
\\	朕: 
\\	言: 
\\	言	
\\	こと・げん・ごんべん・ことば	
\\	形声 
\\	嗣	シ		あき, つかさ, つぎ, つぐ, ひで	
\\	口: 
\\	冊: 
\\	司: 
\\	口	
\\	くち・くちへん	
\\	会意 
\\	畝	ボウ、ホ、モ、ム	せ、うね		
\\	亠: 
\\	田: 
\\	久: 
\\	田	
\\	た・たへん	
\\	形声 
\\	抄	ショウ		さ, り	
\\	手: 
\\	少: 
\\	手	
\\	て・てへん	
\\	形声 
\\	爽	ソウ	あき.らか、さわ.やか、たがう		
\\	爽やか(さわやか): 
\\	大: 
\\	爻: 
\\	爻: 
\\	爻	
\\	こう・めめ	
\\	会意 
\\	惰	ダ			
\\	心: 
\\	左: 
\\	月: 
\\	心	
\\	こころ・したごころ・りっしんべん	
\\	形声 
\\	蛮	バン	えびす		
\\	亦: 
\\	虫: 
\\	虫	
\\	むし・むしへん	蠻	形声 
\\	旺	オウ、キョウ、ゴウ	かがや.き、うつくし.い、さかん		
\\	日: 
\\	玉: 
\\	日	
\\	ひ・ひへん・にち・にちへん	
\\	形声 
\\	壱	イチ、イツ	ひとつ	い, かず	
\\	一)			
\\	土: 
\\	冖: 
\\	匕: 
\\	士	
\\	さむらい	壹	象形 
\\	瑠	ル、リュウ		るり	
\\	玉: 
\\	留: 
\\	玉	
\\	たま・たまへん ・ おう・おうへん・あうさま	
\\	形声 
\\	侯	コウ			
\\	人: 
\\	矢: 
\\	人	
\\	ひと・にんべん・ひとやね	
\\	会意 
\\	弧	コ			
\\	括弧(かっこ): 
\\	弓: 
\\	瓜: 
\\	弓	
\\	ゆみ・ゆみへん	
\\	形声 
\\	附	フ	つ.ける、つ.く	ずき, づけ	
\\	寄附(きふ): 
\\	附属(ふぞく): 
\\	阜: 
\\	付: 
\\	阜	
\\	こざと・こざとへん・おか・ぎふのふ	
\\	形声 
\\	但	タン	ただ.し	たじ	
\\	但し(ただし): 
\\	人: 
\\	旦: 
\\	人	
\\	ひと・にんべん・ひとやね	
\\	形声 
\\	芋	ウ	いも		
\\	艸: 
\\	于: 
\\	艸	
\\	くさ・くさかんむり	
\\	形声 
\\	婆	バ	ばば、ばあ		
\\	波: 
\\	女: 
\\	女	
\\	おんな・おんなへん	
\\	形声 
\\	倣	ホウ	なら.う		
\\	模倣(もほう): 
\\	倣う(ならう): 
\\	人: 
\\	放: 
\\	人	
\\	ひと・にんべん・ひとやね	
\\	形声 
\\	倹	ケン	つま.しい、つづまやか		
\\	倹約(けんやく): 
\\	人: 
\\	僉: 
\\	人	
\\	ひと・にんべん・ひとやね	儉	形声 
\\	繭	ケン	まゆ、きぬ		
\\	艸: 
\\	冂: 
\\	糸: 
\\	虫: 
\\	糸	
\\	いと・いとへん	
\\	会意 
\\	謁	エツ			
\\	言: 
\\	曷: 
\\	言	
\\	こと・げん・ごんべん・ことば	謁	形声 
\\	箇	カ、コ			
\\	箇所(かしょ): 
\\	箇条書き(かじょうがき): 
\\	竹: 
\\	固: 
\\	竹	
\\	たけ・たけかんむり	
\\	形声 
\\	且	ショ、ソ、ショウ	か.つ	あき, かつ	
\\	且つ(かつ): 
\\	月: 
\\	一: 
\\	一	
\\	いち	
\\	象形 
\\	伎	ギ、キ	わざ、わざおぎ		
\\	人: 
\\	支: 
\\	人	
\\	ひと・にんべん・ひとやね	
\\	形声 
\\	斤	キン			
\\	斤: 
\\	斤	
\\	おの・おのづくり・きん	
\\	象形 
\\	虞	グ	おそれ、おもんぱか.る、はか.る、うれ.える、あざむ.く、あやま.る、のぞ.む、たの.しむ	すけ, もち, やす	
\\	虍: 
\\	呉: 
\\	虍	
\\	とら・とらがしら・とらかんむり	
\\	形声 
\\	墾	コン			
\\	豸: 
\\	艮: 
\\	土: 
\\	土	
\\	つち・つちへん・どへん	
\\	形声 
\\	采	サイ	と.る、いろどり		
\\	爪: 
\\	木: 
\\	釆	
\\	のごめ・のごめへん	
\\	会意 
\\	璽	ジ			
\\	爾: 
\\	玉: 
\\	玉	
\\	たま・たまへん ・ おう・おうへん・あうさま	
\\	形声 
\\	爵	シャク			
\\	爪: 
\\	网: 
\\	艮: 
\\	寸: 
\\	爪	
\\	つめ・つめがしら・つめかんむり・そうにょう・のつ	
\\	象形 
\\	遵	ジュン			
\\	尊: 
\\	辶: 
\\	辵	
\\	しんにょう・しんにゅう	
\\	形声 
\\	塑	ソ	でく		
\\	朔: 
\\	土: 
\\	土	
\\	つち・つちへん・どへん	
\\	形声 
\\	汰	タ、タイ	おご.る、にご.る、よな.げる		
\\	ご無沙汰(ごぶさた): 
\\	無沙汰(ぶさた): 
\\	水: 
\\	太: 
\\	水	
\\	みず・したみず・さんずい	
\\	形声 
\\	旦	タン、ダン	あき.らか、あきら、ただし、あさ、あした		
\\	旦那(だんな): 
\\	一旦(いったん): 
\\	日: 
\\	一: 
\\	日	
\\	ひ・ひへん・にち・にちへん	
\\	象形 
\\	朕	チン			
\\	肉: 
\\	天: 
\\	月	
\\	つき・つきへん	
\\	会意 
\\	痘	トウ			
\\	疒: 
\\	豆: 
\\	疒	
\\	やまい・やまいだれ	
\\	形声 
\\	弐	ニ、ジ	ふた.つ、そえ		
\\	二 
\\	一: 
\\	弋: 
\\	二: 
\\	貝	
\\	かい・かいへん・こがい	貳	形声 
\\	賦	フ、ブ		うた	
\\	月賦(げっぷ): 
\\	貝: 
\\	武: 
\\	貝	
\\	かい・かいへん・こがい	
\\	形声 
\\	丙	ヘイ	ひのえ		
\\	一: 
\\	内: 
\\	一	
\\	いち	
\\	象形 
\\	耗	モウ、コウ、カウ			
\\	消耗(しょうもう): 
\\	耒: 
\\	毛: 
\\	耒	
\\	すき・すきへん・らいすき・らいへん	
\\	形声 
\\	冶	ヤ	い.る	じ	
\\	冫: 
\\	台: 
\\	冫	
\\	にすい	
\\	形声 
\\	濫	ラン	みだ.りに、みだ.りがましい		
\\	氾濫(はんらん): 
\\	水: 
\\	監: 
\\	水	
\\	みず・したみず・さんずい	
\\	形声 
\\	吏	リ		さと, し	
\\	捕吏(ほり): 
\\	丈: 
\\	口: 
\\	口	
\\	くち・くちへん	
\\	会意 
\\	璃	リ		あき	
\\	玉: 
\\	亠: 
\\	凶: 
\\	禸: 
\\	玉	
\\	たま・たまへん ・ おう・おうへん・あうさま	
\\	形声 
\\	瞭	リョウ	あきらか		
\\	明瞭(めいりょう): 
\\	目: 
\\	大: 
\\	日: 
\\	小: 
\\	目	
\\	め・めへん・よこめ	
\\	形声 
\\	韓	カン	から、いげた		
\\	十: 
\\	早: 
\\	韋: 
\\	韋	
\\	なめしがわ	
\\	会意 
\\	岡	コウ	おか		
\\	冂: 
\\	山: 
\\	山	
\\	やま・やまへん・やまかんむり	
\\	会意 
\\	阪	ハン	さか		
\\	阜: 
\\	反: 
\\	阜	
\\	こざと・こざとへん・おか・ぎふのふ	
\\	形声 
\\	狙	ソ、ショ	ねら.う、ねら.い		
\\	狙い(ねらい): 
\\	狙う(ねらう): 
\\	犬: 
\\	且: 
\\	犬	
\\	いぬ・けものへん	
\\	形声 
\\	埼	キ	さき、さい、みさき		
\\	土: 
\\	奇: 
\\	土	
\\	つち・つちへん・どへん	
\\	形声 
\\	茨	シ、ジ	いばら、かや、くさぶき	あし, えばら, ばら, まつ, まん	
\\	艸: 
\\	次: 
\\	艸	
\\	くさ・くさかんむり	
\\	形声 
\\	曽	ソウ	かつ、かつて、すなわち		
\\	八: 
\\	田: 
\\	日: 
\\	曰	
\\	ひらび・いわく	曾	象形 
\\	栃		とち		
\\	木: 
\\	厂: 
\\	万: 
\\	木	
\\	き・きへん	
\\	国字 
\\	阜	フ、フウ		おか	
\\	𠂤: 
\\	十: 
\\	阜	
\\	こざと・こざとへん・おか・ぎふのふ	
\\	象形 
\\	葛	カツ、カチ	つづら、くず	か, かず, かずら, かっ, かつら	
\\	艸: 
\\	曷: 
\\	匂: 
\\	艸	
\\	くさ・くさかんむり	
\\	形声 
\\	畿	キ	みやこ		
\\	戌: 
\\	田	
\\	た・たへん	
\\	形声 
\\	尻	コウ	しり	がみ	
\\	尻(しり): 
\\	尻尾(しっぽ): 
\\	尸: 
\\	九: 
\\	尸	
\\	しかばね・かばね・かばねだれ	
\\	形声 
\\	柿	シ	かき		
\\	木: 
\\	市: 
\\	木	
\\	き・きへん	
\\	形声 
\\	腎	ジン			
\\	臣: 
\\	又: 
\\	肉: 
\\	肉	
\\	にく・にくづき	
\\	形声 
\\	釜	フ	かま		
\\	釜(かま): 
\\	父: 
\\	金: 
\\	金	
\\	かね・かねへん	
\\	形声 
\\	脇	キョウ	わき、わけ		
\\	脇(わき): 
\\	肉: 
\\	力: 
\\	力: 
\\	力: 
\\	肉	
\\	にく・にくづき	
\\	形声 
\\	鍋	カ	なべ		
\\	鍋(なべ): 
\\	金: 
\\	咼: 
\\	金	
\\	かね・かねへん	
\\	形声 
\\	瓦	ガ	かわら、ぐらむ		
\\	瓦(かわら): 
\\	瓦斯(ガス): 
\\	瓦(グラム): 
\\	煉瓦(れんが): 
\\	丶: 
\\	瓦	
\\	かわら	
\\	象形 
\\	挫	ザ、サ	くじ.く、くじ.ける		
\\	手: 
\\	坐: 
\\	手	
\\	て・てへん	
\\	形声 
\\	誰	スイ	だれ、たれ、た		
\\	誰(だれ): 
\\	誰か(だれか): 
\\	言: 
\\	隹: 
\\	言	
\\	こと・げん・ごんべん・ことば	
\\	形声 
\\	俺	エン	おれ、われ		
\\	俺(おれ): 
\\	人: 
\\	奄: 
\\	人	
\\	ひと・にんべん・ひとやね	
\\	形声 
\\	痕	コン	あと		
\\	疒: 
\\	艮: 
\\	疒	
\\	やまい・やまいだれ	
\\	形声 
\\	袖	シュウ	そで		
\\	袖(そで): 
\\	衣: 
\\	由: 
\\	衣	
\\	ころも・ころもへん	
\\	形声 
\\	妖	ヨウ	あや.しい、なま.めく、わざわ.い		
\\	女: 
\\	夭: 
\\	女	
\\	おんな・おんなへん	
\\	形声 
\\	闇	アン、オン	やみ、くら.い		
\\	闇(やみ): 
\\	無闇に(むやみに): 
\\	門: 
\\	音: 
\\	門	
\\	もん・もんがまえ・かどがまえ	
\\	形声 
\\	捉	ソク、サク	とら.える		
\\	手: 
\\	足: 
\\	手	
\\	て・てへん	
\\	形声 
\\	賭	ト	か.ける、かけ		
\\	賭け(かけ): 
\\	賭ける(かける): 
\\	貝: 
\\	者: 
\\	貝	
\\	かい・かいへん・こがい	
\\	形声 
\\	汎	ハン、ブ、フウ、ホウ、ホン	ただよ.う、ひろ.い	ひろ, ひろし, みな	
\\	水: 
\\	凡: 
\\	水	
\\	みず・したみず・さんずい	
\\	形声 
\\	頃	ケイ、キョウ	ころ、ごろ、しばら.く	ころも	
\\	頃(ころ): 
\\	この頃(このごろ): 
\\	近頃(ちかごろ): 
\\	年頃(としごろ): 
\\	一頃(ひところ): 
\\	その頃(そのころ): 
\\	匕: 
\\	頁: 
\\	頁	
\\	おおがい・いちのかい	
\\	会意 
\\	謎	メイ、ベイ	なぞ		
\\	謎(なぞ): 
\\	謎々(なぞなぞ): 
\\	言: 
\\	迷: 
\\	言	
\\	こと・げん・ごんべん・ことば	
\\	形声 
\\	爪	ソウ	つめ、つま-		
\\	爪(つめ): 
\\	丿: 
\\	爪	
\\	つめ・つめがしら・つめかんむり・そうにょう・のつ	
\\	象形 
\\	鍵	ケン	かぎ		
\\	鍵(かぎ): 
\\	金: 
\\	建: 
\\	金	
\\	かね・かねへん	
\\	形声 
\\	臼	キュウ、グ	うす、うすづ.く		
\\	臼: 
\\	臼	
\\	うす	
\\	象形 
\\	牙	ガ、ゲ	きば、は		
\\	亅: 
\\	丿: 
\\	牙	
\\	きば・きばへん	
\\	象形 
\\	枕	チン、シン	まくら		
\\	枕(まくら): 
\\	木: 
\\	冖: 
\\	八: 
\\	木	
\\	き・きへん	
\\	形声 
\\	丼	トン、タン、ショウ、セイ	どんぶり		
\\	丼(どんぶり): 
\\	井: 
\\	丶: 
\\	丶	
\\	てん・ちょぼ・ちゅ・ちゅてん	
\\	象形 
\\	惧	ク	おそ.れる		
\\	心: 
\\	具: 
\\	心	
\\	こころ・したごころ・りっしんべん	
\\	形声 
\\	膳	ゼン、セン	かしわ、すす.める、そな.える	ぜ, よし	
\\	膳(ぜん): 
\\	肉: 
\\	善: 
\\	肉	
\\	にく・にくづき	
\\	形声 
\\	呪	ジュ、シュ、シュウ、ズ	まじな.う、のろ.い、まじな.い、のろ.う		
\\	口: 
\\	兄: 
\\	口	
\\	くち・くちへん	
\\	会意 
\\	斬	ザン、サン、セン、ゼン	き.る		
\\	斬る(きる): 
\\	車: 
\\	斤: 
\\	斤	
\\	おの・おのづくり・きん	
\\	会意 
\\	怨	エン、オン、ウン	うら.む、うらみ、うら.めしい		
\\	夕: 
\\	卩: 
\\	心: 
\\	心	
\\	こころ・したごころ・りっしんべん	
\\	形声 
\\	串	カン、ケン、セン	くし、つらぬ.く		
\\	口: 
\\	中: 
\\	丨	
\\	ぼう・たてぼう	
\\	象形 
\\	腫	シュ、ショウ	は.れる、は.れ、は.らす、く.む、はれもの		
\\	腫れる(はれる): 
\\	肉: 
\\	重: 
\\	肉	
\\	にく・にくづき	
\\	形声 
\\	餅	ヘイ、ヒョウ	もち、もちい		
\\	餅(もち): 
\\	食: 
\\	并: 
\\	食	
\\	しょく・しょくへん	餠	形声 
\\	箸	チョ、チャク	はし		
\\	箸(はし): 
\\	竹: 
\\	者: 
\\	竹	
\\	たけ・たけかんむり	
\\	形声 
\\	斑	ハン	ふ、まだら	い	
\\	斑(むら): 
\\	斑(ぶち): 
\\	玉: 
\\	文: 
\\	玉: 
\\	文	
\\	ぶん・ぶんにょう	
\\	会意 
\\	摯	シ	いた.る		
\\	執: 
\\	手: 
\\	手	
\\	て・てへん	
\\	形声 
\\	堆	タイ、ツイ	うずたか.い		
\\	土: 
\\	隹: 
\\	土	
\\	つち・つちへん・どへん	
\\	形声 
\\	芯	シン			
\\	芯(しん): 
\\	艸: 
\\	心: 
\\	艸	
\\	くさ・くさかんむり	
\\	形声 
\\	蜜	ミツ、ビツ			
\\	蜜(みつ): 
\\	蜂蜜(はちみつ): 
\\	宀: 
\\	必: 
\\	虫: 
\\	虫	
\\	むし・むしへん	
\\	形声 
\\	股	コ	また、もも		
\\	股(また): 
\\	股(もも): 
\\	肉: 
\\	殳: 
\\	肉	
\\	にく・にくづき	
\\	形声 
\\	匂		にお.う、にお.い、にお.わせる	おり, こう, さぎ	
\\	匂い(におい): 
\\	匂う(におう): 
\\	勹: 
\\	匕: 
\\	勹	
\\	つつみがまえ・くがまえ	
\\	国字 
\\	玩	ガン	もちあそ.ぶ、もてあそ.ぶ		
\\	玩具(おもちゃ): 
\\	玉: 
\\	元: 
\\	玉	
\\	たま・たまへん ・ おう・おうへん・あうさま	
\\	形声 
\\	蜂	ホウ	はち	ほ	
\\	蜂蜜(はちみつ): 
\\	虫: 
\\	夂: 
\\	三: 
\\	丨: 
\\	虫	
\\	むし・むしへん	
\\	形声 
\\	椅	イ			
\\	椅子(いす): 
\\	木: 
\\	奇: 
\\	木	
\\	き・きへん	
\\	形声 
\\	叱	シツ、 シチ	しか.る		
\\	叱る(しかる): 
\\	叱(しっ): 
\\	口: 
\\	七: 
\\	形声 
\\	挨	アイ	ひら.く		
\\	挨拶(あいさつ): 
\\	手: 
\\	矣: 
\\	手	
\\	て・てへん	
\\	形声 
\\	餌	ジ、ニ	え、えば、えさ、もち		
\\	餌(えさ): 
\\	食: 
\\	耳: 
\\	食	
\\	しょく・しょくへん	
\\	形声 
\\	脊	セキ	せ、せい		
\\	人: 
\\	二: 
\\	二: 
\\	肉: 
\\	肉	
\\	にく・にくづき	
\\	会意 
\\	凄	セイ、サイ	さむ.い、すご.い、すさ.まじい		
\\	凄い(すごい): 
\\	物凄い(ものすごい): 
\\	冫: 
\\	妻: 
\\	冫	
\\	にすい	
\\	形声 
\\	嫉	シツ	そね.む、ねた.む、にく.む		
\\	嫉妬(しっと): 
\\	女: 
\\	疾: 
\\	女	
\\	おんな・おんなへん	
\\	形声 
\\	膝	シツ	ひざ		
\\	膝(ひざ): 
\\	肉: 
\\	木: 
\\	水: 
\\	肉	
\\	にく・にくづき	
\\	形声 
\\	恣	シ	ほしいまま		
\\	次: 
\\	心: 
\\	心	
\\	こころ・したごころ・りっしんべん	
\\	形声 
\\	麺	メン、ベン	むぎこ		
\\	麥: 
\\	面: 
\\	麥	
\\	むぎ・むぎへん・ぼく・ばくにょう	麵	形声 
\\	戴	タイ	いただ.く		
\\	戴く(いただく): 
\\	頂戴(ちょうだい): 
\\	裁: 
\\	戈	
\\	ほこ・ほこづくり・ほこがまえ・かのほこ	
\\	形声 
\\	裾	キョ、コ	すそ		
\\	裾(すそ): 
\\	衣: 
\\	居: 
\\	衣	
\\	ころも・ころもへん	
\\	形声 
\\	冥	メイ、ミョウ	くら.い		
\\	冖: 
\\	日: 
\\	六: 
\\	冖	
\\	わかんむり	
\\	象形 
\\	麓	ロク	ふもと		
\\	麓(ふもと): 
\\	林: 
\\	鹿: 
\\	鹿	
\\	しか・しかへん	
\\	形声 
\\	稽	ケイ	かんが.える、とど.める		
\\	稽古(けいこ): 
\\	滑稽(こっけい): 
\\	禾: 
\\	尤: 
\\	旨: 
\\	禾	
\\	のぎ・のぎへん	
\\	会意 
\\	蹴	シュク、シュウ	け.る		
\\	蹴る(ける): 
\\	蹴飛ばす(けとばす): 
\\	足: 
\\	就: 
\\	足	
\\	あし・あしへん	
\\	形声 
\\	訃	フ	しらせ		
\\	言: 
\\	卜: 
\\	言	
\\	こと・げん・ごんべん・ことば	
\\	形声 
\\	剥	ハク、 ホク	へ.ぐ, へず.る, む.く, む.ける, は.がれる, は.ぐ, は.げる, は.がす		
\\	剥く(むく): 
\\	剥げる(はげる): 
\\	剥ぐ(はぐ): 
\\	剥がす(はがす): 
\\	彑: 
\\	水: 
\\	刀: 
\\	会意 
\\	蓋	ガイ、カイ、コウ	ふた、けだ.し、おお.う、かさ、かこう		
\\	蓋(ふた): 
\\	艸: 
\\	盍: 
\\	艸	
\\	くさ・くさかんむり	
\\	形声 
\\	畏	イ	おそ.れる、かしこま.る、かしこ、かしこ.し		
\\	畏まりました(かしこまりました): 
\\	田: 
\\	一: 
\\	田	
\\	た・たへん	
\\	象形 
\\	喉	コウ	のど		
\\	喉(のど): 
\\	口: 
\\	侯: 
\\	口	
\\	くち・くちへん	
\\	形声 
\\	拭	ショク、シキ	ぬぐ.う、ふ.く		
\\	手拭い(てぬぐい): 
\\	拭く(ふく): 
\\	手: 
\\	式: 
\\	手	
\\	て・てへん	
\\	形声 
\\	頬	キョウ	ほお, ほほ		
\\	頬(ほほ): 
\\	頬っぺた(ほっぺた): 
\\	夾: 
\\	頁: 
\\	形声 
\\	貼	テン、チョウ	は.る、つ.く		
\\	貼る(はる): 
\\	貝: 
\\	占: 
\\	貝	
\\	かい・かいへん・こがい	
\\	形声 
\\	諦	テイ、タイ	あきら.める、つまびらか、まこと		
\\	諦める(あきらめる): 
\\	諦め(あきらめ): 
\\	言: 
\\	帝: 
\\	言	
\\	こと・げん・ごんべん・ことば	
\\	形声 
\\	煎	セン	せん.じる、い.る、に.る	いり	
\\	煎る(いる): 
\\	前: 
\\	火: 
\\	火	
\\	ひ・ひへん・れっか・れんが	
\\	形声 
\\	緻	チ	こまか.い		
\\	糸: 
\\	致: 
\\	糸	
\\	いと・いとへん	
\\	形声 
\\	哺	ホ	はぐく.む、ふく.む		
\\	口: 
\\	甫: 
\\	口	
\\	くち・くちへん	
\\	形声 
\\	罵	バ	ののし.る		
\\	罵る(ののしる): 
\\	网: 
\\	馬: 
\\	网	
\\	あみ・あみがしら・あみめ・よこめ・よんかしら	
\\	形声 
\\	乞	コツ、キツ、キ、キケ、コチ	こ.う		
\\	乙: 
\\	乙	
\\	おつ・おつにょう・つりばり	
\\	象形 
\\	嗅	キュウ	か.ぐ		
\\	嗅ぐ(かぐ): 
\\	口: 
\\	臭: 
\\	口	
\\	くち・くちへん	
\\	会意 
\\	蔑	ベツ	ないがしろ、なみ.する、くらい、さげす.む		
\\	軽蔑(けいべつ): 
\\	艸: 
\\	网: 
\\	戍: 
\\	艸	
\\	くさ・くさかんむり	
\\	会意 
\\	宛	エン	あ.てる、-あて、-づつ、あたか.も		
\\	宛(あて): 
\\	宛名(あてな): 
\\	宛てる(あてる): 
\\	宀: 
\\	夕: 
\\	卩: 
\\	宀	
\\	うかんむり	
\\	会意 
\\	萎	イ	な、しお.れる、しな.びる、しぼ.む、な.える		
\\	萎びる(しなびる): 
\\	艸: 
\\	委: 
\\	艸	
\\	くさ・くさかんむり	
\\	形声 
\\	咽	イン、エン、エツ	むせ.ぶ、むせ.る、のど、の.む		
\\	口: 
\\	因: 
\\	口	
\\	くち・くちへん	
\\	形声 
\\	淫	イン	ひた.す、ほしいまま、みだ.ら、みだ.れる、みだり		
\\	水: 
\\	爪: 
\\	壬: 
\\	水	
\\	みず・したみず・さんずい	
\\	形声 
\\	臆	オク、ヨク	むね、おくする		
\\	臆病(おくびょう): 
\\	肉: 
\\	意: 
\\	肉	
\\	にく・にくづき	
\\	形声 
\\	苛	カ	いじ.める、さいな.む、いらだ.つ、からい、こまかい		
\\	苛める(いじめる): 
\\	苛々(いらいら): 
\\	艸: 
\\	可: 
\\	艸	
\\	くさ・くさかんむり	
\\	形声 
\\	崖	ガイ、ゲ、ギ	がけ、きし、はて		
\\	崖(がけ): 
\\	山: 
\\	厂: 
\\	圭: 
\\	山	
\\	やま・やまへん・やまかんむり	
\\	形声 
\\	骸	ガイ、カイ	むくろ		
\\	骨: 
\\	亥: 
\\	骨	
\\	ほね・ほねへん	
\\	形声 
\\	顎	ガク	あご、あぎと		
\\	顎(あご): 
\\	咢: 
\\	頁: 
\\	頁	
\\	おおがい・いちのかい	
\\	形声 
\\	僅	キン、ゴン	わず.か		
\\	僅か(わずか): 
\\	人: 
\\	艸: 
\\	三: 
\\	人	
\\	ひと・にんべん・ひとやね	
\\	形声 
\\	巾	キン、フク	おお.い、ちきり、きれ		
\\	雑巾(ぞうきん): 
\\	布巾(ふきん): 
\\	巾: 
\\	巾	
\\	はば・はばへん・きんべん	
\\	象形 
\\	窟	クツ、コツ	いわや、いはや、あな		
\\	穴: 
\\	屈: 
\\	穴	
\\	あな・あなかんむり	
\\	形声 
\\	詣	ケイ、ゲイ	けい.する、まい.る、いた.る、もう.でる		
\\	言: 
\\	旨: 
\\	言	
\\	こと・げん・ごんべん・ことば	
\\	形声 
\\	隙	ゲキ、キャク、ケキ	すき、す.く、す.かす、ひま		
\\	隙(すき): 
\\	隙間(すきま): 
\\	阜: 
\\	小: 
\\	日: 
\\	小: 
\\	阜	
\\	こざと・こざとへん・おか・ぎふのふ	
\\	会意 
\\	桁	コウ	けた		
\\	桁(けた): 
\\	木: 
\\	行: 
\\	木	
\\	き・きへん	
\\	形声 
\\	舷	ゲン	ふなばた		
\\	舟: 
\\	玄: 
\\	舟	
\\	ふね・ふねへん	
\\	形声 
\\	勾	コウ、ク	かぎ、ま.がる	まがり	
\\	勹: 
\\	厶: 
\\	勹	
\\	つつみがまえ・くがまえ	
\\	会意 
\\	梗	コウ、キョウ	ふさぐ、やまにれ、おおむね		
\\	木: 
\\	更: 
\\	木	
\\	き・きへん	
\\	形声 
\\	塞	ソク、サイ	ふさ.ぐ、とりで、み.ちる		
\\	塞がる(ふさがる): 
\\	塞ぐ(ふさぐ): 
\\	宀: 
\\	三: 
\\	八: 
\\	土: 
\\	土	
\\	つち・つちへん・どへん	
\\	会意 
\\	柵	サク、サン	しがら.む、しがらみ、とりで、やらい	ませ, やな	
\\	柵(さく): 
\\	木: 
\\	冊: 
\\	木	
\\	き・きへん	
\\	形声 
\\	拶	サツ	せま.る		
\\	挨拶(あいさつ): 
\\	手: 
\\	巛: 
\\	夕: 
\\	手	
\\	て・てへん	
\\	会意 
\\	醒	セイ	さ.ます、さ.める		
\\	酉: 
\\	星: 
\\	酉	
\\	とり・とりへん・ひよみのとり・さけのとり	
\\	形声 
\\	戚	ソク、セキ	いた.む、うれ.える、みうち		
\\	親戚(しんせき): 
\\	厂: 
\\	戈: 
\\	上: 
\\	小: 
\\	戈	
\\	ほこ・ほこづくり・ほこがまえ・かのほこ	
\\	形声 
\\	羨	セン、エン	うらや.む、あまり		
\\	羨ましい(うらやましい): 
\\	羨む(うらやむ): 
\\	羊: 
\\	水: 
\\	欠: 
\\	羊	
\\	ひつじ・ひつじへん	
\\	形声 
\\	腺	セン			
\\	肉: 
\\	泉: 
\\	肉	
\\	にく・にくづき	
\\	形声 
\\	詮	セン	せん.ずる、かい、あき.らか	あき	
\\	言: 
\\	全: 
\\	言	
\\	こと・げん・ごんべん・ことば	
\\	形声 
\\	遡	ソ、サク	さかのぼ.る		
\\	遡る(さかのぼる): 
\\	朔: 
\\	辶: 
\\	辵	
\\	しんにょう・しんにゅう	
\\	形声 
\\	痩	ソウ、チュウ、シュウ、シュ	や.せる		
\\	痩せる(やせる): 
\\	疒: 
\\	叟: 
\\	疒	
\\	やまい・やまいだれ	瘦	形声 
\\	遜	ソン	したが.う、へりくだ.る、ゆず.る		
\\	謙遜(けんそん): 
\\	孫: 
\\	辶: 
\\	辵	
\\	しんにょう・しんにゅう	
\\	形声 
\\	唾	ダ、タ	つば、つばき		
\\	唾(つば): 
\\	口: 
\\	垂: 
\\	口	
\\	くち・くちへん	
\\	形声 
\\	綻	タン	ほころ.びる		
\\	綻びる(ほころびる): 
\\	糸: 
\\	定: 
\\	糸	
\\	いと・いとへん	
\\	形声 
\\	酎	チュウ、チュ	かも.す		
\\	酉: 
\\	寸: 
\\	酉	
\\	とり・とりへん・ひよみのとり・さけのとり	
\\	形声 
\\	捗	チョク、ホ	はかど.る		
\\	捗る(はかどる): 
\\	手: 
\\	歩: 
\\	手	
\\	て・てへん	
\\	形声 
\\	潰	カイ、エ	つぶ.す、つぶ.れる、つい.える		
\\	潰す(つぶす): 
\\	潰れる(つぶれる): 
\\	水: 
\\	貴: 
\\	水	
\\	みず・したみず・さんずい	
\\	形声 
\\	溺	デキ、ジョウ、ニョウ	いばり、おぼ.れる		
\\	溺れる(おぼれる): 
\\	水: 
\\	弱: 
\\	水	
\\	みず・したみず・さんずい	
\\	形声 
\\	填	テン、 チン	は.まる, は.める, うず.める, しず.める, ふさ.ぐ	まき	
\\	土: 
\\	真: 
\\	形声 
\\	妬	ト、ツ	ねた.む、そね.む、つも.る、ふさ.ぐ		
\\	嫉妬(しっと): 
\\	妬む(ねたむ): 
\\	女: 
\\	石: 
\\	女	
\\	おんな・おんなへん	
\\	形声 
\\	頓	トン、トツ	にわか.に、とん.と、つまず.く、とみ.に、ぬかずく		
\\	屯: 
\\	頁: 
\\	頁	
\\	おおがい・いちのかい	
\\	形声 
\\	捻	ネン、ジョウ	ね.じる、ねじ.る、ひね.くる、ひね.る		
\\	捻る(ひねる): 
\\	捻子(ねじ): 
\\	捻じれる(ねじれる): 
\\	手: 
\\	念: 
\\	手	
\\	て・てへん	
\\	形声 
\\	氾	ハン	ひろ.がる		
\\	氾濫(はんらん): 
\\	水: 
\\	卩: 
\\	水	
\\	みず・したみず・さんずい	
\\	形声 
\\	肘	チュウ	ひじ		
\\	肘(ひじ): 
\\	肉: 
\\	寸: 
\\	肉	
\\	にく・にくづき	
\\	形声 
\\	蔽	ヘイ、ヘツ、フツ	おお.う、おお.い		
\\	艸: 
\\	敝: 
\\	艸	
\\	くさ・くさかんむり	
\\	形声 
\\	貌	ボウ、バク	かたち、かたどる		
\\	豸: 
\\	皃: 
\\	豸	
\\	むじな・むじなへん	
\\	象形 
\\	勃	ボツ、ホツ	おこ.る、にわかに		
\\	孛: 
\\	力: 
\\	力	
\\	ちから・りきづくり	
\\	形声 
\\	昧	マイ、バイ	くら.い、むさぼ.る		
\\	曖昧(あいまい): 
\\	日: 
\\	未: 
\\	日	
\\	ひ・ひへん・にち・にちへん	
\\	形声 
\\	沃	ヨウ、ヨク、オク	そそ.ぐ		
\\	水: 
\\	夭: 
\\	水	
\\	みず・したみず・さんずい	
\\	形声 
\\	侶	リョ、ロ	とも		
\\	人: 
\\	呂: 
\\	人	
\\	ひと・にんべん・ひとやね	
\\	形声 
\\	賂	ロ	まいな.い、まいな.う		
\\	貝: 
\\	各: 
\\	貝	
\\	かい・かいへん・こがい	
\\	形声 
\\	弄	ロウ、ル	いじく.る、ろう.する、いじ.る、ひねく.る、たわむ.れる、もてあそ.ぶ		
\\	弄る(いじる): 
\\	玉: 
\\	廾: 
\\	廾	
\\	にじゅう・にじゅうあし・こまぬき	
\\	会意 
\\	傲	ゴウ	おご.る、あなど.る		
\\	人: 
\\	敖: 
\\	人	
\\	ひと・にんべん・ひとやね	
\\	形声 
\\	刹	セチ、セツ、サツ			
\\	乂: 
\\	木: 
\\	刀: 
\\	刀	
\\	かたな・りっとう	
\\	会意 
\\	喩	ユ	たと.える、さと.す		
\\	口: 
\\	兪: 
\\	口	
\\	くち・くちへん	
\\	形声 
\\	嘲	チョウ、トウ	あざけ.る		
\\	口: 
\\	朝: 
\\	口	
\\	くち・くちへん	
\\	形声 
\\	毀	キ	こぼ.つ、こわ.す、こぼ.れる、こわ.れる、そし.る、やぶ.る		
\\	臼: 
\\	土: 
\\	殳: 
\\	殳	
\\	るまた・ほこづくり	
\\	会意 
\\	彙	イ	はりねずみ		
\\	語彙(ごい): 
\\	彑: 
\\	冖: 
\\	果: 
\\	彐	
\\	けい・けいがしら・いのこがしら	
\\	象形 
\\	慄	リツ	ふる.える、おそ.れる、おのの.く		
\\	心: 
\\	栗: 
\\	心	
\\	こころ・したごころ・りっしんべん	
\\	形声 
\\	憬	ケイ	あこが.れる		
\\	心: 
\\	景: 
\\	心	
\\	こころ・したごころ・りっしんべん	
\\	形声 
\\	拉	ラツ、ラ、ロウ	らっ.する、ひし.ぐ、くだ.く		
\\	手: 
\\	立: 
\\	手	
\\	て・てへん	
\\	形声 
\\	曖	アイ	くら.い		
\\	曖昧(あいまい): 
\\	日: 
\\	愛: 
\\	日	
\\	ひ・ひへん・にち・にちへん	
\\	形声 
\\	楷	カイ			
\\	木: 
\\	皆: 
\\	木	
\\	き・きへん	
\\	形声 
\\	鬱	ウツ	うっ.する、ふさ.ぐ、しげ.る		
\\	鬱陶しい(うっとうしい): 
\\	憂鬱(ゆううつ): 
\\	缶: 
\\	木: 
\\	木: 
\\	冖: 
\\	鬯: 
\\	彡: 
\\	鬯	
\\	ちょう・においざけ・かおりぐさ	
\\	会意 
\\	璧	ヘキ	たま		
\\	完璧(かんぺき): 
\\	辟: 
\\	玉: 
\\	玉	
\\	たま・たまへん ・ おう・おうへん・あうさま	
\\	形声 
\\	瘍	ヨウ	かさ		
\\	疒: 
\\	昜: 
\\	疒	
\\	やまい・やまいだれ	
\\	形声 
\\	箋	セン	ふだ		
\\	便箋(びんせん): 
\\	竹: 
\\	戔: 
\\	竹	
\\	たけ・たけかんむり	
\\	形声 
\\	籠	ロウ、ル	かご、こ.める、こも.る、こ.む	ごめ, もり	
\\	籠(かご): 
\\	籠もる(こもる): 
\\	竹: 
\\	龍: 
\\	竹	
\\	たけ・たけかんむり	
\\	形声 
\\	羞	シュウ	はじ.る、すすめ.る、は.ずかしい		
\\	羊: 
\\	丿: 
\\	丑: 
\\	羊	
\\	ひつじ・ひつじへん	
\\	会意 
\\	諧	カイ	かな.う、やわ.らぐ		
\\	言: 
\\	皆: 
\\	言	
\\	こと・げん・ごんべん・ことば	
\\	形声 
\\	貪	タン、トン	むさぼ.る		
\\	今: 
\\	貝: 
\\	貝	
\\	かい・かいへん・こがい	
\\	会意 
\\	踪	ソウ、ショウ	あと		
\\	足: 
\\	宗: 
\\	足	
\\	あし・あしへん	
\\	形声 
\\	辣	ラツ	から.い		
\\	辛: 
\\	束: 
\\	辛	
\\	しん・からい	
\\	形声 
\\	錮	コ	ふさ.ぐ		
\\	金: 
\\	固: 
\\	金	
\\	かね・かねへん	
\\	形声 
\\	丑		うし	ひろ		
\\	一: 
\\	丞	ジョウ	すく.う, たす.ける	すけ	
\\	了: 
\\	水: 
\\	一: 
\\	乃	ナイ	の, すなわ.ち, なんじ	おさむ, お, のり	
\\	乃至(ないし): 
\\	丿: 
\\	之	シ	の, これ, おいて, ゆく, この	ゆき, いたる, あき, つな, ゆみ	
\\	亠: 
\\	乎		か, ああ, かな, や, よ, を				
\\	凝乎と(じっと): 
\\	丿: 
\\	亅: 
\\	乘		の.る, の.せる				
\\	禾: 
\\	木: 
\\	北: 
\\	也	ヤ	なり, か, また	し	
\\	可也(かなり): 
\\	丨: 
\\	乙: 
\\	云	ウン	い.う, ここに		
\\	云々(うんぬん): 
\\	二: 
\\	厶: 
\\	亘		わた.る, もと.める	せん, のぶ, とうる, わたる, ひさし		
\\	二: 
\\	旦: 
\\	亙	コウ	わた.る, もと.める	のぶ	
\\	互: 
\\	彑: 
\\	丶: 
\\	丶: 
\\	些	サ	ち.と, ち.っと, いささか		
\\	些とも(ちっとも): 
\\	些事(さじ): 
\\	此: 
\\	二: 
\\	亞		つ.ぐ				
\\	二: 
\\	亥		い	り		
\\	亠: 
\\	人: 
\\	亦		また				
\\	亠: 
\\	丿: 
\\	亅: 
\\	亨	コウ	とお.る	とうる, みち, ゆき, あきら, なが, あき, とおる, とおるふ	
\\	亠: 
\\	口: 
\\	了: 
\\	亮	リョウ	あきらか	あき, すけ, まこと, あきら, よし, きょう, たすく, ふさ	
\\	亠: 
\\	口: 
\\	冖: 
\\	八: 
\\	仔	シ	こ, た.える		
\\	人: 
\\	子: 
\\	伊	イ	かれ	これ, ただ, よし, いと, だ	
\\	伊達(だて): 
\\	人: 
\\	尹: 
\\	伍	ゴ	いつつ	くみ, あつむ	
\\	伍(ご): 
\\	人: 
\\	五: 
\\	伶		わざおぎ	れ		
\\	人: 
\\	令: 
\\	伽	カ	とぎ		
\\	人: 
\\	加: 
\\	佃	テン	つくだ		
\\	人: 
\\	田: 
\\	佑	ユウ	たす.ける	すけ, たすく, ゆ	
\\	人: 
\\	右: 
\\	佛	ブツ	ほとけ	さらぎ	
\\	人: 
\\	弗: 
\\	侃	カン	つよ.い	あきら, ただし	
\\	人: 
\\	口: 
\\	巛: 
\\	來	ライ	く.る, きた.る, きた.す, き.たす, き.たる		
\\	木: 
\\	从: 
\\	侑	ユウ	すす.める, たす.ける	あつむ, すすむ, ゆき, ゆ	
\\	人: 
\\	有: 
\\	俄	ガ	にわか		
\\	俄(にわか): 
\\	人: 
\\	我: 
\\	俐		かしこ.い				
\\	人: 
\\	利: 
\\	俠		おとこだて				
\\	俣	また	また	ばた	
\\	人: 
\\	口: 
\\	天: 
\\	俱		ともに				
\\	倖	コウ	しあわ.せ, さいわ.い		
\\	人: 
\\	幸: 
\\	倦		あき.る, あぐ.む, あぐ.ねる, う.む, つか.れる			
\\	人: 
\\	巻: 
\\	倭		したが.う	まさ, やす, やまと		
\\	人: 
\\	委: 
\\	偲	サイ	しの.ぶ		
\\	人: 
\\	思: 
\\	傭	ヨウ	やと.う, あた.い, ひと.しい		
\\	人: 
\\	庸: 
\\	傳	テン	つた.わる, つた.える, つた.う, つて				
\\	人: 
\\	專: 
\\	僞		いつわ.る, いつわ.り, にせ				
\\	人: 
\\	爲: 
\\	價		あたい				
\\	人: 
\\	賈: 
\\	儉		つま.しい, つづまやか				
\\	人: 
\\	僉: 
\\	儲	チョ	もう.ける, もう.かる, もうけ, たくわ.える		
\\	儲かる(もうかる): 
\\	儲ける(もうける): 
\\	人: 
\\	諸: 
\\	允	イン	じょう, まこと.に, ゆるす	まこと, のぶ, まさ, みつ, すけ, よし, ちか, とも	
\\	厶: 
\\	儿: 
\\	兎	ト	うさぎ		
\\	兎(うさぎ): 
\\	兎に角(とにかく): 
\\	兎も角(ともかく): 
\\	兎角(とかく): 
\\	丿: 
\\	儿: 
\\	丶: 
\\	兒	ジ	こ	ちご	
\\	臼: 
\\	儿: 
\\	兜	トウ	かぶと		
\\	白: 
\\	儿: 
\\	其	キ	それ, その		
\\	其方(そちら): 
\\	其れでは(それでは): 
\\	其処(そこ): 
\\	其の(その): 
\\	其れ(それ): 
\\	其れから(それから): 
\\	其れで(それで): 
\\	其れ共(それとも): 
\\	其れに(それに): 
\\	其れ程(それほど): 
\\	其の儘(そのまま): 
\\	其れでも(それでも): 
\\	其れなら(それなら): 
\\	其処で(そこで): 
\\	其処ら(そこら): 
\\	其れ故(それゆえ): 
\\	其れと(それと): 
\\	甘: 
\\	八: 
\\	冨	フ	と.む, とみ	と	
\\	冖: 
\\	一: 
\\	口: 
\\	田: 
\\	冴	ゴ	さ.える, こお.る, ひ.える	さえ, さえる	
\\	冴える(さえる): 
\\	冫: 
\\	牙: 
\\	凉	リョウ	すず.しい, すず.む, すず.やか, うす.い, ひや.す, まことに		
\\	冫: 
\\	京: 
\\	凌	リョウ	しの.ぐ		
\\	凌ぐ(しのぐ): 
\\	冫: 
\\	土: 
\\	儿: 
\\	夂: 
\\	凛	リン	きびし.い		
\\	冫: 
\\	禀: 
\\	凜		きびし.い				
\\	冫: 
\\	稟: 
\\	凧	いかのぼり	いかのぼり, たこ		
\\	几: 
\\	巾: 
\\	凪		なぎ, な.ぐ			
\\	几: 
\\	止: 
\\	凰	オオ	おおとり		
\\	几: 
\\	皇: 
\\	凱	ガイ	かちどき, やわらぐ		
\\	豈: 
\\	几: 
\\	函	カン	はこ, い.れる		
\\	一: 
\\	水: 
\\	凵: 
\\	剩		あまつさえ, あま.り, あま.る				
\\	乘: 
\\	禾: 
\\	北: 
\\	刀: 
\\	劉	リュウ	ころ.す	らう, のぶ, みずち	
\\	卯: 
\\	金: 
\\	刀: 
\\	劍		つるぎ				
\\	僉: 
\\	刀: 
\\	劫	コウ	おびや.かす		
\\	劫(こう): 
\\	去: 
\\	力: 
\\	勁	ケイ	つよ.い		
\\	一: 
\\	巛: 
\\	工: 
\\	力: 
\\	勳		いさお	いさ			
\\	熏: 
\\	力: 
\\	勺					
\\	勹: 
\\	丶: 
\\	勿	モチ	なか.れ, なし		
\\	勿論(もちろん): 
\\	勿体ない(もったいない): 
\\	勹: 
\\	丿: 
\\	丿: 
\\	匁		もんめ, め			
\\	勹: 
\\	乂: 
\\	匡	キョウ	すく.う	ただ, ただし, ただす, まさ, まさし, きよ, ひと, やす	
\\	匚: 
\\	玉: 
\\	卜	ボク	うらな.う, うらない	うら	
\\	卜: 
\\	卯	ボウ	う	あきら, しげる, い	
\\	丿: 
\\	卩: 
\\	卷		ま.く, まき				
\\	二: 
\\	人: 
\\	卩: 
\\	卽		つく, すなわち, もし				
\\	卿	ケイ	きみ		
\\	丿: 
\\	即: 
\\	厨	シュウ	くりや		
\\	厂: 
\\	豆: 
\\	寸: 
\\	厩	キュウ	うまや	まや, まら	
\\	厂: 
\\	既: 
\\	叉	サ	また		
\\	又: 
\\	丶: 
\\	叡	エイ	あき.らか		
\\	睿: 
\\	又: 
\\	叢		くさむら, むら.がる, むら			
\\	丶: 
\\	羊: 
\\	取: 
\\	只	シ	ただ		
\\	只(ただ): 
\\	只管(ひたすら): 
\\	口: 
\\	八: 
\\	叶	キョウ	かな.える, かな.う	かの, かのう	
\\	叶う(かなう): 
\\	叶える(かなえる): 
\\	口: 
\\	十: 
\\	吞		のむ				
\\	吻		くちわき, くちさき			
\\	口: 
\\	勿: 
\\	吾	ゴ	われ, わが-, あ-	あ, あが	
\\	五: 
\\	口: 
\\	哉	サイ	かな, や	か, すけ, とし, ちか, はじめ	
\\	土: 
\\	戈: 
\\	口: 
\\	哨	ショウ	みはり		
\\	口: 
\\	肖: 
\\	哩		まいる			
\\	口: 
\\	里: 
\\	啄	タク	ついば.む, つつ.く	くち, はし, ばし	
\\	口: 
\\	豕: 
\\	喋		しゃべ.る, ついば.む			
\\	お喋り(おしゃべり): 
\\	喋る(しゃべる): 
\\	口: 
\\	世: 
\\	木: 
\\	喧	ケン	やかま.しい, かまびす.しい		
\\	喧嘩(けんか): 
\\	喧しい(やかましい): 
\\	口: 
\\	宣: 
\\	喬	キョウ	たか.い	たかし, たか	
\\	呑: 
\\	冂: 
\\	口: 
\\	單		ひとえ				
\\	口: 
\\	口: 
\\	甲: 
\\	一: 
\\	喰	く.う	く.う, く.らう	じき	
\\	口: 
\\	食: 
\\	嘉	カ	よみ.する, よい	ひろ, よし, よしみ, かず, よ	
\\	吉: 
\\	加: 
\\	嘗	ショウ	かつ.て, こころ.みる, な.める		
\\	嘗て(かつて): 
\\	尚: 
\\	旨: 
\\	嘩	カ	かまびす.しい		
\\	喧嘩(けんか): 
\\	口: 
\\	華: 
\\	噂	ソン	うわさ		
\\	噂(うわさ): 
\\	口: 
\\	尊: 
\\	噌	ソウ	かまびす.しい		
\\	味噌(みそ): 
\\	口: 
\\	曽: 
\\	嚴		おごそ.か, きび.しい, いか.めしい, いつくし				
\\	口: 
\\	口: 
\\	厂: 
\\	敢: 
\\	圃		はたけ, にわ			
\\	囗: 
\\	甫: 
\\	圈		かこ.い				
\\	囗: 
\\	卷: 
\\	國	コク	くに	こ	
\\	囗: 
\\	或: 
\\	圓	エン	まる.い, まる, まど, まど.か, まろ.やか	つぶら	
\\	囗: 
\\	員: 
\\	團	ダン	かたまり, まる.い		
\\	囗: 
\\	專: 
\\	圭	ケイ		か, きよ, たま, よし, かど, きよし	
\\	土: 
\\	土: 
\\	坐	ザ	すわ.る, おわす, そぞろに, まします		
\\	从: 
\\	土: 
\\	坦	タン	たいら	やす	
\\	土: 
\\	旦: 
\\	埜	ヤ	の			
\\	林: 
\\	土: 
\\	埴	ショク	はに, へな		
\\	土: 
\\	直: 
\\	堯	ギョウ	たか.い		
\\	土: 
\\	土: 
\\	土: 
\\	兀: 
\\	堰	エン	せき, せ.く		
\\	土: 
\\	匚: 
\\	日: 
\\	女: 
\\	堵	ト	かき		
\\	土: 
\\	者: 
\\	堺	カイ	さかい		
\\	土: 
\\	界: 
\\	塙	カク	はなわ, かた.い	はな	
\\	土: 
\\	高: 
\\	增		ます, ふえる, ふやす				
\\	壕	コウ	ほり		
\\	土: 
\\	豪: 
\\	壘		とりで				
\\	田: 
\\	田: 
\\	田: 
\\	土: 
\\	壞		こわ.す, こわ.れる, やぶ.る				
\\	土: 
\\	衣: 
\\	网: 
\\	二: 
\\	二: 
\\	壬		みずのえ				
\\	玉: 
\\	壯	ソウ	さかん	たけし			
\\	爿: 
\\	士: 
\\	壽	ジュ	ことぶき, ことぶ.く, ことほ.ぐ	かず, じ, とし, ひさ, ひさし	
\\	土: 
\\	工: 
\\	一: 
\\	口: 
\\	寸: 
\\	夷	イ	えびす, えみし, ころ.す, たい.らげる	し	
\\	大: 
\\	弓: 
\\	奄	エン	おお.う, たちまち	あま	
\\	大: 
\\	日: 
\\	奎	ケイ			
\\	大: 
\\	圭: 
\\	套	トウ	かさ.ねる		
\\	大: 
\\	長: 
\\	奧		おく.まる, くま				
\\	冂: 
\\	釆: 
\\	大: 
\\	奬		すす.める				
\\	將: 
\\	大: 
\\	姥	ボ	うば		
\\	女: 
\\	老: 
\\	姪	テツ	めい, おい		
\\	姪(めい): 
\\	女: 
\\	至: 
\\	娃		うつく.しい	い		
\\	女: 
\\	圭: 
\\	娩	ベン			
\\	女: 
\\	免: 
\\	嬉	キ	うれ.しい, たの.しむ	うらし, うれし	
\\	嬉しい(うれしい): 
\\	女: 
\\	喜: 
\\	孃		むすめ				
\\	女: 
\\	襄: 
\\	孜		つと.める			
\\	子: 
\\	攴: 
\\	孟	モウ	かしら	たけし, たけ, はる, はじめ, つとむ, おさ, はい	
\\	子: 
\\	皿: 
\\	宋	ソウ			
\\	宀: 
\\	木: 
\\	宏	コウ	ひろ.い	あつ, ひろ, ひろし	
\\	宀: 
\\	厶: 
\\	宕	トウ	すぎる		
\\	宀: 
\\	石: 
\\	宥	ユウ	なだ.める, ゆる.す		
\\	宀: 
\\	有: 
\\	寅	イン	とら	とも, のぶ	
\\	宀: 
\\	由: 
\\	八: 
\\	寓	グウ	ぐう.する, かこつ.ける, よ.せる, よ.る, かりずまい		
\\	宀: 
\\	禺: 
\\	寢		ね.る, ね.かす, い.ぬ, みたまや, や.める				
\\	宀: 
\\	爿: 
\\	彑: 
\\	冖: 
\\	又: 
\\	實	ジツ	み, みの.る, まこと.に, みの.り, みち.る	さな, さね, みつ	
\\	宀: 
\\	貫: 
\\	寬		ひろい, ゆるやか, くつろぐ				
\\	寵		めぐ.み, めぐ.む			
\\	宀: 
\\	龍: 
\\	將		まさ.に, はた, ひきい.る, もって	まさる			
\\	爿: 
\\	月: 
\\	寸: 
\\	專		もっぱ.ら				
\\	十: 
\\	寸: 
\\	尖	セン	とが.る, さき, するど.い		
\\	尖る(とがる): 
\\	小: 
\\	大: 
\\	尤		もっと.も, とが.める			
\\	尤も(もっとも): 
\\	尢: 
\\	丶: 
\\	尭		たか.い	たかし, たか, のり, あき		
\\	卉: 
\\	兀: 
\\	八: 
\\	屑		くず, いさぎよ.い			
\\	屑(くず): 
\\	紙屑(かみくず): 
\\	尸: 
\\	肖: 
\\	峨	ガ	けわ.しい		
\\	山: 
\\	我: 
\\	峯	ホウ	みね, ね		
\\	山: 
\\	夂: 
\\	三: 
\\	丨: 
\\	峻	シュン	けわ.しい, たか.い	たか, たかし, ちか, とし, おか	
\\	山: 
\\	允: 
\\	夂: 
\\	峽		はざま	かい			
\\	山: 
\\	夾: 
\\	崚					
\\	山: 
\\	土: 
\\	儿: 
\\	夂: 
\\	嵩	スウ	かさ, かさ.む, たか.い	たかし, たか, たけ	
\\	嵩む(かさむ): 
\\	嵩張る(かさばる): 
\\	山: 
\\	高: 
\\	嵯	サ			
\\	山: 
\\	差: 
\\	嶋	トウ	しま		
\\	山: 
\\	鳥: 
\\	嶺	レイ		みね, ね	
\\	山: 
\\	領: 
\\	巌	ガン	いわ, いわお, けわ.しい	よし	
\\	山: 
\\	厳: 
\\	巖	ガン	いわ, いわお, けわ.しい	よし	
\\	山: 
\\	嚴: 
\\	巢						
\\	已	イ	や.む, すで.に, のみ, はなはだ	み	
\\	己: 
\\	巳	シ	み		
\\	己: 
\\	巴	ハ	ともえ, うずまき	とも	
\\	欧羅巴(ヨーロッパ): 
\\	巳: 
\\	巷	コウ	ちまた		
\\	共: 
\\	己: 
\\	巽	ソン	たつみ		
\\	己: 
\\	己: 
\\	共: 
\\	帖	チョウ	かきもの		
\\	巾: 
\\	占: 
\\	帶		お.びる, おび				
\\	廿: 
\\	冖: 
\\	巾: 
\\	幌	コウ	ほろ, とばり		
\\	巾: 
\\	晃: 
\\	幡	マン	はた	は, わた	
\\	巾: 
\\	番: 
\\	庄	ショウ		まさ	
\\	广: 
\\	土: 
\\	庇	ヒ	ひさし, おお.う, かば.う		
\\	庇う(かばう): 
\\	广: 
\\	比: 
\\	庚	コウ	かのえ		
\\	广: 
\\	彑: 
\\	人: 
\\	庵	アン	いおり, いお	あ, い, いほり	
\\	广: 
\\	奄: 
\\	廟	ビョウ	たまや, みたまや, やしろ		
\\	广: 
\\	朝: 
\\	廣	コウ	ひろ.い, ひろ.まる, ひろ.める, ひろ.がる, ひろ.げる		
\\	广: 
\\	黄: 
\\	廳		やくしょ				
\\	广: 
\\	聽: 
\\	廻	カイ	まわ.る, まわ.す, もとお.る, めぐ.る, めぐ.らす	ざこ, じゃく	
\\	回: 
\\	廴: 
\\	廿	ジュウ	にじゅう	はつ	
\\	廾: 
\\	弘	コウ	ひろ.い	ひろ, ひろし, ひろむ, みつ, お, こお	
\\	弓: 
\\	厶: 
\\	弛	チ	たる.む, たる.める, たゆ.む, ゆる.む, ゆる.み		
\\	弛む(たるむ): 
\\	弛み(たるみ): 
\\	弓: 
\\	也: 
\\	彈		ひ.く, はず.む, たま, はじ.く, はじ.ける, ただ.す				
\\	弓: 
\\	單: 
\\	彌	ミ	いや, や, あまねし, いよいよ, とおい, ひさし, ひさ.しい, わた.る	ゆ			
\\	弓: 
\\	爾: 
\\	彗	スイ	ほうき	とし	
\\	三: 
\\	丨: 
\\	三: 
\\	丨: 
\\	彑: 
\\	彦	ゲン	ひこ	よし, こ, ひろ, やす	
\\	立: 
\\	厂: 
\\	彡: 
\\	彪		あや	たけし, たけ, あきら, かおる, たけき, つよし, とら		
\\	虎: 
\\	彡: 
\\	彬	ヒン	うるわ.しい, あき.らか	あきら, あき, よし	
\\	林: 
\\	杉: 
\\	從		したが.う, したが.える, より				
\\	彳: 
\\	从: 
\\	徠		きた.す, きた.る, く.る				
\\	彳: 
\\	來: 
\\	徵		しるし, めす				
\\	德		おしえ				
\\	徽	キ	しるし		
\\	彳: 
\\	山: 
\\	糸: 
\\	攵: 
\\	忽	コツ	たちま.ち, ゆるが.せ	ぬ	
\\	忽ち(たちまち): 
\\	勿: 
\\	心: 
\\	怜	レイ	あわ.れむ, さと.い	さと, さとし	
\\	心: 
\\	令: 
\\	恆		つね.に	わたる, ひさし			
\\	心: 
\\	亙: 
\\	恕	ジョ	ゆる.す		
\\	如: 
\\	心: 
\\	恢	カイ	ひろ.い		
\\	心: 
\\	灰: 
\\	恰	コウ	あたか.も		
\\	心: 
\\	合: 
\\	悉		つ.きる, ことごと, ことごと.く, つ.くす, つぶさ.に			
\\	悉く(ことごとく): 
\\	釆: 
\\	心: 
\\	悌			とも, やす, やすし, よし, ちか		
\\	心: 
\\	弟: 
\\	惇	シュン	あつ.い	あつし, あつ, まこと, とし, つとむ	
\\	心: 
\\	享: 
\\	惚	コツ	ほけ.る, ぼ.ける, ほ.れる		
\\	自惚れ(うぬぼれ): 
\\	惚ける(とぼける): 
\\	心: 
\\	忽: 
\\	惟	イ	おも.んみる, これ, おも.うに	ただ, よし, のぶ	
\\	心: 
\\	隹: 
\\	惠	ケイ	めぐ.む, めぐ.み		
\\	十: 
\\	日: 
\\	心: 
\\	惡		わる.い, あ.し, にく.い, ああ, いずくに, いずくんぞ, にく.む				
\\	亞: 
\\	心: 
\\	惣	ソウ	いそが.しい, そうじて	ふさ, そ	
\\	物: 
\\	心: 
\\	惹	ジャク	ひ.く		
\\	若: 
\\	心: 
\\	惺		さと.る				
\\	心: 
\\	星: 
\\	愼		つつし.む, つつし.み				
\\	心: 
\\	眞: 
\\	慧	ケイ	さとい	さと, さとし, さとる, あきら, とし	
\\	彗: 
\\	心: 
\\	憐	レン	あわ.れむ, あわ.れ		
\\	心: 
\\	米: 
\\	舛: 
\\	應	オウ	あた.る, まさに, こた.える		
\\	雁: 
\\	心: 
\\	懷		ふところ, なつ.かしい, なつ.かしむ, なつ.く, なつ.ける, いだ.く, おも.う				
\\	心: 
\\	衣: 
\\	网: 
\\	二: 
\\	二: 
\\	戊	ボ	つちのえ		
\\	丿: 
\\	戈: 
\\	或	ワク	あ.る, あるい, あるいは		
\\	或(ある): 
\\	或いは(あるいは): 
\\	戈: 
\\	口: 
\\	一: 
\\	戟		ほこ				
\\	十: 
\\	早: 
\\	戈: 
\\	戰		いくさ, たたか.う, おのの.く, そよぐ				
\\	單: 
\\	戈: 
\\	戲		たわむ.れる, たわむ.れ				
\\	虍: 
\\	豆: 
\\	戈: 
\\	托		たく.する, たの.む			
\\	手: 
\\	丿: 
\\	七: 
\\	拂		はら.う				
\\	手: 
\\	弗: 
\\	拔		ぬ.く, ぬ.ける, ぬ.かす, ぬ.かる				
\\	手: 
\\	犬: 
\\	拜		おが.む, おろが.む				
\\	手: 
\\	一: 
\\	一: 
\\	一: 
\\	一: 
\\	丨: 
\\	按		おさ.える, しら.べる			
\\	手: 
\\	安: 
\\	挺	チョウ	ぬ.く		
\\	手: 
\\	廷: 
\\	挽	バン	ひ.く	ひき	
\\	手: 
\\	免: 
\\	捧	ホウ	ささ.げる		
\\	捧げる(ささげる): 
\\	手: 
\\	奉: 
\\	捲		ま.く, ま.くる, まく.る, めく.る, まく.れる				
\\	捲る(まくる): 
\\	捲る(めくる): 
\\	手: 
\\	巻: 
\\	捷	ショウ	はや.い	かつ	
\\	手: 
\\	彑: 
\\	捺	ナツ	さ.す, お.す		
\\	手: 
\\	奈: 
\\	掠		かす.める, かす.る, かす.れる	ぐら		
\\	手: 
\\	京: 
\\	掬		きく.す, むす.ぶ, すく.う, たなごころ				
\\	掬う(すくう): 
\\	手: 
\\	勹: 
\\	米: 
\\	揃	セン	そろ.える, そろ.う, そろ.い, き.る		
\\	揃う(そろう): 
\\	揃える(そろえる): 
\\	揃い(そろい): 
\\	手: 
\\	前: 
\\	揭		かかげる				
\\	搖		ゆ.れる, ゆ.らぐ, ゆ.るぐ, ゆ.する, ゆ.さぶる, ゆ.すぶる, うご.く				
\\	手: 
\\	月: 
\\	缶: 
\\	搜		さが.す				
\\	手: 
\\	叟: 
\\	摑		つか.む, つか.まえる, つか.まる				
\\	摺	ショウ	す.る, ひだ	する, ずり	
\\	手: 
\\	習: 
\\	撒	サン	ま.く		
\\	撒く(まく): 
\\	手: 
\\	散: 
\\	撞		つ.く			
\\	手: 
\\	童: 
\\	撫	ブ	な.でる	なで, なでし, む	
\\	撫でる(なでる): 
\\	手: 
\\	無: 
\\	播	ハ	ま.く	はり	
\\	手: 
\\	番: 
\\	撰	サン	せん.する, えら.む, えら.ぶ		
\\	手: 
\\	巽: 
\\	擊		う.つ				
\\	擢	テキ	ぬ.く, ぬき.んでる		
\\	手: 
\\	羽: 
\\	隹: 
\\	攝		おさ.める, かね.る, と.る				
\\	手: 
\\	聶: 
\\	收	シュウ	おさ.める, おさ.まる		
\\	丨: 
\\	攴: 
\\	敍		つい.ず, ついで				
\\	余: 
\\	攴: 
\\	敦	トン	あつ.い	あつ, あつし, つる, のぶ, のり	
\\	享: 
\\	攴: 
\\	斐	ヒ		あや	
\\	生き甲斐(いきがい): 
\\	非: 
\\	文: 
\\	斡	アツ	めぐ.る, めぐ.らす		
\\	斡旋(あっせん): 
\\	十: 
\\	早: 
\\	斗: 
\\	斧	フ	おの		
\\	父: 
\\	斤: 
\\	斯	シ	か, こう, か.く, この, これ, ここに		
\\	瓦斯(ガス): 
\\	斯う(こう): 
\\	斯うして(こうして): 
\\	其: 
\\	斤: 
\\	於	オ	おい.て, お.ける, ああ, より		
\\	於て(おいて): 
\\	方: 
\\	冫: 
\\	旭	キョク	あさひ	あきら, あき, てる, ひ	
\\	九: 
\\	日: 
\\	昂		あ.がる, たか.い, たか.ぶる	あき, あきら, たか, たかし, のぼる		
\\	日: 
\\	卩: 
\\	昊	コウ	そら		
\\	日: 
\\	天: 
\\	昌	ショウ	さかん	まさ, まさし, よし, あき, あきら, さかえ, あつ, すけ	
\\	日: 
\\	日: 
\\	昏	コン	くら.い, くれ		
\\	氏: 
\\	日: 
\\	昴		すばる			
\\	日: 
\\	卯: 
\\	晃	コウ	あきらか	あき, あきら, てる, ひかる, みつ	
\\	日: 
\\	光: 
\\	晄	コウ	あきらか		
\\	日: 
\\	光: 
\\	晋	シン	すす.む	すすむ, ゆき, くに, すすみ, のぶ	
\\	二: 
\\	日: 
\\	晏		おそ.い			
\\	日: 
\\	安: 
\\	晒		さら.す, さらし			
\\	日: 
\\	襾: 
\\	晚		くれ, おそい				
\\	晝	チュウ	ひる				
\\	聿: 
\\	旦: 
\\	晟		あきらか			
\\	日: 
\\	成: 
\\	晦	カイ	つごもり, くら.い, みそか, くら.む	もり	
\\	日: 
\\	毎: 
\\	晨		あした, とき, あさ			
\\	日: 
\\	辰: 
\\	智	チ		さと, さとし, さとる, さとい, とも, のり, とし, あきら, じ, とみ, ひと, もと, よも	
\\	知: 
\\	日: 
\\	暉	キ	かが.やく	てる	
\\	日: 
\\	軍: 
\\	暢	チョウ	のび.る	いたる, のぶ, のぶる, なが, とうる, とおる, のり, まさ, みつる, よう	
\\	暢気(のんき): 
\\	申: 
\\	昜: 
\\	曆		こよみ				
\\	曉		あかつき, さと.る	あき			
\\	日: 
\\	堯: 
\\	曙	ショ	あけぼの		
\\	日: 
\\	署: 
\\	曝	バク	さら.す		
\\	日: 
\\	暴: 
\\	曳	エイ	ひ.く	ひき, びき	
\\	曰: 
\\	丨: 
\\	丿: 
\\	曾	ソウ、 ソ、 ゾウ	かつ.て, か.って, すなわち		
\\	八: 
\\	日: 
\\	朋	ホウ	とも		
\\	肉: 
\\	肉: 
\\	朔	サク	ついたち	たち	
\\	屮: 
\\	肉: 
\\	李	リ	すもも	もも, い	
\\	木: 
\\	子: 
\\	杏	キョウ	あんず	りょう	
\\	木: 
\\	口: 
\\	杖	ジョウ	つえ		
\\	杖(つえ): 
\\	木: 
\\	丈: 
\\	杜	ト	もり, ふさ.ぐ, やまなし		
\\	木: 
\\	土: 
\\	杭	コウ	くい	わたる	
\\	木: 
\\	亢: 
\\	杵	ショ	きね	き	
\\	木: 
\\	午: 
\\	杷	ハ	つか	わ	
\\	木: 
\\	巴: 
\\	枇	ビ			
\\	木: 
\\	比: 
\\	柊	シュ	ひいらぎ		
\\	木: 
\\	冬: 
\\	柏	ハク	かしわ	かい, かし	
\\	木: 
\\	日: 
\\	柑	コン			
\\	木: 
\\	甘: 
\\	柘	シャ	そ, つげ, やまぐわ	つ	
\\	木: 
\\	石: 
\\	柚	ユ	ゆず		
\\	木: 
\\	由: 
\\	柴	サイ	しば		
\\	此: 
\\	木: 
\\	柾	まさ	まさ, まさめ, まさき		
\\	木: 
\\	正: 
\\	栖	セイ	す.む	す, すみ	
\\	木: 
\\	襾: 
\\	栗	リツ	くり, おののく	くる, りっ	
\\	襾: 
\\	木: 
\\	栞		しおり			
\\	幵: 
\\	木: 
\\	桂	ケイ	かつら	かつ, よし, か	
\\	木: 
\\	圭: 
\\	桐	トウ	きり	ひさ, き	
\\	木: 
\\	同: 
\\	桔	キツ		き	
\\	木: 
\\	吉: 
\\	桧	カイ	ひのき, ひ		
\\	木: 
\\	会: 
\\	桶	ヨウ	おけ		
\\	木: 
\\	甬: 
\\	梁	リョウ	はり, うつばり, うちばり, やな, はし		
\\	水: 
\\	刀: 
\\	丶: 
\\	木: 
\\	梓	シ	あずさ		
\\	木: 
\\	辛: 
\\	梛		なぎ				
\\	木: 
\\	那: 
\\	條	ジョウ	えだ, すじ		
\\	攸: 
\\	木: 
\\	梢	ショウ	こずえ, くすのき		
\\	梢(こずえ): 
\\	木: 
\\	肖: 
\\	梧		あおぎり			
\\	木: 
\\	吾: 
\\	梯	テイ	はしご	だい	
\\	梯子(はしご): 
\\	木: 
\\	弟: 
\\	梶	ビ	かじ, こずえ		
\\	木: 
\\	尾: 
\\	棲	セイ	す.む	ずみ	
\\	木: 
\\	妻: 
\\	椀	ワン	はち		
\\	椀(わん): 
\\	木: 
\\	宛: 
\\	椋	リョウ	むく	ぐら	
\\	木: 
\\	京: 
\\	椛		かば, もみじ				
\\	木: 
\\	花: 
\\	椰	ヤ	やし		
\\	木: 
\\	耶: 
\\	椿	チン	つばき	つば	
\\	木: 
\\	春: 
\\	楊	ヨウ	かわ, やなぎ	やな	
\\	木: 
\\	昜: 
\\	楓	フウ	かえで		
\\	木: 
\\	風: 
\\	楕	ダ			
\\	楕円(だえん): 
\\	木: 
\\	左: 
\\	月: 
\\	楚	ソ	いばら, しもと, すわえ		
\\	林: 
\\	疋: 
\\	楠	ナン	くす, くすのき	くず, な	
\\	木: 
\\	南: 
\\	楢	シュウ	なら		
\\	木: 
\\	酋: 
\\	楯		たて			
\\	木: 
\\	盾: 
\\	榊	さかき	さかき		
\\	木: 
\\	神: 
\\	榎	カ	えのき	え, えの	
\\	木: 
\\	夏: 
\\	榛	シン	はしばみ, はり	はい, はる	
\\	木: 
\\	秦: 
\\	榮	エイ	さか.える, は.える, え		
\\	火: 
\\	火: 
\\	冖: 
\\	木: 
\\	槇	テン	まき, こずえ		
\\	木: 
\\	眞: 
\\	槌	ツイ	つち		
\\	金槌(かなづち): 
\\	木: 
\\	追: 
\\	槍	ソウ	やり	うつ	
\\	木: 
\\	倉: 
\\	槙	テン	まき, こずえ		
\\	木: 
\\	真: 
\\	槻	キ	つき		
\\	木: 
\\	規: 
\\	樂	ガク	たの.しい, たの.しむ, この.む		
\\	白: 
\\	幺: 
\\	幺: 
\\	木: 
\\	樋	トウ	ひ, とい	て, と, とよ, とわ	
\\	木: 
\\	通: 
\\	樟	ショウ	くす		
\\	木: 
\\	章: 
\\	樣		さま				
\\	木: 
\\	羊: 
\\	永: 
\\	樫	かし	かし		
\\	木: 
\\	堅: 
\\	樺	カ	かば, かんば	から, かも, かん	
\\	木: 
\\	華: 
\\	樽	ソン	たる		
\\	木: 
\\	尊: 
\\	橘	キツ	たちばな	きっ	
\\	木: 
\\	矛: 
\\	冏: 
\\	橙		だいだい				
\\	木: 
\\	登: 
\\	橫		よこ, よこたわる, よこたえる				
\\	檀	ダン	まゆみ		
\\	木: 
\\	亶: 
\\	檎	キン			
\\	木: 
\\	禽: 
\\	檜	カイ	ひのき, ひ		
\\	木: 
\\	會: 
\\	檢		しら.べる				
\\	木: 
\\	僉: 
\\	櫂		かい, かじ				
\\	木: 
\\	羽: 
\\	隹: 
\\	櫓	ロ	やぐら, おおだて		
\\	木: 
\\	魯: 
\\	櫛	シツ	くし, くしけず.る		
\\	櫛(くし): 
\\	木: 
\\	節: 
\\	櫻	オウ	さくら		
\\	木: 
\\	嬰: 
\\	欣	キン	よろこ.ぶ, よろこ.び	やすし, よし, し	
\\	斤: 
\\	欠: 
\\	欽	キン	つつし.む	よし, ひとし	
\\	金: 
\\	欠: 
\\	歎		なげ.く, なげ.き			
\\	艸: 
\\	口: 
\\	夫: 
\\	欠: 
\\	此	シ	これ, この, ここ		
\\	彼方此方(あちらこちら): 
\\	此方(こちら): 
\\	此れから(これから): 
\\	此処(ここ): 
\\	此の(この): 
\\	彼此(あれこれ): 
\\	此れ(これ): 
\\	此間(こないだ): 
\\	止: 
\\	匕: 
\\	步		あるく, あゆむ, あゆみ				
\\	歷						
\\	殆	タイ	ほとほと, ほとん.ど, あやうい		
\\	殆ど(ほとんど): 
\\	歹: 
\\	台: 
\\	毅	キ	つよ.い	つよし, こわし, たけし, たけ, はたす, あつし, つよ, とし, み	
\\	立: 
\\	豕: 
\\	殳: 
\\	每		つね				
\\	毘	ヒ	たす.ける		
\\	田: 
\\	比: 
\\	毬	キュウ	いが, まり		
\\	毛: 
\\	求: 
\\	氣	キ	いき		
\\	气: 
\\	米: 
\\	汀	テイ	みぎわ, なぎさ	て	
\\	水: 
\\	丁: 
\\	汐	セキ	しお, うしお, せい	いそ	
\\	水: 
\\	夕: 
\\	汝	ジョ	なんじ, なれ, い, うぬ, いまし, し, しゃ, な, なむち, まし, みまし		
\\	水: 
\\	女: 
\\	汲	キュウ	く.む	くみ	
\\	汲む(くむ): 
\\	水: 
\\	及: 
\\	沌	トン	くら.い		
\\	水: 
\\	屯: 
\\	沓		くつ				
\\	水: 
\\	日: 
\\	沫		あわ, しぶき, つばき			
\\	水: 
\\	末: 
\\	洛	ラク			
\\	水: 
\\	各: 
\\	洲	シュウ	しま		
\\	水: 
\\	州: 
\\	洵		の.ぶ, まこと.に			
\\	水: 
\\	旬: 
\\	洸	コウ		ひろ, ひろし, たけし, ひかり	
\\	水: 
\\	光: 
\\	浩	コウ	おおき.い, ひろ.い	ひろ, ひろし, ゆたか, こお	
\\	水: 
\\	告: 
\\	浬		かいり, のっと			
\\	水: 
\\	里: 
\\	涉		わたる, かかわる				
\\	淀	テン	よど.む		
\\	水: 
\\	定: 
\\	淋	リン	さび.しい, さみ.しい		
\\	水: 
\\	林: 
\\	淚		なみだ				
\\	淨		きよ.める, きよ.い				
\\	水: 
\\	爭: 
\\	淳	ジュン	あつ.い	あつ, あつし, きよ, きよし, まこと, すなお	
\\	水: 
\\	享: 
\\	淵	エン	ふち, かた.い, はなわ		
\\	水: 
\\	丿: 
\\	丨: 
\\	渚	ショ	なぎさ	なぎ	
\\	渚(なぎさ): 
\\	水: 
\\	者: 
\\	渥	アク	あつ.い, うるお.う	あつし	
\\	水: 
\\	屋: 
\\	渴		かわく, かわき				
\\	湊	ソウ	みなと, あつ.まる		
\\	水: 
\\	奏: 
\\	湘	ショウ			
\\	水: 
\\	相: 
\\	湛	タン	しず.む, たた.える	かん, きよ, たたう, たたえ, やす	
\\	水: 
\\	甚: 
\\	溜	リュウ	た.まる, たま.る, た.める, したた.る, たまり, ため		
\\	溜める(ためる): 
\\	溜まる(たまる): 
\\	溜まり(たまり): 
\\	水: 
\\	留: 
\\	溢	イツ	こぼ.れる, あふ.れる, み.ちる		
\\	溢れる(あふれる): 
\\	水: 
\\	益: 
\\	溫		あたたか, あたためる				
\\	滉		ひろ.い			
\\	水: 
\\	晃: 
\\	滯		とどこお.る				
\\	水: 
\\	帶: 
\\	漕	ソウ	こ.ぐ, はこ.ぶ		
\\	漕ぐ(こぐ): 
\\	水: 
\\	曹: 
\\	漣		さざなみ				
\\	水: 
\\	連: 
\\	漱	ソウ	くちすす.ぐ, くちそそ.ぐ, うがい, すす.ぐ		
\\	水: 
\\	束: 
\\	欠: 
\\	澁	ジュウ	しぶ, しぶ.い, しぶ.る				
\\	水: 
\\	止: 
\\	止: 
\\	止: 
\\	澪		みお			
\\	水: 
\\	零: 
\\	濕		しめ.る, しめ.す, うるお.う, うるお.す				
\\	水: 
\\	日: 
\\	幺: 
\\	幺: 
\\	火: 
\\	濡	ジュ	ぬれ.る, ぬら.す, ぬ.れる, ぬ.らす, うるお.い, うるお.う, うるお.す		
\\	濡れる(ぬれる): 
\\	濡らす(ぬらす): 
\\	ずぶ濡れ(ずぶぬれ): 
\\	水: 
\\	需: 
\\	瀕	ヒン	ほとり		
\\	水: 
\\	頻: 
\\	瀧	ロウ	たき		
\\	水: 
\\	龍: 
\\	瀨		せ				
\\	灘	タン	なだ, せ	だな, なん	
\\	水: 
\\	難: 
\\	灸	キュウ	やいと		
\\	久: 
\\	火: 
\\	灼	シャク	あらた, やく		
\\	火: 
\\	勺: 
\\	烏	ウ	からす, いずくんぞ, なんぞ		
\\	鳥: 
\\	焚	フン	た.く, や.く, やきがり		
\\	焚く(たく): 
\\	焚き火(たきび): 
\\	林: 
\\	火: 
\\	焰		ほのお				
\\	煉	レン	ね.る		
\\	煉瓦(れんが): 
\\	火: 
\\	東: 
\\	煌		きらめ.く, きら.めく, かがや.く				
\\	煌びやか(きらびやか): 
\\	煌々(こうこう): 
\\	火: 
\\	皇: 
\\	煤		すす			
\\	火: 
\\	某: 
\\	熙	キ	たのし.む, ひか.る, ひろ.い, よろこ.ぶ, かわ.く, あきらか, ひろ.める, ひろ.まる		
\\	匚: 
\\	巳: 
\\	火: 
\\	燈	トウ	ひ, ほ-, ともしび, とも.す, あかり		
\\	火: 
\\	登: 
\\	燎		かがりび			
\\	火: 
\\	大: 
\\	日: 
\\	小: 
\\	燒		や.く, や.ける	やき			
\\	火: 
\\	堯: 
\\	燕	エン	つばめ, つばくら, つばくろ		
\\	廿: 
\\	口: 
\\	北: 
\\	火: 
\\	燦	サン	さん.たる, あき.らか, きらめ.く, きら.めく		
\\	火: 
\\	粲: 
\\	燭	ソク	ともしび		
\\	蝋燭(ろうそく): 
\\	火: 
\\	蜀: 
\\	燿		かがや.く, ひかり			
\\	火: 
\\	羽: 
\\	隹: 
\\	爭		あらそ.う, いか.でか				
\\	爪: 
\\	彑: 
\\	亅: 
\\	爲		ため, な.る, な.す, す.る, たり, つく.る, なり				
\\	爪: 
\\	勹: 
\\	火: 
\\	爾	ジ	なんじ, しかり, その, のみ, おれ, しか	ちか, み	
\\	一: 
\\	八: 
\\	冂: 
\\	爻: 
\\	爻: 
\\	牒		ふだ			
\\	片: 
\\	世: 
\\	木: 
\\	牟	ボウ			
\\	厶: 
\\	牛: 
\\	牡	ボ	おす, お-, おん-		
\\	牛: 
\\	土: 
\\	牽	ケン	ひ.く		
\\	玄: 
\\	牛: 
\\	犀					
\\	尸: 
\\	二: 
\\	二: 
\\	牛: 
\\	狀						
\\	狹		せま.い, せば.める, せば.まる, さ				
\\	犬: 
\\	夾: 
\\	狼	ロウ	おおかみ		
\\	犬: 
\\	良: 
\\	猪	チョ	い, いのしし	いの	
\\	犬: 
\\	者: 
\\	獅	シ	しし		
\\	犬: 
\\	師: 
\\	獸		けもの, けだもの				
\\	口: 
\\	口: 
\\	田: 
\\	口: 
\\	犬: 
\\	玖	キュウ		たま, き, ひさ	
\\	玉: 
\\	久: 
\\	玲	レイ		たま, あきら, あき, りょう, れ	
\\	玉: 
\\	令: 
\\	珀	ハク			
\\	玉: 
\\	日: 
\\	珂	カ		が	
\\	玉: 
\\	可: 
\\	珈	カ	かみかざり		
\\	珈琲(コーヒー): 
\\	玉: 
\\	加: 
\\	珊		センチ, さんち			
\\	玉: 
\\	冊: 
\\	琉	リュウ			
\\	玉: 
\\	㐬: 
\\	琢	タク	みが.く	あや, たか	
\\	玉: 
\\	豕: 
\\	琥	コ			
\\	玉: 
\\	虎: 
\\	琳					
\\	玉: 
\\	林: 
\\	琵	ビ			
\\	玉: 
\\	玉: 
\\	比: 
\\	琶	ハ			
\\	玉: 
\\	玉: 
\\	巴: 
\\	瑚					
\\	玉: 
\\	胡: 
\\	瑛	エイ		よう, あき, あきら, てる, え	
\\	玉: 
\\	英: 
\\	瑞	ズイ	みず-, しるし	たま, ず, みつ	
\\	玉: 
\\	山: 
\\	而: 
\\	瑳		みが.く			
\\	玉: 
\\	差: 
\\	瑶		たま			
\\	玉: 
\\	爪: 
\\	缶: 
\\	瓜	カ	うり	う	
\\	丿: 
\\	厶: 
\\	瓢	ヒョウ	ひさご, ふくべ		
\\	票: 
\\	瓜: 
\\	甥	セイ	おい, むこ		
\\	生: 
\\	男: 
\\	甫		はじ.めて	はじめ, とし, なみ, すけ, よし		
\\	用: 
\\	丶: 
\\	畠	はたけ	はたけ, はた	はな	
\\	日: 
\\	田: 
\\	畢		おわ.る, あみ, おわ.り, ことごとく			
\\	田: 
\\	丨: 
\\	疊		たた.む, たたみ, かさ.なる, かさ.ねる				
\\	田: 
\\	田: 
\\	田: 
\\	冖: 
\\	且: 
\\	疋	ヒキ	あし		
\\	卜: 
\\	人: 
\\	疏		あら.い, うと.い, うと.む, とお.る, とお.す, まばら			
\\	疋: 
\\	㐬: 
\\	瘦		やせ.る				
\\	皐	コウ	さつき	たかし	
\\	日: 
\\	二: 
\\	二: 
\\	十: 
\\	皓	コウ	しろ.い, ひか.る	あきら, てる, あき, ひろ, ひろし	
\\	日: 
\\	告: 
\\	盃	ハイ	さかずき				
\\	不: 
\\	皿: 
\\	盜		ぬす.む				
\\	水: 
\\	欠: 
\\	皿: 
\\	盡		つ.くす, つ.きる, つ.かす, さかづき, ことごと.く	まま			
\\	聿: 
\\	火: 
\\	皿: 
\\	眞	シン	ま, まこと	さな, ち, まこ, まさ, まつ	
\\	匕: 
\\	目: 
\\	八: 
\\	眸		ひとみ			
\\	目: 
\\	牟: 
\\	瞥	ベツ			
\\	敝: 
\\	目: 
\\	矩		かね, かねざし, さしがね	のり, つね, ただし		
\\	矢: 
\\	巨: 
\\	砥	シ	と, といし, と.ぐ, みが.く, たいら.にする		
\\	石: 
\\	氏: 
\\	一: 
\\	砦	サイ	とりで		
\\	此: 
\\	石: 
\\	砧		きぬた				
\\	石: 
\\	占: 
\\	硯		すずり			
\\	石: 
\\	見: 
\\	碎		くだ.く, くだ.ける				
\\	石: 
\\	卒: 
\\	碓		たし.か, かく.たる	う, うす		
\\	石: 
\\	隹: 
\\	碗	ワン	こばち		
\\	碗(わん): 
\\	石: 
\\	宛: 
\\	碧	ヘキ		みどり, あお, たま	
\\	玉: 
\\	日: 
\\	石: 
\\	碩	セキ	おお.きい		
\\	石: 
\\	頁: 
\\	磐	バン	いわ	いわお, わ	
\\	般: 
\\	石: 
\\	磯	キ	いそ	し, そ	
\\	石: 
\\	幾: 
\\	祁						
\\	示: 
\\	邑: 
\\	祇	ギ	くにつかみ, ただ, まさに		
\\	示: 
\\	氏: 
\\	祐	ユウ	たす.ける	すけ, さち, よし, たすく, ひろ, ひろし, まさ	
\\	示: 
\\	右: 
\\	祕		ひ.める, かく.す				
\\	示: 
\\	必: 
\\	祢	ネ			
\\	示: 
\\	尓: 
\\	祷	トウ	いの.る, いの.り, まつ.る		
\\	示: 
\\	寿: 
\\	祿		さいわ.い, ふち				
\\	示: 
\\	彑: 
\\	水: 
\\	禄	ロク	さいわ.い, ふち	とし, よし	
\\	貫禄(かんろく): 
\\	示: 
\\	彑: 
\\	水: 
\\	禎	テイ	さいわ.い	さだ, ただし, よし, さち, とも, のり	
\\	示: 
\\	貞: 
\\	禪		しずか, ゆず.る				
\\	示: 
\\	單: 
\\	禮	レイ		ひろ, れ			
\\	示: 
\\	豊: 
\\	禰	ネ			
\\	示: 
\\	爾: 
\\	禱		いの.る, まつ.る				
\\	禽	キン	とり, とりこ		
\\	人: 
\\	亠: 
\\	凶: 
\\	禸: 
\\	禾	カ	いね		
\\	丿: 
\\	木: 
\\	秤	ショウ	はかり		
\\	秤(はかり): 
\\	禾: 
\\	平: 
\\	秦	シン	はた	たい, はだ, やす	
\\	三: 
\\	大: 
\\	禾: 
\\	稀	キ	まれ, まばら		
\\	稀(まれ): 
\\	禾: 
\\	希: 
\\	稔	ネン	みの.る, みのり	とし, なる, なり, みのる, ね, み	
\\	禾: 
\\	念: 
\\	稜	リョウ	いつ, かど		
\\	禾: 
\\	土: 
\\	儿: 
\\	夂: 
\\	稟		こめぐら				
\\	亠: 
\\	回: 
\\	禾: 
\\	稻		いね, いな-				
\\	禾: 
\\	爪: 
\\	臼: 
\\	穗		ほ	お			
\\	禾: 
\\	惠: 
\\	穣	ジョウ	わら, ゆたか	しげ, みのる	
\\	禾: 
\\	襄: 
\\	穰		わら, ゆたか	みのる			
\\	禾: 
\\	襄: 
\\	穹		あめ, そら				
\\	穴: 
\\	弓: 
\\	穿	セン	うが.つ, は.く		
\\	穴: 
\\	牙: 
\\	窄	サク	すぼ.める, つぼ.める, せま.い		
\\	見窄らしい(みすぼらしい): 
\\	穴: 
\\	乍: 
\\	窪	ワ	くぼ.む, くぼ.み, くぼ.まる, くぼ		
\\	穴: 
\\	水: 
\\	圭: 
\\	窺	キ	うかが.う, のぞく		
\\	穴: 
\\	規: 
\\	竣	ドウ	わらわ, わらべ, おわ.る		
\\	立: 
\\	允: 
\\	夂: 
\\	竪	ジュ	たて, た.てる, こども		
\\	臣: 
\\	又: 
\\	立: 
\\	竺			あつし		
\\	竹: 
\\	二: 
\\	竿	カン	さお		
\\	竿(さお): 
\\	竹: 
\\	干: 
\\	笈			おい		
\\	竹: 
\\	及: 
\\	笙		ふえ			
\\	竹: 
\\	生: 
\\	笠	リュウ	かさ		
\\	竹: 
\\	立: 
\\	笹	ささ	ささ	さ, さき, しの, じね	
\\	竹: 
\\	世: 
\\	筈	カツ	はず, やはず		
\\	筈(はず): 
\\	手筈(てはず): 
\\	竹: 
\\	舌: 
\\	筑	チク		つく, づき	
\\	竹: 
\\	工: 
\\	凡: 
\\	箔	ハク	すだれ		
\\	竹: 
\\	泊: 
\\	箕	キ	み	みの	
\\	竹: 
\\	其: 
\\	篇	ヘン			
\\	竹: 
\\	扁: 
\\	篠	ゾウ	しの, ささ, すず	の	
\\	竹: 
\\	條: 
\\	簞		はこ				
\\	簾	レン	すだれ, す	みす	
\\	竹: 
\\	廉: 
\\	籾		もみ			
\\	米: 
\\	刃: 
\\	粟	ゾク	あわ, もみ	あ, さっ, そう	
\\	襾: 
\\	米: 
\\	粥	イク	かゆ, かい, ひさ.ぐ		
\\	粥(かゆ): 
\\	弓: 
\\	米: 
\\	弓: 
\\	粹		いき				
\\	米: 
\\	卒: 
\\	糊	コ	のり		
\\	糊(のり): 
\\	米: 
\\	胡: 
\\	紐	チュウ	ひも		
\\	紐(ひも): 
\\	糸: 
\\	丑: 
\\	紗	サ	うすぎぬ	たえ, すず	
\\	糸: 
\\	少: 
\\	紘	コウ	おおづな, つな, つなぐ	ひろ, ひろし	
\\	糸: 
\\	厶: 
\\	紬	チュウ	つむぎ, つむ.ぐ		
\\	糸: 
\\	由: 
\\	絃	ゲン	いと		
\\	糸: 
\\	玄: 
\\	絆	ハン	きずな, ほだ.す, つな.ぐ		
\\	糸: 
\\	半: 
\\	絢	ケン		じゅん, あや	
\\	糸: 
\\	旬: 
\\	綜	ソウ	おさ.める, す.べる		
\\	糸: 
\\	宗: 
\\	綠		みどり				
\\	綴	テイ	と.じる, つづ.る, つづり, すみ.やか		
\\	綴じる(とじる): 
\\	糸: 
\\	又: 
\\	又: 
\\	又: 
\\	又: 
\\	綸	リン	いと		
\\	糸: 
\\	侖: 
\\	綺	キ	あや		
\\	糸: 
\\	奇: 
\\	綾	リン	あや	りょう	
\\	糸: 
\\	土: 
\\	儿: 
\\	夂: 
\\	緋	ヒ	あけ, あか		
\\	糸: 
\\	非: 
\\	緖		お, いとぐち				
\\	緣		ふち, へり, よる				
\\	縞	コウ	しま, しろぎぬ		
\\	縞(しま): 
\\	糸: 
\\	高: 
\\	縣	ケン	か.ける	あがた, がた			
\\	県: 
\\	系: 
\\	縱		たて				
\\	糸: 
\\	從: 
\\	繡		ぬいとり				
\\	繫		つな.ぐ, かか.る, か.ける				
\\	纂		あつ.める			
\\	算: 
\\	糸: 
\\	纏	テン	まつ.わる, まと.う, まと.める, まと.まる, まと.い		
\\	纏め(まとめ): 
\\	纏める(まとめる): 
\\	纏まる(まとまる): 
\\	纏まり(まとまり): 
\\	糸: 
\\	廛: 
\\	纖						
\\	糸: 
\\	从: 
\\	戈: 
\\	韭: 
\\	羚		かもしか				
\\	羊: 
\\	令: 
\\	翔	ショウ	かけ.る, と.ぶ	か	
\\	翔る(かける): 
\\	羊: 
\\	羽: 
\\	翠		かわせみ, みどり	あきら		
\\	羽: 
\\	卒: 
\\	耀	ヨウ	かがや.く, ひかり	あかる	
\\	光: 
\\	羽: 
\\	隹: 
\\	而	ジ	しこ.うして, しか.して, しか.も, しか.れども, すなわち, なんじ, しかるに		
\\	冂: 
\\	耶	ヤ	か		
\\	耳: 
\\	邑: 
\\	耽		ふ.ける			
\\	耽る(ふける): 
\\	耳: 
\\	冖: 
\\	八: 
\\	聡	ソウ	さと.い, みみざと.い	さと, さとし, さとる, あき, あきら, とし, さた, みのる	
\\	耳: 
\\	公: 
\\	心: 
\\	聽		き.く, ゆる.す				
\\	耳: 
\\	玉: 
\\	十: 
\\	网: 
\\	心: 
\\	肇	チョウ	はじ.める, はじめ	とし, ただし, はつ	
\\	戸: 
\\	攵: 
\\	聿: 
\\	肋		あばら			
\\	肉: 
\\	力: 
\\	肴	コウ	さかな		
\\	乂: 
\\	有: 
\\	胡	ウ	なんぞ	えびす, くる	
\\	胡椒(こしょう): 
\\	古: 
\\	肉: 
\\	胤		たね	つぎ, つぐ, かず		
\\	儿: 
\\	幺: 
\\	月: 
\\	脩	シュウ	おさ.める, なが.い, ほじし	おさむ, おさ, なが, のぶ, はる	
\\	攸: 
\\	攸: 
\\	脹		は.れる, ふく.らむ, ふく.れる			
\\	膨脹(ぼうちょう): 
\\	肉: 
\\	長: 
\\	腔	コウ			
\\	肉: 
\\	空: 
\\	膏		あぶら			
\\	高: 
\\	月: 
\\	臟		はらわた				
\\	肉: 
\\	藏: 
\\	臥	ガ	ふせ.る, ふ.せる, ふ.す		
\\	草臥れる(くたびれる): 
\\	臣: 
\\	人: 
\\	與	ヨ	あた.える, あずか.る, くみ.する, ともに		
\\	臼: 
\\	八: 
\\	舜	シュン		みつ	
\\	爪: 
\\	冖: 
\\	舛: 
\\	舵	ダ	かじ		
\\	舟: 
\\	它: 
\\	芙	フ		はす	
\\	艸: 
\\	夫: 
\\	芥	カイ	からし, ごみ, あくた		
\\	艸: 
\\	介: 
\\	芦	ロ	あし, よし	あ	
\\	艸: 
\\	戸: 
\\	芭	バ			
\\	艸: 
\\	巴: 
\\	芹	キン	せり	せい, よし	
\\	艸: 
\\	斤: 
\\	苑	エン	その, う.つ	あや	
\\	艸: 
\\	夕: 
\\	卩: 
\\	苔	タイ	こけ, こけら		
\\	艸: 
\\	台: 
\\	苺	バイ	いちご		
\\	艸: 
\\	母: 
\\	茄	カ			
\\	艸: 
\\	加: 
\\	茅	ボウ	かや, ちがや	ち, じ	
\\	艸: 
\\	矛: 
\\	茉	マツ		み	
\\	艸: 
\\	末: 
\\	茜	セン	あかね		
\\	艸: 
\\	襾: 
\\	茸	ジョウ	きのこ, たけ, しげ.る		
\\	艸: 
\\	耳: 
\\	荻	テキ	おぎ		
\\	艸: 
\\	狄: 
\\	莉	リ			
\\	艸: 
\\	利: 
\\	莊		ほうき, おごそ.か				
\\	艸: 
\\	壯: 
\\	莞		い			
\\	艸: 
\\	完: 
\\	莫	バク	くれ, なか.れ, なし		
\\	莫大(ばくだい): 
\\	艸: 
\\	日: 
\\	大: 
\\	菅	カン	すげ	す, すが, すご	
\\	艸: 
\\	官: 
\\	菖	ショウ			
\\	艸: 
\\	昌: 
\\	菩	ボ			
\\	艸: 
\\	立: 
\\	口: 
\\	菫		すみれ			
\\	艸: 
\\	三: 
\\	菱	リョウ	ひし		
\\	艸: 
\\	土: 
\\	儿: 
\\	夂: 
\\	萄	ドウ			
\\	葡萄(ぶどう): 
\\	艸: 
\\	勹: 
\\	缶: 
\\	萊		あかざ, あれわ, こうがい				
\\	萌	ホウ	も.える, きざ.す, めばえ, きざ.し	もえ, きざし	
\\	艸: 
\\	明: 
\\	萠		も.える, きざ.す, めばえ, きざ.し	めぐむ			
\\	艸: 
\\	朋: 
\\	萩	シュウ	はぎ	は	
\\	艸: 
\\	秋: 
\\	萬	マン	よろず	かず, ま, ゆる, よし	
\\	艸: 
\\	禺: 
\\	萱		かや, かんぞう	か			
\\	艸: 
\\	宣: 
\\	葡	ブ			
\\	葡萄(ぶどう): 
\\	艸: 
\\	匍: 
\\	董	トウ	ただ.す		
\\	骨董品(こっとうひん): 
\\	艸: 
\\	重: 
\\	葦	イ	しお.れる, しな.びる, しぼ.む, な.える	あし	
\\	艸: 
\\	韋: 
\\	葵	キ	あおい	まもる, け	
\\	艸: 
\\	癸: 
\\	葺	シュウ	あし, ふ.く, ふき		
\\	艸: 
\\	口: 
\\	耳: 
\\	蒐	シュウ	あかね, あつ.まる, あつ.める		
\\	艸: 
\\	鬼: 
\\	蒔	シ	う.える, まく	まい, まか, まき	
\\	散蒔く(ばらまく): 
\\	艸: 
\\	時: 
\\	蒙	モウ	こうむ.る, おお.う, くら.い		
\\	艸: 
\\	冖: 
\\	豕: 
\\	蒲	ホ	がま, かば, かま		
\\	艸: 
\\	浦: 
\\	蒼	ソウ	あお.い		
\\	艸: 
\\	倉: 
\\	蓉	ヨウ		はす, よ	
\\	艸: 
\\	容: 
\\	蓑		みの			
\\	艸: 
\\	衰: 
\\	蓬	ホウ	よもぎ		
\\	艸: 
\\	逢: 
\\	蓮	レン	はす, はちす		
\\	蓮(はす): 
\\	艸: 
\\	連: 
\\	蔓	マン	はびこ.る, つる		
\\	艸: 
\\	曼: 
\\	蔣		まこも, はげ.ます				
\\	蔦	チョウ	つた	たつ	
\\	艸: 
\\	鳥: 
\\	蔭	イン	かげ		
\\	お蔭様で(おかげさまで): 
\\	お蔭(おかげ): 
\\	艸: 
\\	陰: 
\\	蕃			しげ, しげる, ば		
\\	艸: 
\\	番: 
\\	蕉	ショウ			
\\	艸: 
\\	焦: 
\\	蕎	キョウ	そば		
\\	蕎麦(そば): 
\\	艸: 
\\	喬: 
\\	蕗	ロ	ふき		
\\	艸: 
\\	路: 
\\	蕨	ケツ	わらび		
\\	艸: 
\\	厥: 
\\	蕪	ブ	かぶ, かぶら, あれる		
\\	艸: 
\\	無: 
\\	蕾	ライ	つぼみ		
\\	蕾(つぼみ): 
\\	艸: 
\\	雷: 
\\	薗	エン		ぞの	
\\	艸: 
\\	園: 
\\	薙	テイ	な.ぐ, なぎ, か.る		
\\	艸: 
\\	雉: 
\\	薩	サツ		さっ	
\\	艸: 
\\	阜: 
\\	産: 
\\	薰		かおる, かおりぐさ, かおり				
\\	藁		わら			
\\	藁(わら): 
\\	蒿: 
\\	木: 
\\	藏	ゾウ	くら, おさ.める, かく.れる		
\\	艸: 
\\	臧: 
\\	藝	ゲイ	う.える, のり, わざ		
\\	艸: 
\\	土: 
\\	八: 
\\	土: 
\\	丸: 
\\	云: 
\\	藥		くすり				
\\	艸: 
\\	樂: 
\\	蘇	ソ	よみがえ.る		
\\	艸: 
\\	魚: 
\\	禾: 
\\	蘭	ラン		か, あららぎ	
\\	艸: 
\\	闌: 
\\	虛		むなしい				
\\	蝦	カ	えび	えみ	
\\	虫: 
\\	又: 
\\	蝶	チョウ			
\\	蝶(ちょう): 
\\	虫: 
\\	世: 
\\	木: 
\\	螺	ラ	にし, にな		
\\	虫: 
\\	累: 
\\	蟬		せみ				
\\	蟹	カイ	かに		
\\	解: 
\\	虫: 
\\	蠟		みつろう, ろうそく				
\\	衞		まも.る	まもり, まもる			
\\	行: 
\\	口: 
\\	巾: 
\\	衿		えり			
\\	衣: 
\\	今: 
\\	袈	ケ			
\\	加: 
\\	衣: 
\\	袴	コ	はかま, ずぼん		
\\	衣: 
\\	夸: 
\\	裝		よそお.う, よそお.い				
\\	壯: 
\\	衣: 
\\	裟	サ			
\\	沙: 
\\	衣: 
\\	裡	リ	うち, うら		
\\	衣: 
\\	里: 
\\	裳	ショウ	も, もすそ		
\\	尚: 
\\	衣: 
\\	襖	オウ	ふすま, あお		
\\	襖(ふすま): 
\\	衣: 
\\	奥: 
\\	覽		み.る				
\\	臣: 
\\	网: 
\\	見: 
\\	訊	ジン	き.く, と.う, たず.ねる		
\\	言: 
\\	乙: 
\\	十: 
\\	訣	ケツ	わかれ, わかれ.る		
\\	言: 
\\	夬: 
\\	註					
\\	言: 
\\	主: 
\\	詢	ジュン	はか.る, まこと		
\\	言: 
\\	旬: 
\\	詫	タ	わび, わび.しい, かこつ, わ.びる, わび.る	たく	
\\	詫び(わび): 
\\	詫びる(わびる): 
\\	言: 
\\	宅: 
\\	誼		よしみ, よい			
\\	言: 
\\	宜: 
\\	諄		ひちくど.い, くど.い, くどくど, ねんご.ろ			
\\	諄い(くどい): 
\\	言: 
\\	享: 
\\	諏	シュ	そう, はか.る		
\\	言: 
\\	取: 
\\	諒	リョウ	あきら.か, まことに	あき, まさ, まこと	
\\	言: 
\\	京: 
\\	諺	ゲン	ことわざ		
\\	諺(ことわざ): 
\\	言: 
\\	彦: 
\\	謂	イ	い.う, いい, おも.う, いわゆる		
\\	所謂(いわゆる): 
\\	言: 
\\	胃: 
\\	謠		うた.う, うた				
\\	言: 
\\	月: 
\\	缶: 
\\	讃	サン	ほ.める, たた.える	さぬ	
\\	言: 
\\	賛: 
\\	讓		ゆず.る				
\\	言: 
\\	襄: 
\\	豹	ヒョウ			
\\	豸: 
\\	勺: 
\\	貰	セイ	もら.う		
\\	"貰う(もらう): 
\\	世: 
\\	貝: 
\\	賑	シン	にぎ.わい, にぎ.やか, にぎ.わす, にぎ.わう		
\\	賑やか(にぎやか): 
\\	賑わう(にぎわう): 
\\	貝: 
\\	辰: 
\\	賣		う.る, う.れる				
\\	士: 
\\	网: 
\\	貝: 
\\	賴		たのむ, たよる, たより				
\\	赳	キュウ		たけ, たけし	
\\	走: 
\\	丨: 
\\	跨		また.がる, またが.る, また.ぐ			
\\	跨ぐ(またぐ): 
\\	跨がる(またがる): 
\\	足: 
\\	夸: 
\\	蹄	テイ	ひづめ		
\\	足: 
\\	帝: 
\\	蹟	セキ	あと		
\\	足: 
\\	責: 
\\	輔	ホ	たす.ける	すけ, たすく, ゆう	
\\	車: 
\\	甫: 
\\	輯		あつ.める, やわ.らぐ			
\\	車: 
\\	口: 
\\	耳: 
\\	輿	ヨ	かご, こし		
\\	臼: 
\\	車: 
\\	八: 
\\	轉	テン	ころ.がる, ころ.げる, ころ.がす, ころ.ぶ, まろ.ぶ, うたた, うつ.る				
\\	車: 
\\	專: 
\\	轟		とどろ.かす, とどろ.く			
\\	車: 
\\	車: 
\\	車: 
\\	辰	シン	たつ	とき, のぶ, のぶる	
\\	厂: 
\\	辻	つじ	つじ		
\\	辻褄(つじつま): 
\\	十: 
\\	辶: 
\\	辿	テン	たど.る, たどり		
\\	辿る(たどる): 
\\	山: 
\\	辶: 
\\	迂	ウ			
\\	于: 
\\	辶: 
\\	迄		まで, およ.ぶ			
\\	飽く迄(あくまで): 
\\	乞: 
\\	辶: 
\\	迦	カ			
\\	加: 
\\	辶: 
\\	迪	テキ	みち, みちび.く, すす.む, いた.る	すすむ, すすみ, いたる, ゆう	
\\	由: 
\\	辶: 
\\	逗	トウ	とど.まる		
\\	豆: 
\\	辶: 
\\	這	シャ	は.う, は.い, むか.える, この		
\\	這う(はう): 
\\	言: 
\\	辶: 
\\	逞		たくま.しい			
\\	逞しい(たくましい): 
\\	呈: 
\\	辶: 
\\	逢	ホウ	あ.う, むか.える	あい, おう	
\\	逢う(あう): 
\\	夂: 
\\	三: 
\\	丨: 
\\	辶: 
\\	遁	トン	のが.れる		
\\	盾: 
\\	辶: 
\\	遙	ヨウ	はる.か		
\\	月: 
\\	缶: 
\\	辶: 
\\	遥	ヨウ	はる.か		
\\	遥か(はるか): 
\\	爪: 
\\	缶: 
\\	辶: 
\\	遼	リョウ		はるか	
\\	大: 
\\	日: 
\\	小: 
\\	辶: 
\\	邑	ユウ	うれ.える, くに, むら	お, おう, おお	
\\	口: 
\\	巴: 
\\	郁	イク		あや, かおる, い, か, ふみ, ゆう	
\\	有: 
\\	邑: 
\\	郞		おとこ				
\\	鄭	テイ			
\\	奠: 
\\	邑: 
\\	酉	ユウ	とり	なが, みのる	
\\	襾: 
\\	一: 
\\	醇	ジュン	もっぱら, こい, あつい		
\\	酉: 
\\	享: 
\\	醉		よ.う, よ.い				
\\	酉: 
\\	卒: 
\\	醍	ダイ			
\\	酉: 
\\	是: 
\\	醐	ゴ			
\\	酉: 
\\	胡: 
\\	醬		ひしお				
\\	釀		かも.す				
\\	酉: 
\\	襄: 
\\	釉	ユウ	うわぐすり		
\\	釆: 
\\	由: 
\\	釘	テイ	くぎ		
\\	釘(くぎ): 
\\	金: 
\\	丁: 
\\	釧	セン	くしろ, うでわ	くし	
\\	金: 
\\	巛: 
\\	銑					
\\	金: 
\\	先: 
\\	鋒	ホウ	きっさき, とかり, ほこさき		
\\	金: 
\\	夂: 
\\	三: 
\\	丨: 
\\	鋸	キョ	のこ, のこぎり		
\\	鋸(のこぎり): 
\\	金: 
\\	居: 
\\	錄		しるす				
\\	錆	ショウ	さび, くわ.しい		
\\	錆(さび): 
\\	錆びる(さびる): 
\\	金: 
\\	青: 
\\	錐		きり			
\\	金: 
\\	隹: 
\\	錘		つむ, おもり			
\\	金: 
\\	垂: 
\\	錫	セキ	すず, たま.う		
\\	金: 
\\	易: 
\\	鍊		ねる				
\\	鍬	ショウ	くわ, すき		
\\	金: 
\\	秋: 
\\	鎧	カイ	よろ.う, よろい		
\\	金: 
\\	豈: 
\\	鎭		しず.める, しず.まる, おさえ				
\\	金: 
\\	眞: 
\\	鑄		い.る				
\\	金: 
\\	壽: 
\\	閃		ひらめ.く, ひらめ.き, うかが.う			
\\	門: 
\\	人: 
\\	閏		うるう				
\\	門: 
\\	玉: 
\\	閤	コウ	くぐりど		
\\	門: 
\\	合: 
\\	阿	ア	おもね.る, くま	ほとり, あず, あわ, おか, きた, な	
\\	阿弗利加(アフリカ): 
\\	阜: 
\\	可: 
\\	陀	タ	けわ.しい, ななめ		
\\	阜: 
\\	它: 
\\	陷		おちい.る, おとしい.れる				
\\	阜: 
\\	勹: 
\\	臼: 
\\	隈	ワイ	くま, すみ		
\\	阜: 
\\	畏: 
\\	險		けわ.しい				
\\	阜: 
\\	僉: 
\\	隼	シュン	はやぶさ	はや	
\\	隹: 
\\	十: 
\\	雀	ジャク	すずめ	ざく	
\\	少: 
\\	隹: 
\\	雁	ガン	かり, かりがね		
\\	厂: 
\\	人: 
\\	隹: 
\\	雛	スウ	ひな, ひよこ		
\\	雛(ひな): 
\\	雛祭(ひなまつり): 
\\	芻: 
\\	隹: 
\\	雜		まじ.える, まじ.る	さい			
\\	亠: 
\\	从: 
\\	木: 
\\	隹: 
\\	雫	ダ	しずく		
\\	雨: 
\\	下: 
\\	霞	カ	かすみ, かす.む		
\\	霞む(かすむ): 
\\	雨: 
\\	又: 
\\	靖	セイ	やす.んじる	のぶ, やす, やすし, しず, おさむ, きよし	
\\	立: 
\\	青: 
\\	靜		しず-, しず.か, しず.まる, しず.める	しずか			
\\	青: 
\\	爭: 
\\	鞄	ハク	かばん		
\\	鞄(かばん): 
\\	革: 
\\	包: 
\\	鞍	アン	くら		
\\	革: 
\\	安: 
\\	鞘	ショウ	さや		
\\	革: 
\\	肖: 
\\	鞠		まり	まい		
\\	鞠(まり): 
\\	革: 
\\	勹: 
\\	米: 
\\	鞭	ベン	むち, むちうつ		
\\	革: 
\\	便: 
\\	頁	ケツ	ぺえじ, おおがい, かしら		
\\	頁(ページ): 
\\	貝: 
\\	頌		かたち, たた.える, ほめ.る	つぐ, のぶ		
\\	公: 
\\	頁: 
\\	頗	ハ	すこぶ.る, かたよ.る		
\\	皮: 
\\	頁: 
\\	顚		いただ.き, たお.れる				
\\	顯		あきらか, あらわ.れる				
\\	日: 
\\	幺: 
\\	幺: 
\\	火: 
\\	頁: 
\\	颯	サツ	さっ.と		
\\	颯と(さっと): 
\\	立: 
\\	風: 
\\	飜		ひるがえ.す, ひるがえ.る				
\\	番: 
\\	飛: 
\\	饗	キョウ	う.ける, もてな.す	あい, あえ	
\\	郷: 
\\	食: 
\\	馨	ケイ	かお.る, かおり	か, かおる, きよ, よし, かほる	
\\	声: 
\\	殳: 
\\	香: 
\\	馳	チ	は.せる		
\\	ご馳走さま(ごちそうさま): 
\\	ご馳走(ごちそう): 
\\	ご馳走様でした(ごちそうさまでした): 
\\	馬: 
\\	也: 
\\	馴	ジュン	な.れる, な.らす, したが.う		
\\	馴れ馴れしい(なれなれしい): 
\\	馴らす(ならす): 
\\	馬: 
\\	巛: 
\\	駈	ク	か.ける, か.る		
\\	馬: 
\\	丘: 
\\	駕	カ	かご, が.する, しのぐ, のる		
\\	加: 
\\	馬: 
\\	駿	シュン	すぐ.れる	する, とし, はやし, はやお	
\\	馬: 
\\	允: 
\\	夂: 
\\	騷		さわ.ぐ, うれい, さわが.しい				
\\	馬: 
\\	蚤: 
\\	驍		たけし, つよ.い	いさ, いさむ, すぐる			
\\	馬: 
\\	堯: 
\\	驗		あかし, しるし, ため.す, ためし				
\\	馬: 
\\	僉: 
\\	髮		かみ				
\\	髟: 
\\	犬: 
\\	魁	カイ	さきがけ, かしら		
\\	鬼: 
\\	斗: 
\\	魯	ロ	おろか		
\\	魚: 
\\	日: 
\\	鮎	デン	あゆ, なまず	あい	
\\	魚: 
\\	占: 
\\	鯉	リ	こい		
\\	魚: 
\\	里: 
\\	鯛	チョウ	たい		
\\	魚: 
\\	周: 
\\	鰯	いわし	いわし		
\\	魚: 
\\	弱: 
\\	鱒	ソン	ます		
\\	魚: 
\\	尊: 
\\	鱗		うろこ, こけ, こけら			
\\	魚: 
\\	米: 
\\	舛: 
\\	鳩	キュウ	はと, あつ.める	やす	
\\	九: 
\\	鳥: 
\\	鳳	ホウ		おおとり, ふげ	
\\	几: 
\\	鳥: 
\\	鳶		とび, とんび			
\\	弋: 
\\	鳥: 
\\	鴨	オウ	かも, あひる		
\\	甲: 
\\	鳥: 
\\	鴻	コウ	おおとり, ひしくい, おおがり		
\\	江: 
\\	鳥: 
\\	鵜	テイ	う		
\\	弟: 
\\	鳥: 
\\	鵬	ホウ	おおとり		
\\	朋: 
\\	鳥: 
\\	鷄		にわとり, とり				
\\	奚: 
\\	鳥: 
\\	鷗		かもめ				
\\	鷲	シュウ		す, わ, わせ	
\\	就: 
\\	鳥: 
\\	鷹	ヨウ	たか		
\\	广: 
\\	鳥: 
\\	鷺	ロ	さぎ		
\\	路: 
\\	鳥: 
\\	麒					
\\	鹿: 
\\	其: 
\\	麟					
\\	鹿: 
\\	米: 
\\	舛: 
\\	麿	まろ	まろ	ま	
\\	麻: 
\\	呂: 
\\	黃		き				
\\	黎	レイ	くろ.い	れ	
\\	黍: 
\\	水: 
\\	黑		くろい, くろ				
\\	里: 
\\	火: 
\\	默		だ.まる, もだ.す				
\\	黒: 
\\	犬: 
\\	黛	タイ	まゆずみ		
\\	代: 
\\	黑: 
\\	鼎	テイ	かなえ		
\\	目: 
\\	齊	セイ	そろ.う, ひと.しい, ひと.しく, あたる, はやい	ひとし	
\\	亠: 
\\	刀: 
\\	二: 
\\	龍	リュウ	たつ	りゅ	
\\	立: 
\\	月: 
\\	欄						
\\	廊						
\\	朗						
\\	虜						
\\	類						
\\	猪		いのしし, い				
\\	神		かみ, こう, たましい				
\\	祥		さいわい				
\\	福		さいわい, ひもろぎ				
\\	諸		もろ, これ				
\\	渚						
\\	琢						
\\	祐						
\\	禎						
\\	悔						
\\	侮						
\\	僧						
\\	勉						
\\	勤						
\\	卑						
\\	嘆						
\\	器						
\\	墨						
\\	層						
\\	憎						
\\	懲						
\\	敏						
\\	暑						
\\	梅						
\\	海						
\\	漢						
\\	煮						
\\	碑						
\\	社						
\\	祉						
\\	祈						
\\	祖						
\\	祝						
\\	禍						
\\	穀						
\\	突						
\\	節						
\\	練						
\\	繁						
\\	署						
\\	者						
\\	臭						
\\	著						
\\	視						
\\	謁						
\\	謹						
\\	賓						
\\	贈						
\\	逸						
\\	都						
\\	難						
\\	響						
\\	煕	キ	たのし.む, ひか.る, ひろ.い, よろこ.ぶ, かわ.く, あきらか, ひろ.める, ひろ.まる	おき, ひろし	
\\	匚: 
\\	巳: 
\\	火: 
\\	渕	エン	ふち, かた.い, はなわ		
\\	水: 
\\	天: 
\\	刀: 
\\	盧	ロ			
\\	趙	チョウ キョウ			
\\	嶌	トウ			
\\	倶	グ	とも.に		
\\	人: 
\\	具: 
\\	呆	ホウ	ほけ.る, ぼ.ける, あき.れる, おろか		
\\	呆ける(ぼける): 
\\	呆れる(あきれる): 
\\	呆気ない(あっけない): 
\\	呆然(ぼうぜん): 
\\	口: 
\\	木: 
\\	粕	ハク	かす		
\\	米: 
\\	日: 
\\	姜	キョウ こう かん			
\\	崔	ガイ スイ			
\\	聯	レン	つら.なる, つら.ねる		
\\	耳: 
\\	幺: 
\\	幺: 
\\	丱: 
\\	麹	キク	こうじ		
\\	麥: 
\\	勹: 
\\	米: 
\\	炒	ソウ い.る			
\\	屏	ヘイ	おお.う, しりぞ.く, びょう.ぶ		
\\	尸: 
\\	并: 
\\	苫	セン	とま		
\\	艸: 
\\	占: 
\\	姑	コ	しゅうとめ, しゅうと, おば, しばらく		
\\	麻姑(まこ): 
\\	女: 
\\	古: 
\\	僑	キョウ		きゅう	
\\	人: 
\\	喬: 
\\	魏	ギ			
\\	荏	ジン え			
\\	銚	チョウ	なべ		
\\	金: 
\\	兆: 
\\	薮	ソウ	やぶ		
\\	艸: 
\\	数: 
\\	鴈	ガン かりがね			
\\	翫	ガン	もてあそ.ぶ			
\\	習: 
\\	元: 
\\	嶽	ガク	たけ	たき	
\\	山: 
\\	獄: 
\\	焉	エン ここに			
\\	苅	ガイ	か.る	かり, かる, かん	
\\	艸: 
\\	刈: 
\\	尹	イン ただ.す ちか			
\\	狐	コ	きつね		
\\	犬: 
\\	瓜: 
\\	鴎	オウ	かもめ		
\\	区: 
\\	鳥: 
\\	瀋	シン			
\\	狛	ハク	こま		
\\	犬: 
\\	日: 
\\	壷		つぼ			
\\	壷(つぼ): 
\\	土: 
\\	冖: 
\\	橿	キョウ	かし, もちのき		
\\	木: 
\\	一: 
\\	田: 
\\	一: 
\\	田: 
\\	一: 
\\	煥						
\\	癌	ガン			
\\	癌(がん): 
\\	疒: 
\\	嵒: 
\\	嘘	キョ	うそ, ふ.く		
\\	嘘(うそ): 
\\	嘘つき(うそつき): 
\\	口: 
\\	虚: 
\\	旛						
\\	篭	ロウ	かご, こ.める, こも.る, こ.む	ごめ	
\\	竹: 
\\	竜: 
\\	娼	ショウ	あそびめ		
\\	女: 
\\	昌: 
\\	篆	テン			
\\	鮫		さめ, みずち			
\\	魚: 
\\	交: 
\\	牌	ハイ	ぱい, ふだ		
\\	片: 
\\	卑: 
\\	蒋	ショウ まこも			
\\	襄						
\\	蛋	タン			
\\	蛋白質(たんぱくしつ): 
\\	疋: 
\\	虫: 
\\	笘						
\\	宍	ニク	しし	し, しん	
\\	宀: 
\\	六: 
\\	囃	サツ はや.す			
\\	簗						
\\	峙	ジ			
\\	舘	カン	やかた, たて	たち	
\\	舍: 
\\	官: 
\\	銕						
\\	鄒	スウ しゅう			
\\	蜷	ケン			
\\	箏	ソウ こと			
\\	鵠	コク くぐい			
\\	諌	カン	いさ.め, いさ.める		
\\	言: 
\\	東: 
\\	牝	ヒン	めす, め-, めん		
\\	牛: 
\\	匕: 
\\	笏						
\\	噺	はなし	はなし		
\\	口: 
\\	新: 
\\	抒						
\\	叩	コウ	たた.く, はた.く, すぎ		
\\	叩く(たたく): 
\\	口: 
\\	卩: 
\\	蔡						
\\	鍾	ショウ	あつ.める, さかずき, かね		
\\	金: 
\\	重: 
\\	仇	キュウ	あだ, あた, かたき, つれあい		
\\	人: 
\\	九: 
\\	瞑	メイ	めい.する, つぶ.る, つむ.る, くら.い		
\\	瞑る(つぶる): 
\\	目: 
\\	冥: 
\\	鍼	シン さ.す			
\\	礒	ギ いそ			
\\	咸	カン みな			
\\	歪	ワイ	いが.む, いびつ, ひず.む, ゆが.む		
\\	歪む(ゆがむ): 
\\	丕: 
\\	正: 
\\	韮	キュウ	にら		
\\	艸: 
\\	韭: 
\\	薔	バ ショク みずたで			
\\	膠	コウ にかわ			
\\	呑	トン	の.む		
\\	夭: 
\\	口: 
\\	茗	ミョウ ちゃ			
\\	瘤	リュウ こぶ			
\\	澤	タク うるお.い つや			
\\	溥						
\\	厦						
\\	冤	エン			
\\	侠	キョウ	きゃん, おとこだて		
\\	人: 
\\	夾: 
\\	贅	セイ			
\\	汪						
\\	猷	ユウ はかりごと					
\\	曼		なが.い			
\\	日: 
\\	网: 
\\	又: 
\\	撚	ネン	よ.る, よ.れる, より, ひね.る		
\\	手: 
\\	然: 
\\	噛	コウ	か.む, か.じる		
\\	噛む(かむ): 
\\	噛る(かじる): 
\\	噛み切る(かみきる): 
\\	口: 
\\	齒: 
\\	桝						
\\	闊	カツ			
\\	浙	セツ			
\\	彭	ホウ			
\\	淘	トウ			
\\	剃	テイ	まい, そ.る, す.る		
\\	剃刀(かみそり): 
\\	剃る(そる): 
\\	弟: 
\\	刀: 
\\	徘	ハイ			
\\	袁						
\\	舩	セン ふな-					
\\	簑						
\\	甦	ソ	よみがえ.る		
\\	甦る(よみがえる): 
\\	更: 
\\	生: 
\\	檄	ケキ ふれぶみ			
\\	姚	ヨウ うつく.しい			
\\	蛭	シツ	ひる	えび	
\\	虫: 
\\	至: 
\\	椙	すぎ	すぎ		
\\	木: 
\\	昌: 
\\	贋	ガン	にせ		
\\	雁: 
\\	貝: 
\\	洒	シャ	すす.ぐ, あら.う		
\\	洒落(しゃれ): 
\\	御洒落(おしゃれ): 
\\	洒落る(しゃれる): 
\\	洒落(しゃらく): 
\\	水: 
\\	襾: 
\\	諜		ちょう.ずる, うかが.う, しめ.す			
\\	言: 
\\	世: 
\\	木: 
\\	蓼	シン リョウ			
\\	甕	オウ みか もたい			
\\	喘	ゼン あえ.ぐ			
\\	鉾	ボウ ほこ			
\\	嗜	シ たしな.み この.み			
\\	偕	カイ			
\\	頚	ケイ	くび		
\\	又: 
\\	土: 
\\	頁: 
\\	胚	ハイ	はらみ, はら.む		
\\	肉: 
\\	丕: 
\\	牢	ロウ	かた.い, ひとや		
\\	宀: 
\\	牛: 
\\	糞	フン	くそ		
\\	米: 
\\	異: 
\\	吊	チョウ	つ.る, つる.す		
\\	吊るす(つるす): 
\\	吊る(つる): 
\\	口: 
\\	巾: 
\\	鮭	カイ	さけ, しゃけ, ふぐ		
\\	魚: 
\\	圭: 
\\	轍	テツ わだち			
\\	醤	ショウ	ひしお		
\\	醤油(しょうゆ): 
\\	将: 
\\	酉: 
\\	舛	ブ まい ます ます			
\\	謳	オウ うた.う			
\\	杞	コ			
\\	狸	リ	たぬき		
\\	犬: 
\\	里: 
\\	什					
\\	人: 
\\	十: 
\\	仄	ソク	ほの.か, ほの-, ほの.めかす, ほの.めく, かたむ.く		
\\	厂: 
\\	人: 
\\	佼		こう.す, うつく.しい			
\\	人: 
\\	交: 
\\	侭		まま, ことごとく			
\\	人: 
\\	尽: 
\\	倅		せがれ			
\\	倅(せがれ): 
\\	人: 
\\	卒: 
\\	做		な.す, つく.る			
\\	人: 
\\	故: 
\\	僻	ヘキ	へき.する, ひが.む		
\\	人: 
\\	辟: 
\\	兇	キョウ	おそ.れる, わる.い		
\\	凶: 
\\	儿: 
\\	凋	チョウ	しぼ.む		
\\	凋む(しぼむ): 
\\	冫: 
\\	周: 
\\	剪		き.る, つ.む			
\\	前: 
\\	刀: 
\\	匕		さじ			
\\	匙		さじ			
\\	匙(さじ): 
\\	是: 
\\	匕: 
\\	厖		おおき.い			
\\	厂: 
\\	尨: 
\\	厭	エン	いや, あ.きる, いと.う, おさ.える		
\\	厂: 
\\	日: 
\\	月: 
\\	犬: 
\\	叛	ハン	そむ.く		
\\	半: 
\\	反: 
\\	吃	キツ	ども.る		
\\	吃逆(しゃっくり): 
\\	吃驚(びっくり): 
\\	口: 
\\	乞: 
\\	吋		インチ			
\\	口: 
\\	寸: 
\\	吠	ハイ	ほえ.る, ほ.える		
\\	吠える(ほえる): 
\\	口: 
\\	犬: 
\\	呎		ふいいと			
\\	口: 
\\	尺: 
\\	咎		とが.める, とが			
\\	咎める(とがめる): 
\\	各: 
\\	咳	カイ	せ.く, しわぶ.く, せき, しわぶき		
\\	咳(せき): 
\\	口: 
\\	亥: 
\\	唖		おし			
\\	口: 
\\	亜: 
\\	嘔		は.く, むかつ.く, うた.う			
\\	口: 
\\	區: 
\\	噸					
\\	口: 
\\	頓: 
\\	嚢	ノウ	ふくろ		
\\	衣: 
\\	口: 
\\	丨: 
\\	冖: 
\\	八: 
\\	囁		ささや.く			
\\	囁く(ささやく): 
\\	口: 
\\	聶: 
\\	坤	コン	つち, ひつじさる		
\\	土: 
\\	申: 
\\	垢	コウ	あか, はじ		
\\	垢(あか): 
\\	土: 
\\	后: 
\\	埠	フ	つか, はとば		
\\	土: 
\\	阜: 
\\	塡		はま.る、うず.める、は.める、ふさ.ぐ		
\\	土	
\\	つち・つちへん・どへん	
\\	塵	チン	ちり, ごみ		
\\	塵(ゴミ): 
\\	塵(ちり): 
\\	微塵(みじん): 
\\	鹿: 
\\	土: 
\\	夙		つとに, はやい			
\\	几: 
\\	歹: 
\\	夥		おびただ.しい			
\\	夥しい(おびただしい): 
\\	果: 
\\	多: 
\\	妓	ギ	わざおぎ, うたいめ		
\\	女: 
\\	支: 
\\	妾		めかけ, そばめ, わらわ			
\\	立: 
\\	女: 
\\	姐	ソ	あね, ねえさん		
\\	女: 
\\	且: 
\\	姦	カン	かん.する, かしま.しい, みだら		
\\	女: 
\\	女: 
\\	女: 
\\	婉		うつく.しい, したが.う			
\\	婉曲(えんきょく): 
\\	女: 
\\	宛: 
\\	婢		はしため			
\\	女: 
\\	卑: 
\\	嬌	キョウ	なまめか.しい		
\\	女: 
\\	喬: 
\\	嬬	ジュ	よわ.い	つま	
\\	女: 
\\	需: 
\\	嬰		ふ.れる, みどりご, あかご			
\\	貝: 
\\	貝: 
\\	女: 
\\	孕	ヨウ	はら.む	はらみ, もと	
\\	乃: 
\\	子: 
\\	屁		へ, おなら			
\\	尸: 
\\	比: 
\\	屍		しかばね			
\\	屍蝋(しろう): 
\\	尸: 
\\	死: 
\\	屠	ト	ほふ.る		
\\	尸: 
\\	者: 
\\	屡		しばしば			
\\	屡々(しばしば): 
\\	屎: 
\\	女: 
\\	嵌	カン	は.める, は.まる, あな		
\\	嵌める(はめる): 
\\	嵌まる(ハマる): 
\\	山: 
\\	甘: 
\\	欠: 
\\	巫		みこ, かんなぎ			
\\	巫山戯る(ふざける): 
\\	工: 
\\	从: 
\\	幟	シ	のぼり		
\\	巾: 
\\	音: 
\\	庖		くりや			
\\	广: 
\\	包: 
\\	廓		くるわ, とりで			
\\	广: 
\\	郭: 
\\	彊	キョウ	つよ.い	つよし	
\\	弓: 
\\	一: 
\\	田: 
\\	一: 
\\	田: 
\\	一: 
\\	怯	キョウ	ひる.む, おびえ.る, おじる, おび.える, おそ.れる		
\\	卑怯(ひきょう): 
\\	怯える(おびえる): 
\\	心: 
\\	去: 
\\	悛		あらた.める			
\\	心: 
\\	允: 
\\	夂: 
\\	悶	モン	もだ.える, もだえ.る		
\\	門: 
\\	心: 
\\	愈		いよいよ, まさ.る			
\\	愈々(いよいよ): 
\\	兪: 
\\	心: 
\\	慾					
\\	欲: 
\\	心: 
\\	憑	ヒョウ	つ.く, つか.れる, よ.る, たの.む		
\\	馮: 
\\	心: 
\\	懺	ザン	くい.る			
\\	心: 
\\	从: 
\\	戈: 
\\	韭: 
\\	截		き.る, たつ			
\\	裁: 
\\	扮	フン	ふん.する, やつ.す, よそお.う		
\\	手: 
\\	分: 
\\	抉	ケツ	えぐ.る, こじ.る, くじ.る		
\\	手: 
\\	夬: 
\\	捌		さば.く, さば.ける, は.け			
\\	手: 
\\	別: 
\\	掟	トウ	おきて		
\\	手: 
\\	定: 
\\	掩		おお.う			
\\	手: 
\\	奄: 
\\	掴	カク	つか.む, つか.まえる, つか.まる		
\\	掴まる(つかまる): 
\\	掴む(つかむ): 
\\	手: 
\\	国: 
\\	掻	ソウ	か.く		
\\	藻掻く(もがく): 
\\	掻き回す(かきまわす): 
\\	掻く(かく): 
\\	引っ掻く(ひっかく): 
\\	手: 
\\	又: 
\\	虫: 
\\	揖	イツ		い	
\\	手: 
\\	口: 
\\	耳: 
\\	撹		みだ.す			
\\	手: 
\\	覚: 
\\	擾		みだ.れる, みだ.す, わずら.わしい			
\\	手: 
\\	憂: 
\\	旱		ひでり			
\\	日: 
\\	干: 
\\	暈		ぼか.す, ぼか.る, かさ, くま, ぼかし, めまい			
\\	日: 
\\	軍: 
\\	暝		くら.い			
\\	日: 
\\	冥: 
\\	曰	エツ	いわ.く, のたま.う, のたま.わく, ここに		
\\	曰: 
\\	杓		ひしゃく			
\\	木: 
\\	勺: 
\\	栂	つが	つが, とが		
\\	木: 
\\	母: 
\\	梟		ふくろう			
\\	鳥: 
\\	木: 
\\	梱	コン	こう.る, こうり, こり, しきみ		
\\	木: 
\\	困: 
\\	梵	ボン			
\\	林: 
\\	凡: 
\\	棉		わた			
\\	木: 
\\	帛: 
\\	棘	キョク	いばら, とげ		
\\	棘(とげ): 
\\	朿: 
\\	朿: 
\\	槃		たら.い			
\\	般: 
\\	木: 
\\	樒		しきみ, じんこう			
\\	木: 
\\	密: 
\\	樗		おおち, おうち	ちょう		
\\	木: 
\\	雨: 
\\	二: 
\\	樵		きこ.る, こ.る, きこり			
\\	木: 
\\	焦: 
\\	橇		そり, かんじき			
\\	橇(そり): 
\\	木: 
\\	毳: 
\\	欝		うっ.する, ふさ.ぐ, しげ.る			
\\	爻: 
\\	木: 
\\	木: 
\\	网: 
\\	艮: 
\\	寸: 
\\	沐	モク	もく.する, あら.う		
\\	水: 
\\	木: 
\\	泪	ルイ	なみだ			
\\	水: 
\\	目: 
\\	洩	エイ	も.らす, の.びる, も.れる		
\\	水: 
\\	曳: 
\\	涅		そ.める			
\\	水: 
\\	日: 
\\	土: 
\\	涜	トク	けが.す, けが.れ, みぞ		
\\	水: 
\\	売: 
\\	渾	コン	すべ.て, にご.る		
\\	水: 
\\	軍: 
\\	溌	ハツ	も.る, とびち.る, そそ.ぐ		
\\	水: 
\\	発: 
\\	滲	シン	し.みる, にじ.む		
\\	滲む(にじむ): 
\\	水: 
\\	參: 
\\	漉	ロク	こ.し, こ.す, す.く		
\\	水: 
\\	鹿: 
\\	潅		そそ.ぐ			
\\	水: 
\\	隹: 
\\	澱	デン	おり, ど.ろ, よど.み, よど.む		
\\	沈澱(ちんでん): 
\\	水: 
\\	殿: 
\\	濠	ゴウ	ほり		
\\	水: 
\\	豪: 
\\	濾		こ.す			
\\	濾す(こす): 
\\	水: 
\\	慮: 
\\	瀞	セイ	とろ, きよ.い		
\\	水: 
\\	静: 
\\	炬	コ		たい	
\\	炬燵(こたつ): 
\\	火: 
\\	巨: 
\\	烙		や.く			
\\	火: 
\\	各: 
\\	烹	ホウ	に.る		
\\	亨: 
\\	火: 
\\	焔		ほのお			
\\	火: 
\\	勹: 
\\	旧: 
\\	煽	セン	あお.る, おだ.てる, おこ.る		
\\	煽てる(おだてる): 
\\	火: 
\\	扇: 
\\	熔		と.ける, と.かす, いがた			
\\	火: 
\\	容: 
\\	燐	リン			
\\	燐寸(マッチ): 
\\	火: 
\\	米: 
\\	舛: 
\\	爺	ヤ	じい, じじい, おやじ, じじ, ちち		
\\	父: 
\\	耶: 
\\	狒		ひひ			
\\	犬: 
\\	弗: 
\\	狗	ク	いぬ		
\\	犬: 
\\	句: 
\\	狡		ずる.い, こす.い, わるがしこ.い			
\\	狡い(ずるい): 
\\	犬: 
\\	交: 
\\	狽					
\\	犬: 
\\	貝: 
\\	猥	ワイ	みだ.ら, みだり.に		
\\	犬: 
\\	畏: 
\\	猾		わるがしこ.い			
\\	犬: 
\\	骨: 
\\	珪		たま			
\\	玉: 
\\	圭: 
\\	琲	ハイ	つらぬく		
\\	珈琲(コーヒー): 
\\	玉: 
\\	非: 
\\	畦		あぜ, うね			
\\	田: 
\\	圭: 
\\	畷	テツ	なわて		
\\	田: 
\\	又: 
\\	又: 
\\	又: 
\\	又: 
\\	疹	シン	はしか		
\\	疒: 
\\	彡: 
\\	痒	ヨウ	かゆ.がる, かさ, かゆ.い		
\\	痒い(かゆい): 
\\	疒: 
\\	羊: 
\\	痔		しもがさ			
\\	疒: 
\\	寺: 
\\	痰					
\\	疒: 
\\	炎: 
\\	痺		しび.れる			
\\	麻痺(まひ): 
\\	痺れる(しびれる): 
\\	疒: 
\\	卑: 
\\	眩		げん.す, くるめ.く, まぶ.しい, くら.む, まど.う, めま.い, まばゆ.い, くれ.る, ま.う			
\\	眩しい(まぶしい): 
\\	目眩(めまい): 
\\	目: 
\\	玄: 
\\	睾	コウ			
\\	血: 
\\	幸: 
\\	砺	レイ	あらと, みが.く	と	
\\	石: 
\\	厂: 
\\	万: 
\\	碇		いかり			
\\	石: 
\\	定: 
\\	碍		さまた.げる			
\\	石: 
\\	旦: 
\\	寸: 
\\	祀	シ	まつ.る, まつり		
\\	示: 
\\	巳: 
\\	祓		はら.う			
\\	示: 
\\	犬: 
\\	丿: 
\\	祟	スイ	たた.る, たた.り		
\\	出: 
\\	示: 
\\	禦		ふせ.ぐ			
\\	御: 
\\	示: 
\\	禿		ちび.る, かむろ, は.げる, はげ			
\\	禾: 
\\	八: 
\\	稗	ハイ	ひえ		
\\	禾: 
\\	卑: 
\\	穆		やわ.らぐ			
\\	禾: 
\\	日: 
\\	小: 
\\	彡: 
\\	穎		ほさき, のぎ			
\\	頃: 
\\	禾: 
\\	頁: 
\\	竃		かまど, かま, へっつい			
\\	穴: 
\\	土: 
\\	日: 
\\	日: 
\\	站					
\\	立: 
\\	占: 
\\	笥	ス	け, はこ		
\\	箪笥(たんす): 
\\	竹: 
\\	司: 
\\	筏	バツ	いかだ		
\\	竹: 
\\	伐: 
\\	箆		へら, の, くし			
\\	竹: 
\\	比: 
\\	箪	タン	はこ		
\\	箪笥(たんす): 
\\	竹: 
\\	単: 
\\	箭	セン	や		
\\	竹: 
\\	前: 
\\	簸	ハ	ひ.る		
\\	竹: 
\\	其: 
\\	皮: 
\\	籤		くじ, かずとり			
\\	籤引き(くじびき): 
\\	籤(くじ): 
\\	竹: 
\\	从: 
\\	戈: 
\\	韭: 
\\	粁		キロメートル			
\\	粁(キロメートル): 
\\	米: 
\\	千: 
\\	粍		ミリ, ミリメートル			
\\	米: 
\\	毛: 
\\	糎		センチ, センチメートル			
\\	米: 
\\	厘: 
\\	糟		ぬか	かす		
\\	米: 
\\	曹: 
\\	糠	コウ	ぬか		
\\	米: 
\\	康: 
\\	絨					
\\	絨毯(じゅうたん): 
\\	糸: 
\\	戎: 
\\	綬	ジュ	ひも		
\\	糸: 
\\	受: 
\\	繋	ケイ	つな.ぐ, かか.る, か.ける		
\\	繋がり(つながり): 
\\	繋がる(つながる): 
\\	繋ぐ(つなぐ): 
\\	繋げる(つなげる): 
\\	車: 
\\	殳: 
\\	糸: 
\\	繍	シュウ	ぬいとり		
\\	刺繍(ししゅう): 
\\	糸: 
\\	粛: 
\\	罫	ケイ			
\\	网: 
\\	卦: 
\\	翰		はね, ふで, やまどり, ふみ			
\\	十: 
\\	早: 
\\	羽: 
\\	聘	ヘイ	へい.する, と.う, め.す		
\\	耳: 
\\	由: 
\\	聚		あつ.まる			
\\	取: 
\\	人: 
\\	聾		ろう.する, つんぼ, みみしい			
\\	龍: 
\\	耳: 
\\	肛	コウ			
\\	肉: 
\\	工: 
\\	肱		かいな, ひじ, まるい			
\\	肉: 
\\	厶: 
\\	胱	コウ			
\\	肉: 
\\	光: 
\\	脆	ゼイ	もろ.い, よわい		
\\	脆い(もろい): 
\\	肉: 
\\	危: 
\\	脛		すね, はぎ			
\\	肉: 
\\	一: 
\\	巛: 
\\	工: 
\\	腑	フ	はらわた		
\\	肉: 
\\	府: 
\\	膿		う.む, うみ			
\\	肉: 
\\	農: 
\\	舅	キュウ	しゅうと		
\\	臼: 
\\	男: 
\\	艘	ショウ	ふね		
\\	艘(そう): 
\\	舟: 
\\	叟: 
\\	芻	スウ	まぐさ		
\\	勹: 
\\	艸: 
\\	勹: 
\\	苓		みみなぐさ			
\\	艸: 
\\	令: 
\\	莨		たばこ			
\\	艸: 
\\	良: 
\\	菰	コ	こも, まこも		
\\	艸: 
\\	孤: 
\\	葱	ソウ	ねぎ		
\\	艸: 
\\	怱: 
\\	蒜	サン	にんにく, ひる, のびる		
\\	艸: 
\\	示: 
\\	示: 
\\	蔚		うち, おとこよもぎ			
\\	艸: 
\\	尉: 
\\	蕊		しべ			
\\	艸: 
\\	心: 
\\	心: 
\\	心: 
\\	蕩		とろ.かす, とろ.ける, うご.く			
\\	蕩ける(とろける): 
\\	蕩蕩(とうとう): 
\\	艸: 
\\	湯: 
\\	薯		いも			
\\	艸: 
\\	署: 
\\	藷		いも			
\\	艸: 
\\	諸: 
\\	虔	ケン	つつし.む		
\\	虍: 
\\	文: 
\\	蚤	ソウ	のみ, はやい		
\\	叉: 
\\	丶: 
\\	丶: 
\\	虫: 
\\	蛎	レイ	かき		
\\	虫: 
\\	厂: 
\\	万: 
\\	蛙	ア	かえる, かわず		
\\	虫: 
\\	圭: 
\\	蛤		はまぐり			
\\	虫: 
\\	合: 
\\	蛸		たこ			
\\	虫: 
\\	肖: 
\\	蛾		ひむし			
\\	虫: 
\\	我: 
\\	蝉	セン	せみ		
\\	虫: 
\\	単: 
\\	蝋		みつろう, ろうそく	あじ		
\\	蝋燭(ろうそく): 
\\	屍蝋(しろう): 
\\	虫: 
\\	鼠: 
\\	蝕	ショク	むしば.む		
\\	食: 
\\	虫: 
\\	蝿		はえ, はい			
\\	虫: 
\\	日: 
\\	日: 
\\	蟄	チツ	ちっ.する, かく.れる		
\\	執: 
\\	虫: 
\\	蟻	ギ	あり		
\\	虫: 
\\	義: 
\\	袷		あわせ			
\\	衣: 
\\	合: 
\\	覗	シ	のぞ.く, うかが.う		
\\	覗く(のぞく): 
\\	司: 
\\	見: 
\\	詑		わ.びる, わ.び, あざむ.く			
\\	言: 
\\	它: 
\\	誹	ヒ	そし.る		
\\	言: 
\\	非: 
\\	謬		あやま.る			
\\	言: 
\\	羽: 
\\	彡: 
\\	譬		たと.える, たと.え			
\\	辟: 
\\	言: 
\\	讐	シュウ	あだ, むく.いる, あ.たる		
\\	隹: 
\\	隹: 
\\	言: 
\\	賎		いや.しい, いや.しむ, いや.しめる, しず, やす.い			
\\	貝: 
\\	戔: 
\\	赫	カク	あかい, あか, かがや.く		
\\	赤: 
\\	赤: 
\\	趨	スウ	しゅ, おもむ.く, はし.る		
\\	走: 
\\	芻: 
\\	跪		ひざまず.く			
\\	足: 
\\	危: 
\\	躇	チョ	ためら.う		
\\	躊躇う(ためらう): 
\\	足: 
\\	著: 
\\	躊	チュウ	ためら.う		
\\	躊躇う(ためらう): 
\\	足: 
\\	壽: 
\\	躓	チ	つまず.く		
\\	躓く(つまずく): 
\\	足: 
\\	質: 
\\	躯		むくろ, からだ			
\\	身: 
\\	区: 
\\	躾		しつ.ける, しつけ			
\\	躾(しつけ): 
\\	躾ける(しつける): 
\\	身: 
\\	美: 
\\	輻		や			
\\	車: 
\\	一: 
\\	口: 
\\	田: 
\\	轡		くつわ, たづな			
\\	車: 
\\	糸: 
\\	糸: 
\\	口: 
\\	逼	フク	せま.る, むかばき		
\\	一: 
\\	口: 
\\	田: 
\\	辶: 
\\	邁		ゆ.く	すす.む		
\\	萬: 
\\	辶: 
\\	酋		おさ, ふるざけ, さけのつかさ			
\\	八: 
\\	酉: 
\\	釦		ぼたん			
\\	釦(ボタン): 
\\	金: 
\\	口: 
\\	鉦	セイ	かね		
\\	金: 
\\	正: 
\\	鋏		はさみ, はさ.む, つるぎ			
\\	鋏(はさみ): 
\\	金: 
\\	夾: 
\\	鋤		す.く, すき, くわ			
\\	金: 
\\	助: 
\\	鋪		しく, みせ	しき		
\\	金: 
\\	甫: 
\\	鋲		びょう			
\\	金: 
\\	兵: 
\\	錨		いかり			
\\	金: 
\\	苗: 
\\	鎚		つち			
\\	金: 
\\	追: 
\\	鏑	テキ	かぶら, かぶらや, やじり		
\\	金: 
\\	亠: 
\\	冂: 
\\	古: 
\\	鑓		やり			
\\	金: 
\\	遣: 
\\	鑿	サク	のみ, うが.つ		
\\	臼: 
\\	殳: 
\\	金: 
\\	雉		きじ	き		
\\	矢: 
\\	隹: 
\\	靭	ジン	うつぼ, しな.やか, ゆぎ	ゆき	
\\	革: 
\\	刃: 
\\	頰		ほお、ほほ		
\\	頁	
\\	おおがい・いちのかい	
\\	顛	テン	いただき, たお.れる		
\\	真: 
\\	頁: 
\\	飴	イ	あめ, やしな.う		
\\	飴(あめ): 
\\	食: 
\\	台: 
\\	餐		の.む, くら.う			
\\	歹: 
\\	又: 
\\	食: 
\\	饅	マン			
\\	食: 
\\	曼: 
\\	駁	ハク	ぶち, まじ.る, まだら		
\\	馬: 
\\	爻: 
\\	騙	ヘン	かた.る, だま.す		
\\	騙す(だます): 
\\	馬: 
\\	扁: 
\\	騨	タン			
\\	馬: 
\\	単: 
\\	驢		うさぎうま			
\\	馬: 
\\	盧: 
\\	髭	シ	ひげ, くちひげ		
\\	髭(ひげ): 
\\	髟: 
\\	此: 
\\	鮒		ふな			
\\	魚: 
\\	付: 
\\	鮨		すし			
\\	魚: 
\\	旨: 
\\	鮪	キ	まぐろ, しび		
\\	魚: 
\\	有: 
\\	鯖	セイ	さば		
\\	魚: 
\\	青: 
\\	鯵	ソウ	あじ		
\\	魚: 
\\	参: 
\\	鰍	シュウ	かじか, いなだ, どじょう		
\\	魚: 
\\	秋: 
\\	鰐		わに			
\\	魚: 
\\	咢: 
\\	鰭		ひれ			
\\	魚: 
\\	耆: 
\\	鰹	ケン	かつお		
\\	魚: 
\\	堅: 
\\	鰻	マン	うなぎ		
\\	魚: 
\\	曼: 
\\	鱈	セツ	たら		
\\	出鱈目(でたらめ): 
\\	矢鱈に(やたらに): 
\\	魚: 
\\	雪: 
\\	鴬		うぐいす			
\\	小: 
\\	冖: 
\\	鳥: 
\\	鵡	ブ			
\\	武: 
\\	鳥: 
\\	鸚	オオ			
\\	嬰: 
\\	鳥: 
\\	鹵		しお, しおち, たて			
\\	卜: 
\\	囗: 
\\	鹸	ケン	あ.く		
\\	石鹸(せっけん): 
\\	鹵: 
\\	僉: 
\\	黍		きび			
\\	禾: 
\\	水: 
\\	鼈	ベツ	すっぽん		
\\	敝: 
\\	黽: 
\\	鼠		ねずみ, ねず			
\\	鼠(ねずみ): 
\\	臼: 
\\	鼾		いびき			
\\	鼾(いびき): 
\\	鼻: 
\\	干: 
\\	剝		はぐ、むく、はげる				
\\	刀	
\\	かたな・りっとう	
\\	𠮟						
\\	口	
\\	くち・くちへん	
\\	咋	サク か.む くら.う くい			
\\	證	ショウ					
\\	潘	ハン			
\\	栩	ク			
\\	覃	タン			
\\	疵	シ			
\\	藪	ソウ			
\\	濱	ヒン					
\\	叟	ソウ おきな			
\\	濤	トウ			
\\	剱	ケン			
\\	鏝	マン			
\\	鰒	フク ふぐ			
\\	熾	シ おこ.す			
\\	熨	イ のし の.す			
\\	蔟	ゾク まぶし			
\\	縷	ル いと			
\\	圀	コク こつ			
\\	蹊	ケイ わた.る			
\\	罠	ビン わな			
\\	譚	タン			
\\	涛	トウ と			
\\	毫	ゴウ ごう.も			
\\	隕	イン ウン			
\\	邊	ヘン ほと.り なべ					
\\	腱	ケン			
\\	爛	ラン			
\\	扁	ヘン			
\\	愕	ガク			
\\	怡	イ			
\\	彷	ホウ			
\\	俑	ヨウ ユウ			
\\	鐸	タク すず			
\\	鉉	ケン			
\\	諫	カン			
\\	蜃	シン			
\\	瞰	カン			
\\	炸	タク サク			
\\	涌	ユウ ユ			
\\	泄	セツ も.れる			
\\	麩	フ			
\\	鑼	ラ			
\\	轢	レキ きし.る			
\\	諤	ガク あご			
\\	褪	タイ あ.せる			
\\	蛯	ケン			
\\	紆	ウ めぐ.る			
\\	糀	こうじ			
\\	磋	サ			
\\	獏	バク			
\\	爬	ハ			
\\	澂	チョウ す.む					
\\	櫃	キ			
\\	曄	ヨウ あき			
\\	彿	フツ			
\\	齋	サイ とき			
\\	飫	ヨウ ヨ			
\\	頸	ケイ			
\\	軋	アツ きし.む			
\\	詭	キ			
\\	稷	ショク			
\\	瀾	ラン			
\\	滔	トウ			
\\	榜	ボウ かじ ふだ			
\\	桓	カン			
\\	曠	コウ むな.しい			
\\	揉	ジュウ も.める			
\\	徊	カイ エ			
\\	媚	ビ こび			
\\	壺	コ			
\\	嗚	ウ ああ			
\\	頷	ガン うなず.く			
\\	鑚	サン			
\\	酩	メイ			
\\	酊	テイ			
\\	螢	ケイ			
\\	薇	ラ ぜんまい			
\\	臍	セイ へそ			
\\	羹	コウ あつもの			
\\	箙	フク			
\\	窈	ヨウ			
\\	祠	シ まつる			
\\	痍	イ			
\\	甑	ソウ こしき			
\\	瑪	メ			
\\	瑙	ノウ			
\\	熏	クン ふす.べる いぶ.る くゆ.らす					
\\	澹	タン あわ.い			
\\	櫨	ロ			
\\	桿	カン			
\\	柩	キュウ ひちぎ			
\\	捏	ネツ ネチ ね.る			
\\	忖	ソン			
\\	幇	ホウ			
\\	嵜	キ さい			
\\	墟	キョ			
\\	啼	テイ			
\\	咤	タ しか.る			
\\	咄	トツ しか.る はな			
\\	劭	ショウ			
\\	佗	タ わび.しい ほか			
\\	饒	ジョウ ゆたか ぬう よ			
\\	餡	アン			
\\	餞	セン			
\\	霍	カク つる			
\\	雖	スイ これ			
\\	隋	ズイ タ			
\\	陜	キョウ せま.い			
\\	鉗	ケン つぐ.む			
\\	鈞	キン			
\\	邱	キュウ			
\\	躬	キュウ み			
\\	誦	ショウ ズ よ.む			
\\	衾	キン			
\\	蠣	レイ			
\\	蜻	セイ			
\\	蜘	チ			
\\	蛛	チュ			
\\	蛉	レイ			
\\	虻	ボウ			
\\	蘊	ウン			
\\	苧	チョ からんし			
\\	臺	ダイ うてな つかさ			
\\	腿	タイ			
\\	翳	エイ かげ.る きぬがさ くもり			
\\	翅	シ つばさ			
\\	纓	エイ			
\\	筬	セイ			
\\	禹	ウ			
\\	礫	レキ こいし			
\\	矮	ワイ ひく.い			
\\	瞽	コ			
\\	瞞	マン バン だま.す			
\\	癬	セン			
\\	瑕	カ きず			
\\	焙	ホウ ほう.じる			
\\	瀝	レキ			
\\	瀚	カン なみ			
\\	瀑	バク ホウ ホク たき			
\\	澎	ホウ			
\\	滸	コ			
\\	溟	メイ くら.い			
\\	湃	ハイ			
\\	涵	カン			
\\	殷	イン さかん			
\\	欅	キョ			
\\	槐	カイ えんじ			
\\	梠	ロ ひさし			
\\	斂	レン			
\\	攫	カク つか.む			
\\	攘	ジョウ はら.う			
\\	揶	ヤ			
\\	揄	ヨウ ユ			
\\	拮	カツ キツ			
\\	拗	オウ イク ねじ.れる す.ねる			
\\	戌	ジュツ			
\\	恪	カク			
\\	恍	コウ ほ.れる			
\\	徨	コウ			
\\	帚	ソウ ほうき			
\\	巍	ギ たかし			
\\	寨	サイ			
\\	娟	エン			
\\	址	シ			
\\	嗤	シ			
\\	嗟	サ あ なげ.く			
\\	唸	テン うなり			
\\	剌	ラツ			
\\	几	キ つくえ			
\\	凖	ジュン じゅん.ずる のり みずもり			
\\	冽	レツ きよ.い			
\\	冑	チュウ よろい			
\\	佞	ネイ よこしま			
\\	鶯	オウ			
\\	鴉	ア			
\\	鱸	ロ			
\\	鰤	シ ぶり かます			
\\	鰓	サイ えら			
\\	鰊	レン			
\\	鰈	チョウ かれい			
\\	鮓	サ			
\\	鬆	ショウ シュ			
\\	髷	キョク わげ			
\\	髑	ドク			
\\	髏	ロ			
\\	驟	シュウ はせ.る			
\\	饌	セン そな.える			
\\	饉	キン			
\\	餃	ギョウ コウ			
\\	飄	ヒョウ つむじかぜ			
\\	顰	ヒン しか.める			
\\	顆	カ			
\\	韶	ショウ つぐ			
\\	韋	イ なめしがわ			
\\	韃	ダツ だっ			
\\	靼	タン タツ なめしがわ			
\\	靱	ジン サ サツ うつお ゆぎ			
\\	雍	ヨウ ふさ.ぐ			
\\	隘	アイ せま.い			
\\	陞	ショウ			
\\	陋	ロウ せま.い			
\\	閔	ビン あわ.れむ			
\\	鐙	トウ たかつき			
\\	鐔	タン つば			
\\	鐐	ハン あしかせ			
\\	鎗	ソウ やり			
\\	鎔	ヨウ と.かす よん			
\\	鈿	デン かんざし			
\\	醂	ラン あわ.す			
\\	鄲	タン			
\\	邯	カン			
\\	邉	ヘン ほと.り なべ					
\\	邇	ジ ちか.い			
\\	邂	カイ			
\\	逡	シュン			
\\	逅	コウ まみ.える			
\\	輦	レン てぐるま			
\\	軛	ヤク くびき			
\\	躙	リン ふ.む			
\\	蹂	ジュウ			
\\	跏	カ			
\\	趾	シ はや.い			
\\	贔	ヒ			
\\	贄	シ			
\\	谿	ケイ たにがわ			
\\	讀	ドク トウ					
\\	謐	ヒツ			
\\	諍	ソウ いさか.う いさ.める			
\\	詈	リ			
\\	訝	ガ いぶか.る			
\\	訛	カ なま.り			
\\	襷	たすき			
\\	裔	エイ			
\\	袰	ほろ			
\\	衒	ゲン てら.う			
\\	蠍	カツ			
\\	蟋	シツ			
\\	蟀	シュツ			
\\	號	コウ よびな					
\\	蘆	ロ よし			
\\	葯	ヤク			
\\	葭	カ あし			
\\	莱	ライ あかざ こうがい			
\\	莪	ガ			
\\	莚	エン			
\\	茹	ジョ ゆ.でる			
\\	茵	イン			
\\	茲	シ ここ.に			
\\	茫	ボウ			
\\	芒	ボウ モウ のぎ			
\\	艸	ソウ くさ			
\\	舳	ジク チュウ へさき			
\\	舫	ホウ もやいぶね			
\\	膵	スイ			
\\	膣	チツ			
\\	膀	ボウ			
\\	腟	チツ			
\\	腓	ヒ こむら			
\\	聰	ソウ みみざと.い あきら			
\\	翹	ギョウ つまだ.てる			
\\	罹	リ かか.る			
\\	繹	エキ たずね.る			
\\	繪	カイ			
\\	繩	ジョウ ただ.す			
\\	繚	リョウ めぐ.る			
\\	緘	カン と.じる			
\\	綏	スイ やす.い			
\\	粂	くめ			
\\	籐	トウ			
\\	籃	ラン			
\\	簒	サン うば.う			
\\	簀	サク セキ			
\\	箴	シン はり			
\\	箟	キン					
\\	箒	ソウ ほうき			
\\	筧	ケン かけひ			
\\	稠	チュウ おお.い			
\\	祗	シ まさに			
\\	碆	ハ			
\\	砿	コウ			
\\	矜	キン ケイ つつし.む			
\\	矗	チョク チク			
\\	瞼	ケン			
\\	睨	ゲイ にら.み			
\\	癈	ハイ			
\\	瘡	ソウ かさ			
\\	痙	ケイ ひきつ.る			
\\	疽	ソ かさ			
\\	疸	タン			
\\	疱	ホウ			
\\	疆	キョウ			
\\	甞	ショウ かつ.て			
\\	瓷	シ			
\\	瑣	サ くさり			
\\	珎	チン たから					
\\	獰	ドウ わる.い			
\\	猊	ゲイ			
\\	牀	ショウ ゆか					
\\	燼	ジン			
\\	燻	クン ふす.べる いぶ.る くゆ.らす			
\\	燵	タツ			
\\	燗	ラン かん.する			
\\	瀟	ショウ			
\\	瀉	シャ は.く だか			
\\	濬	シュン			
\\	漲	チョウ			
\\	滓	シ おり			
\\	滄	ソウ			
\\	溪	ケイ たにがわ			
\\	游	ユウ あそ.び およ.ぐ			
\\	渠	キョ なんぞ みぞ だかり			
\\	淺	セン					
\\	淇	キ			
\\	涸	コ か.れる こお.る			
\\	涎	セン よだれ			
\\	洙	シュ			
\\	泗	シ			
\\	泓	オオ ふか.い			
\\	沁	シン			
\\	檻	カン おばしま			
\\	樸	ボク ホク きじ			
\\	樓	ロウ			
\\	槿	キン			
\\	榧	ヒ			
\\	楡	ユ			
\\	椒	ショウ			
\\	棹	トウ さお			
\\	棣	テイ にわざくら			
\\	棗	ソウ			
\\	栫	セン			
\\	栢	ハク かしわ かや			
\\	杳	ヨウ はるか			
\\	朧	ロウ			
\\	會	カイ あ.う あつ.まる あい			
\\	曵	エイ びき					
\\	斌	ヒン うるわ.しい たけ			
\\	敲	コウ のぼる			
\\	摸	モ バク マク			
\\	拇	ボ			
\\	戮	リク ロク ク けず.る			
\\	戎	ジュウ つわもの えび			
\\	戀	レン こい			
\\	憚	タン はばか.る			
\\	憔	ショウ ソウ やせ.る			
\\	愾	ガイ ケ ためいき			
\\	惻	ソク いた.む			
\\	悸	キ			
\\	悴	スイ やつ.れる			
\\	恂	ジュン まこと			
\\	弩	ド いしゆみ			
\\	廼	ナイ ノ すなわ.ち			
\\	廠	ショウ			
\\	廈	カ いえ			
\\	嶼	ショ			
\\	崗	コウ			
\\	崑	コン			
\\	岱	タイ			
\\	屹	キツ			
\\	屓	キ			
\\	寇	コウ					
\\	學	ガク さとる					
\\	孵	フ			
\\	娜	ダ			
\\	娑	シャ			
\\	姶	オオ アイ			
\\	奢	シャ おご.り			
\\	奘	ジョウ さかん			
\\	夘	ボウ			
\\	壜	ドン びん			
\\	埒	レツ ラチ かこ.い			
\\	囿	ユウ			
\\	囀	テン			
\\	噪	ソウ			
\\	噤	キン			
\\	喨	リョウ			
\\	啖	タン くらわ.す			
\\	哥	カ あに			
\\	呵	カ しか.る			
\\	吽	コウ イン グ			
\\	卉	キ くさ			
\\	匣	コウ			
\\	冲	チュウ おきつ わく			
\\	册	サツ ふみ			
\\	儡	ライ			
\\	儘	ジン ことごとく			
\\	僭	シン せん.する			
\\	傅	フ つく でん			
\\	傀	カイ			
\\	假	カ かり			
\\	俥	くるま			
\\	俟	シ			
\\	佇	チョ			
\\	踝										
\\	踵										
\\	皺										
\\	頤										
\\	戈										
\\	兔										
\\	猴										
\\	豬										
\\	鈎										
\\	鯆										
\\	鮠										
\\	鯊										
\\	鯰										
\\	鰌										
\\	鮃										
\end{CJK}
\end{document}