\documentclass[8pt]{extreport} 
\usepackage{hyperref}
\usepackage{CJKutf8}
\begin{document}
\begin{CJK}{UTF8}{min}
\\	四つ	四[よっ]つ	よっつ	
\\	私は腕時計を四つ持っています。	私[わたし]は 腕時計[うでどけい]を 四[よっ]つ 持[も]っています。	わたし は うでどけい を よっつ もって います	
\\	私[わたし]は 腕時計[うでどけい]を
\\	持[も]っています。			
\\	四	四[し]	し	
\\	四月に大学に入学しました。	四[し] 月[がつ]に 大学[だいがく]に 入学[にゅうがく]しました。	しがつ に だいがく に にゅうがく しました	
\\	月[がつ]に 大学[だいがく]に 入学[にゅうがく]しました。			
\\	四	四[よん]	よん	
\\	ハワイは四回目です。	ハワイは 四[よん] 回目[かいめ]です。	はわい は よんかいめ です	
\\	ハワイは
\\	回目[かいめ]です。			
\\	時	時[とき]	とき	
\\	時の経つのは早い。	時[とき]の 経[た]つのは 早[はや]い。	とき の たつ の は はやい	
\\	の 経[た]つのは 早[はや]い。			
\\	時々	時々[ときどき]	ときどき	
\\	彼は時々遅刻します。	彼[かれ]は 時々[ときどき] 遅刻[ちこく]します。	かれ は ときどき ちこく します	
\\	彼[かれ]は
\\	遅刻[ちこく]します。			
\\	日	日[にち]	にち	
\\	私たちは先月11日に結婚しました。	私[わたし]たちは 先月11[せんげつ じゅういち] 日[にち]に 結婚[けっこん]しました。	わたしたち は せんげつ じゅういち にち に けっこん しました	
\\	私[わたし]たちは 先月11[せんげつ じゅういち]
\\	に 結婚[けっこん]しました。			
\\	四日	四日[よっか]	よっか	
\\	新学期は来月の四日からです。	新学期[しんがっき]は 来月[らいげつ]の 四日[よっか]からです。	しんがっき は らいげつ の よっか から です	
\\	新学期[しんがっき]は 来月[らいげつ]の
\\	からです。			
\\	月	月[つき]	つき	
\\	今夜は月がとてもきれいです。	今夜[こんや]は 月[つき]がとてもきれいです。	こんや は つき が とても きれい です	
\\	今夜[こんや]は
\\	がとてもきれいです。			
\\	水	水[みず]	みず	
\\	水を一杯ください。	水[みず]を 一杯[いっぱい]ください。	みず を いっぱい ください	
\\	を 一杯[いっぱい]ください。			
\\	日曜日	日曜日[にちようび]	にちようび	
\\	日曜日は海に行きました。	日曜日[にちようび]は 海[うみ]に 行[い]きました。	にちようび は うみ に いきました	
\\	は 海[うみ]に 行[い]きました。			
\\	土曜日	土曜日[どようび]	どようび	
\\	土曜日の夜はクラブに行きます。	土曜日[どようび]の 夜[よる]はクラブに 行[い]きます。	どようび の よる は くらぶ に いきます	
\\	の 夜[よる]はクラブに 行[い]きます。			
\\	月曜日	月曜日[げつようび]	げつようび	
\\	月曜日に会いましょう。	月曜日[げつようび]に 会[あ]いましょう。	げつようび に あいましょう	
\\	に 会[あ]いましょう。			
\\	木曜日	木曜日[もくようび]	もくようび	
\\	木曜日は仕事が休みです。	木曜日[もくようび]は 仕事[しごと]が 休[やす]みです。	もくようび は しごと が やすみ です	
\\	は 仕事[しごと]が 休[やす]みです。			
\\	曜日	曜日[ようび]	ようび	
\\	曜日を間違えました。	曜日[ようび]を 間違[まちが]えました。	ようび を まちがえました	
\\	を 間違[まちが]えました。			
\\	水曜日	水曜日[すいようび]	すいようび	
\\	水曜日はバイトがあります。	水曜日[すいようび]はバイトがあります。	すいようび は ばいと が あります	
\\	はバイトがあります。			
\\	年	年[とし]	とし	
\\	新しい年が始まりました。	新[あたら]しい 年[とし]が 始[はじ]まりました。	あたらしい とし が はじまりました	
\\	新[あたら]しい
\\	が 始[はじ]まりました。			
\\	来る	来[く]る	くる	
\\	彼は昼過ぎに来ます。	彼[かれ]は 昼過[ひるす]ぎに 来[き]ます。	かれ は ひるすぎ に きます	
\\	彼[かれ]は 昼過[ひるす]ぎに
\\	来年	来年[らいねん]	らいねん	
\\	来年一緒に旅行しましょう。	来年[らいねん] 一緒[いっしょ]に 旅行[りょこう]しましょう。	らいねん いっしょ に りょこう しましょう	
\\	一緒[いっしょ]に 旅行[りょこう]しましょう。			
\\	帰る	帰[かえ]る	かえる	
\\	家に帰ろう。	家[うち]に 帰[かえ]ろう。	うち に かえろう	
\\	家[うち]に
\\	大きい	大[おお]きい	おおきい	
\\	あの大きい建物は何ですか。	あの 大[おお]きい 建物[たてもの]は 何[なん]ですか。	あの おおきい たてもの は なん です か	
\\	あの
\\	建物[たてもの]は 何[なん]ですか。			
\\	小さい	小[ちい]さい	ちいさい	
\\	小さい花が咲いています。	小[ちい]さい 花[はな]が 咲[さ]いています。	ちいさい はな が さいて います	
\\	花[はな]が 咲[さ]いています。			
\\	少ない	少[すく]ない	すくない	
\\	今年は雨が少ないです。	今年[ことし]は 雨[あめ]が 少[すく]ないです。	ことし は あめ が すくない です 。	
\\	今年[ことし]は 雨[あめ]が
\\	です。			
\\	少し	少[すこ]し	すこし	
\\	少し疲れました。	少[すこ]し 疲[つか]れました。	すこし つかれました	
\\	疲[つか]れました。			
\\	多い	多[おお]い	おおい	
\\	京都にはお寺が多い。	京都[きょうと]にはお 寺[てら]が 多[おお]い。	きょうと に は おてら が おおい	
\\	京都[きょうと]にはお 寺[てら]が
\\	多分	多分[たぶん]	たぶん	
\\	彼女は多分家で寝ています。	彼女[かのじょ]は 多分[たぶん] 家[いえ]で 寝[ね]ています。	かのじょ は たぶん いえ で ねて います	
\\	彼女[かのじょ]は
\\	家[いえ]で 寝[ね]ています。			
\\	右	右[みぎ]	みぎ	
\\	右のポケットにハンカチが入っています。	右[みぎ]のポケットにハンカチが 入[はい]っています。	みぎ の ぽけっと に はんかち が はいって います	
\\	のポケットにハンカチが 入[はい]っています。			
\\	左	左[ひだり]	ひだり	
\\	そこを左に曲がってください。	そこを 左[ひだり]に 曲[ま]がってください。	そこ を ひだり に まがって ください	
\\	そこを
\\	に 曲[ま]がってください。			
\\	方	方[ほう]	ほう	
\\	彼は私の方を見ました。	彼[かれ]は 私[わたし]の 方[ほう]を 見[み]ました。	かれ は わたし の ほう を みました	
\\	彼[かれ]は 私[わたし]の
\\	を 見[み]ました。			
\\	大人	大人[おとな]	おとな	
\\	お酒は大人になってから。	お 酒[さけ]は 大人[おとな]になってから。	おさけ は おとな に なって から	
\\	お 酒[さけ]は
\\	になってから。			
\\	本	本[ほん]	ほん	
\\	本を1冊買いました。	本[ほん]を 1冊買[いっさつ か]いました。	ほん を いっさつ かいました	
\\	を 1冊買[いっさつ か]いました。			
\\	口	口[くち]	くち	
\\	口を大きく開けてください。	口[くち]を 大[おお]きく 開[あ]けてください。	くち を おおきく あけて ください	
\\	を 大[おお]きく 開[あ]けてください。			
\\	手	手[て]	て	
\\	分かった人は手を上げてください。	分[わ]かった 人[ひと]は 手[て]を 上[あ]げてください。	わかった ひと は て を あげて ください	
\\	分[わ]かった 人[ひと]は
\\	を 上[あ]げてください。			
\\	女	女[おんな]	おんな	
\\	店員は若い女の人でした。	店員[てんいん]は 若[わか]い 女[おんな]の 人[ひと]でした。	てんいん は わかい おんな の ひと でした	
\\	店員[てんいん]は 若[わか]い
\\	の 人[ひと]でした。			
\\	子供	子供[こども]	こども	
\\	電車で子供が騒いでいた。	電車[でんしゃ]で 子供[こども]が 騒[さわ]いでいた。	でんしゃ で こども が さわいで いた	
\\	電車[でんしゃ]で
\\	が 騒[さわ]いでいた。			
\\	好き	好[す]き	すき	
\\	私はワインが好きです。	私[わたし]はワインが 好[す]きです。	わたし は わいん が すき です	
\\	私[わたし]はワインが
\\	です。			
\\	大好き	大好[だいす]き	だいすき	
\\	私は犬が大好きだ。	私[わたし]は 犬[いぬ]が 大好[だいす]きだ。	わたし は いぬ が だいすき だ	
\\	私[わたし]は 犬[いぬ]が
\\	だ。			
\\	家	家[うち]	うち	
\\	家に遊びに来てください。	家[うち]に 遊[あそ]びに 来[き]てください。	うち に あそび に きて ください	
\\	に 遊[あそ]びに 来[き]てください。			
\\	気	気[き]	き	
\\	彼は意外に気が小さい。	彼[かれ]は 意外[いがい]に 気[き]が 小[ちい]さい。	かれ は いがい に き が ちいさい	
\\	彼[かれ]は 意外[いがい]に
\\	が 小[ちい]さい。			
\\	天気	天気[てんき]	てんき	
\\	今日はいい天気ですね。	今日[きょう]はいい 天気[てんき]ですね。	きょう は いい てんき です ね	
\\	今日[きょう]はいい
\\	ですね。			
\\	晴れる	晴[は]れる	はれる	
\\	明日は晴れるといいですね。	明日[あした]は 晴[は]れるといいですね。	あした は はれる と いい です ね	
\\	明日[あした]は
\\	といいですね。			
\\	昨日	昨日[きのう]	きのう	
\\	昨日、友達に会った。	昨日[きのう]、 友達[ともだち]に 会[あ]った。	きのう ともだち に あった	
\\	、 友達[ともだち]に 会[あ]った。			
\\	時間	時間[じかん]	じかん	
\\	今は時間がありません。	今[いま]は 時間[じかん]がありません。	いま は じかん が ありません	
\\	今[いま]は
\\	がありません。			
\\	安い	安[やす]い	やすい	
\\	この服はとても安かった。	この 服[ふく]はとても 安[やす]かった。	この ふく は とても やすかった	
\\	この 服[ふく]はとても
\\	後	後[あと]	あと	
\\	仕事の後、映画を見た。	仕事[しごと]の 後[あと]、 映画[えいが]を 見[み]た。	しごと の あと えいが を みた	
\\	仕事[しごと]の
\\	、 映画[えいが]を 見[み]た。			
\\	後ろ	後[うし]ろ	うしろ	
\\	後ろを向いて。	後[うし]ろを 向[む]いて。	うしろ を むいて	
\\	を 向[む]いて。			
\\	朝	朝[あさ]	あさ	
\\	気持ちのいい朝です。	気持[きも]ちのいい 朝[あさ]です。	きもち の いい あさ です	
\\	気持[きも]ちのいい
\\	です。			
\\	昼	昼[ひる]	ひる	
\\	私は昼のドラマを毎日見ます。	私[わたし]は 昼[ひる]のドラマを 毎日見[まいにち み]ます。	わたし は ひる の どらま を まいにち みます	
\\	私[わたし]は
\\	のドラマを 毎日見[まいにち み]ます。			
\\	晩	晩[ばん]	ばん	
\\	晩ご飯は食べましたか。	晩[ばん]ご 飯[はん]は 食[た]べましたか。	ばんごはん は たべました か	
\\	ご 飯[はん]は 食[た]べましたか。			
\\	夜	夜[よる]	よる	
\\	きのうの夜は家にいました。	きのうの 夜[よる]は 家[いえ]にいました。	きのう の よる は いえ に いました	
\\	きのうの
\\	は 家[いえ]にいました。			
\\	書く	書[か]く	かく	
\\	彼に手紙を書きました。	彼[かれ]に 手紙[てがみ]を 書[か]きました。	かれ に てがみ を かきました	
\\	彼[かれ]に 手紙[てがみ]を
\\	場合	場合[ばあい]	ばあい	
\\	分からない場合は私に聞いてください。	分[わ]からない 場合[ばあい]は 私[わたし]に 聞[き]いてください。	わからない ばあい は わたし に きいて ください	
\\	分[わ]からない
\\	は 私[わたし]に 聞[き]いてください。			
\\	止める	止[や]める	やめる	
\\	話すのを止めてください。	話[はな]すのを 止[や]めてください。	はなす の を やめて ください	
\\	話[はな]すのを
\\	ください。			
\\	歩く	歩[ある]く	あるく	
\\	駅まで歩きましょう。	駅[えき]まで 歩[ある]きましょう。	えき まで あるきましょう	
\\	駅[えき]まで
\\	広い	広[ひろ]い	ひろい	
\\	彼の家はとても広い。	彼[かれ]の 家[いえ]はとても 広[ひろ]い。	かれ の いえ は とても ひろい	
\\	彼[かれ]の 家[いえ]はとても
\\	国	国[くに]	くに	
\\	私の国について少しお話しましょう。	私[わたし]の 国[くに]について 少[すこ]しお 話[はなし]しましょう。	わたし の くに に ついて すこし おはなし しましょう	
\\	私[わたし]の
\\	について 少[すこ]しお 話[はなし]しましょう。			
\\	未だ	未[ま]だ	まだ	
\\	宿題は未だ終わっていません。	宿題[しゅくだい]は 未[ま]だ 終[お]わっていません。	しゅくだい は まだ おわって いません	
\\	宿題[しゅくだい]は
\\	終[お]わっていません。			
\\	有る	有[あ]る	ある	
\\	私の机の上に書類がたくさん有ります。	私[わたし]の 机[つくえ]の 上[うえ]に 書類[しょるい]がたくさん 有[あ]ります。	わたし の つくえ の うえ に しょるい が たくさん あります	
\\	私[わたし]の 机[つくえ]の 上[うえ]に 書類[しょるい]がたくさん
\\	消す	消[け]す	けす	
\\	昼間は電気を消してください。	昼間[ひるま]は 電気[でんき]を 消[け]してください。	ひるま は でんき を けして ください	
\\	昼間[ひるま]は 電気[でんき]を
\\	ください。			
\\	売る	売[う]る	うる	
\\	彼は家を売った。	彼[かれ]は 家[いえ]を 売[う]った。	かれ は いえ を うった	
\\	彼[かれ]は 家[いえ]を
\\	店	店[みせ]	みせ	
\\	私はこの店によく来ます。	私[わたし]はこの 店[みせ]によく 来[き]ます。	わたし は この みせ に よく きます	
\\	私[わたし]はこの
\\	によく 来[き]ます。			
\\	春	春[はる]	はる	
\\	今年の春は暖かいね。	今年[ことし]の 春[はる]は 暖[あたた]かいね。	ことし の はる は あたたかい ね	
\\	今年[ことし]の
\\	は 暖[あたた]かいね。			
\\	夏	夏[なつ]	なつ	
\\	私は夏が大好き。	私[わたし]は 夏[なつ]が 大好[だいす]き。	わたし は なつ が だいすき	
\\	私[わたし]は
\\	が 大好[だいす]き。			
\\	暑い	暑[あつ]い	あつい	
\\	今日はとても暑い。	今日[きょう]はとても 暑[あつ]い。	きょう は とても あつい	
\\	今日[きょう]はとても
\\	寒い	寒[さむ]い	さむい	
\\	この部屋は寒いです。	この 部屋[へや]は 寒[さむ]いです。	この へや は さむい です	
\\	この 部屋[へや]は
\\	です。			
\\	暖かい	暖[あたた]かい	あたたかい	
\\	このコートはとても暖かい。	このコートはとても 暖[あたた]かい。	この こーと は とても あたたかい。	
\\	このコートはとても
\\	新しい	新[あたら]しい	あたらしい	
\\	彼の車は新しい。	彼[かれ]の 車[くるま]は 新[あたら]しい。	かれ の くるま は あたらしい	
\\	彼[かれ]の 車[くるま]は
\\	古い	古[ふる]い	ふるい	
\\	私は古い車が好きです。	私[わたし]は 古[ふる]い 車[くるま]が 好[す]きです。	わたし は ふるい くるま が すき です	
\\	私[わたし]は
\\	車[くるま]が 好[す]きです。			
\\	悪い	悪[わる]い	わるい	
\\	たばこは体に悪い。	たばこは 体[からだ]に 悪[わる]い。	たばこ は からだ に わるい	
\\	たばこは 体[からだ]に
\\	思う	思[おも]う	おもう	
\\	私もそう思います。	私[わたし]もそう 思[おも]います。	わたし も そう おもいます	
\\	私[わたし]もそう
\\	忘れる	忘[わす]れる	わすれる	
\\	約束を忘れないでください。	約束[やくそく]を 忘[わす]れないでください。	やくそく を わすれない で ください	
\\	約束[やくそく]を
\\	ください。			
\\	決める	決[き]める	きめる	
\\	帰国することに決めました。	帰国[きこく]することに 決[き]めました。	きこく する こと に きめました	
\\	帰国[きこく]することに
\\	決まる	決[き]まる	きまる	
\\	旅行の日程が決まりました。	旅行[りょこう]の 日程[にってい]が 決[き]まりました。	りょこう の にってい が きまりました	
\\	旅行[りょこう]の 日程[にってい]が
\\	名前	名前[なまえ]	なまえ	
\\	あなたの名前を教えてください。	あなたの 名前[なまえ]を 教[おし]えてください。	あなた の なまえ を おしえて ください	
\\	あなたの
\\	を 教[おし]えてください。			
\\	待つ	待[ま]つ	まつ	
\\	あなたが来るのを待っています。	あなたが 来[く]るのを 待[ま]っています。	あなた が くる の を まって います	
\\	あなたが 来[く]るのを
\\	持つ	持[も]つ	もつ	
\\	私は車を持っています。	私[わたし]は 車[くるま]を 持[も]っています。	わたし は くるま を もって います	
\\	私[わたし]は 車[くるま]を
\\	気持ち	気持[きも]ち	きもち	
\\	彼の気持ちが分からない。	彼[かれ]の 気持[きも]ちが 分[わ]からない。	かれ の きもち が わからない	
\\	彼[かれ]の
\\	が 分[わ]からない。			
\\	大学	大学[だいがく]	だいがく	
\\	大学に行ってもっと勉強したいです。	大学[だいがく]に 行[い]ってもっと 勉強[べんきょう]したいです。	だいがく に いって もっと べんきょう したい です	
\\	に 行[い]ってもっと 勉強[べんきょう]したいです。			
\\	学生	学生[がくせい]	がくせい	
\\	彼は真面目な学生です。	彼[かれ]は 真面目[まじめ]な 学生[がくせい]です。	かれ は まじめ な がくせい です	
\\	彼[かれ]は 真面目[まじめ]な
\\	です。			
\\	大学生	大学生[だいがくせい]	だいがくせい	
\\	姉は大学生です。	姉[あね]は 大学生[だいがくせい]です。	あね は だいがくせい です	
\\	姉[あね]は
\\	です。			
\\	学校	学校[がっこう]	がっこう	
\\	学校は8時半に始まります。	学校[がっこう]は 8時半[はちじはん]に 始[はじ]まります。	がっこう は はちじはん に はじまります	
\\	は 8時半[はちじはん]に 始[はじ]まります。			
\\	教える	教[おし]える	おしえる	
\\	彼は数学を教えています。	彼[かれ]は 数学[すうがく]を 教[おし]えています。	かれ は すうがく を おしえて います	
\\	彼[かれ]は 数学[すうがく]を
\\	強い	強[つよ]い	つよい	
\\	今日は風が強い。	今日[きょう]は 風[かぜ]が 強[つよ]い。	きょう は かぜ が つよい	
\\	今日[きょう]は 風[かぜ]が
\\	弱い	弱[よわ]い	よわい	
\\	その子は体が少し弱い。	その 子[こ]は 体[からだ]が 少[すこ]し 弱[よわ]い。	その こ は からだ が すこし よわい	
\\	その 子[こ]は 体[からだ]が 少[すこ]し
\\	引く	引[ひ]く	ひく	
\\	このドアは引いてください。	このドアは 引[ひ]いてください。	この どあ は ひいて ください	
\\	このドアは
\\	ください。			
\\	数	数[かず]	かず	
\\	グラスの数が足りません。	グラスの 数[かず]が 足[た]りません。	ぐらす の かず が たりません	
\\	グラスの
\\	が 足[た]りません。			
\\	本当に	本当[ほんとう]に	ほんとうに	
\\	あなたが本当に好きです。	あなたが 本当[ほんとう]に 好[す]きです。	あなた が ほんとうに すき です	
\\	あなたが
\\	好[す]きです。			
\\	時計	時計[とけい]	とけい	
\\	時計を見たらちょうど3時だった。	時計[とけい]を 見[み]たらちょうど 3時[さんじ]だった。	とけい を みたら ちょうど さんじ だった	
\\	を 見[み]たらちょうど 3時[さんじ]だった。			
\\	払う	払[はら]う	はらう	
\\	私が払いましょう。	私[わたし]が 払[はら]いましょう。	わたし が はらいましょう	
\\	私[わたし]が
\\	変える	変[か]える	かえる	
\\	旅行の日程を変えました。	旅行[りょこう]の 日程[にってい]を 変[か]えました。	りょこう の にってい を かえました	
\\	旅行[りょこう]の 日程[にってい]を
\\	座る	座[すわ]る	すわる	
\\	私は窓側の席に座った。	私[わたし]は 窓側[まどがわ]の 席[せき]に 座[すわ]った。	わたし は まどがわ の せき に すわった	
\\	私[わたし]は 窓側[まどがわ]の 席[せき]に
\\	次	次[つぎ]	つぎ	
\\	次はいつ会いましょうか。	次[つぎ]はいつ 会[あ]いましょうか。	つぎ は いつ あいましょう か	
\\	はいつ 会[あ]いましょうか。			
\\	早い	早[はや]い	はやい	
\\	まだ学校へ行くには早い時間です。	まだ 学校[がっこう]へ 行[い]くには 早[はや]い 時間[じかん]です。	まだ がっこう へ いく に は はやい じかん です	
\\	まだ 学校[がっこう]へ 行[い]くには
\\	時間[じかん]です。			
\\	始める	始[はじ]める	はじめる	
\\	テストを始めてください。	テストを 始[はじ]めてください。	てすと を はじめて ください	
\\	テストを
\\	ください。			
\\	始まる	始[はじ]まる	はじまる	
\\	新しい仕事が始まりました。	新[あたら]しい 仕事[しごと]が 始[はじ]まりました。	あたらしい しごと が はじまりました	
\\	新[あたら]しい 仕事[しごと]が
\\	楽しむ	楽[たの]しむ	たのしむ	
\\	今日は一人の時間を楽しみたい。	今日[きょう]は 一人[ひとり]の 時間[じかん]を 楽[たの]しみたい。	きょう は ひとり の じかん を たのしみたい	
\\	今日[きょう]は 一人[ひとり]の 時間[じかん]を
\\	楽しい	楽[たの]しい	たのしい	
\\	彼はとても楽しい人です。	彼[かれ]はとても 楽[たの]しい 人[ひと]です。	かれ は とても たのしい ひと です	
\\	彼[かれ]はとても
\\	人[ひと]です。			
\\	歌う	歌[うた]う	うたう	
\\	私たちは大きな声で歌いました。	私[わたし]たちは 大[おお]きな 声[こえ]で 歌[うた]いました。	わたしたち は おおき な こえ で うたいました	
\\	私[わたし]たちは 大[おお]きな 声[こえ]で
\\	欲しい	欲[ほ]しい	ほしい	
\\	僕は新しい靴が欲しいです。	僕[ぼく]は 新[あたら]しい 靴[くつ]が 欲[ほ]しいです。	ぼく は あたらしい くつ が ほしい です	
\\	僕[ぼく]は 新[あたら]しい 靴[くつ]が
\\	です。			
\\	書き直す	書[か]き 直[なお]す	かきなおす	
\\	この書類を書き直してください。	この 書類[しょるい]を 書[か]き 直[なお]してください。	この しょるい を かきなおして ください	
\\	この 書類[しょるい]を
\\	ください。			
\\	曲がる	曲[ま]がる	まがる	
\\	そこを左に曲がってください。	そこを 左[ひだり]に 曲[ま]がってください。	そこ を ひだり に まがって ください	
\\	そこを 左[ひだり]に
\\	ください。			
\\	同じ	同[おな]じ	おなじ	
\\	彼の日本語のレベルは私と同じ位だ。	彼[かれ]の 日本語[にほんご]のレベルは 私[わたし]と 同[おな]じ 位[くらい]だ。	かれ の にほんご の れべる は わたし と おなじ くらい だ	
\\	彼[かれ]の 日本語[にほんご]のレベルは 私[わたし]と
\\	位[くらい]だ。			
\\	図書館	図書館[としょかん]	としょかん	
\\	図書館で料理の本を借りた。	図書館[としょかん]で 料理[りょうり]の 本[ほん]を 借[か]りた。	としょかん で りょうり の ほん を かりた	
\\	で 料理[りょうり]の 本[ほん]を 借[か]りた。			
\\	泊まる	泊[と]まる	とまる	
\\	今日はこのホテルに泊まります。	今日[きょう]はこのホテルに 泊[と]まります。	きょう は この ほてる に とまります	
\\	今日[きょう]はこのホテルに
\\	服	服[ふく]	ふく	
\\	昨日、新しい服を買った。	昨日[きのう]、 新[あたら]しい 服[ふく]を 買[か]った。	きのう あたらしい ふく を かった	
\\	昨日[きのう]、 新[あたら]しい
\\	を 買[か]った。			
\\	母	母[はは]	はは	
\\	昨日、母と話をしました。	昨日[きのう]、 母[はは]と 話[はなし]をしました。	きのう はは と はなし を しました	
\\	昨日[きのう]、
\\	と 話[はなし]をしました。			
\\	姉	姉[あね]	あね	
\\	姉は大学生です。	姉[あね]は 大学生[だいがくせい]です。	あね は だいがくせい です	
\\	は 大学生[だいがくせい]です。			
\\	妹	妹[いもうと]	いもうと	
\\	私の妹は小学生です。	私[わたし]の 妹[いもうと]は 小学生[しょうがくせい]です。	わたし の いもうと は しょうがくせい です	
\\	私[わたし]の
\\	は 小学生[しょうがくせい]です。			
\\	弟	弟[おとうと]	おとうと	
\\	弟は野球が好きです。	弟[おとうと]は 野球[やきゅう]が 好[す]きです。	おとうと は やきゅう が すき です	
\\	は 野球[やきゅう]が 好[す]きです。			
\\	娘	娘[むすめ]	むすめ	
\\	私の娘はアメリカにいます。	私[わたし]の 娘[むすめ]はアメリカにいます。	わたし の むすめ は あめりか に います	
\\	私[わたし]の
\\	はアメリカにいます。			
\\	息子	息子[むすこ]	むすこ	
\\	うちの息子は大学1年生です。	うちの 息子[むすこ]は 大学1年生[だいがく いちねんせい]です。	うち の むすこ は だいがく いちねんせい です	
\\	うちの
\\	は 大学1年生[だいがく いちねんせい]です。			
\\	彼女	彼女[かのじょ]	かのじょ	
\\	彼女は
\\	です。	彼女[かのじょ]は 
\\	[おーえる]です。	かのじょ は おーえる です	
\\	は 
\\	[おーえる]です。			
\\	彼	彼[かれ]	かれ	
\\	彼は私の上司です。	彼[かれ]は 私[わたし]の 上司[じょうし]です。	かれ は わたし の じょうし です	
\\	は 私[わたし]の 上司[じょうし]です。			
\\	死ぬ	死[し]ぬ	しぬ	
\\	犬が病気で死にました。	犬[いぬ]が 病気[びょうき]で 死[し]にました。	いぬ が びょうき で しにました。	
\\	犬[いぬ]が 病気[びょうき]で
\\	日記	日記[にっき]	にっき	
\\	私は毎日、日記を付けています。	私[わたし]は 毎日[まいにち]、 日記[にっき]を 付[つ]けています。	わたし は まいにち にっき を つけて います	
\\	私[わたし]は 毎日[まいにち]、
\\	を 付[つ]けています。			
\\	感じる	感[かん]じる	かんじる	
\\	膝に痛みを感じます。	膝[ひざ]に 痛[いた]みを 感[かん]じます。	ひざ に いたみ を かんじます	
\\	膝[ひざ]に 痛[いた]みを
\\	探す	探[さが]す	さがす	
\\	彼は郵便局を探していました。	彼[かれ]は 郵便局[ゆうびんきょく]を 探[さが]していました。	かれ は ゆうびんきょく を さがして いました	
\\	彼[かれ]は 郵便局[ゆうびんきょく]を
\\	汚い	汚[きたな]い	きたない	
\\	彼の部屋はとても汚い。	彼[かれ]の 部屋[へや]はとても 汚[きたな]い。	かれ の へや は とても きたない	
\\	彼[かれ]の 部屋[へや]はとても
\\	太い	太[ふと]い	ふとい	
\\	彼女は足が太い。	彼女[かのじょ]は 足[あし]が 太[ふと]い。	かのじょ は あし が ふとい	
\\	彼女[かのじょ]は 足[あし]が
\\	曇る	曇[くも]る	くもる	
\\	明日は昼頃から曇るでしょう。	明日[あす]は 昼頃[ひるごろ]から 曇[くも]るでしょう。	あす は ひるごろ から くもる でしょう	
\\	明日[あす]は 昼頃[ひるごろ]から
\\	でしょう。			
\\	建てる	建[た]てる	たてる	
\\	私たちは来年、家を建てます。	私[わたし]たちは 来年[らいねん]、 家[いえ]を 建[た]てます。	わたしたち は らいねん いえ を たてます	
\\	私[わたし]たちは 来年[らいねん]、 家[いえ]を
\\	易しい	易[やさ]しい	やさしい	
\\	この問題はかなり易しいです。	この 問題[もんだい]はかなり 易[やさ]しいです。	この もんだい は かなり やさしい です	
\\	この 問題[もんだい]はかなり
\\	です。			
\\	戻る	戻[もど]る	もどる	
\\	今、会社に戻ります。	今[いま]、 会社[かいしゃ]に 戻[もど]ります。	いま かいしゃ に もどります	
\\	今[いま]、 会社[かいしゃ]に
\\	寝る	寝[ね]る	ねる	
\\	もう寝よう。	もう 寝[ね]よう。	もう ねよう	
\\	もう
\\	夫	夫[おっと]	おっと	
\\	私の夫はサラリーマンです。	私[わたし]の 夫[おっと]はサラリーマンです。	わたし の おっと は さらりーまん です	
\\	私[わたし]の
\\	はサラリーマンです。			
\\	妻	妻[つま]	つま	
\\	今日は妻の誕生日だ。	今日[きょう]は 妻[つま]の 誕生日[たんじょうび]だ。	きょう は つま の たんじょうび だ	
\\	今日[きょう]は
\\	の 誕生日[たんじょうび]だ。			
\\	愛する	愛[あい]する	あいする	
\\	私は家族を愛しています。	私[わたし]は 家族[かぞく]を 愛[あい]しています。	わたし は かぞく を あいして います	
\\	私[わたし]は 家族[かぞく]を
\\	呼ぶ	呼[よ]ぶ	よぶ	
\\	ウェイターを呼びましょう。	ウェイターを 呼[よ]びましょう。	うぇいたー を よびましょう	
\\	ウェイターを
\\	平仮名	平仮名[ひらがな]	ひらがな	
\\	私は平仮名を全部読めます。	私[わたし]は 平仮名[ひらがな]を 全部読[ぜんぶ よ]めます。	わたし は ひらがな を ぜんぶ よめます	
\\	私[わたし]は
\\	を 全部読[ぜんぶ よ]めます。			
\\	悲しい	悲[かな]しい	かなしい	
\\	その映画はとても悲しかった。	その 映画[えいが]はとても 悲[かな]しかった。	その えいが は とても かなしかった	
\\	その 映画[えいが]はとても
\\	授業	授業[じゅぎょう]	じゅぎょう	
\\	今日は日本語の授業があります。	今日[きょう]は 日本語[にほんご]の 授業[じゅぎょう]があります。	きょう は にほんご の じゅぎょう が あります	
\\	今日[きょう]は 日本語[にほんご]の
\\	があります。			
\\	手伝う	手伝[てつだ]う	てつだう	
\\	私が手伝いましょう。	私[わたし]が 手伝[てつだ]いましょう。	わたし が てつだいましょう	
\\	私[わたし]が
\\	嫌い	嫌[きら]い	きらい	
\\	私はタバコが嫌いです。	私[わたし]はタバコが 嫌[きら]いです。	わたし は たばこ が きらい です	
\\	私[わたし]はタバコが
\\	です。			
\\	浴びる	浴[あ]びる	あびる	
\\	私は朝、シャワーを浴びます。	私[わたし]は 朝[あさ]、シャワーを 浴[あ]びます。	わたし は あさ しゃわー を あびます	
\\	私[わたし]は 朝[あさ]、シャワーを
\\	掛ける	掛[か]ける	かける	
\\	夫の服をハンガーに掛けた。	夫[おっと]の 服[ふく]をハンガーに 掛[か]けた。	おっと の ふく を はんがー に かけた	
\\	夫[おっと]の 服[ふく]をハンガーに
\\	大丈夫	大丈夫[だいじょうぶ]	だいじょうぶ	
\\	大丈夫ですか。	大丈夫[だいじょうぶ]ですか。	だいじょうぶ です か	
\\	ですか。			
\\	奇麗	奇麗[きれい]	きれい	
\\	彼女はとても奇麗だ。	彼女[かのじょ]はとても 奇麗[きれい]だ。	かのじょ は とても きれい だ	
\\	彼女[かのじょ]はとても
\\	だ。			
\\	嬉しい	嬉[うれ]しい	うれしい	
\\	彼に会えて嬉しかった。	彼[かれ]に 会[あ]えて 嬉[うれ]しかった。	かれ に あえて うれしかった	
\\	彼[かれ]に 会[あ]えて
\\	寺	寺[てら]	てら	
\\	あそこに古いお寺があります。	あそこに 古[ふる]いお 寺[てら]があります。	あそこ に ふるい おてら が あります	
\\	あそこに 古[ふる]いお
\\	があります。			
\\	日	日[ひ]	ひ	
\\	夏は日が長い。	夏[なつ]は 日[ひ]が 長[なが]い。	なつ は ひ が ながい	
\\	夏[なつ]は
\\	が 長[なが]い。			
\\	木	木[き]	き	
\\	台風で木が倒れた。	台風[たいふう]で 木[き]が 倒[たお]れた。	たいふう で き が たおれた	
\\	台風[たいふう]で
\\	が 倒[たお]れた。			
\\	来月	来月[らいげつ]	らいげつ	
\\	来月から大学生になります。	来月[らいげつ]から 大学生[だいがくせい]になります。	らいげつ から だいがくせい に なります	
\\	から 大学生[だいがくせい]になります。			
\\	来週	来週[らいしゅう]	らいしゅう	
\\	続きは来週やりましょう。	続[つづ]きは 来週[らいしゅう]やりましょう。	つづき は らいしゅう やりましょう	
\\	続[つづ]きは
\\	やりましょう。			
\\	帰り	帰[かえ]り	かえり	
\\	仕事の帰りにビールを飲んだ。	仕事[しごと]の 帰[かえ]りにビールを 飲[の]んだ。	しごと の かえり に びーる を のんだ	
\\	仕事[しごと]の
\\	にビールを 飲[の]んだ。			
\\	大きさ	大[おお]きさ	おおきさ	
\\	この大きさの封筒が欲しいのですが。	この 大[おお]きさの 封筒[ふうとう]が 欲[ほ]しいのですが。	この おおきさ の ふうとう が ほしい の です が	
\\	この
\\	の 封筒[ふうとう]が 欲[ほ]しいのですが。			
\\	大分	大分[だいぶ]	だいぶ	
\\	大分ピアノが上手くなりました。	大分[だいぶ]ピアノが 上手[うま]くなりました。	だいぶ ぴあの が うまく なりました	
\\	ピアノが 上手[うま]くなりました。			
\\	少年	少年[しょうねん]	しょうねん	
\\	少年たちがサッカーをしている。	少年[しょうねん]たちがサッカーをしている。	しょうねんたち が さっかー を して いる	
\\	たちがサッカーをしている。			
\\	少しも	少[すこ]しも	すこしも	
\\	あなたは少しも悪くない。	あなたは 少[すこ]しも 悪[わる]くない。	あなた は すこしも わるく ない	
\\	あなたは
\\	悪[わる]くない。			
\\	少々	少々[しょうしょう]	しょうしょう	
\\	塩を少々入れてください。	塩[しお]を 少々[しょうしょう] 入[い]れてください。	しお を しょうしょう いれて ください	
\\	塩[しお]を
\\	入[い]れてください。			
\\	多く	多[おお]く	おおく	
\\	毎年多くの人が海外へ旅行する。	毎年[まいとし] 多[おお]くの 人[ひと]が 海外[かいがい]へ 旅行[りょこう]する。	まいとし おおく の ひと が かいがい へ りょこう する	
\\	毎年[まいとし]
\\	の 人[ひと]が 海外[かいがい]へ 旅行[りょこう]する。			
\\	年上	年上[としうえ]	としうえ	
\\	彼は私より年上です。	彼[かれ]は 私[わたし]より 年上[としうえ]です。	かれ は わたし より としうえ です	
\\	彼[かれ]は 私[わたし]より
\\	です。			
\\	年下	年下[としした]	としした	
\\	彼は奥さんより年下です。	彼[かれ]は 奥[おく]さんより 年下[としした]です。	かれ は おくさん より としした です	
\\	彼[かれ]は 奥[おく]さんより
\\	です。			
\\	方	方[かた]	かた	
\\	次の方、どうぞ。	次[つぎ]の 方[かた]、どうぞ。	つぎ の かた どうぞ	
\\	次[つぎ]の
\\	、どうぞ。			
\\	大人しい	大人[おとな]しい	おとなしい	
\\	私の彼女はとても大人しいです。	私[わたし]の 彼女[かのじょ]はとても 大人[おとな]しいです。	わたし の かのじょ は とても おとなしい です	
\\	私[わたし]の 彼女[かのじょ]はとても
\\	です。			
\\	外人	外人[がいじん]	がいじん	
\\	この町には外人が少ない。	この 町[まち]には 外人[がいじん]が 少[すく]ない。	この まち に は がいじん が すくない	
\\	この 町[まち]には
\\	が 少[すく]ない。			
\\	外	外[そと]	そと	
\\	外は暑いよ。	外[そと]は 暑[あつ]いよ。	そと は あついよ	
\\	は 暑[あつ]いよ。			
\\	右手	右手[みぎて]	みぎて	
\\	私は右手で字を書きます。	私[わたし]は 右手[みぎて]で 字[じ]を 書[か]きます。	わたし は みぎて で じ を かきます	
\\	私[わたし]は
\\	で 字[じ]を 書[か]きます。			
\\	左手	左手[ひだりて]	ひだりて	
\\	彼女は左手で字を書く。	彼女[かのじょ]は 左手[ひだりて]で 字[じ]を 書[か]く。	かのじょ は ひだりて で じ を かく	
\\	彼女[かのじょ]は
\\	で 字[じ]を 書[か]く。			
\\	山	山[やま]	やま	
\\	山の空気はきれいだ。	山[やま]の 空気[くうき]はきれいだ。	やま の くうき は きれい だ	
\\	の 空気[くうき]はきれいだ。			
\\	川	川[かわ]	かわ	
\\	小さな川を渡りました。	小[ちい]さな 川[かわ]を 渡[わた]りました。	ちいさ な かわ を わたりました	
\\	小[ちい]さな
\\	を 渡[わた]りました。			
\\	海外	海外[かいがい]	かいがい	
\\	彼は海外での生活が長いです。	彼[かれ]は 海外[かいがい]での 生活[せいかつ]が 長[なが]いです。	かれ は かいがい で の せいかつ が ながい です	
\\	彼[かれ]は
\\	での 生活[せいかつ]が 長[なが]いです。			
\\	海	海[うみ]	うみ	
\\	海は広くて大きい。	海[うみ]は 広[ひろ]くて 大[おお]きい。	うみ は ひろく て おおきい	
\\	は 広[ひろ]くて 大[おお]きい。			
\\	毎日	毎日[まいにち]	まいにち	
\\	私たちは毎日散歩をします。	私[わたし]たちは 毎日[まいにち] 散歩[さんぽ]をします。	わたしたち は まいにち さんぽ を します	
\\	私[わたし]たちは
\\	散歩[さんぽ]をします。			
\\	毎年	毎年[まいとし]	まいとし	
\\	私は毎年、海外旅行に行きます。	私[わたし]は 毎年[まいとし]、 海外旅行[かいがい りょこう]に 行[い]きます。	わたし は まいとし かいがい りょこう に いきます	
\\	私[わたし]は
\\	、 海外旅行[かいがい りょこう]に 行[い]きます。			
\\	毎年	毎年[まいねん]	まいねん	
\\	毎年給料が上がる。	毎年[まいねん] 給料[きゅうりょう]が 上[あ]がる。	まいねん きゅうりょう が あがる	
\\	給料[きゅうりょう]が 上[あ]がる。			
\\	毎週	毎週[まいしゅう]	まいしゅう	
\\	私は毎週母に電話をします。	私[わたし]は 毎週[まいしゅう] 母[はは]に 電話[でんわ]をします。	わたし は まいしゅう はは に でんわ を します	
\\	私[わたし]は
\\	母[はは]に 電話[でんわ]をします。			
\\	毎月	毎月[まいつき]	まいつき	
\\	私は毎月貯金をしています。	私[わたし]は 毎月[まいつき] 貯金[ちょきん]をしています。	わたし は まいつき ちょきん を して います	
\\	私[わたし]は
\\	貯金[ちょきん]をしています。			
\\	林	林[はやし]	はやし	
\\	私たちは林の中に入っていった。	私[わたし]たちは 林[はやし]の 中[なか]に 入[はい]っていった。	わたしたち は はやし の なか に はいって いった	
\\	私[わたし]たちは
\\	の 中[なか]に 入[はい]っていった。			
\\	森	森[もり]	もり	
\\	私は森を歩くのが好きです。	私[わたし]は 森[もり]を 歩[ある]くのが 好[す]きです。	わたし は もり を あるく の が すき です	
\\	私[わたし]は
\\	を 歩[ある]くのが 好[す]きです。			
\\	子	子[こ]	こ	
\\	その子は日本語が分からない。	その 子[こ]は 日本語[にほんご]が 分[わ]からない。	その こ は にほんご が わからない	
\\	その
\\	は 日本語[にほんご]が 分[わ]からない。			
\\	女の子	女[おんな]の 子[こ]	おんなのこ	
\\	あの女の子を知っていますか。	あの 女[おんな]の 子[こ]を 知[し]っていますか。	あの おんなのこ を しって います か	
\\	あの
\\	を 知[し]っていますか。			
\\	家	家[いえ]	いえ	
\\	ここが私の家です。	ここが 私[わたし]の 家[いえ]です。	ここ が わたし の いえ です	
\\	ここが 私[わたし]の
\\	です。			
\\	家内	家内[かない]	かない	
\\	家内は九州出身です。	家内[かない]は 九州出身[きゅうしゅう しゅっしん]です。	かない は きゅうしゅう しゅっしん です	
\\	は 九州出身[きゅうしゅう しゅっしん]です。			
\\	客	客[きゃく]	きゃく	
\\	その店は若い客が多いです。	その 店[みせ]は 若[わか]い 客[きゃく]が 多[おお]いです。	その みせ は わかい きゃく が おおい です	
\\	その 店[みせ]は 若[わか]い
\\	が 多[おお]いです。			
\\	気に入る	気[き]に 入[い]る	きにいる	
\\	新しい靴がとても気に入りました。	新[あたら]しい 靴[くつ]がとても 気[き]に 入[い]りました。	あたらしい くつ が とても きにいりました	
\\	新[あたら]しい 靴[くつ]がとても
\\	晴れ	晴[は]れ	はれ	
\\	明日の天気は晴れです。	明日[あす]の 天気[てんき]は 晴[は]れです。	あす の てんき は はれ です	
\\	明日[あす]の 天気[てんき]は
\\	です。			
\\	明らか	明[あき]らか	あきらか	
\\	明らかに彼が悪い。	明[あき]らかに 彼[かれ]が 悪[わる]い。	あきらか に かれ が わるい	
\\	に 彼[かれ]が 悪[わる]い。			
\\	明るい	明[あか]るい	あかるい	
\\	彼女は明るい性格です。	彼女[かのじょ]は 明[あか]るい 性格[せいかく]です。	かのじょ は あかるい せいかく です	
\\	彼女[かのじょ]は
\\	性格[せいかく]です。			
\\	明日	明日[あした]	あした	
\\	明日、会社を休みます。	明日[あした]、 会社[かいしゃ]を 休[やす]みます。	あした かいしゃ を やすみます	
\\	、 会社[かいしゃ]を 休[やす]みます。			
\\	暗い	暗[くら]い	くらい	
\\	東の空が暗いです。	東[ひがし]の 空[そら]が 暗[くら]いです。	ひがし の そら が くらい です	
\\	東[ひがし]の 空[そら]が
\\	です。			
\\	昨年	昨年[さくねん]	さくねん	
\\	昨年は地震が多い年でした。	昨年[さくねん]は 地震[じしん]が 多[おお]い 年[とし]でした。	さくねん は じしん が おおい とし でした	
\\	は 地震[じしん]が 多[おお]い 年[とし]でした。			
\\	東	東[ひがし]	ひがし	
\\	東の空が暗いです。	東[ひがし]の 空[そら]が 暗[くら]いです。	ひがし の そら が くらい です	
\\	の 空[そら]が 暗[くら]いです。			
\\	方向	方向[ほうこう]	ほうこう	
\\	あの人たちは皆、同じ方向を見ている。	あの 人[ひと]たちは 皆[みな]、 同[おな]じ 方向[ほうこう]を 見[み]ている。	あの ひとたち は みな おなじ ほうこう を みて いる	
\\	あの 人[ひと]たちは 皆[みな]、 同[おな]じ
\\	を 見[み]ている。			
\\	向かう	向[む]かう	むかう	
\\	今、会社に向かっています。	今[いま]、 会社[かいしゃ]に 向[む]かっています。	いま かいしゃ に むかって います	
\\	今[いま]、 会社[かいしゃ]に
\\	向こう	向[む]こう	むこう	
\\	友達は向こうにいます。	友達[ともだち]は 向[む]こうにいます。	ともだち は むこう に います	
\\	友達[ともだち]は
\\	にいます。			
\\	向く	向[む]く	むく	
\\	こっちを向いてください。	こっちを 向[む]いてください。	こっち を むいて ください	
\\	こっちを
\\	ください。			
\\	年間	年間[ねんかん]	ねんかん	
\\	年間5万人がここを訪れます。	年間[ねんかん] 5万人[ごまんにん]がここを 訪[おとず]れます。	ねんかん ごまんにん が ここ を おとずれます	
\\	5万人[ごまんにん]がここを 訪[おとず]れます。			
\\	最大	最大[さいだい]	さいだい	
\\	これは世界最大の船です。	これは 世界[せかい] 最大[さいだい]の 船[ふね]です。	これ は せかい さいだい の ふね です	
\\	これは 世界[せかい]
\\	の 船[ふね]です。			
\\	最初	最初[さいしょ]	さいしょ	
\\	5ページの最初を見てください。	5[ご]ページの 最初[さいしょ]を 見[み]てください。	ごぺーじ の さいしょ を みて ください	
\\	5[ご]ページの
\\	を 見[み]てください。			
\\	後	後[のち]	のち	
\\	後に彼は総理大臣になりました。	後[のち]に 彼[かれ]は 総理大臣[そうり だいじん]になりました。	のち に かれ は そうり だいじん に なりました	
\\	に 彼[かれ]は 総理大臣[そうり だいじん]になりました。			
\\	最後	最後[さいご]	さいご	
\\	今日が夏休み最後の日だ。	今日[きょう]が 夏休[なつやす]み 最後[さいご]の 日[ひ]だ。	きょう が なつやすみ さいご の ひ だ	
\\	今日[きょう]が 夏休[なつやす]み
\\	の 日[ひ]だ。			
\\	明後日	明後日[あさって]	あさって	
\\	明後日は休日です。	明後日[あさって]は 休日[きゅうじつ]です。	あさって は きゅうじつ です 。	
\\	は 休日[きゅうじつ]です。			
\\	毎朝	毎朝[まいあさ]	まいあさ	
\\	私は毎朝ジョギングをします。	私[わたし]は 毎朝[まいあさ]ジョギングをします。	わたし は まいあさ じょぎんぐ を します	
\\	私[わたし]は
\\	ジョギングをします。			
\\	昼休み	昼休[ひるやす]み	ひるやすみ	
\\	昼休みに公園に行った。	昼休[ひるやす]みに 公園[こうえん]に 行[い]った。	ひるやすみ に こうえん に いった	
\\	に 公園[こうえん]に 行[い]った。			
\\	昼前	昼前[ひるまえ]	ひるまえ	
\\	昼前に会議があった。	昼前[ひるまえ]に 会議[かいぎ]があった。	ひるまえ に かいぎ が あった	
\\	に 会議[かいぎ]があった。			
\\	昼間	昼間[ひるま]	ひるま	
\\	昼間は仕事で忙しいです。	昼間[ひるま]は 仕事[しごと]で 忙[いそが]しいです。	ひるま は しごと で いそがしい です	
\\	は 仕事[しごと]で 忙[いそが]しいです。			
\\	毎晩	毎晩[まいばん]	まいばん	
\\	姉は毎晩日記を書いています。	姉[あね]は 毎晩[まいばん] 日記[にっき]を 書[か]いています。	あね は まいばん にっき を かいて います	
\\	姉[あね]は
\\	日記[にっき]を 書[か]いています。			
\\	昨夜	昨夜[ゆうべ]	ゆうべ	
\\	昨夜、流れ星を見ました。	昨夜[ゆうべ]、 流[なが]れ 星[ぼし]を 見[み]ました。	ゆうべ ながれぼし を みました	
\\	、 流[なが]れ 星[ぼし]を 見[み]ました。			
\\	夜中	夜中[よなか]	よなか	
\\	夜中に電話がありました。	夜中[よなか]に 電話[でんわ]がありました。	よなか に でんわ が ありました	
\\	に 電話[でんわ]がありました。			
\\	夕方	夕方[ゆうがた]	ゆうがた	
\\	夕方そちらに着きます。	夕方[ゆうがた]そちらに 着[つ]きます。	ゆうがた そちら に つきます	
\\	そちらに 着[つ]きます。			
\\	昼食	昼食[ちゅうしょく]	ちゅうしょく	
\\	昼食に寿司を食べました。	昼食[ちゅうしょく]に 寿司[すし]を 食[た]べました。	ちゅうしょく に すし を たべました	
\\	に 寿司[すし]を 食[た]べました。			
\\	朝食	朝食[ちょうしょく]	ちょうしょく	
\\	朝食に納豆を食べました。	朝食[ちょうしょく]に 納豆[なっとう]を 食[た]べました。	ちょうしょく に なっとう を たべました	
\\	に 納豆[なっとう]を 食[た]べました。			
\\	夕食	夕食[ゆうしょく]	ゆうしょく	
\\	夕食は7時です。	夕食[ゆうしょく]は 7時[しちじ]です。	ゆうしょく は しちじ です	
\\	は 7時[しちじ]です。			
\\	夕飯	夕飯[ゆうはん]	ゆうはん	
\\	夕飯は寿司でした。	夕飯[ゆうはん]は 寿司[すし]でした。	ゆうはん は すし でした	
\\	は 寿司[すし]でした。			
\\	文字	文字[もじ]	もじ	
\\	壁に文字が書いてあった。	壁[かべ]に 文字[もじ]が 書[か]いてあった。	かべ に もじ が かいて あった	
\\	壁[かべ]に
\\	が 書[か]いてあった。			
\\	字	字[じ]	じ	
\\	もっと大きく字を書いてください。	もっと 大[おお]きく 字[じ]を 書[か]いてください。	もっと おおきく じ を かいて ください	
\\	もっと 大[おお]きく
\\	を 書[か]いてください。			
\\	書き方	書[か]き 方[かた]	かきかた	
\\	彼はその漢字の書き方が分からない。	彼[かれ]はその 漢字[かんじ]の 書[か]き 方[かた]が 分[わ]からない。	かれ は その かんじ の かきかた が わからない	
\\	彼[かれ]はその 漢字[かんじ]の
\\	が 分[わ]からない。			
\\	大会	大会[たいかい]	たいかい	
\\	夏には川辺で花火大会があります。	夏[なつ]には 川辺[かわべ]で 花火[はなび] 大会[たいかい]があります。	なつ に は かわべ で はなびたいかい が あります	
\\	夏[なつ]には 川辺[かわべ]で 花火[はなび]
\\	があります。			
\\	合う	合[あ]う	あう	
\\	この靴は私の足に合っている。	この 靴[くつ]は 私[わたし]の 足[あし]に 合[あ]っている。	この くつ は わたし の あし に あって いる	
\\	この 靴[くつ]は 私[わたし]の 足[あし]に
\\	大事	大事[だいじ]	だいじ	
\\	お体をお大事に。	お 体[からだ]をお 大事[だいじ]に。	おからだ を おだいじ に	
\\	お 体[からだ]をお
\\	に。			
\\	工事	工事[こうじ]	こうじ	
\\	工事の音がうるさい。	工事[こうじ]の 音[おと]がうるさい。	こうじ の おと が うるさい	
\\	の 音[おと]がうるさい。			
\\	工場	工場[こうじょう]	こうじょう	
\\	彼は食品工場で働いています。	彼[かれ]は 食品[しょくひん] 工場[こうじょう]で 働[はたら]いています。	かれ は しょくひん こうじょう で はたらいて います	
\\	彼[かれ]は 食品[しょくひん]
\\	で 働[はたら]いています。			
\\	水道	水道[すいどう]	すいどう	
\\	東京は水道の水が不味い。	東京[とうきょう]は 水道[すいどう]の 水[みず]が 不味[まず]い。	とうきょう は すいどう の みず が まずい	
\\	東京[とうきょう]は
\\	の 水[みず]が 不味[まず]い。			
\\	土地	土地[とち]	とち	
\\	ここは父の土地です。	ここは 父[ちち]の 土地[とち]です。	ここ は ちち の とち です	
\\	ここは 父[ちち]の
\\	です。			
\\	地図	地図[ちず]	ちず	
\\	地図を見て来てください。	地図[ちず]を 見[み]て 来[き]てください。	ちず を みて きて ください	
\\	を 見[み]て 来[き]てください。			
\\	止める	止[と]める	とめる	
\\	車を止めて。	車[くるま]を 止[と]めて。	くるま を とめて	
\\	車[くるま]を
\\	止まる	止[と]まる	とまる	
\\	今朝、事故で電車が止まりました。	今朝[けさ]、 事故[じこ]で 電車[でんしゃ]が 止[と]まりました。	けさ じこ で でんしゃ が とまりました	
\\	今朝[けさ]、 事故[じこ]で 電車[でんしゃ]が
\\	止む	止[や]む	やむ	
\\	雨が止みました。	雨[あめ]が 止[や]みました。	あめ が やみました	
\\	雨[あめ]が
\\	歩道	歩道[ほどう]	ほどう	
\\	歩道を歩きましょう。	歩道[ほどう]を 歩[ある]きましょう。	ほどう を あるきましょう	
\\	を 歩[ある]きましょう。			
\\	年度	年度[ねんど]	ねんど	
\\	売り上げは年度によって違います。	売[う]り 上[あ]げは 年度[ねんど]によって 違[ちが]います。	うりあげ は ねんど に よって ちがいます	
\\	売[う]り 上[あ]げは
\\	によって 違[ちが]います。			
\\	最近	最近[さいきん]	さいきん	
\\	それは最近話題の本ですね。	それは 最近[さいきん] 話題[わだい]の 本[ほん]ですね。	それ は さいきん わだい の ほん です ね	
\\	それは
\\	話題[わだい]の 本[ほん]ですね。			
\\	広がる	広[ひろ]がる	ひろがる	
\\	留学してから私の世界が広がった。	留学[りゅうがく]してから 私[わたし]の 世界[せかい]が 広[ひろ]がった。	りゅうがく して から わたし の せかい が ひろがった	
\\	留学[りゅうがく]してから 私[わたし]の 世界[せかい]が
\\	広さ	広[ひろ]さ	ひろさ	
\\	その家の広さはどれ位ですか。	その 家[いえ]の 広[ひろ]さはどれ 位[くらい]ですか。	その いえ の ひろさ は どれ くらい です か	
\\	その 家[いえ]の
\\	はどれ 位[くらい]ですか。			
\\	安全	安全[あんぜん]	あんぜん	
\\	安全が第一です。	安全[あんぜん]が 第一[だいいち]です。	あんぜん が だいいち です	
\\	が 第一[だいいち]です。			
\\	国内	国内[こくない]	こくない	
\\	この携帯電話が使えるのは国内だけです。	この 携帯電話[けいたい でんわ]が 使[つか]えるのは 国内[こくない]だけです。	この けいたい でんわ が つかえる の は こくない だけ です	
\\	この 携帯電話[けいたい でんわ]が 使[つか]えるのは
\\	だけです。			
\\	外国	外国[がいこく]	がいこく	
\\	母はまだ外国に行ったことがありません。	母[はは]はまだ 外国[がいこく]に 行[い]ったことがありません。	はは は まだ がいこく に いった こと が ありません	
\\	母[はは]はまだ
\\	に 行[い]ったことがありません。			
\\	国会	国会[こっかい]	こっかい	
\\	国会が再開した。	国会[こっかい]が 再開[さいかい]した。	こっかい が さいかい した	
\\	が 再開[さいかい]した。			
\\	帰国	帰国[きこく]	きこく	
\\	彼は帰国しました。	彼[かれ]は 帰国[きこく]しました。	かれ は きこく しました	
\\	彼[かれ]は
\\	しました。			
\\	外国人	外国人[がいこくじん]	がいこくじん	
\\	日本に住む外国人が増えています。	日本[にほん]に 住[す]む 外国人[がいこくじん]が 増[ふ]えています。	にほん に すむ がいこくじん が ふえて います	
\\	日本[にほん]に 住[す]む
\\	が 増[ふ]えています。			
\\	外国語	外国語[がいこくご]	がいこくご	
\\	外国語を習うのは難しい。	外国語[がいこくご]を 習[なら]うのは 難[むずか]しい。	がいこくご を ならう の は むずかしい	
\\	を 習[なら]うのは 難[むずか]しい。			
\\	地下鉄	地下鉄[ちかてつ]	ちかてつ	
\\	私は地下鉄で通勤しています。	私[わたし]は 地下鉄[ちかてつ]で 通勤[つうきん]しています。	わたし は ちかてつ で つうきん して います	
\\	私[わたし]は
\\	で 通勤[つうきん]しています。			
\\	本屋	本屋[ほんや]	ほんや	
\\	駅前に本屋があります。	駅前[えきまえ]に 本屋[ほんや]があります。	えきまえ に ほんや が あります	
\\	駅前[えきまえ]に
\\	があります。			
\\	味	味[あじ]	あじ	
\\	この料理は味が薄い。	この 料理[りょうり]は 味[あじ]が 薄[うす]い。	この りょうり は あじ が うすい	
\\	この 料理[りょうり]は
\\	が 薄[うす]い。			
\\	未来	未来[みらい]	みらい	
\\	未来は誰にも分からない。	未来[みらい]は 誰[だれ]にも 分[わ]からない。	みらい は だれ に も わからない	
\\	は 誰[だれ]にも 分[わ]からない。			
\\	料理	料理[りょうり]	りょうり	
\\	母は料理が得意です。	母[はは]は 料理[りょうり]が 得意[とくい]です。	はは は りょうり が とくい です	
\\	母[はは]は
\\	が 得意[とくい]です。			
\\	消える	消[き]える	きえる	
\\	突然、電気が消えた。	突然[とつぜん]、 電気[でんき]が 消[き]えた。	とつぜん でんき が きえた	
\\	突然[とつぜん]、 電気[でんき]が
\\	消しゴム	消[け]しゴム	けしごむ	
\\	消しゴムを貸して下さい。	消[け]しゴムを 貸[か]して 下[くだ]さい。	けしごむ を かして ください	
\\	を 貸[か]して 下[くだ]さい。			
\\	売れる	売[う]れる	うれる	
\\	今年の夏はクーラーがよく売れた。	今年[ことし]の 夏[なつ]はクーラーがよく 売[う]れた。	ことし の なつ は くーらー が よく うれた	
\\	今年[ことし]の 夏[なつ]はクーラーがよく
\\	売り場	売[う]り 場[ば]	うりば	
\\	くつ売り場はどこですか。	くつ 売[う]り 場[ば]はどこですか。	くつうりば は どこ です か	
\\	くつ
\\	はどこですか。			
\\	店員	店員[てんいん]	てんいん	
\\	あの店員はとても親切です。	あの 店員[てんいん]はとても 親切[しんせつ]です。	あの てんいん は とても しんせつ です	
\\	あの
\\	はとても 親切[しんせつ]です。			
\\	売店	売店[ばいてん]	ばいてん	
\\	駅の売店で雑誌を買った。	駅[えき]の 売店[ばいてん]で 雑誌[ざっし]を 買[か]った。	えき の ばいてん で ざっし を かった	
\\	駅[えき]の
\\	で 雑誌[ざっし]を 買[か]った。			
\\	商品	商品[しょうひん]	しょうひん	
\\	この商品はよく売れている。	この 商品[しょうひん]はよく 売[う]れている。	この しょうひん は よく うれて いる	
\\	この
\\	はよく 売[う]れている。			
\\	段階	段階[だんかい]	だんかい	
\\	この段階では、決断するのはまだ早い。	この 段階[だんかい]では、 決断[けつだん]するのはまだ 早[はや]い。	この だんかい で は けつだん する の は まだ はやい	
\\	この
\\	では、 決断[けつだん]するのはまだ 早[はや]い。			
\\	段々	段々[だんだん]	だんだん	
\\	段々仕事が楽しくなってきました。	段々[だんだん] 仕事[しごと]が 楽[たの]しくなってきました。	だんだん しごと が たのしく なって きました	
\\	仕事[しごと]が 楽[たの]しくなってきました。			
\\	合格	合格[ごうかく]	ごうかく	
\\	娘が入学試験に合格しました。	娘[むすめ]が 入学試験[にゅうがく しけん]に 合格[ごうかく]しました。	むすめ が にゅうがく しけん に ごうかく しました	
\\	娘[むすめ]が 入学試験[にゅうがく しけん]に
\\	しました。			
\\	夏休み	夏休[なつやす]み	なつやすみ	
\\	今日が夏休み最後の日だ。	今日[きょう]が 夏休[なつやす]み 最後[さいご]の 日[ひ]だ。	きょう が なつやすみ さいご の ひ だ	
\\	今日[きょう]が
\\	最後[さいご]の 日[ひ]だ。			
\\	四季	四季[しき]	しき	
\\	日本には四季がある。	日本[にほん]には 四季[しき]がある。	にほん に は しき が ある	
\\	日本[にほん]には
\\	がある。			
\\	暑さ	暑[あつ]さ	あつさ	
\\	今年の夏は暑さが厳しい。	今年[ことし]の 夏[なつ]は 暑[あつ]さが 厳[きび]しい。	ことし の なつ は あつさ が きびしい	
\\	今年[ことし]の 夏[なつ]は
\\	が 厳[きび]しい。			
\\	寒さ	寒[さむ]さ	さむさ	
\\	今日は厳しい寒さになるでしょう。	今日[きょう]は 厳[きび]しい 寒[さむ]さになるでしょう。	きょう は きびしい さむさ に なる でしょう	
\\	今日[きょう]は 厳[きび]しい
\\	になるでしょう。			
\\	暖める	暖[あたた]める	あたためる	
\\	今、車を暖めています。	今[いま]、 車[くるま]を 暖[あたた]めています。	いま くるま を あたためています	
\\	今[いま]、 車[くるま]を
\\	暖まる	暖[あたた]まる	あたたまる	
\\	まだ部屋が暖まらない。	まだ 部屋[へや]が 暖[あたた]まらない。	まだ へや が あたたまらない。	
\\	まだ 部屋[へや]が
\\	気温	気温[きおん]	きおん	
\\	今日の気温は26度です。	今日[きょう]の 気温[きおん]は 26度[にじゅうろくど]です。	きょう の きおん は にじゅうろくど です	
\\	今日[きょう]の
\\	は 26度[にじゅうろくど]です。			
\\	台	台[だい]	だい	
\\	そこにちょうど良い台がある。	そこにちょうど 良[い]い 台[だい]がある。	そこ に ちょうど いい だい が ある	
\\	そこにちょうど 良[い]い
\\	がある。			
\\	台風	台風[たいふう]	たいふう	
\\	台風が近づいている。	台風[たいふう]が 近[ちか]づいている。	たいふう が ちかづいて いる	
\\	が 近[ちか]づいている。			
\\	情報	情報[じょうほう]	じょうほう	
\\	学生たちはインターネットでいろいろな情報を集めた。	学生[がくせい]たちはインターネットでいろいろな 情報[じょうほう]を 集[あつ]めた。	がくせいたち は いんたーねっと で いろいろな じょうほう を あつめた	
\\	学生[がくせい]たちはインターネットでいろいろな
\\	を 集[あつ]めた。			
\\	報告	報告[ほうこく]	ほうこく	
\\	昨日の会議について報告があります。	昨日[きのう]の 会議[かいぎ]について 報告[ほうこく]があります。	きのう の かいぎ に ついて ほうこく が あります	
\\	昨日[きのう]の 会議[かいぎ]について
\\	があります。			
\\	新聞	新聞[しんぶん]	しんぶん	
\\	今日の新聞、どこに置いた?	今日[きょう]の 新聞[しんぶん]、どこに 置[お]いた?	きょう の しんぶん どこ に おいた	
\\	今日[きょう]の
\\	、どこに 置[お]いた?			
\\	新年	新年[しんねん]	しんねん	
\\	新年明けましておめでとうございます。	新年[しんねん] 明[あ]けましておめでとうございます。	しんねん あけまして おめでとう ございます	
\\	明[あ]けましておめでとうございます。			
\\	安心	安心[あんしん]	あんしん	
\\	それを聞いて安心しました。	それを 聞[き]いて 安心[あんしん]しました。	それ を きいて あんしん しました	
\\	それを 聞[き]いて
\\	しました。			
\\	思い出す	思[おも]い 出[だ]す	おもいだす	
\\	大切な用事を思い出しました。	大切[たいせつ]な 用事[ようじ]を 思[おも]い 出[だ]しました。	たいせつ な ようじ を おもいだしました	
\\	大切[たいせつ]な 用事[ようじ]を
\\	思い出	思[おも]い 出[で]	おもいで	
\\	旅行で楽しい思い出ができました。	旅行[りょこう]で 楽[たの]しい 思[おも]い 出[で]ができました。	りょこう で たのしい おもいで が できました	
\\	旅行[りょこう]で 楽[たの]しい
\\	ができました。			
\\	可能	可能[かのう]	かのう	
\\	20キロのダイエットは可能だと思いますか。	20[にじゅっ]キロのダイエットは 可能[かのう]だと 思[おも]いますか。	にじゅっきろ の だいえっと は かのう だ と おもいます か	
\\	20[にじゅっ]キロのダイエットは
\\	だと 思[おも]いますか。			
\\	可	可[か]	か	
\\	このアルバイトは「学生可」ですね。	このアルバイトは
\\	学生[がくせい] 可[か]」ですね。	この あるばいと は がくせい か です ね	
\\	このアルバイトは
\\	学生[がくせい]
\\	ですね。			
\\	場所	場所[ばしょ]	ばしょ	
\\	会社の場所を教えてください。	会社[かいしゃ]の 場所[ばしょ]を 教[おし]えてください。	かいしゃ の ばしょ を おしえて ください	
\\	会社[かいしゃ]の
\\	を 教[おし]えてください。			
\\	台所	台所[だいどころ]	だいどころ	
\\	お母さんは台所にいます。	お 母[かあ]さんは 台所[だいどころ]にいます。	おかあさん は だいどころ に います	
\\	お 母[かあ]さんは
\\	にいます。			
\\	有名	有名[ゆうめい]	ゆうめい	
\\	ボルドーはワインの生産で有名だ。	ボルドーはワインの 生産[せいさん]で 有名[ゆうめい]だ。	ぼるどー は わいん の せいさん で ゆうめい だ	
\\	ボルドーはワインの 生産[せいさん]で
\\	だ。			
\\	名字	名字[みょうじ]	みょうじ	
\\	あなたの名字は何ですか。	あなたの 名字[みょうじ]は 何[なん]ですか。	あなた の みょうじ は なん です か	
\\	あなたの
\\	は 何[なん]ですか。			
\\	氏名	氏名[しめい]	しめい	
\\	ここに住所と氏名を書いてください。	ここに 住所[じゅうしょ]と 氏名[しめい]を 書[か]いてください。	ここ に じゅうしょ と しめい を かいて ください	
\\	ここに 住所[じゅうしょ]と
\\	を 書[か]いてください。			
\\	各国	各国[かっこく]	かっこく	
\\	各国の代表がニューヨークに集まった。	各国[かっこく]の 代表[だいひょう]がニューヨークに 集[あつ]まった。	かっこく の だいひょう が にゅーよーく に あつまった	
\\	の 代表[だいひょう]がニューヨークに 集[あつ]まった。			
\\	朝御飯	朝御飯[あさごはん]	あさごはん	
\\	七時に朝御飯を食べました。	七時[しちじ]に 朝御飯[あさごはん]を 食[た]べました。	しちじ に あさごはん を たべました	
\\	七時[しちじ]に
\\	を 食[た]べました。			
\\	品物	品物[しなもの]	しなもの	
\\	その店は色々な品物を売っている。	その 店[みせ]は 色々[いろいろ]な 品物[しなもの]を 売[う]っている。	その みせ は いろいろ な しなもの を うって いる	
\\	その 店[みせ]は 色々[いろいろ]な
\\	を 売[う]っている。			
\\	忘れ物	忘[わす]れ 物[もの]	わすれもの	
\\	学校に忘れ物をしました。	学校[がっこう]に 忘[わす]れ 物[もの]をしました。	がっこう に わすれもの を しました	
\\	学校[がっこう]に
\\	をしました。			
\\	心配	心配[しんぱい]	しんぱい	
\\	明日のプレゼンテーションが心配だ。	明日[あす]のプレゼンテーションが 心配[しんぱい]だ。	あす の ぷれぜんてーしょん が しんぱい だ	
\\	明日[あす]のプレゼンテーションが
\\	だ。			
\\	受ける	受[う]ける	うける	
\\	彼は就職試験を受けた。	彼[かれ]は 就職試験[しゅうしょく しけん]を 受[う]けた。	かれ は しゅうしょく しけん を うけた	
\\	彼[かれ]は 就職試験[しゅうしょく しけん]を
\\	受け取る	受[う]け 取[と]る	うけとる	
\\	彼からメールを受け取りました。	彼[かれ]からメールを 受[う]け 取[と]りました。	かれ から めーる を うけとりました	
\\	彼[かれ]からメールを
\\	書き取る	書[か]き 取[と]る	かきとる	
\\	話しのポイントを書き取った。	話[はな]しのポイントを 書[か]き 取[と]った。	はなし の ぽいんと を かきとった	
\\	話[はな]しのポイントを
\\	届く	届[とど]く	とどく	
\\	昨日、父から手紙が届いた。	昨日[きのう]、 父[ちち]から 手紙[てがみ]が 届[とど]いた。	きのう ちち から てがみ が とどいた	
\\	昨日[きのう]、 父[ちち]から 手紙[てがみ]が
\\	届ける	届[とど]ける	とどける	
\\	これを彼に届けてください。	これを 彼[かれ]に 届[とど]けてください。	これ を かれ に とどけて ください	
\\	これを 彼[かれ]に
\\	ください。			
\\	持つ	持[も]つ	もつ	
\\	この車はよく持っているね。	この 車[くるま]はよく 持[も]っているね。	この くるま は よく もって いる ね	
\\	この 車[くるま]はよく
\\	ね。			
\\	持って行く	持[も]って 行[い]く	もっていく	
\\	水を持って行きましょう。	水[みず]を 持[も]って 行[い]きましょう。	みず を もっていきましょう	
\\	水[みず]を
\\	持って来る	持[も]って 来[く]る	もってくる	
\\	そのいすを持って来てください。	そのいすを 持[も]って 来[き]てください。	その いす を もって きて ください	
\\	そのいすを
\\	ください。			
\\	打つ	打[う]つ	うつ	
\\	転んでひざを打ちました。	転[ころ]んでひざを 打[う]ちました。	ころんで ひざ を うちました	
\\	転[ころ]んでひざを
\\	投げる	投[な]げる	なげる	
\\	ボールをこっちに投げてください。	ボールをこっちに 投[な]げてください。	ぼーる を こっち に なげて ください	
\\	ボールをこっちに
\\	ください。			
\\	女性	女性[じょせい]	じょせい	
\\	そのパーティーに女性は何人来ますか。	そのパーティーに 女性[じょせい]は 何人来[なんにん き]ますか。	その ぱーてぃー に じょせい は なんにん きます か	
\\	そのパーティーに
\\	は 何人来[なんにん き]ますか。			
\\	小学生	小学生[しょうがくせい]	しょうがくせい	
\\	うちの息子は来年、小学生になります。	うちの 息子[むすこ]は 来年[らいねん]、 小学生[しょうがくせい]になります。	うち の むすこ は らいねん しょうがくせい に なります	
\\	うちの 息子[むすこ]は 来年[らいねん]、
\\	になります。			
\\	小学校	小学校[しょうがっこう]	しょうがっこう	
\\	家の近くに小学校があります。	家[いえ]の 近[ちか]くに 小学校[しょうがっこう]があります。	いえ の ちかく に しょうがっこう が あります	
\\	家[いえ]の 近[ちか]くに
\\	があります。			
\\	校長	校長[こうちょう]	こうちょう	
\\	あの人は高校の校長だ。	あの 人[ひと]は 高校[こうこう]の 校長[こうちょう]だ。	あの ひと は こうこう の こうちょう だ	
\\	あの 人[ひと]は 高校[こうこう]の
\\	だ。			
\\	教会	教会[きょうかい]	きょうかい	
\\	私たちは教会で結婚式をしました。	私[わたし]たちは 教会[きょうかい]で 結婚式[けっこんしき]をしました。	わたしたち は きょうかい で けっこんしき を しました	
\\	私[わたし]たちは
\\	で 結婚式[けっこんしき]をしました。			
\\	教育	教育[きょういく]	きょういく	
\\	彼は海外で教育を受けました。	彼[かれ]は 海外[かいがい]で 教育[きょういく]を 受[う]けました。	かれ は かいがい で きょういく を うけました	
\\	彼[かれ]は 海外[かいがい]で
\\	を 受[う]けました。			
\\	強さ	強[つよ]さ	つよさ	
\\	風の強さに驚きました。	風[かぜ]の 強[つよ]さに 驚[おどろ]きました。	かぜ の つよさ に おどろきました	
\\	風[かぜ]の
\\	に 驚[おどろ]きました。			
\\	取引	取引[とりひき]	とりひき	
\\	私たちは中国の会社と取引しています。	私[わたし]たちは 中国[ちゅうごく]の 会社[かいしゃ]と 取引[とりひき]しています。	わたしたち は ちゅうごく の かいしゃ と とりひき して います	
\\	私[わたし]たちは 中国[ちゅうごく]の 会社[かいしゃ]と
\\	しています。			
\\	引き出し	引[ひ]き 出[だ]し	ひきだし	
\\	財布は引き出しの中にあります。	財布[さいふ]は 引[ひ]き 出[だ]しの 中[なか]にあります。	さいふ は ひきだし の なか に あります	
\\	財布[さいふ]は
\\	の 中[なか]にあります。			
\\	押さえる	押[お]さえる	おさえる	
\\	ドアを押さえてください。	ドアを 押[お]さえてください。	どあ を おさえて ください	
\\	ドアを
\\	ください。			
\\	押し入れ	押[お]し 入[い]れ	おしいれ	
\\	布団を押し入れにしまいました。	布団[ふとん]を 押[お]し 入[い]れにしまいました。	ふとん を おしいれ に しまいました	
\\	布団[ふとん]を
\\	にしまいました。			
\\	慣れる	慣[な]れる	なれる	
\\	新しい家にはもう慣れましたか。	新[あたら]しい 家[いえ]にはもう 慣[な]れましたか。	あたらしい いえ に は もう なれました か	
\\	新[あたら]しい 家[いえ]にはもう
\\	か。			
\\	問題	問題[もんだい]	もんだい	
\\	問題が一つあります。	問題[もんだい]が 一[ひと]つあります。	もんだい が ひとつ あります	
\\	が 一[ひと]つあります。			
\\	数字	数字[すうじ]	すうじ	
\\	数字は苦手です。	数字[すうじ]は 苦手[にがて]です。	すうじ は にがて です	
\\	は 苦手[にがて]です。			
\\	数学	数学[すうがく]	すうがく	
\\	兄は数学の先生です。	兄[あに]は 数学[すうがく]の 先生[せんせい]です。	あに は すうがく の せんせい です	
\\	兄[あに]は
\\	の 先生[せんせい]です。			
\\	数える	数[かぞ]える	かぞえる	
\\	いすの数を数えてください。	いすの 数[かず]を 数[かぞ]えてください。	いす の かず を かぞえて ください	
\\	いすの 数[かず]を
\\	ください。			
\\	回る	回[まわ]る	まわる	
\\	月は地球のまわりを回っています。	月[つき]は 地球[ちきゅう]のまわりを 回[まわ]っています。	つき は ちきゅう の まわり を まわって います	
\\	月[つき]は 地球[ちきゅう]のまわりを
\\	回す	回[まわ]す	まわす	
\\	ねじは左に回すと外れます。	ねじは 左[ひだり]に 回[まわ]すと 外[はず]れます。	ねじ は ひだり に まわす と はずれます	
\\	ねじは 左[ひだり]に
\\	と 外[はず]れます。			
\\	担当	担当[たんとう]	たんとう	
\\	私はセールスを担当しています。	私[わたし]はセールスを 担当[たんとう]しています。	わたし は せーるす を たんとう して います	
\\	私[わたし]はセールスを
\\	しています。			
\\	当たる	当[あ]たる	あたる	
\\	ボールが彼の頭に当たった。	ボールが 彼[かれ]の 頭[あたま]に 当[あ]たった。	ぼーる が かれ の あたま に あたった	
\\	ボールが 彼[かれ]の 頭[あたま]に
\\	当時	当時[とうじ]	とうじ	
\\	彼女は当時、まだ3才だった。	彼女[かのじょ]は 当時[とうじ]、まだ 3才[さんさい]だった。	かのじょ は とうじ まだ さんさい だった	
\\	彼女[かのじょ]は
\\	、まだ 3才[さんさい]だった。			
\\	本当	本当[ほんとう]	ほんとう	
\\	その話は本当ですか。	その 話[はなし]は 本当[ほんとう]ですか。	その はなし は ほんとう です か	
\\	その 話[はなし]は
\\	ですか。			
\\	当然	当然[とうぜん]	とうぜん	
\\	彼女が怒るのも当然だ。	彼女[かのじょ]が 怒[おこ]るのも 当然[とうぜん]だ。	かのじょ が おこる の も とうぜん だ	
\\	彼女[かのじょ]が 怒[おこ]るのも
\\	だ。			
\\	方法	方法[ほうほう]	ほうほう	
\\	いい方法を思いつきました。	いい 方法[ほうほう]を 思[おも]いつきました。	いい ほうほう を おもいつきました	
\\	いい
\\	を 思[おも]いつきました。			
\\	法律	法律[ほうりつ]	ほうりつ	
\\	新しい法律ができた。	新[あたら]しい 法律[ほうりつ]ができた。	あたらしい ほうりつ が できた	
\\	新[あたら]しい
\\	ができた。			
\\	株	株[かぶ]	かぶ	
\\	最近株を始めました。	最近[さいきん] 株[かぶ]を 始[はじ]めました。	さいきん かぶ を はじめました	
\\	最近[さいきん]
\\	を 始[はじ]めました。			
\\	工業	工業[こうぎょう]	こうぎょう	
\\	そこは工業都市だ。	そこは 工業[こうぎょう] 都市[とし]だ。	そこ は こうぎょう とし だ	
\\	そこは
\\	都市[とし]だ。			
\\	商業	商業[しょうぎょう]	しょうぎょう	
\\	この町では商業が盛んだ。	この 町[まち]では 商業[しょうぎょう]が 盛[さか]んだ。	この まち で は しょうぎょう が さかん だ	
\\	この 町[まち]では
\\	が 盛[さか]んだ。			
\\	技術	技術[ぎじゅつ]	ぎじゅつ	
\\	彼は非常に高い技術を持っている。	彼[かれ]は 非常[ひじょう]に 高[たか]い 技術[ぎじゅつ]を 持[も]っている。	かれ は ひじょう に たかい ぎじゅつ を もって いる	
\\	彼[かれ]は 非常[ひじょう]に 高[たか]い
\\	を 持[も]っている。			
\\	手術	手術[しゅじゅつ]	しゅじゅつ	
\\	父は胸の手術をした。	父[ちち]は 胸[むね]の 手術[しゅじゅつ]をした。	ちち は むね の しゅじゅつ を した	
\\	父[ちち]は 胸[むね]の
\\	をした。			
\\	必ず	必[かなら]ず	かならず	
\\	必ずシートベルトを着けて下さい。	必[かなら]ずシートベルトを 着[つ]けて 下[くだ]さい。	かならず しーとべると を つけて ください	
\\	シートベルトを 着[つ]けて 下[くだ]さい。			
\\	必要	必要[ひつよう]	ひつよう	
\\	私にはたくさんのお金が必要だ。	私[わたし]にはたくさんのお 金[かね]が 必要[ひつよう]だ。	わたし に は たくさん の おかね が ひつよう だ	
\\	私[わたし]にはたくさんのお 金[かね]が
\\	だ。			
\\	引き算	引[ひ]き 算[ざん]	ひきざん	
\\	娘は学校で引き算を習っている。	娘[むすめ]は 学校[がっこう]で 引[ひ]き 算[ざん]を 習[なら]っている。	むすめ は がっこう で ひきざん を ならって いる	
\\	娘[むすめ]は 学校[がっこう]で
\\	を 習[なら]っている。			
\\	残る	残[のこ]る	のこる	
\\	料理がたくさん残りました。	料理[りょうり]がたくさん 残[のこ]りました。	りょうり が たくさん のこりました	
\\	料理[りょうり]がたくさん
\\	残す	残[のこ]す	のこす	
\\	彼女はメッセージを残しました。	彼女[かのじょ]はメッセージを 残[のこ]しました。	かのじょ は めっせーじ を のこしました	
\\	彼女[かのじょ]はメッセージを
\\	期待	期待[きたい]	きたい	
\\	みんな私たちに期待しています。	みんな 私[わたし]たちに 期待[きたい]しています。	みんな わたしたち に きたい して います	
\\	みんな 私[わたし]たちに
\\	しています。			
\\	期間	期間[きかん]	きかん	
\\	テスト期間は10日から15日までだ。	テスト 期間[きかん]は 10日[とおか]から 15日[じゅうごにち]までだ。	てすと きかん は とおか から じゅうごにち まで だ	
\\	テスト
\\	は 10日[とおか]から 15日[じゅうごにち]までだ。			
\\	時期	時期[じき]	じき	
\\	今はあなたにとって大事な時期です。	今[いま]はあなたにとって 大事[だいじ]な 時期[じき]です。	いま は あなた に とって だいじ な じき です	
\\	今[いま]はあなたにとって 大事[だいじ]な
\\	です。			
\\	急ぐ	急[いそ]ぐ	いそぐ	
\\	私たちは駅へ急ぎました。	私[わたし]たちは 駅[えき]へ 急[いそ]ぎました。	わたしたち は えき へ いそぎました	
\\	私[わたし]たちは 駅[えき]へ
\\	急に	急[きゅう]に	きゅうに	
\\	急に用事を思い出した。	急[きゅう]に 用事[ようじ]を 思[おも]い 出[だ]した。	きゅうに ようじ を おもいだした	
\\	用事[ようじ]を 思[おも]い 出[だ]した。			
\\	急	急[きゅう]	きゅう	
\\	急な坂道を上った。	急[きゅう]な 坂道[さかみち]を 上[のぼ]った。	きゅう な さかみち を のぼった	
\\	な 坂道[さかみち]を 上[のぼ]った。			
\\	急行	急行[きゅうこう]	きゅうこう	
\\	ちょうど急行電車が来た。	ちょうど 急行[きゅうこう] 電車[でんしゃ]が 来[き]た。	ちょうど きゅうこう でんしゃ が きた	
\\	ちょうど
\\	電車[でんしゃ]が 来[き]た。			
\\	売り切れる	売[う]り 切[き]れる	うりきれる	
\\	その本は直ぐ売り切れた。	その 本[ほん]は 直[す]ぐ 売[う]り 切[き]れた。	その ほん は すぐ うりきれた	
\\	その 本[ほん]は 直[す]ぐ
\\	売り切れ	売[う]り 切[き]れ	うりきれ	
\\	チケットはもう売り切れだって。	チケットはもう 売[う]り 切[き]れだって。	ちけっと は もう うりきれ だって	
\\	チケットはもう
\\	だって。			
\\	大切	大切[たいせつ]	たいせつ	
\\	これは母が大切にしていた指輪です。	これは 母[はは]が 大切[たいせつ]にしていた 指輪[ゆびわ]です。	これ は はは が たいせつ に して いた ゆびわ です	
\\	これは 母[はは]が
\\	にしていた 指輪[ゆびわ]です。			
\\	家賃	家賃[やちん]	やちん	
\\	ここの家賃は12万円です。	ここの 家賃[やちん]は 12万円[じゅうにまんえん]です。	ここ の やちん は じゅうにまんえん です	
\\	ここの
\\	は 12万円[じゅうにまんえん]です。			
\\	時代	時代[じだい]	じだい	
\\	今は便利さとスピードの時代だ。	今[いま]は 便利[べんり]さとスピードの 時代[じだい]だ。	いま は べんりさ と すぴーど の じだい だ	
\\	今[いま]は 便利[べんり]さとスピードの
\\	だ。			
\\	指	指[ゆび]	ゆび	
\\	彼は指が太い。	彼[かれ]は 指[ゆび]が 太[ふと]い。	かれ は ゆび が ふとい	
\\	彼[かれ]は
\\	が 太[ふと]い。			
\\	決定	決定[けってい]	けってい	
\\	会議で重要な決定がありました。	会議[かいぎ]で 重要[じゅうよう]な 決定[けってい]がありました。	かいぎ で じゅうよう な けってい が ありました	
\\	会議[かいぎ]で 重要[じゅうよう]な
\\	がありました。			
\\	定期券	定期券[ていきけん]	ていきけん	
\\	定期券は1万2千円でした。	定期券[ていきけん]は 1万2千円[いちまんにせんえん]でした。	ていきけん は いちまんにせんえん でした	
\\	は 1万2千円[いちまんにせんえん]でした。			
\\	天気予報	天気予報[てんきよほう]	てんきよほう	
\\	明日の天気予報は雨です。	明日[あした]の 天気予報[てんきよほう]は 雨[あめ]です。	あした の てんきよほう は あめ です	
\\	明日[あした]の
\\	は 雨[あめ]です。			
\\	変わる	変[か]わる	かわる	
\\	信号が青に変わりました。	信号[しんごう]が 青[あお]に 変[か]わりました。	しんごう が あお に かわりました。	
\\	信号[しんごう]が 青[あお]に
\\	大変	大変[たいへん]	たいへん	
\\	大変なことが起こりました。	大変[たいへん]なことが 起[お]こりました。	たいへん な こと が おこりました	
\\	なことが 起[お]こりました。			
\\	変	変[へん]	へん	
\\	変な音が聞こえます。	変[へん]な 音[おと]が 聞[き]こえます。	へん な おと が きこえます	
\\	な 音[おと]が 聞[き]こえます。			
\\	変化	変化[へんか]	へんか	
\\	今年は変化の多い年でした。	今年[ことし]は 変化[へんか]の 多[おお]い 年[とし]でした。	ことし は へんか の おおい とし でした	
\\	今年[ことし]は
\\	の 多[おお]い 年[とし]でした。			
\\	強化	強化[きょうか]	きょうか	
\\	国は国語教育を強化しています。	国[くに]は 国語教育[こくご きょういく]を 強化[きょうか]しています。	くに は こくご きょういく を きょうか して います	
\\	国[くに]は 国語教育[こくご きょういく]を
\\	しています。			
\\	文化	文化[ぶんか]	ぶんか	
\\	私はこの国の文化を勉強しています。	私[わたし]はこの 国[くに]の 文化[ぶんか]を 勉強[べんきょう]しています。	わたし は この くに の ぶんか を べんきょう して います	
\\	私[わたし]はこの 国[くに]の
\\	を 勉強[べんきょう]しています。			
\\	増える	増[ふ]える	ふえる	
\\	この町は人口が増えた。	この 町[まち]は 人口[じんこう]が 増[ふ]えた。	この まち は じんこう が ふえた	
\\	この 町[まち]は 人口[じんこう]が
\\	増やす	増[ふ]やす	ふやす	
\\	あの町は緑を増やしています。	あの 町[まち]は 緑[みどり]を 増[ふ]やしています。	あの まち は みどり を ふやして います	
\\	あの 町[まち]は 緑[みどり]を
\\	役に立つ	役[やく]に 立[た]つ	やくにたつ	
\\	私は人々の役に立ちたいと思っています。	私[わたし]は 人々[ひとびと]の 役[やく]に 立[た]ちたいと 思[おも]っています。	わたし は ひとびと の やくにたちたい と おもって います	
\\	私[わたし]は 人々[ひとびと]の
\\	と 思[おも]っています。			
\\	席	席[せき]	せき	
\\	この席、空いてますか。	この 席[せき]、 空[あ]いてますか。	この せき あいてます か	
\\	この
\\	、 空[あ]いてますか。			
\\	欠席	欠席[けっせき]	けっせき	
\\	風邪のため今日は欠席します。	風邪[かぜ]のため 今日[きょう]は 欠席[けっせき]します。	かぜ の ため きょう は けっせき します	
\\	風邪[かぜ]のため 今日[きょう]は
\\	します。			
\\	次男	次男[じなん]	じなん	
\\	次男は今、海外にいます。	次男[じなん]は 今[いま]、 海外[かいがい]にいます。	じなん は いま かいがい に います	
\\	は 今[いま]、 海外[かいがい]にいます。			
\\	次女	次女[じじょ]	じじょ	
\\	うちの次女は春から中学生です。	うちの 次女[じじょ]は 春[はる]から 中学生[ちゅうがくせい]です。	うち の じじょ は はる から ちゅうがくせい です	
\\	うちの
\\	は 春[はる]から 中学生[ちゅうがくせい]です。			
\\	活動	活動[かつどう]	かつどう	
\\	彼は地域の活動に参加した。	彼[かれ]は 地域[ちいき]の 活動[かつどう]に 参加[さんか]した。	かれ は ちいき の かつどう に さんか した	
\\	彼[かれ]は 地域[ちいき]の
\\	に 参加[さんか]した。			
\\	早く	早[はや]く	はやく	
\\	なるべく早く来て下さい。	なるべく 早[はや]く 来[き]て 下[くだ]さい。	なるべく はやく きて ください	
\\	なるべく
\\	来[き]て 下[くだ]さい。			
\\	早口	早口[はやくち]	はやくち	
\\	彼女は早口だ。	彼女[かのじょ]は 早口[はやくち]だ。	かのじょ は はやくち だ	
\\	彼女[かのじょ]は
\\	だ。			
\\	始めに	始[はじ]めに	はじめに	
\\	始めにスープが出ます。	始[はじ]めにスープが 出[で]ます。	はじめに すーぷ が でます	
\\	スープが 出[で]ます。			
\\	実現	実現[じつげん]	じつげん	
\\	夢を実現するには努力が必要です。	夢[ゆめ]を 実現[じつげん]するには 努力[どりょく]が 必要[ひつよう]です。	ゆめ を じつげん する に は どりょく が ひつよう です	
\\	夢[ゆめ]を
\\	するには 努力[どりょく]が 必要[ひつよう]です。			
\\	実施	実施[じっし]	じっし	
\\	現在、スペシャルキャンペーンを実施中です。	現在[げんざい]、スペシャルキャンペーンを 実施[じっし] 中[ちゅう]です。	げんざい すぺしゃるきゃんぺーん を じっしちゅう です	
\\	現在[げんざい]、スペシャルキャンペーンを
\\	中[ちゅう]です。			
\\	実行	実行[じっこう]	じっこう	
\\	彼はその計画を実行した。	彼[かれ]はその 計画[けいかく]を 実行[じっこう]した。	かれ は その けいかく を じっこう した	
\\	彼[かれ]はその 計画[けいかく]を
\\	した。			
\\	実験	実験[じっけん]	じっけん	
\\	科学の授業で実験をした。	科学[かがく]の 授業[じゅぎょう]で 実験[じっけん]をした。	かがく の じゅぎょう で じっけん を した	
\\	科学[かがく]の 授業[じゅぎょう]で
\\	をした。			
\\	昼過ぎ	昼過[ひるす]ぎ	ひるすぎ	
\\	彼は昼過ぎに来ます。	彼[かれ]は 昼過[ひるす]ぎに 来[き]ます。	かれ は ひるすぎ に きます	
\\	彼[かれ]は
\\	に 来[き]ます。			
\\	手紙	手紙[てがみ]	てがみ	
\\	友人から手紙をもらいました。	友人[ゆうじん]から 手紙[てがみ]をもらいました。	ゆうじん から てがみ を もらいました	
\\	友人[ゆうじん]から
\\	をもらいました。			
\\	歌	歌[うた]	うた	
\\	私はその歌を知らなかった。	私[わたし]はその 歌[うた]を 知[し]らなかった。	わたし は その うた を しらなかった 。	
\\	私[わたし]はその
\\	を 知[し]らなかった。			
\\	歌手	歌手[かしゅ]	かしゅ	
\\	その歌手は歌が下手だ。	その 歌手[かしゅ]は 歌[うた]が 下手[へた]だ。	その かしゅ は うた が へた だ	
\\	その
\\	は 歌[うた]が 下手[へた]だ。			
\\	欲しがる	欲[ほ]しがる	ほしがる	
\\	子供がジュースを欲しがっています。	子供[こども]がジュースを 欲[ほ]しがっています。	こども が じゅーす を ほしがって います	
\\	子供[こども]がジュースを
\\	映画	映画[えいが]	えいが	
\\	彼はよく映画を見ます。	彼[かれ]はよく 映画[えいが]を 見[み]ます。	かれ は よく えいが を みます	
\\	彼[かれ]はよく
\\	を 見[み]ます。			
\\	形	形[かたち]	かたち	
\\	その椅子は変わった形をしている。	その 椅子[いす]は 変[か]わった 形[かたち]をしている。	その いす は かわった かたち を して いる	
\\	その 椅子[いす]は 変[か]わった
\\	をしている。			
\\	大型	大型[おおがた]	おおがた	
\\	大型のテレビを買った。	大型[おおがた]のテレビを 買[か]った。	おおがた の てれび を かった	
\\	のテレビを 買[か]った。			
\\	四角	四角[しかく]	しかく	
\\	紙を四角に切ってください。	紙[かみ]を 四角[しかく]に 切[き]ってください。	かみ を しかく に きって ください	
\\	紙[かみ]を
\\	に 切[き]ってください。			
\\	四角い	四角[しかく]い	しかくい	
\\	こっちの四角いテーブルを買おうよ。	こっちの 四角[しかく]いテーブルを 買[か]おうよ。	こっち の しかくい てーぶる を かおうよ	
\\	こっちの
\\	テーブルを 買[か]おうよ。			
\\	四つ角	四[よ]つ 角[かど]	よつかど	
\\	あそこの四つ角を左に曲がってください。	あそこの 四[よ]つ 角[かど]を 左[ひだり]に 曲[ま]がってください。	あそこ の よつかど を ひだり に まがって ください	
\\	あそこの
\\	を 左[ひだり]に 曲[ま]がってください。			
\\	曲	曲[きょく]	きょく	
\\	私はこの曲が大好きです。	私[わたし]はこの 曲[きょく]が 大好[だいす]きです。	わたし は この きょく が だいすき です	
\\	私[わたし]はこの
\\	が 大好[だいす]きです。			
\\	曲げる	曲[ま]げる	まげる	
\\	ひざを曲げてください。	ひざを 曲[ま]げてください。	ひざ を まげて ください	
\\	ひざを
\\	ください。			
\\	曲がり角	曲[ま]がり 角[かど]	まがりかど	
\\	ポストはそこの曲がり角にあります。	ポストはそこの 曲[ま]がり 角[かど]にあります。	ぽすと は そこ の まがりかど に あります	
\\	ポストはそこの
\\	にあります。			
\\	同様	同様[どうよう]	どうよう	
\\	私たちは彼を家族同様に思っている。	私[わたし]たちは 彼[かれ]を 家族[かぞく] 同様[どうよう]に 思[おも]っている。	わたしたち は かれ を かぞく どうよう に おもって いる	
\\	私[わたし]たちは 彼[かれ]を 家族[かぞく]
\\	に 思[おも]っている。			
\\	旅行	旅行[りょこう]	りょこう	
\\	彼女は旅行が好きです。	彼女[かのじょ]は 旅行[りょこう]が 好[す]きです。	かのじょ は りょこう が すき です	
\\	彼女[かのじょ]は
\\	が 好[す]きです。			
\\	大使館	大使館[たいしかん]	たいしかん	
\\	彼は大使館に勤めています。	彼[かれ]は 大使館[たいしかん]に 勤[つと]めています。	かれ は たいしかん に つとめて います	
\\	彼[かれ]は
\\	に 勤[つと]めています。			
\\	旅館	旅館[りょかん]	りょかん	
\\	京都では旅館に泊まりました。	京都[きょうと]では 旅館[りょかん]に 泊[と]まりました。	きょうと で は りょかん に とまりました	
\\	京都[きょうと]では
\\	に 泊[と]まりました。			
\\	映画館	映画館[えいがかん]	えいがかん	
\\	彼と近くの映画館に行きました。	彼[かれ]と 近[ちか]くの 映画館[えいがかん]に 行[い]きました。	かれ と ちかく の えいがかん に いきました	
\\	彼[かれ]と 近[ちか]くの
\\	に 行[い]きました。			
\\	宿題	宿題[しゅくだい]	しゅくだい	
\\	友達と一緒に宿題をした。	友達[ともだち]と 一緒[いっしょ]に 宿題[しゅくだい]をした。	ともだち と いっしょ に しゅくだい を した	
\\	友達[ともだち]と 一緒[いっしょ]に
\\	をした。			
\\	泊める	泊[と]める	とめる	
\\	友達をうちに泊めてあげました。	友達[ともだち]をうちに 泊[と]めてあげました。	ともだち を うち に とめて あげました	
\\	友達[ともだち]をうちに
\\	洋服	洋服[ようふく]	ようふく	
\\	今日は洋服を買いに行きます。	今日[きょう]は 洋服[ようふく]を 買[か]いに 行[い]きます。	きょう は ようふく を かい に いきます	
\\	今日[きょう]は
\\	を 買[か]いに 行[い]きます。			
\\	教室	教室[きょうしつ]	きょうしつ	
\\	私の教室は3階にあります。	私[わたし]の 教室[きょうしつ]は 3階[さんがい]にあります。	わたし の きょうしつ は さんがい に あります	
\\	私[わたし]の
\\	は 3階[さんがい]にあります。			
\\	図書室	図書室[としょしつ]	としょしつ	
\\	図書室で勉強した。	図書室[としょしつ]で 勉強[べんきょう]した。	としょしつ で べんきょう した	
\\	で 勉強[べんきょう]した。			
\\	家族	家族[かぞく]	かぞく	
\\	うちは五人家族です。	うちは 五人[ごにん] 家族[かぞく]です。	うち は ごにん かぞく です	
\\	うちは 五人[ごにん]
\\	です。			
\\	姉さん	姉[ねえ]さん	ねえさん	
\\	姉さん、ごめんね。	姉[ねえ]さん、ごめんね。	ねえさん ごめん ね	
\\	、ごめんね。			
\\	業者	業者[ぎょうしゃ]	ぎょうしゃ	
\\	引っ越しを業者に頼んだ。	引[ひ]っ 越[こ]しを 業者[ぎょうしゃ]に 頼[たの]んだ。	ひっこし を ぎょうしゃ に たのんだ	
\\	引[ひ]っ 越[こ]しを
\\	に 頼[たの]んだ。			
\\	彼ら	彼[かれ]ら	かれら	
\\	彼らはバスケットの選手です。	彼[かれ]らはバスケットの 選手[せんしゅ]です。	かれら は ばすけっと の せんしゅ です	
\\	はバスケットの 選手[せんしゅ]です。			
\\	果物	果物[くだもの]	くだもの	
\\	デザートに果物を食べましょう。	デザートに 果物[くだもの]を 食[た]べましょう。	でざーと に くだもの を たべましょう	
\\	デザートに
\\	を 食[た]べましょう。			
\\	市民	市民[しみん]	しみん	
\\	市民の安全は大切だ。	市民[しみん]の 安全[あんぜん]は 大切[たいせつ]だ。	しみん の あんぜん は たいせつ だ	
\\	の 安全[あんぜん]は 大切[たいせつ]だ。			
\\	対する	対[たい]する	たいする	
\\	その質問に対する答えが見つからなかった。	その 質問[しつもん]に 対[たい]する 答[こた]えが 見[み]つからなかった。	その しつもん に たいする こたえ が みつからなかった	
\\	その 質問[しつもん]に
\\	答[こた]えが 見[み]つからなかった。			
\\	対立	対立[たいりつ]	たいりつ	
\\	その2社は対立しています。	その 2社[にしゃ]は 対立[たいりつ]しています。	その にしゃ は たいりつ して います	
\\	その 2社[にしゃ]は
\\	しています。			
\\	普通	普通[ふつう]	ふつう	
\\	彼女は普通の女の子だ。	彼女[かのじょ]は 普通[ふつう]の 女[おんな]の 子[こ]だ。	かのじょ は ふつう の おんな の こ だ	
\\	彼女[かのじょ]は
\\	の 女[おんな]の 子[こ]だ。			
\\	平和	平和[へいわ]	へいわ	
\\	この国は平和です。	この 国[くに]は 平和[へいわ]です。	この くに は へいわ です	
\\	この 国[くに]は
\\	です。			
\\	大学院	大学院[だいがくいん]	だいがくいん	
\\	彼は大学院に進みました。	彼[かれ]は 大学院[だいがくいん]に 進[すす]みました。	かれ は だいがくいん に すすみました	
\\	彼[かれ]は
\\	に 進[すす]みました。			
\\	歯医者	歯医者[はいしゃ]	はいしゃ	
\\	私は歯医者が嫌いです。	私[わたし]は 歯医者[はいしゃ]が 嫌[きら]いです。	わたし は はいしゃ が きらい です	
\\	私[わたし]は
\\	が 嫌[きら]いです。			
\\	歯	歯[は]	は	
\\	私の歯は丈夫です。	私[わたし]の 歯[は]は 丈夫[じょうぶ]です。	わたし の は は じょうぶ です	
\\	私[わたし]の
\\	は 丈夫[じょうぶ]です。			
\\	歯ブラシ	歯[は]ブラシ	はぶらし	
\\	新しい歯ブラシが必要だ。	新[あたら]しい 歯[は]ブラシが 必要[ひつよう]だ。	あたらしい はぶらし が ひつよう だ	
\\	新[あたら]しい
\\	が 必要[ひつよう]だ。			
\\	教科書	教科書[きょうかしょ]	きょうかしょ	
\\	日本語の教科書を忘れた。	日本語[にほんご]の 教科書[きょうかしょ]を 忘[わす]れた。	にほんご の きょうかしょ を わすれた	
\\	日本語[にほんご]の
\\	を 忘[わす]れた。			
\\	忙しい	忙[いそが]しい	いそがしい	
\\	忙しいので手伝ってください。	忙[いそが]しいので 手伝[てつだ]ってください。	いそがしい の で てつだって ください	
\\	ので 手伝[てつだ]ってください。			
\\	存在	存在[そんざい]	そんざい	
\\	宇宙人は存在すると思いますか。	宇宙人[うちゅうじん]は 存在[そんざい]すると 思[おも]いますか。	うちゅうじん は そんざい する と おもいます か	
\\	宇宙人[うちゅうじん]は
\\	すると 思[おも]いますか。			
\\	注目	注目[ちゅうもく]	ちゅうもく	
\\	私たちはその会社に注目している。	私[わたし]たちはその 会社[かいしゃ]に 注目[ちゅうもく]している。	わたしたち は その かいしゃ に ちゅうもく して いる	
\\	私[わたし]たちはその 会社[かいしゃ]に
\\	している。			
\\	注文	注文[ちゅうもん]	ちゅうもん	
\\	レストランでピザを注文しました。	レストランでピザを 注文[ちゅうもん]しました。	れすとらん で ぴざ を ちゅうもん しました	
\\	レストランでピザを
\\	しました。			
\\	意味	意味[いみ]	いみ	
\\	それはどういう意味ですか。	それはどういう 意味[いみ]ですか。	それ は どういう いみ です か	
\\	それはどういう
\\	ですか。			
\\	意見	意見[いけん]	いけん	
\\	あなたの意見が聞きたいです。	あなたの 意見[いけん]が 聞[き]きたいです。	あなた の いけん が ききたい です	
\\	あなたの
\\	が 聞[き]きたいです。			
\\	注意	注意[ちゅうい]	ちゅうい	
\\	車に注意してください。	車[くるま]に 注意[ちゅうい]してください。	くるま に ちゅうい して ください	
\\	車[くるま]に
\\	してください。			
\\	機能	機能[きのう]	きのう	
\\	このソフトにはいろいろな機能があります。	このソフトにはいろいろな 機能[きのう]があります。	この そふと に は いろいろな きのう が あります	
\\	このソフトにはいろいろな
\\	があります。			
\\	機械	機械[きかい]	きかい	
\\	新しい機械が壊れた。	新[あたら]しい 機械[きかい]が 壊[こわ]れた。	あたらしい きかい が こわれた	
\\	新[あたら]しい
\\	が 壊[こわ]れた。			
\\	材料	材料[ざいりょう]	ざいりょう	
\\	サラダの材料をそろえました。	サラダの 材料[ざいりょう]をそろえました。	さらだ の ざいりょう を そろえました	
\\	サラダの
\\	をそろえました。			
\\	基づく	基[もと]づく	もとづく	
\\	この話は真実に基づいています。	この 話[はなし]は 真実[しんじつ]に 基[もと]づいています。	この はなし は しんじつ に もとづいて います	
\\	この 話[はなし]は 真実[しんじつ]に
\\	基本	基本[きほん]	きほん	
\\	今、ジャズダンスの基本を習っています。	今[いま]、ジャズダンスの 基本[きほん]を 習[なら]っています。	いま じゃず だんす の きほん を ならって います	
\\	今[いま]、ジャズダンスの
\\	を 習[なら]っています。			
\\	基準	基準[きじゅん]	きじゅん	
\\	判断の基準が示された。	判断[はんだん]の 基準[きじゅん]が 示[しめ]された。	はんだん の きじゅん が しめされた	
\\	判断[はんだん]の
\\	が 示[しめ]された。			
\\	施設	施設[しせつ]	しせつ	
\\	そのホテルにはレジャー施設がたくさんある。	そのホテルにはレジャー 施設[しせつ]がたくさんある。	その ほてる に は れじゃー しせつ が たくさん ある	
\\	そのホテルにはレジャー
\\	がたくさんある。			
\\	小説	小説[しょうせつ]	しょうせつ	
\\	私は月に3冊くらい小説を読みます。	私[わたし]は 月[つき]に 3冊[さんさつ]くらい 小説[しょうせつ]を 読[よ]みます。	わたし は つき に さんさつ くらい しょうせつ を よみます	
\\	私[わたし]は 月[つき]に 3冊[さんさつ]くらい
\\	を 読[よ]みます。			
\\	国際	国際[こくさい]	こくさい	
\\	ここで国際会議が開かれます。	ここで 国際[こくさい] 会議[かいぎ]が 開[ひら]かれます。	ここ で こくさい かいぎ が ひらかれます	
\\	ここで
\\	会議[かいぎ]が 開[ひら]かれます。			
\\	実際	実際[じっさい]	じっさい	
\\	彼は実際にはあまり背が高くない。	彼[かれ]は 実際[じっさい]にはあまり 背[せ]が 高[たか]くない。	かれ は じっさい に は あまり せ が たかく ない	
\\	彼[かれ]は
\\	にはあまり 背[せ]が 高[たか]くない。			
\\	島	島[しま]	しま	
\\	日本は島国です。	日本[にっぽん]は 島[しま] 国[ぐに]です。	にっぽん は しまぐに です	
\\	日本[にっぽん]は
\\	国[ぐに]です。			
\\	完成	完成[かんせい]	かんせい	
\\	新しいホームページが完成した。	新[あたら]しいホームページが 完成[かんせい]した。	あたらしい ほーむぺーじ が かんせい した	
\\	新[あたら]しいホームページが
\\	した。			
\\	平成	平成[へいせい]	へいせい	
\\	彼女は平成3年生まれです。	彼女[かのじょ]は 平成[へいせい] 3年生[さんねんう]まれです。	かのじょ は へいせい さんねん うまれ です	
\\	彼女[かのじょ]は
\\	3年生[さんねんう]まれです。			
\\	成功	成功[せいこう]	せいこう	
\\	ついに実験が成功した。	ついに 実験[じっけん]が 成功[せいこう]した。	ついに じっけん が せいこう した	
\\	ついに 実験[じっけん]が
\\	した。			
\\	投資	投資[とうし]	とうし	
\\	私は4つの会社に投資しています。	私[わたし]は 4[よっ]つの 会社[かいしゃ]に 投資[とうし]しています。	わたし は よっつ の かいしゃ に とうし して います	
\\	私[わたし]は 4[よっ]つの 会社[かいしゃ]に
\\	しています。			
\\	正確	正確[せいかく]	せいかく	
\\	彼の計算は正確です。	彼[かれ]の 計算[けいさん]は 正確[せいかく]です。	かれ の けいさん は せいかく です	
\\	彼[かれ]の 計算[けいさん]は
\\	です。			
\\	正しい	正[ただ]しい	ただしい	
\\	それは正しい答えです。	それは 正[ただ]しい 答[こた]えです。	それ は ただしい こたえ です	
\\	それは
\\	答[こた]えです。			
\\	正月	正月[しょうがつ]	しょうがつ	
\\	お正月にはたいてい、家族が集まる。	お 正月[しょうがつ]にはたいてい、 家族[かぞく]が 集[あつ]まる。	おしょうがつ に は たいてい かぞく が あつまる	
\\	お
\\	にはたいてい、 家族[かぞく]が 集[あつ]まる。			
\\	正直	正直[しょうじき]	しょうじき	
\\	彼女はとても正直だ。	彼女[かのじょ]はとても 正直[しょうじき]だ。	かのじょ は とても しょうじき だ	
\\	彼女[かのじょ]はとても
\\	だ。			
\\	強調	強調[きょうちょう]	きょうちょう	
\\	彼は良いところだけを強調した。	彼[かれ]は 良[よ]いところだけを 強調[きょうちょう]した。	かれ は よい ところ だけ を きょうちょう した	
\\	彼[かれ]は 良[よ]いところだけを
\\	した。			
\\	季節	季節[きせつ]	きせつ	
\\	私の一番好きな季節は春です。	私[わたし]の 一番好[いちばん す]きな 季節[きせつ]は 春[はる]です。	わたし の いちばん すき な きせつ は はる です	
\\	私[わたし]の 一番好[いちばん す]きな
\\	は 春[はる]です。			
\\	提供	提供[ていきょう]	ていきょう	
\\	彼がパーティー会場を提供してくれました。	彼[かれ]がパーティー 会場[かいじょう]を 提供[ていきょう]してくれました。	かれ が ぱーてぃー かいじょう を ていきょう して くれました	
\\	彼[かれ]がパーティー 会場[かいじょう]を
\\	してくれました。			
\\	提案	提案[ていあん]	ていあん	
\\	そのアイデアは彼の提案です。	そのアイデアは 彼[かれ]の 提案[ていあん]です。	その あいであ は かれ の ていあん です	
\\	そのアイデアは 彼[かれ]の
\\	です。			
\\	案内	案内[あんない]	あんない	
\\	私が中をご案内します。	私[わたし]が 中[なか]をご 案内[あんない]します。	わたし が なか を ごあんない します	
\\	私[わたし]が 中[なか]をご
\\	します。			
\\	治る	治[なお]る	なおる	
\\	けがはもう治りましたか。	けがはもう 治[なお]りましたか。	けが は もう なおりました か	
\\	けがはもう
\\	か。			
\\	政治	政治[せいじ]	せいじ	
\\	私は政治に関心がある。	私[わたし]は 政治[せいじ]に 関心[かんしん]がある。	わたし は せいじ に かんしん が ある	
\\	私[わたし]は
\\	に 関心[かんしん]がある。			
\\	治す	治[なお]す	なおす	
\\	早く風邪を治してください。	早[はや]く 風邪[かぜ]を 治[なお]してください。	はやく かぜ を なおして ください	
\\	早[はや]く 風邪[かぜ]を
\\	ください。			
\\	政府	政府[せいふ]	せいふ	
\\	そのデモについて政府は何もしなかった。	そのデモについて 政府[せいふ]は 何[なに]もしなかった。	その でも に ついて せいふ は なにも しなかった	
\\	そのデモについて
\\	は 何[なに]もしなかった。			
\\	対策	対策[たいさく]	たいさく	
\\	一緒に対策を考えましょう。	一緒[いっしょ]に 対策[たいさく]を 考[かんが]えましょう。	いっしょ に たいさく を かんがえましょう	
\\	一緒[いっしょ]に
\\	を 考[かんが]えましょう。			
\\	政策	政策[せいさく]	せいさく	
\\	新しい政策はあまり良いとは思えません。	新[あたら]しい 政策[せいさく]はあまり 良[い]いとは 思[おも]えません。	あたらしい せいさく は あまり いい と は おもえません	
\\	新[あたら]しい
\\	はあまり 良[い]いとは 思[おも]えません。			
\\	改革	改革[かいかく]	かいかく	
\\	彼は行政を改革したいと思っている。	彼[かれ]は 行政[ぎょうせい]を 改革[かいかく]したいと 思[おも]っている。	かれ は ぎょうせい を かいかく したい と おもって いる	
\\	彼[かれ]は 行政[ぎょうせい]を
\\	したいと 思[おも]っている。			
\\	命令	命令[めいれい]	めいれい	
\\	彼女は命令に従わなかった。	彼女[かのじょ]は 命令[めいれい]に 従[したが]わなかった。	かのじょ は めいれい に したがわなかった	
\\	彼女[かのじょ]は
\\	に 従[したが]わなかった。			
\\	拡大	拡大[かくだい]	かくだい	
\\	この図を拡大コピーしてください。	この 図[ず]を 拡大[かくだい]コピーしてください。	この ず を かくだい こぴー して ください	
\\	この 図[ず]を
\\	コピーしてください。			
\\	従来	従来[じゅうらい]	じゅうらい	
\\	このプリンターは従来のものより速い。	このプリンターは 従来[じゅうらい]のものより 速[はや]い。	この ぷりんたー は じゅうらい の もの より はやい	
\\	このプリンターは
\\	のものより 速[はや]い。			
\\	成績	成績[せいせき]	せいせき	
\\	成績が上がりました。	成績[せいせき]が 上[あ]がりました。	せいせき が あがりました	
\\	が 上[あ]がりました。			
\\	採用	採用[さいよう]	さいよう	
\\	その会社は女性を多く採用している。	その 会社[かいしゃ]は 女性[じょせい]を 多[おお]く 採用[さいよう]している。	その かいしゃ は じょせい を おおく さいよう して いる	
\\	その 会社[かいしゃ]は 女性[じょせい]を 多[おお]く
\\	している。			
\\	就職	就職[しゅうしょく]	しゅうしょく	
\\	最近、若い人たちの就職が難しくなっています。	最近[さいきん]、 若[わか]い 人[ひと]たちの 就職[しゅうしょく]が 難[むずか]しくなっています。	さいきん わかい ひとたち の しゅうしょく が むずかしく なって います	
\\	最近[さいきん]、 若[わか]い 人[ひと]たちの
\\	が 難[むずか]しくなっています。			
\\	契約	契約[けいやく]	けいやく	
\\	その選手は新しいチームと契約した。	その 選手[せんしゅ]は 新[あたら]しいチームと 契約[けいやく]した。	その せんしゅ は あたらしい ちーむ と けいやく した	
\\	その 選手[せんしゅ]は 新[あたら]しいチームと
\\	した。			
\\	条件	条件[じょうけん]	じょうけん	
\\	この条件では厳し過ぎます。	この 条件[じょうけん]では 厳[きび]し 過[す]ぎます。	この じょうけん で は きびし すぎます	
\\	この
\\	では 厳[きび]し 過[す]ぎます。			
\\	増加	増加[ぞうか]	ぞうか	
\\	島の人口は年々増加しています。	島[しま]の 人口[じんこう]は 年々[ねんねん] 増加[ぞうか]しています。	しま の じんこう は ねんねん ぞうか して います	
\\	島[しま]の 人口[じんこう]は 年々[ねんねん]
\\	しています。			
\\	比べる	比[くら]べる	くらべる	
\\	今月と先月の売上を比べた。	今月[こんげつ]と 先月[せんげつ]の 売上[うりあげ]を 比[くら]べた。	こんげつ と せんげつ の うりあげ を くらべた	
\\	今月[こんげつ]と 先月[せんげつ]の 売上[うりあげ]を
\\	批判	批判[ひはん]	ひはん	
\\	彼は同僚を批判した。	彼[かれ]は 同僚[どうりょう]を 批判[ひはん]した。	かれ は どうりょう を ひはん した	
\\	彼[かれ]は 同僚[どうりょう]を
\\	した。			
\\	対象	対象[たいしょう]	たいしょう	
\\	このアンケートは大学生が対象です。	このアンケートは 大学生[だいがくせい]が 対象[たいしょう]です。	この あんけーと は だいがくせい が たいしょう です	
\\	このアンケートは 大学生[だいがくせい]が
\\	です。			
\\	故障	故障[こしょう]	こしょう	
\\	冷蔵庫が故障しました。	冷蔵庫[れいぞうこ]が 故障[こしょう]しました。	れいぞうこ が こしょう しました	
\\	冷蔵庫[れいぞうこ]が
\\	しました。			
\\	換える	換[か]える	かえる	
\\	車のタイヤを換えた。	車[くるま]のタイヤを 換[か]えた。	くるま の たいや を かえた	
\\	車[くるま]のタイヤを
\\	壊す	壊[こわ]す	こわす	
\\	彼女が私のケータイを壊した。	彼女[かのじょ]が 私[わたし]のケータイを 壊[こわ]した。	かのじょ が わたし の けーたい を こわした	
\\	彼女[かのじょ]が 私[わたし]のケータイを
\\	壊れる	壊[こわ]れる	こわれる	
\\	会社のパソコンが壊れた。	会社[かいしゃ]のパソコンが 壊[こわ]れた。	かいしゃ の ぱそこん が こわれた	
\\	会社[かいしゃ]のパソコンが
\\	救急車	救急車[きゅうきゅうしゃ]	きゅうきゅうしゃ	
\\	誰か救急車を呼んでください。	誰[だれ]か 救急車[きゅうきゅうしゃ]を 呼[よ]んでください。	だれか きゅうきゅうしゃ を よんで ください	
\\	誰[だれ]か
\\	を 呼[よ]んでください。			
\\	殺す	殺[ころ]す	ころす	
\\	私は生き物を殺すのが嫌いだ。	私[わたし]は 生[い]き 物[もの]を 殺[ころ]すのが 嫌[きら]いだ。	わたし は いきもの を ころす の が きらい だ	
\\	私[わたし]は 生[い]き 物[もの]を
\\	のが 嫌[きら]いだ。			
\\	戦争	戦争[せんそう]	せんそう	
\\	2003年にイラクで戦争があった。	2003年[にせんさんねん]にイラクで 戦争[せんそう]があった。	にせんさんねん に いらく で せんそう が あった	
\\	2003年[にせんさんねん]にイラクで
\\	があった。			
\\	大統領	大統領[だいとうりょう]	だいとうりょう	
\\	フランスの大統領は誰ですか。	フランスの 大統領[だいとうりょう]は 誰[だれ]ですか。	ふらんす の だいとうりょう は だれ です か	
\\	フランスの
\\	は 誰[だれ]ですか。			
\\	捨てる	捨[す]てる	すてる	
\\	ゴミを捨ててください。	ゴミを 捨[す]ててください。	ごみ を すてて ください	
\\	ゴミを
\\	ください。			
\\	拾う	拾[ひろ]う	ひろう	
\\	道で財布を拾った。	道[みち]で 財布[さいふ]を 拾[ひろ]った。	みち で さいふ を ひろった	
\\	道[みち]で 財布[さいふ]を
\\	池	池[いけ]	いけ	
\\	池に鯉がいます。	池[いけ]に 鯉[こい]がいます。	いけ に こい が います	
\\	に 鯉[こい]がいます。			
\\	浅い	浅[あさ]い	あさい	
\\	この川は浅いです。	この 川[かわ]は 浅[あさ]いです。	この かわ は あさい です	
\\	この 川[かわ]は
\\	です。			
\\	泳ぐ	泳[およ]ぐ	およぐ	
\\	彼女はダイエットのために泳いでいる。	彼女[かのじょ]はダイエットのために 泳[およ]いでいる。	かのじょ は だいえっと の ため に およいで いる	
\\	彼女[かのじょ]はダイエットのために
\\	水泳	水泳[すいえい]	すいえい	
\\	母は健康のために水泳をしている。	母[はは]は 健康[けんこう]のために 水泳[すいえい]をしている。	はは は けんこう の ため に すいえい を して いる	
\\	母[はは]は 健康[けんこう]のために
\\	をしている。			
\\	流れる	流[なが]れる	ながれる	
\\	ラジオから美しい音楽が流れています。	ラジオから 美[うつく]しい 音楽[おんがく]が 流[なが]れています。	らじお から うつくしい おんがく が ながれて います	
\\	ラジオから 美[うつく]しい 音楽[おんがく]が
\\	流行る	流行[はや]る	はやる	
\\	去年はスニーカーが流行りました。	去年[きょねん]はスニーカーが 流行[はや]りました。	きょねん は すにーかー が はやりました	
\\	去年[きょねん]はスニーカーが
\\	洗う	洗[あら]う	あらう	
\\	早く顔を洗いなさい。	早[はや]く 顔[かお]を 洗[あら]いなさい。	はやく かお を あらいなさい	
\\	早[はや]く 顔[かお]を
\\	洗面所	洗面所[せんめんじょ]	せんめんじょ	
\\	洗面所で顔を洗った。	洗面所[せんめんじょ]で 顔[かお]を 洗[あら]った。	せんめんじょ で かお を あらった	
\\	で 顔[かお]を 洗[あら]った。			
\\	油	油[あぶら]	あぶら	
\\	水と油は混ざらない。	水[みず]と 油[あぶら]は 混[ま]ざらない。	みず と あぶら は まざらない	
\\	水[みず]と
\\	は 混[ま]ざらない。			
\\	沈む	沈[しず]む	しずむ	
\\	ボートが川に沈んだ。	ボートが 川[かわ]に 沈[しず]んだ。	ぼーと が かわ に しずんだ	
\\	ボートが 川[かわ]に
\\	氷	氷[こおり]	こおり	
\\	グラスに氷を入れてください。	グラスに 氷[こおり]を 入[い]れてください。	ぐらす に こおり を いれて ください	
\\	グラスに
\\	を 入[い]れてください。			
\\	汚す	汚[よご]す	よごす	
\\	彼は服を汚した。	彼[かれ]は 服[ふく]を 汚[よご]した。	かれ は ふく を よごした	
\\	彼[かれ]は 服[ふく]を
\\	汚れ	汚[よご]れ	よごれ	
\\	靴の汚れを落としました。	靴[くつ]の 汚[よご]れを 落[お]としました。	くつ の よごれ を おとしました	
\\	靴[くつ]の
\\	を 落[お]としました。			
\\	汚れる	汚[よご]れる	よごれる	
\\	エプロンをしないと服が汚れます。	エプロンをしないと 服[ふく]が 汚[よご]れます。	えぷろん を しない と ふく が よごれます	
\\	エプロンをしないと 服[ふく]が
\\	景色	景色[けしき]	けしき	
\\	ここは景色がきれいですね。	ここは 景色[けしき]がきれいですね。	ここ は けしき が きれい です ね	
\\	ここは
\\	がきれいですね。			
\\	影響	影響[えいきょう]	えいきょう	
\\	私は彼から大きな影響を受けました。	私[わたし]は 彼[かれ]から 大[おお]きな 影響[えいきょう]を 受[う]けました。	わたし は かれ から おおき な えいきょう を うけました	
\\	私[わたし]は 彼[かれ]から 大[おお]きな
\\	を 受[う]けました。			
\\	太る	太[ふと]る	ふとる	
\\	私の姉はすぐ太ります。	私[わたし]の 姉[あね]はすぐ 太[ふと]ります。	わたし の あね は すぐ ふとります	
\\	私[わたし]の 姉[あね]はすぐ
\\	太陽	太陽[たいよう]	たいよう	
\\	太陽が雲に隠れた。	太陽[たいよう]が 雲[くも]に 隠[かく]れた。	たいよう が くも に かくれた	
\\	が 雲[くも]に 隠[かく]れた。			
\\	星	星[ほし]	ほし	
\\	今夜は星がよく見えます。	今夜[こんや]は 星[ほし]がよく 見[み]えます。	こんや は ほし が よく みえます	
\\	今夜[こんや]は
\\	がよく 見[み]えます。			
\\	地球	地球[ちきゅう]	ちきゅう	
\\	地球は丸い。	地球[ちきゅう]は 丸[まる]い。	ちきゅう は まるい	
\\	は 丸[まる]い。			
\\	曇り	曇[くも]り	くもり	
\\	今日は一日曇りでした。	今日[きょう]は 一日[いちにち] 曇[くも]りでした。	きょう は いちにち くもり でした	
\\	今日[きょう]は 一日[いちにち]
\\	でした。			
\\	地震	地震[じしん]	じしん	
\\	日本は地震が多いです。	日本[にほん]は 地震[じしん]が 多[おお]いです。	にほん は じしん が おおい です	
\\	日本[にほん]は
\\	が 多[おお]いです。			
\\	振る	振[ふ]る	ふる	
\\	犬がしっぽを振っている。	犬[いぬ]がしっぽを 振[ふ]っている。	いぬ が しっぽ を ふって いる	
\\	犬[いぬ]がしっぽを
\\	揺れる	揺[ゆ]れる	ゆれる	
\\	風で木が揺れています。	風[かぜ]で 木[き]が 揺[ゆ]れています。	かぜ で き が ゆれて います	
\\	風[かぜ]で 木[き]が
\\	います。			
\\	年寄り	年寄[としよ]り	としより	
\\	あの村にはお年寄りがたくさん住んでいます。	あの 村[むら]にはお 年寄[としよ]りがたくさん 住[す]んでいます。	あの むら に は おとしより が たくさん すんで います	
\\	あの 村[むら]にはお
\\	がたくさん 住[す]んでいます。			
\\	歴史	歴史[れきし]	れきし	
\\	私は歴史に興味があります。	私[わたし]は 歴史[れきし]に 興味[きょうみ]があります。	わたし は れきし に きょうみ が あります	
\\	私[わたし]は
\\	に 興味[きょうみ]があります。			
\\	建設	建設[けんせつ]	けんせつ	
\\	新しいビルの建設が始まった。	新[あたら]しいビルの 建設[けんせつ]が 始[はじ]まった。	あたらしい びる の けんせつ が はじまった	
\\	新[あたら]しいビルの
\\	が 始[はじ]まった。			
\\	建物	建物[たてもの]	たてもの	
\\	これは日本一古い建物です。	これは 日本一古[にほんいち ふる]い 建物[たてもの]です。	これ は にほんいち ふるい たてもの です	
\\	これは 日本一古[にほんいち ふる]い
\\	です。			
\\	建つ	建[た]つ	たつ	
\\	ここに来年、家が建ちます。	ここに 来年[らいねん]、 家[いえ]が 建[た]ちます。	ここ に らいねん いえ が たちます	
\\	ここに 来年[らいねん]、 家[いえ]が
\\	構成	構成[こうせい]	こうせい	
\\	システムの構成を変えてみました。	システムの 構成[こうせい]を 変[か]えてみました。	しすてむ の こうせい を かえて みました	
\\	システムの
\\	を 変[か]えてみました。			
\\	構造	構造[こうぞう]	こうぞう	
\\	この建物の構造は複雑です。	この 建物[たてもの]の 構造[こうぞう]は 複雑[ふくざつ]です。	この たてもの の こうぞう は ふくざつ です	
\\	この 建物[たてもの]の
\\	は 複雑[ふくざつ]です。			
\\	橋	橋[はし]	はし	
\\	あの橋は日本で一番長い。	あの 橋[はし]は 日本[にほん]で 一番長[いちばん なが]い。	あの はし は にほん で いちばん ながい	
\\	あの
\\	は 日本[にほん]で 一番長[いちばん なが]い。			
\\	柱	柱[はしら]	はしら	
\\	この家の柱は太い。	この 家[いえ]の 柱[はしら]は 太[ふと]い。	この いえ の はしら は ふとい	
\\	この 家[いえ]の
\\	は 太[ふと]い。			
\\	周辺	周辺[しゅうへん]	しゅうへん	
\\	この周辺には大学が多い。	この 周辺[しゅうへん]には 大学[だいがく]が 多[おお]い。	この しゅうへん に は だいがく が おおい	
\\	この
\\	には 大学[だいがく]が 多[おお]い。			
\\	横	横[よこ]	よこ	
\\	横の長さは1メートルです。	横[よこ]の 長[なが]さは 1[いち]メートルです。	よこ の ながさ は いちめーとる です	
\\	の 長[なが]さは 1[いち]メートルです。			
\\	横書き	横書[よこが]き	よこがき	
\\	この本は横書きです。	この 本[ほん]は 横書[よこが]きです。	この ほん は よこがき です	
\\	この 本[ほん]は
\\	です。			
\\	断る	断[ことわ]る	ことわる	
\\	私は彼のプロポーズを断った。	私[わたし]は 彼[かれ]のプロポーズを 断[ことわ]った。	わたし は かれ の ぷろぽーず を ことわった	
\\	私[わたし]は 彼[かれ]のプロポーズを
\\	横断歩道	横断歩道[おうだんほどう]	おうだんほどう	
\\	あそこに横断歩道があります。	あそこに 横断歩道[おうだんほどう]があります。	あそこ に おうだんほどう が あります	
\\	あそこに
\\	があります。			
\\	大幅	大幅[おおはば]	おおはば	
\\	計画を大幅に変更した。	計画[けいかく]を 大幅[おおはば]に 変更[へんこう]した。	けいかく を おおはば に へんこう した	
\\	計画[けいかく]を
\\	に 変更[へんこう]した。			
\\	暖房	暖房[だんぼう]	だんぼう	
\\	冬は暖房が必要です。	冬[ふゆ]は 暖房[だんぼう]が 必要[ひつよう]です。	ふゆ は だんぼう が ひつよう です	
\\	冬[ふゆ]は
\\	が 必要[ひつよう]です。			
\\	文房具	文房具[ぶんぼうぐ]	ぶんぼうぐ	
\\	新しい文房具を買いました。	新[あたら]しい 文房具[ぶんぼうぐ]を 買[か]いました。	あたらしい ぶんぼうぐ を かいました	
\\	新[あたら]しい
\\	を 買[か]いました。			
\\	指輪	指輪[ゆびわ]	ゆびわ	
\\	彼女に指輪をプレゼントしました。	彼女[かのじょ]に 指輪[ゆびわ]をプレゼントしました。	かのじょ に ゆびわ を ぷれぜんと しました	
\\	彼女[かのじょ]に
\\	をプレゼントしました。			
\\	往復	往復[おうふく]	おうふく	
\\	往復切符をください。	往復[おうふく] 切符[きっぷ]をください。	おうふく きっぷ を ください	
\\	切符[きっぷ]をください。			
\\	復習	復習[ふくしゅう]	ふくしゅう	
\\	昨日の復習をしましたか。	昨日[きのう]の 復習[ふくしゅう]をしましたか。	きのう の ふくしゅう を しました か	
\\	昨日[きのう]の
\\	をしましたか。			
\\	書留	書留[かきとめ]	かきとめ	
\\	これを書留で送りたいのですが。	これを 書留[かきとめ]で 送[おく]りたいのですが。	これ を かきとめ で おくりたい の です が	
\\	これを
\\	で 送[おく]りたいのですが。			
\\	守る	守[まも]る	まもる	
\\	彼は約束を守る人です。	彼[かれ]は 約束[やくそく]を 守[まも]る 人[ひと]です。	かれ は やくそく を まもる ひと です	
\\	彼[かれ]は 約束[やくそく]を
\\	人[ひと]です。			
\\	早起き	早起[はやお]き	はやおき	
\\	祖父は早起きです。	祖父[そふ]は 早起[はやお]きです。	そふ は はやおき です	
\\	祖父[そふ]は
\\	です。			
\\	昼寝	昼寝[ひるね]	ひるね	
\\	私の子供は毎日昼寝をします。	私[わたし]の 子供[こども]は 毎日[まいにち] 昼寝[ひるね]をします。	わたし の こども は まいにち ひるね を します	
\\	私[わたし]の 子供[こども]は 毎日[まいにち]
\\	をします。			
\\	暇	暇[ひま]	ひま	
\\	明日は暇ですか。	明日[あした]は 暇[ひま]ですか。	あした は ひま です か	
\\	明日[あした]は
\\	ですか。			
\\	向こう側	向[む]こう 側[がわ]	むこうがわ	
\\	私の家は川の向こう側にあります。	私[わたし]の 家[いえ]は 川[かわ]の 向[む]こう 側[がわ]にあります。	わたし の いえ は かわ の むこうがわ に あります	
\\	私[わたし]の 家[いえ]は 川[かわ]の
\\	にあります。			
\\	外側	外側[そとがわ]	そとがわ	
\\	白線の外側を歩かないでください。	白線[はくせん]の 外側[そとがわ]を 歩[ある]かないでください。	はくせん の そとがわ を あるかない で ください	
\\	白線[はくせん]の
\\	を 歩[ある]かないでください。			
\\	左側	左側[ひだりがわ]	ひだりがわ	
\\	画面の左側を見てください。	画面[がめん]の 左側[ひだりがわ]を 見[み]てください。	がめん の ひだりがわ を みて ください	
\\	画面[がめん]の
\\	を 見[み]てください。			
\\	右側	右側[みぎがわ]	みぎがわ	
\\	彼女はいつも私の右側を歩きます。	彼女[かのじょ]はいつも 私[わたし]の 右側[みぎがわ]を 歩[ある]きます。	かのじょ は いつも わたし の みぎがわ を あるきます	
\\	彼女[かのじょ]はいつも 私[わたし]の
\\	を 歩[ある]きます。			
\\	朝刊	朝刊[ちょうかん]	ちょうかん	
\\	今日の朝刊に面白い記事があった。	今日[きょう]の 朝刊[ちょうかん]に 面白[おもしろ]い 記事[きじ]があった。	きょう の ちょうかん に おもしろい きじ が あった	
\\	今日[きょう]の
\\	に 面白[おもしろ]い 記事[きじ]があった。			
\\	夕刊	夕刊[ゆうかん]	ゆうかん	
\\	そのニュースは夕刊で見ました。	そのニュースは 夕刊[ゆうかん]で 見[み]ました。	その にゅーす は ゆうかん で みました	
\\	そのニュースは
\\	で 見[み]ました。			
\\	検討	検討[けんとう]	けんとう	
\\	今日中にこの問題を検討してください。	今日中[きょう じゅう]にこの 問題[もんだい]を 検討[けんとう]してください。	きょう じゅう に この もんだい を けんとう して ください	
\\	今日中[きょう じゅう]にこの 問題[もんだい]を
\\	してください。			
\\	塗る	塗[ぬ]る	ぬる	
\\	壁にペンキを塗っています。	壁[かべ]にペンキを 塗[ぬ]っています。	かべ に ぺんき を ぬって います	
\\	壁[かべ]にペンキを
\\	受け付ける	受[う]け 付[つ]ける	うけつける	
\\	郵便物は5時まで受け付けています。	郵便物[ゆうびんぶつ]は 5時[ごじ]まで 受[う]け 付[つ]けています。	ゆうびんぶつ は ごじ まで うけつけて います	
\\	郵便物[ゆうびんぶつ]は 5時[ごじ]まで
\\	受付	受付[うけつけ]	うけつけ	
\\	受付は9時からです。	受付[うけつけ]は 9時[くじ]からです。	うけつけ は くじ から です	
\\	は 9時[くじ]からです。			
\\	気を付ける	気[き]を 付[つ]ける	きをつける	
\\	体に気を付けてください。	体[からだ]に 気[き]を 付[つ]けてください。	からだ に き を つけて ください	
\\	体[からだ]に
\\	ください。			
\\	残念	残念[ざんねん]	ざんねん	
\\	その試合は残念な結果になった。	その 試合[しあい]は 残念[ざんねん]な 結果[けっか]になった。	その しあい は ざんねん な けっか に なった	
\\	その 試合[しあい]は
\\	な 結果[けっか]になった。			
\\	困る	困[こま]る	こまる	
\\	ケータイをなくして困っています。	ケータイをなくして 困[こま]っています。	けーたい を なくして こまって います	
\\	ケータイをなくして
\\	幸せ	幸[しあわ]せ	しあわせ	
\\	良い友達がいて私は幸せだ。	良[い]い 友達[ともだち]がいて 私[わたし]は 幸[しあわ]せだ。	いい ともだち が いて わたし は しあわせ だ	
\\	良[い]い 友達[ともだち]がいて 私[わたし]は
\\	だ。			
\\	塩	塩[しお]	しお	
\\	もうちょっと塩を入れて。	もうちょっと 塩[しお]を 入[い]れて。	もう ちょっと しお を いれて	
\\	もうちょっと
\\	を 入[い]れて。			
\\	塩辛い	塩辛[しおから]い	しおからい	
\\	海の水は塩辛い。	海[うみ]の 水[みず]は 塩辛[しおから]い。	うみ の みず は しおからい	
\\	海[うみ]の 水[みず]は
\\	引っ越す	引[ひ]っ 越[こ]す	ひっこす	
\\	来月、大阪に引っ越します。	来月[らいげつ]、 大阪[おおさか]に 引[ひ]っ 越[こ]します。	らいげつ おおさか に ひっこします	
\\	来月[らいげつ]、 大阪[おおさか]に
\\	引っ越し	引[ひ]っ 越[こ]し	ひっこし	
\\	去年、引っ越ししました。	去年[きょねん]、 引[ひ]っ 越[こ]ししました。	きょねん ひっこし しました	
\\	去年[きょねん]、
\\	しました。			
\\	改札口	改札口[かいさつぐち]	かいさつぐち	
\\	改札口で会いましょう。	改札口[かいさつぐち]で 会[あ]いましょう。	かいさつぐち で あいましょう	
\\	で 会[あ]いましょう。			
\\	失礼	失礼[しつれい]	しつれい	
\\	ではそろそろ失礼します。	ではそろそろ 失礼[しつれい]します。	では そろそろ しつれい します	
\\	ではそろそろ
\\	します。			
\\	注射	注射[ちゅうしゃ]	ちゅうしゃ	
\\	彼は注射があまり好きではありません。	彼[かれ]は 注射[ちゅうしゃ]があまり 好[す]きではありません。	かれ は ちゅうしゃ が あまり すき で は ありません	
\\	彼[かれ]は
\\	があまり 好[す]きではありません。			
\\	導入	導入[どうにゅう]	どうにゅう	
\\	会社で新しいシステムを導入した。	会社[かいしゃ]で 新[あたら]しいシステムを 導入[どうにゅう]した。	かいしゃ で あたらしい しすてむ を どうにゅう した	
\\	会社[かいしゃ]で 新[あたら]しいシステムを
\\	した。			
\\	怒る	怒[おこ]る	おこる	
\\	彼女が嘘をついたので、彼は怒った。	彼女[かのじょ]が 嘘[うそ]をついたので、 彼[かれ]は 怒[おこ]った。	かのじょ が うそ を ついた の で かれ は おこった	
\\	彼女[かのじょ]が 嘘[うそ]をついたので、 彼[かれ]は
\\	招く	招[まね]く	まねく	
\\	両親を食事に招いた。	両親[りょうしん]を 食事[しょくじ]に 招[まね]いた。	りょうしん を しょくじ に まねいた	
\\	両親[りょうしん]を 食事[しょくじ]に
\\	招待	招待[しょうたい]	しょうたい	
\\	高校の時の先生を結婚式に招待した。	高校[こうこう]の 時[とき]の 先生[せんせい]を 結婚式[けっこんしき]に 招待[しょうたい]した。	こうこう の とき の せんせい を けっこんしき に しょうたい した	
\\	高校[こうこう]の 時[とき]の 先生[せんせい]を 結婚式[けっこんしき]に
\\	した。			
\\	夫婦	夫婦[ふうふ]	ふうふ	
\\	その夫婦はとても仲がいい。	その 夫婦[ふうふ]はとても 仲[なか]がいい。	その ふうふ は とても なか が いい	
\\	その
\\	はとても 仲[なか]がいい。			
\\	奥	奥[おく]	おく	
\\	はさみは机の奥にあった。	はさみは 机[つくえ]の 奥[おく]にあった。	はさみ は つくえ の おく に あった	
\\	はさみは 机[つくえ]の
\\	にあった。			
\\	奥さん	奥[おく]さん	おくさん	
\\	彼の奥さんはきれいな方です。	彼[かれ]の 奥[おく]さんはきれいな 方[かた]です。	かれ の おくさん は きれい な かた です	
\\	彼[かれ]の
\\	はきれいな 方[かた]です。			
\\	国籍	国籍[こくせき]	こくせき	
\\	私は日本国籍です。	私[わたし]は 日本[にほん] 国籍[こくせき]です。	わたし は にほん こくせき です	
\\	私[わたし]は 日本[にほん]
\\	です。			
\\	愛	愛[あい]	あい	
\\	彼女は愛をこめて手紙を書いた。	彼女[かのじょ]は 愛[あい]をこめて 手紙[てがみ]を 書[か]いた。	かのじょ は あい を こめて てがみ を かいた	
\\	彼女[かのじょ]は
\\	をこめて 手紙[てがみ]を 書[か]いた。			
\\	可愛い	可愛[かわい]い	かわいい	
\\	彼女の赤ちゃんは可愛いです。	彼女[かのじょ]の 赤[あか]ちゃんは 可愛[かわい]いです。	かのじょ の あかちゃん は かわいい です	
\\	彼女[かのじょ]の 赤[あか]ちゃんは
\\	です。			
\\	恋人	恋人[こいびと]	こいびと	
\\	彼は恋人を失った。	彼[かれ]は 恋人[こいびと]を 失[うしな]った。	かれ は こいびと を うしなった	
\\	彼[かれ]は
\\	を 失[うしな]った。			
\\	夢	夢[ゆめ]	ゆめ	
\\	昨夜恐ろしい夢を見た。	昨夜[ゆうべ] 恐[おそ]ろしい 夢[ゆめ]を 見[み]た。	ゆうべ おそろしい ゆめ を みた	
\\	昨夜[ゆうべ] 恐[おそ]ろしい
\\	を 見[み]た。			
\\	泣く	泣[な]く	なく	
\\	妹はすぐに泣く。	妹[いもうと]はすぐに 泣[な]く。	いもうと は すぐ に なく	
\\	妹[いもうと]はすぐに
\\	喜ぶ	喜[よろこ]ぶ	よろこぶ	
\\	彼女はとても喜びました。	彼女[かのじょ]はとても 喜[よろこ]びました。	かのじょ は とても よろこびました	
\\	彼女[かのじょ]はとても
\\	恥ずかしい	恥[は]ずかしい	はずかしい	
\\	とても恥ずかしかった。	とても 恥[は]ずかしかった。	とても はずかしかった	
\\	とても
\\	弁当	弁当[べんとう]	べんとう	
\\	今日は弁当を持ってきました。	今日[きょう]は 弁当[べんとう]を 持[も]ってきました。	きょう は べんとう を もって きました	
\\	今日[きょう]は
\\	を 持[も]ってきました。			
\\	患者	患者[かんじゃ]	かんじゃ	
\\	患者は眠っています。	患者[かんじゃ]は 眠[ねむ]っています。	かんじゃ は ねむって います	
\\	は 眠[ねむ]っています。			
\\	地域	地域[ちいき]	ちいき	
\\	この地域は雨が多い。	この 地域[ちいき]は 雨[あめ]が 多[おお]い。	この ちいき は あめ が おおい	
\\	この
\\	は 雨[あめ]が 多[おお]い。			
\\	政権	政権[せいけん]	せいけん	
\\	政権が交代した。	政権[せいけん]が 交代[こうたい]した。	せいけん が こうたい した	
\\	が 交代[こうたい]した。			
\\	得意	得意[とくい]	とくい	
\\	彼は歌が得意です。	彼[かれ]は 歌[うた]が 得意[とくい]です。	かれ は うた が とくい です	
\\	彼[かれ]は 歌[うた]が
\\	です。			
\\	新幹線	新幹線[しんかんせん]	しんかんせん	
\\	新幹線で京都に行きました。	新幹線[しんかんせん]で 京都[きょうと]に 行[い]きました。	しんかんせん で きょうと に いきました	
\\	で 京都[きょうと]に 行[い]きました。			
\\	海岸	海岸[かいがん]	かいがん	
\\	海岸を散歩しましょう。	海岸[かいがん]を 散歩[さんぽ]しましょう。	かいがん を さんぽ しましょう	
\\	を 散歩[さんぽ]しましょう。			
\\	家庭	家庭[かてい]	かてい	
\\	彼は家庭を大切にしている。	彼[かれ]は 家庭[かてい]を 大切[たいせつ]にしている。	かれ は かてい を たいせつ に して いる	
\\	彼[かれ]は
\\	を 大切[たいせつ]にしている。			
\\	庭	庭[にわ]	にわ	
\\	庭にバラを植えました。	庭[にわ]にバラを 植[う]えました。	にわ に ばら を うえました	
\\	にバラを 植[う]えました。			
\\	桜	桜[さくら]	さくら	
\\	桜は三月か四月に咲きます。	桜[さくら]は 三月[さんがつ]か 四月[しがつ]に 咲[さ]きます。	さくら は さんがつ か しがつ に さきます	
\\	は 三月[さんがつ]か 四月[しがつ]に 咲[さ]きます。			
\\	咲く	咲[さ]く	さく	
\\	桜の花が咲きました。	桜[さくら]の 花[はな]が 咲[さ]きました。	さくら の はな が さきました	
\\	桜[さくら]の 花[はな]が
\\	吹く	吹[ふ]く	ふく	
\\	今日は北風が吹いている。	今日[きょう]は 北風[きたかぜ]が 吹[ふ]いている。	きょう は きたかぜ が ふいて いる	
\\	今日[きょう]は 北風[きたかぜ]が
\\	散歩	散歩[さんぽ]	さんぽ	
\\	私のお祖父さんは毎日散歩します。	私[わたし]のお 祖父[じい]さんは 毎日[まいにち] 散歩[さんぽ]します。	わたし の おじいさん は まいにち さんぽ します 。	
\\	私[わたし]のお 祖父[じい]さんは
\\	します。			
\\	植える	植[う]える	うえる	
\\	庭にバラを植えました。	庭[にわ]にバラを 植[う]えました。	にわ に ばら を うえました	
\\	庭[にわ]にバラを
\\	屋根	屋根[やね]	やね	
\\	屋根にカラスが止まっています。	屋根[やね]にカラスが 止[と]まっています。	やね に からす が とまって います	
\\	にカラスが 止[と]まっています。			
\\	掲示板	掲示板[けいじばん]	けいじばん	
\\	掲示板のお知らせを見ましたか。	掲示板[けいじばん]のお 知[し]らせを 見[み]ましたか。	けいじばん の おしらせ を みました か	
\\	のお 知[し]らせを 見[み]ましたか。			
\\	吸う	吸[す]う	すう	
\\	彼は大きく息を吸った。	彼[かれ]は 大[おお]きく 息[いき]を 吸[す]った。	かれ は おおきく いき を すった	
\\	彼[かれ]は 大[おお]きく 息[いき]を
\\	普及	普及[ふきゅう]	ふきゅう	
\\	ゴミのリサイクルが普及している。	ゴミのリサイクルが 普及[ふきゅう]している。	ごみ の りさいくる が ふきゅう して いる	
\\	ゴミのリサイクルが
\\	している。			
\\	折れる	折[お]れる	おれる	
\\	強風で木の枝が折れた。	強風[きょうふう]で 木[き]の 枝[えだ]が 折[お]れた。	きょうふう で き の えだ が おれた	
\\	強風[きょうふう]で 木[き]の 枝[えだ]が
\\	折る	折[お]る	おる	
\\	祖父は足の骨を折りました。	祖父[そふ]は 足[あし]の 骨[ほね]を 折[お]りました。	そふ は あし の ほね を おりました	
\\	祖父[そふ]は 足[あし]の 骨[ほね]を
\\	撮る	撮[と]る	とる	
\\	写真をたくさん撮りました。	写真[しゃしん]をたくさん 撮[と]りました。	しゃしん を たくさん とりました	
\\	写真[しゃしん]をたくさん
\\	放送	放送[ほうそう]	ほうそう	
\\	その番組は来週放送されます。	その 番組[ばんぐみ]は 来週[らいしゅう] 放送[ほうそう]されます。	その ばんぐみ は らいしゅう ほうそう されます	
\\	その 番組[ばんぐみ]は 来週[らいしゅう]
\\	されます。			
\\	悲しむ	悲[かな]しむ	かなしむ	
\\	父は友だちの死を悲しんでいます。	父[ちち]は 友[とも]だちの 死[し]を 悲[かな]しんでいます。	ちち は ともだち の し を かなしんで います	
\\	父[ちち]は 友[とも]だちの 死[し]を
\\	固い	固[かた]い	かたい	
\\	私の上司は頭が固い。	私[わたし]の 上司[じょうし]は 頭[あたま]が 固[かた]い。	わたし の じょうし は あたま が かたい	
\\	私[わたし]の 上司[じょうし]は 頭[あたま]が
\\	教師	教師[きょうし]	きょうし	
\\	彼は高校教師だ。	彼[かれ]は 高校[こうこう] 教師[きょうし]だ。	かれ は こうこう きょうし だ	
\\	彼[かれ]は 高校[こうこう]
\\	だ。			
\\	教授	教授[きょうじゅ]	きょうじゅ	
\\	彼は化学の教授です。	彼[かれ]は 化学[かがく]の 教授[きょうじゅ]です。	かれ は かがく の きょうじゅ です	
\\	彼[かれ]は 化学[かがく]の
\\	です。			
\\	声	声[こえ]	こえ	
\\	彼は大きな声で話した。	彼[かれ]は 大[おお]きな 声[こえ]で 話[はな]した。	かれ は おおき な こえ で はなした	
\\	彼[かれ]は 大[おお]きな
\\	で 話[はな]した。			
\\	柔道	柔道[じゅうどう]	じゅうどう	
\\	私は柔道を習っています。	私[わたし]は 柔道[じゅうどう]を 習[なら]っています。	わたし は じゅうどう を ならって います	
\\	私[わたし]は
\\	を 習[なら]っています。			
\\	柔らかい	柔[やわ]らかい	やわらかい	
\\	布団がとても柔らかい。	布団[ふとん]がとても 柔[やわ]らかい。	ふとん が とても やわらかい	
\\	布団[ふとん]がとても
\\	柔らか	柔[やわ]らか	やわらか	
\\	彼の声は柔らかだ。	彼[かれ]の 声[こえ]は 柔[やわ]らかだ。	かれ の こえ は やわらか だ	
\\	彼[かれ]の 声[こえ]は
\\	だ。			
\\	引っ張る	引[ひ]っ 張[ぱ]る	ひっぱる	
\\	娘が私の手を引っ張った。	娘[むすめ]が 私[わたし]の 手[て]を 引[ひ]っ 張[ぱ]った。	むすめ が わたし の て を ひっぱった	
\\	娘[むすめ]が 私[わたし]の 手[て]を
\\	壁	壁[かべ]	かべ	
\\	壁に絵が掛かっている。	壁[かべ]に 絵[え]が 掛[か]かっている。	かべ に え が かかって いる	
\\	に 絵[え]が 掛[か]かっている。			
\\	弾く	弾[ひ]く	ひく	
\\	彼はギターを弾きます。	彼[かれ]はギターを 弾[ひ]きます。	かれ は ぎたー を ひきます	
\\	彼[かれ]はギターを
\\	攻撃	攻撃[こうげき]	こうげき	
\\	2003年にアメリカはイラクを攻撃した。	2003年[にせんさんねん]にアメリカはイラクを 攻撃[こうげき]した。	にせんさんねん に あめりか は いらく を こうげき した	
\\	2003年[にせんさんねん]にアメリカはイラクを
\\	した。			
\\	嫌	嫌[いや]	いや	
\\	私は待つのが嫌だ。	私[わたし]は 待[ま]つのが 嫌[いや]だ。	わたし は まつ の が いや だ	
\\	私[わたし]は 待[ま]つのが
\\	だ。			
\\	大嫌い	大嫌[だいきら]い	だいきらい	
\\	私はテストが大嫌い。	私[わたし]はテストが 大嫌[だいきら]い。	わたし は てすと が だいきらい	
\\	私[わたし]はテストが
\\	大抵	大抵[たいてい]	たいてい	
\\	朝食は大抵7時頃に食べます。	朝食[ちょうしょく]は 大抵[たいてい] 7時頃[しちじごろ]に 食[た]べます。	ちょうしょく は たいてい しちじごろ に たべます	
\\	朝食[ちょうしょく]は
\\	7時頃[しちじごろ]に 食[た]べます。			
\\	大勢	大勢[おおぜい]	おおぜい	
\\	大勢で食事に出かけました。	大勢[おおぜい]で 食事[しょくじ]に 出[で]かけました。	おおぜい で しょくじ に でかけました	
\\	で 食事[しょくじ]に 出[で]かけました。			
\\	姿	姿[すがた]	すがた	
\\	遠くに彼女の姿が見えた。	遠[とお]くに 彼女[かのじょ]の 姿[すがた]が 見[み]えた。	とおく に かのじょ の すがた が みえた	
\\	遠[とお]くに 彼女[かのじょ]の
\\	が 見[み]えた。			
\\	姿勢	姿勢[しせい]	しせい	
\\	あの子はいつも姿勢が悪い。	あの 子[こ]はいつも 姿勢[しせい]が 悪[わる]い。	あの こ は いつも しせい が わるい	
\\	あの 子[こ]はいつも
\\	が 悪[わる]い。			
\\	恐ろしい	恐[おそ]ろしい	おそろしい	
\\	昨夜恐ろしい夢を見た。	昨夜[ゆうべ] 恐[おそ]ろしい 夢[ゆめ]を 見[み]た。	ゆうべ おそろしい ゆめ を みた	
\\	昨夜[ゆうべ]
\\	夢[ゆめ]を 見[み]た。			
\\	怖い	怖[こわ]い	こわい	
\\	私は犬が怖いです。	私[わたし]は 犬[いぬ]が 怖[こわ]いです。	わたし は いぬ が こわい です	
\\	私[わたし]は 犬[いぬ]が
\\	です。			
\\	孫	孫[まご]	まご	
\\	昨日、孫が生まれました。	昨日[きのう]、 孫[まご]が 生[う]まれました。	きのう まご が うまれました	
\\	昨日[きのう]、
\\	が 生[う]まれました。			
\\	木綿	木綿[もめん]	もめん	
\\	彼女は木綿のシャツを着ています。	彼女[かのじょ]は 木綿[もめん]のシャツを 着[き]ています。	かのじょ は もめん の しゃつ を きて います	
\\	彼女[かのじょ]は
\\	のシャツを 着[き]ています。			
\\	机	机[つくえ]	つくえ	
\\	新しい机を買ってもらいました。	新[あたら]しい 机[つくえ]を 買[か]ってもらいました。	あたらしい つくえ を かって もらいました	
\\	新[あたら]しい
\\	を 買[か]ってもらいました。			
\\	棚	棚[たな]	たな	
\\	大きな棚はとても便利です。	大[おお]きな 棚[たな]はとても 便利[べんり]です。	おおき な たな は とても べんり です	
\\	大[おお]きな
\\	はとても 便利[べんり]です。			
\\	本棚	本棚[ほんだな]	ほんだな	
\\	これはとても大きな本棚ですね。	これはとても 大[おお]きな 本棚[ほんだな]ですね。	これ は とても おおき な ほんだな です ね	
\\	これはとても 大[おお]きな
\\	ですね。			
\\	方針	方針[ほうしん]	ほうしん	
\\	今後の方針が決まった。	今後[こんご]の 方針[ほうしん]が 決[き]まった。	こんご の ほうしん が きまった	
\\	今後[こんご]の
\\	が 決[き]まった。			
\\	寿司	寿司[すし]	すし	
\\	彼女は寿司を初めて食べました。	彼女[かのじょ]は 寿司[すし]を 初[はじ]めて 食[た]べました。	かのじょ は すし を はじめて たべました 。	
\\	彼女[かのじょ]は
\\	を 初[はじ]めて 食[た]べました。			
\\	泥棒	泥棒[どろぼう]	どろぼう	
\\	近所に泥棒が入った。	近所[きんじょ]に 泥棒[どろぼう]が 入[はい]った。	きんじょ に どろぼう が はいった	
\\	近所[きんじょ]に
\\	が 入[はい]った。			
\\	沸く	沸[わ]く	わく	
\\	お風呂が沸きました。	お 風呂[ふろ]が 沸[わ]きました。	お ふろ が わきました。	
\\	お 風呂[ふろ]が
\\	沸かす	沸[わ]かす	わかす	
\\	お湯を沸かしてください。	お 湯[ゆ]を 沸[わ]かしてください。	おゆ を わかして ください	
\\	お 湯[ゆ]を
\\	ください。			
\\	洗濯機	洗濯機[せんたくき]	せんたくき	
\\	新しい洗濯機を買いました。	新[あたら]しい 洗濯機[せんたっき]を 買[か]いました。	あたらしい せんたっき を かいました	
\\	新[あたら]しい
\\	を 買[か]いました。			
\\	洗濯	洗濯[せんたく]	せんたく	
\\	一週間、洗濯をしていない。	一週間[いっしゅうかん]、 洗濯[せんたく]をしていない。	いっしゅうかん せんたく を して いない	
\\	一週間[いっしゅうかん]、
\\	をしていない。			
\\	喫茶店	喫茶店[きっさてん]	きっさてん	
\\	喫茶店でコーヒーを飲んだ。	喫茶店[きっさてん]でコーヒーを 飲[の]んだ。	きっさてん で こーひー を のんだ	
\\	でコーヒーを 飲[の]んだ。			
\\	怠ける	怠[なま]ける	なまける	
\\	怠けていないで、手伝って。	怠[なま]けていないで、 手伝[てつだ]って。	なまけて いない で てつだって	
\\	、 手伝[てつだ]って。			
\\	天井	天井[てんじょう]	てんじょう	
\\	この部屋は天井が高いですね。	この 部屋[へや]は 天井[てんじょう]が 高[たか]いですね。	この へや は てんじょう が たかい です ね	
\\	この 部屋[へや]は
\\	が 高[たか]いですね。			
\\	毛	毛[け]	け	
\\	猫の毛がセーターに付いた。	猫[ねこ]の 毛[け]がセーターに 付[つ]いた。	ねこ の け が せーたー に ついた	
\\	猫[ねこ]の
\\	がセーターに 付[つ]いた。			
\\	居る	居[い]る	いる	
\\	今日は一日中家に居ました。	今日[きょう]は 一日中家[いちにちじゅう うち]に 居[い]ました。	きょう は いちにちじゅう うち に いました	
\\	今日[きょう]は 一日中家[いちにちじゅう うち]に
\\	履く	履[は]く	はく	
\\	彼女はブーツを履いています。	彼女[かのじょ]はブーツを 履[は]いています。	かのじょ は ぶーつ を はいて います	
\\	彼女[かのじょ]はブーツを
\\	戸	戸[と]	と	
\\	部屋の戸が開いています。	部屋[へや]の 戸[と]が 開[あ]いています。	へや の と が あいて います	
\\	部屋[へや]の
\\	が 開[あ]いています。			
\\	扇風機	扇風機[せんぷうき]	せんぷうき	
\\	暑いから扇風機をつけよう。	暑[あつ]いから 扇風機[せんぷうき]をつけよう。	あつい から せんぷうき を つけよう	
\\	暑[あつ]いから
\\	をつけよう。			
\\	寝坊	寝坊[ねぼう]	ねぼう	
\\	今朝は寝坊しました。	今朝[けさ]は 寝坊[ねぼう]しました。	けさ は ねぼう しました	
\\	今朝[けさ]は
\\	しました。			
\\	旗	旗[はた]	はた	
\\	旗が風に揺れている。	旗[はた]が 風[かぜ]に 揺[ゆ]れている。	はた が かぜ に ゆれて いる	
\\	が 風[かぜ]に 揺[ゆ]れている。			
\\	本箱	本箱[ほんばこ]	ほんばこ	
\\	雑誌を本箱に入れました。	雑誌[ざっし]を 本箱[ほんばこ]に 入[い]れました。	ざっし を ほんばこ に いれました	
\\	雑誌[ざっし]を
\\	に 入[い]れました。			
\\	手袋	手袋[てぶくろ]	てぶくろ	
\\	寒いので手袋をしました。	寒[さむ]いので 手袋[てぶくろ]をしました。	さむい の で てぶくろ を しました	
\\	寒[さむ]いので
\\	をしました。			
\\	毛布	毛布[もうふ]	もうふ	
\\	この毛布は暖かい。	この 毛布[もうふ]は 暖[あたた]かい。	この もうふ は あたたかい	
\\	この
\\	は 暖[あたた]かい。			
\\	布団	布団[ふとん]	ふとん	
\\	母が布団を干している。	母[はは]が 布団[ふとん]を 干[ほ]している。	はは が ふとん を ほして いる	
\\	母[はは]が
\\	を 干[ほ]している。			
\\	小包	小包[こづつみ]	こづつみ	
\\	フランスの友達から小包が届いた。	フランスの 友達[ともだち]から 小包[こづつみ]が 届[とど]いた。	ふらんす の ともだち から こづつみ が とどいた	
\\	フランスの 友達[ともだち]から
\\	が 届[とど]いた。			
\\	手帳	手帳[てちょう]	てちょう	
\\	新しい手帳を買いました。	新[あたら]しい 手帳[てちょう]を 買[か]いました。	あたらしい てちょう を かいました	
\\	新[あたら]しい
\\	を 買[か]いました。			
\\	封筒	封筒[ふうとう]	ふうとう	
\\	その手紙を封筒に入れた。	その 手紙[てがみ]を 封筒[ふうとう]に 入[い]れた。	その てがみ を ふうとう に いれた	
\\	その 手紙[てがみ]を
\\	に 入[い]れた。			
\\	掛かる	掛[か]かる	かかる	
\\	壁に大きな時計が掛かっています。	壁[かべ]に 大[おお]きな 時計[とけい]が 掛[か]かっています。	かべ に おおき な とけい が かかって います	
\\	壁[かべ]に 大[おお]きな 時計[とけい]が
\\	掛け算	掛[か]け 算[ざん]	かけざん	
\\	弟は掛け算を習っている。	弟[おとうと]は 掛[か]け 算[ざん]を 習[なら]っている。	おとうと は かけざん を ならって いる	
\\	弟[おとうと]は
\\	を 習[なら]っている。			
\\	拍手	拍手[はくしゅ]	はくしゅ	
\\	大きな拍手が上がった。	大[おお]きな 拍手[はくしゅ]が 上[あ]がった。	おおき な はくしゅ が あがった	
\\	大[おお]きな
\\	が 上[あ]がった。			
\\	掃除	掃除[そうじ]	そうじ	
\\	週末は部屋の掃除をしました。	週末[しゅうまつ]は 部屋[へや]の 掃除[そうじ]をしました。	しゅうまつ は へや の そうじ を しました	
\\	週末[しゅうまつ]は 部屋[へや]の
\\	をしました。			
\\	掃く	掃[は]く	はく	
\\	床をほうきで掃きました。	床[ゆか]をほうきで 掃[は]きました。	ゆか を ほうき で はきました	
\\	床[ゆか]をほうきで
\\	掃除機	掃除機[そうじき]	そうじき	
\\	掃除機が壊れた。	掃除機[そうじき]が 壊[こわ]れた。	そうじき が こわれた	
\\	が 壊[こわ]れた。			
\\	握る	握[にぎ]る	にぎる	
\\	少女は母親の手を握った。	少女[しょうじょ]は 母親[ははおや]の 手[て]を 握[にぎ]った。	しょうじょ は ははおや の て を にぎった	
\\	少女[しょうじょ]は 母親[ははおや]の 手[て]を
\\	握手	握手[あくしゅ]	あくしゅ	
\\	彼らは握手をした。	彼[かれ]らは 握手[あくしゅ]をした。	かれら は あくしゅ を した	
\\	彼[かれ]らは
\\	をした。			
\\	幾つ	幾[いく]つ	いくつ	
\\	娘さんは幾つになりましたか。	娘[むすめ]さんは 幾[いく]つになりましたか。	むすめさん は いくつ に なりました か	
\\	娘[むすめ]さんは
\\	になりましたか。			
\\	幾ら	幾[いく]ら	いくら	
\\	この靴は幾らですか。	この 靴[くつ]は 幾[いく]らですか。	この くつ は いくら です か	
\\	この 靴[くつ]は
\\	ですか。			
\\	寂しい	寂[さび]しい	さびしい	
\\	これは寂しい曲ですね。	これは 寂[さび]しい 曲[きょく]ですね。	これ は さびしい きょく です ね	
\\	これは
\\	曲[きょく]ですね。			
\\	可哀相	可哀相[かわいそう]	かわいそう	
\\	その可哀相な子供たちは食べるものがない。	その 可哀相[かわいそう]な 子供[こども]たちは 食[た]べるものがない。	その かわいそう な こどもたち は たべる もの が ない	
\\	その
\\	な 子供[こども]たちは 食[た]べるものがない。			
\\	怪我	怪我[けが]	けが	
\\	彼女は腕を怪我した。	彼女[かのじょ]は 腕[うで]を 怪我[けが]した。	かのじょ は うで を けがした	
\\	彼女[かのじょ]は 腕[うで]を
\\	した。			
\\	我慢	我慢[がまん]	がまん	
\\	彼のわがままには我慢できません。	彼[かれ]のわがままには 我慢[がまん]できません。	かれ の わがまま に は がまん できません	
\\	彼[かれ]のわがままには
\\	できません。			
\\	幼稚園	幼稚園[ようちえん]	ようちえん	
\\	娘は幼稚園に通っています。	娘[むすめ]は 幼稚園[ようちえん]に 通[かよ]っています。	むすめ は ようちえん に かよって います	
\\	娘[むすめ]は
\\	に 通[かよ]っています。			
\\	日陰	日陰[ひかげ]	ひかげ	
\\	暑いので日陰で休みましょう。	暑[あつ]いので 日陰[ひかげ]で 休[やす]みましょう。	あつい の で ひかげ で やすみましょう	
\\	暑[あつ]いので
\\	で 休[やす]みましょう。			
\\	尋ねる	尋[たず]ねる	たずねる	
\\	彼女は道を尋ねた。	彼女[かのじょ]は 道[みち]を 尋[たず]ねた。	かのじょ は みち を たずねた	
\\	彼女[かのじょ]は 道[みち]を
\\	床	床[ゆか]	ゆか	
\\	床がぬれている。	床[ゆか]がぬれている。	ゆか が ぬれて いる	
\\	がぬれている。			
\\	床屋	床屋[とこや]	とこや	
\\	昨日床屋で髪を切った。	昨日[きのう] 床屋[とこや]で 髪[かみ]を 切[き]った。	きのう とこや で かみ を きった	
\\	昨日[きのう]
\\	で 髪[かみ]を 切[き]った。			
\\	干す	干[ほ]す	ほす	
\\	母は洗濯物を干しています。	母[はは]は 洗濯物[せんたくもの]を 干[ほ]しています。	はは は せんたくもの を ほして います	
\\	母[はは]は 洗濯物[せんたくもの]を
\\	います。			
\\	帽子	帽子[ぼうし]	ぼうし	
\\	暑いので帽子を被りましょう。	暑[あつ]いので 帽子[ぼうし]を 被[かぶ]りましょう。	あつい の で ぼうし を かぶりましょう	
\\	暑[あつ]いので
\\	を 被[かぶ]りましょう。			
\\	是非	是非[ぜひ]	ぜひ	
\\	是非、うちに来てください。	是非[ぜひ]、うちに 来[き]てください。	ぜひ うち に きて ください	
\\	、うちに 来[き]てください。			
\\	敬語	敬語[けいご]	けいご	
\\	お客様には敬語を使いなさい。	お 客様[きゃくさま]には 敬語[けいご]を 使[つか]いなさい。	おきゃくさま に は けいご を つかいなさい	
\\	お 客様[きゃくさま]には
\\	を 使[つか]いなさい。			
\\	尊敬	尊敬[そんけい]	そんけい	
\\	祖父は家族みんなに尊敬されています。	祖父[そふ]は 家族[かぞく]みんなに 尊敬[そんけい]されています。	そふ は かぞく みんな に そんけい されて います	
\\	祖父[そふ]は 家族[かぞく]みんなに
\\	されています。			
\\	敷く	敷[し]く	しく	
\\	生まれて初めて布団を敷いた。	生[う]まれて 初[はじ]めて 布団[ふとん]を 敷[し]いた。	うまれて はじめて ふとん を しいた	
\\	生[う]まれて 初[はじ]めて 布団[ふとん]を
\\	文章	文章[ぶんしょう]	ぶんしょう	
\\	彼は文章がとてもうまい。	彼[かれ]は 文章[ぶんしょう]がとてもうまい。	かれ は ぶんしょう が とても うまい	
\\	彼[かれ]は
\\	がとてもうまい。			
\\	昭和	昭和[しょうわ]	しょうわ	
\\	私の両親は昭和生まれです。	私[わたし]の 両親[りょうしん]は 昭和[しょうわ] 生[う]まれです。	わたし の りょうしん は しょうわうまれ です	
\\	私[わたし]の 両親[りょうしん]は
\\	生[う]まれです。			
\\	梅雨	梅雨[つゆ]	つゆ	
\\	梅雨は6月頃です。	梅雨[つゆ]は 6月頃[ろくがつごろ]です。	つゆ は ろくがつごろ です	
\\	は 6月頃[ろくがつごろ]です。			
\\	桃	桃[もも]	もも	
\\	私の一番好きな果物は桃です。	私[わたし]の 一番好[いちばん す]きな 果物[くだもの]は 桃[もも]です。	わたし の いちばん すき な くだもの は もも です	
\\	私[わたし]の 一番好[いちばん す]きな 果物[くだもの]は
\\	です。			
\\	枕	枕[まくら]	まくら	
\\	私は低い枕が好きです。	私[わたし]は 低[ひく]い 枕[まくら]が 好[す]きです。	わたし は ひくい まくら が すき です	
\\	私[わたし]は 低[ひく]い
\\	が 好[す]きです。			
\\	嘘	嘘[うそ]	うそ	
\\	嘘をついてはいけません。	嘘[うそ]をついてはいけません。	うそ を ついて は いけません	
\\	をついてはいけません。			
\\	叱る	叱[しか]る	しかる	
\\	父親が子供を叱っている。	父親[ちちおや]が 子供[こども]を 叱[しか]っている。	ちちおや が こども を しかって いる	
\\	父親[ちちおや]が 子供[こども]を
\\	年賀状	年賀状[ねんがじょう]	ねんがじょう	
\\	昨日、年賀状を出しました。	昨日[きのう]、 年賀状[ねんがじょう]を 出[だ]しました。	きのう ねんがじょう を だしました	
\\	昨日[きのう]、
\\	を 出[だ]しました。			
\\	拭く	拭[ふ]く	ふく	
\\	タオルで体を拭きました。	タオルで 体[からだ]を 拭[ふ]きました。	たおる で からだ を ふきました	
\\	タオルで 体[からだ]を
\\	挨拶	挨拶[あいさつ]	あいさつ	
\\	彼女は笑顔で挨拶した。	彼女[かのじょ]は 笑顔[えがお]で 挨拶[あいさつ]した。	かのじょ は えがお で あいさつ した	
\\	彼女[かのじょ]は 笑顔[えがお]で
\\	した。			
\\	巻く	巻[ま]く	まく	
\\	彼は頭にタオルを巻いていた。	彼[かれ]は 頭[あたま]にタオルを 巻[ま]いていた。	かれ は あたま に たおる を まいて いた	
\\	彼[かれ]は 頭[あたま]にタオルを
\\	歯磨き	歯磨[はみが]き	はみがき	
\\	歯磨きはしましたか。	歯磨[はみが]きはしましたか。	はみがき は しました か	
\\	はしましたか。			
\\	廊下	廊下[ろうか]	ろうか	
\\	廊下は走らないでください。	廊下[ろうか]は 走[はし]らないでください。	ろうか は はしらない で ください	
\\	は 走[はし]らないでください。			
\\	喧嘩	喧嘩[けんか]	けんか	
\\	喧嘩はやめて。	喧嘩[けんか]はやめて。	けんか は やめて	
\\	はやめて。			
\\	叩く	叩[たた]く	たたく	
\\	彼は子供のおしりを叩いた。	彼[かれ]は 子供[こども]のおしりを 叩[たた]いた。	かれ は こども の おしり を たたいた	
\\	彼[かれ]は 子供[こども]のおしりを
\\	噛む	噛[か]む	かむ	
\\	もっとよく噛みなさい。	もっとよく 噛[か]みなさい。	もっと よく かみなさい	
\\	もっとよく
\\	味噌汁	味噌汁[みそしる]	みそしる	
\\	私は毎日味噌汁を飲みます。	私[わたし]は 毎日[まいにち] 味噌汁[みそしる]を 飲[の]みます。	わたし は まいにち みそしる を のみます 。	
\\	私[わたし]は 毎日[まいにち]
\\	を 飲[の]みます。			
\\	姪	姪[めい]	めい	
\\	私の姪は3才です。	私[わたし]の 姪[めい]は 3才[さんさい]です。	わたし の めい は さんさい です	
\\	私[わたし]の
\\	は 3才[さんさい]です。			
\\	椅子	椅子[いす]	いす	
\\	そのお年寄りは椅子に座った。	そのお 年寄[としよ]りは 椅子[いす]に 座[すわ]った。	その お としより は いす に すわった 。	
\\	そのお 年寄[としよ]りは
\\	に 座[すわ]った。			
\\	月日	月日[がっぴ]	がっぴ	
\\	ここに生年月日を記入してください。	ここに 生年[せいねん] 月日[がっぴ]を 記入[きにゅう]してください。	ここ に せいねんがっぴ を きにゅう して ください	
\\	ここに 生年[せいねん]
\\	を 記入[きにゅう]してください。			
\\	帰す	帰[かえ]す	かえす	
\\	学校は生徒たちを午前中に帰したね。	学校[がっこう]は 生徒[せいと]たちを 午前中[ごぜんちゅう]に 帰[かえ]したね。	がっこう は せいとたち を ごぜんちゅう に かえした ね	
\\	学校[がっこう]は 生徒[せいと]たちを 午前中[ごぜんちゅう]に
\\	ね。			
\\	大いに	大[おお]いに	おおいに	
\\	大いに学び、大いに遊びなさい。	大[おお]いに 学[まな]び、 大[おお]いに 遊[あそ]びなさい。	おおいに まなび おおいに あそびなさい	
\\	学[まな]び、 大[おお]いに 遊[あそ]びなさい。			
\\	大げさ	大[おお]げさ	おおげさ	
\\	彼の話は大げさだ。	彼[かれ]の 話[はなし]は 大[おお]げさだ。	かれ の はなし は おおげさ だ	
\\	彼[かれ]の 話[はなし]は
\\	だ。			
\\	大水	大水[おおみず]	おおみず	
\\	その年、この地域では大水がありました。	その 年[とし]、この 地域[ちいき]では 大水[おおみず]がありました。	その とし この ちいき で は おおみず が ありました	
\\	その 年[とし]、この 地域[ちいき]では
\\	がありました。			
\\	水中	水中[すいちゅう]	すいちゅう	
\\	このカメラなら水中の写真が撮れますね。	このカメラなら 水中[すいちゅう]の 写真[しゃしん]が 撮[と]れますね。	この かめら なら すいちゅう の しゃしん が とれます ね	
\\	このカメラなら
\\	の 写真[しゃしん]が 撮[と]れますね。			
\\	小	小[しょう]	しょう	
\\	この箱の小をください。	この 箱[はこ]の 小[しょう]をください。	この はこ の しょう を ください	
\\	この 箱[はこ]の
\\	をください。			
\\	少なくとも	少[すく]なくとも	すくなくとも	
\\	この仕事には少なくとも2週間必要でしょう。	この 仕事[しごと]には 少[すく]なくとも 2週間必要[にしゅうかん ひつよう]でしょう。	この しごと に は すくなくとも にしゅうかん ひつよう でしょう	
\\	この 仕事[しごと]には
\\	2週間必要[にしゅうかん ひつよう]でしょう。			
\\	少々	少々[しょうしょう]	しょうしょう	
\\	少々のことは我慢します。	少々[しょうしょう]のことは 我慢[がまん]します。	しょうしょう の こと は がまん します	
\\	のことは 我慢[がまん]します。			
\\	左右	左右[さゆう]	さゆう	
\\	左右を見てから横断歩道を渡りなさい。	左右[さゆう]を 見[み]てから 横断歩道[おうだん ほどう]を 渡[わた]りなさい。	さゆう を みてから おうだん ほどう を わたりなさい	
\\	を 見[み]てから 横断歩道[おうだん ほどう]を 渡[わた]りなさい。			
\\	四方	四方[しほう]	しほう	
\\	火が四方に広がったんだよ。	火[ひ]が 四方[しほう]に 広[ひろ]がったんだよ。	ひ が しほう に ひろがった ん だ よ	
\\	火[ひ]が
\\	に 広[ひろ]がったんだよ。			
\\	方々	方々[かたがた]	かたがた	
\\	大勢の方々にご出席いただきました。	大勢[おおぜい]の 方々[かたがた]にご 出席[しゅっせき]いただきました。	おおぜい の かたがた に ごしゅっせき いただきました	
\\	大勢[おおぜい]の
\\	にご 出席[しゅっせき]いただきました。			
\\	大人	大人[おとな]	おとな	
\\	あなたもだいぶ大人になったね。	あなたもだいぶ 大人[おとな]になったね。	あなた も だいぶ おとな に なった ね	
\\	あなたもだいぶ
\\	になったね。			
\\	外出	外出[がいしゅつ]	がいしゅつ	
\\	午後は外出の予定です。	午後[ごご]は 外出[がいしゅつ]の 予定[よてい]です。	ごご は がいしゅつ の よてい です	
\\	午後[ごご]は
\\	の 予定[よてい]です。			
\\	水力	水力[すいりょく]	すいりょく	
\\	この島は発電を水力に頼っているんだ。	この 島[しま]は 発電[はつでん]を 水力[すいりょく]に 頼[たよ]っているんだ。	この しま は はつでん を すいりょく に たよって いる ん だ	
\\	この 島[しま]は 発電[はつでん]を
\\	に 頼[たよ]っているんだ。			
\\	口げんか	口[くち]げんか	くちげんか	
\\	きのう、弟と口げんかしました。	きのう、 弟[おとうと]と 口[くち]げんかしました。	きのう おとうと と くちげんか しました	
\\	きのう、 弟[おとうと]と
\\	しました。			
\\	大手	大手[おおて]	おおて	
\\	彼は大手のメーカーに勤めています。	彼[かれ]は 大手[おおて]のメーカーに 勤[つと]めています。	かれ は おおて の めーかー に つとめて います	
\\	彼[かれ]は
\\	のメーカーに 勤[つと]めています。			
\\	小川	小川[おがわ]	おがわ	
\\	小川がさらさら流れています。	小川[おがわ]がさらさら 流[なが]れています。	おがわ が さらさら ながれて います	
\\	がさらさら 流[なが]れています。			
\\	海水	海水[かいすい]	かいすい	
\\	海水から塩を作ります。	海水[かいすい]から 塩[しお]を 作[つく]ります。	かいすい から しお を つくります	
\\	から 塩[しお]を 作[つく]ります。			
\\	海上	海上[かいじょう]	かいじょう	
\\	海上で衝突事故が発生した。	海上[かいじょう]で 衝突事故[しょうとつ じこ]が 発生[はっせい]した。	かいじょう で しょうとつ じこ が はっせい した	
\\	で 衝突事故[しょうとつ じこ]が 発生[はっせい]した。			
\\	水田	水田[すいでん]	すいでん	
\\	窓の外に水田が広がっていたよ。	窓[まど]の 外[そと]に 水田[すいでん]が 広[ひろ]がっていたよ。	まど の そと に すいでん が ひろがって いた よ	
\\	窓[まど]の 外[そと]に
\\	が 広[ひろ]がっていたよ。			
\\	森林	森林[しんりん]	しんりん	
\\	世界中で森林が失われています。	世界中[せかいじゅう]で 森林[しんりん]が 失[うしな]われています。	せかいじゅう で しんりん が うしなわれて います	
\\	世界中[せかいじゅう]で
\\	が 失[うしな]われています。			
\\	女らしい	女[おんな]らしい	おんならしい	
\\	彼女は女らしい。	彼女[かのじょ]は 女[おんな]らしい。	かのじょ は おんならしい	
\\	彼女[かのじょ]は
\\	少女	少女[しょうじょ]	しょうじょ	
\\	少女は母親の手を握った。	少女[しょうじょ]は 母親[ははおや]の 手[て]を握[にぎ]った。	しょうじょ は ははおや の て を にぎった	
\\	は 母親[ははおや]の 手[て]を 握[にぎ]った。			
\\	女子	女子[じょし]	じょし	
\\	このクラスの女子は18人です。	このクラスの 女子[じょし]は 18人[じゅうはちにん]です。	この くらす の じょし は じゅうはちにん です	
\\	このクラスの
\\	は 18人[じゅうはちにん]です。			
\\	好む	好[この]む	このむ	
\\	彼女は背の高い男性を好みますね。	彼女[かのじょ]は 背[せ]の 高[たか]い 男性[だんせい]を 好[この]みますね。	かのじょ は せ の たかい だんせい を このみます ね	
\\	彼女[かのじょ]は 背[せ]の 高[たか]い 男性[だんせい]を
\\	ね。			
\\	好み	好[この]み	このみ	
\\	姉と私は服の好みが似ています。	姉[あね]と 私[わたし]は 服[ふく]の 好[この]みが 似[に]ています。	あね と わたし は ふく の このみ が にて います	
\\	姉[あね]と 私[わたし]は 服[ふく]の
\\	が 似[に]ています。			
\\	家出	家出[いえで]	いえで	
\\	彼の息子が家出したそうよ。	彼[かれ]の 息子[むすこ]が 家出[いえで]したそうよ。	かれ の むすこ が いえで した そう よ	
\\	彼[かれ]の 息子[むすこ]が
\\	したそうよ。			
\\	大家	大家[おおや]	おおや	
\\	ここの大家は近くに住んでいますよ。	ここの 大家[おおや]は 近[ちか]くに 住[す]んでいますよ。	ここ の おおや は ちかく に すんで います よ	
\\	ここの
\\	は 近[ちか]くに 住[す]んでいますよ。			
\\	天の川	天[あま]の 川[がわ]	あまのがわ	
\\	今夜は天の川が見えますね。	今夜[こんや]は 天[あま]の 川[がわ]が 見[み]えますね。	こんや は あまのがわ が みえます ね	
\\	今夜[こんや]は
\\	が 見[み]えますね。			
\\	気分	気分[きぶん]	きぶん	
\\	今日は最高にいい気分だよ。	今日[きょう]は 最高[さいこう]にいい 気分[きぶん]だよ。	きょう は さいこう に いい きぶん だ よ	
\\	今日[きょう]は 最高[さいこう]にいい
\\	だよ。			
\\	気体	気体[きたい]	きたい	
\\	水が沸騰して気体になったんだ。	水[みず]が 沸騰[ふっとう]して 気体[きたい]になったんだ。	みず が ふっとう して きたい に なった ん だ	
\\	水[みず]が 沸騰[ふっとう]して
\\	になったんだ。			
\\	気力	気力[きりょく]	きりょく	
\\	彼は気力にあふれていますね。	彼[かれ]は 気力[きりょく]にあふれていますね。	かれ は きりょく に あふれて います ね	
\\	彼[かれ]は
\\	にあふれていますね。			
\\	大雨	大雨[おおあめ]	おおあめ	
\\	各地で大雨が降っています。	各地[かくち]で 大雨[おおあめ]が 降[ふ]っています。	かくち で おおあめ が ふって います	
\\	各地[かくち]で
\\	が 降[ふ]っています。			
\\	小雨	小雨[こさめ]	こさめ	
\\	小雨なので傘はいりません。	小雨[こさめ]なので 傘[かさ]はいりません。	こさめ な の で かさ は いりません	
\\	なので 傘[かさ]はいりません。			
\\	大雪	大雪[おおゆき]	おおゆき	
\\	10年振りの大雪です。	10年振[じゅうねん ぶ]りの 大雪[おおゆき]です。	じゅうねん ぶり の おおゆき です	
\\	10年振[じゅうねん ぶ]りの
\\	です。			
\\	明日	明日[あす]	あす	
\\	明日のプレゼンテーションが心配だ。	明日[あす]のプレゼンテーションが 心配[しんぱい]だ。	あす の ぷれぜんてーしょん が しんぱい だ	
\\	の プレゼンテーションが 心配[しんぱい]だ。			
\\	明ける	明[あ]ける	あける	
\\	もうすぐ夜が明けるね。	もうすぐ 夜[よ]が 明[あ]けるね。	もうすぐ よ が あける ね	
\\	もうすぐ 夜[よ]が
\\	ね。			
\\	明かり	明[あ]かり	あかり	
\\	部屋の明かりを点けましょう。	部屋[へや]の 明[あ]かりを 点[つ]けましょう。	へや の あかり を つけましょう	
\\	部屋[へや]の
\\	を 点[つ]けましょう。			
\\	明け方	明[あ]け 方[がた]	あけがた	
\\	明け方に雨が降り始めましたね。	明[あ]け 方[がた]に 雨[あめ]が 降[ふ]り 始[はじ]めましたね。	あけがた に あめ が ふりはじめました ね	
\\	に 雨[あめ]が 降[ふ]り 始[はじ]めましたね。			
\\	昨日	昨日[さくじつ]	さくじつ	
\\	昨日は雨でしたね。	昨日[さくじつ]は 雨[あめ]でしたね。	さくじつ は あめ でした ね	
\\	は 雨[あめ]でしたね。			
\\	向上	向上[こうじょう]	こうじょう	
\\	全員で技術の向上に努めています。	全員[ぜんいん]で 技術[ぎじゅつ]の 向上[こうじょう]に 努[つと]めています。	ぜんいん で ぎじゅつ の こうじょう に つとめて います	
\\	全員[ぜんいん]で 技術[ぎじゅつ]の
\\	に 努[つと]めています。			
\\	客間	客間[きゃくま]	きゃくま	
\\	お客さんを客間にお通ししたわよ。	お 客[きゃく]さんを 客間[きゃくま]にお 通[とお]ししたわよ。	おきゃくさん を きゃくま に おとおし した わ よ	
\\	お 客[きゃく]さんを
\\	にお 通[とお]ししたわよ。			
\\	最高	最高[さいこう]	さいこう	
\\	これまでで最高の結果が出たよ。	これまでで 最高[さいこう]の 結果[けっか]が 出[で]たよ。	これまで で さいこう の けっか が でた よ	
\\	これまでで
\\	の 結果[けっか]が 出[で]たよ。			
\\	最低	最低[さいてい]	さいてい	
\\	これは今までで最低の記録だ。	これは 今[いま]までで 最低[さいてい]の 記録[きろく]だ。	これ は いま まで で さいてい の きろく だ	
\\	これは 今[いま]までで
\\	の 記録[きろく]だ。			
\\	最小	最小[さいしょう]	さいしょう	
\\	これは世界で最小のコンピューターです。	これは 世界[せかい]で 最小[さいしょう]のコンピューターです。	これ は せかい で さいしょう の こんぴゅーたー です	
\\	これは 世界[せかい]で
\\	のコンピューターです。			
\\	最上	最上[さいじょう]	さいじょう	
\\	このホテルでは最上のサービスが受けられます。	このホテルでは 最上[さいじょう]のサービスが 受[う]けられます。	この ほてる で は さいじょう の さーびす が うけられます	
\\	このホテルでは
\\	のサービスが 受[う]けられます。			
\\	最中	最中[さいちゅう]	さいちゅう	
\\	夕食の最中に電話がかかってきたの。	夕食[ゆうしょく]の 最中[さいちゅう]に 電話[でんわ]がかかってきたの。	ゆうしょく の さいちゅう に でんわ が かかって きた の	
\\	夕食[ゆうしょく]の
\\	に 電話[でんわ]がかかってきたの。			
\\	後方	後方[こうほう]	こうほう	
\\	彼は後方の座席に着いたの。	彼[かれ]は 後方[こうほう]の 座席[ざせき]に 着[つ]いたの。	かれ は こうほう の ざせき に ついた の	
\\	彼[かれ]は
\\	の 座席[ざせき]に 着[つ]いたの。			
\\	後ろ向き	後[うし]ろ 向[む]き	うしろむき	
\\	彼は車を後ろ向きに駐車したの。	彼[かれ]は 車[くるま]を 後[うし]ろ 向[む]きに 駐車[ちゅうしゃ]したの。	かれ は くるま を うしろむき に ちゅうしゃ した の	
\\	彼[かれ]は 車[くるま]を
\\	に 駐車[ちゅうしゃ]したの。			
\\	明々後日	明々後日[しあさって]	しあさって	
\\	会議は明々後日に延期された。	会議[かいぎ]は 明々後日[しあさって]に 延期[えんき]された。	かいぎ は しあさって に えんき された	
\\	会議[かいぎ]は
\\	に 延期[えんき]された。			
\\	後半	後半[こうはん]	こうはん	
\\	ドラマの後半は来週放送されるんだ。	ドラマの 後半[こうはん]は 来週放送[らいしゅう ほうそう]されるんだ。	どらま の こうはん は らいしゅう ほうそう される ん だ	
\\	ドラマの
\\	は 来週放送[らいしゅう ほうそう]されるんだ。			
\\	朝日	朝日[あさひ]	あさひ	
\\	朝日が昇りましたよ。	朝日[あさひ]が 昇[のぼ]りましたよ。	あさひ が のぼりました よ	
\\	が 昇[のぼ]りましたよ。			
\\	昨晩	昨晩[さくばん]	さくばん	
\\	昨晩の雪がまだ庭に残っている。	昨晩[さくばん]の 雪[ゆき]がまだ 庭[にわ]に 残[のこ]っている。	さくばん の ゆき が まだ にわ に のこって いる	
\\	の 雪[ゆき]がまだ 庭[にわ]に 残[のこ]っている。			
\\	昨夜	昨夜[さくや]	さくや	
\\	昨夜はテレビで喜劇を見たよ。	昨夜[さくや]はテレビで 喜劇[きげき]を 見[み]たよ。	さくや は てれび で きげき を みた よ	
\\	は テレビで 喜劇[きげき]を 見[み]たよ。			
\\	外食	外食[がいしょく]	がいしょく	
\\	たまには外食しましょう。	たまには 外食[がいしょく]しましょう。	たま に は がいしょく しましょう	
\\	たまには
\\	しましょう。			
\\	外来語	外来語[がいらいご]	がいらいご	
\\	外来語は一般にカタカナで書かれます。	外来語[がいらいご]は 一般[いっぱん]にカタカナで 書[か]かれます。	がいらいご は いっぱんに かたかな で かかれます	
\\	は 一般[いっぱん]にカタカナで 書[か]かれます。			
\\	後書き	後書[あとが]き	あとがき	
\\	後書きをよく読んで下さい。	後書[あとが]きをよく 読[よ]んで 下[くだ]さい。	あとがき を よく よんで ください	
\\	をよく 読[よ]んで 下[くだ]さい。			
\\	合わせる	合[あ]わせる	あわせる	
\\	日時はご都合に合わせます。	日時[にちじ]はご 都合[つごう]に 合[あ]わせます。	にちじ は ごつごう に あわせます	
\\	日時[にちじ]はご 都合[つごう]に
\\	合わす	合[あ]わす	あわす	
\\	赤に黄色を合わすと何色になりますか。	赤[あか]に 黄色[きいろ]を 合[あ]わすと 何色[なにいろ]になりますか。	あか に きいろ を あわすと なにいろ に なります か	
\\	赤[あか]に 黄色[きいろ]を
\\	と 何色[なにいろ]になりますか。			
\\	家事	家事[かじ]	かじ	
\\	母は毎日てきぱきと家事をしているよ。	母[はは]は 毎日[まいにち]てきぱきと 家事[かじ]をしているよ。	はは は まいにち てきぱき と かじ を して いる よ	
\\	母[はは]は 毎日[まいにち]てきぱきと
\\	をしているよ。			
\\	外交	外交[がいこう]	がいこう	
\\	政府は外交に力を入れているの。	政府[せいふ]は 外交[がいこう]に 力[ちから]を 入[い]れているの。	せいふ は がいこう に ちから を いれて いる の	
\\	政府[せいふ]は
\\	に 力[ちから]を 入[い]れているの。			
\\	大通り	大通[おおどお]り	おおどおり	
\\	大通りでパレードが始まるよ。	大通[おおどお]りでパレードが 始[はじ]まるよ。	おおどおり で ぱれーど が はじまる よ	
\\	でパレードが 始[はじ]まるよ。			
\\	書道	書道[しょどう]	しょどう	
\\	書道をすると姿勢も良くなります。	書道[しょどう]をすると 姿勢[しせい]も 良[よ]くなります。	しょどう を する と しせい も よく なります	
\\	をすると 姿勢[しせい]も 良[よ]くなります。			
\\	図	図[ず]	ず	
\\	図を描いて説明しましょう。	図[ず]を 描[か]いて 説明[せつめい]しましょう。	ず を かいて せつめい しましょう	
\\	を 描[か]いて 説明[せつめい]しましょう。			
\\	合図	合図[あいず]	あいず	
\\	車掌が発車の合図をしたよ。	車掌[しゃしょう]が 発車[はっしゃ]の 合図[あいず]をしたよ。	しゃしょう が はっしゃ の あいず を した よ	
\\	車掌[しゃしょう]が 発車[はっしゃ]の
\\	をしたよ。			
\\	外部	外部[がいぶ]	がいぶ	
\\	これは外部には秘密です。	これは 外部[がいぶ]には 秘密[ひみつ]です。	これ は がいぶ に は ひみつ です	
\\	これは
\\	には 秘密[ひみつ]です。			
\\	国家	国家[こっか]	こっか	
\\	首相は国家のリーダーだ。	首相[しゅしょう]は 国家[こっか]のリーダーだ。	しゅしょう は こっか の りーだー だ	
\\	首相[しゅしょう]は
\\	のリーダーだ。			
\\	国々	国々[くにぐに]	くにぐに	
\\	そのマラソン大会にはたくさんの国々から選手が集まったよ。	そのマラソン 大会[たいかい]にはたくさんの 国々[くにぐに]から 選手[せんしゅ]が 集[あつ]まったよ。	その まらそん たいかい に は たくさん の くにぐに から せんしゅ が あつまった よ	
\\	そのマラソン 大会[たいかい]にはたくさんの
\\	から 選手[せんしゅ]が 集[あつ]まったよ。			
\\	国外	国外[こくがい]	こくがい	
\\	犯人は国外に逃げたようです。	犯人[はんにん]は 国外[こくがい]に 逃[に]げたようです。	はんにん は こくがい に にげた よう です	
\\	犯人[はんにん]は
\\	に 逃[に]げたようです。			
\\	国土	国土[こくど]	こくど	
\\	わが国の国土は70
\\	が森林です。	わが 国[くに]の 国土[こくど]は 
\\	[ななじゅっぱーせんと]が 森林[しんりん]です。	わがくに の こくど は ななじゅっぱーせんと が しんりん です	
\\	わが 国[くに]の
\\	は 
\\	[ななじゅっぱーせんと]が 森林[しんりん]です。			
\\	国語	国語[こくご]	こくご	
\\	今日の1時間目は国語です。	今日[きょう]の 1時間目[いちじかんめ]は 国語[こくご]です。	きょう の いちじかんめ は こくご です	
\\	今日[きょう]の 1時間目[いちじかんめ]は
\\	です。			
\\	国交	国交[こっこう]	こっこう	
\\	あの国とは国交がない。	あの 国[くに]とは 国交[こっこう]がない。	あの くに と は こっこう が ない	
\\	あの 国[くに]とは
\\	がない。			
\\	国道	国道[こくどう]	こくどう	
\\	この道をまっすぐ進むと国道に出ます。	この 道[みち]をまっすぐ 進[すす]むと 国道[こくどう]に 出[で]ます。	この みち を まっすぐ すすむ と こくどう に でます	
\\	この 道[みち]をまっすぐ 進[すす]むと
\\	に 出[で]ます。			
\\	国鉄	国鉄[こくてつ]	こくてつ	
\\	父は以前、国鉄に勤めていました。	父[ちち]は 以前[いぜん]、 国鉄[こくてつ]に 勤[つと]めていました。	ちち は いぜん こくてつ に つとめて いました	
\\	父[ちち]は 以前[いぜん]、
\\	に 勤[つと]めていました。			
\\	家屋	家屋[かおく]	かおく	
\\	私は木造の家屋が好きです。	私[わたし]は 木造[もくぞう]の 家屋[かおく]が 好[す]きです。	わたし は もくぞう の かおく が すき です	
\\	私[わたし]は 木造[もくぞう]の
\\	が 好[す]きです。			
\\	屋上	屋上[おくじょう]	おくじょう	
\\	屋上から富士山が見えました。	屋上[おくじょう]から 富士山[ふじさん]が 見[み]えました。	おくじょう から ふじさん が みえました	
\\	から 富士山[ふじさん]が 見[み]えました。			
\\	味わう	味[あじ]わう	あじわう	
\\	母の手料理をゆっくり味わいました。	母[はは]の 手料理[てりょうり]をゆっくり 味[あじ]わいました。	はは の てりょうり を ゆっくり あじわいました	
\\	母[はは]の 手料理[てりょうり]をゆっくり
\\	地味	地味[じみ]	じみ	
\\	今日、彼女は地味な服装をしていますね。	今日[きょう]、 彼女[かのじょ]は 地味[じみ]な 服装[ふくそう]をしていますね。	きょう かのじょ は じみ な ふくそう を して います ね	
\\	今日[きょう]、 彼女[かのじょ]は
\\	な 服装[ふくそう]をしていますね。			
\\	月末	月末[げつまつ]	げつまつ	
\\	月末までに申込書を送ってください。	月末[げつまつ]までに 申込書[もうしこみしょ]を 送[おく]ってください。	げつまつ まで に もうしこみしょ を おくって ください	
\\	までに 申込書[もうしこみしょ]を 送[おく]ってください。			
\\	末っ子	末[すえ]っ 子[こ]	すえっこ	
\\	彼は5人兄弟の末っ子です。	彼[かれ]は 5人兄弟[ごにん きょうだい]の 末[すえ]っ 子[こ]です。	かれ は ごにん きょうだい の すえっこ です	
\\	彼[かれ]は 5人兄弟[ごにん きょうだい]の
\\	です。			
\\	末	末[すえ]	すえ	
\\	長い話合いの末、やっと同意に至った。	長[なが]い 話合[はなしあ]いの 末[すえ]、やっと 同意[どうい]に 至[いた]った。	ながい はなしあい の すえ やっと どうい に いたった	
\\	長[なが]い 話合[はなしあ]いの
\\	、やっと 同意[どうい]に 至[いた]った。			
\\	有りのまま	有[あ]りのまま	ありのまま	
\\	有りのままを話して下さい。	有[あ]りのままを 話[はな]して 下[くだ]さい。	ありのまま を はなして ください	
\\	を 話[はな]して 下[くだ]さい。			
\\	国費	国費[こくひ]	こくひ	
\\	彼は国費で留学しています。	彼[かれ]は 国費[こくひ]で 留学[りゅうがく]しています。	かれ は こくひ で りゅうがく して います	
\\	彼[かれ]は
\\	で 留学[りゅうがく]しています。			
\\	売り上げ	売[う]り 上[あ]げ	うりあげ	
\\	この会社の売り上げは昨年の2倍ね。	この 会社[かいしゃ]の 売[う]り 上[あ]げは 昨年[さくねん]の 2倍[に ばい]ね。	この かいしゃ の うりあげ は さくねん の に ばい ね	
\\	この 会社[かいしゃ]の
\\	は 昨年[さくねん]の 2倍[に ばい]ね。			
\\	売り出す	売[う]り 出[だ]す	うりだす	
\\	新しい車が売り出された。	新[あたら]しい 車[くるま]が 売[う]り 出[だ]された。	あたらしい くるま が うりだされた	
\\	新[あたら]しい 車[くるま]が
\\	書店	書店[しょてん]	しょてん	
\\	駅前に新しい書店ができました。	駅前[えきまえ]に 新[あたら]しい 書店[しょてん]ができました。	えきまえ に あたらしい しょてん が できました	
\\	駅前[えきまえ]に 新[あたら]しい
\\	ができました。			
\\	小売店	小売店[こうりてん]	こうりてん	
\\	この商品は小売店でも買えます。	この 商品[しょうひん]は 小売店[こうりてん]でも 買[か]えます。	この しょうひん は こうりてん で も かえます	
\\	この 商品[しょうひん]は
\\	でも 買[か]えます。			
\\	商社	商社[しょうしゃ]	しょうしゃ	
\\	兄は商社に勤めています。	兄[あに]は 商社[しょうしゃ]に 勤[つと]めています。	あに は しょうしゃ に つとめて います	
\\	兄[あに]は
\\	に 勤[つと]めています。			
\\	商店	商店[しょうてん]	しょうてん	
\\	この通りには商店が多いね。	この 通[とお]りには 商店[しょうてん]が 多[おお]いね。	この とおり に は しょうてん が おおい ね	
\\	この 通[とお]りには
\\	が 多[おお]いね。			
\\	商売	商売[しょうばい]	しょうばい	
\\	彼の商売は儲かっているな。	彼[かれ]の 商売[しょうばい]は 儲[もう]かっているな。	かれ の しょうばい は もうかって いる な	
\\	彼[かれ]の
\\	は 儲[もう]かっているな。			
\\	商人	商人[しょうにん]	しょうにん	
\\	商人は数字に強いね。	商人[しょうにん]は 数字[すうじ]に 強[つよ]いね。	しょうにん は すうじ に つよい ね	
\\	は 数字[すうじ]に 強[つよ]いね。			
\\	品	品[しな]	しな	
\\	そちらの品は半額になっています。	そちらの 品[しな]は 半額[はんがく]になっています。	そちら の しな は はんがく に なって います	
\\	そちらの
\\	は 半額[はんがく]になっています。			
\\	手段	手段[しゅだん]	しゅだん	
\\	彼は目的のためには手段を選ばなかったわね。	彼[かれ]は 目的[もくてき]のためには 手段[しゅだん]を 選[えら]ばなかったわね。	かれ は もくてき の ため に は しゅだん を えらばなかった わ ね	
\\	彼[かれ]は 目的[もくてき]のためには
\\	を 選[えら]ばなかったわね。			
\\	格好	格好[かっこう]	かっこう	
\\	格好のいい青年に会ったよ。	格好[かっこう]のいい 青年[せいねん]に 会[あ]ったよ。	かっこう の いい せいねん に あった よ	
\\	のいい 青年[せいねん]に 会[あ]ったよ。			
\\	春分	春分[しゅんぶん]	しゅんぶん	
\\	春分の日は毎年3月20日頃です。	春分[しゅんぶん]の 日[ひ]は 毎年3月20日頃[まいとし さんがつ はつかごろ]です。	しゅんぶん の ひ は まいとし さんがつ はつかごろ です	
\\	の 日[ひ]は 毎年3月20日頃[まいとし さんがつ はつかごろ]です。			
\\	春夏秋冬	春夏秋冬[しゅんかしゅうとう]	しゅんかしゅうとう	
\\	春夏秋冬の移り変わりを見るのが大好きです。	春夏秋冬[しゅんかしゅうとう]の 移[うつ]り 変[か]わりを 見[み]るのが 大好[だいす]きです。	しゅんかしゅうとう の うつりかわり を みる の が だいすき です	
\\	の 移[うつ]り 変[か]わりを 見[み]るのが 大好[だいす]きです。			
\\	夏季	夏季[かき]	かき	
\\	夏季講習に申し込みした?	夏季[かき] 講習[こうしゅう]に 申[もう]し 込[こ]みした?	かき こうしゅう に もうしこみ した 
\\	講習[こうしゅう]に 申[もう]し 込[こ]みした?			
\\	寒気	寒気[さむけ]	さむけ	
\\	何だか寒気がします。	何[なん]だか 寒気[さむけ]がします。	なんだか さむけ が します	
\\	何[なん]だか
\\	がします。			
\\	暖か	暖[あたた]か	あたたか	
\\	最近は暖かです。	最近[さいきん]は 暖[あたた]かです。	さいきん は あたたか です	
\\	最近[さいきん]は
\\	です。			
\\	情熱	情熱[じょうねつ]	じょうねつ	
\\	父は情熱を持って仕事に打ち込んでいます。	父[ちち]は 情熱[じょうねつ]を 持[も]って 仕事[しごと]に 打[う]ち 込[こ]んでいます。	ちち は じょうねつ を もって しごと に うちこんで います	
\\	父[ちち]は
\\	を 持[も]って 仕事[しごと]に 打[う]ち 込[こ]んでいます。			
\\	広告	広告[こうこく]	こうこく	
\\	その広告を新聞で見ました。	その 広告[こうこく]を 新聞[しんぶん]で 見[み]ました。	その こうこく を しんぶん で みました	
\\	その
\\	を 新聞[しんぶん]で 見[み]ました。			
\\	新た	新[あら]た	あらた	
\\	新たな計画が進んでいます。	新[あら]たな 計画[けいかく]が 進[すす]んでいます。	あらた な けいかく が すすんで います	
\\	な 計画[けいかく]が 進[すす]んでいます。			
\\	新聞社	新聞社[しんぶんしゃ]	しんぶんしゃ	
\\	このビルは新聞社です。	このビルは 新聞社[しんぶんしゃ]です。	この びる は しんぶんしゃ です	
\\	このビルは
\\	です。			
\\	新人	新人[しんじん]	しんじん	
\\	彼は今日入ったばかりの新人です。	彼[かれ]は 今日入[きょう はい]ったばかりの 新人[しんじん]です。	かれ は きょう はいった ばかり の しんじん です	
\\	彼[かれ]は 今日入[きょう はい]ったばかりの
\\	です。			
\\	最悪	最悪[さいあく]	さいあく	
\\	何とか最悪の事態を避けることができました。	何[なん]とか 最悪[さいあく]の 事態[じたい]を 避[さ]けることができました。	なんとか さいあく の じたい を さける こと が できました	
\\	何[なん]とか
\\	の 事態[じたい]を 避[さ]けることができました。			
\\	悪用	悪用[あくよう]	あくよう	
\\	彼は地位を悪用しています。	彼[かれ]は 地位[ちい]を 悪用[あくよう]しています。	かれ は ちい を あくよう して います	
\\	彼[かれ]は 地位[ちい]を
\\	しています。			
\\	悪	悪[あく]	あく	
\\	彼は悪を憎んでいます。	彼[かれ]は 悪[あく]を 憎[にく]んでいます。	かれ は あく を にくんで います	
\\	彼[かれ]は
\\	を 憎[にく]んでいます。			
\\	悪女	悪女[あくじょ]	あくじょ	
\\	彼は悪女に騙されたんだ。	彼[かれ]は 悪女[あくじょ]に 騙[だま]されたんだ。	かれ は あくじょ に だまされた ん だ	
\\	彼[かれ]は
\\	に 騙[だま]されたんだ。			
\\	心理	心理[しんり]	しんり	
\\	顧客心理を理解することは重要です。	顧客[こきゃく] 心理[しんり]を 理解[りかい]することは 重要[じゅうよう]です。	こきゃく しんり を りかい する こと は じゅうよう です	
\\	顧客[こきゃく]
\\	を 理解[りかい]することは 重要[じゅうよう]です。			
\\	心	心[こころ]	こころ	
\\	彼は素直な心を持っている。	彼[かれ]は 素直[すなお]な 心[こころ]を 持[も]っている。	かれ は すなお な こころ を もって いる	
\\	彼[かれ]は 素直[すなお]な
\\	を 持[も]っている。			
\\	思い	思[おも]い	おもい	
\\	必死の思いで彼に頼んだよ。	必死[ひっし]の 思[おも]いで 彼[かれ]に 頼[たの]んだよ。	ひっし の おもい で かれ に たのんだ よ	
\\	必死[ひっし]の
\\	で 彼[かれ]に 頼[たの]んだよ。			
\\	思わず	思[おも]わず	おもわず	
\\	嬉しくて思わず涙が出ました。	嬉[うれ]しくて 思[おも]わず 涙[なみだ]が 出[で]ました。	うれしくて おもわず なみだ が でました	
\\	嬉[うれ]しくて
\\	涙[なみだ]が 出[で]ました。			
\\	思いがけない	思[おも]いがけない	おもいがけない	
\\	彼から思いがけないことを聞いた。	彼[かれ]から 思[おも]いがけないことを 聞[き]いた。	かれ から おもいがけない こと を きいた	
\\	彼[かれ]から
\\	ことを 聞[き]いた。			
\\	思いやり	思[おも]いやり	おもいやり	
\\	彼女の思いやりが嬉しかった。	彼女[かのじょ]の 思[おも]いやりが 嬉[うれ]しかった。	かのじょ の おもいやり が うれしかった	
\\	彼女[かのじょ]の
\\	が 嬉[うれ]しかった。			
\\	決して	決[けっ]して	けっして	
\\	このことを決して忘れないでください。	このことを 決[けっ]して 忘[わす]れないでください。	この こと を けっして わすれない で ください	
\\	このことを
\\	忘[わす]れないでください。			
\\	決心	決心[けっしん]	けっしん	
\\	今度こそタバコを止める決心をしました。	今度[こんど]こそタバコを 止[や]める 決心[けっしん]をしました。	こんど こそ たばこ を やめる けっしん を しました	
\\	今度[こんど]こそタバコを 止[や]める
\\	をしました。			
\\	決まり	決[き]まり	きまり	
\\	決まりを守ることは大切です。	決[き]まりを 守[まも]ることは 大切[たいせつ]です。	きまり を まもる こと は たいせつ です	
\\	を 守[まも]ることは 大切[たいせつ]です。			
\\	才能	才能[さいのう]	さいのう	
\\	彼は芸術的な才能にあふれているね。	彼[かれ]は 芸術的[げいじゅつてき]な 才能[さいのう]にあふれているね。	かれ は げいじゅつてき な さいのう に あふれて いる ね	
\\	彼[かれ]は 芸術的[げいじゅつてき]な
\\	にあふれているね。			
\\	小便	小便[しょうべん]	しょうべん	
\\	ちょっと小便しに行って来る。	ちょっと 小便[しょうべん]しに 行[い]って 来[く]る。	ちょっと しょうべん し に いって くる	
\\	ちょっと
\\	しに 行[い]って 来[く]る。			
\\	局	局[きょく]	きょく	
\\	彼女はラジオ局で働いています。	彼女[かのじょ]はラジオ 局[きょく]で 働[はたら]いています。	かのじょ は らじおきょく で はたらいて います	
\\	彼女[かのじょ]はラジオ
\\	で 働[はたら]いています。			
\\	氏	氏[し]	し	
\\	会長は田中氏に決定。	会長[かいちょう]は 田中[たなか] 氏[し]に 決定。[けってい]	かいちょう は たなかし に けってい	
\\	会長[かいちょう]は 田中[たなか]
\\	に 決定。[けってい]			
\\	国名	国名[こくめい]	こくめい	
\\	アジアの国名をいくつ知っていますか。	アジアの 国名[こくめい]をいくつ 知[し]っていますか。	あじあ の こくめい を いくつ しって います か	
\\	アジアの
\\	をいくつ 知[し]っていますか。			
\\	各地	各地[かくち]	かくち	
\\	各地で大雨が降っています。	各地[かくち]で 大雨[おおあめ]が 降[ふ]っています。	かくち で おおあめ が ふって います	
\\	で 大雨[おおあめ]が 降[ふ]っています。			
\\	市内	市内[しない]	しない	
\\	明日は市内を観光する予定です。	明日[あした]は 市内[しない]を 観光[かんこう]する 予定[よてい]です。	あした は しない を かんこう する よてい です 。	
\\	明日[あした]は
\\	を 観光[かんこう]する 予定[よてい]です。			
\\	市長	市長[しちょう]	しちょう	
\\	新しい市長が選ばれました。	新[あたら]しい 市長[しちょう]が 選[えら]ばれました。	あたらしい しちょう が えらばれました	
\\	新[あたら]しい
\\	が 選[えら]ばれました。			
\\	市場	市場[いちば]	いちば	
\\	市場で新鮮な魚を買ってきました。	市場[いちば]で 新鮮[しんせん]な 魚[さかな]を 買[か]ってきました。	いちば で しんせん な さかな を かって きました	
\\	で 新鮮[しんせん]な 魚[さかな]を 買[か]ってきました。			
\\	市場	市場[しじょう]	しじょう	
\\	デジカメ市場は急速に拡大している。	デジカメ 市場[しじょう]は 急速[きゅうそく]に 拡大[かくだい]している。	でじかめ しじょう は きゅうそく に かくだい して いる	
\\	デジカメ
\\	は 急速[きゅうそく]に 拡大[かくだい]している。			
\\	市外	市外[しがい]	しがい	
\\	祖父は市外の病院に通っているの。	祖父[そふ]は 市外[しがい]の 病院[びょういん]に 通[かよ]っているの。	そふ は しがい の びょういん に かよって いる の	
\\	祖父[そふ]は
\\	の 病院[びょういん]に 通[かよ]っているの。			
\\	市	市[し]	し	
\\	その市の人口は減り続けているの。	その 市[し]の 人口[じんこう]は 減[へ]り 続[つづ]けているの。	その し の じんこう は へりつづけて いる の	
\\	その
\\	の 人口[じんこう]は 減[へ]り 続[つづ]けているの。			
\\	様々	様々[さまざま]	さまざま	
\\	その都市には様々な人種が集まっているわ。	その 都市[とし]には 様々[さまざま]な 人種[じんしゅ]が 集[あつ]まっているわ。	その とし に は さまざま な じんしゅ が あつまって いる わ	
\\	その 都市[とし]には
\\	な 人種[じんしゅ]が 集[あつ]まっているわ。			
\\	書物	書物[しょもつ]	しょもつ	
\\	彼は書物に囲まれて生活しているの。	彼[かれ]は 書物[しょもつ]に 囲[かこ]まれて 生活[せいかつ]しているの。	かれ は しょもつ に かこまれて せいかつ して いる の	
\\	彼[かれ]は
\\	に 囲[かこ]まれて 生活[せいかつ]しているの。			
\\	気軽	気軽[きがる]	きがる	
\\	いつでも気軽に遊びに来て下さい。	いつでも 気軽[きがる]に 遊[あそ]びに 来[き]て 下[くだ]さい。	いつでも きがる に あそび に きて ください	
\\	いつでも
\\	に 遊[あそ]びに 来[き]て 下[くだ]さい。			
\\	少量	少量[しょうりょう]	しょうりょう	
\\	泡立てたクリームに少量のブランデーを加えます。	泡立[あわだ]てたクリームに 少量[しょうりょう]のブランデーを 加[くわ]えます。	あわだてた くりーむ に しょうりょう の ぶらんでー を くわえます	
\\	泡立[あわだ]てたクリームに
\\	のブランデーを 加[くわ]えます。			
\\	小量	小量[しょうりょう]	しょうりょう	
\\	私はコーヒー豆を小量で買うようにしています。	私[わたし]はコーヒー 豆[まめ]を 小量[しょうりょう]で 買[か]うようにしています。	わたし は こーひー まめ を しょうりょう で かうよう に しています 。	
\\	私[わたし]はコーヒー 豆[まめ]を
\\	で 買[か]うようにしています。			
\\	受け入れる	受[う]け 入[い]れる	うけいれる	
\\	私は彼の意見を受け入れました。	私[わたし]は 彼[かれ]の 意見[いけん]を 受[う]け 入[い]れました。	わたし は かれ の いけん を うけいれました	
\\	私[わたし]は 彼[かれ]の 意見[いけん]を
\\	受け止める	受[う]け 止[と]める	うけとめる	
\\	ボールが速過ぎて受け止められなかったの。	ボールが 速過[はや す]ぎて 受[う]け 止[と]められなかったの。	ぼーる が はや すぎて うけとめられ なかった の	
\\	ボールが 速過[はや す]ぎて
\\	の。			
\\	受かる	受[う]かる	うかる	
\\	第一志望の大学に受かりました。	第一志望[だいいち しぼう]の 大学[だいがく]に 受[う]かりました。	だいいち しぼう の だいがく に うかりました	
\\	第一志望[だいいち しぼう]の 大学[だいがく]に
\\	受け取り	受[う]け 取[と]り	うけとり	
\\	受け取りに判子をお願いします。	受[う]け 取[と]りに 判子[はんこ]をお 願[ねが]いします。	うけとり に はんこ を おねがい します	
\\	に 判子[はんこ]をお 願[ねが]いします。			
\\	書き取り	書[か]き 取[と]り	かきとり	
\\	僕たちは毎朝漢字の書き取りをします。	僕[ぼく]たちは 毎朝漢字[まいあさ かんじ]の 書[か]き 取[と]りをします。	ぼくたち は まいあさ かんじ の かきとり を します	
\\	僕[ぼく]たちは 毎朝漢字[まいあさ かんじ]の
\\	をします。			
\\	受け持つ	受[う]け 持[も]つ	うけもつ	
\\	1年生を受け持っています。	1年生[いちねんせい]を 受[う]け 持[も]っています。	いちねんせい を うけもって います	
\\	1年生[いちねんせい]を
\\	打ち上げる	打[う]ち 上[あ]げる	うちあげる	
\\	夏祭りで花火を打ち上げます。	夏祭[なつまつ]りで 花火[はなび]を 打[う]ち 上[あ]げます。	なつまつり で はなび を うちあげます	
\\	夏祭[なつまつ]りで 花火[はなび]を
\\	打ち合わせ	打[う]ち 合[あ]わせ	うちあわせ	
\\	午後に打ち合わせをしましょう。	午後[ごご]に 打[う]ち 合[あ]わせをしましょう。	ごご に うちあわせ を しましょう	
\\	午後[ごご]に
\\	をしましょう。			
\\	打ち明ける	打[う]ち 明[あ]ける	うちあける	
\\	親友に悩みを打ち明けたの。	親友[しんゆう]に 悩[なや]みを 打[う]ち 明[あ]けたの。	しんゆう に なやみ を うちあけた の	
\\	親友[しんゆう]に 悩[なや]みを
\\	の。			
\\	打ち合わせる	打[う]ち 合[あ]わせる	うちあわせる	
\\	来週の予定を打ち合わせましょう。	来週[らいしゅう]の 予定[よてい]を 打[う]ち 合[あ]わせましょう。	らいしゅう の よてい を うちあわせましょう	
\\	来週[らいしゅう]の 予定[よてい]を
\\	打ち消し	打[う]ち 消[け]し	うちけし	
\\	彼はうわさを打ち消したわよ。	彼[かれ]はうわさを 打[う]ち 消[け]したわよ。	かれ は うわさ を うちけした わ よ	
\\	彼[かれ]はうわさを
\\	たわよ。			
\\	市役所	市役所[しやくしょ]	しやくしょ	
\\	市役所で書類をもらって来たの。	市役所[しやくしょ]で 書類[しょるい]をもらって 来[き]たの。	しやくしょ で しょるい を もらって きた の	
\\	で 書類[しょるい]をもらって 来[き]たの。			
\\	性能	性能[せいのう]	せいのう	
\\	今度のパソコンは性能がすごく良い。	今度[こんど]のパソコンは 性能[せいのう]がすごく 良[い]い。	こんど の ぱそこん は せいのう が すごく いい	
\\	今度[こんど]のパソコンは
\\	がすごく 良[い]い。			
\\	性格	性格[せいかく]	せいかく	
\\	僕と姉の性格は正反対です。	僕[ぼく]と 姉[あね]の 性格[せいかく]は 正反対[せいはんたい]です。	ぼく と あね の せいかく は せいはんたい です	
\\	僕[ぼく]と 姉[あね]の
\\	は 正反対[せいはんたい]です。			
\\	性	性[せい]	せい	
\\	この会社では性による差別はありません。	この 会社[かいしゃ]では 性[せい]による 差別[さべつ]はありません。	この かいしゃ で は せい に よる さべつ は ありません	
\\	この 会社[かいしゃ]では
\\	による 差別[さべつ]はありません。			
\\	国産	国産[こくさん]	こくさん	
\\	このワインは国産です。	このワインは 国産[こくさん]です。	この わいん は こくさん です	
\\	このワインは
\\	です。			
\\	活用	活用[かつよう]	かつよう	
\\	彼女はインターネットを活用しているの。	彼女[かのじょ]はインターネットを 活用[かつよう]しているの。	かのじょ は いんたーねっと を かつよう して いる の	
\\	彼女[かのじょ]はインターネットを
\\	しているの。			
\\	活字	活字[かつじ]	かつじ	
\\	新聞の活字が読みやすくなったね。	新聞[しんぶん]の 活字[かつじ]が 読[よ]みやすくなったね。	しんぶん の かつじ が よみ やすく なった ね	
\\	新聞[しんぶん]の
\\	が 読[よ]みやすくなったね。			
\\	学会	学会[がっかい]	がっかい	
\\	彼は学会で論文を発表したよ。	彼[かれ]は 学会[がっかい]で 論文[ろんぶん]を 発表[はっぴょう]したよ。	かれ は がっかい で ろんぶん を はっぴょう した よ	
\\	彼[かれ]は
\\	で 論文[ろんぶん]を 発表[はっぴょう]したよ。			
\\	学年	学年[がくねん]	がくねん	
\\	彼は私より一学年上です。	彼[かれ]は 私[わたし]より 一[ひと] 学年[がくねん] 上[うえ]です。	かれ は わたし より ひと がくねん うえ です	
\\	彼[かれ]は 私[わたし]より 一[ひと]
\\	上[うえ]です。			
\\	工学	工学[こうがく]	こうがく	
\\	彼は大学で工学を勉強しました。	彼[かれ]は 大学[だいがく]で 工学[こうがく]を 勉強[べんきょう]しました。	かれ は だいがく で こうがく を べんきょう しました	
\\	彼[かれ]は 大学[だいがく]で
\\	を 勉強[べんきょう]しました。			
\\	学長	学長[がくちょう]	がくちょう	
\\	入学式で学長の挨拶がありました。	入学式[にゅうがくしき]で 学長[がくちょう]の 挨拶[あいさつ]がありました。	にゅうがくしき で がくちょう の あいさつ が ありました	
\\	入学式[にゅうがくしき]で
\\	の 挨拶[あいさつ]がありました。			
\\	工学部	工学部[こうがくぶ]	こうがくぶ	
\\	彼は工学部の教授です。	彼[かれ]は 工学部[こうがくぶ]の 教授[きょうじゅ]です。	かれ は こうがくぶ の きょうじゅ です	
\\	彼[かれ]は
\\	の 教授[きょうじゅ]です。			
\\	学費	学費[がくひ]	がくひ	
\\	彼はアルバイトをして学費を稼いだんだ。	彼[かれ]はアルバイトをして 学費[がくひ]を 稼[かせ]いだんだ。	かれ は あるばいと を して がくひ を かせいだ ん だ 。	
\\	彼[かれ]はアルバイトをして
\\	を 稼[かせ]いだんだ。			
\\	学部	学部[がくぶ]	がくぶ	
\\	彼は経済学部の学生です。	彼[かれ]は 経済[けいざい] 学部[がくぶ]の 学生[がくせい]です。	かれ は けいざい がくぶ の がくせい です	
\\	彼[かれ]は 経済[けいざい]
\\	の 学生[がくせい]です。			
\\	学力	学力[がくりょく]	がくりょく	
\\	学力を付けてその大学に進みたい。	学力[がくりょく]を 付[つ]けてその 大学[だいがく]に 進[すす]みたい。	がくりょく を つけて その だいがく に すすみたい	
\\	を 付[つ]けてその 大学[だいがく]に 進[すす]みたい。			
\\	教員	教員[きょういん]	きょういん	
\\	彼は高校の教員です。	彼[かれ]は 高校[こうこう]の 教員[きょういん]です。	かれ は こうこう の きょういん です	
\\	彼[かれ]は 高校[こうこう]の
\\	です。			
\\	教わる	教[おそ]わる	おそわる	
\\	私は両親から多くを教わりました。	私[わたし]は 両親[りょうしん]から 多[おお]くを 教[おそ]わりました。	わたし は りょうしん から おおく を おそわりました	
\\	私[わたし]は 両親[りょうしん]から 多[おお]くを
\\	教え	教[おし]え	おしえ	
\\	父の教えは「自分に厳しく」です。	父[ちち]の 教[おし]えは
\\	自分[じぶん]に 厳[きび]しく」です。	ちち の おしえ は じぶん に きびしく です	
\\	父[ちち]の
\\	は
\\	自分[じぶん]に 厳[きび]しく」です。			
\\	強力	強力[きょうりょく]	きょうりょく	
\\	これは強力な接着剤ね。	これは 強力[きょうりょく]な 接着剤[せっちゃくざい]ね。	これ は きょうりょく な せっちゃくざい ね	
\\	これは
\\	な 接着剤[せっちゃくざい]ね。			
\\	強制	強制[きょうせい]	きょうせい	
\\	彼らは労働を強制されたんだ。	彼[かれ]らは 労働[ろうどう]を 強制[きょうせい]されたんだ。	かれら は ろうどう を きょうせい された ん だ	
\\	彼[かれ]らは 労働[ろうどう]を
\\	されたんだ。			
\\	最強	最強[さいきょう]	さいきょう	
\\	彼は最強チームの一員です。	彼[かれ]は 最強[さいきょう]チームの 一員[いちいん]です。	かれ は さいきょう ちーむ の いちいん です	
\\	彼[かれ]は
\\	チームの 一員[いちいん]です。			
\\	心強い	心強[こころづよ]い	こころづよい	
\\	あなたが一緒にいてくれると心強い。	あなたが 一緒[いっしょ]にいてくれると 心強[こころづよ]い。	あなた が いっしょ に いて くれる と こころづよい	
\\	あなたが 一緒[いっしょ]にいてくれると
\\	強引	強引[ごういん]	ごういん	
\\	友人の強引な誘いを断れませんでした。	友人[ゆうじん]の 強引[ごういん]な 誘[さそ]いを 断[ことわ]れませんでした。	ゆうじん の ごういん な さそい を ことわれません でした	
\\	友人[ゆうじん]の
\\	な 誘[さそ]いを 断[ことわ]れませんでした。			
\\	引用	引用[いんよう]	いんよう	
\\	論文にその本を引用したの。	論文[ろんぶん]にその 本[ほん]を 引用[いんよう]したの。	ろんぶん に その ほん を いんよう した の	
\\	論文[ろんぶん]にその 本[ほん]を
\\	したの。			
\\	字引	字引[じびき]	じびき	
\\	この漢字を字引で引いてみて。	この 漢字[かんじ]を 字引[じびき]で 引[ひ]いてみて。	この かんじ を じびき で ひいて みて	
\\	この 漢字[かんじ]を
\\	で 引[ひ]いてみて。			
\\	学習	学習[がくしゅう]	がくしゅう	
\\	今日は野外で学習した。	今日[きょう]は 野外[やがい]で 学習[がくしゅう]した。	きょう は やがい で がくしゅう した	
\\	今日[きょう]は 野外[やがい]で
\\	した。			
\\	受験	受験[じゅけん]	じゅけん	
\\	日本語能力試験を受験したんだ。	日本語能力試験[にほんご のうりょく しけん]を 受験[じゅけん]したんだ。	にほんご のうりょく しけん を じゅけん した ん だ	
\\	日本語能力試験[にほんご のうりょく しけん]を
\\	したんだ。			
\\	性質	性質[せいしつ]	せいしつ	
\\	この犬は穏やかな性質だよ。	この 犬[いぬ]は 穏[おだ]やかな 性質[せいしつ]だよ。	この いぬ は おだやか な せいしつ だ よ	
\\	この 犬[いぬ]は 穏[おだ]やかな
\\	だよ。			
\\	悪質	悪質[あくしつ]	あくしつ	
\\	最近は悪質な事件が多いですね。	最近[さいきん]は 悪質[あくしつ]な 事件[じけん]が 多[おお]いですね。	さいきん は あくしつ な じけん が おおい です ね	
\\	最近[さいきん]は
\\	な 事件[じけん]が 多[おお]いですね。			
\\	学問	学問[がくもん]	がくもん	
\\	彼は少年の頃から学問が好きでした。	彼[かれ]は 少年[しょうねん]の 頃[ころ]から 学問[がくもん]が 好[す]きでした。	かれ は しょうねん の ころ から がくもん が すき でした	
\\	彼[かれ]は 少年[しょうねん]の 頃[ころ]から
\\	が 好[す]きでした。			
\\	有り難い	有[あ]り 難[がた]い	ありがたい	
\\	彼の助けは本当に有り難いな。	彼[かれ]の 助[たす]けは 本当[ほんとう]に 有[あ]り 難[がた]いな。	かれ の たすけ は ほんとう に ありがたい な	
\\	彼[かれ]の 助[たす]けは 本当[ほんとう]に
\\	な。			
\\	弱点	弱点[じゃくてん]	じゃくてん	
\\	彼の弱点はスタミナが足りないところです。	彼[かれ]の 弱点[じゃくてん]はスタミナが 足[た]りないところです。	かれ の じゃくてん は すたみな が たりない ところ です	
\\	彼[かれ]の
\\	はスタミナが 足[た]りないところです。			
\\	少数	少数[しょうすう]	しょうすう	
\\	その計画に反対の人はほんの少数だったよ。	その 計画[けいかく]に 反対[はんたい]の 人[ひと]はほんの 少数[しょうすう]だったよ。	その けいかく に はんたい の ひと は ほんの しょうすう だった よ	
\\	その 計画[けいかく]に 反対[はんたい]の 人[ひと]はほんの
\\	だったよ。			
\\	小数	小数[しょうすう]	しょうすう	
\\	小数は切り捨てて計算して下さい。	小数[しょうすう]は 切[き]り 捨[す]てて 計算[けいさん]して 下[くだ]さい。	しょうすう は きりすてて けいさん して ください	
\\	は 切[き]り 捨[す]てて 計算[けいさん]して 下[くだ]さい。			
\\	回路	回路[かいろ]	かいろ	
\\	コンピュータの電子回路が故障した。	コンピュータの 電子[でんし] 回路[かいろ]が 故障[こしょう]した。	こんぴゅーた の でんし かいろ が こしょう した	
\\	コンピュータの 電子[でんし]
\\	が 故障[こしょう]した。			
\\	回数	回数[かいすう]	かいすう	
\\	最近はテレビを見る回数が減りました。	最近[さいきん]はテレビを 見[み]る 回数[かいすう]が 減[へ]りました。	さいきん は てれび を みる かいすう が へりました	
\\	最近[さいきん]はテレビを 見[み]る
\\	が 減[へ]りました。			
\\	後回し	後回[あとまわ]し	あとまわし	
\\	おしゃべりは後回しにしましょう。	おしゃべりは 後回[あとまわ]しにしましょう。	おしゃべり は あとまわし に しましょう	
\\	おしゃべりは
\\	にしましょう。			
\\	決勝	決勝[けっしょう]	けっしょう	
\\	僕たちは頑張って決勝まで進んだよ。	僕[ぼく]たちは 頑張[がんば]って 決勝[けっしょう]まで 進[すす]んだよ。	ぼくたち は がんばって けっしょう まで すすんだ よ	
\\	僕[ぼく]たちは 頑張[がんば]って
\\	まで 進[すす]んだよ。			
\\	担ぐ	担[かつ]ぐ	かつぐ	
\\	彼は大きな荷物を担いで来たの。	彼[かれ]は 大[おお]きな 荷物[にもつ]を 担[かつ]いで 来[き]たの。	かれ は おおき な にもつ を かついで きた の	
\\	彼[かれ]は 大[おお]きな 荷物[にもつ]を
\\	来[き]たの。			
\\	当てる	当[あ]てる	あてる	
\\	彼はくじ引きで一等賞を当てたよ。	彼[かれ]はくじ 引[び]きで 一等賞[いっとう しょう]を 当[あ]てたよ。	かれ は くじびき で いっとう しょう を あてた よ	
\\	彼[かれ]はくじ 引[び]きで 一等賞[いっとう しょう]を
\\	よ。			
\\	当たり前	当[あ]たり 前[まえ]	あたりまえ	
\\	あなたの成績が下がったのは当たり前です。	あなたの 成績[せいせき]が 下[さ]がったのは 当[あ]たり 前[まえ]です。	あなた の せいせき が さがった の は あたりまえ です	
\\	あなたの 成績[せいせき]が 下[さ]がったのは
\\	です。			
\\	当たり	当[あ]たり	あたり	
\\	彼の予想は大当たりでした。	彼[かれ]の 予想[よそう]は 大[おお] 当[あ]たりでした。	かれ の よそう は おおあたり でした	
\\	彼[かれ]の 予想[よそう]は 大[おお]
\\	でした。			
\\	株式	株式[かぶしき]	かぶしき	
\\	彼は株式の売買で多額の利益を得たのさ。	彼[かれ]は 株式[かぶしき]の 売買[ばいばい]で 多額[たがく]の 利益[りえき]を 得[え]たのさ。	かれ は かぶしき の ばいばい で たがく の りえき を えた の さ	
\\	彼[かれ]は
\\	の 売買[ばいばい]で 多額[たがく]の 利益[りえき]を 得[え]たのさ。			
\\	式	式[しき]	しき	
\\	彼らは教会で式を挙げました。	彼[かれ]らは 教会[きょうかい]で 式[しき]を 挙[あ]げました。	かれら は きょうかい で しき を あげました	
\\	彼[かれ]らは 教会[きょうかい]で
\\	を 挙[あ]げました。			
\\	業界	業界[ぎょうかい]	ぎょうかい	
\\	私は
\\	業界で働いています。	私[わたし]は
\\	業界[ぎょうかい]で 働[はたら]いています。	わたし は 
\\	ぎょうかい で はたらいて います	
\\	私[わたし]は
\\	で 働[はたら]いています。			
\\	営業	営業[えいぎょう]	えいぎょう	
\\	彼は営業を担当しています。	彼[かれ]は 営業[えいぎょう]を 担当[たんとう]しています。	かれ は えいぎょう を たんとう しています 。	
\\	彼[かれ]は
\\	を 担当[たんとう]しています。			
\\	必ずしも	必[かなら]ずしも	かならずしも	
\\	親切は必ずしも喜ばれるわけではない。	親切[しんせつ]は 必[かなら]ずしも 喜[よろこ]ばれるわけではない。	しんせつ は かならずしも よろこばれる わけ で は ない	
\\	親切[しんせつ]は
\\	喜[よろこ]ばれるわけではない。			
\\	求人	求人[きゅうじん]	きゅうじん	
\\	彼は求人広告で仕事を見つけたんだ。	彼[かれ]は 求人[きゅうじん] 広告[こうこく]で 仕事[しごと]を 見[み]つけたんだ。	かれ は きゅうじん こうこく で しごと を みつけた ん だ	
\\	彼[かれ]は
\\	広告[こうこく]で 仕事[しごと]を 見[み]つけたんだ。			
\\	合計	合計[ごうけい]	ごうけい	
\\	合計金額を計算してください。	合計[ごうけい] 金額[きんがく]を 計算[けいさん]してください。	ごうけい きんがく を けいさん して ください	
\\	金額[きんがく]を 計算[けいさん]してください。			
\\	家計	家計[かけい]	かけい	
\\	彼女は家計を任されているの。	彼女[かのじょ]は 家計[かけい]を 任[まか]されているの。	かのじょ は かけい を まかされて いる の	
\\	彼女[かのじょ]は
\\	を 任[まか]されているの。			
\\	寒暖計	寒暖計[かんだんけい]	かんだんけい	
\\	壁に寒暖計が掛かっていました。	壁[かべ]に 寒暖計[かんだんけい]が 掛[か]かっていました。	かべ に かんだんけい が かかって いました	
\\	壁[かべ]に
\\	が 掛[か]かっていました。			
\\	差	差[さ]	さ	
\\	都心と地方では収入に大きな差があるね。	都心[としん]と 地方[ちほう]では 収入[しゅうにゅう]に 大[おお]きな 差[さ]があるね。	としん と ちほう で は しゅうにゅう に おおき な さ が ある ね	
\\	都心[としん]と 地方[ちほう]では 収入[しゅうにゅう]に 大[おお]きな
\\	があるね。			
\\	格差	格差[かくさ]	かくさ	
\\	貧富の格差が大きくなっているな。	貧富[ひんぷ]の 格差[かくさ]が 大[おお]きくなっているな。	ひんぷ の かくさ が おおきく なって いる な	
\\	貧富[ひんぷ]の
\\	が 大[おお]きくなっているな。			
\\	差し出す	差[さ]し 出[だ]す	さしだす	
\\	彼は握手をしようと手を差し出したの。	彼[かれ]は 握手[あくしゅ]をしようと 手[て]を 差[さ]し 出[だ]したの。	かれ は あくしゅ を しようと て を さしだした の	
\\	彼[かれ]は 握手[あくしゅ]をしようと 手[て]を
\\	の。			
\\	時差	時差[じさ]	じさ	
\\	日本とフランスの時差は8時間です。	日本[にほん]とフランスの 時差[じさ]は 8時間[はちじかん]です。	にほん と ふらんす の じさ は はちじかん です	
\\	日本[にほん]とフランスの
\\	は 8時間[はちじかん]です。			
\\	差す	差[さ]す	さす	
\\	雨が降ってきたので傘を差しました。	雨[あめ]が 降[ふ]ってきたので 傘[かさ]を 差[さ]しました。	あめ が ふって きた の で かさ を さしました	
\\	雨[あめ]が 降[ふ]ってきたので 傘[かさ]を
\\	差し上げる	差[さ]し 上[あ]げる	さしあげる	
\\	こちらを差し上げます。	こちらを 差[さ]し 上[あ]げます。	こちら を さしあげます	
\\	こちらを
\\	学割	学割[がくわり]	がくわり	
\\	学割だとだいぶ安いな。	学割[がくわり]だとだいぶ 安[やす]いな。	がくわり だ と だいぶ やすい な	
\\	だとだいぶ 安[やす]いな。			
\\	時間割り	時間割[じかんわ]り	じかんわり	
\\	明日の授業は時間割り通りです。	明日[あした]の 授業[じゅぎょう]は 時間割[じかんわ]り 通[どお]りです。	あした の じゅぎょう は じかんわり どおり です	
\\	明日[あした]の 授業[じゅぎょう]は
\\	通[どお]りです。			
\\	残業	残業[ざんぎょう]	ざんぎょう	
\\	昨日は遅くまで残業しました。	昨日[きのう]は 遅[おそ]くまで 残業[ざんぎょう]しました。	きのう は おそく まで ざんぎょう しました	
\\	昨日[きのう]は 遅[おそ]くまで
\\	しました。			
\\	残暑	残暑[ざんしょ]	ざんしょ	
\\	今年も残暑が厳しかった。	今年[ことし]も 残暑[ざんしょ]が 厳[きび]しかった。	ことし も ざんしょ が きびしかった	
\\	今年[ことし]も
\\	が 厳[きび]しかった。			
\\	支店	支店[してん]	してん	
\\	彼は支店に転勤したよ。	彼[かれ]は 支店[してん]に 転勤[てんきん]したよ。	かれ は してん に てんきん した よ	
\\	彼[かれ]は
\\	に 転勤[てんきん]したよ。			
\\	支持	支持[しじ]	しじ	
\\	彼は国民の支持を得たのよ。	彼[かれ]は 国民[こくみん]の 支持[しじ]を 得[え]たのよ。	かれ は こくみん の しじ を えた の よ	
\\	彼[かれ]は 国民[こくみん]の
\\	を 得[え]たのよ。			
\\	支出	支出[ししゅつ]	ししゅつ	
\\	今月のわが家の支出は15万円です。	今月[こんげつ]のわが 家[や]の 支出[ししゅつ]は 15万円[じゅうごまんえん]です。	こんげつ の わがや の ししゅつ は じゅうごまんえん です	
\\	今月[こんげつ]のわが 家[や]の
\\	は 15万円[じゅうごまんえん]です。			
\\	支配	支配[しはい]	しはい	
\\	その権力者による支配は50年以上続いたんです。	その 権力者[けんりょくしゃ]による 支配[しはい]は 50年以上続[ごじゅうねん いじょう つづ]いたんです。	その けんりょくしゃ に よる しはい は ごじゅうねん いじょう つづいた ん です	
\\	その 権力者[けんりょくしゃ]による
\\	は 50年以上続[ごじゅうねん いじょう つづ]いたんです。			
\\	支度	支度[したく]	したく	
\\	支度ができたら出かけましょう。	支度[したく]ができたら 出[で]かけましょう。	したく が できたら でかけましょう	
\\	ができたら 出[で]かけましょう。			
\\	支える	支[ささ]える	ささえる	
\\	父親には一家を支える責任がある。	父親[ちちおや]には 一家[いっか]を 支[ささ]える 責任[せきにん]がある。	ちちおや に は いっか を ささえる せきにん が ある	
\\	父親[ちちおや]には 一家[いっか]を
\\	責任[せきにん]がある。			
\\	支社	支社[ししゃ]	ししゃ	
\\	来月大阪に支社を開設します。	来月大阪[らいげつ おおさか]に 支社[ししゃ]を 開設[かいせつ]します。	らいげつ おおさか に ししゃ を かいせつ します	
\\	来月大阪[らいげつ おおさか]に
\\	を 開設[かいせつ]します。			
\\	支払う	支払[しはら]う	しはらう	
\\	カウンターで料金を支払った。	カウンターで 料金[りょうきん]を 支払[しはら]った。	かうんたー で りょうきん を しはらった	
\\	カウンターで 料金[りょうきん]を
\\	支払い	支払[しはら]い	しはらい	
\\	お支払いはカードもお使いいただけます。	お 支払[しはら]いはカードもお 使[つか]いいただけます。	おしはらい は かーど も お つかい いただけます	
\\	お
\\	はカードもお 使[つか]いいただけます。			
\\	打ち込む	打[う]ち 込[こ]む	うちこむ	
\\	彼は研究に打ち込んでいます。	彼[かれ]は 研究[けんきゅう]に 打[う]ち 込[こ]んでいます。	かれ は けんきゅう に うちこんで います	
\\	彼[かれ]は 研究[けんきゅう]に
\\	思い込む	思[おも]い 込[こ]む	おもいこむ	
\\	彼は騙されたと思い込んでいるようです。	彼[かれ]は 騙[だま]されたと 思[おも]い 込[こ]んでいるようです。	かれ は だまされた と おもいこんで いる よう です	
\\	彼[かれ]は 騙[だま]されたと
\\	ようです。			
\\	学期	学期[がっき]	がっき	
\\	新学期が始まったね。	新[しん] 学期[がっき]が 始[はじ]まったね。	しんがっき が はじまった ね	
\\	新[しん]
\\	が 始[はじ]まったね。			
\\	後期	後期[こうき]	こうき	
\\	後期の授業が始まりました。	後期[こうき]の 授業[じゅぎょう]が 始[はじ]まりました。	こうき の じゅぎょう が はじまりました	
\\	の 授業[じゅぎょう]が 始[はじ]まりました。			
\\	期日	期日[きじつ]	きじつ	
\\	代金を期日までにお支払いください。	代金[だいきん]を 期日[きじつ]までにお 支払[しはら]いください。	だいきん を きじつ まで に おしはらい ください	
\\	代金[だいきん]を
\\	までにお 支払[しはら]いください。			
\\	新学期	新学期[しんがっき]	しんがっき	
\\	今日から新学期が始まります。	今日[きょう]から 新学期[しんがっき]が 始[はじ]まります。	きょう から しんがっき が はじまります	
\\	今日[きょう]から
\\	が 始[はじ]まります。			
\\	期限	期限[きげん]	きげん	
\\	期限までに申し込みました。	期限[きげん]までに 申[もう]し 込[こ]みました。	きげん まで に もうしこみました	
\\	までに 申[もう]し 込[こ]みました。			
\\	大急ぎ	大急[おおいそ]ぎ	おおいそぎ	
\\	大急ぎでその仕事を仕上げたよ。	大急[おおいそ]ぎでその 仕事[しごと]を 仕上[しあ]げたよ。	おおいそぎ で その しごと を しあげた よ	
\\	でその 仕事[しごと]を 仕上[しあ]げたよ。			
\\	急用	急用[きゅうよう]	きゅうよう	
\\	彼は急用で帰りました。	彼[かれ]は 急用[きゅうよう]で 帰[かえ]りました。	かれ は きゅうよう で かえりました	
\\	彼[かれ]は
\\	で 帰[かえ]りました。			
\\	思い切って	思[おも]い 切[き]って	おもいきって	
\\	思い切って彼に相談します。	思[おも]い 切[き]って 彼[かれ]に 相談[そうだん]します。	おもいきって かれ に そうだん します	
\\	彼[かれ]に 相談[そうだん]します。			
\\	品切れ	品切[しなぎ]れ	しなぎれ	
\\	牛乳は品切れだったよ。	牛乳[ぎゅうにゅう]は 品切[しなぎ]れだったよ。	ぎゅうにゅう は しなぎれ だった よ	
\\	牛乳[ぎゅうにゅう]は
\\	だったよ。			
\\	思い切り	思[おも]い 切[き]り	おもいきり	
\\	カラオケで思い切り歌ったの。	カラオケで 思[おも]い 切[き]り 歌[うた]ったの。	からおけ で おもいきり うたった の	
\\	カラオケで
\\	歌[うた]ったの。			
\\	回数券	回数券[かいすうけん]	かいすうけん	
\\	バスの回数券を買いました。	バスの 回数券[かいすうけん]を 買[か]いました。	ばす の かいすうけん を かいました	
\\	バスの
\\	を 買[か]いました。			
\\	古代	古代[こだい]	こだい	
\\	古代の歴史について勉強しました。	古代[こだい]の 歴史[れきし]について 勉強[べんきょう]しました。	こだい の れきし に ついて べんきょう しました	
\\	の 歴史[れきし]について 勉強[べんきょう]しました。			
\\	指す	指[さ]す	さす	
\\	時計が12時を指してる。	時計[とけい]が 12時[じゅうにじ]を 指[さ]してる。	とけい が じゅうにじ を さして る	
\\	時計[とけい]が 12時[じゅうにじ]を
\\	小指	小指[こゆび]	こゆび	
\\	小指を切ってしまいました。	小指[こゆび]を 切[き]ってしまいました。	こゆび を きって しまいました	
\\	を 切[き]ってしまいました。			
\\	安定	安定[あんてい]	あんてい	
\\	彼は精神の安定が必要よ。	彼[かれ]は 精神[せいしん]の 安定[あんてい]が 必要[ひつよう]よ。	かれ は せいしん の あんてい が ひつよう よ	
\\	彼[かれ]は 精神[せいしん]の
\\	が 必要[ひつよう]よ。			
\\	定める	定[さだ]める	さだめる	
\\	税に関する新しい法律が定められたぞ。	税[ぜい]に 関[かん]する 新[あたら]しい 法律[ほうりつ]が 定[さだ]められたぞ。	ぜい に かんする あたらしい ほうりつ が さだめられた ぞ	
\\	税[ぜい]に 関[かん]する 新[あたら]しい 法律[ほうりつ]が
\\	ぞ。			
\\	指定	指定[してい]	してい	
\\	指定された席にお座りください。	指定[してい]された 席[せき]にお 座[すわ]りください。	してい された せき に お すわり ください	
\\	された 席[せき]にお 座[すわ]りください。			
\\	悪化	悪化[あっか]	あっか	
\\	手の傷が悪化した。	手[て]の 傷[きず]が 悪化[あっか]した。	て の きず が あっか した	
\\	手[て]の 傷[きず]が
\\	した。			
\\	消化	消化[しょうか]	しょうか	
\\	彼は消化不良を起こしたんだ。	彼[かれ]は 消化[しょうか] 不良[ふりょう]を 起[お]こしたんだ。	かれ は しょうか ふりょう を おこした ん だ	
\\	彼[かれ]は
\\	不良[ふりょう]を 起[お]こしたんだ。			
\\	更に	更[さら]に	さらに	
\\	彼は更に質問を続けたの。	彼[かれ]は 更[さら]に 質問[しつもん]を 続[つづ]けたの。	かれ は さらに しつもん を つづけた の	
\\	彼[かれ]は
\\	質問[しつもん]を 続[つづ]けたの。			
\\	急増	急増[きゅうぞう]	きゅうぞう	
\\	最近ヨガをやる人が急増しています。	最近[さいきん]ヨガをやる 人[ひと]が 急増[きゅうぞう]しています。	さいきん よが を やる ひと が きゅうぞう して います	
\\	最近[さいきん]ヨガをやる 人[ひと]が
\\	しています。			
\\	市立	市立[しりつ]	しりつ	
\\	娘は市立の学校に通っています。	娘[むすめ]は 市立[しりつ]の 学校[がっこう]に 通[かよ]っています。	むすめ は しりつ の がっこう に かよって います	
\\	娘[むすめ]は
\\	の 学校[がっこう]に 通[かよ]っています。			
\\	国立	国立[こくりつ]	こくりつ	
\\	新しい国立劇場が完成しました。	新[あたら]しい 国立[こくりつ] 劇場[げきじょう]が 完成[かんせい]しました。	あたらしい こくりつ げきじょう が かんせい しました	
\\	新[あたら]しい
\\	劇場[げきじょう]が 完成[かんせい]しました。			
\\	座席	座席[ざせき]	ざせき	
\\	飛行機の座席はゆったりしていたよ。	飛行機[ひこうき]の 座席[ざせき]はゆったりしていたよ。	ひこうき の ざせき は ゆったり して いた よ	
\\	飛行機[ひこうき]の
\\	はゆったりしていたよ。			
\\	客席	客席[きゃくせき]	きゃくせき	
\\	私たちは客席に座ったんだ。	私[わたし]たちは 客席[きゃくせき]に 座[すわ]ったんだ。	わたしたち は きゃくせき に すわった ん だ	
\\	私[わたし]たちは
\\	に 座[すわ]ったんだ。			
\\	欠点	欠点[けってん]	けってん	
\\	欠点のない人間はいません。	欠点[けってん]のない 人間[にんげん]はいません。	けってん の ない にんげん は いません	
\\	のない 人間[にんげん]はいません。			
\\	欠ける	欠[か]ける	かける	
\\	お気に入りのカップが欠けてしまいました。	お 気[き]に 入[い]りのカップが 欠[か]けてしまいました。	おきにいり の かっぷ が かけて しまいました	
\\	お 気[き]に 入[い]りのカップが
\\	欠く	欠[か]く	かく	
\\	彼の態度は誠意を欠いています。	彼[かれ]の 態度[たいど]は 誠意[せいい]を 欠[か]いています。	かれ の たいど は せいい を かいて います	
\\	彼[かれ]の 態度[たいど]は 誠意[せいい]を
\\	次回	次回[じかい]	じかい	
\\	次回の会議は2週間後に行います。	次回[じかい]の 会議[かいぎ]は 2週間後[に しゅうかん ご]に 行[おこな]います。	じかい の かいぎ は に しゅうかん ご に おこないます	
\\	の 会議[かいぎ]は 2週間後[に しゅうかん ご]に 行[おこな]います。			
\\	回転	回転[かいてん]	かいてん	
\\	彼はボールに回転を掛けたんだ。	彼[かれ]はボールに 回転[かいてん]を 掛[か]けたんだ。	かれ は ぼーる に かいてん を かけた ん だ	
\\	彼[かれ]はボールに
\\	を 掛[か]けたんだ。			
\\	急速	急速[きゅうそく]	きゅうそく	
\\	あの国の経済は急速に発展しているのね。	あの 国[くに]の 経済[けいざい]は 急速[きゅうそく]に 発展[はってん]しているのね。	あの くに の けいざい は きゅうそく に はってん して いる の ね	
\\	あの 国[くに]の 経済[けいざい]は
\\	に 発展[はってん]しているのね。			
\\	早速	早速[さっそく]	さっそく	
\\	では早速書類をお送りします。	では 早速[さっそく] 書類[しょるい]をお 送[おく]りします。	では さっそく しょるい を おおくり します	
\\	では
\\	書類[しょるい]をお 送[おく]りします。			
\\	時速	時速[じそく]	じそく	
\\	新幹線の最高時速は300キロです。	新幹線[しんかんせん]の 最高[さいこう] 時速[じそく]は 300[さんびゃく]キロです。	しんかんせん の さいこう じそく は さんびゃくきろ です	
\\	新幹線[しんかんせん]の 最高[さいこう]
\\	は 300[さんびゃく]キロです。			
\\	最終	最終[さいしゅう]	さいしゅう	
\\	東京行きの最終電車は何時ですか。	東京行[とうきょうゆ]きの 最終[さいしゅう] 電車[でんしゃ]は 何時[なんじ]ですか。	とうきょうゆき の さいしゅう でんしゃ は なんじ です か	
\\	東京行[とうきょうゆ]きの
\\	電車[でんしゃ]は 何時[なんじ]ですか。			
\\	始終	始終[しじゅう]	しじゅう	
\\	その部屋は始終、人が出入りしているね。	その 部屋[へや]は 始終[しじゅう]、 人[ひと]が 出入[でい]りしているね。	その へや は しじゅう ひと が でいり して いる ね	
\\	その 部屋[へや]は
\\	、 人[ひと]が 出入[でい]りしているね。			
\\	在学	在学[ざいがく]	ざいがく	
\\	姉は大学に在学しています。	姉[あね]は 大学[だいがく]に 在学[ざいがく]しています。	あね は だいがく に ざいがく して います	
\\	姉[あね]は 大学[だいがく]に
\\	しています。			
\\	実は	実[じつ]は	じつは	
\\	あれは実は私の勘違いでした。	あれは 実[じつ]は 私[わたし]の 勘違[かんちが]いでした。	あれ は じつは わたし の かんちがい でした	
\\	あれは
\\	私[わたし]の 勘違[かんちが]いでした。			
\\	実用	実用[じつよう]	じつよう	
\\	これはとても実用的なサイトだね。	これはとても 実用[じつよう] 的[てき]なサイトだね。	これ は とても じつようてき な さいと だ ね	
\\	これはとても
\\	的[てき]なサイトだね。			
\\	実力	実力[じつりょく]	じつりょく	
\\	二人の実力は互角です。	二人[ふたり]の 実力[じつりょく]は 互角[ごかく]です。	ふたり の じつりょく は ごかく です	
\\	二人[ふたり]の
\\	は 互角[ごかく]です。			
\\	実習	実習[じっしゅう]	じっしゅう	
\\	今日は料理の実習があった。	今日[きょう]は 料理[りょうり]の 実習[じっしゅう]があった。	きょう は りょうり の じっしゅう が あった	
\\	今日[きょう]は 料理[りょうり]の
\\	があった。			
\\	実物	実物[じつぶつ]	じつぶつ	
\\	何かの説明をする時は実物を使うとわかりやすいの。	何[なに]かの 説明[せつめい]をする 時[とき]は 実物[じつぶつ]を 使[つか]うとわかりやすいの。	なにか の せつめい を する とき は じつぶつ を つかう と わかり やすい の	
\\	何[なに]かの 説明[せつめい]をする 時[とき]は
\\	を 使[つか]うとわかりやすいの。			
\\	実に	実[じつ]に	じつに	
\\	昨日のコンサートは実に素晴らしかったよ。	昨日[さくじつ]のコンサートは 実[じつ]に 素晴[すば]らしかったよ。	さくじつ の こんさーと は じつに すばらしかった よ	
\\	昨日[さくじつ]のコンサートは
\\	素晴[すば]らしかったよ。			
\\	活発	活発[かっぱつ]	かっぱつ	
\\	活発な意見が交されたの。	活発[かっぱつ]な 意見[いけん]が 交[かわ]されたの。	かっぱつ な いけん が かわされた の	
\\	な 意見[いけん]が 交[かわ]されたの。			
\\	始発	始発[しはつ]	しはつ	
\\	始発の電車に乗りました。	始発[しはつ]の 電車[でんしゃ]に 乗[の]りました。	しはつ の でんしゃ に のりました	
\\	の 電車[でんしゃ]に 乗[の]りました。			
\\	気楽	気楽[きらく]	きらく	
\\	将来は気楽な生活がしたいです。	将来[しょうらい]は 気楽[きらく]な 生活[せいかつ]がしたいです。	しょうらい は きらく な せいかつ が したい です	
\\	将来[しょうらい]は
\\	な 生活[せいかつ]がしたいです。			
\\	映る	映[うつ]る	うつる	
\\	水面に月が映っているね。	水面[みなも]に 月[つき]が 映[うつ]っているね。	みなも に つき が うつって いる ね	
\\	水面[みなも]に 月[つき]が
\\	ね。			
\\	映す	映[うつ]す	うつす	
\\	彼女は自分の姿を鏡に映したの。	彼女[かのじょ]は 自分[じぶん]の 姿[すがた]を 鏡[かがみ]に 映[うつ]したの。	かのじょ は じぶん の すがた を かがみ に うつした の	
\\	彼女[かのじょ]は 自分[じぶん]の 姿[すがた]を 鏡[かがみ]に
\\	の。			
\\	地面	地面[じめん]	じめん	
\\	地面に何か絵が描いてあるぞ。	地面[じめん]に 何[なに]か 絵[え]が 描[か]いてあるぞ。	じめん に なにか え が かいて ある ぞ	
\\	に 何[なに]か 絵[え]が 描[か]いてあるぞ。			
\\	水面	水面[すいめん]	すいめん	
\\	湖の水面に小さく波が立っているな。	湖[みずうみ]の 水面[すいめん]に 小[ちい]さく 波[なみ]が 立[た]っているな。	みずうみ の すいめん に ちいさく なみ が たって いる な	
\\	湖[みずうみ]の
\\	に 小[ちい]さく 波[なみ]が 立[た]っているな。			
\\	形式	形式[けいしき]	けいしき	
\\	書類は形式を守って作ってください。	書類[しょるい]は 形式[けいしき]を 守[まも]って 作[つく]ってください。	しょるい は けいしき を まもって つくって ください	
\\	書類[しょるい]は
\\	を 守[まも]って 作[つく]ってください。			
\\	小型	小型[こがた]	こがた	
\\	小型のスーツケースを買いました。	小型[こがた]のスーツケースを 買[か]いました。	こがた の すーつけーす を かいました	
\\	のスーツケースを 買[か]いました。			
\\	型	型[かた]	かた	
\\	新しい型のカメラを買いました。	新[あたら]しい 型[かた]のカメラを 買[か]いました。	あたらしい かた の かめら を かいました	
\\	新[あたら]しい
\\	のカメラを 買[か]いました。			
\\	各種	各種[かくしゅ]	かくしゅ	
\\	図書館には各種の雑誌が揃っています。	図書館[としょかん]には 各種[かくしゅ]の 雑誌[ざっし]が 揃[そろ]っています。	としょかん に は かくしゅ の ざっし が そろって います	
\\	図書館[としょかん]には
\\	の 雑誌[ざっし]が 揃[そろ]っています。			
\\	書類	書類[しょるい]	しょるい	
\\	書類を10枚コピーしました。	書類[しょるい]を 10枚[じゅうまい]コピーしました。	しょるい を じゅうまい こぴー しました	
\\	を 10枚[じゅうまい]コピーしました。			
\\	接近	接近[せっきん]	せっきん	
\\	台風が接近していますね。	台風[たいふう]が 接近[せっきん]していますね。	たいふう が せっきん して います ね	
\\	台風[たいふう]が
\\	していますね。			
\\	接する	接[せっ]する	せっする	
\\	子供が動物に接するのは良いことです。	子供[こども]が 動物[どうぶつ]に 接[せっ]するのは 良[よ]いことです。	こども が どうぶつ に せっする の は よい こと です	
\\	子供[こども]が 動物[どうぶつ]に
\\	のは 良[よ]いことです。			
\\	曲線	曲線[きょくせん]	きょくせん	
\\	この道は、ゆるい曲線を描いているよ。	この 道[みち]は、ゆるい 曲線[きょくせん]を 描[えが]いているよ。	この みち は ゆるい きょくせん を えがいて いる よ	
\\	この 道[みち]は、ゆるい
\\	を 描[えが]いているよ。			
\\	合同	合同[ごうどう]	ごうどう	
\\	3社合同で新作の発表会を開きました。	3社[さんしゃ] 合同[ごうどう]で 新作[しんさく]の 発表会[はっぴょうかい]を 開[ひら]きました。	さんしゃ ごうどう で しんさく の はっぴょうかい を ひらきました	
\\	3社[さんしゃ]
\\	で 新作[しんさく]の 発表会[はっぴょうかい]を 開[ひら]きました。			
\\	海洋	海洋[かいよう]	かいよう	
\\	その昔航海士達は未踏の地を求め海洋に乗り出しました。	その 昔航海士達[むかし こうかいしたち]は 未踏[みとう]の 地[ち]を 求[もと]め 海洋[かいよう]に 乗[の]り 出[だ]しました。	その むかし こうかいしたち は みとう の ち を もとめ かいよう に のりだしました	
\\	その 昔航海士達[むかし こうかいしたち]は 未踏[みとう]の 地[ち]を 求[もと]め
\\	に 乗[の]り 出[だ]しました。			
\\	室内	室内[しつない]	しつない	
\\	雨の日は子供を室内で遊ばせます。	雨[あめ]の 日[ひ]は 子供[こども]を 室内[しつない]で 遊[あそ]ばせます。	あめ の ひ は こども を しつない で あそばせます	
\\	雨[あめ]の 日[ひ]は 子供[こども]を
\\	で 遊[あそ]ばせます。			
\\	水族館	水族館[すいぞくかん]	すいぞくかん	
\\	ここの水族館にはイルカがいます。	ここの 水族館[すいぞくかん]にはイルカがいます。	ここ の すいぞくかん に は いるか が います	
\\	ここの
\\	にはイルカがいます。			
\\	歳末	歳末[さいまつ]	さいまつ	
\\	デパートの歳末大売出しが始まったよ。	デパートの 歳末[さいまつ] 大売出[おおうりだ]しが 始[はじ]まったよ。	でぱーと の さいまつ おおうりだし が はじまった よ	
\\	デパートの
\\	大売出[おおうりだ]しが 始[はじ]まったよ。			
\\	姉妹	姉妹[しまい]	しまい	
\\	うちは3人姉妹です。	うちは 3人[さんにん] 姉妹[しまい]です。	うち は さんにん しまい です	
\\	うちは 3人[さんにん]
\\	です。			
\\	次第に	次第[しだい]に	しだいに	
\\	その事件は次第に忘れられていったのさ。	その 事件[じけん]は 次第[しだい]に 忘[わす]れられていったのさ。	その じけん は しだいに わすれられて いった の さ	
\\	その 事件[じけん]は
\\	忘[わす]れられていったのさ。			
\\	次第	次第[しだい]	しだい	
\\	連絡があり次第出発します。	連絡[れんらく]があり 次第[しだい] 出発[しゅっぱつ]します。	れんらく が あり しだい しゅっぱつ します	
\\	連絡[れんらく]があり
\\	出発[しゅっぱつ]します。			
\\	息	息[いき]	いき	
\\	大きく息を吸ってください。	大[おお]きく 息[いき]を 吸[す]ってください。	おおきく いき を すって ください	
\\	大[おお]きく
\\	を 吸[す]ってください。			
\\	学者	学者[がくしゃ]	がくしゃ	
\\	彼は作家であり学者です。	彼[かれ]は 作家[さっか]であり 学者[がくしゃ]です。	かれ は さっか で あり がくしゃ です	
\\	彼[かれ]は 作家[さっか]であり
\\	です。			
\\	後者	後者[こうしゃ]	こうしゃ	
\\	正しい答は後者です。	正[ただ]しい 答[こたえ]は 後者[こうしゃ]です。	ただしい こたえ は こうしゃ です	
\\	正[ただ]しい 答[こたえ]は
\\	です。			
\\	新婚	新婚[しんこん]	しんこん	
\\	妹夫婦は新婚です。	妹夫婦[いもうとふうふ]は 新婚[しんこん]です。	いもうとふうふ は しんこん です	
\\	妹夫婦[いもうとふうふ]は
\\	です。			
\\	婚約	婚約[こんやく]	こんやく	
\\	二人は婚約しています。	二人[ふたり]は 婚約[こんやく]しています。	ふたり は こんやく して います	
\\	二人[ふたり]は
\\	しています。			
\\	各自	各自[かくじ]	かくじ	
\\	ごみは各自で持ち帰ってください。	ごみは 各自[かくじ]で 持[も]ち 帰[かえ]ってください。	ごみ は かくじ で もちかえって ください	
\\	ごみは
\\	で 持[も]ち 帰[かえ]ってください。			
\\	国民	国民[こくみん]	こくみん	
\\	国民の安全が最も大切です。	国民[こくみん]の 安全[あんぜん]が 最[もっと]も 大切[たいせつ]です。	こくみん の あんぜん が もっとも たいせつ です	
\\	の 安全[あんぜん]が 最[もっと]も 大切[たいせつ]です。			
\\	応じる	応[おう]じる	おうじる	
\\	私は彼の要望に応じました。	私[わたし]は 彼[かれ]の 要望[ようぼう]に 応[おう]じました。	わたし は かれ の ようぼう に おうじました	
\\	私[わたし]は 彼[かれ]の 要望[ようぼう]に
\\	応用	応用[おうよう]	おうよう	
\\	このレシピはいろいろ応用できます。	このレシピはいろいろ 応用[おうよう]できます。	この れしぴ は いろいろ おうよう できます	
\\	このレシピはいろいろ
\\	できます。			
\\	回答	回答[かいとう]	かいとう	
\\	アンケートに回答しました。	アンケートに 回答[かいとう]しました。	あんけーと に かいとう しました	
\\	アンケートに
\\	しました。			
\\	差別	差別[さべつ]	さべつ	
\\	彼は差別をなくす運動をしています。	彼[かれ]は 差別[さべつ]をなくす 運動[うんどう]をしています。	かれ は さべつ を なくす うんどう を して います	
\\	彼[かれ]は
\\	をなくす 運動[うんどう]をしています。			
\\	性別	性別[せいべつ]	せいべつ	
\\	出席者を性別で分けてください。	出席者[しゅっせきしゃ]を 性別[せいべつ]で 分[わ]けてください。	しゅっせきしゃ を せいべつ で わけて ください	
\\	出席者[しゅっせきしゃ]を
\\	で 分[わ]けてください。			
\\	合理的	合理的[ごうりてき]	ごうりてき	
\\	彼女は合理的な考え方をする人です。	彼女[かのじょ]は 合理的[ごうりてき]な 考[かんが]え 方[かた]をする 人[ひと]です。	かのじょ は ごうりてき な かんがえかた を する ひと です	
\\	彼女[かのじょ]は
\\	な 考[かんが]え 方[かた]をする 人[ひと]です。			
\\	形式的	形式的[けいしきてき]	けいしきてき	
\\	彼は形式的な質問をしただけだった。	彼[かれ]は 形式的[けいしきてき]な 質問[しつもん]をしただけだった。	かれ は けいしきてき な しつもん を した だけ だった	
\\	彼[かれ]は
\\	な 質問[しつもん]をしただけだった。			
\\	実用的	実用的[じつようてき]	じつようてき	
\\	彼の発明品はみな実用的だよ。	彼[かれ]の 発明品[はつめいひん]はみな 実用的[じつようてき]だよ。	かれ の はつめいひん は みな じつようてき だ よ	
\\	彼[かれ]の 発明品[はつめいひん]はみな
\\	だよ。			
\\	女性的	女性的[じょせいてき]	じょせいてき	
\\	彼は言葉遣いが少し女性的だね。	彼[かれ]は 言葉遣[ことばづか]いが 少[すこ]し 女性的[じょせいてき]だね。	かれ は ことばづかい が すこし じょせいてき だ ね	
\\	彼[かれ]は 言葉遣[ことばづか]いが 少[すこ]し
\\	だね。			
\\	水平	水平[すいへい]	すいへい	
\\	この棚は水平になっていませんね。	この 棚[たな]は 水平[すいへい]になっていませんね。	この たな は すいへい に なって いません ね	
\\	この 棚[たな]は
\\	になっていませんね。			
\\	水平線	水平線[すいへいせん]	すいへいせん	
\\	水平線に太陽が沈んでいった。	水平線[すいへいせん]に 太陽[たいよう]が 沈[しず]んでいった。	すいへいせん に たいよう が しずんで いった	
\\	に 太陽[たいよう]が 沈[しず]んでいった。			
\\	急病	急病[きゅうびょう]	きゅうびょう	
\\	知り合いが急病で倒れた。	知[し]り 合[あ]いが 急病[きゅうびょう]で 倒[たお]れた。	しりあい が きゅうびょう で たおれた	
\\	知[し]り 合[あ]いが
\\	で 倒[たお]れた。			
\\	学科	学科[がっか]	がっか	
\\	彼の得意な学科は数学です。	彼[かれ]の 得意[とくい]な 学科[がっか]は 数学[すうがく]です。	かれ の とくい な がっか は すうがく です	
\\	彼[かれ]の 得意[とくい]な
\\	は 数学[すうがく]です。			
\\	外科	外科[げか]	げか	
\\	友人が骨折して外科に入院しました。	友人[ゆうじん]が 骨折[こっせつ]して 外科[げか]に 入院[にゅういん]しました。	ゆうじん が こっせつ して げか に にゅういん しました	
\\	友人[ゆうじん]が 骨折[こっせつ]して
\\	に 入院[にゅういん]しました。			
\\	歯科	歯科[しか]	しか	
\\	彼は歯科医師です。	彼[かれ]は 歯科[しか] 医師[いし]です。	かれ は しか いし です	
\\	彼[かれ]は
\\	医師[いし]です。			
\\	死	死[し]	し	
\\	死を恐れるのは自然なことです。	死[し]を 恐[おそ]れるのは 自然[しぜん]なことです。	し を おそれる の は しぜん な こと です	
\\	を 恐[おそ]れるのは 自然[しぜん]なことです。			
\\	死者	死者[ししゃ]	ししゃ	
\\	その事故で30人の死者が出たの。	その 事故[じこ]で 30人[さんじゅうにん]の 死者[ししゃ]が 出[で]たの。	その じこ で さんじゅうにん の ししゃ が でた の	
\\	その 事故[じこ]で 30人[さんじゅうにん]の
\\	が 出[で]たの。			
\\	死体	死体[したい]	したい	
\\	公園で死体が見つかったの。	公園[こうえん]で 死体[したい]が 見[み]つかったの。	こうえん で したい が みつかった の	
\\	公園[こうえん]で
\\	が 見[み]つかったの。			
\\	死亡	死亡[しぼう]	しぼう	
\\	その事故では2人死亡したの。	その 事故[じこ]では 2人[ふたり] 死亡[しぼう]したの。	その じこ で は ふたり しぼう した の	
\\	その 事故[じこ]では 2人[ふたり]
\\	したの。			
\\	合意	合意[ごうい]	ごうい	
\\	両社が合併に合意しました。	両社[りょうしゃ]が 合併[がっぺい]に 合意[ごうい]しました。	りょうしゃ が がっぺい に ごうい しました	
\\	両社[りょうしゃ]が 合併[がっぺい]に
\\	しました。			
\\	意向	意向[いこう]	いこう	
\\	彼の意向を聞いてみましょう。	彼[かれ]の 意向[いこう]を 聞[き]いてみましょう。	かれ の いこう を きいて みましょう	
\\	彼[かれ]の
\\	を 聞[き]いてみましょう。			
\\	意欲	意欲[いよく]	いよく	
\\	彼は仕事に意欲を燃やしています。	彼[かれ]は 仕事[しごと]に 意欲[いよく]を 燃[も]やしています。	かれ は しごと に いよく を もやして います	
\\	彼[かれ]は 仕事[しごと]に
\\	を 燃[も]やしています。			
\\	決意	決意[けつい]	けつい	
\\	彼の決意は堅いな。	彼[かれ]の 決意[けつい]は 堅[かた]いな。	かれ の けつい は かたい な	
\\	彼[かれ]の
\\	は 堅[かた]いな。			
\\	意図	意図[いと]	いと	
\\	あなたの意図はよく分かりました。	あなたの 意図[いと]はよく 分[わ]かりました。	あなた の いと は よく わかりました	
\\	あなたの
\\	はよく 分[わ]かりました。			
\\	意外	意外[いがい]	いがい	
\\	意外にも彼は独身です。	意外[いがい]にも 彼[かれ]は 独身[どくしん]です。	いがい に も かれ は どくしん です	
\\	にも 彼[かれ]は 独身[どくしん]です。			
\\	意義	意義[いぎ]	いぎ	
\\	この事業には大きな意義があります。	この 事業[じぎょう]には 大[おお]きな 意義[いぎ]があります。	この じぎょう に は おおき な いぎ が あります	
\\	この 事業[じぎょう]には 大[おお]きな
\\	があります。			
\\	好意	好意[こうい]	こうい	
\\	彼は彼女に好意をもっています。	彼[かれ]は 彼女[かのじょ]に 好意[こうい]をもっています。	かれ は かのじょ に こうい を もって います	
\\	彼[かれ]は 彼女[かのじょ]に
\\	をもっています。			
\\	意地悪	意地悪[いじわる]	いじわる	
\\	彼は時々意地悪な質問をする。	彼[かれ]は 時々[ときどき] 意地悪[いじわる]な 質問[しつもん]をする。	かれ は ときどき いじわる な しつもん を する 。	
\\	彼[かれ]は 時々[ときどき]
\\	な 質問[しつもん]をする。			
\\	機会	機会[きかい]	きかい	
\\	家族で話し合う機会を持ちました。	家族[かぞく]で 話[はな]し 合[あ]う 機会[きかい]を 持[も]ちました。	かぞく で はなしあう きかい を もちました	
\\	家族[かぞく]で 話[はな]し 合[あ]う
\\	を 持[も]ちました。			
\\	機長	機長[きちょう]	きちょう	
\\	機長の放送があったの。	機長[きちょう]の 放送[ほうそう]があったの。	きちょう の ほうそう が あった の	
\\	の 放送[ほうそう]があったの。			
\\	時機	時機[じき]	じき	
\\	あせらずに時機を待つべきです。	あせらずに 時機[じき]を 待[ま]つべきです。	あせらず に じき を まつ べき です	
\\	あせらずに
\\	を 待[ま]つべきです。			
\\	楽器	楽器[がっき]	がっき	
\\	私が好きな楽器はギターです。	私[わたし]が 好[す]きな 楽器[がっき]はギターです。	わたし が すき な がっき は ぎたー です	
\\	私[わたし]が 好[す]きな
\\	はギターです。			
\\	器用	器用[きよう]	きよう	
\\	彼はかなり器用な人です。	彼[かれ]はかなり 器用[きよう]な 人[ひと]です。	かれ は かなり きよう な ひと です	
\\	彼[かれ]はかなり
\\	な 人[ひと]です。			
\\	受話器	受話器[じゅわき]	じゅわき	
\\	受話器を取ってもらえますか。	受話器[じゅわき]を 取[と]ってもらえますか。	じゅわき を とって もらえます か	
\\	を 取[と]ってもらえますか。			
\\	器	器[うつわ]	うつわ	
\\	この器は上等ね。	この 器[うつわ]は 上等[じょうとう]ね。	この うつわ は じょうとう ね	
\\	この
\\	は 上等[じょうとう]ね。			
\\	器械	器械[きかい]	きかい	
\\	体育館で器械を使って運動したの。	体育館[たいいくかん]で 器械[きかい]を 使[つか]って 運動[うんどう]したの。	たいいくかん で きかい を つかって うんどう した の	
\\	体育館[たいいくかん]で
\\	を 使[つか]って 運動[うんどう]したの。			
\\	取材	取材[しゅざい]	しゅざい	
\\	テレビも取材に来たほど有名なレストランです。	テレビも 取材[しゅざい]に 来[き]たほど 有名[ゆうめい]なレストランです。	てれび も しゅざい に きた ほど ゆうめい な れすとらん です	
\\	テレビも
\\	に 来[き]たほど 有名[ゆうめい]なレストランです。			
\\	材木	材木[ざいもく]	ざいもく	
\\	船から材木が降ろされていますね。	船[ふね]から 材木[ざいもく]が 降[お]ろされていますね。	ふね から ざいもく が おろされて います ね	
\\	船[ふね]から
\\	が 降[お]ろされていますね。			
\\	器具	器具[きぐ]	きぐ	
\\	これはスポーツ施設用の器具です。	これはスポーツ 施設用[しせつよう]の 器具[きぐ]です。	これ は すぽーつ しせつよう の きぐ です	
\\	これはスポーツ 施設用[しせつよう]の
\\	です。			
\\	家具	家具[かぐ]	かぐ	
\\	部屋の家具を動かしました。	部屋[へや]の 家具[かぐ]を 動[うご]かしました。	へや の かぐ を うごかしました	
\\	部屋[へや]の
\\	を 動[うご]かしました。			
\\	基地	基地[きち]	きち	
\\	ここは昔、軍事基地でした。	ここは 昔[むかし]、 軍事[ぐんじ] 基地[きち]でした。	ここ は むかし ぐんじ きち でした	
\\	ここは 昔[むかし]、 軍事[ぐんじ]
\\	でした。			
\\	水準	水準[すいじゅん]	すいじゅん	
\\	今年の応募作品は水準が高かったね。	今年[ことし]の 応募作品[おうぼ さくひん]は 水準[すいじゅん]が 高[たか]かったね。	ことし の おうぼ さくひん は すいじゅん が たかかった ね	
\\	今年[ことし]の 応募作品[おうぼ さくひん]は
\\	が 高[たか]かったね。			
\\	基準	基準[きじゅん]	きじゅん	
\\	この建物は建築の基準に達していないよ。	この 建物[たてもの]は 建築[けんちく]の 基準[きじゅん]に 達[たっ]していないよ。	この たてもの は けんちく の きじゅん に たっしていない よ 。	
\\	この 建物[たてもの]は 建築[けんちく]の
\\	に 達[たっ]していないよ。			
\\	学説	学説[がくせつ]	がくせつ	
\\	それは最新の学説ね。	それは 最新[さいしん]の 学説[がくせつ]ね。	それ は さいしん の がくせつ ね	
\\	それは 最新[さいしん]の
\\	ね。			
\\	学園	学園[がくえん]	がくえん	
\\	彼女は郊外の学園に通っているね。	彼女[かのじょ]は 郊外[こうがい]の 学園[がくえん]に 通[かよ]っているね。	かのじょ は こうがい の がくえん に かよって いる ね	
\\	彼女[かのじょ]は 郊外[こうがい]の
\\	に 通[かよ]っているね。			
\\	国際的	国際的[こくさいてき]	こくさいてき	
\\	彼女は国際的に有名な歌手です。	彼女[かのじょ]は 国際的[こくさいてき]に 有名[ゆうめい]な 歌手[かしゅ]です。	かのじょ は こくさいてき に ゆうめい な かしゅ です	
\\	彼女[かのじょ]は
\\	に 有名[ゆうめい]な 歌手[かしゅ]です。			
\\	国際化	国際化[こくさいか]	こくさいか	
\\	この大学も国際化してきたな。	この 大学[だいがく]も 国際化[こくさいか]してきたな。	この だいがく も こくさいか して きた な	
\\	この 大学[だいがく]も
\\	してきたな。			
\\	完全	完全[かんぜん]	かんぜん	
\\	この古い寺院は今でも完全な形を保っているんだ。	この 古[ふる]い 寺院[じいん]は 今[いま]でも 完全[かんぜん]な 形[かたち]を 保[たも]っているんだ。	この ふるい じいん は いま で も かんぜん な かたち を たもって いる ん だ	
\\	この 古[ふる]い 寺院[じいん]は 今[いま]でも
\\	な 形[かたち]を 保[たも]っているんだ。			
\\	成長	成長[せいちょう]	せいちょう	
\\	庭の木、大きく成長したわね。	庭[にわ]の 木[き]、 大[おお]きく 成長[せいちょう]したわね。	にわ の き、 おおきく せいちょう した わ ね	
\\	庭[にわ]の 木[き]、 大[おお]きく
\\	したわね。			
\\	成立	成立[せいりつ]	せいりつ	
\\	あの会社との契約が成立しました。	あの 会社[かいしゃ]との 契約[けいやく]が 成立[せいりつ]しました。	あの かいしゃ と の けいやく が せいりつ しました	
\\	あの 会社[かいしゃ]との 契約[けいやく]が
\\	しました。			
\\	形成	形成[けいせい]	けいせい	
\\	今は骨が形成される大切な時期です。	今[いま]は 骨[ほね]が 形成[けいせい]される 大切[たいせつ]な 時期[じき]です。	いま は ほね が けいせい される たいせつ な じき です	
\\	今[いま]は 骨[ほね]が
\\	される 大切[たいせつ]な 時期[じき]です。			
\\	成果	成果[せいか]	せいか	
\\	厳しい練習が良い成果に結び付きました。	厳[きび]しい 練習[れんしゅう]が 良[よ]い 成果[せいか]に 結[むす]び 付[つ]きました。	きびしい れんしゅう が よい せいか に むすびつきました	
\\	厳[きび]しい 練習[れんしゅう]が 良[よ]い
\\	に 結[むす]び 付[つ]きました。			
\\	合成	合成[ごうせい]	ごうせい	
\\	このソフトで画像を合成できます。	このソフトで 画像[がぞう]を 合成[ごうせい]できます。	この そふと で がぞう を ごうせい できます	
\\	このソフトで 画像[がぞう]を
\\	できます。			
\\	成人	成人[せいじん]	せいじん	
\\	娘が今年成人します。	娘[むすめ]が 今年[ことし] 成人[せいじん]します。	むすめ が ことし せいじん します	
\\	娘[むすめ]が 今年[ことし]
\\	します。			
\\	成年	成年[せいねん]	せいねん	
\\	成年になると独立した戸籍を作れます。	成年[せいねん]になると 独立[どくりつ]した 戸籍[こせき]を 作[つく]れます。	せいねん に なる と どくりつ した こせき を つくれます	
\\	になると 独立[どくりつ]した 戸籍[こせき]を 作[つく]れます。			
\\	失う	失[うしな]う	うしなう	
\\	彼は地震で親を失いました。	彼[かれ]は 地震[じしん]で 親[おや]を 失[うしな]いました。	かれ は じしん で おや を うしないました	
\\	彼[かれ]は 地震[じしん]で 親[おや]を
\\	失業	失業[しつぎょう]	しつぎょう	
\\	友達のお父さんが突然失業したの。	友達[ともだち]のお 父[とう]さんが 突然[とつぜん] 失業[しつぎょう]したの。	ともだち の おとうさん が とつぜん しつぎょう した の	
\\	友達[ともだち]のお 父[とう]さんが 突然[とつぜん]
\\	したの。			
\\	失敗	失敗[しっぱい]	しっぱい	
\\	一度の失敗であきらめてはいけないよ。	一度[いちど]の 失敗[しっぱい]であきらめてはいけないよ。	いちど の しっぱい で あきらめては いけない よ	
\\	一度[いちど]の
\\	であきらめてはいけないよ。			
\\	正式	正式[せいしき]	せいしき	
\\	正式な招待状を受け取りました。	正式[せいしき]な 招待状[しょうたいじょう]を 受[う]け 取[と]りました。	せいしき な しょうたいじょう を うけとりました	
\\	な 招待状[しょうたいじょう]を 受[う]け 取[と]りました。			
\\	正面	正面[しょうめん]	しょうめん	
\\	その家の正面には大きなバルコニーがあるの。	その 家[いえ]の 正面[しょうめん]には 大[おお]きなバルコニーがあるの。	その いえ の しょうめん に は おおき な ばるこにー が ある の	
\\	その 家[いえ]の
\\	には 大[おお]きなバルコニーがあるの。			
\\	正午	正午[しょうご]	しょうご	
\\	昼休みは正午からです。	昼休[ひるやす]みは 正午[しょうご]からです。	ひるやすみ は しょうご から です	
\\	昼休[ひるやす]みは
\\	からです。			
\\	正義	正義[せいぎ]	せいぎ	
\\	この世に正義はないのだろうか。	この 世[よ]に 正義[せいぎ]はないのだろうか。	この よ に せいぎ は ない の だろう か	
\\	この 世[よ]に
\\	はないのだろうか。			
\\	正門	正門[せいもん]	せいもん	
\\	受験生は正門から入って下さい。	受験生[じゅけんせい]は 正門[せいもん]から 入[はい]って 下[くだ]さい。	じゅけんせい は せいもん から はいって ください	
\\	受験生[じゅけんせい]は
\\	から 入[はい]って 下[くだ]さい。			
\\	正解	正解[せいかい]	せいかい	
\\	10問中9問正解しました。	10問中9問[じゅうもんちゅう きゅうもん] 正解[せいかい]しました。	じゅうもんちゅう きゅうもん せいかい しました	
\\	10問中9問[じゅうもんちゅう きゅうもん]
\\	しました。			
\\	正方形	正方形[せいほうけい]	せいほうけい	
\\	正方形の紙を用意しましょう。	正方形[せいほうけい]の 紙[かみ]を 用意[ようい]しましょう。	せいほうけい の かみ を ようい しましょう	
\\	の 紙[かみ]を 用意[ようい]しましょう。			
\\	正	正[せい]	せい	
\\	書類は正と副の2通あります。	書類[しょるい]は 正[せい]と 副[ふく]の 2通[につう]あります。	しょるい は せい と ふく の につう あります	
\\	書類[しょるい]は
\\	と 副[ふく]の 2通[につう]あります。			
\\	正座	正座[せいざ]	せいざ	
\\	彼はきちんと正座して待っていたね。	彼[かれ]はきちんと 正座[せいざ]して 待[ま]っていたね。	かれ は きちんと せいざ して まって いた ね	
\\	彼[かれ]はきちんと
\\	して 待[ま]っていたね。			
\\	正当	正当[せいとう]	せいとう	
\\	これは正当な要求です。	これは 正当[せいとう]な 要求[ようきゅう]です。	これ は せいとう な ようきゅう です	
\\	これは
\\	な 要求[ようきゅう]です。			
\\	正常	正常[せいじょう]	せいじょう	
\\	患者の呼吸は正常です。	患者[かんじゃ]の 呼吸[こきゅう]は 正常[せいじょう]です。	かんじゃ の こきゅう は せいじょう です	
\\	患者[かんじゃ]の 呼吸[こきゅう]は
\\	です。			
\\	意識	意識[いしき]	いしき	
\\	彼は意識を失いました。	彼[かれ]は 意識[いしき]を 失[うしな]いました。	かれ は いしき を うしないました	
\\	彼[かれ]は
\\	を 失[うしな]いました。			
\\	常識	常識[じょうしき]	じょうしき	
\\	そんなの常識だよ。	そんなの 常識[じょうしき]だよ。	そんな の じょうしき だ よ	
\\	そんなの
\\	だよ。			
\\	好調	好調[こうちょう]	こうちょう	
\\	今月はエアコンの売り上げが好調です。	今月[こんげつ]はエアコンの 売[う]り 上[あ]げが 好調[こうちょう]です。	こんげつ は えあこん の うりあげ が こうちょう です	
\\	今月[こんげつ]はエアコンの 売[う]り 上[あ]げが
\\	です。			
\\	整備	整備[せいび]	せいび	
\\	車は整備に出しています。	車[くるま]は 整備[せいび]に 出[だ]しています。	くるま は せいび に だして います	
\\	車[くるま]は
\\	に 出[だ]しています。			
\\	整理	整理[せいり]	せいり	
\\	古い服を整理しました。	古[ふる]い 服[ふく]を 整理[せいり]しました。	ふるい ふく を せいり しました	
\\	古[ふる]い 服[ふく]を
\\	しました。			
\\	検査	検査[けんさ]	けんさ	
\\	私は今日、目の検査を受けます。	私[わたし]は 今日[きょう]、 目[め]の 検査[けんさ]を 受[う]けます。	わたし は きょう め の けんさ を うけます	
\\	私[わたし]は 今日[きょう]、 目[め]の
\\	を 受[う]けます。			
\\	案	案[あん]	あん	
\\	もっと案を出し合いましょう。	もっと 案[あん]を 出[だ]し 合[あ]いましょう。	もっと あん を だしあいましょう	
\\	もっと
\\	を 出[だ]し 合[あ]いましょう。			
\\	案外	案外[あんがい]	あんがい	
\\	彼は案外いい人かも知れない。	彼[かれ]は 案外[あんがい]いい 人[ひと]かも 知[し]れない。	かれ は あんがい いい ひと かも しれない	
\\	彼[かれ]は
\\	いい 人[ひと]かも 知[し]れない。			
\\	案の定	案[あん]の 定[じょう]	あんのじょう	
\\	案の定、彼は遅刻したな。	案[あん]の 定[じょう]、 彼[かれ]は 遅刻[ちこく]したな。	あんのじょう かれ は ちこく した な	
\\	、 彼[かれ]は 遅刻[ちこく]したな。			
\\	国連	国連[こくれん]	こくれん	
\\	国連の本部はニューヨークにあります。	国連[こくれん]の 本部[ほんぶ]はニューヨークにあります。	こくれん の ほんぶ は にゅーよーく に あります	
\\	の 本部[ほんぶ]はニューヨークにあります。			
\\	接続	接続[せつぞく]	せつぞく	
\\	コンピューターをネットワークに接続しました。	コンピューターをネットワークに 接続[せつぞく]しました。	こんぴゅーたー を ねっとわーく に せつぞく しました	
\\	コンピューターをネットワークに
\\	しました。			
\\	外相	外相[がいしょう]	がいしょう	
\\	外相は来週訪米の予定です。	外相[がいしょう]は 来週訪米[らいしゅう ほうべい]の 予定[よてい]です。	がいしょう は らいしゅう ほうべい の よてい です	
\\	は 来週訪米[らいしゅう ほうべい]の 予定[よてい]です。			
\\	暗記	暗記[あんき]	あんき	
\\	試験の前に英文を暗記したんだ。	試験[しけん]の 前[まえ]に 英文[えいぶん]を 暗記[あんき]したんだ。	しけん の まえ に えいぶん を あんき した ん だ	
\\	試験[しけん]の 前[まえ]に 英文[えいぶん]を
\\	したんだ。			
\\	機関	機関[きかん]	きかん	
\\	台風で交通機関がストップしている。	台風[たいふう]で 交通[こうつう] 機関[きかん]がストップしている。	たいふう で こうつう きかん が すとっぷ して いる	
\\	台風[たいふう]で 交通[こうつう]
\\	がストップしている。			
\\	実態	実態[じったい]	じったい	
\\	その会社の経営の実態を調査中だ。	その 会社[かいしゃ]の 経営[けいえい]の 実態[じったい]を 調査中[ちょうさちゅう]だ。	その かいしゃ の けいえい の じったい を ちょうさちゅう だ	
\\	その 会社[かいしゃ]の 経営[けいえい]の
\\	を 調査中[ちょうさちゅう]だ。			
\\	政治家	政治家[せいじか]	せいじか	
\\	大きくなったら政治家になりたいです。	大[おお]きくなったら 政治家[せいじか]になりたいです。	おおきく なったら せいじか に なりたい です	
\\	大[おお]きくなったら
\\	になりたいです。			
\\	治まる	治[おさ]まる	おさまる	
\\	咳が少し治まりました。	咳[せき]が 少[すこ]し 治[おさ]まりました。	せき が すこし おさまりました	
\\	咳[せき]が 少[すこ]し
\\	政党	政党[せいとう]	せいとう	
\\	選挙では3つの政党が争っています。	選挙[せんきょ]では 3[みっ]つの 政党[せいとう]が 争[あらそ]っています。	せんきょ で は みっつ の せいとう が あらそって います	
\\	選挙[せんきょ]では 3[みっ]つの
\\	が 争[あらそ]っています。			
\\	挙げる	挙[あ]げる	あげる	
\\	例を幾つか挙げてみましょう。	例[れい]を 幾[いく]つか 挙[あ]げてみましょう。	れい を いくつ か あげて みましょう	
\\	例[れい]を 幾[いく]つか
\\	気候	気候[きこう]	きこう	
\\	日本の気候は温暖です。	日本[にほん]の 気候[きこう]は 温暖[おんだん]です。	にほん の きこう は おんだん です	
\\	日本[にほん]の
\\	は 温暖[おんだん]です。			
\\	朝顔	朝顔[あさがお]	あさがお	
\\	紫の朝顔が咲きました。	紫[むらさき]の 朝顔[あさがお]が 咲[さ]きました。	むらさき の あさがお が さきました	
\\	紫[むらさき]の
\\	が 咲[さ]きました。			
\\	改正	改正[かいせい]	かいせい	
\\	近く交通法が改正されます。	近[ちか]く 交通法[こうつう ほう]が 改正[かいせい]されます。	ちかく こうつう ほう が かいせい されます	
\\	近[ちか]く 交通法[こうつう ほう]が
\\	されます。			
\\	改良	改良[かいりょう]	かいりょう	
\\	日本では絶えず米の品種を改良しているの。	日本[にほん]では 絶[た]えず 米[こめ]の 品種[ひんしゅ]を 改良[かいりょう]しているの。	にほん で は たえず こめ の ひんしゅ を かいりょう して いる の	
\\	日本[にほん]では 絶[た]えず 米[こめ]の 品種[ひんしゅ]を
\\	しているの。			
\\	改める	改[あらた]める	あらためる	
\\	彼は悪い習慣を改めようとしているわね。	彼[かれ]は 悪[わる]い 習慣[しゅうかん]を 改[あらた]めようとしているわね。	かれ は わるい しゅうかん を あらためよう と して いる わ ね	
\\	彼[かれ]は 悪[わる]い 習慣[しゅうかん]を
\\	としているわね。			
\\	改造	改造[かいぞう]	かいぞう	
\\	首相は内閣の改造を行いました。	首相[しゅしょう]は 内閣[ないかく]の 改造[かいぞう]を 行[おこな]いました。	しゅしょう は ないかく の かいぞう を おこないました	
\\	首相[しゅしょう]は 内閣[ないかく]の
\\	を 行[おこな]いました。			
\\	改めて	改[あらた]めて	あらためて	
\\	改めてあなたのご意見を聞かせて下さい。	改[あらた]めてあなたのご 意見[いけん]を 聞[き]かせて 下[くだ]さい。	あらためて あなた の ごいけん を きかせ て ください	
\\	あなたのご 意見[いけん]を 聞[き]かせて 下[くだ]さい。			
\\	改まる	改[あらた]まる	あらたまる	
\\	年号が改まりました。	年号[ねんごう]が 改[あらた]まりました。	ねんごう が あらたまりました	
\\	年号[ねんごう]が
\\	命	命[いのち]	いのち	
\\	命より大切なものは無いよ。	命[いのち]より 大切[たいせつ]なものは 無[な]いよ。	いのち より たいせつ な もの は ない よ	
\\	より 大切[たいせつ]なものは 無[な]いよ。			
\\	推進	推進[すいしん]	すいしん	
\\	その会社はリサイクルを推進しているね。	その 会社[かいしゃ]はリサイクルを 推進[すいしん]しているね。	その かいしゃ は りさいくる を すいしん して いる ね	
\\	その 会社[かいしゃ]はリサイクルを
\\	しているね。			
\\	委員会	委員会[いいんかい]	いいんかい	
\\	明日、委員会が開かれます。	明日[あした]、 委員会[いいんかい]が 開[ひら]かれます。	あした いいんかい が ひらかれます	
\\	明日[あした]、
\\	が 開[ひら]かれます。			
\\	委員	委員[いいん]	いいん	
\\	彼は委員に選ばれました。	彼[かれ]は 委員[いいん]に 選[えら]ばれました。	かれ は いいん に えらばれました	
\\	彼[かれ]は
\\	に 選[えら]ばれました。			
\\	従う	従[したが]う	したがう	
\\	上司の指示に従った。	上司[じょうし]の 指示[しじ]に 従[したが]った。	じょうし の しじ に したがった	
\\	上司[じょうし]の 指示[しじ]に
\\	従業員	従業員[じゅうぎょういん]	じゅうぎょういん	
\\	会社は従業員の数を増やす予定だ。	会社[かいしゃ]は 従業員[じゅうぎょういん]の 数[かず]を 増[ふ]やす 予定[よてい]だ。	かいしゃ は じゅうぎょういん の かず を ふやす よてい だ	
\\	会社[かいしゃ]は
\\	の 数[かず]を 増[ふ]やす 予定[よてい]だ。			
\\	実績	実績[じっせき]	じっせき	
\\	彼は営業で実績を上げたんだ。	彼[かれ]は 営業[えいぎょう]で 実績[じっせき]を 上[あ]げたんだ。	かれ は えいぎょう で じっせき を あげた ん だ	
\\	彼[かれ]は 営業[えいぎょう]で
\\	を 上[あ]げたんだ。			
\\	業績	業績[ぎょうせき]	ぎょうせき	
\\	彼の今月の業績は素晴らしいです。	彼[かれ]の 今月[こんげつ]の 業績[ぎょうせき]は 素晴[すば]らしいです。	かれ の こんげつ の ぎょうせき は すばらしい です	
\\	彼[かれ]の 今月[こんげつ]の
\\	は 素晴[すば]らしいです。			
\\	応募	応募[おうぼ]	おうぼ	
\\	求人に多数の応募があった。	求人[きゅうじん]に 多数[たすう]の 応募[おうぼ]があった。	きゅうじん に たすう の おうぼ が あった	
\\	求人[きゅうじん]に 多数[たすう]の
\\	があった。			
\\	採算	採算[さいさん]	さいさん	
\\	コストがこんなに高くては採算が取れません。	コストがこんなに 高[たか]くては 採算[さいさん]が 取[と]れません。	こすと が こんなに たかくて は さいさん が とれません	
\\	コストがこんなに 高[たか]くては
\\	が 取[と]れません。			
\\	採点	採点[さいてん]	さいてん	
\\	先生は試験の採点が終わったようね。	先生[せんせい]は 試験[しけん]の 採点[さいてん]が 終[お]わったようね。	せんせい は しけん の さいてん が おわった よう ね	
\\	先生[せんせい]は 試験[しけん]の
\\	が 終[お]わったようね。			
\\	月給	月給[げっきゅう]	げっきゅう	
\\	月給は毎月25日に支給されます。	月給[げっきゅう]は 毎月25日[まいつき にじゅうごにち]に 支給[しきゅう]されます。	げっきゅう は まいつき にじゅうごにち に しきゅう されます	
\\	は 毎月25日[まいつき にじゅうごにち]に 支給[しきゅう]されます。			
\\	時給	時給[じきゅう]	じきゅう	
\\	この仕事は時給1000円です。	この 仕事[しごと]は 時給[じきゅう] 1000円[せんえん]です。	この しごと は じきゅう せんえん です	
\\	この 仕事[しごと]は
\\	1000円[せんえん]です。			
\\	就任	就任[しゅうにん]	しゅうにん	
\\	彼は新首相に就任しましたね。	彼[かれ]は 新首相[しんしゅしょう]に 就任[しゅうにん]しましたね。	かれ は しんしゅしょう に しゅうにん しました ね	
\\	彼[かれ]は 新首相[しんしゅしょう]に
\\	しましたね。			
\\	条約	条約[じょうやく]	じょうやく	
\\	2国間で条約が結ばれました。	2国間[にこくかん]で 条約[じょうやく]が 結[むす]ばれました。	にこくかん で じょうやく が むすばれました	
\\	2国間[にこくかん]で
\\	が 結[むす]ばれました。			
\\	感じ	感[かん]じ	かんじ	
\\	あの子は感じの良い子です。	あの 子[こ]は 感[かん]じの 良[い]い 子[こ]です。	あの こ は かんじ の いい こ です	
\\	あの 子[こ]は
\\	の 良[い]い 子[こ]です。			
\\	感情	感情[かんじょう]	かんじょう	
\\	感情とは複雑なものです。	感情[かんじょう]とは 複雑[ふくざつ]なものです。	かんじょう と は ふくざつ な もの です	
\\	とは 複雑[ふくざつ]なものです。			
\\	感覚	感覚[かんかく]	かんかく	
\\	冷えて指の感覚がない。	冷[ひ]えて 指[ゆび]の 感覚[かんかく]がない。	ひえて ゆび の かんかく が ない	
\\	冷[ひ]えて 指[ゆび]の
\\	がない。			
\\	感動	感動[かんどう]	かんどう	
\\	感動する映画でした。	感動[かんどう]する 映画[えいが]でした。	かんどう する えいが でした	
\\	する 映画[えいが]でした。			
\\	実感	実感[じっかん]	じっかん	
\\	子供が歩き始めたとき、子供の成長を実感した。	子供[こども]が 歩[ある]き 始[はじ]めたとき、 子供[こども]の 成長[せいちょう]を 実感[じっかん]した。	こども が あるきはじめた とき こども の せいちょう を じっかん した	
\\	子供[こども]が 歩[ある]き 始[はじ]めたとき、 子供[こども]の 成長[せいちょう]を
\\	した。			
\\	感心	感心[かんしん]	かんしん	
\\	彼の我慢強さには感心しました。	彼[かれ]の 我慢強[がまんづよ]さには 感心[かんしん]しました。	かれ の がまんづよさ に は かんしん しました	
\\	彼[かれ]の 我慢強[がまんづよ]さには
\\	しました。			
\\	思想	思想[しそう]	しそう	
\\	人には思想の自由がある。	人[ひと]には 思想[しそう]の 自由[じゆう]がある。	ひと に は しそう の じゆう が ある	
\\	人[ひと]には
\\	の 自由[じゆう]がある。			
\\	感想	感想[かんそう]	かんそう	
\\	ご感想をお聞かせ下さい。	ご 感想[かんそう]をお 聞[き]かせ 下[くだ]さい。	ごかんそう を おきかせ ください	
\\	ご
\\	をお 聞[き]かせ 下[くだ]さい。			
\\	気象	気象[きしょう]	きしょう	
\\	テレビで明日の気象情報を確認したよ。	テレビで 明日[あす]の 気象[きしょう] 情報[じょうほう]を 確認[かくにん]したよ。	てれび で あす の きしょう じょうほう を かくにん した よ	
\\	テレビで 明日[あす]の
\\	情報[じょうほう]を 確認[かくにん]したよ。			
\\	水害	水害[すいがい]	すいがい	
\\	水害でたくさんの人が家を失ったの。	水害[すいがい]でたくさんの 人[ひと]が 家[いえ]を 失[うしな]ったの。	すいがい で たくさん の ひと が いえ を うしなった の	
\\	でたくさんの 人[ひと]が 家[いえ]を 失[うしな]ったの。			
\\	害	害[がい]	がい	
\\	お酒の飲み過ぎは健康に害があります。	お 酒[さけ]の 飲[の]み 過[す]ぎは 健康[けんこう]に 害[がい]があります。	お さけ の のみすぎ は けんこう に がい が あります 。	
\\	お 酒[さけ]の 飲[の]み 過[す]ぎは 健康[けんこう]に
\\	があります。			
\\	救う	救[すく]う	すくう	
\\	彼女は通りがかりの人に救われたよ。	彼女[かのじょ]は 通[とお]りがかりの 人[ひと]に 救[すく]われたよ。	かのじょ は とおりがかり の ひと に すくわれた よ	
\\	彼女[かのじょ]は 通[とお]りがかりの 人[ひと]に
\\	よ。			
\\	救い	救[すく]い	すくい	
\\	娘の存在が私の救いでした。	娘[むすめ]の 存在[そんざい]が 私[わたし]の 救[すく]いでした。	むすめ の そんざい が わたし の すくい でした	
\\	娘[むすめ]の 存在[そんざい]が 私[わたし]の
\\	でした。			
\\	救助	救助[きゅうじょ]	きゅうじょ	
\\	プールで男の子が救助されました。	プールで 男[おとこ]の 子[こ]が 救助[きゅうじょ]されました。	ぷーる で おとこ の こ が きゅうじょ されました	
\\	プールで 男[おとこ]の 子[こ]が
\\	されました。			
\\	支援	支援[しえん]	しえん	
\\	彼の支援がなかったらどうなっていたか。	彼[かれ]の 支援[しえん]がなかったらどうなっていたか。	かれ の しえん が なかったら どう なって いた か	
\\	彼[かれ]の
\\	がなかったらどうなっていたか。			
\\	援助	援助[えんじょ]	えんじょ	
\\	その国には物資の援助が必要です。	その 国[くに]には 物資[ぶっし]の 援助[えんじょ]が 必要[ひつよう]です。	その くに に は ぶっし の えんじょ が ひつよう です	
\\	その 国[くに]には 物資[ぶっし]の
\\	が 必要[ひつよう]です。			
\\	応援	応援[おうえん]	おうえん	
\\	大勢が応援に駆けつけてくれたよ。	大勢[おおぜい]が 応援[おうえん]に 駆[か]けつけてくれたよ。	おおぜい が おうえん に かけつけて くれた よ	
\\	大勢[おおぜい]が
\\	に 駆[か]けつけてくれたよ。			
\\	小遣い	小遣[こづか]い	こづかい	
\\	おじいちゃんにお小遣いをもらったよ。	おじいちゃんにお 小遣[こづか]いをもらったよ。	おじいちゃん に おこづかい を もらった よ	
\\	おじいちゃんにお
\\	をもらったよ。			
\\	強盗	強盗[ごうとう]	ごうとう	
\\	強盗がカメラに写っていました。	強盗[ごうとう]がカメラに 写[うつ]っていました。	ごうとう が かめら に うつって いました	
\\	がカメラに 写[うつ]っていました。			
\\	殺人	殺人[さつじん]	さつじん	
\\	その殺人事件は白昼に起こったんだ。	その 殺人[さつじん] 事件[じけん]は 白昼[はくちゅう]に 起[お]こったんだ。	その さつじん じけん は はくちゅう に おこった ん だ	
\\	その
\\	事件[じけん]は 白昼[はくちゅう]に 起[お]こったんだ。			
\\	奪う	奪[うば]う	うばう	
\\	その男は彼女のバッグを奪ったぞ。	その 男[おとこ]は 彼女[かのじょ]のバッグを 奪[うば]ったぞ。	その おとこ は かのじょ の ばっぐ を うばった ぞ	
\\	その 男[おとこ]は 彼女[かのじょ]のバッグを
\\	ぞ。			
\\	戦後	戦後[せんご]	せんご	
\\	戦後の日本は混乱していました。	戦後[せんご]の 日本[にほん]は 混乱[こんらん]していました。	せんご の にほん は こんらん して いました	
\\	の 日本[にほん]は 混乱[こんらん]していました。			
\\	戦場	戦場[せんじょう]	せんじょう	
\\	祖父は戦場に行ったことがあるそうです。	祖父[そふ]は 戦場[せんじょう]に 行[い]ったことがあるそうです。	そふ は せんじょう に いった こと が ある そう です	
\\	祖父[そふ]は
\\	に 行[い]ったことがあるそうです。			
\\	戦前	戦前[せんぜん]	せんぜん	
\\	戦前の生活は今と全く違いました。	戦前[せんぜん]の 生活[せいかつ]は 今[いま]と 全[まった]く 違[ちが]いました。	せんぜん の せいかつ は いま と まったく ちがいました	
\\	の 生活[せいかつ]は 今[いま]と 全[まった]く 違[ちが]いました。			
\\	戦死	戦死[せんし]	せんし	
\\	祖父は戦死しました。	祖父[そふ]は 戦死[せんし]しました。	そふ は せんし しました	
\\	祖父[そふ]は
\\	しました。			
\\	捜す	捜[さが]す	さがす	
\\	警察がその男を捜しているの。	警察[けいさつ]がその 男[おとこ]を 捜[さが]しているの。	けいさつ が その おとこ を さがして いる の	
\\	警察[けいさつ]がその 男[おとこ]を
\\	の。			
\\	海流	海流[かいりゅう]	かいりゅう	
\\	ここで2つの海流が出合っている。	ここで 2[ふた]つの 海流[かいりゅう]が 出合[であ]っている。	ここ で ふたつ の かいりゅう が であって いる	
\\	ここで 2[ふた]つの
\\	が 出合[であ]っている。			
\\	洪水	洪水[こうずい]	こうずい	
\\	洪水でたくさんの家が流されたの。	洪水[こうずい]でたくさんの 家[いえ]が 流[なが]されたの。	こうずい で たくさん の いえ が ながされた の	
\\	でたくさんの 家[いえ]が 流[なが]されたの。			
\\	崩れる	崩[くず]れる	くずれる	
\\	大雨で崖が崩れたね。	大雨[おおあめ]で 崖[がけ]が 崩[くず]れたね。	おおあめ で がけ が くずれた ね	
\\	大雨[おおあめ]で 崖[がけ]が
\\	ね。			
\\	崩す	崩[くず]す	くずす	
\\	彼女は体調を崩しています。	彼女[かのじょ]は 体調[たいちょう]を 崩[くず]しています。	かのじょ は たいちょう を くずして います	
\\	彼女[かのじょ]は 体調[たいちょう]を
\\	水洗	水洗[すいせん]	すいせん	
\\	今はほとんどのトイレが水洗ですよ。	今[いま]はほとんどのトイレが 水洗[すいせん]ですよ。	いま は ほとんど の といれ が すいせん です よ	
\\	今[いま]はほとんどのトイレが
\\	ですよ。			
\\	洗い物	洗[あら]い 物[もの]	あらいもの	
\\	母は台所で洗い物をしています。	母[はは]は 台所[だいどころ]で 洗[あら]い 物[もの]をしています。	はは は だいどころ で あらいもの を して います	
\\	母[はは]は 台所[だいどころ]で
\\	をしています。			
\\	油絵	油絵[あぶらえ]	あぶらえ	
\\	趣味で油絵を描いています。	趣味[しゅみ]で 油絵[あぶらえ]を 描[か]いています。	しゅみ で あぶらえ を かいています	
\\	趣味[しゅみ]で
\\	を 描[か]いています。			
\\	浮かぶ	浮[う]かぶ	うかぶ	
\\	沖にボートが浮かんでいます。	沖[おき]にボートが 浮[う]かんでいます。	おき に ぼーと が うかんで います	
\\	沖[おき]にボートが
\\	浮かべる	浮[う]かべる	うかべる	
\\	お風呂に花を浮かべて入ったの。	お 風呂[ふろ]に 花[はな]を 浮[う]かべて 入[はい]ったの。	おふろ に はな を うかべて はいった の	
\\	お 風呂[ふろ]に 花[はな]を
\\	入[はい]ったの。			
\\	浮く	浮[う]く	うく	
\\	氷は水に浮きます。	氷[こおり]は 水[みず]に 浮[う]きます。	こおり は みず に うきます	
\\	氷[こおり]は 水[みず]に
\\	沈める	沈[しず]める	しずめる	
\\	彼女はソファーに体を沈めたんだ。	彼女[かのじょ]はソファーに 体[からだ]を 沈[しず]めたんだ。	かのじょ は そふぁー に からだ を しずめた ん だ	
\\	彼女[かのじょ]はソファーに 体[からだ]を
\\	んだ。			
\\	将来	将来[しょうらい]	しょうらい	
\\	将来はパイロットになりたいです。	将来[しょうらい]はパイロットになりたいです。	しょうらい は ぱいろっと に なりたい です 。	
\\	はパイロットになりたいです。			
\\	永遠	永遠[えいえん]	えいえん	
\\	平和は人類の永遠のテーマです。	平和[へいわ]は 人類[じんるい]の 永遠[えいえん]のテーマです。	へいわ は じんるい の えいえん の てーま です	
\\	平和[へいわ]は 人類[じんるい]の
\\	のテーマです。			
\\	永久	永久[えいきゅう]	えいきゅう	
\\	彼は永久に帰らぬ人となったのよ。	彼[かれ]は 永久[えいきゅう]に 帰[かえ]らぬ 人[ひと]となったのよ。	かれ は えいきゅう に かえらぬ ひと と なった の よ	
\\	彼[かれ]は
\\	に 帰[かえ]らぬ 人[ひと]となったのよ。			
\\	河口	河口[かこう]	かこう	
\\	この川の河口は太平洋に注いでいます。	この 川[かわ]の 河口[かこう]は 太平洋[たいへいよう]に 注[そそ]いでいます。	この かわ の かこう は たいへいよう に そそいで います	
\\	この 川[かわ]の
\\	は 太平洋[たいへいよう]に 注[そそ]いでいます。			
\\	心臓	心臓[しんぞう]	しんぞう	
\\	祖母は心臓が悪いんだ。	祖母[そぼ]は 心臓[しんぞう]が 悪[わる]いんだ。	そぼ は しんぞう が わるい ん だ	
\\	祖母[そぼ]は
\\	が 悪[わる]いんだ。			
\\	快い	快[こころよ]い	こころよい	
\\	彼女は快い眠りについています。	彼女[かのじょ]は 快[こころよ]い 眠[ねむ]りについています。	かのじょ は こころよい ねむり に ついて います	
\\	彼女[かのじょ]は
\\	眠[ねむ]りについています。			
\\	快晴	快晴[かいせい]	かいせい	
\\	今日は快晴ですね。	今日[きょう]は 快晴[かいせい]ですね。	きょう は かいせい です ね	
\\	今日[きょう]は
\\	ですね。			
\\	最適	最適[さいてき]	さいてき	
\\	ここは子育てに最適な環境です。	ここは 子育[こそだ]てに 最適[さいてき]な 環境[かんきょう]です。	ここ は こそだて に さいてき な かんきょう です	
\\	ここは 子育[こそだ]てに
\\	な 環境[かんきょう]です。			
\\	指摘	指摘[してき]	してき	
\\	ご指摘いただきありがとうございます。	ご 指摘[してき]いただきありがとうございます。	ごしてき いただき ありがとう ございます	
\\	ご
\\	いただきありがとうございます。			
\\	汚染	汚染[おせん]	おせん	
\\	その川の水は汚染されています。	その 川[かわ]の 水[みず]は 汚染[おせん]されています。	その かわ の みず は おせん されて います	
\\	その 川[かわ]の 水[みず]は
\\	されています。			
\\	景気	景気[けいき]	けいき	
\\	景気が回復してきたね。	景気[けいき]が 回復[かいふく]してきたね。	けいき が かいふく して きた ね	
\\	が 回復[かいふく]してきたね。			
\\	影	影[かげ]	かげ	
\\	窓に男性の影が映っています。	窓[まど]に 男性[だんせい]の 影[かげ]が 映[うつ]っています。	まど に だんせい の かげ が うつって います	
\\	窓[まど]に 男性[だんせい]の
\\	が 映[うつ]っています。			
\\	境界	境界[きょうかい]	きょうかい	
\\	ここは隣の市との境界です。	ここは 隣[となり]の 市[し]との 境界[きょうかい]です。	ここ は となり の し と の きょうかい です	
\\	ここは 隣[となり]の 市[し]との
\\	です。			
\\	国境	国境[こっきょう]	こっきょう	
\\	あの山のすぐ近くが国境です。	あの 山[やま]のすぐ 近[ちか]くが 国境[こっきょう]です。	あの やま の すぐ ちかく が こっきょう です	
\\	あの 山[やま]のすぐ 近[ちか]くが
\\	です。			
\\	境	境[さかい]	さかい	
\\	2つの市の境に川が流れているの。	2[ふた]つの 市[し]の 境[さかい]に 川[かわ]が 流[なが]れているの。	ふたつ の し の さかい に かわ が ながれて いる の	
\\	2[ふた]つの 市[し]の
\\	に 川[かわ]が 流[なが]れているの。			
\\	外観	外観[がいかん]	がいかん	
\\	そのモダンな外観の建物が大使館です。	そのモダンな 外観[がいかん]の 建物[たてもの]が 大使館[たいしかん]です。	その もだん な がいかん の たてもの が たいしかん です	
\\	そのモダンな
\\	の 建物[たてもの]が 大使館[たいしかん]です。			
\\	客観的	客観的[きゃっかんてき]	きゃっかんてき	
\\	彼は自分の状況を客観的に見てみたのね。	彼[かれ]は 自分[じぶん]の 状況[じょうきょう]を 客観的[きゃっかんてき]に 見[み]てみたのね。	かれ は じぶん の じょうきょう を きゃっかんてき に みて みた の ね	
\\	彼[かれ]は 自分[じぶん]の 状況[じょうきょう]を
\\	に 見[み]てみたのね。			
\\	推測	推測[すいそく]	すいそく	
\\	それは彼の推測にすぎない。	それは 彼[かれ]の 推測[すいそく]にすぎない。	それ は かれ の すいそく に すぎない	
\\	それは 彼[かれ]の
\\	にすぎない。			
\\	宇宙	宇宙[うちゅう]	うちゅう	
\\	宇宙の謎は限りなく大きいの。	宇宙[うちゅう]の 謎[なぞ]は 限[かぎ]りなく 大[おお]きいの。	うちゅう の なぞ は かぎり なく おおきい の	
\\	の 謎[なぞ]は 限[かぎ]りなく 大[おお]きいの。			
\\	振動	振動[しんどう]	しんどう	
\\	車の振動で棚の荷物が落ちた。	車[くるま]の 振動[しんどう]で 棚[たな]の 荷物[にもつ]が 落[お]ちた。	くるま の しんどう で たな の にもつ が おちた 。	
\\	車[くるま]の
\\	で 棚[たな]の 荷物[にもつ]が 落[お]ちた。			
\\	好奇心	好奇心[こうきしん]	こうきしん	
\\	子供は好奇心でいっぱいだね。	子供[こども]は 好奇心[こうきしん]でいっぱいだね。	こども は こうきしん で いっぱい だ ね	
\\	子供[こども]は
\\	でいっぱいだね。			
\\	奇跡	奇跡[きせき]	きせき	
\\	彼のマジックはまるで奇跡です。	彼[かれ]のマジックはまるで 奇跡[きせき]です。	かれ の まじっく は まるで きせき です	
\\	彼[かれ]のマジックはまるで
\\	です。			
\\	奇数	奇数[きすう]	きすう	
\\	3は奇数です。	3[さん]は 奇数[きすう]です。	さん は きすう です	
\\	3[さん]は
\\	です。			
\\	学歴	学歴[がくれき]	がくれき	
\\	その職は大卒の学歴が必要だ。	その 職[しょく]は 大卒[だいそつ]の 学歴[がくれき]が 必要[ひつよう]だ。	その しょく は だいそつ の がくれき が ひつよう だ	
\\	その 職[しょく]は 大卒[だいそつ]の
\\	が 必要[ひつよう]だ。			
\\	建築	建築[けんちく]	けんちく	
\\	彼らは家を建築中です。	彼[かれ]らは 家[いえ]を 建築[けんちく] 中[ちゅう]です。	かれら は いえ を けんちくちゅう です	
\\	彼[かれ]らは 家[いえ]を
\\	中[ちゅう]です。			
\\	新築	新築[しんちく]	しんちく	
\\	彼は去年、家を新築しました。	彼[かれ]は 去年[きょねん]、 家[いえ]を 新築[しんちく]しました。	かれ は きょねん いえ を しんちく しました	
\\	彼[かれ]は 去年[きょねん]、 家[いえ]を
\\	しました。			
\\	構想	構想[こうそう]	こうそう	
\\	彼は新しい小説の構想を練っているの。	彼[かれ]は 新[あたら]しい 小説[しょうせつ]の 構想[こうそう]を 練[ね]っているの。	かれ は あたらしい しょうせつ の こうそう を ねって いる の	
\\	彼[かれ]は 新[あたら]しい 小説[しょうせつ]の
\\	を 練[ね]っているの。			
\\	構える	構[かま]える	かまえる	
\\	彼はあの通りに店を構えているの。	彼[かれ]はあの 通[とお]りに 店[みせ]を 構[かま]えているの。	かれ は あの とおり に みせ を かまえて いる の	
\\	彼[かれ]はあの 通[とお]りに 店[みせ]を
\\	の。			
\\	構う	構[かま]う	かまう	
\\	子供に構い過ぎてはいけない。	子供[こども]に 構[かま]い 過[す]ぎてはいけない。	こども に かまいすぎて は いけない	
\\	子供[こども]に
\\	過[す]ぎてはいけない。			
\\	周囲	周囲[しゅうい]	しゅうい	
\\	大声で話すと周囲の人に迷惑ですよ。	大声[おおごえ]で 話[はな]すと 周囲[しゅうい]の 人[ひと]に 迷惑[めいわく]ですよ。	おおごえ で はなす と しゅうい の ひと に めいわく です よ	
\\	大声[おおごえ]で 話[はな]すと
\\	の 人[ひと]に 迷惑[めいわく]ですよ。			
\\	囲む	囲[かこ]む	かこむ	
\\	久しぶりに家族全員で食卓を囲みました。	久[ひさ]しぶりに 家族全員[かぞく ぜんいん]で 食卓[しょくたく]を 囲[かこ]みました。	ひさしぶり に かぞく ぜんいん で しょくたく を かこみました	
\\	久[ひさ]しぶりに 家族全員[かぞく ぜんいん]で 食卓[しょくたく]を
\\	横断	横断[おうだん]	おうだん	
\\	道路を横断するときは注意して。	道路[どうろ]を 横断[おうだん]するときは 注意[ちゅうい]して。	どうろ を おうだん する とき は ちゅうい して	
\\	道路[どうろ]を
\\	するときは 注意[ちゅうい]して。			
\\	欧米	欧米[おうべい]	おうべい	
\\	その会社は欧米に進出しているよね。	その 会社[かいしゃ]は 欧米[おうべい]に 進出[しんしゅつ]しているよね。	その かいしゃ は おうべい に しんしゅつ して いる よ ね	
\\	その 会社[かいしゃ]は
\\	に 進出[しんしゅつ]しているよね。			
\\	州	州[しゅう]	しゅう	
\\	来月、隣の州に引っ越します。	来月[らいげつ]、 隣[となり]の 州[しゅう]に 引[ひ]っ 越[こ]します。	らいげつ となり の しゅう に ひっこします	
\\	来月[らいげつ]、 隣[となり]の
\\	に 引[ひ]っ 越[こ]します。			
\\	極めて	極[きわ]めて	きわめて	
\\	これは極めて重要な問題です。	これは 極[きわ]めて 重要[じゅうよう]な 問題[もんだい]です。	これ は きわめて じゅうよう な もんだい です	
\\	これは
\\	重要[じゅうよう]な 問題[もんだい]です。			
\\	消極的	消極的[しょうきょくてき]	しょうきょくてき	
\\	消極的な人は成功しないよ。	消極的[しょうきょくてき]な 人[ひと]は 成功[せいこう]しないよ。	しょうきょくてき な ひと は せいこう しない よ	
\\	な 人[ひと]は 成功[せいこう]しないよ。			
\\	極端	極端[きょくたん]	きょくたん	
\\	それは極端な意見だね。	それは 極端[きょくたん]な 意見[いけん]だね。	それ は きょくたん な いけん だ ね	
\\	それは
\\	な 意見[いけん]だね。			
\\	最先端	最先端[さいせんたん]	さいせんたん	
\\	そのカメラには最先端の技術が使われています。	そのカメラには 最先端[さいせんたん]の 技術[ぎじゅつ]が 使[つか]われています。	その かめら に は さいせんたん の ぎじゅつ が つかわれて います	
\\	そのカメラには
\\	の 技術[ぎじゅつ]が 使[つか]われています。			
\\	外貨	外貨[がいか]	がいか	
\\	外貨を両替しました。	外貨[がいか]を 両替[りょうがえ]しました。	がいか を りょうがえ しました	
\\	を 両替[りょうがえ]しました。			
\\	回復	回復[かいふく]	かいふく	
\\	体がすっかり回復した。	体[からだ]がすっかり 回復[かいふく]した。	からだ が すっかり かいふく した	
\\	体[からだ]がすっかり
\\	した。			
\\	帯	帯[おび]	おび	
\\	この帯は長過ぎます。	この 帯[おび]は 長過[なが す]ぎます。	この おび は なが すぎます	
\\	この
\\	は 長過[なが す]ぎます。			
\\	守備	守備[しゅび]	しゅび	
\\	そのチームは守備が甘いね。	そのチームは 守備[しゅび]が 甘[あま]いね。	その ちーむ は しゅび が あまい ね	
\\	そのチームは
\\	が 甘[あま]いね。			
\\	帰宅	帰宅[きたく]	きたく	
\\	夜の11時に帰宅しました。	夜[よる]の 11時[じゅういちじ]に 帰宅[きたく]しました。	よる の じゅういちじ に きたく しました	
\\	夜[よる]の 11時[じゅういちじ]に
\\	しました。			
\\	宛先	宛先[あてさき]	あてさき	
\\	宛先不明で手紙が戻ってきたの。	宛先[あてさき] 不明[ふめい]で 手紙[てがみ]が 戻[もど]ってきたの。	あてさき ふめい で てがみ が もどって きた の	
\\	不明[ふめい]で 手紙[てがみ]が 戻[もど]ってきたの。			
\\	宛名	宛名[あてな]	あてな	
\\	手紙に宛名を書き込んだよ。	手紙[てがみ]に 宛名[あてな]を 書[か]き 込[こ]んだよ。	てがみ に あてな を かきこんだ よ	
\\	手紙[てがみ]に
\\	を 書[か]き 込[こ]んだよ。			
\\	後戻り	後戻[あともど]り	あともどり	
\\	彼女は途中で後戻りしました。	彼女[かのじょ]は 途中[とちゅう]で 後戻[あともど]りしました。	かのじょ は とちゅう で あともどり しました	
\\	彼女[かのじょ]は 途中[とちゅう]で
\\	しました。			
\\	寝室	寝室[しんしつ]	しんしつ	
\\	寝室の壁紙を張り替えました。	寝室[しんしつ]の 壁紙[かべがみ]を 張[は]り 替[か]えました。	しんしつ の かべがみ を はりかえました	
\\	の 壁紙[かべがみ]を 張[は]り 替[か]えました。			
\\	月刊	月刊[げっかん]	げっかん	
\\	この雑誌は月刊ですか。	この 雑誌[ざっし]は 月刊[げっかん]ですか。	この ざっし は げっかん です か	
\\	この 雑誌[ざっし]は
\\	ですか。			
\\	心細い	心細[こころぼそ]い	こころぼそい	
\\	夜道の一人歩きは心細いね。	夜道[よみち]の 一人歩[ひとりある]きは 心細[こころぼそ]いね。	よみち の ひとりあるき は こころぼそい ね	
\\	夜道[よみち]の 一人歩[ひとりある]きは
\\	ね。			
\\	掲示	掲示[けいじ]	けいじ	
\\	大会のスローガンを掲示したよ。	大会[たいかい]のスローガンを 掲示[けいじ]したよ。	たいかい の すろーがん を けいじ した よ	
\\	大会[たいかい]のスローガンを
\\	したよ。			
\\	気付く	気付[きづ]く	きづく	
\\	彼はやっと問題点に気付きました。	彼[かれ]はやっと 問題点[もんだいてん]に 気付[きづ]きました。	かれ は やっと もんだいてん に きづきました	
\\	彼[かれ]はやっと 問題点[もんだいてん]に
\\	思い付く	思[おも]い 付[つ]く	おもいつく	
\\	彼は名案を思い付いたの。	彼[かれ]は 名案[めいあん]を 思[おも]い 付[つ]いたの。	かれ は めいあん を おもいついた の	
\\	彼[かれ]は 名案[めいあん]を
\\	の。			
\\	後片付け	後片付[あとかたづ]け	あとかたづけ	
\\	食事の後片付けを手伝ったの。	食事[しょくじ]の 後片付[あとかたづ]けを 手伝[てつだ]ったの。	しょくじ の あとかたづけ を てつだった の	
\\	食事[しょくじ]の
\\	を 手伝[てつだ]ったの。			
\\	所属	所属[しょぞく]	しょぞく	
\\	学校では音楽部に所属していました。	学校[がっこう]では 音楽部[おんがくぶ]に 所属[しょぞく]していました。	がっこう で は おんがくぶ に しょぞく して いました	
\\	学校[がっこう]では 音楽部[おんがくぶ]に
\\	していました。			
\\	大蔵省	大蔵省[おおくらしょう]	おおくらしょう	
\\	彼は大蔵省に勤務しているんだよ。	彼[かれ]は 大蔵省[おおくらしょう]に 勤務[きんむ]しているんだよ。	かれ は おおくらしょう に きんむ して いる ん だ よ	
\\	彼[かれ]は
\\	に 勤務[きんむ]しているんだよ。			
\\	外務省	外務省[がいむしょう]	がいむしょう	
\\	ビザについて外務省に問い合わせた。	ビザについて 外務省[がいむしょう]に 問[と]い 合[あ]わせた。	びざ に ついて がいむしょう に といあわせた	
\\	ビザについて
\\	に 問[と]い 合[あ]わせた。			
\\	帰省	帰省[きせい]	きせい	
\\	来週、帰省します。	来週[らいしゅう]、 帰省[きせい]します。	らいしゅう きせい します	
\\	来週[らいしゅう]、
\\	します。			
\\	概念	概念[がいねん]	がいねん	
\\	インターネットは情報の概念を変えたよね。	インターネットは 情報[じょうほう]の 概念[がいねん]を 変[か]えたよね。	いんたーねっと は じょうほう の がいねん を かえた よ ね	
\\	インターネットは 情報[じょうほう]の
\\	を 変[か]えたよね。			
\\	整列	整列[せいれつ]	せいれつ	
\\	体育館に行って整列しなさい。	体育館[たいいくかん]に 行[い]って 整列[せいれつ]しなさい。	たいいくかん に いって せいれつ しなさい	
\\	体育館[たいいくかん]に 行[い]って
\\	しなさい。			
\\	実例	実例[じつれい]	じつれい	
\\	実例を使って説明してください。	実例[じつれい]を 使[つか]って 説明[せつめい]してください。	じつれい を つかって せつめい して ください	
\\	を 使[つか]って 説明[せつめい]してください。			
\\	既に	既[すで]に	すでに	
\\	そのことは既にみんな知っています。	そのことは 既[すで]にみんな 知[し]っています。	その こと は すでに みんな しって います	
\\	そのことは
\\	みんな 知[し]っています。			
\\	既製	既製[きせい]	きせい	
\\	彼の体型じゃ既製のサイズに合わないよ。	彼[かれ]の 体型[たいけい]じゃ 既製[きせい]のサイズに 合[あ]わないよ。	かれ の たいけい じゃ きせい の さいず に あわない よ	
\\	彼[かれ]の 体型[たいけい]じゃ
\\	のサイズに 合[あ]わないよ。			
\\	時刻	時刻[じこく]	じこく	
\\	ただ今の時刻は6時35分です。	ただ 今[いま]の 時刻[じこく]は 6時35分[ろくじ さんじゅうごふん]です。	ただいま の じこく は ろくじ さんじゅうごふん です	
\\	ただ 今[いま]の
\\	は 6時35分[ろくじ さんじゅうごふん]です。			
\\	栄える	栄[さか]える	さかえる	
\\	ここはかつてゴールドラッシュで栄えた町だよ。	ここはかつてゴールドラッシュで 栄[さか]えた 町[まち]だよ。	ここ は かつて ごーるど らっしゅ で さかえた まち だ よ	
\\	ここはかつてゴールドラッシュで
\\	町[まち]だよ。			
\\	栄養	栄養[えいよう]	えいよう	
\\	豆腐は栄養のある食べ物です。	豆腐[とうふ]は 栄養[えいよう]のある 食[た]べ 物[もの]です。	とうふ は えいよう の ある たべもの です	
\\	豆腐[とうふ]は
\\	のある 食[た]べ 物[もの]です。			
\\	教養	教養[きょうよう]	きょうよう	
\\	彼女はとても教養のある人ですね。	彼女[かのじょ]はとても 教養[きょうよう]のある 人[ひと]ですね。	かのじょ は とても きょうよう の ある ひと です ね	
\\	彼女[かのじょ]はとても
\\	のある 人[ひと]ですね。			
\\	困難	困難[こんなん]	こんなん	
\\	困難にあってもあきらめてはいけないよ。	困難[こんなん]にあってもあきらめてはいけないよ。	こんなん に あって も あきらめて は いけない よ	
\\	にあってもあきらめてはいけないよ。			
\\	幸い	幸[さいわ]い	さいわい	
\\	幸い、電車に嵐の影響はなかった。	幸[さいわ]い、 電車[でんしゃ]に 嵐[あらし]の 影響[えいきょう]はなかった。	さいわい でんしゃ に あらし の えいきょう は なかった	
\\	、 電車[でんしゃ]に 嵐[あらし]の 影響[えいきょう]はなかった。			
\\	幸運	幸運[こううん]	こううん	
\\	幸運にもチケットを手に入れました。	幸運[こううん]にもチケットを 手[て]に 入[い]れました。	こううん に も ちけっと を て に いれました	
\\	にもチケットを 手[て]に 入[い]れました。			
\\	幸福	幸福[こうふく]	こうふく	
\\	彼女は幸福な日々を過ごしているわ。	彼女[かのじょ]は 幸福[こうふく]な 日々[ひび]を 過[す]ごしているわ。	かのじょ は こうふく な ひび を すごして いる わ	
\\	彼女[かのじょ]は
\\	な 日々[ひび]を 過[す]ごしているわ。			
\\	圧力	圧力[あつりょく]	あつりょく	
\\	相手会社から強い圧力が掛かった。	相手会社[あいてがいしゃ]から 強[つよ]い 圧力[あつりょく]が 掛[か]かった。	あいてがいしゃ から つよい あつりょく が かかった	
\\	相手会社[あいてがいしゃ]から 強[つよ]い
\\	が 掛[か]かった。			
\\	気圧	気圧[きあつ]	きあつ	
\\	高い山は気圧が低いね。	高[たか]い 山[やま]は 気圧[きあつ]が 低[ひく]いね。	たかい やま は きあつ が ひくい ね	
\\	高[たか]い 山[やま]は
\\	が 低[ひく]いね。			
\\	札	札[さつ]	さつ	
\\	彼はカバンから札の束を取り出したんだ。	彼[かれ]はカバンから 札[さつ]の 束[たば]を 取[と]り 出[だ]したんだ。	かれ は かばん から さつ の たば を とりだした ん だ 。	
\\	彼[かれ]はカバンから
\\	の 束[たば]を 取[と]り 出[だ]したんだ。			
\\	改札	改札[かいさつ]	かいさつ	
\\	改札で3時に会おう。	改札[かいさつ]で 3時[さんじ]に 会[あ]おう。	かいさつ で さんじ に あおう	
\\	で 3時[さんじ]に 会[あ]おう。			
\\	感謝	感謝[かんしゃ]	かんしゃ	
\\	家族に感謝しています。	家族[かぞく]に 感謝[かんしゃ]しています。	かぞく に かんしゃ して います	
\\	家族[かぞく]に
\\	しています。			
\\	月謝	月謝[げっしゃ]	げっしゃ	
\\	先生に月謝を渡しましたか。	先生[せんせい]に 月謝[げっしゃ]を 渡[わた]しましたか。	せんせい に げっしゃ を わたしました か	
\\	先生[せんせい]に
\\	を 渡[わた]しましたか。			
\\	射す	射[さ]す	さす	
\\	今日は久しぶりに日が射してるね。	今日[きょう]は 久[ひさ]しぶりに 日[ひ]が 射[さ]してるね。	きょう は ひさしぶり に ひ が さして る ね	
\\	今日[きょう]は 久[ひさ]しぶりに 日[ひ]が
\\	ね。			
\\	女優	女優[じょゆう]	じょゆう	
\\	彼女はずっと女優になるのが夢でした。	彼女[かのじょ]はずっと 女優[じょゆう]になるのが 夢[ゆめ]でした。	かのじょ は ずっと じょゆう に なる の が ゆめ でした	
\\	彼女[かのじょ]はずっと
\\	になるのが 夢[ゆめ]でした。			
\\	指導	指導[しどう]	しどう	
\\	彼は生徒の指導が上手ね。	彼[かれ]は 生徒[せいと]の 指導[しどう]が 上手[じょうず]ね。	かれ は せいと の しどう が じょうず ね	
\\	彼[かれ]は 生徒[せいと]の
\\	が 上手[じょうず]ね。			
\\	希望	希望[きぼう]	きぼう	
\\	彼は本社で働くことを希望しています。	彼[かれ]は 本社[ほんしゃ]で 働[はたら]くことを 希望[きぼう]しています。	かれ は ほんしゃ で はたらく こと を きぼう して います	
\\	彼[かれ]は 本社[ほんしゃ]で 働[はたら]くことを
\\	しています。			
\\	失望	失望[しつぼう]	しつぼう	
\\	彼女は結婚生活に失望していたの。	彼女[かのじょ]は 結婚生活[けっこん せいかつ]に 失望[しつぼう]していたの。	かのじょ は けっこん せいかつ に しつぼう して いた の	
\\	彼女[かのじょ]は 結婚生活[けっこん せいかつ]に
\\	していたの。			
\\	意志	意志[いし]	いし	
\\	彼は意志の強い人です。	彼[かれ]は 意志[いし]の 強[つよ]い 人[ひと]です。	かれ は いし の つよい ひと です	
\\	彼[かれ]は
\\	の 強[つよ]い 人[ひと]です。			
\\	志す	志[こころざ]す	こころざす	
\\	私は医者を志しています。	私[わたし]は 医者[いしゃ]を 志[こころざ]しています。	わたし は いしゃ を こころざして います	
\\	私[わたし]は 医者[いしゃ]を
\\	怒り	怒[いか]り	いかり	
\\	彼ったら怒り爆発だったよ。	彼[かれ]ったら 怒[いか]り 爆発[ばくはつ]だったよ。	かれ ったら いかり ばくはつ だった よ	
\\	彼[かれ]ったら
\\	爆発[ばくはつ]だったよ。			
\\	心身	心身[しんしん]	しんしん	
\\	私は心身共に疲れていました。	私[わたし]は 心身[しんしん] 共[とも]に 疲[つか]れていました。	わたし は しんしん ともに つかれて いました	
\\	私[わたし]は
\\	共[とも]に 疲[つか]れていました。			
\\	受け身	受[う]け 身[み]	うけみ	
\\	彼はいつも受け身の姿勢で、自分からは何もしないんだ。	彼[かれ]はいつも 受[う]け 身[み]の 姿勢[しせい]で、 自分[じぶん]からは 何[なに]もしないんだ。	かれ は いつも うけみ の しせい で じぶん からは なに も しない ん だ	
\\	彼[かれ]はいつも
\\	の 姿勢[しせい]で、 自分[じぶん]からは 何[なに]もしないんだ。			
\\	工夫	工夫[くふう]	くふう	
\\	いろいろ工夫して仕事をやりとげたさ。	いろいろ 工夫[くふう]して 仕事[しごと]をやりとげたさ。	いろいろ くふう して しごと を やりとげた さ	
\\	いろいろ
\\	して 仕事[しごと]をやりとげたさ。			
\\	奥様	奥様[おくさま]	おくさま	
\\	社長の奥様はきれいな方です。	社長[しゃちょう]の 奥様[おくさま]はきれいな 方[かた]です。	しゃちょう の おくさま は きれい な かた です	
\\	社長[しゃちょう]の
\\	はきれいな 方[かた]です。			
\\	愛情	愛情[あいじょう]	あいじょう	
\\	子供はたくさんの愛情が必要です。	子供[こども]はたくさんの 愛情[あいじょう]が 必要[ひつよう]です。	こども は たくさん の あいじょう が ひつよう です	
\\	子供[こども]はたくさんの
\\	が 必要[ひつよう]です。			
\\	可愛らしい	可愛[かわい]らしい	かわいらしい	
\\	彼女は娘に可愛らしい服を作りましたね。	彼女[かのじょ]は 娘[むすめ]に 可愛[かわい]らしい 服[ふく]を 作[つく]りましたね。	かのじょ は むすめ に かわいらしい ふく を つくりました ね	
\\	彼女[かのじょ]は 娘[むすめ]に
\\	服[ふく]を 作[つく]りましたね。			
\\	可愛がる	可愛[かわい]がる	かわいがる	
\\	彼女は猫を可愛がっています。	彼女[かのじょ]は 猫[ねこ]を 可愛[かわい]がっています。	かのじょ は ねこ を かわいがって います	
\\	彼女[かのじょ]は 猫[ねこ]を
\\	恋	恋[こい]	こい	
\\	彼女は恋をしてきれいになったね。	彼女[かのじょ]は 恋[こい]をしてきれいになったね。	かのじょ は こい を して きれい に なった ね	
\\	彼女[かのじょ]は
\\	をしてきれいになったね。			
\\	失恋	失恋[しつれん]	しつれん	
\\	彼は最近、失恋したらしいの。	彼[かれ]は 最近[さいきん]、 失恋[しつれん]したらしいの。	かれ は さいきん しつれん した らしい の	
\\	彼[かれ]は 最近[さいきん]、
\\	したらしいの。			
\\	恋する	恋[こい]する	こいする	
\\	恋する気持ちを歌にしました。	恋[こい]する 気持[きも]ちを 歌[うた]にしました。	こいする きもち を うた に しました	
\\	気持[きも]ちを 歌[うた]にしました。			
\\	延長	延長[えんちょう]	えんちょう	
\\	国会の会期が延長されたわね。	国会[こっかい]の 会期[かいき]が 延長[えんちょう]されたわね。	こっかい の かいき が えんちょう された わ ね	
\\	国会[こっかい]の 会期[かいき]が
\\	されたわね。			
\\	延期	延期[えんき]	えんき	
\\	運動会は雨で延期されました。	運動会[うんどうかい]は 雨[あめ]で 延期[えんき]されました。	うんどうかい は あめ で えんき されました	
\\	運動会[うんどうかい]は 雨[あめ]で
\\	されました。			
\\	慎重	慎重[しんちょう]	しんちょう	
\\	もう一度慎重に見直しましょう。	もう 一度[いちど] 慎重[しんちょう]に 見直[みなお]しましょう。	もういちど しんちょう に みなおしましょう	
\\	もう 一度[いちど]
\\	に 見直[みなお]しましょう。			
\\	大喜び	大喜[おおよろこ]び	おおよろこび	
\\	弟は新しい自転車に大喜びです。	弟[おとうと]は 新[あたら]しい 自転車[じてんしゃ]に 大喜[おおよろこ]びです。	おとうと は あたらしい じてんしゃ に おおよろこび です	
\\	弟[おとうと]は 新[あたら]しい 自転車[じてんしゃ]に
\\	です。			
\\	権利	権利[けんり]	けんり	
\\	私たちには知る権利があります。	私[わたし]たちには 知[し]る 権利[けんり]があります。	わたしたち に は しる けんり が あります	
\\	私[わたし]たちには 知[し]る
\\	があります。			
\\	権力	権力[けんりょく]	けんりょく	
\\	彼はこの国で大きな権力を持っているわ。	彼[かれ]はこの 国[くに]で 大[おお]きな 権力[けんりょく]を 持[も]っているわ。	かれ は この くに で おおき な けんりょく を もって いる わ	
\\	彼[かれ]はこの 国[くに]で 大[おお]きな
\\	を 持[も]っているわ。			
\\	棄権	棄権[きけん]	きけん	
\\	彼は試合の途中で棄権したぞ。	彼[かれ]は 試合[しあい]の 途中[とちゅう]で 棄権[きけん]したぞ。	かれ は しあい の とちゅう で きけん した ぞ	
\\	彼[かれ]は 試合[しあい]の 途中[とちゅう]で
\\	したぞ。			
\\	完了	完了[かんりょう]	かんりょう	
\\	仕事は全て完了しました。	仕事[しごと]は 全[すべ]て 完了[かんりょう]しました。	しごと は すべて かんりょう しました	
\\	仕事[しごと]は 全[すべ]て
\\	しました。			
\\	承認	承認[しょうにん]	しょうにん	
\\	これは政府の承認を受けた資格です。	これは 政府[せいふ]の 承認[しょうにん]を 受[う]けた 資格[しかく]です。	これ は せいふ の しょうにん を うけた しかく です	
\\	これは 政府[せいふ]の
\\	を 受[う]けた 資格[しかく]です。			
\\	承知	承知[しょうち]	しょうち	
\\	そのことは承知しております。	そのことは 承知[しょうち]しております。	その こと は しょうち して おります	
\\	そのことは
\\	しております。			
\\	所得	所得[しょとく]	しょとく	
\\	ここに去年の所得をご記入ください。	ここに 去年[きょねん]の 所得[しょとく]をご 記入[きにゅう]ください。	ここ に きょねん の しょとく を ご きにゅう ください	
\\	ここに 去年[きょねん]の
\\	をご 記入[きにゅう]ください。			
\\	得る	得[え]る	える	
\\	彼は大金を得ましたよ。	彼[かれ]は 大金[たいきん]を 得[え]ましたよ。	かれ は たいきん を えました よ	
\\	彼[かれ]は 大金[たいきん]を
\\	よ。			
\\	幹部	幹部[かんぶ]	かんぶ	
\\	あの会社の幹部は皆とても優秀だね。	あの 会社[かいしゃ]の 幹部[かんぶ]は 皆[みんな]とても 優秀[ゆうしゅう]だね。	あの かいしゃ の かんぶ は みんな とても ゆうしゅう だ ね	
\\	あの 会社[かいしゃ]の
\\	は 皆[みんな]とても 優秀[ゆうしゅう]だね。			
\\	水素	水素[すいそ]	すいそ	
\\	水は水素と酸素でできています。	水[みず]は 水素[すいそ]と 酸素[さんそ]でできています。	みず は すいそ と さんそ で できて います	
\\	水[みず]は
\\	と 酸素[さんそ]でできています。			
\\	岩	岩[いわ]	いわ	
\\	あの岩まで泳ごう。	あの 岩[いわ]まで 泳[およ]ごう。	あの いわ まで およごう	
\\	あの
\\	まで 泳[およ]ごう。			
\\	岸	岸[きし]	きし	
\\	船がやっと岸に着いたよ。	船[ふね]がやっと 岸[きし]に 着[つ]いたよ。	ふね が やっと きし に ついた よ	
\\	船[ふね]がやっと
\\	に 着[つ]いたよ。			
\\	校庭	校庭[こうてい]	こうてい	
\\	陸上部は校庭で練習しています。	陸上部[りくじょうぶ]は 校庭[こうてい]で 練習[れんしゅう]しています。	りくじょうぶ は こうてい で れんしゅう して います	
\\	陸上部[りくじょうぶ]は
\\	で 練習[れんしゅう]しています。			
\\	植物	植物[しょくぶつ]	しょくぶつ	
\\	休日は植物の世話をして過ごします。	休日[きゅうじつ]は 植物[しょくぶつ]の 世話[せわ]をして 過[す]ごします。	きゅうじつ は しょくぶつ の せわ を して すごします	
\\	休日[きゅうじつ]は
\\	の 世話[せわ]をして 過[す]ごします。			
\\	植民地	植民地[しょくみんち]	しょくみんち	
\\	この国はイギリスの植民地でした。	この 国[くに]はイギリスの 植民地[しょくみんち]でした。	この くに は いぎりす の しょくみんち でした	
\\	この 国[くに]はイギリスの
\\	でした。			
\\	植木	植木[うえき]	うえき	
\\	植木に水をやりました。	植木[うえき]に 水[みず]をやりました。	うえき に みず を やりました	
\\	に 水[みず]をやりました。			
\\	植物園	植物園[しょくぶつえん]	しょくぶつえん	
\\	植物園には珍しい花がたくさんありますね。	植物園[しょくぶつえん]には 珍[めずら]しい 花[はな]がたくさんありますね。	しょくぶつえん に は めずらしい はな が たくさん あります ね	
\\	には 珍[めずら]しい 花[はな]がたくさんありますね。			
\\	根拠	根拠[こんきょ]	こんきょ	
\\	何を根拠にそんな事を言うのですか。	何[なに]を 根拠[こんきょ]にそんな 事[こと]を 言[い]うのですか。	なに を こんきょ に そんな こと を いう の です か	
\\	何[なに]を
\\	にそんな 事[こと]を 言[い]うのですか。			
\\	根本	根本[こんぽん]	こんぽん	
\\	問題の根本を見直しましょう。	問題[もんだい]の 根本[こんぽん]を 見直[みなお]しましょう。	もんだい の こんぽん を みなおしましょう	
\\	問題[もんだい]の
\\	を 見直[みなお]しましょう。			
\\	板	板[いた]	いた	
\\	父は長い板を買って来たんだ。	父[ちち]は 長[なが]い 板[いた]を 買[か]って 来[き]たんだ。	ちち は ながい いた を かって きた ん だ	
\\	父[ちち]は 長[なが]い
\\	を 買[か]って 来[き]たんだ。			
\\	木の葉	木[こ]の 葉[は]	このは	
\\	秋には木の葉が赤くなります。	秋[あき]には 木[こ]の 葉[は]が 赤[あか]くなります。	あき に は このは が あかく なります	
\\	秋[あき]には
\\	が 赤[あか]くなります。			
\\	書き言葉	書[か]き 言葉[ことば]	かきことば	
\\	書き言葉と話し言葉はだいぶ違うことがあります。	書[か]き 言葉[ことば]と 話[はな]し 言葉[ことば]はだいぶ 違[ちが]うことがあります。	かきことば と はなしことば は だいぶ ちがう こと が あります	
\\	と 話[はな]し 言葉[ことば]はだいぶ 違[ちが]うことがあります。			
\\	吸収	吸収[きゅうしゅう]	きゅうしゅう	
\\	彼は知識の吸収が早いですね。	彼[かれ]は 知識[ちしき]の 吸収[きゅうしゅう]が 早[はや]いですね。	かれ は ちしき の きゅうしゅう が はやい です ね	
\\	彼[かれ]は 知識[ちしき]の
\\	が 早[はや]いですね。			
\\	呼吸	呼吸[こきゅう]	こきゅう	
\\	ゆっくり呼吸してください。	ゆっくり 呼吸[こきゅう]してください。	ゆっくり こきゅう して ください	
\\	ゆっくり
\\	してください。			
\\	吸い込む	吸[す]い 込[こ]む	すいこむ	
\\	ほこりを吸い込んじゃった。	ほこりを 吸[す]い 込[こ]んじゃった。	ほこり を すいこんじゃった	
\\	ほこりを
\\	扱う	扱[あつか]う	あつかう	
\\	この荷物は丁寧に扱って下さい。	この 荷物[にもつ]は 丁寧[ていねい]に 扱[あつか]って 下[くだ]さい。	この にもつ は ていねい に あつかって ください	
\\	この 荷物[にもつ]は 丁寧[ていねい]に
\\	下[くだ]さい。			
\\	気の毒	気[き]の 毒[どく]	きのどく	
\\	彼らは気の毒な生活をしている。	彼[かれ]らは 気[き]の 毒[どく]な 生活[せいかつ]をしている。	かれら は きのどく な せいかつ を して いる	
\\	彼[かれ]らは
\\	な 生活[せいかつ]をしている。			
\\	撮影	撮影[さつえい]	さつえい	
\\	撮影は3ヶ月かけて行われました。	撮影[さつえい]は 3ヶ月[さんかげつ]かけて 行[おこな]われました。	さつえい は さんかげつ かけて おこなわれました	
\\	は 3ヶ月[さんかげつ]かけて 行[おこな]われました。			
\\	描く	描[えが]く	えがく	
\\	彼は人物を描くのがうまいな。	彼[かれ]は 人物[じんぶつ]を 描[えが]くのがうまいな。	かれ は じんぶつ を えがく の が うまい な	
\\	彼[かれ]は 人物[じんぶつ]を
\\	のがうまいな。			
\\	活躍	活躍[かつやく]	かつやく	
\\	彼の活躍で優勝したよ。	彼[かれ]の 活躍[かつやく]で 優勝[ゆうしょう]したよ。	かれ の かつやく で ゆうしょう した よ	
\\	彼[かれ]の
\\	で 優勝[ゆうしょう]したよ。			
\\	喜劇	喜劇[きげき]	きげき	
\\	昨夜はテレビで喜劇を見たよ。	昨夜[さくや]はテレビで 喜劇[きげき]を 見[み]たよ。	さくや は てれび で きげき を みた よ	
\\	昨夜[さくや]はテレビで
\\	を 見[み]たよ。			
\\	悲しみ	悲[かな]しみ	かなしみ	
\\	突然の悲しみが一家を襲いました。	突然[とつぜん]の 悲[かな]しみが 一家[いっか]を 襲[おそ]いました。	とつぜん の かなしみ が いっか を おそいました	
\\	突然[とつぜん]の
\\	が 一家[いっか]を 襲[おそ]いました。			
\\	固定	固定[こてい]	こてい	
\\	棒をテープで固定しなさい。	棒[ぼう]をテープで 固定[こてい]しなさい。	ぼう を てーぷ で こてい しなさい	
\\	棒[ぼう]をテープで
\\	しなさい。			
\\	固める	固[かた]める	かためる	
\\	私はもう決心を固めたの。	私[わたし]はもう 決心[けっしん]を 固[かた]めたの。	わたし は もう けっしん を かためた の	
\\	私[わたし]はもう 決心[けっしん]を
\\	の。			
\\	固まる	固[かた]まる	かたまる	
\\	もうプリンは固まったかな。	もうプリンは 固[かた]まったかな。	もう ぷりん は かたまった か な	
\\	もうプリンは
\\	かな。			
\\	固体	固体[こたい]	こたい	
\\	氷は固体です。	氷[こおり]は 固体[こたい]です。	こおり は こたい です	
\\	氷[こおり]は
\\	です。			
\\	固有	固有[こゆう]	こゆう	
\\	これは日本固有の鳥です。	これは 日本[にほん] 固有[こゆう]の 鳥[とり]です。	これ は にほん こゆう の とり です	
\\	これは 日本[にほん]
\\	の 鳥[とり]です。			
\\	古典	古典[こてん]	こてん	
\\	私は古典を読むのが好きです。	私[わたし]は 古典[こてん]を 読[よ]むのが 好[す]きです。	わたし は こてん を よむ の が すき です	
\\	私[わたし]は
\\	を 読[よ]むのが 好[す]きです。			
\\	殊に	殊[こと]に	ことに	
\\	ロックは殊に若者に人気だ。	ロックは 殊[こと]に 若者[わかもの]に 人気[にんき]だ。	ろっく は ことに わかもの に にんき だ	
\\	ロックは
\\	若者[わかもの]に 人気[にんき]だ。			
\\	微か	微[かす]か	かすか	
\\	階下から微かな音が聞こえた。	階下[かいか]から 微[かす]かな 音[おと]が 聞[き]こえた。	かいか から かすか な おと が きこえた	
\\	階下[かいか]から
\\	な 音[おと]が 聞[き]こえた。			
\\	形容詞	形容詞[けいようし]	けいようし	
\\	「大きい」は形容詞です。	
\\	大[おお]きい」は 形容詞[けいようし]です。	おおきい は けいようし です	
\\	大[おお]きい」は
\\	です。			
\\	司会	司会[しかい]	しかい	
\\	彼は司会が上手ですね。	彼[かれ]は 司会[しかい]が 上手[じょうず]ですね。	かれ は しかい が じょうず です ね	
\\	彼[かれ]は
\\	が 上手[じょうず]ですね。			
\\	技師	技師[ぎし]	ぎし	
\\	彼はレントゲン技師です。	彼[かれ]はレントゲン 技師[ぎし]です。	かれ は れんとげん ぎし です	
\\	彼[かれ]はレントゲン
\\	です。			
\\	師走	師走[しわす]	しわす	
\\	師走に入ると忙しくなります。	師走[しわす]に 入[はい]ると 忙[いそが]しくなります。	しわす に はいる と いそがしく なります	
\\	に 入[はい]ると 忙[いそが]しくなります。			
\\	小鳥	小鳥[ことり]	ことり	
\\	誕生日に小鳥を買ってもらいました。	誕生日[たんじょうび]に 小鳥[ことり]を 買[か]ってもらいました。	たんじょうび に ことり を かって もらいました	
\\	誕生日[たんじょうび]に
\\	を 買[か]ってもらいました。			
\\	大声	大声[おおごえ]	おおごえ	
\\	私たちは大声で歌を歌ったの。	私[わたし]たちは 大声[おおごえ]で 歌[うた]を 歌[うた]ったの。	わたしたち は おおごえ で うた を うたった の	
\\	私[わたし]たちは
\\	で 歌[うた]を 歌[うた]ったの。			
\\	歌声	歌声[うたごえ]	うたごえ	
\\	校舎から歌声が聞こえて来たよ。	校舎[こうしゃ]から 歌声[うたごえ]が 聞[き]こえて 来[き]たよ。	こうしゃ から うたごえ が きこえて きた よ	
\\	校舎[こうしゃ]から
\\	が 聞[き]こえて 来[き]たよ。			
\\	急激	急激[きゅうげき]	きゅうげき	
\\	これから高齢化が急激に進みます。	これから 高齢化[こうれいか]が 急激[きゅうげき]に 進[すす]みます。	これから こうれいか が きゅうげき に すすみます	
\\	これから 高齢化[こうれいか]が
\\	に 進[すす]みます。			
\\	感激	感激[かんげき]	かんげき	
\\	感激して泣いてしまいました。	感激[かんげき]して 泣[な]いてしまいました。	かんげき して ないて しまいました	
\\	して 泣[な]いてしまいました。			
\\	坂	坂[さか]	さか	
\\	この坂を上るのはすごくきついね。	この 坂[さか]を 上[のぼ]るのはすごくきついね。	この さか を のぼる の は すごく きつい ね	
\\	この
\\	を 上[のぼ]るのはすごくきついね。			
\\	徐行	徐行[じょこう]	じょこう	
\\	この先は徐行して下さい。	この 先[さき]は 徐行[じょこう]して 下[くだ]さい。	この さき は じょこう して ください	
\\	この 先[さき]は
\\	して 下[くだ]さい。			
\\	柔軟	柔軟[じゅうなん]	じゅうなん	
\\	彼は柔軟に対応をした。	彼[かれ]は 柔軟[じゅうなん]に 対応[たいおう]をした。	かれ は じゅうなん に たいおう を した	
\\	彼[かれ]は
\\	に 対応[たいおう]をした。			
\\	拡張	拡張[かくちょう]	かくちょう	
\\	その会社は店舗を拡張していますね。	その 会社[かいしゃ]は 店舗[てんぽ]を 拡張[かくちょう]していますね。	その かいしゃ は てんぽ を かくちょう して います ね	
\\	その 会社[かいしゃ]は 店舗[てんぽ]を
\\	していますね。			
\\	核	核[かく]	かく	
\\	核戦争は絶対に防ぐべきよ。	核[かく] 戦争[せんそう]は 絶対[ぜったい]に 防[ふせ]ぐべきよ。	かくせんそう は ぜったい に ふせぐ べき よ	
\\	戦争[せんそう]は 絶対[ぜったい]に 防[ふせ]ぐべきよ。			
\\	専攻	専攻[せんこう]	せんこう	
\\	大学では物理を専攻していました。	大学[だいがく]では 物理[ぶつり]を 専攻[せんこう]していました。	だいがく で は ぶつり を せんこう して いました	
\\	大学[だいがく]では 物理[ぶつり]を
\\	していました。			
\\	攻める	攻[せ]める	せめる	
\\	彼は積極的に攻めたが勝てなかったな。	彼[かれ]は 積極的[せっきょくてき]に 攻[せ]めたが 勝[か]てなかったな。	かれ は せっきょくてき に せめた が かてなかった な	
\\	彼[かれ]は 積極的[せっきょくてき]に
\\	が 勝[か]てなかったな。			
\\	撃つ	撃[う]つ	うつ	
\\	彼は拳銃で撃たれたわ。	彼[かれ]は 拳銃[けんじゅう]で 撃[う]たれたわ。	かれ は けんじゅう で うたれた わ	
\\	彼[かれ]は 拳銃[けんじゅう]で
\\	わ。			
\\	暴れる	暴[あば]れる	あばれる	
\\	彼は悪酔いして暴れたんだ。	彼[かれ]は 悪酔[わるよ]いして 暴[あば]れたんだ。	かれ は わるよい して あばれた ん だ	
\\	彼[かれ]は 悪酔[わるよ]いして
\\	んだ。			
\\	気絶	気絶[きぜつ]	きぜつ	
\\	彼女は驚いて気絶してしまったの。	彼女[かのじょ]は 驚[おどろ]いて 気絶[きぜつ]してしまったの。	かのじょ は おどろいて きぜつ して しまった の	
\\	彼女[かのじょ]は 驚[おどろ]いて
\\	してしまったの。			
\\	嫌う	嫌[きら]う	きらう	
\\	父はラッシュアワーを嫌っています。	父[ちち]はラッシュアワーを 嫌[きら]っています。	ちち は らっしゅあわー を きらって います	
\\	父[ちち]はラッシュアワーを
\\	嫌がる	嫌[いや]がる	いやがる	
\\	彼はタバコの煙を嫌がるの。	彼[かれ]はタバコの 煙[けむり]を 嫌[いや]がるの。	かれ は たばこ の けむり を いやがる の	
\\	彼[かれ]はタバコの 煙[けむり]を
\\	の。			
\\	機嫌	機嫌[きげん]	きげん	
\\	彼女は大変機嫌がいいね。	彼女[かのじょ]は 大変[たいへん] 機嫌[きげん]がいいね。	かのじょ は たいへん きげん が いい ね	
\\	彼女[かのじょ]は 大変[たいへん]
\\	がいいね。			
\\	好き嫌い	好[す]き 嫌[きら]い	すききらい	
\\	食べ物の好き嫌いは特にありません。	食[た]べ 物[もの]の 好[す]き 嫌[きら]いは 特[とく]にありません。	たべもの の すききらい は とくに ありません	
\\	食[た]べ 物[もの]の
\\	は 特[とく]にありません。			
\\	抗議	抗議[こうぎ]	こうぎ	
\\	彼の発言に対してたくさんの抗議があったよ。	彼[かれ]の 発言[はつげん]に 対[たい]してたくさんの 抗議[こうぎ]があったよ。	かれ の はつげん に たいして たくさん の こうぎ が あった よ	
\\	彼[かれ]の 発言[はつげん]に 対[たい]してたくさんの
\\	があったよ。			
\\	権威	権威[けんい]	けんい	
\\	博士はその道の権威です。	博士[はかせ]はその 道[みち]の 権威[けんい]です。	はかせ は そのみち の けんい です	
\\	博士[はかせ]はその 道[みち]の
\\	です。			
\\	威張る	威張[いば]る	いばる	
\\	彼は威張ってなんかいません。	彼[かれ]は 威張[いば]ってなんかいません。	かれ は いばって なんか いません	
\\	彼[かれ]は
\\	なんかいません。			
\\	情勢	情勢[じょうせい]	じょうせい	
\\	私は世界情勢を知るために毎日ニュースを見るわ。	私[わたし]は 世界[せかい] 情勢[じょうせい]を 知[し]るために 毎日[まいにち]ニュースを 見[み]るわ。	わたし は せかい じょうせい を しる ため に まいにち にゅーす を みる わ	
\\	私[わたし]は 世界[せかい]
\\	を 知[し]るために 毎日[まいにち]ニュースを 見[み]るわ。			
\\	恐れ	恐[おそ]れ	おそれ	
\\	叔父には心臓病の恐れがあります。	叔父[おじ]には 心臓病[しんぞうびょう]の 恐[おそ]れがあります。	おじ に は しんぞうびょう の おそれ が あります	
\\	叔父[おじ]には 心臓病[しんぞうびょう]の
\\	があります。			
\\	恐れる	恐[おそ]れる	おそれる	
\\	彼は受験の失敗を恐れています。	彼[かれ]は 受験[じゅけん]の 失敗[しっぱい]を 恐[おそ]れています。	かれ は じゅけん の しっぱい を おそれて います	
\\	彼[かれ]は 受験[じゅけん]の 失敗[しっぱい]を
\\	恐らく	恐[おそ]らく	おそらく	
\\	明日は恐らく晴れるでしょう。	明日[あした]は 恐[おそ]らく 晴[は]れるでしょう。	あした は おそらく はれる でしょう	
\\	明日[あした]は
\\	晴[は]れるでしょう。			
\\	怖がる	怖[こわ]がる	こわがる	
\\	彼女はクモを怖がります。	彼女[かのじょ]はクモを 怖[こわ]がります。	かのじょ は くも を こわがります	
\\	彼女[かのじょ]はクモを
\\	巨大	巨大[きょだい]	きょだい	
\\	あの巨大な建物は博物館です。	あの 巨大[きょだい]な 建物[たてもの]は 博物館[はくぶつかん]です。	あの きょだい な たてもの は はくぶつかん です	
\\	あの
\\	な 建物[たてもの]は 博物館[はくぶつかん]です。			
\\	拒否	拒否[きょひ]	きょひ	
\\	彼女は出席を拒否した。	彼女[かのじょ]は 出席[しゅっせき]を 拒否[きょひ]した。	かのじょ は しゅっせき を きょひ した	
\\	彼女[かのじょ]は 出席[しゅっせき]を
\\	した。			
\\	子孫	子孫[しそん]	しそん	
\\	彼は織田信長の子孫だよ。	彼[かれ]は 織田信長[おだのぶなが]の 子孫[しそん]だよ。	かれ は おだのぶなが の しそん だ よ	
\\	彼[かれ]は 織田信長[おだのぶなが]の
\\	だよ。			
\\	孤独	孤独[こどく]	こどく	
\\	彼は孤独な人生を送っていたんだ。	彼[かれ]は 孤独[こどく]な 人生[じんせい]を 送[おく]っていたんだ。	かれ は こどく な じんせい を おくって いた ん だ	
\\	彼[かれ]は
\\	な 人生[じんせい]を 送[おく]っていたんだ。			
\\	柄	柄[がら]	がら	
\\	彼は犬の柄の
\\	シャツを着ているよ。	彼[かれ]は 犬[いぬ]の 柄[がら]の
\\	シャツを 着[き]ているよ。	かれ は いぬ の がら の 
\\	しゃつ を きて いる よ	
\\	彼[かれ]は 犬[いぬ]の
\\	の
\\	シャツを 着[き]ているよ。			
\\	柄	柄[え]	え	
\\	この傘は柄が丈夫だな。	この 傘[かさ]は 柄[え]が 丈夫[じょうぶ]だな。	この かさ は え が じょうぶ だ な	
\\	この 傘[かさ]は
\\	が 丈夫[じょうぶ]だな。			
\\	枝	枝[えだ]	えだ	
\\	強風で木の枝が折れた。	強風[きょうふう]で 木[き]の 枝[えだ]が 折[お]れた。	きょうふう で き の えだ が おれた	
\\	強風[きょうふう]で 木[き]の
\\	が 折[お]れた。			
\\	枯れる	枯[か]れる	かれる	
\\	花瓶の花が枯れました。	花瓶[かびん]の 花[はな]が 枯[か]れました。	かびん の はな が かれました	
\\	花瓶[かびん]の 花[はな]が
\\	木枯らし	木枯[こが]らし	こがらし	
\\	外は木枯らしが吹いているよ。	外[そと]は 木枯[こが]らしが 吹[ふ]いているよ。	そと は こがらし が ふいて いる よ	
\\	外[そと]は
\\	が 吹[ふ]いているよ。			
\\	寿命	寿命[じゅみょう]	じゅみょう	
\\	亀の寿命は長いんだ。	亀[かめ]の 寿命[じゅみょう]は 長[なが]いんだ。	かめ の じゅみょう は ながい ん だ	
\\	亀[かめ]の
\\	は 長[なが]いんだ。			
\\	海水浴	海水浴[かいすいよく]	かいすいよく	
\\	夏休みには海水浴に行きます。	夏休[なつやす]みには 海水浴[かいすいよく]に 行[い]きます。	なつやすみ に は かいすいよく に いきます	
\\	夏休[なつやす]みには
\\	に 行[い]きます。			
\\	沿岸	沿岸[えんがん]	えんがん	
\\	今日は沿岸の波が荒いでしょう。	今日[きょう]は 沿岸[えんがん]の 波[なみ]が 荒[あら]いでしょう。	きょう は えんがん の なみ が あらい でしょう	
\\	今日[きょう]は
\\	の 波[なみ]が 荒[あら]いでしょう。			
\\	沖	沖[おき]	おき	
\\	沖に小島が見えます。	沖[おき]に 小島[こじま]が 見[み]えます。	おき に こじま が みえます	
\\	に 小島[こじま]が 見[み]えます。			
\\	泉	泉[いずみ]	いずみ	
\\	森の中にきれいな泉があるの。	森[もり]の 中[なか]にきれいな 泉[いずみ]があるの。	もり の なか に きれい な いずみ が ある の	
\\	森[もり]の 中[なか]にきれいな
\\	があるの。			
\\	叫び	叫[さけ]び	さけび	
\\	彼女の心の叫びに誰も気付かなかったよ。	彼女[かのじょ]の 心[こころ]の 叫[さけ]びに 誰[だれ]も 気付[きづ]かなかったよ。	かのじょ の こころ の さけび に だれ も きづかなかった よ	
\\	彼女[かのじょ]の 心[こころ]の
\\	に 誰[だれ]も 気付[きづ]かなかったよ。			
\\	叫ぶ	叫[さけ]ぶ	さけぶ	
\\	彼女は助けを求めて大声で叫んだの。	彼女[かのじょ]は 助[たす]けを 求[もと]めて 大声[おおごえ]で 叫[さけ]んだの。	かのじょ は たすけ を もとめて おおごえ で さけんだ の	
\\	彼女[かのじょ]は 助[たす]けを 求[もと]めて 大声[おおごえ]で
\\	の。			
\\	喫煙	喫煙[きつえん]	きつえん	
\\	ここでは喫煙できません。	ここでは 喫煙[きつえん]できません。	ここ で は きつえん できません 。	
\\	ここでは
\\	できません。			
\\	懸ける	懸[か]ける	かける	
\\	彼は仕事に命を懸けているの。	彼[かれ]は 仕事[しごと]に 命[いのち]を 懸[か]けているの。	かれ は しごと に いのち を かけている の 。	
\\	彼[かれ]は 仕事[しごと]に 命[いのち]を
\\	の。			
\\	恩	恩[おん]	おん	
\\	このご恩は一生忘れません。	このご 恩[おん]は 一生忘[いっしょう わす]れません。	この ごおん は いっしょう わすれません	
\\	このご
\\	は 一生忘[いっしょう わす]れません。			
\\	市街	市街[しがい]	しがい	
\\	夕方の市街は車が渋滞するよ。	夕方[ゆうがた]の 市街[しがい]は 車[くるま]が 渋滞[じゅうたい]するよ。	ゆうがた の しがい は くるま が じゅうたい する よ	
\\	夕方[ゆうがた]の
\\	は 車[くるま]が 渋滞[じゅうたい]するよ。			
\\	小麦	小麦[こむぎ]	こむぎ	
\\	小麦は色々な食べ物に使われている。	小麦[こむぎ]は 色々[いろいろ]な 食[た]べ 物[もの]に 使[つか]われている。	こむぎ は いろいろ な たべもの に つかわれて いる	
\\	は 色々[いろいろ]な 食[た]べ 物[もの]に 使[つか]われている。			
\\	小麦粉	小麦粉[こむぎこ]	こむぎこ	
\\	うどんは小麦粉から作られます。	うどんは 小麦粉[こむぎこ]から 作[つく]られます。	うどん は こむぎこ から つくられます	
\\	うどんは
\\	から 作[つく]られます。			
\\	大麦	大麦[おおむぎ]	おおむぎ	
\\	大麦はビールの原料になります。	大麦[おおむぎ]はビールの 原料[げんりょう]になります。	おおむぎ は びーる の げんりょう に なります	
\\	はビールの 原料[げんりょう]になります。			
\\	暦	暦[こよみ]	こよみ	
\\	暦の上では今日から冬ですね。	暦[こよみ]の 上[うえ]では 今日[きょう]から 冬[ふゆ]ですね。	こよみ の うえ で は きょう から ふゆ です ね	
\\	の 上[うえ]では 今日[きょう]から 冬[ふゆ]ですね。			
\\	毛皮	毛皮[けがわ]	けがわ	
\\	彼女は毛皮のコートを着ていたの。	彼女[かのじょ]は 毛皮[けがわ]のコートを 着[き]ていたの。	かのじょ は けがわ の こーと を きて いた の	
\\	彼女[かのじょ]は
\\	のコートを 着[き]ていたの。			
\\	毛糸	毛糸[けいと]	けいと	
\\	彼女は毛糸のセーターを編みました。	彼女[かのじょ]は 毛糸[けいと]のセーターを 編[あ]みました。	かのじょ は けいと の せーたー を あみました	
\\	彼女[かのじょ]は
\\	のセーターを 編[あ]みました。			
\\	尾	尾[お]	お	
\\	尾の長い鳥が飛んでいますね。	尾[お]の 長[なが]い 鳥[とり]が 飛[と]んでいますね。	お の ながい とり が とんで います ね	
\\	の 長[なが]い 鳥[とり]が 飛[と]んでいますね。			
\\	唇	唇[くちびる]	くちびる	
\\	寒くて唇が青くなってしまった。	寒[さむ]くて 唇[くちびる]が 青[あお]くなってしまった。	さむく て くちびる が あおく なって しまった	
\\	寒[さむ]くて
\\	が 青[あお]くなってしまった。			
\\	居間	居間[いま]	いま	
\\	父は居間でテレビを見ている。	父[ちち]は 居間[いま]でテレビを 見[み]ている。	ちち は いま で てれび を みて いる	
\\	父[ちち]は
\\	でテレビを 見[み]ている。			
\\	居る	居[お]る	おる	
\\	母は今、うちに居りません。	母[はは]は 今[いま]、うちに 居[お]りません。	はは は いま うち に おりません	
\\	母[はは]は 今[いま]、うちに
\\	戸籍	戸籍[こせき]	こせき	
\\	結婚すると新しい戸籍が作られます。	結婚[けっこん]すると 新[あたら]しい 戸籍[こせき]が 作[つく]られます。	けっこん する と あたらしい こせき が つくられます	
\\	結婚[けっこん]すると 新[あたら]しい
\\	が 作[つく]られます。			
\\	新鮮	新鮮[しんせん]	しんせん	
\\	この店では新鮮な野菜が買えますよ。	この 店[みせ]では 新鮮[しんせん]な 野菜[やさい]が 買[か]えますよ。	この みせ で は しんせん な やさい が かえます よ	
\\	この 店[みせ]では
\\	な 野菜[やさい]が 買[か]えますよ。			
\\	君	君[きみ]	きみ	
\\	この本を君にあげます。	この 本[ほん]を 君[きみ]にあげます。	この ほん を きみ に あげます	
\\	この 本[ほん]を
\\	にあげます。			
\\	山脈	山脈[さんみゃく]	さんみゃく	
\\	列車の窓から雄大な山脈が見えたんだよ。	列車[れっしゃ]の 窓[まど]から 雄大[ゆうだい]な 山脈[さんみゃく]が 見[み]えたんだよ。	れっしゃ の まど から ゆうだい な さんみゃく が みえた ん だ よ	
\\	列車[れっしゃ]の 窓[まど]から 雄大[ゆうだい]な
\\	が 見[み]えたんだよ。			
\\	暮らす	暮[く]らす	くらす	
\\	将来は海の近くで暮らしたいな。	将来[しょうらい]は 海[うみ]の 近[ちか]くで 暮[く]らしたいな。	しょうらい は うみ の ちかく で くらしたい な	
\\	将来[しょうらい]は 海[うみ]の 近[ちか]くで
\\	な。			
\\	暮らし	暮[く]らし	くらし	
\\	彼女は毎日の暮らしを楽しんでいますね。	彼女[かのじょ]は 毎日[まいにち]の 暮[く]らしを 楽[たの]しんでいますね。	かのじょ は まいにち の くらし を たのしんで います ね	
\\	彼女[かのじょ]は 毎日[まいにち]の
\\	を 楽[たの]しんでいますね。			
\\	暮れ	暮[く]れ	くれ	
\\	暮れは用事が多くて忙しいです。	暮[く]れは 用事[ようじ]が 多[おお]くて 忙[いそが]しいです。	くれ は ようじ が おおくて いそがしい です	
\\	は 用事[ようじ]が 多[おお]くて 忙[いそが]しいです。			
\\	暮れる	暮[く]れる	くれる	
\\	日が暮れる前に帰りましょう。	日[ひ]が 暮[く]れる 前[まえ]に 帰[かえ]りましょう。	ひ が くれる まえ に かえりましょう	
\\	日[ひ]が
\\	前[まえ]に 帰[かえ]りましょう。			
\\	推薦	推薦[すいせん]	すいせん	
\\	彼は会長に推薦されたよ。	彼[かれ]は 会長[かいちょう]に 推薦[すいせん]されたよ。	かれ は かいちょう に すいせん された よ	
\\	彼[かれ]は 会長[かいちょう]に
\\	されたよ。			
\\	慌ただしい	慌[あわ]ただしい	あわただしい	
\\	今日は慌ただしい一日でした。	今日[きょう]は 慌[あわ]ただしい 一日[いちにち]でした。	きょう は あわただしい いちにち でした	
\\	今日[きょう]は
\\	一日[いちにち]でした。			
\\	慌てる	慌[あわ]てる	あわてる	
\\	そんなに慌ててどこに行くの。	そんなに 慌[あわ]ててどこに 行[い]くの。	そんなに あわてて どこ に いく の	
\\	そんなに
\\	どこに 行[い]くの。			
\\	国旗	国旗[こっき]	こっき	
\\	日本の国旗は描くのが簡単です。	日本[にほん]の 国旗[こっき]は 描[か]くのが 簡単[かんたん]です。	にほん の こっき は かく の が かんたん です	
\\	日本[にほん]の
\\	は 描[か]くのが 簡単[かんたん]です。			
\\	座布団	座布団[ざぶとん]	ざぶとん	
\\	この座布団は座り心地がいいね。	この 座布団[ざぶとん]は 座[すわ]り 心地[ごこち]がいいね。	この ざぶとん は すわり ごこち が いい ね	
\\	この
\\	は 座[すわ]り 心地[ごこち]がいいね。			
\\	抱える	抱[かか]える	かかえる	
\\	彼は大きな荷物を抱えているわ。	彼[かれ]は 大[おお]きな 荷物[にもつ]を 抱[かか]えているわ。	かれ は おおき な にもつ を かかえて いる わ	
\\	彼[かれ]は 大[おお]きな 荷物[にもつ]を
\\	わ。			
\\	抱く	抱[いだ]く	いだく	
\\	少年よ大志を抱け。	少年[しょうねん]よ 大志[たいし]を 抱[いだ]け。	しょうねん よ たいし を いだけ	
\\	少年[しょうねん]よ 大志[たいし]を
\\	句	句[く]	く	
\\	この句はどんな意味でしょうか。	この 句[く]はどんな 意味[いみ]でしょうか。	この く は どんな いみ でしょう か	
\\	この
\\	はどんな 意味[いみ]でしょうか。			
\\	慣用句	慣用句[かんようく]	かんようく	
\\	「手が空く」は慣用句です。	
\\	手[て]が 空[す]く」は 慣用句[かんようく]です。	て が すく は かんようく です	
\\	手[て]が 空[す]く」は
\\	です。			
\\	旧	旧[きゅう]	きゅう	
\\	旧ソビエトは今はロシアと呼ばれている。	旧[きゅう]ソビエトは 今[いま]はロシアと 呼[よ]ばれている。	きゅうそびえと は いま は ろしあ と よばれて いる	
\\	ソビエトは 今[いま]はロシアと 呼[よ]ばれている。			
\\	小児科	小児科[しょうにか]	しょうにか	
\\	子供を小児科に連れて行くところです。	子供[こども]を 小児科[しょうにか]に 連[つ]れて 行[い]くところです。	こども を しょうにか に つれて いく ところ です	
\\	子供[こども]を
\\	に 連[つ]れて 行[い]くところです。			
\\	姓名	姓名[せいめい]	せいめい	
\\	あなたの姓名を教えてください。	あなたの 姓名[せいめい]を 教[おし]えてください。	あなた の せいめい を おしえて ください	
\\	あなたの
\\	を 教[おし]えてください。			
\\	姓	姓[せい]	せい	
\\	結婚して姓が変わりました。	結婚[けっこん]して 姓[せい]が 変[か]わりました。	けっこん して せい が かわりました	
\\	結婚[けっこん]して
\\	が 変[か]わりました。			
\\	幼い	幼[おさな]い	おさない	
\\	彼女には幼い息子がいます。	彼女[かのじょ]には 幼[おさな]い 息子[むすこ]がいます。	かのじょ に は おさない むすこ が います	
\\	彼女[かのじょ]には
\\	息子[むすこ]がいます。			
\\	居眠り	居眠[いねむ]り	いねむり	
\\	彼はソファーで居眠りをしているよ。	彼[かれ]はソファーで 居眠[いねむ]りをしているよ。	かれ は そふぁー で いねむり を して いる よ	
\\	彼[かれ]はソファーで
\\	をしているよ。			
\\	垂直	垂直[すいちょく]	すいちょく	
\\	彼らは垂直のがけを登り始めたの。	彼[かれ]らは 垂直[すいちょく]のがけを 登[のぼ]り 始[はじ]めたの。	かれら は すいちょく の がけ を のぼりはじめた の	
\\	彼[かれ]らは
\\	のがけを 登[のぼ]り 始[はじ]めたの。			
\\	心掛ける	心掛[こころが]ける	こころがける	
\\	安全運転を心掛けてください。	安全運転[あんぜん うんてん]を 心掛[こころが]けてください。	あんぜん うんてん を こころがけて ください	
\\	安全運転[あんぜん うんてん]を
\\	ください。			
\\	拝む	拝[おが]む	おがむ	
\\	仏像に手を合わせて拝みました。	仏像[ぶつぞう]に 手[て]を 合[あ]わせて 拝[おが]みました。	ぶつぞう に て を あわせて おがみました	
\\	仏像[ぶつぞう]に 手[て]を 合[あ]わせて
\\	括弧	括弧[かっこ]	かっこ	
\\	括弧の部分は省略できます。	括弧[かっこ]の 部分[ぶぶん]は 省略[しょうりゃく]できます。	かっこ の ぶぶん は しょうりゃく できます	
\\	の 部分[ぶぶん]は 省略[しょうりゃく]できます。			
\\	指揮	指揮[しき]	しき	
\\	彼がそのプロジェクトの指揮を取ったの。	彼[かれ]がそのプロジェクトの 指揮[しき]を 取[と]ったの。	かれ が その ぷろじぇくと の しき を とった の	
\\	彼[かれ]がそのプロジェクトの
\\	を 取[と]ったの。			
\\	抑える	抑[おさ]える	おさえる	
\\	彼は怒りを抑えていたの。	彼[かれ]は 怒[いか]りを 抑[おさ]えていたの。	かれ は いかり を おさえて いた の	
\\	彼[かれ]は 怒[いか]りを
\\	の。			
\\	城	城[しろ]	しろ	
\\	今回の旅行ではヨーロッパの城を見て回ります。	今回[こんかい]の 旅行[りょこう]ではヨーロッパの 城[しろ]を 見[み]て 回[まわ]ります。	こんかい の りょこう で は よーろっぱ の しろ を みて まわります	
\\	今回[こんかい]の 旅行[りょこう]ではヨーロッパの
\\	を 見[み]て 回[まわ]ります。			
\\	栽培	栽培[さいばい]	さいばい	
\\	この地方ではみかんの栽培が盛んです。	この 地方[ちほう]ではみかんの 栽培[さいばい]が 盛[さか]んです。	この ちほう で は みかん の さいばい が さかん です	
\\	この 地方[ちほう]ではみかんの
\\	が 盛[さか]んです。			
\\	幾ら	幾[いく]ら	いくら	
\\	幾ら呼んでも彼は返事をしなかったわ。	幾[いく]ら 呼[よ]んでも 彼[かれ]は 返事[へんじ]をしなかったわ。	いくら よんで も かれ は へんじ を しなかった わ	
\\	呼[よ]んでも 彼[かれ]は 返事[へんじ]をしなかったわ。			
\\	後悔	後悔[こうかい]	こうかい	
\\	後悔しても、しょうがない。	後悔[こうかい]しても、しょうがない。	こうかい して も しょうがない	
\\	しても、しょうがない。			
\\	悔しい	悔[くや]しい	くやしい	
\\	試合に負けてとても悔しい。	試合[しあい]に 負[ま]けてとても 悔[くや]しい。	しあい に まけて とても くやしい	
\\	試合[しあい]に 負[ま]けてとても
\\	宗教	宗教[しゅうきょう]	しゅうきょう	
\\	宗教を持たない人もたくさんいるわ。	宗教[しゅうきょう]を 持[も]たない 人[ひと]もたくさんいるわ。	しゅうきょう を もたない ひと も たくさん いる わ	
\\	を 持[も]たない 人[ひと]もたくさんいるわ。			
\\	審議	審議[しんぎ]	しんぎ	
\\	その問題は審議中です。	その 問題[もんだい]は 審議[しんぎ] 中[ちゅう]です。	その もんだい は しんぎちゅう です	
\\	その 問題[もんだい]は
\\	中[ちゅう]です。			
\\	憲法	憲法[けんぽう]	けんぽう	
\\	憲法を改正することは難しいわね。	憲法[けんぽう]を 改正[かいせい]することは 難[むずか]しいわね。	けんぽう を かいせい する こと は むずかしい わ ね	
\\	を 改正[かいせい]することは 難[むずか]しいわね。			
\\	惜しむ	惜[お]しむ	おしむ	
\\	私たちはみな彼の死を惜しんだの。	私[わたし]たちはみな 彼[かれ]の 死[し]を 惜[お]しんだの。	わたしたち は みな かれ の し を おしんだ の	
\\	私[わたし]たちはみな 彼[かれ]の 死[し]を
\\	の。			
\\	惜しい	惜[お]しい	おしい	
\\	惜しい、もう少しで優勝だった。	惜[お]しい、もう 少[すこ]しで 優勝[ゆうしょう]だった。	おしい もうすこし で ゆうしょう だった	
\\	、もう 少[すこ]しで 優勝[ゆうしょう]だった。			
\\	恨み	恨[うら]み	うらみ	
\\	彼女は長年の恨みを晴らした。	彼女[かのじょ]は 長年[ながねん]の 恨[うら]みを 晴[は]らした。	かのじょ は ながねん の うらみ を はらした	
\\	彼女[かのじょ]は 長年[ながねん]の
\\	を 晴[は]らした。			
\\	恨む	恨[うら]む	うらむ	
\\	彼を恨んではいけません。	彼[かれ]を 恨[うら]んではいけません。	かれ を うらんで は いけません	
\\	彼[かれ]を
\\	はいけません。			
\\	怪しい	怪[あや]しい	あやしい	
\\	その男の行動は怪しかったわ。	その 男[おとこ]の 行動[こうどう]は 怪[あや]しかったわ。	その おとこ の こうどう は あやしかった わ	
\\	その 男[おとこ]の 行動[こうどう]は
\\	わ。			
\\	怪しむ	怪[あや]しむ	あやしむ	
\\	警察はそのグループを怪しんでいます。	警察[けいさつ]はそのグループを 怪[あや]しんでいます。	けいさつ は その ぐるーぷ を あやしんで います	
\\	警察[けいさつ]はそのグループを
\\	歓迎	歓迎[かんげい]	かんげい	
\\	温かい歓迎を受けました。	温[あたた]かい 歓迎[かんげい]を 受[う]けました。	あたたかい かんげい を うけました	
\\	温[あたた]かい
\\	を 受[う]けました。			
\\	塊	塊[かたまり]	かたまり	
\\	道に土の塊ができてたよ。	道[みち]に 土[つち]の 塊[かたまり]ができてたよ。	みち に つち の かたまり が できてた よ	
\\	道[みち]に 土[つち]の
\\	ができてたよ。			
\\	基礎	基礎[きそ]	きそ	
\\	今ドイツ語の基礎を学んでいます。	今[いま]ドイツ 語[ご]の 基礎[きそ]を 学[まな]んでいます。	いま どいつご の きそ を まなんで います	
\\	今[いま]ドイツ 語[ご]の
\\	を 学[まな]んでいます。			
\\	洗剤	洗剤[せんざい]	せんざい	
\\	床に洗剤をこぼしてしまいました。	床[ゆか]に 洗剤[せんざい]をこぼしてしまいました。	ゆか に せんざい を こぼして しまいました	
\\	床[ゆか]に
\\	をこぼしてしまいました。			
\\	強烈	強烈[きょうれつ]	きょうれつ	
\\	彼女は強烈な個性の持ち主ですよ。	彼女[かのじょ]は 強烈[きょうれつ]な 個性[こせい]の 持[も]ち 主[ぬし]ですよ。	かのじょ は きょうれつ な こせい の もちぬし です よ	
\\	彼女[かのじょ]は
\\	な 個性[こせい]の 持[も]ち 主[ぬし]ですよ。			
\\	官庁	官庁[かんちょう]	かんちょう	
\\	その古い建物は官庁です。	その 古[ふる]い 建物[たてもの]は 官庁[かんちょう]です。	その ふるい たてもの は かんちょう です	
\\	その 古[ふる]い 建物[たてもの]は
\\	です。			
\\	擦る	擦[こす]る	こする	
\\	冷えた手を擦って温めた。	冷[ひ]えた 手[て]を 擦[こす]って 温[あたた]めた。	ひえた て を こすって あたためた	
\\	冷[ひ]えた 手[て]を
\\	温[あたた]めた。			
\\	汗	汗[あせ]	あせ	
\\	彼は額に汗をかいていたの。	彼[かれ]は 額[ひたい]に 汗[あせ]をかいていたの。	かれ は ひたい に あせ を かいていた の	
\\	彼[かれ]は 額[ひたい]に
\\	をかいていたの。			
\\	後輩	後輩[こうはい]	こうはい	
\\	彼は大学の後輩です。	彼[かれ]は 大学[だいがく]の 後輩[こうはい]です。	かれ は だいがく の こうはい です	
\\	彼[かれ]は 大学[だいがく]の
\\	です。			
\\	合唱	合唱[がっしょう]	がっしょう	
\\	私たちは校歌を合唱したの。	私[わたし]たちは 校歌[こうか]を 合唱[がっしょう]したの。	わたしたち は こうか を がっしょう した の	
\\	私[わたし]たちは 校歌[こうか]を
\\	したの。			
\\	敬う	敬[うやま]う	うやまう	
\\	両親を敬うことは大切です。	両親[りょうしん]を 敬[うやま]うことは 大切[たいせつ]です。	りょうしん を うやまう こと は たいせつ です	
\\	両親[りょうしん]を
\\	ことは 大切[たいせつ]です。			
\\	座敷	座敷[ざしき]	ざしき	
\\	明日はお座敷での宴会になります。	明日[あした]はお 座敷[ざしき]での 宴会[えんかい]になります。	あした は おざしき で の えんかい に なります	
\\	明日[あした]はお
\\	での 宴会[えんかい]になります。			
\\	敷金	敷金[しききん]	しききん	
\\	マンションの敷金を払いました。	マンションの 敷金[しききん]を 払[はら]いました。	まんしょん の しききん を はらいました	
\\	マンションの
\\	を 払[はら]いました。			
\\	暑中見舞い	暑中見舞[しょちゅうみま]い	しょちゅうみまい	
\\	先生に暑中見舞いを出しました。	先生[せんせい]に 暑中見舞[しょちゅうみま]いを 出[だ]しました。	せんせい に しょちゅうみまい を だしました	
\\	先生[せんせい]に
\\	を 出[だ]しました。			
\\	埋める	埋[う]める	うめる	
\\	庭に穴を掘ってそれを埋めました。	庭[にわ]に 穴[あな]を 掘[ほ]ってそれを 埋[う]めました。	にわ に あな を ほって それ を うめました	
\\	庭[にわ]に 穴[あな]を 掘[ほ]ってそれを
\\	埋める	埋[うず]める	うずめる	
\\	パレードと観衆が道を埋めていたよ。	パレードと 観衆[かんしゅう]が 道[みち]を 埋[うず]めていたよ。	ぱれーど と かんしゅう が みち を うずめて いた よ	
\\	パレードと 観衆[かんしゅう]が 道[みち]を
\\	よ。			
\\	墨	墨[すみ]	すみ	
\\	服に墨がついちゃった。	服[ふく]に 墨[すみ]がついちゃった。	ふく に すみ が ついちゃった	
\\	服[ふく]に
\\	がついちゃった。			
\\	奨学金	奨学金[しょうがくきん]	しょうがくきん	
\\	彼女は奨学金で大学に行きました。	彼女[かのじょ]は 奨学金[しょうがくきん]で 大学[だいがく]に 行[い]きました。	かのじょ は しょうがくきん で だいがく に いきました	
\\	彼女[かのじょ]は
\\	で 大学[だいがく]に 行[い]きました。			
\\	地獄	地獄[じごく]	じごく	
\\	地震の後、街は地獄のようだったよ。	地震[じしん]の 後[あと]、 街[まち]は 地獄[じごく]のようだったよ。	じしん の あと まち は じごく の よう だった よ	
\\	地震[じしん]の 後[あと]、 街[まち]は
\\	のようだったよ。			
\\	改善	改善[かいぜん]	かいぜん	
\\	彼は食生活を改善しました。	彼[かれ]は 食生活[しょくせいかつ]を 改善[かいぜん]しました。	かれ は しょくせいかつ を かいぜん しました	
\\	彼[かれ]は 食生活[しょくせいかつ]を
\\	しました。			
\\	口紅	口紅[くちべに]	くちべに	
\\	赤い口紅を買いました。	赤[あか]い 口紅[くちべに]を 買[か]いました。	あかい くちべに を かいました	
\\	赤[あか]い
\\	を 買[か]いました。			
\\	梅	梅[うめ]	うめ	
\\	梅の花が咲きました。	梅[うめ]の 花[はな]が 咲[さ]きました。	うめ の はな が さきました	
\\	の 花[はな]が 咲[さ]きました。			
\\	梅干	梅干[うめぼし]	うめぼし	
\\	うちでは、朝食には必ず梅干が出ます。	うちでは、 朝食[ちょうしょく]には 必[かなら]ず 梅干[うめぼし]が 出[で]ます。	うち で は ちょうしょく に は かならず うめぼし が でます	
\\	うちでは、 朝食[ちょうしょく]には 必[かなら]ず
\\	が 出[で]ます。			
\\	巣	巣[す]	す	
\\	アリは土の中に巣を作ります。	アリは 土[つち]の 中[なか]に 巣[す]を 作[つく]ります。	あり は つち の なか に す を つくります	
\\	アリは 土[つち]の 中[なか]に
\\	を 作[つく]ります。			
\\	囲碁	囲碁[いご]	いご	
\\	彼の趣味は囲碁です。	彼[かれ]の 趣味[しゅみ]は 囲碁[いご]です。	かれ の しゅみ は いご です	
\\	彼[かれ]の 趣味[しゅみ]は
\\	です。			
\\	嘘つき	嘘[うそ]つき	うそつき	
\\	嘘つきは泥棒の始まりよ。	嘘[うそ]つきは 泥棒[どろぼう]の 始[はじ]まりよ。	うそつき は どろぼう の はじまり よ 。	
\\	は 泥棒[どろぼう]の 始[はじ]まりよ。			
\\	崖	崖[がけ]	がけ	
\\	大雨で崖が崩れたんだ。	大雨[おおあめ]で 崖[がけ]が 崩[くず]れたんだ。	おおあめ で がけ が くずれた ん だ	
\\	大雨[おおあめ]で
\\	が 崩[くず]れたんだ。			
\\	嵐	嵐[あらし]	あらし	
\\	嵐で庭の木が折れたよ。	嵐[あらし]で 庭[にわ]の 木[き]が 折[お]れたよ。	あらし で にわ の き が おれた よ 。	
\\	で 庭[にわ]の 木[き]が 折[お]れたよ。			
\\	海峡	海峡[かいきょう]	かいきょう	
\\	その海峡に橋が掛けられました。	その 海峡[かいきょう]に 橋[はし]が 掛[か]けられました。	その かいきょう に はし が かけられました	
\\	その
\\	に 橋[はし]が 掛[か]けられました。			
\\	噂	噂[うわさ]	うわさ	
\\	その噂は本当ですか。	その 噂[うわさ]は 本当[ほんとう]ですか。	その うわさ は ほんとう です か	
\\	その
\\	は 本当[ほんとう]ですか。			
\\	校舎	校舎[こうしゃ]	こうしゃ	
\\	古い校舎の修理が必要です。	古[ふる]い 校舎[こうしゃ]の 修理[しゅうり]が 必要[ひつよう]です。	ふるい こうしゃ の しゅうり が ひつよう です	
\\	古[ふる]い
\\	の 修理[しゅうり]が 必要[ひつよう]です。			
\\	娯楽	娯楽[ごらく]	ごらく	
\\	テレビは彼のいちばんの娯楽です。	テレビは 彼[かれ]のいちばんの 娯楽[ごらく]です。	てれび は かれ の いちばん の ごらく です	
\\	テレビは 彼[かれ]のいちばんの
\\	です。			
\\	汽車	汽車[きしゃ]	きしゃ	
\\	汽車で街まで行った。	汽車[きしゃ]で 街[まち]まで 行[い]った。	きしゃ で まち まで いった	
\\	で 街[まち]まで 行[い]った。			
\\	揚げる	揚[あ]げる	あげる	
\\	彼女は夕食に天ぷらを揚げました。	彼女[かのじょ]は 夕食[ゆうしょく]に 天[てん]ぷらを 揚[あ]げました。	かのじょ は ゆうしょく に てんぷら を あげました	
\\	彼女[かのじょ]は 夕食[ゆうしょく]に 天[てん]ぷらを
\\	故郷	故郷[こきょう]	こきょう	
\\	彼女は久しぶりに故郷に帰りました。	彼女[かのじょ]は 久[ひさ]しぶりに 故郷[こきょう]に 帰[かえ]りました。	かのじょ は ひさしぶり に こきょう に かえりました	
\\	彼女[かのじょ]は 久[ひさ]しぶりに
\\	に 帰[かえ]りました。			
\\	歌舞伎	歌舞伎[かぶき]	かぶき	
\\	歌舞伎の芝居を見に行きました。	歌舞伎[かぶき]の 芝居[しばい]を 見[み]に 行[い]きました。	かぶき の しばい を み に いきました 。	
\\	の 芝居[しばい]を 見[み]に 行[い]きました。			
\\	囁く	囁[ささや]く	ささやく	
\\	「この会議は退屈だ」と同僚が私に囁いたの。	「この 会議[かいぎ]は 退屈[たいくつ]だ」と 同僚[どうりょう]が 私[わたし]に 囁[ささや]いたの。	
\\	この かいぎ は たいくつ だ 
\\	と どうりょう が わたし に ささやいた の 。	
\\	「この 会議[かいぎ]は 退屈[たいくつ]だ」と 同僚[どうりょう]が 私[わたし]に
\\	の。			
\\	咳	咳[せき]	せき	
\\	咳が止まらないので病院に行ってきたの。	咳[せき]が 止[と]まらないので 病院[びょういん]に 行[い]ってきたの。	せき が とまらない ので びょういん に いってきた の 。	
\\	が 止[と]まらないので 病院[びょういん]に 行[い]ってきたの。			
\\	噛み付く	噛[か]み 付[つ]く	かみつく	
\\	犬が手に噛み付きました。	犬[いぬ]が 手[て]に 噛[か]み 付[つ]きました。	いぬ が て に かみつきました。	
\\	犬[いぬ]が 手[て]に
\\	屑	屑[くず]	くず	
\\	彼の背広に糸屑がついているわ。	彼[かれ]の 背広[せびろ]に 糸[いと] 屑[くず]がついているわ。	かれ の せびろ に いとくず が ついて いる わ	
\\	彼[かれ]の 背広[せびろ]に 糸[いと]
\\	がついているわ。			
\\	掻く	掻[か]く	かく	
\\	背中をお母さんに掻いてもらったの。	背中[せなか]をお 母[かあ]さんに 掻[か]いてもらったの。	せなか を おかあさん に かいて もらった の	
\\	背中[せなか]をお 母[かあ]さんに
\\	の。			
\\	掻き回す	掻[か]き 回[まわ]す	かきまわす	
\\	母は鍋のシチューを掻き回しているよ。	母[はは]は 鍋[なべ]のシチューを 掻[か]き 回[まわ]しているよ。	はは は なべ の しちゅー を かきまわしている よ 。	
\\	母[はは]は 鍋[なべ]のシチューを
\\	よ。			
\\	憧れ	憧[あこが]れ	あこがれ	
\\	海外に住むのは私の憧れです。	海外[かいがい]に 住[す]むのは 私[わたし]の 憧[あこが]れです。	かいがい に すむ の は わたし の あこがれ です	
\\	海外[かいがい]に 住[す]むのは 私[わたし]の
\\	です。			
\\	憧れる	憧[あこが]れる	あこがれる	
\\	彼はパイロットの職に憧れているんだ。	彼[かれ]はパイロットの 職[しょく]に 憧[あこが]れているんだ。	かれ は ぱいろっと の しょく に あこがれて いる ん だ	
\\	彼[かれ]はパイロットの 職[しょく]に
\\	んだ。			
\\	御無沙汰	御無沙汰[ごぶさた]	ごぶさた	
\\	長いこと御無沙汰いたしました。	長[なが]いこと 御無沙汰[ごぶさた]いたしました。	ながい こと ごぶさた いたしました	
\\	長[なが]いこと
\\	いたしました。			
\\	汲む	汲[く]む	くむ	
\\	小さなバケツで水を汲んだの。	小[ちい]さなバケツで 水[みず]を 汲[く]んだの。	ちいさ な ばけつ で みず を くんだ の	
\\	小[ちい]さなバケツで 水[みず]を
\\	の。			
\\	曖昧	曖昧[あいまい]	あいまい	
\\	彼女は曖昧な返事をしたね。	彼女[かのじょ]は 曖昧[あいまい]な 返事[へんじ]をしたね。	かのじょ は あいまい な へんじ を した ね	
\\	彼女[かのじょ]は
\\	な 返事[へんじ]をしたね。			
\\	大晦日	大晦日[おおみそか]	おおみそか	
\\	日本では、大晦日にそばを食べます。	日本[にっぽん]では、 大晦日[おおみそか]にそばを 食[た]べます。	にっぽん で は おおみそか に そば を たべます	
\\	日本[にっぽん]では、
\\	にそばを 食[た]べます。			
\\	柿	柿[かき]	かき	
\\	庭に柿の実がなりました。	庭[にわ]に 柿[かき]の 実[み]がなりました。	にわ に かき の み が なりました	
\\	庭[にわ]に
\\	の 実[み]がなりました。			
\\	嗅ぐ	嗅[か]ぐ	かぐ	
\\	犬がお皿の匂いをクンクン嗅いでいるね。	犬[いぬ]がお 皿[さら]の 匂[にお]いをクンクン 嗅[か]いでいるね。	いぬ が おさら の におい を くんくん かいで いる ね	
\\	犬[いぬ]がお 皿[さら]の 匂[にお]いをクンクン
\\	ね。			
\\	日ごろ	日[ひ]ごろ	ひごろ	
\\	彼女には日ごろからお世話になっています。	彼女[かのじょ]には 日[ひ]ごろからお 世話[せわ]になっています。	かのじょ に は ひごろ から おせわ に なって います	
\\	彼女[かのじょ]には
\\	からお 世話[せわ]になっています。			
\\	日ソ	日[にっ]ソ	にっそ	
\\	当時、日ソ会談が開かれた。	当時[とうじ]、 日[にっ]ソ 会談[かいだん]が 開[ひら]かれた。	とうじ にっそ かいだん が ひらかれた	
\\	当時[とうじ]、
\\	会談[かいだん]が 開[ひら]かれた。			
\\	日	日[にち]	にち	
\\	日仏の共同研究が始まったな。	日[にち] 仏[ふつ]の 共同研究[きょうどう けんきゅう]が 始[はじ]まったな。	にちふつ の きょうどう けんきゅう が はじまった な	
\\	仏[ふつ]の 共同研究[きょうどう けんきゅう]が 始[はじ]まったな。			
\\	日時	日時[にちじ]	にちじ	
\\	試写会の日時を教えてください。	試写会[ししゃかい]の 日時[にちじ]を 教[おし]えてください。	ししゃかい の にちじ を おしえて ください	
\\	試写会[ししゃかい]の
\\	を 教[おし]えてください。			
\\	日日	日日[ひにち]	ひにち	
\\	ミーティングの日日を間違えました。	ミーティングの 日日[ひにち]を 間違[まちが]えました。	みーてぃんぐ の ひにち を まちがえました	
\\	ミーティングの
\\	を 間違[まちが]えました。			
\\	日々	日々[ひび]	ひび	
\\	日々の努力が大切です。	日々[ひび]の 努力[どりょく]が 大切[たいせつ]です。	ひび の どりょく が たいせつ です	
\\	の 努力[どりょく]が 大切[たいせつ]です。			
\\	月日	月日[つきひ]	つきひ	
\\	月日が経つのは早いものです。	月日[つきひ]が 経[た]つのは 早[はや]いものです。	つきひ が たつ の は はやい もの です	
\\	が 経[た]つのは 早[はや]いものです。			
\\	土	土[つち]	つち	
\\	土を掘って木を植えました。	土[つち]を 掘[ほ]って 木[き]を 植[う]えました。	つち を ほって き を うえました	
\\	を 掘[ほ]って 木[き]を 植[う]えました。			
\\	年月	年月[としつき]	としつき	
\\	あれから長い年月が経ちました。	あれから 長[なが]い 年月[としつき]が 経[た]ちました。	あれ から ながい としつき が たちました	
\\	あれから 長[なが]い
\\	が 経[た]ちました。			
\\	年月日	年月日[ねんがっぴ]	ねんがっぴ	
\\	申請の年月日を西暦で書いてください。	申請[しんせい]の 年月日[ねんがっぴ]を 西暦[せいれき]で 書[か]いてください。	しんせい の ねんがっぴ を せいれき で かいて ください	
\\	申請[しんせい]の
\\	を 西暦[せいれき]で 書[か]いてください。			
\\	年月	年月[ねんげつ]	ねんげつ	
\\	そのお寺は長い年月をかけて建てられた。	そのお 寺[てら]は 長[なが]い 年月[ねんげつ]をかけて 建[た]てられた。	その おてら は ながい ねんげつ を かけて たてられた	
\\	そのお 寺[てら]は 長[なが]い
\\	をかけて 建[た]てられた。			
\\	年々	年々[ねんねん]	ねんねん	
\\	東京の人口は年々増えています。	東京[とうきょう]の 人口[じんこう]は 年々[ねんねん] 増[ふ]えています。	とうきょう の じんこう は ねんねん ふえて います	
\\	東京[とうきょう]の 人口[じんこう]は
\\	増[ふ]えています。			
\\	来日	来日[らいにち]	らいにち	
\\	有名なバンドが来日していますね。	有名[ゆうめい]なバンドが 来日[らいにち]していますね。	ゆうめい な ばんど が らいにち して います ね	
\\	有名[ゆうめい]なバンドが
\\	していますね。			
\\	日帰り	日帰[ひがえ]り	ひがえり	
\\	私たちは日帰りで京都に行きました。	私[わたし]たちは 日帰[ひがえ]りで 京都[きょうと]に 行[い]きました。	わたしたち は ひがえり で きょうと に いきました	
\\	私[わたし]たちは
\\	で 京都[きょうと]に 行[い]きました。			
\\	大して	大[たい]して	たいして	
\\	彼は大して嬉しそうには見えなかったよね。	彼[かれ]は 大[たい]して 嬉[うれ]しそうには 見[み]えなかったよね。	かれ は たいして うれし そう に は みえなかった よ ね	
\\	彼[かれ]は
\\	嬉[うれ]しそうには 見[み]えなかったよね。			
\\	大金	大金[たいきん]	たいきん	
\\	このかばんには大金が入っています。	このかばんには 大金[たいきん]が 入[はい]っています。	この かばん に は たいきん が はいって います	
\\	このかばんには
\\	が 入[はい]っています。			
\\	大	大[だい]	だい	
\\	チーズケーキの大を一つ下さい。	チーズケーキの 大[だい]を 一[ひと]つ 下[くだ]さい。	ちーずけーき の だい を ひとつ ください	
\\	チーズケーキの
\\	を 一[ひと]つ 下[くだ]さい。			
\\	日中	日中[にっちゅう]	にっちゅう	
\\	日中はずっと海で泳いでいました。	日中[にっちゅう]はずっと 海[うみ]で 泳[およ]いでいました。	にっちゅう は ずっと うみ で およいで いました	
\\	はずっと 海[うみ]で 泳[およ]いでいました。			
\\	日中	日中[にっちゅう]	にっちゅう	
\\	日中貿易は急激に伸びているわね。	日中[にっちゅう] 貿易[ぼうえき]は 急激[きゅうげき]に 伸[の]びているわね。	にっちゅう ぼうえき は きゅうげき に のびて いる わ ね	
\\	貿易[ぼうえき]は 急激[きゅうげき]に 伸[の]びているわね。			
\\	年中	年中[ねんじゅう]	ねんじゅう	
\\	叔母は年中旅行しています。	叔母[おば]は 年中[ねんじゅう] 旅行[りょこう]しています。	おば は ねんじゅう りょこう して います	
\\	叔母[おば]は
\\	旅行[りょこう]しています。			
\\	大小	大小[だいしょう]	だいしょう	
\\	応募作品の大小は問いません。	応募作品[おうぼ さくひん]の 大小[だいしょう]は 問[と]いません。	おうぼ さくひん の だいしょう は といません	
\\	応募作品[おうぼ さくひん]の
\\	は 問[と]いません。			
\\	多少	多少[たしょう]	たしょう	
\\	このソフトには多少問題がある。	このソフトには 多少[たしょう] 問題[もんだい]がある。	この そふと に は たしょう もんだい が ある	
\\	このソフトには
\\	問題[もんだい]がある。			
\\	方々	方々[ほうぼう]	ほうぼう	
\\	彼の連絡先を方々に問い合わせたんだ。	彼[かれ]の 連絡先[れんらくさき]を 方々[ほうぼう]に 問[と]い 合[あ]わせたんだ。	かれ の れんらくさき を ほうぼう に といあわせた ん だ	
\\	彼[かれ]の 連絡先[れんらくさき]を
\\	に 問[と]い 合[あ]わせたんだ。			
\\	日の入り	日[ひ]の 入[い]り	ひのいり	
\\	今日の日の入りは午後6時でした。	今日[きょう]の 日[ひ]の 入[い]りは 午後6時[ごご 
\\	じ]でした。	きょう の ひのいり は ごご 
\\	じ でした	
\\	今日[きょう]の
\\	は 午後6時[ごご 
\\	じ]でした。			
\\	日の出	日[ひ]の 出[で]	ひので	
\\	日の出がとてもきれいですね。	日[ひ]の 出[で]がとてもきれいですね。	ひので が とても きれい です ね	
\\	がとてもきれいですね。			
\\	外す	外[はず]す	はずす	
\\	彼はメガネを外しました。	彼[かれ]はメガネを 外[はず]しました。	かれ は めがね を はずしました	
\\	彼[かれ]はメガネを
\\	外れる	外[はず]れる	はずれる	
\\	びんのふたが外れません。	びんのふたが 外[はず]れません。	びん の ふた が はずれません	
\\	びんのふたが
\\	外れ	外[はず]れ	はずれ	
\\	このくじは外れです。	このくじは 外[はず]れです。	この くじ は はずれ です	
\\	このくじは
\\	です。			
\\	本来	本来[ほんらい]	ほんらい	
\\	彼女はプレッシャーから解放されて本来の自分に戻ったな。	彼女[かのじょ]はプレッシャーから 解放[かいほう]されて 本来[ほんらい]の 自分[じぶん]に 戻[もど]ったな。	かのじょ は ぷれっしゃー から かいほう されて ほんらい の じぶん に もどった な	
\\	彼女[かのじょ]はプレッシャーから 解放[かいほう]されて
\\	の 自分[じぶん]に 戻[もど]ったな。			
\\	本人	本人[ほんにん]	ほんにん	
\\	それは本人に聞いてください。	それは 本人[ほんにん]に 聞[き]いてください。	それ は ほんにん に きいて ください	
\\	それは
\\	に 聞[き]いてください。			
\\	本年	本年[ほんねん]	ほんねん	
\\	会社の本年の目標が発表されたよ。	会社[かいしゃ]の 本年[ほんねん]の 目標[もくひょう]が 発表[はっぴょう]されたよ。	かいしゃ の ほんねん の もくひょう が はっぴょう された よ	
\\	会社[かいしゃ]の
\\	の 目標[もくひょう]が 発表[はっぴょう]されたよ。			
\\	本日	本日[ほんじつ]	ほんじつ	
\\	本日のランチはハンバーグでございます。	本日[ほんじつ]のランチはハンバーグでございます。	ほんじつ の らんち は はんばーぐ で ございます	
\\	のランチはハンバーグでございます。			
\\	大体	大体[だいたい]	だいたい	
\\	大体、初めから無理な計画だったのだ。	大体[だいたい]、 初[はじ]めから 無理[むり]な 計画[けいかく]だったのだ。	だいたい はじめ から むり な けいかく だった の だ	
\\	、 初[はじ]めから 無理[むり]な 計画[けいかく]だったのだ。			
\\	手入れ	手入[てい]れ	ていれ	
\\	母は庭の手入れをしています。	母[はは]は 庭[にわ]の 手入[てい]れをしています。	はは は にわ の ていれ を して います	
\\	母[はは]は 庭[にわ]の
\\	をしています。			
\\	手本	手本[てほん]	てほん	
\\	手本を見ながら習字をしました。	手本[てほん]を 見[み]ながら 習字[しゅうじ]をしました。	てほん を みながら しゅうじ を しました	
\\	を 見[み]ながら 習字[しゅうじ]をしました。			
\\	手足	手足[てあし]	てあし	
\\	あの人は手足が長い。	あの 人[ひと]は 手足[てあし]が 長[なが]い。	あの ひと は てあし が ながい	
\\	あの 人[ひと]は
\\	が 長[なが]い。			
\\	手元	手元[てもと]	てもと	
\\	説明書は手元にありますか。	説明書[せつめいしょ]は 手元[てもと]にありますか。	せつめいしょ は てもと に あります か	
\\	説明書[せつめいしょ]は
\\	にありますか。			
\\	天	天[てん]	てん	
\\	天から恵みの雨が降ったね。	天[てん]から 恵[めぐ]みの 雨[あめ]が 降[ふ]ったね。	てん から めぐみ の あめ が ふった ね	
\\	から 恵[めぐ]みの 雨[あめ]が 降[ふ]ったね。			
\\	本気	本気[ほんき]	ほんき	
\\	いや、僕は本気なんだ。	いや、 僕[ぼく]は 本気[ほんき]なんだ。	いや ぼく は ほんき な ん だ	
\\	いや、 僕[ぼく]は
\\	なんだ。			
\\	明日	明日[みょうにち]	みょうにち	
\\	明日、会議を開きます。	明日[みょうにち]、 会議[かいぎ]を 開[ひら]きます。	みょうにち かいぎ を ひらきます	
\\	、 会議[かいぎ]を 開[ひら]きます。			
\\	東西	東西[とうざい]	とうざい	
\\	東西に大きな道路が通っています。	東西[とうざい]に 大[おお]きな 道路[どうろ]が 通[とお]っています。	とうざい に おおき な どうろ が とおって います	
\\	に 大[おお]きな 道路[どうろ]が 通[とお]っています。			
\\	向ける	向[む]ける	むける	
\\	彼は上司に怒りの目を向けたんだよ。	彼[かれ]は 上司[じょうし]に 怒[いか]りの 目[め]を 向[む]けたんだよ。	かれ は じょうし に いかり の め を むけた ん だ よ	
\\	彼[かれ]は 上司[じょうし]に 怒[いか]りの 目[め]を
\\	んだよ。			
\\	向き	向[む]き	むき	
\\	花瓶の向きを変えたの。	花瓶[かびん]の 向[む]きを 変[か]えたの。	かびん の むき を かえた の	
\\	花瓶[かびん]の
\\	を 変[か]えたの。			
\\	向かい	向[む]かい	むかい	
\\	向かいの席が空いていますよ。	向[む]かいの 席[せき]が 空[あ]いていますよ。	むかい の せき が あいて います よ	
\\	の 席[せき]が 空[あ]いていますよ。			
\\	手間	手間[てま]	てま	
\\	これはとても手間のかかる料理です。	これはとても 手間[てま]のかかる 料理[りょうり]です。	これ は とても てま の かかる りょうり です	
\\	これはとても
\\	のかかる 料理[りょうり]です。			
\\	安っぽい	安[やす]っぽい	やすっぽい	
\\	そのシャツは安っぽいね。	そのシャツは 安[やす]っぽいね。	その しゃつ は やすっぽい ね	
\\	そのシャツは
\\	ね。			
\\	最も	最[もっと]も	もっとも	
\\	彼は世界で最も早い男です。	彼[かれ]は 世界[せかい]で 最[もっと]も 早[はや]い 男[おとこ]です。	かれ は せかい で もっとも はやい おとこ です	
\\	彼[かれ]は 世界[せかい]で
\\	早[はや]い 男[おとこ]です。			
\\	月初め	月初[つきはじ]め	つきはじめ	
\\	いつも月初めに彼と会います。	いつも 月初[つきはじ]めに 彼[かれ]と 会[あ]います。	いつも つきはじめ に かれ と あいます	
\\	いつも
\\	に 彼[かれ]と 会[あ]います。			
\\	手前	手前[てまえ]	てまえ	
\\	駅の手前に郵便局があります。	駅[えき]の 手前[てまえ]に 郵便局[ゆうびんきょく]があります。	えき の てまえ に ゆうびんきょく が あります	
\\	駅[えき]の
\\	に 郵便局[ゆうびんきょく]があります。			
\\	明朝	明朝[みょうちょう]	みょうちょう	
\\	明朝10時からまた会議です。	明朝[みょうちょう] 
\\	時[じ]からまた 会議[かいぎ]です。	みょうちょう 
\\	じ から また かいぎ です	
\\	時[じ]からまた 会議[かいぎ]です。			
\\	晩年	晩年[ばんねん]	ばんねん	
\\	彼は晩年を故郷で過ごしたんだ。	彼[かれ]は 晩年[ばんねん]を 故郷[こきょう]で 過[す]ごしたんだ。	かれ は ばんねん を こきょう で すごした ん だ	
\\	彼[かれ]は
\\	を 故郷[こきょう]で 過[す]ごしたんだ。			
\\	夜間	夜間[やかん]	やかん	
\\	夜間は裏口から入ってください。	夜間[やかん]は 裏口[うらぐち]から 入[はい]ってください。	やかん は うらぐち から はいって ください	
\\	は 裏口[うらぐち]から 入[はい]ってください。			
\\	夜空	夜空[よぞら]	よぞら	
\\	ふたりで夜空を見上げたの。	ふたりで 夜空[よぞら]を 見上[みあ]げたの。	ふたり で よぞら を みあげた の	
\\	ふたりで
\\	を 見上[みあ]げたの。			
\\	夜明け	夜明[よあ]け	よあけ	
\\	夜明けと共に目が覚めたんだ。	夜明[よあ]けと 共[とも]に 目[め]が 覚[さ]めたんだ。	よあけ と とも に め が さめた ん だ	
\\	と 共[とも]に 目[め]が 覚[さ]めたんだ。			
\\	月夜	月夜[つきよ]	つきよ	
\\	散歩にいい月夜ですね。	散歩[さんぽ]にいい 月夜[つきよ]ですね。	さんぽ に いい つきよ です ね	
\\	散歩[さんぽ]にいい
\\	ですね。			
\\	夜	夜[よ]	よ	
\\	あと1時間で夜が明けますね。	あと1 時間[じかん]で 夜[よ]が 明[あ]けますね。	あと 
\\	じかん で よ が あけます ね	
\\	あと1 時間[じかん]で
\\	が 明[あ]けますね。			
\\	夕日	夕日[ゆうひ]	ゆうひ	
\\	夕日が西の空に沈んだね。	夕日[ゆうひ]が 西[にし]の 空[そら]に 沈[しず]んだね。	ゆうひ が にし の そら に しずんだ ね	
\\	が 西[にし]の 空[そら]に 沈[しず]んだね。			
\\	月見	月見[つきみ]	つきみ	
\\	9月には月見を楽しみます。	
\\	月[がつ]には 月見[つきみ]を 楽[たの]しみます。	
\\	がつ に は つきみ を たのしみます	
\\	月[がつ]には
\\	を 楽[たの]しみます。			
\\	方言	方言[ほうげん]	ほうげん	
\\	彼は方言で話します。	彼[かれ]は 方言[ほうげん]で 話[はな]します。	かれ は ほうげん で はなします	
\\	彼[かれ]は
\\	で 話[はな]します。			
\\	文明	文明[ぶんめい]	ぶんめい	
\\	多くの文明は川の近くで始まった。	多[おお]くの 文明[ぶんめい]は 川[かわ]の 近[ちか]くで 始[はじ]まった。	おおく の ぶんめい は かわ の ちかく で はじまった	
\\	多[おお]くの
\\	は 川[かわ]の 近[ちか]くで 始[はじ]まった。			
\\	本文	本文[ほんぶん]	ほんぶん	
\\	本文をよく読んで答えてください。	本文[ほんぶん]をよく 読[よ]んで 答[こた]えてください。	ほんぶん を よく よんで こたえて ください	
\\	をよく 読[よ]んで 答[こた]えてください。			
\\	文	文[ぶん]	ぶん	
\\	この文は意味が分かりません。	この 文[ぶん]は 意味[いみ]が 分[わ]かりません。	この ぶん は いみ が わかりません	
\\	この
\\	は 意味[いみ]が 分[わ]かりません。			
\\	本社	本社[ほんしゃ]	ほんしゃ	
\\	今日は本社で会議があります。	今日[きょう]は 本社[ほんしゃ]で 会議[かいぎ]があります。	きょう は ほんしゃ で かいぎ が あります	
\\	今日[きょう]は
\\	で 会議[かいぎ]があります。			
\\	大工	大工[だいく]	だいく	
\\	私の父は大工です。	私[わたし]の 父[ちち]は 大工[だいく]です。	わたし の ちち は だいく です	
\\	私[わたし]の 父[ちち]は
\\	です。			
\\	場	場[ば]	ば	
\\	この場でお礼を言わせてください。	この 場[ば]でお 礼[れい]を 言[い]わせてください。	この ば で おれい を いわせて ください	
\\	この
\\	でお 礼[れい]を 言[い]わせてください。			
\\	地上	地上[ちじょう]	ちじょう	
\\	この電車は地上を走ります。	この 電車[でんしゃ]は 地上[ちじょう]を 走[はし]ります。	この でんしゃ は ちじょう を はしります	
\\	この 電車[でんしゃ]は
\\	を 走[はし]ります。			
\\	地	地[ち]	ち	
\\	彼はその地で残りの生涯を過ごしたんだ。	彼[かれ]はその 地[ち]で 残[のこ]りの 生涯[しょうがい]を 過[す]ごしたんだ。	かれ は その ち で のこり の しょうがい を すごした ん だ	
\\	彼[かれ]はその
\\	で 残[のこ]りの 生涯[しょうがい]を 過[す]ごしたんだ。			
\\	地下道	地下道[ちかどう]	ちかどう	
\\	地下道を通って行きましょう。	地下道[ちかどう]を 通[とお]って 行[い]きましょう。	ちかどう を とおって いきましょう	
\\	を 通[とお]って 行[い]きましょう。			
\\	地下	地下[ちか]	ちか	
\\	スタジオは地下にあります。	スタジオは 地下[ちか]にあります。	すたじお は ちか に あります	
\\	スタジオは
\\	にあります。			
\\	地方	地方[ちほう]	ちほう	
\\	この地方は漁業が盛んです。	この 地方[ちほう]は 漁業[ぎょぎょう]が 盛[さか]んです。	この ちほう は ぎょぎょう が さかん です	
\\	この
\\	は 漁業[ぎょぎょう]が 盛[さか]んです。			
\\	図る	図[はか]る	はかる	
\\	これからは経営の合理化を図りたいと思います。	これからは 経営[けいえい]の 合理化[ごうりか]を 図[はか]りたいと 思[おも]います。	これから は けいえい の ごうりか を はかりたい と おもいます	
\\	これからは 経営[けいえい]の 合理化[ごうりか]を
\\	と 思[おも]います。			
\\	図書	図書[としょ]	としょ	
\\	これは児童図書です。	これは 児童[じどう] 図書[としょ]です。	これ は じどうとしょ です	
\\	これは 児童[じどう]
\\	です。			
\\	度々	度々[たびたび]	たびたび	
\\	彼から度々メールが来ます。	彼[かれ]から 度々[たびたび]メールが 来[き]ます。	かれ から たびたび めーる が きます	
\\	彼[かれ]から
\\	メールが 来[き]ます。			
\\	毎度	毎度[まいど]	まいど	
\\	毎度ありがとうございます。	毎度[まいど]ありがとうございます。	まいど ありがとう ございます	
\\	ありがとうございます。			
\\	年長	年長[ねんちょう]	ねんちょう	
\\	彼がこのグループで一番年長です。	彼[かれ]がこのグループで 一番[いちばん] 年長[ねんちょう]です。	かれ が この ぐるーぷ で いちばん ねんちょう です	
\\	彼[かれ]がこのグループで 一番[いちばん]
\\	です。			
\\	広げる	広[ひろ]げる	ひろげる	
\\	電車の中では新聞を広げないで。	電車[でんしゃ]の 中[なか]では 新聞[しんぶん]を 広[ひろ]げないで。	でんしゃ の なか で は しんぶん を ひろげない で	
\\	電車[でんしゃ]の 中[なか]では 新聞[しんぶん]を
\\	広場	広場[ひろば]	ひろば	
\\	広場に子供が沢山集まっていたよ。	広場[ひろば]に 子供[こども]が 沢山集[たくさん あつ]まっていたよ。	ひろば に こども が たくさん あつまって いた よ	
\\	に 子供[こども]が 沢山集[たくさん あつ]まっていたよ。			
\\	広まる	広[ひろ]まる	ひろまる	
\\	その噂はすぐに広まったよ。	その 噂[うわさ]はすぐに 広[ひろ]まったよ。	その うわさ は すぐ に ひろまった よ	
\\	その 噂[うわさ]はすぐに
\\	よ。			
\\	広める	広[ひろ]める	ひろめる	
\\	誰が噂を広めたんだろう。	誰[だれ]が 噂[うわさ]を 広[ひろ]めたんだろう。	だれ が うわさ を ひろめた ん だろう	
\\	誰[だれ]が 噂[うわさ]を
\\	んだろう。			
\\	本部	本部[ほんぶ]	ほんぶ	
\\	その事件の直後、捜査本部が設置された。	その 事件[じけん]の 直後[ちょくご]、 捜査[そうさ] 本部[ほんぶ]が 設置[せっち]された。	その じけん の ちょくご そうさ ほんぶ が せっち された	
\\	その 事件[じけん]の 直後[ちょくご]、 捜査[そうさ]
\\	が 設置[せっち]された。			
\\	大部分	大部分[だいぶぶん]	だいぶぶん	
\\	絵の大部分が水に濡れてしまったな。	絵[え]の 大部分[だいぶぶん]が 水[みず]に 濡[ぬ]れてしまったな。	え の だいぶぶん が みず に ぬれて しまった な	
\\	絵[え]の
\\	が 水[みず]に 濡[ぬ]れてしまったな。			
\\	大国	大国[たいこく]	たいこく	
\\	その国は経済大国よ。	その 国[くに]は 経済[けいざい] 大国[たいこく]よ。	その くに は けいざい たいこく よ	
\\	その 国[くに]は 経済[けいざい]
\\	よ。			
\\	本国	本国[ほんごく]	ほんごく	
\\	彼女は本国に帰りました。	彼女[かのじょ]は 本国[ほんごく]に 帰[かえ]りました。	かのじょ は ほんごく に かえりました	
\\	彼女[かのじょ]は
\\	に 帰[かえ]りました。			
\\	天国	天国[てんごく]	てんごく	
\\	死んだら天国に行きたいです。	死[し]んだら 天国[てんごく]に 行[い]きたいです。	しんだら てんごく に いきたい です	
\\	死[し]んだら
\\	に 行[い]きたいです。			
\\	明白	明白[めいはく]	めいはく	
\\	彼が犯人なのは明白です。	彼[かれ]が 犯人[はんにん]なのは 明白[めいはく]です。	かれ が はんにん な の は めいはく です	
\\	彼[かれ]が 犯人[はんにん]なのは
\\	です。			
\\	日米	日米[にちべい]	にちべい	
\\	テレビで日米野球をやっていますよ。	テレビで 日米[にちべい] 野球[やきゅう]をやっていますよ。	てれび で にちべい やきゅう を やって います よ	
\\	テレビで
\\	野球[やきゅう]をやっていますよ。			
\\	味方	味方[みかた]	みかた	
\\	母はいつも私の味方です。	母[はは]はいつも 私[わたし]の 味方[みかた]です。	はは は いつも わたし の みかた です	
\\	母[はは]はいつも 私[わたし]の
\\	です。			
\\	年末	年末[ねんまつ]	ねんまつ	
\\	年末のセールはいつも込んでいるね。	年末[ねんまつ]のセールはいつも 込[こ]んでいるね。	ねんまつ の せーる は いつも こんで いる ね	
\\	のセールはいつも 込[こ]んでいるね。			
\\	末	末[まつ]	まつ	
\\	今月末にカナダに行きます。	今月[こんげつ] 末[まつ]にカナダに 行[い]きます。	こんげつ まつ に かなだ に いきます	
\\	今月[こんげつ]
\\	にカナダに 行[い]きます。			
\\	料金	料金[りょうきん]	りょうきん	
\\	まだ料金は払っていないけど。	まだ 料金[りょうきん]は 払[はら]っていないけど。	まだ りょうきん は はらって いない けど	
\\	まだ
\\	は 払[はら]っていないけど。			
\\	地理	地理[ちり]	ちり	
\\	彼は地理に詳しいの。	彼[かれ]は 地理[ちり]に 詳[くわ]しいの。	かれ は ちり に くわしい の	
\\	彼[かれ]は
\\	に 詳[くわ]しいの。			
\\	有力	有力[ゆうりょく]	ゆうりょく	
\\	あの都市はオリンピックの有力な候補地です。	あの 都市[とし]はオリンピックの 有力[ゆうりょく]な 候補地[こうほち]です。	あの とし は おりんぴっく の ゆうりょく な こうほち です	
\\	あの 都市[とし]はオリンピックの
\\	な 候補地[こうほち]です。			
\\	有する	有[ゆう]する	ゆうする	
\\	資格を有する人のみ応募できます。	資格[しかく]を 有[ゆう]する 人[ひと]のみ 応募[おうぼ]できます。	しかく を ゆうする ひと のみ おうぼ できます	
\\	資格[しかく]を
\\	人[ひと]のみ 応募[おうぼ]できます。			
\\	有料	有料[ゆうりょう]	ゆうりょう	
\\	このトイレは有料です。	このトイレは 有料[ゆうりょう]です。	この といれ は ゆうりょう です	
\\	このトイレは
\\	です。			
\\	大使	大使[たいし]	たいし	
\\	彼は昔、ドイツの大使でした。	彼[かれ]は 昔[むかし]、ドイツの 大使[たいし]でした。	かれ は むかし どいつ の たいし でした	
\\	彼[かれ]は 昔[むかし]、ドイツの
\\	でした。			
\\	安売り	安売[やすう]り	やすうり	
\\	あの店で野菜の安売りをしていましたよ。	あの 店[みせ]で 野菜[やさい]の 安売[やすう]りをしていましたよ。	あの みせ で やさい の やすうり を して いました よ	
\\	あの 店[みせ]で 野菜[やさい]の
\\	をしていましたよ。			
\\	売買	売買[ばいばい]	ばいばい	
\\	彼は不動産の売買をしています。	彼[かれ]は 不動産[ふどうさん]の 売買[ばいばい]をしています。	かれ は ふどうさん の ばいばい を して います	
\\	彼[かれ]は 不動産[ふどうさん]の
\\	をしています。			
\\	本店	本店[ほんてん]	ほんてん	
\\	ここはチェーン店の本店です。	ここはチェーン 店[てん]の 本店[ほんてん]です。	ここ は ちぇーんてん の ほんてん です	
\\	ここはチェーン 店[てん]の
\\	です。			
\\	日用品	日用品[にちようひん]	にちようひん	
\\	今日は日用品の買い物をした。	今日[きょう]は 日用品[にちようひん]の 買[か]い 物[もの]をした。	きょう は にちようひん の かいもの を した	
\\	今日[きょう]は
\\	の 買[か]い 物[もの]をした。			
\\	段	段[だん]	だん	
\\	この階段は18段ありますね。	この 階[かい] 段[だん]は18 段[だん]ありますね。	この かいだん は 
\\	だん あります ね	
\\	この 階[かい]
\\	は18 段[だん]ありますね。			
\\	地価	地価[ちか]	ちか	
\\	東京の地価は上がり続けているんだ。	東京[とうきょう]の 地価[ちか]は 上[あ]がり 続[つづ]けているんだ。	とうきょう の ちか は あがりつづけて いる ん だ	
\\	東京[とうきょう]の
\\	は 上[あ]がり 続[つづ]けているんだ。			
\\	情けない	情[なさ]けない	なさけない	
\\	こんなことも知らないとは情けない。	こんなことも 知[し]らないとは 情[なさ]けない。	こんな こと も しらない と は なさけない	
\\	こんなことも 知[し]らないとは
\\	報道	報道[ほうどう]	ほうどう	
\\	夜中もテレビで台風の報道をしていた。	夜中[よなか]もテレビで 台風[たいふう]の 報道[ほうどう]をしていた。	よなか も てれび で たいふう の ほうどう を して いた	
\\	夜中[よなか]もテレビで 台風[たいふう]の
\\	をしていた。			
\\	古本	古本[ふるほん]	ふるほん	
\\	おととい古本を3冊買いました。	おととい 古本[ふるほん]を3 冊買[さつ か]いました。	おととい ふるほん を 
\\	さつ かいました	
\\	おととい
\\	を3 冊買[さつ か]いました。			
\\	昔	昔[むかし]	むかし	
\\	彼は昔は貧乏だった。	彼[かれ]は 昔[むかし]は 貧乏[びんぼう]だった。	かれ は むかし は びんぼう だった	
\\	彼[かれ]は
\\	は 貧乏[びんぼう]だった。			
\\	悪口	悪口[わるくち]	わるくち	
\\	彼は決して人の悪口を言わないの。	彼[かれ]は 決[けっ]して 人[ひと]の 悪口[わるくち]を 言[い]わないの。	かれ は けっして ひと の わるくち を いわない の	
\\	彼[かれ]は 決[けっ]して 人[ひと]の
\\	を 言[い]わないの。			
\\	忘年会	忘年会[ぼうねんかい]	ぼうねんかい	
\\	明日は会社の忘年会があります。	明日[あした]は 会社[かいしゃ]の 忘年会[ぼうねんかい]があります。	あした は かいしゃ の ぼうねんかい が あります	
\\	明日[あした]は 会社[かいしゃ]の
\\	があります。			
\\	度忘れ	度忘[どわす]れ	どわすれ	
\\	彼の名前を度忘れしたぞ。	彼[かれ]の 名前[なまえ]を 度忘[どわす]れしたぞ。	かれ の なまえ を どわすれ した ぞ	
\\	彼[かれ]の 名前[なまえ]を
\\	したぞ。			
\\	未知	未知[みち]	みち	
\\	ここからは未知の領域です。	ここからは 未知[みち]の 領域[りょういき]です。	ここ から は みち の りょういき です	
\\	ここからは
\\	の 領域[りょういき]です。			
\\	天才	天才[てんさい]	てんさい	
\\	彼は笑いの天才だね。	彼[かれ]は 笑[わら]いの 天才[てんさい]だね。	かれ は わらい の てんさい だ ね	
\\	彼[かれ]は 笑[わら]いの
\\	だね。			
\\	本能	本能[ほんのう]	ほんのう	
\\	動物は本能のまま動くね。	動物[どうぶつ]は 本能[ほんのう]のまま 動[うご]くね。	どうぶつ は ほんのう の まま うごく ね	
\\	動物[どうぶつ]は
\\	のまま 動[うご]くね。			
\\	有能	有能[ゆうのう]	ゆうのう	
\\	彼女はとても有能な部下です。	彼女[かのじょ]はとても 有能[ゆうのう]な 部下[ぶか]です。	かのじょ は とても ゆうのう な ぶか です	
\\	彼女[かのじょ]はとても
\\	な 部下[ぶか]です。			
\\	大便	大便[だいべん]	だいべん	
\\	病院で大便の検査をした。	病院[びょういん]で 大便[だいべん]の 検査[けんさ]をした。	びょういん で だいべん の けんさ を した	
\\	病院[びょういん]で
\\	の 検査[けんさ]をした。			
\\	所々	所々[ところどころ]	ところどころ	
\\	この本はページが所々破れているね。	この 本[ほん]はページが 所々[ところどころ] 破[やぶ]れているね。	この ほん は ぺーじ が ところどころ やぶれて いる ね	
\\	この 本[ほん]はページが
\\	破[やぶ]れているね。			
\\	名	名[な]	な	
\\	彼は名の通った会社に就職しました。	彼[かれ]は 名[な]の 通[とお]った 会社[かいしゃ]に 就職[しゅうしょく]しました。	かれ は な の とおった かいしゃ に しゅうしょく しました	
\\	彼[かれ]は
\\	の 通[とお]った 会社[かいしゃ]に 就職[しゅうしょく]しました。			
\\	名人	名人[めいじん]	めいじん	
\\	彼は釣りの名人です。	彼[かれ]は 釣[つ]りの 名人[めいじん]です。	かれ は つり の めいじん です	
\\	彼[かれ]は 釣[つ]りの
\\	です。			
\\	地名	地名[ちめい]	ちめい	
\\	その地名は聞いたことがないなあ。	その 地名[ちめい]は 聞[き]いたことがないなあ。	その ちめい は きいた こと が ない なあ	
\\	その
\\	は 聞[き]いたことがないなあ。			
\\	名所	名所[めいしょ]	めいしょ	
\\	ここは桜の名所です。	ここは 桜[さくら]の 名所[めいしょ]です。	ここ は さくら の めいしょ です	
\\	ここは 桜[さくら]の
\\	です。			
\\	村	村[むら]	むら	
\\	私は隣の村から来ました。	私[わたし]は 隣[となり]の 村[むら]から 来[き]ました。	わたし は となり の むら から きました	
\\	私[わたし]は 隣[となり]の
\\	から 来[き]ました。			
\\	様子	様子[ようす]	ようす	
\\	彼女の様子を見てきます。	彼女[かのじょ]の 様子[ようす]を 見[み]てきます。	かのじょ の ようす を みて きます	
\\	彼女[かのじょ]の
\\	を 見[み]てきます。			
\\	本物	本物[ほんもの]	ほんもの	
\\	これは本物のダイヤモンドです。	これは 本物[ほんもの]のダイヤモンドです。	これ は ほんもの の だいやもんど です	
\\	これは
\\	のダイヤモンドです。			
\\	名物	名物[めいぶつ]	めいぶつ	
\\	この町の名物はぶどうです。	この 町[まち]の 名物[めいぶつ]はぶどうです。	この まち の めいぶつ は ぶどう です	
\\	この 町[まち]の
\\	はぶどうです。			
\\	手軽	手軽[てがる]	てがる	
\\	手軽に作れる料理を教えてください。	手軽[てがる]に 作[つく]れる 料理[りょうり]を 教[おし]えてください。	てがる に つくれる りょうり を おしえて ください	
\\	に 作[つく]れる 料理[りょうり]を 教[おし]えてください。			
\\	多量	多量[たりょう]	たりょう	
\\	その事故で多量のガス漏れがあったね。	その 事故[じこ]で 多量[たりょう]のガス 漏[も]れがあったね。	その じこ で たりょう の がすもれ が あった ね	
\\	その 事故[じこ]で
\\	のガス 漏[も]れがあったね。			
\\	大量	大量[たいりょう]	たいりょう	
\\	昨日大量のゴミが出たの。	昨日[きのう] 大量[たいりょう]のゴミが 出[で]たの。	きのう たいりょう の ごみ が でた の	
\\	昨日[きのう]
\\	のゴミが 出[で]たの。			
\\	年配	年配[ねんぱい]	ねんぱい	
\\	年配の人に席を譲りました。	年配[ねんぱい]の 人[ひと]に 席[せき]を 譲[ゆず]りました。	ねんぱい の ひと に せき を ゆずりました	
\\	の 人[ひと]に 席[せき]を 譲[ゆず]りました。			
\\	取れる	取[と]れる	とれる	
\\	このナスは畑で取れたばかりです。	このナスは 畑[はたけ]で 取[と]れたばかりです。	この なす は はたけ で とれた ばかり です	
\\	このナスは 畑[はたけ]で
\\	ばかりです。			
\\	取消し	取消[とりけ]し	とりけし	
\\	彼は免許取消しの処分を受けたよ。	彼[かれ]は 免許[めんきょ] 取消[とりけ]しの 処分[しょぶん]を 受[う]けたよ。	かれ は めんきょ とりけし の しょぶん を うけた よ	
\\	彼[かれ]は 免許[めんきょ]
\\	の 処分[しょぶん]を 受[う]けたよ。			
\\	届け	届[とど]け	とどけ	
\\	郵便局に引っ越しの届けを出したよ。	郵便局[ゆうびんきょく]に 引[ひ]っ 越[こ]しの 届[とど]けを 出[だ]したよ。	ゆうびんきょく に ひっこし の とどけ を だした よ	
\\	郵便局[ゆうびんきょく]に 引[ひ]っ 越[こ]しの
\\	を 出[だ]したよ。			
\\	待ち合わせ	待[ま]ち 合[あ]わせ	まちあわせ	
\\	明日の待ち合わせは11時です。	明日[あす]の 待[ま]ち 合[あ]わせは11 時[じ]です。	あす の まちあわせ は 
\\	じ です	
\\	明日[あす]の
\\	は11 時[じ]です。			
\\	待ち遠しい	待[ま]ち 遠[どお]しい	まちどおしい	
\\	入学式が待ち遠しいです。	入学式[にゅうがくしき]が 待[ま]ち 遠[どお]しいです。	にゅうがくしき が まちどおしい です	
\\	入学式[にゅうがくしき]が
\\	です。			
\\	待ち合わせる	待[ま]ち 合[あ]わせる	まちあわせる	
\\	彼と新宿で待ち合わせました。	彼[かれ]と 新宿[しんじゅく]で 待[ま]ち 合[あ]わせました。	かれ と しんじゅく で まちあわせました	
\\	彼[かれ]と 新宿[しんじゅく]で
\\	持ち物	持[も]ち 物[もの]	もちもの	
\\	持ち物には名前を書いてください。	持[も]ち 物[もの]には 名前[なまえ]を 書[か]いてください。	もちもの に は なまえ を かいて ください	
\\	には 名前[なまえ]を 書[か]いてください。			
\\	投書	投書[とうしょ]	とうしょ	
\\	その事件について新聞に投書したんだ。	その 事件[じけん]について 新聞[しんぶん]に 投書[とうしょ]したんだ。	その じけん に ついて しんぶん に とうしょ した ん だ	
\\	その 事件[じけん]について 新聞[しんぶん]に
\\	したんだ。			
\\	役所	役所[やくしょ]	やくしょ	
\\	役所で住民票をもらってきた。	役所[やくしょ]で 住民票[じゅうみんひょう]をもらってきた。	やくしょ で じゅうみんひょう を もらって きた	
\\	で 住民票[じゅうみんひょう]をもらってきた。			
\\	役人	役人[やくにん]	やくにん	
\\	叔父は役人として30年働きました。	叔父[おじ]は 役人[やくにん]として30 年働[ねん はたら]きました。	おじ は やくにん と して 
\\	ねん はたらきました	
\\	叔父[おじ]は
\\	として30 年働[ねん はたら]きました。			
\\	役目	役目[やくめ]	やくめ	
\\	私は無事に役目を終えたよ。	私[わたし]は 無事[ぶじ]に 役目[やくめ]を 終[お]えたよ。	わたし は ぶじ に やくめ を おえた よ	
\\	私[わたし]は 無事[ぶじ]に
\\	を 終[お]えたよ。			
\\	役	役[やく]	やく	
\\	彼女は弁護士の役を演じているんだ。	彼女[かのじょ]は 弁護士[べんごし]の 役[やく]を 演[えん]じているんだ。	かのじょ は べんごし の やく を えんじて いる ん だ	
\\	彼女[かのじょ]は 弁護士[べんごし]の
\\	を 演[えん]じているんだ。			
\\	土産	土産[みやげ]	みやげ	
\\	土産に日本酒をもらった。	土産[みやげ]に 日本酒[にほんしゅ]をもらった。	みやげ に にほんしゅ を もらった	
\\	に 日本酒[にほんしゅ]をもらった。			
\\	徒歩	徒歩[とほ]	とほ	
\\	家から駅まで徒歩3分です。	家[いえ]から 駅[えき]まで 徒歩[とほ]3 分[ぷん]です。	いえ から えき まで とほ 
\\	ぷん です	
\\	家[いえ]から 駅[えき]まで
\\	分[ぷん]です。			
\\	学ぶ	学[まな]ぶ	まなぶ	
\\	私は哲学を学んでいます。	私[わたし]は 哲学[てつがく]を 学[まな]んでいます。	わたし は てつがく を まなんで います	
\\	私[わたし]は 哲学[てつがく]を
\\	文学	文学[ぶんがく]	ぶんがく	
\\	彼女は文学に興味を持っているのよ。	彼女[かのじょ]は 文学[ぶんがく]に 興味[きょうみ]を 持[も]っているのよ。	かのじょ は ぶんがく に きょうみ を もっている の よ 。	
\\	彼女[かのじょ]は
\\	に 興味[きょうみ]を 持[も]っているのよ。			
\\	強まる	強[つよ]まる	つよまる	
\\	雨はだんだん強まります。	雨[あめ]はだんだん 強[つよ]まります。	あめ は だんだん つよまります	
\\	雨[あめ]はだんだん
\\	強める	強[つよ]める	つよめる	
\\	火を強めてください。	火[ひ]を 強[つよ]めてください。	ひ を つよめて ください	
\\	火[ひ]を
\\	ください。			
\\	強気	強気[つよき]	つよき	
\\	彼女は強気な女性ですね。	彼女[かのじょ]は 強気[つよき]な 女性[じょせい]ですね。	かのじょ は つよき な じょせい です ね	
\\	彼女[かのじょ]は
\\	な 女性[じょせい]ですね。			
\\	弱まる	弱[よわ]まる	よわまる	
\\	夜になって風が弱まったね。	夜[よる]になって 風[かぜ]が 弱[よわ]まったね。	よる に なって かぜ が よわまった ね	
\\	夜[よる]になって 風[かぜ]が
\\	ね。			
\\	弱める	弱[よわ]める	よわめる	
\\	火を弱めてください。	火[ひ]を 弱[よわ]めてください。	ひ を よわめて ください	
\\	火[ひ]を
\\	ください。			
\\	弱る	弱[よわ]る	よわる	
\\	彼は病気で弱っているんだ。	彼[かれ]は 病気[びょうき]で 弱[よわ]っているんだ。	かれ は びょうき で よわって いる ん だ	
\\	彼[かれ]は 病気[びょうき]で
\\	んだ。			
\\	弱み	弱[よわ]み	よわみ	
\\	彼は私の弱みを握っているんだ。	彼[かれ]は 私[わたし]の 弱[よわ]みを 握[にぎ]っているんだ。	かれ は わたし の よわみ を にぎって いる ん だ	
\\	彼[かれ]は 私[わたし]の
\\	を 握[にぎ]っているんだ。			
\\	弱気	弱気[よわき]	よわき	
\\	彼は少し弱気になっています。	彼[かれ]は 少[すこ]し 弱気[よわき]になっています。	かれ は すこし よわき に なって います	
\\	彼[かれ]は 少[すこ]し
\\	になっています。			
\\	引き受ける	引[ひ]き 受[う]ける	ひきうける	
\\	新しい仕事を引き受けたよ。	新[あたら]しい 仕事[しごと]を 引[ひ]き 受[う]けたよ。	あたらしい しごと を ひきうけた よ	
\\	新[あたら]しい 仕事[しごと]を
\\	よ。			
\\	引き上げる	引[ひ]き 上[あ]げる	ひきあげる	
\\	沈んだ船を引き上げたんだ。	沈[しず]んだ 船[ふね]を 引[ひ]き 上[あ]げたんだ。	しずんだ ふね を ひきあげた ん だ	
\\	沈[しず]んだ 船[ふね]を
\\	んだ。			
\\	引き出す	引[ひ]き 出[だ]す	ひきだす	
\\	先生が私の能力を引き出してくれました。	先生[せんせい]が 私[わたし]の 能力[のうりょく]を 引[ひ]き 出[だ]してくれました。	せんせい が わたし の のうりょく を ひきだしてくれました	
\\	先生[せんせい]が 私[わたし]の 能力[のうりょく]を
\\	引き取る	引[ひ]き 取[と]る	ひきとる	
\\	彼女は息子を引き取ったの。	彼女[かのじょ]は 息子[むすこ]を 引[ひ]き 取[と]ったの。	かのじょ は むすこ を ひきとった の	
\\	彼女[かのじょ]は 息子[むすこ]を
\\	の。			
\\	引きずる	引[ひ]きずる	ひきずる	
\\	彼はまだ失恋を引きずっています。	彼[かれ]はまだ 失恋[しつれん]を 引[ひ]きずっています。	かれ は まだ しつれん を ひきずって います	
\\	彼[かれ]はまだ 失恋[しつれん]を
\\	引き分け	引[ひ]き 分[わ]け	ひきわけ	
\\	この勝負は引き分けです。	この 勝負[しょうぶ]は 引[ひ]き 分[わ]けです。	この しょうぶ は ひきわけ です	
\\	この 勝負[しょうぶ]は
\\	です。			
\\	引き止める	引[ひ]き 止[と]める	ひきとめる	
\\	帰ろうとする友達を引き止めたんだ。	帰[かえ]ろうとする 友達[ともだち]を 引[ひ]き 止[と]めたんだ。	かえろう と する ともだち を ひきとめた ん だ	
\\	帰[かえ]ろうとする 友達[ともだち]を
\\	んだ。			
\\	慣れ	慣[な]れ	なれ	
\\	仕事には慣れも必要です。	仕事[しごと]には 慣[な]れも 必要[ひつよう]です。	しごと に は なれ も ひつよう です	
\\	仕事[しごと]には
\\	も 必要[ひつよう]です。			
\\	慣らす	慣[な]らす	ならす	
\\	水の温度に体を慣らしてから、潜ったほうがいいぞ。	水[みず]の 温度[おんど]に 体[からだ]を 慣[な]らしてから、 潜[もぐ]ったほうがいいぞ。	みず の おんど に からだ を ならして から もぐった ほう が いい ぞ	
\\	水[みず]の 温度[おんど]に 体[からだ]を
\\	から、 潜[もぐ]ったほうがいいぞ。			
\\	品質	品質[ひんしつ]	ひんしつ	
\\	このメーカーの製品は高品質だわね。	このメーカーの 製品[せいひん]は 高[こう] 品質[ひんしつ]だわね。	この めーかー の せいひん は こうひんしつ だ わ ね	
\\	このメーカーの 製品[せいひん]は 高[こう]
\\	だわね。			
\\	本質	本質[ほんしつ]	ほんしつ	
\\	彼は仕事の本質をよく理解しているわね。	彼[かれ]は 仕事[しごと]の 本質[ほんしつ]をよく 理解[りかい]しているわね。	かれ は しごと の ほんしつ を よく りかい して いる わ ね	
\\	彼[かれ]は 仕事[しごと]の
\\	をよく 理解[りかい]しているわね。			
\\	問う	問[と]う	とう	
\\	応募者の年齢は問いません。	応募者[おうぼしゃ]の 年齢[ねんれい]は 問[と]いません。	おうぼしゃ の ねんれい は といません	
\\	応募者[おうぼしゃ]の 年齢[ねんれい]は
\\	問い合わせる	問[と]い 合[あ]わせる	といあわせる	
\\	保険会社に問い合わせます。	保険会社[ほけん がいしゃ]に 問[と]い 合[あ]わせます。	ほけん がいしゃ に といあわせます	
\\	保険会社[ほけん がいしゃ]に
\\	問い	問[と]い	とい	
\\	この問いに答えられますか。	この 問[と]いに 答[こた]えられますか。	この とい に こたえられます か	
\\	この
\\	に 答[こた]えられますか。			
\\	問屋	問屋[とんや]	とんや	
\\	この街には家具の問屋がたくさんあります。	この 街[まち]には 家具[かぐ]の 問屋[とんや]がたくさんあります。	この まち に は かぐ の とんや が たくさん あります	
\\	この 街[まち]には 家具[かぐ]の
\\	がたくさんあります。			
\\	地点	地点[ちてん]	ちてん	
\\	もうすぐ目標の地点に到達します。	もうすぐ 目標[もくひょう]の 地点[ちてん]に 到達[とうたつ]します。	もうすぐ もくひょう の ちてん に とうたつ します	
\\	もうすぐ 目標[もくひょう]の
\\	に 到達[とうたつ]します。			
\\	多数	多数[たすう]	たすう	
\\	その仕事に多数の応募があったよ。	その 仕事[しごと]に 多数[たすう]の 応募[おうぼ]があったよ。	その しごと に たすう の おうぼ が あった よ	
\\	その 仕事[しごと]に
\\	の 応募[おうぼ]があったよ。			
\\	日数	日数[にっすう]	にっすう	
\\	今月は出勤日数が多いです。	今月[こんげつ]は 出勤[しゅっきん] 日数[にっすう]が 多[おお]いです。	こんげつ は しゅっきん にっすう が おおい です	
\\	今月[こんげつ]は 出勤[しゅっきん]
\\	が 多[おお]いです。			
\\	手数	手数[てすう]	てすう	
\\	お手数ですがよろしくお願いします。	お 手数[てすう]ですがよろしくお 願[ねが]いします。	おてすう です が よろしく おねがい します	
\\	お
\\	ですがよろしくお 願[ねが]いします。			
\\	回り	回[まわ]り	まわり	
\\	先生の周りに集まってください。	先生[せんせい]の 周[まわ]りに 集[あつ]まってください。	せんせい の まわり に あつまって ください 。	
\\	先生[せんせい]の
\\	に 集[あつ]まってください。			
\\	回り道	回[まわ]り 道[みち]	まわりみち	
\\	今日は回り道して帰ろう。	今日[きょう]は 回[まわ]り 道[みち]して 帰[かえ]ろう。	きょう は まわりみち して かえろう	
\\	今日[きょう]は
\\	して 帰[かえ]ろう。			
\\	枚数	枚数[まいすう]	まいすう	
\\	コピーの枚数を数えてください。	コピーの 枚数[まいすう]を 数[かぞ]えてください。	こぴー の まいすう を かぞえて ください	
\\	コピーの
\\	を 数[かぞ]えてください。			
\\	当初	当初[とうしょ]	とうしょ	
\\	当初の計画ではもっと早く終わるはずでした。	当初[とうしょ]の 計画[けいかく]ではもっと 早[はや]く 終[お]わるはずでした。	とうしょ の けいかく で は もっと はやく おわる はず でした	
\\	の 計画[けいかく]ではもっと 早[はや]く 終[お]わるはずでした。			
\\	当局	当局[とうきょく]	とうきょく	
\\	その事件については当局が調査しています。	その 事件[じけん]については 当局[とうきょく]が 調査[ちょうさ]しています。	その じけん に ついて は とうきょく が ちょうさ して います	
\\	その 事件[じけん]については
\\	が 調査[ちょうさ]しています。			
\\	当日	当日[とうじつ]	とうじつ	
\\	入場券は当日でも買えますよ。	入場券[にゅうじょうけん]は 当日[とうじつ]でも 買[か]えますよ。	にゅうじょうけん は とうじつ で も かえます よ	
\\	入場券[にゅうじょうけん]は
\\	でも 買[か]えますよ。			
\\	手当て	手当[てあ]て	てあて	
\\	彼女は急いで怪我の手当てをしたよ。	彼女[かのじょ]は 急[いそ]いで 怪我[けが]の 手当[てあ]てをしたよ。	かのじょ は いそいで けが の てあて を した よ	
\\	彼女[かのじょ]は 急[いそ]いで 怪我[けが]の
\\	をしたよ。			
\\	当分	当分[とうぶん]	とうぶん	
\\	彼女は当分学校を休むそうです。	彼女[かのじょ]は 当分[とうぶん] 学校[がっこう]を 休[やす]むそうです。	かのじょ は とうぶん がっこう を やすむ そう です	
\\	彼女[かのじょ]は
\\	学校[がっこう]を 休[やす]むそうです。			
\\	当人	当人[とうにん]	とうにん	
\\	当人は意外に平気なようね。	当人[とうにん]は 意外[いがい]に 平気[へいき]なようね。	とうにん は いがい に へいき なよう ね 。	
\\	は 意外[いがい]に 平気[へいき]なようね。			
\\	当番	当番[とうばん]	とうばん	
\\	今日は私が掃除の当番です。	今日[きょう]は 私[わたし]が 掃除[そうじ]の 当番[とうばん]です。	きょう は わたし が そうじ の とうばん です	
\\	今日[きょう]は 私[わたし]が 掃除[そうじ]の
\\	です。			
\\	日当たり	日当[ひあ]たり	ひあたり	
\\	この部屋は日当たりがいい。	この 部屋[へや]は 日当[ひあ]たりがいい。	この へや は ひあたり が いい	
\\	この 部屋[へや]は
\\	がいい。			
\\	天然	天然[てんねん]	てんねん	
\\	ここは天然の温泉です。	ここは 天然[てんねん]の 温泉[おんせん]です。	ここ は てんねん の おんせん です	
\\	ここは
\\	の 温泉[おんせん]です。			
\\	文法	文法[ぶんぽう]	ぶんぽう	
\\	今日は英語の文法を勉強します。	今日[きょう]は 英語[えいご]の 文法[ぶんぽう]を 勉強[べんきょう]します。	きょう は えいご の ぶんぽう を べんきょう します	
\\	今日[きょう]は 英語[えいご]の
\\	を 勉強[べんきょう]します。			
\\	法学部	法学部[ほうがくぶ]	ほうがくぶ	
\\	彼女は法学部の学生です。	彼女[かのじょ]は 法学部[ほうがくぶ]の 学生[がくせい]です。	かのじょ は ほうがくぶ の がくせい です	
\\	彼女[かのじょ]は
\\	の 学生[がくせい]です。			
\\	法	法[ほう]	ほう	
\\	国民は法に従わなければならないよ。	国民[こくみん]は 法[ほう]に 従[したが]わなければならないよ。	こくみん は ほう に したがわなければ ならない よ	
\\	国民[こくみん]は
\\	に 従[したが]わなければならないよ。			
\\	法則	法則[ほうそく]	ほうそく	
\\	勝利するには法則があるそうだ。	勝利[しょうり]するには 法則[ほうそく]があるそうだ。	しょうり する に は ほうそく が ある そう だ	
\\	勝利[しょうり]するには
\\	があるそうだ。			
\\	有利	有利[ゆうり]	ゆうり	
\\	資格があると就職に有利です。	資格[しかく]があると 就職[しゅうしょく]に 有利[ゆうり]です。	しかく が ある と しゅうしょく に ゆうり です	
\\	資格[しかく]があると 就職[しゅうしょく]に
\\	です。			
\\	左利き	左利[ひだりき]き	ひだりきき	
\\	私の息子は左利きです。	私[わたし]の 息子[むすこ]は 左利[ひだりき]きです。	わたし の むすこ は ひだりきき です	
\\	私[わたし]の 息子[むすこ]は
\\	です。			
\\	有益	有益[ゆうえき]	ゆうえき	
\\	昨日の話し合いは有益でした。	昨日[きのう]の 話[はな]し 合[あ]いは 有益[ゆうえき]でした。	きのう の はなしあい は ゆうえき でした	
\\	昨日[きのう]の 話[はな]し 合[あ]いは
\\	でした。			
\\	年収	年収[ねんしゅう]	ねんしゅう	
\\	税金の額は年収によって変わります。	税金[ぜいきん]の 額[がく]は 年収[ねんしゅう]によって 変[か]わります。	ぜいきん の がく は ねんしゅう に よって かわります	
\\	税金[ぜいきん]の 額[がく]は
\\	によって 変[か]わります。			
\\	木造	木造[もくぞう]	もくぞう	
\\	隣に木造の家が建ったね。	隣[となり]に 木造[もくぞう]の 家[いえ]が 建[た]ったね。	となり に もくぞう の いえ が たった ね	
\\	隣[となり]に
\\	の 家[いえ]が 建[た]ったね。			
\\	必然	必然[ひつぜん]	ひつぜん	
\\	私と彼が出会ったのは必然だったの。	私[わたし]と 彼[かれ]が 出会[であ]ったのは 必然[ひつぜん]だったの。	わたし と かれ が であった の は ひつぜん だった の	
\\	私[わたし]と 彼[かれ]が 出会[であ]ったのは
\\	だったの。			
\\	求める	求[もと]める	もとめる	
\\	子供は親の愛を求めます。	子供[こども]は 親[おや]の 愛[あい]を 求[もと]めます。	こども は おや の あい を もとめます	
\\	子供[こども]は 親[おや]の 愛[あい]を
\\	日差し	日差[ひざ]し	ひざし	
\\	今日は日差しが強いですね。	今日[きょう]は 日差[ひざ]しが 強[つよ]いですね。	きょう は ひざし が つよい です ね	
\\	今日[きょう]は
\\	が 強[つよ]いですね。			
\\	役割	役割[やくわり]	やくわり	
\\	みんなで役割を分担しましょう。	みんなで 役割[やくわり]を 分担[ぶんたん]しましょう。	みんな で やくわり を ぶんたん しましょう	
\\	みんなで
\\	を 分担[ぶんたん]しましょう。			
\\	残り	残[のこ]り	のこり	
\\	仕事の残りは家でします。	仕事[しごと]の 残[のこ]りは 家[いえ]でします。	しごと の のこり は いえ で します	
\\	仕事[しごと]の
\\	は 家[いえ]でします。			
\\	残らず	残[のこ]らず	のこらず	
\\	ゴミを残らず拾ったよ。	ゴミを 残[のこ]らず 拾[ひろ]ったよ。	ごみ を のこらず ひろった よ	
\\	ゴミを
\\	拾[ひろ]ったよ。			
\\	払い	払[はら]い	はらい	
\\	飲み屋の払いがたまっているんだ。	飲[の]み 屋[や]の 払[はら]いがたまっているんだ。	のみや の はらい が たまって いる ん だ	
\\	飲[の]み 屋[や]の
\\	がたまっているんだ。			
\\	引き返す	引[ひ]き 返[かえ]す	ひきかえす	
\\	雨が強かったので引き返したよ。	雨[あめ]が 強[つよ]かったので 引[ひ]き 返[かえ]したよ。	あめ が つよかった の で ひきかえした よ	
\\	雨[あめ]が 強[つよ]かったので
\\	よ。			
\\	持ち込む	持[も]ち 込[こ]む	もちこむ	
\\	機内に荷物を持ち込んだの。	機内[きない]に 荷物[にもつ]を 持[も]ち 込[こ]んだの。	きない に にもつ を もちこんだ の	
\\	機内[きない]に 荷物[にもつ]を
\\	の。			
\\	払い込む	払[はら]い 込[こ]む	はらいこむ	
\\	授業料を学校に払い込みました。	授業料[じゅぎょうりょう]を 学校[がっこう]に 払[はら]い 込[こ]みました。	じゅぎょうりょう を がっこう に はらいこみました	
\\	授業料[じゅぎょうりょう]を 学校[がっこう]に
\\	引っ込む	引[ひ]っ 込[こ]む	ひっこむ	
\\	ダイエットをしてお腹が引っ込みました。	ダイエットをしてお 腹[なか]が 引[ひ]っ 込[こ]みました。	だいえっと を して おなか が ひっこみました	
\\	ダイエットをしてお 腹[なか]が
\\	有限	有限[ゆうげん]	ゆうげん	
\\	宇宙は有限だと思いますか。	宇宙[うちゅう]は 有限[ゆうげん]だと 思[おも]いますか。	うちゅう は ゆうげん だ と おもいます か	
\\	宇宙[うちゅう]は
\\	だと 思[おも]いますか。			
\\	年代	年代[ねんだい]	ねんだい	
\\	私と彼は同じ年代です。	私[わたし]と 彼[かれ]は 同[おな]じ 年代[ねんだい]です。	わたし と かれ は おなじ ねんだい です	
\\	私[わたし]と 彼[かれ]は 同[おな]じ
\\	です。			
\\	指差す	指差[ゆびさ]す	ゆびさす	
\\	みんなが彼の指差す方を見たんだ。	みんなが 彼[かれ]の 指差[ゆびさ]す 方[ほう]を 見[み]たんだ。	みんな が かれ の ゆびさす ほう を みたんだ	
\\	みんなが 彼[かれ]の
\\	方[ほう]を 見[み]たんだ。			
\\	定年	定年[ていねん]	ていねん	
\\	彼は来年定年を迎える。	彼[かれ]は 来年[らいねん] 定年[ていねん]を 迎[むか]える。	かれ は らいねん ていねん を むかえる	
\\	彼[かれ]は 来年[らいねん]
\\	を 迎[むか]える。			
\\	定員	定員[ていいん]	ていいん	
\\	降りてください、定員オーバーです。	降[お]りてください、 定員[ていいん]オーバーです。	おりて ください ていいん おーばー です	
\\	降[お]りてください、
\\	オーバーです。			
\\	定期	定期[ていき]	ていき	
\\	定期演奏会は年に4回あります。	定期[ていき] 演奏会[えんそうかい]は 年[ねん]に4 回[かい]あります。	ていきえんそうかい は ねん に 
\\	かい あります	
\\	演奏会[えんそうかい]は 年[ねん]に4 回[かい]あります。			
\\	定価	定価[ていか]	ていか	
\\	この本の定価は525円です。	この 本[ほん]の 定価[ていか]は525 円[えん]です。	この ほん の ていか は 
\\	えん です	
\\	この 本[ほん]の
\\	は525 円[えん]です。			
\\	未定	未定[みてい]	みてい	
\\	この件の担当者は未定です。	この 件[けん]の 担当者[たんとうしゃ]は 未定[みてい]です。	この けん の たんとうしゃ は みてい です	
\\	この 件[けん]の 担当者[たんとうしゃ]は
\\	です。			
\\	定食	定食[ていしょく]	ていしょく	
\\	昼の定食は3種類あります。	昼[ひる]の 定食[ていしょく]は3 種類[しゅるい]あります。	ひる の ていしょく は 
\\	しゅるい あります	
\\	昼[ひる]の
\\	は3 種類[しゅるい]あります。			
\\	定休日	定休日[ていきゅうび]	ていきゅうび	
\\	この店は水曜が定休日です。	この 店[みせ]は 水曜[すいよう]が 定休日[ていきゅうび]です。	この みせ は すいよう が ていきゅうび です	
\\	この 店[みせ]は 水曜[すいよう]が
\\	です。			
\\	束	束[たば]	たば	
\\	これは一束で300円です。	これは 一[ひと] 束[たば]で300 円[えん]です。	これ は ひとたば で 
\\	えん です	
\\	これは 一[ひと]
\\	で300 円[えん]です。			
\\	変更	変更[へんこう]	へんこう	
\\	予定が変更になりました。	予定[よてい]が 変更[へんこう]になりました。	よてい が へんこう に なりました	
\\	予定[よてい]が
\\	になりました。			
\\	夜更かし	夜更[よふ]かし	よふかし	
\\	最近の子供たちは夜更かしです。	最近[さいきん]の 子供[こども]たちは 夜更[よふ]かしです。	さいきん の こども たち は よふかし です	
\\	最近[さいきん]の 子供[こども]たちは
\\	です。			
\\	増す	増[ま]す	ます	
\\	大雨で川の水かさが増しているな。	大雨[おおあめ]で 川[かわ]の 水[みず]かさが 増[ま]しているな。	おおあめ で かわ の みずかさ が まして いる な	
\\	大雨[おおあめ]で 川[かわ]の 水[みず]かさが
\\	な。			
\\	増大	増大[ぞうだい]	ぞうだい	
\\	生産コスト増大のため、値上げします。	生産[せいさん]コスト 増大[ぞうだい]のため、 値上[ねあ]げします。	せいさん こすと ぞうだい の ため ねあげ します	
\\	生産[せいさん]コスト
\\	のため、 値上[ねあ]げします。			
\\	増減	増減[ぞうげん]	ぞうげん	
\\	この数年、体重は増減していません。	この 数年[すうねん]、 体重[たいじゅう]は 増減[ぞうげん]していません。	この すうねん たいじゅう は ぞうげん して いません	
\\	この 数年[すうねん]、 体重[たいじゅう]は
\\	していません。			
\\	水着	水着[みずぎ]	みずぎ	
\\	水着に着替えました。	水着[みずぎ]に 着替[きが]えました。	みずぎ に きがえました	
\\	に 着替[きが]えました。			
\\	役立つ	役立[やくだ]つ	やくだつ	
\\	学校で勉強したことが役立った。	学校[がっこう]で 勉強[べんきょう]したことが 役立[やくだ]った。	がっこう で べんきょう した こと が やくだった	
\\	学校[がっこう]で 勉強[べんきょう]したことが
\\	夕立	夕立[ゆうだち]	ゆうだち	
\\	帰宅中、夕立にあったの。	帰宅中[きたくちゅう]、 夕立[ゆうだち]にあったの。	きたくちゅう ゆうだち に あった の	
\\	帰宅中[きたくちゅう]、
\\	にあったの。			
\\	次々に	次々[つぎつぎ]に	つぎつぎに	
\\	走者が次々にゴールしました。	走者[そうしゃ]が 次々[つぎつぎ]にゴールしました。	そうしゃ が つぎつぎに ごーる しました	
\\	走者[そうしゃ]が
\\	ゴールしました。			
\\	次ぐ	次[つ]ぐ	つぐ	
\\	彼はトップに次ぐ好成績でした。	彼[かれ]はトップに 次[つ]ぐ 好成績[こうせいせき]でした。	かれ は とっぷ に つぐ こうせいせき でした	
\\	彼[かれ]はトップに
\\	好成績[こうせいせき]でした。			
\\	早朝	早朝[そうちょう]	そうちょう	
\\	私は早朝のジョギングを日課にしています。	私[わたし]は 早朝[そうちょう]のジョギングを 日課[にっか]にしています。	わたし は そうちょう の じょぎんぐ を にっか に して います	
\\	私[わたし]は
\\	のジョギングを 日課[にっか]にしています。			
\\	早める	早[はや]める	はやめる	
\\	集合時間を30分早めました。	集合時間[しゅうごうじかん]を30 分[ぷん] 早[はや]めました。	しゅうごうじかん を 
\\	ぷん はやめました	
\\	集合時間[しゅうごうじかん]を30 分[ぷん]
\\	早まる	早[はや]まる	はやまる	
\\	早まらないでよく考えましょう。	早[はや]まらないでよく 考[かんが]えましょう。	はやまらない で よく かんがえましょう	
\\	よく 考[かんが]えましょう。			
\\	始め	始[はじ]め	はじめ	
\\	私たちの旅は始めはよかったんだ。	私[わたし]たちの 旅[たび]は 始[はじ]めはよかったんだ。	わたしたち の たび は はじめ は よかった ん だ	
\\	私[わたし]たちの 旅[たび]は
\\	はよかったんだ。			
\\	始まり	始[はじ]まり	はじまり	
\\	いよいよ劇の始まりですね。	いよいよ 劇[げき]の 始[はじ]まりですね。	いよいよ げき の はじまり です ね	
\\	いよいよ 劇[げき]の
\\	ですね。			
\\	年始	年始[ねんし]	ねんし	
\\	部下の方が年始の挨拶に見えましたよ。	部下[ぶか]の 方[かた]が 年始[ねんし]の 挨拶[あいさつ]に 見[み]えましたよ。	ぶか の かた が ねんし の あいさつ に みえました よ	
\\	部下[ぶか]の 方[かた]が
\\	の 挨拶[あいさつ]に 見[み]えましたよ。			
\\	実る	実[みの]る	みのる	
\\	やっと努力が実りました。	やっと 努力[どりょく]が 実[みの]りました。	やっと どりょく が みのりました	
\\	やっと 努力[どりょく]が
\\	実	実[み]	み	
\\	庭の木が赤い実をつけた。	庭[にわ]の 木[き]が 赤[あか]い 実[み]をつけた。	にわ の き が あかい み を つけた	
\\	庭[にわ]の 木[き]が 赤[あか]い
\\	をつけた。			
\\	楽しみ	楽[たの]しみ	たのしみ	
\\	旅行は父の老後の楽しみです。	旅行[りょこう]は 父[ちち]の 老後[ろうご]の 楽[たの]しみです。	りょこう は ちち の ろうご の たのしみ です	
\\	旅行[りょこう]は 父[ちち]の 老後[ろうご]の
\\	です。			
\\	楽	楽[らく]	らく	
\\	彼には楽な仕事が与えられたよ。	彼[かれ]には 楽[らく]な 仕事[しごと]が 与[あた]えられたよ。	かれ に は らく な しごと が あたえられた よ	
\\	彼[かれ]には
\\	な 仕事[しごと]が 与[あた]えられたよ。			
\\	欲	欲[よく]	よく	
\\	あまり欲を出しちゃだめだよ。	あまり 欲[よく]を 出[だ]しちゃだめだよ。	あまり よく を だし ちゃ だめ だ よ	
\\	あまり
\\	を 出[だ]しちゃだめだよ。			
\\	欲求	欲求[よっきゅう]	よっきゅう	
\\	時には自分の欲求を抑えることも必要です。	時[とき]には 自分[じぶん]の 欲求[よっきゅう]を 抑[おさ]えることも 必要[ひつよう]です。	ときには じぶん の よっきゅう を おさえる こと も ひつよう です	
\\	時[とき]には 自分[じぶん]の
\\	を 抑[おさ]えることも 必要[ひつよう]です。			
\\	場面	場面[ばめん]	ばめん	
\\	ここがいちばん面白い場面です。	ここがいちばん 面白[おもしろ]い 場面[ばめん]です。	ここ が いちばん おもしろい ばめん です	
\\	ここがいちばん 面白[おもしろ]い
\\	です。			
\\	方面	方面[ほうめん]	ほうめん	
\\	沖縄方面にお出かけの方は台風にご注意ください。	沖縄[おきなわ] 方面[ほうめん]にお 出[で]かけの 方[かた]は 台風[たいふう]にご 注意[ちゅうい]ください。	おきなわ ほうめん に おでかけ の かた は たいふう に ごちゅうい ください	
\\	沖縄[おきなわ]
\\	にお 出[で]かけの 方[かた]は 台風[たいふう]にご 注意[ちゅうい]ください。			
\\	水色	水色[みずいろ]	みずいろ	
\\	箱に水色のリボンがかかっていたの。	箱[はこ]に 水色[みずいろ]のリボンがかかっていたの。	はこ に みずいろ の りぼん が かかって いた の	
\\	箱[はこ]に
\\	のリボンがかかっていたの。			
\\	地形	地形[ちけい]	ちけい	
\\	ここはなだらかな地形です。	ここはなだらかな 地形[ちけい]です。	ここ は なだらか な ちけい です	
\\	ここはなだらかな
\\	です。			
\\	方角	方角[ほうがく]	ほうがく	
\\	私と彼は帰る方角が同じです。	私[わたし]と 彼[かれ]は 帰[かえ]る 方角[ほうがく]が 同[おな]じです。	わたし と かれ は かえる ほうがく が おなじ です	
\\	私[わたし]と 彼[かれ]は 帰[かえ]る
\\	が 同[おな]じです。			
\\	同時	同時[どうじ]	どうじ	
\\	二人の走者は同時にゴールしたよ。	二人[ふたり]の 走者[そうしゃ]は 同時[どうじ]にゴールしたよ。	ふたり の そうしゃ は どうじ に ごーる した よ	
\\	二人[ふたり]の 走者[そうしゃ]は
\\	にゴールしたよ。			
\\	同一	同一[どういつ]	どういつ	
\\	この人とその人は、同一人物ですか。	この 人[ひと]とその 人[ひと]は、 同一[どういつ] 人物[じんぶつ]ですか。	この ひと と その ひと は どういつ じんぶつ です か	
\\	この 人[ひと]とその 人[ひと]は、
\\	人物[じんぶつ]ですか。			
\\	同情	同情[どうじょう]	どうじょう	
\\	友人は私に同情してくれたよ。	友人[ゆうじん]は 私[わたし]に 同情[どうじょう]してくれたよ。	ゆうじん は わたし に どうじょう して くれた よ	
\\	友人[ゆうじん]は 私[わたし]に
\\	してくれたよ。			
\\	同性	同性[どうせい]	どうせい	
\\	同性の友達より異性の友達のほうが多いよ。	同性[どうせい]の 友達[ともだち]より 異性[いせい]の 友達[ともだち]のほうが 多[おお]いよ。	どうせい の ともだち より いせい の ともだち の ほう が おおい よ	
\\	の 友達[ともだち]より 異性[いせい]の 友達[ともだち]のほうが 多[おお]いよ。			
\\	旅	旅[たび]	たび	
\\	姉はよく旅をします。	姉[あね]はよく 旅[たび]をします。	あね は よく たび を します	
\\	姉[あね]はよく
\\	をします。			
\\	旅客	旅客[りょかく]	りょかく	
\\	その便は外国人の旅客が多かったよ。	その 便[びん]は 外国人[がいこくじん]の 旅客[りょかく]が 多[おお]かったよ。	その びん は がいこくじん の りょかく が おおかった よ	
\\	その 便[びん]は 外国人[がいこくじん]の
\\	が 多[おお]かったよ。			
\\	旅費	旅費[りょひ]	りょひ	
\\	父が旅費を出してくれました。	父[ちち]が 旅費[りょひ]を 出[だ]してくれました。	ちち が りょひ を だして くれました	
\\	父[ちち]が
\\	を 出[だ]してくれました。			
\\	和らげる	和[やわ]らげる	やわらげる	
\\	ユーモアは場の雰囲気を和らげるわね。	ユーモアは 場[ば]の 雰囲気[ふんいき]を 和[やわ]らげるわね。	ゆーもあ は ば の ふんいき を やわらげる わ ね	
\\	ユーモアは 場[ば]の 雰囲気[ふんいき]を
\\	わね。			
\\	和語	和語[わご]	わご	
\\	日本で生まれた言葉を和語といいます。	日本[にっぽん]で 生[う]まれた 言葉[ことば]を 和語[わご]といいます。	にっぽん で うまれた ことば を わご と いいます	
\\	日本[にっぽん]で 生[う]まれた 言葉[ことば]を
\\	といいます。			
\\	和風	和風[わふう]	わふう	
\\	夕食に和風パスタを作ったよ。	夕食[ゆうしょく]に 和風[わふう]パスタを 作[つく]ったよ。	ゆうしょく に わふう ぱすた を つくった よ	
\\	夕食[ゆうしょく]に
\\	パスタを 作[つく]ったよ。			
\\	和食	和食[わしょく]	わしょく	
\\	私は和食が好きです。	私[わたし]は 和食[わしょく]が 好[す]きです。	わたし は わしょく が すき です	
\\	私[わたし]は
\\	が 好[す]きです。			
\\	和やか	和[なご]やか	なごやか	
\\	彼らは和やかに食事をした。	彼[かれ]らは 和[なご]やかに 食事[しょくじ]をした。	かれら は なごやか に しょくじ を した	
\\	彼[かれ]らは
\\	に 食事[しょくじ]をした。			
\\	和らぐ	和[やわ]らぐ	やわらぐ	
\\	この曲を聞くと気持ちが和らぎます。	この 曲[きょく]を 聞[き]くと 気持[きも]ちが 和[やわ]らぎます。	この きょく を きく と きもち が やわらぎます	
\\	この 曲[きょく]を 聞[き]くと 気持[きも]ちが
\\	和式	和式[わしき]	わしき	
\\	あの家のトイレは和式です。	あの 家[うち]のトイレは 和式[わしき]です。	あの うち の といれ は わしき です	
\\	あの 家[うち]のトイレは
\\	です。			
\\	和英	和英[わえい]	わえい	
\\	私は和英辞書をよく使います。	私[わたし]は 和英[わえい] 辞書[じしょ]をよく 使[つか]います。	わたし は わえい じしょ を よく つかいます	
\\	私[わたし]は
\\	辞書[じしょ]をよく 使[つか]います。			
\\	東洋	東洋[とうよう]	とうよう	
\\	彼は東洋文化を研究しているよ。	彼[かれ]は 東洋[とうよう] 文化[ぶんか]を 研究[けんきゅう]しているよ。	かれ は とうようぶんか を けんきゅう して いる よ	
\\	彼[かれ]は
\\	文化[ぶんか]を 研究[けんきゅう]しているよ。			
\\	洋風	洋風[ようふう]	ようふう	
\\	私は洋風の家に住んでいます。	私[わたし]は 洋風[ようふう]の 家[いえ]に 住[す]んでいます。	わたし は ようふう の いえ に すんで います	
\\	私[わたし]は
\\	の 家[いえ]に 住[す]んでいます。			
\\	洋画	洋画[ようが]	ようが	
\\	私は週に3本洋画を見ます。	私[わたし]は 週[しゅう]に3 本[ぼん] 洋画[ようが]を 見[み]ます。	わたし は しゅう に 
\\	ぼん ようが を みます	
\\	私[わたし]は 週[しゅう]に3 本[ぼん]
\\	を 見[み]ます。			
\\	洋式	洋式[ようしき]	ようしき	
\\	彼の家のトイレは洋式です。	彼[かれ]の 家[いえ]のトイレは 洋式[ようしき]です。	かれ の いえ の といれ は ようしき です	
\\	彼[かれ]の 家[いえ]のトイレは
\\	です。			
\\	洋食	洋食[ようしょく]	ようしょく	
\\	昨日の晩御飯は洋食でした。	昨日[きのう]の 晩御飯[ばんごはん]は 洋食[ようしょく]でした。	きのう の ばんごはん は ようしょく でした	
\\	昨日[きのう]の 晩御飯[ばんごはん]は
\\	でした。			
\\	洋間	洋間[ようま]	ようま	
\\	彼の家には洋間があります。	彼[かれ]の 家[いえ]には 洋間[ようま]があります。	かれ の いえ に は ようま が あります	
\\	彼[かれ]の 家[いえ]には
\\	があります。			
\\	和服	和服[わふく]	わふく	
\\	彼女は和服がよく似合う。	彼女[かのじょ]は 和服[わふく]がよく 似合[にあ]う。	かのじょ は わふく が よく にあう	
\\	彼女[かのじょ]は
\\	がよく 似合[にあ]う。			
\\	待合室	待合室[まちあいしつ]	まちあいしつ	
\\	待合室はとても込んでいたよ。	待合室[まちあいしつ]はとても 込[こ]んでいたよ。	まちあいしつ は とても こんで いた よ	
\\	はとても 込[こ]んでいたよ。			
\\	和室	和室[わしつ]	わしつ	
\\	この和室の天井は低いね。	この 和室[わしつ]の 天井[てんじょう]は 低[ひく]いね。	この わしつ の てんじょう は ひくい ね	
\\	この
\\	の 天井[てんじょう]は 低[ひく]いね。			
\\	洋室	洋室[ようしつ]	ようしつ	
\\	このテーブルは洋室に合わない。	このテーブルは 洋室[ようしつ]に 合[あ]わない。	この てーぶる は ようしつ に あわない	
\\	このテーブルは
\\	に 合[あ]わない。			
\\	母親	母親[ははおや]	ははおや	
\\	彼女は2才の子の母親です。	彼女[かのじょ]は2 才[さい]の 子[こ]の 母親[ははおや]です。	かのじょ は 
\\	さい の こ の ははおや です	
\\	彼女[かのじょ]は2 才[さい]の 子[こ]の
\\	です。			
\\	歩行者	歩行者[ほこうしゃ]	ほこうしゃ	
\\	歩行者は道の右側を歩いてください。	歩行者[ほこうしゃ]は 道[みち]の 右側[みぎがわ]を 歩[ある]いてください。	ほこうしゃ は みち の みぎがわ を あるいて ください	
\\	は 道[みち]の 右側[みぎがわ]を 歩[ある]いてください。			
\\	文学者	文学者[ぶんがくしゃ]	ぶんがくしゃ	
\\	彼は有名な文学者です。	彼[かれ]は 有名[ゆうめい]な 文学者[ぶんがくしゃ]です。	かれ は ゆうめい な ぶんがくしゃ です	
\\	彼[かれ]は 有名[ゆうめい]な
\\	です。			
\\	歩行者天国	歩行者天国[ほこうしゃてんごく]	ほこうしゃてんごく	
\\	日曜日はこの通りが歩行者天国になります。	日曜日[にちようび]はこの 通[とお]りが 歩行者天国[ほこうしゃてんごく]になります。	にちようび は この とおり が ほこうしゃてんごく に なります	
\\	日曜日[にちようび]はこの 通[とお]りが
\\	になります。			
\\	未婚	未婚[みこん]	みこん	
\\	彼はまだ未婚です。	彼[かれ]はまだ 未婚[みこん]です。	かれ は まだ みこん です	
\\	彼[かれ]はまだ
\\	です。			
\\	果たす	果[は]たす	はたす	
\\	彼はしっかりと責任を果たしました。	彼[かれ]はしっかりと 責任[せきにん]を 果[は]たしました。	かれ は しっかり と せきにん を はたしました	
\\	彼[かれ]はしっかりと 責任[せきにん]を
\\	果たして	果[は]たして	はたして	
\\	果たして彼は現れるだろうか。	果[は]たして 彼[かれ]は 現[あらわ]れるだろうか。	はたして かれ は あらわれるだろう か	
\\	彼[かれ]は 現[あらわ]れるだろうか。			
\\	日課	日課[にっか]	にっか	
\\	犬の散歩は私の日課です。	犬[いぬ]の 散歩[さんぽ]は 私[わたし]の 日課[にっか]です。	いぬ の さんぽ は わたし の にっか です	
\\	犬[いぬ]の 散歩[さんぽ]は 私[わたし]の
\\	です。			
\\	有効	有効[ゆうこう]	ゆうこう	
\\	私の免許は来年まで有効です。	私[わたし]の 免許[めんきょ]は 来年[らいねん]まで 有効[ゆうこう]です。	わたし の めんきょ は らいねん まで ゆうこう です	
\\	私[わたし]の 免許[めんきょ]は 来年[らいねん]まで
\\	です。			
\\	民間	民間[みんかん]	みんかん	
\\	その土地は民間企業に売却されたんだ。	その 土地[とち]は 民間[みんかん] 企業[きぎょう]に 売却[ばいきゃく]されたんだ。	その とち は みんかん きぎょう に ばいきゃく された ん だ	
\\	その 土地[とち]は
\\	企業[きぎょう]に 売却[ばいきゃく]されたんだ。			
\\	民族	民族[みんぞく]	みんぞく	
\\	私は民族の歴史に興味があります。	私[わたし]は 民族[みんぞく]の 歴史[れきし]に 興味[きょうみ]があります。	わたし は みんぞく の れきし に きょうみ が あります	
\\	私[わたし]は
\\	の 歴史[れきし]に 興味[きょうみ]があります。			
\\	持ち主	持[も]ち 主[ぬし]	もちぬし	
\\	この自転車の持ち主は誰ですか。	この 自転車[じてんしゃ]の 持[も]ち 主[ぬし]は 誰[だれ]ですか。	この じてんしゃ の もちぬし は だれ です か	
\\	この 自転車[じてんしゃ]の
\\	は 誰[だれ]ですか。			
\\	民主	民主[みんしゅ]	みんしゅ	
\\	民主主義について勉強しました。	民主[みんしゅ] 主義[しゅぎ]について 勉強[べんきょう]しました。	みんしゅしゅぎ に ついて べんきょう しました	
\\	主義[しゅぎ]について 勉強[べんきょう]しました。			
\\	家主	家主[やぬし]	やぬし	
\\	家主は1階に住んでいます。	家主[やぬし]は1 階[かい]に 住[す]んでいます。	やぬし は 
\\	かい に すんで います	
\\	は1 階[かい]に 住[す]んでいます。			
\\	定義	定義[ていぎ]	ていぎ	
\\	美しさを定義してください。	美[うつく]しさを 定義[ていぎ]してください。	うつくしさ を ていぎ して ください	
\\	美[うつく]しさを
\\	してください。			
\\	対話	対話[たいわ]	たいわ	
\\	親子の対話は大切だよ。	親子[おやこ]の 対話[たいわ]は 大切[たいせつ]だよ。	おやこ の たいわ は たいせつ だ よ	
\\	親子[おやこ]の
\\	は 大切[たいせつ]だよ。			
\\	対	対[つい]	つい	
\\	このズボンは上着と対になっています。	このズボンは 上着[うわぎ]と 対[つい]になっています。	この ずぼん は うわぎ と つい に なって います	
\\	このズボンは 上着[うわぎ]と
\\	になっています。			
\\	対応	対応[たいおう]	たいおう	
\\	彼はいつも素早い対応をするね。	彼[かれ]はいつも 素早[すばや]い 対応[たいおう]をするね。	かれ は いつも すばやい たいおう を する ね	
\\	彼[かれ]はいつも 素早[すばや]い
\\	をするね。			
\\	問答	問答[もんどう]	もんどう	
\\	あなたと問答している暇はないの。	あなたと 問答[もんどう]している 暇[ひま]はないの。	あなた と もんどう して いる ひま は ない の	
\\	あなたと
\\	している 暇[ひま]はないの。			
\\	専門家	専門家[せんもんか]	せんもんか	
\\	教授はフランス文学の専門家。	教授[きょうじゅ]はフランス 文学[ぶんがく]の 専門家[せんもんか]。	きょうじゅ は ふらんす ぶんがく の せんもんか	
\\	教授[きょうじゅ]はフランス 文学[ぶんがく]の
\\	専門	専門[せんもん]	せんもん	
\\	法律は私の専門です。	法律[ほうりつ]は 私[わたし]の 専門[せんもん]です。	ほうりつ は わたし の せんもん です	
\\	法律[ほうりつ]は 私[わたし]の
\\	です。			
\\	専用	専用[せんよう]	せんよう	
\\	これは女性専用の車両です。	これは 女性[じょせい] 専用[せんよう]の 車両[しゃりょう]です。	これ は じょせい せんよう の しゃりょう です	
\\	これは 女性[じょせい]
\\	の 車両[しゃりょう]です。			
\\	本格的	本格的[ほんかくてき]	ほんかくてき	
\\	彼は絵を本格的に勉強しているんだ。	彼[かれ]は 絵[え]を 本格的[ほんかくてき]に 勉強[べんきょう]しているんだ。	かれ は え を ほんかくてき に べんきょう して いる ん だ	
\\	彼[かれ]は 絵[え]を
\\	に 勉強[べんきょう]しているんだ。			
\\	文化的	文化的[ぶんかてき]	ぶんかてき	
\\	この国は文化的な事業に力を入れています。	この 国[くに]は 文化的[ぶんかてき]な 事業[じぎょう]に 力[ちから]を 入[い]れています。	この くに は ぶんかてき な じぎょう に ちから を いれて います	
\\	この 国[くに]は
\\	な 事業[じぎょう]に 力[ちから]を 入[い]れています。			
\\	普段	普段[ふだん]	ふだん	
\\	私は普段は
\\	シャツとジーンズを着ています。	私[わたし]は 普段[ふだん]は 
\\	[てぃー]シャツとジーンズを 着[き]ています。	わたし は ふだん は てぃーしゃつ と じーんず を きて います	
\\	私[わたし]は
\\	は 
\\	[てぃー]シャツとジーンズを 着[き]ています。			
\\	平面	平面[へいめん]	へいめん	
\\	このメガネをかけると平面が立体に見えます。	このメガネをかけると 平面[へいめん]が 立体[りったい]に 見[み]えます。	この めがね を かける と へいめん が りったい に みえます	
\\	このメガネをかけると
\\	が 立体[りったい]に 見[み]えます。			
\\	平気	平気[へいき]	へいき	
\\	彼女は平気な顔をしていた。	彼女[かのじょ]は 平気[へいき]な 顔[かお]をしていた。	かのじょ は へいき な かお を して いた	
\\	彼女[かのじょ]は
\\	な 顔[かお]をしていた。			
\\	平ら	平[たい]ら	たいら	
\\	その建物の屋根は平らだね。	その 建物[たてもの]の 屋根[やね]は 平[たい]らだね。	その たてもの の やね は たいら だ ね	
\\	その 建物[たてもの]の 屋根[やね]は
\\	だね。			
\\	平行	平行[へいこう]	へいこう	
\\	平行に線を引いてください。	平行[へいこう]に 線[せん]を 引[ひ]いてください。	へいこう に せん を ひいて ください	
\\	に 線[せん]を 引[ひ]いてください。			
\\	地平線	地平線[ちへいせん]	ちへいせん	
\\	地平線に夕日が沈むところだったの。	地平線[ちへいせん]に 夕日[ゆうひ]が 沈[しず]むところだったの。	ちへいせん に ゆうひ が しずむ ところ だった の	
\\	に 夕日[ゆうひ]が 沈[しず]むところだったの。			
\\	平野	平野[へいや]	へいや	
\\	広い平野が一面雪で真っ白でした。	広[ひろ]い 平野[へいや]が 一面雪[いちめん ゆき]で 真[ま]っ 白[しろ]でした。	ひろい へいや が いちめん ゆき で まっしろ でした	
\\	広[ひろ]い
\\	が 一面雪[いちめん ゆき]で 真[ま]っ 白[しろ]でした。			
\\	平たい	平[ひら]たい	ひらたい	
\\	平たいお皿を一枚取って。	平[ひら]たいお 皿[さら]を 一枚取[いちまい と]って。	ひらたい おさら を いちまい とって	
\\	お 皿[さら]を 一枚取[いちまい と]って。			
\\	平方	平方[へいほう]	へいほう	
\\	この土地の面積は約100平方メートルです。	この 土地[とち]の 面積[めんせき]は 約100[やく 
\\	平方[へいほう]メートルです。	この とち の めんせき は やく 
\\	へいほうめーとる です	
\\	この 土地[とち]の 面積[めんせき]は 約100[やく 
\\	メートルです。			
\\	平日	平日[へいじつ]	へいじつ	
\\	彼は平日がお休みです。	彼[かれ]は 平日[へいじつ]がお 休[やす]みです。	かれ は へいじつ が おやすみ です	
\\	彼[かれ]は
\\	がお 休[やす]みです。			
\\	平均	平均[へいきん]	へいきん	
\\	平均で一日に8時間ぐらい働いています。	平均[へいきん]で 一日[いちにち]に8 時間[じかん]ぐらい 働[はたら]いています。	へいきん で いちにち に 
\\	じかん ぐらい はたらいて います	
\\	で 一日[いちにち]に8 時間[じかん]ぐらい 働[はたら]いています。			
\\	平等	平等[びょうどう]	びょうどう	
\\	あの先生は生徒をみな平等に扱います。	あの 先生[せんせい]は 生徒[せいと]をみな 平等[びょうどう]に 扱[あつか]います。	あの せんせい は せいと を みな びょうどう に あつかいます	
\\	あの 先生[せんせい]は 生徒[せいと]をみな
\\	に 扱[あつか]います。			
\\	同等	同等[どうとう]	どうとう	
\\	彼には大学生と同等の学力があります。	彼[かれ]には 大学生[だいがくせい]と 同等[どうとう]の 学力[がくりょく]があります。	かれ に は だいがくせい と どうとう の がくりょく が あります	
\\	彼[かれ]には 大学生[だいがくせい]と
\\	の 学力[がくりょく]があります。			
\\	対等	対等[たいとう]	たいとう	
\\	その子供は大人と対等に話していたよ。	その 子供[こども]は 大人[おとな]と 対等[たいとう]に 話[はな]していたよ。	その こども は おとな と たいとう に はなして いた よ	
\\	その 子供[こども]は 大人[おとな]と
\\	に 話[はな]していたよ。			
\\	必死	必死[ひっし]	ひっし	
\\	学生たちは授業についていくのに必死です。	学生[がくせい]たちは 授業[じゅぎょう]についていくのに 必死[ひっし]です。	がくせいたち は じゅぎょう に ついて いく の に ひっし です	
\\	学生[がくせい]たちは 授業[じゅぎょう]についていくのに
\\	です。			
\\	必死に	必死[ひっし]に	ひっしに	
\\	必死に単語を暗記したよ。	必死[ひっし]に 単語[たんご]を 暗記[あんき]したよ。	ひっしに たんご を あんき した よ	
\\	単語[たんご]を 暗記[あんき]したよ。			
\\	多忙	多忙[たぼう]	たぼう	
\\	彼女は多忙な人です。	彼女[かのじょ]は 多忙[たぼう]な 人[ひと]です。	かのじょ は たぼう な ひと です	
\\	彼女[かのじょ]は
\\	な 人[ひと]です。			
\\	日本酒	日本酒[にほんしゅ]	にほんしゅ	
\\	珍しい日本酒が手に入りました。	珍[めずら]しい 日本酒[にほんしゅ]が 手[て]に 入[はい]りました。	めずらしい にほんしゅ が て に はいりました	
\\	珍[めずら]しい
\\	が 手[て]に 入[はい]りました。			
\\	存じる	存[ぞん]じる	ぞんじる	
\\	郵便局はどこかご存じですか。	郵便局[ゆうびんきょく]はどこかご 存[ぞん]じですか。	ゆうびんきょく は どこ か ごぞんじ です か	
\\	郵便局[ゆうびんきょく]はどこかご
\\	ですか。			
\\	注ぐ	注[そそ]ぐ	そそぐ	
\\	みんなのグラスにジュースを注いだよ。	みんなのグラスにジュースを 注[そそ]いだよ。	みんな の ぐらす に じゅーす を そそいだ よ	
\\	みんなのグラスにジュースを
\\	よ。			
\\	注	注[ちゅう]	ちゅう	
\\	詳しくは注を読んでください。	詳[くわ]しくは 注[ちゅう]を 読[よ]んでください。	くわしくは ちゅう を よんで ください	
\\	詳[くわ]しくは
\\	を 読[よ]んでください。			
\\	同意	同意[どうい]	どうい	
\\	彼の意見には同意できません。	彼[かれ]の 意見[いけん]には 同意[どうい]できません。	かれ の いけん に は どうい できません	
\\	彼[かれ]の 意見[いけん]には
\\	できません。			
\\	明確	明確[めいかく]	めいかく	
\\	彼女には明確な目標があるね。	彼女[かのじょ]には 明確[めいかく]な 目標[もくひょう]があるね。	かのじょ に は めいかく な もくひょう が ある ね	
\\	彼女[かのじょ]には
\\	な 目標[もくひょう]があるね。			
\\	旅客機	旅客機[りょかくき]	りょかくき	
\\	旅客機が墜落したよ。	旅客機[りょかくき]が 墜落[ついらく]したよ。	りょかくき が ついらく した よ	
\\	が 墜落[ついらく]したよ。			
\\	島	島[とう]	とう	
\\	私たちはハワイのマウイ島に旅行したの。	私[わたし]たちはハワイのマウイ 島[とう]に 旅行[りょこう]したの。	わたしたち は はわい の まういとう に りょこう した の	
\\	私[わたし]たちはハワイのマウイ
\\	に 旅行[りょこう]したの。			
\\	成り立つ	成[な]り 立[た]つ	なりたつ	
\\	この島は観光で成り立っています。	この 島[しま]は 観光[かんこう]で 成[な]り 立[た]っています。	この しま は かんこう で なりたって います	
\\	この 島[しま]は 観光[かんこう]で
\\	います。			
\\	未成年	未成年[みせいねん]	みせいねん	
\\	未成年はお酒を飲めません。	未成年[みせいねん]はお 酒[さけ]を 飲[の]めません。	みせいねん は おさけ を のめません	
\\	はお 酒[さけ]を 飲[の]めません。			
\\	敗れる	敗[やぶ]れる	やぶれる	
\\	私のチームは1回戦で敗れたよ。	私[わたし]のチームは1 回戦[かいせん]で 敗[やぶ]れたよ。	わたし の ちーむ は 
\\	かいせん で やぶれた よ	
\\	私[わたし]のチームは1 回戦[かいせん]で
\\	よ。			
\\	因る	因[よ]る	よる	
\\	彼の病気は過労に因るものです。	彼[かれ]の 病気[びょうき]は 過労[かろう]に 因[よ]るものです。	かれ の びょうき は かろう に よる もの です	
\\	彼[かれ]の 病気[びょうき]は 過労[かろう]に
\\	ものです。			
\\	大正	大正[たいしょう]	たいしょう	
\\	祖母は大正生まれです。	祖母[そぼ]は 大正[たいしょう] 生[う]まれです。	そぼ は たいしょう うまれ です	
\\	祖母[そぼ]は
\\	生[う]まれです。			
\\	正に	正[まさ]に	まさに	
\\	彼は正に英雄ね。	彼[かれ]は 正[まさ]に 英雄[えいゆう]ね。	かれ は まさに えいゆう ね	
\\	彼[かれ]は
\\	英雄[えいゆう]ね。			
\\	常に	常[つね]に	つねに	
\\	彼は常に姿勢がいい。	彼[かれ]は 常[つね]に 姿勢[しせい]がいい。	かれ は つねに しせい が いい	
\\	彼[かれ]は
\\	姿勢[しせい]がいい。			
\\	日常	日常[にちじょう]	にちじょう	
\\	音楽は私の日常の一部です。	音楽[おんがく]は 私[わたし]の 日常[にちじょう]の 一部[いちぶ]です。	おんがく は わたし の にちじょう の いちぶ です	
\\	音楽[おんがく]は 私[わたし]の
\\	の 一部[いちぶ]です。			
\\	整える	整[ととの]える	ととのえる	
\\	彼はスピーチの前に服装を整えた。	彼[かれ]はスピーチの 前[まえ]に 服装[ふくそう]を 整[ととの]えた。	かれ は すぴーち の まえ に ふくそう を ととのえた	
\\	彼[かれ]はスピーチの 前[まえ]に 服装[ふくそう]を
\\	整う	整[ととの]う	ととのう	
\\	パーティーの準備が整いました。	パーティーの 準備[じゅんび]が 整[ととの]いました。	ぱーてぃー の じゅんび が ととのいました	
\\	パーティーの 準備[じゅんび]が
\\	提出	提出[ていしゅつ]	ていしゅつ	
\\	課題は7月5日までに提出してください。	課題[かだい]は7 月5日[がつ 
\\	か]までに 提出[ていしゅつ]してください。	かだい は 
\\	がつ 
\\	か まで に ていしゅつ して ください	
\\	課題[かだい]は7 月5日[がつ 
\\	か]までに
\\	してください。			
\\	投票	投票[とうひょう]	とうひょう	
\\	私は朝早く投票を済ませました。	私[わたし]は 朝早[あさ はや]く 投票[とうひょう]を 済[す]ませました。	わたし は あさ はやく とうひょう を すませました	
\\	私[わたし]は 朝早[あさ はや]く
\\	を 済[す]ませました。			
\\	標準	標準[ひょうじゅん]	ひょうじゅん	
\\	ニュースでは標準語が使われるの。	ニュースでは 標準[ひょうじゅん] 語[ご]が 使[つか]われるの。	にゅーす で は ひょうじゅんご が つかわれる の	
\\	ニュースでは
\\	語[ご]が 使[つか]われるの。			
\\	手続き	手続[てつづ]き	てつづき	
\\	入国手続きが終わりました。	入国[にゅうこく] 手続[てつづ]きが 終[お]わりました。	にゅうこく てつづき が おわりました	
\\	入国[にゅうこく]
\\	が 終[お]わりました。			
\\	対談	対談[たいだん]	たいだん	
\\	雑誌にその女優の対談が載っていたよ。	雑誌[ざっし]にその 女優[じょゆう]の 対談[たいだん]が 載[の]っていたよ。	ざっし に その じょゆう の たいだん が のって いた よ	
\\	雑誌[ざっし]にその 女優[じょゆう]の
\\	が 載[の]っていたよ。			
\\	山登り	山登[やまのぼ]り	やまのぼり	
\\	明日は友達と山登りに行きます。	明日[あした]は 友達[ともだち]と 山登[やまのぼ]りに 行[い]きます。	あした は ともだち と やまのぼり に いきます	
\\	明日[あした]は 友達[ともだち]と
\\	に 行[い]きます。			
\\	態度	態度[たいど]	たいど	
\\	あの男の態度にみんなあきれてたよ。	あの 男[おとこ]の 態度[たいど]にみんなあきれてたよ。	あの おとこ の たいど に みんな あきれて た よ	
\\	あの 男[おとこ]の
\\	にみんなあきれてたよ。			
\\	明治	明治[めいじ]	めいじ	
\\	祖父は明治の生まれです。	祖父[そふ]は 明治[めいじ]の 生[う]まれです。	そふ は めいじ の うまれ です	
\\	祖父[そふ]は
\\	の 生[う]まれです。			
\\	府	府[ふ]	ふ	
\\	彼は大阪府に住んでいます。	彼[かれ]は 大阪[おおさか] 府[ふ]に 住[す]んでいます。	かれ は おおさかふ に すんで います	
\\	彼[かれ]は 大阪[おおさか]
\\	に 住[す]んでいます。			
\\	府立	府立[ふりつ]	ふりつ	
\\	彼女は府立大学に通っています。	彼女[かのじょ]は 府立[ふりつ] 大学[だいがく]に 通[かよ]っています。	かのじょ は ふりつ だいがく に かよって います	
\\	彼女[かのじょ]は
\\	大学[だいがく]に 通[かよ]っています。			
\\	当選	当選[とうせん]	とうせん	
\\	彼は選挙に当選しました。	彼[かれ]は 選挙[せんきょ]に 当選[とうせん]しました。	かれ は せんきょ に とうせん しました	
\\	彼[かれ]は 選挙[せんきょ]に
\\	しました。			
\\	天候	天候[てんこう]	てんこう	
\\	ここは天候の変化が激しいですね。	ここは 天候[てんこう]の 変化[へんか]が 激[はげ]しいですね。	ここ は てんこう の へんか が はげしい です ね	
\\	ここは
\\	の 変化[へんか]が 激[はげ]しいですね。			
\\	手首	手首[てくび]	てくび	
\\	手首の関節をひねっちゃった。	手首[てくび]の 関節[かんせつ]をひねっちゃった。	てくび の かんせつ を ひねっちゃった	
\\	の 関節[かんせつ]をひねっちゃった。			
\\	悩む	悩[なや]む	なやむ	
\\	彼は受験のことで悩んでいます。	彼[かれ]は 受験[じゅけん]のことで 悩[なや]んでいます。	かれ は じゅけん の こと で なやんで います	
\\	彼[かれ]は 受験[じゅけん]のことで
\\	います。			
\\	悩み	悩[なや]み	なやみ	
\\	彼は大きな悩みを抱えていました。	彼[かれ]は 大[おお]きな 悩[なや]みを 抱[かか]えていました。	かれ は おおき な なやみ を かかえて いました	
\\	彼[かれ]は 大[おお]きな
\\	を 抱[かか]えていました。			
\\	命じる	命[めい]じる	めいじる	
\\	急に出張を命じられました。	急[きゅう]に 出張[しゅっちょう]を 命[めい]じられました。	きゅう に しゅっちょう を めいじられました	
\\	急[きゅう]に 出張[しゅっちょう]を
\\	担任	担任[たんにん]	たんにん	
\\	私は3年生のクラスを担任しています。	私[わたし]は3 年生[ねんせい]のクラスを 担任[たんにん]しています。	わたし は 
\\	ねんせい の くらす を たんにん して います	
\\	私[わたし]は3 年生[ねんせい]のクラスを
\\	しています。			
\\	採る	採[と]る	とる	
\\	この山ではきのこが採れますよ	この 山[やま]ではきのこが 採[と]れますよ	この やま で は きのこ が とれます よ	
\\	この 山[やま]ではきのこが
\\	よ			
\\	就く	就[つ]く	つく	
\\	今年から新しい仕事に就きます。	今年[ことし]から 新[あたら]しい 仕事[しごと]に 就[つ]きます。	ことし から あたらしい しごと に つきます	
\\	今年[ことし]から 新[あたら]しい 仕事[しごと]に
\\	早退	早退[そうたい]	そうたい	
\\	具合が悪かったので仕事を早退しました。	具合[ぐあい]が 悪[わる]かったので 仕事[しごと]を 早退[そうたい]しました。	ぐあい が わるかった の で しごと を そうたい しました	
\\	具合[ぐあい]が 悪[わる]かったので 仕事[しごと]を
\\	しました。			
\\	惨め	惨[みじ]め	みじめ	
\\	彼は惨めな気持ちになったの。	彼[かれ]は 惨[みじ]めな 気持[きも]ちになったの。	かれ は みじめ な きもち に なった の	
\\	彼[かれ]は
\\	な 気持[きも]ちになったの。			
\\	比率	比率[ひりつ]	ひりつ	
\\	業界は女性の比率が低い。	
\\	業界[あいてぃーぎょうかい]は 女性[じょせい]の 比率[ひりつ]が 低[ひく]い。	あいてぃーぎょうかい は じょせい の ひりつ が ひくい	
\\	業界[あいてぃーぎょうかい]は 女性[じょせい]の
\\	が 低[ひく]い。			
\\	対比	対比[たいひ]	たいひ	
\\	この絵は赤と黒の対比が美しいですね。	この 絵[え]は 赤[あか]と 黒[くろ]の 対比[たいひ]が 美[うつく]しいですね。	この え は あか と くろ の たいひ が うつくしい です ね	
\\	この 絵[え]は 赤[あか]と 黒[くろ]の
\\	が 美[うつく]しいですね。			
\\	比較的	比較的[ひかくてき]	ひかくてき	
\\	今年は比較的景気がいい。	今年[ことし]は 比較的[ひかくてき] 景気[けいき]がいい。	ことし は ひかくてき けいき が いい	
\\	今年[ことし]は
\\	景気[けいき]がいい。			
\\	比較	比較[ひかく]	ひかく	
\\	去年の売り上げと比較しましょう。	去年[きょねん]の 売[う]り 上[あ]げと 比較[ひかく]しましょう。	きょねん の うりあげ と ひかく しましょう	
\\	去年[きょねん]の 売[う]り 上[あ]げと
\\	しましょう。			
\\	批評	批評[ひひょう]	ひひょう	
\\	その映画はよい批評を得ているんだ。	その 映画[えいが]はよい 批評[ひひょう]を 得[え]ているんだ。	その えいが は よい ひひょう を えて いる ん だ	
\\	その 映画[えいが]はよい
\\	を 得[え]ているんだ。			
\\	想像	想像[そうぞう]	そうぞう	
\\	そんなことは想像できないよ。	そんなことは 想像[そうぞう]できないよ。	そんな こと は そうぞう できない よ	
\\	そんなことは
\\	できないよ。			
\\	抽象的	抽象的[ちゅうしょうてき]	ちゅうしょうてき	
\\	彼は抽象的な絵が好きだね。	彼[かれ]は 抽象的[ちゅうしょうてき]な 絵[え]が 好[す]きだね。	かれ は ちゅうしょうてき な え が すき だ ね	
\\	彼[かれ]は
\\	な 絵[え]が 好[す]きだね。			
\\	変換	変換[へんかん]	へんかん	
\\	ひらがなをカタカナに変換しました。	ひらがなをカタカナに 変換[へんかん]しました。	ひらがな を かたかな に へんかん しました	
\\	ひらがなをカタカナに
\\	しました。			
\\	天災	天災[てんさい]	てんさい	
\\	天災を防ぐことはできません。	天災[てんさい]を 防[ふせ]ぐことはできません。	てんさい を ふせぐ こと は できません	
\\	を 防[ふせ]ぐことはできません。			
\\	有害	有害[ゆうがい]	ゆうがい	
\\	この材料は有害だよ。	この 材料[ざいりょう]は 有害[ゆうがい]だよ。	この ざいりょう は ゆうがい だ よ	
\\	この 材料[ざいりょう]は
\\	だよ。			
\\	派手	派手[はで]	はで	
\\	雪道で派手に転んでしまったの。	雪道[ゆきみち]で 派手[はで]に 転[ころ]んでしまったの。	ゆきみち で はで に ころんで しまった の	
\\	雪道[ゆきみち]で
\\	に 転[ころ]んでしまったの。			
\\	派出所	派出所[はしゅつじょ]	はしゅつじょ	
\\	派出所にだれもいないな。	派出所[はしゅつじょ]にだれもいないな。	はしゅつじょ に だれ も いない な	
\\	にだれもいないな。			
\\	派遣	派遣[はけん]	はけん	
\\	彼はイギリスに派遣されました。	彼[かれ]はイギリスに 派遣[はけん]されました。	かれ は いぎりす に はけん されました	
\\	彼[かれ]はイギリスに
\\	されました。			
\\	捕まる	捕[つか]まる	つかまる	
\\	彼女はついに捕まりました。	彼女[かのじょ]はついに 捕[つか]まりました。	かのじょ は ついに つかまりました	
\\	彼女[かのじょ]はついに
\\	捕まえる	捕[つか]まえる	つかまえる	
\\	少年は網でその蝶を捕まえた。	少年[しょうねん]は 網[あみ]でその 蝶[ちょう]を 捕[つか]まえた。	しょうねん は あみ で その ちょう を つかまえた	
\\	少年[しょうねん]は 網[あみ]でその 蝶[ちょう]を
\\	戦う	戦[たたか]う	たたかう	
\\	彼は最後まで戦ったよ。	彼[かれ]は 最後[さいご]まで 戦[たたか]ったよ。	かれ は さいご まで たたかった よ	
\\	彼[かれ]は 最後[さいご]まで
\\	よ。			
\\	挑戦	挑戦[ちょうせん]	ちょうせん	
\\	彼は新しいことに挑戦している。	彼[かれ]は 新[あたら]しいことに 挑戦[ちょうせん]している。	かれ は あたらしい こと に ちょうせん して いる	
\\	彼[かれ]は 新[あたら]しいことに
\\	している。			
\\	敗戦	敗戦[はいせん]	はいせん	
\\	敗戦の原因は何だろう。	敗戦[はいせん]の 原因[げんいん]は 何[なん]だろう。	はいせん の げんいん は なん だろう	
\\	の 原因[げんいん]は 何[なん]だろう。			
\\	戦い	戦[たたか]い	たたかい	
\\	長い戦いが終わった。	長[なが]い 戦[たたか]いが 終[お]わった。	ながい たたかい が おわった	
\\	長[なが]い
\\	が 終[お]わった。			
\\	大戦	大戦[たいせん]	たいせん	
\\	大戦で多くの人が亡くなりました。	大戦[たいせん]で 多[おお]くの 人[ひと]が 亡[な]くなりました。	たいせん で おおく の ひと が なくなりました	
\\	で 多[おお]くの 人[ひと]が 亡[な]くなりました。			
\\	捜査	捜査[そうさ]	そうさ	
\\	その殺人事件の捜査は2年間続きました。	その 殺人事件[さつじん じけん]の 捜査[そうさ]は2 年間続[ねんかん つづ]きました。	その さつじん じけん の そうさ は 
\\	ねんかん つづきました	
\\	その 殺人事件[さつじん じけん]の
\\	は2 年間続[ねんかん つづ]きました。			
\\	注意深い	注意深[ちゅういぶか]い	ちゅういぶかい	
\\	彼は注意深い人です。	彼[かれ]は 注意深[ちゅういぶか]い 人[ひと]です。	かれ は ちゅういぶかい ひと です	
\\	彼[かれ]は
\\	人[ひと]です。			
\\	段落	段落[だんらく]	だんらく	
\\	次の段落を読んでください。	次[つぎ]の 段落[だんらく]を 読[よ]んでください。	つぎ の だんらく を よんで ください	
\\	次[つぎ]の
\\	を 読[よ]んでください。			
\\	波	波[なみ]	なみ	
\\	今日の海は波が穏やかです。	今日[きょう]の 海[うみ]は 波[なみ]が 穏[おだ]やかです。	きょう の うみ は なみ が おだやか です	
\\	今日[きょう]の 海[うみ]は
\\	が 穏[おだ]やかです。			
\\	流れ	流[なが]れ	ながれ	
\\	川の上流は流れが速いよ。	川[かわ]の 上流[じょうりゅう]は 流[なが]れが 速[はや]いよ。	かわ の じょうりゅう は ながれ が はやい よ	
\\	川[かわ]の 上流[じょうりゅう]は
\\	が 速[はや]いよ。			
\\	流通	流通[りゅうつう]	りゅうつう	
\\	今日は流通の仕組みを勉強しましょう。	今日[きょう]は 流通[りゅうつう]の 仕組[しく]みを 勉強[べんきょう]しましょう。	きょう は りゅうつう の しくみ を べんきょう しましょう	
\\	今日[きょう]は
\\	の 仕組[しく]みを 勉強[べんきょう]しましょう。			
\\	流す	流[なが]す	ながす	
\\	彼女は涙を流したんだ。	彼女[かのじょ]は 涙[なみだ]を 流[なが]したんだ。	かのじょ は なみだ を ながした ん だ	
\\	彼女[かのじょ]は 涙[なみだ]を
\\	んだ。			
\\	流行	流行[はやり]	はやり	
\\	この服は今の流行です。	この 服[ふく]は 今[いま]の 流行[はやり]です。	この ふく は いま の はやり です	
\\	この 服[ふく]は 今[いま]の
\\	です。			
\\	流行	流行[りゅうこう]	りゅうこう	
\\	このスタイルは今年の流行です。	このスタイルは 今年[ことし]の 流行[りゅうこう]です。	この すたいる は ことし の りゅうこう です	
\\	このスタイルは 今年[ことし]の
\\	です。			
\\	洗面	洗面[せんめん]	せんめん	
\\	洗面用具を忘れた。	洗面[せんめん] 用具[ようぐ]を 忘[わす]れた。	せんめん ようぐ を わすれた	
\\	用具[ようぐ]を 忘[わす]れた。			
\\	手洗い	手洗[てあら]い	てあらい	
\\	風邪をひかないように手洗いとうがいをしましょう。	風邪[かぜ]をひかないように 手洗[てあら]いとうがいをしましょう。	かぜ を ひかない よう に てあらい と うがい を しましょう	
\\	風邪[かぜ]をひかないように
\\	とうがいをしましょう。			
\\	洗面器	洗面器[せんめんき]	せんめんき	
\\	洗面器でハンカチを洗ったの。	洗面器[せんめんき]でハンカチを 洗[あら]ったの。	せんめんき で はんかち を あらった の	
\\	でハンカチを 洗[あら]ったの。			
\\	沈没	沈没[ちんぼつ]	ちんぼつ	
\\	船は沈没しました。	船[ふね]は 沈没[ちんぼつ]しました。	ふね は ちんぼつ しました	
\\	船[ふね]は
\\	しました。			
\\	染める	染[そ]める	そめる	
\\	髪を赤に染めてみた。	髪[かみ]を 赤[あか]に 染[そ]めてみた。	かみ を あか に そめて みた	
\\	髪[かみ]を 赤[あか]に
\\	廃止	廃止[はいし]	はいし	
\\	その制度は廃止されました。	その 制度[せいど]は 廃止[はいし]されました。	その せいど は はいし されました	
\\	その 制度[せいど]は
\\	されました。			
\\	日光	日光[にっこう]	にっこう	
\\	この部屋は日光がよく当たるね。	この 部屋[へや]は 日光[にっこう]がよく 当[あ]たるね。	この へや は にっこう が よく あたる ね	
\\	この 部屋[へや]は
\\	がよく 当[あ]たるね。			
\\	振り返る	振[ふ]り 返[かえ]る	ふりかえる	
\\	学生時代を懐かしく振り返ったんだ。	学生時代[がくせい じだい]を 懐[なつ]かしく 振[ふ]り 返[かえ]ったんだ。	がくせい じだい を なつかしく ふりかえった ん だ	
\\	学生時代[がくせい じだい]を 懐[なつ]かしく
\\	んだ。			
\\	振り向く	振[ふ]り 向[む]く	ふりむく	
\\	彼女は振り向いて俺に微笑んだんだ。	彼女[かのじょ]は 振[ふ]り 向[む]いて 俺[おれ]に 微笑[ほほえ]んだんだ。	かのじょ は ふりむいて おれ に ほほえんだ ん だ	
\\	彼女[かのじょ]は
\\	俺[おれ]に 微笑[ほほえ]んだんだ。			
\\	振り	振[ふ]り	ふり	
\\	彼はバットの振りが大きすぎる。	彼[かれ]はバットの 振[ふ]りが 大[おお]きすぎる。	かれ は ばっと の ふり が おおき すぎる	
\\	彼[かれ]はバットの
\\	が 大[おお]きすぎる。			
\\	密か	密[ひそ]か	ひそか	
\\	彼女の誕生日パーティーを密かに計画しています。	彼女[かのじょ]の 誕生日[たんじょうび]パーティーを 密[ひそ]かに 計画[けいかく]しています。	かのじょ の たんじょうび ぱーてぃー を ひそか に けいかく して います	
\\	彼女[かのじょ]の 誕生日[たんじょうび]パーティーを
\\	に 計画[けいかく]しています。			
\\	寄せる	寄[よ]せる	よせる	
\\	車を塀に寄せたよ。	車[くるま]を 塀[へい]に 寄[よ]せたよ。	くるま を へい に よせた よ	
\\	車[くるま]を 塀[へい]に
\\	よ。			
\\	寄る	寄[よ]る	よる	
\\	帰りに叔母の家に寄ります。	帰[かえ]りに 叔母[おば]の 家[いえ]に 寄[よ]ります。	かえり に おば の いえ に よります	
\\	帰[かえ]りに 叔母[おば]の 家[いえ]に
\\	寄り道	寄[よ]り 道[みち]	よりみち	
\\	今日は寄り道してから帰ります。	今日[きょう]は 寄[よ]り 道[みち]してから 帰[かえ]ります。	きょう は よりみち して から かえります	
\\	今日[きょう]は
\\	してから 帰[かえ]ります。			
\\	寄り集まる	寄[よ]り 集[あつ]まる	よりあつまる	
\\	ニューヨークには芸術家が寄り集まっているの。	ニューヨークには 芸術家[げいじゅつか]が 寄[よ]り 集[あつ]まっているの。	にゅーよーく に は げいじゅつか が よりあつまって いる の	
\\	ニューヨークには 芸術家[げいじゅつか]が
\\	いるの。			
\\	歴史的	歴史的[れきしてき]	れきしてき	
\\	今日は歴史的な日です。	今日[きょう]は 歴史的[れきしてき]な 日[ひ]です。	きょう は れきしてき な ひ です	
\\	今日[きょう]は
\\	な 日[ひ]です。			
\\	宝石	宝石[ほうせき]	ほうせき	
\\	私が一番好きな宝石はダイヤモンドなの。	私[わたし]が 一番好[いちばん す]きな 宝石[ほうせき]はダイヤモンドなの。	わたし が いちばん すき な ほうせき は だいやもんど なの	
\\	私[わたし]が 一番好[いちばん す]きな
\\	はダイヤモンドなの。			
\\	建て前	建[た]て 前[まえ]	たてまえ	
\\	本音と建て前は違うことが多いよ。	本音[ほんね]と 建[た]て 前[まえ]は 違[ちが]うことが 多[おお]いよ。	ほんね と たてまえ は ちがう こと が おおい よ	
\\	本音[ほんね]と
\\	は 違[ちが]うことが 多[おお]いよ。			
\\	地位	地位[ちい]	ちい	
\\	彼女は会社で高い地位に就いているよ。	彼女[かのじょ]は 会社[かいしゃ]で 高[たか]い 地位[ちい]に 就[つ]いているよ。	かのじょ は かいしゃ で たかい ちい に ついて いる よ	
\\	彼女[かのじょ]は 会社[かいしゃ]で 高[たか]い
\\	に 就[つ]いているよ。			
\\	横切る	横切[よこぎ]る	よこぎる	
\\	目の前を猫が横切ったんだ。	目[め]の 前[まえ]を 猫[ねこ]が 横切[よこぎ]ったんだ。	め の まえ を ねこ が よこぎった ん だ	
\\	目[め]の 前[まえ]を 猫[ねこ]が
\\	んだ。			
\\	横顔	横顔[よこがお]	よこがお	
\\	彼女の横顔は素敵だ。	彼女[かのじょ]の 横顔[よこがお]は 素敵[すてき]だ。	かのじょ の よこがお は すてき だ	
\\	彼女[かのじょ]の
\\	は 素敵[すてき]だ。			
\\	断水	断水[だんすい]	だんすい	
\\	地震のために1週間、断水したの。	地震[じしん]のために1 週間[しゅうかん]、 断水[だんすい]したの。	じしん の ため に 
\\	しゅうかん だんすい した の	
\\	地震[じしん]のために1 週間[しゅうかん]、
\\	したの。			
\\	油断	油断[ゆだん]	ゆだん	
\\	少しの油断が大きな事故につながります。	少[すこ]しの 油断[ゆだん]が 大[おお]きな 事故[じこ]につながります。	すこし の ゆだん が おおき な じこ に つながります	
\\	少[すこ]しの
\\	が 大[おお]きな 事故[じこ]につながります。			
\\	断定	断定[だんてい]	だんてい	
\\	まだ原因は断定できません。	まだ 原因[げんいん]は 断定[だんてい]できません。	まだ げんいん は だんてい できません	
\\	まだ 原因[げんいん]は
\\	できません。			
\\	幅広い	幅広[はばひろ]い	はばひろい	
\\	彼は幅広い知識を持っています。	彼[かれ]は 幅広[はばひろ]い 知識[ちしき]を 持[も]っています。	かれ は はばひろい ちしき を もって います	
\\	彼[かれ]は
\\	知識[ちしき]を 持[も]っています。			
\\	幅	幅[はば]	はば	
\\	この道は幅が狭いので気をつけて運転してください。	この 道[みち]は 幅[はば]が 狭[せま]いので 気[き]をつけて 運転[うんてん]してください。	この みち は はば が せまい の で き を つけて うんてん して ください	
\\	この 道[みち]は
\\	が 狭[せま]いので 気[き]をつけて 運転[うんてん]してください。			
\\	大陸	大陸[たいりく]	たいりく	
\\	ユーラシアは世界で最も大きい大陸です。	ユーラシアは 世界[せかい]で 最[もっと]も 大[おお]きい 大陸[たいりく]です。	ゆーらしあ は せかい で もっとも おおきい たいりく です	
\\	ユーラシアは 世界[せかい]で 最[もっと]も 大[おお]きい
\\	です。			
\\	女房	女房[にょうぼう]	にょうぼう	
\\	女房は実家に帰っています。	女房[にょうぼう]は 実家[じっか]に 帰[かえ]っています。	にょうぼう は じっか に かえって います	
\\	は 実家[じっか]に 帰[かえ]っています。			
\\	復活	復活[ふっかつ]	ふっかつ	
\\	その選手は怪我を乗り越えて復活したわね。	その 選手[せんしゅ]は 怪我[けが]を 乗[の]り 越[こ]えて 復活[ふっかつ]したわね。	その せんしゅ は けが を のりこえて ふっかつ した わ ね	
\\	その 選手[せんしゅ]は 怪我[けが]を 乗[の]り 越[こ]えて
\\	したわね。			
\\	操作	操作[そうさ]	そうさ	
\\	この携帯電話は操作が簡単です。	この 携帯電話[けいたい でんわ]は 操作[そうさ]が 簡単[かんたん]です。	この けいたい でんわ は そうさ が かんたん です	
\\	この 携帯電話[けいたい でんわ]は
\\	が 簡単[かんたん]です。			
\\	操縦	操縦[そうじゅう]	そうじゅう	
\\	このボートは操縦が簡単です。	このボートは 操縦[そうじゅう]が 簡単[かんたん]です。	この ぼーと は そうじゅう が かんたん です	
\\	このボートは
\\	が 簡単[かんたん]です。			
\\	地帯	地帯[ちたい]	ちたい	
\\	この都市は工業地帯です。	この 都市[とし]は 工業[こうぎょう] 地帯[ちたい]です。	この とし は こうぎょう ちたい です	
\\	この 都市[とし]は 工業[こうぎょう]
\\	です。			
\\	戻す	戻[もど]す	もどす	
\\	話を戻しましょう。	話[はなし]を 戻[もど]しましょう。	はなし を もどしましょう	
\\	話[はなし]を
\\	払い戻す	払[はら]い 戻[もど]す	はらいもどす	
\\	飛行機の運賃が払い戻されたの。	飛行機[ひこうき]の 運賃[うんちん]が 払[はら]い 戻[もど]されたの。	ひこうき の うんちん が はらいもどされた の	
\\	飛行機[ひこうき]の 運賃[うんちん]が
\\	の。			
\\	寝かす	寝[ね]かす	ねかす	
\\	いつも9時に子供を寝かします。	いつも9 時[じ]に 子供[こども]を 寝[ね]かします。	いつも 
\\	じ に こども を ねかします	
\\	いつも9 時[じ]に 子供[こども]を
\\	寝過ごす	寝過[ねす]ごす	ねすごす	
\\	うっかり寝過ごしてしまったんだ。	うっかり 寝過[ねす]ごしてしまったんだ。	うっかり ねすごして しまった ん だ	
\\	うっかり
\\	んだ。			
\\	早寝	早寝[はやね]	はやね	
\\	早寝は健康のためによいことです。	早寝[はやね]は 健康[けんこう]のためによいことです。	はやね は けんこう の ため に よい こと です	
\\	は 健康[けんこう]のためによいことです。			
\\	寝かせる	寝[ね]かせる	ねかせる	
\\	赤ちゃんをベッドに寝かせた。	赤[あか]ちゃんをベッドに 寝[ね]かせた。	あかちゃん を べっど に ねかせた	
\\	赤[あか]ちゃんをベッドに
\\	寝転ぶ	寝転[ねころ]ぶ	ねころぶ	
\\	土手に寝転んで空をながめました。	土手[どて]に 寝転[ねころ]んで 空[そら]をながめました。	どて に ねころんで そら を ながめました	
\\	土手[どて]に
\\	空[そら]をながめました。			
\\	名付ける	名付[なづ]ける	なづける	
\\	子猫にトラと名付けました。	子猫[こねこ]にトラと 名付[なづ]けました。	こねこ に とら と なづけました	
\\	子猫[こねこ]にトラと
\\	日付け	日付[ひづ]け	ひづけ	
\\	今日の日付けは6月19日です。	今日[きょう]の 日付[ひづ]けは6 月19日[がつ 
\\	にち]です。	きょう の ひづけ は 
\\	がつ 
\\	にち です	
\\	今日[きょう]の
\\	は6 月19日[がつ 
\\	にち]です。			
\\	属する	属[ぞく]する	ぞくする	
\\	私は市民オーケストラに属しています。	私[わたし]は 市民[しみん]オーケストラに 属[ぞく]しています。	わたし は しみん おーけすとら に ぞくして います	
\\	私[わたし]は 市民[しみん]オーケストラに
\\	大概	大概[たいがい]	たいがい	
\\	大概、風邪は寝ていれば治ります。	大概[たいがい]、 風邪[かぜ]は 寝[ね]ていれば 治[なお]ります。	たいがい かぜ は ねて いれば なおります	
\\	、 風邪[かぜ]は 寝[ね]ていれば 治[なお]ります。			
\\	含める	含[ふく]める	ふくめる	
\\	私を含めて10人が参加しました。	私[わたし]を 含[ふく]めて10 人[にん]が 参加[さんか]しました。	わたし を ふくめて 
\\	にん が さんか しました	
\\	私[わたし]を
\\	人[にん]が 参加[さんか]しました。			
\\	含む	含[ふく]む	ふくむ	
\\	その食品は有害物質を含んでいるぞ。	その 食品[しょくひん]は 有害物質[ゆうがい ぶっしつ]を 含[ふく]んでいるぞ。	その しょくひん は ゆうがい ぶっしつ を ふくんで いる ぞ	
\\	その 食品[しょくひん]は 有害物質[ゆうがい ぶっしつ]を
\\	ぞ。			
\\	比例	比例[ひれい]	ひれい	
\\	努力と結果が比例していないの。	努力[どりょく]と 結果[けっか]が 比例[ひれい]していないの。	どりょく と けっか が ひれい して いない の	
\\	努力[どりょく]と 結果[けっか]が
\\	していないの。			
\\	富む	富[と]む	とむ	
\\	彼の人生は変化に富んでいるな。	彼[かれ]の 人生[じんせい]は 変化[へんか]に 富[と]んでいるな。	かれ の じんせい は へんか に とんで いる な	
\\	彼[かれ]の 人生[じんせい]は 変化[へんか]に
\\	な。			
\\	模様	模様[もよう]	もよう	
\\	彼女は水玉模様のスカートをはいているね。	彼女[かのじょ]は 水玉[みずたま] 模様[もよう]のスカートをはいているね。	かのじょ は みずたま もよう の すかーと を はいて いる ね	
\\	彼女[かのじょ]は 水玉[みずたま]
\\	のスカートをはいているね。			
\\	模範	模範[もはん]	もはん	
\\	彼は全校生徒の模範です。	彼[かれ]は 全校生徒[ぜんこう せいと]の 模範[もはん]です。	かれ は ぜんこう せいと の もはん です	
\\	彼[かれ]は 全校生徒[ぜんこう せいと]の
\\	です。			
\\	引き伸ばす	引[ひ]き 伸[の]ばす	ひきのばす	
\\	この写真を引き伸ばしてください。	この 写真[しゃしん]を 引[ひ]き 伸[の]ばしてください。	この しゃしん を ひきのばして ください	
\\	この 写真[しゃしん]を
\\	ください。			
\\	抜ける	抜[ぬ]ける	ぬける	
\\	彼はグループから抜けました。	彼[かれ]はグループから 抜[ぬ]けました。	かれ は ぐるーぷ から ぬけました	
\\	彼[かれ]はグループから
\\	抜く	抜[ぬ]く	ぬく	
\\	ワインのコルクを抜きました。	ワインのコルクを 抜[ぬ]きました。	わいん の こるく を ぬきました	
\\	ワインのコルクを
\\	昇る	昇[のぼ]る	のぼる	
\\	太陽は東から昇ります。	太陽[たいよう]は 東[ひがし]から 昇[のぼ]ります。	たいよう は ひがし から のぼります	
\\	太陽[たいよう]は 東[ひがし]から
\\	札	札[ふだ]	ふだ	
\\	店の外にまだ営業中の札がでているよ。	店[みせ]の 外[そと]にまだ 営業中[えいぎょう ちゅう]の 札[ふだ]がでているよ。	みせ の そと に まだ えいぎょう ちゅう の ふだ が でている よ	
\\	店[みせ]の 外[そと]にまだ 営業中[えいぎょう ちゅう]の
\\	がでているよ。			
\\	名札	名札[なふだ]	なふだ	
\\	生徒たちは校内では名札をつけます。	生徒[せいと]たちは 校内[こうない]では 名札[なふだ]をつけます。	せいとたち は こうない で は なふだ を つけます	
\\	生徒[せいと]たちは 校内[こうない]では
\\	をつけます。			
\\	日程	日程[にってい]	にってい	
\\	試験の日程が発表されました。	試験[しけん]の 日程[にってい]が 発表[はっぴょう]されました。	しけん の にってい が はっぴょう されました	
\\	試験[しけん]の
\\	が 発表[はっぴょう]されました。			
\\	導く	導[みちび]く	みちびく	
\\	先生は私たちを導いてくれます。	先生[せんせい]は 私[わたし]たちを 導[みちび]いてくれます。	せんせい は わたしたち を みちびいて くれます	
\\	先生[せんせい]は 私[わたし]たちを
\\	くれます。			
\\	望む	望[のぞ]む	のぞむ	
\\	彼は私との結婚を望んでいます。	彼[かれ]は 私[わたし]との 結婚[けっこん]を 望[のぞ]んでいます。	かれ は わたし と の けっこん を のぞんで います	
\\	彼[かれ]は 私[わたし]との 結婚[けっこん]を
\\	望ましい	望[のぞ]ましい	のぞましい	
\\	夜は10時までに寝るのが望ましいの。	夜[よる]は10 時[じ]までに 寝[ね]るのが 望[のぞ]ましいの。	よる は 
\\	じ まで に ねる の が のぞましい の	
\\	夜[よる]は10 時[じ]までに 寝[ね]るのが
\\	の。			
\\	望み	望[のぞ]み	のぞみ	
\\	私の望みは海外で暮らすことです。	私[わたし]の 望[のぞ]みは 海外[かいがい]で 暮[く]らすことです。	わたし の のぞみ は かいがい で くらす こと です	
\\	私[わたし]の
\\	は 海外[かいがい]で 暮[く]らすことです。			
\\	有望	有望[ゆうぼう]	ゆうぼう	
\\	彼は有望な社員です。	彼[かれ]は 有望[ゆうぼう]な 社員[しゃいん]です。	かれ は ゆうぼう な しゃいん です	
\\	彼[かれ]は
\\	な 社員[しゃいん]です。			
\\	欲望	欲望[よくぼう]	よくぼう	
\\	彼は欲望が強い人です。	彼[かれ]は 欲望[よくぼう]が 強[つよ]い 人[ひと]です。	かれ は よくぼう が つよい ひと です	
\\	彼[かれ]は
\\	が 強[つよ]い 人[ひと]です。			
\\	待ち望む	待[ま]ち 望[のぞ]む	まちのぞむ	
\\	その国の人々は平和を待ち望んでいるの。	その 国[くに]の 人々[ひとびと]は 平和[へいわ]を 待[ま]ち 望[のぞ]んでいるの。	その くに の ひとびと は へいわ を まちのぞんで いる の	
\\	その 国[くに]の 人々[ひとびと]は 平和[へいわ]を
\\	の。			
\\	夫人	夫人[ふじん]	ふじん	
\\	スミス夫人がいらっしゃいました。	スミス 夫人[ふじん]がいらっしゃいました。	すみすふじん が いらっしゃいました	
\\	スミス
\\	がいらっしゃいました。			
\\	婦人	婦人[ふじん]	ふじん	
\\	婦人服売り場は5階でございます。	婦人[ふじん] 服売[ふく う]り 場[ば]は5 階[かい]でございます。	ふじんふく うりば は 
\\	かい で ございます	
\\	服売[ふく う]り 場[ば]は5 階[かい]でございます。			
\\	夫妻	夫妻[ふさい]	ふさい	
\\	昨日の夜、社長ご夫妻と食事をしました。	昨日[きのう]の 夜[よる]、 社長[しゃちょう]ご 夫妻[ふさい]と 食事[しょくじ]をしました。	きのう の よる しゃちょうごふさい と しょくじ を しました	
\\	昨日[きのう]の 夜[よる]、 社長[しゃちょう]ご
\\	と 食事[しょくじ]をしました。			
\\	我々	我々[われわれ]	われわれ	
\\	我々の決意は固いです。	我々[われわれ]の 決意[けつい]は 固[かた]いです。	われわれ の けつい は かたい です	
\\	の 決意[けつい]は 固[かた]いです。			
\\	我が国	我[わ]が 国[くに]	わがくに	
\\	彼は我が国を代表する作家です。	彼[かれ]は 我[わ]が 国[くに]を 代表[だいひょう]する 作家[さっか]です。	かれ は わがくに を だいひょう する さっか です	
\\	彼[かれ]は
\\	を 代表[だいひょう]する 作家[さっか]です。			
\\	我が家	我[わ]が 家[や]	わがや	
\\	ぜひ我が家に遊びに来てください。	ぜひ 我[わ]が 家[や]に 遊[あそ]びに 来[き]てください。	ぜひ わがや に あそび に きて ください 。	
\\	ぜひ
\\	に 遊[あそ]びに 来[き]てください。			
\\	年齢	年齢[ねんれい]	ねんれい	
\\	彼女の年齢は27です。	彼女[かのじょ]の 年齢[ねんれい]は27です。	かのじょ の ねんれい は 
\\	です	
\\	彼女[かのじょ]の
\\	は27です。			
\\	恋愛	恋愛[れんあい]	れんあい	
\\	彼女は恋愛にあこがれる年ごろです。	彼女[かのじょ]は 恋愛[れんあい]にあこがれる 年[とし]ごろです。	かのじょ は れんあい に あこがれる とし ごろ です	
\\	彼女[かのじょ]は
\\	にあこがれる 年[とし]ごろです。			
\\	延びる	延[の]びる	のびる	
\\	工事の予定が1ヶ月延びてしまった。	工事[こうじ]の 予定[よてい]が1 ヶ月[かげつ] 延[の]びてしまった。	こうじ の よてい が 
\\	かげつ のびて しまった	
\\	工事[こうじ]の 予定[よてい]が1 ヶ月[かげつ]
\\	しまった。			
\\	延ばす	延[の]ばす	のばす	
\\	出発を一週間延ばしたの。	出発[しゅっぱつ]を 一週間[いっしゅうかん] 延[の]ばしたの。	しゅっぱつ を いっしゅうかん のばした の	
\\	出発[しゅっぱつ]を 一週間[いっしゅうかん]
\\	の。			
\\	引き延ばす	引[ひ]き 延[の]ばす	ひきのばす	
\\	司会者は話を引き延ばしたわ。	司会者[しかいしゃ]は 話[はなし]を 引[ひ]き 延[の]ばしたわ。	しかいしゃ は はなし を ひきのばした わ	
\\	司会者[しかいしゃ]は 話[はなし]を
\\	わ。			
\\	夢中	夢中[むちゅう]	むちゅう	
\\	うちの子はその本に夢中です。	うちの 子[こ]はその 本[ほん]に 夢中[むちゅう]です。	うち の こ は その ほん に むちゅう です	
\\	うちの 子[こ]はその 本[ほん]に
\\	です。			
\\	泣き顔	泣[な]き 顔[がお]	なきがお	
\\	彼女は泣き顔になったの。	彼女[かのじょ]は 泣[な]き 顔[がお]になったの。	かのじょ は なきがお に なった の	
\\	彼女[かのじょ]は
\\	になったの。			
\\	喜び	喜[よろこ]び	よろこび	
\\	人々は喜びに沸いた。	人々[ひとびと]は 喜[よろこ]びに 沸[わ]いた。	ひとびと は よろこび に わいた	
\\	人々[ひとびと]は
\\	に 沸[わ]いた。			
\\	喜ばす	喜[よろこ]ばす	よろこばす	
\\	私は人を喜ばすのが大好きです。	私[わたし]は 人[ひと]を 喜[よろこ]ばすのが 大好[だいす]きです。	わたし は ひと を よろこばす の が だいすき です	
\\	私[わたし]は 人[ひと]を
\\	のが 大好[だいす]きです。			
\\	恥	恥[はじ]	はじ	
\\	間違えることは恥ではありません。	間違[まちが]えることは 恥[はじ]ではありません。	まちがえる こと は はじ で は ありません	
\\	間違[まちが]えることは
\\	ではありません。			
\\	弁論	弁論[べんろん]	べんろん	
\\	弁論大会で優勝したことがあります。	弁論[べんろん] 大会[たいかい]で 優勝[ゆうしょう]したことがあります。	べんろん たいかい で ゆうしょう した こと が あります	
\\	大会[たいかい]で 優勝[ゆうしょう]したことがあります。			
\\	弁護	弁護[べんご]	べんご	
\\	友人が私を弁護してくれました。	友人[ゆうじん]が 私[わたし]を 弁護[べんご]してくれました。	ゆうじん が わたし を べんご して くれました	
\\	友人[ゆうじん]が 私[わたし]を
\\	してくれました。			
\\	同士	同士[どうし]	どうし	
\\	彼と私はいとこ同士です。	彼[かれ]と 私[わたし]はいとこ 同士[どうし]です。	かれ と わたし は いとこ どうし です	
\\	彼[かれ]と 私[わたし]はいとこ
\\	です。			
\\	弁護士	弁護士[べんごし]	べんごし	
\\	父は弁護士です。	父[ちち]は 弁護士[べんごし]です。	ちち は べんごし です	
\\	父[ちち]は
\\	です。			
\\	否定	否定[ひてい]	ひてい	
\\	彼、友達の意見を否定した。	彼[かれ]、 友達[ともだち]の 意見[いけん]を 否定[ひてい]した。	かれ ともだち の いけん を ひてい した	
\\	彼[かれ]、 友達[ともだち]の 意見[いけん]を
\\	した。			
\\	文化財	文化財[ぶんかざい]	ぶんかざい	
\\	この建物は国の文化財です。	この 建物[たてもの]は 国[くに]の 文化財[ぶんかざい]です。	この たてもの は くに の ぶんかざい です	
\\	この 建物[たてもの]は 国[くに]の
\\	です。			
\\	得する	得[とく]する	とくする	
\\	ネットで得する情報を見つけたよ。	ネットで 得[とく]する 情報[じょうほう]を 見[み]つけたよ。	ねっと で とく する じょうほう を みつけた よ	
\\	ネットで
\\	情報[じょうほう]を 見[み]つけたよ。			
\\	得	得[とく]	とく	
\\	この車を今買うとお得ですよ。	この 車[くるま]を 今買[いま か]うとお 得[とく]ですよ。	この くるま を いま かう と お とく です よ	
\\	この 車[くるま]を 今買[いま か]うとお
\\	ですよ。			
\\	損	損[そん]	そん	
\\	パチンコで5000円損しました。	パチンコで5000 円[えん] 損[そん]しました。	ぱちんこ で 
\\	えん そん しました	
\\	パチンコで5000 円[えん]
\\	しました。			
\\	損害	損害[そんがい]	そんがい	
\\	町は台風で大きな損害を受けたんだ。	町[まち]は 台風[たいふう]で 大[おお]きな 損害[そんがい]を 受[う]けたんだ。	まち は たいふう で おおきな そんがい を うけた ん だ	
\\	町[まち]は 台風[たいふう]で 大[おお]きな
\\	を 受[う]けたんだ。			
\\	損する	損[そん]する	そんする	
\\	わざわざ行って損した。	わざわざ 行[い]って 損[そん]した。	わざわざ いって そん した	
\\	わざわざ 行[い]って
\\	日焼け	日焼[ひや]け	ひやけ	
\\	海で日焼けしたんだ。	海[うみ]で 日焼[ひや]けしたんだ。	うみ で ひやけ した ん だ	
\\	海[うみ]で
\\	したんだ。			
\\	夕焼け	夕焼[ゆうや]け	ゆうやけ	
\\	今日は夕焼けがきれいです。	今日[きょう]は 夕焼[ゆうや]けがきれいです。	きょう は ゆうやけ が きれい です	
\\	今日[きょう]は
\\	がきれいです。			
\\	幹	幹[みき]	みき	
\\	この木の幹はとても太いよ。	この 木[き]の 幹[みき]はとても 太[ふと]いよ。	この き の みき は とても ふとい よ	
\\	この 木[き]の
\\	はとても 太[ふと]いよ。			
\\	散らばる	散[ち]らばる	ちらばる	
\\	机の上に書類が散らばっている。	机[つくえ]の 上[うえ]に 書類[しょるい]が 散[ち]らばっている。	つくえ の うえ に しょるい が ちらばって いる	
\\	机[つくえ]の 上[うえ]に 書類[しょるい]が
\\	散る	散[ち]る	ちる	
\\	風で桜の花が散ってるね。	風[かぜ]で 桜[さくら]の 花[はな]が 散[ち]ってるね。	かぜ で さくら の はな が ちってる ね	
\\	風[かぜ]で 桜[さくら]の 花[はな]が
\\	ね。			
\\	散らかる	散[ち]らかる	ちらかる	
\\	弟の部屋はいつも散らかっているんだ。	弟[おとうと]の 部屋[へや]はいつも 散[ち]らかっているんだ。	おとうと の へや は いつも ちらかって いる ん だ	
\\	弟[おとうと]の 部屋[へや]はいつも
\\	んだ。			
\\	散らかす	散[ち]らかす	ちらかす	
\\	部屋を散らかさないでください。	部屋[へや]を 散[ち]らかさないでください。	へや を ちらかさない で ください	
\\	部屋[へや]を
\\	ください。			
\\	根	根[ね]	ね	
\\	この木の根はとても太いな。	この 木[き]の 根[ね]はとても 太[ふと]いな。	この き の ね は とても ふとい な	
\\	この 木[き]の
\\	はとても 太[ふと]いな。			
\\	大根	大根[だいこん]	だいこん	
\\	大根は白くて長い野菜です。	大根[だいこん]は 白[しろ]くて 長[なが]い 野菜[やさい]です。	だいこん は しろくて ながい やさい です	
\\	は 白[しろ]くて 長[なが]い 野菜[やさい]です。			
\\	呼び出す	呼[よ]び 出[だ]す	よびだす	
\\	親が学校に呼び出されたんだ。	親[おや]が 学校[がっこう]に 呼[よ]び 出[だ]されたんだ。	おや が がっこう に よびだされた ん だ	
\\	親[おや]が 学校[がっこう]に
\\	んだ。			
\\	同級生	同級生[どうきゅうせい]	どうきゅうせい	
\\	私たちは同級生です。	私[わたし]たちは 同級生[どうきゅうせい]です。	わたしたち は どうきゅうせい です	
\\	私[わたし]たちは
\\	です。			
\\	容易	容易[ようい]	ようい	
\\	彼はその問題を容易に解決したわ。	彼[かれ]はその 問題[もんだい]を 容易[ようい]に 解決[かいけつ]したわ。	かれ は その もんだい を ようい に かいけつ した わ	
\\	彼[かれ]はその 問題[もんだい]を
\\	に 解決[かいけつ]したわ。			
\\	容器	容器[ようき]	ようき	
\\	容器のふたはきちんと閉めましょう。	容器[ようき]のふたはきちんと 閉[し]めましょう。	ようき の ふた は きちんと しめましょう	
\\	のふたはきちんと 閉[し]めましょう。			
\\	治療	治療[ちりょう]	ちりょう	
\\	今、歯を治療しています。	今[いま]、 歯[は]を 治療[ちりょう]しています。	いま は を ちりょう して います	
\\	今[いま]、 歯[は]を
\\	しています。			
\\	毒	毒[どく]	どく	
\\	飲み過ぎは体に毒ですよ。	飲[の]み 過[す]ぎは 体[からだ]に 毒[どく]ですよ。	のみ すぎ は からだ に どく です よ	
\\	飲[の]み 過[す]ぎは 体[からだ]に
\\	ですよ。			
\\	有毒	有毒[ゆうどく]	ゆうどく	
\\	その工場は有毒ガスを出しているのね。	その 工場[こうじょう]は 有毒[ゆうどく]ガスを 出[だ]しているのね。	その こうじょう は ゆうどくがす を だして いる の ね	
\\	その 工場[こうじょう]は
\\	ガスを 出[だ]しているのね。			
\\	放射能	放射能[ほうしゃのう]	ほうしゃのう	
\\	この地区は放射能に汚染されたんだ。	この 地区[ちく]は 放射能[ほうしゃのう]に 汚染[おせん]されたんだ。	この ちく は ほうしゃのう に おせん された ん だ	
\\	この 地区[ちく]は
\\	に 汚染[おせん]されたんだ。			
\\	放す	放[はな]す	はなす	
\\	公園で犬を放したの。	公園[こうえん]で 犬[いぬ]を 放[はな]したの。	こうえん で いぬ を はなした の	
\\	公園[こうえん]で 犬[いぬ]を
\\	の。			
\\	放る	放[ほう]る	ほうる	
\\	ボールを空中に放ったの。	ボールを 空中[くうちゅう]に 放[ほう]ったの。	ぼーる を くうちゅう に ほうった の	
\\	ボールを 空中[くうちゅう]に
\\	の。			
\\	服装	服装[ふくそう]	ふくそう	
\\	そのパーティーはカジュアルな服装で大丈夫です。	そのパーティーはカジュアルな 服装[ふくそう]で 大丈夫[だいじょうぶ]です。	その ぱーてぃー は かじゅある な ふくそう で だいじょうぶ です	
\\	そのパーティーはカジュアルな
\\	で 大丈夫[だいじょうぶ]です。			
\\	振り仮名	振[ふ]り 仮名[がな]	ふりがな	
\\	名前に振り仮名をつけてください。	名前[なまえ]に 振[ふ]り 仮名[がな]をつけてください。	なまえ に ふりがな を つけて ください	
\\	名前[なまえ]に
\\	をつけてください。			
\\	悲劇	悲劇[ひげき]	ひげき	
\\	あの悲劇を繰り返してはいけない。	あの 悲劇[ひげき]を 繰[く]り 返[かえ]してはいけない。	あの ひげき を くりかえして は いけない	
\\	あの
\\	を 繰[く]り 返[かえ]してはいけない。			
\\	団体	団体[だんたい]	だんたい	
\\	サッカーは団体競技です。	サッカーは 団体[だんたい] 競技[きょうぎ]です。	さっかー は だんたい きょうぎ です	
\\	サッカーは
\\	競技[きょうぎ]です。			
\\	団地	団地[だんち]	だんち	
\\	私の弟は団地に住んでいます。	私[わたし]の 弟[おとうと]は 団地[だんち]に 住[す]んでいます。	わたし の おとうと は だんち に すんで います	
\\	私[わたし]の 弟[おとうと]は
\\	に 住[す]んでいます。			
\\	展開	展開[てんかい]	てんかい	
\\	話の展開についていけない。	話[はなし]の 展開[てんかい]についていけない。	はなし の てんかい に ついていけない	
\\	話[はなし]の
\\	についていけない。			
\\	微妙	微妙[びみょう]	びみょう	
\\	彼は会社で微妙な立場にあります。	彼[かれ]は 会社[かいしゃ]で 微妙[びみょう]な 立場[たちば]にあります。	かれ は かいしゃ で びみょう な たちば に あります	
\\	彼[かれ]は 会社[かいしゃ]で
\\	な 立場[たちば]にあります。			
\\	名詞	名詞[めいし]	めいし	
\\	「学校」は名詞です。	
\\	学校」[がっこう]は 名詞[めいし]です。	がっこう は めいし です	
\\	学校」[がっこう]は
\\	です。			
\\	手伝い	手伝[てつだ]い	てつだい	
\\	会議の準備に手伝いが必要です。	会議[かいぎ]の 準備[じゅんび]に 手伝[てつだ]いが 必要[ひつよう]です。	かいぎ の じゅんび に てつだい が ひつよう です	
\\	会議[かいぎ]の 準備[じゅんび]に
\\	が 必要[ひつよう]です。			
\\	悲鳴	悲鳴[ひめい]	ひめい	
\\	外から悲鳴が聞こえたな。	外[そと]から 悲鳴[ひめい]が 聞[き]こえたな。	そと から ひめい が きこえた な	
\\	外[そと]から
\\	が 聞[き]こえたな。			
\\	泣き声	泣[な]き 声[ごえ]	なきごえ	
\\	赤ちゃんの泣き声が聞こえますね。	赤[あか]ちゃんの 泣[な]き 声[ごえ]が 聞[き]こえますね。	あかちゃん の なきごえ が きこえます ね	
\\	赤[あか]ちゃんの
\\	が 聞[き]こえますね。			
\\	名刺	名刺[めいし]	めいし	
\\	私たちは名刺を交換しました。	私[わたし]たちは 名刺[めいし]を 交換[こうかん]しました。	わたしたち は めいし を こうかん しました	
\\	私[わたし]たちは
\\	を 交換[こうかん]しました。			
\\	斜め	斜[なな]め	ななめ	
\\	ここに斜めに線を引いてください。	ここに 斜[なな]めに 線[せん]を 引[ひ]いてください。	ここ に ななめ に せん を ひいて ください	
\\	ここに
\\	に 線[せん]を 引[ひ]いてください。			
\\	柔らかい	柔[やわ]らかい	やわらかい	
\\	柔らかい日差しが気持ちいいね。	柔[やわ]らかい 日差[ひざ]しが 気持[きも]ちいいね。	やわらかい ひざし が きもち いい ね	
\\	日差[ひざ]しが 気持[きも]ちいいね。			
\\	張る	張[は]る	はる	
\\	疲れて肩が張っています。	疲[つか]れて 肩[かた]が 張[は]っています。	つかれて かた が はって います	
\\	疲[つか]れて 肩[かた]が
\\	張り切る	張[は]り 切[き]る	はりきる	
\\	母は張り切ってお弁当を用意したの。	母[はは]は 張[は]り 切[き]ってお 弁当[べんとう]を 用意[ようい]したの。	はは は はりきって おべんとう を ようい した の	
\\	母[はは]は
\\	お 弁当[べんとう]を 用意[ようい]したの。			
\\	欲張り	欲張[よくば]り	よくばり	
\\	彼女は欲張りです。	彼女[かのじょ]は 欲張[よくば]りです。	かのじょ は よくばり です	
\\	彼女[かのじょ]は
\\	です。			
\\	墜落	墜落[ついらく]	ついらく	
\\	飛行機の墜落事故があったんだ。	飛行機[ひこうき]の 墜落[ついらく] 事故[じこ]があったんだ。	ひこうき の ついらく じこ が あった ん だ	
\\	飛行機[ひこうき]の
\\	事故[じこ]があったんだ。			
\\	武器	武器[ぶき]	ぶき	
\\	彼らは武器を取り、立ち上がった。	彼[かれ]らは 武器[ぶき]を 取[と]り、 立[た]ち 上[あ]がった。	かれら は ぶき を とり たちあがった	
\\	彼[かれ]らは
\\	を 取[と]り、 立[た]ち 上[あ]がった。			
\\	武士	武士[ぶし]	ぶし	
\\	彼の家柄は武士でした。	彼[かれ]の 家柄[いえがら]は 武士[ぶし]でした。	かれ の いえがら は ぶし でした	
\\	彼[かれ]の 家柄[いえがら]は
\\	でした。			
\\	弾	弾[たま]	たま	
\\	彼はピストルに弾を込めたんだ。	彼[かれ]はピストルに 弾[たま]を 込[こ]めたんだ。	かれ は ぴすとる に たま を こめた ん だ	
\\	彼[かれ]はピストルに
\\	を 込[こ]めたんだ。			
\\	弾む	弾[はず]む	はずむ	
\\	このボールはよく弾みますね。	このボールはよく 弾[はず]みますね。	この ぼーる は よく はずみます ね	
\\	このボールはよく
\\	ね。			
\\	日の丸	日[ひ]の 丸[まる]	ひのまる	
\\	日本の国旗は日の丸と呼ばれています。	日本[にっぽん]の 国旗[こっき]は 日[ひ]の 丸[まる]と 呼[よ]ばれています。	にっぽん の こっき は ひのまる と よばれて います	
\\	日本[にっぽん]の 国旗[こっき]は
\\	と 呼[よ]ばれています。			
\\	暴落	暴落[ぼうらく]	ぼうらく	
\\	昨日株価が暴落しました。	昨日株価[きのう かぶか]が 暴落[ぼうらく]しました。	きのう かぶか が ぼうらく しました	
\\	昨日株価[きのう かぶか]が
\\	しました。			
\\	暴力	暴力[ぼうりょく]	ぼうりょく	
\\	暴力はいけません。	暴力[ぼうりょく]はいけません。	ぼうりょく は いけません	
\\	はいけません。			
\\	妨害	妨害[ぼうがい]	ぼうがい	
\\	彼に営業を妨害されました。	彼[かれ]に 営業[えいぎょう]を 妨害[ぼうがい]されました。	かれ に えいぎょう を ぼうがい されました	
\\	彼[かれ]に 営業[えいぎょう]を
\\	されました。			
\\	徹夜	徹夜[てつや]	てつや	
\\	ゆうべは徹夜しました。	ゆうべは 徹夜[てつや]しました。	ゆうべ は てつや しました	
\\	ゆうべは
\\	しました。			
\\	底	底[そこ]	そこ	
\\	コップの底が濡れていますよ。	コップの 底[そこ]が 濡[ぬ]れていますよ。	こっぷ の そこ が ぬれて います よ	
\\	コップの
\\	が 濡[ぬ]れていますよ。			
\\	徹底的	徹底的[てっていてき]	てっていてき	
\\	部屋の中を徹底的に探しました。	部屋[へや]の 中[なか]を 徹底的[てっていてき]に 探[さが]しました。	へや の なか を てっていてき に さがしました	
\\	部屋[へや]の 中[なか]を
\\	に 探[さが]しました。			
\\	抵抗	抵抗[ていこう]	ていこう	
\\	犯人は警察に抵抗したの。	犯人[はんにん]は 警察[けいさつ]に 抵抗[ていこう]したの。	はんにん は けいさつ に ていこう した の	
\\	犯人[はんにん]は 警察[けいさつ]に
\\	したの。			
\\	敵	敵[てき]	てき	
\\	あそこに敵がひそんでいる。	あそこに 敵[てき]がひそんでいる。	あそこ に てき が ひそんで いる	
\\	あそこに
\\	がひそんでいる。			
\\	態勢	態勢[たいせい]	たいせい	
\\	作業を始める態勢は整っています。	作業[さぎょう]を 始[はじ]める 態勢[たいせい]は 整[ととの]っています。	さぎょう を はじめる たいせい は ととのって います	
\\	作業[さぎょう]を 始[はじ]める
\\	は 整[ととの]っています。			
\\	文系	文系[ぶんけい]	ぶんけい	
\\	彼女は文系です。	彼女[かのじょ]は 文系[ぶんけい]です。	かのじょ は ぶんけい です	
\\	彼女[かのじょ]は
\\	です。			
\\	日韓	日韓[にっかん]	にっかん	
\\	日韓合同のコンサートが開かれたよ。	日韓[にっかん] 合同[ごうどう]のコンサートが 開[ひら]かれたよ。	にっかん ごうどう の こんさーと が ひらかれた よ	
\\	合同[ごうどう]のコンサートが 開[ひら]かれたよ。			
\\	枠	枠[わく]	わく	
\\	枠の中に答えを書いてください。	枠[わく]の 中[なか]に 答[こた]えを 書[か]いてください。	わく の なか に こたえ を かいて ください	
\\	の 中[なか]に 答[こた]えを 書[か]いてください。			
\\	棒	棒[ぼう]	ぼう	
\\	この棒は何に使うのですか。	この 棒[ぼう]は 何[なに]に 使[つか]うのですか。	この ぼう は なに に つかう の です か	
\\	この
\\	は 何[なに]に 使[つか]うのですか。			
\\	沿う	沿[そ]う	そう	
\\	川に沿って歩いたんだ。	川[かわ]に 沿[そ]って 歩[ある]いたんだ。	かわ に そって あるいた ん だ	
\\	川[かわ]に
\\	歩[ある]いたんだ。			
\\	浜	浜[はま]	はま	
\\	今晩、浜で花火大会がありますよ。	今晩[こんばん]、 浜[はま]で 花火大会[はなび たいかい]がありますよ。	こんばん はま で はなび たいかい が あります よ	
\\	今晩[こんばん]、
\\	で 花火大会[はなび たいかい]がありますよ。			
\\	浜辺	浜辺[はまべ]	はまべ	
\\	浜辺できれいな貝がらを拾いました。	浜辺[はまべ]できれいな 貝[かい]がらを 拾[ひろ]いました。	はまべ で きれい な かいがら を ひろいました	
\\	できれいな 貝[かい]がらを 拾[ひろ]いました。			
\\	泥	泥[どろ]	どろ	
\\	靴が泥だらけになったよ。	靴[くつ]が 泥[どろ]だらけになったよ。	くつ が どろだらけ に なった よ	
\\	靴[くつ]が
\\	だらけになったよ。			
\\	吐く	吐[は]く	はく	
\\	彼は乱暴な言葉を吐いたぞ。	彼[かれ]は 乱暴[らんぼう]な 言葉[ことば]を 吐[は]いたぞ。	かれ は らんぼう な ことば を はいた ぞ	
\\	彼[かれ]は 乱暴[らんぼう]な 言葉[ことば]を
\\	ぞ。			
\\	嘆く	嘆[なげ]く	なげく	
\\	嘆いていても何も変わりません。	嘆[なげ]いていても 何[なに]も 変[か]わりません。	なげいていて も なに も かわりません	
\\	も 何[なに]も 変[か]わりません。			
\\	忠実	忠実[ちゅうじつ]	ちゅうじつ	
\\	犬は飼い主に忠実です。	犬[いぬ]は 飼[か]い 主[ぬし]に 忠実[ちゅうじつ]です。	いぬ は かいぬし に ちゅうじつ です	
\\	犬[いぬ]は 飼[か]い 主[ぬし]に
\\	です。			
\\	忠告	忠告[ちゅうこく]	ちゅうこく	
\\	先生からの忠告を聞くべきだよ。	先生[せんせい]からの 忠告[ちゅうこく]を 聞[き]くべきだよ。	せんせい から の ちゅうこく を きく べき だ よ	
\\	先生[せんせい]からの
\\	を 聞[き]くべきだよ。			
\\	恵まれる	恵[めぐ]まれる	めぐまれる	
\\	当日は天気に恵まれました。	当日[とうじつ]は 天気[てんき]に 恵[めぐ]まれました。	とうじつ は てんき に めぐまれました	
\\	当日[とうじつ]は 天気[てんき]に
\\	同居	同居[どうきょ]	どうきょ	
\\	私はまだ両親と同居しています。	私[わたし]はまだ 両親[りょうしん]と 同居[どうきょ]しています。	わたし は まだ りょうしん と どうきょ して います	
\\	私[わたし]はまだ 両親[りょうしん]と
\\	しています。			
\\	掘る	掘[ほ]る	ほる	
\\	ここに穴を掘りましょう。	ここに 穴[あな]を 掘[ほ]りましょう。	ここ に あな を ほりましょう	
\\	ここに 穴[あな]を
\\	塀	塀[へい]	へい	
\\	猫が塀の上で寝ているぞ。	猫[ねこ]が 塀[へい]の 上[うえ]で 寝[ね]ているぞ。	ねこ が へい の うえ で ねて いる ぞ	
\\	猫[ねこ]が
\\	の 上[うえ]で 寝[ね]ているぞ。			
\\	大層	大層[たいそう]	たいそう	
\\	彼は大層喜んでいました。	彼[かれ]は 大層[たいそう] 喜[よろこ]んでいました。	かれ は たいそう よろこんで いました	
\\	彼[かれ]は
\\	喜[よろこ]んでいました。			
\\	履歴	履歴[りれき]	りれき	
\\	最近使ったファイルは、履歴からすぐ開けます。	最近使[さいきん つか]ったファイルは、 履歴[りれき]からすぐ 開[ひら]けます。	さいきん つかった ふぁいる は、 りれき から すぐ ひらけます	
\\	最近使[さいきん つか]ったファイルは、
\\	からすぐ 開[ひら]けます。			
\\	履歴書	履歴書[りれきしょ]	りれきしょ	
\\	面接のために履歴書を書きました。	面接[めんせつ]のために 履歴書[りれきしょ]を 書[か]きました。	めんせつ の ため に りれきしょ を かきました	
\\	面接[めんせつ]のために
\\	を 書[か]きました。			
\\	履物	履物[はきもの]	はきもの	
\\	履物は靴箱に入れてください。	履物[はきもの]は 靴箱[くつばこ]に 入[い]れてください。	はきもの は くつばこ に いれて ください	
\\	は 靴箱[くつばこ]に 入[い]れてください。			
\\	戸棚	戸棚[とだな]	とだな	
\\	この皿を戸棚にしまってください。	この 皿[さら]を 戸棚[とだな]にしまってください。	この さら を とだな に しまって ください	
\\	この 皿[さら]を
\\	にしまってください。			
\\	扉	扉[とびら]	とびら	
\\	彼は扉を開けたんだ。	彼[かれ]は 扉[とびら]を 開[あ]けたんだ。	かれ は とびら を あけた ん だ	
\\	彼[かれ]は
\\	を 開[あ]けたんだ。			
\\	殴る	殴[なぐ]る	なぐる	
\\	彼は思わず友人を殴ったの。	彼[かれ]は 思[おも]わず 友人[ゆうじん]を 殴[なぐ]ったの。	かれ は おもわず ゆうじん を なぐった の	
\\	彼[かれ]は 思[おも]わず 友人[ゆうじん]を
\\	の。			
\\	大胆	大胆[だいたん]	だいたん	
\\	彼女はずいぶん大胆なことを言うね。	彼女[かのじょ]はずいぶん 大胆[だいたん]なことを 言[い]うね。	かのじょ は ずいぶん だいたん な こと を いう ね	
\\	彼女[かのじょ]はずいぶん
\\	なことを 言[い]うね。			
\\	幕	幕[まく]	まく	
\\	ステージの幕が上がったんだ。	ステージの 幕[まく]が 上[あ]がったんだ。	すてーじ の まく が あがった ん だ	
\\	ステージの
\\	が 上[あ]がったんだ。			
\\	夕暮れ	夕暮[ゆうぐ]れ	ゆうぐれ	
\\	夕暮れの空がきれいですね。	夕暮[ゆうぐ]れの 空[そら]がきれいですね。	ゆうぐれ の そら が きれい です ね	
\\	の 空[そら]がきれいですね。			
\\	墓	墓[はか]	はか	
\\	祖父の墓は近くにあります。	祖父[そふ]の 墓[はか]は 近[ちか]くにあります。	そふ の はか は ちかく に あります	
\\	祖父[そふ]の
\\	は 近[ちか]くにあります。			
\\	墓地	墓地[ぼち]	ぼち	
\\	寺のとなりに墓地があります。	寺[てら]のとなりに 墓地[ぼち]があります。	てら の となり に ぼち が あります	
\\	寺[てら]のとなりに
\\	があります。			
\\	墓参り	墓参[はかまい]り	はかまいり	
\\	明日は家族で墓参りに行きます。	明日[あした]は 家族[かぞく]で 墓参[はかまい]りに 行[い]きます。	あした は かぞく で はかまいり に いきます	
\\	明日[あした]は 家族[かぞく]で
\\	に 行[い]きます。			
\\	憎む	憎[にく]む	にくむ	
\\	彼はもう彼女を憎んではいないよ。	彼[かれ]はもう 彼女[かのじょ]を 憎[にく]んではいないよ。	かれ は もう かのじょ を にくんで は いない よ	
\\	彼[かれ]はもう 彼女[かのじょ]を
\\	はいないよ。			
\\	憎しみ	憎[にく]しみ	にくしみ	
\\	彼の心は憎しみに満ちていたんだ。	彼[かれ]の 心[こころ]は 憎[にく]しみに 満[み]ちていたんだ。	かれ の こころ は にくしみ に みちて いた ん だ	
\\	彼[かれ]の 心[こころ]は
\\	に 満[み]ちていたんだ。			
\\	憎らしい	憎[にく]らしい	にくらしい	
\\	妹は時々憎らしいことを言うんだ。	妹[いもうと]は 時々[ときどき] 憎[にく]らしいことを 言[い]うんだ。	いもうと は ときどき にくらしい こと を いう ん だ	
\\	妹[いもうと]は 時々[ときどき]
\\	ことを 言[い]うんだ。			
\\	憎い	憎[にく]い	にくい	
\\	彼が憎いですか。	彼[かれ]が 憎[にく]いですか。	かれ が にくい です か	
\\	彼[かれ]が
\\	ですか。			
\\	坊さん	坊[ぼう]さん	ぼうさん	
\\	私の友人はお坊さんをしているの。	私[わたし]の 友人[ゆうじん]はお 坊[ぼう]さんをしているの。	わたし の ゆうじん は おぼうさん を して いる の	
\\	私[わたし]の 友人[ゆうじん]はお
\\	をしているの。			
\\	名簿	名簿[めいぼ]	めいぼ	
\\	これが参加者の名簿です。	これが 参加者[さんかしゃ]の 名簿[めいぼ]です。	これ が さんかしゃ の めいぼ です	
\\	これが 参加者[さんかしゃ]の
\\	です。			
\\	布	布[ぬの]	ぬの	
\\	この布はカーテンに使えます。	この 布[ぬの]はカーテンに 使[つか]えます。	この ぬの は かーてん に つかえます	
\\	この
\\	はカーテンに 使[つか]えます。			
\\	抱く	抱[だ]く	だく	
\\	祖母がうちの猫を抱いているわよ。	祖母[そぼ]がうちの 猫[ねこ]を 抱[だ]いているわよ。	そぼ が うち の ねこ を だいて いる わ よ	
\\	祖母[そぼ]がうちの 猫[ねこ]を
\\	いるわよ。			
\\	文句	文句[もんく]	もんく	
\\	彼女はいつも文句ばかり言う。	彼女[かのじょ]はいつも 文句[もんく]ばかり 言[い]う。	かのじょ は いつも もんく ばかり いう	
\\	彼女[かのじょ]はいつも
\\	ばかり 言[い]う。			
\\	平凡	平凡[へいぼん]	へいぼん	
\\	彼はごく平凡な人です。	彼[かれ]はごく 平凡[へいぼん]な 人[ひと]です。	かれ は ごく へいぼん な ひと です	
\\	彼[かれ]はごく
\\	な 人[ひと]です。			
\\	妊娠	妊娠[にんしん]	にんしん	
\\	彼女が妊娠したそうです。	彼女[かのじょ]が 妊娠[にんしん]したそうです。	かのじょ が にんしん した そう です	
\\	彼女[かのじょ]が
\\	したそうです。			
\\	同姓	同姓[どうせい]	どうせい	
\\	日本では夫婦同姓が一般的よ。	日本[にっぽん]では 夫婦[ふうふ] 同姓[どうせい]が 一般的[いっぱんてき]よ。	にっぽん で は ふうふ どうせい が いっぱんてき よ	
\\	日本[にっぽん]では 夫婦[ふうふ]
\\	が 一般的[いっぱんてき]よ。			
\\	嫁	嫁[よめ]	よめ	
\\	姉が嫁に行ったの。	姉[あね]が 嫁[よめ]に 行[い]ったの。	あね が よめ に いった の	
\\	姉[あね]が
\\	に 行[い]ったの。			
\\	幼児	幼児[ようじ]	ようじ	
\\	幼児は入場無料です。	幼児[ようじ]は 入場無料[にゅうじょう むりょう]です。	ようじ は にゅうじょう むりょう です	
\\	は 入場無料[にゅうじょう むりょう]です。			
\\	同封	同封[どうふう]	どうふう	
\\	手紙に写真が同封されていたよ。	手紙[てがみ]に 写真[しゃしん]が 同封[どうふう]されていたよ。	てがみ に しゃしん が どうふう されて いた よ	
\\	手紙[てがみ]に 写真[しゃしん]が
\\	されていたよ。			
\\	封	封[ふう]	ふう	
\\	手紙に封をしたよ。	手紙[てがみ]に 封[ふう]をしたよ。	てがみ に ふう を した よ	
\\	手紙[てがみ]に
\\	をしたよ。			
\\	呼び掛ける	呼[よ]び 掛[か]ける	よびかける	
\\	友人達に協力を呼び掛けたの。	友人達[ゆうじんたち]に 協力[きょうりょく]を 呼[よ]び 掛[か]けたの。	ゆうじんたち に きょうりょく を よびかけた の	
\\	友人達[ゆうじんたち]に 協力[きょうりょく]を
\\	の。			
\\	引っ掛かる	引[ひ]っ 掛[か]かる	ひっかかる	
\\	魚の骨がのどに引っ掛かった。	魚[さかな]の 骨[ほね]がのどに 引[ひ]っ 掛[か]かった。	さかな の ほね が のど に ひっかかった	
\\	魚[さかな]の 骨[ほね]がのどに
\\	引っ掛ける	引[ひ]っ 掛[か]ける	ひっかける	
\\	その選手はハードルに足を引っ掛けたんだ。	その 選手[せんしゅ]はハードルに 足[あし]を 引[ひ]っ 掛[か]けたんだ。	その せんしゅ は はーどる に あし を ひっかけた ん だ	
\\	その 選手[せんしゅ]はハードルに 足[あし]を
\\	んだ。			
\\	寄り掛かる	寄[よ]り 掛[か]かる	よりかかる	
\\	彼はフェンスに寄り掛かったの。	彼[かれ]はフェンスに 寄[よ]り 掛[か]かったの。	かれ は ふぇんす に よりかかった の 。	
\\	彼[かれ]はフェンスに
\\	の。			
\\	措置	措置[そち]	そち	
\\	被災者を救うための特別な措置が取られました。	被災者[ひさいしゃ]を 救[すく]うための 特別[とくべつ]な 措置[そち]が 取[と]られました。	ひさいしゃ を すくう ため の とくべつ な そち が とられました	
\\	被災者[ひさいしゃ]を 救[すく]うための 特別[とくべつ]な
\\	が 取[と]られました。			
\\	拝見	拝見[はいけん]	はいけん	
\\	あなたの著書を拝見しました。	あなたの 著書[ちょしょ]を 拝見[はいけん]しました。	あなた の ちょしょ を はいけん しました	
\\	あなたの 著書[ちょしょ]を
\\	しました。			
\\	控える	控[ひか]える	ひかえる	
\\	最近甘いものを控えています。	最近[さいきん] 甘[あま]いものを 控[ひか]えています。	さいきん あまい もの を ひかえて います	
\\	最近[さいきん] 甘[あま]いものを
\\	控え室	控[ひか]え 室[しつ]	ひかえしつ	
\\	ここはお客様用の控え室です。	ここはお 客様用[きゃくさま よう]の 控[ひか]え 室[しつ]です。	ここ は おきゃくさま よう の ひかえしつ です 。	
\\	ここはお 客様用[きゃくさま よう]の
\\	です。			
\\	握り締める	握[にぎ]り 締[し]める	にぎりしめる	
\\	彼は両手を握り締めたの。	彼[かれ]は 両手[りょうて]を 握[にぎ]り 締[し]めたの。	かれ は りょうて を にぎりしめた の	
\\	彼[かれ]は 両手[りょうて]を
\\	の。			
\\	揃う	揃[そろ]う	そろう	
\\	朝は家族全員が揃って食事します。	朝[あさ]は 家族全員[かぞく ぜんいん]が 揃[そろ]って 食事[しょくじ]します。	あさ は かぞく ぜんいん が そろって しょくじ します	
\\	朝[あさ]は 家族全員[かぞく ぜんいん]が
\\	食事[しょくじ]します。			
\\	揃える	揃[そろ]える	そろえる	
\\	花は長さを揃えて花瓶にさしましょう。	花[はな]は 長[なが]さを 揃[そろ]えて 花瓶[かびん]にさしましょう。	はな は ながさ を そろえて かびん に さしましょう	
\\	花[はな]は 長[なが]さを
\\	花瓶[かびん]にさしましょう。			
\\	巡る	巡[めぐ]る	めぐる	
\\	明日から温泉を巡る旅に出ます。	明日[あした]から 温泉[おんせん]を 巡[めぐ]る 旅[たび]に 出[で]ます。	あした から おんせん を めぐる たび に でます	
\\	明日[あした]から 温泉[おんせん]を
\\	旅[たび]に 出[で]ます。			
\\	待遇	待遇[たいぐう]	たいぐう	
\\	社員たちは待遇の改善を要求している。	社員[しゃいん]たちは 待遇[たいぐう]の 改善[かいぜん]を 要求[ようきゅう]している。	しゃいんたち は たいぐう の かいぜん を ようきゅう して いる	
\\	社員[しゃいん]たちは
\\	の 改善[かいぜん]を 要求[ようきゅう]している。			
\\	敏感	敏感[びんかん]	びんかん	
\\	彼女は流行に敏感だね。	彼女[かのじょ]は 流行[りゅうこう]に 敏感[びんかん]だね。	かのじょ は りゅうこう に びんかん だ ね	
\\	彼女[かのじょ]は 流行[りゅうこう]に
\\	だね。			
\\	寮	寮[りょう]	りょう	
\\	彼は会社の寮に住んでいます。	彼[かれ]は 会社[かいしゃ]の 寮[りょう]に 住[す]んでいます。	かれ は かいしゃ の りょう に すんで います	
\\	彼[かれ]は 会社[かいしゃ]の
\\	に 住[す]んでいます。			
\\	宣伝	宣伝[せんでん]	せんでん	
\\	その商品の宣伝をよく見かけます。	その 商品[しょうひん]の 宣伝[せんでん]をよく 見[み]かけます。	その しょうひん の せんでん を よく みかけます	
\\	その 商品[しょうひん]の
\\	をよく 見[み]かけます。			
\\	展覧会	展覧会[てんらんかい]	てんらんかい	
\\	昨日、絵の展覧会に行ってきました。	昨日[きのう]、 絵[え]の 展覧会[てんらんかい]に 行[い]ってきました。	きのう え の てんらんかい に いって きました	
\\	昨日[きのう]、 絵[え]の
\\	に 行[い]ってきました。			
\\	大臣	大臣[だいじん]	だいじん	
\\	大臣が汚職で逮捕されました。	大臣[だいじん]が 汚職[おしょく]で 逮捕[たいほ]されました。	だいじん が おしょく で たいほ されました	
\\	が 汚職[おしょく]で 逮捕[たいほ]されました。			
\\	派閥	派閥[はばつ]	はばつ	
\\	あの大学には派閥がたくさんあります。	あの 大学[だいがく]には 派閥[はばつ]がたくさんあります。	あの だいがく に は はばつ が たくさん あります	
\\	あの 大学[だいがく]には
\\	がたくさんあります。			
\\	懐かしい	懐[なつ]かしい	なつかしい	
\\	ここは私にとって懐かしい場所です。	ここは 私[わたし]にとって 懐[なつ]かしい 場所[ばしょ]です。	ここ は わたし に とって なつかしい ばしょ です	
\\	ここは 私[わたし]にとって
\\	場所[ばしょ]です。			
\\	噴火	噴火[ふんか]	ふんか	
\\	島で火山が噴火したよ。	島[しま]で 火山[かざん]が 噴火[ふんか]したよ。	しま で かざん が ふんか した よ	
\\	島[しま]で 火山[かざん]が
\\	したよ。			
\\	噴水	噴水[ふんすい]	ふんすい	
\\	公園の噴水の前で会いましょう。	公園[こうえん]の 噴水[ふんすい]の 前[まえ]で 会[あ]いましょう。	こうえん の ふんすい の まえ で あいましょう	
\\	公園[こうえん]の
\\	の 前[まえ]で 会[あ]いましょう。			
\\	愉快	愉快[ゆかい]	ゆかい	
\\	彼らはとても愉快な人たちです。	彼[かれ]らはとても 愉快[ゆかい]な 人[ひと]たちです。	かれら は とても ゆかい な ひとたち です	
\\	彼[かれ]らはとても
\\	な 人[ひと]たちです。			
\\	幼稚	幼稚[ようち]	ようち	
\\	その考えは少し幼稚だと思う。	その 考[かんが]えは 少[すこ]し 幼稚[ようち]だと 思[おも]う。	その かんがえ は すこし ようち だ と おもう	
\\	その 考[かんが]えは 少[すこ]し
\\	だと 思[おも]う。			
\\	名称	名称[めいしょう]	めいしょう	
\\	みんなで製品の名称を考えました。	みんなで 製品[せいひん]の 名称[めいしょう]を 考[かんが]えました。	みんな で せいひん の めいしょう を かんがえました	
\\	みんなで 製品[せいひん]の
\\	を 考[かんが]えました。			
\\	悪賢い	悪賢[わるがしこ]い	わるがしこい	
\\	彼は悪賢い男だな。	彼[かれ]は 悪賢[わるがしこ]い 男[おとこ]だな。	かれ は わるがしこい おとこ だ な	
\\	彼[かれ]は
\\	男[おとこ]だな。			
\\	沸騰	沸騰[ふっとう]	ふっとう	
\\	水が沸騰したよ。	水[みず]が 沸騰[ふっとう]したよ。	みず が ふっとう した よ	
\\	水[みず]が
\\	したよ。			
\\	庁	庁[ちょう]	ちょう	
\\	彼は県庁で働いています。	彼[かれ]は 県[けん] 庁[ちょう]で 働[はたら]いています。	かれ は けんちょう で はたらいて います	
\\	彼[かれ]は 県[けん]
\\	で 働[はたら]いています。			
\\	府庁	府庁[ふちょう]	ふちょう	
\\	彼は府庁で働いています。	彼[かれ]は 府庁[ふちょう]で 働[はたら]いています。	かれ は ふちょう で はたらいて います	
\\	彼[かれ]は
\\	で 働[はたら]いています。			
\\	摩擦	摩擦[まさつ]	まさつ	
\\	その二つの国に摩擦が生じたんだ。	その 二[ふた]つの 国[くに]に 摩擦[まさつ]が 生[しょう]じたんだ。	その ふたつ の くに に まさつ が しょうじた ん だ	
\\	その 二[ふた]つの 国[くに]に
\\	が 生[しょう]じたんだ。			
\\	大衆	大衆[たいしゅう]	たいしゅう	
\\	能は元々大衆の文化でした。	能[のう]は 元々[もともと] 大衆[たいしゅう]の 文化[ぶんか]でした。	のう は もともと たいしゅう の ぶんか でした	
\\	能[のう]は 元々[もともと]
\\	の 文化[ぶんか]でした。			
\\	古里	古里[ふるさと]	ふるさと	
\\	いつも古里を懐かしく思い出すの。	いつも 古里[ふるさと]を 懐[なつ]かしく 思[おも]い 出[だ]すの。	いつも ふるさと を なつかしく おもいだす の	
\\	いつも
\\	を 懐[なつ]かしく 思[おも]い 出[だ]すの。			
\\	彫る	彫[ほ]る	ほる	
\\	彼は木の像を彫ったの。	彼[かれ]は 木[き]の 像[ぞう]を 彫[ほ]ったの。	かれ は き の ぞう を ほった の	
\\	彼[かれ]は 木[き]の 像[ぞう]を
\\	の。			
\\	報酬	報酬[ほうしゅう]	ほうしゅう	
\\	その仕事の報酬として20万円もらいました。	その 仕事[しごと]の 報酬[ほうしゅう]として20 万円[まんえん]もらいました。	その しごと の ほうしゅう と して 
\\	まんえん もらいました	
\\	その 仕事[しごと]の
\\	として20 万円[まんえん]もらいました。			
\\	天皇	天皇[てんのう]	てんのう	
\\	日本には天皇がいます。	日本[にっぽん]には 天皇[てんのう]がいます。	にっぽん に は てんのう が います	
\\	日本[にっぽん]には
\\	がいます。			
\\	召し上がる	召[め]し 上[あ]がる	めしあがる	
\\	どうぞ召し上がってください。	どうぞ 召[め]し 上[あ]がってください。	どうぞ めしあがって ください	
\\	どうぞ
\\	ください。			
\\	哲学	哲学[てつがく]	てつがく	
\\	哲学は興味深い学問です。	哲学[てつがく]は 興味深[きょうみぶか]い 学問[がくもん]です。	てつがく は きょうみぶかい がくもん です	
\\	は 興味深[きょうみぶか]い 学問[がくもん]です。			
\\	梅雨明け	梅雨明[つゆあ]け	つゆあけ	
\\	梅雨明けは来週だそうです。	梅雨明[つゆあ]けは 来週[らいしゅう]だそうです。	つゆあけ は らいしゅう だ そう です	
\\	は 来週[らいしゅう]だそうです。			
\\	梅雨入り	梅雨入[つゆい]り	つゆいり	
\\	梅雨入りは6月17日でした。	梅雨入[つゆい]りは6 月17日[がつ 
\\	にち]でした。	つゆいり は 
\\	がつ 
\\	にち でした	
\\	は6 月17日[がつ 
\\	にち]でした。			
\\	欄	欄[らん]	らん	
\\	お名前は上の欄にお書きください。	お 名前[なまえ]は 上[うえ]の 欄[らん]にお 書[か]きください。	お なまえ は うえ の らん に お かき ください	
\\	お 名前[なまえ]は 上[うえ]の
\\	にお 書[か]きください。			
\\	栓抜き	栓抜[せんぬ]き	せんぬき	
\\	栓抜きはどこですか。	栓抜[せんぬ]きはどこですか。	せんぬき は どこ です か	
\\	はどこですか。			
\\	峠	峠[とうげ]	とうげ	
\\	この峠を越えると村があります。	この 峠[とうげ]を 越[こ]えると 村[むら]があります。	この とうげ を こえる と むら が あります	
\\	この
\\	を 越[こ]えると 村[むら]があります。			
\\	婆さん	婆[ばあ]さん	ばあさん	
\\	彼女は元気なお婆さんね。	彼女[かのじょ]は 元気[げんき]なお 婆[ばあ]さんね。	かのじょ は げんき な おばあさん ね	
\\	彼女[かのじょ]は 元気[げんき]なお
\\	ね。			
\\	年賀	年賀[ねんが]	ねんが	
\\	年賀葉書が売り出されたね。	年賀[ねんが] 葉書[はがき]が 売[う]り 出[だ]されたね。	ねんがはがき が うりだされた ね	
\\	葉書[はがき]が 売[う]り 出[だ]されたね。			
\\	挟む	挟[はさ]む	はさむ	
\\	ドアに指を挟んだ。	ドアに 指[ゆび]を 挟[はさ]んだ。	どあ に ゆび を はさんだ	
\\	ドアに 指[ゆび]を
\\	塔	塔[とう]	とう	
\\	あの塔の高さはどれくらいですか。	あの 塔[とう]の 高[たか]さはどれくらいですか。	あの とう の たかさ は どれ くらい です か	
\\	あの
\\	の 高[たか]さはどれくらいですか。			
\\	慰める	慰[なぐさ]める	なぐさめる	
\\	友人が慰めてくれました。	友人[ゆうじん]が 慰[なぐさ]めてくれました。	ゆうじん が なぐさめて くれました	
\\	友人[ゆうじん]が
\\	朗らか	朗[ほが]らか	ほがらか	
\\	彼はとても朗らかな人です。	彼[かれ]はとても 朗[ほが]らかな 人[ひと]です。	かれ は とても ほがらか な ひと です	
\\	彼[かれ]はとても
\\	な 人[ひと]です。			
\\	喉	喉[のど]	のど	
\\	風邪をひいて喉が痛いです。	風邪[かぜ]をひいて 喉[のど]が 痛[いた]いです。	かぜ を ひいて のど が いたい です	
\\	風邪[かぜ]をひいて
\\	が 痛[いた]いです。			
\\	唾	唾[つば]	つば	
\\	彼は地面に唾を吐いたよ。	彼[かれ]は 地面[じめん]に 唾[つば]を 吐[は]いたよ。	かれ は じめん に つば を はいた よ	
\\	彼[かれ]は 地面[じめん]に
\\	を 吐[は]いたよ。			
\\	吠える	吠[ほ]える	ほえる	
\\	どこかで犬が吠えていますね。	どこかで 犬[いぬ]が 吠[ほ]えていますね。	どこか で いぬ が ほえて います ね	
\\	どこかで 犬[いぬ]が
\\	ね。			
\\	吊るす	吊[つ]るす	つるす	
\\	木にブランコを吊るしました。	木[き]にブランコを 吊[つ]るしました。	き に ぶらんこ を つるしました	
\\	木[き]にブランコを
\\	味噌	味噌[みそ]	みそ	
\\	きゅうりに味噌をつけて食べたんだ。	きゅうりに 味噌[みそ]をつけて 食[た]べたんだ。	きゅうり に みそ を つけて たべた ん だ	
\\	きゅうりに
\\	をつけて 食[た]べたんだ。			
\\	塞がる	塞[ふさ]がる	ふさがる	
\\	傷口はもう塞がりました。	傷口[きずぐち]はもう 塞[ふさ]がりました。	きずぐち は もう ふさがりました	
\\	傷口[きずぐち]はもう
\\	塞ぐ	塞[ふさ]ぐ	ふさぐ	
\\	あまりにうるさかったので耳を塞いだわ。	あまりにうるさかったので 耳[みみ]を 塞[ふさ]いだわ。	あまりに うるさかった の で みみ を ふさいだ わ	
\\	あまりにうるさかったので 耳[みみ]を
\\	わ。			
\\	撫でる	撫[な]でる	なでる	
\\	猫の頭を撫でてあげた。	猫[ねこ]の 頭[あたま]を 撫[な]でてあげた。	ねこ の あたま を なでて あげた	
\\	猫[ねこ]の 頭[あたま]を
\\	海苔	海苔[のり]	のり	
\\	旅館の朝食にのりが出ました。	旅館[りょかん]の 朝食[ちょうしょく]にのりが 出[で]ました。	りょかん の ちょうしょく に のり が でました 。	
\\	旅館[りょかん]の 朝食[ちょうしょく]に
\\	が 出[で]ました。			
\\	明瞭	明瞭[めいりょう]	めいりょう	
\\	彼は話し方が明瞭ね。	彼[かれ]は 話[はな]し 方[かた]が 明瞭[めいりょう]ね。	かれ は はなしかた が めいりょう ね	
\\	彼[かれ]は 話[はな]し 方[かた]が
\\	ね。			
\end{CJK}
\end{document}