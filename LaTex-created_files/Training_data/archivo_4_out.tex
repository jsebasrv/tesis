\documentclass[8pt]{extreport} 
\usepackage{hyperref}
\usepackage{CJKutf8}
\begin{document}
\begin{CJK}{UTF8}{min}
\\	大人	
\\	おとな	
\\	これは、大人のりょうきんです。	
\\	大人は三人だけです。	
\\	大人たちはいざかやにいった。	
\\	(おとな)? 
\\	大, 人	
\\	一人	
\\	ひとり	
\\	一人でカラオケにいきました。	
\\	一人はアメリカ人です。	
\\	むすめは、きょうはじめて一人でおつかいにいきました。	
\\	一つ 
\\	一, 人	
\\	人工	
\\	じんこう	
\\	の 
\\	あなたののうは、ほんとうに人工ののうなんですか?	
\\	この川は、人工の川です。	
\\	にわに人工しばをしいてみました。	
\\	人, 工	
\\	大きい	
\\	おおきい	い 
\\	このぼうしは大きすぎます。	
\\	その大きいももを二つください。	
\\	わたしのあかちゃんは、こうしとおなじくらい大きい。	
\\	い 
\\	おお 
\\	おお 
\\	大	
\\	下	
\\	した	
\\	つくえの下にはなにがありますか。	
\\	下に下りてきてください。	
\\	ぼくは
\\	オフィスの下のかいにすんでいる。	
\\	した.	下	
\\	八	
\\	はち	
\\	がくせいは、きょううちで、八ページをべんきょうします。	
\\	あめを八こカバンに入れました。	
\\	八じ?ほんとうに、八じ?	
\\	八	
\\	八つ	
\\	やっつ	
\\	ふくろには、じゃがいもが八つ入っています。	
\\	八つしかありませんよ。	
\\	八つのかおと六つのうでがあるもの、なーんだ。	
\\	つ 
\\	つ. 
\\	つ), 
\\	つ 
\\	つ 
\\	(やっつ). 
\\	つ, 
\\	つ 
\\	つ 
\\	八	
\\	入り口	
\\	いりぐち	
\\	入り口のところでまっています。	
\\	入り口が三つあります。	
\\	入り口にジョニーデップがいるよ。	
\\	入り 
\\	入れる. 
\\	(いり). 
\\	口 
\\	く 
\\	くち 
\\	ぐ 
\\	ぐち 
\\	くち 
\\	入, 口	
\\	大した	
\\	たいした	
\\	それは大したことはない。	
\\	まったく大した女だよ。	
\\	マクドナルドのしゃちょうになれれば大したものだ。	
\\	"した 
\\	する 
\\	大 
\\	大した 
\\	たいした.	大	
\\	山	
\\	やま	
\\	ともだちといっしょに、山にのぼります。	
\\	これはせかい一大きい人工の山なんですよ。	
\\	あさ八じに、はだかではげ山にしゅうごうしましょう。	
\\	(やま). 
\\	山	
\\	口	
\\	くち	
\\	口にソースがついていますよ。	
\\	これを口に入れてください。	
\\	きのうから口がとじられない。	
\\	(くち) 
\\	(こう). くちくちこう!くちくちこう! 
\\	口	
\\	ふじ山	
\\	ふじさん	
\\	ふじ山に、ごぜん十一じごろまでにつきたいです。	
\\	ふじ山にはいつも大人のフランス人が三人とちいさいアメリカ人が二人とカナダ人が一人います。	
\\	ふじ山はしろいとりのふんでおおわれていた。	
\\	さん, ちゃん, 
\\	山 
\\	ふじ 
\\	山. 
\\	山	
\\	九	
\\	きゅう, く	
\\	まいにち九じかんぐらいべんきょうします。	
\\	九ページをひらいてください。	
\\	九わの九かんちょうをかっていました。	
\\	九, 
\\	九	
\\	九つ	
\\	ここのつ	
\\	きょうのテストには、じしょをつかうもんだいが九つあります。	
\\	ねこには九つのいのちがあるってほんとう?	
\\	ドーナツを九つつくった。	
\\	つ 
\\	つ. 
\\	つ), 
\\	つ 
\\	つ 
\\	(ここの). 
\\	九	
\\	一	
\\	いち	
\\	一ど、あいましょう。	
\\	一いはアメリカ人でした。	
\\	ぼくはせかいで一ばんよわい。	
\\	一	
\\	一つ	
\\	ひとつ	
\\	こたえを一つえらびます。	
\\	お一人さま一つかぎりです。	
\\	きのう、一つめこぞうにあった。	
\\	つ 
\\	つ. 
\\	つ), 
\\	つ 
\\	つ 
\\	(ひと). 
\\	一	
\\	人	
\\	ひと	
\\	やじるしの人はなんといいますか。	
\\	あの人はなに人ですか?	
\\	人びとは、かれをばかな人だという。	
\\	(ひと). 
\\	人	
\\	下さい	
\\	ください	
\\	ここでは、きっぷがいりますから、きっぷをかってきて下さい。	
\\	りんごを二こ下さい。	
\\	フグさん、そっとつねって下さいね。	
\\	下 
\\	さい 
\\	ください 
\\	くだ 
\\	さい 
\\	下	
\\	人口	
\\	じんこう	
\\	人口のはんぶんはだんせいです。	
\\	このむらの人口はたったの二十三にんです。	
\\	じぶんのまちの人口をしらないにほんじんはおおい。	
\\	人, 口	
\\	力	
\\	ちから	
\\	おなかがすいて、力がでない。	
\\	あのアメリカ人の女にはふじ山をうごかす力がある。	
\\	おかあさんは、のこりのかぞくがたばになってもかなわない力をもっている。	
\\	(ちから). 
\\	力	
\\	川	
\\	かわ	
\\	この川には、さかながたくさんおよいでいます。	
\\	にほんいちながい川は、しなの川です。	
\\	川をみるといつもいかりがこみあげてくる。	
\\	川	
\\	七	
\\	なな, しち	
\\	あしたのあさ、七じにえきであいませんか。	
\\	みかんが七こあります。	
\\	七にんのこびととともだちになりたい。	
\\	なな 
\\	七	
\\	七つ	
\\	ななつ	
\\	はこのなかにりんごが七つあります。	
\\	二人でメロンを七つたべました。	
\\	七つかぞえると、わたしは女になる。	
\\	つ 
\\	つ. 
\\	つ), 
\\	つ 
\\	つ 
\\	(なな). 
\\	七	
\\	大きさ	
\\	おおきさ	
\\	このテーブルの大きさは、よこ100センチ、たて120センチです。	
\\	テレビの大きさは三十二インチです。	
\\	ロナルド・マクドナルドはいつも大きさのちがうくつをはいていた。	
\\	さ 
\\	大きい) 
\\	大きい 
\\	大きさ, 
\\	大, 
\\	""おお 
\\	おお 
\\	大	
\\	十	
\\	じゅう	
\\	わたしは、ツバメを十わかっています。	
\\	あと二かで、十二がつですね。	
\\	十ねんかん十にんのじゅうにんとくらした。	
\\	じゅう 
\\	十	
\\	三	
\\	さん	
\\	このもんだいようしは、ぜんぶで三ページあります。	
\\	わたしは三十二さいです。	
\\	三かいまわってワンとほえろ。	
\\	さん.	三	
\\	三人	
\\	さんにん	
\\	わたしには、きょうだいが三人います。	
\\	その川には二、三人しかいませんでした。	
\\	三人のうちで、だれがいちばんタイプ?	
\\	三, 人	
\\	三つ	
\\	みっつ	
\\	三つのせんたくしがおんせいできこえます。	
\\	ここにボールを三つ入れてください。	
\\	三つもかにかまれた。	
\\	つ 
\\	つ. 
\\	つ), 
\\	つ 
\\	つ 
\\	三	
\\	上る	
\\	のぼる	
\\	わたしがやねに上りましょう。	
\\	十かいまでかいだんを上った。	
\\	サケが川を上っているのをみた。	
\\	う 
\\	(る, 
\\	上 
\\	上げる 
\\	上がる). 
\\	上 
\\	上	
\\	入る	
\\	はいる	
\\	プールに入るまえに、じゅんびうんどうをしてください。	
\\	ちゃんと入り口から入ってください。	
\\	へやに入って、いっしょにまゆげをかこうよ!	
\\	""う
\\	る, 
\\	""う
\\	入 
\\	入る 
\\	(はい) 
\\	入	
\\	下がる	
\\	さがる	
\\	せかいじゅうのかぶかが下がった。	
\\	ねつが三十七どに下がりました。	
\\	ズボンが下がるようにベルトをゆるめなさい。	
\\	う 
\\	(さ) 
\\	下	
\\	入れる	
\\	いれる	
\\	しかくになにを入れますか。	
\\	ここにおりがみを十二まい入れてください。	
\\	わたしはハンバーガーをひつぎに入れた。	
\\	入る 
\\	入る (はいる/
\\	入. 
\\	""い!
\\	入	
\\	下げる	
\\	さげる	
\\	けつあつを下げるほうほうをおしえてください。	
\\	あのみせは、入り口にふうりんを一つ下げています。	
\\	きおんを下げるために、あまごいダンスをおどってみた。	
\\	う 
\\	(さ). 
\\	下	
\\	上げる	
\\	あげる	
\\	このはこをれいぞうこの上に上げてください。	
\\	三センチ上げてください。	
\\	りょうてを上に上げて、大きくおならをした。	
\\	う 
\\	(あ) 
\\	上	
\\	上がる	
\\	あがる	
\\	どんどんねつが上がっています。	
\\	三かいまでエレベーターで上がりましょう。	
\\	けむりが上がってる。	
\\	う 
\\	あ 
\\	""あ!
\\	上	
\\	二	
\\	に	
\\	二十一ページの二ばんはクラスでします。	
\\	わたしのこどもは二さいです。	
\\	わたしのおじは、ねこを二ひきかっています。	
\\	二	
\\	二人	
\\	ふたり	
\\	を、あねと二人でみています。	
\\	あの二人はいつも二人でくみたがる。	
\\	かのじょは二人ごろしでたいほされた。	
\\	二つ
\\	二 
\\	(ふた). 
\\	り 
\\	人 
\\	一人). 
\\	一人 
\\	二, 人	
\\	二つ	
\\	ふたつ	
\\	二つのかいわです。	
\\	このカゴにリンゴを二つ入れてください。	
\\	かれの二つめのあくぎょうは、かえるのあんさつだ。	
\\	つ 
\\	つ. 
\\	つ), 
\\	つ 
\\	つ 
\\	(ふた). 
\\	に)? 
\\	二	
\\	上	
\\	うえ	
\\	の 
\\	上のひょうをみてください。	
\\	このフロアの三かい上にトーフグのオフィスがあります。	
\\	くつ下のかたほうは、かいだんの上にあったわよ。	
\\	(うえ), 
\\	上	
\\	女	
\\	おんな	
\\	あの女はだれ?	
\\	わたしは女です。	
\\	おれはこどもをせわする女がほしい。	
\\	(おんな), 
\\	女	
\\	又	
\\	また	
\\	又してもおまえにやられてしまったぜ...とほほ!	
\\	わたしはアメリカ人です。又、わたしはフランス人でもあります。	
\\	又来たぜ!	
\\	又	
\\	四月	
\\	しがつ	
\\	たんじょうびは、四月九日です。	
\\	四月はさくらがさくので、王子は日本にいきたがっています。	
\\	四月にりょうしんがりこんしたストレスで、おれのガールフレンドにはすこしひげがはえた。	
\\	"月 
\\	がつ 
\\	四, 月	
\\	玉	
\\	たま	
\\	うんどうかいで玉入れをしました。	
\\	ここに、白の玉が九つあります。	
\\	けいとの玉のようなどうぶつをみかけませんでしたか?	
\\	たま.	玉	
\\	本	
\\	ほん	
\\	この本はどこでかいましたか。	
\\	コウイチがあやしいフランス人からかった本のカバーは本がわのカバーでした。	
\\	ここで本をよみたければ、きんをわたしてもらおう。	
\\	本 
\\	本	
\\	白人	
\\	はくじん	
\\	の 
\\	白人だんせいとけっこんしました。	
\\	わたしはケニア出しんのこく人ですが、アメリカの白人かぞくにそだてられました。	
\\	おおくの白人たちは、日本のなつが大のにが手だ。	
\\	白, 人	
\\	丸	
\\	まる	
\\	これは、ピカソがえがいた丸です。	
\\	このちずにかいてある丸はなんですか。	
\\	わたしのくろラブは、おどろいて、めが丸になった。	
\\	丸	
\\	丸い	
\\	まるい	い 
\\	山田さんがつくったのは、丸いおにぎりです。	
\\	ちきゅうは丸い。	
\\	わたしのかおは丸いとはいいがたい。	
\\	い 
\\	い-
\\	丸? 
\\	(円), 
\\	丸, 
\\	い 
\\	丸	
\\	正しい	
\\	ただしい	い 
\\	かれは、じぶんがいつも正しいとおもっている。	
\\	王女はいつも正しくて、王子はいつもまちがっている。	
\\	ぼくのこころのとびらをあけるには、正しいかぎがひつようだよ、ベイビー。	
\\	い 
\\	い-
\\	正 
\\	ただ 
\\	正	
\\	犬	
\\	いぬ	
\\	犬とねこがいます。	
\\	その王は犬を六ぴきかっています。	
\\	わたしの犬は、白さいがすきだ。	
\\	犬	
\\	八日	
\\	ようか	
\\	わたしのなつやすみは、八日かんです。	
\\	すみませんが、水を下さいませんか。もう八日もなにもたべていないんです。	
\\	らい月の八日にじの手じゅつをするよていです。	
\\	(ようか). 
\\	八, 日	
\\	夕べ	
\\	ゆうべ	
\\	夕べ、パーティーにいきました。	
\\	デンマークの王子はフランス人二人をおんがくの夕べにしょうたいしました。	
\\	夕べ、ひたいに肉とかいてみた。	
\\	べ). 
\\	夕	
\\	出口	
\\	でぐち	
\\	そこは出口ではありません。	
\\	ふじ山にはひみつの入り口と出口があるらしいですよ。	
\\	このきゅうじょうの出口はあちらです。	
\\	出口 
\\	口 (くち) 
\\	出 
\\	(で). 
\\	くち 
\\	ぐち.
\\	出, 口	
\\	目	
\\	め	
\\	はなこさんの目はいいですか?	
\\	アメリカには目が三つある人々がすんでいる山があって、そこではまいとし二月八日におまつりがある。	
\\	うつくしいあおい目のおとこが大すき!	
\\	目	
\\	目玉	
\\	めだま	
\\	このじゃがいもを、ゆきだるまの目玉にします。	
\\	王女は目玉をコレクションするしゅみがあります。	
\\	ようかいの目玉のようなぶどうだとおもう。	
\\	目玉, 
\\	目, 玉	
\\	二月	
\\	にがつ	
\\	二月三日にけっこんしました。	
\\	二月は、日本いきのチケットがとてもやすいです。	
\\	きょねんの二月に、わたしたちのふりんははじまったの。	
\\	月 
\\	月 
\\	がつ 
\\	げつ 
\\	二, 月	
\\	五日	
\\	いつか	
\\	五日もしごとをやすみました。	
\\	六月五日はともだちのたんじょう日です。	
\\	五日めに、「これでおしまい。」といった。	
\\	(いつ) 
\\	(か). 
\\	五, 日	
\\	五十	
\\	ごじゅう	
\\	かいてんずしで、五十さらたべた。	
\\	ある天才いしは、たったの五十えんでつうふうをなおすちりょうほうをはつめいしました。	
\\	ほぼ五十ねんかんあなたのことがすきでしたが、きのうさめました。	
\\	五, 十	
\\	火	
\\	ひ	
\\	はやく火をけしてください。	
\\	石をつかって火をおこしたことはありますか?	
\\	火は、かれのかみのけをぜんぶやいてしまった。	
\\	(ひ). 
\\	火	
\\	五	
\\	ご	
\\	トムさんは、コンビニでおにぎりを五こかいました。	
\\	わたしのともだちはしょうわ五十五ねん五月いつかうまれです。	
\\	むかしむかし、五ひきの子ぶたをうんだははぶたがいました。	
\\	五	
\\	五つ	
\\	いつつ	
\\	わたしは五つもおもちをたべました。	
\\	五十人の白人のアメリカ人女子が、五つのおまんじゅうをてにいれようときそいあっています。	
\\	人せいには五つのくるしみがあるが、その一つはへんずつうである。	
\\	つ 
\\	五, 
\\	(いつ) 
\\	五	
\\	四	
\\	よん, し	
\\	もんだい四は、えなどがありません。	
\\	王のへやには大きい、白い一千万円玉が四こもある。	
\\	おやしらずを四ほんともばっしした。	
\\	よん 
\\	し 
\\	し 
\\	し. 
\\	よん. 
\\	四	
\\	天才	
\\	てんさい	
\\	かれはしょうぎの天才です。	
\\	あのフランス人の女は天才だ。	
\\	そのゴリラは、まぎれもなく天才だ。	
\\	天, 才	
\\	女子	
\\	じょし	
\\	そっちは女子トイレですよ。	
\\	わたしは女子ではなく、一人の女せいです。	
\\	女子力をみがくひつようがある。	
\\	女, 子	
\\	女の子	
\\	おんなのこ	
\\	リボンをつけている女の子がはな子ちゃんです。	
\\	とてもかわいい女の子ですね。	
\\	ぼくのはつこいの女の子は、にんじんがきらいだった。	
\\	の, 
\\	子 
\\	女. 
\\	女. 
\\	子. 
\\	子, 
\\	女, 子	
\\	上手	
\\	じょうず	
\\	な 
\\	さとうさんは、うたを上手にうたいます。	
\\	オリーブオイルのつかいかたが上手な王子をしっていますか。	
\\	わたしはへんがおをつくるのが上手です。	
\\	手 
\\	(じょう) 
\\	(ず). 
\\	上, 手	
\\	手	
\\	て	
\\	手がカサカサになってしまいました。	
\\	おれの女に手を出すな。	
\\	せんせいの手はどうしてそんなにけぶかいの?	
\\	手	
\\	天	
\\	てん	
\\	ポチは天のほしになりました。	
\\	わたしがコウイチから四千えんをぬすんだことを、天のかみさまはしっかりとみていました。	
\\	じゃんけんにかって、わたしは天にものぼるきもちだった。	
\\	天	
\\	入力	
\\	にゅうりょく	
\\	する 
\\	データの入力をおえました。	
\\	ここに王子のソーシャルセキュリティーナンバーを入力してください。	
\\	「ローマじ入力」と「かな入力」、どちらをつかっていますか?	
\\	入, 力	
\\	一月	
\\	いちがつ	
\\	わたしは一月にかんじを千こおぼえました。	
\\	一月うまれのアメリカ人と六月うまれのイタリア人では、どちらのほうがおおいとおもいますか。	
\\	一月にたべるアイスクリームもわるくない。	
\\	月 
\\	がつ 
\\	一, 月	
\\	日本	
\\	にほん, にっぽん	
\\	日本のカレーはおいしいです。	
\\	十月から、日本にすみます。	
\\	日本のかたちをしたケーキをやいた。	
\\	本? 
\\	日本.
\\	日 
\\	にち 
\\	に.	日, 本	
\\	六月	
\\	ろくがつ	
\\	六月にとうきょうへいきました。	
\\	五月に刀を二本かって、六月にトーフグのオフィスでつじぎりをします。	
\\	六月のてんきは、くもりときどきミートボールです。	
\\	月 
\\	がつ 
\\	六, 月	
\\	子	
\\	こ	
\\	あなたは、わたしのほんとうの子ではないの。	
\\	そのロシア人の子は、大人になって、さけよりも上手に川を上れるようになりました。	
\\	わたしの子、ビヨンセににているとおもわない?	
\\	(こ). 
\\	子 
\\	子	
\\	王	
\\	おう	
\\	このくにの王はわたしだ。	
\\	王の刀をみせてください。	
\\	王さまでもカレーをたべるのだ。	
\\	王	
\\	左右	
\\	さゆう	
\\	する 
\\	それは大いにてんきに左右される。	
\\	ビエトのサスペンダー、左右のいろがちがっていておしゃれだね。	
\\	みちをわたるまえに、左右をよくたしかめましょう。	
\\	左, 右	
\\	左	
\\	ひだり	
\\	の 
\\	わたしは、左ききです。	
\\	左の山のなまえはなんですか?	
\\	あのむすめが左みみのピアスをあけたときのことがわすれられない。	
\\	(ひだり) 
\\	左	
\\	五月	
\\	ごがつ	
\\	五月四日にくにへかえりました。	
\\	五月にはゴールデンウィークがあります。	
\\	五月に川へいきます。	
\\	がつ 
\\	五, 月	
\\	中	
\\	なか	
\\	いえの中は、エアコンがきいていました。	
\\	おいそがしい中、おへんじを下さりありがとうございました。	
\\	りょうしんはこどもを中にしてすわっていた。	
\\	(なか), 
\\	中	
\\	〜円	
\\	えん	
\\	このきっぷは三千円です。	
\\	王女があやしいアメリカ人と、王しつのかね一千まん円をもってよにげしてしまった。	
\\	円をアメリカドルにりょうがえしたいんです。	
\\	円	
\\	月	
\\	つき	
\\	こんやは月がきれいですね。	
\\	六月はふじ山がやまびらきする月です。	
\\	ピンクの月が出たら、出ぱつのあいずだ。	
\\	(つき), 
\\	月	
\\	九日	
\\	ここのか	
\\	九日に、いっしょにサッカーをみにいきませんか。	
\\	コウイチは九日もニッカにあっていないので、さびしくてしにそうだ。	
\\	マミのたんじょう日は四がつ九日だ。	
\\	九 
\\	九つ, 
\\	九, 日	
\\	十月	
\\	じゅうがつ	
\\	さらいねんの十月で、六十さいになります。	
\\	十月にはだんだんすずしくなってくるでしょう。	
\\	かれがおかしいのは、十月なのにすいえいにさそったわたしのせきにんだ。	
\\	がつ 
\\	十, 月	
\\	一日	
\\	いちにち, ついたち	
\\	きのうは一日テストのべんきょうをしていました。	
\\	六月一日にアメリカ人とフランス人のふうふ二人がいっしょにエベレストをのぼりきったことは人々のきおくにあたらしい。	
\\	本とうに一日でこの本をよんだんですか?	
\\	(ついたち) 
\\	一, 日	
\\	一千	
\\	いっせん	
\\	そのライブには、一千まん人があつまった。	
\\	その刀は、一千えんです。	
\\	一千フィートのたかさでおならをした。	
\\	一. 
\\	ち 
\\	っ, 
\\	いっせん.	一, 千	
\\	千円	
\\	せんえん	
\\	このスカートは、千円です。	
\\	ぎゅうにゅうのねだんが千円上がった。	
\\	あのやきゅうせんしゅのねんぽうは五千円だ。	
\\	千, 円	
\\	玉ねぎ	
\\	たまねぎ	
\\	玉ねぎをきっていたら、なみだが出てきた。	
\\	この玉ねぎを立たせることができたあなたは天才です。	
\\	きみは玉ねぎをいくつわきにはさめるかい?	
\\	"ねぎ 
\\	玉	
\\	人々	
\\	ひとびと	
\\	の 
\\	なぜ人々はくるまをかうのでしょうか?	
\\	おおくのパプアニューギニアの人々は水中でもいきができるらしい。	
\\	このちいきの人々は、みんないい人ばかりです。	
\\	人 
\\	びと.	人, 々	
\\	王子	
\\	おうじ	
\\	王子は、じょうばをはじめました。	
\\	あの王子の左手は大きい。	
\\	王子がかえるにキスをすると、そのかえるは王子をたべてしまった。	
\\	王, 
\\	子 (し) 
\\	じ.	王, 子	
\\	王女	
\\	おうじょ	
\\	王子は、王女のうつくしさにうっとりしました。	
\\	王は王女に木のつくえをかってあげました。	
\\	王女さまはよろこびをバクてんでひょうげんしているようです。	
\\	女王, 
\\	女 
\\	王 
\\	女 
\\	王, 女	
\\	子犬	
\\	こいぬ	
\\	けさ、子犬がうまれました。	
\\	その子犬は王子にもらった九つの玉をくびから下げて入り口に立っていた。	
\\	子犬ごっこをしましょうよ。	
\\	子 
\\	犬 
\\	犬 
\\	子 
\\	こ 
\\	子, 犬	
\\	女王	
\\	じょおう	
\\	それは、女王のいすです。	
\\	女王は目が大きくてかわいい王女のことが大きらいです。	
\\	エリザベス女王と文つうをはじめた。	
\\	女, 王	
\\	田	
\\	た	
\\	あしたはかぞくみんなで田をたがやします。	
\\	そのフランス人は四十ユーロで田を一つかいました。	
\\	わたしのファーストキスは、田んぼのかかしにうばわれた。	
\\	田	
\\	右	
\\	みぎ	
\\	右のしゃしんは、とうきょうでとったしゃしんです。	
\\	右には王の好きなメイドが五十人、ひだりには王子のすうはいする天才かがくしゃが一人いる。	
\\	うまが右にまがったとき、ロバは右をチラリとみた。	
\\	(みぎ) 
\\	右	
\\	円い	
\\	まるい	い 
\\	円いおさらをつくえにならべてください。	
\\	円いまどがあるのが、王女のへやです。	
\\	わたしは円い月がこわい。	
\\	い 
\\	円 
\\	円 
\\	丸. 
\\	丸 
\\	まる 
\\	まる!	円	
\\	二日	
\\	ふつか	
\\	らい月の二日に、あねとえいがをみにいきます。	
\\	王子が子犬をおってふじ山のじゅかいにきえてから二日がたちました。	
\\	ひらがななんて、しんけんにべんきょうすれば二日あれば十ぶんさ。	
\\	(ふつか) 
\\	二, 日	
\\	七日	
\\	なのか	
\\	かぜで、七日かんもねこんでいました。	
\\	まい月七日に、おはかまいりをします。	
\\	たった七日でかいしゃをやめた。	
\\	(なのか), 
\\	七, 日	
\\	六	
\\	ろく	
\\	山本さんのでんわばんごうは、六一二の七七三六ですね?	
\\	六じにふじ山であいましょう。	
\\	六かける六は三十六ですよね?	
\\	六	
\\	十六	
\\	じゅうろく	
\\	十六さいのたんじょうびにふじ山にのぼりました。	
\\	もう十六ねんもふじ山の入り口からうごかないアメリカ人がいます。	
\\	わたしはえいえんの十六さいです。	
\\	十, 六	
\\	六日	
\\	むいか	
\\	二人は、せん月の六日に、どこであいましたか?	
\\	六日もがっこうをやすんで、いったいなにをしてたんだい?	
\\	月の六日目に、くまがバーにやってきた。	
\\	むい) 
\\	(か). 
\\	六, 日	
\\	六つ	
\\	むっつ	
\\	わたしのへやには、テーブルが一つと、いすが六つあります。	
\\	川から石を六つもってかえりました。	
\\	わたしは、六つのカップをかいました。	
\\	つ 
\\	(むっつ). 
\\	六	
\\	小さい	
\\	ちいさい	い 
\\	このふくは、わたしには小さいです。	
\\	このクラスには、まだ小さい子もいれば、大人のように大きい子もいる。	
\\	ぼくのりそうのマリモは、小さい玉ねぎのようなマリモだ。	
\\	い 
\\	小 
\\	ちい 
\\	(ちい) 
\\	小	
\\	土	
\\	つち	
\\	これは、こうしえんの土です。	
\\	このビー玉を七つ土にうめたら、天のこえがきこえます。	
\\	マットのしゅみは、土をほりあげることだ。	
\\	(つち). 
\\	土	
\\	日	
\\	ひ	
\\	そろそろ日がのぼるじかんだ。	
\\	きょうは、トーフグがけっこんするたいせつな日です。	
\\	日がおちると、すぐにあたりはまっくらになって、わたしは本をよむことができなくなってしまった。	
\\	(ひ).	日	
\\	刀	
\\	かたな	
\\	あのさむらいに、刀をぬすまれたんだ。	
\\	あのアメリカ人は刀を二ほんもっている。	
\\	スラリと刀をぬいたあのおとこはだれだ?	
\\	かたな 
\\	刀	
\\	十日	
\\	とおか	
\\	このくすりを、十日かんのんでください。	
\\	十日でかんせいさせてください。	
\\	七月十日はそふのめい日です。	
\\	(とお) 
\\	十, 日	
\\	三日	
\\	みっか	
\\	二人は、せんせん月の三日にどこであいましたか?	
\\	三日も水をのんでいません。	
\\	三月三日はひなまつりです。	
\\	三 
\\	三つ. 
\\	日: 
\\	(み) 
\\	(か). 
\\	三, 日	
\\	千	
\\	せん	
\\	このりょうりの本は、千円です。	
\\	そのフランス人は、千に一人の天才です。	
\\	白玉を千こたべた。	
\\	千	
\\	正す	
\\	ただす	
\\	しせいを正しなさい。	
\\	王子はまちがいを正されてプライドがきずついた。	
\\	あやまりを正すことができれば、ごほうびをあげるわ。	
\\	う 
\\	(ただ) 
\\	正	
\\	出る	
\\	でる	
\\	あちらの出口から出てください。	
\\	くしゃみが出そうで出ない。	
\\	わたしは、パチンコやからいそいで出た。	
\\	う 
\\	出す, 
\\	る 
\\	(で) 
\\	出	
\\	立つ	
\\	たつ	
\\	あそこに立っているのは、ぼくのあにです。	
\\	その石の上に立ってください。	
\\	あそこに立っているのが、あなたのみらいの子どもです。	
\\	う 
\\	た 
\\	(た)!	立	
\\	立てる	
\\	たてる	
\\	ロウソクは、なん本立てますか?	
\\	このしゃしん立てを立てるためのスタンドをさがしています。	
\\	ぼくのつくえにえんぴつを立ててくれませんか?	
\\	う 
\\	立つ 
\\	た 
\\	(た)!
\\	立	
\\	出す	
\\	だす	
\\	つくえの上に、じしょを出してください。	
\\	そのアメリカ人は、あしから二ひきのガラガラヘビにかまれたどくをぜんぶ出した。	
\\	どろぼうはぬすんだものをすべて出してくばった。	
\\	う 
\\	す 
\\	スーを出す.
\\	だ 
\\	(だ) 
\\	出	
\\	木	
\\	き	
\\	いえのにわには、さくらの木があります。	
\\	ともだちのアメリカ人のうちは、木とほねでできています。	
\\	あのりんごの木からおちたのは、はじめてじゃない。	
\\	(き) 
\\	木	
\\	水中	
\\	すいちゅう	
\\	の 
\\	水中でしか使えない。	
\\	右をみると、水中のせかいがひろがっていました。	
\\	水中でめをあけることができません。	
\\	水, 中	
\\	下手	
\\	へた	
\\	な 
\\	ギターはまだ下手ですが、まいばんれんしゅうしています。	
\\	アメリカ人のまねが下手な人は、フランス人のまねが上手です。	
\\	わたしははをみがくのが下手です。	
\\	(へた). 
\\	下, 手	
\\	中々	
\\	なかなか	
\\	中々できません。	
\\	あの王子も中々やりますね。	
\\	マリオカートをきわめることは中々かんたんではない。	
\\	々 
\\	中々 
\\	中中. 
\\	(なかなか).
\\	中, 々	
\\	大いに	
\\	おおいに	
\\	大いにまなびましょう。	
\\	王は王子と王女に、「大いにこくみんのぜいきんをつかっていいぞ」といった。	
\\	きょうは大いにのみたまえ。	
\\	(に) 
\\	大きい. 
\\	大	
\\	火山	
\\	かざん	
\\	火山のふんかもありました。	
\\	ふじ山は火山です。いまから四十二じかんごにふんかします。	
\\	きっと火山にちがいない。	
\\	山 
\\	ざん 
\\	さん 
\\	火, 山	
\\	水	
\\	みず	
\\	水かおちゃをもらえますか。	
\\	ふじ山にのぼるときは、水をわすれないようにしましょう。	
\\	コップ一ぱいの水とあいをください。	
\\	(みず) 
\\	水	
\\	白	
\\	しろ	
\\	テーブルに、白のおさらをならべてください。	
\\	七つの石のうち、三つは白でした。	
\\	そのおとこは、ゆきのように白いので、白ゆきおとことなづけられた。	
\\	(しろ) 
\\	白	
\\	文	
\\	ぶん	
\\	はやく文をよんでください。	
\\	おどろいたことに、そのアメリカ人の文はイギリス人のわたしの文よりえいごが上手にかけていました。	
\\	かなしい文をよむと、いつもなみだが二ふん二十びょうだけとまらない。	
\\	ぶん 
\\	文	
\\	〜才	
\\	さい	
\\	わたしは五才です。	
\\	らい月二日に、六十七才になります。	
\\	ちちの五十五才のたんじょう日に、ホグワーツへしょうたいされた。	
\\	才	
\\	少ない	
\\	すくない	い 
\\	ひゃくさいまで生きる人は少ない。	
\\	りょうりのレパートリーが少ないんです。	
\\	びようしさんから、かみのけが少ないとよくいわれます。	
\\	い. 
\\	すく 
\\	(すく). 
\\	少	
\\	少し	
\\	すこし	
\\	ぎゅうにゅうは、少ししかありません。	
\\	少ししかのこってないよ。	
\\	かれのプロポーズは少しあいまいだった。	
\\	少 
\\	少し! 
\\	少	
\\	アメリカ人	
\\	あめりかじん, アメリカじん	
\\	おれはアメリカ人か?	
\\	アメリカ人が二人います。	
\\	アメリカ人の女とともだちになった。	
\\	アメリカ 
\\	人 
\\	人 
\\	じん, 
\\	人 
\\	人	
\\	矢	
\\	や	
\\	矢はなん本のこっていますか?	
\\	こいのキューピッドが矢をはなとうとしているよ。	
\\	ほくべい土人の矢じりがとんできた。	
\\	矢	
\\	〜台	
\\	だい	
\\	バスはなん台いりますか?	
\\	し力がよくなるきかいを三台ちゅうもんしました。	
\\	あのかねもちしゃちょうはこうきゅうしゃを三台もっています。	
\\	台	
\\	市	
\\	し	
\\	大さか市で生まれました。	
\\	市のあたらしいマスコットキャラクターがけっていしました。	
\\	きょうとの市バスの中で、はいてしまった。	
\\	市	
\\	牛	
\\	うし	
\\	しゃちょうは牛をかっているらしい。	
\\	かの女は、牛がらのドレスをきて、牛どんやに行ってきました。	
\\	デイブは、牛にゅうのために牛をかっている。	
\\	(うし) 
\\	牛	
\\	戸口	
\\	とぐち	
\\	戸口に手ぶくろがおちていました。	
\\	たまたま、戸口でぬすみぎきしてしまったんです。	
\\	戸口に三枚のトーストがあった。	
\\	出口 
\\	入り口 
\\	口 
\\	戸
\\	口 
\\	く 
\\	ぐ.	戸, 口	
\\	外れ	
\\	はずれ	
\\	かれはまち外れの小さないえにすんでいます。	
\\	王女は、きせつ外れのドレスをきていた。	
\\	「あたりが出ると、もう一本」とかかれたアイスをかったが、ぜんぶ外れだった。	
\\	外 
\\	れ 
\\	れ
\\	外 
\\	れ 
\\	(はず) 
\\	外	
\\	太い	
\\	ふとい	い 
\\	山田さんのまゆげはとても太いです。	
\\	このおみせのめんはおもったより太かったです。	
\\	おれの母ちゃんは、ちょう太いんだぜ。	
\\	い 
\\	ふと 
\\	(ふと).	太	
\\	お父さん	
\\	おとうさん	
\\	お父さんはまい日リンゴをたべます。	
\\	お父さんにだって分からないことはある。	
\\	お父さんのねこはかわいい。	
\\	(お) 
\\	父, 
\\	さん. 
\\	お父様, 
\\	おとうさま. 
\\	父 
\\	(とう). 
\\	父	
\\	父	
\\	ちち	
\\	父がいるから、さびしくありません。	
\\	わたしの父は、ドイツ人です。	
\\	父はうしのちちをしぼるのがうまい。	
\\	父	
\\	五台	
\\	ごだい	
\\	たいじゅうけいを五台かいました。	
\\	あのいえのまえに、パトカーが五台もとまっている。	
\\	五台のバンがげきとつしてめりこんだ。	
\\	五, 台	
\\	外人	
\\	がいじん	
\\	田中さんのむすめさんは外人さんとけっこんしたんだよ。	
\\	でんしゃでよっぱらった外人にからまれた。	
\\	日本人と外人のあしのながさはかなりちがう。	
\\	外国人 
\\	国.
\\	人 
\\	じん 
\\	(じん).	外, 人	
\\	フランス人	
\\	ふらんすじん, フランスじん	
\\	フランス人のゆうじんができました。	
\\	二人はアメリカ人で、あとの三人はフランス人です。	
\\	とてもセクシーなフランス人シェフにあった。	
\\	"フランス 
\\	人 
\\	人. 
\\	人 
\\	人	
\\	生	
\\	なま	
\\	の 
\\	生チョコをつくりました。	
\\	生のやさいはにが手です。	
\\	生ビール一ちょう、下さーい。	
\\	生 
\\	生たまご 
\\	(なま) 
\\	生	
\\	友人	
\\	ゆうじん	
\\	友人のけっこんしきで、とうきょうにいっていました。	
\\	さらい月、アメリカ人の友人が日本にあそびにきてくれます。	
\\	あにの友人に、こく白された。	
\\	友, 人	
\\	毛	
\\	け	
\\	あしの毛がボーボーです。	
\\	うちの犬の毛はよくぬける。	
\\	今一ばん大切なのはあたまの毛だ。	
\\	(け) 
\\	毛	
\\	少女	
\\	しょうじょ, おとめ	
\\	その少女は、ゆうめいなアイドルです。	
\\	久しぶりに、天才少女ゴルファーがあらわれました。	
\\	犬と少女が、はまべをかけていった。	
\\	少, 女	
\\	公用	
\\	こうよう	
\\	今、公用でパリにきています。	
\\	これは公用のアカウントです。	
\\	おれたちがぬすむのは公用のかねだ。	
\\	公, 用	
\\	半分	
\\	はんぶん	
\\	この本を半分までよみました。	
\\	このクラスのがくせいの半分は女子です。	
\\	ざっくり日本人の半分ぐらいは、にんじゃです。	
\\	半, 分	
\\	半	
\\	はん	
\\	わたしはこのえいがを三かい半みました。	
\\	まであと一ヵ月半しかありません。	
\\	かの女は半じゅくたまごのようにかわいいね。	
\\	半	
\\	心	
\\	こころ	
\\	せつなさで、心がしめつけられた。	
\\	おもわず心の中でさけんでしまった。	
\\	もしアンパンマンがれいこくな心をもっていたらどうしよう。	
\\	(こころ) 
\\	心	
\\	今日は	
\\	こんにちは	
\\	今日は、おげんきですか?	
\\	「今日は」はひらがなでかくことの方がおおい。	
\\	大きなこえで、「今日は」といいましょう。	
\\	こんにちは, 
\\	きょうは! 
\\	こんにちは, 
\\	は 
\\	は 
\\	わ.	今, 日	
\\	大切	
\\	たいせつ	
\\	な 
\\	これは、メアリーが大切にしている人ぎょうです。	
\\	そばにいてくれる友人やこい人、かぞくを大切にした方がいいですよ。	
\\	大切なしょるいときくと、いつも中みをみたくなる。	
\\	大, 切	
\\	内	
\\	うち	
\\	の 
\\	はこの内がわをピンクいろにぬってください。	
\\	てきは内にあり。	
\\	内にひめたおもいを、日きにつづった。	
\\	(うち)! 
\\	内	
\\	戸	
\\	と	
\\	戸をたたかないでください。	
\\	うちのねこは戸をカリカリするのがすきです。	
\\	この戸は立てつけがわるい。	
\\	戸	
\\	ビー玉	
\\	びーだま, ビーだま	
\\	ビー玉をつかってゲームをしましょう。	
\\	アメリカではまいとし一月にビー玉をつかったゲームが上手なフランス人がつどうコンテストがあります。	
\\	かれは、けっこんのあかしに一つぶのビー玉をくれた。	
\\	ビー 
\\	玉, 
\\	玉	
\\	分	
\\	ふん, ぶん	
\\	ここからえきまで、あるいて十五分かかります。	
\\	その玉ねぎは女王の分だ。	
\\	ふぐを七分かんくすぐればどうなるかしっている?	
\\	ふん. 
\\	ぶん. 
\\	分	
\\	お母さん	
\\	おかあさん	
\\	お母さんはどこ?	
\\	ひこうき代なら、お母さんが出してあげるよ。	
\\	お母さんのはなげはながい。	
\\	お 
\\	母 
\\	さん 
\\	お母様, 
\\	おかあさま. 
\\	かあ 
\\	(かあ) 
\\	母	
\\	母	
\\	はは	
\\	母はタクシーにのりました。	
\\	いもうとは、どんどん母ににてきました。	
\\	母はとうふとふぐがすきだ。	
\\	母	
\\	市立	
\\	しりつ	
\\	の 
\\	そぼは市立びょういんに入いんしています。	
\\	市立としょかんまでのみちをおしえてください。	
\\	大さか市立大がくのまえでさか立ちをした。	
\\	市, 立	
\\	北	
\\	きた	
\\	えきの北には、びょういんがあります。	
\\	北ちょうせんが又ミサイルをはっしゃした。	
\\	北アメリカのゆうれいはとてもこわい。	
\\	(きた).	北	
\\	今	
\\	いま	
\\	今、しんぶんをよんでいます。	
\\	おれは今までなにをしていたんだ。	
\\	今から、ヘルメットにおはようといいます。	
\\	今	
\\	〜人	
\\	にん	
\\	ぜんぶで十人いました。	
\\	トーフグにはなん人のしゃいんがいますか?	
\\	わたしは三人子どもがいます。	
\\	7人 
\\	3人, 
\\	一人 
\\	二人 
\\	にん, 
\\	5人 
\\	ごにん, 10人 
\\	じゅうにん, 
\\	にん 
\\	人	
\\	大人しい	
\\	おとなしい	い 
\\	さとうさんの犬は、とても大人しいです。	
\\	なんなのアイツ。がっこうでは大人しいいんキャのくせに。	
\\	大人しい子どもだったよ。	
\\	大人 
\\	い?). 
\\	大人! 
\\	大, 人	
\\	古い	
\\	ふるい	い 
\\	そのふくは古いので、もういりません。	
\\	古いものにしがみついていては、あたらしいものをうけとることができません。	
\\	ガレージで、古いパンツをみつけた。	
\\	い 
\\	(ふる) 
\\	古	
\\	十万	
\\	じゅうまん	
\\	パチンコで十万円かちました。	
\\	このアイフォンは、十万六千九ひゃく円でした。	
\\	ウォッカをのむと、すう十万でん子ボルトものエネルギーがみなぎってくる。	
\\	十, 万	
\\	一台	
\\	いちだい	
\\	このテレビは、一台四万九千円です。	
\\	今はパソコン一台でかせぐじだいですからね。	
\\	タクシーを一台よんでください。	
\\	一, 台	
\\	人生	
\\	じんせい	
\\	人生ってすばらしい。	
\\	人生オワタ 
\\	人生はバナナのようなものさ。どれだけながいのかだれもしらないのさ。	
\\	人, 生	
\\	元	
\\	もと	
\\	きのう、元つまにあってきました。	
\\	元のきじはけされてしまいました。	
\\	元は月の出しんなんです。	
\\	(もと) 
\\	元	
\\	外	
\\	そと	
\\	外でまちましょう。	
\\	ねこたちは、まどの外にいるとりをジーッとみています。	
\\	今日は一日中外であくびをしていた。	
\\	(そと) 
\\	外	
\\	シアトル市	
\\	しあとるし, シアトルし	
\\	今はシアトル市にすんでいます。	
\\	わたしのお母さんは、シアトル市でそだちました。	
\\	あなたのへんがおしゃしんを、シアトル市におくります。	
\\	(シアトル/しあとる) 
\\	市 
\\	市	
\\	中古	
\\	ちゅうこ	
\\	の 
\\	中古ですが、ベンツをかいました。	
\\	このあたりは、中古ショップがじゅうじつしています。	
\\	中古のかぐのような人とはけっこんしたくない。	
\\	中, 古	
\\	中止	
\\	ちゅうし	
\\	する 
\\	やきゅうのしあいは、あめで中止になりました。	
\\	ざんねんですが、このきかくはいったん中止することになりました。	
\\	たのしみにしていた生ほうそうが中止になった。	
\\	中, 止	
\\	用	
\\	よう	
\\	なにも用がない。	
\\	なにかご用ですか?	
\\	用もないのにでんわしてくんじゃねーよ。	
\\	用	
\\	十台	
\\	じゅうだい	
\\	パソコンを十台ちゅう文しました。	
\\	ぼうはんカメラを十台せっちしました。	
\\	たんじょう日にはでんたくが十台ほしい。	
\\	十, 台	
\\	一万	
\\	いちまん	
\\	わたしのぜんざいさんは一万円です。	
\\	一万
\\	いったら、こく白します。	
\\	ちくしょう!にせの一万円さつをつかまされた。	
\\	一, 万	
\\	万	
\\	まん	
\\	日本へのひこうきだいは、十万五千円でした。	
\\	まいとしコウイチは、バレンタインデーに万たんいのチョコレートをもらいます。	
\\	ごしゅうぎのそうばは大たい三万円です。	
\\	万	
\\	今月	
\\	こんげつ	
\\	今月は大じなしけんがあります。	
\\	おそくとも、今月の中じゅんまでにかんせいさせてください。	
\\	今月は、ちょうのちょう子があまりよくない。	
\\	今, 月	
\\	生まれる	
\\	うまれる	
\\	日よう日、ヤンさんの子どもが生まれました。	
\\	マイケルせん手の、力づよいフルスイングから生まれるホームランのかずかずをごらんください。	
\\	生まれたばかりの子どもに、さけをのませるな。	
\\	う 
\\	生? 
\\	生む 
\\	う 
\\	(う) 
\\	生	
\\	切れる	
\\	きれる	
\\	テープが切れてしまいました。	
\\	でんわが切れるおとをきくとさみしくなる。	
\\	この小刀はよく切れる。	
\\	う 
\\	き 
\\	木 (き) 
\\	切	
\\	外れる	
\\	はずれる	
\\	あみ戸が外れてしまいました。	
\\	けがをして、スタメンから外れることになっちゃったんだ。	
\\	プラグが外れたので、
\\	ロボットはありがとうといえなくなった。	
\\	う 
\\	はず 
\\	(はず) 
\\	外	
\\	切る	
\\	きる	
\\	つぎに、かみをハサミで切ります。	
\\	いとを切るはさみをさがしています。	
\\	うっかりかみのけをみじかく切りすぎちゃったけど、そのあといぜんよりもよくナンパされるようになった。	
\\	う 
\\	木, 
\\	き. 
\\	きる 
\\	き.	切	
\\	今日	
\\	きょう	
\\	今日のしゅくだいは、二十一ページの二ばんです。	
\\	今日はいつもよりうり上げが少ない。	
\\	あら、今日が私のたん生日だったんですか?	
\\	(きょう). 
\\	今, 日	
\\	太る	
\\	ふとる	
\\	さいきん、たべすぎで少し太りました。	
\\	たべても太らない人がうらやましい。	
\\	太れば太るほど、わたしはおなかがすく。	
\\	う 
\\	ふと 
\\	(ふと) 
\\	太	
\\	生む	
\\	うむ	
\\	かれのはつげんは、ぎもんを生んだ。	
\\	ごかいを生むようなとうこうはしないでください。	
\\	おっとの子ではなかったが、生むしかなかった。	
\\	う 
\\	生まれる 
\\	う 
\\	生まれる, 
\\	(う).
\\	生	
\\	生きる	
\\	いきる	
\\	しつれんで生きるき力をうしなった。	
\\	今、かれが生きているのかすら分かりません。	
\\	わたしの大好きなアニメキャラが、今生きるかしぬかのせとぎわにいる。	
\\	う 
\\	い 
\\	(いき 
\\	き 
\\	生	
\\	引く	
\\	ひく	
\\	十引く四は六です。	
\\	どうせ、女のきを引くためにいってるだけでしょう。	
\\	引きがねを引いて、フライパンをうった。	
\\	う 
\\	(ひ) 
\\	引	
\\	分ける	
\\	わける	
\\	このにくまんを二人で分けましょう。	
\\	生しを分けるけつだんになった。	
\\	ピザを半分に分けるなら、大きい方を下さい。	
\\	う 
\\	る) 
\\	分かる 
\\	(わけ) 
\\	(わ). 
\\	分	
\\	止まる	
\\	とまる	
\\	あかしんごうで、くるまは止まりました。	
\\	心ぞうが止まるかとおもった。	
\\	こおりの上で、止まることができなかった。	
\\	う 
\\	(と) 
\\	(ま) 
\\	止	
\\	止める	
\\	とめる	
\\	おんがくを止めてください。	
\\	三びょうかんいきを止めてください。	
\\	かれがはしりさるのを止めるひつようがある。	
\\	う 
\\	(と), 
\\	止	
\\	分かる	
\\	わかる	
\\	おいしいかどうか分かりません。	
\\	分かるような、分からないような。	
\\	日本ごは分かりますか?	
\\	(か) 
\\	(わ) 
\\	分	
\\	用いる	
\\	もちいる	
\\	ににたなまえを用いるものがいます。	
\\	北ちょうせんは、かくへいきをおどしに用いる。	
\\	えをかくために用いるふでで、かの女をくすぐった。	
\\	う 
\\	もち 
\\	もちいる: 
\\	用	
\\	二万	
\\	にまん	
\\	このぶどうは二万円もしました。	
\\	二万フォロワーおめでとうございます。	
\\	二万ドルかけるよ。	
\\	二, 万	
\\	二台	
\\	にだい	
\\	どうしてけいたいを二台もっているんですか。	
\\	コウイチの母のいえには、ピアノが二台ある。	
\\	うちのトイレにはテレビが二台しかない。	
\\	二, 台	
\\	方	
\\	かた, ほう	
\\	どちらの方がすきですか。	
\\	あちらの方に行ってください 。	
\\	かの女のわらい方がきらいだ。	
\\	方 
\\	(かた) 
\\	方	
\\	広い	
\\	ひろい	い 
\\	友だちのへやは、広くてきれいです。	
\\	あのフランス人はおでこが広い。	
\\	すげー広いな!	
\\	い 
\\	ひろ, 
\\	ひろい.	広	
\\	冬	
\\	ふゆ	
\\	ことしの冬は、ゆきがたくさんふりました。	
\\	きょねんの冬にかいたえです。	
\\	冬をなめるといたい目にあうぞ!	
\\	冬	
\\	女の人	
\\	おんなのひと	
\\	わたしは女の人にいじめられた。	
\\	そのカフェは女の人だらけでした。	
\\	女の人としゃべるのが苦手です。	
\\	の 
\\	人 
\\	女. 
\\	女, 人	
\\	他人	
\\	たにん	
\\	他人のかねにたよるな。	
\\	他人のあなたには分からないでしょうが、この玉ねぎはわたしのいのちのおん人なんです!	
\\	他人とは思えないほど親切にしてくれたんだけど、単に自分の叔母さんと私を間違えていただけみたい。	
\\	他, 人	
\\	早々	
\\	そうそう, はやばや	
\\	早々におへんじいただきありがとうございます。	
\\	年まつだからか、人々が早々とすぎさっていく。	
\\	新年早々、借金を催促されるなんてひどすぎる。	
\\	早 
\\	々 
\\	早, 々	
\\	大気	
\\	たいき	
\\	大気おせんマップを毎日チェックしています。	
\\	そらがあおいのは、ちきゅうの大気の分子があおとみどりのひかりをさんらんするからです。	
\\	今日も、ぜんこく的に大気のじょうたいがふあんていです。	
\\	大, 気	
\\	赤ちゃん	
\\	あかちゃん	
\\	ぶじに、女の赤ちゃんがうまれました。	
\\	このかわいい女の子の赤ちゃんの名まえはエマちゃんです。	
\\	あなたの赤ちゃんの頃の写真が見てみたいわ。	
\\	ちゃん 
\\	ちゃん 
\\	あかちゃん 
\\	赤 
\\	赤	
\\	竹	
\\	たけ	
\\	このおはしは竹でできています。	
\\	竹は、一メートルぐらいのたかさまで切ってしまうと、ねまでかれてしまいます。	
\\	竹細工が本当に好きなんだ!うちのテレビも俺が竹から作ったんだよ。まあ、動かないけどね。でも、見た目がいいんだよ。	
\\	竹	
\\	竹の子	
\\	たけのこ	
\\	竹の子をゆでているところです。	
\\	ふじ山の入り口から北の方にのぼっていくと、と中に竹の子が山のようにとれるところがあります。	
\\	「あ、お兄さん。わかりました。わかりました。竹の子、好きなだけ持って行ってください。」とパンダが言った。	
\\	竹 
\\	子 
\\	竹, 子	
\\	太字	
\\	ふとじ	
\\	の 
\\	この字は太字にしてください。	
\\	マンションのエレベーターにのったら、でんこうけいじばんに赤い太字で「じしんです今すぐ外に出てください」とかひょうじが出てきて、マジびびったよ。	
\\	太字にすることで、その言葉を強調しましょう。	
\\	太 
\\	ふと 
\\	ふとい 
\\	字 
\\	太, 字	
\\	少年	
\\	しょうねん	
\\	少年は、やきゅうが大すきでした。	
\\	けいさつかんは、一ぱんてきに、少年はんざいのかがいしゃを、けいさつの少年かにれんこうします。	
\\	私は腕白な少年ではなく、気が弱くて無口な少年でした。	
\\	少, 年	
\\	虫	
\\	むし	
\\	虫はにが手です。	
\\	あの虫、口から白い糸をはいているよ!	
\\	ううん。その虫は噛まないよ。でも、もし君がそれでもそいつを殺すって言うなら、僕が君のことを噛むよ。	
\\	虫	
\\	子牛	
\\	こうし	
\\	白い子牛をかっているんです。	
\\	この子牛は、とても元気です。	
\\	お父さん!サンタ・クロースが昨夜子牛をくれたわ!	
\\	子 
\\	牛 
\\	子, 牛	
\\	平気	
\\	へいき	
\\	な 
\\	よく平気でいられますね。	
\\	王女はペットの子犬をなくしても平気な人です。	
\\	夫が会社を首になったが、意外と平気だ。	
\\	平, 気	
\\	見える	
\\	みえる	
\\	あそこの赤いはたが見えますか。	
\\	他人からどう見えるのかがすごく気になってし方がない。	
\\	太陽が眩しくて、風船がどのくらいの高さまで上がっていったのかほとんど見えなかった。	
\\	う 
\\	える 
\\	え 
\\	見	
\\	車	
\\	くるま	
\\	日よう日に、車でうみにいきました。	
\\	だれか車出してくれないかな?	
\\	私の車は、フロリダまでもたないだろう。	
\\	(くるま) 
\\	車	
\\	中央	
\\	ちゅうおう	
\\	の 
\\	中央に立っているのが、田中さんのおねえさんです。	
\\	大さかえきの、中央かいさつ口でまっています。	
\\	私の頭の中央部分には小さな凹みがある。	
\\	中, 央	
\\	字	
\\	じ	
\\	もっとていねいに字をかいてください。	
\\	八の字まゆ毛の男せいとけっこんしました。	
\\	この字は何という意味ですか?	
\\	字	
\\	耳	
\\	みみ	
\\	このねこの耳はちゃいろです。	
\\	パンの耳はわたしの大こうぶつです。	
\\	わたしは、左耳をミミと名付けた。右耳はよく聞こえないので、ママと名付けた。	
\\	耳	
\\	早い	
\\	はやい	い 
\\	よていより、十五分早くつきました。	
\\	かの女にこく白するのはまだ早いとおもうんだ。	
\\	豆腐屋さんの朝はいつも早い。	
\\	い 
\\	はや 
\\	""はや!
\\	早	
\\	元気	
\\	げんき	
\\	な 
\\	お父さんは、元気ですか。	
\\	かなちゃんは、元気いっぱいですね。	
\\	彼は元気そうに見えたが、実はひどい下りだったらしい。	
\\	元, 気	
\\	気分	
\\	きぶん	
\\	今日はあさから気分がいいです。	
\\	なんとなくピクニックに出かけたい気分です。	
\\	君が僕と結婚してくれたら、残りの人生ずっといい気分ですごせると思うよ。	
\\	気, 分	
\\	花火	
\\	はなび	
\\	花火が上がるのはきんよう日だけです。	
\\	昨日、かぞくで花火大かいに出かけました。	
\\	花火ってめっちゃいい匂いだよね?	
\\	ひ 
\\	び. 
\\	花 
\\	火. 
\\	花, 火	
\\	先ず	
\\	まず	
\\	先ず、下ごしらえをします。	
\\	先ず、その五十ねんかけてあつめた切手コレクションからだんしゃりしましょう。	
\\	先ずは、チキンの羽をむしりとって、それぞれの羽に名前をつけます。	
\\	(まず).	先	
\\	一年生	
\\	いちねんせい	
\\	川下さんは、小がく一年生のおとうとがいます。	
\\	ことしの四月、わたしは中がく一年生になり、わたしのいもうとは小学一年生、兄はこうこう一年生、母と父は大学一年生になります。	
\\	まだ小学一年生が学ぶ漢字ですら知りません。	
\\	二年生, 
\\	一, 年, 生	
\\	五百	
\\	ごひゃく	
\\	三月まつまでに、かん字を五百こおぼえてください。	
\\	日本には五百円玉というものがあります。	
\\	「どうやって五百頭もの馬を車に乗せるっていうの?」と女が聞いた。「そうだよね。代わりにバンを持ってきた方が良かったかな。」と男は答えた。	
\\	五, 百	
\\	平ら	
\\	たいら	
\\	な 
\\	パソコンを、どこか平らなところにおいてください。	
\\	その女の子は赤のビー玉を八つ平らなところにおいた。	
\\	平らで細長いパスタの名前覚えてない?	
\\	(たいら) 
\\	平	
\\	花	
\\	はな	
\\	ヒマワリは、とてもきれいな花です。	
\\	この花の花ことばは、はな毛ボーボーです。	
\\	何か花を使った料理が知りたいんだけど。花のスープとかって聞いたことある?	
\\	花	
\\	足	
\\	あし	
\\	足にタトゥーがある。	
\\	わたしは子どものとき、手のことを足、足のことを手だとおもっていました。	
\\	もし私が足を切り落としても、すぐにまた生えてくるんだよね?	
\\	(あし). 
\\	足	
\\	四十	
\\	よんじゅう	
\\	シャツを四十まいもっています。	
\\	あの天才アメリカ人は四十才です。	
\\	四十だいにもなると、ぜんしんがかゆい。	
\\	四, 
\\	四, 十	
\\	四百	
\\	よんひゃく	
\\	このしりょうを四百枚コピーしてください。	
\\	アメリカのシアトル市で、一やにして四百人がしっそうするというじけんがあった。	
\\	いったいどうして四百ドルも一本の爪楊枝を買うのに使ったんだい?	
\\	百 
\\	四, 百	
\\	四日	
\\	よっか	
\\	らい月の四日に、わたしと一しょにやきゅうをみにいきませんか?	
\\	七月四日はアメリカのどくりつきねん日です。	
\\	あのくまに出あって四日ご、ぼくのせなかにかびがはえた。	
\\	四, 
\\	ん 
\\	っ, 
\\	(よっ) 
\\	(か) 
\\	(いつ) 
\\	(五
\\	いつ).	四, 日	
\\	四つ	
\\	よっつ	
\\	このはこには、みかんが四つあります。	
\\	小さい玉ねぎを四つ下さい。	
\\	今朝、クロワッサンを四つか五つ食べました。	
\\	つ 
\\	ん 
\\	よん 
\\	っ 
\\	四	
\\	四千	
\\	よんせん	
\\	フォロワーが四千人になりました。	
\\	これは、ちゅうごく四千ねんのれきしがつくり上げた、せかいに一つしかないとうふです。	
\\	四千ぼんのでんちゅうをいっせいにたおせるくらい、きみのことをあいしているよ。	
\\	千 
\\	四 
\\	四, 千	
\\	主人	
\\	しゅじん	
\\	の 
\\	きのう、ご主人とどこかへいきましたか?	
\\	ご主人さま、それは少しかいすぎではないでしょうか。	
\\	私の主人は何の信念も無い政治家なんです。	
\\	主, 人	
\\	休日	
\\	きゅうじつ	
\\	休日は、えいがを見ることがおおいです。	
\\	たのしい休日をすごしてくださいね。	
\\	休日をツリーハウスで過ごした。	
\\	日 
\\	じつ.	休, 日	
\\	百	
\\	ひゃく	
\\	一しゅうかんで、百ページよみました。	
\\	王子がレベル百になったら王か女王にてんしょくできますよ。	
\\	百円で色々なものが買えるんですね。	
\\	百	
\\	氷	
\\	こおり	
\\	氷をつくっておいてください。	
\\	タピオカミルクティーを氷ぬきでちゅう文した。	
\\	「どうしてあの女は俺のグラスに氷を入れるのを忘れたんだ?」「彼女は今日自動車にひかれたんです。きっと飲み物を作っている時にまだふらふらしていたんだと思います。」	
\\	氷	
\\	丸ごと	
\\	まるごと	
\\	わたしは、そのうたを丸ごとあんきしています。	
\\	その牛、半分だけじゃなくて丸ごと一とう下さい。	
\\	このケーキには林檎が丸ごと入っています。	
\\	(ごと) 
\\	丸	
\\	不正	
\\	ふせい	
\\	な 
\\	不正をはたらいたのはだれですか。	
\\	中ごくからの不正アクセスにより、トーフグの
\\	アカウントがぜんぶのっとられてしまった。	
\\	俺の叔母は、いつも自分が今までにした不正やごまかしについて話をする。	
\\	不, 正	
\\	車内	
\\	しゃない	
\\	の 
\\	ちょっと車内でまっていてください。	
\\	シアトル市のバスの車内にスタバのコップをもったきょどうふしんな外人がいたとおもったら、ただのドバイのイケメン王子さまだった。	
\\	誰かが車内にベーコンをいくつか忘れていったようだ。	
\\	車, 内	
\\	宝石	
\\	ほうせき	
\\	の 
\\	あの赤い宝石のゆびわは、だれかのわすれものです。	
\\	あの大ふごうは、メイドに宝石があしらわれたそうじどうぐをつかわせている。	
\\	お姫様は美しい宝石のついた腕輪を時計回りに回しました。	
\\	宝, 石	
\\	先月	
\\	せんげつ	
\\	先月、田中さんにあいました。	
\\	先月、市立びょういんであなたのお兄さんを見ましたよ。	
\\	彼らは先月コスプレパーティーをしました。	
\\	先, 月	
\\	去年	
\\	きょねん	
\\	去年の四月に日本にきました。	
\\	去年、はじめて夏コミにさんせんしました。	
\\	彼はハンバーガーを毎日食べるので、去年よりも太ってしまった。	
\\	去, 年	
\\	左手	
\\	ひだりて	
\\	あねの左手には、ほくろがある。	
\\	左手を上げてください。	
\\	左手に、でんきゅうのタトゥーをほった。	
\\	手, 
\\	左, 手	
\\	文字	
\\	もじ	
\\	文字はきいろにしてください。	
\\	マイケルは八じかんかけてコウイチにしょうけい文字で手がみをしたためた。	
\\	あなたからのメール、文字化けしちゃったみたいなんだけど、何て書いてたの?	
\\	文字 
\\	文 
\\	もん, 
\\	も. 
\\	もん 
\\	も.
\\	文, 字	
\\	一代	
\\	いちだい	
\\	かれは、一代でここまでかいしゃを大きくした。	
\\	コウイチは一世一代の大しょうぶに打って出た。	
\\	彼は一代で巨額の財産を作ったが、彼の息子がさらにそれを二倍にした。	
\\	一, 代	
\\	主に	
\\	おもに	
\\	主に日本ごをべんきょうしています。	
\\	ポートランドでは、赤い車と白い車が主にうれています。	
\\	彼は研究者で、主にバナナに興味があります。	
\\	主 
\\	に 
\\	(おも). 
\\	主	
\\	男	
\\	おとこ	
\\	この男はだれですか。	
\\	今日は男友だちとパチンコを打ちにいくよていです。	
\\	知らない男と花火大会で意気投合した。	
\\	子 (おとこ), 
\\	男	
\\	名人	
\\	めいじん	
\\	かれは、けんどうの名人です。	
\\	かの女は、このくにに四人しかいないはいくの名人の一人です。	
\\	彼はチェスの名人であるだけでなく、トランプの名人でもある。	
\\	名, 人	
\\	仕方	
\\	しかた	
\\	おじぎの仕方をおしえてください。	
\\	フランス人の友人にフランスしきのあいさつの仕方をおしえてもらいました。	
\\	「運転の仕方」という本は今年のベストセラーとなった。	
\\	仕, 
\\	方. 
\\	仕, 方	
\\	百万	
\\	ひゃくまん	
\\	この
\\	は、百万まいうれたそうです。	
\\	で百万ドルもうけたので、きんでできたトーフグのおきものをつくるつもりです。	
\\	48の裁判を見に、百万人の人々が法廷に出席した。	
\\	百, 万	
\\	先々月	
\\	せんせんげつ	
\\	先々月、キューバにりょこうにいきました。	
\\	先々月、シアトル市で大あめがふったせいで、市のめいぶつの玉ねぎばたけがぜんめつしてしまった。	
\\	お母さんが先々月の携帯電話代を払ってくれた。	
\\	々 
\\	先, 々, 月	
\\	月見	
\\	つきみ	
\\	月見のおだん子をつくっています。	
\\	今日のよる、友人たちと、お月見をするよていです。	
\\	そ父は、月見の日に餅でのどを詰まらせて死にました。	
\\	見 
\\	月 
\\	月 
\\	つき 
\\	つきみ.	月, 見	
\\	名	
\\	な	
\\	この花の名をおしえてください。	
\\	きょ年のなつ、「きみの名は。」というえいがが日本ではやりました。	
\\	まだ彼女の名も知らない。	
\\	(な), 
\\	名	
\\	号	
\\	ごう	
\\	なん号車にのるよていですか。	
\\	天才かがくしゃと天才こうがくしゃの手によって、われわれのみらいを百パーセントよそうてきちゅうできるうらないロボットだい一号がかんせいしました。	
\\	日本では新聞の号外は大抵無料で配られます。	
\\	号	
\\	一気	
\\	いっき	
\\	にがいくすりを一気にのみほしました。	
\\	元気そうな父のかおを見たら、一気に気がぬけました。	
\\	僕のひいお婆さんは、ぎょう子を二百こ一気にたべた。	
\\	いち 
\\	一 
\\	いっ.	一, 気	
\\	一本気	
\\	いっぽんぎ	
\\	な 
\\	かれは、とても一本気なせいかくです。	
\\	モテないだん子には、じゅんじょうで一本気なおとこがおおい気がする。	
\\	俺の親友は、とても一本気な男で、90歳になった今でも初恋の女を思い続けている。	
\\	いち 
\\	いっ, ほん 
\\	ぽん, 
\\	き 
\\	ぎ. 
\\	一本気, 
\\	一, 本, 気	
\\	他	
\\	ほか	
\\	の 
\\	他にしつもんはありませんか。	
\\	クラスの他のがく生はぜんぜん大したことのないように見えた。	
\\	この卵をあいつの家に投げるのは、他の日にしようぜ。	
\\	(ほか)! 
\\	他	
\\	休止	
\\	きゅうし	
\\	する 
\\	の 
\\	わたしの大すきなバンドが、かつどうを休止しました。	
\\	パソコンのスリープと休止じょうたいのちがいをおしえてください。	
\\	しごとのと中で、小まめに小休止をとってエクササイズをするようにしています。	
\\	休, 止	
\\	皿	
\\	さら	
\\	お皿をあらってから、ベッドにいきました。	
\\	あとは、この竹の子を皿にもりつければかんせいです。
\\	(*皿
\\	(*皿
\\	うちのいいお皿を射撃の標的に使うのはやめてくれ!	
\\	皿	
\\	人気	
\\	にんき	
\\	な 
\\	の 
\\	このビールはとても人気があります。	
\\	あのどう人しの人気のひみつをさぐっている。	
\\	私がこの辺りで一番人気の高いベリー採集者であります。」と熊が言った。	
\\	人, 気	
\\	切手	
\\	きって	
\\	それから、ゆうびんきょくへいって、切手をかいます。	
\\	あの外人さんは切手を一万まいもかっていったよ。	
\\	彼女を失ってから、切手集めへの興味も失ってしまった。	
\\	切 
\\	手 
\\	っ, 
\\	切, 手	
\\	先	
\\	さき	
\\	の 
\\	分かるもんだいから先にときましょう。	
\\	この先のみちをまっすぐいって川をこえると、北アメリカに入ります。	
\\	申し訳ないけど、今日は先に元彼とした約束があるので、あなたとは会えません。	
\\	(さき). 
\\	先ず (まず) 
\\	さき).
\\	先	
\\	赤	
\\	あか	
\\	その赤のくつをとってください。	
\\	つぎは、まえがみのいろを赤にしようとおもっています。	
\\	オーランド・ブルームという名前を聞くだけで、今でも頬が赤くなる。	
\\	赤	
\\	赤い	
\\	あかい	い 
\\	赤いネクタイをください。	
\\	今年は赤いマニキュアがはやっています。	
\\	赤いニット帽を被った女の子がスケートをしている。	
\\	い 
\\	赤	
\\	休み	
\\	やすみ	
\\	かとうさんのつぎの休みはいつですか?	
\\	ひる休みにカフェでかき氷をたべました。	
\\	明日は休みなので、一日中一人でマリオカートをする予定です。	
\\	(やす).
\\	休	
\\	右手	
\\	みぎて	
\\	山下さんは、右手にくろいかさをもっています。	
\\	小さい女の子が、右手にビー玉をにぎりしめて立っていた。	
\\	右あし、右手、右めをつかうほうがすきだ。	
\\	手, 
\\	右, 手	
\\	二世	
\\	にせい	
\\	かれは、日けいアメリカ人二世の一人です。	
\\	せいこうしている二世タレントは少ない。	
\\	私は、父が日系イギリス人二世だということを知らなかった。	
\\	二, 世	
\\	本気	
\\	ほんき	
\\	な 
\\	の 
\\	いつになったら、本気を出すんですか。	
\\	ついにコウイチが本気を出してきた。	
\\	私があの男の頭の毛をむしったなんて、本気で思ってないよね?少しは私を信用してよ。	
\\	本, 気	
\\	かき氷	
\\	かきごおり	
\\	かき氷は、あついなつによくたべられます。	
\\	この赤いシロップがかかったかき氷はトマトあじです。	
\\	世界で一番いい気分になる時は、嫌いなやつに冷たいかき氷をぶちまけてやった時だよ。	
\\	かき 
\\	氷 
\\	(こおり) 
\\	ごおり 
\\	かき, 
\\	氷	
\\	貝	
\\	かい	
\\	かえりに、貝をかってきましょうか。	
\\	貝るいやこうかくるいをたべると、からだがかゆくなります。	
\\	私は、牡蠣の貝殻の中に置かれていることに気付かずに、婚約指輪を飲み込んでしまった。	
\\	貝	
\\	不足	
\\	ふそく	
\\	する 
\\	な 
\\	おつりが不足しています。	
\\	しきん不足で中央せいふのきかくしたプロジェクトが中止になった。	
\\	経験が不足しているうえにいつも寝不足なので、あなたは首です。	
\\	足 
\\	不, 足	
\\	小皿	
\\	こざら	
\\	小皿をテーブルにならべてください。	
\\	かくじ、小皿にとり分けてたべましょう。	
\\	このどでかいステーキを、その小皿に載せるのは無理だ。	
\\	"さら 
\\	ざら 
\\	小 
\\	(子) 
\\	小, 皿	
\\	気	
\\	き	
\\	かのじょはとつぜん、気をうしなった。	
\\	気をつけてかえってきてね。	
\\	おまえ、気は確かか?	
\\	気	
\\	石	
\\	いし	
\\	人にむかって石をなげるんじゃない。	
\\	石が一つくつに入った。	
\\	石と石がぶつかるおとにいらいらする。	
\\	石	
\\	一文字	
\\	いちもんじ	
\\	女の子は、おこって口を一文字にむすびました。	
\\	ペーパータオルでさかなの水気をふきとったら、かわに一文字の切り目を入れてください。	
\\	白鳥たちは、うちの風呂場からその湖へ、真一文字になって飛んでいった。	
\\	一, 文, 字	
\\	一打	
\\	いちだ	
\\	ボールは、一打でグリーンにのりました。	
\\	テニスでは、さいしょの一打を「サービス」とよびます。	
\\	一打逆転の大チャンスだ。	
\\	一, 打	
\\	代用	
\\	だいよう	
\\	する 
\\	の 
\\	サワークリームをヨーグルトで代用しました。	
\\	きもののこしひもは、ストッキングでも代用かのうです。	
\\	糊がないので飯粒で代用した。	
\\	代, 用	
\\	名字	
\\	みょうじ	
\\	の 
\\	さくらさんの名字は何といいますか。	
\\	コウイチの名字をおしえてくれたら、お礼にてつ人二十八号のフィギュアをあげるよ。	
\\	私の名字はあなたには絶対教えません。	
\\	名 
\\	みょう. 
\\	名, 字	
\\	先生	
\\	せんせい	
\\	先生がアンナさんに手がみをかきました。	
\\	ダリン先生のつぎの休みはいつですか?	
\\	先生と生徒の恋愛についてどう思いますか?	
\\	先, 生	
\\	申し申し	
\\	もうしもうし, もしもし	
\\	申し申し、花子ちゃん?	
\\	あの、申し申し?きいていますか???	
\\	「申し申し。鈴木と申しますが、佐藤さんはいらっしゃいますか?」	
\\	申し申し 
\\	申	
\\	お礼	
\\	おれい	
\\	あなたにお礼をいいたかったんです。	
\\	お礼にビールをもらいました。	
\\	この町が平和を取り戻したことについて、スパイダーマンにお礼を言いたい。	
\\	お 
\\	礼	
\\	先日	
\\	せんじつ	
\\	先日はどうもありがとうございました。	
\\	先日いった北かいどうのおまつりは二十日かんもつづくものらしい。	
\\	彼女は先日からずっと病気で寝ていると嘘をついた。	
\\	日 
\\	じつ.	先, 日	
\\	三世	
\\	さんせい	
\\	ルイ三世は、フランスのきぞくです。	
\\	ざい日三世のほとんどはかんこくごがはなせません。	
\\	僕はアニメルパン三世を毎日見ます。	
\\	三, 世	
\\	糸	
\\	いと	
\\	糸は、ここにまいてください。	
\\	うんめいの赤い糸をしんじますか?	
\\	私は足の傷口を縫うのに針と糸を使った。	
\\	糸	
\\	引き分け	
\\	ひきわけ	
\\	バスケのしあいは、引き分けでした。	
\\	アメリカのベストピックアップトラックをきめるたたかいは、今のところトヨタとフォードの引き分けのじょうたいがつづいている。	
\\	私は、じゃんけんが三対三の引き分けに終わったことにとても腹を立てた。	
\\	引, 分	
\\	足りる	
\\	たりる	
\\	これだけで、足りるのかな。	
\\	モナコの王女はせけんしらずなので、車をかうのに十万円もあれば足りるとおもっている。	
\\	今夜、ワインコルクで何か工作をするのに足りるだけのワインを飲むことを誓います。	
\\	う 
\\	た 
\\	た 
\\	(た). 
\\	足	
\\	写る	
\\	うつる	
\\	あなたのお兄さんは、このしゃしんに写っていますか。	
\\	今年とった五百まいの写しんの内、なんと百三まいの写しんに去年しんだ父のすがたが写っていた。	
\\	この写真に写っている男の人は、私のファーストキスです。	
\\	う 
\\	る. 
\\	うつ 
\\	うつうつうつ 
\\	写	
\\	気に入る	
\\	きにいる	
\\	このぼうし、とっても気に入りました。	
\\	玉ねぎ王子に気に入ってもらうために、アメリカ中の玉ねぎをかいしめて玉ねぎごてんをたてました。	
\\	君の新しい子犬が気に入ってもらえるといいんだけど。	
\\	気 
\\	入る 
\\	はいる 
\\	入れる/いれる? 
\\	いる. 
\\	はいる 
\\	いれる 
\\	いる.	気, 入	
\\	写す	
\\	うつす	
\\	となりのページに、手本の字を写してください。	
\\	ビエトの写す写しんの大ファンです。	
\\	お経を書き写すには、千円支払う必要があります。	
\\	う 
\\	す. 
\\	(す) 
\\	うつ 
\\	うつうつうつ 
\\	写	
\\	見分ける	
\\	みわける	
\\	本ものかにせものか見分けられない。	
\\	こい人がサイコパスかどうか見分ける方ほうはありますか?	
\\	首輪をつけておくべきだったよね。だって、どっちの犬がどっちの犬か見分けるのが難しくなっちゃったよ。	
\\	(分ける) 
\\	見 
\\	分ける 
\\	見, 分	
\\	出かける	
\\	でかける	
\\	母は、デパートへかいものに出かけました。	
\\	おっとが子どもをつれて出かけたので、今日はドラクエ三まいです。	
\\	日曜日は家族で海に出かけるの。	
\\	かける 
\\	かける 
\\	出かける
\\	出る, 
\\	出	
\\	打つ	
\\	うつ	
\\	カブレラがホームランを打ちました。	
\\	ぼくの先生は、キーボードを打つのが世かいで一ばんはやい。	
\\	ママは狂って、人参で僕のおしりを何度もひどく打った。	
\\	う 
\\	(う). 
\\	打	
\\	代わる	
\\	かわる	
\\	もしよければ、代わりましょうか。	
\\	王子は、けん玉に代わるあたらしいあそびを見つけた。	
\\	私はスティーブ・ジョブズに代わる天才を探しています。	
\\	う 
\\	か 
\\	(か). 
\\	代	
\\	休む	
\\	やすむ	
\\	わたしはいえで休んでいます。	
\\	やるときにきちんとやることをやって、休むときはしっかり休んだ方がいいよ。	
\\	一日で休むことができるのは、トイレに座っている時だけだ。	
\\	う 
\\	(やす).	休	
\\	申す	
\\	もうす	
\\	トーフグのコウイチと申します。	
\\	トーフグが日本のわかものにモノ申す!	
\\	申すまでもなく、私の上司は性転換手術をなされました。今の彼女は大変嬉しいそうだ。	
\\	う 
\\	ます 
\\	申します. 
\\	申	
\\	見る	
\\	みる	
\\	このもんだいでは、えを見ながらしつもんをきいてください。	
\\	その女がのっていた車のうんてん手は、よく見るとわたしの主人でした。	
\\	ちょっと僕の手を見て、それが縮んでいるかどうか教えてくれませんか?	
\\	う 
\\	見	
\\	見せる	
\\	みせる	
\\	みせの人は、なんのえを女の人に見せましたか?	
\\	今年のなつは、すはだをチラリと見せるトップスが人気のようです。	
\\	もし俺がこのポーカーゲームでかったら、お前らに俺のリザードンのカードを見せてやるよ。	
\\	う 
\\	せ 
\\	見	
\\	町	
\\	まち	
\\	この町には、コンビニがおおいです。	
\\	よる、一人でこの町をあるくのはきけんだ。	
\\	私の町は、最高のエアギターを作ることで有名です。	
\\	(まち) 
\\	町	
\\	宝	
\\	たから	
\\	宝をどこにかくしたんだ?	
\\	この宝のちず、三ばいにかくだいしてプリントアウトしておいてくれない?	
\\	宝箱に隠しておいたサマージャンボ宝くじがなくなってる!	
\\	(たから), 
\\	宝	
\\	二十日	
\\	はつか	
\\	二十日にテストがあります。	
\\	あと二十日で2019年です。	
\\	毎月二十日は給料日です。	
\\	二十 
\\	日 
\\	(はつ) 
\\	(か) 
\\	二, 十, 日	
\\	二百	
\\	にひゃく	
\\	オーブンを、二百どによねつしてください。	
\\	フェイスブックで二百もいいねをもらいました。	
\\	僕は、レジの人に二百ドルをあげて、これで制汗剤を買うように言いました。	
\\	二, 百	
\\	花見	
\\	はなみ	
\\	する 
\\	ことしは花見にいかなかった。	
\\	北りくち方の花見の名しょをごしょうかいします。	
\\	死ぬ前にもう一度花見をしながら宴会がしたいなあ。	
\\	花, 見	
\\	村	
\\	むら	
\\	ちかくに村があります。	
\\	あなたのお母さんはこの村で生まれたのよ。	
\\	ええ、私はあなたの村で一度本当に馬糞を踏んでしまったんです。	
\\	村	
\\	村人	
\\	むらびと	
\\	村人は、年よりばかりです。	
\\	なんかファンタジーゲームの村人っぽいセリフだよね。	
\\	悲観主義者は「もうコップ半分しかない」と言うが、その村人は「おかわり!」と言って私を驚かせた。	
\\	村 
\\	人 
\\	びと 
\\	村, 人	
\\	見方	
\\	みかた	
\\	このひょうの見方が分かりません。	
\\	いろんな見方やとらえ方があります。	
\\	別の見方をすれば、トーフグは河豚味の豆腐だと考えることもできる。	
\\	見, 方	
\\	天気	
\\	てんき	
\\	今日は天気がわるいです。	
\\	天気よほうをかくにんしましょう。	
\\	天気と気候の違いは何ですか?	
\\	天, 気	
\\	平日	
\\	へいじつ	
\\	平日は仕ごとをしています。	
\\	ざんねんだけど、平日だからたぶんいけないなぁ。	
\\	私は料理人で、平日は仕事で忙しいので、家庭ではふだん料理をしません。	
\\	日 
\\	じつ 
\\	平, 日	
\\	白い	
\\	しろい	い 
\\	外では、白いゆきがふっていました。	
\\	うちの牛は白いのよりもくろいのの方が元気がいい。	
\\	フワフワで白いウサギが野原でピョンピョン跳ねていました。	
\\	い 
\\	白 
\\	白	
\\	年内	
\\	ねんない	
\\	年内には、かんせいするはずです。	
\\	年内にフォロワー10万人目ざします。	
\\	この暗殺は、年内に遂行されなくてはならない。	
\\	年, 内	
\\	世	
\\	よ	
\\	このアプリは、まだ世に出ていません。	
\\	まったく世も末だね。	
\\	暗殺者は別にして、スズメ蜂ほど怖いものはこの世の中にない。	
\\	(よ).
\\	世	
\\	年	
\\	とし	
\\	どうぞよいお年をおむかえください。	
\\	今年もぶじに年をとりました。	
\\	あくる年、エリカは白い牛に乗った王子様に出会いました。	
\\	(とし) 
\\	年	
\\	2011年	
\\	にせんじゅういちねん	
\\	2011年に、日本にいきました。	
\\	主人は、2011年から火せいではたらいています。	
\\	父は、2011年にようやくドラッグの使用を認めました。	
\\	(二千/にせん), 
\\	(十一/じゅういち). 
\\	年 
\\	年	
\\	年中	
\\	ねんじゅう	
\\	この町のそらは、年中どんよりしています。	
\\	トーフグのしゃいんは、二十四じかん年中むきゅうではたらいている。	
\\	この国では一年中雨が降るんだが、妻もまた年中その文句を言うんだ。	
\\	中 
\\	じゅう. 
\\	ちゅう 
\\	ぢゅう, 
\\	ぢ 
\\	じ 
\\	じゅう. 
\\	年, 中	
\\	休学	
\\	きゅうがく	
\\	する 
\\	大学を一年かん休学していたんです。	
\\	ビエトは大学のとき、学こうを一年かん休学してじ分さがしのたびに出た。	
\\	「どうしてケリーが三週間も休学しているか知ってる?」「知ってるよ。ケリーは結膜炎がひどいらしいよ。」	
\\	休, 学	
\\	仕草	
\\	しぐさ	
\\	かの女が花をつむ仕草が大すきです。	
\\	男せいがすきな女せいの仕草はなんですか。	
\\	煙草を吸う大人の仕草にドキッとした。	
\\	くさ 
\\	ぐさ.	仕, 草	
\\	作用	
\\	さよう	
\\	する 
\\	アルコールの作用をおしえてください。	
\\	今日、学こうで、作用・はん作用のほうそくを学んだ。	
\\	このケーキの味がする薬には、副作用はありませんか?	
\\	作 
\\	さ 
\\	さく? 
\\	作, 用	
\\	空気	
\\	くうき	
\\	ちょっとは空気をよんでください。	
\\	冬のあさの空気はつめたい。	
\\	海賊の宇宙船に乗っている猿は、三時間後に大気圏に突入する予定だ。猿は、肺に好きなだけ空気を吸い込むことが待ちきれない。	
\\	ここの空気はいいね!	
\\	空, 気	
\\	万人	
\\	ばんにん	
\\	このはいゆうは、万人に人気がある。	
\\	セキセイインコのにおいは万人にうけるにおいらしい。	
\\	これは万人に共通の感想だろう。	
\\	万 
\\	ばん 
\\	まん. 
\\	万, 人	
\\	古来	
\\	こらい	
\\	の 
\\	カッパは、日本古来のようかいです。	
\\	古来、この池は、「ワニカニの池」という名まえでしられていました。	
\\	トイレは、古来から現代へと受け継がれてきた素晴らしい発明品だ。	
\\	古, 来	
\\	大作	
\\	たいさく	
\\	このえいがは、ヒッチコックの大作だ。	
\\	今世き一ばんの大作ともいわれているえいがのタイトルは「フグの一生」です。	
\\	ベートーベンが作曲した曲は、どれも大作だった。	
\\	大, 作	
\\	角	
\\	かく, かど	
\\	つぎの角を、左にまがってください。	
\\	あの先生は、とうふの角にあたまをぶつけて気ぜつしたらしい。	
\\	正三角形では,三つの角の大きさがみんな同じです。	
\\	かく 
\\	(かど), 
\\	(かど). 
\\	角	
\\	考古学	
\\	こうこがく	
\\	の 
\\	わたしは、日本の考古学にきょうみがあります。	
\\	今月末でこの考古学のしごとをやめるんです。	
\\	この大学では、考古学を学べます。	
\\	考, 古, 学	
\\	当たり	
\\	あたり	
\\	たからくじで当たりが出るかくりつはとてもひくいです。	
\\	あんなしょっぼい当たりのゴロでヒットになるわけがない。	
\\	僕が開いた父さんの50歳の誕生日パーティーは大当たりで、父さんはそこで新しい彼女まで見つけてしまった。	
\\	あ 
\\	(あ).	当	
\\	今すぐ	
\\	いますぐ	
\\	今すぐ、会いに行きましょう。	
\\	テキストフグのシーズン1なら、むりょうで今すぐよめますよ。	
\\	いいから黙って今すぐ来い!	
\\	今 
\\	すぐ 
\\	今 
\\	いま), 
\\	今	
\\	牛肉	
\\	ぎゅうにく	
\\	の 
\\	いえのちかくのえきででんしゃをおりて、スーパーに牛肉をかいに行きました。	
\\	去年、えいこくさん牛肉のゆ入がさいかいされました。	
\\	牛肉はありえないくらい生焼けだ。この牛肉のせいで今夜はキッチンの悪夢を見ることになりそうだ。	
\\	牛, 肉	
\\	生まれ	
\\	うまれ	
\\	わたしは日本人ですが、生まれはサンフランシスコです。	
\\	つまは、アメリカ生まれの日本人なんです。	
\\	生まれは大阪ですが、育ちは奈良です。	
\\	生まれる 
\\	生まれる. 
\\	る 
\\	生	
\\	青い	
\\	あおい	い 
\\	青いとりがいます。	
\\	そのリンゴ、まだ青いんじゃないの。	
\\	うわっ、この青いペンキ、あの赤いペンキと全く同じ臭いがする。	
\\	い 
\\	あお 
\\	(あお).	青	
\\	体	
\\	からだ	
\\	の 
\\	おふろに入ると、はじめに体をあらいます。	
\\	ジャマルって何でそんなに体がやわらかいの?	
\\	なぜだか、体中がかゆいんだ。	
\\	(からだ). 
\\	からだ.	体	
\\	男の子	
\\	おとこのこ	
\\	まえにいる男の子は、おか田さんの子どもです。	
\\	おめでとうございます、お母さん。かわいい男の子ですよ。	
\\	あの男の子は元気が良すぎる。今さっきも叔母さんの頭に卵を二十個投げつけて、うまく逃げ切ったんだ。	
\\	の. 
\\	男, 子	
\\	〜斤	
\\	きん	
\\	ストレスで4斤のパンを一気に食べてしまった。	
\\	食パンを五斤買うなら、自分で焼いた方が安いよ。	
\\	8斤のパンで、卵サンドが23個とツナサンドが23個できる予定だ。	
\\	斤	
\\	兄弟	
\\	きょうだい	
\\	わたしには、兄弟が二人います。	
\\	マリオとルイージって兄弟だったの!?	
\\	あの兄弟は好きだけど、あいつらのもっているぬいぐるみは好きじゃない。	
\\	兄, 弟	
\\	毛虫	
\\	けむし	
\\	キャベツに毛虫がついていました。	
\\	毛虫かと思ったら妹のつけまだった。	
\\	私は巨大な毛虫と巨大なムカデが戦っているところをビデオに収めた。	
\\	虫 
\\	毛 
\\	毛 
\\	毛, 虫	
\\	近い	
\\	ちかい	い 
\\	わたしのいえはえきから近いです。	
\\	今日は、ほぼすっぴんに近いです。	
\\	アメリカのビールって、カヌーの上ででんぐり返しをするみたいなものだよね。超水に近いんだもん。	
\\	い 
\\	ちか 
\\	(ちか) 
\\	近	
\\	色	
\\	いろ	
\\	何色がすきですか。	
\\	ピーマンはうれるとみどり色から赤色になります。	
\\	僕は色盲なので、赤色はオレンジ色に、青色は黒色のように見えます。	
\\	色	
\\	会社	
\\	かいしゃ	
\\	会社へは、何で行っていますか。	
\\	トーフグは日本ごの学しゅうきょうざいをつくったり、日本についてのじょうほうをはいしんしたりする会社です。	
\\	私が働く会社の社長は、ハムスターのような容貌をしている。	
\\	会, 社	
\\	作文	
\\	さくぶん	
\\	する 
\\	この作文を、中川先生にわたしてください。	
\\	作文のテストで百てんをもらいました。	
\\	僕の作文は、まだ間違いだらけです。	
\\	作, 文	
\\	工作	
\\	こうさく	
\\	する 
\\	工作はとくいではありません。	
\\	アリバイ工作にしっぱいした。	
\\	私は小学生の頃、夏休みの自由工作の宿題で、アイスクリームを作る機械を作った。	
\\	工, 作	
\\	大会	
\\	たいかい	
\\	の 
\\	日よう日は、サッカーの大会に出じょうします。	
\\	花火大会では、アイスクリームがよく売れます。	
\\	私がマラソン大会で優勝したのは、オナラを止めることができず、他のランナーがその臭いを嗅ぐたびにペースを落としていったからだ。	
\\	大, 会	
\\	図	
\\	ず	
\\	下の図を見てください。	
\\	クリックすると、かく大図がひょうじされます。	
\\	あなたのお兄さんは、自分と彼女の関係を簡単な図に表しました。	
\\	図	
\\	外交	
\\	がいこう	
\\	今日は、日本の外交についてはなしあいをしました。	
\\	外交かんが不足している。	
\\	日本は外交があまり上手ではない。	
\\	外, 交	
\\	下町	
\\	したまち	
\\	わたしは、下町でそだちました。	
\\	とうきょうの下町にあるむかしながらのさかばに行ってみたいです。	
\\	下町とは、日本の古いスタイルのダウンタウンのことで、現代のものとは異なっている。	
\\	下, 町	
\\	工学	
\\	こうがく	
\\	わたしのせんもんは、工学です。	
\\	ハーバード大学の工学のクラスは目玉がとび出るほどむずかしかった。	
\\	僕はこの仕事にぴったりだと思うんだけど、応募するには工学の学位がいるんだ。	
\\	工, 学	
\\	毎日	
\\	まいにち	
\\	わたしは日本ご学こうで、毎日べんきょうしています。	
\\	名古やから、大さかにあるアメリカ村に毎日かよっています。	
\\	毎日17から21個のチーズバーガーを食べます。	
\\	毎, 日	
\\	毎月	
\\	まいつき, まいげつ	
\\	うちの学こうでは、毎月テストがあります。	
\\	ゆうふくないえではありませんでしたが、わたしが大学生のとき、りょうしんは毎月生かつひをしおくりしてくれました。	
\\	毎月どのくらい家賃を払っていますか?	
\\	毎 
\\	月. 
\\	まいつき. 
\\	まいげつ 
\\	毎, 月	
\\	毎回	
\\	まいかい	
\\	このマラソン大会には、毎回出じょうしています。	
\\	コウイチは毎回ミーティングのと中でクネクネうごく。	
\\	笑おうとすると、毎回歯茎から血が出始める。	
\\	毎回?	
\\	毎, 回	
\\	毎年	
\\	まいとし, まいねん	
\\	山田さんは、毎年どこにりょ行に行きますか?	
\\	あのか手は毎年引たいを口にしているので、きっと今回もただのつりでしょう。	
\\	私は、ジョンソンさん一家のために、10年間毎年新しい扉を建てている。どうしてジョンソンさんたちはトロールに扉を使わせないようにしないのだろうか。	
\\	毎 
\\	年 
\\	年 
\\	まいねん 
\\	まいとし 
\\	毎, 年	
\\	羽	
\\	はね	
\\	小とりは、羽をパタパタうごかしました。	
\\	手羽先のからあげはたべたことがあるけど、ちょうちょうの羽のからあげはたべたことがありません。	
\\	「この羽、どの鳥の羽か分かるかな?」と彼は自分の鼻毛を持ちながら聞いてきた。	
\\	羽	
\\	林	
\\	はやし	
\\	毎あさ、林をさんぽしています。	
\\	林先生が早足で林をあるいているのを見ちゃった。	
\\	家に帰るまで待てずに、林の中に濡れた靴下を残した。	
\\	林	
\\	〜形	
\\	けい	
\\	です・ます形ではなしをするとていねいにきこえます。	
\\	ひてい形から会わに入る人はコミュしょうのかのうせいがたかい。	
\\	どうして今、過去形で言ったの?	
\\	形	
\\	〜年来	
\\	ねんらい	
\\	かれは、わたしの三十年来の友人です。	
\\	わたしの年来のきぼうがついにかないました。	
\\	数年来彼女をデートに誘っているんだが、いつも「髪の毛を洗うのに忙しいから」と断られる。	
\\	来年 
\\	来 
\\	5年来 
\\	年, 来	
\\	大体	
\\	だいたい	
\\	大体あっています。	
\\	大体同じぐらいのせです。	
\\	私は大体6時に起きます。	
\\	大, 体	
\\	金	
\\	きん	
\\	金のネックレスをプレゼントしました。	
\\	お金があったら、金正ウンとおそろいの、金のロレックスをかいたいとおもっています。	
\\	この毛糸は24時間後に金に変わるんだけど、クリスマスプレゼントとして少しあんたにあげるよ。感謝しろよな。	
\\	お金, 
\\	金	
\\	草	
\\	くさ	
\\	そろそろ草をからなくてはなりません。	
\\	お金がないから草をたべて生きています。	
\\	どうして隣の人の草はあんなに青いんだろう。どうして彼の家はあんなに大きいんだろう。	
\\	草	
\\	本社	
\\	ほんしゃ	
\\	本社はとうきょうにあります。	
\\	月よう日に、トーフグの本社におじゃましてきました。	
\\	今日は本社がやけに静かだな。俺が働いていない音まで聞こえるぜ。	
\\	本, 社	
\\	里心	
\\	さとごころ	
\\	ちょっと里心がついちゃったんです。	
\\	なつかしいメロディーに、里心がわいた。	
\\	この仔犬は里心がついて泣いているのかもしれないけど、そんなのかまうもんか。	
\\	里 
\\	心 
\\	心 
\\	ごころ. 
\\	里, 心	
\\	里	
\\	さと	
\\	いつ里にかえってくるんだい。	
\\	そんな言ばづかいをしていると、お里がしれますよ。	
\\	ふる里に帰るんだけど、顔見知りに会わないことを願っているよ。	
\\	里	
\\	何日	
\\	なんにち	
\\	何日仕ごとを休んでいたんですか。	
\\	何日がごつごうよろしいですか?	
\\	あなたはその島から何日間出られなかったんですか?	
\\	何, 日	
\\	何人	
\\	なんにん	
\\	そのパーティーには、何人の人がさんかしましたか。	
\\	今までこの池で何人の人々がさつ人ぎょにころされましたか。	
\\	夕食にはいったい何人の人がやって来るんですか?	
\\	何, 
\\	何 
\\	人. 
\\	人 
\\	にん 
\\	何, 人	
\\	何回	
\\	なんかい	
\\	日本へ来るのは、今回で何回目ですか。	
\\	このビデオ、おもしろすぎて何回も見てしまうわ。	
\\	事をはっきりさせるまでに、いったい何回この会話を繰り返さないといけねえんだよ?だから俺はリューク・スカイウォーカーじゃねえんだよ。映画でその役を演じただけだっつーの。俺の名前はマークだ。シカゴに住んでる。	
\\	何, 
\\	何, 回	
\\	外来	
\\	がいらい	
\\	の 
\\	外来の言ばは、カタカナでかきます。	
\\	毎日外来に来ていたかんじゃさんが、来なくなったので心ぱいしています。	
\\	日本のものと全然違うので、私はとても外来品に興味がある。	
\\	外, 来	
\\	自立	
\\	じりつ	
\\	する 
\\	の 
\\	むすこもようやく自立しはじめました。	
\\	自立しんけいしっちょうしょうがなおったら、自立しようとおもっています。	
\\	もしあなたが強くで自立した日本人女性で、男なんて必要ないというなら、これをシェアしてください。	
\\	自, 立	
\\	体内	
\\	たいない	
\\	の 
\\	ちが、体内でどんなはたらきをしているのかしっていますか。	
\\	カボチャに多くふくまれるβカロテンは体内でビタミン
\\	にかわります。	
\\	人間の体か吸血鬼の体かにかかわらず、血液は体内を循環する。	
\\	体, 内	
\\	皮肉	
\\	ひにく	
\\	な 
\\	の 
\\	まったく皮肉なものですね。	
\\	あの肉やさんは、大した皮肉やだよ。	
\\	という曲の中に、うまい皮肉はひとつも見当たらなかった。	
\\	皮, 肉	
\\	入社	
\\	にゅうしゃ	
\\	する 
\\	この会社には、四月に入社しました。	
\\	トーフグの入社しけんはグーグルの入社しけんよりもむずかしい。	
\\	大学を卒業したらすぐに、大きな金融機関に入社するか、アメリカ合衆国の大統領になれたらいいなと思っている。	
\\	入, 社	
\\	大声	
\\	おおごえ	
\\	の 
\\	そんな大声を出さないでください。	
\\	おもわず大声を出してわらってしまった。	
\\	クリスティーナは大声の持ち主で、残念なことにそれを使うのを全くはばからない。	
\\	声 
\\	ごえ 
\\	こえ). 
\\	大 
\\	大きい (おおきい). 
\\	""おお!.
\\	大, 声	
\\	光	
\\	ひかり	
\\	太ようの光がまぶしいので、カーテンをしめてください。	
\\	月の光にてらされて、かなえちゃんの歯がまるでパールのように見えるよ。	
\\	北朝鮮で行方不明になったあなたの子供には、まだ微かに希望の光が残っている。	
\\	(ひかり) 
\\	光	
\\	光年	
\\	こうねん	
\\	ちきゅうから月までは、一おく分の四光年です。	
\\	ポートランドの中しんから北の方に四光年すすんだら、どこにつきますか。	
\\	光年とは、光の速度で1年間進み続け到達出来る距離だ。貴様が俺と対決しようなんて、一万光年早いわ。	
\\	光, 年	
\\	文学	
\\	ぶんがく	
\\	文学をよむいみとは何だろうか。	
\\	わたしは大学で、イギリス文学をせんこうしていました。	
\\	村上春樹の作品は、純文学のふりをした大衆文学だと思います。	
\\	文, 学	
\\	男の人	
\\	おとこのひと	
\\	ああ!この男の人が、すず木さんですか。	
\\	あそこに立っている外人の男の人のとなりにいる女の人があのゆう名なまつい一代さんです。	
\\	背の高い男の人は
\\	シャツを着ていて、背の低い男の人はセーターを着ています。	
\\	の 
\\	人 
\\	男. 
\\	男, 人	
\\	多い	
\\	おおい	い 
\\	このりょうりは多すぎます。	
\\	「自分は正しい」とおもいこんでいる人は多い。	
\\	電車にはとても多くの乗客が乗っていたので、眠ることはできなかったが、大声で歌うことはできた。	
\\	い 
\\	""おお!!!
\\	多	
\\	多分	
\\	たぶん	
\\	かさは、多分外にあります。	
\\	多分、レイヤーさんなのかもしれないよ。	
\\	明日の夜は多分雪になるから、雪だるまを作ろう。	
\\	多, 分	
\\	肉	
\\	にく	
\\	水よう日は、肉がやすいです。	
\\	牛肉とぶた肉ととり肉の中ではどれが一ばんすきですか。	
\\	お肉は、二枚の食パンに挟まれる、もっとも重要な具材です。	
\\	肉	
\\	会	
\\	かい	
\\	きんようびの夜に、女子会をします。	
\\	そのメンバーの会はとてもおも白そうですね。	
\\	この会、また開こうよ!	
\\	会	
\\	お金	
\\	おかね	
\\	この写しんは、お金よりも大切です。	
\\	お金をなめているとバチが当たるよ!	
\\	この機械は、お金を取りあげるけどなんの見返りもくれないから、気をつけた方がいいよ。	
\\	お 
\\	金 
\\	(かね). 
\\	金	
\\	月光	
\\	げっこう	
\\	そのよるは、月光がとてもきれいでした。	
\\	一りんの白い花に、月光がさえざえとふりそそいでいた。	
\\	月光に照らされた彼女は、とても美しかった。	
\\	月 
\\	げつ 
\\	げっ. 
\\	つ 
\\	月, 光	
\\	来月	
\\	らいげつ	
\\	来月たん生日会をするので、かなえさんも来てください。	
\\	花火大会の日は来月の何日でしたっけ?	
\\	今日海に落とした眼鏡を探すために、来月スキューバダイビングを始めることにしたよ。	
\\	来, 月	
\\	来年	
\\	らいねん	
\\	来年はアメリカにかえるつもりです。	
\\	今年も来年もさ来年もわたしはニートをつづけるつもりです。	
\\	来年の今頃に来てくれよ。ハウストレーラー用のキャンプ場で、一番でかいトレーラーが俺のだよ。	
\\	来, 年	
\\	北西	
\\	ほくせい	
\\	の 
\\	ひこうきは、北西にむかっています。	
\\	ポートランドから車で一じかん半ほど北西の方にいくと、シーサイドというまちにつきます。	
\\	ここから北西の方角のどこかに、パンストを落とした。	
\\	北, 西	
\\	兄	
\\	あに	
\\	山下さんはわたしの兄ににていますね。	
\\	一ヶ月まえに兄がくろラブをつれていえにかえってきました。	
\\	兄は、とうふにコーヒーをかけてたべるのがすきだ。	
\\	(あに), 
\\	兄	
\\	自分	
\\	じぶん	
\\	の 
\\	ちゃんと自分でしゅくだいをしました。	
\\	自分に自しんがないとえらべない色だよね。	
\\	もちろん私は自分と話しますよ。だって、時には専門家の意見が必要ですもの。	
\\	自, 分	
\\	一回	
\\	いっかい	
\\	もう一回、言ってください。	
\\	平日は一日に一回シャワーをあびますが、休日はぜんぜんあびません。	
\\	もう一回欲しいのかい、ハニー?	
\\	一 
\\	いっ.	一, 回	
\\	空車	
\\	くうしゃ	
\\	タクシーのひょうじが「空車」となっているときは、おきゃくさんはだれものっていません。	
\\	この辺りのコインパーキングがいつも空車なのがふしぎでしょうがない。	
\\	空車のタクシーが全然通らない。	
\\	空, 車	
\\	本来	
\\	ほんらい	
\\	の 
\\	このサイトは本来かめのせいそくちとしてつかわれていた。	
\\	その言ばの本来のいみをしっていますか。	
\\	本来漢字は悪いものではない。	
\\	本, 来	
\\	体力	
\\	たいりょく	
\\	わたしは体力があまりありません。	
\\	コウイチには、うで立てふせを八十回する体力はありません。	
\\	その熱が彼の体力を弱めたが、まだ自分で水を飲む体力は残っていた。	
\\	体, 力	
\\	公社	
\\	こうしゃ	
\\	たしか、かれは公社ではたらいています。	
\\	このりょきゃくてつどうは、みなみアフリカりょきゃくてつどう公社がうんえいしています。	
\\	4月1日に国営公社である日本郵政公社が業務を開始したが、エイプリルフールのジョークではなかった。	
\\	公, 社	
\\	雨	
\\	あめ	
\\	雨はまだふっていません。	
\\	一年中雨がふるくにの王子として生まれたことをくいています。	
\\	私は雨の臭いが好きで、あなたはそんな臭いがする。だからあなたのことが大好きなの。	
\\	雨	
\\	本当	
\\	ほんとう	
\\	な 
\\	それは本当ですか?	
\\	さいきんの大学生はスマホでレポートをかくってきいたけど、本当かなあ。	
\\	本当に下手なのです。	
\\	本, 当	
\\	近年	
\\	きんねん	
\\	近年、日本しゅがまたブームになっています。	
\\	近年のディズニーランドは、あまりこんざつしていないというのはかんぜんなるデマです。	
\\	豆腐を食べている河豚の存在が、近年徐々に明らかとなってきた。	
\\	近, 年	
\\	米	
\\	こめ	
\\	米をかうのをわすれてしまいました。	
\\	1おく円のおもさはやく10キロときいたので、10キロのお米のふくろをだきしめながら、1おく円をもっているところをそうぞうしてみた。	
\\	私の一番好きな朝ごはんは、生米と味噌汁です。	
\\	ごはん. 
\\	(こめ) 
\\	米	
\\	同じ	
\\	おなじ	
\\	この文と大体同じいみの文があります。	
\\	今回のしけんは、ぜん回とほぼ同じ内ようでした。	
\\	あの店と同じ味のラーメンがどうしても作りたいんだ。	
\\	(おな). 
\\	同	
\\	同日	
\\	どうじつ	
\\	それらは、同日にはつばいされました。	
\\	すきなバンドのライブがちがうはこで同日にかぶっていて、どれにいこうかちょうぜつまよっています。	
\\	父の死と同日に、アポロ11号が月面着陸に成功した。	
\\	日 
\\	じつ, 
\\	同, 日	
\\	入学	
\\	にゅうがく	
\\	する 
\\	これは、入学のおいわいのプレゼントです。	
\\	父が入学金をはらうのをわすれてしまったんです。	
\\	トムはオックスフォード大学に入学できたと思っていたんだけど、どうもその合格証明書はトムの両親がエイプリルフールのジョークとして作ったものだったようだぜ。	
\\	入, 学	
\\	学年	
\\	がくねん	
\\	かれは、兄と同じ学年です。	
\\	トーフグには、小学こうてい学年のファンもいれば、100才をこえるファンもいる。	
\\	あの給食の後、三年生の学年の三分の一が吐いた。	
\\	学, 年	
\\	皮	
\\	かわ	
\\	にわに、みかんの皮がおちていました。	
\\	らく天で、天ねんの牛の皮でできた小さいラグをかった。	
\\	俺のジャケットは、100
\\	竜の皮でできているんだ。	
\\	皮	
\\	空	
\\	そら	
\\	空にきれいなにじがかかっている。	
\\	女心とあきの空といいますが、男心だってかわりやすいんですよ。	
\\	今夜映画
\\	の有名なシーンがちゃんと見れるくらい、空が晴れ渡っているといいんだが。	
\\	(そら). 
\\	空	
\\	水色	
\\	みずいろ	
\\	の 
\\	あの水色のバッグがほしいんですが。	
\\	トーフグの
\\	シャツは、さわやかな水色です。	
\\	どうして今日の空は水色なんだろう?	
\\	水 
\\	色. 
\\	水 
\\	色 
\\	水, 色	
\\	近日	
\\	きんじつ	
\\	このゲームは、近日はつばいよていです。	
\\	田中さんがたのしみにしていたえいがが、ついに近日こうかいします。	
\\	近日中にお会いして、ネットで見たことを一緒に語り合いましょう。	
\\	日 
\\	じつ.	近, 日	
\\	音	
\\	おと	
\\	二かいから、大きな音がしました。	
\\	となりのいえから、すごく上手なリコーダーの音が聞こえてきます。	
\\	金魚が泣いている音を聞いたことはある?	
\\	(おと). 
\\	おと 
\\	音	
\\	四角	
\\	しかく	
\\	な 
\\	の 
\\	じゃがいもを、四角に切ってください。	
\\	ツイッターのアイコンの形を今の丸から四角にもどしてほしい。	
\\	どうしてほとんどの窓は真四角か長四角なんだろう。イタリア型の窓とかも作ったらいいのに。	
\\	四, 角	
\\	学生	
\\	がくせい	
\\	学生は今日、うちでどこをべんきょうしますか。	
\\	学生わり引ってありますか?	
\\	「我々はみな人生の学生だ。」と言いながら、父はおならをした。	
\\	学, 生	
\\	日光	
\\	にっこう	
\\	日光をあびすぎるのは体によくない。	
\\	できるだけちょくしゃ日光にさらされないよう気をつけています。	
\\	何故か日光を見るといつもくしゃみが出ます。	
\\	にち 
\\	にっ, 
\\	日, 光	
\\	金玉	
\\	きんたま	
\\	金玉がヒリヒリするんです。	
\\	おれの金玉はぜったい他の男のばいはたらいてるとおもうぜ。	
\\	金玉を金色にする手術は、六ヶ月先まで予約がいっぱいです。	
\\	金 
\\	玉. 
\\	金, 玉	
\\	今回	
\\	こんかい	
\\	今回は、やめておきます。	
\\	今回のマラソン大会では一いでした。	
\\	前回お前が運転した時は、岸にぶつかりそうになったじゃん。だから今回は俺が釣り船を運転するよ。	
\\	今, 回	
\\	何千	
\\	なんぜん	
\\	何千回も、れんしゅうしました。	
\\	とつぜん、何千ものベーコンがそらからふってきた。	
\\	巨人攻撃の被害者は、何千人にものぼった。	
\\	千 
\\	ぜん 
\\	せん 
\\	何, 千	
\\	〜回	
\\	かい	
\\	日本に一回行ったことがあります。	
\\	アメリカから日本までおよいでわたるけいかくを立てたが、もう八回ほどと中で中止になっている。	
\\	何回砂を食べたか自分でも数えきれないよ。	
\\	1回, 2回, 3回, 
\\	100回 
\\	回	
\\	当たる	
\\	あたる	
\\	人に石が当たったらどうするんだ。	
\\	まんいんでん車で、手がとなりの女のしりに当たってしまい、ちかんにまちがえられたらどうしようと不あんでいっぱいになった。	
\\	自分の予想が当たって、嬉しいです。	
\\	う 
\\	あ 
\\	(あ) 
\\	当	
\\	行う	
\\	おこなう	
\\	よていどおり、日よう日にうんどう会を行います。	
\\	来しゅうの月よう日は、学生が、えきでうちわのむりょうはいふを行うよていです。	
\\	東京病院に豚の心臓を運ぶ仕事を任されましたが、もし私がその仕事を行わなければ、誰かが料理を完成させることができなくなってしまいます。	
\\	(う) 
\\	おこな 
\\	(おこな) 
\\	行	
\\	来る	
\\	くる	
\\	アメリカから来ました。	
\\	来る、きっと来る。	
\\	二度と戻って来るんじゃねーぞ!	
\\	う 
\\	(く) 
\\	来	
\\	行く	
\\	いく	
\\	ごぜん中は、かいものに行ったり、そうじをしたりします。	
\\	ようじができて、行けなくなってしまいました。	
\\	タコベルが無いので火星には行きたくない。	
\\	う 
\\	い 
\\	い 
\\	(い) 
\\	行	
\\	生える	
\\	はえる	
\\	あごひげがよく生えます。	
\\	「くさ生える」とは、おも白いとおもったときにつかう日本のネットスラングで、
\\	をくさに見立てたことからつかわれはじめました。	
\\	こんなところにキノコが生えるのかね。	
\\	う 
\\	生きる 
\\	(は) 
\\	生	
\\	学ぶ	
\\	まなぶ	
\\	わたしは、どく学で日本ごを学んでいます。	
\\	日本ごをべんきょうするのはやめて、トーフグといっしょにへんがおとへんなポーズを学びませんか?	
\\	学ぶことは大切だけど、おそらくもっと大切なのはどのように学ぶかということじゃないだろうか。	
\\	う 
\\	習う. 
\\	学ぶ 
\\	習う, 
\\	まな 
\\	(まな). 
\\	学	
\\	作る	
\\	つくる	
\\	今日は、ばんごはんにやきそばを作ります。	
\\	フェイスブックのアカウントをふくすう作る方ほうをおしえてください。	
\\	テスラ・コイルをどうやって作るのかを知っていたら良かったんだけどな。	
\\	う 
\\	つく 
\\	(つく) 
\\	作	
\\	会う	
\\	あう	
\\	日よう日に会う。	
\\	ポートランドくうこうでばったりコウイチに会いました。	
\\	明日の午後、大好きな俳優に会えるんです!	
\\	う 
\\	あ 
\\	う 
\\	(あう) 
\\	会	
\\	交ぜる	
\\	まぜる	
\\	トランプをよく交ぜてください。	
\\	どうしてたくさんの男たちの中に女を一人交ぜたんだ?	
\\	豆腐と河豚を交ぜるのは危険です。	
\\	う 
\\	交じる 
\\	交わる 
\\	交じる, 
\\	ま 
\\	(ま).	交	
\\	回る	
\\	まわる	
\\	今日は、おきゃくさんと18ホールを回りました。	
\\	百円玉はまだつくえの上で回っています。	
\\	惑星が太陽の周りを回るように、その子猫たちが親猫の周りを回っている。	
\\	う 
\\	(まわ)
\\	回	
\\	走る	
\\	はしる	
\\	えいがにおくれそうだったので、走りました。	
\\	コウイチはカメよりもはやく走れる。	
\\	人生走り続けるより、侍に首をはねられる方がずっとましだという結論に達しました。	
\\	う 
\\	はし 
\\	(はし) 
\\	走	
\\	言う	
\\	いう	
\\	ゆっくり言ってください。	
\\	いつもありがとうってえいごで何て言うの?	
\\	ジェイクに、足を踏んでごめんって言うつもりなの?	
\\	う 
\\	""い
\\	言	
\\	考える	
\\	かんがえる	
\\	少し考えさせてください。	
\\	考えるな!かんじろ!	
\\	私の娘は、もしタイタニック号が沈まなければ、みんなその代わりにディズニーランドに行くことができて、もっといい映画になったのにと考える。	
\\	う 
\\	思う.
\\	(かんが). 
\\	考	
\\	行き	
\\	ゆき, いき	
\\	しん大さか行きのしんかんせんのきっぷをください。	
\\	行きもかえりもあるくつもりです。	
\\	この電車は、
\\	オフィス行きです。	
\\	き 
\\	(き) 
\\	いき, 
\\	ゆき 
\\	ゆ 
\\	(ゆ) 
\\	行	
\\	走行	
\\	そうこう	
\\	する 
\\	あなたの車の走行マイルをおしえてください。	
\\	バスの走行中にせきを立つのはきけんです。	
\\	その電車は、ストライキが終わり次第、ホグワーツ魔法魔術学校への走行を再開する。	
\\	走, 行	
\\	三角	
\\	さんかく	
\\	三角のクラッカーをかってきてください。	
\\	今、三角かんけいになっちゃってるんだよね。	
\\	カナダ人はピカソにかなり刺激を受けたに違いない。四角い箱に、丸いピザ、三角のピザスライスがあちこちにある。	
\\	三, 角	
\\	大学	
\\	だいがく	
\\	わたしの大学には、中ごく人がたくさんいます。	
\\	アメリカの大学と日本の大学の主なちがいはなんですか。	
\\	大学で解剖学を勉強したが、教科書は一度も開かなかった。	
\\	大, 学	
\\	今まで	
\\	いままで	
\\	今まで何をしていたんですか。	
\\	コウイチが今まで付きあったかの女は百万人に上るが、ようやく一人の女の人におちつきつつある。	
\\	お前が今までの中では一番の彼女だよ。	
\\	今 
\\	まで 
\\	今. 
\\	今	
\\	谷	
\\	たに	
\\	あそこの谷で、くまを見ました。	
\\	はこねには大わく谷というむかし「じごく谷」といわれた谷があって、かんこうの名しょになっています。	
\\	死の谷で自殺した男は、ダース・ベイダーのコスプレをしていた。	
\\	谷	
\\	色々	
\\	いろいろ	
\\	な 
\\	の 
\\	色々な小せつをよみました。	
\\	色々考えたけっか、日よう日にたこやきパーティーをすることにしました。	
\\	私のベッドの下に住んでいるトロールには色々な種類がいます。	
\\	々 
\\	色, 々	
\\	声	
\\	こえ	
\\	はし本さんの声は、とてもひくいです。	
\\	友人の声とその子どもの声がそっくりだ。	
\\	あんたの声、ひどいな!	
\\	声	
\\	西	
\\	にし	
\\	びょういんの西のたてものがわたしのアパートです。	
\\	ダリンは、西にむかっています。	
\\	その頃、西ポートランドでは、鰐蟹がその道を渡っていた。	
\\	(にし) 
\\	西	
\\	何	
\\	なに, なん	
\\	の 
\\	何と言いますか。	
\\	あなた、何さまのつもりですか。	
\\	僕達がタイのホテルで飲んだ飲み物の名前は何?	
\\	何	
\\	何月	
\\	なんがつ	
\\	学こうは何月にはじまりますか。	
\\	それって何年何月何日何びょう、ちきゅうが何回回ったとき?	
\\	田中さんは何月生まれですか?	
\\	10月, 
\\	何 
\\	月 
\\	何, 月	
\\	何年	
\\	なんねん	
\\	何年生まれですか。	
\\	おたくのお子さんは、何年に大学に入りましたか。	
\\	アメリカの南北戦争は、何年間続いたんですか?	
\\	何, 年	
\\	麦	
\\	むぎ	
\\	わたしは、麦アレルギーがあります。	
\\	大麦と小麦のちがいは何ですか。	
\\	「スクワット、スクワット、スクワット、麦とスクワット
\\	屈伸をしながら彼は歌った。	
\\	麦	
\\	一生	
\\	いっしょう	
\\	の 
\\	このゆびわは、一生、大切にします。	
\\	わたしの父は、しぬまで一生どくしんでした。	
\\	一生かけて償います。	
\\	生 
\\	(しょう). 
\\	一, 生	
\\	社内	
\\	しゃない	
\\	の 
\\	つぎの日よう日に、社内のバスケットボール大会があります。	
\\	花さんは、社内けっこんをしてたいしょくしました。	
\\	最近、社内で変なことが流行っている。従業員が、会社の冷蔵庫の中の食べ物に名前をつけるのだ。今日は、俺は、「鈴木」と名付けられたオニギリを食べた。	
\\	社, 内	
\\	毛糸	
\\	けいと	
\\	みどり色の毛糸はありますか。	
\\	今日はとてもさむいので、友人があんでくれた毛糸のセーターをきています。	
\\	「毛糸の玉をほどいてください。お願いします。お願いします。」と、その幽霊は何度も私に頼んできた。	
\\	毛 
\\	糸 
\\	毛, 糸	
\\	弟	
\\	おとうと	
\\	わたしの弟は、テレビを見ない。	
\\	その青いかみの青年はぼくの弟です。	
\\	弟がビリヤードのボールを頭に投げつけてきたので、昨夜僕は病院に行くはめになった。	
\\	(おとうと) 
\\	弟	
\\	青年	
\\	せいねん	
\\	とてもさわやかな青年でした。	
\\	その青年は、ぼう走ぞくの一いんです。	
\\	どうして日本の青年は、下心があっていやらしい笑みを浮かべていることを隠そうとするのか?	
\\	青, 年	
\\	後で	
\\	あとで	
\\	後でいっしょにお茶でもしませんか。	
\\	後で、「言わんこっちゃない」って言わせないでよ。	
\\	その映画を見終わった後で、もう一度それを見るために違う映画館へ向かった。	
\\	(あと) 
\\	後ろ 
\\	で 
\\	(で) 
\\	後	
\\	全て	
\\	すべて	
\\	の 
\\	全て上手く行きました。	
\\	申しわけございませんが、げんざい全てのおせきがうまっているじょうきょうでして...。	
\\	僕が先だよ。ウィキペディアが僕の宿題を全て真似したんだ。	
\\	すべ 
\\	(すべ) 
\\	全	
\\	全日本	
\\	ぜんにほん, ぜんにっぽん	
\\	これは、全日本バレーボール大会のしあいの写しんです。	
\\	コウイチは、けん道の全日本せん手けんのしあいを見るために、日本に行っています。	
\\	全日本が生クリームに包まれている。	
\\	日本 
\\	日本. 
\\	全 
\\	全, 日, 本	
\\	午前	
\\	ごぜん	
\\	このデパートは、午前十じにひらきます。	
\\	わたしのむす子は、まだ午前中にあさねをするひつようがあります。	
\\	今日は午前9時にテレパシーで投稿する予定にしているので、もし君がその時何か面白いことを考えたら、それは私の仕業だから。	
\\	午, 前	
\\	米国	
\\	べいこく	
\\	の 
\\	米国の大とうりょうの名まえを知っていますか。	
\\	米国には元々何もきたいなんてしてなかったよ。	
\\	多くの人は、単にマクドナルドのスーパーサイズコンボのために米国に移住する。	
\\	アメリカ, 
\\	米, 国	
\\	足首	
\\	あしくび	
\\	足首をひねってしまいました。	
\\	右の足首がすごくいたいんです!	
\\	宙に舞う羽から逃げる途中で、足首を捻挫した。	
\\	足 
\\	首 
\\	足, 首	
\\	年次	
\\	ねんじ	
\\	今年の年次けいかくを立てましょう。	
\\	マリ子は年次有きゅう休か、何日もらえるの?	
\\	生徒たちの日本語がどれだけ上達したかに関する年次報告書を作成する。	
\\	年, 次	
\\	後ろ	
\\	うしろ	
\\	あなたの後ろのドアをしめてください。	
\\	チャイルドシートは、かならず後ろのざせきにせっちしてください。	
\\	振り返って後ろを見た後、さらにもう一度振り返ったら、僕のことが見えるよ。	
\\	後で 
\\	ろ 
\\	(ろ) 
\\	うし, 
\\	後	
\\	出血	
\\	しゅっけつ	
\\	する 
\\	なかなか出血が止まりません。	
\\	はぜっさん出血大サービス中です。	
\\	指を抜いていれば、鼻からの出血は止まると思います。	
\\	しゅつ 
\\	出 
\\	しゅっ. 
\\	つ 
\\	出, 血	
\\	血	
\\	ち	
\\	シャツに血がついていますよ。	
\\	うんちをするときにおしりから血が出るのはなぜなんだ!	
\\	血が滴るぐらいのステーキが好きです。	
\\	血	
\\	両方	
\\	りょうほう	
\\	の 
\\	姉も妹も、両方とも大さかにすんでいます。	
\\	じかんはあるんだから、両方やってみたら?	
\\	ダメだよ!両方とも僕のだよ。	
\\	両方, 
\\	両, 方	
\\	両日	
\\	りょうじつ	
\\	両日とも、会ぎが入っています。	
\\	とうふまつりは、両日ともせい天にめぐまれた。	
\\	の使い方についての講義に両日とも参加したが、まだ使い方が分からない。	
\\	日 
\\	じつ 
\\	にち.	両, 日	
\\	両手	
\\	りょうて	
\\	の 
\\	両手を上げて、バンザイをしました。	
\\	ビエトは両手に花ですね。	
\\	バランスの良い食事とは、両手にチョコレートクッキーを持つことです。	
\\	両, 手	
\\	明るい	
\\	あかるい	い 
\\	このへやは明るい。	
\\	トーフグの社いんは、みんな明るい。	
\\	未来はまだまだ明るいよ、おじいちゃん。	
\\	い 
\\	あか 
\\	あか 
\\	あか). 
\\	明	
\\	茶色	
\\	ちゃいろ	
\\	の 
\\	に入るばんごうを、茶色にぬります。	
\\	明るい茶色にカラーリングしてください。	
\\	私の白い
\\	シャツ全ての脇のところに、茶色のシミがある。	
\\	茶, 色	
\\	仕方がない	
\\	しかたがない	
\\	かれがおこるのは、仕方がないよ。	
\\	泣いてばかりいても仕方がないよ。	
\\	お前は本当に仕方がないやつだ!	
\\	仕方がない, 
\\	仕, 方	
\\	安い	
\\	やすい	い 
\\	そして、でん車は安い方がいいです。	
\\	タイのぶっかはアメリカよりも安いです。	
\\	アパートの屋根から投げて粉々になるのを見てみたかったので、昨日すごく安いラジオを買った。	
\\	い 
\\	100円 
\\	安い 
\\	(やす) 
\\	安	
\\	中国	
\\	ちゅうごく	
\\	来月は、出ちょうで中国に行きます。	
\\	冬休みに、中国でりゅう星ぐんを見ました。	
\\	どうして日本人は中国を支那と呼ぶことをやめて中国と呼び始めたんだろう。	
\\	中, 国	
\\	社長	
\\	しゃちょう	
\\	社長は会ぎ中です。	
\\	あの人気サービスのねだんを下げるなんて、トーフグの社長はどうかしてるよ。	
\\	全社員は社長のことを嫌っているが、私は単に彼は少し意地悪なだけだと思っている。	
\\	社長 
\\	社, 長	
\\	活用	
\\	かつよう	
\\	する 
\\	今日は、どうしの活用をべんきょうします。	
\\	ちょこっとあいたじかんを活用して、ワニカニで日本ごをべんきょうします。	
\\	このデータをうまく活用して、何かできないかな。	
\\	活, 用	
\\	国	
\\	くに	
\\	お国はどちらですか。	
\\	この国は安全だと思っていたのに、がっかりです。	
\\	私の国には、猫が焼くアップルパイのお店がある。	
\\	(くに), 
\\	国	
\\	文化	
\\	ぶんか	
\\	日本の文化にきょうみがあります。	
\\	この国の文化にはなれましたか?	
\\	コスプレは日本の文化の一つですか?	
\\	文, 化	
\\	死体	
\\	したい	
\\	の 
\\	だれが死体を見つけましたか。	
\\	だれが死体いきのようぎでたいほされたって?	
\\	私が初めて死体を見たのは、六歳の時です。	
\\	死, 体	
\\	死	
\\	し	
\\	その死は早すぎました。	
\\	人は死をのがれることはできない…コウイチい外は。	
\\	愛犬の死に打ちひしがれています。	
\\	死	
\\	方言	
\\	ほうげん	
\\	の 
\\	方言のじしょを作っています。	
\\	ああっもうっ!かれしの方言がうつっちゃってるずら。	
\\	私の国の東側出身の人達は変わった方言で喋るので、彼らが何を言っているのか全く分かりません。	
\\	方 
\\	方, 言	
\\	方向	
\\	ほうこう	
\\	わたしは、方向が分からなくなりました。	
\\	めっちゃ方向音ちなんですよ。	
\\	こっちの方向であってるの?チャッキー・チーズは反対方向だと思ってたんだけど。	
\\	方, 向	
\\	早く	
\\	はやく	
\\	きのうはどうして早く学こうに行きましたか。	
\\	ゆっくり休んで早くよくなってください。	
\\	早く食べなさい。	
\\	早い, 
\\	早	
\\	地	
\\	ち	
\\	一まいのはっぱが、ひらりと地におちた。	
\\	コウイチはむかしはおおものだったけど、イノシシとのたたかいにまけてから、ひょうばんが地におちたよね。	
\\	しかし、コウイチ氏は意外と地に足をしっかりつけ続けています。	
\\	地	
\\	東	
\\	ひがし	
\\	えきの東がわにあるきっ茶店で会いましょう。	
\\	どっちの方角が東?	
\\	空港の東側に住んでいます。	
\\	(ひがし) 
\\	東	
\\	東方	
\\	とうほう	
\\	東方の空が明るくなりだした。	
\\	あの外国人は、大さかえきの東方やく五キロのところに住んでいる。	
\\	日本はイギリスの東方にある。	
\\	東, 方	
\\	活気	
\\	かっき	
\\	この町は、とても活気があります。	
\\	ニッカはいつも活気にあふれている。	
\\	市場は活気にあふれていた。	
\\	かつ 
\\	かっ. 
\\	つ, 
\\	活, 気	
\\	有名	
\\	ゆうめい	な 
\\	いえの近くに、有名なすしやがあります。	
\\	このきじはトーフグを一やくちょう有名にした。	
\\	俺は有名だけど、お前はどうよ?	
\\	有, 名	
\\	名曲	
\\	めいきょく	
\\	クリスマスソングの名曲といえば、何だと思いますか。	
\\	この曲は名曲だと言われているけど、わたしには他の曲と大してかわりない。	
\\	ベートーベンの「スモークサーモン」は名曲だ。	
\\	名, 曲	
\\	早口	
\\	はやくち, はやぐち	
\\	の 
\\	わたしの兄は、とても早口です。	
\\	お前のことが一ばん大すきだよって、何でそんなに早口で言ったの?	
\\	こんなに早口の赤ん坊なんて見たことないよ。	
\\	はやくち. 
\\	はやぐち 
\\	早, 口	
\\	外国	
\\	がいこく	
\\	の 
\\	わたしたちはまだ外国に行ったことがありません。	
\\	外国へでんわをするのはこれがはじめてです。	
\\	もし外国にいるなら、有名人になっても構わない。	
\\	外, 国	
\\	外国人	
\\	がいこくじん	
\\	今日、外国人の友だちと、かいものに行きました。	
\\	その外国人の目は血走っていた。	
\\	いいえ、私は人種差別主義者ではありません。世界中の全ての外国人に会って、それぞれ個人的に嫌いになっただけです。	
\\	外人 
\\	外, 国, 人	
\\	四十二	
\\	よんじゅうに	
\\	とうなんアジアを、四十二日かんりょこうしました。	
\\	フランス人グルメは五月だけでさらうどんを四十二さらたべた。	
\\	そぼがトイレットペーパーを四十二ロールもかってきた。	
\\	四 
\\	四, 十, 二	
\\	前	
\\	まえ	
\\	前は花やで、花やのとなりはさかなやです。	
\\	毎あさ、会社に行く前に、スクワットを五百回します。	
\\	赤信号の前にいる時、「もうすぐ青に変わるぞ!」と思うとかなり興奮する。	
\\	(まえ) 
\\	前	
\\	全力	
\\	ぜんりょく	
\\	の 
\\	えきまで、全力で走りました。	
\\	その中国人の女は、王子が自分とけっこんしてくれるよう全力でせっとくしたが、しっぱいにおわった。	
\\	そのゾンビは、死から蘇ろうと全力を尽くしていた。	
\\	全, 力	
\\	正直	
\\	しょうじき	
\\	な 
\\	川田さんは、とても正直な人です。	
\\	かれの正直なところがすきなんです。	
\\	僕は水中で息をしている火星からきた王子だし、他の人の模範となる必要があるので、いつも正直でいようと心がけています。	
\\	正, 直	
\\	本州	
\\	ほんしゅう	
\\	この後、本州に行くつもりです。	
\\	本州とは、いったいぜんたい日本のどこにあるんですか?	
\\	本州がどこか分からなければ、パソコンでしらべたらどうですか?	
\\	本, 州	
\\	私	
\\	わたし, わたくし	
\\	の 
\\	メーリングリストから私を外してください。	
\\	私がえいぎょう部のせきにん者です。	
\\	昨日の夜、バーデビューしちゃった。私もちょっとずつ大人の階段登ってるって感じ?	
\\	わたし. 
\\	し?
\\	し.
\\	私	
\\	国王	
\\	こくおう	
\\	の 
\\	あの国王は、とてもよくばりです。	
\\	わたしは、国王のことをじょじょにりかいしてきました。	
\\	18歳という若さで国王になった。	
\\	国, 王	
\\	天国	
\\	てんごく	
\\	の 
\\	ポチはきっと天国にいるんだよ。	
\\	天国がこんなとこだったなんて、本当にがっかりです。	
\\	タバコを天国で吸いましょう。	
\\	国 
\\	ごく, 
\\	天, 国	
\\	九州	
\\	きゅうしゅう	
\\	九州のりょうりはおいしいです。	
\\	九州までは、フェリーでかた道一じかんです。	
\\	九州には、綺麗な女性が多い。	
\\	九, 州	
\\	土地	
\\	とち	
\\	ここは、あなたの土地ではありません。	
\\	コウイチはマインクラフトに広大な土地をもっている。	
\\	もし自分の土地があれば、ようやくそこに自分のベッドルームを手に入れることができるぜ!	
\\	土 
\\	と 
\\	土, 地	
\\	生活	
\\	せいかつ	
\\	する 
\\	アメリカと日本の生活のちがいをおしえてください。	
\\	あなたとの生活をもう一どやり直したいんです。	
\\	毎日の生活でアニメを見ながらマシュマロを食べることができたら、とても幸せです。	
\\	生活 
\\	生, 活	
\\	長い	
\\	ながい	い 
\\	このえいがは長かったです。	
\\	あそこのドアの近くに立ってるかみの長い女性が、たん当しゃの方です。	
\\	長い距離のランニングを避けるため、自分の足の骨を折ってしまおうと思っている。	
\\	い 
\\	(なが) 
\\	長	
\\	宝くじ	
\\	たからくじ	
\\	「妹が宝くじに当たったのよ。」「すごいね!で、いくら?」「一万円。」	
\\	きぼうてきかんそくでは、来年あたりに宝くじに当たるはずです。	
\\	父親の秘書が宝くじに当たったんだけど、それをフェイスブックに書いた時に秘書を秘密と書き間違えてしまい、みんなが勘違いしてたくさんおめでとうコメントを残してくれました。	
\\	宝 
\\	くじ 
\\	宝 
\\	たから.	宝	
\\	小文字	
\\	こもじ	
\\	どうやって小文字と大文字を切りかえますか。	
\\	パスワードには、かならず大文字、小文字、すう字を一つい上つかってください。	
\\	君の
\\	メールアドレスって、全部小文字なの?本当に?ちょっとおかしいんじゃないの?	
\\	文字 
\\	小 
\\	こ. 
\\	文字 
\\	こもじ.
\\	小, 文, 字	
\\	地図	
\\	ちず	
\\	スマートフォンで地図を出しました。	
\\	地図をかた手に東京を歩きました。	
\\	先生、もし世界地図を忘れたのなら、私の額にタトゥーの世界地図がありますよ。	
\\	地, 図	
\\	草地	
\\	くさち	
\\	一とうの牛が草地をのんびり歩いていました。	
\\	アイルランドには、ぼく草地や草地が多い。	
\\	この金は、あそこの草地に隠そう。	
\\	草, 地	
\\	水星	
\\	すいせい	
\\	の 
\\	水星には大気がありません。	
\\	こんなはずじゃなかったんですが、もう十年も水星でホームレスをしています。	
\\	水星は、日の出直前、または日没直後しか見ることができない。	
\\	水, 星	
\\	中東	
\\	ちゅうとう	
\\	な 
\\	けさ、テレビで中東のニュースを見ました。	
\\	きんきゅうじたいで中東に行かなくてはいけなかったんだから、けっこんしきに来れなかったのは仕方がないよ。	
\\	中東で一番楽しまれているゲームは何ですか?	
\\	中, 東	
\\	死亡	
\\	しぼう	
\\	の 
\\	やさいをたくさん食べると、死亡リスクがへるそうです。	
\\	山田さんの弟さんの死亡のげんいんはなんですか?	
\\	チーズバーガーの発明により、全先進国の死亡率が三倍に跳ね上がった。	
\\	死, 亡	
\\	国宝	
\\	こくほう	
\\	私のきおくが正しければ、これは国宝ではありません。	
\\	「どうしてこのそうじきは国宝なんですか?」「ニコラス・ケイジがつかったことがあるからだよ。」	
\\	トーフグオフィスに国宝が隠されている。	
\\	国, 宝	
\\	全国	
\\	ぜんこく	
\\	の 
\\	全国でインフルエンザがはやっています。	
\\	リンゴのしゅうかくりょうが全国で一ばん多いけんはどこですか?	
\\	彼は全国規模のサイバーいじめのスキャンダルに関与していた。	
\\	全, 国	
\\	首	
\\	くび	
\\	パソコンのしすぎで、首がいたいです。	
\\	首をねちがえました。	
\\	腺ペストって、首から胸が生えてくる病気のことですか?	
\\	首	
\\	海王星	
\\	かいおうせい	
\\	海王星は、太ようから八ばん目のわく星です。	
\\	海王星に行くことをずっとゆめ見てきたんです。	
\\	海王星は1846年9月23日に発見されました。	
\\	海, 王, 星	
\\	次	
\\	つぎ	
\\	の 
\\	次のでん車にのりましょう。	
\\	次のよていは、決まり次だいお知らせします。	
\\	このバスツアーの次の目的地は、マイケル・ジャクソンの家です。	
\\	(つぎ) 
\\	次	
\\	次回	
\\	じかい	
\\	次回の作ひんも、たのしみにしています。	
\\	次回なんてないんだよ。人生は一どきりなんだから。	
\\	兄貴、心配するなよ。次回は絶対あいつらを仕留めてやろうぜ。	
\\	次, 回	
\\	夜	
\\	よる	
\\	きのうの夜、パーティーに行きました。	
\\	きのうの夜は、こうふんしてほとんどねむれませんでした。	
\\	夜には、モンスターたちが床から這い上がり、私のベッドの下でお茶会を開くんだよ。	
\\	(よる) 
\\	夜	
\\	直行	
\\	ちょっこう	
\\	する 
\\	の 
\\	このひ行きは、東京に直行します。	
\\	今日は、日本との会ぎの後、トーフグとのアポに直行します。	
\\	ごめんなさい、今は話せません。トイレへ直行しているところなのです。	
\\	ちょく 
\\	直 
\\	っ, 
\\	直, 行	
\\	東北	
\\	とうほく	
\\	東北で、ボランティアをしていました。	
\\	東北で一人ぐらしをしています。	
\\	マグニチュード8.9の地震が、日本の東北地方の海岸を襲った。	
\\	東, 北	
\\	足りない	
\\	たりない	
\\	お金が十円足りないんです。	
\\	もしスペースが足りなければ、べつの用しについきして、それをてんぷしてください。	
\\	ビル・ゲイツは私の息子なのだが、それを証明するための証拠がまだ足りない。	
\\	足りる 
\\	ない). 
\\	足りる 
\\	足りる (たりる) 
\\	足	
\\	海	
\\	うみ	
\\	海の公えんで、先生が学生にはなしています。	
\\	海でおよいだあと、目がじゅう血しました。	
\\	映画「水の世界」は、実は、海の底に住んでいる超頭のいい魚によって書かれた作品だ。	
\\	(うみ). 
\\	海	
\\	年上	
\\	としうえ	
\\	の 
\\	年上の人と、デートするのははじめてです。	
\\	人生はよそく不かのうだ。お前が年上のび人とけっこんするかのうせいだってゼロパーセントではない。	
\\	お父さんが、お母さんのお母さんはピラミッドよりも年上だと言うんだよ。	
\\	年, 上	
\\	姉	
\\	あね	
\\	姉のいえに行くところです。	
\\	わたしは、姉の五ばいのりょうの本をもっています。	
\\	姉はスチュワーデスをしています。	
\\	(あね). 
\\	姉	
\\	お姉さん	
\\	おねえさん	
\\	みどりちゃんのお姉さんはめがねをかけています。	
\\	あなたのこと、てっきりビエトくんのお姉さんだと思ってたんですが、ちがったんですね。	
\\	私のお姉さんは、バイクを買うことが彼氏を見つけることに繋がると考えています。	
\\	お 
\\	(ねえ).	姉	
\\	一歩	
\\	いっぽ	
\\	ようやく、ゆめに向かって一歩ふみ出しました。	
\\	千里の道も一歩からって言うでしょ?	
\\	最初の一歩を踏み出すのに勇気がいる。	
\\	歩 
\\	歩, 
\\	いち 
\\	いっ 
\\	ほ 
\\	ぽ. 
\\	一, 歩	
\\	海外	
\\	かいがい	
\\	の 
\\	しょう来は、海外ではたらきたいです。	
\\	ほけんぎょうかいに八年つとめた後、アメリカへ海外りゅう学しました。	
\\	私はどこか海外で住みたいと思っているが、旅行は嫌いだ。	
\\	海, 外	
\\	平安	
\\	へいあん	
\\	な 
\\	平安じだいはいつですか。	
\\	平安をみだす人がいないワニカニフォーラムは、わたしの心に平安をあたえてくれる。	
\\	ブリティッシュコロンビアの古の森を歩く時、心に平安を感じる。	
\\	平, 安	
\\	私生活	
\\	しせいかつ	
\\	コウイチの私生活は、ひみつのベールにつつまれています。	
\\	生まれつき人よりめんえき力が弱いので、私生活ではいつもマスクをしています。	
\\	コウイチ、仕事で相当ストレスがたまっているみたいで、私生活が荒れてきてるんだよね。ちょっと相談相手になってあげてくれないかな。誰か聞いてくれる人が必要だと思うんだよね。	
\\	生活 
\\	私, 生, 活	
\\	午後	
\\	ごご	
\\	女の人は、あしたの午後、何をしますか。	
\\	直近の午後で空いているじかんは何じですか?	
\\	午後三時に、足の巻き爪を取り除くためのお医者さんの予約がある。	
\\	午, 後	
\\	点	
\\	てん	
\\	ねつが、三十八点九どあります。	
\\	このクローゼットにある二点五ばいのりょうのふくが他のへやにもちらばっています。	
\\	あの映画が駄作だったという点ではあなたに同意しますが、時間の無駄だったと気づいたのは映画が終わってからでした。	
\\	点	
\\	私立	
\\	しりつ	
\\	の 
\\	私立のようちえんはクリスマスの日はお休みだよ。	
\\	ここを出て、前の通りを左の方へ三つ目の交さ点まで歩くと、その私立病院に着きますよ。	
\\	コウイチがカフェでこっそり会っていた男の人は、私立探偵だったということが判明しました。	
\\	私, 立	
\\	車両	
\\	しゃりょう	
\\	ただいま、車両の点けんをしています。	
\\	日本では、車両は左がわを走ります。	
\\	この車両は女性専用車両です。	
\\	車, 両	
\\	安心	
\\	あんしん	
\\	する 
\\	な 
\\	これで安心です。	
\\	あの子は次の行どうがよそくできないから、安心できないのよね。	
\\	お父さんの癌が無くなったことを聞いて、とても安心した。	
\\	安, 心	
\\	〜室	
\\	しつ	
\\	きょう室に入ってください。	
\\	このゲーム、むずかしすぎるよ。このめんで、1569号室まで行くには、どうがんばってもじかんが足りないんだ。	
\\	保健室で眠っていました。	
\\	室	
\\	安全	
\\	あんぜん	
\\	な 
\\	ここまで来れば、安全です。	
\\	ゆっくりでかまいませんので、安全うんてんでおねがいします。	
\\	安全第一が私のモットーです。	
\\	安, 全	
\\	土星	
\\	どせい	
\\	どれが土星か、分かりますか。	
\\	土星にはどうやって行けばいいですか?	
\\	私は土星出身の宇宙人です。	
\\	土, 星	
\\	科学	
\\	かがく	
\\	かの女のせんもんは、科学です。	
\\	日本は科学ぎじゅつというめんにおいてとてもすぐれています。	
\\	我々に蚊除け剤を作らせてくれる科学は、本当に偉大だ。	
\\	科, 学	
\\	羊	
\\	ひつじ	
\\	どうぶつえんで、羊にえさをやりました。	
\\	いつ何がおこるか分からないんだから、羊がかいたいなら、かいたいと思ったときにかった方がいいよ。	
\\	毛を刈られた時、羊も恥ずかしい思いをしていると思う?	
\\	(ひつじ) 
\\	羊	
\\	四国	
\\	しこく	
\\	四国にはまだ行ったことがありません。	
\\	四国での生活にはなれましたか?	
\\	最高の饂飩の作り方を求めて、四国を旅していた。	
\\	四, 国	
\\	店	
\\	みせ	
\\	えきからその店までの地図をかいてください。	
\\	この店は年内はすでにずっとよやくでいっぱいだ。	
\\	この店のこと絶対気に入るぜ。世界一うまいミートローフがあるんだよ。	
\\	(みせ). 
\\	店	
\\	姉妹	
\\	しまい	
\\	二人の姉妹はとてもよくにています。	
\\	何か、二人って姉妹みたいだよね。	
\\	単なる女友達には無い絆が、姉妹の間にはある。	
\\	姉, 妹	
\\	世の中	
\\	よのなか	
\\	この世の中には、色んな人がいます。	
\\	世の中には、しんじつを知らない方がいいことだってあるんだよ。	
\\	我々は、警察よりも先にピザが家に到着するような素晴らしい世の中に住んでいる。	
\\	世, 中	
\\	南	
\\	みなみ	
\\	としょかんの南に、交ばんがあります。	
\\	どうして東日本りょかくてつ道と西日本りょかくてつ道はあるのに、北日本りょかくてつ道と南日本りょかくてつ道はないんだ?	
\\	南球場でブラッド・ピットを見た。	
\\	(みなみ) 
\\	南	
\\	南東	
\\	なんとう	
\\	の 
\\	車は、南東に向かってすすみました。	
\\	てっきり今日は日が南東にしずむもんだとおもっていたよ。	
\\	南東の方角に、流れ星を見た。	
\\	南, 東	
\\	星	
\\	ほし	
\\	夜には、星がとてもきれいでした。	
\\	今夜はたくさんながれ星が見れるといいんですが。	
\\	お父さん、どうして今日は星型の眼鏡をかけているの?	
\\	(ほし) 
\\	星	
\\	州	
\\	しゅう	
\\	ニューヨーク州の出しんです。	
\\	アメリカがっしゅう国には、五十の州があります。	
\\	オレゴン州ポートランド市には、
\\	の本社がある。	
\\	カリフォルニア州, ニューヨーク州, 
\\	州	
\\	代わり	
\\	かわり	
\\	の 
\\	代わりの先生をさがしています。	
\\	この世のどこにも、かれの代わりはいないんです。	
\\	日本語を教わる代わりに、英語を教えることができますよ。	
\\	代わる 
\\	う 
\\	代わる 
\\	代	
\\	目次	
\\	もくじ	
\\	目次のページがやぶれています。	
\\	先ずはじめに、目次をごせつ明いたします。	
\\	この本全部読まないとだめ?目次だけ読むってことは無理かな?	
\\	目 
\\	もく 
\\	め. 
\\	(もく) 
\\	目, 次	
\\	お茶	
\\	おちゃ	
\\	わたしはお茶をのみませんでした。	
\\	死ぬ前に、ベーコンのお茶づけが食べたい。	
\\	何故かおならがしたくなるので、もうお茶を飲むことはできない。	
\\	お 
\\	お. 
\\	茶	
\\	今年	
\\	ことし, こんねん	
\\	山下さんのかぞくは、今年の冬休みはどこに行きますか?	
\\	おきなわのせい人しきは今年も大あれだった。	
\\	ようやく今年の年を2013年と書くことを覚えたところで、年が2014年に変わった。	
\\	こん 
\\	今 
\\	こ. 
\\	年 
\\	こんねん 
\\	今, 年	
\\	三百	
\\	さんびゃく	
\\	まんがを三百さつもっています。	
\\	ひるねをしていた時に、だれかがインターホンを三百回もならした。	
\\	一年は三百六十五日です。	
\\	ひゃく 
\\	びゃく. 
\\	三 
\\	三, 百	
\\	足す	
\\	たす	
\\	三足す二は五です。	
\\	このスープ、もう少し水を足した方がいいんじゃないかな。	
\\	ここにもう少し文章を書き足すことはできますか?	
\\	足りる? 
\\	足りる. 
\\	(す) 
\\	足りる, 
\\	足	
\\	出来る	
\\	できる	
\\	やっと
\\	1にごうかくすることが出来ました。	
\\	ここまではできましたが、最後の問題が難しくて出来ませんでした。	
\\	私は内面を美しくするために化粧品を食べました。あなたにもそれが出来ますか?	
\\	(出) 
\\	(来る) 
\\	出 
\\	で 
\\	来る 
\\	きる. 
\\	来る (くる), 
\\	出, 来	
\\	直る	
\\	なおる	
\\	こしょうした車が直るのはいつですか。	
\\	それでアヤの気げんが直るなら、おれはぜんぜんかまわないよ。	
\\	農夫の壊れたトラクターは直った。	
\\	う 
\\	なお 
\\	(なお). 
\\	直す, 
\\	直	
\\	死ぬ	
\\	しぬ	
\\	死ぬかと思いました。	
\\	わたしのそ母は、がんで死にました。	
\\	私は絶対に死なないと約束するよ。	
\\	う 
\\	死	
\\	食べる	
\\	たべる	
\\	毎あさ、なっとうを食べます。	
\\	いつも、けんこうにいいものを食べるじかんがないんです。	
\\	私は、寝ながら食べるのが好きだ。夢遊病のような感じで、それは夢遊食事病と呼ばれる。	
\\	う 
\\	た 
\\	(た). 
\\	食	
\\	向く	
\\	むく	
\\	こっちを向いてください。	
\\	このへやは、南を向いているので日当たりがいい。	
\\	今度は、お父さんの方を向いて、どうして赤ちゃんの格好をしているのか聞いてみましょう。	
\\	(む).	向	
\\	直す	
\\	なおす	
\\	パソコンがこわれたんですが、直せますか。	
\\	考え直した方がいいですよ。	
\\	もし壊れていなければ、それを直そうとする必要はないよ。	
\\	う 
\\	なお 
\\	(なお). 
\\	直る, 
\\	直	
\\	首になる	
\\	くびになる	
\\	先月、会社を首になりました。	
\\	しゅう金したお金をつかいこんだら、首になるに決まってるじゃん!	
\\	マーチングバンドを職場に引き連れてきたことが理由で、私は昨日首になりました。	
\\	になる 
\\	首	
\\	有る	
\\	ある	
\\	マヨネーズなら、まだ少し有ります。	
\\	一じかんに一本しかでん車は有りません。	
\\	彼の家には金も有るが、屋根の上に銀も有る。	
\\	う 
\\	(あ).
\\	有	
\\	知る	
\\	しる	
\\	何も知らない。	
\\	ルーターのさいきどうの仕方を知っていますか?	
\\	あなたも知っているように、コウイチはロボットです。	
\\	う 
\\	(し). 
\\	知	
\\	東京	
\\	とうきょう	
\\	東京えきから中川えきまででんしゃで行って、中川えきからみかん山までバスで行きます。	
\\	今から東京に行かなくてはいけないので、そろそろお会けいおねがいします。	
\\	東京についてすぐに、マクドナルドに行った。	
\\	京都 
\\	東, 京	
\\	今夜	
\\	こんや	
\\	今夜、いっしょにのみに行きませんか。	
\\	明日しけんがあるから、今夜はてつ夜でがんばります。	
\\	今夜は晩ご飯に手羽先が食べたい気分だ。	
\\	今, 夜	
\\	亡くなる	
\\	なくなる	
\\	まつ本さんが、亡くなったそうです。	
\\	わたしの妹は、五才のときに、はいえんで亡くなりました。	
\\	もし母さんが亡くなったら、僕の料理を作ってくれる人がいなくなってしまう。	
\\	う 
\\	亡く 
\\	なる 
\\	(死ぬ).
\\	な 
\\	亡く, 
\\	(なく) 
\\	亡	
\\	代える	
\\	かえる	
\\	かんとくは、ピッチャーを代えるだろう。	
\\	ふだんは、ヨーグルトをサワークリームに代えてつかっています。	
\\	この指輪をお金に代えることはできません。	
\\	え. 
\\	代わる 
\\	代	
\\	思う	
\\	おもう	
\\	かれも来ると思いますよ。	
\\	ハリケーンのきせつはメキシコに行くのをさけた方がいいと思いますか。	
\\	思うだけならまだ違法じゃないよ。	
\\	う 
\\	おも 
\\	(おも) 
\\	思	
\\	歩く	
\\	あるく	
\\	学こうまで歩くつもりです。	
\\	あなたと手をつないで歩くなんて、何年ぶりかしら。	
\\	私はとても太いので、私が道を歩いる時にペニーを踏んだら、女王様の鼻から鼻くそが飛び出ます。	
\\	う 
\\	ある 
\\	(ある). 
\\	歩	
\\	曲	
\\	きょく	
\\	その曲の名まえ、何だったっけ。	
\\	何かおすすめの曲はありますか?	
\\	彼がピアノで一曲弾いてくれたんですが、何故かそのピアノはギターの様な音がしたんです。	
\\	曲	
\\	地中	
\\	ちちゅう	
\\	の 
\\	地中は、冬にはあたたかく、なつはつめたいです。	
\\	多くのみずうみは、地中の水ろでつながっています。	
\\	地中の温度を測るにはどうすればいいですか。	
\\	地, 中	
\\	地下	
\\	ちか	
\\	の 
\\	このたてものの地下に、トイレがあります。	
\\	デパ地下でたくさんし食をしました。	
\\	ヨハネの黙示録がもうすぐ実現されるから、私は地下室を建設しているのだよ。	
\\	地, 下	
\\	不安	
\\	ふあん	
\\	な 
\\	ごうかくしているか不安です。	
\\	出さんを目前に不安な気もちでいっぱいです。	
\\	あの殺人事件を目撃してからというもの、不安を取り除くことができない。	
\\	不, 安	
\\	不明	
\\	ふめい	
\\	な 
\\	このはなしの作しゃは不明です。	
\\	この牛にゅうのしょうみきげんは不明です。	
\\	え?寝る時、玉ねぎを枕にしたの?あんたの言ってること、意味不明なんだけど。	
\\	不, 明	
\\	大文字	
\\	おおもじ	
\\	さいしょの字は、大文字でかいてください。	
\\	タイトルを大文字でかき直してください。	
\\	何故だかわからないけど、小文字より大文字の方が好きです。	
\\	大 
\\	大きい (おおきい). 
\\	大. 
\\	大, 文, 字	
\\	金星	
\\	きんせい	
\\	あそこに見えているのは、金星です。	
\\	金星の学こうには、うんどう会はありますか?	
\\	人類はいつか金星に移住することになるだろう。	
\\	金, 星	
\\	耳打ち	
\\	みみうち	
\\	する 
\\	かれは、わたしにこっそり耳打ちしました。	
\\	ミーティングのときに、コウイチがニッカになにか耳打ちしてるのをみてしまった。	
\\	私が彼に、何かセクシーな言葉を耳打ちしてというと、彼は「テープや
\\	は好きかい?」と囁いてくれた。	
\\	耳, 打	
\\	冬休み	
\\	ふゆやすみ	
\\	冬休みに、かぞくで山の中のホテルにとまりました。	
\\	冬休みは子どもたちがいえで兄弟げんかばっかりしていて大へんでした。	
\\	冬休みは、毎日一日中ジョギングをしていました。	
\\	冬, 休	
\\	羊毛	
\\	ようもう	
\\	の 
\\	クリスマスプレゼントに、羊毛のブランケットをもらいました。	
\\	この羊毛フェルトの人形に出会えて本当によかった。	
\\	羊毛の手袋は、枕の詰め物に適している。	
\\	羊, 毛	
\\	手首	
\\	てくび	
\\	ころんで手首をひねった。	
\\	あぶらが手首にはねてヤケドしてしまった。	
\\	双子の妹には、手首にホクロがあるんだ。	
\\	手, 首	
\\	年下	
\\	としした	
\\	の 
\\	もり田さんには、年下のかの女がいます。	
\\	コウイチはアーノルドシュワルツネッガーより三十八才年下だ。	
\\	あの子は年下なのにとても大人びている。	
\\	年 
\\	下.	年, 下	
\\	妹	
\\	いもうと	
\\	田中さんの妹はどの人ですか。	
\\	妹にうら切られた気分です。	
\\	私の妹は、バービー人形にウィッグを作ってあげたいからと、自分の髪の毛を切った。	
\\	(いもうと). 
\\	妹	
\\	以後	
\\	いご	
\\	「以後、このようなことがないように、気をつけるんだぞ。」「はい、先生。以後、もっと気をつけます。」	
\\	10月29日のトーフグの日以後、ビエトは行方をくらましているんだ。	
\\	それ以後、その作家は酒を一滴も飲まなくなった。	
\\	以, 後	
\\	先回り	
\\	さきまわり	
\\	する 
\\	先回りして、まちぶせしよう!	
\\	リクはとてもせっかちで、いつも人の話をかっ手によそうして先回りして話すから、すごくイライラするんだよね。	
\\	彼は先回りをして富士山の頂上に着いており、私のことをのろまと笑った。	
\\	(回り), 
\\	先 
\\	回る 
\\	回り). 
\\	先, 回	
\\	辺	
\\	へん	
\\	その辺に、ホテルはありますか。	
\\	あの辺はごう雪地たいです。	
\\	正三角形の辺の長さから、高さと面積が計算できます。	
\\	辺	
\\	辺り	
\\	あたり	
\\	辺りはしーんとしずまりかえっていた。	
\\	そうだ、ちょっと言いわすれていたんだけど、この辺りにえきはないからね。	
\\	彼のトイレはこの辺りに違いない。	
\\	(あた).	辺	
\\	この辺	
\\	このへん	
\\	の 
\\	この辺は、人が少ないのでしずかです。	
\\	この辺に、スタバはありますか?	
\\	この辺では、言葉に気をつけた方がいいぜ。町のこちら側に住んでいるやつらは、口笛を吹いただけでも顔に唾をぶっかけてきやがるからな。	
\\	"この 
\\	この辺 
\\	辺 
\\	辺	
\\	作家	
\\	さっか	
\\	あこがれの作家さんから、サインをもらいました。	
\\	この作家、ずいぶん思い切ったね。	
\\	あのベストセラー作家の趣味はサッカーだ。	
\\	さく 
\\	作 
\\	さっ. 
\\	作, 家	
\\	風船	
\\	ふうせん	
\\	あのピエロが、風船で犬を作ってくれたんだ。	
\\	ひ行きによったば合は、この風船をエチケットぶくろの代わりにごしようください。	
\\	風船を使って実際に空飛ぶ家を作った人たちがいます。	
\\	風, 船	
\\	近々	
\\	ちかぢか, きんきん, ちかじか	
\\	それについては、近々はっぴょうがあると思いますよ。	
\\	近々雪がふるはずなので、たくさんつもったらかまくらを作ります。	
\\	もしタバコを吸い続けるなら、あなたは近々癌を発症するでしょう。	
\\	々 
\\	近い 
\\	近 
\\	ぢか, 
\\	近, 々	
\\	〜札	
\\	さつ	
\\	一万円札を小ぜににくずしてもらえませんか?	
\\	この二千円札、千円札二枚と両がえしてくれない?	
\\	トーフグのオフィスのあちこちに、百ドル札が隠されているらしいぜ。	
\\	札	
\\	鳥	
\\	とり	
\\	この公えんには、きれいな鳥がたくさんいます。	
\\	けさカーテンをあけたら、何もかもが鳥のふんにおおわれていました。	
\\	自分が鳥だったらいいのになと思うよ。だって、そうすればもう鳥みたいに歩いているとからかわれないだろうからね。	
\\	(とり) 
\\	鳥	
\\	黒い	
\\	くろい	い 
\\	その上の、黒いぼう子をとってください。	
\\	あの黒ひげの男ははら黒いことで有名です。	
\\	私の一番好きなテレビ番組は、「黒い画面」です。この番組は、停電になった時だけ放送されます。	
\\	い 
\\	黒 
\\	い 
\\	(くろ). 
\\	黒	
\\	黒人	
\\	こくじん	
\\	の 
\\	なぜ未だに黒人をさべつする人がいるのか、りかいできません。	
\\	おれは黒人であいつは白人だけど、それがどうしたの?	
\\	ピーナッツバターは、ジョージ・ワシントン・カーヴァーという名の黒人によって発明された。	
\\	白人 
\\	黒, 人	
\\	青空	
\\	あおぞら	
\\	青空のせいで海は青いの?	
\\	未来の子どもたちに、今と同じくらいうつくしい青空をのこしてあげましょうよ。	
\\	今日は一日中晴れて、青空が広がっていました。	
\\	空 
\\	ぞら.
\\	青, 空	
\\	船	
\\	ふね	
\\	次の船は、何時に出ぱつしますか。	
\\	船にのると必ず船よいしちゃうんだよね。	
\\	俺様が船のすべてを取り仕切る船長だ。	
\\	(ふね). 
\\	船	
\\	教室	
\\	きょうしつ	
\\	学生は、教室ではじめに何をしますか。	
\\	かれは教室でタバコをくわえたところを、先生に目げきされました。	
\\	先生がいなくなったのはたったの五分間だけだったが、大学生たちは完璧に教室を滅茶苦茶にしてしまった。	
\\	教, 室	
\\	活用形	
\\	かつようけい	
\\	このどうしの活用形を教えてください。	
\\	昨日から、じょどうしの活用形をべんきょうし直してます。	
\\	この辞書アプリでは、活用形から動詞の逆引き検索も可能です。	
\\	活用 
\\	〜形 
\\	活, 用, 形	
\\	日記	
\\	にっき	
\\	あなたの日記も、この本と一しょにここにおいておきます。	
\\	北米には、こうかん日記みたいな文化はないんだよ。	
\\	昨日の夜、フグの日記を読んだんだけどさ、実はとても悲しい日記だったんだよね。	
\\	日 (にち) 
\\	っ, 
\\	にっき.	日, 記	
\\	〜丁目	
\\	ちょうめ	
\\	ぼくのかの女は、同じ町の二丁目にすんでいるミイコちゃんです。	
\\	その台風で、一丁目と二丁目がしん水してしまいました。	
\\	この桃、三丁目の田中さんから頂いたの。	
\\	丁, 目	
\\	人形	
\\	にんぎょう	
\\	たん生日プレゼントに、人形をもって行きます。	
\\	その人形、ちゃんと元のところにもどしておいてね。	
\\	17歳まで、私の親友は靴下で作った腕人形だった。	
\\	人 
\\	形 
\\	(ぎょう) 
\\	人, 形	
\\	小学生	
\\	しょうがくせい	
\\	小学生だからって、あまく見てるとけがするぜ!	
\\	きょうしつのドアをあけると、一人の小学生が石ころみたいにぐっすりねむっていました。	
\\	小学生の頃、いつもコインを投げて自分の行動を決めていました。	
\\	学生 
\\	小, 学, 生	
\\	月末	
\\	げつまつ	
\\	月末までに五キロやせます。	
\\	ここ、ブラックきぎょうすぎるから、月末ににげ出すわ。	
\\	私の上司は、コアラで給料を払います。毎月末、わたしには二匹のコアラが支払われます。	
\\	月 
\\	末 
\\	月, 末	
\\	魚	
\\	さかな	
\\	魚はすきですが、生魚はにが手です。	
\\	バランスのいいちょう食をとるために、いつもグラノーラと魚を食べます。	
\\	もう絶対にメキシコの軽トラの積み荷から魚を買うもんか。	
\\	(さかな).	魚	
\\	必死	
\\	ひっし	な 
\\	の 
\\	必死にがんばっても、どうにもならないこともある。	
\\	わたしのそ父は、必死のかくごでせんじょうに向かったそうです。	
\\	「隊長、私はどうしても火星に戻る必要があるんです。アイフォンを忘れてしまったのです。」彼は必死に頼んだ。	
\\	必 (ひつ) 
\\	ひっ. 
\\	つ 
\\	っ 
\\	必, 死	
\\	自由	
\\	じゆう	
\\	な 
\\	一人ぐらしをはじめて、自由なじかんができました。	
\\	自由になるための道のりはまだまだとおいですが、ぜったいにあきらめません。	
\\	日本のサラリーマンが自由に使えるお金は、毎月平均たった三万五千円だ。	
\\	自, 由	
\\	氏名	
\\	しめい	
\\	自分の氏名ぐらいは、日本ごでかけるようになりたいよね。	
\\	ごさん考までに、わたしの氏名は、高 一子です。	
\\	そちらの誓約書に氏名をご記入頂ければ、大砲で貴方の頭の上の林檎を撃ちぬいて差し上げます。	
\\	氏, 名	
\\	未来	
\\	みらい	
\\	の 
\\	タイムスリップできるなら、未来に行きたいな。	
\\	週末は、未来とか去を行ったり来たりしていました。	
\\	コウイチにはうちゅうひ行しとしての大きな未来があるんだから、たのむからじゃましないでやってくれよ。	
\\	未, 来	
\\	地理	
\\	ちり	
\\	一人で地理のべん強をするのもすきですが、友だちとべん強する方がたのしいです。	
\\	アメリカ人の友人は地理が全くのにが手で、さい近まで日本が中国の首とだとおもっていたらしい。	
\\	俺たちが道に迷ったってどういうことだよ?お前、地理の先公じゃねえのかよ?	
\\	地, 理	
\\	金魚	
\\	きんぎょ	
\\	これは、五年前の夏まつりの金魚すくいですくった金魚です。	
\\	金魚もかぜをひくんですか?	
\\	金魚は、三秒間しか記憶がないので、自分たちがお城に住んでいることに気がづく度に驚いています。	
\\	金, 魚	
\\	組	
\\	くみ	
\\	これから、みなさんを二組に分けます。	
\\	オレたちの組はこの辺じゃ有名なんだぜ。	
\\	三年二組のコウイチくん、またこく白されたらしいよ。	
\\	組み 
\\	くみ).	組	
\\	手作り	
\\	てづくり	
\\	の 
\\	このテーブルは、林さんの手作りです。	
\\	毎日手作りのバランスのいい食じをとるようにしています。	
\\	手作りのプレゼントは、どれだけ暇な時間があるかってことがバレてしまうので、結構怖い。	
\\	(手 
\\	作る). 
\\	る 
\\	作る 
\\	り 
\\	つ 
\\	つくり 
\\	づ 
\\	手, 作	
\\	大空	
\\	おおぞら	
\\	大空を見ていたら、いやな気もちがふきとんだ。	
\\	アイリちゃんは大空にうかぶお星さまになったんだよ。	
\\	鷲が大空を悠々と飛んでいた。	
\\	そら 
\\	ぞら 
\\	大, 空	
\\	高校	
\\	こうこう	
\\	高校まで、車で行けば十分ですよ。	
\\	高校ではいつも、おひるにどんなものを食べていますか?	
\\	高校に行って初めて話をし始めたんです。そして、今は、話を止めることができません。	
\\	高, 校	
\\	家	
\\	いえ, うち	
\\	の 
\\	みなさん、家にあそびに来てください。	
\\	いいかげん、家をかた付けた方がいいよ。	
\\	家とは、みんなの心と食べ物がある場所です。	
\\	(いえ). 
\\	うち. 
\\	いえ, 
\\	家	
\\	作り方	
\\	つくりかた	
\\	おいしいマカロニチーズの作り方を教えてください。	
\\	麦茶の作り方には、おゆでわかしながらに出す方ほう、ふっとうさせたおゆでジワジワおゆ出しする方ほう、つめたい水でジワジワ水出しする方ほうの三しゅるいありますが、それぞれメリットデメリットがあります。	
\\	手裏剣の作り方を知りたいですか?	
\\	作, 方	
\\	船体	
\\	せんたい	
\\	船体にらくがきをしたのはだれだ!?	
\\	クラーケンがしょく手で船体をつきやぶったんだ。	
\\	船体は思った以上に大きかった。	
\\	船, 体	
\\	大きく	
\\	おおきく	
\\	大きくいきをついてください。	
\\	このトウフ&フグのピザチェーンは、大きくきぼをかく大している。	
\\	円の価値が大きく上昇した。	
\\	大きい, 
\\	大	
\\	先週	
\\	せんしゅう	
\\	先週は学校はお休みでした。	
\\	大きくなったねー。先週はまだあんなに小さかったのに。	
\\	先週出席した経済相互援助会議で、喋る犬にインタビューされたんだが、あいつは本当に嫌なやつだった。	
\\	先, 週	
\\	以下	
\\	いか	
\\	五千円以下の本だなをさがしています。	
\\	コウイチのすきなネコのしゅるいは以下の通りです。	
\\	以下同文です。	
\\	以, 下	
\\	手紙	
\\	てがみ	
\\	の 
\\	外国に手紙を出しました。	
\\	その金正日からの手紙、ネットオークションに出しちゃえば?きっとだれかしらかうと思うよ。	
\\	私はコカコーラ宛に、心のこもった手紙を書いて、どうして彼らの製品にたくさん砂糖が入っているのかについて尋ねたいと思います。	
\\	がみ, 
\\	かみ.	手, 紙	
\\	小声	
\\	こごえ	
\\	の 
\\	二人とも、どうして小声ではなしているの?	
\\	「ビエトに学ランをきてもらいたいんだ。に合うと思うんだよね。」と、わたしのもうそうの中でコウイチがビエトに小声で言った。	
\\	男は、小声で話し始めた。	
\\	大声, 
\\	こえ 
\\	ごえ 
\\	小, 声	
\\	心理	
\\	しんり	
\\	ねこの心理を思うがままにあやつれるようになりたい。	
\\	春人以外の男どもには、女せいのせんさいな心理は分からないんだよ。	
\\	サーモンと彼女に翻弄される人々の心理を、君の小説で描いてみてもらえないかな?人気が出ると思うんだけど。	
\\	心, 理	
\\	中学生	
\\	ちゅうがくせい	
\\	中学生にもなって、まだお母さんといっしょにおふろに入ってるんですか?	
\\	かち目のないたたかいはしない方がいいということは、中学生でも知っているよ。	
\\	中学生の頃、自分自身に、「書かなくても覚えられるから大丈夫」という大法螺をふいてしまいました。	
\\	学生 
\\	中, 学, 生	
\\	〜氏	
\\	し	
\\	やはり、アベ氏のはつげんはいつも全くブレませんね。	
\\	コウイチ氏とビエト氏はとても気が合う。	
\\	おお、ホワイト氏、ちょっと座れば?	
\\	氏	
\\	以上	
\\	いじょう	
\\	このえいがは、十八才以上しか見れません。	
\\	手つづきは以上です。	
\\	100歳以上の人は、必ずブラジャーを着用し無くてはならない。	
\\	以, 上	
\\	近く	
\\	ちかく	
\\	今は、学校の近くに、姉とすんでいます。	
\\	あのアイドル、近くで見るとあんまりかわいくないね。	
\\	近くの焼き鳥屋で、一杯ひっかけていかないか。	
\\	近い, 
\\	近	
\\	来週	
\\	らいしゅう	
\\	来週の日よう日にレストランに行きます。	
\\	来週はどれぐらい雪がふるかな?	
\\	飲み会は来週に延期になりました。	
\\	来, 週	
\\	未だ	
\\	まだ	
\\	未だ二百才まで生きた人げんはいない。	
\\	未だ、時さボケがあるんですよね。	
\\	仕事が決まったのかどうかは未だ分からないんだ。向こうは未だ、十二人も面接しなきゃいけないみたい。	
\\	未, 
\\	未	
\\	海魚	
\\	かいぎょ, うみざかな	
\\	さけは海魚ですか?それとも川魚ですか?	
\\	海中には、色とりどりの海魚がおよいでいました。	
\\	水族館には沢山の種類の海魚がいますが、スタッフは僕達に一匹も食べてはいけませんと言いました。	
\\	海, 魚	
\\	〜時	
\\	じ	
\\	三時になったらおやつ休けいにしましょう。	
\\	コウイチの家の台どころで二時半にまちあわせね。	
\\	いつも何時にワニカニのレビューをしますか?	
\\	1時 
\\	23時 
\\	時	
\\	お兄さん	
\\	おにいさん	
\\	どれがたかしのお兄さんですか。	
\\	今日のたっきゅうびんのお兄さんのかおがタイプすぎた。	
\\	お兄さんのはなげがきになってしかたがない。	
\\	お 
\\	兄 
\\	さん 
\\	兄).
\\	兄 
\\	(にい), 
\\	お兄さん!
\\	兄	
\\	一斤	
\\	いっきん	
\\	一斤のパンは6枚切りか8枚切りで売られていることが多い。	
\\	たった一斤のパンで給料日までを生き延びなければいけない。	
\\	食パン一斤と、ニュテラがあれば生きていけるよ。	
\\	一, 斤	
\\	一時	
\\	いちじ	
\\	毎日一時かんぐらいおふろに入ります。	
\\	一時までにおもちゃをかたづけなさい。	
\\	もし私のところで働くなら、あなたのシフトは午後一時から遅くとも三時までになりますが、よろしいですか?	
\\	一, 時	
\\	以外	
\\	いがい	
\\	ビール以外ならなんでもいいです。	
\\	マンガ以外は何もよみません。	
\\	式の間中新郎が花嫁の胸元を上から眺めていたこと以外は、とても素敵な結婚式でした。	
\\	以, 外	
\\	紙	
\\	かみ	
\\	紙とえんぴつとけしゴムをつくえに出してください。	
\\	あたらしい紙をかうのがたのしみでしかたがない。	
\\	昨日、紙の切れ端に詩を書いている狼を見かけた。	
\\	紙	
\\	時代	
\\	じだい	
\\	今の時代は、インターネットで何でもかうことができます。	
\\	明里は、わたしの大学時代の友人です。	
\\	紫式部は、平安時代中期の女性作家かつ歌人です。	
\\	時, 代	
\\	以前	
\\	いぜん	
\\	以前、パーティーで一どお会いしましたよね。	
\\	以前はよく船よいしていたんですが、今はさっぱりなくなりました。	
\\	私の親友は、以前は天文学者に知られていなかった惑星の出身です。	
\\	以, 前	
\\	心理学	
\\	しんりがく	
\\	えいぎょうをするなら、心理学を学んだ方がいいよ。	
\\	心理学のせつ明によると「あい手にひどいことが言えるということは、その人にいぞんしてるということ」らしいです。	
\\	私は大学で犯罪心理学を専攻していました。	
\\	心, 理, 学	
\\	本当に	
\\	ほんとうに	
\\	日本の人たちは、本当にやさしいです。	
\\	このせいひん、本当に高くなっちゃったねー。	
\\	本当にこの道であっていますか?	
\\	本当 
\\	本当に! 
\\	本当 
\\	本, 当	
\\	理由	
\\	りゆう	
\\	とくに理由はありません。	
\\	未だけいひのせいさんをしてないのには、何か理由があるんですか?	
\\	論理的な理由から、彼は行方不明になったゴム手袋の事件の犯人が風船屋であることを導いた。	
\\	理, 由	
\\	〜号室	
\\	ごうしつ	
\\	おきゃくさまのおへやは、302号室になります。	
\\	このマンションの444号室はゆうれいが出ることで有名です。	
\\	102号室の坂本と申します。	
\\	号, 室	
\\	失礼	
\\	しつれい	
\\	する 
\\	な 
\\	失礼ですが、日本ごははなせますか。	
\\	じっさいの会話で「あなた」という言ばを使うと、失礼にきこえることが多いです。	
\\	もし私があなたに太いと言えば、失礼だと思いますか?	
\\	失, 礼	
\\	同時	
\\	どうじ	
\\	の 
\\	先生が教室に入ってきたと同時に、しぎょうのチャイムがなりました。	
\\	でん子レンジとトースターを同時につかったから、ブレーカーがおちちゃったんだ。	
\\	「今日はよくできましたね。それではまた来週同じ時間に。」と、コスプレのプロとアニメオタクが同時に言った。	
\\	同, 時	
\\	学校	
\\	がっこう	
\\	わたしは歩いて学校に行っています。	
\\	学校にてい出するので、ちえんしょう明しょを下さい。	
\\	今日はどうして学校で授業中にホットドッグをバーベキューするの?	
\\	学 (がく) 
\\	っ 
\\	がっこう.	学, 校	
\\	形	
\\	かたち	
\\	つまとそうだんして、形より内ようをおもんじたけっか、この形にきめました。	
\\	これでなんとか形がつきました。	
\\	ねえ、私の、お尻の形をしたお腹、撫でたい?	
\\	(かた) 
\\	(ち) 
\\	形	
\\	欠点	
\\	けってん	
\\	だれだって、何かしらの欠点をもってるものさ。	
\\	そうだ、ちょっと思い出したんだけど、かの女って欠点がないんだよ。	
\\	私は欠点がありすぎるので、絶対にプロのサッカー選手にはなれません。たとえば、私には足がありません。	
\\	けつ 
\\	けっ 
\\	欠, 点	
\\	雪	
\\	ゆき	
\\	日本でも、雪がふりますか?	
\\	そっちも雪、つもった?	
\\	私が初めて雪を見たのは、四十六歳の時でした。	
\\	雪	
\\	社会	
\\	しゃかい	
\\	今家を出ても、社会のじゅぎょうに間にあわないと思います。	
\\	国さい社会の一員として、めしテロにはだんじてくっしません。	
\\	社会問題として、若者がお葬式で自分撮りをすることについて考える必要がある。	
\\	会社, 
\\	社, 会	
\\	強力	
\\	きょうりょく	
\\	な 
\\	このボンドは、かなり強力なので気をつけてください。	
\\	強力なミニウォーターポンプの作り方をおしえてもらいました。	
\\	彼女を見つめると、まるで十人力ぐらい強力になったように感じる。	
\\	強, 力	
\\	強い	
\\	つよい	い 
\\	この中で一ばん強いのはだれですか。	
\\	クチバシと羽の色からさっするに、こいつはおそらくめちゃくちゃ強いカラスだね。	
\\	海流はとても強く、その猫を川まで運んだ。	
\\	い 
\\	(つよ). 
\\	強	
\\	夏	
\\	なつ	
\\	今年の夏は、さく年よりもあついです。	
\\	近いうちに夏ワンピをかいに行きたいんだよね。	
\\	夏にして楽しいことは三つだけだ。映画、ビデオゲーム、それから木彫の鴨を作ること。	
\\	夏	
\\	夏休み	
\\	なつやすみ	
\\	夏休みは、一ヶ月あります。	
\\	夏休みにおきなわでダイビングをした時、しんじられないくらいたく山のしゅるいの魚を見ることができました!	
\\	小さいころ、夏休みに両親がディズニーランドに連れて行ってくれました。でも、帰り際、両親は何と私のことを忘れて帰ってしまったのです。あの時だけは、迷子になったことを嬉しく思いましたね。	
\\	夏 
\\	休 
\\	夏 
\\	休み 
\\	夏, 休	
\\	必ず	
\\	かならず	
\\	かれは必ずもどってくるので、大じょうぶです。	
\\	必ずねる前にコンタクトを外した方がいいよ。でないとし力がさらにおちるよ。	
\\	何があっても必ずさい後までワニカニをやりとげます。	
\\	(かなら). 
\\	必	
\\	白鳥	
\\	はくちょう	
\\	自どりをするためだけに白鳥をみずうみから引きずり出すなんてひどすぎるよ。	
\\	白ワインは魚りょうりに合うっていうけど、白鳥りょうりに合うワインは何か知っていますか?	
\\	白鳥はとても美しいが、汚らしい言葉遣いをする。	
\\	白, 鳥	
\\	札	
\\	ふだ	
\\	このお札は、げんかんのドアの内がわにはっておいてください。	
\\	自分の札を見せないようにした方がいいよ。	
\\	ジェロってほんとにキモい。ドクロマークの絵が描かれた札でも付けておくべきだよ。	
\\	(ふだ)!	札	
\\	高い	
\\	たかい	い 
\\	この店は、高いですがおいしいです。	
\\	わたしには高すぎてとどかないので、とてもせが高いジェームズ・ステュアートを天国からよんできました。	
\\	どうして、「ビッグアンドトール」のスーツはあんなに値段が高いんだ?	
\\	い 
\\	(たか).	高	
\\	末	
\\	すえ	
\\	こんなあそびがはやるとは、世も末だな。	
\\	おたくのむす子さんは末が本当にたのしみですね。	
\\	激しい口論の末、末息子は私の金をどこかに埋めるために出て行ってしまいました。	
\\	末	
\\	一体	
\\	いったい	
\\	一体どうなっているんだ。	
\\	「あいとにくしみはひょうり一体」って、一体全体どういういみなんだ!?	
\\	一体俺が何をしたって言うんだよ?	
\\	いち 
\\	いっ 
\\	一, 体	
\\	国民	
\\	こくみん	
\\	本当に国民はわれわれにさんせいしてくれるのでしょうか。	
\\	どうすれば日本国民になれますか。	
\\	昨夜、
\\	4の新ゲームを買うために並んでいた国民たちは、凍って死にました。	
\\	国, 民	
\\	今週	
\\	こんしゅう	
\\	今週は仕ごとがたくさんあります。	
\\	今週の火よう日の午前十一じに、しぶ谷のハチ公前でまち合わせをしましょう。	
\\	新しいゲームを4つ買ったので、今週はかなり忙しくなる。	
\\	今, 週	
\\	考え	
\\	かんがえ	
\\	いつも自分の考えを上手くせつめいできません。	
\\	あなたの考えはあますぎます。	
\\	あなたの考えを百円で買わせてもらえませんか?	
\\	考える, 
\\	る!	考	
\\	時	
\\	とき	
\\	れいぼうをつける時は、白いボタンをおしてください。	
\\	時がかい決してくれるのをまつしかないよ。	
\\	幸せな時、いつも時が止まればいいのに、と思う。	
\\	(とき) 
\\	時	
\\	付ける	
\\	つける	
\\	グリルドチーズにはケチャップを付けるはですか?	
\\	どうしてワンちゃんに、ニッカって名前を付けたんですか?	
\\	どうして頭に蛭を付けているんですか?	
\\	う 
\\	付	
\\	見当たる	
\\	みあたる	
\\	さいふが見当たりません。	
\\	さがしたけど、どこにも見当たらなかったんです。	
\\	俺の老眼鏡が見当たらないんだ。	
\\	う 
\\	見, 当	
\\	欠ける	
\\	かける	
\\	自てん車でこけて、前ばが欠けてしまった。	
\\	たとえばの話なんだけど、もしコウイチのお気に入りのきゅうすが口が欠けたせいでわたしたちがかいこされてしまっても、しつぎょうほけんをしんせいすることってできるのかな。	
\\	私には、セーラームーンを見るのを止めるための強い意志が欠けている。	
\\	う 
\\	(か). 
\\	欠	
\\	失う	
\\	うしなう	
\\	もうおれには失うものが何もないんだ。	
\\	トーフグがギャンブルで全ざいさんを失ったというじょうほうは、しんぴょうせいに欠けている。	
\\	人生最悪の日は、私が
\\	48のフィギュアを失った日です。	
\\	牛 
\\	(うし 
\\	な) 
\\	失	
\\	通る	
\\	とおる	
\\	このみちは、ほそくて通りにくいです。	
\\	日本ごがペラペラになるなんてありえないと思うかもしれませんが、それはあたらしい言ごを学ぶだれもが通る道です。	
\\	お茶会のためのメープルシロップを買うために、カナダを通ってイギリスに行った。	
\\	う 
\\	(とお) 
\\	通	
\\	光る	
\\	ひかる	
\\	星がきらきら光っている。	
\\	右のくつに、光るとりょうが付いてるよ。	
\\	あそこで確かに何かが光るのを見たんだ。	
\\	う 
\\	光? 
\\	り 
\\	る 
\\	光	
\\	教える	
\\	おしえる	
\\	こたえ方を教えてください。	
\\	ワニカニで日本ごを教える人の中には、「鬼コーチ」はいないよね?	
\\	皆さん、お早うございます。今日は、皆さんに、茄子を食べることの危険性について教えたいと思います。	
\\	う 
\\	(おし).
\\	教	
\\	町民	
\\	ちょうみん	
\\	今すぐ町民をひなんさせるべきです!	
\\	この町民たちは、みんなちゃんと後かた付けができる。	
\\	この町の人達はみんな、この町の町民であることを誇りに思っています。	
\\	町, 民	
\\	交通	
\\	こうつう	
\\	する 
\\	ここでは、交通じこがよくおきるんです。	
\\	トーフグのオフィスが入ったビルの前の通りは交通がはげしい。	
\\	もし俺の車にヘリコプターのプロペラがあれば、空を飛んで交通渋滞を回避することができるのになぁ。	
\\	交, 通	
\\	三角形	
\\	さんかくけい, さんかっけい	
\\	の 
\\	どうすれば、三角形のオニギリをにぎれるんですか?	
\\	夏の夜空には、天の川をはさんでことざとわしざと白鳥ざの一とう星をむすんだ大三角形が広がります。	
\\	そこには、三角形の公園があります。	
\\	三角 
\\	三, 角, 形	
\\	二斤	
\\	にきん	
\\	二斤買えば100円もお得です。	
\\	パンを二斤も焼いたのに、すぐに食べ切ってしまった。	
\\	賞味期限が明後日だけど、二斤も食べきれるかな?	
\\	二, 斤	
\\	二時半	
\\	にじはん	
\\	そのえいがのかいえん時かんは二時半です。	
\\	二時半に大じなやくそくがあるのにねぼうしちゃって、ちょっと今かなりテンパってます。	
\\	「私は明日午後二時半に、私のお母さんのお父さんの娘の姪っ子に会います。」「なんで、単にお姉ちゃんとランチに行くって言わないの?ありえないんだけど!」	
\\	二, 時, 半	
\\	台風	
\\	たいふう	
\\	今回の台風は弱いので、あまり心ぱいしなくてもよさそうです。	
\\	ひじょうに強い台風トウフ号が、月よう日、西日本に上りくしました。	
\\	大型の台風が日本列島に近づいています。	
\\	だい 
\\	台 
\\	たい. 
\\	(たい) 
\\	台, 風	
\\	大学生	
\\	だいがくせい	
\\	男の大学生と女の大学生がはなしています。	
\\	あの大学生の女せいは、ダイエット中です。	
\\	マカロニチーズしか食べていないっていうのに、いったいどうして大学生は一日中勉強して夜中中お酒を飲む元気があるっていうんだ?	
\\	学生 
\\	大学 
\\	大, 学, 生	
\\	不人気	
\\	ふにんき	
\\	な 
\\	今年、もっとも不人気だったアニメは何だと思いますか。	
\\	コウイチたいマイケルのすもうのし合は、
\\	で見るとはく力に欠けるので不人気です。	
\\	「お前の母ちゃんに言ってやる」が口癖だったため、彼はクラスメートに不人気だった。	
\\	人気 
\\	不 
\\	不, 人, 気	
\\	言い方	
\\	いいかた	
\\	そうは言っても、言い方ってもんがあるだろ!	
\\	そうだ、話はかわるんだけど、コウイチのネコって言ばの言い方、めっちゃかわいくない?	
\\	あなたの物の言い方はとても馬鹿です。	
\\	言う. 
\\	方 
\\	言, 方	
\\	考え方	
\\	かんがえかた	
\\	考え方は、人それぞれです。	
\\	そもそも、父と母は考え方がまったくちがうんです。	
\\	自分のことを悪い人間だと思うのは、もし自分が実際に悪女だったとしても、良くない考え方です。	
\\	考, 方	
\\	弱い	
\\	よわい	い 
\\	わたしは、あさに弱いです。	
\\	ユイは体は弱いけど、人のいたみがわかる人げんです。	
\\	インフルエンザにかかると、スーパーマンでさえもとても弱くなります。	
\\	い 
\\	(よわ) 
\\	弱	
\\	弱点	
\\	じゃくてん	
\\	ついにアイツの弱点を見つけたぜ!	
\\	あの体そうせん手の弱点は、せいかくがじゅうなんせいに欠けているところだ。	
\\	私の従業員としての弱点は、恐らく弱点が一つも無いことです。	
\\	弱, 点	
\\	週末	
\\	しゅうまつ	
\\	週末は、おきなわにりょ行に行っていました。	
\\	せん週末、雪が十センチつもりました。	
\\	宝くじに当たったんだけど、週末ローマにピザでも食べに行かない?	
\\	週, 末	
\\	何時	
\\	なんじ	
\\	何時のひ行きにのるんですか?	
\\	目まいをかんじはじめたのは何時ですか?	
\\	今夜何時にボーリングに行くの?	
\\	何, 時	
\\	風	
\\	かぜ	
\\	外では、つめたい風がふいていました。	
\\	風が強くなってきましたね。	
\\	風が彼女のカツラを吹き飛ばした時に、私は自分が坊主の女性にとても魅力を感じることに気が付きました。	
\\	かぜ 
\\	風	
\\	年末	
\\	ねんまつ	
\\	すみませんが、年末は、ちょっとバタバタしてじかんがとれなさそうです。	
\\	マイケルは、年末にトーフグのだい二えいぎょうぶにいどうになりました。	
\\	年末までには、氷中毒から立ち直れるといいんだけど。	
\\	年, 末	
\\	黄色	
\\	きいろ	
\\	な 
\\	黄色かみどり色のペンはありますか?	
\\	アメリカのカラスは、美しい黄色の羽をもっていることで有名です。	
\\	もし尿のいろが透明から黄色に変わってしまったら、もっと水を飲んだ方が良いという合図です。	
\\	黄, 色	
\\	地上	
\\	ちじょう	
\\	の 
\\	木のはが地上におちた。	
\\	コウイチは地上1
\\	の所に、ひみつき地をもっている。	
\\	東京タワーは、地上から見上げると物凄い迫力がありますね。	
\\	地下, 
\\	地, 上	
\\	住所	
\\	じゅうしょ	
\\	こちらにあなたの家の住所をかいてください。	
\\	クリスマスカードをおくりたいので住所をおしえてくれませんか?	
\\	私が月に購入した土地には、住所はありますか?自分宛に酸素を送りたいので、よろしくお願い致します。	
\\	住, 所	
\\	入所	
\\	にゅうしょ	
\\	する 
\\	わたしの母は、来年のはるから、かいごしせつに入所することになりました。	
\\	こちらに入所年月日を記入してください。	
\\	俺の兄貴は今入所中だ。	
\\	入, 所	
\\	助言	
\\	じょげん	
\\	する 
\\	の 
\\	あなたの助言、ぜんぜん役に立たなかったんだけど。	
\\	かの女の助言にしたがって、コウイチはついにタバコをやめたんだよ。	
\\	そのチョコレートをくれたら、助言をあげるわ。	
\\	助, 言	
\\	全米	
\\	ぜんべい	
\\	これは、全米がなみだした、大ヒットチョ~大作えいがです。	
\\	まつ山ひできは、2017年の全米オープンで二いに入った。	
\\	全米ヒットチャートで今一位の曲は何ですか?	
\\	米国 
\\	米 
\\	全日本 
\\	全, 米	
\\	思い出	
\\	おもいで	
\\	思い出って、むねにそっとしまっておくものなんじゃないの?	
\\	子どものころ、日がくれるまで友だちとあそんでいたことは、今となってはとてもなつかしいいい思い出です。今はもうかん全ボッチですからね。	
\\	私のお父さんが毎週日曜日の朝にチョコレートでコーティングされたベーコンを作ってくれたのは、とてもいい思い出です。	
\\	思う? 
\\	思.	思, 出	
\\	〜間	
\\	かん	
\\	気がついたら、五時間ぐらいねむっていました。	
\\	たのむから三分間だけだまっててくれないか?	
\\	試験の中に、一定時間息を止めるというものがありました。一番難しかったのは、三時間息を止めるというものだったけれど、私はなんとかやり切りました。	
\\	二年間 
\\	一時間 
\\	〜間 
\\	二分間.	間	
\\	他所	
\\	たしょ, よそ	
\\	ここじゃなくて、どこか他所の店に行こうよ。	
\\	他所は他所、家は家!	
\\	私は暴力夫から逃れるために、他所に移り住んだ。	
\\	他, 所	
\\	答え	
\\	こたえ	
\\	答えは、1、2、3、4から、一番いいものを一つえらんでください。	
\\	山口さんの答えは、いつも通りでした。	
\\	テストの最初の質問の答え、何て書いた?「名前をここに書きなさい」ってやつだよ。自分の名前を書き込んだ?それとも「名前」ってそこに書いた?	
\\	う-
\\	い-
\\	答え 
\\	答	
\\	助力	
\\	じょりょく	
\\	する 
\\	そのせつは、ご助力をたまわりかんしゃ申し上げます。	
\\	のどうがさい生カウントをのばすため、コウイチは助力をもとめてタランティーノに電話をかけた。	
\\	スパイダーマンの助力を求める必要はない。	
\\	助, 力	
\\	作者	
\\	さくしゃ	
\\	この小せつの作者の名前がどうしても思い出せません。	
\\	ついにテキストフグの作者に会えました。	
\\	俺の兄貴がギリシャ神話の作者だったんだが、今は失業している。	
\\	作, 者	
\\	電池	
\\	でんち	
\\	このおもちゃの電車は、電池でうごきます。	
\\	ママ、見てー!電池が取れちゃった。	
\\	自作のロボット犬の電池を買いに、ちょっくら町へ行ってくるよ。	
\\	電, 池	
\\	身	
\\	み	
\\	リストラとりこんで、身も心もズタズタです。	
\\	この魚の身はたん白でおいしいね。	
\\	現金は常に身につけておく方がいい。	
\\	(み).	身	
\\	心身	
\\	しんしん	
\\	の 
\\	かれは、心身ともに、つかれきっていたんだと思います。	
\\	取りえは、心身ともにけん全であることだけです。	
\\	ヨガは、心身共にとても健康にしてくれる。	
\\	心, 身	
\\	両者	
\\	りょうしゃ	
\\	それは、両者にひがあると思うよ。	
\\	コウイチせん手とマイケルせん手、今のところ両者のじつ力はほぼご角のようです。	
\\	車両事故に関与している両者は、共に衝突時に携帯電話で話をしていた。	
\\	者 
\\	両, 者	
\\	〜君	
\\	くん	
\\	今年の夏、私は山田君と、ディズニーランドの中にあるホテルにとまりました。	
\\	ビエト君って、せんたくきを一日に七十七回も回すらしいよ!	
\\	日本の男の子の名前の後ろには「君」という言葉が付けられるそうだが、ラという名前の男の子があまりいないというのは残念ですね。	
\\	さん) 
\\	こういち君) 
\\	君	
\\	支局	
\\	しきょく	
\\	千ば支局にてんきんすることになりました。	
\\	東京支局の支局員のたこやきパーティーによんでもらった。	
\\	テレビ大阪京都支局の前でくしゃみが出た。	
\\	支店 
\\	支, 局	
\\	支店	
\\	してん	
\\	わたしのおっとは、大さか支店の支店長をつとめています。	
\\	この支店には、一人で行どうすることがすきな人もはたらいている。	
\\	支店長の髭は少し長すぎる。	
\\	支局 
\\	支, 店	
\\	朝ごはん	
\\	あさごはん	
\\	朝ごはんにパンとたまごをたべた。	
\\	朝ごはんにはベーコンに決まってるだろ?こんな当たり前のことも分からないのか。	
\\	滅茶苦茶お腹が空いていて、朝ごはんに腐った卵を6つも食べてしまったこと、今までにある?	
\\	ごはん 
\\	あさ 
\\	朝	
\\	局	
\\	きょく	
\\	ゆうびん局とびょういんもあります。	
\\	あのテレビ局、のっとっちゃおうぜ!	
\\	彼は、マクドナルドのドライブスルーで待たされすぎたからといって、アメリカ連邦捜査局に電話をかけようとした。	
\\	局	
\\	雲	
\\	くも	
\\	雲一つない青空でした。	
\\	この間おごってくれたおかえしに、あそこの雲をあげるよ。	
\\	飛行機は雲の後ろに消え、もう二度とその姿を現すことはなかった。	
\\	雲	
\\	楽	
\\	らく	
\\	な 
\\	あんたはそうやって楽することばっかり考えて!	
\\	このいすめっちゃ楽だわー。	
\\	私は愛犬トトと一緒にいる時が一番楽です。	
\\	楽	
\\	全く	
\\	まったく	
\\	これじゃあ、全く話にならないよ。	
\\	コウイチは、全く見かえりをもとめずに人助けをします。	
\\	コンドームも全く安全だって訳はない。俺の友達はゴムを着けてたけど、軽トラにはねられたしな。	
\\	(まった). 
\\	っ, 
\\	全	
\\	対立	
\\	たいりつ	
\\	する 
\\	の 
\\	あととりあらそいで、兄と弟は対立してしまったんです。	
\\	対立するんじゃなく、一しょにきょう力してがんばろうよ!	
\\	もう一回俺のことを対立を避けるような人間だと言ってみろ!お前の顔面をぶん殴ってやる。	
\\	対, 立	
\\	会話	
\\	かいわ	
\\	する 
\\	男の人と女の人が、大学のろう下で会話をしています。	
\\	ペースのはやい日本ごの会話は、まだりかい出来ません。	
\\	どうして私達の会話はいつも野球についてなのでしょうか?	
\\	会, 話	
\\	決	
\\	けつ	
\\	それでは、決をとりたいと思います。	
\\	なかなか決が出ないみたいだね。	
\\	ジョン、ホットドッグアイスなんてものを作り始める決を下したのはお前だろう?だから首になったんだよ。	
\\	決	
\\	決心	
\\	けっしん	
\\	する 
\\	あなたのおかげで、せい形手じゅつをする決心がつきました。	
\\	インスタセレブになるというトーフグの決心はトウフと同じぐらいかたい。	
\\	ついに、あなたに、どこから私の誕生日プレゼントの酒を買ってきてもらうか決心したよ。それは…日本だ。楽しんできてね!	
\\	つ. 
\\	けつ 
\\	けっ.	決, 心	
\\	医者	
\\	いしゃ	
\\	今日は医者に行かなくちゃいけない。	
\\	ある夜、とあるあやしいレストランに、もくもくとサメを食べる三人の医者がいました。	
\\	僕は医者になりたいけど、人に触ることが嫌いなんだ。それってやっぱり問題かな?	
\\	医, 者	
\\	役目	
\\	やくめ	
\\	おれのゆう者としての役目は、もうおわったんだ。	
\\	サラさん、来客のめんどうを見るのはあなたの役目でしょう?	
\\	役目を果たせなかったことを残念に思っています。	
\\	役 
\\	目. 
\\	役, 目	
\\	東口	
\\	ひがしぐち	
\\	えきの東口で大道げいをしている人がいたよ。	
\\	東口でとった写しん、たぶん手ブレしたみたいで、ちょっとブレちゃってる。	
\\	緊急の際には、このビルの東口から避難してください。そこでは、コーヒーとドーナツが支給される予定です。	
\\	入り口 
\\	出口 
\\	口 
\\	口, 
\\	くち 
\\	ぐち.	東, 口	
\\	電気	
\\	でんき	
\\	電気をつけてください。	
\\	電気をつけっぱなしにしないでって言ったじゃん!かせぎが少ないんだから、ちゃんと電気代せつやくにきょう力してよ。	
\\	子供の頃は暗闇を恐れていたが、今は電気料金の請求書を受け取る度に、光が怖いと思うよ。	
\\	電, 気	
\\	電力	
\\	でんりょく	
\\	2016年4月から日本でスタートした、電力自由化についてくわしく教えてください。	
\\	あの電力会社には電気代をはらいたくないよ。	
\\	電力が約三時間で復旧したので、私は面接の前に電動バリカンで頭を刈って整えることができました。	
\\	電, 力	
\\	電子	
\\	でんし	
\\	電子のおもさはいくらですか?	
\\	その電子ピアノを直すのに、どれくらい時間がかかると思いますか?	
\\	ティラノサウルスは、1万電子ボルトの衝撃を受けた。	
\\	電, 子	
\\	絵文字	
\\	えもじ	
\\	みなさんは絵文字はつかいますか?	
\\	しいて言えば、この絵文字がすきです。	
\\	一番よく使う絵文字を教えてください。	
\\	絵, 文, 字	
\\	工学者	
\\	こうがくしゃ	
\\	「ルンバ」をはつ明したロボット工学者はだれですか?	
\\	あの工学者のことはきらいですが、そうするべきだと思ったから、助けてあげました。	
\\	工学者になるためには、長期間勉強をする必要があります。	
\\	工学 
\\	工学. 
\\	工学 
\\	工学者 
\\	工, 学, 者	
\\	入場	
\\	にゅうじょう	
\\	する 
\\	会場に入場するには、チケットが必ようです。	
\\	さいしゅう入場時かんは午後五時です。	
\\	ちくしょう!「確実な死への部屋」への入場門が塞がれている。	
\\	入, 場	
\\	毎朝	
\\	まいあさ	
\\	毎朝、シャワーをあびます。	
\\	毎朝、本当に楽しかったです。ありがとうございました。	
\\	毎朝ラジオ体操をします。	
\\	毎, 朝	
\\	工場	
\\	こうじょう, こうば	
\\	トヨタの工場に行きたいんですけど、何で行くのがべんりですか。	
\\	毎ばん、ベーコン工場のせじょうをするのは、わたしの主人の役目です。	
\\	私が働きたいと思う唯一の工場は、クッキー工場です。	
\\	工, 場	
\\	名所	
\\	めいしょ	
\\	ポートランドのかん光名所を教えてください。	
\\	名所めぐりの日にちが近くなってきたら、またれんらくするね。	
\\	私は、スターバックスやデニーズ、それからニューヨーク郵便局など数々の名所を見るために、ニューヨークへ行きます。	
\\	名所 
\\	名, 所	
\\	対外	
\\	たいがい	
\\	せん後、日本の対外せいさくはどのようにかわっていったのでしょうか。	
\\	日本の対外せいさくについてのポッドキャストをレコーディングしています。	
\\	アメリカの対外債務が5000億ドルを突破した時に、俺の個人負債が500万ドルを突破したのは、単なる偶然なのか?	
\\	対, 外	
\\	楽しい	
\\	たのしい	い 
\\	このゲームは、とても楽しくて、おも白いです。	
\\	テレビを見ながら食べる方が、しゃべりながら食べるより楽しいよ。	
\\	飛行機から飛び降りるって、物凄く楽しいよ!	
\\	い 
\\	(たの) 
\\	楽	
\\	北海道	
\\	ほっかいどう	
\\	ぼくは、北海道で雪合せんをしたことがあります。	
\\	北海道についたどー!本もののすしを食べるどー!	
\\	北海道のお土産といえば、何が有名でしょうか。	
\\	ほく 
\\	北, 海, 道	
\\	馬	
\\	うま	
\\	あの馬に、ニンジンやリンゴをあげてもいいですか?	
\\	わたしの馬は、また一つこんなんをのりこえました。	
\\	お年寄りたちが次から次へと馬から落ちたの、めちゃくちゃ面白いと思うんだけど〜!	
\\	(うま) 
\\	馬	
\\	馬力	
\\	ばりき	
\\	自どう車の馬力を上げるには、どうすればいいですか。	
\\	アサヒのじいちゃんって、本当に馬力があるよね。	
\\	約四年間いじくり回したおかげで、今俺の芝刈り機は450馬力もあるんだぜ。	
\\	力 
\\	りき 
\\	馬, 力	
\\	人間	
\\	にんげん	
\\	さっきカフェで、人間のようにコーヒーをのんでるしば犬を見かけたよ。	
\\	わたしのすきなしは、あい田みつをの、「つまづいたって いいじゃないか にんげんだもの」というしです。人間というかん字がひらがなでかかれているところがすきです。	
\\	人間はいつも人生の目的について悩んでいる。でも、私はその答えを見つけた気がします。
\\	で勉強することです!	
\\	人 
\\	間 
\\	(げん) 
\\	人, 間	
\\	大役	
\\	たいやく	
\\	この大役をしっかりはたせるよう、一生けんめいがんばります!	
\\	ずい分とまた大役を買って出たものだね。	
\\	ロード・オブ・ザ・リングでフロドは大役を任されたが、靴も履かずに見事にやってのけた。	
\\	大, 役	
\\	間	
\\	あいだ, ま	
\\	たいてい、ねている間は、だんぼうを切ります。	
\\	その間に、いそいでひるごはんをかきこみました。	
\\	筏を作っている間に、俺達みんな餓死にしちまうよ!泳いで川を渡った方が無難なんじゃないか?	
\\	〜間 
\\	〜間 
\\	(あいだ). 
\\	(あいだ) 
\\	間	
\\	究明	
\\	きゅうめい	
\\	する 
\\	お前は本当にしんそうを究明するつもりがあるのか?	
\\	コウイチのおかしな言どうのしんのどうきはまだ究明されていない。	
\\	私が大学で取っている授業の一つで、現在「鳥が先か卵が先か」について究明しているんです。	
\\	究, 明	
\\	研究	
\\	けんきゅう	
\\	する 
\\	研究にねっ中しすぎて、気がついたら12時をすぎていました。	
\\	わたしは、トーフグの研究者で、バイオ医やくひんのせいぞう研究にたずさわっています。	
\\	現在、我々の巨人に関する研究に参加を希望する申込者を受付けている。	
\\	研, 究	
\\	この前	
\\	このまえ	
\\	の 
\\	そういえば、この前、しょうてんがいでさくら子ちゃんとバッタリ会ったよ。	
\\	この前、日本ごの曲の
\\	を買ったんだよね。	
\\	この前、思わずノートをとるぐらい完璧な日没を見たのはいつだったかな?	
\\	"この 
\\	前 
\\	前 
\\	前. 
\\	この 
\\	前.	前	
\\	前回	
\\	ぜんかい	
\\	前回のじゅぎょうでは、カエルのかいぼうをしました。	
\\	それでは、前回のつづきからはじめたいと思います。	
\\	前回のレッスンの復習は家でちゃんとやってきましたか?	
\\	前, 回	
\\	長さ	
\\	ながさ	
\\	おかしいなー、ちゃんと長さをはかったつもりなんだけどなー。	
\\	こちらのヘビは3センチたんいの長さで切り売りはん売しています。	
\\	僕がよろめいてしまった主な原因は橋の長さです。僕にはちょっと長すぎました。	
\\	長い 
\\	大きい 
\\	大きさ, 
\\	大きさ 
\\	大きい 
\\	長い 
\\	長さ 
\\	長い 
\\	さ 
\\	長	
\\	生地	
\\	きじ	
\\	マリメッコの生地でクッションを作りました。	
\\	このカーテンは、パラシュート生地でできています。	
\\	冷凍のパイ生地を買ってきてください。	
\\	生 
\\	地. 
\\	(き)
\\	(じ)! 
\\	生, 地	
\\	数学	
\\	すうがく	
\\	川下さんは、石田中学校で数学の先生をしています。	
\\	数学のしけん、明日だっていうのに、海人はホントききかんがねーよな。	
\\	数学さんよ、どうか大人になって自分の問題は自分で解決してくれよな。	
\\	数, 学	
\\	医学	
\\	いがく	
\\	さい近は、医学の本を月に三さつぐらいよんでいます。	
\\	わたしの兄は、医学ぶを目ざしています。	
\\	医学はいつも最新の薬を発見し続けている。昨日は、医者が皿洗いをするようになる薬を私にくれたよ。	
\\	医, 学	
\\	医大	
\\	いだい	
\\	わたしは、医大を目ざしてべん強しています。	
\\	医者のたまごをかれ氏にするため、医大生のふりをして、医大にせん入してみました。	
\\	私は、ポケモンコレクションを売って、医大の学費を払った。	
\\	大学 
\\	大 
\\	医, 大	
\\	地中海	
\\	ちちゅうかい	
\\	さい近、地中海りょうりにハマっているんですよね。	
\\	ただ今、地中海からきたく中でーす。	
\\	夏休みに、家族で地中海をクルーズしてきました。	
\\	地, 中, 海	
\\	〜道	
\\	どう	
\\	先生が、じゅう道のルールややり方を教えます。	
\\	北海道で、けん道をきわめるつもりです。	
\\	武士道と騎士道の違いについて説明してください。	
\\	どう 
\\	道, 
\\	道	
\\	君主国	
\\	くんしゅこく	
\\	天のうのいる日本は君主国ですか、それとも民主国ですか?	
\\	やっとこの君主国にもどって来たぜ!	
\\	イギリス君主国は、別のラップ音楽のビデオを作る締め切りをまた大幅に過ぎている。	
\\	君, 主, 国	
\\	朝	
\\	あさ	
\\	朝のあいさつをしてください。	
\\	朝は毎日、トロトロの半じゅく目玉やきを食べます。	
\\	明日の朝は早起きをしなくてはいけません。だって、
\\	をプレイしまくるという長い一日が待っているから。	
\\	朝	
\\	朝日	
\\	あさひ	
\\	明日の朝、一しょに山にのぼって朝日を見ませんか?	
\\	どうしてあの朝日の絵をかべから外してしまったんですか?	
\\	日本は「日出ずる国」として知られているが、それは「朝日国旗」に見事に描写されている。	
\\	朝 (あさ) 
\\	日 (ひ). 
\\	あさひ!	朝, 日	
\\	音楽	
\\	おんがく	
\\	べん強してから音楽をききます。	
\\	音楽をききながら仕ごとをすることはよくないと思いますか?	
\\	違ったタイプの音楽が創りだされることはもう無いのかなって思ったりするんだよね。あなたは、人類は既に全ジャンルの音楽を全て創りだしちゃったと思う?	
\\	不音楽 
\\	楽 
\\	(がく) 
\\	音, 楽	
\\	名前	
\\	なまえ	
\\	あの人の名前は何ですか。	
\\	名前はきいたことがあるけど、あまりよく知らないかな。	
\\	私は名前を「ああ」に、名字を「そうそう」に変えたい。	
\\	名 (な) 
\\	前 (まえ) 
\\	なまえ.	名, 前	
\\	当たり前	
\\	あたりまえ	
\\	な 
\\	の 
\\	こんな当たり前のこと、どうして分からないの?	
\\	でも、これが当たり前のことだとは思わないでくださいね。	
\\	今日のような当たり前の日をすごせることに、かんしゃしなきゃいけませんよね。	
\\	当たり 
\\	当, 前	
\\	お知らせ	
\\	おしらせ	
\\	する 
\\	あのう、お知らせしたいことがあるんですが、今、いいですか?	
\\	学校からのお知らせはもうよみましたか?	
\\	残念なお知らせがあります。昨夜誰かがこの家に侵入し、冷凍庫に入っていたアイスクリームを全部盗んでしまいました。	
\\	知 
\\	らせ 
\\	知る? 
\\	(し). 
\\	知	
\\	数	
\\	かず	
\\	おかしいわ、どうしても数が合わないのよ。	
\\	1985年におきたニューヨーク大地しんの死しゃの数は何人でしたか?	
\\	この映画には、数々の日本映画のお兄ちゃん役で知られているコウイチが、数を数えるのが得意な数学教師の役で出演しています。	
\\	(かず). 
\\	数	
\\	番号	
\\	ばんごう	
\\	あなたのじゅけん番号は何番ですか?	
\\	三十分で、この番号をおぼえてください。	
\\	今、めちゃくちゃ落ち込んでるんだよ。だって、彼女に偽の電話番号を渡されたんだよ。	
\\	番, 号	
\\	一番	
\\	いちばん	
\\	イチゴあじのあめが、この中で一番すきです。	
\\	わたしが人生で一番さいしょによんだ教かしょは、テキストフグです。	
\\	シドニー・クロスビーは、世界で一番うまいホッケー選手だ。	
\\	一, 番	
\\	二番	
\\	にばん	
\\	き末しけんで、学年で二番になりました。	
\\	これは二番目に古いゴジラえいがです。	
\\	カルロス・スリムは、世界大富豪番付けで二番に順位を下げた。	
\\	一番, 
\\	二, 番	
\\	数字	
\\	すうじ	
\\	数字がのびるからといって、下ひんなきじをけいさいするのはよした方がいい。	
\\	この子は、一万五千六百七十二けたの数字をあん記している天才じです。	
\\	数字の十一は、私がバレーボールをする時にいつも身に付ける背番号です。	
\\	数, 字	
\\	次々	
\\	つぎつぎ	
\\	しんじられないような出来ごとが次々とおこったんです。	
\\	コウイチかんとくは、さく夜、
\\	で次々とヒット作を生み出すゆめを見ました。	
\\	ヨーロッパの街は、次々と疫病に侵されていった。	
\\	次 
\\	つぎ), 
\\	次, 々	
\\	反対	
\\	はんたい	
\\	する 
\\	な 
\\	の 
\\	その考えには反対です。	
\\	不自由の男神は、アッパーわんをはさんで、自由の女神の反対がわに立っています。	
\\	チェンジビルに引っ越してからというもの、あいつは昔のあいつとは正反対の人間になっちまったんだ。	
\\	反, 対	
\\	向こう	
\\	むこう	
\\	向こうのパンやで、パンを買いましょう。	
\\	一花は、向こうの世界からやって来たのかもしれない。	
\\	もし君が道路の向こう側に立つのであれば、私の秘密を君に打ち明けるよ。	
\\	向 
\\	こういち.
\\	こういち 
\\	(む). 
\\	こういち 
\\	向く 
\\	向	
\\	絵	
\\	え	
\\	この絵は、クリックでかく大します。	
\\	この絵をかべにかけ直してくれませんか?	
\\	昨夜美術館からピカソの絵が五枚も盗まれたが、誰も気にしていないようだ。	
\\	絵	
\\	未決	
\\	みけつ	
\\	の 
\\	しょうさいは未決です。	
\\	未決しょるいは、コウイチのデスクの上の赤いトレイに入れてください。	
\\	彼女の裁判はもう十三年近くも未決となっています。そんなに待たなくてはいけないなんて、とても不公平です。	
\\	未, 決	
\\	私自身	
\\	わたしじしん, わたくしじしん	
\\	私自身は、その動物園の問題にはあまり関心がないんです。	
\\	私自身は、そんなにきょくたんなことじゃなければ、どんな食生活でも良いと思っています。	
\\	あなたがもし私自身よりも私のことを知っていると思っているんだとしたら、 それはとんだ大間違いよ。	
\\	私 
\\	わたし 
\\	私, 自, 身	
\\	投手	
\\	とうしゅ	
\\	そのダルビッシュ投手の人形はいくらですか?	
\\	二丁目には、田中投手が住んでいます。	
\\	今日の先発投手は誰になると思いますか?	
\\	手 
\\	(しゅ). 
\\	投, 手	
\\	所	
\\	ところ	
\\	分からない所があれば、手を上げてください。	
\\	まだおとし所をさがしていると中です。	
\\	七十二回も同じ所を歩いているので、私達は道に迷ったんだと思います。	
\\	(ところ). 
\\	所	
\\	場所	
\\	ばしょ	
\\	いつもは教室にあつまりますが、明日は場所がちがうので、まちがえないでくださいね。	
\\	自分がたたかうべき場所をきちんと知っておくべきだ。	
\\	この場所は私が初めてローズに会った場所です。でも、あの日どうして靴を脱いだのかは思い出せません。	
\\	場 
\\	(ば) 
\\	場, 所	
\\	点数	
\\	てんすう	
\\	ワニカニに百点まん点で点数をつけるなら、何点になると思いますか?	
\\	この点数は、こないだくれたプレゼントのおかえしだよ。	
\\	「どうしてそのテストでこんなに低い点数をとったの?」「欠席のせい。つまり、僕の隣の席の子が休んでたんだよ。」	
\\	点, 数	
\\	交番	
\\	こうばん	
\\	交番で、大しかんまでの道をききました。	
\\	新しく入って来たヒマリちゃんは、この交番のアイドルです。	
\\	すみません、お巡りさん。一番近い交番はどこですか?ある男が私のことをデブと言ったので、警察を呼ぶ必要があるんです。	
\\	交, 番	
\\	池	
\\	いけ	
\\	この池は大きすぎる。	
\\	池に竹の矢をおとしたら、中からかみさまがあらわれて、「おまえがおとしたのはこの竹の矢か、それともこの金の矢か。」ときかれたので、正じきに「竹の矢ですが、その金の矢もほしいです。」といった。	
\\	うちのご近所さんは、鳥を撃ち落としては、うちの池に投げ入れる。	
\\	(いけ)! 
\\	池	
\\	私立大学	
\\	しりつだいがく	
\\	私立大学は公立大学よりじゅ業料が高いです。	
\\	わせ田大学は日本の有名な私立大学です。	
\\	あの出火した私立大学は、私が通っている学校だよ。	
\\	私立, 
\\	大学, 
\\	私立 
\\	大学 
\\	私, 立, 大, 学	
\\	番組	
\\	ばんぐみ	
\\	この番組はおも白くありませんでしたが、わたしのすきなはいゆうが出ていました。	
\\	きのうの番組で、あの二人はついに両しゃがなっとくするだきょうあんにたどりついたよ。	
\\	最近はバラエティ番組しか見ません。	
\\	番 
\\	組. 
\\	番, 組	
\\	役人	
\\	やくにん	
\\	せいじ家や役人には下手にさからわない方がいい。	
\\	コウイチはトーフグの社長であるだけでなく、文ぶ科学しょうの役人でもある。	
\\	私の街の役人の一人が、カメラの前で嘘をついたということで、調べを受けている。	
\\	役, 人	
\\	住人	
\\	じゅうにん	
\\	こんなしん夜に大声でけんかをしたら、近りんの住人にめいわくだろう。	
\\	話し合いの末、森の住人とみずうみの住人は、一しょに百本のけんを買うことにした。	
\\	私は地球の住人です。パーティーにようこそ。	
\\	住, 人	
\\	道	
\\	みち	
\\	男の人が、交番で道をたずねています。	
\\	ほら、千里の道も一歩からって言うでしょ?	
\\	この道を車で20分ぐらい行けば、未確認動物ビッグフットが目撃された場所に辿り着くよ。	
\\	〜道, 
\\	(みち). 
\\	道	
\\	役	
\\	やく	
\\	コウイチが、クママン役でドラマに出えんすることになりました。	
\\	コウイチは役作りのために、一年間ホームレスになってみました。	
\\	マイケル、めちゃくちゃ役になりきってるじゃん。	
\\	役	
\\	学者	
\\	がくしゃ	
\\	の 
\\	かれのお父さんは、立ぱな学者です。	
\\	あの学者はけいれきさしょうをしています。	
\\	私は世界でも有名なハロー・キティ学者です。	
\\	学, 者	
\\	数年	
\\	すうねん	
\\	数年がすぎさった。	
\\	数年前まで、日本に住んでいました。	
\\	この絵が完成するには、あと数年かかります。	
\\	数, 年	
\\	歩道	
\\	ほどう	
\\	どうして自てん車が歩道を走ってはいけないんですか?	
\\	まもなくうごく歩道がおわりますので、足元にごちゅういください。	
\\	自転車は、忙しい車道じゃなく、歩道を走るべきだ。	
\\	歩, 道	
\\	間もなく	
\\	まもなく	
\\	間もなくかいえんです!	
\\	大学いん研究生のしょう学金についてのへんじが、間もなく来るはずです。	
\\	もう間もなく夕飯が出来上がる頃ですが、クッキーを食べる手が止まりません。	
\\	もなく 
\\	あいだ), 
\\	ま. 
\\	""間もなく, 
\\	間	
\\	出所	
\\	しゅっしょ, でどころ	
\\	する 
\\	ジェニーの出所はいつですか?	
\\	出所したら、ちゃんとかしをかえしてくれよな。	
\\	引用部分の出所を明示してください。	
\\	しゅつ 
\\	しゅっ. 
\\	でどころ 
\\	出どころ 
\\	出, 所	
\\	南米	
\\	なんべい	
\\	学生時代に、バックパッカーとして南米をりょ行しました。	
\\	このホテルの水道水は、南米で今までのんだ水の中で一番おいしい。	
\\	ごめんなさい。南米はアメリカの一部だと思っていました。	
\\	米国 
\\	南, 米	
\\	南口	
\\	みなみぐち	
\\	そこに行くなら、南口から出るのが一番近いです。	
\\	このモールの南口で今
\\	が見られるよ!	
\\	新宿南口でナンパされた。	
\\	入り口 
\\	口 
\\	南 
\\	口, 
\\	くち 
\\	ぐち.	南, 口	
\\	空間	
\\	くうかん	
\\	あの人と同じ空間にいるだけで心がいやされるんです。	
\\	日本人の中には、ベビーカーは空間をたくさんとるので、まんいん電車ではきん止にするべきだとかんがえる人もいます。	
\\	宇宙空間には空気がないけど、それでもやっぱり月に住みたい。	
\\	空, 間	
\\	科目	
\\	かもく	
\\	とくいな科目とにが手な科目を教えてください。	
\\	多分、その科目、さいじゅこうしないといけないと思う。	
\\	好きな科目は何ですか?	
\\	科, 目	
\\	自決	
\\	じけつ	
\\	する 
\\	あのさむらいは自決した。	
\\	コウイチ社長は、トーフグは自決けんを強く心からしじするということを、何どもくりかえした。	
\\	彼は、獄中で絶望し、自決してしまったんです。	
\\	自, 決	
\\	電話	
\\	でんわ	
\\	する 
\\	の 
\\	電話で家ぞくと話しています。	
\\	夜おそくに電話してごめんね。	
\\	アレクサンダー・グラハム・ベルが自分の電話の着信音を何にしていたのかが気になります。	
\\	電, 話	
\\	身体	
\\	しんたい, からだ	
\\	の 
\\	明日は、学校で身体そくていがあります。	
\\	この図しょかんには、身体しょうがいしゃがつかえるトイレはありますか?	
\\	空こうで、まやくをかくしもっていないか、身体けんさをされました。	
\\	身, 体	
\\	人数	
\\	にんずう	
\\	カラオケに行く人数は全ぶで何人ですか。	
\\	トーフグのあたらしいライターのぼしゅう人数は何人ですか?	
\\	沢山の人数の人が、この国の人は同じズボンを履くべきだという私の提案に反対しました。	
\\	人数, 
\\	すう 
\\	ずう.	人, 数	
\\	時間	
\\	じかん	
\\	先生はいつ時間がありますか。	
\\	今日の午後でしんさつしてもらえる時間ってありますか?	
\\	スーパーマンは、アクアマンに、最も真剣な質問をした。「もう一度、ウォータースライダー に乗る時間、ある?」	
\\	時間?	
\\	時, 間	
\\	答える	
\\	こたえる	
\\	このもんだいに、じしょをつかって答えてください。	
\\	どうしてしつもんに答えられないの?	
\\	面接官に今までに嘘をついたことがあるかどうか聞かれた時に、何て答えればいいのか分からなかったよ。	
\\	う 
\\	答	
\\	曲がる	
\\	まがる	
\\	三つ目のしんごうを、右に曲がってください。	
\\	そのカレー色のネクタイ、ちょっと曲がってるよ。	
\\	野球のボールが当たってから、僕の小指はちょっとだけ左に曲がっています。	
\\	う 
\\	曲	
\\	出来上がる	
\\	できあがる	
\\	おいしいカレーが出来上がりました。	
\\	日本りょ行のけいかくは、ギリギリになってようやく出来上がりました。	
\\	豆腐と河豚のハンバーガーが出来上がりました。	
\\	出来る 
\\	上がる 
\\	出来る 
\\	上がる. 
\\	出, 来, 上	
\\	交じる	
\\	まじる	
\\	空と海が交じるところが、世かいのはてなのさ。	
\\	きたいと不安が交じった気もちです。	
\\	わたしのさいふをすった後、そのスリはす早く人ごみに交じってきえました。	
\\	交ぜる 
\\	じ 
\\	交じる 
\\	交ぜる. 
\\	ま 
\\	交	
\\	曲げる	
\\	まげる	
\\	次は、右足のひざを90ど曲げてください。	
\\	この
\\	ビデオでは、コウイチがスプーンを百本ひたすら曲げつづけます。	
\\	スーパーマンはなんでも曲げられます。すでに曲がっているものでさえも。	
\\	う 
\\	げ 
\\	(ま). 
\\	曲	
\\	役に立つ	
\\	やくにたつ	
\\	日本ごのべん強の役に立つサイトを見つけました。	
\\	ハンガリーには、にげるははじだが役に立つ、ということわざがあります。	
\\	ゾンビがいっぱいの町を歩くなら、銃よりも剣の方が役に立つだろうか?	
\\	役 
\\	立つ 
\\	立つ 
\\	役, 立	
\\	買う	
\\	かう	
\\	トイレットペーパーと肉とやさいとしょうゆを同じ日に買いたいです。	
\\	どうしてそのスマホを買うことにしたんですか?	
\\	コウイチのボブルヘッド人形を買うために、長い行列ができていた。	
\\	う 
\\	買	
\\	対する	
\\	たいする	する 
\\	たまごの白身の黄身に対するわり合は、2対1です。	
\\	お前って、両しんに対する思いやりに欠けてるよね。	
\\	あいつらの関係が修復不可能だってことは明らかだね。だって、スタートレックに対する意見が全く違っているんだから。	
\\	(する) 
\\	対	
\\	数える	
\\	かぞえる	
\\	ねられないとき、羊を数えるのはなぜですか。	
\\	いのちがおしけりゃ、十数えるうちにきえろ。	
\\	スミス先生、生徒が二人足りません。数えましたが38人しかいないんです。	
\\	う 
\\	(かぞ) 
\\	数	
\\	決める	
\\	きめる	
\\	大学をそつぎょうしたら何をするか、もう決めましたか。	
\\	おなかが空きすぎてて何食べるか決められないから、よかったらユイちゃんが決めちゃって。	
\\	残りの人生をどう過ごすか、みんなは一体どうやって決めるんだい?選びきれないほど仕事があるじゃないか。	
\\	う 
\\	(き) 
\\	決	
\\	近づく	
\\	ちかづく	
\\	カメラマンが近づくとカメはにげてしまった。	
\\	おれの女に近づくな!	
\\	私達に今近づいてきているゾンビ、めちゃくちゃ私のタイプなんだけど。	
\\	う 
\\	近い. 
\\	近	
\\	当てる	
\\	あてる	
\\	この貝がら、耳に当てるとなみの音がするの。	
\\	ひる休みに、だれがうそをついているのかを当てるゲームをしました。	
\\	ボールをバットに上手く当てたと言いたかったんだけど、間違えてお尻に当てたと言ってしまった。	
\\	当 
\\	当たる 
\\	あてる. 
\\	あ.	当	
\\	助ける	
\\	たすける	
\\	ワニカニとトーフグがおぼれていたらどちらを助ける?	
\\	このかわいそうなすて犬ちゃんを助けてあげることができる人はいませんか?	
\\	チリチーズブリトーを食べるのを助けて欲しい人はいますか?私はそれがとても得意ですよ!	
\\	う 
\\	助かる 
\\	助ける 
\\	(ける).
\\	(たす) 
\\	助	
\\	交わる	
\\	まじわる	
\\	オレンジとあい色が交わる、夕ぐれの空が大すきなの。	
\\	このオフィスでは、時々、日本とアメリカの文化が交わります。	
\\	婆ちゃんの乳はとても長いので、たまに相互に交わることすらできる。	
\\	交ぜる 
\\	わ 
\\	交ぜる 
\\	ま 
\\	じ 
\\	じ 
\\	(まじ) 
\\	交	
\\	住む	
\\	すむ	
\\	わたしは、父と母と一しょに住んでいます。	
\\	青木ヶ原のじゅ海は、住むにはおそろしすぎる場所です。	
\\	ホビットの穴に住んでみたいのですが、どうやって掘ればいいのか分かりません。	
\\	う 
\\	(す) 
\\	住	
\\	明日	
\\	あした, あす, みょうにち	
\\	明日、びょういんへ行きましょう。	
\\	明日は明日の風がふく。	
\\	一ヶ月イタリアに行くことを知らせるために、明日学校に電話しなくてはいけません。	
\\	あす 
\\	みょうにち: あす 
\\	みょうにち 
\\	あした 
\\	明, 日	
\\	見直す	
\\	みなおす	
\\	もう一ど、けいかくを見直しましょう。	
\\	この間のはいひん回しゅうの日、主人が思ったよりたよりになって、ちょっと見直しちゃった。	
\\	大学の授業の選択を見直す方がいいと思うよ。だって、愚鳩を捕まえるのが、いいキャリアになるとは思えないもん。	
\\	直す 
\\	見, 直	
\\	思い出す	
\\	おもいだす	
\\	わたしは、家を出る時、いつも家の前でひかれた子犬のことを思い出します。	
\\	四けたのあんしょう番号が思い出せなくて、めっちゃあせったよ。	
\\	人に思い出してもらう一番の方法は、その人たちからお金を借りることだ。	
\\	思う 
\\	思い 
\\	出す 
\\	思う 
\\	出す 
\\	思, 出	
\\	走り回る	
\\	はしりまわる	
\\	ニッカはいつもオフィス内を走り回っています。	
\\	うちの子なら、あの交通じこからもうすっかり立ち直って、元気に走り回っていますよ。	
\\	コウイチが蜂の巣を撃った後、蜂から逃げるために約十分間走り回った。	
\\	走る 
\\	回る 
\\	走る 
\\	回る. 
\\	走, 回	
\\	話す	
\\	はなす	
\\	女の人と男の人が写しんを見ながら話しています。	
\\	早く日本ごがうまく話せるようになりたいです。	
\\	あなたが話すのを聞いていると、なぜかいつも、紐無しで橋からバンジー・ジャンプをしているような気になります。	
\\	う 
\\	花
\\	(はな) 
\\	話	
\\	支える	
\\	ささえる	
\\	やっぱり、おれのしゅう入だけで、一家を支えるのはきびしいよ。	
\\	アメリカのけいざいは、トーフグによって支えられている。	
\\	壁に寄りかかって体を支えるのがとても楽しかった。	
\\	う 
\\	(ささ 
\\	支	
\\	投げる	
\\	なげる	
\\	このあたらしいボールは、とても投げやすいです。	
\\	どうしてあの人たちは、かんそうした大ずを投げているんですか?	
\\	ピッチャーがボールを投げたんですけど、ちょうどバッターの前を飛んでいた鳩に当たったんですよね。	
\\	う 
\\	(な) 
\\	投	
\\	化ける	
\\	ばける	
\\	かの女は、化しょうをすれば、び人に化けることで有名です。	
\\	あのようかいは、人間に化けることができる。	
\\	私の家はリフォームの後で全く違う家に化けてしまったので、息子は学校が終わった後にどの家に帰るのか分からなくなってしまいました。	
\\	う 
\\	化	
\\	回す	
\\	まわす	
\\	きのうは一日中、人さしゆびでピザを回すれんしゅうをしていました。	
\\	ドアノブを回す音で、赤ちゃんの目がさめてしまいました。	
\\	私がこの一ドル硬貨を十分速く回したら、もしかすると百ドル札になるかもしれない。	
\\	回る, 
\\	す 
\\	回る (まわ). 
\\	回	
\\	点ける	
\\	つける	
\\	電気を点けてくれ。	
\\	ラジオの点け方が分かりません。	
\\	テレビを点けてもらえますか。	
\\	う 
\\	(つ). 
\\	点	
\\	向ける	
\\	むける	
\\	おれはとつぜんカメラを向けられるのが大っきらいなんだよ!	
\\	スマホじゃなくて子どもに、目を向けてあげてください。	
\\	その男はゆっくりと俺に銃口を向けた。	
\\	向く 
\\	け. 
\\	向く 
\\	向	
\\	電車	
\\	でんしゃ	
\\	下のひょうは、「電車の時間」と「バスの時間」です。	
\\	この電車にのって、トトロえきでおりて、ネコバスにのりかえてください。	
\\	ホグワーツ特急は電車じゃなくて汽車です。	
\\	電, 車	
\\	水道	
\\	すいどう	
\\	この水道の水は、のんでも平気ですよ。	
\\	水道代がはらえなかったので、水道が止められてしまいました。	
\\	水道が凍結してしまいました。	
\\	水, 道	
\\	全身	
\\	ぜんしん	
\\	の 
\\	わたし、全身毛むくじゃらのワイルドな男せいがすきなんです。	
\\	コウイチは全身にタトゥーを入れようと思っているみたいなんだけど、どう思う?	
\\	昨日七ヶ月ぶりにランニングしたから、今日は全身筋肉痛だよ。	
\\	全, 身	
\\	森	
\\	もり	
\\	土よう日に、森でピクニックをしました。	
\\	お母さんは木を見て森を見ずなんだよ!	
\\	赤ずきんちゃんが森の中を歩いていると、死んだ狼に出くわしました。赤ずきんちゃんはその場を立ち去り、訪れたお婆さんの家で素晴らしい時間を過ごしました。	
\\	森	
\\	お前	
\\	おまえ	
\\	の 
\\	一生お前といっしょに生きていきたいんだ。	
\\	女のことお前ってよぶ男本当にきらいだわ。	
\\	お前の作る餃子は本当に美味しいな。	
\\	お 
\\	前 
\\	前. 
\\	前	
\\	君	
\\	きみ	
\\	君とぼくは、いつもいっしょだよ。	
\\	君はけっこうせがひくいから、シャワーヘッドをフックから外せるか心ぱいです。	
\\	君は、エレベーターで一緒に閉じ込められたくない人ナンバーワンだ。	
\\	(きみ)! 
\\	君	
\\	全部	
\\	ぜんぶ	
\\	全部ですか。	
\\	わたしのベーコンクッキーを一はこ全部食べちゃったのはだれ!?	
\\	この靴下、全部穴があいていますよ。	
\\	全, 部	
\\	天使	
\\	てんし	
\\	ニッカちゃん、マジ天使だわー。	
\\	あの天使と、少しきょりをおきたいと思っています。	
\\	天使のように可愛い女の子に会った。	
\\	天, 使	
\\	角度	
\\	かくど	
\\	の 
\\	べつの角度から、もう一度見てみましょう。	
\\	ピサのしゃとうはたしかにへんな角度で立っている。	
\\	もしマイリー・サイラスが一番好きな歌を歌っているのをこの角度から見ることができたら、俺の夢を叶えてくれたってことで神様の存在を信じると思う。	
\\	角, 度	
\\	発表	
\\	はっぴょう	
\\	する 
\\	今夜、コウイチから、重大な発表があります。	
\\	ピカチュウは、今朝、ポケモンから引たいすることを発表した。	
\\	明日、みんなの前で
\\	の研究成果について発表をしなければならないので、緊張しています。	
\\	ひょう 
\\	ぴょう 
\\	はつ 
\\	発, 表	
\\	記事	
\\	きじ	
\\	わたしは、アメリカの教いくについて、記事をかきました。	
\\	クリスティンの記事、今月号のナショナル・ジオグラフィックで見たよ!	
\\	この記事はデタラメだ。	
\\	記, 事	
\\	仮定	
\\	かてい	
\\	する 
\\	でも、それはたんなる仮定の話でしょう。	
\\	たとえば、一ドル百円と仮定すると、一本一ドルのコカコーラは百円になりますよね?	
\\	彼女のベッドに積まれた札束のせいで、今とんでもない仮定が俺の頭の中を飛び交っている。	
\\	仮, 定	
\\	通り	
\\	とおり	
\\	この通りをまっすぐに行って、三つ目のしん号を左に曲がってください。	
\\	この通りでは、一日中車が行き交っています。	
\\	チーズベーコンバーガー通りには、ガソリンスタンドってあったっけ?	
\\	通る 
\\	通る, 
\\	お 
\\	通	
\\	美しい	
\\	うつくしい	い 
\\	美しいスカーフ、ありがとうございます。大切にします。	
\\	クリステン王女の赤いかみ、ほんと、美しすぎてなみだが出る。	
\\	なんてこった、君の目は宝石のように美しいよ、クリスティーン。	
\\	い 
\\	(うつくしい). 
\\	うつくしいしい 
\\	美	
\\	美人	
\\	びじん	
\\	あなたのおくさんは、美人だし、とても上ひんですね。	
\\	物は言いようなんだから、馬か正直にならすに、美人ですねって言っておけばよかったんだよ。	
\\	彼は誰の忠告も絶対に受けないなんて言っていたが、美人の忠告なら話は別だろう。	
\\	美, 人	
\\	気を付けて	
\\	きをつけて	
\\	あぶらを使ったちょうりをするさいは、火事にならないよう、気をつけてください。	
\\	このマグカップ、あついから、気を付けてね。	
\\	これから雪を食べる時はもっと気を付けてよ。色盲だから、色が黄色でも分かんないんだから。	
\\	気 
\\	付ける 
\\	気 
\\	付ける. 
\\	気, 付	
\\	苦い	
\\	にがい	い 
\\	このベーコン、なんか苦い。	
\\	このくすりはすっごく苦いから、気をつけてね。	
\\	カラメルは、焦がさないように気を付けてください。でないと、苦い味になってしまいます。	
\\	苦 
\\	(にが). 
\\	苦	
\\	白黒	
\\	しろくろ	
\\	の 
\\	白黒ハッキリさせようぜ。	
\\	この白黒の写しん、すてきですね。	
\\	カナエちゃんがいつもオシャレな服装をしているってことは認めるけどさ、今日の白黒のドレスはちょっとなんかパンダみたいじゃない?	
\\	白 
\\	黒い 
\\	白, 黒	
\\	体重	
\\	たいじゅう	
\\	アンナさんの体重は、何キロですか?	
\\	次の身体そく定で体重をはかるのが待ちきれない。	
\\	どうして私があなたの体重を知っているのか聞かないで。ジェシカが口を滑らせたことは絶対に誰にも言わないって、あの子と約束したんだから。	
\\	体, 重	
\\	本屋	
\\	ほんや	
\\	よくその本屋に行きます。	
\\	この本屋には、ツアーガイドブックがたく山おいてあります。	
\\	本屋のレジでアルバイトをしています。	
\\	本, 屋	
\\	肉屋	
\\	にくや	
\\	あそこの肉屋で、合いびき肉を100
\\	買ってきてちょうだい。	
\\	あそこのお肉屋さん、新しいあじのビーガンとうふバーガーをかいはつしてるらしいよ。	
\\	私の小さな町には「お肉屋さん」とい名前のレストランがあったんだけど、本当の「お肉屋さん」がどんなものかを知った時にはとてもショックでした。それで、今、私はベジタリアンなんです。	
\\	肉 
\\	にく 
\\	屋 
\\	肉, 屋	
\\	服	
\\	ふく	
\\	服を引き出しの中に入れてください。	
\\	かわいい!おそろいの服をきてるんですね。	
\\	今日はチョー暑くて、あなたはあまりにも多くの服を着ているよ。	
\\	服	
\\	部室	
\\	ぶしつ	
\\	そのまんがなら、部室に全かんそろってるよ。	
\\	何ぜか部室に魚を口にくわえたネコがいるんです。	
\\	部室に忘れ物をしちゃった。	
\\	部, 室	
\\	工事	
\\	こうじ	
\\	する 
\\	きのうてつ夜で仕事だったんですが、朝から工事の音がうるさくてぜんぜんねむれてないんです。	
\\	町中で一日中工事しているので、通行止めだらけでいやになるよ。	
\\	この工事は今日終わったんですよね?	
\\	工, 事	
\\	相談	
\\	そうだん	
\\	する 
\\	相談の時間は、午後六時から午後九時です。	
\\	こ人的に相談したいことがあるんですが、ちょっとお時間よろしいですか?	
\\	私は医者に痔の相談をした。	
\\	相, 談	
\\	対談	
\\	たいだん	
\\	する 
\\	コウイチとスティーヴ・ブシェミのゆめの対談が実げんしました!	
\\	ざんねんながら、そろそろこの対談をおわらせる時間になってしまいました。	
\\	有名な作家と対談してみたい。	
\\	対, 談	
\\	決定	
\\	けってい	
\\	する 
\\	クママンまんがの、アニメ化が決定しました!	
\\	これはさいしゅう決定になりますので、その後の変更はできませんが、よろしいでしょうか?	
\\	君の妹の結婚式と俺の母さんの葬式、どちらに出席するかの決定は君に任せるよ。	
\\	けつ 
\\	決. 
\\	けっ.	決, 定	
\\	〜度	
\\	ど	
\\	弟は、かぜで38度のねつがあります。	
\\	エアコンのおん度はいつも何度ぐらいにしていますか?	
\\	24度ぐらいがちょうどいいと思うんだけど。	
\\	度	
\\	〜部	
\\	ぶ	
\\	大しきゅう、へんしゅう部の部長をよんでくれないか。	
\\	中学の時は、けん道部でした。	
\\	経理部の部長をしています。	
\\	部	
\\	出発	
\\	しゅっぱつ	
\\	する 
\\	出発まで五分です。	
\\	二週間後に、東京に出発する予定です。	
\\	ほとんど喋ったことのない同僚の出発の門出を祝う飲み会にお誘い頂き、どうも有難う御座います。	
\\	しゅつ 
\\	はつ 
\\	ぱつ. 
\\	出, 発	
\\	不自由	
\\	ふじゆう	
\\	な 
\\	何不自由ない生活をしてるくせに、ぜいたくだよ!	
\\	このせきは、お年よりや体の不自由な方のためのゆう先ざせきですよ。	
\\	私の兄は、足が不自由です。	
\\	自由 
\\	不, 自, 由	
\\	発見	
\\	はっけん	
\\	する 
\\	自たくのベランダで、イモムシを発見しました。	
\\	がんは、早き発見することがひじょうに重ようです。	
\\	でお金を稼ぐ方法を発見したぜ。しかも超簡単なの。まず、アカウント設定に行くだろ、それからアカウントを停止して、仕事に行くんだ!	
\\	はつ 
\\	はっ. 
\\	見 
\\	見, 
\\	けん, 
\\	けん 
\\	発, 見	
\\	二重	
\\	にじゅう	
\\	の 
\\	分からないかん字の下に、二重せんを引いてください。	
\\	日本は二重国せきをみとめていません。	
\\	「それでは、世間話はそれぐらいにして、本題に入ろう。今日の議題は何だったかな?」「今日はあなたの二重あごについて話しをすることになっています。」「そうだったな。それでは始めようか。」	
\\	二重 
\\	二, 重	
\\	毎度	
\\	まいど	
\\	毎度おさわがせして、申しわけありません。	
\\	毎度のことだが、空は今日もちゅう文を決めるのがおそい。	
\\	毎度有難う御座います。	
\\	毎, 度	
\\	楽勝	
\\	らくしょう	
\\	する 
\\	の 
\\	今回のテストは、楽勝だったね。	
\\	5はどうだった?」「ワニカニのおかげで、楽勝だったよ。」	
\\	戦争では、より高度な技術を持っている側が通常楽勝する。	
\\	楽, 勝	
\\	小学校	
\\	しょうがっこう	
\\	小学校で、小さな子どもたちに、色々な絵本をよんであげています。	
\\	あの小学校には、コウイチ気取りの六年生の男の子がいます。	
\\	小学校の頃から、君のことずっと気になっていたんだ。	
\\	学校 
\\	小 
\\	学校 
\\	小学校.	小, 学, 校	
\\	用事	
\\	ようじ	
\\	用事をおねがいしてもいいですか。	
\\	ごめんなさい!ちょっと用事が出来てしまって、今日のレッスンをキャンセルしないといけなくなってしまいました。	
\\	大事な用事があるので、明日は出勤することができません。	
\\	事 
\\	用, 事	
\\	重要	
\\	じゅうよう	
\\	な 
\\	重要なしょるいなので、なくさないようちゅういしてください。	
\\	この国にとって、国民一人当たり一羽の白鳥をかうことはとても重要なことなんです。	
\\	私にとって人生で一番重要なものはビールですって?そんなこと言ってないわよ。言いもしないことを言ったなんて言わないでよ。	
\\	重, 要	
\\	丁度	
\\	ちょうど	
\\	な 
\\	丁度おふろから出たところです。	
\\	丁度いいゆかげんです。	
\\	まあ、偶然!今丁度あなたに電話をしようと思っていたところよ!	
\\	丁, 度	
\\	試験	
\\	しけん	
\\	する 
\\	おっと!明日試験だっけ?忘れるところだったよ。	
\\	何度か試験してみたんですが、まだ一度も成功してないんですよ。	
\\	「ところで、試験はどうだった?」「パパ、その話はやめてよ。」	
\\	試, 験	
\\	体験	
\\	たいけん	
\\	する 
\\	いねかり体験教室への応募者は、五百人に達しました。	
\\	日本では戦争体験をした人の数が年々少なくなっています。	
\\	これは、コウイチ自身が日本へ留学した時の体験を元にした作品です。	
\\	体, 験	
\\	実験	
\\	じっけん	
\\	する 
\\	実験のために、車を一台借りました。	
\\	どんなことを言ったらビエトがおこるのか、実験してみた。	
\\	私達の新製品「フグ風味の豆腐」の実験台になりたい人はいますか?	
\\	実, 験	
\\	火事	
\\	かじ	
\\	火事があった場所は、アパートの五かいです。	
\\	何か臭うんだけど。もしかして、火事じゃない?	
\\	昨日、家が火事になる夢を見たんです。	
\\	火, 事	
\\	魚屋	
\\	さかなや	
\\	魚屋で金目だいを買ってきたの。	
\\	この魚屋のゆかはすべりやすいから気を付けてね。	
\\	あの魚屋さんは百円おまけをしてくれた。	
\\	魚 
\\	さかな) 
\\	屋 (や). 
\\	魚 
\\	魚, 屋	
\\	発売	
\\	はつばい	
\\	する 
\\	この夏、
\\	から、「ベーコンとおこのみやきのチーズタルト」が発売されることが決定しました!	
\\	コウイチのつめのあかをせんじたポートランド名物トーフグ茶がトーフグのウェブサイトででぜっさん発売中です!	
\\	のデート用抱きまくらが発売されるのが待ち遠しい。	
\\	発, 売	
\\	表	
\\	おもて	
\\	表はピンク色にぬってください。	
\\	コウイチの表のかおはトーフグ社長であるが、うらのかおは世界でも有数のジャグリングせん手である。	
\\	十円玉の表はどちらですか?	
\\	(おもて)
\\	表	
\\	家具	
\\	かぐ	
\\	今週末は、新きょの家具を買いに行く予定です。	
\\	王女の家具は本当にすばらしいですね!さすがです!	
\\	家具はあなたの家にお運びしましたが、もう少し腰を軽くして扉と窓を早く取り付ける方がいいのではないかと思います。	
\\	家, 具	
\\	お客さん	
\\	おきゃくさん	
\\	お客さんが入ってくると、このベルがなります。	
\\	東京えきで、お客さんと会いました。	
\\	お客さんを獲得するにはどうすればいいと思いますか?	
\\	さん 
\\	さん. 
\\	お客さん, 
\\	お客さま 
\\	さま 
\\	さん).
\\	客	
\\	客室	
\\	きゃくしつ	
\\	大すきなアイドルと、ろ天ぶろ付の客室にとまることがゆめなんです。	
\\	このホテルでは全ての客室にコンシェルジュサービスが付いてきます。	
\\	ちょうど客室のリフォームが終わったところなんですの。是非うちにいらして、ご意見を聞かせてくださいな。	
\\	客, 室	
\\	ハート形	
\\	はーとがた, ハートがた	
\\	ネイル用に、ハート形のストーンをさがしています。	
\\	クリスティンは、くぎをくわえながら、ハート形のハンマーをふり回していた。	
\\	フラミンゴがキスをする時、二羽の頭と首がハート形を描く。	
\\	"ハート 
\\	形 
\\	形 
\\	ち 
\\	がた. 
\\	形	
\\	重い	
\\	おもい	い 
\\	そのにもつは重そうですね。	
\\	この車、重すぎてうでがもげそう。	
\\	「足を踏んでしまってごめんなさい。」「ちょっと重かったけど、どうってことないよ。」	
\\	い 
\\	(おも) 
\\	重	
\\	高さ	
\\	たかさ	
\\	あいつのテンションの高さはやばいよな。	
\\	トーフグのサボテンは、1.5メートルの高さにまでせい長しました。	
\\	「この建物の高さについてどう思いますか?少し高すぎますかねえ?」「申し訳ありませんが、分かりません。私には難しすぎます。」	
\\	高い 
\\	大きい 
\\	大きさ, 
\\	高さ 
\\	高い 
\\	いくらですか? 
\\	高い (たかい) 
\\	高	
\\	高校生	
\\	こうこうせい	
\\	今日は、高校生たちが、このパン工場の見学に来ます。	
\\	わたしは高校生の時、毎日のように海に行っては魚をとって食べていました。	
\\	あいつはただの高校生デビューだよ。	
\\	高校 
\\	中学生 
\\	小学生 
\\	高校 
\\	高, 校, 生	
\\	家事	
\\	かじ	
\\	家事、ほんとにめんどくさいけど、がんばる。	
\\	主人はいつもわたしにばかり家事をおし付けてくるから本当にこまります。	
\\	家事全般が苦手なんですよね。	
\\	家, 事	
\\	何度	
\\	なんど	
\\	人生には本当に大切なしゅん間が何度かおとずれる。	
\\	トーフグは、おなかが空いたサメに食べられそうになったことが何度もある。	
\\	何度同じことを言えば分かるんだ?	
\\	何, 度	
\\	大事	
\\	だいじ	
\\	な 
\\	大事なのは、きちんとあいさつをすることです。	
\\	日本ごのべんきょうのために、毎日ワニカニにログインしてレビューをするのはとても大事なことです。	
\\	リラックスする時間をとることは、とても大事だよ。	
\\	大, 事	
\\	和風	
\\	わふう	
\\	和風のあじ付けがすきなんですよ。	
\\	かえりにスーパーで、和風ごまドレッシングを買ってきてください。	
\\	これが和風ホットドッグです。	
\\	和, 風	
\\	和服	
\\	わふく	
\\	和服もに合ってるね。	
\\	何ちゃく和服をもっていますか?	
\\	和服でマラソンをする日本人を見たことがありますか?	
\\	和服.	
\\	和, 服	
\\	和食	
\\	わしょく	
\\	朝ごはんは、和食がいいですね。	
\\	アメリカ人の中にもさい近、よう食よりも和食がすきな人がふえている。	
\\	和食イコール寿司ってわけじゃない。	
\\	和食?
\\	和, 食	
\\	和室	
\\	わしつ	
\\	和室でねるのは、これがはじめてです。	
\\	この和室、ドッグフードとコウイチの足のにおいがまざったようなにおいがしない?	
\\	私のお兄ちゃんは、和室で大きな蜘蛛を飼っています。	
\\	和室 
\\	和, 室	
\\	仮名	
\\	かな	
\\	まずは、仮名をかくれんしゅうから始めましょう。	
\\	ワニカニでかん字をべん強するのに、まずは仮名を全部知っている必要がある。	
\\	この書類の漢字全部に、振り仮名を振ってもらえませんか?	
\\	仮 
\\	名 
\\	(な), 
\\	名 (な). 
\\	仮, 名	
\\	生保	
\\	せいほ	
\\	生保はもちろんつみ立てですよね?	
\\	日本ではけんこう保険のせい度が整っているんだから、わざわざ生保に入る必要はあまりないよ。	
\\	日本では、女性の生命保険販売員は「生保レディ」と呼ばれますが、彼女たちの中には大きなコミッションや昇進のために枕営業をする人がいるという噂があります。	
\\	生, 保	
\\	生物	
\\	せいぶつ	
\\	かわらでなぞの生物を発見した。	
\\	地きゅう上でもっとも大きい生物はシロナガスクジラではなく、シロナガスフグです。	
\\	こないだの生物の授業のノート、見せてくれないかな。	
\\	生, 物	
\\	名物	
\\	めいぶつ	
\\	ポートランドの名物は何ですか。	
\\	トーフグの名物社長から、東京名物の東京ばななをおみやげにもらいました。	
\\	たこ焼きは絶対に大阪名物の一つです。	
\\	名物! 
\\	名, 物	
\\	要点	
\\	ようてん	
\\	もう少し要点をせいりしてからいけんをのべてください。	
\\	ユウサクが要点をはあくしているとは思えないんだけど。	
\\	いいから!早く要点を言ってよ!	
\\	要, 点	
\\	勝負	
\\	しょうぶ	
\\	する 
\\	勝負は引き分けにおわった。	
\\	これは、一発勝負のコンテストです。	
\\	この一年は、勝負の年だと思っています。	
\\	ふ 
\\	ぶ 
\\	勝, 負	
\\	食事	
\\	しょくじ	
\\	今、食事中なので、後でまたかけます。	
\\	どうぞごゆっくりお食事をお楽しみください。	
\\	もし残りの人生毎日同じ食事しか食べられないとしたら、何を選びますか?	
\\	食, 事	
\\	中学校	
\\	ちゅうがっこう	
\\	それより、中学校の前でとった写しんがいいよ。	
\\	ちょっとまってて。その中学校のば所を今しらべてあげる。	
\\	ええっと、私は中学校の教師ですよね。そこで、時間を無駄にしないためにも、とりあえず私は絶対に間違ったことを言わない、ということにしておいてください。	
\\	学校 
\\	学校 
\\	中. 
\\	がく 
\\	がっ, 
\\	中, 学, 校	
\\	名古屋	
\\	なごや	
\\	名古屋にはおいしいものがたくさんあります。	
\\	今日も名古屋で仕事があった。	
\\	名古屋にいる友人を訪れた際、矢場とんというとても美味しいトンカツをご馳走してもらいました。	
\\	名 
\\	屋, 
\\	な 
\\	名前 (なまえ) 
\\	な 
\\	(名前) 
\\	こ 
\\	ご 
\\	名, 古, 屋	
\\	必要	
\\	ひつよう	
\\	な 
\\	バスをおりてから、少し歩く必要があります。	
\\	のめんせつに合かくするために必要なことって何ですか。	
\\	日本に行ったら必要になるものって何だろう?	
\\	必, 要	
\\	付近	
\\	ふきん	
\\	この付近に、コンビニはありますか?	
\\	付近には、外国人見物客もたくさんいました。	
\\	この付近に公衆便所はありますか。	
\\	付 
\\	(ふ). 
\\	付, 近	
\\	新しい	
\\	あたらしい	い 
\\	新しいえんぴつですね。	
\\	じゃあ、あの新しいスタバでね!	
\\	あの男、バンジージャンプするのに私の新しいスカーフを使うなんて、信じられない!	
\\	い 
\\	新	
\\	新年	
\\	しんねん	
\\	新年、明けましておめでとうございます。	
\\	新年のほうふはもう決めましたか?	
\\	新年の挨拶に、親戚の家を回りました。	
\\	新, 年	
\\	苦しい	
\\	くるしい	い 
\\	食べすぎて苦しい。	
\\	トーフグにあいをこく白できなくて苦しいよ!	
\\	絶望のどん底に落ちて、苦しい思いをしている、そこのあなた!もしや、銀行があなたの家を差し押さえちゃったとか?もし、そうなら、我々はあなたのためにとっておきのものをご用意しております…その名も「インスタント・アルツハイマー」!我々のこの新薬を使えば、あなたは辛い思い出を忘ることができますよ!	
\\	い 
\\	(くる) 
\\	苦	
\\	相手	
\\	あいて	
\\	まずは、相手の話をちゃんときいてあげようよ。	
\\	うらないによると、わたしのけっこん相手はどこかの国の王子さまだそうです。本当かな?	
\\	馬鹿な話し相手は欲しくない。	
\\	手 
\\	手, 
\\	相, 
\\	(あい) 
\\	相, 手	
\\	平和	
\\	へいわ	
\\	な 
\\	いつになったら、世界は平和になるんだろうか。	
\\	安全で平和な地いきに住みたいです。	
\\	どうして日本ではみんなあの平和のシンボルを写真にかざすの?	
\\	平, 和	
\\	予定	
\\	よてい	
\\	する 
\\	ねえ、日よう日、予定を入れてもいい?	
\\	その小わく星は地きゅうに向かっていて、北大西ようのどこかにしょうとつする予定だ。	
\\	私がせっかく予定を立てても、誰もその通りに動いてくれないことが嫌です。	
\\	予, 定	
\\	部分	
\\	ぶぶん	
\\	この部分は、何色にしますか?	
\\	のうの前の部分の名前って何だっけ?	
\\	この映画、共感出来る部分だらけでめちゃくちゃ泣けた。	
\\	部, 分	
\\	〜県	
\\	けん	
\\	何県の出身ですか。	
\\	なら県の小学校では、せい服をきないといけない学校が多いです。	
\\	今まで日本の何県に行ったことがあるの?	
\\	けん 
\\	県	
\\	発音	
\\	はつおん	
\\	する 
\\	の 
\\	このたんごの発音を教えてください。	
\\	タイラーさんは日本ごの発音がとても上手ですね。	
\\	わあ!スペイン語の小さな家って言葉の発音、日本語の「お腹すいた」にめちゃくちゃ似てる!	
\\	発, 音	
\\	部首	
\\	ぶしゅ	
\\	このかん字の部首は何ですか?	
\\	部首をわらうものは部首になく。	
\\	私は全部の部首を覚えています。	
\\	首 
\\	(しゅ). 
\\	部, 首	
\\	返事	
\\	へんじ	
\\	する 
\\	まだ返事をおくっていません。	
\\	出せきのお返事、ありがとうございます。	
\\	私は、お金が欲しいというメールを百通以上ビル・ゲーツに送りましたが、まだ返事をもらっていません。	
\\	返, 事	
\\	住民	
\\	じゅうみん	
\\	市はひなん所をかいせつして、住民にひなんをよびかけています。	
\\	住民たちは、みんなすぐに打ちとけました。	
\\	あのアパートの住民には無料のエアコンが与えられるが、何故か冬の間だけなのである。	
\\	住, 民	
\\	保持	
\\	ほじ	
\\	する 
\\	コウイチは何かの世界記録保持者だと聞きました。	
\\	かくへい器の保持についてどうお考えですか?	
\\	は、火星の治安を保持することに、微力ながら貢献したいと思います。	
\\	保, 持	
\\	部屋	
\\	へや	
\\	妹は部屋のそうじをしました。	
\\	わたしの学校には、音楽室の中に夜だけにあらわれる部屋があるそうです。	
\\	部屋を片付けようとすると、いつも途中で見つけた別のものに気を取られてしまって、片付けることができません。	
\\	屋 
\\	(や) 
\\	部 
\\	へ. 
\\	(へ) 
\\	部, 屋	
\\	試食	
\\	ししょく	
\\	する 
\\	文句を言う前に、試食してみてよ。	
\\	コウイチの昔なつかしいソフトクリームチーズ商店の試食会におよばれしました。	
\\	スーパーで試食販売をしている人は、日本の食品業界では「マネキンさん」と呼ばれる。	
\\	試, 食	
\\	〜屋	
\\	や	
\\	近くにくすり屋と肉屋もあります。	
\\	王女にてんしょくする前は花屋ではたらいていました。	
\\	焼肉屋さんを経営している人と結婚するのが夢です。	
\\	屋	
\\	必勝	
\\	ひっしょう	
\\	じん社で必勝をきがんしてきました。	
\\	空手のし合で、「必勝」とかかれたハチマキをしている人を見ました。	
\\	もし虎と七面鳥が戦争することになったら、それはもう虎の必勝になるだろうね。それから、ついでに、養鶏所で働く人も喜ぶだろうね。	
\\	つ 
\\	ひつ 
\\	っ, 
\\	必, 勝	
\\	泳ぎ	
\\	およぎ	
\\	マイケルは魚よりも泳ぎが上手です。	
\\	カエルの泳ぎにあこがれて必死に平泳ぎをれんしゅうしていたら、手と足がほそ長くなりました。	
\\	泳ぎはあまり得意ではありません。	
\\	泳	
\\	受験	
\\	じゅけん	
\\	する 
\\	兄は来年大学受験です。	
\\	最近は、毎日朝から晩までずっと受験勉強をしています。	
\\	私は受験票を持って行くのを忘れてしまいましたが、母が届けてくれました。	
\\	受, 験	
\\	教え	
\\	おしえ	
\\	あなたは、父さんの教えをしっかりまもって生きるのよ。	
\\	それはすばらしい教えですね。	
\\	君の教えに従うと、ろくなことがない。	
\\	教える, 
\\	教	
\\	茶屋	
\\	ちゃや	
\\	ポートランドでは、トーフグ茶屋の、ふぐもちが人気です。	
\\	きょうとにりょ行したとき、コウイチは毎ばん茶屋あそびを楽しみました。	
\\	日本人は現代的なカフェを「カフェ」、昭和の香りのするコーヒーラウンジを「喫茶店」、伝統的なお茶屋さんを「茶屋」と呼びます。	
\\	茶, 屋	
\\	仮に	
\\	かりに	
\\	仮にコウイチが『トーフグはめすだ』と言っても、多くのユーザーはそのいけんに反対するんじゃないだろうか。	
\\	仮にトーフグにさいあくのじたいがおきたとしたら、ライフタイムメンバーの人たちはどうなるんですか?	
\\	え?あの娘の胸は風船で作られているって?仮にそうだとしても、だから何なんだよ!	
\\	に 
\\	(かり) 
\\	仮	
\\	世界	
\\	せかい	
\\	しかし、この世界にはどちらもありません。	
\\	ふじ山は世界いさんにとうろくされています。	
\\	あいつは自分の世界に閉じこもっていて、そこから出てこようとしないんだ。	
\\	世界一番 
\\	世, 界	
\\	物	
\\	もの	
\\	マサルさんが持っている物は何ですか。	
\\	うちのワンちゃんを物あつかいしないでください!	
\\	私の好きな物を三つ挙げるなら、チキンと揚げ物とご飯です。	
\\	(もの). 
\\	物	
\\	事	
\\	こと	
\\	スモモさんが、ピンク色のいしょうをきる事が多いのはなぜですか?	
\\	後はいが言っている事がいつもよく分からなくてこまっています。	
\\	すまないが、ちょっと今はする事がたくさんあって、忙しいんだ。	
\\	(こと) 
\\	事	
\\	今度	
\\	こんど	
\\	じゃあ、やっぱり今度にしよう。	
\\	今度は本当に本気の本気で日本ごをべんきょうします!	
\\	今度俺がお前を泣かせる時は、嬉し泣きだ。	
\\	今回.
\\	今, 度	
\\	千円札	
\\	せんえんさつ	
\\	千円札のおつりが不足しています。	
\\	これは、水そうから紙のひしゃくで千円札をすくうゲームです。	
\\	「千ドル札を燃やしてるの?」「いや、ただの千円札だよ。」	
\\	千円 
\\	千, 円, 札	
\\	付く	
\\	つく	
\\	今このパソコンを買うと、オマケが付くので、おとくですよ。	
\\	マイケルの
\\	シャツにはドレッシングのしみがたくさん付いています。	
\\	あなたの耳たぶに付いてるそれ、何?	
\\	付ける 
\\	付 
\\	く 
\\	付ける, 
\\	付	
\\	売れる	
\\	うれる	
\\	このゲーム、ぜったい売れるだろ。	
\\	その千円札、五百円でネットオークションにかけたら、よゆうで売れるっしょ。	
\\	売れる小説の書き方を教えてあげよう。	
\\	う 
\\	売	
\\	表す	
\\	あらわす	
\\	でも、ろこつに不かいかんを表すのはよくないよ。	
\\	国民はみんな、高がくのぜい金に不まんを表していますよ。	
\\	このアップルシナモンウィスキーがどれだけ美味しいかを言葉で表すことはできない。	
\\	う 
\\	(あらわす). 
\\	表	
\\	見付ける	
\\	みつける	
\\	むしさされって、見付けるととたんにかゆくなるよな。	
\\	このぼう子は、コウイチのかつらを見付けるための手がかりになるかもしれません。	
\\	私は今朝、虹の端にあると言われている金の壺を見付けるために四時間以上も費やしました。	
\\	付ける 
\\	見 
\\	見る 
\\	付ける. 
\\	見, 付	
\\	乗せる	
\\	のせる	
\\	子どもを助手せきに乗せるのはきけんです。	
\\	机に足を乗せるのは行ぎがわるいですよ。	
\\	男は、女性の自転車が完璧に壊れているのを見て、「町まで車に乗せてあげるよ」と言った。	
\\	う 
\\	(乗る 
\\	せ 
\\	せ 
\\	(せ) 
\\	乗	
\\	持つ	
\\	もつ	
\\	あらきさんは、けしゴムを持っていますか?	
\\	コウイチは、実はなんしきやきゅうしんぱん員のしかくを持っています...というのは実はうそです。	
\\	一等に当選したことを知った時、宝くじを持つ手の震えが止まらなかった。	
\\	う 
\\	(も). 
\\	持	
\\	通す	
\\	とおす	
\\	お肉には、しっかり火を通してください。	
\\	きちんとすじを通すべきだよ。	
\\	針に糸を通すのは難しい。	
\\	う 
\\	通る 
\\	(す) 
\\	通る. 
\\	る 
\\	す 
\\	通	
\\	負ける	
\\	まける	
\\	すきなチームが負けたので、くやしいです。	
\\	食パンマンは、アンパンマンに、チェスでこてんぱんに負けました。	
\\	「今夜は、どこで食事したい?」「あなたが決めてよ。私はじゃんけんに負けたんだから。」	
\\	う 
\\	(ま). 
\\	負	
\\	欠かす	
\\	かかす	
\\	イメージトレーニングとじゅんびうんどうは欠かすな!	
\\	どの社いんもみんな、トーフグにとって欠かすことのできない社いんです。	
\\	私の生活に、携帯電話を欠かすことはできない。	
\\	欠ける 
\\	(かす) 
\\	欠	
\\	要る	
\\	いる	
\\	玉ねぎとトマトと大こんが要ります。	
\\	ふくろは要りますか?	
\\	要るものと要らないものをリストに書き出しなさい。	
\\	う 
\\	要	
\\	道具	
\\	どうぐ	
\\	これが、ロブさんがかりたいと言っていた道具です。	
\\	ドラえもんの道具の中で一番使ってみたい道具は何ですか。やっぱりどこでもドアですか?	
\\	昨日買った道具入れの中に、なんとベイブ・ルースのルーキーカードが一枚入っていたんだ!	
\\	道, 具	
\\	保つ	
\\	たもつ	
\\	わかさを保つひけつは何ですか?	
\\	ビエトのおかげで、コウイチはメンツを保つことができた。	
\\	あのお医者さんに、「体調を保つためには、日々の運動が大切だよ」なんて言われたけど、そのお医者さん自体がすっごく太ってるんだよね。	
\\	う 
\\	(たも), 
\\	保	
\\	受ける	
\\	うける	
\\	しけんを受ける人は、13:00までに大学の受付に来てください。	
\\	トーフグの言ばをまともに受ける必要はないよ。	
\\	僕の手は他の人のよりも15倍も大きいけど、それでも野球ボールをうまく受けることができないんだ。	
\\	う 
\\	(う). 
\\	受	
\\	返す	
\\	かえす	
\\	今日、としょかんに本を返しに行かなくちゃいけないんです。	
\\	返す言ばが見当たらないよ。	
\\	「昨日あげたクッキー返してくれない?」「えっ、吐きだせってこと?」	
\\	う 
\\	帰る, 
\\	(かえ) 
\\	返	
\\	乗る	
\\	のる	
\\	バスに乗って、海へ行きました。	
\\	東京・しな川方めん行きではなくて、上野・池ぶくろ方めん行きに乗ってください。	
\\	ジェットコースターに乗るのは大好きなんだけど、どの席もいつもゲロ臭いんだよね。	
\\	う 
\\	乗	
\\	売る	
\\	うる	
\\	ここに、日本ごのじしょは売っていますか。	
\\	この村では、人々は大麦を売って生けいを立てています。	
\\	請求書の支払いをするために車を売る必要があるが、そうすれば今度は仕事にいけなくなってしまう。	
\\	う 
\\	(う)! 
\\	売	
\\	送る	
\\	おくる	
\\	スミスさんに手紙を送りました。	
\\	後で写しん
\\	で送るね。	
\\	遅くなったね。今日はここまでにしよう。家まで送るよ。	
\\	う 
\\	(おく) 
\\	送	
\\	泳ぐ	
\\	およぐ	
\\	ここで泳いではいけません。	
\\	あんた、目が泳いでるよ。	
\\	私はベッドの上にある見えないプールで泳いで運動しているんです。	
\\	う 
\\	泳	
\\	試みる	
\\	こころみる	
\\	ビエトに言われた方法を五度試みたんですが、ダメでした。	
\\	ビエトの手下が、けいむ所からのだっ走を試みたらしい。	
\\	は、常に新しい事を試みる人を探しています。	
\\	う 
\\	心 (こころ) 
\\	試	
\\	使う	
\\	つかう	
\\	友だちのじしょを使いたいです。	
\\	いたいのはとても苦手なので、ますいを使ってください。	
\\	そんなに屁っ放り腰にならないで!知ってる日本語をできるだけ使うようにしなよ。	
\\	う 
\\	(つか), 
\\	使	
\\	勝つ	
\\	かつ	
\\	サッカーのし合では、1対0で勝つことができました。	
\\	四つの時に、し合には勝ったが勝ぶに負けた。	
\\	明日のドッグ・レースにペットのウサギの参加を申し込むつもりなんだけど、確実に勝てると思うんだよね。	
\\	う 
\\	(か)!	勝	
\\	実力	
\\	じつりょく	
\\	コウイチのフラダンスの実力は本物です。	
\\	ビエトが本気の実力を出した所を見た人はまだだれもいません。	
\\	運も実力のうちってよく言うだろ。	
\\	実, 力	
\\	事実	
\\	じじつ	
\\	事実かもしれないけど、なっとくはできないな。	
\\	あのせいじ家の言っていることは、事実に反します。	
\\	口髭を生やした男が気味悪く見えるのは、科学的な事実です。	
\\	事, 実	
\\	実	
\\	じつ	
\\	の 
\\	実は、わたし、こうみえてもオリンピックに出たことがあるんです。	
\\	実のところ、おれたちはお前の実のおやじゃないんだよ。	
\\	すまない、お前。実は、俺達の貯金を全部、プロ野球カードと風船ガムにはたいちまったんだ。	
\\	実	
\\	使用	
\\	しよう	
\\	する 
\\	このクーポンの使用の仕方が分かりません。	
\\	明日、九時から会ぎ室を使用したいんですが、空いていますか?	
\\	「へー!使用人って被雇用者のことも言うんだ。てっきり召使いって意味だと思ってた。」「そうだよ。使用者も、利用者だけじゃなくて雇用主って意味もあるしね。」	
\\	使, 用	
\\	弱虫	
\\	よわむし	
\\	な 
\\	の 
\\	あんなに弱虫でなき虫だったむす子も、もう一人前の大人です。	
\\	多くのアメリカ人はなっとうごときが食べられない弱虫です。	
\\	弱虫毛虫は、挟んで捨てろ!	
\\	弱い 
\\	虫 
\\	弱, 虫	
\\	弱々しい	
\\	よわよわしい	い 
\\	なんか、弱々しいうたごえだったから、ちょっと心ぱいになっちゃった。	
\\	むかしは強く見えた父が年とともにだんだん弱々しく見えてきました。	
\\	その弱々しい患者は河豚の毒で苦しんでいた。	
\\	い. 
\\	弱い 
\\	らしい 
\\	しい 
\\	らしい 
\\	よわ 
\\	しい 
\\	弱, 々	
\\	勝者	
\\	しょうしゃ	
\\	勝者だけが次のステージにすすめます。	
\\	せんそうに勝者はいないのですから、どうかそんなむいみなたたかいは止めてください。	
\\	ビエトは、72時間耐久スクラブルゲームの勝者となった。	
\\	勝, 者	
\\	学院	
\\	がくいん	
\\	2014年に、トーフグ学院をそつぎょうしました。	
\\	かん西学院大学の多くの学生はざい学中にりゅう学をけいけんします。	
\\	やあ!こいつはワニカニ。あだ名はクラビゲーターだよ。トーフグ工業学院で働いてるんだ。	
\\	学, 院	
\\	足し算	
\\	たしざん	
\\	わたしは足し算ができない。	
\\	その足し算コンテストに、かん客としてさんかすることはできますか?	
\\	今日、学校で足し算の勉強をしました。	
\\	足す 
\\	足 
\\	算. 
\\	さん 
\\	ざん. 
\\	足, 算	
\\	進行	
\\	しんこう	
\\	する 
\\	げんざい、六つのプロジェクトが進行中です。	
\\	おいしゃさんに、進行がんがあると言われました。	
\\	フグエキスから作られた新しい薬がその病気の進行を抑えた。	
\\	進, 行	
\\	農業	
\\	のうぎょう	
\\	の 
\\	わたしたちは、さく年、む人とうで農業を始めました。	
\\	最近日本では、農業にきょうみのあるわか者がふえています。	
\\	僕は今はセックスにあまり興味が無いんだよ。しばらくはただ農業に集中したいんだよね。	
\\	農, 業	
\\	路地	
\\	ろじ	
\\	うちのお店は、ウェスタン通りから、少し路地に入ったところにあります。	
\\	その路地は、夜は少しきけんだから気を付けてね。	
\\	警察官は、路地まで苺泥棒を追いかけた。	
\\	じ 
\\	生地 
\\	路, 地	
\\	算数	
\\	さんすう	
\\	の 
\\	明日の算数のじゅぎょう、定ぎと分度きわすれずにね!	
\\	立ち入ったことをお聞きしますが、よみ・かき・算数、どれが一番とくいですか?	
\\	「連絡するのに時間がかかっちゃってごめんなさい。最近、仕事で忙しかったものだから。」「君の仕事って何だっけ?」「小学校の算数の先生よ。」	
\\	算, 数	
\\	助手	
\\	じょしゅ	
\\	ワトソンはかせは、シャーロック・ホームズの助手ではなく、あくまで友人です。	
\\	わたしの助手は、年末に肉ばなれになりました。	
\\	もし私が今首になった助手の助手だとしたら、それってつまり私も首になっちゃったってこと?	
\\	助, 手	
\\	早速	
\\	さっそく	
\\	早速使わせていただきました。	
\\	早速のごへんしん、ありがとうございます。	
\\	私はその茄子を早速彼に送ります。	
\\	速 (そく) 
\\	早 
\\	(さっ), 
\\	早, 速	
\\	心配	
\\	しんぱい	
\\	する 
\\	な 
\\	かれが元気がないのはめずらしいので心配です。	
\\	この船の進行方向が西なのか東なのか心配になった。	
\\	心配しないで。すべて上手くいくよ。私の言うことを信じて。	
\\	心配しないでね!	
\\	ぱい 
\\	配 
\\	はい, 
\\	心, 配	
\\	自転車	
\\	じてんしゃ	
\\	わたしは自転車で会社に行っています。	
\\	自転車を買ってあげるよ。お返しは要らないからね!	
\\	「ごめん。てっきり君が僕の自転車を壊したのかと思ってたよ。」 「何それ!ひどい。私がそんなことするはずないでしょ。」	
\\	自, 転, 車	
\\	千葉	
\\	ちば	
\\	かれは千葉県人です。	
\\	わたしの母は、千葉県さんの落花生が大すきです。	
\\	ここから千葉までだと、片道1時間30分から2時間はぐらいかかりますね。	
\\	千 
\\	(ち) 
\\	千, 葉	
\\	親友	
\\	しんゆう	
\\	ま、親友には反対されたんですけどね。	
\\	コウイチとビエトは親友で、友だち以上、こい人未まんのかんけいです。	
\\	「ねぇ、知ってる?サラがベスと付き合ってるのよ。」「え、知らなかった。サラのこと親友だと思ってたけど、私そのこと何も聞いてないわ。」	
\\	親, 友	
\\	集金	
\\	しゅうきん	
\\	する 
\\	みんなから集金して、花たばと色紙を買いました。	
\\	集金のことはもう心配するなよ。なるようになるさ。	
\\	やらなければならない仕事が山のようにあるよ。とりあえず、お客さんから集金することから始めようかな。	
\\	集金 
\\	集, 金	
\\	集中	
\\	しゅうちゅう	
\\	する 
\\	の 
\\	会ぎに集中しすぎて終電をのがしてしまいました。	
\\	集中集中!気合いを入れろ!	
\\	もう!まとわりつくのは止めてよ!集中できないでしょ。	
\\	集, 中	
\\	調子	
\\	ちょうし	
\\	この調子でがんばるぞ!	
\\	最近、トーフグの調子はどう?	
\\	あの男はみんなに調子のいいことばかり言うから嫌いよ。	
\\	調, 子	
\\	曲線	
\\	きょくせん	
\\	このドレスは、曲線を美しく出してくれます。	
\\	ブーメランパンツから出てるコウイチのおしりの曲線…いいね!	
\\	あなたが怒った時の、ほおの柔らかく歪んだ曲線が好きよ。	
\\	曲, 線	
\\	要求	
\\	ようきゅう	
\\	する 
\\	労働者たちは、ちん上げを要求した。	
\\	会社からの要求が前近代的すぎて気持ち悪いんだけど。	
\\	この要求はいくらなんでも無茶だ。	
\\	要, 求	
\\	開発	
\\	かいはつ	
\\	する 
\\	の 
\\	開発チームがみなさんのごしつもんにお答えします。	
\\	トーフグの開発ひ話を教えてください。	
\\	コウイチは新しい日本語学習サイトを開発するために徹夜をした。	
\\	開, 発	
\\	〜病	
\\	びょう	
\\	メニエール病をわずらっているので、よくめまいが起きるんです。	
\\	ぼくがワニカニレベル1から全ぜん進めないのはきっと外国語できない病にかかったからにちがいない。	
\\	彼の言ったことを大袈裟に考えないで。難病で苦しんでるなんて明らかに嘘なんだから。	
\\	病	
\\	電鉄	
\\	でんてつ	
\\	東きゅう電鉄によりますと、東きゅう東よこ線は人身じこのえいきょうで、げんざい運転を見合わせています。	
\\	電鉄会社ではたらくのがゆめです。	
\\	「はい、どうぞ。あなたが好きな阪急電鉄の写真よ。」「うわぁ!やったぁ!俺の大好きな阪急電鉄じゃん!」	
\\	電, 鉄	
\\	強調	
\\	きょうちょう	
\\	する 
\\	の記事は超ドきゅうにおも白くて役に立つことを強調しておきたい。	
\\	大事な用語は、けい光ペンで強調するとおぼえやすいよ。	
\\	この仕事では身支度をきちんとしていることと礼儀正しいことが重要になることを改めて強調させて頂きます。	
\\	強, 調	
\\	終了	
\\	しゅうりょう	
\\	する 
\\	大さか公えん1日目終了しました!	
\\	主人公が死んでゲームが終了してしまいました。	
\\	「もし私がまたデブになったらどうする?」 「まぁ、俺達の関係が終了しないのは確実だね。」	
\\	終, 了	
\\	進化	
\\	しんか	
\\	する 
\\	日に日に進化してるのが実かんできて、うれしいです。	
\\	みなさんのおかげで、トーフグは大きな進化をとげることができました。	
\\	おい、お前って本当に阿呆だな。ダーウィンの進化論はディズニー映画のタイトルなんかじゃねえよ。	
\\	進, 化	
\\	顔付き	
\\	かおつき	
\\	このころのたえ子ちゃんは、まだ顔付きが子どもっぽいね。	
\\	あのミニチュアハスキーは、とてもりりしい顔付きをしているね。	
\\	うわぁ、すっかり大きくなったね!顔付きはお父さんそっくりそのままだ。	
\\	顔, 付	
\\	目医者	
\\	めいしゃ	
\\	きのう目がいたくなって、目医者に行ったら、ものもらいだった。	
\\	週末も開いている目医者をさがしています。	
\\	「よお、目医者さん。もう何もかもばれているんだよ。だから、目について何も知らないふりをする必要はないぜ。」	
\\	医者 
\\	医者 (いしゃ) 
\\	目 (め), 
\\	目, 医, 者	
\\	顔	
\\	かお	
\\	きのう会ったりょ行会社の人の顔をもうわすれました。	
\\	コウイチって、顔がこいよね?	
\\	「おい、お前。何か文句でもあんのか?ジロジロ見てんじゃねぇぞコラ。」「えっと、あなたの顔、かなり青く見えるんですが、大丈夫ですか?」	
\\	顔	
\\	農場	
\\	のうじょう	
\\	日よう日、小麦のしゅうかくを体けんしに、農場へ行ってきました。	
\\	農場で、引きつづきはたらかないかと聞かれました。	
\\	オーガニック農場を始めるつもりです。	
\\	農, 場	
\\	速い	
\\	はやい	い 
\\	新かん線はふつうの電車より速いです。	
\\	ポートランドから東京まで、だれが一番速く泳げるかきょうそうしました。	
\\	うちの家族は、ご飯を食べるのがとても速いです。	
\\	早い 
\\	速	
\\	一番目	
\\	いちばんめ	
\\	このごうかな朝食を食べ終わって、一番目のかんそうは、やっぱりベーコンはおいしい、です。	
\\	一番目のらんには、すきな人の名前をかいてください。	
\\	「僕は毎朝一番目に、5キロのランニングをするんだぜ。」「大したことないよ!僕は10キロ走るよ。」	
\\	一番 
\\	(目) 
\\	一番 
\\	目. 
\\	一, 番, 目	
\\	外来語	
\\	がいらいご	
\\	げんざいの中国語では、たくさんの日本語が外来語として使われています。	
\\	リュックサックは、外来語ですか?それとも和せいえい語ですか?	
\\	ねえ、煙草って外来語なのに漢字があるって知ってた?	
\\	外来 
\\	外来語, 
\\	外, 来, 語	
\\	フランス語	
\\	ふらんすご, フランスご	
\\	わたしのむすめは、フランス語もえい語も話せます。	
\\	アフリカの多くの国では公用語にえい語とフランス語が使用されています。	
\\	フランス語の学習は僕には難しすぎるよ。もうたく山だ!勘弁してくれよ!	
\\	フランス 
\\	語 
\\	日本語 
\\	中国語 
\\	スペイン語 
\\	語	
\\	青葉	
\\	あおば	
\\	青葉が目にしみるきせつとなりました。	
\\	家のにわの青葉に虫がいっぱい付いてる!	
\\	お前の頭、ちょっと変なにおいがするぞ。 乾燥した青葉か何かみたいな。	
\\	青, 
\\	葉 
\\	は 
\\	ば 
\\	あおば.	青, 葉	
\\	大学院	
\\	だいがくいん	
\\	大学院をそつぎょうしたんだけど、仕事が見つからないんだ。	
\\	この大学院の自どうはん売きは、つめたい飲み物しか売っていません。	
\\	私は大学院進学を考えております。	
\\	大学 
\\	大, 学, 院	
\\	頭	
\\	あたま	
\\	頭をあらってあげましょう。	
\\	ひるねをすると、いつも頭がすっきりします。	
\\	頭が痛いので、喋らないでください。	
\\	頭	
\\	求人	
\\	きゅうじん	
\\	する 
\\	求人については、後ほどあらためてせつ明をいたします。	
\\	近年は、ほぼ全てのき業が求人にインターネットを使います。	
\\	トーフグのブログで拝見した求人の件でお電話しました。プログラマー職に応募したいのですが、ビエトさんはまだ生きていらっしゃいますか?今朝方刺客を送り込んでおいたのですが。	
\\	求, 人	
\\	病院	
\\	びょういん	
\\	の 
\\	病院までタクシーで、千円ぐらいです。	
\\	この病院の院長は、九九もできない。	
\\	この病院で医者をすることには本当に嫌気が差すね。患者は馬鹿ばっかりなんだ。	
\\	病, 院	
\\	入院	
\\	にゅういん	
\\	する 
\\	病じょうがあっ化し、入院する事になりました。	
\\	入院中は、たいくつであくびが止まらなかったよ。	
\\	今度、乳癌で入院することになったんです。	
\\	入, 院	
\\	工業	
\\	こうぎょう	
\\	日本では工業が発たつしています。	
\\	工業デザイナーの中山さんは、ワニカニをストレス発さんに使っています。	
\\	「工業技術プロジェクトの進み具合はどう?」「今のところは順調だよ。」	
\\	工, 業	
\\	親しい	
\\	したしい	い 
\\	まだ親しい友人にしか話してないの。	
\\	「親しき仲にも礼ぎあり」って言うでしょ?いくら親しいからってトイレはのぞかないで!	
\\	ちょっと、コウイチ。落ち着いてよ。 親しい友人たちに誰か80セント貸してあげれる人がいないか聞いてみてあげるから。	
\\	(した). 
\\	親	
\\	日本語	
\\	にほんご	
\\	の 
\\	いいですか。英語じゃなくて日本語ですよ。	
\\	今ワニカニを買うと、もれなく日本語でかかれたトーフグのサインが付いてきますよ。	
\\	間違うことを恐れるなよ。日本語で喋ってみろって。やってみろよ。君ならできると思うよ。	
\\	日本 
\\	日本 (にほん) 
\\	語. 
\\	日, 本, 語	
\\	日本酒	
\\	にほんしゅ, にっぽんしゅ	
\\	日本酒はあまりすきではないんです。	
\\	その日本酒のボトルから、値札を外してもらってもいいですか?	
\\	「何飲んでるの?」「日本酒だよ。一口飲んでみる?」「それじゃ、お言葉に甘えて一口もらってみようかな。」	
\\	日本 
\\	日本
\\	酒, 
\\	日, 本, 酒	
\\	漢字	
\\	かんじ	
\\	わたしは漢字をかくのが下手です。	
\\	わたしの名前の漢字には、ウィスコンシン州でくらせますように、といういみが込められています。	
\\	「これ、君への誕生日プレゼントだよ。」「わあ!漢字のテキストじゃない!どうもありがとう。あなたのおかげで本当に幸せな気分になったわ。」	
\\	漢, 字	
\\	親切	
\\	しんせつ	
\\	な 
\\	いつも親切にしていただいて、ありがとうございます。	
\\	トーフグのような親切な会社と出会ったのははじめてです。	
\\	わぁ。すごい!なんてクリーミーなの。 信じられない。今まで食べた中で一番のチーズケーキだわ。 あなたってとても親切だから、きっと最後の一切れは私にくれちゃうわよね?でしょ?有難う、有難う!	
\\	親, 切	
\\	台所	
\\	だいどころ	
\\	の 
\\	母は、台所でりょうりをしています。	
\\	マイケルの家の台所にはなっとうの空きばこが三百万八千九百十七こおいてあるんだって。	
\\	台所から焦げ臭い臭いがするよ。	
\\	台 
\\	所. 
\\	台, 所	
\\	研究室	
\\	けんきゅうしつ	
\\	わたしは子どもの時、研究室ではたらくのがゆめでした。	
\\	くどいようですが、山田さんのかつらについて話すのは、この研究室ではタブーですからね。	
\\	蛙の解剖を始めるので、皆さんには研究室に移動して頂きたいと思います。	
\\	研究 
\\	研, 究, 室	
\\	言語	
\\	げんご	
\\	今人気のプログラミング言語は何ですか?	
\\	言語オタクたちが朝まで言語について語りつくすオフ会にさんかしてきました。	
\\	新しいコンピュータ言語の勉強を始めました。	
\\	言, 語	
\\	最終	
\\	さいしゅう	
\\	の 
\\	大すきなドラマの最終回を見のがして泣いた。	
\\	今乗ってきた電車が最終だったから、もうかえれないよ。	
\\	これが我が社の最終提案です。後はあなたの決断次第です。	
\\	最, 終	
\\	最後	
\\	さいご	
\\	の 
\\	お前に金をやるのは、これが最後だ。	
\\	うちのつま、いつも最後に一言多いんですよね。	
\\	とにかく、最後まで聞いてよ。そうすれば分かるから。	
\\	最, 後	
\\	終点	
\\	しゅうてん	
\\	電車でばくすいしてしまって、終点でうん転手さんに起こされたことはよくありますね。	
\\	このシカゴ発のアムトラックの終点の駅は何て言う名前ですか。	
\\	お願いがあるの。私がたとえ何と言っても、何をしようとも、終点に着くまで私が電車から降りないようにして。分かった?	
\\	終, 点	
\\	葉	
\\	は	
\\	きのうは風が強かったので、木から葉が全部落ちてしまいました。	
\\	あきになって、木の葉の色が色んな色にかわってきた。	
\\	大根の葉でふりかけを作りました。	
\\	葉	
\\	軽い	
\\	かるい	い 
\\	このはこは、あのはこよりも軽い。	
\\	この毛ぬき、びっくりするぐらい軽いんだけど。	
\\	毎日軽い朝ごはんを食べる。	
\\	い 
\\	軽	
\\	線	
\\	せん	
\\	分数をかく時にまん中に引く線の名前は、かつ線です。	
\\	しぶ谷で山手線に乗ってください。	
\\	本当にこの線が完璧だって思ったの?しかも、全然真っ直ぐじゃないじゃない!	
\\	線	
\\	楽しみ	
\\	たのしみ	
\\	な 
\\	あなたに会うのが楽しみです。	
\\	夏休みの楽しみと言えば、海にキャンプに色々あると思いますよ。	
\\	私達、みんな亀を燃やすのを楽しみにしているんだよね?	
\\	楽しい, 
\\	み 
\\	楽しい, 
\\	楽	
\\	運がいい	
\\	うんがいい	
\\	い 
\\	最近みょうに運がいいんだよな。	
\\	こんな美人のよめさんもらって、お前って本当に運がいいな。	
\\	日本語の授業で
\\	が取れたんだけど、ただ単に運がよかっただけだと思うの。	
\\	いい 
\\	運	
\\	〜向け	
\\	むけ	
\\	これは子ども向けの作ひんではないですよね。	
\\	このカレーは、日本人向けのあじですね。	
\\	コウイチが履いてるパンツは、子供向けなんじゃないのか?ツンツルテンだよ。	
\\	向く 
\\	向ける 
\\	向け 
\\	向く 
\\	向	
\\	近所	
\\	きんじょ	
\\	の 
\\	近所の人はしずかですか?	
\\	トーフグのオフィスの近所には、おしゃれなお店が多く集まっています。	
\\	近所にコインランドリーがあるわよ。	
\\	しょ 
\\	じょ.	近, 所	
\\	新聞	
\\	しんぶん	
\\	の 
\\	父は、毎朝新聞を読みます。	
\\	今朝の地元の新聞の一めんでトーフグとどーも君がコラボしていたよ。	
\\	この新聞すっげー専門用語だらけじゃん。スコットがこの新聞を何回も読まなきゃいけないのは当然だな。	
\\	新, 聞	
\\	番号札	
\\	ばんごうふだ	
\\	番号札をとって、お待ちください。	
\\	すみません、この辺りに番号札は落ちていませんでしたか?	
\\	こちらがお客様のお荷物の番号札となります。	
\\	番号 
\\	(番号) 
\\	(ふだ). 
\\	番, 号, 札	
\\	開業	
\\	かいぎょう	
\\	する 
\\	うちも、開業当時はしゃっ金だらけだったんだよ。	
\\	この目医者は、今年で開業して三十五年になります。	
\\	コウイチの開業準備を手伝うために、昨日は学校をサボったんだ。	
\\	開, 業	
\\	公開	
\\	こうかい	
\\	する 
\\	の 
\\	どうがが公開されるのはいつですか?	
\\	発売を記ねんして、コウイチの半生をドキュメンタリーにしたえいがが公開になりました。	
\\	「この写真を公開したのは、写真より実物の花子の方がずっと可愛いんだってことをみんなに教えたかったからだよ。」「あんたって本当口からでまかせばっかりよね!」	
\\	公開.	
\\	公, 開	
\\	親	
\\	おや	
\\	わたしの親は、先週病院の近くに引っこしました。	
\\	うちの親、かほごすぎると思うんだよね。	
\\	これって、すごく不公平だわ! ミドリは私が欲しいものを全部手に入れて、その上私のパパとママが親じゃないだなんて!	
\\	(おや) 
\\	親	
\\	道路	
\\	どうろ	
\\	大雪で、高速道路が通行止めになっているそうですよ。	
\\	ミネソタ州は夏に道路工事ばかりしています。	
\\	この町は本当につまらない。基本的に道路と家しかないし。あぁ〜もう!どこか楽しいところはないのかな?	
\\	道, 路	
\\	農民	
\\	のうみん	
\\	の 
\\	え戸時代の農民でも、全てがまずしかったわけではないはずだ。	
\\	この村の農民たちは、クロールの泳ぎ方を知りたがっている。	
\\	おぉ!あの農民の腹筋すげぇな。	
\\	農, 民	
\\	当て字	
\\	あてじ	
\\	「リキュール」には、「利休酒」という当て字が使われます。	"""利休酒
\\	近ごろは、当て字を使ったキラキラネームを持つ子どもがたくさんいます。	
\\	時々、漢字は、亜米利加 (アメリカ)のようにそれぞれの意味を無視して、読み方だけで選ばれます。これらの漢字を当て字と言います。	
\\	亜米利加 (アメリカ). 
\\	当て 
\\	字 
\\	字 
\\	当て. 
\\	当てる 
\\	当, 字	
\\	私大	
\\	しだい	
\\	私大の授業料は高いもんね。	
\\	とある大阪の私大で英語を教えています。	
\\	こないだ言ってた私大の学園祭の件なんだけど、今回は私の事数にいれないで。ごめん。でも悪く思わないでね。時間ができたら連絡するから。	
\\	私 
\\	大 
\\	大学? 
\\	私, 大	
\\	私鉄	
\\	してつ	
\\	東京の私鉄はふくざつでよく分かりません。	
\\	で新宿駅まで行ったら私鉄に乗りかえてください。	
\\	私は、とある日本の大手私鉄会社で働いております。	
\\	鉄 
\\	私, 鉄	
\\	線路	
\\	せんろ	
\\	大雨で、線路わきの土しゃがくずれ、線路が土しゃでうまってしまっているもよう。	
\\	だれかが線路に立ち入ったため、電車が止まってしまった。	
\\	「あぁ、ごめん。今はあんまりそういう気分じゃないんだ。」「あなた、どうしたの?」「実は、さっきそこの線路で男が飛び込み自殺をするのを目撃したんだ。」	
\\	線, 路	
\\	読み方	
\\	よみかた	
\\	楽ふの読み方を知らない。	
\\	三だんらく目の七行目のこのたん語の読み方は、なんですか?	
\\	あなたの名前の漢字の読み方を教えてください。	
\\	方 
\\	方 
\\	見方 
\\	作り方.	読, 方	
\\	最近	
\\	さいきん	
\\	の 
\\	女の人は、最近、どのぐらい本を読んでいますか。	
\\	最近のトレンドファッションにはもう付いていけません。	
\\	「ござる」という言葉は最近はあまり使われなくなったようだ。	
\\	最, 近	
\\	研究所	
\\	けんきゅうしょ, けんきゅうじょ	
\\	とあるアメリカの研究所が、日本語がペラペラになるくすりを開発したそうだ。	
\\	今日は、みそしる研究所で、ぼうさいくんれんがあります。	
\\	私はバーガーキング研究所で働いています。私の仕事は、フライドポテトを食べ、その後自分の体重を測ることです。	
\\	研究 
\\	研究 
\\	所 
\\	しょ 
\\	じょ, 
\\	研, 究, 所	
\\	回転	
\\	かいてん	
\\	する 
\\	ペットのハムスターがものすごいいきおいで回し車を回転させています。	
\\	あそこの回転ずし、もう行った?なんか、そこのすししょく人は皿がきらいらしく、すしが回転してあつくなってるコンベアベルトの上に直でおかれているらしいよ。	
\\	月は地球の周り回転していて、地球は太陽の周りを回転している。	
\\	""回転ずし?
\\	回, 転	
\\	お酒	
\\	おさけ	
\\	わたしはパーティーでお酒を飲んだ。	
\\	今、そのお酒はシーズンではないので、申しわけありませんがご用いできません。	
\\	「割り勘にしよう。」「いや。今日は俺の奢りだよ。しかも、俺はお酒をたくさん飲んだけど、お前は一滴も飲まなかったじゃないか。」	
\\	酒 
\\	お 
\\	さけ, 
\\	酒	
\\	近道	
\\	ちかみち	
\\	する 
\\	上手な絵をも写する事は、上たつの近道です。	
\\	ワニカニで漢字をおぼえることは、本当に日本語習とくの近道になると思いますか?	
\\	「ありえねぇ!お前って本当脳みそ無いのな。なんで一方通行を逆走してるんだよ。」「近道になると思ったんだよ。」	
\\	ちか 
\\	近 
\\	近い) 
\\	道 (みち), 
\\	近, 道	
\\	病気	
\\	びょうき	
\\	の 
\\	母の病気は、まだよくなりません。	
\\	おしりからおならが止まらない病気にかかってしまい、外に出られなくなってしまいました。	
\\	「先生、僕のママの病気、直せるの?」「やってみるけど、当てにしないでね。」	
\\	病, 気	
\\	病人	
\\	びょうにん	
\\	うつったらいやだから、病人には近よりたくない。	
\\	うらないしに、前世は病人だったと言われました。	
\\	「調子はどうだい?」「絶好調だよ。」「それは良かった。病人からその言葉を聞けるとは思ってもみなかったがな。」	
\\	にん 
\\	人. 
\\	病, 人	
\\	横	
\\	よこ	
\\	きっ茶店の横に、食どうがあります。	
\\	もし強とうがトーフグのオフィスにやって来たら、コウイチのデスクまで走ってその横にしゃがみこむんだ。そこが最も安全な場所だからね。	
\\	午後の仮眠の間に横向きに寝る。	
\\	右側.
\\	横	
\\	歌手	
\\	かしゅ	
\\	歌手になったのは、母のえいきょうが大きいです。	
\\	あの歌手、かなり日やけしたね。	
\\	子どもの頃は、歌手だったのよ。なんちゃってね。今でも歌手なんですけど。	
\\	手 
\\	かしゅ 
\\	歌, 手	
\\	歌	
\\	うた	
\\	でも、日本の歌はあまり知りません。	
\\	これはたんなる歌じゃなく、ひみつのあん号でもあるんじゃないかという気がしています。	
\\	「ジェイソン!あんた、アマンダのために歌を作ったの?」「えーと、これはそういうのじゃないんだ。」「もう!とぼけないでよ。」	
\\	(うた) 
\\	歌	
\\	スペイン語	
\\	すぺいんご, スペインご	
\\	明日はスペイン語の小テストだぜ!	
\\	そのスペイン語のテキストは今売り切れているので、ご用いできません。	
\\	ニンニクはスペイン語で
\\	っていうんだけど、「アホ」って発音されるんだよ。「アホ」は日本語で
\\	って意味だよ。	
\\	スペイン 
\\	語	
\\	速度	
\\	そくど	
\\	インターネットの速度がおそすぎてイライラしています。	
\\	もう少し速度をおとして走行してください。	
\\	ドアを開けたら、冬の突風が物凄い速度で吹きこんできた。	
\\	速, 度	
\\	立ち飲み	
\\	たちのみ	
\\	夏には、テラスせきで立ち飲みもできます。	
\\	東京の新ばしえき近辺には立ち飲み屋がたく山あります。	
\\	昨日、あの立ち飲み屋の後、誰かとやった?	
\\	立つ 
\\	飲む 
\\	立, 飲	
\\	開始	
\\	かいし	
\\	する 
\\	このゲームは、明日はいしん開始です。	
\\	の先行予やく受付を開始しました。	
\\	現在、
\\	は枕業界で巻き返しを図っており、新しいモデルの枕の販売を開始した。	
\\	開, 始	
\\	話	
\\	はなし	
\\	それから話を聞いて、もんだい用紙の1から4の中から、一番いいものを一つえらんでください。	
\\	話がちがうよ!	
\\	「それから、私、他にも言いたいことがあるのよ、ブリタニー!」「ジェレミー、いい加減黙ってくれないか!君の話はもう聞き飽きたよ。」	
\\	話 
\\	話.	
\\	(はなし) 
\\	話す
\\	話	
\\	鉄人	
\\	てつじん	
\\	うちの社長は、鉄人すぎます。	
\\	わたしが一番すきな番組はりょうりの鉄人です。	
\\	私はスーパーで赤ちゃんに「いないいないばぁ」をしている鉄人を見た。	
\\	料理の鉄人, 
\\	鉄, 人	
\\	主語	
\\	しゅご	
\\	の 
\\	主語をハッキリさせてくれないかな。	
\\	この主語を明かくにする新サービスは、本当に役に立つ。	
\\	日本語は、主語ー目的語ー動詞の言語ですが、英語は主語ー動詞ー目的語の言語です。	
\\	主, 語	
\\	引き算	
\\	ひきざん	
\\	もちろん、わたしは引き算ができます!	
\\	この引き算のじゅぎょうは長すぎて、あくびが止まりません。	
\\	引き算を習うのは、小学校何年生ですか?	
\\	引く 
\\	引 
\\	算. 
\\	さん 
\\	ざん. 
\\	引, 算	
\\	地下鉄	
\\	ちかてつ	
\\	地下鉄に乗って行きます。	
\\	東京の地下鉄はかなりふくざつです。	
\\	ああ、よかった!地下鉄に乗り遅れちゃうかと思ったよ。	
\\	地, 下, 鉄	
\\	最高	
\\	さいこう	
\\	な 
\\	の 
\\	仕事終わりのビールは最高だな。	
\\	東京わんの夜けいを見ながら最高の一時を楽しんでいます。	
\\	「うわ!あんたの作ったの、今までで最高のチョコレートケーキだわ。でも、一体どうしちゃったの?お菓子作りなんてしたことなかったじゃない。」「別に。何か新しいことをしてみたかっただけだよ。」	
\\	最, 高	
\\	最も	
\\	もっとも	
\\	げしとは、一年で最も夜がみじかくなる日のことだ。	
\\	ふじ山は日本で最も高い山です。	
\\	彼、今日の夜何食べたか知ってる?焼き肉食べ放題に行ったんだよ。最も臭いおならに備えておいた方が身のためだよ。用心に越したことはないんだから。	
\\	(もっと) 
\\	最	
\\	今朝	
\\	けさ	
\\	今朝、くすりを飲むのをわすれました。	
\\	今朝から両足がいたくて思うように歩けません。	
\\	今朝ここで玉突き事故があったんだ。	
\\	(けさ) 
\\	今, 朝	
\\	進む	
\\	すすむ	
\\	仕事は進んでいますか?	
\\	トーフグは、いつも一歩先を進んでいるよね。	
\\	お嬢さんは他の学生よりずっと進んでいる。	
\\	う 
\\	進	
\\	決まる	
\\	きまる	
\\	仕事が決まってからは、きょうみがなくなりました。	
\\	コウイチは毎朝かみがたが決まらないせいで会社にちこくする。	
\\	「それは本当にいい考えだね。そうしよう。」「よし、じゃあ、それで話は決まりだ。」	
\\	決める 
\\	決める, 
\\	決	
\\	始める	
\\	はじめる	
\\	何から始めるのがいいのか、分かりません。	
\\	むすめさん、四年生になったら、何かならい事を始めるの?	
\\	生地をフライパンに流し始める前に、材料を全て混ぜる必要があります。	
\\	う 
\\	(はじ). 
\\	始	
\\	助かる	
\\	たすかる	
\\	まだ助かる見こみはあるんですか。	
\\	コウイチ先生にもらったれんあいのアドバイスのおかげで、助かりました。	
\\	「で、私達付き合ってるのよね?どうなの?はっきりさせてよ。」「えーと…」<その時、インターホンが鳴った。:ピンポーン
\\	「ふぅ、間一髪で助かった。」	
\\	助ける 
\\	か 
\\	助ける, 
\\	助	
\\	病む	
\\	やむ	
\\	あっぱくめんせつって、終わった後、ほんと病むよな。	
\\	しょく場の同りょうは、仕事のストレスが原いんで病んでしまいました。	
\\	まさかアイツが心を病むなんてな。	
\\	う 
\\	(や) 
\\	病	
\\	化かす	
\\	ばかす	
\\	たぬきやきつねは、人間を化かすのがとくいだ。	
\\	わたしの友人はいつも化しょうで男の人を化かしています。	
\\	「遅くなってすみません。いやあ、きつねに化かされちまってねえ。」「心配しないでください。それはどうしようもないことですからね。」	
\\	化ける 
\\	(かす) 
\\	化ける, 
\\	化	
\\	算定する	
\\	さんていする	する 
\\	所とくぜいがくを算定するためのけいさんしきを知っていますか。	
\\	ワニカニのユーザーのぞうげんを算定する
\\	をやといました。	
\\	このプログラムは、該当ユーザーに選択されたバナー広告の掲示数に応じて、広告費用を算定するんだ。	
\\	算, 定	
\\	運ぶ	
\\	はこぶ	
\\	あ、それから、つくえを運ぶのをわすれないでください。	
\\	私の父親としての役目は、食べ物をテーブルに運んで、娘と一緒にお茶会をすることだ。	
\\	この教科書を運ぶための鞄がほしい。	
\\	う 
\\	(はこ). 
\\	運	
\\	鳴く	
\\	なく	
\\	子犬がワンワン鳴いています。	
\\	あそこでコケコッコーとおんどりが鳴くまねをしているのが、うちの社長です。	
\\	今夜、あの新しい洒落た日本食レストランに行こうよ。知ってた?あそこには、ホーホケキョと美しい声で鳴く鶯も何羽かいるんだぜ。	
\\	う 
\\	鳴	
\\	集める	
\\	あつめる	
\\	青いシャツをきている人を集めてください。	
\\	ユーザーからのフィードバックを集めています。	
\\	私は外国の食べ物の写真を集めている。	
\\	う 
\\	(あつ). 
\\	集	
\\	配る	
\\	くばる	
\\	バレンタインデーには、友だちに、手作りのおかしを配るんです。	
\\	しんぶんを配るバイトって月にいくらくらいかせげるのかな。	
\\	このパーティーマジでつまんない。道でティシュでも配ってる方がマシだよ。	
\\	う 
\\	(くば) 
\\	配	
\\	飲む	
\\	のむ	
\\	のどがかわいたので、何か飲みましょう。	
\\	この水道水は飲めますか?	
\\	「何をお飲みになりますか? ジン、ウォッカ、ビール、ワイン、日本酒、それとも焼酎?  何でもおっしゃってください。お酒なら何でも揃っていますから。」	
\\	う 
\\	飲	
\\	運転する	
\\	うんてんする	する 
\\	休日には、とくにあてもなく運転するのがすきなんだ。	
\\	運転する時は、ぜっ対に後ろはふり返りません。サメと同じで、前しか見ないんです。	
\\	このフグモデルの車は最高だね!とても気に入ったよ。毎日運転するのが待ち遠しいよ。	
\\	運, 転	
\\	終わる	
\\	おわる	
\\	学校が終わったら、買い物に行きます。	
\\	ギャンブルにまけて人生終わってしまった。	
\\	おい、こら!引っ込んでろ!原子爆弾の解体が終わるまでは、私に近づくなと言っただろ。	
\\	う 
\\	(おわ) 
\\	終	
\\	楽しむ	
\\	たのしむ	
\\	まどから、海の美しいけ色を楽しむことができます。	
\\	このプロジェクト、みんなすごく楽しみながらやってます。	
\\	うさぎちゃん、いい加減目を覚ましなって。タキシード仮面のことなんか忘れて、うさぎちゃんはうさぎちゃんの人生を楽しんだらいいんだよ!	
\\	楽しい, 
\\	う 
\\	楽しい, 
\\	楽	
\\	落ちる	
\\	おちる	
\\	あなたのけしゴム、つくえの下に落ちていますよ。	
\\	しけんに落ちた日に、ひろうから、はがぬけ落ちました。	
\\	「何してるの?私も入れて?」「本当?俺たち、ステアフォールっていう、棒人間がひたすら階段から落ちるゲームをしてるんだけど。」「おもしろそうじゃん。」	
\\	う 
\\	(お)!
\\	落	
\\	聞く	
\\	きく	
\\	テキストを聞きます。	
\\	この日本語は分からないから先生に聞きましょう。	
\\	ねえ、聞いて、エイミー。これはあなたのせいじゃないわ。だって、ケンカしたからって、他の人と寝ていい正当な理由にはならないでしょ。	
\\	う 
\\	(き) 
\\	聞	
\\	調べる	
\\	しらべる	
\\	今日の午後は、としょかんでせんそうについて調べました。	
\\	けいさつがコウイチの車の中を調べると、中から酒によったフグが出てきました。	
\\	カピバラが食べられるのかどうか、ちょっと調べてくれない?	
\\	う 
\\	(しら) 
\\	調	
\\	習う	
\\	ならう	
\\	日本に来てから、いろんな言葉を習いました。	
\\	うちのむす子は、しょ道を習っています。	
\\	今日は、世界一のフグクッキーの作り方を習いに行くのよ。それ、子供たち大好きよね。	
\\	う 
\\	習	
\\	出会う	
\\	であう	
\\	わたしたち、ちょっと出会うのが早すぎたみたいね。	
\\	きのうの夜、クラブで、れんあいかんじょうぬきでただ楽しみたいっていう男に出会ってさ、ハマっちゃいけないのは分かってるけど、めちゃくちゃタイプだったんだよね。	
\\	「ニコールって、今、付き合っている人いるの?」「ええ。エリックと1年間付き合っているわ。」「そうなんだ!彼とは、どこで出会ったの?」	
\\	出る 
\\	会う, 
\\	出 
\\	出 
\\	出合う. 
\\	出, 会	
\\	口調	
\\	くちょう	
\\	かの女は口調がキツイのでちょっと苦手です。	
\\	コウイチは、とてもきびしい口調で、「うんも実力のうちだ」と言った。	
\\	厳しい口調で責め立てられた。	
\\	口 
\\	口? 
\\	ち 
\\	く. 
\\	口, 調	
\\	開ける	
\\	あける	
\\	日本語のじしょを開けてください。	
\\	わたしのかれ氏は、車に乗る時は、いつもドアを開けてくれるよ。	
\\	「ねぇ、このワインボトルを開けてもらえないかな?」「いいよ。朝飯前だよ。」	
\\	う 
\\	(あ). 
\\	あ 
\\	開	
\\	読む	
\\	よむ	
\\	次の文を読んで、しつもんに答えてください。	
\\	ビエトとコウイチのひみつの交かん日記を内しょで読んでしまいました。	
\\	あんたの彼氏が書いた「旅の注意書き」読んだ?二、三本傘を持参するべきって書いてあるけど。	
\\	う 
\\	読	
\\	転がる	
\\	ころがる	
\\	あんたのけいたいなら、その辺に転がってたわよ。	
\\	ダリンは、トーフグオフィスの前にフグの死体が転がっているのを発見しました。	
\\	転がる石?イギリスのロックバンド、ザ・ローリング・ストーンズのことですか?	
\\	(ころ) 
\\	転	
\\	語る	
\\	かたる	
\\	次のセミナーでは、何について語る予定ですか?	
\\	田中さん、時間がないので、ぼうけんだんを語るのはこの辺で止めてください。	
\\	この事件を人に語ることはできない。	
\\	う 
\\	(かた). 
\\	語	
\\	投げ付ける	
\\	なげつける	
\\	むすめから、頭に花びんを投げ付けられて、七はりをぬう大けがを負いました。	
\\	せつ分の日にコウイチにまめを投げ付けてじゃ気をはらいました。	
\\	「サーモン、僕は君に夢中なんだ。」「フグ、私はあんたに怒り心頭よ。どうして私に向かってニンニクなんて投げ付けるのよ!」「ご…ごめん。鮭はニンニクの臭いが好きって聞いたことがあったんだ。」	
\\	投げる 
\\	付ける. 
\\	投, 付	
\\	起きる	
\\	おきる	
\\	父は、毎朝六時に起きます。	
\\	いつも朝起きるのめっちゃ早いですね。まだ四時ですよ。	
\\	三段腹を無くすために、毎朝起きたらすぐに100回腹筋をします。	
\\	う 
\\	起	
\\	求める	
\\	もとめる	
\\	トーフグは、ポーカーフェイスができる人を求めています。	
\\	時々、返金を求めるメールがとどきます。	
\\	の顧客は、ますます環境に優しい日本語学習製品を求めるようになってきている。	
\\	(もと) 
\\	求	
\\	役場	
\\	やくば	
\\	役場でしょう明しょをもらってきてください。	
\\	役場のかべ紙は、かべからはがれてきています。	
\\	役場に行って転出届を出してきたよ。	
\\	役, 場	
\\	転送	
\\	てんそう	
\\	する 
\\	そのメール、私にも転送してくれない?	
\\	海外に住んでいる日本人は、よく日本のウェブサイトから物を買って、海外の住所に転送しています。	
\\	郵便物をこちら住所の方へ転送して頂きたいんですが。	
\\	転, 送	
\\	思わず	
\\	おもわず	
\\	思わず本音を口にしてしまいました。	
\\	コウイチがすんげー大きな音のオナラをしたから、みんな思わず耳をふさいだんだけど、はなをふさいだ方がよかったね。	
\\	美味しそうな写真を見て、思わず涎が出た。	
\\	思	
\\	作業	
\\	さぎょう	
\\	する 
\\	なかなか作業がはかどらないんです。	
\\	きのうは、一日中作業ハンドブックを作る作業をしていました。	
\\	あなたのせいで、製造作業が全て停止してしまったってこと、分かっていますか?	
\\	作, 業	
\\	上級	
\\	じょうきゅう	
\\	上級クラスについていけずにこまっています。	
\\	コウイチ子ちゃん、上級生からいじめられているみたいなんです。	
\\	英語が母国語でない人で、ネイティブレベルや上級の英語を話す人もいれば、そうでない人もいます。	
\\	上, 級	
\\	味方	
\\	みかた	
\\	する 
\\	の 
\\	トーフグだけは、味方してくれると思ってたんだけどな。	
\\	実は、コウイチは味方のふりをしたてきだったんだよ。	
\\	お前は一体どっちの味方なんだ?	
\\	味 
\\	方. 
\\	味, 方	
\\	売り上げ	
\\	うりあげ	
\\	だんだん売り上げがへってきた。	
\\	アメリカ人人気はいゆうが主えんしたハリウッドえいがの
\\	は、日本だけで売り上げが八百万まいになった。	
\\	あの性差別の広告のせいで、売り上げが激減した。	
\\	(上げ) 
\\	売 
\\	売る. 
\\	上げ 
\\	上げる, 
\\	売, 上	
\\	不良	
\\	ふりょう	
\\	な 
\\	の 
\\	ビエト、今日はちょっと不良な感じのかみがただね。	
\\	ひんしつ不良のポテトチップスが半額で売ってたので買って食べたらおなかをこわしました。	
\\	あの不良少女は、急に成績が大幅に下がった後、ギャングの一員になったんです。	
\\	不, 良	
\\	初歩	
\\	しょほ	
\\	の 
\\	ワニカニの使い方の初歩を教えるビデオを作りました。	
\\	父はわたしに、日本のしょうぎというゲームの初歩をおしえてくれました。	
\\	フランス語を初歩から学び始めました。	
\\	初, 歩	
\\	戦い	
\\	たたかい	
\\	今日のコウイチ、なんだか戦いにくたびれてしまったへいしみたいだね。	
\\	しずかだろうがさわがしかろうが、戦いはきらいなんだよ。	
\\	彼はマクドナルドに行く途中、血みどろの戦いに遭遇したが、その後そのままビッグマックを食べに行った。	
\\	い, 
\\	(たたか)!
\\	戦	
\\	戦場	
\\	せんじょう	
\\	戦場にいたって、そりゃあだれだって命はおしいさ。	
\\	ビエトのせいで、トーフグのオフィスはヤクザの戦場と化した。	
\\	コウイチは頼れるやつだ。たとえ戦場にいたとしても、俺達を決して見捨てやしないよ。	
\\	戦, 場	
\\	予め	
\\	あらかじめ	
\\	こうなることは分かっていたので、予め用意しておきました。	
\\	ひ行きに乗りおくれないように、予め前日にオンラインでチェックインしておきました。	
\\	予め予習しておいた甲斐がありました。	
\\	(あらかじ) 
\\	予	
\\	初級	
\\	しょきゅう	
\\	の 
\\	来週、初級から中級に進級します。	
\\	エトエトは、初級、中級、上級の三つのレベルに分かれていました。	
\\	私の日本語はまだ初級レベルですが、一生懸命勉強しています。	
\\	中級 
\\	上級 
\\	初, 級	
\\	伝記	
\\	でんき	
\\	の 
\\	この投手の伝記、おも白かったよ。	
\\	その作家の伝記に書かれていたことは、とても意外だった。	
\\	まあ、そう思うのは、最近読んだ伝記の影響かもしれません。	
\\	伝, 記	
\\	生物学	
\\	せいぶつがく	
\\	わたしの一番すきな科目は生物学と生化学です。	
\\	川田さんのお兄さんは生物学をせんもんとする学者さんです。	
\\	デザイナーになりたいなら、生物学も学んだ方がいい。	
\\	生物 
\\	生, 物, 学	
\\	乗り場	
\\	のりば	
\\	ゴンドラ乗り場はどこですか?	
\\	空港のタクシー乗り場でタクシーを待っているところです。	
\\	今、家の前のバス乗り場で通り過ぎる車を見ながらダラダラしているよ。	
\\	乗る 
\\	乗る 
\\	乗り.	乗, 場	
\\	注意	
\\	ちゅうい	
\\	する 
\\	今日、上しに注意されちゃったよ。	
\\	日本語で何かを読んでいる時には、おも白い語句くや文ぽうに注意を払って下さい。	
\\	私は彼が間違った道を進んでるんじゃないかと心配しているのよ。注意したんだけど、聞く耳持たずって感じで。	
\\	注, 意	
\\	洋服	
\\	ようふく	
\\	洋服を貸してくれて有難う。	
\\	和服にしようか洋服にしようかまよってます。	
\\	早く洋服を着ないと、学校に遅れるわよ。	
\\	洋, 服	
\\	共通点	
\\	きょうつうてん	
\\	コウイチとは共通点が全くないから、何を話せばいいか分からないよ。	
\\	トウフとフグの、おどろくべき共通点を発見しました。	
\\	「コウイチが熟女好きっていう噂を聞いたんだけど。」「マジかよ!もしそれが本当なら、俺とコウイチに共通点ができるってことになるぜ。」	
\\	共, 通, 点	
\\	仲間	
\\	なかま	
\\	今はテスト週間中だから、終わるまで仲間とあそぶのはおあずけです。	
\\	仲間たちとは、兄弟みたいにいつも一しょにいて、ふざけ合って、おたがいにからかい合ってるよ。	
\\	学校での厳しい一日を終えて、仲間たちとリラックスをした。	
\\	仲 
\\	間 (ま) 
\\	(ま). 
\\	仲, 間	
\\	競争	
\\	きょうそう	
\\	する 
\\	の 
\\	今日の朝ごはんは、昨日のパン食い競争のパンです。	
\\	今は競争とかあんまりする気分じゃないんだけど。	
\\	あいつら、もうちょっと落ち着いた方がいいよ。二人共負けず嫌いのくせに、いっつも二人で競争してるんだぜ。	
\\	競, 争	
\\	便利	
\\	べんり	な 
\\	このアプリ、すごく便利だよね。	
\\	何か子育てに便利なアイテムを知っていたらぜひ教えて下さい。	
\\	生活は日に日に便利になっていく。	
\\	便, 利	
\\	共同	
\\	きょうどう	
\\	する 
\\	の 
\\	トーフグと日清が、豆腐と河豚味のカップラーメンを、共同開発するらしいよ。	
\\	トーフグは、北朝せんと共同して、金色の金正日人形を生さんしています。	
\\	彼は共同の貯金をこっそり使い果たしてしまった。	
\\	共, 同	
\\	別	
\\	べつ	
\\	の 
\\	コウイチが、別の女と歩いているところを見たの。	
\\	それとこれとは話が別だろう。	
\\	その場合は別として、このやり方で男女を別にするのは通じょうもんだいありません。	
\\	別	
\\	別人	
\\	べつじん	
\\	コウイチはお酒を止めてから、すっかり別人になったね。	
\\	友だちだと思って声をかけたら、かん全に別人だった。	
\\	コウイチは昔は
\\	48をよくディスっていたのに、今はすっかり大ファンになっちまった。まるで別人のようだよ。	
\\	別, 人	
\\	教育	
\\	きょういく	
\\	する 
\\	の 
\\	今日行くのは教育センターです。	
\\	日本では、小学校教育と中学校教育がぎむ教育です。	
\\	「教育があって常識がないよりも、教育がなくて常識がある方がはるかによい。」って誰の言葉だっけ?	
\\	教, 育	
\\	入学試験	
\\	にゅうがくしけん	
\\	入学試験を終えた息子が、ニコニコしながら飛んで帰ってきました。	
\\	えーっ!入学試験代ってこんなにかかるの?は産しちゃうよ〜。	
\\	ええっ!入学試験に落ちちゃったの?	
\\	(入学) 
\\	(試験). 
\\	入, 学, 試, 験	
\\	特に	
\\	とくに	
\\	特に病院ではしずかにしなければいけません。	
\\	このメロンは特にかたいですね。まるでカボチャのようだ。	
\\	「お夕飯は何が食べたい?」「何でもいいよ。特にこだわりはないよ。」	
\\	(に) 
\\	特	
\\	運命	
\\	うんめい	
\\	運命をしんじますか。	
\\	二人は運命の赤い糸でむすばれているって思ってたけど、だんだん心がはなれていってるみたい。	
\\	さん天国行き。用を足した後のトイレに落としちゃいました。まぁ、これが
\\	さんの運命だったのかもしれないけどさ。でも、猛烈に悲しいぜよ...。	
\\	運, 命	
\\	心持ち	
\\	こころもち	
\\	何事も心持ち次だいだと思うんだよね。	
\\	今日あった男に心持ちほねっぽいねって言われたんだけど、何なのアイツ。	
\\	あの子は、心持ちのよい素直な子です。	
\\	心, 持	
\\	気持ち	
\\	きもち	
\\	あなたの気持ちはよく分かります。	
\\	気持ちがわるくてはきそうです。	
\\	これ、つまらないものですが。ささやかな感謝の気持ちです。どうぞ、開けてみてください。	
\\	気持ちいい 
\\	気 
\\	持ち 
\\	気, 持	
\\	指	
\\	ゆび	
\\	指がきれいですね。	
\\	ええっ、この間指のえがき方のレッスンを受けてから、もう一ヶ月もたったの!?	
\\	突然中指が痛くなったんだけど。	
\\	(ゆび) 
\\	指	
\\	首位	
\\	しゅい	
\\	の 
\\	学力テストで首位に立った。	
\\	首位打者がケガをしたせいで、首位のチームに6ゲームさを付けられてしまった。	
\\	その老人は、両チームの首位争いには全く興味が無いようだった。	
\\	首, 位	
\\	一位	
\\	いちい	
\\	の 
\\	一位になったら、一週間毎日からあげを作ってあげる。	
\\	子供がピアノコンクールで一位をとりました。	
\\	お母さん!僕、あのテストで滅茶苦茶良い点をとったんだよ!なんとクラスで一位だったんだ!	
\\	一, 位	
\\	昔話	
\\	むかしばなし	
\\	これは日本の昔話ですか。	
\\	まるで昔話の中に出てくるおひめさまみたいね。	
\\	バス待合所でバスを待ちながら、昔話の絵本を子どもに読み聞かせた。	
\\	はなし 
\\	ばなし.	昔, 話	
\\	食べ物	
\\	たべもの	
\\	安田さんは、食べ物をおねがいします。	
\\	かなしい時や、おちこんでいる時は、いつも食べ物がのどを通らなくなってしまうんです。	
\\	食べ物を粗末にするとバチが当たりますよ。	
\\	食, 物	
\\	友好	
\\	ゆうこう	
\\	このやきいもは、友好のしるしです。	
\\	トーフグは、日米の友好をそくしんする役わりをはたした。	
\\	カナさんって、とても友好的で経験豊富な日本語教師じゃない?でもそれだけじゃなく、実はヤクザの隠れメンバーでもあるんだよ。	
\\	友, 好	
\\	仕返し	
\\	しかえし	
\\	する 
\\	くやしいから、なんか仕返ししたいな。	
\\	キャットウーマン!あんたって本当に最低の女だね!いつか仕返ししてやるわ!	
\\	プリンを食べられた仕返しに、コウイチの大すきなヨーグルトを食べてやった。	
\\	仕 
\\	返し 
\\	返す 
\\	仕, 返	
\\	神	
\\	かみ	
\\	あなたは神をしんじますか?	
\\	神をいつまで待たせるつもりなの?	
\\	神様は水の上を歩けるかもしれないけど、俺は陸の上を泳ぐことができるんだぜ。	
\\	風 
\\	神	
\\	良い	
\\	よい, いい	い 
\\	お羊ざと天びんざは相しょうが良い。	
\\	けんかするほど仲が良いって言うだろ?	
\\	良い考えだけど、実際はそれをするには資金が足りないんだよね。	
\\	い 
\\	(よ). 
\\	良	
\\	気持ちいい	
\\	きもちいい	
\\	い 
\\	この羽毛ぶとん、気持ちいい。	
\\	家がきれいだと気持ちいいよね!	
\\	このマッサージ、気持ちいい。このいた気持ちよさがたまらない。	
\\	(気持ち) 
\\	いい 
\\	気持ち, 
\\	気 
\\	持ち 
\\	気, 持	
\\	努力	
\\	どりょく	
\\	する 
\\	「努力はうら切らない」という言葉があるけど、あれはうそだね。	
\\	努力のかいもむなしく、たからくじには当たらなかった。	
\\	努力しても、結果が出なくては意味が無い。	
\\	努, 力	
\\	苦労	
\\	くろう	
\\	する 
\\	な 
\\	ワニカニに出会うまでは、おれの人生は苦労ばかりだった。	
\\	「このコウイチのサイン、苦労してやっと手に入れたんだよ。」「それはご苦労なことですね。」	
\\	うちのお風呂って、シャワーの位置が固定されちゃってるから、浴槽を洗って流すときにかなり苦労するんだよね。	
\\	苦, 労	
\\	苦手	
\\	にがて	
\\	な 
\\	ちんもくが苦手です。	
\\	「コウイチが苦手なことって何ですか?」「ぼくは歌が苦手です。」	
\\	俺とお前は昔は切っても切れない仲だっただろ。でもさ、 ほら、今は昔とは違うというか何というか、えっと…正直いって、お前のこと苦手なんだよ。	
\\	苦い 
\\	手 
\\	苦い (にがい) 
\\	手 (て). 
\\	にがい 
\\	苦, 手	
\\	高級	
\\	こうきゅう	
\\	な 
\\	の 
\\	高級バージョンのワニカニがものすごく売れたたおかげで、トーフグの名前に箔が付きました。	
\\	高級茶を一ぱいごち走させて下さい。	
\\	この高級レストランはビップ専用となっているので、入店するには招待状か紹介状が必要になります。	
\\	高, 級	
\\	平仮名	
\\	ひらがな	
\\	平仮名は読めるけど書けません。	
\\	平仮名でかいていただけませんか?	
\\	ビエトが、平仮名を五分で全部おぼえられるアプリを開発しました。	
\\	仮名 
\\	平, 仮, 名	
\\	使い方	
\\	つかいかた	
\\	もう少し、言葉の使い方に気を付けた方がいいでしょ。	
\\	アイフォンの使い方が分かりません。	
\\	一緒に仕事ができるように、俺の犬に金槌の使い方を訓練してみようと思うんだ。	
\\	使う 
\\	使う 
\\	方. 
\\	使, 方	
\\	〜付き	
\\	つき	
\\	このそうじきは、一年間のほしょうしょ付きですよ。	
\\	一ぱく二食付きのりょかんをさがしています。	
\\	この部屋は冷蔵庫、洗濯機付きです。	
\\	付 
\\	付く 
\\	付き, 
\\	付く 
\\	付ける.	付	
\\	不便	
\\	ふべん	
\\	な 
\\	この辺りでは、車を持っていないとかなり不便ですよ。	
\\	ワニカニの使い勝手はいかがですか?ご不便はございませんか?	
\\	俺の隠れ家へようこそ。ビールは冷蔵庫に入ってるぜ。ちょっと不便かもしれねぇが、煙草は外で吸ってくれよな。	
\\	不, 便	
\\	中級	
\\	ちゅうきゅう	
\\	の 
\\	中級のなんい度はどのぐらいですか?	
\\	明日から、また中級クラスにぎゃくもどりです。	
\\	英語力が中級や初級の人たちと会話をする能力は、実は、訓練して身につくスキルなんです。	
\\	中, 級	
\\	仕事	
\\	しごと	
\\	する 
\\	の 
\\	この仕事は、交通ひこみですか?	
\\	赤ちゃんの仕事は泣くことです。	
\\	「時間内に、この仕事を終えられると思うか?」「さあ、それはどうかな。」	
\\	仕 
\\	事. 
\\	事 
\\	こと, 
\\	ごと.	仕, 事	
\\	丁度いい	
\\	ちょうどいい	
\\	い 
\\	あーっ丁度いい所に!	
\\	大きすぎず、小すぎず、本当に丁度いいサイズです。	
\\	このぐらいの温度が丁度いいですね。	
\\	丁度 
\\	丁度 
\\	丁, 度	
\\	労働者	
\\	ろうどうしゃ	
\\	コウイチ社長は、肉体労働者の不足についてどうお考えなんですか?	
\\	一体、お前と労働者たちの間に、何があったんだよ。	
\\	ちょっとコピー機使わせてもらってもいい?新しい労働者を雇うために、チラシを作ってるんだけど。	
\\	労, 働, 者	
\\	命	
\\	いのち	
\\	コウイチはわたしたちみんなの命のおん人です。	
\\	おねがいですから、ワニカニでかん字をべん強することよりもご自分の命を大切にしてください。	
\\	セーラームーンの命が危ない!	
\\	(いのち) 
\\	命	
\\	好き	
\\	すき	
\\	な 
\\	テキストフグのことが好きです。でも、ワニカニはもっと好きです。	
\\	好きな人と家でゴロゴロするのが大好きです。	
\\	そのままの君が好きなんだ。だから、他のやつのために自分を変えようなんてしないでくれよ。	
\\	好 
\\	好	
\\	文字通り	
\\	もじどおり	の 
\\	かれは文字通り、「天才」です。	
\\	トーフグの社いんになるには、文字通りとうふとフグが好物でなければならない。	
\\	ツンデレとは、文字通り「人前ではツンツンしているけど二人の時はデレデレすること」です。	
\\	文字 
\\	通り? 
\\	文字 
\\	通り, 
\\	とおり 
\\	どおり 
\\	文, 字, 通	
\\	昔	
\\	むかし	
\\	の 
\\	コウイチは昔はけんけんをするのがとく意だった。	
\\	昔、一回だけ、右足に魚の目ができたことがあるよ。	
\\	「あなた、今夜の予定は?」「昔の飲み仲間とちょっと出かけてくる予定だよ。」	
\\	昔々に… 
\\	むかし 
\\	昔	
\\	負け犬	
\\	まけいぬ	
\\	自分が、一日中家にいて、テレビばっかりみてる負け犬だってことは自かくしているよ。	
\\	あの負け犬が明日からおれの仕事を引きつぐとか、本当にがまんできないよ。	
\\	この負け犬め!キモいんだよ!	
\\	負ける 
\\	負ける 
\\	犬 
\\	負, 犬	
\\	低い	
\\	ひくい	い 
\\	低い声の男の人が好きです。	
\\	うちのむすめは、平きんよりやや身長が低いんです。	
\\	「ねぇ、ゆきちゃん。あんた、あまり元気じゃなさそうね。」「うん。そうなの。低血圧症って診断されたわ。」「何それ?」「つまり、私の血圧がめちゃくちゃ低いってこと。」「どれくらい低いの?」「えっと、今朝は上が80で下が40ぐらいだったかな。」	
\\	い 
\\	低	
\\	労働	
\\	ろうどう	
\\	する 
\\	の 
\\	あなたの労働時間は週何時間ですか?	
\\	うちの主人は年なので、肉体労働はちょっとさすがにきびしいかも。	
\\	俺には何が必要か知ってるかい?趣味の部屋と新しい車だよ。労働の対価をきちんと受け取る必要があるだろ?	
\\	労, 働	
\\	意味	
\\	いみ	
\\	する 
\\	コウイチのあの意味あり気なかおを見て。	
\\	この目が何を意味するかは分かりますか?	
\\	「あなたって、もっと賢いと思っていたわ。」「おい、それってどういう意味だよ!?」	
\\	意, 味	
\\	新た	
\\	あらた	
\\	な 
\\	コウイチが新たなオムツせいぞうぎじゅつを発明しました。	
\\	九月は自分にとって新たなスタートになりそうな予かんがする。	
\\	この小説は、読み返す度に新たな発見がある。	
\\	(あら) 
\\	新	
\\	別に	
\\	べつに	
\\	温かいものはふくろを別にしてください。	
\\	別に会社で人間関係に苦労はしてないよ。	
\\	「おお、フグ。最近どうよ?」「別に変わりないね。ワニ君、君はどうだい?」	
\\	(に), 
\\	別	
\\	発売中	
\\	はつばいちゅう	
\\	の 
\\	年末ジャンボたからくじはまだ発売中ですか?	
\\	これは、トーフグストアで発売中の手ぬぐいです。	
\\	少年ジャンプ6月号、ただ今絶賛発売中。	
\\	発売, 
\\	中 
\\	発売 
\\	中 
\\	発, 売, 中	
\\	老人	
\\	ろうじん	
\\	の 
\\	日本では、老人の一人ぐらしや二人ぐらしがふえています。	
\\	あの老人とは仲直りした?	
\\	うわ!あの老人めちゃくちゃイケメンでマッチョなんだけど。	
\\	古い人, 
\\	古い 
\\	老, 人	
\\	意見	
\\	いけん	
\\	する 
\\	の 
\\	意見の出しおしみはしないで下さい。	
\\	いくら仲の良い友だちでも、ちょっと正直には意見しづらいよね。	
\\	これは俺の個人的な意見なんだけど、たまに、日本は本当に変わった国だなぁと思うんだよね。	
\\	意見, 
\\	意 
\\	見
\\	(けん) 
\\	発見 
\\	意, 見	
\\	注文	
\\	ちゅうもん	
\\	する 
\\	の 
\\	スタバの注文の仕方がいまいちよく分からないんだけど。	
\\	まさか注文するのをわすれてたりしないよね。	
\\	それは無理な注文だが、やるっきゃない。	
\\	注, 文	
\\	命令	
\\	めいれい	
\\	する 
\\	コウイチ王の命令には全てしたがわなくてはいけない。	
\\	はぁ?おねがいだった?ふざけんなよ。こっちには命令にしか聞こえなかったっつーの。	
\\	「ほら!何をダラダラ待ってるの? 走って!」「俺に命令するんじゃねぇよ!」	
\\	命, 令	
\\	太平洋	
\\	たいへいよう	
\\	トーフグチームは、ごうか客せんで太平洋をわたる社いんりょ行をけいかくしています。	
\\	冷ぞうこの中の太平洋クロマグロは、カナエちゃんのお土産です。	
\\	現在、ほとんどの河豚は繁殖期である春に収穫され、太平洋に浮かぶ籠の中で養殖されます。	
\\	太, 平, 洋	
\\	用意	
\\	ようい	
\\	する 
\\	万が一の時にそなえて、マッチとろうそくを用意しました。	
\\	毎週金よう日はきゅう食がないから自分でおべん当を用意しないといけないんです。	
\\	出所祝いの鍋パーティーの用意は全てこちらで致しますので、あなたは保釈金の用意をお願いします。	
\\	用, 意	
\\	支度	
\\	したく	
\\	する 
\\	学校へ行く支度ができたらよんでください。	
\\	母は台所で夕はんの支度を、父は部屋でりょ行の支度をしています。	
\\	いつまで寝てるの!早く起きて支度しなさい!!	
\\	度 
\\	(したく). 
\\	支, 度	
\\	公共	
\\	こうきょう	
\\	の 
\\	ぼう公共しせつでボランティアをしていたことがあります。	
\\	公共とは一体何なのか、今一度考え直してみてほしい。	
\\	ねぇ、レイとアミ。これは私が口出しするようなことじゃないだろうけど、あんた達、公共の場で一緒にいるところを見られちゃいけないんじゃなかったの?だって、セレブでしょ?	
\\	公, 共	
\\	〜位	
\\	い	
\\	今年は何位だろう?	
\\	コウイチは黒ニンニクのしょうひりょうにかんしては、ポートランドで一位かもしれない。	
\\	マラソン大会で一位になったので、両親がマウンテンバイクを買ってくれた。	
\\	位	
\\	電波	
\\	でんぱ	
\\	このビルはコンクリでできてるので、む線の電波はとどきません。	
\\	公きょうの電波でそんな事言うなんて、最低だね。	
\\	ごめん、よく聞こえないよ。ここは電波が悪いみたい。	
\\	は 
\\	波 
\\	ぱ, 
\\	電, 波	
\\	受付	
\\	うけつけ	
\\	あの受付の女せい、左足に大きな魚の目があるらしいよ。	
\\	明日から、シュワちゃんがトーフグの受付をしてくれます。	
\\	え?知らない。受付で聞いてない?	
\\	受ける 
\\	付ける. 
\\	る
\\	受 
\\	うけ 
\\	付 
\\	つけ 
\\	うけつけ. 
\\	受, 付	
\\	仲直り	
\\	なかなおり	
\\	する 
\\	過去のことは水に流して、仲直りしようよ。	
\\	そんなにかんたんに仲直りは出来ないよ。	
\\	一体全体、コウイチはどうやってマリオブラザーズを仲直りさせたの?	
\\	仲 
\\	直り 
\\	直る. 
\\	仲, 直	
\\	仲	
\\	なか	
\\	母親とは、あんまり仲が良くないんだよね。	
\\	きっ茶店で仲むつまじい老ふうふをみて、ちょっとほっこりしました。	
\\	キンニクマ:「うわ!お前らちょっと仲良すぎじゃね?」 ワニカニとトーフグ:「なんで?ただのマブダチだよ。」	
\\	仲	
\\	金持ち	
\\	かねもち	
\\	かれ、お金持ちなんだけど…	
\\	ちょうお金持ちとけっこんしたい。	
\\	若い頃、お金を稼ぐために、近所のお金持ちの人たちの使い走りをしていました。	
\\	お金? 
\\	金 
\\	持ち 
\\	も 
\\	持つ. 
\\	かね 
\\	お金 
\\	持ち 
\\	お金持ち 
\\	金, 持	
\\	屋上	
\\	おくじょう	
\\	今日、屋上でバーベキューをします。	
\\	夜きん明けにもかかわらず、屋上できんトレをしました。	
\\	屋上に上がって、天からふり注ぐ星の光を一緒にあびないかい?	
\\	屋 
\\	(おく). 
\\	屋, 上	
\\	海岸	
\\	かいがん	
\\	の 
\\	花火大会の日に、ゆかたをきて海岸ぞいを歩いているカップルを見るのがすきです。	
\\	海岸にあるいいカフェを知ってるんだけど、よかったら一しょにお茶しない?	
\\	「今日、私達と一緒に海岸に行かない?」「またの機会にするよ。」	
\\	海, 岸	
\\	秒	
\\	びょう	
\\	コウイチにラインしたら秒で返事きてわらえた。	
\\	大体いつも、ふとんに入ったら三秒でねてる。	
\\	あなたの発音をたったの十二秒でネイティブレベルにしてみせますよ。	
\\	秒	
\\	安売り	
\\	やすうり	
\\	する 
\\	あんまり自分を安売りすると、後で辛くなりますよ。	
\\	このジャケット、めっちゃ好きだけど、ちょっとよゆうがないんですよね。近々安売りするよていとかってないですかね?	
\\	そうだよ。君に売った時にはそのテレビは壊れていたさ。でも、だからこそ安売りしたんじゃないか。	
\\	安い 
\\	売る 
\\	安, 売	
\\	売り手	
\\	うりて	
\\	あの売り手にだまされました。	
\\	空ぜんの売り手市場ですね。	
\\	その家の売り手は、「近所の人が本当にひどいやつなので、家を手放すことにしたんだ」と、私に話してくれました。	
\\	売, 手	
\\	別の	
\\	べつの	の 
\\	別のアプローチをトライしてみない?	
\\	これとそれは、別のもんだいだと思います。	
\\	私は家を二軒所有している。三ヶ月ほど前に購入した場所と、谷底にもう一件別の家がある。	
\\	の 
\\	別 
\\	別の漢字, 
\\	別	
\\	別々	
\\	べつべつ	
\\	な 
\\	の 
\\	二人はふた子だが、いつもそれぞれ別々なことを考えてる。	
\\	サボテンとは別々にしてうえて下さい。	
\\	「私達、本当に別々の道を進まなきゃいけないの?」「ああ。でもまたすぐ会えるさ。」	
\\	べつ 
\\	別, 々	
\\	共有	
\\	きょうゆう	
\\	する 
\\	の 
\\	「あのヨットは素敵ですね。田中さんのですか?」 「兄貴と共有なんですよ。」	
\\	このワンピースのフィギュアは、コウイチとビエトの共有のざいさんです。	
\\	子育ては、ふうふ間でじょうほうをきちんと共有した方が上手く行きますよ。	
\\	共, 有	
\\	神道	
\\	しんとう	
\\	これは、神道のぎしきの一つです。	
\\	きのうの夜、ひたすら神道についてコウイチに語るゆめを見ました。	
\\	どうして神道には絶対的な神がいないのですか。	
\\	神 
\\	道 
\\	(どう) 
\\	とう 
\\	とう, 
\\	神, 道	
\\	神社	
\\	じんじゃ	
\\	神社に初もうでに行ったら、コウイチとビエトがなぐり合いのけんかをしていた。	
\\	日本の神社で買ったおまもりのこと、すっかりわすれてた。	
\\	よぉ、中々イケてる自撮りじゃん。でも、お前、なんで学校じゃなくて神社になんていたんだよ?	
\\	しん 
\\	神.
\\	神 
\\	じん 
\\	(じんじゃ). 
\\	神, 社	
\\	買い物	
\\	かいもの	
\\	の 
\\	わたしはスーパーで買い物をしています。	
\\	買い物はいつもネットショッピングですませています。	
\\	今日はクリスマスの買い物に行くの!すごくワクワクするわ!	
\\	買う 
\\	買い 
\\	買う 
\\	買い 
\\	物 
\\	もの, 
\\	買, 物	
\\	見物	
\\	けんぶつ	
\\	する 
\\	おも白そうだから、見物してみましょう。	
\\	オフィスのまどから、クリスマスパレードを見物しました。	
\\	母ちゃんなら、東京見物に出かけたぜ。	
\\	見, 物	
\\	売り切れ	
\\	うりきれ	
\\	の 
\\	そのざっしは、売り切れです。	
\\	トーフグのこうえん会のチケットが売り切れとしってなきそうになった。	
\\	ジョーンズさん、本当に申し訳ないのですが、動物クッキーは売り切れてしまいました。	
\\	売, 切	
\\	時々	
\\	ときどき	
\\	の 
\\	私は時々トーフグホテルの最上階のレストランでトーフグの記事を読みます。	
\\	はま辺では、時々、そよ風がわたしのほほをやさしくなでていった。	
\\	トーフグって、時々お客さんともめるんだよね。あるトラブルが、死人を出すなぐり合いのけんかになったこともあるから、ほんと、気を付けた方がいいよ。	
\\	時, 
\\	時 
\\	時, 々	
\\	特別	
\\	とくべつ	
\\	な 
\\	の 
\\	タランティーノの新作の、特別ゆう待けんをもらいました。	
\\	これがトーフグデザインの特別な車ですか。	
\\	あの子って、特別かわいいわけではないのにモテるよね?	
\\	特, 別	
\\	見事	
\\	みごと	
\\	な 
\\	見事なまつの木ですね。	
\\	お見事!いやー、実に見事なシュートだった!	
\\	気持ちいいぐらい見事にフラれたよ。	
\\	見, 事	
\\	成功	
\\	せいこう	
\\	する 
\\	気にするなって。失ぱいは成功のもとだよ。	
\\	新しいお仕事での成功をおいのりしています。	
\\	「ビジネスで成功したければ、型にはまらない考え方をしなさい」と言われて、カレー味の歯磨き粉を思いついたんだ。	
\\	成, 功	
\\	作戦	
\\	さくせん	
\\	午後九時に明日の作戦について話したい。	
\\	もう一度、作戦をねり直そう。	
\\	コウイチは、トーフグには新しいはん売作戦がひつようだと考えている。	
\\	作, 戦	
\\	物語	
\\	ものがたり	
\\	する 
\\	の 
\\	物語の主人公にでもなったつもりでいるんだろう。	
\\	竹とり物語は日本の有名な物語です。	
\\	今物語を書いているんです。悲しい子犬フェチの男の話です。	
\\	物 
\\	もの, 
\\	語 
\\	ものがたり 
\\	(がたり) 
\\	物, 語	
\\	戦車	
\\	せんしゃ	
\\	うちの市では、ゴミしゅう集車の代わりに戦車が使われることになりました。	
\\	この戦車、そろそろせん車しないとなぁ。	
\\	「俺の戦車のタイヤ、どう?」「もう少し空気を入れるか、戦車用の履帯に換えた方がいいと思いますよ。」	
\\	戦, 車	
\\	味	
\\	あじ	
\\	味がお口に合うといいんですが。	
\\	しょう味きげんをすぎたから、味はちょっとかわってるかもしれないけど、まだ食べられると思うよ。	
\\	このフグの刺し身の味、最高だよ!まさにこれだって感じだね!	
\\	(あじ). 
\\	味	
\\	最初	
\\	さいしょ	
\\	の 
\\	最初はとりあえず生で!	
\\	コウイチの最初のかの女の名前はコウイチ子でした。	
\\	お前本当に下手くそだな!初心者でもこんなに下手くそじゃねぇだろ。俺が最初にこのゲームをプレイした時は、お前よりずっと上手かったわ。	
\\	最, 初	
\\	初回	
\\	しょかい	
\\	コウイチの新曲の、初回げん定ばんの
\\	を手に入れました。	
\\	初回のお客さまには、特別かかくでごていきょうしています。	
\\	えっ、君が日本語学習クラブに参加するのは今回が初回なんだ?よし、みんな。彼は初心者だから手加減してあげようぜ。	
\\	初, 回	
\\	最低	
\\	さいてい	
\\	な 
\\	の 
\\	よくこんな最低最あくの酒がのめるよね。	
\\	今までで最低のキスだったかも。	
\\	えぇっ!あの試験で最低点を取ったの?参ったなぁ。	
\\	最, 低	
\\	本物	
\\	ほんもの	
\\	の 
\\	さい近の
\\	は、パッと見たかんじ本物にしか見えないのがマジですごいと思う。	
\\	トーフグが本物そっくりのにせ札を作っているという黒いうわさをききました。	
\\	小さいころ、父さんはいつも僕に「本物の男は誠実でいつづける」って言ってたけど、結局浮気をして、両親は離婚することになったんだ。	
\\	本 
\\	物). 
\\	本, 物	
\\	西洋	
\\	せいよう	
\\	の 
\\	西洋には、トーフグのファンはどのぐらいいますか。	
\\	西洋の国々にくらべて、インドのりこんりつはとても低い。	
\\	これは、西洋の妖怪です。	
\\	西, 洋	
\\	指定する	
\\	していする	する 
\\	ミーティングの場所は、お客さまが指定しました。	
\\	ウェブでチケットを予やくしても、自分でせきを指定することはできますか。	
\\	向こうに指定された喫煙エリアがあるから、私にじゃなくてそこに行って喫煙者がどれだけ臭いかについて文句を言った方がいいんじゃない。	
\\	する 
\\	指定 
\\	する 
\\	指, 定	
\\	試す	
\\	ためす	
\\	この薬を試してみます。	
\\	どのくらい速く日本語の早口言葉を言えるか、試してみようよ。	
\\	肘を舐めれるかどうかは分からないけど、試しにやってみるよ。	
\\	う 
\\	(ため) 
\\	試	
\\	成る	
\\	なる	
\\	取りこし苦労だよ。成るように成るさ。	
\\	日食を直に見てしまったため、目が不自由に成りました。	
\\	うわーー!!コウイチがとうとうアメリカ大統領に成ったよ。「為せば成る。為さねばならぬ何事も、成らぬは人の為さぬなりけり」だね!	
\\	う 
\\	(な).	成	
\\	育つ	
\\	そだつ	
\\	わたしは東京で生まれ育ちました。	
\\	わたしはとてもきびしい家ていで、ベーコンを食べて育ちました。	
\\	私は明治神宮がとっても好きなの。森が、自然の森として育つように計算して造られているので、とても美しいし、自然的な空間なの。	
\\	う 
\\	育てる. 
\\	(そだ). 
\\	育	
\\	返る	
\\	かえる	
\\	われに返ると、かれはトーフグオフィスのソファーに横たわっていた。	
\\	失くしたと思っていた大事な母の形見のペンダントが気づいたら手元に返ってきていた。	
\\	貸したお金がちゃんと返ってくると思ったら大間違いよ。	
\\	返す, 
\\	る 
\\	る 
\\	返す, 
\\	返	
\\	伝わる	
\\	つたわる	
\\	トーフグに入社する前のビエトの様々な伝せつが社員に伝わっています。	
\\	日本語が上手に話せなくても心で話せばきっと相手に伝わりますよ。	
\\	忍者トーフグ組が夜に城へ忍び寄ったということが、コウイチ将軍に伝わった。	
\\	う 
\\	伝える.	伝	
\\	放送する	
\\	ほうそうする	する 
\\	このドラマが放送されるのは、久しぶりです。	
\\	そのし合のさい放送なら、今日五時から
\\	で放送される予定ですよ。	
\\	コウイチ社長は今晩テレビで重大な決定を放送することになっている。	
\\	する 
\\	する 
\\	放, 送	
\\	追う	
\\	おう	
\\	日本には、二とを追うものは一とをもえず、ということわざがあります。	
\\	コウイチは、いつも最新のりゅう行を追っている。	
\\	どんなに大変でも、私は自分の夢を追い続けるつもりなの。もし私の夫になりたいなら、そのことを理解してもらいたいの。	
\\	う 
\\	追	
\\	通う	
\\	かよう	
\\	えっ、北海道からおきなわまで毎日通ってるの?	
\\	コウイチは、トーフグのオフィスまでハーレーで通っています。	
\\	私の妹は、四月からトーフグ学園に通うことになります。	
\\	う 
\\	通る 
\\	(かよ). 
\\	通	
\\	争う	
\\	あらそう	
\\	わたしの事で争うのはよして!	
\\	トーフグの社いんたちは、社長のざをねらって争っています。	
\\	あなたって本当に落ち着いているし楽しいし、あなたと遊ぶのは最高だわ。しかも、私たち言い争ったこともまだ一度も無いわよね。それってすごくない?	
\\	う 
\\	(あらそ) 
\\	争	
\\	競う	
\\	きそう	
\\	子どもたちは、新しい先生の所に競うようにかけよった。	
\\	トーフグはユニークすぎて、競う相手がいないんですよ。いわゆる一人勝ちってやつですね。	
\\	コウイチとビエトは、年に一度トーフグオフィスで開かれる剣道の大会で、優れた腕前を披露しようと競い合いました。	
\\	う 
\\	(きそ) 
\\	競	
\\	集中する	
\\	しゅうちゅうする	する 
\\	みんなが集中すればこの仕事は五時までに終わるはずだよ。	
\\	今は、一番大事な仕事に注意を集中させた方がいいよ。	
\\	現在、トーフグチームは
\\	という秘密のプロジェクトに集中し、全力で取りかかっています。	
\\	集, 中	
\\	決定する	
\\	けっていする	する 
\\	コウイチの赤ちゃんの名前が決定したら、すぐにお知らせしますね。	
\\	コウイチは、じゅう業いんのきゅうりょうを決定するけん利を持っている。	
\\	クソワロタ
\\	このテレビ番組クソほど面白いじゃねーかよ。俺の一番好きな番組に決定したわ。	
\\	決定 
\\	する 
\\	決定 
\\	する 
\\	決定 
\\	決, 定	
\\	便所	
\\	べんじょ	
\\	へをこく時は便所に行けよ。	
\\	サトシ、ようやく便所そうじのスイッチが入ったみたいだね。	
\\	よっしゃ!完璧なシュートが決まったぜ!俺、ここの便所のバスケットゴール付きのゴミ箱めっちゃ好きだわ。	
\\	便, 所	
\\	保持する	
\\	ほじする	する 
\\	コウイチは、ブロッコリー早食い王者の世界チャンピオンの座を保持している。	
\\	この辺りの治安を保持するには、ビエトの組の力が必要だ。	
\\	自国の地位を保持することをまず優先した方がいい。	
\\	保持, 
\\	保持, 
\\	保, 持	
\\	開放する	
\\	かいほうする	する 
\\	このそうこをひなん所として開放することになりました。	
\\	このプールは、夏の間、一ぱんに開放されます。	
\\	あるゲームにアクセスするために、ポートを開放する方法が知りたいんだ。	
\\	開, 放	
\\	拾う	
\\	ひろう	
\\	すてる神あれば拾う神あり。	
\\	かえりにあそこの歩道で五百円玉を拾いました。	
\\	だれかが道端にソファを捨ててたんだけど、まだ使えそうだったから、拾ってきたよ。	
\\	う 
\\	拾	
\\	指す	
\\	さす	
\\	矢じるしは、どちらの方向を指していますか?	
\\	人を指で指すのは、日本では失礼です。	
\\	ああ、あなたは色盲なんですか。ということは、どっちが赤色でどっちが緑色かを指すことはできませんね。	
\\	(さ). 
\\	指	
\\	注ぐ	
\\	そそぐ, つぐ	
\\	もしかして、わたしのコーラに牛にゅう注いだ?	
\\	コウイチは、今年は
\\	48をおうえんすることに全力を注いでいます。	
\\	「もう少しワインを注いでいただけますか?」「もちろん。“そこまで”って言ってくださいね。」「あ、そこまで!それくらいで結構です。有難うございます。」	
\\	う 
\\	(そそ).	注	
\\	気付く	
\\	きづく	
\\	気付かないふりはしないでください。	
\\	コウイチ子へのおもいに気付いたのはかの女と出会って三日目のことだったと思います。	
\\	あんたが私にアイスクリームをくれるなんて、本当に信じられないよ。あんたがそんなに優しいなんて今まで気付かなかったわ!	
\\	気を付けて 
\\	気 
\\	付く 
\\	付く 
\\	気, 付	
\\	放す	
\\	はなす	
\\	放せよ!	
\\	三十分後、コウイチはついにビエトの手を放した。	
\\	フグを海に放すといつも、泳いで逃げちゃって、私が呼んでも知らんぷりするの。どうすれば逃げないように教えることができるのかしら。	
\\	う 
\\	花 (はな). 
\\	放れる 
\\	放	
\\	見送る	
\\	みおくる	
\\	空こうまで見送るよ。	
\\	見送る方と見送られる方って、どっちの方がつらいんだろうね。	
\\	コウイチは、彼らを見送った後、どっと涙の雨を流した。	
\\	見る 
\\	送る 
\\	見る 
\\	送る. 
\\	見, 送	
\\	苦しむ	
\\	くるしむ	
\\	トーフグの考え方は、理かいに苦しみます。	
\\	はんざいに手をそめるようそそのかしてくるコウイチからの強れつなプレッシャーに苦しんでいます。	
\\	オレは家にいるおまえの新妻のことを考えて言ってるだけだよ。分かるか?彼女がお前の死で一生苦しむことになっても、本当にいいのか?だから、自殺なんてするなって。	
\\	苦しい 
\\	苦しい, 
\\	苦	
\\	伝える	
\\	つたえる	
\\	トーフグは、日本語の発音をそのままダイレクトにのうに伝えるきかいを発明しました。	
\\	先生には、授業中にねむってしまうぐらいつかれているって事前にちゃんと伝えてたんだよ。	
\\	お父様とお母様にも、よろしくお伝えくださいね。	
\\	う 
\\	(つた)! 
\\	伝	
\\	働く	
\\	はたらく	
\\	頭はガンガンするし、おなかはペコペコだったが、一日中働かなければいけなかった。	
\\	日本で働くために日本語をべん強しています。	
\\	「金曜日、
\\	の記事を読みに来ませんか?」「行きたいんですが、金曜日の夜は遅くまで働かないといけないんです。」「それは残念ですね。それでは、また今度。」	
\\	う 
\\	働	
\\	利く	
\\	きく	
\\	自転車のブレーキが利かない。	
\\	今日は、うでの利くりょうり人にりょうりを作ってもらいました。	
\\	コウイチは中々融通の利く男だ。	
\\	(き) 
\\	利	
\\	意外	
\\	いがい	
\\	な 
\\	意外にも、父はすぐにさん成してくれました。	
\\	ゆですぎのやわらかいパスタをこのんで食べるのは、カナダの意外な文化でした。	
\\	フグは物凄く走るの速いよ。海ではあまり良いスイマーじゃなかったってことと、大学に入るまで陸上での競技に出たことがなかったってのは、本当に意外だよ。	
\\	意, 外	
\\	乗り物	
\\	のりもの	
\\	乗り物の中でタバコをすわないでください。	
\\	乗り物よいしちゃった。	
\\	「あの乗り物、どうだった?」「ああ、怖かったさ!小便チビっちゃうくらい怖かったぜ。」	
\\	乗り 
\\	物 
\\	乗, 物	
\\	勝ち	
\\	かち	
\\	コウイチはベリーダンス対決でボロ勝ちした。	
\\	雪合せん日本代表はぎゃく転勝ちをおさめた。	
\\	早い者勝ちなんだだから、急ごうよ!あのフグのキーホルダーすごく欲しいんだよね!この機会を逃したくないんだ。	
\\	勝つ 
\\	勝つ. 
\\	勝	
\\	戦争	
\\	せんそう	
\\	する 
\\	戦争に行くのは半年ぶりです。	
\\	戦争をしない方がいい理由を十こあげてください。	
\\	戦争が始まる前に、腹筋を六つに割りたい。	
\\	戦, 争	
\\	波	
\\	なみ	
\\	今日の波、すごく高いね。	
\\	大へんだ!コウイチが波に飲みこまれたぞー!	
\\	波にゆらゆら揺られるのが気持ちいい。	
\\	(なみ). 
\\	波	
\\	洋食	
\\	ようしょく	
\\	この洋食レストランは、有名なすし屋が共同して立ち上げたものです。	
\\	いわゆる昔ながらの洋食屋さんで、バイトをしています。	
\\	ええっ!あそこのレストランで洋食ランチセットを食べたの?ははは。どおりでまだお腹がいっぱいなはずだわ。	
\\	洋, 食	
\\	洋風	
\\	ようふう	
\\	な 
\\	の 
\\	今日の夕食は、洋風ポトフです。	
\\	洋風なかんじのオシャレな曲が、洋風のカフェでながれていた。	
\\	私の家は、家具を全て洋風に揃えているので、畳はありません。	
\\	風 
\\	和風? 
\\	洋, 風	
\\	洋室	
\\	ようしつ	
\\	コウイチの発言で、洋室にいた全いんがこおりついてしまった。	
\\	この洋室、はじめて来た場所なのに、なんだかよく知ってる部屋のような気がする。	
\\	私の部屋は六畳の洋室で、壁の色は私の好きな水色です。	
\\	洋, 室	
\\	行動	
\\	こうどう	
\\	する 
\\	の 
\\	これはあなたの行動がまねいたけっかです。	
\\	いつもコウイチの行動パターンは決まっている。	
\\	あの人は発言と行動が一致していないからいつも惑わされて困る。	
\\	行, 動	
\\	活動	
\\	かつどう	
\\	する 
\\	クリステンは、週末はコスプレイヤーとして活動しています。	
\\	午前と午後は大学で生物学を勉強し、夕方からはしゅう教活動をしています。	
\\	活動動詞とは動詞を相の観点から四つに分けた内の一つです。	
\\	活, 動	
\\	空港	
\\	くうこう	
\\	この間、成田空港で大げんかをしているカップルを見ました。	
\\	おむつがえシートのあるトイレはこの空港にありますか?	
\\	空港で借りた車の中に、パスポートを忘れた気がする。	
\\	空, 港	
\\	酒飲み	
\\	さけのみ	
\\	子どものくせにスルメがすきだなんて、こりゃぁこの子はしょう来酒飲みになるねぇ。	
\\	コウイチはお酒は好きだけど、酒飲みってほどじゃあないですよ。	
\\	彼は大酒飲みだが、赤ワインが苦手だ。	
\\	飲む, 
\\	お酒 
\\	飲む 
\\	み 
\\	飲, 
\\	酒, 飲	
\\	全員	
\\	ぜんいん	
\\	全員男せいです。	
\\	トーフグの社員は、全員がベーコンソムリエのしけんを受けることになりました。	
\\	全員がさんかするのはむずかしそうですね。	
\\	全, 員	
\\	動物	
\\	どうぶつ	
\\	動物は苦手なんです。	
\\	何か動物はかっていますか。	
\\	コウイチがトーフグ動物園に100匹の重要な動物を集めてパーティーを開催したんだけど、何故か私は招待されなかったんだよね。私もコウイチにとっての重要動物の一匹だと思っていたんだけど。	
\\	動, 物	
\\	鳴き声	
\\	なきごえ	
\\	きりんの鳴き声を聞いたことがありますか。	
\\	外からコウイチの声が聞こえると思ったら、小鳥の鳴き声だった。	
\\	自動販売機の下から子犬の鳴き声が聞こえるんです。	
\\	声 
\\	ごえ, 
\\	鳴, 声	
\\	勝手	
\\	かって	
\\	な 
\\	ほんと、勝手だよね!	
\\	勝手なことばっかり言わないでよ!	
\\	「サーモン!フグは五回も浮気したんだよ。そんな男のこと、信じちゃダメだってば。」「分かってるんだけど、でも、彼、私がいないと生きていけないって言うの。」「じゃあ、もう、勝手にすれば!」	
\\	手 
\\	っ 
\\	勝, 手	
\\	都合	
\\	つごう	
\\	都合がつき次第、こちらかられんらくします。	
\\	都合次第では、コウイチのけっこんしきはえん期になるかもしれません。	
\\	「土曜日と日曜日、どちらの方が都合がいい?」「どちらでもいいよ。」	
\\	都 
\\	(つごう) 
\\	合 (ごう) 
\\	ごう.
\\	都, 合	
\\	悪い	
\\	わるい	い 
\\	この食べ放だいのすし屋のせっ客は、しんじられないほど悪い。	
\\	悪いけど、この野さいちょっと蒸しといてくれない?	
\\	牛乳は悪くなっていたけど、俺は気にしなかった。	
\\	い 
\\	(わる)? 
\\	悪	
\\	悪人	
\\	あくにん	
\\	の 
\\	あいつは悪人だが、ガーデニングがしゅ味なんだ。	
\\	コウイチは悪人をえんじているだけだよ。	
\\	お前、本当に善人と悪人を区別することが可能だと思っているのか?	
\\	悪, 人	
\\	野球	
\\	やきゅう	
\\	これから、プロ野球のし合を見に行くところなんです。	
\\	野球は好きだけど、野球部は体育会けいって聞いたから入りたくないんだよ。	
\\	野球をしている時に膝を骨折しちゃったんだよね。	
\\	野, 球	
\\	飲み物	
\\	のみもの	
\\	コウイチの飲んでいる飲み物は何ですか?	
\\	にんしん中は、アルコールをふくんだ飲み物は飲んではいけない。	
\\	「何か冷たい飲み物はいかがですか?」「それでは、牛乳パックを二本頂けますか?」	
\\	飲, 物	
\\	血族	
\\	けつぞく	
\\	の 
\\	おそうしきの後に、血族の集まりがあります。	
\\	コウイチの血族は、みんなとても身長が高い。	
\\	私の父は血族じゃないが、それでも本当の父のように思っている。	
\\	血, 族	
\\	親分	
\\	おやぶん	
\\	地下道でヤクザの親分とすれちがったんだ。	
\\	コウイチとビエトではどちらが親分でどちらが子分ですか。	
\\	親分にバレたら大目玉を食らうぞ。	
\\	親 
\\	分. 
\\	親, 分	
\\	息	
\\	いき	
\\	あんたの息、死ぬほどくさいよ。	
\\	びっくりしすぎて、息が止まるかと思ったよ。	
\\	水中で息はしないでください。	
\\	(いき) 
\\	息	
\\	商売	
\\	しょうばい	
\\	する 
\\	一しょに商売して一もうけしようぜ!	
\\	最近はあのコンビニのせいで、商売あがったりだね。	
\\	学校卒業後に商売を始めたんですが、すぐに商売をすることは簡単ではないことを学びました。	
\\	商, 売	
\\	心配事	
\\	しんぱいごと	
\\	心配事をわらいとばしてくれてありがとう。	
\\	なんだか心配事があるみたいだよ。	
\\	私の一番の心配事は、町に降り落ちる火山灰です。	
\\	心配 
\\	心配 
\\	事 
\\	(こと) 
\\	ごと 
\\	心配事, 
\\	心, 配, 事	
\\	虫歯	
\\	むしば	
\\	の 
\\	今日、歯医者で虫歯をぬいてもらいました。	
\\	虫歯のげんいんであるミュータンスきんは子どもの成長とともにまわりの大人からうつってしまいます。	
\\	虫歯がいっぱいあるんだけど、歯科保険が無いので、治療代を捻出することができないんだ。	
\\	は 
\\	ば, 
\\	虫, 歯	
\\	第二章	
\\	だいにしょう	
\\	の 
\\	第一章はおもしろかったけど、第二章はいまいちだった。	
\\	人生の第二章のまく開けですね。	
\\	まじで?君もその本読んでるの?俺、昨日ちょうど第二章読み終わったところなんだけど。もしそれより先に進んでるなら、何も言わないでね。	
\\	第, 二, 章	
\\	陽気	
\\	ようき	
\\	な 
\\	りょう理上手で子ども好きで陽気なおくさんがいるなんて、お前は本当にしあわせものだよ。	
\\	はるの陽気にさそわれて、いつの間にかねむってしまっていました。	
\\	うちの家族は貧乏だが陽気だ。	
\\	陽, 気	
\\	童話	
\\	どうわ	
\\	この童話、意外と泣けるんだよね。	
\\	やっと、童話のしっぴつに、本気モードになってきました。	
\\	宮沢賢治の童話「祭の晩」は、私が子供の頃大好きなお話でした。	
\\	話 
\\	童, 話	
\\	音読み	
\\	おんよみ	
\\	する 
\\	この漢字の音読みって何だっけ?	
\\	初めて見た地名はふ通は音読みしちゃうよね。	
\\	日本語の音読みと訓読みの違いは何ですか?	
\\	音 
\\	音読み. 読み 
\\	音, 読	
\\	都会	
\\	とかい	
\\	の 
\\	都会のいそがしい生活にはもううんざりだよ。	
\\	これだから都会育ちはきらいなんだよ!	
\\	こんな田舎じゃなく、私は都会で働きたいのよ。	
\\	都, 会	
\\	都市	
\\	とし	
\\	の 
\\	このコウイチのどうぞうは、この都市のシンボルになりました。	
\\	いなかよりも、都市での生活の方がいいですね。	
\\	何かおもしろい都市伝説を知っていますか?	
\\	都, 市	
\\	寒い	
\\	さむい	い 
\\	寒すぎてふとんから出られない。	
\\	この寒い中、一時間もバスを待たなきゃならなかったんだよ。	
\\	僕がハワイに行くのは、寒さから逃れるためだけではなく家族に会うためでもあります。	
\\	い 
\\	(さむ)). 
\\	寒	
\\	会社員	
\\	かいしゃいん	
\\	わたしの夫は、ごくふ通の会社員です。	
\\	わたしたちはみんな、会社員である前に、一人の人間なんですよ。	
\\	あの会社員は麻薬所持で逮捕された。	
\\	(会社) 
\\	会, 社, 員	
\\	落ち	
\\	おち	
\\	落ちをじゃまされてムカついた。	
\\	そういう落ちだったのね。	
\\	あいつの話には落ちが無いんだよな。	
\\	落	
\\	大根	
\\	だいこん	
\\	の 
\\	一度、大根を育ててみようとしたんですが、全部かれてしまったんです。	
\\	うちの家族は、クリスマスの時期にくつ下とかんそうさせた大根をぶら下げるんだ。	
\\	おお、これいいんじゃねえ?めちゃくちゃいい大根だよ。お前も、一本欲しいか?	
\\	大 
\\	だい 
\\	大, 根	
\\	深い	
\\	ふかい	い 
\\	水深が急に深くなりましたね。	
\\	コウイチがデスクで深いため息をついてるのを見たんだけど、何かあったの?	
\\	魚たちが、親や兄弟、高校時代の友人たちとの深い繋がりを保っていることが多いって、本当ですか?	
\\	い 
\\	深	
\\	最深	
\\	さいしん	
\\	な 
\\	の 
\\	日本で最深な地下鉄は、地下48mの場所がある大え戸線です。	
\\	しっとは人間の最深のつみだ。	
\\	サーモンと俺は、光が到達できる海の最深部に位置するロマンチックな秘密の洞穴で、結婚式をあげた。	
\\	最, 深	
\\	歯医者	
\\	はいしゃ	
\\	半年に一度は歯医者に行った方がいい。	
\\	花田さんは、この歯医者にはもっ体ないよ。	
\\	彼はただ不良に憧れてるだけだと思うよ。絶対に本物の不良にはならないね。だって、歯医者さんに行くのですら怖がってるんだよ!	
\\	医者 
\\	歯医者 
\\	歯, 医, 者	
\\	消化	
\\	しょうか	
\\	する 
\\	の 
\\	このくすりは、消化をそく進してくれます。	
\\	コウイチは下り気味だから、消化の良いものを食べさせてあげてね。	
\\	昼に食べたラーメンをまだ消化しきっていない。	
\\	消, 化	
\\	二倍	
\\	にばい	
\\	このかぶのかぶかはたったの二日で二倍になった。	
\\	めんつゆを二倍にうすめるのを忘れないでください。	
\\	うわ!君のお兄さんって、君の二倍の体重があるんだね!	
\\	三倍, 百倍, 
\\	四十二倍 
\\	二, 倍	
\\	運転手	
\\	うんてんしゅ	
\\	タクシーに乗ったら運転手がトーフグのコウイチだった。	
\\	バスの運転手と言い争いになってしまった。	
\\	運転手は大丈夫だったみたいだけど、車は事故でめちゃくちゃになっていたよ。	
\\	運転 
\\	手 
\\	手 
\\	運, 転, 手	
\\	社員	
\\	しゃいん	
\\	全社員がコウイチと仲良くするためのイロハを心えている。	
\\	れいの社員さんとは最近どうなの?仲直りはできた?	
\\	この会社の社員は、全員奇抜で素晴らしい。	
\\	社, 員	
\\	民族	
\\	みんぞく	
\\	の 
\\	わたしは民族のアイデンティティーを失いたくありません。	
\\	この民族は、とてもしん重なことでよく知られている。	
\\	インドには多様な民族がいますが、みんながみんなカレーを食べるのでしょうか。	
\\	民族?	
\\	民, 族	
\\	悪女	
\\	あくじょ	
\\	あの悪女、耳だけはいいんだよな。	
\\	かの女は悪女だが、いつも時間はきっちり守る。	
\\	その悪女は息をのむほど美しかった。僕は彼女に夢中だった。	
\\	悪, 女	
\\	運動	
\\	うんどう	
\\	する 
\\	最近、運動不足気味です。	
\\	運動したいけど、時間がないんだよね。	
\\	私は読書よりも運動の方が好きなので、本も、新聞も、雑誌も読みません。	
\\	運, 動	
\\	期待	
\\	きたい	
\\	する 
\\	あの大学院生には、もっと期待してたのに。	
\\	この前わざわざ日本からとりよせた新しいおやつは期待外れでした。	
\\	ウォール・ストリートは常にトップレベルの人材を求めていますが、コウイチならきっとそこでも期待に応えられると思います。	
\\	期, 待	
\\	落ち葉	
\\	おちば	
\\	の 
\\	目の前には、落ち葉のじゅうたんが広がっていました。	
\\	落ち葉をひろって天ぷらにしました。	
\\	「聞いてよ。最初は、お姉ちゃんの彼氏だから、フグと仲良くしようとしてただけだったの。でも、次々にいろんなことが重なって、知らない間に、お…落ち葉を一緒に熊手で集めてたの。」「ツ…ツナったら、なんてことなの。」「サーモン、待って。一度だけよ。それに、私は何とも思わなかったわ。」	
\\	落ちる 
\\	葉 
\\	ば. 
\\	落, 葉	
\\	家庭	
\\	かてい	
\\	の 
\\	いつかはけっこんして、あたたかい家庭をきずきたいと思っています。	
\\	わたしのしゅ味は、かく家庭から出るゴミをす手でしゅう集することです。	
\\	日本女性と比べると、外国の女性は家庭の時間と仕事の時間を分けることには、かなり厳格です。	
\\	家, 庭	
\\	家族	
\\	かぞく	
\\	の 
\\	わたしの家族は、庭でキュウリやトマト、ズッキーニなどたくさんの野さいを育てています。	
\\	家族がなくなった場合、そのじゅう業いんは死亡発生日の次の出きん日から三日間けいちょう休かを取得できます。	
\\	今度、俺の日本への家族旅行のことを君に話したいな。	
\\	家, 族	
\\	父親	
\\	ちちおや	
\\	わたしの父親はへなちょこレスラーです。	
\\	今朝、父親が部屋でこっそり泣いているのを見てしまいました。	
\\	家族はみんな父親のこと嫌ってるから、父はいつも一人裏通りで夕飯を食べてるわ。	
\\	父 (ちち) 
\\	親 (おや).
\\	父, 親	
\\	祭	
\\	まつり	
\\	祭の屋台で売ってるりんごあめがすきなんです。	
\\	こないだの日よう日のお祭りで、かれ氏と初めて手をつないだんだ。	
\\	日本の多くのお祭は、仏教や神道と結びついています。	
\\	(まつり) 
\\	祭	
\\	一階	
\\	いっかい	
\\	ヘビたちは一階でかっているんですか?	
\\	一階はまだまだびちょびちょだよ。	
\\	「荷物はどこに置いたらいいかな?」「一階でも、二階でも、下でも、どこでも。あなたの好きなところに置いてくれていいよ。」「わかったわ。有難う。」	
\\	一 
\\	っ 
\\	一, 階	
\\	第一位	
\\	だいいちい	
\\	トーフグはユニークな日本語学習教ざいとして、五年れんぞく第一位のざをかくとくしています。	
\\	トーフグチームの中で一番げん付乗りまわしていてほしい人第一位はわたしてきにはクリステンなんだよね。	
\\	あきらが入社するまで、ヒロは五年続けて、社内で売り上げ第一位だった。	
\\	第, 一, 位	
\\	一流	
\\	いちりゅう	
\\	の 
\\	さすが、一流のピアニストは違うね。	
\\	わたしの妹は、お金持ちで一流のしんしとこいに落ちました。	
\\	これでコウイチは一流社会の仲間入りだ。たくさんのセレブに会ったり、シャンパンパーティにもすぐ行き始めるだろう。	
\\	一, 流	
\\	〜階	
\\	かい	
\\	「何階ですか?」「四階をおねがいします。」「このエレベーターはその階には止まりませんよ。」	
\\	コウイチは、九十九階だてのマンションの最上階に住んでいます。	
\\	このアパートの十三階に住んでいます。	
\\	階	
\\	四十二階	
\\	よんじゅうにかい	
\\	四十二階まで競争よ!	
\\	コウイチは東京に四十二階だてのマンションを持っている。	
\\	引っ越しって、色々事務手続きがあって結構面倒くさい上に、対応の悪い担当者とかもいたりするよね。でも、今回の引っ越しで最悪だったのは、引っ越しの日にエレベーターが故障中だったことだよ。俺の部屋、四十二階なのに。	
\\	四十二 
\\	四, 十, 二, 階	
\\	庭	
\\	にわ	
\\	今日は、庭の地ならしをします。	
\\	庭で家庭さいえんをしています。	
\\	うちには美しい庭があるんだが、隣の人の鶏がいつも花を食べてしまうんだ。	
\\	(にわ) 
\\	庭	
\\	集まり	
\\	あつまり	
\\	今日は研究室の集まりがある。	
\\	なんか今日の飲み会、集まりも悪かったし、グダグダだったね。	
\\	正月の集まりに弟が彼女を連れてきやがった。	
\\	集	
\\	大学院生	
\\	だいがくいんせい	
\\	この大学院生は、えんげいが上手だ。	
\\	あのまんるいホームランを打ったせん手、げん役の大学院生らしいよ。	
\\	大学院生ってどんなことをするんですか?	
\\	大学院 
\\	大学生 
\\	大, 学, 院, 生	
\\	港	
\\	みなと	
\\	ベラクルスという港町に行ってみたいです。	
\\	ごう雨のため、港への道は通行止めになっています。	
\\	不思議な船が港に漂着した。	
\\	(みなと) 
\\	港	
\\	聞こえる	
\\	きこえる	
\\	聞こえたんだったら、ちゃんと返事してよ!	
\\	オナラ、聞こえちゃったよね?	
\\	きつい言い方に聞こえることは分かってるんだけど、つまらない言い訳はしたくないの。	
\\	聞く 
\\	こえ 
\\	聞く, 
\\	聞	
\\	広島	
\\	ひろしま	
\\	広島に、登山にいい山はありますか?	
\\	広島空港から市バスで市内に向かってるところです。	
\\	広島と長崎に投下された原爆は、一瞬で十万人以上の人々の命を奪った。	
\\	広, 島	
\\	温泉	
\\	おんせん	
\\	この温泉は、タトゥーのある人も入よくできます。	
\\	かるく運動してあせを流した後、温泉に入ろう。	
\\	私の父も母も温泉が好きじゃありません。	
\\	温, 泉	
\\	湯	
\\	ゆ	
\\	お湯であらってみたら、落ちるかもしれないよ。	
\\	コウイチは大きい方をした後、お湯と石けんで手をあらいます。	
\\	お湯を沸かしてくれないか?	
\\	湯	
\\	暑い	
\\	あつい	い 
\\	暑すぎてとけそう。	
\\	日本では、このところ暑い日がつづいています。	
\\	クソ暑いし蒸し蒸しするしたまんねぇな!この暑さでは寝られないよ。プールに行こうぜ!	
\\	暑	
\\	僕	
\\	ぼく	
\\	の 
\\	僕に任せろ。	
\\	医者から慰謝料をもらうというダジャレを作ったのは僕だ。	
\\	これは僕の缶珈琲だぞ!	
\\	僕	
\\	利息	
\\	りそく	
\\	もし利息の支はらいがおくれたら、ビエトがヤクザの友だちをつれてお前のところにやって来ると思うぜ。	
\\	コウイチは利息で生活できるぐらい十分なお金をぎん行にちょ金しています。	
\\	あいにく今、その質問に答えられる者がいないのですが、もしインターネットにアクセス可能なら利息のレートはそちらでもご確認頂けますよ。	
\\	利, 息	
\\	始めに	
\\	はじめに	
\\	このもんだいでは、始めにしつもんをきいてください。	
\\	始めにカナエがたん生日ケーキにしょうゆをこぼしてしまったこと以外は、コウイチのサプライズバースデーパーティーは大せいこうでした。	
\\	聖書の一番始めには何が書かれているの?	
\\	に 
\\	始める, 
\\	始	
\\	島	
\\	しま	
\\	この島のはま辺では、水着は着用できません。	
\\	わたしの家内は、この島で一番の美人です。	
\\	この島で一週間生活をしてもらいます。	
\\	島	
\\	顔文字	
\\	かおもじ	
\\	今の気持ちを顔文字で表してください。	
\\	ビエトがコウイチにあてた手紙を内しょで読んだら、全てが顔文字でかかれていた。	
\\	上司へのメールに顔文字を使うんじゃない!	
\\	文字 
\\	顔 
\\	文字 
\\	顔, 文, 字	
\\	京都	
\\	きょうと	
\\	の「そうだ、京都行こう」の広告を見て京都に行きたくなったが、そんなお金はなかった。	
\\	京都に行ったら、湯葉を食べたいです。	
\\	コウイチは本日は既に京都へ発ってしまいました。明日の朝一番に折り返しお電話をさせましょうか?	
\\	京, 都	
\\	終電	
\\	しゅうでん	
\\	終電、何時だっけ?	
\\	もうすぐ終電だから、そろそろかえらなきゃ。	
\\	「もし終電に乗り遅れたら、いつでも電話してくれていいよ。」「有難う。覚えておくよ。」	
\\	電 
\\	終, 電	
\\	運	
\\	うん	
\\	運が良ければ、トーフグのコウイチに会えるかもね。	
\\	たまたま運が良かっただけだよ。	
\\	こんなことで使い果たしたくはないけど、これで運を使い果たしちゃったかも。	
\\	運	
\\	本流	
\\	ほんりゅう	
\\	の 
\\	このフグは、トーフ川本流のワニカニはしの下でつれました。	
\\	コウイチは、ファッションの本流に大きなえいきょうをあたえることで有名です。	
\\	日本文学の本流と言って、最初に思い浮かぶのは誰ですか。	
\\	本, 流	
\\	会員	
\\	かいいん	
\\	あの会員はかなり心配しょうのようなんです。	
\\	見えないかもしれませんが、そこにもう一人小さな会員がいるんですよ。	
\\	会員カードはお持ちですか。	
\\	会, 員	
\\	商人	
\\	しょうにん	
\\	の 
\\	商人の血がさわぎます。	
\\	わたしの父は、おろし商人です。	
\\	彼、私に自分は商人だって言ってたんだけど、でも実は貧乏なホームレスで、仕事はなんと空き缶収集だったの。	
\\	商, 人	
\\	深夜	
\\	しんや	
\\	こんな深夜に、どうしたの?	
\\	深夜に小ばらが空いてきた。	
\\	環境難民となった熊たちは、新しい国で食べ物を探すため、深夜、国境を越えた。	
\\	深, 夜	
\\	母親	
\\	ははおや	
\\	の 
\\	母親と仲直りがしたい。	
\\	母親とそのかれ氏、気持ちがだんだんはなれていっちゃったみたいで、けっきょく別れたんだよね。	
\\	「ごめんなさい。明日のランチ、キャンセルしなくちゃいけなくなったの。母親がちょっと体調崩しちゃって。」 「気にしないで。また今度、会いましょう。それより、お母様、早く良くなるといいわね。」	
\\	母 
\\	親. 
\\	母, 親	
\\	登山	
\\	とざん	
\\	する 
\\	ふじ山に登山してみたいです。	
\\	もう登山はこりごりです。	
\\	「登山に行くってのはどう?」「さあ、どうしようかなあ。」	
\\	と 
\\	とう 
\\	と 
\\	さん 
\\	ざん 
\\	登, 山	
\\	祭日	
\\	さいじつ	
\\	の 
\\	その祭日の予定はまだありません。	
\\	祭日なのに、仕事なの?	
\\	祭日だってことを忘れていて、会社に来てしまった。	
\\	祭, 日	
\\	根気	
\\	こんき	
\\	これは根気のいる仕事だよな。	
\\	ほとんどの失ぱいのげんいんは、根気のなさなんだよね。	
\\	根気よく日本語の勉強を続けていれば、いつかはペラペラになれるさ。	
\\	根, 気	
\\	期間	
\\	きかん	
\\	テスト期間中はあまりねれません。	
\\	わたしは人生のある期間を山おくですごした。	
\\	コウイチの髭を期間限定で販売しています。	
\\	期, 間	
\\	植物	
\\	しょくぶつ	
\\	コウイチは植物の言っている事が分かるらしい。	
\\	これは、植物でできたふ葉土です。	
\\	園芸の才能は無いけど植物を育てるのは好きなんですよ。	
\\	物 
\\	ぶつ, 
\\	植, 物	
\\	屋根	
\\	やね	
\\	来年、屋根をしゅう理する必要があります。	
\\	屋根にあなが開いているので、雨もりしてるんですよ。	
\\	お昼までに屋根の雪を取り除いて下さい。	
\\	根 
\\	ね. 
\\	(ね). 
\\	屋, 根	
\\	根本	
\\	こんぽん, ねもと	
\\	の 
\\	そもそも、根本から間ちがってるんだよ。	
\\	そのふうふは、子どもの教育についての根本の考え方が合わなかったんです。	
\\	クリステンはベーコンを食べませんが、この問題の根本の原因はどこにあると思いますか?	
\\	本 
\\	ぽん, 
\\	根, 本	
\\	根	
\\	ね	
\\	この花の根は食べられますよ。	
\\	このもんだいは相当根が深いようですね。	
\\	彼は素敵な人だと思うかもしれないが、根がひどい男なのだ。	
\\	根	
\\	船員	
\\	せんいん	
\\	の 
\\	今日は夕方に船員さんがおむかえに来てくれるからね。	
\\	そ父は海ぐんの船員でした。	
\\	今日、あの船員がチェックアウトした部屋に、部屋を変えてもらえませんか?	
\\	船, 員	
\\	学期	
\\	がっき	
\\	の 
\\	次の学期にもフグと日本語と着物のクラスをとるつもりです。	
\\	この学期が終わったら、コウイチとコウイチ子はハワイでけっこんしきをあげる予定です。	
\\	学期末試験は三月なので、まだまだ勉強する時間はたっぷりあります。	
\\	がく 
\\	がっ, 
\\	がく 
\\	学, 期	
\\	二階	
\\	にかい	
\\	歯医者さんなら、このビルの二階にあります。	
\\	二階だてのマイホームを買いました。	
\\	毎日午後に二階にある同じ教室に行って、英語を二時間教えます。	
\\	二, 階	
\\	店員	
\\	てんいん	
\\	の 
\\	昔、あの店員の弟に家庭教しのバイトで英語を教えていました。	
\\	あの店員さん、とってもお茶目だったね。	
\\	えっと、レジの順番待ちの列に並んでいる時に、ある店員さんが「万引きです!捕まえて!」と叫ぶのが聞こえたので、彼を捕まえたんです。	
\\	店, 員	
\\	短い	
\\	みじかい	い 
\\	そのスカート、ちょっと短すぎるんじゃない?	
\\	赤ちゃんが短い手足をバタバタさせてるのは本当にかわいい。	
\\	わぁ!あんたの尻尾が私より短いなんて、知らなかったわ。	
\\	い 
\\	(みじか).	短	
\\	短期	
\\	たんき	
\\	の 
\\	まずは、短期の目ひょうをいくつか作ってみてください。	
\\	こんな短期間でこんなにおぼえるとか、頭がパンクするっつーの。	
\\	私は父に、短期的なコスト削減のためにリーダーシップ研修を縮小している企業は、一文惜しみの百知らずだと教えられました。	
\\	短, 期	
\\	短刀	
\\	たんとう	
\\	その短刀、どこで買ったの?	
\\	ビエトは、短刀を使ってとても上手にりょう理をします。	
\\	私の父は、母に短刀で刺されて死にました。	
\\	短, 刀	
\\	息子	
\\	むすこ	
\\	息子が明日のしけんのために一夜づけしているため、今夜はとてもしずかだ。	
\\	うちの息子は、こうりつてきに働いてだれよりも早くたい社するんだ。	
\\	全くひどい息子をもったもんだよ。あいつは昨夜俺のケーキの残りをこっそり盗み食いしやがったんだ。	
\\	子 
\\	こ, 
\\	息 
\\	むす 
\\	(むす). 
\\	息, 子	
\\	球	
\\	たま, きゅう	
\\	ああっ!球がバンカーに入っちゃったよ。	
\\	球の体せきをもとめる公しきって何だっけ?	
\\	ねぇ、新しい野球の球を買ったんだけど、一緒にキャッチボールしない?	
\\	(たま). 
\\	玉. 
\\	きゅう 
\\	球	
\\	泉	
\\	いずみ	
\\	わたしたちは、泉からのわき水を飲んでいます。	
\\	その泉にはがちょうがたくさんいます。	
\\	わぁ!すごく綺麗。何か特別なダイエットでもしたの?それとも、若返りの泉でも見つけたとか?	
\\	(いずみ), 
\\	泉	
\\	着々	
\\	ちゃくちゃく	
\\	その会社員は、いつも着々と仕事をこなしていた。	
\\	プロジェクトは着々進んでいるようだね。	
\\	手はずは着々と整っていますよ。	
\\	着, 々	
\\	湯気	
\\	ゆげ	
\\	ビエトのおでこから出ている湯気、なんだかいいにおいがするんだよね。	
\\	それ、まだ湯気が立つほどあついから、気を付けてね。	
\\	美味しい寿司飯を作るには、しゃもじを使って、湯気が立つ炊きたての熱いごはんに酢を混ぜる必要があります。信じられないかもしれませんが、実際に米一粒一粒を潰さずに酢を混ぜるには技術が必要です。	
\\	ゆ 
\\	げ 
\\	気 
\\	(げ). 
\\	(ゆげ). 
\\	湯, 気	
\\	勉強	
\\	べんきょう	
\\	する 
\\	ありがとうございます!勉強になります。	
\\	最近、だらけちゃってて、あんまりちゃんと勉強してないんだよね。	
\\	英語の勉強、頑張って下さいね。私も日本語をずっと勉強しているので苦労はよくわかります。	
\\	勉, 強	
\\	太陽	
\\	たいよう	
\\	太陽のようなコウイチのえ顔に、こいに落ちてしまいました。	
\\	太陽の温度って何度ぐらいあるか知っていますか?	
\\	日本万国博覧会のシンボルは、岡本太郎が作った「太陽の塔」で、それは今も日本の大阪の吹田にある万博公園にそびえ立っています。	
\\	太, 陽	
\\	田代島	
\\	たしろじま	
\\	田代島の人口は百人以下です。	
\\	田代島にはガソリンスタンドがありません。	
\\	最も有名な「猫島」といえば、日本の宮城県にある田代島という島です。	
\\	代 
\\	(しろ). 
\\	田, 代, 島	
\\	茶の湯	
\\	ちゃのゆ	
\\	茶の湯の道具が買えるお店をさがしています。	
\\	この水は、茶の湯にてきした水として知られています。	
\\	茶の湯の世界には、「一つ一つの出会いを大切に」という意味の「一期一会」という言葉があります。この言葉は、映画「フォレスト・ガンプ」の副タイトルとしても使われました。	
\\	茶, 湯	
\\	温度	
\\	おんど	
\\	アメリカ人と日本人って、温度のかんじ方がちがうと思いますか?	
\\	エアコンの温度を二十四度まで上げてもらえませんか?	
\\	醸造中に天気や温度がほんの少し変化しただけで、酒の味に影響することがあるため、酒を作ることは非常に難しいと言われています。	
\\	温, 度	
\\	気温	
\\	きおん	
\\	気温が急に下がった。	
\\	今日の最高気温は四十度です。	
\\	朝ばんの気温のさがはげしいんです。	
\\	気, 温	
\\	地球	
\\	ちきゅう	
\\	の 
\\	地球はお金では買えないよ。	
\\	コウイチは今ごろ地球のうらがわで何をしてるんだろう。	
\\	地球という名前を考えたのは誰だ?	
\\	地, 球	
\\	終わり	
\\	おわり	
\\	これで終わりだ。	
\\	終わり良ければ全て良し、でしょ?	
\\	今週で仕事も終わりで、来週からはクリスマス休暇だー!あー、待ちきれない。	
\\	終わる 
\\	終わる. 
\\	終	
\\	第一	
\\	だいいち	
\\	自分のけんこうが第一ですよ。	
\\	第一きぼうは二月十四日で、第二きぼうは二月二十一日です。	
\\	あの有名なフグ一家の第一印象はどうでしたか?	
\\	第, 一	
\\	最悪	
\\	さいあく	
\\	な 
\\	今日は最悪な一日だったよ。	
\\	最悪、すいみん時間をけずって仕事をしなければならないですが、まぁなんとかなるでしょう。	
\\	「昨夜、自動車を盗まれちゃったんだよ。」「げっ、最悪じゃん。」	
\\	最, 悪	
\\	三番目	
\\	さんばんめ	
\\	この曲、ビートルズの中で三番目くらいにすき。	
\\	コウイチなら左から三番目のトイレでハリーポッターを読んでるよ。	
\\	日本の有人島の多さは、インドネシア、フィリピンに次いで、世界で三番目だってこと、知ってた?	
\\	一番目? 
\\	一番目 
\\	三, 番, 目	
\\	時期	
\\	じき	
\\	時期が来るまで、見守ってやらないか?	
\\	もうすぐハリケーンの時期ですね。	
\\	タケシは単によくあるフェーズにさしかかっているだけだよ。ほら、外国で暮らす人はたいてい、カルチャーショックの時期を経験するっていうじゃない?	
\\	時間 
\\	時, 期	
\\	〜倍	
\\	ばい	
\\	ワニカニのレビューの数が四倍になってしまった。	
\\	クリスマスの日ぐらい、きゅうりょうを十倍にしてくれたっていいじゃないか!	
\\	回転寿司の普及に伴い、日本におけるサーモンの消費量は輸入開始から三倍となった。	
\\	3倍 
\\	倍	
\\	着く	
\\	つく	
\\	今、成田空港に着きました。	
\\	着く時間が分かったら、またれんらくして。	
\\	電車が時間通りに着かなかったんです。	
\\	(着る) 
\\	着 
\\	着る, 
\\	く. 
\\	(く) 
\\	(つ) 
\\	着	
\\	始まる	
\\	はじまる	
\\	そのえいがは何時から始まりますか。	
\\	アメリカと中国のしれつなけいざいせんそうが始まりました。	
\\	銃声が、大食いパイコンテストの始まりの合図となります。	
\\	う 
\\	始める.	始	
\\	登る	
\\	のぼる	
\\	この長くてきゅうなさかを登るのはかなりキツイっすね。	
\\	コウイチは、さるのようにスイスイと木に登り、いともかんたんにバナナを取っておりてきました。	
\\	一生の内、少なくとも一度は富士山に登ることが夢なんです。	
\\	う 
\\	上る 
\\	登	
\\	集まる	
\\	あつまる	
\\	クラウドファンディングで、こんなにたくさんのお金が集まりました。	
\\	社員旅行は、全社員が集まるいいき会になります。	
\\	見て!あそこにヤンキーたちがたくさん集まってるよ。	
\\	集める 
\\	集める 
\\	集	
\\	要求する	
\\	ようきゅうする	する 
\\	あんまりむ理なことを要求するなよ?	
\\	先方は、か格の点で妥協するようこちらに要求してきています。	
\\	彼にお金を要求されたから、ふざけるなって言ってやったんだけど、そしたらボコボコにされて、財布から無理やりお金を盗られちゃったよ。	
\\	要求 
\\	要求. 
\\	要, 求	
\\	落とす	
\\	おとす	
\\	今朝、コウイチのデスクの下に生たまごを落としてしまいました。	
\\	そんなしん用を落とすようなことは今すぐ止めた方がいいよ。	
\\	「今日、『男を落とす方法』っていう本を買ったんだけど、帰り道にどこかで落としちゃったみたいなんだよね。」「ええ〜。それは運が悪かったわね。でも、そんなに肩を落とさないで。そのうち良いことあるわよ。」	
\\	落ちる 
\\	(とす) 
\\	落ちる. 
\\	落	
\\	消す	
\\	けす	
\\	消したい過去があるんです。	
\\	間違えてデータを全部消しちゃったんだよ。	
\\	教室を出た後は電気を消すのを忘れないで下さい。	
\\	う 
\\	消	
\\	転ぶ	
\\	ころぶ	
\\	サヨナラホームランを打った後であのせん手が転んだのはおも白かったね。	
\\	今日のコウイチは、久しぶりに転んでもなかなかった。	
\\	コウイチはもう少しで地面に転ぶところだったが、スーパーマンのコスプレをしたビエトに助けられた。	
\\	転がる? 
\\	ぶ 
\\	転がる. 
\\	転	
\\	流す	
\\	ながす	
\\	トイレの水、流すのわすれたでしょう?	
\\	どうしてあの人たちがぬま地にあぶらを流してるか知っている?	
\\	トーフグの風習に、河豚の形に切りとった紙を、コウイチの誕生日に川に流すというものがあります。	
\\	う 
\\	流	
\\	東京都	
\\	とうきょうと	
\\	東京都には、一流のレストランがたくさんある。	
\\	東京都の30代会社員、山田太ろうさん(仮名)は、げきむと長時間ろうどうで知られるぎょうかいで働いています。	
\\	スカイツリーがそびえ立つ東京都墨田区では、今年は昨年よりも早く木の葉が落ち始めた。	
\\	東京 
\\	都 (と). 
\\	東, 京, 都	
\\	動く	
\\	うごく	
\\	あそこで何か動いたよ。	
\\	トーフグのサーバーは二十四時間動いているので、いつでもアクセスできますよ。	
\\	くそ〜。パソコンが動かなくなっちゃったぜ。どうしてこうなったんだろう?!	
\\	う 
\\	(うご) 
\\	動	
\\	起こる	
\\	おこる	
\\	まさかこんなことが起こるなんて、思ってもみなかったよ。	
\\	今起こったことについて、落ち着いてよく考えてみてください。	
\\	「ねぇ、僕にキスしたい?」「ありえない。100万年経ってもあんたとキスなんて絶対にしたくないから。天地がひっくり返っても、起こりっこないよ。」	
\\	起きる 
\\	(こる) 
\\	起きる 
\\	起	
\\	開く	
\\	あく	
\\	社会のまどが開いていますよ。	
\\	このお店は何時に開きますか。	
\\	「すみません。トーフグ博物館は明日は開いてますか?」「開いてますよ。でも、混雑が予想されるので、早めにお越しになられることをお勧めします。」「分かりました。有難うございます。」	
\\	開ける 
\\	開ける. 
\\	開	
\\	歯	
\\	は	
\\	七本目の歯が生えてきました。	
\\	タバコのすいすぎで歯が黄色くなってしまった。	
\\	電動歯ブラシを買うことを考えているんだが、どれも値段が高いんだよね。	
\\	歯	
\\	植える	
\\	うえる	
\\	トーフグオフィスの屋上にはサボテンが植えてあります。	
\\	チューリップ、ちょっと植えすぎなぐらい植えたんじゃない?	
\\	この種をどこに植えるつもりだい。	
\\	う 
\\	植 
\\	(う) 
\\	植	
\\	鳴る	
\\	なる	
\\	お腹が鳴ってしまった。	
\\	だれかの電話が鳴ってるね。	
\\	雷が鳴ったが、雨は降らなかった。	
\\	鳴く, 
\\	る 
\\	(る). 
\\	る 
\\	鳴く. 
\\	鳴	
\\	歌う	
\\	うたう	
\\	歌う時、ふっきんに力を入れるようにするといいよ。	
\\	人前で歌うのはすきじゃないんです。	
\\	「外は雨が降ってるね。外出せずに家で歌でも歌いたいな。」「じゃあ、こうしよう。今夜は出かけるのをやめよう。で、私が夕食を作るから、あなたは私のために歌ってよ。」	
\\	う 
\\	歌. 
\\	う 
\\	歌	
\\	合う	
\\	あう	
\\	肉りょうりには赤ワインが合います。	
\\	合うかどうか、しちゃくしてかくにんした方がいいよ。	
\\	アメリカのインド料理は、アメリカ人の口に合うよう、実際のインドでの調理方法とはかなり違った方法で作られている。	
\\	う 
\\	(あ). 
\\	合	
\\	待つ	
\\	まつ	
\\	きっ茶店で、五時間待っていましたが、かれは来ませんでした。	
\\	スタバで待ってますね。	
\\	ちょっと待って!財布を家に忘れたかも。	
\\	う 
\\	(ま) 
\\	待	
\\	着る	
\\	きる	
\\	パニクってて、
\\	シャツをおもてうらはんたいに着てたわ!	
\\	コウイチは毎日エプロンを着て出きんしている。	
\\	一昨日の彼氏の誕生日会には、黒の膝丈のドレスを着ていったんだ。	
\\	う 
\\	着く 
\\	る 
\\	(き)? 
\\	着	
\\	心配する	
\\	しんぱいする	する 
\\	あの会社は安全第一だから、心配しなくても大丈夫ですよ。	
\\	トーフグの業せきを心配してくれているのであれば、どうか手ぬぐいを買ってください。	
\\	外では雪がしんしんと降っており、地面が凍りだしているので、彼氏の家に行くときに丘を車で登りきれるのかどうか心配しています。	
\\	心配 
\\	心配 
\\	心, 配	
\\	流行	
\\	りゅうこう, はやり	
\\	する 
\\	の 
\\	最近、アメリカの女せいの間でアジアの化しょうひんが流行しています。	
\\	ポケモン
\\	は
\\	しで2016年度に最も流行したものの一つにえらばれました。	
\\	ねぇ、フグ。魚達向けに、流行のヘアスタイルを掲載した雑誌を出版したらどうかなって思ってるんだけど、あなたはどう思う?	
\\	流, 行	
\\	旅行	
\\	りょこう	
\\	する 
\\	の 
\\	夏休みは、家族旅行でキューバに行く予定です。	
\\	大学をそつ業する前に、東南アジアを旅行したいんだよね。	
\\	私の旅行の話を聞きたいのは分かるんだけど、それより、まずは私がいない間どうしてたのか教えてよ。	
\\	旅行 
\\	旅, 行	
\\	旅	
\\	たび	
\\	する 
\\	世界中を旅するのがゆめです。	
\\	さいふとパスポートをぬすまれるは、ひ行きに乗りおくれるは、マジで最悪の旅だったよ。	
\\	私はもうすっかり旅支度ができているよ。	
\\	(たび) 
\\	旅	
\\	温かい	
\\	あたたかい	い 
\\	何か温かいものが飲みたいなぁ。	
\\	初心者なので、どうか温かい目で見まもってください。	
\\	「デザートに温かいアイスクリームはいかがですか?」 「いいえ、もう結構です。」	
\\	い 
\\	(あたた). 
\\	温	
\\	水深	
\\	すいしん	
\\	水深100フィートまでもぐったことがあります。	
\\	このプールの水深はあさいから、心配しなくていいよ。	
\\	ダイビングをしている際に、水深十メートルのところで虹色の海藻を見つけました。	
\\	水, 深	
\\	話し方	
\\	はなしかた	
\\	話し方が本当にイライラするのよね。	
\\	コウイチの話し方って、かなりしぶいよね。	
\\	うわぁ!君の話し方、まるで母国語が日本語の人みたいだね。	
\\	話す 
\\	方. 
\\	話, 方	
\\	言葉	
\\	ことば	
\\	エルフをやといたいのは山々なんですが、言葉のかべがあるのでむずかしいんですよ。	
\\	うまく言葉にできないけど、コウイチはわたしにとってすごく特別な人なの。	
\\	私は言葉に詰まった。	
\\	言 
\\	(こと). 
\\	言, 葉	
\\	文章	
\\	ぶんしょう	
\\	この文章をかいたのはだれですか?	
\\	コウイチは文章が本当に上手だ。	
\\	私は、村上春樹の気障で気取った文章があまり好きではありません。	
\\	文, 章	
\\	感心	
\\	かんしん	
\\	する 
\\	な 
\\	感心なお子さんですね。	
\\	毎日休まずに部屋をそうじするなんて、感心、感心。	
\\	ケイティは、真面目で熱心な働きと細かい気配りで、同僚たちを感心させた。	
\\	感, 心	
\\	入館料	
\\	にゅうかんりょう	
\\	入口で入館料をはらってください。	
\\	トーフグはく物館の入館料は高すぎます。	
\\	入館料、もう10%値引きしていただけませんかねぇ?	
\\	入, 館, 料	
\\	養子	
\\	ようし	
\\	あの駅長の子は、どうやら養子らしいです。	
\\	わたしの友人は九月に養子をむかえいれることにした。	
\\	養子の態度が悪いから、叱ってやったんだ。	
\\	養, 子	
\\	整理	
\\	せいり	
\\	する 
\\	このアパートの住民の部屋はみんなきちんと整理されています。	
\\	整理整とんすることは大切です。	
\\	別れた彼氏の写真を整理しました。	
\\	整, 理	
\\	〜放題	
\\	ほうだい	
\\	食べ放題に行くといつも食べすぎてしまいます。	
\\	カラオケに行くなら、朝まで歌い放題のプランにしようよ。	
\\	もしテキストフグの永久購読権を購読されるのであれば、日本語の学習がし放題ですよ。	
\\	放, 題	
\\	暗殺	
\\	あんさつ	
\\	する 
\\	そのヤクザの組長は暗殺されてしまいました。	
\\	もう暗殺教室の映画は見ましたか?とってもおも白いですよね〜。	
\\	ねぇ、ビエト!どうやってこんなに面白い暗殺計画を思いついたの?	
\\	暗, 殺	
\\	銀行	
\\	ぎんこう	
\\	の 
\\	銀行にお金を下ろしに行かないといけないんです。	
\\	一円を笑う者は一円に泣くということわざがあるので、銀行に一円をあずけに行くことにしました。	
\\	銀行を強盗した人は牢屋行きなのに、銀行が人々からお金を奪ったら、ボーナスがもらえるんだよ。それってすごく不公平じゃない?	
\\	銀, 行	
\\	初めに	
\\	はじめに	
\\	まず初めに牛肉をスライスして、それからフライパンでいためてください。	
\\	初めに言わせてもらいますが、アドリブえんそうはしないでください。	
\\	よ安らかに眠ってくれ。今月初めに、僕の一番好きな日本語学習サイトが倒産しちゃったんだ。	
\\	に). 
\\	(はじ). 
\\	初	
\\	橋	
\\	はし	
\\	かれに橋の上でキスされちゃった。	
\\	橋をわたる間、むすめと歌を一しょに歌ってくれませんか?	
\\	その1603年に建てられた橋の名前は日本橋といいます。日本橋は、その名の通りまさに日本の中心でした。	
\\	橋
\\	橋	
\\	賞金	
\\	しょうきん	
\\	実はコウイチの首には賞金がかけられているって知っていましたか。	
\\	ビエトは今回の日本旅行で当てたけい馬の賞金が一億円もあるらしい。	
\\	私達は変顔コンテストの賞金をいつもらえるんですか?	
\\	賞, 金	
\\	事情	
\\	じじょう	
\\	大人には大人の事情があるんですよ。	
\\	警察はすべての事情を考慮しなければならない。	
\\	サマンサが母親の入院ひのためにワニカニのサービス利用代金がはらえなかったという事情はむしされてしまった。	
\\	事, 情	
\\	器	
\\	うつわ	
\\	この器は母の形見なんです。	
\\	どうしよう?コウイチが大切にしていたフグの器をわってしまった。	
\\	まず最初にごはんを器に入れて、次にハンバーグをのせて、それからグレービーソースをかけて、最後に卵をのせたら…ジャーン!ロコモコの出来上がり!	
\\	(うつわ) 
\\	器	
\\	暗い	
\\	くらい	い 
\\	もうちょっと
\\	の画面を暗くしてくれませんか。	
\\	田中さんは暗やみきょうふしょうなので、部屋が暗いとねむれないんですって。	
\\	最近、暗いニュースばっかりだけど、こんな暗い世の中でも滅茶苦茶良いことだってあるんだよ。えっと、それで…うーん。何を言いかけたのか忘れちゃったんだけど、とにかく、僕と結婚してくれませんか?	
\\	い 
\\	(くら). 
\\	暗	
\\	器用	
\\	きよう	
\\	な 
\\	コウイチは手先がとても器用です。	
\\	ビエトはワニカニの着ぐるみを着たまま、器用にコカ・コーラを飲んだ。	
\\	彼女はいつも明るくて良い子なんだけど、あまり器用じゃないので、首にしようか迷ってるんです。	
\\	器, 用	
\\	疑問	
\\	ぎもん	
\\	コウイチが本当に肉を食べられないかは疑問です。	
\\	田代さんのあの日の行動には疑問がのこります。	
\\	フグ市長は好きですが、私たちの地域に何をしてくれたかって聞かれると、まだ疑問です。	
\\	疑, 問	
\\	酒好き	
\\	さけずき	
\\	な 
\\	かれ氏は酒好きなんで、心配なんですよね。	
\\	あの酒好き、今日も朝から引っかけてやがるぜ。	
\\	私の妻は酒好きなので、アル中にならないか心配だ。	
\\	(酒 
\\	好き) 
\\	すき 
\\	ずき.	酒, 好	
\\	飲み会	
\\	のみかい	
\\	トーフグ主さいの飲み会はとても楽しいですよ!	
\\	最近、飲み会で一気飲みを強要されて急せいアルコール中どくで死ぼうする学生がふえているようです。	
\\	お袋が飲み会には行っちゃダメだって言うんだよね。お前んとこの親がアルコールとかに対して厳しくないの、羨ましいよ。	
\\	飲む 
\\	会 
\\	飲む 
\\	会. 
\\	飲, 会	
\\	熱心	
\\	ねっしん	
\\	な 
\\	子どもは熱心に平仮名をかいていた。	
\\	かれが毎朝駅前でスキップの練習をしている熱心なすがたに心打たれました。	
\\	トーフグの新入社員はみんな、仕事にやる気を見せることに熱心で、寝てもいい時間でさえも、仕事をしていた。	
\\	ねっ 
\\	ねつ).	熱, 心	
\\	全然	
\\	ぜんぜん	
\\	ハリケーンとの天気予ほうがあったのに、全然雨がふらなかった。	
\\	「すみませんが、ちょっとたのみ事をしてもいいですか。」「はい、全然いいですよ!」	
\\	「分かりましたか?」 「申し訳ないのですが、チンプンカンプンで全然分かりません。今言ったことをもう少しゆっくり言っていただけませんか?」	
\\	全然. 
\\	全, 然	
\\	皆さん	
\\	みなさん	
\\	それは皆さんが決めてください。	
\\	わたしは皆さんと仲良くなりたいだけなんです。決して下心なんてありません。	
\\	皆さん、こんにちは。今週中には、トーフグセミナーの日程を確定させたいと思っています。数日中にビエトと話しをして、その後で皆さんに日程をお伝えさせて頂きます。	
\\	さん 
\\	皆	
\\	皆様	
\\	みなさま	
\\	皆様、間もなく当ひ行きはロサンゼルス空港に着りくします。	
\\	皆様のごけんこうとご多こうをおいのり申し上げます。	
\\	「皆様、ご着席いただけますか?」「お言葉ですが、それは良い考えではないと思います。何故だか分かりませんが、我々の椅子に接着剤が付いているのです。」	
\\	皆, 様	
\\	予想	
\\	よそう	
\\	する 
\\	の 
\\	わたしは未来を予想できるんです。	
\\	今日もコウイチはちこくするであろうというわたしの予想は見事に外れてしまった。	
\\	予想外の出来事が生じているため、今日は氷の舟で溶岩のプールを渡ることはできません。	
\\	予, 想	
\\	表情	
\\	ひょうじょう	
\\	そんなこわい表情をしないでください。	
\\	ワニカニの売り上げレポートを見たコウイチの表情はとてもうれしそうだった。	
\\	子供達の表情は、みんなとても活き活きとしていた。	
\\	表, 情	
\\	料金	
\\	りょうきん	
\\	日本はガスの料金は高いですか。	
\\	けいたい電話の料金プランがふくざつすぎて分かりません。分かりやすくせつ明してくれませんか。	
\\	「申し訳ありませんが、この料金が精一杯です。」「うーん。まぁ、妥当な値段のようですね。分かりました。これで手を打ちましょう。」	
\\	料, 金	
\\	感じ	
\\	かんじ	
\\	あの人感じ悪いよね。	
\\	きのうオフィスに行った時にはコウイチ子がにんしんしている感じはしませんでしたよ。	
\\	コウイチさん、社長になるって、どんな感じですか?	
\\	感	
\\	感情	
\\	かんじょう	
\\	の 
\\	おこると感情のコントロールがうまく出来ません。	
\\	これは大事なことですから、感情に左右されずにれいせいにはんだんしてくださいね。	
\\	感情に流されすぎてはいけないよ。自分の感情に負けることには何の価値もないよ。	
\\	感, 情	
\\	熱	
\\	ねつ	
\\	今日は少し熱があるので外出できません。	
\\	なんてこった!ちょっと熱っぽいだけだと思ってたら、体温40度以上あるじゃん!	
\\	この五百度の熱を持った鉄をさわってみて、その体験を感想文に書いてください。	
\\	熱	
\\	指先	
\\	ゆびさき	
\\	の 
\\	マリアって、左の小指の指先のホクロがセクシーだよね?	
\\	コウイチは、考え事をしている時、サッカーボールを指先でクルクルと回すくせがあります。	
\\	指先がかじかんでいます。	
\\	指, 先	
\\	〜様	
\\	さま	
\\	コウイチ様は、何もかもが自分の思い通りにならないと気がすまないんですよ。	
\\	日本で手紙を送る時には相手の名前に様をつけないと失礼になります。	
\\	トーフグのコウイチ様御一行九名様ですね。お席のご用意ができましたのでどうぞ。	
\\	様	
\\	初めて	
\\	はじめて	
\\	の 
\\	初めてオバマ大とうりょうと会ったしゅんかんから意気投合した。	
\\	初めての銀行業むに四苦八苦しています。	
\\	「もしかして、日本はこれが初めて?」「はい。そうなんです。」「そうなんだ。日本はどう?」「かなり気に入ってるわ。」	
\\	初 
\\	めて 
\\	(はじ). 
\\	初	
\\	仲良く	
\\	なかよく	
\\	さとみちゃんとは、仲良くなれそう?	
\\	トーフグの社いんはあまり仲良くなく、おたがいほとんど話をしません。	
\\	仲良くなれば、彼女が中々面白い酒飲みだということが分かるよ。	
\\	仲, 良	
\\	駅前	
\\	えきまえ	
\\	駅前のマンションをさがしています。	
\\	せんきょの期間は多くのせいじ家が駅前でスピーチをする。	
\\	なんで私に、駅前で撮ったこんな変な自撮り写メを送ってきたの?	
\\	駅 
\\	前 
\\	駅, 前	
\\	福島	
\\	ふくしま	
\\	このバンドのボーカルは、福島の出身です。	
\\	福島の子どもたちがかいた絵の中の、この銀色の星空を見て下さい。	
\\	福島問題に関するメールは、全て私を
\\	に入れてもらえますか?	
\\	福, 島	
\\	目標	
\\	もくひょう	
\\	川村さんの人生の目標はなんですか。	
\\	五十メートル先の目標に向かって矢を放ちました。	
\\	トーフグの目標について、もう少し具体的に話をして頂けませんか?	
\\	目 
\\	目次 
\\	目, 標	
\\	神様	
\\	かみさま	
\\	あの神様、まだおむかえを待ってるの?	
\\	ギターの神様と言えば、だれを思いうかべますか?	
\\	あれは多分神様のいたずらだったのだろう。	
\\	神, 様	
\\	女神	
\\	めがみ, じょしん	
\\	この女神の人形の前がみを少し短くしてください。	
\\	あの女神の身勝手さには、開いた口がふさがらなかったね。	
\\	女神のように美しい女声に出会った。	
\\	女 
\\	目 (め) 
\\	女, 神	
\\	仲良し	
\\	なかよし	
\\	な 
\\	の 
\\	あの二人、仲良しなふりをしているだけで、本当は仲が悪いと思うよ。	
\\	先日、日本に住んでいる仲良しの友人から、けっこんしきのしょう待じょうがとどきました。	
\\	「今何しているの?」「うんこしてるんだよ。放っておいて。」「はぁ?いくら仲良しだからって、そんなことまで教えてくれなくても良いわよ。親しき仲にも礼儀ありって言うでしょ!」	
\\	仲, 良	
\\	緑色	
\\	みどりいろ	
\\	な 
\\	の 
\\	この緑色なゆで玉子は、ちょっと食べる気になれませんね。	
\\	これはかん全にこ人てきな好みなんですけど、コウイチにはかみを緑色にそめてもらいたいんですよねー。	
\\	「このネクタイ、どうかな?」 
\\	「そうねぇ。こっちの緑色のをしてみてよ。」 
\\	「わかったよ。これでどう?」 
\\	「うん。そっちの方がいいわね。」	
\\	緑, 色	
\\	緑	
\\	みどり	
\\	わたしの主人の目の色は、緑がかったはい色です。	
\\	緑のじゅうたんとか、ちょっと悪しゅ味だよね。	
\\	「僕はこの赤いのが気に入ったよ。」 
\\	「私は緑の方がいいわ。」 
\\	「ああ、そうかい。じゃあ、好きなようにしろよ。」	
\\	緑	
\\	心願	
\\	しんがん	
\\	サーモンの心願には心が打たれますね。	
\\	今回のハリケーンのひがいが少なくてすむようにというのがわたしの心願です。	
\\	私はその神社で心願成就の御札を買いました。	
\\	心, 願	
\\	宿題	
\\	しゅくだい	
\\	ななめから見てるから、宿題の答えが見えづらいんだよね。	
\\	今日のわたしの宿題、ちょっと見てもらえないかな?	
\\	「でも、ママ!僕…」「つべこべ言わないの。さあ、自分の部屋へ行って、宿題をしなさい。」	
\\	宿, 題	
\\	熱い	
\\	あつい	い 
\\	おふろを四十一度まで熱くしてください。	
\\	お茶をいれようかと思うんですけど、熱いのとつめたいの、どっちがいいですか。	
\\	「どうぞ、お熱いうちにお召し上がりくださいね。」「有難うございます。でも、ええっと、ちょっと熱すぎるので、まだ食べられません。」	
\\	暑い. 
\\	暑い, 
\\	暑 
\\	あつ. 
\\	熱	
\\	人情	
\\	にんじょう	
\\	人情のあたたかさに生かされています。	
\\	コウイチ様は人情味にあふれているすばらしいお方ですが、ビエト様はひとかけらの人情もないごく悪ヤクザです。	
\\	私達は
\\	の毎日のブログ記事を通して、日本人の生き方や人情を描き出すことを目指しています。	
\\	人, 情	
\\	億	
\\	おく	
\\	このおく歯までしっかりとどく歯ブラシで、今年は五億円を売り上げました。	
\\	あんなに小さかった子がこんなに大きくなって、今や毎年三億ドルをかせぐ大物になっているなんて、本当にしんじられないよ。	
\\	出張の領収書は全部とっておいてください。経費は二億円までなら後で精算されますので。	
\\	億	
\\	想定	
\\	そうてい	
\\	する 
\\	あの手がたくさんある神様の絵がここにかざられることを想定しています。	
\\	まだ手に入れていないものを、手に入れたとの想定で行動するのは良くない。	
\\	地震を想定した避難訓練を行います。	
\\	想, 定	
\\	画像	
\\	がぞう	
\\	勝手にわたしの画像を使わないでくださいよ!	
\\	けいさつにコウイチの暗殺事けんにかんするしょうこ画像をてい出しました。	
\\	残念。この画像はちょっとボケてるので使えないよ。	
\\	画, 像	
\\	感想	
\\	かんそう	
\\	する 
\\	マイケルの日かは、毎朝同りょう全員に前日に見たアニメの感想を話すことです。	
\\	文学がきらいなわたしの弱点の一つは、読んだ小せつの感想を聞かれても「よかった」としか言えないことである。	
\\	私達のサイトについて、感想をお聞かせください。どんなご意見でも結構です。	
\\	感, 想	
\\	旅館	
\\	りょかん	
\\	日本に旅行に行ったら旅館にとまってみたいです。	
\\	来月日本に旅行に行くんだけど、旅館とホテルとどっちにとまった方がいいと思う?	
\\	「なんでこの旅館、こんなにゴキブリが出るの?」「最近雨ばっかり降ってるからじゃない?」	
\\	旅, 館	
\\	神風	
\\	かみかぜ	
\\	この神社には神風がふいています。	
\\	第二次世界大戦の末期には日本ぐんは神風とっこうたいを使ってアメリカぐんをこうげきしました。	
\\	五年前、神風が吹き、コウイチはビエトの軍隊を海岸から後退させた。	
\\	風 
\\	神 
\\	神, 風	
\\	王様	
\\	おうさま	
\\	王様は料理が大好きです。	
\\	あの王様は、とんでもなくケチなんだ。	
\\	王様は、力士と尻相撲をしたがっています。	
\\	王 
\\	様. 
\\	王, 様	
\\	左利き	
\\	ひだりきき	
\\	の 
\\	なぜかいつも左利きの女せいにひかれるんです。	
\\	左利きの人向けの仕事に力を入れています。	
\\	「どういったご用件でしょうか?」「あなたが左利きかどうかだけ知りたいんですが。」「あいにくそれはお伝えできないことになっております。」	
\\	左 
\\	利く 
\\	左, 利	
\\	小指	
\\	こゆび	
\\	なんでグラスで飲み物を飲む時に小指を立てるの?	
\\	いってぇー、そこのタンスの角に右足の小指ぶつけた。	
\\	またシュートミスっちまったぜ。突き指してる小指のせいで今日はまじでダメだわ。	
\\	小, 指	
\\	宿	
\\	やど	
\\	十年前にそこをおとずれた時には、こんな宿は全くありませんでした。	
\\	宿がまだ決まってないんだよ。	
\\	この宿は歴史的にも有名で、宿泊客だけでなく見物客も訪れる。	
\\	(やど) 
\\	宿	
\\	本館	
\\	ほんかん	
\\	本館には、ようじ教室の部屋がへいせつされています。	
\\	本館のれきしをゼロから教える本を作ります。	
\\	あの百貨店の本館で働いている店員は、別館で働いている人たちに比べて、なんか働かされてるって感じだった。愛想がやけに悪いっていうかさ。	
\\	本 
\\	本, 館	
\\	暗記	
\\	あんき	
\\	する 
\\	短期間で丸暗記しても、すぐわすれてしまうだろう?	
\\	さくら木さんは、円しゅうりつを何けた目まで暗記していますか?	
\\	彼はクラスのお調子者で、いつもボケているが、暗記力はすごい。	
\\	暗, 記	
\\	水銀	
\\	すいぎん	
\\	の 
\\	海魚の体が大きいほど体内にたくさん水銀がたまっています。	
\\	水星と水銀はえい語で共にマーキュリーと言いますが、これはぐう然でしょうか。	
\\	昔はみんな水銀体温計をお尻に差し込んでいたんだよね。めっちゃうける。	
\\	水, 銀	
\\	中指	
\\	なかゆび	
\\	の良犬に中指をかまれました。	
\\	右手の中指に出来たタコが、かなりかたくなってきてます。	
\\	中指を立てるんじゃない!	
\\	中, 指	
\\	鏡	
\\	かがみ	
\\	鏡の横の植木ばちに花を植えました。	
\\	コウイチは毎日のように鏡の中の自分に向かって話しかけているあやしい人です。	
\\	「今朝、鏡が盗まれたよ。数週間前に歯ブラシと櫛を盗まれたばかりなのに!」「まぁ、人生そんなもんだよ。」	
\\	鏡	
\\	映画	
\\	えいが	
\\	の 
\\	もうジブリの最新映画は見ましたか。	
\\	わたしはロマンス映画が好きなのに、主人は戦争映画ばっかり見たがるんです。	
\\	「僕は映画スターになって、自家用ジェット機を買うよ。それから、君は僕のお嫁さんになるんだ。」 「おじいちゃん、何寝ぼけたこと言ってんの。」	
\\	映, 画	
\\	殺人	
\\	さつじん	
\\	の 
\\	あの母親の殺人の動きは分からなくもない。	
\\	今まで、オンライン日本語学習教ざいを専門としていた会社が、 初めて本かくてきな殺人ゲームを発売しました。	
\\	一人の殺人者として言わせてもらうけど、俺はあまり彼女を評価していないよ。殺人方法がつまらないからね。	
\\	殺, 人	
\\	楽器	
\\	がっき	
\\	の 
\\	楽器にも色々ありますが、どんな楽器が好きですか。私は打楽器が好きです。	
\\	まずは楽器を楽ふ通りにえんそうしてみてください。	
\\	「あなたの
\\	を持ってくるのを忘れてしまったわ。」「別にいいよ、またの機会で。でも、その代わりに
\\	として何か楽器を弾いてくれない。」	
\\	楽 
\\	音楽 
\\	楽 
\\	がく 
\\	がっ, 
\\	楽, 器	
\\	当然	
\\	とうぜん	な 
\\	の 
\\	「すごい!あのテストで百点とったの?」「当然さ。」	
\\	かん字が上手くなりたければ毎日少しずつ練習するのは当然でしょう!	
\\	あんな素敵な奥さんがいるなんて、お前は本当に運がいいよな。当然のことだなんて思うなよ。	
\\	当然!	
\\	当, 然	
\\	自然	
\\	しぜん	
\\	な 
\\	日本の自然が大好きです。	
\\	コウイチとコウイチ子がこいに落ちるのはとても自然なことでした。	
\\	私達はアメリカ政府から、フグを自然に返すように言われました。	
\\	青葉 
\\	な 
\\	自然な, 
\\	自 
\\	し 
\\	じ! 
\\	し. 
\\	自, 然	
\\	一億	
\\	いちおく	
\\	コウイチは東京に一億円のマンションを持っています。	
\\	日本という小さな島国に一億もの人が住んでいることがしんじられない。	
\\	宝くじで一億ドル当たったけど、私は既に兆万長者なので全部慈善活動に寄付したよ。	
\\	一, 億	
\\	一億円	
\\	いちおくえん	
\\	一億円とか大きな注文は入ってないけど、コンスタントに注文はあるよ。	
\\	トーフグがわたしに一億円をめぐんでくれることが決定したというお知らせを受けとりました。	
\\	「さっき一億円が入ったスーツケースを無くしちゃったんだけど。」「えぇっ!見つかるといいね。」	
\\	一, 億, 円	
\\	整然	
\\	せいぜん	
\\	東京は人が多いけれど整然としています。	
\\	ダリンはいつも理路整然としたせつ明をしてくれます。	
\\	オフィスの棚に整然と並べられていたワインボトルを、マミが全部空けてしまった。	
\\	整, 然	
\\	情熱	
\\	じょうねつ	
\\	な 
\\	情熱さえあれば何だってできるよ!	
\\	ビエトがこんなに情熱てきだったなんて、信じられません。	
\\	コウイチは、日本とアメリカの伝統を融合させることに、生涯情熱を捧げることを決意しました。	
\\	情, 熱	
\\	体育	
\\	たいいく	
\\	小学校の時、体育のマラソンが大きらいでした。	
\\	次の体育のじゅ業の前に、つめを切ってきてください。	
\\	「体育の日に、一緒にバレーボールしない?」「ああ、確か祝日だったよね。十月十日とかだっけ?」「そうそう。祝日。でも、もう十月十日じゃないよ。ハッピー・マンデー制度ができてから、十月の第二月曜日になったんだよ。」	
\\	教育 
\\	育 
\\	体, 育	
\\	情け	
\\	なさけ	
\\	コウイチ子は、コウイチが情けのかけらもない男だと知って、別れました。	
\\	日本語には、「旅は道づれ世は情け」ということわざがあります。	
\\	テストは70点が合格点だったが、私は69点だったのにも関わらずお情けで合格させてもらえた。	
\\	(け 
\\	情	
\\	お願いします	
\\	おねがいします	
\\	神様、お願いします!日本語が上手くなりたいんです。	
\\	ゆうれい会員でもかまわないので、お願いします。	
\\	通路側の席をお願いします。窓から外の景色を眺めるのが嫌いなんです。	
\\	願	
\\	詩	
\\	し	
\\	詩が大好きで、毎日詩を読んでいます。	
\\	わたしの心のやみを詩で表げんしてみたんだけど、ちょっと読んでみてくれない?	
\\	今日の宿題は、日本語で詩を書くことです。	
\\	詩	
\\	詩人	
\\	しじん	
\\	サーモンは今世紀を代表する詩人です。	
\\	わたしの友人に詩人が一人いるんですが、かれの家の中はかべ中詩でおおわれているんです。	
\\	君は詩人で、暗喩を使うことが好きなのかもしれないけど、もっと分かりやすく話さないと相手はわからないよ。	
\\	詩, 人	
\\	詩歌	
\\	しいか, しか	
\\	の 
\\	毎日詩歌を楽しんでいます。	
\\	フグにはすばらしい詩歌を作るかくれた才のうがあります。	
\\	詩歌とは日本の詩のことですが、一般的には漢詩や和歌、俳句のことを指します。	
\\	い 
\\	しか 
\\	しいか 
\\	詩, 歌	
\\	練習	
\\	れんしゅう	
\\	する 
\\	一年間練習して、ようやく有気音とむ気音を聞き分けられるようになりました。	
\\	日々の練習が実をむすび、アドリブでギターがひけるようになりました。	
\\	僕が世界ランキング一位を維持し続けているのは、日本語を毎日七時間練習しているからだと思います。	
\\	練, 習	
\\	選手	
\\	せんしゅ	
\\	あのハンサムな選手はだれですか。	
\\	最近、日本人のプロ野球選手がメジャーリーグでもプレイするすがたがよく見られます。	
\\	コウイチは素晴らしい剣道の選手です。	
\\	手 
\\	運転手 
\\	選 
\\	手 
\\	手 
\\	選, 手	
\\	問題	
\\	もんだい	
\\	ワニカニの売り上げが落ちるのはトーフグにとって大きな問題です。	
\\	この数学の問題、三十秒間どう考えても分からなかったからわたしの代わりにちょっとやっておいてくれない?	
\\	「お願いがあるんだけど。」「いいけど、何?」「週末、私の犬の世話をしてもらえないかな?」「問題ないよ。任せといて。」	
\\	問題.	
\\	問, 題	
\\	実感	
\\	じっかん	
\\	する 
\\	この売り上げ表を見ると、ワニカニが売れているんだなと実感できます。	
\\	母が死んだという知らせを聞いても、すぐには実感がわかなかった。	
\\	アメリカへ行って初めて、日本はなんと小さい国かと実感した。	
\\	実 
\\	じっ 
\\	実, 感	
\\	大好き	
\\	だいすき	
\\	な 
\\	かれしが前がみをいじる仕草が大好きです。	
\\	なんだかんだ言っても、フグはワニカニのことが大好きなんですよ。	
\\	天然ボケで、一緒にいていつも楽しいなって思うんだ。俺、お前のこと大好きだよ。	
\\	好き 
\\	大 
\\	好き 
\\	大 
\\	大, 好	
\\	映像	
\\	えいぞう	
\\	の 
\\	でマイケルの映像を四千四百四十四回見ると、のろいにかかるらしいよ。	
\\	映画リングを見てからというもの、古い戸からさだ子が出てくる映像が頭からはなれない。	
\\	あの新しい日本映画見た?映画自体はいまいちだったんだけど、映像が物凄く美しかったよ。	
\\	映, 像	
\\	回想	
\\	かいそう	
\\	する 
\\	の 
\\	この映画のこの回想のシーンがすごくすきなんだよね。	
\\	コウイチはえんがわでネコと一しょにとりとめのない回想にふけっていた。	
\\	ビエトが時々、仕事中にヤクザ時代のことを回想しているのをわたしは知っている。	
\\	回, 想	
\\	右利き	
\\	みぎきき	
\\	の 
\\	わたしの息子は右利きではありません。	
\\	ぼくは左利きなのに、お母さんが間ちがえて右利きのグローブを買っちゃったんだ。	
\\	彼女のこと、すごく好きだったんだけど、右利きだって分かってから冷めちゃったんだよね。どうしてかは分からないけど。	
\\	右 
\\	利く 
\\	右, 利	
\\	輪	
\\	わ	
\\	早くこの輪投げの輪をなげてくださいよ!	
\\	パーティーに行くと、友だちの輪が広がります。	
\\	うわっ!コウイチ、なんで金の輪っかのイヤリングなんて付けてるの?	
\\	輪	
\\	学問	
\\	がくもん	
\\	する 
\\	学問と文学は切っても切れない関係にある。	
\\	こ人てきには、大学って学問する所じゃなくて、「大そつ」っていうしかくを取りに行くための所だと思ってるんだよね。	
\\	学問を疎かにしてはいけない。	
\\	学, 問	
\\	課長	
\\	かちょう	
\\	先月、課長にしょう進しました。	
\\	課長になれたのはうれしいんだけど、ざん業代が出ないから実はきゅう料が下がったんだよね。	
\\	課長が少しうざいんだよね。他の社員に良い印象を与えようといつも頑張っちゃってさ。	
\\	社長 
\\	長 
\\	課 
\\	課, 長	
\\	新宿	
\\	しんじゅく	
\\	この新宿のホテルのシャワーは、水かぬるま湯しか出ない。	
\\	新宿は毎年、わたしの行ってみたい場所ランキングの中で最下位です。	
\\	新宿二丁目は今はゲイタウンとして知られていますが、昔は危険な香りのする赤線地区でした。	
\\	しゅく 
\\	じゅく. 
\\	新, 宿	
\\	銀	
\\	ぎん	
\\	の 
\\	けっこん指わは、銀か金、どちらにする?	
\\	いくつになっても銀せいひんがすきです。	
\\	赤信号で、銀の軽トラが私の横に止まった。	
\\	銀	
\\	銀色	
\\	ぎんいろ	
\\	の 
\\	初めにお伝えしていたように、これを銀色にぬり直してください。	
\\	コウイチの銀色のヒゲをわたしに売ってくれませんか?	
\\	私のダンスの相手は、あの銀色のコスチュームを着たフグって男よ。	
\\	銀, 色	
\\	特選	
\\	とくせん	
\\	する 
\\	の 
\\	親分の絵が、コンクールで特選に入賞したぞ!	
\\	こちらが、特選されたコスプレイヤーの方々です。	
\\	コウイチは、毎晩寝る前に特選の極上黒大蒜を食べるそうだ。	
\\	特, 選	
\\	駅	
\\	えき	
\\	新宿駅でおりて下さい。	
\\	上野駅の駅の前にはコウイチが一億円かけて出店したトーフグやきぎょうざの店があります。	
\\	ようやく直接お話することができてよかったです。まさか駅でバッタリお会いするなんて思ってもいませんでしたよ。	
\\	えき 
\\	駅	
\\	駅長	
\\	えきちょう	
\\	すきなアイドルが一日駅長にえらばれました。	
\\	駅長と川田さんは親友だからこっそりと切ぷ代をわり引してもらっているんだって。	
\\	うわっ!あの駅長、今電車に唾はかなかったか?お前、見たか?	
\\	長 
\\	社長, 
\\	長 
\\	駅, 長	
\\	自殺	
\\	じさつ	
\\	する 
\\	東アジアの国の中で最も自殺りつが高い国はかん国です。	
\\	(シロナガスクジラ)」という名前の
\\	ゲームで、十代のロシア人少年少女たちが130人以上も自殺したそうです。	
\\	日本の自殺名所のリストを、メールで送ってもらえますか?あと、ファイルを送る際に圧縮して頂けると助かります。	
\\	自, 殺	
\\	力士	
\\	りきし	
\\	力士のサインがほしいなら両国に行ってみたら?	
\\	初デートで、力士はずっと前がみをなでつけてたんだよね。	
\\	お集まりの皆さん、本日は度重なる力士の不祥事に関する会議にご出席いただき有難うございます。	
\\	力, 士	
\\	同感	
\\	どうかん	
\\	する 
\\	わたしはコウイチの意見に同感ですね。	
\\	今までにこんなに同感したことがないぐらい、思いっきり同感しています。	
\\	「私、フグのこと嫌い。」「同感だわ。」	
\\	同, 感	
\\	食器	
\\	しょっき	
\\	この食器をテーブルの上にならべてくれませんか。	
\\	あなたが池に落としたのはこの金の食器ですか、それともこの銀の食器ですか。	
\\	「どうして、食器が一式多いの?」「お前がフグを招待したからだろ。」「あぁ、そっか。 ごめんなさい。 フグは来ないのよ。 数日前に元カノが街に遊びにやってきて、よりを戻したから、サンクスギビングは彼女と一緒に過ごすんだって。」	
\\	しょく.	食, 器	
\\	感謝	
\\	かんしゃ	
\\	する 
\\	な 
\\	感謝祭の日にはみんなに感謝しましょう。	
\\	言葉には出さないけど、つまにはいつも感謝しているんです。	
\\	私達のお客様全員に感謝したいと思います。私達は、皆さん無しには、ここには居られません。	
\\	感, 謝	
\\	五感	
\\	ごかん	
\\	五感のうちどれか一つを失ってしまうとしたら、どれを選びますか?また、その理由も教えてください。	
\\	さく夜のゆめで、今までにかいだことのないにおいを感じました。五感ってねている間も働いているんでしょうか。	
\\	五感に関する日本語の単語と動詞の他に、第六感に関する語彙も学習してください。	
\\	五, 感	
\\	思想	
\\	しそう	
\\	コウイチの思想にはついていけません。	
\\	きのう、ニュースを見ていたら、ビエトが思想犯としてたいほされたニュースが出て本当にびっくりしたよ。	
\\	コウイチはトーフグの創始者であり、近代日本を代表する啓蒙思想家である。	
\\	思, 想	
\\	親指	
\\	おやゆび	
\\	親指を放してくれ!	
\\	きんちょうすると、親指がふるえます。	
\\	親指にタコができました。	
\\	親, 指	
\\	車輪	
\\	しゃりん	
\\	バイクの車輪が外れてしまいました。	
\\	暗殺のきょうきには車輪が使われたそうです。	
\\	今日、ちょっと自転車貸してもらえないかな?今、車輪を取り替えてもらっているので、自分のが無いんだよね。	
\\	車, 輪	
\\	謝る	
\\	あやまる	
\\	コウイチちゃんに今すぐ謝りなさい!	
\\	何も悪いことをしてないのに謝られるとぎゃくにこっちがこまります。	
\\	君が満足する結果が得られないのはかわいそうだけど、自分がしたことじゃないのに謝るつもりはないよ。	
\\	う 
\\	(あやま) 
\\	謝	
\\	感動する	
\\	かんどうする	する 
\\	感動してなみだが止まりません。	
\\	トーフグのせつ立ひ話を聞いて、とても感動しました。	
\\	なんてこった!今までの人類の歴史上で最高の日本語学習教材を作っちまったぜ!みんなマジで感動するだろうな!	
\\	する. 
\\	感, 動	
\\	映る	
\\	うつる	
\\	スクリーンに社内用しりょうが映っています。	
\\	鏡に映った自分のかおを見て、「なんて美しいんだ」と毎日思っています。	
\\	コウイチは湖の水面に映る美しい山に感動し、ホロリと涙を流した。	
\\	う 
\\	(る) 
\\	打つ (うつ) 
\\	うつうつうつうつ. 
\\	映	
\\	放れる	
\\	はなれる	
\\	馬が馬小屋から放れました。	
\\	苦労して働いて、ようやくしゃっ金から放れることができました。	
\\	ようやく子どもから手が放れたと思ったんだが、五十三歳の妻がまた妊娠してしまってね。	
\\	う 
\\	放す 
\\	(れる) 
\\	放す. 
\\	放	
\\	育てる	
\\	そだてる	
\\	ビエトとコウイチは、オフィスで野さいを育てるのが大好きです。	
\\	この子犬をわたしの代わりに育ててくれませんか。わたしが育てたいんですが、来月から海王星に転きんになってしまったんです。	
\\	小学校の時、クラスで何を育ててた?うちのクラスはおたまじゃくしとハムスターとなすを育ててたんだけど、ハムスターはみんな逃げてねこに食べられちゃった。	
\\	育つ, 
\\	てる. 
\\	(てる) 
\\	育つ. 
\\	育	
\\	追い付く	
\\	おいつく	
\\	前日にシアトルまでと歩で出発したコウイチに車で五分もかからず追い付きました。	
\\	100メートル走でクリステンに追い付くことができたらオリンピックに出られるよ。	
\\	の塗り絵は人気すぎて、生産が需要に追い付かなかった。	
\\	追う 
\\	付く 
\\	追, 付	
\\	選ぶ	
\\	えらぶ	
\\	だれを殺人するかなんて、選べないよ。	
\\	先生の話を聞かずに、ずっとどっちの女の子を選ぼうか考えていました。	
\\	良いフグのコスチュームをお選びになりましたね。	
\\	う 
\\	(えら).
\\	選	
\\	感じる	
\\	かんじる	
\\	見るのではなく、心で感じてください。	
\\	ワニカニの台頭により、トーフグは会社に自分のい場所がなくなっていると感じている。	
\\	最近兄に冷たくされていると感じます。	
\\	う 
\\	感	
\\	戦う	
\\	たたかう	
\\	第二次世界大戦で日本とアメリカは戦いました。	
\\	わたしたちは日本語が上手くなるために、毎日このかん字というきょ大な化け物と戦っているんです。	
\\	剣道は通常、一対一で戦う武道だが、コウイチは五年前、二人の男を同時に相手にしなくてはいけなかった。	
\\	戦い, 
\\	戦い 
\\	戦	
\\	追いかける	
\\	おいかける	
\\	田中さんは山田さんのことが大すきで、毎日追いかけています。	
\\	大へんだ!コウイチったらパスポートを持たずに日本へ出発してしまったよ。今すぐ追いかけてこのパスポートをわたしてあげてくれない?	
\\	シャチの群れを追いかけてるフグを見たよ。	
\\	追う 
\\	かける 
\\	追う 
\\	追	
\\	養う	
\\	やしなう	
\\	ワニカニがトーフグを養っているの?それともトーフグがワニカニを養っているの?	
\\	コウイチのにんじゅつにだまされない目を養う必要があります。	
\\	たまにはご主人にも感謝しろよ。あいつだって、立派に家族を養ってきたじゃないか。	
\\	う 
\\	(やしな) 
\\	養	
\\	手伝う	
\\	てつだう	
\\	れいぞうこをかた付けるの、手伝ってくれない?	
\\	ビエトに、コウイチに一億ドルのほけん金をかけたので、やつを暗殺するのを手伝ってほしいとたのまれたんだ。	
\\	ごめんなさい。お手伝いしてあげたいんだけど、私にはどうにもできないんです。	
\\	た 
\\	だ 
\\	手, 伝	
\\	想像する	
\\	そうぞうする	する 
\\	コウイチは仕事中、たまに想像できないような動きをする。	
\\	てらや神社をめぐってごしゅいんちょうにごしゅいんを集める仕事って、想像できる?	
\\	もし指を喉に突っ込む以外でゲロを吐く方法を学びたいなら、裸のコウイチを想像するといいよ。	
\\	想, 像	
\\	殺す	
\\	ころす	
\\	お前なんか殺してやる!	
\\	魚は殺さずに生きたまま食べた方がおいしいんだよ。	
\\	私、シンデレラは実際、継母のこと殺したくて仕方がなかったと思うんだよね。	
\\	う 
\\	(ころ) 
\\	殺	
\\	見返る	
\\	みかえる	
\\	か去を見返ってるひまはないんだよ。	
\\	赤さまの見返るおすがたがただただかわいすぎてモエまくっておりまする
\\	ガシャーンと大きな音がして、私は思わず後ろを見返った。	
\\	見 
\\	返る, 
\\	見, 返	
\\	注文する	
\\	ちゅうもんする	する 
\\	銀色のラメの入ったラッピングペーパーを注文しました。	
\\	アマゾンで、間ちがえてコウイチの使用ずみブリーフを注文してしまった。	
\\	外食でも、お持ち帰りで注文しても、僕はどちらでも構わないよ。	
\\	注文 
\\	注文 
\\	注, 文	
\\	話題	
\\	わだい	
\\	の 
\\	友だちと話題のフグをかいに行きました。	
\\	明くる朝、オフィスに行くと、コウイチのけっこんしきとよく日の電げきりこんの話題で持ちきりだった。	
\\	ここまでは理解していただけたかと思うのですが。それでは、次の話題に移りましょうか。トマトは、原産地が南米なだけでなく、フグの一番好きな野菜でもあるってことは知っていましたか?	
\\	話, 題	
\\	別れる	
\\	わかれる	
\\	コウイチ子はコウイチと別れてビエトと付き合いました。	
\\	コミュニケーション不足がげんいんで別れました。	
\\	彼女と別れる時に、涙が止まらなかった。	
\\	う 
\\	分ける? 
\\	わける 
\\	わかれる. 
\\	け 
\\	る 
\\	け, 
\\	かれ, 
\\	わかれる.
\\	別	
\\	共有する	
\\	きょうゆうする	する 
\\	不かく定な情ほうは共有しないでください。	
\\	自動車を共有するという考えにはさん成できません。	
\\	動画を共有する方法を教えてください。	
\\	共有 
\\	共有 
\\	共, 有	
\\	駅員	
\\	えきいん	
\\	その駅員の黒目は普通の人よりも大きいと思います。	
\\	この駅の駅員は、みんなお昼に一杯二百八十円のかけそばを食べる。	
\\	あの駅員、こう見えてすっごくお金持ちなのよ。	
\\	駅, 員	
\\	様々	
\\	さまざま	
\\	な 
\\	わたしの大学は大きくて有名なので、様々な国からりゅう学生が集まっています。	
\\	その発言に対しては様々な受けとり方ができると思います。	
\\	トーフグの社員は、どこの出身かによって、肌の色も体型も大きさも様々で、みんなそれぞれユニークな個性をもっています。	
\\	々). 
\\	ざま 
\\	さま 
\\	様, 々	
\\	能力	
\\	のうりょく	
\\	わたしには、グーグルをとてもこうりつてきに使う能力があります。	
\\	ワニカニの例文作りには、へんな文章をかく能力が求められる。	
\\	私は英語のネイティブスピーカーだが、日本語でテレパシーをする能力がある。	
\\	能, 力	
\\	約	
\\	やく	
\\	ゴボウは地上約2メートルまでのびます。	
\\	その前に、約三年間かぶしきとり引の分野で働いていた経験があります。	
\\	今までで、約百万人の人が
\\	のフェイスブックページに「いいね」をした。	
\\	約	
\\	知り合い	
\\	しりあい	
\\	知り合いにたのんで、トーフグのコウイチからサインをもらいました。	
\\	今知り合いにれんあい相だんしてるから、カラオケには行けないかも。ごめんね!	
\\	ここだけの話なんだけど、上司が知り合いの人妻と浮気しているのよね。	
\\	知る 
\\	合う 
\\	知る 
\\	合う. 
\\	知, 合	
\\	〜的	
\\	てき	
\\	な 
\\	自分的には上手にアメリカ的なスピーチができたと思ってたんだけどな。	
\\	コウイチの神ぴ的なえみが、こ人的にめっちゃツボなんだよね。	
\\	的秘密計画って一体何ですか?もう少し具体的におっしゃって頂けませんか?	
\\	日本 
\\	的	
\\	周年	
\\	しゅうねん	
\\	トーフグは、来年で十周年になるので、コウイチは周年プロジェクトをけい画しています。	
\\	うちの実家は、にゅう牛の周年放ぼくをしています。今年は、放ぼく事業を始めてから、ちょうど二十周年です。	
\\	今度の日曜日に、私達はマクドナルドで
\\	の五周年記念を祝います。	
\\	周, 年	
\\	芸術	
\\	げいじゅつ	
\\	おか本太ろうは、「芸術はばく発だ」って言ってたけど、あれってどういう意味?	
\\	わたしの妹は、芸大でタイ芸術を学んでいます。	
\\	アヤは、芸術と音楽への愛をもっと多くの人に伝えるべきだ。	
\\	芸, 術	
\\	雰囲気	
\\	ふんいき	
\\	トーフグオフィスの雰囲気ってかなりくつろげるって感じだよね。	
\\	マイケルは、トーフグオフィスの雰囲気をもり上げるのが上手です。	
\\	あーっ、なんかさっきの会議、煙たい雰囲気で終わったよね。まさか最後に社長が屁をこくなんてね。まぁ、おもしろかったけどさ。	
\\	雰, 囲, 気	
\\	自動	
\\	じどう	
\\	の 
\\	ボーナスで、全自動のフロントロードしきのせんたくきを買う予定です。	
\\	スマホでとった写しんは、全て自動でクラウドにアップロードしてほぞんされるようにせっ定しています。	
\\	の方針で、許可無くコウイチのベーコンを食べた社員は、自動的に退職処分となる。	
\\	自, 動	
\\	悪例	
\\	あくれい	
\\	日本語を学ぶ時には、いい例だけでなく、悪例を学ぶことも必要です。	
\\	コウイチは悪例のかたまりだ。	
\\	僕の両親は、今後の寝る時間に悪例を残すことになるからって、絶対に夜更かしさせてくれないんだ。	
\\	悪, 例	
\\	気持ち悪い	
\\	きもちわるい	い 
\\	二日よいで、頭はガンガンするし、めっちゃ気持ち悪いし、最悪だよ。	
\\	コウイチは気持ち悪いネコなで声で、ビエトをよんだ。	
\\	ちょっと、そんな風に腕を曲げるのやめてくれない?めちゃくちゃ気持ち悪いんだけど。	
\\	気持ち 
\\	気持ち 
\\	悪い. 
\\	気, 持, 悪	
\\	骨	
\\	ほね	
\\	あの骨を的にしてねらおう!	
\\	クリステンは、転んでうでの骨を折った時もれいせいに行動していました。	
\\	昨夜、
\\	のオフィスに放火する夢を見たんだけど、焼け跡の灰には、大量の魚の骨があったよ。	
\\	(ほね) 
\\	骨	
\\	骨折	
\\	こっせつ	
\\	する 
\\	これはふくざつ骨折ですね。	
\\	コウイチは去年ひどい自動車事こにあってね。足を骨折しただけじゃなく、頭まで失くしてしまって、今はロボットになっちゃったんですよ。	
\\	しばらくみんなで一緒に会ってないよね〜。最近どうしてるの?私は足の小指を骨折した以外は特に変わりないよ〜。(笑)	
\\	こつ 
\\	こっ 
\\	骨, 折	
\\	束	
\\	たば	
\\	コウイチは札束にうもれるほど金を持っている。	
\\	記おくが正しければ、たしかブロッコリーは束で売ってたと思うよ。	
\\	どうしてもコピー用紙がいるんだけど、この花束とそのコピー用紙一束を交換してもらえないかな?	
\\	(たば) 
\\	束	
\\	人参	
\\	にんじん	
\\	この人参スープには失望しました。	
\\	酒好きの王様に、人参リキュールをプレゼントしたが、よろこばれなかった。	
\\	お願いします。何でも食べるので、人参だけは勘弁してください。	
\\	参 
\\	(じん) 
\\	人, 参	
\\	場合	
\\	ばあい	
\\	タオルや石けん、おかし、おそばなどをもっていく場合が多いです。	
\\	出来るかぎり早く
\\	メールにへんしんできるよう努めておりますが、すぐにへんしんできない場合もありますのでごりゅういください。	
\\	その場合には警察を呼んだ方がいいと思う。	
\\	場, 合	
\\	人格	
\\	じんかく	
\\	ビエトの人格を疑います!	
\\	フグってとっても温こうな人格の持ち主だね。	
\\	コウイチは人格者です。毎週日曜日に孤児院に出向いて、ベーコンを寄付しているのです。	
\\	人, 格	
\\	周り	
\\	まわり	
\\	周りがギザギザになっているコインを集めています。	
\\	大好きな事やむ中になれる事があるなら、周りが何と言おうと、それをやりつづけるべきだよ。	
\\	周りの目が気になるんです。	
\\	(まわ).
\\	周	
\\	芸人	
\\	げいにん	
\\	その芸人は、まるでねむれる森の美女のようにねむっています。	
\\	あの芸人、いつもおしい所まで行くんだけど、ゆう勝はできないんだよね。	
\\	彼は根っからの芸人かもしれない。	
\\	芸, 人	
\\	完全	
\\	かんぜん	
\\	な 
\\	の 
\\	トーフグは日々少しずつ完全のいきに近づいている。	
\\	残念ながら、コウイチには完全なアリバイがあります。	
\\	ああ、もう!お前のせいで完全に気が散っちゃったじゃないか!話したいことって何なんだよ。オフィスで飼う犬だっけか?	
\\	完, 全	
\\	完成	
\\	かんせい	
\\	する 
\\	完成前に起こしてもらえますか?	
\\	トーフグはこの数日間で、かなりのりょうを完成させた。	
\\	我々の新しい日本語学習アプリは、あと一ヶ月程で完成します。	
\\	完, 成	
\\	完了	
\\	かんりょう	
\\	する 
\\	の 
\\	作戦完了!	
\\	この町でしばらくつづいていたさい開発工事がようやく完了した。	
\\	携帯の充電が完了してから折り返してもいいかな?	
\\	完, 了	
\\	妥協	
\\	だきょう	
\\	する 
\\	妥協したくないの。	
\\	うまく妥協することができなくて、ちょっとせい神的に参っちゃったんだよね。	
\\	この仕事に、妥協は許されないよ。	
\\	妥, 協	
\\	具合	
\\	ぐあい	
\\	母の具合があまりよくないんです。	
\\	なんかめっちゃ具合わるそうだけど、大じょうぶ?	
\\	ちょっと昨日から具合が良くなくて。	
\\	具 
\\	合. 
\\	具, 合	
\\	料理	
\\	りょうり	
\\	する 
\\	メニューには魚料理がたくさんございますが、みなさんどれになさいますか?	
\\	マミは、カリフラワーを白いブロッコリーだとかんちがいして料理しました。	
\\	簡単な料理しかできません。	
\\	料, 理	
\\	協力	
\\	きょうりょく	
\\	する 
\\	の 
\\	これは皆様のご協力のたまものです。	
\\	コウイチが協力したかいもあって、ビエトはぶ事にシアトル市の市ぎ会ぎ員になることができました。	
\\	我々に協力するか、少なくとも協力するふりをしてくれませんか?	
\\	協, 力	
\\	周期	
\\	しゅうき	
\\	ふり子の周期についてのお返事、お待ちいたしております。	
\\	実はさぁ、クリステンにきいてもらいたくて、「月の周期」っていう曲を練習してるんだよね。	
\\	東北の天気は短い周期で変わる。	
\\	周, 期	
\\	深さ	
\\	ふかさ	
\\	このプールの深さを教えて下さい。	
\\	コウイチの知えの深さははかり知れない。	
\\	ビエトが、週末に百メートルの深さの落としあなをほったと自まんしていました。	
\\	深い 
\\	深	
\\	願望	
\\	がんぼう	
\\	する 
\\	の 
\\	世界が平和であってほしいと強く願望しています。	
\\	「お金をためて早期リタイアして毎日一日中ゲーム三まいというゆめを見ました。」「ゆめは願望の表れって言いますよね。それがあなたの願望なんですよ。」	
\\	結婚願望はあるんだけど、なかなか理想通りの相手が見つからなくてさ。	
\\	願, 望	
\\	失望	
\\	しつぼう	
\\	する 
\\	ホイップクリームたっぷりでって言ったのに、全然たっぷりじゃなくて失望しています。	
\\	あの先生は、期待していたよりも英語が上手じゃなくて失望しました。	
\\	マジ、自分に超失望中だよ〜。ノースリーブのシャツが着たいんだけど、ポッチャリ二の腕が気になってさ。しかも、腰まわりもたるんできたし。彼氏は私のポッチャリした指が好きだとか言うんだけど、そんなの全然嬉しくないってのさ。	
\\	失, 望	
\\	消しゴム	
\\	けしごむ, けしゴム	
\\	日本の消しゴムがこいしいです。	
\\	トーフグチームは、コウイチのたん生日に、消しゴム付きのえんぴつを百本分プレゼントしました。	
\\	「コウイチ、お前に謝るチャンスをあげてもいいよ。」「何に謝れっていうんだよ、ビエト? お前が俺をトーフグのオフィスから追放したことに対して謝れってか?」「そもそもお前が俺の消しゴムを盗んだからそうなったんだろ?いいから謝れよ!」	
\\	消す, 
\\	ゴム. ゴム 
\\	消す, 
\\	消	
\\	悪気	
\\	わるぎ	
\\	悪気がないのは知ってるけど、そういうことを言われるとこっちもきずつくんだよ?	
\\	悪気があってしたわけじゃないんですが、けっかてきにそういう風に感じさせてしまう事になって本当にごめんなさい。	
\\	ところで、そもそも、何の経験もなくて、どうやってその職を得たの?あ、悪気は無いよ。	
\\	悪気.	
\\	悪気 
\\	悪 
\\	悪い. 
\\	気 
\\	悪, 気	
\\	例文	
\\	れいぶん	
\\	例文という語いを使ったいい例文が思いつきません。	
\\	ワニカニの例文の中で一番好きな文章を教えてください。	
\\	何故だか分からないんだけど、僕の先生は例文を作る時にいつも僕の名前を使うんだよね。	
\\	例, 文	
\\	例外	
\\	れいがい	
\\	の 
\\	このレストランの料理は例外なくどれも美味しいです。	
\\	きわめて例外のケースをのぞいて、この病院がその手術に成功したのをほとんど見たことがありません。	
\\	コウイチはいつも朝遅くまで寝ていますが、金曜日はスーツを着なくてはいけないので例外です。	
\\	例, 外	
\\	女性	
\\	じょせい	
\\	の 
\\	近年、日本でも女性のせいじ家の数が少しずつふえています。	
\\	この女性はりょう手でしっかりオニギリをにぎれるので、トーフグはかの女をやとうべきだと思うんだ。	
\\	私は女の子なんかじゃないよ。私はれっきとした女性だよ。	
\\	女, 性	
\\	美術	
\\	びじゅつ	
\\	の 
\\	大学で美術のクラスをとっています。	
\\	き業のリーダーには美術のセンスも必要とされるが、コウイチにはそんなセンスは全くない。	
\\	アヤは、人間の剥製という美しい美術作品が並べられたアジトを
\\	の恐ろしい美術館」と名付けた。そして、コウイチとビエトもその剥製の一つにされようとしていた。	
\\	美, 術	
\\	例えば	
\\	たとえば	
\\	例えば、昔はすごく人気があったけど、今は落ちぶれてしまった人がいたとしましょう。	
\\	例えば、舌をかみそうになった時、コウイチなら何て言うと思う?	
\\	なぁ、さくらちゃん、例えば、週に一度や二度ぐらい、うちに来ないか?	
\\	""例えば, 
\\	例えば 
\\	(たと) 
\\	例	
\\	基本	
\\	きほん	
\\	の 
\\	この公しきが数学の基本だというダリン君の主ちょうは、基本的に正しいと思います。	
\\	コウイチは、ビエトからヤクザの対外けいざいせいさくの基本について学んでいます。	
\\	私は日本語の基本中の基本も知らない超初心者です。	
\\	基本 
\\	基, 本	
\\	試合	
\\	しあい	
\\	する 
\\	良かったらもう一試合しませんか?	
\\	けん道の試合の結果はどうでしたか?	
\\	「あの野球の試合は本当にひどかったね。」「本当にそうだね。」	
\\	試合 
\\	し 
\\	試 
\\	あい 
\\	合. 
\\	試, 合	
\\	芸者	
\\	げいしゃ	
\\	あの芸者とわたしの間には、何一つ共通点がありません。	
\\	日本に旅行に行ったらぜひ芸者のかっ好をして京都の町を練り歩いてみたいです。	
\\	コウイチ、昨日は私達が芸者遊びをしている間、フグと私の子どもを見ててくれてありがとう。再度お礼が言いたくて。	
\\	芸, 者	
\\	性	
\\	せい	
\\	性はんざいをゆるしてはいけません。	
\\	性による違いをはあくしたうえで、男女平どうを考えることが大事だと思います。	
\\	性のところの、男性、女性、その他のいずれかにチェックをしてください。	
\\	性	
\\	良好	
\\	りょうこう	
\\	な 
\\	トーフグと日本せいふは、良好なかんけいをたもっています。	
\\	コウイチ先生は、ムッツリスケベであることは生物学的に良好なことだっておっしゃってましたよ。	
\\	その患者さんの経過は過去六ヶ月間良好で、普段通りの生活を送ることができています。	
\\	良, 好	
\\	卒業	
\\	そつぎょう	
\\	する 
\\	の 
\\	日本では三月が卒業のシーズンです。	
\\	ビエトは
\\	48というアイドルグループを作ってずっとセンターをつとめているが、いつまでたっても卒業する気配がない。もう58才なのに。	
\\	先生は、日本語が話せると、卒業生の就職の可能性が高くなると思いますか?	
\\	卒, 業	
\\	固い	
\\	かたい	い 
\\	トーフグオフィスの入ったビルは、固い地ばんにたてられています。	
\\	その固い牛肉をかんだ時、なぜか、自分がとてもちっぽけなそんざいであることを思い知って、なみだがこみ上がってきました。	
\\	あなたは口が固いから教えてあげるけど、私、アヤがキンニクマの洗濯板のような腹筋と岩のように硬い胸筋を触っているのを見ちゃったんだよね。	
\\	い 
\\	(かた). 
\\	固	
\\	念願	
\\	ねんがん	
\\	する 
\\	の 
\\	コウイチは、世界の平和を念願して、出家しました。	
\\	ついに長年の念願だったアフリカ旅行を実げんしました。	
\\	やったー!ついに念願の彼氏ができたよー!やっぱ一生懸命願えば叶うもんだね!超幸せー!	
\\	念, 願	
\\	希望	
\\	きぼう	
\\	する 
\\	回転し金がそこをついてしまったが、まだ希望はあるとしんじているんだ。	
\\	この方は、とく名を希望しているようですね。	
\\	「お座席のご希望はございますか?」「はい。窓側の席を取っていただけますか?」「かしこまりました。」	
\\	希, 望	
\\	人性	
\\	じんせい	
\\	私は、大学で人性学を勉強しています。	
\\	私は、キリストは神性と人性という二つの本性を持つという両性せつを支持します。	
\\	人性論における人間の本性において、私は結局人の性質は善悪を共に備えていると思います。	
\\	人, 性	
\\	消化不良	
\\	しょうかふりょう	
\\	の 
\\	消化不良のくすりをさがしています。	
\\	コウイチくんにこく白したんだけど、返事が全然はっきりしなかったから、消化不良ですごくモヤモヤしてるんだよね。	
\\	今、消化不良で苦しんでるの。胸焼けもするし、膨満感もあるし、すごく気持ち悪い。しばらく放っておいてくれない。	
\\	消化 
\\	消化 
\\	不良 
\\	消, 化, 不, 良	
\\	材料	
\\	ざいりょう	
\\	かれが材料調たつのたん当者です。	
\\	材料の二倍の水を入れてください。	
\\	私の好きな食べ物は饂飩ですが、麺の材料は、小麦粉と塩と水だけなんですよ。	
\\	材, 料	
\\	一例	
\\	いちれい	
\\	これはほんの一例にすぎません。	
\\	一例としては、ごはんやじゃがいもを食べるりょうをへらしたからといって、やせられるわけではないということがあげられます。	
\\	秘書としてサーモンを迎え入れた後で、彼女が職場で突然感情を爆発させる性格であることが判明したのです。例えば、自分のデスクの電話を引き抜いて、粉々にブチ壊しちゃったりとか、まあ、ほんの一例を挙げればそんな感じです。	
\\	一, 例	
\\	本能	
\\	ほんのう	
\\	の 
\\	動物は本能的に火をおそれるが、コウイチは例外だった。	
\\	いつも、生まれもった本能のおももくままにできるだけしたがうようにしています。	
\\	熊は、動物の本能で危険を察知した。	
\\	本, 能	
\\	意図的	
\\	いとてき	な 
\\	コウイチは、トーフグ社員の基本給を、意図的に引き上げた。	
\\	でもそれちょっと、意図的な作り話っぽくない?	
\\	意図的に情報を操作している奴がいる。	
\\	図 
\\	(と) 
\\	意, 図, 的	
\\	着物	
\\	きもの	
\\	カオリさんは、着物のどこがすきですか?	
\\	お着物が汚れていますよ。	
\\	母親に、既婚女性か未婚女性かで着物の種類が異なると教えてもらいました。	
\\	温泉 
\\	着, 物	
\\	材木	
\\	ざいもく	
\\	なんで材木にクイックルワイパーをかけてるの?	
\\	あの材木おき場に夜の八時に行くと、神様に会えるらしいよ。	
\\	材木が来たらすぐに犬小屋作りに取り掛からないとな。	
\\	材, 木	
\\	男性	
\\	だんせい	
\\	の 
\\	あの男性の名前を知っていますか。	
\\	このオフィスには男性の会社員が約百名、女性の会社員が約七十名います。	
\\	この会社で働きたかったら、そこにいる男性に腕相撲で勝利してみせなさい。	
\\	男, 性	
\\	意地悪	
\\	いじわる	
\\	な 
\\	折角の時間をあんな意地悪なやつのためにむだにするなんてもったいないよ。	
\\	このようせいは、かわいいけど意地悪です。	
\\	どうして妹に意地悪ばっかりするの!	
\\	意 
\\	地, 
\\	悪. 
\\	意, 地, 悪	
\\	近代的	
\\	きんだいてき	な 
\\	近代的なデザインが気に入りました。	
\\	とても近代的な建物ですね。	
\\	最近のプリクラはちょっと近代的になりすぎていると思う。	
\\	近, 代, 的	
\\	山登り	
\\	やまのぼり	
\\	する 
\\	山登りがしゅ味なんですか?	
\\	ものしずかな友人と、山登りに行ってきました。	
\\	お互い色々あったんだけど、山登りをしながらじっくり話し合ってさ。なんと復縁することになったよ。	
\\	登る 
\\	山 
\\	登る 
\\	山登り. 
\\	山, 登	
\\	動き	
\\	うごき	
\\	ヤクザのまやくがらみの動きが知りたいなら、ビエトと話をしに行ったらいいよ。	
\\	コウイチとビエトは、今週は市場の動きのかんさつ方ほうを学んでいます。	
\\	あの人、何だか動きが不自然ですね。	
\\	動く. 
\\	動	
\\	一周	
\\	いっしゅう	
\\	する 
\\	コウイチは、最後の一周で十人のランナーをゴボウぬきした。	
\\	コウイチはグラウンド一周を三分台で走ります。	
\\	トーフグチームは世界一周旅行に出かけるので、オフィスは約一年間不在となります。	
\\	一周?	
\\	いち 
\\	いっ 
\\	(一階, 一回, 
\\	一, 周	
\\	残業	
\\	ざんぎょう	
\\	する 
\\	ちゃんときゅう料が支はらわれるなら、残業してもいいですよ。	
\\	残業中に、材木がおちてきて、九死に一生をえることになりました。	
\\	残業後にイタメシ屋でお一人様しちゃったよ〜。なかなか寂しかったぜぃ(笑)	
\\	残, 業	
\\	周囲	
\\	しゅうい	
\\	の 
\\	あの女性の周囲にはいつもヤクザ風の男がいる。	
\\	コウイチは周囲の目をけいかいしながら黒いたて物に入っていきました。	
\\	富士山の火口の周囲は約三キロメートルで、深さは二百三十七メートルだと聞きました。	
\\	周, 囲	
\\	性格	
\\	せいかく	
\\	コウイチは性格はおっとりとしているが、頭の回転が速いだけに、よくしたが回る。	
\\	かれ、性格はそんなに悪くないんだけど、一しょにいても会話が広がらないんだよね。	
\\	私の旦那さんの性格はのんびり屋で、忘れっぽくて、優柔不断です。彼のことは大好きだけど、たまにイラッとくるんだよね。	
\\	性, 格	
\\	松	
\\	まつ	
\\	その松買うのに、二十分間も待ったんだよ。	
\\	松の木から松ぼっくりが落ちてきてわたしの頭に当たった。	
\\	わたしの小学校では、クラスを一組、二組、とは分けずに、松組、竹組で分けていました。	
\\	松	
\\	松葉	
\\	まつば	
\\	かんそうした松葉が一本私の部屋のゆかに落ちていました。	
\\	その松葉は、クルクルと十回転した後、水たまりへ落ちました。	
\\	今度の昇給が得られれば、ようやく両親に松葉ガニを買ってやることができる。	
\\	葉 
\\	ば 
\\	松, 葉	
\\	流行語	
\\	りゅうこうご	
\\	どの流行語が、今年の流行語大賞にえらばれると思う?	
\\	毎年ほんとに色んな流行語が生まれるよね。	
\\	ここ数年の間に、「ミーム」はインターネットでの流行語となりました。	
\\	流, 行, 語	
\\	流行歌	
\\	りゅうこうか	
\\	今時は「流行りの歌」って言うよね。「流行歌」っていうとなんかしょう和の歌よう曲って感じがする。	
\\	ツイッターでフォローして、最新の流行歌情ほうをチェックしてね。	
\\	私は、流行歌ではないけど良い曲を見つけるのが好きです。	
\\	流行 
\\	流, 行, 歌	
\\	約束	
\\	やくそく	
\\	する 
\\	土よう日は、コウイチとの約束があるからむりなのよね。日よう日はどう?	
\\	あの男には、心中の約束をうら切られたの。	
\\	日程の件で色々と振り回しちゃって本当にごめんね。四日なら都合がつきそうです。次は今回みたいなドタキャンは絶対しないって約束します!	
\\	約, 束	
\\	目的	
\\	もくてき	
\\	一体、目的は何なんですか?	
\\	着付けが出来るようになるのが目的で、この料ていで働き始めました。	
\\	私たちの目的は、一緒に楽しみながら皆さんに日本語を勉強してもらうことです。	
\\	目 
\\	目標 
\\	目次, 
\\	目的, 
\\	もく 
\\	目, 的	
\\	理性	
\\	りせい	
\\	お酒を飲みすぎて、完全に理性を失ってしまっていました。	
\\	ビエトは、今時は、ヤクザもぼう力にたよるんじゃなく、理性にうったえなくっちゃダメだって言ってたよ。	
\\	ダメだと分かっていたが、理性を保てなかったんだ。	
\\	理, 性	
\\	残り	
\\	のこり	
\\	の 
\\	残りの水はどこに行ったの?	
\\	さく夜のコウイチのツイートは、一にぎりの残りのフォロワーさえもおこらせてしまいました。	
\\	今夜はデートがあるので、晩御飯は昼御飯の残りを温めて食べてね。ママより。	
\\	り 
\\	(のこ), 
\\	残	
\\	予約	
\\	よやく	
\\	する 
\\	の 
\\	パーマとカットの予約をしたいんですけど。	
\\	2020年に日本へ行くなら早めに予約した方がいいですよ。その年は東京オリンピックがありますから。	
\\	ただ今、コウイチのサイン入りブロマイドの先行予約を受け付けております。予約された方の中から抽選で三名の方にコウイチのファンミーティング
\\	ハワイの
\\	チケットをプレゼントいたします。皆様どしどしご応募ください!	
\\	予, 約	
\\	回転ずし	
\\	かいてんずし	
\\	新しく回転ずしのお店がオープンした。	
\\	その回転ずしは、毎秒三十二回転していました。	
\\	ねぇ、まだ準備出来てないの?お腹ペコペコなんだけど。三分以内に準備出来なかったら、先に回転寿司に行っちゃうからね!	
\\	回転 
\\	すし 
\\	ずし). 
\\	回転, 
\\	回, 転	
\\	文化祭	
\\	ぶんかさい	
\\	うちの学校の文化祭にもあそびに来てくださいね。	
\\	「文化祭の練習、何時にする?」「おれは一日中ひまだから、そっちの都合の良い時なら何時でもいいよ。」	
\\	文化祭の管理諸経費は全体の約三十五パーセントになると見積っています。	
\\	文化 
\\	文, 化, 祭	
\\	性病	
\\	せいびょう	
\\	の 
\\	こまってるんだったら、性病ちりょうのお金、少しだったら都合してあげられるよ。	
\\	全く都合の悪い時に性病にかかったもんだよ。	
\\	性病を移してくれてどうも有難う。とても嬉しいです。	
\\	性, 病	
\\	格好	
\\	かっこう	
\\	の 
\\	ずい文変わった格好だね。	
\\	格好のいい車ですね。	
\\	「何でミカサがエレンのことを好きなのか私には分からないわ。だって、エレンって傲慢だし、思いやりがないし、それに格好良くもないのに。」「恋は盲目っていうじゃない。」	
\\	かく 
\\	かっ.	格, 好	
\\	芸能界	
\\	げいのうかい	
\\	折角東京まで行ったのに、芸能界には入れなかった。	
\\	わたしのむすめは思いつきで物事を決めがちなんですが、今度は芸能界で働きたいって言い出してるんですよね。	
\\	芸能界のゴシップなんて全然興味ねーよ。どうでもいいっつーの。	
\\	芸, 能, 界	
\\	合図	
\\	あいず	
\\	する 
\\	わたしが合図をしたら、大声で「エイドリアーン!」とさけんでください。	
\\	わたしに、売りか買いかの合図の出し方をおしえてくれませんか?	
\\	彼女は彼に、秘密の合図を送った。	
\\	合 
\\	図. 
\\	合, 図	
\\	骨格	
\\	こっかく	
\\	の 
\\	ガッシリした骨格の男性がタイプです。	
\\	昨日は、この記事の骨格をまとめ終えました。	
\\	コーラが骨格の発育に悪影響を与えるという噂は本当ですか?	
\\	こつ 
\\	こっ.	骨, 格	
\\	技能	
\\	ぎのう	
\\	カナエは、先月末にウェブデザイン技能検定に合格した。	
\\	リーディング、ライティング、リスニング、スピーキングの四つの技能をバランスよく上たつさせるのはむずかしい。	
\\	逃走するために、俺は
\\	諜報員としての年月で習得した技能を駆使した。	
\\	技, 能	
\\	特技	
\\	とくぎ	
\\	わたしの特技は、お湯をわかすことです。それ以外の特技は何一つありません。	
\\	私の六人ぐらい後ろにならんでる青い服の男の子、高校の同級生なんだけど、なんとオナラでセーラームーンのテーマをかなでるという特技を持ってるのよ。	
\\	コウイチの特技はお菓子作りです。お菓子作りが好きな理由は、みんなの嬉しそうな顔を見るのが好きだからだそうです。	
\\	特技?	
\\	特, 技	
\\	言葉つき	
\\	ことばつき	
\\	そのようせいの声は子どもの声だったが、言葉つきはまるで大人だった。	
\\	あの社員は言葉つきがハキハキしていて好感がもてますね。	
\\	しばらくぶりに会った友達の言葉つきが随分変わっていてびっくりした。	
\\	言葉 
\\	つき 
\\	(つき) 
\\	言, 葉	
\\	運動会	
\\	うんどうかい	
\\	悪天こうで、折角楽しみにしていた運動会が中止になってしまいました。	
\\	運動会の日、晴れるといいですね。	
\\	運動会の日の朝に、今日は都合が悪いから運動会を明日に変えることはできないかって聞いてきた親がいて、びっくりしたよ。	
\\	運動 
\\	運, 動, 会	
\\	待ちぼうけ	
\\	まちぼうけ	
\\	返事を待っていましたが、待ちぼうけを食らいました。	
\\	おかげで、さびれたえきで、一時間も待ちぼうけをする羽目になったよ。	
\\	彼が待ちぼうけを食らったと思わないように、明日の会議が中止になったことを伝えてあげてくれるかな?	
\\	待ち 
\\	ぼうけ? 
\\	ボケ 
\\	待ち 
\\	待つ. 
\\	待	
\\	頑固	
\\	がんこ	
\\	な 
\\	子どもが頑固でこまっています。	
\\	頑固なクリステンは、コウイチとビエトは死ぬまでずっと仲良しだと思うと言いはった。	
\\	フグは根っからの頑固者で、自分のやり方にこだわります。	
\\	頑, 固	
\\	水着	
\\	みずぎ	
\\	水着を持ってくるの、わすれちゃった。	
\\	なんでコウイチは女せい用水着を着てるの?	
\\	水着はありますか?とびきりセクシーなのを探しているんですが。	
\\	着 
\\	き 
\\	ぎ 
\\	水, 着	
\\	才能	
\\	さいのう	
\\	この映画に出えんしているはいゆうたちは、みんな才能があります。	
\\	外国語を勉強するのに才能がいると思いますか。	
\\	君はゴミをゴミ箱に投げ入れる才能があるね。	
\\	才, 能	
\\	技	
\\	わざ	
\\	これはまたきわどい技をマスターしたね!	
\\	コウイチはニッカに新しい技を仕こんでいるんですか?	
\\	柔道では力より技のほうが大切であります。	
\\	(わざ) 
\\	技	
\\	四季	
\\	しき	
\\	ビエトのとった写しんは、四季の美しさをとてもよくびょう写しています。	
\\	わたしの弟は、昔は四季を通じて下たをはいていました。	
\\	日本の四季についてと、それぞれの季節にどんなことをしたりどんなものを食べるのかについて、教えてもらえませんか?	
\\	四 
\\	し 
\\	死), 
\\	四, 季	
\\	間に合う	
\\	まにあう	
\\	今すぐタクシーに乗れば、まだ間に合うかもしれません。	
\\	ごめん!じゅうたいにはまっちゃって、待ち合わせに間に合いそうにない。	
\\	「間に合うと思う?」「もうすぐ7時なんだから、後は自分で考えればわかるだろ。」	
\\	(合う). 
\\	間 
\\	間もなく. 
\\	合う 
\\	合う. 
\\	間, 合	
\\	落ち着く	
\\	おちつく	
\\	かの女はとらえどころのない女性ですが、一しょにいると落ち着くんです。	
\\	ビエトは赤ちゃんにスコッチをあげて落ち着かせた。	
\\	「ちくしょう、上司のやつ!もう、頭にきた!こんな仕事、辞めてやる。」「おいおいおい、そう焦るなって。ちょっと落ち着けよ。 
\\	落ちる 
\\	着く. 
\\	落ち着く 
\\	落, 着	
\\	参る	
\\	まいる	
\\	いざ、参らん!	
\\	今週末は、母のおはかに参るつもりです。	
\\	私は毎日神社に参ります。	
\\	(まい) 
\\	参	
\\	消化する	
\\	しょうかする	する 
\\	このネコは、魚の皮をうまく消化できずにはいてしまったみたいです。	
\\	トーフグは、全社員が有きゅう休かを完全消化することをすいしょうしています。	
\\	今、フグの刺し身を百切れ食べたよ。消化するのにちょっと時間がかかりそう。(ゲップ)	
\\	消化 
\\	消化 
\\	消, 化	
\\	消える	
\\	きえる	
\\	ワニカニにしゅう入の大半が消えます。	
\\	あせでまゆ毛が消えてしまいました。	
\\	彼は、彼女が視界から完全に消えるまで、さようならと手を振り続けた。	
\\	消す 
\\	消す? 
\\	える 
\\	(える). 
\\	消す, 
\\	(き) 
\\	消	
\\	期待する	
\\	きたいする	する 
\\	病気がちな体しつなので、あまり期待しないでください。	
\\	つまは、けっこん記念日にはおっとがみかんを買ってかえってくるものだと期待していた。	
\\	今年は母親以外の女子からのバレンタインのチョコレートを期待しています。	
\\	期待 
\\	期待, 
\\	期, 待	
\\	折る	
\\	おる	
\\	ぎょうざの皮を半分に折って折り目を作って下さい。	
\\	きっ茶店で折り紙を折っていたら、一人の女の子が近よってきた。	
\\	その紙を折った時に、指を切ってしまったんだ。	
\\	う 
\\	(お) 
\\	折	
\\	合わせる	
\\	あわせる	
\\	全ての赤と白のワイヤーを一たばに合わせました。	
\\	男が女のペースに合わせてゆっくり歩くのは、その女のことが好きな時だけさ。	
\\	おい、スティーヴ!お前はもうずいぶん長く日本に住んでいるんだから、そろそろ日本の習慣とかに合わせたほうがいいんじゃないのか?好き嫌いは止めて、魚も食べるようにしろよ!	
\\	合 
\\	合う, 
\\	合	
\\	待たせる	
\\	またせる	
\\	先ぱいを待たせるのはよくないよ。	
\\	一体何時まで待たせるつもりですか?	
\\	お待たせしてすみません。いや〜バスにえらく長いこと待たされましてね。	
\\	う 
\\	たせ 
\\	待つ, 
\\	待	
\\	動かす	
\\	うごかす	
\\	友だちのほとんどが、トラックを手でおして動かすことができるよ。	
\\	コウイチったら、急にむくっと起き上がって、ねぼけてあのきょ大なグランドピアノを動かそうとしだしたんだよ。	
\\	あのバイク事故のせいで、もう脚を動かすことができなくなっちゃったんだ。	
\\	動く 
\\	(かす) 
\\	動く. 
\\	動	
\\	残念	
\\	ざんねん	
\\	な 
\\	それは残念すぎる。	
\\	折角のチャンスだったのに、残念です。	
\\	「このテレビ番組のために私が蜘蛛を食べなくちゃいけないって、本当?」「残念ながらそのようです。」	
\\	残, 念	
\\	合格する	
\\	ごうかくする	する 
\\	折角苦労して合格したのに、何でそんなすぐに大学やめちゃったの?	
\\	今朝の時点では、まだ合格したのかどうか分からなかったんです。	
\\	表向きは入学試験に合格するために学習塾に通ってたんだけど、私は実は友達と一緒にいたかっただけなんです。	
\\	合, 格	
\\	着ける	
\\	つける	
\\	久しぶりにイヤリングを着けてみた。	
\\	コウイチのしゅ味は、夜な夜な自分の部屋にこもって世界かく地のお面を着けてあそぶことです。	
\\	このお守りを肌身離さず身に着けると約束してくれないか?	
\\	着る 
\\	着く 
\\	着ける 
\\	着る, 
\\	付ける, 
\\	着く
\\	付ける, 
\\	着	
\\	残る	
\\	のこる	
\\	自分がミステリアスだと思う人は、ここに残ってください。	
\\	私たちはポートランドに向けて出発したが、コウイチはまだ日本に残っていました。	
\\	新しいテキストフグは、今までにない最高の日本語学習教材として、歴史に残るでしょう。	
\\	う 
\\	(のこ) 
\\	残	
\\	回転する	
\\	かいてんする	する 
\\	このコウイチ人形は、このビエト人形の周りを、時けい仕かけで高速で回転します。	
\\	マイケルは、ふれずにそのボールを回転させることができます。	
\\	この回転椅子はこんな風に回転するんだ。	
\\	回転 
\\	回転 
\\	回, 転	
\\	流れる	
\\	ながれる	
\\	自転車のレースの後、ビエトの顔をあせがたきのように流れていました。	
\\	この川は、どの地いきを流れているんですか?	
\\	流れる水の音が好きなので、涙が君の頬を流れるままにしておいてくれよ。ほら、とても美しいよ。	
\\	う 
\\	流す 
\\	流す 
\\	(す) 
\\	流れる 
\\	(れる) 
\\	流す. 
\\	流	
\\	勉強する	
\\	べんきょうする	する 
\\	毎日ワニカニにログインしてかん字を勉強することが大事です。	
\\	勉強することは好きなんですが、本を読むことはきらいなんです。	
\\	ちょっと忙し過ぎて、アインシュタインの相対性理論について勉強する時間がないだけなんです。	
\\	勉強 
\\	勉強. 
\\	勉, 強	
\\	囲む	
\\	かこむ	
\\	正かいの番号を全て丸で囲みなさい。	
\\	時々、もう一度記者に囲まれたいなと思うことがあります。	
\\	日本は海に囲まれているんだから、塩はいつでも手に入ったんじゃないんですか?	
\\	う 
\\	(かこ) 
\\	囲	
\\	旅行する	
\\	りょこうする	する 
\\	アヤはかん国を旅行するだけでなく、日本にも旅行する予定です。	
\\	旅行する前に美よう院に行ったんだけど、思ってたより短く切られちゃったんだよね。	
\\	マジ、終わってる。一緒に旅行をした後、彼が一週間も電話を折り返してこないの。きっと、私寝っ屁をこきまくってたんだわ。	
\\	旅行 
\\	旅行, 
\\	旅, 行	
\\	起こす	
\\	おこす	
\\	折角起こしてあげたのに、意味なかったね。	
\\	赤ちゃんを起こしたくないので、インターホンは鳴らさないでください。	
\\	ビエットが100
\\	正しいよ。今すぐコウイチ起こすべきだよ。	
\\	起きる 
\\	起きる? 
\\	す 
\\	(す) 
\\	起きる, 
\\	起	
\\	温める	
\\	あたためる	
\\	その器は電子レンジで温めないでください。	
\\	夜ねる前に牛にゅうを温めてホットミルクにして飲むのが好きです。	
\\	水に入れて温めると、昆布のうまみ成分がしみ出てきます。	
\\	温かい 
\\	温かい. 
\\	温	
\\	望む	
\\	のぞむ	
\\	コウイチは自分の足のつめをなめたいと強く望んでいた。	
\\	今は
\\	で何でも望むものが見られていいよね。	
\\	髪切ったんだけど、逆に童顔に磨きがかかっちゃってさあ。年相応のお洒落な三十五歳の女性になることを望んでたのに。。。とほほ。	
\\	(のぞ). 
\\	望	
\\	旅行者	
\\	りょこうしゃ	
\\	最初、ぶあい想な旅行者だなぁと思ったんだけど、話をしてみると意外におも白い人だったのよね。	
\\	ふだんはこういった対おうはいたしませんが、旅行者でいらっしゃるということですので、今回は特別に対おうさせていただきます。	
\\	ねぇ、見て見て。あの旅行者の服装、全然年相応じゃないと思わない?	
\\	旅行 
\\	旅行 
\\	旅, 行, 者	
\\	下着	
\\	したぎ	
\\	今日はコウイチは勝負下着をはいている。	
\\	全ての下着の注文ひんを、き日通りに発送しないといけません。	
\\	どの下着が欲しい?植物柄?それともアニマル柄?好きなものを選んでいいよ。	
\\	着 
\\	着る, 
\\	き 
\\	ぎ 
\\	下, 着	
\\	協会	
\\	きょうかい	
\\	息子が折角コツコツためたお金は、協会ひの支はらいに消えてしまいました。	
\\	明日は協会の集まりがあるので、六時に起こしてください。	
\\	こんにちは。僕はコウイチといいます。オレゴン州ポートランド市のポートランド剣道協会で、イベントマネージャーをしています。	
\\	協, 会	
\\	悪口	
\\	わるくち, わるぐち, あっこう	
\\	ビエトは意地悪なえみをうかべながらコウイチに大声で悪口を言い始めた。	
\\	子どもの前では、おっとの悪口を言ったことがありません。	
\\	友達がみんなで私の悪口を言っているのを聞いちゃったんだよね。	
\\	悪い 
\\	口 
\\	わるぐち 
\\	悪, 口	
\\	日本的	
\\	にほんてき	な 
\\	日本的なくらしにあこがれています。	
\\	このアニメはとても日本的ですね。	
\\	フグは、サーモンの中に「日本的な優しさ」を見出した。	
\\	日本的.	
\\	日本 
\\	日, 本, 的	
\\	待合	
\\	まちあい	
\\	する 
\\	合図をしたら、待合でスクワットを始めてください。	
\\	待合する場所がさむすぎて、ぼうこうえんになりました。	
\\	病院の待合でばったり久しぶりに友人に会って、昔話に花がさきました。	
\\	待ち 
\\	待, 合い 
\\	合 
\\	待, 合	
\\	望み	
\\	のぞみ	
\\	あなたの望みは何ですか。	
\\	時間をかけて努力すればするほど、望みをかなえることに近づくのよって母に教えられました。	
\\	サーモンの望み通り、フグは彼女にプロポーズをした。	
\\	(のぞ). 
\\	のぞみ.
\\	望	
\\	折角	
\\	せっかく	
\\	折角ですが、これ以上妥協することは出来かねます。	
\\	おれ様の折角のちゅうこくをむだにするつもりなのか?	
\\	うーん。折角イメチェンするいい機会だし、後ろ髪を三十センチくらい切ってもらおうかな。	
\\	せつ 
\\	せっ, 
\\	折, 角	
\\	木材	
\\	もくざい	
\\	ひまだったら、木材をはこぶのを手伝ってくれない?	
\\	「木材」は、まだほとんど加工されていない原料としての木のことを指し、「材木」は、その木材を加工してすぐ使える状態に仕上げたものを指します。	
\\	二日なら時間を作れるよ。いつもの木材置き場で話さない?	
\\	木, 材	
\\	紀元後	
\\	きげんご	
\\	この原始人は、紀元後にようやくスランプをだっ出するまで、あせらずじっとたえていました。	
\\	力ずくでネジをしめるとななめになりやすいってこと、紀元後になるまで知らなかったんだよね。	
\\	紀元前五十年から紀元後二十年の間に作られたとされるこの地域の土器に、同じようなフグの模様が見られるんです。	
\\	紀, 元, 後	
\\	飲み放題	
\\	のみほうだい	
\\	の 
\\	食べ放題と飲み放題だったらどっちの方がおとくかな?	
\\	「カナエ、ちょっと飲み会をき画してくれない?」「おっけー!どんなテーマの飲み会?」「うーんと、ビエトが日本で買ってきたコスプレコレクションのおひろ目会なんだけど。」「分かった。めっちゃ安い飲み放題のい酒屋でもいい?」	
\\	「じゃあ、あなたは昨夜飲み放題に行っていた訳ですね。他には何かしましたか?」「いや、まあそんなところだよ。」	
\\	放題 (ほうだい) 
\\	飲む (のむ). 
\\	飲, 放, 題	
\\	食べ放題	
\\	たべほうだい	
\\	の 
\\	一しょに食べ放題に行ったのが、とおい昔のような気がするよ。	
\\	日本で、ケーキ食べ放題のカフェに行ってみたんですが、思っていたよりもたくさん食べられました。	
\\	よし、今がチャンスだ。さあ、焼き肉食べ放題へ行こう。	
\\	食べる 
\\	放題. 
\\	食, 放, 題	
\\	変人	
\\	へんじん	
\\	あの変人のクローゼット、見た?	
\\	変人的なふるまいをしつづける世界記ろくにちょう戦したが、おしくも世界記ろくにはあと十分足りなかった。	
\\	あなたの上司はちょっと変人ですが、すぐに馴染めると思いますよ。	
\\	変, 人	
\\	建築	
\\	けんちく	
\\	する 
\\	あの会社が建築したビルは、たいしん基じゅんをみたしていません。	
\\	この建築ひ用の内わけを教えてもらえますか?	
\\	多くの人が京都や奈良に残る日本の伝統建築に魅了されますが、私もその一人です。	
\\	建, 築	
\\	お願い	
\\	おねがい	
\\	する 
\\	お願い!手伝って!	
\\	「ちょっとお願いがあるんだけど……。」「もちろん。言ってみて。」	
\\	中小企業は小回りが利くから、こういう仕事をお願いしても平気だろう。	
\\	い 
\\	お願いします. 
\\	願	
\\	軍	
\\	ぐん	
\\	トーフグ軍のしんの指きかんはコウイチじゃなくてビエトだ。	
\\	どうやっててき軍の追げきをかわせばいいんだ。	
\\	とりあえず軍の拠点に戻ってみましょう。	
\\	軍	
\\	秋	
\\	あき	
\\	秋はわたしの一番好きなきせつです。	
\\	秋と言えば何を思いうかべますか。わたしはやっぱり食よくの秋ですね。	
\\	はぁ?まだこの秋に一回ご飯食べに行っただけなの?あんた頭イッちゃってるんじゃないの?どう考えても彼氏って呼ぶのはおかしいでしょ。それは付き合ってるとは言えないと思うよ。	
\\	あき.	秋	
\\	上司	
\\	じょうし	
\\	その上司のコートは、とても丈夫な生地で出来ていました。	
\\	うちの上司は、日本のれきしに大変きょう味があるようです。	
\\	シャークがあなたのこと、名ばかりの上司だと言ってましたよ。	
\\	上, 司	
\\	岩	
\\	いわ	
\\	あの岩のところまでどっちが先に着けるか競争しよう!	
\\	トーフグにはコウイチが岩みたいなかおをしている時は、話しかけないでそっとしておくという決まりがあります。	
\\	コウイチは、夢の中で大きな岩が頭上に落ちてくるところで、目が覚めた。	
\\	岩	
\\	英国	
\\	えいこく	
\\	あの頑固な英国首相をせっとくするのはとても大変だったでしょう?	
\\	毎年、大変な数の人たちが、ゴルフをしに英国をおとずれます。	
\\	英国と米国は敵対関係にある。	
\\	イギリス 
\\	英, 国	
\\	仏	
\\	ほとけ	
\\	ビエトはヤクザをやめて、仏の道に入ることを決意しました。	
\\	コウイチは、いつも仏のようにれいせいで、えんま大王のようにきびしい。	
\\	「知らぬが仏」や「仏の顔も三度まで」など、仏に関する諺はいくつかありますね。	
\\	(ほとけ). 
\\	仏	
\\	仏教	
\\	ぶっきょう	
\\	仏教とヒンドゥー教のちがいを正かくにせつ明できますか?	
\\	そんなに信心深い人間じゃないんですが、一おう神道と仏教を信こうしています。	
\\	仏教に伴って、お香は六世紀に日本に伝来しました。	
\\	仏 
\\	ぶっ 
\\	仏, 教	
\\	計算	
\\	けいさん	
\\	する 
\\	ビエトは今、飲み放題のパーティーにかかるひ用を計算しています。	
\\	ああ見えて、コウイチは計算がものすごく速いんです。	
\\	お母さん!こんなに沢山の計算をしたら、数学の天才になるかわりに逆に頭がおかしくなっちゃうよ。	
\\	計, 算	
\\	猫	
\\	ねこ	
\\	うちの猫は、暗算がとく意なんです。	
\\	このクローゼットにあるりょうの四倍の猫の服が家にあります。	
\\	助けてくれ!俺の飼い猫がスクラブルの 
\\	のコマを飲み込んじゃったんだ!	
\\	猫	
\\	世紀	
\\	せいき	
\\	「今って何世紀だっけ?」「二十一世紀だよ。」	
\\	コウイチは少なくとも一世紀は生きたいと思っている。	
\\	あいにく、その日のスケジュールは少しきついですね。今世紀最大の日本語学習祭り関連の会議がいくつか入っているんですよ。	
\\	世, 紀	
\\	式	
\\	しき	
\\	いったい式場はどこなんだ?	
\\	開会式は九時にスタートする予定です。	
\\	その儀式は退屈だったので、たくさん方程式を解いて時間を潰しました。	
\\	しき, 
\\	式	
\\	変化	
\\	へんか	
\\	する 
\\	コウイチは、天こうの変化にとてもびん感で、雨の日はよくへん頭痛をわずらっています。	
\\	十年前におとずれた時とくらべて、東京はずいぶんと変化しましたね。	
\\	うわ!すごいね。既に変化が現れているのが分かるよ。	
\\	変, 化	
\\	晴れ	
\\	はれ	
\\	の 
\\	天気予ほうでは、明日も明後日も明々後日も晴れでしょうって言ってたよ。	
\\	予ほうは「晴れ後くもり」だし、かさはいらないんじゃない?	
\\	折角の誕生日だし、晴れて良かったですね。	
\\	晴	
\\	不器用	
\\	ぶきよう, ぶきっちょ	
\\	な 
\\	十代は、不器用な私にとって、とても大変な時期でした。	
\\	不器用な助手が、不器用な手つきで、いつものようにクソ不味いコーヒーをいれてくれた。	
\\	私の主人は堅苦しくて不器用ですが、正直者です。	
\\	器用 
\\	不 
\\	ふ 
\\	ぶ. 
\\	(ぶ), 
\\	不, 器, 用	
\\	自信	
\\	じしん	
\\	する 
\\	コウイチは大変な自信家です。	
\\	もっと自信を持った方がいいよ!	
\\	ビエトは、誰よりもコウイチのことをよく知っているので、誰よりもうまくコウイチの顔を描ける自信があると言っていた。	
\\	自信 
\\	自, 信	
\\	勇気	
\\	ゆうき	
\\	ワニカニユーザーは、コウイチのスピーチに勇気づけられました。	
\\	他の人の意見に反対するにはふ通は勇気がいると思うけど、トーフグで働いている時はそんな心配はいらないよ。	
\\	「フグ、私のお父さんは素晴らしい人だったのよ。いつも私に、流れに逆らって泳ぐには力と勇気が必要だ。浮かぶだけなら死んだ魚でもできるって言ってたわ。」「サーモン、それって、サミュエル・スマイルズの名言じゃなかったっけ。」	
\\	勇, 気	
\\	地区	
\\	ちく	
\\	の 
\\	来月から、この地区のたん当者が変わるので、あらかじめお知らせさせていただきます。	
\\	この地区には、りん時ふくしきゅう付金を受きゅうしている人が三十八人います。	
\\	コウイチは、近日中に地区検察局に自白内容を収めたビデオテープを送ることを計画しているようですよ。	
\\	地, 区	
\\	区	
\\	く	
\\	私のおじは、中央区の区役所で働いています。	
\\	ここはちゅう車きん止区なので、車は止められませんよ。	
\\	ここ最近、横浜の外国人居留区では、暴力行為がかつてないほど横行している。	
\\	区	
\\	不信	
\\	ふしん	
\\	の 
\\	あの変人が不信の目でおれのことを見てくるんだよ。	
\\	ワニカニユーザーのトーフグへの不信はつのるばかりだった。	
\\	新しい日本語学習教材を部分的に公開することは、我々の会社が何か事実を隠蔽しようとしているとの不信を招くかもしれないじゃないですか?	
\\	不, 信	
\\	区分	
\\	くぶん	
\\	する 
\\	の 
\\	この町では、校区が三つに区分されています。	
\\	これは予算区分の問題なので、私にはどうしようもありません。	
\\	私は、テロ行為と合法的な戦争を区分する必要は無いと思います。	
\\	区, 分	
\\	英会話	
\\	えいかいわ	
\\	私の兄は、英会話は上手だけど、計算は下手なんです。	
\\	まさか。また英会話のテストに落ちちゃったよ。ショックすぎる。	
\\	英会話の試験でミスをしたことで動揺してしまい、さらに別のミスを誘発することになってしまいました。	
\\	英, 会, 話	
\\	英和	
\\	えいわ	
\\	英和ほんやくの仕事をしています。	
\\	新しい英和対やくの本が出ました。	
\\	おっと、キャメロンが泣きそうだ!ちょっとあの英和辞典を探して来て、キャメロンに渡してあげてくれない?	
\\	英, 和	
\\	英語	
\\	えいご	
\\	の 
\\	英語の発音は日本語の発音よりむずかしいと思う。	
\\	もしこのレストランの店員も英語が話せるようだったら、今後わざわざあの店まで行かなくてすむようになるね。	
\\	「僕たちは皆、英語と数学の授業が嫌いだ。」「ちょっと、一緒にしないでよ。私は両方共好きだし。」	
\\	英, 語	
\\	英文	
\\	えいぶん	
\\	ここに日本語で文をかいて、その下に英文でやくを付けてください。	
\\	わたしの母は高校の英語の先生になるために、大学で英文科をそつ業したそうです。	
\\	ねぇ、忙しいところ悪いんだけどさ、この英文の翻訳にちょっと手こずっちゃってるんだよね。手伝ってもらえないかな?	
\\	英, 文	
\\	毎晩	
\\	まいばん	
\\	体を丈夫にするために、毎晩やく用酒を飲んでいます。	
\\	私の夫は、毎晩五時に夕食を食べて、十時には寝ます。	
\\	毎晩泣く子を抱きながら、子供の世話をするには大変な忍耐が必要だということを学びました。	
\\	毎, 晩	
\\	信心	
\\	しんじん	
\\	する 
\\	の 
\\	私の母は仏教を信心しています。	
\\	信心ぶって他人をひはんしてくるから、あの子のことあんまり好きじゃないんだよね。	
\\	彼の家族はとても信心深くてスピリチュアルなので、他のお酒を飲まずにスピリッツだけ飲みます。	
\\	しん 
\\	じん.	信, 心	
\\	食中毒	
\\	しょくちゅうどく	
\\	食中毒で会社を休んでしまい、大変申しわけありません。	
\\	ハーゲンダッツを食べて食中毒を乗りこえるというとても役立つ情ほうを教えてくれて、本当にありがとうございました。	
\\	フグが私に食中毒の危険性をレクチャーしてくれたんだけどさぁ、説明がすっごく分かりにくかったんだよね。	
\\	食, 中, 毒	
\\	大仏	
\\	だいぶつ	
\\	昨日、ならの大仏をおとずれました。	
\\	大仏をすんげーしんけんに見つめてるあの男、だれ?	
\\	「彼女、大仏さんと付き合っているのよ。」「とても信じられないわね。だって、彼女の父親と言ってもいいくらい年上でしょ?」	
\\	大, 仏	
\\	今晩は	
\\	こんばんは	
\\	今晩は。こんな所で何してるの?	
\\	「あら、今晩は。」「今晩は。」「昨日の社長とのミーティング、どうだった?」「それが....てっきりしょう進するものだと信じてたんですが、首になりました。」	
\\	今晩は、コウイチおじさん。今日は私とフグの赤ちゃんの面倒をみてくれて有難うございます。あ、ところで、電子レンジは使わなかったよね? 母乳だから、電子レンジ使ったらいけないんだよ。	
\\	こんにちは, 
\\	今日は, 
\\	は 
\\	は 
\\	わ.	今, 晩	
\\	文法	
\\	ぶんぽう	
\\	の 
\\	日本語の文法の本をさがしています。	
\\	文法を学べば学ぶほど、文章をかくのが上手になりますよ。	
\\	私の上司、見た目はできる男なのに、いつも間違った文法を使うのが残念すぎるよ。	
\\	ほう 
\\	ぽう.	文, 法	
\\	丈	
\\	たけ	
\\	ユニクロで、七分丈のパンツを買いました。	
\\	少し見ないうちに、ずいぶん丈がのびたな!	
\\	話を変える訳じゃないんだけど、そのスカートはちょっと丈が長過ぎるんじゃないかしら?	
\\	竹? 
\\	(たけ) 
\\	丈	
\\	夫	
\\	おっと	
\\	私の夫の歯は、とても丈夫です。	
\\	私の夫は、大変すばらしい男性です。	
\\	ほら、また始まった。あんたって本当にいつも自分の夫の不満ばかり言っているよね。	
\\	(おっと), 
\\	(おっと) 
\\	夫	
\\	不法	
\\	ふほう	
\\	な 
\\	の 
\\	ビエトが行っている不法のやみとり引のことを、トーフグの社員はだれも知らない。	
\\	アメリカにはげん時てんで何人の不法たいざい者がいると思いますか?	
\\	不法な捜査で得た証拠は法廷では使えませんよ。	
\\	不, 法	
\\	事変	
\\	じへん	
\\	上海事変は何年に起こりましたか。	
\\	事変という言葉を使った例文をさがしていたら、東京事変というバンドがあることを知りました。	
\\	満州事変について説明をしてもらえますか?	
\\	(満州事変), 
\\	変事) 
\\	(変) 
\\	事, 変	
\\	伝	
\\	つて	
\\	母が亡くなったことを伝えたいんですが、おじさんにはれんらくする伝がないんです。	
\\	伝をたよって、トーフグにしゅうしょくしました。	
\\	誰かの伝があるといいんだけど。	
\\	(つて) 
\\	伝	
\\	子猫	
\\	こねこ	
\\	その子猫は、えだからえだを軽かいにとびうつりました。	
\\	子猫が時計回りにクルクル歩き回っている動画を
\\	にアップロードしました。	
\\	私達、ちょうどこの子猫を見つけて、飼い主をさがしているんです。	
\\	子 (こ) 
\\	(ねこ).	子, 猫	
\\	昨年	
\\	さくねん	
\\	この二十四時間スーパーは、昨年オープンしました。	
\\	昨年は旅行でグアムに行きました。	
\\	昨年、医者から中性しぼうのあたいが高いと言われちゃったんですよ。	
\\	昨, 年	
\\	列	
\\	れつ	
\\	「列の最後びはここですか?」「いいえ、私は列の先頭です。」	
\\	私のせきは、前から二列目でした。	
\\	ベーコンショップの前には長い列ができていた。	
\\	列	
\\	昼休み	
\\	ひるやすみ	
\\	大変だ!あと五分で昼休みだ。	
\\	今日の昼休みに食べたカツどんは、大変美味しかったです。	
\\	昼休みにゆかたに着がえるつもりです。	
\\	休み 
\\	昼, 休	
\\	ご飯	
\\	ごはん	
\\	最近わたしたち、ご飯を食べるタイミングが合わなくてすれちがってばかりだよね。	
\\	しまった!ご飯こぼしちゃった。	
\\	どうしてコウイチのデスクの引き出しには乾いたご飯の塊が入っているの?	
\\	ご 
\\	ご飯
\\	飯	
\\	飯	
\\	めし	
\\	しまった!飯をたくのをわすれてた。	
\\	コウイチは三度の飯よりトーフグの仕事をするのが好きです。	
\\	あそこの飯は本当に臭かったぜ。二度と塀の中には戻りたくねえな。	
\\	(めし) 
\\	飯	
\\	方法	
\\	ほうほう	
\\	ビエトには、その方法はきかないっていうか、むしろぎゃくこうかですよ。	
\\	赤ちゃんをぐっすりベッドでねむらせる方法が知りたいです。	
\\	シャーロック・ホームズって、古びた犯罪調査方法を使っているよね?	
\\	方法? 
\\	方, 法	
\\	情けない	
\\	なさけない	い 
\\	情けないですが、すぐに人を信じる性格なので、よくだまされるんです。	
\\	ツテでしゅうしょくするなんて、情けない。実力で勝負しろよ。	
\\	日本の政治家には、本当に情けない奴が多い。	
\\	情け 
\\	ない 
\\	情	
\\	映画館	
\\	えいがかん	
\\	次にこの町にくる時には、この映画館が完成して町なみもずいぶん変わってるんだろうね。	
\\	れん休も終わったし、あの映画館多分しばらくガラガラだろうね。	
\\	「今夜、映画館に行かない?」「そうできればいいんだけど、このレポートを仕上げなくちゃいけないんだ。」	
\\	映画 
\\	(映画) 
\\	映, 画, 館	
\\	人殺し	
\\	ひとごろし	
\\	あいつ、人殺しのくせに、仏教となんだぜ。	
\\	計算ずくの人生なんてつまらないと思って好き勝手していたら、いつの間にか人殺しになってたんだ。	
\\	おい、昨日駅でお前が彼女といるのを見かけたぜ。お前の彼女、人殺しとも付き合えないぐらい不細工だな。	
\\	人
\\	殺す
\\	ころし 
\\	ごろし 
\\	人, 殺	
\\	殺人者	
\\	さつじんしゃ	
\\	その殺人者は、とう走中に大変な失ぱいをおかしてしまった。	
\\	殺人者とうたわがわれていましたが、ついにようぎが晴れました。	
\\	今の今まで、俺の友達の人生は信じられない苦難の連続でした。彼の人生は殺人者の子どもとして始まりました。	
\\	殺人 
\\	殺, 人, 者	
\\	海軍	
\\	かいぐん	
\\	の 
\\	海軍にいた時、おれ的にはお前には十分すぎるぐらいやさしくしてたつもりなんだけどな。	
\\	この海軍のしきたりでは、本当に悪かったなと思って反せいしている人は、自らすすんでトイレそうじをすることでせい意を見せることになっています。	
\\	電話に出れなくてごめん。信じられないかもしれないけど、実は今海軍の船の上なんだよね。降りたら電話するね。	
\\	海, 軍	
\\	新築	
\\	しんちく	
\\	する 
\\	の 
\\	新築の家でも買えば、気分が晴れるよ。	
\\	新築のマンション以外には住みたくありません。	
\\	悲しいことに夫に捨てられちゃいました。新築の一軒家を買ったばっかりなのに。話を聞いてほしいんだけど。	
\\	新, 築	
\\	晩	
\\	ばん	
\\	昨日の晩はフグの毒にあたって死ぬところでした。	
\\	三日前の晩から毎晩コウイチがゆめに出てきてわたしに
\\	を買わない不こうになると耳打ちするんです。	
\\	「よお!今晩、飲みに行かないか?」「ごめん。今、禁酒中なんだよ。」	
\\	晩	
\\	昼	
\\	ひる	
\\	今日の昼、あそこの定食屋でお昼一しょに食べようよ。	
\\	目ざまし時計が鳴らなくて、起きたらすでに昼だった。	
\\	今から一時間昼休憩を取るので、電話は取り次がないようにしてください。	
\\	昼	
\\	計画	
\\	けいかく	
\\	する 
\\	夏休みに海に行こうと計画しています。	
\\	計画を立てるのは大好きなんですけど、計画したことを実行するのはきらいなんです。	
\\	もし俺たちがエッフェル塔を盗むというなら、かなりしっかりした計画を練る必要がある。	
\\	画 
\\	(かく), 
\\	計, 画	
\\	毒	
\\	どく	
\\	働きすぎは体に毒ですよ。	
\\	ビエトのつくえの引き出しには、コウイチをいつでも暗殺できるように毒の矢が仕こんである。	
\\	フグはとても美味しく、食べる人の心を豊かにしてくれますが、一方で猛毒も持っています。	
\\	毒	
\\	毒ガス	
\\	どくがす, どくガス	
\\	毒ガスは第一次世界大戦で使われ始めたらしいです。	
\\	午前十時からの毒ガスの使い方のレッスンを取っていたはずなんですが、もう一度かくにんしてもらえますか?	
\\	「この野郎!また屁をこきやがったな!毒ガスのように臭いじゃねえか。」「すみません。もう二度とおならはしません。」「いいや、きっとまたこくさ。もう堪忍袋の緒が切れた。お前は首だ。」「そんな〜。」	
\\	ガス 
\\	毒	
\\	中毒	
\\	ちゅうどく	
\\	する 
\\	の 
\\	食中毒でかこきゅうになり、きゅうきゅう病院へはん送されました。	
\\	ガス中毒にならないよう気を付けてくださいね。	
\\	フグは麻薬中毒から逃れようとしてリハビリセンターに行ったが、結局また麻薬中毒になってしまった。	
\\	中, 毒	
\\	有毒	
\\	ゆうどく	
\\	な 
\\	の 
\\	ガラガラヘビって有毒なんですか?	
\\	ダリンが道ばたに生えていた有毒キノコを間ちがって食べて、病院にはこばれてしまいました。	
\\	ビエトだけがコウイチのオナラが有毒ガスのような臭いがすることを知っている。	
\\	有, 毒	
\\	返信	
\\	へんしん	
\\	する 
\\	やばい、返信するのわすれてた。ちょっと大変なことになってしまうかも。	
\\	き読になったのに、いつまでたっても返信が来ないんだよね。	
\\	お疲れ〜。返信したつもりだったんだけど、どうやら力尽きて、送信ボタンを押す前に寝てたみたい。ごめんね、許して〜。	
\\	返, 信	
\\	指輪	
\\	ゆびわ	
\\	大事な指輪を失くしてしまい、気分が晴れません。	
\\	指輪が指から外れなくなってしまいました。	
\\	「ボーナスが出たら、ダイヤモンドの指輪を買ってやるよ。」「あなたはいつも口先ばかりでしょ。」	
\\	指 
\\	輪. 
\\	ゆびわ.
\\	指, 輪	
\\	輪ゴム	
\\	わごむ, わゴム	
\\	この輪ゴムはとても丈夫です。	
\\	輪ゴムどろぼうのうたがいをかけられていたが、ついに晴れの身となりました。	
\\	わぁ!そのワンピマジで可愛い!カラフルな輪ゴムの絵がたくさんプリントしてあるんだね!そんなワンピ初めてみたよ。	
\\	(ゴム) 
\\	輪 
\\	わ, 
\\	わゴム.	輪	
\\	大変	
\\	たいへん	
\\	な 
\\	日本語を習とくするのは大変ですが、がんばってくださいね。	
\\	大変丈夫な子供が生まれました。	
\\	災害時は、電話回線もネット回線も混み合って、繋がりづらくなるので、ブログの更新が大変です。	
\\	大, 変	
\\	浅い	
\\	あさい	い 
\\	この海は浅いから子どもも安心して泳げます。	
\\	考えの浅い人間にはなりたくありません。	
\\	酔っ払うと、眠りが浅くなっちゃって、早朝覚醒しちゃうんだよね。	
\\	い 
\\	あさい	浅	
\\	単に	
\\	たんに	
\\	コウイチは単に大変な努力家なだけですよ。	
\\	「トーフグは、私にとっては単にお金をかせぐためのものではないんです。トーフグこそが私の人生なんです。」と、コウイチは言った。	
\\	剣道では、単に相手を倒すことだけではなく、自らの心を鍛えることも重要です。	
\\	(に) 
\\	単	
\\	坂	
\\	さか	
\\	その寺なら、この坂を上ったところにありますよ。	
\\	ビエトは自転車でゆるやかな坂を下るのが好きです。	
\\	スター選手であったフグを失ってから、
\\	野球チームの調子は下り坂となってしまった。	
\\	坂	
\\	軍人	
\\	ぐんじん	
\\	おきなわにはアメリカ軍のき地があるから、軍人さんがたくさんいます。	
\\	その軍人の勇気ある行動が、多くの人の命をすくいました。	
\\	昨日、上司に俺はリストラで首になるって言われちゃってさ。だから、今、軍人になることを真剣に考えてるんだ。	
\\	軍, 人	
\\	頑丈	
\\	がんじょう	
\\	な 
\\	土台はできるだけ頑丈にしておいた方がいいですよ。	
\\	この車はとても頑丈にできています。	
\\	眼鏡屋さんで、とても頑丈な眼鏡を作ってもらいました。	
\\	頑, 丈	
\\	春	
\\	はる	
\\	わたしは虫が苦手だから春はきらいです。	
\\	「お前のチーム、春の大会ではどうだったの?」「へへ。大勝したっての!」	
\\	春巻きという料理に「春」の字が使われているのは、春に芽が出る野菜を使って作る食べ物だかららしいですよ。	
\\	はる, 
\\	春	
\\	春休み	
\\	はるやすみ	
\\	春休みキャンペーン期間中は、学生の入場料は一りつ千円です。	
\\	春休み中だったら、この時間たいは空いてます。	
\\	そろそろ風邪治った?春休みに風邪引くなんて最悪だよね。もしまだ辛いなら、夜更かししちゃだめだよ。早く良くなってね。	
\\	夏休み 
\\	冬休み. 
\\	春 
\\	休み 
\\	休み, 
\\	春, 休	
\\	変	
\\	へん	
\\	な 
\\	はしらのかげに変な人がいると思ったらただのコウイチだった。	
\\	同じ仕事をしているのに男性の方が女性よりきゅう料が高いなんてぜっ対に変だよ!	
\\	変だなぁ。僕のメールボックスには、君が送ったと言っているメールがどこにも見当たらないよ。	
\\	変な猫!.
\\	変	
\\	変事	
\\	へんじ	
\\	とつ然の変事で、今日のかぶ式市場は大変なことになっています。	
\\	トーフグに何か変事が起きたんじゃはないかと、全ワニカニユーザーが心配していたんですよ。	
\\	私は変事の前の不気味な静けさにいつもワクワクする。	
\\	事変 
\\	事変 
\\	変事 
\\	変, 事	
\\	丈夫	
\\	じょうぶ	な 
\\	私のつまは、あまり体が丈夫ではなく、よく病気になります。	
\\	このいすはとても丈夫なので、すもう取りがすわってもこわれません。	
\\	彼が丈夫な男だってことは分かってるけど、暴飲暴食の習慣がちょっと心配なんだよね。	
\\	丈, 夫	
\\	同期中	
\\	どうきちゅう	
\\	同期中に電源が切れて軽くあせった。	
\\	、いつも写しん同期中にフリーズしてしまうのはなぜ。	
\\	たぶんもうすぐ反映されると思う。私のエバーノート今同期中だし。	
\\	同, 期, 中	
\\	美味しい	
\\	おいしい	い 
\\	コウイチとビエトは、オフィスに美味しいウィスキーをかくし持っている。	
\\	少しずつ、色んなしゅるいの美味しいものが食べたいんです。	
\\	最近、
\\	のオフィス内に、料理の美味しいパブがたくさん開店してるんだよね。	
\\	おいしい 
\\	(おい). 
\\	美, 味	
\\	寺	
\\	てら	
\\	昨夜、あのお寺で変死体が発見されたらしいよ。	
\\	子どものころ、寺のとなりに住んでいたので、仏教の勉強には大変都合が良かったんです。	
\\	日本の地元のお寺で誕生日会を開いたんだけど、すぐに追い出されちゃった。お寺ってロック・コンサートをやる場所じゃないってこと、知っておけばよかったんだけど。	
\\	お寺.	寺	
\\	昨今	
\\	さっこん	
\\	昨今はあなたの様な大物にお目にかかれることはめったになくなりましたよ、コウイチさん。	
\\	昨今の流行りにあやかって、コウイチはオフィスでしば犬とチンド犬をかうことに決めました。	
\\	昨今は就職氷河期だと言われているが、会社を選ばなければどこかには就職できるはずである。	
\\	さく 
\\	さっ 
\\	昨, 今	
\\	整える	
\\	ととのえる	
\\	ビエトは、口ひげが三角になるように毎日整えています。	
\\	父のそうぎの手はずは、兄が全て整えてくれました。	
\\	ホルモンバランスを整える方法をご紹介致します。	
\\	う 
\\	(ととの)!
\\	整	
\\	整理する	
\\	せいりする	する 
\\	本だなを整理してくれたこと、大変感謝しています。	
\\	この書るいを全て整理するには、大変な時間と労力が必要である。	
\\	ちょっと話を整理させてね。つまり、男の人が浮気をするのはいいけど、女の人がするのはだめってこと?それがあなたの言いたいこと?	
\\	整, 理	
\\	好む	
\\	このむ	
\\	この日本酒は、大変酒好きだった私の父が生前に好んで飲んでいたものです。	
\\	コウイチは、黒ニンニクを好んで食べます。	
\\	コウイチさん、申し訳ありませんが、好むと好まざるとに関わらず、あなたの秘密は暴露されてしまうかもしれません。	
\\	好き 
\\	好き 
\\	この, 
\\	(この) 
\\	好	
\\	信じる	
\\	しんじる	
\\	信じられないかもしれないが、実はビエトはスパイだったんだよ。	
\\	自分が目にしたもの以外は信じない主ぎなんです。	
\\	自分がもはやサンタクロースを信じていないということに、最近イライラしている...と思うんだよね。	
\\	う 
\\	信 
\\	しん 
\\	信	
\\	建てる	
\\	たてる	
\\	この男は、とても丈夫な家を建てることで有名です。	
\\	その王は、山の上に信じられないほど美しいおしろを建てました。	
\\	我が社は、外国の子どもたちに学校を建ててあげるために貯金をしています。	
\\	う 
\\	たてる. 
\\	建	
\\	変える	
\\	かえる	
\\	大人になると、考え方を変えることはむずかしいです。	
\\	あっ、フグのくちびるの形は変えないでください。それはコウイチの好物のタラコをモチーフにしているんです。変えたらコウイチが悲しみます。	
\\	眉毛の形を変えると、まるで別人のようだね。	
\\	(か).	変	
\\	晴れる	
\\	はれる	
\\	今は雨がふってますけど、二時までには晴れると思いますよ。	
\\	昨日はよく晴れて良かったですね。	
\\	ようやくフグと結婚することになったので、是非式に来てほしいなと思っています。詳細を同封した招待状を後ほどお送りしますね。パーティーは野外の予定なので、当日晴れるといいんだけどな。	
\\	う 
\\	晴	
\\	泣く	
\\	なく	
\\	ヤバイ。うれしすぎて、泣きそう。	
\\	情けないことに、晴れのぶ台で泣いてしまった。	
\\	減給になっちゃって泣いてます。もっと給料のいい会社に移りたいよ。	
\\	う 
\\	泣	
\\	放つ	
\\	はなつ	
\\	コウイチは、するどく変化するサーブを放つバレーボール選手として知られている。	
\\	その選手は、ホームランを放ったしゅん間、くさいオナラも放った。	
\\	この鳥をいつか籠から放つつもりですか。	
\\	う 
\\	放す 
\\	放れる 
\\	放	
\\	見返す	
\\	みかえす	
\\	トーフグの社長になって、コウイチを見返してやるぞ!	
\\	子猫にイワシをあげたら、「本当にいいの?」って感じの顔でマジマジと見返してきたのがマジかわいかった。	
\\	答案は提出する前に見返したほうがいい。	
\\	見 
\\	返す 
\\	見, 返	
\\	暗記する	
\\	あんきする	する 
\\	クリステンはセーラームーンのテーマ曲の歌しを暗記している。	
\\	明日までに、台本に書いてあるセリフを一つ残らず暗記しなければいけません。	
\\	文を丸ごと暗記する方が、単語だけを暗記するより効率がいい。	
\\	暗記 
\\	暗記. 
\\	暗, 記	
\\	練習する	
\\	れんしゅうする	する 
\\	昨日、肉って言葉の発音を六時間も練習したんだぜ。	
\\	「おんをあだで返すような子に育てたおぼえはないよ!」というしばいのセリフを言うのを練習しています。	
\\	「俺達、惨敗しちまったな。」「ああ、惨敗しちまった。もっとたくさん練習しないとな。」	
\\	練習, 
\\	練習. 
\\	練, 習	
\\	映す	
\\	うつす	
\\	スクリーンにスライドを映し始めてもらえますか?	
\\	この白いかべをスクリーン代わりにして映画を映すのはどうでしょう。	
\\	ビエトだけが、コウイチが鏡に全身を映すのが好きだということを知っている。	
\\	映る 
\\	映る. 
\\	映る 
\\	映	
\\	帰る	
\\	かえる	
\\	カナエちゃんは、日本から帰ってきて、オフィスの人におみやげを配りました。	
\\	「好む」という単語をか去形にしなさい、という問題の答えが分からないまま、家に帰りました。	
\\	「お腹が空いたし足が痛いよ」と、ホビット庄に帰るちょっと前にビルボーは言った。	
\\	う 
\\	帰	
\\	合計	
\\	ごうけい	
\\	する 
\\	ビエトのさぎにひっかかったひがい者の合計は、百人になった。	
\\	合計金がくを出してもらっていいですか?	
\\	何度も足し算をしたんですが、その度に合計が違ったんですよね。私、馬鹿なんでしょうか?	
\\	合, 計	
\\	信号	
\\	しんごう	
\\	する 
\\	赤信号で止まって、青信号でわたるのよ。	
\\	あぶないから、信号をむししてはいけませんよ。	
\\	いいか、よく聞けよ。六つ目の信号で左折した後、八つ目の角まで歩いて右折したら、そこで会った人に向かってお尻を振るんだ。それが秘密のサインだからな。そしたら、奴はお前を俺たちのアジトまで連れて行ってくれるさ。	
\\	信, 号	
\\	信用	
\\	しんよう	
\\	する 
\\	トーフグチームは、コウイチの言うことを全めん的に信用しています。	
\\	コウイチの不しょう事が、トーフグの信用をきずつけてしまった。	
\\	私は、掃除や料理、マッサージや寝る前の歌なんかで、妻から信用点をつけられていると思っていたが、なんと逆に離婚されてしまったよ。	
\\	信, 用	
\\	変死	
\\	へんし	
\\	する 
\\	けいさつは、コウイチの変死についてちょうさしています。	
\\	「なんかパニクってるけど、どうしたの?」「コウイチは変死して、トーフグがとうさんするって聞いたんだよ。最悪だよ。」	
\\	コウイチが謎の変死を遂げたというニュースは、全国民を震撼させた。	
\\	変, 死	
\\	時計	
\\	とけい	
\\	コウイチはビエトにもらった時計を船から落としてしまいました。	
\\	時計台の下で、三時に待ち合わせね!	
\\	ねぇ、あなたの時計に唾が着いてない?	
\\	計 
\\	時 
\\	と 
\\	時 (とき), 
\\	(とけい) 
\\	時, 計	
\\	単語	
\\	たんご	
\\	の 
\\	日本語の単語の単語ちょうを作りました。	
\\	単語の意味が分からない時は、じしょを引きます。	
\\	日本語の単語ゲーム作ったんだけど、やってみない?	
\\	単, 語	
\\	昨日	
\\	きのう, さくじつ	
\\	昨日の夜は、よくねむれた?	
\\	昨日はアメリカとカナダで労働者の日でした。	
\\	昨日は、大学の課題で、沢山の叙事詩を読まなくてはいけませんでした。	
\\	(きのう), 
\\	昨, 日	
\\	留守	
\\	るす	
\\	する 
\\	母は買い物に出かけているので、私と妹が留守をしています。	
\\	コウイチの家にいってみたが、留守でした。	
\\	私の夫は出張で留守がちです。夫が留守にしてる間、とても寂しいので、一日少なくとも五回は電話をします。	
\\	留, 守	
\\	料理人	
\\	りょうりにん	
\\	コウイチからの手紙を読んで、料理人になる決意を固めました。	
\\	料理人は、「料理が気に入らなかったら、残してくださっていいですよ」と悲しそうに言いました。	
\\	その料理人は、冷凍庫のドアから栓抜きを剥ぎ取った。	
\\	料理 
\\	料, 理, 人	
\\	入門	
\\	にゅうもん	
\\	する 
\\	の 
\\	コウイチは有名な剣道の先生のところに入門するそうですよ。	
\\	入門手つづきをまだ終えていません。	
\\	日本語に馴染みのない私にとっても、テキストフグは絶好の入門書だと思う?買った方がいいかな?	
\\	入, 門	
\\	冒険	
\\	ぼうけん	
\\	する 
\\	ビエトは仕事を残したまま、冒険に出かけた。	
\\	明日からの冒険のために、力を残しておいた方がいいよ。	
\\	私たち、ナイル川を遡るっていう冒険旅行をしたんだけど、彼氏が西ナイル熱にかかっちゃってさあ。未だに病院で入院してるんだよね。	
\\	冒, 険	
\\	技術	
\\	ぎじゅつ	
\\	私は医者ではありませんが、ある医者から、心ぞういしょくの技術を学びました。	
\\	以前に働いていた会社で、えい業技術を身につけました。	
\\	私は技術的に優れたボクサーと結婚した。	
\\	技 
\\	技, 術	
\\	品	
\\	ひん, しな	
\\	この品は、まとめ買いがおとくですよ。	
\\	彼は男らしくて良い人だけど、品のないしゃべり方をするので少し苦手です。	
\\	コウイチさんから開業祝いの品が届いたので、机の上に置いておいたよ。ちなみにメッセージカードには、「あまりにも早く独立開業を成し遂げたあなたのパワーに圧倒されています」って書いあるよ。	
\\	品	
\\	芸術家	
\\	げいじゅつか	
\\	この芸術家の作品には、苦笑いがかくせません。	
\\	その芸術家は、弟子に留守番をたのんで、ヨーロッパに留学しました。	
\\	再来月のマラソン大会、芸術家達と一緒に参加するってのはどう?	
\\	芸術 
\\	芸術家 
\\	芸, 術, 家	
\\	自動的	
\\	じどうてき	な 
\\	それは、ボットによる自動的なツイートですよ。	
\\	この電車のドアは、自動的には開きません。	
\\	もし誰かがあなたの家に侵入したら、この機械が自動的に警備会社に通報します。	
\\	(自動 
\\	的) 
\\	自動 
\\	自, 動, 的	
\\	自動車	
\\	じどうしゃ	
\\	あの自動車ディーラーにうら切られた気分です。	
\\	新しい自動車を買うために、お金をためるようにしています。	
\\	自動車の運転免許を失くしてしまいました。	
\\	自動 
\\	自動 
\\	車
\\	自, 動, 車	
\\	作品	
\\	さくひん	
\\	この箱庭の作品は、マッチ箱で作られています。	
\\	トーフグオフィスの入り口にあるヘンテコな像は、コウイチの十五才の時の作品ですよ。	
\\	久々にいい作品に出会ったわ。進撃の巨人は、私の好きなアニメの歴代三位に入るんじゃないかってくらい良かったよ。マジでお薦めするわ。	
\\	作, 品	
\\	弁当	
\\	べんとう	
\\	私のかばんの中には、さいふと、弁当と、タバコが一箱入っています。	
\\	お弁当のおかずを少し残しておいて、子猫にあげました。	
\\	明日あなたにお弁当作ってあげようと思うんだけど、何か苦手なものとかあるの?	
\\	弁 
\\	当 
\\	べんとう. 
\\	弁, 当	
\\	箱	
\\	はこ	
\\	コウイチは、大切にしていたダンボール箱がすてられたことを知って、顔面そう白になりました。	
\\	夫がトマトを箱で買ってきて、食べきれなくてこまっています。	
\\	トーフグのオフィスに届いた箱の中には、辞書と死んだフグが詰まっていた。	
\\	箱	
\\	新品	
\\	しんぴん	
\\	の 
\\	線路に新品のそろばんを落としてしまいました。	
\\	新品のジーパンになっとうをこぼしてしまって、テンションめっちゃ下がってます。	
\\	「新しいパソコンを買ったんだって?」「うーん。新品ではないんだけどね。ビエトのお下がりだよ。」	
\\	ひん 
\\	ぴん. 
\\	新, 品	
\\	仏典	
\\	ぶってん	
\\	その仏典はこの箱の中にしまってもらえますか?	
\\	こんざつ時は、ゆずり合って仏典をご利用ください。	
\\	仏典がどのようにしてサンスクリット語から漢語に訳されたのかご存じですか?	
\\	ぶつ 
\\	ぶっ.	仏, 典	
\\	字典	
\\	じてん	
\\	コウイチは、ぬれた字典について、ビエトにせつ明を要求しました。	
\\	とつ然大雨になったので、手でもっていた字典がびしょぬれになっちゃったんです。	
\\	この書道字典はあそこの古本屋で買ったんです。	
\\	字, 典	
\\	古典的	
\\	こてんてき	な 
\\	コウイチは写真で古典的なポーズをとるのが好きなようだね。	
\\	こんな古典的なさぎに引っかかってしまって、本当に情けないです。	
\\	あの日本の古典的なスタイルの画家は、私が知っている人の中で、一番好みがうるさいです。どうしてそんなに色々気にするのか理解不能です。	
\\	古, 典, 的	
\\	全治	
\\	ぜんち, ぜんじ	
\\	する 
\\	医者には、全治するのは難しいと思う、多分後いしょうが残るだろうって言われました。	
\\	交通きそくをきちんと守っていたのに、トラックにはねられて全治一年の大ケガをしました。	
\\	ビエトが暴力団抗争で全治六カ月の重傷を負った時、コウイチは特殊治癒能力を使ってその傷を治した。	
\\	全, 治	
\\	関心	
\\	かんしん	
\\	日本のアニメの声ゆうは好きですが、海外のアニメの声ゆうには関心がありません。	
\\	げんざいの最大の関心は何ですか?	
\\	当社トーフグ、ならびに当社製品に関心をお寄せいただき、有難うございます。	
\\	関, 心	
\\	関係	
\\	かんけい	
\\	する 
\\	の 
\\	ビエトはヤクザだが、そのはんざいには関係してないとだん言できます。	
\\	ビエトは、ヤクザとの関係をたってトーフグの仕事に専念するかどうかまよっています。	
\\	「彼は前に結婚していて、子供も3人もいるのよ。」「知ってるわよ。でも、それとこれとは関係ないでしょ。」	
\\	関, 係	
\\	保守的	
\\	ほしゅてき	な 
\\	保守的な学校だったので、学生はみんな校そく通りにせい服を着用するよう要求されました。	
\\	日本国民は、保守的な国民だと思いますか?	
\\	コウイチはビエトに、年をとるにつれて保守的になる人が多いけど、俺はそうはなりたくない、と言った。	
\\	守. 
\\	しゅ, 
\\	(ほ) 
\\	(しゅ) 
\\	保, 守, 的	
\\	取材	
\\	しゅざい	
\\	する 
\\	取材にご協力いただけませんか。	
\\	テレビの取材を受けた時に、金が全てではないと答えたところが、ニュース番組に使われていました。	
\\	無理は承知なんですが、どうにかご主人を取材させてもらうことはできないでしょうか?もし何とかして頂ければ、本当に本当に感謝します。	
\\	取, 材	
\\	下品	
\\	げひん	
\\	な 
\\	私は下品な英語の基本をマスターしました。	
\\	はるか遠くから、下品な笑い声が聞こえてきた。	
\\	あなたの下品なメールは会社に戻ってからじっくり読ませてもらいます。	
\\	下 
\\	(げ). 
\\	下, 品	
\\	危ない	
\\	あぶない	い 
\\	危ないから、車に荷物をつむのは力の強いビエトにやってもらおう。	
\\	四つ葉のクローバーを見つけたら幸運がおとずれるって言うけど、ぎゃくに危ない目にあっちゃったよ。	
\\	危ない会社のこんなに危ない計画に巻き込まれることに、誰も反対しなかったなんて信じられないよ!	
\\	い 
\\	(あぶ) 
\\	危ないよ!!!!, 
\\	危	
\\	曜日	
\\	ようび	
\\	水曜日は、ネイティブが話すスピードで日本語を話す日にしませんか?	
\\	今日って何曜日だっけ?	
\\	肝心なのは、この日本語の課題を金曜日までに終わらせなければならないということだ。	
\\	曜日 
\\	曜 
\\	日. 
\\	ひ 
\\	日 
\\	び 
\\	曜, 日	
\\	〜弁	
\\	べん	
\\	日本語で一番面白い方言は東北弁でしょうか?	
\\	私は日本語の共通語も関西弁も話せるからバイリンガルなんですよ。	
\\	博多弁で喋る女性はとても可愛いと思います。	
\\	東京弁, 
\\	弁	
\\	辞書	
\\	じしょ	
\\	コウイチは、トーフグにたくさんの辞書を残して死んだ。	
\\	これは、世界で一番売れている辞書です。	
\\	「愛」という字を辞書で引いて、その意味の上に彼氏の名前を書いておいたの。そしたら、彼に見つかっちゃって、それで振られちゃったのさ。	
\\	辞, 書	
\\	辞典	
\\	じてん	
\\	その辞典、使い終わったらちゃんと箱の中にもどしておいてね。	
\\	電子辞書には色んな辞典が入っています。	
\\	辞典の使い方が分かりません。	
\\	辞, 典	
\\	事典	
\\	じてん	
\\	コウイチなら、サンルームの日光浴用のイスで事典とにらめっこしていますよ。	
\\	あなたが事典を万引きしたしゅん間がカメラに映っているんですよ。	
\\	本としてはちょっと大きすぎたので、私は百科事典を枕として使っていました。	
\\	事, 典	
\\	証	
\\	あかし	
\\	友情の証として、このふじ山のキーホルダーをあげます。	
\\	あいつの身の証を立てるのはそうかん単なことじゃないかもしれない。	
\\	君の両親だって本当は叱りたくないけど、叱っているんだよ。愛の証だよ。	
\\	(あかし) 
\\	証	
\\	遠足	
\\	えんそく	
\\	する 
\\	学校の遠足で、動物園に行きました。	
\\	春になったら、海に遠足しようよ。	
\\	遠足で阪神甲子園球場に野球観戦に行って、阪神タイガースを応援しました。	
\\	遠, 足	
\\	生存	
\\	せいぞん	
\\	われわれの肉体は、生存のためにつねに水分を必要としている。	
\\	その世界では、きょ人たちが人るい生存のきょういとなっていた。	
\\	は、生存者を求めて、夜通し
\\	オフィスの中を捜索した。	
\\	生, 存	
\\	急行	
\\	きゅうこう	
\\	する 
\\	の 
\\	けいさつは、はんざいげん場へ急行しました。	
\\	私にやさしくしてくれた人たちに、急行便でお礼の品を送りました。	
\\	あーーーしまったーーーー。急行列車が行ってしまった。ってことは私の遅刻も確定だ。コウイチから受ける処罰がすんごく怖いんですけど。	
\\	急行 
\\	急, 行	
\\	外面	
\\	がいめん, そとづら	
\\	ビエトは外面はいい人そうですが、実はとてもおそろしい一面を持っています。	
\\	外面だけがいい男性とけっこんし、外面は立ぱだけど内面はボロボロの中古住たくを買ってしまい、今とても苦労しているんです。	
\\	彼の外面に騙されない方がいいですよ。私がインストラクターのことが嫌いだったのは、上司にはいい顔をして陰で後輩のことを小突き回していたからです。	
\\	そとづら 
\\	外, 面	
\\	顔面	
\\	がんめん	
\\	の 
\\	トーフグの社員たちは、みんな顔面へんさちが高い。	
\\	ストレスが原因の顔面神けいつうになやまされています。	
\\	私はアスファルトに顔面を強打した。	
\\	顔 
\\	(がん). 
\\	(がんめん). 
\\	顔, 面	
\\	不合格	
\\	ふごうかく	
\\	カンニングしたのがバレて不合格になってしまった。	
\\	本命の公立大学の入学しけんは不合格だったが、すべり止めで受けた私立大学に合格しました。	
\\	あんなに簡単な試験が不合格だったなんて本当に恥ずかしいよ。穴があったら入りたいぐらいだ。	
\\	合格する 
\\	合格, 
\\	不 
\\	不, 合, 格	
\\	遠い	
\\	とおい	い 
\\	コウイチのアパートはトーフグオフィスから三時間ほど行った、遠い林の中にあります。	
\\	あの新築マンション、とてもよかったんですけど、駅から遠すぎて買うのをあきらめました。	
\\	東京は、ポートランドからどれくらい遠いんですか?歩くにはちょっと遠すぎますか?	
\\	い 
\\	(とお)! 
\\	遠	
\\	世辞	
\\	せじ	
\\	コウイチは心にもないお世辞を言うのが下手です。	
\\	お世辞で日本語が上手だと言われるのは好きではありません。	
\\	「わぉ!今夜の君はとてもきれいだね。まるで、お姫様みたいだよ。」「パパ、お世辞を言っても何も出ないわよ。」	
\\	世, 辞	
\\	折り目	
\\	おりめ	
\\	このズボンには、折り目がなかったということを、ここに証明いたします。	
\\	折り紙を半分に折って、折り目をつけてからまた開いてください。	
\\	ズボンには折り目を付けてもらえますか?	
\\	折る 
\\	折り. 
\\	目, 
\\	折, 目	
\\	前面	
\\	ぜんめん	
\\	ビエトはいつもいい人さを前面に出している。	
\\	前面も後面も両方とも緑色の箱をさがしてください。	
\\	手紙や小包を送る時には、いつも前面に
\\	のステッカーを貼ります。	
\\	外面 
\\	前, 面	
\\	基本的	
\\	きほんてき	な 
\\	基本的に政治に関心がないんです。	
\\	上司に、基本的な事も知らない役立たずだと言われました。	
\\	どのくらい頻繁に自分の超基本的な間違いにびっくりしますか?	
\\	(基本) 
\\	(的) 
\\	基本 
\\	基, 本, 的	
\\	ゴミ箱	
\\	ごみばこ, ゴミばこ	
\\	ゴミ箱に、新せんな箱づめの魚がすてられていました。	
\\	そのゴミ箱の中にけっこん指輪があるかもしれないというかすかな希望がありました。	
\\	ゴミ箱のゴミを捨てる時のあなたってとってもセクシーだと思うのよね。だから、ちょっと私のためにゴミ箱のゴミを捨ててくれないかな?	
\\	"ゴミ 
\\	はこ 
\\	ばこ.	箱	
\\	園	
\\	えん, その	
\\	保育園を建てる前に、労働者たちが地面を固めました。	
\\	リンゴ農園でリンゴを箱づめするアルバイトをしていました。	
\\	私は植物園の近くに住んでいます。	
\\	園	
\\	門	
\\	もん	
\\	私たちの家は、茶色い門のある家のすぐとなりです。	
\\	あそこの大きい門をくぐると神でんが見えるよ。	
\\	門には鍵がかかっているんだ。	
\\	門	
\\	幸運	
\\	こううん	
\\	な 
\\	幸運になるおまじないよ。	
\\	このフォーチュンクッキーには、だれかの代わりに引っこしの荷ときをしてくれたら、あなたに幸運がおとずれるでしょう、って書いてあるよ。	
\\	コウイチの名前の意味は、一番幸運な人っていう意味なんじゃないかとすいそくしてるんだけど、どう思う?	
\\	幸, 運	
\\	政府	
\\	せいふ	
\\	アメリカ政府は、国きょうのぼうびを固めなくてはいけないと言っています。	
\\	政府は、自分たちのひみつは固く守るが、国民との約束はちっとも守らない。	
\\	私達の記事が政府から出版を禁止されているなんて信じられません。
\\	には政府を批判する権利があります。	
\\	政, 府	
\\	卒業式	
\\	そつぎょうしき	
\\	卒業式の後で、橋の上で一しょに写真をとろうよ!	
\\	はあ?卒業式にビヨンセが来て歌を歌ってくれた?まさか!冗談でしょう?	
\\	卒業式の後の旅行超楽しみだよ〜!ゆっくり露天風呂に入って、山の幸をいっぱい食べ尽くそうね。ところで、念のため確認だけど、宿の予約してくれたんだよね?	
\\	卒業 
\\	卒業 
\\	卒, 業, 式	
\\	食料品	
\\	しょくりょうひん	
\\	食料品コーナーで働いている人たちが、ちん上げを要求しているらしいよ。	
\\	「サーモン、万が一にそなえて、食料品を買っておかなくちゃ。」「フグ、買わなくても大丈夫よ!食料品ならここにあるじゃない。」「え、どこに?」「食料にこまったら共食いすればいいだけじゃない。」	
\\	食料品売場はどこにありますか。	
\\	食, 料, 品	
\\	地面	
\\	じめん	
\\	の 
\\	海の中じゃなくて地面の中に住めたらいいのに、とフグは言いました。	
\\	家の前の庭は草が生えていないので、夏の間は地面がとても熱くなります。	
\\	コウイチはフグを優しく撫でてから、地面に降ろしてあげました。	
\\	面 
\\	地 
\\	じ, 
\\	地, 面	
\\	保証	
\\	ほしょう	
\\	する 
\\	うまく組しきを固められる保証はありませんよ。	
\\	私はワニカニのライフタイムメンバーですが、トーフグが五年後もあるなんて保証はないから本当にライフタイムでワニカニが使えるなんて思っていませんよ。	
\\	恐らく灰色熊が近寄ってくるかと思いますが、とても安全なのでご安心ください。ご保証致します。	
\\	保 
\\	保, 証	
\\	阪神	
\\	はんしん	
\\	阪神電車カラーのゼリーを氷でひやして固めているところです。	
\\	今度、阪神地区に新築の家を買おうと思ってるんですが、相場はいくらぐらいなんですか。	
\\	うわぁ、阪神タイガースのファンたち、恵比寿橋からあの汚い川にダイブしたんだって?すげぇな、あいつらの頭大丈夫かよ!	
\\	大阪 
\\	神戸 
\\	阪神! 
\\	阪, 神	
\\	幸福	
\\	こうふく	
\\	な 
\\	幸福の科学というしゅう教団体について調べてみました。	
\\	トーフグに幸福をもたらすために、コウイチはじゅう業員全員に幸福のフグのおまもりを一万ドルで買わせようとしました。	
\\	彼女は幸運の星のもとに生まれたため、幸福すぎて毎日涙が止まらないんだよ。	
\\	幸, 福	
\\	幸せ	
\\	しあわせ	
\\	な 
\\	末ながくお幸せにね。	
\\	ビエトは、うらないしから今年中に幸せがおとずれると言われました。	
\\	「私はお金で幸せは買えると思います。何故なら、私は花火を見ていると幸せな気持ちになるからです。もし私が大金持ちだったら、毎日花火を楽しんでずっと幸せな気持ちでいられることができます。」彼女は幸せそうな口調でそう言った。	
\\	幸. 
\\	(しあわせ) 
\\	幸	
\\	記念日	
\\	きねんび	
\\	記念日をわすれるなんて、最低よ!	
\\	今年のけっこん記念日には、夫から全自動せんたくきをもらいました。	
\\	どく立記念日は、フリマで服を売っています。	
\\	ひ 
\\	び.	記, 念, 日	
\\	仏像	
\\	ぶつぞう	
\\	日本で仏像を買ったら、申告しないといけませんか?	
\\	仏像はよく金色に作られています。	
\\	誰かがうちのオフィスに毎月仏像を一体ずつ送ってくるんです。心当たりのある方は、どうか名乗り出て下さい。	
\\	仏, 像	
\\	不完全	
\\	ふかんぜん	
\\	な 
\\	不完全な形のあいってどんなものですか。	
\\	完全なものよりも不完全なものの方が人情味があってよくない?	
\\	勝ち試合が雨で中止になり、
\\	ソフトボールチームのシーズンが不完全に終わった。	
\\	完全 
\\	完全.	不, 完, 全	
\\	不治	
\\	ふじ, ふち	
\\	の 
\\	不治のがんで、よ命三ヶ月だとせん告されました。	
\\	私の息子は、不治ののうしっかんと戦っています。	
\\	コウイチは一度不治の病に倒れ、「俺たちは必ず、また会えるよ」とビエトに言い残し亡くなったのですが、言葉通り次の日にはオフィスに出勤し、以前と全く同じように業務をこなしました。	
\\	不, 治	
\\	保険	
\\	ほけん	
\\	の 
\\	どうして日本では、女性が保険の外交をすることが多いんですか?	
\\	保険にか入するのに、出生証明書は必要ですか?	
\\	しまった。また家に健康保険証を忘れてきちゃった。	
\\	保険?	
\\	保, 険	
\\	面白い	
\\	おもしろい	い 
\\	あいつは面白いやつだけど、ちょっとしたことですぐ切れるのがネックだよな。	
\\	このマンガは、ちょっと下品だけどとても面白いんです。	
\\	あはははは!超ウケるんだけど。今まで聞いた中で、一番面白い話だわ。	
\\	面 
\\	(白い 
\\	白い). 
\\	(おも) 
\\	面, 白	
\\	冗談	
\\	じょうだん	
\\	の 
\\	つまらない冗談に係わってるひまはないんだよ。	
\\	トーフグはポッドキャストで冗談ばかり言っているが、時々冗談の通じない人からいかりのメールをもらうことがあります。	
\\	冗談はヤメてよ!何でフレンチレストランなんて行きたいの?私達の柄じゃないっしょ。飲めればどこでもいいじゃん?違う?	
\\	冗, 談	
\\	関西	
\\	かんさい	
\\	マイケルは、関西一の芸人としての地位を固めつつある。	
\\	関東の人と関西の人って折が合わないって聞いたんですが、本当ですか。	
\\	こいつは関西は完全に初心者だから、俺らが案内してやらないとな。	
\\	西 
\\	さい 
\\	せい. 
\\	(さい), 
\\	関, 西	
\\	図書館	
\\	としょかん	
\\	図書館から、電子書せきをかりることもできるって知っていましたか。	
\\	この図書館のトイレは、ここを真っ直ぐ行ったところの一番おくにあります。	
\\	「年末年始のお休みいつからいつまでなの?一緒に図書館に行こうよ。」「十二月三十日から五日間だけど、図書館も閉まってるはずだよ。」	
\\	書館 
\\	図 
\\	(と). 
\\	図, 書, 館	
\\	特急	
\\	とっきゅう	
\\	ノリで特急に乗ることに決めました。	
\\	けんこうしんだんで、心電図に異常があったので、月曜日に特急に乗って大阪の大学病院まで行って、せいみつけんさを受けなければいけません。	
\\	特急券を先に買っておいたので、今出れば三時五十分の特急に間に合うよ。	
\\	とく 
\\	とっ.	特, 急	
\\	荷物	
\\	にもつ	
\\	ぜい関で荷物を調べられないか心配です。	
\\	荷物でいっぱいのベビーカーが公園におきっ放しになっていました。	
\\	「誰かが車から荷物を降ろさないといけないんだけど…。」 「わかったから、それ以上言わなくていいよ。」	
\\	物 
\\	もつ, 
\\	荷, 物	
\\	仮面	
\\	かめん	
\\	ハロウィンの日の夜、仮面ライダーのコスプレをしながら、チョコレートを一箱平らげたビエトを見ました。	
\\	わぁ、びっくりした!何で仮面なんてかぶってるの?	
\\	ヴェネツィアの仮面舞踏会に行って白馬の王子様に会うことが私の夢です。	
\\	仮, 面	
\\	明治	
\\	めいじ	
\\	明治生まれのそ母は、生前は仏教の教えをちゅう実に守っていました。	
\\	明治い新についてのろん文を書いています。	
\\	明治時代の生活を描いたドラマを見ているんですが、当時は一万円が大金だったということに驚きました。	
\\	めいじ. 
\\	明, 治	
\\	商品	
\\	しょうひん	
\\	ワニカニは、トーフグのヒット商品です。	
\\	あの会社の商品にはいつも保証がついてるから、私はそこで買うことが多いかな。	
\\	この新商品は大成功を収めると信じているんだ。そして、君にはこの商品の店内キャンペーンでの販売促進に取り組んでもらいたいと思っているんだ。頼りにしているよ。	
\\	商, 品	
\\	美術館	
\\	びじゅつかん	
\\	その美術館にかざられていたノアの箱ぶねの絵が好きでした。	
\\	母が残してくれた指輪を、美術館にき付しました。	
\\	「どうやって美術館のツアーガイドになれるの?芸術の専門家か何かじゃなくちゃいけないんじゃないの?」「いいや、そうでもないよ。 美術館が情報を全部くれるから、まぁ、台詞を覚えるようなもんだよ。」	
\\	美術 
\\	美, 術, 館	
\\	折り紙	
\\	おりがみ	
\\	今日のコウイチの弁当箱の中には、なぜか食べ物の代わりに折り紙が入っていました。	
\\	私はプロの折り紙アーティストです。	
\\	私のことをこの数年間支えてくれたみなさん、本当に有難うございます。お陰様で、ついに折り紙で河豚が折れるようになりました。	
\\	折る 
\\	折る 
\\	紙. 
\\	折, 紙	
\\	大阪	
\\	おおさか	
\\	大阪では、せいげん速度を守って走る車は少ないと聞いたんですが、本当ですか?	
\\	大阪旅行が楽しみすぎて、勉強がすっかりお留守になってしまっていました。	
\\	あいにくですが、来週の水曜日はちょっと都合がつかないんです。代わりに来週の木曜日に大阪に行くってのはどうでしょうか?	
\\	大 
\\	おお 
\\	おおきい. 
\\	阪 
\\	坂, 
\\	坂 
\\	おおさか. 
\\	おお 
\\	大, 阪	
\\	写真	
\\	しゃしん	
\\	コウイチの高校時代の写真が流出してしまいました。	
\\	この写真、見て!お父さん、昔のまんま。変わったことと言えば、ちょっと太ったことくらいじゃない?	
\\	ご依頼のフグのヌード写真集を同封いたします。	
\\	写, 真	
\\	真っ黒	
\\	まっくろ	
\\	な 
\\	コウイチは真っ黒なニンニクが好きで、よくしたから血が出るまで食べています。	
\\	ビエトはハワイのビーチで日やけして帰ってきたんですが、まるでこげたトーストのようにものの見事に真っ黒でした。	
\\	やばい、超絶に可愛い真っ黒なワンピース見つけちゃったんだけど。	
\\	真 
\\	(ま). 
\\	真, 黒	
\\	政治	
\\	せいじ	
\\	ついに、大学で政治を学ぶという私の長年の希望がかないました。	
\\	サーモンの政治に対する考えはとても保守的です。	
\\	ワシントン
\\	で起きている政治騒動なんてお構いなしに、トーフグは共和党と民主党の両地域で雇用を生み出している。	
\\	政, 治	
\\	保存	
\\	ほぞん	
\\	する 
\\	コウイチのデスクのかぎのかかった引き出しに、重要文書が保存されています。	
\\	母乳は、れいとうすれば保存がききます。	
\\	「このベーコンは、着色保存料を一切使用しておりません。」「ってことは、本当のベーコンではありませんね。」	
\\	保, 存	
\\	証明	
\\	しょうめい	
\\	する 
\\	の 
\\	クリステンが正真正めいの箱入りむすめであることが証明された。	
\\	何か身分を証明できる物は持っていますか?	
\\	テキストフグとワニカニの各コースが修了するごとに、修了証明書のようなものが受け取れた方がいいですか?	
\\	証, 明	
\\	治安	
\\	ちあん	
\\	この辺りは治安も悪くないけど、あの辺りは治安が良くないからね。	
\\	ビエトの親分は、治安ぼうがいでたいほされた。	
\\	こないだ、酔っ払ったコウイチが、「俺は実はスーパーマンなんだ。この辺りの治安を守っているのはこの俺様なんだ」って豪語していたよ。	
\\	治, 安	
\\	公園	
\\	こうえん	
\\	この通りを真っ直ぐ行くと、公園があります。	
\\	公園ですなあそびをしていたら、手が真っ黒になってしまった。	
\\	「フグと一緒にいった公園、どうだった?」「正直今までに行った公園の中で一番楽しかったよ。」	
\\	公, 園	
\\	危険	
\\	きけん	
\\	な 
\\	まだ危険なじょうきょうだが、回ふくの希望はある。	
\\	駅ではよく、「危険ですから黄色い線の内がわまでお下がりください」というアナウンスが流れています。	
\\	我が社が現在債務不履行の危険性に瀕しているのは真実だが、リスクを覚悟のうえで、信頼に基づいて行動し続けるしかないと思うんだ。	
\\	危, 険	
\\	場面	
\\	ばめん	
\\	の 
\\	つまが荷物をまとめて出ていく場面が自分にかさなりつらくなりました。	
\\	このフレーズは、ビジネスの場面で有効です。	
\\	最後の場面で泣いてしまいました。	
\\	場, 面	
\\	画面	
\\	がめん	
\\	ずっとコンピューターの画面ばかり見ていると目が悪くなるよ。	
\\	最近、よく犬のエッチな画像が自動で画面に出てくるからみんなこまっているんだよね。	
\\	変なんだよ。ビエトはいつも通り画面に釘付けになって何時間も座ったままでいるんだけど、肝心の画面は今朝から真っ黒なんだよね。	
\\	画, 面	
\\	笑顔	
\\	えがお	
\\	する 
\\	あの笑顔がコウイチのせい実さを証明しているよね。	
\\	この国の王子は笑顔で国民を殺すとてもひどい男です。	
\\	この笑顔の素敵な男は俺の連れで、コウイチっていうの。	
\\	笑顔
\\	(え).
\\	笑, 顔	
\\	専門	
\\	せんもん	
\\	の 
\\	それは私の専門外です。	
\\	私の姉は、ペット専門の目医者です。	
\\	私の専門は天文学です。新しい星を発見したいです。	
\\	専, 門	
\\	留学	
\\	りゅうがく	
\\	する 
\\	「兄は親の言いつけを守らず、留学してしまいました。」「留学先はどこなんですか?」「アメリカです。」	
\\	留学先で、日本語を自然にしゃべる練習をたくさんしました。	
\\	約束を来週の日曜日に変更していただけませんか?土曜日に、留学の書類を取りに日本総領事館に行かなければならなくなってしまって。	
\\	留 
\\	りゅう 
\\	(りゅう) 
\\	留, 学	
\\	急死	
\\	きゅうし	
\\	する 
\\	父が急死した時、私は受けん生でした。	
\\	昨日、上司が、のう出血のため急死しました。	
\\	彼とはよく喧嘩になっちゃって、それが本当に嫌だったんだけど、今では喧嘩もできなくなっちゃってすっごく悲しいよ。まさか心臓発作で急死しちゃうなんて思ってもみなかったからさ。	
\\	急, 死	
\\	急に	
\\	きゅうに	
\\	急に空からマッチ箱が一箱ふってきたんです。	
\\	えっ、クリステン!急にそんなことを言われても、困るよ...。	
\\	ファミリー・ガイは、急に馬鹿げた回想シーンを使うことで良く知られている。	
\\	急	
\\	日光浴	
\\	にっこうよく	
\\	する 
\\	トトロの大きなおなかの上に乗って日光浴がしたいです。	
\\	私は平日は日光浴だけで入浴はしない主ぎなんです。	
\\	トーフグのワンコは、ベランダで日光浴をするのが大好きです。	
\\	日光 
\\	日, 光, 浴	
\\	荷札	
\\	にふだ	
\\	旅行かばんに付ける荷札をさがしています。	
\\	荷札の書き方を教えてください。	
\\	加茂錦の「荷札酒」の荷札を集めるのが趣味なんです。	
\\	荷, 札	
\\	証言	
\\	しょうげん	
\\	する 
\\	全トーフグ社員が、コウイチがビエトのミートボールを食べるのを見たと証言しました。	
\\	マイケルだけは、コウイチに不利な証言はしたくないと言って、証言を拒否しました。	
\\	フグが先生の林檎に河豚毒を塗ったと疑われた時、クラスメート全員が彼は無実だと証言した。	
\\	証, 言	
\\	教科書	
\\	きょうかしょ	
\\	その日本語の教科書は高すぎて買えません。	
\\	教科書はシェアしないで一人一さつ買ってくださいね。	
\\	教科書を持ってくるのを忘れたせいで、放課後一時間学校に残されました。	
\\	教, 科, 書	
\\	浴びる	
\\	あびる	
\\	晩飯の前に、ひとっ風ろ浴びてくるよ。	
\\	放しゃ能を浴びるのが怖くて、この町ににげてきました。	
\\	フグは私に親友の天然鯛を紹介してくれました。鯛君は、深海に住んでいるので日光をあまり浴びないんだと言っていました。	
\\	う 
\\	(あ) 
\\	浴	
\\	関する	
\\	かんする	する 
\\	このけんに関しては、妥協のよ地はありません。	
\\	クリステンは、日本文学に関する研究を行っています。	
\\	ビエトは、
\\	のオフィスで働いている時に、アジトでの発砲に関する電話を受けた。	
\\	関 
\\	する 
\\	関	
\\	係わる	
\\	かかわる	
\\	ビエトがやばいそしきと係わっているのは知っているけど、そのはんざいに係わっていたとはやっぱり思えないんだよな。	
\\	あのれん中には係わらない方がいいよ。	
\\	もし毒抜きをせずに河豚を食べたら、命に係わりますよ。	
\\	(かか) 
\\	係	
\\	折れる	
\\	おれる	
\\	そんなことを言われると、心が折れるよ。	
\\	重い荷物をかたにかついで運んでいたら、かたの骨が折れてしまいました。	
\\	うちのチューリップ、毎年花が咲いた後、根元からポッキリ折れるんですよね。	
\\	折る 
\\	折る, 
\\	折	
\\	取れる	
\\	とれる	
\\	ボタンが取れちゃった。	
\\	このしみは取れないかもしれないね。	
\\	ビジネス街にオフィスを設けるとなると賃料はかなり高くつきますが、我々のイメージにとって重要な事でもありますし、一年ぐらいで元は取れるのではないかと思うんです。	
\\	う 
\\	取る.	取	
\\	妥協する	
\\	だきょうする	する 
\\	十万円で妥協しました。	
\\	コウイチとビエトは、マクドナルドに行くことで妥協しました。	
\\	「ねぇ、サーモン。あんたとフグが一緒に住み始めた時、やっぱり二人とも妥協しなきゃいけなかった?」「うーん。私はしなかったけど、フグはね!」	
\\	妥協 
\\	妥協, 
\\	妥, 協	
\\	治す	
\\	なおす	
\\	生理つうを治すにはアスピリンを飲むのがいいよ。	
\\	ハーブりょう法でがんを治すことについて、どう思われますか?	
\\	ビエトだけが、コウイチが怪我や病気を治すことができる不思議な力を持っていることを知っている。	
\\	う 
\\	治る 
\\	(す) 
\\	(なお). 
\\	直す, 
\\	なおす 
\\	治	
\\	卒業する	
\\	そつぎょうする	する 
\\	大学を卒業した後、ぶ事に第一希望の会社にしゅうしょくすることができました。	
\\	出せき日数が足りなくて高校を卒業できないだって?冗談だろう?	
\\	私の兄は障害を乗り越え、大学を卒業しました。私は兄をとても誇りに思います。大学卒業は彼の決意と努力の賜物です。	
\\	卒業 
\\	卒業.	卒, 業	
\\	急ぐ	
\\	いそぐ	
\\	コウイチは急いでビエトに書きおきを残した後、トイレに一時間こもった。	
\\	ビエトがか労でたおれたと聞いて、コウイチは病院へ急ぎました。	
\\	「もう!急いでよ!遅れちゃうわ。」「今起きたから、すぐ行くわ。」	
\\	う 
\\	(いそ).	急	
\\	存じる	
\\	ぞんじる	
\\	社員たちがしょうきゅうを要求していることは存じておりました。	
\\	つまが弟にお金の返さいを要求していたことは存じませんでした。	
\\	年末でお忙しいとは存じますが、この記事を読んで頂ければ嬉しく存じます。	
\\	存じる 
\\	存	
\\	東京弁	
\\	とうきょうべん	
\\	東京弁は共通語とほとんど同じですよ。	
\\	東京弁でしゃべってほしいというコウイチの要求をしりぞけました。	
\\	「ねえ、無料で東京弁のレッスンを受けたいんだけど。」「コウイチに聞けよ。アイツは何でもするぞ。」	
\\	東京 
\\	弁 
\\	東京 
\\	東, 京, 弁	
\\	笑う	
\\	わらう	
\\	その冒険家は、後世に名を残したいんだと笑って言った。	
\\	クリステンは、笑いながらコウイチに謝ざいを要求しました。	
\\	「昨夜、フグとデートをしたの。」「どうだった?」「楽しかった。めちゃくちゃ笑ったわ。」	
\\	う 
\\	笑	
\\	残す	
\\	のこす	
\\	このクリーミーチキンポットパイ、明日に残しておかない?	
\\	コウイチは、ビエトに多がくのしゃっ金を残して死んでしまいました。	
\\	昨夜
\\	オフィスに泥棒が入り、
\\	ステッカーを何枚か盗んで行ったのですが、馬鹿な泥棒たちは、指紋を残すという初歩的な過ちも犯しました。	
\\	う 
\\	す. 
\\	残る. 
\\	残	
\\	固める	
\\	かためる	
\\	ねん土は熱で固めることができるよ。	
\\	ビエトは毎朝一時間かけてリーゼントをワックスで固めます。	
\\	コウイチは、日本に軍隊を派遣する決意を固めた。	
\\	う 
\\	固まる, 
\\	(める) 
\\	固 (かた) 
\\	(かた) 
\\	固	
\\	約束する	
\\	やくそくする	する 
\\	「大きくなったらけっこんしよう。」「約束する?」「うん、約束するよ。」	
\\	ぜっ対にだれにも言わないって、約束してくれる?	
\\	二人は、「指切拳万、嘘ついたら針千本呑ます」という歌を歌いながら指切りをし、日曜日に一緒にホットケーキを作ろうと約束した。	
\\	約束 
\\	する 
\\	約束 
\\	する 
\\	約束 
\\	約, 束	
\\	守る	
\\	まもる	
\\	分からないかもしれませんが、かれはいつもあなたのそばにいて、あなたのことを守ってくれているんですよ。	
\\	コウイチの犬は、コウイチを危険から守りました。	
\\	いいか、わが愛犬よ、よく聞け。家族を守るのは、お前の責任だ。	
\\	う 
\\	(まも).	守	
\\	辞める	
\\	やめる	
\\	医者のアドバイスにしたがって、仕事を辞めました。	
\\	マイケルがへん集長を辞めたなんて、だれが言ったの?	
\\	フグが会社を辞めるので、お餞別に浮き輪を買おうと思っています。一緒に買いたい人は連絡ください。	
\\	う 
\\	(や) 
\\	辞	
\\	希望する	
\\	きぼうする	する 
\\	日本への留学を希望しています。	
\\	希望していた大学に入学することができました。	
\\	雨が降っているので、待ち合わせを地下に変更することを希望しますが、どう思われますか?	
\\	希望 
\\	希望. 
\\	希, 望	
\\	取る	
\\	とる	
\\	運転めんきょ証がやっと取れました!	
\\	たなからマグカップを取ってくれない?	
\\	私がこの行動の責任はすべて取ります。本当にごめんなさい。	
\\	う 
\\	(と). 
\\	取	
\\	書く	
\\	かく	
\\	日本人でさえ漢字を書けなくなっている人がふえているので、ワニカニは漢字の読みと意味を重ししています。	
\\	コウイチは、お世辞にも字を書くのが上手だとは言えません。	
\\	よっしゃ!今コウイチのフェイスブックアカウントをハッキングしてやったぜ!これから、コウイチのポストとして、何でも好きなことを書くことができる。へへへ。	
\\	う 
\\	か. 
\\	(か) 
\\	書	
\\	真実	
\\	しんじつ	
\\	な 
\\	の 
\\	トーフグが、あるワニカニユーザーから5万ドルのそんがいばいしょうを要求されているというのは真実ですか?	
\\	真実のあいを見つけたいという私の希望は打ちくだかれてしまいました。	
\\	真実から目を背けないで。結局マザコンだってこと、認めなさいよ。あんたのお母さんはあんたのことを愛しているし、あんたもお母さんのことを愛しているんでしょ。	
\\	真, 実	
\\	不幸	
\\	ふこう	
\\	な 
\\	まるで不幸のどん底のようにいるような顔をしてるけど、大丈夫?	
\\	家族に不幸があったので、明日、明後日会社を休ませてください。	
\\	不幸は仲間を好むってよく言うけど、それって私があなたと付き合っている理由をうまく説明しているよね。	
\\	不, 幸	
\\	残品	
\\	ざんぴん	
\\	残品は半がくでしょ分しました。	
\\	残品はこちらで引き取るので、こちらのお店ではん売させてもらえませんか。	
\\	今夜店を出発する前に残品を数えておきましょう。	
\\	ひん 
\\	ぴん.	残, 品	
\\	急	
\\	きゅう	な 
\\	あの男は、つまと二人の子を残して急に死んでしまったんだよ。	
\\	急な仕事を引き受けちゃったので、今日のおむかえはおそくなります。	
\\	急な依頼ですみません。どうしても断れないお客様からの急ぎの注文だったんです。	
\\	急	
\\	待合室	
\\	まちあいしつ	
\\	今、歯医者の待合室にいます。	
\\	空港には、団体で旅行するグループのための待合室があります。	
\\	数日前、病院の待合室で経済についての本を読んだんだけど、結構良かったよ。	
\\	待合 
\\	室 
\\	待, 合, 室	
\\	証人	
\\	しょうにん	
\\	証人が、ビエトが有ざいであることを証明してくれました。	
\\	私の夫のさいばんでは、夫の口座がある銀行の行員の人たちが証人として出ていしてくれました。	
\\	証人として、遠回しに言わずに率直に君の質問に答えさせてもらうよ。ああ、あのフグが確かに君のプリンを食べていたね。	
\\	証, 人	
\\	動物園	
\\	どうぶつえん	
\\	ヤクザだって動物園に来ることもあるさ。	
\\	動物園で間近で見たフラミンゴは、とてもきれいでした。	
\\	多くのコアラはコアラレトロウィルスというウイルスを先天的に保持しており、ストレスなどを受けると、リンパ腫など死につながる病気を発症してしまいます。ですので、動物園に行かれた際は、カメラのフラッシュや大声を出すなどしないよう、十分気をつけてください。	
\\	動物 
\\	動, 物, 園	
\\	欠席	
\\	けっせき	
\\	する 
\\	今日は薬学はかせは欠席なんですね。	
\\	無だん欠席が三回つづいたので、首にしました。	
\\	大変申し訳ないのですが、急な出張が入ってしまったので、今夜の同窓会は欠席させて頂きます。	
\\	けつ 
\\	けっ 
\\	つ
\\	欠, 席	
\\	是非	
\\	ぜひ	
\\	また是非日本にお来しくださいね。	
\\	トーフグのパーティーには是非出席させていただきたいと思っています。	
\\	の製品に興味がありそうな人を知っていたら、是非私達に知らせてくださいね。	
\\	是, 非	
\\	無茶	
\\	むちゃ	
\\	な 
\\	こんなに車がビュンビュン通っていて横だん歩道もないのに、向こう側へわたるなんて、無茶だよ。	
\\	ビエトは無茶なりょうのあわもりを飲んでいました。	
\\	カレーソースでいっぱいのプールで泳ごうなんて、彼も無茶な決定をしたものだ。	
\\	""無コーヒー
\\	無, 茶	
\\	結局	
\\	けっきょく	
\\	の 
\\	結局、梅干しはつけませんでした。	
\\	目覚まし時計で一たんは目が覚めたが、結局またねむりに落ちてしまいました。	
\\	バレンタインの友チョコを買いに行ったんだけど、結局自分用チョコしか買わなかったわ。	
\\	けつ 
\\	けっ 
\\	つ
\\	結, 局	
\\	愛知県	
\\	あいちけん	
\\	愛知県には、美しい梅林があります。	
\\	愛知県は、そろそろ梅雨入りです。	
\\	フグが日本の愛知県に引っ越した後、サーモンが切ない思いをしていることを、誰一人知らなかった。	
\\	愛, 知, 県	
\\	常に	
\\	つねに	
\\	うちの母は、常に父に愛情を持っているわけではなさそうでした。	
\\	最近は、常に十時にはねていますね。	
\\	自分の親が常に正しいわけではないことを知るのに、時間がかかる子どももいます。	
\\	(つね) 
\\	常	
\\	無名	
\\	むめい	
\\	の 
\\	フグは、無名のまま死にたくないと思っている。	
\\	若手のアーティストの中には、無名でも良い作品を作る人がたくさんいます。	
\\	アリアナ・グランデは自分の才能を少しも鼻にかけず、無名の新人歌手にも優しく接するらしいので、とても好感がもてます。	
\\	無, 名	
\\	建築家	
\\	けんちくか	
\\	お父さん、あんたが建築家になったから、鼻が高いのよ。	
\\	医者は、その建築家をあやまって胃がんだとしんだんしました。	
\\	私の家は有名な建築家によって設計されたんですよ。	
\\	建築 
\\	建築 
\\	建, 築, 家	
\\	兵器	
\\	へいき	
\\	兵器使用の是非について語り合いました。	
\\	化学兵器の研究室で働いていたことがあります。	
\\	核兵器の所在地の移動に関する
\\	メールが、とある
\\	ユーザーから送られてきた。	
\\	兵, 器	
\\	相変わらず	
\\	あいかわらず	
\\	相変わらずかっこいい鼻輪をつけてるね。でも、今日は鼻毛が出てるよ。	
\\	相変わらず面白いことをしてるね。	
\\	ビエトは相変わらず、つぶやくような小声で、ヒドイことを言いまくってるね。	
\\	相, 変	
\\	原子	
\\	げんし	
\\	しょう来は、原子エネルギーの勉強をしたいと思っています。	
\\	この原子の原子番号と原子記号は何だったっけ?	
\\	老朽化した原子力発電所は、多くの国々にとって悩みの種です。	
\\	原, 子	
\\	出席	
\\	しゅっせき	
\\	する 
\\	申しわけありませんが、出席は不定期になると思います。	
\\	コウイチのたん生日パーティーには、出席しない予定です。	
\\	別の便に乗せていただけますか?どうしても、主要国首脳会議に出席しなくてはならないんです。	
\\	しゅつ 
\\	しゅっ, 
\\	つ
\\	出, 席	
\\	意識	
\\	いしき	
\\	する 
\\	道をわたる前に、車が来てないことをかくにんするよう意識してください。	
\\	果物や野さいはけんこうに良いので、意識して食べるようにしています。	
\\	サーモンは、フグが他の女の子に興味を示そうものなら、意識を失う。	
\\	意, 識	
\\	不味い	
\\	まずい	い 
\\	なんか不味そうなサンドイッチだね。	
\\	あの字はちょっと不味いんじゃない?	
\\	こんなに不味い味噌汁を飲んだことはありません。	
\\	美味しい 
\\	美 
\\	不 
\\	まずい. 
\\	(まず) 
\\	不, 味	
\\	紀元前	
\\	きげんぜん	
\\	きょうりゅうがいたのは、紀元前何年ごろまでですか。	
\\	今日のコウイチは、自分は紀元前の古代ローマからタイムスリップしてきた男だって言いはってきかないんです。	
\\	もしタイムマシーンがあったなら、紀元前三千年紀に行ってみたいな。	
\\	紀元後 
\\	(前) 
\\	紀元後 
\\	前.	紀, 元, 前	
\\	両側	
\\	りょうがわ	
\\	の 
\\	両側の耳がかゆくてむずむずします。	
\\	二ひきの犬が、両側から私に鼻をすりつけてきました。	
\\	両側に女の子をはべらせたビエトの写真がパパラッチに売りつけられた。	
\\	両, 側	
\\	底	
\\	そこ	
\\	くつの底にガムがついてるよ。	
\\	底無しぬまに泳ぎに行くなんて、無茶だよ。	
\\	トーフグのお陰で、日本語がペラペラになりました。心の底から感謝しています。	
\\	底	
\\	心底	
\\	しんそこ	
\\	コウイチの不治の病が治って、心底うれしいよ。	
\\	コウイチには心底からきらわれていると思います。	
\\	新しい部署、優しい人がいっぱいで、心底ホッとしたよ。	
\\	心底? 
\\	心, 底	
\\	朝ご飯	
\\	あさごはん	
\\	朝ご飯の美味しそうなにおいに鼻がひくひくしました。	
\\	わが家の朝ご飯は、ご飯と味そしる、玉子やきに干し魚と決まっています。	
\\	「ねぇ、さっきコウイチにサプライズ朝ごはんパーティーの事を教えたでしょ?」「えっ!サプライズだったなんて知らなかった!」	
\\	晩ご飯 
\\	朝 
\\	あさ 
\\	ご飯 
\\	ごはん.	朝, 飯	
\\	イギリス人	
\\	いぎりすじん, イギリスじん	
\\	あの鼻すじの通ったイギリス人は、鼻にかかった声で話します。	
\\	このイギリス人は、梅ずを飲んだことがないそうです。	
\\	知らないイギリス人から友達申請が来たよ。	
\\	英国 
\\	英国 
\\	人 
\\	じん 
\\	人.	人	
\\	建物	
\\	たてもの	
\\	すてきなデザインの建物ですね。	
\\	建てられてからまだ一年しかたってないのに、あの建物はもうずい分よごれているね。	
\\	「あなたとその建物を一緒に見に行きたいと思うのですが、いつがご都合よろしいですか?」「ええっと、あなたのご都合に合わせますよ。」	
\\	建てる 
\\	建てる (たてる). 
\\	たてもの. 
\\	建物 
\\	建, 物	
\\	食堂	
\\	しょくどう	
\\	この食堂の店主は、古くさい男で、男は家庭のことに干渉すべきではないと主ちょうするんです。	
\\	さっきトーフグ食堂に行ったんだけど、無名の歌手が無料のワンマンライブをしてたよ。	
\\	お昼ごはん、食堂で食べない?もしよければ、私二限目のクラスが無いから、席取っておけるよ。	
\\	食, 堂	
\\	原因	
\\	げんいん	
\\	する 
\\	愛情からではなく金のためにけっこんしたのが原因で、けんかがたえず、結局りこんしてしまいました。	
\\	原因不明の山火事で、町は大変なことになっています。	
\\	心配ないよ。ビエトがすでに不具合の原因を突き止めて、現在修復作業中だよ。	
\\	原, 因	
\\	喜劇	
\\	きげき	
\\	の 
\\	その映画は悲しすぎて、喜劇というよりは悲劇だ。	
\\	喜劇王チャップリンのドキュメンタリーを観て自分もお笑いの道を目指そうと思いました。	
\\	すごく暑かったので、喜劇を観ながらずっと自分をうちわで扇いでいました。	
\\	喜, 劇	
\\	常識	
\\	じょうしき	
\\	常識のある人たちばかりで、安心しました。	
\\	常識テストを受けてみたが、結果はさんざんだった。	
\\	常識は脇に置いて、少しだけ既存の考えに囚われずに考えてみませんか?	
\\	常, 識	
\\	完結	
\\	かんけつ	
\\	する 
\\	この小説は、次号で完結する予定です。	
\\	しつ問しようと思ってたけど、勝手に自こ完結しちゃったんだよね。	
\\	この漫画はまだ完結していません。	
\\	完, 結	
\\	自覚	
\\	じかく	
\\	する 
\\	ぎ理の母から、母親としての自覚が無いって言われて、すっごくはらが立ちました。	
\\	自分がすばらしいワニカニユーザーなんだという自覚を持って、日本語の勉強にはげんでください。	
\\	禁煙して三ヶ月になるけど、まだ時々煙草を吸いたいという衝動に駆られるね。興味深いことに、煙草が吸いたくなる時、自分ではそれをあまり自覚してないんだ。自分の人差し指を噛みしめているのに気づいて、ようやく自覚するんだよ。	
\\	自, 覚	
\\	泣き虫	
\\	なきむし	
\\	泣き虫コウちゃーん、ほら、コウちゃんの好きなトンネルだよーっ!	
\\	コウイチだけがビエトが泣き虫だということを知っている。	
\\	娘の高熱が中々下がりません。普段は泣き虫なんですが、今は泣く元気も無いようで、心配です。	
\\	泣く 
\\	泣き 
\\	虫 
\\	泣く 
\\	虫. 
\\	泣, 虫	
\\	泣き声	
\\	なきごえ	
\\	つまの泣き声を聞いて、ようやくつらい一日をすごしたんだなということが分かりました。	
\\	コウイチがビエトに泣き声で話しをしてるのを聞いてしまったんだよね。	
\\	赤ん坊の泣き声を聞くのは耐え難いことだというのは分かるのですが、だからって赤ちゃんを電車に乗せない方がいいってことにはならないと思うのですが?	
\\	泣く 
\\	声. 
\\	泣, 声	
\\	一昨日	
\\	おととい, いっさくじつ, おとつい	
\\	今さらですが一昨日の5月5日はルフィのたん生日でした。	
\\	一昨日だったら空いてたんだけど、今月は他の日は全てうまってしまってるんだよね。	
\\	一昨日、夫の不倫現場を見てしまったんです。	
\\	昨日 
\\	(おととい). 
\\	一, 昨, 日	
\\	詳しい	
\\	くわしい	い 
\\	ずい分シメサバについて詳しいんですね。	
\\	すみませんが、私はじゅうについて詳しくないので下手に触らない方がいいと思います。	
\\	「おお、そういえば、あいつら結婚するって聞いた?」「おお、マジかよ? 詳しく教えてくれよ。」	
\\	い 
\\	(くわ)?? 
\\	詳	
\\	詳細	
\\	しょうさい	
\\	な 
\\	このユーザーは、いつも詳細なフィードバックを送ってくれます。	
\\	コウイチさんは貧しい家に生まれたと聞いたんですが、もう少し詳細を教えてもらえませんか?	
\\	詳細を調べている時間がありません。	
\\	細かい 
\\	詳, 細	
\\	晩ご飯	
\\	ばんごはん	
\\	今日の晩ご飯のオカズは、天日干しで作ったカレイの干物です。	
\\	晩ご飯の前に、ビールを一気にグビッと飲み干しました。	
\\	晩ご飯の後の散らかった台所を見て、コウイチは気が狂いそうだった。	
\\	晩, 飯	
\\	外交官	
\\	がいこうかん	
\\	あの外交官は、どうしてけっこんして身を固めないんだろう。	
\\	私のおじは外交官で、世界に数台しかないギターを他の国の外交官からプレゼントされました。	
\\	私は外交官で、どうしても今日日本に飛ばなければいけないんです。空席待ちをさせてもらえますか?	
\\	外, 交, 官	
\\	劇	
\\	げき	
\\	オペラ劇を観劇しに行く天皇へい下をお見送りしました。	
\\	この劇の第2幕第3場で馬の役をしているのがオレの兄貴なんだ。	
\\	隣の席の人が劇の最中にずっと指をポキポキ鳴らしてて、すっごく苛々しました。	
\\	劇	
\\	劇的	
\\	げきてき	
\\	な 
\\	ダイエットを始めて三ヶ月で体型が劇的に変化しました。	
\\	劇的とまではいかないけれど、かなりの売り上げがあったことは間違いないようです。	
\\	応援してるチームが劇的なさよならホームランで勝利を納めた時、思わずウルッときてしまいました。	
\\	劇, 的	
\\	干天	
\\	かんてん	
\\	れんぞく干天日数が二十九日をむかえ、フグは干からびるすん前だった。	
\\	無じ悲な干天がつづき、はたけのやさいが全てかれてしまった。	
\\	コウイチから腹ペコの社員への差し入れのベーコンは、まさに干天の慈雨だった。	
\\	天気 
\\	天 
\\	干, 天	
\\	東側	
\\	ひがしがわ	
\\	寒天工場なら、その道路の東側にあります。	
\\	こちらは、地中海東側の地いきで食べられている料理です。	
\\	飛行場の東側なので、ダンダス通りを真っ直ぐ東側に三十分ぐらい行くと着きますよ。	
\\	東口 
\\	ひがしぐち? 
\\	東 
\\	ひがし. 側 
\\	東, 側	
\\	非常	
\\	ひじょう	
\\	な 
\\	の 
\\	この非常ブレーキ、こわれてるかもしれないよ。	
\\	急に非常ベルが鳴り出して、びっくりしました。何かの非常事たいだったんですか?	
\\	いつでも家に非常食を備えて置くことは良いことです。	
\\	非, 常	
\\	虚しい	
\\	むなしい	い 
\\	毎日泣いてばかりいるのは虚しい。	
\\	今日は何だか虚しい気分です。	
\\	毎日何度も何度も同じことの繰り返しで、虚しくなりませんか。	
\\	い 
\\	(むな), 
\\	虚	
\\	日常	
\\	にちじょう	
\\	の 
\\	コウイチの日常の仕事がどんなものなのか知りたいです。	
\\	私の日常はとても平ぼんです。	
\\	新しい言語を学ぶ際、まず始めに日常で一般的に使われる実用的な言葉から覚える方が効率的だ。	
\\	日, 常	
\\	悪因悪果	
\\	あくいんあっか	
\\	ぜん因ぜん果悪因悪果って言うだろ?当然のむくいだよ。	
\\	悪いことをしたら、悪因悪果で後々ひどい目にあうよ。	
\\	聖書には、業について何て書いてあるんだっけ?悪因悪果とかだっけか?	
\\	悪果 
\\	悪 
\\	悪, 因, 果	
\\	説明	
\\	せつめい	
\\	する 
\\	の 
\\	電車のスピードを急に落とした理由を説明してください。	
\\	あんな無茶苦茶な説明で、よく会社の社長がつとまるよな。	
\\	ごめん。さっぱり分からないよ。要約すると、説明の趣旨は何なの?	
\\	説, 明	
\\	失敗	
\\	しっぱい	
\\	する 
\\	の 
\\	こっちの選手はバントを失敗したと思ったら、今度はこっちの選手がとうるいに失敗したんだよ。	
\\	失敗して後かいする方が、何もせずに後かいするよりマシだと思うんだ。	
\\	正直言って、君が作ったこの映画は失敗作だね。でも、一度失敗したからといって、くよくよしてちゃ駄目だよ。誰かも、「成功とは、失敗から失敗へと熱意を失うことなく進んでいける能力である」って言ってただろ?	
\\	失 
\\	失敗.	失, 敗	
\\	幻想	
\\	げんそう	
\\	の 
\\	フグはいくつになっても幻想の世界の住人なんだな。	
\\	時々、コウイチが自分の息子だというあまい幻想をいだいています。	
\\	私の兄は、自分が河豚の王だという馬鹿げた幻想を持ち続けている。	
\\	幻, 想	
\\	無料	
\\	むりょう	
\\	の 
\\	コストコで無料のサンプルを食べ歩くのが好きです。	
\\	ワニカニは、レベル四まで無料で利用できます。	
\\	クリスマス、一人ぼっちで過ごしたくないんだけど、もう予定立てちゃった?もしまだなら、ポール・マッカートニーのチケットが二枚あるんだけど、無料で一枚プレゼントするよ。	
\\	無, 料	
\\	果物	
\\	くだもの	
\\	果物の中では、無花果が一番好きです。	
\\	うちの子は野さいはきらいですが、トマトだけは果物だと思っているようで食べてくれるんです。	
\\	「果物屋さんが閉まるまでに、仕事終わりそう?」「大丈夫。もし終わらなかったら明日に回すよ。」	
\\	果 
\\	(くだ) 
\\	果, 物	
\\	栄光	
\\	えいこう	
\\	今回、勝利の栄光にかがやいたのは、トーフグチームです!	
\\	か去の栄光をなつかしそうに語る大人にだけはなりたくありません。	
\\	コウイチの栄光の瞬間は、二十二歳でブロガーのチャンピオンになった時にやって来た。	
\\	栄, 光	
\\	官金	
\\	かんきん	
\\	間違っても官金に手を出んじゃないぞ!	
\\	保育園まで公用車に子どもを乗せて行くことは、ある意味で官金の私消に当たると思いますか?	
\\	金曜日に親友が一万ドルの公的資金を濫用したとして官金私消罪で逮捕されたのだが、私には彼女がそんなことをしたとは思えない。	
\\	官, 金	
\\	幻覚	
\\	げんかく	
\\	コウイチは時々、ポッドキャストをレコーディングしている時に幻覚を起こしてしまうんだ。	
\\	この映像を見る時は、部屋を明るくしてはなれて見ないと、幻覚が見えてくるらしいよ。	
\\	フグは幻覚作用のあるキノコを食べたせいで、幻覚に襲われていたに違いない。	
\\	幻, 覚	
\\	失恋	
\\	しつれん	
\\	する 
\\	ぺちゃんこの赤い鼻が原因で、失恋しました。	
\\	失恋なんて、日常茶飯事だよ。	
\\	失恋しちゃって、超切ないよ。ストレス解消に買い物に行きたい。今週末バーゲンに行ってお昼も一緒に食べようよ。	
\\	失, 恋	
\\	光栄	
\\	こうえい	
\\	な 
\\	あべさんにインタビューさせていただけることをとても光栄に思っています。	
\\	コウイチ王と直せつお話ができることは、大変光栄なことなんですよ。	
\\	の本社で勤務する機会を頂いたことを、とても光栄に思います。皆さんのお役に立てるよう精一杯頑張りますので、どうぞよろしくお願いいたします。	
\\	栄光! 
\\	光, 栄	
\\	鼻歌	
\\	はなうた	
\\	コウイチは、鼻をふくらませながら、鼻歌を歌っていた。	
\\	ビエトは鼻歌まじりで楽しそうに自転車に乗っていました。	
\\	爆笑!お前鼻歌下手すぎだっつーの。	
\\	鼻 
\\	鼻
\\	(はな) 
\\	歌
\\	歌 (うた). 
\\	はなうた.	鼻, 歌	
\\	仮説	
\\	かせつ	
\\	の 
\\	この土地は、みかんやレモンのようなかんきつるいの果物に向いているんじゃないかという仮説を立てました。	
\\	この仮説の正しさを立証するには、コウイチの力が必要です。	
\\	面白いとは思うのですが、もしそれを題材に記事を書きたいのであれば、その仮説を裏付ける資料を見つけてくださいね。	
\\	仮, 説	
\\	無知	
\\	むち	
\\	な 
\\	周りの人から、無知なくせに知ったかぶりをするやつだと思われていたらどうしようと心配です。	
\\	無知が原因で、間ちがいをおかしてしまいました。	
\\	無知をさらけ出したからって恥ずかしがることなんか無いよ。あいつらこそ、自分たちの無知を隠していることを恥じるべきさ。	
\\	無, 知	
\\	幻	
\\	まぼろし	
\\	この薬を飲むと、幻が見えますよ。	
\\	幻の魚をつかまえたぞ!	
\\	恍惚状態の中でサーモンの幻を何度も見た後に、フグはようやく彼女が12年前に死んだことを思い出し、自分が幻を見ていたことに気づいた。	
\\	(まぼろし). 
\\	幻	
\\	図説	
\\	ずせつ	
\\	図説付きの参考書をさがしています。	
\\	アヤが、トーフグ社内の人間関係の図説を作ってくれました。	
\\	もし解説と共に図説があれば、より分かりやすくなるかと思うんですが、如何でしょうか?	
\\	図, 説	
\\	無理	
\\	むり	
\\	する 
\\	な 
\\	コウイチはニンニクは便ぴを治す薬だって言ってたけど、ニンニクで便ぴを治すのは無理でしょ。	
\\	コウイチが無理なん題をふっかけても、トーフグチームは無理してなんとかやってのける。	
\\	自分の肘を舐めるのは無理です。	
\\	無, 理	
\\	察知	
\\	さっち	
\\	する 
\\	昨夜、ビエトがコンピューターの不正利用を察知しました。	
\\	ビエトと手下たちは、警察の手入れがあることを事前に察知していたようだ。	
\\	僕達の赤ちゃんは、母親であるサーモンの感情や不安をすごく敏感に察知するんです。	
\\	察 
\\	さっ.	察, 知	
\\	内側	
\\	うちがわ	
\\	の 
\\	りそな銀行の内側で待っています。	
\\	右の内側のかかとにトゲがささったみたいです。	
\\	ゾンビを中に入れないために、金属製のシャッターを下ろして、内側から鍵を掛ける必要があります。	
\\	側 
\\	内 
\\	うち. 
\\	うちがわ. 
\\	うち 
\\	(うち) 
\\	内, 側	
\\	干渉	
\\	かんしょう	
\\	する 
\\	コウイチは、私生活への干渉はしないよう社員たちに注意しました。	
\\	両国とも、ぶ力干渉はさけたいはずだと思うよ。	
\\	コウイチは、どんなことにも干渉してくるタイプの上司ではありませんが、何故かビエトが何をしているのかについては口出しするのが好きなようです。	
\\	干, 渉	
\\	内因	
\\	ないいん	
\\	内因性ぜんそくだとしんだんされました。	
\\	トラウマになる出来事が起きると、その苦しみを和らげるために、内因性オピオイドというのう内ま薬が分ぴつされます。	
\\	私は長い間内因性精神疾患に悩まされています。	
\\	内, 因	
\\	知識	
\\	ちしき	
\\	コウイチって、知識のほうこみたいなやつだよな。	
\\	トーフグの社員は、みんな知識よくがおうせいです。	
\\	日本語の知識を身に付けたいですか?恐がらなくてもいいですよ。簡単です。あなたの知識を別の言語で表現する方法を学べばいいだけなんですから。	
\\	知, 識	
\\	左側	
\\	ひだりがわ	
\\	の 
\\	このエスカレーターでは、みんな左側に立っていますね。	
\\	日本は、イギリスと同じで、車は左側通行ですよ。	
\\	寝る時は、左側を下にして横になる方が好きです。	
\\	左 
\\	側 
\\	ひだりがわ.	左, 側	
\\	伝説	
\\	でんせつ	
\\	アイルランドの伝説の英ゆうと言われれば、だれを思いうかべますか?	
\\	お前めちゃくちゃモテるな!マジで伝説的だよ。	
\\	伝説によると、この海には豆腐を食べるフグが住んでいるそうです。	
\\	伝 
\\	でん 
\\	(でん) 
\\	伝, 説	
\\	敗者	
\\	はいしゃ	
\\	まだ敗者ふっ活戦が残っていますよ。	
\\	その歯医者は、生まれながらの敗者であった。	
\\	スポーツを見るとき、全力を尽くした負け犬と自分を重ね合わせてみなよ。それって楽しい?達成感はある?何か意味はあるの?意味無いよね?敗者から勝者に生まれ変わらなきゃ。	
\\	敗, 者	
\\	愛	
\\	あい	
\\	サボテンの花は、母性愛のしょうちょうです。	
\\	社内恋愛をもう想するのが好きなんですけど、一番最近の私のもう想の中ではコウイチとビエトの間に愛がめ生えちゃったんです!	
\\	私はお金は愛より大切だと思います。よく「愛があれば山でも動かせる」なんて言いますが、実際はお金が無いとそんなことできないでしょ。	
\\	愛	
\\	恋愛	
\\	れんあい	
\\	する 
\\	の 
\\	恋愛したら、相手に対していつも愛情を感じすぎてしまうんだよね。	
\\	うちの子は、最近恋愛小説ばかり読んでいます。	
\\	恋愛での別れ際は、切なくて甘酸っぱい。	
\\	恋, 愛	
\\	恋	
\\	こい	
\\	今まで、恋に恋したことはあっても、本当にだれかに恋したことは無かったんです。	
\\	サーモンちゃん、なんかかわいくなったよな。恋する女って感じじゃない?	
\\	フグのことが頭から離れないの。これって恋の始まりなのかな?	
\\	(こい) 
\\	恋	
\\	愛情	
\\	あいじょう	
\\	の 
\\	私の両親は、愛情の深い人たちでした。	
\\	愛情の無いけっこんはしたくありません。	
\\	愛情たっぷりのビーフシチューです。	
\\	愛, 情	
\\	愛人	
\\	あいじん	
\\	コウイチの愛人は、少し鼻にかかったみ力的な声をしていました。	
\\	お前は愛人にすらなれなかったんだよ。	
\\	愛人の是非については、判断しかねますね。	
\\	愛, 人	
\\	昼ご飯	
\\	ひるごはん	
\\	かれったら、私と一しょに昼ご飯を食べている時に、他のかわいい女を見て鼻の下をのばしてたんだから。	
\\	あ、まずい!昼ご飯はお好みやきにしようと思ってたのに、ソースが切れてるのわすれてた。	
\\	しまったぁ、またトーストを焦がしちゃったよ。最後の一枚だったのに。何か別のものを昼ご飯に探さなくちゃ。	
\\	昼 
\\	ご飯. 
\\	昼, 飯	
\\	主因	
\\	しゅいん	
\\	下りの主因はおそらくさっき食べたソフトクリームです。	
\\	ニアミスの主因は、この電車がスピードを上げるべきではないタイミングで上げたことです。	
\\	癌や自閉症など、多くの病気の主因は未だに分かっていません。	
\\	主, 因	
\\	薬用	
\\	やくよう	
\\	の 
\\	にきびが気になるので、薬用の石けんで顔をあらうようにしています。	
\\	この薬局に、薬用リップクリームはおいていますか?	
\\	咳がひどいんですが、妊娠中に薬用せき止めドロップを服用しても大丈夫でしょうか?	
\\	薬, 用	
\\	薬	
\\	くすり	
\\	鼻をきつくつまみながら薬を飲みます。	
\\	あの男は、薬づけの生活を送っている。	
\\	私は薬の中で座薬が一番好きです。	
\\	(くすり) 
\\	薬	
\\	薬物	
\\	やくぶつ	
\\	「今すぐ薬物をやめろ!」とさけびながら、コウイチはビエトの鼻を人さし指で指した。	
\\	ビエトは、ヤクザの世界からも、薬物からも、完全に足をあらった。	
\\	今までに何か薬物によってアレルギー反応が引き起こされたことはありますか?	
\\	薬, 物	
\\	兵員	
\\	へいいん	
\\	兵員たちはみんな、かぜにかかって鼻がつまっている。	
\\	兵員がにげ出さないよう、ドアには外側からかぎがかけられていた。	
\\	トーフグ軍の改革のため、コウイチは兵員の数を削減をしなくてはならなかった。	
\\	兵, 員	
\\	水兵	
\\	すいへい	
\\	あの水兵が、船長にふるえ声で何かを話しているのを見てしまったんだ。	
\\	かわいい水兵服と水兵ぼうが売っていたので、まごに買ってあげました。	
\\	その演劇で水兵を演じた素晴らしい俳優は、観客を大いに沸かせた。	
\\	水, 兵	
\\	交渉	
\\	こうしょう	
\\	する 
\\	の 
\\	クリステンは、交渉のプロです。	
\\	なんとか交渉してみますが、期待はしないでくださいね。	
\\	新婚旅行の際、フグはサーモンが値段の交渉が得意だということを知ってとても感心した。	
\\	交, 渉	
\\	無休	
\\	むきゅう	
\\	コウイチは大てい無休で働いている。	
\\	カナとマミは、無休でワニカニの例文を作っています。	
\\	のウェブサイトは、24時間年中無休で開いております。	
\\	無, 休	
\\	鼻	
\\	はな	
\\	ぼくの鼻のことにはふれずに、そっとしておいてくれませんか?	
\\	鼻をほじると、鼻のあなは大きくなりますか?	
\\	鼻の病気のせいでストレスが溜まっているからって、私たちに八つ当たりしないでよ!	
\\	鼻	
\\	鼻血	
\\	はなぢ, はなじ	
\\	鼻の下にひげをはやしていると、鼻血が出た時に大変ですね。	
\\	すみません。ちょっと一言、言わせてもらいたいんだけど、あなた、鼻血が出てますよ。	
\\	電車に駆け込み乗車をしようとして鼻が挟まれて、鼻血が出ました。	
\\	鼻
\\	血 
\\	はなち, 
\\	ち 
\\	ぢ. 
\\	はなじ. 
\\	鼻, 血	
\\	小説	
\\	しょうせつ	
\\	の 
\\	この小説マジで大好きだわ。これはホンモノだよ。	
\\	お父さん、小説が売れて鼻高々だね。	
\\	「ほら、これでなみだと鼻をふきなよ」と言いながら、コウイチは読んでいた小説のページを一枚やぶって、クリステンにわたした。	
\\	小, 説	
\\	願い事	
\\	ねがいごと	
\\	あなたの願い事は、きっとすぐにかなうと思います。	
\\	コウイチはいつも、願い事を書いた紙をそばにおいてねます。	
\\	日本人は、人生の節目節目に、願い事だけでなく感謝の気持ちを伝えるために、神社を訪れます。	
\\	お願い 
\\	事 
\\	お願い 
\\	事. 
\\	願, 事	
\\	空席	
\\	くうせき	
\\	の 
\\	アメリカでクリスマスに映画を見に行ったら、ほとんど空席でびっくりしました。	
\\	空席の出そうなじゅ業があれば教えてもらえませんか?	
\\	まだ空席はたくさんありますか?できれば前の方の席がいいのですが。	
\\	空, 席	
\\	歌劇	
\\	かげき	
\\	の 
\\	その歌劇の主人公は生粋のフランス人です。	
\\	とある歌劇で、奴隷の役を演じました。	
\\	歌劇のチケットの値段を2
\\	値上げします。	
\\	歌, 劇	
\\	原作	
\\	げんさく	
\\	の 
\\	マンガが原作の映画は、原作の良さを失ってしまうことが多いです。	
\\	シェイクスピアを原作で読んだことはありますか?	
\\	この漫画は、テレビアニメとドラマの原作です。	
\\	原, 作	
\\	外側	
\\	そとがわ, がいそく	
\\	の 
\\	外側に開くと思ったら、内側に開くドアだったんだね。	
\\	いつもは外側のポケットにかぎを入れておくんですけど...おかしいなぁ。	
\\	サーモンのフライを作ろうとしたんだけど、外側だけ焦げちゃって内側は血生臭いオレンジ色のままになっちゃった。油がちょっと熱すぎたのかも。	
\\	外 
\\	外, 側	
\\	愛国心	
\\	あいこくしん	
\\	愛国心がやたらに強い男が、トーフグに入社しました。	
\\	マイケルが、愛国心あふれるチップチューンの曲を作曲しました。	
\\	あの国が、誤った教育を通じて子どもたちに愛国心を植え付けようとしているのは好ましくないと思います。	
\\	愛, 国, 心	
\\	警察	
\\	けいさつ	
\\	の 
\\	警察しょに入ると、梅酒のにおいが鼻をついた。	
\\	その警察は、ビエトの警告をあざ笑ったため、後からひどい目にあいました。	
\\	警察署での出会いから一時間後、その警察官と泥棒はすっかり意気投合した。	
\\	警, 察	
\\	無力	
\\	むりょく	
\\	な 
\\	の 
\\	自分がどれだけちっぽけで無力な存ざいであるかを思い知らされました。	
\\	「無力の証明」というゆうぎ王カードを手に入れました。	
\\	警察はそのストーカーに対して無力で何をすることもできず、結局彼女は殺されてしまった。	
\\	力 
\\	りょく 
\\	(りょく) 
\\	無, 力	
\\	薬方	
\\	やくほう	
\\	医者にもらった薬方を失くしてしまいました。	
\\	字がきたなすぎて、薬方に何が書いてあるのかさっぱり分かりません。	
\\	古い薬方の中には、危険であるということで現代医療からは除かれているものもある。	
\\	薬, 方	
\\	建前	
\\	たてまえ	
\\	ビエトは本音と建前を使い分けるのが上手い。	
\\	しょうかいじょうが無いとここの医者にみてもらえないというのは建前です。	
\\	よく「本音と建前」は日本独特の文化だと言われますが、実際は世界中に存在します。	
\\	建物 
\\	たて, 
\\	建てる 
\\	て 
\\	建てる 
\\	て 
\\	建, 前	
\\	結果	
\\	けっか	
\\	今年の
\\	の選きょの結果が気になります。	
\\	コウイチは、結果が一番大事だとよく口にしています。	
\\	その結果には全然満足していませんよ。彼女の役に立とうという計画は、いつも裏目裏目に出るんです。今回もまた然りでした。	
\\	結 
\\	けっ.	結, 果	
\\	右側	
\\	みぎがわ	
\\	の 
\\	日本の自動車のハンドルは通常右側にあります。	
\\	コウイチのつくえの右側の引き出しには、塩がたくさん入っています。	
\\	心霊写真を撮っちゃったの。ほら。この写真の右側に、その場にはいなかった幼い女の子が写ってるのよ。分かる?	
\\	右 
\\	側 
\\	側.	右, 側	
\\	川底	
\\	かわぞこ	
\\	あの辺りは川底が深いので気を付けてください。	
\\	この川は、川底にかなりのすながたまってしまっています。	
\\	「サーモンはゴールデンウィーク期間中何してたの?」「結婚指輪を川で無くしちゃって、ずっと探してたの。三日もかかったけど、ようやく川底に沈んでいる指輪を見つけたわ。」	
\\	底!	川, 底	
\\	無事	
\\	ぶじ	
\\	な 
\\	みなさんの無事をおいのりしています。	
\\	両親は、たいしょくして、平おん無事な生活を送っています。	
\\	本当に赤ん坊は私達の言葉が分からないと言い切れるんですか?もし私の双子ちゃんが、「お医者さんが今ママに、僕達がママの子宮から無事に出られるかどうかは、五分五分だって言ったみたいだね。」なんて話をしてたらどうするんですか?ああ、私の赤ちゃん、きっと先生の言葉に震え上がってるに違いないわ。	
\\	無 
\\	む 
\\	ぶ. 
\\	(ぶ) 
\\	無, 事	
\\	塩	
\\	しお	
\\	もう少し塩が多い方が良いかもしれませんね。	
\\	一才になるまで赤ちゃんには塩をあげない方がいいですよ。	
\\	けしからん!一体どうやってそんなに馬鹿なことが出来るんだ?砂糖と塩を入れ替えるなんて、いくら今日がエイプリルフールだからといっても、ちっともおもしろくないぞ。お前が馬鹿者だ!	
\\	塩	
\\	塩水	
\\	しおみず, えんすい	
\\	鼻が利く方だと思っていたんだが、どちらが塩水か分からなかった。	
\\	塩水につけて天日干ししたフグってきっと美味しくなると思うんだけどなあ。	
\\	気温はそんなに高くないけど、日差しは強いから、熱射病対策で塩水を持って行った方がいいんじゃない?	
\\	塩水.	
\\	塩
\\	水
\\	水. 
\\	しおみず.	塩, 水	
\\	塩味	
\\	しおあじ, しおみ	
\\	の 
\\	もっと塩味を利かせてもよかったかもね。	
\\	うす塩味のポテトチップスは、あまり人気がありません。	
\\	「ねぇねぇ!コウイチに塩味のジェロを作ってあげたら、テキストフグのアカウントが無料でもらえるって聞いたんだけど。」「信じないほうがいいよ。それって嘘だと思うよ。」	
\\	塩 
\\	味 
\\	しおあじ.	塩, 味	
\\	席	
\\	せき	
\\	おそれ入りますが、お席をご利用されるさいは、メニューから何か一つご注文をお願いします。	
\\	私の右側の席にみんなの荷物をおきましょう。	
\\	「すみませんが、席を詰めて私を座らせてもらえませんか?」「申し訳ないんですが、この席は空いてないんです。」	
\\	席	
\\	自愛	
\\	じあい	
\\	する 
\\	今日は自愛デーにすることに自分で決めたので、一日中自愛しまくっちゃいます。	
\\	自愛をぎせいにして成り立つものを生きがいにするべきではないと思うんです。	
\\	時節柄、お風邪など召しませぬよう、くれぐれも御自愛くださいませ。	
\\	自, 愛	
\\	感覚	
\\	かんかく	
\\	する 
\\	したの感覚がまひしています。	
\\	トーフグのユーモアの感覚が好きです。	
\\	もし本当に煙草を止めたいなら、肺を病む人特有の感覚を知ってみることだね。	
\\	感, 覚	
\\	性愛	
\\	せいあい	
\\	大学で、性愛学について勉強しています。	
\\	他人の夫婦の性愛に口出しするなんて、下品にもほどがあるよ。	
\\	性欲が無いのか性愛恐怖症なのか自分では判断できません。	
\\	性, 愛	
\\	側	
\\	がわ, そば	
\\	もう少しだけ、側にいてもいい?	
\\	お母さんの側をはなれちゃだめよ!	
\\	ソファの右側に植木を置こうかなと思ってるの。	
\\	側	
\\	署名	
\\	しょめい	
\\	する 
\\	ここに署名をお願いします。	
\\	このたん願書に、一万にんの署名を集めることが目標です。	
\\	細則を読まずに契約書に署名しない方が良かったことは分かってたんですが、一見したところ大家さんがいい人そうだったので、つい。	
\\	署, 名	
\\	無言	
\\	むごん	
\\	毎晩無言電話がかかってくるんです。	
\\	トーフグの社員たちは、無言でもくもくと仕事をしているように見えますが、実はスラックというアプリでチャットしています。	
\\	知らないオッサンから急に母親ってのは自分の子どもに愛情を感じるものだろうとか説教されてさぁ、びっくりして無言になっちゃったよ。	
\\	言 
\\	(ごん). 
\\	無, 言	
\\	鼻くそ	
\\	はなくそ	
\\	かぎ鼻の男が、ほじくり出した鼻くそを、ねむっている低い鼻の男の鼻の穴に入れるのを見ました。	
\\	指でほじくり出した鼻くそをティッシュの上に集めています。	
\\	フグが僕のことを自分勝手だって責めるんだ。目くそ鼻くそを笑うだよね。	
\\	くそ 
\\	はなくそ.	鼻	
\\	兵士	
\\	へいし	
\\	あの兵士の泣き声が頭からはなれません。	
\\	その兵士はコウイチの車に火を付けた。	
\\	仲間がいないといつもの道が遠く感じると思いながら、兵士は足早に歩いた。	
\\	兵, 士	
\\	兵	
\\	へい	
\\	あの兵が、無力な子供たちにドーナツを投げつけたんだ!	
\\	たくさんの兵たちが、梅毒に苦しんでいました。	
\\	ここで、アメリカ兵たちがよく屯しています。	
\\	兵	
\\	薬学	
\\	やくがく	
\\	の 
\\	どうしてコウイチは最近薬学の本をデスクにおいているの?	
\\	かれは、東京大学の薬学部の学生です。	
\\	コウイチが、薬学者の募集を検討しているって、本当ですか?	
\\	薬, 学	
\\	劇場	
\\	げきじょう	
\\	の 
\\	劇場のロビーで待ち合わせしましょう。	
\\	それでは劇場のトイレで会いましょう。	
\\	劇場までの毎日の通勤に、片道二時間以上かかります。	
\\	劇, 場	
\\	細い	
\\	ほそい	い 
\\	この糸はちょっと細すぎますね。	
\\	ピアニストみたいに細くて長い指だね。	
\\	私のダイエット計画は、基本的に友達全員にカップケーキを作ることです。つまり、みんなを太らせれば太らせるほど、私が細く見えるってわけ。	
\\	い 
\\	(ほそ)! 
\\	細	
\\	鼻先	
\\	はなさき	
\\	鼻先にマスタードがついてるよ。	
\\	その警察犬は、鼻先をひくひくさせて、見知らぬ人のにおいをかぎ始めた。	
\\	フグはサーモンの鼻先が赤くなってるのを見て、とても可愛いと感じた。	
\\	鼻 
\\	先 
\\	鼻, 先	
\\	変わる	
\\	かわる	
\\	だれがベーコンをやいてもおいしさはそんなに変わらないと言うのはうそです。	
\\	習かんは国によって変わりますから、なれるしかないですよ。	
\\	化粧でこんなに変わるものかね。	
\\	変える 
\\	変	
\\	敗れる	
\\	やぶれる	
\\	ブルージェイズは、レッドソックスに三対一で敗れました。	
\\	みんなが予想していた通り、コウイチは今回ものど自まんコンテストで敗れてしまいました。	
\\	オリンピックの予選第一回戦で敗れた日本代表のアイスホッケーチームに、怒ってカレーを投げつけた男の話聞いた?	
\\	う 
\\	(やぶ) 
\\	敗	
\\	勇む	
\\	いさむ	
\\	あんなに勇んでたら、あの馬は買えないなぁ。	
\\	コウイチとビエトは、勇んでオフィスに入って来ました。	
\\	少年は、喜び勇んで旅に出発した。	
\\	う 
\\	(いさ)! 
\\	勇	
\\	建つ	
\\	たつ	
\\	東京には、新しい高そうビルがぞくぞくと建っています。	
\\	ホワイトハウスに、コウイチのどうぞうが建ちました。	
\\	サーモンと僕は、浜辺の北端に建つ家を買おうと相談してるんですが、ちょっと私達には値段が高いんですよね。	
\\	建てる 
\\	つ, 
\\	(つ) 
\\	建てる, 
\\	建てる 
\\	建	
\\	悲しむ	
\\	かなしむ	
\\	かの女は失れんしてとても悲しんでいます。	
\\	人間だけではなく、動物も仲間の死を悲しみます。	
\\	そう悲しむなって。お前の日本語は今はそれで十分だよ。ただ、勉強することを投げ出さずに、少しずつ上手くなればいいさ。	
\\	う 
\\	(かな). 
\\	悲	
\\	干す	
\\	ほす	
\\	セーターはかんそうきに入れないで、干してねって言ったじゃない。	
\\	あの芸人、ぜっ対最近干されてるよね。	
\\	洗濯物を外に干したかったんだけど、物干し竿が壊れちゃって。	
\\	う 
\\	(ほす) 
\\	干	
\\	果てる	
\\	はてる	
\\	三日もねてないから、つかれ果てたよ。	
\\	こんなところで果てるわけにはいかないと思ったんです。	
\\	いつ果てるともなく続く会議に、社員は全員ウンザリしていた。	
\\	(てる) 
\\	(は)!
\\	果	
\\	栄える	
\\	さかえる	
\\	この街はもっと栄えてるのかと思ってました。	
\\	この町は、かつては銀山で栄えていました。	
\\	中国には、覚せい剤の生産で栄えていた村がありました。	
\\	う 
\\	(さか) 
\\	栄	
\\	察する	
\\	さっする	する 
\\	ふだんの言動から察するに、コウイチは鼻の下が非常に長い。	
\\	うちの夫は、私の気持ちを察するのが下手です。	
\\	いくつかの記事を読んで、私は
\\	チームには何か暗黙の構想があることを察しました。	
\\	う 
\\	さつ 
\\	さっ. 
\\	つ.	察	
\\	愛する	
\\	あいする	する 
\\	コウイチはトーフグのことをとても愛していて、「ぼくの愛するトーフグちゃん」とかってよくつぶやいています。	
\\	愛する人をきずつけてしまった。	
\\	主人は息子のことは愛していましたが、愛情を示したことは一度もありませんでした。きっと、どうすれば良いのか分からなかったんだと思います。	
\\	愛 
\\	する 
\\	愛	
\\	喜ぶ	
\\	よろこぶ	
\\	やっとラケットの真ん中にボールを当てることができて、息子はとても喜んでいました。	
\\	うちの母、初めて白がぞめをしたんだけど、十才はわか返って見えるって言って、本人はすごく喜んでるよ。	
\\	日本語を勉強することがあなたの人生で唯一の喜びだと聞いて、私達はとても嬉しく思っています。喜んでアドバイスするので、質問があればいつでも聞いてくださいね。	
\\	う 
\\	(よろこ)!
\\	喜	
\\	悲劇	
\\	ひげき	
\\	の 
\\	それが悲劇の始まりだった。	
\\	あんた、悲劇の主人公にでもなったつもり?いつまでもクヨクヨしてたって仕方ないでしょ?	
\\	あなたには悲劇がお似合いよ。	
\\	悲, 劇	
\\	梅	
\\	うめ	
\\	梅の花がさき始めました。	
\\	「こんなにたくさんの梅の実、どうしたの?」「今日、梅園に行ってきたんだよ。」	
\\	給料日が来たら、一粒一万円もする高級梅干を買ってあげるよ。	
\\	梅	
\\	梅酒	
\\	うめしゅ	
\\	風ろ上がりに梅酒を開けると、一本飲み切るまで止まりません。	
\\	日本では、電車の中で梅酒を飲んでもいいんですか?	
\\	よっしゃ〜花金だ〜。今夜の自家製梅酒が待ち遠しい。	
\\	梅, 酒	
\\	不詳	
\\	ふしょう	
\\	な 
\\	の 
\\	昨日、年れい不詳な美人女性に出会ったんだ。	
\\	ワニカニは、性別不詳のモンスターです。	
\\	彼ってほんと職業不詳だよね。	
\\	不, 詳	
\\	米兵	
\\	べいへい	
\\	あの米兵は、どうして鼻めがねをかけているの?	
\\	ここは米兵がたくさん通るから、通行のじゃまをしないようにしてください。	
\\	「すごい。今日は本当にいいお天気だね!今日こそ腹筋三百回が出来そうな気がする。」と、米兵は言った。	
\\	米国 
\\	米 
\\	兵 
\\	米, 兵	
\\	虚栄心	
\\	きょえいしん	
\\	虚栄心の強い人は、男でも女でも苦手です。	
\\	自分の虚栄心に度々きずつけられます。	
\\	ソフトボールチームに入る際の第一規則は、虚栄心をベンチに置いておくことです。	
\\	虚, 栄, 心	
\\	勝敗	
\\	しょうはい	
\\	その戦いの勝敗なら、一昨日の朝決まりましたよ。	
\\	コウイチのその一言が、勝敗の分かれ目になったんです。	
\\	私の両親はいつも、全力を尽くしさえすれば、勝敗は問題ではないと言っていたけれど、私はそれが好きじゃありませんでした。もちろん、私は勝敗をとても気にします。だって、ベストを尽くした後に負けるなんて、意味が無いじゃないですか。	
\\	勝, 敗	
\\	虚弱	
\\	きょじゃく	
\\	な 
\\	の 
\\	虚弱な子どもを見ると、心配になります。	
\\	私はいちょうが虚弱なんです。	
\\	私の息子は生まれつき虚弱体質で、運動はできません。	
\\	虚, 弱	
\\	西側	
\\	にしがわ	
\\	の 
\\	この町の西側のゆう便局で働いています。	
\\	西側しょ国とみなされるのはどの国ですか。	
\\	今年の二月三日に巻き寿司を食べるなら、西側を向いて食べなくてはいけませんよ。	
\\	西 
\\	にし 
\\	側 
\\	がわ 
\\	西, 側	
\\	無し	
\\	なし	
\\	このブドウはたね無しだから食べやすいですよ。	
\\	文く無しのす晴らしいスピーチだったね。	
\\	この話は、一旦無しにしてもらえますか。	
\\	(なし) 
\\	無	
\\	予告	
\\	よこく	
\\	する 
\\	大丈夫だよ、コウイチは予告なしにかいこしたりしないよ。	
\\	三ヶ月家ちんをたいのうしてしまい、立ちのきを予告されました。	
\\	殺人予告を告げる手紙が、
\\	オフィスに届けられた。	
\\	予, 告	
\\	広告	
\\	こうこく	
\\	する 
\\	の 
\\	図書館から借りた本に、スーパーの広告が一枚はさまっていました。	
\\	クーポンが付いた広告を忘れたので、家にもどらなければいけません。	
\\	ビエトはコウイチに、アヤが
\\	の広告にものすごく変わった絵を描いたことを告げ口しました。	
\\	広. 
\\	こういち 
\\	広, 告	
\\	笑い	
\\	わらい	
\\	日本ではクリスマスディナーにケンタッキーを食べると言ったら、なぜか笑いが起きました。	
\\	もうかれ自身多分何を笑ってるのか分からないけどただ笑いが止まらないっていうじょうたいなんだと思うよ。	
\\	映画を見て泣くなんて、お前らしくないじゃないか。いつもどの映画のどんな深刻なシーンにも大笑いしてるのに。	
\\	笑う 
\\	う 
\\	笑う 
\\	い 
\\	笑う, 
\\	笑	
\\	弓道	
\\	きゅうどう	
\\	弓道はむずかしいという結ろんに達しました。	
\\	日本の弓道の弓って、どうしてそんなに強いの?	
\\	「一生懸命何しているの?」「弓道だよ。」	
\\	弓, 道	
\\	取り決め	
\\	とりきめ	
\\	ちゃんと取り決めを守ってもらえるのか、不安です。	
\\	ルームメイトと生活上の取り決めをしておいた方がいいよ。	
\\	サーモンさんと我が社で取り交わした契約によると、残念なことに彼女が競合他社の
\\	に出演することについて我々は何の取り決めも行っておりませんでした。	
\\	取り 
\\	取る 
\\	決め 
\\	決める. 
\\	とりきめ!	取, 決	
\\	黒煙	
\\	こくえん, くろけむり	
\\	黒煙のせいで、熱気がこもって息苦しかった。	
\\	あなただけがあの黒煙をすいこんだわけじゃないのよ。	
\\	私はバスの運転手になるのを楽しみにしていたんですが、息子から、「バスが黒煙をまき散らしていることを知りながら、それでもバスの運転手を続けたいなら、排気管からの黒煙を少し吸ってみるといい」と言われました。	
\\	黒, 煙	
\\	借用	
\\	しゃくよう	
\\	する 
\\	これで、借用金が百万円に達してしまった。	
\\	ちゃんと借用証を書いてくれるんだったら、お金をかしてもいいよ。	
\\	日本語には、別の言語から借用された言葉がたくさんあります。	
\\	借, 用	
\\	弓	
\\	ゆみ	
\\	コウイチの弓がどこにも見当たらないんだ。	
\\	弓を引くには、まず上半身を安定させなくてはいけません。	
\\	うん、ローソンは確かに二十四時間営業だよ。でもさぁ、あのー、多分弓と矢は置いてないと思うよ?	
\\	(ゆみ). 
\\	弓	
\\	脳	
\\	のう	
\\	アイツは本当に脳が弱いよな。	
\\	ちょっとは脳を使って考えろよ!	
\\	この国の人たちは脳を食べるってこと、知ってる?珍味なんだよ。	
\\	脳	
\\	脳死	
\\	のうし	
\\	の 
\\	母が脳死じょうたいになった時、父は「何が起きるか分からない、それが人生だ」と自分に言い聞かせるように言いました。	
\\	脳内出血は、脳卒中につながり、最悪脳死を引き起こす可能性があります。	
\\	昨年イギリスで、脳死と判定された人のうち2~3人の人が息を吹き返し、さらに昏睡状態の間の会話を一部覚えていたという。	
\\	脳, 死	
\\	脳みそ	
\\	のうみそ	
\\	自分の体のどこかを変えられるとしたら、ぜっ対に脳みそを変えるね。	
\\	今日はつかれすぎて脳みそが全然働いていません。	
\\	「ドタマかち割って脳みそストローでチューチュー吸うたろか!」とお化けは怒って言った。	
\\	みそ 
\\	脳 
\\	脳	
\\	洗脳	
\\	せんのう	
\\	する 
\\	洗脳という言葉は1950年代に生まれた。	
\\	トーフグで最初に受けるトレーニングの中には、トーフグマインドへの洗脳もふくまれている。	
\\	多くの日本のネット住民は、日本のテレビ番組に韓国コンテンツが急増していることを洗脳行為のようなものだとなぞらえる。	
\\	洗, 脳	
\\	仏僧	
\\	ぶっそう	
\\	あの仏僧は、あと一ヶ月で九十九才に達します。	
\\	コウイチのダジャレがつまらなさすぎて、仏僧ははとがまめ鉄っぽうを食ったような顔をしていた。	
\\	その魔女は仏僧の心をも燃え上がらせた。	
\\	仏, 僧	
\\	平静	
\\	へいせい	
\\	な 
\\	人類は自らぜつめつをまねくかもしれないと聞いて、平静ではいられなかった。	
\\	今は自分でもびっくりするぐらい、平静な気持ちなんです。	
\\	便秘の時には、心の平静を失いやすくなる。	
\\	平, 静	
\\	飴	
\\	あめ	
\\	飴を一つ下さい。	
\\	日本では、カレー味の飴が流行っています。	
\\	飴は舐める派ですか?それとも飲み込む派?	
\\	飴	
\\	留守番	
\\	るすばん	
\\	する 
\\	一日二万円で留守番の仕事を引き受けた。	
\\	アメリカでは小さい子どもだけで留守番させるのはい法です。	
\\	2ヶ月間旅行したければ、留守番を雇う必要があります。	
\\	留守 
\\	留守 
\\	留, 守, 番	
\\	洗車	
\\	せんしゃ	
\\	する 
\\	コウイチの車の洗車をしたいという方がいたら、
\\	までメールしてください。	
\\	洗車が終わるのを待っている間、ガソリンスタンドのバイトがずっとダラダラ話しかけてきて、かなりウザかった。	
\\	誰がマネー・ロンダリングするのに洗車ビジネスなんて買うっていうのさ?	
\\	戦車 
\\	戦車 
\\	洗車...	洗, 車	
\\	証明書	
\\	しょうめいしょ	
\\	電車の中に証明書が入ったかばんを忘れてきてしまった。	
\\	証明書のていじは必要ですか?	
\\	身分証明書をご提示頂けますか?	
\\	証明 
\\	証明書 
\\	証明
\\	証, 明, 書	
\\	お守り	
\\	おまもり	
\\	このお守り、どのぐらいお借りしていてもいいですか?	
\\	お守りを入れるのを忘れちゃったみたい。	
\\	たくさんのお守りを同時に付けない方がいいですよ。逆に不幸をもたらすかもしれないですからね。	
\\	守る 
\\	守る, 
\\	守	
\\	胸	
\\	むね	
\\	希望に胸をふくらませながら、トーフグで働き始めました。	
\\	かにさされて胸がすごくかゆいことは、だれにも言わずに、胸にしまっておいた。	
\\	コウイチが毎朝胸を張って歌う
\\	のテーマソングの歌詞は胸にぐっときます。	
\\	胸	
\\	分類	
\\	ぶんるい	
\\	する 
\\	トーフグの社員をお笑いのレベルで松、竹、梅に分類してください。	
\\	動物うらないで、コウイチは自分が羊に分類されたことに大喜びした。	
\\	テロリストは人質の扱い方の違いで大きく2つに分類されます。「フレンドリーな奴」と「フレンドリーじゃない奴」です。	
\\	分, 類	
\\	お笑い	
\\	おわらい	
\\	マイケルがトーフグに入る時に間ちがえてメイン州のポートランドに引っこしてしまったことは、今でもみんなのお笑い草です。	
\\	コウイチは毎日、マイケルからお笑いの特くんを受けています。	
\\	夢の中で、私はお笑い芸人で、テレビでお客さんの前でオナラをしなくちゃいけなかったんだけど、うまくできなかったの。それで、その番組の司会を務めてた先輩芸人にすごく怒られたのよね。	
\\	笑う 
\\	笑い). 
\\	お 
\\	お 
\\	笑う 
\\	笑い. 
\\	笑	
\\	文句	
\\	もんく	
\\	コウイチは、トーフグ社員の
\\	の試験結果に文句タラタラだった。	
\\	オンラインの日本語学習教材の中では、トーフグが文句無しで一番いいと思っています。	
\\	日本語で、文句という言葉が、英語の僧侶という単語と発音が似ているということは面白いですね。	
\\	文, 句	
\\	告白	
\\	こくはく	
\\	する 
\\	の 
\\	「どうして告白にそんなに時間がかかったの?」「告白の仕方が分からなかったんだ。」	
\\	ジャマルはずっとドラクエファンだと公言していたのに、最近ついにファイナルファンタジーのかくれファンだったことを告白した。	
\\	会計事務所でのアシスタントの仕事を得た後に、彼女は実は数学があまり得意ではないことを告白した。	
\\	告, 白	
\\	可愛い	
\\	かわいい	い 
\\	コウイチって可愛い声で話すよね。	
\\	あの可愛い赤ちゃんはだれの子ですか?	
\\	フグ: サ…サーモンっ?へーっ...す、すっごく可愛い名前だね。 サーモン: そう思う? 私はずっと自分の名前があまり好きじゃないんだよね。	
\\	かわいい. 
\\	はいい
\\	(かわい), 
\\	可, 愛	
\\	借金	
\\	しゃっきん	
\\	する 
\\	新しいけいけんをしてみたくて、借金を作りました。	
\\	コウイチは、ビエトに五千万円借金している。	
\\	失敬、失敬、通らせて。借金取りに追われているんだ。	
\\	しゃく 
\\	しゃっ 
\\	借, 金	
\\	辞書形	
\\	じしょけい	
\\	食べたの辞書形は食べるです。	
\\	辞書形は分かるんですが、活用形がよく分かりません。	
\\	この言葉の辞書形を書きなさい。	
\\	辞書 
\\	〜形 
\\	辞, 書, 形	
\\	可分	
\\	かぶん	
\\	今日のトーフグポッドキャストのテーマは、生と死の不可分性です。	
\\	英文けい約書には、一ぱんじょうこうの一つに、可分性というじょうこうがよくあります。	
\\	預貯金などの可分債権は、原則として遺産分割の対象にはなりません。	
\\	可, 分	
\\	書類	
\\	しょるい	
\\	この書類は不びがあります。	
\\	書類を家のつくえの上においてきてしまったっぽい!	
\\	「いつ
\\	の機密書類が欲しい?」「早ければ早いに越したことはないな。」	
\\	書, 類	
\\	上品	
\\	じょうひん	
\\	な 
\\	母は時々上品な仕草を見せる時があります。	
\\	私の友達はミセスコンテストで上品な言葉づかいでスピーチした。	
\\	カナエはいつも上品にナイフとフォークを使うように両親からキツく言われている。	
\\	下品 
\\	上, 品	
\\	等しい	
\\	ひとしい	い 
\\	そんなに平等にこだわるなら、全社員の仕事りょうを等しくするべきです。	
\\	左の金玉と右の金玉の重さって等しいと思いますか?	
\\	オタクだけではなく他の人にとっても、ネット上で大文字を使うことは大声を出すことに等しいですよ。	
\\	い 
\\	(ひと) 
\\	等	
\\	対等	
\\	たいとう	
\\	な 
\\	の 
\\	コウイチは、社員達を対等の人間としてあつかってくれます。	
\\	ビエトは、コウイチと対等な共同けいえい者だと思っていました。	
\\	あなたとは対等に話をすることができません。	
\\	対, 等	
\\	等号	
\\	とうごう	
\\	ここに等号を書き忘れているよ。	
\\	コウイチは等号の上と下の線が同じ長さじゃないとげきどする。	
\\	等号の右側の数式が正か負かについて、注意する必要があります。	
\\	等, 号	
\\	取り分け	
\\	とりわけ	
\\	取り分けコウイチがどなっている時に、ビエトは一人静かに笑っていました。	
\\	ついに
\\	1に合格したんですが、トーフグには取り分け感謝しています。	
\\	今日は取り分け寒いという訳ではない。	
\\	取る 
\\	分ける. 
\\	取, 分	
\\	留学生	
\\	りゅうがくせい	
\\	その留学生は、くつひもを結ぶのをいつも忘れてしまいます。	
\\	あのクラスの留学生、何だかあどけない感じがするよね。	
\\	あの留学生、いつも私と彼氏の写真に変顔とか変なポーズで写り込んでくるんだけど。ちょっとうざいんだよね。	
\\	留学, 
\\	学生, 
\\	学), 
\\	留, 学, 生	
\\	遠く	
\\	とおく	
\\	の 
\\	今はめがねをかけてないので、遠くの物がよく見えません。	
\\	私は、家族から、遠くはなれて住んでいます。	
\\	遠くの方で銃声が響いた。	
\\	遠い. 
\\	遠	
\\	一等	
\\	いっとう	
\\	の 
\\	一等車に乗ってるとは、さすがコウイチだね。	
\\	銀座の一等地にオフィスをかまえるなんて、よっぽどもうけてるんだなぁ。	
\\	このベーコンのキーホルダーは、私がベーコンのコスプレ大会で一等賞を取った時にもらったんです。	
\\	いち 
\\	いっ 
\\	一, 等	
\\	五枚	
\\	ごまい	
\\	ゆう便局に行って、葉書を五枚買いました。	
\\	コピー用紙は三枚じゃなくて五枚必要だって言ったでしょ!どうしてこんなやさしいこともできないの、コウイチは?!	
\\	いいですね。五枚のステッカーをオマケに付けておきますね。	
\\	枚 
\\	五, 枚	
\\	〜枚	
\\	まい	
\\	日曜日のサッカーの試合のチケットを十枚持っています。	
\\	いつもは朝ご飯は食パン一枚だけです。	
\\	のステッカーを何枚か買いたいのですが、一枚いくらになりますか?	
\\	枚	
\\	禁止	
\\	きんし	
\\	する 
\\	ビエトは手下たちに、とばくは禁止だと言い聞かせた。	
\\	トーフグは、
\\	のけんえつによって動画の公開を禁止されました。	
\\	宇宙服を脱ぎっぱなしにするんじゃない!禁止事項の一つだろうが!	
\\	禁, 止	
\\	予報	
\\	よほう	
\\	する 
\\	今日は午後から雨の予報だったことをすっかり忘れていました。	
\\	お天気お姉さんは、明日から暑くなるって予報してたけどね。	
\\	天気予報によると、カナダは今年はずっと雪が降る見通しです。	
\\	天気予報.	
\\	予, 報	
\\	金曜日	
\\	きんようび	
\\	金曜日に飲みに行く約束、忘れてないよね?	
\\	ビエトは、毎週金曜日の夜に、情報のバックアップを作成します。	
\\	「フグのコスチュームですが、いつ頃仕上がりますかねえ?」「金曜日の午後までにはできていると思います。」	
\\	曜日 
\\	曜日 
\\	金, 曜, 日	
\\	友達	
\\	ともだち	
\\	あなたのお友達リストに私をついかしてください。	
\\	友達が、ここはちゅう車禁止エリアだって言ってたよ。	
\\	「側にいてくれてありがとう。」「何言ってるのよ!友達じゃないの!」	
\\	(とも) 
\\	(だち) 
\\	(ともだち) 
\\	友, 達	
\\	銀座	
\\	ぎんざ	
\\	銀座付近で、じゅうたいをぬけるのにすごく時間がかかった。	
\\	いとこは銀座を歩いていた時に、テレビ局からインタビューを受けました。	
\\	もし賢い愛人を見つけたいのなら、東京の銀座に行くのがいいかもね。	
\\	銀, 座	
\\	品物	
\\	しなもの	
\\	こちらのツイートへ、ほしい品物の名前を書いて返信してください。	
\\	この時間のスーパーには、品物がほとんど無い。	
\\	品物が到着次第、ご連絡差し上げます。	
\\	(しな) 
\\	品, 物	
\\	落書き	
\\	らくがき	
\\	する 
\\	コウイチの落書きは今やオークションで百万ドルで落札されるまでになった。	
\\	新宿駅の地下道には、昔、たくさんの落書きがしてありましたが、今はほとんど見当たりません。	
\\	スプレーでの落書きアートは既に少し時代遅れです。新しい芸術家達は、壁の汚れをゴシゴシ擦って洗い落としながらアートを創りだします。	
\\	書, 
\\	が 
\\	落, 書	
\\	手荷物	
\\	てにもつ	
\\	手荷物を忘れてませんか?	
\\	フロントで手荷物をあずかってもらえるよ。	
\\	税関と手荷物受取所を通るのに、絶対に三十分以上かかると思うよ。	
\\	荷物 
\\	荷物 
\\	手, 
\\	手, 荷, 物	
\\	幸い	
\\	さいわい	
\\	な 
\\	幸いなことに、ちょうど空車のタクシーがやって来ました。	
\\	お役に立てたようで、幸いです。	
\\	幸い命に別状はありませんでした。	
\\	(さいわ)?
\\	幸	
\\	高等	
\\	こうとう	
\\	な 
\\	の 
\\	ずい分高等な技術を持っているんだね。	
\\	うちの息子はまだ小学生ですが、じゅくではすでに高等の数学を勉強しています。	
\\	あなたの国の高等教育について教えて下さい。	
\\	高, 等	
\\	何枚	
\\	なんまい	
\\	ベーコンは美味しいから、何枚でも食べれるよ。	
\\	チケット、今のところ何枚ぐらい売れたの?	
\\	ステッカーは何枚欲しいんですか?大量買いされるなら割引できますよ。	
\\	何回 
\\	何年 
\\	何, 枚	
\\	不可欠	
\\	ふかけつ	
\\	な 
\\	ビーチに行く時は、日焼け止めが不可欠です。	
\\	日本語がペラペラになるのに不可欠なことは何だと思いますか?	
\\	美味しいベーコンは、サクサク仕事を進めるのに不可欠です。	
\\	不, 可, 欠	
\\	不等	
\\	ふとう	
\\	な 
\\	の 
\\	コウイチは、トーフグの利えきを不等に分配するのがいやなんだよ。	
\\	不等辺三角形の面せきを求める公式を知っていますか?	
\\	私は連立不等式が得意です。	
\\	不, 等	
\\	情報	
\\	じょうほう	
\\	やつらがベッドの上でざん殺されていたというのはだれからの情報ですか?	
\\	トーフグには、情報の取りあつかいきそくがありますか。	
\\	えーっと、ワニカニプロジェクトの件で情報を交換したいんですが、後ほどお伺いしてもよろしいですか。	
\\	情, 報	
\\	関西弁	
\\	かんさいべん	
\\	関西弁で、ウンチのことをババというらしいです。	
\\	関西弁で文句を言われると、かなりこわいよね。	
\\	彼、私に向かって、「お前のこと嫌いやねん」とか「帰れや」とか言ったのよ。ひどいでしょ?でもさ、実は関西弁で言われたもんだから、ちょっと格好いいなんて思っちゃったんだよね。何か、すごいセクシーだったの。	
\\	関西, 
\\	関西
\\	弁, 
\\	関, 西, 弁	
\\	種類	
\\	しゅるい	
\\	世の中には色んな種類の人がいるよね。	
\\	このバラとあのバラとでは、また種類がちがうんだよ。	
\\	ビエトは、どんな不測の事態にも対応できるように、あらゆる種類のプログラムをダウンロードしています。	
\\	種, 類	
\\	叩頭	
\\	こうとう	
\\	する 
\\	叩頭とは、いわゆる日本の土下座のことです。	
\\	部屋の中にいた全員が、王に向かって叩頭した。	
\\	その男は、叩頭を拒否したため、刑務所に入れられた。	
\\	頭 
\\	とうきょう. 
\\	とうきょう 
\\	叩, 頭	
\\	訓読み	
\\	くんよみ	
\\	する 
\\	この漢字は、訓読みでも音読みでも読めます。	
\\	この漢字は、訓読みすることができますか?	
\\	いつも訓読みと音読みを間違えてしまう。	
\\	読み 
\\	読む) 
\\	訓 
\\	くんよみ. 
\\	訓, 読	
\\	喉頭	
\\	こうとう	
\\	喉頭は、喉ぼとけの一部です。	
\\	喉頭ガンはたばこや飲酒が原因で起こることが多い。	
\\	喉頭炎で息をしたり、水を飲んだりするのさえ苦しい。	
\\	頭 
\\	とうきょう. 
\\	喉, 頭	
\\	教訓	
\\	きょうくん	
\\	する 
\\	コウイチの話から、どんな教訓がえられましたか?	
\\	この教訓を無だにはしません。	
\\	前職から学んだ最大の教訓を教えてください。	
\\	教, 訓	
\\	子守歌	
\\	こもりうた	
\\	クリステンの子守歌は、プロのいきに達している。	
\\	アメリカで一番有名な子守歌は何ですか?	
\\	子供の頃に歌ってもらった子守歌なんです。	
\\	守 
\\	(もり). 
\\	子, 守, 歌	
\\	人類	
\\	じんるい	
\\	人類の最大のてきは何だと思いますか?	
\\	お前は人類のはじだよ!	
\\	アヤは、ベーコンは人類が後世に残すべき最も重要な遺産の一つだと主張した。	
\\	人, 類	
\\	達人	
\\	たつじん	
\\	一つのことに全てをかけないと、真の達人にはなれないぞ。	
\\	ビエトは、カメラの達人です。	
\\	父は、ピザ配達の分野で、世界レベルの達人であると評判になっています。	
\\	達, 人	
\\	月曜日	
\\	げつようび	
\\	月曜日に、お風ろの水を止めるのを忘れてて、湯船からあふれて、家中水びたしになってしまったんだ。	
\\	受験料は、月曜日までにふりこんでください。	
\\	月曜日、熱々のロマンスとか、ゾッとするホラーみたいな感じの楽しい映画をあなたと一緒に観たいなって思うんだけど、どうかな?	
\\	曜日 
\\	曜日 
\\	月, 曜, 日	
\\	報道	
\\	ほうどう	
\\	する 
\\	同りょう達は、みんなその報道のことをすでに知っていました。	
\\	今回のテロのことは、日本ではどのように報道されていますか?	
\\	報道によると、本日早朝に起きた玉突き事故で、20匹の魚が地域の病院へ運ばれたそうだよ。	
\\	報, 道	
\\	禁煙	
\\	きんえん	
\\	する 
\\	の 
\\	ここが禁煙スペースだってこと、忘れてるわけないじゃない。	
\\	毎週月曜日は禁煙の日って決めているんです。	
\\	母親に禁煙するように言ったんですが、逆に母親から、禁煙したらおそらく気が狂って、父親と離婚することになると言われました。	
\\	禁, 煙	
\\	枚数	
\\	まいすう	
\\	折り紙の枚数を数えてもらえますか?	
\\	今コウイチのさいふの中にあるクレジットカードの枚数を当ててみてください。	
\\	ステッカーの枚数については、何枚くらい注文できるかはまだはっきり分かりませんが、少なくとも百枚にはなると思います。	
\\	枚, 数	
\\	大阪弁	
\\	おおさかべん	
\\	あの大阪弁の上司は、いつも前おきが長いので、うんざりします。	
\\	大阪弁のかれ氏とのかこは、もう忘れました。	
\\	「心配しないで。君が大阪弁を喋るってことは誰にも言わないよ。秘密は守るよ。」「おおきに!」	
\\	大阪 
\\	大阪 
\\	弁 
\\	大, 阪, 弁	
\\	書き方	
\\	かきかた	
\\	この漢字の書き方が思い出せない。	
\\	トーフグにはどうして書き方を教えるアプリが無いんですか?	
\\	平仮名の綺麗な書き方を勉強することには、心底飽き飽きしてるんだよね!	
\\	見方 
\\	書き方? 
\\	書く 
\\	方 
\\	書, 方	
\\	許可	
\\	きょか	
\\	する 
\\	ワニカニの例文を他で使いたいならマミの許可を取ってくださいね。	
\\	トーフグのコアなファンとのインタビューの許可が下りました。	
\\	もしコウイチと一緒に写真を撮りたいなら、ビエトの許可が必要です。	
\\	許, 可	
\\	〜達	
\\	たち	
\\	コウイチ達、一体どこに行ったんだろう。	
\\	あそこのギャング達は、みんなビエトの兄弟分です。	
\\	あなた達はまだ未成年でしょう。	
\\	たつ? 
\\	たち. 
\\	(たち) 
\\	達	
\\	可能	
\\	かのう	
\\	な 
\\	今年の春から、むすめを保育園にあずけることが可能だと聞いて、胸のつかえが下りました。	
\\	もう少し実げん可能なかい決さくをていあんしてください。	
\\	私は年間一億ドルを稼ぐことは実現可能だと思います。	
\\	可, 能	
\\	葉書	
\\	はがき	
\\	ちょっとその葉書をお借りしてもいいですか?	
\\	ゆう便局で葉書を買ったんだけど、お金をはらったあと、そのまま忘れてそこにおいてきちゃったみたい。	
\\	コウイチが実はとある秘密の国の王子様で、のっぴきならない理由のためポートランドオレゴンで身分を隠して
\\	の社長として生きているという奇妙な葉書を受け取った。	
\\	葉 
\\	は 
\\	書 
\\	かき 
\\	き 
\\	書き 
\\	書き 
\\	葉, 書	
\\	祈願	
\\	きがん	
\\	する 
\\	の 
\\	無事にビザが取れるよう、祈願しています。	
\\	コウイチは、火星人の戦勝祈願のおどりをおどることができます。	
\\	「サーモンの手術が上手くいきますように!」と、フグは神様に祈願した。	
\\	祈, 願	
\\	祈念	
\\	きねん	
\\	する 
\\	世界中の全ての人が、祈念しています。	
\\	明日は原ばくぎせい者いれい平和祈念式典に出席します。	
\\	の益々の発展と繁栄を祈念して、乾杯!	
\\	祈, 念	
\\	公告	
\\	こうこく	
\\	する 
\\	トーフグの決算公告、もう見ましたか。	
\\	トーフグは、日本せいふの官報にコウイチについての重大な公告をのせることにしました。	
\\	日本には、電子公告規則という法律がありますが、あなたの国はどうですか?	
\\	公, 告	
\\	静か	
\\	しずか	な 
\\	静かな通りにある家の方がいいです。	
\\	下りなのは分かりますが、もう少し静かにしていただけますか?	
\\	コウイチとビエトが乗り込むと、そのタクシーは静かに動き出した。	
\\	な 
\\	(しず) 
\\	静	
\\	人種	
\\	じんしゅ	
\\	の 
\\	この町には、色々な人種の人がいます。	
\\	人種をネタにしたじょうだんは、あんまり好きじゃありません。	
\\	僕のお父さんは人種差別主義者のくせに、僕には人種で人を判断してはいけないって言うんだ。	
\\	人, 種	
\\	洗練	
\\	せんれん	
\\	する 
\\	コウイチの洗練されたおもかげを、二度と忘れることはないでしょう。	
\\	コウイチは十月に行われる詩のコンテストのために下書き中の詩を洗練しようとしている。	
\\	サーモンがあんな風に洗練された日本語を話すなんて知らなかった。	
\\	洗, 練	
\\	親類	
\\	しんるい	
\\	の 
\\	まずは年配の親類に相談してみます。	
\\	実は、コウイチとオバマ大とうりょうは遠い親類なんですよ。	
\\	彼は全親類と縁を切った孤独な男だ。	
\\	親, 類	
\\	借家	
\\	しゃくや, しゃっか	
\\	する 
\\	借家をたん保にお金を借りることは出来ませんよ。	
\\	借家けんって、単じゅんに、「借家するけん利」のことでしょう?	
\\	家を買う代わりに借家を借りる提案をサーモンに切り出そうとしてるんだけど、彼女、その話題には触れないようにしてるみたいなんだよね。	
\\	家 
\\	(しゃくや) 
\\	しゃく 
\\	しゃっ. 
\\	しゃくや 
\\	借, 家	
\\	報告	
\\	ほうこく	
\\	する 
\\	コウイチに報告した方がいいんじゃない?	
\\	コウイチは、トーフグの年次報告を、社員たちに口頭で報告しました。	
\\	彼が逃げ出さないように、めちゃくちゃ面白い妊娠の報告方法を考えているの。	
\\	報, 告	
\\	焼き鳥	
\\	やきとり	
\\	焼き鳥は私の生活に不可欠です。	
\\	主人は焼き鳥屋のけいえいが忙しすぎて、けっこん記念日のことはすっかり忘れていました。	
\\	私のせいにしないで。あなたの焼き鳥なんか食べてないわよ!	
\\	焼き, 
\\	鳥 
\\	とり. 
\\	焼, 鳥	
\\	焼き肉	
\\	やきにく	
\\	焼き肉屋にカメラをおいてきちゃった。	
\\	去る者より残る者の方がつらいというのは、焼き肉を食べてる時には当てはまらないよね。	
\\	私が特にイライラするのは、焼き肉をクチャクチャ音を立てて食べる人です。	
\\	焼肉 
\\	焼き 
\\	肉 
\\	焼, 肉	
\\	下書き	
\\	したがき	
\\	する 
\\	いい記事を書くためには下書きをすることが大事です。	
\\	コウイチのオフィスのつくえの引き出しには、高校の時に書いた自作の詩の下書きが大切にしまってあるって知ってる?	
\\	アヤって、いつも最初に鉛筆で下書きしてるのかな。それとも、下書き無しで絵を描いてるのかな。	
\\	下 
\\	書く 
\\	下, 書	
\\	大丈夫	
\\	だいじょうぶ	
\\	な 
\\	この日焼け止めは、赤ちゃんに使っても大丈夫ですか。	
\\	「ハックション!」「大丈夫?」	
\\	いつでも大丈夫だよ。こっちは君の都合に合わせるから、お好きな時にどうぞ。	
\\	""大丈夫
\\	大, 丈, 夫	
\\	土曜日	
\\	どようび	
\\	土曜日のつかれはもうすっかり取れました。	
\\	土曜日に、美人は得をするということを実感しました。	
\\	クソ上司に土曜日も働かなきゃいけないと言われた上に、風邪をひいてしまい、さらには彼女にも振られたよ。俺の人生はクソだ。	
\\	曜日 
\\	曜日 
\\	土, 曜, 日	
\\	座席	
\\	ざせき	
\\	の 
\\	この座席のクッション、めっちゃいい感じ。	
\\	座席の予約が必要だったとは、知りませんでした。	
\\	車内で、サーモンが座席から身を乗り出してフグにキスをした瞬間、信号が青に変わった。	
\\	座, 席	
\\	正座	
\\	せいざ	
\\	する 
\\	何でそんなに長い間正座できるの?	
\\	正座させられると思っていましたが、仏僧のように心が広いコウイチは、私のちこくを許してくれました。	
\\	法要の間長時間正座をしていたので、足がすごく痺れています。	
\\	正, 座	
\\	煙	
\\	けむり	
\\	ポートランドの町は火と煙につつまれた。	
\\	ちょっと!バーベキューグリルから煙が出ているよ!!	
\\	ビエトは、コウイチのパソコンの向こうから立ち上っている煙を見て驚きました。何故なら、オフィスは禁煙だったからです。	
\\	(けむり), 
\\	煙	
\\	喫煙	
\\	きつえん	
\\	する 
\\	トーフグの社員には、喫煙する人が一人もいません。	
\\	喫煙室はどこにあるかご存知ですか?	
\\	やばーい!ちょっと面白すぎっしょ〜!何で鼻で喫煙できるなんて思っちゃったわけ?爆笑してるんですけど。	
\\	喫, 煙	
\\	汽船	
\\	きせん	
\\	この汽船は、あなたの国で何番目に大きい汽船ですか?	
\\	汽船を一台借り切って、シンガポールへ行きました。	
\\	汽船での世界一周旅行の費用なんて、どうやって捻出するんだよ?	
\\	汽, 船	
\\	汽車	
\\	きしゃ	
\\	その駅員は、汽車を静止させるのを忘れていたことに気が付きました。	
\\	大好きな汽車だったので、何枚も何枚も写真をとってしまいました。	
\\	「すみません。汽車は何分おきに来ますか?」「大体20分おきくらいです。」	
\\	電車
\\	汽, 車	
\\	静止	
\\	せいし	
\\	する 
\\	の 
\\	カナさんの言葉に胸をハッとつかれて、思わず静止してしまいました。	
\\	これはただの静止画像なのに、びみょうに動いて見える不思ぎな絵です。	
\\	弓道では、矢を放つ前に、弓を引ききった状態で一瞬静止する必要がありますが、西洋のアーチェリーも同じですか?	
\\	静, 止	
\\	飴細工	
\\	あめざいく	
\\	私の父は飴細工の名人です。	
\\	飴細工作ったことある?	
\\	その金魚の飴細工は、形も味も本物の金魚にそっくりなんだって。	
\\	工 
\\	(く)! 
\\	細, 
\\	飴, 細, 工	
\\	日焼け	
\\	ひやけ	
\\	する 
\\	どうやらうちの娘は、あの日焼けした青年への恋に胸をこがしているようです。	
\\	日焼けでせ中が真っ赤になってるよ。	
\\	ハワイでは強い日焼け止めが必需品で、人達を日焼けから守るのにシートで覆ったりもします。	
\\	ひ 
\\	日 
\\	や 
\\	焼け 
\\	日, 焼	
\\	日曜日	
\\	にちようび	
\\	日曜日、じゅ業参かんの後、親のみのこん親会があります。	
\\	日曜日に教会で、一人ずつ自こしょうかいをしました。	
\\	毎週日曜日は、彼女のサーモンとテニスをするとかアイスを食べるというようなことをする。	
\\	曜日 
\\	曜日 
\\	日, 曜	
\\	夕焼け	
\\	ゆうやけ	
\\	夕焼けを見ていると何だか虚しい気持ちになります。	
\\	私の友達はずっと朝焼けのことも夕焼けだと言ってはばからない。	
\\	昨日、海岸から見た夕焼けはとても眩しくて天国のようだったわ。そのせいで左目を失明しちゃったけどね。	
\\	夕 
\\	焼け 
\\	ゆうやけ.	夕, 焼	
\\	禁句	
\\	きんく	
\\	本当に申しわけないです。どうして禁句を言ってしまったんだろう。ごめんなさい。	
\\	コウイチが言った禁句が頭からはなれません。	
\\	ソフトボールチームに入る際の第二規則は、お腹がすきすぎちゃうので、昼食前には「ベーコン」という言葉は禁句だということを肝に銘じることです。	
\\	禁, 句	
\\	喫茶店	
\\	きっさてん	
\\	アイフォンがぬすまれちゃってさ。今、喫茶店の電話を借りてかけてるんだ。	
\\	喫茶店でコーヒーを飲んでたんだけど、レジの人がおつりをわたし忘れてて私もそれに気づかず出てきちゃったんだよね。	
\\	「去年、ブラジルの喫茶店で偶然コウイチに会ったよ。」 「本当? そう言えば、長いことあいつの顔見てないなぁ。」	
\\	きつ 
\\	きっ 
\\	茶, 
\\	さ 
\\	さ 
\\	ちゃ, 
\\	(さ) 
\\	喫, 茶, 店	
\\	僧院	
\\	そういん	
\\	この僧院では、良い食生活は健康に不可欠だと考えているので、僧りょ達は毎日美味しい料理を食べます。	
\\	映画「ゴジラ対キングコウイチ」のロケ地めぐりをしたんですが、この僧院も私がおとずれた場所の一つです。	
\\	初めまして。今日からこの僧院で働く予定になっているコウイチです。よろしくお願いいたします。	
\\	僧, 院	
\\	喉	
\\	のど	
\\	猫は、喉でゴロゴロという音を出します。	
\\	最近、喉がやせたんじゃない?	
\\	喉が痛いので、私の代わりに歌ってくれませんか?	
\\	こういち 
\\	(のど)
\\	喉	
\\	木曜日	
\\	もくようび	
\\	木曜日に、ガスパイプにヒビが入ってしまったんだ。	
\\	洗たくは、木曜日にしようと思ってます。	
\\	木曜日に、雪だるまを作るとかかまくらを作るみたいな、何か楽しいことをしたくない?	
\\	曜日 
\\	曜日 
\\	木, 曜, 日	
\\	告げる	
\\	つげる	
\\	恋人に別れを告げました。	
\\	今日の朝会で、コウイチは明日からじで入院することを社員に告げた。	
\\	ビエトはコウイチに、マミはみんなの想像の斜め上を行く変人だと告げた。	
\\	う 
\\	(つ) 
\\	告	
\\	達する	
\\	たっする	する 
\\	少しおくれたが、無事に目的地に達した。	
\\	コウイチのひげは、ついにこしまで達した。	
\\	目標人数に達するまで、あと少しだ。	
\\	達 
\\	する 
\\	達 
\\	たつ, 
\\	たっ. 
\\	達	
\\	焼く	
\\	やく	
\\	チーズケーキを百こ焼くという目標を達しました。	
\\	その手紙は焼いもと一しょに焼いてしまいました。	
\\	あなたの旦那さんも、ケーキを焼くの?	
\\	う 
\\	焼	
\\	借りる	
\\	かりる	
\\	アクを取りたいので、ちょっとお玉を借りてもいいですか?	
\\	クリステンのルーズリーフ、一枚借りたよ。	
\\	「ごめんね。君から借りていた恋愛小説をなくしちゃったんだ。」「気にしないでいいよ。それはもう必要ないから。彼女ができたんだ。」	
\\	う 
\\	(か)? 
\\	借	
\\	切り取る	
\\	きりとる	
\\	このキャベツのしんを切り取ってもらえますか?	
\\	今日は、一枚の紙から、星の形を切り取る方法をお教えします。	
\\	使わないのにどうしていちいち新聞から割引券を切り取るのさ?	
\\	切る 
\\	取る. 
\\	切, 取	
\\	配達する	
\\	はいたつする	する 
\\	コウイチは、ぜっ対に六時までに配達してみせると言って、胸をたたきました。	
\\	この花を今日中に配達しなければいけないのを、すっかり忘れていました。	
\\	僕のお父さんはピザの配達人です。誰よりも速くピザを配達することができます。	
\\	配, 達	
\\	忘れる	
\\	わすれる	
\\	ごめんね、カナエちゃん。みんなカナエちゃんがどれだけ努力したのかすっかり忘れちゃってたんだ。	
\\	危うくコウイチの顔を忘れる所だったよ。	
\\	もう忘れなって。過ぎてしまったことは今さらどうにもならないんだから。	
\\	う 
\\	(わす) 
\\	忘	
\\	固まる	
\\	かたまる	
\\	みんな何でそんなすみっこに固まってるの?	
\\	ようやくトーフグをやめる決心が固まりました。	
\\	そろそろプリンが固まる頃かしら。	
\\	固める 
\\	固	
\\	許す	
\\	ゆるす	
\\	重大発表があります。今この壊れたエレベーターの中でオナラをしてしまいました。みなさん、どうかお許しください。	
\\	ここはしゅりょうが許されている地いきだと思っていました。	
\\	の社員は、月曜日は青色、茶色、鼠色などの地味な色の服を着る事しか許されません。	
\\	う 
\\	許さない 
\\	(ゆる) 
\\	許	
\\	取り出す	
\\	とりだす	
\\	天才外科医が、コウイチの脳を取り出して、ビエトにい植しました。	
\\	黒い服を着た男が、マイケルのかばんからさいふを取り出すのを見たよ。	
\\	オーブンからフグを取り出しておいてくれない?あと五分で着くから。	
\\	取る 
\\	出す. 
\\	取, 出	
\\	祈る	
\\	いのる	
\\	ハリケーンのひがい者のために祈りました。	
\\	国家試験に無事合格できるように祈っています。	
\\	日本の国技である相撲が、かつては米の豊作を祈る儀式だったことを知ってる日本人がたくさんいるとは思えないな。	
\\	う 
\\	(いの), 
\\	祈	
\\	禁じる	
\\	きんじる	
\\	これは、コウイチが自分の禁じられた恋について書いた私小説です。	
\\	医者は、にんぷにアルコールときつ煙を禁じた。	
\\	この部屋では、私語を禁じる。	
\\	う 
\\	禁	
\\	報じる	
\\	ほうじる	
\\	スーパーマンの死去が報じられ、全米が涙した。	
\\	コウイチの無事が報じられて、胸をなで下ろしました。	
\\	は、初音ミクの独占取材を報じた。	
\\	う 
\\	報	
\\	書き直す	
\\	かきなおす	
\\	主人公をコウイチからビエトに変こうして、この話を書き直してください。	
\\	試験終了間ぎわになって、受験番号が間ちがっているのに気づいて、あわてて書き直しました。	
\\	「これを書き直す時間があると思う?」「私はそうは思わないね。」	
\\	書く 
\\	直す. 
\\	書, 直	
\\	座る	
\\	すわる	
\\	大事な書るいの上に座らないでもらえますか。	
\\	コウイチとビエトはいつもブーブークッションでいたずらするのでオフィスのいすに座る時は気をつけている。	
\\	ずっと座っていると、お尻が汗ばんできます。	
\\	う 
\\	(すわ). 
\\	座	
\\	叩く	
\\	たたく	
\\	ドラムを叩く才能がありますね。	
\\	開ける前に、ドアを叩いてって言ったよね!?	
\\	釘バットで叩かれるのって、本当に痛い!	
\\	う 
\\	(たた) 
\\	たた! 
\\	叩	
\\	洗う	
\\	あらう	
\\	家に帰ったら手を洗ってうがいをしてください。	
\\	ビエトは大切にはいていたビンテージのジーンズをかの女に洗われたショックで今日は休んでいます。	
\\	ああ、私達は無洗米を使ってるので、洗う必要は無いですよ。	
\\	う 
\\	(あら) 
\\	洗	
\\	書き入れる	
\\	かきいれる	
\\	この書類のここに社会ほしょう番号を書き入れてもらえれば、あとは全て上手くやっておきますよ。	
\\	けっこんしょう明書に名前を書き入れました。	
\\	日本語の学習予定を書き入れる
\\	手帳をコウイチに作ってもらいたいな。	
\\	書く 
\\	入れる. 
\\	書, 入	
\\	訓練	
\\	くんれん	
\\	する 
\\	昨日、トーフグのオフィスで、火さい訓練をしました。	
\\	今日は学校で、おはしの使い方を訓練しました。	
\\	日本語の訓練所を破壊する秘密組織があると聞きました。	
\\	訓, 練	
\\	駅弁	
\\	えきべん	
\\	いそがしすぎて、駅弁を買うのを忘れてました。	
\\	日本国内の旅行の楽しみの一つは、色々な駅弁を食べくらべることです。	
\\	あっ、おにぎりを持ってくるのを忘れたの?じゃあ、お昼は駅弁を買わなくちゃいけませんね。	
\\	弁当 
\\	弁 
\\	駅, 弁	
\\	伝達	
\\	でんたつ	
\\	する 
\\	伝えたかったことが、ちゃんと伝達されたのか心配です。	
\\	伝達事こうは以上です。	
\\	私は、人々が新しい言語を学んでいる時の、ニューロンの信号伝達について研究しています。	
\\	伝, 達	
\\	火曜日	
\\	かようび	
\\	ある晴れた火曜日の朝、家の花に水やりをしていると、白いひげを生やしたおじいさんに話しかけられました。	
\\	すみませんが、来週の月曜日と火曜日はいそがしいんです。	
\\	火曜日は野球でコテンパにやられたよ。	
\\	曜日 
\\	曜日 
\\	火, 曜, 日	
\\	警告	
\\	けいこく	
\\	する 
\\	の 
\\	借りた金を早く返すよう警告した。	
\\	あの選手はどの試合でも必ず警告のイエローカードを出されるね。	
\\	警告を無視するつもりなのか?	
\\	警, 告	
\\	水曜日	
\\	すいようび	
\\	水曜日にするべきことは、大きく分けて三つあります。	
\\	毎週水曜日は、学校の後プールに行って泳いでいます。	
\\	水曜日の夜、彼はかなり酔っていて、トラックに轢かれてしまったんだ。	
\\	曜日 
\\	曜日 
\\	水, 曜, 日	
\\	忘年会	
\\	ぼうねんかい	
\\	今日の忘年会であった出来事は、忘れる事にします。	
\\	元々忘れっぽいんで、私には「忘年会」は必要ありません。	
\\	多くの日本企業では、新入社員は忘年会で一発芸を披露しなくてはならない。	
\\	忘, 年, 会	
\\	座禅	
\\	ざぜん	
\\	座禅は、そこでクライマックスに達した。	
\\	京都に旅行に行ったら、ぜひ座禅の体験をしてみたいなあ。	
\\	私の友達はお坊さんで、自分のお寺で座禅の会を開いています。彼は、座禅によって、リラックスできるだけでなく、欲望や執着心が無くなる悟りの境地に至ることもできると教えてくれました。	
\\	座, 禅	
\\	禅僧	
\\	ぜんそう	
\\	その禅僧は、すきっぱらに不味いものは無いという言葉を実感しました。	
\\	努力は全て無だに終わったが、禅僧は少しもはらが立たなかった。	
\\	もし禅僧になりたければ、とんでもなくきつい修行を終える必要があります。	
\\	禅, 僧	
\\	禅寺	
\\	ぜんでら, ぜんじ	
\\	あの禅寺の名前は何だっけ?	
\\	あの禅寺、さいむちょうかの会社のかぶに投しをして、痛い目にあったらしいよ。	
\\	天龍寺という、十四世紀に京都市西部に建てられた禅寺が好きです。	
\\	禅 
\\	寺 
\\	てら 
\\	でら, 
\\	ぜんじ, 
\\	禅寺 
\\	ぜんじ. 
\\	禅, 寺	
\\	学歴	
\\	がくれき	
\\	日本やかん国は学歴社会ですが、アメリカはちがうんですか?	
\\	学歴はほとんど無いけど、立ぱな人もいるよ。	
\\	「お母さん、私ね、今日もテストで満点とったの。」「それがどうしたのよ?あんた本気で学歴が全てだと思ってるわけ?」	
\\	学, 歴	
\\	形容詞	
\\	けいようし	
\\	形容詞は、な形容詞、い形容詞、の形容詞の三つに分けられます。	
\\	このシチュエーションにぴったりのいい形容詞が思いつきません。	
\\	その面接試験で、自分を表現するのに最も適した形容詞を選んで、その理由を説明しなくちゃいけなかったんだ。	
\\	形, 容, 詞	
\\	五十音順	
\\	ごじゅうおんじゅん	
\\	町名は五十音順にならんでいます。	
\\	この名ぼでは、学生の名前は名字で五十音順に記さいされています。	
\\	日本で初めて五十音順に配列された国語辞典は、『和訓栞』です。	
\\	あいうえお 
\\	五, 十, 音, 順	
\\	猫舌	
\\	ねこじた	
\\	私が猫舌だってこと、忘れてない?	
\\	ごめんなさい。私、猫舌なんです。	
\\	コウイチは猫舌なので、スープをふうふうしないと食べられません。本当でしょうか嘘でしょうか?	
\\	猫 
\\	舌 
\\	ねこじた. 
\\	舌 
\\	じた, 
\\	した.	猫, 舌	
\\	信徒	
\\	しんと	
\\	トーフグの信徒数は一万人をこえました。	
\\	かん国ではキリスト教の信徒の方が仏教の信徒よりも多いって知っていましたか。	
\\	私の祖父は、オウム真理教の敬虔な信徒でした。	
\\	信, 徒	
\\	得	
\\	とく	
\\	な 
\\	お得なクーポン冊子はいかがですか?	
\\	そんな事して、一体何の得があるっていうの?	
\\	ルックスがいい人は絶対に得だと思います。	
\\	とく!	得	
\\	毛布	
\\	もうふ	
\\	母が、毛布を忘れないようにすることを思い出させてくれました。	
\\	その毛布、ダンボール箱に入れてもらっていい?	
\\	毎日毛布を百枚も洗うなんて想像もつかないよ。	
\\	毛, 布	
\\	〜冊	
\\	さつ	
\\	市立図書館には、三万三千三百冊以上のぞう書があります。	
\\	友達にもあげたかったので、その本を十冊買いました。	
\\	すみません、ちょっとお声が遠いようなのですが。日本語の教科書が何冊いるとおっしゃいましたか?	
\\	冊	
\\	仏教徒	
\\	ぶっきょうと	
\\	この国ではキリスト教徒が多いので、仏教徒はかた身のせまい思いをしています。	
\\	その仏教徒が、仏様に向かって舌を出して「ベーッ」と言っているのを見てしまいました。	
\\	どうして日本人は、ほとんどの人が仏教徒なのに、クリスマスをお祝いするの?	
\\	仏教 
\\	徒 
\\	仏, 教, 徒	
\\	多忙	
\\	たぼう	
\\	な 
\\	最近多忙でつかれていて、よく舌がもつれるんです。	
\\	多忙な生活を送っています。	
\\	ご多忙な所申し訳無いんですが、少しの間代わりに列の順番を取っておいてもらえませんか?	
\\	多, 忙	
\\	忙しい	
\\	いそがしい	い 
\\	忙しくて、頭がクラクラしています。	
\\	コウイチはとても忙しいので、コウイチからすぐメールの返事がくるとは思わないでください。	
\\	今日は忙しかったね。なんてったって、今日だけで千語も新しい日本語を学んだんだから。	
\\	い 
\\	(いそが) 
\\	忙	
\\	年代順	
\\	ねんだいじゅん	
\\	の 
\\	マイケルが、ビエトとヤクザの関係について、年代順に細かく説明してくれました。	
\\	記事の年代順リストとかってやっぱり作った方がいいのかな?	
\\	年代順の詳細な経過については、こちらの年表をご参照ください。	
\\	年, 代, 順	
\\	布	
\\	ぬの	
\\	その布、新しい布に変えてもらえるかな?	
\\	布が絡んじゃって、チャックがうまく開かないのよ。	
\\	布おむつの方が紙おむつよりも健康にいいと聞いたことがあります。	
\\	(ぬの) 
\\	布	
\\	連中	
\\	れんちゅう, れんぢゅう, れんじゅう	
\\	うちの息子は、最近悪い連中と付き合っているみたいで、少し心配です。	
\\	よっぱらい連中には近づかない方がいいぜ。	
\\	あの連中はいつもくだらない話ばかりしている。	
\\	連, 中	
\\	混乱	
\\	こんらん	
\\	する 
\\	はん人は、混乱にまぎれてにげてしまった。	
\\	その町は、山火事で大混乱となっていた。	
\\	申し訳ないけど、そんな中途半端なやり方じゃ新入社員が混乱しちゃう気がします。	
\\	混, 乱	
\\	内容	
\\	ないよう	
\\	トーフグチームは、コウイチの話の内容に細かい注意をはらいました。	
\\	コウイチのプレゼンテーションには、いつも内容があるようで無いよう。	
\\	一部の
\\	ファンは、内容と正確さについて非常に厳しいことでよく知られています。	
\\	内, 容	
\\	対比	
\\	たいひ	
\\	する 
\\	りょうしは、二ひきの魚を対比してみました。	
\\	日本の生活とアメリカの生活じゃ、いい対比にならないよ。	
\\	誰がコウイチの鼻の写真を撮ったの?その光と影の対比がすごく好きなんだけど。	
\\	対, 比	
\\	借財	
\\	しゃくざい	
\\	する 
\\	返す目どが立っていないのに、借財をするのは良くないよ。	
\\	トーフグが多がくの借財をかかえているというのは本当ですか?	
\\	フグが暗殺された後、サーモンに残されたのは借財だけだということが明らかになった。	
\\	ざい 
\\	さい, 
\\	借, 財	
\\	大きい順	
\\	おおきいじゅん	
\\	うちの子は、何でも大きい順に食べるんです。	
\\	次の国を面せきが大きい順にならびかえなさい。	
\\	モデル達の写真を顔が大きい順に並べました。	
\\	大きい 
\\	順
\\	大, 順	
\\	細かい	
\\	こまかい	い 
\\	じゃあ、このキャベツを細かくきざんでください。	
\\	細かいお金しかありません。	
\\	木を見て森を見ず、だぜ。細かいことにばかり気をとられずに、もっと大きいスケールで物を考えてくれよ。	
\\	細い 
\\	細 
\\	かい 
\\	(こま).	細	
\\	劇団	
\\	げきだん	
\\	練習の後、劇団員はみんな汗だくだった。	
\\	劇団四季に入るのが私の目標です。	
\\	ドラマでとある劇団に所属する俳優を好きになってから、彼の出演する舞台を観に劇場にまで足を運ぶようになりました。	
\\	劇, 団	
\\	容易	
\\	ようい	
\\	な 
\\	新人だからといって、容易な仕事ばかりをやらされるのはいやです。	
\\	こうなることは容易に予想できただろう。	
\\	日本語学習は容易ではないよ。必死で頑張らないといけないよ。	
\\	容, 易	
\\	易しい	
\\	やさしい	い 
\\	肉をうすく切るのはあまり易しいことではありません。	
\\	きめの細かい肌を保つのは、コウイチにとっては易しいことだった。	
\\	試験の時、易しい問題から先に解いてしまう方が好きなんです。	
\\	い 
\\	(やさ) 
\\	易	
\\	八冊	
\\	はっさつ	
\\	学生は八人だから、教科書は八冊必要です。	
\\	お金がなくていすが買えないので、本を八冊つみ上げていすの代わりにしています。	
\\	小説をたった八冊出版したからって、自分の事を小説家だなんて呼びませんよ。五十冊出版すれば、もしかするとそう呼ぶかもしれないけどね。	
\\	はち 
\\	はっ. 
\\	冊 
\\	八, 冊	
\\	非常口	
\\	ひじょうぐち	
\\	非常口はそこの通路を左に行ったところにあります。	
\\	不良高校生がクラスをサボってくつろぐ場所と言えばふ通は非常口のかいだんの下ですよね?	
\\	コートの下にアヤが描いた絵を隠すと、泥棒は非常口からこっそりと抜け出した。	
\\	非常 
\\	口 
\\	入り口, 
\\	非常 
\\	口 
\\	非, 常, 口	
\\	連日	
\\	れんじつ	
\\	連日車を乗り回していたら、タイヤがすり減ってしまいました。	
\\	連日、この時間になるとねむくなるんです。	
\\	祭りは連日大賑わいだ。	
\\	連, 日	
\\	笛	
\\	ふえ	
\\	その笛には、ツタが絡まっていました。	
\\	このはと笛、お母さんがおまつりで買ってくれたの。	
\\	あの笛を持っている男の人がコウイチさんだよ。	
\\	笛	
\\	二枚舌	
\\	にまいじた	
\\	の 
\\	政治家達の二枚舌の政さくにはもううんざりだよ。	
\\	あいつは二枚舌だから気をつけた方がいいよ。	
\\	クイズ: ビエトは等身大の初音ミク人形を買うお金を儲けるために、二枚舌を使ってコウイチを騙しました。
\\	か×か?	
\\	舌 
\\	じた, 
\\	した.	二, 枚, 舌	
\\	布団	
\\	ふとん	
\\	この布団の品しつには、いくらか改善が見られると思います。	
\\	はずかしすぎて布団を頭からかぶってかくれたいよー!	
\\	布団の勧誘って、何であんなにしつこいのかなぁ。何度来られても買わないものは買わないっつうの。ほんと、鬱陶しい!	
\\	団 
\\	布, 団	
\\	善悪	
\\	ぜんあく	
\\	善悪を見きわめることって、大人でもむずかしいですよね。	
\\	一才をすぎたころから、だんだん善悪が分かるようになってきました。	
\\	彼は赤ちゃんなので、まだ善悪の区別がつきません。	
\\	善, 悪	
\\	得意	
\\	とくい	
\\	な 
\\	の 
\\	一番得意な料理は何ですか。	
\\	ダリンはプログラミングより山登りの方が得意です。	
\\	コウイチもビエトも、サルサはあまり得意ではない。	
\\	得, 意	
\\	助詞	
\\	じょし	
\\	この助詞についての説明は、細かい点がいくつかしょうりゃくされています。	
\\	今日、助詞の使い方を学校で習ったんですが、まだ頭が混乱しています。	
\\	日本語の文法で、修得するのが最も難しいもののうちの一つが助詞です。	
\\	助, 詞	
\\	集団	
\\	しゅうだん	
\\	集団で行動するのが苦手なんです。	
\\	その国に行くなら、テロリスト集団に気をつけなくちゃいけませんよ。	
\\	コウイチは終盤で集団を抜け出し、ホノルルマラソンで優勝した。	
\\	集, 団	
\\	歴史	
\\	れきし	
\\	の 
\\	歴史を勉強しておくと、しょう来得をするよ。	
\\	そのころは、高校の歴史の先生になりたいと思っていたんです。	
\\	の歴史について簡単に説明させてください。	
\\	歴, 史	
\\	改善	
\\	かいぜん	
\\	する 
\\	の 
\\	この絵はまだ改善のよ地があると思うんだよね。	
\\	コウイチは、トーフグの体しつ改善を図っています。	
\\	私達のサイトに関して、何か改善すべき点などがございましたら、どうぞご連絡くださいませ。	
\\	改, 善	
\\	小さい順	
\\	ちいさいじゅん	
\\	すみませんが、きゅう料はせが小さい順にわたしてくれますか。	
\\	トーフグの社員を口が小さい順にならべたら一番になれる自信があります。	
\\	空欄を埋めて、数字を小さい順に並び替えなさい。	
\\	小さい 
\\	順.	小, 順	
\\	自動詞	
\\	じどうし	
\\	自動詞「進む」の可能形は、「進める」です。	
\\	私が大の苦手としていることの一つに、他動詞と自動詞の使い分けがあります。	
\\	「受動態にできるかどうか?」で判断すると、自動詞と他動詞が簡単に見分けられますよ。	
\\	付く 
\\	消える, 
\\	自, 動, 詞	
\\	いい加減	
\\	いいかげん	な 
\\	このサービスはいい加減すぎます。いい加減、改善をするべきですよ。	
\\	あんないい加減な人とけっこんして二十年もけっこん生活が続けられるなんて信じられない。	
\\	ねえ、お父さん。お母さんさぁ、近所の井戸端会議で仕入れたいい加減な情報に踊らされるの、いい加減に止めるべきだと思わない?	
\\	いい 
\\	加, 減	
\\	入団	
\\	にゅうだん	
\\	する 
\\	入団するための申しこみ書は、細かい活字でいんさつされていました。	
\\	その教団に入団しても、何の得にもならないよ。	
\\	十月二十三日金曜日、
\\	野球チームは、
\\	オフィスで入団テストを実施します。	
\\	入, 団	
\\	喜び	
\\	よろこび	
\\	ニッカはシッポをふって喜びを表します。	
\\	ついにワニカニを正式にローンチすることができ、コウイチは喜びで顔をかがやかせていました。	
\\	喜びを表す、面白い顔文字を探しています。	
\\	喜ぶ 
\\	喜	
\\	昆布	
\\	こんぶ, こぶ	
\\	昆布でだしを取りました。	
\\	昆布と若布のちがいを教えてください。	
\\	「よくも私に対して昆布を投げつけてくれたわね。」「ごめん。そんな大ごとだとは思わなかったんだよ。」	
\\	布 
\\	ぶ, 
\\	(ぶ).	昆, 布	
\\	恋人	
\\	こいびと	
\\	私の恋人は、舌がこえているので、料理にはうるさいです。	
\\	恋人とあい人は意味が全ぜんちがうので使い方を間ちがえないようにしてくださいね。	
\\	「先月、恋人と別れたんだ。」「大丈夫だよ。お前には他にもっといい子がいるよ。」	
\\	恋 
\\	人 
\\	こいびと. 
\\	恋, 人	
\\	乱戦	
\\	らんせん	
\\	あの試合はまれに見る乱戦だった。	
\\	コウイチのスピーチのせいでかん客の間に乱戦が起こってしまった。	
\\	セール初日、モールは人でごった返しになり、まるで大乱戦のようだった。	
\\	乱, 戦	
\\	宙	
\\	ちゅう	
\\	コウイチのヘリコプターのラジコンは宙をまって地面にげきとつした。	
\\	ポケットから札束を取り出し、宙にバラまく時のコウイチの顔が好きだ。	
\\	あの問題は宙に浮いたままで、誰も議論しようとしない。	
\\	宙	
\\	混血	
\\	こんけつ	
\\	する 
\\	の 
\\	動物は、別の種が混血することはめったにありません。	
\\	コウイチは日本人とアメリカ人の混血です。	
\\	日本人の新生児の三十人に一人は混血だそうですね。	
\\	混, 血	
\\	名詞	
\\	めいし	
\\	この名詞は、の形容詞になりますか。	
\\	これは可算名詞ですか、不可算名詞ですか?	
\\	このゲームでは固有名詞は使えませんよ。	
\\	名, 詞	
\\	原子力	
\\	げんしりょく	
\\	の 
\\	今は原子力の是非について話をしているんだ。関係無い話を持ち出して頭を混乱させないでくれ。	
\\	私の父は、以前は原子力発電所で仕事をしていました。	
\\	コウイチはゴジラと戦うために、原子力で稼働する
\\	ロボットを作った。	
\\	(原子) 
\\	原, 子, 力	
\\	数詞	
\\	すうし	
\\	日本語には数詞がたくさんありすぎて、ちょっとパニクってます。	
\\	コウイチは、どうやって数詞を覚えたんですか?	
\\	日本語の数詞が勉強できるゲームがあればいいのに。	
\\	数, 詞	
\\	続々	
\\	ぞくぞく	
\\	オフィスの出口から社員が続々と出てきた。	
\\	ワニカニのサブスクリプションを申し込みたいという人が続々と集まってきて、うれしいけどちょっと困っています。	
\\	ワニカニに新しい単語が続々と登場する日が待ち遠しい。	
\\	続, 々	
\\	一冊	
\\	いっさつ	
\\	このざっしは一冊四百八十円です。	
\\	これは、世界に一冊しかない絵本です。	
\\	あなたのために旅の間に読める本を一冊買っておいたの。タイトルは『地図』よ。	
\\	一 
\\	いっ 
\\	一, 冊	
\\	順番	
\\	じゅんばん	
\\	コウイチは順番に
\\	に返信をします。	
\\	今、スーパーで支はらいの順番を待っている所です。	
\\	押さずに順番を待ってください!	
\\	順, 番	
\\	乱交	
\\	らんこう	
\\	する 
\\	の 
\\	かの女は、乱交のと中で、舌をだらりと出して気ぜつしてしまったんです。	
\\	いい加減、乱交するのは止めなよ。みんなドン引きだよ。	
\\	「何か今日の俺、マジでイケてる。ひょっとしたらアイツらの乱交パーティにも誘われちゃうかも。」「ははは〜。そうだといいね。」	
\\	乱, 交	
\\	宇宙	
\\	うちゅう	
\\	宇宙での生活には、いい加減あきてしまいました。	
\\	宇宙の果てには何があると思いますか。	
\\	フグ、サーモンはあなたが復縁したいって知らないのよ。 もし知ったら、彼女も違うように感じるかもしれないわ。もしかしたら、他の魚と月を見に宇宙へなんて行かないかもしれないわよ。	
\\	宇, 宙	
\\	参加	
\\	さんか	
\\	する 
\\	の 
\\	やっと順番がまわってきて、ゲームに参加することができました。	
\\	あなたもトーフグの忘年会に参加しませんか。きっと楽しいですよ。	
\\	久々の飲み会参加でエンジョイしちゃってまーす。今日は超リア充って感じだよ〜。	
\\	参, 加	
\\	品詞	
\\	ひんし	
\\	品詞にはどんなものがあるか教えてください。	
\\	品詞についての本を書いたんですが、今、どうすればたくさんの冊数を売ることができるのか考えています。	
\\	この文章を品詞に分解しなさい。	
\\	品, 詞	
\\	履歴書	
\\	りれきしょ	
\\	アルバイトにおうぼするために、履歴書を作成しました。	
\\	あの教じゅは履歴書さぎで首になってしまったよ。	
\\	フグは、自分が非常に優秀な魚だということが物凄く目立つ履歴書を作成した。	
\\	履, 歴, 書	
\\	説得	
\\	せっとく	
\\	する 
\\	これ以上説得のよ地は無いよ。	
\\	コウイチを説得するのは岩を手でわるのよりもむずかしいと言われています。	
\\	日本に住むために、まず最初に自分の親を説得しなくてはいけません。	
\\	説 
\\	せっ.	説, 得	
\\	梅干	
\\	うめぼし	
\\	梅干は乗り物よいにきくんですよ。	
\\	梅干は何に混ぜて食べても美味しい。	
\\	すみません、 前を通らせてもらってもよろしいでしょうか。あの梅干しが入った小瓶を買いたいので。	
\\	うめ 
\\	ぼし, 
\\	干す 
\\	し 
\\	干 
\\	梅, 干	
\\	財閥	
\\	ざいばつ	
\\	の 
\\	彼はこの財閥のあと取りです。	
\\	おそかれ早かれ、
\\	は財閥として日本のビジネス界に君りんするでしょう。	
\\	日本には、第二次世界大戦前に三井財閥、三菱財閥、住友財閥、安田財閥などの財閥がありました。	
\\	財, 閥	
\\	警官	
\\	けいかん	
\\	の 
\\	その警官の名前は、舌をかみそうな名前だった。	
\\	警官は、いい加減な返事をして言いのがれました。	
\\	「昨夜、フグと別れたの。」「サーモン、何があったの?」「フグが、私が警官だってことを知って、それで、もう私とは会わないって言われたの。」「そんなに簡単に?」「ええ。あっけないものね。」	
\\	警察 
\\	警察 
\\	警, 官	
\\	警察署	
\\	けいさつしょ	
\\	警察署へ行って、何の得があるの?	
\\	財布をひろったので警察署にとどけたら、指もんをとられました。	
\\	日本の警察署に英語が喋れる人が常駐してたらいいのにな。	
\\	(警察) 
\\	警, 察, 署	
\\	私財	
\\	しざい	
\\	コウイチの私財がいくらあるか知っていますか。	
\\	私は仏僧になるために私財を全部すてて京都へ旅立った。	
\\	コウイチは、私財を投じて河豚毒研究センターを作った。	
\\	私, 財	
\\	比例	
\\	ひれい	
\\	する 
\\	芸能人は比例代表選のシステムを利用して政治家になるケースが多い。	
\\	コウイチの美しさに比例してワニカニユーザーの数もふえるってこと、ちゃんと自覚してくださいね。	
\\	外国語能力は、いつでも時間変化率に比例するという訳ではない。	
\\	比, 例	
\\	順位	
\\	じゅんい	
\\	われわれの順位は重々しょう知しております。	
\\	日本語の試験の順位が発表されました。	
\\	順位なんかいちいち気にしてたらやってられないよ。	
\\	順, 位	
\\	果たして	
\\	はたして	
\\	この決だんは果たして正しかったのだろうか。	
\\	外に出ると、果たして見覚えのある風けいが広がっていた。	
\\	あのカフェでコウイチに会えるかもしれないと思っていたら、果たしてその通りになった。	
\\	(は) 
\\	果	
\\	関連	
\\	かんれん	
\\	する 
\\	の 
\\	関連の仕事をしています。	
\\	この問題に関連したメールは、全部コウイチに回してください。	
\\	大気汚染と異常気象の関連を調査する仕事をしています。	
\\	関, 連	
\\	改正	
\\	かいせい	
\\	する 
\\	の 
\\	四月から、
\\	山手線のダイヤが改正されます。	
\\	インターネットの改正料金、もうチェックしましたか?	
\\	市民団体は、公共の場でおならを禁止する法律の改正を求めた。	
\\	改, 正	
\\	暴走	
\\	ぼうそう	
\\	する 
\\	フグは最近ちょっと暴走気味だよね。	
\\	暴走しているコウイチをビエトはやさしくつつみこんだ。	
\\	暴走したトラックが、トーフグオフィスに突っ込んで来る夢を見た。	
\\	暴, 走	
\\	悲しい	
\\	かなしい	い 
\\	それは悲しすぎるね。	
\\	悲しいけど、今ので運を使い切ってしまった気がする。	
\\	桃ちゃんが悲しい気持ちの時は、お父さんもお母さんも辛い気持ちになってしまうのよ。	
\\	い 
\\	悲しむ, 
\\	悲	
\\	悲しみ	
\\	かなしみ	
\\	コウイチは、チーズケーキを失った悲しみで、気もくるわんばかりだった。	
\\	ポチが死んだという知らせを聞いて、悲しみに打ちひしがれていました。	
\\	悲しみで食欲が全くわかなかった。	
\\	悲しむ, 
\\	悲	
\\	季節	
\\	きせつ	
\\	の 
\\	このマーケットには、季節の野さいがたくさん売っています。	
\\	その服そうは、ちょっと季節外れじゃない?	
\\	心配しないで。季節の変わり目に風邪をひくのは、私の習慣のようなものなの。	
\\	季節.	
\\	季, 節	
\\	節句	
\\	せっく	
\\	もうすぐももの節句なので、ひな人形を出しました。	
\\	五節句を全部言えますか。	
\\	「端午の節句」と呼ばれる日本の男の子の日に欠かせないのが、鯉をかたどった「鯉のぼり」と呼ばれる吹き流しです。	
\\	せつ 
\\	せっ 
\\	つ
\\	節, 句	
\\	席順	
\\	せきじゅん	
\\	あいうえお順にすれば、容易に席順を決められますよ。	
\\	席順はクジで決めたいと思いますが、目の悪い人は先に申し出てください。	
\\	月曜日に席順を変更したいと思います。	
\\	席, 順	
\\	別冊	
\\	べっさつ	
\\	別冊の付ろくをいつも楽しみにしています。	
\\	かつて、別冊少年ジャンプというざっしがありました。	
\\	1985年、雑誌
\\	の別冊本に、生まれたばかりのコウイチの写真が掲載されました。	
\\	べつ 
\\	べっ, 
\\	つ 
\\	つ
\\	別, 冊	
\\	連続	
\\	れんぞく	
\\	する 
\\	この乗り放題切ぷは、連続する三日間のみ有こうです。	
\\	朝のニュースは、連続殺人事けんの話で持ち切りでした。	
\\	君との人生は本当に驚きの連続だよ。まさか三百六十五日連続で遅刻するなんて思ってもみなかったよ。	
\\	連, 続	
\\	歌詞	
\\	かし	
\\	この歌詞はだれが書いたの?	
\\	とっても若者らしい歌詞ですね。	
\\	のテーマ曲の歌詞は、とても独創的でユニークでなきゃね。	
\\	カラオケ?	
\\	歌, 詞	
\\	汽笛	
\\	きてき	
\\	まどの外から、汽車の汽笛が聞こえてきました。	
\\	いいか?汽笛が鳴ったら、船から飛び下りるんだぞ。	
\\	昨日の夢で、幸運にもタイタニック号が汽笛を鳴らすところを見たよ。	
\\	汽, 笛	
\\	生徒	
\\	せいと	
\\	の 
\\	その生徒は、先生に向かって舌打ちをしました。	
\\	高校までは生徒のことを「生徒」とよぶけど、大学では学生のことを「生徒」とはよばないよ。	
\\	私の趣味は、教師と生徒の禁断の愛をテーマにした本を読むことです。	
\\	先生 
\\	生徒, 
\\	生, 徒	
\\	減法	
\\	げんぽう	
\\	先日、加法をマスターしたので今日からは減法を学びます。	
\\	私の算数で一番得意な分野は減法です。	
\\	物理の問題で、ベクトル減法はあまり見ませんが、たまに出てくることがあります。	
\\	ほう 
\\	ぽう. 
\\	(ぽう) 
\\	減, 法	
\\	容疑	
\\	ようぎ	
\\	あの容疑者には、善悪の区別がないんだよ。	
\\	コウイチの証言で、ビエトの容疑は晴れました。	
\\	コウイチは、許可無くコウイチのベーコンを食べたという容疑でマミを起訴する準備を進めている。	
\\	容, 疑	
\\	三冊	
\\	さんさつ	
\\	この図書館では、本は三冊まで借りれます。	
\\	コウイチは、ヌード写真集を三冊だけ発行しました。	
\\	「今日は日本の小説を三冊読むぞ!」「はは。出来るもんならやってみな。」	
\\	三, 冊	
\\	得る	
\\	える	
\\	コウイチは全てを失ってしまったが、ビエトという親友を得ることができた。	
\\	人の信らいを得るのは容易なことではない。	
\\	日本で英語を教えるという経験から、たくさんのことが得られました。	
\\	う 
\\	(え) 
\\	得	
\\	暴れる	
\\	あばれる	
\\	その店員に、一万円札を細かくしてくれるようたのんだら、急に暴れ出したんだよ。	
\\	フグがおこって暴れたのは一度しか見たことがありません。	
\\	ビエトは酔って暴れたが、コウイチがなんとか落ち着かせた。	
\\	う 
\\	(あば) 
\\	暴	
\\	加える	
\\	くわえる	
\\	牛にゅうを二カップ加えて、ゆっくりかき回してください。	
\\	コウイチ君も仲間に加えてあげてよ。	
\\	オバマ大統領が作った料理に、塩をひとつまみ加えた。	
\\	う 
\\	(くわ).	加	
\\	減る	
\\	へる	
\\	十日で体重がニキロ減りました。	
\\	今月は、ワニカニの売り上げが、五パーセントも減ってしまいました。	
\\	いつも右の靴の踵が左よりも速く擦り減るんだ。	
\\	う 
\\	(へる) 
\\	減	
\\	絡む	
\\	からむ	
\\	よっぱらって絡んでくるやつが大きらいなんだよ。	
\\	かみにボタンが絡んじゃって取れないの。	
\\	最悪!ネチャネチャの風船ガムが釣り糸に絡んじゃった!どうやって取ればいいんだろう。	
\\	う 
\\	む, 
\\	(む) 
\\	からて 
\\	絡	
\\	乱れる	
\\	みだれる	
\\	最近、校風が乱れています。	
\\	トーフグ社員はコウイチの前で一糸乱れぬおどりをひろうしなければならない。	
\\	サーモンと離婚した後、フグは生活が乱れた。	
\\	乱す 
\\	れる 
\\	(みだ) 
\\	乱	
\\	改まる	
\\	あらたまる	
\\	日本の年号が、豆河豚に改まりました。	
\\	ビエトの生活たい度が改まって、コウイチはとてもよろこびました。	
\\	そう改まるなよ。コウイチ大統領は、とても気さくなんだから。	
\\	う 
\\	(あらた) 
\\	改	
\\	比べる	
\\	くらべる	
\\	去年と比べてワニカニユーザーの数が減っているんです。どうしましょう?	
\\	このコウイチのつめのあかが入ったお茶と鼻くそが入ったお茶を飲み比べて感想を言ってください。	
\\	主人と私は、私達の双子をできるだけ比べないようにはしていますが、やっぱり難しいですね。	
\\	う 
\\	(くら). 
\\	比	
\\	連絡する	
\\	れんらくする	する 
\\	会社を首になったら連絡するよ。	
\\	コウイチとはえんを切りたいので、金輪ざい連絡してこないでください。	
\\	ごめん、行かなきゃ。予定を調べて、後で折り返し連絡するよ。	
\\	連, 絡	
\\	続く	
\\	つづく	
\\	五日も雨が続いています。	
\\	コウイチのジョークの後に、気まずいちんもくが続きました。	
\\	金玉を捻ってしまい、痛みが五時間以上も続いています。	
\\	う 
\\	(つづ) 
\\	続	
\\	混ざる	
\\	まざる	
\\	水と油は混ざりません。	
\\	色が混ざってにじんでしまいました。	
\\	みんなの会話に混ざることができなくてとても悲しかったし、寂しかったよ。	
\\	う 
\\	(ざる) 
\\	(ま) 
\\	混	
\\	覚える	
\\	おぼえる	
\\	よくそんなに細かい事を覚えていますね。	
\\	ワニカニを使えば、漢字を覚えるのは容易です。	
\\	十年前、お前が俺に殺人の容疑をかけたことは、今でも覚えているぞ。	
\\	う 
\\	(おぼ). 
\\	覚	
\\	混ぜる	
\\	まぜる	
\\	混ぜるな危険!	
\\	トンカツソースとマヨネーズを混ぜたら最強に美味しいよ!	
\\	ねぇ、気をつけた方がいいよ。あの男がさっきあんたの飲み物に何か混ぜてたのを見たよ。	
\\	う 
\\	(ぜる) 
\\	(ま) 
\\	交ぜる 
\\	混	
\\	舌	
\\	した	
\\	舌のしびれはもう取れましたか?	
\\	ビタミン
\\	が足りてないからか、最近舌があれてるんだよね。	
\\	いいえ、結構です。昨夜、舌を焼けどしたので、コーヒーは飲みたくないんです。	
\\	""した.
\\	舌	
\\	履く	
\\	はく	
\\	ゆかたを着て、下たを履きました。	
\\	くつ下を履かずにスニーカーを履かないで!スニーカーがくさくなるって何度言ったら分かるの?	
\\	道路上のウンチを踏まないようにするために、人々がハイヒールを履き始めたって知ってた?	
\\	う 
\\	(は) 
\\	履	
\\	連れる	
\\	つれる	
\\	コウイチは女を連れていました。	
\\	よく母親に連れられて、支えんセンターへ行っていたのを覚えています。	
\\	フグを動物保護施設に連れていくしかありませんね。	
\\	う 
\\	(つ) 
\\	連	
\\	大失敗	
\\	だいしっぱい	
\\	する 
\\	静けさの中でおならをするという大失敗をおかしてしまいました。	
\\	マミにコウイチのベーコンを見ていてもらうというアイデアは大失敗でした。マミがコウイチのベーコンを全部食べちゃったんですよ。	
\\	昨日の会議で、シャークの奴大失敗しちゃったんだよ。もしかすると今のポジションから降ろされちゃうかも。	
\\	失敗 
\\	失敗 
\\	大, 失, 敗	
\\	無意識	
\\	むいしき	
\\	な 
\\	無意識のうちに、舌で虫歯の穴をさわってしまいます。	
\\	その無意識につめをかむくせは直した方がいいよ。	
\\	我々は素晴らしい。何故なら、無意識にクリエイティブになることができるからだ。	
\\	意識 
\\	意識, 
\\	無, 意, 識	
\\	動詞	
\\	どうし	
\\	の 
\\	英語は「主語ー動詞ー目的語」の言語です。	
\\	あなたの日本語で一番お気に入りの動詞は何?	
\\	もし自分を表現するために動詞を一つだけ選ばなくてはいけないとしたら、どの動詞を選びますか?	
\\	動, 詞	
\\	非常に	
\\	ひじょうに	
\\	イチゴは今季節外れなので、非常に高いんです。	
\\	舌は非常に重要な発声器官です。	
\\	コウイチ大統領の支持率は非常に高い。	
\\	(非常), 
\\	非常, 
\\	非, 常	
\\	乱暴	
\\	らんぼう	
\\	する 
\\	な 
\\	乱暴なやつは大きらいだ!	
\\	その男は少女に乱暴したつみでたいほされた。	
\\	「父さん、友達が僕のこと、乱暴だって言うんだ。」「それがどうした。そんな奴の言うことなんかどうでもいいよ。そんな馬鹿たれ、こてんぱにしてやれ!」	
\\	乱, 暴	
\\	暴力	
\\	ぼうりょく	
\\	コウイチに暴力でたまごっちを取られました。	
\\	家庭内暴力にたえかねて、家を出ました。	
\\	親は本当に、暴力的な映画やビデオゲームが子供の目に触れる事を制限すべきなのでしょうか?	
\\	暴, 力	
\\	若布	
\\	わかめ	
\\	かん国ではたん生日には若布スープを飲む習かんがあります。	
\\	忙しいので、かん単に若布ときゅうりの和え物を作りました。	
\\	私のお母さんが、若布と豆腐のお味噌汁を作ったんだけど、食べる?	
\\	若い 
\\	わか 
\\	若, 布	
\\	財布	
\\	さいふ	
\\	財布を家に忘れたおかげで、十ドル得をしました。	
\\	母は風水の教えにしたがって、毎年財布を買いかえています。	
\\	「ちょっと待って。財布を落としましたよ!」「ありがとう。でも、もし私がわざと落としたって言ったら、どうしますか?」	
\\	財, 布	
\\	説明書	
\\	せつめいしょ	
\\	昨日買ったアイスクリームメーカーに説明書がついていませんでした。	
\\	新しいゲームをする時に説明書を読んでから始めるタイプですか。それとも、説明書は読まずにいきなり始めるタイプですか。	
\\	私達は現在、日本語で書かれた説明書を英語に翻訳してくださる方を探しています。	
\\	(説明) 
\\	説, 明, 書	
\\	若い	
\\	わかい	い 
\\	ビエトは若い時、あるやみ取引でかなり利えきを上げました。	
\\	コウイチは若いころ、日本に留学していたことがあります。	
\\	あの若い女性は、あなたには到底手の届かない高嶺の花だってことに気づいてるわよね。	
\\	い 
\\	若	
\\	若者	
\\	わかもの	
\\	の 
\\	その若者は、公園でセミを細かくかんさつしていました。	
\\	この件には、あの若者の金が絡んでいるらしいよ。	
\\	この若者はすごいよ。大物になる前にサインをもらっておいた方がいいよ。	
\\	若 
\\	(わか), 
\\	者 
\\	もの, 
\\	(もの) 
\\	若, 者	
\\	若々しい	
\\	わかわかしい	い 
\\	コウイチっていつまでたっても若々しいよね。	
\\	そのファッション、ちょっと若々しすぎない?	
\\	私の祖母は九十代ですが、六十代の人よりも若く見えます。若さの秘訣は、若々しい気持ちでいることと、よく噛むことだそうです。	
\\	若, 々	
\\	共犯者	
\\	きょうはんしゃ	
\\	あの共犯者は、まさに飛んで火にいる夏の虫だったな。	
\\	共犯者は国外に飛んでしまった可能性が高い。	
\\	彼はとても素晴らしい犯罪者で、自分一人でも、いとも簡単にとても危ない状態から抜け出すことが出来ました。そして、共犯者は一生必要ないとも豪語していました。	
\\	共, 犯, 者	
\\	正確	
\\	せいかく	
\\	な 
\\	コウイチはワニカニの正確なユーザー数を知りません。	
\\	マイケルはコウイチのランチリクエストを正確に記おくし、注文するというにんむを終えました。	
\\	君って、ああ言えばこう言うし、しかも言うことがどれもいちいち正確なのが結構頭にくるんだけど。	
\\	正, 確	
\\	余震	
\\	よしん	
\\	まだ余震が続いているので、予だんは許されない。	
\\	今日は何回ぐらい余震を感じましたか?	
\\	その地震の後に、余震が相次ぎました。	
\\	余, 震	
\\	飛行機	
\\	ひこうき	
\\	飛行機のチケット、もう取っちゃったんだよね。	
\\	飛行機でコウイチのとなりの席になった時、めちゃくちゃこうふんしました。	
\\	私は飛行機が苦手なので、一生日本に行けることはないでしょう。	
\\	飛, 行, 機	
\\	留守番電話	
\\	るすばんでんわ	
\\	留守番電話にコウイチから明日飛行機でハワイに行くというメッセージが入っていました。	
\\	あっちが留守番電話で議論をふかっけてきたんだぜ。	
\\	のオフィスに何度も何度も電話をしていますが、いつも留守番電話です。	
\\	(留守番). 
\\	(電話) 
\\	留, 守, 番, 電, 話	
\\	人類学	
\\	じんるいがく	
\\	コウイチのお父さんの専門は文化人類学です。	
\\	研究対しょうのアフリカの民族に気に入ってもらうために、その人類学者ははだかになって村人と交流を始めた。	
\\	どうしてあなたの妹は人類学の授業を受けたかったのかな。	
\\	(人類) 
\\	人, 類, 学	
\\	議論	
\\	ぎろん	
\\	する 
\\	コウイチ達は、日本まで飛行機で行くか船で行くかで議論してるんだ。	
\\	マイケルが書いた記事は、大いに議論を呼んでいます。	
\\	コウイチとビエトは、夕飯に何を食べるかについて数時間に渡って議論に議論を重ねたが、結論には至らなかった。	
\\	議, 論	
\\	倒産	
\\	とうさん	
\\	する 
\\	会社が倒産すると知った時、空を見上げたら、きれいな飛行機雲が見えたんです。	
\\	あの会社には、倒産の嫌疑がかかっている。	
\\	昨年はこの街だけで100余りの企業が倒産した。	
\\	倒, 産	
\\	血圧	
\\	けつあつ	
\\	お互い年だし血圧が上がっちゃうから、もう非難し合うのは止めにしない?	
\\	低血圧だから朝起きられなくて困っているんです。今日は血圧をはかってもらったら、上が85で下が30でした。コウイチ先生、何とかしてもらえませんか?	
\\	血圧の薬を飲んでも血圧が高いままであることを医者に言うと、血圧を抑えるにはストレスを減らすことが重要だよと教えられました。	
\\	血, 圧	
\\	借り手	
\\	かりて	
\\	そうは言っても、借り手が見つからない可能性もありますよね?	
\\	コウイチが昔住んでいた部屋だと言ったら、すぐに借り手が見つかりました。	
\\	あのマンションは、幽霊屋敷であるという評判のせいで借り手がつかない。	
\\	借りる 
\\	手 
\\	借, 手	
\\	弓矢	
\\	ゆみや	
\\	弓矢でシカを仕留めるには、にんたいがいります。	
\\	その夜、両親は、弓矢を売るべきかどうか議論しました。	
\\	昔はうさぎ狩りに弓矢を使っていました。	
\\	弓 
\\	矢, 
\\	弓, 矢	
\\	お尻	
\\	おしり	
\\	高いところに行くと、お尻がかゆくなるんです。	
\\	赤ちゃんのお尻ほど可愛いものはこの世に無い。	
\\	ほとんどの日本人が誕生日にお尻を叩くなんて聞いた事がないと思うよ。	
\\	お 
\\	尻	
\\	確かに	
\\	たしかに	
\\	コウイチが百メートルほど飛ぶのを、この目で確かに見ました。	
\\	この機械をそう作するのは確かに難しそうだね。	
\\	確かに
\\	は、伝統的な日本語学習業界にダブルパンチを食らわせました。	
\\	確	
\\	議長	
\\	ぎちょう	
\\	議長は危機をだっしました。	
\\	あのバスの最後尾の席に座っている黒いスーツの男性が議長です。	
\\	私がこの会議の議長を務めさせていただきます。	
\\	議, 長	
\\	会議室	
\\	かいぎしつ	
\\	コウイチは、ビエトを助けるために、会議室まで飛んで行きました。	
\\	ビエトとコウイチが会議室に入ったまんまずっと出てこないんだ。	
\\	今は会議室で寝泊まりしています。	
\\	会, 議, 室	
\\	穴子	
\\	あなご	
\\	穴子が減ってきたから、そろそろ注文しないといけないな。	
\\	アメリカでは穴子は手に入りにくいので、代わりにウナギが使われています。	
\\	「穴子の握り、もう一貫頼む?」「うん!あたり前田のクラッカーだよ。」	
\\	穴 
\\	子 
\\	あなご.	穴, 子	
\\	喉飴	
\\	のどあめ	
\\	マヨネーズ味の喉飴があったら良いのになあ。	
\\	喉飴持ってない?	
\\	喉飴って理科室みたいな味がすると思わない?	
\\	喉, 飴	
\\	夢中	
\\	むちゅう	
\\	な 
\\	の 
\\	穴をつくろうのに夢中な母のすがたを見て、うれしくなりました。	
\\	皿洗いの男に夢中のウェイトレスのことを好きになってしまったんだ。	
\\	「サーモン。あのさ、多分知らなかっただろうけど、高校生の頃、き、きみに夢中だったんだ。」「知ってたわよ。フグ。」	
\\	夢, 中	
\\	犯罪	
\\	はんざい	
\\	の 
\\	犯罪にまきこまれて命を落としそうになりましたが、危機一ぱつの所で助かりました。	
\\	通報を受けて、警察は犯罪の現場に飛んで行った。	
\\	ああいう子供がそのまま大きくなるから、若者の犯罪が増えるんだよ。そう思わない?	
\\	犯, 罪	
\\	犯人	
\\	はんにん	
\\	犯人と議論したんだけど、中々かみあわなくってね。	
\\	その犯人は、口は災いの元だということを思い知った。	
\\	私は抜け目のない犯人にやられてしまったんだよ。私は全てを失って、犯人は指紋一つさえ残さなかったんだ。	
\\	犯, 人	
\\	危機	
\\	きき	
\\	の 
\\	できるだけ早い危機のこく服を期待しています。	
\\	トーフグは今、危機的じょうきょうにあります。	
\\	2008年の世界的な金融危機では、
\\	も経営難に陥ったのですか?	
\\	危, 機	
\\	非難	
\\	ひなん	
\\	する 
\\	その夫は、妻のことを家事ができないと言って非難してばかりいた。	
\\	その記事によって、トーフグは世間の非難にさらされることになった。	
\\	君に非難の目つきで見られることは予想していたけど、それでもやっぱり悲しかったんだ。	
\\	非, 難	
\\	被害	
\\	ひがい	
\\	オフィスにどろぼうが入ってしまい、トーフグは大きな被害を受けました。	
\\	この町は、山火事の被害は受けませんでした。	
\\	私達は、洪水の被害があった地域に、今日の利益を募金としてお送りします。	
\\	被, 害	
\\	議会	
\\	ぎかい	
\\	あいつら、何とかして議会の進行を妨害するつもりだぞ。	
\\	正直、そのことについて議会で説明するのはかなり面倒です。	
\\	もう議会で決定したことなんだ。	
\\	議, 会	
\\	難しい	
\\	むずかしい	い 
\\	そのかん者は、難しい手術の後のとうげをこしました。	
\\	このみぞを飛んでわたるのは難しいと思うんだ。	
\\	「これはどうすればいいの?」「ロケット科学みたいに難しいものじゃあないよ。貸して。やってみせてあげるよ。」	
\\	い 
\\	(むずか). 
\\	難	
\\	嫌悪	
\\	けんお	
\\	する 
\\	の 
\\	クリステンはクイン・ベリルを嫌悪している。	
\\	コウイチは嫌悪の目で私をにらんできた。	
\\	あいつの顔を見るだけで、嫌悪を催す。	
\\	悪 
\\	(お), 
\\	嫌, 悪	
\\	皿洗い	
\\	さらあらい	
\\	皿洗いをしているクリステンが可愛すぎて、穴があくほど見つめてしまいました。	
\\	皿洗いをしている時は、時間が経つのがおそく感じます。	
\\	食事代を支払う代わりに皿洗いをすることができるラーメン屋さんが日本にあるって本当ですか?	
\\	皿 
\\	洗う 
\\	皿 
\\	洗う 
\\	皿, 洗	
\\	嫌い	
\\	きらい	
\\	な 
\\	私の嫌いな穴うめ問題がテストに出ました。	
\\	単じゅん作業は嫌いです。そんなものは機械にまかせるべきだと思います。	
\\	色々と振り回してごめんなさい。嫌なやつと思ってるかもしれないけど、嫌いにならないで。	
\\	嫌 
\\	(きら) 
\\	嫌	
\\	嫌	
\\	いや	
\\	な 
\\	嫌な仕事を嫌々するのは体に良くないよ。	
\\	嫌嫌嫌嫌嫌!無理無理無理無理無理!クモがいる!気持ち悪い!	
\\	「これ以上は安くできないよ。嫌ならやめて帰ってくれ。」 「分かったよ。買うことにするよ。」	
\\	嫌い 
\\	いや, 
\\	嫌	
\\	立入禁止	
\\	たちいりきんし	
\\	の 
\\	このドアには、関係者以外立入禁止って書かれているよ。	
\\	スーパーのバックヤードは、たいていじゅう業員以外立入禁止です。	
\\	それは大胆だね!君が本当に立入禁止って書いてる場所に入ってションベンしたなんて信じられないよ!	
\\	(禁止). 
\\	立 
\\	たち 
\\	入 
\\	いり. 
\\	立つ 
\\	入れる 
\\	いり 
\\	入れる). 
\\	禁止 
\\	立, 入, 禁, 止	
\\	夢	
\\	ゆめ	
\\	あなたは夢を売っているそうですが、一個いくらですか?	
\\	昨日、コウイチに指圧をしてもらっている夢を見たんだ。	
\\	私の夢は漫画家になることです。叶うといいな。	
\\	(ゆめ). 
\\	夢	
\\	地震	
\\	じしん	
\\	新聞は、政府の地震対さくをはげしく非難しました。	
\\	ふだん地震のない所でごくたまに地震が起きると警察に電話してくる人が出る。	
\\	「昨夜の地震について話したいんだけど。」「ぜひ聞きたいから話してよ。」	
\\	地 
\\	じ 
\\	ち. 
\\	地, 震	
\\	震災	
\\	しんさい	
\\	震災で、地面が穴ぼこだらけになってしまいました。	
\\	震災にあった日のことが、今でもわすれられないトラウマになっています。	
\\	日本に行く時に、2011年の東北大震災の大津波に耐えた一本の松の木を見に行きたいです。	
\\	災 
\\	さい 
\\	震, 災	
\\	震度	
\\	しんど	
\\	震度は3でしたがけっこうゆれました。	
\\	マグニチュードと震度って、何が違うの?	
\\	2011年に震度七を記録した東日本大震災の余震は、今後百年程は続き、巨大余震もあるだろうと言われています。	
\\	震度 
\\	震, 度	
\\	電子機器	
\\	でんしきき	
\\	機内ではお手持ちの電子機器の電げんはお切りになるか機内モードにごせっ定くださいますよう、ご協力お願い申し上げます。	
\\	電子機器に欠かんが無いか、一個ずつ手作業で確認しています。	
\\	どうしていくつかの航空会社は、飛行中の電子機器の使用についての規制を緩和し始めたんでしょうか?	
\\	(電子) 
\\	電, 子, 機, 器	
\\	最後尾	
\\	さいこうび	
\\	大人気のゲームを発売日に買うために開店前にお店に行ったけど、すでにお店の前から列の最後尾まで200
\\	ぐらいあった。	
\\	昨日通りかかったお店の列の最後尾にクリステンがいたよ。仕事をサボって何買いに行ってたんだろう?	
\\	行列の最後尾はこちらです。	
\\	最後 
\\	最後 
\\	最, 後, 尾	
\\	存在	
\\	そんざい	
\\	する 
\\	の 
\\	そもそも、この町には水害防止対さくが存在していなかったんだよ。	
\\	じゃあ、最初はオゾンそうに穴は存在していなかったんですか?	
\\	「コウイチさんが存在するとは思えない…」とフグは言った。	
\\	存, 在	
\\	経験	
\\	けいけん	
\\	する 
\\	私は、飛行機で旅行をした経験がありません。	
\\	友達を助けようとしたら、今度は非難のほこ先が自分に向いてしまうという経験をしたことがある人はいますか?	
\\	「そうだ、結婚カウンセリングは試してみたの?」 「もちろん、とっくに経験済みだわよ。でも、上手くいかなかったわ。」	
\\	経, 験	
\\	火災	
\\	かさい	
\\	どうやって火災が大水害を引き起こしたの?	
\\	今日中に火災で空いた穴をふさがなくてはいけません。	
\\	彼は深夜二時に換気扇を回さずにフライパンでステーキを焼いて、火災警報装置を鳴らした。	
\\	火, 災	
\\	防火	
\\	ぼうか	
\\	する 
\\	の 
\\	私なりの防火の心得を文章にまとめました。	
\\	木材にぬると完全防火すると料を発明することができれば、大金持ちになれますよ。	
\\	はどのくらいの頻度で防火訓練を実施しますか?	
\\	防, 火	
\\	水害	
\\	すいがい	
\\	ビエトとダリンが、水害で飛んだデータをふく元しています。	
\\	その水害で、5,000びきの猫が家を失いました。	
\\	私は、水害の起きやすい島で育ちました。	
\\	水, 害	
\\	忘れ物	
\\	わすれもの	
\\	単に忘れ物をしたことを非難してるわけじゃないんだよ。	
\\	コウイチの飛行機が成田空港に着いた時、コウイチは忘れ物に気づいたのでそのまま同じ飛行機でポートランドへと引き返しました。	
\\	今日何か忘れ物をしたような気がするんだけど、それが何だったかをどうしても思い出せないんだよな。	
\\	忘れる 
\\	忘 
\\	わす, 
\\	忘, 物	
\\	自在	
\\	じざい	
\\	な 
\\	の 
\\	コウイチは、自由自在な発想でいつもみんなをおどろかせてくれます。	
\\	しんしゅく自在の犬用リードを探しています。	
\\	あの男は、他人の心を自在に操れる。	
\\	自, 在	
\\	有罪	
\\	ゆうざい	
\\	の 
\\	犯人には、有罪かつ死けいのはん決が下りました。	
\\	有罪の五名のうち二名は大しん院に上告した。	
\\	飲酒運転で有罪になるのはご免なので、運転はできません。	
\\	有, 罪	
\\	手洗い	
\\	てあらい	
\\	コウイチは何度言っても社員達が手洗い、うがいをしないので機嫌が悪い。	
\\	お手洗いをお借りしてもいいですか?	
\\	トーフグ
\\	シャツに洗濯機は使えません。手洗いをしなくてはいけません。	
\\	洗う 
\\	手 
\\	洗う 
\\	手, 洗	
\\	困難	
\\	こんなん	
\\	な 
\\	フグがちょうぼにあけた大穴を今年度中に埋めるのは困難でしょう。	
\\	困難な仕事に直面すると、ワクワクするんです。	
\\	私の祖母は歩くのが困難なので、今日は買い物に行くのを手伝ってあげないといけません。	
\\	困, 難	
\\	無害	
\\	むがい	
\\	な 
\\	の 
\\	マグニチュード6.2の地震がありましたが、無害でした。	
\\	無害のツイートにまでいちいちつっかかってるやつ、なんなの?	
\\	心配しなくていいよ。
\\	に毒は無いから。彼は無害な河豚だよ。	
\\	無, 害	
\\	胸焼け	
\\	むねやけ	
\\	する 
\\	胸焼けしている時に、なわとびで何回飛べますか?	
\\	年を取ったせいか、あげ物を食べると胸焼けがするんです。	
\\	朝からカツ丼なんて、胸焼けしないの?	
\\	胸 
\\	焼く. 
\\	胸, 焼	
\\	穴	
\\	あな	
\\	はずかしくて、穴があったら入りたいよ!	
\\	あそこのほら穴には、コウモリが住んでいるらしいよ。	
\\	彼が深くて素晴らしい穴を掘れるのは、持って生まれた才能です。	
\\	穴	
\\	穴場	
\\	あなば	
\\	ここは、フグ料理の穴場なんですよ。	
\\	このい酒屋、カガヤって言うんだけど、東京でも穴場中の穴場だよ。	
\\	あのことがあって以来、もう二度とああいう思いをしたくないし、周りにも迷惑をかけたくないので、フグ釣りの穴場に行くことは止めました。	
\\	穴, 場	
\\	夫妻	
\\	ふさい	
\\	その夫妻は、良い夫ふ関係には議論が必要かどうかについて、はげしく議論していました。	
\\	みんながコウイチ夫妻のことを非難がましい顔で見ていました。	
\\	フグ夫妻はともにとても口が達者で、近所の人達に大量の海藻を売りつけた。	
\\	夫, 妻	
\\	不可分	
\\	ふかぶん	
\\	な 
\\	の 
\\	ビエトとコウイチは、不可分の関係にあるんですよ。	
\\	けい約では、それは不可分なさいむだったはずですよ。	
\\	ご存知のように、豆腐と河豚は不可分です。	
\\	(可分), 
\\	可分.	不, 可, 分	
\\	論理	
\\	ろんり	
\\	その論理でいくと意味が通じないよ。	
\\	三だん論法の論理についてだれか教えてくれる人はいますか?	
\\	彼女の不満は理解できたんだけど、私はただ論理的に私達の問題を考えてほしかっただけなんだよね。そしたら今度は、私がいつも誰のせいかってことを決めたがるとか言ってきて。全然そんな事ないのに。	
\\	論, 理	
\\	機械	
\\	きかい	
\\	こちらは、地面に穴をほる機械で、そちらは、はりの穴に糸を通す機械です。	
\\	今日は機械の調子があんまり良くないんだよね。	
\\	コウイチは会社用にバッティングセンターの機械を買ったの?	
\\	機, 械	
\\	機	
\\	き	
\\	コウイチは節ぜい対さくのために、飛行機を一機こう入しました。	
\\	コウイチがアメリカ合しゅう国の大とうりょうになる機はじゅくしました。	
\\	今はまだ、機を窺っているんだ。	
\\	機	
\\	会議	
\\	かいぎ	
\\	する 
\\	の 
\\	会議中に、おなかが減ってきました。	
\\	会議をサボって連ドラを見てるとは、何事だ!	
\\	この会議の目的は、我々の現状と新しい計画についての話し合いをすることです。	
\\	会, 議	
\\	議員	
\\	ぎいん	
\\	国会議員になるのがマイケルの長年の夢でした。	
\\	大学生の時にある市議会議員のところでインターンをしていました。とてもいい経験でした。	
\\	議員バッヂを失くしてしまった。	
\\	議, 員	
\\	災難	
\\	さいなん	
\\	私は、運良く海外にいたので、その災難をまぬがれることができました。	
\\	今日学校で、「災難は勇気を試す試金石である。」ということわざを習いました。	
\\	どうしてあいつはあんなに災難に見舞われてばかりいるのかな?本当に偶然なのか、それともあいつ自体に何か問題があるんだろうか。	
\\	災, 難	
\\	機嫌	
\\	きげん	
\\	コウイチは、機械のように、毎日三時のおやつの時間になると機嫌が良くなる。	
\\	今日のお母さんの機嫌はどう?	
\\	「彼女は朝、機嫌が悪いわね。」「ええ、全くその通りね。」	
\\	けん 
\\	げん, 
\\	機, 嫌	
\\	災害	
\\	さいがい	
\\	自然災害ほどおそろしいものはありません。	
\\	そもそも、どうして災害が起こる可能性の高い場所に原子力発電所を建てたのかが理かいできません。	
\\	それぞれの部屋には、災害の場合に備えて懐中電灯とペットボトルのお水が何本か用意してあります。	
\\	災, 害	
\\	悪夢	
\\	あくむ	
\\	主役が急病になったんだけど、代役が見つからなくてね。まさに悪夢だったよ。	
\\	太った猫になって歩けないという悪夢を見た。	
\\	腕を胸の前で組んで寝ると悪夢が見れるって聞いたから試したら、朝から晩までひたすらトイレ掃除をずっとしているという夢を見たよ。	
\\	悪, 夢	
\\	鼻の穴	
\\	はなのあな	
\\	「コウイチの鼻の穴の形ってちょっと変じゃない?」「そう?私はかわいいと思うけど。」	
\\	コウイチは、鼻血が出てくる時は鼻の穴にチクワをさしこんで止血します。	
\\	「君の鼻の穴…。」「何?聞こえないわ。」「ええっと、やっぱり何でもないよ。大したことじゃないから。」「教えてくれなきゃだめよ。何を言おうとしてたの?」	
\\	鼻 
\\	穴. 
\\	鼻, 穴	
\\	妨害	
\\	ぼうがい	
\\	する 
\\	の 
\\	これ以上妨害するのを止めないと、えい業妨害でうったえるぞ!	
\\	ここまで来れば、てきの妨害の危険はほぼ無いだろう。	
\\	私はいつか映画のように公務執行妨害で逮捕されてみたい。	
\\	妨, 害	
\\	一個	
\\	いっこ	
\\	おく歯に一個、大きな穴が空いているね。	
\\	そのイチゴは、一個百円にもかかわらず、飛ぶように売れました。	
\\	ビエトはシュークリームがとっても好きなので、毎日少なくとも一個は食べます。	
\\	個, 
\\	いち 
\\	一, 個	
\\	音訓	
\\	おんくん	
\\	かん字の音訓の見分け方を教えてください。	
\\	これは常用かん字の音訓表です。	
\\	漢字の音訓ってすごく紛らわしいので嫌いです。	
\\	音読み 
\\	訓読み. 
\\	音, 訓	
\\	在外	
\\	ざいがい	
\\	の 
\\	在外し産はどのようにかん理していますか?	
\\	在外ほう人の安ぴはげん在確にん中です。	
\\	在外投票をする前に、申し込み用紙を日本大使館か日本領事館に提出する必要があります。	
\\	在, 外	
\\	人達	
\\	ひとたち	
\\	この人達、みんな本当によく舌が回るよね。	
\\	テキサスの人達ってとっても親切だね!	
\\	お父さんがピザの配達人であることを笑う人達もいますが、僕はお父さんのことをとても誇りに思っています。	
\\	達 
\\	あの人達 
\\	男の人達, その人達, 向こうの人達, 
\\	人 
\\	だち 
\\	友達, 
\\	たち 
\\	(たち) 
\\	人, 達	
\\	比率	
\\	ひりつ	
\\	機械ほんやくにはどれぐらいの比率でたよっていますか?	
\\	トーフグのオフィスの男女比率は、5:2です。	
\\	の自己資本比率は100%だと思います。	
\\	比, 率	
\\	個人	
\\	こじん	
\\	の 
\\	私個人の意見ですが、ビエトにはいいおよめさんが見つかってほしいです。	
\\	アメリカ人の国民性なのか個人の問題なのかは分からないけど、コウイチはビールが大好きだよね。	
\\	個人主義と自分勝手の違いがよく分かりません。	
\\	個, 人	
\\	災い	
\\	わざわい	
\\	これは、災いを防いでくれるお守りです。	
\\	災い転じて福となすという言葉を信じてがんばりました。	
\\	シャークは他人の災いを願っているようだね。	
\\	(わざわ). 
\\	い 
\\	災	
\\	公害	
\\	こうがい	
\\	の 
\\	日本では1950~1960年代にかけて、公害が大きな社会問題になりました。	
\\	私達はみんな、公害の加害者ですよ。	
\\	私の息子は、公害による呼吸器疾患に苦しんでいます。	
\\	公, 害	
\\	一夫多妻	
\\	いっぷたさい	
\\	の 
\\	日本もかつては一夫多妻がみとめられていました。	
\\	「一夫多妻せいとか、ありえないよな」とビエトが非難するように言うと、「その非難、自分に向けた方がいいんじゃねぇか?」とコウイチが笑って言い返した。	
\\	全ての男性が一夫多妻制を支持する訳じゃない。	
\\	一 
\\	いっ 
\\	夫 
\\	ぷ 
\\	ふう 
\\	一, 夫, 多, 妻	
\\	可能性	
\\	かのうせい	
\\	戦争になる可能性は低いと思うけど、心配だよね。	
\\	今年のクリスマスは雪になる可能性が高いみたいです。	
\\	思いついた考えは、どれも実現の可能性は無さそうでしたが、私達はそう簡単には諦めませんでした。	
\\	(可能) 
\\	可能 
\\	可, 能, 性	
\\	圧力	
\\	あつりょく	
\\	圧力なべをばく発させてしまいましたが、幸いケガはありませんでした。	
\\	ロケットが大気にとつ入するときの圧力は相当なものだ。	
\\	ワニカニがベータ版から脱するよう、我々はコウイチに圧力をかけ続けなくてはいけない。	
\\	圧, 力	
\\	予防	
\\	よぼう	
\\	する 
\\	の 
\\	危機かん理がもっとしっかりしていたら、この事故は事前に予防することができたはずだ。	
\\	その国に行くなら、事前に必ずマラリア予防の薬を飲んでください。	
\\	日本人も十年ごとに破傷風の予防接種を受けるのですか?	
\\	予, 防	
\\	防止	
\\	ぼうし	
\\	する 
\\	警察がこの町の犯罪の拡大を防止してくれています。	
\\	今日は学校で、地球温だん化防止について話し合いをしました。	
\\	日本の通貨は、世界一優れた偽造防止技術が使われていることで知られていますが、ビエトはそれを打ち負かそうとしています。	
\\	防, 止	
\\	個室	
\\	こしつ	
\\	小さい子がいると個室い酒屋の存在はありがたいです。	
\\	コウイチはシェアハウスで自分の個室がないという生活をもう五年も続けている。	
\\	忘年会に個室を予約しました。	
\\	個, 室	
\\	確率	
\\	かくりつ	
\\	コウイチがあやまってこの穴に落ちる確率は、20
\\	ぐらいだろうね。	
\\	コウイチが友情に厚い人である確率はどのぐらいだと思いますか?	
\\	ビエトはコウイチに、
\\	が成功する確率は二百パーセントだと言いきった。	
\\	確, 率	
\\	静々	
\\	しずしず	
\\	私の母は、いつも静々と機械的に家事をこなしていました。	
\\	いかりくるう妻の横を尻目に、ぼくはただ静々とそうじをしていた。	
\\	公爵夫人は静々と部屋に入って、公爵にキスをした。	
\\	静か 
\\	しずか? 
\\	""しずしず.
\\	静, 々	
\\	読者	
\\	どくしゃ	
\\	トーフグの読者が、飛行機代を出してくれたんです。	
\\	トーフグには世界中にたくさんの読者がいます。	
\\	の読者のうち、日本語学習者はどのくらいいますか?	
\\	読 
\\	どく, 
\\	(どく). 
\\	読, 者	
\\	実在	
\\	じつざい	
\\	する 
\\	穴ぐまって、実在するんですね。	
\\	どんな時に、神が実在すると感じますか。	
\\	の顔は、実在の人物をモデルに作られたという噂があります。	
\\	実, 在	
\\	残余	
\\	ざんよ	
\\	の 
\\	今月のさがく残余がくは現時点でいくらになっていますか?	
\\	残余のねん料では、地球まで引き返すことは難しいだろう。	
\\	「残余リスクについてもちゃんと議論したんですか?」「まだです。みんな残余リスクの計算式を忘れてしまって、議論にならないんです。」	
\\	残, 余	
\\	在留	
\\	ざいりゅう	
\\	する 
\\	の 
\\	日本在留のアメリカ人と友達になりました。	
\\	在留期間をえん長するには、一たん国外に出ないといけません。	
\\	ビザと在留資格は別物です。	
\\	在, 留	
\\	高等学校	
\\	こうとうがっこう	
\\	コウイチは高等学校を卒業してすぐ海軍に入りました。	
\\	私がまだ小さかった時に住んでいたマンションの近くに芸能人が通うことで有名な高等学校がありました。	
\\	高等学校で教頭をしております。	
\\	高等 
\\	学校 
\\	高校 
\\	高, 等, 学, 校	
\\	指圧	
\\	しあつ	
\\	する 
\\	指圧はストレスや不安を和らげてくれます。	
\\	指圧りょう法しのかれ氏が出来たけど、お金をはらわないと指圧してくれないんです。	
\\	指圧をしすぎると体に良くないこともあると言う人もいます。	
\\	指, 圧	
\\	罪	
\\	つみ	
\\	な 
\\	マミがコウイチのベーコンを食べた罪は思っている以上に重い。	
\\	私の会社の社長は会計さぎの罪で有罪になった。	
\\	コウイチは罪なき一般市民を殺害した人は雇わないと思うよ。	
\\	(つみ) 
\\	罪	
\\	煙い	
\\	けむい	い 
\\	山火事のせいで町中が煙くなっているんです。	
\\	花火が煙くて、むせてしまいました。	
\\	煙いなあ!一体いつになったら煙草やめるんだよ?本当に臭いんだけど。	
\\	い 
\\	(けむ) 
\\	煙	
\\	臭い	
\\	くさい	い 
\\	水害地の臭いは、とても臭かったです。	
\\	ニンニク臭くてごめんね。すごい臭いよね。さっきラーメン食べたんだ。	
\\	夫の朝の口臭が臭くて耐えられないので、離婚を決意したんです。	
\\	い 
\\	臭	
\\	余裕	
\\	よゆう	
\\	トーフグは今年は予算にかなりの余裕があります。	
\\	まだ二、三人分ぐらい余裕があるよ。	
\\	新しいパソコンと
\\	48のフィギュアを買いたいが、今は余裕が無い。	
\\	余, 裕	
\\	嫌疑	
\\	けんぎ	
\\	ビエトがヤクザだという嫌疑がぬぐえません。	
\\	警察に嫌疑をかけられているんだ。	
\\	ビエトがコウイチにかかった殺人の嫌疑を晴らした。	
\\	嫌, 疑	
\\	尾	
\\	お	
\\	コウイチは、魚の尾を見ただけで魚の種類を言うのが特技です。	
\\	尾が無いと、鳥は上手く着りく出来ません。	
\\	どうして犬の尾を短くする習慣が産まれたのでしょうか。	
\\	(お) 
\\	尾	
\\	理論	
\\	りろん	
\\	コウイチの理論はだれからも支持されませんでした。	
\\	コウイチはよい会社の社長になるために、リーダーシップ理論を学んでいます。	
\\	理論的に説明するのは苦手です。	
\\	理, 論	
\\	論文	
\\	ろんぶん	
\\	この論文、終わりの方で五ページぐらい飛んでたよ。	
\\	友達は大学院在学中に論文をたくさん出ぱんしたツワモノです。	
\\	うまくいけば、大学の卒業論文今日中に書き終わるかなって感じ。	
\\	論, 文	
\\	厚い	
\\	あつい	い 
\\	どうやってこの厚いへいに穴が空いたんだろう。	
\\	トーフグオフィスをおとずれたさい、とても厚いもてなしを受けました。	
\\	コウイチは、厚い苔の上で休んでいる病気のミミズを見つけた。	
\\	い 
\\	厚	
\\	三個	
\\	さんこ	
\\	コーヒーにはミルクとシロップを三個ずつ入れます。	
\\	三個入りの石けんを一箱買いました。	
\\	チーズバーガー三個と照り焼きバーガー三個とナゲットをいくつか食べたので、今とても気持ちが悪い。	
\\	個 
\\	三, 個	
\\	焼ける	
\\	やける	
\\	火事で家が焼けてしまいました。	
\\	このパンはこんがりとよく焼けています。	
\\	「この肉は厚いから、焼けるまでに時間がかかりそうだね。」「あなたが皿洗いをしている間に、焼けると思うけど。」	
\\	焼く 
\\	焼ける 
\\	焼く, 
\\	焼	
\\	困る	
\\	こまる	
\\	タンがのどに絡んで困っています。	
\\	日本語を勉強していますが、名詞や助詞などの意味が分からなくて困っています。	
\\	「どうして遅くまでお店を開けておけないの?」「すみませんが、私に言われても困ります。私はただここで働いているだけなので。」	
\\	う 
\\	子
\\	子
\\	子
\\	困	
\\	余る	
\\	あまる	
\\	穴づりでつった魚が余っているので、良かったら少しいかがですか?	
\\	くらしぶりからして、あの夫妻、相当お金が余ってそうだと思いませんか。	
\\	私達の果樹園にはあり余るほどの林檎があります。	
\\	う 
\\	あま).
\\	余	
\\	犯す	
\\	おかす	
\\	これから戦時中に犯した罪を告白します。	
\\	法りつを犯してまでするような事じゃないよね。	
\\	ビエトがミスを犯すことなどない。ビエトに不可能は無い。	
\\	う 
\\	(おか). 
\\	犯	
\\	防ぐ	
\\	ふせぐ	
\\	どうやって事こを防ぐかについて、議論する必要があります。	
\\	ビエト達がてきを防いでいる間に、クリステンはコウイチ王子を日本へにがしました。	
\\	このダウンジャケットは、寒さをとってもよく防いでくれますよ。	
\\	(ふせ) 
\\	防	
\\	飛ぶ	
\\	とぶ	
\\	公園につくと、たくさんのシャボン玉が空を飛んでいました。	
\\	ああ見えて、コウイチはすごく高く飛ぶことができるんだ。	
\\	の最新の
\\	ブックは、飛ぶように売れています。	
\\	う 
\\	(と). 
\\	飛	
\\	産む	
\\	うむ	
\\	あの人、去年の冬に赤ちゃんを産んだばかりなのに今年の秋にもまた赤ちゃんを産むんですって。	
\\	カマキリが産んだたまごの中からたくさんの赤ちゃんカマキリが出てきました。	
\\	コウイチ先生:よし、もう一度いきむ準備はいいかい?君は双子の赤ちゃんを産むんだよ! サーモン:もう一人生んだじゃない。放っておいてよ。 フグ:大丈夫、大丈夫。頑張れ、サーモン!君ならできるよ!	
\\	う 
\\	産む 
\\	(生む). 
\\	産む 
\\	(生む). 
\\	産	
\\	倒す	
\\	たおす	
\\	穴じゃく子、倒したのだれ?	
\\	あいつの事はおれが倒してみせるから、余計な心配はするな。	
\\	日曜日、僕は父と一緒に木を何本か切り倒した後、それを叩き切って薪にしました。	
\\	う 
\\	(たお) 
\\	倒	
\\	妨げる	
\\	さまたげる	
\\	あの会社は、トーフグの成功を妨げようとしているらしい。	
\\	生産性を妨げるものは、すててしまった方がいいよ。	
\\	どうして人々は宇宙人が地球にやってくる事を妨げようとするのか理解できません。	
\\	う 
\\	(さまた) 
\\	妨	
\\	経つ	
\\	たつ	
\\	時間が経つのを忘れて、ゲームに夢中になっていました。	
\\	トーフグができてから、もう何年経ちましたか?	
\\	「あいつら、幸せな結婚生活を送れると思う?」「時間が経てばわかるよ。」	
\\	経 
\\	(た), 
\\	経	
\\	被る	
\\	かぶる	
\\	夏は外に出る時はぼうしを被らなくちゃ。	
\\	あそこでコナンのお面を被って無じゃ気にあそんでいるのがコウイチで、そのとなりでやさしく見守っているのがビエトです。	
\\	大阪で雨の日に、中年のおばさんがビニール袋を被りながら自転車に乗っているのを見たことがあります。	
\\	う 
\\	(かぶ). 
\\	被	
\\	面倒	
\\	めんどう	
\\	な 
\\	面倒な仕事をおし付けられたと思ったことは、一度もありません。	
\\	こりゃまた面倒な仕事を引き受けたもんだね。	
\\	今に見てなさい。あのフグ少年はきっと面倒を引き起こすわよ。	
\\	とう 
\\	どう, 
\\	面倒.	面, 倒	
\\	余計	
\\	よけい	
\\	な 
\\	このイス、一つ余計だったね。	
\\	車のそう音に安みんを妨害されて、余計にはらが立ったんです。	
\\	他人の事情に首を突っ込むのは、まったく余計なお世話だってことは分かっていたんですが、つい好奇心に負けてしまいました。	
\\	余, 計	
\\	防水	
\\	ぼうすい	
\\	する 
\\	の 
\\	水害被災者達は、防水の家具をそろえておけばよかったと口をそろえて言いました。	
\\	完全防水するには、これを使うしかないよ。	
\\	この防水ブーツ、防水されているとはとても言えないな。この辺りを歩いてるだけで、足がビショビショになっちゃったよ。	
\\	防水 
\\	防, 水	
\\	裕福	
\\	ゆうふく	
\\	な 
\\	不けい気を尻目に、コウイチはどんどん裕福になっていった。	
\\	ビエトって、裕福な家庭の出身だと思うんだよね。	
\\	僕の両親はド金持ちって訳じゃないけど、比較的裕福な方ではあります。	
\\	裕, 福	
\\	天気予報	
\\	てんきよほう	
\\	天気予報を事前にチェックするのが面倒くさかったんです。	
\\	天気予報では午後から雨になってたよ。	
\\	天気予報は見事に外れた。	
\\	天気, 
\\	予報 
\\	天, 気, 予, 報	
\\	妻	
\\	つま	
\\	妻のたん生日に、水上飛行機をプレゼントしました。	
\\	妻がコウイチのコンサートに行ってきたんですが、最後尾の席だったけど、すごく良かったと言ってましたよ。	
\\	元妻が、「私みたいな女にはもう巡り合わないわよ」とか言うんだけどさ…っていうか、それが俺の計画なんですけど、みたいな?	
\\	(つま), 
\\	妻	
\\	妻子	
\\	さいし	
\\	家では妻子がおなかをすらせて待っているんです。	
\\	コウイチは妻子ある身なのに毎晩行きつけのホステスのところに通っているんだよ。ひどいと思わない?	
\\	新しい支店に転勤になって、やる気満々だよ〜。イイ男探しを日課に頑張ってます。妻子持ちだろうが関係ないかんね。笑	
\\	妻, 子	
\\	不思議	
\\	ふしぎ	
\\	な 
\\	休みの日の時間って、不思議と飛ぶようにすぎるよね。	
\\	ビエトの周りにいると不思議なことがよく起こるよね!この前も列の最後尾にならんでいたのになぜか前の人が次々とゆずってくれるし。	
\\	私達はコウイチをすごく頼りにしていますが、どうしていつもそんなに陽気にしてるのか不思議にも思っています。	
\\	不, 思, 議	
\\	実際	
\\	じっさい	
\\	の 
\\	今現在、実際に
\\	は何名の顧客を抱えているのですか?	
\\	つ 
\\	っ.	実, 際	
\\	宇宙人	
\\	うちゅうじん	
\\	コウイチの写真は、火星で干ばつと大恐慌の犠牲となった宇宙人たちを正直に、かつ同情を込めてとらえている。	
\\	宇宙 
\\	宇宙 
\\	宇, 宙, 人	
\\	分解	
\\	ぶんかい	
\\	する 
\\	の 
\\	電気分解の分解効率を向上させることはできる、お父さん?	
\\	分, 解	
\\	大敵	
\\	たいてき	
\\	分かっていると思うけど、油断大敵だよ。「猿も木から落ちる」って言うだろう?	
\\	大, 敵	
\\	権利	
\\	けんり	
\\	フェミニストの中には、生理中に血を垂れ流しにすることが女性の権利だと主張する人まで出てきました。	
\\	権, 利	
\\	総体的	
\\	そうたいてき	
\\	な 
\\	我々は、お客様に総体的に満足して頂けるよう努力しています。	
\\	総, 体, 的	
\\	暴走族	
\\	ぼうそうぞく	
\\	ビエトさん、あなたが暴走族をしていた頃のエピソードについて、早急にお返事いただけると幸いです。	
\\	(暴走) 
\\	暴, 走, 族	
\\	書評	
\\	しょひょう	
\\	する 
\\	その本についての批判的な書評があまりにも多いので、買う気が失せたよ。	
\\	書, 評	
\\	得体	
\\	えたい	
\\	こんな得体の知れないもの、食べられないよ。	
\\	得体の知れない男がさっきからこっちを見ていると思ったら、トーフグのコウイチだった。	
\\	得体の知れない何かが、猛スピードで迫ってくるような感覚に襲われ、冷や汗が止まらなくなりました。	
\\	得体が知れない/得体の知れない, 
\\	得, 体	
\\	検査	
\\	けんさ	
\\	する 
\\	北米でギョウ虫検査があまり一般的ではないなんて、驚きです。	
\\	検, 査	
\\	布地	
\\	ぬのじ, きれじ	
\\	レースを布地にたたきつける。	
\\	布, 地	
\\	理解	
\\	りかい	
\\	する 
\\	愛してるよ、ソファー。お前だけは俺のことを理解してくれる。	
\\	理, 解	
\\	強制	
\\	きょうせい	
\\	する 
\\	の 
\\	母は、フグと私がこのアール・デコ様式の建物に引っ越すよう強制しました。	
\\	強, 制	
\\	条件	
\\	じょうけん	
\\	フグ、金曜の夜の飲み会に行ってもいいけど、一つだけ条件があるわ。お酒は飲まないでね。	
\\	条, 件	
\\	確認	
\\	かくにん	
\\	する 
\\	確認画面が出たら
\\	とパスワードを入力してログインをしてください。	
\\	確, 認	
\\	建設	
\\	けんせつ	
\\	する 
\\	私はかつて古代ローマが建設された丘の近くに住んでいた。	
\\	建, 設	
\\	正解	
\\	せいかい	
\\	する 
\\	コウイチには内緒にしておいて、正解だったね。	
\\	正解!!!	
\\	正, 解	
\\	暴力団	
\\	ぼうりょくだん	
\\	暴力団と関東連合の違いは何ですか?	
\\	暴, 力, 団	
\\	批判	
\\	ひはん	
\\	する 
\\	批判の的となっても仕方がないと思っている。	
\\	批, 判	
\\	評論	
\\	ひょうろん	
\\	する 
\\	映画の中で十歳の女の子が評論をしているのが可愛いのよね。	
\\	評, 論	
\\	いい加減にしろ	
\\	いいかげんにしろ	
\\	ビエトはあまりにも怒っていたので、コウイチを殴って「いい加減にしろ!」と叫ばずにはいられなかった。	
\\	いい加減 
\\	いい加減, 
\\	加, 減	
\\	定義	
\\	ていぎ	
\\	する 
\\	スパムメールの定義は何ですか?	
\\	定, 義	
\\	任務	
\\	にんむ	
\\	任務完了前に撤退することは許されないとコウイチに言われました。	
\\	任, 務	
\\	経済	
\\	けいざい	
\\	する 
\\	コウイチが大統領に就任してからというもの、アメリカの経済は急ピッチで回復している。	
\\	さい 
\\	ざい, 
\\	経, 済	
\\	選挙	
\\	せんきょ	
\\	する 
\\	の 
\\	コウイチは、選挙演説中にブルースを歌って声がつぶれた。	
\\	選, 挙	
\\	敵	
\\	てき, かたき	
\\	「フグ、あなたはどっちの味方なの?私の敵なの?」「僕は君の味方だよ、サーモン。」	
\\	敵	
\\	羨望	
\\	せんぼう	
\\	する 
\\	ハチは、犬界での羨望の的だ。	
\\	彼は、あまりにも美味しそうなカレーパンを食べていたため羨望の視線を浴びた。	
\\	ゴードンへの羨望の気持ちが抑えきれず、料理の道へ進みました。	
\\	羨, 望	
\\	審査	
\\	しんさ	
\\	する 
\\	私のパスポートに上陸許可の判子を押してくれた入国審査官は、コウイチにそっくりでした。	
\\	審, 査	
\\	解説	
\\	かいせつ	
\\	する 
\\	の 
\\	あの先生の解説はとても分かり易いね。	
\\	解, 説	
\\	評判	
\\	ひょうばん	
\\	の 
\\	この地域は昔はスラム街でしたが、政府の戦略としての居住地域の高級化が功を奏し、住環境ががらっと変わったんですよ。今は評判もすこぶるいいです。	
\\	はん 
\\	ばん, 
\\	評, 判	
\\	履き物	
\\	はきもの	
\\	日本語で、屋内の履き物は「上履き」、屋外の履き物は「外履き」と呼ばれます。	
\\	履く 
\\	履く 
\\	物 
\\	履, 物	
\\	資本	
\\	しほん	
\\	ベンチャー・キャピタル、いわゆる冒険資本が不足するには、訳があります。	
\\	資, 本	
\\	資金	
\\	しきん	
\\	資金が底をつき始めたときに、パリでホテル代を払ってくれるというあの素晴らしいお方にお会いしたのです。	
\\	資, 金	
\\	立派	
\\	りっぱ	
\\	な 
\\	私の猫は立派な髯を生やしているので、「ヒゲ」と名付けました。	
\\	立派だ!	
\\	りつ 
\\	りっ 
\\	は 
\\	ぱ 
\\	立, 派	
\\	人権	
\\	じんけん	
\\	の 
\\	服装規定が人権侵害にあたることがあるという人もいますが、それについてどう思いますか?	
\\	人, 権	
\\	事件	
\\	じけん	
\\	この「盗まれたコウイチの下着」事件について、できるだけ早く報告してください。	
\\	事, 件	
\\	増加	
\\	ぞうか	
\\	する 
\\	お正月に餅を食べ過ぎて、体重が五十キロも増加した。	
\\	増, 加	
\\	国際	
\\	こくさい	
\\	の 
\\	コウイチは国際会議で
\\	の開発成果について発表する予定です。	
\\	国, 際	
\\	投資	
\\	とうし	
\\	する 
\\	リスクの高い投資商品にまで手を出したくはないんです。	
\\	投, 資	
\\	主義	
\\	しゅぎ	
\\	フグは肉を食べないよ。菜食主義者だからさ。	
\\	主, 義	
\\	判断	
\\	はんだん	
\\	する 
\\	私の母はいつも人を見た目で判断してはいけないと言っていたのに、父が禿げたことだけを理由に父と離婚しました。	
\\	判, 断	
\\	審判	
\\	しんぱん	
\\	する 
\\	私は審判でもあり、スポーツ解説者でもあります。	
\\	審, 判	
\\	正義	
\\	せいぎ	
\\	チームは正義の味方です。	
\\	正, 義	
\\	心得	
\\	こころえ	
\\	先輩が就職活動の心得を教えてくれました。	
\\	心, 得	
\\	素敵	
\\	すてき	な 
\\	彼の他の人を助けたいっていう熱意が、素敵だなって思うのよね。	
\\	素 
\\	(す). 
\\	素, 敵	
\\	資料	
\\	しりょう	
\\	の 
\\	もしよろしければ、いくつか資料をお送りさせて頂けませんでしょうか?	
\\	資, 料	
\\	委員	
\\	いいん	
\\	風紀委員は、保健体育の授業で教師が夢精について生徒たちに教える必要は無いと判断した。	
\\	委, 員	
\\	〜務省	
\\	むしょう	
\\	ビエト財務省国際局次長は現在ダイエット中であり、砂糖の代わりにサッカリンを用いているため、彼の食事には砂糖を使用しないでください。	
\\	務, 省	
\\	形容動詞	
\\	けいようどうし	
\\	今日の授業は形容動詞でした。	
\\	形容詞 
\\	動詞 
\\	な (便利な, 
\\	形容詞 
\\	動詞 
\\	形, 容, 動, 詞	
\\	義務	
\\	ぎむ	
\\	の 
\\	家事をするのは、夫と子どもに対する妻の義務だと言う人が多くいますが、果たしてそれは本当でしょうか?	
\\	義, 務	
\\	事務所	
\\	じむしょ	
\\	当社トフグの事務所を簡単にご案内させてください。	
\\	事, 務, 所	
\\	岡山県	
\\	おかやまけん	
\\	岡山県には、土地特有の建築物はありますか?	
\\	(岡山) 
\\	県 
\\	岡, 山, 県	
\\	判子	
\\	はんこ	
\\	この書類には、コウイチによって公的な判子が押されています。	
\\	判 
\\	子 
\\	判, 子	
\\	任意	
\\	にんい	の 
\\	名前と住所は法律で答えなくてはいけないと決まっているものですか?それとも任意ですか?	
\\	任, 意	
\\	総合	
\\	そうごう	
\\	する 
\\	の 
\\	オリンピックの総合メダル獲得数では、アメリカが一位だった。	
\\	総, 合	
\\	参加者	
\\	さんかしゃ	
\\	映画好きのパーティーの最年長参加者は、なんと百歳だった。	
\\	参加 
\\	参, 加, 者	
\\	企画	
\\	きかく	
\\	する 
\\	初めまして。私は経営企画部部長のコウイチと申します。	
\\	かく 
\\	計画 
\\	企, 画	
\\	警察官	
\\	けいさつかん	
\\	警察官が容疑者を発見した際、男は丘の上で美しい風景画を描いていました。	
\\	(警察) 
\\	警察署, 
\\	警, 察, 官	
\\	総理	
\\	そうり	
\\	は総理大臣賞を受賞した。	
\\	総, 理	
\\	手続き	
\\	てつづき	
\\	する 
\\	誰だって面倒な手続きは避けたい。	
\\	手 
\\	続く 
\\	手, 続	
\\	固有名詞	
\\	こゆうめいし	
\\	の 
\\	しりとりに固有名詞は使えません。	
\\	(名詞) 
\\	名詞 
\\	固, 有, 名, 詞	
\\	公設	
\\	こうせつ	
\\	の 
\\	もし沖縄を旅行するなら、牧志公設市場に行った方がいいよ。	
\\	公, 設	
\\	資格	
\\	しかく	
\\	一流の資格を持っていたって、食いっぱぐれることはあるさ。	
\\	資, 格	
\\	素材	
\\	そざい	
\\	このレストランでは、素材そのものの味を楽しんでもらうために味付けは控えめにしています。	
\\	素 
\\	そ 
\\	素, 材	
\\	反省	
\\	はんせい	
\\	する 
\\	反省しているので、そんなにきつく当たらないでください。	
\\	反 
\\	省 
\\	(せい) 
\\	反, 省	
\\	責任	
\\	せきにん	
\\	日本のサラリーマンは責任を取る事を嫌がる傾向にある。	
\\	責, 任	
\\	自制	
\\	じせい	
\\	する 
\\	の 
\\	私としては、断然フランボワイヤン様式の方が良かったんですが、そこはまぁ自制しています。	
\\	自, 制	
\\	義理	
\\	ぎり	
\\	の 
\\	義理の父に義理チョコを買いました。	
\\	義, 理	
\\	派手	
\\	はで	
\\	な 
\\	ドレスが派手すぎるので、ブライズ・メイドには絶対なりたくありません。	
\\	て 
\\	で.	派, 手	
\\	際	
\\	きわ	
\\	小さい頃、友人達と毎日水際で遊んでいました。	
\\	(きわ) 
\\	際	
\\	解決	
\\	かいけつ	
\\	する 
\\	トフグの株式を上場させる前に、解決しなければならない難題がまだ山積みになっている。	
\\	解, 決	
\\	宇宙船	
\\	うちゅうせん	
\\	十六台の宇宙船のほとんど全てが、保持され改装されました。	
\\	宇宙 
\\	船 
\\	宇宙 
\\	宇, 宙, 船	
\\	急増	
\\	きゅうぞう	
\\	する 
\\	振り付け師になるのが夢だという子どもが、急増しています。	
\\	急, 増	
\\	容疑者	
\\	ようぎしゃ	
\\	その容疑者は、どういう訳か十代の若者たちに大志と楽観主義のメッセージを伝えた。	
\\	(容疑) 
\\	容疑 
\\	容, 疑, 者	
\\	制度	
\\	せいど	
\\	の 
\\	これは、新制度についての初心者向け案内書です。	
\\	制, 度	
\\	税金	
\\	ぜいきん	
\\	私の両親があなたを養子縁組したのは、ただ税金の控除を受けたかったのが理由です。	
\\	税, 金	
\\	無税	
\\	むぜい	
\\	の 
\\	お酒が無税だったらいいのにな。	
\\	無, 税	
\\	増税	
\\	ぞうぜい	
\\	する 
\\	さっき届いた請求書の固定資産税が、50
\\	も増税されてたんだけど、間違いだよね。そうであってほしい。	
\\	増, 税	
\\	混乱する	
\\	こんらんする	する 
\\	君はおそらく混乱すると思うけど、弟は今、死への恐怖を克服しようと必死で努力しているところなの。だから、そっとしておいてあげてくれないかな?	
\\	混乱 
\\	混乱 
\\	混, 乱	
\\	混む	
\\	こむ	
\\	電車は毎日混みますが、あなたのせいではありません。自分を責める必要はありませんよ。	
\\	混じる 
\\	混ぜる 
\\	混ざる 
\\	混む, 
\\	混 
\\	(む) 
\\	(こむ) 
\\	混	
\\	解ける	
\\	とける	
\\	くつひもが解けているよ。危ないから、早く結んだ方がいいよ。	
\\	う 
\\	(ける) 
\\	(と). 
\\	解	
\\	責める	
\\	せめる	
\\	「自分でちゃんと確かめるべきだったよ。」「そんなに自分を責めないで。あなたのせいじゃないわよ。」	
\\	う 
\\	(せ) 
\\	責	
\\	済む	
\\	すむ	
\\	謝ったんだけど、「これは謝って済む問題じゃない」なんて言われちゃって、それで頭にきて彼の顔をひっぱたいちゃったのよね。	
\\	う 
\\	(す) 
\\	済	
\\	続ける	
\\	つづける	
\\	一日中寝続けられるので、週末は最高だ。	
\\	続く 
\\	ける, 
\\	く 
\\	続く, 
\\	続	
\\	減らす	
\\	へらす	
\\	その疫病は、河豚の数をそれまでの水準の半分に減らした。	
\\	う 
\\	減る 
\\	(らす) 
\\	減る, 
\\	減	
\\	羨む	
\\	うらやむ	
\\	コウイチは、世界が羨む男だ。	
\\	人は他人の幸せを羨む傾向がある。	
\\	いつか誰もが羨む名声と富を手に入れたいです。	
\\	う 
\\	(うらや). 
\\	羨	
\\	設ける	
\\	もうける	
\\	コウイチは、新しい従業員規則を設けることを検討しているようです。	
\\	う 
\\	(もう). 
\\	設	
\\	設定する	
\\	せっていする	する 
\\	あなたの新しいビジネス用のメールアカウントを、グーグルに設定しました。	
\\	設 
\\	せっ.	設, 定	
\\	混じる	
\\	まじる	
\\	悪魔からのメッセージは偽りだが、我々を混乱させるために嘘と真実が入り混じっている。	
\\	混む 
\\	混ぜる 
\\	混ざる 
\\	混じる, 
\\	(じ) 
\\	混 
\\	混	
\\	増える	
\\	ふえる	
\\	ここ5年間、ビエトの一日当たりの飲酒量が増えてきています。	
\\	う 
\\	(える) 
\\	(ふ) 
\\	増	
\\	検問する	
\\	けんもんする	する 
\\	警察が私の車を検問している時、どきどきしました。	
\\	検, 問	
\\	省く	
\\	はぶく	
\\	このロボットが多くの人手を省くはずだったが、状況は変化しないままである。	
\\	う 
\\	(はぶ) 
\\	省	
\\	参加する	
\\	さんかする	する 
\\	今週の日曜日、摩天楼の見学ツアーに参加します。	
\\	参加 
\\	する 
\\	参加.	参, 加	
\\	説得する	
\\	せっとくする	する 
\\	ロックミュージシャンになるには、親父を説得する必要がある。	
\\	説得 
\\	説得.	説, 得	
\\	企てる	
\\	くわだてる	
\\	「何それ?」「わっ!何でもないよ。」「何か企てているでしょう?」	
\\	う 
\\	(くわだ) 
\\	企	
\\	乱す	
\\	みだす	
\\	注意欠陥過活動性障害の子どもは、生活のリズムを乱されることに耐えられないことが多いです。	
\\	乱れる 
\\	(す) 
\\	乱れる, 
\\	乱	
\\	認める	
\\	みとめる	
\\	認めるのは悔しいけど、今回はフグはいい仕事をしていると思うよ。	
\\	う 
\\	(みと). 
\\	認	
\\	断る	
\\	ことわる	
\\	コウイチ大統領はビエトを次期国務長官に任命したが、ビエトはそれを断った。	
\\	う 
\\	事 
\\	事 
\\	(ことわる).	断	
\\	挙がる	
\\	あがる	
\\	自分がヒゲ大賞の候補に挙がっているなんて、信じられないよ。	
\\	上がる. 
\\	挙	
\\	目覚める	
\\	めざめる	
\\	この裏ワザを使うと、朝、すっきり目覚めることができます。	
\\	覚 
\\	(ざ). 
\\	目, 覚	
\\	務める	
\\	つとめる	
\\	父は去年まで四十年間警察官を務めました。	
\\	コウイチは色々な星からやってきた宇宙人をお世話するという大役を務めた。	
\\	田村さんは田中さんの個人秘書を一年間務めたんだけど、あまりうまくいかなかったんだよね。	
\\	う 
\\	(つと)! 
\\	務	
\\	他動詞	
\\	たどうし	
\\	その動詞が他動詞か自動詞かを学ぶには、まる暗記するしかないのかな。	
\\	自動詞 
\\	(動詞) 
\\	他, 動, 詞	
\\	条約	
\\	じょうやく	
\\	コウイチの今年の抱負は、ベルサイユ条約の全文を暗記することです。	
\\	条, 約	
\\	制服	
\\	せいふく	
\\	アヤに、制服を着た女の子の絵を、立体派の画風を取り込みながら描いてもらいたいんだよね。	
\\	制, 服	
\\	無敵	
\\	むてき	
\\	な 
\\	の 
\\	しりとりで、コウイチは無敵だ。	
\\	無, 敵	
\\	口笛	
\\	くちぶえ	
\\	口笛が下手くそなので、私が口笛を吹いても犬が来てくれません。	
\\	口 
\\	口 (くち) 
\\	笛 (ふえ) 
\\	ふえ 
\\	ぶえ, 
\\	口, 笛	
\\	女権	
\\	じょけん	
\\	西洋と違い、日本では歴史上、大きな女権拡張運動は起きていない。	
\\	女, 権	
\\	副詞	
\\	ふくし	
\\	の 
\\	私は昨日、日本語の時を表す副詞を学びました。	
\\	副, 詞	
\\	罰金	
\\	ばっきん	
\\	罰金については、追ってご連絡を差し上げます。	
\\	ばつ 
\\	ばっ 
\\	罰金!).	罰, 金	
\\	待機	
\\	たいき	
\\	する 
\\	の 
\\	少子化なのにどうして待機児童が増えるんですか?	
\\	待, 機	
\\	策	
\\	さく	
\\	本日のプレゼンでは、主に3つの策を取り上げます。	
\\	策	
\\	反応	
\\	はんのう, はんおう	
\\	する 
\\	反応しないアプリケーションを強制終了する方法を教えてくれませんか?	
\\	反 
\\	応 
\\	のう. 
\\	はんおう 
\\	はんおう 
\\	はんのう 
\\	はんおう 
\\	反, 応	
\\	援助	
\\	えんじょ	
\\	する 
\\	の 
\\	このオナラ計画が成功するには君の援助がぜひ必要です。	
\\	援, 助	
\\	態度	
\\	たいど	
\\	の 
\\	あいつ、誰と話をしているかによって、態度を180度変えるからね。信じられないよ。	
\\	態, 度	
\\	観客	
\\	かんきゃく	
\\	サーモンとコンサートに行ったんだけど、なんと他の観客は全員お相撲さんだったんだ。	
\\	観, 客	
\\	赤ん坊	
\\	あかんぼう	
\\	あーどうしたら赤ん坊を泣き止ませることができるんだろう。ミルクもあげてみたし、オムツもかえてみたでしょ。それに、げっぷもさせてみた。あ、これはどうかな?赤ちゃんを体に引き寄せて抱っこして、左右に素早く揺らしてみるって書いてあるわ。	
\\	赤 
\\	坊 
\\	赤, 坊	
\\	不機嫌	
\\	ふきげん	
\\	な 
\\	花江は今、不機嫌なのよ。	
\\	機嫌 
\\	不, 機, 嫌	
\\	罰ゲーム	
\\	ばつげーむ, ばつゲーム	
\\	この罰ゲームについて、がっかりもしてるんだけど、ワクワクしてもいるんだよね。	
\\	(ゲーム) 
\\	罰	
\\	誕生	
\\	たんじょう	
\\	する 
\\	私の誕生日は四月八日で、お釈迦様の誕生日と同じ日です。	
\\	誕 
\\	生 
\\	(じょう). 
\\	誕, 生	
\\	失態	
\\	しったい	
\\	社長は大失態を犯し、それを隠蔽しようとしたが、さらにその隠蔽工作が裏目に出てしまった。	
\\	しつ 
\\	しっ, 
\\	失, 態	
\\	区域	
\\	くいき	
\\	日本政府は、その区域の正確な大気中の放射能濃度を公表しているのですか?	
\\	区, 域	
\\	坊主	
\\	ぼうず	
\\	息子よ。現実的になれ。いい加減俺の後を継いで坊主になったらどうなんだ?	
\\	坊 
\\	主 
\\	ず. 
\\	坊, 主	
\\	営業	
\\	えいぎょう	
\\	する 
\\	トーフグの営業チームに配ぞくされました。	
\\	トーフグオフィスの営業時間は西海岸時間で大体朝九時から夕方六時までです。	
\\	営, 業	
\\	社費	
\\	しゃひ	
\\	私は社費で日本語を勉強させてもらっています。	
\\	社, 費	
\\	状態	
\\	じょうたい	
\\	妹は、精神状態が不安定であるということを私に打ち明けた。	
\\	状, 態	
\\	消費	
\\	しょうひ	
\\	する 
\\	の 
\\	豆乳や納豆の消費が増加しています。	
\\	消, 費	
\\	防犯	
\\	ぼうはん	
\\	の 
\\	本日参りました理由は、当社の最新防犯グッズをご紹介するためです。	
\\	防, 犯	
\\	大勢	
\\	おおぜい	
\\	の 
\\	大勢の人の前で話す事に慣れていないので、今からもう手が震えています。	
\\	大 
\\	おお 
\\	大きい). 
\\	勢 
\\	ぜい. 
\\	大, 勢	
\\	脱線	
\\	だっせん	
\\	する 
\\	電車が脱線しましたが、怪我人はありませんでした。	
\\	脱, 線	
\\	領域	
\\	りょういき	
\\	の 
\\	誰でも、前進するためには未知の領域に足を踏み入れなくてはいけない。	
\\	領, 域	
\\	各々	
\\	おのおの	
\\	どうやら各々独自のポテトチップスの食べ方をもっているようだ。	
\\	各, 々	
\\	各〜	
\\	かく	
\\	まず始めに、各チームの代表者を決めて下さい。	
\\	各々 
\\	各	
\\	各地	
\\	かくち	
\\	の 
\\	父の仕事の都合で、私の家族は各地を転々としなくてはいけませんでした。	
\\	各, 地	
\\	各自	
\\	かくじ	
\\	遠足には、各自お弁当とお菓子、水筒を持参してください。	
\\	各, 自	
\\	評価	
\\	ひょうか	
\\	する 
\\	私のホームドクターは、いつも急いでいて、自分の症状について調べてきた患者の話を聞かずにけんもほろろに追い出すので、評価が低い。	
\\	評, 価	
\\	費用	
\\	ひよう	
\\	日本では引っ越し費用の平均って大体どれくらいなの?	
\\	費, 用	
\\	経験者	
\\	けいけんしゃ	
\\	コウイチは
\\	に犯罪経験者を雇うと思いますか?	
\\	(経験) 
\\	経, 験, 者	
\\	外観	
\\	がいかん	
\\	の 
\\	このアパートメントの外観が好きです。	
\\	外, 観	
\\	姿	
\\	すがた	
\\	彼が私に微笑みかけている姿が頭に焼きついて離れない。	
\\	(すがた)??? 
\\	姿	
\\	勢い	
\\	いきおい	
\\	私達の犬は、誰と競争している訳でもないのに、いつもすごい勢いでミルクボーンに向かって行きます。	
\\	い 
\\	(いきおい), 
\\	勢	
\\	官営	
\\	かんえい	
\\	日本郵政をどうして官営から民営に変える必要があったのか、未だに分かりません。	
\\	官, 営	
\\	案内	
\\	あんない	
\\	する 
\\	「二人なんですが、席は空いていますか?」「お席にご案内するまで少しお待ちください。」	
\\	案, 内	
\\	大嫌い	
\\	だいきらい	
\\	な 
\\	コウイチはあなたのことが大嫌いなので、彼に代わってビエトがお会いします。	
\\	嫌い 
\\	大, 
\\	嫌い 
\\	嫌い 
\\	大, 嫌	
\\	高血圧	
\\	こうけつあつ	
\\	フグは高血圧で、塩分の少ない食事を取る必要があるとお医者さんから言われたのに、まだ毎日ハンバーガーを食べているの。	
\\	(血圧)? 
\\	高, 血, 圧	
\\	嫌味	
\\	いやみ	
\\	な 
\\	彼はちょっと頭が悪いよね。どうしてあれが嫌味だってことに気づかないのかしら。	
\\	味 
\\	嫌 
\\	いや. 
\\	いや.
\\	嫌, 味	
\\	観念	
\\	かんねん	
\\	する 
\\	の 
\\	彼女は芸術においては素晴らしい美の観念を持っているのに、どうしてあんなに不細工な男性と付き合っているのかが理解できない。	
\\	観, 念	
\\	不可能	
\\	ふかのう	
\\	な 
\\	あのポーカーのトーナメントで、悪い予感はしてたんだけど、捨てるのは不可能な手札を持っていたんだよ。	
\\	可能 
\\	可能 
\\	不, 可, 能	
\\	産業	
\\	さんぎょう	
\\	の 
\\	日本で最も重要な産業は何ですか?	
\\	産, 業	
\\	勢力	
\\	せいりょく	
\\	その武装勢力は、急速に勢力を増大している。	
\\	勢, 力	
\\	指示	
\\	しじ	
\\	する 
\\	指示を無視するのはよくありませんが、指示待ち族にはなってもらいたくはありません。私の言っている意味が分かりますか?	
\\	指, 示	
\\	対応	
\\	たいおう	
\\	する 
\\	あの医者の対応は素晴らしかった。	
\\	対, 応	
\\	寝不足	
\\	ねぶそく	
\\	な 
\\	「今日も残業するつもりなの?」「えっと…ですね、今日は寝不足なのでどちらかというと早くあがりたいと思っています。」	
\\	不足 
\\	ふ 
\\	不 
\\	ぶ, 
\\	寝, 不, 足	
\\	地価	
\\	ちか	
\\	幸運なことに、私の両親は地価が高騰する前にその土地を買い、最高値で売り抜けました。	
\\	地, 価	
\\	領袖	
\\	りょうしゅう	
\\	領袖の言うことは絶対だ。	
\\	その政治家こそが、党の領袖だ。	
\\	「領袖」と言う言葉は、現代では、政治的な使われることが多い。	
\\	領, 袖	
\\	坊さん	
\\	ぼうさん	
\\	くそ、彼女が付き合ってた坊さんたちには太刀打ちできないよ。あいつら俺とはかなりレベルが違うんだ。	
\\	"さん 
\\	ぼうさん!	坊	
\\	昼寝	
\\	ひるね	
\\	する 
\\	コウイチはただ今昼寝中です。おかけになって少々お待ちください。	
\\	昼, 寝	
\\	年賀状	
\\	ねんがじょう	
\\	年賀状以外の挨拶状は出さない。	
\\	年, 賀, 状	
\\	脱字	
\\	だつじ	
\\	誤字脱字を見つけたら、遠慮せずに教えてくださいね。	
\\	脱, 字	
\\	態と	
\\	わざと	
\\	彼女は、彼に態と気のないふりをした。	
\\	態	
\\	機会	
\\	きかい	
\\	「土曜日に夕食にいらっしゃいませんか?河豚のパイ生地包み焼きを作る予定なんですよ。」「喜んで行きたいのですが、会議がありますので。またの機会にお願いします。」	
\\	機, 会	
\\	寝坊	
\\	ねぼう	
\\	する 
\\	あいにくですが、トフグのオフィスには今、ご対応できる者がおりません。社員がみんな寝坊してしまったのです。	
\\	寝, 坊	
\\	宮	
\\	みや	
\\	主人は宮仕えでとても忙しくしており、私は四週間も顔も見ていません。	
\\	宮	
\\	過去	
\\	かこ	
\\	玉ねぎが目にしみたことは過去に一度もありません。	
\\	過, 去	
\\	罰	
\\	ばつ	
\\	クラスの生徒全員がカンニングで罰を受けたなんて、信じられない。	
\\	罰	
\\	変態	
\\	へんたい	
\\	する 
\\	の 
\\	実は、ちょっと恥ずかしい話なんだけど、俺の親父めちゃくちゃ変態で、児童ポルノの一斉捜査で捕まっちゃったんだよね。	
\\	変, 態	
\\	政策	
\\	せいさく	
\\	お集まりの皆さん、本日は我々の諸政策についてのプレゼンテーションの機会をいただき、有難うございます。	
\\	政, 策	
\\	位置	
\\	いち	
\\	する 
\\	位置について、用意、ドン!	
\\	位, 置	
\\	姿勢	
\\	しせい	
\\	恐らく姿勢が悪いことが首の痛みの原因でしょう。	
\\	姿, 勢	
\\	提出	
\\	ていしゅつ	
\\	する 
\\	トフグライターへの応募者の提出物について教えて頂けますか?また、必要なスキルは何ですか?	
\\	提, 出	
\\	大統領	
\\	だいとうりょう	
\\	ほら、彼はここの大統領としての仕事がとても好きだったんです。 それに、現実を見つめてください。彼がしてた仕事と同じくらい良い仕事をする人を見つけられないでしょう。 そうじゃないですか?	
\\	大, 統, 領	
\\	価格	
\\	かかく	
\\	この価格はちょっと高すぎやしないかい。	
\\	価, 格	
\\	物価	
\\	ぶっか	
\\	日本では、物価が上がり始めたようだ。	
\\	ぶつ 
\\	ぶっ, 
\\	物, 価	
\\	値札	
\\	ねふだ	
\\	この品物には、値札シールがありません。	
\\	王様は、いつも値札を見ずに買い物をして、女王様に怒られます。	
\\	値札を確かめた後に、買うのを止めました。	
\\	値, 札	
\\	私営	
\\	しえい	
\\	の 
\\	私の両親は、私営の葱園を所有しています。	
\\	私, 営	
\\	提案	
\\	ていあん	
\\	する 
\\	「提案はどうだった?」「うーん。まあまあって感じかな。」	
\\	提, 案	
\\	公営	
\\	こうえい	
\\	の 
\\	もし収入がものすごく低いのであれば、安い公営のアパートに住めるかもしれませんよ。	
\\	私営 
\\	公, 営	
\\	公示	
\\	こうじ	
\\	する 
\\	毎年いつ頃日本の地価公示価格が公表されるのか教えて頂けませんか?	
\\	公, 示	
\\	受領書	
\\	じゅりょうしょ	
\\	先週購入した
\\	製品の受領書を送って頂けませんか?	
\\	受, 領, 書	
\\	地域	
\\	ちいき	
\\	スーパーマンはその地域へ食料と新鮮な水を急送した。	
\\	地, 域	
\\	諦観	
\\	ていかん	
\\	する 
\\	父は、様々な経験の後に諦観に至った。	
\\	やる気はやがて、諦観となった。	
\\	君は諦観するにはまだ若い。	
\\	諦, 観	
\\	市営	
\\	しえい	
\\	の 
\\	大阪市は、舞州にすごくユニークな市営のゴミ焼却炉を建設しました。	
\\	市, 営	
\\	県営	
\\	けんえい	
\\	の 
\\	フグは県営図書館でサーモンといちゃついていて、図書館司書さんに怒られた。	
\\	県, 営	
\\	副業	
\\	ふくぎょう	
\\	副業で、私の彼氏はネットワークビジネスをしている。	
\\	副, 業	
\\	観光	
\\	かんこう	
\\	する 
\\	の 
\\	観光案内のパンフレットありますか?	
\\	観, 光	
\\	袖	
\\	そで	
\\	ねえ、袖が汚れているよ。	
\\	このポロシャツ、袖が短すぎてノースリーブに間違われたんだよね。	
\\	先輩の袖から見えた腕の血管に見とれて車にひかれそうになったなんて誰にも言えない。	
\\	(そで). 
\\	袖	
\\	お土産	
\\	おみやげ	
\\	まだお土産を選んでいません。	
\\	(みやげ) 
\\	土, 産	
\\	主観	
\\	しゅかん	
\\	彼女の文章はとても主観的だが、だからこそとてもユニークでもある。	
\\	主, 観	
\\	副題	
\\	ふくだい	
\\	昨晩、私の本にぴったりのすごくいい副題を思いついたの。	
\\	副, 題	
\\	応援	
\\	おうえん	
\\	する 
\\	あの熱心に声を出す応援者はいったい誰なんだ?	
\\	応, 援	
\\	支援	
\\	しえん	
\\	する 
\\	心あるオナラ臭い支援を有難うございます。	
\\	支, 援	
\\	尻尾	
\\	しっぽ	
\\	あの可愛い柴犬が、尻尾を振りながらコウイチに近づき、飛びついた。	
\\	尻 
\\	しり 
\\	しっ. 
\\	尾 
\\	(ぽ) 
\\	尻, 尾	
\\	脱税	
\\	だつぜい	
\\	する 
\\	の 
\\	私は小さい頃、両親から親切な人になるよう教わりましたが、二人は脱税容疑で逮捕されました。	
\\	脱, 税	
\\	一応	
\\	いちおう	
\\	一応念のため、全データのバックアップをこのハードディスクに取っておきました。	
\\	一, 応	
\\	領土	
\\	りょうど	
\\	の 
\\	ブログでは、国家間の領土問題にはあまり触れたくありません。	
\\	領, 土	
\\	有り難う	
\\	ありがとう	
\\	どうして素直に有り難うと言うことができないんだろうね。	
\\	有難うございます? 
\\	有難う.	
\\	ありがとう 
\\	有, 難	
\\	非難する	
\\	ひなんする	する 
\\	子どもたちは両親が不祥事を隠したことを非難した。	
\\	非難 
\\	非難 
\\	非, 難	
\\	脱走する	
\\	だっそうする	する 
\\	重い刑罰を言い渡された犯罪者が、刑務所から脱走した。	
\\	だつ 
\\	だっ, 
\\	脱, 走	
\\	過ぎる	
\\	すぎる	
\\	彼は私には優し過ぎます。	
\\	う 
\\	(す) 
\\	過	
\\	存在する	
\\	そんざいする	する 
\\	私は死後の世界は存在すると思います。	
\\	存在 
\\	存, 在	
\\	倒れる	
\\	たおれる	
\\	そのマラソン選手は、ゴールの直前で疲れきって倒れてしまいました。	
\\	倒す 
\\	(れる) 
\\	倒す. 
\\	倒す 
\\	倒	
\\	諦める	
\\	あきらめる	
\\	死んでも漢字の勉強を諦めないで!	
\\	コウイチはついにポケモンマスターになる夢を諦めた。	
\\	山のようなレビューを諦めずにやるかリセットするかは、とても難しい選択です。	
\\	う 
\\	(あきら)!
\\	諦	
\\	お手洗い	
\\	おてあらい	
\\	お手洗いは、真っすぐ行って左です。	
\\	手洗い 
\\	お 
\\	手洗い, 
\\	手, 洗	
\\	示す	
\\	しめす	
\\	約束なんてしないで。言葉ではなく行いで示してよ。	
\\	う 
\\	(しめ) 
\\	示	
\\	飛ばす	
\\	とばす	
\\	フグは、センターオーバーの三塁打をかっ飛ばした。	
\\	飛ぶ 
\\	(ばす). 
\\	飛ぶ, 
\\	飛	
\\	確かめる	
\\	たしかめる	
\\	ちゃんとドアに鍵をかけたかどうか確かめた方がいいですよ。	
\\	確か, 
\\	確かに, 
\\	確	
\\	圧倒する	
\\	あっとうする	する 
\\	この新しいサイトは他社の日本語学習サイトを圧倒するだろう。	
\\	あつ 
\\	あっ, 
\\	圧, 倒	
\\	罰する	
\\	ばっする	する 
\\	コウイチは
\\	の秘密を漏らした従業員を罰した。	
\\	つ 
\\	っ 
\\	罰	
\\	置く	
\\	おく	
\\	フグと私は今距離を置いてるの。	
\\	(お). 
\\	置	
\\	在留する	
\\	ざいりゅうする	する 
\\	外国人が日本に在留する時に必要になるものは何ですか?	
\\	在留 
\\	在留 
\\	在, 留	
\\	応じる	
\\	おうじる	
\\	先方は、我々の二割りの値下げ要求に応じるでしょうか?	
\\	う 
\\	応	
\\	営む	
\\	いとなむ	
\\	私の両親はマッサージ業を営んでいるので、私はそこで「マッサージ受け放題」なんです。	
\\	(いとな).
\\	営	
\\	震える	
\\	ふるえる	
\\	アル中の父の手は、いつも震えていた。	
\\	う 
\\	(ふる) 
\\	震	
\\	寝る	
\\	ねる	
\\	よし、そろそろ寝るね〜。おやすみ〜。また明日〜。	
\\	う 
\\	寝	
\\	吸う	
\\	すう	
\\	これを最後に、きっぱりとタバコを吸うのを止めるつもりなんだ。	
\\	う 
\\	(す) 
\\	吸	
\\	脱ぐ	
\\	ぬぐ	
\\	「分かったよ。気温が何度だろうが、服を脱いでやるよ。」「いいぞ。その意気だ。」	
\\	う 
\\	(ぬ).	脱	
\\	観る	
\\	みる	
\\	どっちの映画を観るか迷っています。	
\\	う 
\\	見る. 
\\	見 
\\	観	
\\	伝統	
\\	でんとう	
\\	の 
\\	あの日本の伝統衣装の名前って何だっけ?単語をすっかり忘れちゃったよ。	
\\	伝統?	
\\	でん 
\\	伝, 
\\	統 
\\	伝, 統	
\\	面倒臭い	
\\	めんどうくさい	い 
\\	毎日鎌で草を刈るのは本当に面倒くさい。	
\\	(面倒) 
\\	(臭い). 
\\	面倒 
\\	臭い 
\\	面, 倒, 臭	
\\	案外	
\\	あんがい	
\\	な 
\\	日本語学習は私にはかなりハードルが高いんじゃないかと思っていたんですが、案外そうでもなかったです。	
\\	案, 外	
\\	統合	
\\	とうごう	
\\	する 
\\	全ての国が一つの政府に統合される日がいつかやってくると思いますか。	
\\	統, 合	
\\	嫌悪感	
\\	けんおかん	
\\	あなたと話していると、いつも嫌悪感を抱くのです。	
\\	嫌, 悪, 感	
\\	価	
\\	あたい	
\\	の 
\\	その価には同意できないな。タダ同然じゃないか。	
\\	(あたい). 
\\	価	
\\	価値	
\\	かち	
\\	コンピュータの無い人生なんて生きてる価値がない。	
\\	価, 値	
\\	値	
\\	ね, あたい	
\\	の 
\\	羽毛布団はやっぱり値が張るね。	
\\	値	
\\	被害者	
\\	ひがいしゃ	
\\	その犯罪は本当に残忍だったので、被害者家族だけではなく全ての人が裁判官が極刑を言い渡すことを願っている。	
\\	被害 
\\	被, 害, 者	
\\	藤	
\\	ふじ	
\\	メニューにある、この藤セットって何ですか?	
\\	藤	
\\	吸収	
\\	きゅうしゅう	
\\	する 
\\	植物が水分を取り入れる時って、どれくらいのスピードで吸収するんですか?	
\\	吸, 収	
\\	演技	
\\	えんぎ	
\\	する 
\\	の 
\\	あの女優の演技力は素晴らしい。	
\\	演, 技	
\\	俳優	
\\	はいゆう	
\\	の 
\\	私の恋人は、この芝居で馬の役を演じている俳優よ。	
\\	俳, 優	
\\	男優	
\\	だんゆう	
\\	ホームレスの男性が、「主演男優賞を狙いたい」と真面目に語った。	
\\	男, 優	
\\	現に	
\\	げんに	
\\	現にさっきそう言ったじゃありませんか。	
\\	現	
\\	沢山	
\\	たくさん	
\\	な 
\\	の 
\\	今月は沢山の金を手に入れた。	
\\	沢, 山	
\\	弁護士	
\\	べんごし	
\\	グッドマン弁護士はウォルトに会いたくなかったので仮病を使った。	
\\	弁, 護, 士	
\\	新幹線	
\\	しんかんせん	
\\	走りすぎる新幹線の鈍い轟を除いては物音一つ聞こえなかった。	
\\	新, 幹, 線	
\\	バス停	
\\	ばすてい, バスてい	
\\	このバス、あのバス停で止まるのか?おい、このバス、あのバス停で止まるのかって聞いてんだよ!バスの運転手さんよお!お前、耳が聞こえねえのか!	
\\	(バス) 
\\	停	
\\	崎	
\\	さき, みさき	
\\	今日志賀直哉の小説『城の崎にて』を読み始めました。	
\\	崎	
\\	城	
\\	しろ	
\\	もし宝くじが当たったら、裏庭に自分のお城を建てるね。	
\\	城	
\\	委員会	
\\	いいんかい	
\\	いい加減、委員会での重要な問題に対して意見をころころ変えるのをやめてくれないか。みんなにいい迷惑だよ。	
\\	(委員) 
\\	委員 
\\	委, 員, 会	
\\	了解	
\\	りょうかい	
\\	する 
\\	彼は弱々しいうなずいて了解の意を示して、立ち去った。	
\\	了, 解	
\\	保守主義	
\\	ほしゅしゅぎ	
\\	私の両親は、二人とも教師で、めちゃくちゃ古風な保守主義者です。	
\\	主義 
\\	保守的 
\\	保, 守, 主, 義	
\\	鬼	
\\	おに	
\\	おい!下手な芝居はやめて正体を現せ!この、鬼め!	
\\	(おに).
\\	鬼	
\\	割り算	
\\	わりざん	
\\	こんな簡単な割り算の問題を間違えちゃったよ。	
\\	割 
\\	算. 
\\	さん 
\\	ざん. 
\\	割, 算	
\\	医師	
\\	いし	
\\	の 
\\	ほんの短い時間だけでしたが、あの医師は二度テレビに出たことがあります。	
\\	医, 師	
\\	住宅	
\\	じゅうたく	
\\	近年でも、多くの日本の住宅で畳が使われています。	
\\	住, 宅	
\\	経済的	
\\	けいざいてき	な 
\\	私は経済的に苦しい立場にあるため、週五日フルタイムの仕事をしているうえに、週六日パートタイムでも働いています。	
\\	経済 
\\	経済 
\\	経, 済, 的	
\\	経済学	
\\	けいざいがく	
\\	経済学のけの字も知らないので、学校に行くことにしたんです。	
\\	経済 
\\	経済 
\\	経, 済, 学	
\\	有職	
\\	ゆうしょく	
\\	私がインターネットで出会った男性は、三十代で有職ですが、子持ちです。	
\\	有, 職	
\\	施行	
\\	しこう, せこう	
\\	する 
\\	この法律の施行の責任を誰が負っていたのか覚えていますか?	
\\	しこう). 
\\	せこう, 
\\	施, 行	
\\	羨ましい	
\\	うらやましい	い 
\\	日本語がペラペラと話せる人が羨ましい。	
\\	ビジネスで成功した上に、サングラスが似合うなんて、私はあなたが羨ましいですよ。	
\\	私に彼氏ができたって言ったら、妹が羨ましがると思うの。	
\\	い, 
\\	羨	
\\	表現	
\\	ひょうげん	
\\	する 
\\	昨夜の夢で、私たちの赤ちゃんに「乳歯が抜けそうだった時、言葉では表現できないほど痒かった」って言われたの。	
\\	表, 現	
\\	贅沢	
\\	ぜいたく	
\\	な 
\\	二十歳の誕生日だからって贅沢な事は何もしたくないわ。良い本が読めればそれでいいの。	
\\	贅, 沢	
\\	反則	
\\	はんそく	
\\	する 
\\	つけ睫毛を付けるなんて反則だよ。ほとんど別人みたいじゃないか。	
\\	反, 則	
\\	優しい	
\\	やさしい	い 
\\	フグはちょっとヤンチャだけど、ああ見えてすごく優しいよ。	
\\	い 
\\	(やさ). 
\\	優	
\\	指導	
\\	しどう	
\\	する 
\\	彼女は不注意ですが、とても頭のいい指導者です。	
\\	指, 導	
\\	俳句	
\\	はいく	
\\	彼が今日詠んだ俳句、すごくロマンチックだったわよね。	
\\	俳, 句	
\\	宅	
\\	たく	
\\	今日、宅配便でラクダが届いた。	
\\	宅	
\\	収入	
\\	しゅうにゅう	
\\	その海外での仕事に就けば、おそらく収入が半分になるだろう。	
\\	収, 入	
\\	施設	
\\	しせつ	
\\	する 
\\	児童養護施設の仕事がきついので辞めたいんですが、一方で今まで築き上げてきた関係も壊したくはないんです。	
\\	施, 設	
\\	求職	
\\	きゅうしょく	
\\	する 
\\	の 
\\	失業なんて気にするなよ。俺なんてかれこれもう約二年間も求職活動中だぜ。	
\\	求, 職	
\\	川崎	
\\	かわさき	
\\	川崎病という子どもの病気があると聞きましたが、それはどんな病気ですか?	
\\	川, 崎	
\\	法律	
\\	ほうりつ	
\\	の 
\\	もし一ヶ月の間に全法律用語を覚えることができたら、百万ドルお支払いしてもいいですよ。	
\\	法, 律	
\\	革	
\\	かわ	
\\	フグは日本で鰐革のサンダルを買った。	
\\	皮 
\\	かわ? 
\\	革	
\\	皮革	
\\	ひかく	
\\	何度か洗濯したら、皮革の色があせ始めたんだけど、もしかしたら洗っちゃいけなかったのかな。	
\\	皮, 革	
\\	各駅停車	
\\	かくえきていしゃ	
\\	の 
\\	この各駅停車はちょうどトンネルを通り過ぎたところです。	
\\	各, 駅, 停, 車	
\\	牛乳	
\\	ぎゅうにゅう	
\\	「おはよう。はい、牛乳よ。」「ああ、ありがとう。やっぱりこれだよね。一日の始まりに、美味しい牛乳に勝るものはないね。」	
\\	牛, 乳	
\\	沢	
\\	さわ	
\\	フロリダのとある沢のほとりで、巨大なワニを見ました。	
\\	さわ 
\\	沢	
\\	長崎	
\\	ながさき	
\\	あなたへのお土産に長崎カステラを買いましたよ。	
\\	長 
\\	長い. 
\\	崎 
\\	ながさき. 
\\	長, 崎	
\\	乳首	
\\	ちくび	
\\	の 
\\	あんたの乳首から血が噴き出しているのを見て、思わずパニクッちゃったよ!	
\\	乳 
\\	ち, 
\\	首 
\\	くび. 
\\	(ちくび). 
\\	乳, 首	
\\	看護師	
\\	かんごし	
\\	その看護師は患者に作り笑いをした。	
\\	看, 護, 師	
\\	職業	
\\	しょくぎょう	
\\	の 
\\	彼女の職業はモデルなの?写真写りがとてもいいんね。	
\\	職, 業	
\\	自宅	
\\	じたく	
\\	フグ、現実を見つめて。今のところ、サーモンはまだあなたのこと好きじゃないわよね!でも私がそれを良くできるわ!まずは彼女を自宅に招待するといいわよ。	
\\	自, 宅	
\\	楽天主義	
\\	らくてんしゅぎ	
\\	あいつほど楽天主義なやつはいない。	
\\	主義 
\\	楽, 天, 主, 義	
\\	演芸	
\\	えんげい	
\\	演芸のチケットはどこで買えますか?	
\\	演, 芸	
\\	上演	
\\	じょうえん	
\\	する 
\\	ライオンキングの上演時間を教えて下さい。	
\\	上, 演	
\\	職員	
\\	しょくいん	
\\	の 
\\	私は教師ではありませんが、学校職員です。	
\\	職, 員	
\\	法則	
\\	ほうそく	
\\	もし兎が高速で走っている電車の車内で真っすぐ上に飛び跳ねても、慣性の法則によって同じ場所に着地することになります。	
\\	法, 則	
\\	職場	
\\	しょくば	
\\	もしこの職場に残りたいのなら、物事をたいそうに考える癖を無くしてください。	
\\	職場 
\\	職, 場	
\\	停電	
\\	ていでん	
\\	する 
\\	どうしてまだ停電してるって分かってるのに、電気のスイッチを何度も付けたり消したりしちゃうんだろう。	
\\	停, 電	
\\	準備	
\\	じゅんび	
\\	する 
\\	おはよう。朝ご飯の準備ができてるわよ。	
\\	準, 備	
\\	現在	
\\	げんざい	
\\	現在一日に何件の
\\	を受信しますか?	
\\	現, 在	
\\	総理府	
\\	そうりふ	
\\	彼女のことについては、総理府で働いているという以外には何も知らなかった。	
\\	(総理) 
\\	総理 
\\	総, 理, 府	
\\	優先	
\\	ゆうせん	
\\	する 
\\	の 
\\	我が社は新しいベーコンフレーバーの開発を優先するべきだ。	
\\	優, 先	
\\	割合	
\\	わりあい	
\\	日本では、蕎麦粉の割合が100
\\	の蕎麦を十割り蕎麦と呼びます。	
\\	割, 合	
\\	現実	
\\	げんじつ	
\\	「頼むから、現実を見てくれよ!これはケーキじゃなくてうんちなんだ!」	
\\	現, 実	
\\	領収書	
\\	りょうしゅうしょ	
\\	これを返品したいのですが。これが領収書です。	
\\	領, 収, 書	
\\	辞職	
\\	じしょく	
\\	する 
\\	上司にはっきりと「あなたの代わりはいくらでもいる」と言われたので、その会社を辞職しました。	
\\	辞, 職	
\\	革命	
\\	かくめい	
\\	の 
\\	革命の後、情勢は好転した。	
\\	革, 命	
\\	律動	
\\	りつどう	
\\	する 
\\	今朝、たくさんの兎が庭で律動的に飛び跳ねているのを見ました。	
\\	リズム 
\\	律動的 
\\	律, 動	
\\	役割	
\\	やくわり	
\\	インターネットは今後の選挙において益々重要な役割を担っていくでしょう。	
\\	役割.	役, 割	
\\	規則	
\\	きそく	
\\	「それが規則だ。分かったか?」「分かりました。」	
\\	規, 則	
\\	規律	
\\	きりつ	
\\	規律に従って酒は一滴たりとも飲んでいないと断言します。	
\\	規, 律	
\\	現場	
\\	げんば	
\\	の 
\\	あの犯罪現場で実際に何が起きたのかがすぐに明るみに出ることを期待しよう。	
\\	現, 場	
\\	責任感	
\\	せきにんかん	
\\	彼は責任感が強いのかもしれないけど、私には少し失礼で自分勝手な気がしました。	
\\	"責任 
\\	感 
\\	責, 任, 感	
\\	台詞	
\\	せりふ	
\\	こんな長い台詞、よく覚えられるわね。	
\\	(せりふ).	台, 詞	
\\	基準	
\\	きじゅん	
\\	の 
\\	私達は、製品の品質において、業界基準を大きく上回っています。	
\\	基, 準	
\\	停止	
\\	ていし	
\\	する 
\\	私の兄は一度心肺停止にまで至ったんですが、外科医が心臓マッサージを施した後に、心臓が再び脈動しだしたんです。	
\\	停, 止	
\\	優秀	
\\	ゆうしゅう	
\\	な 
\\	優秀なシステム管理者は、ハッキングによるいかなる攻撃の可能性にも備えます。	
\\	優, 秀	
\\	教師	
\\	きょうし	
\\	の 
\\	あの教師は歴史認識をころころ変えるからあまり好きじゃない。	
\\	教, 師	
\\	一割	
\\	いちわり	
\\	「この河豚、ちょっと高すぎますよ。まけてもらえませんか?」「一割引きにしてあげましょう。」	
\\	一, 割	
\\	現れる	
\\	あらわれる	
\\	私の猫はおそらく夕飯時頃に現れるでしょう。	
\\	(あらわ) 
\\	現	
\\	呼ぶ	
\\	よぶ	
\\	彼はあなたの命の恩人なんだから、誕生日会に呼んだ方がいいわよ。	
\\	う 
\\	(よ)! 
\\	呼	
\\	確認する	
\\	かくにんする	する 
\\	我が社がついに赤字を脱したことを確認した時、私は胸を撫で下ろしました。	
\\	確認 
\\	確認 
\\	確, 認	
\\	保護する	
\\	ほごする	する 
\\	あそこで虫をついばんでいる鳥は、国際間で保護することが取り決められた鳥です。	
\\	保, 護	
\\	批判する	
\\	ひはんする	する 
\\	私の彼は、散々私を批判した後で飴をくれた。	
\\	批判 
\\	批判, 
\\	批, 判	
\\	断つ	
\\	たつ	
\\	酔っぱらった時に何をしでかしたかを考慮すると、あなたは絶対にお酒を断つべきですよ。	
\\	う 
\\	断る 
\\	つ, 
\\	(つ) 
\\	断る, 
\\	たつ. 
\\	(たつ). 
\\	断	
\\	割る	
\\	わる	
\\	7割る2が3で余りが1になることを解くのに一時間かかった。	
\\	う 
\\	わ 
\\	わる.	割	
\\	選挙する	
\\	せんきょする	する 
\\	の会長を毎年選挙しています。	
\\	選挙 
\\	選挙 
\\	選, 挙	
\\	検査する	
\\	けんさする	する 
\\	このクーポン券を使えば、血液を検査するのに10ドルの節約になります。	
\\	検査 
\\	検査 
\\	検, 査	
\\	秀でる	
\\	ひいでる	
\\	私は鼻くそをほじくることだけには秀でています。実際、鼻血を出すはめになったこともありません。	
\\	う 
\\	(ひい) 
\\	秀	
\\	済ます	
\\	すます	
\\	私は本当に何もやっていません。今朝そこで、父のためにちょっと用事を済ませていただけなんです。	
\\	済む 
\\	(ます) 
\\	済む, 
\\	済	
\\	増やす	
\\	ふやす	
\\	平日に家族と過ごす時間を増やすために、宿題は全部週末にやることにしてるんだ。	
\\	増える 
\\	(やす) 
\\	増える. 
\\	増	
\\	裁く	
\\	さばく	
\\	その男が殺人犯であることがわかったが、時効を迎えているので法的には誰も彼を裁く事ができない。	
\\	う 
\\	(さば). 
\\	裁	
\\	判断する	
\\	はんだんする	する 
\\	聞いたことから判断すると、真打ちは最後にあるようだね。	
\\	判断 
\\	判断 
\\	判, 断	
\\	心得る	
\\	こころえる	
\\	このお方をどなたと心得るか。	
\\	心得 
\\	心, 得	
\\	導く	
\\	みちびく	
\\	あなたは父のことを誤解していると思います。彼は決して誰かを凶悪犯罪へと導くような人ではありません。	
\\	う 
\\	道 
\\	みち. 
\\	道 
\\	(みちび)!	導	
\\	演ずる	
\\	えんずる	
\\	こんな大役を演ずることができて、とても光栄です。	
\\	演 
\\	演	
\\	備える	
\\	そなえる	
\\	気持ちを引き締めて敵の攻撃に備えろ!	
\\	う 
\\	(そな) 
\\	備	
\\	挙げる	
\\	あげる	
\\	私は結婚式を市役所で挙げましたが、そのことを恥ずかしいとは思っていません。	
\\	挙がる, 
\\	挙がる, 
\\	挙	
\\	解く	
\\	とく	
\\	プレゼントのリボンを解くのが好きなんです。	
\\	う 
\\	(く) 
\\	解ける, 
\\	解	
\\	張る	
\\	はる	
\\	バレーボールのネットを張ったのは誰ですか?	
\\	う 
\\	(は) 
\\	張	
\\	理解する	
\\	りかいする	する 
\\	お前が彼のことを愛しているのは理解するけど、あいつがお前にキスをするのは許せないよ。	
\\	理解 
\\	理解 
\\	理, 解	
\\	幹	
\\	みき	
\\	木の幹に絡まっているツタを取り除いた方がいいよ。	
\\	幹	
\\	裁判	
\\	さいばん	
\\	する 
\\	の 
\\	この貯金箱には小遣いからためた80ドルが入っているんだけど、裁判を起こすには到底足りないと思うんだよね。	
\\	はん 
\\	ばん 
\\	裁, 判	
\\	無職	
\\	むしょく	
\\	の 
\\	彼は完璧な夫ですが、一つだけ問題があるんです。十年間ずっと無職なんです。	
\\	無, 職	
\\	不文律	
\\	ふぶんりつ	
\\	学校では日本語でしか話してはいけないという不文律のようなものがあります。	
\\	不, 文, 律	
\\	吸血鬼	
\\	きゅうけつき	
\\	あの吸血鬼の野郎、俺の事、血を吸いたくない程インチキ臭すぎるって言うんだ。	
\\	吸血 
\\	鬼 
\\	き. 
\\	吸, 血, 鬼	
\\	優勝	
\\	ゆうしょう	
\\	する 
\\	ジャイアンツは去年はボロ負けだったが、今年はリーグ優勝を果たした。	
\\	優, 勝	
\\	素晴らしい	
\\	すばらしい	い 
\\	「そんな風に自分を安売りしないで。君は素晴らしい女性だよ。」	
\\	晴れる 
\\	はれる. 
\\	は 
\\	ば.	素, 晴	
\\	上述	
\\	じょうじゅつ	する 
\\	の 
\\	上述のように、雇用者、従業員のどちらも、いついかなる理由においても契約を終了することができる。	
\\	上, 述	
\\	年額	
\\	ねんがく	
\\	前回のサービスも良かったとは思いますが、今回の新サービスはもっとよくなったと思うんです。それでいて、会員費は変わらず年額80ドルのままなんですよ。	
\\	年, 額	
\\	触角	
\\	しょっかく	
\\	触角の先端に目があるなんて知らなかったので、その蝸牛の触角を切り落としてしまった。	
\\	しょく 
\\	しょっ.	触, 角	
\\	腕	
\\	うで	
\\	「フグのこと、どう思う?」「正直に言うと、私の好みじゃないわ。私はキンニクマみたいに筋肉モリモリの腕の男が好きなの。」	
\\	腕	
\\	誕生日	
\\	たんじょうび	
\\	「君の誕生日会に友達を連れて行ってもいい?」「もちろん! 人数が多ければ多いほど楽しいしね。」	
\\	誕生 
\\	日 
\\	ひ, 
\\	び 
\\	誕, 生, 日	
\\	輸血	
\\	ゆけつ	
\\	する 
\\	彼女はちょうど東京に向かおうとしていた矢先に交通事故に巻き込まれ、現在病院で輸血を受けている。	
\\	輸, 血	
\\	血管	
\\	けっかん	
\\	の 
\\	彼は血管が切れそうなぐらい怒っていた。	
\\	けつ 
\\	けっ.	血, 管	
\\	両腕	
\\	りょううで, もろうで	
\\	両腕に注射の跡がたくさんあります。	
\\	両 
\\	腕. 
\\	両, 腕	
\\	境界	
\\	きょうかい	
\\	母と私は、いつもお互いの個人的な境界線のようなものを気にしています。	
\\	境, 界	
\\	境	
\\	さかい	
\\	この川がちょうど両州の境となっているんですよ。	
\\	(さかい) 
\\	境	
\\	負担	
\\	ふたん	
\\	する 
\\	10キロのお米を持ち上げることは、彼の腰の負担となった。	
\\	負, 担	
\\	可燃ゴミ	
\\	かねんごみ, かねんゴミ	
\\	あなたが私から貰うクリスマスプレゼントは、可燃ごみだけだよ。	
\\	ゴミ 
\\	可, 燃	
\\	特質	
\\	とくしつ	
\\	笑いは人間だけの特質なのかな?それとも動物も笑うのかな?	
\\	特, 質	
\\	子供	
\\	こども	
\\	お前の子供、イケメンすぎて、お前の子供だとは思えねーよ。	
\\	子 
\\	こ, 
\\	供 
\\	ども, 
\\	子ども 
\\	子供 
\\	子, 供	
\\	親展	
\\	しんてん	
\\	の 
\\	もし「親展」と書かれている封書を勝手に開けたら、信書開封罪とみなされて一年以下の懲役又は二十万円以下の罰金に課せられるかもしれませんよ。	
\\	親, 展	
\\	対策	
\\	たいさく	
\\	する 
\\	コーヒーを入れようとしてボタンを押したら、コーヒーメーカーが火を噴いたの。不具合の原因を見つけて、再発防止対策を講じなきゃいけないわ。	
\\	対, 策	
\\	記述	
\\	きじゅつ	
\\	する 
\\	もしその男に関する記述が誤植だったら、間違った人を逮捕してしまうかもしれないよ。	
\\	記, 述	
\\	違い	
\\	ちがい	
\\	清掃員と副社長の違いなんて全く分からないね。	
\\	い 
\\	違	
\\	差別	
\\	さべつ	
\\	する 
\\	私は偏見や差別は好きではありませんが、そういう冗談は好きです。	
\\	差, 別	
\\	象	
\\	ぞう	
\\	あの象、別の動物園から来た雌の象と昨夜交尾をしたよ。	
\\	ぞう 
\\	ぞう 
\\	象	
\\	環境	
\\	かんきょう	
\\	の 
\\	環境保護の観点では、固形石けんの方が液体石けんよりも優れている。	
\\	環, 境	
\\	本質	
\\	ほんしつ	
\\	の 
\\	この男は、物事の本質を一瞬で見抜くことができる。	
\\	本, 質	
\\	輸出	
\\	ゆしゅつ	
\\	する 
\\	な 
\\	の 
\\	出張の際にヘロウィンを密輸出しようとしたため、彼は試験期間のうちに契約を打ち切られた。	
\\	輸, 出	
\\	遠視	
\\	えんし	
\\	の 
\\	遠視が進んできたので、そろそろ老眼鏡を買わなきゃいけないかもしれない。	
\\	遠, 視	
\\	定額	
\\	ていがく	
\\	の 
\\	あの〜、定額貯金の口座を開設したいんですが。	
\\	定, 額	
\\	定量	
\\	ていりょう	
\\	の 
\\	どうにかしてこのリスクを定量化する方法はないのかな。	
\\	定, 量	
\\	額	
\\	がく	
\\	この絵に合う額を探しています。	
\\	額	
\\	燃料	
\\	ねんりょう	
\\	単なる燃料切れだったよ。	
\\	燃, 料	
\\	良質	
\\	りょうしつ	
\\	な 
\\	の 
\\	良質のワインを飲む事が私の唯一の楽しみだ。	
\\	良, 質	
\\	官庁	
\\	かんちょう	
\\	主に官庁内で使われる特別な用語ってありますか?	
\\	官, 庁	
\\	半額	
\\	はんがく	
\\	コウイチ、そろそろ起きる時間よ!ウォールマートの半額セールに行かなきゃいけないんだから。	
\\	半, 額	
\\	高値	
\\	たかね	
\\	の 
\\	コウイチの秘密の日記が、オークションで歴代最高値で落札された。	
\\	高, 値	
\\	祝日	
\\	しゅくじつ	
\\	の 
\\	コウイチは祝日はいつもかなり派手な格好をしている。	
\\	祝, 日	
\\	自家製	
\\	じかせい	
\\	の 
\\	私がホテルで自家製ビールを楽しんでいる時に、警察が自宅へ襲撃を決行したようです。	
\\	自, 家, 製	
\\	担当	
\\	たんとう	
\\	する 
\\	もしこの染みが抜けなかったら、私はこのプロジェクトの担当から外されてしまうかもしれない。	
\\	担, 当	
\\	交差点	
\\	こうさてん	
\\	次の交差点を左に曲がって下さい。	
\\	交, 差, 点	
\\	多額	
\\	たがく	
\\	な 
\\	の 
\\	最終的に会議はうまくいって、なんとか我々の科学調査への多額の寄付金を約束してもらうことができたよ。	
\\	多, 額	
\\	大量	
\\	たいりょう	
\\	な 
\\	の 
\\	台所に蟻が大量発生しているのを見て、完全に取り乱してしまいました。	
\\	大, 量	
\\	朝寝坊	
\\	あさねぼう	
\\	する 
\\	私は朝寝坊だが、妹は私と違って早起きだから、君の犬の散歩に行けるかもしれないよ。	
\\	寝坊 
\\	朝 
\\	寝坊 
\\	朝, 寝, 坊	
\\	法規	
\\	ほうき	
\\	海戦に関する国際法規について書かれた本は、この図書館にありますか。	
\\	法, 規	
\\	〜層	
\\	そう	
\\	ホッケーの競技場さえも、オゾン層の破壊に関係しているって、知ってたかい?	
\\	層	
\\	左腕	
\\	ひだりうで, さわん	
\\	左腕を骨折しました。	
\\	わん. 
\\	さわん 
\\	(わん) 
\\	左, 腕	
\\	長袖	
\\	ながそで	
\\	の 
\\	今日は、長袖着ていった方がいいかな?	
\\	長袖のパーカーを買いに行くのに、付き合ってくれない?	
\\	私の彼氏は、タトゥー隠すため夏でも長袖を着ることがある。	
\\	長, 袖	
\\	安値	
\\	やすね	
\\	の 
\\	私はこの長靴を驚くほどの安値で買いましたが、防水加工がされてませんでした。	
\\	安, 値	
\\	アメリカ製	
\\	あめりかせい, アメリカせい	
\\	アメリカ製の寿司は食べられないだって?好き嫌いするんじゃないよ!	
\\	製	
\\	中国製	
\\	ちゅうごくせい	
\\	インターネット上で、中国製の日本語学習サイトがどれだけあるのかについては、正確な数字は今は分かりません。調べてのちほどお答えします。	
\\	(中国) 
\\	中国 
\\	中, 国, 製	
\\	日本製	
\\	にほんせい	
\\	その日本製の刀の柄には、美しい宝石が散りばめられている。	
\\	(日本) 
\\	日本 
\\	日, 本, 製	
\\	通販	
\\	つうはん	
\\	する 
\\	の 
\\	今まで通販で買い物したことないけど、ちょっとやってみるよ。	
\\	通, 販	
\\	管理	
\\	かんり	
\\	する 
\\	時間管理法の授業の講師が遅刻しているなんて、信じられない。	
\\	管, 理	
\\	製作	
\\	せいさく	
\\	する 
\\	の 
\\	村田製作所は、電子部品を製造している日本の会社です。	
\\	製, 作	
\\	武道	
\\	ぶどう	
\\	武道の経験が無い日本人はたくさんいます。	
\\	武, 道	
\\	品質	
\\	ひんしつ	
\\	彼女は品質管理マネージャーとして入社した。	
\\	品, 質	
\\	都庁	
\\	とちょう	
\\	東京都庁の展望台に行った事はありますか。	
\\	都, 庁	
\\	警視庁	
\\	けいしちょう	
\\	私の兄は、東京都警視庁公安部に勤めています。	
\\	警, 視, 庁	
\\	型	
\\	かた	
\\	どうして日本人は血液型をそんなに気にするんですか?	
\\	(かた) 
\\	型	
\\	月額	
\\	げつがく	
\\	月額たったの5ドルでできる新しい日本語学習体験を促進するための広告キャンペーンを打ち始めます。	
\\	月, 額	
\\	一層	
\\	いっそう	
\\	の 
\\	かわいそうなことに、彼の視力はレーシック手術の後に一層悪化してしまった。	
\\	いち 
\\	いっ.	一, 層	
\\	狭い	
\\	せまい	い 
\\	コウイチ、おもちゃを片付けなさい!さもないと、暗くて狭いクローゼットに閉じ込めるわよ。	
\\	い 
\\	狭	
\\	国境	
\\	こっきょう, くにざかい	
\\	の 
\\	国境警備隊の隊員がアメリカに不法入国した外国人を国境のそばの茂みで見つけることはよくあることです。	
\\	こく 
\\	こっ.	国, 境	
\\	警察庁	
\\	けいさつちょう	
\\	警視庁と警察庁の違いは何ですか?	
\\	警視庁, 
\\	(警察) 
\\	警, 察, 庁	
\\	近視	
\\	きんし	
\\	の 
\\	私は近視と乱視の両方があります。	
\\	近, 視	
\\	不燃ゴミ	
\\	ふねんごみ, ふねんゴミ	
\\	不燃ゴミは火曜日に出してください。	
\\	(ごみ). 
\\	不, 燃	
\\	提供	
\\	ていきょう	
\\	する 
\\	日本で臓器提供者になる同意書にサインをするには、どうすればいいですか?	
\\	提, 供	
\\	営業中	
\\	えいぎょうちゅう	
\\	営業中に電話しないでって言ったじゃない。	
\\	営業 
\\	営業 
\\	営, 業, 中	
\\	過去形	
\\	かこけい	
\\	この場合は過去形は使わないんですか?	
\\	形容詞), 
\\	過, 去, 形	
\\	現象	
\\	げんしょう	
\\	日本に
\\	現象を調査している
\\	団体はありますか?	
\\	現, 象	
\\	管	
\\	くだ	
\\	妹は二歳の時に甘い咳止めシロップを一気飲みしてしまい、鼻から管を通して胃を洗浄するはめになった。	
\\	だ,
\\	管	
\\	置き場	
\\	おきば	
\\	洗面台の下にタオル置き場があるから、シャワーの後勝手に取って使ってね。	
\\	置く 
\\	置く 
\\	場 
\\	置, 場	
\\	所載	
\\	しょさい	
\\	『古事記』所載の日本最古の和歌について何か知っていますか?	
\\	所, 載	
\\	製品	
\\	せいひん	
\\	我が社の新製品に関する会議を火曜日の午後に延期することは可能かな?	
\\	製, 品	
\\	原型	
\\	げんけい	
\\	の 
\\	私たちはまず初めに、粘土で彫刻の原型を作るんです。	
\\	原, 型	
\\	質	
\\	しつ	
\\	外国語指導助手の質は人によって全然違うので、心配です。	
\\	質	
\\	量	
\\	りょう	
\\	仕事の量を考えると、この給料じゃ割りに合わない。	
\\	量	
\\	質問	
\\	しつもん	
\\	する 
\\	の 
\\	彼はある観客からの予想外の質問にたじろいだ。	
\\	質, 問	
\\	記載	
\\	きさい	
\\	する 
\\	小論文を提出したが、日付を記載するのを忘れてしまった。	
\\	記, 載	
\\	残額	
\\	ざんがく	
\\	残額は五百円あるので、次の旅行に繰り越すこともできるし、半日市内バス観光をすることもできます。	
\\	残, 額	
\\	応援団	
\\	おうえんだん	
\\	ああ、君が応援団長だね?噂はかねがね伺っておるよ。	
\\	応援 
\\	応, 援, 団	
\\	販売	
\\	はんばい	
\\	する 
\\	我が社の販売員の多くは、八月に休暇を取ります。	
\\	販, 売	
\\	武士	
\\	ぶし	
\\	コウイチは武士のようにストイックに走り続け、フルマラソンを完走した。	
\\	武, 士	
\\	視覚	
\\	しかく	
\\	の 
\\	視覚障害のある社員を差別したとの理由から、上司が解雇された。	
\\	視, 覚	
\\	半袖	
\\	はんそで	
\\	の 
\\	半袖で寒くない?	
\\	コンマリの本を読んでから、半袖のシャツは一枚しか持っていない。	
\\	今年のファッショントレンドは半袖の革ジャンです。	
\\	半 
\\	袖. 
\\	半, 袖	
\\	肩	
\\	かた	
\\	肩がこっているんだけど、ちょっと揉んでくれない?	
\\	肩	
\\	規準	
\\	きじゅん	
\\	の 
\\	彼は世間の規準からはちょっと外れた、ユニークな人です。	
\\	規, 準	
\\	株式	
\\	かぶしき	
\\	株式市場が底なしの下落に陥ったようだというニュースは、彼に大きな精神的ダメージを与えた。	
\\	株, 式	
\\	株	
\\	かぶ	
\\	コウイチは株でガッポリ儲けたので、車のタイヤをどでかいホイール・タイヤに変えて、シートも革張りにしてカッチョ良くしちゃおうと決めた。	
\\	株	
\\	主観的	
\\	しゅかんてき	な 
\\	これはとても主観的な見方かもしれないけど、僕にとっては君は働き過ぎだと思うんだよね。近々休みを取った方がいいよ。	
\\	主観 
\\	主観.	主, 観, 的	
\\	対象	
\\	たいしょう	
\\	の 
\\	勤続三十年の従業員は、勤続手当として二週間連続有給休暇の対象となる。	
\\	対, 象	
\\	時差	
\\	じさ	
\\	まだ時差ボケに苦しんでいるんです。	
\\	時, 差	
\\	援助する	
\\	えんじょする	する 
\\	俺はポテトチップス一袋を買うのに、他人からの援助を必要とするほど貧しい。	
\\	援助 
\\	援助. 
\\	援, 助	
\\	届く	
\\	とどく	
\\	からの手紙がちょうど今届きました。	
\\	う 
\\	(く), 
\\	とど!	届	
\\	違う	
\\	ちがう	
\\	ベテラン社員とは違って、新入社員の基本給はあまり高くない。	
\\	う 
\\	違	
\\	載る	
\\	のる	
\\	新入社員の代表として受けた私のインタビュー記事が、社内報の春季号に載りました。	
\\	う 
\\	(の)! 
\\	載	
\\	燃やす	
\\	もやす	
\\	何かを完全に灰になるまで燃やすことが大好きだ。	
\\	う 
\\	も 
\\	(も), 
\\	燃	
\\	担ぐ	
\\	かつぐ	
\\	日本に行ったら、お祭りで神輿を担いでみたい。	
\\	う 
\\	担う 
\\	(かつ). 
\\	担	
\\	祝う	
\\	いわう	
\\	日曜日におじいちゃんの還暦を祝うために、中華を食べに行きます。	
\\	う 
\\	祝う!	
\\	いわう 
\\	いい 
\\	いい
\\	いい
\\	いい 
\\	祝	
\\	発展する	
\\	はってんする	する 
\\	年寄りの風邪は肺炎に発展しやすいんだよ。	
\\	はつ 
\\	はっ, 
\\	発, 展	
\\	無視する	
\\	むしする	する 
\\	休暇届を申請したが、上司に無視された。	
\\	無, 視	
\\	供える	
\\	そなえる	
\\	明日お墓に供える花を、帰りに買ってきてくれない?	
\\	(そな), 
\\	供	
\\	量る	
\\	はかる	
\\	体重計で自分の体重を量るのが恐い。	
\\	う 
\\	(はか) 
\\	量	
\\	述べる	
\\	のべる	
\\	上司に全く正反対の意見を述べるのは気が引けます。	
\\	う 
\\	(の) 
\\	述	
\\	過ぎ	
\\	すぎ	
\\	短時間の仕事で良い給料をもらい過ぎだよ。	
\\	過ぎる 
\\	過ぎる. 
\\	過ぎ.	過	
\\	寝坊する	
\\	ねぼうする	する 
\\	「しまった!寝坊した!」と思ったんですが、有り難いことに日曜日でした。	
\\	寝坊 
\\	寝坊.	寝, 坊	
\\	指差す	
\\	ゆびさす	
\\	人を指差すのは失礼だよ。	
\\	指, 差	
\\	提案する	
\\	ていあんする	する 
\\	本日は、新しい投資信託をご提案させて頂きたいと思います。	
\\	提案, 
\\	提, 案	
\\	応援する	
\\	おうえんする	する 
\\	どちらの力士を応援しているんだい?	
\\	"応援 
\\	応援. 
\\	応, 援	
\\	営業する	
\\	えいぎょうする	する 
\\	道ばたで営業する占い師が、どれくらいのお金を稼ぐのか知ってる?	
\\	営業 
\\	営業.	営, 業	
\\	差す	
\\	さす	
\\	窓の外を見ると、もう雨が上がって薄日が差していた。	
\\	さ).	差	
\\	支援する	
\\	しえんする	する 
\\	何が起ころうが君を支援しようとしたけど、できなかったんだ。	
\\	支, 援	
\\	全額	
\\	ぜんがく	
\\	お菓子代は、全額自己負担でお願いします。	
\\	全, 額	
\\	触る	
\\	さわる	
\\	上司が私のお尻を触った翌日、私は会社を一日休んだ。	
\\	う 
\\	(さわ).	触	
\\	感触	
\\	かんしょく	
\\	する 
\\	このストールの感触、シルクみたい。	
\\	感, 触	
\\	観光客	
\\	かんこうきゃく	
\\	その観光客は、アメリカで医者の診断書をもらうために50ドルを支払わなければならなかった。	
\\	(観光) 
\\	観光 
\\	観, 光, 客	
\\	展示会	
\\	てんじかい	
\\	私の兄は、カフェで初めての美術展示会を開催した。	
\\	展, 示, 会	
\\	伝統的	
\\	でんとうてき	な 
\\	褌は、伝統的な日本の男性用下着です。	
\\	的 
\\	伝, 統, 的	
\\	輸送	
\\	ゆそう	
\\	する 
\\	輸送コストを教えて下さい。	
\\	輸, 送	
\\	典型的	
\\	てんけいてき	な 
\\	どうして彼のことを典型的な日本人だっていうの?	
\\	典, 型, 的	
\\	展開	
\\	てんかい	
\\	する 
\\	中小企業が海外展開をする利点とは何でしょうか。	
\\	展, 開	
\\	価値観	
\\	かちかん	
\\	結婚相手と価値観が近いことって結構大切だと思うんだよねー。	
\\	(価値) 
\\	価, 値, 観	
\\	自動販売機	
\\	じどうはんばいき	
\\	10月23日の自動販売機に関する会議は、コウイチの誕生日会のため、10月24日に変更されました。	
\\	(自動) 
\\	自, 動, 販, 売, 機	
\\	副大統領	
\\	ふくだいとうりょう	
\\	やってみるべきだと思うけど。大統領は無理でも、副大統領ぐらいなら狙えるんじゃない?	
\\	大統領 
\\	大統領 
\\	副, 大, 統, 領	
\\	腰	
\\	こし	
\\	腰に激痛が走ってるんだ。もう一歩も歩けないよ。	
\\	腰	
\\	武器	
\\	ぶき	
\\	全ての警察官に武器を使う資格があるわけではないというのは本当ですか?	
\\	武, 器	
\\	気象	
\\	きしょう	
\\	の 
\\	気象予報士と気象学者の違いが説明できますか。	
\\	気, 象	
\\	木製	
\\	もくせい	
\\	の 
\\	俺の木製バットは、あいつの内角速球によって折れてしまったんだ。	
\\	木, 製	
\\	腕時計	
\\	うでどけい	
\\	俺の無くなった腕時計についてちょっと話がしたいんだけど、いいか?あまり時間は取らせないよ。	
\\	時計 
\\	腕 
\\	時計. 
\\	腕, 時, 計	
\\	告訴	
\\	こくそ	
\\	する 
\\	の 
\\	毎月きちんとお金を払って頂けるなら、告訴はしません。	
\\	告, 訴	
\\	女優	
\\	じょゆう	
\\	の 
\\	「ホグワーツ魔法魔術学校にはもう行きたくない。女優になりたいの。」「もう、馬鹿なこと言わないで。ハーマイオニーったら。」	
\\	男優 
\\	女, 優	
\\	効果	
\\	こうか	
\\	の 
\\	ジムで水泳をすることが、最も効果的なダイエット方法だと聞きました。	
\\	効, 果	
\\	逮捕	
\\	たいほ	
\\	する 
\\	ラッシュ時の電車内で痴漢行為を働いたとして、危うく誤認逮捕されるところだった。	
\\	逮, 捕	
\\	所属	
\\	しょぞく	
\\	する 
\\	の 
\\	所属部署に病欠のための診断書を提出しておいた方がいいよ。	
\\	所, 属	
\\	票	
\\	ひょう	
\\	コウイチは票の大半を得て相手候補者に圧勝し、アメリカ合衆国の大統領となった。	
\\	票	
\\	賞与金	
\\	しょうよきん	
\\	給料日は毎月5日と20日で、会社の業績次第で賞与金が年末に支払われます。	
\\	賞, 与, 金	
\\	景況	
\\	けいきょう	
\\	日本の景況感が大きく上向くのは、まだまだ先のことだろう。	
\\	景, 況	
\\	候補者	
\\	こうほしゃ	
\\	学歴的には、候補者として十分通用するはずが、彼は性格に難があるからね。	
\\	候, 補, 者	
\\	気候	
\\	きこう	
\\	気候は荒れ模様だが、新しい仕事ではいいスタートが切れた。	
\\	気, 候	
\\	全景	
\\	ぜんけい	
\\	私は息を飲むような富士の全景を期待していたが、実際はかなりお粗末なものだった。	
\\	全, 景	
\\	状況	
\\	じょうきょう	
\\	の 
\\	しっかりしろよ。状況はお前が思っているほど悪くないみたいだぜ。	
\\	状, 況	
\\	捜査	
\\	そうさ	
\\	する 
\\	の 
\\	本日は捜査に関する重要書類を作成する予定です。	
\\	捜, 査	
\\	習慣	
\\	しゅうかん	
\\	の 
\\	「たばこ、止めたんじゃなかったの?」「まあ、ほら、『古い習慣はなかなかなくならない』って言うだろ。」	
\\	習, 慣	
\\	絞殺	
\\	こうさつ	
\\	する 
\\	「あんた、もう少しであの男を絞殺しかけたのよ!」「ごめん。やりすぎたよ。」	
\\	絞, 殺	
\\	割引	
\\	わりびき	
\\	する 
\\	の 
\\	ちょうどホームページへ割引クーポンの追加をしようと思ってたところだよ。	
\\	き 
\\	ひき 
\\	引き 
\\	びき 
\\	割, 引	
\\	反響	
\\	はんきょう	
\\	する 
\\	この記事への反響はとても大きかった。	
\\	反, 響	
\\	効力	
\\	こうりょく	
\\	あなたの今日のプレゼンテーションは、前回に比べて効力が少し弱いような気がしました。	
\\	効, 力	
\\	効率	
\\	こうりつ	
\\	効率が第一だということを、しっかり心に刻んで置いてください。	
\\	効, 率	
\\	年輩	
\\	ねんぱい	
\\	の 
\\	年輩の上司が、牛の乳搾り体験の費用を払ってくれた。	
\\	はい 
\\	ぱい, 
\\	年, 輩	
\\	結構	
\\	けっこう	
\\	な 
\\	「今日は他には何か、ございませんか?」「いいえ。それだけで結構です。」	
\\	結構です 
\\	結構大きい 
\\	結構です 
\\	けつ 
\\	けっ.	結, 構	
\\	上巻	
\\	じょうかん	
\\	上巻の終わりは、すごく続きが気になる感じで終わった。	
\\	上, 巻	
\\	前景	
\\	ぜんけい	
\\	このカメラでは、どうすれば前景の被写体にピントを合わせることができるんですか?	
\\	前, 景	
\\	新鮮	
\\	しんせん	な 
\\	前の職場では、止めろと言われるまでいつも残業していたので、毎日五時に退社するのはとても新鮮です。	
\\	新, 鮮	
\\	鮮魚	
\\	せんぎょ	
\\	俺の魚屋で働いている限り、従業員は福利厚生として鮮魚をもらう権利を、放棄することはできないぜ。	
\\	鮮, 魚	
\\	満員	
\\	まんいん	
\\	の 
\\	毎朝、新宿駅に着くまで、満員電車の中に立っていなくてはいけないのが苦痛です。	
\\	満, 員	
\\	満月	
\\	まんげつ	
\\	の 
\\	明日の夜は満月なんだって。曇らないといいね。	
\\	満, 月	
\\	含意	
\\	がんい	
\\	する 
\\	誰にとっても、含意を汲み取るのは難しいことだ。	
\\	含, 意	
\\	無限	
\\	むげん	な 
\\	の 
\\	今は絶好調のように思えるが、需要は無限には増加しないだろう?	
\\	無, 限	
\\	影響	
\\	えいきょう	
\\	残業は健康に悪影響を及ぼす可能性があるということを考えた方がいいですよ。	
\\	影, 響	
\\	不規則	
\\	ふきそく	
\\	な 
\\	不規則な生活で体調を崩してしまった。	
\\	規則 
\\	不, 規, 則	
\\	後輩	
\\	こうはい	
\\	お前は俺のお気に入りの後輩だから、どんなことがあってもちゃんと面倒みてやるよ。	
\\	後 
\\	こう 
\\	こういち.	後, 輩	
\\	下巻	
\\	げかん	
\\	上巻を買ったつもりが下巻を買ってしまっていた。	
\\	下 
\\	げ 
\\	巻 
\\	下, 巻	
\\	訴訟	
\\	そしょう	
\\	する 
\\	の 
\\	その訴訟に関する書類は、弁護士によって全てきちんとファイルされている。	
\\	訴, 訟	
\\	革ジャン	
\\	かわじゃん, かわジャン	
\\	お洒落大好き人間にアドバイスされたってだけで、あんな高い革ジャンは買うわけないじゃん。	
\\	革 
\\	ジャン 
\\	ジャンパー 
\\	革, 
\\	革	
\\	限界	
\\	げんかい	
\\	の 
\\	その野球選手は、自分の限界を感じて引退した。	
\\	限, 界	
\\	限度	
\\	げんど	
\\	このクレジットカードの限度額がいくらだったか忘れちゃったんだよね。	
\\	限, 度	
\\	限定	
\\	げんてい	
\\	する 
\\	の 
\\	お母さん、あの限定販売の
\\	人形が欲しいんだけど、百万ドルくれない?	
\\	限, 定	
\\	肥料	
\\	ひりょう	
\\	肥料工場での社員間のいざこざについて、どう対処していいか全くわかりませんでした。	
\\	肥, 料	
\\	金属	
\\	きんぞく	
\\	厳しい景況にもかかわらず、あの金属メッキ工場は成長を続けている。	
\\	金, 属	
\\	掛け算	
\\	かけざん	
\\	明日、塾で掛け算のテストがあるんです。	
\\	掛 
\\	算. 
\\	さん 
\\	ざん. 
\\	掛, 算	
\\	長崎県	
\\	ながさきけん	
\\	私は長崎県の出身で、家族の墓も全部そこにあります。	
\\	長 
\\	長い, 
\\	崎 
\\	県.	長, 崎, 県	
\\	肥満	
\\	ひまん	
\\	する 
\\	の 
\\	あの肥満上司が、私たちに対する全権限を持っていて、先週は二名を解雇しやがったんだよ。	
\\	肥, 満	
\\	一巻	
\\	いっかん	
\\	一巻では、全ての引用に脚注がつけられ出典が記されています。	
\\	いち 
\\	いっ.	一, 巻	
\\	構成	
\\	こうせい	
\\	する 
\\	の 
\\	この映画は二部構成になっている。	
\\	構, 成	
\\	模様	
\\	もよう	
\\	私の妻は、小さな花の模様が好きです。	
\\	様 
\\	(よう). 
\\	模, 様	
\\	満点	
\\	まんてん	
\\	の 
\\	満点をとったら、先生に花丸をもらえるの。	
\\	満, 点	
\\	模型	
\\	もけい	
\\	の 
\\	祖父は、最初は従業員一人の小売模型店からこのビジネスを始めました。	
\\	模, 型	
\\	豊か	
\\	ゆたか	な 
\\	毎日満員電車にぎゅうぎゅう詰めになって乗り込む人がたくさんいる以上、日本の生活がとても豊かだと言う事はできないね。	
\\	豊, 
\\	(ゆた) 
\\	豊	
\\	豊満	
\\	ほうまん	
\\	な 
\\	彼女は豊満な胸をもっており、いつも美しい谷間を覗かせている。	
\\	豊, 満	
\\	不況	
\\	ふきょう	
\\	な 
\\	の 
\\	不況時に金の価格は上昇しやすい。	
\\	不, 況	
\\	規則正しい	
\\	きそくただしい	い 
\\	規則正しい生活を送って下さい。	
\\	規則 
\\	正しい, 
\\	規則 
\\	正しい, 
\\	規, 則, 正	
\\	隠居	
\\	いんきょ	
\\	する 
\\	お前は本当に隠居したいのか?	
\\	隠, 居	
\\	革命家	
\\	かくめいか	
\\	あの革命家は、ネット弁けいだ。	
\\	革命家になりたければ、まずは効果的な話し方を身につけないといけないよ。	
\\	彼女は情熱的なフランス人革命家と恋に落ちた。	
\\	(革命) 
\\	革命 
\\	革, 命, 家	
\\	律動的	
\\	りつどうてき	な 
\\	この音楽はとても律動的ですね。	
\\	律動, 
\\	的 
\\	律動. 
\\	的 
\\	律, 動, 的	
\\	満足	
\\	まんぞく	
\\	する 
\\	な 
\\	ビジネスが順調なので、ようやく満足しています。	
\\	足 
\\	ぞく 
\\	満, 足	
\\	規模	
\\	きぼ	
\\	あなたにとってはただの規模の小さな会社かもしれませんが、私は両親の会社に誇りをもっています。	
\\	規, 模	
\\	景色	
\\	けしき	
\\	の 
\\	ちょっといつもと違った景色を見るために、今日は回り道をしてみない?	
\\	けい 
\\	け 
\\	色 
\\	しき. 
\\	景, 色	
\\	景観	
\\	けいかん	
\\	私は京都の景観がとても好きです。	
\\	景, 観	
\\	時候	
\\	じこう	
\\	日本では、通常手紙は時候の挨拶から書き始められる。	
\\	時, 候	
\\	二巻	
\\	にかん	
\\	二巻では、引用の出典に関する別表が作成されました。	
\\	二, 巻	
\\	先輩	
\\	せんぱい	
\\	俺の先輩、悪い人では無いんだけどさ、ただちょっと俺とは反りが合わないんだよね。	
\\	(せんぱい).	
\\	先, 輩	
\\	影	
\\	かげ	
\\	「私、もう何週間もフグを見てないわ。」「ほら、見て。ちょうど今フグが来たわよ。」「あら、噂をすれば影がさす、ね。こんにちは。私たち、今ちょうどあなたのことを話していたのよ。」	
\\	(かげ). 
\\	影	
\\	光景	
\\	こうけい	
\\	このような絵のように美しい光景は今までみたことがありません。	
\\	光, 景	
\\	期限	
\\	きげん	
\\	ビザの期限が切れるので、来月日本を出なくてはいけません。	
\\	期, 限	
\\	天候	
\\	てんこう	
\\	繁忙期に旅行をすれば費用が嵩むのは分かっているけど、そうはいっても天候もその時期が一番いいからね。	
\\	天, 候	
\\	時限	
\\	じげん	
\\	の 
\\	時限爆弾がきちんと引き出しの中にしまわれていることを確認した。	
\\	時, 限	
\\	吸収する	
\\	きゅうしゅうする	する 
\\	全社員が新システムに関する知識を吸収する良い機会です。	
\\	吸収 
\\	吸, 収	
\\	居る	
\\	いる	
\\	うちの猫は大抵二階の寝室に居るんです。	
\\	う 
\\	いる? 
\\	居	
\\	呼ばれる	
\\	よばれる	
\\	おばちゃんと呼ばれる年になってしまったことがちょっと悲しかった。	
\\	呼ぶ 
\\	呼ぶ, 
\\	呼	
\\	捕まる	
\\	つかまる	
\\	最初スピード違反で止められたんだけど、その後飲酒運転で捕まっちゃったんだよ。	
\\	う 
\\	捕える, 
\\	つか, 
\\	(つか) 
\\	捕	
\\	慣れる	
\\	なれる	
\\	裾上げをお願いできますか?自分ではあまり縫い物に慣れてなくて。	
\\	う 
\\	(な) 
\\	慣	
\\	割れる	
\\	われる	
\\	大便がしたくてたまらなかった彼は、ドアを拳で割れるほど強く叩いた。	
\\	割る 
\\	割る.	割	
\\	効く	
\\	きく	
\\	肺癌の治療に最も良く効く薬を探しています。	
\\	う 
\\	(き). 
\\	効	
\\	属する	
\\	ぞくする	
\\	する 
\\	四年生の時は誰のグループに属していたの?	
\\	う 
\\	属	
\\	替わる	
\\	かわる	
\\	もしご迷惑でなければ、席を替わってもらえませんか?彼女と一緒なんですが、席が別々になってしまったんです。	
\\	う 
\\	替える, 
\\	わる 
\\	替	
\\	構う	
\\	かまう	
\\	「こちらの品は百万円ほどお高くなります。」「それでも構わないよ。それを頂くよ。」	
\\	う 
\\	(かま). 
\\	構	
\\	捉える	
\\	とらえる	
\\	コップの半分まで入れられた水は、人によって「半分しかない」とも「半分もある」とも捉えられます。	
\\	日本人のクライアントの本音を捉えるのは難しい。	
\\	そのサーカス団は、開演してすぐに観客の心を捉えた。	
\\	う 
\\	捉	
\\	渡る	
\\	わたる	
\\	橋を渡り終わったところで、右に曲がって下さい。	
\\	う 
\\	渡	
\\	響く	
\\	ひびく	
\\	大きな契約を勝ち取るには、顧客の心に響く素晴らしい提案書が必要だ。	
\\	う 
\\	(ひび) 
\\	(ひび) 
\\	(ひび) 
\\	(ひび) 
\\	(ひび) 
\\	響	
\\	抜く	
\\	ぬく	
\\	ワサビ抜きにしてもらえますか?	
\\	う 
\\	抜	
\\	与える	
\\	あたえる	
\\	まず初めに、お忙しい中足をお運び頂き誠に有り難うございます。また、このような素晴らしいプレゼンテーションの機会を与えて頂いたことにも、感謝申し上げます。	
\\	う 
\\	(あた) 
\\	与	
\\	渡す	
\\	わたす	
\\	彼が私にペンを渡してくれた時、胸がどきどきした。	
\\	う 
\\	す? 
\\	渡る 
\\	す 
\\	渡	
\\	掛ける	
\\	かける	
\\	彼の死後、くれた絵をどこに掛けたらいいかしら。	
\\	う 
\\	(ける) 
\\	掛	
\\	隠す	
\\	かくす	
\\	秘密の花園への入口は隠されている。	
\\	う 
\\	(かく)! 
\\	隠	
\\	含む	
\\	ふくむ	
\\	アルカリを含む温泉にはどのような効能があるのですか?	
\\	う 
\\	(ふく) 
\\	含	
\\	増す	
\\	ます	
\\	砂糖でさらに甘みが増しますよ。	
\\	増! 
\\	増やす 
\\	(ま)! 
\\	増	
\\	限る	
\\	かぎる	
\\	やっぱり、日本語の勉強は
\\	に限るよ。	
\\	う 
\\	(かぎ) 
\\	限	
\\	見渡す	
\\	みわたす	
\\	見渡す限り、向日葵畑が続いていた。	
\\	う 
\\	見, 渡	
\\	頑張る	
\\	がんばる	
\\	「そのまま頑張って。あなたなら大丈夫よ。」	
\\	張る 
\\	頑 
\\	張る 
\\	張る 
\\	頑, 張	
\\	準備する	
\\	じゅんびする	する 
\\	準備する前に先に朝ご飯食べちゃいなさい。	
\\	準備 
\\	準備.	準, 備	
\\	替える	
\\	かえる	
\\	お父さんに電球を替えるようにもう頼んだよ。	
\\	う 
\\	(える). 
\\	替	
\\	巻く	
\\	まく	
\\	二時間かけて髪を巻いたが、それでもあまりうまく巻けなかった。	
\\	う 
\\	(ま). 
\\	巻	
\\	捜す	
\\	さがす	
\\	祖母を捜しているんです。アルツハイマーがあって、昨日から家に帰ってないんです。	
\\	う 
\\	(さが) 
\\	捜	
\\	現す	
\\	あらわす	
\\	彼女は最後には本性を現すと思ってたよ。	
\\	う 
\\	現れる? 
\\	現	
\\	収める	
\\	おさめる	
\\	争いを収めるにはどうすればいいのだろう。	
\\	(おさ). 
\\	収	
\\	絞める	
\\	しめる	
\\	弟は羊に首を絞められた。	
\\	絞る 
\\	める 
\\	める 
\\	(し)! 
\\	絞	
\\	訴える	
\\	うったえる	
\\	多くの女子社員がセクハラを訴えたため、私の直属の上司は二、三週間前に解雇されました。	
\\	う 
\\	(うった) 
\\	訴	
\\	補う	
\\	おぎなう	
\\	欠員を補うために、ボーリングゲームに参加しました。	
\\	う 
\\	(おぎな) 
\\	ぎ 
\\	補	
\\	優れる	
\\	すぐれる	
\\	あのころは、景子先輩より優れている人はだれもいなかったの。	
\\	トーフグはユニークさにおいて、他の日本語学習教材より優れています。	
\\	コウイチは納豆の模型を作るのに優れている。	
\\	今すぐ 
\\	今すぐ!!!
\\	優	
\\	投票する	
\\	とうひょうする	する 
\\	この会社では、毎年
\\	を選ぶために全社員が投票することになっている。	
\\	投 
\\	投資 
\\	投, 票	
\\	絞る	
\\	しぼる	
\\	ぞうきんを絞って作ったジュースです。	
\\	会議でアイデアを絞りましょう。	
\\	絞める 
\\	絞る, 
\\	(める) 
\\	し 
\\	絞める, 
\\	ぼ 
\\	しぼ. 
\\	(ぼる). 
\\	絞	
\\	家庭教師	
\\	かていきょうし	
\\	家庭教師のアルバイトをしています。	
\\	家庭 
\\	教師. 
\\	家, 庭, 教, 師	
\\	鮮やか	
\\	あざやか	な 
\\	新しい人事部長は、強制的なサービス残業に関する問題を鮮やかに解決してのけた。	
\\	(あざ) 
\\	鮮	
\\	事故	
\\	じこ	
\\	ひき逃げ事故のせいで、買ったばかりの牛肉が挽き肉になった。	
\\	事, 故	
\\	針医	
\\	はりい	
\\	針医の技術の進歩には驚くばかりです。	
\\	針 
\\	医. 
\\	針 
\\	(はり). 
\\	針, 医	
\\	再度	
\\	さいど	
\\	相手は、埋め立て計画に反対する請願書を再度持ってくるかもしれませんよ。	
\\	い 
\\	再, 度	
\\	再び	
\\	ふたたび	
\\	もしできるなら、再び肘を舐めてみるべきだ。	
\\	再, 
\\	再	
\\	独り	
\\	ひとり	
\\	彼は独り住居だ。	
\\	一人 
\\	独り 
\\	ひとり 
\\	独	
\\	間違い	
\\	まちがい	
\\	新しい市場に参入したことは間違いじゃなかった。	
\\	間 (ま 
\\	間に合う) 
\\	違う 
\\	間, 違	
\\	人造	
\\	じんぞう	
\\	の 
\\	その外交官は人造人間十八号に刺された。	
\\	人, 造	
\\	獣	
\\	けもの	
\\	あの男の人、獣のような目をしていて、恐いわ。	
\\	(物, もの) 
\\	(毛, け), 
\\	けもの 
\\	けもの. 
\\	獣	
\\	獣類	
\\	じゅうるい	
\\	の 
\\	引越しを考えてるんだけど、どこも鳥獣類禁止ばっかりなの。	
\\	獣, 類	
\\	大違い	
\\	おおちがい	
\\	堅い木の椅子と柔らかい快適な椅子じゃあ大違いだよ。	
\\	違い 
\\	違う 
\\	大 
\\	大きい (おお) 
\\	(お), 
\\	大, 違	
\\	酒造	
\\	しゅぞう	
\\	ジムの会員権や食事だけではなく酒まで無料の特典として提供されるので、私はこの酒造での仕事が好きです。	
\\	酒, 造	
\\	名刺	
\\	めいし	
\\	彼が受話器に戻ってきて会社の住所を尋ねたので、名刺に記載されていることをお伝えしました。	
\\	名, 刺	
\\	運河	
\\	うんが	
\\	冬には運河の上でスケートができますよ。	
\\	か 
\\	が, 
\\	運, 河	
\\	お菓子	
\\	おかし	
\\	小さい頃、お菓子の家に住むことが夢でした。	
\\	菓, 子	
\\	特徴	
\\	とくちょう	
\\	優柔不断と事なかれ主義が彼の特徴だ。	
\\	特, 徴	
\\	比較	
\\	ひかく	
\\	する 
\\	の 
\\	偽物のダイヤと本物のダイヤを直接比較してみれば、どっちがどっちだかすぐに分かるよ。	
\\	比, 較	
\\	お祝い	
\\	おいわい	
\\	する 
\\	引っ越しお祝いを渡さなくちゃいけないわね。	
\\	お 
\\	祝	
\\	株式会社	
\\	かぶしきがいしゃ, かぶしきかいしゃ	
\\	株式会社に電話を入れて、ファックスの送信がきちんと完了しているかどうか確認してくれる?	
\\	(会社) 
\\	株, 式, 会, 社	
\\	気違い	
\\	きちがい	
\\	の 
\\	この吹雪の中、パンツ一丁で踊るなんて、気違いにも程がある。	
\\	気 
\\	違い.	気, 違	
\\	創造	
\\	そうぞう	
\\	する 
\\	私の妹は、創造力がとても豊かだ。	
\\	創, 造	
\\	討論	
\\	とうろん	
\\	する 
\\	の 
\\	私は給与削減と年金減額に関する討論に加わった。	
\\	討, 論	
\\	独裁	
\\	どくさい	
\\	する 
\\	の 
\\	俺は世界一の独裁者になるんだ。	
\\	独, 裁	
\\	直接	
\\	ちょくせつ	
\\	な 
\\	の 
\\	君と彼が僕のベーコンを食べたって話は、彼から直接聞いたんだよ。	
\\	直, 接	
\\	障害	
\\	しょうがい	
\\	する 
\\	障害があることがバレて、内定が取り消された。	
\\	障, 害	
\\	不振	
\\	ふしん	
\\	な 
\\	私の
\\	はしばらく売上不振に苦しんでいたが、ラジオで紹介されてから今月は売上げが上昇しています。	
\\	不, 振	
\\	従業	
\\	じゅうぎょう	
\\	する 
\\	ここの従業員たちは、大阪支店の従業員たちより楽観的なようだ。	
\\	従, 業	
\\	激励	
\\	げきれい	
\\	する 
\\	の 
\\	テレビ番組が放映された後、何百にも及ぶ激励の電話が、数時間に渡って掛かってきた。	
\\	激, 励	
\\	創立	
\\	そうりつ	
\\	する 
\\	電車に乗り遅れて、会社の創立記念のイベントに遅れそうだ。	
\\	創, 立	
\\	激しい	
\\	はげしい	い 
\\	時々、夜になると激しい不安を感じるの。	
\\	い 
\\	(はげ). 
\\	激	
\\	第一印象	
\\	だいいちいんしょう	
\\	オペレーターは、電話をかけてくださったお客様にとって、会社の 第一印象になるということを教えられました。	
\\	(第一) 
\\	第, 一, 印, 象	
\\	占い	
\\	うらない	
\\	星占いによると、フグと私ってあまり相性が良くないみたいなんだよね。	
\\	い 
\\	(うらな) 
\\	占	
\\	万年筆	
\\	まんねんひつ	
\\	万年筆を落とされましたよ。	
\\	万, 年, 筆	
\\	河豚	
\\	ふぐ, ふく	
\\	「豆腐と河豚のサンドイッチをお願いします。」 「同じく。」	
\\	ふぐ 
\\	河, 豚	
\\	怪談	
\\	かいだん	
\\	アメリカで有名な学校の怪談はありますか。	
\\	怪, 談	
\\	怪獣	
\\	かいじゅう	
\\	スーパーマンは、怪獣に数分遅れて現れた。	
\\	怪, 獣	
\\	氷河	
\\	ひょうが	
\\	地球温暖化のために、毎年溶けている氷河が増えている。	
\\	氷 
\\	(ひょう).	氷, 河	
\\	保障	
\\	ほしょう	
\\	する 
\\	日本にはどんな社会保障がありますか。	
\\	保, 障	
\\	お誕生日おめでとう	
\\	おたんじょうびおめでとう	
\\	遅れたとしても、「お誕生日おめでとう」って言う方が言わないよりもいいよ。	
\\	"誕生日 
\\	おめでとう 
\\	誕生日, 
\\	誕, 生, 日	
\\	我	
\\	われ	
\\	の 
\\	我関せずって感じのアイツの態度、超嫌なんだけど。	
\\	(われ) 
\\	我	
\\	輸入	
\\	ゆにゅう	
\\	する 
\\	の 
\\	兄の会社が輸入家具マーケットにビジネスに広げるのですが、私はそのプロジェクトマネージャーに抜擢されました。	
\\	輸, 入	
\\	独立	
\\	どくりつ	
\\	する 
\\	な 
\\	会社を辞めて独立した人で成功した人は多くないって言うなら、そのデータやその人たちの計画、そしてどうしてそれが失敗したのかを示してくださいよ。	
\\	独, 立	
\\	人差し指	
\\	ひとさしゆび	
\\	電気コードに躓いた時に、人差し指を切っちゃったんだよね。	
\\	人, 
\\	差す, 
\\	指. 
\\	人, 差, 指	
\\	間接	
\\	かんせつ	
\\	の 
\\	寝室用の落ち着いた感じのロマンチックな間接照明を探しているんです。	
\\	間, 接	
\\	故意	
\\	こい	
\\	コウイチの秘密のベーコンを故意に食べた罪を認めます。	
\\	故, 意	
\\	面接	
\\	めんせつ	
\\	する 
\\	面接の合否に関して、印刷物の郵送は行っておりません。	
\\	面, 接	
\\	障子	
\\	しょうじ	
\\	障子から差し込む光で、朝目を覚ますのが好きです。	
\\	子 
\\	し 
\\	じ, 
\\	障, 子	
\\	河童	
\\	かっぱ	
\\	まるで河童が存在していることを信じているかのようにリアクションしてもらえますか?	
\\	河 
\\	童 
\\	河, 童	
\\	造園	
\\	ぞうえん	
\\	大学を卒業した時、造園会社から内定をもらっていました。	
\\	造, 園	
\\	鉛	
\\	なまり	
\\	今日はすごくやる気が無くて、体が鉛のように重く感じる。	
\\	(なまり).
\\	鉛	
\\	鉛管	
\\	えんかん	
\\	水道の配管には、かつては鉛管がよく使われていました。	
\\	鉛, 管	
\\	鉛毒	
\\	えんどく	
\\	の 
\\	かつてたくさんの歌舞伎役者が鉛毒中毒に苦しみました。	
\\	鉛, 毒	
\\	授業	
\\	じゅぎょう	
\\	する 
\\	昨日病気で欠席したため、今日は授業の遅れを取り戻すために忙しい。	
\\	授, 業	
\\	郵便	
\\	ゆうびん	
\\	郵便業においては、効率とスピードが全てだ。	
\\	便 (びん) 
\\	(びん). 
\\	郵, 便	
\\	故障	
\\	こしょう	
\\	する 
\\	あのファックス機は故障中だよ。	
\\	故, 障	
\\	管理人	
\\	かんりにん	
\\	忙しい一日を終えて、管理人に「早く仕事を切り上げて帰りたい」という魔がさした。	
\\	"管理 
\\	管, 理, 人	
\\	製造	
\\	せいぞう	
\\	する 
\\	の 
\\	織り物の製造元でアメリカで事業を拡張しているところがいくつかあるって聞いたんだけど、一体どうやって拡張してるのか知ってる?	
\\	製, 造	
\\	印	
\\	しるし	
\\	私は手紙の最後にハートの印を付けるのが好きです。	
\\	(しるし)!	印	
\\	読み違い	
\\	よみちがい	
\\	今日、超恥ずかしい読み違いをしてしまった。	
\\	読, 違	
\\	人違い	
\\	ひとちがい	
\\	する 
\\	はい、私達も結合双生児ではありますが、申し訳ありませんがあなたは人違いをしていると思います。	
\\	人違い!	
\\	人 
\\	違い 
\\	人, 違	
\\	独占	
\\	どくせん	
\\	する 
\\	うちの家族の場合は、父がテレビを独占しています。	
\\	独, 占	
\\	怪物	
\\	かいぶつ	
\\	この山には、心の優しい怪物が住んでいるという言い伝えがある。	
\\	怪, 物	
\\	再来月	
\\	さらいげつ	
\\	再来月に一週間の休暇が取れるんだけど、どこか一緒にいかない?	
\\	(来月) 
\\	再, 来, 月	
\\	従順	
\\	じゅうじゅん	
\\	な 
\\	あいつは犬のように従順だ。	
\\	従, 順	
\\	独創	
\\	どくそう	
\\	する 
\\	彼が独創性に欠けていることは非常に残念です。	
\\	独, 創	
\\	鉛筆	
\\	えんぴつ	
\\	そのパグは、この
\\	の鉛筆で描いたんだよ。	
\\	鉛, 筆	
\\	豚	
\\	ぶた	
\\	上司に豚呼ばわりされて腹が立ったので、速攻で人事課に退職届けを叩きつけてやった。	
\\	豚	
\\	豚肉	
\\	ぶたにく	
\\	の 
\\	ランチに食べた豚肉に当たったかもしれません。すごく気持ち悪いんですが、早退させてもらえませんかね。	
\\	豚, 肉	
\\	故	
\\	ゆえ	
\\	それ故、兄は両親から勘当されてしまったのです。	
\\	故	
\\	再建	
\\	さいけん	
\\	する 
\\	潰れかかった会社を再建するまで、彼は毎日規定よりも長い時間働いた。	
\\	い!	再, 建	
\\	回復	
\\	かいふく	
\\	する 
\\	インターネットへの接続はまだ回復していません。	
\\	回, 復	
\\	改造	
\\	かいぞう	
\\	する 
\\	言うのは恥ずかしいんですが、あの改造バイクに乗ってるのは私の兄です。	
\\	改, 造	
\\	再開	
\\	さいかい	
\\	する 
\\	コウイチとビエトの交渉は、近々再開される予定だ。	
\\	い!	再, 開	
\\	復習	
\\	ふくしゅう	
\\	する 
\\	予習と復習はどちらも大切です。	
\\	復, 習	
\\	右腕	
\\	みぎうで, うわん	
\\	奴の右腕には、龍のタトゥーがあるハズだ。	
\\	右, 腕	
\\	河	
\\	かわ	
\\	毎日夕方頃、河の土手で犬を散歩させる。	
\\	川 
\\	河	
\\	往復	
\\	おうふく	
\\	する 
\\	の 
\\	ここから大阪まで、往復でいくらですか?	
\\	往, 復	
\\	貯金	
\\	ちょきん	
\\	する 
\\	もし退職金が一括で支払われることになったら、それを使い切ってしまわずにちゃんと貯金することができますか。	
\\	貯, 金	
\\	秒針	
\\	びょうしん	
\\	腕時計の秒針が折れちゃってるの。	
\\	秒, 針	
\\	無我	
\\	むが	
\\	私は鮫に追いかけられて、無我夢中で泳ぎました。	
\\	無, 我	
\\	独学	
\\	どくがく	
\\	する 
\\	の 
\\	インターネットの普及によって、独学で外国語を勉強することは昔よりもずっと簡単になった。	
\\	独, 学	
\\	独身	
\\	どくしん	
\\	の 
\\	私は独身で、婚活中です。とりあえずそんな感じですが、何か他に聞きたいことがあれば、遠慮なくご連絡ください。	
\\	独, 身	
\\	刺激	
\\	しげき	
\\	する 
\\	彼女には会ったんだけど、お互いにいまいちピンと来なかったんだよな。俺としては、もうちょっと刺激的な女性の方がいいっていうかさ。	
\\	刺, 激	
\\	株式市場	
\\	かぶしきしじょう	
\\	世界的な景気の減速は、私たちの株式市場にも影響を及ぼした。	
\\	(場) 
\\	株, 式, 市, 場	
\\	怪事件	
\\	かいじけん	
\\	ホームズでさえもその怪事件の糸口を見つけることはできないだろう。	
\\	(事件) 
\\	怪, 事, 件	
\\	構造	
\\	こうぞう	
\\	今日はエッセイの構造とルールについて授業をします。	
\\	構, 造	
\\	突然	
\\	とつぜん	
\\	な 
\\	の 
\\	突然、猫のお化けが台所に現れて、「ニャーニャーソング」を歌い始めたんだ。	
\\	突, 然	
\\	怪しい	
\\	あやしい	い 
\\	彼女は同僚に比べて病欠が多いし、本当に病気なのか怪しいよね。	
\\	い 
\\	怪	
\\	汗	
\\	あせ	
\\	彼の額は汗でピカリと光っていた。	
\\	(あせ).	汗	
\\	汗臭い	
\\	あせくさい	い 
\\	私は彼の汗臭い枕が好きだった。	
\\	(臭い). 
\\	汗, 臭	
\\	象徴	
\\	しょうちょう	
\\	する 
\\	の 
\\	鳩は世界共通の平和の象徴だ。	
\\	象, 徴	
\\	教授	
\\	きょうじゅ	
\\	する 
\\	の 
\\	その教授は、ファイルよりも書類整理箱を好んだ。	
\\	教, 授	
\\	燃え付く	
\\	もえつく	
\\	化学の実験をしている最中に、前髪に火が燃え付いた。	
\\	燃える 
\\	付く 
\\	燃, 付	
\\	届ける	
\\	とどける	
\\	今夜お前のところに俺のパソコンと外付けハードディスクを届けるよ。	
\\	"届く 
\\	(ける) 
\\	届く 
\\	届	
\\	差別する	
\\	さべつする	する 
\\	誰かに差別されたことはありますか。	
\\	差別 
\\	差, 別	
\\	励ます	
\\	はげます	
\\	落ち込んでいる人を励ますのは難しい。	
\\	う 
\\	(はげ). 
\\	励	
\\	検討する	
\\	けんとうする	する 
\\	分厚いノートパソコンから薄型タブレットへの買い替えを検討している。	
\\	検, 討	
\\	輸出する	
\\	ゆしゅつする	する 
\\	彼らはライオン、トラ、くまなどたくさんの動物を輸出する。	
\\	輸出 
\\	輸出.	輸, 出	
\\	触れる	
\\	ふれる	
\\	誰かが肩に触れるのを感じて振り返ったが、そこには誰もいなかった。	
\\	"触る 
\\	(れる) 
\\	触る 
\\	(ふれ).	触	
\\	燃える	
\\	もえる	
\\	城は燃える木のにおいで充満しているようでした。	
\\	燃やす 
\\	える 
\\	燃やす.	燃	
\\	差し上げる	
\\	さしあげる	
\\	パソコンの動作を速くするのに役立つアプリケーションを無料で差し上げます。	
\\	差す 
\\	上げる 
\\	差, 上	
\\	障る	
\\	さわる	
\\	彼女の金切り声が時々癪に障るんだよね。	
\\	(さわ). 
\\	障	
\\	造る	
\\	つくる	
\\	私の最近の趣味は、手作りのビールを造ることなんですよ。	
\\	作る, 
\\	作る 
\\	造	
\\	狭める	
\\	せばめる	
\\	もう少しここの幅を狭めることはできますか?	
\\	う 
\\	(せば) 
\\	狭	
\\	従う	
\\	したがう	
\\	もし彼の命令に従わなければ、どうなると思う?	
\\	う 
\\	が (したが) 
\\	従	
\\	占める	
\\	しめる	
\\	がベーコン市場を買い占めるつもりだと聞きましたよ。	
\\	う 
\\	(し). 
\\	占	
\\	載せる	
\\	のせる	
\\	自分の写真をインターネットに載せるのには、ちょっと抵抗があります。	
\\	載る 
\\	(せる) 
\\	載る, 
\\	載	
\\	貯える	
\\	たくわえる	
\\	大きなビジネスを始める前に、元手を貯える必要がある。	
\\	う 
\\	(たくわ) 
\\	貯	
\\	振る	
\\	ふる	
\\	「彼に振られたなんて信じられない。」「あんた、浮気してたんでしょ。自業自得だよ。」	
\\	う 
\\	(ふる), 
\\	振	
\\	過ごす	
\\	すごす	
\\	電気料金の支払いが三ヶ月遅れて電気を止められちゃって、あのクソ暑い日をエアコン無しで過ごさなきゃならなかったんだ。	
\\	過ぎる 
\\	過ごす 
\\	過ぎる. 
\\	過	
\\	突く	
\\	つく	
\\	「それじゃあ、いくらお金があれば幸せになれるっていうの?」 「う~ん。痛いところを突かれたな。」	
\\	う 
\\	(つ) 
\\	突	
\\	刺す	
\\	さす	
\\	母親をナイフで刺したにもかかわらず、私はその夜熟睡した。	
\\	う 
\\	(さ). 
\\	刺	
\\	担う	
\\	になう	
\\	我が社の純利益は前年比で70%も上昇したが、それを達成するために私が重要な役割を担ったんだ。	
\\	担ぐ 
\\	担う 
\\	(う) 
\\	(にな).	担	
\\	独特	
\\	どくとく	
\\	な 
\\	の 
\\	この香辛料は独特の香りがする。	
\\	独, 特	
\\	従来	
\\	じゅうらい	
\\	の 
\\	従来、一年目の新入社員はこの会社では有給休暇は一日ももらえません。	
\\	の. 
\\	従, 来	
\\	自販機	
\\	じはんき	
\\	父親が誕生日に自販機をプレゼントしてくれた。	
\\	自動販売機? 
\\	自販機.	
\\	自動販売機 
\\	自, 販, 機	
\\	獣医	
\\	じゅうい	
\\	の 
\\	あの獣医さんに行く一番のメリットは、もちろん先生が優しくて賢いってこともあるけど、それだけじゃなくて治療費を分割で支払えるのよ。	
\\	獣, 医	
\\	振動	
\\	しんどう	
\\	する 
\\	車の中で、エンジンの振動を感じるのが好きなんです。	
\\	振, 動	
\\	違反	
\\	いはん	
\\	する 
\\	スピード違反で捕まった。	
\\	違 
\\	(い). 
\\	違, 反	
\\	気象庁	
\\	きしょうちょう	
\\	やっと気象庁に電話が繋がったと思ったら、切られたんだよ。ありえなくね?	
\\	気, 象, 庁	
\\	再来週	
\\	さらいしゅう	
\\	コーヒーメーカー とか、キッチン向けの小型電気器具を売ってるいい感じのお店を見つけたんだけど、再来週一緒にいってみない?	
\\	来週 
\\	再, 来, 週	
\\	何故	
\\	なぜ	
\\	何故復讐しようと思ったんですか。	
\\	何 
\\	な 
\\	なに. 
\\	(ぜ) 
\\	何, 故	
\\	野獣	
\\	やじゅう	
\\	の 
\\	あのカップル、まさに「美女と野獣」って感じだね。	
\\	野, 獣	
\\	筆	
\\	ふで	
\\	筆マークのアイコンの上にカーソルを合わせて、マウスの左のボタンをカチカチッと2回クリックしてください。	
\\	(ふで) 
\\	筆	
\\	再来年	
\\	さらいねん	
\\	再来年まで継続する二年契約を提案するべきではないでしょうか。	
\\	(来年), 
\\	再, 来, 年	
\\	給与	
\\	きゅうよ	
\\	毎月の給与は、指定の銀行口座にお振り込み致します。	
\\	給, 与	
\\	健忘症	
\\	けんぼうしょう	
\\	健忘症に苦しんでいる親戚がいます。	
\\	健, 忘, 症	
\\	南極	
\\	なんきょく	
\\	の 
\\	私の両親は、この夏に南極に行ってみることを決心しました。	
\\	南, 極	
\\	訪問	
\\	ほうもん	
\\	する 
\\	の 
\\	日本では、通常歯医者に一回の訪問で治療を済ませてもらうことはできません。	
\\	訪, 問	
\\	悩み	
\\	なやみ	
\\	悩み事の相談に乗るのは構わないんだけど、金欠なんだよね。外食したいなら、あんたの奢りだよ。	
\\	悩	
\\	悪影響	
\\	あくえいきょう	
\\	悪影響が自分の子どもにまで及んでいたことに気が付き、彼女は泣き出した。	
\\	影響 
\\	悪, 影, 響	
\\	野郎	
\\	やろう	
\\	あのクソ野郎がセールス・マネージャーに昇進するなんて本当に気に食わねぇ。	
\\	野, 郎	
\\	腹	
\\	はら	
\\	お前ちょっと腹が出てきたんじゃねぇか?	
\\	(はら)! 
\\	腹	
\\	長靴	
\\	ながぐつ	
\\	病院のベッドに横になりながら、彼は明日妻にいい長靴を買ってやろうと思った。	
\\	長い 
\\	靴 
\\	靴 
\\	長, 靴	
\\	退屈	
\\	たいくつ	
\\	する 
\\	な 
\\	果てしなく続く空の下で、生徒たちは果てしなく長くて退屈な校長の話を聞かなくてはならなかった。	
\\	退, 屈	
\\	着替え	
\\	きがえ	
\\	する 
\\	母は、私の着替えを持ってくるのを忘れた事を謝りました。	
\\	替える 
\\	着る 
\\	替える 
\\	か 
\\	が.	着, 替	
\\	手掛かり	
\\	てがかり	
\\	警察は消えたベーコンの謎について何か手掛かりを見つけましたか?	
\\	手 
\\	掛かる 
\\	手, 掛	
\\	腰抜け	
\\	こしぬけ	
\\	管理職に昇進する前に辞職する腰抜け社員が多すぎる。	
\\	腰, 抜	
\\	濃度	
\\	のうど	
\\	あなたの血中アルコール濃度については追って連絡致します。	
\\	濃, 度	
\\	〜症	
\\	しょう	
\\	な 
\\	妻の神経性ヒステリー症について、医者に相談しなくてはいけません。	
\\	症	
\\	お構いなく	
\\	おかまいなく	
\\	すぐ帰りますので、お構いなく。	
\\	構う 
\\	構う, 
\\	構	
\\	頑張れ	
\\	がんばれ	
\\	まあ、結果を気にせずに、頑張れよ!	
\\	頑張る? 
\\	頑張る, 
\\	頑張れ, 
\\	頑張る.	頑, 張	
\\	端	
\\	はし	
\\	私は大粒の涙がコウイチの目の端からこぼれ落ちるのを見てしまった。	
\\	(はし) 
\\	端	
\\	効果的	
\\	こうかてき	な 
\\	成果主義の給与制度は、良い社員を引き込む最も効果的で公平な方法である。	
\\	効果 
\\	効, 果, 的	
\\	就業	
\\	しゅうぎょう	
\\	する 
\\	就業規則についての話し合いをするために、 私たちは月一のペースで会っている。	
\\	就, 業	
\\	極端	
\\	きょくたん	
\\	な 
\\	君は奇跡を起こすことができるかもしれない。でも、僕は君には極端なことは避けてもらいたいんだよ。	
\\	極, 端	
\\	就職	
\\	しゅうしょく	
\\	する 
\\	「いくつか就職の面接をしたんだけど、まだどこからも内定が出ないんだ。」「お互い大変だな。」	
\\	就, 職	
\\	段々	
\\	だんだん	
\\	私の日本語が段々うまくなっていると言ってもらえて、すごく嬉しいです。	
\\	段, 々	
\\	中途半端	
\\	ちゅうとはんぱ	
\\	な 
\\	の 
\\	俺は中途半端な生き方をしてきたかもしれないけど、マフィアに入りたいって思ったことは一度もないよ。	
\\	ぱ 
\\	中, 途, 半, 端	
\\	頭痛	
\\	ずつう	
\\	頭痛が頻発することを母親に相談しました。	
\\	痛 
\\	頭 
\\	(ず) 
\\	頭, 痛	
\\	健康	
\\	けんこう	
\\	な 
\\	会社側が負担する健康保険は、諸手当の中に含まれているのでしょうか。	
\\	健, 康	
\\	休暇	
\\	きゅうか	
\\	いつもどのくらい前から休暇の予定を立てますか?	
\\	休, 暇	
\\	間抜け	
\\	まぬけ	
\\	な 
\\	彼に間抜けな田舎者のお上りさんだと思われたくないの。	
\\	抜く 
\\	ま 
\\	間 
\\	抜く.	間, 抜	
\\	不眠症	
\\	ふみんしょう	
\\	彼と仕事をするのがストレスで、不眠症になってしまいました。	
\\	不, 眠, 症	
\\	居酒屋	
\\	いざかや	
\\	扉が開き、居酒屋の中に老婆が一人ぼっちで入ってきた。	
\\	居, 酒, 屋	
\\	昇進	
\\	しょうしん	
\\	する 
\\	の 
\\	「聞いて聞いて、昇進したんだ!」「本当?すごいね!おめでとう!」	
\\	昇, 進	
\\	大規模	
\\	だいきぼ	
\\	な 
\\	細川は、原発再稼働に反対する大規模デモの実現に協力してくれたことについて、児島に礼を述べた。	
\\	規模 
\\	大, 規, 模	
\\	早退	
\\	そうたい	
\\	する 
\\	一緒に学校を早退しない?	
\\	早, 退	
\\	限定販売	
\\	げんていはんばい	
\\	日本では三台のみの限定販売になるそうなんだが、寝台車付き特急の実物大型模型を買おうか迷っているんだよ。	
\\	(販売) 
\\	(限定). 
\\	限, 定, 販, 売	
\\	居間	
\\	いま	
\\	父は居間でテレビを観ています。	
\\	居, 間	
\\	迷子	
\\	まいご	
\\	迷子の子どもの両親を探してあげようとしていただけなのに、誘拐未遂で起訴されてしまいました。	
\\	迷 
\\	ご 
\\	こ. 
\\	まい 
\\	(まい) 
\\	迷, 子	
\\	迷路	
\\	めいろ	
\\	する 
\\	の 
\\	誰一人としてその迷路の出口を見つけることができませんでした。	
\\	迷, 路	
\\	両替	
\\	りょうがえ	
\\	する 
\\	後で、今日習った日本語での両替の仕方について復習したいと思います。	
\\	両替 
\\	か 
\\	が 
\\	がえ. 
\\	(がえ).	両, 替	
\\	結構です	
\\	けっこうです	
\\	いいえ、結構です。夕食はいりません。申し訳ありませんが、ストレスで食欲がないんです。	
\\	"結構 
\\	結構.	結, 構	
\\	途中	
\\	とちゅう	
\\	「今すぐに来られますか?」「ええ。今向かっている途中です。」「え?ということは、場所をご存知なんですね?」「いいえ、分かりません。」	
\\	途, 中	
\\	組織	
\\	そしき	
\\	する 
\\	私がこの組織で昇格の決定を担っている桜庭と申します。	
\\	組, 織	
\\	痛い	
\\	いたい	い 
\\	この部屋じゃ煙が目に入って痛いので、会社は屋外に喫煙場所を指定するべきだと思うね。	
\\	い 
\\	(いた) 
\\	痛	
\\	極楽	
\\	ごくらく	
\\	間違いで極楽に行った男についての小説が、映画化されました。	
\\	極, 楽	
\\	減給	
\\	げんきゅう	
\\	する 
\\	の 
\\	減給されるし、上司も嫌な奴だし、会社を辞めて転職先を探すことを本気で考えてるんだよね。	
\\	減, 給	
\\	昇給	
\\	しょうきゅう	
\\	する 
\\	業界の景気回復に伴い、昇給を願い出たんだが、うまくいかなかったよ。	
\\	昇, 給	
\\	迫害	
\\	はくがい	
\\	する 
\\	全ての人々が、世界の迫害と差別の歴史を学ぶべきだと思います。	
\\	迫, 害	
\\	圧迫	
\\	あっぱく	
\\	する 
\\	圧迫に耐えないので、宇宙飛行士になる夢を諦めた。	
\\	あつ 
\\	っ, 
\\	はく 
\\	ぱく.	圧, 迫	
\\	値段	
\\	ねだん	
\\	テレビとタイアップした書籍を手がけているのは本当だが、まだ値段は決まっていない。	
\\	値 
\\	段. 
\\	値 
\\	ね 
\\	(ね) 
\\	値, 段	
\\	貸し切り	
\\	かしきり	
\\	この宿には、貸し切りの温泉があります。	
\\	貸す 
\\	切る 
\\	貸切り 
\\	貸切 
\\	貸, 切	
\\	引退	
\\	いんたい	
\\	する 
\\	そろそろ
\\	界を引退しようと考えてるんだ。	
\\	引 
\\	(いん) 
\\	引, 退	
\\	巻きずし	
\\	まきずし	
\\	「君も巻きずしを食べてみるべきだよ。きっと気に入ると思うよ。」「わかったわ。あなたがそう言うのなら。」	
\\	"巻く 
\\	ずし 
\\	すし. 
\\	巻く.	巻	
\\	切腹	
\\	せっぷく	
\\	する 
\\	その年老いた武士は、切腹をすることに固執した。	
\\	切, 腹	
\\	靴	
\\	くつ	
\\	「ああ、足が棒だよ。痛くて死にそう。」「言わんこっちゃない。だから、もっと歩きやすい靴を履けって最初に言ったじゃないか。」	
\\	靴	
\\	靴屋	
\\	くつや	
\\	靴屋での仕事に内定が出たが、丁重にお断りさせてもらったよ。	
\\	靴, 屋	
\\	睡眠	
\\	すいみん	
\\	の 
\\	彼は、給与の良い管理職へ昇格したいがために、睡眠時間を削って一生懸命働いている。	
\\	睡, 眠	
\\	眠い	
\\	ねむい	い 
\\	渋滞に巻き込まれた時、目も開けていられないくらい眠い状態でもありました。	
\\	い 
\\	(ねむ).	眠	
\\	小規模	
\\	しょうきぼ	
\\	な 
\\	私達はみんな、啓介に小規模農場なんて始めない方がいいと警告したんです。	
\\	"規模 
\\	小, 規, 模	
\\	靴下	
\\	くつした	
\\	ピンクの靴下を取ろうと一番上の棚に手を伸ばしたが、背丈が足りずに届かなかった。	
\\	靴, 下	
\\	暇	
\\	ひま	
\\	な 
\\	私達は空気洗浄機とか家電を販売してるんですが、お客さんがあまりこないので、お店では正直暇してます。	
\\	(ひま), 
\\	暇	
\\	春巻き	
\\	はるまき	
\\	「私の春巻き食べたでしょ?」「ごめん。その通りだよ。」	
\\	"巻く 
\\	春 
\\	巻く.	春, 巻	
\\	段階	
\\	だんかい	
\\	彼氏は、私が妊娠からかなり経った段階まで妊娠の事実を伝えなかったことについて私を責めました。	
\\	段, 階	
\\	階段	
\\	かいだん	
\\	充は、私が階段の掃除を手伝った事に礼を述べた。	
\\	段階, 
\\	階, 段	
\\	構え	
\\	かまえ	
\\	ムエタイではあなたの打撃の構えが最も重要な事である。	
\\	構う, 
\\	構	
\\	胃	
\\	い	
\\	ストレスで、胃がキリキリします。	
\\	胃	
\\	胃痛	
\\	いつう	
\\	禁煙してから、激しい胃痛も治まっている。	
\\	胃, 痛	
\\	迷信	
\\	めいしん	
\\	の 
\\	年寄りは迷信を信じやすい。	
\\	迷, 信	
\\	供給	
\\	きょうきゅう	
\\	する 
\\	最優先事項は供給を安定させることだ。	
\\	供, 給	
\\	症状	
\\	しょうじょう	
\\	の 
\\	私の症状について、一度同僚に相談しました。	
\\	症, 状	
\\	誘惑	
\\	ゆうわく	
\\	する 
\\	フグは、サーモンの唇を奪いたいという誘惑に抗うことはできなかった。	
\\	誘, 惑	
\\	第一段	
\\	だいいちだん	
\\	第一段階として、まずは基本の復習をしなくちゃいけないかもしれない。	
\\	第一 
\\	段 
\\	第, 一, 段	
\\	理屈	
\\	りくつ	
\\	理屈ばっかりこねて、本当に嫌味な人ね。	
\\	理, 屈	
\\	濃い	
\\	こい	い 
\\	私は彼の髪は濃い紫色にしようかなって思ってるんだよね。	
\\	い 
\\	子 
\\	濃	
\\	先端	
\\	せんたん	
\\	の 
\\	そのお箸の先端は焦げています。	
\\	先, 端	
\\	逮捕する	
\\	たいほする	する 
\\	もし今この場を去らなければ、公務執行妨害で逮捕するぞ。	
\\	逮捕 
\\	逮捕 
\\	逮, 捕	
\\	昇る	
\\	のぼる	
\\	私は天にも昇る気持ちで、大学からの合格通知を何度も読み直した。	
\\	う 
\\	登る 
\\	上る 
\\	のぼる, 
\\	昇	
\\	招く	
\\	まねく	
\\	死者の霊魂を招く歌を作ってみた。	
\\	う 
\\	(まね). 
\\	招	
\\	退院する	
\\	たいいんする	する 
\\	すぐに退院することができるといいんだけど。	
\\	退, 院	
\\	抜ける	
\\	ぬける	
\\	最初の放射線療法から十日目に、髪の毛が抜け始めました。	
\\	抜く 
\\	(ける).
\\	抜く.	抜	
\\	隠れる	
\\	かくれる	
\\	「やばい。多分あれ元彼だ。速く隠れなきゃ。」「まじで?フグってこと?」「違うよ。サメだよ。」
\\	サメ?サメって誰のこと?てか、あんた元カレ多すぎだっつーの!」	
\\	"隠す 
\\	隠れる 
\\	(れる) 
\\	隠す. 
\\	隠	
\\	掛かる	
\\	かかる	
\\	ルーヴル美術館に行った時、モナリザが曲がって掛かっていた。	
\\	"掛ける 
\\	(かる) 
\\	掛ける, 
\\	掛	
\\	惑う	
\\	まどう	
\\	惑う鹿がぐるぐる回っていたので、首を銃で撃ち抜いた。	
\\	う 
\\	(まど) 
\\	惑	
\\	極める	
\\	きわめる	
\\	日本語を極めるのは、生易しい事ではない。	
\\	う 
\\	(きわ) 
\\	極	
\\	捕える	
\\	とらえる	
\\	ご来光のしゅん間をカメラで捕えることができました。	
\\	私の猫は鼠を捕えることが得意なんです。	
\\	捕まる 
\\	捕まる. 
\\	取る 
\\	取る 
\\	捕	
\\	締結する	
\\	ていけつする	する 
\\	どうしてあの国が突然平和友好条約を締結したのか不思議に思うよ。	
\\	締, 結	
\\	取り替える	
\\	とりかえる	
\\	そろそろ我々の
\\	機器を最新のものに取り替える方がいいんじゃないでしょうか。	
\\	取る 
\\	替える 
\\	取, 替	
\\	怒る	
\\	おこる	
\\	「フグ、どうしてサーモンがお前のことを怒っているのか分からないんだろ?」「うん。さっぱり分からないよ。」「そうだと思ったよ。」	
\\	う 
\\	(おこ). 
\\	怒	
\\	迷う	
\\	まよう	
\\	無糖珈琲を買おうかどうか迷っています。	
\\	う 
\\	(まよ). 
\\	迷	
\\	手渡す	
\\	てわたす	
\\	プレゼントを手渡す勇気はありません。	
\\	渡す 
\\	手, 渡	
\\	招待する	
\\	しょうたいする	する 
\\	どのお友達を誕生日会に招待するか決めましたか。	
\\	招, 待	
\\	誘う	
\\	さそう	
\\	「ついに彼女をデートに誘ったんだけど、なんと
\\	の返事をもらえたんだよ。だから、今夜、食事をしに出かけるんだ。」「そうなんだ!楽しんできてね!くれぐれも悪いことはしないようにね。」	
\\	う 
\\	(さそ) 
\\	誘	
\\	貸す	
\\	かす	
\\	「君の車を貸してくれない?」「絶対にだめだよ。まだ免許持ってないじゃないか。」	
\\	う 
\\	貸	
\\	隠居する	
\\	いんきょする	する 
\\	隠居してから、祖父は毎朝自宅で瞑想をしています。	
\\	隠居 
\\	隠居 
\\	隠, 居	
\\	見抜く	
\\	みぬく	
\\	うちの母親は父親の嘘を見抜くのがうまい。	
\\	見る 
\\	抜く 
\\	見, 抜	
\\	抜き出す	
\\	ぬきだす	
\\	彼は背広から銃をそっと抜き出した。	
\\	抜く 
\\	出す 
\\	抜く 
\\	出す, 
\\	抜, 出	
\\	就く	
\\	つく	
\\	どうして日当で給与を支払う会社に就きたいの?	
\\	う 
\\	(つ) 
\\	就	
\\	怒鳴る	
\\	どなる	
\\	「キンニクマってよく怒鳴るよね。」「そうだね。でも、心配しないで。彼は口は悪いけど根は悪い人ではないから。」	
\\	怒 
\\	鳴る. 
\\	怒, 鳴	
\\	屈む	
\\	かがむ	
\\	屈むと腰が痛いんです。	
\\	う 
\\	(かが), 
\\	屈	
\\	締める	
\\	しめる	
\\	ガスの元栓、ちゃんと締めた?	
\\	う 
\\	(し), 
\\	締	
\\	迫る	
\\	せまる	
\\	彼に金を返すよう迫ったが、逃げられてしまった。	
\\	う 
\\	(せま)!	迫	
\\	訪ねる	
\\	たずねる	
\\	家を訪ねる前に、電話してみた方がいいよ。	
\\	う 
\\	(たず) 
\\	訪	
\\	織る	
\\	おる	
\\	私の母親は、最近手織り機でカーペットを織ることにはまっているんです。	
\\	う 
\\	(お) 
\\	織	
\\	退く	
\\	しりぞく	
\\	ソフトボールチームのヘッドコーチのポジションを退くには、まだ早過ぎますよ。	
\\	う 
\\	尻
\\	(しりぞ). 
\\	尻
\\	退	
\\	悩む	
\\	なやむ	
\\	「どうしたの?何を悩んでいるの?」「昨日、彼女に振られたんだ。」	
\\	う 
\\	悩	
\\	迷惑	
\\	めいわく	
\\	な 
\\	日本語が下手なので、あなたに迷惑をかけてしまうかもしれません。	
\\	迷, 惑	
\\	究極	
\\	きゅうきょく	
\\	の 
\\	あの榊氏までもが、私が宇宙の究極の原理を解明したことに祝いの言葉を述べてくれました。	
\\	究, 極	
\\	切迫	
\\	せっぱく	
\\	する 
\\	の 
\\	ヤクザからお金を借りなければいけないという切迫した危機に直面している。	
\\	せつ 
\\	切 
\\	っ. 
\\	はく 
\\	ぱく.	切, 迫	
\\	給料	
\\	きゅうりょう	
\\	「君の上司が、君の給料を上げるつもりだって聞いたよ。」「まさか!そんなことあり得ないよ。」	
\\	給, 料	
\\	絶対	
\\	ぜったい	
\\	学校のある夜は、外出できないって言ってるでしょ。だめだと言ったら絶対にだめだからね!	
\\	絶, 対	
\\	進撃	
\\	しんげき	
\\	する 
\\	進撃の巨人にはまっています。	
\\	進撃 
\\	進撃の巨人 
\\	進, 撃	
\\	前売り券	
\\	まえうりけん	
\\	前売り券は他の割引とは併用できませんって書いてあるよ。	
\\	前 売る 
\\	券 
\\	前, 売, 券	
\\	巨人	
\\	きょじん	
\\	「何してるの?」「俺達、進撃の巨人を見てるんだよ。って、あぁ、しまった。最後の字幕見逃しちゃった。ちょっと巻き戻してくれない?」「了解。ストップって言ってね。」「ストップ!」	
\\	巨, 人	
\\	軍隊	
\\	ぐんたい	
\\	この鰐蟹軍隊の成功のためには、あなたが必要なんです。どうか入隊してください。	
\\	軍, 隊	
\\	矢印	
\\	やじるし	
\\	あの矢印の道路標識は何という意味ですか。	
\\	矢 
\\	印 
\\	印 
\\	印, 
\\	矢, 印	
\\	金庫	
\\	きんこ	
\\	彼が私の金庫を盗んでいた時、私はその物音が聞こえないふりをしていました。	
\\	金, 庫	
\\	星占い	
\\	ほしうらない	
\\	今日の星占いで、頭にたんこぶができないように気をつけた方がいいと出ていたの。	
\\	"占い 
\\	星 
\\	占い 
\\	星, 占	
\\	攻撃	
\\	こうげき	
\\	する 
\\	今まで見た中で一番セクシーなミサイル攻撃だったよ。	
\\	攻, 撃	
\\	浜辺	
\\	はまべ	
\\	一緒に浜辺を散歩しないか。	
\\	辺 
\\	べ! 
\\	べ!	浜, 辺	
\\	浜	
\\	はま	
\\	エンジンの調子がおかしくなったので、浜に着くほんの手前で車を止めなきゃいけなくなっちゃったんだよ。	
\\	海.	
\\	浜	
\\	何故なら	
\\	なぜなら	
\\	何故なら、彼は魔法使いだったのです。	
\\	何故, 
\\	""何故なら 
\\	何故, 
\\	何, 故	
\\	有益	
\\	ゆうえき	
\\	な 
\\	非常に有益な情報を提供して頂き、誠に有り難うございます。	
\\	有, 益	
\\	大間違い	
\\	おおまちがい	
\\	彼女をデートに誘ったが、それは大間違いだった。	
\\	間違い 
\\	大 
\\	間違い 
\\	大, 間, 違	
\\	回数券	
\\	かいすうけん	
\\	電車の回数券は金券ショップでも買えますよ。	
\\	回, 数, 券	
\\	菓子屋	
\\	かしや	
\\	お母さん、心配しなくていいよ。あのお菓子屋さんなら僕、一人でも行けるよ。	
\\	"お菓子 
\\	お菓子 
\\	菓, 子, 屋	
\\	幼年時代	
\\	ようねんじだい	
\\	「昨日、幼年時代の友人に偶然会ったの。彼には何年も会っていなかったんだけど。」「そう言えば、サーモンって、幼年時代の友人のフグと結婚したのよね。」	
\\	時代 
\\	幼, 年, 時, 代	
\\	幼い	
\\	おさない	い 
\\	私が幼い頃は、何事もうまくいっていたんだ。	
\\	(おさな) 
\\	幼	
\\	幼稚	
\\	ようち	
\\	な 
\\	の 
\\	どうして幼稚な喋り方をする日本人女性が多いのですか?	
\\	幼, 稚	
\\	幼児	
\\	ようじ	
\\	あの幼児のテディベアの毛はベルベットのように滑らかだった。	
\\	幼, 児	
\\	児童	
\\	じどう	
\\	児童の九割が欠席しました。	
\\	児, 童	
\\	清潔	
\\	せいけつ	
\\	な 
\\	実際には清潔だけど、見た目は汚らしくてちょっと影がある男の人がタイプです。	
\\	清, 潔	
\\	冷たい	
\\	つめたい	い 
\\	冷たいコーヒーのことを、日本語では「アイスコーヒー」と呼びます。	
\\	寒い).	
\\	冷	
\\	比較的	
\\	ひかくてき	
\\	彼は新しいプログラミング言語を比較的速く修得します。	
\\	比較 
\\	的 
\\	比較.	比, 較, 的	
\\	憲法	
\\	けんぽう	
\\	の 
\\	憲法改正案の見直しで依然として忙しい。	
\\	憲, 法	
\\	憲政	
\\	けんせい	
\\	憲政の危機って何ですか。	
\\	憲, 政	
\\	冷静	
\\	れいせい	
\\	な 
\\	冷静にネット上の荒らしに対処する方法を学ぶ必要があります。	
\\	冷, 静	
\\	反攻	
\\	はんこう	
\\	する 
\\	いつ誰に対して織田軍が反攻を開始したのか覚えてる?	
\\	反, 攻	
\\	従兄弟	
\\	いとこ	
\\	従兄弟同士は結婚できるんだぜ。	
\\	兄弟 
\\	いいと子, 
\\	いとこ 
\\	いとこ 
\\	従, 兄, 弟	
\\	創造的	
\\	そうぞうてき	な 
\\	これで、あなたには創造的な仕事にもっと集中してもらえます。	
\\	創造 
\\	的 
\\	創, 造, 的	
\\	創造力	
\\	そうぞうりょく	
\\	あなたのような創造力のある人と仕事ができることを、とても楽しみにしています。	
\\	創造 
\\	創, 造, 力	
\\	処理	
\\	しょり	
\\	する 
\\	彼は、上司が顧客のクレーム処理をしないことについてぶつぶつ文句を言っている。	
\\	処, 理	
\\	程度	
\\	ていど	
\\	彼は実際は35歳だが、3歳児と同程度の知能水準しかない。	
\\	程, 度	
\\	微か	
\\	かすか	な 
\\	彼女の微かな香水の香りに、妙に興奮してしまった。	
\\	(かす) 
\\	微	
\\	絶望	
\\	ぜつぼう	
\\	する 
\\	の 
\\	相手チームに15点差で負けているので、かなり絶望的な状況です。	
\\	絶, 望	
\\	直接的	
\\	ちょくせつてき	
\\	な 
\\	幸運なことに、私の家族はあの日本の大地震や大津波による直接的な被害は受けませんでした。	
\\	直接 
\\	的 
\\	直接 
\\	直, 接, 的	
\\	処分	
\\	しょぶん	
\\	する 
\\	私の息子は、ある生徒に苛めを行い、退学処分を受けました。	
\\	処, 分	
\\	入隊	
\\	にゅうたい	
\\	する 
\\	日本人男性がアメリカ軍に入隊することはできますか。	
\\	入, 隊	
\\	程	
\\	ほど	
\\	まだ夜明けまでは一時間程あったが、驚く程青い目をした男は既に起床し、珈琲を作り始めていた。	
\\	(ほど) 
\\	程	
\\	不潔	
\\	ふけつ	
\\	な 
\\	「わっ!君にビールをこぼしちゃったよね!本当にごめんなさい。」「大丈夫よ!このスカート、妹のなんだけど、妹はそもそも不潔だから多分気にしないと思うわ。」	
\\	不, 潔	
\\	凍結	
\\	とうけつ	
\\	する 
\\	彼女は道路の凍結についてよく文句を言っています。	
\\	凍, 結	
\\	冷凍庫	
\\	れいとうこ	
\\	良い冷凍庫が見つかるといいですね。	
\\	冷蔵庫, 
\\	冷, 凍, 庫	
\\	凍死	
\\	とうし	
\\	する 
\\	警察が昨夜私の見込み客が雪山で凍死しているのを発見したので、私に事情聴取をしたいそうです。	
\\	凍, 死	
\\	振り仮名	
\\	ふりがな	
\\	漢字に振り仮名を振る仕事を見つけました。	
\\	"振る 
\\	仮名 
\\	振る 
\\	仮名 
\\	振, 仮, 名	
\\	博打	
\\	ばくち	
\\	博打を打つことは日本では違法です。	
\\	はく 
\\	ばく, 
\\	ち 
\\	(ばくち), 
\\	博, 打	
\\	車庫	
\\	しゃこ	
\\	両親の車を車庫に入れる時に、柱にぶつけてしまいました。	
\\	車, 庫	
\\	大衆	
\\	たいしゅう	
\\	の 
\\	大衆の感じ方次第です。	
\\	大, 衆	
\\	身振り	
\\	みぶり	
\\	日本人の人たちと、身振り手振りで何とかコミュニケーションをとることができた。	
\\	身 
\\	振る 
\\	振り 
\\	ぶり.	身, 振	
\\	家政婦	
\\	かせいふ	
\\	家政婦を雇おうかと思っているんです。	
\\	家, 政, 婦	
\\	衆議院	
\\	しゅうぎいん	
\\	ある衆議院議員が公費でボートを借りたことが明らかになりました。	
\\	衆, 議, 院	
\\	主婦	
\\	しゅふ	
\\	あの主婦の肌はものすごくスベスベしている。	
\\	主, 婦	
\\	巨大	
\\	きょだい	
\\	な 
\\	彼はとっても忙しい巨大企業の社長さんなんだから、アポの約束は絶対に早目に取っておいた方がいいよ。	
\\	巨, 大	
\\	夫婦	
\\	ふうふ	
\\	ねぇ、あなた。これから先、私たち夫婦の間で色々なことがうまくいくといいわね。	
\\	夫, 婦	
\\	移民	
\\	いみん	
\\	する 
\\	の 
\\	私は、自分が違法移民であるという秘密を彼女に告白した。	
\\	移, 民	
\\	並列	
\\	へいれつ	
\\	する 
\\	の 
\\	もし自分の彼氏が後ろ向きで入る並列駐車が上手くできなかったら、がっかりしちゃうと思う。	
\\	並, 列	
\\	婦人	
\\	ふじん	
\\	あの婦人、俺の手には負えなくなってきたよ。どうしたらいいかな?	
\\	婦, 人	
\\	郵便箱	
\\	ゆうびんばこ	
\\	どうやって私の小さな郵便箱にこの本が入ったんだろう。	
\\	郵便 
\\	郵, 便, 箱	
\\	専攻	
\\	せんこう	
\\	する 
\\	の 
\\	私はファイン・アートを専攻しています。	
\\	専, 攻	
\\	修士	
\\	しゅうし	
\\	の 
\\	アルバイトをしながら、修士号を取るために勉強しています。	
\\	修, 士	
\\	微生物	
\\	びせいぶつ	
\\	微生物学の専門家は何時に来る予定ですか。	
\\	(生物 
\\	微, 生, 物	
\\	移住	
\\	いじゅう	
\\	する 
\\	の 
\\	それはあなたがこの国に移住して来るか来ないかによります。	
\\	移, 住	
\\	移動	
\\	いどう	
\\	する 
\\	バスで長距離の移動は疲れる。	
\\	移, 動	
\\	博物館	
\\	はくぶつかん	
\\	「サーモン、昨日の博物館デート、どうだった?」「ひどかったわ!」「どうして?何があったの?」「フグったら、ずっと他の女の子をじろじろ見てたのよ!すごくむかついたわ。」	
\\	博, 物, 館	
\\	国益	
\\	こくえき	
\\	日本人の英会話力不足によって、将来国益が損なわれる可能性がある。	
\\	国, 益	
\\	微妙	
\\	びみょう	
\\	な 
\\	うーん…なんとも微妙な感じですね。	
\\	微, 妙	
\\	妙	
\\	みょう	
\\	な 
\\	妙だな。今日はまだお袋から電話がかかってこないよ。	
\\	妙	
\\	逆効果	
\\	ぎゃくこうか, ぎゃっこうか	
\\	彼女に焼き餅を妬かせようとして他の女にキスをしたら、全く逆効果になってしまった。	
\\	"効果 
\\	効果 
\\	逆, 効, 果	
\\	並	
\\	なみ	
\\	運動神経は並だったが、頭はすこぶる良かった。	
\\	(なみ) 
\\	並	
\\	逆説	
\\	ぎゃくせつ	
\\	の 
\\	逆説的に聞こえるかもしれないが、長い結婚生活を送ってきた年配の人達は、ちょっと違ったことに取り組むことは夫婦関係の刺激になると認めている。	
\\	逆, 説	
\\	並行	
\\	へいこう	
\\	する 
\\	の 
\\	複数の言語を並行して勉強することは、ひとつだけを学習するよりも効率的だと聞いたことがあります。	
\\	並, 行	
\\	旅券	
\\	りょけん	
\\	ホテルの金庫に貴重品と旅券を預けたいのですが。	
\\	旅, 券	
\\	郵便番号	
\\	ゆうびんばんごう	
\\	まだ新しい郵便番号を覚えていません。	
\\	郵便 
\\	番号 
\\	郵, 便, 番, 号	
\\	郵便局	
\\	ゆうびんきょく	
\\	郵便局のサービスに対して苦情を言いました。	
\\	郵便 
\\	局 
\\	郵便 
\\	郵, 便, 局	
\\	精度	
\\	せいど	
\\	プリンタは発明されたものの、我々の高精度の研削盤に技術が追いつくにはまだ時間がかかるだろう。	
\\	精, 度	
\\	過程	
\\	かてい	
\\	普通の主婦が無差別に脅迫状を送りつける気違いに成り下がった過程を説明してもらえますか。	
\\	過, 程	
\\	利益	
\\	りえき	
\\	する 
\\	この新しいタイプの下着が、莫大な年間利益を上げることを予想しています。	
\\	利, 益	
\\	公益	
\\	こうえき	
\\	の 
\\	私は公益法人で働いています。	
\\	公, 益	
\\	清い	
\\	きよい	い 
\\	彼女は清い心を持っており、人々をとても思いやります。	
\\	い 
\\	(きよ).
\\	清	
\\	潔い	
\\	いさぎよい	い 
\\	俺にもお前みたいな潔い心があればいいんだがな。	
\\	い 
\\	(いさぎ)... 
\\	良い 
\\	潔	
\\	記録	
\\	きろく	
\\	する 
\\	彼女は、バイク事故で投げ飛ばされる飛距離の世界記録を樹立した。	
\\	記, 録	
\\	登録	
\\	とうろく	
\\	する 
\\	三宅結花です。携帯電話を変えたので、新しい電話番号とメールアドレスの登録をお願いします。	
\\	登, 録	
\\	修理	
\\	しゅうり	
\\	する 
\\	の 
\\	「おい、修理屋。これはすぐに直せるか?」「って、君たちの関係ってこと?悪いけど、それはできないよ。」	
\\	修, 理	
\\	逆	
\\	ぎゃく	
\\	な 
\\	私はホットドッグのことが好きで、逆もまた同じ。こないだ夢であるホットドッグに
\\	好きだ
\\	ってちゃっかり言われちゃったんだから。	
\\	逆	
\\	修辞学	
\\	しゅうじがく	
\\	私は修辞学の先生に恋をしています。	
\\	修, 辞, 学	
\\	刺身	
\\	さしみ	
\\	会合の後、素晴らしい刺身弁当があるレストランで昼食を予定しています。	
\\	刺す 
\\	さし. 
\\	身 
\\	身
\\	刺, 身	
\\	貯金箱	
\\	ちょきんばこ	
\\	兄は、十年前に私の貯金箱を盗んだ事を白状しました。	
\\	貯金 
\\	貯, 金, 箱	
\\	日程	
\\	にってい	
\\	演奏会の日程は、次の日曜日に変更されました。	
\\	日, 程	
\\	兵隊	
\\	へいたい	
\\	兵隊ごっこをしているの?面白そう。私も入れてよ。	
\\	兵, 隊	
\\	録音	
\\	ろくおん	
\\	する 
\\	誰かがいびき録音用枕の特許を取得したというのは本当ですよ。	
\\	録, 音	
\\	精神	
\\	せいしん	
\\	先生は「精神を養わなくてはならない」と言った。	
\\	精, 神	
\\	〜隊	
\\	たい	
\\	海上自衛隊音楽隊に採用された時、彼女は他の自衛隊員と同様に肉体的にも厳しいトレーニングを受けなくてはいけないことを知らなかった。	
\\	隊	
\\	絶景	
\\	ぜっけい	
\\	世界の息を飲むような絶景をたくさん紹介している有名な
\\	ページがあります。	
\\	絶, 景	
\\	独り言	
\\	ひとりごと	
\\	君が女性に対して口がうまいのは知ってたけど、その練習のためにそんな風にブツブツ独り言を言ってるなんてこれっぽっちも知らなかったよ。	
\\	独り 
\\	言 
\\	独り 
\\	独り. 
\\	言 
\\	ごと, 
\\	こと. 
\\	独, 言	
\\	故に	
\\	ゆえに	
\\	度重なる無断遅刻故に、 休暇日数が減らされてしまった。	
\\	故 
\\	故 
\\	故	
\\	〜券	
\\	けん	
\\	三ヶ月の定期券をもらえますか?	
\\	券	
\\	攻める	
\\	せめる	
\\	そのピッチャーはバッターを変化球で攻めることが多い。	
\\	攻める!
\\	(せ). 
\\	攻	
\\	並ぶ	
\\	ならぶ	
\\	エコバッグを買うために十二時間も行列に並ぶなんて信じられないよ。	
\\	う 
\\	(ぶ), 
\\	(なら), 
\\	並	
\\	比較する	
\\	ひかくする	する 
\\	同じ年齢の女の子と比較すると、私の娘は体力があまりありません。	
\\	比較 
\\	比較 
\\	比, 較	
\\	凍る	
\\	こおる	
\\	オレンジジュースを凍らせて氷を作ったことはありますか。	
\\	う 
\\	氷, 
\\	こおり, 
\\	り 
\\	る 
\\	こおる, 
\\	氷
\\	凍	
\\	逆らう	
\\	さからう	
\\	「うわ!お前目のとこに青たんできてるじゃん。」「そうなんだよ。昨日の夜お袋に逆らったら、こてんぱに殴られてさ。」	
\\	う 
\\	(さか). 
\\	逆	
\\	輸入する	
\\	ゆにゅうする	する 
\\	私はヨーロッパからワインとチョコレートを輸入する仕事をしています。	
\\	輸入 
\\	輸入 
\\	輸, 入	
\\	面接する	
\\	めんせつする	する 
\\	今日私が面接した男はすごく馬鹿な奴で、オリンピックは三年毎に開催されるって言ってたよ。	
\\	面接 
\\	面接 
\\	面, 接	
\\	間違える	
\\	まちがえる	
\\	こんな簡単な問題を間違えるなんて、悔しいよ。	
\\	間違い 
\\	間違い! 
\\	間, 違	
\\	回復する	
\\	かいふくする	する 
\\	彼が意識を回復することは難しいだろうと医者に言われました。	
\\	回復 
\\	回復, 
\\	回, 復	
\\	絶つ	
\\	たつ	
\\	酒を絶つことが約束できますか。	
\\	(た) 
\\	絶	
\\	移す	
\\	うつす	
\\	私は苺をミキサーに移しました。	
\\	う 
\\	(うつうつうつうつうつ) 
\\	うつうつうつうつうつ
\\	移	
\\	撃つ	
\\	うつ	
\\	動くな。じっとしていないと、撃つぞ!	
\\	う 
\\	(う). 
\\	撃	
\\	無我夢中	
\\	むがむちゅう	
\\	の 
\\	彼は無我夢中で働き、二十代で家を建てたが、三十代で墓を建てることとなった。	
\\	夢中 
\\	無, 我, 夢, 中	
\\	研修	
\\	けんしゅう	
\\	する 
\\	その研修は午後3時に始まる予定です。	
\\	研, 修	
\\	処置	
\\	しょち	
\\	する 
\\	誰か彼に応急処置ができる人はいませんか。	
\\	処, 置	
\\	傘	
\\	かさ	
\\	私は傘をクロークに預けました。	
\\	傘	
\\	傘立て	
\\	かさたて	
\\	傘立てに濡れた傘を立てないで下さい。	
\\	傘 
\\	立つ 
\\	傘, 立	
\\	庫	
\\	くら	
\\	お宅の庫は何階建てですか?	
\\	(くら) 
\\	庫	
\\	我々	
\\	われわれ	
\\	の 
\\	我々はその問題を熟考する必要がある。	
\\	我 
\\	我, 
\\	我, 々	
\\	針金	
\\	はりがね	
\\	の 
\\	もし値ごろ感が重視されるのなら、我々の針金が圧勝することになるだろう。	
\\	針 
\\	金 
\\	お金, 
\\	針, 金	
\\	妙薬	
\\	みょうやく	
\\	これは二日酔いに良く効く妙薬です。	
\\	妙, 薬	
\\	横浜	
\\	よこはま	
\\	「横浜から大阪までずうっと乗せてくれて、本当に有難うございます。」「どういたしまして!」	
\\	横 
\\	浜 
\\	よこはま.	横, 浜	
\\	省略	
\\	しょうりゃく	
\\	する 
\\	時間の都合上、今は詳細については省略せざるを得ませんが、また後ほどご説明させて頂きます。	
\\	省, 略	
\\	略語	
\\	りゃくご	
\\	あなたの略語辞典を貸してもらえませんか?	
\\	略, 語	
\\	板	
\\	いた	
\\	の野球板に書き込みをしたんですが、板違いだと言われてしまいました。	
\\	痛い (いた).	板	
\\	怒気	
\\	どき	
\\	彼の鼻の穴は怒気を含んで膨らんでいる。	
\\	怒, 気	
\\	怒り	
\\	いかり	
\\	私の怒りは三分以上続かない。	
\\	怒る. 
\\	(いかり) 
\\	怒	
\\	痛み	
\\	いたみ	
\\	俺の伯母さん、クソしてる時に胸にするどい痛みが走って、119に電話したんだ。	
\\	"痛い 
\\	痛い. 
\\	痛	
\\	面積	
\\	めんせき	
\\	彼のおでこの面積は100平方メートルある。	
\\	面, 積	
\\	添付	
\\	てんぷ	
\\	する 
\\	の 
\\	人事部長様、 恐れ入りますが、この
\\	メールに添付された履歴書にお目通し頂けますでしょうか。お忙しいとは存じますが、よろしくお願い申し上げます。	
\\	添 
\\	付 
\\	ふ, 
\\	ぷ 
\\	(ぷ). 
\\	添, 付	
\\	航空	
\\	こうくう	
\\	の 
\\	あの航空会社は四月に新路線を増やす予定だ。	
\\	航, 空	
\\	宴会	
\\	えんかい	
\\	の 
\\	宴会で嫌な奴とお酒を飲むのは、会議で議事録を記録するより退屈な仕事だ。	
\\	宴, 会	
\\	宴	
\\	うたげ	
\\	俺はその宴で、ドイツ語訛りの英語を話すとても美しい女性に出会ったんだ。	
\\	歌 
\\	(うたげ), 
\\	宴	
\\	お腹	
\\	おなか	
\\	「お夕飯の支度ができましたよ!いらっしゃい。」「ああ、よかった!お腹ペコペコだよ。」	
\\	中 (なか), 
\\	腹	
\\	黒板	
\\	こくばん	
\\	黒板が使えないので、ホワイトボードを持参しています。	
\\	はん 
\\	ばん.	黒, 板	
\\	官僚	
\\	かんりょう	
\\	あの官僚が我々の素晴らしい提案を断ったなんて信じられないよ。	
\\	官, 僚	
\\	閣議	
\\	かくぎ	
\\	さっきの閣議では何が議題だったんだっけ?忘れちゃった。	
\\	閣, 議	
\\	閣僚	
\\	かくりょう	
\\	彼は閣僚のふりをしたが、嘘丸出しだった。	
\\	閣, 僚	
\\	中欧	
\\	ちゅうおう	
\\	中欧を旅してみたいなと随分長い間思っていたんですよ。	
\\	中, 欧	
\\	織物	
\\	おりもの	
\\	政治学が専門なのですが、何故か織物会社に就職しました。	
\\	織 
\\	織る. 
\\	物 
\\	もの 
\\	ぶつ. 
\\	織, 物	
\\	倒壊	
\\	とうかい	
\\	する 
\\	私の祖父母の家はとても古いので、雪の重みで倒壊しかねません。	
\\	倒, 壊	
\\	同僚	
\\	どうりょう	
\\	残念な事に、私の同僚はあの電車の事故によって大きな商談のスケジュールの組み直しを余儀なくされた。	
\\	同, 僚	
\\	全壊	
\\	ぜんかい	
\\	する 
\\	マイホームを持っていたんですが、台風で全壊しちゃったんですよ。	
\\	全, 壊	
\\	寄与	
\\	きよ	
\\	する 
\\	が世界平和に寄与していることを知っているかい?	
\\	寄, 与	
\\	寄付	
\\	きふ	
\\	する 
\\	毎年少しだけ、母校へお金を寄付しています。	
\\	寄, 付	
\\	好奇心	
\\	こうきしん	
\\	子どもの頃、私は好奇心旺盛でした。	
\\	好, 奇, 心	
\\	娘	
\\	むすめ	
\\	「私の娘は人魚なんだ。」「ええっ、それってつまり、あんた魚とやっちまったってことか?」	
\\	娘	
\\	締切	
\\	しめきり	
\\	うまく締切を明日まで延ばせたの?	
\\	締める, 
\\	締める 
\\	切る 
\\	締め切り, 
\\	締切 
\\	締, 切	
\\	要請	
\\	ようせい	
\\	する 
\\	労働組合は賃上げを要請した。	
\\	要, 請	
\\	請求	
\\	せいきゅう	
\\	する 
\\	請求書の金額が間違っているように思うので私はその請求を取り消す。	
\\	請, 求	
\\	診断	
\\	しんだん	
\\	する 
\\	の 
\\	私の息子は四歳の時に喘息だと診断されました。	
\\	診, 断	
\\	乾季	
\\	かんき	
\\	の 
\\	いつもはこの時期は既に乾季になっているが、依然として雨が降り続けている。	
\\	乾, 季	
\\	江戸	
\\	えど	
\\	だらだらしてないで、さっさと二階に行って宿題をしなさい。日本の江戸時代についてレポートを書かないといけないって言ってたでしょ!	
\\	江, 戸	
\\	欧米	
\\	おうべい	
\\	の 
\\	欧米でピンクの薔薇を栽培している所がないか探しているんですが。	
\\	欧, 米	
\\	日欧	
\\	にちおう	
\\	日欧貿易に関する会議は、司会者の不在によって横道に逸れてしまった。	
\\	日 
\\	日 
\\	日本). 
\\	日, 欧	
\\	極めて	
\\	きわめて	
\\	君がこの問題の解決方法について具体的には何も考えていないというのは、極めて明白な事実だ。	
\\	"極める 
\\	極めて 
\\	極める.	極	
\\	猛烈	
\\	もうれつ	な 
\\	このプロジェクトのリーダーを買って出たことを、今猛烈に後悔しています。	
\\	猛, 烈	
\\	猛〜	
\\	もう	
\\	バイクが猛スピードで突っ込んできた。	
\\	猛	
\\	腹切り	
\\	はらきり	
\\	侍の腹切りをテーマにした映画は来週公開されます。	
\\	腹 
\\	切る. 
\\	腹, 切	
\\	英雄	
\\	えいゆう	
\\	この言葉は、英雄たちについて書かれたとある本からの引用です。	
\\	英, 雄	
\\	迷い	
\\	まよい	
\\	なんとか迷いを断ち切りました。	
\\	迷う 
\\	迷う! 
\\	う 
\\	い.	迷	
\\	街道	
\\	かいどう	
\\	東海道、中山道、日光街道、甲州街道、奥州街道は、五街道と呼ばれ、江戸時代の五大陸上交通路であった。	
\\	かい 
\\	がい) 
\\	街, 道	
\\	腹が減った	
\\	はらがへった	
\\	腹が減ったな。この件については、昼飯を食べながら話さないか?	
\\	腹 
\\	減る 
\\	はらがへった. 
\\	腹減った, 
\\	が. 
\\	腹, 減	
\\	熱烈	
\\	ねつれつ	
\\	な 
\\	彼女はどうやらお前の熱烈なファンみたいだな。	
\\	熱, 烈	
\\	索引	
\\	さくいん	
\\	の 
\\	英語索引も付けた方がいいんじゃないかな。	
\\	(いん). 
\\	索, 引	
\\	入り江	
\\	いりえ	
\\	バルト海の北の入江で知らない人からプロポーズされたんですが、とてもロマンチックだったのでつい承諾してしまったんです。	
\\	入り 
\\	江 
\\	入, 江	
\\	日韓	
\\	にっかん	
\\	この風習は日韓両国でよく見られます。	
\\	日 
\\	日本, 
\\	日 
\\	日, 韓	
\\	乾杯	
\\	かんぱい	
\\	する 
\\	さあ、乾杯、乾杯!それ、ぐいと一杯、飲み干したまえ。	
\\	乾, 杯	
\\	検索	
\\	けんさく	
\\	する 
\\	後でネットで検索してみるよ。	
\\	検, 索	
\\	診察	
\\	しんさつ	
\\	する 
\\	先生は診察の予定がみっちり詰まっています。	
\\	診, 察	
\\	大臣	
\\	だいじん	
\\	彼女は大臣たちに会議の前に資料を配るのを忘れてしまったために、首になっちゃったんだよ。	
\\	大, 臣	
\\	居眠り	
\\	いねむり	
\\	する 
\\	運転中に居眠りしてたみたいなんだ。	
\\	居, 眠	
\\	航法	
\\	こうほう	
\\	どこで空中航法を学んだんですか。	
\\	航, 法	
\\	直航	
\\	ちょっこう	
\\	する 
\\	この船は神戸港に直航します。	
\\	直行 
\\	直航 
\\	直, 航	
\\	催告	
\\	さいこく	
\\	する 
\\	サラ金から、借金の返済を求める催告状が何度も何度も届いています。	
\\	催, 告	
\\	監視	
\\	かんし	
\\	する 
\\	監視カメラは持ってるんですが、まだテープを確認した事はないんですよね。ちゃんと撮れてるのかな。	
\\	監, 視	
\\	奇数	
\\	きすう	
\\	の 
\\	奇数とは2で割り切れない数字のことです。	
\\	奇, 数	
\\	一杯	
\\	いっぱい	
\\	な 
\\	の 
\\	「ちょっと一杯飲みに行かない?」「いいね。」	
\\	一, 杯	
\\	一人娘	
\\	ひとりむすめ	
\\	「何でのっけから彼女に一人娘が売春しているなんて言ったんだ?」「ごめん。軽率だったよ。」	
\\	(一人) 
\\	一人 
\\	娘 
\\	一, 人, 娘	
\\	痛み止め	
\\	いたみどめ	
\\	彼は痛み止めを飲んで、また畑を耕し始めた。	
\\	"痛み 
\\	止める 
\\	痛み 
\\	止める 
\\	ど.	痛, 止	
\\	請願	
\\	せいがん	
\\	する 
\\	の 
\\	もしご迷惑でなければ請願書に署名して頂けませんでしょうか。	
\\	請, 願	
\\	恐怖症	
\\	きょうふしょう	
\\	私も父も高所恐怖症です。	
\\	恐, 怖, 症	
\\	綺麗	
\\	きれい	な 
\\	「君みたいに綺麗な女性に会ったのは初めてだよ、サーモン。」「フグったら、嘘ばっかり。」	
\\	綺麗. 
\\	綺麗. 
\\	綺, 麗	
\\	宗教	
\\	しゅうきょう	
\\	彼女が、世界宗教会議の記録を取る書記を務めます。	
\\	宗, 教	
\\	貸し	
\\	かし	
\\	今は貸し倉庫で寝泊まりしているんです。	
\\	貸す. 
\\	貸す.	貸	
\\	略図	
\\	りゃくず	
\\	の 
\\	あなたの家からうちまでの略図を書いて、
\\	メールでお送りします。	
\\	略, 図	
\\	怖い	
\\	こわい	い 
\\	火星でマグニチュード7の地震が起きたんだってさ。怖いね!	
\\	い 
\\	(こわ). 
\\	怖	
\\	恐ろしい	
\\	おそろしい	い 
\\	恐ろしいと思うかもしれないが、自然の法則には逆らえないんだよ。仕方がないさ。	
\\	い 
\\	(おそ) 
\\	恐	
\\	捜索	
\\	そうさく	
\\	する 
\\	警察は捜索活動を続けています。	
\\	捜, 索	
\\	宗派	
\\	しゅうは	
\\	の 
\\	この宗派には、会議で議長を務めるための規則があります。	
\\	宗, 派	
\\	商店街	
\\	しょうてんがい	
\\	私の弟は正真正銘の馬鹿だよ。昨日、鼻で牛乳を飲んでみたかったみたいで、商店街に牛乳を買いに行ったんだけど、そこで道に迷って結局牛乳買わずじまいだったんだから。	
\\	商, 店, 街	
\\	招き猫	
\\	まねきねこ	
\\	ここを出る時、電動招き猫のコンセントを挿しっぱなしにしないでね。	
\\	"招く 
\\	招, 猫	
\\	眠り薬	
\\	ねむりぐすり	
\\	おい、お前、今の見てたか?俺、今ドリブルで三人のバックを抜いたんだぜ!俺ってば、ほんと、最高!相手チームにあげた眠り薬入りのオレンジジュースがうまく効いてたみたいだな。	
\\	眠い 
\\	薬 
\\	眠, 薬	
\\	睡眠薬	
\\	すいみんやく	
\\	さっき飲んだ睡眠薬のせいで、手足が思うように動かないんだけど。	
\\	睡, 眠, 薬	
\\	韓国	
\\	かんこく	
\\	この韓国ドラマ、再放送されるのかな。	
\\	韓, 国	
\\	迷惑メール	
\\	めいわくめーる, めいわくメール	
\\	迷惑メールが次から次へと送られてくる。	
\\	(迷惑) 
\\	迷惑, 
\\	迷, 惑	
\\	主催	
\\	しゅさい	
\\	する 
\\	競馬場の芝についてのシンポジウムが市当局の主催で開かれた。	
\\	主, 催	
\\	看板	
\\	かんばん	
\\	酔っぱらいの男が殴ったせいで、お店の看板がへこんでいるんです。	
\\	看, 板	
\\	奇妙	
\\	きみょう	
\\	な 
\\	今年の夏が例年になく暑かったせいなのかは分かりませんが、奇妙な発疹が胸に出ましてね。	
\\	奇, 妙	
\\	緊張	
\\	きんちょう	
\\	する 
\\	「明日、就職の面接があるんで、ちょっと緊張してるんだよね。」「え〜そうなの?うまくいくといいね。」	
\\	緊, 張	
\\	強烈	
\\	きょうれつ	
\\	な 
\\	ひっさびさにタバコ吸ったら、強烈なヤニくらくらっちまったよ。	
\\	強, 烈	
\\	大略	
\\	たいりゃく	
\\	まずは事件の大略を説明してくれないか?	
\\	大, 略	
\\	監督	
\\	かんとく	
\\	する 
\\	の 
\\	あの映画監督の名前が思い出せないんだけど。何て名前だっけ。もうここまで出かかってるんだけどな。あー、気持ち悪い!	
\\	監, 督	
\\	恐怖	
\\	きょうふ	
\\	する 
\\	ヤクザにお釣りが百円足りないと言われた時に、とてつもない恐怖を感じました。	
\\	恐, 怖	
\\	三杯	
\\	さんばい	
\\	その時ちょっとへこみ気味だったもんで、ビールを三杯一気飲みしちゃったんだよね。	
\\	三, 杯	
\\	略す	
\\	りゃくす	
\\	日本の若者はすぐ単語やフレーズを略したがる。	
\\	う 
\\	りゃくす.	略	
\\	積もる	
\\	つもる	
\\	体重を減らすために、日常の食事の量を極端に減らす必要はないよ。それよりも、バランスの良い、健康な食事を取ることの方が重要だ。ちりも積もれば山となるって言うじゃないか。	
\\	う 
\\	(つ) 
\\	積	
\\	怒らせる	
\\	おこらせる	
\\	父さんを怒らせると後が面倒よ。	
\\	怒る 
\\	怒る, 
\\	怒	
\\	添える	
\\	そえる	
\\	そのシェフは、ステーキにバニラアイスクリームを添えた。	
\\	う 
\\	(そ) 
\\	添	
\\	寄る	
\\	よる	
\\	学校の帰りに、図書館に寄るつもりです。	
\\	う 
\\	(よ)! 
\\	寄	
\\	構える	
\\	かまえる	
\\	ポーカーをする時は、気長に構えることが大切です。	
\\	構え, 
\\	構え, 
\\	構	
\\	痛む	
\\	いたむ	
\\	昨日の夜から歯が酷く痛むんですが、診てもらえますか。	
\\	"痛い 
\\	痛	
\\	詰まる	
\\	つまる	
\\	魚の骨が喉に詰まっちゃった。	
\\	う 
\\	まる 
\\	詰	
\\	壊す	
\\	こわす	
\\	この河豚を自分なりに料理してみたんですが、お腹を壊してしまいました。	
\\	う 
\\	(す) 
\\	(こわ). 
\\	壊	
\\	診る	
\\	みる	
\\	赤ちゃんがちゃんと育っているのか、お医者さんに診てもらいたいんです。	
\\	見る, 
\\	み 
\\	診	
\\	乾く	
\\	かわく	
\\	洗濯物はまだ乾いていません。	
\\	う 
\\	川 (かわ) 
\\	乾	
\\	道に迷う	
\\	みちにまよう	
\\	私は道に迷うのが得意です。	
\\	迷う? 
\\	道 
\\	迷う.	道, 迷	
\\	腹立つ	
\\	はらだつ	
\\	そんなに腹立てるなって。大したことじゃないじゃん。放っておきなよ。	
\\	腹 
\\	立つ 
\\	たつ 
\\	だつ, 
\\	腹, 立	
\\	見極める	
\\	みきわめる	
\\	自分のこの目で真実を見極めたいんです。	
\\	"極める 
\\	見る 
\\	極める 
\\	見, 極	
\\	眠る	
\\	ねむる	
\\	「ぐっすり眠れた?」「ああ。爆睡したよ。」	
\\	"眠い 
\\	眠い.	眠	
\\	詰める	
\\	つめる	
\\	お弁当にご飯を詰めといてくれない?	
\\	う 
\\	(める), 
\\	詰	
\\	請ける	
\\	うける	
\\	お前の顔を立てるためだけにこの仕事を請けたんだからな。	
\\	受ける? 
\\	請	
\\	誘惑する	
\\	ゆうわくする	する 
\\	彼女はたくさんの男を誘惑する悪魔だ。	
\\	誘惑 
\\	誘惑 
\\	誘, 惑	
\\	締まる	
\\	しまる	
\\	初めての管理職ということで、身の引き締まる思いをしています。	
\\	う 
\\	(し). 
\\	締	
\\	促す	
\\	うながす	
\\	お天気のお姉さんが、視聴者に傘を持って行くよう促した。	
\\	う 
\\	(うなが). 
\\	促	
\\	訪れる	
\\	おとずれる	
\\	もしあのラーメン工場に訪れたいのであれば、できるだけ早くに予約を入れる方がいいですよ。	
\\	う 
\\	訪ねる, 
\\	訪れる, 
\\	訪れる. 
\\	訪ねる. 
\\	訪ねる, 
\\	おとず, 
\\	(おとず) 
\\	訪	
\\	二杯	
\\	にはい	
\\	電車が運行を見合わせている間に、珈琲を二杯飲みました。	
\\	二女一杯, 
\\	二, 杯	
\\	不健康	
\\	ふけんこう	
\\	な 
\\	不健康な野球選手についてのテレビ番組は、全国放送される予定ですよ。	
\\	健康 
\\	健康 
\\	不, 健, 康	
\\	催促	
\\	さいそく	
\\	する 
\\	どうすれば顧客に未払分の支払手続きを丁寧かつ強く催促することができるのでしょうか。	
\\	催, 促	
\\	緊急	
\\	きんきゅう	
\\	な 
\\	の 
\\	緊急地震速報を携帯で受信した三秒後に、地震が起きました。	
\\	緊, 急	
\\	体積	
\\	たいせき	
\\	父さんにガレージの体積を測るように言われたんだけど、理由は知らないよ。何か変だよね。	
\\	体, 積	
\\	禅宗	
\\	ぜんしゅう	
\\	その禅宗に関する映画は、この映画館で上映中です。	
\\	禅, 宗	
\\	浮気	
\\	うわき	
\\	する 
\\	な 
\\	「彼は浮気をして、それを携帯電話を持っていないあなたのせいにしたってこと?」「要約するとそんなところね。」	
\\	浮, 気	
\\	魅力	
\\	みりょく	
\\	私はゴルフの魅力に取り憑かれてしまいました。	
\\	魅, 力	
\\	一覧	
\\	いちらん	
\\	する 
\\	連絡先の一覧から中々上司の名前が見つからないんだけど。	
\\	一, 覧	
\\	変更	
\\	へんこう	
\\	する 
\\	上司に、もっとお互いのメリットを強調するような企画書に変更するように言われました。	
\\	変, 更	
\\	遊園地	
\\	ゆうえんち	
\\	「スカイダイビングをして、その後遊園地に行くんだけど、一緒に行かない?」「えーっ、絶対に嫌!」	
\\	遊, 園, 地	
\\	適当	
\\	てきとう	
\\	な 
\\	居酒屋で、同僚がウェイトレスに「適当に頼みます」と言って注文をしたことに驚きました。	
\\	適, 当	
\\	服飾	
\\	ふくしょく	
\\	の 
\\	私は服飾専門学校を卒業しました。	
\\	服, 飾	
\\	先程	
\\	さきほど	
\\	先程あなたからの手紙を読ませて頂きました。	
\\	先 
\\	程, 
\\	先, 程	
\\	背後	
\\	はいご	
\\	の 
\\	背後に敵が潜んでいたことに全く気づかなかった。	
\\	背, 後	
\\	背景	
\\	はいけい	
\\	の本社を背景にして写真を撮ってもらえませんか。	
\\	背, 景	
\\	預金	
\\	よきん	
\\	する 
\\	彼は帆立の養殖で稼いだ金は全て預金している。	
\\	預, 金	
\\	官僚的	
\\	かんりょうてき	
\\	な 
\\	この会社では官僚的な形式主義がまかり通りすぎている。	
\\	官僚 
\\	的 
\\	官僚 
\\	官, 僚, 的	
\\	強盗	
\\	ごうとう	
\\	銀行強盗をやってみないか?今を逃がすと一生ないよ。	
\\	強 
\\	(ごう) 
\\	強, 盗	
\\	版権	
\\	はんけん	
\\	この手紙には、版権に関する契約更新予定日が記載されています。	
\\	版, 権	
\\	電飾	
\\	でんしょく	
\\	省エネのために電飾に
\\	タップを使ってるって聞いたんだけど、タコ足配線にするのは危ないわよ。	
\\	電 
\\	電, 飾	
\\	移住者	
\\	いじゅうしゃ	
\\	国外移住者になるのは容易なことではない。	
\\	移, 住, 者	
\\	博覧会	
\\	はくらんかい	
\\	この博覧会は無料で入場できます。	
\\	博, 覧, 会	
\\	初版	
\\	しょはん	
\\	彼の小説の初版は半年で売り切れた。	
\\	初, 版	
\\	旗	
\\	はた	
\\	私は可愛い爪楊枝の旗を衝動買いしてしまいました。	
\\	(はた). 
\\	旗	
\\	更に	
\\	さらに	
\\	留学後、更に英語を上達させるには、ネイティブスピーカーと喋り続ける必要があります。	
\\	皿 (さら). 
\\	更	
\\	背	
\\	せ	
\\	紫色のドレスを着ている背の高い女性は、あなたのご友人ですか?	
\\	(せ) 
\\	背	
\\	快速	
\\	かいそく	
\\	な 
\\	の 
\\	コウイチは、昨晩、150キロの快速球を投げ込んで才能の違いを見せつけている夢を見た。	
\\	快速 
\\	快, 速	
\\	新婚旅行	
\\	しんこんりょこう	
\\	の 
\\	新婚旅行でハワイに行くのが待ちきれないよ〜。円高の恩恵にあやかって、お金使い果たすぐらい買い物しまくるつもり!笑	
\\	旅行 
\\	新, 婚, 旅, 行	
\\	照明	
\\	しょうめい	
\\	する 
\\	の 
\\	照明器具の周りを蠅が2-3匹ブンブン飛び交っている。	
\\	照, 明	
\\	成程	
\\	なるほど	
\\	成程、これなら誰にも見つからずに手紙を出すことができる。	
\\	成 
\\	成る 
\\	る 
\\	成, 程	
\\	日系	
\\	にっけい	
\\	の 
\\	私の妻は日系アメリカ人の三世です。	
\\	日 
\\	日 
\\	日本, 
\\	日, 系	
\\	幼稚園	
\\	ようちえん	
\\	幼稚園のクリスマス会のスケジュールをまた微調整しなくちゃいけないかもしれないの。	
\\	幼稚 
\\	幼稚 
\\	幼, 稚, 園	
\\	欠乏	
\\	けつぼう	
\\	する 
\\	ビタミン
\\	が欠乏しているかどうかはどうすれば分かりますか。	
\\	欠, 乏	
\\	漏出	
\\	ろうしゅつ	
\\	する 
\\	原油の海への漏出事故を防ぐために、どんな手段を講じていますか。	
\\	漏, 出	
\\	ご覧	
\\	ごらん	
\\	娘さんの試験の結果はもうご覧になられましたか。	
\\	見る 
\\	ご覧になって下さい, 
\\	ご覧 
\\	覧	
\\	結婚	
\\	けっこん	
\\	する 
\\	の 
\\	「シャーク、私がフグと結婚したってどうして分かったの?」 「ただ人づてに聞いただけさ。」	
\\	結, 婚	
\\	求婚	
\\	きゅうこん	
\\	する 
\\	ホタルは、どの求婚者にも、結婚の条件として無理難題を突き付けた。	
\\	求, 婚	
\\	主	
\\	ぬし	
\\	この池には、古池の主と呼ばれる大きな殿様蛙がいると言われています。	
\\	主人 
\\	主, 
\\	(ぬし), 
\\	主	
\\	精神病	
\\	せいしんびょう	
\\	僕の友達は精神病で頭が少しイッっちゃてるけど、良い人だよ。	
\\	精神 
\\	精神 
\\	精, 神, 病	
\\	国旗	
\\	こっき	
\\	やっべーな。お前この巨大国旗を買ったのかよ?いったいどこに掲げるつもりなの?	
\\	こく 
\\	こっ, 
\\	国, 旗	
\\	騒音	
\\	そうおん	
\\	騒音問題について、お隣さんにどうやって切り出したらいいんだろう。	
\\	騒, 音	
\\	懐かしい	
\\	なつかしい	い 
\\	懐かしい曲がスピーカーから鳴り響いていました。	
\\	い 
\\	懐かしい.	懐	
\\	越権	
\\	えっけん	
\\	の 
\\	お巡りさん、差し出がましいようですがそれは越権行為では無いでしょうか。	
\\	越, 権	
\\	撮影	
\\	さつえい	
\\	する 
\\	お待たせして申し訳ございません。それでは、撮影を始めましょうか。	
\\	撮影 
\\	撮, 影	
\\	枕	
\\	まくら	
\\	結局夫のために高い枕を買うはめになりました。	
\\	枕	
\\	盗作	
\\	とうさく	
\\	する 
\\	他人の作品をパクったら、盗作で訴えられるかもしれないよ。	
\\	盗, 作	
\\	快適	
\\	かいてき	
\\	な 
\\	エアコンのある生活は快適だ。	
\\	快, 適	
\\	快い	
\\	こころよい	い 
\\	彼女の声は小川のせせらぎのように耳にとても快い。	
\\	い 
\\	(心), 
\\	こころ. 
\\	よい 
\\	いい, 
\\	よい). 
\\	こころよい 
\\	快	
\\	快楽	
\\	かいらく	
\\	の 
\\	もう少しで、快楽に溺れるところだったわ。	
\\	快, 楽	
\\	快感	
\\	かいかん	
\\	人に恥をかかせることで快感を味合うような人にはなってもらいたくないわ。	
\\	快, 感	
\\	貧乏	
\\	びんぼう	
\\	な 
\\	サーモンって本当に物欲が強い女性だよね。フグが言ってるほどお金持って無いって知った途端、彼のこと貧乏って呼んで、振っちゃったんだもんね。	
\\	貧, 乏	
\\	貧しい	
\\	まずしい	い 
\\	もしまた貧しい環境に陥ったとしても、また億万長者になれるように努力すると思います。	
\\	い 
\\	先ず (まず)? 
\\	(まず), 
\\	貧	
\\	延長	
\\	えんちょう	
\\	する 
\\	延長コードがごちゃごちゃになっているのが許せないの。	
\\	延, 長	
\\	程よく	
\\	ほどよく	
\\	ひき肉を程よくこねます。	
\\	(よく) 
\\	程	
\\	出版	
\\	しゅっぱん	
\\	する 
\\	その本を出版するのに二年もかかりました。	
\\	出, 版	
\\	購入	
\\	こうにゅう	
\\	する 
\\	この店で偽のシャネル
\\	°5を購入しました。	
\\	購, 入	
\\	購買	
\\	こうばい	
\\	する 
\\	私は消費者の購買意欲について研究をしています。	
\\	購 
\\	買. 
\\	買 
\\	(ばい), 
\\	購, 買	
\\	押し	
\\	おし	
\\	俺は押しに弱いんだよなあ。	
\\	押	
\\	更生	
\\	こうせい	
\\	する 
\\	彼女は自分の息子は更生の見込みがあると信じている。	
\\	更, 生	
\\	更新	
\\	こうしん	
\\	する 
\\	先方は、いくつかの修正を伴う契約更新を提案してきているんだが、あまり気が進まないんだよね。	
\\	更, 新	
\\	乏しい	
\\	とぼしい	い 
\\	日本語能力だけじゃなく、対人能力も乏しいんです。	
\\	い 
\\	(とぼ) 
\\	乏	
\\	盗撮	
\\	とうさつ	
\\	する 
\\	電車内で居眠りしているところを盗撮されたんです。	
\\	盗, 撮	
\\	太陽系	
\\	たいようけい	
\\	太陽系には惑星はいくつあるんだっけ?	
\\	(太陽) 
\\	太陽.	太, 陽, 系	
\\	精神的	
\\	せいしんてき	な 
\\	精神的に病んでいた時、毎日一人で公園をブラブラしていました。	
\\	精神 
\\	的 
\\	精, 神, 的	
\\	盗品	
\\	とうひん	
\\	の 
\\	ここではみんな盗品を売ってるので、「泥棒市場」と呼ばれています。	
\\	盗, 品	
\\	購読	
\\	こうどく	
\\	する 
\\	の 
\\	全アメリカ人が
\\	のニュースレターを購読しているという噂が、日本中に広まった。	
\\	読 (どく) 
\\	(どく) 
\\	購, 読	
\\	体系的	
\\	たいけいてき	な 
\\	私達は、体系的な日本語学習教材を提供しています。	
\\	的 
\\	体, 系, 的	
\\	翌日	
\\	よくじつ	
\\	列車は翌日平常運転に戻りました。	
\\	翌, 日	
\\	翌月	
\\	よくげつ	
\\	彼は翌月黄色いドアがある小さな家に引っ越しています。	
\\	翌, 月	
\\	翌朝	
\\	よくあさ, よくちょう	
\\	申し訳ありませんが、お客様の足つぼマッサージのご予約は、本日の夜9時から翌朝9時30分にご変更されております。	
\\	よくあさ, 
\\	よくちょう. 
\\	よくちょう 
\\	よくあさ 
\\	翌, 朝	
\\	翌年	
\\	よくねん, よくとし	
\\	翌年彼は起業しました。	
\\	年 
\\	よくねん 
\\	よくとし, 
\\	翌, 年	
\\	切符	
\\	きっぷ	
\\	そのツアーの切符は、現地で購入することもできます。	
\\	切る 
\\	きっ, 
\\	ふ 
\\	ぷ. 
\\	切, 符	
\\	攻撃する	
\\	こうげきする	する 
\\	「コウイチ!怒り狂ったライオンの群れが俺たちの町に向かって突撃してくるよ。俺たち、もうすぐ攻撃されちまうよ。どうしたらいいんだ?」「心配するな。俺に任せとけ。」	
\\	攻撃 
\\	攻撃 
\\	攻, 撃	
\\	渇く	
\\	かわく	
\\	ええっと、喉がカラカラに渇いてたから、ビールを飲み過ぎちゃったんだよね。俺は今、ベロベロだ〜。	
\\	渇	
\\	匂う	
\\	におう	
\\	この香水は、全く匂わない。	
\\	その工場からは、チョコレートの香りが匂います。	
\\	春の花畑では、色んな種類のチューリップが匂い、人がたくさん訪れて香りを楽しみます。	
\\	う 
\\	匂	
\\	移る	
\\	うつる	
\\	臭いが服に移るから、焼肉は好きじゃないの。	
\\	う 
\\	(る) 
\\	移す, 
\\	移	
\\	冷やす	
\\	ひやす	
\\	指をドアで挟んじゃって、今氷で冷やしてるの。	
\\	"冷たい 
\\	(ひ).	冷	
\\	飾る	
\\	かざる	
\\	クリスマスツリーの飾り付けをするのが好きです。今年はツリーを暖炉の横に飾りました。	
\\	う 
\\	(かざ). 
\\	飾	
\\	預ける	
\\	あずける	
\\	え?今、いくら預けたって言った?	
\\	う 
\\	(あず). 
\\	預	
\\	診断する	
\\	しんだんする	する 
\\	その医者が私に癌だと診断するほんの少し前に、ある看護婦さんが私の事をちらっと振り返って見たんです。	
\\	診断 
\\	診, 断	
\\	絶える	
\\	たえる	
\\	メンバーが笑い茸と呼ばれるキノコを食べてからというもの、
\\	のオフィスでは、笑いが絶えることはない。	
\\	う 
\\	(える), 
\\	絶つ, 
\\	絶	
\\	照れる	
\\	てれる	
\\	えっ、有り難う。照れるなぁ。	
\\	手 (て). 
\\	照	
\\	浮く	
\\	うく	
\\	スイカも南瓜も水に浮く。	
\\	う 
\\	う!	浮	
\\	越える	
\\	こえる	
\\	日本語では、線路は続くよどこまでもの歌は、「線路は続くよどこまでも、野を越え山越え谷を越え」という感じで始まります。	
\\	う 
\\	(こ) 
\\	越	
\\	照らす	
\\	てらす	
\\	ちょっと懐中電灯であそこの方を照らしてくれない?	
\\	う 
\\	照れる 
\\	らす 
\\	(らす) 
\\	(手) 
\\	照	
\\	漏れる	
\\	もれる	
\\	ねえ、なんかガスが漏れてる臭いがしない?	
\\	う 
\\	(も). 
\\	漏	
\\	並べる	
\\	ならべる	
\\	ボーリングのピンを並べるのにどうしてそんなに時間がかかったんですか。	
\\	並ぶ 
\\	並べる 
\\	(べる), 
\\	並ぶ, 
\\	並	
\\	騒ぐ	
\\	さわぐ	
\\	騒ぐな!死にたいのか?死にたくなけりゃ黙ってろ。	
\\	沢 (さわ) 
\\	騒	
\\	遊ぶ	
\\	あそぶ	
\\	「宿題が済むまで、外で遊んじゃだめだからね。」「言われなくても分かってるよ!」	
\\	う 
\\	(あそ), 
\\	遊	
\\	延期する	
\\	えんきする	する 
\\	その洪水は広範囲に及ぶ被害を引き起こしたため、被害地域の高校は期末試験を一ヶ月延期した。	
\\	延, 期	
\\	延ばす	
\\	のばす	
\\	この写真を延ばして壁に飾りたいなと思ってるんです。	
\\	う 
\\	(の), 
\\	延	
\\	押す	
\\	おす	
\\	すみません。今朝タイムカードを押し忘れちゃったんですが、どうすればいいですか?	
\\	う 
\\	押	
\\	登録する	
\\	とうろくする	する 
\\	臓器提供者として登録をしたきっかけは何ですか。	
\\	登録 
\\	登, 録	
\\	盗む	
\\	ぬすむ	
\\	あいつは絶対に中途半端なことはしない奴だから、3106カラットのダイヤモンドを盗むための壮大な計画を立てているに違いない。	
\\	う 
\\	(ぬす)? 
\\	盗	
\\	撮る	
\\	とる	
\\	写真を撮るために、彼は野球帽を後ろ向きに被り直した。	
\\	う 
\\	取る, 
\\	とる, 
\\	撮	
\\	濡らす	
\\	ぬらす	
\\	髪をよく濡らしてから、シャンプーをするといいですよ。	
\\	大事な
\\	なので、コーヒーや水で濡らさないで下さいね。	
\\	先輩にフラれてから、毎日涙で枕を濡らしています。	
\\	う 
\\	濡	
\\	研修生	
\\	けんしゅうせい	
\\	こちらが、新しい研修生の三浦くんです。	
\\	"研修 
\\	研修 
\\	研, 修, 生	
\\	〜系	
\\	けい	
\\	私、アキバ系のオタッキーな男性に何故か魅力を感じるんだよね。	
\\	系	
\\	浮世絵	
\\	うきよえ	
\\	その浮世絵についてお話をしませんか?	
\\	浮 
\\	うき 
\\	浮く), 世 
\\	よ, 
\\	絵 
\\	え. 
\\	うきよえ! 
\\	浮, 世, 絵	
\\	未婚	
\\	みこん	
\\	の 
\\	結局、未婚の女性ではないという真実を彼に話すはめになりました。	
\\	未, 婚	
\\	不快	
\\	ふかい	
\\	な 
\\	上司の不快なジョークに笑いたくもないのに笑っている自分が嫌いです。	
\\	不, 快	
\\	不適	
\\	ふてき	
\\	な 
\\	この職が私にとって適不適かを考えているんですが、まだ答えは出ていません。	
\\	不, 適	
\\	逆さま	
\\	さかさま	
\\	な 
\\	の 
\\	どうして蝶々を逆さまに描いたの?	
\\	さま 
\\	様, 
\\	逆らう 
\\	様, 
\\	逆	
\\	観覧	
\\	かんらん	
\\	する 
\\	の撮影現場を観覧しました。	
\\	観, 覧	
\\	漏水	
\\	ろうすい	
\\	する 
\\	どんなに値段が高くても、その漏水検出器を買うべきだと思います。	
\\	漏, 水	
\\	精一杯	
\\	せいいっぱい	
\\	今は古い写真を捨てることが私にできる精一杯です。	
\\	一杯 
\\	精, 一, 杯	
\\	驚嘆	
\\	きょうたん	
\\	する 
\\	その太陽光発電所の規模に驚嘆しました。	
\\	驚, 嘆	
\\	航空券	
\\	こうくうけん	
\\	博多までの航空券を手配してもらえますか。	
\\	(航空) 
\\	航, 空, 券	
\\	既に	
\\	すでに	
\\	私が電話した時、彼は既に退社してしまってました。	
\\	(すで). 
\\	既	
\\	救急車	
\\	きゅうきゅうしゃ	
\\	救急車が急発進して男性にぶつかったため、もう一台別の救急車も必要になった。	
\\	救, 急, 車	
\\	鑑定	
\\	かんてい	
\\	する 
\\	の 
\\	お前が赤ん坊の本当の父親かどうかを調べるために、
\\	鑑定を受けた方がいいんじゃないか。	
\\	鑑, 定	
\\	散歩	
\\	さんぽ	
\\	する 
\\	散歩をする時は、小銭しか持ち歩きません。	
\\	ほ 
\\	ぽ, 
\\	散, 歩	
\\	平均	
\\	へいきん	
\\	する 
\\	2014年のセンター試験の平均点が出ました。	
\\	平, 均	
\\	ハチの巣	
\\	はちのす, ハチのす	
\\	ハチの巣の周りで沢山のハチがブンブン飛び回っている。	
\\	"ハチ 
\\	巣	
\\	帯	
\\	おび	
\\	白い帯はお手入れが大変です。	
\\	(おび)! 
\\	帯	
\\	血脈	
\\	けつみゃく	
\\	我々の血脈を絶やさせるべきではない。	
\\	血, 脈	
\\	用心棒	
\\	ようじんぼう	
\\	私の用心棒をしてくれない?	
\\	用, 心, 棒	
\\	解散	
\\	かいさん	
\\	する 
\\	私達にとって、衆議院を解散するメリットとデメリットは何ですか?	
\\	解, 散	
\\	中華	
\\	ちゅうか	の 
\\	中華料理が食べたい気分だ。今夜一緒に中華食べに行かないか?	
\\	中国 
\\	中 
\\	中, 華	
\\	掃除	
\\	そうじ	
\\	する 
\\	まずやるべきことをやらなくてはね。部屋の掃除から始めようか。	
\\	除 
\\	じ. 
\\	(じ). 
\\	掃, 除	
\\	大陸	
\\	たいりく	
\\	の 
\\	ムー大陸って本当に実在したんですか?	
\\	大, 陸	
\\	廊下	
\\	ろうか	
\\	彼が廊下でバナナの皮に滑るのを見て、彼女は思わず吹き出してしまった。	
\\	廊, 下	
\\	通貨	
\\	つうか	
\\	の 
\\	私はこの通貨をよく知っています。	
\\	通, 貨	
\\	既決	
\\	きけつ	
\\	の 
\\	あなたのケースでは、既決囚救済制度が適用される可能性もなきにしもあらずだと思います。	
\\	既, 決	
\\	子孫	
\\	しそん	
\\	彼は宮家の子孫です。	
\\	子, 孫	
\\	直径	
\\	ちょっけい	
\\	直径100フィートもある巨大な新種のクラゲが、ビーチをドロドロにしてしまいました。	
\\	直, 径	
\\	離婚	
\\	りこん	
\\	する 
\\	「彼とは離婚するつもりなの。」「ちょっ…早まらないで。決断する前によく考えて。」	
\\	離, 婚	
\\	編集	
\\	へんしゅう	
\\	する 
\\	の 
\\	どうしてブログ記事を編集するのに一時間もかかったんですか。	
\\	編, 集	
\\	編者	
\\	へんしゃ	
\\	こちらの雑誌の編者を務めてくれた丸山君です。	
\\	編, 者	
\\	均等	
\\	きんとう	
\\	な 
\\	キャリアを積む機会は、男女に均等に与えられるべきです。	
\\	均, 等	
\\	除外	
\\	じょがい	
\\	する 
\\	その馬はソエにより競争除外となりました。	
\\	除, 外	
\\	感嘆符	
\\	かんたんふ	
\\	メールで語尾に必ず感嘆符を付ける友達がいます。	
\\	感, 嘆, 符	
\\	既存	
\\	きそん, きぞん	
\\	の 
\\	どれでもいいから、既存のファイルを開いてみてください。	
\\	既, 存	
\\	恐れ	
\\	おそれ	
\\	土砂崩れの恐れがあります。	
\\	恐	
\\	融資	
\\	ゆうし	
\\	する 
\\	銀行の融資を受けるのは思っていたよりも難しかったです。	
\\	融, 資	
\\	華道	
\\	かどう	
\\	小さい頃からずっと華道を習っています。	
\\	華, 道	
\\	壊れ物	
\\	こわれもの	
\\	小包に、「壊れ物、取扱注意」って書いておいた方がいいかな。	
\\	壊れる 
\\	物 
\\	壊, 物	
\\	貨物	
\\	かもつ	
\\	毎日札幌貨物ターミナルと福岡貨物ターミナル間を走行する貨物列車があります。	
\\	物 
\\	もつ. 
\\	(持つ 
\\	貨, 物	
\\	融合	
\\	ゆうごう	
\\	する 
\\	の 
\\	俺たちのバンドは、ジャズとロックが融合した音楽を演奏しています。	
\\	融, 合	
\\	華美	
\\	かび	
\\	な 
\\	華美な服装は控えた方がいいんじゃないですか。	
\\	華, 美	
\\	幾何学	
\\	きかがく	
\\	幾何学のテストで
\\	プラス取ったの?すごいじゃん、よくやったね!	
\\	学 
\\	幾 
\\	何 
\\	か. 
\\	""何か????
\\	か. 
\\	幾 
\\	(き) 
\\	か?
\\	幾, 何, 学	
\\	孫	
\\	まご	
\\	新しいコンピューターを接続することは、 私の孫にとってはお茶の子さいさいだ。	
\\	(まご). 
\\	孫	
\\	墓	
\\	はか	
\\	僕たちの墓を太平洋に浮かぶ小さな島に作ることを、いつも夢見てきたんだよ。	
\\	(はか) 
\\	墓	
\\	墓地	
\\	ぼち	
\\	うちの庭は墓地に隣接している。	
\\	墓, 地	
\\	花粉症	
\\	かふんしょう	
\\	プロポリスが花粉症に効くって聞いたよ。	
\\	花. 
\\	花 
\\	か, 
\\	花, 粉, 症	
\\	幾つ	
\\	いくつ	
\\	録音に適したマイクを幾つ知っていますか。	
\\	""つ.
\\	幾	
\\	幾度	
\\	いくど, いくたび	
\\	彼は、幾度失敗しても、諦めなかった。	
\\	幾, 度	
\\	幾ら	
\\	いくら	
\\	送料が幾らになるのか教えて頂けませんでしょうか。	
\\	幾	
\\	真似	
\\	まね	
\\	する 
\\	四歳になる娘が、見るもの全てを真似するので困っています。	
\\	真 
\\	似. 
\\	真, 似	
\\	陸軍	
\\	りくぐん	
\\	私達は陸軍訓練を見に来たただの見学です。	
\\	陸, 軍	
\\	巣立ち	
\\	すだち	
\\	どうして鳥は飛べるようになる前に巣立ちをするのですか。	
\\	巣 
\\	立つ 
\\	巣, 立	
\\	豪華	
\\	ごうか	
\\	な 
\\	うわっ!めっちゃ豪華やん。美味しそう。(関西弁)	
\\	豪, 華	
\\	雄犬	
\\	おすいぬ	
\\	「パパ、僕、雄犬がほしいな。雄犬を飼える?」「様子を見ようか。でも、どうして雌犬じゃないの?」	
\\	犬, 
\\	雄 
\\	(押す) 
\\	雄, 犬	
\\	外務大臣	
\\	がいむだいじん	
\\	とにかく、手短に言うとね、フグと僕は外務大臣について口論をして、それ以来、僕は彼に会っていないんだ。	
\\	(大臣) 
\\	外, 務, 大, 臣	
\\	道徳	
\\	どうとく	
\\	の 
\\	どうして動物には道徳感がないって断言できるの?	
\\	道, 徳	
\\	山脈	
\\	さんみゃく	
\\	ヒマラヤ山脈は多種多様な自然に恵まれています。	
\\	山, 脈	
\\	富士山	
\\	ふじさん	
\\	コウイチ:明日富士山に登るなんて、もう気が狂っちゃいそうだよ。 ビエト:分かる分かる。俺もめちゃくちゃ不安。	
\\	富, 士, 山	
\\	泥	
\\	どろ	
\\	の 
\\	「おい、ちょっと待て!お前今、床に泥を付けただろ?」「えっ!泥なんて付けてませんよ。私のうんこなら付けましたが。」	
\\	(どろ) 
\\	泥	
\\	泥水	
\\	でいすい, どろみず	
\\	「ちょっと!今私の雑誌にコーヒーをこぼしたでしょ?」「ああ、ごめん!悪かったよ。」「あ〜ぁ〜…ジョニーデップの股間のところに泥水みたいな染みができちゃったじゃない!」	
\\	泥, 水	
\\	巣	
\\	す	
\\	巣の中では、ツバメの雛が五羽、ぴいぴい鳴いていました。	
\\	巣	
\\	普通	
\\	ふつう	
\\	の 
\\	普通、人間は他人に影響されるものだと考えられています。	
\\	普, 通	
\\	人脈	
\\	じんみゃく	
\\	人脈作りは、ビジネスに関わる殆ど全ての人にとってとても重要なスキルですが、事業家にとっては殊更です。	
\\	人, 脈	
\\	画廊	
\\	がろう	
\\	彼女は自分の画廊を開いた。	
\\	画, 廊	
\\	棒	
\\	ぼう	
\\	もし私があんたのお父さんなら、この棒でお尻をぶっ叩いてやったのに。	
\\	棒	
\\	粉	
\\	こな, こ	
\\	最近私は粉ものを食べ過ぎています。	
\\	(こな) 
\\	こ 
\\	粉	
\\	粉状	
\\	ふんじょう	
\\	の 
\\	新鮮な鷹の爪があるので粉状の唐辛子にしたいなと思っているんですが、やり方がよく分かりません。	
\\	粉, 状	
\\	総理大臣	
\\	そうりだいじん	
\\	よぉみんな!コウイチ総理大臣がここに登場だぜ!チェケラ!	
\\	(総理) 
\\	総理 
\\	大臣 
\\	総, 理, 大, 臣	
\\	探偵	
\\	たんてい	
\\	する 
\\	の 
\\	彼女は結局探偵になりました。	
\\	探, 偵	
\\	恐らく	
\\	おそらく	
\\	犯人は恐らくこの辺りに潜んでいるのでしょう。	
\\	(らく) 
\\	恐	
\\	尋問	
\\	じんもん	
\\	する 
\\	警察は関係者全員を尋問しました。	
\\	尋, 問	
\\	半径	
\\	はんけい	
\\	半径3
\\	の球体の体積は何立方センチメートルですか。	
\\	半, 径	
\\	探究	
\\	たんきゅう	
\\	する 
\\	我々はその原因を探究している。	
\\	探, 究	
\\	経路	
\\	けいろ	
\\	このアパートから東大までの一番速くて安い通学経路を教えてください。	
\\	経, 路	
\\	恐い	
\\	こわい	
\\	い 
\\	俺の母親は世界で一番恐い。	
\\	い 
\\	(こわ). 
\\	恐	
\\	内偵	
\\	ないてい	
\\	する 
\\	警察はどんな時に内偵捜査をするのですか。	
\\	内, 偵	
\\	印鑑	
\\	いんかん	
\\	公的な印鑑が押された書類が必要になります。	
\\	印, 鑑	
\\	分離	
\\	ぶんり	
\\	する 
\\	クリームを牛乳から分離させる方法をググってみるよ。	
\\	分, 離	
\\	華やか	
\\	はなやか	
\\	な 
\\	いつか綺麗なドレスを着て、華やかなカクテルパーティーに行ってみたいな。	
\\	花
\\	(はな). 
\\	華	
\\	嘆息	
\\	たんそく	
\\	する 
\\	他人よりも自分がずっと短足だと気づいた時、彼は深々と嘆息して、日本短足協会へと向かった。	
\\	嘆, 息	
\\	クモの巣	
\\	くものす, クモのす	
\\	の 
\\	屋根裏部屋で、クモの巣が掛かった祖父の未完成の絵を見つけたんです。	
\\	クモ 
\\	巣	
\\	倉庫	
\\	そうこ	
\\	倉庫の鍵を閉めるには、カチッという音が聞こえるまでレバーを上げる必要があります。	
\\	倉, 庫	
\\	鼻詰まり	
\\	はなづまり	
\\	私は高熱と鼻詰まりに苦しんでいます。	
\\	"詰まる 
\\	鼻 
\\	詰まる 
\\	詰まり 
\\	鼻, 詰	
\\	均整	
\\	きんせい	
\\	彼女はとても均整のとれた体形をしている.	
\\	均, 整	
\\	壊れる	
\\	こわれる	
\\	寝過ごして朝ごはんを食べ損ねた上に、車が壊れたんだ。本当に今日はついてないよ。	
\\	"壊す 
\\	壊れる 
\\	壊す, 
\\	壊	
\\	富む	
\\	とむ	
\\	カナダはカナディアンベーコンに富む。	
\\	う 
\\	(む). 
\\	富	
\\	怖がる	
\\	こわがる	
\\	私の主人は逆さ睫毛の手術を受けるのをとても怖がっていたんですよ。	
\\	"怖い 
\\	怖い.	怖	
\\	散る	
\\	ちる	
\\	「ねえ、聞いてるの?」「あっ、うん。聞いてるよ。ごめん、ごめん。でも、ちょっと気が散っちゃってた。」	
\\	う 
\\	(ち), 
\\	散	
\\	驚く	
\\	おどろく	
\\	「フグ?お前、フグか?」「サンマか?こいつは驚いた。高校以来だね。元気かい?」	
\\	う 
\\	(おどろ). 
\\	驚	
\\	思い詰める	
\\	おもいつめる	
\\	私は失業中ですが、そのことを思い詰めてはいません。	
\\	"詰める 
\\	思い 
\\	詰める 
\\	思, 詰	
\\	除く	
\\	のぞく	
\\	赤い紙を除く全ての紙を、その折り目まで折ってください。	
\\	う 
\\	(のぞ) 
\\	除	
\\	恐れる	
\\	おそれる	
\\	目が見えなくなること以外は、何も恐れることはないよ。	
\\	う 
\\	恐ろしい, 
\\	恐	
\\	尋ねる	
\\	たずねる	
\\	自分のウンコを食べてみて、それがすっごい苦かったんだけど、どうして苦いのかって先生に尋ねてみたら、「胆汁酸がいっぱい含まれてるから苦いんだよ」って教えてくれたよ。	
\\	う 
\\	(たず) 
\\	尋	
\\	絶やす	
\\	たやす	
\\	夜の間、火を絶やさないように気をつけるんだぞ。	
\\	"絶える 
\\	絶やす 
\\	(やす).
\\	絶える.	絶	
\\	編む	
\\	あむ	
\\	ブラジャーを編んでみようと思ったきっかけはなんですか。	
\\	う 
\\	(あ). 
\\	編	
\\	徳川	
\\	とくがわ	
\\	彼女は歴女で、徳川幕府にはとても精通しています。	
\\	徳 
\\	川 
\\	徳, 川	
\\	徳島県	
\\	とくしまけん	
\\	知事は、徳島県をどういう風に変えていくおつもりですか?私は今のままがいいと思うのですが。	
\\	徳, 島, 県	
\\	嘆く	
\\	なげく	
\\	どんなにくよくよ嘆いても、死んだ人はもう戻っては来ないんだから、彼女のことはさっさと忘れて、代わりに私と結婚してよ。	
\\	(投げる) 
\\	嘆	
\\	迷わす	
\\	まよわす	
\\	もし俺の娘を路頭に迷わすようなことがあったら、一生許さないぞ。	
\\	迷う 
\\	(わす) 
\\	迷う, 
\\	迷	
\\	探す	
\\	さがす	
\\	三本足の犬がバーにやって来てこう言った。「俺の足を打った男を探しているんだ。半分河豚で半分豆腐みたいな男なんだが。」	
\\	う 
\\	(さが) 
\\	探	
\\	積む	
\\	つむ	
\\	それじゃあ2人組を作って、トラックに七面鳥を積み始めて下さい。	
\\	"積もる 
\\	(む). 
\\	積もる, 
\\	積	
\\	救う	
\\	すくう	
\\	友人のブログが荒らされています。友人を救うための知恵を貸してください。	
\\	う 
\\	すく 
\\	(すくう) 
\\	救	
\\	似る	
\\	にる	
\\	お前が彼女の兄貴だって?嘘つけ!全然似てないじゃねえか。	
\\	う 
\\	(に) 
\\	似	
\\	離れる	
\\	はなれる	
\\	親元から離れて一人暮らしをすることは思っていたよりも簡単でした。	
\\	う 
\\	(れる) 
\\	れる 
\\	花
\\	(はな) 
\\	離	
\\	主催する	
\\	しゅさいする	する 
\\	うちの学校で土曜日にカラオケパーティーを主催するんだけど、よかったら来ない?	
\\	主催 
\\	主催 
\\	主, 催	
\\	見詰める	
\\	みつめる	
\\	その犬は、真っすぐ上へ上へと飛んで行く風船をじっと見詰めていた。	
\\	(詰める) 
\\	見 
\\	詰める.	見, 詰	
\\	緊張する	
\\	きんちょうする	する 
\\	赤ちゃんが逆子で産まれてくるって分かってたので、出産時は緊張しました。	
\\	緊張 
\\	緊, 張	
\\	掃く	
\\	はく	
\\	会議が始まる前に、会議室をちょっと掃いておいてもらっていいかな。	
\\	う 
\\	(は) 
\\	掃	
\\	催促する	
\\	さいそくする	する 
\\	催促するようで申し訳ありませんが、東京タワーの写真を何枚か撮って月曜日までに送ってもらえませんでしょうか。	
\\	催促 
\\	催促 
\\	催, 促	
\\	地下街	
\\	ちかがい	
\\	ちょっくら地下街に行くところだよ。	
\\	(地下) 
\\	地, 下, 街	
\\	普段	
\\	ふだん	の 
\\	私は普段殆ど電話をしないことを、母に謝りました。	
\\	普, 段	
\\	野菜	
\\	やさい	
\\	の 
\\	どんなにお腹が苦しくても、野菜なら食べられます。	
\\	野, 菜	
\\	菜食	
\\	さいしょく	
\\	する 
\\	の 
\\	菜食主義者の人は偽善者が多いっていう意見に賛成してくれる人はいますか。	
\\	菜, 食	
\\	倉	
\\	くら	
\\	倉で見つけた地図を広げてみた。	
\\	(くら) 
\\	倉	
\\	富	
\\	とみ	
\\	巨万の富を築くことは思ったより簡単ではありません。	
\\	(とみ) 
\\	富	
\\	富裕	
\\	ふゆう	
\\	な 
\\	どうして人は富裕層と貧困層に分けられるのでしょうか。	
\\	富, 裕	
\\	地帯	
\\	ちたい	
\\	ここは、あなた以外には安全地帯です。	
\\	地, 帯	
\\	功績	
\\	こうせき	
\\	成功が誰の功績かを判断するのは難しい。	
\\	功, 績	
\\	賛成	
\\	さんせい	
\\	する 
\\	の 
\\	日本が新たな移民を受け入れることには賛成ですか。	
\\	賛, 成	
\\	先祖	
\\	せんぞ	
\\	の 
\\	うちの家族には、六世代に渡って先祖代々受け継がれてきた特別なおむつがあります。	
\\	先, 祖	
\\	食欲	
\\	しょくよく	
\\	の 
\\	ダイエットをしたいんですが、食欲があり過ぎるんですよね。	
\\	食, 欲	
\\	恩人	
\\	おんじん	
\\	コウイチは私の恩人です。私がベーコンに溺れていた時、私を救ってくれたんです。	
\\	恩, 人	
\\	血液	
\\	けつえき	
\\	どうして血液は赤いのですか。	
\\	血, 液	
\\	ゆで卵	
\\	ゆでたまご	
\\	僕が君のゆで卵を盗んだと思っているのなら、とんだお門違いだよ。	
\\	ゆ 
\\	湯, 
\\	卵	
\\	火傷	
\\	やけど, かしょう	
\\	する 
\\	カレーを零して、手に火傷を負った。	
\\	(やけど) 
\\	火, 傷	
\\	背広	
\\	せびろ	
\\	背広のボタンをなくした。	
\\	び 
\\	ひ.	背, 広	
\\	肉欲	
\\	にくよく	
\\	の 
\\	どうしてキリスト教徒と仏教徒の中には、肉欲を罪深いものだとして否定する人がいるのですか。	
\\	肉, 欲	
\\	桜んぼ	
\\	さくらんぼ	
\\	この桜んぼ、やばいくらい美味しいんだけど。	
\\	桜	
\\	大騒ぎ	
\\	おおさわぎ	
\\	する 
\\	どこかの国で、子供用の股の部分が開いたセクシー下着が販売されて大騒ぎになっていたことがあったよね。	
\\	騒ぐ 
\\	大 
\\	騒ぐ 
\\	大, 騒	
\\	押入れ	
\\	おしいれ	
\\	彼女を誘拐して押入れに閉じ込めたあの日から、僕はずっと彼女のことばかり考えているんだ。	
\\	押し 
\\	し 
\\	押) 
\\	入れる
\\	押, 入	
\\	銭	
\\	ぜに	
\\	私は妻のために死ぬまで働いて銭を稼ぐだけの奴隷ですよ。	
\\	せん, 
\\	ぜに. 
\\	ぜに 
\\	銭	
\\	共産党	
\\	きょうさんとう	
\\	最初に気づいたのは、彼が共産党員だということでした。	
\\	共, 産, 党	
\\	複雑	
\\	ふくざつ	な 
\\	当時、家庭環境が複雑だったので、高校には行っていません。	
\\	複, 雑	
\\	便秘	
\\	べんぴ	
\\	する 
\\	の 
\\	便秘がひどくてさぁ。もう一週間以上ウンコしていないんだよ!	
\\	便 
\\	便所, 
\\	ひ 
\\	ぴ 
\\	便, 秘	
\\	伝染病	
\\	でんせんびょう	
\\	世界で一番危険な伝染病は何ですか。	
\\	伝, 染, 病	
\\	複写	
\\	ふくしゃ	
\\	する 
\\	一枚目はお手元にお持ちください。私達は複写の方を頂きますね。	
\\	複, 写	
\\	飾り	
\\	かざり	
\\	そろそろ正月飾りの準備をしておかなくちゃ。	
\\	飾	
\\	志望	
\\	しぼう	
\\	する 
\\	この仕事を志望する動機は何ですか?	
\\	志, 望	
\\	机	
\\	つくえ	
\\	「君の机の上に、コウイチの写真を置いておいたよ。」「ありがとう。恩に着るよ。」	
\\	机	
\\	汚水	
\\	おすい	
\\	雪の日に汚水の配管が詰まってしまいました。	
\\	汚, 水	
\\	卵	
\\	たまご	
\\	「卵がかえる前にひなを数えるな」という諺は、日本語では「とらぬ狸の皮算用をするな」です。	
\\	卵	
\\	英訳	
\\	えいやく	
\\	する 
\\	日本語から英訳することしかできません。	
\\	英, 訳	
\\	眼	
\\	め, まなこ	
\\	コンピュータを使う仕事は眼を酷使しやすい。	
\\	目.	
\\	(目), 
\\	眼	
\\	眼球	
\\	がんきゅう	
\\	の 
\\	私は自分の左の眼球をあちこち捜したが、なかなか見つけられなかった。	
\\	眼, 球	
\\	老眼	
\\	ろうがん	
\\	の 
\\	老眼は大体いつ頃から始まるんですか。	
\\	老眼.	
\\	老, 眼	
\\	永久	
\\	えいきゅう	
\\	な 
\\	の 
\\	脇毛を永久脱毛したから、今脇はツルツルだよ。	
\\	永, 久	
\\	永遠	
\\	えいえん	
\\	な 
\\	の 
\\	もう永遠に日本には帰ってこないの?	
\\	永久, 
\\	永, 遠	
\\	成績	
\\	せいせき	
\\	家庭教師の先生のお陰で、成績がグンと上がりました。	
\\	成, 績	
\\	祖父	
\\	そふ	
\\	の 
\\	う〜ん、何て言ったらいいのかな。僕の祖父は単なる気分屋なんだよね。機嫌がいい時もあれば、悪い時もあるっていうか。だから、あんまり気にしないで。	
\\	祖 
\\	父. 
\\	父 
\\	ふ, 
\\	(ふ) 
\\	祖, 父	
\\	祖母	
\\	そぼ	
\\	の 
\\	私の祖母はいつも指にド派手な指輪を嵌めています。	
\\	祖 
\\	母 
\\	母 
\\	祖母, 
\\	(ぼ), 
\\	祖, 母	
\\	傷心	
\\	しょうしん	
\\	の 
\\	彼に振られて傷心気味なので、パリに傷心旅行に行くんです。	
\\	傷, 心	
\\	桜肉	
\\	さくらにく	
\\	お前も一緒に桜肉パーティーに来いよ。イカしたパーティーになるぜ。	
\\	桜 
\\	肉 
\\	桜, 肉	
\\	衛生	
\\	えいせい	
\\	の 
\\	その歯科衛生士は、お口を開けてください、と私に言いながらにやりと笑い、日焼けした顔に白い歯をキラリと見せた。	
\\	衛, 生	
\\	感染	
\\	かんせん	
\\	する 
\\	の 
\\	歯周病は感染症の一種だとは知りませんでした。	
\\	感, 染	
\\	傷者	
\\	しょうしゃ	
\\	その事故による死者は三名、傷者は二百人以上にのぼるという。	
\\	傷, 者	
\\	志	
\\	こころざし	
\\	ビエトは高い志を持っており、どんどん夢を実現していってる。	
\\	こころざし (心刺し). 
\\	こころざし.	志	
\\	興味	
\\	きょうみ	
\\	俺はエロゲに全く興味が無いふりをしている。	
\\	興, 味	
\\	通訳	
\\	つうやく	
\\	する 
\\	もし私が全能の神だったら、この会話を通訳してあげることができるけど、そうじゃないから無理だわ。ごめんよ。	
\\	通, 訳	
\\	和訳	
\\	わやく	
\\	する 
\\	いくつか小さな間違いはあるものの、うまく和訳できていると思いますよ。	
\\	和 
\\	和, 訳	
\\	自民党	
\\	じみんとう	
\\	我らが自民党が今年は熱いな。もう誰も我々のことを「愚民党」なんて呼ぶことはないだろう。	
\\	自, 民, 党	
\\	液体	
\\	えきたい	
\\	機内に持ち込み可能な液体の量はどれぐらいですか。	
\\	液, 体	
\\	久しい	
\\	ひさしい	い 
\\	わたしがベーコンをたべなくなって久しい。	
\\	へいわボケして久しい日本人には、おそらくりかいできないだろう。	
\\	かれが生のきんぎょをたべてしんでから久しい。	
\\	い 
\\	(ひさ) 
\\	久	
\\	結婚式	
\\	けっこんしき	
\\	「もう明日だけど、でも、やっぱり、私たちの結婚式を中止するべきだと思うの。」「僕もそれを言おうと思っていたんだ。」	
\\	結婚 
\\	結, 婚, 式	
\\	雑費	
\\	ざっぴ	
\\	日本の大学生の食費と雑費の平均はいくらぐらいですか。	
\\	ざつ 
\\	ざっ 
\\	ひ 
\\	ぴ, 
\\	(ざっぴ), 
\\	雑, 費	
\\	複数	
\\	ふくすう	
\\	の 
\\	私には複数のニックネームとアカウントがあります。	
\\	複, 数	
\\	雑音	
\\	ざつおん	
\\	この音声ファイルの雑音を消したいんです。	
\\	雑, 音	
\\	撮影禁止	
\\	さつえいきんし	
\\	この幼稚園では、子どもたちのお遊戯会は撮影禁止です。	
\\	撮影 
\\	禁止 
\\	撮影 
\\	禁止.	撮, 影, 禁, 止	
\\	背中	
\\	せなか	
\\	背中のニキビを無くす方法が知りたいよ。	
\\	背 
\\	中 
\\	背, 中	
\\	酸素	
\\	さんそ	
\\	の 
\\	毎日午後に有酸素運動をしています。	
\\	酸, 素	
\\	党員	
\\	とういん	
\\	ある共和党の党員の演説にとても感動しました。	
\\	党, 員	
\\	桜色	
\\	さくらいろ	
\\	自分が二歳か三歳の頃に、一丁前に桜色のタキシードを着てる写真を見ました。	
\\	ピンク, 
\\	桜 
\\	色 
\\	桜, 色	
\\	遊び	
\\	あそび	
\\	最近学校で流行っている遊びを教えて下さい。	
\\	遊ぶ. 
\\	遊	
\\	政党	
\\	せいとう	
\\	の 
\\	日本での政党の成立要件を教えてください。	
\\	政, 党	
\\	汚染	
\\	おせん	
\\	する 
\\	自国の
\\	5汚染レベルはどうすれば分かるのでしょうか。	
\\	汚, 染	
\\	貧乏人	
\\	びんぼうにん	
\\	お前は自分のことを素晴らしいと思っているみたいだけどな、ただの貧乏家族の貧乏人だってことを、よ〜く肝に銘じておくんだな。	
\\	貧乏 
\\	貧乏 
\\	貧, 乏, 人	
\\	序文	
\\	じょぶん	
\\	の 
\\	私はいつも本の序文は飛ばします。	
\\	序, 文	
\\	採算	
\\	さいさん	
\\	もし別の惑星で金を見つけることができて、それを地球に持ち帰って販売した場合、採算は取れるのでしょうか。	
\\	採, 算	
\\	銭湯	
\\	せんとう	
\\	祖父の一日は、銭湯から始まります。	
\\	湯, 
\\	(とう), 
\\	銭, 湯	
\\	出版社	
\\	しゅっぱんしゃ	
\\	その出版社では、入館の際に身分証明書を提示しなければなりません。	
\\	出版 
\\	出版 
\\	出, 版, 社	
\\	押しボタン	
\\	おしぼたん, おしボタン	
\\	押しボタン式のドアでも、自動ドアって呼ぶのかな。	
\\	押し 
\\	ボタン 
\\	押す.	押	
\\	生卵	
\\	なまたまご	
\\	「生卵アレルギーはありますか?」「私の知る限りではありません。」	
\\	生 
\\	生 
\\	卵 
\\	生卵.	生, 卵	
\\	訳	
\\	わけ	
\\	どうして便器の上で逆立ちをしようとしたのか訳を聞かせてください。	
\\	(わけ) 
\\	訳	
\\	恩賞	
\\	おんしょう	
\\	あのベーコン泥棒を引っ捕えた者には恩賞を与えよう。	
\\	功労に準じて恩賞が付与されるということを覚えておいてくださいね。	
\\	後醍醐天皇による建武の新政は、公家にはたくさん恩賞が出たのに武士には少なかったため、武士からの評判が悪かった。	
\\	恩, 賞	
\\	桜	
\\	さくら	
\\	相手の気を引いたりうまく釣ったりする人のことを英語ではデコイと言うが、日本語では桜、ネット用語では釣りと言う。	
\\	桜	
\\	衛星	
\\	えいせい	
\\	の 
\\	落下する衛星を避けることは、不可能ではありません。	
\\	衛, 星	
\\	匂い	
\\	におい	
\\	先輩、革ジャンの匂いは好きですか?	
\\	トイレは匂いが強すぎて、鼻で息をしたくない。	
\\	この温泉は、バナナの匂いがする。	
\\	匂	
\\	秘密	
\\	ひみつ	
\\	な 
\\	の 
\\	秘密は誰にも言わないよ。ちゃんと、口にチャックしておくよ。	
\\	秘, 密	
\\	密か	
\\	ひそか	
\\	な 
\\	僕たちは、お互いの家族にも知らせず、密かに入籍をしたんだ。	
\\	(ひそ) 
\\	密	
\\	密告	
\\	みっこく	
\\	する 
\\	とある会社の脱税について、税務署に密告したいと思ってるんです。	
\\	みつ 
\\	みっ.	密, 告	
\\	密会	
\\	みっかい	
\\	する 
\\	私達は密会によくこの部屋を利用しています。	
\\	密, 会	
\\	自衛	
\\	じえい	
\\	する 
\\	の 
\\	自衛のための殺人は合法なのでしょうか。	
\\	自, 衛	
\\	順序	
\\	じゅんじょ	
\\	ヤフオクで入札した後の手続きの順序についてブログでまとめてみました。	
\\	順, 序	
\\	飾り気	
\\	かざりけ	
\\	あいつは本当に飾り気のない女だ。	
\\	気 
\\	(け). 
\\	飾, 気	
\\	密輸	
\\	みつゆ	
\\	する 
\\	俺は一度麻薬の密輸をする話を持ちかけられたことがあるよ。もちろん断ったけどね。	
\\	密, 輸	
\\	社会党	
\\	しゃかいとう	
\\	社会党の主張は現実的ではないような気がする。	
\\	(社会) 
\\	社, 会, 党	
\\	酸っぱい	
\\	すっぱい	い 
\\	最初に食べた食べ物は腐りかけていて酸っぱかったです。	
\\	(す) 
\\	酸	
\\	汚点	
\\	おてん	
\\	人生の汚点は日本語では黒歴史とも呼ばれる。	
\\	汚, 点	
\\	背丈	
\\	せたけ	
\\	背丈の低い男が千鳥足でバーから出てきて、私の目の前でゲロを吐いたんです。	
\\	背 
\\	丈, 
\\	背, 丈	
\\	厳しい	
\\	きびしい	い 
\\	インターネットでは厳しいコメントを書く人が多い。	
\\	い 
\\	(きび). 
\\	厳	
\\	厳禁	
\\	げんきん	
\\	する 
\\	従業員同士の私語は厳禁です。	
\\	厳, 禁	
\\	背が高い	
\\	せがたかい	
\\	い 
\\	こんなに背が高いホビットは見たことがないよ。	
\\	背 
\\	高い 
\\	背, 高	
\\	照り焼き	
\\	てりやき	
\\	「コウイチ、忘れ物は無い?」「無いよ。」「あ、ちょっと待って。あんたのお弁当に照り焼きサーモン入れるの忘れてたわ。」「ええっ!母さん、早くしてよ。」	
\\	照れる 
\\	焼く. 
\\	照, 焼	
\\	冷える	
\\	ひえる	
\\	私は冷え性で、特に足がいつも冷えています。	
\\	冷やす 
\\	(える). 
\\	冷やす. 
\\	冷	
\\	込む	
\\	こむ	
\\	休日だろうが何だろうがディズニーランドはいつでも込んでるよ。	
\\	う 
\\	込	
\\	浮かれる	
\\	うかれる	
\\	たくさんの人がオリンピックに浮かれすぎている。	
\\	"浮く 
\\	(かれる) 
\\	浮く, 
\\	浮	
\\	秘める	
\\	ひめる	
\\	彼は内に秘めるタイプなので、心配しています。	
\\	秘	
\\	傷める	
\\	いためる	
\\	染毛剤は髪の毛を傷める。	
\\	痛い? 
\\	いためる, 
\\	痛い
\\	傷	
\\	染める	
\\	そめる	
\\	髪を何色に染めるか迷っています。	
\\	う 
\\	(そ) 
\\	染	
\\	火照る	
\\	ほてる	
\\	高熱で体が火照っています。	
\\	火 
\\	(ほてる) 
\\	火, 照	
\\	採る	
\\	とる	
\\	僕のお父さんはカブト虫を採る名人です。	
\\	う 
\\	取る, 
\\	とる.	採	
\\	汚れる	
\\	よごれる	
\\	汚れやすいので白いコントローラーはお勧めしません。	
\\	う 
\\	(れる) 
\\	(よご) 
\\	汚	
\\	濡れる	
\\	ぬれる	
\\	やれやれ。雨で尻尾まで濡れちゃったよ。	
\\	私の
\\	はウォータープルーフだから、一緒にお風呂に入っても濡れないの。	
\\	あなた、髪がまだ濡れているわよ。乾かしてから、寝なさい。	
\\	う 
\\	濡	
\\	越す	
\\	こす	
\\	どうすれば
\\	選手のように身長が6.5フィートを越すのでしょうか。今、6フィートなのですが、6.5を越したいのです。	
\\	う 
\\	越える. 
\\	越	
\\	飼う	
\\	かう	
\\	何か爬虫類をペットに飼いたいな。	
\\	か 
\\	飼	
\\	漏らす	
\\	もらす	
\\	もう限界だ!君の行動にはうんざりだよ、フグ。どうしていつも小便を漏らすんだ?ああ、また海出身だからなんて言わないでくれよ。君のお決まりの言い訳はもう聞き飽きているからね。	
\\	漏れる 
\\	(らす) 
\\	漏れる, 
\\	漏	
\\	結婚する	
\\	けっこんする	する 
\\	お前、もう元カノについて考えるのはよした方がいいぞ。特に、サーモンのお父さんに彼女と結婚していいかどうか聞いて、許しも得ちまったんだからな。	
\\	結婚 
\\	結婚 
\\	結, 婚	
\\	今更	
\\	いまさら	
\\	今更ですが、マイ箸を持ち歩き始めました。	
\\	今 
\\	更に
\\	更 
\\	今, 更	
\\	迎える	
\\	むかえる	
\\	うちに仔犬を迎えたいなと思ってるんですが、無駄吠え防止の訓練をしやすいのは何犬でしょうか。	
\\	迎	
\\	延長する	
\\	えんちょうする	する 
\\	ビザを延長するにはどこに行ったらいいのか知ってる?	
\\	延長 
\\	延, 長	
\\	出版する	
\\	しゅっぱんする	する 
\\	彼は有名な小説家だが、時々自費出版することもあるんだよ。	
\\	出版 
\\	出版, 
\\	出, 版	
\\	購読する	
\\	こうどくする	する 
\\	その雑誌は購読する価値がありますよ。	
\\	購読 
\\	購, 読	
\\	捨てる	
\\	すてる	
\\	私の主人は、たとえ一日でも賞味期限が過ぎていたら捨てるんですよ。	
\\	う 
\\	捨	
\\	訳す	
\\	やくす	
\\	この単語を訳すのは難しい。	
\\	う 
\\	やく 
\\	訳	
\\	訳語	
\\	やくご	
\\	日本語の中には、良い英語の訳語が無い時がある。	
\\	訳, 語	
\\	訳者	
\\	やくしゃ	
\\	この本の訳者の名前、何て読むか知ってる?	
\\	訳, 者	
\\	採用	
\\	さいよう	
\\	する 
\\	よくあるように、その会社は女性は顔で採用するそうですよ。	
\\	採, 用	
\\	採決	
\\	さいけつ	
\\	する 
\\	じゃあ、多数決で採決しよう。夕食にてっちり(ふぐ鍋)が食べたい人は?	
\\	採, 決	
\\	欲しい	
\\	ほしい	い 
\\	「欲しいものを買っていいよ。金額の制限は無しだよ。」「まあ!熱でもあるの? 何かあったの?」	
\\	い 
\\	(ほ), 
\\	欲	
\\	欲求	
\\	よっきゅう	
\\	する 
\\	の 
\\	私の妻は、頻繁に欲求不満になります。	
\\	よく 
\\	よっ, 
\\	欲, 求	
\\	温暖	
\\	おんだん	
\\	な 
\\	今年の冬は例年よりも寒くて雪も凄く降ったけど、本当に地球は温暖化してるのかね。	
\\	温, 暖	
\\	暖かい	
\\	あたたかい, あったかい	い 
\\	今日はポカポカと暖かい一日だった。	
\\	い 
\\	温かい, 
\\	あたたかい. 
\\	温かい, 
\\	暖かい 
\\	あったかい 
\\	あたたかい. 
\\	暖	
\\	白旗	
\\	しらはた, はっき, しろはた	
\\	白旗を上げるということは、敵に降伏するということだ。	
\\	しらはた 
\\	しら 
\\	しろ. 
\\	はっき, 
\\	白, 旗	
\\	意欲	
\\	いよく	
\\	金メダル獲得に意欲をみせた。	
\\	意, 欲	
\\	意志	
\\	いし	
\\	自分が意志の弱い人間だってことは自覚してますよ。	
\\	意, 志	
\\	傷	
\\	きず	
\\	ウクレレに傷がついちゃったんです。	
\\	(きず) 
\\	傷	
\\	異常	
\\	いじょう	
\\	な 
\\	異常気象が続いています。	
\\	異, 常	
\\	異状	
\\	いじょう	
\\	子宮がん検診の結果が「異状あり」だったのでびびっています。	
\\	異, 状	
\\	忠告	
\\	ちゅうこく	
\\	する 
\\	の 
\\	彼の忠告を侮るでないぞ、さもなくば命を落としかねんぞ。	
\\	忠, 告	
\\	裏通り	
\\	うらどおり	
\\	レーシングカーは、裏通りを縫うようにして走っていった。	
\\	裏 
\\	通り 
\\	裏, 通	
\\	武装	
\\	ぶそう	
\\	する 
\\	の 
\\	どうやら、武装勢力に襲われたようだ。	
\\	武, 装	
\\	灰	
\\	はい	
\\	鹿児島の桜島の火山の噴火で、近くの街は灰だらけになってしまいました。	
\\	灰	
\\	灰皿	
\\	はいざら	
\\	念のため灰皿を買っておきました。	
\\	灰 
\\	皿, 
\\	灰, 皿	
\\	服装	
\\	ふくそう	
\\	彼のことは好きなんだけど、服装があまり好きじゃないのよね。	
\\	服, 装	
\\	装い	
\\	よそおい	
\\	もうすっかり春の装いだね。	
\\	装	
\\	著者	
\\	ちょしゃ	
\\	なんと私の姉もその本の著者でした。	
\\	著, 者	
\\	裏口	
\\	うらぐち	
\\	の 
\\	彼女の家には隠された裏口があるという彼の話は、信憑性を増した。	
\\	裏 
\\	口 
\\	裏, 口	
\\	裏	
\\	うら	
\\	私の紹介はパンフレットの裏表紙に記されています。	
\\	裏	
\\	裏切り	
\\	うらぎり	
\\	奴らは裏切り者を殺すのにこの施設を利用している。	
\\	裏 
\\	切る
\\	裏, 切	
\\	体操	
\\	たいそう	
\\	する 
\\	の 
\\	この体操は見た目よりもずっときついんですよ。	
\\	体, 操	
\\	操	
\\	みさお	
\\	私の兄は、女の子は結婚するまで操を守るべきだって言うの。	
\\	(みさお), 
\\	操	
\\	熟語	
\\	じゅくご	
\\	昨日その熟語を学んだところです。	
\\	熟, 語	
\\	皇太子	
\\	こうたいし	
\\	皇太子殿下に「うまくいくといいですね」と言った。	
\\	皇, 太, 子	
\\	否定	
\\	ひてい	
\\	する 
\\	の 
\\	その女優はダイエット中であることを否定した。	
\\	否, 定	
\\	砂漠	
\\	さばく	
\\	私は砂漠探索の経験が豊富です。	
\\	砂, 漠	
\\	異義	
\\	いぎ	
\\	この単語に異義はありますか?	
\\	異, 義	
\\	変装	
\\	へんそう	
\\	する 
\\	変装教室にギリギリ間に合いました。	
\\	変, 装	
\\	編集者	
\\	へんしゅうしゃ	
\\	その本の編集者として任命されてから、過労気味です。	
\\	編集. 
\\	編集 
\\	編, 集, 者	
\\	天皇	
\\	てんのう	
\\	天皇陛下に「プラス思考であれば、何でも最後にはうまくいきますよ」なんて助言した奴は一体どこの誰だ?	
\\	皇 
\\	のう 
\\	(のう).	天, 皇	
\\	拡張	
\\	かくちょう	
\\	する 
\\	どうやってそんなにピアスの穴を拡張したの?	
\\	かく. 
\\	拡, 張	
\\	装置	
\\	そうち	
\\	する 
\\	今日は電気制御装置の使い方を学びました。	
\\	装, 置	
\\	除いて	
\\	のぞいて	
\\	彼女は厄介な仕事をうまくやってのけた…あることを除いて、だが。	
\\	"除く 
\\	除く. 
\\	除	
\\	忠実	
\\	ちゅうじつ	な 
\\	私達の子どもはあなたの子と違って、忠実に言うことを聞くんです。	
\\	忠, 実	
\\	小麦粉	
\\	こむぎこ	
\\	うわっ最悪。さっき買った高級小麦粉、電車に忘れちゃった。	
\\	小 
\\	(子), 
\\	こ. 
\\	小, 麦, 粉	
\\	貨物船	
\\	かもつせん	
\\	私の貨物船の中で騒ぐんじゃない!	
\\	物 
\\	もつ 
\\	持つ, 
\\	貨, 物, 船	
\\	果糖	
\\	かとう	
\\	ブドウ糖果糖液糖は体に悪いって言うじゃない?	
\\	果, 糖	
\\	諸君	
\\	しょくん	
\\	諸君、
\\	オフィスで独身の 
\\	人がみんな集まるのってこれがたぶん最後だってこと、気がついていたかい?	
\\	諸, 君	
\\	華々しい	
\\	はなばなしい	い 
\\	その左利きの投手は、大リーグで華々しいデビューを飾りました。	
\\	"華やか 
\\	華やか. 
\\	はな 
\\	ばな, 
\\	はなばなしい.	華, 々	
\\	墓場	
\\	はかば	
\\	墓場の隣に住んだら、お化けがドアをノックすることもあるかもしれない。	
\\	場 
\\	墓 
\\	はか 
\\	(はか) 
\\	墓, 場	
\\	灰色	
\\	はいいろ	
\\	の 
\\	この灰色のお守りは、悪い気を追い払ってくれます。	
\\	灰, 色	
\\	敬語	
\\	けいご	
\\	あの女が私に敬語を使わないなんて、本当に屈辱的だわ。	
\\	敬, 語	
\\	家賃	
\\	やちん	
\\	安月給だけど、この家賃だったら何とか暮らしていけそうかな。	
\\	昨日、アパートの契約こう新したんだけど、去年から家賃が500ドルも上がったってどういうこと?一しゅん引っ越そうかと思ったよ。	
\\	フグが私にどんな扱いをしてたか考えてみると、彼がどれだけ最低男だったかが分かるわ。私は彼の家賃から電話代から食費から何から何まで払ってあげていたのに、あの男は他の女と浮気してたのよ!	
\\	家 
\\	や, 
\\	(や), 
\\	家, 賃	
\\	皇族	
\\	こうぞく	
\\	時々電車は、皇族専用の駅に特別に止まることがあります。	
\\	皇, 族	
\\	皇室	
\\	こうしつ	
\\	皇室はここ最近、かなり大変な状況に陥っています。	
\\	皇, 室	
\\	物真似	
\\	ものまね	
\\	する 
\\	「いつも何してるの?」「毎日一日中物真似の練習をしてるよ。」	
\\	"真似 
\\	物 
\\	真似 
\\	物, 真, 似	
\\	地蔵	
\\	じぞう	
\\	お地蔵さんの頭がへこんでるよ。	
\\	地, 蔵	
\\	散らし	
\\	ちらし	
\\	スーパーの散らし、ここに置いてなかった?	
\\	(らし)? 
\\	散る, 
\\	散	
\\	肺	
\\	はい	
\\	肺気胸は日本ではイケメン病としても知られている。	
\\	肺	
\\	肺がん	
\\	はいがん	
\\	君の辛さはすごく分かるよ。僕も肺がんなんだ。	
\\	"肺 
\\	がん 
\\	(がん) 
\\	肺	
\\	肺病	
\\	はいびょう	
\\	の 
\\	肺病の治療のために、三日後に入院するんです。	
\\	肺, 病	
\\	雑誌	
\\	ざっし	
\\	この雑誌の表紙のインターバル撮影された写真すごい好きかも。	
\\	雑, 誌	
\\	筋肉	
\\	きんにく	
\\	の 
\\	毎日七時に仕事を終え、筋肉を鍛えるために帰り道にジムに寄ります。	
\\	筋, 肉	
\\	否	
\\	いな, いいえ, いえ, いや	
\\	そのビデオを見るか否か、まだ決めきれていない。	
\\	いな. 
\\	いいえ 
\\	いや, 
\\	否	
\\	未熟	
\\	みじゅく	
\\	な 
\\	の 
\\	俺も昔は未熟で好き勝手してたよ。	
\\	未, 熟	
\\	開閉	
\\	かいへい	
\\	する 
\\	開閉時刻の変更は、小さな混乱と誤解を招いた。	
\\	開, 閉	
\\	操作	
\\	そうさ	
\\	他人にスマフォの遠隔操作を可能にさせるバックドアを発見しました。	
\\	操, 作	
\\	異性	
\\	いせい	
\\	の 
\\	今日は異性といつもより長く話しをしました。	
\\	異, 性	
\\	納入	
\\	のうにゅう	
\\	する 
\\	授業料の納入期限はどこに書いてありますか。	
\\	急に大きいお金を準備できなかったから、分割納入制度を利用しました。	
\\	五日以内にその製品を納入して頂きたいのですが、可能でしょうか。	
\\	納, 入	
\\	宣言	
\\	せんげん	
\\	する 
\\	彼のお気に入りのアイドルが、年末に引退するって宣言したから、彼も他のファンもショックを受けちゃってるのよ。	
\\	宣, 言	
\\	宣伝	
\\	せんでん	
\\	する 
\\	の広告宣伝費はいくらぐらいですか。	
\\	宣, 伝	
\\	冷蔵庫	
\\	れいぞうこ	
\\	なんでタトゥーで冷蔵庫なんて彫ったの?	
\\	冷, 蔵, 庫	
\\	否決	
\\	ひけつ	
\\	する 
\\	今朝新聞で上院は議案を否決したって読んだよ。	
\\	否, 決	
\\	著しい	
\\	いちじるしい	い 
\\	親馬鹿かもしれませんが、娘の英語力の上達が著しい気がするんですよね。	
\\	(いちじる) 
\\	著	
\\	賃貸	
\\	ちんたい	
\\	する 
\\	の 
\\	友人が賃貸のアパートを探してるんですが、どこかいい物件を知りませんかね。	
\\	貸 
\\	たい 
\\	(たい) 
\\	賃, 貸	
\\	尊敬	
\\	そんけい	
\\	する 
\\	彼の事、知れば知る程尊敬するのよね。	
\\	尊, 敬	
\\	尊い	
\\	とうとい, たっとい	い 
\\	息子に、「命ってどれくらい尊いの?」なんて聞かれちゃってさ。言葉に窮しちゃったよ。	
\\	い 
\\	尊	
\\	成熟	
\\	せいじゅく	
\\	する 
\\	の 
\\	成熟した女性の脳と成熟した男性の脳の相違点について、論文を書いています。	
\\	成, 熟	
\\	泥棒	
\\	どろぼう	
\\	する 
\\	泥棒になるためのトレーニングは、ブートキャンプ並にハードです。	
\\	泥 
\\	棒 
\\	泥, 棒	
\\	噂	
\\	うわさ	
\\	する 
\\	の 
\\	サイラスというあの男は、実はロボットだという噂がある。	
\\	あなたって本当、噂をするのが好きね。	
\\	ネットで噂の「加賀屋」っていう居酒屋に行かない?	
\\	噂	
\\	砂	
\\	すな	
\\	小さい頃、あなたとビーチで作った砂のお城を思い出します。	
\\	(すな) 
\\	砂	
\\	簡易	
\\	かんい	
\\	な 
\\	の 
\\	もしうちに泊まりたいなら、簡易ベッドがあるよ。	
\\	簡, 易	
\\	簡単	
\\	かんたん	
\\	な 
\\	はノートパソコンよりも持ち運びが簡単です。	
\\	簡, 単	
\\	蒸気	
\\	じょうき	
\\	珈琲メーカーからの蒸気でやけどをしちゃった。	
\\	蒸, 気	
\\	収納	
\\	しゅうのう	
\\	する 
\\	私の部屋にはあまり収納スペースがないので、ちょっとした工夫が必要です。	
\\	収, 納	
\\	閉店	
\\	へいてん	
\\	する 
\\	「営業時間は何時から何時までですか?」「午前10時開店で、午後8時閉店となります。」	
\\	閉, 店	
\\	蔵	
\\	くら	
\\	今、蔵の掃除をしているんです。	
\\	(倉) 
\\	(くら).	蔵	
\\	砂糖	
\\	さとう	
\\	ブラウンシュガーは日本語では粗糖と呼ばれ、黒砂糖とは区別されます。	
\\	砂, 糖	
\\	無糖	
\\	むとう	
\\	の 
\\	さっきも言ったけど、珈琲は無糖派なの。	
\\	無, 糖	
\\	筋	
\\	すじ	
\\	ジョギングをしてる時に足の筋を痛めちゃってさ。	
\\	今日は久々に牛筋のに込みでも作ってみますか。	
\\	ほら、肉を焼く前にちゃんと筋を切っておかなかったから焼いたら縮んじゃったでしょ?	
\\	筋	
\\	窓口	
\\	まどぐち	
\\	窓口の仕事ではあったが、彼女はその面接をとてもうまくこなした。	
\\	窓 
\\	口.	窓, 口	
\\	裏切る	
\\	うらぎる	
\\	彼女に裏切られて落ち込んでいるんです。	
\\	裏 
\\	切る 
\\	裏, 切	
\\	蒸れる	
\\	むれる	
\\	この靴、可愛いんだけど、足が蒸れて臭うんだよね。	
\\	(む) 
\\	蒸	
\\	掃除する	
\\	そうじする	する 
\\	「今朝は4時から起きてるよ。」「嘘でしょ?いったい何してるの?」「えっと…家を大掃除してるの。」	
\\	掃除 
\\	掃除 
\\	掃, 除	
\\	閉める	
\\	しめる	
\\	旦那がペットボトルのキャップを閉める時、キツく締めすぎるのが嫌です。	
\\	う 
\\	(し) 
\\	閉	
\\	垂らす	
\\	たらす	
\\	あの娘が涎を垂らしながら一服しているブルドッグの絵を描いたんだけど、それがすっごく可愛くってさ。	
\\	う 
\\	(た) 
\\	垂	
\\	異なる	
\\	ことなる	
\\	それぞれの薬には、異なる副作用があります。	
\\	う 
\\	(こと), 
\\	異	
\\	乾かす	
\\	かわかす	
\\	あなたを愛しています。僕にはあなたが必要なんです。太陽さん、どうか僕の濡れたシャツを今すぐ乾かしてくれませんか?お願いします!!!	
\\	"乾く 
\\	乾く.	乾	
\\	編集する	
\\	へんしゅうする	する 
\\	今オフィスで資料の要約を編集しているところよ。	
\\	編集 
\\	編, 集	
\\	拡がる	
\\	ひろがる	
\\	薄暗い部屋の床に、血が静かに拡がっていった。	
\\	う 
\\	広い, 
\\	広い, 
\\	拡がる 
\\	ひろがる.	拡	
\\	散歩する	
\\	さんぽする	する 
\\	妻が、今日は早目に仕事を切り上げて犬の散歩をしてほしいと言っているのですが、よろしいでしょうか。	
\\	"散歩 
\\	散歩 
\\	散, 歩	
\\	盛る	
\\	もる	
\\	日本人の中には、不幸や悪霊を追い払うために玄関先に塩を盛る人がいます。	
\\	う 
\\	(もる), 
\\	盛	
\\	暮らす	
\\	くらす	
\\	彼は今は幸せに暮らしているが、ああ見えて苦難を経験してるんだぜ。	
\\	う 
\\	(く). 
\\	暮	
\\	操る	
\\	あやつる	
\\	クマが裏でみんなを操っている黒幕だっていう噂があるんだ。	
\\	う 
\\	(あやつ) 
\\	操	
\\	熟れる	
\\	うれる	
\\	そろそろ桃が熟れる頃だ。	
\\	う 
\\	(う), 
\\	熟	
\\	散らかす	
\\	ちらかす	
\\	私の部屋には脱ぎっぱなしの服が床に散らかっています。掃除しても、次の日にはまたすぐ散らかしちゃうんだよね。	
\\	"散る 
\\	(らかす) 
\\	散 
\\	散る. 
\\	散	
\\	離す	
\\	はなす	
\\	ごめん。いま手が離せないんだ。後でかけ直してもいいかな?	
\\	離れる 
\\	離す, 
\\	(す) 
\\	離れる.	離	
\\	似合う	
\\	にあう	
\\	そのワンピには絶対デニムのジャケットが似合うと思うよ。	
\\	似る 
\\	合う, 
\\	似, 合	
\\	納める	
\\	おさめる	
\\	血税の使い道が理にかなってさえいれば、政府に税金を納めるのは私は一向に構いませんよ。	
\\	う 
\\	納	
\\	驚かす	
\\	おどろかす	
\\	23歳の新入社員は82歳の
\\	と結婚して、一緒に仕事をする人全員を驚かした。	
\\	驚く 
\\	驚く. 
\\	驚	
\\	野暮	
\\	やぼ	
\\	うちの学校は本当に野暮だから、生徒たちのソーシャルメディアの利用を禁止してるんだよ。	
\\	野, 暮	
\\	掃除機	
\\	そうじき	
\\	科学者は、宇宙はある巨大な掃除機だと言うでしょ?だったら、それをカーペットを掃除するのに使ったらいいんじゃない??	
\\	(掃除) 
\\	掃, 除, 機	
\\	漠然	
\\	ばくぜん	
\\	漠然とした不安を感じながら、私は川の流れを見詰めていました。	
\\	漠, 然	
\\	諸〜	
\\	しょ	
\\	各出席者の方に、まず諸書類をお配り致します。	
\\	諸	
\\	垂直	
\\	すいちょく	
\\	な 
\\	の 
\\	ロケットは垂直に発射した。	
\\	垂, 直	
\\	窓	
\\	まど	
\\	この窓は何で作られているんですか。	
\\	窓	
\\	著作	
\\	ちょさく	
\\	する 
\\	彼女は少なくとも五十は著作があります。	
\\	著, 作	
\\	拡大	
\\	かくだい	
\\	する 
\\	写真を拡大した時に、彼氏の鼻毛が出てることを発見しちゃったんだよね。どん引きだったわ。	
\\	拡, 大	
\\	純毛	
\\	じゅんもう	
\\	新しい純毛のセーター、チクチクして痒いんだよね。	
\\	純, 毛	
\\	沿線	
\\	えんせん	
\\	の 
\\	私の弟は、京王沿線に住んでいます。	
\\	沿, 線	
\\	承認	
\\	しょうにん	
\\	する 
\\	委員会から承認を得られる自信はあまりありません。	
\\	承, 認	
\\	歓迎	
\\	かんげい	
\\	する 
\\	の 
\\	カウンターの上に「チップ大歓迎」と書かれた箱がありましたよ。	
\\	歓, 迎	
\\	小豆	
\\	あずき	
\\	餡こは小豆から作られているが、けっこう糖分が高い。	
\\	(あずき). 
\\	小, 豆	
\\	幕府	
\\	ばくふ	
\\	幕府が俺の家族を皆殺しにした。	
\\	幕 
\\	(ばく). 
\\	幕, 府	
\\	豆	
\\	まめ	
\\	お豆さん、お豆さん、魔法の野菜。たくさん食べれば食べるほど、たくさん屁が出るぷっぷぷぷ。たくさんぷっぷぷ屁が出れば、気持ちが良くなるウッフフフ。だから毎食お豆を食べなさい。	
\\	(まめ).
\\	豆	
\\	牛丼	
\\	ぎゅうどん	
\\	「この牛丼は、私のおごりね。」「わあ、ありがとう。じゃあ、お言葉に甘えます。」	
\\	牛, 丼	
\\	聖書	
\\	せいしょ	
\\	の 
\\	聖書を半分ぐらい読み終わりました。	
\\	聖, 書	
\\	紅茶	
\\	こうちゃ	
\\	お紅茶はどのようにおいれ致しましょうか。	
\\	紅, 茶	
\\	血液型	
\\	けつえきがた	
\\	私は血液型が
\\	型の人が苦手です。	
\\	血液 
\\	型 
\\	血, 液, 型	
\\	枝	
\\	えだ	
\\	枝は真っ赤な林檎でたわんでいた。	
\\	だ. 
\\	だ.	枝	
\\	爪	
\\	つめ	
\\	彼女の右の小指の爪の長さはなんと六メートルもあります。	
\\	爪 
\\	つめ.	爪	
\\	沿岸	
\\	えんがん	
\\	の 
\\	今度の日曜日にヨットで沿岸を一緒に帆走しないかって彼に誘われたんだ。	
\\	沿, 岸	
\\	沿海	
\\	えんかい	
\\	の 
\\	沿海部での生活について、月一でコラムを書いてみませんか。	
\\	沿, 海	
\\	指揮	
\\	しき	
\\	する 
\\	の 
\\	アメリカ軍を指揮している司令官に一度会ってみたいな。	
\\	指, 揮	
\\	通勤	
\\	つうきん	
\\	する 
\\	通勤は時間を浪費するので、私は職場の近くに住む方がいいですね。	
\\	通, 勤	
\\	承知	
\\	しょうち	
\\	する 
\\	わしの娘がお前と結婚することは承知できん。	
\\	承, 知	
\\	腐食	
\\	ふしょく	
\\	する 
\\	ステンレスは腐食作用への耐久性があるんじゃなかったっけ?	
\\	腐, 食	
\\	人込み	
\\	ひとごみ	
\\	あいつら、人込みに紛れて俺に浣腸しようとしてやがるんだ。お前らは俺を守ってくれるよなあ?	
\\	込む 
\\	人 
\\	込む 
\\	こ 
\\	込み 
\\	ご 
\\	人, 込	
\\	幕	
\\	まく	
\\	その劇では、女主人公が死んで幕が下りました。	
\\	幕	
\\	損害	
\\	そんがい	
\\	する 
\\	トラックの衝突で積荷がダメージを受けたが、小さな損害で済んだ。	
\\	損, 害	
\\	腐敗	
\\	ふはい	
\\	する 
\\	の 
\\	多くの人々が腐敗した政治組織にうんざりしています。	
\\	腐, 敗	
\\	紅	
\\	くれない, べに	
\\	その紅色の花の名前は何ですか?	
\\	(べに) 
\\	紅 
\\	ない (くれない). 
\\	紅	
\\	損	
\\	そん	
\\	な 
\\	ポーカーで五百ドル損をした。	
\\	損	
\\	使い捨て	
\\	つかいすて	
\\	の 
\\	私にとって、使い捨てカイロは冬には無くてはならないものです。	
\\	使う 
\\	捨てる, 
\\	使, 捨	
\\	奴	
\\	やつ	
\\	の 
\\	宝くじが当たってから、奴は数週間最高の気分を味わったんだ。	
\\	(やつ) 
\\	奴	
\\	震源地	
\\	しんげんち	
\\	震源地に近いところに住んでる友達が心配だよ。	
\\	震, 源, 地	
\\	推定	
\\	すいてい	
\\	する 
\\	の 
\\	そのサバンの男は群衆の規模を534人と推定したのだが、実際の数とピッタリ当っていた。	
\\	推, 定	
\\	言い訳	
\\	いいわけ	
\\	する 
\\	分かってもらえることを願って、言い訳を書いた
\\	メールを送りました。	
\\	言う 
\\	訳 
\\	言, 訳	
\\	常勤	
\\	じょうきん	
\\	する 
\\	子どもがもうちょっと大きくなるまでは、常勤でのお仕事はちょっと無理かな。	
\\	常, 勤	
\\	源氏	
\\	げんじ	
\\	どうして娼婦やホステス、キャバ嬢、芸者さんなどが使う偽名を「源氏名」って言うんですか?	
\\	源, 氏	
\\	降車	
\\	こうしゃ	
\\	する 
\\	「あのジェットコースター、どうだった?」「すっごく良かったから、降車した瞬間にゲロっちゃったよ。」	
\\	降, 車	
\\	眼鏡	
\\	めがね, がんきょう	
\\	眼鏡を拭いた方がいいですよ。	
\\	眼 
\\	眼. 
\\	鏡 
\\	金 
\\	がんきょう, 
\\	めがね!	眼, 鏡	
\\	祖父母	
\\	そふぼ	
\\	私の祖父母はまだ小児科医にかかっています。	
\\	祖父 
\\	祖母 
\\	祖父 
\\	祖母, 
\\	祖, 父, 母	
\\	心臓	
\\	しんぞう	
\\	電話恐怖症みたいなのがあって、電話で話をする時に心臓がバクバクするんです。	
\\	心, 臓	
\\	大損	
\\	おおぞん	
\\	する 
\\	大儲けするつもりでパチンコに行ったが、結局大損してしまった。	
\\	おお 
\\	大 
\\	損. 
\\	そん 
\\	ぞん 
\\	大, 損	
\\	以降	
\\	いこう	
\\	返事は来週以降になります。	
\\	八月三日以降にもう一度ご連絡していただけますか?	
\\	夜九時以降の外出は父に禁止されているんです。	
\\	以, 降	
\\	神聖	
\\	しんせい	
\\	な 
\\	村人たちは、自然に対し神聖な敬意を持っています。	
\\	神, 聖	
\\	聖日	
\\	せいじつ	
\\	今日はユダヤ教の聖日にあたります。	
\\	聖, 日	
\\	聖地	
\\	せいち	
\\	この道は聖地に続いています。	
\\	聖, 地	
\\	不純	
\\	ふじゅん	
\\	な 
\\	私は不純な動機から日本語を勉強し始めました。	
\\	不, 純	
\\	内臓	
\\	ないぞう	
\\	の 
\\	ご近所さんのガレージで鹿の内臓を取り出すんです。	
\\	内.	
\\	内, 臓	
\\	痩身	
\\	そうしん	
\\	の 
\\	そのエステは痩身を専門にしているんだって。	
\\	一年くらいマッサージに通ったんだけど、痩身の効果はなかったよ。	
\\	防犯カメラに写っていたのは、痩身の男性でした。	
\\	痩, 身	
\\	縦横	
\\	たてよこ, じゅうおう	
\\	壁の縦横の長さを測りたいんだけどちょっと手伝ってくれない。	
\\	縦, 横	
\\	久しぶり	
\\	ひさしぶり	な 
\\	の 
\\	久しぶりですね!	
\\	久しぶりのビールは、なみだが出るほどおいしかった。	
\\	久しぶり!とあったことのない人にいわれた。	
\\	久 
\\	しぶり. 
\\	久しい. 
\\	久	
\\	肺臓	
\\	はいぞう	
\\	私の祖父は肺臓炎で亡くなりました。	
\\	肺, 臓	
\\	磁石	
\\	じしゃく	
\\	の 
\\	お土産に可愛い磁石をいくつか買ったわよ。	
\\	磁 
\\	しゃく 
\\	石 
\\	(しゃく) 
\\	磁, 石	
\\	磁場	
\\	じば	
\\	地球にはとても強い磁場があります。	
\\	磁, 場	
\\	磁気	
\\	じき	
\\	磁気カードと磁石を一緒にしない方がいいよ。	
\\	磁, 気	
\\	迎え	
\\	むかえ	
\\	駅まで迎えをよこすよ。	
\\	迎	
\\	誤算	
\\	ごさん	
\\	する 
\\	彼女は英語がペラペラだと思ったが、とんだ誤算だった。	
\\	誤, 算	
\\	誤解	
\\	ごかい	
\\	する 
\\	君は誤解しているよ。あれは俺の姉貴だよ。	
\\	誤, 解	
\\	誤用	
\\	ごよう	
\\	する 
\\	薬の誤用は命に関わることがあります。	
\\	誤, 用	
\\	爪切り	
\\	つめきり	
\\	犬の爪を爪切りで切ってあげようとしてたんですけど、つい携帯に気を取られちゃいました。	
\\	爪 
\\	切る.	爪, 切	
\\	納豆	
\\	なっとう	
\\	誤解しないでね。日本料理は大好きなんだけど、納豆だけはどうしても食べられないの。	
\\	豆 
\\	納 
\\	のう 
\\	なっ. 
\\	(なっとう) 
\\	納, 豆	
\\	興味がない	
\\	きょうみがない	
\\	俺は顧客と個人的に仲良くなることには興味がないんだ。	
\\	興味 
\\	(がない) 
\\	興味.	興, 味	
\\	兄貴	
\\	あにき	
\\	お前の兄貴はイケてるけど、俺の兄貴はダサイからな。	
\\	兄 
\\	貴. 
\\	兄, 貴	
\\	源	
\\	みなもと	
\\	音楽が私の日々の生活のパワーの源です。	
\\	(皆) 
\\	(元). 
\\	皆元!	源	
\\	飼い主	
\\	かいぬし	
\\	もし犬の飼い主が水疱瘡になったら、その犬も水疱瘡になる可能性はあるんでしょうか。	
\\	飼う 
\\	(飼う), 
\\	飼う 
\\	主 
\\	飼, 主	
\\	歓楽街	
\\	かんらくがい	
\\	歓楽街のツアーには何が含まれていますか。	
\\	歓, 楽, 街	
\\	複数形	
\\	ふくすうけい	
\\	日本語では英語に比べて単語を複数形にすることは少ないです。	
\\	複数 
\\	形 
\\	複数 
\\	複, 数, 形	
\\	芋	
\\	いも	
\\	このお芋、レンジでチンしてくれない?	
\\	芋	
\\	電源	
\\	でんげん	
\\	出かける時は、完全に電源を切るために全部のコンセントを抜きます。	
\\	電, 源	
\\	貴重	
\\	きちょう	
\\	な 
\\	貴重なご意見を有難うございます。	
\\	重 
\\	(ちょう) 
\\	貴, 重	
\\	純粋	
\\	じゅんすい	な 
\\	の 
\\	純粋に金銭的な理由だけで、夫とまだ一緒にいるんです。	
\\	純, 粋	
\\	放射	
\\	ほうしゃ	
\\	する 
\\	放射冷却のせいで今朝起きた時はものすごい冷え込みだったよ。	
\\	放, 射	
\\	推理	
\\	すいり	
\\	する 
\\	の 
\\	日付の書き方から、その脅迫状はアメリカ人によって書かれたものだと推理しました。	
\\	推, 理	
\\	推薦	
\\	すいせん	
\\	する 
\\	の 
\\	あなたを、優秀な屁こき野郎として、屁こき大会に推薦しておきましたよ。	
\\	推, 薦	
\\	縮小	
\\	しゅくしょう	
\\	する 
\\	の 
\\	うちの会社、他社との競争に勝つために、利鞘を縮小することを検討しているんだ。	
\\	縮, 小	
\\	反射	
\\	はんしゃ	
\\	する 
\\	雪野原は、巨大な鏡のように、冬の太陽の光をキラキラと反射します。	
\\	反, 射	
\\	丼	
\\	どんぶり	
\\	朝から丼なんて食べてるの?あんたって、本当に朝型人間ね。	
\\	どん. 
\\	どんぶり. 
\\	ぶり 
\\	(ぶり) 
\\	丼	
\\	焼き芋	
\\	やきいも	
\\	私の彼氏、本当にムカつくんだけど!私がすっごくすっごく楽しみしてた焼き芋、食べちゃったの。	
\\	焼く 
\\	芋 
\\	焼, 芋	
\\	自薦	
\\	じせん	
\\	する 
\\	今回のモデルへの応募について、自薦・他薦は問いません。	
\\	自, 薦	
\\	深刻	
\\	しんこく	
\\	な 
\\	甲状腺異常のある日本の子どもたちの数が増えているのは、深刻な問題です。	
\\	深, 刻	
\\	注射	
\\	ちゅうしゃ	
\\	する 
\\	腰が痛かったんですが、お医者さんが痛み止めの注射を打ってくれました。	
\\	注, 射	
\\	単純	
\\	たんじゅん	
\\	な 
\\	日本語で単純な文章なら作れますが、難しいのはまだ無理です。	
\\	単, 純	
\\	奴隷	
\\	どれい	
\\	の 
\\	そう取り乱すなよ。心配ないって。俺らはまだ奴隷だけど、そのうち何とかなるよ。	
\\	奴, 隷	
\\	染み	
\\	しみ	
\\	この染みなんとか取れないかなあ。	
\\	染. 
\\	(し) 
\\	染	
\\	粋	
\\	いき	
\\	な 
\\	自分で粋な会話をしてるって思ってるところがムカツクんだよね。	
\\	いきます (いき) 
\\	粋	
\\	字幕	
\\	じまく	
\\	字幕のせいで気が散ることもあります。	
\\	字, 幕	
\\	飛び込み自殺	
\\	とびこみじさつ	
\\	あいつ、飛び込み自殺をするために家を出る前、俺のプレイステーションをあいつのテレビに繋いだんだよな。何か意味があったのかな?	
\\	(自殺), 
\\	飛ぶ, 込む, 
\\	自殺. 
\\	飛, 込, 自, 殺	
\\	降参	
\\	こうさん	
\\	する 
\\	私はまだ降参していません。	
\\	降, 参	
\\	貴族	
\\	きぞく	
\\	の 
\\	ああ、もし私が貴族の家柄に生まれていたら、彼女と話をすることも可能だったかもしれないのに。	
\\	貴, 族	
\\	時刻表	
\\	じこくひょう	
\\	通勤・通学者にとって、電車の時刻表は絶対に必要なものです。	
\\	時, 刻, 表	
\\	賛成する	
\\	さんせいする	する 
\\	それがどんな汚い手を使ってでも勝たなきゃいけない試合だってことには賛成するよ。	
\\	賛成 
\\	賛, 成	
\\	申し込む	
\\	もうしこむ	
\\	私は麻疹に免疫がないので、病院に麻疹の予防接種を受けるための申込をしました。	
\\	申す 
\\	込む, 
\\	申, 込	
\\	傷む	
\\	いたむ	
\\	このアボカド、昨日買ったところなのにもう傷んでるよ。	
\\	傷める 
\\	傷める. 
\\	傷	
\\	染まる	
\\	そまる	
\\	子供たちがこの悪い環境に染まってしまうことを心配してるのよ。	
\\	染める 
\\	(まる) 
\\	染める.	染	
\\	勤める	
\\	つとめる	
\\	娘が小学校に上がったので、私は図書館に勤め始めました。	
\\	(務める). 
\\	勤	
\\	刻む	
\\	きざむ	
\\	葱を刻むのは面倒くさいよ。	
\\	う 
\\	(きざ) 
\\	刻	
\\	発揮する	
\\	はっきする	する 
\\	問題解決にリーダーシップを一番発揮していたのは誰ですか。	
\\	はつ 
\\	はっ, 
\\	発, 揮	
\\	降る	
\\	ふる	
\\	お昼過ぎ、にわか雨が降る中、私は愛犬の散歩をしていました。	
\\	う 
\\	(ふる) 
\\	降	
\\	書き込む	
\\	かきこむ	
\\	急いで申込書に書き込んだ方がいいよ。あと十分で事務所は閉まっちゃうんだから。	
\\	書く 
\\	込む 
\\	書, 込	
\\	傷つく	
\\	きずつく	
\\	おばちゃんって呼ばれる年齢かなあ。傷つくなあ。	
\\	傷 
\\	つく
\\	傷.	傷	
\\	痩せる	
\\	やせる	
\\	「痩せたい」と、言い始めてから二年が経ちます。	
\\	私の村にいる野良犬のほとんどは痩せていますよ。	
\\	会話が途切れて気まずくなった時は、とりあえず「最近、痩せた?」と言うことにしている。	
\\	う 
\\	(や). 
\\	痩	
\\	損なう	
\\	そこなう	
\\	働きすぎてあんたが健康を損なうんじゃないかって、心配してるのよ。	
\\	(そこ) 
\\	損	
\\	拝見する	
\\	はいけんする	する 
\\	あなたが描いた日陰に座っている少女の絵を拝見しました。	
\\	拝, 見	
\\	承る	
\\	うけたまわる	
\\	ご伝言を承りましょうか。	
\\	(受けた) 
\\	(回る) 
\\	承	
\\	傷つける	
\\	きずつける	
\\	コンタクトレンズを傷つけちゃったかも。	
\\	(つける). 
\\	傷. 
\\	傷	
\\	込める	
\\	こめる	
\\	ショットガンに弾を込める時間がなかったんだ。	
\\	込む 
\\	(める) 
\\	込む. 
\\	込	
\\	発射する	
\\	はっしゃする	する 
\\	ミサイルは予定通り発射された。	
\\	発, 射	
\\	汚す	
\\	よごす	
\\	綺麗な涎かけが無くなっちゃった。うちの赤ちゃんが全部汚しちゃって。	
\\	"汚れる 
\\	(す) 
\\	汚れる, 
\\	汚	
\\	引っ越す	
\\	ひっこす	
\\	「パパ、フグと私、結婚して日本に引っ越すつもりよ。」「俺の目の黒いうちはそんなことはさせないぞ。」	
\\	"越す 
\\	引く 
\\	越す 
\\	っ.	引, 越	
\\	追い越す	
\\	おいこす	
\\	他の自転車走者を追い越す時、いつも挨拶代わりに手を振ります。	
\\	"越す 
\\	追 
\\	追う 
\\	越す 
\\	追, 越	
\\	吐く	
\\	はく	
\\	「遅れてごめんなさい。朝、少し気分が悪くて吐いてしまって。」「大丈夫だよ。遅れても来ないよりはましだよ。」	
\\	う 
\\	吐	
\\	薦める	
\\	すすめる	
\\	この番組を彼に薦められたんだけど、アニメを観るのはあまり好きじゃないんだよね。	
\\	う 
\\	(すす). 
\\	薦	
\\	腐る	
\\	くさる	
\\	牛乳は腐りやすいんだから冷蔵庫の外に出しっ放しにしないでね。	
\\	う 
\\	臭い, 
\\	くさい.
\\	腐	
\\	沿う	
\\	そう	
\\	川の両岸に沿って、ずうっと、美しい桜並木があるんです。	
\\	う 
\\	(そう) 
\\	沿	
\\	縮まる	
\\	ちぢまる, ちじまる	
\\	雷がゴロゴロ鳴っている間、うちの犬は怖がって縮まっていました。	
\\	う 
\\	(ちぢ), 
\\	縮	
\\	詰め込む	
\\	つめこむ	
\\	なんとか洋服を全部スーツケースに詰め込めました。	
\\	詰める 
\\	込む 
\\	詰, 込	
\\	全損	
\\	ぜんそん	
\\	車が全損してしまった場合、修理代は保険で全額支払われないことが多いです。	
\\	全, 損	
\\	投げ捨てる	
\\	なげすてる	
\\	今朝、六時に目覚ましがなった時に、その時計を投げ捨ててしまいました。	
\\	投げる 
\\	捨てる, 
\\	投, 捨	
\\	拝む	
\\	おがむ	
\\	私の家族は、毎朝仏壇の前に集まって仏様に拝みます。	
\\	う 
\\	(おが), 
\\	拝	
\\	誤字	
\\	ごじ	
\\	とりあえず今は誤字脱字を見つけてくれればそれでいいから。	
\\	誤, 字	
\\	貴い	
\\	とうとい, たっとい	い 
\\	命ってのはベーコンと同じくらい貴いものなんだよ。	
\\	い 
\\	(とうと). 
\\	とう) 
\\	と). 
\\	尊い, 
\\	貴	
\\	縦	
\\	たて	
\\	縦100
\\	横160
\\	の窓に合う良いカーテンを探しています。	
\\	縦	
\\	縦書	
\\	たてがき	
\\	文章を縦書にしたいんです。	
\\	縦 
\\	かき 
\\	がき), 
\\	き 
\\	縦, 書	
\\	じゃが芋	
\\	じゃがいも	
\\	「じゃが芋の皮、剥ける?」「心配しないで。朝飯前だよ。」	
\\	じゃが 
\\	芋	
\\	黒幕	
\\	くろまく	
\\	の 
\\	結局、あんたが黒幕だったんだな。どうりてあの女を殺した時手慣れていたと思ったよ。	
\\	黒 
\\	幕, 
\\	幕 
\\	黒, 幕	
\\	貴様	
\\	きさま	
\\	の 
\\	貴様がうるさかったせいで、昨日全然寝れなかったじゃねぇか。	
\\	きさま 
\\	貴 
\\	様.	貴, 様	
\\	交互	
\\	こうご	
\\	の 
\\	彼女は二種類の味のアイスクリームを交互に舐めた。	
\\	交, 互	
\\	破産	
\\	はさん	
\\	する 
\\	彼の祖父は江戸時代の破産した百姓だという噂だ。	
\\	産 
\\	(倒産). 
\\	破, 産	
\\	兵舎	
\\	へいしゃ	
\\	その兵舎には異臭が漂っており、息をするのもやっとだった。	
\\	兵, 舎	
\\	熊	
\\	くま	
\\	ありえない!今温泉に浸かっている熊を見たんだけど!	
\\	熊	
\\	遅刻	
\\	ちこく	
\\	する 
\\	の 
\\	三日連続で遅刻してしまいました。	
\\	遅, 刻	
\\	彼氏	
\\	かれし	
\\	の 
\\	ねぇ、私、あなたと喋るだけで、いい気分になるわ。あなたって、私にとって本当に理想の彼氏だわ。	
\\	彼, 氏	
\\	三つ編み	
\\	みつあみ	
\\	おいおい、頼むよ。明らかにお前たちの相性全然よくなかっただろ!絶対にあの三つ編みの女に電話なんてしない方がいいって!	
\\	"編む 
\\	み 
\\	三つ 
\\	つ 
\\	編む 
\\	みっつあみ, 
\\	みつあみ. 
\\	三, 編	
\\	青銅	
\\	せいどう	
\\	青銅でコウイチの像を鋳造するっていうのはどうよ。	
\\	青, 銅	
\\	杉	
\\	すぎ	
\\	うちのお店では、国産杉を使った家具をお手頃なお値段で提供しております。	
\\	杉.	杉	
\\	推測	
\\	すいそく	
\\	する 
\\	の 
\\	私の推測では、彼は宝くじが当たって、バハマに引っ越した。	
\\	推, 測	
\\	納得	
\\	なっとく	
\\	する 
\\	どうせお前は娘を金ずくで納得させたんだろう?	
\\	納. 
\\	(なっ), 
\\	納, 得	
\\	銅山	
\\	どうざん	
\\	日本では、銅山は次々に閉山していっています。	
\\	銅, 山	
\\	銅像	
\\	どうぞう	
\\	誰かが銅像を粉々に破壊しました。	
\\	銅, 像	
\\	著作権	
\\	ちょさくけん	
\\	著作権の対象となるかどうかに関わらず、他人の作品は使用しないでください。	
\\	"著作 
\\	著作 
\\	著, 作, 権	
\\	田舎	
\\	いなか	
\\	の 
\\	田舎道で死にかけのフクロネズミを見ました。	
\\	中 (いなか) 
\\	田, 舎	
\\	原油	
\\	げんゆ	
\\	原油の値段は上がり続けている。	
\\	原, 油	
\\	泥酔	
\\	でいすい	
\\	する 
\\	完璧に泥酔していて、方向感覚を失っていました。	
\\	泥, 酔	
\\	行為	
\\	こうい	
\\	私がそのような残虐行為を繰り返したのは、その度に達成感を得られることができたからです。	
\\	行, 為	
\\	遅延	
\\	ちえん	
\\	する 
\\	の 
\\	バスの大幅な遅延により会議に遅れてしまった。	
\\	遅, 延	
\\	遅滞	
\\	ちたい	
\\	する 
\\	その動画を遅滞なく削除せよ。	
\\	遅, 滞	
\\	破壊	
\\	はかい	
\\	する 
\\	兎小屋は火事によって破壊されました。	
\\	破, 壊	
\\	恥ずかしい	
\\	はずかしい	い 
\\	こんなことを言うのは恥ずかしいんですが、私は安物買いの銭失いみたいなところがあるんですよね。	
\\	い 
\\	(は) 
\\	恥	
\\	炎	
\\	ほのお	
\\	山火事の炎は、強風にあおられ、さらに大きくなった。	
\\	たき火の炎を見つめていると、色々なことを思い出します。	
\\	ある超能力者が、ホワイトハウスが一時間もしないうちに炎に飲み込まれてしまう予知夢を見たと噂になっています。	
\\	(ほのお), 
\\	炎	
\\	為に	
\\	ために	
\\	夜勤の為に寝ておきたかったが、電車の音がうるさすぎた為に眠れなかった。	
\\	(ため) 
\\	為	
\\	果汁	
\\	かじゅう	
\\	俺が果汁百パーセントのオレンジジュースをストローですすっていると、誰かが俺の肩をトントンと叩いた。	
\\	果, 汁	
\\	給油	
\\	きゅうゆ	
\\	する 
\\	給油ポンプが車に届きません。	
\\	給, 油	
\\	庁舎	
\\	ちょうしゃ	
\\	庁舎に雷が落ちたみたいだよ。	
\\	庁, 舎	
\\	福寿	
\\	ふくじゅ	
\\	福寿と書かれた年賀状を受け取りました。	
\\	福, 寿	
\\	彼	
\\	かれ	
\\	しまった!急須を壊しちゃった。彼が知ったらめちゃくちゃ怒るだろうな。	
\\	彼	
\\	地獄	
\\	じごく	
\\	死後地獄に堕ちる事を願って、ありとあらゆる拷問について研究しています。	
\\	地, 獄	
\\	同音異義語	
\\	どうおんいぎご	
\\	同音異義語に関する
\\	ビデオを作ったので、よかったら見てみてください。	
\\	(異義), 
\\	同, 音, 異, 義, 語	
\\	尊敬語	
\\	そんけいご	
\\	よく尊敬語と謙譲語がゴチャゴチャになってしまいます。	
\\	尊敬 
\\	尊敬 
\\	語 
\\	尊, 敬, 語	
\\	蒸し暑い	
\\	むしあつい	い 
\\	今年の夏は絶対去年よりも蒸し暑いよね。	
\\	"蒸れる 
\\	蒸れる 
\\	暑い. 
\\	蒸, 暑	
\\	入獄	
\\	にゅうごく	
\\	する 
\\	初めて入獄した時、何故か安心感を得たんですよね。	
\\	入, 獄	
\\	油断	
\\	ゆだん	
\\	する 
\\	コウイチには油断をするな。奴は中々手強い男だぞ。	
\\	油, 断	
\\	炎症	
\\	えんしょう	
\\	筋肉の炎症には、ノンステロイドの痛み止めを処方しておきますね。	
\\	炎, 症	
\\	獄内	
\\	ごくない	
\\	の 
\\	あなたの獄内での話、最後はどうなるの?	
\\	獄, 内	
\\	介入	
\\	かいにゅう	
\\	する 
\\	警察の介入により、事態は一層複雑化した。	
\\	介, 入	
\\	紹介	
\\	しょうかい	
\\	する 
\\	の 
\\	「こんにちは。サーモンいますか?妹なんですけど。」「わぁ、イクラじゃない!!!」「わぁ、紹介してよ!」「こちらが、フグ。(フグを指さしながら)」「初めまして。」「鰐蟹とキンニクマは知ってるわよね!」「やあ、イクラちゃん。久しぶり。」	
\\	紹, 介	
\\	脱獄	
\\	だつごく	
\\	する 
\\	彼の母親は彼に脱獄を強要した。	
\\	脱, 獄	
\\	剣道	
\\	けんどう	
\\	コウイチには大きな目標がある。剣道で世界一になりたいんだ。	
\\	剣, 道	
\\	熊本県	
\\	くまもとけん	
\\	夜行バスは、熊本県で発生した事故のため一時不通になった。	
\\	熊
\\	本 
\\	県.
\\	熊, 本, 県	
\\	湖	
\\	みずうみ	
\\	仕事に向かう前に、湖のほとりに沿って寄り道をするのが好きだ。	
\\	みず 
\\	うみ 
\\	みずうみ.
\\	湖	
\\	山中湖	
\\	やまなかこ	
\\	安物のカメラで撮ったにしては、山中湖の写真すごく綺麗に撮れてるじゃん。	
\\	山 
\\	中, 
\\	湖.	山, 中, 湖	
\\	講演	
\\	こうえん	
\\	する 
\\	その講師、講演会場にリムジンで登場したんだよ。	
\\	講, 演	
\\	講義	
\\	こうぎ	
\\	する 
\\	講義の頭に、教授は私達がナイフにニックネームを付けているかどうかを尋ねました。	
\\	講, 義	
\\	講師	
\\	こうし	
\\	その大学の講師としての資格を得ました。	
\\	講, 師	
\\	暮らし	
\\	くらし	
\\	田舎暮らしは退屈だ。	
\\	暮らす 
\\	暮らす.	暮	
\\	寿命	
\\	じゅみょう	
\\	携帯の電池の寿命が短くなってきてるんだけど。	
\\	寿, 
\\	命 
\\	(みょう). 
\\	寿, 命	
\\	長寿	
\\	ちょうじゅ	
\\	の 
\\	婆ちゃんにいつも、「バナナを食べることは長寿に繋がる」と言われるんだよね。	
\\	長, 寿	
\\	味噌	
\\	みそ	
\\	味噌と醤油は和食には欠かせません。	
\\	味, 噌	
\\	互い	
\\	たがい	
\\	あの二人は、「どうぞお座りください」とお互いに席の譲り合いをしたんだ。	
\\	(たが) 
\\	互	
\\	相互	
\\	そうご	
\\	の 
\\	相互リンクしてもらえませんか。	
\\	相, 互	
\\	否定形	
\\	ひていけい	
\\	この文章を否定形にすることができますか。	
\\	否定 
\\	形 
\\	否, 定, 形	
\\	観測	
\\	かんそく	
\\	する 
\\	南極観測隊のメンバーになるにはどうすればいいですか。	
\\	観, 測	
\\	石油	
\\	せきゆ	
\\	石油価格が高騰している。	
\\	石, 油	
\\	油	
\\	あぶら	
\\	その割りには油は綺麗だね。	
\\	(あぶら) 
\\	油	
\\	油田	
\\	ゆでん	
\\	残念ですが、どの油田も私のものではありません。	
\\	田 
\\	(でん), 
\\	油, 田	
\\	己	
\\	おのれ	
\\	の 
\\	俺は、みんな死ぬ前に己の墓は己で建てるべきだと思うけどね。	
\\	(おのれ), 
\\	己	
\\	白熊	
\\	しろくま	
\\	どうしてここに白熊がいるんだ?	
\\	白, 熊	
\\	鍋	
\\	なべ	
\\	彼女が彼氏にキムチ鍋を作らせた。	
\\	鍋	
\\	予測	
\\	よそく	
\\	する 
\\	科学者たちは、大地震の予測に失敗した。	
\\	予, 測	
\\	出獄	
\\	しゅつごく	
\\	する 
\\	そして、出獄の日が近づくにつれて、憂鬱になって、何故か危機感を感じたんです。	
\\	出, 獄	
\\	彫刻	
\\	ちょうこく	
\\	する 
\\	彫刻を落として、真っ二つに割ってしまいました。	
\\	彫, 刻	
\\	海亀	
\\	うみがめ	
\\	その種の海亀は、絶滅の危機にあります。	
\\	うみがめ, 
\\	かめ 
\\	海, 亀	
\\	自己	
\\	じこ	
\\	の 
\\	彼は、「俺様が一番好きな歌手は俺様自身だよ」と自己満足の笑みを浮かべながら豪語していた。	
\\	自, 己	
\\	真剣	
\\	しんけん	
\\	な 
\\	売上げを損ねずに値上げする方法を真剣に考えているんです。	
\\	真, 剣	
\\	喜寿	
\\	きじゅ	
\\	バイトのみんなで、店長の喜寿をコンビニでお祝いする予定なんだよ。	
\\	喜 
\\	(き) 
\\	喜, 寿	
\\	恥	
\\	はじ	
\\	あいつ、俺にみんなの前で恥をかかされたって思ってるみたいでさ。	
\\	恥	
\\	破船	
\\	はせん	
\\	江戸時代の有名な刀を載せた船は、嵐によって破船してしまいました。	
\\	破, 船	
\\	醤油	
\\	しょうゆ	
\\	これは普通の醤油じゃなくて、最高の醤油だよ!	
\\	醤, 油	
\\	有意	
\\	ゆうい	
\\	な 
\\	の 
\\	卒論のための研究で、統計学的に有意な値を導けなかったんです。	
\\	有, 意	
\\	一筋	
\\	ひとすじ	
\\	うちの長男は野球一筋なのよ。女っ気も無くてさ。	
\\	一 
\\	ひと 
\\	筋 
\\	すじ. 
\\	一, 筋	
\\	汁	
\\	しる	
\\	熱々の豚汁をもらえますか。	
\\	(しる) 
\\	汁	
\\	寿司	
\\	すし	
\\	近所にお寿司屋さんがあるんだけど、すごく旨いちらし寿司をたったの10ドルで出しているんだよ。これ以上のものはないね。	
\\	寿, 司	
\\	剣	
\\	けん	
\\	私達は午前二時に剣で戦うことにした。	
\\	剣	
\\	彼ら	
\\	かれら	
\\	の 
\\	彼らはみんな、電車の切符を買っていなかったため、無賃乗車者として捕まった。	
\\	彼	
\\	酔う	
\\	よう	
\\	みんな船に酔ってしまい、そこら中に吐き散らかしていました。	
\\	酔ってる時に車を運転するのは絶対にダメだよ。	
\\	う 
\\	(よ). 
\\	酔	
\\	遅れる	
\\	おくれる	
\\	もし飛行機が遅れたら、補償金のようなものは支払われたりするんでしょうか。	
\\	(おく). 
\\	遅	
\\	滞る	
\\	とどこおる	
\\	家主が六ヶ月間家賃が滞っている男の家に出向くと、男は家の中で死んでいた。	
\\	う 
\\	(とど) 
\\	(こお) 
\\	滞	
\\	納まる	
\\	おさまる	
\\	金を払うだけで、彼らが納まるとは思えないけどな。	
\\	納める 
\\	(まる 
\\	納める.	納	
\\	閉まる	
\\	しまる	
\\	突然、扉がひとりでに閉まりました。	
\\	閉める 
\\	閉める.	閉	
\\	垂れる	
\\	たれる	
\\	ほら!母乳だと乳が垂れるって言うじゃない?	
\\	垂らす 
\\	(れる), 
\\	垂らす.	垂	
\\	否定する	
\\	ひていする	する 
\\	私は納豆がご飯に合うって説は否定しますね。	
\\	否定 
\\	否定 
\\	否, 定	
\\	装う	
\\	よそおう	
\\	予想していた昇進が無かったが、彼は平気を装って昇進した社員たちにおめでとうと言った。	
\\	(よそお) 
\\	装	
\\	暮れる	
\\	くれる	
\\	早くしないと、日が暮れるわよ。	
\\	暮らす 
\\	(らす) 
\\	暮らす, 
\\	暮	
\\	盛り上げる	
\\	もりあげる	
\\	俺の友達はメチャクチャ面白くて、パーティーを盛り上げるのが得意なんだよ。	
\\	"盛る 
\\	(上げる) 
\\	盛る 
\\	上げる. 
\\	盛, 上	
\\	測る	
\\	はかる	
\\	その看護婦さんは、血圧を測るのが上手です。	
\\	う 
\\	量る, 
\\	測	
\\	払う	
\\	はらう	
\\	えっと、シャンパンとキャビアの代金を払えるほどお金を持っていない気がするんですが。	
\\	う 
\\	払	
\\	宣言する	
\\	せんげんする	する 
\\	あの時、日本政府は非常事態を宣言するべきだったと思わないかい。	
\\	宣言 
\\	宣, 言	
\\	蒸し返す	
\\	むしかえす	
\\	またかよ。女はすぐ昔の話を蒸し返すよな。やってられないわ。	
\\	蒸れる 
\\	返す. 
\\	蒸, 返	
\\	尊敬する	
\\	そんけいする	する 
\\	尊敬する人はいますか?	
\\	尊敬 
\\	尊, 敬	
\\	閉じる	
\\	とじる	
\\	彼女はノートパソコンを閉じて、両腕を伸ばし、欠伸をした。	
\\	閉.	
\\	閉める 
\\	しまる. 
\\	とじる. 
\\	(と) 
\\	閉	
\\	破る	
\\	やぶる	
\\	うちのワンコがクリスマスプレゼントの包装をビリビリ破っているビデオを撮ったのよ。	
\\	う 
\\	(やぶ) 
\\	破	
\\	裏切り者	
\\	うらぎりもの	
\\	彼は裏切り者なんかじゃないよ。ただ、生まれつきちょっととぼけているだけさ。	
\\	(裏切り) 
\\	裏切り 
\\	者. 
\\	(物) 
\\	裏, 切, 者	
\\	厄介	
\\	やっかい	
\\	な 
\\	ビエットの顔に落書きなんてしたら、後々厄介なことになるよ。	
\\	やく 
\\	やっ.	厄, 介	
\\	亀	
\\	かめ	
\\	なんかその亀に親近感を感じるんだよな。	
\\	亀	
\\	諸々	
\\	もろもろ	
\\	諸々の理由がありまして、しばらく一時的に店を閉めることにしました。	
\\	(もろ) 
\\	諸, 々	
\\	酢	
\\	す	
\\	お酢は常温で保管しています。	
\\	酢	
\\	酢の物	
\\	すのもの	
\\	大人になったら酢の物が好きになりました。	
\\	酢 
\\	物 
\\	酢, 物	
\\	廃止	
\\	はいし	
\\	する 
\\	の 
\\	死刑制度廃止には、賛成ですか、反対ですか。	
\\	廃, 止	
\\	受諾	
\\	じゅだく	
\\	する 
\\	コウイチはアメリカ合衆国の大統領になることを渋々受諾した。	
\\	受, 諾	
\\	同盟	
\\	どうめい	
\\	する 
\\	の 
\\	ああ、そういうことか。
\\	は一番のライバル会社と同盟を組んだんだよ。	
\\	同, 盟	
\\	大幅	
\\	おおはば	
\\	な 
\\	会議の資料が大幅に変更されました。	
\\	大, 幅	
\\	献血	
\\	けんけつ	
\\	する 
\\	車が移動献血車に衝突するのを見ました。	
\\	献, 血	
\\	債券	
\\	さいけん	
\\	債券市場は何時まで開いてますか。	
\\	債, 券	
\\	主将	
\\	しゅしょう	
\\	どちらの主将もそんなに強くはなかった。	
\\	主, 将	
\\	指揮者	
\\	しきしゃ	
\\	どちらの指揮者もまだ来てないんだよ。	
\\	指揮 
\\	指揮 
\\	者.	指, 揮, 者	
\\	承諾	
\\	しょうだく	
\\	する 
\\	の 
\\	父は、私達の結婚を二つ返事で承諾した。	
\\	承, 諾	
\\	許諾	
\\	きょだく	
\\	する 
\\	もし娘さんとの結婚を許可してくださるのなら、あなたに我が社の登録商標の使用を許諾します。	
\\	許, 諾	
\\	諾否	
\\	だくひ	
\\	「諾否を御一報ください」という言葉は、仕事の手紙や
\\	などでよく使われるフレーズです。	
\\	諾, 否	
\\	変換	
\\	へんかん	
\\	する 
\\	太陽エネルギーを電気に変換する技術を開発した人に会ってみたいです。	
\\	海外でもパソコンを使えるように変換プラグを買ってきました。	
\\	平仮名がうまく漢字に変換できない時は、大体タイプミスが原因です。	
\\	変, 換	
\\	遺体	
\\	いたい	
\\	どうして犯人が遺体の唇を舐めたのかは不明です。	
\\	遺, 体	
\\	債権	
\\	さいけん	
\\	債権者は破産手続きによっていくらかお金を受け取れることが多いが、株主は滅多に受け取ることがない。	
\\	債, 権	
\\	舞踏	
\\	ぶとう	
\\	する 
\\	の 
\\	半年ぶりの舞踏会でした。	
\\	舞, 踏	
\\	舞	
\\	まい	
\\	これは、日本最古の舞です。	
\\	(まい) 
\\	舞	
\\	鹿	
\\	しか	
\\	牡鹿と牝鹿の区別がつきません。	
\\	(しか) 
\\	鹿	
\\	依存	
\\	いぞん, いそん	
\\	する 
\\	自分が携帯依存症だってことは認めます。	
\\	依, 存	
\\	普及	
\\	ふきゅう	
\\	する 
\\	思想を普及するための歌を作曲しました。	
\\	普, 及	
\\	汚い	
\\	きたない	い 
\\	汚い雑巾が、トイレに落ちていました。	
\\	汚, 
\\	きたない. 
\\	""来た、な
\\	汚	
\\	献金	
\\	けんきん	
\\	する 
\\	政治献金としていくらかお金を包んだ。	
\\	献, 金	
\\	枝豆	
\\	えだまめ	
\\	「お早うございます。何にいたしましょうか?」「いつものやつ頼むよ。」「かしこまりました。ビールと枝豆でございますね。」「そうだよ。」	
\\	枝 
\\	豆 
\\	枝, 豆	
\\	うなぎ丼	
\\	うなぎどんぶり, うなぎどん	
\\	「じゃあね、たまには連絡してね。」「もちろん。てか、近いうちにうなぎ丼でも食べに行こうよ。」「いいね。」「よかった。じゃあ、また連絡するね。」「ええ、またね。」	
\\	丼	
\\	電磁場	
\\	でんじば	
\\	私の父は、電磁場発生装置の特許を持っているんですが、私はそれが何のためのものなのかよく分かりません。	
\\	磁場 
\\	磁場 
\\	電, 磁, 場	
\\	廃絶	
\\	はいぜつ	
\\	する 
\\	核廃絶運動家たちは、核兵器を廃絶することは実現可能だと信じている。	
\\	廃, 絶	
\\	牙	
\\	きば	
\\	ハロウィン用の偽物の牙の在庫が切れそうです。	
\\	(きば).	牙	
\\	姓名	
\\	せいめい	
\\	私の姓名を使用しないでください。	
\\	姓, 名	
\\	将来	
\\	しょうらい	
\\	将来いつか、もっといい飼い犬の名前を思いつくかもしれない。	
\\	将, 来	
\\	一般	
\\	いっぱん	
\\	の 
\\	私みたいな一般庶民には、そのパンプスはちょっと高すぎるかな。	
\\	はん 
\\	ぱん 
\\	般.	一, 般	
\\	大将	
\\	たいしょう	
\\	の 
\\	大将が机の角に頭をぶつけて、たんこぶを作っちまったんだよ。	
\\	大, 将	
\\	遺伝	
\\	いでん	
\\	する 
\\	な 
\\	の 
\\	日本では、遺伝子組み換え食品にはパッケージへの表示が義務づけられています。	
\\	伝 
\\	(でん). 
\\	遺, 伝	
\\	馬鹿	
\\	ばか	
\\	な 
\\	お前がそこまで馬鹿じゃなければ、お前がどれだけ馬鹿なのか説明することができるんだけどなあ。	
\\	馬, 鹿	
\\	遺産	
\\	いさん	
\\	父の遺産を巡る家族間の争いが起きてるんです。	
\\	遺, 産	
\\	象牙	
\\	ぞうげ	
\\	印鑑を高級な象牙の物に変えました。	
\\	象, 牙	
\\	縄文	
\\	じょうもん	
\\	縄文時代にもおしゃぶりみたいなものはあったのかな。	
\\	縄, 文	
\\	歌舞伎	
\\	かぶき	
\\	歌舞伎のチケットが見当たらないんだけど。	
\\	歌, 舞, 伎	
\\	鹿児島県	
\\	かごしまけん	
\\	鹿児島県で、今までで運転した中で一番デッコボコの道路をドライブしました。	
\\	児 
\\	ご 
\\	鹿, 児, 島, 県	
\\	神奈川県	
\\	かながわけん	
\\	神奈川県では漫画がブームです。	
\\	か 
\\	神 
\\	み). 
\\	神, 奈, 川, 県	
\\	連盟	
\\	れんめい	
\\	の 
\\	スケート選手たちは、日本スケート連盟について不平を漏らしています。	
\\	連, 盟	
\\	口紅	
\\	くちべに	
\\	の 
\\	口紅をポケットに入れっぱなしにしてたみたいで、乾燥機の中で溶けちゃったのよね。	
\\	口 
\\	紅 
\\	口, 紅	
\\	遺失	
\\	いしつ	
\\	する 
\\	この空港に遺失物取扱所があるかどうかをご存知ですか。	
\\	遺, 失	
\\	旧姓	
\\	きゅうせい	
\\	いつもパスワードに母親の旧姓を使うんですよ。	
\\	旧, 姓	
\\	維持	
\\	いじ	
\\	する 
\\	車は維持費が高いし、今はまだそんな余裕はないな。	
\\	維, 持	
\\	治療	
\\	ちりょう	
\\	する 
\\	の 
\\	虫歯の治療をするために、五年ぶりに歯医者に行きました。	
\\	治, 療	
\\	医療	
\\	いりょう	
\\	の 
\\	日本で医者になりたいので、今は医療用語を勉強しています。	
\\	医, 療	
\\	誤り	
\\	あやまり	
\\	ようやく誤りに気がついた。	
\\	(あやま).
\\	誤	
\\	奈良	
\\	なら	
\\	私は歯のしつこい着色汚れを奈良の歯医者さんで落としました。	
\\	奈 
\\	良 
\\	なら 
\\	奈, 良	
\\	国債	
\\	こくさい	
\\	ちょうど国債を売ろうとしたところで、値が急激に下がって、売り損ねちゃったんだ。	
\\	国, 債	
\\	吐き気	
\\	はきけ	
\\	ビールでいい気分になったんだけど、その後吐き気がひどくてさ。	
\\	吐く 
\\	気 
\\	吐, 気	
\\	核実験	
\\	かくじっけん	
\\	予算の都合上、核実験への臨時スタッフの採用は増やせない。	
\\	(実験) 
\\	核, 実, 験	
\\	核兵器	
\\	かくへいき	
\\	核兵器を使用する前に、説明書に書かれている操作方法を必ずお読みください。	
\\	(へいき), 
\\	核, 兵, 器	
\\	核	
\\	かく	
\\	の 
\\	どうして核は細胞の中心部に位置するんですか?	
\\	沖田君は小学生の時に原子核について興味を持ち、今ではこのチームの核として働いてくれています。	
\\	核ミサイルの発射が、もう予定より三時間も遅れています。	
\\	核	
\\	沖合	
\\	おきあい	
\\	ピンク色の怪獣は海岸から約8
\\	沖合いにいるとの報告を受けました。	
\\	合 
\\	あい 
\\	沖, 合	
\\	親子丼	
\\	おやこどんぶり, おやこどん	
\\	てか、あの親子丼超まずかったよね!トイレで吐いちゃったよ!	
\\	親子丼 
\\	親 
\\	子 
\\	丼 
\\	親, 子, 丼	
\\	及第	
\\	きゅうだい	
\\	する 
\\	の 
\\	ねぇ、及第点って何点だったか覚えてる?	
\\	及, 第	
\\	盟約	
\\	めいやく	
\\	する 
\\	トーフグと盟約を結びたいんです。	
\\	盟, 約	
\\	指摘	
\\	してき	
\\	する 
\\	ちょっと間違いを指摘させてもらってもいいですか。	
\\	指, 摘	
\\	放射能	
\\	ほうしゃのう	
\\	放射能に関する会議を招集する前に、出席者を決める方がいいだろう。	
\\	"放射 
\\	放, 射, 能	
\\	薦め	
\\	すすめ	
\\	する 
\\	そのスリッパは私のお薦めです。	
\\	薦める 
\\	薦める. 
\\	薦	
\\	信頼	
\\	しんらい	
\\	する 
\\	彼は信頼できるような気がするし、早合点はしたくないね。	
\\	信, 頼	
\\	依頼	
\\	いらい	
\\	する 
\\	依頼は断られちゃうかもしれないけど、聞くだけ聞いてみたら。	
\\	依, 頼	
\\	維新	
\\	いしん	
\\	学校で明治維新について勉強はしたんですが、名前以外は何にも覚えていません。	
\\	維, 新	
\\	復旧	
\\	ふっきゅう	
\\	する 
\\	の 
\\	ビエトが
\\	のシステム復旧のために今一生懸命頑張っています。	
\\	復, 旧	
\\	縄	
\\	なわ	
\\	私にはどっちの縄も同じに見えますね。違いは全く分かりません。	
\\	(なわ), 
\\	縄	
\\	将軍	
\\	しょうぐん	
\\	つまり、敵の将軍と付き合っているってこと。	
\\	将, 軍	
\\	舞台	
\\	ぶたい	
\\	すごくいい声持ってるね。舞台で歌を歌ってみようと思った事はないの?	
\\	台 
\\	台 
\\	たい 
\\	だい.	舞, 台	
\\	継父	
\\	けいふ	
\\	小さい頃、継父にいつもおんぶをしてもらっていました。	
\\	継, 父	
\\	依然	
\\	いぜん	
\\	依然父の消息は不明です。	
\\	依, 然	
\\	継承	
\\	けいしょう	
\\	する 
\\	王位継承者の暗殺を企てている者がいるとの噂があります。	
\\	継, 承	
\\	超自然的	
\\	ちょうしぜんてき	な 
\\	そういえば、私の超自然的な力についてお話したことってありましたっけ?	
\\	(自然的) 
\\	超, 自, 然, 的	
\\	超音速	
\\	ちょうおんそく	
\\	僕のパンダロボットは、超音速で空を飛べるんだぜ。	
\\	超, 音, 速	
\\	甘い	
\\	あまい	
\\	い 
\\	甘いケーキと美味しいワインを買って、うちに招待すれば、必ず友達を作れるよ。	
\\	い 
\\	甘	
\\	甘党	
\\	あまとう	
\\	私は甘党で、ケーキには目がないの。	
\\	甘, 党	
\\	源氏物語	
\\	げんじものがたり	
\\	私は源氏物語を漫画でしか読んだことがありません。	
\\	源氏? 
\\	物語. 
\\	源氏 
\\	物語 
\\	源, 氏, 物, 語	
\\	奴ら	
\\	やつら	
\\	奴ら、俺たちは働いてるってのに、呑気に日光浴なんかしてやがるぜ。	
\\	奴 
\\	ら 
\\	奴.	奴	
\\	頼む	
\\	たのむ	
\\	「茹でたワニカニ、いつまでに必要ですか?」「できるだけ早く頼むよ。」	
\\	う 
\\	頼む 
\\	楽しい 
\\	頼	
\\	廃れる	
\\	すたれる	
\\	ビエトはカップケーキブームは既に廃れていると主張した。	
\\	う 
\\	廃	
\\	伸びる	
\\	のびる	
\\	どうして彼女の英語力がグングン伸びてるのか知ってる?	
\\	う 
\\	(びる). 
\\	伸	
\\	損害する	
\\	そんがいする	する 
\\	彼の言葉を信じて投資信託を買ったんですが、結局大きく損害してしまいました。	
\\	損害 
\\	損害 
\\	損, 害	
\\	舞う	
\\	まう	
\\	彼氏の部屋、埃がすっごい舞ってて最悪だったんだけど、綺麗な花びらが舞っているんだって思い込むように頑張ってみたさ。ま、無理だったけど。	
\\	う 
\\	(まう).	舞	
\\	換える	
\\	かえる	
\\	一つ大きいサイズの靴に換えてもらえませんか。	
\\	(か).	換	
\\	豆腐	
\\	とうふ	
\\	うーん。どっちの豆腐もいまいちかな。実は豆腐はあんまり好きじゃないんだよね。	
\\	豆, 腐	
\\	降りる	
\\	おりる	
\\	彼の奥さんだと思っていた女性が実はスパイでね、電車から飛び降りたんだよ。	
\\	(りる) 
\\	(お).	降	
\\	継ぐ	
\\	つぐ	
\\	日本では、通常は長男が一家の跡を継ぎます。	
\\	う 
\\	(つ) 
\\	つ 
\\	次, 
\\	継	
\\	甘く見る	
\\	あまくみる	
\\	シャツを入れないと、お客さんに甘く見られるかもしれないよ。	
\\	甘い 
\\	見る. 
\\	甘, 見	
\\	摘む	
\\	つむ	
\\	彼女は薔薇の花びらを一枚一枚丁寧に手で摘んでいきます。	
\\	う 
\\	(つ).	摘	
\\	及ぶ	
\\	およぶ	
\\	その火山の噴火の被害は、広範囲に及んだ。	
\\	う 
\\	及	
\\	吐き出す	
\\	はきだす	
\\	男は舐めていた飴を吐き出しました。	
\\	吐く 
\\	出す 
\\	吐く 
\\	出す.	吐, 出	
\\	踏む	
\\	ふむ	
\\	うわぁ!うんこ踏んじゃったの?きもーい。	
\\	う 
\\	(ふ)! 
\\	踏	
\\	降参する	
\\	こうさんする	する 
\\	降参しようと思ったことは今まで一度もないんですか。	
\\	"降参 
\\	降参 
\\	降, 参	
\\	摘発する	
\\	てきはつする	する 
\\	私は警察官ですが、麻薬の密売組織を摘発するような機会に出くわしたことがありません。	
\\	摘, 発	
\\	貿易	
\\	ぼうえき	
\\	する 
\\	貿易事務の仕事を探しています。	
\\	貿 
\\	易 
\\	(駅). 
\\	えき.	貿, 易	
\\	津波	
\\	つなみ	
\\	たくさんの人達が津波に流されてしまいました。	
\\	津 
\\	波 
\\	津, 波	
\\	超〜	
\\	ちょう	
\\	今超貧乏でさ〜。砂糖も切らしてるんだけど、それすら新しいの買えないんだわ。	
\\	超	
\\	超音波	
\\	ちょうおんぱ	
\\	超音波なら脂肪細胞を永久に破壊することができると言われたので、この3000ドルもする機械を買ったんですよ。	
\\	は 
\\	ぱ 
\\	波.	超, 音, 波	
\\	貴重品	
\\	きちょうひん	
\\	貴重品はご自身で管理して下さい。	
\\	貴重 
\\	貴, 重, 品	
\\	換気	
\\	かんき	
\\	する 
\\	の 
\\	ちょっと窓を開けて部屋の換気をしない?	
\\	換, 気	
\\	幅	
\\	はば	
\\	うちのお母さん、肩幅が広いから、私はそれが遺伝したの。	
\\	幅	
\\	勤め	
\\	つとめ	
\\	お勤め先はどちらですか?	
\\	勤める 
\\	勤める.	勤	
\\	狙い	
\\	ねらい	
\\	どうせアフィ狙いかステマの釣り記事だろ。	
\\	(ねら) 
\\	狙	
\\	抗体	
\\	こうたい	
\\	どうしてボスに逆らって、そのウィルスへの抗体を作り出そうとするのですか?予防接種を受ける!	
\\	抗, 体	
\\	患者	
\\	かんじゃ	
\\	その患者は、左足が麻痺してもなお、サッカー選手になれるという希望に固執していた。	
\\	患, 者	
\\	応募	
\\	おうぼ	
\\	する 
\\	応募用紙は裁判所でもらえますよ。	
\\	応, 募	
\\	陣	
\\	じん	
\\	コウイチ大統領は、報道陣と会見をする予定でしたが、食中毒にかかってしまったため中止となりました。	
\\	陣	
\\	執着	
\\	しゅうちゃく	
\\	する 
\\	異常なほどお金に執着する男の小説を書いてみたいんだよね。	
\\	着, 
\\	ちゃく. 
\\	(ちゃく). 
\\	執, 着	
\\	塁	
\\	るい	
\\	彼が塁を踏んだのかどうか、よく見えなかった。	
\\	塁	
\\	塁審	
\\	るいしん	
\\	あの一塁の塁審、試合中に屁をこきまくってたぜ。	
\\	塁, 審	
\\	塁打	
\\	るいだ	
\\	彼が塁打を打つのは久しぶりだね。いつもはホームランだからな。	
\\	二塁打 
\\	三塁打 
\\	塁, 打	
\\	戦闘	
\\	せんとう	
\\	する 
\\	の 
\\	実際のところ、忍者は直接戦闘することをできるだけ避けた。	
\\	戦, 闘	
\\	爆弾	
\\	ばくだん	
\\	爆弾を分解することは危険だってことは分かってたんだけど、すごく興味があったんだ。	
\\	爆, 弾	
\\	爆撃	
\\	ばくげき	
\\	する 
\\	そのチョコレート工場は、爆撃を受けて粉々になりました。	
\\	爆, 撃	
\\	眉間	
\\	みけん	
\\	の 
\\	社長は眉間にシワを寄せながら「この資料は誰が書いた?」と言った。	
\\	私のパグのチャームポイントはやはり眉間のシワです。	
\\	眉間を黒いマジックで塗りつぶしてください。	
\\	眉, 間	
\\	弾丸	
\\	だんがん	
\\	俺は弾丸の雨をとても上手くかわした。	
\\	弾 
\\	丸, 
\\	がん. 
\\	(がん). 
\\	弾, 丸	
\\	弾	
\\	たま	
\\	先日、銃の弾のようなかたちの便が出た。	
\\	球? 
\\	弾	
\\	陣営	
\\	じんえい	
\\	敵の陣営が見えてくると、恐怖を感じ始めました。	
\\	陣, 営	
\\	契機	
\\	けいき	
\\	吐血したのを契機に、煙草を止めました。	
\\	契, 機	
\\	崩壊	
\\	ほうかい	
\\	する 
\\	今朝の俺のクソが大き過ぎて、トイレの床が崩壊したんだ。	
\\	崩, 壊	
\\	契約	
\\	けいやく	
\\	する 
\\	あの事件が先方に契約を解除させたんだよ。	
\\	契, 約	
\\	火葬	
\\	かそう	
\\	する 
\\	火葬のシステムは自動的に停止する訳ではありません。係の人が小窓から骨の様子を見ながら、終わったと思う時に止めるのです。	
\\	火, 葬	
\\	死刑	
\\	しけい	
\\	死刑執行は五時間も遅れました。	
\\	死, 刑	
\\	削除	
\\	さくじょ	
\\	する 
\\	の 
\\	インターネット上からあんたの写真を全部削除するのは、時間はかかるだろうけど、俺の手に掛かれば不可能じゃないね。	
\\	削, 除	
\\	派遣	
\\	はけん	
\\	する 
\\	その派遣社員は、新しい職場の雰囲気をすぐに掴みました。	
\\	派, 遣	
\\	酔っ払い	
\\	よっぱらい	
\\	昨日の夜酔っ払いに絡まれた時、心臓がドキドキしました。	
\\	酔う 
\\	払い, 
\\	っ 
\\	酔, 払	
\\	弾力	
\\	だんりょく	
\\	お肌の弾力が無くなってきているの。	
\\	弾, 力	
\\	急患	
\\	きゅうかん	
\\	人員不足でごった返しになっていた病院が急患の受け入れを断ったせいで、男性が死亡したそうなんだけど、でもそれって病院だけのせいじゃないよね。	
\\	急, 患	
\\	証跡	
\\	しょうせき	
\\	部屋には犯罪の証跡はなかったが、警察は殺人事件ではないかと考えていた。	
\\	証, 跡	
\\	試験地獄	
\\	しけんじごく	
\\	試験地獄に苦しむ日本の学生にとっては、電車で勉強するのは当たり前のことです。	
\\	試, 験, 地, 獄	
\\	爆発	
\\	ばくはつ	
\\	する 
\\	彼女の一言が彼の怒りを爆発させました。	
\\	爆, 発	
\\	爆笑	
\\	ばくしょう	
\\	する 
\\	母さんと父さんを爆笑させるような、何かいいエイプリルフールの案はある?	
\\	笑 
\\	(しょう)! 
\\	爆, 笑	
\\	眉	
\\	まゆ	
\\	眉の形は人によって違う。	
\\	早く、眉にピアス開けたいな。	
\\	眉の動きに注目すると表情を捉えやすいって知ってましたか?	
\\	(まゆ)
\\	眉	
\\	闘志	
\\	とうし	
\\	その戦闘ロボットは、メラメラと闘志を燃やしていた。	
\\	闘, 志	
\\	上旬	
\\	じょうじゅん	
\\	気象庁は六月上旬に梅雨が始まると言っている。	
\\	上, 旬	
\\	漁師	
\\	りょうし	
\\	の 
\\	私はその漁師を知っていますよ。私の義理の兄ですから。	
\\	りょう 
\\	漁, 師	
\\	漁船	
\\	ぎょせん	
\\	昨夜漁船を盗まれました。	
\\	漁, 船	
\\	漁業	
\\	ぎょぎょう	
\\	の 
\\	ここは小さな漁村で、ここに住むほとんどの人は漁業に従事しています。	
\\	漁, 業	
\\	香り	
\\	かおり	
\\	どの香りを付けるかを決めるのは難しいですね。毎朝短くとも一時間はそれに時間を費やします。	
\\	香, 
\\	顔 (かお). 
\\	香	
\\	募金	
\\	ぼきん	
\\	する 
\\	の 
\\	一万円札を崩してもらうことはできますか?もしそれができるなら、千円札を募金します。	
\\	募, 金	
\\	葬式	
\\	そうしき	
\\	の 
\\	もし誰かから電話があったら、葬式に出ていて三時に戻ると伝えてください。	
\\	葬, 式	
\\	給油所	
\\	きゅうゆじょ, きゅうゆしょ	
\\	この給油所ではどのくらい働いているんですか?	
\\	(給油) 
\\	給, 油, 所	
\\	大抵	
\\	たいてい	
\\	な 
\\	の 
\\	大抵の外国人は納豆が苦手かと思いますが、私もその一人です。	
\\	大, 抵	
\\	串焼き	
\\	くしやき	
\\	最近、ケールの串焼きにはまっています。	
\\	金曜日の夜は、串焼き屋さんで夕飯を食べました。	
\\	うちのインコは「串焼きにしないで!」が口癖です。	
\\	(焼く) 
\\	串, 焼	
\\	湾	
\\	わん	
\\	伯父は、殺されて、ドラム缶の中でコンクリート詰めにされて東京湾に沈められました。	
\\	湾	
\\	二日酔い	
\\	ふつかよい	
\\	する 
\\	二日酔いだし、足もつるし今日は最悪だよ。	
\\	二日 
\\	酔い 
\\	酔う).	二, 日, 酔	
\\	聴力	
\\	ちょうりょく	
\\	今日の午後に、盲目の人の聴力に関するレポートを提出するつもりです。	
\\	聴, 力	
\\	香港	
\\	ほんこん	
\\	週の前半は出張で香港にいます。	
\\	ほんこん. 
\\	香, 港	
\\	旬	
\\	しゅん	
\\	の 
\\	苺は旬ではありませんが、クリスマスケーキにたくさん使われます。	
\\	旬	
\\	刑務所	
\\	けいむしょ	
\\	この刑務所は今満員なので、キャンセル待ちとさせて頂きますね。	
\\	刑, 務, 所	
\\	香川県	
\\	かがわけん	
\\	フグ夫妻は二人とも香川県には住んでいるが、別居中です。	
\\	香 
\\	川 
\\	県. 
\\	香 
\\	(か). 
\\	香, 川, 県	
\\	下旬	
\\	げじゅん	
\\	四月の下旬に日本に戻って、六月上旬にまたアメリカに戻ってくる予定です。	
\\	下, 旬	
\\	紹介状	
\\	しょうかいじょう	
\\	この病院で予約をするには、お医者さんからの紹介状が要ります。	
\\	年賀状 
\\	(紹介), 
\\	紹介 
\\	紹, 介, 状	
\\	終身刑	
\\	しゅうしんけい	
\\	終身刑を言い渡されたのは初めてです。	
\\	終, 身, 刑	
\\	聴者	
\\	ちょうしゃ	
\\	電話が壊れてるのかも。だって、聴者から電話が一本もないなんて、おかしいよ。	
\\	リスナー 
\\	聴, 者	
\\	一人暮らし	
\\	ひとりぐらし	
\\	一人暮らしをしているので、冬時間が始まった時に、時計を一時間遅らせることを忘れてしまいました。	
\\	暮らす 
\\	一人, 
\\	一人 
\\	暮らす.	一, 人, 暮	
\\	中旬	
\\	ちゅうじゅん	
\\	予定日は十一月の中旬です。	
\\	中, 旬	
\\	奇跡	
\\	きせき	
\\	の 
\\	奇跡でも起こらない限り、百人が同時におならをするのは不可能でしょう。	
\\	奇, 跡	
\\	味噌汁	
\\	みそしる	
\\	「明日までに、この味噌汁を飲んでしまわなければならないよ。」「じゃあ、さっさと片付けちまおうぜ。」	
\\	味噌 
\\	汁 
\\	味噌 
\\	汁 
\\	味, 噌, 汁	
\\	掲示	
\\	けいじ	
\\	する 
\\	アヤは、レストランが臨時休業の掲示をしている紙に余白があるのを見つけては、そこに落書きをします。	
\\	掲, 示	
\\	油絵	
\\	あぶらえ	
\\	この油絵はフリマで千五百円で買ったんですよ。	
\\	油 
\\	絵 
\\	油, 絵	
\\	前兆	
\\	ぜんちょう	
\\	の 
\\	この雲と風は嵐の前兆です。	
\\	前, 兆	
\\	支払い	
\\	しはらい	
\\	支払いはカードでお願いします。	
\\	支 
\\	払. 
\\	支, 払	
\\	払い	
\\	はらい	
\\	現金払いにして頂けるなら、5
\\	引きにさせて頂きます。	
\\	払う 
\\	払う, 
\\	払	
\\	香水	
\\	こうすい	
\\	学校では香水が禁止されているにもかかわらず、その男子生徒はいつも甘い香りの香水を付けていて、教師たちには自然な脇の臭いだと言っていた。	
\\	香, 水	
\\	刑事	
\\	けいじ	
\\	その刑事は俺の彼女の腕を突然掴んで、彼女にキスをしたんだ。だから、俺はそいつを殴りつけてやったんだけど、そしたら逮捕されたんだよ。そんなの不公平だよ。	
\\	刑, 事	
\\	〜鍋	
\\	なべ	
\\	豆乳鍋を食べてみたことはありますか?	
\\	鍋	
\\	刑期	
\\	けいき	
\\	その囚人は、十五年間の刑期を務めた後で釈放されたが、行く当てがなく自殺をしてしまいました。	
\\	刑, 期	
\\	掲載	
\\	けいさい	
\\	に掲載されているブログ記事を何部かコピーして、日本語を学習している私の生徒たちに配っても構いませんか?	
\\	掲, 載	
\\	刑罰	
\\	けいばつ	
\\	の 
\\	犯罪を根絶するには、火あぶりの刑みたいな残酷な刑罰を全部復活させるべきだと思うね。	
\\	刑, 罰	
\\	募集	
\\	ぼしゅう	
\\	する 
\\	はまだブロガーを募集中ですよ。	
\\	募, 集	
\\	削減	
\\	さくげん	
\\	する 
\\	コストの削減に苦戦しています。	
\\	削, 減	
\\	弁償	
\\	べんしょう	
\\	する 
\\	撮影中に漁師さんの網を誤って破いてしまったため、弁償しなくてはならなかった。	
\\	弁, 償	
\\	抗戦	
\\	こうせん	
\\	する 
\\	学生たちは、権威に対して徹底抗戦を強いられました。	
\\	抗, 戦	
\\	遺跡	
\\	いせき	
\\	古代の遺跡で、フグの形をした石像を見つけました。	
\\	遺, 跡	
\\	船酔い	
\\	ふなよい	
\\	する 
\\	この薬で船酔いを克服しました。	
\\	船 
\\	ふな 
\\	酔い 
\\	酔う.	船, 酔	
\\	臨海	
\\	りんかい	
\\	の 
\\	日本の学生の多くは、夏に臨海学校に行く事を経験します。	
\\	臨, 海	
\\	自己紹介	
\\	じこしょうかい	
\\	する 
\\	自己紹介では、いつもすぐ言うことがなくなっちゃうんだよね。	
\\	(自己) 
\\	(紹介) 
\\	自, 己, 紹, 介	
\\	恥知らず	
\\	はじしらず	
\\	な 
\\	あの俺の兄貴は恥知らずの嘘つきだ。何の事だかサッパリ分からないなんて言ってたくせに、俺のベーコンを平らげたのはアイツだったんだよ。	
\\	"恥 
\\	知らず 
\\	恥 
\\	知る 
\\	恥, 知	
\\	彼女	
\\	かのじょ	
\\	彼女は服のセンスはいいが、体型がいまいちだ。	
\\	彼 
\\	(かの), 
\\	彼, 女	
\\	狙撃	
\\	そげき	
\\	する 
\\	男は政治家を狙撃するために、そのホテルの部屋が見える場所までやってきた。	
\\	狙, 撃	
\\	昭和	
\\	しょうわ	
\\	昭和何年生まれですか。	
\\	昭, 和	
\\	兆候	
\\	ちょうこう	
\\	の 
\\	彼女は、私に下痢の兆候がきたと告げるやいなや、トイレに駆け込んでいった。	
\\	兆, 候	
\\	串	
\\	くし	
\\	今度の
\\	には100本くらい串が必要です。	
\\	肉とジャガイモに串を刺してくれませんか?	
\\	焼き鳥とは、串に刺さった鶏肉のことです。	
\\	串	
\\	遅い	
\\	おそい	い 
\\	私のパソコンはハードディスクの容量が無くなってきているので、動作がすごく遅いです。	
\\	い 
\\	(おそ).	遅	
\\	汁物	
\\	しるもの	
\\	あいつ、あのウェイトレスさんに三回もこのレストランに汁物はないのかって聞きやがったんだぜ。ウェイトレスさんその度にありませんって言ってさ、困惑してたよ。	
\\	汁.	汁, 物	
\\	自爆	
\\	じばく	
\\	する 
\\	の 
\\	今月末までに、自爆テロを実行しなくてはなりません。	
\\	自, 爆	
\\	甘酢	
\\	あまず	
\\	この間、給食の時に生徒が何故か机の中から甘酢の瓶を取り出したのよ。	
\\	甘い 
\\	酢, 
\\	す 
\\	ず 
\\	甘, 酢	
\\	台湾	
\\	たいわん	
\\	の 
\\	感染源を台湾のアヒルの群れだと突き止めました。	
\\	たい 
\\	台, 
\\	わん 
\\	だいわん, 
\\	たいわん.	台, 湾	
\\	臨時	
\\	りんじ	
\\	の 
\\	マミは、臨時のボーナスが出たので、さらにベーコンを購入した。	
\\	臨, 時	
\\	募る	
\\	つのる	
\\	妻への不信感は募るばかりです。	
\\	みんなで募金を募りましょう。	
\\	(つの), 
\\	募	
\\	患う	
\\	わずらう	
\\	シェルターには、結核を患うホームレスの人がたくさんいます。	
\\	う 
\\	(わずら), 
\\	患	
\\	恥ずかしがる	
\\	はずかしがる	
\\	ストリッパーになると自分で決断したにも関わらず、彼女はまだ人前で裸で踊ることを恥ずかしがっているんだよ。	
\\	"恥ずかしい 
\\	恥ずかしい.	恥	
\\	弾む	
\\	はずむ	
\\	彼との会話がこんなに弾むなんて、思ってもみなかったわ。	
\\	(はず) 
\\	弾	
\\	葬る	
\\	ほうむる	
\\	その事件の真相が闇に葬られるのではないかと心配しています。	
\\	う 
\\	(ほうむ) 
\\	葬	
\\	償う	
\\	つぐなう	
\\	たとえあなたが死をもって罪を償っても、問題は解決しないでしょう。	
\\	(つぐなう).	償	
\\	遣う	
\\	つかう	
\\	義理の両親と暮らすのって、毎日気を遣うだろうし、疲れちゃうと思うのよね。	
\\	使う, 
\\	遣	
\\	対向する	
\\	たいこうする	する 
\\	対向するバイクの存在を完璧に見落としていました。	
\\	対, 向	
\\	崩す	
\\	くずす	
\\	「1ドル札を崩してもらえませんか?」「ええっと…25セント硬貨が2枚と10セント硬貨が5枚あります。これでよろしいですか?」「はい。有難うございます。」	
\\	(す) 
\\	(くず), 
\\	崩	
\\	恵む	
\\	めぐむ	
\\	本日の
\\	の収益の全額が、恵まれない子どもたちに寄付されます。	
\\	う 
\\	(めぐ). 
\\	恵	
\\	臨む	
\\	のぞむ	
\\	の本社は、オレゴン州という、太平洋に臨むアメリカ北西部の州に位置しています。	
\\	(のぞ) 
\\	臨	
\\	破れる	
\\	やぶれる	
\\	ズボンのお尻の部分がいつも最初に破れるんだけど、どうしてだろう。	
\\	破る 
\\	破る.	破	
\\	抱く	
\\	だく, いだく	
\\	母熊は自分の赤ん坊を抱き上げ、顔中を舐め回した。	
\\	う 
\\	抱いて抱いて!	
\\	(だく), 
\\	(だ). 
\\	を抱く?	抱	
\\	紹介する	
\\	しょうかいする	する 
\\	お前にどの娘を紹介するかを選ぶのは難しいよ。	
\\	紹介 
\\	紹, 介	
\\	跳ぶ	
\\	とぶ	
\\	彼女は完璧なトリプルアクセルを跳んだのに、スコアはあまり伸びなかった。	
\\	う 
\\	飛ぶ 
\\	跳	
\\	聴く	
\\	きく	
\\	その癒し系の曲を聴いてると、眠くなってきたよ。	
\\	う 
\\	聞く, 
\\	聴	
\\	掲げる	
\\	かかげる	
\\	ジェームズは、鉛筆削りの技術のレベルを上げるという目標を掲げた。	
\\	う 
\\	(かか). 
\\	掲	
\\	反抗する	
\\	はんこうする	する 
\\	犬が私に反抗したので、「お前は悪い犬だ!」って叱りつけてやりましたよ。	
\\	反, 抗	
\\	抵抗する	
\\	ていこうする	する 
\\	どうしてあのケーキを食べるのを抵抗したんだ?	
\\	抵, 抗	
\\	戻る	
\\	もどる	
\\	母親が病気になったので、私は国に戻ることにしました。	
\\	う 
\\	戻	
\\	戻す	
\\	もどす	
\\	「彼にあんなひどいこと言わなければよかったわ。」「そうね。でも、言ってしまったことはもう元には戻せないわ。」	
\\	う 
\\	(す) 
\\	戻	
\\	闘う	
\\	たたかう	
\\	この子たちは、みんな病気と闘っているんです。	
\\	う 
\\	戦う.
\\	(たたか). 
\\	闘	
\\	盛り上がる	
\\	もりあがる	
\\	子どもっていうのは、どんなささいなことでも手当たり次第に盛り上がることができるのよね。	
\\	"盛り上げる 
\\	上がる 
\\	盛り 
\\	上がる. 
\\	盛, 上	
\\	執る	
\\	とる	
\\	あなたにこのプロジェクトの指揮を執ってもらいたいんです。	
\\	取る. 
\\	執	
\\	削る	
\\	けずる	
\\	歯を削る音ってのは、私にとっては最も不快な音の一つだね。	
\\	う 
\\	(けず) 
\\	削	
\\	跡	
\\	あと	
\\	私は、ワイングラスに付いた彼女の唇の跡を指でなぞった。	
\\	後 (あと) 
\\	跡	
\\	盗聴	
\\	とうちょう	
\\	する 
\\	の 
\\	誰かが
\\	のオフィスに盗聴器を仕掛けたのですが、私達は犯人も理由も分かりません。	
\\	盗, 聴	
\\	知恵	
\\	ちえ	
\\	無い知恵を絞ってこのラブレターを書いたよ。	
\\	知, 恵	
\\	執筆	
\\	しっぴつ	
\\	する 
\\	最後に短編小説を執筆したのはいつですか?	
\\	執 
\\	しつ 
\\	しっ. 
\\	執, 筆	
\\	馬鹿馬鹿しい	
\\	ばかばかしい	い 
\\	馬鹿馬鹿しいこと言わないでよ。	
\\	馬鹿 
\\	馬鹿, 
\\	馬, 鹿	
\\	馬鹿らしい	
\\	ばからしい	い 
\\	全く馬鹿らしいにもほどがある!!!	
\\	"馬鹿 
\\	(らしい) 
\\	馬鹿.	馬, 鹿	
\\	頼み	
\\	たのみ	
\\	コウイチはビエトに窓を開けるよう頼みましたが、ビエトはコウイチの頼みを拒絶しました。	
\\	頼む 
\\	頼む.	頼	
\\	冷房	
\\	れいぼう	
\\	する 
\\	暑すぎだよ〜。ちょっと冷房付けてもらえないかな。	
\\	冷, 房	
\\	及び	
\\	および	
\\	このセキュリティー対策アプリは、
\\	及び
\\	のどちらでもご利用頂けます。	
\\	及ぶ 
\\	及ぶ, 
\\	及	
\\	舞踏会	
\\	ぶとうかい	
\\	今回の舞踏会には、前回よりも明るい色のドレスを着て行くつもりです。	
\\	(舞踏) 
\\	舞, 踏, 会	
\\	基盤	
\\	きばん	
\\	ベーコン事業に先立ち、まずは資金基盤を築くつもりです。	
\\	基, 盤	
\\	受託	
\\	じゅたく	
\\	する 
\\	我が社にとってその受託契約を受注することがどんなに重要か、分かっているよな。	
\\	受, 託	
\\	避妊	
\\	ひにん	
\\	する 
\\	の 
\\	今年の新年の抱負の一つは、ちゃんと避妊をすること、です。	
\\	避, 妊	
\\	湯豆腐	
\\	ゆどうふ	
\\	最初は、どんなもんかなと思って湯豆腐を食べてみただけなんですが、そしたら好きになっちゃったんですよね。	
\\	豆腐 
\\	湯 
\\	豆腐 
\\	湯, 豆, 腐	
\\	賄賂	
\\	わいろ	
\\	彼にはビッグマック以外の賄賂は効かないよ。	
\\	賄, 賂	
\\	贈収賄	
\\	ぞうしゅうわい	
\\	父は、私が三歳の時に、「その贈収賄事件には私は関与していない」と母に言うように教えた。	
\\	贈, 収, 賄	
\\	依頼人	
\\	いらいにん	
\\	もう少し我慢すれば、依頼人が誰だか分かったのに。	
\\	依, 頼, 人	
\\	房	
\\	ふさ	
\\	この葡萄、三房で298円だったんですよ。	
\\	(ふさ) 
\\	房	
\\	考慮	
\\	こうりょ	
\\	する 
\\	私の家族ではなく自分の家族のことを考慮してください。	
\\	考, 慮	
\\	配慮	
\\	はいりょ	
\\	する 
\\	お母さん、大人って子供への配慮が欠けていると思わない?	
\\	配, 慮	
\\	委託	
\\	いたく	
\\	する 
\\	そいつがさ、男の目隠しまではしたけど、残りの殺人業務は他の暗殺者に委託したって言うんだよ。	
\\	委, 託	
\\	抑制	
\\	よくせい	
\\	する 
\\	の 
\\	生理前は、月経前症候群のせいで、感情をうまく抑制することができません。	
\\	抑, 制	
\\	日刊	
\\	にっかん	
\\	の 
\\	アヤが描いたパグのイラストが、ある日刊紙のマスコットキャラクターに選ばれました。	
\\	にち 
\\	にっ.	日, 刊	
\\	需要	
\\	じゅよう	
\\	シャツは需要は大きいかもしれないけど、コスパがあまり良くないんだよね。	
\\	需, 要	
\\	描写	
\\	びょうしゃ	
\\	する 
\\	アヤは、
\\	殺人事件を細部に渡って描写した。	
\\	描, 写	
\\	奥底	
\\	おくそこ	
\\	心の奥底でほくそ笑みました。	
\\	奥, 底	
\\	抑止	
\\	よくし	
\\	する 
\\	どの国が一番素晴らしい犯罪抑止モデルを有していると思われますか。	
\\	抑, 止	
\\	円盤	
\\	えんばん	
\\	の 
\\	しばらくすると、ジェームズは円盤を脇に抱えて戻ってきて、
\\	を見つけたんだと言いました。	
\\	円, 盤	
\\	併殺	
\\	へいさつ	
\\	する 
\\	「ゲッツー」というのは、併殺を意味する和製英語です。	
\\	併, 殺	
\\	描画	
\\	びょうが	
\\	する 
\\	クロード・モネは、点画法という、点描画法とは少し違った技法を得意としていました。	
\\	描, 画	
\\	懸命	
\\	けんめい	
\\	な 
\\	足が泥にはまって、一生懸命抜こうとしてみたんですが、できなかったんです。	
\\	懸, 命	
\\	高齢者	
\\	こうれいしゃ	
\\	高齢者たちは、雇用について難しい問題を抱えています。	
\\	高, 齢, 者	
\\	逃亡	
\\	とうぼう	
\\	する 
\\	私の弟は優し過ぎて犯罪者達の逃亡を助けてしまい、そのせいで逮捕されてしまいました。	
\\	逃, 亡	
\\	避難	
\\	ひなん	
\\	する 
\\	その避難梯子はひどく痛んでいるので、交換する必要があります。	
\\	避, 難	
\\	夕刊	
\\	ゆうかん	
\\	どの新聞の夕刊でも、株式市場の大暴落が一面記事で報じられた。	
\\	夕, 刊	
\\	奥	
\\	おく	
\\	奥にもっと小さいサイズのものも置いてますよ。	
\\	奥	
\\	致命的	
\\	ちめいてき	な 
\\	生徒とのスキャンダルは、彼の教師生命にとって致命的でした。	
\\	致, 命, 的	
\\	一人称	
\\	いちにんしょう	
\\	このホラー映画は、一人称の視点から撮影されています。	
\\	一, 人, 称	
\\	全般的	
\\	ぜんぱんてき	な 
\\	全般的に、犬は猫を怖がるものじゃないんですかね。	
\\	全, 般, 的	
\\	遺伝子	
\\	いでんし	
\\	本当に彼らは父親を鑑定する遺伝子検査を受けたいんでしょうか。	
\\	遺伝 
\\	遺, 伝, 子	
\\	宜しい	
\\	よろしい	い 
\\	ここに鞄を置かせてもらっても宜しいですか?	
\\	宜	
\\	暖房	
\\	だんぼう	
\\	する 
\\	何で暖房消したの?パイプが全部凍っちゃったじゃない。	
\\	暖, 房	
\\	膝	
\\	ひざ	
\\	ピザを膝の上で食べるのが好きです。	
\\	レンタル彼氏に膝をついてプロポーズしてもらうには追加料金が10万円もかかる。	
\\	私の母は膝で歩くレースの世界記録保持者だ。	
\\	膝	
\\	膝頭	
\\	ひざがしら	
\\	膝頭の毛が気になって眠れることすらできない。	
\\	コウイチが寝ている間に、膝頭に目を書いた。	
\\	今年アップル社は、膝頭認証の開発を始めるらしい。	
\\	頭 
\\	(がしら). 
\\	膝, 頭	
\\	緩やか	
\\	ゆるやか	な 
\\	その小川が緩やかに東にカーブしているところがあるんだけど、そこが釣りの穴場なんだよ。	
\\	緩	
\\	緩い	
\\	ゆるい	い 
\\	このシャツは首回りが緩いんですよね。	
\\	い 
\\	緩	
\\	明治維新	
\\	めいじいしん	
\\	蛍光灯は明治維新の頃にはまだ無かったんだよね。	
\\	(明治), 
\\	(維新). 
\\	明, 治, 維, 新	
\\	併合	
\\	へいごう	
\\	する 
\\	アメリカ合衆国は、私が生まれた年にテキサスを併合しました。	
\\	合 
\\	ごう 
\\	ごういち.	併, 合	
\\	月刊	
\\	げっかん	
\\	の 
\\	月刊誌『ベーコン』は、エープリルフールのジョークとして、「ノーベルベーコン賞はマミに授与された。」と報じた。	
\\	月 
\\	月, 刊	
\\	朝刊	
\\	ちょうかん	
\\	今日の朝刊の大ニュースには、誰もが多かれ少なかれ興味を抱いているんじゃないかな。	
\\	朝 
\\	(ちょう) 
\\	朝, 刊	
\\	奏楽	
\\	そうがく	
\\	する 
\\	千円くらいの、奏楽用のハンドベルはありますか?	
\\	奏, 楽	
\\	演奏	
\\	えんそう	
\\	する 
\\	バンドが舞台で演奏をしている最中に、電球が切れてしまいました。	
\\	演, 奏	
\\	奈良県	
\\	ならけん	
\\	私が奈良県に住んでいる時は、彼女は親友の一人でしたが、私が大阪に引っ越してからは疎遠になりつつあります。	
\\	(奈良)? 
\\	奈, 良, 県	
\\	必需品	
\\	ひつじゅひん	
\\	新生児にとっての必需品って何だろう。	
\\	必, 需, 品	
\\	沖縄	
\\	おきなわ	
\\	うちの娘は沖縄が大好きで、一体いつ戻ってくるのかさっぱり分かりません。	
\\	沖, 縄	
\\	老齢	
\\	ろうれい	
\\	の 
\\	日本の老齢年金の仕組みについてご説明してもらえませんでしょうか。	
\\	老, 齢	
\\	却って	
\\	かえって	
\\	善かれと思ってしたことが、却って仇となって友人のストレスになってしまいました。	
\\	変える, 
\\	却	
\\	抑圧	
\\	よくあつ	
\\	する 
\\	警察は、その暴動を抑圧するのにしばらく時間がかかりました。	
\\	抑, 圧	
\\	一般的	
\\	いっぱんてき	な 
\\	大学卒業後も実家に住み続けることは、日本ではそこそこ一般的です。	
\\	一般 
\\	的 
\\	一般 
\\	的.	一, 般, 的	
\\	妊婦	
\\	にんぷ	
\\	の 
\\	妊婦さんのためのカウンセラーという、臨時の仕事が見つかりました。	
\\	妊, 婦	
\\	奥深い	
\\	おくぶかい, おくふかい	い 
\\	狩人とは非常に奥深い職業です。	
\\	おく 
\\	ふかい 
\\	深い, 
\\	ふ 
\\	ぶ, 
\\	奥, 深	
\\	生還	
\\	せいかん	
\\	する 
\\	行方不明の英語教師は、足が凍傷になっていたものの、無事に生還しました。	
\\	生, 還	
\\	却下	
\\	きゃっか	
\\	する 
\\	メンバー全員がバンジージャンプに挑戦するというアイディアは、速攻で却下されました。	
\\	却, 下	
\\	遠慮	
\\	えんりょ	
\\	する 
\\	な 
\\	質問があればどんなものでも遠慮なくおっしゃって下さいね。	
\\	遠, 慮	
\\	返還	
\\	へんかん	
\\	する 
\\	優勝旗が毎年返還されなくちゃいけないなんて、知らなかったよ。	
\\	返, 還	
\\	還元	
\\	かんげん	
\\	する 
\\	の 
\\	我々は円高還元セールを実施します。	
\\	還, 元	
\\	縄張り	
\\	なわばり	
\\	する 
\\	案の定、その場所はヤクザの縄張りだった。	
\\	縄 
\\	張る 
\\	縄, 張	
\\	内緒	
\\	ないしょ	
\\	の 
\\	私は
\\	レコーダーで会話を内緒で録音します。	
\\	内, 緒	
\\	選択	
\\	せんたく	
\\	する 
\\	私の大学では、第二外国語は自由に選べるんですが、どの言語を選択するか迷いに迷っています。	
\\	選, 択	
\\	対称	
\\	たいしょう	
\\	の 
\\	アシメっていう左右非対称のショートカットにしてみようかなって思ってるんだよね。	
\\	対, 称	
\\	お見舞い	
\\	おみまい	
\\	病院にお見舞いに行くと、彼女の部屋はたくさんの花で飾られていました。	
\\	舞い 
\\	舞う.	見, 舞	
\\	傾向	
\\	けいこう	
\\	する 
\\	日本人は外国人と喋る時、英語訛りの日本語を話す傾向があります。	
\\	傾, 向	
\\	廃止する	
\\	はいしする	する 
\\	教師陣が制服を廃止するかどうか検討しているそうですよ。	
\\	廃止 
\\	廃止 
\\	廃, 止	
\\	伴う	
\\	ともなう	
\\	技術の進歩に伴って人々は自然環境にだんだん関心がなくなってきた。	
\\	う 
\\	""友, 
\\	(ともなう) 
\\	伴	
\\	同伴する	
\\	どうはんする	する 
\\	今日は奥さんが同伴するのかと思ってたよ。	
\\	同, 伴	
\\	避ける	
\\	さける, よける	
\\	明日車で母を東京の名所へ案内する予定なんだけど、渋滞をうまく避ける方法とか知らないよね?	
\\	う 
\\	(さけ) 
\\	(よ) 
\\	避	
\\	傾く	
\\	かたむく	
\\	その松の木は、少し南の方向へ傾いています。	
\\	う 
\\	(く) 
\\	(かたむ), 
\\	傾	
\\	妊娠する	
\\	にんしんする	する 
\\	彼女は尿で妊娠反応検査をして、お医者さんから妊娠していると言われました。	
\\	妊, 娠	
\\	致す	
\\	いたす	
\\	もしよろしければ私が町をご案内致しますよ。	
\\	う 
\\	(いた), 
\\	致	
\\	描く	
\\	かく, えがく	
\\	うちのロゴを違う色でも描いてみてもらえますか。	
\\	う 
\\	書く, 
\\	えがく, 
\\	絵を書く, 
\\	描	
\\	託す	
\\	たくす	
\\	探偵を託して写真を全部渡したのが間違いでした。	
\\	たく, 
\\	託	
\\	逃げる	
\\	にげる	
\\	怒り狂った妻から逃げ回っているうちに、スーツがボロボロに痛んでしまいました。	
\\	う 
\\	(に). 
\\	逃	
\\	超す	
\\	こす	
\\	彼はあまり自分の気持ちを言葉で表さない人なので、度を超す程私に執着していることを隠していました。	
\\	う 
\\	子 (こ)! 
\\	子. 
\\	超	
\\	及ぼす	
\\	およぼす	
\\	放射能汚染は人体にどのような影響を及ぼすと考えられているのでしょうか。	
\\	及ぶ 
\\	及ぶ, 
\\	及	
\\	一緒	
\\	いっしょ	
\\	の 
\\	「ご一緒してもよろしいですか?」「全然、構いませんよ。どうぞお座りください。」	
\\	一 
\\	いっ.	一, 緒	
\\	贈る	
\\	おくる	
\\	ファンタジーベースボールの優勝者には、何か賞品が贈られます。	
\\	う 
\\	/送る (おくる) 
\\	贈	
\\	扱う	
\\	あつかう	
\\	夫の連れ子にまるで他人の用に扱われて、すごくストレスが溜まります。	
\\	う 
\\	扱	
\\	緩む	
\\	ゆるむ	
\\	そのトンネルのボルトは、何本か緩んできています。	
\\	う 
\\	(む)...	
\\	緩	
\\	維持する	
\\	いじする	する 
\\	健康を維持するために、何か運動をする方がいいよ。	
\\	(維持)
\\	維, 持	
\\	一致する	
\\	いっちする	する 
\\	偽のベーコンに残された指紋は、マミのものと一致しました。	
\\	いち 
\\	いっ.	一, 致	
\\	賄う	
\\	まかなう	
\\	もし今赤ちゃんができたら、俺の収入だけで家族を賄うのは無理だと思うんだよ。	
\\	う 
\\	(まかなう) 
\\	賄	
\\	指摘する	
\\	してきする	する 
\\	何名かの歩行者が、運転者が赤信号を無視したことを指摘しました。	
\\	指摘 
\\	指摘.	指, 摘	
\\	刊行する	
\\	かんこうする	する 
\\	マミはアヤに、ベーコン雑誌を一緒に刊行しないかと尋ねた。	
\\	刊, 行	
\\	頼る	
\\	たよる	
\\	彼の事を頼っていいものか決めかねています。	
\\	"頼む 
\\	頼む, 
\\	たよ, 
\\	(たよ). 
\\	頼	
\\	繰る	
\\	くる	
\\	おっと。すみません。トランプを繰るのが苦手なんです。	
\\	う 
\\	繰	
\\	懸ける	
\\	かける	
\\	何か命を懸けるほどのものはありますか。	
\\	う 
\\	(か). 
\\	懸	
\\	踏み込む	
\\	ふみこむ	
\\	ルームメートが、私のプライバシーに踏み込みすぎだと思うのよね。	
\\	(踏む) 
\\	(込む) 
\\	踏む 
\\	込む, 
\\	踏, 込	
\\	伸ばす	
\\	のばす	
\\	祖母は死ぬ間際、私の顔に手を伸ばし、「愛してるよ」と言ってくれました。	
\\	伸びる 
\\	伸ばす 
\\	伸びる, 
\\	伸	
\\	降ろす	
\\	おろす	
\\	荷物を降ろすのを手伝ってくれ。	
\\	う 
\\	降りる 
\\	降	
\\	乗り換える	
\\	のりかえる	
\\	飛行機をうまく乗り換えることができるかどうか心配しています。	
\\	乗る 
\\	換える.	乗, 換	
\\	貿易会社	
\\	ぼうえきがいしゃ	
\\	三百人以上の人がその貿易会社の職に応募したが、誰も採用されなかったそうだよ。	
\\	(貿易) 
\\	(会社) 
\\	易 
\\	(えき).	貿, 易, 会, 社	
\\	盤	
\\	ばん	
\\	私はビートルズのビニール盤を収集しています。	
\\	盤	
\\	信託	
\\	しんたく	
\\	する 
\\	の 
\\	私はその証券マンを信頼して、投資信託に大金をつぎ込みました。	
\\	信, 託	
\\	奥さん	
\\	おくさん	
\\	「おまえの奥さんって、どんな人?」「彼女は何でもバリバリやるタイプだよ。」	
\\	さん 
\\	奥	
\\	仮称	
\\	かしょう	
\\	する 
\\	タイトルはまだ単なる仮称です。	
\\	仮, 称	
\\	欠伸	
\\	あくび	
\\	退屈すぎて欠伸が出てきた。	
\\	(あく) 
\\	(び). 
\\	欠, 伸	
\\	年齢	
\\	ねんれい	
\\	「年齢はおいくつですか?」「あまり言いたくないです。」「じゃあ、別にいいよ。ただ、百歳を超えてるかどうかだけ教えてくれない?」	
\\	年, 齢	
\\	活躍	
\\	かつやく	
\\	リレーのレースで、彼は五人をごぼう抜きにする活躍をみせ、彼のチームは見事一等賞に輝いた。	
\\	活, 躍	
\\	小遣い	
\\	こづかい	
\\	お子さんには月々お小遣いはいくら渡しているんですか。	
\\	遣う 
\\	小 
\\	遣う
\\	小, 遣	
\\	南極圏	
\\	なんきょくけん	
\\	南極圏にはペンギン以外にも動物はいるんですか?	
\\	南極 
\\	南, 極, 圏	
\\	原子爆弾	
\\	げんしばくだん	
\\	の 
\\	私の家は火災保険に入っていますが、もし誰かが原子爆弾を落として、それで家が燃えちゃった場合はどうなるんでしょうか。それでも保険金は降りるんでしょうか。	
\\	原子 
\\	爆弾 
\\	原, 子, 爆, 弾	
\\	控え	
\\	ひかえ	
\\	控えの選手には誰がいたっけ?	
\\	控	
\\	渋い	
\\	しぶい	い 
\\	コウイチは
\\	攻撃に渋い顔をして、拳で机をバンバンと叩いた。	
\\	い 
\\	渋	
\\	片言	
\\	かたこと	
\\	の 
\\	私は片言の日本語でなんとかパスポートの再発行をしてもらいました。	
\\	片 
\\	こと 
\\	言葉.	片, 言	
\\	掲示板	
\\	けいじばん	
\\	この掲示板は学校のものです。	
\\	掲示 
\\	掲, 示, 板	
\\	免状	
\\	めんじょう	
\\	今度お会いする際に免状をお渡ししますね。	
\\	免, 状	
\\	言葉遣い	
\\	ことばづかい	
\\	する 
\\	誰かの言葉遣いを真似することは、外国語の良い勉強法です。	
\\	(遣う) 
\\	(言葉) 
\\	言葉 
\\	遣う 
\\	言, 葉, 遣	
\\	連邦	
\\	れんぽう	
\\	の 
\\	ちょっと待って。もしかして、連邦税と州税どっちも払わなきゃいけないの?	
\\	連, 邦	
\\	御飯	
\\	ごはん	
\\	映画を見ながら御飯を食べています。	
\\	御, 飯	
\\	群集	
\\	ぐんしゅう	
\\	する 
\\	小さな男の子がジャスティン・ビーバーの物真似をし出したので、群集は大笑いをしました。	
\\	群, 集	
\\	水仙	
\\	すいせん	
\\	お墓の真後ろに、水仙の花々が咲いていました。	
\\	水, 仙	
\\	銃殺	
\\	じゅうさつ	
\\	する 
\\	犬が飼い主に銃殺されるという悲しい事件がありました。	
\\	銃, 殺	
\\	慎重	
\\	しんちょう	
\\	な 
\\	どうしてそんなに慎重なんですか。	
\\	重. 
\\	(ちょう)
\\	慎, 重	
\\	呼び鈴	
\\	よびりん	
\\	大便の最中に呼び鈴が鳴り、男はお尻も拭かずに玄関へ急いだ。	
\\	呼ぶ 
\\	鈴
\\	呼, 鈴	
\\	甲斐	
\\	かい	
\\	努力の甲斐があって、そのニュースレターは廃刊になるのを免れた。	
\\	甲 
\\	(か). 
\\	甲, 斐	
\\	雇用	
\\	こよう	
\\	する 
\\	かつては多くの会社が終身雇用を保障していましたが、今ではそれが当たり前では無くなっています。	
\\	雇, 用	
\\	英語圏	
\\	えいごけん	
\\	シンガポールが英語圏だとは思ってもみなかったよ。	
\\	英, 語, 圏	
\\	免除	
\\	めんじょ	
\\	する 
\\	私の祖父は心臓病を患っていたため、兵役を免除されました。	
\\	免, 除	
\\	眉毛	
\\	まゆげ	
\\	三時間前に眉毛をそったばかりなのに、もう眉毛がつながりそうだ。	
\\	ドレスの色に合わせて、眉毛を染めたの。	
\\	眉毛を書くのに毎朝二時間くらいかかる。	
\\	眉, 毛	
\\	邦人	
\\	ほうじん	
\\	アメリカには在留邦人はどのくらいいるんですか。	
\\	邦, 人	
\\	群れ	
\\	むれ	
\\	春にヨーロッパ旅行をした時に、生まれて初めて真っ白な羊の群れを見ました。	
\\	(む)! 
\\	群	
\\	足跡	
\\	あしあと	
\\	雪の上に熊の足跡を見つけた瞬間、ポケットの中のナイフを握りしめました。	
\\	足, 跡	
\\	片〜	
\\	かた	
\\	片目だけ視力が悪いんです。	
\\	片	
\\	枠組み	
\\	わくぐみ	
\\	の日本語学習教材改革のための枠組みが出来上がりました。	
\\	枠 
\\	組み. 
\\	枠, 組	
\\	枠	
\\	わく	
\\	窓枠を木からステンレスのものに取り替えたい。	
\\	枠	
\\	充実	
\\	じゅうじつ	
\\	する 
\\	お客様と末永い関係が築けるよう、充実したアフターサービスをご提供しております。	
\\	充, 実	
\\	岐阜	
\\	ぎふ	
\\	週の後半に岐阜の両親のとこに顔を見せに行くつもりだよ。	
\\	岐, 阜	
\\	棋院	
\\	きいん	
\\	毎週月曜日にこの棋院に来て、囲碁を練習しています。	
\\	歴史ある棋院の前で記念写真を撮りました。	
\\	母はいつも棋院の真ん前に車を止めて、私を降ろしてくれます。	
\\	棋, 院	
\\	銃	
\\	じゅう	
\\	外出中に、部屋に置いてあった銃が母親に見つかってしまったんです。	
\\	銃	
\\	群馬県	
\\	ぐんまけん	
\\	群馬県の実家から帰ってきたら、知らせてくれよな。	
\\	群 
\\	馬 
\\	ま, 
\\	ま 
\\	うま. 
\\	群, 馬, 県	
\\	跡継ぎ	
\\	あとつぎ	
\\	父親が跡継ぎに弟を選んだという知らせは、彼にとって大きな打撃でした。	
\\	跡 
\\	継ぐ 
\\	跡, 継	
\\	仙人	
\\	せんにん	
\\	「わしゃ、お前の真上におるぞ。」と、仙人が雲の上から言いました。	
\\	仙, 人	
\\	本塁打	
\\	ほんるいだ	
\\	今日は彼の二百本目の本塁打を見るのが楽しみでしょうがないよ。	
\\	本, 塁, 打	
\\	謙譲語	
\\	けんじょうご	
\\	どうしてそれが謙譲語だと思ったんですか。	
\\	謙, 譲, 語	
\\	要項	
\\	ようこう	
\\	詳細については、添付の募集要項をご確認くださいませ。	
\\	要, 項	
\\	項目	
\\	こうもく	
\\	「それでは、アジェンダの次の項目へ進みましょう。」「ちょっと待って下さい。次へ進む前に、10分間の休憩を取りませんか?」	
\\	目. 
\\	(もく) 
\\	項, 目	
\\	跳躍	
\\	ちょうやく	
\\	する 
\\	の 
\\	この授業は、生徒たちの跳躍力を伸ばすことを目的としています。	
\\	跳, 躍	
\\	縄跳び	
\\	なわとび	
\\	802を聞きながら縄跳びをしています。	
\\	縄 
\\	跳ぶ, 
\\	縄, 跳	
\\	片仮名	
\\	かたかな	
\\	平仮名は分かりますが、片仮名はまだ覚えきれていません。	
\\	片 
\\	かた, 
\\	仮名 
\\	かな. 
\\	片, 仮, 名	
\\	飛躍	
\\	ひやく	
\\	する 
\\	彼は陰謀論が好きなんですが、彼の論理はよく飛躍するんですよね。	
\\	飛, 躍	
\\	免許	
\\	めんきょ	
\\	する 
\\	「あぁ、自分がどれだけドライブが好きか忘れてたわ。免許更新しなくちゃ。」「おい、待てよ!お前の免許失効してるのか?ありえねえ。よし、それまで!今すぐ横につけて!」	
\\	免, 許	
\\	躍動	
\\	やくどう	
\\	する 
\\	アヤの最新の
\\	キャラのイラストは、とても躍動感に溢れています。	
\\	躍, 動	
\\	事項	
\\	じこう	
\\	用紙に必要事項を記載のうえ、郵送をお願い致します。	
\\	事, 項	
\\	謙虚	
\\	けんきょ	
\\	な 
\\	その謙虚な男性が実はマフィアの一員だったなんて、信じられません。	
\\	謙, 虚	
\\	募集中	
\\	ぼしゅうちゅう	
\\	そいつ、パソコンオタクではあるけど、見た目はイケメンで、今彼女募集中なんだ。どう、会ってみない?	
\\	募, 集, 中	
\\	片手	
\\	かたて	
\\	の 
\\	彼は重いお米の袋を片手でヒョイと持ち上げました。	
\\	片, 手	
\\	片道	
\\	かたみち	
\\	の 
\\	大阪から東京までの片道切符を二枚もらえますか。	
\\	片 
\\	道 
\\	みち.	片, 道	
\\	圏外	
\\	けんがい	
\\	しまった!兄ちゃんが俺の携帯に電話するって言ってたのに、ここ圏外だわ。	
\\	圏, 外	
\\	短銃	
\\	たんじゅう	
\\	しばらく時間はかかったが、ようやく犯行の凶器となった短銃を見つけることができました。	
\\	短, 銃	
\\	隆盛	
\\	りゅうせい	
\\	な 
\\	の 
\\	アメリカで最も隆盛を極めて入る州は何州ですか。	
\\	隆, 盛	
\\	隆起	
\\	りゅうき	
\\	する 
\\	の 
\\	喉頭隆起は、その近くにある骨の形が仏様に似ていることから、喉仏としても知られています。	
\\	起. 
\\	(き). 
\\	隆, 起	
\\	充電	
\\	じゅうでん	
\\	する 
\\	もしよければ俺の車で携帯の充電ができるよ。	
\\	充, 電	
\\	勧告	
\\	かんこく	
\\	する 
\\	の 
\\	人事部に辞職勧告を受けた後もなお仕事を続けていたら、窓際族に追いやられてしまいました。	
\\	勧, 告	
\\	拒否	
\\	きょひ	
\\	する 
\\	の 
\\	彼は戦場に行く事を拒否したため、臆病者と呼ばれた。	
\\	拒, 否	
\\	稲作	
\\	いなさく	
\\	彼は稲作に没頭しています。	
\\	いな 
\\	作 
\\	稲, 作	
\\	稲田	
\\	いなだ	
\\	友達はもっと大きい稲田をもっていますよ。	
\\	いな 
\\	稲 
\\	田 
\\	稲, 田	
\\	稲	
\\	いね	
\\	メンバーはみんな稲刈りで忙しいので、
\\	のブログは休刊中です。	
\\	いね 
\\	""いいね!
\\	い).
\\	稲	
\\	銃弾	
\\	じゅうだん	
\\	少し待って貰ってもいいですか?ちょっと銃弾を取りに行かなくてはならないので。	
\\	銃, 弾	
\\	躍進	
\\	やくしん	
\\	する 
\\	が今年、大躍進を遂げたのは何故だと思いますか。	
\\	躍, 進	
\\	埼玉県	
\\	さいたまけん	
\\	どうして埼玉県に来たんですか。	
\\	埼 
\\	玉, 
\\	県.	埼, 玉, 県	
\\	仙台	
\\	せんだい	
\\	仙台に行くまでに地図を買っておきます。	
\\	仙, 台	
\\	将棋	
\\	しょうぎ	
\\	指せば指すほど将棋が好きになります。	
\\	将, 棋	
\\	棋士	
\\	きし	
\\	その棋士は、この建物に自分専用の練習部屋を構えています。	
\\	棋, 士	
\\	鈴	
\\	すず	
\\	退職をきっかけに、鈴作りを始めました。	
\\	(すず) 
\\	鈴	
\\	勧誘	
\\	かんゆう	
\\	する 
\\	大学生活の最初は、部活の勧誘をたくさん受けました。	
\\	勧, 誘	
\\	御免	
\\	ごめん	
\\	御免。流石に今日の今日で、空港まで迎えに行くのは無理だわ。もうちょっと早く言ってくれたらよかったのに。	
\\	御, 免	
\\	鋼	
\\	はがね	
\\	そのゲームで鋼の鎧っていくらだったっけ?	
\\	ね?
\\	(はがね). 
\\	鋼	
\\	製鋼	
\\	せいこう	
\\	する 
\\	賃金カットなんかしたら、製鋼所の工員達がストライキを起こすかもしれませんよ。	
\\	製, 鋼	
\\	鋼材	
\\	こうざい	
\\	鋼材不足には早急に対処する必要があります。	
\\	鋼, 材	
\\	補充	
\\	ほじゅう	
\\	する 
\\	の 
\\	帰る前にプリンタの紙を補充しておいてもらえるかな。	
\\	補, 充	
\\	免税	
\\	めんぜい	
\\	する 
\\	の 
\\	免税店がどこにあるかご存知ですか?	
\\	免, 税	
\\	狙う	
\\	ねらう	
\\	一応忠告しておくと、奴はお前のポジションを狙ってるみたいだぞ。	
\\	狙い 
\\	狙い, 
\\	狙	
\\	応募する	
\\	おうぼする	する 
\\	あいつ、ビキニ姿のギャルが見れるからっていう理由だけで、市民プールの監視員の仕事に応募したんだよ。	
\\	応募 
\\	応募
\\	応, 募	
\\	慎む	
\\	つつしむ	
\\	電車の乗客は、携帯での会話を慎むべきだ。	
\\	(つつし) 
\\	慎	
\\	埋める	
\\	うめる, うずめる	
\\	きっと誰かがもう話をしてると思うんだけど、私達、あの肉屋をあんたの庭に埋めたのさ。良かったかな。	
\\	う 
\\	埋	
\\	譲る	
\\	ゆずる	
\\	彼は、絶版になっている古い小説を私に譲ってくれました。	
\\	(ゆず). 
\\	譲	
\\	崩れる	
\\	くずれる	
\\	ああっ!またお化粧が崩れてきちゃった。	
\\	う 
\\	崩す, 
\\	崩	
\\	削除する	
\\	さくじょする	する 
\\	今お送りしたメッセージを削除して頂いてもよろしいでしょうか。別の方に送信するはずだったものなんです。	
\\	削除 
\\	削, 除	
\\	雇う	
\\	やとう	
\\	あなたを雇いたいのは山々ですが、就業許可証が発給されるのを待ってはいられないのです。	
\\	う 
\\	(やと). 
\\	雇	
\\	免れる	
\\	まぬかれる, まぬがれる	
\\	後部座席に座ってシートベルトを締めていたので、私達は重傷を免れました。	
\\	(まぬか) 
\\	免	
\\	解雇する	
\\	かいこする	する 
\\	シャークは事あるごとに規則違反をするので、そろそろ解雇する必要があると思います。	
\\	解, 雇	
\\	跳ねる	
\\	はねる	
\\	あのピョコピョコ飛び跳ねている兎が超可愛くてやばい。	
\\	跳ぶ, 
\\	は 
\\	(ねる) 
\\	跳	
\\	躍る	
\\	おどる	
\\	ずっと探していた60年代に生産中止になったクラッシックカーを見つけた時は、嬉しくて胸が躍りました。	
\\	う 
\\	(おど) 
\\	躍	
\\	御覧になる	
\\	ごらんになる	
\\	もう三階へ行かれて新しい部屋を御覧になられましたか?	
\\	(なる) 
\\	を御覧になって下さい!	
\\	御, 覧	
\\	弾く	
\\	ひく	
\\	彼がまたピアノを弾き始めたというニュースに、私は胸を躍らせました。	
\\	引く, 
\\	弾	
\\	掲載する	
\\	けいさいする	する 
\\	このコンテンツは、過激すぎて掲載することはできません。	
\\	掲載 
\\	掲載 
\\	掲, 載	
\\	勧める	
\\	すすめる	
\\	とある仙人から、仙台に行くことを勧められたんだよ。	
\\	友達にワニカニを始めるようにいつも勧めています。	
\\	こないだ仙田さんが勧めてくれた仙川のケーキ屋さん、とってもおいしかったよ!	
\\	う 
\\	薦める, 
\\	勧	
\\	拒む	
\\	こばむ	
\\	私のかかりつけのお医者さんは、私が大腸癌検査を受けるのを拒みました。	
\\	う 
\\	""子 
\\	子 
\\	拒	
\\	払い戻す	
\\	はらいもどす	
\\	恐れ入りますが、お客様は鍵を失くされたので、保証金の払い戻しは出来かねます。	
\\	"払い 
\\	戻す 
\\	払い 
\\	戻す.	払, 戻	
\\	奪う	
\\	うばう	
\\	母親は、少女に、悪霊が犬の命を奪ったのだと教えました。	
\\	う 
\\	奪	
\\	取り戻す	
\\	とりもどす	
\\	コンピュータを出荷時の設定に初期化することでこの問題は解決しますが、バックアップがなければ失われたデータを取り戻すことはできません。	
\\	取る 
\\	戻す. 
\\	取, 戻	
\\	渋滞	
\\	じゅうたい	
\\	する 
\\	もうすぐ日も暮れるぜ。全くこの馬鹿げた渋滞のせいで今日は一日何にもできなかったな。	
\\	渋, 滞	
\\	移譲	
\\	いじょう	
\\	する 
\\	校長が権力を他の人に移譲したがらないんだよ。	
\\	移, 譲	
\\	邦訳	
\\	ほうやく	
\\	する 
\\	ロード・オブ・ザ・リングの邦訳、もう図書館に返却しちゃった?	
\\	邦, 訳	
\\	甲	
\\	こう	
\\	左足の甲にタトゥーを彫りました。	
\\	甲	
\\	壁	
\\	かべ	
\\	スムージーを飲みながら壁をペンキで塗ってるところです。	
\\	壁	
\\	壁紙	
\\	かべがみ	
\\	いつも壁紙はアヤのイラストにしています。	
\\	壁 
\\	紙, 
\\	壁, 紙	
\\	戻り道	
\\	もどりみち	
\\	戻り道がどれだか分からなかったんです。	
\\	戻る 
\\	戻る 
\\	戻り 
\\	戻, 道	
\\	福祉	
\\	ふくし	
\\	日本の社会福祉制度の問題点とは何だと思いますか。	
\\	福, 祉	
\\	獲得	
\\	かくとく	
\\	する 
\\	の 
\\	今日ゲームをプレイすると、いつもの二倍のコインが獲得できるよ。	
\\	獲, 得	
\\	戒告	
\\	かいこく	
\\	する 
\\	彼女が私の尻を触ったことはセクハラに当たります。即刻戒告処分にしてください。	
\\	戒, 告	
\\	顧問	
\\	こもん	
\\	の 
\\	彼は良いバスケ部の顧問だとの評判があります。	
\\	顧, 問	
\\	主唱	
\\	しゅしょう	
\\	する 
\\	ベーコンの平和を唱える主唱者にお会いすることはできますか?	
\\	主, 唱	
\\	贈り物	
\\	おくりもの	
\\	「これ、君への贈り物。干しふぐだよ。気に入ってもらえるといいけど。」「有難う。聞いただけで既にとっても気に入ったわ。」	
\\	贈る, 
\\	贈る 
\\	物.	贈, 物	
\\	雅致	
\\	がち	
\\	壁の絵はとても雅致がありますね。どなたの作品ですか?	
\\	雅, 致	
\\	宜しくお願いします	
\\	よろしくおねがいします	
\\	どうして日本人はいつも「宜しくお願いします」って言うの?	
\\	宜しい 
\\	宜しい 
\\	お願いします.	宜, 願	
\\	宜しく	
\\	よろしく	
\\	お母様に宜しくお伝えくださいね。	
\\	宜	
\\	繁殖	
\\	はんしょく	
\\	する 
\\	の 
\\	本当に俺たちの柴犬を繁殖させることができると思う?	
\\	繁, 殖	
\\	贈賄	
\\	ぞうわい	
\\	する 
\\	これは、ある国では「プレゼント」と呼ばれるかもしれないが、この国では「贈賄」である。	
\\	贈, 賄	
\\	捕獲	
\\	ほかく	
\\	する 
\\	このメモ帳に、巨人の捕獲についての思いついたことを全部メモしています。	
\\	捕, 獲	
\\	合唱団	
\\	がっしょうだん	
\\	合唱団員は、みんなで手を繋いで歌っています。	
\\	合 
\\	がっ 
\\	ごう. 
\\	合, 唱, 団	
\\	薄情	
\\	はくじょう	
\\	な 
\\	彼は薄情者だし、どっちの味方かサッパリ分からないよ。	
\\	薄, 情	
\\	衝突	
\\	しょうとつ	
\\	する 
\\	高速道路で、車が次々と衝突しました。	
\\	衝, 突	
\\	兼用	
\\	けんよう	
\\	する 
\\	の 
\\	この食事は朝昼兼用にします。	
\\	兼, 用	
\\	内緒話	
\\	ないしょばなし	
\\	当人達は内緒話をしているつもりのようだが、声がでかいので全部筒抜けだよ。	
\\	内緒 
\\	話, 
\\	内, 緒, 話	
\\	新鋭	
\\	しんえい	
\\	な 
\\	の 
\\	その会社は最新鋭のパソコンを購入することを決定しました。	
\\	新, 鋭	
\\	洗剤	
\\	せんざい	
\\	どうして洗剤一本買うだけで三時間もかかるの。	
\\	洗, 剤	
\\	不孝	
\\	ふこう	
\\	な 
\\	自分の家族の不孝になるようなことは、絶対にしたくない。	
\\	不, 孝	
\\	堀	
\\	ほり	
\\	堀があれば、ゾンビとうまく闘えると思わないか?	
\\	堀	
\\	排水	
\\	はいすい	
\\	する 
\\	記念硬貨を排水溝に落っことしちゃった。	
\\	排, 水	
\\	優雅	
\\	ゆうが	
\\	な 
\\	彼女は自称「優雅なマダム」です。	
\\	優, 雅	
\\	排出	
\\	はいしゅつ	
\\	する 
\\	の 
\\	植物は二酸化炭素を吸収し、酸素を排出します。	
\\	排, 出	
\\	排除	
\\	はいじょ	
\\	する 
\\	の 
\\	各下水道施設は、下水排除規準を満たす必要があります。	
\\	排, 除	
\\	排他	
\\	はいた	
\\	日本記者クラブはかなり排他的だよ。	
\\	排, 他	
\\	排気	
\\	はいき	
\\	する 
\\	自動車の排気ガスの臭いが好きなのは、私達の共通点ですね。	
\\	排, 気	
\\	殿様	
\\	とのさま	
\\	暇つぶしにイケメンの殿様の絵を描きました。	
\\	殿 
\\	(との). 
\\	殿, 様	
\\	孝行	
\\	こうこう	
\\	な 
\\	女房が死ぬ前に、もっと孝行しておくんだったな。	
\\	孝, 行	
\\	頻度	
\\	ひんど	
\\	どれくらいの頻度で家に掃除機をかけますか?	
\\	頻, 度	
\\	頻発	
\\	ひんぱつ	
\\	する 
\\	みんな、頻発する地震に、ストレスが溜まりに溜まっているんだよ。	
\\	頻, 発	
\\	頻繁	
\\	ひんぱん	な 
\\	お前は頻繁にくだらない事を気にし過ぎなんだって。	
\\	頻, 繁	
\\	頻りに	
\\	しきりに	
\\	彼は「ライオンキングは子どもから大人まで楽しめるショーだ」ということを頻りに強調した。	
\\	(しき).	頻	
\\	逃亡者	
\\	とうぼうしゃ	
\\	逃亡者は逃走中に病気の子どもに会い、自首を決意したそうです。	
\\	逃, 亡, 者	
\\	俊才	
\\	しゅんさい	
\\	この問題集は俊才児を対象にして作られているので、あなたのお子さんには難しすぎますよ。	
\\	俊, 才	
\\	嬉しい	
\\	うれしい	い 
\\	今日は何だか嬉しそうですね。	
\\	百年前の今日、あなたが生まれてくれて嬉しかった。お誕生日おめでとう。	
\\	最近、嬉しいことが全然ない。	
\\	い 
\\	(うれ), 
\\	嬉	
\\	扱い	
\\	あつかい	
\\	する 
\\	ワイングラスの扱いに気をつけるようにと言われた直後に、案の定一つ割ってしまった。	
\\	扱う 
\\	扱う, 
\\	扱	
\\	嬉々	
\\	きき	
\\	子供ができればすぐ、嬉々として報告してくれるだろう。	
\\	父は、久しぶりに旧友に会い、嬉々として近況を語り合った。	
\\	コウイチが無事に帰ってきて、村には嬉々たる声が響いた。	
\\	と(して) 
\\	たる 
\\	嬉, 々	
\\	栄誉	
\\	えいよ	
\\	で仕事をするという栄誉を得たとき、そのオファーに感激して言葉に詰まりました。	
\\	栄, 誉	
\\	名誉	
\\	めいよ	
\\	な 
\\	あいつは名誉を得ることには興味がないなんて言ってるけど、俺は嘘だと思うね。	
\\	名, 誉	
\\	内堀	
\\	うちぼり	
\\	私は彼を内堀に通すために道をあけた。	
\\	内 
\\	堀 
\\	内, 堀	
\\	雅楽	
\\	ががく	
\\	雅楽の演奏会の後は、何をしましょうか。	
\\	雅, 楽	
\\	膝小僧	
\\	ひざこぞう	
\\	自転車で転んで、膝小僧が痛い。	
\\	最近、右の膝小僧にタトゥーを入れたんです。	
\\	膝小僧の調子はどうですか?	
\\	膝頭. 
\\	膝 
\\	小僧. 
\\	膝, 小, 僧	
\\	〜殿	
\\	どの	
\\	信長殿にお伝え頂きたい。	
\\	""おはようございます!こういち殿!
\\	(どの). 
\\	殿	
\\	繁茂	
\\	はんも	
\\	する 
\\	庭に雑草が繁茂しているので、この週末に草刈りをします。	
\\	繁, 茂	
\\	褒賞	
\\	ほうしょう	
\\	する 
\\	褒賞を受け取るために舞台に上がろうとした時、彼女は階段につまづきました。	
\\	褒, 賞	
\\	〜剤	
\\	ざい	
\\	解毒剤の極秘開発を行う予定だとちょいと小耳に挟みましたよ。	
\\	剤	
\\	薬剤	
\\	やくざい	
\\	の 
\\	薬剤を使い終わったら、必ず薬剤入れに戻してください。	
\\	薬, 剤	
\\	回顧録	
\\	かいころく	
\\	なんてつまらない回顧録なんだ。	
\\	回, 顧, 録	
\\	隣	
\\	となり	
\\	の 
\\	子育ては隣町でしてたんですが、子どもたちがみんな大きくなったので主人と私はこの町に越して来たんですよ。	
\\	(となり) 
\\	隣	
\\	隣人	
\\	りんじん	
\\	の 
\\	私の隣人は、幸運を呼ぶお守りを持っていることで有名です。	
\\	隣, 人	
\\	近隣	
\\	きんりん	
\\	の 
\\	コウイチ大統領は、近隣諸国との関係をないがしろにはしなかった。	
\\	近, 隣	
\\	隣国	
\\	りんごく	
\\	どうして隣国の肩を持つのさ?日本人じゃないの?	
\\	隣国 
\\	隣, 国	
\\	隣家	
\\	りんか	
\\	の 
\\	隣家ではどうやら窓は新聞紙で拭くようです。	
\\	隣, 家	
\\	沖縄県	
\\	おきなわけん	
\\	我々は、人々が沖縄県から去るのを食い止めた。	
\\	沖縄 
\\	沖縄 
\\	県 
\\	沖, 縄, 県	
\\	外堀	
\\	そとぼり	
\\	ちょっと気が変わって、外堀は必要がないようと思うようになりました。	
\\	堀 
\\	外, 堀	
\\	過敏	
\\	かびん	
\\	な 
\\	過敏性大腸症候群に苦しんでいます。	
\\	過, 敏	
\\	殿堂	
\\	でんどう	
\\	お前にとって、殿堂入りを果たした選手といえば、誰になる?	
\\	殿, 堂	
\\	駐車場	
\\	ちゅうしゃじょう	
\\	ショッピングモールの駐車場で渋滞に巻き込まれちゃってるよ。クリスマス前の日曜日なんかに、こんなところに来るんじゃなかったよ。	
\\	駐, 車, 場	
\\	巡回	
\\	じゅんかい	
\\	する 
\\	の 
\\	天気予報が外れて、巡回中に雨が降り出しました。	
\\	巡, 回	
\\	薬剤師	
\\	やくざいし	
\\	薬剤師は別に白衣を着る必要はないと思うんだよな。	
\\	薬, 剤, 師	
\\	巡礼	
\\	じゅんれい	
\\	する 
\\	世界中の聖地を巡礼しようと決心しました。	
\\	巡, 礼	
\\	柱	
\\	はしら	
\\	そのスーパーで買い物をしてる間、私は犬を柱に繋ぐんですが、いつもちゃんと大人しく待ってますよ。	
\\	(はしら) 
\\	柱	
\\	巡査	
\\	じゅんさ	
\\	その巡査は、スコッチを一気にゴクリと半分飲み干した。	
\\	巡, 査	
\\	携帯	
\\	けいたい	
\\	する 
\\	の 
\\	携帯の電源を入れるには、電源ボタンを五秒間長押しする必要があります。	
\\	携, 帯	
\\	防腐剤	
\\	ぼうふざい	
\\	古い防腐剤はどこに捨てればいいですか。	
\\	防, 腐, 剤	
\\	繁栄	
\\	はんえい	
\\	する 
\\	信じられないかもしれませんが、私が生まれ育ったこの港町は、かつては繁栄していたんですよ。	
\\	繁, 栄	
\\	機敏	
\\	きびん	
\\	な 
\\	ウェイトレスってのは機敏に動くものなのに、彼女はとんでもなく鈍いんだよ。	
\\	機, 敏	
\\	生殖	
\\	せいしょく	
\\	する 
\\	の 
\\	今、生殖器が二つある女の子と付き合ってるんだよ。	
\\	生, 殖	
\\	屋敷	
\\	やしき	
\\	男の妻は毎日お屋敷で何もせずただブラブラしている。	
\\	屋, 敷	
\\	駐在	
\\	ちゅうざい	
\\	する 
\\	海外駐在員と一緒に外国に来ている妻は、駐在さんと呼ばれることがある。	
\\	駐, 在	
\\	駐日	
\\	ちゅうにち	
\\	今夜は駐日アメリカ大使主催のパーティーに行かなくちゃいけないんです。	
\\	日本 
\\	駐, 日	
\\	褒美	
\\	ほうび	
\\	頑張ったご褒美がもらえることを期待しています。	
\\	褒, 美	
\\	犠打	
\\	ぎだ	
\\	犠打にならないか心配です。	
\\	打 
\\	だ. 
\\	犠, 打	
\\	〜房	
\\	ふさ	
\\	葡萄を三房買ってきたわよ。	
\\	房? 
\\	房!	房	
\\	敷金	
\\	しききん	
\\	敷金を全額返金してもらいました。	
\\	敷, 金	
\\	敏感	
\\	びんかん	
\\	な 
\\	耳がすっごい敏感だから、コショコショ話はしないで。	
\\	敏, 感	
\\	鋭い	
\\	するどい	い 
\\	私は森で熊に遭遇し、鋭い爪で顔を引っ掻かれました。	
\\	い 
\\	(するど) 
\\	鋭	
\\	鋭利	
\\	えいり	
\\	な 
\\	通り魔は、鋭利な刃物で私の全身を切りつけた。	
\\	鋭, 利	
\\	鋭敏	
\\	えいびん	
\\	な 
\\	その犬の鋭敏な嗅覚は、何人もの麻薬密売人を捕らえました。	
\\	鋭, 敏	
\\	衝撃	
\\	しょうげき	
\\	する 
\\	あなたは爆発の衝撃で気絶したのよ。	
\\	衝, 撃	
\\	兼業	
\\	けんぎょう	
\\	する 
\\	最近では、兼業ヤクザが増えてきています。	
\\	兼, 業	
\\	歌唱	
\\	かしょう	
\\	する 
\\	あなたの歌唱力をみんなに見せつけるために、このビデオを
\\	に投稿してもいいですか?	
\\	歌, 唱	
\\	唱歌	
\\	しょうか	
\\	する 
\\	愛国唱歌は聴きたくないね。	
\\	唱, 歌	
\\	独唱	
\\	どくしょう	
\\	する 
\\	花粉症なのに、発表会で独唱しなくちゃいけないんです。	
\\	独, 唱	
\\	獲物	
\\	えもの	
\\	都会には、我々の詐欺の獲物がたくさんいるんだよ。	
\\	獲 
\\	(え)? 
\\	獲, 物	
\\	駐留軍	
\\	ちゅうりゅうぐん	
\\	駐留軍向けの洗濯サービスとかはありますか?	
\\	駐, 留, 軍	
\\	座敷	
\\	ざしき	
\\	お座敷のどこかでコオロギが鳴いている。	
\\	座, 敷	
\\	電柱	
\\	でんちゅう	
\\	今日車を電柱にぶつけちゃったんだよね。	
\\	電, 柱	
\\	薄い	
\\	うすい	い 
\\	コピー機のインクの色が薄くなってきちゃった。新しいトナーを買わなくっちゃ。	
\\	い 
\\	(うす) 
\\	薄	
\\	茂る	
\\	しげる	
\\	妖婆の庭には雑草が生い茂っていた。	
\\	う 
\\	(しげ) 
\\	茂	
\\	透ける	
\\	すける	
\\	光っている電球の上に銀色のスクラッチカードを当てると、その銀色の部分の下に何が書かれてるかが透けて見えちゃうことがあるんだよ。	
\\	う 
\\	透	
\\	吹く	
\\	ふく	
\\	我慢するんだ!諦めるなよ。明日は明日の風が吹くんだからさ!	
\\	う 
\\	吹	
\\	唱える	
\\	となえる	
\\	背の高い魔法使いが呪文を唱えると、辺りは一瞬真っ暗になった。	
\\	う 
\\	(とな). 
\\	唱	
\\	兼ねる	
\\	かねる	
\\	できれば台所と居間を兼ねるような部屋がある方がいいなぁ。	
\\	う 
\\	(か) 
\\	兼	
\\	遠慮する	
\\	えんりょする	する 
\\	明日、健康診断なので、できれば今日はお酒はご遠慮させてください。	
\\	遠慮 
\\	遠慮.	遠, 慮	
\\	携わる	
\\	たずさわる	
\\	同僚の評価に携わるのは嫌だったが、選択の余地はなかった。	
\\	(たずさ) 
\\	携	
\\	避難する	
\\	ひなんする	する 
\\	雨が突然降り出したので、先輩と私は桜の木の下で避難しました。	
\\	避難 
\\	避難.	避, 難	
\\	駆ける	
\\	かける	
\\	先生が腕を組んだまま廊下を駆けて行くのを見て、大笑いしました。	
\\	う 
\\	(か).
\\	駆	
\\	巡る	
\\	めぐる	
\\	今日は五つのお寺を巡る予定です。	
\\	う 
\\	(めぐ) 
\\	巡	
\\	懸かる	
\\	かかる	
\\	彼の安否が気に懸かります。	
\\	(か), 
\\	懸	
\\	傾ける	
\\	かたむける	
\\	みんな、コウイチの話に、一心に耳を傾けた。	
\\	う 
\\	傾く, 
\\	傾	
\\	逃す	
\\	のがす	
\\	主人がクシャミをしたので、シャッターチャンスを逃してしまいました。	
\\	逃げる 
\\	(のが)! 
\\	逃	
\\	顧みる	
\\	かえりみる	
\\	あなた、我が子のことをもう少し顧みてあげてくれませんか。	
\\	う 
\\	帰る. 
\\	帰る 
\\	帰る 
\\	かえり.	顧	
\\	緩める	
\\	ゆるめる	
\\	どうぞネクタイをお緩めになってください。	
\\	"緩む 
\\	緩む, 
\\	緩	
\\	取り扱う	
\\	とりあつかう	
\\	その話題については、次の記事で取り扱う予定です。	
\\	取る 
\\	扱う.	取, 扱	
\\	殖える	
\\	ふえる	
\\	なんだか毛が殖えてるなあと思ったら、植毛手術をしたんだぁ!なるほどねぇ。	
\\	う 
\\	(ふ) 
\\	増える, 
\\	殖	
\\	称える	
\\	たたえる	
\\	ワニカニのレビューを毎日していたら、先生から勤勉だと褒め称えられた。	
\\	う 
\\	(たた)! 
\\	称	
\\	褒める	
\\	ほめる	
\\	どう返答していいか分からないので、人に褒められるのは苦手です。	
\\	う 
\\	(ほ) 
\\	褒	
\\	繰り返す	
\\	くりかえす	
\\	同じ失敗を何度も繰り返すのはやめようじゃないか。	
\\	"繰る 
\\	返す 
\\	繰る 
\\	返す. 
\\	繰, 返	
\\	選択する	
\\	せんたくする	する 
\\	結婚相手にどちらの女性を選択するか決めましたか?	
\\	"選択 
\\	選択.	選, 択	
\\	敷く	
\\	しく	
\\	彼に、部屋にカーペットを敷くよう説得しました。	
\\	う 
\\	(し)! 
\\	敷	
\\	戒める	
\\	いましめる	
\\	誰かが彼女の無茶苦茶な運転を戒める必要があるんじゃない?	
\\	(いまし) 
\\	戒	
\\	透明	
\\	とうめい	
\\	な 
\\	の 
\\	透明人間の男は、バスタオルで自分の体をよーく拭きました。	
\\	透, 明	
\\	警戒	
\\	けいかい	
\\	の 
\\	ちょっと警戒していることがあるんです。	
\\	警, 戒	
\\	訓戒	
\\	くんかい	
\\	する 
\\	いつも勉強会で居眠りする同僚がいてさ、ついに上司が今日それについて訓戒を与えてたよ。	
\\	訓, 戒	
\\	薄弱	
\\	はくじゃく	
\\	な 
\\	お兄さんに言われたわ。あなたは意志薄弱だから私と別れることができないけど、本当は別れたいんだって。	
\\	薄, 弱	
\\	一生懸命	
\\	いっしょうけんめい	
\\	な 
\\	遅れを取り戻すために一生懸命働かなくてはいけないのは分かっているんですが、中々難しいです。	
\\	(懸命). 
\\	生 
\\	生 
\\	しょう, 
\\	(しょう), 
\\	一, 生, 懸, 命	
\\	奥様	
\\	おくさま	
\\	お話中恐れ入りますが、奥様がエレベーターに閉じ込められてしまったようなんです。	
\\	奥, 様	
\\	投棄	
\\	とうき	
\\	する 
\\	それが不法投棄になるとは知らなかったんです。	
\\	投 
\\	投資 
\\	投, 棄	
\\	航空母艦	
\\	こうくうぼかん	
\\	その航空母艦は突然視界から消えたんです。	
\\	(航空) 
\\	航, 空, 母, 艦	
\\	徹夜	
\\	てつや	
\\	する 
\\	の 
\\	英語を徹夜で勉強するなんて、いかにもあの子らしいよね。	
\\	徹, 夜	
\\	凄い	
\\	すごい	い 
\\	生後半年で話せるようになったなんて、凄い!	
\\	雨、凄いですね。	
\\	東大に受かったなんて、凄いじゃないですか!	
\\	い 
\\	(すご). 
\\	凄	
\\	鉱山	
\\	こうざん	
\\	鉱山で働いてる時は、いつも背中に痛みがありました。	
\\	鉱, 山	
\\	廃棄	
\\	はいき	
\\	する 
\\	古いテレビの廃棄に支払うお金、五千円もあれば十分なんじゃないの?	
\\	廃, 棄	
\\	原爆	
\\	げんばく	
\\	原爆のことを言ってるんじゃないよ。	
\\	原 
\\	原, 爆	
\\	手の甲	
\\	てのこう	
\\	とても綺麗な手の甲を持っていることが彼女の自慢です。	
\\	手 
\\	甲.	手, 甲	
\\	根拠	
\\	こんきょ	
\\	する 
\\	彼が高所恐怖症だと思ったのには何か根拠があるんですか?	
\\	根, 拠	
\\	軍艦	
\\	ぐんかん	
\\	軍艦プラモの専門店を開く予定なんだよ。	
\\	軍艦!	
\\	軍, 艦	
\\	蜂	
\\	はち	
\\	スズメ蜂に腕を刺され、大きく腫れ上がりました。	
\\	蜂	
\\	偽装	
\\	ぎそう	
\\	する 
\\	の 
\\	伯父は、自社製品の表示を偽装して逮捕され、現在は刑務所にいます。	
\\	偽, 装	
\\	炭素	
\\	たんそ	
\\	の 
\\	一酸化炭素と二酸化炭素の違いは何ですか?	
\\	炭, 素	
\\	祝儀	
\\	しゅうぎ	
\\	封筒の表には「御祝儀」と自分の名前を、裏には金額を書いたよ。	
\\	祝 
\\	しゅう 
\\	祝, 儀	
\\	儀式	
\\	ぎしき	
\\	の 
\\	昨夜我が家で生贄の儀式が行われた。	
\\	儀, 式	
\\	更衣室	
\\	こういしつ	
\\	今更衣室であの女のロッカーを探しているところよ。	
\\	更, 衣, 室	
\\	炭	
\\	すみ	
\\	彼の口車に乗って、炭をしこたま購入してしまった。	
\\	(すみ)?	炭	
\\	拳骨	
\\	げんこつ	
\\	する 
\\	父は、僕に拳骨を食らわせるふりをしました。	
\\	拳 
\\	げん 
\\	げん 
\\	けん 
\\	拳, 骨	
\\	衣	
\\	ころも, きぬ	
\\	衣替えをするのって、とても面倒くさいと思うんだけど、どう思う?	
\\	ころも 
\\	ころも 
\\	""こども (子供) 
\\	(きぬ)!
\\	ころも 
\\	衣	
\\	衣服	
\\	いふく	
\\	あなたが彼女の衣服を盗んだんですね。	
\\	衣, 服	
\\	石炭	
\\	せきたん	
\\	彼は、石炭ビジネスでの成功を自慢しました。	
\\	石 
\\	宝石. 
\\	石, 炭	
\\	炭鉱	
\\	たんこう	
\\	うーん。この炭鉱が閉山するってのは、本当に残念ですね。	
\\	炭, 鉱	
\\	思い遣り	
\\	おもいやり	
\\	こんなに臭い部屋でいつまでも待たせるなんて、あいつはなんて思い遣りのないやつなんだ。	
\\	思う, 
\\	遣り 
\\	(やり). 
\\	(や), 
\\	思, 遣	
\\	潜水	
\\	せんすい	
\\	する 
\\	うちの妹は今年の夏に潜水士免許を取得したんだよ。	
\\	潜, 水	
\\	雇用者	
\\	こようしゃ	
\\	雇用者に新しい電話番号を何度も教えてるのに、いっつも古い方の番号に掛けてくるんだよな。	
\\	(雇用) 
\\	雇, 用, 者	
\\	凄絶	
\\	せいぜつ	
\\	な 
\\	殺人現場は凄絶だった。	
\\	私の祖母は戦争を経験し、凄絶な時代を生きました。	
\\	たった今、世界の凄絶さを目にした気がした。	
\\	凄, 絶	
\\	偽	
\\	にせ	
\\	の 
\\	どうして俺が偽札を偽札じゃないということが皆の為になるのさ?	
\\	(にせ) 
\\	偽
\\	偽	
\\	田畑	
\\	たはた	
\\	今時田畑を機械を使わずに耕すなんて、時間の無駄だよ。	
\\	畑 
\\	はた 
\\	田, 畑	
\\	〜畑	
\\	はたけ, はた	
\\	私は、苺畑づたいにヨロヨロと歩いている老人を見かけました。	
\\	畑	
\\	蛍	
\\	ほたる	
\\	蛍の光は、誰にとっても見ていて心地の良いものです。	
\\	蛍	
\\	拳	
\\	こぶし	
\\	彼は拳についた血をハンカチで拭いました。	
\\	(こぶし) 
\\	拳	
\\	偽造	
\\	ぎぞう	
\\	する 
\\	の 
\\	全く同じ日に、二種類の偽造紙幣が東京で出回り出しました。	
\\	偽, 造	
\\	破片	
\\	はへん	
\\	の 
\\	地面はガラスの破片だらけなので、気をつけて下さい。	
\\	破 
\\	片 
\\	(へん), 
\\	破, 片	
\\	果樹	
\\	かじゅ	
\\	果樹園での仕事は結構退屈です。	
\\	果, 樹	
\\	葬儀	
\\	そうぎ	
\\	中村は、葬儀に一人で行くと言って聞かなかった。	
\\	葬, 儀	
\\	岐阜県	
\\	ぎふけん	
\\	岐阜県に行くには、電車の乗り換えが必要ですか?	
\\	岐阜 
\\	岐, 阜, 県	
\\	郷里	
\\	きょうり	
\\	郷里を訪れるのはお金と時間の無駄です。	
\\	郷 
\\	里 
\\	(り), 
\\	(り) 
\\	郷, 里	
\\	故郷	
\\	こきょう, ふるさと	
\\	聞いたところによると、彼女、離婚後故郷に戻ったらしいぜ。	
\\	故, 郷	
\\	蜂蜜	
\\	はちみつ	
\\	彼女は私の蜂蜜プリンを一口味見しました。	
\\	蜂, 蜜	
\\	蜜	
\\	みつ	
\\	その熊は、蜜の入った瓶を五本、次々飲み干しました。	
\\	蜜	
\\	仁	
\\	じん	
\\	私の一番好きな日本の諺は、「身を殺して仁を成す」です。	
\\	仁	
\\	仁義	
\\	じんぎ	
\\	盗人には仁義はない。	
\\	仁, 義	
\\	謙遜	
\\	けんそん	
\\	する 
\\	な 
\\	の 
\\	その四十歳の
\\	社員は、自分の事を謙遜して「過去の遺物だ」なんて言っていたが、実際は全然そんなことないからね。	
\\	謙, 遜	
\\	必至	
\\	ひっし	な 
\\	の 
\\	子どもへの性教育はある時期では必至になると思うんですが、日本はそうした教育については通常どのように行われるんですか。	
\\	必, 至	
\\	侵害	
\\	しんがい	
\\	する 
\\	髭を禁止することは人権侵害です。	
\\	侵, 害	
\\	侵入	
\\	しんにゅう	
\\	する 
\\	の 
\\	このコンピュータウィルスは、大晦日の夜に時計の針が12時を告げると、何千台ものパソコンに侵入をするんですよ。	
\\	侵, 入	
\\	侵攻	
\\	しんこう	
\\	する 
\\	奴らが山のてっぺんにある俺たちの村を侵攻しようとした時、敵陣が全員高山病に罹ってしまったのには笑えたよ。	
\\	侵, 攻	
\\	鉄鉱	
\\	てっこう	
\\	マイクラでたくさん鉄鉱石を採掘した。	
\\	鉄, 鉱	
\\	躍り	
\\	おどり	
\\	私は小躍りして喜びました。	
\\	躍る 
\\	躍る.	躍	
\\	包丁	
\\	ほうちょう	
\\	包丁を研いで二、三時間潰しました。	
\\	包, 丁	
\\	嘘吐き	
\\	うそつき	
\\	の 
\\	ウソップは嘘吐きだ。	
\\	フグ子ちゃん、大嘘つきの彼氏がいて、よくケンカするんだって。	
\\	その時代、嘘吐きは全員燃やされていた。	
\\	(吐く) 
\\	吐 
\\	(つ). 
\\	嘘, 吐	
\\	嘘	
\\	うそ	
\\	の 
\\	嘘ですよね?	
\\	「嘘をつかない」は私のモットーです。	
\\	宝くじで1億円当たったなんて、嘘の話みたいで信じられない!	
\\	嘘	
\\	忠誠	
\\	ちゅうせい	
\\	な 
\\	今振り返ってみると、ハチ公は絶対的な忠誠心を持ったいい犬だったよな。	
\\	忠, 誠	
\\	礼儀	
\\	れいぎ	
\\	日本の礼儀作法について学んでいます。	
\\	礼, 儀	
\\	措置	
\\	そち	
\\	する 
\\	その生徒の今後の措置についてはどうお考えですか。	
\\	措, 置	
\\	鉱物	
\\	こうぶつ	
\\	これは今までで最高の鉱物さ。これに敵うものはないね。	
\\	鉱, 物	
\\	鉱業	
\\	こうぎょう	
\\	彼は鉱業の過去十年間を振り返りました。	
\\	鉱, 業	
\\	艦隊	
\\	かんたい	
\\	敵の艦隊からもっと激しい抵抗があると思っていたので、なんだか拍子抜けです。	
\\	艦, 隊	
\\	原潜	
\\	げんせん	
\\	原子力潜水艦は、略して原潜と呼ばれることもある。	
\\	原, 潜	
\\	鉱石	
\\	こうせき	
\\	どんな鉱石でも、彼が触ると金に変わってしまうのさ。	
\\	石 
\\	宝石. 
\\	鉱, 石	
\\	証拠	
\\	しょうこ	
\\	私が不法侵入をしたという証拠はあるんですか。	
\\	拠 
\\	こ 
\\	証, 拠	
\\	拠点	
\\	きょてん	
\\	我々は生産拠点を海外に移します。	
\\	拠, 点	
\\	瀬	
\\	せ	
\\	「沈む瀬あれば浮かぶ瀬あり」なんていう言い回しがあるのは知ってるけど、今はマジで沈んでるわ。	
\\	瀬	
\\	生き甲斐	
\\	いきがい	
\\	ロッククライミングは私の生き甲斐です。	
\\	甲斐 
\\	甲斐 
\\	生き, 
\\	生, 甲, 斐	
\\	渋々	
\\	しぶしぶ	
\\	渋々自分がやったことを認めました。	
\\	渋い. 
\\	渋, 々	
\\	放棄	
\\	ほうき	
\\	する 
\\	私は典型的な育児放棄をされた子どもで、孤児院で育ちました。	
\\	放, 棄	
\\	甲斐性	
\\	かいしょう	
\\	どうしてあの甲斐性無しが彼女の愛を復活させることができたのか、全く理解できないね。	
\\	(甲斐) 
\\	甲斐 
\\	性. 
\\	甲, 斐, 性	
\\	御手洗	
\\	おてあらい	
\\	何時までに御手洗からバスに戻らなくてはいけませんか?行列ができているので、遅れないか心配してるんですが。	
\\	手洗い, 
\\	御 
\\	お, 
\\	てあらい, 
\\	い 
\\	洗い.	御, 手, 洗	
\\	拳銃	
\\	けんじゅう	
\\	私は拳銃を派手なスパンコールで飾り付けしてみました。	
\\	拳, 銃	
\\	廃墟	
\\	はいきょ	
\\	昨日廃墟で鬼ごっこをしたので、筋肉痛になりました。	
\\	廃, 墟	
\\	瀬戸	
\\	せと	
\\	今日市場ですごくいい瀬戸物を見つけたんだよ。	
\\	瀬, 戸	
\\	高瀬	
\\	たかせ	
\\	この川はこの辺りは高瀬になってるから、きっと足が底に着くと思うよ。	
\\	高い 
\\	瀬.	高, 瀬	
\\	渋谷	
\\	しぶや	
\\	私はよく渋谷駅で外国人を見ます。	
\\	渋 
\\	渋い. 
\\	谷 
\\	(や)
\\	渋, 谷	
\\	焼酎	
\\	しょうちゅう	
\\	この焼酎のもっと大きい瓶ってありますか。	
\\	焼 
\\	酎 
\\	焼 
\\	焼, 酎	
\\	包囲	
\\	ほうい	
\\	する 
\\	の 
\\	その建物はじきに警察に包囲されるでしょう。	
\\	包, 囲	
\\	誠実	
\\	せいじつ	
\\	な 
\\	彼は僕の誠実な友人だから、心配する必要は無いよ。絶対に取引から手を引いたりなんかしないさ。	
\\	誠, 実	
\\	誠意	
\\	せいい	
\\	誠意のこもった説明が聞けたので、今回は彼女の言い分を聞いて上げることにしました。	
\\	誠, 意	
\\	誠	
\\	まこと	
\\	裁判官が彼の心神喪失による刑事責任能力の回避の申し立てを受け入れたことは、誠に遺憾です。	
\\	(こと) 
\\	(ま) 
\\	事 (まこと), 
\\	誠	
\\	社会福祉	
\\	しゃかいふくし	
\\	明日なら、家族の墓参りをした後に、その社会福祉施設に寄れなくもないよ。	
\\	福祉 
\\	社会 
\\	社, 会, 福, 祉	
\\	御免なさい	
\\	ごめんなさい	
\\	御免なさい。このTシャツの
\\	サイズは売り切れちゃったんです。	
\\	御免? 
\\	御免. 
\\	御, 免	
\\	潜在意識	
\\	せんざいいしき	
\\	の 
\\	夢ってのは、僕たちの潜在意識を反映しているとよく言われるよね。	
\\	意識 
\\	潜, 在, 意, 識	
\\	潜水艦	
\\	せんすいかん	
\\	多分あの潜水艦の中で片方のイヤリングを失くしてしまったの。	
\\	潜, 水, 艦	
\\	夏至	
\\	げし	
\\	の 
\\	父に夏至から11日目のことを何と呼ぶか知っているかと聞かれたので、得意気に「ハゲ症」って答えたんですけど、それって「ハゲ頭の症状」って意味になるみたいで。父は笑って、正しい答えは「半夏生」だよって教えてくれましたけどね。	
\\	夏 
\\	(げ) 
\\	夏, 至	
\\	至上	
\\	しじょう	
\\	彼は名門大学を主席で卒業した頭のいい男性なんでしょうが、明らかに白人至上主義者なのでそこが嫌いなんですよ。	
\\	至, 上	
\\	早瀬	
\\	はやせ	
\\	小舟は早瀬で横転してしまいました。	
\\	早い 
\\	瀬.	早, 瀬	
\\	徹底	
\\	てってい	
\\	する 
\\	彼女は徹底した化粧を施さないと、外出することができません。	
\\	底 
\\	(てい) 
\\	徹, 底	
\\	虎	
\\	とら	
\\	恐い虎の顔が描かれたお揃いの
\\	シャツを着たカップルが、今目の前を腕を組みながら歩いています。	
\\	虎	
\\	控える	
\\	ひかえる	
\\	お酒を控えるなんてらしくないね。もしかして、妊娠したとか?	
\\	う 
\\	控	
\\	至る	
\\	いたる	
\\	そういう訳で、あの化け物は今日に至るまで牢屋にぶちこまれることになったんですよ。	
\\	(いた) 
\\	至	
\\	拠る	
\\	よる	
\\	皆様のお力に拠って、成功することができました。感謝の気持ちをこめて、ベーコンをプレゼントさせて頂きます。	
\\	拠	
\\	辞儀する	
\\	じぎする	する 
\\	日本人はどうしてペコペコお辞儀ばっかりするんですか?	
\\	辞, 儀	
\\	充電する	
\\	じゅうでんする	する 
\\	充電するのを忘れたから携帯の電源が切れて、目覚ましが鳴らなかったんだ。	
\\	充電 
\\	充電.	充, 電	
\\	埋め合わせる	
\\	うめあわせる	
\\	どうにかこの損失を埋め合わせることはできないだろうか。	
\\	"埋める 
\\	合わせる 
\\	埋める 
\\	合わせる. 
\\	埋, 合	
\\	徹する	
\\	てっする	する 
\\	ギャンブルに徹する覚悟を決めました。	
\\	徹	
\\	群れる	
\\	むれる	
\\	どうして日本では、不良少年少女は、コンビニの前で群れる傾向があるんでしょうか。	
\\	群れ 
\\	群れ, 
\\	群	
\\	埋もれる	
\\	うもれる	
\\	昨夜、雪が30
\\	も積もり、車が埋もれてしまいました。	
\\	埋める 
\\	(もれる) 
\\	埋める.	埋	
\\	伺う	
\\	うかがう	
\\	残って最後まで彼の話を伺いたいのは山々なんですが、残念ながらどうしても行かなくてはいけないんです。	
\\	(うかが). 
\\	伺	
\\	侵す	
\\	おかす	
\\	たくさんの人たちが、プライバシーを侵されたことに憤慨しています。	
\\	う 
\\	(おか)? 
\\	侵	
\\	偽る	
\\	いつわる	
\\	日本では「言わぬが花」だと聞いたので、日本にいる間中自分は口がきけないと偽りました。	
\\	う 
\\	(いつわ)??? 
\\	偽	
\\	潜む	
\\	ひそむ	
\\	ワニカニはあなたの部屋の片隅に潜んでいるかもしれませんよ。	
\\	う 
\\	(ひそ) 
\\	潜	
\\	帰郷する	
\\	ききょうする	する 
\\	ほとんどの博打打ちが思っているように、彼女も自分には天分があると思っていたんだが、結局あるポーカーのゲームで全財産を失って、帰郷するはめになったんだ。	
\\	帰 
\\	(き) 
\\	帰, 郷	
\\	包む	
\\	つつむ, くるむ	
\\	睡眠薬をビニールラップで包みました。	
\\	う 
\\	(つつ) 
\\	包	
\\	樹皮	
\\	じゅひ	
\\	その樹皮は食べられますか?	
\\	樹, 皮	
\\	植樹	
\\	しょくじゅ	
\\	する 
\\	結婚の記念として、主人と私はメープルシュガーの木を植樹しました。	
\\	植, 樹	
\\	脱衣	
\\	だつい	
\\	する 
\\	脱衣する前に、鞄を脇へ置いたのは覚えています。	
\\	脱, 衣	
\\	艦船	
\\	かんせん	
\\	いったいまたどうして艦船に住むことになったんですか?月々のお家賃はいくらぐらいですか。	
\\	艦, 船	
\\	麦畑	
\\	むぎばたけ	
\\	麦畑が見えてきたとたんに、くしゃみが出始めました。	
\\	麦, 畑	
\\	冬至	
\\	とうじ	
\\	の 
\\	どうして日本では冬至の日にカボチャを食べると良いというのですか?	
\\	冬 
\\	とうきょう (とう). 
\\	冬, 至	
\\	撤去	
\\	てっきょ	
\\	する 
\\	瓦礫の撤去に関するニュースをネットで読みました。	
\\	てつ.	撤, 去	
\\	撤回	
\\	てっかい	
\\	する 
\\	の 
\\	ごめん。前言撤回するよ。	
\\	てつ.	撤, 回	
\\	撤兵	
\\	てっぺい	
\\	する 
\\	私はその撤兵を恥じた事は一度もありません。	
\\	てっ 
\\	撤, 兵	
\\	措辞	
\\	そじ	
\\	君の詩の措辞が素晴らしいことに感動しました。	
\\	措, 辞	
\\	やり甲斐	
\\	やりがい	
\\	この仕事は、誰かのために役に立つことができるので、とてもやり甲斐があります。	
\\	やり 
\\	やる 
\\	(甲斐) 
\\	甲, 斐	
\\	挑戦	
\\	ちょうせん	
\\	する 
\\	ありえねえ!お前ってほんと、超天然キャラだよな。なんで蝉の鳴き真似に挑戦なんてしてるんだよ?	
\\	挑, 戦	
\\	一括	
\\	いっかつ	
\\	する 
\\	の 
\\	日本では焼き菓子作りとか料理の計量方法が違うんだよね。いっそ世界中の計量方法が一括で同じだったらいいのにね。	
\\	いち.	一, 括	
\\	解析	
\\	かいせき	
\\	する 
\\	私は踊りながらそれを解析してみました。	
\\	解, 析	
\\	分析	
\\	ぶんせき	
\\	する 
\\	それは先生の息の質の分析結果です。	
\\	分, 析	
\\	円弧	
\\	えんこ	
\\	父親は息子に円弧を描くように命令しました。	
\\	円, 弧	
\\	到着	
\\	とうちゃく	
\\	する 
\\	の 
\\	日本に到着した時、私はちょっとテンションが高すぎましたよね。	
\\	着 
\\	執着.	到, 着	
\\	軸	
\\	じく	
\\	地軸の延長が北方で天球と交わる点で会いましょう。	
\\	軸	
\\	中軸	
\\	ちゅうじく	
\\	昨日、珊瑚虫の中軸骨格を夢に見た。	
\\	中, 軸	
\\	枢軸	
\\	すうじく	
\\	「ぐるぐるバット」は、おでこをバットの先に付けて、そのバットを枢軸としてぐるぐる回転した後に、誰が目的地まで速く辿り着けるかを競う遊びです。	
\\	枢, 軸	
\\	双眼鏡	
\\	そうがんきょう	
\\	この双眼鏡の黒はありますか?	
\\	鏡 
\\	鏡). 
\\	(きょう)!	双, 眼, 鏡	
\\	糾弾	
\\	きゅうだん	
\\	する 
\\	この金融危機について、誰が糾弾されるべきだと思われますか?	
\\	糾, 弾	
\\	中枢	
\\	ちゅうすう	
\\	の 
\\	どうしてそれが彼女の中枢神経系を刺激したの?	
\\	中, 枢	
\\	通信網	
\\	つうしんもう	
\\	時計が時刻を告げた瞬間、全ての通信網が途絶えました。	
\\	通, 信, 網	
\\	紛糾	
\\	ふんきゅう	
\\	する 
\\	この問題をこれ以上紛糾させたくないんです。	
\\	紛, 糾	
\\	克服	
\\	こくふく	
\\	する 
\\	徐々にホームシックを克服していってます。	
\\	克, 服	
\\	克明	
\\	こくめい	
\\	な 
\\	十年前の旧正月のことは、克明に覚えていますよ。	
\\	克, 明	
\\	不孝者	
\\	ふこうもの	
\\	あの不孝者は、ダイビングか何だか知らないが馬鹿な事をしに行って、潜水病になって死んじまったんですよ。	
\\	不孝 
\\	不孝 
\\	もの 
\\	者 
\\	不, 孝, 者	
\\	円滑	
\\	えんかつ	な 
\\	英語で円滑なコミュニケーションを取るために、最も重要なことは何だと思いますか?	
\\	円, 滑	
\\	範	
\\	はん	
\\	もしその野菜が全部安全だというのなら、まず先に彼らがそれを食べて我々に範を垂れるべきだ。	
\\	範	
\\	発掘	
\\	はっくつ	
\\	する 
\\	ある日、金を求めて穴を掘っていると、大量の古代ギリシャのお宝を発掘しました。そこで、その幸運を分かち合う妻を娶ろうと決意しました。	
\\	発 
\\	発, 掘	
\\	模範	
\\	もはん	
\\	君を模範としよう。	
\\	模, 範	
\\	範囲	
\\	はんい	
\\	彼は守備範囲の広いセンターで、大活躍すると踏んでいたんだが、とんだお門違いだったよ。	
\\	範, 囲	
\\	親孝行	
\\	おやこうこう	
\\	な 
\\	何か両親に親孝行してあげたいと思っています。	
\\	孝行 
\\	孝行 
\\	親. 
\\	親, 孝, 行	
\\	床	
\\	ゆか	
\\	テレビを見てる時は、大抵足を組みながら床に寝転がっています。	
\\	(ゆか) 
\\	(ゆか). 
\\	床	
\\	焦点	
\\	しょうてん	
\\	の 
\\	先生が顕微鏡の焦点の合わせ方について説明をしてくれていた時、私は上の空でした。	
\\	焦, 点	
\\	斎場	
\\	さいじょう	
\\	ちょうど斎場に到着するところです。	
\\	場 
\\	入場 
\\	斎, 場	
\\	ゴミ袋	
\\	ごみぶくろ, ゴミぶくろ	
\\	駅員さんに、電車に自転車を持ち込む場合は、持ち運び用の袋の使用が義務づけられているって言われたんだけど、代わりにゴミ袋を使っちゃだめなのかなぁ。	
\\	(ゴミ) 
\\	(ふくろ) 
\\	袋	
\\	握力	
\\	あくりょく	
\\	自分の父親なんだから、善意に解釈してあげるべきよ。例えば、年を取って握力が弱っているせいで、あなたのビンテージのバカラのグラスを割っちゃったんだ、っていう風に。	
\\	握, 力	
\\	握手	
\\	あくしゅ	
\\	する 
\\	を買うと、オマケで握手券が二枚ついてきた。	
\\	手 
\\	握, 手	
\\	掛軸	
\\	かけじく	
\\	アヤに、掛け軸のための絵を描いてもらいたいんです。	
\\	掛 
\\	掛ける, 
\\	かけ 
\\	軸 
\\	掛, 軸	
\\	肝炎	
\\	かんえん	
\\	の 
\\	あの
\\	型肝炎の患者は愛に飢えており、いつも医者や看護師の気を引こうとしている。	
\\	肝, 炎	
\\	温床	
\\	おんしょう	
\\	この公園は地元では犯罪の温床になっていることで知られているので、近づかない方がいいですよ。	
\\	温, 床	
\\	暫く	
\\	しばらく	
\\	の 
\\	悪阻がひどくて、暫く全然家事ができてないんですよね。	
\\	(しばらく). 
\\	暫	
\\	親不孝	
\\	おやふこう	
\\	な 
\\	もう手遅れですが、私は本当に親不孝な息子でしたよ。	
\\	不孝 
\\	不孝 
\\	親
\\	親, 不, 孝	
\\	潟	
\\	かた	
\\	去年の夏、秋田県の八郎潟にバス釣りに行きました。	
\\	潟	
\\	芝生	
\\	しばふ	
\\	の 
\\	この公園では、芝生の上を歩くのは禁止されています。	
\\	芝 
\\	生 
\\	ふ, 
\\	(ふ) 
\\	芝, 生	
\\	芝草	
\\	しばくさ	
\\	この芝草が、芝草クッキーを緑色にしてくれるんです。	
\\	芝, 草	
\\	芝	
\\	しば	
\\	芝に水は撒いてくれた?	
\\	芝	
\\	肝	
\\	きも	
\\	トイレに行く前に、鳥の肝をもう一口かじりました。	
\\	(きも) 
\\	肝	
\\	肝臓	
\\	かんぞう	
\\	「もし世界中の食料が無くなって、お前がすっげぇお腹がすいてるとしたら、俺の肝臓、食べるか?」「さあ、どうだろう。」「さあ、どうだろうって、それ、どういう意味だよ。」「今は分かんないってこと。」	
\\	肝, 臓	
\\	宿泊	
\\	しゅくはく	
\\	する 
\\	まだ今夜の宿泊先を探しているんです。	
\\	宿, 泊	
\\	喪失	
\\	そうしつ	
\\	する 
\\	介護施設に、どうやら記憶喪失らしいお婆さんがいるんだけど、その人がすっごく面白いのよ。	
\\	喪, 失	
\\	喪	
\\	も	
\\	の 
\\	大変申し訳ありませんが、喪に服しているため年賀状が送れないのです。	
\\	(も). 
\\	喪	
\\	福袋	
\\	ふくぶくろ	
\\	元旦に、福袋を買う列に並ぶため、妹と私は朝の四時に起きました。	
\\	福 
\\	袋, 
\\	福, 袋	
\\	国柄	
\\	くにがら	
\\	それぞれの国にはそれぞれのお国柄ってものがあって、私達はそれをお互いに尊重し合うべきだと思います。	
\\	国 
\\	くに 
\\	柄 
\\	国, 柄	
\\	網	
\\	あみ	
\\	うちの娘が虫取り網が欲しいと鼻を鳴らしてせがむもんだから、ついつい買ってあげたんだよ。	
\\	(あみ) 
\\	網	
\\	一泊	
\\	いっぱく	
\\	する 
\\	の 
\\	ねぇ、そう堅いこと言わないでさ。うちに泊まってけばいいじゃん。たったの一泊だけだよ。	
\\	一 
\\	一, 泊	
\\	不透明	
\\	ふとうめい	
\\	な 
\\	不透明な窓がある女子の更衣室です。	
\\	透明 
\\	不, 透, 明	
\\	双	
\\	そう	
\\	妹と私は二卵性双生児ですが、外見がすごく似ています。	
\\	双	
\\	括弧	
\\	かっこ	
\\	する 
\\	の 
\\	左の括弧を読む時は単に「括弧」と言いますが、右の括弧を読む時は「括弧閉じる」と言います。	
\\	括, 弧	
\\	柄	
\\	がら	
\\	彼女は私にヒョウ柄の名刺をくれました。	
\\	柄	
\\	人柄	
\\	ひとがら	
\\	な 
\\	本の内容を表紙で判断するみたいに、人の人柄を見た目で決めつけないでよ。	
\\	人 
\\	柄 
\\	人, 柄	
\\	哲学	
\\	てつがく	
\\	彼は哲学者だが、理路整然と考える人ではない。	
\\	哲, 学	
\\	携帯電話	
\\	けいたいでんわ	
\\	の 
\\	「あなたの携帯電話をお借りしてもいいですか?」「ええ、どうぞ。」	
\\	携帯 
\\	電話 
\\	携帯, 
\\	携帯 
\\	電話 
\\	携, 帯, 電, 話	
\\	挑発	
\\	ちょうはつ	
\\	する 
\\	抗議者の一人が、警察を挑発するために火炎瓶を投げつけました。	
\\	挑, 発	
\\	起床	
\\	きしょう	
\\	する 
\\	の 
\\	彼は起床時間になっても、まだいびきをかいています。	
\\	起, 
\\	(き). 
\\	起, 床	
\\	綱	
\\	つな	
\\	二つの気球の間を綱渡りした男の人の話はもうお聞きになりましたか?	
\\	綱	
\\	袋	
\\	ふくろ	
\\	子どもが学校で使う用の袋を縫いました。	
\\	袋	
\\	犠飛	
\\	ぎひ	
\\	野球用語の犠飛と犠打の意味について教えてもらえませんか?	
\\	飛 
\\	(ひ) 
\\	犠, 飛	
\\	動揺	
\\	どうよう	
\\	する 
\\	彼女の自然な美しさと魅力的な笑顔に息も出来なくなり、会議中ずっと動揺していました。	
\\	動, 揺	
\\	病床	
\\	びょうしょう	
\\	の 
\\	彼は病床に臥してついにまた起ちません。	
\\	病, 床	
\\	小柄	
\\	こがら	
\\	な 
\\	の 
\\	その店では、一人の小柄なおばあさんが買い物をしていた。	
\\	小, 柄	
\\	堅い	
\\	かたい	い 
\\	レシピには、美味しくてしっとりしたクッキーって書いてあったのに、出来上がったクッキーはすっごく堅かったんだよね。	
\\	い 
\\	堅	
\\	最新鋭	
\\	さいしんえい	
\\	の 
\\	かつて、少し前までは、3
\\	プリンタは最新鋭の機械でした。	
\\	"最新 
\\	最, 新, 鋭	
\\	書斎	
\\	しょさい	
\\	今日は一日中書斎で重要書類を探していたんですが、まだ見つからないんですよ。	
\\	書, 斎	
\\	総括	
\\	そうかつ	
\\	する 
\\	の 
\\	私は情報を総括し、上司の注意を促すために報告しましたが、彼は聞く耳を持ちませんでした。	
\\	総, 括	
\\	暫定	
\\	ざんてい	
\\	の 
\\	こちらの料金設定は暫定的ですのでご留意下さいませ。	
\\	暫, 定	
\\	干潟	
\\	ひがた	
\\	交際一ヶ月記念の日に、干潟に潮干狩りに行ってきたの。	
\\	干 
\\	(ひ) 
\\	干, 潟	
\\	焦げる	
\\	こげる	
\\	一定にかき混ぜることで、鍋の底が焦げることが防げるとお母さんが言っていたわ。	
\\	う 
\\	(子). 
\\	焦	
\\	吹き出す	
\\	ふきだす	
\\	俺の新しい髪型を見て、あいつプッと吹き出したんだぜ。失礼だよな。	
\\	吹く 
\\	出す 
\\	吹く 
\\	出す.	吹, 出	
\\	荒れる	
\\	あれる	
\\	美容師として働いている限りは、手が荒れるのはどうしようもありません。	
\\	荒	
\\	透き通る	
\\	すきとおる	
\\	彼女の手は透き通るほど白かった。	
\\	透ける 
\\	通る. 
\\	透, 通	
\\	吹き飛ばす	
\\	ふきとばす	
\\	パーティーに行って、憂鬱なんて吹き飛ばしちゃおうぜ。	
\\	吹く 
\\	飛ばす 
\\	吹く 
\\	飛ばす, 
\\	吹, 飛	
\\	吹き込む	
\\	ふきこむ	
\\	家はおんぼろで、嵐が来る度に雨風が吹き込むのが現実です。	
\\	吹く 
\\	込む 
\\	吹く 
\\	込む.	吹, 込	
\\	挑む	
\\	いどむ	
\\	私はこの日曜日、有名なベーコン鑑定家に挑みます。	
\\	う 
\\	(いど) 
\\	(いど), 
\\	挑	
\\	掘る	
\\	ほる	
\\	クママンは中年の男やもめで、家業で金を掘っています。	
\\	う 
\\	(ほ).	掘	
\\	薄める	
\\	うすめる	
\\	この味噌汁はしょっぱすぎるよ。もうちょっと薄めてもらってもいいかな?	
\\	う 
\\	薄い, 
\\	薄	
\\	紛らす	
\\	まぎらす	
\\	不快な気分を紛らすために、裁縫針で耳にピアスの穴を開けました。	
\\	う 
\\	(まぎ).	紛	
\\	逃れる	
\\	のがれる	
\\	万が一ゾンビが実際に現れた場合を想定して、どうやってゾンビから逃れるのかを考えているんです。	
\\	逃げる 
\\	逃す. 
\\	逃れる. 
\\	逃す, 
\\	逃	
\\	握る	
\\	にぎる	
\\	初めて人に会うのは手をぎゅっと握りました。	
\\	う 
\\	(にぎ). 
\\	にぎ.
\\	握	
\\	揚げる	
\\	あげる	
\\	ちょうど今からトンカツを揚げるところです。	
\\	う 
\\	あげる 
\\	上げる 
\\	あげる 
\\	挙げる 
\\	あげる 
\\	揚	
\\	逃がす	
\\	にがす	
\\	あの犯人を逃がすことはできない。	
\\	逃す 
\\	逃す 
\\	逃がす.	
\\	(にが). 
\\	逃	
\\	駆け回る	
\\	かけまわる	
\\	レシピにサフランがいるって書いてあったから、あちこち駆け回って探したんだけど見つからなくて、結局今回はサフランは使わなかったの。	
\\	駆ける 
\\	回る 
\\	駆ける 
\\	回る.	駆, 回	
\\	駆け込む	
\\	かけこむ	
\\	終電に駆け込み乗車をしようとしましたが、目の前でドアが閉まりました。	
\\	"駆ける 
\\	込む 
\\	駆ける 
\\	込む.	駆, 込	
\\	駆け出す	
\\	かけだす	
\\	彼に質問を尋ねた時、駆け出しただけだ。	
\\	"駆ける 
\\	出す 
\\	駆ける 
\\	出す.	駆, 出	
\\	揺る	
\\	ゆる	
\\	カヌーを揺らないで!	
\\	う 
\\	(ゆ), 
\\	揺	
\\	東芝	
\\	とうしば	
\\	東芝のお客様センターから届いたメールは、全文文字化けしていました。	
\\	東, 芝	
\\	滑る	
\\	すべる	
\\	雪解けは始まったけど、まだ滑りやすい場所もあるから気をつけてね。	
\\	う 
\\	(すべ)! 
\\	滑	
\\	泊まる	
\\	とまる	
\\	今までで一番面白い最高傑作のお笑いの
\\	を買ったんだけど、今日うちに泊まって一緒に観ない?	
\\	う 
\\	(と) 
\\	(と) 
\\	泊	
\\	括る	
\\	くくる	
\\	どうしてこの文章を括弧で括ったんですか?	
\\	う 
\\	(くく) 
\\	括	
\\	口笛を吹く	
\\	くちぶえをふく	
\\	口笛を吹くことにどんな魅力を感じますか。	
\\	口笛 
\\	口笛 
\\	吹.	口, 笛, 吹	
\\	交通網	
\\	こうつうもう	
\\	大地震が首都圏の交通網を麻痺させました。	
\\	交, 通, 網	
\\	綱引き	
\\	つなひき	
\\	綱引きは、日本と同じくらいアメリカでも人気がありますか?	
\\	綱 
\\	引く.	綱, 引	
\\	双子	
\\	ふたご	
\\	「男の子ですか?女の子ですか?」「実は双子の男の子なんです。」	
\\	二人, 
\\	ふた 
\\	子, 
\\	双, 子	
\\	二泊	
\\	にはく	
\\	する 
\\	の 
\\	継母がうちに二泊泊めてくれって言ってるんだけど、どう思う?	
\\	二, 泊	
\\	統括	
\\	とうかつ	
\\	する 
\\	俺たちはいつもリスク統括部の奴らとは意見が合わないんだよ。	
\\	統, 括	
\\	荒い	
\\	あらい	い 
\\	私の初めての彼氏はいつも鼻が詰まっていて、電話をする時は息づかいが荒かったです。	
\\	い 
\\	あら 
\\	荒	
\\	横綱	
\\	よこづな	
\\	今は単なる凡人のように見えるかもしれないけど、こう見えてもかつては横綱だったんだよ。	
\\	横, 綱	
\\	空襲	
\\	くうしゅう	
\\	する 
\\	の 
\\	僕が空を見上げると、ちょうど空襲が始まるところだった。	
\\	空, 襲	
\\	御札	
\\	おふだ	
\\	どうしてここに御札をはっているんですか?	
\\	これは、地元の神社でもらった御札です。	
\\	英語圏の国にも、御札ってあるんですか?	
\\	札 
\\	御 
\\	お 
\\	ふだ 
\\	御, 札	
\\	球威	
\\	きゅうい	
\\	野球のシーズンが到来する春に近づくにつれて、あの投手の球威は益々増しています。	
\\	球, 威	
\\	原子炉	
\\	げんしろ	
\\	原子炉の建設について、政府に忠告をした人もいました。	
\\	原子 
\\	炉 
\\	原, 子, 炉	
\\	襲撃	
\\	しゅうげき	
\\	する 
\\	警察の車は、暴走族に襲撃された。	
\\	襲, 撃	
\\	権威	
\\	けんい	
\\	権威を失ったのにまだ過去の栄光にしがみついていて、全く往生際が悪いったらありゃしないよ。	
\\	権, 威	
\\	凄く	
\\	すごく	
\\	ベトナムのコーヒー、凄く甘くて驚きました。	
\\	今日は、先生の気分が凄く悪くなったため授業が中止になった。	
\\	この長靴が凄く欲しいんだけど、
\\	でオーダーする価値あるかな?	
\\	凄い. 
\\	凄	
\\	軍艦島	
\\	ぐんかんじま	
\\	アイツはぜんぜん面白くないな。話題といったら軍艦島の都市伝説だけだし。	
\\	軍艦 
\\	島 
\\	軍 
\\	艦, 
\\	島. 
\\	軍, 艦, 島	
\\	高炉	
\\	こうろ	
\\	今朝の朝刊で、その鉄鋼会社は世界一大きな高炉の建設を計画していると読んだよ。	
\\	高, 炉	
\\	沼	
\\	ぬま	
\\	どっちにしろ沼に飛び込まなきゃいけないと思うしね。	
\\	ぬま.	沼	
\\	泥沼	
\\	どろぬま	
\\	コスタリカの泥沼で、忍者トカゲとしても知られているバシリスクを見ました。	
\\	泥 
\\	沼.	泥, 沼	
\\	決裂	
\\	けつれつ	
\\	する 
\\	あの男は奴らの決裂を、牢屋の中で舌なめずりをしながら待っていたに違いない。	
\\	決, 裂	
\\	朗らか	
\\	ほがらか	な 
\\	朗らかな声をした知らない人から国際電話がかかってきました。	
\\	(ほが).	朗	
\\	明朗	
\\	めいろう	な 
\\	彼はかつては明朗快活な青年だった。	
\\	明, 朗	
\\	包み	
\\	つつみ	
\\	日付が変わる瞬間に包みを開けてもいい?	
\\	(つつ). 
\\	つつ.	包	
\\	国籍	
\\	こくせき	
\\	日本国籍がないんですが、司法試験を受験することは可能ですか?	
\\	国, 籍	
\\	慰謝料	
\\	いしゃりょう	
\\	慰謝料をよこせと脅されています。	
\\	慰謝 
\\	慰, 謝, 料	
\\	慰問	
\\	いもん	
\\	する 
\\	私達は毎週日曜日、セラピー犬たちと一緒に老人ホームを慰問します。	
\\	慰, 問	
\\	慰謝	
\\	いしゃ	
\\	する 
\\	保険会社から受け取る慰謝料は、税務署に申告する必要がありますか。	
\\	慰, 謝	
\\	貢献	
\\	こうけん	
\\	する 
\\	どうして
\\	20は財政支援への貢献を否認したのですか。	
\\	貢, 献	
\\	逆襲	
\\	ぎゃくしゅう	
\\	する 
\\	パックマンは、追ってくるモンスターを逆襲して食べ始めた。	
\\	逆, 襲	
\\	珍味	
\\	ちんみ	
\\	私は痛風なのでビールを飲まない方がいいのは分かっているんですが、友人からビールに良く合う珍味をもらったので今日ばかりは我慢できませんね。	
\\	珍, 味	
\\	旨い	
\\	うまい	い 
\\	めっちゃ旨いお好み焼きの店見つけたから、今度一緒に行かへん?	
\\	い 
\\	(うま) 
\\	旨	
\\	壊滅	
\\	かいめつ	
\\	する 
\\	の 
\\	その都市は原爆により壊滅しました。	
\\	壊, 滅	
\\	絶滅	
\\	ぜつめつ	
\\	する 
\\	の 
\\	全ての種の鯨が絶滅の危機に瀕している訳ではない。	
\\	絶, 滅	
\\	露	
\\	つゆ	
\\	葉っぱの上に乗っている露を集めているんだ。	
\\	(つゆ). 
\\	露	
\\	威厳	
\\	いげん	
\\	彼はそれを威厳をもって言いました。	
\\	威, 厳	
\\	懲罰	
\\	ちょうばつ	
\\	する 
\\	の 
\\	違反者には懲罰が与えられる。	
\\	懲, 罰	
\\	暴露	
\\	ばくろ	
\\	する 
\\	暴露本の出版については現在再考中です。	
\\	暴 
\\	(ばく) 
\\	暴, 露	
\\	幻滅	
\\	げんめつ	
\\	する 
\\	彼氏が道端で立ちションしてるのを見ちゃってさ。もう、完璧に幻滅した。	
\\	幻, 滅	
\\	距離	
\\	きょり	
\\	年を取ったら長距離のフライトはきつくなるだろうね。	
\\	距, 離	
\\	滅亡	
\\	めつぼう	
\\	する 
\\	もしその超巨大火山が噴火をすれば、我々は滅亡を免れないだろう。	
\\	滅, 亡	
\\	封筒	
\\	ふうとう	
\\	金はきっちり人数分に山分けして、お前の取り分はその封筒に入れておいたぜ。	
\\	封, 筒	
\\	露出	
\\	ろしゅつ	
\\	する 
\\	思い直して、肌の露出が激しいドレスを着るのは控えることにしました。	
\\	露, 出	
\\	撲滅	
\\	ぼくめつ	
\\	する 
\\	私達はみんな、飲酒運転撲滅運動をサポートしています。	
\\	撲, 滅	
\\	偽物	
\\	にせもの	
\\	の 
\\	あの男、俺に偽物のグッチの財布を自慢してきたんだぜ。	
\\	偽, 物	
\\	濡れ衣	
\\	ぬれぎぬ	
\\	僕がおならしたって?そんなの濡れ衣だよ。	
\\	コウイチに濡れ衣を被せられた。	
\\	濡れ衣だったと信じてくれたのは、飼い犬だけだった。	
\\	濡れる 
\\	衣 
\\	濡, 衣	
\\	戸籍	
\\	こせき	
\\	私は戸籍謄本の写しが必要です。	
\\	戸 
\\	と 
\\	こ. 
\\	(こ).	戸, 籍	
\\	垣	
\\	かき	
\\	犬が逃げないように、家の周りに垣を巡らすことしたんです。	
\\	垣	
\\	封建主義	
\\	ほうけんしゅぎ	
\\	今日は歴史の授業で封建主義について学びました。	
\\	主義 
\\	封, 建, 主, 義	
\\	暖炉	
\\	だんろ	
\\	うちの子がおしっこ漏らしをしちゃって、ズボンを洗ったから、今それを暖炉の前で乾かしているのよ。	
\\	暖, 炉	
\\	露骨	
\\	ろこつ	
\\	な 
\\	彼女があまりに露骨にものを言うもので、少々面食らいました。	
\\	露, 骨	
\\	摩擦	
\\	まさつ	
\\	する 
\\	摩擦によって静電気が生じるんですか?	
\\	摩, 擦	
\\	懇話	
\\	こんわ	
\\	する 
\\	来る10月22日に、マンションの住人の皆様との懇話会を予定しております。	
\\	懇, 話	
\\	懇親	
\\	こんしん	
\\	その会社の株主総会の後には、懇親会があります。	
\\	懇, 親	
\\	柔らかい	
\\	やわらかい	い 
\\	あなたのセーターはとっても柔らかいですね。	
\\	い 
\\	(やわ) 
\\	柔	
\\	柔和	
\\	にゅうわ	
\\	な 
\\	あそこで柔和な笑みを浮かべている理髪師は、実はこの辺りのポン引き男どもの親分なんだ。	
\\	柔 
\\	(にゅう) 
\\	柔, 和	
\\	旨	
\\	むね	
\\	先生には、家庭の方針として、主人と私は子どもにそうした漫画は読ませたくないと思っているという旨を伝えました。	
\\	胸 
\\	むね.	旨	
\\	趣	
\\	おもむき	
\\	こんなに趣のあるお持ち帰り用の容器を見たのは初めてです。	
\\	(思) 
\\	(む) 
\\	(き). 
\\	趣	
\\	朗報	
\\	ろうほう	
\\	朗報を伝えるためにここに来ました。	
\\	朗, 報	
\\	垣根	
\\	かきね	
\\	男は垣根の上にそっと義手を置きました。	
\\	垣 
\\	根 
\\	垣, 根	
\\	満潮	
\\	まんちょう, みちしお	
\\	一番釣りに適しているのは満潮時でしょうか?	
\\	満, 潮	
\\	趣味	
\\	しゅみ	
\\	私の趣味は家の大掃除です。	
\\	趣, 味	
\\	即座	
\\	そくざ	
\\	の 
\\	即座の結果は期待しないでください。	
\\	即, 座	
\\	即〜	
\\	そく	
\\	すごく可愛いティディベアを見つけたので、即買いしちゃいました。	
\\	即 
\\	即	
\\	即効	
\\	そっこう	
\\	即効で目の疲れを取る方法をお教えしましょう。	
\\	即, 効	
\\	即興	
\\	そっきょう	
\\	の 
\\	みんなは彼が即興で作った詩をすごいねと褒めたが、口先だけなのは見え見えでした。	
\\	そく.	即, 興	
\\	誠に	
\\	まことに	
\\	誠に申し訳ありませんが、当ホテルは全室禁煙となっております。	
\\	事 (まこと) 
\\	誠	
\\	懇談	
\\	こんだん	
\\	する 
\\	授業参観日の先生との懇談の間、彼女はずっとハンカチを握りしめてていました。	
\\	懇, 談	
\\	伺い	
\\	うかがい	
\\	あなたのブログの
\\	をお伺いしてもよろしいですか?	
\\	伺う 
\\	伺う.	伺	
\\	即死	
\\	そくし	
\\	する 
\\	の 
\\	彼は潮津波に乗ってサーフィンをしようとしたが、残念ながら即死した。	
\\	即, 死	
\\	懇意	
\\	こんい	
\\	な 
\\	その弁護士さんとはもう何十年も懇意にしていましてね。	
\\	懇, 意	
\\	柔道	
\\	じゅうどう	
\\	もし柔道が何かを知らないって言うなら、ここに超簡単な入門書があるよ。	
\\	柔, 道	
\\	琴	
\\	こと	
\\	今けちって安い琴を買っても、どうせ後から質のいい方が欲しくなるんだから。	
\\	琴	
\\	牧場	
\\	ぼくじょう, まきば	
\\	この牧場には、絞り立ての牛乳を使った自慢のアイスクリーム屋さんがあるんだ。	
\\	牧, 場	
\\	沼地	
\\	ぬまち	
\\	の 
\\	もしかしたら、彼らがこの沼地にはドラゴンがいると言ってたのは、本当のことだったのかも。	
\\	沼, 地	
\\	沼田	
\\	ぬまた, ぬた	
\\	沼田を裸足で歩いたことはありますか。	
\\	沼, 田	
\\	山岳	
\\	さんがく	
\\	の 
\\	この山岳地帯では、少なくともあと三ヶ月は冬の季節が続くだろうね。	
\\	山, 岳	
\\	滋養	
\\	じよう	
\\	の 
\\	風邪を引いたら、大豆だとか林檎だとか、滋養のある物を食べなさい。	
\\	滋, 養	
\\	炉心	
\\	ろしん	
\\	どうして奴らは原子炉の炉心が溶解してしまったことを隠そうとしたんだ?	
\\	炉, 心	
\\	牧師	
\\	ぼくし	
\\	の 
\\	牧師さんはもう今日は教会を出ちゃいましたか?	
\\	牧, 師	
\\	牧草	
\\	ぼくそう	
\\	氷点下の日々が続いているせいで、牧草が全てやられてしまった。	
\\	草 
\\	(そう) 
\\	牧, 草	
\\	牧野	
\\	ぼくや, まきの	
\\	二、三年くらいしたら、父親が死んで、俺は父さんの牧野を相続する事になると思うよ。	
\\	牧, 野	
\\	安泰	
\\	あんたい	
\\	な 
\\	あの忌まわしい家族が未来永劫安泰でいられるなんて、誰が信じたいと思う?	
\\	安, 泰	
\\	岳	
\\	たけ	
\\	富山県には、剣の山という意味の、剣岳と呼ばれる山があります。	
\\	竹 (たけ). 
\\	岳	
\\	筒	
\\	つつ	
\\	使い終わったら、体温計を筒の中に戻しておいてね。	
\\	(つつ) 
\\	筒	
\\	要旨	
\\	ようし	
\\	授業前にその本の要旨を理解しておくと、プラスになると思うよ。	
\\	要, 旨	
\\	論旨	
\\	ろんし	
\\	彼の話の論旨があまりに不明瞭だったので、苛々しました。	
\\	論, 旨	
\\	印刷	
\\	いんさつ	
\\	する 
\\	我々の広告を上質の紙に料理のレシピと一緒に印刷すれば、みんなそれを冷蔵庫に貼っておいてくれるんじゃないかな。	
\\	印, 刷	
\\	珍	
\\	ちん	
\\	な 
\\	あなたの珍増税政策の理論的根拠を述べてくださいよ。	
\\	珍	
\\	珍しい	
\\	めずらしい	い 
\\	この辺ではベーコンの乗ったドーナッツはそう珍しいものではありませんよ。	
\\	い 
\\	(めずら). 
\\	珍	
\\	即日	
\\	そくじつ	
\\	この回転肉焼き器は、注文日の即日に発送致します。	
\\	即 
\\	日 
\\	じつ 
\\	(じつ) 
\\	即, 日	
\\	封	
\\	ふう	
\\	自分が六歳の子どもだってことは分かってたけど、父さんが勝手に封を開けて友達からの手紙を読んだことにはすごく腹が立ったわ。	
\\	ふう. 
\\	封	
\\	封書	
\\	ふうしょ	
\\	契約書にサインをして封筒に入れてありますが、まだその封書を郵送していません。	
\\	封, 書	
\\	一斉	
\\	いっせい	
\\	生徒達は一斉に温度計に目をやりました。	
\\	一.	一, 斉	
\\	慰安	
\\	いあん	
\\	する 
\\	うちの会社では、毎年強制参加の慰安旅行があります。	
\\	慰, 安	
\\	誰か	
\\	だれか	
\\	うちの家族の誰かが、母の死にあの善良な外科医が関与していると訴えたんですか?	
\\	"誰 
\\	誰か 
\\	誰	
\\	分裂	
\\	ぶんれつ	
\\	する 
\\	その政党は三派に分裂したと思っていたんだけど、実際は四派に分裂していたんだね。	
\\	分, 裂	
\\	沈滞	
\\	ちんたい	
\\	する 
\\	景気の沈滞により、たくさんの労働者が首を切られました。	
\\	沈, 滞	
\\	石垣	
\\	いしがき	
\\	彼は義足を付けていたので、石垣を乗り越えられませんでした。	
\\	石 
\\	垣 
\\	石, 垣	
\\	露店	
\\	ろてん	
\\	あの露店で買ったアクセは全部すぐ壊れちゃったよ。	
\\	露, 店	
\\	相撲	
\\	すもう	
\\	今夜ビールを飲みながら相撲を見るってのはどうだい?	
\\	すもう.	相, 撲	
\\	奇襲	
\\	きしゅう	
\\	する 
\\	の 
\\	敵の奇襲を予想できなかった。	
\\	奇, 襲	
\\	沼沢	
\\	しょうたく	
\\	この黄色い花を咲かせる植物は、沼沢地に生息します。	
\\	(しょう) 
\\	沼, 沢	
\\	潮	
\\	しお	
\\	インターネットで調べたら潮の満ち引きの時刻が分かるんじゃない?	
\\	(しお).	潮	
\\	風潮	
\\	ふうちょう	
\\	彼らは、富裕層に反対する社会的風潮を生み出した。	
\\	風, 潮	
\\	潮流	
\\	ちょうりゅう	
\\	荒い潮流に逆らってボートを漕ぐのは簡単ではないが、コテージに戻るにはそれしか方法がない。	
\\	潮, 流	
\\	襲う	
\\	おそう	
\\	俺は文字通りズボンを下げたところで、突然の激しい腹痛に襲われました。	
\\	う 
\\	襲	
\\	埋まる	
\\	うまる	
\\	カナダでは、十階建てのアパートが雪で埋まる。	
\\	埋める 
\\	埋	
\\	威張る	
\\	いばる	
\\	昨年最高額の売上を上げて歴代最年少の部長に昇進したんだから、きっと優秀なセールスマンなんだろうけど、確実に今は組織のトップのような顔をして威張り散らしているからね。私はあいつの事が嫌いだし、今後も尊敬することはないと思うよ。	
\\	威 
\\	張る.	威, 張	
\\	擦れる	
\\	すれる, こすれる	
\\	ペティコートは、スカートが肌と擦れることを防ぎます。	
\\	(す) 
\\	擦	
\\	慰める	
\\	なぐさめる	
\\	落ち込んでいると、いつもうちのワン子が慰めてくれます。	
\\	う 
\\	(なぐさ) 
\\	慰	
\\	滅ぼす	
\\	ほろぼす	
\\	奴らはメディアで情報操作を行って、日本を滅ぼそうとしているんだ。	
\\	う 
\\	滅	
\\	群がる	
\\	むらがる	
\\	空港に到着するとファンが群がっていて、驚きましたよ。	
\\	群れる, 
\\	群れる, 
\\	ら 
\\	村 (むら). 
\\	村 
\\	群	
\\	懲りる	
\\	こりる	
\\	彼女に三度も振られてるっていうのに、お前まだ懲りてないのか。	
\\	(こ) 
\\	懲	
\\	潜る	
\\	くぐる, もぐる	
\\	うちの犬は、フラフープをジャンプして潜る技を身につけました。	
\\	う 
\\	(くぐ) 
\\	潜	
\\	刷る	
\\	する	
\\	今年は年賀状何枚刷ればいいかなあ?	
\\	う 
\\	(す) 
\\	刷	
\\	沈める	
\\	しずめる	
\\	彼はソファに身を沈めながら、アカデミー賞の授賞式の生中継を観ています。	
\\	う 
\\	(しず) 
\\	沈	
\\	沈む	
\\	しずむ	
\\	なんだか沈んだ顔をしているけど、何を思い詰めてるの?	
\\	う 
\\	む, 
\\	(しず) 
\\	沈	
\\	裂く	
\\	さく	
\\	私は彼女の才能を妬んでおり、彼女が絵画コンクールに向けて絵を描いていたキャンバスを切り裂きました。	
\\	う 
\\	(さく). 
\\	裂	
\\	開封	
\\	かいふう	
\\	する 
\\	開封前の商品であればいつでも返品可って書いてあるけど、開封後に中身が壊れてたり、その商品が好きじゃなかった場合はどうなんだろう。	
\\	開, 封	
\\	誰	
\\	だれ	
\\	の 
\\	「彼女は誰かしら。」「名前を知らないの?クラスであなたの後ろに座っているよ!」	
\\	誰	
\\	浴衣	
\\	ゆかた	
\\	彼女は自分の古い考え方にしがみついて、毎晩パジャマの代わりに浴衣を着て過ごした。	
\\	ゆかた.	浴, 衣	
\\	挑戦者	
\\	ちょうせんしゃ	
\\	腕相撲の挑戦者は、握力を見せつけるために、右手で林檎を潰してみせました。	
\\	(挑戦) 
\\	挑, 戦, 者	
\\	寝床	
\\	ねどこ	
\\	「さぁ、そろそろ寝床に着く時間じゃぞ。」子供の頃、毎晩祖父にこう言われました。	
\\	床. 
\\	とこ 
\\	どこ. 
\\	(どこ) 
\\	寝, 床	
\\	刃	
\\	は	
\\	心配は要りませんよ、患者さん。私のこの手術用メスの鋭い刃で、すぐに盲腸を取ってあげますからね。	
\\	刃	
\\	金髪	
\\	きんぱつ	
\\	の 
\\	金髪になりたかったので、コカコーラで髪を脱色しました。	
\\	髪 
\\	はつ, 
\\	(はつ) 
\\	金, 髪	
\\	本棚	
\\	ほんだな	
\\	彼の本棚に、食い逃げの仕方について書かれた本を見つけたんだよね。	
\\	本, 棚	
\\	両翼	
\\	りょうよく	
\\	二遊間、外野の両翼のポジションはまだ空いています。	
\\	両, 翼	
\\	缶ビール	
\\	かんびーる, かんビール	
\\	ああ、この缶ビールはマジでうますぎる。	
\\	(ビール) 
\\	缶	
\\	缶コーヒー	
\\	かんこーひー, かんコーヒー	
\\	彼らは、とても美味い缶コーヒーを販売して、大儲けしている。	
\\	(コーヒー) 
\\	缶	
\\	大砲	
\\	たいほう	
\\	大砲がズドーンと鳴り響き、男は三人の子どもたちを残したまま逝ってしまいました。	
\\	大, 砲	
\\	携帯ストラップ	
\\	けいたいすとらっぷ, けいたいストラップ	
\\	あの携帯ストラップが気色悪すぎて、考えるだけでも吐きそうだわ。	
\\	携帯 
\\	(ストラップ) 
\\	携帯.	携, 帯	
\\	紛らわしい	
\\	まぎらわしい	い 
\\	結局それはバスケットボールだったんですが、とても紛らわしかったです。私は完全に妊婦さんの大きなお腹だと思っていました。	
\\	(まぎ) 
\\	(まぎ). 
\\	紛	
\\	笠	
\\	かさ	
\\	私は彼女の笠にこっそりエーデルワイスの花を飾りました。	
\\	笠	
\\	管制塔	
\\	かんせいとう	
\\	もし巨大な竜巻がこの地域を襲ったら、管制塔はどうなるんでしょうか。	
\\	管, 制, 塔	
\\	戸棚	
\\	とだな	
\\	今日買った単三電池は戸棚の上に置いておいたよ。	
\\	戸, 棚	
\\	恐竜	
\\	きょうりゅう	
\\	恐竜の化石を見つけるのに一番いい方法はなんでしょうか。	
\\	恐, 竜	
\\	竜	
\\	りゅう	
\\	その小説家は、55冊目の竜を題材にした小説でようやく大儲けすることができた。	
\\	竜	
\\	縁	
\\	ふち	
\\	縁が無い眼鏡はありますか?	
\\	(ふ) 
\\	(ち) 
\\	縁	
\\	刃物	
\\	はもの	
\\	刃物をちらつかせながら、男は低い声で私に「こっちへ来い」と言いました。	
\\	刃
\\	物 
\\	刃, 物	
\\	刃先	
\\	はさき	
\\	刃先を俺の首に当てながら、奴は、皆にいつもご機嫌をとられてるってのはどんな気分なのかと聞いてきたんだ。	
\\	刃 
\\	先 
\\	刃, 先	
\\	エッフェル塔	
\\	えっふぇるとう, エッフェルとう	
\\	エッフェル塔のどんなところが好きでしたか?	
\\	(エッフェル) 
\\	塔	
\\	空き缶	
\\	あきかん	
\\	三人のホームレスの老人が、公園で空き缶拾い競争を行った。	
\\	空き 
\\	開く 
\\	開く 
\\	空く, 
\\	空き.
\\	空, 缶	
\\	絶叫	
\\	ぜっきょう	
\\	する 
\\	誰かがジェットコースターの上で、「覚悟しろ!」と絶叫した。	
\\	絶, 叫	
\\	釣り	
\\	つり	
\\	父親同様、魚釣りは彼のお気に入りの気晴らしだ。蛙の子は蛙ってことだね。	
\\	釣	
\\	四匹	
\\	よんひき	
\\	うちの四匹のマルチーズは、昨日、誕生日だったので、とても美味しい晩ご飯を食べました。	
\\	四, 匹	
\\	揚げ	
\\	あげ	
\\	どんな食べ物でも、彼女の手にかかれば、美しく美味しい金色の揚げ物になるんです。	
\\	揚げる 
\\	揚げる 
\\	揚	
\\	手袋	
\\	てぶくろ	
\\	どうして裸んぼで手袋だけはめて走り回ってるの?あんたは本当に馬鹿ね。	
\\	手 
\\	袋 
\\	手, 袋	
\\	吉	
\\	きち	
\\	の 
\\	みんな、結婚式の日に吉日を選びたいのよ。	
\\	きち 
\\	吉	
\\	粒	
\\	つぶ	
\\	私は一粒の岩塩をオブラートで包みました。	
\\	(つぶ). 
\\	粒	
\\	挨拶	
\\	あいさつ	
\\	する 
\\	の 
\\	まだ小さいのに挨拶ができて良い子だね。	
\\	挨拶はコミュニケーションに欠かせない要素です。	
\\	近くにいたので、挨拶しに来ました。	
\\	挨, 拶	
\\	握り	
\\	にぎり	
\\	怒ったコウイチは、握りこぶしで机をバンバン叩いた。	
\\	握る 
\\	握る, 
\\	握	
\\	髪	
\\	かみ	
\\	生まれたての赤ちゃんは、毎日お風呂に入れてあげて髪を洗ってあげた方がいいのかな。	
\\	髪	
\\	髪型	
\\	かみがた	
\\	ビートルズみたいな髪型だね。	
\\	髪 
\\	型 
\\	髪, 型	
\\	髪の毛	
\\	かみのけ	
\\	「ねえ、サーモン。今何してんの?」「髪の毛セットしてるとこ。」「終わったら、ちょっと時間ある?」「無理〜。これが終わったらネイルもしなきゃなの。」	
\\	髪 
\\	毛, 
\\	髪, 毛	
\\	丘	
\\	おか	
\\	初めてマニュアルで運転した時、丘のてっぺんでエンストしてしまいました。	
\\	岡 
\\	丘	
\\	俺	
\\	おれ	
\\	の 
\\	やあ、コウイチ!俺、久しぶりにお前に会ったから、話したいことが山ほどあるよ。	
\\	俺	
\\	斗	
\\	と	
\\	一斗のお米が入った袋は置いてますか?	
\\	斗	
\\	左翼	
\\	さよく	
\\	の 
\\	取材であの左翼支持者の方とお会いできるのがとても楽しみです。	
\\	左 
\\	さ. 
\\	(さ) 
\\	左, 翼	
\\	縁談	
\\	えんだん	
\\	高卒以下の学歴の男との縁談を私に持って寄越すなんて、一体どういう神経してる訳?	
\\	縁, 談	
\\	寸法	
\\	すんぽう	
\\	この寸法は正しいですか?	
\\	ほう 
\\	ぽう.	寸, 法	
\\	喪服	
\\	もふく	
\\	私の妹は拒食症なのですが、喪服を着ているとより一層痩せ細ってみえました。	
\\	服 
\\	喪, 
\\	喪. 
\\	喪, 服	
\\	謎	
\\	なぞ	
\\	あの映画のその場面がいつも謎なんだよね。	
\\	謎	
\\	辛勝	
\\	しんしょう	
\\	する 
\\	は著作権侵害の裁判で辛勝しました。	
\\	辛, 勝	
\\	忍者	
\\	にんじゃ	
\\	コウイチは忍者に、正々堂々と勝負で勝った。	
\\	忍, 者	
\\	縁起	
\\	えんぎ	
\\	カラスは、日本だけでなくヨーロッパでも、縁起の悪い鳥とみなされます。	
\\	起 
\\	ぎ 
\\	縁, 起	
\\	不吉	
\\	ふきつ	
\\	な 
\\	すごく不吉な夢を見たもんで、あんたのことが心配になってこうして電話をした訳さ。	
\\	不, 吉	
\\	一匹	
\\	いっぴき	
\\	「なあ、今夜俺たちと遊ばないか?金曜日の東京の繁華街はめちゃくちゃ楽しいぜ。」「俺は一匹狼なんだよ。放っておいてくれ。」	
\\	一, 匹	
\\	寸前	
\\	すんぜん	
\\	私の会社は破産寸前である。	
\\	寸, 前	
\\	泊まり	
\\	とまり	
\\	うちの子、幼稚園のお泊まり会をすごく楽しみにしてたんだけど、風邪を引いちゃって、今抗生物質を飲ませたところなんです。だから、残念ながら今夜は欠席させて頂きます。	
\\	泊まる 
\\	泊まる, 
\\	泊	
\\	一翼	
\\	いちよく	
\\	大きなプロジェクトの一翼を担えることになりました。	
\\	一, 翼	
\\	粒子	
\\	りゅうし	
\\	光が粒子なのか波なのかを考えていて、昨日は眠れませんでした。	
\\	粒, 子	
\\	辛抱	
\\	しんぼう	
\\	する 
\\	場違いだって思うかもしれませんが、私のためだと思って辛抱してください。	
\\	抱
\\	(ぼう).	辛, 抱	
\\	桃	
\\	もも	
\\	大リーグの試合で、ピッチャーがボールの代わりに桃を投げました。	
\\	桃	
\\	桃色	
\\	ももいろ	
\\	の 
\\	桃色のバンで母親が迎えにきた時は、恥ずかしかったですね。	
\\	桃, 色	
\\	梨	
\\	なし	
\\	この洋梨のマカロンは、お好みで冷やしても美味しく御賞味頂けます。	
\\	梨	
\\	哲学者	
\\	てつがくしゃ	
\\	その哲学者は強風を通って歩いたので、自分のベレー帽をしっかりと掴まなくてはならなかった。	
\\	(学者) 
\\	哲, 学, 者	
\\	滑り台	
\\	すべりだい	
\\	甥っ子の誕生日祝いに滑り台を作ってるんですが、少々予定が遅れ気味なんですよね。	
\\	滑る 
\\	台 
\\	滑る 
\\	台.	滑, 台	
\\	娯楽	
\\	ごらく	
\\	する 
\\	の 
\\	娯楽小説を読むことの何が悪いって言うんですか?	
\\	娯, 楽	
\\	姫様	
\\	ひめさま	
\\	お迎えにあがりましたよ、僕の小さなお姫様。	
\\	姫 
\\	様 
\\	姫, 様	
\\	姫	
\\	ひめ	
\\	わかったわ。じゃあ、これからはアヤのこと、アヤ姫って呼ばないといけないの?	
\\	姫	
\\	朱印	
\\	しゅいん	
\\	この紙に御朱印を押してもらえますか?	
\\	朱, 印	
\\	謎々	
\\	なぞなぞ	
\\	子どもはみんな謎々遊びをするのが好きだろう。	
\\	謎, 々	
\\	右翼	
\\	うよく	
\\	の 
\\	自分は、極端な右翼思想を持って生まれたように感じます。	
\\	右, 
\\	(う), 
\\	右, 翼	
\\	侍	
\\	さむらい	
\\	「今の侍としての仕事が気に入らないのなら、他の仕事を探したら?」 「口で言うほど簡単じゃないよ。」	
\\	さむらい.	侍	
\\	砂丘	
\\	さきゅう	
\\	その砂丘には駱駝もいるんですよ。	
\\	砂, 丘	
\\	網戸	
\\	あみど	
\\	あの古い網戸は本当に気持ち悪いね。	
\\	網 
\\	戸 
\\	あみど.	網, 戸	
\\	棚	
\\	たな	
\\	フグが僕のことをデブって呼ぶんだよね。自分のことを棚に上げてよく言うよ。	
\\	棚	
\\	〜匹	
\\	ひき	
\\	大きい方の犬は車庫にいて、もう一匹は室内にいるよ。	
\\	匹	
\\	手堅い	
\\	てがたい	い 
\\	中々手堅い仕事みたいだから、葬儀屋になるための勉強をしようかなって思ってるんだよね。	
\\	堅い 
\\	手 
\\	堅い 
\\	手, 堅	
\\	香辛料	
\\	こうしんりょう	
\\	クミンシードは、インドカレーには欠かせない香辛料のうちの一つです。	
\\	香, 辛, 料	
\\	辛い	
\\	からい, つらい	い 
\\	「やばい〜!超辛い~!でも、このカレーすっげ〜美味しい!」	
\\	からい 
\\	つらい. 
\\	からい 
\\	(から) 
\\	つらい 
\\	(つら) 
\\	辛	
\\	芽	
\\	め	
\\	ニンニクの芽はいつもどうやって料理に使いますか?	
\\	芽	
\\	嵐	
\\	あらし	
\\	「彼女、もしかして本当は全然怒ってないんじゃない?」「ただの嵐の前の静けさだよ。」	
\\	嵐	
\\	荒波	
\\	あらなみ	
\\	の 
\\	アイツはまだ社会の荒波に揉まれた事のないただの若造だよ。	
\\	荒 
\\	波 
\\	荒, 波	
\\	寸	
\\	すん	
\\	1寸は1尺の10分の1で、約3.03センチです。	
\\	寸	
\\	涙	
\\	なみだ	
\\	面白過ぎて片腹が痛いし、涙も出てきたよ。	
\\	波だ (なみだ).	涙	
\\	感涙	
\\	かんるい	
\\	どんな時に人は感涙にむせび泣くと思いますか。	
\\	感, 涙	
\\	雷	
\\	かみなり	
\\	雷の音で、洗濯物を取り入れなくちゃいけないことを思い出しました。	
\\	雷 
\\	神成り (かみなり) 
\\	雷	
\\	落雷	
\\	らくらい	
\\	する 
\\	落雷が直撃したのに、奇跡的に無傷だった。	
\\	落, 雷	
\\	雷雨	
\\	らいう	
\\	今朝の強烈な雷雨で停電になったので、今はインターネットが使えません。	
\\	雨, 
\\	う. 
\\	雷, 雨	
\\	缶	
\\	かん	
\\	私のお父さんは桃の缶詰工場で働いています。	
\\	缶	
\\	紛れる	
\\	まぎれる	
\\	解熱剤が他の薬に紛れちゃってて、どれがどれだか分かりません。	
\\	紛らす 
\\	紛らす.	紛	
\\	焦る	
\\	あせる	
\\	焦るなよ。小便に行きたい以外は、急ぐ必要はないよ。一日一日を着実にな。	
\\	(焦げる) 
\\	る, 
\\	(あせ) 
\\	焦	
\\	揺れる	
\\	ゆれる	
\\	今あの子のおっぱいがボインボイン揺れてるのを見てたよね?	
\\	揺る 
\\	(れる) 
\\	揺る.	揺	
\\	焦がす	
\\	こがす	
\\	いつもトーストを真っ黒に焦がしてしまう。	
\\	焦げる 
\\	(がす) 
\\	焦げる, 
\\	焦	
\\	克服する	
\\	こくふくする	する 
\\	三歳の頃犬に噛まれてから、ずっと犬に恐怖心を抱いていたんですが、この小さな仔犬に出会って犬嫌いを克服しました。	
\\	"克服 
\\	克, 服	
\\	忍ぶ	
\\	しのぶ	
\\	父さん、どうして僕たちはアイツらの侮辱を忍ばなきゃならないの?	
\\	う 
\\	(しの).	忍	
\\	取り逃がす	
\\	とりにがす	
\\	犯罪者を取り逃がすふりをするのは、ちょっとやり過ぎじゃない?	
\\	取る 
\\	逃がす.	取, 逃	
\\	釣る	
\\	つる	
\\	後で何匹釣れたかお知らせしますね。	
\\	う 
\\	釣	
\\	荒らす	
\\	あらす	
\\	昨夜車上荒らしにあって、マックを盗まれました。	
\\	"荒れる 
\\	荒れる.	荒	
\\	泊める	
\\	とめる	
\\	うちに知らない人を泊めるのは嫌です。	
\\	"泊まる 
\\	泊まる.	泊	
\\	見逃す	
\\	みのがす	
\\	その法律に抜け穴があることを今の今まで見逃していました。	
\\	"逃す 
\\	見 
\\	逃す.	見, 逃	
\\	竜巻	
\\	たつまき	
\\	の 
\\	昨夜、竜巻警報が出て、停電にもなりました。	
\\	竜 
\\	(立つ) 
\\	まき 
\\	巻く, 
\\	竜, 巻	
\\	魚雷	
\\	ぎょらい	
\\	小包を開けると、中には自動追尾魚雷が入っていた。	
\\	魚, 雷	
\\	叱る	
\\	しかる	
\\	二重瞼の整形手術をした時、親から貰った顔に傷をつけたと両親からこっぴどく叱られました。	
\\	う 
\\	叱	
\\	揺する	
\\	ゆする	
\\	僕の伯父さんは、今まで蜂に刺されたことがなくて、どんなものか試してみたかったから、蜂の巣がある木を揺すったんだそうです。	
\\	揺る.	揺	
\\	揺さぶる	
\\	ゆさぶる	
\\	コウイチが木を揺さぶったので、ビエットは枝に掴まった。	
\\	う 
\\	揺る.	揺	
\\	叫ぶ	
\\	さけぶ	
\\	「感情を吐き出すために叫ぶのが終わったら、ちゃんと電気を消してね。」「了解。」	
\\	う 
\\	(さけ).	叫	
\\	塔	
\\	とう	
\\	五重の塔の売買契約の手付金を支払いました。	
\\	塔	
\\	バベルの塔	
\\	ばべるのとう, バベルのとう	
\\	バベルの塔に辿り着く一番速い方法はなんですか?	
\\	バベル
\\	塔 
\\	塔	
\\	無縁	
\\	むえん	
\\	な 
\\	の 
\\	この田舎道は、渋滞とは無縁です。	
\\	無, 縁	
\\	朱	
\\	あけ, しゅ	
\\	この布を朱に染めたいんです。	
\\	朱.	
\\	あけ 
\\	開ける, 
\\	(あけ) 
\\	朱	
\\	俺たち	
\\	おれたち	
\\	俺たち、生き別れになっていた妹に会いに、来月北海道に行くんだ。	
\\	たち 
\\	私達? 
\\	俺 
\\	俺	
\\	俺ら	
\\	おれら	
\\	の 
\\	ごめん。 俺らがこの会話をするのは明らかにすごく早すぎたよな。でも、ただ、俺はお前と結婚したいと思ってるってこと、言いたかったんだ。	
\\	ら 
\\	俺たち 
\\	俺	
\\	荒々しい	
\\	あらあらしい	い 
\\	彼は荒々しい声で、寝ている時に奥さんが毎晩顔に小便を掛けてくることに怒り心頭していることを公言しました。	
\\	荒い 
\\	荒々しい 
\\	荒い. 
\\	々 
\\	荒, 々	
\\	翼	
\\	つばさ	
\\	つけたい翼を選んで下さいね。	
\\	(つばさ). 
\\	翼	
\\	いつ頃	
\\	いつごろ	
\\	はいつ頃
\\	ブックマーケットに参入するつもりですか?	
\\	"いつ 
\\	頃 
\\	頃	
\\	警鐘	
\\	けいしょう	
\\	研究者たちは、どれだけ速く氷山が溶け出しているかについて警鐘を鳴らしている。	
\\	警, 鐘	
\\	年頃	
\\	としごろ	
\\	うちの年頃の娘に一体何をしてくれたんだ?	
\\	年 
\\	頃, 
\\	年, 頃	
\\	頃	
\\	ころ, ごろ	
\\	今日のお昼の三時頃、この公園でツキノワグマが目撃されました。	
\\	頃	
\\	凶悪	
\\	きょうあく	な 
\\	誰もが、その凶悪犯が刑務所で大人しく刑期を務めるのは無理だろうと思っていました。	
\\	凶, 悪	
\\	細菌	
\\	さいきん	
\\	最近真剣に細菌について勉強をし始めました。	
\\	細, 菌	
\\	悪趣味	
\\	あくしゅみ	
\\	な 
\\	母は、父が悪趣味な人形を集めることを禁止しました。	
\\	趣味 
\\	悪, 趣, 味	
\\	鐘	
\\	かね	
\\	私が旅行者だと言うと、男は鐘の形をしたお守りの値段を明らかに釣り上げたが、何だかそいつに悪い気がして、そのまま何も言わずに言われた通りの値段を払ったんだよ。	
\\	(かね) 
\\	鐘	
\\	賭け	
\\	かけ	
\\	する 
\\	あいつとの賭けに勝ったよ。	
\\	賭	
\\	舟	
\\	ふね	
\\	いさり舟をハンマーで叩いて破壊した。	
\\	船). 
\\	船.	舟	
\\	小舟	
\\	こぶね	
\\	太平洋上には小舟がひしめきあっていた。	
\\	小 
\\	こ 
\\	小, 舟	
\\	嫁	
\\	よめ	
\\	靴下を裏返しに履いちゃうと、嫁が怒ってうるさいんだわ。	
\\	嫁	
\\	電卓	
\\	でんたく	
\\	頼むから電卓を叩くのをやめてくれ。気が狂いそうになるだろ!	
\\	電, 卓	
\\	暦	
\\	こよみ	
\\	うちの会社は暦通りの営業になります。	
\\	子読み (こよみ) 
\\	暦	
\\	天井	
\\	てんじょう	
\\	ある日天井に小さな穴が空いてるのを見つけたんですが、ストーカーがたまに屋根裏に上ってその穴から私のことを覗いていたことが後から判明したんです。	
\\	天, 井	
\\	霊園	
\\	れいえん	
\\	家族でうちの犬をペット専用霊園に連れて行きました。	
\\	霊, 園	
\\	西暦	
\\	せいれき	
\\	昭和は西暦何年に終わったんですか?	
\\	カレンダー 
\\	西, 暦	
\\	曇り	
\\	くもり	
\\	実は、眩しい午後の日差しの下よりも、曇りの日の方が写真撮影には適しています。	
\\	曇	
\\	全裸	
\\	ぜんら	
\\	の 
\\	一度しかない人生だろ!なぁ!一緒に全裸でスカイダイビングしに行こうぜ。すんげぇいい経験になると思うんだよね。	
\\	全, 裸	
\\	可也	
\\	かなり	
\\	な 
\\	日本で初恋の人にバッタリ出くわした時は、可也驚きました。	
\\	可, 也	
\\	塾	
\\	じゅく	
\\	塾に何か問題でもあるの?	
\\	塾	
\\	塾生	
\\	じゅくせい	
\\	彼は塾生たちの中で一番記憶力が悪かった。	
\\	学生 
\\	塾, 生	
\\	呪い	
\\	のろい	
\\	魔女はお姫様に、コップを逆さまにしか持てなくする呪いをかけた。	
\\	呪	
\\	凶器	
\\	きょうき	
\\	「アチョーーーーーー!」と叫び声を上げながら、コウイチはビエトの手から凶器を叩き落とした。	
\\	凶, 器	
\\	食卓	
\\	しょくたく	
\\	の 
\\	私がお皿を洗うから、あなたは食卓を綺麗にしてくれる?いいかしら?	
\\	食, 卓	
\\	排水溝	
\\	はいすいこう	
\\	お風呂の排水溝が髪の毛で詰まっちゃったみたいなの。	
\\	排, 水, 溝	
\\	入籍	
\\	にゅうせき	
\\	する 
\\	結婚式の前に入籍をする予定です。	
\\	戸籍 
\\	入, 籍	
\\	悪霊	
\\	あくりょう, あくれい	
\\	悪霊がいると聞いたので、トンネルの中ではスピードを上げて運転しました。	
\\	霊 
\\	りょう, 
\\	悪, 霊	
\\	疲労	
\\	ひろう	
\\	する 
\\	今納税の手続きをしているんだけど、もう疲労困憊だよ。	
\\	疲, 労	
\\	肌色	
\\	はだいろ	
\\	の 
\\	その男は強引な押し売りで私に肌色のクレヨンを一箱売りつけました。	
\\	肌 
\\	色 
\\	肌, 色	
\\	ばい菌	
\\	ばいきん	
\\	ホームレスだからって、俺がまるでばい菌のように人にジロジロ見られるのは、とても腹立たしい。	
\\	""ばい
\\	ばい 
\\	黴, 
\\	菌	
\\	亡霊	
\\	ぼうれい	
\\	亡霊はめげずに、人々を怖がらせ続けました。	
\\	亡, 霊	
\\	陰気	
\\	いんき	
\\	な 
\\	アイツの
\\	ビデオを見るまでは、てっきり陰気な男だと思っていたよ。	
\\	陰, 気	
\\	鳥肌	
\\	とりはだ	
\\	「ゴキブリ」という名前を聞くだけで、鳥肌が立ちます。	
\\	鳥 
\\	肌.	鳥, 肌	
\\	溝	
\\	みぞ	
\\	コインパーキングの横の溝に、万札が落ちてるのを見つけたんだよ。	
\\	水. 
\\	みぞ 
\\	みず, 
\\	溝	
\\	半裸	
\\	はんら	
\\	きゃー!たった今夢が叶っちゃった!半裸のイケメンが淹れたての珈琲を飲みながら、私が目を覚ました瞬間に「おはよう」って言ってくれるっていう夢が!	
\\	半, 裸	
\\	肌	
\\	はだ	
\\	その温泉に入った後は、お肌がツルツルになるよ。	
\\	肌	
\\	加湿器	
\\	かしつき	
\\	最近加湿器の調子が悪いんだよね。	
\\	加, 湿, 器	
\\	狩人	
\\	かりゅうど	
\\	狩人が週休二日制だってのは初耳だな。	
\\	狩, 人	
\\	狩り	
\\	かり	
\\	来週はずっと山で狩りをしている予定です。	
\\	狩	
\\	霊感	
\\	れいかん	
\\	彼女は非常に霊感の強い霊能力者で、あの世の人と話をすることができます。	
\\	霊, 感	
\\	黒潮	
\\	くろしお	
\\	黒潮の流域とは正確にはどこになりますか?	
\\	黒 
\\	潮.	黒, 潮	
\\	近頃	
\\	ちかごろ	
\\	近頃は製菓工場で夜勤の仕事をしているんだ。	
\\	近い 
\\	頃 
\\	近, 頃	
\\	脚	
\\	あし	
\\	夜になると脚がムズムズして、よく眠れないんです。	
\\	足, 
\\	脚	
\\	遠距離	
\\	えんきょり	
\\	の 
\\	彼女はとても素晴らしい遠距離走選手だよ。ほら、今も急にペースを上げたから、前のランナーに追いつくと思う。	
\\	距離 
\\	遠, 距, 離	
\\	塊	
\\	かたまり	
\\	店に入ると、カウンターにドンと置かれた肉の塊がまず一番に目についた。	
\\	塊	
\\	旧暦	
\\	きゅうれき	
\\	旧暦の正月といえばいつになりますかね。	
\\	旧, 暦	
\\	狂気	
\\	きょうき	
\\	の 
\\	その申し出を断るなんて狂気の沙汰だよ。	
\\	狂, 気	
\\	手頃	
\\	てごろ	
\\	な 
\\	手頃な値段ですね。	
\\	手, 頃	
\\	湿気	
\\	しっけ	
\\	日本の梅雨は湿気が多いので嫌いです。	
\\	気.	湿, 気	
\\	裸	
\\	はだか	
\\	の 
\\	キモイ!今あそこの横断歩道で、裸に袖無しエプロンを着た男を見かけたんだけど。警察に通報した方がいいかな。	
\\	(はだか), 
\\	裸	
\\	核分裂	
\\	かくぶんれつ	
\\	核分裂の過程について、分かりやすいスライドショーを作ってもらえますか。	
\\	"分裂 
\\	核, 分, 裂	
\\	海溝	
\\	かいこう	
\\	鰻がフィリピン海溝の近くで産卵することは、フィリピン人に知られているのでしょうか。	
\\	海, 溝	
\\	若い頃	
\\	わかいころ	
\\	若い頃は食べ物について気にしなかったけど、今はお医者さんに糖尿病予備軍だって言われたので、食べ物に気をつけるようになったし、いつもレシピに書かれてるよりも少ない砂糖を入れるようにもなりました。	
\\	若い 
\\	頃 
\\	若, 頃	
\\	小包	
\\	こづつみ	
\\	この小包は、ほんの友情の印です。	
\\	小 
\\	包み, 
\\	み 
\\	づ 
\\	小, 包	
\\	磨き	
\\	みがき	
\\	のスキルに磨きをかけて、仕事が見つかればいいな、と思っています。	
\\	磨	
\\	義塾	
\\	ぎじゅく	
\\	義塾に通い始めた初日、思いっきり場違いだと感じました。	
\\	義, 塾	
\\	眺望	
\\	ちょうぼう	
\\	する 
\\	そのホテルの部屋からは美しく青い海を眺望することができました。	
\\	眺, 望	
\\	滝川	
\\	たきがわ	
\\	祖母に、滝川を泳ぐのは危険だと言われました。	
\\	滝 
\\	川. 
\\	滝, 川	
\\	先頃	
\\	さきごろ	
\\	つい先頃までとても仲良くしてたのに、彼が急によそよそしくなったんだけど、理由が全然分からないの。	
\\	先 
\\	頃 
\\	頃 
\\	ごろ 
\\	先, 頃	
\\	この頃	
\\	このごろ	
\\	この頃、風味を出すために砕いたヘーゼルナッツを珈琲に加えてるんです。	
\\	(この) 
\\	頃 
\\	頃	
\\	硬直	
\\	こうちょく	
\\	する 
\\	死後どのぐらいで死後硬直が起きるんでしょうか。	
\\	硬, 直	
\\	元凶	
\\	げんきょう, がんきょう	
\\	人類の半分を一掃した凶悪なウィルスの元凶は、ウォッカのつまった西瓜だった。	
\\	元, 凶	
\\	脚本	
\\	きゃくほん	
\\	他の脚本では出来なかった事を、彼の脚本はやってのけたんだよ。	
\\	脚, 本	
\\	陰	
\\	かげ	
\\	陰でこそこそ私の陰口言うの、止めてくれない?言いたいことがあるなら、はっきり言ってよ!	
\\	(かげ), 
\\	陰	
\\	近距離	
\\	きんきょり	
\\	タクシーの運ちゃんは大体みんな近距離乗車は嫌がるよね。	
\\	距離 
\\	近 
\\	近, 距, 離	
\\	短距離	
\\	たんきょり	
\\	ほんの数年後、彼女はようやく自分がその短距離走で優勝した短距離走者だったことを認めました。	
\\	距離 
\\	短, 距, 離	
\\	一斉に	
\\	いっせいに	
\\	派遣社員たちは、景気の悪化で一斉に解雇されることを心配しています。	
\\	"一斉 
\\	に 
\\	一斉.	一, 斉	
\\	魂	
\\	たましい	
\\	小さな天使とベーコンのビデオを観て、魂が震えました。	
\\	(たましい). 
\\	魂	
\\	矛	
\\	ほこ	
\\	この美しい矛は、白金とプラチナで作られています。	
\\	子. 
\\	矛	
\\	〜魂	
\\	こん	
\\	死者の霊魂に捧げる為のイラストを描いてほしいとアヤに頼みました。	
\\	魂	
\\	殺菌	
\\	さっきん	
\\	する 
\\	の 
\\	トイレを殺菌する一番いい方法は何ですか?	
\\	殺, 菌	
\\	無菌	
\\	むきん	
\\	の 
\\	のステッカーが無菌かどうかの問い合わせのメールが来たんだけど。	
\\	無, 菌	
\\	硬い	
\\	かたい	い 
\\	なんて硬いベーコンなんだと思ってたら、夢を見ながらスマートフォンを噛んでたみたいなのよね。どうりて硬いはずだわさ。	
\\	い 
\\	固い. 
\\	硬	
\\	卓	
\\	たく	
\\	麻雀するなら、うちに雀卓があるけど。	
\\	卓	
\\	卓球	
\\	たっきゅう	
\\	あの卓球の音のせいで気が変になりそうだ!	
\\	たく
\\	卓, 球	
\\	肌触り	
\\	はだざわり	
\\	このタオルはとても柔らかくて心地よい肌触りです。	
\\	肌 
\\	触る
\\	肌, 触	
\\	潮時	
\\	しおどき	
\\	今こそタバコをやめる潮時だ。今すぐニコチンパッチを買いに行くんだ。	
\\	潮 
\\	時. 
\\	潮, 時	
\\	襲撃する	
\\	しゅうげきする	する 
\\	飛行機で巨人を襲撃することができるようになるとは、知る由もなかった。	
\\	襲撃 
\\	襲, 撃	
\\	澄む	
\\	すむ	
\\	池の水、昨日は濁って緑色だったのに、今はすっかり澄んでいるじゃないか。一体何があったの?	
\\	(す) 
\\	澄	
\\	湿る	
\\	しめる	
\\	昨日の夜、寝汗をたくさんかいちゃったから、ベットが何だか湿ってるのよね。	
\\	う 
\\	(しめ). 
\\	湿	
\\	滅びる	
\\	ほろびる	
\\	地球がじきに滅びるんだとしたら、どうしてベーコンを食べるのを我慢しなくちゃいけないの?	
\\	う 
\\	(びる). 
\\	(ほろ) 
\\	滅	
\\	疲れる	
\\	つかれる	
\\	トイレが詰まって、すっぽんで詰まり抜きをする前に水が溢れちゃって、朝早くから掃除をしなくちゃいけなかったから、とっても疲れました。	
\\	う 
\\	(つか) 
\\	疲	
\\	呪う	
\\	のろう	
\\	夫が心的外傷後ストレス障害と診断された後、私はそれが誰であれその戦争を始めた人を呪いました。	
\\	う 
\\	呪	
\\	稼ぐ	
\\	かせぐ	
\\	現職の大統領が、前大統領よりもどれだけ多くのお金を稼いでいるか知っているかい?	
\\	う 
\\	(かせ). 
\\	稼	
\\	翔る	
\\	かける	
\\	それは、まるで天空を高く翔る不死鳥のようでした。	
\\	う 
\\	翔	
\\	賭ける	
\\	かける	
\\	彼女は競馬で、中々の大金を賭けるんだよ。	
\\	う 
\\	賭	
\\	曇る	
\\	くもる	
\\	ラーメンの湯気で眼鏡が曇っちゃうのが好きじゃないんだよね。	
\\	う 
\\	雲, 
\\	曇	
\\	狂う	
\\	くるう	
\\	「サーモン、今週末、ラスベガスで結婚しよう。」「フグ、気でも狂ったの?」	
\\	う 
\\	(来る) 
\\	狂	
\\	嫁ぐ	
\\	とつぐ	
\\	嫁ぎたての頃は、息子の嫁はかまととぶっていました。	
\\	う 
\\	(とつ), 
\\	嫁	
\\	磨く	
\\	みがく	
\\	何で切れた電球なんか磨いてるの?	
\\	う 
\\	磨	
\\	擦る	
\\	する, こする	
\\	花粉症で目を擦ってたら、片方のコンタクトレンズが破けました。	
\\	う 
\\	擦れる 
\\	擦れる 
\\	こする. 
\\	する
\\	すれる
\\	擦	
\\	眺める	
\\	ながめる	
\\	うちの甥っ子は新幹線を見るのが好きで、一日中眺めていることもできます。	
\\	う 
\\	長い, 
\\	なが, 
\\	ながめる
\\	眺	
\\	引き裂く	
\\	ひきさく	
\\	あのカップルの仲をうまく引き裂く事ができたら、お知らせ致します。	
\\	裂く 
\\	引く 
\\	裂く.	引, 裂	
\\	滝	
\\	たき	
\\	一位のチームには、賞品として滝修行が授与されます。	
\\	滝	
\\	井戸	
\\	いど	
\\	よく冷えた井戸水を飲むと、冷たいアイスクリームを食べた時のように頭がキーンと痛みました。	
\\	戸, 
\\	ど 
\\	と. 
\\	井 
\\	(い), 
\\	井, 戸	
\\	湿地	
\\	しっち	
\\	の 
\\	確か、北海道には湿地が多いんだよね?	
\\	湿, 地	
\\	包み紙	
\\	つつみがみ	
\\	この包み紙には点字が施してあるわ。	
\\	包み 
\\	紙 
\\	包み 
\\	紙 
\\	包, 紙	
\\	瞬き	
\\	まばたき, またたき	
\\	する 
\\	瞬きせずに、この絵を見続けてください。	
\\	(まばた) 
\\	瞬	
\\	気泡	
\\	きほう	
\\	一番最初にしなくちゃいけないことが、陶芸用の粘土から気泡を取り除くことです。	
\\	気, 泡	
\\	錬金術	
\\	れんきんじゅつ	
\\	ある錬金術士が鉛を金に変えようとして失敗した日、外は雨がザザ降りでした。	
\\	錬, 金, 術	
\\	瞬間	
\\	しゅんかん	
\\	蜘蛛の巣にぶつかった瞬間に、人は突然空手の達人へと変身する。	
\\	瞬, 間	
\\	一瞬	
\\	いっしゅん	
\\	「お前には捕まらないぞ!」「捕まえたっ!」「うわっ!速い!一瞬だったな。でも、大人げないよ。子どもなんだからもう少し手加減してよ。」	
\\	一, 瞬	
\\	歳入	
\\	さいにゅう	
\\	どの政府にも、特別歳入基金はあるものなのですか。	
\\	歳, 入	
\\	癖	
\\	くせ	
\\	突き指が癖になってしまっているんです。	
\\	癖	
\\	万歳	
\\	ばんざい	
\\	する 
\\	トフグのコウイチに万歳三唱をしましょう!万歳!万歳!万歳!	
\\	万
\\	ばん 
\\	まん. 
\\	万, 歳	
\\	零下	
\\	れいか	
\\	の 
\\	零下十度だろうが二十度だろうがに関係なく、洗い物はここでします。	
\\	零, 下	
\\	魔術	
\\	まじゅつ	
\\	の 
\\	"「好き嫌いしないの!黒魔術だけじゃなくて白魔術もちゃんと勉強しなさい!」「はいはい。いつも同じこと言ってるよね。」「""はい""は一回でしょ!」
\\	魔, 術	
\\	鈍器	
\\	どんき	
\\	毎月の鈍器代はいくらですか。	
\\	鈍, 器	
\\	寮生	
\\	りょうせい	
\\	「新しい寮生はセクシー?」「ああ、セクシーだよ。フグなんて、すでに彼女に熱を上げているよ。」	
\\	学生) 
\\	寮, 生	
\\	書架	
\\	しょか	
\\	わっ!うちの父さん、お前のと全く同じ書架を持ってるぜ。お前も
\\	で買ったの?	
\\	書, 架	
\\	泡	
\\	あわ	
\\	マリファナを一服した瞬間、男は口から泡を吹いて地面に倒れ込んだ。	
\\	(あわ) 
\\	泡	
\\	穏やか	
\\	おだやか	
\\	な 
\\	彼女は穏やかに眠っているようだった。	
\\	な 
\\	(おだ) 
\\	穏	
\\	平穏	
\\	へいおん	
\\	な 
\\	の 
\\	俺は平穏無事に暮らすことが嫌いだ。	
\\	平, 穏	
\\	紛れもない	
\\	まぎれもない	い 
\\	彼女は紛れもない田舎っぺの女の子だ。	
\\	"紛らわしい 
\\	ない), 
\\	紛らわしい.	紛	
\\	椅子	
\\	いす	
\\	する 
\\	この椅子の色違いは置いてますか?	
\\	子 
\\	す, 
\\	(す) 
\\	椅, 子	
\\	租界	
\\	そかい	
\\	昔の上海とフランス租界について知りたいのなら、彼がそれについては詳しいよ。	
\\	租, 界	
\\	幽閉	
\\	ゆうへい	
\\	する 
\\	専業主婦になってからというもの、まるで捕虜になって、台所に幽閉されてでもいるかのようだわ。	
\\	幽, 閉	
\\	架設	
\\	かせつ	
\\	する 
\\	今まで一度も橋梁を架設したことがありません。	
\\	架, 設	
\\	涼しい	
\\	すずしい	い 
\\	残念ながら、今日はとても涼しかった。	
\\	い 
\\	(すず). 
\\	涼	
\\	涼風	
\\	りょうふう	
\\	窓を開けると、心地よい涼風とともに美しいピアノの音色が舞い込んできた。	
\\	涼, 風	
\\	綿	
\\	わた, めん	
\\	綿でも、ジャケットは手洗いにした方がいいと思う。	
\\	わた 
\\	(わた) 
\\	綿	
\\	木綿	
\\	もめん	
\\	木綿のハンカチをプレゼントしてもらいました。	
\\	木. 
\\	く 
\\	もく 
\\	もめん.	木, 綿	
\\	綿布	
\\	めんぷ	
\\	水をよく吸収するので、私は綿布を布巾に使います。	
\\	綿, 布	
\\	揚げ出し	
\\	あげだし	
\\	これは出来たてホカホカの美味しい揚げ出し豆腐ですよ。	
\\	揚げる 
\\	揚げる 
\\	出す
\\	揚, 出	
\\	魔	
\\	ま	
\\	今は「魔の二歳児」ですからねー。	
\\	魔	
\\	悪魔	
\\	あくま	
\\	今夜は悪魔と出かける気分じゃないの。	
\\	悪, 魔	
\\	寮	
\\	りょう	
\\	もしこの寮にご両親を宿泊させたいのであれば、この宿泊カードに記入をお願いします。	
\\	寮	
\\	鈍い	
\\	にぶい, のろい	い 
\\	現在妊娠中で、お腹に時々鈍い痛みを感じるんですが、これは普通ですか。	
\\	い 
\\	(にぶ) 
\\	にぶい 
\\	のろい, 
\\	(のろ) 
\\	鈍	
\\	素粒子	
\\	そりゅうし	
\\	「素粒子」って言葉は物理の教科書で見た事がある気がするよ。	
\\	(粒子) 
\\	素, 粒, 子	
\\	皇帝	
\\	こうてい	
\\	皇帝はお酒には目がない。	
\\	皇, 帝	
\\	帝国	
\\	ていこく	
\\	帝国では、殺人事件の時効は廃止されました。	
\\	帝, 国	
\\	歳暮	
\\	せいぼ	
\\	本当にちゃんと上司へのお歳暮持ったの?	
\\	せい, 
\\	歳, 暮	
\\	誇張	
\\	こちょう	
\\	する 
\\	レビューで今までで最高のカメラだとか何とか書かれてたけど、それはやっぱり誇張だったね。	
\\	誇, 張	
\\	誇大	
\\	こだい	
\\	な 
\\	四人に一人が広告は誇張されていると思っていることは知っていますが、私は全部が全部誇大広告ではないと思うんです。	
\\	誇, 大	
\\	眼孔	
\\	がんこう	
\\	蒸しタオルで眼孔を覆うと気持ち良いですよ。	
\\	眼, 孔	
\\	口癖	
\\	くちぐせ	
\\	「何々が旬の季節です」ってあのシェフの口癖だよね。	
\\	癖 
\\	口, 癖	
\\	水泡	
\\	すいほう	
\\	データを保存する前にブラウザがクラッシュしてしまったため、長文のメッセージを書いていた努力が水泡に帰してしまった。	
\\	水, 泡	
\\	発泡	
\\	はっぽう	
\\	する 
\\	の 
\\	発泡スチロールはこの袋に入れてもらってもいい?他のゴミとは分別して捨てるんだよね。	
\\	発, 泡	
\\	幽霊	
\\	ゆうれい	
\\	「今日、学校サボってゲーセン行こうぜ。人生は一度っきりだしな!」「そうかな。僕はよく幽霊を見るから、死後の世界を信じてるんだけど。」	
\\	幽, 霊	
\\	碁	
\\	ご	
\\	年寄りたちと碁は打ちたくないんだよね。	
\\	碁	
\\	囲碁	
\\	いご	
\\	彼は囲碁トーナメントの決勝に進みました。	
\\	囲, 碁	
\\	碁盤	
\\	ごばん	
\\	濡れ衣だよ。じいちゃんの碁盤を壊したのは俺じゃないよ。	
\\	碁, 盤	
\\	碁会所	
\\	ごかいしょ, ごかいじょ	
\\	今朝俺たちの碁会所を覗き見していた男を取っ捕まえてやったよ。	
\\	碁, 会, 所	
\\	穂	
\\	ほ	
\\	私達はお腹がペコペコだったので、麦の穂を集めてそのまま生で食べたんです。	
\\	穂	
\\	稲穂	
\\	いなほ	
\\	お米は私達の主食ですが、だからといって日本人全員が風に靡く美しい金色の稲穂を見た事がある訳ではありません。	
\\	稲, 穂	
\\	病棟	
\\	びょうとう	
\\	あんたの病棟で、かっこいい男は見た?	
\\	病, 棟	
\\	吾輩	
\\	わがはい	
\\	の 
\\	吾輩が貴殿に久しく手紙を書いていなかったことをどうかお許しください。	
\\	吾, 輩	
\\	黙殺	
\\	もくさつ	
\\	する 
\\	警察はその地域では多くの犯罪を黙殺しています。	
\\	黙, 殺	
\\	焦り	
\\	あせり	
\\	会議に行かなくちゃいけなかったのに、脚がやばいくらいつって全然動けなかったので、焦りを感じましたよ。	
\\	焦る, 
\\	焦る.	焦	
\\	帝政	
\\	ていせい	
\\	の 
\\	ローマ帝政時代、皇帝はその村の人達に目をつけていました。	
\\	帝, 政	
\\	墨	
\\	すみ	
\\	おっと。墨が校章についちゃった。	
\\	墨	
\\	墨絵	
\\	すみえ	
\\	の 
\\	彼女は、完成したと同時にその墨絵を真っすぐ立てました。	
\\	墨, 絵	
\\	帝	
\\	みかど	
\\	人々は、帝がお出でになるということで、国旗を掲げました。	
\\	帝 
\\	(みかど) 
\\	帝	
\\	鍵	
\\	かぎ	
\\	いや〜鍵を中に置いたまま車に鍵をかけちゃってさ、今鍵屋さんが来るのを待ってるんだよ。	
\\	鍵	
\\	斬殺	
\\	ざんさつ	
\\	する 
\\	僕のお気に入りのアニメキャラクターが斬殺されてしまった。	
\\	斬, 殺	
\\	魔法	
\\	まほう	
\\	わぁ、
\\	級の新しい魔法を習得中なの?上手くいけば、万歳ものだね!	
\\	魔, 法	
\\	庄園	
\\	しょうえん	
\\	庄園制度導入への準備は全て整っています。	
\\	庄, 園	
\\	零時	
\\	れいじ	
\\	零時頃、彼女が町中を一人で猛スピードで歩いているのを見かけました。	
\\	零, 時	
\\	瞬時	
\\	しゅんじ	
\\	言いたかったことを彼が瞬時にズバッと言ってくれたので、とても感心しました。	
\\	瞬, 時	
\\	寸暇	
\\	すんか	
\\	夫は、いつも寸暇を惜しんで働いています。	
\\	寸, 暇	
\\	猿	
\\	さる	
\\	その飼育員さんが、猿たちが囲いから出ることを防ぎました。	
\\	猿	
\\	猿真似	
\\	さるまね	
\\	猿真似をするのはよせ。みっともない。	
\\	真似 
\\	猿 
\\	真似 
\\	猿, 真, 似	
\\	辛子	
\\	からし	
\\	あなたすごく考えすぎてるのよ。さっさと辛子を口の中に放り込んじゃいなさいよ。きっとそこまで辛くないわよ。	
\\	辛い (からい) 
\\	子 
\\	辛, 子	
\\	斬新	
\\	ざんしん	
\\	な 
\\	すごく斬新なアイディアなので、後者よりも最初の案の方がいいと思います。	
\\	斬, 新	
\\	お盆	
\\	おぼん	
\\	今日の宿題は、英語で「お盆」の説明を考えてくる事です。	
\\	盆 
\\	お 
\\	盆	
\\	阻止	
\\	そし	
\\	する 
\\	その証人が法廷で証言するのを、何としてでも阻止するんだ。	
\\	阻, 止	
\\	阻害	
\\	そがい	
\\	する 
\\	の 
\\	カフェインを含む飲み物は、睡眠を阻害します。	
\\	阻, 害	
\\	お握り	
\\	おにぎり	
\\	私は息子がコンビニのお握りを食べることを禁止しました。	
\\	"握り 
\\	お 
\\	握り.	握	
\\	鳩	
\\	はと	
\\	鳩たちは、一斉に飛び立ちました。	
\\	鳩	
\\	丁寧	
\\	ていねい	
\\	な 
\\	有名人なのに、私の説明に丁寧に答えてくれて、すごく感激しました。	
\\	丁 
\\	てい 
\\	ちょう, 
\\	(てい) 
\\	丁, 寧	
\\	穏当	
\\	おんとう	
\\	な 
\\	彼が嫌いだと言うわけじゃないんだけど、あそこであの発言はちょっと穏当でないなと思ったわけよ。	
\\	穏, 当	
\\	瞳孔	
\\	どうこう	
\\	いつ、どうして瞳孔の縮小と拡張が起きるのか、説明できますか?	
\\	瞳, 孔	
\\	瞳	
\\	ひとみ	
\\	父の再婚相手の気怠い瞳を見ると、いつもたじろいでしまいます。	
\\	(ひと) 
\\	(見) 
\\	ひとみ.	瞳	
\\	寧ろ	
\\	むしろ	
\\	お姫様よりも寧ろ猫になりたい。	
\\	(むし), 
\\	寧	
\\	清涼	
\\	せいりょう	
\\	な 
\\	十分間にどれだけの炭酸入り清涼飲料水を飲めると思いますか?	
\\	清, 涼	
\\	土俵	
\\	どひょう	
\\	誰かが土俵にパンの欠片を投げ込みました。	
\\	土, 俵	
\\	俵	
\\	たわら	
\\	俵先生は、自分の名前が俵なので、あの金の米俵のキーホルダーを鞄に付けていると言っていましたよ。	
\\	(たわら). 
\\	俵	
\\	叫び	
\\	さけび	
\\	この人形に単四電池を入れて電源を入れると、叫び出すんです。	
\\	"叫ぶ 
\\	叫ぶ.	叫	
\\	叫び声	
\\	さけびごえ	
\\	マミは叫び声を上げながら、ベーコンを上下に動かしました。	
\\	叫ぶ 
\\	声, 
\\	叫, 声	
\\	沈黙	
\\	ちんもく	
\\	する 
\\	の 
\\	気まずい沈黙の後、私は塩の入った小瓶を掴みました。	
\\	沈, 黙	
\\	辛口	
\\	からくち	
\\	の 
\\	カレーはいつも、甘口、中辛、辛口、どれを買いますか?	
\\	庄司さんは、辛口の酒が好きです。	
\\	みんなは辛口コメントだったけど、私はこの映画好きだな。	
\\	辛, 口	
\\	新芽	
\\	しんめ	
\\	このペースじゃ、夕方までにお茶の新芽を摘み終わるのは無理じゃないかな。	
\\	新, 芽	
\\	担架	
\\	たんか	
\\	あそこ家の息子さん、交通事故に巻き込まれて担架で病院に担ぎ込まれたんだけど、結局植物状態になっちゃったんだって。	
\\	担, 架	
\\	入れ墨	
\\	いれずみ	
\\	する 
\\	その入れ墨には今までいくら費やしているんですか。	
\\	入れる 
\\	墨 
\\	入, 墨	
\\	租税	
\\	そぜい	
\\	田んぼには特定の租税が課されているんですか?	
\\	租, 税	
\\	鍛錬	
\\	たんれん	
\\	する 
\\	発音が一晩で上手くなる人なんていません。発音を上達させるには、継続的かつ正しい鍛錬が必要です。	
\\	鍛, 錬	
\\	鈍感	
\\	どんかん	
\\	な 
\\	あんたって、ほんと鈍感。今はフグの話はしたくないの。それくらい空気を読んでよ。	
\\	鈍, 感	
\\	歳月	
\\	さいげつ	
\\	それから半世紀の歳月が流れ、彼のことなんてほとんど忘れかけていた時に、私達は小さなパン屋さんで偶然再会したんです。	
\\	歳, 月	
\\	誇る	
\\	ほこる	
\\	この短編映画では、日本が世界に誇るロボット産業をテーマとしています。	
\\	う 
\\	子 (ほこ) 
\\	誇	
\\	瞬く	
\\	またたく, まばたく	
\\	辺りの景色は瞬く間に雪に覆われ、真っ白になった。	
\\	う 
\\	(またた) 
\\	瞬	
\\	斬る	
\\	きる	
\\	その侍は、早口言葉が上手く言えなければお前の舌を斬るぞ、と私に言いました。	
\\	う 
\\	切る, 
\\	斬	
\\	鍛える	
\\	きたえる	
\\	生まれたての頃からミッチリ鍛えたおかげで、息子は今ピアノの天才児と呼ばれています。	
\\	う 
\\	(きた) 
\\	鍛	
\\	揺らぐ	
\\	ゆらぐ	
\\	それっぽっちのことで、気持ちが揺らぐんだ。	
\\	揺る, 揺する, 
\\	揺れる 
\\	揺	
\\	黙る	
\\	だまる	
\\	どうしたの?どうして黙っているの?	
\\	う 
\\	(だま) 
\\	黙	
\\	零す	
\\	こぼす	
\\	「あんた、今カーペットに珈琲を零したでしょう?」「しまった!ごめん!」「ごめんで済んだら警察はいらないわよ!」	
\\	う 
\\	零	
\\	阻む	
\\	はばむ	
\\	誰かが、彼が今日娘さんを連れてバージンロードを歩くのを阻もうとしているようなんです。	
\\	う 
\\	阻	
\\	二十歳	
\\	はたち	
\\	「僕は二十歳になる前に社長になるよ。」「そうだといいけどね。」	
\\	はたち. 
\\	(はたち) 
\\	二, 十, 歳	
\\	不穏	
\\	ふおん	
\\	な 
\\	今夜の三日月は、何だか不穏な感じがするわ。	
\\	不, 穏	
\\	白菊	
\\	しらぎく	
\\	家の前に大量の白菊を並べるなんて、迷惑行為以上のことだよ。だって、白菊は日本では死者を弔う花だからね。言ってる意味が分かりますか?	
\\	白 
\\	しら 
\\	しろ, 
\\	白, 菊	
\\	魔女	
\\	まじょ	
\\	の 
\\	「ちょっと言ってもいいかな?」「もちろん。」「あのさ…やっぱいいや。」「ほら、白状しなって!」「えっと、実は私魔女なんだよね。」	
\\	魔, 女	
\\	〜歳	
\\	さい	
\\	三十歳を過ぎてから、全身がすごく痒いんだよね。	
\\	歳	
\\	零	
\\	れい	
\\	の 
\\	うちの母親は、僕が零点を取ると、キーッと癇癪を起こします。	
\\	零 
\\	零	
\\	零点	
\\	れいてん	
\\	最大公約数を公分母にするということが全く理解できず、試験で零点を取ってしまった。	
\\	零, 点	
\\	補佐	
\\	ほさ	
\\	する 
\\	の 
\\	私は
\\	営業部の補佐として働きました。	
\\	補, 佐	
\\	裸足	
\\	はだし	
\\	の 
\\	水虫になりたくないなら、スリッパを裸足で履かない方がいいですよ。	
\\	裸 (はだか) 
\\	足 (あし) 
\\	裸 
\\	はだ 
\\	足 
\\	し. 
\\	(はだし) 
\\	裸, 足	
\\	胴体	
\\	どうたい	
\\	の 
\\	君は今までみた中で一番胴体が長いよ。	
\\	胴, 体	
\\	挿絵	
\\	さしえ	
\\	アイツの話によると、彼女、本の挿絵を描いてるんだってさ。	
\\	挿絵 
\\	そう 
\\	さし. 
\\	絵, 
\\	挿, 絵	
\\	箸	
\\	はし	
\\	口コミがあっという間に広がって、その折りたたみ式の陶器のお箸はとても人気が出ました。	
\\	箸	
\\	明瞭	
\\	めいりょう	
\\	な 
\\	さすがベテランアナウンサーなだけあって、彼女はとてもよく聞き取れる明瞭な声でその質問に答えました。	
\\	明, 瞭	
\\	粘土	
\\	ねんど	
\\	粘土を買おうかなと思ってるの?	
\\	粘, 土	
\\	崖	
\\	がけ	
\\	崖っぷちではさすがに携帯圏外だったわ。	
\\	(がけ) 
\\	崖	
\\	粘着	
\\	ねんちゃく	
\\	する 
\\	どうして粘着テープを宝石箱に仕舞っているの?	
\\	着, 
\\	執着. 
\\	粘, 着	
\\	大佐	
\\	たいさ	
\\	大佐にお会いできるのを楽しみにしています。	
\\	大, 佐	
\\	川柳	
\\	せんりゅう	
\\	ここにあなたの川柳を書いてもらえませんか?	
\\	川 
\\	(せん), 
\\	川, 柳	
\\	炊事	
\\	すいじ	
\\	する 
\\	僕たちは、炊事をする男を雇った。	
\\	炊, 事	
\\	自炊	
\\	じすい	
\\	する 
\\	の 
\\	忙しいけど、ちゃんと自炊してるよ。実際、昨日の夜はラタトゥイユっていう野菜の煮込み料理を作ったよ。	
\\	自, 炊	
\\	芯	
\\	しん	
\\	お買い得だったので、
\\	のシャー芯を百ケース買いました。	
\\	芯	
\\	〜畳	
\\	じょう	
\\	5畳で毎月のお家賃が一万円のこの部屋は、早い者勝ちの物件です。	
\\	畳 
\\	畳	
\\	布巾	
\\	ふきん	
\\	誰かが棒のてっぺんに、日の丸の代わりに布巾をはためかせたみたいですね。	
\\	布, 巾	
\\	門扉	
\\	もんぴ	
\\	私が門扉の前で待っていると、誰かが割り込んできました。	
\\	門, 扉	
\\	水滴	
\\	すいてき	
\\	車の窓には水滴がついています。	
\\	水, 滴	
\\	雑巾	
\\	ぞうきん	
\\	コウイチに雑巾を売り込もうと電話を掛けたんだけど、全く相手にされなかったよ。	
\\	雑 
\\	ざつ 
\\	ぞう. 
\\	象 (ぞう). 
\\	雑, 巾	
\\	稼ぎ	
\\	かせぎ	
\\	稼ぎがそんなに多くないので、何によくお金を使っているか習慣を見直して節約するために家計簿をつけています。	
\\	"稼ぐ 
\\	稼ぐ.	稼	
\\	扇風機	
\\	せんぷうき	
\\	時計のチクタクという音と、扇風機のブーンという音しか聞こえなかった。	
\\	扇, 風, 機	
\\	伊達	
\\	だて	
\\	な 
\\	伊達巻きはおせち料理の中で私が好きなものの一つです。	
\\	(だて) 
\\	伊, 達	
\\	挿話	
\\	そうわ	
\\	挿話を十カテゴリーに分類する予定です。	
\\	挿, 話	
\\	墜落	
\\	ついらく	
\\	する 
\\	あのジェット機の墜落は人為的ミスが原因だったと思う?	
\\	墜, 落	
\\	霧	
\\	きり	
\\	七夕の日、朝は霧がかかっていましたが、お昼には晴れました。	
\\	霧	
\\	扇子	
\\	せんす	
\\	この扇子は特価で買ったんですよ。	
\\	扇, 子	
\\	詐欺	
\\	さぎ	
\\	の 
\\	詐欺を働いて告訴された経験はおありですか。	
\\	詐, 欺	
\\	扉	
\\	とびら	
\\	おい、ふざけている場合か!扉を開けろ!	
\\	(飛び) 
\\	(ら). 
\\	扉	
\\	お婆さん	
\\	おばあさん	
\\	この度は本当にご愁傷様です。あなたのお婆さんはとても素敵な人だったので、あなたのお爺さんの代わりに私が一緒にいられればいいのにといつも願っていたほどですよ。残念ながら願いは叶いませんでしたがね。	
\\	お 
\\	(ばあ).	婆	
\\	掌握	
\\	しょうあく	
\\	する 
\\	恐らく気づいていないと思うが、コウイチ大統領は全てを掌握しているんだよ。	
\\	掌, 握	
\\	帽子	
\\	ぼうし	
\\	こないだ海でフグって名前のめちゃくちゃ大きい男に会ったんだけど、その人シルクハットみたいな帽子被っててさぁ、…あ、そういう感じの帽子、分かる?	
\\	帽, 子	
\\	〜狩り	
\\	かり, がり	
\\	鹿狩り用の見張り台は、今は大きなブルーシートで覆われています。	
\\	狩り 
\\	狩り, 
\\	狩	
\\	挿入	
\\	そうにゅう	
\\	する 
\\	このパラグラフの間にアヤのイラストを挿入するのはどうでしょうか。	
\\	挿, 入	
\\	点滴	
\\	てんてき	
\\	する 
\\	点滴を受けた後、彼は綿密な診察を受けました。	
\\	点, 滴	
\\	伊勢	
\\	いせ	
\\	伊勢神宮には、一度行ってみた方がいいよ。	
\\	せ, 
\\	せい 
\\	伊, 勢	
\\	塊魂	
\\	かたまりだましい	
\\	最近は塊魂というゲームをプレイしています。	
\\	塊 
\\	魂 
\\	塊, 魂	
\\	欠如	
\\	けつじょ	
\\	する 
\\	ビタミン
\\	1が欠如すると、脚気になるかもしれないですよ。	
\\	欠, 如	
\\	中佐	
\\	ちゅうさ	
\\	中佐は、日本への移住に反対することを決めた。	
\\	中, 佐	
\\	躍如	
\\	やくじょ	
\\	そいつにチャンピオンの面目躍如たるところを見せつけてやれ!	
\\	躍, 如	
\\	唇	
\\	くちびる	
\\	たらこ唇にコンプレックスを感じるって言うけど、僕は君の唇、とっても可愛いと思うよ。	
\\	唇	
\\	愛憎	
\\	あいぞう	
\\	な 
\\	の 
\\	この愛憎の入り混じった気持ちをコントロールするのはもう無理だよ。	
\\	愛, 憎	
\\	下唇	
\\	したくちびる, かしん	
\\	赤ちゃんは、木綿のミトンをしていたにも関わらず、下唇を引っ掻いてしまった。	
\\	下 
\\	唇.	下, 唇	
\\	少佐	
\\	しょうさ	
\\	その少佐は、大佐に飲み物を奢ると言ってきかなかった。	
\\	少, 佐	
\\	化粧	
\\	けしょう	
\\	する 
\\	出産後お肌の調子が悪くってさ〜。前に言ってたおススメの化粧品の名前教えてくれない?	
\\	化 
\\	け. 
\\	(けしょう) 
\\	化, 粧	
\\	詐称	
\\	さしょう	
\\	する 
\\	庭師は彼女が年齢を詐称していることにすぐに気がつきましたが、何も言いませんでした。	
\\	詐, 称	
\\	貨幣	
\\	かへい	
\\	の 
\\	私は実は、昔の貨幣の収集家なんです。	
\\	貨, 幣	
\\	朝霧	
\\	あさぎり	
\\	チャリティーオークションで、朝霧から山々が浮き出ている素晴らしく美しい絵を落札しました。	
\\	朝, 霧	
\\	土塀	
\\	どべい	
\\	あなたの家の土塀をよじ登っている人がいたので、それをお知らせするためにお電話させて頂きました。	
\\	土, 塀	
\\	老婆	
\\	ろうば	
\\	あの老婆は絶対に賭博を認めようとしないんだ。	
\\	老, 婆	
\\	掌	
\\	てのひら	
\\	掌サイズの小さな電子辞書を買いました。	
\\	手の
\\	(てのひら). 
\\	掌	
\\	紙幣	
\\	しへい	
\\	アメリカに行くと、お財布に紙幣がギッシリ入るのでお金持ちになった気がするんですよね。まあ、それは全部一ドル札なんですが。	
\\	紙 
\\	(し)? 
\\	紙, 幣	
\\	可哀想	
\\	かわいそう	
\\	な 
\\	お酒を付き合い程度にしか飲まない人のこと、なんだか可哀想だなって思うんだよね。	
\\	哀 
\\	(わい) 
\\	可, 哀, 想	
\\	哀れ	
\\	あわれ	
\\	な 
\\	衣食住もままならない彼らの事を哀れに思います。	
\\	(あわ), 
\\	哀	
\\	矛先	
\\	ほこさき	
\\	でも、どうしてハッカーは攻撃の矛先を俺たちに向けてきたんだ?	
\\	矛 
\\	先 
\\	矛, 先	
\\	粉砕	
\\	ふんさい	
\\	する 
\\	私の従兄弟は、転んで左足の膝のお皿を粉砕骨折しました。	
\\	粉, 砕	
\\	虹	
\\	にじ	
\\	はためく日本国旗の上には、美しい虹が掛かっていた。	
\\	虹	
\\	虹色	
\\	にじいろ	
\\	の 
\\	虹色の山葵を購入してみたんですが、味はまんま山葵ですね。	
\\	虹 
\\	色.	虹, 色	
\\	花柳	
\\	かりゅう	
\\	彼は花柳小説の有名な著者ですよ。	
\\	花, 柳	
\\	爽やか	
\\	さわやか	
\\	な 
\\	爽やかなイケメンが横を通り過ぎたので、思わず三度見をしちゃいました。	
\\	爽	
\\	痛恨	
\\	つうこん	
\\	な 
\\	の 
\\	痛恨のミスをしてしまい、彼にアドバイスしてもらうはめになりました。	
\\	痛, 恨	
\\	休憩	
\\	きゅうけい	
\\	する 
\\	今までに、休憩を取ることを考えたことはありますか。	
\\	休, 憩	
\\	炊飯器	
\\	すいはんき	
\\	私達はみんな、剛志が炊飯器を買うのを楽しみにしています。	
\\	炊, 飯, 器	
\\	砕石	
\\	さいせき	
\\	する 
\\	この道は現在砕石舗装工事中です。	
\\	砕, 石	
\\	尺	
\\	しゃく	
\\	うちのチワワに二平方尺の犬小屋を作ってあげました。	
\\	尺	
\\	撃墜	
\\	げきつい	
\\	する 
\\	で軍事飛行機がヘリコプターを撃墜する瞬間を撮影しました。	
\\	撃, 墜	
\\	悲哀	
\\	ひあい	
\\	ベーコンを取り上げられたプードルは、悲哀と悲しみに満ちた声でクーンクーンと鳴いていました。	
\\	悲, 哀	
\\	割り箸	
\\	わりばし	
\\	日本の首相を歓迎するために、通りは割り箸で飾られました。	
\\	割る 
\\	箸 
\\	割, 箸	
\\	粘々	
\\	ねばねば	
\\	する 
\\	な 
\\	綿棒で犬の耳垢を取ったら何だか粘々していたんですが、それは普通ですか。	
\\	粘る. 
\\	粘, 々	
\\	突如	
\\	とつじょ	
\\	彼女が突如彼をセクハラで訴え始めたのには理由があります。	
\\	突, 如	
\\	詐取	
\\	さしゅ	
\\	する 
\\	私は偽の売買契約書で、お金を詐取されました。	
\\	詐, 取	
\\	巻尺	
\\	まきじゃく	
\\	この巻尺で胸囲を測るのを手伝ってもらえませんか。	
\\	巻 
\\	巻く 
\\	尺 
\\	尺. 
\\	巻, 尺	
\\	畳	
\\	たたみ	
\\	私は畳の上に脱ぎ捨てられた、表裏が逆になった皺々のワイシャツを見つけました。	
\\	たたみ 
\\	畳	
\\	お陰で	
\\	おかげで	
\\	皆様のお陰で、我々は甲子園球場で開催されている高校野球のトーナメントで、準々決勝にまで進出することができました。	
\\	"陰 
\\	お 
\\	陰. 
\\	陰	
\\	黒い霧	
\\	くろいきり	
\\	ホームステイ先のお父さんが、今日の夕飯の際に黒い霧事件について説明をしてくれました。	
\\	黒い 
\\	霧.	黒, 霧	
\\	疲れ	
\\	つかれ	
\\	うちの町が大雪に見舞われて、昨日は一日中雪かきしなきゃいけなかったんだよ。もう疲れてクタクタだよ。	
\\	"疲れる 
\\	疲れる.	疲	
\\	粘る	
\\	ねばる	
\\	ええっ!あのセールスマンまだうちの母ちゃんと話をしてるよ。粘るね〜!	
\\	う 
\\	(ねば) 
\\	粘	
\\	歯を磨く	
\\	はをみがく	
\\	「どうして歯を磨かないの?」「だって、食べ物の味が口に残っているのが好きなんだもん。それっていけないことかしら?」	
\\	"磨く 
\\	歯 
\\	磨く, 
\\	歯, 磨	
\\	澄ます	
\\	すます	
\\	私は、一つ上の階からコンスタントに聞こえてくる気味の悪い金切り声を聞くために、耳を澄ましました。	
\\	"澄む 
\\	(ます) 
\\	澄む, 
\\	澄	
\\	炊く	
\\	たく	
\\	ご飯を炊きながら、妹は私に家計が火の車だとぼやきました。	
\\	う 
\\	(たく) 
\\	炊	
\\	欺く	
\\	あざむく	
\\	孝志は彼女をうまく欺いた。	
\\	う 
\\	(あざむ)... 
\\	欺	
\\	滴る	
\\	したたる	
\\	彼の額から、血が滴り落ちました。	
\\	う 
\\	(したた), 
\\	滴	
\\	扇ぐ	
\\	あおぐ	
\\	重ね着しすぎてすごく暑いんだけど、ちょっと私に向かって扇いでくれない?	
\\	う 
\\	青 (あお) 
\\	扇	
\\	憎む	
\\	にくむ	
\\	私は母を憎んだが、母も同様に私のことを憎んでいました。	
\\	う 
\\	肉 (にく). 
\\	憎	
\\	恨む	
\\	うらむ	
\\	どうしてあのスタントマンを恨んでいるんだい?	
\\	う 
\\	(うら). 
\\	恨	
\\	湿らせる	
\\	しめらせる	
\\	唇を舐めて湿らせることは、結局その後乾燥させてしまうことになりうるので、あまりいいことではありません。	
\\	湿る 
\\	湿る.	湿	
\\	憩う	
\\	いこう	
\\	この山小屋でちょっと憩わないか。	
\\	う 
\\	(いこ) 
\\	憩	
\\	胴	
\\	どう	
\\	剣道の試合で、コウイチは竹刀で相手の胴を完璧に討ち取りました。	
\\	胴	
\\	砕く	
\\	くだく	
\\	クッキーが粉々になるまで、伸ばし棒で叩いて砕きました。	
\\	う 
\\	だ (くだ). 
\\	砕	
\\	車掌	
\\	しゃしょう	
\\	車掌は、乗客たちを待たせていることについて、お詫びをした。	
\\	車 
\\	電車, 
\\	車, 掌	
\\	上唇	
\\	うわくちびる, じょうしん	
\\	疲れると、上唇がよく腫れちゃうんですよね。	
\\	上 
\\	うわ 
\\	うえ. 
\\	うえ, 
\\	上, 唇	
\\	眺め	
\\	ながめ	
\\	シャワーの後、髪の毛が半乾きのまましばらく部屋の窓からの眺めを見つめていました。	
\\	"眺める 
\\	眺める.	眺	
\\	塀	
\\	へい	
\\	ブロック塀の取付け費用がいくらかかるのかが知りたいんですが。	
\\	塀	
\\	柳	
\\	やなぎ	
\\	あの居酒屋の柳模様の暖簾が好きなのよね。	
\\	柳	
\\	お嬢さん	
\\	おじょうさん	
\\	部長のお嬢さんは確か歌手志望じゃなかったですっけ?	
\\	嬢さん, 
\\	嬢さん 
\\	嬢	
\\	遂行	
\\	すいこう	
\\	する 
\\	恥ずかしいことに、その任務の遂行に失敗してしまいました。	
\\	遂, 行	
\\	癖に	
\\	くせに	
\\	彼は陰では顧客のことを「カモ」と呼んでいる癖に、表ではペコペコ頭を下げておべっかを言っている。	
\\	癖.	癖	
\\	麻酔	
\\	ますい	
\\	の 
\\	麻酔でまだ寝ぼけています。	
\\	麻, 酔	
\\	畜産	
\\	ちくさん	
\\	このトウモロコシは畜産用飼料なので、人間用ではありません。	
\\	畜, 産	
\\	塗布	
\\	とふ	
\\	する 
\\	昨日は一日中石膏を乾式工法の壁に塗布していました。	
\\	塗, 布	
\\	脇	
\\	わき	
\\	俺が逆らうと、兄貴はよく脇をこしょばしてきました。	
\\	(わき)?	脇	
\\	未遂	
\\	みすい	
\\	の 
\\	彼は取り付け騒ぎを防ぐ為にもう少しで預金者に嘘をつくところだったが、結局未遂に終わった。	
\\	未, 遂	
\\	後ろ盾	
\\	うしろだて	
\\	私は友人の後ろ盾に勇気づけられました。	
\\	後ろ 
\\	盾 
\\	後, 盾	
\\	盆踊り	
\\	ぼんおどり	
\\	みんなが、私に一緒に盆踊りをやってみるようにしつこく言ってきたんです。	
\\	盆 
\\	躍り 
\\	盆, 踊	
\\	鉢	
\\	はち	
\\	私は鉢にピーナッツを植えました。	
\\	鉢	
\\	火鉢	
\\	ひばち	
\\	私は正座をして、火鉢に手をかざしました。	
\\	火 
\\	鉢, 
\\	火, 鉢	
\\	光輝	
\\	こうき	
\\	彼は光輝燦然と旗を掲げました。	
\\	光, 輝	
\\	迷彩	
\\	めいさい	
\\	今日迷彩柄のズボンを買いに行ったんだけど、ひとつも見つけられなかったよ。	
\\	迷, 彩	
\\	麻	
\\	あさ	
\\	これは麻の木じゃないって、何回説明したら分かってもらえるんですか?	
\\	朝. 
\\	朝.	麻	
\\	騎兵	
\\	きへい	
\\	その騎兵の写真、なんだか若かりし頃を思い出すな。	
\\	騎, 兵	
\\	賢い	
\\	かしこい	い 
\\	僕はあまり賢くはないけど、やるときはちゃんとやってみせるよ。	
\\	い 
\\	(かしこ).	賢	
\\	塗装	
\\	とそう	
\\	する 
\\	遅くなって、本当にすみません。トイレの壁の塗装が思ったよりも長引いてしまって。	
\\	塗, 装	
\\	色彩	
\\	しきさい	
\\	先生は、その生徒は鋭い色彩感覚を持っていると判断しました。	
\\	色 
\\	しき 
\\	景色?). 
\\	彩 
\\	色, 彩	
\\	返り咲き	
\\	かえりざき	
\\	九回裏から逆転ホームランで見事返り咲いた。	
\\	返る 
\\	咲. 
\\	返, 咲	
\\	矛盾	
\\	むじゅん	
\\	する 
\\	の 
\\	それってすごく矛盾した状況じゃない?	
\\	矛, 盾	
\\	隙間	
\\	すきま	
\\	二台の自販機の隙間に、小銭が2-3枚落ちているのを見つけました。	
\\	隙 
\\	間 
\\	間に合う 
\\	隙, 間	
\\	隙	
\\	すき	
\\	彼女がマシンガントークを繰り広げていたので、私が会話に入る隙はこれっぽっちもありませんでした。	
\\	隙	
\\	培養	
\\	ばいよう	
\\	する 
\\	細胞培養を作るには、何を準備する必要がありますか。	
\\	培, 養	
\\	畜生	
\\	ちくしょう	
\\	この畜生め!何でシラチャソースを俺のベッドにぶちまけやがったんだ!	
\\	畜, 生	
\\	踊り	
\\	おどり	
\\	盆踊りの練習をして時間を潰しました。	
\\	踊	
\\	踊り場	
\\	おどりば	
\\	彼女はクラブでベロベロに酔っ払い、踊り場でトロンとした目つきでろれつの回らない会話を繰り広げていた。	
\\	踊, 場	
\\	暗闇	
\\	くらやみ	
\\	暗闇で戦う時は、相手の出方を見て、その上でどうするか判断した方がいいよ。	
\\	暗い 
\\	闇.	暗, 闇	
\\	闇	
\\	やみ	
\\	の 
\\	僕はある梟と目を交換したから、暗闇でも目が見えるんだよ。	
\\	闇	
\\	餓死	
\\	がし	
\\	する 
\\	警察は、その寝たきりの老人は餓死したと考えている。	
\\	餓, 死	
\\	斜め	
\\	ななめ	
\\	な 
\\	の 
\\	水平な横線を引くように言ったのに、ちょっと斜めになっています。	
\\	七 (なな) 
\\	斜	
\\	恥辱	
\\	ちじょく	
\\	彼女を雑誌の表紙に起用することは、恥辱を受けるようなものです。	
\\	恥 
\\	ち 
\\	恥, 辱	
\\	屈辱	
\\	くつじょく	
\\	前年度優勝校と一回戦で当たってしまい、屈辱的な敗北を味わったが、相手が悪かった。	
\\	屈, 辱	
\\	家畜	
\\	かちく	
\\	日本独特の家畜っていうのはいるんですか?	
\\	家, 畜	
\\	耐久性	
\\	たいきゅうせい	
\\	太陽光発電のパネルの耐久性がどんなもんなのか気になっています。	
\\	耐, 久, 性	
\\	土俵際	
\\	どひょうぎわ	
\\	その力士は土俵際で踏ん張り、相手力士に打っ棄りを食らわせた。	
\\	(土俵) 
\\	際 
\\	土, 俵, 際	
\\	尽力	
\\	じんりょく	
\\	する 
\\	の社長として、コウイチはサービスの向上に尽力致します。	
\\	尽, 力	
\\	電灯	
\\	でんとう	
\\	うちの嫁さんは全然電灯を消さないんだよね。きっとスイッチがあることを知らないんだと思うよ。	
\\	電, 灯	
\\	備蓄	
\\	びちく	
\\	する 
\\	地下室に一年分のトイレットペーパーの備蓄をしてるんです。	
\\	備, 蓄	
\\	忍耐	
\\	にんたい	
\\	する 
\\	俺は社長の腰巾着かもしれないが、それって俺にはすごい忍耐力があるってことにもならないかな?	
\\	忍, 耐	
\\	耐火	
\\	たいか	
\\	の 
\\	この消防士さんの人形は、実際に耐火性の服と手袋を身につけています。	
\\	耐, 火	
\\	紛れ	
\\	まぐれ, まぎれ	
\\	そのクイズの答えが当たったのはただの紛れです。	
\\	"紛れる 
\\	紛れる 
\\	まぐれ. 
\\	まぎれる, 
\\	ぎ 
\\	ぐ.	紛	
\\	霜	
\\	しも	
\\	お父さんなら今うちの冷蔵庫の霜取りをしてるよ。	
\\	霜	
\\	憶測	
\\	おくそく	
\\	する 
\\	人々は、その試合は八百長だったのではないかと憶測している。	
\\	憶, 測	
\\	収穫	
\\	しゅうかく	
\\	する 
\\	お米の収穫期は今年はいつ頃になりそうですか?	
\\	収, 穫	
\\	首班	
\\	しゅはん	
\\	どいつが新入生の首班を占めることになると思う?	
\\	首, 班	
\\	鉢巻	
\\	はちまき	
\\	する 
\\	そのラーメン屋の店主は、いつも色あせた綿の鉢巻を絞めていました。	
\\	鉢 
\\	巻く, 
\\	鉢, 巻	
\\	耐熱	
\\	たいねつ	
\\	の 
\\	友達が耐熱容器の漫画を書いてるんだけど、主人公の容器の「耐熱容器としての限界を感じた」っていう台詞が気に入ってるのよね。	
\\	耐, 熱	
\\	大麻	
\\	たいま	
\\	の 
\\	君が失業したのは本当に気の毒に思うけど、でも一体どうして大麻なんて庭で育てていたんだい?	
\\	大, 麻	
\\	殴打	
\\	おうだ	
\\	する 
\\	突然、背後から知らない男に後頭部を殴打されました。	
\\	殴, 打	
\\	騎手	
\\	きしゅ	
\\	その騎手は、自分の初めての
\\	1レースの日に朝寝坊をしました。	
\\	騎, 手	
\\	騎馬	
\\	きば	
\\	運動会の騎馬戦の練習があるので、明日は早起きしなくてはいけません。	
\\	騎, 馬	
\\	帝国主義	
\\	ていこくしゅぎ	
\\	の 
\\	あなたの国は、帝国主義国家ですか?	
\\	(帝国) 
\\	(主義) 
\\	帝国 
\\	主義.	帝, 国, 主, 義	
\\	斜体	
\\	しゃたい	
\\	の 
\\	どうしてこの単語の文字のフォントを斜体に変えたんですか。	
\\	斜, 体	
\\	騎士	
\\	きし	
\\	その騎士は、アメリカ国旗を掲げることを断った。	
\\	騎, 士	
\\	灯り	
\\	あかり	
\\	その部屋には、小さなロウソクの灯りだけが灯っていました。	
\\	明るい, 
\\	あか 
\\	明るい (あか)? 
\\	赤 (あか) 
\\	灯	
\\	遅咲き	
\\	おそざき	
\\	冬が長引いたため、今年はうちのお庭の桜は遅咲きでした。	
\\	花見.	
\\	遅い 
\\	咲く 
\\	咲 
\\	遅, 咲	
\\	溶岩	
\\	ようがん	
\\	火山が溶岩を吹き出していて危険なので、これ以上近づくことはできません。	
\\	岩, 
\\	がん. 
\\	溶, 岩	
\\	脇見	
\\	わきみ	
\\	する 
\\	いてぇ。どこに目ぇつけて歩いてるんだ!脇見しながら歩いてんじゃねぇぞ!	
\\	脇 
\\	見る.	脇, 見	
\\	輝度	
\\	きど	
\\	例えば、太陽の輝度値はとても高いです。	
\\	輝, 度	
\\	メモ帳	
\\	めもちょう, メモちょう	
\\	話の途中で申し訳ありませんが、あなたのメモ帳のことでどうしても質問したいんです。	
\\	(メモ) 
\\	メモ.	帳	
\\	記憶	
\\	きおく	
\\	する 
\\	ある臭いが、ある記憶を引き起こすって、聞いたことある?俺は今まさに、下水道に落ちた時のことを思い出しているんだけど。	
\\	記, 憶	
\\	蚊	
\\	か	
\\	根っからの蚊取り線香派なんで、電気のやつを使うのは好きじゃないんですよね。	
\\	蚊	
\\	蚊帳	
\\	かや	
\\	十人用の大きな蚊帳はありますか?	
\\	蚊, 
\\	や. 
\\	や, 
\\	(や) 
\\	蚊, 帳	
\\	手帳	
\\	てちょう	
\\	手帳を盗み見すると、彼女は物凄く怒りました。	
\\	手, 帳	
\\	油彩	
\\	ゆさい	
\\	油彩画のコツを掴んできました。	
\\	油, 彩	
\\	塗料	
\\	とりょう	
\\	刷毛に蛍光塗料を付けました。	
\\	塗, 料	
\\	迅速	
\\	じんそく	
\\	な 
\\	こちらの件に関して、迅速かつ丁寧にご対応頂き誠に有り難うございます。	
\\	迅, 速	
\\	記帳	
\\	きちょう	
\\	する 
\\	帳簿に記帳をし忘れたため、上司に叱られました。	
\\	記, 帳	
\\	後悔	
\\	こうかい	
\\	する 
\\	の 
\\	美人局にいとも簡単に引っ掛かってしまったことを後悔しています。	
\\	後, 悔	
\\	悔しい	
\\	くやしい	い 
\\	他の人が日本語で何を話しているのか理解できない時、私はいつも悔しい思いをします。	
\\	(くや), 
\\	悔	
\\	貯蓄	
\\	ちょちく	
\\	する 
\\	うちの祖母は葬式代分しか貯蓄していないと言っていました。	
\\	貯, 蓄	
\\	盾	
\\	たて	
\\	盾を買いたいんだけど、どこに行けば買えるのか知りませんか?	
\\	(立て) 
\\	盾	
\\	斜面	
\\	しゃめん	
\\	うちの私道は急斜面になってるので、上まで上がってくるにはスノータイヤがいります。	
\\	斜, 面	
\\	蛇	
\\	へび	
\\	この町で蛇がペットとして普及した時には、身震いしましたよ。	
\\	蛇	
\\	班	
\\	はん	
\\	あの班は、五輪の左端の輪っかに間違った色を選びました。	
\\	班	
\\	班長	
\\	はんちょう	
\\	彼は社交的でみんなから好かれているので、班長に選ばれました。	
\\	班, 長	
\\	飢餓	
\\	きが	
\\	どうしてそんな風にクッキーを踏みつけたりしたんだ?毎日何千何百人という人が飢餓で亡くなるのを知らないのか。	
\\	飢, 餓	
\\	街灯	
\\	がいとう	
\\	大量の蛾が、各街灯の周りを飛び交っています。	
\\	街, 灯	
\\	脇役	
\\	わきやく	
\\	の 
\\	彼女、オーディションに受かったって言いふらしてるけど、ただの脇役みたいだよ。	
\\	脇
\\	役
\\	脇, 役	
\\	電話帳	
\\	でんわちょう	
\\	この電話帳を整理してしまう方がいいと思います。	
\\	(電話) 
\\	電, 話, 帳	
\\	脅し	
\\	おどし	
\\	寿司屋に行く途中、寿司に醤油をかけたら殺すぞ、という脅しを受けた。	
\\	(おど) 
\\	脅	
\\	脅迫	
\\	きょうはく	
\\	する 
\\	脅迫電話を受けた直後に警察に電話をしましたが、彼らは何だか対応したくなさそうでした。	
\\	脅, 迫	
\\	蓄える	
\\	たくわえる	
\\	彼はとても立派な口髭を蓄えています。	
\\	う 
\\	(たくわ) 
\\	蓄	
\\	尽きる	
\\	つきる	
\\	長い髪を手櫛で梳かしながら、彼女は僕に、あんたにはもう愛想が尽きたわと言った。	
\\	う 
\\	(つ) 
\\	尽	
\\	咲く	
\\	さく	
\\	この桃の木は、今までに一度も花が咲いた事がないんですよ。	
\\	う 
\\	咲	
\\	彩る	
\\	いろどる	
\\	環境に優しいアイテムに切り替えることは、あなたの暮らしを彩ります。	
\\	う 
\\	(いろど).	彩	
\\	涼む	
\\	すずむ	
\\	うちのワンコはあそこの木陰で涼んでいます。	
\\	う 
\\	(すず) 
\\	涼	
\\	培う	
\\	つちかう	
\\	高校時代のクラブ活動を通じて、友情を培うことの大切さについてたくさんのことを学びました。	
\\	う 
\\	(つちか) 
\\	培	
\\	踊る	
\\	おどる	
\\	海の中の魚達は、みんな太鼓の音に合わせて一斉に踊り出しました。	
\\	う 
\\	踊	
\\	耐える	
\\	たえる	
\\	彼のひどい寝起きの口臭に、耐えられなかったんです。	
\\	う 
\\	(た), 
\\	耐	
\\	辱める	
\\	はずかしめる	
\\	みんなの前で私の事を辱めるのが、どうしてそんなに面白いんですか?	
\\	う 
\\	恥ずかしい? 
\\	辱	
\\	溶かす	
\\	とかす	
\\	電子レンジでもバターを溶かすことができますよ。	
\\	う 
\\	(かす) 
\\	(と). 
\\	溶	
\\	塗る	
\\	ぬる	
\\	マニュキアを塗り終わってからまた電話してもいい?	
\\	う 
\\	(ぬ) 
\\	塗	
\\	貼る	
\\	はる	
\\	肩がこったので、湿布を貼りました。	
\\	う 
\\	貼	
\\	殴る	
\\	なぐる	
\\	ゴングが鳴った瞬間、そのボクサーは相手の顎を思い切り殴り付けた。	
\\	う 
\\	(なぐ) 
\\	殴	
\\	輝く	
\\	かがやく	
\\	明るく輝く月の下で、飲酒運転取締り検問所にひっかかり、飲酒運転の罪で速攻逮捕されてしまいました。	
\\	う 
\\	輝	
\\	飢える	
\\	うえる	
\\	長年日本語を独学してきたんですが、今は誰かと日本語の会話をすることに飢えています。	
\\	う 
\\	(う) 
\\	飢	
\\	理不尽	
\\	りふじん	
\\	な 
\\	超理不尽な上司とやっていかなくちゃいけなくて、理想の仕事が一気に悪夢になったよ。	
\\	理, 不, 尽	
\\	無闇に	
\\	むやみに	
\\	無闇に女の子に声をかけるのはよした方がいいよ。	
\\	無, 闇	
\\	水彩画	
\\	すいさいが	
\\	彼女は素晴らしい水彩画家だそうですよ。	
\\	水, 彩, 画	
\\	水溶性	
\\	すいようせい	
\\	水溶性のビタミン
\\	はすぐに体内から排出されるので、私達の体に蓄えることはできないと読みましたよ。	
\\	水, 溶, 性	
\\	車椅子	
\\	くるまいす	
\\	あの車椅子に座ってる女の子、めちゃくちゃセクシーじゃない?	
\\	椅子 
\\	車 
\\	くるま. 
\\	椅子 
\\	いす.	車, 椅, 子	
\\	賢明	
\\	けんめい	
\\	な 
\\	このビールを買うのは賢明なご決断ですよ。コクがあるだけでなく、喉ごしもとてもいいですからね。	
\\	賢, 明	
\\	賢人	
\\	けんじん	
\\	あの教授、外では賢人って言われてるけど、実は内弁慶らしいぜ。	
\\	賢, 人	
\\	魔法使い	
\\	まほうつかい	
\\	約束を守らないと、みんなから相手にされなくなるぞ。いいから、お前が先日会ったっていう、その魔法使いを連れてこいよ!	
\\	魔法 
\\	魔法 
\\	使い 
\\	使う.	魔, 法, 使	
\\	抽象	
\\	ちゅうしょう	
\\	彼女のレビューは何だか抽象的だったので、あまりよく分かりませんでした。	
\\	抽, 象	
\\	概算	
\\	がいさん	
\\	する 
\\	の 
\\	旅行費用の概算を教えてもらえますか。	
\\	概, 算	
\\	煮物	
\\	にもの	
\\	彼女は小食ですが、煮物ならモリモリ食べます。	
\\	煮, 物	
\\	仏壇	
\\	ぶつだん	
\\	仏壇の前でうつ伏せになって寝転がっていると、母がやって来てそれは失礼なことだと怒られました。	
\\	仏, 壇	
\\	珠算	
\\	しゅざん	
\\	今日は珠算を勉強するって気分じゃないんだよね。	
\\	珠, 算	
\\	転覆	
\\	てんぷく	
\\	する 
\\	ワニだらけの沼沢で私のカヌーが転覆した時は、どっと冷や汗をかきましたよ。	
\\	転, 覆	
\\	陶芸	
\\	とうげい	
\\	彼女の陶芸作品のルーツは、日本文化にあります。	
\\	陶, 芸	
\\	駒	
\\	こま	
\\	父さんが日曜日にチェスの駒の動かし方を教えてくれるって約束してくれたんだ。	
\\	駒	
\\	俗語	
\\	ぞくご	
\\	の 
\\	この俗語はアメリカ独特の表現です。	
\\	俗, 語	
\\	征服	
\\	せいふく	
\\	する 
\\	が世界征服をしようとしているということを又聞きしてしまいました。	
\\	征, 服	
\\	陰謀	
\\	いんぼう	
\\	彼は秘密結社が世界を支配しているという陰謀論を信じています。	
\\	陰, 謀	
\\	鶴	
\\	つる	
\\	たくさんの折り紙の鶴を醤油につけて、日本人の女の子の部屋に置いてみるってのはどうかな。	
\\	鶴	
\\	皇太子妃	
\\	こうたいしひ	
\\	皇太子妃が軍に志願するはずねぇだろうが。	
\\	皇太子 
\\	皇, 太, 子, 妃	
\\	貨幣価値	
\\	かへいかち	
\\	その時からどれぐらい貨幣価値が変わったか、知っているかい?	
\\	"貨幣 
\\	価値 
\\	貨, 幣, 価, 値	
\\	衰退	
\\	すいたい	
\\	する 
\\	どの専門家も、投資市場の衰退は信用収縮によって誘導されたと口を揃えて言うけれど、それってつまりはどういう意味なの?	
\\	衰, 退	
\\	拘置	
\\	こうち	
\\	する 
\\	その女性は豆芝を膝に載せたまま酔っぱらっていたので、見かねた警察がその犬を保護拘置し、後から返してあげたそうです。	
\\	拘, 置	
\\	劣化	
\\	れっか	
\\	する 
\\	女優やモデルの見た目が以前よりも老けた時、日本のネット用語で「劣化した」と言います。	
\\	劣, 化	
\\	不明瞭	
\\	ふめいりょう	
\\	な 
\\	説明書を読んだんですが、あまりに不明瞭だったので、新モデルの実力についてちょっと不安に思っています。	
\\	"明瞭 
\\	不, 明, 瞭	
\\	侵食	
\\	しんしょく	
\\	する 
\\	木々を切り倒すことで、土壌の侵食が引き起こされることがあります。	
\\	侵, 食	
\\	隔月	
\\	かくげつ	
\\	の 
\\	日本文化の授業が隔月で開講されるんですが、今月は餅つきをする予定です。	
\\	隔, 月	
\\	隔週	
\\	かくしゅう	
\\	このアニメは隔週で放送されていて、私はそのオープニング曲とエンディング曲が好きです。	
\\	隔, 週	
\\	遠征	
\\	えんせい	
\\	する 
\\	の 
\\	豚インフルエンザの流行により、遠征は中止となりました。	
\\	遠, 征	
\\	抽出	
\\	ちゅうしゅつ	
\\	する 
\\	の 
\\	五時までに抽出データを整理してもらうことはできますか。	
\\	抽, 出	
\\	淡い	
\\	あわい	い 
\\	話を変えて悪いけど、淡いピンク色のチーク持ってなかったっけ?	
\\	い 
\\	(あわ) 
\\	淡	
\\	団扇	
\\	うちわ	
\\	同じ模様の団扇だけど、私の物の方があなたのより少し小さくて軽いみたいね。	
\\	団 
\\	内
\\	扇 
\\	(わ) 
\\	内 
\\	団, 扇	
\\	花壇	
\\	かだん	
\\	その花壇を見ると、いつも母を思い出します。	
\\	花, 壇	
\\	扇	
\\	おうぎ	
\\	あの、紫色の扇をもったご婦人は一体どなたですか。	
\\	(おうぎ), 
\\	扇	
\\	民俗	
\\	みんぞく	
\\	彼は、歴史民俗資料館の館長です。	
\\	民, 俗	
\\	基礎	
\\	きそ	
\\	の 
\\	基礎の日本語であれば教えられないこともないですが。	
\\	基, 礎	
\\	礎	
\\	いしずえ	
\\	ここにいる人は皆、総合格闘技の礎を築いた故人の死を悼んでいると思います。	
\\	(いし). 
\\	(ずえ) 
\\	礎	
\\	淡水	
\\	たんすい	
\\	の 
\\	金魚は淡水魚だから、水槽に塩を入れてはいけませんよ。	
\\	淡, 水	
\\	概念	
\\	がいねん	
\\	ほとんどの無神論者は、神や神々の概念に対して否定的です。	
\\	概, 念	
\\	お婆ちゃん	
\\	おばあちゃん	
\\	お婆ちゃんに、嘘をついたことで責められた。	
\\	ちゃん 
\\	お 
\\	ば 
\\	ばあ, 
\\	(ばあ).	婆	
\\	恨み	
\\	うらみ	
\\	アポを取ろうとしたんだけど、彼はどうやらうちの社長に何か恨みがあるみたいで、時間をくれなかったんだよね。	
\\	"恨む 
\\	恨む, 
\\	恨	
\\	憎らしい	
\\	にくらしい	い 
\\	候補者のコウイチ氏は、大統領候補討論会の前、憎らしいほど落ち着いていました。	
\\	"憎む 
\\	憎む 
\\	憎	
\\	憎い	
\\	にくい	い 
\\	こんなに誰かのことを憎いと思ったのは生まれて初めてです。	
\\	憎む 
\\	憎む 
\\	憎	
\\	憎しみ	
\\	にくしみ	
\\	てっきり彼らはお互いに憎しみ合っているものだと思っていました。	
\\	憎む 
\\	憎む.	憎	
\\	劣悪	
\\	れつあく	
\\	な 
\\	劣悪な職場環境の会社は、日本ではしばしば「ブラック企業」と呼ばれます。	
\\	劣, 悪	
\\	劣等感	
\\	れっとうかん	
\\	彼女が私に対して劣等感を抱いていると誤解していました。	
\\	劣, 等, 感	
\\	大概	
\\	たいがい	
\\	の 
\\	万引きをするのは大概、退屈で孤独を感じている女性だと言う人がいますが、ある調査が男性の方が女性よりも万引きをすることが多い可能性を示唆しているそうですよ。	
\\	大, 概	
\\	勘	
\\	かん	
\\	成功の秘訣は、鋭い勘をもつことです。	
\\	勘	
\\	隔離	
\\	かくり	
\\	する 
\\	の 
\\	彼は二日酔いであまりにも酒臭すぎるので、別の部屋に隔離しています。	
\\	隔, 離	
\\	抽選	
\\	ちゅうせん	
\\	する 
\\	私はあの病院の抽選で、胃のバイパス手術が当たったんです。	
\\	抽, 選	
\\	覆面	
\\	ふくめん	
\\	する 
\\	覆面のスーパーマンは、ボッロボロのマントを身につけていました。	
\\	覆, 面	
\\	唯物論	
\\	ゆいぶつろん	
\\	母国語でも唯物論について説明できないのに、どうして外国語で説明できるというのか。	
\\	唯, 物, 論	
\\	勘違い	
\\	かんちがい	
\\	する 
\\	「すみません、間違えました。違う人と 勘違いしました。」「正確には誰と間違えたんですか?お願いします。教えてください。」	
\\	勘, 違	
\\	草刈り	
\\	くさかり	
\\	昨日は俺が悪かったよ。本当にごめん。お詫びの印に、今日は草刈りをするよ。	
\\	草 
\\	刈 
\\	か 
\\	刈り 
\\	草, 刈	
\\	桑畑	
\\	くわばたけ	
\\	私の地元には、桑畑があちこちにあります。	
\\	桑 
\\	畑. 
\\	桑, 畑	
\\	桑原	
\\	くわばら	
\\	かつては桑畑がたくさんあったため、この場所は「桑原」と呼ばれるようになりました。	
\\	原 
\\	(ばら) 
\\	桑, 原	
\\	桑	
\\	くわ	
\\	桑の実と林檎と胡桃の組み合わせが最強に美味しいんだよね。	
\\	桑	
\\	雑煮	
\\	ぞうに	
\\	お雑煮そろそろ煮えたんじゃないかな。	
\\	煮 
\\	雑 
\\	雑巾, 
\\	雑, 煮	
\\	尼	
\\	あま	
\\	彼女は尼さんなのにお酒をたくさん飲みます。	
\\	(あま) 
\\	尼	
\\	尼僧	
\\	にそう	
\\	その尼僧は、寝起きはいつも顔がむくんでいます。	
\\	尼, 僧	
\\	唯一	
\\	ゆいつ, ゆいいつ	
\\	の 
\\	あいつは奴の家族の中で唯一まともな人間だよ。	
\\	唯 
\\	ゆい 
\\	ゆ. 
\\	一 
\\	一! 
\\	(いつ) 
\\	いつ 
\\	一 
\\	唯, 一	
\\	概要	
\\	がいよう	
\\	青年海外協力隊についての概要は、こちらのページでお読み頂けます。	
\\	概, 要	
\\	恐慌	
\\	きょうこう	
\\	する 
\\	私が留守だった日、村は恐慌に包まれた。	
\\	人生で二度もの経済恐慌を経験するとは思っていなかったよ。	
\\	国全体が恐慌状態に陥った。	
\\	恐, 慌	
\\	勘弁	
\\	かんべん	
\\	する 
\\	勘弁してちゃぶ台。今日は忙かったし、滅茶苦茶疲れているんだ。	
\\	勘, 弁	
\\	真珠	
\\	しんじゅ	
\\	真珠を身につける人は減ってきています。	
\\	真, 珠	
\\	浸透	
\\	しんとう	
\\	する 
\\	こちらの化粧水は、お肌の奥深くまで浸透するということで、すごく人気が出てきています。	
\\	浸, 透	
\\	壇	
\\	だん	
\\	壇上で、小さな可愛いフラワーガールの女の子が、しばらく花嫁の人気をさらいました。	
\\	壇	
\\	陶器	
\\	とうき	
\\	の 
\\	もう使わないんだったら、白い陶器は全部誰かにあげちゃえば?	
\\	陶, 器	
\\	妃	
\\	ひ	
\\	フグは、わざとサーモン妃にぶつかりました。	
\\	妃	
\\	利潤	
\\	りじゅん	
\\	日本には資源超過利潤税は無かったと思うよ。	
\\	利, 潤	
\\	紫	
\\	むらさき	
\\	暗闇の中でキラキラ光る紫色の光を見た瞬間、「母だ」と思ったんです。	
\\	""村先
\\	(むらさき) 
\\	紫	
\\	王妃	
\\	おうひ	
\\	私は王妃と友情を育んだ。	
\\	王, 妃	
\\	無謀	
\\	むぼう	
\\	な 
\\	私の娘を止めてもらえませんか。無謀にもスパイ活動をしようとしているんです。	
\\	無, 謀	
\\	開拓	
\\	かいたく	
\\	する 
\\	畑は違えど、新規市場を開拓するという点においては、前職とさほど変わりはないよ。	
\\	開, 拓	
\\	推奨	
\\	すいしょう	
\\	する 
\\	このカボチャのパイは、焼き上がりでも美味しく召し上がって頂けるのですが、私たちは御賞味の前に少し冷蔵庫に入れて頂くことを推奨しています。	
\\	推, 奨	
\\	花柳界	
\\	かりゅうかい	
\\	花柳界を経験してみるのが夢だったんだ。	
\\	(花柳) 
\\	花, 柳, 界	
\\	覚悟	
\\	かくご	
\\	する 
\\	私は痩身手術を受ける覚悟を決めました。	
\\	覚, 悟	
\\	憩い	
\\	いこい	
\\	私はうちの食堂を忙しいサラリーマンの憩いの場にしたいんです。	
\\	憩う 
\\	憩う, 
\\	憩	
\\	休憩所	
\\	きゅうけいじょ	
\\	逮捕された男は、休憩所に不法侵入した疑いがかけられていた。	
\\	(休憩) 
\\	休, 憩, 所	
\\	数珠	
\\	じゅず	
\\	高額な数珠を私に売りつけようとした男の名前が思い出せません。	
\\	(じゅず), 
\\	数, 珠	
\\	奨学金	
\\	しょうがくきん	
\\	その占い師に、私は奨学金をもらえるだろうって予言されたんだけど、当たったよ。	
\\	奨, 学, 金	
\\	老衰	
\\	ろうすい	
\\	する 
\\	の 
\\	母親が老衰で亡くなったとき、妻が「厄介者がいなくなってせいせいした」と独り言を言っているのを私は聞き漏らしませんでした。	
\\	老, 衰	
\\	俗	
\\	ぞく	
\\	な 
\\	の 
\\	貧乏から金持ちに成り上がった人は、俗に成金と呼ばれます。	
\\	俗	
\\	唯	
\\	ただ	
\\	の 
\\	彼女は私の母ではなく唯の付き添い人です。	
\\	(ただ) 
\\	唯	
\\	間隔	
\\	かんかく	
\\	生理痛程度の陣痛が15分から20分ぐらいの間隔で始まって、今行った方がいいのか病院に電話をしました。	
\\	間, 隔	
\\	割り勘	
\\	わりかん	
\\	私は下戸でお酒を飲まないので、お酒のお勘定は割り勘じゃない方が嬉しいんですが。	
\\	割る 
\\	勘, 
\\	割, 勘	
\\	粘り	
\\	ねばり	
\\	一生懸命粘り強く努力すればいつかは報われると思っていましたが、それは間違いでした。	
\\	粘る 
\\	粘る.	粘	
\\	気概	
\\	きがい	
\\	その教師は、教室で見て見ぬふりをされている問題に取り組む気概がなかった。	
\\	気, 概	
\\	浸水	
\\	しんすい	
\\	する 
\\	の 
\\	気温が上がるのは素晴らしいことだけど、私達は雪解け水による浸水を心配しています。	
\\	浸, 水	
\\	詐欺師	
\\	さぎし	
\\	「ねぇ、今俺のこと、嘘つきって言った?」「そうは言ってないけど、詐欺師とは言ったよ。」「はぁ?何言ってんだよ。マジ、信じらんねぇ。」	
\\	詐欺 
\\	(詐欺) 
\\	詐欺 
\\	詐, 欺, 師	
\\	勘案	
\\	かんあん	
\\	する 
\\	様々な状況を勘案すると、私にとって仕事を辞めることがベストな選択だったんです。	
\\	自分がされたり言われたりしたらどう思うかってことも勘案に入れるようにしてます。	
\\	この投資に関して、どのようなリスクを勘案していますか。	
\\	勘, 案	
\\	お疲れ様	
\\	おつかれさま	
\\	私の妻は私の背広をハンガーに掛る時に、「今日も一日お仕事お疲れ様でした」といつも言ってくれるんです。	
\\	疲れ 
\\	様 
\\	お疲れ様でした! 
\\	疲れ 
\\	様, 
\\	疲, 様	
\\	劣る	
\\	おとる	
\\	今、彼は私のこと嫌っているから、私の立場は挨拶程度の人よりも劣っているってのははっきりしてるわ。	
\\	う 
\\	(おと) 
\\	劣	
\\	潤う	
\\	うるおう	
\\	このローションを使えば、あなたのお肌はプルプルに潤いますよ。	
\\	う 
\\	(うるお) 
\\	潤	
\\	覆る	
\\	くつがえる	
\\	スポーツはずっとそんなに好きではなかったんですが、一度野球の試合を見に行った時にそれが覆りまして、今ではマリナーズの大ファンです。	
\\	う 
\\	靴 
\\	(くつがえ). 
\\	靴 
\\	覆	
\\	浸る	
\\	ひたる	
\\	お風呂で図書館から借りたホーソンの『巌の顔』を読んでいたんですが、うっかりお湯に浸ってしまったんです。	
\\	う 
\\	(ひた). 
\\	浸	
\\	煮る	
\\	にる	
\\	今日は、この脂ののった秋鮭を煮てみたいと思います。	
\\	う 
\\	煮	
\\	謀る	
\\	はかる	
\\	あいつらはフィレンツェのダビデ像の前でのドラマチックな暗殺を謀っているんだ。	
\\	う 
\\	(はか) 
\\	謀	
\\	衰える	
\\	おとろえる	
\\	数年前にこの牛肉を食べた時は、非常に良く熟成されて霜が降っていることに感動したんですが、今はちょっと味が衰えたみたいでとても残念ですね。	
\\	う 
\\	(おとろ), 
\\	衰	
\\	挿入する	
\\	そうにゅうする	する 
\\	はここから挿入できるよ。	
\\	"挿入 
\\	挿入, 
\\	挿, 入	
\\	隔てる	
\\	へだてる	
\\	私のアパートでは、浴槽と洗面所はシャワーカーテン一枚で隔てられています。	
\\	う 
\\	(へだ) 
\\	隔	
\\	刈り取る	
\\	かりとる	
\\	私の実家は農家で米を育てているので、今週末は稲の刈り取りを手伝わなきゃいけないんです。	
\\	刈 
\\	取る.	刈, 取	
\\	慌てる	
\\	あわてる	
\\	慌てないで、安全運転をお願いします。	
\\	火事だと思い、慌てて水をかけたら、ただのロウソクだった。	
\\	慌てず、慎重にね。	
\\	う 
\\	(あわ)?
\\	慌	
\\	哀れむ	
\\	あわれむ	
\\	彼女のお婆さんが、「あの子の人見知りは哀れむべきものだ」とぼやくのを聞きました。	
\\	(あわ), 
\\	哀れ.	哀	
\\	刈る	
\\	かる	
\\	草を刈り終わったから、放してやってよ。	
\\	う 
\\	刈	
\\	悟る	
\\	さとる	
\\	背中の曲がったお婆さんは、不法滞在者として拘置所に拘留された時、そこで自分が死を迎えることになることを悟りました。	
\\	(さと) 
\\	悟	
\\	紫外線	
\\	しがいせん	
\\	あんたはまずその紫外線恐怖症を克服しなきゃいけないね。	
\\	紫, 外, 線	
\\	一概に	
\\	いちがいに	
\\	定額給付金は一概に良い事だとは言えない。	
\\	一, 概	
\\	剛健	
\\	ごうけん	
\\	な 
\\	彼は剛健な昔の武士のような見た目だが、彼の足にある血豆を誤って押してしまった時に、泣き叫んでいました。	
\\	剛, 健	
\\	紫色	
\\	むらさきいろ	
\\	の 
\\	彼女はいつも制服の上に紫色のベストを着ている。	
\\	紫 
\\	""村先
\\	(むらさき) 
\\	紫, 色	
\\	誓い	
\\	ちかい	
\\	結婚の誓いの言葉を述べている時、自分が間違った男の人と結婚しようとしていることは分かっていました。	
\\	近い (ちかい). 
\\	(近い).	誓	
\\	誓約	
\\	せいやく	
\\	する 
\\	こちらが誓約書となっておりまして、こちらとこちらにサインをして頂く必要があります。	
\\	誓, 約	
\\	陛下	
\\	へいか	
\\	陛下、恐れ入りますが会議にご出席のご予定だったのでは。	
\\	陛, 下	
\\	帳簿	
\\	ちょうぼ	
\\	帳簿を燃やしてしまうという考えは思いつきませんでした。	
\\	帳, 簿	
\\	巧い	
\\	うまい	い 
\\	話が巧すぎるぜ。何か企んでるだろ?	
\\	旨い, 
\\	巧	
\\	鰐	
\\	わに	
\\	の 
\\	この鰐の置物、邪魔なんだけど。	
\\	鰐蟹って漢字、何も見ずに手書きで書けるの?すごいね!	
\\	鰐の剥製が部屋に飾ってあります。	
\\	鰐	
\\	古墳	
\\	こふん	
\\	近年その古墳を訪れる人が増えています。	
\\	古, 墳	
\\	伯母	
\\	おば	
\\	伯母はどうして私達の結婚式に来ようとしなかったんだろう。	
\\	母 
\\	は 
\\	はは, 
\\	伯, 母	
\\	銘柄	
\\	めいがら	
\\	後どれくらいでお客さんに推奨する銘柄が決まりそうですか。	
\\	銘, 柄	
\\	偶然	
\\	ぐうぜん	
\\	な 
\\	の 
\\	空港で偶然、高校時代の友人に会ったんですよ。	
\\	偶, 然	
\\	搬出	
\\	はんしゅつ	
\\	する 
\\	日曜の夜七時までに全ての作品を搬出する必要があります。	
\\	搬, 出	
\\	洞穴	
\\	どうけつ, ほらあな	
\\	その洞穴で躓いて、足首を捻挫しました。	
\\	洞 
\\	穴 
\\	(けつ). 
\\	ほらあな, 
\\	穴, 
\\	洞, 穴	
\\	洞	
\\	ほら	
\\	その薄気味悪い洞は、村の北外れにひっそりと佇んでいました。	
\\	(ほら), 
\\	洞	
\\	翻意	
\\	ほんい	
\\	する 
\\	この結婚について、絶対に翻意することがないことを誓います。	
\\	翻, 意	
\\	伯	
\\	はく	
\\	フグの爵位を伯爵にして、フグ伯って呼ぶことにするのはどうかな。	
\\	はく 
\\	伯	
\\	風邪	
\\	かぜ	
\\	「なんで母親の葬式に来ないの?」「風邪ひいてるからだよ。」	
\\	風 (かぜ). 
\\	風, 邪	
\\	彩り	
\\	いろどり	
\\	する 
\\	父の弁当は、彩りについて全く考慮がされておらず、全部茶色だったので嫌でしたね。	
\\	いろ 
\\	色) 
\\	ど, 
\\	色 
\\	色 
\\	彩	
\\	漫才	
\\	まんざい	
\\	漫才を見て大声で笑うことはストレス解消になる。	
\\	漫 
\\	漫, 才	
\\	訂正	
\\	ていせい	
\\	する 
\\	の 
\\	もし私の日本語に間違いがあれば、その場で訂正してもらってもいいですか?	
\\	訂, 正	
\\	〜把	
\\	わ	
\\	わっ!ほうれん草一把で九十八円だって。	
\\	わ 
\\	(わ)!
\\	把	
\\	法廷	
\\	ほうてい	
\\	の 
\\	あとどれくらい私達がこの法廷にいることになるか分かりますか?	
\\	法, 廷	
\\	蟹	
\\	かに	
\\	お腹がすいてきた。晩ご飯の蟹が楽しみだ。	
\\	鰐 
\\	蟹	
\\	水晶	
\\	すいしょう	
\\	この水晶は大中小の三つの大きさがあります。	
\\	水, 晶	
\\	堰	
\\	せき	
\\	私の息子は、ある日突然、堰を切ったように日本語をペラペラ話し始めたんです。	
\\	堰	
\\	感銘	
\\	かんめい	
\\	する 
\\	私達は彼のピアノの演奏に深く感銘を受けました。	
\\	感, 銘	
\\	悪賢い	
\\	わるがしこい	い 
\\	あいつは悪賢い男で、猫かぶりが大のお得意なんだ。	
\\	(賢い) 
\\	悪い 
\\	賢い 
\\	悪, 賢	
\\	漂流	
\\	ひょうりゅう	
\\	する 
\\	帆が折れて、ヨットが海に漂流し始めた時は、ちょっと悲観的でしたね。	
\\	漂, 流	
\\	漂着	
\\	ひょうちゃく	
\\	する 
\\	傷ついた赤ちゃん鯨が、とある砂浜に漂着するところを想像してみてください。	
\\	着 
\\	執着 
\\	漂, 着	
\\	銘々	
\\	めいめい	
\\	うちの家族の銘々がその写真を持っています。	
\\	銘, 々	
\\	堤防	
\\	ていぼう	
\\	する 
\\	水が堤防を越えそうになっているから、近づかない方がいいですよ。	
\\	堤, 防	
\\	堤	
\\	つつみ	
\\	村長は、堤が切れるのを防ぐため、堤に人柱を埋めることを決意しました。	
\\	包み, 
\\	包み (つつみ) 
\\	堤	
\\	興奮	
\\	こうふん	
\\	する 
\\	止まれの標識を無視する度に、興奮するんです。	
\\	興奮 
\\	きょう 
\\	こう, 
\\	こういち. 
\\	こういち 
\\	興, 奮	
\\	殴り合い	
\\	なぐりあい	
\\	俺の連れは今クラブで殴り合いの喧嘩の真っ最中さ。	
\\	殴る, 
\\	殴る 
\\	合う, 
\\	殴, 合	
\\	無駄	
\\	むだ	
\\	な 
\\	言うだけ無駄だよ。フグは決して忠告には耳を貸さないんだから。	
\\	無, 駄	
\\	下駄	
\\	げた	
\\	こちらの下駄は二色のお色をご用意しております。	
\\	駄 
\\	だ 
\\	た. 
\\	下, 駄	
\\	鬱陶しい	
\\	うっとうしい	い 
\\	「この女マジ鬱陶しいんだけど。」「よー。気にすんなって。」	
\\	鬱, 陶	
\\	鬱気	
\\	うっき	
\\	伯父は、鬱気で家でふさぎ込んでいる。	
\\	鬱気を吹き飛ばしてくれるような歌ですね。	
\\	今日は鬱気が邪魔をして、日本語を勉強する気になれません。	
\\	鬱 
\\	鬱, 気	
\\	把握	
\\	はあく	
\\	する 
\\	正直、まだ事態の把握すら出来ていない状況です。	
\\	把 
\\	把, 握	
\\	墳墓	
\\	ふんぼ	
\\	ツタンカーメンの墳墓は絶対に訪れた方がいいよ。	
\\	墳, 墓	
\\	麻布	
\\	あさぬの	
\\	麻布の服が好きなんだけど、結構お値段が高いんだよね。	
\\	麻 
\\	布 
\\	麻, 布	
\\	邪魔	
\\	じゃま	
\\	する 
\\	な 
\\	私の邪魔をするな。一人で出来るよ!	
\\	邪, 魔	
\\	空洞	
\\	くうどう	
\\	白熊が木の空洞の中でうたた寝をしていると、近くに木こりがやって来ました。	
\\	空, 洞	
\\	飢え	
\\	うえ	
\\	飢えに苦しむ子どもたちの写真を見る度に胸が痛みます。	
\\	飢える 
\\	飢える.	飢	
\\	免疫	
\\	めんえき	
\\	いつも微熱とひどい頭痛があるのでお医者さんに行ったら、免疫機能が弱ってきていると言われました。	
\\	免, 疫	
\\	宮廷	
\\	きゅうてい	
\\	の 
\\	宮廷では咳風邪が流行っています。	
\\	宮 
\\	(きゅう) 
\\	宮, 廷	
\\	疫病	
\\	えきびょう	
\\	の 
\\	その村の人々は、全員疫病で亡くなりました。	
\\	疫, 病	
\\	巧妙	
\\	こうみょう	
\\	な 
\\	ある邪悪な魔法使いが、巧妙に偶然を装って女性に近づきました。	
\\	巧, 妙	
\\	無邪気	
\\	むじゃき	
\\	な 
\\	子供達は無邪気に笑い合っていた。	
\\	無, 邪, 気	
\\	皇后	
\\	こうごう	
\\	の 
\\	彼は誤って皇后であった母親を殺害してしまいました。	
\\	皇 
\\	皇, 后	
\\	液晶	
\\	えきしょう	
\\	俺の投げたライターがテレビに当たって、液晶画面にヒビが入りました。	
\\	液, 晶	
\\	壮大	
\\	そうだい	
\\	な 
\\	なんて壮大な曲なんだ。凄く感動しています。	
\\	壮, 大	
\\	漫画	
\\	まんが	
\\	漫画には色々な形式があります。	
\\	漫, 画	
\\	連峰	
\\	れんぽう	
\\	今日の連峰はよく晴れていて見晴らしがいいです。	
\\	連, 峰	
\\	駄目	
\\	だめ	
\\	な 
\\	男の子と旅行に行くのは、絶対に駄目ですからね!	
\\	駄, 目	
\\	偶に	
\\	たまに	
\\	偶にぎっくり腰になるんですよね。	
\\	弾 (たま), 
\\	偶	
\\	生涯	
\\	しょうがい	
\\	生涯ずっと独身なんて嫌だよう。	
\\	生, 涯	
\\	軌道	
\\	きどう	
\\	ロケットが軌道に入ったというニュースを聞いて、誰もが喜びました。	
\\	軌, 道	
\\	仰々しい	
\\	ぎょうぎょうしい	い 
\\	仰々しい言葉遣いをする人は嫌いです。	
\\	しい 
\\	仰, 々	
\\	壮年	
\\	そうねん	
\\	彼のお兄さんは壮年期に心臓発作を起こして亡くなりました。	
\\	壮, 年	
\\	表彰	
\\	ひょうしょう	
\\	する 
\\	表彰式には出席せずに帰ってきてしまいました。	
\\	表, 彰	
\\	検疫	
\\	けんえき	
\\	する 
\\	オーストラリアの検疫法はすっごく厳しいから、食べ物は何も持って行かない方が身のためだよ。	
\\	検, 疫	
\\	諮問	
\\	しもん	
\\	する 
\\	の 
\\	その日は経済財政諮問会議が開催される予定です。	
\\	諮, 問	
\\	輝き	
\\	かがやき	
\\	私はそのダイヤモンドの輝きに目がくらんでしまいました。	
\\	輝く 
\\	輝く.	輝	
\\	殴り込み	
\\	なぐりこみ	
\\	殴り込みの前に早めの昼食を取りました。	
\\	殴る 
\\	込む.	殴, 込	
\\	名簿	
\\	めいぼ	
\\	彼は仰向けに寝転がって、名簿の全ての名前に目を通した。	
\\	名, 簿	
\\	信仰	
\\	しんこう	
\\	する 
\\	私はキリスト教を信仰していて、神への揺るぎない思いを持っています。	
\\	仰 
\\	ぎょう 
\\	こう. 
\\	こういち.	信, 仰	
\\	〜亭	
\\	てい	
\\	多くの人が、コウイチは最初のレストランを
\\	亭」と名付けるのではないかと踏んでいる。	
\\	亭	
\\	改訂版	
\\	かいていばん	
\\	バスの時刻が改訂されたなら、改訂版の時刻表をもらわなきゃ。	
\\	改, 訂, 版	
\\	壮行	
\\	そうこう	
\\	の 
\\	彼のために金曜日壮行会を予定してるんですが、ご都合は如何ですか?	
\\	壮, 行	
\\	配偶者	
\\	はいぐうしゃ	
\\	新しい配偶者を探そうかとずっと考えているんです。	
\\	配, 偶, 者	
\\	奮起	
\\	ふんき	
\\	する 
\\	彼女の一言が、彼を奮起させ、受験勉強をさせた。	
\\	奮, 起	
\\	唐突	
\\	とうとつ	
\\	な 
\\	彼女、唐突に車の修理をしてほしいって俺に頼んできたんだよ。	
\\	唐, 突	
\\	峰	
\\	みね	
\\	妹と私は、雲の上にそびえる富士山の峰に登頂しました。	
\\	(みね) 
\\	峰	
\\	仰天	
\\	ぎょうてん	
\\	する 
\\	私は真美がどれだけ速くベーコンを食べれるのかってことにびっくり仰天しました。	
\\	仰, 天	
\\	軌跡	
\\	きせき	
\\	今日は、昨年の英語学習の軌跡を辿っています。	
\\	軌, 跡	
\\	遂げる	
\\	とげる	
\\	諦めないで!やり遂げるんだ!お前なら出来るよ!	
\\	う 
\\	(と)... 
\\	遂	
\\	灯る	
\\	ともる	
\\	あの部屋に明かりが灯るのを見たんだ。	
\\	う 
\\	(とも) 
\\	灯	
\\	諮る	
\\	はかる	
\\	この問題は取締役会に諮る必要があります。	
\\	う 
\\	(はか) 
\\	諮	
\\	漂う	
\\	ただよう	
\\	焼き芋を食べた後のおならは、他のおならよりも長い間空気中を漂う気がするよ。	
\\	う 
\\	(ただよ)! 
\\	漂	
\\	尽くす	
\\	つくす	
\\	「これ、明日までにできるかな?」「わかりました。最善を尽くします。」	
\\	"尽きる 
\\	尽くす 
\\	尽きる.	尽	
\\	翻る	
\\	ひるがえる	
\\	彼女のスカートが風で翻ったのは分かるんですが、その時の彼の心象を翻訳するのは私には難しすぎます。	
\\	う 
\\	(ひるがえ) 
\\	翻	
\\	仰ぐ	
\\	あおぐ	
\\	色んな人から、彼の指導を仰げば間違いないと言われましたが、それは明らかに間違いでした。	
\\	う 
\\	青 (あお). 
\\	仰	
\\	溶ける	
\\	とける	
\\	あの光り輝く星々が、夜空に溶けるのを想像してみてください。	
\\	"溶かす 
\\	溶ける 
\\	溶かす.	溶	
\\	悔やむ	
\\	くやむ	
\\	ガンガン響く頭と共に目を覚ます度に、前日の夜飲み過ぎたことを悔やみます。	
\\	う 
\\	悔しい.	悔	
\\	脅す	
\\	おどす	
\\	その教師は何名かの男子生徒を刃物で脅したとして解雇されました。	
\\	う 
\\	(おど). 
\\	脅	
\\	翻訳	
\\	ほんやく	
\\	する 
\\	ちょうど君の翻訳を見させてもらってたところだよ。	
\\	翻, 訳	
\\	運搬	
\\	うんぱん	
\\	する 
\\	このレストランには、食事やその他の細々したものを運搬するためのミニエレベータがあります。	
\\	運, 搬	
\\	搬送	
\\	はんそう	
\\	する 
\\	搬送作業時に左肩を脱臼してしまいました。	
\\	搬, 送	
\\	伯父	
\\	おじ	
\\	伯父は銀行員です。	
\\	父 
\\	(おじ)
\\	伯, 父	
\\	邪	
\\	よこしま	な 
\\	神様は私の邪な心を見透かしていらっしゃったんだわ。	
\\	横島 (よこしま).	邪	
\\	淀川	
\\	よどがわ	
\\	汚い水にも怯まず、彼女は淀川で泳いだ。	
\\	淀 
\\	川. 
\\	淀, 川	
\\	慈愛	
\\	じあい	
\\	彼女は慈愛に満ちた微笑を僕に向けた。	
\\	慈, 愛	
\\	耕地	
\\	こうち	
\\	この辺りには耕地が広がっています。	
\\	耕, 地	
\\	喚起	
\\	かんき	
\\	する 
\\	大学の入学式で、学生たちに飲酒やカルト宗教についての注意喚起が行われました。	
\\	喚, 起	
\\	浦	
\\	うら	
\\	駿河湾西沿岸は田子ノ浦と呼ばれます。	
\\	浦	
\\	沸点	
\\	ふってん	
\\	沸点は何度ですか。	
\\	沸, 点	
\\	瓶	
\\	びん	
\\	瓶ビールを一本ください。	
\\	瓶	
\\	瓶詰	
\\	びんづめ	
\\	する 
\\	の 
\\	メープルシロップを瓶詰しています。	
\\	瓶 
\\	詰 
\\	詰める. 
\\	詰 
\\	め 
\\	詰める 
\\	瓶, 詰	
\\	勘定	
\\	かんじょう	
\\	する 
\\	お勘定をお願いします。	
\\	定 
\\	(じょう) 
\\	勘, 定	
\\	慈善	
\\	じぜん	
\\	の 
\\	この本は慈善事業支援の目的で出版されました。	
\\	慈, 善	
\\	貞操	
\\	ていそう	
\\	あの男が彼女の貞操を奪ったのよ。	
\\	貞, 操	
\\	貞節	
\\	ていせつ	
\\	な 
\\	彼女は最後まで貞節を守ったんだ。	
\\	貞, 節	
\\	襟	
\\	えり	
\\	襟が曲がっていますよ。	
\\	襟	
\\	慈悲	
\\	じひ	
\\	祖母は本当に慈悲深い人だった。	
\\	慈, 悲	
\\	片隅	
\\	かたすみ	
\\	そのことが心の片隅にずっと引っかかっていたんです。	
\\	片, 隅	
\\	隅	
\\	すみ	
\\	この部屋の隅に、少女の霊がいます。	
\\	隅	
\\	耕作	
\\	こうさく	
\\	する 
\\	この土地を耕作するには許可がいります。	
\\	耕, 作	
\\	乾燥	
\\	かんそう	
\\	する 
\\	乾燥対策でマスクをつけています。	
\\	乾, 燥	
\\	洗濯粉	
\\	せんたくこ	
\\	洗濯粉がきれちゃった。	
\\	洗濯 
\\	粉. 
\\	粉 
\\	麦粉. 
\\	洗, 濯, 粉	
\\	郡	
\\	ぐん	
\\	この遺跡には、竪穴住居郡があります。	
\\	郡	
\\	洗濯屋	
\\	せんたくや	
\\	洗濯屋にワイシャツを取りに行ってきます。	
\\	洗濯 
\\	屋. 
\\	洗, 濯, 屋	
\\	枯渇	
\\	こかつ	
\\	する 
\\	もうすぐ石油は枯渇する。	
\\	渇 
\\	(かつ). 
\\	枯, 渇	
\\	軒	
\\	のき	
\\	ここには多くのラーメン屋台が軒を連ねています。	
\\	(のき). 
\\	軒	
\\	愛媛県	
\\	えひめけん	
\\	私は愛媛県の出身です。	
\\	愛 
\\	媛 
\\	ひめ, 
\\	県 
\\	けん. 
\\	愛: 
\\	(え). 
\\	愛, 媛, 県	
\\	空き瓶	
\\	あきびん, からびん	
\\	空瓶はこちらに捨ててください。	
\\	空 
\\	空き缶. 
\\	空瓶 
\\	からびん! 
\\	空き瓶 
\\	あきびん) 
\\	空, 瓶	
\\	玄関	
\\	げんかん	
\\	の 
\\	玄関で靴を脱いでください。	
\\	玄, 関	
\\	釈明	
\\	しゃくめい	
\\	する 
\\	これについて、正式に釈明をする必要がある。	
\\	釈, 明	
\\	解釈	
\\	かいしゃく	
\\	する 
\\	その解釈は間違っていると思います。	
\\	解, 釈	
\\	農耕	
\\	のうこう	
\\	僕のおじいさんは農耕をしています。	
\\	農, 耕	
\\	脂肪	
\\	しぼう	
\\	太ももの脂肪を落としたいんです。	
\\	脂, 肪	
\\	柔軟	
\\	じゅうなん	
\\	な 
\\	このホテルは色々な要望に柔軟に対応してくれます。	
\\	柔, 軟	
\\	四隅	
\\	よすみ	
\\	写真の四隅を暗く加工するにはどうすればいいのか教えてください。	
\\	四, 隅	
\\	偉い	
\\	えらい	い 
\\	社長がそんなに偉いんですか?	
\\	い 
\\	(えら). 
\\	偉	
\\	偉大	
\\	いだい	
\\	な 
\\	人類史上最も偉大な発明は何ですか。	
\\	偉, 大	
\\	偉人	
\\	いじん	
\\	世界の偉人たちの名言を集めました。	
\\	偉, 人	
\\	邸内	
\\	ていない	
\\	これは、邸内の写真です。	
\\	邸, 内	
\\	軟禁	
\\	なんきん	
\\	する 
\\	彼女は自宅軟禁下に置かれました。	
\\	軟, 禁	
\\	〜軒	
\\	けん	
\\	この人は、マンションを五軒持っています。	
\\	軒	
\\	慌ただしい	
\\	あわただしい	い 
\\	十二月はレストランにとって慌ただしい時期だ。	
\\	本当、カツオくんのお姉さんって慌ただしい性格ね。	
\\	出発が慌ただしく、コーヒーも飲めなかった。	
\\	い 
\\	慌	
\\	焦燥	
\\	しょうそう	
\\	する 
\\	度重なる事業の失敗で、すっかり焦燥しきっているんだ。	
\\	焦, 燥	
\\	蓮花	
\\	れんげ	
\\	この池にはいつから蓮花があるのですか。	
\\	花 
\\	(げ). 
\\	蓮, 花	
\\	邸宅	
\\	ていたく	
\\	この邸宅は、幽霊が出ることで有名です。	
\\	邸, 宅	
\\	潤い	
\\	うるおい	
\\	この化粧水が、あなたのお肌の潤いを保ちます。	
\\	潤う, 
\\	潤	
\\	火炎瓶	
\\	かえんびん	
\\	男は火炎瓶を投げつけた。	
\\	火, 炎, 瓶	
\\	塚	
\\	つか	
\\	塚を見下ろすことのは失礼です。	
\\	塚	
\\	会釈	
\\	えしゃく	
\\	する 
\\	会釈をしたら会釈を返すのが礼儀でしょう。	
\\	会 
\\	(え) 
\\	会, 釈	
\\	隅々	
\\	すみずみ	
\\	今日は部屋を隅々まで掃除しました。	
\\	隅, 々	
\\	注釈	
\\	ちゅうしゃく	
\\	する 
\\	その場合は、注釈を付けてください。	
\\	注, 釈	
\\	公邸	
\\	こうてい	
\\	大統領公邸が何者かに襲撃されました。	
\\	公, 邸	
\\	官邸	
\\	かんてい	
\\	首相官邸ツアーに参加しました。	
\\	官, 邸	
\\	肯定	
\\	こうてい	
\\	する 
\\	どうして肯定も否定もしないんですか。	
\\	肯, 定	
\\	私邸	
\\	してい	
\\	この豪華な建物は、大統領の私邸です。	
\\	私, 邸	
\\	樹脂	
\\	じゅし	
\\	の 
\\	これを透明の樹脂で固めました。	
\\	樹, 脂	
\\	蓮	
\\	はす	
\\	蓮の花の花言葉は何ですか。	
\\	(はす) 
\\	蓮	
\\	苗	
\\	なえ	
\\	四月に苗を植えました。	
\\	(なえ). 
\\	苗	
\\	〜隻	
\\	せき	
\\	港には、七隻の船が止まっています。	
\\	隻	
\\	皮膚	
\\	ひふ	
\\	の 
\\	ギターを弾くので指先の皮膚が厚いんです。	
\\	皮, 膚	
\\	聡い	
\\	さとい	い 
\\	この子は本当に聡い子だよ。	
\\	い 
\\	(さと) 
\\	聡	
\\	郊外	
\\	こうがい	
\\	の 
\\	都心と郊外、どちらに住みたいですか。	
\\	郊, 外	
\\	近郊	
\\	きんこう	
\\	私は東京近郊に住んでいます。	
\\	近, 郊	
\\	山頂	
\\	さんちょう	
\\	富士山の山頂には住所がありません。	
\\	山, 頂	
\\	召喚	
\\	しょうかん	
\\	する 
\\	の 
\\	魔法使いを召喚しました。	
\\	召, 喚	
\\	苗字	
\\	みょうじ	
\\	の 
\\	その苗字、初めて聞きました。	
\\	苗, 字	
\\	世界恐慌	
\\	せかいきょうこう	
\\	私の祖父は世界恐慌が起きた年に生まれました。	
\\	世界恐慌は様々な国に影響を与えた。	
\\	世界恐慌は失業率と共に自殺率までも上げた。	
\\	"世界 
\\	恐慌 
\\	世, 界, 恐, 慌	
\\	沸く	
\\	わく	
\\	私はお湯が沸くのを待っている。	
\\	う 
\\	(わ). 
\\	沸	
\\	覆う	
\\	おおう	
\\	私の庭は落ち葉に覆われています。	
\\	う 
\\	""おお!
\\	おお 
\\	覆	
\\	頂く	
\\	いただく	
\\	お祝いを頂き、本当に有難うございます。	
\\	いただきます? 
\\	いただく!	頂	
\\	挟む	
\\	はさむ	
\\	パンにハムと卵を挟みました。	
\\	う 
\\	挟	
\\	召す	
\\	めす	
\\	あなたのお姉さまは神に召されたのですよ。	
\\	う 
\\	目 (め) 
\\	目 
\\	召	
\\	喚く	
\\	わめく	
\\	酔っぱらいが喚いているんだよ。	
\\	う 
\\	(わめ) 
\\	喚	
\\	浸す	
\\	ひたす	
\\	どうして野菜を水に浸しているの?	
\\	う 
\\	浸る, 
\\	浸	
\\	耕す	
\\	たがやす	
\\	この機械で畑を耕すんですよ。	
\\	う 
\\	(たがや). 
\\	耕	
\\	洗濯する	
\\	せんたくする	する 
\\	セーターをお湯で洗濯したら縮んでしまった。	
\\	洗濯が好きだと言っていたから結婚したのに、今まで一度も洗濯をしてくれたことがありません。	
\\	「俺のワイシャツ、今洗濯してる?」「ああ、それなら口紅が付いていたので洗濯屋さんに出しておいたわよ。」	
\\	洗, 濯	
\\	枯れる	
\\	かれる	
\\	せっかく植えた木が枯れてしまいました。	
\\	う 
\\	(か). 
\\	枯	
\\	不貞	
\\	ふてい	
\\	な 
\\	の 
\\	妻が不貞をはたらいたので、離婚しました。	
\\	不, 貞	
\\	玄米	
\\	げんまい	
\\	玄米は体に良い。	
\\	米 
\\	(まい)! 
\\	玄, 米	
\\	花瓶	
\\	かびん	
\\	これは素敵な花瓶ですね。	
\\	花, 瓶	
\\	渦中	
\\	かちゅう	
\\	彼は離婚騒動で渦中の人だよ。	
\\	渦, 中	
\\	洗濯機	
\\	せんたくき, せんたっき	
\\	洗濯機が壊れたみたい。	
\\	洗, 濯, 機	
\\	渦	
\\	うず	
\\	海に渦が発生するのはどうして?	
\\	(うず). 
\\	渦	
\\	聡明	
\\	そうめい	
\\	な 
\\	三村崇は非常に聡明な少年だった。	
\\	聡, 明	
\\	殊勝	
\\	しゅしょう	
\\	な 
\\	あいつは殊勝な女だったよ。	
\\	殊, 勝	
\\	不倫	
\\	ふりん	
\\	な 
\\	の 
\\	どうやら妻が不倫をしているようなんです。	
\\	不, 倫	
\\	没後	
\\	ぼつご	
\\	没後50年を記念した展覧会を予定しています。	
\\	没, 後	
\\	覇気	
\\	はき	
\\	うちの息子は、若いのに覇気が無い。	
\\	覇, 気	
\\	出没	
\\	しゅつぼつ	
\\	する 
\\	日本各地でクマが大量出没している。	
\\	出, 没	
\\	擁立	
\\	ようりつ	
\\	する 
\\	どうせ出馬者を擁立しようとしているだけだろう。	
\\	擁, 立	
\\	お風呂	
\\	おふろ	
\\	ゆっくりお風呂に浸かりたいなあ。	
\\	風, 
\\	ふ. 
\\	(ふ) 
\\	う 
\\	風, 呂	
\\	封鎖	
\\	ふうさ	
\\	する 
\\	銀行が潰れて、預金が封鎖されてしまったんです。	
\\	封, 鎖	
\\	陥没	
\\	かんぼつ	
\\	する 
\\	地震であそこの道路が陥没したそうです。	
\\	陥, 没	
\\	鎖	
\\	くさり	
\\	逃げないように鎖で繋いでいます。	
\\	(くさり). 
\\	鎖	
\\	連鎖	
\\	れんさ	
\\	する 
\\	まずはこの負の連鎖を断ち切る必要があります。	
\\	連, 鎖	
\\	突貫	
\\	とっかん	
\\	する 
\\	突貫工事で犬小屋を建てました。	
\\	突, 貫	
\\	慢性	
\\	まんせい	
\\	の 
\\	私は慢性胃腸炎を患っているんです。	
\\	慢, 性	
\\	陳腐	
\\	ちんぷ	
\\	な 
\\	陳腐な台詞しか思い浮かびません。	
\\	ふ 
\\	ぷ.	陳, 腐	
\\	閉鎖	
\\	へいさ	
\\	する 
\\	の 
\\	このプールは今月で閉鎖されます。	
\\	閉, 鎖	
\\	冷遇	
\\	れいぐう	
\\	する 
\\	彼は実力があるのにどうしてこんなに冷遇されているんですか。	
\\	冷, 遇	
\\	制覇	
\\	せいは	
\\	する 
\\	わが校の野球部が、秋のリーグ戦を制覇しました。	
\\	制, 覇	
\\	恒常	
\\	こうじょう	
\\	このシステムで部屋の温度を恒常に保っています。	
\\	恒, 常	
\\	欠陥	
\\	けっかん	
\\	消費者からこの商品の安全性の欠陥が指摘されています。	
\\	欠, 陥	
\\	遭難	
\\	そうなん	
\\	する 
\\	危うく遭難するところでした。	
\\	遭, 難	
\\	没	
\\	ぼつ	
\\	こないだの原稿は没になってしまった。	
\\	没	
\\	抱擁	
\\	ほうよう	
\\	する 
\\	その映画の、主役の二人が甘い抱擁を交わすシーンがとても美しかったよ。	
\\	抱 
\\	(ほう)! 
\\	抱, 擁	
\\	遭遇	
\\	そうぐう	
\\	する 
\\	悲惨な交通事故に遭遇しました。	
\\	遭, 遇	
\\	我慢	
\\	がまん	
\\	する 
\\	おしっこを我慢するのは体に悪い。	
\\	我, 慢	
\\	境遇	
\\	きょうぐう	
\\	自分の不幸を境遇のせいにするんじゃない。	
\\	境, 遇	
\\	噴火	
\\	ふんか	
\\	する 
\\	の 
\\	富士山が噴火したらどうなると思いますか。	
\\	噴, 火	
\\	恒例	
\\	こうれい	
\\	うちの会社では、毎年恒例のクリスマス会があるんだよ。	
\\	恒, 例	
\\	倫理	
\\	りんり	
\\	の 
\\	ちゃんと職業倫理を守ってください。	
\\	倫, 理	
\\	陳列	
\\	ちんれつ	
\\	する 
\\	あなたは、パン屋のむき出しの陳列についてどう思いますか。	
\\	陳, 列	
\\	膨張	
\\	ぼうちょう	
\\	する 
\\	の 
\\	宇宙は膨張しています。	
\\	膨, 張	
\\	隼	
\\	はやぶさ	
\\	俺の一番好きなバイクは、スズキの隼です。	
\\	隼	
\\	陥落	
\\	かんらく	
\\	する 
\\	トヨタが世界一から陥落したのは何故だと思いますか。	
\\	陥, 落	
\\	没収	
\\	ぼっしゅう	
\\	する 
\\	の 
\\	先生に携帯電話を没収されました。	
\\	没, 収	
\\	覇権	
\\	はけん	
\\	の 
\\	あの男が党内の覇権を握っているのです。	
\\	覇, 権	
\\	膨大	
\\	ぼうだい	
\\	な 
\\	の 
\\	これからこの膨大な資料に目を通さなくてはいけないんだ。	
\\	膨, 大	
\\	猟師	
\\	りょうし	
\\	私の叔父は猟師です。	
\\	猟, 師	
\\	狩猟	
\\	しゅりょう	
\\	する 
\\	の 
\\	この鳥の狩猟は禁じられています。	
\\	狩 
\\	(しゅ). 
\\	狩, 猟	
\\	猟	
\\	りょう	
\\	の 
\\	これから猟に行くところだよ。	
\\	猟	
\\	猟犬	
\\	りょうけん	
\\	一般的に猟犬はよく吠えると言われています。	
\\	犬 
\\	(けん) 
\\	猟, 犬	
\\	没頭	
\\	ぼっとう	
\\	する 
\\	好きなことに没頭できるなんて、幸せですね。	
\\	没, 頭	
\\	埋没	
\\	まいぼつ	
\\	する 
\\	整形手術で二重の埋没手術をしました。	
\\	埋 
\\	(まい). 
\\	埋, 没	
\\	必須	
\\	ひっす	
\\	な 
\\	の 
\\	デフォルトでは、名前と生年月日、メールアドレスの入力が必須になっています。	
\\	必, 須	
\\	惰性	
\\	だせい	
\\	僕は、勉強への意欲は無く、ただ惰性で大学へ通っていました。	
\\	惰, 性	
\\	孤独	
\\	こどく	
\\	な 
\\	の 
\\	ずっと孤独な人生を送ってきたんです。	
\\	孤, 独	
\\	孤立	
\\	こりつ	
\\	する 
\\	ある事件があって、職場で孤立してしまったんです。	
\\	孤, 立	
\\	噴射	
\\	ふんしゃ	
\\	する 
\\	催涙スプレーの噴射方法を教えてください。	
\\	噴, 射	
\\	戦没	
\\	せんぼつ	
\\	する 
\\	たくさんの戦没した兵士たちがここに埋められています。	
\\	戦, 没	
\\	怠惰	
\\	たいだ	
\\	な 
\\	どうすれば怠惰を克服できるのでしょうか。	
\\	怠, 惰	
\\	偏狭	
\\	へんきょう	
\\	な 
\\	ステレオタイプは人間を偏狭にします。	
\\	狭 
\\	(きょう). 
\\	偏, 狭	
\\	鎖国	
\\	さこく	
\\	する 
\\	日本が鎖国していたのはいつですか。	
\\	鎖, 国	
\\	怠慢	
\\	たいまん	
\\	な 
\\	の 
\\	それは警察の怠慢なんじゃないんですか。	
\\	怠, 慢	
\\	一貫	
\\	いっかん	
\\	する 
\\	最高級のトロを一貫だけ注文しました。	
\\	一, 貫	
\\	秩序	
\\	ちつじょ	
\\	秩序を乱すようなことはしないでください。	
\\	秩, 序	
\\	発祥	
\\	はっしょう	
\\	する 
\\	デコポンは日本発祥のみかんの品種です。	
\\	発, 祥	
\\	孤児	
\\	こじ	
\\	の 
\\	妻と相談して、孤児を引き取ることにしました。	
\\	孤, 児	
\\	芳香	
\\	ほうこう	
\\	の 
\\	お風呂あがりには、アロマオイルの芳香を楽しみます。	
\\	芳, 香	
\\	恒久	
\\	こうきゅう	
\\	の 
\\	恒久の平和を祈っています。	
\\	恒, 久	
\\	貫徹	
\\	かんてつ	
\\	する 
\\	大変だったけど、みんなで決めた計画を貫徹することができた。	
\\	十年前の夢を貫徹できなかったことを残念に思います。	
\\	初志貫徹の心構えで頑張ります。	
\\	貫, 徹	
\\	陳情	
\\	ちんじょう	
\\	する 
\\	請願・陳情は、住民の意見や要望を反映させるための制度です。	
\\	陳, 情	
\\	貫通	
\\	かんつう	
\\	する 
\\	の 
\\	ブラジャーのワイヤーが弾丸の貫通を防いでくれた。	
\\	貫, 通	
\\	密猟	
\\	みつりょう	
\\	する 
\\	どうすれば象の密猟を防止できると思いますか。	
\\	密, 猟	
\\	優遇	
\\	ゆうぐう	
\\	する 
\\	の 
\\	高齢者を優遇しすぎじゃないでしょうか。	
\\	優, 遇	
\\	偏見	
\\	へんけん	
\\	それが偏見というものです。	
\\	偏見が全くない世界なんてものは存在しない。	
\\	彼女はアジア人に偏見があるようだ。	
\\	偏, 見	
\\	自慢	
\\	じまん	
\\	する 
\\	の 
\\	自分の不倫を得意気に自慢するとか、ありえないんだけど。	
\\	自, 慢	
\\	擁護	
\\	ようご	
\\	する 
\\	君はどうして彼を擁護するような発言をしたんだい。	
\\	擁, 護	
\\	食糧	
\\	しょくりょう	
\\	食糧の備蓄は十分にあります。	
\\	食, 糧	
\\	颯と	
\\	さっと	
\\	茜は颯と涙を拭った。	
\\	さっ 
\\	さつ. 
\\	さつ 
\\	颯	
\\	賠償	
\\	ばいしょう	
\\	する 
\\	被害総額の全額を賠償してください。	
\\	賠, 償	
\\	没落	
\\	ぼつらく	
\\	する 
\\	本当に日本は没落していると思いますか?	
\\	没, 落	
\\	不祥事	
\\	ふしょうじ	
\\	警察は自分たちの不祥事を隠蔽しようとしているんだ。	
\\	不, 祥, 事	
\\	沈没	
\\	ちんぼつ	
\\	する 
\\	タイタニック号の沈没を予言していた人がいるんです。	
\\	沈, 没	
\\	緩慢	
\\	かんまん	
\\	な 
\\	あのウエイトレスは動きが緩慢なので首にしたよ。	
\\	緩 
\\	(かん). 
\\	緩, 慢	
\\	孤島	
\\	ことう	
\\	今は孤島を舞台とした小説を書いています。	
\\	島 
\\	(とう). 
\\	孤, 島	
\\	特殊	
\\	とくしゅ	
\\	な 
\\	の 
\\	この車のタイヤはちょっと特殊なんだよ。	
\\	特, 殊	
\\	噴出	
\\	ふんしゅつ	
\\	する 
\\	ブラックホールからガスが噴出しているって本当ですか?	
\\	噴, 出	
\\	恒星	
\\	こうせい	
\\	の 
\\	ブラックホールが、恒星をばらばらに引き裂くことがあるらしい。	
\\	恒, 星	
\\	陳述	
\\	ちんじゅつ	
\\	する 
\\	まず、被告人が冒頭陳述を行います。	
\\	陳, 述	
\\	連覇	
\\	れんぱ	
\\	する 
\\	今までに甲子園を三連覇した高校はありますか?	
\\	は 
\\	ぱ.	連, 覇	
\\	示唆	
\\	しさ	
\\	する 
\\	メッシが、移籍を示唆するコメントをしたようだ。	
\\	示 
\\	(し). 
\\	示, 唆	
\\	日没	
\\	にちぼつ	
\\	今出発すれば、日没までには着けると思います。	
\\	日, 没	
\\	覇者	
\\	はしゃ	
\\	あいつらが前年度のこの大会の覇者です。	
\\	覇, 者	
\\	〜貫	
\\	かん	
\\	マグロを二貫ください。	
\\	貫	
\\	芳しい	
\\	かんばしい	い 
\\	この季節は、梅の香りがとても芳しいですね。	
\\	い 
\\	(かんば) 
\\	芳	
\\	茨	
\\	いばら	
\\	この程度の苦難で、茨の道を歩んでいるような人生だとか、よく言えるよね。	
\\	茨	
\\	偏る	
\\	かたよる	
\\	栄養が偏るのは良くない。	
\\	う 
\\	(かたよ)! 
\\	偏	
\\	怠る	
\\	おこたる	
\\	この失敗は確認を怠った私の責任です。	
\\	う 
\\	(おこた)
\\	怠	
\\	陥る	
\\	おちいる	
\\	このままでは、価格競争に陥るのがオチですよ。	
\\	う 
\\	(おちい)!!
\\	陥	
\\	貫く	
\\	つらぬく	
\\	父は最後まで自分の信念を貫き通しました。	
\\	う 
\\	(つらぬ). 
\\	貫	
\\	擁する	
\\	ようする	する 
\\	優勝候補は、天才バッターを擁するトフグ学園です。	
\\	擁	
\\	遭う	
\\	あう	
\\	露骨な人種差別に遭ったことはありますか。	
\\	う 
\\	会う 
\\	あう 
\\	遭	
\\	膨れる	
\\	ふくれる	
\\	食べ過ぎてお腹がパンパンに膨れています。	
\\	う 
\\	(ふく) 
\\	膨	
\\	待遇	
\\	たいぐう	
\\	する 
\\	このカードのお陰で、
\\	待遇を味わうことができました。	
\\	待, 遇	
\\	処遇	
\\	しょぐう	
\\	する 
\\	不正を行った社員の処遇を考えなくてはいけません。	
\\	処, 遇	
\\	偏食	
\\	へんしょく	
\\	する 
\\	子供が偏食がちなので困っています。	
\\	偏, 食	
\\	犠牲	
\\	ぎせい	
\\	だれかを犠牲にしてまで成功したいかい?	
\\	犠, 牲	
\\	噴煙	
\\	ふんえん	
\\	噴煙を上げる桜島の写真を撮りました。	
\\	噴, 煙	
\\	鰐蟹	
\\	わにかに	
\\	鰐蟹で漢字を猛勉強中です。	
\\	鰐, 蟹	
\\	噴水	
\\	ふんすい	
\\	あそこの噴水で待ち合わせをしましょう。	
\\	噴, 水	
\\	虐待	
\\	ぎゃくたい	
\\	する 
\\	の 
\\	虐待を受けた子供は、自己評価が非常に低くなることが多い。	
\\	虐, 待	
\\	軒並	
\\	のきなみ	
\\	関連の株価が軒並み下落しました。	
\\	軒 
\\	並 
\\	軒, 並	
\\	稿料	
\\	こうりょう	
\\	わずかですが、稿料をもらいました。	
\\	稿, 料	
\\	披露	
\\	ひろう	
\\	する 
\\	誕生日会で、手品を披露しました。	
\\	露 
\\	(ろう) 
\\	披, 露	
\\	随時	
\\	ずいじ	
\\	新しい情報が入れば、随時ご連絡致します。	
\\	随, 時	
\\	虐殺	
\\	ぎゃくさつ	
\\	する 
\\	南京大虐殺が捏造だったというなら、その証拠を説明してください。	
\\	虐, 殺	
\\	真鯉	
\\	まごい	
\\	真鯉の刺青を背中に彫りました。	
\\	真, 鯉	
\\	搭乗	
\\	とうじょう	
\\	する 
\\	ご搭乗の際は、足元にお気をつけください。	
\\	乗 
\\	(じょう), 
\\	搭, 乗	
\\	競艇	
\\	きょうてい	
\\	の 
\\	競艇で大儲けしました。	
\\	競, 艇	
\\	沸騰	
\\	ふっとう	
\\	する 
\\	スープが沸騰してグツグツいってますよ。	
\\	沸, 騰	
\\	同胞	
\\	どうほう, どうぼう	
\\	お前は同胞を裏切るつもりか。	
\\	同, 胞	
\\	錦	
\\	にしき	
\\	高級錦織で作られた着物を買いました。	
\\	錦	
\\	搭載	
\\	とうさい	
\\	する 
\\	今度の機種には、どんな新機能が搭載されるんでしょうか。	
\\	搭, 載	
\\	鯉	
\\	こい	
\\	鯉に餌をあげないでください。	
\\	鯉	
\\	細胞	
\\	さいぼう, さいほう	
\\	の 
\\	私は、人工多能性幹細胞
\\	細胞)の研究をしています。	
\\	細, 胞	
\\	浄水	
\\	じょうすい	
\\	三十万円の浄水器を売りつけられそうになりました。	
\\	浄, 水	
\\	襟元	
\\	えりもと	
\\	襟元にファンデーションがついていますよ。	
\\	襟, 元	
\\	錦鯉	
\\	にしきごい	
\\	庭の池に錦鯉を放ちました。	
\\	錦, 鯉	
\\	総帥	
\\	そうすい	
\\	それは、ヒトラー総帥の指令だったのです。	
\\	総, 帥	
\\	頂	
\\	いただき	
\\	頂きものなんですが、良かったらどうぞ。	
\\	頂く. 
\\	頂	
\\	残虐	
\\	ざんぎゃく	
\\	な 
\\	これは非常に残虐な犯行です。	
\\	残, 虐	
\\	惨敗	
\\	ざんぱい	
\\	する 
\\	日本はワールドカップで惨敗してしまった。	
\\	惨, 敗	
\\	胡瓜	
\\	きゅうり	
\\	たまに胡瓜の糠漬けが無性に食べたくなるんだ。	
\\	(きゅうり)! 
\\	胡, 瓜	
\\	曙	
\\	あけぼの	
\\	春の曙の空はいくら見ても見飽きません。	
\\	曙	
\\	枯れ木	
\\	かれき	
\\	枯れ木に花を咲かせましょう。	
\\	枯木 
\\	枯, 木	
\\	繊細	
\\	せんさい	
\\	な 
\\	上品で繊細な味の日本食が好きです。	
\\	繊, 細	
\\	皮膚科	
\\	ひふか	
\\	私は皮膚科に通院しています。	
\\	皮膚 
\\	皮, 膚, 科	
\\	経緯	
\\	けいい, いきさつ	
\\	彼は、これまでの経緯を静かに語りだした。	
\\	経, 緯	
\\	丹念	
\\	たんねん	
\\	な 
\\	熟練の染師が、一つ一つ丹念に染め上げました。	
\\	丹, 念	
\\	惨事	
\\	さんじ	
\\	惨事を招くことは避けたい。	
\\	惨, 事	
\\	悲惨	
\\	ひさん	
\\	な 
\\	残された加害者の家族の末路は、悲惨なものだった。	
\\	悲, 惨	
\\	惨状	
\\	さんじょう	
\\	東北大震災の惨状を忘れることができません。	
\\	惨, 状	
\\	啓発	
\\	けいはつ	
\\	する 
\\	最近は、自己啓発本ばかり読んでいます。	
\\	啓, 発	
\\	啓蒙	
\\	けいもう	
\\	する 
\\	我々が大衆を啓蒙するのです。	
\\	啓, 蒙	
\\	随筆	
\\	ずいひつ	
\\	随筆の書き方を教えてください。	
\\	随, 筆	
\\	随所	
\\	ずいしょ	
\\	この寿司屋には、店の随所に、日本の美を感じさせるインテリアが飾られています。	
\\	随, 所	
\\	過剰	
\\	かじょう	
\\	な 
\\	過剰包装は紙の無駄です。	
\\	過, 剰	
\\	玄人	
\\	くろうと	
\\	の 
\\	水商売の玄人と素人を見分けるのが趣味です。	
\\	(くろうと). 
\\	玄, 人	
\\	脂身	
\\	あぶらみ	
\\	お肉の脂身って苦手なんですよね。	
\\	脂 
\\	(あぶら).	脂, 身	
\\	繊維	
\\	せんい	
\\	の 
\\	毎日サラダを食べて、ちゃんと食物繊維を摂るようにしています。	
\\	繊, 維	
\\	追随	
\\	ついずい	
\\	する 
\\	この会社は、圧倒的な強さで、他社の追随を許しませんでした。	
\\	追 
\\	(つい)! 
\\	追, 随	
\\	元帥	
\\	げんすい	
\\	これが、あの有名な最高司令官のマッカ
\\	サ
\\	元帥です。	
\\	元, 帥	
\\	徐行	
\\	じょこう	
\\	する 
\\	住宅地や学校の近くでは、徐行で運転してください。	
\\	徐, 行	
\\	徐々	
\\	じょじょ	
\\	の 
\\	二学期に入って、徐々に成績が上がり始めました。	
\\	徐, 々	
\\	本舗	
\\	ほんぽ	
\\	ベビー用品は、いつも赤ちゃん本舗で買います。	
\\	本, 舗	
\\	葵	
\\	あおい	
\\	徳川氏の家紋といえば、「三つ葉葵」が一番有名だ。	
\\	葵	
\\	頂戴	
\\	ちょうだい	
\\	する 
\\	ママのこの指輪、大きくなったら私に頂戴ね。	
\\	たい 
\\	だい 
\\	頂, 戴	
\\	緯度	
\\	いど	
\\	の 
\\	で経度や緯度が検索できます。	
\\	緯, 度	
\\	蓮根	
\\	れんこん, はすね	
\\	蓮根と人参のキンピラを作りました。	
\\	蓮, 根	
\\	原稿	
\\	げんこう	
\\	原稿に珈琲を零してしまいました。	
\\	原, 稿	
\\	瓜	
\\	うり	
\\	これは瓜の漬物です。	
\\	(うり)! 
\\	瓜	
\\	艦艇	
\\	かんてい	
\\	海軍の艦艇の乗組員が不足しています。	
\\	艦, 艇	
\\	惨め	
\\	みじめ	な 
\\	そんなことを言われて、人がどれだけ惨めな気持ちになるのか分かる?	
\\	短い (みじ) 
\\	惨	
\\	蒙古	
\\	もうこ	
\\	この絵には、蒙古襲来の時の様子が描かれています。	
\\	蒙, 古	
\\	北緯	
\\	ほくい	
\\	江ノ島とギリシャのクレタ島の北緯は同じです。	
\\	北, 緯	
\\	草稿	
\\	そうこう	
\\	スピーチの草稿を作成しています。	
\\	草, 稿	
\\	苗床	
\\	なえどこ	
\\	今日は夫と苗床を作りました。	
\\	苗, 床	
\\	緯線	
\\	いせん	
\\	緯度0度の緯線のことを、赤道と呼びます。	
\\	緯, 線	
\\	舗装	
\\	ほそう	
\\	する 
\\	この道路はまだ舗装されていない。	
\\	舗, 装	
\\	清浄	
\\	せいじょう, しょうじょう	
\\	な 
\\	の 
\\	真に清浄な心を持った人間などいるのでしょうか。	
\\	清, 浄	
\\	浄土	
\\	じょうど	
\\	そこはまさに極楽浄土のような所だった。	
\\	浄, 土	
\\	浄化	
\\	じょうか	
\\	する 
\\	の 
\\	私は、放射能汚染水を浄化する施設を設計しています。	
\\	浄, 化	
\\	平壌	
\\	へいじょう, ぴょんやん	
\\	平壌で英語を教えていたことがあります。	
\\	ピョンヤン, 
\\	平, 壌	
\\	批准	
\\	ひじゅん	
\\	する 
\\	新戦略兵器削減条約の批准を最優先課題としています。	
\\	批, 准	
\\	緋鯉	
\\	ひごい	
\\	ニシキゴイは緋鯉をもとに改良されたものです。	
\\	緋, 鯉	
\\	余剰	
\\	よじょう	
\\	の 
\\	余剰資金で株を買いました。	
\\	余, 剰	
\\	啓示	
\\	けいじ	
\\	する 
\\	それはまるで神の啓示のようだった。	
\\	啓, 示	
\\	襟巻き	
\\	えりまき	
\\	ピンクの襟巻きを巻いているのが私の妹です。	
\\	巻く 
\\	襟, 巻	
\\	苗木	
\\	なえぎ	
\\	この盆栽は苗木から育てたんです。	
\\	苗, 木	
\\	自浄	
\\	じじょう	
\\	する 
\\	この事件が業界の自浄を促すだろう。	
\\	自, 浄	
\\	店舗	
\\	てんぽ	
\\	空港の近くにここのレンタカーの店舗はありますか?	
\\	店, 舗	
\\	舗	
\\	ほ, ぽ	
\\	おかげ様で自分の舗を構えることができました。	
\\	舗	
\\	胡座	
\\	あぐら	
\\	こんなところで胡座をかいてないで、さっさと仕事にとりかかりなさい。	
\\	(あぐら). 
\\	胡, 座	
\\	皮膚病	
\\	ひふびょう	
\\	息子は皮膚病を患っています。	
\\	皮膚 
\\	皮, 膚, 病	
\\	土壌	
\\	どじょう	
\\	土壌が悪い時は育てられない。	
\\	土, 壌	
\\	南緯	
\\	なんい	
\\	理論上は、南緯45度が南極点から赤道までの距離の中間地点であるとされる。	
\\	南, 緯	
\\	胞子	
\\	ほうし	
\\	の 
\\	キノコの胞子はとても小さいので、肉眼で形は分かりません。	
\\	胞, 子	
\\	投稿	
\\	とうこう	
\\	する 
\\	フェイスブックに犬の写真を投稿しました。	
\\	投, 稿	
\\	高騰	
\\	こうとう	
\\	する 
\\	どうして株価がこんなに高騰しているんだ。	
\\	高, 騰	
\\	急騰	
\\	きゅうとう	
\\	する 
\\	ドルが再び急騰しています。	
\\	急, 騰	
\\	暴騰	
\\	ぼうとう	
\\	する 
\\	住宅価格が暴騰している。	
\\	暴, 騰	
\\	統帥	
\\	とうすい	
\\	する 
\\	今日は学校で、天皇の統帥権について学びました。	
\\	統, 帥	
\\	剰余金	
\\	じょうよきん	
\\	会社の利益剰余金を資本金に振替えるつもりです。	
\\	剰, 余, 金	
\\	合繊	
\\	ごうせん	
\\	原油安やナフサ安を受けて、合繊の価格も下落しています。	
\\	合, 繊	
\\	化繊	
\\	かせん	
\\	掃除には、化繊箒を使っています。	
\\	化, 繊	
\\	教諭	
\\	きょうゆ	
\\	する 
\\	あの教諭、昔の教え子と結婚したらしいよ。	
\\	教, 諭	
\\	諭す	
\\	さとす	
\\	うちの父は俺のことを叱るけど、本当は諭すのが正解だと思う。	
\\	う 
\\	(さと) 
\\	諭	
\\	沸かす	
\\	わかす	
\\	お風呂を沸かしておきましたよ。	
\\	沸く 
\\	沸く, 
\\	沸	
\\	挟まる	
\\	はさまる	
\\	アスパラガスが歯に挟まっています。	
\\	挟む 
\\	挟む, 
\\	挟	
\\	据える	
\\	すえる	
\\	今度、この機械を工場に据えようと思っているんだ。	
\\	う 
\\	据	
\\	寛大	
\\	かんだい	
\\	な 
\\	それは寛大な処置だったと思います。	
\\	寛, 大	
\\	寛容	
\\	かんよう	
\\	な 
\\	自分には厳しいけど、他人には寛容です。	
\\	寛, 容	
\\	枯らす	
\\	からす	
\\	水をやりすぎて、サボテンを枯らしてしまいました。	
\\	枯れる 
\\	枯れる, 
\\	枯	
\\	肯く	
\\	うなずく, うなづく	
\\	母は話を聞きながら、ウンウンと肯きました。	
\\	(うなず). 
\\	肯	
\\	虐げる	
\\	しいたげる	
\\	あの男は、妻と子供をずっと虐げてきたのよ。	
\\	う 
\\	(しいた). 
\\	い 
\\	い. 
\\	虐	
\\	寛ぐ	
\\	くつろぐ	
\\	まるで自分の家のように寛いでいました。	
\\	う 
\\	(くつろ). 
\\	寛	
\\	召し上がる	
\\	めしあがる	
\\	もしよければお召し上がり下さい。	
\\	召す 
\\	上がる, 
\\	召す 
\\	上がる 
\\	召す 
\\	召し 
\\	上がる. 
\\	召, 上	
\\	暴虐	
\\	ぼうぎゃく	
\\	な 
\\	あいつらは暴虐の限りを尽くしたんだ。	
\\	暴, 虐	
\\	不浄	
\\	ふじょう	
\\	な 
\\	の 
\\	インドでは、左手を不浄の手として、右手でご飯を食べる。	
\\	不, 浄	
\\	随分	
\\	ずいぶん	
\\	な 
\\	随分大胆なことをしたんですね。	
\\	随, 分	
\\	洗浄	
\\	せんじょう	
\\	する 
\\	の 
\\	電気ポットはどうやって洗浄してますか?	
\\	洗, 浄	
\\	西瓜	
\\	すいか	
\\	夏といえば西瓜ですよね。	
\\	西 
\\	(すい) 
\\	西, 瓜	
\\	渦巻き	
\\	うずまき	
\\	蚊取り線香はどうして渦巻き状になっているんですか。	
\\	渦 
\\	巻く 
\\	渦 
\\	巻く. 巻く 
\\	巻き 
\\	渦 
\\	渦, 巻	
\\	丹誠	
\\	たんせい	
\\	する 
\\	お婆ちゃんが丹誠込めて育てたトマトだよ。	
\\	丹, 誠	
\\	顕在	
\\	けんざい	
\\	する 
\\	この絵は、人間の顕在意識と潜在意識を象徴しています。	
\\	顕, 在	
\\	逸話	
\\	いつわ	
\\	の 
\\	この学校には面白い逸話があってね。	
\\	逸, 話	
\\	杏	
\\	あんず	
\\	干し杏のジャムを作りました。	
\\	杏	
\\	兵糧	
\\	ひょうろう	
\\	兵糧攻めをするつもりなんだ。	
\\	(ひょうろう) 
\\	兵, 糧	
\\	自叙伝	
\\	じじょでん	
\\	電子書籍で自叙伝を販売しようと思っています。	
\\	自, 叙, 伝	
\\	哺乳瓶	
\\	ほにゅうびん	
\\	哺乳瓶を熱湯消毒しています。	
\\	哺, 乳, 瓶	
\\	風呂場	
\\	ふろば	
\\	風呂場で滑ってお尻の骨を折りました。	
\\	風呂 
\\	風呂 
\\	場. 
\\	風, 呂, 場	
\\	偏り	
\\	かたより	
\\	あなたって、考え方にすごく偏りがあるよね。	
\\	偏る, 
\\	偏	
\\	空欄	
\\	くうらん	
\\	空欄に答えを書き込んでください。	
\\	空, 欄	
\\	盆栽	
\\	ぼんさい	
\\	アメリカでも盆栽が買えるんですか?	
\\	盆, 栽	
\\	栞	
\\	しおり	
\\	本を読む時に栞は使いますか。	
\\	栞	
\\	和尚	
\\	おしょう	
\\	昔々あるところに、水飴が大好きな和尚がおりました。	
\\	和 
\\	(お). 
\\	和, 尚	
\\	欄	
\\	らん	
\\	この欄は空けておいてください。	
\\	欄	
\\	叙勲	
\\	じょくん	
\\	毎年、春と秋に春秋叙勲が実施されます。	
\\	叙, 勲	
\\	紋章	
\\	もんしょう	
\\	これは、王家の紋章です。	
\\	紋, 章	
\\	王冠	
\\	おうかん	
\\	の 
\\	彼は恋をして王冠を捨てたんだ。	
\\	王, 冠	
\\	冷酷	
\\	れいこく	
\\	な 
\\	あなたのこと、ずっと冷酷な人だと思っていました。	
\\	冷, 酷	
\\	酷い	
\\	ひどい	い 
\\	彼女の作った料理は、どれも酷い味だった。	
\\	い 
\\	(ひど) 
\\	酷	
\\	残酷	
\\	ざんこく	
\\	な 
\\	私は、ひどく残酷な決断を迫られました。	
\\	残, 酷	
\\	栽培	
\\	さいばい	
\\	する 
\\	庭でトマトを栽培しているんです。	
\\	栽, 培	
\\	勲章	
\\	くんしょう	
\\	どうして勲章の受賞を辞退したんですか。	
\\	勲, 章	
\\	痴呆	
\\	ちほう	
\\	の 
\\	祖母の痴呆が始まったんです。	
\\	痴, 呆	
\\	過疎	
\\	かそ	
\\	の 
\\	日本には、過疎に悩む地域がたくさんあります。	
\\	過, 疎	
\\	叙述	
\\	じょじゅつ	
\\	する 
\\	の 
\\	この授業では、挿し絵や写真と叙述を結び付けることを学びます。	
\\	叙, 述	
\\	逸脱	
\\	いつだつ	
\\	する 
\\	任務を逸脱する行為は慎み給え。	
\\	逸, 脱	
\\	露顕	
\\	ろけん	
\\	する 
\\	どうして我々の陰謀が露顕したのだ。	
\\	露, 顕	
\\	疾患	
\\	しっかん	
\\	の 
\\	私の妹は、精神疾患を抱えています。	
\\	疾, 患	
\\	無秩序	
\\	むちつじょ	
\\	な 
\\	戸棚の中には、何千本というマニキュアが無秩序に並べられていた。	
\\	秩序 
\\	無, 秩, 序	
\\	叙事詩	
\\	じょじし	
\\	『ギルガメシュ叙事詩』は、古代メソポタミアの文学作品です。	
\\	叙, 事, 詩	
\\	殊に	
\\	ことに	
\\	殊に、来年はコウイチ様生誕100周年を記念するメモリアル・イヤーとなります。	
\\	に, 
\\	(こと). 
\\	殊	
\\	必須条件	
\\	ひっすじょうけん	
\\	お肌の張りと艶は、モテる女の必須条件です。	
\\	必須 
\\	条件 
\\	必, 須, 条, 件	
\\	疎遠	
\\	そえん	
\\	な 
\\	の 
\\	それっきり、姉とは疎遠になっているんです。	
\\	疎, 遠	
\\	疎外	
\\	そがい	
\\	する 
\\	学校ではずっと疎外感を感じていました。	
\\	疎, 外	
\\	悠久	
\\	ゆうきゅう	
\\	な 
\\	の 
\\	悠久な大自然で、人生について色々なことを考えてみました。	
\\	悠, 久	
\\	倫理的	
\\	りんりてき	な 
\\	うちの会社は、世界で最も倫理的な企業の一つに選ばれました。	
\\	倫, 理, 的	
\\	倫理学	
\\	りんりがく	
\\	医学倫理学の講座を受けています。	
\\	倫理 
\\	倫, 理, 学	
\\	疎開	
\\	そかい	
\\	する 
\\	空襲を避けるために田舎へ疎開していました。	
\\	疎, 開	
\\	秀逸	
\\	しゅういつ	
\\	な 
\\	この広告は非常に秀逸だと評判です。	
\\	秀, 逸	
\\	応酬	
\\	おうしゅう	
\\	する 
\\	両チームのファンからやじの応酬が続いた。	
\\	応, 酬	
\\	酷使	
\\	こくし	
\\	する 
\\	労働者を酷使するブラック企業をこのまま野放しにしていてもいいんですか。	
\\	酷, 使	
\\	露呈	
\\	ろてい	
\\	する 
\\	あいつがしていたインチキが全部露呈したんだよ。	
\\	露, 呈	
\\	紋	
\\	もん	
\\	新郎が紋付袴を着ている写真はありますか。	
\\	紋	
\\	指紋	
\\	しもん	
\\	ドアノブの指紋を綺麗に拭き取った。	
\\	指, 紋	
\\	民謡	
\\	みんよう	
\\	最近は毎日ロシア民謡を聞いています。	
\\	民, 謡	
\\	阿呆	
\\	あほう, あほ	
\\	な 
\\	学生の頃は、いつも阿呆なことばかりしてたな。	
\\	阿, 呆	
\\	愚	
\\	ぐ	
\\	な 
\\	まさに愚の骨頂だね。	
\\	愚	
\\	尚	
\\	なお	
\\	温めて食べたら、尚、おいしくなりますよ。	
\\	(なお)!	尚	
\\	疾風	
\\	しっぷう, はやて	
\\	真っ黒な犬が、疾風のように河原を駆け抜けていった。	
\\	疾, 風	
\\	逸品	
\\	いっぴん	
\\	この黒毛和牛のステーキは、シェフのこだわりの逸品です。	
\\	逸, 品	
\\	庶務	
\\	しょむ	
\\	こんど、庶務課に配属されることになりました。	
\\	庶, 務	
\\	銀杏	
\\	いちょう, ぎんなん	
\\	銀杏を食べ過ぎると中毒症状が出ます。	
\\	ぎんこう 
\\	いちょう. 
\\	銀 (ぎん) 
\\	(なん). 
\\	銀, 杏	
\\	厳粛	
\\	げんしゅく	
\\	な 
\\	結果を厳粛に受け止め、次回に臨みたい。	
\\	厳, 粛	
\\	欄干	
\\	らんかん	
\\	その欄干は古いからもたれないほうがいい。	
\\	欄, 干	
\\	高尚	
\\	こうしょう	
\\	な 
\\	高尚な趣味をお持ちなんですね。	
\\	高, 尚	
\\	別荘	
\\	べっそう	
\\	この春に別荘を売ってしまうつもりです。	
\\	別, 荘	
\\	顕彰	
\\	けんしょう	
\\	する 
\\	先日、献血50回を労る顕彰をいただきました。	
\\	顕, 彰	
\\	茨城県	
\\	いばらきけん	
\\	茨城県に実家があります。	
\\	茨, 
\\	城 
\\	県. 
\\	城 
\\	(き)! 
\\	茨, 城, 県	
\\	愚痴	
\\	ぐち	
\\	愚痴ばかり言っても、何も変わらないよ。	
\\	愚, 痴	
\\	疾病	
\\	しっぺい	
\\	がん、急性心筋梗塞、脳卒中のことを三大疾病といいます。	
\\	病 
\\	(ぺい) 
\\	疾, 病	
\\	酷暑	
\\	こくしょ	
\\	今年の夏は、酷暑が続きました。	
\\	暑 
\\	(しょ) 
\\	酷, 暑	
\\	誘拐	
\\	ゆうかい	
\\	する 
\\	社長の息子を誘拐するつもりだ。	
\\	誘, 拐	
\\	悠長	
\\	ゆうちょう	な 
\\	そんな悠長なことを言ってる時間はないよ。	
\\	悠, 長	
\\	叙情	
\\	じょじょう	
\\	の 
\\	うまく叙情できるよう勉強しています。	
\\	叙, 情	
\\	欄外	
\\	らんがい	
\\	の 
\\	サインが欄外にはみ出てしまったんですが、大丈夫でしょうか。	
\\	欄, 外	
\\	庶民	
\\	しょみん	
\\	の 
\\	私みたいな庶民には関係のない話です。	
\\	庶, 民	
\\	殊勲	
\\	しゅくん	
\\	攻守ともに殊勲を立てました。	
\\	殊, 勲	
\\	新陳代謝	
\\	しんちんたいしゃ	
\\	する 
\\	新陳代謝がアップするエクササイズをご紹介します。	
\\	新, 陳, 代, 謝	
\\	痴漢	
\\	ちかん	
\\	今朝、痴漢を逮捕しました。	
\\	漢 
\\	痴, 漢	
\\	最優遇	
\\	さいゆうぐう	
\\	最優遇貸出金利はいまどのぐらいですか。	
\\	最, 優, 遇	
\\	山荘	
\\	さんそう	
\\	ある閉ざされた山荘で悲劇は起きた。	
\\	山, 荘	
\\	疎通	
\\	そつう	
\\	する 
\\	彼とは、意思疎通が全然できないんです。	
\\	疎, 通	
\\	童謡	
\\	どうよう	
\\	幼稚園で新しい童謡を習ってきました。	
\\	童, 謡	
\\	哺育	
\\	ほいく	
\\	する 
\\	の 
\\	乳児哺育の方法は大きく変化しています。	
\\	哺, 育	
\\	孤児院	
\\	こじいん	
\\	私は孤児院で幼少期を過ごしました。	
\\	孤児 
\\	孤, 児, 院	
\\	実践	
\\	じっせん	
\\	する 
\\	の 
\\	理論は実践に移してこそ意味があるのです。	
\\	実, 践	
\\	時期尚早	
\\	じきしょうそう	
\\	な 
\\	の 
\\	その話を持ち出すのは時期尚早だろう。	
\\	時期 
\\	時, 期, 尚, 早	
\\	贈呈	
\\	ぞうてい	
\\	する 
\\	の 
\\	主役を演じた友人に花束を贈呈しました。	
\\	贈, 呈	
\\	進呈	
\\	しんてい	
\\	する 
\\	の 
\\	訪問者の多いブログには、
\\	ポイントが進呈されます。	
\\	進, 呈	
\\	傲慢	
\\	ごうまん	
\\	な 
\\	あんな傲慢な男は初めてだよ。	
\\	傲, 慢	
\\	風呂屋	
\\	ふろや	
\\	風呂屋に来るのは初めてです。	
\\	風呂 
\\	風呂 
\\	屋. 
\\	風, 呂, 屋	
\\	悠々	
\\	ゆうゆう	
\\	退職後は田舎で悠々と暮らしたいな。	
\\	悠, 々	
\\	顕著	
\\	けんちょ	
\\	な 
\\	我が社の業績は顕著に回復しています。	
\\	顕, 著	
\\	賠償金	
\\	ばいしょうきん	
\\	賠償金をお支払い頂く可能性があります。	
\\	賠, 償, 金	
\\	酷似	
\\	こくじ	
\\	する 
\\	の 
\\	三冊の本は、どれも内容が酷似している。	
\\	似 
\\	(じ), 
\\	酷, 似	
\\	報酬	
\\	ほうしゅう	
\\	このプロジェクトが成功すれば、報酬として、クリステンがほっぺにキスをしてくれます。	
\\	報, 酬	
\\	波紋	
\\	はもん	
\\	大臣の不用意な発言が波紋を呼んだ。	
\\	波, 紋	
\\	自粛	
\\	じしゅく	
\\	する 
\\	役員の不祥事発覚を受け、本日は全社員が営業活動を自粛致します。	
\\	自, 粛	
\\	酷	
\\	こく	
\\	な 
\\	それをあいつに頼むのは、ちょっと酷すぎやしないか。	
\\	酷	
\\	酷評	
\\	こくひょう	
\\	する 
\\	の 
\\	いちいち酷評を気にしてたら、何も書けないよ。	
\\	酷, 評	
\\	過酷	
\\	かこく	
\\	な 
\\	彼らは、不当に過酷な労働を強いられています。	
\\	過, 酷	
\\	陳列室	
\\	ちんれつしつ	
\\	その陳列室には、いったい何が並べられているのですか。	
\\	陳, 列, 室	
\\	鎌	
\\	かま	
\\	鎌をすっぽ抜けただけです。	
\\	鎌	
\\	静粛	
\\	せいしゅく	
\\	な 
\\	静粛に願います。	
\\	静, 粛	
\\	荘厳	
\\	そうごん	
\\	な 
\\	その儀式は荘厳に行われました。	
\\	厳 
\\	(ごん) 
\\	荘, 厳	
\\	歌謡	
\\	かよう	
\\	今日のカラオケは、昭和歌謡縛りにしよう。	
\\	歌, 謡	
\\	疎ら	
\\	まばら	
\\	な 
\\	昼間は賑やかですが、夜になると人通りも疎らになります。	
\\	(まば) 
\\	疎	
\\	疾走	
\\	しっそう	
\\	する 
\\	少年は、駅の構内を全力疾走していた。	
\\	疾, 走	
\\	茎	
\\	くき	
\\	薔薇の茎を食べることはできますか?	
\\	茎	
\\	噴き出す	
\\	ふきだす	
\\	今までためていた怒りが一気に噴き出した。	
\\	吹き出す? 
\\	噴, 出	
\\	膨らむ	
\\	ふくらむ	
\\	希望は膨らむばかりだった。	
\\	う 
\\	(ふく) 
\\	膨	
\\	貫き通す	
\\	つらぬきとおす	
\\	彼は最後まで自分の信念を貫き通しました。	
\\	貫く 
\\	通す 
\\	貫, 通	
\\	呆ける	
\\	ぼける, ほうける	
\\	あの芸人は呆けるのが本当にうまい。	
\\	う 
\\	(ぼ) 
\\	呆	
\\	音痴	
\\	おんち	
\\	な 
\\	の 
\\	音痴なので、カラオケは苦手です。	
\\	音, 痴	
\\	怠ける	
\\	なまける	
\\	努力する人は希望を語るが、怠ける人は不満を語るのさ。	
\\	う 
\\	(なま). 
\\	怠	
\\	卸す	
\\	おろす	
\\	うちにもその商品を卸してもらえませんか。	
\\	う 
\\	卸	
\\	逸らす	
\\	そらす	
\\	彼女は颯と目を逸らした。	
\\	う 
\\	(そ). 
\\	逸	
\\	冠	
\\	かん, かむり, かんむり	
\\	この冠を被れるのは王だけです。	
\\	(かん) 
\\	(かむり), 
\\	(かんむり) 
\\	冠	
\\	空疎	
\\	くうそ	
\\	な 
\\	こんな空疎な論争を続けるのは時間の無駄だ。	
\\	空, 疎	
\\	犠牲者	
\\	ぎせいしゃ	
\\	その飛行機事故では、事故で多くの犠牲者がでました。	
\\	犠牲 
\\	犠, 牲, 者	
\\	鯨	
\\	くじら	
\\	鯨のベーコンを食べました。	
\\	(くじら). 
\\	鯨	
\\	鯨肉	
\\	げいにく	
\\	美味しい鯨肉の店を知っているよ。	
\\	鯨, 肉	
\\	捕鯨	
\\	ほげい	
\\	私は捕鯨には反対です。	
\\	捕, 鯨	
\\	卸	
\\	おろし	
\\	うちは雑貨の卸をしています。	
\\	卸	
\\	卸売	
\\	おろしうり	
\\	の 
\\	卸売市場の見学をしてきました。	
\\	卸売り 
\\	卸, 売	
\\	卸値	
\\	おろしね	
\\	それは卸値で購入することができたんです。	
\\	卸, 値	
\\	累積	
\\	るいせき	
\\	する 
\\	の 
\\	日本政府が巨額の国債を累積していることについて、あなたはどのようにお考えですか。	
\\	累, 積	
\\	癒着	
\\	ゆちゃく	
\\	する 
\\	の 
\\	手術後の癒着リスクについて、きちんと説明はなされましたか。	
\\	癒, 着	
\\	一抹	
\\	いちまつ	
\\	一抹の不安が胸をよぎった。	
\\	一, 抹	
\\	伏兵	
\\	ふくへい	
\\	伏兵に襲われたんだ。	
\\	伏, 兵	
\\	憂慮	
\\	ゆうりょ	
\\	する 
\\	ズバリ、社長が現在抱える憂慮とは何ですか。	
\\	憂, 慮	
\\	奉仕	
\\	ほうし	
\\	する 
\\	今年の夏休みは、老人ホームで奉仕活動をしました。	
\\	奉, 仕	
\\	信奉	
\\	しんぽう	
\\	する 
\\	私は日本国民全員が神道を信奉すべきだと思います。	
\\	信, 奉	
\\	骨髄	
\\	こつずい	
\\	の 
\\	骨髄バンクに登録しました。	
\\	骨, 髄	
\\	栓抜き	
\\	せんぬき	
\\	栓抜きはそこの引き出しに入ってます。	
\\	栓 
\\	抜, 
\\	抜く. 
\\	栓, 抜	
\\	虜	
\\	とりこ	
\\	彼女は、毎日ブログで男を虜にするテクニックについて書いています。	
\\	鳥子 (とりこ), 
\\	虜	
\\	循環	
\\	じゅんかん	
\\	する 
\\	の 
\\	血液の循環には二種類の経路があります。	
\\	循, 環	
\\	粗悪	
\\	そあく	
\\	な 
\\	の 
\\	粗悪なガソリンは、車の寿命を早める。	
\\	粗, 悪	
\\	粗い	
\\	あらい	い 
\\	私はいつも、トウモロコシ、麦、豆などをひき割って粗い粉にしたものをスムージーに入れます。	
\\	い 
\\	(あら). 
\\	粗	
\\	潜伏	
\\	せんぷく	
\\	する 
\\	我々はマフィアの潜伏捜査をすることになった。	
\\	潜, 伏	
\\	哀悼	
\\	あいとう	
\\	する 
\\	の 
\\	哀悼の意を表して、黙祷しました。	
\\	哀, 悼	
\\	栓	
\\	せん	
\\	栓抜きが無くても瓶ビールの栓を抜く裏ワザを知っています。	
\\	栓	
\\	該当	
\\	がいとう	
\\	する 
\\	この商品も、キャンペーンの対象に該当しますか?	
\\	該, 当	
\\	累計	
\\	るいけい	
\\	する 
\\	このサイトでは、人気アプリの累計ダウンロードランキングを毎日更新しています。	
\\	累, 計	
\\	奉納	
\\	ほうのう	
\\	する 
\\	の 
\\	どうして首相が靖国神社に玉串料を奉納することが批判されるのですか。	
\\	奉, 納	
\\	憂鬱	
\\	ゆううつ	
\\	な 
\\	会社へ出勤するのが憂鬱です。	
\\	憂, 鬱	
\\	洗浄剤	
\\	せんじょうざい	
\\	洗浄剤なら、洗い場の下の棚にあります。	
\\	洗浄 
\\	洗, 浄, 剤	
\\	抹消	
\\	まっしょう	
\\	する 
\\	彼は怪我で一軍登録を抹消されたんだ。	
\\	抹, 消	
\\	神髄	
\\	しんずい	
\\	の 
\\	プロの神髄を見せつけられました。	
\\	神, 髄	
\\	真髄	
\\	しんずい	
\\	の 
\\	京料理の真髄はハモ料理だと思います。	
\\	真, 髄	
\\	佳作	
\\	かさく	
\\	私の描いた絵が佳作で入選しました。	
\\	佳, 作	
\\	伏線	
\\	ふくせん	
\\	伏線の張り方がすごい。	
\\	伏, 線	
\\	来賓	
\\	らいひん	
\\	来賓の方々をはじめ、皆様にはお忙しい中、ご臨席を賜り誠にありがとうございます。	
\\	来, 賓	
\\	賓客	
\\	ひんきゃく, ひんかく	
\\	海外の賓客を一流の和食でもてなしました。	
\\	賓, 客	
\\	治癒	
\\	ちゆ	
\\	する 
\\	虫歯が自然治癒することはありません。	
\\	治, 癒	
\\	重鎮	
\\	じゅうちん	
\\	彼の会社は、業界の重鎮達を次々と役員に迎え入れた。	
\\	重, 鎮	
\\	傍観	
\\	ぼうかん	
\\	する 
\\	私はイジメをただ傍観していました。	
\\	傍, 観	
\\	惜敗	
\\	せきはい	
\\	する 
\\	わが校は、惜敗でベスト8入りを逃しました。	
\\	惜, 敗	
\\	旦那	
\\	だんな	
\\	旦那は出張で今週ずっと留守にしています。	
\\	旦, 那	
\\	抹茶	
\\	まっちゃ	
\\	毎朝抹茶ラテを飲んでいます。	
\\	抹, 茶	
\\	傍受	
\\	ぼうじゅ	
\\	する 
\\	警察は、犯罪捜査のために、通信の傍受を行うことがあります。	
\\	傍, 受	
\\	追悼	
\\	ついとう	
\\	する 
\\	の 
\\	みんなで奴の追悼コンサートを開かないか?	
\\	追 
\\	(つい) 
\\	追, 悼	
\\	赴任	
\\	ふにん	
\\	する 
\\	今度、中国に赴任することになりました。	
\\	赴, 任	
\\	抹殺	
\\	まっさつ	
\\	する 
\\	あの男は、組織に歯向かったため、社会から抹殺されたんだよ。	
\\	抹, 殺	
\\	傍	
\\	そば, はた	
\\	いつまでも君の傍にいるよ。	
\\	(そば) 
\\	傍	
\\	元旦	
\\	がんたん	
\\	元旦には家族で初詣に行きます。	
\\	元 
\\	(がん). 
\\	元, 旦	
\\	貴賓	
\\	きひん	
\\	社長が外国の貴賓を接待する際に、通訳として同行しました。	
\\	貴, 賓	
\\	老舗	
\\	しにせ, ろうほ	
\\	の 
\\	彼らは、まさに老舗の意地を見せたんだ。	
\\	(しにせ). 
\\	老, 舗	
\\	一旦	
\\	いったん	
\\	この話は一旦なかったことにしてもらえませんか。	
\\	一, 旦	
\\	輪郭	
\\	りんかく	
\\	涙で輪郭が霞んで見えました。	
\\	輪, 郭	
\\	瓜実顔	
\\	うりざねがお	
\\	瓜実顔で雰囲気のある女優を探しています。	
\\	瓜 
\\	顔, 
\\	実 
\\	(ざね) 
\\	瓜, 実, 顔	
\\	鎮痛剤	
\\	ちんつうざい	
\\	医者に鎮痛剤を注射してもらいました。	
\\	鎮, 痛, 剤	
\\	憂国	
\\	ゆうこく	
\\	憂国の若きサムライたちよ、今こそ立ち上がれ!	
\\	憂, 国	
\\	愉快	
\\	ゆかい	
\\	な 
\\	昔、「僕は殺しが愉快でたまらない」と言った少年殺人犯がいました。	
\\	愉, 快	
\\	捕虜	
\\	ほりょ	
\\	の 
\\	ごぼうを捕虜に食べさせて有罪になった日本兵がいます。	
\\	捕, 虜	
\\	繁昌	
\\	はんじょう	
\\	する 
\\	商売が繁昌するように、神社でお祈りをしてきました。	
\\	繁, 昌	
\\	南瓜	
\\	かぼちゃ	
\\	南瓜の煮物を炊きました。	
\\	(かぼちゃ). 
\\	南, 瓜	
\\	遺憾	
\\	いかん	
\\	な 
\\	裁判は誠に遺憾な結果に終わりました。	
\\	遺, 憾	
\\	鎮魂	
\\	ちんこん	
\\	する 
\\	星祭りは、平和を祈り、戦死者を鎮魂する儀式です。	
\\	鎮, 魂	
\\	粗野	
\\	そや	
\\	な 
\\	彼女は、決して他人に対して粗野に振る舞うことがなかった。	
\\	粗, 野	
\\	奉公	
\\	ほうこう	
\\	する 
\\	お礼奉公とは看護婦が、看護学生時代に借りた奨学金の返済制度の一つです。	
\\	奉, 公	
\\	素朴	
\\	そぼく	
\\	な 
\\	みなさんの素朴な疑問にお答えします。	
\\	素, 朴	
\\	凝固	
\\	ぎょうこ	
\\	する 
\\	煮詰めたジャムは、冷めてくると粘性が強まり、ゼリー状に凝固します。	
\\	凝, 固	
\\	脊髄	
\\	せきずい	
\\	の 
\\	事故で脊髄に傷がついてしまったのです。	
\\	脊, 髄	
\\	凝視	
\\	ぎょうし	
\\	する 
\\	タイプの女性がいたので、思わず凝視してしまいました。	
\\	凝, 視	
\\	国賓	
\\	こくひん	
\\	アメリカ大統領が国賓として来日しました。	
\\	国, 賓	
\\	累進	
\\	るいしん	
\\	する 
\\	彼は若くして部長に累進した。	
\\	累, 進	
\\	鎮圧	
\\	ちんあつ	
\\	する 
\\	の 
\\	警察が三日がかりでデモ隊を鎮圧しました。	
\\	鎮, 圧	
\\	降伏	
\\	こうふく	
\\	する 
\\	絶対にあいつらを降伏させてみせる。	
\\	降, 伏	
\\	之	
\\	これ	
\\	よかったら之も持って行きなさい。	
\\	之	
\\	鎮める	
\\	しずめる	
\\	神の怒りを鎮めるには、生贄を捧げる必要があります。	
\\	う 
\\	(しず) 
\\	鎮	
\\	凝る	
\\	こる	
\\	妻の料理の盛り付けは、いつも凝っています。	
\\	う 
\\	子
\\	(こ). 
\\	凝	
\\	憂える	
\\	うれえる	
\\	親が子供の将来を憂えるのは当然のことだ。	
\\	う 
\\	(うれ)! 
\\	憂	
\\	惜しむ	
\\	おしむ	
\\	お前がちょっとした手間を惜しんだために、全ての計画がおじゃんになったんだぞ!	
\\	う 
\\	(おし). 
\\	惜	
\\	栃木県	
\\	とちぎけん	
\\	栃木県の出身なんですか?	
\\	栃 
\\	木 
\\	県, 
\\	き 
\\	ぎ 
\\	栃, 木, 県	
\\	伏せる	
\\	ふせる	
\\	どうして弟が犯罪者であることを伏せていたんだ。	
\\	う 
\\	(ふ) 
\\	伏	
\\	据え付ける	
\\	すえつける	
\\	ここに戸棚を据え付けるつもりだよ。	
\\	据える, 
\\	付ける, 
\\	据, 付	
\\	起伏	
\\	きふく	
\\	する 
\\	生理前はどうしても感情の起伏が激しくなってしまう。	
\\	起, 伏	
\\	尿	
\\	にょう	
\\	病院で尿検査をして、妊娠していると言われました。	
\\	尿	
\\	山葵	
\\	わさび	
\\	やはり、すりたての山葵は美味しいね。	
\\	山, 葵	
\\	披露宴	
\\	ひろうえん	
\\	友人の披露宴に出席する際のマナーを教えてください。	
\\	披露 
\\	披, 露, 宴	
\\	弥生	
\\	やよい	
\\	弥生時代のはじまりは紀元前300年頃です。	
\\	(やよい) 
\\	弥, 生	
\\	尚更	
\\	なおさら	
\\	あなたのことが尚更嫌いになりました。	
\\	尚, 更	
\\	尚且つ	
\\	なおかつ	
\\	彼女は、美しく、尚且つ頭も良いが、性格は悪い。	
\\	尚, 且	
\\	拍手	
\\	はくしゅ	
\\	する 
\\	舞台が終わった後、拍手が鳴り止まなかった。	
\\	拍, 手	
\\	匠	
\\	たくみ	
\\	この商品のデザインは、匠の技をもって初めて実現することができました。	
\\	(たくみ)!
\\	匠	
\\	嘉日	
\\	かじつ	
\\	本日は、皇太子殿下御生誕の嘉日なり。	
\\	嘉, 日	
\\	〜拍	
\\	はく	
\\	4分の4拍子の曲では、通常、1拍目と3拍目の頭に強拍が来ます。	
\\	拍	
\\	呉越同舟	
\\	ごえつどうしゅう	
\\	利害が一致すれば、呉越同舟も厭わない。	
\\	舟 
\\	(しゅう) 
\\	呉, 越, 同, 舟	
\\	弦	
\\	つる, げん	
\\	ギターの弦が緩んでいます。	
\\	(つる)!
\\	弦	
\\	蝶	
\\	ちょう	
\\	父は、毎日色の違う蝶ネクタイをしていた。	
\\	蝶	
\\	大尉	
\\	たいい	
\\	大尉の死を決して無駄にはしません。	
\\	大, 尉	
\\	摂氏	
\\	せっし	
\\	の 
\\	今日の最高気温は、摂氏二十五度です。	
\\	摂, 氏	
\\	平凡	
\\	へいぼん	
\\	な 
\\	平凡な毎日に退屈していたんです。	
\\	平, 凡	
\\	錯乱	
\\	さくらん	
\\	する 
\\	薬で意識が錯乱しているようでした。	
\\	錯, 乱	
\\	満悦	
\\	まんえつ	
\\	する 
\\	ご満悦のようですね。	
\\	満, 悦	
\\	意匠	
\\	いしょう	
\\	これは、意匠が凝らされた素晴らしい作品です。	
\\	意, 匠	
\\	窮地	
\\	きゅうち	
\\	窮地に追い込まれ、ようやく尻尾を見せました。	
\\	窮, 地	
\\	直轄	
\\	ちょっかつ	
\\	する 
\\	の 
\\	それじゃあ、ここも政府が直轄している機関なんですか。	
\\	直, 轄	
\\	窮状	
\\	きゅうじょう	
\\	役所に窮状を訴えたのですが、聞き入れてもらえませんでした。	
\\	窮, 状	
\\	摂理	
\\	せつり	
\\	する 
\\	の 
\\	誰も自然の摂理には逆らえないよ。	
\\	摂, 理	
\\	呉服	
\\	ごふく	
\\	彼は呉服屋の一人息子でした。	
\\	呉, 服	
\\	弊害	
\\	へいがい	
\\	子供にゲームをさせないことによる弊害も考えておくべきです。	
\\	弊, 害	
\\	疲弊	
\\	ひへい	
\\	する 
\\	人口が増えない限り、地方経済の疲弊が回復することはないでしょう。	
\\	疲, 弊	
\\	搾取	
\\	さくしゅ	
\\	する 
\\	このエージェントは、ギャラをかなり搾取することで有名だ。	
\\	搾, 取	
\\	猶予	
\\	ゆうよ	
\\	する 
\\	お前に三日間の猶予を与えてやろう。	
\\	猶, 予	
\\	遥か	
\\	はるか	
\\	な 
\\	その鐘の音は、遥か彼方まで響き渡った。	
\\	か 
\\	遥	
\\	柴	
\\	しば	
\\	おじいさんは、山で柴刈りをしていました。	
\\	柴	
\\	中尉	
\\	ちゅうい	
\\	我が軍の中尉が敵軍に拉致され、処刑されました。	
\\	中, 尉	
\\	洪水	
\\	こうずい	
\\	台風で川が氾濫し、洪水になりました。	
\\	すい 
\\	ずい 
\\	洪, 水	
\\	愚か	
\\	おろか	
\\	な 
\\	あいつも愚かなことをしたもんだよ。	
\\	(おろ), 
\\	愚	
\\	無報酬	
\\	むほうしゅう	
\\	無報酬でもかまいませんよ。	
\\	報酬 
\\	無, 報, 酬	
\\	凛々しい	
\\	りりしい	い 
\\	少年は、凛々しい青年に成長した。	
\\	々 
\\	りんりん 
\\	ん
\\	りり!	凛, 々	
\\	紳士	
\\	しんし	
\\	彼は、英国紳士の鏡のような人でした。	
\\	紳, 士	
\\	飽食	
\\	ほうしょく	
\\	する 
\\	日本は、戦後の食料不足から飽食の時代へ大きく変わりました。	
\\	飽, 食	
\\	穀物	
\\	こくもつ	
\\	の 
\\	このスープには五種類の穀物が入っています。	
\\	穀, 物	
\\	穀類	
\\	こくるい	
\\	の 
\\	私は穀類を毎食摂るようにしています。	
\\	穀, 類	
\\	墓碑	
\\	ぼひ	
\\	誰かが墓碑を倒したんです。	
\\	墓, 碑	
\\	丘陵	
\\	きゅうりょう	
\\	今日は天気が良かったので、矢田丘陵を歩きました。	
\\	丘, 陵	
\\	錯覚	
\\	さっかく	
\\	する 
\\	の 
\\	それがまるで自分のもののような錯覚に襲われたんです。	
\\	錯, 覚	
\\	舶来	
\\	はくらい	
\\	の 
\\	舶来のブランデーだと叔父はよく自慢していた。	
\\	舶, 来	
\\	碑文	
\\	ひぶん	
\\	の 
\\	この慰霊碑の碑文は誰が考えたのですか。	
\\	碑, 文	
\\	摂取	
\\	せっしゅ	
\\	する 
\\	一日の摂取カロリーを計算してみました。	
\\	摂, 取	
\\	管轄	
\\	かんかつ	
\\	する 
\\	ここは、南署の管轄です。	
\\	管, 轄	
\\	所轄	
\\	しょかつ	
\\	する 
\\	の 
\\	とにかく、まずは所轄の警察署に連絡をいれましょう。	
\\	所, 轄	
\\	怠け者	
\\	なまけもの	
\\	彼は会社で一番の怠け者だが、一番成績を上げている。	
\\	怠, 者	
\\	寂しい	
\\	さびしい, さみしい	い 
\\	やっぱり一人暮らしは寂しいです。	
\\	い 
\\	(さび). 
\\	寂	
\\	巨匠	
\\	きょしょう	
\\	この女優は、多くの巨匠たちを唸らせた。	
\\	巨, 匠	
\\	師匠	
\\	ししょう	
\\	師匠と呼ばせてもらえませんか。	
\\	師, 匠	
\\	凡庸	
\\	ぼんよう	
\\	な 
\\	の 
\\	彼は凡庸な男ですが、愛に満ちあふれていました。	
\\	凡, 庸	
\\	搾乳	
\\	さくにゅう	
\\	する 
\\	牧場で搾乳体験をしてきました。	
\\	搾, 乳	
\\	錯誤	
\\	さくご	
\\	ちょっとそれは時代錯誤じゃないか。	
\\	錯, 誤	
\\	交錯	
\\	こうさく	
\\	する 
\\	色々な感情が交錯し、結論を出せずにいました。	
\\	交, 錯	
\\	記念碑	
\\	きねんひ	
\\	これは何の記念碑ですか。	
\\	記, 念, 碑	
\\	疎か	
\\	おろそか	
\\	な 
\\	基本を疎かにしてはいけません。	
\\	(おろそか). 
\\	疎	
\\	窒素	
\\	ちっそ	
\\	の 
\\	飛行機のタイヤには窒素が入っています。	
\\	窒, 素	
\\	方向音痴	
\\	ほうこうおんち	
\\	私は方向音痴で、すぐ道に迷います。	
\\	方向 
\\	音痴. 
\\	方, 向, 音, 痴	
\\	凡人	
\\	ぼんじん	
\\	あんたみたいな凡人が書いた本、誰が読むのさ。	
\\	凡, 人	
\\	倒錯	
\\	とうさく	
\\	する 
\\	の 
\\	倒錯思考はまず現実否定から始まる。	
\\	倒, 錯	
\\	庶民的	
\\	しょみんてき	な 
\\	ここの饂飩はこの庶民的な味がいいんだよ。	
\\	庶民 
\\	庶, 民, 的	
\\	歌謡曲	
\\	かようきょく	
\\	心に残る昭和の歌謡曲ベスト100のリストが公表されました。	
\\	歌謡 
\\	歌, 謡, 曲	
\\	窮乏	
\\	きゅうぼう	
\\	する 
\\	生活が窮乏して、仕方なく盗みを働いたんです。	
\\	窮, 乏	
\\	困窮	
\\	こんきゅう	
\\	する 
\\	両親は生活に困窮していましたが、私にも彼らを助ける余裕はありませんでした。	
\\	困, 窮	
\\	宰相	
\\	さいしょう	
\\	英国宰相から直々に手紙を頂きました。	
\\	相 
\\	(しょう). 
\\	宰, 相	
\\	非凡	
\\	ひぼん	
\\	な 
\\	みんな、彼の非凡な才能を妬んでいたんだと思います。	
\\	非, 凡	
\\	脈拍	
\\	みゃくはく	
\\	の 
\\	すでに脈拍が止まっている。	
\\	脈, 拍	
\\	恭しい	
\\	うやうやしい	い 
\\	二人は神前で恭しく頭を下げた。	
\\	い 
\\	恭	
\\	束縛	
\\	そくばく	
\\	する 
\\	彼氏の束縛が厳しいんだよね。	
\\	束, 縛	
\\	米穀	
\\	べいこく	
\\	我が社は、米国で米穀を売っています。	
\\	米, 穀	
\\	飽和	
\\	ほうわ	
\\	する 
\\	の 
\\	市場は今飽和状態になっているんだよ。	
\\	飽, 和	
\\	少尉	
\\	しょうい	
\\	私の曽祖父は、旧日本陸軍少尉でした。	
\\	少, 尉	
\\	裁縫	
\\	さいほう	
\\	する 
\\	裁縫はあまり得意ではありません。	
\\	裁, 縫	
\\	縫製	
\\	ほうせい	
\\	する 
\\	このミシンで縫製加工していきます。	
\\	縫, 製	
\\	柴犬	
\\	しばいぬ, しばけん	
\\	ニッカは柴犬ではなく珍島犬です。	
\\	柴, 犬	
\\	船舶	
\\	せんぱく	
\\	の 
\\	その船舶は、どこの国に登録されているものですか。	
\\	船, 舶	
\\	静寂	
\\	せいじゃく	
\\	な 
\\	赤ん坊の泣き声が静寂を破った。	
\\	静, 寂	
\\	怠い	
\\	だるい	い 
\\	インフルエンザによる高熱で、体が怠いんです。	
\\	い 
\\	(だる) 
\\	怠	
\\	石碑	
\\	せきひ	
\\	津波が襲った高さの地点に石碑が建てられているのです。	
\\	石, 碑	
\\	弦楽	
\\	げんがく	
\\	学生時代に数年ほど弦楽をやっていました。	
\\	楽 
\\	音楽 
\\	楽 
\\	弦, 楽	
\\	窒息	
\\	ちっそく	
\\	する 
\\	の 
\\	死因は首を絞められたことによる窒息です。	
\\	窒, 息	
\\	主宰	
\\	しゅさい	
\\	する 
\\	いつか自分の劇団を主宰するのが夢です。	
\\	主, 宰	
\\	窮屈	
\\	きゅうくつ	
\\	な 
\\	窮屈な下着は体によくない。	
\\	窮, 屈	
\\	呆れる	
\\	あきれる	
\\	アイドルの度重なるプッツン発言に、世間はみんな呆れました。	
\\	う 
\\	(あき)!
\\	呆	
\\	飽きる	
\\	あきる	
\\	新しい恋人に飽きてきたところなの。	
\\	う 
\\	(あ).
\\	飽	
\\	縫う	
\\	ぬう	
\\	家庭科の授業で、雑巾を縫いました。	
\\	う 
\\	(ぬ). 
\\	縫	
\\	搾る	
\\	しぼる	
\\	レモンを唐揚げに搾ってもいいですか?	
\\	う 
\\	(しぼ) 
\\	搾	
\\	逸れる	
\\	それる, はぐれる	
\\	すぐ話が脇道へ逸れてしまう。	
\\	逸らす 
\\	(それ) 
\\	逸	
\\	縛る	
\\	しばる	
\\	靴紐が解けないようにしっかり縛った。	
\\	う 
\\	(しば) 
\\	縛	
\\	智	
\\	ち	
\\	そんな子供の姿を見て、智を開く思いがしました。	
\\	智	
\\	靖国神社	
\\	やすくにじんじゃ	
\\	靖国神社に参拝したことはありますか。	
\\	神社 
\\	神社. 
\\	靖, 国, 神, 社	
\\	甲乙	
\\	こうおつ	
\\	どちらも甲乙つけがたい出来栄えだ。	
\\	甲, 乙	
\\	堕胎	
\\	だたい	
\\	する 
\\	彼女が堕胎を拒んでいるみたいよ。	
\\	堕, 胎	
\\	約款	
\\	やっかん	
\\	きちんと約款を読みましたか?	
\\	約, 款	
\\	年俸	
\\	ねんぽう	
\\	この選手は、破格の年俸をオファーされましたが、それを蹴ってチームに留まりました。	
\\	ほう 
\\	ぽう 
\\	年, 俸	
\\	盲腸	
\\	もうちょう	
\\	の 
\\	手術で盲腸を取ってしまいました。	
\\	盲, 腸	
\\	定款	
\\	ていかん	
\\	この雛形を作れば簡単に定款が作成できます。	
\\	定, 款	
\\	平衡	
\\	へいこう	
\\	する 
\\	彼女は突然体の平衡を失い、ひっくり返った。	
\\	平, 衡	
\\	山賊	
\\	さんぞく	
\\	の 
\\	山賊たちと友だちになりました。	
\\	山, 賊	
\\	浴槽	
\\	よくそう	
\\	毎日掃除をしないと、浴槽には垢がたまります。	
\\	浴, 槽	
\\	鼓動	
\\	こどう	
\\	する 
\\	少女の顔が赤くなり、鼓動が高鳴りました。	
\\	鼓, 動	
\\	恐喝	
\\	きょうかつ	
\\	する 
\\	不倫をダシにヤクザに恐喝されています。	
\\	恐, 喝	
\\	盲目	
\\	もうもく	
\\	な 
\\	の 
\\	私達は、盲目の犬を飼っているんです。	
\\	盲, 目	
\\	盲人	
\\	もうじん	
\\	その家には、生まれつき盲人の男が住んでいます。	
\\	盲, 人	
\\	盲点	
\\	もうてん	
\\	犯人は捜査の盲点を突いてきた。	
\\	盲, 点	
\\	敢然	
\\	かんぜん	
\\	あの男は、まるで自分の意志で敢然と難局に立ち向かっているつもりなのよ。	
\\	敢, 然	
\\	勇敢	
\\	ゆうかん	な 
\\	お前の父さんは、とても勇敢な戦士だったよ。	
\\	勇, 敢	
\\	醸成	
\\	じょうせい	
\\	する 
\\	ここで酒を醸成しているんですね。	
\\	醸, 成	
\\	醸造	
\\	じょうぞう	
\\	する 
\\	スコッチの醸造所を見学してきました。	
\\	醸, 造	
\\	閲覧	
\\	えつらん	
\\	する 
\\	このウェブサイトには閲覧規制がかかっています。	
\\	閲, 覧	
\\	検閲	
\\	けんえつ	
\\	する 
\\	の 
\\	この国ではネット検閲が行われています。	
\\	検, 閲	
\\	海峡	
\\	かいきょう	
\\	このフェリーで海峡を渡ります。	
\\	海, 峡	
\\	循環器	
\\	じゅんかんき	
\\	当院には、循環器専門医が五人います。	
\\	循環 
\\	循, 環, 器	
\\	熊之実	
\\	くまのみ	
\\	ニモやその家族は、隠熊之実という魚だよ。	
\\	之 
\\	実 
\\	(のみ)! 
\\	熊, 之, 実	
\\	色盲	
\\	しきもう	
\\	の 
\\	色盲って治るんですか。	
\\	色, 盲	
\\	羅針盤	
\\	らしんばん	
\\	彼は私にとって、人生の羅針盤だったのです。	
\\	羅, 針, 盤	
\\	受胎	
\\	じゅたい	
\\	する 
\\	エレインという馬が、ステイゴールドの最後の子を受胎しました。	
\\	受, 胎	
\\	弔意	
\\	ちょうい	
\\	する 
\\	の 
\\	弔意を表して花を贈りました。	
\\	弔, 意	
\\	角膜	
\\	かくまく	
\\	の 
\\	レーシック手術では、角膜をレーザーで削るんですよ。	
\\	角, 膜	
\\	堕落	
\\	だらく	
\\	する 
\\	兄は酒と女に溺れ、敢然に堕落した生活を送っています。	
\\	堕, 落	
\\	網羅	
\\	もうら	
\\	する 
\\	あの男はたった一ヶ月で、ここでの業務を網羅した。	
\\	網, 羅	
\\	烏	
\\	からす	
\\	烏がゴミをあさるんですよ。	
\\	烏	
\\	敢えて	
\\	あえて	
\\	あなただからこそ、敢えて厳しいことを言わせてもらったんだよ。	
\\	(あ)!
\\	敢	
\\	濃紺	
\\	のうこん	
\\	黒と濃紺のスーツが一着ずつあると便利ですよ。	
\\	濃, 紺	
\\	敢行	
\\	かんこう	
\\	する 
\\	本当にこの作戦を敢行するつもりですか。	
\\	敢, 行	
\\	太鼓	
\\	たいこ	
\\	の 
\\	お腹がパンパンで太鼓みたいになっています。	
\\	太, 鼓	
\\	鼓膜	
\\	こまく	
\\	の 
\\	そんなに大声を出されたら鼓膜が破れるよ。	
\\	鼓, 膜	
\\	胎児	
\\	たいじ	
\\	の 
\\	エコーで胎児が足を伸ばしたり縮めたりするところが見れました。	
\\	胎, 児	
\\	羅列	
\\	られつ	
\\	する 
\\	一見すると意味不明な文字の羅列のように見える。	
\\	羅, 列	
\\	酵素	
\\	こうそ	
\\	の 
\\	今若い女の子の間では、酵素ダイエットが流行っています。	
\\	酵, 素	
\\	均衡	
\\	きんこう	
\\	する 
\\	不倫でバランスをとって、なんとか心の均衡を保っているんです。	
\\	均, 衡	
\\	発酵	
\\	はっこう	
\\	する 
\\	の 
\\	ワインの甘みを残して発酵を止める方法を教えてください。	
\\	発, 酵	
\\	敢闘	
\\	かんとう	
\\	する 
\\	優勝候補を相手に、よく敢闘したと思います。	
\\	敢, 闘	
\\	水槽	
\\	すいそう	
\\	どのくらい頻繁に水槽の掃除をしますか。	
\\	水, 槽	
\\	萌芽	
\\	ほうが	
\\	する 
\\	あなたの役目は、二人の恋の萌芽を絶つことよ。	
\\	芽 
\\	(が).	萌, 芽	
\\	憂い	
\\	うれい	
\\	彼女は、憂いを帯びた顔でこちらを見ていた。	
\\	い 
\\	憂える. 
\\	憂	
\\	腸	
\\	ちょう, わた, はらわた	
\\	腸の働きをよくする運動をご紹介します。	
\\	腸	
\\	鼓	
\\	つづみ	
\\	あの女性の鼓の打ち方は絶品だ。	
\\	(つづみ). 
\\	鼓	
\\	何遍	
\\	なんべん	
\\	この映画は何遍みても泣けます。	
\\	へん 
\\	べん 
\\	何, 遍	
\\	鼓舞	
\\	こぶ	
\\	する 
\\	あの監督は、選手たちの士気を鼓舞するのがうまい。	
\\	鼓, 舞	
\\	遮断	
\\	しゃだん	
\\	する 
\\	サイトへのアクセスが遮断されました。	
\\	遮, 断	
\\	大腸	
\\	だいちょう	
\\	の 
\\	大腸がんが見つかりました。	
\\	大, 腸	
\\	凸版	
\\	とっぱん	
\\	の 
\\	我が社では、特殊な技術を用いて、樹脂凸版の直接製版を格安で行っています。	
\\	はん 
\\	ぱん 
\\	凸, 版	
\\	借款	
\\	しゃっかん	
\\	日本が多くの国に円借款を提供しているのはご存知ですか。	
\\	借, 款	
\\	伐採	
\\	ばっさい	
\\	する 
\\	森林を伐採する際は、事前に届け出が必要です。	
\\	伐, 採	
\\	奉行	
\\	ぶぎょう	
\\	する 
\\	彼氏がいつも鍋奉行をしてくれます。	
\\	(ぶぎょう). 
\\	奉, 行	
\\	楓	
\\	かえで, ふう	
\\	楓の葉も色づいてきました。	
\\	楓	
\\	楓糖	
\\	ふうとう, かえでとう	
\\	料理には砂糖の代わりに楓糖を使うことが多いです。	
\\	ふうとう. 
\\	風 
\\	(ふう).	楓, 糖	
\\	旋律	
\\	せんりつ	
\\	の 
\\	あの家からは、いつも美しいピアノの旋律が聞こえていました。	
\\	旋, 律	
\\	膜	
\\	まく	
\\	牛乳を温めると薄い膜が張ることがあります。	
\\	膜	
\\	弔辞	
\\	ちょうじ	
\\	誰が弔辞を読むのですか。	
\\	弔, 辞	
\\	紺	
\\	こん	
\\	少女の白い肌に、紺のワンピースがよく似合っていた。	
\\	紺	
\\	一遍	
\\	いっぺん	
\\	一遍だけでいいから、キャビアを食べてみたかったの。	
\\	へん 
\\	ぺん 
\\	一, 遍	
\\	烏龍茶	
\\	うーろんちゃ, ウーロンちゃ	
\\	烏龍茶を氷抜きでください。	
\\	龍 
\\	烏 
\\	(うーろん) 
\\	烏, 龍, 茶	
\\	海賊	
\\	かいぞく	
\\	の 
\\	航海の途中で海賊に襲われました。	
\\	海, 賊	
\\	胎盤	
\\	たいばん	
\\	の 
\\	出産後、妻の胎盤を食べてみました。	
\\	胎, 盤	
\\	峡谷	
\\	きょうこく	
\\	この峡谷に架かる吊橋を渡ったことはありますか?	
\\	谷 
\\	(こく). 
\\	峡, 谷	
\\	迎賓館	
\\	げいひんかん	
\\	私は、迎賓館に招待されました。	
\\	迎, 賓, 館	
\\	惜しい	
\\	おしい	い 
\\	惜しい人を亡くしたものだ。	
\\	い 
\\	(お) 
\\	惜	
\\	果敢	
\\	かかん	
\\	な 
\\	彼は、世界各地で果敢な取材をしていた。	
\\	果, 敢	
\\	網膜	
\\	もうまく	
\\	の 
\\	網膜とは、カメラのフィルムに似た、眼の奥の膜のことです。	
\\	網, 膜	
\\	校閲	
\\	こうえつ	
\\	する 
\\	原稿を校閲してもらえませんか。	
\\	校, 閲	
\\	旋回	
\\	せんかい	
\\	する 
\\	ガレージで八の字旋回の練習をしてみました。	
\\	旋, 回	
\\	減俸	
\\	げんぽう	
\\	する 
\\	会社から、三割の減俸を言い渡されました。	
\\	ほう 
\\	ぽう 
\\	減, 俸	
\\	粗塩	
\\	あらじお, あらしお	
\\	お風呂に粗塩を入れると体にいいらしいですよ。	
\\	粗, 塩	
\\	乙	
\\	おつ	
\\	な 
\\	どーも、報告乙です。	
\\	合同会社
\\	を甲とし、
\\	を乙として、甲の業務の委託に関して、次の通り契約を締結する。	
\\	こうして山頂から生まれ故郷を見渡すのも、また乙なものですねぇ。	
\\	乙	
\\	烏賊	
\\	いか	
\\	烏賊の刺し身が大好物です。	
\\	いか 
\\	いか! 
\\	烏, 賊	
\\	胃腸	
\\	いちょう	
\\	な 
\\	の 
\\	胃腸がそんなに丈夫な方ではありません。	
\\	胃, 腸	
\\	盗賊	
\\	とうぞく	
\\	今度こそ盗賊の隠れ家を突き止めてやる。	
\\	盗, 賊	
\\	醸す	
\\	かもす	
\\	アイドルの大胆な発言が物議を醸している。	
\\	う 
\\	(かも). 
\\	醸	
\\	遮る	
\\	さえぎる	
\\	人の話を遮るのは失礼です。	
\\	う 
\\	(さえぎ). 
\\	遮	
\\	弔う	
\\	とむらう	
\\	ここには、殉職者の魂を弔う石碑が建てられています。	
\\	う 
\\	(とむら). 
\\	弔	
\\	鎮まる	
\\	しずまる	
\\	とりあえず、今は父の怒りが鎮まるのを待ちましょう。	
\\	鎮める 
\\	(しず) 
\\	鎮	
\\	漬ける	
\\	つける	
\\	こうやって醤油に漬けて食べてください。	
\\	う 
\\	漬	
\\	遍歴	
\\	へんれき	
\\	する 
\\	過去の男性遍歴を語ってくる女ってマジでうざいよな。	
\\	遍, 歴	
\\	間伐	
\\	かんばつ	
\\	する 
\\	間伐を行うことで、残った木は幹が太く枝葉がしっかりとした健全な木に育ちます。	
\\	間, 伐	
\\	坪	
\\	つぼ	
\\	この土地は何坪ぐらいですか?	
\\	坪	
\\	凹凸	
\\	おうとつ	
\\	な 
\\	の 
\\	凹凸のはっきりした顔立ちですね。	
\\	凹, 凸	
\\	普遍	
\\	ふへん	な 
\\	普遍的なテーマは、多くの人に受け入れられやすい。	
\\	普, 遍	
\\	不愉快	
\\	ふゆかい	
\\	な 
\\	このビデオは、見ていて不愉快です。	
\\	愉快 
\\	不, 愉, 快	
\\	悪循環	
\\	あくじゅんかん	
\\	まずは、あなたの心の悪循環を断ち切ることが大切です。	
\\	循環 
\\	悪, 循, 環	
\\	傍ら	
\\	かたわら	
\\	の 
\\	新聞局で働く傍ら、夜間学校に通っています。	
\\	(かたわ)! 
\\	傍	
\\	旋風	
\\	せんぷう	
\\	トフグは、日本語学習業界に旋風を巻き起こしています。	
\\	ふう 
\\	ぷう 
\\	旋, 風	
\\	酵母	
\\	こうぼ	
\\	自作酵母を使ってパンを作っています。	
\\	酵, 母	
\\	扶助	
\\	ふじょ	
\\	する 
\\	ほとんどの学生は、父母の扶助を受けて生活している。	
\\	扶, 助	
\\	恩赦	
\\	おんしゃ	
\\	大統領は、七面鳥に恩赦を与えた。	
\\	恩, 赦	
\\	憤慨	
\\	ふんがい	
\\	する 
\\	君の身勝手な言動に憤慨しているんだよ。	
\\	憤, 慨	
\\	解剖	
\\	かいぼう	
\\	する 
\\	解剖をしてみないと何とも言えません。	
\\	解, 剖	
\\	蝶々	
\\	ちょうちょう	
\\	彼女は蝶々のように自由な女性でした。	
\\	蝶, 々	
\\	触媒	
\\	しょくばい	
\\	の 
\\	つまり彼は二人にとって触媒のような役割を果たしたということだ。	
\\	触, 媒	
\\	更迭	
\\	こうてつ	
\\	する 
\\	監督の更迭が求められている。	
\\	更, 迭	
\\	鶏	
\\	にわとり	
\\	育てたひよこが鶏になりました。	
\\	鳥 (にわとり). 
\\	鳥 
\\	鶏	
\\	鶏卵	
\\	けいらん	
\\	鶏卵は日本が自給自足できる唯一の蛋白源です。	
\\	卵 
\\	(らん) 
\\	鶏, 卵	
\\	養鶏	
\\	ようけい	
\\	養鶏経営の難しさを痛感しました。	
\\	養, 鶏	
\\	鶏肉	
\\	とりにく, けいにく	
\\	できるだけ鶏肉のささみを食べるようにしています。	
\\	とりにく. 
\\	鳥 
\\	とり? 
\\	鶏, 肉	
\\	年譜	
\\	ねんぷ	
\\	この作家の年譜を作ってくれないか。	
\\	年, 譜	
\\	暁	
\\	あかつき	
\\	暁の冷たい空気に、気持ちが張り詰めた。	
\\	暁	
\\	老朽	
\\	ろうきゅう	
\\	する 
\\	の 
\\	その建物はかなり老朽しているので、立ち入らない方がいい。	
\\	老, 朽	
\\	戯曲	
\\	ぎきょく	
\\	の 
\\	これは、戯曲を原作としたミュージカルです。	
\\	戯, 曲	
\\	呆気	
\\	あっけ	
\\	彼女のおかしな言動に、みんな呆気に取られていました。	
\\	呆, 気	
\\	地殻	
\\	ちかく	
\\	の 
\\	日本列島は、千年に一度の大地殻変動期に突入しています。	
\\	地, 殻	
\\	享受	
\\	きょうじゅ	
\\	する 
\\	我が社は円安の恩恵をもろに享受しているんだ。	
\\	享, 受	
\\	晩酌	
\\	ばんしゃく	
\\	する 
\\	ダイエットのために晩酌を控えています。	
\\	晩, 酌	
\\	藩主	
\\	はんしゅ	
\\	藩主の墓に参ってきた。	
\\	藩, 主	
\\	藩	
\\	はん	
\\	江戸時代中期には、どこの藩も財政が苦しくなりました。	
\\	藩	
\\	瑞々しい	
\\	みずみずしい	い 
\\	彼女の瑞々しい歌声が人々を魅了した。	
\\	瑞, 々	
\\	遊戯	
\\	ゆうぎ	
\\	する 
\\	園児たちが音楽にあわせて遊戯をしています。	
\\	遊, 戯	
\\	棋譜	
\\	きふ	
\\	棋譜の付け方を教えてください。	
\\	棋, 譜	
\\	系譜	
\\	けいふ	
\\	この巻紙に一族の系譜が描かれています。	
\\	系, 譜	
\\	紳士協定	
\\	しんしきょうてい	
\\	あいつらとは紳士協定を結んでいるはずだ。	
\\	紳士 
\\	紳, 士, 協, 定	
\\	淑やか	
\\	しとやか	
\\	な 
\\	未だに淑やかな女性が好きな男性は多い。	
\\	(しと). 
\\	淑	
\\	手錠	
\\	てじょう	
\\	犯人と間違われて手錠を掛けられたんだ。	
\\	手, 錠	
\\	縛り首	
\\	しばりくび	
\\	江戸時代には縛り首という刑罰があった。	
\\	縛る 
\\	縛る 
\\	首. 
\\	る 
\\	り 
\\	縛, 首	
\\	豪傑	
\\	ごうけつ	
\\	日本史上、 最強の豪傑は誰だと思いますか。	
\\	豪, 傑	
\\	呆け	
\\	ぼけ	
\\	あいつは色呆けしてるんだよ。	
\\	呆ける? 
\\	る 
\\	呆	
\\	不朽	
\\	ふきゅう	
\\	の 
\\	これは不朽の名作です。	
\\	不, 朽	
\\	媒介	
\\	ばいかい	
\\	する 
\\	鳥が伝染病の媒介となると仮定しよう。	
\\	媒, 介	
\\	陪審	
\\	ばいしん	
\\	日本で陪審裁判が導入された経緯を教えてください。	
\\	陪, 審	
\\	淑女	
\\	しゅくじょ	
\\	紳士淑女のみなさん 、こんばんは。	
\\	淑, 女	
\\	錠	
\\	じょう	
\\	この薬を必ず一日一錠飲んで下さい。	
\\	錠	
\\	傑作	
\\	けっさく	
\\	な 
\\	これはピカソの傑作だね。	
\\	傑, 作	
\\	媒酌	
\\	ばいしゃく	
\\	する 
\\	私どもの結婚式の媒酌の労をおとりいただき、本当にありがとうございました。	
\\	媒, 酌	
\\	媒体	
\\	ばいたい	
\\	紙媒体で仕入れた情報の方が何故か安心するんだよね。	
\\	媒, 体	
\\	濁流	
\\	だくりゅう	
\\	濁流に飲み込まれそうになったのを、彼が救ってくれたのです。	
\\	濁, 流	
\\	譜面	
\\	ふめん	
\\	譜面を無料でダウンロードできるサイトを探しています。	
\\	譜, 面	
\\	管弦楽	
\\	かんげんがく	
\\	子供に管弦楽を習わせたいと思っています。	
\\	弦楽 
\\	管, 弦, 楽	
\\	容赦	
\\	ようしゃ	
\\	する 
\\	罪人は容赦なく罰せられなくてはならない。	
\\	容, 赦	
\\	嘱託	
\\	しょくたく	
\\	する 
\\	あの人は、再雇用制度を利用した嘱託社員なんです。	
\\	嘱, 託	
\\	錠剤	
\\	じょうざい	
\\	錠剤を飲み込むことができないので、砕いて飲んでいます。	
\\	錠, 剤	
\\	汚濁	
\\	おだく	
\\	する 
\\	湖の水質汚濁を防止するために規制が設けられた。	
\\	汚, 濁	
\\	肖像	
\\	しょうぞう	
\\	の 
\\	あの画家に私の肖像を描かせてみよう。	
\\	肖, 像	
\\	赦免	
\\	しゃめん	
\\	する 
\\	彼は特別に罪の赦免が認められた。	
\\	赦, 免	
\\	憤り	
\\	いきどおり	
\\	職場では理不尽なことが多くて憤りを感じています。	
\\	(いきどお) 
\\	憤	
\\	奔走	
\\	ほんそう	
\\	する 
\\	うちの社長は資金繰りに奔走しています。	
\\	奔, 走	
\\	帆	
\\	ほ	
\\	そろそろ帆を替えるなくちゃいかんなあ。	
\\	(ほ). 
\\	帆	
\\	帆走	
\\	はんそう	
\\	する 
\\	の 
\\	ほら、あそこにヨットが帆走しているよ。	
\\	帆, 走	
\\	楽譜	
\\	がくふ	
\\	ピアノは弾けますが、楽譜は読めません。	
\\	楽, 譜	
\\	縫目	
\\	ぬいめ	
\\	自分で縫ったので縫い目がガタガタなんです。	
\\	縫う 
\\	縫う 
\\	目. 
\\	う 
\\	い 
\\	縫い目 
\\	縫, 目	
\\	殻	
\\	から	
\\	蟹の殻はこの皿に入れてください。	
\\	(から) 
\\	殻	
\\	絹	
\\	きぬ	
\\	妻は、美しい絹のガウンを羽織って死んでいました。	
\\	(き) 
\\	(ぬ). 
\\	絹	
\\	絹糸	
\\	けんし, きぬいと	
\\	西陣の糸屋で絹糸を買ってきてちょうだい。	
\\	糸 
\\	(し)! 
\\	絹, 糸	
\\	脊椎	
\\	せきつい	
\\	の 
\\	交通事故で脊椎を損傷し、下半身不随になったのです。	
\\	脊, 椎	
\\	弦楽器	
\\	げんがっき	
\\	私の叔父は、弦楽器の修理をしています。	
\\	楽器 
\\	弦, 楽, 器	
\\	感慨	
\\	かんがい	
\\	感慨がこめられた彼女の歌声に感動しました。	
\\	感, 慨	
\\	硫酸	
\\	りゅうさん	
\\	どこで硫酸を手に入れたんだ。	
\\	硫, 酸	
\\	扶養	
\\	ふよう	
\\	する 
\\	扶養の範囲内で働かせてください。	
\\	扶, 養	
\\	執行猶予	
\\	しっこうゆうよ	
\\	の 
\\	あいつは執行猶予期間中に他の刑事事件を起こしたんだよ。	
\\	猶予 
\\	執, 行, 猶, 予	
\\	起訴猶予	
\\	きそゆうよ	
\\	起訴猶予処分がくだされました。	
\\	猶予 
\\	起, 訴, 猶, 予	
\\	拍子	
\\	ひょうし	
\\	拍子抜けしてしまったよ。	
\\	拍 
\\	(ひょう), 
\\	拍, 子	
\\	窃盗	
\\	せっとう	
\\	する 
\\	の 
\\	ストレス解消に窃盗を繰り返していたそうだ。	
\\	窃, 盗	
\\	金縛り	
\\	かなしばり	
\\	金縛りにあったことはありますか。	
\\	縛る 
\\	金 
\\	縛る. 
\\	金 
\\	かな 
\\	かね. 
\\	金, 縛	
\\	呆れ返る	
\\	あきれかえる	
\\	部屋があまりに汚かったので、呆れ返ってしまいました。	
\\	呆れる 
\\	返る 
\\	呆れる 
\\	返る. 呆れる 
\\	呆れ 
\\	返る 
\\	呆, 返	
\\	朽ちる	
\\	くちる	
\\	結局成功もせず、女もできずに朽ちていくはめになった。	
\\	う 
\\	(く) 
\\	朽	
\\	凌ぐ	
\\	しのぐ	
\\	貧乏な時は、雑草を食べて飢えを凌いでいました。	
\\	う 
\\	凌	
\\	濁す	
\\	にごす	
\\	核心をついた質問に、社長は言葉を濁した。	
\\	う 
\\	(にご) 
\\	濁	
\\	酌む	
\\	くむ	
\\	急いでバケツに水を酌んできてください。	
\\	う 
\\	(く)! 
\\	酌	
\\	飽くまでも	
\\	あくまでも	
\\	飽くまでも、これは個人的な見解です。	
\\	くまでも. 
\\	飽きる 
\\	飽	
\\	戯れる	
\\	たわむれる	
\\	子犬と戯れている少女の写真を撮りました。	
\\	う 
\\	(たわ). 
\\	戯	
\\	飽き	
\\	あき	
\\	私って、飽きっぽいのよね。	
\\	飽きる? 
\\	る 
\\	飽	
\\	奔放	
\\	ほんぽう	
\\	な 
\\	あの自由奔放な性格に惹かれてしまったんだ。	
\\	ほう 
\\	ぽう 
\\	奔, 放	
\\	環礁	
\\	かんしょう	
\\	その時、日本のマグロ漁船「第五福竜丸」がたまたまビキニ環礁を航行しており、乗組員が被ばくしてしまった。	
\\	環, 礁	
\\	一遍に	
\\	いっぺんに	
\\	一遍にいろんなことを言われても、分からないわ。	
\\	一遍 
\\	に 
\\	一遍 
\\	一, 遍	
\\	叔母	
\\	おば	
\\	叔母は学校の先生をしています。	
\\	伯母 
\\	叔, 母	
\\	慶事	
\\	けいじ	
\\	親しい友人の慶事は自分のことの様にうれしい。	
\\	慶, 事	
\\	憂き目	
\\	うきめ	
\\	失恋の憂き目に合いました。	
\\	憂, 
\\	(れ) 
\\	う 
\\	う 
\\	うれ 
\\	憂, 目	
\\	大胆	
\\	だいたん	
\\	な 
\\	グループは、大胆にも真っ昼間から犯行に及んだのです。	
\\	大, 胆	
\\	岬	
\\	みさき	
\\	岬の突端にカフェがあるんです。	
\\	岬	
\\	鋳造	
\\	ちゅうぞう	
\\	する 
\\	の 
\\	ここで、プラチナ・コインの鋳造をします。	
\\	鋳, 造	
\\	変遷	
\\	へんせん	
\\	する 
\\	1996年から2013年までのトップ日本語学習サイト20の変遷を見てみましょう。	
\\	変, 遷	
\\	慶祝	
\\	けいしゅく	
\\	する 
\\	慶祝の意を表して、乾杯!	
\\	慶, 祝	
\\	慶弔	
\\	けいちょう	
\\	慶弔用の礼服は持っているのか。	
\\	慶, 弔	
\\	危篤	
\\	きとく	
\\	の 
\\	父が危篤だという電話がありました。	
\\	危, 篤	
\\	遮断機	
\\	しゃだんき	
\\	遮断機が降りている時に、線路を渡ってはいけません。	
\\	遮断 
\\	遮, 断, 機	
\\	酪農	
\\	らくのう	
\\	ここは日本酪農発祥の地です。	
\\	酪, 農	
\\	回忌	
\\	かいき	
\\	今年は祖父の三回忌です。	
\\	回, 忌	
\\	左遷	
\\	させん	
\\	する 
\\	田舎の支店に左遷されました。	
\\	左, 遷	
\\	閑散	
\\	かんさん	
\\	な 
\\	駅の南裏は、いつも閑散としています。	
\\	閑, 散	
\\	落胆	
\\	らくたん	
\\	する 
\\	息子は、受験に失敗して落胆しているんだ。	
\\	落, 胆	
\\	擬装	
\\	ぎそう	
\\	する 
\\	の 
\\	食品会社の擬装を許してはいけません。	
\\	擬, 装	
\\	東亜	
\\	とうあ	
\\	太平洋戦争のことを、大東亜戦争と呼ぶのは何故ですか。	
\\	東, 亜	
\\	大腸菌	
\\	だいちょうきん	
\\	大腸菌を美少女に擬人化させたカードアプリ・ゲームがあります。	
\\	大腸 
\\	大, 腸, 菌	
\\	雌	
\\	めす	
\\	の 
\\	この犬は雌ですか?	
\\	雌	
\\	雌花	
\\	めばな	
\\	の 
\\	ヒマラヤスギは、樹齢30年を超えないと雌花を付けません。	
\\	雌, 花	
\\	一周忌	
\\	いっしゅうき	
\\	七月には、妻の父の一周忌の法事があります。	
\\	一周 
\\	一, 周, 忌	
\\	取り敢えず	
\\	とりあえず	
\\	取り敢えず枝豆とビールをください。	
\\	取, 敢	
\\	親睦	
\\	しんぼく	
\\	する 
\\	親睦を深めるためのパーティーを準備しています。	
\\	親, 睦	
\\	胆石	
\\	たんせき	
\\	胆嚢炎は、ほとんどの場合、胆石が原因です。	
\\	胆, 石	
\\	胆	
\\	たん, きも	
\\	お前はもう少し胆を練る必要がある。	
\\	胆	
\\	不均衡	
\\	ふきんこう	
\\	な 
\\	先進国と新興国・発展途上国間の経常収支の不均衡は是正されるのでしょうか。	
\\	均衡 
\\	不, 均, 衡	
\\	模擬	
\\	もぎ	
\\	の 
\\	模擬試験の結果はどうだったの?	
\\	模, 擬	
\\	甚大	
\\	じんだい	
\\	な 
\\	大型台風が甚大な被害をもたらした。	
\\	甚, 大	
\\	侮辱	
\\	ぶじょく	
\\	する 
\\	の 
\\	こんな侮辱を受けて、黙っていろというのか。	
\\	侮, 辱	
\\	漆黒	
\\	しっこく	
\\	の 
\\	漆黒の闇に青い地球が浮かんでいた。	
\\	漆, 黒	
\\	漆	
\\	うるし	
\\	漆の葉を触ると手がかぶれますよ。	
\\	(うるし). 
\\	漆	
\\	漆器	
\\	しっき	
\\	漆器の手入れの仕方を教えてください。	
\\	漆, 器	
\\	法曹	
\\	ほうそう	
\\	法曹を目指そうと思ったきっかけは何ですか。	
\\	法, 曹	
\\	崇高	
\\	すうこう	
\\	な 
\\	自然の崇高な美しさに感動しました。	
\\	崇, 高	
\\	婚姻	
\\	こんいん	
\\	する 
\\	の 
\\	婚姻届を役所に提出しました。	
\\	婚, 姻	
\\	卑劣	
\\	ひれつ	
\\	な 
\\	卑劣な手段を使いやがって。	
\\	卑, 劣	
\\	忌	
\\	き	
\\	忌が明けたら、みんなで旅行でも行きましょうか。	
\\	忌	
\\	遷都	
\\	せんと	
\\	する 
\\	来年は、遷都1500年記念のイベントが開催されます。	
\\	遷, 都	
\\	ご無沙汰	
\\	ごぶさた	
\\	する 
\\	ご無沙汰しております。	
\\	無, 沙, 汰	
\\	〜漬け	
\\	づけ	
\\	胡瓜の浅漬けならありますよ。	
\\	漬ける 
\\	る) 
\\	漬ける 
\\	る 
\\	漬	
\\	魂胆	
\\	こんたん	
\\	する 
\\	あわよくばおこぼれに預かりたいという魂胆が見え見えだよ。	
\\	魂, 胆	
\\	閑静	
\\	かんせい	
\\	な 
\\	この辺りは、閑静な住宅街になっています。	
\\	閑, 静	
\\	暗礁	
\\	あんしょう	
\\	船長は、暗礁に気をつけてと言って、僕に操縦を任せた。	
\\	捜査は暗礁に乗り上げてしまったな。	
\\	暗, 礁	
\\	岩礁	
\\	がんしょう	
\\	岩礁を埋め立て、人工島に造り替えたんです。	
\\	岩, 礁	
\\	早乙女	
\\	さおとめ	
\\	昔は稲の苗を水田に植えつける女性のことを「早乙女」や「植女」と呼んでいました。	
\\	さ 
\\	早速, 
\\	おつ 
\\	おと, 
\\	女 
\\	め 
\\	さおとめ. 
\\	早, 乙, 女	
\\	峠	
\\	とうげ	
\\	この峠で転ぶと、三年だけしか生きられないという言い伝えがあります。	
\\	(とうげ) 
\\	峠	
\\	浪人	
\\	ろうにん	
\\	する 
\\	の 
\\	受験に失敗し、浪人生活に突入しました。	
\\	お前みたいな浪人の要求には応じられないな。	
\\	浪, 人	
\\	禍根	
\\	かこん	
\\	できれば禍根を残すことは避けたい。	
\\	禍, 根	
\\	侮蔑	
\\	ぶべつ	
\\	する 
\\	叔母の言葉は、いつも軽い侮蔑を帯びている。	
\\	侮, 蔑	
\\	軽蔑	
\\	けいべつ	
\\	する 
\\	私、あの人のこと軽蔑しているの。	
\\	軽 
\\	(けい) 
\\	軽, 蔑	
\\	吟味	
\\	ぎんみ	
\\	する 
\\	結婚相手はよく吟味してから決めたい。	
\\	吟, 味	
\\	紡績	
\\	ぼうせき	
\\	する 
\\	牽切紡績とは、化学繊維特有の紡績方法で、綿からの紡績ではありません。	
\\	紡, 績	
\\	紡織	
\\	ぼうしょく	
\\	日本繊維産業は、綿紡織、化繊を中心として復興しました。	
\\	織 
\\	(しょく) 
\\	紡, 織	
\\	萌え	
\\	もえ	
\\	このブログでは、萌え系を中心に、僕の好きなアニメや漫画を紹介していきます。	
\\	(もえ) 
\\	萌	
\\	駐屯	
\\	ちゅうとん	
\\	する 
\\	あそこには米軍が駐屯しているんだよ。	
\\	駐, 屯	
\\	弁慶	
\\	べんけい	
\\	弁慶の泣き所を棚にぶつけたんだ。	
\\	弁慶 
\\	弁, 慶	
\\	卑屈	
\\	ひくつ	
\\	な 
\\	どうしてすぐに卑屈になるの?	
\\	卑, 屈	
\\	禁忌	
\\	きんき	
\\	する 
\\	の 
\\	よく、禁忌を犯すと手足をもがれると言われますが、これは本当でしょうか。	
\\	禁, 忌	
\\	凹む	
\\	へこむ	
\\	失恋で凹んでいます。	
\\	早速新車を凹ませてしまって、凹んでいます。	
\\	う 
\\	(へこ). 
\\	凹	
\\	漬かる	
\\	つかる	
\\	温泉に漬かってゆっくりした。	
\\	漬ける 
\\	漬ける, 
\\	漬	
\\	詠む	
\\	よむ	
\\	俳句を詠む人のことを俳人と呼びます。	
\\	う 
\\	(よ). 
\\	詠	
\\	侮る	
\\	あなどる	
\\	子供だからといって侮らない方がいい。	
\\	う 
\\	(あなど) 
\\	侮	
\\	堪える	
\\	たえる, こたえる, こらえる	
\\	この作品は読むに堪えない出来栄えだ。	
\\	う 
\\	(た). 
\\	堪	
\\	憧れる	
\\	あこがれる	
\\	あの先輩に憧れているんです。	
\\	う 
\\	憧	
\\	蔑む	
\\	さげすむ	
\\	あの男は、無意識に人のことを蔑んでいるのよ。	
\\	う 
\\	(さげす) 
\\	蔑	
\\	紡ぐ	
\\	つむぐ	
\\	昔はこうやって糸を紡いでいたんですよ。	
\\	う 
\\	(つむ) 
\\	紡	
\\	漬物	
\\	つけもの	
\\	この漬物は誰が漬けたんですか。	
\\	漬ける 
\\	る 
\\	け 
\\	(もの) 
\\	漬け物 
\\	漬, 物	
\\	叔父	
\\	おじ	
\\	叔父はこの学校の校長です。	
\\	伯父 
\\	叔, 父	
\\	凸凹	
\\	でこぼこ	
\\	する 
\\	な 
\\	の 
\\	床がちょっと凸凹しているのが気になるね。	
\\	凹凸 
\\	(でこぼこ). 
\\	凸, 凹	
\\	普遍的	
\\	ふへんてき	な 
\\	とても普遍的なテーマだと思います。	
\\	普遍 
\\	普, 遍, 的	
\\	忌まわしい	
\\	いまわしい	い 
\\	この忌まわしい記憶をどうか消し去ってください。	
\\	い 
\\	(い) 
\\	忌	
\\	稚拙	
\\	ちせつ	
\\	な 
\\	稚拙な文章ですがもしよければ読んでください。	
\\	稚, 拙	
\\	流浪	
\\	るろう	
\\	する 
\\	全てのジプシーが貧しい流浪の民というわけではない。	
\\	流, 浪	
\\	浮浪者	
\\	ふろうしゃ	
\\	浮浪者でも7割の人は何らかの仕事をして日銭を稼いでいる。	
\\	浮 
\\	(ふ). 
\\	浮, 浪, 者	
\\	甚だ	
\\	はなはだ	
\\	そのようなことが事実だとすれば、甚だ遺憾である。	
\\	(はなは).
\\	甚	
\\	放浪	
\\	ほうろう	
\\	する 
\\	の 
\\	当時は、逃げるように各地を転々と放浪していました。	
\\	放, 浪	
\\	戦禍	
\\	せんか	
\\	戦禍を逃れて田舎に疎開した。	
\\	戦, 禍	
\\	浪費	
\\	ろうひ	
\\	する 
\\	の 
\\	妻の浪費癖がなかなか治らないんです。	
\\	浪, 費	
\\	惜しまない	
\\	おしまない	い 
\\	目的を成し遂げるには、努力を惜しまないことが大切だと思います。	
\\	惜しむ 
\\	ない, 
\\	惜	
\\	崇拝	
\\	すうはい	
\\	する 
\\	どうして偶像崇拝を禁止するのですか。	
\\	崇, 拝	
\\	憧れ	
\\	あこがれ	
\\	の 
\\	憧れのダービー、なんとしても勝ちたいです。	
\\	憧	
\\	解剖学	
\\	かいぼうがく	
\\	大学では解剖学を研究しています。	
\\	解剖 
\\	解, 剖, 学	
\\	匿名	
\\	とくめい	
\\	匿名で投書をしました。	
\\	匿, 名	
\\	蛮行	
\\	ばんこう	
\\	あいつらはそこで蛮行を繰り返しているんだ。	
\\	蛮, 行	
\\	花婿	
\\	はなむこ	
\\	花婿が結婚式場から逃げ出したらしいわよ。	
\\	花, 婿	
\\	煩忙	
\\	はんぼう	
\\	な 
\\	日々の煩忙から逃れ、ハワイにやって来ました。	
\\	煩, 忙	
\\	某	
\\	ぼう	
\\	某私立高校に伝わる、奇妙な噂があるんだ。	
\\	某	
\\	吸い殻	
\\	すいがら	
\\	吸い殻が二本たまったら、すぐに灰皿を交換してください。	
\\	吸う 
\\	殻 
\\	吸う 
\\	殻. 
\\	う 
\\	い 
\\	吸, 殻	
\\	煩雑	
\\	はんざつ	
\\	な 
\\	煩雑な手続きに苛々します。	
\\	煩, 雑	
\\	死刑囚	
\\	しけいしゅう	
\\	死刑囚はこの階段を登って死刑台へと向かうんだ。	
\\	死刑 
\\	死, 刑, 囚	
\\	矯正	
\\	きょうせい	
\\	する 
\\	歯の矯正をしていました。	
\\	矯, 正	
\\	逝去	
\\	せいきょ	
\\	する 
\\	社長の逝去は本当に大きなショックです。	
\\	逝, 去	
\\	感慨無量	
\\	かんがいむりょう	
\\	な 
\\	あんなに小さかった娘の成長を見て、感慨無量でした。	
\\	感慨 
\\	感, 慨, 無, 量	
\\	感慨深い	
\\	かんがいぶかい	い 
\\	初めて自分の書いたものが出版されるのは、やっぱり感慨深いものがあります。	
\\	感慨 
\\	深い 
\\	ふ 
\\	ぶ, 
\\	感, 慨, 深	
\\	妄想	
\\	もうそう	
\\	する 
\\	いつも大金持ちになる妄想ばかりしています。	
\\	妄, 想	
\\	下痢	
\\	げり	
\\	する 
\\	の 
\\	朝から下痢が止まりません。	
\\	下, 痢	
\\	殉職	
\\	じゅんしょく	
\\	する 
\\	その刑事は、銃弾を浴びて殉職しました。	
\\	殉, 職	
\\	罷免	
\\	ひめん	
\\	する 
\\	あの大臣は罷免されたよ。	
\\	罷, 免	
\\	赤痢	
\\	せきり	
\\	の 
\\	男女計5人から細菌性の赤痢が検出されました。	
\\	赤 
\\	(せき). 
\\	赤, 痢	
\\	漸く	
\\	ようやく	
\\	漸く作家としてのスタートラインに立つことができました。	
\\	漸	
\\	狐	
\\	きつね	
\\	かわいい狐が茂みからひょこっと顔を出した。	
\\	狐	
\\	升	
\\	ます	
\\	作文の書き出しは、一升開けて下さい。	
\\	升	
\\	謹賀新年	
\\	きんがしんねん	
\\	年賀状には謹賀新年と書いてありました。	
\\	新年 
\\	謹, 賀, 新, 年	
\\	隠匿	
\\	いんとく	
\\	する 
\\	彼女は犯人を隠匿していたんだよ。	
\\	隠, 匿	
\\	婿	
\\	むこ	
\\	こんないいお婿さんにもらわれて、娘は幸せ者です。	
\\	婿	
\\	弾劾	
\\	だんがい	
\\	する 
\\	大統領の弾劾を求める大規模な反政府デモが開かれました。	
\\	弾, 劾	
\\	藍	
\\	あい	
\\	藍染の藍は天然ですが、ジーンズに使われる合成インディゴは人工的なものです。	
\\	藍	
\\	湖畔	
\\	こはん	
\\	の 
\\	夏は、静かな湖畔のロッジで読書をしています。	
\\	湖, 畔	
\\	長唄	
\\	ながうた	
\\	長唄教室に通い始めました。	
\\	長, 唄	
\\	廉価	
\\	れんか	
\\	な 
\\	廉価版のゲームのパッケージって安物感丸出しだよな。	
\\	廉, 価	
\\	帆柱	
\\	ほばしら	
\\	この木は帆柱にちょうどいいぞ。	
\\	帆, 柱	
\\	渓流	
\\	けいりゅう	
\\	ラフティングで渓流を下りました。	
\\	渓, 流	
\\	煩い	
\\	うるさい	い 
\\	彼がポテトチップスを食べる音が煩くて苛々します。	
\\	い 
\\	(うるさい). 
\\	煩	
\\	管弦楽団	
\\	かんげんがくだん	
\\	小さな管弦楽団でバイオリンを演奏しています。	
\\	管弦楽 
\\	管, 弦, 楽, 団	
\\	桟橋	
\\	さんばし	
\\	船は桟橋に繋いでいます。	
\\	桟 
\\	橋. 
\\	桟, 橋	
\\	戯れ	
\\	たわむれ	
\\	戯れに文章を綴っただけなのに、賞を獲ってしまった。	
\\	戯れる 
\\	戯	
\\	囚人	
\\	しゅうじん	
\\	どうして囚人は縞々の服を着ているんですか。	
\\	囚, 人	
\\	容赦なく	
\\	ようしゃなく	
\\	父は弟を容赦なく殴りつけました。	
\\	容赦 
\\	なく 
\\	なく 
\\	容, 赦	
\\	帆船	
\\	はんせん, ほぶね	
\\	店内は帆船をイメージした内装となっております。	
\\	帆, 船	
\\	野蛮	
\\	やばん	
\\	な 
\\	そんな野蛮な人たちと付き合ってはいけません。	
\\	野, 蛮	
\\	醜聞	
\\	しゅうぶん	
\\	誰がこんな根も葉もない醜聞を流したんだ。	
\\	醜, 聞	
\\	藻	
\\	も	
\\	藻が絡まってしまったみたいだ。	
\\	(も) 
\\	藻	
\\	海藻	
\\	かいそう	
\\	この海藻は食べられますか。	
\\	海, 藻	
\\	分泌	
\\	ぶんぴつ, ぶんぴ	
\\	する 
\\	の 
\\	胃液は1日1~2
\\	も分泌されています。	
\\	分, 泌	
\\	醜態	
\\	しゅうたい	
\\	とんだ醜態を晒してしまったよ。	
\\	醜, 態	
\\	寡黙	
\\	かもく	な 
\\	寡黙で読書好きな男性がタイプです。	
\\	寡, 黙	
\\	唄	
\\	うた	
\\	この唄は、私と兄が一緒に作りました。	
\\	唄	
\\	娘婿	
\\	むすめむこ	
\\	いずれは娘婿を社長にするつもりです。	
\\	娘, 婿	
\\	南蛮	
\\	なんばん	
\\	チキン南蛮の簡単でおいしいレシピがあれば教えてください。	
\\	南, 蛮	
\\	湧水	
\\	ゆうすい, わきみず	
\\	地下水が地表に自然に出てきたものを、湧水といいます。	
\\	湧, 水	
\\	升目	
\\	ますめ	
\\	升目を数えると、全部で400ありました。	
\\	升, 目	
\\	硫黄	
\\	いおう	
\\	の 
\\	この温泉には硫黄が含まれています。	
\\	(いおう)!! 
\\	硫, 黄	
\\	悪戯	
\\	いたずら	
\\	する 
\\	な 
\\	の 
\\	今子供達の間で流行っている悪戯ですよ。	
\\	(いたずら), 
\\	悪, 戯	
\\	倹約	
\\	けんやく	
\\	する 
\\	な 
\\	の 
\\	いくらマイホームを買いたいからって、度を過ぎた倹約を強いるのはどうかと思います。	
\\	倹, 約	
\\	慕う	
\\	したう	
\\	私は彼のことを兄のように慕っていました。	
\\	う 
\\	(した)... 
\\	慕	
\\	濁る	
\\	にごる	
\\	煙草の煙で部屋の空気が濁っています。	
\\	濁す 
\\	濁す. 
\\	濁	
\\	拷問	
\\	ごうもん	
\\	する 
\\	の 
\\	俺の姉は、マフィアから残虐な拷問の受けたんだ。	
\\	拷, 問	
\\	唄う	
\\	うたう	
\\	私のおじさんは、唄うアコーディオン弾きです。	
\\	う 
\\	歌う 
\\	唄	
\\	湧く	
\\	わく	
\\	悔し涙が湧きました。	
\\	う 
\\	(わ) 
\\	湧	
\\	坑道	
\\	こうどう	
\\	地下20
\\	のところに坑道が掘られています。	
\\	坑, 道	
\\	醜悪	
\\	しゅうあく	
\\	な 
\\	老婆の幽霊が、醜悪な顔でこちらを睨んでいた。	
\\	醜, 悪	
\\	醜い	
\\	みにくい	い 
\\	あの女は、顔は美しいが心は醜い。	
\\	い 
\\	(みにく). 
\\	醜	
\\	泌尿器	
\\	ひにょうき	
\\	泌尿器の病気で入院することになりました。	
\\	泌, 尿, 器	
\\	渓谷	
\\	けいこく	
\\	温泉湯舎から渓谷美を一望することができます。	
\\	渓, 谷	
\\	寡婦	
\\	かふ	
\\	寡婦控除について詳しく教えてください。	
\\	寡, 婦	
\\	思慕	
\\	しぼ	
\\	する 
\\	異国の地で、母への思慕を重ねています。	
\\	思, 慕	
\\	亅	
\\	大	
\\	工	
\\	十	
\\	丶	
\\	入	
\\	ハ	
\\	一	
\\	亠	
\\	山	
\\	口	
\\	九	
\\	人	
\\	大. 
\\	力	
\\	勹	
\\	川	
\\	七	
\\	丿	
\\	日	
\\	ト	
\\	ト 
\\	ト, 
\\	木	
\\	二	
\\	女	
\\	本	
\\	弓	
\\	子	
\\	牛	
\\	土	
\\	犬	
\\	夕	
\\	目	
\\	火	
\\	五	
\\	尸	
\\	彡	
\\	手	
\\	冂	
\\	天	
\\	王	
\\	儿	
\\	中	
\\	月	
\\	ナ	
\\	ナ 
\\	(ナ) 
\\	(メ), 
\\	ム	
\\	田	
\\	小	
\\	立	
\\	石	
\\	又	
\\	止	
\\	丁	
\\	刀	
\\	(力), 
\\	千	
\\	メ	
\\	水	
\\	白	
\\	文	
\\	矢	
\\	广	
\\	方	
\\	戸	
\\	一 
\\	一 
\\	干	
\\	父	
\\	扌	
\\	(手) 
\\	毛	
\\	心	
\\	生	
\\	今	
\\	古	
\\	元	
\\	幺	
\\	匕	
\\	用	
\\	巾	
\\	夂	
\\	也	
\\	竹	
\\	巴	
\\	支	
\\	車	
\\	弋	
\\	厂	
\\	耳	
\\	(目), 
\\	气	
\\	艹	
\\	足	
\\	冖	
\\	(冂) 
\\	虫	
\\	イ	
\\	人), 
\\	彳	
\\	(イ) 
\\	主	
\\	寸	
\\	不	
\\	兀	
\\	皿	
\\	赤	
\\	米	
\\	(田). 
\\	宀	
\\	士	
\\	見	
\\	貝	
\\	ネ	
\\	糸	
\\	乚	
\\	世	
\\	斤	
\\	青	
\\	耂	
\\	色	
\\	羽	
\\	行	
\\	金	
\\	冫	
\\	乍	
\\	开	
\\	可	
\\	肉	
\\	会	
\\	冋	
\\	雨	
\\	走	
\\	(足), 
\\	言	
\\	自	
\\	氵	
\\	(丶), 
\\	廾	
\\	十 
\\	十... 
\\	里	
\\	西	
\\	毋	
\\	田 
\\	ヨ	
\\	ホ	
\\	犭	
\\	血	
\\	(皿). 
\\	灬	
\\	凵	
\\	市	
\\	亡	
\\	食	
\\	占	
\\	禾	
\\	未	
\\	刂	
\\	欠	
\\	斗	
\\	長	
\\	マ	
\\	(マ). 
\\	首	
\\	羊	
\\	舌	
\\	歹	
\\	鳥	
\\	黒	
\\	舟	
\\	周	
\\	氏	
\\	魚	
\\	門	
\\	夫	
\\	己	
\\	豕	
\\	矛	
\\	卩	
\\	几	
\\	孝	
\\	寺	
\\	且	
\\	勿	
\\	黄	
\\	匚	
\\	云	
\\	比	
\\	易	
\\	穴	
\\	ユ	
\\	馬	
\\	鳥 
\\	罒	
\\	釆	
\\	禾 
\\	巳	
\\	合	
\\	殳	
\\	反	
\\	聿	
\\	阝	
\\	咅	
\\	疋	
\\	正	
\\	永	
\\	重	
\\	覀	
\\	史	
\\	介	
\\	台	
\\	辛	
\\	癶	
\\	酉	
\\	豆	
\\	束	
\\	丙	
\\	失	
\\	頁	
\\	各	
\\	辰	
\\	曲	
\\	売	
\\	疒	
\\	乃	
\\	圣	
\\	隹	
\\	戈	
\\	及	
\\	(乃)? 
\\	少	
\\	申	
\\	兄	
\\	令	
\\	艮	
\\	単	
\\	皮	
\\	音	
\\	亜	
\\	予	
\\	業	
\\	其	
\\	早	
\\	免	
\\	直	
\\	示	
\\	求	
\\	者	
\\	共	
\\	廴	
\\	弟	
\\	身	
\\	冊	
\\	央	
\\	丸	
\\	東	
\\	象	
\\	果	
\\	良	
\\	百	
\\	意	
\\	原	
\\	品	
\\	畐	
\\	尺	
\\	忄	
\\	坴	
\\	才	
\\	分	
\\	勺	
\\	公	
\\	井	
\\	丈	
\\	男	
\\	句	
\\	去	
\\	歩	
\\	付	
\\	壴	
\\	面	
\\	谷	
\\	楽	
\\	非	
\\	祭	
\\	吉	
\\	兑	
\\	無	
\\	苟	
\\	虍	
\\	争	
\\	尭	
\\	幸	
\\	辛, 
\\	昔	
\\	曽	
\\	午	
\\	先	
\\	凶	
\\	成	
\\	复	
\\	司	
\\	我	
\\	感	
\\	由	
\\	右	
\\	巛	
\\	飛	
\\	至	
\\	兼	
\\	(美). 
\\	義	
\\	角	
\\	刃	
\\	半	
\\	啇	
\\	次	
\\	番	
\\	平	
\\	斉	
\\	能	
\\	充	
\\	呂	
\\	安	
\\	弗	
\\	爰	
\\	丬	
\\	衣	
\\	鬼	
\\	革	
\\	朝, 
\\	亭	
\\	道	
\\	丩	
\\	受	
\\	制	
\\	旦	
\\	客	
\\	韋	
\\	然	
\\	両	
\\	京	
\\	友	
\\	莫	
\\	冓	
\\	禹	
\\	交	
\\	甫	
\\	告	
\\	章	
\\	万	
\\	倉	
\\	召	
\\	化	
\\	出	
\\	農	
\\	民	
\\	而	
\\	代	
\\	余	
\\	屯	
\\	巨	
\\	麗	
\\	奇	
\\	専	
\\	上	
\\	責	
\\	尞	
\\	区	
\\	乙	
\\	臣	
\\	系	
\\	北	
\\	片	
\\	監	
\\	冘	
\\	(大), 
\\	甲	
\\	鬲	
\\	牙	
\\	郎	
\\	扁	
\\	並	
\\	离	
\\	卬	
\\	必	
\\	舎	
\\	夜	
\\	同	
\\	夋	
\\	内	
\\	間	
\\	享	
\\	蔵	
\\	喿	
\\	広	
\\	堇	
\\	呉	
\\	爪	
\\	員	
\\	亥	
\\	帯	
\\	将	
\\	為	
\\	彑	
\\	亀	
\\	明	
\\	舛	
\\	発	
\\	南	
\\	甘	
\\	貴	
\\	兆	
\\	賞	
\\	串	
\\	暴	
\\	包	
\\	有	
\\	辟	
\\	県	
\\	空	
\\	思	
\\	歯	
\\	君	
\\	旧	
\\	岡	
\\	㐮	
\\	真	
\\	巻	
\\	三	
\\	雇	
\\	毎	
\\	保	
\\	喜	
\\	夆	
\\	妻	
\\	骨	
\\	瓜	
\\	屋	
\\	豸	
\\	龍	
\\	耒	
\\	缶	
\\	竜	
\\	風	
\\	更	
\\	家	
\\	守	
\\	太	
\\	恵	
\\	敝	
\\	波	
\\	容	
\\	玄	
\\	光	
\\	甚	
\\	(匹) 
\\	(其) 
\\	回	
\\	荒	
\\	名	
\\	禺	
\\	胃	
\\	夌	
\\	高	
\\	疑	
\\	凹	
\\	凸	
\\	下	
\\	勇	
\\	上	
\\	じょう	うえ, あ, のぼ, うわ, かみ	ト, 一	
\\	(じょう).	
\\	じょう. 
\\	じょ. 
\\	じょう 
\\	下	
\\	か, げ	した, さ, くだ, お	一, ト	
\\	(か). 
\\	げ, 
\\	下 
\\	げ. 
\\	大	
\\	たい, だい	おお	大	
\\	(たい, だい) 
\\	工	
\\	こう, く	
\\	工	
\\	こう 
\\	こういち 
\\	こう, 
\\	こういち. 
\\	(こういち), 
\\	こういち 
\\	こういち
\\	こういち 
\\	工 
\\	こう 
\\	八	
\\	はち	や, よう	ハ	
\\	(はち) 
\\	入	
\\	にゅう	はい, い	入	
\\	(にゅう) 
\\	山	
\\	さん	やま	山	
\\	(さん) 
\\	口	
\\	こう, く	くち	口	
\\	こういち 
\\	工 
\\	こういち
\\	こういち 
\\	九	
\\	く, きゅう	ここの	九	
\\	(く) 
\\	(きゅう). 
\\	一	
\\	いち, いつ	ひと	一	
\\	(いち).	
\\	人	
\\	にん, じん	ひと, と	人	
\\	(にん) 
\\	(じん) 
\\	力	
\\	りょく, りき	ちから	力	
\\	(りき, りょく) 
\\	(りき), 
\\	(りょく) 
\\	川	
\\	せん	かわ	川	
\\	(かわ) 
\\	七	
\\	しち	なな, なの	七	
\\	(しち) 
\\	(しち). 
\\	十	
\\	じゅう	とお	十	
\\	(じゅう) 
\\	三	
\\	さん	み	一, 二	
\\	(さん). 
\\	二	
\\	に	ふた	二	
\\	(に) 
\\	女	
\\	じょ	おんな, め	女	
\\	(じょ), 
\\	じょ 
\\	じょ 
\\	じょう 
\\	じょ 
\\	じょう 
\\	又	
\\	また	又	
\\	(また) 
\\	また 
\\	玉	
\\	ぎょく	たま	王, 丶	
\\	(たま).	
\\	本	
\\	ほん	もと	本	
\\	(ほん) 
\\	子	
\\	し, す	こ	子	
\\	し 
\\	す 
\\	(し) 
\\	(す) 
\\	丸	
\\	がん	まる	九, 丶	
\\	正	
\\	せい, しょう	ただ, まさ	一, 止	
\\	(しょう) 
\\	(せい) 
\\	土	
\\	ど, と	つち	土	
\\	(ど) 
\\	犬	
\\	けん	いぬ	犬	
\\	夕	
\\	せき	ゆう	夕	
\\	(ゆう) 
\\	ゆう 
\\	出	
\\	しゅつ	で, だ	山	
\\	(しゅつ) 
\\	目	
\\	もく	め	目	
\\	(め)!
\\	了	
\\	りょう	
\\	亅	
\\	(りょう) 
\\	火	
\\	か	ひ, ほ	火	
\\	か, 
\\	(か)! 
\\	五	
\\	ご	いつ	五	
\\	(ご) 
\\	四	
\\	し	よん, よ	口, 儿	
\\	(し). 
\\	才	
\\	さい	
\\	一, 亅, 丿	
\\	(さい).	
\\	手	
\\	しゅ, ず	て	手	
\\	て 
\\	(て) 
\\	天	
\\	てん	あま	天	
\\	(てん) 
\\	王	
\\	おう	
\\	王	
\\	(おう) 
\\	王 
\\	おう 
\\	おう 
\\	左	
\\	さ	ひだり	ナ, 工	
\\	(さ). 
\\	中	
\\	ちゅう	なか	中	
\\	ちゅう 
\\	中, 
\\	月	
\\	げつ, がつ	つき	月	
\\	げつ 
\\	がつ 
\\	(げつ) 
\\	々	
\\	のま	勹, 丶	
\\	人々 
\\	人人. 
\\	のま 
\\	ノ 
\\	マ 
\\	(のま) 
\\	々
\\	田	
\\	でん	た	田	
\\	た, 
\\	(た), 
\\	右	
\\	ゆう, う	みぎ	ナ, 口	
\\	(ゆう). 
\\	ゆう 
\\	六	
\\	ろく	む	亠, ハ	
\\	ろく 
\\	(ろく) 
\\	小	
\\	しょう	ちい, こ, お	小	
\\	(しょう). 
\\	う 
\\	立	
\\	りつ, りゅう	た	立	
\\	(りつ).	
\\	丁	
\\	ちょう, てい	
\\	丁	
\\	(ちょう).	
\\	ちょう 
\\	日	
\\	にち, じつ	ひ, か, び	日	
\\	にち. 
\\	(にち), 
\\	じつ. 
\\	刀	
\\	とう	かたな	刀	
\\	とうきょう 
\\	とうきょう 
\\	とうきょう.	
\\	とうきょう 
\\	千	
\\	せん	ち	千	
\\	千? 
\\	(せん). 
\\	木	
\\	もく, ぼく	き, こ	木	
\\	(もく) 
\\	水	
\\	すい	みず	水	
\\	(すい). 
\\	すい 
\\	白	
\\	はく	しろ, しら	白	
\\	(はく) 
\\	文	
\\	ぶん, もん	
\\	文	
\\	(ぶん) 
\\	(もん) 
\\	円	
\\	えん	まる	冂, 亠	
\\	えん. 
\\	えん!
\\	えん 
\\	矢	
\\	し	や	矢	
\\	や 
\\	(や) 
\\	市	
\\	し	いち	亠, 巾	
\\	(し) 
\\	牛	
\\	ぎゅう	うし	牛	
\\	わぎゅう 
\\	ぎゅう 
\\	ぎゅう.	
\\	わぎゅう 
\\	切	
\\	せつ	き	七, 刀	
\\	(せつ) 
\\	方	
\\	ほう	かた	方	
\\	(ほう). 
\\	戸	
\\	こ	と	戸	
\\	(と), 
\\	太	
\\	たい, た	ふと	大, 丶	
\\	(たい).	
\\	大 
\\	父	
\\	ふ	ちち	父	
\\	(ちち) 
\\	ちちちちちちちち〜!	
\\	少	
\\	しょう	すこ	小, 丿	
\\	(しょう) 
\\	友	
\\	ゆう	とも	ナ, 又	
\\	(ゆう)
\\	毛	
\\	もう	け	毛	
\\	(もう) 
\\	半	
\\	はん	なか	干		
\\	(はん), 
\\	心	
\\	しん	こころ	心	
\\	(しん).	
\\	内	
\\	ない	うち	冂, 人	
\\	(ない) 
\\	生	
\\	せい, しょう	い, なま, う, は, き	生	
\\	(せい). 
\\	台	
\\	だい, たい	
\\	ム, 口	
\\	(だい) 
\\	母	
\\	ぼ	はは	日, 丶	
\\	(はは)!
\\	午	
\\	ご	
\\	丿, 干	
\\	(ご) 
\\	北	
\\	ほく	きた	扌, 匕	
\\	(ほく) 
\\	今	
\\	こん	いま	今	
\\	(こん) 
\\	古	
\\	こ	ふる	古	
\\	子
\\	(こ) 
\\	子 
\\	こ 
\\	子 
\\	(古) 
\\	こ.	
\\	元	
\\	げん, がん	もと	元	
\\	(げん). 
\\	外	
\\	がい	そと, はず	夕, ト	
\\	(がい) 
\\	分	
\\	ぶん, ふん, ぶ	わ	ハ, 刀	
\\	(ぶん) 
\\	(ふん) 
\\	(ぶん) 
\\	ふん 
\\	公	
\\	こう	
\\	ハ, ム	
\\	こういち
\\	こう, 
\\	こういち, 
\\	こういち 
\\	こういち
\\	こういち--
\\	こういち--
\\	引	
\\	いん	ひ	弓		
\\	(ひ) 
\\	止	
\\	し	と, や	止	
\\	(し) 
\\	用	
\\	よう	もち	用	
\\	(よう). 
\\	万	
\\	まん, ばん	
\\	刀		
\\	(まん)! 
\\	広	
\\	こう	ひろ	广, ム	
\\	(ひろ)!	
\\	冬	
\\	とう	ふゆ	夂, 二	
\\	夂 
\\	(ふゆ).	
\\	ふゆ ふゆ ふゆ...	
\\	竹	
\\	たけ	竹	
\\	(たけ) 
\\	車	
\\	しゃ	くるま	車	
\\	しゃ, 
\\	(しゃ)? 
\\	央	
\\	おう	
\\	大, 冂	
\\	王
\\	(おう) 
\\	王 
\\	写	
\\	しゃ	うつ	冖, 一		
\\	しゃ, 
\\	(しゃ) 
\\	仕	
\\	し	つか	イ, 士	
\\	(し). 
\\	耳	
\\	じ	みみ	耳	
\\	(みみ) 
\\	早	
\\	そう	はや, さ	日, 十	
\\	そう 
\\	(そう). 
\\	気	
\\	き, け	いき	气, メ	
\\	(き) 
\\	平	
\\	へい, ひょう, びょう	たいら, ひら	干		
\\	(へい). 
\\	花	
\\	か, け	はな	艹, イ, 匕	
\\	(はな) 
\\	足	
\\	そく	あし, た	足	
\\	(そく).	
\\	打	
\\	だ	う, ぶ	扌, 丁	
\\	(だ) 
\\	百	
\\	ひゃく	もも	日		
\\	(ひゃく)!
\\	氷	
\\	ひょう	こおり	水, 丶	
\\	(こおり)? 
\\	虫	
\\	ちゅう, き	むし	虫	
\\	(むし) 
\\	字	
\\	じ	
\\	宀, 子	
\\	(じ). 
\\	男	
\\	だん, なん	おとこ	田, 力	
\\	(だん). 
\\	主	
\\	しゅ	おも, ぬし	主	
\\	(しゅ) 
\\	名	
\\	めい, みょう	な	夕, 口	
\\	(めい). 
\\	不	
\\	ふ	
\\	不	
\\	ふ 
\\	(ふ).
\\	号	
\\	ごう	
\\	口, 一, 勹	
\\	(ごういち). 
\\	ごう 
\\	ごういち 
\\	こういち 
\\	こう 
\\	こう 
\\	ごう 
\\	ごういち.
\\	ごういち 
\\	ごういち 
\\	他	
\\	た	ほか	イ, 也	
\\	た, 
\\	(た) 
\\	去	
\\	きょ, こ	さ	土, ム	
\\	きょ, 
\\	きょ 
\\	きょう, 
\\	(きょ). 
\\	子 (こ) 
\\	子 
\\	皿	
\\	さら	皿	
\\	(さら). 
\\	先	
\\	せん	さき, まず	丿, 土, 儿	
\\	(せん)! 
\\	赤	
\\	せき	あか	赤	
\\	休	
\\	きゅう	やす	イ, 木	
\\	(きゅう) 
\\	申	
\\	しん	もう	十, 口	
\\	(もう).	
\\	見	
\\	けん	み	見	
\\	(み)!
\\	貝	
\\	かい	貝	
\\	(かい) 
\\	石	
\\	せき	いし	石	
\\	(せき).	
\\	代	
\\	だい	か, かわ, しろ	イ, 弋	
\\	(だい) 
\\	礼	
\\	れい	
\\	ネ, 乚	
\\	(れい) 
\\	糸	
\\	し	いと	糸	
\\	(いと) 
\\	町	
\\	ちょう	まち	田, 丁	
\\	(ちょう) 
\\	宝	
\\	ほう	たから	宀, 王, 丶	
\\	(ほう). 
\\	村	
\\	そん	むら	木, 寸	
\\	(むら) 
\\	世	
\\	せ, せい	よ	世	
\\	(せい). 
\\	年	
\\	ねん	とし	牛		
\\	(ねん). 
\\	角	
\\	かく	かど, つの	勹, 用	
\\	(かく) 
\\	斤	
\\	きん	
\\	斤	
\\	(きん) 
\\	青	
\\	せい, しょう	あお	青	
\\	(しょう) 
\\	(せい). 
\\	体	
\\	たい	からだ	イ, 本	
\\	(たい). 
\\	色	
\\	しき, しょく	いろ	色	
\\	(いろ) 
\\	来	
\\	らい	く	一, 米	
\\	(らい) 
\\	社	
\\	しゃ	やしろ	ネ, 土	
\\	(しゃ), 
\\	当	
\\	とう	あ	ヨ		
\\	とう, 
\\	とうきょう!
\\	とうきょう 
\\	図	
\\	ず, と	え, はか	口, メ, 冫	
\\	(ず).	
\\	毎	
\\	まい	ごと	毋		
\\	(まい) 
\\	羽	
\\	う	はね, は, わ	羽	
\\	はね 
\\	林	
\\	りん	はやし	木	
\\	行	
\\	こう, ぎょう	い, おこな, ゆ	行	
\\	こういち. 
\\	(ぎょう).
\\	金	
\\	きん	かね	金	
\\	(きん) 
\\	草	
\\	そう	くさ	艹, 日, 十	
\\	(くさ) 
\\	里	
\\	り	さと	里	
\\	(さと). 
\\	作	
\\	さく, さ	つく	イ, 乍	
\\	(さく) 
\\	さ. 
\\	さく, 
\\	多	
\\	た	おお	夕	
\\	た 
\\	肉	
\\	にく	
\\	肉	
\\	(にく) 
\\	会	
\\	かい	あ	会	
\\	(かい)! 
\\	交	
\\	こう	まじ, ま, か	亠, 父	
\\	こういち. 
\\	こういち 
\\	近	
\\	きん, こん	ちか	斤		
\\	(きん) 
\\	兄	
\\	きょう	あに	口, 儿	
\\	きょうと. きょうと 
\\	きょうと 
\\	雨	
\\	う	あめ, あま	雨	
\\	(あめ) 
\\	あめ 
\\	あめ, 
\\	米	
\\	べい, まい	こめ	米	
\\	(べい), 
\\	走	
\\	そう	はし	走	
\\	(そう) 
\\	同	
\\	どう	おな	冋, 一	
\\	どう, 
\\	(どう) 
\\	言	
\\	げん, ごん	い, こと	言	
\\	(げん) 
\\	自	
\\	じ, し	
\\	自	
\\	じ 
\\	(じ) 
\\	形	
\\	けい, ぎょう	かた, かたち	开, 彡	
\\	(けい)! 
\\	皮	
\\	ひ	かわ	丿, 支	
\\	(ひ) 
\\	空	
\\	くう	そら, あ, から, す	宀, 儿, 工	
\\	(くう) 
\\	くう.
\\	音	
\\	おん	おと, ね	立, 日	
\\	(おん)!	
\\	学	
\\	がく	まな	子		
\\	(がく) 
\\	光	
\\	こう	ひかり, ひか	兀		
\\	(こういち)
\\	こういち
\\	こういち
\\	こういち
\\	考	
\\	こう	かんが	耂		
\\	こういち! 
\\	こういち 
\\	こういち 
\\	こういち. 
\\	こういち 
\\	回	
\\	かい	まわ	口	
\\	(かい) 
\\	谷	
\\	こく	たに	ハ, 口		
\\	(たに). 
\\	声	
\\	せい	こえ	士, 尸		
\\	(こえ) 
\\	西	
\\	せい, さい	にし	西	
\\	(せい) 
\\	何	
\\	か	なに, なん	イ, 可	
\\	(なん) 
\\	(なに) 
\\	なん), 
\\	なに). 
\\	麦	
\\	ばく	むぎ	生, 夂	
\\	(むぎ).	
\\	むぎ 
\\	むぎ.	
\\	弟	
\\	だい, で, てい	おとうと	弓, 丿	
\\	(だい). 
\\	(で) 
\\	全	
\\	ぜん	すべ, まった	王		
\\	(ぜん) 
\\	後	
\\	ご, こう	うし, あと, のち	彳, 幺, 夂	
\\	(ご) 
\\	こういち 
\\	こういち 
\\	血	
\\	けつ	ち	血	
\\	けつ, 
\\	(けつ).
\\	両	
\\	りょう	
\\	一, 冂, 山	
\\	(りょう) 
\\	明	
\\	めい	あ, あか, あき	日, 月	
\\	(めい).	
\\	京	
\\	きょう	みやこ	亠, 口, 小	
\\	きょう 
\\	きょうと. 
\\	きょうと!	
\\	化	
\\	か	ば	イ, 匕	
\\	(か).	
\\	か, 
\\	国	
\\	こく	くに	口, 王, 丶	
\\	(こく). 
\\	科	
\\	か	
\\	禾, 斗	
\\	(か). 
\\	死	
\\	し	し	歹, 匕	
\\	(し). 
\\	亡	
\\	ぼう	な	亡	
\\	(ぼう) 
\\	画	
\\	が, かく	
\\	田, 凵		
\\	(が) 
\\	(かく). 
\\	地	
\\	ち, じ	
\\	土, 也	
\\	(ち). 
\\	東	
\\	とう	ひがし	木, 日	
\\	とうきょう. 
\\	とうきょう 
\\	とうきょう. 
\\	とうきょう 
\\	食	
\\	しょく	た, く	食	
\\	(しょく) 
\\	直	
\\	ちょく, じき	なお	
\\	十, 目	
\\	(ちょく) 
\\	(じき). 
\\	前	
\\	ぜん	まえ	一, 月, 刂		
\\	(ぜん) 
\\	有	
\\	ゆう, う	あ	ナ, 月	
\\	(ゆう).	
\\	私	
\\	し	わたし	禾, ム	
\\	(し) 
\\	知	
\\	ち	し	矢, 口	
\\	ち 
\\	(ち) 
\\	活	
\\	かつ	い	氵, 舌	
\\	(かつ) 
\\	長	
\\	ちょう	なが	長	
\\	(ちょう), 
\\	長 
\\	ちょう.	
\\	曲	
\\	きょく	ま	口, 廾	
\\	(きょく) 
\\	首	
\\	しゅ	くび	首	
\\	(くび). 
\\	次	
\\	じ	つぎ	冫, 欠	
\\	じ, 
\\	(じ), 
\\	夜	
\\	や	よ, よる	亠, イ, 夂, 丶	
\\	(や) 
\\	姉	
\\	し	お, あね, ねえ	女, 市	
\\	(し) 
\\	点	
\\	てん	つ	占, 灬	
\\	(てん) 
\\	安	
\\	あん	やす	宀, 女	
\\	(あん).	
\\	室	
\\	しつ	
\\	宀, 一, ム, 土	
\\	(しつ). 
\\	海	
\\	かい	うみ	氵, 毋		
\\	(かい) 
\\	羊	
\\	よう	ひつじ	羊	
\\	よう, 
\\	(よう). 
\\	店	
\\	てん	みせ	广, 占	
\\	(てん) 
\\	南	
\\	なん	みなみ	十, 冂, 干		
\\	(なん).	
\\	星	
\\	せい	ほし	日, 生	
\\	(せい). 
\\	州	
\\	しゅう	
\\	丶, 川	
\\	(しゅう) 
\\	茶	
\\	ちゃ, さ	
\\	艹, ホ		
\\	思	
\\	し	おも	田, 心	
\\	(し). 
\\	歩	
\\	ほ	ある, あゆ	止, 小, 丿	
\\	(ほ) 
\\	向	
\\	こう	む	丶, 冂, 口	
\\	こういち.	
\\	(こういち),
\\	妹	
\\	まい	いもうと	女, 未	
\\	(まい) 
\\	辺	
\\	へん	あた, べ	刀		
\\	(へん) 
\\	付	
\\	ふ	つ	イ, 寸	
\\	つ. 
\\	(つ) 
\\	札	
\\	さつ	ふだ	木, 乚	
\\	(さつ) 
\\	鳥	
\\	ちょう	とり	鳥	
\\	(ちょう).	
\\	黒	
\\	こく	くろ	黒	
\\	(こく)! 
\\	船	
\\	せん	ふね	舟, ハ, 口	
\\	(せん). 
\\	必	
\\	ひつ	かなら	心, 丿	
\\	(ひつ) 
\\	末	
\\	まつ	すえ	未	
\\	(まつ) 
\\	氏	
\\	し	うじ	氏	
\\	(し). 
\\	失	
\\	しつ	うしな	丿, 夫	
\\	(しつ) 
\\	魚	
\\	ぎょ	さかな	魚	
\\	(ぎょ). 
\\	以	
\\	い	
\\	丶, 人	
\\	(い) 
\\	組	
\\	そ	くみ	糸, 且	
\\	(そ) 
\\	家	
\\	か, け	いえ, や, うち	宀, 豕	
\\	(か). 
\\	欠	
\\	けつ	か	欠	
\\	けつ, 
\\	(けつ) 
\\	未	
\\	み	ま, いま, ひつじ	未	
\\	(み) 
\\	紙	
\\	し	かみ	糸, 氏	
\\	(かみ), 
\\	通	
\\	つう	とお, かよ	マ, 用		
\\	(つう)!
\\	民	
\\	みん	たみ	口, 氏	
\\	(みん) 
\\	理	
\\	り	ことわり	王, 里	
\\	(り).	
\\	由	
\\	ゆう	よし, よ	十, 口	
\\	(ゆう) 
\\	校	
\\	こう	
\\	木, 亠, 父	
\\	こういち
\\	こういち
\\	こういち
\\	こういち
\\	雪	
\\	せつ	ゆき	雨, ヨ	
\\	(ゆき).
\\	ゆき.
\\	強	
\\	きょう	つよ	弓, ム, 虫	
\\	きょうと. きょうと 
\\	きょうと 
\\	夏	
\\	げ, か, が	なつ	目, 夂		
\\	(なつ) 
\\	高	
\\	こう	たか	亠, 口, 冋	
\\	こういち
\\	こういち 
\\	こういち
\\	こういち
\\	教	
\\	きょう	おし, おそ	孝, 夂	
\\	きょうと. きょうと 
\\	きょうと 
\\	きょうと!	
\\	時	
\\	じ	とき	日, 寺	
\\	(じ). 
\\	弱	
\\	じゃく	よわ	弓, 冫	
\\	(じゃく) 
\\	週	
\\	しゅう	
\\	周		
\\	風	
\\	ふう, ふ	かぜ	几, 丿, 虫	
\\	(ふう). 
\\	記	
\\	き	しる	言, 己	
\\	(き). 
\\	黄	
\\	おう	き	黄	
\\	(き) 
\\	答	
\\	とう	こた	竹, 合	
\\	(こた) 
\\	反	
\\	はん	
\\	厂, 又	
\\	(はん) 
\\	君	
\\	くん	きみ	ヨ, 丿, 口	
\\	局	
\\	きょく	
\\	尸, 口		
\\	(きょく) 
\\	きょく 
\\	きょく 
\\	買	
\\	ばい	か	罒, 貝	
\\	(か). 
\\	雲	
\\	うん	くも	雨, 云	
\\	(くも). 
\\	楽	
\\	らく, がく	たの	白, 冫, 木	
\\	(らく) 
\\	数	
\\	すう	かぞ, かず	米, 女, 夂	
\\	決	
\\	けつ	き	氵, 人, ユ	
\\	けつ 
\\	(けつ).
\\	絵	
\\	え	
\\	糸, 会	
\\	(え)
\\	住	
\\	じゅう	す	イ, 主	
\\	(じゅう). 
\\	電	
\\	でん	
\\	雨, 田, 乚	
\\	森	
\\	しん	もり	木	
\\	(もり) 
\\	助	
\\	じょ	たす, すけ	且, 力	
\\	(じょ).	
\\	じょ 
\\	(じょう), 
\\	馬	
\\	ば	うま	馬	
\\	(ば) 
\\	間	
\\	かん, けん	あいだ, ま	門, 日	
\\	(かん). 
\\	場	
\\	じょう	ば	土, 易	
\\	(じょう), 
\\	上 
\\	医	
\\	い	
\\	匚, 矢	
\\	い 
\\	(い) 
\\	朝	
\\	ちょう	あさ	十, 日, 月	
\\	(あさ) 
\\	番	
\\	ばん	
\\	釆, 田	
\\	所	
\\	しょ	ところ	戸, 斤	
\\	(しょ). 
\\	池	
\\	ち	いけ	也, 氵	
\\	(ち). 
\\	究	
\\	きゅう	きわ	穴, 九	
\\	(きゅう)!	
\\	道	
\\	どう	みち	首		
\\	どう 
\\	(どう) 
\\	役	
\\	やく	
\\	彳, 殳	
\\	(やく) 
\\	研	
\\	けん	と	石, 开	
\\	(けん), 
\\	身	
\\	しん	み	自, 丿	
\\	(しん). 
\\	者	
\\	しゃ	もの	耂, 日	
\\	しゃ 
\\	(しゃ), 
\\	支	
\\	し	ささ	支	
\\	(し) 
\\	話	
\\	わ	はな, はなし	言, 舌	
\\	(わ) 
\\	投	
\\	とう	な	扌, 殳	
\\	とうきょう 
\\	とうきょう. 
\\	対	
\\	たい	
\\	文, 寸	
\\	(たい). 
\\	受	
\\	じゅ	う	冖, 又		
\\	(じゅ) 
\\	事	
\\	じ	こと, つか	十, 口, 聿	
\\	(じ) 
\\	美	
\\	び, み	うつく	王, 大		
\\	(び). 
\\	予	
\\	よ	あらかじ	マ, 丁	
\\	(よ) 
\\	服	
\\	ふく	
\\	月, 卩, 又	
\\	(ふく) 
\\	度	
\\	ど, たく	たび	又		
\\	(ど).	
\\	発	
\\	はつ	
\\	癶, 开	
\\	(はつ) 
\\	定	
\\	てい, じょう	さだ	宀, 正	
\\	(てい) 
\\	談	
\\	だん	
\\	言, 火	
\\	(だん), 
\\	表	
\\	ひょう	あらわ, おもて	生		
\\	(ひょう) 
\\	ひょう, 
\\	客	
\\	きゃく	
\\	宀, 夂, 口	
\\	(きゃく) 
\\	重	
\\	じゅう, ちょう	おも, かさ, え	重	
\\	(じゅう). 
\\	持	
\\	じ	も	扌, 寺	
\\	(じ). 
\\	負	
\\	ふ	ま, お	勹, 貝	
\\	(ふ). 
\\	相	
\\	そう, しょう	あい	木, 目	
\\	(そう). 
\\	要	
\\	よう	い, かなめ	覀, 女	
\\	(よう)! 
\\	新	
\\	しん	あたら, あら, にい	立, 木, 斤	
\\	部	
\\	ぶ	へ	咅, 阝	
\\	(ぶ). 
\\	和	
\\	わ, お	なご, やわ	禾, 口	
\\	(わ) 
\\	県	
\\	けん	
\\	目, 
\\	小	
\\	(けん) 
\\	保	
\\	ほ	たも	イ, 口, 木	
\\	(ほ) 
\\	返	
\\	へん	かえ	反		
\\	(へん)!	
\\	乗	
\\	じょう	の	禾, 口	
\\	(の) 
\\	屋	
\\	おく	や	尸, 一, ム, 土	
\\	(や) 
\\	売	
\\	ばい	う	士, 冖, 儿	
\\	(ばい)!
\\	送	
\\	そう	おく	天	
\\	(そう) 
\\	苦	
\\	く	くる, にが	艹, 古	
\\	(く).	
\\	泳	
\\	えい	およ	氵, 永	
\\	(およ)! 
\\	仮	
\\	か	かり	イ, 反	
\\	(か). 
\\	験	
\\	けん	ため, ためし	馬		
\\	(けん) 
\\	(けん) 
\\	物	
\\	ぶつ, もつ	もの	牛, 勿	
\\	(ぶつ). 
\\	具	
\\	ぐ	
\\	目, 一, ハ	
\\	(ぐ) 
\\	実	
\\	じつ	み	宀		
\\	(じつ) 
\\	試	
\\	し	こころ, ため	言, 弋, 工	
\\	(し) 
\\	使	
\\	し	つか	イ, 一, 史	
\\	(し). 
\\	勝	
\\	しょう	か	月, 力		
\\	(しょう) 
\\	界	
\\	かい	
\\	田, 介	
\\	(かい).	
\\	進	
\\	しん	すす	隹		
\\	(しん) 
\\	酒	
\\	しゅ	さけ, さか	氵, 酉	
\\	(しゅ), 
\\	始	
\\	し	はじ	女, 台	
\\	(し) 
\\	業	
\\	ぎょう	
\\	羊, ハ		
\\	(ぎょう) 
\\	算	
\\	さん	そろ	竹, 目, 廾	
\\	(さん) 
\\	運	
\\	うん	はこ	冖, 車		
\\	(うん). 
\\	漢	
\\	かん	
\\	氵		
\\	(かん) 
\\	鳴	
\\	めい	な	口, 鳥	
\\	(な). 
\\	集	
\\	しゅう	あつ	隹, 木	
\\	(しゅう). 
\\	配	
\\	はい	くば	酉, 己	
\\	(はい): 
\\	飲	
\\	いん	の	食, 欠	
\\	(の) 
\\	終	
\\	しゅう	おわ, お	糸, 夂, 二	
\\	(しゅう). 
\\	顔	
\\	がん	かお	立, 厂, 彡, 頁	
\\	(かお). 
\\	落	
\\	らく	お	艹, 氵, 各	
\\	(らく) 
\\	農	
\\	のう	
\\	曲, 辰	
\\	(のう), 
\\	速	
\\	そく	はや	束		
\\	(そく). 
\\	頭	
\\	ず, とう	あたま	豆, 頁	
\\	(あたま) 
\\	聞	
\\	ぶん, もん	き	門, 耳	
\\	(ぶん).” 
\\	院	
\\	いん	
\\	阝, 宀, 元	
\\	(いん)!	
\\	調	
\\	ちょう	しら	言, 周	
\\	(ちょう)!
\\	鉄	
\\	てつ	
\\	金, 失	
\\	(てつ) 
\\	語	
\\	ご	かた	言, 五, 口	
\\	(ご) 
\\	葉	
\\	よう	は, ば	艹, 世, 木	
\\	(は)!
\\	は 
\\	習	
\\	しゅう	なら	羽, 白	
\\	(しゅう). 
\\	軽	
\\	けい	かる, かろ	車, 圣	
\\	(かる). 
\\	線	
\\	せん	
\\	糸, 白, 水	
\\	(せん). 
\\	最	
\\	さい	もっと	日, 耳, 又	
\\	(さい). 
\\	開	
\\	かい	あ, ひら	門, 开	
\\	(かい) 
\\	親	
\\	しん	おや, した	立, 木, 見	
\\	(しん) 
\\	読	
\\	とう, どく	よ	言, 売	
\\	(よ) 
\\	求	
\\	きゅう	もと	一, 水, 丶	
\\	(きゅう).	
\\	転	
\\	てん	ころ	車, 云	
\\	(てん) 
\\	路	
\\	ろ	じ, みち	足, 各	
\\	(ろ). 
\\	病	
\\	びょう	や, やまい	疒, 丙	
\\	(びょう). 
\\	横	
\\	おう	よこ	木, 黄	
\\	(よこ) 
\\	歌	
\\	か	うた	可, 欠	
\\	(か).	
\\	起	
\\	き	お	走, 己	
\\	(お) 
\\	功	
\\	こう	
\\	工, 力	
\\	こういち, 
\\	こういち. 
\\	工 
\\	こう 
\\	成	
\\	せい	な	戈, 刀	
\\	(せい)! 
\\	岸	
\\	がん	きし	山, 厂, 干	
\\	(がん). 
\\	競	
\\	きょう	きそ	立, 兄	
\\	きょうと 
\\	きょうと, 
\\	きょうと... 
\\	争	
\\	そう	あらそ	勹, ヨ, 亅	
\\	(そう). 
\\	便	
\\	べん, びん	たよ	イ, 一, 田, メ	
\\	(べん). 
\\	老	
\\	ろう	
\\	耂, 匕	
\\	(ろう). 
\\	命	
\\	めい, みょう	いのち	令, 口	
\\	(めい). 
\\	指	
\\	し	ゆび, さ	扌, 匕, 日	
\\	(し) 
\\	初	
\\	しょ	はじ, はつ, そ, ぞ	ネ, 刀	
\\	(しょ) 
\\	味	
\\	み	あじ	口, 未	
\\	(み)? 
\\	追	
\\	つい	お		
\\	(お). 
\\	神	
\\	しん	かみ	ネ, 申	
\\	(しん). 
\\	良	
\\	りょう	よ, い	丶, 艮	
\\	(りょう). 
\\	意	
\\	い	
\\	音, 心	
\\	(い) 
\\	労	
\\	ろう	いたわ	力		
\\	(ろう) 
\\	級	
\\	きゅう	
\\	糸, 及	
\\	(きゅう).	
\\	好	
\\	こう	す, この	女, 子	
\\	こういち
\\	こういち
\\	こういち 
\\	こういち 
\\	昔	
\\	むかし	日		
\\	(むかし). 
\\	低	
\\	てい	ひく	イ, 氏, 一	
\\	(てい) 
\\	育	
\\	いく	そだ, はぐく	月		
\\	""行く (いく), 行く!
\\	行く
\\	行く 
\\	令	
\\	れい	
\\	令	
\\	(れい) 
\\	れい. 
\\	拾	
\\	ひろ	扌, 合	
\\	(ひろ)! 
\\	注	
\\	ちゅう	そそ, さ, つ	氵, 主	
\\	(ちゅう) 
\\	利	
\\	り	き	禾, 刂	
\\	(り) 
\\	位	
\\	い	くらい	イ, 立	
\\	(い) 
\\	仲	
\\	ちゅう	なか	イ, 中	
\\	(なか) 
\\	放	
\\	ほう	はな, ほう	方, 夂	
\\	(ほう). 
\\	秒	
\\	びょう	
\\	禾, 少	
\\	(びょう) 
\\	別	
\\	べつ	わか	口, 刀, 刂	
\\	(べつ) 
\\	特	
\\	とく	
\\	牛, 寺	
\\	(とく) 
\\	共	
\\	きょう	とも	ハ		
\\	きょうと 
\\	きょうと 
\\	努	
\\	ど	つと	女, 又, 力	
\\	(ど). 
\\	伝	
\\	でん	つた, つて	イ, 云	
\\	(でん) 
\\	戦	
\\	せん	たたか	単, 戈	
\\	(ツ)_/
\\	(せん). 
\\	波	
\\	は	なみ	氵, 皮	
\\	(は)!
\\	洋	
\\	よう	
\\	氵, 羊	
\\	(よう). 
\\	働	
\\	どう	はたら	イ, 重, 力	
\\	(どう). 
\\	悪	
\\	あく, お	わる	亜, 心	
\\	(あく).	
\\	息	
\\	そく	いき	自, 心	
\\	(そく) 
\\	章	
\\	しょう	
\\	立, 早	
\\	(しょう). 
\\	登	
\\	とう, と	のぼ	癶, 豆	
\\	とうきょう! 
\\	とうきょう. 
\\	とうきょう 
\\	とうきょう 
\\	とうきょう 
\\	寒	
\\	かん	さむ	宀, 冫		
\\	深	
\\	しん	ふか	氵, 兀, 木	
\\	(しん) 
\\	倍	
\\	ばい	
\\	イ, 咅	
\\	(ばい). 
\\	勉	
\\	べん	
\\	免, 力	
\\	(べん) 
\\	消	
\\	しょう	き, け	氵, 月		
\\	(しょう). 
\\	祭	
\\	さい	まつり, まつ	癶, 示	
\\	(さい) 
\\	野	
\\	や	の	里, 予	
\\	(や) 
\\	階	
\\	かい	
\\	阝, 比, 白	
\\	(かい) 
\\	庭	
\\	てい	にわ	广, 廴, 王	
\\	(てい) 
\\	港	
\\	こう	みなと	氵, 共, 己	
\\	こういち! 
\\	こういち 
\\	暑	
\\	しょ	あつ	日, 者	
\\	(あつ) 
\\	湯	
\\	とう	ゆ	氵, 易	
\\	(ゆ) 
\\	僕	
\\	ぼく	
\\	イ, 業	
\\	(ぼく) 
\\	ぼく 
\\	ぼく 
\\	ぼく.	
\\	島	
\\	とう	しま	鳥, 山	
\\	(しま), 
\\	童	
\\	どう	
\\	立, 里	
\\	(どう). 
\\	員	
\\	いん	
\\	口, 貝	
\\	(いん). 
\\	商	
\\	しょう	あきな	立, 冋, 儿	
\\	(しょう) 
\\	都	
\\	と, つ	みやこ	者, 阝	
\\	(と). 
\\	動	
\\	どう	うご	重, 力	
\\	(どう). 
\\	第	
\\	だい	
\\	竹, 弟	
\\	(だい) 
\\	期	
\\	き	
\\	其, 月	
\\	(き). 
\\	植	
\\	しょく	う	木, 直	
\\	(しょく). 
\\	根	
\\	こん	ね	木, 艮	
\\	ね.	
\\	(こん). 
\\	短	
\\	たん	みじか	矢, 豆	
\\	(たん). 
\\	球	
\\	きゅう	たま	王, 求	
\\	(きゅう) 
\\	泉	
\\	せん	いずみ	白, 水	
\\	(せん). 
\\	流	
\\	りゅう, る	なが	氵, 川		
\\	(りゅう) 
\\	合	
\\	ごう, がっ	あ, あい	合	
\\	ごういち.	
\\	ごういち 
\\	陽	
\\	よう	ひ	阝, 易	
\\	(よう) 
\\	歯	
\\	し	は	止, 凵, 米	
\\	(は)! 
\\	族	
\\	ぞく	
\\	方, 矢		
\\	(ぞく) 
\\	旅	
\\	りょ	たび	方	
\\	(りょ)! 
\\	待	
\\	たい	ま	彳, 寺	
\\	(たい). 
\\	温	
\\	おん	あたた	氵, 日, 皿	
\\	(おん) 
\\	着	
\\	ちゃく	き, つ	王, 丿, 目		
\\	(ちゃく) 
\\	皆	
\\	かい	みな, みんな	比, 白	
\\	(みな). 
\\	謝	
\\	しゃ	あやま	言, 身, 寸	
\\	(しゃ). 
\\	整	
\\	せい	ととの	束, 夂, 正	
\\	(せい) 
\\	橋	
\\	きょう	はし	木, 天, 口, 冋	
\\	(はし) 
\\	選	
\\	せん	えら	己, 共		
\\	(せん). 
\\	想	
\\	そう	
\\	木, 目, 心	
\\	(そう) 
\\	器	
\\	き	うつわ	大, 品	
\\	(き). 
\\	暗	
\\	あん	くら, くれ	日, 立	
\\	(あん). 
\\	疑	
\\	ぎ	うたが	匕, マ, 矢, 疋	
\\	(ぎ)! 
\\	料	
\\	りょう	
\\	米, 斗	
\\	(りょう) 
\\	感	
\\	かん	
\\	丿, 戈, 心		
\\	(かん) 
\\	情	
\\	じょう	なさけ	忄, 青	
\\	(じょう) 
\\	様	
\\	よう	さま	木, 羊, 水	
\\	(さま). 
\\	養	
\\	よう	やしな	羊, 良, ハ	
\\	(よう). 
\\	緑	
\\	りょく	みどり	糸, ヨ, 水	
\\	(みどり) 
\\	熱	
\\	ねつ	あつ	坴, 丸, 灬	
\\	(ねつ) 
\\	億	
\\	おく	
\\	イ, 意	
\\	(おく). 
\\	殺	
\\	さつ	ころ	メ, 木, 殳	
\\	(さつ) 
\\	宿	
\\	しゅく	やど	宀, イ, 百	
\\	(しゅく) 
\\	福	
\\	ふく	
\\	ネ, 畐	
\\	(ふく) 
\\	鏡	
\\	きょう	かがみ	金, 立, 見	
\\	(かがみ). 
\\	然	
\\	ぜん, ねん	しか, さ	月, 犬, 灬	
\\	(ぜん) 
\\	詩	
\\	し	うた	言, 寺	
\\	(し) 
\\	練	
\\	れん	ね	糸, 東	
\\	(れん), 
\\	賞	
\\	しょう	
\\	口, 貝		
\\	(しょう). 
\\	問	
\\	もん	と, とん	門, 口	
\\	(もん). 
\\	館	
\\	かん	
\\	食, 宀		
\\	(かん). 
\\	映	
\\	えい	うつ, は	日, 央	
\\	(えい), 
\\	願	
\\	がん	ねが, ねがい	原, 頁	
\\	(がん). 
\\	士	
\\	し	さむらい	士	
\\	(し) 
\\	課	
\\	か	
\\	言, 果	
\\	(か). 
\\	標	
\\	ひょう	しるし	木, 覀, 示	
\\	(ひょう)!!	
\\	銀	
\\	ぎん	
\\	金, 艮	
\\	(ぎん) 
\\	駅	
\\	えき	
\\	馬, 尺	
\\	(えき) 
\\	像	
\\	ぞう	
\\	イ, 象	
\\	(ぞう). 
\\	題	
\\	だい	
\\	日, 正, 頁	
\\	(だい) 
\\	輪	
\\	りん	わ	車, 一, 冊		
\\	(りん) 
\\	能	
\\	のう	
\\	ム, 月, 匕	
\\	(のう)! 
\\	芸	
\\	げい	
\\	艹, 云	
\\	(げい). 
\\	術	
\\	じゅつ	
\\	行, ホ, 丶	
\\	(じゅつ)!!!	
\\	雰	
\\	ふん	
\\	雨, 分	
\\	分
\\	分, 
\\	(ふん), 
\\	骨	
\\	こつ	ほね	冋, 冖, 月	
\\	(こつ). 
\\	束	
\\	そく	たば	束	
\\	(そく). 
\\	周	
\\	しゅう	まわ	周	
\\	(しゅう). 
\\	協	
\\	きょう	
\\	十, 力	
\\	きょうと. 
\\	きょうと 
\\	きょうと. 
\\	きょうと!	
\\	例	
\\	れい	たと	イ, 歹, 刂	
\\	(れい) 
\\	折	
\\	せつ	お	扌, 斤	
\\	(せつ), 
\\	基	
\\	き	もと	其, 土	
\\	(き) 
\\	性	
\\	せい, しょう	
\\	忄, 生	
\\	(しょう) 
\\	(せい) 
\\	妥	
\\	だ	
\\	女		
\\	(だ). 
\\	卒	
\\	そつ	
\\	亠, 人, 十	
\\	(そつ) 
\\	固	
\\	こ	かた	口, 古	
\\	子 (こ) 
\\	子 
\\	望	
\\	ぼう	のぞ	亡, 月, 王	
\\	(ぼう) 
\\	材	
\\	ざい	
\\	木, 才	
\\	(ざい)! 
\\	参	
\\	さん, しん	まい	ム, 大, 彡	
\\	(さん)! 
\\	完	
\\	かん	
\\	宀, 元	
\\	(かん). 
\\	松	
\\	しょう	まつ	木, 公	
\\	(まつ) 
\\	約	
\\	やく	
\\	糸, 勺	
\\	(やく) 
\\	残	
\\	ざん	のこ	歹		
\\	(ざん).	
\\	季	
\\	き	
\\	禾, 子	
\\	(き). 
\\	技	
\\	ぎ	わざ	扌, 支	
\\	(ぎ). 
\\	格	
\\	かく	
\\	木, 各	
\\	(かく), 
\\	頑	
\\	がん	
\\	元, 頁	
\\	(がん). 
\\	囲	
\\	い	かこ	口, 井	
\\	(い) 
\\	的	
\\	てき	まと	白, 勺	
\\	(てき) 
\\	念	
\\	ねん	
\\	今, 心	
\\	年 (ねん) 
\\	年 
\\	希	
\\	き	まれ	メ, ナ, 巾	
\\	(き). 
\\	紀	
\\	き	
\\	糸, 己	
\\	(き). 
\\	軍	
\\	ぐん	
\\	冖, 車	
\\	(ぐん) 
\\	秋	
\\	あき	禾, 火	
\\	(あき)!
\\	信	
\\	しん	
\\	イ, 言	
\\	(しん). 
\\	岩	
\\	がん	いわ	山, 石	
\\	仏	
\\	ぶつ	ほとけ	イ, ム	
\\	(ぶつ). 
\\	建	
\\	けん	た	廴, 聿	
\\	(けん) 
\\	猫	
\\	ねこ	犭, 艹, 田	
\\	(ねこ). 
\\	変	
\\	へん	か	赤, 夂	
\\	(へん). 
\\	晴	
\\	せい	は	日, 青	
\\	(は)! 
\\	築	
\\	ちく	きず	竹, 木		
\\	(ちく) 
\\	勇	
\\	ゆう	いさ	マ, 男	
\\	(ゆう) 
\\	泣	
\\	きゅう	な	氵, 立	
\\	(な). 
\\	司	
\\	し	つかさど		
\\	(し). 
\\	区	
\\	く	
\\	匚, メ	
\\	(く) 
\\	英	
\\	えい	
\\	艹, 央	
\\	(えい) 
\\	丈	
\\	じょう	たけ	丈	
\\	(じょう). 
\\	夫	
\\	ふう, ふ	おっと	夫	
\\	(ふう)! 
\\	飯	
\\	はん	めし	食, 反	
\\	(はん) 
\\	計	
\\	けい	はか	言, 十	
\\	(けい). 
\\	法	
\\	ほう	
\\	氵, 去	
\\	(ほう) 
\\	晩	
\\	ばん	
\\	日, 免	
\\	(ばん) 
\\	昼	
\\	ちゅう	ひる	尺, 日, 一	
\\	(ひる) 
\\	毒	
\\	どく	
\\	生, 毋	
\\	(どく) 
\\	昨	
\\	さく	
\\	日, 乍	
\\	(さく) 
\\	帰	
\\	き	かえ	刂, ヨ, 冖, 巾	
\\	(かえ) 
\\	式	
\\	しき	
\\	弋, 工	
\\	(しき) 
\\	列	
\\	れつ	
\\	歹, 刂	
\\	(れつ) 
\\	浅	
\\	せん	あさ	氵		
\\	(あさ), 
\\	単	
\\	たん	
\\	単	
\\	(たん). 
\\	坂	
\\	はん	さか	土, 反	
\\	(さか). 
\\	春	
\\	しゅん	はる	日		
\\	(はる) 
\\	寺	
\\	じ	てら	寺	
\\	(てら) 
\\	浴	
\\	よく	あ	氵, 谷	
\\	(よく) 
\\	箱	
\\	はこ	竹, 木, 目	
\\	(はこ) 
\\	係	
\\	けい	かか, かかり	イ, 一, 糸	
\\	(けい). 
\\	治	
\\	じ, ち	なお	氵, 台	
\\	(じ) 
\\	(ち). 
\\	危	
\\	き	あぶ, あや	勹, 厂, 巳	
\\	(き). 
\\	冒	
\\	ぼう	おか	日, 目	
\\	(ぼう). 
\\	留	
\\	る, りゅう	と	ム, 刀, 田	
\\	(りゅう) 
\\	(る) 
\\	弁	
\\	べん	
\\	ム, 廾	
\\	(べん) 
\\	証	
\\	しょう	あかし	言, 正	
\\	(しょう), 
\\	存	
\\	そん, ぞん	
\\	ナ, 子		
\\	(そん). 
\\	(ぞん) 
\\	面	
\\	めん	おも, おもて	面	
\\	(めん). 
\\	遠	
\\	えん	とお		
\\	(えん). 
\\	園	
\\	えん	
\\	口		
\\	(えん).	
\\	門	
\\	もん	
\\	門	
\\	(もん). 
\\	府	
\\	ふ	
\\	广, 付	
\\	(ふ). 
\\	幸	
\\	こう	しあわ, さいわ, さち	亠, 辛	
\\	こういち, 
\\	こういち 
\\	こういち 
\\	阪	
\\	はん	さか	阝, 反	
\\	(はん) 
\\	急	
\\	きゅう	いそ	勹, ヨ, 心	
\\	(きゅう) 
\\	笑	
\\	しょう	わら, え	竹, 天	
\\	(わら) 
\\	荷	
\\	か	に	艹, イ, 可	
\\	(に). 
\\	政	
\\	せい	
\\	正, 夂	
\\	(せい). 
\\	品	
\\	ひん	しな	品	
\\	(ひん). 
\\	守	
\\	す, しゅ	まも, もり	宀, 寸	
\\	(す)! 
\\	辞	
\\	じ	や	舌, 辛	
\\	(じ). 
\\	真	
\\	しん	ま	一, ハ		
\\	(しん). 
\\	関	
\\	かん	かか, せき	門, 天		
\\	(かん) 
\\	険	
\\	けん	けわ	阝		
\\	(けん) 
\\	典	
\\	てん	
\\	曲, ハ	
\\	(てん) 
\\	専	
\\	せん	もっぱ	十, 田, 寸	
\\	(せん) 
\\	冗	
\\	じょう	
\\	冖, 几	
\\	(じょう), 
\\	取	
\\	しゅ	と	耳, 又	
\\	(しゅ)! 
\\	曜	
\\	よう	
\\	日, ヨ, 隹	
\\	(よう). 
\\	書	
\\	しょ	か	聿, 日	
\\	(しょ)
\\	是	
\\	ぜ	
\\	日, 正	
\\	(ぜ) 
\\	結	
\\	けつ	むす, ゆ	糸, 吉	
\\	(けつ) 
\\	底	
\\	てい	そこ	广, 氏, 一	
\\	(そこ)!	
\\	因	
\\	いん	よ	口, 大	
\\	(いん) 
\\	詳	
\\	しょう	くわ	言, 羊	
\\	(しょう). 
\\	識	
\\	しき	
\\	言, 音, 戈	
\\	(しき) 
\\	劇	
\\	げき	
\\	虍, 豕, 刂	
\\	(げき), 
\\	干	
\\	かん	ほ, ひ	干	
\\	(かん). 
\\	敗	
\\	はい	やぶ	貝, 夂	
\\	(はい) 
\\	渉	
\\	しょう	わた	氵, 歩	
\\	(しょう). 
\\	果	
\\	か	くだ, は	果	
\\	(か) 
\\	官	
\\	かん	
\\	宀		
\\	(かん). 
\\	署	
\\	しょ	
\\	罒, 者	
\\	(しょ). 
\\	察	
\\	さつ	
\\	宀, 祭	
\\	(さつ) 
\\	堂	
\\	どう	
\\	口, 土		
\\	(どう). 
\\	幻	
\\	げん	まぼろし	幺		
\\	(げん). 
\\	非	
\\	ひ	
\\	非	
\\	(ひ) 
\\	愛	
\\	あい	まな	冖, 心, 夂		
\\	(あい) 
\\	薬	
\\	やく	くすり	艹, 楽	
\\	(やく). 
\\	覚	
\\	かく	おぼ, さ	見		
\\	(かく) 
\\	鼻	
\\	び	はな	自, 田, 廾	
\\	花 (はな). 
\\	花 
\\	花 
\\	無	
\\	む, ぶ	な	無	
\\	(む) 
\\	常	
\\	じょう	つね	口, 巾		
\\	(じょう) 
\\	原	
\\	げん	はら	原	
\\	(げん), 
\\	栄	
\\	えい	さか	木		
\\	(えい). 
\\	喜	
\\	き	よろこ	壴, 口	
\\	(き) 
\\	恋	
\\	れん	こい	赤, 心	
\\	(れん) 
\\	悲	
\\	ひ	かな	非, 心	
\\	(ひ). 
\\	塩	
\\	えん	しお	土, 口, 皿		
\\	(しお) 
\\	席	
\\	せき	
\\	巾		
\\	(せき) 
\\	側	
\\	そく	がわ, そば	イ, 貝, 刂	
\\	川 (かわ), 
\\	がわ.	
\\	かわ 
\\	がわ 
\\	兵	
\\	へい, ひょう	
\\	斤, 一, ハ	
\\	(へい). 
\\	説	
\\	せつ	と	言, 兑	
\\	(せつ) 
\\	細	
\\	さい	ほそ, こま	糸, 田	
\\	(さい). 
\\	梅	
\\	ばい	うめ	木, 毋		
\\	(うめ) 
\\	虚	
\\	きょ, こ	むな	虍		
\\	(きょ). 
\\	警	
\\	けい	
\\	苟, 夂, 言	
\\	(けい). 
\\	告	
\\	こく	つ	丿, 土, 口	
\\	(こく). 
\\	達	
\\	たつ	たち	幸		
\\	(たつ) 
\\	焼	
\\	しょう	や	火, 尭	
\\	(や). 
\\	借	
\\	しゃく	か	イ, 昔	
\\	(しゃく) 
\\	弓	
\\	きゅう	ゆみ	弓	
\\	(きゅう). 
\\	脳	
\\	のう	
\\	月, 凶		
\\	(のう). 
\\	飴	
\\	あめ	食, 台	
\\	雨 (あめ) 
\\	雨. 
\\	胸	
\\	きょう	むね	月, 勹, 凶	
\\	(むね). 
\\	喫	
\\	きつ	の	口, 生, 刀, 大	
\\	(きつ) 
\\	等	
\\	とう	ひと, など	竹, 寺	
\\	とうきょう. 
\\	とうきょう 
\\	とうきょう 
\\	枚	
\\	まい	
\\	木, 夂	
\\	(まい) 
\\	忘	
\\	ぼう	わす	亡, 心	
\\	(ぼう). 
\\	訓	
\\	くん	よ	言, 川	
\\	(くん) 
\\	種	
\\	しゅ	たね, ぐさ	禾, 重	
\\	(しゅ). 
\\	報	
\\	ほう	むく	幸, 卩, 又	
\\	(ほう). 
\\	句	
\\	く	
\\	句	
\\	(く).	
\\	許	
\\	きょ	ゆる	言, 午	
\\	(きょ). 
\\	きょ 
\\	可	
\\	か	
\\	可	
\\	(か). 
\\	祈	
\\	き	いの	ネ, 斤	
\\	(き) 
\\	僧	
\\	そう	
\\	イ, 曽	
\\	(そう). 
\\	禁	
\\	きん	
\\	木, 示	
\\	(きん) 
\\	静	
\\	せい	しず	青, 争	
\\	(せい) 
\\	座	
\\	ざ	すわ	广, 人, 土	
\\	(ざ) 
\\	煙	
\\	えん	けむ, けむり	火, 覀, 土	
\\	(えん) 
\\	汽	
\\	き	
\\	氵, 气	
\\	(き). 
\\	叩	
\\	こう	たた	口, 卩	
\\	こういち 
\\	こういち 
\\	喉	
\\	こう	のど	口, イ, 矢, ユ	
\\	こういち! 
\\	こういち 
\\	類	
\\	るい	たぐ	米, 大, 頁	
\\	(るい). 
\\	洗	
\\	せん	あら	氵, 先	
\\	(せん), 
\\	禅	
\\	ぜん	
\\	ネ, 単	
\\	(ぜん), 
\\	得	
\\	とく	え, う	彳, 日, 寺	
\\	(とく) 
\\	加	
\\	か	くわ	力, 口	
\\	(か). 
\\	冊	
\\	さつ	
\\	冊	
\\	(さつ) 
\\	履	
\\	り	は	尸, 彳, 复	
\\	(り) 
\\	忙	
\\	ぼう	いそが	忄, 亡	
\\	(ぼう) 
\\	閥	
\\	ばつ	
\\	門, イ, 戈	
\\	(ばつ) 
\\	布	
\\	ふ	ぬの	ナ, 巾	
\\	(ふ). 
\\	比	
\\	ひ	くら	比	
\\	(ひ) 
\\	歴	
\\	れき	へ	厂, 木, 止	
\\	(れき) 
\\	続	
\\	ぞく	つづ	糸, 売	
\\	(ぞく) 
\\	減	
\\	げん	へ	氵, 感	
\\	(げん) 
\\	昆	
\\	こん	
\\	日, 比	
\\	(こん) 
\\	易	
\\	い, えき	やさ	易	
\\	い! 
\\	絡	
\\	らく	から	糸, 各	
\\	(らく). 
\\	笛	
\\	てき	ふえ	竹, 由	
\\	(てき) 
\\	容	
\\	よう	
\\	宀, 谷	
\\	(よう) 
\\	団	
\\	だん, とん	
\\	口, 寸	
\\	(だん) 
\\	(とん) 
\\	史	
\\	し	
\\	史	
\\	(し). 
\\	徒	
\\	と	
\\	彳, 走	
\\	(と) 
\\	宙	
\\	ちゅう	
\\	宀, 由	
\\	(ちゅう) 
\\	混	
\\	こん	ま	氵, 日, 比	
\\	(こん) 
\\	善	
\\	ぜん	
\\	羊, 一, 口		
\\	(ぜん) 
\\	順	
\\	じゅん	
\\	川, 頁	
\\	(じゅん) 
\\	宇	
\\	う	
\\	宀, 干	
\\	(う) 
\\	詞	
\\	し	
\\	言, 司	
\\	(し)?
\\	改	
\\	かい	あらた	己, 夂	
\\	(かい)! 
\\	乱	
\\	らん	みだ	舌, 乚	
\\	(らん). 
\\	節	
\\	せつ	ふし	竹, 艮, 卩	
\\	(せつ). 
\\	連	
\\	れん	つ, つら	車		
\\	(れん) 
\\	舌	
\\	ぜつ	した	舌	
\\	(した). 
\\	暴	
\\	ぼう	あば, ばく	日, 共, 水	
\\	(ぼう). 
\\	財	
\\	さい, ざい	
\\	貝, 才	
\\	(さい) 
\\	(ざい)!	
\\	若	
\\	じゃく	わか, も	艹, 右	
\\	(わか). 
\\	裕	
\\	ゆう	
\\	ネ, 谷	
\\	(ゆう) 
\\	尻	
\\	しり	尸, 九	
\\	(しり), 
\\	確	
\\	かく	たし	石, 冖, 隹	
\\	(かく) 
\\	械	
\\	かい	かせ	木, 戈, 廾	
\\	(かい) 
\\	犯	
\\	はん	おか	犭, 巳	
\\	(はん) 
\\	害	
\\	がい	
\\	宀, 生, 口	
\\	(がい), 
\\	議	
\\	ぎ	
\\	言, 義	
\\	(ぎ). 
\\	難	
\\	なん	むずか	隹		
\\	(なん). 
\\	災	
\\	さい	わざわ	巛, 火	
\\	(さい)! 
\\	嫌	
\\	けん, げん	いや, きら	女, 兼	
\\	(けん) 
\\	困	
\\	こん	こま	口, 木	
\\	(こん). 
\\	夢	
\\	む	ゆめ	艹, 罒, 冖, 夕	
\\	(む), 
\\	震	
\\	しん	ふる	雨, 辰	
\\	(しん) 
\\	在	
\\	ざい	
\\	ナ, 土		
\\	(ざい) 
\\	飛	
\\	ひ	と	飛	
\\	(ひ) 
\\	産	
\\	さん	う	立, 厂, 生	
\\	(さん)! 
\\	罪	
\\	ざい	つみ	罒, 非	
\\	(ざい) 
\\	穴	
\\	けつ	あな	穴	
\\	(あな) 
\\	被	
\\	ひ	かぶ	ネ, 皮	
\\	(ひ) 
\\	個	
\\	こ	
\\	イ, 口, 古	
\\	子
\\	(こ) 
\\	子
\\	機	
\\	き	はた	木, 幺, 戈, 人	
\\	(き). 
\\	妨	
\\	ぼう	さまた	女, 方	
\\	(ぼう) 
\\	倒	
\\	とう	たお	イ, 至, 刂	
\\	とうきょう. 
\\	とうきょう.	
\\	とうきょう 
\\	経	
\\	けい	た, へ	糸, 圣	
\\	(けい). 
\\	率	
\\	りつ, そつ	ひき	亠, 幺, 十		
\\	(りつ) 
\\	圧	
\\	あつ	
\\	厂, 土	
\\	(あつ)!
\\	防	
\\	ぼう	ふせ	阝, 方	
\\	(ぼう) 
\\	臭	
\\	しゅう	くさ	自, 大	
\\	草 (くさ). 
\\	草 
\\	草 
\\	草 
\\	余	
\\	よ	あま	示		
\\	(よ) 
\\	尾	
\\	び	お, ぽ	尸, 毛	
\\	(び)! 
\\	論	
\\	ろん	
\\	言, 一, 冊		
\\	(ろん)?
\\	厚	
\\	こう	あつ	厂, 日, 子	
\\	(あつ). 
\\	妻	
\\	さい	つま	一, 聿, 女	
\\	(さい) 
\\	責	
\\	せき	せ	生, 貝	
\\	(せき). 
\\	条	
\\	じょう	
\\	夂, 木	
\\	(じょう), 
\\	済	
\\	さい	す	氵, 斉	
\\	(さい) 
\\	委	
\\	い	
\\	禾, 女	
\\	(い) 
\\	省	
\\	しょう, せい	はぶ	少, 目	
\\	(しょう) 
\\	制	
\\	せい	
\\	生, 巾, 刂	
\\	(せい) 
\\	批	
\\	ひ	
\\	扌, 比	
\\	(ひ) 
\\	断	
\\	だん	ことわ, た	
\\	米, 斤	
\\	(だん). 
\\	任	
\\	にん	まか	イ, 王	
\\	(にん), 
\\	素	
\\	そ, す	
\\	生, 糸	
\\	(そ) 
\\	敵	
\\	てき	かな, かたき	啇, 夂	
\\	(てき), 
\\	羨	
\\	せん	うらや	王, 次		
\\	(せん)! 
\\	設	
\\	せつ	もう	言, 殳	
\\	(せつ). 
\\	評	
\\	ひょう	
\\	言, 平	
\\	(ひょう) 
\\	検	
\\	けん	
\\	木		
\\	(けん) 
\\	岡	
\\	おか	冂, 一, 山		
\\	(おか), 
\\	増	
\\	ぞう	ふ, ま	土, 曽	
\\	(ぞう) 
\\	査	
\\	さ	
\\	木, 且	
\\	(さ) 
\\	審	
\\	しん	
\\	宀, 番	
\\	(しん). 
\\	判	
\\	はん	
\\	半, 刂	
\\	(はん) 
\\	件	
\\	けん	
\\	イ, 牛	
\\	(けん) 
\\	際	
\\	さい	きわ	阝, 祭	
\\	(さい). 
\\	企	
\\	き	くわだ	止		
\\	(き) 
\\	挙	
\\	きょ	あ	一, ハ, 手		
\\	(きょ). 
\\	認	
\\	にん	みと	言, 刃, 心	
\\	(にん).	
\\	資	
\\	し	
\\	次, 貝	
\\	(し). 
\\	義	
\\	ぎ	
\\	義	
\\	(ぎ).	
\\	権	
\\	けん	
\\	木, 矢, 隹	
\\	(けん) 
\\	派	
\\	は	
\\	氵, 厂		
\\	(は)!
\\	務	
\\	む	つと	矛, 夂, 力	
\\	(む)!
\\	税	
\\	ぜい	
\\	禾, 兑	
\\	(ぜい). 
\\	解	
\\	かい	と	角, 刀, 牛	
\\	(かい) 
\\	総	
\\	そう	
\\	糸, 公, 心	
\\	(そう) 
\\	援	
\\	えん	
\\	扌, 爰	
\\	(えん) 
\\	態	
\\	たい	わざ	能, 心	
\\	(たい) 
\\	誕	
\\	たん	
\\	言, 廴, 正	
\\	(たん).	
\\	状	
\\	じょう	
\\	丬, 犬	
\\	(じょう), 
\\	賀	
\\	が	
\\	力, 口, 貝	
\\	(が). 
\\	各	
\\	かく	おの	各	
\\	(かく) 
\\	費	
\\	ひ	つい	弗, 貝	
\\	(ひ) 
\\	姿	
\\	し	すがた	次, 女	
\\	(し) 
\\	勢	
\\	せい, せ	いきお	坴, 丸, 力	
\\	(せい)! 
\\	諦	
\\	てい	あきら	言, 立, 巾	
\\	(てい) 
\\	示	
\\	じ, し	しめ	示	
\\	(じ). 
\\	寝	
\\	しん	ね	宀, 丬, ヨ, 冖, 又	
\\	(ね) 
\\	営	
\\	えい	いとな	呂		
\\	(えい). 
\\	坊	
\\	ぼう	
\\	土, 方	
\\	(ぼう) 
\\	罰	
\\	ばつ	ばっ	罒, 言, 刂	
\\	(ばつ).	
\\	案	
\\	あん	
\\	安, 木	
\\	(あん) 
\\	策	
\\	さく	
\\	竹, 木, 冂	
\\	(さく) 
\\	提	
\\	てい	
\\	扌, 日, 正	
\\	(てい) 
\\	置	
\\	ち	お	罒, 直	
\\	(ち) 
\\	域	
\\	いき	
\\	土, 戈, 口, 一	
\\	(いき) 
\\	応	
\\	おう	
\\	广, 心	
\\	王 (おう).	
\\	王 
\\	宮	
\\	きゅう	みや	宀, 呂	
\\	袖	
\\	しゅう	そで	ネ, 由	
\\	(しゅう) 
\\	吸	
\\	きゅう	す	口, 及	
\\	(きゅう) 
\\	過	
\\	か	す, あやま	冋		
\\	(か)! 
\\	領	
\\	りょう	
\\	令, 頁	
\\	(りょう) 
\\	脱	
\\	だつ	ぬ	月, 兑	
\\	(だつ) 
\\	統	
\\	とう	す	糸, 充	
\\	とうきょう. 
\\	とうきょう. 
\\	とうきょう!	
\\	価	
\\	か	あたい	イ, 覀	
\\	(か) 
\\	値	
\\	ち	ね, あたい	イ, 直	
\\	(ち), 
\\	副	
\\	ふく	
\\	畐, 刂	
\\	(ふく) 
\\	観	
\\	かん	み	矢, 隹, 見	
\\	(かん) 
\\	藤	
\\	とう, どう	ふじ	艹, 月, 水		
\\	(ふじ).	
\\	呼	
\\	こ	よ	口, 平	
\\	(よ) 
\\	崎	
\\	き	さき	山, 大, 可	
\\	(さき). 
\\	施	
\\	し	ほどこ	方, 也		
\\	(し) 
\\	城	
\\	じょう	しろ	土, 成	
\\	(しろ). 
\\	護	
\\	ご	
\\	言, 艹, 隹, 又	
\\	(ご). 
\\	鬼	
\\	き	おに	鬼	
\\	(き). 
\\	割	
\\	かつ	わり, わ	宀, 生, 口, 刂	
\\	(わり)! 
\\	わ 
\\	わり
\\	わ
\\	職	
\\	しょく	
\\	耳, 音, 戈	
\\	(しょく) 
\\	秀	
\\	しゅう	ひい	禾, 乃	
\\	(しゅう).	
\\	俳	
\\	はい	
\\	イ, 非	
\\	(はい). 
\\	停	
\\	てい	
\\	イ, 亭	
\\	(てい) 
\\	宅	
\\	たく	
\\	宀, 丿, 七	
\\	(たく), 
\\	裁	
\\	さい	さば, た	十, 戈, 衣	
\\	(さい) 
\\	律	
\\	りつ	
\\	彳, 聿	
\\	(りつ) 
\\	導	
\\	どう	みちび	道, 寸	
\\	(どう) 
\\	革	
\\	かく	かわ	革	
\\	(かく). 
\\	贅	
\\	ぜい	いぼ	土, 方, 夂, 貝	
\\	(ぜい)!	
\\	乳	
\\	にゅう	ちち	子, 乚		
\\	(にゅう) 
\\	収	
\\	しゅう	おさ	丩, 又	
\\	(しゅう). 
\\	演	
\\	えん	
\\	氵, 宀, 田, ハ		
\\	(えん), 
\\	現	
\\	げん	あらわ	王, 見	
\\	(げん). 
\\	備	
\\	び	そな	イ, 艹, 厂, 用	
\\	(び).	
\\	則	
\\	そく	のっと	貝, 刂	
\\	(そく) 
\\	規	
\\	き	
\\	夫, 見	
\\	(き) 
\\	準	
\\	じゅん	
\\	氵, 隹, 十	
\\	(じゅん) 
\\	張	
\\	ちょう	は	弓, 長	
\\	(ちょう). 
\\	優	
\\	ゆう	やさ, すぐ	イ, 百, 冖, 心, 夂	
\\	(ゆう). 
\\	沢	
\\	たく	さわ	氵, 尺	
\\	(たく) 
\\	師	
\\	し	
\\	丶, 一, 巾		
\\	(し). 
\\	幹	
\\	かん	みき	干	
\\	(かん) 
\\	看	
\\	かん	
\\	手, 目	
\\	(かん) 
\\	庁	
\\	ちょう	
\\	广, 丁	
\\	(ちょう) 
\\	丁, 
\\	額	
\\	がく	ひたい	客, 頁	
\\	(がく) 
\\	腕	
\\	わん	うで	月, 宀, 夕, 巳	
\\	(うで)?! 
\\	境	
\\	きょう	さかい	土, 立, 見	
\\	(きょう) 
\\	燃	
\\	ねん	も	火, 然	
\\	(ねん) 
\\	担	
\\	たん	にな, かつ	扌, 旦	
\\	(たん). 
\\	祝	
\\	しゅく, しゅう	いわ	ネ, 兄	
\\	(しゅく) 
\\	届	
\\	とど	尸, 由	
\\	(とど). 
\\	違	
\\	い	ちが	韋		
\\	(ちが). 
\\	差	
\\	さ	さ	王, 丿, 工		
\\	(さ).	
\\	象	
\\	ぞう, しょう	
\\	象	
\\	(ぞう). 
\\	(しょう) 
\\	ぞう 
\\	しょう 
\\	展	
\\	てん	のぶ, のび	尸	
\\	(てん). 
\\	層	
\\	そう	
\\	尸, 曽	
\\	(そう) 
\\	視	
\\	し	み	ネ, 見	
\\	(し) 
\\	環	
\\	かん	
\\	王, 罒		
\\	(かん) 
\\	製	
\\	せい	
\\	制, 衣	
\\	(せい). 
\\	述	
\\	じゅつ	の	ホ, 丶		
\\	(じゅつ)!
\\	武	
\\	ぶ, む	たけ	一, 弋, 止	
\\	(ぶ)!	
\\	型	
\\	けい	かた	开, 刂, 土	
\\	(けい)! 
\\	狭	
\\	きょう	せま, せば	犭, 夫		
\\	(せま) 
\\	管	
\\	かん	くだ	竹, 宀		
\\	(かん), 
\\	載	
\\	さい	の	十, 戈, 車	
\\	(さい) 
\\	質	
\\	しつ, しち	
\\	斤, 貝	
\\	(しつ) 
\\	量	
\\	りょう	はか	旦, 里	
\\	販	
\\	はん	
\\	貝, 反	
\\	(はん) 
\\	供	
\\	きょう	とも, そな	イ, 共	
\\	きょうと, 
\\	きょうと 
\\	きょうと 
\\	肩	
\\	けん	かた	戸, 月	
\\	(かた)!
\\	株	
\\	しゅ	かぶ	木, 丿, 未	
\\	(かぶ)! 
\\	触	
\\	しょく	さわ, ふ	角, 虫	
\\	(しょく). 
\\	輸	
\\	ゆ	
\\	車		
\\	(ゆ). 
\\	腰	
\\	よう	こし	月, 覀, 女	
\\	(こし)! 
\\	慣	
\\	かん	な	忄, 毋, 貝	
\\	(かん). 
\\	居	
\\	きょ	い	尸, 古	
\\	(きょ).	
\\	逮	
\\	たい	
\\	聿, 水		
\\	(たい) 
\\	票	
\\	ひょう	
\\	覀, 示	
\\	(ひょう), 
\\	属	
\\	ぞく	
\\	尸, 禹	
\\	(ぞく) 
\\	捉	
\\	そく	とら	扌, 足	
\\	(とら) 
\\	捕	
\\	ほ	つか, とら	扌, 甫	
\\	(ほ) 
\\	候	
\\	こう	
\\	イ, ユ, 矢		
\\	こういち.	
\\	こういち 
\\	輩	
\\	はい	
\\	非, 車	
\\	(はい). 
\\	況	
\\	きょう	
\\	氵, 兄	
\\	きょうと 
\\	きょうと 
\\	響	
\\	きょう	ひび	幺, 艮, 阝, 音	
\\	きょうと. 
\\	きょうと 
\\	きょうと 
\\	効	
\\	こう	き	交, 力	
\\	こういち
\\	こういち
\\	抜	
\\	ばつ, はつ, はい	ぬ	扌, 友	
\\	(ぬ). 
\\	鮮	
\\	せん	あざ	羊, 魚	
\\	(せん), 
\\	満	
\\	まん	み	氵, 艹, 両	
\\	(まん) 
\\	与	
\\	よ	あた	一		
\\	(よ)!” 
\\	掛	
\\	がい	か	扌, 土, ト	
\\	(か). 
\\	隠	
\\	いん	かく	阝, ヨ, 心		
\\	(いん).	
\\	模	
\\	も, ぼ	
\\	木, 莫	
\\	(も) 
\\	(ぼ) 
\\	含	
\\	がん	ふく	今, 口	
\\	(がん) 
\\	訟	
\\	しょう	
\\	言, 公	
\\	(しょう). 
\\	限	
\\	げん	かぎ	阝, 艮	
\\	(げん). 
\\	肥	
\\	ひ	こ, こえ	月, 巴	
\\	(ひ)! 
\\	豊	
\\	ほう	ゆた	曲, 豆	
\\	(ほう). 
\\	替	
\\	たい	か	夫, 日	
\\	(か) 
\\	景	
\\	けい	
\\	日, 京	
\\	(けい). 
\\	巻	
\\	かん	ま	己		
\\	(かん)! 
\\	捜	
\\	そう	さが	扌, 申, 又	
\\	(そう) 
\\	構	
\\	こう	かま	木, 冓	
\\	こういち! 
\\	こういち 
\\	こういち 
\\	こう.
\\	影	
\\	えい	かげ	日, 京, 彡	
\\	(えい) 
\\	絞	
\\	こう	し, しぼ	糸, 交	
\\	こういち, 
\\	訴	
\\	そ	うった	言, 斤, 丶	
\\	(そ). 
\\	補	
\\	ほ	おぎな	ネ, 甫	
\\	(ほ). 
\\	渡	
\\	と	わた	氵, 又		
\\	(わた). 
\\	接	
\\	せつ	つ	扌, 立, 女	
\\	(せつ). 
\\	再	
\\	さ, さい	ふたた	用		
\\	(さ) 
\\	(さい).	
\\	独	
\\	どく	ひと	犭, 虫	
\\	(どく) 
\\	獣	
\\	じゅう	けもの	田, 犬	
\\	(じゅう), 
\\	菓	
\\	か	
\\	艹, 果	
\\	(か) 
\\	果.	
\\	討	
\\	とう	
\\	言, 寸	
\\	とうきょう. 
\\	とうきょう?! 
\\	故	
\\	こ	ゆえ	古, 夂	
\\	子 (こ), 
\\	較	
\\	かく, こう	
\\	車, 交	
\\	(かく). 
\\	創	
\\	そう	
\\	倉, 刂	
\\	(そう). 
\\	造	
\\	ぞう	つく	告		
\\	(ぞう) 
\\	往	
\\	おう	
\\	彳, 主	
\\	王 (おう) 
\\	王 
\\	励	
\\	れい	はげ	厂, 万, 力	
\\	(れい) 
\\	激	
\\	げき	はげ	氵, 白, 方, 夂	
\\	(げき), 
\\	占	
\\	せん	うらな, し	占	
\\	(せん) 
\\	障	
\\	しょう	さわ	阝, 章	
\\	(しょう).	
\\	我	
\\	が	われ	我	
\\	(が) 
\\	徴	
\\	ちょう	
\\	彳, 山, 王, 夂	
\\	(ちょう) 
\\	授	
\\	じゅ	さず	扌, 受	
\\	(じゅ) 
\\	鉛	
\\	えん	なまり	金, ハ, 口	
\\	(えん).	
\\	郵	
\\	ゆう	
\\	車, 阝	
\\	(ゆう) 
\\	針	
\\	しん	はり	金, 十	
\\	(しん). 
\\	従	
\\	じゅう	したが	彳, 正		
\\	(じゅう)!	
\\	豚	
\\	とん	ぶた	月, 豕	
\\	(ぶた). 
\\	復	
\\	ふく	
\\	彳, 复	
\\	(ふく) 
\\	河	
\\	か	かわ	氵, 可	
\\	(か) 
\\	貯	
\\	ちょ	たくわ	貝, 宀, 丁	
\\	(ちょ) 
\\	印	
\\	いん	しるし	卩		
\\	(いん). 
\\	振	
\\	しん	ふ	扌, 辰	
\\	(しん). 
\\	突	
\\	とつ	つ	穴, 大	
\\	(とつ) 
\\	刺	
\\	し	さ	木, 冂, 刂	
\\	(し). 
\\	怪	
\\	かい, け	あや	忄, 圣	
\\	(かい). 
\\	(け) 
\\	汗	
\\	かん	あせ	氵, 干	
\\	(あせ).	
\\	筆	
\\	ひつ	ふで	竹, 聿	
\\	(ひつ) 
\\	怒	
\\	ど	おこ, いか	女, 又, 心	
\\	(ど). 
\\	昇	
\\	しょう	のぼ	日, 丿, 廾	
\\	(しょう) 
\\	迷	
\\	めい	まよ	米		
\\	めい, 
\\	招	
\\	しょう	まね	扌, 召	
\\	(しょう). 
\\	腹	
\\	ふく	はら, なか	月, 复	
\\	(ふく) 
\\	睡	
\\	すい	
\\	目, 車	
\\	(すい) 
\\	康	
\\	こう	
\\	广, 聿, 水	
\\	こういち. 
\\	"こういち 
\\	端	
\\	たん	はし, はた	立, 山, 而	
\\	(たん). 
\\	極	
\\	きょく, ごく	きわ	木, 口, 又, 一		
\\	きょく, 
\\	(ごく) 
\\	(きょく). 
\\	郎	
\\	ろう	
\\	良, 阝	
\\	(ろう).	
\\	健	
\\	けん	
\\	イ, 廴, 聿	
\\	(けん) 
\\	誘	
\\	ゆう	さそ	言, 禾, 乃	
\\	(ゆう) 
\\	貸	
\\	たい	か	代, 貝	
\\	(か). 
\\	惑	
\\	わく	まど	戈, 口, 一, 心	
\\	(わく). 
\\	痛	
\\	つう	いた	疒, マ, 用	
\\	(つう) 
\\	退	
\\	たい	しりぞ, ひ, ど, の	艮		
\\	(たい). 
\\	途	
\\	と	
\\	余		
\\	(と) 
\\	給	
\\	きゅう	たま	糸, 合	
\\	(きゅう). 
\\	就	
\\	しゅう	つ	京, 犬	
\\	(しゅう). 
\\	靴	
\\	か	くつ	革, 化	
\\	(くつ). 
\\	眠	
\\	みん	ねむ	目, 民	
\\	(みん) 
\\	民, 
\\	暇	
\\	か	ひま, いとま	日, 匚, 又		
\\	(か). 
\\	段	
\\	だん	
\\	殳		
\\	(だん) 
\\	胃	
\\	い	
\\	田, 月	
\\	(い) 
\\	症	
\\	しょう	
\\	疒, 正	
\\	(しょう) 
\\	濃	
\\	のう	こ	氵, 農	
\\	(のう) 
\\	締	
\\	てい	し	糸, 立, 巾	
\\	(てい) 
\\	迫	
\\	はく	せま	白		
\\	(はく) 
\\	白 
\\	はく.	
\\	訪	
\\	ほう	たず, おとず	言, 方	
\\	(ほう).
\\	織	
\\	しき, しょく	お	糸, 音, 戈	
\\	(しき) 
\\	悩	
\\	のう	なや	忄, 凶		
\\	(なや). 
\\	屈	
\\	くつ	かが	尸, 出	
\\	(くつ)! 
\\	攻	
\\	こう	せ	工, 夂	
\\	こういち, 
\\	こういち, 
\\	工 
\\	こう 
\\	撃	
\\	げき	う	車, 殳, 手	
\\	(げき), 
\\	浜	
\\	ひん	はま	氵, 斤, 一, ハ	
\\	(はま). 
\\	綺	
\\	き	
\\	糸, 奇	
\\	(き). 
\\	益	
\\	えき	
\\	一, ハ, 皿		
\\	駅 
\\	駅 
\\	駅 
\\	児	
\\	じ	こ	日, 儿		
\\	(じ) 
\\	憲	
\\	けん	
\\	宀, 生, 罒, 心	
\\	(けん) 
\\	冷	
\\	れい	つめ, ひ, さ	冫, 令	
\\	(れい) 
\\	れい.
\\	処	
\\	しょ	ところ	夂, 几	
\\	(しょ)? 
\\	微	
\\	び	かす	彳, 山, 兀, 夂	
\\	(び), 
\\	修	
\\	しゅう	おさ	イ, 夂, 彡		
\\	(しゅう) 
\\	博	
\\	はく, ばく	
\\	十, 専	
\\	(はく) 
\\	程	
\\	てい	ほど	禾, 口, 王	
\\	(てい) 
\\	絶	
\\	ぜつ	た	糸, 色	
\\	(ぜつ). 
\\	凍	
\\	とう	こお, こご	冫, 東	
\\	とうきょう. とうきょう 
\\	とうきょう 
\\	巨	
\\	きょ	
\\	巨	
\\	(きょ), 
\\	稚	
\\	ち	
\\	禾, 隹	
\\	(ち). 
\\	幼	
\\	よう	おさな	幺, 力	
\\	(よう). 
\\	並	
\\	へい	なら, なみ	一	
\\	(へい) 
\\	麗	
\\	れい	うるわ	麗	
\\	(れい) 
\\	奇	
\\	き	
\\	奇	
\\	(き).	
\\	衆	
\\	しゅう	
\\	血, 彡		
\\	(しゅう). 
\\	清	
\\	せい, しょう, しん	きよ	氵, 青	
\\	(せい)! 
\\	潔	
\\	けつ	いさぎよ	氵, 生, 刀, 糸	
\\	(けつ) 
\\	録	
\\	ろく	
\\	金, ヨ, 水	
\\	(ろく) 
\\	逆	
\\	ぎゃく	さか	屯, 丶		
\\	(ぎゃく). 
\\	移	
\\	い	うつ	禾, 夕	
\\	(い) 
\\	精	
\\	せい	
\\	米, 青	
\\	(せい), 
\\	隊	
\\	たい	
\\	阝, 豕		
\\	(たい) 
\\	庫	
\\	こ	くら	广, 車	
\\	子 (こ) 
\\	子
\\	子 
\\	子 
\\	妙	
\\	みょう	たえ	女, 少	
\\	(みょう) 
\\	券	
\\	けん	
\\	刀		
\\	(けん) 
\\	傘	
\\	さん	かさ	人, 十		
\\	(かさ) 
\\	婦	
\\	ふ	
\\	女, ヨ, 冖, 巾	
\\	(ふ). 
\\	略	
\\	りゃく	
\\	田, 各	
\\	(りゃく) 
\\	積	
\\	せき	つ	禾, 責	
\\	(せき) 
\\	添	
\\	てん	そ	氵, 天, 小, 丶	
\\	(てん) 
\\	寄	
\\	き	よ	宀, 奇	
\\	(き).	
\\	宴	
\\	えん	うたげ	宀, 日, 女	
\\	(えん) 
\\	板	
\\	はん	いた	木, 反	
\\	(はん) 
\\	壊	
\\	かい	こわ	土, 十, 罒, 衣	
\\	(かい) 
\\	督	
\\	とく	
\\	上, 小, 又, 目	
\\	(とく) 
\\	僚	
\\	りょう	
\\	イ, 尞	
\\	(りょう) 
\\	杯	
\\	はい	
\\	木, 不	
\\	(はい) 
\\	娘	
\\	むすめ	女, 良	
\\	(むすめ) 
\\	診	
\\	しん	み	言, 彡		
\\	(しん) 
\\	乾	
\\	かん	かわ, ほ	乙	
\\	(かん)! 
\\	欧	
\\	おう	
\\	区, 欠	
\\	王 (おう) 
\\	王
\\	恐	
\\	きょう	おそ, こわ	心		
\\	きょうと. 
\\	きょうと, 
\\	きょうと. 
\\	きょうと 
\\	猛	
\\	もう	
\\	犭, 子, 皿	
\\	(もう) 
\\	江	
\\	こう	え	氵, 工	
\\	(え)? 
\\	韓	
\\	かん	
\\	韋		
\\	(かん). 
\\	雄	
\\	ゆう	おす	ナ, ム, 隹	
\\	(ゆう)! 
\\	航	
\\	こう	
\\	舟, 亠, 几	
\\	こういち, 
\\	こういち 
\\	監	
\\	かん	
\\	臣, 一, 皿		
\\	(かん), 
\\	宗	
\\	しゅう	
\\	宀, 示	
\\	(しゅう). 
\\	請	
\\	せい, しん, しょう	う, こ	言, 青	
\\	(せい) 
\\	怖	
\\	ふ	こわ	忄, ナ, 巾	
\\	(ふ) 
\\	索	
\\	さく	
\\	十, 冖, 糸	
\\	(さく). 
\\	臣	
\\	しん, じん	
\\	臣	
\\	(しん) 
\\	(じん) 
\\	催	
\\	さい	もよお	イ, 山, 隹	
\\	(さい). 
\\	街	
\\	がい, かい	まち	土, 行	
\\	(がい) 
\\	詰	
\\	きつ, きち	つ, づ	言, 吉	
\\	(つ) 
\\	緊	
\\	きん	
\\	臣, 又, 糸	
\\	(きん) 
\\	閣	
\\	かく	
\\	門, 各	
\\	(かく). 
\\	促	
\\	そく	うなが	イ, 足	
\\	(そく) 
\\	足, 
\\	烈	
\\	れつ	はげ	歹, 刂, 灬	
\\	(れつ) 
\\	更	
\\	こう	さら, ふ	一, 田, メ	
\\	こういち 
\\	こういち 
\\	魅	
\\	み	
\\	鬼, 未	
\\	(み). 
\\	背	
\\	はい	せ, そむ	北, 月	
\\	(はい) 
\\	騒	
\\	そう	さわ	馬, 又, 虫	
\\	(そう) 
\\	飾	
\\	しょく	かざ	食, 巾		
\\	(しょく). 
\\	預	
\\	よ	あず	予, 頁	
\\	(よ). 
\\	版	
\\	はん	
\\	片, 反	
\\	(はん) 
\\	旗	
\\	き	はた	方, 其		
\\	(き). 
\\	浮	
\\	ふ	う, うわ	氵, 子		
\\	(う) 
\\	(うわ)” 
\\	うに 
\\	越	
\\	えつ	こ	走, 成	
\\	(えつ) 
\\	照	
\\	しょう	て	日, 召, 灬	
\\	(しょう). 
\\	漏	
\\	ろう	も	氵, 尸, 雨	
\\	(ろう) 
\\	系	
\\	けい	
\\	系	
\\	(けい). 
\\	覧	
\\	らん	
\\	臣, 一, 見		
\\	(らん) 
\\	婚	
\\	こん	
\\	女, 氏, 日	
\\	(こん) 
\\	懐	
\\	かい	なつ	忄, 十, 罒, 衣	
\\	(なつ) 
\\	撮	
\\	さつ	と	扌, 日, 耳, 又	
\\	(さつ) 
\\	枕	
\\	しん	まくら	木, 冘	
\\	(まくら). 
\\	遊	
\\	ゆう	あそ	方, 子	
\\	(ゆう), 
\\	快	
\\	かい	こころよ	忄, 人, ユ	
\\	(かい). 
\\	貧	
\\	びん, ひん	まず	分, 貝	
\\	(びん). 
\\	延	
\\	えん	のば, の	廴, 正	
\\	(えん) 
\\	押	
\\	おう	お	扌, 甲	
\\	(お). 
\\	乏	
\\	ぼう	とぼ		
\\	(ぼう) 
\\	匂	
\\	にお	勹, 匕	
\\	(にお). 
\\	盗	
\\	とう	ぬす	次, 皿	
\\	とうきょう 
\\	とうきょう 
\\	とうきょう. 
\\	購	
\\	こう	
\\	貝, 冓	
\\	こういち 
\\	こういち 
\\	こういち 
\\	こう.
\\	適	
\\	てき	
\\	啇		
\\	(てき), 
\\	翌	
\\	よく	
\\	羽, 立	
\\	(よく) 
\\	渇	
\\	かつ	かわ	氵, 日, 勹, 匕	
\\	(かわ). 
\\	符	
\\	ふ	
\\	竹, 付	
\\	(ふ)! 
\\	濡	
\\	ぬ	氵, 雨, 而	
\\	(ぬ)!	
\\	帯	
\\	たい	おび	山, 一, 冖, 巾	
\\	(たい) 
\\	廊	
\\	ろう	
\\	广, 郎	
\\	(ろう).	
\\	離	
\\	り	はな	离, 隹	
\\	り.	
\\	径	
\\	けい	
\\	彳, 圣	
\\	(けい) 
\\	融	
\\	ゆう	
\\	鬲, 虫	
\\	(ゆう). 
\\	均	
\\	きん	ひと	土, 勺, 丶	
\\	(きん) 
\\	除	
\\	じょ, じ	のぞ	阝, 余	
\\	(じょ).	
\\	貨	
\\	か	
\\	化, 貝	
\\	(か). 
\\	孫	
\\	そん	まご	子, 系	
\\	(そん) 
\\	墓	
\\	ぼ	はか	土, 莫	
\\	(ぼ) 
\\	幾	
\\	き	いく	幺, 戈, 人	
\\	(いく) 
\\	尋	
\\	じん	たず, ひろ	ヨ, 工, 口, 寸	
\\	(じん). 
\\	編	
\\	へん	あ	糸, 扁	
\\	(へん) 
\\	陸	
\\	りく	
\\	阝, 坴	
\\	(りく) 
\\	探	
\\	たん	さが, さぐ	扌, 兀, 木	
\\	(たん). 
\\	豪	
\\	ごう	
\\	亠, 口, 冖, 豕	
\\	(ごういち). 
\\	鑑	
\\	かん	
\\	金, 監	
\\	(かん). 
\\	泥	
\\	でい	どろ	氵, 尸, 匕	
\\	(でい) 
\\	巣	
\\	そう	す	果		
\\	(す). 
\\	普	
\\	ふ	
\\	並, 日	
\\	(ふ), 
\\	棒	
\\	ぼう	
\\	木, 干		
\\	(ぼう). 
\\	粉	
\\	ふん	こな, こ	米, 分	
\\	(ふん). 
\\	既	
\\	き	すで	艮, 牙	
\\	(き).	
\\	救	
\\	きゅう	すく	求, 夂	
\\	(きゅう).
\\	似	
\\	ね, じ	に	イ, 
\\	丶, 人	
\\	(ね) 
\\	富	
\\	ふ	と, とみ	宀, 畐	
\\	(ふ)! 
\\	散	
\\	さん	ち	月, 夂		
\\	(さん). 
\\	華	
\\	か	はな	艹, 田, 十	
\\	(か)!	
\\	嘆	
\\	たん	なげ	口		
\\	(たん). 
\\	偵	
\\	てい	
\\	イ, ト, 貝	
\\	(てい) 
\\	驚	
\\	きょう	おどろ	苟, 夂, 馬	
\\	きょうと. 
\\	きょうと 
\\	きょうと 
\\	きょうと 
\\	掃	
\\	そう	は	扌, ヨ, 冖, 巾	
\\	(そう). 
\\	菜	
\\	さい	な	艹, 木		
\\	(さい). 
\\	脈	
\\	みゃく	
\\	月, 厂		
\\	(みゃく) 
\\	徳	
\\	とく	
\\	彳, 十, 罒, 心	
\\	(とく) 
\\	倉	
\\	そう	くら	倉	
\\	(そう). 
\\	酸	
\\	さん	す	酉, 夋	
\\	(さん) 
\\	賛	
\\	さん	
\\	夫, 貝	
\\	(さん) 
\\	祖	
\\	そ	
\\	ネ, 且	
\\	(そ) 
\\	銭	
\\	せん	ぜに	金		
\\	(せん). 
\\	込	
\\	こ	入		
\\	子 (こ) 
\\	子 
\\	衛	
\\	えい	
\\	行, 韋	
\\	(えい). 
\\	机	
\\	き	つくえ	木, 几	
\\	(つくえ)?” 
\\	汚	
\\	お	よご, きたな, けが	氵, 一		
\\	(お) 
\\	飼	
\\	し	か	食, 司	
\\	(か) 
\\	複	
\\	ふく	
\\	ネ, 复	
\\	(ふく) 
\\	染	
\\	せん	しみ, そ, し	氵, 九, 木	
\\	(せん). 
\\	卵	
\\	らん	たまご	勺	
\\	(たまご), 
\\	永	
\\	えい	
\\	永	
\\	(えい) 
\\	績	
\\	せき	
\\	糸, 責	
\\	(せき). 
\\	眼	
\\	がん	め	目, 艮	
\\	(がん). 
\\	液	
\\	えき	
\\	氵, 夜	
\\	駅 (えき) 
\\	駅 
\\	駅 
\\	採	
\\	さい	と	扌, 木		
\\	(さい) 
\\	志	
\\	し	こころざし	士, 心	
\\	(し) 
\\	興	
\\	きょう, こう	
\\	同, 一, ハ		
\\	きょうと. 
\\	きょうと 
\\	きょうと, 
\\	きょうと 
\\	恩	
\\	おん	
\\	口, 大, 心	
\\	(おん) 
\\	久	
\\	きゅう, く	ひさ	勹, 丿	
\\	(きゅう) 
\\	党	
\\	とう	
\\	兄		
\\	とうきょう. 
\\	とうきょう 
\\	とうきょう. 
\\	序	
\\	じょ	つい, ついで	广, 予	
\\	(じょ), 
\\	雑	
\\	ざつ, ぞう	
\\	九, 木, 隹	
\\	(ざつ) 
\\	桜	
\\	おう, よう	さくら	木, 女		
\\	(さくら). 
\\	密	
\\	みつ	ひそ	宀, 必, 山	
\\	(みつ) 
\\	秘	
\\	ひ	ひ	禾, 必	
\\	(ひ).	
\\	厳	
\\	げん, ごん	きび, おごそ	厂, 夂	
\\	(げん)?! 
\\	捨	
\\	しゃ	す	扌, 舎	
\\	(す). 
\\	訳	
\\	やく	わけ	言, 尺	
\\	(やく). 
\\	欲	
\\	よく	ほ	谷, 欠	
\\	(よく) 
\\	暖	
\\	だん	あたた	日, 爰	
\\	(だん). 
\\	迎	
\\	げい	むか	卬		
\\	(げい) 
\\	傷	
\\	しょう	きず, いた	イ, 易		
\\	(しょう). 
\\	灰	
\\	かい	はい	厂, 火	
\\	(はい). 
\\	装	
\\	そう, しょう	よそお	丬, 士, 衣	
\\	(そう). 
\\	著	
\\	ちょ	いちじる, あらわ	艹, 者	
\\	(ちょ), 
\\	裏	
\\	り	うら	亠, 里		
\\	(うら). 
\\	閉	
\\	へい	し, と	門, 才	
\\	(へい) 
\\	垂	
\\	すい	た	車	
\\	車. 
\\	(すい) 
\\	漠	
\\	ばく	
\\	氵, 莫	
\\	(ばく) 
\\	異	
\\	い	こと	田, 共	
\\	(い) 
\\	皇	
\\	こう	
\\	白, 王	
\\	こういち. こういち 
\\	こういち 
\\	こういち 
\\	こう.	
\\	拡	
\\	かく	ひろ	扌, 広	
\\	(かく) 
\\	暮	
\\	ぼ	く	莫, 日	
\\	(ぼ) 
\\	忠	
\\	ちゅう	
\\	中, 心	
\\	(ちゅう), 
\\	中 
\\	肺	
\\	はい	
\\	月, 市	
\\	月 
\\	(はい). 
\\	誌	
\\	し	
\\	言, 士, 心	
\\	(し) 
\\	操	
\\	そう	あやつ, みさお	扌, 喿	
\\	(そう). 
\\	筋	
\\	きん	すじ	竹, 月, 力	
\\	(きん) 
\\	否	
\\	ひ	いな, いや	不, 口	
\\	(ひ). 
\\	盛	
\\	せい, じょう	も, さか	成, 皿	
\\	(せい) 
\\	宣	
\\	せん	のたま	宀, 一, 旦	
\\	(せん). 
\\	賃	
\\	ちん	
\\	イ, 王, 貝	
\\	(ちん).	
\\	敬	
\\	けい	うやま	苟, 夂	
\\	(けい)!	
\\	尊	
\\	そん	とうと, たっと	酉, 寸		
\\	(そん) 
\\	熟	
\\	じゅく	う	享, 丸, 灬	
\\	(じゅく) 
\\	噂	
\\	そん	うわさ	口, 酉, 寸		
\\	(うわさ)...
\\	うわさ, 
\\	砂	
\\	さ	すな	石, 少	
\\	(さ)! 
\\	簡	
\\	かん	
\\	竹, 間	
\\	(かん) 
\\	蒸	
\\	じょう	む	艹, 灬		
\\	(じょう), 
\\	蔵	
\\	ぞう	くら	蔵	
\\	(ぞう).	
\\	糖	
\\	とう	
\\	米, 广, 聿, 口	
\\	とうきょう. 
\\	とうきょう 
\\	とうきょう. 
\\	納	
\\	のう	おさ, なっ	糸, 内	
\\	(のう) 
\\	諸	
\\	しょ	もろ	言, 者	
\\	(しょ). 
\\	窓	
\\	そう	まど	穴, ム, 心	
\\	(まど)! 
\\	豆	
\\	とう	まめ	豆	
\\	とうきょう. 
\\	とうきょう 
\\	とうきょう. 
\\	とうきょう 
\\	とうきょう 
\\	枝	
\\	し	えだ	木, 支	
\\	(し). 
\\	揮	
\\	き	
\\	扌, 冖, 車	
\\	(き)! 
\\	刻	
\\	こく	きざ	亥, 刂	
\\	(こく). 
\\	爪	
\\	そう	つま, つめ	爪	
\\	(つま), 
\\	(つめ) 
\\	承	
\\	しょう	うけたまわ	子, 二, 水	
\\	(しょう). 
\\	幕	
\\	まく, ばく	とばり	莫, 巾	
\\	(まく). 
\\	紅	
\\	こう	べに, くれない	工, 糸	
\\	こういち, 
\\	"こういち 
\\	工 
\\	こう 
\\	歓	
\\	かん	
\\	矢, 隹, 欠	
\\	(かん) 
\\	降	
\\	こう	お, ふ	阝, 夂, 牛	
\\	こういち
\\	こういち
\\	奴	
\\	ど	やつ	女, 又	
\\	ど 
\\	聖	
\\	せい	
\\	耳, 口, 王	
\\	(せい) 
\\	推	
\\	すい	お	扌, 隹	
\\	(すい) 
\\	臓	
\\	ぞう	
\\	月, 蔵	
\\	蔵?). 
\\	(ぞう) 
\\	損	
\\	そん	そこ	扌, 員	
\\	(そん). 
\\	磁	
\\	じ	
\\	石, 一, 幺		
\\	(じ)! 
\\	誤	
\\	ご	あやま	言, 呉	
\\	(ご), 
\\	源	
\\	げん	みなもと	氵, 原	
\\	(げん), 
\\	芋	
\\	いも	艹, 干	
\\	(いも) 
\\	純	
\\	じゅん	
\\	糸, 屯	
\\	(じゅん), 
\\	薦	
\\	せん	すす	艹, 广, 覀, 鳥	
\\	(せん) 
\\	丼	
\\	どん	どんぶり	井, 丶	
\\	(どん) 
\\	腐	
\\	ふ	くさ	广, 付, 肉	
\\	(ふ). 
\\	沿	
\\	えん	そ	氵, ハ, 口	
\\	(えん).	
\\	射	
\\	しゃ	い, さ, う	身, 寸	
\\	(しゃ). 
\\	縮	
\\	しゅく	ちぢ, ちじ	糸, 宀, イ, 百	
\\	(しゅく) 
\\	隷	
\\	れい	
\\	士, 示, 聿, 水	
\\	(れい) 
\\	粋	
\\	すい	いき	米, 九, 十	
\\	(すい) 
\\	痩	
\\	そう	や	疒, 申, 又	
\\	(そう) 
\\	吐	
\\	と	は, つ	口, 土	
\\	(は)!
\\	貴	
\\	き	とうと	中, 一, 貝	
\\	(き). 
\\	縦	
\\	じゅう	たて	糸, 彳, 正		
\\	(たて) 
\\	勤	
\\	きん	つと	堇, 力	
\\	(きん) 
\\	拝	
\\	はい	おが	扌, 干	
\\	(はい). 
\\	熊	
\\	くま	能, 灬	
\\	(くま) 
\\	噌	
\\	そ	
\\	口, 曽	
\\	(そ) 
\\	彫	
\\	ちょう	ほ	周, 彡	
\\	(ちょう). 
\\	杉	
\\	すぎ	木, 彡	
\\	(すぎ), 
\\	銅	
\\	どう	あかがね	金, 同	
\\	(どう). 
\\	舎	
\\	しゃ	
\\	舎	
\\	(しゃ). 
\\	酔	
\\	すい	よ	酉, 九, 十	
\\	水? 
\\	すい, 
\\	水, 
\\	水 
\\	水. 
\\	炎	
\\	えん	ほのお	火	
\\	(えん).	
\\	彼	
\\	ひ	かれ, かの	彳, 皮	
\\	(かれ). 
\\	紹	
\\	しょう	
\\	糸, 召	
\\	(しょう) 
\\	介	
\\	かい	
\\	介	
\\	(かい).	
\\	湖	
\\	こ	みずうみ	氵, 古, 月	
\\	子 (こ).
\\	子 
\\	子 
\\	講	
\\	こう	
\\	言, 冓	
\\	こういち! 
\\	こういち 
\\	こう.
\\	寿	
\\	じゅ, す	ことぶき	寸		
\\	(じゅ) 
\\	測	
\\	そく	はか	氵, 貝, 刂	
\\	(そく). 
\\	互	
\\	ご	たが	一, 彑	
\\	(ご) 
\\	油	
\\	ゆ	あぶら	氵, 由	
\\	(ゆ). 
\\	己	
\\	こ, き	おのれ	己	
\\	子 (こ) 
\\	払	
\\	ふつ	はら	扌, ム	
\\	(はら)! 
\\	鍋	
\\	か	なべ	金, 冋	
\\	(なべ),” 
\\	獄	
\\	ごく	
\\	犭, 言, 犬	
\\	(ごく). 
\\	為	
\\	い	ため, な, す	為	
\\	(い) 
\\	恥	
\\	ち	は, はじ	耳, 心	
\\	(はじ). 
\\	(は) 
\\	遅	
\\	ち	おそ, おく	尸, 羊		
\\	(ち) 
\\	汁	
\\	じゅう	しる	氵, 十	
\\	(じゅう) 
\\	醤	
\\	しょう	
\\	将, 酉	
\\	(しょう) 
\\	滞	
\\	たい	とどこお	帯, 氵	
\\	(たい). 
\\	剣	
\\	けん	つるぎ	刂		
\\	(けん) 
\\	破	
\\	は	やぶ	石, 皮	
\\	(は)!
\\	亀	
\\	き	かめ	亀	
\\	(かめ). 
\\	厄	
\\	やく	
\\	厂, 巳	
\\	(やく) 
\\	酢	
\\	さく	す	酉, 乍	
\\	(す). 
\\	諾	
\\	だく	
\\	言, 艹, 右	
\\	(だく) 
\\	盟	
\\	めい	
\\	明, 皿	
\\	(めい). 
\\	将	
\\	しょう	
\\	将	
\\	(しょう)! 
\\	舞	
\\	ぶ	まい, ま	無, 舛	
\\	(ぶ). 
\\	債	
\\	さい	
\\	イ, 責	
\\	(さい) 
\\	伎	
\\	き	わざ	イ, 支	
\\	(き). 
\\	鹿	
\\	ろく	か, しか	广, 覀, 比	
\\	(か) 
\\	換	
\\	かん	か	扌, 勹, 口, 儿, 大	
\\	(かん). 
\\	牙	
\\	げ, が	きば	牙	
\\	(げ)! 
\\	旧	
\\	きゅう	
\\	日		
\\	(きゅう).	
\\	般	
\\	はん	
\\	舟, 殳	
\\	(はん) 
\\	津	
\\	しん	つ	氵, 聿	
\\	(つ) 
\\	療	
\\	りょう	
\\	疒, 尞	
\\	(りょう) 
\\	継	
\\	けい	つ	糸, 
\\	米	
\\	(けい), 
\\	遺	
\\	い	のこ	貴		
\\	(い). 
\\	維	
\\	い	
\\	糸, 隹	
\\	(い) 
\\	奈	
\\	な	
\\	大, 示	
\\	(な). 
\\	核	
\\	かく	
\\	木, 亥	
\\	(かく). 
\\	廃	
\\	はい	すた	广, 発	
\\	(はい). 
\\	献	
\\	けん, こん	たてまつ	南, 犬	
\\	(けん) 
\\	沖	
\\	ちゅう	おき	氵, 中	
\\	(おき) 
\\	摘	
\\	てき	つ	扌, 啇	
\\	(てき), 
\\	及	
\\	きゅう	およ	及	
\\	(きゅう).	
\\	依	
\\	い	よ	イ, 衣	
\\	(い) 
\\	縄	
\\	じょう	なわ	糸, 亀	
\\	(じょう). 
\\	踏	
\\	とう	ふ	足, 水, 日	
\\	とうきょう. 
\\	とうきょう, 
\\	とうきょう. 
\\	伸	
\\	しん	の	イ, 申	
\\	(の) 
\\	姓	
\\	せい, しょう	
\\	女, 生	
\\	(せい) 
\\	甘	
\\	かん	あま	甘	
\\	(あま) 
\\	貿	
\\	ぼう	
\\	ム, 刀, 貝	
\\	(ぼう). 
\\	頼	
\\	らい	たの, たよ	束, 頁	
\\	(らい) 
\\	超	
\\	ちょう	こ	走, 召	
\\	(ちょう) 
\\	幅	
\\	ふく	はば	巾, 畐	
\\	(はば) 
\\	患	
\\	かん	わずら	串, 心	
\\	(かん) 
\\	狙	
\\	そ	ねら	犭, 且	
\\	(そ) 
\\	陣	
\\	じん	
\\	阝, 車	
\\	(じん). 
\\	塁	
\\	るい	
\\	田, 土		
\\	(るい). 
\\	弾	
\\	だん	ひ, はず, たま	弓, 単	
\\	(だん). 
\\	葬	
\\	そう	ほうむ	艹, 歹, 匕, 廾	
\\	(そう) 
\\	抗	
\\	こう	あらが	扌, 亠, 几	
\\	こういち. 
\\	こういち! 
\\	こういち 
\\	崩	
\\	ほう	くず	山, 月	
\\	(ほう), 
\\	遣	
\\	けん	つか, や	虫	
\\	(けん) 
\\	掲	
\\	けい	かか	扌, 日, 勹, 匕	
\\	(けい). 
\\	爆	
\\	ばく	は	火, 暴	
\\	(ばく) 
\\	眉	
\\	み	まゆ	尸, 目		
\\	(み). 
\\	恵	
\\	え, けい	めぐ	十, 田, 心	
\\	(え)? 
\\	漁	
\\	ぎょ, りょう	あさ	氵, 魚	
\\	(ぎょ) 
\\	(りょう) 
\\	魚.	
\\	香	
\\	こう, きょう	かお, か	禾, 日	
\\	こういち 
\\	こういち
\\	こういち
\\	湾	
\\	わん	
\\	氵, 赤, 弓	
\\	(わん). 
\\	跳	
\\	ちょう	と, は	足, 兆	
\\	(ちょう) 
\\	抱	
\\	ほう	だ, かか, いだ	扌, 包	
\\	(だ)! 
\\	旬	
\\	しゅん, じゅん	
\\	勹, 日	
\\	(しゅん), 
\\	(じゅん) 
\\	聴	
\\	ちょう	き	耳, 十, 罒, 心	
\\	(ちょう), 
\\	臨	
\\	りん	のぞ	臣, 品		
\\	(りん). 
\\	兆	
\\	ちょう	きざ	兆	
\\	(ちょう) 
\\	契	
\\	けい	
\\	生, 刀, 大	
\\	(けい), 
\\	刑	
\\	けい	
\\	开, 刂	
\\	(けい)! 
\\	募	
\\	ぼ	つの	莫, 力	
\\	(ぼ) 
\\	償	
\\	しょう	つぐな	イ, 賞	
\\	(しょう) 
\\	抵	
\\	てい	
\\	扌, 氏, 一	
\\	(てい) 
\\	戻	
\\	れい	もど	戸, 大	
\\	(もど). 
\\	昭	
\\	しょう	
\\	日, 召	
\\	(しょう). 
\\	串	
\\	くし	串	
\\	(くし). 
\\	闘	
\\	とう	たたか	門, 豆, 寸	
\\	とうきょう. 
\\	とうきょう, 
\\	とうきょう!
\\	執	
\\	しゅう, しつ	と	幸, 丸	
\\	(しつ) 
\\	(しゅう). 
\\	跡	
\\	せき	あと	足, 赤	
\\	(せき) 
\\	削	
\\	さく	けず	月, 刂		
\\	(さく).	
\\	伴	
\\	はん	ともな	イ, 半	
\\	(はん) 
\\	齢	
\\	れい	よわい	歯, 令	
\\	(れい) 
\\	れい.	
\\	宜	
\\	ぎ	よろ	宀, 且	
\\	(よろ) 
\\	賂	
\\	ろ	
\\	貝, 各	
\\	(ろ). 
\\	賄	
\\	わい	まかな	貝, 有	
\\	(わい)?!
\\	房	
\\	ぼう	ふさ	戸, 方	
\\	(ぼう) 
\\	慮	
\\	りょ	おもんぱく, おもんぱか	虍, 思	
\\	(りょ), 
\\	託	
\\	たく	かこ	言, 丿, 七	
\\	(たく). 
\\	却	
\\	きゃく	かえって	去, 卩	
\\	(きゃく) 
\\	需	
\\	じゅ	
\\	雨, 而	
\\	(じゅ)! 
\\	致	
\\	ち	いた	至, 夂	
\\	(ち).	
\\	避	
\\	ひ	さ, よ	辟		
\\	(ひ) 
\\	描	
\\	びょう	か, えが	扌, 艹, 田	
\\	(びょう). 
\\	刊	
\\	かん	
\\	干, 刂	
\\	(かん) 
\\	逃	
\\	とう	に, のが, の	兆		
\\	とうきょう. 
\\	とうきょう 
\\	とうきょう 
\\	扱	
\\	きゅう	あつか	扌, 及	
\\	""暑か?
\\	(あつか). 
\\	暑か?! 暑か?!?! 暑か?!?!?!” 
\\	暑か? 
\\	奥	
\\	おう	おく	丶, 冂, 米, 大	
\\	(おく) 
\\	併	
\\	へい	あわ	イ, 开		
\\	(へい). 
\\	膝	
\\	ひざ	月, 木, 水		
\\	(ひざ)! 
\\	傾	
\\	けい	かたむ	化, 頁	
\\	(けい) 
\\	緩	
\\	かん	ゆる	糸, 爰	
\\	(ゆる) 
\\	奏	
\\	そう	かな	天		
\\	(そう) 
\\	娠	
\\	しん	
\\	女, 辰	
\\	(しん). 
\\	妊	
\\	にん	
\\	女, 王	
\\	(にん). 
\\	贈	
\\	ぞう	おく	貝, 曽	
\\	(ぞう). 
\\	択	
\\	たく	えら	扌, 尺	
\\	(たく)!	
\\	還	
\\	かん	かえ	罒	
\\	(かん). 
\\	繰	
\\	そう	く	糸, 喿	
\\	(く)! 
\\	抑	
\\	よく	おさ	扌, 卬	
\\	(よく). 
\\	懸	
\\	けん	か	県, 系, 心	
\\	(けん) 
\\	称	
\\	しょう	とな, たた, ほめ	禾, 小		
\\	(しょう) 
\\	緒	
\\	しょ	お	糸, 者	
\\	(しょ). 
\\	盤	
\\	ばん	
\\	舟, 殳, 皿	
\\	(ばん) 
\\	控	
\\	こう	ひか	扌, 空	
\\	(ひか) 
\\	充	
\\	じゅう	あ, み	充	
\\	(じゅう) 
\\	渋	
\\	じゅう	しぶ	氵, 止		
\\	(じゅう) 
\\	岐	
\\	き, ぎ	
\\	山, 支	
\\	(き). 
\\	(ぎ).	
\\	埋	
\\	まい	う	土, 里	
\\	(う) 
\\	鈴	
\\	りん	すず	金, 令	
\\	(りん)! 
\\	埼	
\\	き	さい	土, 奇	
\\	(さい). 
\\	棋	
\\	き	ご	木, 其	
\\	(き), 
\\	譲	
\\	じょう	ゆず	言, 㐮	
\\	(じょう), 
\\	雇	
\\	こ	やと	隹, 戸	
\\	子 (こ) 
\\	免	
\\	めん	まぬか	免	
\\	(めん)! 
\\	群	
\\	ぐん	む, むら	君, 羊	
\\	(ぐん) 
\\	枠	
\\	わく	木, 九, 十	
\\	(わく) 
\\	銃	
\\	じゅう	
\\	金, 充	
\\	(じゅう)! 
\\	仙	
\\	せん	
\\	イ, 山	
\\	せん. 
\\	邦	
\\	ほう	くに	三, 丿, 阝	
\\	(ほう)!	
\\	御	
\\	ご	お	彳, 正, 卩		
\\	(ご). 
\\	慎	
\\	しん	つつし	忄, 真	
\\	(しん). 
\\	躍	
\\	やく	おど	足, ヨ, 隹	
\\	(やく) 
\\	謙	
\\	けん	
\\	言, 兼	
\\	(けん) 
\\	阜	
\\	ふ	
\\	丶, 十		
\\	(ふ) 
\\	片	
\\	へん	かた	片	
\\	(かた).	
\\	項	
\\	こう	
\\	工, 頁	
\\	こういち, 
\\	こういち 
\\	こういち 
\\	工 
\\	こう 
\\	斐	
\\	い	
\\	非, 文	
\\	(い). 
\\	隆	
\\	りゅう	
\\	阝, 夂, 生	
\\	(りゅう) 
\\	圏	
\\	けん	
\\	口, 巻	
\\	(けん) 
\\	勧	
\\	かん	すす	矢, 隹, 力	
\\	(かん) 
\\	拒	
\\	きょ	こば	扌, 巨	
\\	(きょ). 
\\	稲	
\\	いね, いな	禾, 旧		
\\	""いな、いね.
\\	いな、いね 
\\	奪	
\\	だつ	うば	大, 隹, 寸	
\\	(うば) 
\\	鋼	
\\	こう	はがね	金, 岡	
\\	こういち. 
\\	甲	
\\	こう, かん	か	甲	
\\	こういち.	
\\	こういち 
\\	壁	
\\	へき	かべ	土, 辟	
\\	(かべ) 
\\	祉	
\\	し	
\\	ネ, 止	
\\	(し). 
\\	敏	
\\	びん	
\\	毎, 夂	
\\	(びん) 
\\	吹	
\\	すい	ふ	口, 欠	
\\	(ふ), 
\\	唱	
\\	しょう	とな	口, 日	
\\	(しょう) 
\\	衝	
\\	しょう	つ	行, 重	
\\	(しょう)! 
\\	戒	
\\	かい	いまし	戈, 廾	
\\	(かい).	
\\	兼	
\\	けん	か	兼	
\\	(けん) 
\\	薄	
\\	はく	うす	艹, 氵, 専	
\\	(はく). 
\\	堀	
\\	くつ	ほり	土, 尸, 出	
\\	(ほり) 
\\	剤	
\\	ざい	
\\	斉, 刂	
\\	(ざい) 
\\	雅	
\\	が	みや	牙, 隹	
\\	(が) 
\\	孝	
\\	こう	
\\	孝	
\\	こういち.	
\\	こういち, 
\\	頻	
\\	ひん	しき	歩, 頁	
\\	(ひん) 
\\	駆	
\\	く	か	馬, 区	
\\	(か). 
\\	俊	
\\	しゅん	
\\	イ, 夋	
\\	(しゅん) 
\\	嬉	
\\	き	うれ	女, 喜	
\\	(き)!
\\	誉	
\\	よ	ほ	一, ハ, 言		
\\	(よ). 
\\	茂	
\\	も	しげ	艹, 丿, 戈	
\\	(も) 
\\	殿	
\\	でん	との, どの	尸, 共, 殳	
\\	(でん)!
\\	殖	
\\	しょく	ふ	歹, 直	
\\	(しょく). 
\\	隣	
\\	りん	となり	阝, 米, 舛	
\\	(りん) 
\\	繁	
\\	はん	しげ	毎, 夂, 糸	
\\	(はん) 
\\	巡	
\\	じゅん	めぐ	巛		
\\	(じゅん). 
\\	柱	
\\	ちゅう	はしら	木, 主	
\\	(ちゅう). 
\\	携	
\\	けい	たずさ	扌, 隹, 乃	
\\	(けい) 
\\	褒	
\\	ほう	ほ	亠, 保		
\\	(ほう), 
\\	排	
\\	はい	
\\	扌, 非	
\\	(はい) 
\\	駐	
\\	ちゅう	
\\	馬, 主	
\\	(ちゅう). 
\\	顧	
\\	こ	かえり	雇, 頁	
\\	子 (こ)!	
\\	犠	
\\	ぎ	
\\	牛, 義	
\\	(ぎ). 
\\	義. 
\\	獲	
\\	かく	え	犭, 艹, 隹, 又	
\\	(かく). 
\\	鋭	
\\	えい	するど	金, 兑	
\\	(えい) 
\\	敷	
\\	ふ	しき, し	十, 田, 方, 夂	
\\	(しき) 
\\	し 
\\	し, 
\\	しき, 
\\	透	
\\	とう	す	禾, 乃		
\\	とうきょう. 
\\	とうきょう 
\\	棄	
\\	き	
\\	果		
\\	(き) 
\\	凄	
\\	せい	すご	冫, 妻	
\\	(せい) 
\\	至	
\\	し	いた	至	
\\	(し) 
\\	拠	
\\	きょ	よ	扌, 夂, 几	
\\	(きょ). 
\\	蜂	
\\	ほう	はち	虫, 夆	
\\	(はち) 
\\	儀	
\\	ぎ	
\\	イ, 義	
\\	(ぎ).	
\\	炭	
\\	たん	すみ	山, 厂, 火	
\\	(たん). 
\\	衣	
\\	い, え	ころも, きぬ	衣	
\\	(い) 
\\	(え) 
\\	潜	
\\	せん	くぐ, ひそ, もぐ	氵, 夫, 日	
\\	(せん). 
\\	偽	
\\	ぎ	にせ, いつわ	イ, 為	
\\	(ぎ) 
\\	畑	
\\	はたけ, はた	火, 田	
\\	(はたけ) 
\\	はた 
\\	(はたけ) 
\\	蛍	
\\	けい	ほたる	虫		
\\	(ほたる) 
\\	拳	
\\	けん, げん	こぶし	手		
\\	(けん) 
\\	郷	
\\	きょう	
\\	幺, 郎	
\\	きょうとう 
\\	きょうとう 
\\	きょうとう. 
\\	蜜	
\\	みつ	
\\	宀, 必, 虫	
\\	(みつ) 
\\	仁	
\\	じん	
\\	イ, 二	
\\	(じん). 
\\	遜	
\\	そん	したが	子, 系		
\\	(そん). 
\\	侵	
\\	しん	おか	イ, ヨ, 冖, 又	
\\	(しん). 
\\	嘘	
\\	うそ	口, 虍		
\\	(うそ) 
\\	鉱	
\\	こう	あらがね	金, 広	
\\	こういち.	
\\	こういち 
\\	鋼, 
\\	伺	
\\	し	うかが	イ, 司	
\\	(うかが). 
\\	徹	
\\	てつ	
\\	彳, 月, 夂		
\\	鉄 (てつ).	
\\	瀬	
\\	らい	せ	氵, 束, 頁	
\\	(せ) 
\\	墟	
\\	きょ	
\\	土, 虍		
\\	(きょ), 
\\	酎	
\\	ちゅう, ちゅ	かも	酉, 寸	
\\	(ちゅう). 
\\	措	
\\	そ	
\\	扌, 昔	
\\	(そ). 
\\	誠	
\\	せい	まこと	言, 成	
\\	(せい) 
\\	虎	
\\	こ	とら	虍, 儿	
\\	(とら). 
\\	艦	
\\	かん	
\\	舟, 監	
\\	(かん), 
\\	撤	
\\	てつ	
\\	扌, 月, 夂		
\\	鉄 (てつ) 
\\	鉄 
\\	鉄 
\\	樹	
\\	じゅ	き	木, 壴, 寸	
\\	(じゅ) 
\\	包	
\\	ほう	つつ, くる	包	
\\	(ほう). 
\\	析	
\\	せき	
\\	木, 斤	
\\	(せき). 
\\	弧	
\\	こ	
\\	弓, 瓜	
\\	子
\\	(こ) 
\\	子 
\\	子 
\\	到	
\\	とう	
\\	至, 刂	
\\	とうきょう 
\\	とうきょう 
\\	とうきょう.	
\\	軸	
\\	じく	
\\	車, 由	
\\	(じく) 
\\	綱	
\\	こう	つな	糸, 岡	
\\	(つな)! 
\\	挑	
\\	ちょう	いど	扌, 兆	
\\	(ちょう) 
\\	焦	
\\	しょう	こ, あせ	隹, 灬	
\\	(しょう) 
\\	掘	
\\	くつ	ほ	扌, 尸, 出	
\\	(くつ). 
\\	紛	
\\	ふん	まぎ, まぐ	糸, 分	
\\	(ふん).	
\\	範	
\\	はん	
\\	竹, 車, 巳	
\\	(はん) 
\\	括	
\\	かつ	くく	扌, 舌	
\\	(かつ). 
\\	床	
\\	しょう	ゆか, とこ	广, 木	
\\	(しょう), 
\\	握	
\\	あく	にぎ	扌, 屋	
\\	(あく)! 
\\	枢	
\\	すう	からくり	木, 区	
\\	(すう) 
\\	揚	
\\	よう	あ	扌, 易	
\\	(よう). 
\\	潟	
\\	せき	かた	氵, 日, 勿	
\\	(かた). 
\\	芝	
\\	し	しば	艹		
\\	(しば) 
\\	肝	
\\	かん	きも	月, 干	
\\	(かん) 
\\	喪	
\\	そう	も	十, 口, 衣	
\\	(そう) 
\\	網	
\\	もう	あみ	糸, 岡	
\\	綱, 
\\	綱 
\\	(もう) 
\\	克	
\\	こく	
\\	古, 儿	
\\	(こく) 
\\	泊	
\\	はく	と	氵, 白	
\\	(はく) 
\\	双	
\\	そう	ふた	又	
\\	(そう). 
\\	柄	
\\	へい	がら	木, 丙	
\\	(がら). 
\\	哲	
\\	てつ	
\\	扌, 斤, 口	
\\	鉄 (てつ). 
\\	鉄 
\\	斎	
\\	さい	いつ	斉, 示	
\\	(さい). 
\\	袋	
\\	たい	ふくろ	代, 衣	
\\	(ふくろ) 
\\	揺	
\\	よう	ゆ	扌		
\\	(よう). 
\\	滑	
\\	かつ	すべ, なめ	氵, 骨	
\\	(かつ) 
\\	堅	
\\	けん	かた	臣, 又, 土	
\\	(かた). 
\\	暫	
\\	ざん	しばら	車, 斤, 日	
\\	(ざん) 
\\	糾	
\\	きゅう	
\\	糸, 丩	
\\	(きゅう) 
\\	荒	
\\	こう	あ, あら	艹, 亡, 川	
\\	(あ), 
\\	(あら) 
\\	襲	
\\	しゅう	おそ	龍, 衣	
\\	(しゅう) 
\\	沼	
\\	しょう	ぬま	氵, 召	
\\	(ぬま) 
\\	朗	
\\	ろう	ほが	良, 月	
\\	(ろう) 
\\	摩	
\\	ま	さす	广, 木, 手	
\\	(ま), 
\\	懲	
\\	ちょう	こ	彳, 山, 王, 夂, 心	
\\	(ちょう).	
\\	潮 
\\	慰	
\\	い	なぐさ	尸, 示, 寸, 心	
\\	懇	
\\	こん	
\\	豸, 艮, 心	
\\	(こん). 
\\	筒	
\\	とう	つつ	竹, 同	
\\	とうきょう. 
\\	とうきょう!	とうきょう 
\\	滅	
\\	めつ	ほろ	氵, 丿, 戈, 一, 火	
\\	(めつ) 
\\	(めつ) 
\\	距	
\\	きょ	
\\	足, 巨	
\\	(きょ) 
\\	籍	
\\	せき	
\\	竹, 耒, 昔	
\\	(せき). 
\\	露	
\\	ろ, ろう	つゆ	雨, 足, 各	
\\	(ろ), 
\\	炉	
\\	ろ	いろり	火, 戸	
\\	(ろ).	
\\	柔	
\\	じゅう, にゅう	やわ	矛, 木	
\\	(じゅう) 
\\	趣	
\\	しゅ	おもむき	走, 耳, 又	
\\	(しゅ). 
\\	擦	
\\	さつ	こす, す	扌, 宀, 祭	
\\	(さつ) 
\\	琴	
\\	こと	王, 今	
\\	事 (こと, 
\\	事 
\\	事 
\\	垣	
\\	かき	土, 一, 旦	
\\	(かき) 
\\	即	
\\	そく	すなわ	艮, 卩	
\\	(そく)! 
\\	威	
\\	い	
\\	丿, 戈, 一, 女	
\\	(い) 
\\	滋	
\\	じ	
\\	氵, 一, 幺		
\\	(じ) 
\\	牧	
\\	ぼく	まき	牛, 夂	
\\	"ぼく 
\\	""ぼく 
\\	泰	
\\	たい	
\\	水		
\\	(たい) 
\\	岳	
\\	がく	たけ	斤, 一, 山	
\\	(がく) 
\\	旨	
\\	し	うま, むね	匕, 日	
\\	(し)!	
\\	刷	
\\	さつ	す	尸, 巾, 刂	
\\	(さつ) 
\\	珍	
\\	ちん	めずら	王, 彡		
\\	(ちん).	
\\	封	
\\	ふう, ほう	
\\	土, 寸	
\\	(ふう) 
\\	ほう. 
\\	(ほう), 
\\	斉	
\\	せい, さい	
\\	斉	
\\	(せい).	
\\	沈	
\\	ちん	しず	氵, 冘	
\\	(ちん) 
\\	撲	
\\	ぼく	
\\	扌, 業	
\\	"ぼく 
\\	ぼく.
\\	ぼく. 
\\	裂	
\\	れつ	さ	歹, 刂, 衣	
\\	(れつ) 
\\	潮	
\\	ちょう	しお	氵, 月		
\\	(ちょう)!	
\\	貢	
\\	こう	みつ	工, 貝	
\\	こういち.	
\\	こういち 
\\	工 
\\	こう 
\\	誰	
\\	だれ	言, 隹	
\\	(だれ) 
\\	刃	
\\	じん, にん	は	刃	
\\	(は)!
\\	缶	
\\	かん	
\\	缶	
\\	かん, 
\\	(かん) 
\\	かん 
\\	砲	
\\	ほう	
\\	石, 包	
\\	(ほう). 
\\	笠	
\\	かさ	竹, 立	
\\	(かさ)!
\\	竜	
\\	りゅう	たつ	竜	
\\	(りゅう).	
\\	拶	
\\	さつ	
\\	扌, 巛, 夕	
\\	(さつ) 
\\	縁	
\\	えん, ねん	ふち	糸, ヨ, 豕	
\\	(えん).	
\\	忍	
\\	にん	しの	刃, 心	
\\	(にん) 
\\	釣	
\\	ちょう	つ	金, 勺	
\\	(つ) 
\\	吉	
\\	きつ, きち	よし	吉	
\\	(きつ) 
\\	(きち). 
\\	粒	
\\	りゅう	つぶ	米, 立	
\\	(りゅう) 
\\	髪	
\\	はつ	かみ	長, 彡, 友	
\\	神 (かみ) 
\\	神. 
\\	丘	
\\	きゅう	おか	斤, 一	
\\	(きゅう). 
\\	俺	
\\	おれ	イ, 大, 田, 乚	
\\	俺 
\\	(おれ).	
\\	斗	
\\	と	
\\	斗	
\\	(と)! 
\\	寸	
\\	すん	
\\	寸	
\\	(すん).	
\\	桃	
\\	とう	もも	木, 兆	
\\	(もも) 
\\	(もも).
\\	梨	
\\	なし	禾, 刂, 木	
\\	(なし) 
\\	姫	
\\	ひめ	女, 臣	
\\	(ひめ) 
\\	挨	
\\	あい	
\\	扌, ム, 矢	
\\	(あい). 
\\	娯	
\\	ご	
\\	女, 呉	
\\	(ご).	
\\	謎	
\\	なぞ	言, 米		
\\	(なぞ)! 
\\	侍	
\\	さむらい	イ, 寺	
\\	叱	
\\	しか	口, 七	
\\	鹿 (しか). 
\\	棚	
\\	ほう	たな	木, 月	
\\	(たな) 
\\	叫	
\\	きょう	さけ	口, 丩	
\\	""きょうと!
\\	きょうと 
\\	きょう 
\\	きょうと!	
\\	匹	
\\	ひき	匚, 儿	
\\	(ひき).	
\\	辛	
\\	しん	から, つら	辛	
\\	(しん) 
\\	芽	
\\	が	め	艹, 牙	
\\	目 (め), 
\\	目
\\	嵐	
\\	あらし	山, 風	
\\	(あらし).	
\\	涙	
\\	るい	なみだ	氵, 戸, 大	
\\	(るい) 
\\	雷	
\\	らい	かみなり	雨, 田	
\\	(らい) 
\\	(らい).	
\\	塔	
\\	とう	
\\	土, 艹, 合	
\\	とうきょう. とうきょう 
\\	とうきょう 
\\	とうきょう 
\\	とうきょう 
\\	朱	
\\	しゅ	あけ	丿, 未	
\\	(しゅ). 
\\	翼	
\\	よく	つばさ	羽, 田, 共	
\\	(よく). 
\\	頃	
\\	ころ, ごろ	匕, 頁	
\\	(ころ) 
\\	菌	
\\	きん	
\\	艹, 口, 禾	
\\	(きん) 
\\	鐘	
\\	しょう	かね	金, 立, 里	
\\	(しょう) 
\\	舟	
\\	しゅう	ふね, ふな	舟	
\\	船, 
\\	ふな. 
\\	ふね.	
\\	ふな 
\\	嫁	
\\	か	よめ, とつ	女, 家	
\\	(よめ), 
\\	暦	
\\	れき	こよみ	厂, 木, 日	
\\	歴史, 
\\	歴史
\\	歴史 
\\	曇	
\\	くも	日, 雨, 云	
\\	雲? 
\\	くも.
\\	也	
\\	や	なり	也	
\\	(なり) 
\\	(なり) 
\\	塾	
\\	じゅく	
\\	享, 丸, 土	
\\	(じゅく) 
\\	呪	
\\	じゅ	のろ	口, 兄	
\\	(のろ) 
\\	湿	
\\	しつ	しめ	氵, 日		
\\	(しつ) 
\\	稼	
\\	か	かせ	禾, 家	
\\	(か). 
\\	疲	
\\	ひ	つか	疒, 皮	
\\	(ひ) 
\\	(ひ) 
\\	翔	
\\	しょう	かけ, と	羊, 羽	
\\	(かけ) 
\\	賭	
\\	か	貝, 者	
\\	(か)!	
\\	霊	
\\	れい, りょう	
\\	雨, 二		
\\	(れい) 
\\	(りょう) 
\\	溝	
\\	こう	みぞ	氵, 冓	
\\	こういち 
\\	こういち 
\\	こう.
\\	狩	
\\	しゅ	か	犭, 守	
\\	(か). 
\\	脚	
\\	きゃく	あし	月, 去, 卩	
\\	(きゃく) 
\\	澄	
\\	ちょう	す	氵, 癶, 豆	
\\	(ちょう), 
\\	塊	
\\	かい	かたまり	土, 鬼	
\\	(かたまり) 
\\	狂	
\\	きょう	くる	犭, 王	
\\	きょうと. 
\\	きょうと, 
\\	嬢	
\\	じょう	むすめ	女, 㐮	
\\	(じょう),
\\	裸	
\\	ら	はだか	ネ, 果	
\\	(ら).	
\\	磨	
\\	ま	みが	广, 木, 石	
\\	(みが) 
\\	陰	
\\	いん	かげ	阝, 今, 云	
\\	(いん) 
\\	肌	
\\	き	はだ	月, 几	
\\	(はだ) 
\\	魂	
\\	こん	たましい	云, 鬼	
\\	(こん). 
\\	矛	
\\	む	ほこ	矛	
\\	(む)!
\\	眺	
\\	ちょう	なが	目, 兆	
\\	(ちょう) 
\\	硬	
\\	こう	かた	石, 更	
\\	こういち 
\\	こういち 
\\	卓	
\\	たく	
\\	ト, 早	
\\	(たく).	
\\	凶	
\\	きょう	
\\	凶	
\\	きょうと 
\\	きょうと! 
\\	滝	
\\	たき	氵, 竜	
\\	(たき).	
\\	井	
\\	しょう	い	井	
\\	(しょう). 
\\	墨	
\\	すみ	黒, 土	
\\	(すみ)!
\\	瞬	
\\	しゅん	またた	目, 冖, 舛		
\\	(しゅん) 
\\	泡	
\\	ほう	あわ	氵, 包	
\\	(ほう), 
\\	穏	
\\	おん	おだ	禾, ヨ, 心		
\\	(おん) 
\\	孔	
\\	こう	あな	子, 乚	
\\	こういち.
\\	こういち 
\\	椅	
\\	い	
\\	木, 奇	
\\	(い). 
\\	菊	
\\	きく	
\\	艹, 勹, 米	
\\	(きく) 
\\	涼	
\\	りょう	すず	氵, 京	
\\	(りょう) 
\\	綿	
\\	めん	わた	糸, 白, 巾	
\\	(めん).	
\\	魔	
\\	ま	
\\	广, 木, 鬼	
\\	(ま).	
\\	寮	
\\	りょう	
\\	宀, 尞	
\\	(りょう) 
\\	鳩	
\\	く	はと	九, 鳥	
\\	(はと) 
\\	鈍	
\\	どん	にぶ, のろ	金, 屯	
\\	(どん) 
\\	鍛	
\\	たん	きた	金, 殳		
\\	(たん).	
\\	碁	
\\	ご	
\\	其, 石	
\\	(ご) 
\\	癖	
\\	へき	くせ	疒, 辟	
\\	(くせ). 
\\	穂	
\\	すい	ほ	禾, 恵	
\\	(ほ). 
\\	吾	
\\	ご	わが	五, 口	
\\	(わが). 
\\	鍵	
\\	けん	かぎ	金, 廴, 聿	
\\	(かぎ) 
\\	盆	
\\	ぼん	
\\	分, 皿	
\\	(ぼん) 
\\	庄	
\\	しょう	
\\	广, 土	
\\	猿	
\\	えん	さる	犭, 土, 口		
\\	(さる) 
\\	棟	
\\	とう	
\\	木, 東	
\\	とうきょう. 
\\	とうきょう 
\\	とうきょう 
\\	誇	
\\	こ	ほこ	言, 大, 一		
\\	子 (こ) 
\\	瞳	
\\	とう, どう	ひとみ	目, 立, 里	
\\	(どう) 
\\	寧	
\\	ねい	むし	宀, 心, 罒, 丁	
\\	(ねい) 
\\	俵	
\\	ひょう	たわら	イ, 生		
\\	(ひょう)! 
\\	幽	
\\	ゆう	
\\	幺, 山	
\\	(ゆう). 
\\	架	
\\	か	か	力, 口, 木	
\\	(か) 
\\	黙	
\\	もく	だま	里, 犬, 灬	
\\	(もく) 
\\	斬	
\\	ざん	き	車, 斤	
\\	(ざん) 
\\	帝	
\\	てい	みかど	立, 巾	
\\	(てい) 
\\	租	
\\	そ	
\\	禾, 且	
\\	(そ).	
\\	錬	
\\	れん	ね	金, 東	
\\	(れん). 
\\	練, 
\\	阻	
\\	そ	はば	阝, 且	
\\	(そ) 
\\	歳	
\\	さい, せい	
\\	止, 丿, 戈, 示	
\\	(せい) 
\\	(さい)!
\\	零	
\\	れい	こぼ	雨, 令	
\\	(れい) 
\\	れい.
\\	幣	
\\	へい	
\\	敝, 巾	
\\	(へい) 
\\	箸	
\\	ちゃく	はし	竹, 者	
\\	(はし) 
\\	瞭	
\\	りょう	
\\	目, 尞	
\\	(りょう) 
\\	崖	
\\	がい	がけ	山, 厂, 土	
\\	(がい)!” 
\\	炊	
\\	すい	た	火, 欠	
\\	(すい) 
\\	粧	
\\	しょう	
\\	米, 广, 土	
\\	(しょう) 
\\	墜	
\\	つい	
\\	阝, 豕, 土		
\\	(つい)!
\\	欺	
\\	ぎ	あざむ	其, 欠	
\\	(ぎ).	
\\	滴	
\\	てき	したた, しずく	氵, 啇	
\\	(てき), 
\\	塀	
\\	へい	
\\	土, 尸, 开		
\\	(へい).	
\\	霧	
\\	む	きり	雨, 矛, 夂, 力	
\\	(きり). 
\\	扇	
\\	せん	おうぎ, あお	戸, 羽	
\\	(せん). 
\\	扉	
\\	ひ	とびら	戸, 非	
\\	(ひ). 
\\	恨	
\\	こん	うら	忄, 艮	
\\	(こん). 
\\	帽	
\\	ぼう	
\\	巾, 日, 目	
\\	(ぼう) 
\\	憎	
\\	ぞう	にく	忄, 曽	
\\	(ぞう).	
\\	佐	
\\	さ	
\\	イ, ナ, 工	
\\	(さ) 
\\	挿	
\\	そう	さ	扌, 千, 日	
\\	(そう) 
\\	伊	
\\	い	だ	イ, ヨ, 丿	
\\	(い) 
\\	詐	
\\	さ	いつわ	言, 乍	
\\	(さ). 
\\	如	
\\	じょ	ごと	女, 口	
\\	(じょ) 
\\	唇	
\\	しん	くちびる	辰, 口	
\\	(くちびる). 
\\	掌	
\\	しょう	てのひら	口, 手		
\\	(しょう)! 
\\	婆	
\\	ば	ばあ	波, 女	
\\	(ば) 
\\	哀	
\\	あい	あわ	亠, 口		
\\	(あい). 
\\	虹	
\\	こう	にじ	虫, 工	
\\	(にじ). 
\\	爽	
\\	そう	さわ	大, メ	
\\	(さわ) 
\\	憩	
\\	けい	いこ	舌, 自, 心	
\\	(けい)! 
\\	尺	
\\	しゃく	
\\	尺	
\\	(しゃく) 
\\	砕	
\\	さい	くだ	石, 九, 十	
\\	(さい) 
\\	粘	
\\	ねん	ねば	米, 占	
\\	(ねん) 
\\	畳	
\\	じょう	たたみ	田, 冖, 且	
\\	(じょう) 
\\	胴	
\\	どう	
\\	月, 同	
\\	(どう) 
\\	巾	
\\	きん	
\\	巾	
\\	(きん) 
\\	きん 
\\	きん 
\\	芯	
\\	しん	
\\	艹, 心	
\\	(しん). 
\\	柳	
\\	りゅう	やなぎ	木, 卩	
\\	(りゅう) 
\\	遂	
\\	すい	と, つい	豕	
\\	(すい), 
\\	蓄	
\\	ちく	たくわ	艹, 玄, 田	
\\	(ちく) 
\\	脇	
\\	きょう	わき	月, 力	
\\	(わき) 
\\	殴	
\\	おう	なぐ	区, 殳	
\\	王
\\	(おう) 
\\	王 
\\	王 
\\	咲	
\\	しょう	さ	口, 天		
\\	(さ) 
\\	鉢	
\\	はち	
\\	金, 本	
\\	(はち) 
\\	賢	
\\	けん	かしこ	臣, 又, 貝	
\\	(けん) 
\\	彩	
\\	さい	いろど	木, 彡		
\\	(さい). 
\\	隙	
\\	げき	すき	阝, 小, 日	
\\	(すき).	
\\	培	
\\	ばい	つちか	土, 咅	
\\	(ばい) 
\\	踊	
\\	よう	おど	足, マ, 用	
\\	(おど). 
\\	闇	
\\	あん, おん	やみ	門, 音	
\\	(やみ)! 
\\	斜	
\\	しゃ	なな	余, 斗	
\\	(しゃ) 
\\	尽	
\\	じん	つ	尺, 冫	
\\	(じん) 
\\	霜	
\\	しも	雨, 木, 目	
\\	(しも). 
\\	穫	
\\	かく	
\\	禾, 艹, 隹, 又	
\\	(かく) 
\\	麻	
\\	ま	あさ	广, 木	
\\	(ま) 
\\	騎	
\\	き	
\\	馬, 奇	
\\	(き). 
\\	辱	
\\	じょく	はずかし	辰, 寸	
\\	(じょく). 
\\	灯	
\\	とう	あかり, とも	火, 丁	
\\	とうきょう. 
\\	とうきょう 
\\	とうきょう, 
\\	畜	
\\	ちく	
\\	玄, 田	
\\	(ちく). 
\\	溶	
\\	よう	と	氵, 容	
\\	(よう). 
\\	蚊	
\\	か	虫, 文	
\\	(か) 
\\	帳	
\\	ちょう	とばり	巾, 長	
\\	(ちょう).
\\	塗	
\\	と	ぬ	氵, 余, 土	
\\	(と) 
\\	貼	
\\	ちょう	は	貝, 占	
\\	(は)!
\\	輝	
\\	き	かがや	光, 冖, 車	
\\	(き) 
\\	憶	
\\	おく	
\\	忄, 意	
\\	(おく). 
\\	悔	
\\	かい	くや, く	忄, 毎	
\\	(かい). 
\\	耐	
\\	たい	た	而, 寸	
\\	(たい). 
\\	盾	
\\	じゅん	たて	厂		
\\	(じゅん) 
\\	蛇	
\\	じゃ	へび	虫, 宀, 匕	
\\	(へび).	
\\	班	
\\	はん	
\\	王, 刂	
\\	(はん) 
\\	餓	
\\	が	う	食, 我	
\\	(が). 
\\	飢	
\\	き	う	食, 几	
\\	(き) 
\\	迅	
\\	じん	
\\	厂, 十		
\\	(じん) 
\\	脅	
\\	きょう	おど, おびや	力, 月	
\\	きょうと. 
\\	きょうと 
\\	きょうと 
\\	概	
\\	がい	おおむ	木, 艮, 牙	
\\	(がい)! 
\\	拘	
\\	こう	かか	扌, 句	
\\	こういち. 
\\	こういち.	
\\	煮	
\\	しゃ	に	者, 灬	
\\	(に) 
\\	覆	
\\	ふく	おお, くつがえ	覀, 彳, 复	
\\	(ふく) 
\\	駒	
\\	く	こま	馬, 句	
\\	(こま). 
\\	悟	
\\	ご	さと	忄, 五, 口	
\\	(ご). 
\\	慌	
\\	こう	あわ	忄, 荒	
\\	こういち 
\\	こういち 
\\	こういち. 
\\	謀	
\\	ぼう	はか	言, 甘, 木	
\\	(ぼう) 
\\	鶴	
\\	かく	つる	宀, 隹, 鳥	
\\	(つる)!
\\	拓	
\\	たく	
\\	扌, 石	
\\	(たく) 
\\	衰	
\\	すい	おとろ	亠, 日		
\\	(すい) 
\\	奨	
\\	しょう	
\\	将, 大	
\\	(しょう) 
\\	淡	
\\	たん	あわ	氵, 火	
\\	(たん) 
\\	礎	
\\	そ	いしずえ	石, 木, 疋	
\\	(そ).	
\\	陛	
\\	へい	
\\	阝, 比, 土	
\\	(へい) 
\\	浸	
\\	しん	ひた	氵, ヨ, 冖, 又	
\\	(しん). 
\\	劣	
\\	れつ	おと	少, 力	
\\	(れつ) 
\\	勘	
\\	かん	
\\	甚, 力	
\\	(かん) 
\\	隔	
\\	かく	へだ	阝, 鬲	
\\	(かく) 
\\	桑	
\\	そう	くわ	又, 木	
\\	(くわ) 
\\	尼	
\\	に	あま	尸, 匕	
\\	(に) 
\\	珠	
\\	しゅ	たま	王, 丿, 未	
\\	(しゅ) 
\\	抽	
\\	ちゅう	
\\	扌, 由	
\\	(ちゅう). 
\\	壇	
\\	だん	
\\	土, 亠, 回, 旦	
\\	(だん), 
\\	陶	
\\	とう	
\\	阝, 勹, 缶	
\\	とうきょう. とうきょう 
\\	とうきょう 
\\	妃	
\\	ひ	
\\	女, 己	
\\	(ひ). 
\\	刈	
\\	か	メ, 刂	
\\	(か). 
\\	紫	
\\	し	むらさき	止, 匕, 糸	
\\	(し).	
\\	唯	
\\	ゆい	ただ	口, 隹	
\\	(ゆい). 
\\	剛	
\\	ごう	
\\	岡, 刂	
\\	ごういち.	
\\	ごういち 
\\	征	
\\	せい	
\\	彳, 正	
\\	(せい)! 
\\	誓	
\\	せい	ちか	扌, 斤, 言	
\\	(せい) 
\\	俗	
\\	ぞく	
\\	イ, 谷	
\\	(ぞく) 
\\	潤	
\\	じゅん	うるお, うる	氵, 門, 王	
\\	(じゅん), 
\\	偶	
\\	ぐう	たま	イ, 禺	
\\	(ぐう) 
\\	巧	
\\	こう	うま, たく	工		
\\	こういち.
\\	こういち. 
\\	工 
\\	こう 
\\	鰐	
\\	わに	魚, 口, 一		
\\	わに 
\\	把	
\\	は, わ	
\\	扌, 巴	
\\	(は)! 
\\	駄	
\\	だ, た	
\\	馬, 太	
\\	(だ).	
\\	洞	
\\	どう	ほら	氵, 同	
\\	(どう).	
\\	同 
\\	同 
\\	洞 
\\	伯	
\\	はく, お	
\\	イ, 白	
\\	白, 
\\	お, 
\\	御, 
\\	お, 
\\	唐	
\\	とう	
\\	广, 聿, 口	
\\	とうきょう. 
\\	とうきょう 
\\	とうきょう? 
\\	彰	
\\	しょう	
\\	章, 彡	
\\	(しょう) 
\\	諮	
\\	し	はか	言, 次, 口	
\\	(し).	
\\	廷	
\\	てい	
\\	廴, 王	
\\	(てい) 
\\	蟹	
\\	かに	角, 刀, 牛, 虫	
\\	かに 
\\	晶	
\\	しょう	
\\	日	
\\	(しょう).	
\\	堰	
\\	えん	せき	土, 匚, 日, 女	
\\	(せき) 
\\	漂	
\\	ひょう	ただよ	氵, 覀, 示	
\\	(ひょう).
\\	淀	
\\	よど	氵, 宀, 正	
\\	(よど) 
\\	堤	
\\	てい	つつみ	土, 日, 正	
\\	(てい) 
\\	后	
\\	こう, ご	きさき	厂, 一, 口	
\\	こういち.
\\	こういち.	
\\	疫	
\\	えき	
\\	疒, 殳	
\\	駅 (えき). 
\\	駅 
\\	駅 
\\	翻	
\\	ほん	ひるがえ	番, 羽	
\\	(ほん) 
\\	本 (ほん) 
\\	鬱	
\\	うつ	
\\	木, 缶		
\\	(うつ) 
\\	涯	
\\	がい	はて	氵, 厂, 土	
\\	(がい) 
\\	銘	
\\	めい	
\\	金, 名	
\\	名, 
\\	めい.
\\	名 
\\	仰	
\\	ぎょう, こう	あお	イ, 卬	
\\	(ぎょう).	
\\	漫	
\\	まん	
\\	氵, 日, 罒, 又	
\\	(まん)!	
\\	簿	
\\	ぼ	
\\	竹, 氵, 専	
\\	(ぼ), 
\\	亭	
\\	てい	
\\	亭	
\\	(てい) 
\\	訂	
\\	てい	
\\	言, 丁	
\\	(てい) 
\\	壮	
\\	そう	
\\	丬, 士	
\\	(そう) 
\\	軌	
\\	き	
\\	車, 九	
\\	(き) 
\\	奮	
\\	ふん	ふる	大, 隹, 田	
\\	(ふん).	
\\	峰	
\\	ほう	みね	山, 夆	
\\	(ほう). 
\\	墳	
\\	ふん	
\\	土, 十, 艹, 貝	
\\	(ふん). 
\\	搬	
\\	はん	
\\	扌, 舟, 殳	
\\	(はん) 
\\	邪	
\\	じゃ	よこし	牙, 阝	
\\	(じゃ) 
\\	肯	
\\	こう	がえんじ	止, 月	
\\	こういち.
\\	こういち 
\\	浦	
\\	ほ	うら	氵, 甫	
\\	(うら) 
\\	挟	
\\	きょう, しょう	はさ	扌, 夫		
\\	(はさ) 
\\	沸	
\\	ふつ	わ	氵, 弗	
\\	(ふつ)! 
\\	瓶	
\\	びん	かめ	开, 万, 丶, 一		
\\	(びん). 
\\	召	
\\	しょう	め	召	
\\	(しょう). 
\\	貞	
\\	てい	さだ	ト, 貝	
\\	(てい) 
\\	亮	
\\	りょう	あきらか	亠, 口, 冖, 儿	
\\	(りょう) 
\\	襟	
\\	きん	えり	ネ, 木, 示	
\\	(えり)!
\\	隅	
\\	ぐう	すみ	阝, 禺	
\\	(すみ)?	
\\	郡	
\\	ぐん	こおり	君, 阝	
\\	(ぐん). 
\\	燥	
\\	そう	はしゃ	火, 喿	
\\	(そう). 
\\	釈	
\\	しゃく, せき	す, とく, ゆる	釆, 尺	
\\	(しゃく). 
\\	脂	
\\	し	あぶら	月, 匕, 日	
\\	(し)!	
\\	偉	
\\	い	えら	イ, 韋	
\\	(い) 
\\	軒	
\\	けん	のき	車, 干	
\\	(けん) 
\\	蓮	
\\	れん	はす, はちす	艹, 車		
\\	(れん).	
\\	慈	
\\	じ	いつく	一, 幺, 心		
\\	(じ) 
\\	塚	
\\	ちょう	つか	土, 冖, 豕	
\\	(つか)! 
\\	玄	
\\	げん	くろ	玄	
\\	(げん). 
\\	肪	
\\	ぼう	
\\	月, 方	
\\	(ぼう) 
\\	耕	
\\	こう	たがや	耒, 井	
\\	こういち
\\	こういち. 
\\	こういち 
\\	こういち
\\	媛	
\\	えん	ひめ	女, 爰	
\\	姫? 
\\	ひめ 
\\	姫, 
\\	邸	
\\	てい	やしき	氏, 一, 阝	
\\	(てい) 
\\	喚	
\\	かん	わめ	口, 勹, 儿, 大	
\\	(かん) 
\\	苗	
\\	みょう, びょう	なえ, なわ	艹, 田	
\\	(みょう), 
\\	隻	
\\	せき	
\\	隹, 又	
\\	(せき)! 
\\	膚	
\\	ふ	はだ	虍, 胃	
\\	(ふ). 
\\	軟	
\\	なん	やわ	車, 欠	
\\	(なん). 
\\	郊	
\\	こう	
\\	交, 阝	
\\	こういち.
\\	こういち 
\\	頂	
\\	ちょう	いただき, いただ	丁, 頁	
\\	(ちょう), 
\\	濯	
\\	たく	すす, ゆす, そそぐ	氵, ヨ, 隹	
\\	(たく). 
\\	渦	
\\	か	うず	氵, 冋	
\\	(か)? 
\\	聡	
\\	そう	さと, みみざと	耳, 公, 心	
\\	(そう), 
\\	枯	
\\	こ	か	木, 古	
\\	子 (こ). 
\\	子 
\\	祥	
\\	しょう	きざ, さいわ, つまび, よ	ネ, 羊	
\\	(しょう) 
\\	呂	
\\	ろ, りょ	せぼね	呂	
\\	(ろ). 
\\	偏	
\\	へん	かたよ	イ, 扁	
\\	(へん) 
\\	茨	
\\	し, じ	いばら	艹, 次	
\\	(いばら)! 
\\	陥	
\\	かん	おちい, おとしい	阝, 勹, 旧	
\\	(かん) 
\\	鎖	
\\	さ	くさり, とざ	金, 貝		
\\	(さ) 
\\	賠	
\\	ばい	
\\	貝, 咅	
\\	(ばい)
\\	恒	
\\	こう	つね, つねに	忄, 一, 旦	
\\	こういち, 
\\	こういち 
\\	綾	
\\	りん	あや	糸, 夌	
\\	(りん) 
\\	没	
\\	ぼつ, もつ	おぼ, しず, ない	氵, 殳	
\\	(ぼつ). 
\\	擁	
\\	よう	
\\	扌, 亠, 幺, 隹	
\\	(よう) 
\\	遭	
\\	そう	あ	一, 曲, 日		
\\	(そう) 
\\	噴	
\\	ふん	ふ	口, 十, 艹, 貝	
\\	(ふん).	
\\	殊	
\\	しゅ	こと	歹, 丿, 未	
\\	(しゅ). 
\\	倫	
\\	りん	
\\	イ, 一, 冊		
\\	(りん) 
\\	陳	
\\	ちん	ひ	阝, 東	
\\	(ちん) 
\\	隼	
\\	しゅん, じゅん	はやぶさ	隹, 十	
\\	(はやぶさ) 
\\	乃	
\\	ない	すなわ, なんじ, の	乃	
\\	(ない) 
\\	輔	
\\	ふ, ほ	たす	車, 甫	
\\	(すけ) 
\\	猟	
\\	りょう, れふ	かり, か	犭, 用		
\\	(りょう) 
\\	唆	
\\	さ	そそのか, そそ	口, 夋	
\\	(さ) 
\\	惰	
\\	だ	
\\	忄, ナ, 工, 月	
\\	(だ). 
\\	怠	
\\	たい	おこた, なま	台, 心	
\\	(たい) 
\\	覇	
\\	は, はく	はたがしら	覀, 革, 月	
\\	(は) 
\\	須	
\\	す, しゅ	すべから, すべし, ひげ, まつ, もち, もと	彡, 頁	
\\	(す) 
\\	牲	
\\	せい	
\\	牛, 生	
\\	(せい). 
\\	秩	
\\	ちつ	
\\	禾, 失	
\\	(ちつ) 
\\	孤	
\\	こ	
\\	瓜, 子	
\\	子 (こ).	
\\	芳	
\\	ほう	かんば	艹, 方	
\\	(ほう) 
\\	貫	
\\	かん	つらぬ, ぬき, ぬ	毋, 貝	
\\	(かん) 
\\	糧	
\\	りょう, ろう	かて	米, 旦, 里	
\\	(りょう) 
\\	(ろう) 
\\	颯	
\\	さつ, そう	さっ	立, 風	
\\	(さつ) 
\\	慢	
\\	まん	
\\	忄, 日, 罒, 又	
\\	(まん). 
\\	膨	
\\	ぼう	ふく	月, 壴, 彡	
\\	(ぼう) 
\\	遇	
\\	ぐう	あ	禺		
\\	(ぐう).	
\\	ぐう. 
\\	諭	
\\	ゆ	さと	言		
\\	(ゆ). 
\\	随	
\\	ずい	したが, まにま	阝, 有		
\\	(ずい).	
\\	胡	
\\	こ, う, ご	なんぞ	古, 月	
\\	子 (こ) 
\\	搭	
\\	とう	
\\	扌, 艹, 合	
\\	とうきょう, 
\\	とうきょう.	
\\	錦	
\\	きん	にしき	金, 白, 巾	
\\	(にしき). 
\\	鯉	
\\	り	こい	魚, 里	
\\	(こい) 
\\	胞	
\\	ほう	
\\	月, 包	
\\	(ほう). 
\\	浄	
\\	じょう, せい	きよ	氵, 争	
\\	(じょう). 
\\	帥	
\\	すい	
\\	丶, 巾		
\\	(すい).	
\\	諒	
\\	りょう	あきら, まことに	言, 京	
\\	(りょう) 
\\	りょう 
\\	蒙	
\\	もう, ぼう	おお, くら, こうむ	艹, 冖, 一, 豕	
\\	(もう) 
\\	曙	
\\	しょ	あけぼの	日, 罒, 者	
\\	(あけぼの). 
\\	惨	
\\	さん, ざん	みじ, いた, むご	忄, ム, 大, 彡	
\\	(さん) 
\\	(ざん) 
\\	稿	
\\	こう	したがき, わら	禾, 高	
\\	こういち, 
\\	こういち 
\\	こういち
\\	啓	
\\	けい	さと, ひら	戸, 夂, 口	
\\	(けい) 
\\	披	
\\	ひ	
\\	扌, 皮	
\\	(ひ) 
\\	繊	
\\	せん	
\\	糸, 十, 戈		
\\	(せん). 
\\	徐	
\\	じょ	おもむ	彳, 余	
\\	(じょ). 
\\	葵	
\\	き	あおい	艹, 癶, 天	
\\	あおい 
\\	あおい 
\\	あおい 
\\	騰	
\\	とう	あが, のぼ	月, 馬		
\\	とうきょう, 
\\	とうきょう 
\\	据	
\\	きょ	す	扌, 尸, 古	
\\	(す) 
\\	莉	
\\	り, れい, らい	
\\	艹, 禾, 刂	
\\	(り)!	
\\	緯	
\\	い	ぬき, よこいと	糸, 韋	
\\	(い) 
\\	瓜	
\\	か, け	うり	瓜	
\\	(か)! 
\\	虐	
\\	ぎゃく	しいた	虍, ヨ	
\\	(ぎゃく). 
\\	戴	
\\	たい	いただ	十, 戈, 田, 共	
\\	(たい) 
\\	艇	
\\	てい	
\\	舟, 廴, 王	
\\	(てい). 
\\	丹	
\\	たん	に	舟	
\\	(たん). 
\\	緋	
\\	ひ	あか, あけ	糸, 非	
\\	(ひ). 
\\	准	
\\	じゅん	
\\	冫, 隹	
\\	(じゅん)! 
\\	舗	
\\	ほ	
\\	舎, 甫	
\\	(ほ) 
\\	壌	
\\	じょう	つち	土, 㐮	
\\	(じょう) 
\\	駿	
\\	しゅん, すん	すぐ	馬, 夋	
\\	(しゅん) 
\\	剰	
\\	じょう	あまつさえ, あま	禾, 口, 刂	
\\	(じょう) 
\\	寛	
\\	かん	くつろ	宀, 艹, 見	
\\	(かん) 
\\	庶	
\\	しょ	
\\	灬		
\\	(しょ) 
\\	且	
\\	しょ, しょう, そ	か	且	
\\	(か) 
\\	顕	
\\	けん	あきらか, あらわ	日, 頁		
\\	(けん)
\\	杏	
\\	あん, きょう, こう	あんず	木, 口	
\\	(あん) 
\\	(ず) 
\\	栞	
\\	かん	しおり	干, 木	
\\	(しおり).
\\	欄	
\\	らん	てすり	木, 門, 東	
\\	(らん) 
\\	冠	
\\	かん	かんむり	冖, 元, 寸	
\\	(かん), 
\\	酷	
\\	こく	ひど	酉, 告	
\\	(こく)! 
\\	叙	
\\	じょ	つい, ついで	余, 又	
\\	(じょ).	
\\	じょ 
\\	じょう). 
\\	逸	
\\	いつ	そ, はぐ	免		
\\	(いつ) 
\\	紋	
\\	もん	
\\	糸, 文	
\\	阿	
\\	あ, お	おもね, くま	阝, 可	
\\	さん,
\\	(あ)!
\\	愚	
\\	ぐ	おろ	禺, 心	
\\	(ぐ). 
\\	尚	
\\	しょう	なお	冋		
\\	(しょう) 
\\	拐	
\\	かい	
\\	扌, 口, 刀	
\\	(かい). 
\\	悠	
\\	ゆう	
\\	イ, 夂, 心		
\\	(ゆう) 
\\	勲	
\\	くん	いさお	重, 力, 灬	
\\	(くん) 
\\	疎	
\\	そ, しょ	うと, まば	疋, 束	
\\	(そ). 
\\	謡	
\\	よう	うた	言		
\\	(よう). 
\\	哺	
\\	ほ	ほぐく, ふく	口, 甫	
\\	(ほ) 
\\	栽	
\\	さい	
\\	耒, 戈	
\\	(さい)!	
\\	践	
\\	せん	ふ	足		
\\	(せん). 
\\	呈	
\\	てい	
\\	口, 王	
\\	(てい) 
\\	傲	
\\	ごう	あなど, おご	イ, 土, 方, 夂	
\\	ごういち. 
\\	ごういち 
\\	疾	
\\	しつ	はや	疒, 矢	
\\	(しつ) 
\\	茜	
\\	せん	あかね	艹, 西	
\\	""赤ね (あかね)?
\\	あかね 
\\	酬	
\\	しゅう, しゅ, とう	むく	酉, 丶, 川	
\\	(しゅう). 
\\	呆	
\\	ほう	あき, おろか, ほけ, ぼ	口, 木	
\\	(ほう) 
\\	鎌	
\\	けん, れん	かま	金, 兼	
\\	(かま) 
\\	粛	
\\	しゅく, すく	つつし	ヨ, 儿, 米		
\\	(しゅく) 
\\	茎	
\\	きょう, けい	くき	艹, 圣	
\\	(くき)! 
\\	痴	
\\	ち	おろか, し	疒, 矢, 口	
\\	(ち). 
\\	荘	
\\	そう, しょう, ちゃん	あごそ, ほうき	艹, 丬, 士	
\\	(そう) 
\\	鯨	
\\	げい	くじら	魚, 京	
\\	(げい). 
\\	卸	
\\	しゃ	おろし, おろ	午, 止, 卩	
\\	(おろし). 
\\	累	
\\	るい	
\\	田, 糸	
\\	(るい) 
\\	伏	
\\	ふく	ふ	イ, 犬	
\\	(ふく)!
\\	虜	
\\	りょ, ろ	とりく, とりこ	虍, 男	
\\	(りょ) 
\\	循	
\\	じゅん	
\\	彳, 厂		
\\	(じゅん). 
\\	粗	
\\	そ	あら	米, 且	
\\	(そ).	
\\	凝	
\\	ぎょう	こ, こご	冫, 疑	
\\	(ぎょう) 
\\	栓	
\\	せん	
\\	木, 王		
\\	(せん). 
\\	瑛	
\\	えい	
\\	王, 艹, 央	
\\	(えい) 
\\	旦	
\\	たん, だん	あきら, あき, あさ, あした, ただし	旦	
\\	(だん)! 
\\	(たん). 
\\	奉	
\\	ほう, ぶ	たてまつ, まつ, ほう	干		
\\	(ほう). 
\\	遼	
\\	りょう	
\\	尞		
\\	(りょう) 
\\	郭	
\\	かく	くるわ	享, 阝	
\\	(かく) 
\\	抹	
\\	まつ	
\\	扌, 未	
\\	(まつ) 
\\	佳	
\\	か	
\\	イ, 土	
\\	(か)! 
\\	惜	
\\	せき	お	忄, 昔	
\\	(せき). 
\\	憂	
\\	ゆう	う, うれ	百, 冖, 心, 夂	
\\	(ゆう). 
\\	悼	
\\	とう	いた	忄, ト, 早	
\\	とうきょう. 
\\	とうきょう, 
\\	とうきょう 
\\	癒	
\\	ゆ	い, いや	疒, 心		
\\	(ゆ). 
\\	栃	
\\	とち	木, 厂, 万	
\\	(とち)! 
\\	龍	
\\	りゅう, りょう, ろう	たつ	龍	
\\	(りゅう) 
\\	弥	
\\	び, み	や	弓, 小		
\\	(や) 
\\	髄	
\\	ずい	
\\	骨, 有		
\\	(ずい) 
\\	傍	
\\	ぼう	かたわ, わき, おか, はた, そば	イ, 立, 方	
\\	(ぼう) 
\\	愉	
\\	ゆ	たの	忄		
\\	(ゆ), 
\\	赴	
\\	ふ	おもむ	走, ト	
\\	(ふ). 
\\	昌	
\\	しょう	さかん	日	
\\	(しょう). 
\\	憾	
\\	かん	うら	忄, 感	
\\	(かん). 
\\	朴	
\\	ぼく	えのき, ほう, ほお	木, ト	
\\	ぼく, 
\\	ぼく ぼく ぼく!
\\	ぼく!	
\\	脊	
\\	せき	せ, せい	二, 人, 月	
\\	(せき) 
\\	該	
\\	がい	
\\	言, 亥	
\\	(がい) 
\\	之	
\\	し	これ, の	丶		
\\	(これ) 
\\	鎮	
\\	ちん	おさえ, しず	金, 真	
\\	(ちん). 
\\	尿	
\\	にょう	
\\	尸, 水	
\\	(にょう)
\\	賓	
\\	ひん	
\\	宀, 一, 少, 貝	
\\	(ひん) 
\\	那	
\\	な, だ	いかん, なに, なんぞ	刀, 二, 阝	
\\	(な)! 
\\	(な)!	
\\	匠	
\\	しょう	たくみ	匚, 斤	
\\	(しょう) 
\\	拍	
\\	はく, ひょう	
\\	扌, 白	
\\	(はく) 
\\	縛	
\\	ばく	しば	糸, 専, 丶	
\\	(ばく) 
\\	飽	
\\	ほう	あ	食, 包	
\\	(ほう). 
\\	柴	
\\	さい, し	しば	止, 匕, 木	
\\	(しば) 
\\	蝶	
\\	ちょう	
\\	虫, 世, 木	
\\	(ちょう).	
\\	弦	
\\	げん	つる	弓, 玄	
\\	(げん). 
\\	凛	
\\	りん	きびし	冫, 亠, 回, 示	
\\	(りん) 
\\	庸	
\\	よう	
\\	广, 聿, 用	
\\	(よう) 
\\	錯	
\\	さく, しゃく	
\\	金, 昔	
\\	(さく) 
\\	轄	
\\	かつ	くさび	車, 宀, 生, 口	
\\	(かつ). 
\\	悦	
\\	えつ	よろこ	忄, 兑	
\\	(えつ) 
\\	窮	
\\	きゅう, きょう	きわ	穴, 身, 弓	
\\	(きゅう). 
\\	嘉	
\\	か	よい, よみ	壴, 力, 口	
\\	(か). 
\\	弊	
\\	へい	
\\	敝, 廾	
\\	(へい). 
\\	遥	
\\	よう	はる		
\\	(はる) 
\\	洪	
\\	こう	
\\	氵, 共	
\\	こういち 
\\	こういち 
\\	紳	
\\	しん	
\\	糸, 申	
\\	(しん) 
\\	呉	
\\	ご	くれ, く	呉	
\\	(ご) 
\\	穀	
\\	こく	
\\	士, 冖, 禾, 殳	
\\	(こく).	
\\	摂	
\\	せつ, しょう	おさ, かね, と	扌, 耳		
\\	(せつ). 
\\	寂	
\\	じゃく, せき	さび, さみ	宀, 上, 小, 又	
\\	(じゃく). 
\\	宰	
\\	さい	
\\	宀, 辛	
\\	(さい). 
\\	陵	
\\	りょう	みささぎ	阝, 夌	
\\	(りょう) 
\\	凡	
\\	ぼん, はん	おうよ, およ, すべ	几, 丶	
\\	(ぼん) 
\\	尉	
\\	い, じょう	
\\	尸, 示, 寸	
\\	(い)! 
\\	靖	
\\	じょう, せい	やす	立, 青	
\\	(やす)
\\	恭	
\\	きょう	うやうや	共, 小, 丶	
\\	うやうや 
\\	縫	
\\	ほう	ぬ	糸, 夆		
\\	(ほう), 
\\	舶	
\\	はく	
\\	舟, 白	
\\	(はく). 
\\	搾	
\\	さく	しぼ	扌, 穴, 乍	
\\	(さく). 
\\	猶	
\\	ゆう, ゆ	なお	犭, 酉		
\\	(ゆう) 
\\	窒	
\\	ちつ	
\\	穴, 至	
\\	(ちつ) 
\\	碑	
\\	ひ	いしぶみ	石, 丶, 田, 十	
\\	(ひ)! 
\\	智	
\\	ち	
\\	矢, 口, 日	
\\	(ち)! 
\\	款	
\\	かん	
\\	士, 示, 欠	
\\	(かん).	
\\	盲	
\\	もう	めくら	亡, 目	
\\	(もう) 
\\	醸	
\\	じょう	かも	酉, 㐮	
\\	(じょう) 
\\	凹	
\\	おう	くぼ, へこ, ぼこ	凹	
\\	王 (おう) 
\\	王. 
\\	王
\\	弔	
\\	ちょう	とぶら, とむら	弓		
\\	(ちょう), 
\\	凸	
\\	とつ	でこ	凸	
\\	(とつ) 
\\	烏	
\\	うお	からす	鳥	
\\	(からす) 
\\	敢	
\\	かん	あ	夂		
\\	(かん) 
\\	堕	
\\	だ	お, くず	阝, 有, 土	
\\	(だ). 
\\	鼓	
\\	こ	つづみ	壴, 支	
\\	子
\\	(こ), 
\\	衡	
\\	こう	
\\	行, 勹, 田, 大	
\\	耕
\\	こういち 
\\	こういち
\\	こういち 
\\	伐	
\\	ばつ	う, き, そむ	イ, 戈	
\\	(ばつ) 
\\	酵	
\\	こう	
\\	酉, 孝	
\\	こういち.	
\\	こういち
\\	閲	
\\	えつ	けみ	門, 兑	
\\	(えつ) 
\\	遮	
\\	しゃ	さえぎ	灬	
\\	(しゃ) 
\\	腸	
\\	ちょう	はらわた	月, 易	
\\	(ちょう).	
\\	瑠	
\\	る, りゅう	
\\	王, ム, 刀, 田	
\\	(る)!
\\	乙	
\\	おつ, いつ	おと, きのと	乙	
\\	(おつ). 
\\	楓	
\\	ふう	かえで	木, 風	
\\	(かえで)!	
\\	膜	
\\	まく	
\\	月, 莫	
\\	(まく) 
\\	紺	
\\	こん	
\\	糸, 甘	
\\	(こん) 
\\	蒼	
\\	そう	あお	艹, 倉	
\\	(そう). 
\\	漬	
\\	し	つ	氵, 生, 貝	
\\	(つ) 
\\	哉	
\\	さい	や, かな	土, 口, 戈	
\\	(や) 
\\	峡	
\\	きょう, こう	はざま	山, 夫		
\\	きょうと 
\\	きょうと, 
\\	賊	
\\	ぞく	
\\	貝, 戈, 十	
\\	(ぞく) 
\\	旋	
\\	せん	
\\	方, 疋		
\\	(せん) 
\\	俸	
\\	ほう	
\\	イ, 干		
\\	(ほう) 
\\	喝	
\\	かつ	
\\	口, 日, 勹, 匕	
\\	(かつ).	
\\	羅	
\\	ら	うすもの	罒, 糸, 隹	
\\	(ら).	
\\	萌	
\\	ほう	きざ, めばえ, も	艹, 明	
\\	(ほう), 
\\	槽	
\\	そう	ふね	木, 一, 曲, 日	
\\	(そう) 
\\	坪	
\\	へい	つぼ	土, 平	
\\	(つぼ). 
\\	遍	
\\	へん	あまね	扁		
\\	(へん) 
\\	胎	
\\	たい	
\\	月, 台	
\\	(たい). 
\\	陪	
\\	ばい	
\\	阝, 咅	
\\	部, 
\\	(ばい) 
\\	扶	
\\	ふ	たす	扌, 夫	
\\	(ふ). 
\\	迭	
\\	てつ	
\\	失		
\\	鉄 (てつ). 
\\	鉄 
\\	鶏	
\\	けい	とり, にわとり	夫, 鳥		
\\	(けい) 
\\	瑞	
\\	すい, ずい	みず, しるし	王, 山, 而	
\\	(みず) 
\\	暁	
\\	きょう, ぎょう	あかつき, さと	日, 尭	
\\	あか 
\\	つき 
\\	あかつき 
\\	あかつき 
\\	剖	
\\	ぼう	
\\	咅, 刂	
\\	(ぼう) 
\\	凌	
\\	りょう	しの	冫, 夌	
\\	(しの) 
\\	藩	
\\	はん	
\\	艹, 氵, 番	
\\	(はん) 
\\	譜	
\\	ふ	
\\	言, 並, 日	
\\	(ふ) 
\\	璃	
\\	り	
\\	王, 离	
\\	(り) 
\\	淑	
\\	しゅく	しと	氵, 上, 小, 又	
\\	(しゅく) 
\\	傑	
\\	けつ	すぐ	イ, 舛, 木	
\\	(けつ) 
\\	殻	
\\	かく, こく, ばい	から, がら	士, 冖, 几, 殳	
\\	(かく) 
\\	錠	
\\	じょう	
\\	金, 宀, 正	
\\	(じょう).	
\\	媒	
\\	ばい	なこうど	女, 甘, 木	
\\	ばい).
\\	濁	
\\	だく, じょく	にご	氵, 罒, 勹, 虫	
\\	(だく). 
\\	椎	
\\	つい, すい	う, つち	木, 隹	
\\	(つい)!
\\	赦	
\\	しゃ	
\\	赤, 夂	
\\	(しゃ) 
\\	戯	
\\	ぎ, げ	ざ, じゃ, たわむ	虍, 戈		
\\	(ぎ). 
\\	享	
\\	きょう, こう	う	享	
\\	きょうと. 
\\	きょうと, 
\\	きょうと 
\\	嘱	
\\	しょく	しょく, たの	口, 尸, 禹	
\\	(しょく) 
\\	肖	
\\	しょう	あやか	月		
\\	(しょう). 
\\	憤	
\\	ふん	いきどお	忄, 十, 艹, 貝	
\\	(ふん). 
\\	漣	
\\	れん, らん	さざなみ	氵, 車		
\\	(れん), 
\\	朽	
\\	きゅう	く	木, 一, 勹	
\\	(きゅう) 
\\	奔	
\\	ほん	はし	大, 十, 廾	
\\	(ほん) 
\\	帆	
\\	はん	ほ	巾, 几, 丶	
\\	(はん) 
\\	菅	
\\	かん, けん	すげ	艹, 宀		
\\	(すが)! 
\\	酌	
\\	しゃく	く	酉, 勺	
\\	(しゃく) 
\\	慨	
\\	がい	
\\	忄, 艮, 牙	
\\	(がい).
\\	絹	
\\	けん	きぬ	糸, 口, 月	
\\	(けん) 
\\	窃	
\\	せつ	ぬす, ひそ	穴, 七, 刀	
\\	(せつ) 
\\	硫	
\\	りゅう	
\\	石, 川		
\\	(りゅう) 
\\	亜	
\\	あ	つ	亜	
\\	(あ). 
\\	屯	
\\	とん	
\\	屯	
\\	(とん). 
\\	岬	
\\	こう	みさき	山, 甲	
\\	(みさき). 
\\	鋳	
\\	ちゅう	い	金, 三, 丿, 寸	
\\	(ちゅう)! 
\\	拙	
\\	せつ	つたな	扌, 出	
\\	(せつ). 
\\	詠	
\\	えい	よ, うた	言, 永	
\\	(えい). 
\\	慶	
\\	けい	よろこ	广, 覀, 亅, 心, 夂	
\\	(けい)!	
\\	酪	
\\	らく	
\\	酉, 各	
\\	(らく). 
\\	篤	
\\	とく	あつ	竹, 馬	
\\	(とく) 
\\	侮	
\\	ぶ	あなず, あなど	イ, 毎	
\\	(ぶ) 
\\	堪	
\\	かん, たん	た, こた, こ	土, 甚	
\\	(た). 
\\	禍	
\\	か	わざわい	ネ, 冋	
\\	(か). 
\\	雌	
\\	し	めす, め, めん	止, 匕, 隹	
\\	(めす) 
\\	睦	
\\	ぼく, もく	むつ	目, 坴	
\\	""ぼく-
\\	ぼく 
\\	ぼく 
\\	胆	
\\	たん	きも	月, 旦	
\\	(たん) 
\\	擬	
\\	ぎ	まが, もど	扌, 疑	
\\	(ぎ). 
\\	漆	
\\	しつ	うるし	氵, 木, 水		
\\	(しつ). 
\\	閑	
\\	かん	
\\	門, 木	
\\	(かん)! 
\\	憧	
\\	しょう, とう, どう	あこが	忄, 立, 里	
\\	(あこが) 
\\	卑	
\\	ひ	いや	丶, 田, 十	
\\	(ひ). 
\\	姻	
\\	いん	
\\	女, 口, 大	
\\	(いん) 
\\	忌	
\\	き	い	己, 心	
\\	(き) 
\\	曹	
\\	そう, ぞう	つかさ, ともがら, へや	一, 曲, 日	
\\	(そう). 
\\	吟	
\\	ぎん	
\\	口, 今	
\\	(ぎん) 
\\	礁	
\\	しょう	
\\	石, 隹, 灬	
\\	(しょう) 
\\	峠	
\\	とうげ	山, 上, 下	
\\	(とうげ). 
\\	沙	
\\	さ, しゃ	すな, よなげる	氵, 少	
\\	(さ) 
\\	蔑	
\\	べつ	さげす	艹, 罒, 戈, 丿, 丶	
\\	(べつ) 
\\	汰	
\\	た, たい	おご, にご, よな	氵, 太	
\\	(た). 
\\	紡	
\\	ぼう	つむ	糸, 方	
\\	(ぼう) 
\\	遷	
\\	せん	うつ, みやこがえ	覀, 大, 己		
\\	(せん) 
\\	叔	
\\	しゅく	
\\	上, 小, 又	
\\	(お) 
\\	甚	
\\	じん	はなは	甚	
\\	(じん). 
\\	浪	
\\	ろう	
\\	氵, 良	
\\	(ろう).	
\\	梓	
\\	し	あずさ	木, 辛	
\\	(あずさ) 
\\	崇	
\\	すう	あが	山, 宀, 示	
\\	(すう) 
\\	煩	
\\	はん, ぼん	うるさ, わずら	火, 頁	
\\	(はん) 
\\	蛮	
\\	ばん	えびす	赤, 虫	
\\	(ばん) 
\\	廉	
\\	れん	
\\	广, 兼	
\\	(れん). 
\\	劾	
\\	がい	
\\	亥, 力	
\\	(がい). 
\\	某	
\\	ぼう	それがし, なにがし	甘, 木	
\\	(ぼう) 
\\	矯	
\\	きょう	た	矢, 天, 口, 冋	
\\	きょうと. 
\\	きょうと, 
\\	囚	
\\	しゅう	とら	口, 人	
\\	(しゅう). 
\\	痢	
\\	り	
\\	疒, 禾, 刂	
\\	(り) 
\\	逝	
\\	せい	い, ゆ	扌, 斤		
\\	(せい) 
\\	狐	
\\	こ	きつね	犭, 瓜	
\\	(きつね) 
\\	漸	
\\	ぜん	ようや, やや	氵, 車, 斤	
\\	(ようや) 
\\	升	
\\	しょう	ます	丿, 廾	
\\	(ます) 
\\	婿	
\\	せい	むこ	女, 疋, 月	
\\	(むこ). 
\\	匿	
\\	とく	かくま	匚, 艹, 右	
\\	(とく) 
\\	謹	
\\	きん	つつし	言, 堇	
\\	(きん) 
\\	藍	
\\	らん	あい	艹, 監	
\\	(あい)! 
\\	桟	
\\	さん, せん	かけはし	木		
\\	(さん) 
\\	殉	
\\	じゅん	
\\	歹, 勹, 日	
\\	(じゅん). 
\\	坑	
\\	こう	
\\	土, 亠, 几	
\\	こういち, 
\\	こういち 
\\	こういち
\\	こういち 
\\	罷	
\\	ひ	や	罒, 能	
\\	(ひ). 
\\	妄	
\\	もう, ぼう	みだ	亡, 女	
\\	(もう) 
\\	藻	
\\	そう	も	艹, 氵, 喿	
\\	(そう) 
\\	泌	
\\	ひ, ひつ	
\\	氵, 必	
\\	(ひ). 
\\	唄	
\\	ばい	うた	口, 貝	
\\	歌 (うた) 
\\	畔	
\\	はん	あぜ, くろ, ほとり	田, 半	
\\	(はん) 
\\	倹	
\\	けん	つづまやか, つま	イ		
\\	(けん) 
\\	(けん) 
\\	拷	
\\	ごう	
\\	扌, 耂		
\\	ごういち, 
\\	醜	
\\	しゅう	しこ, みにく	酉, 鬼	
\\	(しゅう) 
\\	渓	
\\	けい	たに, たにがわ	氵, 夫		
\\	(けい)! 
\\	湧	
\\	ゆう, ゆ, よう	わ	氵, 勇	
\\	(ゆう)! 
\\	寡	
\\	か	
\\	宀, 頁, 一, 刀	
\\	(か), 
\\	慕	
\\	ぼ	した	莫, 小, 丶	
\\	(ぼ) 
\end{CJK}
\end{document}