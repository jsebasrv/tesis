\documentclass[8pt]{extreport} 
\usepackage{hyperref}
\usepackage{CJKutf8}
\begin{document}
\begin{CJK}{UTF8}{min}
\\	先生はジョンを叱った。	
\\	先生[せんせい]はジョンを 叱[しか]った。
\\	ジョンは先生に叱られた。	
\\	ジョンは 先生[せんせい]に 叱[しか]られた。
\\	ビルはマーさをぶった。	
\\	ビルはマーさをぶった。
\\	マーサはビルにぶたれた。	
\\	マーサはビルにぶたれた。
\\	私はメアリーがトムに会った事実を知っている。	
\\	私[わたし]はメアリーがトムに 会[あ]った 事実[じじつ]を 知[し]っている。
\\	私はジョンに行って欲しい。	
\\	私[わたし]はジョンに 行[い]って 欲[ほ]しい。
\\	この辞書は使いやすい。	
\\	この 辞書[じしょ]は 使[つか]いやすい。
\\	私はすしが食べたい。	
\\	私[わたし]はすしが 食[た]べたい。
\\	ベスは大学を出たらしい。	
\\	ベスは 大学[だいがく]を 出[で]たらしい。
\\	花子は淋しいようだ。	
\\	花子[はなこ]は 淋[さび]しいようだ。
\\	このお菓子はおいしそうだ。	
\\	このお 菓子[かし]はおいしそうだ。
\\	ビルは今手紙を書いている。	
\\	ビルは 今[いま] 手紙[てがみ]を 書[か]いている。
\\	窓が開けてある。	
\\	窓[まど]が 開[あ]けてある。
\\	僕は宿題をしてしまった。	
\\	僕[ぼく]は 宿題[しゅくだい]をしてしまった。
\\	私は友達にお金を貸してあげた。	
\\	私[わたし]は 友達[ともだち]にお 金[かね]を 貸[か]してあげた。
\\	このコンピューターは高すぎる。	
\\	このコンピューターは 高[たか]すぎる。
\\	ジョージはスポーツカーを欲しがっている。	
\\	ジョージはスポーツカーを 欲[ほ]しがっている。
\\	あっ!雨が降ってきた!	
\\	あっ! 雨[あめ]が 降[ふ]ってきた!
\\	僕は泳いだがスミスさんは泳がなかった。	
\\	僕[ぼく]は 泳[およ]いだがスミスさんは 泳[およ]がなかった。
\\	吉田さんは東京に行って鈴木さんに会った。	
\\	吉田[よしだ]さんは 東京[とうきょう]に 行[い]って 鈴木[すずき]さんに 会[あ]った。
\\	ジョンは来たが、ビルは来なかった。	
\\	ジョンは 来[き]たが、ビルは 来[こ]なかった。
\\	走っていますが、ちっともやせません。	
\\	走[はし]っていますが、ちっともやせません。
\\	山口先生は学生をよくほめる。	
\\	山口[やまぐち] 先生[せんせい]は 学生[がくせい]をよくほめる。
\\	かおりは一郎をだました。	
\\	かおりは 一郎[いちろう]をだました。
\\	僕は本を書いた。	
\\	僕[ぼく]は 本[ほん]を 書[か]いた。
\\	一郎はみどりにスカーフをやった。	
\\	一郎[いちろう]はみどりにスカーフをやった。
\\	私はドアを開けておいた。	
\\	私[わたし]はドアを 開[あ]けておいた。
\\	花子は一郎の大学入学を喜んだ。	
\\	花子[はなこ]は 一郎[いちろう]の 大学[だいがく] 入学[にゅうがく]を 喜[よろこ]んだ。
\\	トムはその時公園を歩いていた。	
\\	トムはその 時[とき] 公園[こうえん]を 歩[ある]いていた。
\\	東京からは田中さんが来た。	
\\	東京[とうきょう]からは 田中[たなか]さんが 来[き]た。
\\	私はミラーさんとも話した。	
\\	私[わたし]はミラーさんとも 話[はな]した。
\\	山田は頭が痛いと言った。	
\\	山田[やまだ]は 頭[あたま]が 痛[いた]いと 言[い]った。
\\	山田は頭が痛いので学校を休んだ。	
\\	山田[やまだ]は 頭[あたま]が 痛[いた]いので 学校[がっこう]を 休[やす]んだ。
\\	山田は外国に行くのが大好きだ。	
\\	山田[やまだ]は 外国[がいこく]に 行[い]くのが 大好[だいす]きだ。
\\	山田は空手を習ったことがある。	
\\	山田[やまだ]は 空手[からて]を 習[なら]ったことがある。
\\	ーどんな辞書が欲しいんですか。 ー小さいのが欲しいんです。	ー
\\	ーどんな 辞書[じしょ]が 欲[ほ]しいんですか。 ー 小[ちい]さいのが 欲[ほ]しいんです。
\\	鈴木さんは走った。	
\\	鈴木[すずき]さんは 走[はし]った。
\\	日本語を読むのは難しい。	
\\	日本語[にほんご]を 読[よ]むのは 難[むずか]しい。
\\	日本語を読むことは難しい。	
\\	日本語[にほんご]を 読[よ]むことは 難[むずか]しい。
\\	私はビルにぶたれた。	
\\	私[わたし]はビルにぶたれた。
\\	太郎は秋子に泣かれた。	
\\	太郎[たろう]は 秋子[あきこ]に 泣[な]かれた。
\\	松本さんはよく映画を見る。	
\\	松本[まつもと]さんはよく 映画[えいが]を 見[み]る。
\\	私の家はスミスさんのより新しい。	
\\	私[わたし]の 家[いえ]はスミスさんのより 新[あたら]しい。
\\	ジョンは日本語の学生です。	
\\	ジョンは 日本語[にほんご]の 学生[がくせい]です。
\\	ジョンが林檎を食べた。	
\\	ジョン が 林檎[りんご] を 食べた[たべた] 。
\\	メアリーが先生にほめられた。	
\\	メアリーが 先生[せんせい]にほめられた。
\\	ナンシーはきれいだ。	
\\	ナンシーはきれいだ。
\\	ドアが開いた。	
\\	ドアが 開[あ]いた。
\\	机が一つある。	
\\	机[つくえ]が 一[ひと]つある。
\\	空が青い。	
\\	空[そら]が 青[あお]い。
\\	中山さんは頭が痛かったので学校を休んだ。	
\\	中山[なかやま]さんは 頭[あたま]が 痛[いた]かったので 学校[がっこう]を 休[やす]んだ。
\\	私は車がある。	
\\	私[わたし]は 車[くるま]がある。
\\	僕はお金がいる。	
\\	僕[ぼく]はお 金[かね]がいる。
\\	スミスさんは中国語が分かる。	
\\	スミスさんは 中国語[ちゅうごくご]が 分[わ]かる。
\\	私はフランス語が少し出来る。	
\\	私[わたし]は フランス 語[ご]が 少[すこ]し 出来[でき]る。
\\	木下さんは東京でお父さんに会った。	
\\	木下[きのした]さんは 東京[とうきょう]でお 父[とう]さんに 会[あ]った。
\\	私にはベルが聞こえなかった。	
\\	私[わたし]にはベルが 聞[き]こえなかった。
\\	ここからは富士さんが見えますよ。	
\\	ここからは 富士[ふじ]さんが 見[み]えますよ。
\\	私たちは新幹線に乗りました。	
\\	私[わたし]たちは 新幹線[しんかんせん]に 乗[の]りました。
\\	私は母に似ているらしい。	
\\	私[わたし]は 母[はは]に 似[に]ているらしい。
\\	僕が行こう。	
\\	僕[ぼく]が 行[い]こう。
\\	僕が行きましょう。	
\\	僕[ぼく]が 行[い]きましょう。
\\	誰が来ましたか。	
\\	誰[だれ]が 来[き]ましたか。
\\	何を食べますか。	
\\	何[なに]を 食[た]べますか。
\\	どこに行きますか。	
\\	どこに 行[い]きますか。
\\	いつ大阪へ帰りますか。	
\\	いつ 大阪[おおさか]へ 帰[かえ]りますか。
\\	どうして買わないんですか。	
\\	どうして 買[か]わないんですか。
\\	東京駅へはどう行きますか。	
\\	東京[とうきょう] 駅[えき]へはどう 行[い]きますか。
\\	昨日のパーティーには誰が来ましたか。	
\\	昨日[きのう]のパーティーには 誰[だれ]が 来[き]ましたか。
\\	誰が昨日のパーティーに来ましたか。	
\\	誰[だれ]が 昨日[きのう]のパーティーに 来[き]ましたか。
\\	日本では何をしましたか。	
\\	日本[にほん]では 何[なに]をしましたか。
\\	何を日本でしました。	
\\	何[なに]を 日本[にほん]でしました。
\\	ー上田さんは来ましたか。 ーはい、来ました。	ー
\\	ー 上田[うえだ]さんは 来[き]ましたか。 ーはい、 来[き]ました。
\\	ー鈴木さんは学生ですか。 ーいいえ、そうじゃありません。	ー
\\	ー 鈴木[すずき]さんは 学生[がくせい]ですか。 ーいいえ、そうじゃありません。
\\	スミスさんは日本で日本語を勉強している。	
\\	スミスさんは 日本[にほん]で 日本語[にほんご]を 勉強[べんきょう]している。
\\	田中さんは学生だ。	
\\	田中[たなか]さんは 学生[がくせい]だ。
\\	田中さんが学生です。	
\\	田中[たなか]さんが 学生[がくせい]です。
\\	山川さんは若い。	
\\	山川[やまかわ]さんは 若[わか]い。
\\	山川さんは元気です。	
\\	山川[やまかわ]さんは 元気[げんき]です。
\\	本田さんはテニスが上手だ。	
\\	本田[ほんだ]さんはテニスが 上手[じょうず]だ。
\\	象は鼻が長いです。	
\\	象[ぞう]は 鼻[はな]が 長[なが]いです。
\\	スミスさんは学校に行く。	
\\	スミスさんは 学校[がっこう]に 行[い]く。
\\	この町は大学が二つあります。	
\\	この 町[まち]は 大学[だいがく]が 二[ふた]つあります。
\\	大学はこの町に二つある。	
\\	大学[だいがく]はこの 町[まち]に 二[ふた]つある。
\\	花子は学生です。	
\\	花子[はなこ]は 学生[がくせい]です。
\\	昔々一人のおじいさんが住んでいました。おじいさんはとても貧乏でした。	
\\	昔々[むかしむかし]、 一人[ひとり]のおじいさんが 住[す]んでいました。おじいさんはとても 貧乏[びんぼう]でした。
\\	これは私のペンです。	
\\	これは 私[わたし]のペンです。
\\	鈴木さんは日本人です。	
\\	鈴木[すずき]さんは 日本人[にほんじん]です。
\\	鯨は哺乳動物です。	
\\	鯨[くじら]は 哺乳[ほにゅう] 動物[どうぶつ]です。
\\	スミスさんは日本へ行きました。	
\\	スミスさんは 日本[にほん]へ 行[い]きました。
\\	その映画はもう見ました。	
\\	その 映画[えいが]はもう 見[み]ました。
\\	日本へはまだ行ったことがない。	
\\	日本[にほん]へはまだ 行[い]ったことがない。
\\	日本はまだ行ったことがない。	
\\	日本[にほん]はまだ 行[い]ったことがない。
\\	日本にはまだ行ったことがない。	
\\	日本[にほん]にはまだ 行[い]ったことがない。
\\	まり子には本を上げた。	
\\	まり 子[こ]には 本[ほん]を 上[あ]げた。
\\	アメリカからはスミスさんが来た。	
\\	アメリカからはスミスさんが 来[き]た。
\\	太郎はまだアメリカに行ったことがない。いつも行きたいと思っているがお金がないから行けない。お父さんは働いてお金をためなさいと言う。でも太郎はお父さんにお金を出して欲しいのだ。	
\\	太郎[たろう]はまだアメリカに 行[い]ったことがない。いつも 行[い]きたいと 思[おも]っているがお 金[かね]がないから 行[い]けない。お 父[とう]さんは 働[はたら]いてお 金[かね]をためなさいと 言[い]う。でも 太郎[たろう]はお 父[とう]さんにお 金[かね]を 出[だ]して 欲[ほ]しいのだ。
\\	ー太郎はその店で何を買いましたか。 ーペンを買いました。	ー
\\	ー 太郎[たろう]はその 店[みせ]で 何[なに]を 買[か]いましたか。 ーペンを 買[か]いました。
\\	ー田中さんは迎えに来てくれましたか。 ーはい、来てくれました。	ー
\\	ー 田中[たなか]さんは 迎[むか]えに 来[き]てくれましたか。 ーはい、 来[き]てくれました。
\\	ーあなたはアメリカにいた時、学生でしたか。 ーはい、そうでした。	ー
\\	ーあなたはアメリカにいた 時[とき]、 学生[がくせい]でしたか。 ーはい、そうでした。
\\	-行きますか。 -はい、行きます。	
\\	行[い]きますか。 はい、 行[い]きます。
\\	-私達は行きましょうか。 -いや、私達はよしましょう。	
\\	私[わたし] 達[たち]は 行[い]きましょうか。 いや、 私[わたし] 達[たち]はよしましょう。
\\	-これは何ですか。 -それは花子からのプレゼントです。	
\\	これは 何[なん]ですか。 それは 花[はな] 子[こ]からのプレゼントです。
\\	私山田と申します。	
\\	私[わたくし] 山田[やまだ]と 申[もう]します。
\\	君は今何年生ですか。	
\\	君[きみ]は 今[いま] 何[なん] 年生[ねんせい]ですか。
\\	この本は面白いよ。	
\\	この 本[ほん]は 面白[おもしろ]いよ。
\\	あの人誰ですか。	
\\	あの 人[ひと] 誰[だれ]ですか。
\\	あ、電車来た。	
\\	あ、 電車[でんしゃ] 来[き]た。
\\	皆さん、お食事が出来ました。	
\\	皆[みな]さん、お 食事[しょくじ]が 出来[でき]ました。
\\	今日田中さんから君に電話があったよ。	
\\	今日[きょう] 田中[たなか]さんから 君[きみ]に 電話[でんわ]があったよ。
\\	もしもし、靴の紐がほどけてますよ。	
\\	もしもし、 靴[くつ]の 紐[ひも]がほどけてますよ。
\\	-コーヒー飲みませんか。 -じゃ、それもらいます。	
\\	コーヒー 飲[の]みませんか。 じゃ、それもらいます。
\\	今晩電話を下さい。	
\\	今晩[こんばん] 電話[でんわ]を 下[くだ]さい。
\\	-きのうパーティにこなかったね。 -うん、ちょっと忙しかったから。	
\\	ーきのうパーティにこなかったね。 ーうん、ちょっと 忙[いそが]しかったから。
\\	田中先生に聞いたらどうですか。	
\\	田中[たなか] 先生[せんせい]に 聞[き]いたらどうですか。
\\	お茶がはいりましたけどいかがですか。	
\\	お 茶[ちゃ]がはいりましたけどいかがですか。
\\	僕は雪子はあまり好きじゃない。しかし春子は好きだ。	
\\	僕[ぼく]は 雪子[ゆきこ]はあまり 好[す]きじゃない。しかし 春子[はるこ]は 好[す]きだ。
\\	僕は雪子はあまり好きじゃない。しかし春子は雪子が好きだ。	
\\	僕[ぼく]は 雪子[ゆきこ]はあまり 好[す]きじゃない。しかし 春子[はるこ]は 雪子[ゆきこ]が 好[す]きだ。
\\	私はいい人を見つけたので、早速田中さんにその人に会ってもらった。	
\\	私[わたし]はいい 人[ひと]を 見[み]つけたので、 早速[さっそく] 田中[たなか]さんにその 人[ひと]に 会[あ]ってもらった。
\\	私はいい人を見つけたので、早速その人に田中さんに会ってもらった。	
\\	私[わたし]はいい 人[ひと]を 見[み]つけたので、 早速[さっそく]その 人[ひと]に 田中[たなか]さんに 会[あ]ってもらった。
\\	忙しい私はテレビも見られない。	
\\	忙[いそが]しい 私[わたし]はテレビも 見[み]られない。
\\	音楽が好きな私は一日中音楽をきいている。	
\\	音楽[おんがく]が 好[す]きな 私[わたし]は 一日[いちにち] 中[じゅう] 音楽[おんがく]をきいている。
\\	若いあなたにはまだ分からないでしょう。	
\\	若い[わかい] あなたにはまだ 分か[わか] らないでしょう 。
\\	私は部屋で友達と話していた。	
\\	私[わたし] は 部屋[へや] で 友達[ともだち] と 話し[はなし] ていた 。
\\	あなたは昨日部屋で友達と何をしていましたか。	
\\	あなたは 昨日[きのう] 部屋[へや] で 友達[ともだち] と 何を[なにを] していましたか 。
\\	あのね、京子昨日一郎さんとデートしたのよ。	
\\	あのね 、 京子[きょうこ] 昨日[きのう] 一郎[いちろう] さんと デート したのよ 。
\\	山田さんは去年アメリカに行った。英語が勉強したかったのだ。	
\\	山田[やまだ] さんは 去年[きょねん] アメリカ に 行[い] った 。 英語[えいご] が 勉強[べんきょう] したかったのだ 。
\\	上田さんは十年前に買った車にまだ乗っている。	
\\	上田[うえだ] さんは 十年前[じゅうねんまえ] に 買[か] った 車[くるま] にまだ 乗って[のって] いる 。
\\	-ボブは昨日来ましたか。 -ええ、来ましたよ。	ー
\\	-ボブ は 昨日[きのう] 来ま[きま] したか 。 -ええ 、 来ま[きま] したよ 。
\\	ー田中さんたち、遅いわねえ。 ー忘れているんじゃないかしら。	ー
\\	ー 田中[たなか] さんたち 、 遅い[おそい] わねえ 。 ー 忘れ[わすれ] ているんじゃないかしら 。
\\	ここから大阪まで車で一時間かかる。	
\\	ここから 大阪[おおさか] まで 車[くるま] で 一時間[いちじかん] かかる 。
\\	今七時です。	
\\	今[いま] 七時[しちじ] です 。
\\	僕の彼女、とてもきれいなんだ。	
\\	僕[ぼく] の 彼女[かのじょ] 、 とてもきれいなんだ 。
\\	大学を出た彼はすぐ結婚した。	
\\	大学[だいがく] を 出た[でた] 彼は[かれは] すぐ 結婚[けっこん] した 。
\\	山田さんは奥さんに逃げられた。	
\\	山田[やまだ] さんは 奥さん[おくさん] に 逃げ[にげ] られた 。
\\	太郎は春子にたばこを吸われた。	
\\	太郎[たろう] は 春子[はるこ] にたばこを 吸わ[すわ] れた 。
\\	木村さんは美人に横に座られてうれしそうだ。	
\\	木村[きむら] さんは 美人[びじん] に 横[よこ] に 座ら[すわら] れてうれしそうだ 。
\\	僕は雨に降られた。	
\\	僕は[ぼくは] 雨[あめ] に 降ら[ふら] れた 。
\\	ジョンはビルにぶたれた。	
\\	ジョン は ビル にぶたれた 。
\\	この絵は十九世紀に描かれた。	
\\	この 絵[え] は 十九[じゅうきゅう] 世紀[せいき] に 描か[かか] れた 。
\\	酒は米から作られる。	
\\	酒[さけ] は 米[こめ] から 作ら[つくら] れる 。
\\	マーガレットはポールにイヤリングをもらった。	
\\	マーガレット は ポール に イヤリング をもらった 。
\\	メアリーはスティーブにアパートに来てもらった。	
\\	メアリー は スティーブ に アパート に 来て[きて] もらった 。
\\	私は父にカメラを買ってもらった。	
\\	私[わたし] は 父[ちち] に カメラ を 買って[かって] もらった 。
\\	山口さんはアメリカに行くことにした。	
\\	山口[やまぐち] さんは アメリカ に 行く[いく] ことにした 。
\\	山口さんはアメリカに行くことになった。	
\\	山口[やまぐち] さんは アメリカ に 行く[いく] ことになった 。
\\	先生はフレッドを停学にした。	
\\	先生[せんせい] は フレッド を 停学[ていがく] にした 。
\\	フレッドは停学になった。	
\\	フレッド は 停学[ていがく] になった 。
\\	先生はアメリカの大学で日本語をお教えになります。	
\\	先生[せんせい] は アメリカ の 大学[だいがく] で 日本語[にほんご] をお 教[おし]え になります 。
\\	先生はアメリカの大学で日本語を教えられます。	
\\	先生[せんせい] は アメリカ の 大学[だいがく] で 日本語[にほんご] を 教[おし]え られます 。
\\	先生はゴルフをなさるとおっしゃいました。	
\\	先生[せんせい] は ゴルフ をなさるとおっしゃいました 。
\\	先生は今ご本をお読みでいらっしゃいます。	
\\	先生[せんせい] は 今[いま] ご 本[ほん] をお 読み[よみ] でいらっしゃいます 。
\\	先生は今ご本を読んでいらっしゃいます。	
\\	先生[せんせい] は 今[いま] ご 本[ほん] を 読んで[よんで] いらっしゃいます 。
\\	先生は今ご本をお読みだ。	
\\	先生[せんせい] は 今[いま] ご 本[ほん] をお 読み[よみ] だ 。
\\	お若いですね。	
\\	お 若い[わかい] ですね 。
\\	お父さんはお元気でいらっしゃいますか。	
\\	お 父さん[とうさん] はお 元気[げんき] でいらっしゃいますか 。
\\	私がそのお荷物をお持ちします。	
\\	私[わたし] がそのお 荷物[にもつ] をお 持ち[もち] します 。
\\	母がお連れいたします。	
\\	母[はは]がお 連[つ]れいたします。
\\	先生はお車をお持ちですか。	
\\	先生[せんせい] はお 車[くるま] をお 持ち[もち] ですか 。
\\	田中先生はご本をお書きになった。	
\\	田中[たなか] 先生[せんせい] はご 本[ほん] をお 書き[かき] になった 。
\\	今晩はお刺身とお吸い物にしましょうか。	
\\	今晩[こんばん] はお 刺身[さしみ] とお 吸い[すい] 物[もの] にしましょうか 。
\\	お手洗いはどこですか。	
\\	お 手洗い[てあらい] はどこですか 。
\\	先生は今日研究室にいらっしゃるはずです。	
\\	先生[せんせい] は 今日[きょう] 研究室[けんきゅうしつ] にいらっしゃるはずです 。
\\	私が先生にご連絡するつもりです。	
\\	私[わたし] が 先生[せんせい] にご 連絡[れんらく] するつもりです 。
\\	山田さんは先生です。	
\\	山田[やまだ] さんは 先生[せんせい] です 。
\\	山田さんは先生ですか。	
\\	山田[やまだ] さんは 先生[せんせい] ですか 。
\\	山田さんは先生ですね。	
\\	山田[やまだ] さんは 先生[せんせい] ですね 。
\\	来てね。	
\\	来て[きて] ね 。
\\	一緒に行きましょうね。	
\\	一緒に[いっしょに] 行き[いき] ましょうね 。
\\	学校へ行きなさいね。	
\\	学校[がっこう] へ 行き[いき] なさいね 。
\\	こっちへ来るな。	
\\	こっちへ 来る[くる] な 。
\\	動くな。	
\\	動く[うごく] な 。
\\	一郎は一年生だな。	
\\	一郎[いちろう] は 一年生[いちねんせい] だな 。
\\	一緒に行こうな。	
\\	一緒に[いっしょに] 行こ[いこ] うな 。
\\	おかしいな。誰もいない。どうしたんだろう。	
\\	おかしいな 。 誰も[だれも] いない 。 どうしたんだろう 。
\\	それはちょっと難しいですな。	
\\	それはちょっと 難しい[むずかしい] ですな 。
\\	私は知りませんよ。	
\\	私[わたし] は 知り[しり] ませんよ 。
\\	俺は負けないぞ。	
\\	俺[おれ] は 負け[まけ] ないぞ 。
\\	これは金だぜ。	
\\	これは 金[かね] だぜ 。
\\	今晩は飲もうよ。	
\\	今晩[こんばん] は 飲も[のも] うよ 。
\\	今晩は飲もうぜ。	
\\	今晩[こんばん] は 飲も[のも] うぜ 。
\\	私はまだ十八ですわ。	
\\	私[わたし] はまだ 十八[じゅうはち] ですわ 。
\\	あたし、うれしいわ。	
\\	あたし 、 うれしいわ 。
\\	これ、高いわね。	
\\	これ 、 高い[たかい] わね 。
\\	私も行くわよ。	
\\	私[わたし] も 行く[いく] わよ 。
\\	きれいだねえ。	
\\	きれいだねえ 。
\\	きれいだわねえ。	
\\	きれいだわねえ 。
\\	鈴木君はよく働くなあ。	
\\	鈴木[すずき] 君[くん] はよく 働く[はたらく] なあ 。
\\	すてきだわあ。	
\\	すてきだわあ 。
\\	これは君の本かい。	
\\	これは 君[きみ] の 本[ほん] かい 。
\\	これは誰の本だい。	
\\	これは 誰[だれ] の 本[ほん] だい 。
\\	友子さん、パーティーに行かないの?	
\\	友子[ともこ] さん 、 パーティー に 行か[いか] ないの 
\\	何を買うの?	
\\	何を[なにを] 買う[かう] の 
\\	私にも出来るかしら。	
\\	私[わたし] にも 出来る[できる] かしら 。
\\	明日は雨かなあ。	
\\	明日[あした] は 雨[あめ] かなあ 。
\\	私は良子に花を上げた。	
\\	私[わたし] は 良子[よしこ] に 花[はな] を 上げ[あげ] た 。
\\	大野さんは山本さんに本を上げた。	
\\	大野[おおの] さんは 山本[やまもと] さんに 本[ほん] を 上げ[あげ] た 。
\\	君はアンに何を上げましたか。	
\\	君[きみ] は アン に 何を[なにを] 上げ[あげ] ましたか 。
\\	花子は私に本をくれた。	
\\	花子[はなこ] は 私[わたし] に 本[ほん] をくれた 。
\\	花子は私の息子に本をくれた。	
\\	花子[はなこ] は 私[わたし] の 息子[むすこ] に 本[ほん] をくれた 。
\\	私は先生に本を差し上げました。	
\\	私[わたし] は 先生[せんせい] に 本[ほん] を 差[さ]し上げ[あげ] ました 。
\\	私は弟に本をやった。	
\\	私[わたし] は 弟[おとうと] に 本[ほん] をやった 。
\\	ひろしは猫にミルクをやった。	
\\	ひろしは 猫[ねこ] に ミルク をやった 。
\\	私はトムにケーキをやりました。	
\\	私[わたし] は トム に ケーキ をやりました 。
\\	私は信男さんにネクタイを買って上げました。	
\\	私[わたし] は 信男[のぶお] さんに ネクタイ を 買って[かって] 上げ[あげ] ました 。
\\	僕は春子さんを慰めて上げた。	
\\	僕は[ぼくは] 春子[はるこ] さんを 慰め[なぐさめ] て 上げ[あげ] た 。
\\	君はお母さんに何をして上げましたか。	
\\	君[きみ] はお 母さん[かあさん] に 何を[なにを] して 上げ[あげ] ましたか 。
\\	田中さんはスミスさんに本を貸して上げた。	
\\	田中[たなか] さんは スミス さんに 本[ほん] を 貸し[かし] て 上げ[あげ] た 。
\\	花子は私に日本語を教えてくれた。	
\\	花子[はなこ] は 私[わたし] に 日本語[にほんご] を 教[おし]え てくれた 。
\\	メアリーは私の娘にペンを買って呉れた。	
\\	メアリーは 私[わたし]の 娘[むすめ]にペンを 買[か]って 呉[く]れた。
\\	私達はジョンのために働いた。	
\\	私達[わたしたち] は ジョン のために 働い[はたらい] た 。
\\	私は先生にピアノを弾いて差し上げました。	
\\	私[わたし] は 先生[せんせい] に ピアノ を 弾い[ひい] て 差し上げ[さしあげ] ました 。
\\	私は妹にレコードを買って遣った。	
\\	私[わたし] は 妹[いもうと] に レコード を 買って[かって] 遣った[やった] 。
\\	じゅんは猫に魚を焼いて遣った。	
\\	じゅんは 猫[ねこ] に 魚[さかな] を 焼い[やい] て 遣った[よった] 。
\\	私はジョンにラジオを貸した遣りました。	
\\	私[わたし] は ジョン に ラジオ を 貸し[かし] た 遣り[やり] ました 。
\\	私がご飯を食べている間山田さんはテレビを見ていました。	
\\	私[わたし] が ご飯[ごはん] を 食べて[たべて] いる 間[あいだ] 山田[やまだ] さんは テレビ を 見て[みて] いました 。
\\	私がご飯を食べている間に山田さんは来た。	
\\	私[わたし] が ご飯[ごはん] を 食べて[たべて] いる 間[あいだ] に 山田[やまだ] さんは 来た[きた] 。
\\	スミスさんは日本にいる間英語を教えていました。	
\\	スミス さんは 日本[にほん] にいる 間[あいだ] 英語[えいご] を 教[おし]え ていました 。
\\	子供達がテレビを見ている間私は本を読んでいました。	
\\	子供達[こどもたち] が テレビ を 見て[みて] いる 間[あいだ] 私[わたし] は 本[ほん] を 読んで[よんで] いました 。
\\	高橋さんはアメリカにいる間にゴルフを覚えました。	
\\	高橋[たかはし] さんは アメリカ にいる 間[あいだ] に ゴルフ を 覚え[おぼえ] ました 。
\\	中川さんのお母さんは中川さんがパリに留学している間に病気になりました。	
\\	中川[なかがわ] さんのお 母さん[かあさん] は 中川[なかがわ] さんが パリ に 留学[りゅうがく] している 間に[あいだ] に 病気[びょうき] に なりました 。
\\	秋子は子供がいない間に本を読む。	
\\	秋子[あきこ] は 子供[こども] がいない 間に[まに] 本[ほん] を 読む[よむ] 。
\\	夏休みの間レストランでアルバイトをしました。	
\\	夏休み[なつやすみ] の 間[あいだ] レストラン で アルバイト をしました 。
\\	夏休みの間にガールフレンドが出来ました。	
\\	夏休み[なつやすみ] の 間[あいだ] に ガールフレンド が 出来ま[できま] した 。
\\	長い間母に手紙を書いていません。	
\\	長い[ながい] 間[あいだ] 母[はは] に 手紙[てがみ] を 書い[かい] ていません 。
\\	若い間に本を出来るだけたくさん読みなさい。	
\\	若い[わかい] 間[あいだ] に 本[ほん] を 出来る[できる] だけ たくさん 読み[よみ] なさい 。
\\	日本とアメリカの間には海しかない。	
\\	日本[にほん]とアメリカの 間[あいだ]には 海[うみ]しかない。
\\	山口と僕との間は決して悪くない。	
\\	山口[やまぐち] と 僕[ぼく] との 間[あいだ] は 決して[けっして] 悪く[わるく] ない 。
\\	若い間にスポーツを色々した方がいいです。	
\\	若い[わかい] 間[あいだ] に スポーツ を 色々[いろいろ] した 方が[ほうが] いい です 。
\\	雪子はご飯を食べている間テレビを見ていた。	
\\	雪子[ゆきこ] は ご飯[ごはん] を 食べて[たべて] いる 間[あいだ] テレビ を 見て[みて] いた 。
\\	雪子はご飯を食べながらテレビを見ていた。	
\\	雪子[ゆきこ] は ご飯[ごはん] を 食べ[たべ] ながら テレビ を 見て[みて] いた 。
\\	この本はあまりよくない。	
\\	この 本[ほん] はあまり よく ない 。
\\	鈴木さんはあまり食べない。	
\\	鈴木[すずき] さんはあまり 食べ[たべ] ない 。
\\	僕はあまり速く走れません。	
\\	僕[ぼく]はあまり 速[はや]く 走[はし]れません。
\\	スコットがあまり失礼なことを言うのでサリーは怒って帰ってしまった。	
\\	スコット があまり 失礼[しつれい] なことを 言う[いう] ので サリー は 怒って[おこって] 帰って[かえって] しまった 。
\\	井上さんはあまり酒を飲んだので病気になった。	
\\	井上[いのうえ] さんはあまり 酒[さけ] を 飲ん[のん] だので 病気[びょうき] になった 。
\\	それはあんまりです。	
\\	それはあんまりです 。
\\	この町には大学が三つあります。	
\\	この 町[まち] には 大学[だいがく] が 三つ[みっつ] あり ます 。
\\	エッフェルトーはパリにある。	
\\	エッフェルトー は パリ にある 。
\\	私のアパートにはテレビが二台ある。	
\\	私[わたし] の アパート には テレビ が 二台[にだい] ある 。
\\	デトロイトはミシガン州にあります。	
\\	デトロイト は ミシガン 州[しゅう] に あり ます 。
\\	和田さんのうちはクーラーがない。	
\\	和田[わだ] さんの うち は クーラー がない 。
\\	ー早稲田大学はどこにありますか。 ー早稲田大学は東京にあります。	ー
\\	ー 早稲田大学[わせだだいがく] は どこ に あり ますか 。 ー 早稲田大学[わせだだいがく] は 東京[とうきょう] に あり ます 。
\\	ー東京には早稲田大学しかありませんか。 ーいいえ、東京には大学がたくさんあります。	ー
\\	ー 東京[とうきょう] には 早稲田大学[わせだだいがく] しかありませんか 。 ー いいえ 、 東京[とうきょう] には 大学[だいがく] が たくさん あり ます 。
\\	この車にはカー・ステレオがある。	
\\	この 車[くるま] には カー ・ ステレオ が ある 。
\\	僕は車がある。	
\\	僕は[ぼくは] 車[くるま] がある 。
\\	僕には子供が三人ある。	
\\	僕[ぼく] には 子供[こども] が 三人[さんにん] ある。
\\	今日サムのうちでパーティーがある。	
\\	今日[きょう]サムのうちでパーティーがある。
\\	ジョンはいい車を持っている。	
\\	ジョン は いい 車[くるま] を 持って[もって] いる 。
\\	この大学はいい図書館を持っている。	
\\	この 大学[だいがく] は いい 図書館[としょかん] を 持って[もって] いる 。
\\	それはもうジョンに話してある。	
\\	それはもう ジョン に 話し[はなし] てある 。
\\	飲み物はもう買ってあります。	
\\	飲み物[のみもの] はもう 買って[かって] あります 。
\\	窓が開いている。	
\\	窓[まど] が 開い[あい] ている 。
\\	僕は窓を開けておく。	
\\	僕は[ぼくは] 窓[まど] を 開け[あけ] ておく 。
\\	メイソンさんは日本へ行った後で病気になりました。	
\\	メイソン さんは 日本[にほん] へ 行[い] った 後で[あとで] 病気[びょうき] になりました 。
\\	僕は授業の後で図書館に行った。	
\\	僕は[ぼくは] 授業[じゅぎょう] の 後で[あとで] 図書館[としょかん] に 行[い] った 。
\\	ご飯を食べた後すぐ勉強しました。	
\\	ご 飯[はん] を 食べた[たべた] 後[あと] すぐ 勉強[べんきょう] しました 。
\\	山田さんはビールを飲んだ後で寝てしまった。	
\\	山田[やまだ] さんは ビール を 飲ん[のん] だ 後で[あとで] 寝て[ねて] しまった 。
\\	戦争が終わった後東京に戻りました。	
\\	戦争[せんそう]が 終[お]わった 後[あと] 東京[とうきょう]に 戻[もど]りました。
\\	食事の後でテニスをした。	
\\	食事[しょくじ] の 後で[あとで] テニス をした 。
\\	私がうちへ帰った後で雨が降った。	
\\	私[わたし] が うち へ 帰った[かえった] 後で[あとで] 雨[あめ] が 降った[ふった] 。
\\	日本へ来てから何年になりますか。	
\\	日本[にほん] へ 来て[きて] から 何年[なんねん] になりますか 。
\\	ご飯を食べたらすぐ勉強しました。	
\\	ご 飯[はん] を 食べた[たべた] らすぐ 勉強[べんきょう] しました 。
\\	山田さんはビールを飲んだら寝てしまった。	
\\	山田[やまだ] さんは ビール を 飲ん[のん] だら 寝て[ねて] しまった 。
\\	戦争が終わったら東京に戻りました。	
\\	戦争[せんそう] が 終わ[おわ] ったら 東京[とうきょう] に 戻り[もどり] ました 。
\\	日本へ行ったらいいでしょう。	
\\	日本[にほん] へ 行[い] った ら いい でしょう 。
\\	これを読んだらどうですか。	
\\	これを 読んだ[よんだ] らどうですか 。
\\	この薬を飲めばよくなる。	
\\	この 薬[くすり] を 飲め[のめ] ば よく なる 。
\\	これは松本先生に聞けば分かります。	
\\	これは 松本[まつもと] 先生[せんせい] に 聞け[きけ] ば 分か[わか] ります 。
\\	その町は車で行けば三十分で行ける。	
\\	その 町[まち] は 車[くるま] で 行け[いけ] ば 三十[さんじゅう] 分[ぷん] で 行け[いけ] る 。
\\	安ければ買います。	
\\	安け[やすけ] れば 買い[かい] ます 。
\\	時間があれば京都へも行きたい。	
\\	時間[じかん] があれば 京都[きょうと] へも 行き[いき] たい 。
\\	見たければ見なさい。	
\\	見た[みた] ければ 見な[みな] さい 。
\\	出来ればこれも遣って下さい。	
\\	出来[でき]ればこれも 遣[や]って 下[くだ]さい。
\\	山本さんは来たら知らせて下さい。	
\\	山本[やまもと] さんは 来た[きた] ら 知らせ[しらせ] て 下[くだ]さい 。
\\	シカゴへ行くのならバスで行ったらどうですか。	
\\	シカゴ へ 行く[いく] のなら バス で 行[い] った らどうですか 。
\\	もっと安ければ買いました。	
\\	もっと 安け[やすけ] れば 買い[かい] ました 。
\\	雨が降ればよく家で本を読んだ物だ。	
\\	雨[あめ] が 降れ[ふれ] ばよく 家[いえ] で 本[ほん] を 読んだ[よんだ] 物[もの] だ 。
\\	日本へ行ったら日本語が上手になった。	
\\	日本[にほん] へ 行[い] った ら 日本語[にほんご] が 上手[じょうず] になった 。
\\	その檻に近づけば危ないですよ。	
\\	その 檻[おり] に 近づ[ちかづ] けば 危な[あぶな] いですよ 。
\\	デザートは食べるばかりになっているいます。	
\\	デザート は 食[た]べるばかりになっているいます 。
\\	友子は遊んでばかりいる。	
\\	友子[ともこ] は 遊んで[あそんで] ばかりいる 。
\\	僕は昼ご飯を食べたばかりだ。	
\\	僕は[ぼくは] 昼[ひる] ご 飯[はん] を 食べた[たべた] ばかりだ 。
\\	このレストランは安いばかりです。	
\\	この レストラン は 安い[やすい] ばかりです 。
\\	デニスはビールばかり飲んでいる。	
\\	デニス は ビール ばかり 飲ん[のん] でいる 。
\\	私は一月ばかりパリにいました。	
\\	私[わたし] は 一月[ひとつき] ばかり パリ にいました 。
\\	このレポートは後結論を書くばかりだ。	
\\	この レポート は 後[あと] 結論[けつろん] を 書く[かく] ばかりだ 。
\\	春江は泣いてばかりいて何も話そうとしない。	
\\	春江[はるえ] は 泣い[ない] てばかりいて 何も[なにも] 話そ[はなそ] うとしない 。
\\	松山さんは今シカゴに着いたばかりです。	
\\	松山[まつやま] さんは 今[いま] シカゴ に 着い[つい] たばかりです 。
\\	この仕事は面倒なばかりであまり儲かりません。	
\\	この 仕事[しごと] は 面倒[めんどう] なばかりであまり 儲か[もうか] りません 。
\\	この寮に住んでいるのは男子学生ばかりだ。	
\\	この 寮[りょう] に 住ん[すん] でいるのは 男子学生[だんしがくせい] ばかりだ 。
\\	フィッシャー先生はリサとばかり話している。	
\\	フィッシャー 先生[せんせい] は リサ とばかり 話し[はなし] ている 。
\\	十人ばかりの友達が手伝ってくれました。	
\\	十人[じゅうにん] ばかりの 友達[ともだち] が 手伝って[てつだって] くれました 。
\\	彼は笑うばかりで何も説明してくれない。	
\\	彼は[かれは] 笑う[わらう] ばかりで 何も[なにも] 説明し[せつめいし] てくれない 。
\\	先生に聞けば良かった。	
\\	先生[せんせい] に 聞け[きけ] ば 良か[よか] った 。
\\	あの本を読めば良かった。	
\\	あの 本[ほん] を 読め[よめ] ば 良か[よか] った 。
\\	ウエルズさんは日本へ行けば良かったね。	
\\	ウエルズ さんは 日本[にほん] へ 行け[いけ] ば 良か[よか] ったね 。
\\	ああ、もっと英語を勉強しておけば良かったなあ。	
\\	ああ 、 もっと 英語[えいご] を 勉強[べんきょう] しておけば 良か[よか] ったなあ 。
\\	ビルは日本へ行けば良かったと思っている。	
\\	ビル は 日本[にほん] へ 行け[いけ] ば 良か[よか] ったと 思って[おもって] いる 。
\\	あの本を読んで良かった。	
\\	あの 本[ほん] を 読んで[よんで] 良か[よか] った 。
\\	どこが静かだい。	
\\	どこが 静か[しずか] だい 。
\\	誰が行くんだい。	
\\	誰が[だれが] 行く[いく] んだい 。
\\	新しい仕事はどうだい。	
\\	新[あたら] しい 仕事[しごと] はどうだい 。
\\	あの人は誰だい。	
\\	あの 人[ひと] は 誰[だれ] だい 。
\\	何が可笑しいんだい。	
\\	何[なに] が 可笑しい[おかしい] んだい 。
\\	どの人が佐藤先生なんだい。	
\\	どの 人[ひと] が 佐藤[さとう] 先生[せんせい] なんだい 。
\\	これは僕のだい。	
\\	これは 僕[ぼく] のだい 。
\\	僕も行くんだい。	
\\	僕[ぼく] も 行く[いく] んだい 。
\\	誰が行く。	
\\	誰が[だれが] 行く[いく] 。
\\	どれが面白い。	
\\	どれが 面白い[おもしろい] 。
\\	新しい仕事はどうだった。	
\\	新[あたら] しい 仕事[しごと] はどうだった 。
\\	どこが静かだった。	
\\	どこ が 静か[しずか] だった 。
\\	スミスさんだけが来ました。	
\\	スミス さんだけが 来ま[きま] した 。
\\	私は日本へ一度だけ行った。	
\\	私[わたし] は 日本[にほん] へ 一度[いちど] だけ 行[い] った 。
\\	雪子さんとはデートしただけです。	
\\	雪子[ゆきこ] さんとは デート しただけです 。
\\	この家は大きいだけだ。	
\\	この 家[いえ] は 大き[おおき] いだけだ 。
\\	このお菓子は色がきれいなだけです。	
\\	このお 菓子[かし] は 色[いろ] がきれい なだけです 。
\\	佐藤さんだけ会議に出ました。	
\\	佐藤[さとう] さんだけ 会議[かいぎ] に 出ま[でま] した 。
\\	小さい和英辞典だけを買いました。	
\\	小さ[ちいさ] い 和英辞典[わえいじてん] だけを 買い[かい] ました 。
\\	僕にだけ話して下さい。	
\\	僕[ぼく] にだけ 話し[はなし] て 下[くだ]さい 。
\\	僕だけに話して下さい。	
\\	僕[ぼく] だけに 話し[はなし] て 下[くだ]さい 。
\\	この車はアルコールでだけ動きます。	
\\	この 車[くるま] は アルコール でだけ 動き[うごき] ます 。
\\	この車はアルコールだけで動きます。	
\\	この 車[くるま] は アルコール だけで 動き[うごき] ます 。
\\	あの先生には一度だけ会いました。	
\\	あの 先生[せんせい] には 一度[いちど] だけ 会[あ] い ました 。
\\	朝はコーヒーを一杯飲むだけです。	
\\	朝[あさ] は コーヒー を 一杯[いっぱい] 飲む[のむ] だけです 。
\\	この本は高いだけで面白くない。	
\\	この 本[ほん] は 高い[たかい] だけで 面白く[おもしろく] ない 。
\\	テニスは好きなだけで上手じゃない。	
\\	テニス は 好き[すき] なだけで 上手[じょうず] じゃない 。
\\	それは学生だけのパーティだった。	
\\	それは 学生[がくせい] だけの パーティ だった 。
\\	出来るだけゆっくり話して下さい。	
\\	出来る[できる] だけゆっくり 話し[はなし] て 下[くだ]さい 。
\\	魚だけを食べた。	
\\	魚[さかな] だけを 食べた[たべた] 。
\\	魚を食べただけだ。	
\\	魚[さかな] を 食べた[たべた] だけだ 。
\\	食べられるだけ食べたい。	
\\	食べ[たべ] られるだけ 食べた[たべた] い 。
\\	踊れるだけ踊ろう。	
\\	踊れ[おどれ] るだけ 踊ろ[おどろ] う 。
\\	あの人はよく勉強するだけでなくよく遊ぶ。	
\\	あの 人[ひと] は よく 勉強[べんきょう] するだけでなく よく 遊ぶ[あそぶ] 。
\\	あの人は頭がいいだけではなくよく勉強もします。	
\\	あの 人[ひと] は 頭[あたま] が いい だけではなく よく 勉強[べんきょう] もします 。
\\	このうちはきれいなだけでなくとても安い。	
\\	この うち は きれい なだけでなくとても 安い[やすい] 。
\\	ジョンだけではなくメアリーも来ました。	
\\	ジョン だけではなく メアリー も 来ま[きま] した 。
\\	あの人は小説を読むだけではなく書きます。	
\\	あの 人[ひと] は 小説[しょうせつ] を 読む[よむ] だけではなく 書き[かき] ます 。
\\	ブラウンさんは日本へ行っただけではなく住んだこともあります。	
\\	ブラウン さんは 日本[にほん] へ 行[い] った だけではなく 住ん[すん] だこともあります 。
\\	このアパートは高いだけでなく大変狭いです。	
\\	この アパート は 高い[たかい] だけでなく 大変[たいへん] 狭い[せまい] です 。
\\	幸子は頭が良かっただけではなく、とても親切でした。	
\\	幸子[さちこ] は 頭[あたま] が 良か[よか] っただけではなく 、 とても 親切[しんせつ] でした 。
\\	この車はきれいなだけではなく、よく走ります。	
\\	この 車[くるま] はきれい なだけではなく 、 よく 走り[はしり] ます 。
\\	日本人だけでなくアメリカ人もよく働きます。	
\\	日本人[にほんじん] だけでなく アメリカ 人[じん] も よく 働き[はたらき] ます 。
\\	中村さんはアメリカやヨーロッパだけではなく東南アジアにもよく出張します。	
\\	中村[なかむら] さんは アメリカ や ヨーロッパ だけではなく 東南アジア[とうなんあじあ] にも よく 出張[しゅっちょう] します 。
\\	ジョンソンさんは日本語だけではなく中国語も話せます。	
\\	ジョンソン さんは 日本語[にほんご] だけではなく 中国語[ちゅうごくご] も 話せ[はなせ] ます 。
\\	手紙は友達だけではなく先生からも来ました。	
\\	手紙[てがみ] は 友達[ともだち] だけではなく 先生[せんせい] からも 来ま[きま] した 。
\\	手紙は友達だけではなく先生にも書きました。	
\\	手紙[てがみ] は 友達[ともだち] だけではなく 先生[せんせい] にも 書き[かき] ました 。
\\	明日までに十課だけでなく十一課も勉強しておきなさい。	
\\	明日[あした] までに 十課[じゅっか] だけでなく 十一課[じゅういっか] も 勉強[べんきょう] しておきなさい 。
\\	アンダーソンさんは日本へ行くでしょう。	
\\	アンダーソン さんは 日本[にほん] へ 行く[いく] でしょう 。
\\	あのアパートは高いでしょう。	
\\	あの アパート は 高い[たかい] でしょう 。
\\	ロジャーはスキーが上手だろう。	
\\	ロジャー は スキー が 上手[じょうず] だろう 。
\\	あの人は中国人だろう。	
\\	あの 人[ひと] は 中国人[ちゅうごくじん] だろう 。
\\	アンダーソンさんは多分日本へ行くだろう。	
\\	アンダーソン さんは 多分[たぶん] 日本[にほん] へ 行く[いく] だろう 。
\\	アンダーソンさんはきっと日本へ行くだろう。	
\\	アンダーソンさんはきっと 日本[にほん]へ 行[い]くだろう。
\\	大丈夫ですか。	
\\	大丈夫[だいじょうぶ] ですか 。
\\	大丈夫でしょうか。	
\\	大丈夫[だいじょうぶ] でしょうか 。
\\	君も行くだろう。	
\\	君[きみ] も 行く[いく] だろう 。
\\	これ、きれいでしょう。	
\\	これ 、 きれい でしょう 。
\\	君も行くね。	
\\	君[きみ] も 行く[いく] ね 。
\\	車が動き出した。	
\\	車[くるま] が 動き[うごき] 出し[だし] た 。
\\	急に雨が降り出した。	
\\	急に[きゅうに] 雨[あめ] が 降り[ふり] 出し[だし] た 。
\\	一歳になって初めて歩き出した。	
\\	一歳[いっさい] になって 初めて[はじめて] 歩き[あるき] 出し[だし] た 。
\\	そのアイディアは誰が考え出したんですか。	
\\	その アイディア は 誰が[だれが] 考え[かんがえ] 出し[だし] たんですか 。
\\	一時間ぐらいかけてとうとうその本屋を探し出した。	
\\	一時間[いちじかん] ぐらいかけてとうとうその 本屋[ほんや] を 探し[さがし] 出し[だし] た 。
\\	そろそろ歩き始めましょうか。	
\\	そろそろ 歩き[あるき] 始め[はじめ] ましょうか 。
\\	私達が歩き始めた時山田達が来た。	
\\	私達[わたしたち] が 歩き[あるき] 始め[はじめ] た 時[とき] 山田[やまだ] 達[たち] が 来た[きた] 。
\\	私達が歩き出した時山田達が来た。	
\\	私達[わたしたち] が 歩き[あるき] 出し[だし] た 時[とき] 山田[やまだ] 達[たち] が 来た[きた] 。
\\	どうしてか分からなかったが、男は急に怒り出した。	
\\	どうしてか 分か[わか] らなかったが 、 男[おとこ] は 急に[きゅうに] 怒り[いかり] 出し[だし] た 。
\\	私達は喫茶店でコーヒーを飲んだ。	
\\	私達[わたしたち] は 喫茶店[きっさてん] で コーヒー を 飲ん[のん] だ 。
\\	ゆり子はデパートで働いています。	
\\	ゆり 子[こ] は デパート で 働い[はたらい] ています 。
\\	オーストラリアでは十二月は夏だ。	
\\	オーストラリア では 十二月[じゅうにがつ] は 夏[なつ] だ 。
\\	島崎さんは日本では元気でした。	
\\	島崎[しまざき] さんは 日本[にほん] では 元気[げんき] でした 。
\\	ヘレンは始めて舞台で歌った。	
\\	ヘレンは 始[はじ]めて 舞台[ぶたい]で 歌[うた]った。
\\	今晩ジムの家でパーティーがあります。	
\\	今晩[こんばん] ジム の 家[いえ] で パーティー があります 。
\\	昨日この部屋でプライス先生の講演があった。	
\\	昨日[きのう] この 部屋[へや] で プライス 先生[せんせい] の 講演[こうえん] があった 。
\\	宮本さんは毎日バスで会社へ行きます。	
\\	宮本[みやもと] さんは 毎日[まいにち] バス で 会社[かいしゃ] へ 行き[いき] ます 。
\\	日本人は箸でご飯を食べる。	
\\	日本人[にほんじん] は 箸[はし] でご 飯[はん] を 食[た]べる 。
\\	私達は日本語で話した。	
\\	私達[わたしたち] は 日本語[にほんご] で 話し[はなし] た 。
\\	私はその映画をテレビで見ました。	
\\	私[わたし] はその 映画[えいが] を テレビ で 見ま[みま] した 。
\\	豆腐は大豆で作ります。	
\\	豆腐[とうふ] は 大豆[だいず] で 作り[つくり] ます 。
\\	ラリーはこのいすを十ドルで買った。	
\\	ラリー はこのいすを 十[じゅう] ドル で 買[か] った 。
\\	ベッツィーはそのレポートを一日で書いたそうだ。	
\\	ベッツィー はその レポート を 一日[いちにち] で 書い[かい] たそうだ 。
\\	本田さんは何で大阪へ行くんですか。	
\\	本田[ほんだ] さんは 何で[なんで] 大阪[おおさか] へ 行く[いく] んですか 。
\\	私達はタクシを使って来ました。	
\\	私達[わたしたち] は タクシ を 使って[つかって] 来ま[きま] した 。
\\	私達はタクシで来ました。	
\\	私達[わたしたち] は タクシ で 来ま[きま] した 。
\\	私は子供を使って家具を動かした。	
\\	私[わたし] は 子供[こども] を 使って[つかって] 家具[かぐ] を 動か[うごか] した 。
\\	山口さんは病気で学校を休んだ。	
\\	山口[やまぐち] さんは 病気[びょうき] で 学校[がっこう] を 休ん[やすん] だ 。
\\	明日は期末試験で大変です。	
\\	明日[あした] は 期末試験[きまつしけん] で 大変[たいへん] です 。
\\	大雨で橋が壊れた。	
\\	大雨[おおあめ] で 橋[はし] が 壊れ[こわれ] た 。
\\	昨日は夜、仕事でとても疲れた。	
\\	昨日[きのう] は 夜[よる] 、 仕事[しごと] でとても 疲れ[つかれ] た 。
\\	父は交通事故で入院しました。	
\\	父[ちち] は 交通事故[こうつうじこ] で 入院[にゅういん] しました 。
\\	山口さんは病気でした。	
\\	山口[やまぐち] さんは 病気[びょうき] でした 。
\\	山口さんは学校を休みました。	
\\	山口[やまぐち] さんは 学校[がっこう] を 休み[やすみ] ました 。
\\	卵でアレルギーになる。	
\\	卵[たまご] で アレルギー になる 。
\\	卵でオムレツを作る。	
\\	卵[たまご] で オムレツ を 作る[つくる] 。
\\	春学期は五月十日で終わります。	
\\	春[はる] 学期[がっき] は 五月[ごがつ] 十日[とおか] で 終わり[おわり] ます 。
\\	このコンサートは十時で終わります。	
\\	この コンサート は 十時[じゅうじ] で 終わり[おわり] ます 。
\\	私のパスポートは六月で切れる。	
\\	私[わたし] の パスポート は 六月[ろくがつ] で 切れ[きれ] る 。
\\	アメリカに来てから今日で三年になる。	
\\	アメリカ に 来て[きて] から 今日[きょう] で 三年[さんねん] になる 。
\\	山田さんは一週間でそのレポートを書いた。	
\\	山田[やまだ] さんは 一週間[いっしゅうかん] でその レポート を 書い[かい] た 。
\\	春学期は五月十日に終わる。	
\\	春[はる] 学期[がっき] は 五月[ごがつ] 十日[とおか] に 終わ[おわ] る 。
\\	ジョンは車でシカゴへ行った。	
\\	ジョン は 車[くるま] で シカゴ へ 行[い] った 。
\\	私達は英語で話した。	
\\	私達[わたしたち] は 英語[えいご] で 話し[はなし] た 。
\\	その家は台風で壊れた。	
\\	その 家[いえ] は 台風[たいふう] で 壊れ[こわれ] た 。
\\	僕は試験で行けなかった。	
\\	僕は[ぼくは] 試験[しけん] で 行け[いけ] なかった 。
\\	私は毛糸で靴下を編んだ。	
\\	私[わたし]は 毛糸[けいと]で 靴下[くつした]を 編[あ]んだ。
\\	道子はいつも図書館で勉強する。	
\\	道子[みちこ] は いつ も 図書館[としょかん] で 勉強[べんきょう] する 。
\\	木村さんは三日でこのレポートを書いた。	
\\	木村[きむら] さんは 三日[みっか] でこの レポート を 書い[かい] た 。
\\	仕事は五時で終わります。	
\\	仕事[しごと] は 五時[ごじ] で 終わり[おわり] ます 。
\\	私はこの本を十ドルで買った。	
\\	私[わたし] はこの 本[ほん] を 十[じゅう] ドル で 買[か] った 。
\\	先生でも間違う。	
\\	先生[せんせい] でも 間違う[まちがう] 。
\\	私は難しい仕事でもします。	
\\	私[わたし] は 難しい[むずかしい] 仕事[しごと] でもします 。
\\	ここからでも富士山が見える。	
\\	ここからでも 富士山[ふじさん] が 見え[みえ] る 。
\\	子供でもそんなことは分かりますよ。	
\\	子供[こども] でもそんなことは 分か[わか] りますよ 。
\\	お会いしたいんですが、日曜日でもかまいませんか。	
\\	お 会[あ] い したいんですが 、 日曜日[にちようび] でも 構い[かまい] ませんか 。
\\	あの人は魚でも肉でも食べます。	
\\	あの 人[ひと] は 魚[さかな] でも 肉[にく] でも 食べ[たべ] ます 。
\\	あの図書館は誰でも入れます。	
\\	あの 図書館[としょかん] は 誰[だれ] でも 入れ[いれ] ます 。
\\	いつでもかまいませんよ。	
\\	いつ でもかまいませんよ 。
\\	日本でもインフレが大きな問題になっている。	
\\	日本[にほん] でも インフレ が 大き[おおき] な 問題[もんだい] になっている 。
\\	そこはバスでも行けますか。	
\\	そこは バス でも 行け[いけ] ますか 。
\\	お母さんはどうですか。	
\\	お 母さん[かあさん] は どう ですか 。
\\	新しいアパートはどうですか。	
\\	新[あたら] しい アパート は どう ですか 。
\\	コーヒーはどうですか。	
\\	コーヒーはどうですか 。
\\	土曜日はどう?	
\\	土曜日[どようび] は どう 
\\	この言葉はどう読みますか。	
\\	この 言葉[ことば] は どう 読み[よみ] ますか 。
\\	どうして日本語を勉強しているんですか。	
\\	どうして 日本語[にほんご] を 勉強[べんきょう] しているんですか 。
\\	この木はどうして枯れたんですか。	
\\	この 木[き] は どうして 枯れ[かれ] たんですか 。
\\	私は先週京都へ旅行に行った。	
\\	私[わたし] は 先週[せんしゅう] 京都[きょうと] へ 旅行[りょこう] に 行[い] った 。
\\	僕は札幌のうちへ飛行機で帰った。	
\\	僕は[ぼくは] 札幌[さっぽろ] のうちへ 飛行機[ひこうき] で 帰った[かえった] 。
\\	地震だったので急いで外へ出た。	
\\	地震[じしん] だったので 急い[いそい] で 外[そと] へ 出た[でた] 。
\\	父へ手紙を出したが、まだ返事が来ない。	
\\	父[ちち] へ 手紙[てがみ] を 出し[だし] たが 、 まだ 返事[へんじ] が 来ない[こない] 。
\\	はるばる東京まで来た。	
\\	はるばる 東京[とうきょう] まで 来た[きた] 。
\\	東京に十二時に着いた。	
\\	東京[とうきょう] に 十二[じゅうに] 時に[ときに] 着い[つい] た 。
\\	東京へ十二時に着いた。	
\\	東京[とうきょう] へ 十二[じゅうに] 時に[ときに] 着い[つい] た 。
\\	ニューヨークへの便はもうありません。	
\\	ニューヨーク への 便[びん] はもうありません 。
\\	ニューヨークまでの便はもうありません。	
\\	ニューヨーク までの 便[びん] はもうありません 。
\\	ここまでの道は悪かったでしょう。	
\\	ここまでの 道[みち] は 悪か[わるか] ったでしょう 。
\\	お風呂に入った。	
\\	お 風呂[ふろ] に 入[はい] った。
\\	机の上へ乗った。	
\\	机[つくえ] の 上[うえ] へ 乗った[のった] 。
\\	黒板に漢字を書いた。	
\\	黒板[こくばん] に 漢字[かんじ] を 書い[かい] た 。
\\	雨が降っている。	
\\	雨[あめ] が 降って[ふって] いる 。
\\	私の部屋にはステレオがあります。	
\\	私[わたし] の 部屋[へや] には ステレオ が あり ます 。
\\	あ、のり子が走っている。	
\\	あ 、 のり 子[こ] が 走って[はしって] いる 。
\\	ーこのレストランは何がおいしいですか。 ーステーキがおいしいです。	ー
\\	ー この レストラン は 何[なに] が おいしい ですか 。 ーステーキ が おいしい です 。
\\	今晩誰が来ますか。	
\\	今晩[こんばん] 誰が[だれが] 来ま[きま] すか 。
\\	僕はデビーはフランスへ行くことを知らなかった。	
\\	僕は[ぼくは] デビー は フランス へ 行く[いく] ことを 知ら[しら] なかった 。
\\	ジーンは僕がアパートを出た時まだ寝ていた。	
\\	ジーン は 僕[ぼく] が アパート を 出た[でた] 時[とき]まだ 寝て[ねて] いた 。
\\	僕が昨日見た映画はドイツの映画だった。	
\\	僕[ぼく] が 昨日[きのう] 見た[みた] 映画[えいが] は ドイツ の 映画[えいが] だった 。
\\	僕の昨日見た映画はドイツの映画だった。	
\\	僕[ぼく] の 昨日[きのう] 見た[みた] 映画[えいが] は ドイツ の 映画[えいが] だった 。
\\	僕はスポーツカーが欲しい。	
\\	僕は[ぼくは] スポーツカー が 欲しい[ほしい] 。
\\	春子はスペイン語が分かる。	
\\	春子[はるこ] は スペイン 語[ご] が 分か[わか] る 。
\\	ジョンは来ましたがメアリーは来ませんでした。	
\\	ジョン は 来ま[きま] したが メアリー は 来ま[きま] せんでした 。
\\	太郎は泳いだが次郎は泳がなかった。	
\\	太郎[たろう] は 泳い[およい] だが 次郎[じろう] は 泳が[およが] なかった 。
\\	私はビールは飲むが酒は飲まない。	
\\	私[わたし] は ビール は 飲む[のむ] が 酒[さけ] は 飲[の] ま ない 。
\\	旅行をしたいがお金がない。	
\\	旅行[りょこう] をしたいがお 金[かね] がない 。
\\	パーティーをしますが来ませんか。	
\\	パーティー をしますが 来ま[きま] せんか 。
\\	田中さんは私のうちにも時々来ますが面白い人ですね。	
\\	田中[たなか] さんは 私[わたし] の うち にも 時々[ときどき] 来ま[きま] すが 面白い[おもしろい] 人[ひと] ですね 。
\\	大丈夫だと思いますが。	
\\	大丈夫[だいじょうぶ] だと 思い[おもい] ますが 。
\\	トムはよく勉強するんですが。	
\\	トム は よく 勉強[べんきょう] するんですが 。
\\	一男はスポーツカーを欲しがった。	
\\	一男[かずお] は スポーツカー を 欲し[ほし] がった 。
\\	上田さんはアイスクリームを食べたがりました。	
\\	上田[うえだ] さんは アイスクリーム を 食べた[たべた] がりました 。
\\	スーザンは一人で寂しがっています。	
\\	スーザン は 一人[ひとり] で 寂し[さびし] がっています 。
\\	子供が眠たがっている。	
\\	子供[こども] が 眠た[ねむた] がっている 。
\\	僕は犬が怖い。	
\\	僕は[ぼくは] 犬[いぬ] が 怖い[こわい] 。
\\	深田さんは犬を怖がっている。	
\\	深田[ふかだ] さんは 犬[いぬ] を 怖が[こわが] っている 。
\\	一男はスポーツカーを欲しがっている。	
\\	一男[かずお] は スポーツカー を 欲し[ほし] がっている 。
\\	深田さんは犬が怖いと言った。	
\\	深田[ふかだ] さんは 犬[いぬ] が 怖い[こわい] と 言った[いった] 。
\\	一男はスポーツカーは欲しいそうだ。	
\\	一男[かずお] は スポーツカー は 欲しい[ほしい] そうだ 。
\\	鈴木さんは朝七時頃うちを出る。	
\\	鈴木[すずき] さんは 朝[あさ] 七時[しちじ] 頃[ごろ] うち を 出る[でる] 。
\\	ー何時頃北京に行きますか。 ー来年の六月頃行きます。	ー
\\	ー 何時頃[なんじごろ] 北京[ぺきん] に 行き[いき] ますか 。 ー 来年[らいねん] の 六月[ろくがつ] 頃[ごろ] 行き[いき] ます 。
\\	ー今朝は何時頃まで寝ていましたか。 ー十時頃まで寝ていました。	ー
\\	ー 今朝[けさ] は 何時頃[なんじごろ] まで 寝て[ねて] いましたか 。 ー 十時[じゅうじ] 頃[ごろ] まで 寝て[ねて] いました 。
\\	再来年の夏頃山中さんに会います。	
\\	再来年[さらいねん] の 夏頃[なつごろ] 山中[やまなか] さんに 会[あ] い ます 。
\\	一昨年の夏頃山中さんに会いました。	
\\	一昨年[おととし] の 夏頃[なつごろ] 山中[やまなか] さんに 会[あ] い ました 。
\\	僕は三時頃に寝ました。	
\\	僕は[ぼくは] 三時[さんじ] 頃に[ごろに] 寝ま[ねま] した 。
\\	僕は三時間ぐらい寝ました。	
\\	僕は[ぼくは] 三時[さんじ] 間[かん] ぐらい 寝ま[ねま] した 。
\\	私は三時間毎に薬を飲みました。	
\\	私[わたし] は 三時[さんじ] 間[かん] 毎に[ごとに] 薬[くすり] を 飲み[のみ] ました 。
\\	正月は家毎に門松を立てる。	
\\	正月[しょうがつ] は 家[いえ] 毎に[ごとに] 門松[かどまつ] を 立て[たて] る 。
\\	一課毎に試験がある。	
\\	一課[いっか] 毎に[ごとに] 試験[しけん] がある 。
\\	学期毎に先生が代わる。	
\\	学期[がっき] 毎[ごと]に 先生[せんせい]が 代[か]わる。
\\	木村さんは会う人毎に挨拶している。	
\\	木村[きむら] さんは 会う[あう] 人[ひと] 毎に[ごとに] 挨拶[あいさつ] している 。
\\	三日毎にテニスをしています。	
\\	三日[みっか] 毎に[ごとに] テニス をしています 。
\\	この電車は二駅おきに止まる。	
\\	この 電車[でんしゃ] は 二駅[ふたえき] おきに 止ま[とま] る 。
\\	この電車は二駅毎に止まる。	
\\	この 電車[でんしゃ] は 二駅[ふたえき] 毎に[ごとに] 止ま[とま] る 。
\\	電車は五分おきに出る。	
\\	電車[でんしゃ] は 五分[ごふん] おきに 出る[でる] 。
\\	電車は五分毎に出る。	
\\	電車[でんしゃ] は 五分[ごふん] 毎に[ごとに] 出る[でる] 。
\\	僕は二日おきに花子と会っている。	
\\	僕は[ぼくは] 二日[ふつか] おきに 花子[はなこ] と 会って[あって] いる 。
\\	僕は二日毎に花子と会っている。	
\\	僕は[ぼくは] 二日[ふつか] 毎に[ごとに] 花子[はなこ] と 会って[あって] いる 。
\\	明日からレポートを書き始めます。	
\\	明日[あした] から レポート を 書き始め[かきはじめ] ます 。
\\	私は八月から源氏の物語を読み始めました。	
\\	私[わたし] は 八月[はちがつ] から 源氏[げんじ] の 物語[ものがたり] を 読み始め[よみはじめ] ました 。
\\	日本の経済は一九六四年の東京オリンピックの頃から強くなり始めた。	
\\	日本[にほん]の 経済[けいざい]は 一九六四年[せんきゅうひゃくろくじゅうよねん]の 東京[とうきょう]オリンピックの 頃[ころ]から 強[つよ]くなり 始[はじ]めた。
\\	ーいつから切手を集め始めたんですか。 ー子供の時から集め始めました。	ー
\\	ー いつ から 切手[きって] を 集め[あつめ] 始め[はじめ] たんですか 。 ー 子供[こども] の 時[とき] から 集め[あつめ] 始め[はじめ] ました 。
\\	先生は本を書き始めた。	
\\	先生[せんせい] は 本[ほん] を 書き始め[かきはじめ] た 。
\\	先生はご本をお書きになり始めた。	
\\	先生[せんせい] はご 本[ほん] をお 書き[かき] になり 始め[はじめ] た 。
\\	先生はご本をお書き始めになった。	
\\	先生[せんせい] はご 本[ほん] をお 書き始め[かきはじめ] になった 。
\\	上田さんは新聞を読み始めました。	
\\	上田[うえだ] さんは 新聞[しんぶん] を 読み始め[よみはじめ] ました 。
\\	人々が会場に着き始めました。	
\\	人々[ひとびと] が 会場[かいじょう] に 着き[つき] 始め[はじめ] ました 。
\\	クラークさんはパーティーに行くはずだ。	
\\	クラークさんはパーティーに 行[い]くはずだ。
\\	大野先生はサンドラを知っているはずです。	
\\	大野[おおの] 先生[せんせい]はサンドラを 知[し]っているはずです。
\\	あの本はたかったはずだ。	
\\	あの 本[ほん]はたかったはずだ。
\\	そのアパートはきれいなはずです。	
\\	そのアパートはきれいなはずです。
\\	カールソンさんは昔先生だったはずだ。	
\\	カールソン さんは 昔[むかし] 先生[せんせい] だったはずだ 。
\\	ーマーサも来ますか。 ーはい、そのはずです。	ー
\\	ーマーサ も 来ま[きま] すか 。 ー はい 、 そのはずです 。
\\	クラークさんはパーティーに行かないはずだ。	
\\	クラーク さんは パーティー に 行か[いか] ないはずだ 。
\\	クラークさんはパーティーに行くはずはない。	
\\	クラーク さんは パーティー に 行く[いく]はずはない 。
\\	高いはずだ。こらは金だ。	
\\	高い[たかい] はずだ 。 こらは 金[きん] だ 。
\\	山口さんに出したはずの手紙が床に落ちていた。	
\\	山口[やまぐち] さんに 出し[だし] たはずの 手紙[てがみ] が 床[ゆか] に 落ち[おち] ていた 。
\\	私はケンほど強くない。	
\\	私[わたし] は ケン ほど 強く[つよく] ない 。
\\	この仕事は子供でも出来るほど易しいです。	
\\	この 仕事[しごと] は 子供[こども] でも 出来る[できる] ほど 易し[やすし] いです 。
\\	僕が君ほど速く歩けません。	
\\	僕[ぼく]が 君[きみ]ほど 速[はや]く 歩[ある]けません。
\\	ジェニファーほどよく勉強する学生はいない。	
\\	ジェニファー ほど よく 勉強[べんきょう] する 学生[がくせい] はいない 。
\\	僕は西田さんがあれほどピアノが弾けるとは思わなかった。	
\\	僕は[ぼくは] 西田[にしだ] さんがあれほど ピアノ が 弾け[ひけ] るとは 思わ[おもわ] なかった 。
\\	ルイスさんの日本語を太田先生も外国人だと気づかなかったほど上手です。	
\\	ルイス さんの 日本語[にほんご] を 太田[おおた] 先生[せんせい] も 外国人[がいこくじん] だと 気づ[きづ] かなかったほど 上手[じょうず] です 。
\\	その仕事はそれほど易しいんですか。	
\\	その 仕事[しごと] はそれほど 易し[やすし] いんですか 。
\\	ビールを三本ほど飲みました。	
\\	ビール を 三本[さんぼん] ほど 飲み[のみ] ました 。
\\	その林檎を二つばかり下さい。	
\\	その 林檎[りんご] を 二つ[ふたつ] ばかり 下[くだ]さい 。
\\	その林檎を二つほど下さい。	
\\	その 林檎[りんご] を 二つ[ふたつ] ほど 下[くだ]さい 。
\\	日本語の本を読んだ方がいいです。	
\\	日本語[にほんご] の 本[ほん] を 読んだ[よんだ] 方が[ほうが] いい です 。
\\	野菜も食べた方がいいよ。	
\\	野菜[やさい] も 食べた[たべた] 方が[ほうが] いい よ 。
\\	もう帰った方がいいですか。	
\\	もう 帰[かえ]った 方[ほう]がいいですか。
\\	和子はもっと英語を勉強した方がいいね。	
\\	和子[かずこ] はもっと 英語[えいご] を 勉強[べんきょう] した 方が[ほうが] いい ね 。
\\	日本語の本を読む方がいい。	
\\	日本語[にほんご] の 本[ほん] を 読む[よむ] 方が[ほうが] いい 。
\\	お前、この頃麻薬を遣ってるそうだな。そんなもの止めた方がいいぜ。	
\\	お 前[まえ] 、この 頃[ごろ] 麻薬[まやく] を 遣って[やって] るそうだな 。 そんなもの 止め[やめ] た 方が[ほうが] いい ぜ 。
\\	コーヒーを飲まない方がいい。	
\\	コーヒー を 飲[の] ま ない 方が[ほうが] いい 。
\\	石田さんの方が私より若いです。	
\\	石田[いしだ] さんの 方が[ほうが] 私[わたし] より 若い[わかい] です 。
\\	僕の方が上田さんよりよく食べる。	
\\	僕[ぼく]の 方[ほう]が 上田[うえだ]さんよりよく 食[た]べる。
\\	車で行く方がバスで行くより安いです。	
\\	車[くるま] で 行く[いく] 方[より] が バス で 行く[いく] より 安い[やすい] です 。
\\	このカメラの方があのカメラより好きです。	
\\	この カメラ の 方が[ほうが] あの カメラ より 好き[すき] です 。
\\	ジェフの方が私より速く走れる。	
\\	ジェフの 方[ほう]が 私[わたし]より 速[はや]く 走[はし]れる。
\\	僕が話す方が君が話すよりいいでしょう。	
\\	僕[ぼく] が 話す[はなす] 方が[ほうが] 君[きみ] が 話す[はなす] より いい でしょう 。
\\	勿論、安い方が高いより嬉しいです。	
\\	勿論[もちろん] 、 安い[やすい] 方が[ほうが] 高い[たかい] より 嬉し[うれし] いです 。
\\	子供は元気な方が静かなより安心だ。	
\\	子供[こども] は 元気[げんき] な 方が[ほうが] 静か[しずか] なより 安心[あんしん] だ 。
\\	僕は女である方が男であるより楽しいと思う。	
\\	僕は[ぼくは] 女[おんな] である 方が[ほうが] 男[おとこ] であるより 楽し[たのし] いと 思う[おもう] 。
\\	ートムとジョンとどちらが強いですか。 ートムの方が強いです。	ー
\\	ートム と ジョン とどちらが 強い[つよい] ですか 。 ートム の 方が[ほうが] 強い[つよい] です 。
\\	僕が話した方が君が話すよりいいでしょう。	
\\	僕[ぼく] が 話し[はなし] た 方が[ほうが] 君[きみ] が 話す[はなす] より いい でしょう 。
\\	僕が話した方が君が話すより良かった。	
\\	僕[ぼく] が 話し[はなし] た 方が[ほうが] 君[きみ] が 話す[はなす] より 良か[よか] った 。
\\	私はビールより酒の方をよく飲む。	
\\	私[わたし] は ビールより 酒[さけ] の 方を[ほうを] よく 飲む[のむ] 。
\\	学生は川田先生より木村先生の方によく質問に行く。	
\\	学生[がくせい] は 川田[かわだ] 先生[せんせい] より 木村[きむら] 先生[せんせい] の 方[ほう] によく 質問[しつもん] に 行く[いく] 。
\\	トムはとても強いです。彼はジョンより強いです。	
\\	トム はとても 強い[つよい] です 。 彼は[かれは] ジョン より 強い[つよい] です 。
\\	私は車が欲しい。	
\\	私[わたし] は 車[くるま] が 欲しい[ほしい] 。
\\	弟は僕の自転車を欲しがっています。	
\\	弟[おとうと] は 僕[ぼく] の 自転車[じてんしゃ] を 欲し[ほし] がっています 。
\\	僕は日本人の友達が欲しい。	
\\	僕は[ぼくは] 日本人[にほんじん] の 友達[ともだち] が 欲しい[ほしい] 。
\\	あなたは今何が欲しいですか。	
\\	あなたは 今[いま] 何[なに] が 欲しい[ほしい] ですか 。
\\	パムはステレオを欲しがっている。	
\\	パム は ステレオ を 欲し[ほし] がっている 。
\\	モーリスはいいステレオが欲しかった。	
\\	モーリス は いい ステレオ が 欲し[ほし] かった 。
\\	ジョイも欲しいと言っている。	
\\	ジョイ も 欲しい[ほしい] と 言って[いって] いる 。
\\	オスカーも欲しいそうだ。	
\\	オスカー も 欲しい[ほしい] そうだ 。
\\	パメラはイヤリングが欲しいんです。	
\\	パメラ は イヤリング が 欲しい[ほしい] んです 。
\\	フランシスは腕時計が欲しいらしい。	
\\	フランシス は 腕時計[うでどけい] が 欲しい[ほしい] らしい 。
\\	コニーは人形が欲しいようだ。	
\\	コニー は 人形[にんぎょう] が 欲しい[ほしい] ようだ 。
\\	私はあなたに英語を教えて欲しい。	
\\	私[わたし] はあなたに 英語[えいご] を 教[おし]え て 欲しい[ほしい] 。
\\	僕は子供達に僕と一緒に住んで欲しい。	
\\	僕は[ぼくは] 子供達[こどもたち] に 僕[ぼく] と 一緒に[いっしょに] 住ん[すん] で 欲しい[ほしい] 。
\\	あなたは誰に来て欲しいですか。	
\\	あなたは 誰[だれ] に 来て[きて] 欲しい[ほしい] ですか 。
\\	僕は吉田先生に来て欲しい。	
\\	僕は[ぼくは] 吉田[よしだ] 先生[せんせい] に 来て[きて] 欲しい[ほしい] 。
\\	アダムスさんはフランシスにこの仕事をして貰いたがっている。	
\\	アダムス さんは フランシス にこの 仕事[しごと] をして 貰い[もらい] たがっている 。
\\	私はあなたに英語を教えて貰いたい。	
\\	私[わたし] はあなたに 英語[えいご] を 教[おし]え て 貰い[もらい] たい 。
\\	私は吉田先生に来て頂きたい。	
\\	私[わたし] は 吉田[よしだ] 先生[せんせい] に 来て[きて] 頂き[いただき] たい 。
\\	クラスで大川さんが一番頭がいい。	
\\	クラス で 大川[おおかわ] さんが 一番[いちばん] 頭[あたま] が いい 。
\\	ーこの中でどの映画が一番面白いですか。 ーこの日本の映画でしょう。	ー
\\	ー この 中[なか] でどの 映画[えいが] が 一番[いちばん] 面白い[おもしろい] ですか 。 ー この 日本[にほん] の 映画[えいが] でしょう 。
\\	ー松本さんと池田さんと清水さんの中で誰が一番出来ますか。 ー池田さんです。	ー
\\	ー 松本[まつもと]さんと 池田[いけだ]さんと 清水[しみず]さんの 中[なか]で 誰[だれ]が 一番[いちばん] 出来[でき]ますか。 ー 池田[いけだ]さんです。
\\	日本で一番きれいな所はどこですか。	
\\	日本[にほん] で 一番[いちばん]きれい な 所[ところ] は どこ ですか 。
\\	田中さんは来週アメリカへ行く。	
\\	田中[たなか] さんは 来週[らいしゅう] アメリカ へ 行く[いく] 。
\\	私は東京から大阪までバスで行った。	
\\	私[わたし] は 東京[とうきょう] から 大阪[おおさか] まで バス で 行[い] った 。
\\	僕は毎朝八時に会社に行く。	
\\	僕は[ぼくは] 毎朝[まいあさ] 八[はち] 時に[じに] 会社[かいしゃ] に 行く[いく] 。
\\	今晩私のうちでパーティーをしますが来ませんか。	
\\	今晩[こんばん] 私[わたし] の うち で パーティー をしますが 来ま[きま] せんか 。
\\	あなたにもその知らせは行きましたか。	
\\	あなたにもその 知らせ[しらせ] は 行き[いき] ましたか 。
\\	その村にもバスは行っている。	
\\	その 村[むら] にも バス は 行って[いって] いる 。
\\	君のうちに息子が来ませんでしたか。	
\\	君[きみ] の うち に 息子[むすこ] が 来ま[きま] せんでしたか 。
\\	スミスさんは来月ロスアンジェルスに行くそうですね。	
\\	スミス さんは 来月[らいげつ] ロスアンジェルス に 行く[いく] そうですね 。
\\	これからは寒くなっていくよ。	
\\	これからは 寒く[さむく] なっていくよ 。
\\	これからは毎日本を一冊読んで行くつもりです。	
\\	これからは 毎日[まいにち] 本[ほん] を 一冊[いっさつ] 読んで[よんで] 行く[いく]つもりです 。
\\	これからは暖かくなっていきますよ。	
\\	これからは 暖か[あたたか] くなっていきますよ 。
\\	その頃から日本の経済は強くなっていった。	
\\	その 頃[ころ] から 日本[にほん] の 経済[けいざい] は 強く[つよく] なっていった 。
\\	分からないことをノートに書いていった。	
\\	分か[わか] らない こと を ノート に 書い[かい] ていった 。
\\	毎日会社にバスに乗って行く。	
\\	毎日[まいにち] 会社[かいしゃ] に バス に 乗って[のって] 行く[いく] 。
\\	あのレストランでコーヒーを飲んで行きましょう。	
\\	あのレストランでコーヒーを 飲[の]んで 行[い]きましょう。
\\	これからは暖かくなって来ますよ。	
\\	これからは 暖か[あたたか] くなって 来ま[きま] すよ 。
\\	その頃から日本の経済は強くなって来た。	
\\	その 頃[ころ] から 日本[にほん] の 経済[けいざい] は 強く[つよく] なって 来た[きた] 。
\\	この町は日本人がたくさんいます。	
\\	この 町[まち] は 日本人[にほんじん] が たくさん いま す 。
\\	リーはこの寮にいる。	
\\	リー はこの 寮[りょう] に いる 。
\\	この動物園にはパンダがいます。	
\\	この 動物園[どうぶつえん] には パンダ が いま す 。
\\	スチーブは今ロビンのアパートにいる。	
\\	スチーブ は 今[いま] ロビン の アパート に いる 。
\\	私には子供が三人いる。	
\\	私[わたし] には 子供[こども] が 三人[さんにん] いる 。
\\	このチームはいいクオーターバックがいる。	
\\	この チーム は いい クオーターバック が いる 。
\\	佐々木さんは酒を飲んでいます。	
\\	佐々木[ささき] さんは 酒[さけ] を 飲ん[のん] でいます 。
\\	和江は新聞を読んでいる。	
\\	和江[かずえ] は 新聞[しんぶん] を 読んで[よんで] いる 。
\\	この林檎は腐っている。	
\\	この 林檎[りんご]は 腐[くさ]っている。
\\	木が倒れている。	
\\	木[き] が 倒れ[たおれ] ている 。
\\	私は鈴木さんを知っています。	
\\	私[わたし] は 鈴木[すずき] さんを 知って[しって] います 。
\\	私は毎日四マイル走っている。	
\\	私[わたし] は 毎日[まいにち] 四[よん] マイル 走って[はしって] いる 。
\\	次郎はアメリカに行っている。	
\\	次郎[じろう] は アメリカ に 行って[いって] いる 。
\\	ベックさんはもう家に帰っています。	
\\	ベック さんはもう 家[いえ] に 帰って[きって] います 。
\\	私は東京に住んでいる。	
\\	私[わたし] は 東京[とうきょう] に 住ん[すん] でいる 。
\\	ウエストさんは日本語は易しいと思っている。	
\\	ウエスト さんは 日本語[にほんご] は 易し[やすし] いと 思って[おもって] いる 。
\\	君達は英和辞典が要ります。	
\\	君達[きみたち] は 英和辞典[えいわじてん] が 要り[いり] ます 。
\\	私は今お金が要る。	
\\	私[わたし] は 今[いま] お 金[かね] が 要る[いる] 。
\\	何か道具が要りますか。	
\\	何か[なにか] 道具[どうぐ] が 要り[いり] ますか 。
\\	この車はガソリンがたくさん要る。	
\\	この 車[くるま]はガソリンがたくさん 要[い]る。
\\	この子にはいい家庭教師が要る。	
\\	この 子[こ] には いい 家庭教師[かていきょうし] が 要る[いる] 。
\\	このプロジェクトには人と金が要る。	
\\	この プロジェクト には 人[ひと] と 金[かね] が 要る[いる] 。
\\	土田は幸子が自分を愛していることを知らなかった。	
\\	土田[つちだ] は 幸子[さちこ] が 自分[じぶん] を 愛[あい] している こと を 知ら[しら] なかった 。
\\	ゆかりは歩きながら考えていた。	
\\	ゆかりは 歩き[あるき] ながら 考え[かんがえ] ていた 。
\\	健一は自分が本当に好きなんだろうか。	
\\	健一[けんいち]は 自分[じぶん]が 本当[ほんとう]に 好[す]きなんだろうか。
\\	中川は自分が京大に入れると思っていなかった。	
\\	中川[なかがわ] は 自分[じぶん] が 京大[きょうだい] に 入れ[いれ] ると 思って[おもって] いなかった 。
\\	一郎は冬子が自分に会いに来た時うちにいなかった。	
\\	一郎[いちろう]は 冬子[ふゆこ]が 自分[じぶん]に 会[あ]いに 来[き]た 時[とき]うちにいなかった。
\\	道子は一男に自分の車で行かせた。	
\\	道子[みちこ] は 一男[かずお] に 自分[じぶん] の 車[くるま] で 行か[いか] せた 。
\\	ジョンはメアリーのことを思っていた。彼女は自分と結婚して呉れるんだろうか。自分を捨てて、ボブと結婚するんだろうか。	
\\	ジョン は メアリー の こと を 思って[おもって] いた 。 彼女[かのじょ] は 自分[じぶん] と 結婚[けっこん] して 呉れ[くれ] るんだろうか 。 自分[じぶん] を 捨て[すて] て 、 ボブ と 結婚[けっこん] するんだろうか 。
\\	道子は一男に自分の車で行って貰った	
\\	道子[みちこ] は 一男[かずお] に 自分[じぶん] の 車[くるま] で 行って[いって] 貰った[もらった]
\\	道子は一男に自分の車で行かれた。	
\\	道子[みちこ] は 一男[かずお] に 自分[じぶん] の 車[くるま] で 行か[いか] れた 。
\\	友子は自分が死ぬ前に子供が大学に入った。	
\\	友子[ともこ] は 自分[じぶん] が 死ぬ[しぬ] 前[まえ] に 子供[こども] が 大学[だいがく] に 入[はい] った。
\\	日本人は自分の国の文化をユニークだと思っている。	
\\	日本人[にほんじん] は 自分[じぶん] の 国[くに] の 文化[ぶんか] を ユニーク だと 思って[おもって] いる 。
\\	メアリーは自分で何でもします。	
\\	メアリー は 自分[じぶん] で 何で[なんで] もします 。
\\	一男は自分を励ました。	
\\	一男[かずお] は 自分[じぶん] を 励まし[はげまし] た 。
\\	自分を知ることが一番難しい。	
\\	自分[じぶん] を 知る[しる] こと が 一番[いちばん] 難しい[むずかしい] 。
\\	マイクはいつも自分の寮の部屋で勉強している。	
\\	マイク は いつ も 自分[じぶん] の 寮[りょう] の 部屋[へや] で 勉強[べんきょう] している 。
\\	小林は自分からしたいと言った。	
\\	小林[こばやし] は 自分[じぶん] からしたいと 言った[いった] 。
\\	先生はご自分の家で私に会って下さった。	
\\	先生[せんせい] はご 自分[じぶん] の 家[いえ] で 私[わたし] に 会って[あって] 下さ[くださ] った 。
\\	私は電車かバスで行きます。	
\\	私[わたし] は 電車[でんしゃ] か バス で 行き[いき] ます 。
\\	トムが行くかメアリーが行くかどちらかだ。	
\\	トム が 行く[いく] か メアリー が 行く[いく] かどちらかだ 。
\\	僕は毎朝ジュースかミルクを飲む。	
\\	僕は[ぼくは] 毎朝[まいあさ] ジュース か ミルク を 飲む[のむ] 。
\\	それはボブかマークがします。	
\\	それは ボブ か マーク がします 。
\\	肉が高かったか一郎が肉が嫌いだったかどちらかだ。	
\\	肉[にく] が 高か[たかか] ったか 一郎[いちろう] が 肉[にく] が 嫌い[きらい] だったかどちらかだ 。
\\	手紙を書くか電話を掛けるかどちらして下さい。	
\\	手紙[てがみ] を 書く[かく] か 電話[でんわ] を 掛け[かけ] るかどちらして 下[くだ]さい 。
\\	食べるか話すかどちらかにしなさい。	
\\	食[た]べるか 話す[はなす] かどちらかにしなさい 。
\\	肉が高かったか、それとも一郎が肉が嫌いだったかどちらかだ。	
\\	肉[にく] が 高か[たかか] ったか 、 それとも 一郎[いちろう] が 肉[にく] が 嫌い[きらい] だったかどちらかだ 。
\\	私と一緒に来ますか。それともここにいますか。	
\\	私[わたし] と 一緒に[いっしょに] 来ま[きま] すか 。 それともここに いま すか 。
\\	よし子は大学へ行きますか。	
\\	よし 子[こ] は 大学[だいがく] へ 行き[いき] ますか 。
\\	僕はテリーにナンシーが日本へ行くかと聞いた。	
\\	僕は[ぼくは] テリー に ナンシー が 日本[にほん] へ 行く[いく] かと 聞い[きい] た 。
\\	あなたは学生ですか。	
\\	あなたは 学生[がくせい] ですか 。
\\	これは何ですか。	
\\	これは 何で[なんで] すか 。
\\	友達は漢字が難しいかと聞いた。	
\\	友達[ともだち] は 漢字[かんじ] が 難しい[むずかしい] かと 聞い[きい] た 。
\\	私はジャンに誰が来たかと尋ねた。	
\\	私[わたし] は ジャン に 誰が[だれが] 来た[きた] かと 尋ね[たずね] た 。
\\	僕は山崎先生が昨日何を言ったか忘れてしまった。	
\\	僕は[ぼくは] 山崎[やまざき] 先生[せんせい] が 昨日[きのう] 何を[なにを] 言った[いった] か 忘れ[わすれ] てしまった 。
\\	僕は健二にお金を貸したかどうか思い出せない。	
\\	僕は[ぼくは] 健二[けんじ] にお 金[かね] を 貸し[かし] たかどうか 思い出せ[おもいだせ] ない 。
\\	よし子は大学へ行きますか。それとも行きませんか。	
\\	よし 子[こ] は 大学[だいがく] へ 行き[いき] ますか 。 それとも 行き[いき] ませんか 。
\\	鈴木さんが大学に入ったか知らない。	
\\	鈴木[すずき] さんが 大学[だいがく] に 入[はい] ったか 知ら[しら] ない 。
\\	小川さんが結婚しているかどうか知っていますか。	
\\	小川[おがわ] さんが 結婚[けっこん] しているかどうか 知って[しって] いますか 。
\\	フォークナーの小説が面白いかどうか知らない。	
\\	フォークナー の 小説[しょうせつ] が 面白い[おもしろい] かどうか 知ら[しら] ない 。
\\	山口先生がお元気かどうか存じません。	
\\	山口[やまぐち] 先生[せんせい] がお 元気[げんき] かどうか 存じ[ぞんじ] ません 。
\\	レストランで何を食べたか覚えていますか。	
\\	レストラン で 何を[なにを] 食べた[たべた] か 覚え[おぼえ] ていますか 。
\\	その時誰と一緒にいたか覚えていますか。	
\\	その 時[とき] 誰[だれ]と 一緒[いっしょ]にいたか 覚[おぼ]えていますか。
\\	仕事を止めるかどうかが問題だった。	
\\	仕事[しごと] を 止める[やめる] かどうかが 問題[もんだい] だった 。
\\	大学院に行くかどうかを今考えています。	
\\	大学院[だいがくいん] に 行く[いく] かどうかを 今[いま] 考え[かんがえ] ています 。
\\	日本語は面白いかい。	
\\	日本語[にほんご] は 面白い[おもしろい] かい 。
\\	上野さんはアメリカへ行くのかい。	
\\	上野[うえの] さんは アメリカ へ 行く[いく] のかい 。
\\	日本語を勉強するかい。	
\\	日本語[にほんご] を 勉強[べんきょう] するかい 。
\\	このアパートは静かかい。	
\\	この アパート は 静か[しずか] かい 。
\\	君達は学生かい。	
\\	君達[きみたち] は 学生[がくせい] かい 。
\\	その本は難しいのかい。	
\\	その 本[ほん] は 難しい[むずかしい] のかい 。
\\	あの人は先生なのかい。	
\\	あの 人[ひと] は 先生[せんせい] なのかい 。
\\	日本語は面白い?	
\\	日本語[にほんご] は 面白い[おもしろい] 
\\	上野さんはアメリカへ行くの?	
\\	上野[うえの] さんは アメリカ へ 行く[いく] の 
\\	あの人は先生なの?	
\\	あの 人[ひと] は 先生[せんせい] なの 
\\	午後雨が降るかもしれない。	
\\	午後[ごご] 雨[あめ] が 降る[ふる] かもしれない 。
\\	あの先生の授業は詰らないかもしれない。	
\\	あの 先生[せんせい] の 授業[じゅぎょう] は 詰ら[つまら] ないかもしれない 。
\\	京都の桜はまだきれいかもしれません。	
\\	京都[きょうと] の 桜[さくら] はまだきれい かもしれません 。
\\	今年の冬は大変寒くなるかもしれません。	
\\	今年[ことし] の 冬[ふゆ] は 大変[たいへん] 寒く[さむく] なるかもしれません 。
\\	あの人は今日のパーティーのことを忘れたかもしれません。	
\\	あの 人[ひと] は 今日[きょう] の パーティー の こと を 忘れ[わすれ] たかもしれません 。
\\	この映画はあなたには詰らないかもしれません。	
\\	この 映画[えいが] はあなたには 詰ら[つまら] ないかもしれません 。
\\	中国語の文法は日本語の文法より簡単かもしれない。	
\\	中国語[ちゅうごくご] の 文法[ぶんぽう] は 日本語[にほんご] の 文法[ぶんぽう] より 簡単[かんたん] かもしれない 。
\\	あれは鈴木先生かもしれないよ。	
\\	あれは 鈴木[すずき] 先生[せんせい] かもしれないよ 。
\\	関東地方、明日は小雨が一日中降るでしょう。	
\\	関東地方[かんとうちほう] 、 明日[あす] は 小雨[こさめ] が 一日中[いちにちじゅう] 降る[ふる] でしょう 。
\\	あっ!このケーキ美味しそうですね。	
\\	あっ 
\\	この ケーキ 美味し[おいし] そうですね 。
\\	パーティーは八時から始まります。	
\\	パーティー は 八[はち] 時[じ] から 始まり[はじまり] ます 。
\\	今日の授業は一時から三時までです。	
\\	今日[きょう] の 授業[じゅぎょう] は 一時[いちじ] から 三時[さんじ] までです 。
\\	このバスはニューヨークから来た。	
\\	この バス は ニューヨーク から 来た[きた] 。
\\	ここから富士山が見えるよ。	
\\	ここから 富士山[ふじさん] が 見え[みえ] るよ 。
\\	そのタイプライターは誰から借りたんですか。	
\\	その タイプライター は 誰か[だれか] ら 借り[かり] たんですか 。
\\	酒は米から作る。	
\\	酒[さけ] は 米[こめ] から 作る[つくる] 。
\\	詰らないことから喧嘩になった。	
\\	詰ら[つまら] ない こと から 喧嘩[けんか] になった 。
\\	雪子は晩ご飯を食べてから映画に行った。	
\\	雪子[ゆきこ] は 晩[ばん] ご 飯[はん] を 食べて[たべて] から 映画[えいが] に 行[い] った 。
\\	私は友達に電話してからうちを出た。	
\\	私[わたし] は 友達[ともだち] に 電話[でんわ] してから うち を 出た[でた] 。
\\	ジョーンズさんはいつもシャワーを浴びてから寝ます。	
\\	ジョーンズ さんは いつ も シャワー を 浴び[あび] てから 寝ま[ねま] す 。
\\	私達がこの家を買ってからもう十年になる。	
\\	私達[わたしたち] がこの 家[いえ] を 買って[かって] からもう 十年[じゅうねん] になる 。
\\	二年前に交通事故を起こしてから、ミラーさんは車に乗らないようにしています。	
\\	二年前[にねんまえ] に 交通事故[こうつうじこ] を 起こ[おこ] してから 、 ミラー さんは 車[くるま] に 乗ら[のら] ないようにしています 。
\\	ジョギングをしてからシャワーを浴びた。	
\\	ジョギング をしてから シャワー を 浴び[あび] た 。
\\	ジョギングをしたからシャワーを浴びた。	
\\	ジョギング をしたから シャワー を 浴び[あび] た 。
\\	勉強が終わってからテニスをしましょう。	
\\	勉強[べんきょう] が 終わ[おわ] ってから テニス をしましょう 。
\\	勉強が終わってから遊びなさい。	
\\	勉強[べんきょう] が 終わ[おわ] ってから 遊び[あそび] なさい 。
\\	来年日本へ行くから日本語を勉強しています。	
\\	来年[らいねん] 日本[にほん] へ 行く[いく] から 日本語[にほんご] を 勉強[べんきょう] しています 。
\\	ーどうして日本語を勉強しているんだ。 ー来年日本へ行くからだ。	ー
\\	ー どうして 日本語[にほんご] を 勉強[べんきょう] しているんだ 。 ー 来年[らいねん] 日本[にほん] へ 行く[いく] からだ 。
\\	春子は十七だからまだお酒を飲めない。	
\\	春子[はるこ] は 十七[じゅうしち] だからまだお 酒[さけ] を 飲め[のめ] ない 。
\\	今日は忙しいですから明日来て下さい。	
\\	今日[きょう]は 忙[いそが]しいですから 明日[あした] 来[き]て 下[くだ]さい。
\\	ーどうして昨日学校を休んだんですか。 ー頭が痛かったからです。	ー
\\	ー どうして 昨日[きのう] 学校[がっこう] を 休ん[やすん] だんですか 。 ー 頭[あたま] が 痛か[いたか] ったからです 。
\\	松本さんは来るかしら。	
\\	松本[まつもと] さんは 来る[くる] かしら 。
\\	あの先生の授業は面白いかしら。	
\\	あの 先生[せんせい] の 授業[じゅぎょう] は 面白い[おもしろい] かしら 。
\\	幸子さんは何が好きかしら。	
\\	幸子[さちこ] さんは 何[なに] が 好き[すき] かしら 。
\\	そこは静かですかしら。	
\\	そこは 静か[しずか] ですかしら 。
\\	私は松本さんは来るかしらと思った。	
\\	私[わたし] は 松本[まつもと] さんは 来る[くる] かしらと 思った[おもった] 。
\\	山本さんは小川さんは先生かしらと思った。	
\\	山本[やまもと] さんは 小川[おがわ] さんは 先生[せんせい] かしらと 思った[おもった] 。
\\	ケーキの作り方を教えて下さい。	
\\	ケーキ の 作り方[つくりかた] を 教[おし]え て 下[くだ]さい 。
\\	この漢字の書き方が分かりません。	
\\	この 漢字[かんじ] の 書き方[かきかた] が 分か[わか] りません 。
\\	あの人の歩き方は面白いですね。	
\\	あの 人[ひと] の 歩き方[あるきかた] は 面白い[おもしろい] ですね 。
\\	日本語の勉強の仕方を教えて下さい。	
\\	日本語[にほんご] の 勉強[べんきょう] の 仕方[しかた] を 教[おし]え て 下[くだ]さい 。
\\	この薬は飲み方が大事です。必ず食前に飲んで下さい。	
\\	この 薬[くすり] は 飲み[のみ] 方が[ほうが] 大事[だいじ] です 。 必ず[かならず] 食前[しょくぜん] に 飲ん[のん] で 下[くだ]さい 。
\\	スープは飲み方が大事です。	
\\	スープ は 飲み[のみ] 方が[ほうが] 大事[だいじ] です 。
\\	日本語の勉強の方法を教えて下さい。	
\\	日本語[にほんご] の 勉強[べんきょう] の 方法[ほうほう] を 教[おし]え て 下[くだ]さい 。
\\	先生の代りに私が教えた。	
\\	先生[せんせい] の 代り[かわり] に 私[わたし] が 教[おし]え た 。
\\	土曜日に仕事をする代りに月曜日は休みます。	
\\	土曜日[どようび] に 仕事[しごと] をする 代り[かわり] に 月曜日[げつようび] は 休み[やすみ] ます 。
\\	ビールの代りに酒を買いました。	
\\	ビール の 代り[かわり] に 酒[さけ] を 買い[かい] ました 。
\\	私の代りに父が行ってもいいですか。	
\\	私[わたし] の 代り[かわり] に 父[ちち] が 行って[いって] も いい ですか 。
\\	今日は夜遅くまで踊る代りに明日は一日中勉強します。	
\\	今日は[きょうは] 夜遅く[よるおそく] まで 踊る[おどる] 代り[かわり] に 明日[あした] は 一日中[いちにちじゅう] 勉強[べんきょう] します 。
\\	手伝って上げる代りに飲ませて下さいよ。	
\\	手伝って[てつだって] 上げ[あげ] る 代り[かわり] に 飲[の] ま せて 下[くだ]さい よ 。
\\	僕のアパートは不便な代りに家賃が安い。	
\\	僕[ぼく] の アパート は 不便[ふべん] な 代り[かわり] に 家賃[やちん] が 安い[やすい] 。
\\	英語を教えて上げた代りに日本語を教えて貰った。	
\\	英語[えいご] を 教[おし]え て 上げ[あげ] た 代り[かわり] に 日本語[にほんご] を 教[おし]え て 貰った[もらった] 。
\\	その車は安かった代りによく故障した。	
\\	その 車[くるま] は 安か[やすか] った 代り[かわり] に よく 故障[こしょう] した 。
\\	僕は甘い物をたくさん食べるけれども歯は強い。	
\\	僕は[ぼくは] 甘い[あまい] 物[もの] を たくさん 食[た]べるけれども 歯[は] は 強い[つよい] 。
\\	僕は甘い物をたくさん食べる代りに歯をよく磨く。	
\\	僕は[ぼくは] 甘い[あまい] 物[もの] を たくさん 食[た]べる 代り[かわり] に 歯を[はを] よく 磨く[みがく] 。
\\	私はいわなかったけれどもトムは知っていました。	
\\	私[わたし] はいわなかったけれども トム は 知って[しって] いました 。
\\	この本は高いけれどもいい本ですよ。	
\\	この 本[ほん] は 高い[たかい] けれども いい 本[ほん] ですよ 。
\\	僕はドイツ語があまり好きじゃないけれども勉強しなければならない。	
\\	僕は[ぼくは] ドイツ 語[ご] があまり 好き[すき] じゃないけれども 勉強[べんきょう] しなければならない 。
\\	大野さんは九十歳だけれどもとても元気だ。	
\\	大野[おおの] さんは 九十歳[きゅうじゅっさい] だけれどもとても 元気[げんき] だ 。
\\	この本は高いですけれどもいい本ですよ。	
\\	この 本[ほん] は 高い[たかい] ですけれども いい 本[ほん] ですよ 。
\\	私は鶯の声がよく聞こえる。	
\\	私[わたし] は 鶯[うぐいす] の 声[こえ] が よく 聞こ[きこ] える 。
\\	その音は小さすぎて聞こえない。	
\\	その 音[おと] は 小さ[ちいさ] すぎて 聞こ[きこ] えない 。
\\	大山さんの声は大きいので隣の部屋の人にもよく聞こえる。	
\\	大山[おおやま] さんの 声[こえ] は 大き[おおき] いので 隣[となり] の 部屋[へや] の 人[ひと] にも よく 聞こ[きこ] える 。
\\	私にはお寺の鐘の音が聞こえたが、弟には聞こえなかった。	
\\	私[わたし] にはお 寺[てら] の 鐘[かね] の 音[おと] が 聞こ[きこ] えたが 、 弟[おとうと] には 聞こ[きこ] えなかった 。
\\	テーラーさんの作った文は変に聞こえる。	
\\	テーラー さんの 作った[つくった] 文[ぶん] は 変に[へんに] 聞こ[きこ] える 。
\\	僕は耳が聞こえない。	
\\	僕は[ぼくは] 耳[みみ] が 聞こ[きこ] えない 。
\\	こんなに後ろに座るとよく聞こえないよ。	
\\	こんなに 後ろ[うしろ] に 座る[すわる] と よく 聞こ[きこ] えないよ 。
\\	いいステレオを買ったからレコードが聞ける。	
\\	いい ステレオ を 買[か] った から レコード が 聞け[きけ] る 。
\\	音楽が煩くて話が聞こえない。	
\\	音楽[おんがく] が 煩[うるさ] くて 話[はなし] が 聞こ[きこ] えない 。
\\	音楽が煩くて話が聞けない。	
\\	音楽[おんがく]が 煩[うるさ]くて 話[はなし]が 聞[き]けない。
\\	私はチーズが嫌いです。	
\\	私[わたし] は チーズ が 嫌い[きらい] です 。
\\	僕は冬が嫌いだ。	
\\	僕は[ぼくは] 冬[ふゆ] が 嫌い[きらい] だ 。
\\	ホワイトさんはフットボールが大嫌いです。	
\\	ホワイト さんは フットボール が 大嫌い[だいきらい] です 。
\\	私がチーズが嫌いなことは皆知っている。	
\\	私[わたし] が チーズ が 嫌い[きらい] な こと は 皆[みんな] 知って[しって] いる 。
\\	僕が嫌いな季節は冬です。	
\\	僕[ぼく] が 嫌い[きらい] な 季節[きせつ] は 冬[ふゆ] です 。
\\	いいことを教えて上げましょう。	
\\	いい こと を 教[おし]え て 上げ[あげ] ましょう 。
\\	論文に書いたことを話して下さい。	
\\	論文[ろんぶん] に 書い[かい] た こと を 話し[はなし] て 下[くだ]さい 。
\\	ブラウンさんは日本の大学のことをよく知っている。	
\\	ブラウン さんは 日本[にほん] の 大学[だいがく] の こと を よく 知って[しって] いる 。
\\	大事なことはもう全部話しました。	
\\	大事[だいじ] なことはもう 全部[ぜんぶ] 話し[はなし] ました 。
\\	先生が言ったことを覚えていますか。	
\\	先生[せんせい] が 言った[いった] こと を 覚え[おぼえ] ていますか 。
\\	試験のことは忘れなさい。	
\\	試験[しけん] の こと は 忘れ[わすれ] なさい 。
\\	黒い物が見えましたか。	
\\	黒い[くろい] 物[もの] が 見え[みえ] ましたか 。
\\	面白いことを話して下さい。	
\\	面白[おもしろ]いことを 話[はな]して 下[くだ]さい。
\\	小説を書くことは難しいです。	
\\	小説[しょうせつ] を 書く[かく] こと は 難しい[むずかしい] です 。
\\	若い時にいい友達を作ることはとても大事だ。	
\\	若い[わかい] 時に[ときに] いい 友達[ともだち] を 作る[つくる] ことはとても 大事[だいじ] だ 。
\\	大学四年の時フランスに留学することを考えています。	
\\	大学[だいがく] 四年[よねん] の 時[とき] フランス に 留学[りゅうがく] する こと を 考え[かんがえ] ています 。
\\	日本の文化が面白いことは分かるが、ユニークだとは思わない。	
\\	日本[にほん] の 文化[ぶんか] が 面白い[おもしろい] こと は 分か[わか] るが 、 ユニーク だとは 思わ[おもわ] ない 。
\\	スイスがきれいなことは写真で知っています。	
\\	スイス が きれい な こと は 写真[しゃしん] で 知って[しって] います 。
\\	あの人がいい人であることは確かです。	
\\	あの 人[ひと] が いい 人[ひと] である こと は 確か[たしか] です 。
\\	僕は静江が泳ぐのを見ていた。	
\\	僕は[ぼくは] 静江[しずえ] が 泳ぐ[およぐ] のを 見て[みて] いた 。
\\	お母さんがこんなに心配しているのが分からないの?	
\\	お 母さん[かあさん] がこんなに 心配[しんぱい] しているのが 分か[わか] らないの 
\\	ジェーンはビルが洗濯するのを手伝った。	
\\	ジェーン は ビル が 洗濯[せんたく] するのを 手伝った[てつだった] 。
\\	この町ではいい音楽を聞くことが出来る。	
\\	この 町[まち] ではいい 音楽[おんがく] を 聞く[きく] こと が 出来る[できる] 。
\\	見ることは信じることだ。	
\\	見る[みる] こと は 信じ[しんじ] る こと だ 。
\\	私はヨーロッパへ行ったことがあります。	
\\	私[わたし] は ヨーロッパ へ 行[い] った こと があります 。
\\	レタスがとても高かったことがある。	
\\	レタス がとても 高か[たかか] った こと がある 。
\\	私は中学校で英語を教えたことがあります。	
\\	私[わたし] は 中学校[ちゅうがっこう] で 英語[えいご] を 教[おし]え た こと があります 。
\\	小川さんはまだゴルフをしたことがない。	
\\	小川[おがわ] さんはまだ ゴルフ をした こと がない 。
\\	私は日本の小説をよく読んだことがある。	
\\	私[わたし] は 日本[にほん] の 小説[しょうせつ] をよく 読んだ[よんだ] こと がある 。
\\	スーザンは一時ジャズがとても好きだったことがある。	
\\	スーザン は 一時[いちじ] ジャズ がとても 好き[すき] だった こと がある 。
\\	私はプロ野球選手だったことがあります。	
\\	私[わたし] は プロ 野球選手[やきゅうせんしゅ] だった こと があります 。
\\	私はロシア語を勉強したことがある。	
\\	私[わたし] は ロシア 語[ご] を 勉強[べんきょう] した こと がある 。
\\	私は五年前に日本へ行ったことがある。	
\\	私[わたし] は 五年[ごねん] 前[まえ] に 日本[にほん] へ 行[い] った こと がある 。
\\	私は朝風呂に入ることがあります。	
\\	私[わたし] は 朝[あさ] 風呂[ふろ] に 入る[はいる] こと があります 。
\\	たかしは朝ご飯を食べずに学校へ行くことがある。	
\\	たかしは 朝[あさ] ご 飯[はん] を 食べ[たべ] ずに 学校[がっこう] へ 行く[いく] こと がある 。
\\	この店のミルクはたまに古いことがある。	
\\	この 店[みせ]のミルクはたまに 古[ふる]いことがある。
\\	アメリカで日本への土産を買うとそれが日本製であることがよくある。	
\\	アメリカ で 日本[にほん] への 土産[みやげ] を 買う[かう] とそれが 日本製[にほんせい] である こと がよくある 。
\\	最近の所謂自然食品は本当の自然食品じゃないことがある。	
\\	最近[さいきん] の 所謂[いわゆる] 自然[しぜん] 食品[しょくひん] は 本当[ほんとう] の 自然[しぜん] 食品[しょくひん] じゃない こと がある。
\\	田口さんは中国語を話すことが出来ます。	
\\	田口[たぐち] さんは 中国語[ちゅうごくご] を 話す[はなす] こと が 出来ま[できま] す 。
\\	新幹線に乗れば大阪まで三時間で行くことが出来る。	
\\	新幹線[しんかんせん] に 乗れ[のれ] ば 大阪[おおさか] まで 三時[さんじ] 間[かん] で 行く[いく] こと が 出来る[できる] 。
\\	小田は六つの時バッハを弾くことが出来た。	
\\	小田[おだ] は 六つ[むっつ] の 時[とき] バッハ を 弾く[ひく] こと が 出来た[できた] 。
\\	ジョンソンさんは日本語で手紙を書くことが出来る。	
\\	ジョンソン さんは 日本語[にほんご] で 手紙[てがみ] を 書く[かく] こと が 出来る[できる] 。
\\	ナンシーはピアノが出来る。	
\\	ナンシー は ピアノ が 出来る[できる] 。
\\	岡本さんはロシア語を話すことが出来る。	
\\	岡本[おかもと] さんは ロシア 語[ご] を 話す[はなす] こと が 出来る[できる] 。
\\	スミスさんは日本語の新聞を読むことが出来る。	
\\	スミス さんは 日本語[にほんご] の 新聞[しんぶん] を 読む[よむ] こと が 出来る[できる] 。
\\	新幹線に乗れば大阪まで三時間で行ける。	
\\	新幹線[しんかんせん] に 乗れ[のれ] ば 大阪[おおさか] まで 三時[さんじ] 間[かん] で 行け[いけ] る 。
\\	小田は六つの時バッハが弾けた。	
\\	小田[おだ]は 六[むっ]つの 時[とき]バッハが 弾[ひ]けた。
\\	ジョンソンさんは日本語で手紙が書ける。	
\\	ジョンソン さんは 日本語[にほんご] で 手紙[てがみ] が 書け[かけ] る 。
\\	私は来年大阪に転勤することになりました。	
\\	私[わたし] は 来年[らいねん] 大阪[おおさか] に 転勤[てんきん] する こと になりました 。
\\	日本では車は道の左側を走ることになっている。	
\\	日本[にほん] では 車[くるま] は 道[みち] の 左側[ひだりがわ] を 走る[はしる] こと になっている 。
\\	僕は来月から会社に勤めることになりました。	
\\	僕は[ぼくは] 来月[らいげつ] から 会社[かいしゃ] に 勤め[つとめ] る こと になりました 。
\\	多分ジャンセンさんは日本で英語を教えることになるでしょう。	
\\	多分[たぶん] ジャンセン さんは 日本[にほん] で 英語[えいご] を 教[おし]え る こと になるでしょう 。
\\	来年六月に結婚することになりました。	
\\	来年[らいねん] 六月[ろくがつ] に 結婚[けっこん] する こと になりました 。
\\	スミスさんは日本で英語を教えることになっている。	
\\	スミス さんは 日本[にほん] で 英語[えいご] を 教[おし]え る こと になっている 。
\\	今日山田先生に会うことになっています。	
\\	今日[きょう] 山田[やまだ] 先生[せんせい] に 会う[あう] こと になっています 。
\\	私は会社を止めることにしました。	
\\	私[わたし] は 会社[かいしゃ] を 止める[やめる] こと にしました 。
\\	僕は毎日三十分ぐらい運動をすることにしている。	
\\	僕は[ぼくは] 毎日[まいにち] 三十分[さんじゅうっぷん] ぐらい 運動[うんどう] をする こと にしている 。
\\	今年の夏は北海道を旅行することにしました。	
\\	今年[ことし] の 夏[なつ] は 北海道[ほっかいどう] を 旅行[りょこう] する こと にしました 。
\\	京都までバスで行くことにしましょう。	
\\	京都[きょうと] まで バス で 行く[いく] こと にしましょう 。
\\	毎日漢字を十覚えることにしました。	
\\	毎日[まいにち] 漢字[かんじ] を 十[じゅう] 覚え[おぼえ] る こと にしました 。
\\	僕は肉をあまり食べないことにしている。	
\\	僕は[ぼくは] 肉[にく] をあまり 食べ[たべ] ない こと にしている 。
\\	ピクニックに行こうと思ったんですが、行かないことにしました。	
\\	ピクニック に 行こ[いこ] うと 思った[おもった] んですが 、 行か[いか] ない こと にしました 。
\\	ー何にしますか。 ーハンバーガーにします。	ー
\\	ー 何[なん] にしますか 。 ーハンバーガー にします 。
\\	僕は大阪に転勤することになりました。	
\\	僕は[ぼくは] 大阪[おおさか] に 転勤[てんきん] する こと になりました 。
\\	僕はたばこを止めることにしました。	
\\	僕は[ぼくは] たばこを 止める[やめる] こと にしました 。
\\	僕は会社を止めることにします。	
\\	僕は[ぼくは] 会社[かいしゃ] を 止める[やめる] こと にします 。
\\	私はテニスをすることはするが上手じゃありません。	
\\	私[わたし] は テニス をする こと はするが 上手[じょうず] じゃありません 。
\\	ジョーンズさんは日本語を話すことは話しますが、簡単なことしか言えません。	
\\	ジョーンズ さんは 日本語[にほんご] を 話す[はなす] こと は 話し[はなし] ますが 、 簡単[かんたん] な こと しか 言え[いえ] ません 。
\\	ボストンシンフォニーの切符は買えたことは買えましたが、大変悪い席でした。	
\\	ボストンシンフォニー の 切符[きっぷ] は 買え[かえ] た こと は 買え[かえ] ましたが 、 大変[たいへん] 悪い[わるい] 席[せき] でした 。
\\	この店は安いことは安いですが、物がよくありません。	
\\	この 店[みせ] は 安い[やすい] こと は 安い[やすい] ですが 、 物[もの] が よく ありません 。
\\	今日の試験は難しかったことは難しかったがよく出来た。	
\\	今日[きょう] の 試験[しけん] は 難しか[むずかしか] った こと は 難しか[むずかしか] ったが よく 出来た[できた] 。
\\	僕のアパートは駅に近くて便利なことは便利ですが、家賃がとても高いです。	
\\	僕[ぼく] の アパート は 駅[えき] に 近く[ちかく] て 便利[べんり] な こと は 便利[べんり] ですが 、 家賃[やちん] がとても 高い[たかい] です 。
\\	その女の子が好きだったことは好きでしたが結婚はしなかったんです。	
\\	その 女の子[おんなのこ] が 好き[すき] だった こと は 好き[すき] でしたが 結婚[けっこん] はしなかったんです 。
\\	あの人はいい人はいい人だったけれど頑固だったね。	
\\	あの 人[ひと] は いい 人[ひと] は いい 人[ひと] だったけれど 頑固[がんこ] だったね 。
\\	ボストンシンフォニーの切符は買えることは買えましたが大変悪い席でした。	
\\	ボストンシンフォニー の 切符[きっぷ] は 買え[かえ] る こと は 買え[かえ] ましたが 大変[たいへん] 悪い[わるい] 席[せき] でした 。
\\	今日の試験は難しいことは難しかったがよく出来た。	
\\	今日[きょう] の 試験[しけん] は 難しい[むずかしい] こと は 難しか[むずかしか] ったが よく 出来た[できた] 。
\\	日本語で書いて下さい。	
\\	日本語[にほんご] で 書い[かい] て 下[くだ]さい 。
\\	英語を使わないで下さい。	
\\	英語[えいご] を 使わ[つかわ] ないで 下[くだ]さい 。
\\	この言葉の意味を教えて下さい。	
\\	この 言葉[ことば] の 意味[いみ] を 教[おし]え て 下[くだ]さい 。
\\	夜遅く電話しないで下さい。	
\\	夜遅く[よるおそく] 電話[でんわ] しないで 下[くだ]さい 。
\\	どうぞ教えて下さい。	
\\	どうぞ 教[おし]え て 下[くだ]さい 。
\\	速く来て。	
\\	速[はや]く 来[き]て。
\\	まだ帰らないで。	
\\	まだ 帰ら[かえら] ないで 。
\\	明日八時に来て下さいませんか。	
\\	明日[あした] 八時[はちじ] に 来て[きて] 下[くだ]さい ませんか 。
\\	僕と一緒に来てくれ。	
\\	僕[ぼく] と 一緒に[いっしょに] 来て[きて] くれ 。
\\	アパートには来ないでくれ。	
\\	アパート には 来ない[こない] でくれ 。
\\	もしもし、一郎君いますか。	
\\	もしもし 、 一郎[いちろう] 君[くん] いま すか 。
\\	田口君が来年結婚するそうだ。	
\\	田口[たぐち] 君[くん] が 来年[らいねん] 結婚[けっこん] するそうだ 。
\\	田口一郎君、昇進おめでとう。	
\\	田口[たぐち] 一郎[いちろう] 君[くん] 、 昇進[しょうしん] おめでとう 。
\\	東京からサンフランシスコまで飛行機で九時間くらい掛かります。	
\\	東京[とうきょう] から サンフランシスコ まで 飛行機[ひこうき] で 九時[くじ] 間[かん] くらい 掛か[かか] ります 。
\\	ーその車はいくらぐらいでしたか。 ー百五十万円くらいでした。	ー
\\	ー その 車[くるま] はいくらぐらい でしたか 。 ー 百[ひゃく] 五十[ごじゅう] 万円[まんえん] くらい でした 。
\\	スミスさんは京都に四か月くらい行っていました。	
\\	スミス さんは 京都[きょうと] に 四[よん] か 月[げつ] くらい 行って[いって] いました 。
\\	山田さんぐらい英語が出来れば楽しいでしょうね。	
\\	山田[やまだ] さん ぐらい 英語[えいご] が 出来れ[できれ] ば 楽し[たのし] いでしょうね 。
\\	私だってそれぐらいのことは分かりますよ。	
\\	私[わたし] だってそれ ぐらい の こと は 分か[わか] りますよ 。
\\	大川さんは私に本を呉れました。	
\\	大川[おおかわ] さんは 私[わたし] に 本[ほん] を 呉れ[くれ] ました 。
\\	ビルは何を呉れましたか。	
\\	ビル は 何を[なにを] 呉れ[くれ] ましたか 。
\\	川村さんは私の娘にレコードを呉れた。	
\\	川村[かわむら] さんは 私[わたし] の 娘[むすめ] に レコード を 呉れ[くれ] た 。
\\	私は中島さんに酒を上げた。	
\\	私[わたし] は 中島[なかじま] さんに 酒[さけ] を 上げ[あげ] た 。
\\	先生は本を下さいました。	
\\	先生[せんせい] は 本[ほん] を 下[くだ]さい ました 。
\\	父は私にカメラを買って呉れました。	
\\	父[ちち] は 私[わたし] に カメラ を 買って[かって] 呉れ[くれ] ました 。
\\	道男は私を慰めて呉れました。	
\\	道男[みちお] は 私[わたし] を 慰め[なぐさめ] て 呉れ[くれ] ました 。
\\	母はケーキを焼いて呉れた。	
\\	母[はは] は ケーキ を 焼い[やい] て 呉れ[くれ] た 。
\\	ウォーカーさんは私の息子に英語を教えて呉れている。	
\\	ウォーカー さんは 私[わたし] の 息子[むすこ] に 英語[えいご] を 教[おし]え て 呉れ[くれ] ている 。
\\	子供達は何をして呉れましたか。	
\\	子供達[こどもたち] は 何を[なにを] して 呉れ[くれ] ましたか 。
\\	知らない人が私にコーラを買って呉れた。	
\\	知ら[しら] ない 人[ひと] が 私[わたし] に コーラ を 買って[かって] 呉れ[くれ] た 。
\\	皆は私のために働いて呉れた。	
\\	皆[みんな]は 私[わたし]のために 働[はたら]いて 呉[く]れた。
\\	先生は私に本を貸して下さった。	
\\	先生[せんせい]は 私[わたし]に 本[ほん]を 貸[か]して 下[くだ]さった。
\\	田中さんが明日うちへ来ます。	
\\	田中[たなか] さんが 明日[あした] うち へ 来ま[きま] す 。
\\	ナンシーは昨日パーティーに来ましたか。	
\\	ナンシー は 昨日[きのう] パーティー に 来ま[きま] したか 。
\\	来週木口さんは名古屋に来るそうだ。	
\\	来週[らいしゅう] 木口[きぐち] さんは 名古屋[なごや] に 来る[くる] そうだ 。
\\	もしもし、家内が来たら直ぐ帰るように言って下さい。	
\\	もしもし、 家内[かない]が 来[き]たら 直[す]ぐ 帰[かえ]るように 言[い]って 下[くだ]さい。
\\	今日はまだ新聞が来ない。	
\\	今日[きょう]はまだ 新聞[しんぶん]が 来[こ]ない。
\\	僕の研究室に明日来て下さい。	
\\	僕[ぼく] の 研究室[けんきゅうしつ] に 明日[あした] 来て[きて] 下[くだ]さい 。
\\	私はコンピューターが少し分かって来ました。	
\\	私[わたし] は コンピューター が 少し[すこし] 分か[わか] って 来ま[きま] した 。
\\	僕は色々日本の歴史書を読んで来た。	
\\	僕は[ぼくは] 色々[いろいろ] 日本[にほん] の 歴史書[れきししょ] を 読んで[よんで] 来た[きた] 。
\\	テニスをしていたら急に雨が降って来た。	
\\	テニス をしていたら 急に[きゅうに] 雨[あめ] が 降って[ふって] 来た[きた] 。
\\	午後から頭が痛くなって来ました。	
\\	午後[ごご] から 頭[あたま] が 痛く[いたく] なって 来ま[きま] した 。
\\	僕はこのごろ太って来ました。	
\\	僕は[ぼくは] このごろ 太って[ふとって] 来ま[きま] した 。
\\	あの子はこの頃随分きれいになって来たね。	
\\	あの 子[こ] はこの 頃[ごろ] 随分[ずいぶん]きれい になって 来た[きた] ね 。
\\	今までたくさん本を読んで来ましたが、これからも読んで行くつもりです。	
\\	今まで[いままで] たくさん 本[ほん] を 読んで[よんで] 来ま[きま] したが 、 これからも 読んで[よんで] 行く[いく]つもりです 。
\\	今まで遊んで来ましたが、これからは一所懸命勉強するつもりです。	
\\	今まで[いままで] 遊んで[あそんで] 来ま[きま] したが 、 これからは 一所懸命[いっしょうけんめい] 勉強[べんきょう] するつもりです 。
\\	私は会社にバスに乗って来ます。	
\\	私[わたし] は 会社[かいしゃ] に バス に 乗って[のって] 来ま[きま] す 。
\\	四時までには帰って来て下さい。	
\\	四時[よじ] までには 帰って[かえって] 来て[きて] 下[くだ]さい 。
\\	おいしいケーキを買って来ました。	
\\	おいしい ケーキ を 買って[かって] 来ま[きま] した 。
\\	友達をうちに連れて来た。	
\\	友達[ともだち] を うち に 連れ[つれ] て 来た[きた] 。
\\	あの本持って来た?	
\\	あの 本[ほん] 持って[もって] 来た[きた] 
\\	ちょっと見て来るよ。	
\\	ちょっと 見て[みて] 来る[くる] よ 。
\\	傘を取って来ます。	
\\	傘[かさ] を 取って[とって] 来ま[きま] す 。
\\	木村はまだ昼ご飯を食べています。	
\\	木村[きむら]はまだ 昼[ひる]ご 飯[はん]を 食[た]べています。
\\	太田さんはまだそのことを知らない。	
\\	太田[おおた]さんはまだそのことを 知[し]らない。
\\	お酒はまだありますか。	
\\	お 酒[さけ]はまだありますか。
\\	僕はまだ日本へ行ったことがない。	
\\	僕[ぼく]はまだ 日本[にほん]へ 行[い]ったことがない。
\\	昨日は三時から五時まで友達とテニスをしました。	
\\	昨日[きのう]は三 時[じ]から五 時[じ]まで 友達[ともだち]とテニスをしました。
\\	神戸から長崎まで新幹線で三時間掛かります。	
\\	神戸[こうべ] から 長崎[ながさき] まで 新幹線[しんかんせん] で 三時[さんじ] 間[かん] 掛か[かか] ります 。
\\	僕は行くまでうちで待っていて下さい。	
\\	僕は[ぼくは] 行く[いく] まで うち で 待って[まって] いて 下[くだ]さい 。
\\	このホールは二千人まで入れる。	
\\	この ホール は 二千[にせん] 人[にん] まで 入れ[はいれ] る 。
\\	あの人は鼠やスカンクは勿論蛇まで好きです。	
\\	あの 人[ひと] は 鼠[ねずみ] や スカンク は 勿論[もちろん] 蛇[へび] まで 好き[すき] です 。
\\	アメリカ人は毎週月曜日から金曜日まで働く。	
\\	アメリカ 人[じん] は 毎週[まいしゅう] 月曜日[げつようび] から 金曜日[きんようび] まで 働く[はたらく] 。
\\	駅から国立博物館までは歩いて十分ぐらいです。	
\\	駅[えき] から 国立[こくり] 博物館[はくぶつかん] までは 歩い[あるい] て 十分[じゅっぷん] ぐらいです 。
\\	スミスさんは刺身は勿論、納豆まで食べるんですよ。	
\\	スミス さんは 刺身[さしみ] は 勿論[もちろん] 、 納豆[なっとう] まで 食[た]べるんですよ 。
\\	飛行機が出るまでロビーで友達と話していた。	
\\	飛行機[ひこうき] が 出る[でる] まで ロビー で 友達[ともだち] と 話し[はなし] ていた 。
\\	私は来週の月曜日まで休みます。	
\\	私[わたし] は 来週[らいしゅう] の 月曜日[げつようび] まで 休み[やすみ] ます 。
\\	僕は十時までに帰ります。	
\\	僕は[ぼくは] 十時[じゅうじ] までに 帰り[かえり] ます 。
\\	学校が始まるまでにこの本を読んでおいて下さい。	
\\	学校[がっこう] が 始ま[はじま] るまでにこの 本[ほん] を 読んで[よんで] おいて 下[くだ]さい 。
\\	ー何時までに空港に行けばいいでしょうか。 ー出発の一時間前までに来て下さい。	ー
\\	ー 何時[なんじ] までに 空港[くうこう] に 行け[いけ] ば いい でしょうか 。 ー 出発[しゅっぱつ] の 一時間前[いちじかんまえ] までに 来て[きて] 下[くだ]さい 。
\\	僕はこのレポートを一月二十日までに書き上げなければならないんだ。	
\\	僕は[ぼくは] この レポート を 一月[いちがつ] 二十日[はつか] までに 書き上げ[かきあげ] なければならないんだ 。
\\	ジェット機がパリに着くまでに本を三冊読んでしまいました。	
\\	ジェット 機[き] が パリ に 着く[つく] までに 本[ほん] を 三冊[さんさつ] 読んで[よんで] しまいました 。
\\	山田は来月いる。	
\\	山田[やまだ] は 来月[らいげつ] いる 。
\\	私は五時まで待っている。	
\\	私[わたし] は 五時[ごじ] まで 待って[まって] いる 。
\\	来月までにこの仕事をします。	
\\	来月[らいげつ] までにこの 仕事[しごと] をします 。
\\	授業前に郵便局に行く。	
\\	授業前[じゅぎょうまえ] に 郵便局[ゆうびんきょく] に 行く[いく] 。
\\	この教科書は二十課までありますが今学期は十課までで終わります。	
\\	この 教科書[きょうかしょ] は 二十課[にじゅっか] までありますが 今学期[こんがっき] は 十課[じゅっか] までで 終わり[おわり] ます 。
\\	この教科書は二十課までありますが大事なことは十課までに全部出て来ます。	
\\	この 教科書[きょうかしょ] は 二十課[にじゅっか] までありますが 大事[だいじ] なことは 十課[じゅっか] までに 全部[ぜんぶ] 出て[でて] 来ま[きま] す 。
\\	ジャクソンさんは日本へ行く前に日本語を勉強した。	
\\	ジャクソン さんは 日本[にほん] へ 行く[いく] 前[まえ] に 日本語[にほんご] を 勉強[べんきょう] した 。
\\	旅行の前に風邪を引いた。	
\\	旅行[りょこう] の 前[まえ] に 風邪[かぜ] を 引い[ひい] た 。
\\	駅の前にタバコ屋があります。	
\\	駅[えき] の 前[まえ] に タバコ 屋[や] があります 。
\\	日本人はご飯を食べる前に「いただきます」と言う。	
\\	日本人[にほんじん] はご 飯[はん] を 食[た]べる 前[まえ] に 
\\	いただきます 
\\	と 言う[いう] 。
\\	テイラーさんは日本へ行く前にハワイに寄りました。	
\\	テイラーさんは 日本[にほん]へ 行[い]く 前[まえ]にハワイに 寄[よ]りました。
\\	僕は試験の前に映画を見に行った。	
\\	僕は[ぼくは] 試験[しけん] の 前[まえ] に 映画[えいが] を 見に行った[みにいった] 。
\\	雨が降らないうちに帰りましょう。	
\\	雨[あめ] が 降ら[ふら] ない うち に 帰り[かえり] ましょう 。
\\	忘れないうちに言っておこう。	
\\	忘[わす]れないうちに 言[い]っておこう。
\\	私は毎日一マイル泳ぎます。	
\\	私[わたし] は 毎日[まいにち] 一[いち] マイル 泳ぎ[およぎ] ます 。
\\	弘は毎晩道子に電話しているそうだ。	
\\	弘[ひろし] は 毎晩[まいばん] 道子[みちこ] に 電話[でんわ] しているそうだ 。
\\	台風は毎時二十キロの速さで北に進んでいる。	
\\	台風[たいふう] は 毎時[まいじ] 二十キロ[にじゅっきろ] の 速さ[はやさ] で 北[きた] に 進ん[すすん] でいる 。
\\	僕は毎年一度は日本へ行く。	
\\	僕は[ぼくは] 毎年[まいとし] 一度[いちど] は 日本[にほん] へ 行く[いく] 。
\\	ジーンは毎食サラダを食べる。	
\\	ジーン は 毎食[まいしょく] サラダ を 食[た]べる 。
\\	ジョンは会う人毎に日本語で挨拶をした。	
\\	ジョン は 会う[あう] 人[ひと] 毎[ごと] に 日本語[にほんご] で 挨拶[あいさつ] をした 。
\\	弟はテレビを点けたまま寝てしまいました。	
\\	弟[おとうと] は テレビ を 点け[つけ] たまま 寝て[ねて] しまいました 。
\\	妹はテレビを点けたままにしておきました。	
\\	妹[いもうと] は テレビ を 点け[つけ] たままにしておきました 。
\\	姉さんはテレビを点けたままだ。	
\\	姉[ねえ]さんはテレビを 点け[つけ] たままだ 。
\\	この部屋は昨日のままだ。	
\\	この 部屋[へや] は 昨日[きのう] のままだ 。
\\	ここはこのままにして置きます。	
\\	ここはこのままにして 置き[おき] ます 。
\\	テレビが点いたままになっている。	
\\	テレビ が 点[てん] いたままになっている 。
\\	帽子を被ったままの学生が教室にいました。	
\\	帽子[ぼうし] を 被った[かぶった] ままの 学生[がくせい] が 教室[きょうしつ] に いま した 。
\\	電気を点けたまま寝てしまいました。	
\\	電気[でんき] を 点け[つけ] たまま 寝て[ねて] しまいました 。
\\	ビールを買ったまま飲まなかった。	
\\	ビール を 買[か] った まま 飲[の] ま なかった 。
\\	聞いたままを友達に話しました。	
\\	聞い[きい] たままを 友達[ともだち] に 話し[はなし] ました 。
\\	あの人はあれからずっと寝たままだ。	
\\	あの 人[ひと] はあれからずっと 寝た[ねた] ままだ 。
\\	高山さんはアメリカへ行ったまま帰らなかった。	
\\	高山[たかやま] さんは アメリカ へ 行[い] った まま 帰ら[かえら] なかった 。
\\	車のエンジンをかけたままにしておいた。	
\\	車[くるま] の エンジン をかけたままにしておいた 。
\\	今のままにしておいて下さい。	
\\	今[いま] のままにしておいて 下[くだ]さい 。
\\	ドアを閉めないまま出かけてしまった。	
\\	ドア を 閉め[しめ] ないまま 出か[でか] けてしまった 。
\\	さようならの挨拶もしないまま行ってしまった。	
\\	さようならの 挨拶[あいさつ] もしないまま 行って[いって] しまった 。
\\	言われるままを払った。	
\\	言[い] われるままを 払[はら] った。
\\	思っていたままを書いて見た。	
\\	思って[おもって] いたままを 書い[かい] て 見た[みた] 。
\\	弟はテレビを点けたままで寝てしまった。	
\\	弟[おとうと] は テレビ を 点け[つけ] たままで 寝て[ねて] しまった 。
\\	ドアを閉めないで出かけてしまった。	
\\	ドア を 閉め[しめ] ないで 出か[でか] けてしまった 。
\\	ドアを閉めずに出かけてしまった。	
\\	ドア を 閉め[しめ] ずに 出か[でか] けてしまった 。
\\	私が彼に話しましょう。	
\\	私[わたし] が 彼[かれ] に 話し[はなし] ましょう 。
\\	警視庁に行きましょう。	
\\	警視庁[けいしちょう] に 行き[いき] ましょう 。
\\	私達が手伝いましょう。	
\\	私達[わたしたち] が 手伝い[てつだい] ましょう 。
\\	飛行機で行きましょうか。	
\\	飛行機[ひこうき] で 行き[いき] ましょうか 。
\\	その鞄を持ちましょう。	
\\	その 鞄[かばん] を 持ち[もち] ましょう 。
\\	結婚式に行くのは止めましょう。	
\\	結婚式[けっこんしき] に 行く[いく] のは 止め[やめ] ましょう 。
\\	フレッドには言わないでおきましょう。	
\\	フレッド には 言わ[いわ] ないでおきましょう 。
\\	村田さんはコーヒーを飲もうと言った。	
\\	村田[むらた] さんは コーヒー を 飲も[のも] うと 言った[いった] 。
\\	僕は来年英国へ行こうと思います。	
\\	僕は[ぼくは] 来年[らいねん] 英国[えいこく] へ 行こ[いこ] うと 思い[おもい] ます 。
\\	私には遠くのものがよく見えます。	
\\	私[わたし] には 遠く[とおく] のものが よく 見え[みえ] ます 。
\\	窓から海が見える。	
\\	窓[まど] から 海[うみ] が 見え[みえ] る 。
\\	今日は山が近く見えます。	
\\	今日は[きょうは] 山[やま] が 近く[ちかく] 見え[みえ] ます 。
\\	僕には木の上の小さな鳥が見えた。	
\\	僕[ぼく] には 木[き] の 上の[うえの] 小さ[ちいさ] な 鳥[とり] が 見え[みえ] た 。
\\	東京タワーからは東京の町がよく見える。	
\\	東京[とうきょう] タワー からは 東京[とうきょう] の 町[まち] がよく 見え[みえ] る 。
\\	あの人は年よりずっと若く見える。	
\\	あの 人[ひと] は 年[とし] よりずっと 若く[わかく] 見え[みえ] る 。
\\	女の人は夜きれいに見える。	
\\	女の人[おんなのひと] は 夜[よる] きれい に 見え[みえ] る 。
\\	中島さんは学生に見える。	
\\	中島[なかじま] さんは 学生[がくせい] に 見え[みえ] る 。
\\	お母さん、お客さんが見えたよ。	
\\	お 母さん[かあさん] 、 お 客[きゃく] さんが 見え[みえ] たよ 。
\\	学会には上田先生も見えた。	
\\	学会[がっかい] には 上田[うえだ] 先生[せんせい] も 見え[みえ] た 。
\\	僕は目が見えない。	
\\	僕は[ぼくは] 目[め] が 見え[みえ] ない 。
\\	テレビを買ったからテレビが見られる。	
\\	テレビ を 買[か] った から テレビ が 見ら[みら] れる 。
\\	あの映画はどこで見られますか。	
\\	あの 映画[えいが] は どこ で 見ら[みら] れますか 。
\\	よく見えないからステージの近くに行こう。	
\\	よく 見え[みえ] ないから ステージ の 近く[ちかく] に 行こ[いこ] う 。
\\	隣に高いビルが建ったので山が見えなくなった。	
\\	隣[となり] に 高い[たかい] ビル が 建った[たった] ので 山[やま] が 見え[みえ] なくなった 。
\\	僕は日本の小説を読んでみます。	
\\	僕は[ぼくは] 日本[にほん] の 小説[しょうせつ] を 読んで[よんで] みます 。
\\	面白そうなコンサートだったので行ってみました。	
\\	面白そ[おもしろそ] うな コンサート だったので 行って[いって] みました 。
\\	ここの刺身はおいしいですよ。食べてみますか。	
\\	ここの 刺身[さしみ] は おいしい ですよ 。 食べて[たべて] みますか 。
\\	ブラウンさんはトムのシャツを着てみた。	
\\	ブラウン さんは トム の シャツ を 着て[きて] みた 。
\\	ブラウンさんはトムのシャツを着ようとした。	
\\	ブラウン さんは トム の シャツ を 着[き] ようとした 。
\\	私も警官だ。	
\\	私[わたし] も 警官[けいかん] だ 。
\\	私はスペイン語も話す。	
\\	私[わたし] は スペイン 語[ご] も 話す[はなす] 。
\\	僕は村山さんにもプレゼントを上げる。	
\\	僕は[ぼくは] 村山[むらやま] さんにも プレゼント を 上げ[あげ] る 。
\\	ハートさんは日本へ行った。ルイスさんも行った。	
\\	ハート さんは 日本[にほん] へ 行[い] った 。 ルイス さんも 行[い] った 。
\\	林さんはテニスをします。彼はゴルフもします。	
\\	林[はやし] さんは テニス をします 。 彼は[かれは] ゴルフ もします 。
\\	この町では新しい魚が買えません。隣の町でも買えません。	
\\	この 町[まち] では 新[あたら] しい 魚[さかな] が 買え[かえ] ません 。 隣[となり] の 町[まち] でも 買え[かえ] ません 。
\\	僕の家は居間が狭い。それに台所も不便だ。	
\\	僕[ぼく] の 家[いえ] は 居間[いま] が 狭い[せまい] 。 それに 台所[だいどころ] も 不便[ふべん] だ 。
\\	彼は医者だ。そして先生でもある。	
\\	彼は[かれは] 医者[いしゃ] だ 。 そして 先生[せんせい] でもある 。
\\	グレーさんはこんな難しい漢字も読める。	
\\	グレー さんはこんな 難しい[むずかしい] 漢字[かんじ] も 読め[よめ] る 。
\\	ハリスさんはこんな易しい漢字も読めません。	
\\	ハリスさんはこんな 易[やさ]しい 漢字[かんじ]も 読[よ]めません。
\\	私は漢字を八千も知っています。	
\\	私[わたし] は 漢字[かんじ] を 八千[はっせん] も 知って[しって] います 。
\\	僕は漢字を一つも知らない。	
\\	僕は[ぼくは] 漢字[かんじ] を 一つ[ひとつ] も 知ら[しら] ない 。
\\	僕は何も食べなかった。	
\\	僕は[ぼくは] 何も[なにも] 食べ[たべ] なかった 。
\\	ナンシーは私の話を聞きもしません。	
\\	ナンシー は 私[わたし] の 話[はなし] を 聞き[きき] もしません 。
\\	春男は一年生の足し算もまともに出来ない。	
\\	春男[はるお] は 一年生[いちねんせい] の 足し算[たしざん] もまともに 出来な[できな] い 。
\\	今日は車が十台も売れた。	
\\	今日は[きょうは] 車[くるま] が 十[じゅう] 台[だい] も 売れ[うれ] た 。
\\	そのパーテイーには女の子は一人も来なかった。	
\\	その パーテイー には 女の子[おんなのこ] は 一人[ひとり] も 来な[こな] かった 。
\\	誰もそのことを知りません。	
\\	誰も[だれも] その こと を 知り[しり] ません 。
\\	僕は木村さんが呉れた本をまだ見もしていない。	
\\	僕[ぼく]は 木村[きむら]さんが 呉[く]れた 本[ほん]をまだ 見[み]もしていない。
\\	グレーさんは非常に難しい漢字も読める。	
\\	グレー さんは 非常に[ひじょうに] 難しい[むずかしい] 漢字[かんじ] も 読め[よめ] る 。
\\	ハリスさんは非常に易しい漢字も読めない。	
\\	ハリスさんは 非常[ひじょう]に 易[やさ]しい 漢字[かんじ]も 読[よ]めない。
\\	大木君はいつも人に親切だ。	
\\	大木[おおき] 君[くん] は いつ も 人[ひと] に 親切[しんせつ] だ 。
\\	小山はいつも約束を守らない。	
\\	小山[こやま] は いつ も 約束[やくそく] を 守ら[まもら] ない 。
\\	私はもう昼ご飯を食べました。	
\\	私[わたし] はもう 昼[ひる] ご 飯[はん] を 食べ[たべ] ました 。
\\	ヒルさんはもう日本語を勉強していない。	
\\	ヒル さんはもう 日本語[にほんご] を 勉強[べんきょう] していない 。
\\	もう宿題をしましたか。	
\\	もう 宿題[しゅくだい] をしましたか 。
\\	田中さんも中山さんもデパートに勤めている。	
\\	田中[たなか] さんも 中山[なかやま] さんも デパート に 勤め[つとめ] ている 。
\\	札幌へはバスでも電車でも行けます。	
\\	札幌[さっぽろ] へは バス でも 電車[でんしゃ] でも 行け[いけ] ます 。
\\	このうちは広くも狭くもありません。	
\\	この うち は 広く[ひろく] も 狭く[せまく] もありません 。
\\	日本人もアメリカ人もよく働く。	
\\	日本人[にほんじん] も アメリカ 人[じん] もよく 働く[はたらく] 。
\\	山川さんは酒もたばこも遣りません。	
\\	山川[やまかわ] さんは 酒[さけ] もたばこも 遣り[やり] ません 。
\\	弟にも妹にも本を買って遣りました。	
\\	弟[おとうと] にも 妹[いもうと] にも 本[ほん] を 買って[かって] 遣り[やり] ました 。
\\	和枝は美人でも不美人でもない。	
\\	和枝[かずえ] は 美人[びじん] でも 不[ふ] 美人[びじん] でもない 。
\\	この建物はきれいでも立派でもありません。	
\\	この 建物[たてもの] は きれい でも 立派[りっぱ] でもありません 。
\\	ーどうして行かないの? ーだって、忙しいもの。	ー
\\	ー どうして 行か[いか] ないの 
\\	ー だって 、 忙し[いそがし] いもの 。
\\	よくあんな男とデート出来るものだ。	
\\	よくあんな 男[おとこ] と デート 出来る[できる] ものだ 。
\\	昔はよく映画を見たものです。	
\\	昔[むかし] はよく 映画[えいが] を 見た[みた] ものです 。
\\	こんないいうちに一度住んで見たいものだ。	
\\	こんないい うち に 一度[いちど] 住ん[すん] で 見た[みた] いものだ 。
\\	朝人に会ったら「おはよう」と言うものだ。	
\\	朝[あさ] 人[ひと] に 会った[あった] ら 
\\	おはよう 
\\	と 言う[いう] ものだ 。
\\	年を取ると昔のことを話したがるものです。	
\\	年[とし] を 取る[とる] と 昔[むかし] の こと を 話し[はなし] たがるものです 。
\\	ーどうして食べないの? ーだって、不味いもの。	ー
\\	ー どうして 食べ[たべ] ないの 
\\	ー だって 、 不味い[まずい] もの 。
\\	ーどうしてその本、読まないんだい? ーだって、よく分からないんだもの。	ー
\\	ー どうして その 本[ほん] 、 読ま[よま] ないんだい 
\\	ー だって 、 よく 分か[わか] らないんだもの 。
\\	よくそんな馬鹿なことをしたものだ!	
\\	よくそんな 馬鹿[ばか] なことをしたものだ 
\\	月日の立つのは速いものだ。	
\\	月日[つきひ]の 立[た]つのは 速[はや]いものだ。
\\	昔はこの川にも魚がいたものだ。	
\\	昔[むかし]はこの 川[かわ]にも 魚[さかな]がいたものだ。
\\	一度是非一緒にテニスをしたいものですね。	
\\	一度[いちど] 是非[ぜひ] 一緒に[いっしょに] テニス をしたいものですね 。
\\	人の家に行く時はお土産を持って行くものです。	
\\	人[ひと] の 家[いえ] に 行く[いく] 時[とき] はお 土産[みやげ] を 持って[もって] 行く[いく] ものです 。
\\	好きな仕事をしていると病気にならないものだ。	
\\	好き[すき] な 仕事[しごと] をしていると 病気[びょうき] にならないものだ 。
\\	金がないもんで映画も見られないんだ。	
\\	金[かね] がないもんで 映画[えいが] も 見ら[みら] れないんだ 。
\\	今薬が飲みたいんです。	
\\	今[いま] 薬[くすり] が 飲み[のみ] たいんです 。
\\	今直ぐ会いに行くんです。	
\\	今[いま] 直ぐ[すぐ] 会[あ] い に 行く[いく] んです 。
\\	私は山本さんに本を貰いました。	
\\	私[わたし] は 山本[やまもと] さんに 本[ほん] を 貰い[もらい] ました 。
\\	和重さんは橋本さんにウイスキーを貰った。	
\\	和重[かずしげ] さんは 橋本[はしもと] さんに ウイスキー を 貰った[もらった] 。
\\	進吉さんに何を貰ったの?	
\\	進吉[しんきち] さんに 何を[なにを] 貰った[もらった] の 
\\	僕は先生に本を頂いた。	
\\	僕[ぼく]は 先生[せんせい]に 本[ほん]を 頂[いただ]いた。
\\	僕は安部さんに本を貰った。	
\\	僕は[ぼくは] 安部[あべ] さんに 本[ほん] を 貰った[もらった] 。
\\	私は日本政府からお金を貰った。	
\\	私[わたし] は 日本政府[にほんせいふ] からお 金[かね] を 貰った[もらった] 。
\\	私は父にカメラを買って貰いました。	
\\	私[わたし] は 父[ちち] に カメラ を 買って[かって] 貰い[もらい] ました 。
\\	友達に来て貰った。	
\\	友達[ともだち] に 来て[きて] 貰った[もらった] 。
\\	饗庭さんは阿川さんに車を貸して貰った。	
\\	饗庭[あえば] さんは 阿川[あがわ] さんに 車[くるま] を 貸し[かし] て 貰った[もらった] 。
\\	ベックさんに何をして貰いましたか。	
\\	ベック さんに 何を[なにを] して 貰い[もらい] ましたか 。
\\	會川さんは稲葉さんに英語を教えて貰っている。	
\\	會川[あいかわ] さんは 稲葉[いなば] さんに 英語[えいご] を 教[おし]え て 貰って[もらって] いる 。
\\	私は隣の人に鉛筆を貸して貰った。	
\\	私[わたし] は 隣[となり] の 人[ひと] に 鉛筆[えんぴつ] を 貸し[かし] て 貰った[もらった] 。
\\	私は先生に本を貸して頂いた。	
\\	私[わたし] は 先生[せんせい] に 本[ほん] を 貸し[かし] て 頂い[いただい] た 。
\\	たばこ吸うな!	
\\	たばこ 吸う[すう] な 
\\	酒をあまり飲むな!	
\\	酒[さけ]をあまり 飲[の]むな!
\\	そんな所にもう行くな!	
\\	そんな 所[ところ] にもう 行く[いく] な 
\\	あんな男とは結婚するな!	
\\	あんな 男[おとこ] とは 結婚[けっこん] するな 
\\	もう帰って来るな!	
\\	もう 帰って[かえって] 来る[くる] な 
\\	酒をあまり飲むなよ。	
\\	酒[さけ]をあまり 飲[の]むなよ。
\\	日本の食べ物の中では鮨や天婦羅などが好きです。	
\\	日本[にほん] の 食べ物[たべもの] の 中[なか] では 鮨[すし] や 天婦羅[てんぷら] などが 好き[すき] です 。
\\	ー冬休みはどこに行きましょうか。 ーハワイなどどうですか。	ー
\\	ー 冬休み[ふゆやすみ] は どこ に 行き[いき] ましょうか 。 ーハワイ など どう ですか 。
\\	大学ではフランス語やドイツ語などを勉強した。	
\\	大学[だいがく] では フランス 語[ご] や ドイツ 語[ご] などを 勉強[べんきょう] した 。
\\	この大学には中国や韓国などから学生が大勢来ます。	
\\	この 大学[だいがく] には 中国[ちゅうごく] や 韓国[かんこく] などから 学生[がくせい] が 大勢[おおぜい] 来ま[きま] す 。
\\	ー経済学を専攻するつもりですか。 ーいいえ、経済学など専攻するつもりはありません。	ー
\\	ー 経済学[けいざいがく] を 専攻[せんこう] するつもりですか 。 ー いいえ 、 経済学[けいざいがく] など 専攻[せんこう] するつもりはありません 。
\\	ーこの翻訳をして下さいませんか。 ーこんな難しいもの、私などには出来ません。	ー
\\	ー この 翻訳[ほんやく] をして 下[くだ]さい ませんか 。 ー こんな 難しい[むずかしい] もの 、 私[わたし] などには 出来ま[できま] せん 。
\\	箸でなど食べられない。	
\\	箸[はし] でなど 食べ[たべ] られない 。
\\	箸などで食べる人もいる。	
\\	箸[はし] などで 食[た]べる 人[ひと] も いる 。
\\	田口さんはいつも新聞を読みながら朝ご飯を食べます。	
\\	田口[たぐち] さんは いつ も 新聞[しんぶん] を 読み[よみ] ながら 朝[あさ] ご 飯[はん] を 食べ[たべ] ます 。
\\	歩きながら本を読むのは危ないです。	
\\	歩き[あるき] ながら 本[ほん] を 読む[よむ] のは 危な[あぶな] いです 。
\\	深沢さんは笑いながら私の話を聞いていた。	
\\	深沢[ふかざわ] さんは 笑い[わらい] ながら 私[わたし] の 話[はなし] を 聞い[きい] ていた 。
\\	僕は寝ながら小説を読むのが好きだ。	
\\	僕は[ぼくは] 寝な[ねな] がら 小説[しょうせつ] を 読む[よむ] のが 好き[すき] だ 。
\\	ものを食べながら話してはいけません。	
\\	ものを 食べ[たべ] ながら 話し[はなし] てはいけません 。
\\	後藤さんは会社で働きながら学校に行っている。	
\\	後藤[ごとう] さんは 会社[かいしゃ] で 働き[はたらき] ながら 学校[がっこう] に 行って[いって] いる 。
\\	ーちょっと話がしたいんですが。 ーじゃ、コーヒーを飲みながら話しましょう。	ー
\\	ー ちょっと 話[はなし] がしたいんですが 。 ー じゃ 、 コーヒー を 飲み[のみ] ながら 話し[はなし] ましょう 。
\\	ロクサンは昨日朝ご飯を食べないで学校へ行きました。	
\\	ロクサン は 昨日[きのう] 朝[あさ] ご 飯[はん] を 食べ[たべ] ないで 学校[がっこう] へ 行き[いき] ました 。
\\	中田さんは沖縄に行かないで九州に行った。	
\\	中田[なかだ] さんは 沖縄[おきなわ] に 行か[いか] ないで 九州[きゅうしゅう] に 行[い] った 。
\\	辞書を使わないで読んで下さい。	
\\	辞書[じしょ] を 使わ[つかわ] ないで 読んで[よんで] 下[くだ]さい 。
\\	まだ帰らないで下さい。	
\\	まだ帰[かえ] らないで 下[くだ]さい 。
\\	電話しないで欲しい。	
\\	電話[でんわ] しないで 欲しい[ほしい] 。
\\	やかましくしないで貰いたい。	
\\	やかましくしないで 貰い[もらい] たい 。
\\	まだあるから買わないでおきました。	
\\	まだあるから 買[か]わないでおきました。
\\	ナンシーは昨日朝ご飯を食べずに学校へ行った。	
\\	ナンシー は 昨日[きのう] 朝[あさ] ご 飯[はん] を 食べ[たべ] ずに 学校[がっこう] へ 行[い] った 。
\\	藤枝さんは仙台に行かずに鹿児島に行った。	
\\	藤枝[ふじえだ] さんは 仙台[せんだい] に 行か[いか] ずに 鹿児島[かごしま] に 行[い] った 。
\\	辞書を使わずに読んで下さい。	
\\	辞書[じしょ] を 使わ[つかわ] ずに 読んで[よんで] 下[くだ]さい 。
\\	まだあるから買わずにおきました。	
\\	まだあるから 買[か] わずにおきました 。
\\	僕は英語が話せないで恥ずかしかった。	
\\	僕は[ぼくは] 英語[えいご] が 話せ[はなせ] ないで 恥ずかし[はずかし] かった 。
\\	スージが遊びに来なくて助かった。	
\\	スージ が 遊[あそ] びに 来[こ] なくて 助[たす] かった 。
\\	静男は帽子を脱がずに部屋に入った。	
\\	静男[しずお] は 帽子[ぼうし] を 脱が[ぬが] ずに 部屋[へや] に 入[はい] った。
\\	見ないで下さい。	
\\	見な[みな] いで 下[くだ]さい 。
\\	食べずにおいた。	
\\	食べ[たべ] ずにおいた 。
\\	私は看護婦ではなくて医学者です。	
\\	私[わたし] は 看護婦[かんごふ] ではなくて 医学者[いがくしゃ] です 。
\\	ー今日は働かなくてもいいんですか。 ーっいぇ、働かなければなりません。	ー
\\	ー 今日[きょう]は 働[はたら] かなくてもいいんですか 。 ー っいぇ 、 働[はたら] かなければなりません 。
\\	バスケットボールの選手は大きくなければならない。	
\\	バスケットボール の 選手[せんしゅ] は 大き[おおき] くなければならない 。
\\	図書館は静かでなければなりません。	
\\	図書館[としょかん] は 静か[しずか] でなければなりません 。
\\	この仕事は学生でなければならない。	
\\	この 仕事[しごと] は 学生[がくせい] でなければならない 。
\\	私は明日朝五時に起きなければなりません。	
\\	私[わたし] は 明日[あした] 朝[あさ] 五時[ごじ] に 起き[おき] なければなりません 。
\\	あの部屋はもう少し広くなければなりません。	
\\	あの 部屋[へや] はもう 少し[すこし] 広く[ひろく] なければなりません 。
\\	この仕事をする為には英語が上手でなければならない。	
\\	この 仕事[しごと] をする 為に[ために] は 英語[えいご] が 上手[じょうず] でなければならない 。
\\	このゴルフ場を使う為にはメンバーでなければならない。	
\\	この ゴルフ 場[じょう] を 使う[つかう] 為に[ために] は メンバー でなければならない 。
\\	もう帰らなければ。	
\\	もう 帰ら[かえら] なければ 。
\\	もう帰らなけりゃ。	
\\	もう 帰ら[かえら] なけりゃ 。
\\	もう帰らなきゃ。	
\\	もう 帰ら[かえら] なきゃ 。
\\	易しい漢字も書けなりました。	
\\	易[やさ]しい 漢字[かんじ]も 書[か]けなりました。
\\	前は酒をよく飲んでいたが、このごろは飲まなくなりました。	
\\	前[まえ] は 酒[さけ] をよく 飲ん[のん] でいたが 、 このごろは 飲[の] ま なくなりました 。
\\	彼は前よく電話をかけて来ましたが、もうかけて来なくなりました。	
\\	彼は[かれは] 前[まえ] よく 電話[でんわ] をかけて 来ま[きま] したが 、 もうかけて 来な[こな] くなりました 。
\\	日本語はもう難しくなくなりました。	
\\	日本語[にほんご] はもう 難しく[むずかしく] なくなりました 。
\\	この郊外も地下鉄が来て不便ではなくなった。	
\\	この 郊外[こうがい] も 地下鉄[ちかてつ] が 来て[きて] 不便[ふべん] ではなくなった 。
\\	やっと大学を出て学生ではなくなりました。	
\\	やっと 大学[だいがく] を 出て[でて] 学生[がくせい] ではなくなりました 。
\\	日本語が話せなくなった。	
\\	日本語[にほんご] が 話せ[はなせ] なくなった 。
\\	日本語が話せないようになった。	
\\	日本語[にほんご] が 話せ[はなせ] ないようになった 。
\\	日本語がもう話せない。	
\\	日本語[にほんご] がもう 話せ[はなせ] ない 。
\\	朝七時に起きられなくて会社に遅れました。	
\\	朝[あさ] 七時[しちじ] に 起き[おき] られなくて 会社[かいしゃ] に 遅れ[おくれ] ました 。
\\	先生の説明が分からなくて困りました。	
\\	先生[せんせい] の 説明[せつめい] が 分か[わか] らなくて 困り[こまり] ました 。
\\	日本では日本語が話せなくて残念でした。	
\\	日本[にほん] では 日本語[にほんご] が 話せ[はなせ] なくて 残念[ざんねん] でした 。
\\	試験は難しくなくてよかったですね。	
\\	試験[しけん] は 難しく[むずかしく] なくてよかったですね 。
\\	字が上手じゃなくて恥ずかしいんです。	
\\	字が[じが] 上手[じょうず] じゃなくて 恥ずかし[はずかし] いんです 。
\\	厳しい先生じゃなくてよかった。	
\\	厳し[きびし] い 先生[せんせい] じゃなくてよかった 。
\\	小川は頭がいいから勉強しなくても東大には入れるよ。	
\\	小川[おがわ] は 頭[あたま] がいいから 勉強[べんきょう] しなくても 東大[とうだい] には 入れ[いれ] るよ 。
\\	この難しい漢字は覚えなくてもいいですか。	
\\	この 難しい[むずかしい] 漢字[かんじ] は 覚え[おぼえ] なくてもいいですか 。
\\	松田が来るなら僕は行かない。	
\\	松田[まつだ] が 来る[くる] なら 僕は[ぼくは] 行か[いか] ない 。
\\	シカゴへ行くのならバスで行きなさい。	
\\	シカゴ へ 行く[いく] のなら バス で 行き[いき] なさい 。
\\	杉田が来たのなら僕は帰る。	
\\	杉田[すぎた] が 来た[きた] のなら 僕は[ぼくは] 帰る[かえる] 。
\\	そんなに高いのなら買えません。	
\\	そんなに 高い[たかい] のなら 買え[かえ] ません 。
\\	山田さんが好きなら手紙を書いたらどうですか。	
\\	山田[やまだ] さんが 好き[すき] なら 手紙[てがみ] を 書い[かい] たら どう ですか 。
\\	高橋先生ならそのことを知っているでしょう。	
\\	高橋[たかはし] 先生[せんせい] なら その こと を 知って[しって] いるでしょう 。
\\	僕が行きたいのなら誰にも言わずに一人で行くよ。	
\\	僕[ぼく] が 行き[いき] たいのなら 誰[だれ] にも 言わ[いわ] ずに 一人[ひとり] で 行く[いく] よ 。
\\	ニューヨークへ行くならリンカーン・センターが面白いですよ。	
\\	ニューヨーク へ 行[い] くなら リンカーン ・ センター が 面白[おもしろ] いですよ 。
\\	ニューヨークへ行くのならリンカーン・センターへ行きます。	
\\	ニューヨーク へ 行く[いく] のなら リンカーン ・ センター へ 行き[いき] ます 。
\\	テープレコーダーを買うならテープを呉れるはずです。	
\\	テープレコーダー を 買う[かう] なら テープ を 呉れ[くれ] るはずです 。
\\	雨が降ったなら試合はなかったはずです。	
\\	雨[あめ] が 降った[ふった] なら 試合[しあい] はなかったはずです 。
\\	もっとたくさん食べなさい。	
\\	もっと たくさん 食べ[たべ] なさい 。
\\	遊ばないで勉強しなさい。	
\\	遊ば[あそば] ないで 勉強[べんきょう] しなさい 。
\\	早くうちに帰って来なさいよ。	
\\	早く[はやく] うち に 帰って[かえって] 来な[きな] さいよ 。
\\	もう遅いから歯を磨いて寝なさい。	
\\	もう 遅い[おそい] から 歯を[はを] 磨い[みがい] て 寝な[ねな] さい 。
\\	先生、文法を教えて下さい。	
\\	先生[せんせい] 、 文法[ぶんぽう] を 教[おし]え て 下[くだ]さい 。
\\	坂本さんはたばこを吸いませんね。	
\\	坂本[さかもと] さんはたばこを 吸い[すい] ませんね 。
\\	ー今日はいい天気ですね。 ー本当にそうですね。	
\\	ー 今日は[きょうは] いい 天気[てんき] ですね 。 ー 本当[ほんとう] にそうですね 。
\\	あなたは学生ですね。	
\\	あなたは 学生[がくせい] ですね 。
\\	パーティーにいらっしゃいますね?	
\\	パーティー にいらっしゃいますね 
\\	読みなさいね。	
\\	読み[よみ] なさいね 。
\\	読んで下さいね。	
\\	読んで[よんで] 下[くだ]さい ね 。
\\	面白い映画でしたねえ!	
\\	面白い[おもしろい] 映画[えいが] でしたねえ 
\\	もしもし、今晩ね、銀座で飲んでから帰るからね、十一時半頃になるよ。	
\\	もしもし、 今晩[こんばん]ね、 銀座[ぎんざ]で 飲[の]んでから 帰[かえ]るからね、 十一[じゅういち] 時半[じはん] 頃[ごろ]になるよ。
\\	英語が分かるよね。	
\\	英語[えいご] が 分か[わか] るよね 。
\\	あの先生はいい先生ですよね。	
\\	あの 先生[せんせい] はいい 先生[せんせい] ですよね 。
\\	山下君は京大に入れますかね。	
\\	山下[やました] 君[くん] は 京大[きょうだい] に 入れ[いれ] ますかね 。
\\	あの先生はいい先生ですかね。	
\\	あの 先生[せんせい] はいい 先生[せんせい] ですかね 。
\\	私は毎朝六時半に起きます。	
\\	私[わたし] は 毎朝[まいあさ] 六[ろく] 時半[じはん] に 起き[おき] ます 。
\\	僕は四月一日に生まれました。	
\\	僕は[ぼくは] 四月[しがつ] 一日[ついたち] に 生まれ[うまれ] ました 。
\\	今朝は五時半にうちを出ました。	
\\	今朝[けさ] は 五時[ごじ] 半[はん] に うち を 出ま[でま] した 。
\\	来年の夏外国旅行をするつもりです。	
\\	来年[らいねん] の 夏[なつ] 外国旅行[がいこくりょこう] をするつもりです 。
\\	子供が学校に行っている間に手紙を書いた。	
\\	子供[こども] が 学校[がっこう] に 行って[いって] いる 間に[まに] 手紙[てがみ] を 書い[かい] た 。
\\	折口は大学にいる時に今の奥さんと出会った。	
\\	折口[おりくち] は 大学[だいがく] に いる 時に[ときに] 今[いま] の 奥さん[おくさん] と 出会った[であった] 。
\\	子供が学校に行っている間手紙を書いていた。	
\\	子供[こども] が 学校[がっこう] に 行って[いって] いる 間[あいだ] 手紙[てがみ] を 書い[かい] ていた 。
\\	今朝五時半ごろ起きました。	
\\	今朝[けさ] 五時[ごじ] 半[はん] ごろ 起き[おき] ました 。
\\	春に結婚します。	
\\	春[はる] に 結婚[けっこん] します 。
\\	僕は母に手紙をよく書きます。	
\\	僕は[ぼくは] 母[はは] に 手紙[てがみ] をよく 書き[かき] ます 。
\\	父は僕に時計を呉れた。	
\\	父[ちち] は 僕[ぼく] に 時計[とけい] を 呉れ[くれ] た 。
\\	加藤先生はアメリカ人の学生に日本文学を教えている。	
\\	加藤[かとう] 先生[せんせい] は アメリカ 人[じん] の 学生[がくせい] に 日本文[にほんぶん] 学[がく] を 教[おし]え ている 。
\\	僕は妹にお金を少し遣りました。	
\\	僕は[ぼくは] 妹[いもうと] にお 金[かね] を 少し[すこし] 遣り[やり] ました 。
\\	今井さんに電話しましたがいませんでした。	
\\	今井[いまい] さんに 電話[でんわ] しましたが いま せんでした 。
\\	大川先生は学生に色々な辞書を見せた。	
\\	大川[おおかわ] 先生[せんせい] は 学生[がくせい] に 色々[いろいろ] な 辞書[じしょ] を 見せ[みせ] た 。
\\	小さい時お母さんは私によく本を読んで呉れた。	
\\	小さ[ちいさ] い 時[とき] お 母さん[かあさん] は 私[わたし] によく 本[ほん] を 読んで[よんで] 呉れ[くれ] た 。
\\	先生は私を褒めて下さいました。	
\\	先生[せんせい]は 私[わたし]を 褒[ほ]めて 下[くだ]さいました。
\\	ジョンはメアリーのパーティーに来て上げた。	
\\	ジョン は メアリー の パーティー に 来て[きて] 上げ[あげ] た 。
\\	一男は友達に手紙を読まれました。	
\\	一男[かずお] は 友達[ともだち] に 手紙[てがみ] を 読ま[よま] れました 。
\\	秋子は浩にご飯を作らせました。	
\\	秋子[あきこ] は 浩[ひろし] にご 飯[はん] を 作ら[つくら] せました 。
\\	僕は父に車を買って貰った。	
\\	僕は[ぼくは] 父[ちち] に 車[くるま] を 買って[かって] 貰った[もらった] 。
\\	ジェーンは山野先生に生け花を習った。	
\\	ジェーン は 山野[やまの] 先生[せんせい] に 生け花[いけばな] を 習った[ならった] 。
\\	その子はお母さんに叱られました。	
\\	その 子[こ] はお 母さん[かあさん] に 叱ら[しから] れました 。
\\	そのネクタイは誰に貰ったんですか。	
\\	その ネクタイ は 誰[だれ] に 貰った[もらった] んですか 。
\\	僕はアメリカ人に英語を教えて貰った。	
\\	僕は[ぼくは] アメリカ 人[ひと] に 英語[えいご] を 教[おし]え て 貰った[もらった] 。
\\	兄は私に五時間も運転させました。	
\\	兄[あに] は 私[わたし] に 五時[ごじ] 間[かん] も 運転[うんてん] させました 。
\\	ー吉田さんが結婚したそうですよ。 ーそうですか。誰に聞きましたか。	ー
\\	ー 吉田[よしだ] さんが 結婚[けっこん] したそうですよ 。 ー そうですか 。 誰[だれ] に 聞き[きき] ましたか 。
\\	友達が手紙を読んだ。	
\\	友達[ともだち] が 手紙[てがみ] を 読んだ[よんだ] 。
\\	浩がご飯を作った。	
\\	浩[ひろし] がご 飯[はん] を 作った[つくった] 。
\\	父が車を買った。	
\\	父[ちち] が 車[くるま] を 買[か] った 。
\\	ヒルさんは文部省から奨学金を貰いました。	
\\	ヒル さんは 文部省[もんぶしょう] から 奨学金[しょうがくきん] を 貰い[もらい] ました 。
\\	子供は紙に絵を描きました。	
\\	子供[こども] は 紙[かみ] に 絵[え] を 描き[かき] ました 。
\\	ここにあなたの名前と住所を書いて下さい。	
\\	ここにあなたの 名前[なまえ] と 住所[じゅうしょ] を 書い[かい] て 下[くだ]さい 。
\\	ヘリコプターが山の上におりました。	
\\	ヘリコプター が 山[やま] の 上に[うえに] おりました 。
\\	オーバーはハンガーにかけて下さい。	
\\	オーバー は ハンガー にかけて 下[くだ]さい 。
\\	そんな所に立っていると危ないですよ。	
\\	そんな 所[ところ] に 立って[たって] いると 危な[あぶな] いですよ 。
\\	道に絵をかいた。	
\\	道[みち]に 絵[え]をかいた。
\\	道で絵を描いた。	
\\	道[みち] で 絵[え] を 描い[かい] た 。
\\	その寺は松本にある。	
\\	その 寺[てら] は 松本[まつもと] にある 。
\\	僕は部屋にいた。	
\\	僕[ぼく]は 部屋[へや]にいた。
\\	私はデパートへ贈り物を買いに行きました。	
\\	私[わたし] は デパート へ 贈り物[おくりもの] を 買い[かい] に 行き[いき] ました 。
\\	学生が質問をしに来た。	
\\	学生[がくせい] が 質問[しつもん] をしに 来た[きた] 。
\\	そこへ何をしに行くんですか。	
\\	そこへ 何を[なにを] しに 行く[いく] んですか 。
\\	村井さんは昼ご飯を食べにうちへ帰った。	
\\	村井[むらい] さんは 昼[ひる] ご 飯[はん] を 食べ[たべ] に うち へ 帰った[かえった] 。
\\	大山さんは仕事しに行った。	
\\	大山[おおやま] さんは 仕事[しごと] しに 行[い] った 。
\\	学生が質問しに来た。	
\\	学生[がくせい] が 質問[しつもん] しに 来た[きた] 。
\\	僕は子供の写真を撮るためにカメラを買った。	
\\	僕は[ぼくは] 子供[こども] の 写真[しゃしん] を 撮る[とる] ために カメラ を 買[か] った 。
\\	私はアメリカ人の友達と話すために英語を勉強している。	
\\	私[わたし] は アメリカ 人[じん] の 友達[ともだち] と 話す[はなす] ために 英語[えいご] を 勉強[べんきょう] している 。
\\	僕は酒を飲みに行った。	
\\	僕は[ぼくは] 酒[さけ] を 飲み[のみ] に 行[い] った 。
\\	僕は酒を飲むために友達のうちに行った。	
\\	僕は[ぼくは] 酒[さけ] を 飲む[のむ] ために 友達[ともだち] の うち に 行[い] った 。
\\	喫茶店へコーヒーを飲みに行った。	
\\	喫茶店[きっさてん] へ コーヒー を 飲み[のみ] に 行[い] った 。
\\	榎戸さんは今梶山さんのアパートにいます。	
\\	榎戸[えのきど] さんは 今[いま] 梶山[かじやま] さんの アパート に いま す 。
\\	僕のクラスには中国人の学生がいます。	
\\	僕[ぼく] の クラス には 中国人[ちゅうごくじん] の 学生[がくせい] が いま す 。
\\	その本はこの学校の図書館にあります。	
\\	その 本[ほん] はこの 学校[がっこう] の 図書館[としょかん] に あり ます 。
\\	この学校にはプールがない。	
\\	この 学校[がっこう] には プール がない 。
\\	加藤さんは大阪に住んでいます。	
\\	加藤[かとう] さんは 大阪[おおさか] に 住ん[すん] でいます 。
\\	庭に桜の木が立っている。	
\\	庭[にわ] に 桜[さくら] の 木[き] が 立って[たって] いる 。
\\	この作文には文法の間違いがたくさん見られる。	
\\	この 作文[さくぶん] には 文法[ぶんぽう] の 間違い[まちがい] が たくさん 見ら[みら] れる 。
\\	花子がみのるの隣に座っている。	
\\	花子[はなこ] がみのるの 隣[となり] に 座って[すわって] いる 。
\\	私のうちにはテレビがない。	
\\	私[わたし] の うち には テレビ がない 。
\\	スミスさんは今会議室にいる。	
\\	スミス さんは 今[いま] 会議室[かいぎしつ] に いる 。
\\	僕はいつも図書館で勉強する。	
\\	僕は[ぼくは] いつ も 図書館[としょかん] で 勉強[べんきょう] する 。
\\	このラケットは日本では五万円ぐらいだ。	
\\	この ラケット は 日本[にほん] では 五万[ごまん] 円[えん] ぐらいだ 。
\\	そのパーティーはトムの家であった。	
\\	その パーティー は トム の 家[いえ] であった 。
\\	僕は沖縄にいい家を見つけた。	
\\	僕は[ぼくは] 沖縄[おきなわ] にいい 家[いえ] を 見つ[みつ] けた 。
\\	私は札幌で仕事を見つけた。	
\\	私[わたし] は 札幌[さっぽろ] で 仕事[しごと] を 見つ[みつ] けた 。
\\	私は昨日喜多方に行きました。	
\\	私[わたし] は 昨日[きのう] 喜多方[きたかた] に 行き[いき] ました 。
\\	ジムは来年アメリカに帰る。	
\\	ジム は 来年[らいねん] アメリカ に 帰る[かえる] 。
\\	いつか私のうちに来ませんか。	
\\	いつ か 私[わたし] の うち に 来ま[きま] せんか 。
\\	彼らは角のレストランに入った。	
\\	彼[かれ]らは 角[かく]のレストランに 入[はい]った。
\\	黒板に絵をかいた。	
\\	黒板[こくばん]に 絵[え]をかいた。
\\	太郎は花子に本を貸した。	
\\	太郎[たろう] は 花子[はなこ] に 本[ほん] を 貸し[かし] た 。
\\	ボブはメアリーに切符を貰った。	
\\	ボブ は メアリー に 切符[きっぷ] を 貰った[もらった] 。
\\	下田さんは今日のことを忘れたに違いありません。	
\\	下田[しもだ] さんは 今日[きょう] のことを 忘れ[わすれ] たに 違い[ちがい] ありません 。
\\	あの先生の試験は難しいに違いない。	
\\	あの 先生[せんせい] の 試験[しけん] は 難しい[むずかしい] に 違い[ちがい] ない 。
\\	ベイリーさんはテニスが上手に違いありません。	
\\	ベイリー さんは テニス が 上手[じょうず] に 違い[ちがい] ありません 。
\\	あの人は日本人に違いない。	
\\	あの 人[ひと] は 日本人[にほんじん] に 違い[ちがい] ない 。
\\	二人は今ごろハワイで楽しく泳いでいるに違いない。	
\\	二人[ふたり] は 今[いま] ごろ ハワイ で 楽し[たのし] く 泳い[およい] でいるに 違い[ちがい] ない 。
\\	一人で外国へ行くのは大変に違いない。	
\\	一人[ひとり] で 外国[がいこく] へ 行く[いく] のは 大変[たいへん] に 違い[ちがい] ない 。
\\	山口さんは頭がいいに違いない。	
\\	山口[やまぐち] さんは 頭[あたま] がいいに 違い[ちがい] ない 。
\\	あれはトンプソンさんに違いない。	
\\	あれは トンプソン さんに 違い[ちがい] ない 。
\\	日本の経済はよくなったのに違いない。	
\\	日本[にほん] の 経済[けいざい] はよくなったのに 違い[ちがい] ない。
\\	家を買うのはあまり難しくないのに違いない。	
\\	家[いえ] を 買う[かう] のはあまり 難しく[むずかしく] ないのに 違い[ちがい] ない 。
\\	この本は大変読みにくいです。	
\\	この 本[ほん] は 大変[たいへん] 読み[よみ] にくいです 。
\\	このドアは開きにくいですね。	
\\	この ドア は 開き[あき] にくいですね 。
\\	あの人の名前は覚えにくい。	
\\	あの 人[ひと] の 名前[なまえ] は 覚え[おぼえ] にくい 。
\\	あの先生は話しにくいです。	
\\	あの 先生[せんせい] は 話し[はなし] にくいです 。
\\	この靴は走りにくいです。	
\\	この 靴[くつ] は 走り[はしり] にくいです 。
\\	高山さんは日本人にしては大きいです。	
\\	高山[たかやま] さんは 日本人[にほんじん] にしては 大き[おおき] いです 。
\\	ボブは日本語をよく勉強しているにしては下手だ。	
\\	ボブ は 日本語[にほんご] をよく 勉強[べんきょう] しているにしては 下手[へた] だ 。
\\	八月にしては涼しいですね。	
\\	八月[はちがつ] にしては 涼し[すずし] いですね 。
\\	これは日本の車にしては大きいですね。	
\\	これは 日本[にほん] の 車[くるま] にしては 大き[おおき] いですね 。
\\	彼はレスラーだったにしては体が小さい。	
\\	彼は[かれは] レスラー だったにしては 体[からだ] が 小さ[ちいさ] い。
\\	青木さんはアメリカに十年いたにしては英語があまり上手じゃない。	
\\	青木[あおき] さんは アメリカ に 十年[じゅうねん] いたにしては 英語[えいご] が あまり 上手[じょうず] じゃない 。
\\	私はこのアパートにします。	
\\	私[わたし] はこの アパート にします 。
\\	ーあなたは何にしますか。 ー僕はステーキにします。	ー
\\	ー あなたは 何[なん] にしますか 。 ー 僕は[ぼくは] ステーキ にします 。
\\	岡田さんはアメリカの車にしました。	
\\	岡田[おかだ] さんは アメリカ の 車[くるま] にしました 。
\\	今度の旅行は沖縄までにします。	
\\	今度[こんど] の 旅行[りょこう] は 沖縄[おきなわ] までにします 。
\\	これは先生の新聞です。	
\\	これは 先生[せんせい] の 新聞[しんぶん] です 。
\\	これは友達からの手紙だ。	
\\	これは 友達[ともだち] からの 手紙[てがみ] だ 。
\\	これは私のです。	
\\	これは 私[わたし] のです 。
\\	僕は大きいのを買った。	
\\	僕は[ぼくは] 大き[おおき] いのを 買[か] った 。
\\	僕は去年買ったのを使った。	
\\	僕は[ぼくは] 去年[きょねん] 買[か] った のを 使った[つかった] 。
\\	ーどんな車が欲しいですか。 ー小さいのが欲しいです。	ー
\\	ー どんな 車[くるま] が 欲しい[ほしい] ですか 。 ー 小さ[ちいさ] いのが 欲しい[ほしい] です 。
\\	友達がワインを飲みたがったので昨日買ったのを出した。	
\\	友達[ともだち] が ワイン を 飲み[のみ] たがったので 昨日[きのう] 買[か] った のを 出し[だし] た 。
\\	今まで聞いたアイデアの中では田村君が言ったのが一番よさそうだ。	
\\	今まで[いままで] 聞い[きい] た アイデア の 中[なか] では 田村[たむら] 君[くん] が 言った[いった] のが 一番[いちばん] よさそうだ 。
\\	日本語を教えるのは難しいです。	
\\	日本語[にほんご] を 教[おし]え るのは 難しい[むずかしい] です 。
\\	僕は雪子さんがビールを飲むのを見た。	
\\	僕は[ぼくは] 雪子[ゆきこ] さんが ビール を 飲む[のむ] のを 見た[みた] 。
\\	日本へ行くのは簡単です。	
\\	日本[にほん] へ 行く[いく] のは 簡単[かんたん] です 。
\\	私は小林さんがピアノを弾いているのを聞いた。	
\\	私[わたし] は 小林[こばやし] さんが ピアノ を 弾い[ひい] ているのを 聞い[きい] た 。
\\	クラークさんがフランスへ行くのを知っていますか。	
\\	クラーク さんが フランス へ 行く[いく] のを 知って[しって] いますか 。
\\	困ったのは彼が来られないこと。	
\\	困った[こまった] のは 彼[かれ] が 来ら[こら] れないこと 。
\\	僕はひろ子さんがピアノを弾いているのを聞いた。	
\\	僕は[ぼくは] ひろ 子[こ] さんが ピアノ を 弾い[ひい] ているのを 聞い[きい] た 。
\\	日本へ行くことは簡単です。	
\\	日本[にほん] へ 行く[いく] ことは 簡単[かんたん] です 。
\\	クラークさんがフランスへ行くことを知っていますか。	
\\	クラーク さんが フランス へ 行く[いく] ことを 知って[しって] いますか 。
\\	私は自分の体が震えているのを感じた。	
\\	私[わたし] は 自分[じぶん] の 体[からだ] が 震え[ふるえ] ているのを 感じ[かんじ] た 。
\\	ーどうして泣いているの? ーお母さんがいないの。	ー
\\	ー どうして 泣い[ない] ているの 
\\	ー お 母さん[かあさん] がいないの 。
\\	あの人はとても優しいの。	
\\	あの 人[ひと] はとても 優しい[やさしい] の 。
\\	母はまだとても元気なの。	
\\	母[はは] はまだとても 元気[げんき] なの 。
\\	主人はエンジニアなの。	
\\	主人[しゅじん] は エンジニア なの 。
\\	ーどうして食べないの? ーおなか空いてないの。	ー
\\	ー どうして 食べ[たべ] ないの 
\\	ー おなか 空い[すい] てないの 。
\\	ーどうしたの? ー頭が痛いの。	ー
\\	ー どうしたの 
\\	ー 頭[あたま] が 痛い[いたい] の 。
\\	うちの子はまだ小学生なの。	
\\	うち の 子[こ] はまだ 小学生[しょうがくせい] なの 。
\\	大学はプリンストンだったの。	
\\	大学[だいがく] は プリンストン だったの 。
\\	ー何をしているのですか。 ー日本語を勉強しているのです。	ー
\\	ー 何を[なにを] しているのですか 。 ー 日本語[にほんご] を 勉強[べんきょう] しているのです 。
\\	ーどうしてお酒を飲まないんですか。 ー僕はまだ十七なんです。	ー
\\	ー どうして お 酒[さけ] を 飲[の] ま ないんですか 。 ー 僕は[ぼくは] まだ 十[じゅう] 七[しち] なんです 。
\\	僕は今日のパーテイーに行けません。宿題がたくさんあるんです。	
\\	僕は[ぼくは] 今日[きょう] の パーテイー に 行け[いけ] ません 。 宿題[しゅくだい] が たくさん あるんです 。
\\	あなたと結婚したいんです。	
\\	あなたと 結婚[けっこん] したいんです 。
\\	あなたは今何をしていますか。	
\\	あなたは 今[いま] 何を[なにを] していますか 。
\\	今日フットボールがあるんですが一緒に行きませんか。	
\\	今日[きょう] フットボール があるんですが 一緒に[いっしょに] 行き[いき] ませんか 。
\\	先生、困っているんです。助けて下さい。	
\\	先生[せんせい] 、 困って[こまって] いるんです 。 助け[たすけ] て 下[くだ]さい 。
\\	日本語の文法は難しいですが面白いんですよ。	
\\	日本語[にほんご] の 文法[ぶんぽう] は 難しい[むずかしい] ですが 面白い[おもしろい] んですよ 。
\\	お酒をたくさん飲んだので眠くなりました。	
\\	お 酒[さけ] を たくさん 飲ん[のん] だので 眠く[ねむく] なりました 。
\\	宿題がたくさんあるのでパーテイーに行けません。	
\\	宿題[しゅくだい] が たくさん あるので パーテイー に 行け[いけ] ません 。
\\	その本は高かったので買わなかった。	
\\	その 本[ほん] は 高か[たかか] ったので 買わ[かわ] なかった 。
\\	私の部屋は静かなのでよく勉強できます。	
\\	私[わたし] の 部屋[へや] は 静か[しずか] なので よく 勉強[べんきょう] できます 。
\\	ジェーンはまだ中学生なので車を運転出来ない。	
\\	ジェーン はまだ 中学生[ちゅうがくせい] なので 車[くるま] を 運転[うんてん] 出来な[できな] い 。
\\	人がたくさん来るだろうから食べ物をたくさん買っておいた。	
\\	人[ひと] が たくさん 来る[くる] だろうから 食べ物[たべもの] を たくさん 買って[かって] おいた 。
\\	この映画はためになるから行きなさい。	
\\	この 映画[えいが] はためになるから 行き[いき] なさい 。
\\	明日のショーは面白いから是非見に来て下さい。	
\\	明日[あした] の ショー は 面白い[おもしろい] から 是非[ぜひ] 見に[みに] 来て[きて] 下[くだ]さい 。
\\	この本はとても面白いから皆も読んだほうがいいよ。	
\\	この 本[ほん] はとても 面白い[おもしろい] から 皆[みんな] も 読んだ[よんだ] ほうがいいよ 。
\\	いい酒を貰ったから一緒に飲みましょうか。	
\\	いい 酒[さけ] を 貰った[もらった] から 一緒に[いっしょに] 飲み[のみ] ましょうか 。
\\	立原さんが行くから僕も行こう。	
\\	立原[たちはら] さんが 行く[いく] から 僕[ぼく] も 行こ[いこ] う 。
\\	僕がよく知っているから大丈夫です。	
\\	僕[ぼく] がよく 知って[しって] いるから 大丈夫[だいじょうぶ] です 。
\\	台風で家がたくさん倒れた。	
\\	台風[たいふう] で 家[いえ] が たくさん 倒れ[たおれ] た 。
\\	強い風が吹いたから家がたくさん倒れた。	
\\	強い[つよい] 風[かぜ] が 吹い[ふい] たから 家[いえ] が たくさん 倒れ[たおれ] た 。
\\	毎日漢字を勉強しているのによく覚えられません。	
\\	毎日[まいにち] 漢字[かんじ] を 勉強[べんきょう] しているのによく 覚え[おぼえ] られません 。
\\	このステーキは高いのに美味しくない。	
\\	この ステーキ は 高い[たかい] のに 美味しく[おいしく] ない 。
\\	清水さんはゴルフが下手なのに大好きです。	
\\	清水[しみず] さんは ゴルフ が 下手[へた] なのに 大好き[だいすき] です 。
\\	ホールさんはアメリカ人なのに肉が嫌いだ。	
\\	ホール さんは アメリカ 人[じん] なのに 肉[にく] が 嫌い[きらい] だ 。
\\	中学と高校で六年間も英語を勉強したのにまだ英語が話せません。	
\\	中学[ちゅうがく] と 高校[こうこう] で 六年[ろくねん] 間[かん] も 英語[えいご] を 勉強[べんきょう] したのにまだ 英語[えいご] が 話せ[はなせ] ません 。
\\	寒いのにオーバーを着ないで出かけた。	
\\	寒い[さむい] のに オーバー を 着な[きな] いで 出か[でか] けた 。
\\	あの人はピアノが上手なのにめったに弾きません。	
\\	あの 人[ひと] は ピアノ が 上手[じょうず] なのにめったに 弾き[ひき] ません 。
\\	父は九十歳なのにまだ働いています。	
\\	父[ちち] は 九十歳[きゅうじゅっさい] なのにまだ 働い[はたらい] ています 。
\\	ーそんなものいらないよ。 ーせっかくあげると言うのに。	ー
\\	ー そんなものいらないよ 。 ー せっかくあげると 言う[いう] のに 。
\\	難しいけれどして見て下さい。	
\\	難しい[むずかしい] けれどして 見て[みて] 下[くだ]さい 。
\\	あまり美味しくないけれど食べて見ませんか。	
\\	あまり 美味しく[おいしく] ないけれど 食べて[たべて] 見ま[みま] せんか 。
\\	寒いけれど外に出ますか。	
\\	寒い[さむい] けれど 外[そと] に 出ま[でま] すか 。
\\	寒いのに外に出るんですか。	
\\	寒い[さむい] のに 外[そと] に 出る[でる] んですか 。
\\	詰らないけれど読みなさい。	
\\	詰[つま]らないけれど 読[よ]みなさい。
\\	下手だけれどしてもいいですか。	
\\	下手[へた] だけれどしてもいいですか 。
\\	分からないけれどよく考えて見るつもりです。	
\\	分か[わか] らないけれどよく 考え[かんがえ] て 見る[みる] 積り[つもり] です 。
\\	分からないけれどよく考えてみようと思います。	
\\	分か[わか] らないけれどよく 考え[かんがえ] てみようと 思い[おもい] ます 。
\\	父は九十歳だが、まだ働いている。	
\\	父[ちち] は 九十歳[きゅうじゅっさい] だが 、 まだ 働い[はたらい] ている 。
\\	私は日本語の新聞を読むのに辞書を使います。	
\\	私[わたし] は 日本語[にほんご] の 新聞[しんぶん] を 読む[よむ] のに 辞書[じしょ] を 使い[つかい] ます 。
\\	寿喜焼を作るのには何が要りますか。	
\\	寿[す] 喜[き] 焼[やき] を 作る[つくる] のには 何[なに] が 要り[いり] ますか 。
\\	このレポートを書くのに一ヶ月掛かりました。	
\\	この レポート を 書く[かく] のに 一ヶ月[いっかげつ] 掛か[かか] りました 。
\\	会社に行くのにバスと電車を使っている。	
\\	会社[かいしゃ] に 行く[いく] のに バス と 電車[でんしゃ] を 使って[つかって] いる 。
\\	寿喜焼を作るには何が要りますか。	
\\	寿[す] 喜[き] 焼[やき] を 作る[つくる] には 何[なに] が 要り[いり] ますか 。
\\	映画を見に銀座へ行った。	
\\	映画[えいが] を 見に[みに] 銀座[ぎんざ] へ 行[い] った 。
\\	映画を見るのに銀座へ行った。	
\\	映画[えいが] を 見る[みる] のに 銀座[ぎんざ] へ 行[い] った 。
\\	映画を見に行った。	
\\	映画[えいが] を 見に行った[みにいった] 。
\\	生きるために食べる。	
\\	生き[いき] るために 食[た]べる 。
\\	この町で生きていくために月二十万円は必要だ。	
\\	この 町[まち] で 生き[いき] ていくために 月[つき] 二十[にじゅう] 万円[まんえん] は 必要[ひつよう] だ 。
\\	私が中国に行ったのは三年前です。	
\\	私[わたし] が 中国[ちゅうごく] に 行[い] った のは 三年前[さんねんまえ] です 。
\\	このクラスで一番頭がいいのは吉田さんだ。	
\\	この クラス で 一番[いちばん] 頭[あたま] がいいのは 吉田[よしだ] さんだ 。
\\	ここで一番きれいなのは山です。	
\\	ここで 一番[いちばん] きれいなのは 山[やま] です 。
\\	大江さん、あなたがシカゴに行ったのはいつですか。	
\\	大江[おおえ] さん 、 あなたが シカゴ に 行[い] ったのはいつですか 。
\\	一昨日遊びに来たのは秋子さんです。	
\\	一昨日[おととい] 遊び[あそび] に 来た[きた] のは 秋[あき] 子[こ] さんです 。
\\	モーツアルトが大好きになったのは大学一年の時です。	
\\	モーツアルト が 大好き[だいすき] になったのは 大学[だいがく] 一年[いちねん] の 時[とき] です 。
\\	父が嫌いなのはテレビだ。	
\\	父[ちち] が 嫌い[きらい] なのは テレビ だ 。
\\	日本でおいしいのは果物だ。	
\\	日本[にほん] で おいしい のは 果物[くだもの] だ 。
\\	八時にうちに来たのは森田さんだ。	
\\	八[はち] 時に[じに] うち に 来た[きた] のは 森田[もりた] さんだ 。
\\	永井さんが持ってきたのはケーキだ。	
\\	永井[ながい] さんが 持って[もって] きたのは ケーキ だ 。
\\	長田さんがケーキを呉れたのは弟だ。	
\\	長田[ながた] さんが ケーキ を 呉れ[くれ] たのは 弟[おとうと] だ 。
\\	荻原さんがうちに来たのは車でだ。	
\\	荻原[おぎわら] さんが うち に 来た[きた] のは 車[くるま] でだ 。
\\	小栗さんが来たのは出雲からだ。	
\\	小栗[おぐり] さんが 来た[きた] のは 出雲[いずも] からだ 。
\\	長与さんが出発したのは米子だ。	
\\	長与[ながよ] さんが 出発[しゅっぱつ] したのは 米子[よなご] だ 。
\\	岡倉さんが飲んだのは田山さんとだ。	
\\	岡[おか] 倉[くら] さんが 飲ん[のん] だのは 田山[たやま] さんとだ 。
\\	森田さんが一緒に飲んだのは田山さんだ。	
\\	森田[もりた] さんが 一緒に[いっしょに] 飲ん[のん] だのは 田山[たやま] さんだ 。
\\	昨日買ったのはステレオです。	
\\	昨日[きのう] 買[か] った のは ステレオ です 。
\\	ジョンがジェーンに東京で八月に会った。	
\\	ジョン が ジェーン に 東京[とうきょう] で 八月[はちがつ] に 会った[あった] 。
\\	ジェーンに鳥取で八月に会ったのはジョンだ。	
\\	ジェーン に 鳥取[とっとり] で 八月[はちがつ] に 会った[あった] のは ジョン だ 。
\\	ジョンが喜多方で八月に会ったのはジェーンだ。	
\\	ジョン が 喜多方[きたかた] で 八月[はちがつ] に 会った[あった] のは ジェーン だ 。
\\	ジョンがジェーンには四月に会ったのは仙台だ。	
\\	ジョン が ジェーン には 四月[しがつ] に 会った[あった] のは 仙台[せんだい] だ 。
\\	ジョンがジェーンに平泉で会ったのは二月だ。	
\\	ジョンがジェーンに 平泉[ひらいずみ]で 会[あ]ったのは 二月[にがつ]だ。
\\	僕が作っているのがロボットです。	
\\	僕[ぼく] が 作って[つくって] いるのが ロボット です 。
\\	ロボットは私が作っています。	
\\	ロボット は 私[わたし] が 作って[つくって] います 。
\\	本を読むのはたいてい夜だ。	
\\	本[ほん] を 読む[よむ] のはたいてい 夜[よる] だ 。
\\	本を読むのはいいことだ。	
\\	本[ほん] を 読む[よむ] のはいいことだ 。
\\	大抵夜本を読む。	
\\	大抵[たいてい] 夜[よる] 本[ほん] を 読む[よむ]。
\\	石田先生は英語でお話になりました。	
\\	石田[いしだ] 先生[せんせい] は 英語[えいご] でお 話[はなし] になりました 。
\\	僕は昨日山崎先生にお会いした。	
\\	僕は[ぼくは] 昨日[きのう] 山崎[やまざき] 先生[せんせい] にお 会[あ] い した 。
\\	今村さんはゴルフがお好きです。	
\\	今村[いまむら] さんは ゴルフ がお 好き[すき] です 。
\\	お飲み物は何が宜しいですか。	
\\	お 飲[の]み 物[もの]は 何[なに]が 宜[よろ]しいですか。
\\	田村先生は今とてもお忙しい。	
\\	田村[たむら] 先生[せんせい] は 今[いま] とてもお 忙し[いそがし] い 。
\\	毎日お暑いですね。	
\\	毎日[まいにち]お 暑[あつ]いですね。
\\	上田先生はテニスがお上手です。	
\\	上田[うえだ] 先生[せんせい] は テニス がお 上手[じょうず] です 。
\\	宮本さんの奥様はとてもおきれいだ。	
\\	宮本[みやもと] さんの 奥様[おくさま] はとてもおきれい だ 。
\\	野村さんからお電話がありました。	
\\	野村[のむら] さんからお 電話[でんわ] がありました 。
\\	お昼ご飯はもう食べましたか。	
\\	お 昼[ひる] ご 飯[はん] はもう 食べ[たべ] ましたか 。
\\	先生がお手紙を下さいました。	
\\	先生[せんせい] がお 手紙[てがみ] を 下[くだ]さい ました 。
\\	明日お電話を差し上げます。	
\\	明日[みょうにち] お 電話[でんわ] を 差し上げ[さしあげ] ます 。
\\	お野菜が高くなりましたねえ。	
\\	お 野菜[やさい] が 高く[たかく] なりましたねえ 。
\\	私は日本語を勉強している。	
\\	私[わたし] は 日本語[にほんご] を 勉強[べんきょう] している 。
\\	前田さんは昨日車を買った。	
\\	前田[まえだ] さんは 昨日[きのう] 車[くるま] を 買[か] った 。
\\	何を飲みますか。	
\\	何を[なにを] 飲み[のみ] ますか 。
\\	僕は英語が分かる。	
\\	僕は[ぼくは] 英語[えいご] が 分か[わか] る 。
\\	ジムはベスに電話した。。	
\\	ジム は ベス に 電話[でんわ] した 。。
\\	僕は昨日田中さんと会った。	
\\	僕は[ぼくは] 昨日[きのう] 田中[たなか] さんと 会った[あった] 。
\\	父は私に酒を飲ませた。	
\\	父[ちち] は 私[わたし] に 酒[さけ] を 飲[の] ま せた 。
\\	友達は私に急な坂を上らせた。	
\\	友達[ともだち] は 私[わたし] に 急な[きゅうな] 坂[さか] を 上ら[のぼら] せた 。
\\	日本語は知りません。	
\\	日本語[にほんご] は 知り[しり] ません 。
\\	その本はもう読みました。	
\\	その 本[ほん] はもう 読み[よみ] ました 。
\\	僕は五番街を歩いた。	
\\	僕は[ぼくは] 五番街[ごばんがい] を 歩い[あるい] た 。
\\	公園を通って帰りましょう。	
\\	公園[こうえん] を 通って[とおって] 帰り[かえり] ましょう 。
\\	鶴が湖の上を飛んでいます。	
\\	鶴[つる] が 湖[みずうみ] の 上[うえ] を 飛ん[とん] でいます 。
\\	日本では車は道の左側を走ります。	
\\	日本[にほん] では 車[くるま] は 道[みち] の 左側[ひだりがわ] を 走り[はしり] ます 。
\\	そこの交差点を右に曲がって下さい。	
\\	そこの 交差点[こうさてん] を 右[みぎ] に 曲がって[まがって] 下[くだ]さい 。
\\	ドンは東京タワーを上った。	
\\	ドンは 東京[とうきょう]タワーを 上[のぼ]った。
\\	ドンは東京タワーに上った。	
\\	ドン は 東京[とうきょう] タワー に 上[のぼ]った 。
\\	私は図書館で勉強した。	
\\	私[わたし] は 図書館[としょかん] で 勉強[べんきょう] した 。
\\	ジョンは川で泳いだ。	
\\	ジョン は 川[かわ] で 泳い[およい] だ 。
\\	ジョンはプールが嫌いだから大抵川で泳ぐ。	
\\	ジョン は プール が 嫌い[きらい] だから 大抵[たいてい] 川[かわ] で 泳ぐ[およぐ] 。
\\	ジョンは川を泳いで逃げた。	
\\	ジョン は 川[かわ] を 泳い[およい] で 逃げ[にげ] た 。
\\	私は朝七時半にうちを出ます。	
\\	私[わたし] は 朝[あさ] 七時[しちじ] 半[はん] に うち を 出ま[でま] す 。
\\	汽車がトンネルを出た。	
\\	汽車[きしゃ] が トンネル を 出た[でた] 。
\\	バスを降りた時友達に会った。	
\\	バス を 降り[おり] た 時[とき] 友達[ともだち] に 会った[あった] 。
\\	日本を離れて外国で暮らしている。	
\\	日本[にほん] を 離れ[はなれ] て 外国[がいこく] で 暮らし[くらし] ている 。
\\	私は今朝八時ごろうちを出た。	
\\	私[わたし] は 今朝[けさ] 八[はち] 時[じ] ごろ うち を 出た[でた] 。
\\	昨日はうちから外に出なかった。	
\\	昨日[きのう] は うち から 外[そと] に 出な[でな] かった 。
\\	次郎は父の死を悲しみました。	
\\	次郎[じろう] は 父[ちち] の 死[し] を 悲しみ[かなしみ] ました 。
\\	次郎は父が死んだことを悲しんだ。	
\\	次郎[じろう] は 父[ちち] が 死ん[しん] だことを 悲し[かなし] んだ 。
\\	僕は浩の大学入学を喜んだ。	
\\	僕は[ぼくは] 浩[ひろし] の 大学入学[だいがくにゅうがく] を 喜んだ[よろこんだ] 。
\\	ヨーロッパ人はまた戦争が起きることを恐れている。	
\\	ヨーロッパ 人[じん]はまた 戦争[せんそう]が 起[お]きることを 恐[おそ]れている。
\\	信子は大津での一年を懐かしんだ。	
\\	信子[のぶこ]は 大津[おおつ]での 一年[いちねん]を 懐[なつ]かしんだ。
\\	林は英語が出来ないことを悩んでいる。	
\\	林[はやし] は 英語[えいご] が 出来な[できな] いことを 悩ん[なやん] でいる 。
\\	次郎は父が死んだので悲しんだ。	
\\	次郎[じろう] は 父[ちち] が 死ん[しん] だので 悲し[かなし] んだ 。
\\	僕は浩が大学に入学したので喜んだ。	
\\	僕は[ぼくは] 浩[ひろし] が 大学[だいがく] に 入学[にゅうがく] したので 喜んだ[よろこんだ] 。
\\	誰でも親の死を悲しむ。	
\\	誰[だれ] でも 親[おや] の 死[し] を 悲し[かなし] む 。
\\	次郎は父の死を悲しがった。	
\\	次郎[じろう] は 父[ちち] の 死[し] を 悲し[かなし] がった 。
\\	日本には大学が多いです。	
\\	日本[にほん] には 大学[だいがく] が 多い[おおい] です 。
\\	京都にはお寺が多いです。	
\\	京都[きょうと] にはお 寺[てら] が 多い[おおい] です 。
\\	ロスさんの作文は間違いが多いです。	
\\	ロス さんの 作文[さくぶん] は 間違い[まちがい] が 多い[おおい] です 。
\\	一月には雪が多い。	
\\	一月[いちがつ] には 雪[ゆき] が 多い[おおい] 。
\\	お寺が多い町は京都です。	
\\	お 寺[てら] が 多い[おおい] 町[まち] は 京都[きょうと] です 。
\\	多くの学生が毎年アジアから日本の大学に来る。	
\\	多く[おおく] の 学生[がくせい] が 毎年[まいとし] アジア から 日本[にほん] の 大学[だいがく] に 来る[くる] 。
\\	この町は車が少ないです。	
\\	この 町[まち] は 車[くるま] が 少ない[すくない] です 。
\\	日本語の新聞が読める学生は少ない。	
\\	日本語[にほんご] の 新聞[しんぶん] が 読め[よめ] る 学生[がくせい] は 少ない[すくない] 。
\\	この部屋は学生が大勢いる。	
\\	この 部屋[へや] は 学生[がくせい] が 大勢[おおぜい] いる 。
\\	この部屋には大勢の学生がいる。	
\\	この 部屋[へや] には 大勢[おおぜい] の 学生[がくせい] がいる 。
\\	学生が大勢来た。	
\\	学生[がくせい] が 大勢[おおぜい] 来た[きた] 。
\\	この部屋は学生がたくさんいる。	
\\	この 部屋[へや] は 学生[がくせい] が たくさん いる 。
\\	この部屋は机がたくさんある。	
\\	この 部屋[へや] は 机[つくえ] が たくさん ある。
\\	今年はたくさん雪が降った。	
\\	今年[ことし] は たくさん 雪[ゆき] が 降った[ふった] 。
\\	学生の多くは男だ。	
\\	学生[がくせい] の 多く[おおく] は 男[おとこ] だ 。
\\	明日パーテイーをするのでビールを買って置きました。	
\\	明日[あした] パーテイー をするので ビール を 買って[かって] 置き[おき] ました 。
\\	今日は昼に客が来るので昼ご飯を早目に食べて置いた。	
\\	今日は[きょうは] 昼[ひる] に 客[きゃく] が 来る[くる] ので 昼[ひる] ご 飯[はん] を 早目[はやめ] に 食べて[たべて] 置いた[おいた] 。
\\	来年日本へ行くから少し日本語を勉強して置きます。	
\\	来年[らいねん] 日本[にほん] へ 行く[いく] から 少し[すこし] 日本語[にほんご] を 勉強[べんきょう] して 置き[おき] ます 。
\\	信子は疲れているからしばらく寝させて置こう。	
\\	信子[のぶこ]は 疲[つか]れているからしばらく 寝[ね]させて 置[お]こう。
\\	飲みたいと言うのなら好きなだけ飲ませて置きなさい。	
\\	飲み[のみ] たいと 言う[いう] のなら 好き[すき] なだけ 飲[の] ま せて 置き[おき] なさい 。
\\	私はボブにビールを飲ませて置いた。	
\\	私[わたし] は ボブ に ビール を 飲[の] ま せて 置いた[おいた] 。
\\	加賀先生はもうお帰りになりました。	
\\	加賀[かが] 先生[せんせい] はもうお 帰り[かえり] になりました 。
\\	この本をもうお読みになりましたか。	
\\	この 本[ほん] をもうお 読み[よみ] になりましたか 。
\\	山部先生は私の名前をお忘れになった。	
\\	山部[やまべ] 先生[せんせい] は 私[わたし] の 名前[なまえ] をお 忘れ[わすれ] になった 。
\\	小山先生は四月にご結婚なさいます。	
\\	小山[こやま] 先生[せんせい] は 四月[しがつ] にご 結婚[けっこん] なさいます 。
\\	森本さんにお電話なさいましたか。	
\\	森本[もりもと] さんにお 電話[でんわ] なさいましたか 。
\\	佐藤先生はもう帰られましたか。	
\\	佐藤[さとう] 先生[せんせい] はもう 帰ら[かえら] れましたか 。
\\	この本をもう読まれましたか。	
\\	この 本[ほん] をもう 読ま[よま] れましたか 。
\\	私は先生のスーツケースをお持ちしました。	
\\	私[わたし] は 先生[せんせい] の スーツケース をお 持ち[もち] しました 。
\\	そのことは僕がお話しましょう。	
\\	そのことは 僕[ぼく] がお 話し[はなし] ましょう 。
\\	山村君は先生に本をお借りした。	
\\	山村[やまむら] 君[くん] は 先生[せんせい] に 本[ほん] をお 借り[かり] した 。
\\	サンプルは明日うちのセールスマンがお届けします。	
\\	サンプル は 明日[みょうにち] うち の セールスマン がお 届け[とどけ] します 。
\\	植村さんをご紹介致します。	
\\	植村[うえむら] さんをご 紹介致し[しょうかいいたし] ます 。
\\	明日お電話致します。	
\\	明日[あした] お 電話[でんわ] 致し[いたし] ます 。
\\	源氏物語をやっと読み終わりました。	
\\	源氏物語[げんじものがたり] をやっと 読み終わ[よみおわ] りました 。
\\	やっと論文を書き終わった。	
\\	やっと 論文[ろんぶん] を 書き[かき] 終わ[おわ] った 。
\\	恰度ご飯を食べ終わったところです。	
\\	恰度[ちょうど] ご 飯[はん] を 食べ[たべ] 終わ[おわ] ったところです 。
\\	一郎は花子に騙されました。	
\\	一郎[いちろう] は 花子[はなこ] に 騙さ[だまさ] れました 。
\\	ジェーンはフレッドに夜遅くアパートに来られた。	
\\	ジェーン は フレッド に 夜遅く[よるおそく] アパート に 来ら[こら] れた 。
\\	私は弟にケーキを食べられました。	
\\	私[わたし] は 弟[おとうと] に ケーキ を 食べ[たべ] られました 。
\\	鈴木先生は日本へ帰られた。	
\\	鈴木[すずき] 先生[せんせい] は 日本[にほん] へ 帰ら[かえら] れた 。
\\	このビルは二年前に建てられた。	
\\	この ビル は 二年前[にねんまえ] に 建て[たて] られた 。
\\	この本は1965年にアメリカで出版された。	
\\	この 本[ほん] は 
\\	年[ねん] に アメリカ で 出版[しゅっぱん] された 。
\\	私は二年前妻にしなれた。	
\\	私[わたし] は 二年前[にねんまえ] 妻[つま] にしなれた 。
\\	原田さんは奥さんに高いコートを買われた。	
\\	原田[はらだ]さんは 奥[おく]さんに 高[たか]いコートを 買[か]われた。
\\	林先生は日本の大学のことを話された。	
\\	林[はやし] 先生[せんせい] は 日本[にほん] の 大学[だいがく] の こと を 話さ[はなさ] れた 。
\\	花子は一郎を騙した。	
\\	花子[はなこ] は 一郎[いちろう] を 騙し[だまし] た 。
\\	ジョンは先生に質問をした。	
\\	ジョン は 先生[せんせい] に 質問[しつもん] をした 。
\\	先生はジョンに質問をされた。	
\\	先生[せんせい] は ジョン に 質問[しつもん] を された 。
\\	この絵はピカソによって書かれた。	
\\	この 絵[え]はピカソによって 書[か]かれた。
\\	電話はベルによって発明された。	
\\	電話[でんわ] は ベル によって 発明[はつめい] された 。
\\	私は学生から日本の大学のことを聞かれた。	
\\	私[わたし] は 学生[がくせい] から 日本[にほん] の 大学[だいがく] の こと を 聞か[きか] れた 。
\\	高橋さんは皆から尊敬されている。	
\\	高橋[たかはし] さんは 皆[みんな] から 尊敬[そんけい] されている 。
\\	太郎は次郎のビールを飲んだ。	
\\	太郎[たろう] は 次郎[じろう] の ビール を 飲ん[のん] だ 。
\\	次郎は太郎にビールを飲まれた。	
\\	次郎[じろう] は 太郎[たろう] に ビール を 飲[の] ま れた 。
\\	田中さんは美人に横に座られてニコニコしている。	
\\	田中[たなか] さんは 美人[びじん] に 横[よこ] に 座ら[すわら] れて ニコニコ している 。
\\	僕はトムにメアリーに電話された。	
\\	僕は[ぼくは] トム に メアリー に 電話[でんわ] された 。
\\	先生は刺身が食べられる。	
\\	先生[せんせい] は 刺身[さしみ] が 食べ[たべ] られる 。
\\	先生は刺身を食べられた。	
\\	先生[せんせい] は 刺身[さしみ] を 食べ[たべ] られた 。
\\	先生は学生に刺身を食べられた。	
\\	先生[せんせい] は 学生[がくせい] に 刺身[さしみ] を 食べ[たべ] られた 。
\\	私は日本語が読めます。	
\\	私[わたし] は 日本語[にほんご] が 読め[よめ] ます 。
\\	この水が飲めない。	
\\	この 水[みず] が 飲め[のめ] ない 。
\\	渡辺さんは刺身が食べられる。	
\\	渡辺[わたなべ] さんは 刺身[さしみ] が 食べ[たべ] られる 。
\\	渡邊さんはテニスが出来る。	
\\	渡邊[わたなべ] さんは テニス が 出来る[できる] 。
\\	この字は読めない。	
\\	この 字[じ] は 読め[よめ] ない 。
\\	僕は英語を話します。	
\\	僕は[ぼくは] 英語[えいご] を 話し[はなし] ます 。
\\	僕は英語が話せます。	
\\	僕は[ぼくは] 英語[えいご] が 話せ[はなせ] ます 。
\\	私はやめようと思えばいつでも今の仕事を止められる。	
\\	私[わたし]はやめようと 思[おも]えばいつでも 今[いま]の 仕事[しごと]を 止[や]められる。
\\	僕はやっとの思いで自分を抑えられた。	
\\	僕は[ぼくは] やっとの 思い[おもい] で 自分[じぶん] を 抑え[おさえ] られた 。
\\	僕は将棋が出来る。	
\\	僕は[ぼくは] 将棋[しょうぎ] が 出来る[できる] 。
\\	僕はあの公園を夜一人で歩けない。	
\\	僕は[ぼくは] あの 公園[こうえん] を 夜[よる] 一人[ひとり] で 歩け[あるけ] ない 。
\\	この道を靴をはかずに歩けますか。	
\\	この 道[みち] を 靴[くつ] をはかずに 歩け[あるけ] ますか 。
\\	今日はうちを出られません。	
\\	今日は[きょうは] うち を 出ら[でら] れません 。
\\	伊藤さんはアメリカへ行くらしいです。	
\\	伊藤[いとう] さんは アメリカ へ 行く[いく] らしいです 。
\\	山本さんはもう帰ったらしいです。	
\\	山本[やまもと] さんはもう 帰った[かえった] らしいです 。
\\	あの学校の入学試験は難しいらしい。	
\\	あの 学校[がっこう] の 入学試験[にゅうがくしけん] は 難しい[むずかしい] らしい 。
\\	この辺りはとても静からしいです。	
\\	この 辺り[あたり] はとても 静か[しずか] らしいです 。
\\	その話は本当らしい。	
\\	その 話[はなし] は 本当[ほんとう] らしい 。
\\	ここは学校らしい。	
\\	ここは 学校[がっこう] らしい 。
\\	中村さんはパーテイーに行かないらしい。	
\\	中村[なかむら] さんは パーテイー に 行か[いか] ないらしい 。
\\	メアリーはビルが好きじゃないらしい。	
\\	メアリー は ビル が 好き[すき] じゃないらしい 。
\\	小林さんは男らしい。	
\\	小林[こばやし] さんは 男[おとこ] らしい 。
\\	僕は男らしい人が好きだ。	
\\	僕は[ぼくは] 男[おとこ] らしい 人[ひと] が 好き[すき] だ 。
\\	斉藤さんは男らしくない。	
\\	斉藤[さいとう] さんは 男[おとこ] らしくない 。
\\	加藤さんは男じゃないらしい。	
\\	加藤[かとう] さんは 男[おとこ] じゃないらしい 。
\\	吉田さんが食べたステーキは高かったです。	
\\	吉田[よしだ] さんが 食べた[たべた] ステーキ は 高か[たかか] ったです 。
\\	ステーキがおいしいレストランを知らないか。	
\\	ステーキ が おいしい レストラン を 知ら[しら] ないか 。
\\	日本語を教えている先生は山田先生です。	
\\	日本語[にほんご] を 教[おし]え ている 先生[せんせい] は 山田[やまだ] 先生[せんせい] です 。
\\	テニスが上手な人を教えて下さい。	
\\	テニス が 上手[じょうず] な 人[ひと] を 教[おし]え て 下[くだ]さい 。
\\	お父さんが医者の学生は三人います。	
\\	お 父さん[とうさん] が 医者[いしゃ] の 学生[がくせい] は 三人[さんにん] います 。
\\	私がいた町は病院がなかった。	
\\	私[わたし] がいた 町[まち] は 病院[びょういん] がなかった 。
\\	道子が行く学校は徳島にあります。	
\\	道子[みちこ] が 行く[いく] 学校[がっこう] は 徳島[とくしま] にあります 。
\\	佐々木さんがあなたの写真を撮ったカメラはこれですか。	
\\	佐々木[ささき] さんがあなたの 写真[しゃしん] を 撮った[とった] カメラ はこれですか 。
\\	山口さんはステーキを食べました。	
\\	山口[やまぐち] さんは ステーキ を 食べ[たべ] ました 。
\\	そのステーキは美味しかったです。	
\\	その ステーキ は 美味しか[おいしか] ったです 。
\\	私が日本語を教えて上げたブラウンさんはよく勉強する。	
\\	私[わたし] が 日本語[にほんご] を 教[おし]え て 上げ[あげ] た ブラウン さんはよく 勉強[べんきょう] する 。
\\	よく働く日本人は嫌われる。	
\\	よく 働く[はたらく] 日本人[にほんじん] は 嫌わ[きらわ] れる 。
\\	このうちの広さはちょうどいいです。	
\\	このうちの 広さ[ひろさ] はちょうどいいです 。
\\	富士山の高さはどのぐらいですか。	
\\	富士山[ふじさん] の 高さ[たかさ] はどのぐらいですか 。
\\	日本語の難しさがよく分かりました。	
\\	日本語[にほんご] の 難し[むずかし] さがよく 分か[わか] りました 。
\\	アメリカのよさはパイオニアスピリットでしょう。	
\\	アメリカ のよさは パイオニアスピリット でしょう 。
\\	楽しみにしています。	
\\	楽しみ[たのしみ] にしています 。
\\	王様の耳は驢馬の耳だ。	
\\	王様[おうさま] の 耳[みみ] は 驢馬[ろば] の 耳[みみ] だ 。
\\	神様を信じますか。	
\\	神様[かみさま] を 信じ[しんじ] ますか 。
\\	花子ちゃん、今夜はお月様もお星様もきれいねえ。	
\\	花子[はなこ] ちゃん 、 今夜[こんや] はお 月[つき] 様[さま] もお 星様[ほしさま] も きれい ねえ 。
\\	お母様、どこにいらっしゃるの?	
\\	お 母[かあ] 様[さま] 、 どこにいらっしゃるの 
\\	お客様が見えたよ。	
\\	お 客様[きゃくさま] が 見え[みえ] たよ 。
\\	ご馳走様。	
\\	ご 馳走[ちそう] 様[さま] 。
\\	お世話ようになりました。	
\\	お 世話[せわ] ようになりました 。
\\	お気の毒様でした。	
\\	お 気の毒[きのどく] 様[さま] でした 。
\\	お疲れ様。	
\\	お 疲れ[つかれ] 様[さま] 。
\\	おかげ様で元気です。	
\\	おかげ 様[さま] で 元気[げんき] です 。
\\	井上さんは娘を大学へ行かせました。	
\\	井上[いのうえ] さんは 娘[むすめ] を 大学[だいがく] へ 行か[いか] せました 。
\\	父が妹にピアノを習わせた。	
\\	父[ちち] が 妹[いもうと] に ピアノ を 習わ[ならわ] せた 。
\\	ひさ子は私を困らせた。	
\\	ひさ 子[こ] は 私[わたし] を 困ら[こまら] せた 。
\\	このゲームはジムに勝たせようと思う。	
\\	この ゲーム は ジム に 勝た[かた] せようと 思う[おもう] 。
\\	友達は僕にチップを払わせた。	
\\	友達[ともだち] は 僕[ぼく] に チップ を 払わ[はらわ] せた 。
\\	木村さんは子供達に好きなだけアイスクリームを食べさせた。	
\\	木村[きむら] さんは 子供達[こどもたち] に 好き[すき] なだけ アイスクリーム を 食べ[たべ] させた 。
\\	僕は冷蔵庫でミルクを凍らせた	
\\	僕は[ぼくは] 冷蔵庫[れいぞうこ] で ミルク を 凍ら[こおら] せた
\\	それは私にさせて下さい。	
\\	それは 私[わたし] にさせて 下[くだ]さい 。
\\	父は私を無理やりパーテイーへ行かせた。	
\\	父[ちち] は 私[わたし] を 無理[むり] やり パーテイー へ 行か[いか] せた 。
\\	僕はハンカチを湿らせた。	
\\	僕[ぼく]はハンカチを 湿[しめ]らせた。
\\	僕は嫌がる息子を無理やり泳がせた。	
\\	僕は[ぼくは] 嫌が[いやが] る 息子[むすこ] を 無理[むり] やり 泳が[およが] せた 。
\\	私は息子に好きなだけ泳がせた。	
\\	私[わたし] は 息子[むすこ] に 好き[すき] なだけ 泳が[およが] せた 。
\\	秋子は勉強はもうしたくないと言ったが私は彼女を大学に行かせることにした。	
\\	秋[あき] 子[こ] は 勉強[べんきょう] はもうしたくないと 言った[いった] が 私[わたし] は 彼女[かのじょ] を 大学[だいがく] に 行か[いか] せることにした 。
\\	秋子はもっと勉強したいと言ったので、私は彼女に大学に行かせることにした。	
\\	秋[あき] 子[こ] はもっと 勉強[べんきょう] したいと 言った[いった] ので 、 私[わたし] は 彼女[かのじょ] に 大学[だいがく] に 行か[いか] せることにした 。
\\	一郎は雪子にビールを飲ませた。	
\\	一郎[いちろう] は 雪子[ゆきこ] に ビール を 飲[の] ま せた 。
\\	一郎は雪子に無理やりビールを飲ませた。	
\\	一郎[いちろう] は 雪子[ゆきこ] に 無理[むり] やり ビール を 飲[の] ま せた 。
\\	僕は春子をパーテイーに行かせて遣った。	
\\	僕は[ぼくは] 春子[はるこ] を パーテイー に 行か[いか] せて 遣った[やった] 。
\\	父は私に酒を飲ませて呉れた。	
\\	父[ちち] は 私[わたし] に 酒[さけ] を 飲[の] ま せて 呉れ[くれ] た 。
\\	ぼくは林さんに果物を食べさすつもりだ。	
\\	ぼくは 林[はやし] さんに 果物[くだもの] を 食べ[たべ] さすつもりだ 。
\\	僕は清水に果物を食べさせるつもりだ。	
\\	僕は[ぼくは] 清水[しみず] に 果物[くだもの] を 食べ[たべ] させるつもりだ 。
\\	父は私を歩いて帰らせた。	
\\	父[ちち] は 私[わたし] を 歩い[あるい] て 帰ら[かえら] せた 。
\\	僕は父に歩いて帰らせられた。	
\\	僕は[ぼくは] 父[ちち] に 歩い[あるい] て 帰ら[かえら] せられた 。
\\	和夫は夏子に酒を飲ませた。	
\\	和夫[かずお] は 夏子[なつこ] に 酒[さけ] を 飲[の] ま せた 。
\\	夏子は和夫に酒を飲まれられた。	
\\	夏子[なつこ] は 和夫[かずお] に 酒[さけ] を 飲[の] ま れられた 。
\\	せっかく会いに行ったのに友達はいませんでした。	
\\	せっかく 会[あ] いに 行[い] ったのに 友達[ともだち] はいませんでした 。
\\	せっかくいい大学に入ったのだからよく勉強するつもりだ。	
\\	せっかくいい 大学[だいがく] に 入[はい] ったのだからよく 勉強[べんきょう] するつもりだ 。
\\	せっかくの日曜日なのに働いた。	
\\	せっかくの 日曜日[にちようび] なのに 働い[はたらい] た 。
\\	せっかくアメリカまで行ったのにニューヨークに行けなくて残念だった。	
\\	せっかく アメリカ まで 行[い] った のに ニューヨーク に 行け[いけ] なくて 残念[ざんねん] だった 。
\\	せっかく日本語を三年間も勉強したのだから、是非一度日本へ行ってみたいと思います。	
\\	せっかく 日本語[にほんご] を 三年[さんねん] 間[かん] も 勉強[べんきょう] したのだから 、 是非一度[ぜひいちど] 日本[にほん] へ 行って[いって] みたいと 思い[おもい] ます 。
\\	せっかくの旅行が病気でだめになりました。	
\\	せっかくの 旅行[りょこう] が 病気[びょうき] でだめになりました 。
\\	せっかくですが今日は忙しくて行けません。	
\\	せっかくですが 今日は[きょうは] 忙し[いそがし] くて 行け[いけ] ません 。
\\	せっかくのチャンスだからデートしたらどうですか。	
\\	せっかくの チャンス だから デート したら どう ですか 。
\\	せっかくですから、遠慮なく頂きます。	
\\	せっかくですから 、 遠慮[えんりょ] なく 頂き[いただき] ます 。
\\	せっかく近くまで来たのだから寄りました。	
\\	せっかく 近[ちか] くまで 来[き] たのだから 寄[よ] りました 。
\\	わざわざ新しい車を買った。	
\\	わざわざ 新[あたら] しい 車[くるま] を 買[か] った 。
\\	わざわざ持って来てくれてありがとう。	
\\	わざわざ 持って[もって] 来て[きて] くれてありがとう 。
\\	今日はテニスもしたし、映画も見ました。	
\\	今日は[きょうは] テニス もしたし 、 映画[えいが] も 見ま[みま] した 。
\\	ここは夏は暑いし、冬は寒い。	
\\	ここは 夏[なつ]は 暑[あつ]いし、 冬[ふゆ]は 寒[さむ]い。
\\	このアパートはきれいだし、安いです。	
\\	この アパート はきれい だし 、 安い[やすい] です 。
\\	遊びたいんですが明日は試験だし遊べません。	
\\	遊び[あそび] たいんですが 明日[あした] は 試験[しけん] だし 遊べ[あそべ] ません 。
\\	仕事もあったし、結婚も出来たし、とても嬉しいです。	
\\	仕事[しごと] もあったし 、 結婚[けっこん] も 出来た[できた] し 、 とても 嬉し[うれし] いです 。
\\	今日は天気もいいし、どこかへ行きましょうか。	
\\	今日は[きょうは] 天気[てんき] もいいし 、 どこかへ 行き[いき] ましょうか 。
\\	あの人は美人だし、頭もいい。	
\\	あの 人[ひと] は 美人[びじん] だし 、 頭[あたま] もいい 。
\\	切符は買ってあるし、是非見に行きましょう。	
\\	切符[きっぷ] は 買って[かって] あるし 、 是非[ぜひ] 見に行き[みにいき] ましょう 。
\\	ー明日映画に行きませんか。 ーえっ、明日ですか。明日は試験があるし...	ー
\\	ー 明日[あした] 映画[えいが] に 行き[いき] ませんか 。 ー えっ 、 明日[あした] ですか 。 明日[あした] は 試験[しけん] があるし 
\\	今日は天気もいいですし、どこかへ行きましょうか。	
\\	今日は[きょうは] 天気[てんき] もいいですし 、 どこかへ 行き[いき] ましょうか 。
\\	戸田さんしかたばこを吸いません。	
\\	戸田[とだ] さんしかたばこを 吸い[すい] ません 。
\\	私は日本語しか知りません。	
\\	私[わたし] は 日本語[にほんご] しか 知り[しり] ません 。
\\	それは江口さんにしか話していません。	
\\	それは 江口[えぐち] さんにしか 話し[はなし] ていません 。
\\	私はご飯を一杯しか食べなかった。	
\\	私[わたし] はご 飯[はん] を 一杯[いっぱい] しか 食べ[たべ] なかった 。
\\	パーテイーには学生しか来なかった。	
\\	パーテイー には 学生[がくせい] しか 来な[こな] かった 。
\\	山崎さんはサラダしか食べなかった。	
\\	山崎[やまざき] さんは サラダ しか 食べ[たべ] なかった 。
\\	僕は日曜日しか来られません。	
\\	僕は[ぼくは] 日曜日[にちようび] しか 来ら[こら] れません 。
\\	この本はこの図書館にしかありません。	
\\	この 本[ほん] はこの 図書館[としょかん] にしかありません 。
\\	そこは車でしか行けない。	
\\	そこは 車[くるま] でしか 行け[いけ] ない 。
\\	私は山崎さんとしか話しをしない。	
\\	私[わたし] は 山崎[やまさき] さんとしか 話し[はなし] をしない 。
\\	この学校は学生が百人しかいない。	
\\	この 学校[がっこう] は 学生[がくせい] が 百人[ひゃくにん] しかいない 。
\\	森さんだけ来た。	
\\	森[もり] さんだけ 来た[きた] 。
\\	阿部さんしか来なかった。	
\\	阿部[あべ] さんしか 来な[こな] かった 。
\\	安部さんだけ来なかった。	
\\	安部[あべ] さんだけ 来な[こな] かった 。
\\	私の家から学校までは車で五分しか掛からない。	
\\	私[わたし] の 家[いえ] から 学校[がっこう] までは 車[くるま] で 五分[ごふん] しか 掛か[かか] らない 。
\\	池田さんはビールばかり飲んだ。	
\\	池田[いけだ] さんは ビール ばかり 飲ん[のん] だ 。
\\	橋本さんはビールしか飲まなかった。	
\\	橋本[はしもと] さんは ビール しか 飲[の] ま なかった 。
\\	山下さんはビールだけ飲んだ。	
\\	山下[やました] さんは ビール だけ 飲ん[のん] だ 。
\\	女の子ばかり来た。	
\\	女の子[おんなのこ] ばかり 来た[きた] 。
\\	子供達だけ来なかった。	
\\	子供達[こどもたち] だけ 来な[こな] かった 。
\\	石川君は三日でその本を読んでしまいました。	
\\	石川[いしかわ] 君[くん] は 三日[みっか] でその 本[ほん] を 読んで[よんで] しまいました 。
\\	私はルームメートのミルクを飲んでしまった。	
\\	私[わたし] は ルームメート の ミルク を 飲ん[のん] でしまった 。
\\	もう宿題をしてしまいましたか。	
\\	もう 宿題[しゅくだい] をしてしまいましたか 。
\\	僕は今日中にそのレポートを書いてしまおうと思っている。	
\\	僕は[ぼくは] 今日中[きょうじゅう] にその レポート を 書い[かい] てしまおうと 思って[おもって] いる 。
\\	速くご飯を食べてしまいなさい。	
\\	速[はや]くご 飯[はん]を 食[た]べてしまいなさい。
\\	シチューを作りすぎてしまいました。	
\\	シチュー を 作り[つくり] すぎてしまいました 。
\\	中島さんはすっかり日本語を忘れてしまった。	
\\	中島[なかじま] さんはすっかり 日本語[にほんご] を 忘れ[わすれ] てしまった 。
\\	僕は持っていた切手を全部友達に上げてしまった。	
\\	僕は[ぼくは] 持って[もって] いた 切手[きって] を 全部[ぜんぶ] 友達[ともだち] に 上げ[あげ] てしまった 。
\\	中島さんは前田さんのケーキを食べてしまった。	
\\	中島[なかしま] さんは 前田[まえだ] さんの ケーキ を 食べて[たべて] しまった 。
\\	僕は違うバスに乗ってしまった。	
\\	僕は[ぼくは] 違う[ちがう] バス に 乗って[のって] しまった 。
\\	雨が降ってしまったのでピクニックに行けなかった。	
\\	雨[あめ] が 降って[ふって] しまったので ピクニック に 行け[いけ] なかった 。
\\	僕はお酒を飲んでしまった。	
\\	僕は[ぼくは] お 酒[さけ] を 飲ん[のん] でしまった 。
\\	ここにおいておくと藤田さんが食べてしまうよ。	
\\	ここにおいておくと 藤田[ふじた] さんが 食べて[たべて] しまうよ 。
\\	昨日その本を読み終わった。	
\\	昨日[きのう] その 本[ほん] を 読み終わ[よみおわ] った 。
\\	今朝くじにやっとレポートを書き終わった。	
\\	今朝[けさ] くじにやっと レポート を 書き[かき] 終わ[おわ] った 。
\\	僕は小川さんの住所を忘れてしまった。	
\\	僕は[ぼくは] 小川[おがわ] さんの 住所[じゅうしょ] を 忘れ[わすれ] てしまった 。
\\	ー日本のことを知っていますか。 ーはい、知っています。	ー
\\	ー 日本[にほん] の こと を 知って[しって] いますか 。 ー はい 、 知って[しって] います 。
\\	岡田さんを知っていますか。	
\\	岡田[おかだ] さんを 知って[しって] いますか 。
\\	後藤さんの電話番号を知っていますか。	
\\	後藤[ごとう] さんの 電話番号[でんわばんごう] を 知って[しって] いますか 。
\\	ー中国語を知っていますか。 ーいいえ、知りません。	
\\	ー 中国語[ちゅうごくご] を 知って[しって] いますか 。 ー いいえ 、 知り[しり] ません 。
\\	私は長谷川さんがアメリカへ行ったことを知らなかった。	
\\	私[わたし] は 長谷川[はせがわ] さんが アメリカ へ 行[い] った こと を 知ら[しら] なかった 。
\\	村上さんはフランス語を勉強しているそうです。	
\\	村上[むらかみ] さんは フランス 語[ご] を 勉強[べんきょう] しているそうです 。
\\	近藤さんはお酒を飲まないそうです。	
\\	近藤[こんどう] さんはお 酒[さけ] を 飲[の] ま ないそうです 。
\\	日本の肉はとても高いそうだ。	
\\	日本[にほん] の 肉[にく] はとても 高い[たかい] そうだ 。
\\	利子さんは英語がとても上手だそうです。	
\\	利子[としこ] さんは 英語[えいご] がとても 上手[じょうず] だそうです 。
\\	石井さんは英語の先生だそうだ。	
\\	石井[いしい] さんは 英語[えいご] の 先生[せんせい] だそうだ 。
\\	新聞によるとフロリダに雪が降ったそうだ。	
\\	新聞[しんぶん] によると フロリダ に 雪[ゆき] が 降った[ふった] そうだ 。
\\	雨が降りそうです。	
\\	雨[あめ] が 降り[ふり] そうです 。
\\	あの車は高そうだ。	
\\	あの 車[くるま] は 高そ[たかそ] うだ 。
\\	この家は強い風が吹いたら倒れそうだ。	
\\	この 家[いえ] は 強い[つよい] 風[かぜ] が 吹い[ふい] たら 倒れ[たおれ] そうだ 。
\\	あのステーキは美味しそうだった。	
\\	あのステーキは 美味[おい]しそうだった。
\\	この辺りは静かそうだ。	
\\	この 辺り[あたり] は 静か[しずか] そうだ 。
\\	このアパートはよさそうだ。	
\\	この アパート はよさそうだ 。
\\	問題はなさそうだ。	
\\	問題[もんだい] はなさそうだ 。
\\	坂本さんの家はあまり新しくなさそうだ。	
\\	坂本[さかもと] さんの 家[いえ] は あまり 新し[あたらし] くなさそうだ 。
\\	遠藤さんは学生じゃなさそうだ。	
\\	遠藤[えんどう] さんは 学生[がくせい] じゃなさそうだ 。
\\	青木さんは車を売りそうに。	
\\	青木[あおき] さんは 車[くるま] を 売り[うり] そうに 。
\\	この問題は学生には出来もない。	
\\	この 問題[もんだい] は 学生[がくせい] には 出来も[できも] ない 。
\\	僕はこのケーキを残しそうだ。	
\\	僕は[ぼくは] この ケーキ を 残し[のこし] そうだ 。
\\	私はとても疲れていて倒れそうだ。	
\\	私[わたし] はとても 疲れ[つかれ] ていて 倒れ[たおれ] そうだ 。
\\	昨日は風邪を引きました。それで学校を休んだんです。	
\\	昨日[きのう] は 風邪[かぜ] を 引き[ひき] ました 。 それで 学校[がっこう] を 休ん[やすん] だんです 。
\\	ちょっと東京で用事がありました。それで昨日いなかったんです。	
\\	ちょっと 東京[とうきょう] で 用事[ようじ] がありました 。 それで 昨日[きのう] いなかったんです 。
\\	ー昨日はちょっと横浜で用事がありました。 ーああ、それでいらっしゃらなかったんですね。	ー
\\	ー 昨日[きのう] はちょっと 横浜[よこはま] で 用事[ようじ] がありました 。 ー ああ 、 それでいらっしゃらなかったんですね 。
\\	昨日が藤井君とピンポンの試合をしたんだ。それで、先週新しいラケットを買ったんですね。	
\\	昨日[きのう] が 藤井[ふじい] 君[くん] と ピンポン の 試合[しあい] をしたんだ 。 それで 、 先週[せんしゅう] 新[あたら] しい ラケット を 買[か] った んですね 。
\\	ちょっと大阪で用事があったので昨日いなかったんです。	
\\	ちょっと 大阪[おおさか] で 用事[ようじ] があったので 昨日[きのう] いなかったんです 。
\\	ちょっと名古屋で用事がありました。だから昨日いなかったんです。	
\\	ちょっと 名古屋[なごや] で 用事[ようじ] がありました 。 だから 昨日[きのう] いなかったんです 。
\\	今日は忙しいです。だから明日来て下さい。	
\\	今日は[きょうは] 忙し[いそがし] いです 。 だから 明日[あした] 来て[きて] 下[くだ]さい 。
\\	ーこのオレンジは甘くないです。 ーそれではこれはどうですか。	ー
\\	ー この オレンジ は 甘く[あまく] ないです 。 ー それではこれは どう ですか 。
\\	ー僕は魚も肉も嫌いです。 ーそれでは何を食べるんですか。	ー
\\	ー 僕は[ぼくは] 魚[さかな] も 肉[にく] も 嫌い[きらい] です 。 ー それでは 何を[なにを] 食[た]べるんですか 。
\\	ー今日の午後テニスをしませんか。 ー今日の午後はちょっと都合が悪いんですが。 ーそれでは明日の午後はどうですか。	ー
\\	ー 今日[きょう] の 午後[ごご] テニス をしませんか 。 ー 今日[きょう] の 午後[ごご] はちょっと 都合[つごう] が 悪い[わるい] んですが 。 ー それでは 明日[あした] の 午後[ごご] は どう ですか 。
\\	それでは二十分ぐらい休みましょう。	
\\	それでは 二十分[にじゅっぷん] ぐらい 休み[やすみ] ましょう 。
\\	それではまた来週の金曜日に来ます。	
\\	それではまた 来週[らいしゅう] の 金曜日[きんようび] に 来ま[きま] す 。
\\	昨日は二時間ぐらい友達と飲んでそれからうちに帰りました。	
\\	昨日[きのう] は 二時間[にじかん] ぐらい 友達[ともだち] と 飲ん[のん] でそれから うち に 帰り[かえり] ました 。
\\	昨日は二時間ぐらい友達と飲んだ。それから本屋によってうちに帰った。	
\\	昨日[きのう] は 二時間[にじかん] ぐらい 友達[ともだち] と 飲ん[のん] だ 。 それから 本屋[ほんや] によって うち に 帰った[かえった] 。
\\	十時まで宿題をしました。それから映画に行きました。	
\\	十時[じゅうじ] まで 宿題[しゅくだい] をしました 。 それから 映画[えいが] に 行き[いき] ました 。
\\	昨日は朝銀座に行って、それから映画を見に行った。	
\\	昨日[きのう]は 朝[あさ] 銀座[ぎんざ]に 行[い]って、それから 映画[えいが]を 見[み]に 行[い]った。
\\	レストランではステーキとサラダとそれからチーズケーキを食べました。	
\\	レストラン では ステーキ と サラダ とそれから チーズケーキ を 食べ[たべ] ました 。
\\	湖の色は初めは青く、それから緑になりました。	
\\	湖[みずうみ] の 色[いろ] は 初め[はじめ] は 青く[あおく] 、 それから 緑[みどり] になりました 。
\\	月曜、火曜、それから木曜もとても忙しいです。	
\\	月曜[げつよう] 、 火曜[かよう] 、 それから 木曜[もくよう] もとても 忙し[いそがし] いです 。
\\	西村さんと福田さんと、ああそうだ、それから太田さんが来たよ。	
\\	西村[にしむら] さんと 福田[ふくだ] さんと 、 ああそうだ 、 それから 太田[おおた] さんが 来た[きた] よ 。
\\	ー今日はどこへ行きましたか。 ーまず東京タワーに上りました。 ーそれから? ー美術館に行きました。 ーそれから? ーデパートに行って、食堂で昼ご飯を食べました。	ー
\\	ー 今日は[きょうは] どこ へ 行き[いき] ましたか 。 ー まず 東京[とうきょう] タワー に 上り[のぼり] ました 。 ー それから 
\\	ー 美術館[びじゅつかん] に 行き[いき] ました 。 ー それから 
\\	ーデパート に 行って[いって] 、 食堂[しょくどう] で 昼[ひる] ご 飯[はん] を 食べ[たべ] ました 。
\\	三浦さんは三時間ゴルフをしてから一時間泳いだ。	
\\	三浦[みうら] さんは 三時[さんじ] 間[かん] ゴルフ をしてから 一時間[いちじかん] 泳い[およい] だ 。
\\	藤原さんは三時間ゴルフをして、それから一時間泳いだ。	
\\	藤原[ふじわら] さんは 三時[さんじ] 間[かん] ゴルフ をして 、 それから 一時間[いちじかん] 泳い[およい] だ 。
\\	今日はテニスをして、それから映画も見た。	
\\	今日は[きょうは] テニス をして 、 それから 映画[えいが] も 見た[みた] 。
\\	今日はテニスもしたし、映画も見た。	
\\	今日は[きょうは] テニス もしたし 、 映画[えいが] も 見た[みた] 。
\\	音楽を聞いて、そうして勉強するのが好きだ。	
\\	音楽[おんがく] を 聞い[きい] て 、 そうして 勉強[べんきょう] するのが 好き[すき] だ 。
\\	音楽を聞いて、それから勉強するのが好きだ。	
\\	音楽[おんがく] を 聞い[きい] て 、 それから 勉強[べんきょう] するのが 好き[すき] だ 。
\\	ー頭が痛いんです。 ーそれならすぐ寝なさい。	ー
\\	ー 頭[あたま] が 痛い[いたい] んです 。 ー それならすぐ 寝な[ねな] さい 。
\\	ー映画を見に行きませんか。 ー明日試験があるんです。 ーそれなら、明後日はどうですか。	ー
\\	ー 映画[えいが] を 見に行き[みにいき] ませんか 。 ー 明日[あした] 試験[しけん] があるんです 。 ー それなら 、 明後日[あさって] は どう ですか 。
\\	ー日本にはどのぐらいいましたか。 ー三年です。 ーそれなら、日本のことはよく知っているでしょうね。	ー
\\	ー 日本[にほん] にはどのぐらいいましたか 。 ー 三年[さんねん] です 。 ー それなら 、 日本[にほん] の こと はよく 知って[しって] いるでしょうね 。
\\	頭が痛いんならすぐ寝なさい。	
\\	頭[あたま] が 痛い[いたい] んならすぐ 寝な[ねな] さい 。
\\	それでは失礼します。	
\\	それでは 失礼[しつれい] します 。
\\	それでは遠慮なく頂きます。	
\\	それでは 遠慮[えんりょ] なく 頂き[いただき] ます 。
\\	この本が間違っているかそれとも私が間違っているかどちらかです。	
\\	この 本[ほん] が 間違って[まちがって] いるかそれとも 私[わたし] が 間違って[まちがって] いるかどちらかです 。
\\	刺身を食べますか。それとも寿喜焼にしますか。	
\\	刺身[さしみ] を 食べ[たべ] ますか 。 それとも 寿[す] 喜[き] 焼[やき] にしますか 。
\\	ぼくがくるか岡本さんが来るかどちらかです。	
\\	ぼくがくるか 岡本[おかもと] さんが 来る[くる] かどちらかです 。
\\	あの人は先生ですか。それとも医者ですか。	
\\	あの 人[ひと] は 先生[せんせい] ですか 。 それとも 医者[いしゃ] ですか 。
\\	刺身?それとも、寿喜焼?	
\\	刺身[さしみ] 
\\	それとも 、 寿[す] 喜[き] 焼[やき] 
\\	今日は札幌に行きました。そして友達に会いました。	
\\	今日は[きょうは] 札幌[さっぽろ] に 行き[いき] ました 。 そして 友達[ともだち] に 会[あ] い ました 。
\\	昨日は朝ゴルフをしました。そして午後はテニスをしました。	
\\	昨日[きのう] は 朝[あさ] ゴルフ をしました 。 そして 午後[ごご] は テニス をしました 。
\\	この映画は詰らない。そして長過ぎる。	
\\	この 映画[えいが] は 詰ら[つまら] ない 。 そして 長過ぎ[ながすぎ] る 。
\\	この花はきれいです。そして安いです。	
\\	この 花[はな] はきれい です 。 そして 安い[やすい] です 。
\\	そうして直すんですか。	
\\	そうして 直す[なおす] んですか 。
\\	今日は神戸に行って、そうして友達に会いました。	
\\	今日は[きょうは] 神戸[こうべ] に 行って[いって] 、 そうして 友達[ともだち] に 会[あ] い ました 。
\\	松田さんは肉を食べ過ぎます。	
\\	松田[まつだ] さんは 肉[にく] を 食べ[たべ] 過ぎ[すぎ] ます 。
\\	このアパートは私達には高過ぎる。	
\\	この アパート は 私達[わたしたち] には 高過ぎ[たかすぎ] る 。
\\	私は今朝寝過ぎて学校に遅れた。	
\\	私[わたし] は 今朝[けさ] 寝過[ねす] ぎて 学校[がっこう] に 遅[おく] れた 。
\\	この机は私の部屋には大き過ぎる。	
\\	この 机[つくえ] は 私[わたし] の 部屋[へや] には 大き過ぎ[おおきすぎ] る 。
\\	中川先生の授業は大変過ぎるので止めました。	
\\	中川[なかがわ] 先生[せんせい] の 授業[じゅぎょう] は 大変[たいへん] 過ぎ[すぎ] るので 止め[やめ] ました 。
\\	中野さんは太り過ぎている。	
\\	中野[なかの]さんは 太[ふと]り 過[す]ぎている。
\\	このアパートはトムには良過ぎる。	
\\	この アパート はトムには 良過ぎ[よすぎ] る 。
\\	ベンは力がなさ過ぎる。	
\\	ベン は 力[ちから] がなさ 過ぎ[すぎ] る 。
\\	友子は野菜を食べなさ過ぎる。	
\\	友子[ともこ] は 野菜[やさい] を 食べ[たべ] なさ 過ぎ[すぎ] る 。
\\	私はステーキが好きです。	
\\	私[わたし] は ステーキ が 好き[すき] です 。
\\	僕は野球が好きだ。	
\\	僕は[ぼくは] 野球[やきゅう] が 好き[すき] だ 。
\\	ジョンソンさんはジャズが大好きです。	
\\	ジョンソン さんは ジャズ が 大好き[だいすき] です 。
\\	僕がステーキが好きなことは皆知っている。	
\\	僕[ぼく]がステーキが 好[す]きなことは 皆[みな] 知[し]っている。
\\	ジョンが好きなスポーツは野球です。	
\\	ジョンが 好[す]きなスポーツは 野球[やきゅう]です。
\\	この町はいいレストランが少ないです。	
\\	この 町[まち] はいい レストラン が 少ない[すくない] です 。
\\	この大学は女子学生が少ないです。	
\\	この 大学[だいがく] は 女子学生[じょしがくせい] が 少ない[すくない] です 。
\\	日本は犯罪が少ない。	
\\	日本[にほん] は 犯罪[はんざい] が 少ない[すくない] 。
\\	日本語が書ける外国人は大変少ない。	
\\	日本語[にほんご] が 書け[かけ] る 外国人[がいこくじん] は 大変[たいへん] 少ない[すくない] 。
\\	木が少ない町に住みたくありません。	
\\	木[き] が 少ない[すくない] 町[まち] に 住み[すみ] たくありません 。
\\	日本語が分かるアメリカ人は少ない。	
\\	日本語[にほんご] が 分か[わか] る アメリカ 人[じん] は 少ない[すくない] 。
\\	日本語が分かるアメリカ人は少しいる。	
\\	日本語[にほんご] が 分か[わか] る アメリカ 人[じん] は 少し[すこし] いる 。
\\	私が持っているお金は僅かだ。	
\\	私[わたし] が 持って[もって] いるお 金[かね] は 僅か[わずか] だ 。
\\	僅かなお金で暮らしている。	
\\	僅[わず]かなお 金[かね]で 暮[く]らしている。
\\	あの人は僅かのことですぐ怒る。	
\\	あの 人[ひと] は 僅か[わずか] の こと ですぐ 怒る[おこる] 。
\\	その時僕は僅か六つでした。	
\\	その 時[とき] 僕は[ぼくは] 僅か[わずか] 六つ[むっつ] でした 。
\\	中山さんはテニスをします。	
\\	中山[なかやま] さんは テニス をします 。
\\	リーズさんは英語の先生をしている。	
\\	リーズ さんは 英語[えいご] の 先生[せんせい] をしている 。
\\	先生がテストを易しくしました。	
\\	先生[せんせい]がテストを 易[やさ]しくしました。
\\	木口は息子を医者にした。	
\\	木口[きぐち] は 息子[むすこ] を 医者[いしゃ] にした 。
\\	僕は中国語を勉強しています。	
\\	僕は[ぼくは] 中国語[ちゅうごくご] を 勉強[べんきょう] しています 。
\\	京子はきれいなスカーフをしている。	
\\	京子[きょうこ] は きれい な スカーフ をしている 。
\\	日本人は大抵土曜日も仕事をする。	
\\	日本人[にほんじん] は 大抵[たいてい] 土曜日[どようび] も 仕事[しごと] をする 。
\\	ビルはハムレッタをするつもりだ。	
\\	ビル は ハムレッタ をするつもりだ 。
\\	陽子は部屋をきれいにした。	
\\	陽子[ようこ] は 部屋[へや] を きれい にした 。
\\	僕はよく車を運転します。	
\\	僕は[ぼくは] よく 車[くるま] を 運転[うんてん] します 。
\\	一男はアメリカ人を妻にした。	
\\	一男[かずお] は アメリカ 人[じん] を 妻[つま] にした 。
\\	いいネクタイをしているね。	
\\	いい ネクタイ をしているね 。
\\	僕は中国語の勉強をしている。	
\\	僕は[ぼくは] 中国語[ちゅうごくご] の 勉強[べんきょう] をしている 。
\\	僕はよく車の運転をします。	
\\	僕は[ぼくは] よく 車[くるま] の 運転[うんてん] をします 。
\\	病気をする。	
\\	病気[びょうき] をする 。
\\	病気になる。	
\\	病気[びょうき] になる 。
\\	若い頃はよく病気をした。	
\\	若い[わかい] 頃[ころ] はよく 病気[びょうき] をした 。
\\	いつも病気をしてすみません。	
\\	いつ も 病気[びょうき] をしてすみません 。
\\	一年に何回ぐらい病気をしますか。	
\\	一年[いちねん] に 何回[なんかい] ぐらい 病気[びょうき] をしますか 。
\\	昨日急に病気になりました。	
\\	昨日[きのう] 急[きゅう]に 病気[びょうき]になりました。
\\	僕はを読みします。	
\\	僕は[ぼくは] を 読み[よみ] します 。
\\	山田は停学になった。	
\\	山田[やまだ] は 停学[ていがく] になった 。
\\	学校は山田を停学にした。	
\\	学校[がっこう] は 山田[やまだ] を 停学[ていがく] にした 。
\\	山田は本を書くことになった。	
\\	山田[やまだ] は 本[ほん] を 書く[かく] こと になった 。
\\	山田は本を書くことにした。	
\\	山田[やまだ] は 本[ほん] を 書く[かく] こと にした 。
\\	陽子は長い足をしています。	
\\	陽子[ようこ] は 長い[ながい] 足[あし] をしています 。
\\	一男は丈夫な体をしています。	
\\	一男[かずお] は 丈夫[じょうぶ] な 体[からだ] をしています 。
\\	この机は丸い形をしている。	
\\	この 机[つくえ] は 丸い[まるい] 形[かたち] をしている 。
\\	陽子は足が長い。	
\\	陽子[ようこ] は 足[あし] が 長い[ながい] 。
\\	子供達の声がしました。	
\\	子供達[こどもたち] の 声[こえ] がしました 。
\\	この魚は変な味がしますね。	
\\	この 魚[さかな] は 変な[へんな] 味[あじ] がしますね 。
\\	このきれはざらざらしている。	
\\	このきれはざらざらしている。
\\	この花はいい匂いがする。	
\\	この 花[はな] はいい 匂い[におい] がする 。
\\	私は寒気がします。	
\\	私[わたし] は 寒気[さむけ] がします 。
\\	この林檎はきれいな色をしている。	
\\	この 林檎[りんご] はきれい な 色[いろ] をしている 。
\\	この林檎は色がきれいだ。	
\\	この 林檎[りんご] は 色[いろ] がきれい だ 。
\\	この時計は十万円します。	
\\	この 時計[とけい] は 十万円[じゅうまんえん] します 。
\\	後一年したら大学を出る。	
\\	後[あと] 一年[いちねん] したら 大学[だいがく] を 出る[でる] 。
\\	ーそれはどのぐらいしましたか。 ー二十五万円ぐらいしました。	ー
\\	ーそれはどのぐらいしましたか 。 ー二十[にじゅう] 五万[ごまん] 円[えん] ぐらいしました 。
\\	もう少しすれば主人が帰ってまいります。	
\\	もう 少[すこ]しすれば 主人[しゅじん]が 帰[かえ]ってまいります。
\\	三年経ちました。	
\\	三年[さんねん] 経ち[たち] ました 。
\\	ジョギングを始めました。するとご飯が美味しくなりました。	
\\	ジョギング を 始め[はじめ] ました 。 するとご 飯[はん] が 美味しく[おいしく] なりました 。
\\	ー今日は月曜日ですよ。 ーするとあのデパートは休みですね。	
\\	ー 今日は[きょうは] 月曜日[げつようび] ですよ 。 ー するとあの デパート は 休み[やすみ] ですね 。
\\	僕は自転車を買いました。すると弟も欲しがりました。	
\\	僕は[ぼくは] 自転車[じてんしゃ] を 買い[かい] ました 。 すると 弟[おとうと] も 欲し[ほし] がりました 。
\\	頭が痛かったのでアスピリンを飲みました。すると痛みがすぐ止まりました。	
\\	頭[あたま] が 痛か[いたか] ったので アスピリン を 飲み[のみ] ました 。 すると 痛み[いたみ] がすぐ 止まり[どまり] ました 。
\\	ー息子は今高校三年です。 ーすると、来年は大学受験ですね。	ー
\\	ー 息子[むすこ] は 今[いま] 高校[こうこう] 三年[さんねん] です 。 ー すると 、 来年[らいねん] は 大学受験[だいがくじゅけん] ですね 。
\\	ー今日は忙しいです。 ーそれでは明日行きましょう。	ー
\\	ー 今日は[きょうは] 忙し[いそがし] いです 。 ー それでは 明日[あした] 行き[いき] ましょう 。
\\	今日は忙しいです。それでは明日来て下さい。	
\\	今日は[きょうは] 忙し[いそがし] いです 。 それでは 明日[あした] 来て[きて] 下[くだ]さい 。
\\	ーあなたが来ないと困るんです。 ーそれでは行きます。	ー
\\	ー あなたが 来ない[こない] と 困る[こまる] んです 。 ー それでは 行き[いき] ます 。
\\	私共は何も存じません。	
\\	私[わたくし] 共[ども]は 何[なに]も 存[ぞん]じません。
\\	あなた方はいついらっしゃいますか。	
\\	あなた 方は[がたは] いつ いらっしゃいますか 。
\\	先生方はいらっしゃらないそうです。	
\\	先生方[せんせいがた] はいらっしゃらないそうです 。
\\	私は日本へ行きたいです。	
\\	私[わたし] は 日本[にほん] へ 行き[いき] たいです 。
\\	僕は今ピザが食べたい。	
\\	僕は[ぼくは] 今[いま] ピザ が 食べた[たべた] い 。
\\	鈴木さんはアメリカへ行きたがっている。	
\\	鈴木[すずき] さんは アメリカ へ 行き[いき] たがっている 。
\\	僕は冷たいビールを飲みたい。	
\\	僕[ぼく]は 冷[つめ]たいビールを 飲[の]みたい。
\\	今日は何が食べたいですか。	
\\	今日は[きょうは] 何[なに] が 食べた[たべた] いですか 。
\\	三木さんは車を買いたがっている。	
\\	三木[みき] さんは 車[くるま] を 買い[かい] たがっている 。
\\	和男はとても行きたかった。	
\\	和男[かずお] はとても 行き[いき] たかった 。
\\	一郎も行きたいと言っている。	
\\	一郎[いちろう] も 行き[いき] たいと 言って[いって] いる 。
\\	利子は日本へ帰りたいそうだ。	
\\	利子[としこ] は 日本[にほん] へ 帰り[かえり] たいそうだ 。
\\	野村さんはあなたと話したいんですよ。	
\\	野村[のむら] さんはあなたと 話し[はなし] たいんですよ 。
\\	村山さんはのり子と踊りたいらしい。	
\\	村山[むらやま] さんはのり 子[こ] と 踊り[おどり] たいらしい 。
\\	早田さんは早く家族に会いたそうだ。	
\\	早田[はやた] さんは 早く[はやく] 家族[かぞく] に 会[あ] い たそうだ 。
\\	私は水を出かける前に飲みたい。	
\\	私[わたし] は 水[みず] を 出か[でか] ける 前[まえ] に 飲み[のみ] たい 。
\\	僕は先生にこの絵を褒められたい。	
\\	僕[ぼく]は 先生[せんせい]にこの 絵[え]を 褒[ほ]められたい。
\\	僕は早くこの電車を降りたい。	
\\	僕は[ぼくは] 早く[はやく] この 電車[でんしゃ] を 降り[おり] たい 。
\\	私は公園を歩きたい。	
\\	私[わたし] は 公園[こうえん] を 歩き[あるき] たい 。
\\	私と一緒に行きたいですか。	
\\	私[わたし] と 一緒に[いっしょに] 行き[いき] たいですか 。
\\	明日食事に来たいですか。	
\\	明日[あした] 食事[しょくじ] に 来た[きた] いですか 。
\\	私と一緒に行きませんか。	
\\	私[わたし] と 一緒に[いっしょに] 行き[いき] ませんか 。
\\	今日は暑くてたまりません。	
\\	今日は[きょうは] 暑く[あつく] てたまりません 。
\\	数学が嫌いでたまりません。	
\\	数学[すうがく] が 嫌い[きらい] でたまりません 。
\\	この本が面白くてたまりません。	
\\	この 本[ほん] が 面白く[おもしろく] てたまりません 。
\\	父が死んで、悲しくてたまりません。	
\\	父[ちち] が 死ん[しん] で 、 悲し[かなし] くてたまりません 。
\\	おいしい魚が食べたくてたまりません。	
\\	おいしい 魚[さかな]が 食[た]べたくてたまりません。
\\	僕はあの子が好きでたまらない。	
\\	僕は[ぼくは] あの 子[こ] が 好き[すき] でたまらない 。
\\	兄は田中先生の授業が嫌でたまらなかった。	
\\	兄[あに] は 田中[たなか] 先生[せんせい] の 授業[じゅぎょう] が 嫌で[いやで] たまらなかった 。
\\	そんな本は読んでも仕方がない。	
\\	そんな 本[ほん] は 読んで[よんで] も 仕方[しかた] がない 。
\\	学生は試験のために勉強します。	
\\	学生[がくせい] は 試験[しけん] のために 勉強[べんきょう] します 。
\\	私は日本のことを知るため日本へ行く。	
\\	私[わたし]は 日本[にほん]のことを 知[し]るため 日本[にほん]へ 行[い]く。
\\	今年は雪があまり降らないためにスキーが出来ません。	
\\	今年[ことし] は 雪[ゆき] が あまり 降ら[ふら] ないために スキー が 出来ま[できま] せん 。
\\	外国人のためのいい辞書がない。	
\\	外国人[がいこくじん] のためのいい 辞書[じしょ] がない 。
\\	日本人は会社のためによく働きます。	
\\	日本人[にほんじん] は 会社[かいしゃ] のためによく 働き[はたらき] ます 。
\\	雪のため学校が休みになった。	
\\	雪[ゆき] のため 学校[がっこう] が 休み[やすみ] になった 。
\\	ジャクソンさんは研究のためドイツに行った。	
\\	ジャクソン さんは 研究[けんきゅう] のため ドイツ に 行[い] った 。
\\	私は父が死んだために大学に行けなかった。	
\\	私[わたし] は 父[ちち] が 死ん[しん] だために 大学[だいがく] に 行け[いけ] なかった 。
\\	体を強くするために毎日プールで泳いでいる。	
\\	体[からだ] を 強く[つよく] するために 毎日[まいにち] プール で 泳い[およい] でいる 。
\\	親が甘かったため子供がだめになった。	
\\	親[おや] が 甘か[あまか] ったため 子供[こども] がだめになった 。
\\	字が下手なために人に笑われた。	
\\	字が[じが] 下手[へた] なために 人[ひと] に 笑わ[わらわ] れた 。
\\	フランスに行ったのは香水を買うためだ。	
\\	フランス に 行[い] った のは 香水[こうすい] を 買う[かう] ためだ 。
\\	仕事のために何もほかのことが出来ない。	
\\	仕事[しごと] のために 何も[なにも] ほかのことが 出来な[できな] い 。
\\	会社のために朝から晩まで働いている。	
\\	会社[かいしゃ] のために 朝[あさ] から 晩[ばん] まで 働い[はたらい] ている 。
\\	今年は雪があまり降らないのでスキーが出来ない。	
\\	今年[ことし] は 雪[ゆき] が あまり 降ら[ふら] ないので スキー が 出来な[できな] い 。
\\	僕は父が死んだから大学に行けなかった。	
\\	僕は[ぼくは] 父[ちち] が 死ん[しん] だから 大学[だいがく] に 行け[いけ] なかった 。
\\	漢字を調べるために辞書を使う。	
\\	漢字[かんじ] を 調べ[しらべ] るために 辞書[じしょ] を 使う[つかう] 。
\\	漢字を調べるのに辞書を使う。	
\\	漢字[かんじ] を 調べ[しらべ] るのに 辞書[じしょ] を 使う[つかう] 。
\\	山田さんが来たら私は帰ります。	
\\	山田[やまだ] さんが 来た[きた] ら 私[わたし] は 帰り[かえり] ます 。
\\	先生に聞いたらすぐ分かった。	
\\	先生[せんせい] に 聞い[きい] たらすぐ 分か[わか] った 。
\\	僕は大学を出たら小学校の先生になります。	
\\	僕は[ぼくは] 大学[だいがく] を 出た[でた] ら 小学校[しょうがっこう] の 先生[せんせい] になります 。
\\	私は忙しいですから、面白かったら読みますが、面白くなかったら読みませんよ。	
\\	私[わたし] は 忙し[いそがし] いですから 、 面白か[おもしろか] ったら 読み[よみ] ますが 、 面白く[おもしろく] なかったら 読み[よみ] ませんよ 。
\\	嫌いだったら残して下さい。	
\\	嫌い[きらい] だったら 残し[のこし] て 下[くだ]さい 。
\\	英語だったら分かると思います。	
\\	英語[えいご] だったら 分か[わか] ると 思い[おもい] ます 。
\\	十二時になったら帰ります。	
\\	十二[じゅうに] 時に[じに] なったら 帰り[かえり] ます 。
\\	もし山田さんが来たら私は帰ります。	
\\	もし 山田[やまだ] さんが 来た[きた] ら 私[わたし] は 帰り[かえり] ます 。
\\	仕事が早く終わったら僕のうちに来なさい。	
\\	仕事[しごと]が 早[はや]く 終[お]わったら 僕[ぼく]のうちに 来[き]なさい。
\\	仕事が早く終わったら僕のうちに来て下さい。	
\\	仕事[しごと]が 早[はや]く 終[お]わったら 僕[ぼく]のうちに 来[き]て 下[くだ]さい。
\\	仕事が早く終わったら僕のうちに来たらどうですか。	
\\	仕事[しごと]が 早[はや]く 終[お]わったら 僕[ぼく]のうちに 来[き]たらどうですか。
\\	仕事が早く終わったら僕のうちに来ませんか。	
\\	仕事[しごと]が 早[はや]く 終[お]わったら 僕[ぼく]のうちに 来[き]ませんか。
\\	仕事が早く終わったらおうちにお伺いします。	
\\	仕事[しごと]が 早[はや]く 終[お]わったらおうちにお 伺[うかが]いします。
\\	お金があったらこんなうちにはいない。	
\\	お 金[かね] があったらこんなうちにはいない 。
\\	あの時お金があったら日本へ行っていたでしょう。	
\\	あの 時[とき] お 金[かね] があったら 日本[にほん] へ 行って[いって] いたでしょう 。
\\	学校へ行ったら、偶然上田さんにあった。	
\\	学校[がっこう] へ 行[い] ったら 、 偶然[ぐうぜん] 上田[うえだ] さんにあった 。
\\	お酒を飲んだら寝てしまった。	
\\	お 酒[さけ] を 飲ん[のん] だら 寝て[ねて] しまった 。
\\	先生に聞くとすぐ分かった。	
\\	先生[せんせい] に 聞く[きく] とすぐ 分か[わか] った 。
\\	キャシーのアパートに行ったらアンディーがいた。	
\\	キャシー の アパート に 行[い] った ら アンディー がいた 。
\\	安かったら買いなさい。	
\\	安か[やすか] ったら 買い[かい] なさい 。
\\	安ければ買いなさい。	
\\	安け[やすけ] れば 買い[かい] なさい 。
\\	安いなら買いなさい。	
\\	安い[やすい] なら 買い[かい] なさい 。
\\	ベンが来たら、私は帰ります。	
\\	ベン が 来た[きた] ら 、 私[わたし] は 帰り[かえり] ます 。
\\	ベンが来れば、私は帰ります。	
\\	ベン が 来れ[くれ] ば 、 私[わたし] は 帰り[かえり] ます 。
\\	ベンが来るなら、私は帰ります。	
\\	ベン が 来る[くる] なら 、 私[わたし] は 帰り[かえり] ます 。
\\	あの時お金があれば日本に行っていただろう。	
\\	あの 時[とき] お 金[かね] があれば 日本[にほん] に 行って[いって] いただろう 。
\\	もっと日本の本を読んだらどうですか。	
\\	もっと 日本[にほん] の 本[ほん] を 読んだ[よんだ] ら どう ですか 。
\\	サラダも食べたらどうですか。	
\\	サラダ も 食べた[たべた] ら どう ですか 。
\\	山村先生に聞いたらどうですか。	
\\	山村[やまむら] 先生[せんせい] に 聞い[きい] たら どう ですか 。
\\	私たちは歌ったり踊ったりしました。	
\\	私たち[わたしたち] は 歌った[うたった] り 踊った[おどった] りしました 。
\\	このレストランのステーキは大きかったり小さかったりする。	
\\	この レストラン の ステーキ は 大き[おおき] かったり 小さ[ちいさ] かったりする 。
\\	ー毎日テープを聴いていますか。 ーいいえ、聞いたり聞かなかったりです。	
\\	ー 毎日[まいにち] テープ を 聴い[きい] ていますか 。 ー いいえ 、 聞い[きい] たり 聞か[きか] なかったりです 。
\\	私は福岡でミュージカルを見たりコンサートを聴いたりした。	
\\	私[わたし] は 福岡[ふくおか] で ミュージカル を 見た[みた] り コンサート を 聴い[きい] たりした 。
\\	トムは来たり来なかったりする。	
\\	トム は 来た[きた] り 来な[こな] かったりする 。
\\	この店の魚は新しかったり古かったりする。	
\\	この 店[みせ] の 魚[さかな] は 新し[あたらし] かったり 古か[ふるか] ったりする 。
\\	明日は雨が降ったりやんだりするでしょう。	
\\	明日[あした] は 雨[あめ] が 降った[ふった] りやんだりするでしょう 。
\\	日本語の先生は日本人だったりアメリカ人だったりします。	
\\	日本語[にほんご] の 先生[せんせい] は 日本人[にほんじん] だったり アメリカ 人[じん] だったりします 。
\\	石川さんは来たり来なかったりであてにならない。	
\\	石川[いしかわ] さんは 来た[きた] り 来な[こな] かったりであてにならない 。
\\	本を読んだり、映画を見たり、テープを聴いたりして日本語を勉強している。	
\\	本[ほん] を 読んだ[よんだ] り 、 映画[えいが] を 見た[みた] り 、 テープ を 聴い[きい] たりして 日本語[にほんご] を 勉強[べんきょう] している 。
\\	新聞を読んだりして友達が来るのを待っていた。	
\\	新聞[しんぶん] を 読んだ[よんだ] りして 友達[ともだち] が 来る[くる] のを 待って[まって] いた 。
\\	歌ったり踊ったりとても楽しかった。	
\\	歌った[うたった] り 踊った[おどった] りとても 楽し[たのし] かった 。
\\	あの人には話したって分かりません。	
\\	あの 人[ひと] には 話し[はなし] たって 分か[わか] りません 。
\\	僕はその切符を高くたって買う。	
\\	僕は[ぼくは] その 切符[きっぷ] を 高く[たかく] たって 買う[かう] 。
\\	お金があったって車は買いたくない。	
\\	お 金[かね] があったって 車[くるま] は 買い[かい] たくない 。
\\	本を買ったって、忙しくて読めない。	
\\	本[ほん] を 買[か] った って 、 忙し[いそがし] くて 読め[よめ] ない 。
\\	汚くたって構いません。	
\\	汚[きたな] くたって 構い[かまい] ません 。
\\	遠くて不便だってマイホームなら構いません。	
\\	遠く[とおく] て 不便[ふべん] だって マイホーム なら 構い[かまい] ません 。
\\	どんなにいい先生だって時々間違います。	
\\	どんなにいい 先生[せんせい] だって 時々[ときどき] 間違い[まちがい] ます 。
\\	どんなに考えたって分からないよ。	
\\	どんなに 考え[かんがえ] たって 分か[わか] らないよ 。
\\	どんなに寒くたって大丈夫です。	
\\	どんなに 寒く[さむく] たって 大丈夫[だいじょうぶ] です 。
\\	お金があっても車は買いたくない。	
\\	お 金[かね] があっても 車[くるま] は 買い[かい] たくない 。
\\	ジムは川崎へ行って勉強しました。	
\\	ジム は 川崎[かわさき] へ 行って[いって] 勉強[べんきょう] しました 。
\\	ここのステーキは安くておいしい。	
\\	ここの ステーキ は 安く[やすく] て おいしい 。
\\	このアパートは静かでいいです。	
\\	この アパート は 静か[しずか] でいいです 。
\\	私の父が先生で高校で英語を教えている。	
\\	私[わたし] の 父[ちち] が 先生[せんせい] で 高校[こうこう] で 英語[えいご] を 教[おし]え ている 。
\\	僕はコートを脱いでハンガーにかけた。	
\\	僕は[ぼくは] コート を 脱い[ぬい] で ハンガー にかけた 。
\\	ワインを飲みすぎて頭が痛い。	
\\	ワイン を 飲み[のみ] すぎて 頭[あたま] が 痛い[いたい] 。
\\	私の部屋は狭くて暗い。	
\\	私[わたし] の 部屋[へや] は 狭く[せまく] て 暗い[くらい] 。
\\	このスープは辛くて飲めない。	
\\	この スープ は 辛く[からく] て 飲め[のめ] ない 。
\\	私はテニスが大好きでよく友達とする。	
\\	私[わたし] は テニス が 大好き[だいすき] でよく 友達[ともだち] とする 。
\\	ゆみ子は今大学三年で専攻は日本文学です。	
\\	ゆみ 子[こ] は 今[いま] 大学[だいがく] 三年[さんねん] で 専攻[せんこう] は 日本文[にほんぶん] 学[がく] です 。
\\	伊藤先生は今週病気で、代りに村田先生が教えた。	
\\	伊藤[いとう] 先生[せんせい] は 今週[こんしゅう] 病気[びょうき] で 、 代り[かわり] に 村田[むらた] 先生[せんせい] が 教[おし]え た 。
\\	男は外で働いて、女はうちで働く。	
\\	男[おとこ] は 外[そと] で 働い[はたらい] て 、 女[おんな] は うち で 働く[はたらく] 。
\\	トムはいつも遊んでいてテストが出来る。	
\\	トム は いつ も 遊んで[あそんで] いて テスト が 出来る[できる] 。
\\	エミーは美しくて明るくて人に親切だ。	
\\	エミー は 美し[うつくし] くて 明る[あかる] くて 人[ひと] に 親切[しんせつ] だ 。
\\	私は雨が降っても行きます。	
\\	私[わたし] は 雨[あめ] が 降って[ふって] も 行き[いき] ます 。
\\	私は寒くても出かける。	
\\	私[わたし] は 寒く[さむく] ても 出か[でか] ける 。
\\	僕がトムでも同じことをしただろう。	
\\	僕[ぼく] が トム でも 同じ[おなじ] ことをしただろう 。
\\	中山さんは本を買っても読みません。	
\\	中山[なかやま] さんは 本[ほん] を 買って[かって] も 読み[よみ] ません 。
\\	私は四時間歩いても疲れなかった。	
\\	私[わたし] は 四時[よじ] 間[かん] 歩い[あるい] ても 疲れ[つかれ] なかった 。
\\	何を食べてもおいしいです。	
\\	何を[なにを] 食べて[たべて] も おいしい です 。
\\	誰に聞いても分からなかった。	
\\	誰[だれ] に 聞い[きい] ても 分か[わか] らなかった 。
\\	ーたばこを吸ってもいいですか。 ーはい、いいです。	ー
\\	ー たばこを 吸って[すって] もいいですか 。 ー はい 、 いいです 。
\\	どんなに日本語が難しくてもやってみます。	
\\	どんなに 日本語[にほんご] が 難しく[むずかしく] てもやってみます 。
\\	どんなに丈夫でも体には気をつけた方がいい。	
\\	どんなに 丈夫[じょうぶ] でも 体[からだ] には 気を[きを] つけた 方が[ほうが] いい 。
\\	井上さんが先輩でも僕ははっきり言うつもりだ。	
\\	井上[いのうえ] さんが 先輩[せんぱい] でも 僕は[ぼくは] はっきり 言う[いう]つもりだ 。
\\	中山さんは本を買うけれど読みません。	
\\	中山[なかやま] さんは 本[ほん] を 買う[かう] けれど 読み[よみ] ません 。
\\	ーここでたばこを吸ってもいいですか。 ーはい、いいです。	ー
\\	ー ここでたばこを 吸って[すって] もいいですか 。 ー はい 、 いいです 。
\\	高くてもいいです。	
\\	高く[たかく] てもいいです 。
\\	学生でもいい。	
\\	学生[がくせい] でもいい 。
\\	この椅子を使ってもいいですか。	
\\	この 椅子[いす]を 使[つか]ってもいいですか。
\\	今日の宿題は作文です。短くてもいいですが面白いのを書いて下さい。	
\\	今日[きょう] の 宿題[しゅくだい] は 作文[さくぶん] です 。 短く[みじかく] てもいいですが 面白い[おもしろい] のを 書い[かい] て 下[くだ]さい 。
\\	ーテニスをしませんか。 ー下手でもいいですか。	ー
\\	ーテニス をしませんか 。 ー 下手[へた] でもいいですか 。
\\	ー何か冷たいものを飲みたいんですが。 ービールでもいいですか。	ー
\\	ー 何か[なにか] 冷た[つめた] いものを 飲み[のみ] たいんですが 。 ービール でもいいですか 。
\\	私はその試験を受けなくてもいい。	
\\	私[わたし] はその 試験[しけん] を 受け[うけ] なくてもいい 。
\\	新しくなくてもいいです。	
\\	新し[あたらし] くなくてもいいです 。
\\	学生じゃなくてもいいです。	
\\	学生[がくせい] じゃなくてもいいです 。
\\	明日休んでもよろしいですか。	
\\	明日[あした] 休ん[やすん] でもよろしいですか 。
\\	たばこを吸っても構いませんか。	
\\	たばこを 吸って[すって] も 構い[かまい] ませんか 。
\\	それは誰がしてもいいです。	
\\	それは 誰が[だれが] してもいいです 。
\\	いくら食べてもいいです。	
\\	いくら 食べて[たべて] もいいです 。
\\	マイクとディックは学生です。	
\\	マイク と ディック は 学生[がくせい] です 。
\\	私は英語と日本語を話す。	
\\	私[わたし] は 英語[えいご] と 日本語[にほんご] を 話す[はなす] 。
\\	ミルズさんは来年ドイツとフランスとスペインへ行くつもりだ。	
\\	ミルズ さんは 来年[らいねん] ドイツ と フランス と スペイン へ 行く[いく]つもりだ 。
\\	ステーキはナイフとフォークで食べる。	
\\	ステーキ は ナイフ と フォーク で 食[た]べる 。
\\	山本さんとスミスさんがテニスをしている。	
\\	山本[やまもと] さんと スミス さんが テニス をしている 。
\\	日本語とトルコ語は似ている。	
\\	日本語[にほんご] と トルコ 語[ご] は 似て[にて] いる 。
\\	山本さんはスミスさんとテニスをしている。	
\\	山本[やまもと] さんは スミス さんと テニス をしている 。
\\	スミスさんは山本さんとテニスをしている。	
\\	スミス さんは 山本[やまもと] さんと テニス をしている 。
\\	ミルズさんやハリスさんが来た。	
\\	ミルズ さんや ハリス さんが 来た[きた] 。
\\	これはスプーンやフォークで食べる。	
\\	これは スプーン や フォーク で 食[た]べる 。
\\	毎朝味噌汁にご飯を食べる。	
\\	毎朝[まいあさ] 味噌汁[みそしる] にご 飯[はん] を 食[た]べる 。
\\	あの人はいつも白いシャツに赤いチョッキを着ている。	
\\	あの 人[ひと] は いつ も 白い[しろい] シャツ に 赤い[あかい] チョッキ を 着て[きて] いる 。
\\	私はアンディーと一緒に広島に行きました。	
\\	私[わたし] は アンディー と 一緒に[いっしょに] 広島[ひろしま] に 行き[いき] ました 。
\\	トムはジョンとよくテニスをする。	
\\	トム は ジョン とよく テニス をする 。
\\	一郎はみどりと結婚した。	
\\	一郎[いちろう] はみどりと 結婚[けっこん] した 。
\\	この車は僕の車と同じです。	
\\	この 車[くるま] は 僕[ぼく] の 車[くるま] と 同じ[おなじ] です 。
\\	日本の胡瓜アメリカの胡瓜と少し違います。	
\\	日本[にほん] の 胡瓜[きゅうり] アメリカ の 胡瓜[きゅうり] と 少し[すこし] 違い[ちがい] ます 。
\\	みどりは一郎と結婚した。	
\\	みどりは 一郎[いちろう] と 結婚[けっこん] した 。
\\	正男は先生と話した。	
\\	正男[まさお] は 先生[せんせい] と 話し[はなし] た 。
\\	正男は先生に話した。	
\\	正男[まさお] は 先生[せんせい] に 話し[はなし] た 。
\\	明は君子とぶつかった。	
\\	明[あきら] は 君子[きみこ] とぶつかった 。
\\	明は君子にぶつかった。	
\\	明[あきら] は 君子[きみこ] にぶつかった 。
\\	明は柱にぶつかった。	
\\	明[あきら] は 柱[はしら] にぶつかった 。
\\	ヒルさんは私に日本語で「こんにちは」と言いました。	
\\	ヒルさんは 私[わたし]に 日本語[にほんご]で「こんにちは」と 言[い]いました。
\\	トムは日本へ行きたいと言っている。	
\\	トム は 日本[にほん] へ 行き[いき] たいと 言って[いって] いる 。
\\	これは十六世紀に立てられたと書いてあります。	
\\	これは 十六[じゅうろく] 世紀[せいき] に 立て[たて] られたと 書い[かい] てあります 。
\\	私はバスで行こうと思う。	
\\	私[わたし] は バス で 行こ[いこ] うと 思う[おもう] 。
\\	子供達はばたばたと走り回った。	
\\	子供達[こどもたち] はばたばたと 走り[はしり] 回った[まわった] 。
\\	真知子はしっかりと私の手を握った。	
\\	真知子[まちこ]はしっかりと 私[わたし]の 手[て]を 握[にぎ]った。
\\	ベンはむっつりと座っている。	
\\	ベン はむっつりと 座って[すわって] いる 。
\\	正男はバタンとドアを閉めた。	
\\	正男[まさお] は バタン と ドア を 閉め[しめ] た 。
\\	埼玉に行くと面白い店がたくさんあります。	
\\	埼玉[さいたま] に 行く[いく] と 面白い[おもしろい] 店[みせ] が たくさん あります 。
\\	それは先生に聞くとすぐ分かった。	
\\	それは 先生[せんせい] に 聞く[きく] とすぐ 分か[わか] った 。
\\	タイヤは古いと危ないですよ。	
\\	タイヤ は 古い[ふるい] と 危な[あぶな] いですよ 。
\\	魚が嫌いだと日本へ行った時困りますか?	
\\	魚[さかな] が 嫌い[きらい] だと 日本[にほん] へ 行[い] った 時[とき] 困り[こまり] ますか 
\\	学生だと割引があります。	
\\	学生[がくせい] だと 割引[わりびき] が あり ます 。
\\	太郎は花子と踊った。	
\\	太郎[たろう] は 花子[はなこ] と 踊った[おどった] 。
\\	これはあれと同じだ。	
\\	これはあれと 同じ[おなじ] だ 。
\\	僕は刺身と天婦羅を食べた。	
\\	僕は[ぼくは] 刺身[さしみ] と 天婦羅[てんぷら] を 食べた[たべた] 。
\\	トムは「こんにちは」と言った。	
\\	トムは「こんにちは」と 言[い]った。
\\	トムは学生だと言った。	
\\	トム は 学生[がくせい] だと 言った[いった] 。
\\	一郎は大丈夫だと思った。	
\\	一郎[いちろう] は 大丈夫[だいじょうぶ] だと 思った[おもった] 。
\\	太郎はばたばたと走った。	
\\	太郎[たろう] はばたばたと 走った[はしった] 。
\\	花子はじっと待っていた。	
\\	花子[はなこ] はじっと 待って[まって] いた 。
\\	北九州へ行くといいレストランがある。	
\\	北九州[きたきゅうしゅう] へ 行く[いく] といい レストラン がある。
\\	ーこの本は田中さんに借りたんです。 ー田中さんと言えばもう病気は治ったのかしら。	ー
\\	ー この 本[ほん] は 田中[たなか] さんに 借り[かり] たんです 。 ー 田中[たなか] さんと 言え[いえ] ばもう 病気[びょうき] は 治った[なおった] のかしら 。
\\	ー次の日曜日は仙台へ行くつもりだ。 ー仙台と言えば、春子が仙台大学の入学試験に通ったそうだ。	ー
\\	ー 次の[つぎの] 日曜日[にちようび] は 仙台[せんだい] へ 行く[いく]つもりだ 。 ー 仙台[せんだい] と 言え[いえ] ば 、 春子[はるこ] が 仙台大学[せんだいだいがく] の 入学試験[にゅうがくしけん] に 通った[とうった] そうだ 。
\\	ー小西君はよく休むねえ。 ーよく休むと言えば、山本君も最近見ませんね。	ー
\\	ー 小西[こにし] 君[くん] はよく 休む[やすむ] ねえ 。 ー よく 休む[やすむ] と 言え[いえ] ば 、 山本[やまもと] 君[くん] も 最近[さいきん] 見ま[みま] せんね 。
\\	よし江ったら私には何も言わないのよ。	
\\	よし 江[え] ったら 私[わたし] には 何も[なにも] 言わ[いわ] ないのよ 。
\\	ーシービーを買いましたよ。 ーシービーって?	ー
\\	ーシービー を 買い[かい] ましたよ 。 ーシービー って 
\\	ー来週から英語を教えるんです。 ー英語教えるって、誰に?	ー
\\	ー 来週[らいしゅう] から 英語[えいご] を 教[おし]え るんです 。 ー 英語[えいご] 教[おし]え るって 、 誰[だれ] に 
\\	「雪国」と言う小説を読みました。	
\\	雪国[ゆきぐに] 
\\	と 言う[いう] 小説[しょうせつ] を 読み[よみ] ました 。
\\	山田さんが入院したと言う知らせを聞いた。	
\\	山田[やまだ] さんが 入院[にゅういん] したと 言う[いう] 知らせ[しらせ] を 聞い[きい] た 。
\\	もっと頑張らなくてはならないと言う気持ちがあります。	
\\	もっと 頑張ら[がんばら] なくてはならないと 言う[いう] 気持ち[きもち] があります 。
\\	「七人の侍」と言う映画を見たことがありますか。	
\\	七人[しちにん] の 侍[さむらい] 
\\	と 言う[いう] 映画[えいが] を 見た[みた] こと がありますか 。
\\	ジューンが日本へ行くと言う話は本当ですか。	
\\	ジューン が 日本[にほん] へ 行く[いく] と 言う[いう] 話[はなし] は 本当[ほんとう] ですか 。
\\	その人に会いたくないと言う気持ちはよく分かります。	
\\	その 人[ひと] に 会[あ] い たくないと 言う[いう] 気持ち[きもち] はよく 分か[わか] ります 。
\\	友達が今日来ると言うことをすっかり忘れていた。	
\\	友達[ともだち] が 今日[きょう] 来る[くる] と 言う[いう] こと をすっかり 忘れ[わすれ] ていた 。
\\	学生の話しによると吉田先生は教え方が非常に上手だと言う。	
\\	学生[がくせい] の 話し[はなし] によると 吉田[よしだ] 先生[せんせい] は 教え方[おしえかた] が 非常に[ひじょうに] 上手[じょうず] だと 言う[いう] 。
\\	私はバッハとかモーツアルトとかベートーベンが好きです。	
\\	私[わたし] は バッハ とか モーツアルト とか ベートーベン が 好き[すき] です 。
\\	疲れた時はお風呂に入るとか早く寝るとかしなさい。	
\\	疲れ[つかれ] た 時[とき] はお 風呂[ふろ] に 入る[はいる] とか 早く[はやく] 寝る[ねる] とかしなさい 。
\\	日本の茶道とか生け花とかいうものをよく知らない。	
\\	日本[にほん] の 茶道[さどう] とか 生け花[いけばな] とかいうものをよく 知ら[しら] ない 。
\\	ーどんな日本の映画を見ましたか。 ー黒沢の「七人の侍」とか「生きる」とか「影武者」を見ました。	ー
\\	ー どんな 日本[にほん] の 映画[えいが] を 見ま[みま] したか 。 ー 黒沢[くろさわ] の 
\\	七人[しちにん] の 侍[さむらい] 
\\	とか 
\\	生き[いき] る 
\\	とか 
\\	影武者[かげむしゃ] 
\\	を 見ま[みま] した 。
\\	ー日曜日にはどんなことをしていますか。 ー音楽を聞くとか、本を読むとかしています。	ー
\\	ー 日曜日[にちようび] にはどんな こと をしていますか 。 ー 音楽[おんがく] を 聞く[きく] とか 、 本[ほん] を 読む[よむ] とかしています 。
\\	佐藤さんは忙しくて行けないとか言っていた。	
\\	佐藤[さとう]さんは 忙[いそが]しくて 行[い]けないとか 言[い]っていた。
\\	野口とか言う人が来た。	
\\	野口[のぐち] とか 言う[いう] 人[ひと] が 来た[きた] 。
\\	函館ではどんなことをしましたか。	
\\	函館[はこだて] ではどんな こと をしましたか 。
\\	子供が部屋を出たり入ったりしている。	
\\	子供[こども] が 部屋[へや] を 出た[でた] り 入[はい] ったりしている 。
\\	私は日本にいた時にお茶を習いました。	
\\	私[わたし]は 日本[にほん]にいた 時[とき]にお 茶[ちゃ]を 習[なら]いました。
\\	ひろしは試験の時風邪を引いた。	
\\	ひろしは 試験[しけん]の 時[とき] 風邪[かぜ]を 引[ひ]いた。
\\	松本さんは朝ご飯を食べる時いつもテレビを見る。	
\\	松本[まつもと] さんは 朝[あさ] ご 飯[はん] を 食[た]べる 時[とき] いつ も テレビ を 見る[みる] 。
\\	テーリーはジュースが安い時にたくさん買っておいた。	
\\	テーリー は ジュース が 安い[やすい] 時に[ときに] たくさん 買って[かって] おいた 。
\\	僕は静かな時しか本を読まない。	
\\	僕は[ぼくは] 静か[しずか] な 時[とき] しか 本[ほん] を 読ま[よま] ない 。
\\	私が大学生だった時妹はまだ三つだった。	
\\	私[わたし] が 大学生[だいがくせい] だった 時[とき] 妹[いもうと] はまだ 三つ[みっつ] だった 。
\\	これは出発の時に渡します。	
\\	これは 出発[しゅっぱつ]の 時[とき]に 渡[わた]します。
\\	メアリーは私が富良野にいた時小樽にいた。	
\\	メアリー は 私[わたし] が 富良野[ふらの] にいた 時[とき] 小樽[おたる] にいた 。
\\	私は学生の時よく勉強した。	
\\	私[わたし] は 学生[がくせい] の 時[とき] よく 勉強[べんきょう] した 。
\\	私は学生の時によく勉強した。	
\\	私[わたし]は 学生[がくせい]の 時[とき]によく 勉強[べんきょう]した。
\\	私は鎌倉にいた時田中先生に会った。	
\\	私[わたし]は 鎌倉[かまくら]にいた 時[とき] 田中[たなか] 先生[せんせい]に 会[あ]った。
\\	肉が高い時は魚も高かった。	
\\	肉[にく] が 高い[たかい] 時[とき] は 魚[さかな] も 高か[たかか] った 。
\\	私はご飯を食べる時手を洗う。	
\\	私[わたし] はご 飯[はん] を 食[た]べる 時[とき] 手[て] を 洗う[あらう] 。
\\	私はご飯を食べた時手を洗う。	
\\	私[わたし]はご 飯[はん]を 食[た]べた 時[とき] 手[て]を 洗[あら]う。
\\	私はご飯を食べる時手を洗った。	
\\	私[わたし]はご 飯[はん]を 食[た]べる 時[とき] 手[て]を 洗[あら]った。
\\	私はご飯を食べた時手を洗った。	
\\	私[わたし]はご 飯[はん]を 食[た]べた 時[とき] 手[て]を 洗[あら]った。
\\	私は沖縄へ行く時鮨を食べるつもりだ。	
\\	私[わたし]は 沖縄[おきなわ]へ 行[い]く 時[とき] 鮨[すし]を 食[た]べるつもりだ。
\\	私は那覇へ行った時鮨を食べるつもりだ。	
\\	私[わたし] は 那覇[なは] へ 行[い] った 時[とき] 鮨[すし] を 食[た]べるつもりだ 。
\\	私は長崎へ行く時鮨を食べた。	
\\	私[わたし] は 長崎[ながさき] へ 行く[いく] 時[とき] 鮨[すし] を 食べた[たべた] 。
\\	私は金沢へ行った時鮨を食べた。	
\\	私[わたし] は 金沢[かなざわ] へ 行[い] った 時[とき] 鮨[すし] を 食べた[たべた] 。
\\	私は別府へ行く時車で行く。	
\\	私[わたし] は 別府[べっぷ] へ 行く[いく] 時[とき] 車[くるま] で 行く[いく] 。
\\	私は奈良へ行く時車で行った。	
\\	私[わたし]は 奈良[なら]へ 行[い]く 時[とき] 車[くるま]で 行[い]った。
\\	私は旭川へ行った時車で行った。	
\\	私[わたし] は 旭川[あさひかわ] へ 行[い] った 時[とき] 車[くるま] で 行[い] った 。
\\	それを見たら驚きますよ。	
\\	それを 見た[みた] ら 驚き[おどろき] ますよ 。
\\	私のうちは駅から歩いて十分の所です。	
\\	私[わたし] の うち は 駅[えき] から 歩い[あるい] て 十分[じゅっぷん] の 所[ところ] です 。
\\	学校はうちからバスで三十分の所です。	
\\	学校[がっこう] は うち から バス で 三十分[さんじゅっぷん] の 所[ところ] です 。
\\	その病院はここから車で十五分の所だ。	
\\	その 病院[びょういん] はここから 車[くるま] で 十五[じゅうご] 分[ふん] の 所[ところ] だ 。
\\	春江は晩ご飯を食べる所です。	
\\	春江[はるえ] は 晩[ばん] ご 飯[はん] を 食[た]べる 所[ところ] です 。
\\	春江は晩ご飯を食べている所だ。	
\\	春江[はるえ] は 晩[ばん] ご 飯[はん] を 食べて[たべて] いる 所[ところ] だ 。
\\	春江は晩ご飯を食べた所です。	
\\	春江[はるえ] は 晩[ばん] ご 飯[はん] を 食べた[たべた] 所[ところ] です 。
\\	春江は晩ご飯を食べていた所だ。	
\\	春江[はるえ] は 晩[ばん] ご 飯[はん] を 食べて[たべて] いた 所[ところ] だ 。
\\	私はもう少しで宿題を忘れる所でした。	
\\	私[わたし] はもう 少し[すこし] で 宿題[しゅくだい] を 忘れ[わすれ] る 所[ところ] でした 。
\\	私は危ない所をジーンに助けて貰った。	
\\	私[わたし] は 危な[あぶな] い 所[ところ] を ジーン に 助け[たすけ] て 貰った[もらった] 。
\\	お仕事中の所をすみません。	
\\	お 仕事中[しごとちゅう] の 所[ところ] をすみません 。
\\	僕は今出かける所です。	
\\	僕は[ぼくは] 今[いま] 出かける[でかける] 所[ところ] です 。
\\	テリーと踊っている所をマーサに見られてしまった。	
\\	テリー と 踊って[おどって] いる 所[ところ] を マーサ に 見ら[みら] れてしまった 。
\\	ご飯を食べ終わった所にまり子がたずねて来た。	
\\	ご 飯[はん] を 食べ[たべ] 終わ[おわ] った 所[ところ] にまり 子[こ] がたずねて 来た[きた] 。
\\	三章まで読んだ所で寝てしまった。	
\\	三[さん] 章[しょう]まで 読[よ]んだ 所[ところ]で 寝[ね]てしまった。
\\	私は危なく溺れる所だった。	
\\	私[わたし]は 危[あぶ]なく 溺[おぼ]れる 所[ところ]だった。
\\	お忙しい所をどうもありがとうございました。	
\\	お 忙し[いそがし] い 所[ところ] をどうもありがとうございました 。
\\	利男は一番大事な所でよく勉強しなかったから何も分からなくなってしまった。	
\\	利男[としお] は 一番大事[いちばんだいじ] な 所[ところ] でよく 勉強[べんきょう] しなかったから 何も[なにも] 分か[わか] らなくなってしまった 。
\\	お休みの所をすみません。	
\\	お 休[やす]みの 所[ところ]をすみません。
\\	僕は電話を掛ける所だ。	
\\	僕は[ぼくは] 電話[でんわ] を 掛け[かけ] る 所[ところ] だ 。
\\	あ、ジョンが走っている。	
\\	あ 、 ジョン が 走って[はしって] いる 。
\\	私は山本先生には一週間前に会ったばかりだ。	
\\	私[わたし]は 山本[やまもと] 先生[せんせい]には 一週間[いっしゅうかん] 前[まえ]に 会[あ]ったばかりだ。
\\	佐藤さんはセールスマンとして採用されました。	
\\	佐藤[さとう] さんは セールスマン として 採用[さいよう] されました 。
\\	私は医者としてあなたに言います。	
\\	私[わたし] は 医者[いしゃ] としてあなたに 言い[いい] ます 。
\\	木村さんとは友達として付合っているだけです。	
\\	木村[きむら] さんとは 友達[ともだち] として 付合[つきあい] っているだけです 。
\\	この部屋は物置として使っている。	
\\	この 部屋[へや] は 物置[ものおき] として 使って[つかって] いる 。
\\	このステーキは日本のステーキとしては安い。	
\\	この ステーキ は 日本[にほん] の ステーキ としては 安い[やすい] 。
\\	小林さんは英語の一年生としては英語が上手だ。	
\\	小林[こばやし] さんは 英語[えいご] の 一年生[いちねんせい] としては 英語[えいご] が 上手[じょうず] だ 。
\\	これは日本のアパートとしては大きい方です。	
\\	これは 日本[にほん]のアパートとしては 大[おお]きい 方[ほう]です。
\\	これは本にしては軽すぎるね。	
\\	これは 本[ほん]にしては 軽[かる]すぎるね。
\\	私は来年結婚するつもりだ。	
\\	私[わたし] は 来年[らいねん] 結婚[けっこん] するつもりだ 。
\\	斎藤さんはまだ若いつもりです。	
\\	斎藤[さいとう] さんはまだ 若い[わかい]つもりです 。
\\	僕はよく読んだつもりです。	
\\	僕[ぼく]はよく 読[よ]んだつもりです。
\\	これはお礼のつもりだ。	
\\	これはお 礼[れい] のつもりだ 。
\\	休みには何をするつもりですか。	
\\	休み[やすみ] には 何を[なにを] するつもりですか 。
\\	僕は大学に行くつもりはない。	
\\	僕は[ぼくは] 大学[だいがく] に 行く[いく]つもりはない 。
\\	あんな人にはもう会わないつもりです。	
\\	あんな 人[ひと] にはもう 会わ[あわ] ないつもりです 。
\\	吉田さんはまだ元気なつもりだ。	
\\	吉田[よしだ] さんはまだ 元気[げんき] なつもりだ 。
\\	話したつもりでしたが、話さなかったんですね。	
\\	話[はな]したつもりでしたが、 話[はな]さなかったんですね。
\\	この仕事は遊びのつもりです。	
\\	この 仕事[しごと]は 遊[あそ]びのつもりです。
\\	それで勉強しているつもりですか。	
\\	それで 勉強[べんきょう] しているつもりですか 。
\\	ー名古屋に行くつもりですか。 ーええ、そのつもりです。	ー
\\	ー 名古屋[なごや]に 行[い]くつもりですか。 ーええ、そのつもりです。
\\	松本さんは食べるつもりだ。	
\\	松本[まつもと] さんは 食[た]べるつもりだ 。
\\	ー今晩の音楽会に行くつもりですか。 ーまだよく分かりませんが、たぶん、行かないつもりです。	ー
\\	ー 今晩[こんばん]の 音楽[おんがく] 会[かい]に 行[い]くつもりですか。 ーまだよく 分[わ]かりませんが、たぶん、 行[い]かないつもりです。
\\	この車はようさそうなので買おうと思いますがいくらですか。	
\\	この 車[くるま]はようさそうなので 買[か]おうと 思[おも]いますがいくらですか。
\\	ステレオの広告を見て買おうと思ったら家内に反対された。	
\\	ステレオ の 広告[こうこく] を 見て[みて] 買おう[かおう] と 思った[おもった] ら 家内[かない] に 反対[はんたい] された 。
\\	アメリカ人ってフットボールが好きですね。	
\\	アメリカ 人[じん] って フットボール が 好き[すき] ですね 。
\\	漢字を覚えるって大変だね。	
\\	漢字[かんじ] を 覚え[おぼえ] るって 大変[たいへん] だね 。
\\	漢字って面白いですよ。	
\\	漢字[かんじ] って 面白い[おもしろい] ですよ 。
\\	日本人ってよく写真を撮りますね。	
\\	日本人[にほんじん] ってよく 写真[しゃしん] を 撮り[とり] ますね 。
\\	外国で暮らすって難しいね。	
\\	外国[がいこく] で 暮ら[くら] すって 難しい[むずかしい] ね 。
\\	木が多いっていいものだね。	
\\	木[き]が 多[おお]いっていいものだね。
\\	家が広いっていいですね。	
\\	家[いえ] が 広い[ひろい] っていいですね 。
\\	健康ってありがたいですね。	
\\	健康[けんこう] ってありがたいですね 。
\\	山崎が医者になったって本当ですか。	
\\	山崎[やまざき] が 医者[いしゃ] になったって 本当[ほんとう] ですか 。
\\	日本の肉が高いってどのぐらい高いんですか。	
\\	日本[にほん] の 肉[にく] が 高い[たかい] ってどのぐらい 高い[たかい] んですか 。
\\	山口さんは先生です。	
\\	山口[やまぐち] さんは 先生[せんせい] です 。
\\	井上さんって変な人ですね。	
\\	井上[いのうえ] さんって 変な人[へんなひと] ですね 。
\\	佐々木さんは踊らないって。	
\\	佐々木[ささき] さんは 踊ら[おどら] ないって 。
\\	今晩は雪が降るって。	
\\	今晩[こんばん] は 雪[ゆき] が 降る[ふる] って 。
\\	僕も行こうかって思いました。	
\\	僕[ぼく] も 行こ[いこ] うかって 思い[おもい] ました 。
\\	林さんは踊らないって言っているよ。	
\\	林[はやし]さんは 踊[おど]らないって 言[い]っているよ。
\\	森さんは踊らないって言った。	
\\	森[もり] さんは 踊ら[おどら] ないって 言った[いった] 。
\\	石川さんは踊らないと言った。	
\\	石川[いしかわ] さんは 踊ら[おどら] ないと 言った[いった] 。
\\	橋本が言っていたけど、中島さんは踊らないって。	
\\	橋本[はしもと] が 言って[いって] いたけど 、 中島[なかしま] さんは 踊ら[おどら] ないって 。
\\	池田さんによると、前田は踊らないそうだ。	
\\	池田[いけだ] さんによると 、 前田[まえだ] は 踊ら[おどら] ないそうだ 。
\\	藤田さんはアメリカにいるうちに英語が上手になりました。	
\\	藤田[ふじた] さんは アメリカ にいるうちに 英語[えいご] が 上手[じょうず] になりました 。
\\	何もしないうちに今年も終わった。	
\\	何も[なにも] しないうちに 今年[ことし] も 終わ[おわ] った 。
\\	走っているうちに御腹が痛くなりました。	
\\	走って[はしって] いるうちに 御腹[おなか] が 痛く[いたく] なりました 。
\\	若いうちに本をたくさん読みなさい。	
\\	若い[わかい] うちに 本[ほん] を たくさん 読み[よみ] なさい 。
\\	雨が降らない内にテニスをして来ます。	
\\	雨[あめ]が 降[ふ]らない 内[うち]にテニスをして 来[き]ます。
\\	忘れない内にいっておきたいことがある。	
\\	忘れ[わすれ] ない 内[うち] にいっておきたいことがある 。
\\	考えている内に分からなくなった。	
\\	考え[かんがえ] ている 内[うち] に 分か[わか] らなくなった 。
\\	働ける内に出来るだけ働きたい。	
\\	働け[はたらけ] る 内[うち] に 出来る[できる] だけ 働き[はたらき] たい 。
\\	温かい内に飲んで下さい。	
\\	温[あたた] かい 内[うち] に 飲ん[のん] で 下[くだ]さい 。
\\	休みの内によく寝ておきます。	
\\	休み[やすみ] の 内[うち] によく 寝て[ねて] おきます 。
\\	子供が学校に行っている内に本を読みます。	
\\	子供[こども] が 学校[がっこう] に 行って[いって] いる 内[うち] に 本[ほん] を 読み[よみ] ます 。
\\	雨が降らない内にテニスをします。	
\\	雨[あめ] が 降ら[ふら] ない 内[うち] に テニス をします 。
\\	冷たい内にビールを飲んで下さい。	
\\	冷た[つめた] い 内[うち] に ビール を 飲ん[のん] で 下[くだ]さい 。
\\	授業の間によく質問をした。	
\\	授業[じゅぎょう] の 間に[まに] よく 質問[しつもん] をした 。
\\	試合の中雨が降って来た。	
\\	試合[しあい] の 中[ちゅう] 雨[あま] が 降って[ふって] 来た[きた] 。
\\	加藤さんは学生です。	
\\	加藤[かとう] さんは 学生[がくせい] です 。
\\	杉田さんは行きますが私は行きません。	
\\	杉田[すぎた] さんは 行き[いき] ますが 私[わたし] は 行き[いき] ません 。
\\	青木さんは今ドイツ語を勉強している。	
\\	青木[あおき] さんは 今[いま] ドイツ 語[ご] を 勉強[べんきょう] している 。
\\	この町には大学が二つある。	
\\	この 町[まち] には 大学[だいがく] が 二つ[ふたつ] ある。
\\	私はビールは飲みますが酒は飲みません。	
\\	私[わたし] は ビール は 飲み[のみ] ますが 酒[さけ] は 飲み[のみ] ません 。
\\	春子には人形を、秋子には絵本を上げた。	
\\	春子[はるこ] には 人形[にんぎょう] を 、 秋[あき] 子[こ] には 絵本[えほん] を 上げ[あげ] た 。
\\	昔々一人のおじいさんが住んでいました。おじいさんはとても優しい人でした。	
\\	昔[むかし]々一 人[にん]のおじいさんが 住[す]んでいました。おじいさんはとても 優[やさ]しい 人[ひと]でした。
\\	太郎はテニスは出来ます。	
\\	太郎[たろう] は テニス は 出来ま[できま] す 。
\\	僕は今日はテニスはしない。	
\\	僕は[ぼくは] 今日は[きょうは] テニス はしない 。
\\	私は昨日福岡へ行かなかった。	
\\	私[わたし] は 昨日[きのう] 福岡[ふくおか] へ 行か[いか] なかった 。
\\	西村さんは昨日は函館へ行かなかった。	
\\	西村[にしむら] さんは 昨日[きのう] は 函館[はこだて] へ 行か[いか] なかった 。
\\	遠藤さんは昨日札幌へは行かなかった。	
\\	遠藤[えんどう] さんは 昨日[きのう] 札幌[さっぽろ] へは 行か[いか] なかった 。
\\	ここではたばこを吸わないで下さい。	
\\	ここではたばこを 吸わ[すわ] ないで 下[くだ]さい 。
\\	三浦さんとはよく会います。	
\\	三浦[みうら] さんとはよく 会[あ] い ます 。
\\	清水さんは私が昨日酒を飲んだことを知っている。	
\\	清水[しみず] さんは 私[わたし] が 昨日[きのう] 酒[さけ] を 飲ん[のん] だことを 知って[しって] いる 。
\\	私が読んだ本は「雪国」です。	
\\	私[わたし] が 読んだ[よんだ] 本[ほん] は 
\\	雪国[ゆきぐに] 
\\	です 。
\\	私も明日のパーティーに行くわ。	
\\	私[わたし] も 明日[あした] の パーティー に 行く[いく] わ 。
\\	大木さんはもう帰りましたわ。	
\\	大木[おおき] さんはもう 帰り[かえり] ましたわ 。
\\	久子はまだ学生ですわ。	
\\	久子[ひさこ] はまだ 学生[がくせい] ですわ 。
\\	この部屋は小さいわ。	
\\	この 部屋[へや] は 小さ[ちいさ] いわ 。
\\	私も行くわね。	
\\	私[わたし] も 行く[いく] わね 。
\\	これは本です。	
\\	これは 本[ほん] です 。
\\	コンサートは八時からだ。	
\\	コンサート は 八[はち] 時[じ] からだ 。
\\	あの人は先生だ。	
\\	あの 人[ひと] は 先生[せんせい] だ 。
\\	私はジョーンズです。	
\\	私[わたし] は ジョーンズ です 。
\\	授業は四時までです。	
\\	授業[じゅぎょう] は 四時[よじ] までです 。
\\	今度の試合はワシントン大学とだ。	
\\	今度[こんど] の 試合[しあい] は ワシントン 大学[だいがく] とだ 。
\\	私は毎朝コーヒーを飲む。家内は紅茶だ。	
\\	私[わたし] は 毎朝[まいあさ] コーヒー を 飲む[のむ] 。 家内[かない] は 紅茶[こうちゃ] だ 。
\\	ー私はハワイへ行きます。 ー私はフロリダです。	ー
\\	ー 私[わたし] は ハワイ へ 行き[いき] ます 。 ー 私[わたし] は フロリダ です 。
\\	このレストランは鮨がおいしい。あのレストランは天婦羅だ。	
\\	この レストラン は 鮨[すし] が おいしい 。 あの レストラン は 天[てん] 婦[ぷ] 羅[ら] だ 。
\\	ヂックはスケートが上手だ。ポールはスキだ。	
\\	ヂック は スケート が 上手[じょうず] だ 。 ポール は スキ だ 。
\\	みつ子はお母さんが病気だ。つとむはお父さんだ。	
\\	みつ 子[こ] はお 母さん[かあさん] が 病気[びょうき] だ 。 つとむはお 父さん[とうさん] だ 。
\\	僕は鰻だ。	
\\	僕[ぼく]は 鰻[うなぎ]だ。
\\	ー僕はケンに頼んだ。 ー僕はアンディーにだ。	ー
\\	ー 僕は[ぼくは] ケン に 頼ん[たのん] だ 。 ー 僕は[ぼくは] アンディー にだ 。
\\	ー僕はアパートで勉強する。 ー僕は図書館だ。	ー
\\	ー 僕は[ぼくは] アパート で 勉強[べんきょう] する 。 ー 僕は[ぼくは] 図書館[としょかん] だ 。
\\	ー僕は病気で休んだ。 ー僕は怪我でだ。	ー
\\	ー 僕は[ぼくは] 病気[びょうき] で 休ん[やすん] だ 。 ー 僕は[ぼくは] 怪我[けが] でだ 。
\\	ー私は一郎と踊りました。 ー私は信男とです。	ー
\\	ー 私[わたし] は 一郎[いちろう] と 踊り[おどり] ました 。 ー 私[わたし] は 信男[のぶお] とです 。
\\	ーそれは何ですか。 ー辞書です。	ー
\\	ー それは 何で[なんで] すか 。 ー 辞書[じしょ] です 。
\\	私は英語が分かります。	
\\	私[わたし] は 英語[えいご] が 分か[わか] ります 。
\\	花子は目がきれいだ。	
\\	花子[はなこ] は 目[め] がきれい だ 。
\\	辞書はウェブスターがいいです。	
\\	辞書[じしょ] は ウェブスター がいいです 。
\\	たかしはテニスが上手だ。	
\\	たかしは テニス が 上手[じょうず] だ 。
\\	私はボイフレンドが欲しい。	
\\	私[わたし] は ボイフレンド が 欲しい[ほしい] 。
\\	僕はフットボールが好きだ。	
\\	僕[ぼく]はフットボールが 好[す]きだ。
\\	象は鼻が長い。	
\\	象[ぞう] は 鼻[はな] が 長い[ながい] 。
\\	私は御腹が空いた。	
\\	私[わたし] は 御[お] 腹[なか] が 空い[すい] た 。
\\	ミラーさんは背が高い。	
\\	ミラー さんは 背が[せが] 高い[たかい] 。
\\	日本はステーキが高い。	
\\	日本[にほん] は ステーキ が 高い[たかい] 。
\\	海は日本海がきれいだ。	
\\	海[うみ] は 日本海[にほんかい] が きれい だ 。
\\	日本の野球はどこが強いですか。	
\\	日本[にほん] の 野球[やきゅう] はどこが 強い[つよい] ですか 。
\\	この部屋に入ってはいけません。	
\\	この 部屋[へや] に 入って[はいって] はいけません 。
\\	教室でものを食べてはいけません。	
\\	教室[きょうしつ] でものを 食べて[たべて] はいけません 。
\\	花子はまだ酒を飲んではいけない。	
\\	花子[はなこ] はまだ 酒[さけ] を 飲ん[のん] ではいけない 。
\\	私はこの言葉が分からない。	
\\	私[わたし] はこの 言葉[ことば] が 分か[わか] らない 。
\\	私にはこの映画の意味が分からない。	
\\	私[わたし] にはこの 映画[えいが] の 意味[いみ] が 分か[わか] らない 。
\\	この問題は難しくて分からない。	
\\	この 問題[もんだい] は 難しく[むずかしく] て 分か[わか] らない 。
\\	明日までにこの仕事が出来るかどうか分かりません。	
\\	明日[あした] までにこの 仕事[しごと] が 出来る[できる] かどうか 分か[わか] りません 。
\\	ー田中さんを知っていますか。 ーええ、知っていますが、あの人はどうもよく分かりません。	ー
\\	ー 田中[たなか] さんを 知って[しって] いますか 。 ー ええ 、 知って[しって] いますが 、 あの 人[ひと] はどうもよく 分か[わか] りません 。
\\	親の死を小さい子供に分からせるのは難しい。	
\\	親[おや] の 死[し] を 小さ[ちいさ] い 子供[こども] に 分か[わか] らせるのは 難しい[むずかしい] 。
\\	ジャックはリンダの気持ちを分かろうとしない。	
\\	ジャックはリンダの 気持[きも]ちを 分[わ]かろうとしない。
\\	友達に聞いて大野の結婚のことを知っていた。	
\\	友達[ともだち]に 聞[き]いて 大野[おおの]の 結婚[けっこん]のことを 知[し]っていた。
\\	よく考えれば分かりますよ。	
\\	よく 考え[かんがえ] れば 分か[わか] りますよ 。
\\	ーいくらか分かりますか。 ーはい、調べれば分かります。	
\\	ー いくらか 分か[わか] りますか 。 ー はい 、 調べ[しらべ] れば 分か[わか] ります 。
\\	ー今晩何をするつもりですか。 ーまだ分かりません。	ー
\\	ー 今晩[こんばん] 何を[なにを] するつもりですか 。 ー まだ 分か[わか] りません 。
\\	毎日三時間も日本語を勉強しているんですか。よく出来るわけですね。	
\\	毎日[まいにち] 三時[さんじ] 間[かん] も 日本語[にほんご] を 勉強[べんきょう] しているんですか 。 よく 出来る[できる] わけですね 。
\\	明日試験ですか。じゃあ今晩忙しいわけですね。	
\\	明日[あした] 試験[しけん] ですか 。 じゃあ 今晩[こんばん] 忙し[いそがし] いわけですね 。
\\	スミスさんは十年間もテニスをしてのだから上手なわけだ。	
\\	スミス さんは 十年[じゅうねん] 間[かん] も テニス をしてのだから 上手[じょうず] なわけだ 。
\\	ー来月から四ヶ月フランスに行きます。 ーすると、六月に帰って来るわけですね。	ー
\\	ー 来月[らいげつ] から 四[よん] ヶ[か] 月[げつ] フランス に 行き[いき] ます 。 ー すると 、 六月[ろくがつ] に 帰って[かえって] 来る[くる] わけですね 。
\\	毎日プールで泳いでいるんですか。丈夫なわけですね。	
\\	毎日[まいにち]プールで 泳[およ]いでいるんですか。 丈夫[じょうぶ]なわけですね。
\\	昨日は三時間しか寝ていない。道理で眠いわけだ。	
\\	昨日[きのう]は 三時間[さんじかん]しか 寝[ね]ていない。 道理[どうり]で 眠[ねむ]いわけだ。
\\	えっ?足立さんが入院したんですか。パーテイーに来なかったわけだ。	
\\	えっ? 足立[あだち]さんが 入院[にゅういん]したんですか。パーテイーに 来[こ]なかったわけだ。
\\	ー山田さんは英語のことは何でも知っています。 ー生き字引と言うわけですか。	ー
\\	ー 山田[やまだ]さんは 英語[えいご]のことは 何[なに]でも 知[し]っています。 ー 生[い]き 字引[じびき]と 言[い]うわけですか。
\\	父の言うことが分からないわけではないが、どうしても医者になりたくない。	
\\	父[ちち]の 言[い]うことが 分[わ]からないわけではないが、どうしても 医者[いしゃ]になりたくない。
\\	会社を止めたそうですが、そのわけを聞かせて下さい。	
\\	会社[かいしゃ]を 止[と]めたそうですが、そのわけを 聞[き]かせて 下[くだ]さい。
\\	ここに来たわけは言えません。	
\\	ここに 来[き]たわけは 言[い]えません。
\\	お母さんと二人だけですか。そういうわけで結婚出来ないんですか。	
\\	お 母[かあ]さんと 二人[ふたり]だけですか。そういうわけで 結婚[けっこん] 出来[でき]ないんですか。
\\	明日は授業がないはずですよ。	
\\	明日[あした] は 授業[じゅぎょう] がないはずですよ 。
\\	あの人は若いはずですよ。	
\\	あの 人[ひと] は 若い[わかい] はずですよ 。
\\	酒屋さんに電話してビールを三ダース持って来て貰った。	
\\	酒屋[さかや]さんに 電話[でんわ]してビールを 三[さん]ダース 持[も]って 来[き]て 貰[もら]った。
\\	あの人は政治家ではなくて政治屋だ。	
\\	あの 人[ひと] は 政治家[せいじか] ではなくて 政治屋[せいじや] だ 。
\\	山本さんやスミスさんがテニスをしている。	
\\	山本[やまもと] さんや スミス さんが テニス をしている 。
\\	僕はビールやワインを飲んだ。	
\\	僕は[ぼくは] ビール や ワイン を 飲ん[のん] だ 。
\\	土田さんは大川さんや鈴木さんに手紙を書いた。	
\\	土田[つちだ] さんは 大川[おおかわ] さんや 鈴木[すずき] さんに 手紙[てがみ] を 書い[かい] た 。
\\	昨日はデパートに行ったり映画を見たりしました。	
\\	昨日[きのう] は デパート に 行[い] った り 映画[えいが] を 見た[みた] りしました 。
\\	学生はジェーンやミッシェルです。	
\\	学生[がくせい] は ジェーン や ミッシェル です 。
\\	ースミスさんはたとえばどんな日本食が好きですか。 ーそうですね。天麩羅とか寿喜焼とか刺身ですね。	ー
\\	ースミスさんはたとえばどんな 日本食[にほんしょく]が 好[す]きですか。 ーそうですね。 天麩羅[てんぷら]とか 寿[す] 喜[き] 焼[やき]とか 刺身[さしみ]ですね。
\\	ー昨日料理で何を食べましたか。 ー天婦羅や寿喜焼や刺身を食べました。	ー
\\	ー 昨日[きのう] 料理[りょうり] で 何を[なにを] 食べ[たべ] ましたか 。 ー 天[てん] 婦[ぷ] 羅[ら] や 寿[す] 喜[き] 焼[やき] や 刺身[さしみ] を 食べ[たべ] ました 。
\\	ベイリーさんはやはり来ませんでしたね。	
\\	ベイリーさんはやはり 来[き]ませんでしたね。
\\	雪子さんは今でもやはりきれいです。	
\\	雪子[ゆきこ]さんは 今[いま]でもやはりきれいです。
\\	お兄さんはテニスが上手ですが、弟さんもやっぱり上手ですよ。	
\\	お 兄[にい]さんはテニスが 上手[じょうず]ですが、 弟[おとうと]さんもやっぱり 上手[じょうず]ですよ。
\\	やはり日本のビールはおいしいですね。	
\\	やはり 日本[にほん] の ビール は おいしい ですね 。
\\	僕はやはり結婚することにしました。	
\\	僕は[ぼくは] やはり 結婚[けっこん] することにしました 。
\\	やはりベイリーさんは来なかったね。	
\\	やはり ベイリー さんは 来な[こな] かったね 。
\\	さすがに日本人はよく働く。	
\\	さすがに 日本人[にほんじん]はよく 働[はたら]く。
\\	さすがアメリカは広い。	
\\	さすがアメリカは 広[ひろ]い。
\\	色々薬を飲んだが結局治らなかった。	
\\	色々[いろいろ] 薬[くすり] を 飲ん[のん] だが 結局[けっきょく] 治ら[なおら] なかった 。
\\	色々薬を飲んだがやはり治らなかった。	
\\	色々[いろいろ] 薬[くすり] を 飲ん[のん] だが やはり 治ら[なおら] なかった 。
\\	やはり行かないことにした。	
\\	やはり 行か[いか] ないことにした 。
\\	結局行かないことにした。	
\\	結局[けっきょく] 行か[いか] ないことにした 。
\\	まさか小川さんが先生になるとは思わなかった。	
\\	まさか 小川[おがわ] さんが 先生[せんせい] になるとは 思わ[おもわ] なかった 。
\\	ートムが入院したそうだ。 ーまさか!昨日テニスをしていたよ。	ー
\\	ートム が 入院[にゅういん] したそうだ 。 ー まさか 
\\	昨日[きのう] テニス をしていたよ 。
\\	この漢字は覚えやすいです。	
\\	この 漢字[かんじ]は 覚[おぼ]えやすいです。
\\	あの人のうちは見つけ易い。	
\\	あの 人[ひと] の うち は 見つ[みつ] け 易い[やすい] 。
\\	佐藤先生は話し易いです。	
\\	佐藤[さとう] 先生[せんせい] は 話し[はなし] 易い[やすい] です 。
\\	このペンは書き易い。	
\\	この ペン は 書き[かき] 易い[やすい] 。
\\	上野さんは一緒に仕事がし易かった。	
\\	上野[うえの] さんは 一緒に[いっしょに] 仕事[しごと] がし 易か[やすか] った 。
\\	豆腐は腐り易い。	
\\	豆腐[とうふ] は 腐り[くさり] 易い[やすい] 。
\\	ーどの先生が話し易いですか。 ー佐藤先生が話し易いです。	ー
\\	ー どの 先生[せんせい] が 話し[はなし] 易い[やすい] ですか 。 ー 佐藤[さとう] 先生[せんせい] が 話し[はなし] 易い[やすい] です 。
\\	坂本さんはたばこを吸わないよ。	
\\	坂本[さかもと] さんはたばこを 吸わ[すわ] ないよ 。
\\	ー日本語は面白いですか。 ーええ、とても面白いですよ。	ー
\\	ー 日本語[にほんご] は 面白い[おもしろい] ですか 。 ー ええ 、 とても 面白い[おもしろい] ですよ 。
\\	ーあの人はそんな本を買わないでしょう。 ーいいえ、買いますよ。	ー
\\	ー あの 人[ひと] はそんな 本[ほん] を 買わ[かわ] ないでしょう 。 ー いいえ 、 買い[かい] ますよ 。
\\	読めよ。	
\\	読め[よめ] よ 。
\\	読みなさいよ。	
\\	読み[よみ] なさいよ 。
\\	読んで下さいよ。	
\\	読んで[よんで] 下[くだ]さい よ 。
\\	この本は面白いですよね。	
\\	この 本[ほん] は 面白い[おもしろい] ですよね 。
\\	アメリカまで十時間は掛かりますよね。	
\\	アメリカ まで 十時[じゅうじかん] 間[かん] は 掛か[かか] りますよね 。
\\	加藤さんはロシア語が出来ますよ、ね?	
\\	加藤[かとう] さんは ロシア 語[ご] が 出来ま[できま] すよ 、 ね 
\\	おいしいですよ。	
\\	おいしい ですよ 。
\\	杉山さんはアメリカへ行くようだ。	
\\	杉山[すぎやま] さんは アメリカ へ 行く[いく] ようだ 。
\\	上田さんはボクシングが好きなようです。	
\\	上田[うえだ] さんは ボクシング が 好き[すき] なようです 。
\\	あの人は田中先生のようだ。	
\\	あの 人[ひと] は 田中[たなか] 先生[せんせい] のようだ 。
\\	ー石井さんはもう帰りましたか。 ーはい、そのようです。	ー
\\	ー 石井[いしい] さんはもう 帰り[かえり] ましたか 。 ー はい 、 そのようです 。
\\	木村さんは昨日お酒を飲んだようだ。	
\\	木村[きむら] さんは 昨日[きのう] お 酒[さけ] を 飲ん[のん] だようだ 。
\\	この問題は学生にはちょっと難しいようだ。	
\\	この 問題[もんだい] は 学生[がくせい] にはちょっと 難しい[むずかしい] ようだ 。
\\	ここは昔学校だったようだ。	
\\	ここは 昔[むかし] 学校[がっこう] だったようだ 。
\\	この酒は水のようだ。	
\\	この 酒[さけ] は 水[みず] のようだ 。
\\	木村さんはまるで酒を飲んだようだ。	
\\	木村[きむら] さんはまるで 酒[さけ] を 飲ん[のん] だようだ 。
\\	あの人はまるで日本人のようです。	
\\	あの 人[ひと] はまるで 日本人[にほんじん] のようです 。
\\	今日田中さんのような人を見ました。	
\\	今日[きょう] 田中[たなか] さんのような 人[ひと] を 見ま[みま] した 。
\\	スミスさんは日本人のように日本語を話します。	
\\	スミス さんは 日本人[にほんじん] のように 日本語[にほんご] を 話し[はなし] ます 。
\\	この本は高いだろう。	
\\	この 本[ほん] は 高い[たかい] だろう 。
\\	この本は高いらしい。	
\\	この 本[ほん] は 高い[たかい] らしい 。
\\	この本は高そうだ。	
\\	この 本[ほん] は 高そ[たかそ] うだ 。
\\	この本は高いようだ。	
\\	この 本[ほん] は 高い[たかい] ようだ 。
\\	あの人は先生だろう。	
\\	あの 人[ひと] は 先生[せんせい] だろう 。
\\	あの人は先生らしい。	
\\	あの 人[ひと] は 先生[せんせい] らしい 。
\\	あの人は先生のようだ。	
\\	あの 人[ひと] は 先生[せんせい] のようだ 。
\\	今日は雨が降るだろう。	
\\	今日は[きょうは] 雨[あめ] が 降る[ふる] だろう 。
\\	今日は雨が降るらしい。	
\\	今日は[きょうは] 雨[あめ] が 降る[ふる] らしい 。
\\	今にも雨が降りそうだ。	
\\	今[いま]にも 雨[あめ]が 降[ふ]りそうだ。
\\	この辺はよく雨が降るようだ。	
\\	この 辺[へん] はよく 雨[あめ] が 降る[ふる] ようだ 。
\\	読めるように字をきれいに書いて下さい。	
\\	読[よ]めるように 字[じ]をきれいに 書[か]いて 下[くだ]さい。
\\	風邪を引かないように気をつけている。	
\\	風邪[かぜ]を 引[ひ]かないように 気[き]をつけている。
\\	僕は分かるようにスミスさんはゆっくり英語を話して呉れた。	
\\	僕[ぼく]は 分[わ]かるようにスミスさんはゆっくり 英語[えいご]を 話[はな]して 呉[く]れた。
\\	遅れないようにタクシーで行きました。	
\\	遅[おく]れないようにタクシーで 行[い]きました。
\\	病気が治るように薬を飲んだ。	
\\	病気[びょうき]が 治[なお]るように 薬[くすり]を 飲[の]んだ。
\\	子供が本を読むように面白そうな本を買って来た。	
\\	子供[こども]が 本[ほん]を 読[よ]むように 面白[おもしろ]そうな 本[ほん]を 買[か]って 来[き]た。
\\	私が言うように書いて下さい。	
\\	私[わたし]が 言[い]うように 書[か]いて 下[くだ]さい。
\\	今日は春になったように暖かい。	
\\	今日[きょう]は 春[はる]になったように 暖[あたた]かい。
\\	ドロシーは日本人のように日本語を話します。	
\\	ドロシーは 日本人[にほんじん]のように 日本語[にほんご]を 話[はな]します。
\\	前に話したように私は来月会社を止めます。	
\\	前[まえ]に 話[はな]したように 私[わたし]は 来月[らいげつ] 会社[かいしゃ]を 止[と]めます。
\\	上田さんのように走れますか。	
\\	上田[うえだ]さんのように 走[はし]れますか。
\\	若い時のようには元気なくなった。	
\\	若[わか]い 時[とき]のようには 元気[げんき]なくなった。
\\	土井さんはいつものように朝六時に起きた。	
\\	土井[どい]さんはいつものように 朝[あさ]六 時[じ]に 起[お]きた。
\\	兄がスポーツが上手なように弟も上手だ。	
\\	兄[あに]がスポーツが 上手[じょうず]なように 弟[おとうと]も 上手[じょうず]だ。
\\	猫が可愛いように犬も可愛い。	
\\	猫[ねこ]が 可愛[かわい]いように 犬[いぬ]も 可愛[かわい]い。
\\	まるで生き返ったように元気になった。	
\\	まるで 生[い]き 返[かえ]ったように 元気[げんき]になった。
\\	あの子はまるで大人のように話す。	
\\	あの 子[こ] はまるで 大人[おとな] のように 話す[はなす] 。
\\	先生は学生に宿題をするように言いました。	
\\	先生[せんせい] は 学生[がくせい] に 宿題[しゅくだい] をするように 言い[いい] ました 。
\\	坂本さんは雪江に図書館の前で待っているように言った。	
\\	坂本[さかもと] さんは 雪[ゆき] 江[え] に 図書館[としょかん] の 前[まえ] で 待って[まって] いるように 言った[いった] 。
\\	すみませんが、小林さんに明日朝九時にここへ来るように言って下さいませんか。	
\\	すみませんが 、 小林[こばやし] さんに 明日[あした] 朝[あさ] 九時[くじ] にここへ 来る[くる] ように 言って[いって] 下[くだ]さい ませんか 。
\\	私は聞こえるようにはっきり言ったんですが...	
\\	私[わたし] は 聞こ[きこ] えるようにはっきり 言った[いった] んですが 
\\	田中さんは山本さんにすぐ来るように言った。	
\\	田中[たなか] さんは 山本[やまもと] さんにすぐ 来る[くる] ように 言った[いった] 。
\\	田中さんは山本さんにすぐ来る頼んだ。	
\\	田中[たなか]さんは 山本[やまもと]さんにすぐ 来[く]る 頼[たの]んだ。
\\	田中さんは山本さんにすぐ来る命じた。	
\\	田中[たなか]さんは 山本[やまもと]さんにすぐ 来[く]る 命[めい]じた。
\\	田中さんは山本さんに「すぐ来て下さい」と言った。	
\\	田中[たなか] さんは 山本[やまもと] さんに 
\\	すぐ 来て[きて] 下[くだ]さい 
\\	と 言った[いった] 。
\\	田中さんは山本さんに「すぐ来い」と命じた。	
\\	田中[たなか]さんは 山本[やまもと]さんに「すぐ 来[こ]い」と 命[めい]じた。
\\	田中さんは山本さんに「すぐ来て下さい」と頼んだ。	
\\	田中[たなか] さんは 山本[やまもと] さんに 
\\	すぐ 来て[きて] 下[くだ]さい 
\\	と 頼ん[たのん] だ 。
\\	ジャクソンさんは日本語が話せるようになりました。	
\\	ジャクソン さんは 日本語[にほんご] が 話せ[はなせ] るようになりました 。
\\	林さんは酒を飲まないようになった。	
\\	林[はやし] さんは 酒[さけ] を 飲[の] ま ないようになった 。
\\	難しい日本語が読めるようになりました。	
\\	難しい[むずかしい] 日本語[にほんご] が 読め[よめ] るようになりました 。
\\	パットは私と話さないようになった。	
\\	パットは 私[わたし]と 話[はな]さないようになった。
\\	もうすぐ面白さが分かるようになりますよ。	
\\	もうすぐ 面白さ[おもしろさ] が 分か[わか] るようになりますよ 。
\\	この道は今通れないようになっている。	
\\	この 道[みち] は 今[いま] 通れ[とおれ] ないようになっている 。
\\	うちの子は最近急に勉強するようになったんですよ。	
\\	うちの 子[こ] は 最近[さいきん] 急に[きゅうに] 勉強[べんきょう] するようになったんですよ 。
\\	漢字が覚えられないようになった。	
\\	漢字[かんじ] が 覚え[おぼえ] られないようになった 。
\\	漢字が覚えられるようにならなかった。	
\\	漢字[かんじ] が 覚え[おぼえ] られるようにならなかった 。
\\	私は大阪に転勤することになりました。	
\\	私[わたし] は 大阪[おおさか] に 転勤[てんきん] することになりました 。
\\	私は英語が話せるようになりました。	
\\	私[わたし] は 英語[えいご] が 話せ[はなせ] るようになりました 。
\\	林さんは酒を飲まなくなった。	
\\	林[はやし] さんは 酒[さけ] を 飲[の] ま なくなった 。
\\	私は毎日運動するようにします。	
\\	私[わたし] は 毎日[まいにち] 運動[うんどう] するようにします 。
\\	出来るだけ日本語で話すようにしています。	
\\	出来る[できる] だけ 日本語[にほんご] で 話す[はなす] ようにしています 。
\\	山川には会わないようにした。	
\\	山川[やまかわ] には 会わ[あわ] ないようにした 。
\\	分からないことは先生に聞くようにしている。	
\\	分か[わか] らないことは 先生[せんせい] に 聞く[きく] ようにしている 。
\\	毎朝八時に来るようにしてください。	
\\	毎朝[まいあさ] 八[はち] 時に[じに] 来る[くる] ようにしてください 。
\\	後ろからもよく見えるようにしました。	
\\	後ろ[うしろ] からもよく 見え[みえ] るようにしました 。
\\	成績がよくなるようにした。	
\\	成績[せいせき] がよくなるようにした 。
\\	ジョンはメアリーが行けるようにした。	
\\	ジョン は メアリー が 行け[いけ] るようにした 。
\\	ジョンは行けるようにした。	
\\	ジョン は 行け[いけ] るようにした 。
\\	小川は山本が勉強出来ないようにした。	
\\	小川[おがわ] は 山本[やまもと] が 勉強[べんきょう] 出来な[できな] いようにした 。
\\	小川は山本が勉強出来るようにしなかった。	
\\	小川[おがわ] は 山本[やまもと] が 勉強[べんきょう] 出来る[できる] ようにしなかった 。
\\	朝早く起きられるようになりました。	
\\	朝早く[あさはやく] 起き[おき] られるようになりました 。
\\	朝早く起きられるようにしています。	
\\	朝早く[あさはやく] 起き[おき] られるようにしています 。
\\	日本語はスペイン語よりも面白いです。	
\\	日本語[にほんご] は スペイン 語[ご] よりも 面白い[おもしろい] です 。
\\	車で行く方がバスで行くより安い。	
\\	車[くるま] で 行く方[ほう] が バス で 行く[いく] より 安い[やすい] 。
\\	私は旅行するよりうちにいたいです。	
\\	私[わたし] は 旅行[りょこう] するより うち にいたいです 。
\\	これより方法はない。	
\\	これより 方法[ほうほう] はない 。
\\	バスで行くよりほかに仕方がありません。	
\\	バス で 行く[いく] よりほかに 仕方[しかた] がありません 。
\\	漢字は平仮名より難しい。	
\\	漢字[かんじ] は 平仮名[ひらがな] より 難しい[むずかしい] 。
\\	林さんは私より早く走れる。	
\\	林[はやし] さんは 私[わたし] より 早く[はやく] 走れ[はしれ] る 。
\\	僕はステーキより魚のほうが好きだ。	
\\	僕は[ぼくは] ステーキ より 魚[さかな] のほうが 好き[すき] だ 。
\\	図書館で勉強するほうがうちで勉強するよりよく出来る。	
\\	図書館[としょかん] で 勉強[べんきょう] するほうが うち で 勉強[べんきょう] するよりよく 出来る[できる] 。
\\	それは赤と言うより茶色に近かった。	
\\	それは 赤[あか] と 言う[いう] より 茶色[ちゃいろ] に 近か[ちかか] った 。
\\	日本語を勉強したかったら、アメリカの学校で勉強するより日本へ行きなさい。	
\\	日本語[にほんご] を 勉強[べんきょう] したかったら 、 アメリカ の 学校[がっこう] で 勉強[べんきょう] するより 日本[にほん] へ 行き[いき] なさい 。
\\	その試験は思ったより易しかった。	
\\	その 試験[しけん]は 思[おも]ったより 易[やさ]しかった。
\\	メキシコは赤道より北にあります。	
\\	メキシコ は 赤道[せきどう] より 北[きた] にあります 。
\\	この線より内側に入ってはいけません。	
\\	この 線[せん]より 内側[うちがわ]に 入[はい]ってはいけません。
\\	三時より前に来て下さい。	
\\	三時[さんじ] より 前[まえ] に 来て[きて] 下[くだ]さい 。
\\	これより先はバスがありません。	
\\	これより 先[さき] は バス がありません 。
\\	八十点より上は合格です。	
\\	八十点[はちじゅってん] より 上[うえ] は 合格[ごうかく] です 。
\\	この線から内側に入ってはいけません。	
\\	この 線[せん] から 内側[うちがわ] に 入って[はいって] はいけません 。
\\	三時より後に来て下さい。	
\\	三時[さんじ] より 後に[あとに] 来て[きて] 下[くだ]さい 。
\\	私は日本歴史を読もうと思います。	
\\	私[わたし] は 日本[にほん] 歴史[れきし] を 読も[よも] うと 思い[おもい] ます 。
\\	私はもう酒を飲むまいと思う。	
\\	私[わたし] はもう 酒[さけ] を 飲む[のむ] まいと 思う[おもう] 。
\\	夏休みに日本アルプスに登と思います。	
\\	夏休み[なつやすみ] に 日本[にほん] アルプス に 登と[のぼと] 思い[おもい] ます 。
\\	森さんはワープロを買おうと思っています。	
\\	森[もり] さんは ワープロ を 買お[かお] うと 思って[おもって] います 。
\\	僕はあの人とはもう話すまいと思う。	
\\	僕は[ぼくは] あの 人[ひと] とはもう 話す[はなす] まいと 思う[おもう] 。
\\	私はその映画を見ようと思います。	
\\	私[わたし] はその 映画[えいが] を 見よ[みよ] うと 思い[おもい] ます 。
\\	車を買えると思う。	
\\	車[くるま] を 買え[かえ] ると 思う[おもう] 。
\\	お金を貰ったら喜ぶだろうと思う。	
\\	お 金[かね] を 貰った[もらった] ら 喜ぶ[よろこぶ] だろうと 思う[おもう] 。
\\	雨に降られると思う。	
\\	雨[あめ] に 降ら[ふら] れると 思う[おもう] 。
\\	たまには先生に褒められようと思う。	
\\	たまには 先生[せんせい]に 褒[ほ]められようと 思[おも]う。
\\	私は漢字を毎日五つずつ覚えます。	
\\	私[わたし] は 漢字[かんじ] を 毎日[まいにち] 五つ[いつつ] ずつ 覚え[おぼえ] ます 。
\\	毎月三冊ずつ本を買っています。	
\\	毎月[まいつき] 三冊[さんさつ] ずつ 本[ほん] を 買って[かって] います 。
\\	ゴルフが少しずつ上手になってきた。	
\\	ゴルフ が 少し[すこし] ずつ 上手[じょうず] になってきた 。
\\	私は子供達に本を二冊ずつ遣った。	
\\	私[わたし] は 子供達[こどもたち] に 本[ほん] を 二冊[にさつ] ずつ 遣った[やった] 。
\\	どのクラスにも女子学生が六人ずついた。	
\\	どの クラス にも 女子学生[じょしがくせい] が 六人[ろくにん] ずついた 。
\\	私は漢字を毎日五つ覚える。	
\\	私[わたし] は 漢字[かんじ] を 毎日[まいにち] 五つ[いつつ] 覚え[おぼえ] る 。
\\	毎月三冊本を買っています。	
\\	毎月[まいつき] 三冊[さんさつ] 本[ほん] を 買って[かって] います 。
\\	こちらは山田さんです。	
\\	こちらは 山田[やまだ] さんです 。
\\	こちのほうが安いよ。	
\\	こちのほうが 安い[やすい] よ 。
\\	お手洗いはこちらです。	
\\	お 手洗い[てあらい] はこちらです 。
\\	こちが僕の車だ。	
\\	こちが 僕[ぼく] の 車[くるま] だ 。
\\	ー昨日車で湖に行って来たよ。 ーその湖には魚がいたかい?	ー
\\	ー 昨日[きのう] 車[くるま] で 湖[みずうみ] に 行って[いって] 来た[きた] よ 。 ー その 湖[みずうみ] には 魚[さかな] がいたかい 
\\	私は辞書をまだ使っている。	
\\	私[わたし] は 辞書[じしょ] をまだ 使って[つかって] いる 。
\\	私は小さな辞書をまだ使っている。	
\\	私[わたし] は 小さ[ちいさ] な 辞書[じしょ] をまだ 使って[つかって] いる 。
\\	私は父が買ってくれた小さな辞書をまだ使っている。	
\\	私[わたし] は 父[ちち] が 買って[かって] くれた 小さ[ちいさ] な 辞書[じしょ] をまだ 使って[つかって] いる 。
\\	私は中学に入った時に父が買ってくれた小さな辞書をまだ使っている。	
\\	私[わたし] は 中学[ちゅうがく] に 入[はい] った時に[ときに] 父[ちち] が 買って[かって] くれた 小さ[ちいさ] な 辞書[じしょ] をまだ 使って[つかって] いる 。
\\	ジョンは今日授業がないと言った。	
\\	ジョン は 今日[きょう] 授業[じゅぎょう] がないと 言った[いった] 。
\\	私は子供が病気だったから行けなかった。	
\\	私[わたし] は 子供[こども] が 病気[びょうき] だったから 行け[いけ] なかった 。
\\	私は鈴木さんが行かなければ行かない。	
\\	私[わたし] は 鈴木[すずき] さんが 行か[いか] なければ 行か[いか] ない 。
\\	山本は雪子が結婚したことを知らない。	
\\	山本[やまもと] は 雪子[ゆきこ] が 結婚[けっこん] したことを 知ら[しら] ない 。
\\	昨日はここにあった本が今日はない。	
\\	昨日[きのう] はここにあった 本[ほん] が 今日は[きょうは] ない 。
\\	宿題がたくさんあったが、疲れていたのですぐ寝てしまった。	
\\	宿題[しゅくだい] が たくさん あったが 、 疲れ[つかれ] ていたのですぐ 寝て[ねて] しまった 。
\\	「将軍」を読むつもりだったが、テレビで見たから読むのを止めた。	
\\	将軍[しょうぐん] 
\\	を 読む[よむ]つもりだったが 、 テレビ で 見た[みた] から 読む[よむ] のを 止め[やめ] た 。
\\	陽子は頭が痛くて仕方がないと言った。	
\\	陽子[ようこ] は 頭[あたま] が 痛く[いたく] て 仕方[しかた] がないと 言った[いった] 。
\\	陽子は京都に行って、前から買おうと思っていた着物を買った。	
\\	陽子[ようこ] は 京都[きょうと] に 行って[いって] 、 前か[ぜんか] ら 買お[かお] うと 思って[おもって] いた 着物[きもの] を 買[か] った 。
\\	リサは来年日本へ行くらしい。	
\\	リサ は 来年[らいねん] 日本[にほん] へ 行く[いく] らしい 。
\\	ボブには日本語は難しすぎるようだ。	
\\	ボブ には 日本語[にほんご] は 難し[むずかし] すぎるようだ 。
\\	あんな所には行きたくないんだ。	
\\	あんな 所[ところ] には 行き[いき] たくないんだ 。
\\	ナンシーは大学を止めるそうだ。	
\\	ナンシー は 大学[だいがく] を 止める[やめる] そうだ 。
\\	人間は考えるあしだとパスカルが言った。	
\\	人間[にんげん]は 考[かんが]えるあしだとパスカルが 言[い]った。
\\	本にペットは老人にいいと書いてあった。	
\\	本[ほん] に ペット は 老人[ろうじん] にいいと 書い[かい] てあった 。
\\	幸子は大学を出た。しかし仕事はなかった。	
\\	幸子[さちこ] は 大学[だいがく] を 出た[でた] 。 しかし 仕事[しごと] はなかった 。
\\	急に胸が苦しくなったんです。だから病院に行きました。	
\\	急に[きゅうに] 胸[むね] が 苦し[くるし] くなったんです 。 だから 病院[びょういん] に 行き[いき] ました 。
\\	明日は朝五時に起きます。五時半の汽車に乗るんです。	
\\	明日[あした] は 朝[あさ] 五時[ごじ] に 起き[おき] ます 。 五時[ごじ] 半[はん] の 汽車[きしゃ] に 乗る[のる] んです 。
\\	私はその晩興奮のあまり寝られなかった。	
\\	私[わたし]はその 晩[ばん] 興奮[こうふん]のあまり 寝[ね]られなかった。
\\	この教科書は文法を重視するあまり面白くないものになってしまった。	
\\	この 教科書[きょうかしょ]は 文法[ぶんぽう]を 重視[じゅうし]するあまり 面白[おもしろ]くないものになってしまった。
\\	秋子は恐怖のあまり声も出なかった。	
\\	秋子[あきこ]は 恐怖[きょうふ]のあまり 声[こえ]も 出[で]なかった。
\\	私は喜びのあまり思わず隣の人に抱きついてしまった。	
\\	私[わたし]は 喜[よろこ]びのあまり 思[おも]わず 隣[となり]の 人[ひと]に 抱[だ]きついてしまった。
\\	日本の英語教育は文法が強調されるあまり会話力の養成が疎かになっているようだ。	
\\	日本[にほん]の 英語[えいご] 教育[きょういく]は 文法[ぶんぽう]が 強調[きょうちょう]されるあまり 会話[かいわ] 力[りょく]の 養成[ようせい]が 疎[おろそ]かになっているようだ。
\\	今度の会合は形式を重んずるあまり内容が乏しくなってしまった。	
\\	今度[こんど]の 会合[かいごう]は 形式[けいしき]を 重[おも]んずるあまり 内容[ないよう]が 乏[とぼ]しくなってしまった。
\\	私は悲しみのあまり涙も出なかった。	
\\	私[わたし]は 悲[かな]しみのあまり 涙[なみだ]も 出[で]なかった。
\\	彼らは心配あまり食事も喉を通らない様子だった。	
\\	彼[かれ]らは 心配[しんぱい]あまり 食事[しょくじ]も 喉[のど]を 通[とお]らない 様子[ようす]だった。
\\	日本語は勉強すればするほど面白くなります。	
\\	日本語[にほんご]は 勉強[べんきょう]すればするほど 面白[おもしろ]くなります。
\\	パーティーは人が多ければ多いほど楽しい。	
\\	パーティーは 人[ひと]が 多[おお]ければ 多[おお]いほど 楽[たの]しい。
\\	機械は簡単であればあるほど壊れにくい。	
\\	機械[きかい]は 簡単[かんたん]であればあるほど 壊[こわ]れにくい。
\\	この酒は飲めば飲むほど美味しくなります。	
\\	この 酒[さけ]は 飲[の]めば 飲[の]むほど 美味[おい]しくなります。
\\	大学はよければよいほど入るのが難しいです。	
\\	大学[だいがく]はよければよいほど 入[はい]るのが 難[むずか]しいです。
\\	アパートは駅に近ければ近いほど家賃が高い。	
\\	アパートは 駅[えき]に 近[ちか]ければ 近[ちか]いほど 家賃[やちん]が 高[たか]い。
\\	普段元気なら元気なほど体に気を付けた方がいいですよ。	
\\	普段[ふだん] 元気[げんき]なら 元気[げんき]なほど 体[からだ]に 気[き]を 付[つ]けた 方[ほう]がいいですよ。
\\	授業料は安ければ安いほどいい。	
\\	授業[じゅぎょう] 料[りょう]は 安[やす]ければ 安[やす]いほどいい。
\\	玄米はかめばかむほど味が出る。	
\\	玄米[げんまい]はかめばかむほど 味[あじ]が 出[で]る。
\\	日本では子供ばかりか大人さえ漫画を読んでいる。	
\\	日本[にほん]では 子供[こども]ばかりか 大人[おとな]さえ 漫画[まんが]を 読[よ]んでいる。
\\	あの人は絵を見て楽しむばかりか、自分でも絵をかく。	
\\	あの 人[ひと]は 絵[え]を 見[み]て 楽[たの]しむばかりか、 自分[じぶん]でも 絵[え]をかく。
\\	アメリカでは大学生ばかりか、中学生、高校生さえ日本語を勉強している。	
\\	アメリカでは 大学生[だいがくせい]ばかりか、 中学生[ちゅうがくせい]、 高校生[こうこうせい]さえ 日本語[にほんご]を 勉強[べんきょう]している。
\\	僕の寮の部屋は狭いばかりか、窓さえないんです。	
\\	僕[ぼく]の 寮[りょう]の 部屋[へや]は 狭[せま]いばかりか、 窓[まど]さえないんです。
\\	トムは漢字が読めないばかりか、平仮名さえ読めない	
\\	トムは 漢字[かんじ]が 読[よ]めないばかりか、 平仮名[ひらがな]さえ 読[よ]めない
\\	父は食べるのが大好きなばかりか、料理はするのも大好きです。	
\\	父[ちち]は 食[た]べるのが 大好[だいす]きなばかりか、 料理[りょうり]はするのも 大好[だいす]きです。
\\	あの人は勉強だけでなくスポーツもよく出来る。	
\\	あの 人[ひと]は 勉強[べんきょう]だけでなくスポーツもよく 出来[でき]る。
\\	この本は面白いばかりではなく、とてもためになる。	
\\	この 本[ほん]は 面白[おもしろ]いばかりではなく、とてもためになる。
\\	私は日本語が話せるどころか、一度も勉強したことがありません。	
\\	私[わたし]は 日本語[にほんご]が 話[はな]せるどころか、一 度[ど]も 勉強[べんきょう]したことがありません。
\\	スミスさんは日本語が書けないどころか、日本語で小説さえ書ける。	
\\	スミスさんは 日本語[にほんご]が 書[か]けないどころか、 日本語[にほんご]で 小説[しょうせつ]さえ 書[か]ける。
\\	クラークさんは日本語が話せるばかりか、韓国語さえ話せる。	
\\	クラークさんは 日本語[にほんご]が 話[はな]せるばかりか、 韓国[かんこく] 語[ご]さえ 話[はな]せる。
\\	ジムは日本語で会話が出来ないどころか、簡単な挨拶も出来ない。	
\\	ジムは 日本語[にほんご]で 会話[かいわ]が 出来[でき]ないどころか、 簡単[かんたん]な 挨拶[あいさつ]も 出来[でき]ない。
\\	この論文は書き直すべきだ。	
\\	この 論文[ろんぶん]は 書き直[かきなお]すべきだ。
\\	そんなことを人に言うべきではありません。	
\\	そんなことを 人[ひと]に 言[い]うべきではありません。
\\	君も来るべきでしたよ。	
\\	君[きみ]も 来[きた]るべきでしたよ。
\\	山田には話すべきじゃなかった。	
\\	山田[やまだ]には 話[はな]すべきじゃなかった。
\\	話すべきことは全部話しました。	
\\	話[はな]すべきことは 全部[ぜんぶ] 話[はな]しました。
\\	自分のことは自分ですべきだ。	
\\	自分[じぶん]のことは 自分[じぶん]ですべきだ。
\\	今、家を買うべきじゃないよ。	
\\	今[いま]、 家[いえ]を 買[か]うべきじゃないよ。
\\	それは課長にも言っておくべきだったね。	
\\	それは 課長[かちょう]にも 言[い]っておくべきだったね。
\\	彼は結婚なんかするべきじゃなかったんだ。	
\\	彼[かれ]は 結婚[けっこん]なんかするべきじゃなかったんだ。
\\	我々はもっと創造的であるべきだ。	
\\	我々[われわれ]はもっと 創造[そうぞう] 的[てき]であるべきだ。
\\	この状態が現実であるべきだ。	
\\	この 状態[じょうたい]が 現実[げんじつ]であるべきだ。
\\	調査の結果、驚くべきことが分かった。	
\\	調査[ちょうさ]の 結果[けっか]、 驚[おどろ]くべきことが 分[わ]かった。
\\	田中は全く軽蔑すべき男だ。	
\\	田中[たなか]は 全[まった]く 軽蔑[けいべつ]すべき 男[おとこ]だ。
\\	あるべき所に記述がない。	
\\	あるべき 所[ところ]に 記述[きじゅつ]がない。
\\	体は年をとっても心は若くあるべきだ。	
\\	体[からだ]は 年[とし]をとっても 心[こころ]は 若[わか]くあるべきだ。
\\	学生は勉強するものだ。	
\\	学生[がくせい]は 勉強[べんきょう]するものだ。
\\	君は勉強するべきだ。	
\\	君[きみ]は 勉強[べんきょう]するべきだ。
\\	君はもっと勉強しなければならない。	
\\	君[きみ]はもっと 勉強[べんきょう]しなければならない。
\\	君はもっと勉強すべきだ。	
\\	君[きみ]はもっと 勉強[べんきょう]すべきだ。
\\	君はもっと勉強した方がいい。	
\\	君[きみ]はもっと 勉強[べんきょう]した 方[ほう]がいい。
\\	ガソリンを十ドル分入れておきました。	
\\	ガソリンを十ドル 分[ぶん] 入[い]れておきました。
\\	私は今日三日分の仕事を片付けた。	
\\	私[わたし]は 今日[きょう] 三日[みっか] 分[ぶん]の 仕事[しごと]を 片付[かたづ]けた。
\\	現金の不足分は小切手で払います。	
\\	現金[げんきん]の 不足[ふそく] 分[ぶん]は 小切手[こぎって]で 払[はら]います。
\\	会議の資料を六人分用意しておいて下さい。	
\\	会議[かいぎ]の 資料[しりょう]を 六[ろく] 人[にん] 分[ぶん] 用意[ようい]しておいて 下[くだ]さい。
\\	ここは後でサインをしますので二行分あけておいて下さい。	
\\	ここは 後[あと]でサインをしますので二 行[ぎょう] 分[ぶん]あけておいて 下[くだ]さい。
\\	私達は四か月分のボーナスをもらった。	
\\	私[わたし] 達[たち]は四か 月[げつ] 分[ぶん]のボーナスをもらった。
\\	トラック三台分のごみが出た。	
\\	トラック 三台[さんだい] 分[ぶん]のごみが 出[で]た。
\\	政府は十万人分の食料を被災地に送った。	
\\	政府[せいふ]は 十万[じゅうまん] 人[にん] 分[ぶん]の 食料[しょくりょう]を 被災[ひさい] 地[ち]に 送[おく]った。
\\	学校当局は諸経費の増加分を授業料の値上げでカバーしようとしている。	
\\	学校[がっこう] 当局[とうきょく]は 諸[しょ] 経費[けいひ]の 増加[ぞうか] 分[ぶん]を 授業[じゅぎょう] 料[りょう]の 値上[ねあ]げでカバーしようとしている。
\\	吉田は鮨を五人前平らげた。	
\\	吉田[よしだ]は 鮨[すし]を 五人前[ごにんまえ] 平[たい]らげた。
\\	日本の経済力は伸びている。だが、いつまで続くかは分からない。	
\\	日本[にほん]の 経済[けいざい] 力[りょく]は 伸[の]びている。だが、いつまで 続[つづ]くかは 分[わ]からない。
\\	都会の生活は便利だ。だが、ストレスが多すぎる。	
\\	都会[とかい]の 生活[せいかつ]は 便利[べんり]だ。だが、ストレスが 多[おお]すぎる。
\\	私は彼女とは初めて会った。だが、前から知っていたような親しみを感じた。	
\\	私[わたし]は 彼女[かのじょ]とは 初[はじ]めて 会[あ]った。だが、 前[まえ]から 知[し]っていたような 親[した]しみを 感[かん]じた。
\\	山本は医者に何度もたばこをやめるように言われた。だが、やめる気はないらしい。	
\\	山本[やまもと]は 医者[いしゃ]に 何[なん] 度[ど]もたばこをやめるように 言[い]われた。だが、やめる 気[き]はないらしい。
\\	あの人には才能がある。だが、その才能を使っていない。	
\\	あの 人[ひと]には 才能[さいのう]がある。だが、その 才能[さいのう]を 使[つか]っていない。
\\	今日の試験のために寝ないで勉強した。だが、さっぱり出来なかった。	
\\	今日[きょう]の 試験[しけん]のために 寝[ね]ないで 勉強[べんきょう]した。だが、さっぱり 出来[でき]なかった。
\\	妹はよく勉強するし、頭もいい。だが、成績はなぜかよくない。	
\\	妹[いもうと]はよく 勉強[べんきょう]するし、 頭[あたま]もいい。だが、 成績[せいせき]はなぜかよくない。
\\	ーあの男はなかなか切れるな。 ーだがね、奴は性格は悪いんだ。	
\\	ーあの 男[おとこ]はなかなか 切[き]れるな。 ーだがね、 奴[やつ]は 性格[せいかく]は 悪[わる]いんだ。
\\	ー仕事はお忙しいでしょう。 ーええ、でも、面白いんです。	ー
\\	ー 仕事[しごと]はお 忙[いそが]しいでしょう。 ーええ、でも、 面白[おもしろ]いんです。
\\	日本へ行きたくて仕方がないんです。ですが、お金がないから、行けません。	
\\	日本[にほん]へ 行[い]きたくて 仕方[しかた]がないんです。ですが、お 金[かね]がないから、 行[い]けません。
\\	今東京に来ていますが、東京は、やはり、活気がありますね。	
\\	今[こん] 東京[とうきょう]に 来[き]ていますが、 東京[とうきょう]は、やはり、 活気[かっき]がありますね。
\\	チェンさんは日本語が苦手だ。テストではいつも半分も出来ない。だからと言って、頭が悪いわけではない。	
\\	チェンさんは 日本語[にほんご]が 苦手[にがて]だ。テストではいつも 半分[はんぶん]も 出来[でき]ない。だからと 言[い]って、 頭[あたま]が 悪[わる]いわけではない。
\\	吉田さんは日本人だ。しかも、小説家だ。だからと言って、日本語が教えられるとは限らない。	
\\	吉田[よしだ]さんは 日本人[にほんじん]だ。しかも、 小説[しょうせつ] 家[か]だ。だからと 言[い]って、 日本語[にほんご]が 教[おし]えられるとは 限[かぎ]らない。
\\	彼は若い。経験も浅い。だからと言って教えられないわけではない。	
\\	彼[かれ]は 若[わか]い。 経験[けいけん]も 浅[あさ]い。だからと 言[い]って 教[おし]えられないわけではない。
\\	ジョンは奥さんのことをちょっとも褒めない。時々口をきかないこともある。だからと言って、奥さんを愛していないわけではない。	
\\	ジョンは 奥[おく]さんのことをちょっとも 褒[ほ]めない。 時々[ときどき] 口[くち]をきかないこともある。だからと 言[い]って、 奥[おく]さんを 愛[あい]していないわけではない。
\\	僕は毎日運動をしている。食べ物にも注意している。しかし、だからと言って、長生きする保証はない。	
\\	僕[ぼく]は 毎日[まいにち] 運動[うんどう]をしている。 食べ物[たべもの]にも 注意[ちゅうい]している。しかし、だからと 言[い]って、 長生[ながい]きする 保証[ほしょう]はない。
\\	日本人は集団行動が好きだと言われる。何をするにも一緒にやる。しかし、だからと言って、個人行動が全くないわけではない。	
\\	日本人[にほんじん]は 集団[しゅうだん] 行動[こうどう]が 好[す]きだと 言[い]われる。 何[なに]をするにも 一緒[いっしょ]にやる。しかし、だからと 言[い]って、 個人[こじん] 行動[こうどう]が 全[まった]くないわけではない。
\\	魚は健康にいい。しかし、だからと言って、魚ばかり食べていたら、体に悪いはずだ。	
\\	魚[さかな]は 健康[けんこう]にいい。しかし、だからと 言[い]って、 魚[さかな]ばかり 食[た]べていたら、 体[からだ]に 悪[わる]いはずだ。
\\	日本語はよく難しい言語だといわれる。文法が複雑だし、漢字を覚えるのも大変だ。だからと言って、外国人が学べないわけではない。	
\\	日本語[にほんご]はよく 難[むずか]しい 言語[げんご]だといわれる。 文法[ぶんぽう]が 複雑[ふくざつ]だし、 漢字[かんじ]を 覚[おぼ]えるのも 大変[たいへん]だ。だからと 言[い]って、 外国[がいこく] 人[じん]が 学[まな]べないわけではない。
\\	君はここに座ってるだけでよい。	
\\	君[きみ]はここに 座[すわ]ってるだけでよい。
\\	私は筆跡を見ただけで誰が書いたか分かった。	
\\	私[わたし]は 筆跡[ひっせき]を 見[み]ただけで 誰[だれ]が 書[か]いたか 分[わ]かった。
\\	この植木は二週間に一度水をやるだけでよい。	
\\	この 植木[うえき]は 二週間[にしゅうかん]に 一度[いちど] 水[みず]をやるだけでよい。
\\	この用紙にサインしていただくだけで結構です。	
\\	この 用紙[ようし]にサインしていただくだけで 結構[けっこう]です。
\\	本当に来るだけでいいんですか。	
\\	本当[ほんとう]に 来[く]るだけでいいんですか。
\\	その会議では私はただ座っているだけでよかった。	
\\	その 会議[かいぎ]では 私[わたし]はただ 座[すわ]っているだけでよかった。
\\	村上君は一週間勉強しただけであの試験に通ったそうだ。	
\\	村上[むらかみ] 君[くん]は 一週間[いっしゅうかん] 勉強[べんきょう]しただけであの 試験[しけん]に 通[とお]ったそうだ。
\\	お金を入れてボタンを押すだけで温かいラーメンが出てくる自動販売機がある。	
\\	お 金[かね]を 入[い]れてボタンを 押[お]すだけで 温[あたた]かいラーメンが 出[で]てくる 自動[じどう] 販売[はんばい] 機[き]がある。
\\	頭金一万円を払うだけで品物をお届けします。	
\\	頭金[あたまきん] 一万[いちまん] 円[えん]を 払[はら]うだけで 品物[しなもの]をお 届[とど]けします。
\\	聞くだけで胸が悪くなるような話だ。	
\\	聞[き]くだけで 胸[むね]が 悪[わる]くなるような 話[はなし]だ。
\\	スミスさんはいつも泥だらけの靴を履いています。	
\\	スミスさんはいつも 泥[どろ]だらけの 靴[くつ]を 履[は]いています。
\\	この作文は間違いだらけです。	
\\	この 作文[さくぶん]は 間違[まちが]いだらけです。
\\	長いこと掃除をしていなかったらしく、床も机の上もほこりだらけだった。	
\\	長[なが]いこと 掃除[そうじ]をしていなかったらしく、 床[ゆか]も 机[つくえ]の 上[うえ]もほこりだらけだった。
\\	病院に担ぎ込んだ時、その男の顔は血だらけだった。	
\\	病院[びょういん]に 担[かつ]ぎ 込[こ]んだ 時[とき]、その 男[おとこ]の 顔[かお]は 血[ち]だらけだった。
\\	泥だらけの足で入ってこないで。	
\\	泥[どろ]だらけの 足[あし]で 入[はい]ってこないで。
\\	借金だらけの生活をしています。	
\\	借金[しゃっきん]だらけの 生活[せいかつ]をしています。
\\	この貝は砂だらけで食べにくい。	
\\	この 貝[かい]は 砂[すな]だらけで 食[た]べにくい。
\\	床の上には札束がゴロゴロしていた。	
\\	床[ゆか]の 上[うえ]には 札束[さつたば]がゴロゴロしていた。
\\	その学生の論文にはいいアイディアがいっぱいあった。	
\\	その 学生[がくせい]の 論文[ろんぶん]にはいいアイディアがいっぱいあった。
\\	私の作文は訂正だらけだった。	
\\	私[わたし]の 作文[さくぶん]は 訂正[ていせい]だらけだった。
\\	僕のズボンは穴だらけだ。	
\\	僕[ぼく]のズボンは 穴[あな]だらけだ。
\\	少年の顔はにきびだらけだった。	
\\	少年[しょうねん]の 顔[かお]はにきびだらけだった。
\\	凸凹だらけの道を三時間も運転して、疲れてしまった。	
\\	凸凹[でこぼこ]だらけの 道[みち]を 三時間[さんじかん]も 運転[うんてん]して、 疲[つか]れてしまった。
\\	この翻訳は一ページで二千円お払いします。	
\\	この 翻訳[ほんやく]は一ページで 二千[にせん] 円[えん]お 払[はら]いします。
\\	このアルバイトは一時間で千円払ってくれます。	
\\	このアルバイトは 一時間[いちじかん]で 千[せん] 円[えん] 払[はら]ってくれます。
\\	昨日は一日で本を五百ページ読んだ。	
\\	昨日[きのう]は 一日[いちにち]で 本[ほん]を 五百[ごひゃく]ページ 読[よ]んだ。
\\	食べて飲んで、五人で、七万円ぐらいでした。	
\\	食[た]べて 飲[の]んで、五人[ごにん]で、七万[ななまん] 円[えん]ぐらいでした。
\\	この林檎は一山で二百円です。	
\\	この 林檎[りんご]は 一山[ひとやま]で 二百[にひゃく] 円[えん]です。
\\	日本経済は今後も成長を続けるであろう。	
\\	日本[にほん] 経済[けいざい]は 今後[こんご]も 成長[せいちょう]を 続[つづ]けるであろう。
\\	このようなブームはもう二度と起こらないであろう。	
\\	このようなブームはもう 二度[にど]と 起[お]こらないであろう。
\\	この次に起こる地震は非常に大きいであろうと予想される。	
\\	この 次[つぎ]に 起[お]こる 地震[じしん]は 非常[ひじょう]に 大[おお]きいであろうと 予想[よそう]される。
\\	その交渉は極めて困難であろう。	
\\	その 交渉[こうしょう]は 極[きわ]めて 困難[こんなん]であろう。
\\	この辺りは昔、湖だったであろうと思われる。	
\\	この 辺[あた]りは 昔[むかし]、 湖[みずうみ]だったであろうと 思[おも]われる。
\\	私達はいつか起こるであろう大地震に対して備えておかなければならない。	
\\	私[わたし] 達[たち]はいつか 起[お]こるであろう 大[だい] 地震[じしん]に 対[たい]して 備[そな]えておかなければならない。
\\	彼はもうここへは来るまい。	
\\	彼[かれ]はもうここへは 来[く]るまい。
\\	「坊っちゃん」は漱石の初期の代表作である。	
\\	坊[ぼ]っちゃん」は 漱石[そうせき]の 初期[しょき]の 代表[だいひょう] 作[さく]である。
\\	今年の水不足は極めて深刻である。	
\\	今年[ことし]の 水不足[みずぶそく]は 極[きわ]めて 深刻[しんこく]である。
\\	人間は考える葦ある。	
\\	人間[にんげん]は 考[かんが]える 葦[あし]ある。
\\	日本語があいまいな言語であるというのは誤りである。	
\\	日本語[にほんご]があいまいな 言語[げんご]であるというのは 誤[あやま]りである。
\\	大型車に一人で乗るのは不経済である。	
\\	大型[おおがた] 車[しゃ]に 一人[ひとり]で 乗[の]るのは 不経済[ふけいざい]である。
\\	彼に何度も手紙を書いたが無駄であった。	
\\	彼[かれ]に 何[なん] 度[ど]も 手紙[てがみ]を 書[か]いたが 無駄[むだ]であった。
\\	日本語では文脈から分かることは原則として言わないのである。	
\\	日本語[にほんご]では 文脈[ぶんみゃく]から 分[わ]かることは 原則[げんそく]として 言[い]わないのである。
\\	彼らが出来ないのは練習していないからである。	
\\	彼[かれ]らが 出来[でき]ないのは 練習[れんしゅう]していないからである。
\\	漱石の代表作の「坊っちゃん」は明治三十九年に書かれた。	
\\	漱石[そうせき]の 代表[だいひょう] 作[さく]の
\\	坊[ぼ]っちゃん」は 明治[めいじ] 三十九年[さんじゅうくねん]に 書[か]かれた。
\\	彼が一級のピアニストであることは疑う余地がない。	
\\	彼[かれ]が 一級[いっきゅう]のピアニストであることは 疑[うたが]う 余地[よち]がない。
\\	これは漱石の書いたものらしい。	
\\	これは 漱石[そうせき]の 書[か]いたものらしい。
\\	この方法の方が効果的であるかもしれない。	
\\	この 方法[ほうほう]の 方[ほう]が 効果[こうか] 的[てき]であるかもしれない。
\\	ジムは日本語の会話が出来ないどころか、簡単なあいさつも出来ない。	
\\	ジムは 日本語[にほんご]の 会話[かいわ]が 出来[でき]ないどころか、 簡単[かんたん]なあいさつも 出来[でき]ない。
\\	スミスさんは日本語の新聞が読めるどころか、平仮名も知らない。	
\\	スミスさんは 日本語[にほんご]の 新聞[しんぶん]が 読[よ]めるどころか、 平仮名[ひらがな]も 知[し]らない。
\\	ジョンソンさんは漢字が書けないどころか、平仮名も書けない。	
\\	ジョンソンさんは 漢字[かんじ]が 書[か]けないどころか、 平仮名[ひらがな]も 書[か]けない。
\\	スミスさんは日本語が書けないどころか、日本語で小説が書けるぐらいだ。	
\\	スミスさんは 日本語[にほんご]が 書[か]けないどころか、 日本語[にほんご]で 小説[しょうせつ]が 書[か]けるぐらいだ。
\\	クラークさんは刺し身が食べられるどころか、納豆まで食べられる。	
\\	クラークさんは 刺し身[さしみ]が 食[た]べられるどころか、 納豆[なっとう]まで 食[た]べられる。
\\	あの人は本を読むどころか、新聞も読まない。	
\\	あの 人[ひと]は 本[ほん]を 読[よ]むどころか、 新聞[しんぶん]も 読[よ]まない。
\\	父は運動をするどころか、家から一歩も出ない。	
\\	父[ちち]は 運動[うんどう]をするどころか、 家[いえ]から 一歩[いっぽ]も 出[で]ない。
\\	今年の八月は暑いどころか、寒かった。	
\\	今年[ことし]の 八月[はちがつ]は 暑[あつ]いどころか、 寒[さむ]かった。
\\	私の父は丈夫などころか、寝たきりです。	
\\	私[わたし]の 父[ちち]は 丈夫[じょうぶ]などころか、 寝[ね]たきりです。
\\	食事はご馳走どころか、豚のえさみたいだった。	
\\	食事[しょくじ]はご 馳走[ちそう]どころか、 豚[ぶた]のえさみたいだった。
\\	この文の意味がどうもよく分からない。	
\\	この 文[ぶん]の 意味[いみ]がどうもよく 分[わ]からない。
\\	鈴木さんはどうも京都大学に入りたいらしい。	
\\	鈴木[すずき]さんはどうも 京都[きょうと] 大学[だいがく]に 入[はい]りたいらしい。
\\	あの先生の授業はどうも面白くない。	
\\	あの 先生[せんせい]の 授業[じゅぎょう]はどうも 面白[おもしろ]くない。
\\	このごろどうも体の調子がよくないんです。	
\\	このごろどうも 体[からだ]の 調子[ちょうし]がよくないんです。
\\	あの人の日本語はどうも聞きにくい。	
\\	あの 人[ひと]の 日本語[にほんご]はどうも 聞[き]きにくい。
\\	先生、この問題の意味がどうもつかめないんです。	
\\	先生[せんせい]、この 問題[もんだい]の 意味[いみ]がどうもつかめないんです。
\\	こんな田舎に住むのはどうも不便だ。	
\\	こんな 田舎[いなか]に 住[す]むのはどうも 不便[ふべん]だ。
\\	うちの子はテレビばかり見て、どうも本を読まない。	
\\	うちの 子[こ]はテレビばかり 見[み]て、どうも 本[ほん]を 読[よ]まない。
\\	あの先生はどうも厳しいようだ。	
\\	あの 先生[せんせい]はどうも 厳[きび]しいようだ。
\\	ジョンは日本語を話すのは上手だが、読むのはどうも下手なようです。	
\\	ジョンは 日本語[にほんご]を 話[はな]すのは 上手[じょうず]だが、 読[よ]むのはどうも 下手[へた]なようです。
\\	その男がどうも犯人に違いないと思っていたが、やっぱりそうだった。	
\\	その 男[おとこ]がどうも 犯人[はんにん]に 違[ちが]いないと 思[おも]っていたが、やっぱりそうだった。
\\	父は症状からしてどうもがんになったらしい。	
\\	父[ちち]は 症状[しょうじょう]からしてどうもがんになったらしい。
\\	彼女が今朝電車の中で僕に言ったことがどうも気になった。	
\\	彼女[かのじょ]が 今朝[けさ] 電車[でんしゃ]の 中[なか]で 僕[ぼく]に 言[い]ったことがどうも 気[き]になった。
\\	先日はどうも。	
\\	先日[せんじつ]はどうも。
\\	どうもすみません。	
\\	どうもすみません。
\\	兄は病気だ治ってどんなにうれしかったことか。	
\\	兄[あに]は 病気[びょうき]だ 治[なお]ってどんなにうれしかったことか。
\\	父は母をどんなに愛していたことか。	
\\	父[ちち]は 母[はは]をどんなに 愛[あい]していたことか。
\\	日本へ行くまで日本がどんなに狭いか分からなかった。	
\\	日本[にほん]へ 行[い]くまで 日本[にほん]がどんなに 狭[せま]いか 分[わ]からなかった。
\\	英夫は由美子に結婚を断られた時にどんなに残念に思ったことか。	
\\	英夫[ひでお]は 由美子[ゆみこ]に 結婚[けっこん]を 断[ことわ]られた 時[とき]にどんなに 残念[ざんねん]に 思[おも]ったことか。
\\	勉強が嫌いだった良夫には大学に入るのがどんなに難しかったか。	
\\	勉強[べんきょう]が 嫌[きら]いだった 良夫[よしお]には 大学[だいがく]に 入[はい]るのがどんなに 難[むずか]しかったか。
\\	圭子はどんなにフランスに留学したかったことか!	
\\	圭子[けいこ]はどんなにフランスに 留学[りゅうがく]したかったことか。
\\	学生達はその厳しい先生をどんなに強く憎んだことか!	
\\	学生[がくせい] 達[たち]はその 厳[きび]しい 先生[せんせい]をどんなに 強[つよ]く 憎[にく]んだことか!
\\	駅前のアパートはどんなに便利だったことか!	
\\	駅前[えきまえ]のアパートはどんなに 便利[べんり]だったことか!
\\	一郎は親のもとを離れたとき、親がどんなに寂しがったが分からなかった。	
\\	一郎[いちろう]は 親[おや]のもとを 離[はな]れたとき、 親[おや]がどんなに 寂[さび]しがったが 分[わ]からなかった。
\\	受けても、どうせ駄目だから、文部省の留学生試験を受けないことにしました。	
\\	受[う]けても、どうせ 駄目[だめ]だから、 文部省[もんぶしょう]の 留学生[りゅうがくせい] 試験[しけん]を 受[う]けないことにしました。
\\	どうせ外国旅行をするんなら、若いうちにした方がいい。	
\\	どうせ 外国[がいこく] 旅行[りょこう]をするんなら、 若[わか]いうちにした 方[ほう]がいい。
\\	どうせこんなたくさんの宿題、出来るわけがないから、遊ぶことにしたよ。	
\\	どうせこんなたくさんの 宿題[しゅくだい]、 出来[でき]るわけがないから、 遊[あそ]ぶことにしたよ。
\\	人間はどうせ死ぬんだから、あくせく働いても仕方がない。	
\\	人間[にんげん]はどうせ 死[し]ぬんだから、あくせく 働[はたら]いても 仕方[しかた]がない。
\\	どうせ捨てられるのに、どうして彼の後を追うの?	
\\	どうせ 捨[す]てられるのに、どうして 彼[かれ]の 後[あと]を 追[お]うの?
\\	どうせあの女性とは結婚出来ないよ。早くあきらめたほうがいいよ。	
\\	どうせあの 女性[じょせい]とは 結婚[けっこん] 出来[でき]ないよ。 早[はや]くあきらめたほうがいいよ。
\\	ー君はこんなことが分からないのかい? ーどうせ私はばかですよ。	ー
\\	ー 君[きみ]はこんなことが 分[わ]からないのかい? ーどうせ 私[わたし]はばかですよ。
\\	どうせ来週日本へ行きますから、その辞書は日本で買います。	
\\	どうせ 来週[らいしゅう] 日本[にほん]へ 行[い]きますから、その 辞書[じしょ]は 日本[にほん]で 買[か]います。
\\	私の家の庭は日本風に大きい石が置いてあります。	
\\	私[わたし]の 家[いえ]の 庭[にわ]は 日本[にほん] 風[ふう]に 大[おお]きい 石[いし]が 置[お]いてあります。
\\	あんな風に勉強していたらいい成績は取れないだろう。	
\\	あんな 風[ふう]に 勉強[べんきょう]していたらいい 成績[せいせき]は 取[と]れないだろう。
\\	私は先生に頭がいいんだからもっと勉強しろという風によく言われた。	
\\	私[わたし]は 先生[せんせい]に 頭[あたま]がいいんだからもっと 勉強[べんきょう]しろという 風[ふう]によく 言[い]われた。
\\	ーこれはどんな風に持ったらいいんですか。 ーこんな風に持ってください。	ー
\\	ーこれはどんな 風[ふう]に 持[も]ったらいいんですか。 ーこんな 風[ふう]に 持[も]ってください。
\\	あんな風に毎日飲んでいたら、きっと病気になるでしょう。	
\\	あんな 風[ふう]に 毎日[まいにち] 飲[の]んでいたら、きっと 病気[びょうき]になるでしょう。
\\	このカレーライスはインド風に、とても辛くしてあります。	
\\	このカレーライスはインド 風[ふう]に、とても 辛[から]くしてあります。
\\	ジェミーが日本風のお辞儀をした時にはびっくりした。	
\\	ジェミーが 日本[にほん] 風[ふう]のお 辞儀[じぎ]をした 時[とき]にはびっくりした。
\\	京子は、今日はお花、あすはダンスという風に、毎日何かを習いに行っている。	
\\	京子[きょうこ]は、 今日[きょう]はお 花[はな]、あすはダンスという 風[ふう]に、 毎日[まいにち] 何[なに]かを 習[なら]いに 行[い]っている。
\\	お母さんは入院なさったという風に人から伺いましたが、いかがですか。	
\\	お 母[かあ]さんは 入院[にゅういん]なさったという 風[ふう]に 人[ひと]から 伺[うかが]いましたが、いかがですか。
\\	教育改善のためなら、いくらお金を使ってもいい、という風になればいいんですがね。	
\\	教育[きょういく] 改善[かいぜん]のためなら、いくらお 金[かね]を 使[つか]ってもいい、という 風[ふう]になればいいんですがね。
\\	田中君は最近授業を休みがちだ。	
\\	田中[たなか] 君[くん]は 最近[さいきん] 授業[じゅぎょう]を 休[やす]みがちだ。
\\	それは日本に短期間滞在した外国人にありがちな誤解だ。	
\\	それは 日本[にほん]に 短期間[たんきかん] 滞在[たいざい]した 外国[がいこく] 人[じん]にありがちな 誤解[ごかい]だ。
\\	私は小さい時病気がちでした。	
\\	私[わたし]は 小[ちい]さい 時[とき] 病気[びょうき]がちでした。
\\	この辺りは昼間留守がちの家が多い。	
\\	この 辺[あた]りは 昼間[ひるま] 留守[るす]がちの 家[か]が 多[おお]い。
\\	夏はややもすると塩分が不足しがちだ。	
\\	夏[なつ]はややもすると 塩分[えんぶん]が 不足[ふそく]しがちだ。
\\	この時計は最近遅れがちだ。	
\\	この 時計[とけい]は 最近[さいきん] 遅[おく]れがちだ。
\\	人はともすると自分の都合のいいように物事を考えがちだ。	
\\	人[ひと]はともすると 自分[じぶん]の 都合[つごう]のいいように 物事[ものごと]を 考[かんが]えがちだ。
\\	若いうちがとかく物事を一途に考えがちだ。	
\\	若[わか]いうちがとかく 物事[ものごと]を 一途[いっと]に 考[かんが]えがちだ。
\\	私は最近週末もうちを空けがちです。	
\\	私[わたし]は 最近[さいきん] 週末[しゅうまつ]もうちを 空[あ]けがちです。
\\	これはアメリカ人の学生が犯しがちな間違いだ。	
\\	これはアメリカ 人[じん]の 学生[がくせい]が 犯[おか]しがちな 間違[まちが]いだ。
\\	明日は曇りがちの天気でしょう。	
\\	明日[あした]は 曇[くも]りがちの 天気[てんき]でしょう。
\\	私のクラスには病気がちの学生が何人かいる。	
\\	私[わたし]のクラスには 病気[びょうき]がちの 学生[がくせい]が 何[なん] 人[にん]かいる。
\\	彼女は遠慮がちに話した。	
\\	彼女[かのじょ]は 遠慮[えんりょ]がちに 話[はな]した。
\\	妙子は最近太りがちだ。	
\\	妙子[たえこ]は 最近[さいきん] 太[ふと]りがちだ。
\\	妙子は最近太りぎみだ。	
\\	妙子[たえこ]は 最近[さいきん] 太[ふと]りぎみだ。
\\	私は今日ちょっと風邪ぎみです。	
\\	私[わたし]は 今日[きょう]ちょっと 風邪[かぜ]ぎみです。
\\	彼のしたことは許しがたい。	
\\	彼[かれ]のしたことは 許[ゆる]しがたい。
\\	この旅行は私にとって忘れがたい思い出になるだろう。	
\\	この 旅行[りょこう]は 私[わたし]にとって 忘[わす]れがたい 思い出[おもいで]になるだろう。
\\	彼の行為は理解しがたい。	
\\	彼[かれ]の 行為[こうい]は 理解[りかい]しがたい。
\\	あの先生は偉すぎて私には近寄りがたい。	
\\	あの 先生[せんせい]は 偉[えら]すぎて 私[わたし]には 近寄[ちかよ]りがたい。
\\	この二つの作品は甲乙付けがたい。	
\\	この 二[ふた]つの 作品[さくひん]は 甲乙[こうおつ] 付[つ]けがたい。
\\	このプロジェクトは成功したとは言いがたい。	
\\	このプロジェクトは 成功[せいこう]したとは 言[い]いがたい。
\\	ジョーンズ氏は得がたい人物だ。	
\\	ジョーンズ 氏[し]は 得[え]がたい 人物[じんぶつ]だ。
\\	我々は彼の犯行に関する動かしがたい証拠をつかんだ。	
\\	我々[われわれ]は 彼[かれ]の 犯行[はんこう]に 関[かん]する 動[うご]かしがたい 証拠[しょうこ]をつかんだ。
\\	私は過去に彼から耐えがたい屈辱を受けた。	
\\	私[わたし]は 過去[かこ]に 彼[かれ]から 耐[た]えがたい 屈辱[くつじょく]を 受[う]けた。
\\	彼のしたことは許せない。	
\\	彼[かれ]のしたことは 許[ゆる]せない。
\\	山田さんの字は読みにくい。	
\\	山田[やまだ]さんの 字[じ]は 読[よ]みにくい。
\\	山田さんの字は読みづらい。	
\\	山田[やまだ]さんの 字[じ]は 読[よ]みづらい。
\\	この戸は開きにくい。	
\\	この 戸[と]は 開[ひら]きにくい。
\\	この木は燃えにくい。	
\\	この 木[き]は 燃[も]えにくい。
\\	このコピーは字が薄くて読みづらい。	
\\	このコピーは 字[じ]が 薄[うす]くて 読[よ]みづらい。
\\	このコピーは字が薄くて読みにくい。	
\\	このコピーは 字[じ]が 薄[うす]くて 読[よ]みにくい。
\\	歯が悪いので堅いステーキは食べづらい。	
\\	歯[は]が 悪[わる]いので 堅[かた]いステーキは 食[た]べづらい。
\\	歯が悪いので堅いステーキは食べにくい。	
\\	歯[は]が 悪[わる]いので 堅[かた]いステーキは 食[た]べにくい。
\\	単語は例文と一緒に覚えれば忘れにくい。	
\\	単語[たんご]は 例文[れいぶん]と 一緒[いっしょ]に 覚[おぼ]えれば 忘[わす]れにくい。
\\	薬を飲んだら、逆に熱が出た。	
\\	薬[くすり]を 飲[の]んだら、 逆[ぎゃく]に 熱[ねつ]が 出[で]た。
\\	叱られると思ったのに、逆にほめられた。	
\\	叱[しか]られると 思[おも]ったのに、 逆[ぎゃく]にほめられた。
\\	会話を強調すると文法が不正確になる。逆に、文法は強調すると話せなくなる。	
\\	会話[かいわ]を 強調[きょうちょう]すると 文法[ぶんぽう]が 不[ふ] 正確[せいかく]になる。 逆[ぎゃく]に、 文法[ぶんぽう]は 強調[きょうちょう]すると 話[はな]せなくなる。
\\	しばらく練習をしなかったら、逆に成績が伸びた。	
\\	しばらく 練習[れんしゅう]をしなかったら、 逆[ぎゃく]に 成績[せいせき]が 伸[の]びた。
\\	寝すぎると、元気にならないで、逆に疲れてしまう。	
\\	寝[ね]すぎると、 元気[げんき]にならないで、 逆[ぎゃく]に 疲[つか]れてしまう。
\\	文句を言ってやろうと思っていたら、逆に文句を言われてしまった。	
\\	文句[もんく]を 言[い]ってやろうと 思[おも]っていたら、 逆[ぎゃく]に 文句[もんく]を 言[い]われてしまった。
\\	一生懸命仕事をしたのに、逆に仲間に嫌われてしまった。	
\\	一生懸命[いっしょうけんめい] 仕事[しごと]をしたのに、 逆[ぎゃく]に 仲間[なかま]に 嫌[きら]われてしまった。
\\	ガールフレンドを喜ばせようとしたのに、逆に怒らせてしまった。	
\\	ガールフレンドを 喜[よろこ]ばせようとしたのに、 逆[ぎゃく]に 怒[おこ]らせてしまった。
\\	その試験に落ちるだろうと思っていたのに、逆に一番で通ってしまった。	
\\	その 試験[しけん]に 落[お]ちるだろうと 思[おも]っていたのに、 逆[ぎゃく]に 一番[いちばん]で 通[とお]ってしまった。
\\	人に親切にすることは一般にいいことだと考えられている。しかし、親切にすることが、逆に人を傷つけることもある。	
\\	人[ひと]に 親切[しんせつ]にすることは 一般[いっぱん]にいいことだと 考[かんが]えられている。しかし、 親切[しんせつ]にすることが、 逆[ぎゃく]に 人[ひと]を 傷[きず]つけることもある。
\\	あの子は優しくすると付け上がる。逆に厳しくするとすぐ泣く。	
\\	あの 子[こ]は 優[やさ]しくすると 付け上[つけあ]がる。 逆[ぎゃく]に 厳[きび]しくするとすぐ 泣[な]く。
\\	いい演技をするためには緊張しすぎてはいけない。しかし、逆にリラックスしすぎてもいい演技は出来ない。	
\\	いい 演技[えんぎ]をするためには 緊張[きんちょう]しすぎてはいけない。しかし、 逆[ぎゃく]にリラックスしすぎてもいい 演技[えんぎ]は 出来[でき]ない。
\\	この音楽は舞台に近いほど聞きやすい。	
\\	この 音楽[おんがく]は 舞台[ぶたい]に 近[ちか]いほど 聞[き]きやすい。
\\	上空に行くほど酸素が薄くなる。	
\\	上空[じょうくう]に 行[い]くほど 酸素[さんそ]が 薄[うす]くなる。
\\	私は難しい仕事ほどやる気が出てくる。	
\\	私[わたし]は 難[むずか]しい 仕事[しごと]ほどやる 気[き]が 出[で]てくる。
\\	子供は小言を言うほど反発するものだ。	
\\	子供[こども]は 小言[こごと]を 言[い]うほど 反発[はんぱつ]するものだ。
\\	私は静かなほど落ち着かない。	
\\	私[わたし]は 静[しず]かなほど 落ち着[おちつ]かない。
\\	駅に近くなるほど家賃が高くなる。	
\\	駅[えき]に 近[ちか]くなるほど 家賃[やちん]が 高[たか]くなる。
\\	運動するほど体の調子が変になる。	
\\	運動[うんどう]するほど 体[からだ]の 調子[ちょうし]が 変[へん]になる。
\\	元気な人ほど無理をしがちだ。	
\\	元気[げんき]な 人[ひと]ほど 無理[むり]をしがちだ。
\\	あまり働かない者ほど不平が多い。	
\\	あまり 働[はたら]かない 者[もの]ほど 不平[ふへい]が 多[おお]い。
\\	私は仕事が難しいほどやる気が出てくる。	
\\	私[わたし]は 仕事[しごと]が 難[むずか]しいほどやる 気[き]が 出[で]てくる。
\\	山田さん以外の人はみんなそのことを知っています。	
\\	山田[やまだ]さん 以外[いがい]の 人[ひと]はみんなそのことを 知[し]っています。
\\	日本語以外に何か外国語が話せますか。	
\\	日本語[にほんご] 以外[いがい]に 何[なに]か 外国[がいこく] 語[ご]が 話[はな]せますか。
\\	私には酒を飲む以外に何も楽しみがない。	
\\	私[わたし]には 酒[さけ]を 飲[の]む 以外[いがい]に 何[なに]も 楽[たの]しみがない。
\\	私は日本酒以外の酒は飲まない。	
\\	私[わたし]は 日本[にほん] 酒[しゅ] 以外[いがい]の 酒[さけ]は 飲[の]まない。
\\	我々の会社では現在オーストラリア以外の国と取引はない。	
\\	我々[われわれ]の 会社[かいしゃ]では 現在[げんざい]オーストラリア 以外[いがい]の 国[くに]と 取引[とりひき]はない。
\\	アメリカ以外の国からもたくさん研究者が来た。	
\\	アメリカ 以外[いがい]の 国[くに]からもたくさん 研究[けんきゅう] 者[しゃ]が 来[き]た。
\\	原因はこれ以外に考えられない。	
\\	原因[げんいん]はこれ 以外[いがい]に 考[かんが]えられない。
\\	私は散歩以外にも毎日軽い運動をしている。	
\\	私[わたし]は 散歩[さんぽ] 以外[いがい]にも 毎日[まいにち] 軽[かる]い 運動[うんどう]をしている。
\\	ここで待っている以外ない。	
\\	ここで 待[ま]っている 以外[いがい]ない。
\\	安い以外に何かいいことがありますか。	
\\	安[やす]い 以外[いがい]に 何[なに]かいいことがありますか。
\\	この文は少し漢字の間違いがあるが、それ以外は完全だ。	
\\	この 文[ぶん]は 少[すこ]し 漢字[かんじ]の 間違[まちが]いがあるが、それ 以外[いがい]は 完全[かんぜん]だ。
\\	従業員以外立入禁止。 
\\	従業[じゅうぎょう] 員[いん] 以外[いがい] 立入禁止[たちいりきんし]。
\\	ほかの日は空いていませんか。	
\\	ほかの 日[ひ]は 空[あ]いていませんか。
\\	ほかは問題ありません。	
\\	ほかは 問題[もんだい]ありません。
\\	ほかに何か質問はありませんか。	
\\	ほかに 何[なに]か 質問[しつもん]はありませんか。
\\	お金以外なら何でも貸してあげるよ。	
\\	お 金[かね] 以外[いがい]なら 何[なに]でも 貸[か]してあげるよ。
\\	お金以外だったら何でも貸してあげるよ。	
\\	お 金[かね] 以外[いがい]だったら 何[なに]でも 貸[か]してあげるよ。
\\	お金以外であれば何でも貸してあげるよ。	
\\	お 金[かね] 以外[いがい]であれば 何[なに]でも 貸[か]してあげるよ。
\\	麻雀以外なら何でもやります。	
\\	麻雀[まーじゃん] 以外[いがい]なら 何[なん]でもやります。
\\	金持ち以外とは付き合わないことにしている。	
\\	金持[かねも]ち 以外[いがい]とは 付き合[つきあ]わないことにしている。
\\	この部屋以外で物を食べないで下さい。	
\\	この 部屋[へや] 以外[いがい]で 物[もの]を 食[た]べないで 下[くだ]さい。
\\	日本に来た以上は、日本語をしっかり勉強したい。	
\\	日本[にほん]に 来[き]た 以上[いじょう]は、 日本語[にほんご]をしっかり 勉強[べんきょう]したい。
\\	学生である以上、勉強すべきだ。	
\\	学生[がくせい]である 以上[いじょう]、 勉強[べんきょう]すべきだ。
\\	日本語を始めた以上、よく話せて、聞けて、読めて、書けるようになるまで頑張ります。	
\\	日本語[にほんご]を 始[はじ]めた 以上[いじょう]、よく 話[はな]せて、 聞[き]けて、 読[よ]めて、 書[か]けるようになるまで 頑張[がんば]ります。
\\	新車を買う以上は出来るだけ燃費のいいのを買いたいです。	
\\	新車[しんしゃ]を 買[か]う 以上[いじょう]は 出来[でき]るだけ 燃費[ねんぴ]のいいのを 買[か]いたいです。
\\	親である以上、子供の教育に関心があるのは当然でしょう。	
\\	親[おや]である 以上[いじょう]、 子供[こども]の 教育[きょういく]に 関心[かんしん]があるのは 当然[とうぜん]でしょう。
\\	もらった以上は、あなたが何と言おうと、私の物です。	
\\	もらった 以上[いじょう]は、あなたが 何[なん]と 言[い]おうと、 私[わたし]の 物[もの]です。
\\	体をよく動かしている以上は、人間の体は衰えないらしい。	
\\	体[からだ]をよく 動[うご]かしている 以上[いじょう]は、 人間[にんげん]の 体[からだ]は 衰[おとろ]えないらしい。
\\	人と約束した以上は、それを守らなければならない。	
\\	人[ひと]と 約束[やくそく]した 以上[いじょう]は、それを 守[まも]らなければならない。
\\	僕が生きている以上、お前に不自由をさせないよ。	
\\	僕[ぼく]が 生[い]きている 以上[いじょう]、お 前[まえ]に 不自由[ふじゆう]をさせないよ。
\\	酒を飲み続けている以上、病気は治らないよ。	
\\	酒[さけ]を 飲[の]み 続[つづ]けている 以上[いじょう]、 病気[びょうき]は 治[なお]らないよ。
\\	日本語のラジオを聞いている以上、日本語を聞く力は低下しないでしょうね。	
\\	日本語[にほんご]のラジオを 聞[き]いている 以上[いじょう]、 日本語[にほんご]を 聞[き]く 力[ちから]は 低下[ていか]しないでしょうね。
\\	アメリカ人の目から見ると、日本の社会はいかにも閉鎖的だ。	
\\	アメリカ 人[じん]の 目[め]から 見[み]ると、 日本[にほん]の 社会[しゃかい]はいかにも 閉鎖[へいさ] 的[てき]だ。
\\	外は雪が降っていて、いかにも寒そうだ。	
\\	外[そと]は 雪[ゆき]が 降[ふ]っていて、いかにも 寒[さむ]そうだ。
\\	彼の書斎の本棚には古今東西の本が詰まっていて、いかにも学者の部屋らしい。	
\\	彼[かれ]の 書斎[しょさい]の 本棚[ほんだな]には 古今[ここん] 東西[とうざい]の 本[ほん]が 詰[つ]まっていて、いかにも 学者[がくしゃ]の 部屋[へや]らしい。
\\	先生は最近いかにもお忙しいよう。	
\\	先生[せんせい]は 最近[さいきん]いかにもお 忙[いそが]しいよう。
\\	日本人は集団行動がいかにも好きではあるが、個人行動をしないわけではない。	
\\	日本人[にほんじん]は 集団[しゅうだん] 行動[こうどう]がいかにも 好[す]きではあるが、 個人[こじん] 行動[こうどう]をしないわけではない。
\\	彼はいかにも紳士であるかのように振る舞っているが、なかなかの策士だ。	
\\	彼[かれ]はいかにも 紳士[しんし]であるかのように 振[ふ]る 舞[ま]っているが、なかなかの 策士[さくし]だ。
\\	彼の発想はいかにも日本的だ。	
\\	彼[かれ]の 発想[はっそう]はいかにも 日本[にほん] 的[てき]だ。
\\	その教授の知識はいかにも百科全書的だ。	
\\	その 教授[きょうじゅ]の 知識[ちしき]はいかにも 百科全書[ひゃっかぜんしょ] 的[てき]だ。
\\	デーヴィッドは文部省の奨学金がもらえて、いかにもうれしそうだった。	
\\	デーヴィッドは 文部省[もんぶしょう]の 奨学[しょうがく] 金[きん]がもらえて、いかにもうれしそうだった。
\\	父は退院して、いかにも元気そうになった。	
\\	父[ちち]は 退院[たいいん]して、いかにも 元気[げんき]そうになった。
\\	彼女は明るく、陽気で、いかにもアメリカ人らしい。	
\\	彼女[かのじょ]は 明[あか]るく、 陽気[ようき]で、いかにもアメリカ 人[じん]らしい。
\\	僕の大学の友人はいかにも金持ちらしく、いつもしゃれた物を着ている。	
\\	僕[ぼく]の 大学[だいがく]の 友人[ゆうじん]はいかにも 金持[かねも]ちらしく、いつもしゃれた 物[もの]を 着[き]ている。
\\	ジョンは恋人と別れて、いかにも落ち込んでいるようだった。	
\\	ジョンは 恋人[こいびと]と 別[わか]れて、いかにも 落ち込[おちこ]んでいるようだった。
\\	みゆきは母を失って、いかにも悲しんでいる様子だった。	
\\	みゆきは 母[はは]を 失[うしな]って、いかにも 悲[かな]しんでいる 様子[ようす]だった。
\\	いかにもおっしゃる通りです。	
\\	いかにもおっしゃる 通[とお]りです。
\\	日本語はいかにも難しい言語ではあるが、マスター出来ないわけではない。	
\\	日本語[にほんご]はいかにも 難[むずか]しい 言語[げんご]ではあるが、マスター 出来[でき]ないわけではない。
\\	彼はいかにも全部分かっているかのように話しているが、その実何も分かっていない。	
\\	彼[かれ]はいかにも 全部[ぜんぶ] 分[わ]かっているかのように 話[はな]しているが、その 実[じつ] 何[なに]も 分[わ]かっていない。
\\	もうよせよ。	
\\	もうよせよ。
\\	ここには十二月三十一日までに払えと書いてある。	
\\	ここには 十二月[じゅうにがつ] 三十一日[さんじゅういちにち]までに 払[はら]えと 書[か]いてある。
\\	黙れ!	
\\	黙[だま]れ!
\\	動くな!	
\\	動[うご]くな!
\\	山中首相は即時退陣せよ!	
\\	山中[やまなか] 首相[しゅしょう]は 即時[そくじ] 退陣[たいじん]せよ!
\\	次の文を英訳せよ。
\\	次[つぎ]の 文[ぶん]を 英訳[えいやく]せよ。 
\\	次の質問に答えよ。
\\	次[つぎ]の 質問[しつもん]に 答[こた]えよ。 
\\	乗るなら飲むな。飲んだら乗るな。	
\\	乗[の]るなら 飲[の]むな。 飲[の]んだら 乗[の]るな。
\\	現金は送るなと書いてある。	
\\	現金[げんきん]は 送[おく]るなと 書[か]いてある。
\\	課長にあまりタクシーは使うなと言われた。	
\\	課長[かちょう]にあまりタクシーは 使[つか]うなと 言[い]われた。
\\	十二月三十一日までに払って下さい。	
\\	十二月[じゅうにがつ] 三十一日[さんじゅういちにち]までに 払[はら]って 下[くだ]さい。
\\	あまりタクシーは使わないでくれないか。	
\\	あまりタクシーは 使[つか]わないでくれないか。
\\	もっと勉強しなさい。	
\\	もっと 勉強[べんきょう]しなさい。
\\	遅刻をしてはいけない。	
\\	遅刻[ちこく]をしてはいけない。
\\	本は一週間以内に返すこと。	
\\	本[ほん]は 一週間[いっしゅうかん] 以内[いない]に 返[かえ]すこと。
\\	閲覧室では話をしないこと。	
\\	閲覧[えつらん] 室[しつ]では 話[はなし]をしないこと。
\\	神田先生は、一方では大学で物理学を教えながら、他方では日本語の研究をなさっている。	
\\	神田[かんだ] 先生[せんせい]は、 一方[いっぽう]では 大学[だいがく]で 物理[ぶつり] 学[がく]を 教[おし]えながら、 他方[たほう]では 日本語[にほんご]の 研究[けんきゅう]をなさっている。
\\	あの男は、一方では静かな日本画を書いたりしているが、他方ではサッカーのような激しいスポーツをしている。	
\\	あの 男[おとこ]は、 一方[いっぽう]では 静[しず]かな 日本[にほん] 画[が]を 書[か]いたりしているが、 他方[たほう]ではサッカーのような 激[はげ]しいスポーツをしている。
\\	氏は、一方で慈善事業をやりながら、他方でかなりあくどい商売をしているという噂だ。	
\\	氏[し]は、 一方[いっぽう]で 慈善[じぜん] 事業[じぎょう]をやりながら、 他方[たほう]でかなりあくどい 商売[しょうばい]をしているという 噂[うわさ]だ。
\\	あの大統領は、一方では減税を約束しておきながら、他方では側近の税金の無駄遣いをあまり重要視していない。	
\\	あの 大統領[だいとうりょう]は、 一方[いっぽう]では 減税[げんぜい]を 約束[やくそく]しておきながら、 他方[たほう]では 側近[そっきん]の 税金[ぜいきん]の 無駄遣[むだづか]いをあまり 重要[じゅうよう] 視[し]していない。
\\	留守番電話は、一方ではかかる方にもかける方にも便利な面もあるが、他方ではお互いに直接話しが出来ないから、不自然な面もある。	
\\	留守番[るすばん] 電話[でんわ]は、 一方[いっぽう]ではかかる 方[ほう]にもかける 方[ほう]にも 便利[べんり]な 面[めん]もあるが、 他方[たほう]ではお 互[たが]いに 直接[ちょくせつ] 話[はな]しが 出来[でき]ないから、 不自然[ふしぜん]な 面[めん]もある。
\\	ストレスは、一方では害になることもあるが、他方では生活のいい刺激にもなるそうだ。	
\\	ストレスは、 一方[いっぽう]では 害[がい]になることもあるが、 他方[たほう]では 生活[せいかつ]のいい 刺激[しげき]にもなるそうだ。
\\	外国生活は、一方では新しい文化に接することが出来て楽しい面もあるが、他方ではカルチャーショックで大変困ることもある。	
\\	外国[がいこく] 生活[せいかつ]は、 一方[いっぽう]では 新[あたら]しい 文化[ぶんか]に 接[せっ]することが 出来[でき]て 楽[たの]しい 面[めん]もあるが、 他方[たほう]ではカルチャーショックで 大変[たいへん] 困[こま]ることもある。
\\	この映画は教育上よくない。	
\\	この 映画[えいが]は 教育[きょういく] 上[じょう]よくない。
\\	便宜上私がこの部屋の鍵を預かっているんです。	
\\	便宜上[べんぎじょう] 私[わたし]がこの 部屋[へや]の 鍵[かぎ]を 預[あず]かっているんです。
\\	時間の制約上細かい説明は省略させていただきます。	
\\	時間[じかん]の 制約[せいやく] 上[じょう] 細[こま]かい 説明[せつめい]は 省略[しょうりゃく]させていただきます。
\\	計算上はこれで正しい。	
\\	計算[けいさん] 上[じょう]はこれで 正[ただ]しい。
\\	健康上の理由で引退することにした。	
\\	健康[けんこう] 上[じょう]の 理由[りゆう]で 引退[いんたい]することにした。
\\	「使用上の注意」をよく読んで下さい。	
\\	使用[しよう] 上[じょう]の 注意[ちゅうい]」をよく 読[よ]んで 下[くだ]さい。
\\	このような行為は道義上許せない。	
\\	このような 行為[こうい]は 道義[どうぎ] 上[じょう] 許[ゆる]せない。
\\	この条件はこれからの取引上極めて不利だ。	
\\	この 条件[じょうけん]はこれからの 取引[とりひき] 上[じょう] 極[きわ]めて 不利[ふり]だ。
\\	仕事の都合上こんな高いマンションに住んでいるんです。	
\\	仕事[しごと]の 都合[つごう] 上[じょう]こんな 高[たか]いマンションに 住[す]んでいるんです。
\\	仕事の関係上、今この町を離れるわけにはいかないんです。	
\\	仕事[しごと]の 関係[かんけい] 上[じょう]、 今[いま]この 町[まち]を 離[はな]れるわけにはいかないんです。
\\	理論上はこうなるはずなのだが、実際どうなるかは分からない。	
\\	理論[りろん] 上[じょう]はこうなるはずなのだが、 実際[じっさい]どうなるかは 分[わ]からない。
\\	法律上は彼の行為は罪にならない。	
\\	法律[ほうりつ] 上[じょう]は 彼[かれ]の 行為[こうい]は 罪[つみ]にならない。
\\	この製品にはデザイン上の欠陥がいくつかある。	
\\	この 製品[せいひん]にはデザイン 上[じょう]の 欠陥[けっかん]がいくつかある。
\\	彼女は一身上の都合で会社を辞めることになった。	
\\	彼女[かのじょ]は 一身上[いっしんじょう]の 都合[つごう]で 会社[かいしゃ]を 辞[や]めることになった。
\\	この部品は製作上いくつかの問題がある。	
\\	この 部品[ぶひん]は 製作[せいさく] 上[じょう]いくつかの 問題[もんだい]がある。
\\	薬を飲んだら、かえって病気がひどくなった。	
\\	薬[くすり]を 飲[の]んだら、かえって 病気[びょうき]がひどくなった。
\\	ー一時間も泳いで、疲れたでしょう。 ーいや、かえって元気になったよ。	ー
\\	ー 一時間[いちじかん]も 泳[およ]いで、 疲[つか]れたでしょう。 ーいや、かえって 元気[げんき]になったよ。
\\	日本へ行ったら日本語が上手になるかと思って、日本へ行ったんですが、日本人と英語でばかり話していたので、かえって、下手になって帰って来ました。	
\\	日本[にほん]へ 行[い]ったら 日本語[にほんご]が 上手[じょうず]になるかと 思[おも]って、 日本[にほん]へ 行[い]ったんですが、 日本人[にほんじん]と 英語[えいご]でばかり 話[はな]していたので、かえって、 下手[へた]になって 帰[かえ]って 来[き]ました。
\\	就職のことで三人の先生にアドバイスをしてもらったのですが、アドバイスが全然違うので、かえって、分からなくなってしまいました。	
\\	就職[しゅうしょく]のことで 三人[さんにん]の 先生[せんせい]にアドバイスをしてもらったのですが、アドバイスが 全然[ぜんぜん] 違[ちが]うので、かえって、 分[わ]からなくなってしまいました。
\\	いわゆる一流大学で勉強するより小さな私立大学で勉強するほうが、かえって、いい教育を受けることが出来る。	
\\	いわゆる 一流[いちりゅう] 大学[だいがく]で 勉強[べんきょう]するより 小[ちい]さな 私立[しりつ] 大学[だいがく]で 勉強[べんきょう]するほうが、かえって、いい 教育[きょういく]を 受[う]けることが 出来[でき]る。
\\	アルコールも適量飲めば、かえって、体にいいそうだ。	
\\	アルコールも 適量[てきりょう] 飲[の]めば、かえって、 体[からだ]にいいそうだ。
\\	あの人は学者というよりはむしろ教育者だと思う。	
\\	あの 人[ひと]は 学者[がくしゃ]というよりはむしろ 教育[きょういく] 者[しゃ]だと 思[おも]う。
\\	今日は涼しいというよりはむしろ寒いです。	
\\	今日[きょう]は 涼[すず]しいというよりはむしろ 寒[さむ]いです。
\\	京都へ行くのは冬よりむしろ春の方がよくありませんか。	
\\	京都[きょうと]へ 行[い]くのは 冬[ふゆ]よりむしろ 春[はる]の 方[ほう]がよくありませんか。
\\	私がここにいる限り心配は無用です。	
\\	私[わたし]がここにいる 限[かぎ]り 心配[しんぱい]は 無用[むよう]です。
\\	田中さんが来ない限りこの会議は始められない。	
\\	田中[たなか]さんが 来[こ]ない 限[かぎ]りこの 会議[かいぎ]は 始[はじ]められない。
\\	これが事実である限り彼は有罪を免れないだろう。	
\\	これが 事実[じじつ]である 限[かぎ]り 彼[かれ]は 有罪[ゆうざい]を 免[まぬか]れないだろう。
\\	今の状態が続く限りプロジェクトは始められない。	
\\	今[いま]の 状態[じょうたい]が 続[つづ]く 限[かぎ]りプロジェクトは 始[はじ]められない。
\\	この事件に関する限り彼は無実だ。	
\\	この 事件[じけん]に 関[かん]する 限[かぎ]り 彼[かれ]は 無実[むじつ]だ。
\\	私の知っている限り彼は正直者です。	
\\	私[わたし]の 知[し]っている 限[かぎ]り 彼[かれ]は 正直[しょうじき] 者[しゃ]です。
\\	その書類は私が読んだ限り誤りはなかった。	
\\	その 書類[しょるい]は 私[わたし]が 読[よ]んだ 限[かぎ]り 誤[あやま]りはなかった。
\\	教育者である限りそんなことは口にすべきではない。	
\\	教育[きょういく] 者[しゃ]である 限[かぎ]りそんなことは 口[くち]にすべきではない。
\\	この試験に通らない限り上級クラスには入れません。	
\\	この 試験[しけん]に 通[とお]らない 限[かぎ]り 上級[じょうきゅう]クラスには 入[い]れません。
\\	事態が変わらない限り今以上の援助は不可能です。	
\\	事態[じたい]が 変[か]わらない 限[かぎ]り 今[いま] 以上[いじょう]の 援助[えんじょ]は 不可能[ふかのう]です。
\\	アメリカ人でない限りこの仕事には就けない。	
\\	アメリカ 人[じん]でない 限[かぎ]りこの 仕事[しごと]には 就[つ]けない。
\\	私達は力の続く限り漕いだ。	
\\	私[わたし] 達[たち]は 力[ちから]の 続[つづ]く 限[かぎ]り 漕[こ]いだ。
\\	出来る限りやってみます。	
\\	出来[でき]る 限[かぎ]りやってみます。
\\	便利なうちは借りておくといいでしょう。	
\\	便利[べんり]なうちは 借[か]りておくといいでしょう。
\\	今度限りで彼のパーティーには行かないつもりだ。	
\\	今度[こんど] 限[かぎ]りで 彼[かれ]のパーティーには 行[い]かない 積[つ]もりだ。
\\	切符は一人二枚限りです。	
\\	切符[きっぷ]は 一人[ひとり] 二枚[にまい] 限[かぎ]りです。
\\	この映画館は今月限りで閉館されます。	
\\	この 映画[えいが] 館[かん]は 今月[こんげつ] 限[かぎ]りで 閉館[へいかん]されます。
\\	今日限りで酒もたばこもやめます。	
\\	今日[きょう] 限[かぎ]りで 酒[さけ]もたばこもやめます。
\\	セールは明日限りです。	
\\	セールは 明日[あす] 限[かぎ]りです。
\\	その場限りの約束はしない方がいい。	
\\	その 場[ば] 限[かぎ]りの 約束[やくそく]はしない 方[ほう]がいい。
\\	貸し出しは一回三冊限りです。	
\\	貸し出[かしだ]しは 一回[いっかい] 三冊[さんさつ] 限[かぎ]りです。
\\	書き直しは一回限りです。	
\\	書き直[かきなお]しは 一回[いっかい] 限[かぎ]りです。
\\	来年日本へ行けるか行けないかまだ分かりません。	
\\	来年[らいねん] 日本[にほん]へ 行[い]けるか 行[い]けないかまだ 分[わ]かりません。
\\	傘を駅に置き忘れたのか事務所に忘れたのか、はっきり覚えていません。	
\\	傘[かさ]を 駅[えき]に 置き忘[おきわす]れたのか 事務所[じむしょ]に 忘[わす]れたのか、はっきり 覚[おぼ]えていません。
\\	大学を出てから就職するか大学院に入るかまだ決めていません。	
\\	大学[だいがく]を 出[で]てから 就職[しゅうしょく]するか 大学院[だいがくいん]に 入[はい]るかまだ 決[き]めていません。
\\	お客さんが肉が好きか魚が好きか、聞いておいて下さい。	
\\	お 客[きゃく]さんが 肉[にく]が 好[す]きか 魚[さかな]が 好[す]きか、 聞[き]いておいて 下[くだ]さい。
\\	夏休みにはヨーロッパを旅行するか、ソウルで仕事をするか、まだ決めていません。	
\\	夏休[なつやす]みにはヨーロッパを 旅行[りょこう]するか、ソウルで 仕事[しごと]をするか、まだ 決[き]めていません。
\\	会議が木曜日だった金曜日だったか、忘れてしまいました。	
\\	会議[かいぎ]が 木曜日[もくようび]だった 金曜日[きんようび]だったか、 忘[わす]れてしまいました。
\\	初めてタマラと会ったのがパーティーでだったか、プールでだったか、覚えていません。	
\\	初[はじ]めてタマラと 会[あ]ったのがパーティーでだったか、プールでだったか、 覚[おぼ]えていません。
\\	木村さんが大学で経済を専攻してのか、政治を専攻したのか、知っていますか。	
\\	木村[きむら]さんが 大学[だいがく]で 経済[けいざい]を 専攻[せんこう]してのか、 政治[せいじ]を 専攻[せんこう]したのか、 知[し]っていますか。
\\	最近は男か、男でないか、分からないような男が多い。	
\\	最近[さいきん]は 男[おとこ]か、 男[おとこ]でないか、 分[わ]からないような 男[おとこ]が 多[おお]い。
\\	会議は月曜か水曜にしてください。	
\\	会議[かいぎ]は 月曜[げつよう]か 水曜[すいよう]にしてください。
\\	春休みは、家にいるか旅行をするか研究所で働くか、まだ決めていません。	
\\	春休[はるやす]みは、 家[いえ]にいるか 旅行[りょこう]をするか 研究所[けんきゅうじょ]で 働[はたら]くか、まだ 決[き]めていません。
\\	今週末には何をしようかな。	
\\	今週[こんしゅう] 末[まつ]には 何[なに]をしようかな。
\\	この問題、君に分かるかな。	
\\	この 問題[もんだい]、 君[くん]に 分[わ]かるかな。
\\	今日は何曜日だったかな。	
\\	今日[きょう]は 何[なに] 曜日[ようび]だったかな。
\\	来学期から日本語を始めようかな。	
\\	来[らい] 学期[がっき]から 日本語[にほんご]を 始[はじ]めようかな。
\\	日本での生活はどうかな。	
\\	日本[にほん]での 生活[せいかつ]はどうかな。
\\	今日の晩ご飯は何かな。	
\\	今日[きょう]の 晩[ばん]ご 飯[はん]は 何[なに]かな。
\\	アパートの家賃は高いかな。	
\\	アパートの 家賃[やちん]は 高[たか]いかな。
\\	先生はお元気かな。	
\\	先生[せんせい]はお 元気[げんき]かな。
\\	音楽会の切符があるんだけど、君は行けるかな。	
\\	音楽[おんがく] 会[かい]の 切符[きっぷ]があるんだけど、 君[きみ]は 行[い]けるかな。
\\	どこへ行きましょうか。	
\\	どこへ 行[い]きましょうか。
\\	どこへ行きますか。	
\\	どこへ 行[い]きますか。
\\	どこへ行こうかな。	
\\	どこへ 行[い]こうかな。
\\	頭のいい人が必ずしも成功するとは限らない。	
\\	頭[あたま]のいい 人[ひと]が 必[かなら]ずしも 成功[せいこう]するとは 限[かぎ]らない。
\\	高い料理が必ずしもおいしいわけではない。	
\\	高[たか]い 料理[りょうり]が 必[かなら]ずしもおいしいわけではない。
\\	記憶力のいい人が必ずしも外国語が上手だとは言えない。	
\\	記憶[きおく] 力[りょく]のいい 人[ひと]が 必[かなら]ずしも 外国[がいこく] 語[ご]が 上手[じょうず]だとは 言[い]えない。
\\	日本人が必ずしもいい日本語の先生だとは思わない。	
\\	日本人[にほんじん]が 必[かなら]ずしもいい 日本語[にほんご]の 先生[せんせい]だとは 思[おも]わない。
\\	結婚しても必ずしも幸福になるとは言えない。	
\\	結婚[けっこん]しても 必[かなら]ずしも 幸福[こうふく]になるとは 言[い]えない。
\\	運動をよくする人が必ずしも長生きするとは限らない。	
\\	運動[うんどう]をよくする 人[ひと]が 必[かなら]ずしも 長生[ながい]きするとは 限[かぎ]らない。
\\	日本へ行った学生が日本語が上手になるかというと、必ずしもそうではない。	
\\	日本[にほん]へ 行[い]った 学生[がくせい]が 日本語[にほんご]が 上手[じょうず]になるかというと、 必[かなら]ずしもそうではない。
\\	お金は人を必ずしも幸福にはしない。	
\\	お 金[かね]は 人[ひと]を 必[かなら]ずしも 幸福[こうふく]にはしない。
\\	優れた研究者が必ずしも優れた教育者であるわけではない。	
\\	優[すぐ]れた 研究[けんきゅう] 者[しゃ]が 必[かなら]ずしも 優[すぐ]れた 教育[きょういく] 者[しゃ]であるわけではない。
\\	良薬は口に苦しと言うが、苦い薬が必ずしもいいとは限らない。	
\\	良薬[りょうやく]は 口[くち]に 苦[にが]しと 言[い]うが、 苦[にが]い 薬[くすり]が 必[かなら]ずしもいいとは 限[かぎ]らない。
\\	めがねをかけて、カメラを下げて、集団で歩いている東洋人が必ずしも日本人ではない。	
\\	めがねをかけて、カメラを 下[さ]げて、 集団[しゅうだん]で 歩[ある]いている 東洋[とうよう] 人[じん]が 必[かなら]ずしも 日本人[にほんじん]ではない。
\\	その夏日本に行った学生は必ずしも日本語が上手にはならなかった。	
\\	その 夏[なつ] 日本[にほん]に 行[い]った 学生[がくせい]は 必[かなら]ずしも 日本語[にほんご]が 上手[じょうず]にはならなかった。
\\	ピアノの名手ウラジミール・ホロビッツの演奏は必ずしも完璧ではなかった。	
\\	ピアノの 名手[めいしゅ]ウラジミール・ホロビッツの 演奏[えんそう]は 必[かなら]ずしも 完璧[かんぺき]ではなかった。
\\	光る物がみんな金ではない。	
\\	光[ひか]る 物[もの]がみんな 金[きん]ではない。
\\	私はいつも元気だというわけじゃない。	
\\	私[わたし]はいつも 元気[げんき]だというわけじゃない。
\\	あの先生の話は全部は分からない。	
\\	あの 先生[せんせい]の 話[はなし]は 全部[ぜんぶ]は 分[わ]からない。
\\	友達はみんな結婚している。	
\\	友達[ともだち]はみんな 結婚[けっこん]している。
\\	宿題は全部やってしまった。	
\\	宿題[しゅくだい]は 全部[ぜんぶ]やってしまった。
\\	私はいつも元気だ。	
\\	私[わたし]はいつも 元気[げんき]だ。
\\	そのことは私には分かりかねますから、部長にお聞きになって下さい。	
\\	そのことは 私[わたし]には 分[わ]かりかねますから、 部長[ぶちょう]にお 聞[き]きになって 下[くだ]さい。
\\	あの男はどんなばかなことでもやりかねない。	
\\	あの 男[おとこ]はどんなばかなことでもやりかねない。
\\	来週はロンドンへ出張しますので、誠に申し訳ございませんが、名古屋での会議には出席できかねます。	
\\	来週[らいしゅう]はロンドンへ 出張[しゅっちょう]しますので、 誠[まこと]に 申し訳[もうしわけ]ございませんが、 名古屋[なごや]での 会議[かいぎ]には 出席[しゅっせき]できかねます。
\\	これだけの書類を一月では処理いたしかねますが。	
\\	これだけの 書類[しょるい]を 一月[ひとつき]では 処理[しょり]いたしかねますが。
\\	大変遺憾に存じますが、ご依頼には応じかねます。	
\\	大変[たいへん] 遺憾[いかん]に 存[ぞん]じますが、ご 依頼[いらい]には 応[おう]じかねます。
\\	こんな高価な贈り物、頂きかねます。	
\\	こんな 高価[こうか]な 贈り物[おくりもの]、 頂[いただ]きかねます。
\\	彼は私の気持ちを量りかねているようだ。	
\\	彼[かれ]は 私[わたし]の 気持[きも]ちを 量[はか]りかねているようだ。
\\	非常にいいにくいことなので、さすがの部長も切り出しかねている。	
\\	非常[ひじょう]にいいにくいことなので、さすがの 部長[ぶちょう]も 切り出[きりだ]しかねている。
\\	みどりさん、ご主人がお待ちかねよ。	
\\	みどりさん、ご 主人[しゅじん]がお 待[ま]ちかねよ。
\\	あいつはとんでもないことを言いかねないから、注意した方がいいよ。	
\\	あいつはとんでもないことを 言[い]いかねないから、 注意[ちゅうい]した 方[ほう]がいいよ。
\\	あの男はちょっとしたことで暴力を振るいかねない。	
\\	あの 男[おとこ]はちょっとしたことで 暴力[ぼうりょく]を 振[ふ]るいかねない。
\\	この事件は内閣総辞職にも発展しかねない。	
\\	この 事件[じけん]は 内閣[ないかく] 総[そう] 辞職[じしょく]にも 発展[はってん]しかねない。
\\	暴動すら起こりかねないような緊迫した情況だった。	
\\	暴動[ぼうどう]すら 起[お]こりかねないような 緊迫[きんぱく]した 情況[じょうきょう]だった。
\\	そのご依頼には、応じかねます。	
\\	そのご 依頼[いらい]には、 応[おう]じかねます。
\\	この仕事は引き受けかねます。	
\\	この 仕事[しごと]は 引[ひ]き 受[う]けかねます。
\\	そのような条件には応じかねる。	
\\	そのような 条件[じょうけん]には 応[おう]じかねる。
\\	そいつはちょっと賛成しにくいな。	
\\	そいつはちょっと 賛成[さんせい]しにくいな。
\\	スミスさんは週刊誌から学術書に至るまで、幅広い日本語が読める。	
\\	スミスさんは 週刊[しゅうかん] 誌[し]から 学術[がくじゅつ] 書[しょ]に 至[いた]るまで、 幅広[はばひろ]い 日本語[にほんご]が 読[よ]める。
\\	その新聞記者は首相の公の生活から私生活に至るまで、何でも知っている。	
\\	その 新聞[しんぶん] 記者[きしゃ]は 首相[しゅしょう]の 公[おおやけ]の 生活[せいかつ]から 私生活[しせいかつ]に 至[いた]るまで、 何[なに]でも 知[し]っている。
\\	私は日本へ行った時、友人の山田さんは空港への出迎えからホテルの予約に至るまで、実に親切にしてくれた。	
\\	私[わたし]は 日本[にほん]へ 行[い]った 時[とき]、 友人[ゆうじん]の 山田[やまだ]さんは 空港[くうこう]への 出迎[でむか]えからホテルの 予約[よやく]に 至[いた]るまで、 実[じつ]に 親切[しんせつ]にしてくれた。
\\	その女の人は私に家族のことから自分の悩みに至るまで、細かに話した。	
\\	その 女[おんな]の 人[ひと]は 私[わたし]に 家族[かぞく]のことから 自分[じぶん]の 悩[なや]みに 至[いた]るまで、 細[こま]かに 話[はな]した。
\\	社長が現れた時には、守衛から副社長に至るまで、門の前で待っていた。	
\\	社長[しゃちょう]が 現[あらわ]れた 時[とき]には、 守衛[しゅえい]から 副[ふく] 社長[しゃちょう]に 至[いた]るまで、 門[もん]の 前[まえ]で 待[ま]っていた。
\\	ルーシーは靴から帽子に至るまで、緑の装束だった。	
\\	ルーシーは 靴[くつ]から 帽子[ぼうし]に 至[いた]るまで、 緑[みどり]の 装束[しょうぞく]だった。
\\	東京から京都まで新幹線で三時間かかる。	
\\	東京[とうきょう]から 京都[きょうと]まで 新幹線[しんかんせん]で 三時間[さんじかん]かかる。
\\	昨日は朝九時から十一時ごろまでテニスをした。	
\\	昨日[きのう]は 朝[あさ]九 時[じ]から十一 時[じ]ごろまでテニスをした。
\\	今年の夏、七月から八月にかけて中国大陸を旅行した。	
\\	今年[ことし]の 夏[なつ]、 七月[しちがつ]から 八月[はちがつ]にかけて 中国[ちゅうごく] 大陸[たいりく]を 旅行[りょこう]した。
\\	今週は木曜から金曜にかけて雪が降るでしょう。	
\\	今週[こんしゅう]は 木曜[もくよう]から 金曜[きんよう]にかけて 雪[ゆき]が 降[ふ]るでしょう。
\\	日本は六月から七月にかけて梅雨が続く。	
\\	日本[にほん]は 六月[ろくがつ]から 七月[しちがつ]にかけて 梅雨[つゆ]が 続[つづ]く。
\\	高気圧が朝鮮半島から九州にかけて張り出している。	
\\	高気圧[こうきあつ]が 朝鮮半島[ちょうせんはんとう]から 九州[きゅうしゅう]にかけて 張[は]り 出[だ]している。
\\	午前九時から午後五時まで水道が止まります。	
\\	午前[ごぜん] 九時[くじ]から 午後[ごご] 五時[ごじ]まで 水道[すいどう]が 止[と]まります。
\\	は新宿駅から渋谷駅まで不通になっています。	
\\	は 新宿[しんじゅく] 駅[えき]から 渋谷[しぶや] 駅[えき]まで 不通[ふつう]になっています。
\\	子供だからと言って許すわけにはいかない。	
\\	子供[こども]だからと 言[い]って 許[ゆる]すわけにはいかない。
\\	何も不平を言わないからと言って現状に満足しているわけではない。	
\\	何[なに]も 不平[ふへい]を 言[い]わないからと 言[い]って 現状[げんじょう]に 満足[まんぞく]しているわけではない。
\\	試験に受からなかったからと言ってそんなに悲観することはない。	
\\	試験[しけん]に 受[う]からなかったからと 言[い]ってそんなに 悲観[ひかん]することはない。
\\	弁償したからと言って済む問題ではない。	
\\	弁償[べんしょう]したからと 言[い]って 済[す]む 問題[もんだい]ではない。
\\	上司の命令だからと言って黙って従うわけにはいかない。	
\\	上司[じょうし]の 命令[めいれい]だからと 言[い]って 黙[だま]って 従[したが]うわけにはいかない。
\\	アメリカへ行ったからと言って勝手に英語が上手になるものではない。	
\\	アメリカへ 行[い]ったからと 言[い]って 勝手[かって]に 英語[えいご]が 上手[じょうず]になるものではない。
\\	毎日授業に出ているからと言って真面目に勉強していることにはならない。	
\\	毎日[まいにち] 授業[じゅぎょう]に 出[で]ているからと 言[い]って 真面目[まじめ]に 勉強[べんきょう]していることにはならない。
\\	こんなことは言うからと言って別に批判しているわけではない。	
\\	こんなことは 言[い]うからと 言[い]って 別[べつ]に 批判[ひはん]しているわけではない。
\\	自分の問題じゃないからと言って知らん顔をしているのはよくない。	
\\	自分[じぶん]の 問題[もんだい]じゃないからと 言[い]って 知らん顔[しらんかお]をしているのはよくない。
\\	女だからと言って侮ってはいけない。	
\\	女[おんな]だからと 言[い]って 侮[あなど]ってはいけない。
\\	ちょっと出来るからって、そんなに威張らなくてもいいでしょう。	
\\	ちょっと 出来[でき]るからって、そんなに 威張[いば]らなくてもいいでしょう。
\\	子供だから許すわけにはいかない。	
\\	子供[こども]だから 許[ゆる]すわけにはいかない。
\\	試験に受からなくてもそんなに悲観することはない。	
\\	試験[しけん]に 受[う]からなくてもそんなに 悲観[ひかん]することはない。
\\	弁償したって済む問題ではない。	
\\	弁償[べんしょう]したって 済[す]む 問題[もんだい]ではない。
\\	十年後の自分を予想するのは難しかろう。	
\\	十年[じゅうねん] 後[ご]の 自分[じぶん]を 予想[よそう]するのは 難[むずか]しかろう。
\\	この研究を一年で完成するのは極めて難しかろう。	
\\	この 研究[けんきゅう]を一 年[ねん]で 完成[かんせい]するのは 極[きわ]めて 難[むずか]しかろう。
\\	日本人の中にも創造性のない教育に反対する人は多かろう。	
\\	日本人[にほんじん]の 中[なか]にも 創造[そうぞう] 性[せい]のない 教育[きょういく]に 反対[はんたい]する 人[ひと]は 多[おお]かろう。
\\	私のように文化は宗教のようなものだと考えている人は少なかろう。	
\\	私[わたし]のように 文化[ぶんか]は 宗教[しゅうきょう]のようなものだと 考[かんが]えている 人[ひと]は 少[すく]なかろう。
\\	アメリカの大学のように、日本の大学でも学生に教師の評価をさせるとよかろう。	
\\	アメリカの 大学[だいがく]のように、 日本[にほん]の 大学[だいがく]でも 学生[がくせい]に 教師[きょうし]の 評価[ひょうか]をさせるとよかろう。
\\	人種偏見ほど人間に根深いものはなかろう。	
\\	人種[じんしゅ] 偏見[へんけん]ほど 人間[にんげん]に 根深[ねぶか]いものはなかろう。
\\	近い将来に日米関係に大きな変化はなかろう。	
\\	近[ちか]い 将来[しょうらい]に 日[にち] 米[べい] 関係[かんけい]に 大[おお]きな 変化[へんか]はなかろう。
\\	あの二人の間柄は親しくはなかろう。	
\\	あの 二人[ふたり]の 間柄[あいだがら]は 親[した]しくはなかろう。
\\	残業手当てを要求することは無理ではなかろう。	
\\	残業[ざんぎょう] 手当[てあ]てを 要求[ようきゅう]することは 無理[むり]ではなかろう。
\\	彼にはアリバイがあるのだから、犯人ではなかろう。	
\\	彼[かれ]にはアリバイがあるのだから、 犯人[はんにん]ではなかろう。
\\	私達は来年もこの共同研究を続けるだろう。	
\\	私[わたし] 達[たち]は 来年[らいねん]もこの 共同[きょうどう] 研究[けんきゅう]を 続[つづ]けるだろう。
\\	ジョンには日本滞在は興味深かったろう。	
\\	ジョンには 日本[にほん] 滞在[たいざい]は 興味深[きょうみぶか]かったろう。
\\	北海道の冬は寒かっただろう。	
\\	北海道[ほっかいどう]の 冬[ふゆ]は 寒[さむ]かっただろう。
\\	林田先生は厳しい教え方をする。	
\\	林田[はやしだ] 先生[せんせい]は 厳[きび]しい 教[おし]え 方[かた]をする。
\\	昭は変わったものの見方をする。	
\\	昭[あきら]は 変[か]わったものの 見方[みかた]をする。
\\	ブラウンさんは日本人のような考え方する。	
\\	ブラウンさんは 日本人[にほんじん]のような 考え方[かんがえかた]する。
\\	誰にでも分かるような書き方をして下さい。	
\\	誰[だれ]にでも 分[わ]かるような 書き方[かきかた]をして 下[くだ]さい。
\\	あのピッチャーは面白い投げ方をする。	
\\	あのピッチャーは 面白[おもしろ]い 投[な]げ 方[かた]をする。
\\	吉田君は乱暴な運転のし方をするので乗せてもらうのが怖い。	
\\	吉田[よしだ] 君[くん]は 乱暴[らんぼう]な 運転[うんてん]のし 方[かた]をするので 乗[の]せてもらうのが 怖[こわ]い。
\\	正はほかの学生と違った勉強のし方をしているようだ。	
\\	正[ただし]はほかの 学生[がくせい]と 違[ちが]った 勉強[べんきょう]のし 方[かた]をしているようだ。
\\	私には野村先生のような考え方は出来ない。	
\\	私[わたし]には 野村[のむら] 先生[せんせい]のような 考え方[かんがえかた]は 出来[でき]ない。
\\	勉強が楽しくなるような教え方をしてほしい。	
\\	勉強[べんきょう]が 楽[たの]しくなるような 教[おし]え 方[かた]をしてほしい。
\\	この問題について私と同じような考え方をする人は多くないでしょう。	
\\	この 問題[もんだい]について 私[わたし]と 同[おな]じような 考え方[かんがえかた]をする 人[ひと]は 多[おお]くないでしょう。
\\	田村はその大臣をよく知っているような話し方をする。	
\\	田村[たむら]はその 大臣[だいじん]をよく 知[し]っているような 話し方[はなしかた]をする。
\\	米田先生は文法をうまく教える。	
\\	米田[よねだ] 先生[せんせい]は 文法[ぶんぽう]をうまく 教[おし]える。
\\	私が言うような書き方をして下さい。	
\\	私[わたし]が 言[い]うような 書き方[かきかた]をして 下[くだ]さい。
\\	グリーンさんは日本人のように日本語を話す。	
\\	グリーンさんは 日本人[にほんじん]のように 日本語[にほんご]を 話[はな]す。
\\	グリーンさんは日本人のような日本語の話し方をする。	
\\	グリーンさんは 日本人[にほんじん]のような 日本語[にほんご]の 話し方[はなしかた]をする。
\\	ブラウンさんはハリスさんのようなものの見方をする。	
\\	ブラウンさんはハリスさんのようなものの 見方[みかた]をする。
\\	日本人はみんなすしが好きかと言うと、そうではない。嫌いな人もいる。	
\\	日本人[にほんじん]はみんなすしが 好[す]きかと 言[い]うと、そうではない。 嫌[きら]いな 人[ひと]もいる。
\\	がんは治らない病気かと言うと、そうではない。早期発見をすれば治ると言われている。	
\\	がんは 治[なお]らない 病気[びょうき]かと 言[い]うと、そうではない。 早期[そうき] 発見[はっけん]をすれば 治[なお]ると 言[い]われている。
\\	日本に行って、二、三年住めば日本語が上手になるかと言うと、そうでもないようだ。かえって下手になることもある。	
\\	日本[にほん]に 行[い]って、二[に] 、三年[さんねん] 住[す]めば 日本語[にほんご]が 上手[じょうず]になるかと 言[い]うと、そうでもないようだ。かえって 下手[へた]になることもある。
\\	大学の時にいい成績の学生が社会で成功するかと言うと、必ずしもそうではないようだ。	
\\	大学[だいがく]の 時[とき]にいい 成績[せいせき]の 学生[がくせい]が 社会[しゃかい]で 成功[せいこう]するかと 言[い]うと、 必[かなら]ずしもそうではないようだ。
\\	毎日運動をすれば長生きをするかと言うと、そうでもなさそうだ。	
\\	毎日[まいにち] 運動[うんどう]をすれば 長生[ながい]きをするかと 言[い]うと、そうでもなさそうだ。
\\	日本語は難しいかと言うと、話したり聞いたりすることはそんなに難しくない。	
\\	日本語[にほんご]は 難[むずか]しいかと 言[い]うと、 話[はな]したり 聞[き]いたりすることはそんなに 難[むずか]しくない。
\\	ボストンでの車なしの生活が不便だったかと言うと、全然そうではなかったんです。	
\\	ボストンでの 車[くるま]なしの 生活[せいかつ]が 不便[ふべん]だったかと 言[い]うと、 全然[ぜんぜん]そうではなかったんです。
\\	日本人はみんなすしが好きかと言うと、全然すしを食べない日本人も時々いる。	
\\	日本人[にほんじん]はみんなすしが 好[す]きかと 言[い]うと、 全然[ぜんぜん]すしを 食[た]べない 日本人[にほんじん]も 時々[ときどき]いる。
\\	漫画がくだらないかと言うと、中にはとてもいい漫画もある。	
\\	漫画[まんが]がくだらないかと 言[い]うと、 中[なか]にはとてもいい 漫画[まんが]もある。
\\	昼間遊ぶ代りに夜勉強するつもりだ。	
\\	昼間[ひるま] 遊[あそ]ぶ 代[かわ]りに 夜[よる] 勉強[べんきょう]するつもりだ。
\\	山田さんにはちょっと余分に働いてもらった代わりに特別手当てを出した。	
\\	山田[やまだ]さんにはちょっと 余分[よぶん]に 働[はたら]いてもらった 代[か]わりに 特別[とくべつ] 手当[てあ]てを 出[だ]した。
\\	前のアパートは設備が悪かった代わりに家賃が安かった。	
\\	前[まえ]のアパートは 設備[せつび]が 悪[わる]かった 代[か]わりに 家賃[やちん]が 安[やす]かった。
\\	今朝は、コーヒーの代わりにココアを飲んだ。	
\\	今朝[けさ]は、コーヒーの 代[か]わりにココアを 飲[の]んだ。
\\	今日は図書館で勉強する代わりに寮の部屋で勉強した。	
\\	今日[きょう]は 図書館[としょかん]で 勉強[べんきょう]する 代[か]わりに 寮[りょう]の 部屋[へや]で 勉強[べんきょう]した。
\\	私はトムに日本語を教えてあげた代わりに彼に英語を教えてもらった。	
\\	私[わたし]はトムに 日本語[にほんご]を 教[おし]えてあげた 代[か]わりに 彼[かれ]に 英語[えいご]を 教[おし]えてもらった。
\\	高い長距離電話をかける代わりに、手紙をよく書いています。	
\\	高[たか]い 長距離[ちょうきょり] 電話[でんわ]をかける 代[か]わりに、 手紙[てがみ]をよく 書[か]いています。
\\	私のアパートは家賃が高い代わりに、駅に近くてとても便利です。	
\\	私[わたし]のアパートは 家賃[やちん]が 高[たか]い 代[か]わりに、 駅[えき]に 近[ちか]くてとても 便利[べんり]です。
\\	私達の日本語の先生は厳しい代わりに学生の面倒見がいい。	
\\	私[わたし] 達[たち]の 日本語[にほんご]の 先生[せんせい]は 厳[きび]しい 代[か]わりに 学生[がくせい]の 面倒[めんどう] 見[み]がいい。
\\	大学の先生は給料が低い代わりに自由がある。	
\\	大学[だいがく]の 先生[せんせい]は 給料[きゅうりょう]が 低[ひく]い 代[か]わりに 自由[じゆう]がある。
\\	父は体が弱い代わりに意志がとても強い。	
\\	父[ちち]は 体[からだ]が 弱[よわ]い 代[か]わりに 意志[いし]がとても 強[つよ]い。
\\	この辺は静かな代りに、店も遠くて不便です。	
\\	この 辺[あたり]は 静[しず]かな 代[かわ]りに、 店[みせ]も 遠[とお]くて 不便[ふべん]です。
\\	日本語の授業にいつもの山田先生の代わりに田中と言う新しい先生がいらっしゃった。	
\\	日本語[にほんご]の 授業[じゅぎょう]にいつもの 山田[やまだ] 先生[せんせい]の 代[か]わりに 田中[たなか]と 言[い]う 新[あたら]しい 先生[せんせい]がいらっしゃった。
\\	昼間遊ぶ代わりに晩遊ぶつもりだ。	
\\	昼間[ひるま] 遊[あそ]ぶ 代[か]わりに 晩[ばん] 遊[あそ]ぶつもりだ。
\\	昼間遊んだ変わりに夜勉強した。	
\\	昼間[ひるま] 遊[あそ]んだ 変[か]わりに 夜[よる] 勉強[べんきょう]した。
\\	昼間遊んだ代わりに夜勉強するつもりだ。	
\\	昼間[ひるま] 遊[あそ]んだ 代[か]わりに 夜[よる] 勉強[べんきょう]するつもりだ。
\\	昼間遊ぶかわりによる遊んだ。	
\\	昼間[ひるま] 遊[あそ]ぶかわりによる 遊[あそ]んだ。
\\	アパートは高い代わりに便利だ。	
\\	アパートは 高[たか]い 代[か]わりに 便利[べんり]だ。
\\	アパートは高い代わりに便利だった。	
\\	アパートは 高[たか]い 代[か]わりに 便利[べんり]だった。
\\	アパートは高かった代わりに便利だった。	
\\	アパートは 高[たか]かった 代[か]わりに 便利[べんり]だった。
\\	今朝は朝ご飯を食べないで会社に行った。	
\\	今朝[けさ]は 朝[あさ]ご 飯[はん]を 食[た]べないで 会社[かいしゃ]に 行[い]った。
\\	友達に電話したけれど、あいにく、家にいなかった。	
\\	友達[ともだち]に 電話[でんわ]したけれど、あいにく、 家[いえ]にいなかった。
\\	スミスさんは日本に行ってことがない。しかし、日本語が上手だ。	
\\	スミスさんは 日本[にほん]に 行[い]ってことがない。しかし、 日本語[にほんご]が 上手[じょうず]だ。
\\	投票の結果、その提案は反対多数で否決された。	
\\	投票[とうひょう]の 結果[けっか]、その 提案[ていあん]は 反対[はんたい] 多数[たすう]で 否決[ひけつ]された。
\\	妻と相談した結果、家を買うことにした。	
\\	妻[つま]と 相談[そうだん]した 結果[けっか]、 家[いえ]を 買[か]うことにした。
\\	相談の結果、今回の旅行は延期することになった。	
\\	相談[そうだん]の 結果[けっか]、 今回[こんかい]の 旅行[りょこう]は 延期[えんき]することになった。
\\	検査の結果、妻の体はどこにも異状がないことが分かった。	
\\	検査[けんさ]の 結果[けっか]、 妻[つま]の 体[からだ]はどこにも 異状[いじょう]がないことが 分[わ]かった。
\\	調査の結果、新しい事実が発見された。	
\\	調査[ちょうさ]の 結果[けっか]、 新[あたら]しい 事実[じじつ]が 発見[はっけん]された。
\\	警察で調べた結果、原因はたばこの火の不始末と分かった。	
\\	警察[けいさつ]で 調[しら]べた 結果[けっか]、 原因[げんいん]はたばこの 火[ひ]の 不[ふ] 始末[しまつ]と 分[わ]かった。
\\	特別のダイエットをした結果、十キロの減量に成功した。	
\\	特別[とくべつ]のダイエットをした 結果[けっか]、十キロ[じゅっきろ]の 減量[げんりょう]に 成功[せいこう]した。
\\	新しい教科書を使った結果、学生の成績が著しく伸びた。	
\\	新[あたら]しい 教科書[きょうかしょ]を 使[つか]った 結果[けっか]、 学生[がくせい]の 成績[せいせき]が 著[いちじる]しく 伸[の]びた。
\\	ゴルフの個人指導を受けた結果、自分の問題点が明らかになった。	
\\	ゴルフの 個人[こじん] 指導[しどう]を 受[う]けた 結果[けっか]、 自分[じぶん]の 問題[もんだい] 点[てん]が 明[あき]らかになった。
\\	試験の結果は来週知らせます。	
\\	試験[しけん]の 結果[けっか]は 来週[らいしゅう] 知[し]らせます。
\\	私が特別のダイエットをした結果をまだ誰も知らない。	
\\	私[わたし]が 特別[とくべつ]のダイエットをした 結果[けっか]をまだ 誰[だれ]も 知[し]らない。
\\	ここのすしは、結構おいしいね。	
\\	ここのすしは、 結構[けっこう]おいしいね。
\\	この車は古いんだけど、結構よく走りますよ。	
\\	この 車[くるま]は 古[ふる]いんだけど、 結構[けっこう]よく 走[はし]りますよ。
\\	あの人は間抜けに見えるでしょう。でも、結構頭がいいんです。	
\\	あの 人[ひと]は 間抜[まぬ]けに 見[み]えるでしょう。でも、 結構[けっこう] 頭[あたま]がいいんです。
\\	四月だと言うのに、結構寒いね。	
\\	四月[しがつ]だと 言[い]うのに、 結構[けっこう] 寒[さむ]いね。
\\	小さい町なのに、結構いいレストランがあるね。	
\\	小[ちい]さい 町[まち]なのに、 結構[けっこう]いいレストランがあるね。
\\	この料理は量は少ないけど、結構胃にもたれるね。	
\\	この 料理[りょうり]は 量[りょう]は 少[すく]ないけど、 結構[けっこう] 胃[い]にもたれるね。
\\	今日は日曜日なのに高速が結構混んでいるね。	
\\	今日[きょう]は 日曜日[にちようび]なのに 高速[こうそく]が 結構[けっこう] 混[こ]んでいるね。
\\	父は楽天的な人でしたが、失職した時には結構悩んだようです。	
\\	父[ちち]は 楽天的[らくてんてき]な 人[ひと]でしたが、 失職[しっしょく]した 時[とき]には 結構[けっこう] 悩[なや]んだようです。
\\	カラオケは初めてだったが、結構楽しかった。	
\\	カラオケは 初[はじ]めてだったが、 結構[けっこう] 楽[たの]しかった。
\\	この雑誌は思ったよりつまらないです。	
\\	この 雑誌[ざっし]は 思[おも]ったよりつまらないです。
\\	このケーキは思ったよりまずいです。	
\\	このケーキは 思[おも]ったよりまずいです。
\\	この雑誌は思ったより面白くないです。	
\\	この 雑誌[ざっし]は 思[おも]ったより 面白[おもしろ]くないです。
\\	このケーキは思ったよりおいしくないです。	
\\	このケーキは 思[おも]ったよりおいしくないです。
\\	案外あの二人は結婚するかもしれないよ。	
\\	案外[あんがい]あの 二人[ふたり]は 結婚[けっこん]するかもしれないよ。
\\	あいつはパーティーにはたいてい来ないけど、今日は夏休みの前だし、案外来るかもしれないぞ。	
\\	あいつはパーティーにはたいてい 来[こ]ないけど、 今日[きょう]は 夏休[なつやす]みの 前[まえ]だし、 案外[あんがい] 来[く]るかもしれないぞ。
\\	案外、彼の方が先に着いているかもしれませんよ。	
\\	案外[あんがい]、 彼[かれ]の 方[ほう]が 先[さき]に 着[つ]いているかもしれませんよ。
\\	意外とあの二人は結婚するかもしれないよ。	
\\	意外[いがい]とあの 二人[ふたり]は 結婚[けっこん]するかもしれないよ。
\\	あいつはパーティーにはたいてい来ないけど、今日は夏休みの前だし、意外と来るかもしれないぞ。	
\\	あいつはパーティーにはたいてい 来[こ]ないけど、 今日[きょう]は 夏休[なつやす]みの 前[まえ]だし、 意外[いがい]と 来[く]るかもしれないぞ。
\\	意外と、彼の方が先に着いているかもしれませんよ。	
\\	意外[いがい]と、 彼[かれ]の 方[ほう]が 先[さき]に 着[つ]いているかもしれませんよ。
\\	ここのすしは、割合とおいしいね。	
\\	ここのすしは、 割合[わりあい]とおいしいね。
\\	あの人は間抜けに見えるでしょう。でも、割と頭がいいんです。	
\\	あの 人[ひと]は 間抜[まぬ]けに 見[み]えるでしょう。でも、 割[わり]と 頭[あたま]がいいんです。
\\	今日は日曜日なのに高速が割合混んでいるね。	
\\	今日[きょう]は 日曜日[にちようび]なのに 高速[こうそく]が 割合[わりあい] 混[こ]んでいるね。
\\	秋山さんはこの春結婚します。	
\\	秋山[あきやま]さんはこの 春[はる] 結婚[けっこん]します。
\\	山下君はこの一週間授業を休んでいます。	
\\	山下[やました] 君[くん]はこの 一週間[いっしゅうかん] 授業[じゅぎょう]を 休[やす]んでいます。
\\	この間吉岡さんに会いました。	
\\	この 間[あいだ] 吉岡[よしおか]さんに 会[あ]いました。
\\	このたびこの会の会員に加えていただきました。	
\\	このたびこの 会[かい]の 会員[かいいん]に 加[くわ]えていただきました。
\\	この際車を買おうか。	
\\	この 際[さい] 車[くるま]を 買[か]おうか。
\\	この辺で妥協したらどうですか。	
\\	この 辺[あたり]で 妥協[だきょう]したらどうですか。
\\	この夏は日本の女流作家の研究をしています。	
\\	この 夏[なつ]は 日本[にほん]の 女流[じょりゅう] 作家[さっか]の 研究[けんきゅう]をしています。
\\	この一週間は忙しくて何も出来ないだろう。	
\\	この 一週間[いっしゅうかん]は 忙[いそが]しくて 何[なに]も 出来[でき]ないだろう。
\\	この次はいつお目にかかれますか。	
\\	この 次[つぎ]はいつお 目[め]にかかれますか。
\\	こうした問題はこの国では聞かれないようである。	
\\	こうした 問題[もんだい]はこの 国[くに]では 聞[き]かれないようである。
\\	こうした行為がどのような結果を招くかは誰の目にも明らかだ。	
\\	こうした 行為[こうい]がどのような 結果[けっか]を 招[まね]くかは 誰[だれ]の 目[め]にも 明[あき]らかだ。
\\	こうした経験は日本へ行ったことのあるものなら誰にでもあるはずだ。	
\\	こうした 経験[けいけん]は 日本[にほん]へ 行[い]ったことのあるものなら 誰[だれ]にでもあるはずだ。
\\	私はこうした話には耳を貸さないことにしている。	
\\	私[わたし]はこうした 話[はなし]には 耳[みみ]を 貸[か]さないことにしている。
\\	私達の活動はこうして始まった。	
\\	私[わたし] 達[たち]の 活動[かつどう]はこうして 始[はじ]まった。
\\	アメリカでは景気が悪くなると、大企業が大量の労働者を解雇することがよく行われる。こうした問題はこの国ではあまり聞かれないようである。	
\\	アメリカでは 景気[けいき]が 悪[わる]くなると、 大[だい] 企業[きぎょう]が 大量[たいりょう]の 労働[ろうどう] 者[しゃ]を 解雇[かいこ]することがよく 行[おこな]われる。こうした 問題[もんだい]はこの 国[くに]ではあまり 聞[き]かれないようである。
\\	ーアメリカでは景気が悪くなると、大企業が大量の労働者を解雇することがよく行われます。 ーそうですか。そうした問題はこの国ではあまり聞かれないようです。	ー
\\	ーアメリカでは 景気[けいき]が 悪[わる]くなると、 大[だい] 企業[きぎょう]が 大量[たいりょう]の 労働[ろうどう] 者[しゃ]を 解雇[かいこ]することがよく 行[おこな]われます。 ーそうですか。そうした 問題[もんだい]はこの 国[くに]ではあまり 聞[き]かれないようです。
\\	ああした問題はこの国ではあまり聞かれないようです。	
\\	ああした 問題[もんだい]はこの 国[くに]ではあまり 聞[き]かれないようです。
\\	これこそ我々が探し求めていたものだ。	
\\	これこそ 我々[われわれ]が 探し求[さがしもと]めていたものだ。
\\	ーどうもすみませんでした。 ーいいえ、こちらこそすみませんでした。	
\\	ーどうもすみませんでした。 ーいいえ、こちらこそすみませんでした。
\\	一人でやってこそ勉強になるのだ。	
\\	一人[ひとり]でやってこそ 勉強[べんきょう]になるのだ。
\\	君が正直に話してくれたからこそ問題は最小で済んだんだ。	
\\	君[きみ]が 正直[しょうじき]に 話[はな]してくれたからこそ 問題[もんだい]は 最小[さいしょう]で 済[す]んだんだ。
\\	ー一体一週間も何をしていたんだ。 ーそう言う君こそ何をしていたんだ。	
\\	ー 一体[いったい] 一週間[いっしゅうかん]も 何[なに]をしていたんだ。 ーそう 言[い]う 君[きみ]こそ 何[なに]をしていたんだ。
\\	こんな時にこそ全員で力を合わせて問題を解決しなければならない。	
\\	こんな 時[とき]にこそ 全員[ぜんいん]で 力[ちから]を 合[あ]わせて 問題[もんだい]を 解決[かいけつ]しなければならない。
\\	今年こそこの試験に通ってみせる。	
\\	今年[ことし]こそこの 試験[しけん]に 通[かよ]ってみせる。
\\	それでこそ我々のリーダーだ。	
\\	それでこそ 我々[われわれ]のリーダーだ。
\\	ここにいてこそいい仕事も見つかるのだ。	
\\	ここにいてこそいい 仕事[しごと]も 見[み]つかるのだ。
\\	親友だからこそこんなことまで君に言うんだよ。	
\\	親友[しんゆう]だからこそこんなことまで 君[きみ]に 言[い]うんだよ。
\\	出来ないからこそ人より余計に練習しなければならないのだ。	
\\	出来[でき]ないからこそ 人[ひと]より 余計[よけい]に 練習[れんしゅう]しなければならないのだ。
\\	ロンドンへこそ行くべきだ。	
\\	ロンドンへこそ 行[い]くべきだ。
\\	この大学にこそ求めているような学者がいるはずだ。	
\\	この 大学[だいがく]にこそ 求[もと]めているような 学者[がくしゃ]がいるはずだ。
\\	発表は十五分以内で行うこと。	
\\	発表[はっぴょう]は 十五分[じゅうごふん] 以内[いない]で 行[おこな]うこと。
\\	プールサイドを走らないこと。	
\\	プールサイドを 走[はし]らないこと。
\\	詳細を二十三ページ参照のこと。	
\\	詳細[しょうさい]を 二十三[にじゅうさん]ページ 参照[さんしょう]のこと。
\\	私語は慎むこと。	
\\	私語[しご]は 慎[つつし]むこと。
\\	先に必ずテープを聞くこと。	
\\	先[さき]に 必[かなら]ずテープを 聞[き]くこと。
\\	辞書は見ないこと。	
\\	辞書[じしょ]は 見[み]ないこと。
\\	裸足で歩き回らないこと。	
\\	裸足[はだし]で 歩き回[あるきまわ]らないこと。
\\	弁当を持参のこと。	
\\	弁当[べんとう]を 持参[じさん]のこと。
\\	制服着用のこと。	
\\	制服[せいふく] 着用[ちゃくよう]のこと。
\\	願書は本人が提出のこと。	
\\	願書[がんしょ]は 本人[ほんにん]が 提出[ていしゅつ]のこと。
\\	明朝八時に東京駅に集合のこと。	
\\	明朝[みょうちょう] 八時[はちじ]に 東京[とうきょう] 駅[えき]に 集合[しゅうごう]のこと。
\\	私は週末にゴルフをすることで気分転換を図っています。	
\\	私[わたし]は 週末[しゅうまつ]にゴルフをすることで 気分[きぶん] 転換[てんかん]を 図[はか]っています。
\\	彼が仲裁の場に出ていったことで事態は余計にこじれた。	
\\	彼[かれ]が 仲裁[ちゅうさい]の 場[ば]に 出[で]ていったことで 事態[じたい]は 余計[よけい]にこじれた。
\\	私が十万円出すことで問題は解決した。	
\\	私[わたし]が 十万[じゅうまん] 円[えん] 出[だ]すことで 問題[もんだい]は 解決[かいけつ]した。
\\	彼女はものを言わないことで私へ精一杯の抵抗を示しているのです。	
\\	彼女[かのじょ]はものを 言[い]わないことで 私[わたし]へ 精一杯[せいいっぱい]の 抵抗[ていこう]を 示[しめ]しているのです。
\\	朝晩簡単な体操をするだけのことですばらしい健康が保てます。	
\\	朝晩[あさばん] 簡単[かんたん]な 体操[たいそう]をするだけのことですばらしい 健康[けんこう]が 保[たも]てます。
\\	私が参加することであなたに迷惑はかかりませんか。	
\\	私[わたし]が 参加[さんか]することであなたに 迷惑[めいわく]はかかりませんか。
\\	彼のパーティーに行かなかったことで彼の気持ちを害したのでなければよいが。	
\\	彼[かれ]のパーティーに 行[い]かなかったことで 彼[かれ]の 気持[きも]ちを 害[がい]したのでなければよいが。
\\	私は英語が下手なことで時々損をしている。	
\\	私[わたし]は 英語[えいご]が 下手[へた]なことで 時々[ときどき] 損[そん]をしている。
\\	この国では外国人であることで得をすることがある。	
\\	この 国[くに]では 外国[がいこく] 人[じん]であることで 得[とく]をすることがある。
\\	私は学生の時あまり勉強しなかったことでよく親に叱られた。	
\\	私[わたし]は 学生[がくせい]の 時[とき]あまり 勉強[べんきょう]しなかったことでよく 親[おや]に 叱[しか]られた。
\\	私は学生の時あまり勉強しなかったために、今とても苦労している。	
\\	私[わたし]は 学生[がくせい]の 時[とき]あまり 勉強[べんきょう]しなかったために、 今[いま]とても 苦労[くろう]している。
\\	これ以上聞くのは彼を余計に苦しめることになる。	
\\	これ 以上[いじょう] 聞[き]くのは 彼[かれ]を 余計[よけい]に 苦[くる]しめることになる。
\\	今これをしておかなかったら後で大変なことになる。	
\\	今[いま]これをしておかなかったら 後[あと]で 大変[たいへん]なことになる。
\\	この手形が落ちなければ会社は倒産と言うことになる。	
\\	この 手形[てがた]が 落[お]ちなければ 会社[かいしゃ]は 倒産[とうさん]と 言[い]うことになる。
\\	こんな結果になったということは我々の準備に手落ちがあったことになる。	
\\	こんな 結果[けっか]になったということは 我々[われわれ]の 準備[じゅんび]に 手落[てお]ちがあったことになる。
\\	今回何も起こらなかったということはこのシステムで大丈夫だということになる。	
\\	今回[こんかい] 何[なに]も 起[お]こらなかったということはこのシステムで 大丈夫[だいじょうぶ]だということになる。
\\	今怠けていると試験の時ひどいことになるよ。	
\\	今[いま] 怠[なま]けていると 試験[しけん]の 時[とき]ひどいことになるよ。
\\	そんなことをしたら二度と人前に出られないことになる。	
\\	そんなことをしたら 二度[にど]と 人前[ひとまえ]に 出[で]られないことになる。
\\	社が百万円寄付してくれれば合計一千万円に達することになる。	
\\	社[しゃ]が 百[ひゃく] 万[まん] 円[えん] 寄付[きふ]してくれれば 合計[ごうけい] 一千[いっせん] 万[まん] 円[えん]に 達[たっ]することになる。
\\	すべてがうまくいけば来年卒業ということになります。	
\\	すべてがうまくいけば 来年[らいねん] 卒業[そつぎょう]ということになります。
\\	葉書が戻って来たということは彼はもうこの住所には住んでいないことになる。	
\\	葉書[はがき]が 戻[もど]って 来[き]たということは 彼[かれ]はもうこの 住所[じゅうしょ]には 住[す]んでいないことになる。
\\	これが一キロ二千円と言うことはこちらの方が高いということになる。	
\\	これが一キロ 二千[にせん] 円[えん]と 言[い]うことはこちらの 方[ほう]が 高[たか]いということになる。
\\	今回の失敗は事前の話し合いが不十分だったことによる。	
\\	今回[こんかい]の 失敗[しっぱい]は 事前[じぜん]の 話し合[はなしあ]いが 不十分[ふじゅうぶん]だったことによる。
\\	進が非行に走ったのは両親の愛情が不足していたことによる。	
\\	進[すすむ]が 非行[ひこう]に 走[はし]ったのは 両親[りょうしん]の 愛情[あいじょう]が 不足[ふそく]していたことによる。
\\	彼の現在の成功は平生の努力を怠らなかったことによる。	
\\	彼[かれ]の 現在[げんざい]の 成功[せいこう]は 平生[へいぜい]の 努力[どりょく]を 怠[おこた]らなかったことによる。
\\	彼の離婚を二人の価値観があまりに違っていたことによる。	
\\	彼[かれ]の 離婚[りこん]を 二人[ふたり]の 価値[かち] 観[かん]があまりに 違[ちが]っていたことによる。
\\	彼が負けたのは勝ちを焦りすぎたことによる。	
\\	彼[かれ]が 負[ま]けたのは 勝[か]ちを 焦[あせ]りすぎたことによる。
\\	彼の才能がこれ程までに開発されたのは鈴木氏に師事したことによる。	
\\	彼[かれ]の 才能[さいのう]がこれ 程[ほど]までに 開発[かいはつ]されたのは 鈴木[すずき] 氏[し]に 師事[しじ]したことによる。
\\	我々が今回失敗したのは事前お話し合いが不十分だったことによる。	
\\	我々[われわれ]が 今回[こんかい] 失敗[しっぱい]したのは 事前[じぜん]お 話し合[はなしあ]いが 不十分[ふじゅうぶん]だったことによる。
\\	彼がパーティーに来なかったのは忙しかったためだ。	
\\	彼[かれ]がパーティーに 来[こ]なかったのは 忙[いそが]しかったためだ。
\\	彼がパーティーへ来るのは春子に会うためだ。	
\\	彼[かれ]がパーティーへ 来[く]るのは 春子[はるこ]に 会[あ]うためだ。
\\	あんな男の言うことを聞くことはない。	
\\	あんな 男[おとこ]の 言[い]うことを 聞[き]くことはない。
\\	彼女が今日の会議を忘れることはないと思います。	
\\	彼女[かのじょ]が 今日[きょう]の 会議[かいぎ]を 忘[わす]れることはないと 思[おも]います。
\\	あんな奴に親切にしてやることはない。	
\\	あんな 奴[やつ]に 親切[しんせつ]にしてやることはない。
\\	何もそんなに慌てることはない。	
\\	何[なに]もそんなに 慌[あわ]てることはない。
\\	君が来ることはないと思います。	
\\	君[きみ]が 来[く]ることはないと 思[おも]います。
\\	彼がわざわざ出て行くことはないでしょう。	
\\	彼[かれ]がわざわざ 出[で]て 行[い]くことはないでしょう。
\\	もうお目にかかることはないかもしれませんね。	
\\	もうお 目[め]にかかることはないかもしれませんね。
\\	多分私が教えることはないと思います。	
\\	多分[たぶん] 私[わたし]が 教[おし]えることはないと 思[おも]います。
\\	彼女が今日の会議を忘れはずはない。	
\\	彼女[かのじょ]が 今日[きょう]の 会議[かいぎ]を 忘[わす]れはずはない。
\\	加奈子の性格は明るく、無邪気だった。	
\\	加奈子[かなこ]の 性格[せいかく]は 明[あか]るく、 無邪気[むじゃき]だった。
\\	せっかく講演をしに行ったのに、聴衆が少なく、がっかりした。	
\\	せっかく 講演[こうえん]をしに 行[い]ったのに、 聴衆[ちょうしゅう]が 少[すく]なく、がっかりした。
\\	カリフォルニアの空は青く、美しかった。	
\\	カリフォルニアの 空[そら]は 青[あお]く、 美[うつく]しかった。
\\	トムの東京のアパートの部屋は暗く、狭い。	
\\	トムの 東京[とうきょう]のアパートの 部屋[へや]は 暗[くら]く、 狭[せま]い。
\\	バッハの音楽は歯切れがよく、幾何学的だ。	
\\	バッハの 音楽[おんがく]は 歯切[はぎ]れがよく、 幾何[きか] 学[がく] 的[てき]だ。
\\	ニューヨークタイムズは質が高く、購読者の数も多い。	
\\	ニューヨークタイムズは 質[しつ]が 高[たか]く、 購読[こうどく] 者[しゃ]の 数[かず]も 多[おお]い。
\\	世界の平和は得られにくく、維持しにくい。	
\\	世界[せかい]の 平和[へいわ]は 得[え]られにくく、 維持[いじ]しにくい。
\\	あの哲学者の思想は分かりやすく、文章も簡潔だ。	
\\	あの 哲学[てつがく] 者[しゃ]の 思想[しそう]は 分[わ]かりやすく、 文章[ぶんしょう]も 簡潔[かんけつ]だ。
\\	去年の冬は雪が多く、車の運転が大変だった。	
\\	去年[きょねん]の 冬[ふゆ]は 雪[ゆき]が 多[おお]く、 車[くるま]の 運転[うんてん]が 大変[たいへん]だった。
\\	私は失敗が恐ろしく、新しいことが何も出来ない。	
\\	私[わたし]は 失敗[しっぱい]が 恐[おそ]ろしく、 新[あたら]しいことが 何[なに]も 出来[でき]ない。
\\	陽子は一人でいるのが寂しく、最近猫を飼い始めた。	
\\	陽子[ようし]は 一人[ひとり]でいるのが 寂[さび]しく、 最近[さいきん] 猫[ねこ]を 飼[か]い 始[はじ]めた。
\\	私は自分の研究を出版したくて、いくつかの出版社に連絡してみた。	
\\	私[わたし]は 自分[じぶん]の 研究[けんきゅう]を 出版[しゅっぱん]したくて、いくつかの 出版[しゅっぱん] 社[しゃ]に 連絡[れんらく]してみた。
\\	僕は若い頃から、小説が書きたくて、うずうずしていた。	
\\	僕[ぼく]は 若[わか]い 頃[ころ]から、 小説[しょうせつ]が 書[か]きたくて、うずうずしていた。
\\	その作家の文章はとても読みにくく、途中で読むのをあきらめてしまった。	
\\	その 作家[さっか]の 文章[ぶんしょう]はとても 読[よ]みにくく、 途中[とちゅう]で 読[よ]むのをあきらめてしまった。
\\	小林先生の文法の説明は分かりやすく、とても助かる。	
\\	小林[こばやし] 先生[せんせい]の 文法[ぶんぽう]の 説明[せつめい]は 分[わ]かりやすく、とても 助[たす]かる。
\\	少女は恋をしているらしく、はつらつとしていた。	
\\	少女[しょうじょ]は 恋[こい]をしているらしく、はつらつとしていた。
\\	字をもっと大きく書いてください。	
\\	字[じ]をもっと 大[おお]きく 書[か]いてください。
\\	日本語が面白くなってきた。	
\\	日本語[にほんご]が 面白[おもしろ]くなってきた。
\\	私達は一歩も歩けないくらい疲れていた。	
\\	私[わたし] 達[たち]は 一歩[いっぽ]も 歩[ある]けないくらい 疲[つか]れていた。
\\	今晩は暖かいのでストーブが要らないくらいだ。	
\\	今晩[こんばん]は 暖[あたた]かいのでストーブが 要[い]らないくらいだ。
\\	山田さんくらいよく物を忘れる人はいない。	
\\	山田[やまだ]さんくらいよく 物[もの]を 忘[わす]れる 人[ひと]はいない。
\\	私は料理は下手ですが、ご飯くらい炊けます。	
\\	私[わたし]は 料理[りょうり]は 下手[へた]ですが、ご 飯[はん]くらい 炊[た]けます。
\\	今この仕事が出来るのは彼くらいのものだ。	
\\	今[いま]この 仕事[しごと]が 出来[でき]るのは 彼[かれ]くらいのものだ。
\\	そんなことをするくらいなら死んだ方がましだ。	
\\	そんなことをするくらいなら 死[し]んだ 方[ほう]がましだ。
\\	その家は直しようがないくらい痛んでいた。	
\\	その 家[いえ]は 直[なお]しようがないくらい 痛[いた]んでいた。
\\	次郎はひどく酔っていて立っていられないくらいだった。	
\\	次郎[じろう]はひどく 酔[よ]っていて 立[た]っていられないくらいだった。
\\	こんな本、十ドルでもまだ高いくらいだ。	
\\	こんな 本[ほん]、十[じゅう]ドルでもまだ 高[たか]いくらいだ。
\\	内田さんくらいかわいそうな人はいない。	
\\	内田[うちだ]さんくらいかわいそうな 人[ひと]はいない。
\\	あの時くらい苦しかった時はない。	
\\	あの 時[とき]くらい 苦[くる]しかった 時[とき]はない。
\\	信頼していた人に裏切られることくらい辛いことはない。	
\\	信頼[しんらい]していた 人[ひと]に 裏切[うらぎ]られることくらい 辛[つら]いことはない。
\\	いくら安い所でもシャワーくらい付いているでしょう。	
\\	いくら 安[やす]い 所[ところ]でもシャワーくらい 付[つ]いているでしょう。
\\	そんな物に二万円も出すのはあなたくらいのものですよ。	
\\	そんな 物[もの]に 二万[にまん] 円[えん]も 出[だ]すのはあなたくらいのものですよ。
\\	あんな男の下で働くくらいなら乞食になった方がましだ。	
\\	あんな 男[おとこ]の 下[した]で 働[はたら]くくらいなら 乞食[こじき]になった 方[ほう]がましだ。
\\	山田さんは一番よく物を忘れる。	
\\	山田[やまだ]さんは 一番[いちばん]よく 物[もの]を 忘[わす]れる。
\\	山田さんは日本人のくせにあまり漢字を知らない。	
\\	山田[やまだ]さんは 日本人[にほんじん]のくせにあまり 漢字[かんじ]を 知[し]らない。
\\	山内はよく知っているくせに何も教えてくれない。	
\\	山内[やまうち]はよく 知[し]っているくせに 何[なに]も 教[おし]えてくれない。
\\	彼は大学生のくせに漫画ばかり読んでいる。	
\\	彼[かれ]は 大学生[だいがくせい]のくせに 漫画[まんが]ばかり 読[よ]んでいる。
\\	良子はまだ学生のくせに高いマンションに住んでいる。	
\\	良子[よしこ]はまだ 学生[がくせい]のくせに 高[たか]いマンションに 住[す]んでいる。
\\	隆司は下手なくせに私とテニスはしたがる。	
\\	隆司[たかし]は 下手[へた]なくせに 私[わたし]とテニスはしたがる。
\\	怖いくせに無理するなよ。	
\\	怖[こわ]いくせに 無理[むり]するなよ。
\\	孝男はお金もないくせに外車を欲しがっている。	
\\	孝男[たかお]はお 金[かね]もないくせに 外車[がいしゃ]を 欲[ほ]しがっている。
\\	吉田は前は私を見ても挨拶もしなかったくせに、私が部長になった途端に急に愛想がよくなった。	
\\	吉田[よしだ]は 前[まえ]は 私[わたし]を 見[み]ても 挨拶[あいさつ]もしなかったくせに、 私[わたし]が 部長[ぶちょう]になった 途端[とたん]に 急[きゅう]に 愛想[あいそ]がよくなった。
\\	昨日まで見習いだったくせに大きな口をきくな。	
\\	昨日[きのう]まで 見習[みなら]いだったくせに 大[おお]きな 口[くち]をきくな。
\\	良子はまだ学生なのに両親は彼女に高いマンションを買ってやった。	
\\	良子[りょうこ]はまだ 学生[がくせい]なのに 両親[りょうしん]は 彼女[かのじょ]に 高[たか]いマンションを 買[か]ってやった。
\\	私は日本人にもかかわらず日本の国家が歌えない。	
\\	私[わたし]は 日本人[にほんじん]にもかかわらず 日本[にほん]の 国家[こっか]が 歌[うた]えない。
\\	何だ、何も知らないくせに。	
\\	何[なん]だ、 何[なに]も 知[し]らないくせに。
\\	まあ、男のくせに。	
\\	まあ、 男[おとこ]のくせに。
\\	これは簡単な計算だから計算器を使うまでもない。	
\\	これは 簡単[かんたん]な 計算[けいさん]だから 計算[けいさん] 器[き]を 使[つか]うまでもない。
\\	言うまでもなくジョージ・ワシントンはアメリカの初代大統領だ。	
\\	言[い]うまでもなくジョージ・ワシントンはアメリカの 初代[しょだい] 大統領[だいとうりょう]だ。
\\	彼がみんなに尊敬されたのは言うまでもない。	
\\	彼[かれ]がみんなに 尊敬[そんけい]されたのは 言[い]うまでもない。
\\	大した用事じゃないからあなたがわざわざ行くまでもないでしょう。	
\\	大[たい]した 用事[ようじ]じゃないからあなたがわざわざ 行[い]くまでもないでしょう。
\\	行けば分かることですから電話して聞くまでもありません。	
\\	行[い]けば 分[わ]かることですから 電話[でんわ]して 聞[き]くまでもありません。
\\	今更申すまでもありませんが、山中先生はこの分野では指導的な立場にあるお方です。	
\\	今更[いまさら] 申[もう]すまでもありませんが、 山中[やまなか] 先生[せんせい]はこの 分野[ぶんや]では 指導[しどう] 的[てき]な 立場[たちば]にあるお 方[かた]です。
\\	それは説明するまでもなく明らかなことだ。	
\\	それは 説明[せつめい]するまでもなく 明[あき]らかなことだ。
\\	待つまでもなく妻は買い物から帰ってきた。	
\\	待[ま]つまでもなく 妻[つま]は 買い物[かいもの]から 帰[かえ]ってきた。
\\	これは簡単な計算だから計算器を使う必要はない。	
\\	これは 簡単[かんたん]な 計算[けいさん]だから 計算[けいさん] 器[き]を 使[つか]う 必要[ひつよう]はない。
\\	このレストランは予約の必要はない。	
\\	このレストランは 予約[よやく]の 必要[ひつよう]はない。
\\	お礼には及びません。	
\\	お 礼[れい]には 及[およ]びません。
\\	ご心配には及びません。	
\\	ご 心配[しんぱい]には 及[およ]びません。
\\	もう橋本には何も頼むまい。	
\\	もう 橋本[はしもと]には 何[なに]も 頼[たの]むまい。
\\	これは恐らく誰も気が付くまい。	
\\	これは 恐[おそ]らく 誰[だれ]も 気[き]が 付[つ]くまい。
\\	参加者はそれほど多くはあるまい。	
\\	参加[さんか] 者[しゃ]はそれほど 多[おお]くはあるまい。
\\	この教え方はあまり効果的ではあるまい。	
\\	この 教[おし]え 方[かた]はあまり 効果[こうか] 的[てき]ではあるまい。
\\	これは何かの間違いではあるまいか。	
\\	これは 何[なに]かの 間違[まちが]いではあるまいか。
\\	清のパーティーにはもう行くまい。	
\\	清[きよし]のパーティーにはもう 行[い]くまい。
\\	日本へ行こうか行くまいか迷った。	
\\	日本[にほん]へ 行[い]こうか 行[い]くまいか 迷[まよ]った。
\\	彼は多分誰の言うことも聞くまい。	
\\	彼[かれ]は 多分[たぶん] 誰[だれ]の 言[い]うことも 聞[き]くまい。
\\	それは今我々には必要ではあるまい。	
\\	それは 今[いま] 我々[われわれ]には 必要[ひつよう]ではあるまい。
\\	この問題はこの学校の学生にはそれほど難しくはあるまい。	
\\	この 問題[もんだい]はこの 学校[がっこう]の 学生[がくせい]にはそれほど 難[むずか]しくはあるまい。
\\	ジョージが真相を知っているのではあるまいか。	
\\	ジョージが 真相[しんそう]を 知[し]っているのではあるまいか。
\\	彼はもう英語は教えまい。	
\\	彼[かれ]はもう 英語[えいご]は 教[おし]えまい。
\\	彼女はそんなことはすまい。	
\\	彼女[かのじょ]はそんなことはすまい。
\\	こんな機会は二度と来まい。	
\\	こんな 機会[きかい]は 二度[にど]と 来[こ]まい。
\\	清のパーティーにはもう行くまいと思って。	
\\	清[きよし]のパーティーにはもう 行[い]くまいと 思[おも]って。
\\	もう橋本には何も頼むまいと心に決めた。	
\\	もう 橋本[はしもと]には 何[なに]も 頼[たの]むまいと 心[こころ]に 決[き]めた。
\\	恵子は清のパーティーにはもう行くまいと思った。	
\\	恵子[けいこ]は 清[きよし]のパーティーにはもう 行[い]くまいと 思[おも]った。
\\	恵子は清のパーティーにはもう行くまい。	
\\	恵子[けいこ]は 清[きよし]のパーティーにはもう 行[い]くまい。
\\	彼は私のハンドバッグを取ろうとしたが、私は取られまいとして脇の下に強くはさんだ。	
\\	彼[かれ]は 私[わたし]のハンドバッグを 取[と]ろうとしたが、 私[わたし]は 取[と]られまいとして 脇の下[わきのした]に 強[つよ]くはさんだ。
\\	私は負けるまいと頑張った。	
\\	私[わたし]は 負[ま]けるまいと 頑張[がんば]った。
\\	まさか美智子があんな男と結婚するとは思わなかった。	
\\	まさか 美智子[みちこ]があんな 男[おとこ]と 結婚[けっこん]するとは 思[おも]わなかった。
\\	まさかあたしの誕生日を忘れたんじゃないでしょうね。	
\\	まさかあたしの 誕生[たんじょう] 日[び]を 忘[わす]れたんじゃないでしょうね。
\\	まさか彼がこんな寒い日に来るはずはないですよ。	
\\	まさか 彼[かれ]がこんな 寒[さむ]い 日[ひ]に 来[く]るはずはないですよ。
\\	最近父から手紙が来ないけれど、まさか具合が悪いのではあるまい。	
\\	最近[さいきん] 父[ちち]から 手紙[てがみ]が 来[こ]ないけれど、まさか 具合[ぐあい]が 悪[わる]いのではあるまい。
\\	ー七十のおじいさんが高校に入ったんだって。 ーまさか。	ー
\\	ー七十[ななじゅう]のおじいさんが 高校[こうこう]に 入[はい]ったんだって。 ーまさか。
\\	まさか自分が交通事故に巻き込まれるとは思いませんでしたよ。	
\\	まさか 自分[じぶん]が 交通[こうつう] 事故[じこ]に 巻き込[まきこ]まれるとは 思[おも]いませんでしたよ。
\\	まさか司法試験に一度でパス出来るとは考えてもいなかった。	
\\	まさか 司法[しほう] 試験[しけん]に 一度[いちど]でパス 出来[でき]るとは 考[かんが]えてもいなかった。
\\	青い顔をしているけど、まさか病気じゃないでしょうね。	
\\	青[あお]い 顔[かお]をしているけど、まさか 病気[びょうき]じゃないでしょうね。
\\	まさかこんな高いダイヤモンドの指輪を買ってくるたんじゃないでしょうね。	
\\	まさかこんな 高[たか]いダイヤモンドの 指輪[ゆびわ]を 買[か]ってくるたんじゃないでしょうね。
\\	まさかあんないい人が人を殺すなんてあるはずがない。	
\\	まさかあんないい 人[ひと]が 人[ひと]を 殺[ころ]すなんてあるはずがない。
\\	まさか彼が日本語の先生になるとは考えてもみなかった。	
\\	まさか 彼[かれ]が 日本語[にほんご]の 先生[せんせい]になるとは 考[かんが]えてもみなかった。
\\	まさかあの人がそんなことを言うはずはないでしょ。	
\\	まさかあの 人[ひと]がそんなことを 言[い]うはずはないでしょ。
\\	もう五月なのだから、まさか雪は降るまい。	
\\	もう 五月[ごがつ]なのだから、まさか 雪[ゆき]は 降[ふ]るまい。
\\	母は入院中だから、まさか私の結婚式に出席するわけにはいくまい。	
\\	母[はは]は 入院[にゅういん] 中[ちゅう]だから、まさか 私[わたし]の 結婚式[けっこんしき]に 出席[しゅっせき]するわけにはいくまい。
\\	ースミスさんは日本語は一年しか勉強していないのに、ぺらぺらですよ。 ーまさか。	
\\	ースミスさんは 日本語[にほんご]は 一年[いちねん]しか 勉強[べんきょう]していないのに、ぺらぺらですよ。 ーまさか。
\\	まさかの時に備えて貯金をしておいた方がいいよ。	
\\	まさかの 時[とき]に 備[そな]えて 貯金[ちょきん]をしておいた 方[ほう]がいいよ。
\\	こんな給料をもらうなら辞めた方がましだ。	
\\	こんな 給料[きゅうりょう]をもらうなら 辞[や]めた 方[ほう]がましだ。
\\	今度の日本語の先生は前の先生よりずっとましだ。	
\\	今度[こんど]の 日本語[にほんご]の 先生[せんせい]は 前[まえ]の 先生[せんせい]よりずっとましだ。
\\	もう少しましなコーヒーはありませんか。	
\\	もう 少[すこ]しましなコーヒーはありませんか。
\\	こんな大学に入るくらいなら、仕事をした方がましだ。	
\\	こんな 大学[だいがく]に 入[はい]るくらいなら、 仕事[しごと]をした 方[ほう]がましだ。
\\	こんな苦しい生活をするくらいなら、死んだ方がましだ。	
\\	こんな 苦[くる]しい 生活[せいかつ]をするくらいなら、 死[し]んだ 方[ほう]がましだ。
\\	学校へ行くくらいなら家でテレビでも見ていた方がましだ。	
\\	学校[がっこう]へ 行[い]くくらいなら 家[いえ]でテレビでも 見[み]ていた 方[ほう]がましだ。
\\	こんなまずいご飯を食べるくらいなら何も食べない方がいい。	
\\	こんなまずいご 飯[はん]を 食[た]べるくらいなら 何[なに]も 食[た]べない 方[ほう]がいい。
\\	あのレストランよりこのレストランの方がましだ。	
\\	あのレストランよりこのレストランの 方[ほう]がましだ。
\\	ここの夏も暑いですが、東京の夏よりましですね。	
\\	ここの 夏[なつ]も 暑[あつ]いですが、 東京[とうきょう]の 夏[なつ]よりましですね。
\\	僕の車もとても古いけど、君のよりましだ。	
\\	僕[ぼく]の 車[くるま]もとても 古[ふる]いけど、 君[きみ]のよりましだ。
\\	給料は二万円でも、ないよりましだ。	
\\	給料[きゅうりょう]は 二万[にまん] 円[えん]でも、ないよりましだ。
\\	お宅は狭いと言っても私の家に比べたらずっとましですよ。	
\\	お 宅[たく]は 狭[せま]いと 言[い]っても 私[わたし]の 家[いえ]に 比[くら]べたらずっとましですよ。
\\	もう少しましな人間になろうと思っています。	
\\	もう 少[すこ]しましな 人間[にんげん]になろうと 思[おも]っています。
\\	このホテルはひどいですね。この辺にもう少しましなホテルはないんでしょうか。	
\\	このホテルはひどいですね。この 辺[へん]にもう 少[すこ]しましなホテルはないんでしょうか。
\\	現金か、または小切手でお払い下さい。	
\\	現金[げんきん]か、または 小切手[こぎって]でお 払[はら]い 下[くだ]さい。
\\	、または
\\	のどちらかを選びなさい。	
\\	、または 
\\	のどちらかを 選[えら]びなさい。
\\	電話番号を聞き違えたか、またはもうこの電話は使われていないのだろう。	
\\	電話[でんわ] 番号[ばんごう]を 聞[き]き 違[ちが]えたか、またはもうこの 電話[でんわ]は 使[つか]われていないのだろう。
\\	黒か、または青のボールペンを使って下さい。	
\\	黒[くろ]か、または 青[あお]のボールペンを 使[つか]って 下[くだ]さい。
\\	三年以下の懲役、または百万円以下の罰金。	
\\	三年[さんねん] 以下[いか]の 懲役[ちょうえき]、または 百[ひゃく] 万[まん] 円[えん] 以下[いか]の 罰金[ばっきん]。
\\	昼は仕事があるから、夜間コースを取るか、または家庭教師を探すしかない。	
\\	昼[ひる]は 仕事[しごと]があるから、 夜間[やかん]コースを 取[と]るか、または 家庭[かてい] 教師[きょうし]を 探[さが]すしかない。
\\	二週間前に出したはずの手紙がまだ先方に着いていない。私の秘書が出し忘れたか、または郵便局が間違えたのだろう。	
\\	二週間[にしゅうかん] 前[まえ]に 出[だ]したはずの 手紙[てがみ]がまだ 先方[せんぽう]に 着[つ]いていない。 私[わたし]の 秘書[ひしょ]が 出[だ]し 忘[わす]れたか、または 郵便[ゆうびん] 局[きょく]が 間違[まちが]えたのだろう。
\\	車で行きますか。それとも、飛行機で行きますか。	
\\	車[くるま]で 行[い]きますか。それとも、 飛行機[ひこうき]で 行[い]きますか。
\\	私の車は右から二台目です。	
\\	私[わたし]の 車[くるま]は 右[みぎ]から 二台[にだい] 目[め]です。
\\	それは上から三つ目の引き出しに入っています。	
\\	それは 上[うえ]から 三[みっ]つ 目[め]の 引き出[ひきだ]しに 入[はい]っています。
\\	山本先生は前から二列目、左から三人目の人だ。	
\\	山本[やまもと] 先生[せんせい]は 前[まえ]から 二列[にれつ] 目[め]、 左[ひだり]から 三人[さんにん] 目[め]の 人[ひと]だ。
\\	一回目は失敗した。	
\\	一回[いっかい] 目[め]は 失敗[しっぱい]した。
\\	この日本語プログラムは今年で五年目になる。	
\\	この 日本語[にほんご]プログラムは 今年[ことし]で 五年[ごねん] 目[め]になる。
\\	五週目からは林先生がこのクラスを教えになります。	
\\	五週[ごしゅう] 目[め]からは 林[はやし] 先生[せんせい]がこのクラスを 教[おし]えになります。
\\	上野さんは一番目に演奏する。	
\\	上野[うえの]さんは 一番目[いちばんめ]に 演奏[えんそう]する。
\\	ー今のバイオリンは何台目ですか。 ー六台目です。	ー
\\	ー 今[いま]のバイオリンは 何[なん] 台[だい] 目[め]ですか。 ー六台[ろくだい] 目[め]です。
\\	彼は学業面では優秀だが、精神面に少し弱さがあるようだ。	
\\	彼[かれ]は 学業[がくぎょう] 面[めん]では 優秀[ゆうしゅう]だが、 精神[せいしん] 面[めん]に 少[すこ]し 弱[よわ]さがあるようだ。
\\	この食堂は衛生面によく気を配っている。	
\\	この 食堂[しょくどう]は 衛生[えいせい] 面[めん]によく 気[き]を 配[くば]っている。
\\	藤田さんは技術面からのみものを見る傾向がある。	
\\	藤田[ふじた]さんは 技術[ぎじゅつ] 面[めん]からのみものを 見[み]る 傾向[けいこう]がある。
\\	ここの土地所有者は税金面で優遇措置を受けている。	
\\	ここの 土地[とち] 所有[しょゆう] 者[しゃ]は 税金[ぜいきん] 面[めん]で 優遇[ゆうぐう] 措置[そち]を 受[う]けている。
\\	あの候補者の演説は政策面での説得力に欠ける。	
\\	あの 候補[こうほ] 者[しゃ]の 演説[えんぜつ]は 政策[せいさく] 面[めん]での 説得[せっとく] 力[りょく]に 欠[か]ける。
\\	彼女は日本へ行っても語学面は心配しなくてもいい。	
\\	彼女[かのじょ]は 日本[にほん]へ 行[い]っても 語学[ごがく] 面[めん]は 心配[しんぱい]しなくてもいい。
\\	この車は性能面を高く評価された。	
\\	この 車[くるま]は 性能[せいのう] 面[めん]を 高[たか]く 評価[ひょうか]された。
\\	彼の研究はいろいろの面で高い性能を受けている。	
\\	彼[かれ]の 研究[けんきゅう]はいろいろの 面[めん]で 高[たか]い 性能[せいのう]を 受[う]けている。
\\	僕は小説を書いて、芥川賞を取ってみせる。	
\\	僕[ぼく]は 小説[しょうせつ]を 書[か]いて、 芥川賞[あくたがわしょう]を 取[と]ってみせる。
\\	僕はこの会社の社長になってみせる。	
\\	僕[ぼく]はこの 会社[かいしゃ]の 社長[しゃちょう]になってみせる。
\\	私はあのハンサムな男の子と結婚してみせるわ。	
\\	私[わたし]はあのハンサムな 男の子[おとこのこ]と 結婚[けっこん]してみせるわ。
\\	三百ページの本を一時間で読んでみせるぞ。	
\\	三百[さんびゃく]ページの 本[ほん]を 一時間[いちじかん]で 読[よ]んでみせるぞ。
\\	今年こそは修士論文を書き上げてみせる。	
\\	今年[ことし]こそは 修士[しゅうし] 論文[ろんぶん]を 書き上[かきあ]げてみせる。
\\	一男は百メートルを十一秒で走ってみせた。	
\\	一男[かずお]は 百[ひゃく]メートルを 十一秒[じゅういちびょう]で 走[はし]ってみせた。
\\	私はその酒を一息で飲みほしてみせた。	
\\	私[わたし]はその 酒[さけ]を 一息[ひといき]で 飲[の]みほしてみせた。
\\	百メートルを十一秒で走ってみせてくれませんか。	
\\	百[ひゃく]メートルを 十一秒[じゅういちびょう]で 走[はし]ってみせてくれませんか。
\\	吉岡さんは今年七十歳だが、毎日一キロ走る。時々泳ぎもする。	
\\	吉岡[よしおか]さんは 今年[ことし]七十 歳[さい]だが、 毎日[まいにち]一[いち]キロ 走[はし]る。 時々[ときどき] 泳[およ]ぎもする。
\\	この本は有益で、その上、面白くもある。	
\\	この 本[ほん]は 有益[ゆうえき]で、その 上[うえ]、 面白[おもしろ]くもある。
\\	この映画は面白くないし、特に教育的でもない。	
\\	この 映画[えいが]は 面白[おもしろ]くないし、 特[とく]に 教育[きょういく] 的[てき]でもない。
\\	奥田氏は弁護士であり、作家でもある。	
\\	奥田[おくだ] 氏[し]は 弁護士[べんごし]であり、 作家[さっか]でもある。
\\	由利子は最近廊下ですれ違っても見向きもしない。	
\\	由利子[ゆりこ]は 最近[さいきん] 廊下[ろうか]ですれ 違[ちが]っても 見向[みむ]きもしない。
\\	猿だっておかしければ笑いもするでしょう。	
\\	猿[さる]だっておかしければ 笑[わら]いもするでしょう。
\\	健一は新婚早々三日も家を空けたが、和代は泣かなかった。そして、怒りもしなかった。	
\\	健一[けんいち]は 新婚[しんこん] 早々[そうそう] 三日[みっか]も 家[いえ]を 空[あ]けたが、 和代[かずよ]は 泣[な]かなかった。そして、 怒[いか]りもしなかった。
\\	あのアパートはあまり便利ではないし安くもない。	
\\	あのアパートはあまり 便利[べんり]ではないし 安[やす]くもない。
\\	周遊券は便利な上、経済的でもある。	
\\	周遊[しゅうゆう] 券[けん]は 便利[べんり]な 上[うえ]、 経済[けいざい] 的[てき]でもある。
\\	田口さんなんて友達でもないのに、どうしてそんなにしてあげるの。	
\\	田口[たぐち]さんなんて 友達[ともだち]でもないのに、どうしてそんなにしてあげるの。
\\	安くもないのに、どうしてそんなものを買うんですか。	
\\	安[やす]くもないのに、どうしてそんなものを 買[か]うんですか。
\\	桜の便りも聞かれる今日このごろですが、いかがお過ごしですか。	
\\	桜[さくら]の 便[たよ]りも 聞[き]かれる 今日[きょう]このごろですが、いかがお 過[す]ごしですか。
\\	行きたくなくもない。	
\\	行[い]きたくなくもない。
\\	彼の言うことは分からなくもないが、賛成は出来ない。	
\\	彼[かれ]の 言[い]うことは 分[わ]からなくもないが、 賛成[さんせい]は 出来[でき]ない。
\\	この美術館は三時間もあれば全部見られる。	
\\	この 美術館[びじゅつかん]は三 時間[じかん]もあれば 全部[ぜんぶ] 見[み]られる。
\\	二万円も持って行けば足りるでしょう。	
\\	二万[にまん] 円[えん]も 持[も]って 行[い]けば 足[た]りるでしょう。
\\	ビールは二ダースも買っておけば大丈夫だ。	
\\	ビールは二ダースも 買[か]っておけば 大丈夫[だいじょうぶ]だ。
\\	一週間もすれば歩けるようになります。	
\\	一週間[いっしゅうかん]もすれば 歩[ある]けるようになります。
\\	二、三回も聞けば大体分かる。	
\\	二[に]、三回[さんかい]も 聞[き]けば 大体[だいたい] 分[わ]かる。
\\	腰が痛くて立ちも座りも出来ない。	
\\	腰[こし]が 痛[いた]くて 立[た]ちも 座[すわ]りも 出来[でき]ない。
\\	私は俳句が大好きで、よく読みもするし自分で作りもする。	
\\	私[わたし]は 俳句[はいく]が 大好[だいす]きで、よく 読[よ]みもするし 自分[じぶん]で 作[つく]りもする。
\\	あのアパートはよくも安くもない。	
\\	あのアパートはよくも 安[やす]くもない。
\\	彼の作品は面白くもあり読みやすくもある。	
\\	彼[かれ]の 作品[さくひん]は 面白[おもしろ]くもあり 読[よ]みやすくもある。
\\	彼女の英語は特に上手でも下手でもない。	
\\	彼女[かのじょ]の 英語[えいご]は 特[とく]に 上手[じょうず]でも 下手[へた]でもない。
\\	彼は親戚でもなければ友達でもない。	
\\	彼[かれ]は 親戚[しんせき]でもなければ 友達[ともだち]でもない。
\\	私達のクラスには中国人の学生もいればスペイン人の学生もいる。	
\\	私[わたし] 達[たち]のクラスには 中国人[ちゅうごくじん]の 学生[がくせい]もいればスペイン 人[じん]の 学生[がくせい]もいる。
\\	私はドイツ語なら読みも話しも出来る。	
\\	私[わたし]はドイツ 語[ご]なら 読[よ]みも 話[はな]しも 出来[でき]る。
\\	この教科書は難しすぎもせずやさしすぎもせず、ちょうどいい。	
\\	この 教科書[きょうかしょ]は 難[むずか]しすぎもせずやさしすぎもせず、ちょうどいい。
\\	ボブの奥さんは料理もしなければ掃除もしない。	
\\	ボブの 奥[おく]さんは 料理[りょうり]もしなければ 掃除[そうじ]もしない。
\\	この物語は面白くもなければおかしくもない。	
\\	この 物語[ものがたり]は 面白[おもしろ]くもなければおかしくもない。
\\	奥田氏は弁護士でもあり、作家でもある。	
\\	奥田[おくだ] 氏[し]は 弁護士[べんごし]でもあり、 作家[さっか]でもある。
\\	その言葉の意味は字引も引いたし友達にも聞いたが結局分からなかった。	
\\	その 言葉[ことば]の 意味[いみ]は 字引[じびき]も 引[ひ]いたし 友達[ともだち]にも 聞[き]いたが 結局[けっきょく] 分[わ]からなかった。
\\	あの人は頭もいいしハンサムでもあるが、どうも好きになれない。	
\\	あの 人[ひと]は 頭[あたま]もいいしハンサムでもあるが、どうも 好[す]きになれない。
\\	現在のコンピュータのほとんどはフォン・ノイマン型と呼ばれているものである。	
\\	現在[げんざい]のコンピュータのほとんどはフォン・ノイマン 型[がた]と 呼[よ]ばれているものである。
\\	社が盲人用障害物感知器の試作品を発表した。この装置は目の不自由な人が皮膚への刺激を通して、前にある障害物などを感知出来るようにしたものである。	
\\	社[しゃ]が 盲人[もうじん] 用[よう] 障害[しょうがい] 物[ぶつ] 感知[かんち] 器[き]の 試作[しさく] 品[ひん]を 発表[はっぴょう]した。この 装置[そうち]は 目[め]の 不自由[ふじゆう]な 人[ひと]が 皮膚[ひふ]への 刺激[しげき]を 通[とお]して、 前[まえ]にある 障害[しょうがい] 物[ぶつ]などを 感知[かんち] 出来[でき]るようにしたものである。
\\	米国の貿易赤字はしばらくこのまま続くものと予想される。	
\\	米国[べいこく]の 貿易[ぼうえき] 赤字[あかじ]はしばらくこのまま 続[つづ]くものと 予想[よそう]される。
\\	この本の内容はいかなる人間社会にも当てはまる一般的なものである。	
\\	この 本[ほん]の 内容[ないよう]はいかなる 人間[にんげん] 社会[しゃかい]にも 当[あ]てはまる 一般[いっぱん] 的[てき]なものである。
\\	これは革命的発見とも言うべきものだ。	
\\	これは 革命[かくめい] 的[てき] 発見[はっけん]とも 言[い]うべきものだ。
\\	社が
\\	型ワープロの五パーセント値下げに踏み切った。これは、最近低下している同社のワープロシェアの巻き返しを狙ったもの。	
\\	社[しゃ]が 
\\	型[がた]ワープロの五[ご]パーセント 値下[ねさ]げに 踏み切[ふみき]った。これは、 最近[さいきん] 低下[ていか]している 同社[どうしゃ]のワープロシェアの 巻き返[まきかえ]しを 狙[ねら]ったもの。
\\	大学入試制度調査会の第一回会合が昨日都内のホテルで行われた。この調査会は、最近しばしば批判の対象になっている、現行の大学入試制度の見直しのために発足したものである。	
\\	大学[だいがく] 入試[にゅうし] 制度[せいど] 調査[ちょうさ] 会[かい]の 第[だい] 一回[いっかい] 会合[かいごう]が 昨日[きのう] 都内[とない]のホテルで 行[おこな]われた。この 調査[ちょうさ] 会[かい]は、 最近[さいきん]しばしば 批判[ひはん]の 対象[たいしょう]になっている、 現行[げんこう]の 大学[だいがく] 入試[にゅうし] 制度[せいど]の 見直[みなお]しのために 発足[ほっそく]したものである。
\\	この問題は時期が来れば自然に解決されるものと見られている。	
\\	この 問題[もんだい]は 時期[じき]が 来[く]れば 自然[しぜん]に 解決[かいけつ]されるものと 見[み]られている。
\\	エレクトロニクスの世界は今後も急速な進歩を続けていくものと予想される。	
\\	エレクトロニクスの 世界[せかい]は 今後[こんご]も 急速[きゅうそく]な 進歩[しんぽ]を 続[つづ]けていくものと 予想[よそう]される。
\\	現在のコンピュータのほとんどはフォン・ノイマン型と呼ばれている。	
\\	現在[げんざい]のコンピュータのほとんどはフォン・ノイマン 型[がた]と 呼[よ]ばれている。
\\	彼はよく食べるなあ。	
\\	彼[かれ]はよく 食[た]べるなあ。
\\	すばらしい映画だなあと思った。	
\\	すばらしい 映画[えいが]だなあと 思[おも]った。
\\	まだ間に合うかなあ。	
\\	まだ 間に合[まにあ]うかなあ。
\\	河合君、早く来ないかなあ。	
\\	河合[かわい] 君[くん]、 早[はや]く 来[こ]ないかなあ。
\\	車があったらいいなあ。	
\\	車[くるま]があったらいいなあ。
\\	よく飲んだなあ。	
\\	よく 飲[の]んだなあ。
\\	あのけちの吉田がよく金を出したなあ。	
\\	あのけちの 吉田[よしだ]がよく 金[きん]を 出[だ]したなあ。
\\	家が欲しいなあ。	
\\	家[いえ]が 欲[ほ]しいなあ。
\\	山内さん、今日は来ないかなあ。	
\\	山内[やまうち]さん、 今日[きょう]は 来[こ]ないかなあ。
\\	何か面白い映画はないかなあ。	
\\	何[なに]か 面白[おもしろ]い 映画[えいが]はないかなあ。
\\	彼女、こんなことを言っても怒らないかなあ。	
\\	彼女[かのじょ]、こんなことを 言[い]っても 怒[おこ]らないかなあ。
\\	タイガース、今日勝たないかなあ。	
\\	タイガース、 今日[きょう] 勝[か]たないかなあ。
\\	きれいだなあ。	
\\	きれいだなあ。
\\	面白かったなあ。	
\\	面白[おもしろ]かったなあ。
\\	僕は幸せ者だなあ。	
\\	僕[ぼく]は 幸[しあわ]せ 者[もの]だなあ。
\\	高いなあ。	
\\	高[たか]いなあ。
\\	いいなあ。	
\\	いいなあ。
\\	かわいそうだなあ。	
\\	かわいそうだなあ。
\\	馬鹿だなあ。	
\\	馬鹿[ばか]だなあ。
\\	ひどいなあ。	
\\	ひどいなあ。
\\	楽しかったなあ、あの頃は。	
\\	楽[たの]しかったなあ、あの 頃[ころ]は。
\\	よく頑張ったなあ、お互いに。	
\\	よく 頑張[がんば]ったなあ、お 互[たが]いに。
\\	うらやましいなあ、書斎があって。	
\\	うらやましいなあ、 書斎[しょさい]があって。
\\	日本の文化はユニークだなどと言う人がいるが、私はそうは思わない。	
\\	日本[にほん]の 文化[ぶんか]はユニークだなどと 言[い]う 人[ひと]がいるが、 私[わたし]はそうは 思[おも]わない。
\\	お金がないから、日本へ行くなどということは夢です。	
\\	お 金[かね]がないから、 日本[にほん]へ 行[い]くなどということは 夢[ゆめ]です。
\\	毎日運動しろなどとは言いませんから、週に二、三度運動したらどうですか。	
\\	毎日[まいにち] 運動[うんどう]しろなどとは 言[い]いませんから、 週[しゅう]に二[に]、三度[さんど] 運動[うんどう]したらどうですか。
\\	ジョンソンさんは日本語で何でも話せるなどと言っている。	
\\	ジョンソンさんは 日本語[にほんご]で 何[なに]でも 話[はな]せるなどと 言[い]っている。
\\	困った時は人が助けてくれるだろうなどと甘く考えてはいけない。	
\\	困[こま]った 時[とき]は 人[ひと]が 助[たす]けてくれるだろうなどと 甘[あま]く 考[かんが]えてはいけない。
\\	日本へ留学しようとしているケントさんは、うまく日本の生活に適応出来るかなどと心配している。	
\\	日本[にほん]へ 留学[りゅうがく]しようとしているケントさんは、うまく 日本[にほん]の 生活[せいかつ]に 適応[てきおう] 出来[でき]るかなどと 心配[しんぱい]している。
\\	みんなが君のことをクラブのホープだなどと言っているよ。	
\\	みんなが 君[きみ]のことをクラブのホープだなどと 言[い]っているよ。
\\	尊敬しています、などと言われると照れくさい。	
\\	尊敬[そんけい]しています、などと 言[い]われると 照[て]れくさい。
\\	田中さんや小川さんなどが来ました。	
\\	田中[たなか]さんや 小川[おがわ]さんなどが 来[き]ました。
\\	すしやてんぷらなどを食べました。	
\\	すしやてんぷらなどを 食[た]べました。
\\	私などにはこんな仕事は出来ません。	
\\	私[わたし]などにはこんな 仕事[しごと]は 出来[でき]ません。
\\	鈴木などにこの問題が分かるはずがない。	
\\	鈴木[すずき]などにこの 問題[もんだい]が 分[わ]かるはずがない。
\\	生卵など、食べないよ。	
\\	生[なま] 卵[たまご]など、 食[た]べないよ。
\\	この部屋は狭いながらも居心地がよい。	
\\	この 部屋[へや]は 狭[せま]いながらも 居心地[いごこち]がよい。
\\	山口はそのことを知りながら黙っていた。	
\\	山口[やまぐち]はそのことを 知[し]りながら 黙[だま]っていた。
\\	あの人は若いながらも、なかなか有能だ。	
\\	あの 人[ひと]は 若[わか]いながらも、なかなか 有能[ゆうのう]だ。
\\	教授のゼミは厳しいながら学ぶところが多かった。	
\\	教授[きょうじゅ]のゼミは 厳[きび]しいながら 学[まな]ぶところが 多[おお]かった。
\\	ベッキーの日本語はたどたどしいながらも、言おうとしていることは分かった。	
\\	ベッキーの 日本語[にほんご]はたどたどしいながらも、 言[い]おうとしていることは 分[わ]かった。
\\	この辺は不便ながらも、車も少なく、空気がきれいだ。	
\\	この 辺[へん]は 不便[ふべん]ながらも、 車[くるま]も 少[すく]なく、 空気[くうき]がきれいだ。
\\	残念ながら、明日の音楽会には行けません。	
\\	残念[ざんねん]ながら、 明日[あした]の 音楽[おんがく] 会[かい]には 行[い]けません。
\\	ひとみは子供ながら、よく考えてものを言う。	
\\	ひとみは 子供[こども]ながら、よく 考[かんが]えてものを 言[い]う。
\\	この車は小型ながらもよく走る。	
\\	この 車[くるま]は 小型[こがた]ながらもよく 走[はし]る。
\\	日本の単身赴任のサラリーマンは苦しみながらも、会社のために働いている。	
\\	日本[にほん]の 単身[たんしん] 赴任[ふにん]のサラリーマンは 苦[くる]しみながらも、 会社[かいしゃ]のために 働[はたら]いている。
\\	ボブは日本にいながら、洋食ばかり食べている。	
\\	ボブは 日本[にほん]にいながら、 洋食[ようしょく]ばかり 食[た]べている。
\\	たばこは体に悪いと分かっていながらも、吸ってしまう。	
\\	たばこは 体[からだ]に 悪[わる]いと 分[わ]かっていながらも、 吸[す]ってしまう。
\\	弟はたくさん本を買っていながら、少しも読んでいない。	
\\	弟[おとうと]はたくさん 本[ほん]を 買[か]っていながら、 少[すこ]しも 読[よ]んでいない。
\\	彼は何度も日本へ行っていながら、日本語が少しも話せない。	
\\	彼[かれ]は 何[なん] 度[ど]も 日本[にほん]へ 行[い]っていながら、 日本語[にほんご]が 少[すこ]しも 話[はな]せない。
\\	彼は倒れながらピストルの引き金を引こうとした。	
\\	彼[かれ]は 倒[たお]れながらピストルの 引き金[ひきがね]を 引[ひ]こうとした。
\\	彼は倒れながらもピストルの引き金を引こうとした。	
\\	彼[かれ]は 倒[たお]れながらもピストルの 引き金[ひきがね]を 引[ひ]こうとした。
\\	ー日本語の新聞は読まないんですか。 ーいいえ、読まないこともないんですが、時たまですね。	
\\	ー 日本語[にほんご]の 新聞[しんぶん]は 読[よ]まないんですか。 ーいいえ、 読[よ]まないこともないんですが、 時[とき]たまですね。
\\	ーこの辺は夏涼しくないんですか。 ーいや、涼しくないこともないんですが、時々ひどく暑くなります。	ー
\\	ーこの 辺[へん]は 夏[なつ] 涼[すず]しくないんですか。 ーいや、 涼[すず]しくないこともないんですが、 時々[ときどき]ひどく 暑[あつ]くなります。
\\	ーお父さんはお元気じゃないんですか。 ーいや、元気じゃないことはないんですが、血圧が少し高いようです。	
\\	ーお 父[とう]さんはお 元気[げんき]じゃないんですか。 ーいや、 元気[げんき]じゃないことはないんですが、 血圧[けつあつ]が 少[すこ]し 高[たか]いようです。
\\	ー日本語は難しくありませんか。 ーいいえ、難しくないこともないんですが、日本語の難しさは強調されすぎていると思いますよ。	
\\	ー 日本語[にほんご]は 難[むずか]しくありませんか。 ーいいえ、 難[むずか]しくないこともないんですが、 日本語[にほんご]の 難[むずか]しさは 強調[きょうちょう]されすぎていると 思[おも]いますよ。
\\	ー山田さんは政治学者じゃないんですね。 ーいや、政治学者じゃないこともないんですが、どちらかというと政治家です。	
\\	ー 山田[やまだ]さんは 政治[せいじ] 学者[がくしゃ]じゃないんですね。 ーいや、 政治[せいじ] 学者[がくしゃ]じゃないこともないんですが、どちらかというと 政治[せいじ] 家[か]です。
\\	日本人は集団行動が好きだとよく言われている。確かに、集団行動をしないことはないのだが、個人行動をとる日本人もけっこういる。	
\\	日本人[にほんじん]は 集団[しゅうだん] 行動[こうどう]が 好[す]きだとよく 言[い]われている。 確[たし]かに、 集団[しゅうだん] 行動[こうどう]をしないことはないのだが、 個人[こじん] 行動[こうどう]をとる 日本人[にほんじん]もけっこういる。
\\	ー酒は飲まないそうじゃありませんか。 ー酒は飲まないことはないんですが、せいぜいおちょこに二、三杯ですね。	ー
\\	ー 酒[しゅ]は 飲[の]まないそうじゃありませんか。 ー 酒[しゅ]は 飲[の]まないことはないんですが、せいぜいおちょこに 二[に]、 三杯[さんぱい]ですね。
\\	ー日本語の新聞は読みますか。 ーそうですね、読むことは読みますが、時たまですね。	ー
\\	ー 日本語[にほんご]の 新聞[しんぶん]は 読[よ]みますか。 ーそうですね、 読[よ]むことは 読[よ]みますが、 時[とき]たまですね。
\\	ーこの辺は夏涼しいんですか。 ーそうですね、涼しいことは涼しいですが、時々ひどく暑くなります。	
\\	ーこの 辺[あたり]は 夏[なつ] 涼[すず]しいんですか。 ーそうですね、 涼[すず]しいことは 涼[すず]しいですが、 時々[ときどき]ひどく 暑[あつ]くなります。
\\	ブラウンさんは日本語がなかなか上手ですね。	
\\	ブラウンさんは 日本語[にほんご]がなかなか 上手[じょうず]ですね。
\\	この生け花はなかなか見事に生けてある。	
\\	この 生け花[いけばな]はなかなか 見事[みごと]に 生[い]けてある。
\\	今年は桜の花がなかなか咲きません。	
\\	今年[ことし]は 桜[さくら]の 花[はな]がなかなか 咲[さ]きません。
\\	あの男は仕事がなかなか出来るねえ。	
\\	あの 男[おとこ]は 仕事[しごと]がなかなか 出来[でき]るねえ。
\\	山田部長の奥さんはなかなかの美人だ。	
\\	山田[やまだ] 部長[ぶちょう]の 奥[おく]さんはなかなかの 美人[びじん]だ。
\\	この庭園はなかなかきれいですね。	
\\	この 庭園[ていえん]はなかなかきれいですね。
\\	この数学の問題はなかなか難しそうだ。	
\\	この 数学[すうがく]の 問題[もんだい]はなかなか 難[むずか]しそうだ。
\\	君はなかなかコンピュータに強いそうじゃないか。	
\\	君[きみ]はなかなかコンピュータに 強[つよ]いそうじゃないか。
\\	この大学の学生はなかなかよく出来る。	
\\	この 大学[だいがく]の 学生[がくせい]はなかなかよく 出来[でき]る。
\\	あのピアニストはモーツァルトをなかなかうまく弾く。	
\\	あのピアニストはモーツァルトをなかなかうまく 弾[ひ]く。
\\	テニスは毎日練習しているのに、なかなか上手にならないんです。	
\\	テニスは 毎日[まいにち] 練習[れんしゅう]しているのに、なかなか 上手[じょうず]にならないんです。
\\	約束の時間になっても友達がなかなか来なかったので家に帰ってしまった。	
\\	約束[やくそく]の 時間[じかん]になっても 友達[ともだち]がなかなか 来[こ]なかったので 家[いえ]に 帰[かえ]ってしまった。
\\	あの人はどんなに説明しても、僕の言うことがなかなか分からないようだ。	
\\	あの 人[ひと]はどんなに 説明[せつめい]しても、 僕[ぼく]の 言[い]うことがなかなか 分[わ]からないようだ。
\\	風邪がなかなか治らないので、困っているんです。	
\\	風邪[かぜ]がなかなか 治[なお]らないので、 困[こま]っているんです。
\\	フグは怖くて、なかなか食べられないんだ。	
\\	フグは 怖[こわ]くて、なかなか 食[た]べられないんだ。
\\	中村先生はなかなか話せる先生だ。	
\\	中村[なかむら] 先生[せんせい]はなかなか 話[はな]せる 先生[せんせい]だ。
\\	私のルームメートはなかなか勉強家で、かなわない。	
\\	私[わたし]のルームメートはなかなか 勉強[べんきょう] 家[か]で、かなわない。
\\	この本はそんなに面白くない。	
\\	この 本[ほん]はそんなに 面白[おもしろ]くない。
\\	この映画は全くつまらない。	
\\	この 映画[えいが]は 全[まった]くつまらない。
\\	君のテニスはとても下手だね。	
\\	君[きみ]のテニスはとても 下手[へた]だね。
\\	十二月になったら、とても寒くなった。	
\\	十二月[じゅうにがつ]になったら、とても 寒[さむ]くなった。
\\	この絵はなかなかよく描けている。	
\\	この 絵[え]はなかなかよく 描[か]けている。
\\	彼はなかなかのプレーボーイだ。	
\\	彼[かれ]はなかなかのプレーボーイだ。
\\	こんな小さい会社なのに、けっこう質のいい製品を出しているね。	
\\	こんな 小[ちい]さい 会社[かいしゃ]なのに、けっこう 質[しつ]のいい 製品[せいひん]を 出[だ]しているね。
\\	あの人は見かけによらず、かなり飲みますね。	
\\	あの 人[ひと]は 見[み]かけによらず、かなり 飲[の]みますね。
\\	東京は日中三十七度もあり、大変暑った。	
\\	東京[とうきょう]は 日中[にっちゅう] 三十七度[さんじゅうしちど]もあり、 大変[たいへん] 暑[あつ]った。
\\	日本語の期末試験はあまり難しくなく、ほっとした。	
\\	日本語[にほんご]の 期末[きまつ] 試験[しけん]はあまり 難[むずか]しくなく、ほっとした。
\\	私の研究対象は現代史ではなく、古代史だ。	
\\	私[わたし]の 研究[けんきゅう] 対象[たいしょう]は 現代[げんだい] 史[し]ではなく、 古代[こだい] 史[し]だ。
\\	今年の冬はあまり寒くなく、オーバーも二、三度しか着なかった。	
\\	今年[ことし]の 冬[ふゆ]はあまり 寒[さむ]くなく、オーバーも二[に]、三度[さんど]しか 着[き]なかった。
\\	昨日見た映画は実に面白くなく、途中で寝てしまった。	
\\	昨日[きのう] 見[み]た 映画[えいが]は 実[じつ]に 面白[おもしろ]くなく、 途中[とちゅう]で 寝[ね]てしまった。
\\	私には日本語は聞くのが容易ではなく、大分苦労した。	
\\	私[わたし]には 日本語[にほんご]は 聞[き]くのが 容易[ようい]ではなく、 大分[だいぶ] 苦労[くろう]した。
\\	彼女と別れた時はそれほど悲しくなく、自分でも驚いた。	
\\	彼女[かのじょ]と 別[わか]れた 時[とき]はそれほど 悲[かな]しくなく、 自分[じぶん]でも 驚[おどろ]いた。
\\	私に日本語を教えてくれたのは日本人ではなく、アメリカ人だった。	
\\	私[わたし]に 日本語[にほんご]を 教[おし]えてくれたのは 日本人[にほんじん]ではなく、アメリカ 人[じん]だった。
\\	僕が好きなのは日本料理ではなくタイ料理だ。	
\\	僕[ぼく]が 好[す]きなのは 日本[にほん] 料理[りょうり]ではなくタイ 料理[りょうり]だ。
\\	うちの子は本を読まなくて困るんです。	
\\	うちの 子[こ]は 本[ほん]を 読[よ]まなくて 困[こま]るんです。
\\	主人は酒をあまり飲まなくて助かっています。	
\\	主人[しゅじん]は 酒[さけ]をあまり 飲[の]まなくて 助[たす]かっています。
\\	うちの子は本を読まなくなりました。	
\\	うちの 子[こ]は 本[ほん]を 読[よ]まなくなりました。
\\	主人は酒を飲まなくなりました。	
\\	主人[しゅじん]は 酒[さけ]を 飲[の]まなくなりました。
\\	腫瘍が悪性でなく、ほっとした。	
\\	腫瘍[しゅよう]が 悪性[あくしょう]でなく、ほっとした。
\\	書いた本が好評でなく、がっかりした。	
\\	書[か]いた 本[ほん]が 好評[こうひょう]でなく、がっかりした。
\\	日本語の新聞が読めず、困った。	
\\	日本語[にほんご]の 新聞[しんぶん]が 読[よ]めず、 困[こま]った。
\\	住所も電話番号も分からず、連絡が出来なかった。	
\\	住所[じゅうしょ]も 電話[でんわ] 番号[ばんごう]も 分[わ]からず、 連絡[れんらく]が 出来[でき]なかった。
\\	簡単な質問に答えられず、恥をかいた。	
\\	簡単[かんたん]な 質問[しつもん]に 答[こた]えられず、 恥[はじ]をかいた。
\\	何でも山本さんは奥さんと別れて、一人で暮らしているそうですよ。	
\\	何[なん]でも 山本[やまもと]さんは 奥[おく]さんと 別[わか]れて、一人[ひとり]で 暮[く]らしているそうですよ。
\\	何でもこの辺は物価が非常に高いそうですよ。	
\\	何[なん]でもこの 辺[へん]は 物価[ぶっか]が 非常[ひじょう]に 高[たか]いそうですよ。
\\	何でもあの人は株で大分もうけたようですよ。	
\\	何[なん]でもあの 人[ひと]は 株[かぶ]で 大分[だいぶ]もうけたようですよ。
\\	何でも戸田さん息子さんはシカゴ大学で経営学修士を取ったらしいですよ。	
\\	何[なん]でも 戸田[とだ]さん 息子[むすこ]さんはシカゴ 大学[だいがく]で 経営[けいえい] 学修[がくしゅう] 士[し]を 取[と]ったらしいですよ。
\\	何でも日本とアメリカの西海岸を五時間ぐらいで飛ぶ飛行機を開発しているという話ですよ。	
\\	何[なん]でも 日本[にほん]とアメリカの 西海岸[にしかいがん]を五 時間[じかん]ぐらいで 飛[と]ぶ 飛行機[ひこうき]を 開発[かいはつ]しているという 話[はなし]ですよ。
\\	何でもあの二人は結婚するらしいですね。	
\\	何[なん]でもあの 二人[ふたり]は 結婚[けっこん]するらしいですね。
\\	うちの父は頑固なんだ。何しろ一度言い出したら絶対に引き下がらないんだから。	
\\	うちの 父[ちち]は 頑固[がんこ]なんだ。 何[なに]しろ 一度[いちど] 言い出[いいだ]したら 絶対[ぜったい]に 引き下[ひきさ]がらないんだから。
\\	ーこの大学の教育はいいらしいね。 ーうん、何しろ学生六人に先生一人だからね。	ー
\\	ーこの 大学[だいがく]の 教育[きょういく]はいいらしいね。 ーうん、 何[なに]しろ 学生[がくせい] 六人[ろくにん]に 先生[せんせい] 一人[ひとり]だからね。
\\	一日中ボスにがなり立てられるんで、何しろ、ストレスが多いんだ。	
\\	一日[いちにち] 中[じゅう]ボスにがなり 立[た]てられるんで、 何[なに]しろ、ストレスが 多[おお]いんだ。
\\	あの人は金持ちですよ。何しろベンツを三台も持っているんですからね。	
\\	あの 人[ひと]は 金持[かねも]ちですよ。 何[なに]しろベンツを 三台[さんだい]も 持[も]っているんですからね。
\\	今年の冬は本当に雪が多かったです。何しろ、雪の降らない日の方が少ないくらいでしたからね。	
\\	今年[ことし]の 冬[ふゆ]は 本当[ほんとう]に 雪[ゆき]が 多[おお]かったです。 何[なに]しろ、 雪[ゆき]の 降[ふ]らない 日[ひ]の 方[ほう]が 少[すく]ないくらいでしたからね。
\\	あの人はよくたばこを吸いますよ。何しろ一日に六十本ぐらい吸うんですから。	
\\	あの 人[ひと]はよくたばこを 吸[す]いますよ。 何[なに]しろ 一日[いちにち]に 六十本[ろくじっぽん]ぐらい 吸[す]うんですから。
\\	ー日本は人が多いねえ。 ーうん、何しろ、面積はアメリカの二十五分の一なのに、人口は二分の一だからね。	
\\	ー 日本[にほん]は 人[ひと]が 多[おお]いねえ。 ーうん、 何[なに]しろ、 面積[めんせき]はアメリカの 二十五分[にじゅうごぶん]の一[いち]なのに、 人口[じんこう]は 二分[にぶん]の一[いち]だからね。
\\	今日は休ませて下さい。何しろ頭が割れそうに痛いので。	
\\	今日[きょう]は 休[やす]ませて 下[くだ]さい。 何[なに]しろ 頭[あたま]が 割[わ]れそうに 痛[いた]いので。
\\	心配しなくてもいいよ。何しろ警官が二十人も見張っているから。	
\\	心配[しんぱい]しなくてもいいよ。 何[なに]しろ 警官[けいかん]が 二十人[にじゅうにん]も 見張[みは]っているから。
\\	とにかく会いに来て下さい。	
\\	とにかく 会[あ]いに 来[き]て 下[くだ]さい。
\\	とにかく手紙を出したらどうですか。	
\\	とにかく 手紙[てがみ]を 出[だ]したらどうですか。
\\	とにかくたばこをやめますか。	
\\	とにかくたばこをやめますか。
\\	病気の母のことが心配でならない。	
\\	病気[びょうき]の 母[はは]のことが 心配[しんぱい]でならない。
\\	日本の歴史をもっと深く知りたくてならない。	
\\	日本[にほん]の 歴史[れきし]をもっと 深[ふか]く 知[し]りたくてならない。
\\	一人で住んでいる母親のことが気になってならない。	
\\	一人[ひとり]で 住[す]んでいる 母親[ははおや]のことが 気[き]になってならない。
\\	夫が単身赴任しているので、寂しくてなりません。	
\\	夫[おっと]が 単身[たんしん] 赴任[ふにん]しているので、 寂[さび]しくてなりません。
\\	二人は婚約が決まって、うれしくてならないようだ。	
\\	二人[ふたり]は 婚約[こんやく]が 決[き]まって、うれしくてならないようだ。
\\	松本清張の推理小説は面白くてならない。	
\\	松本[まつもと] 清張[せいちょう]の 推理[すいり] 小説[しょうせつ]は 面白[おもしろ]くてならない。
\\	隣の家のステレオがやかましくてならない。	
\\	隣[となり]の 家[いえ]のステレオがやかましくてならない。
\\	山田先生がこの大学をお辞めになるので、残念でならない。	
\\	山田[やまだ] 先生[せんせい]がこの 大学[だいがく]をお 辞[や]めになるので、 残念[ざんねん]でならない。
\\	東京の夏は暑くてならない。	
\\	東京[とうきょう]の 夏[なつ]は 暑[あつ]くてならない。
\\	駅の階段を急いで上がったら、息が苦しくてならなかった。	
\\	駅[えき]の 階段[かいだん]を 急[いそ]いで 上[あ]がったら、 息[いき]が 苦[くる]しくてならなかった。
\\	買ってきたばかりのテレビがすぐ壊れてしまい、腹が立ってならなかった。	
\\	買[か]ってきたばかりのテレビがすぐ 壊[こわ]れてしまい、 腹[はら]が 立[た]ってならなかった。
\\	仕事がうまく行っていないので、気が滅入ってならない。	
\\	仕事[しごと]がうまく 行[い]っていないので、 気[き]が 滅入[めい]ってならない。
\\	私の家は狭すぎる。	
\\	私[わたし]の 家[いえ]は 狭[せま]すぎる。
\\	日本の物価は高すぎる。	
\\	日本[にほん]の 物価[ぶっか]は 高[たか]すぎる。
\\	真知子はひとみの美貌がうらやましくてならない。	
\\	真知子[まちこ]はひとみの 美貌[びぼう]がうらやましくてならない。
\\	暑いので、喉が渇いてたまらない。	
\\	暑[あつ]いので、 喉[のど]が 渇[かわ]いてたまらない。
\\	アレルギーのためか、しょっちゅうくしゃみが出て仕方がない。	
\\	アレルギーのためか、しょっちゅうくしゃみが 出[で]て 仕方[しかた]がない。
\\	駅まで走ったので足が痛くてたまらない。	
\\	駅[えき]まで 走[はし]ったので 足[あし]が 痛[いた]くてたまらない。
\\	文法が分からなかったら、私なり鈴木先生なりに質問しなさい。	
\\	文法[ぶんぽう]が 分[わ]からなかったら、 私[わたし]なり 鈴木[すずき] 先生[せんせい]なりに 質問[しつもん]しなさい。
\\	読めない漢字は辞書を引くなり、日本人に聞くなりしなさい。	
\\	読[よ]めない 漢字[かんじ]は 辞書[じしょ]を 引[ひ]くなり、 日本人[にほんじん]に 聞[き]くなりしなさい。
\\	私なり山田なりが空港に迎えに参ります。	
\\	私[わたし]なり 山田[やまだ]なりが 空港[くうこう]に 迎[むか]えに 参[まい]ります。
\\	何か連絡することがありましたか電話なりファックスなりでお願いします。	
\\	何[なに]か 連絡[れんらく]することがありましたか 電話[でんわ]なりファックスなりでお 願[ねが]いします。
\\	両親なり親友なりに会って、ゆっくり話したいんです。	
\\	両親[りょうしん]なり 親友[しんゆう]なりに 会[あ]って、ゆっくり 話[はな]したいんです。
\\	今は忙しくて旅行が出来ないが、春休みなり夏休みなりにするつもりだ。	
\\	今[いま]は 忙[いそが]しくて 旅行[りょこう]が 出来[でき]ないが、 春休[はるやす]みなり 夏休[なつやす]みなりにするつもりだ。
\\	テニスがしたかったら、トムなりチャールズなりとしたらいいでしょう。	
\\	テニスがしたかったら、トムなりチャールズなりとしたらいいでしょう。
\\	アパートを探しているんなら、不動産屋に行くなり、新聞の広告を見るなりしたらどうですか。	
\\	アパートを 探[さが]しているんなら、 不動産[ふどうさん] 屋[や]に 行[い]くなり、 新聞[しんぶん]の 広告[こうこく]を 見[み]るなりしたらどうですか。
\\	散歩するなり、泳ぐなり、何か運動をした方がいいですよ。	
\\	散歩[さんぽ]するなり、 泳[およ]ぐなり、 何[なに]か 運動[うんどう]をした 方[ほう]がいいですよ。
\\	床屋に行くなりして、もっと髪をきちんとしたらどうですか。	
\\	床屋[とこや]に 行[い]くなりして、もっと 髪[かみ]をきちんとしたらどうですか。
\\	私は漢字の読み方が分からない時は、先生なり日本人なりに聞くようにした。	
\\	私[わたし]は 漢字[かんじ]の 読み方[よみかた]が 分[わ]からない 時[とき]は、 先生[せんせい]なり 日本人[にほんじん]なりに 聞[き]くようにした。
\\	分からなかったら、図書館で調べるなり、先生に聞くなりしろよ。	
\\	分[わ]からなかったら、 図書館[としょかん]で 調[しら]べるなり、 先生[せんせい]に 聞[き]くなりしろよ。
\\	昨日はテニスはしたり
\\	を聞いたりした。	
\\	昨日[きのう]はテニスはしたり 
\\	を 聞[き]いたりした。
\\	酒を飲んだり、歌を歌ったり、とても楽しかった。	
\\	酒[さけ]を 飲[の]んだり、 歌[うた]を 歌[うた]ったり、とても 楽[たの]しかった。
\\	私はビールとかワインとか、何か飲み物がほしいです。	
\\	私[わたし]はビールとかワインとか、 何[なに]か 飲み物[のみもの]がほしいです。
\\	私は政治とか宗教とかいうものは嫌いです。	
\\	私[わたし]は 政治[せいじ]とか 宗教[しゅうきょう]とかいうものは 嫌[きら]いです。
\\	あの人はあの人なりによく考えているらしい。	
\\	あの 人[ひと]はあの 人[ひと]なりによく 考[かんが]えているらしい。
\\	動物は動物なりのコミュニケーションが出来る。	
\\	動物[どうぶつ]は 動物[どうぶつ]なりのコミュニケーションが 出来[でき]る。
\\	小さい大学は小さい大学なりによさがある。	
\\	小[ちい]さい 大学[だいがく]は 小[ちい]さい 大学[だいがく]なりによさがある。
\\	私は私なりに、人生観を持っています。	
\\	私[わたし]は 私[わたし]なりに、 人生[じんせい] 観[かん]を 持[も]っています。
\\	自転車は自転車なりに、車は車なりに、長所、短所がある。	
\\	自転車[じてんしゃ]は 自転車[じてんしゃ]なりに、 車[くるま]は 車[くるま]なりに、 長所[ちょうしょ]、 短所[たんしょ]がある。
\\	健康な人は健康な人なりに、体に気をつけた方がいい。	
\\	健康[けんこう]な 人[ひと]は 健康[けんこう]な 人[ひと]なりに、 体[からだ]に 気[き]をつけた 方[ほう]がいい。
\\	私は老人が老人なりに生活を楽しめる社会が好きだ。	
\\	私[わたし]は 老人[ろうじん]が 老人[ろうじん]なりに 生活[せいかつ]を 楽[たの]しめる 社会[しゃかい]が 好[す]きだ。
\\	私は私なりに一生懸命働くつもりです。	
\\	私[わたし]は 私[わたし]なりに 一生懸命[いっしょうけんめい] 働[はたら]くつもりです。
\\	この問題について、父は父なりの考えを持っているようです。	
\\	この 問題[もんだい]について、 父[ちち]は 父[ちち]なりの 考[かんが]えを 持[も]っているようです。
\\	ジャズ音楽はジャズ音楽なりの魅力がある。	
\\	ジャズ 音楽[おんがく]はジャズ 音楽[おんがく]なりの 魅力[みりょく]がある。
\\	金持ちは金持ちなりの悩みがあるのだ。	
\\	金持[かねも]ちは 金持[かねも]ちなりの 悩[なや]みがあるのだ。
\\	小さい大学はそれなりによさがある。	
\\	小[ちい]さい 大学[だいがく]はそれなりによさがある。
\\	ジャズ音楽はそれなりの魅力がある。	
\\	ジャズ 音楽[おんがく]はそれなりの 魅力[みりょく]がある。
\\	実力なしでは世の中は渡れない。	
\\	実力[じつりょく]なしでは 世の中[よのなか]は 渡[わた]れない。
\\	あなたなしでは生きていけない。	
\\	あなたなしでは 生[い]きていけない。
\\	お金なしではいい教育は受けられない。	
\\	お 金[かね]なしではいい 教育[きょういく]は 受[う]けられない。
\\	暇なしでは人生はつまらなくなってしまう。	
\\	暇[ひま]なしでは 人生[じんせい]はつまらなくなってしまう。
\\	仕事なしでは生活が出来ない。	
\\	仕事[しごと]なしでは 生活[せいかつ]が 出来[でき]ない。
\\	基礎研究なしは科学は発展しない。	
\\	基礎[きそ] 研究[けんきゅう]なしは 科学[かがく]は 発展[はってん]しない。
\\	車なしではちょっと不便だ。	
\\	車[くるま]なしではちょっと 不便[ふべん]だ。
\\	あなたなしでも生きていける。	
\\	あなたなしでも 生[い]きていける。
\\	実力なしでも世の中は渡れる。	
\\	実力[じつりょく]なしでも 世の中[よのなか]は 渡[わた]れる。
\\	ワープロなしではいい論文が書けない。	
\\	ワープロなしではいい 論文[ろんぶん]が 書[か]けない。
\\	ワープロなしには文筆作業は考えられない。	
\\	ワープロなしには 文筆[ぶんぴつ] 作業[さぎょう]は 考[かんが]えられない。
\\	仕事がなければ探しなさい。	
\\	仕事[しごと]がなければ 探[さが]しなさい。
\\	先生がいなければ自分で勉強しなさい。	
\\	先生[せんせい]がいなければ 自分[じぶん]で 勉強[べんきょう]しなさい。
\\	お金がなければ、どうしたらいいですか。	
\\	お 金[かね]がなければ、どうしたらいいですか。
\\	日本を知るためには一度は日本へ行かねばならない。	
\\	日本[にほん]を 知[し]るためには 一度[いちど]は 日本[にほん]へ 行[い]かねばならない。
\\	日本は市場をもっと広く開放せねばならない。	
\\	日本[にほん]は 市場[しじょう]をもっと 広[ひろ]く 開放[かいほう]せねばならない。
\\	今後も同じ研究を続けねばならない。	
\\	今後[こんご]も 同[おな]じ 研究[けんきゅう]を 続[つづ]けねばならない。
\\	人種偏見をなくさねばならない。	
\\	人種[じんしゅ] 偏見[へんけん]をなくさねばならない。
\\	会議の内容を帰国後本社に報告せねばならなかった。	
\\	会議[かいぎ]の 内容[ないよう]を 帰国[きこく] 後[ご] 本社[ほんしゃ]に 報告[ほうこく]せねばならなかった。
\\	自然保護を叫ばねばならない。	
\\	自然[しぜん] 保護[ほご]を 叫[さけ]ばねばならない。
\\	日本はこれから世界をリードせねばなりません。	
\\	日本[にほん]はこれから 世界[せかい]をリードせねばなりません。
\\	理想は高くなければならない。	
\\	理想[りそう]は 高[たか]くなければならない。
\\	この問題に関して我々は慎重であらねばならない。	
\\	この 問題[もんだい]に 関[かん]して 我々[われわれ]は 慎重[しんちょう]であらねばならない。
\\	昨日来れば、夏子に会えたでしょうに。	
\\	昨日[きのう] 来[く]れば、 夏子[なつこ]に 会[あ]えたでしょうに。
\\	老人が一人で暮らすのはさぞ寂しいだろうに。	
\\	老人[ろうじん]が 一人[ひとり]で 暮[く]らすのはさぞ 寂[さび]しいだろうに。
\\	あと一年ぐらい日本にいたら日本語がもっと上手になるでしょうに。	
\\	あと 一年[いちねん]ぐらい 日本[にほん]にいたら 日本語[にほんご]がもっと 上手[じょうず]になるでしょうに。
\\	お金がたくさんあればこんな家でも買えるでしょうに。	
\\	お 金[かね]がたくさんあればこんな 家[いえ]でも 買[か]えるでしょうに。
\\	僕にいい仕事があったら美智子さんと結婚出来ただろうに。	
\\	僕[ぼく]にいい 仕事[しごと]があったら 美智子[みちこ]さんと 結婚[けっこん] 出来[でき]ただろうに。
\\	飛行機で行けば簡単に行ける所でしょうに。	
\\	飛行機[ひこうき]で 行[い]けば 簡単[かんたん]に 行[い]ける 所[ところ]でしょうに。
\\	彼はもう少し慎重に金を使えば、貯金出来るだろうに。	
\\	彼[かれ]はもう 少[すこ]し 慎重[しんちょう]に 金[きん]を 使[つか]えば、 貯金[ちょきん] 出来[でき]るだろうに。
\\	駅の近くのアパートに住んだら便利でしょうに。	
\\	駅[えき]の 近[ちか]くのアパートに 住[す]んだら 便利[べんり]でしょうに。
\\	子供を二人も大学に行かせるのにはお金がかかるでしょうに。	
\\	子供[こども]を 二人[ふたり]も 大学[だいがく]に 行[い]かせるのにはお 金[かね]がかかるでしょうに。
\\	飛行機の切符をなくして、さぞ困ったことでしょうに。	
\\	飛行機[ひこうき]の 切符[きっぷ]をなくして、さぞ 困[こま]ったことでしょうに。
\\	僕にいい仕事があったら美智子さんと結婚出来たのに。	
\\	僕[ぼく]にいい 仕事[しごと]があったら 美智子[みちこ]さんと 結婚[けっこん] 出来[でき]たのに。
\\	新入社員の入社に当たって社長はホテルでパーティーを開いた。	
\\	新入[しんにゅう] 社員[しゃいん]の 入社[にゅうしゃ]に 当[あ]たって 社長[しゃちょう]はホテルでパーティーを 開[ひら]いた。
\\	和也は大学を卒業するに当たり、指導教官に今後の進路について相談した。	
\\	和也[かずや]は 大学[だいがく]を 卒業[そつぎょう]するに 当[あ]たり、 指導[しどう] 教官[きょうかん]に 今後[こんご]の 進路[しんろ]について 相談[そうだん]した。
\\	私の留学に当たり、父は自分の経験を話してくれた。	
\\	私[わたし]の 留学[りゅうがく]に 当[あ]たり、 父[ちち]は 自分[じぶん]の 経験[けいけん]を 話[はな]してくれた。
\\	日本の経済を研究するに当たって、国会図書館で資料集めをした。	
\\	日本[にほん]の 経済[けいざい]を 研究[けんきゅう]するに 当[あ]たって、 国会図書館[こっかいとしょかん]で 資料[しりょう] 集[あつ]めをした。
\\	総理はサミット出席に当たり、閣僚と会議をした。	
\\	総理[そうり]はサミット 出席[しゅっせき]に 当[あ]たり、 閣僚[かくりょう]と 会議[かいぎ]をした。
\\	小説家は時代小説を書くに当たり、その時代の歴史を詳しく調べた。	
\\	小説[しょうせつ] 家[か]は 時代[じだい] 小説[しょうせつ]を 書[か]くに 当[あ]たり、その 時代[じだい]の 歴史[れきし]を 詳[くわ]しく 調[しら]べた。
\\	私は就寝に当たって少量の洋酒を喫することを習慣にしている。	
\\	私[わたし]は 就寝[しゅうしん]に 当[あ]たって 少量[しょうりょう]の 洋酒[ようしゅ]を 喫[きっ]することを 習慣[しゅうかん]にしている。
\\	進入社員の入社に当たってパーティーに社長も出席した。	
\\	進入[しんにゅう] 社員[しゃいん]の 入社[にゅうしゃ]に 当[あ]たってパーティーに 社長[しゃちょう]も 出席[しゅっせき]した。
\\	私は日本にいた時日本語を勉強した。	
\\	私[わたし]は 日本[にほん]にいた 時[とき] 日本語[にほんご]を 勉強[べんきょう]した。
\\	田中さんは大学を卒業した時すぐ会社に入った。	
\\	田中[たなか]さんは 大学[だいがく]を 卒業[そつぎょう]した 時[とき]すぐ 会社[かいしゃ]に 入[はい]った。
\\	日本人はご飯を食べる時「いただきます」と言う。	
\\	日本人[にほんじん]はご 飯[はん]を 食[た]べる 時[とき]「いただきます」と 言[い]う。
\\	寝る時シャワーを浴びる。	
\\	寝[ね]る 時[とき]シャワーを 浴[あ]びる。
\\	伸子は寝る前にテレビを見る。	
\\	伸子[のぶこ]は 寝[ね]る 前[まえ]にテレビを 見[み]る。
\\	伸子は入院するに当たって家族といろいろ相談した。	
\\	伸子[のぶこ]は 入院[にゅういん]するに 当[あ]たって 家族[かぞく]といろいろ 相談[そうだん]した。
\\	大阪に転勤するに際して妻子は東京に残した。	
\\	大阪[おおさか]に 転勤[てんきん]するに 際[さい]して 妻子[さいし]は 東京[とうきょう]に 残[のこ]した。
\\	日本の近代化の研究の際に日本の国会図書館を利用した。	
\\	日本[にほん]の 近代[きんだい] 化[か]の 研究[けんきゅう]の 際[さい]に 日本[にほん]の 国会[こっかい] 図書館[としょかん]を 利用[りよう]した。
\\	今年の東京国際マラソンは大方の予想に反して無名の選手が優勝した。	
\\	今年[ことし]の 東京[とうきょう] 国際[こくさい]マラソンは 大方[おおかた]の 予想[よそう]に 反[はん]して 無名[むめい]の 選手[せんしゅ]が 優勝[ゆうしょう]した。
\\	我が社はテレビの売れ行きが伸びているのに反してオーディオ製品の売れ行きが下がっている。	
\\	我[わ]が 社[しゃ]はテレビの 売れ行[うれゆ]きが 伸[の]びているのに 反[はん]してオーディオ 製品[せいひん]の 売れ行[うれゆ]きが 下[さ]がっている。
\\	日本ではいい大学に入るのは難しいが卒業するのはやさしいと言われている。これに反して、アメリカではいい大学でも入学は比較的やさしいが卒業するには相当勉強しなければならない。	
\\	日本[にほん]ではいい 大学[だいがく]に 入[はい]るのは 難[むずか]しいが 卒業[そつぎょう]するのはやさしいと 言[い]われている。これに 反[はん]して、アメリカではいい 大学[だいがく]でも 入学[にゅうがく]は 比較的[ひかくてき]やさしいが 卒業[そつぎょう]するには 相当[そうとう] 勉強[べんきょう]しなければならない。
\\	妙子は意に反する結婚を押しつけられた。	
\\	妙子[たえこ]は 意[い]に 反[はん]する 結婚[けっこん]を 押[お]しつけられた。
\\	浩は両親の期待に反して高校を出てからコックになった。	
\\	浩[ひろし]は 両親[りょうしん]の 期待[きたい]に 反[はん]して 高校[こうこう]を 出[で]てからコックになった。
\\	彼は自分の意思に反して賄賂を受け取ってしまった。	
\\	彼[かれ]は 自分[じぶん]の 意思[いし]に 反[はん]して 賄賂[わいろ]を 受け取[うけと]ってしまった。
\\	この規則に反して従業員を働かせた場合は一か月の営業停止になる。	
\\	この 規則[きそく]に 反[はん]して 従業[じゅうぎょう] 員[いん]を 働[はたら]かせた 場合[ばあい]は 一か[いっか] 月[げつ]の 営業[えいぎょう] 停止[ていし]になる。
\\	奥村さんのうちはご主人が無口なのに反して奥さんが人一倍のおしゃべりだ。	
\\	奥村[おくむら]さんのうちはご 主人[しゅじん]が 無口[むくち]なのに 反[はん]して 奥[おく]さんが 人一倍[ひといちばい]のおしゃべりだ。
\\	日本での初等教育は生徒に知識を与えることに主眼が置かれている。これに反して、アメリカでは生徒の創造性や個性を伸ばすことに重点が置かれている。	
\\	日本[にほん]での 初等[しょとう] 教育[きょういく]は 生徒[せいと]に 知識[ちしき]を 与[あた]えることに 主眼[しゅがん]が 置[お]かれている。これに 反[はん]して、アメリカでは 生徒[せいと]の 創造[そうぞう] 性[せい]や 個性[こせい]を 伸[の]ばすことに 重点[じゅうてん]が 置[お]かれている。
\\	今回の選挙は一般の予想に反する結果に終わった。	
\\	今回[こんかい]の 選挙[せんきょ]は 一般[いっぱん]の 予想[よそう]に 反[はん]する 結果[けっか]に 終[お]わった。
\\	社が教育産業を重視しているのに対して
\\	社はレジャー産業に力を入れている。	
\\	社[しゃ]が 教育[きょういく] 産業[さんぎょう]を 重視[じゅうし]しているのに 対[たい]して 
\\	社[しゃ]はレジャー 産業[さんぎょう]に 力[ちから]を 入[い]れている。
\\	あの人の言葉は美辞麗句にほかならない。	
\\	あの 人[ひと]の 言葉[ことば]は 美辞麗句[びじれいく]にほかならない。
\\	日本語を勉強しているのは将来日本で仕事をしたいからにほかならない。	
\\	日本語[にほんご]を 勉強[べんきょう]しているのは 将来[しょうらい] 日本[にほん]で 仕事[しごと]をしたいからにほかならない。
\\	外国語学習はほかの国の人の考え方を学ぶことにほかならない。	
\\	外国[がいこく] 語[ご] 学習[がくしゅう]はほかの 国[くに]の 人[ひと]の 考え方[かんがえかた]を 学[まな]ぶことにほかならない。
\\	モーツァルトの音楽は天使の声にほかならない。	
\\	モーツァルトの 音楽[おんがく]は 天使[てんし]の 声[こえ]にほかならない。
\\	結婚は人生の墓場にほかならない。	
\\	結婚[けっこん]は 人生[じんせい]の 墓場[はかば]にほかならない。
\\	彼の言動は自己宣伝にほかならない。	
\\	彼[かれ]の 言動[げんどう]は 自己[じこ] 宣伝[せんでん]にほかならない。
\\	親が子供を厳しくしつけるのは子供を愛しているからにほかならない。	
\\	親[おや]が 子供[こども]を 厳[きび]しくしつけるのは 子供[こども]を 愛[あい]しているからにほかならない。
\\	彼女が僕に寄って来るのは僕を利用したいからにほかならない。	
\\	彼女[かのじょ]が 僕[ぼく]に 寄[よ]って 来[く]るのは 僕[ぼく]を 利用[りよう]したいからにほかならない。
\\	私の日本語がこれほど上達したのは大学の時の日本語の先生のおかげにほかならない。	
\\	私[わたし]の 日本語[にほんご]がこれほど 上達[じょうたつ]したのは 大学[だいがく]の 時[とき]の 日本語[にほんご]の 先生[せんせい]のおかげにほかならない。
\\	この大学の学生の質がいいのは選抜が厳しいからにほかならない。	
\\	この 大学[だいがく]の 学生[がくせい]の 質[しつ]がいいのは 選抜[せんばつ]が 厳[きび]しいからにほかならない。
\\	ほかならない君のためだから、やってあげよう。	
\\	ほかならない 君[きみ]のためだから、やってあげよう。
\\	核兵器の使用は人類の破滅を招くことにほかならない。	
\\	核兵器[かくへいき]の 使用[しよう]は 人類[じんるい]の 破滅[はめつ]を 招[まね]くことにほかならない。
\\	それは噂に過ぎない。	
\\	それは 噂[うわさ]に 過[す]ぎない。
\\	吉田は平凡な会社員に過ぎない。	
\\	吉田[よしだ]は 平凡[へいぼん]な 会社[かいしゃ] 員[いん]に 過[す]ぎない。
\\	僕のアルバイト料は五百円に過ぎない。	
\\	僕[ぼく]のアルバイト 料[りょう]は 五百[ごひゃく] 円[えん]に 過[す]ぎない。
\\	私の睡眠時間は四、五時間に過ぎない。	
\\	私[わたし]の 睡眠[すいみん] 時間[じかん]は四[し]、五時間[ごじかん]に 過[す]ぎない。
\\	見合い結婚は日本に限らずほかの国でも行われている。	
\\	見合[みあ]い 結婚[けっこん]は 日本[にほん]に 限[かぎ]らずほかの 国[くに]でも 行[おこな]われている。
\\	音楽はクラシックに限らず何でも聞きます。	
\\	音楽[おんがく]はクラシックに 限[かぎ]らず 何[なに]でも 聞[き]きます。
\\	このバーは男性だけに限らず女性の間にも人気がある。	
\\	このバーは 男性[だんせい]だけに 限[かぎ]らず 女性[じょせい]の 間[あいだ]にも 人気[にんき]がある。
\\	果物は何に限らず好きです。	
\\	果物[くだもの]は 何[なに]に 限[かぎ]らず 好[す]きです。
\\	何事に限らず仕事は誠意をもって行うことが大切だ。	
\\	何事[なにごと]に 限[かぎ]らず 仕事[しごと]は 誠意[せいい]をもって 行[おこな]うことが 大切[たいせつ]だ。
\\	見合い結婚は日本だけではなくほかの国でも言われている。	
\\	見合[みあ]い 結婚[けっこん]は 日本[にほん]だけではなくほかの 国[くに]でも 言[い]われている。
\\	音楽はクラシックだけでなく何でも聞きます。	
\\	音楽[おんがく]はクラシックだけでなく 何[なに]でも 聞[き]きます。
\\	このバーは男性だけではなく女性の間にも人気がある。	
\\	このバーは 男性[だんせい]だけではなく 女性[じょせい]の 間[あいだ]にも 人気[にんき]がある。
\\	うちの子供に限ってそんなひどいことはしませんよ。	
\\	うちの 子供[こども]に 限[かぎ]ってそんなひどいことはしませんよ。
\\	中野先生はたいてい研究室にいらっしゃるのに、今日に限っていらっしゃらなかった。	
\\	中野[なかの] 先生[せんせい]はたいてい 研究[けんきゅう] 室[しつ]にいらっしゃるのに、 今日[きょう]に 限[かぎ]っていらっしゃらなかった。
\\	傘を持ってこない日に限って雨が降るんですよ。	
\\	傘[かさ]を 持[も]ってこない 日[ひ]に 限[かぎ]って 雨[あめ]が 降[ふ]るんですよ。
\\	急ぐ時に限って、バスがなかなか来ない。	
\\	急[いそ]ぐ 時[とき]に 限[かぎ]って、バスがなかなか 来[こ]ない。
\\	嫌なことがある時に限って、嫌なことが重なる。	
\\	嫌[いや]なことがある 時[とき]に 限[かぎ]って、 嫌[いや]なことが 重[かさ]なる。
\\	この小説家の書いたものに限って、つまらないものはない。	
\\	この 小説[しょうせつ] 家[か]の 書[か]いたものに 限[かぎ]って、つまらないものはない。
\\	消費税に関して与野党の意見が対立した。	
\\	消費[しょうひ] 税[ぜい]に 関[かん]して 与野党[よやとう]の 意見[いけん]が 対立[たいりつ]した。
\\	人間の言語習得に関してはまだまだ不明のことが多い。	
\\	人間[にんげん]の 言語[げんご] 習得[しゅうとく]に 関[かん]してはまだまだ 不明[ふめい]のことが 多[おお]い。
\\	先週、東京で超伝導に関する学会が開かれた。	
\\	先週[せんしゅう]、 東京[とうきょう]で 超[ちょう] 伝導[でんどう]に 関[かん]する 学会[がっかい]が 開[ひら]かれた。
\\	その学会で吉岡氏の発表に関して多くの批判がなされた。	
\\	その 学会[がっかい]で 吉岡[よしおか] 氏[し]の 発表[はっぴょう]に 関[かん]して 多[おお]くの 批判[ひはん]がなされた。
\\	その件に関しましては、まだ発表できる段階ではありません。	
\\	その 件[けん]に 関[かん]しましては、まだ 発表[はっぴょう]できる 段階[だんかい]ではありません。
\\	最近老人問題に関する記事が目に付く。	
\\	最近[さいきん] 老人[ろうじん] 問題[もんだい]に 関[かん]する 記事[きじ]が 目[め]に 付[つ]く。
\\	大気汚染に関する報告書が委員会に提出された。	
\\	大気[たいき] 汚染[おせん]に 関[かん]する 報告[ほうこく] 書[しょ]が 委員[いいん] 会[かい]に 提出[ていしゅつ]された。
\\	病気の山田先生に代って、鈴木先生が教えて下さった。	
\\	病気[びょうき]の 山田[やまだ] 先生[せんせい]に 代[かわ]って、 鈴木[すずき] 先生[せんせい]が 教[おし]えて 下[くだ]さった。
\\	戦後ガラスに代わってプラスチックが出てきた。	
\\	戦後[せんご]ガラスに 代[か]わってプラスチックが 出[で]てきた。
\\	レコードに代わって
\\	が出て、音質がとてもよくなった。	
\\	レコードに 代[か]わって 
\\	が 出[で]て、 音質[おんしつ]がとてもよくなった。
\\	私が父に代わって、空港までお客さんを迎えに行った。	
\\	私[わたし]が 父[ちち]に 代[か]わって、 空港[くうこう]までお 客[きゃく]さんを 迎[むか]えに 行[い]った。
\\	御出席の皆様に代わりました、一言ご挨拶を述べさせていただきます。	
\\	御[ご] 出席[しゅっせき]の 皆様[みなさま]に 代[か]わりました、 一言[ひとこと]ご 挨拶[あいさつ]を 述[の]べさせていただきます。
\\	申し訳ございませんが、社長は今日都合が悪いので、私が社長に代わってまいりました。	
\\	申し訳[もうしわけ]ございませんが、 社長[しゃちょう]は 今日[きょう] 都合[つごう]が 悪[わる]いので、 私[わたし]が 社長[しゃちょう]に 代[か]わってまいりました。
\\	エコノミークラスではガラスの食器の代わりにプラスチックの食器が出る。	
\\	エコノミークラスではガラスの 食器[しょっき]の 代[か]わりにプラスチックの 食器[しょっき]が 出[で]る。
\\	欧米に比べて日本の住宅事情は極めて貧困だ。	
\\	欧米[おうべい]に 比[くら]べて 日本[にほん]の 住宅[じゅうたく] 事情[じじょう]は 極[きわ]めて 貧困[ひんこん]だ。
\\	日本に来る前に比べると今は大分日本語でものが言えるようになった。	
\\	日本[にほん]に 来[く]る 前[まえ]に 比[くら]べると 今[いま]は 大分[おおいた] 日本語[にほんご]でものが 言[い]えるようになった。
\\	今年は去年に比べて雨の日が多いようだ。	
\\	今年[ことし]は 去年[きょねん]に 比[くら]べて 雨[あめ]の 日[ひ]が 多[おお]いようだ。
\\	十年前の暮らしに比べると今の暮らしは天国です。	
\\	十年[じゅうねん] 前[まえ]の 暮[く]らしに 比[くら]べると 今[いま]の 暮[く]らしは 天国[てんごく]です。
\\	この新しい(コンピュータ)モデルは、古いモデルと比べて演算が三倍速くなっている。	
\\	この 新[あたら]しい(コンピュータ)モデルは、 古[ふる]いモデルと 比[くら]べて 演算[えんざん]が 三倍[さんばい] 速[はや]くなっている。
\\	懸命な努力にもかかわらず、健一は大学入試に失敗した。	
\\	懸命[けんめい]な 努力[どりょく]にもかかわらず、 健一[けんいち]は 大学[だいがく] 入試[にゅうし]に 失敗[しっぱい]した。
\\	あの人はよく運動をするのにもかかわらず、太っている。	
\\	あの 人[ひと]はよく 運動[うんどう]をするのにもかかわらず、 太[ふと]っている。
\\	試験が難しかったにもかかわらず、よく出来た。	
\\	試験[しけん]が 難[むずか]しかったにもかかわらず、よく 出来[でき]た。
\\	冬山は危険なのにもかかわらず、一郎は山へ出かけた。	
\\	冬山[ふゆやま]は 危険[きけん]なのにもかかわらず、 一郎[いちろう]は 山[やま]へ 出[で]かけた。
\\	激しい雨だったにもかかわらず、サッカーの試合は続いた。	
\\	激[はげ]しい 雨[あめ]だったにもかかわらず、サッカーの 試合[しあい]は 続[つづ]いた。
\\	安い値段にもかかわらず、その家はまだ売れていない。	
\\	安[やす]い 値段[ねだん]にもかかわらず、その 家[いえ]はまだ 売[う]れていない。
\\	宏は美香を愛していたのにもかかわらず、結婚しなかった。	
\\	宏[ひろし]は 美香[みか]を 愛[あい]していたのにもかかわらず、 結婚[けっこん]しなかった。
\\	昨日は徹夜して勉強してにもかかわらず、試験は出来なかった。	
\\	昨日[きのう]は 徹夜[てつや]して 勉強[べんきょう]してにもかかわらず、 試験[しけん]は 出来[でき]なかった。
\\	キャロルは日本に三年も住んでいたのにもかかわらず、日本語は大変下手だ。	
\\	キャロルは 日本[にほん]に三 年[ねん]も 住[す]んでいたのにもかかわらず、 日本語[にほんご]は 大変[たいへん] 下手[へた]だ。
\\	日本の経済力は強くなっているにもかかわらず、日本人はそれを実感出来ない。	
\\	日本[にほん]の 経済[けいざい] 力[りょく]は 強[つよ]くなっているにもかかわらず、 日本人[にほんじん]はそれを 実感[じっかん] 出来[でき]ない。
\\	私のアパートは広いにもかかわらず、家賃が安い。	
\\	私[わたし]のアパートは 広[ひろ]いにもかかわらず、 家賃[やちん]が 安[やす]い。
\\	旅行が好きなのにもかかわらず、どこに行く暇もない。	
\\	旅行[りょこう]が 好[す]きなのにもかかわらず、どこに 行[い]く 暇[ひま]もない。
\\	健康なのにもかかわらず、働かない。	
\\	健康[けんこう]なのにもかかわらず、 働[はたら]かない。
\\	事実に基づいてお話します。	
\\	事実[じじつ]に 基[もと]づいてお 話[はなし]します。
\\	これは五百年前の史料に基づく研究だ。	
\\	これは 五百[ごひゃく] 年[ねん] 前[まえ]の 史料[しりょう]に 基[もと]づく 研究[けんきゅう]だ。
\\	この製品はアメリカ製だがすべて
\\	に基づいて設計されている。	
\\	この 製品[せいひん]はアメリカ 製[せい]だがすべて 
\\	に 基[もと]づいて 設計[せっけい]されている。
\\	このビルはある有名な建築家の設計に基づいて建てられた。	
\\	このビルはある 有名[ゆうめい]な 建築[けんちく] 家[か]の 設計[せっけい]に 基[もと]づいて 建[た]てられた。
\\	この報告は去年の行われたアンケート調査に基づいて書かれたものである。	
\\	この 報告[ほうこく]は 去年[きょねん]の 行[おこな]われたアンケート 調査[ちょうさ]に 基[もと]づいて 書[か]かれたものである。
\\	野村氏の講演は氏の二十年間の教育経験に基づく話で、極めて示唆に富むものであった。	
\\	野村[のむら] 氏[し]の 講演[こうえん]は 氏[し]の 二十年間[にじゅうねんかん]の 教育[きょういく] 経験[けいけん]に 基[もと]づく 話[はなし]で、 極[きわ]めて 示唆[しさ]に 富[と]むものであった。
\\	これは一つの仮説に基づいた議論です。	
\\	これは 一[ひと]つの 仮説[かせつ]に 基[もと]づいた 議論[ぎろん]です。
\\	夏になると、摂氏三十度を越す日が多いです。	
\\	夏[なつ]になると、 摂氏[せっし] 三十度[さんじゅうど]を 越[こ]す 日[ひ]が 多[おお]いです。
\\	父は食べ物のことになると、とてもやかましかった。	
\\	父[ちち]は 食べ物[たべもの]のことになると、とてもやかましかった。
\\	四月半ばになると、この辺は桜がきれいに咲きます。	
\\	四月[しがつ] 半[なか]ばになると、この 辺[へん]は 桜[さくら]がきれいに 咲[さ]きます。
\\	ニューイングランドと言われるアメリカの東北地方は、秋になると、紅葉がとてもきれいです。	
\\	ニューイングランドと 言[い]われるアメリカの 東北[とうほく] 地方[ちほう]は、 秋[あき]になると、 紅葉[こうよう]がとてもきれいです。
\\	私は夜十一時になると、頭が働かなくなる。	
\\	私[わたし]は 夜[よる] 十一時[じゅういちじ]になると、 頭[あたま]が 働[はたら]かなくなる。
\\	子供の頃、夏になると、両親は僕を海に連れて行ってくれた。	
\\	子供[こども]の 頃[ころ]、 夏[なつ]になると、 両親[りょうしん]は 僕[ぼく]を 海[うみ]に 連[つ]れて 行[い]ってくれた。
\\	大学生の時は、休みになると、よく旅行をした。	
\\	大学生[だいがくせい]の 時[とき]は、 休[やす]みになると、よく 旅行[りょこう]をした。
\\	彼はフットボールの話になると夢中だ。	
\\	彼[かれ]はフットボールの 話[はなし]になると 夢中[むちゅう]だ。
\\	社長は娘のことになると、人が変わったように甘くなる。	
\\	社長[しゃちょう]は 娘[むすめ]のことになると、 人[ひと]が 変[か]わったように 甘[あま]くなる。
\\	雪になると、この空港はよく閉鎖になる。	
\\	雪[ゆき]になると、この 空港[くうこう]はよく 閉鎖[へいさ]になる。
\\	私が女優になると、両親は嫌がるかもしれません。	
\\	私[わたし]が 女優[じょゆう]になると、 両親[りょうしん]は 嫌[いや]がるかもしれません。
\\	季節が春になると人の心は自然にはずんでくる。	
\\	季節[きせつ]が 春[はる]になると 人[ひと]の 心[こころ]は 自然[しぜん]にはずんでくる。
\\	来年の総会はシカゴのヒルトン・ホテルにおいて行われる。	
\\	来年[らいねん]の 総会[そうかい]はシカゴのヒルトン・ホテルにおいて 行[おこな]われる。
\\	コンピュータは近い将来においてほとんどの家庭に行き渡るだろう。	
\\	コンピュータは 近[ちか]い 将来[しょうらい]においてほとんどの 家庭[かてい]に 行き渡[いきわた]るだろう。
\\	先月の東京における環境保護国際会議には全世界から大勢の学者が参加した。	
\\	先月[せんげつ]の 東京[とうきょう]における 環境[かんきょう] 保護[ほご] 国際[こくさい] 会議[かいぎ]には 全[ぜん] 世界[せかい]から 大勢[たいせい]の 学者[がくしゃ]が 参加[さんか]した。
\\	1992年のオリンピックはバルセロナにおいて開催された。	
\\	年[ねん]のオリンピックはバルセロナにおいて 開催[かいさい]された。
\\	過去においてはこのようなことはあまり問題にならなかった。	
\\	過去[かこ]においてはこのようなことはあまり 問題[もんだい]にならなかった。
\\	この作文は文法においてはあまり問題はない。	
\\	この 作文[さくぶん]は 文法[ぶんぽう]においてはあまり 問題[もんだい]はない。
\\	本校における男女学生の比率は二対一である。	
\\	本校[ほんこう]における 男女[だんじょ] 学生[がくせい]の 比率[ひりつ]は 二対[にたい]一[いち]である。
\\	木村博士は遺伝学における権威者として知られている。	
\\	木村[きむら] 博士[はかせ]は 遺伝[いでん] 学[がく]における 権威[けんい] 者[しゃ]として 知[し]られている。
\\	過去五年間における彼の業績は実にすばらしいものであった。	
\\	過去[かこ] 五年間[ごねんかん]における 彼[かれ]の 業績[ぎょうせき]は 実[じつ]にすばらしいものであった。
\\	十月十二日における彼のアリバイは成立していない。	
\\	十月[じゅうがつ] 十二日[じゅうににち]における 彼[かれ]のアリバイは 成立[せいりつ]していない。
\\	私は昨日ダウンタウンの映画館で映画を見た。	
\\	私[わたし]は 昨日[きのう]ダウンタウンの 映画[えいが] 館[かん]で 映画[えいが]を 見[み]た。
\\	山田さんのうちでの誕生パーティーにはたくさんの友達が集まった。	
\\	山田[やまだ]さんのうちでの 誕生[たんじょう]パーティーにはたくさんの 友達[ともだち]が 集[あつ]まった。
\\	その会議は1991年に開かれた。	
\\	その 会議[かいぎ]は1991 年[ねん]に 開[ひら]かれた。
\\	この表は1991年におけるアジア諸国の
\\	を示している。	
\\	この 表[ひょう]は1991 年[ねん]におけるアジア 諸国[しょこく]の 
\\	を 示[しめ]している。
\\	日本の経済力が強くなるに従って日本語学習者が増えてきた。	
\\	日本[にほん]の 経済[けいざい] 力[りょく]が 強[つよ]くなるに 従[したが]って 日本語[にほんご] 学習[がくしゅう] 者[しゃ]が 増[ふ]えてきた。
\\	契約に従い雇用期間を三年とする。	
\\	契約[けいやく]に 従[したが]い 雇用[こよう] 期間[きかん]を 三年[さんねん]とする。
\\	収入が増えるに従って、支出も増える。	
\\	収入[しゅうにゅう]が 増[ふ]えるに 従[したが]って、 支出[ししゅつ]も 増[ふ]える。
\\	年を取るに従い、体力が衰える。	
\\	年[とし]を 取[と]るに 従[したが]い、 体力[たいりょく]が 衰[おとろ]える。
\\	文明が進むに従い、人間のストレスが多くなる。	
\\	文明[ぶんめい]が 進[すす]むに 従[したが]い、 人間[にんげん]のストレスが 多[おお]くなる。
\\	日がたつに従って、父の病気は回復してきた。	
\\	日[ひ]がたつに 従[したが]って、 父[ちち]の 病気[びょうき]は 回復[かいふく]してきた。
\\	太陽が沈んで行くに従って、温度がどんどん下がった。	
\\	太陽[たいよう]が 沈[しず]んで 行[い]くに 従[したが]って、 温度[おんど]がどんどん 下[さ]がった。
\\	約束に従い、借金を一か月後に返した。	
\\	約束[やくそく]に 従[したが]い、 借金[しゃっきん]を 一か月[いっかげつ] 後[ご]に 返[かえ]した。
\\	社長の命令に従って、彼はただちにマニラに飛んだ。	
\\	社長[しゃちょう]の 命令[めいれい]に 従[したが]って、 彼[かれ]はただちにマニラに 飛[と]んだ。
\\	窓を開けたら、涼しい風が入ってきた。	
\\	窓[まど]を 開[あ]けたら、 涼[すず]しい 風[かぜ]が 入[はい]ってきた。
\\	たばこをやめたら、体の調子がよくなった。	
\\	たばこをやめたら、 体[からだ]の 調子[ちょうし]がよくなった。
\\	これは私の私見に過ぎない。	
\\	これは 私[わたし]の 私見[しけん]に 過[す]ぎない。
\\	今期の売り上げの伸びは三パーセントに過ぎない。	
\\	今期[こんき]の 売り上[うりあ]げの 伸[の]びは 三[さん]パーセントに 過[す]ぎない。
\\	彼は私のボーイフレンドの一人に過ぎない。	
\\	彼[かれ]は 私[わたし]のボーイフレンドの 一人[ひとり]に 過[す]ぎない。
\\	これは数ある中のほんの一例に過ぎない。	
\\	これは 数[かず]ある 中[なか]のほんの 一例[いちれい]に 過[す]ぎない。
\\	今言ったことは私の希望に過ぎませんので、あまり深刻に考えないで下さい。	
\\	今[いま] 言[い]ったことは 私[わたし]の 希望[きぼう]に 過[す]ぎませんので、あまり 深刻[しんこく]に 考[かんが]えないで 下[くだ]さい。
\\	私の収入はアルバイトを入れても年二万ドルに過ぎない。	
\\	私[わたし]の 収入[しゅうにゅう]はアルバイトを 入[い]れても 年[とし] 二万[にまん]ドルに 過[す]ぎない。
\\	私を支持してくれる人は身内を入れても五十人程度に過ぎない。	
\\	私[わたし]を 支持[しじ]してくれる 人[ひと]は 身内[みうち]を 入[い]れても 五十人[ごじゅうにん] 程度[ていど]に 過[す]ぎない。
\\	あの子はまだ十五に過ぎないが、なかなかしっかりしている。	
\\	あの 子[こ]はまだ十五に 過[す]ぎないが、なかなかしっかりしている。
\\	彼はただ人の意見を受け売りしているだけに過ぎない。	
\\	彼[かれ]はただ 人[ひと]の 意見[いけん]を 受け売[うけう]りしているだけに 過[す]ぎない。
\\	彼の演説は原稿を読み上げたに過ぎない。	
\\	彼[かれ]の 演説[えんぜつ]は 原稿[げんこう]を 読み上[よみあ]げたに 過[す]ぎない。
\\	私と山野氏は初対面の時、簡単な挨拶を交わしただけに過ぎない。	
\\	私[わたし]と 山野[やまの] 氏[し]は 初対面[しょたいめん]の 時[とき]、 簡単[かんたん]な 挨拶[あいさつ]を 交[か]わしただけに 過[す]ぎない。
\\	あの男はただ言われたことをしているだけだ。	
\\	あの 男[おとこ]はただ 言[い]われたことをしているだけだ。
\\	これは私のただの私見だ。	
\\	これは 私[わたし]のただの 私見[しけん]だ。
\\	彼は高校の先生に過ぎない。	
\\	彼[かれ]は 高校[こうこう]の 先生[せんせい]に 過[す]ぎない。
\\	彼はただの高校の先生だ。	
\\	彼[かれ]はただの 高校[こうこう]の 先生[せんせい]だ。
\\	今期の売り上げの伸びはただの三パーセントだ。	
\\	今期[こんき]の 売り上[うりあ]げの 伸[の]びはただの三[さん]パーセントだ。
\\	これまで日本は外国に対して閉鎖的な政策を取ってきた。	
\\	これまで 日本[にほん]は 外国[がいこく]に 対[たい]して 閉鎖[へいさ] 的[てき]な 政策[せいさく]を 取[と]ってきた。
\\	日本の大学は入学するのが難しいのに対して、アメリカの大学は卒業するのが難しい。	
\\	日本[にほん]の 大学[だいがく]は 入学[にゅうがく]するのが 難[むずか]しいのに 対[たい]して、アメリカの 大学[だいがく]は 卒業[そつぎょう]するのが 難[むずか]しい。
\\	アメリカでは離婚に対する考え方が大分代わってきた。	
\\	アメリカでは 離婚[りこん]に 対[たい]する 考え方[かんがえかた]が 大分[だいぶ] 代[か]わってきた。
\\	鈴木さんは誰に対しても丁寧だ。	
\\	鈴木[すずき]さんは 誰[だれ]に 対[たい]しても 丁寧[ていねい]だ。
\\	日本は外国に対して市場をもっと開放すべきだ。	
\\	日本[にほん]は 外国[がいこく]に 対[たい]して 市場[しじょう]をもっと 開放[かいほう]すべきだ。
\\	私は政治に対して強い関心がある。	
\\	私[わたし]は 政治[せいじ]に 対[たい]して 強[つよ]い 関心[かんしん]がある。
\\	手数料は一万円に対して五百円です。	
\\	手数料[てすうりょう]は 一万[いちまん] 円[えん]に 対[たい]して 五百[ごひゃく] 円[えん]です。
\\	市民は増税に対して強く反対している。	
\\	市民[しみん]は 増税[ぞうぜい]に 対[たい]して 強[つよ]く 反対[はんたい]している。
\\	私は高校生の時から、日本の文化に対して興味があった。	
\\	私[わたし]は 高校生[こうこうせい]の 時[とき]から、 日本[にほん]の 文化[ぶんか]に 対[たい]して 興味[きょうみ]があった。
\\	去年は自動車事故が少なかったのに対して、今年は大変多かった。	
\\	去年[きょねん]は 自動車[じどうしゃ] 事故[じこ]が 少[すく]なかったのに 対[たい]して、 今年[ことし]は 大変[たいへん] 多[おお]かった。
\\	日本人は褒められた時、よく「とんでもない」と言う、これに対して、アメリカ人はよく「ありがとう」という。	
\\	日本人[にほんじん]は 褒[ほ]められた 時[とき]、よく「とんでもない」と 言[い]う、これに 対[たい]して、アメリカ 人[じん]はよく「ありがとう」という。
\\	女性の地位は日本ではまだ低いのに対して、欧米では比較的高い。	
\\	女性[じょせい]の 地位[ちい]は 日本[にほん]ではまだ 低[ひく]いのに 対[たい]して、 欧米[おうべい]では 比較的[ひかくてき] 高[たか]い。
\\	東ヨーロッパの国々が民主化に向かっているのに対して、中国大陸と北朝鮮は民主化がまだ遅れているようだ。	
\\	東[ひがし]ヨーロッパの 国々[くにぐに]が 民主[みんしゅ] 化[か]に 向[む]かっているのに 対[たい]して、 中国[ちゅうごく] 大陸[たいりく]と 北朝鮮[きたちょうせん]は 民主[みんしゅ] 化[か]がまだ 遅[おく]れているようだ。
\\	ロシアに対してのアメリカ人の見方は最近友好的になった。	
\\	ロシアに 対[たい]してのアメリカ 人[じん]の 見方[みかた]は 最近[さいきん] 友好[ゆうこう] 的[てき]になった。
\\	部長は岡田に対して有利な発言をした。	
\\	部長[ぶちょう]は 岡田[おかだ]に 対[たい]して 有利[ゆうり]な 発言[はつげん]をした。
\\	その経済学者の説に対する反論はなかった。	
\\	その 経済[けいざい] 学者[がくしゃ]の 説[せつ]に 対[たい]する 反論[はんろん]はなかった。
\\	彼は女性に対してとても親切だ。	
\\	彼[かれ]は 女性[じょせい]に 対[たい]してとても 親切[しんせつ]だ。
\\	淑子が好きな男性は豪快な男性なのに対して、由美子が好きな男性は優しい男性だ。	
\\	淑子[よしこ]が 好[す]きな 男性[だんせい]は 豪快[ごうかい]な 男性[だんせい]なのに 対[たい]して、 由美子[ゆみこ]が 好[す]きな 男性[だんせい]は 優[やさ]しい 男性[だんせい]だ。
\\	これは我々にとって無視出来ない問題だ。	
\\	これは 我々[われわれ]にとって 無視[むし] 出来[でき]ない 問題[もんだい]だ。
\\	この大学で教えられることは私にとって極めて名誉なことです。	
\\	この 大学[だいがく]で 教[おし]えられることは 私[わたし]にとって 極[きわ]めて 名誉[めいよ]なことです。
\\	町の人々にとってこの工場の閉鎖は死活問題だ。	
\\	町[まち]の 人々[ひとびと]にとってこの 工場[こうじょう]の 閉鎖[へいさ]は 死活[しかつ] 問題[もんだい]だ。
\\	今日は私達にとって忘れられない日になるでしょう。	
\\	今日[きょう]は 私[わたし] 達[たち]にとって 忘[わす]れられない 日[ひ]になるでしょう。
\\	今度のアメリカの輸入政策は日本企業にとって大きな打撃になる。	
\\	今度[こんど]のアメリカの 輸入[ゆにゅう] 政策[せいさく]は 日本[にほん] 企業[きぎょう]にとって 大[おお]きな 打撃[だげき]になる。
\\	今回の合併は我が社にとってあまり益はない。	
\\	今回[こんかい]の 合併[がっぺい]は 我[わ]が 社[しゃ]にとってあまり 益[えき]はない。
\\	ここは君にとって一番安全な場所だ。	
\\	ここは 君[きみ]にとって 一番[いちばん] 安全[あんぜん]な 場所[ばしょ]だ。
\\	今村さんは私にとって遠い親戚に当たる。	
\\	今村[いまむら]さんは 私[わたし]にとって 遠[とお]い 親戚[しんせき]に 当[あ]たる。
\\	この仕事は私には出来ない。	
\\	この 仕事[しごと]は 私[わたし]には 出来[でき]ない。
\\	あの字は私には見えない。	
\\	あの 字[じ]は 私[わたし]には 見[み]えない。
\\	太郎にとっての問題は、彼が全然英語が分からないことだ。	
\\	太郎[たろう]にとっての 問題[もんだい]は、 彼[かれ]が 全然[ぜんぜん] 英語[えいご]が 分[わ]からないことだ。
\\	私は日本の政治制度について研究しています。	
\\	私[わたし]は 日本[にほん]の 政治[せいじ] 制度[せいど]について 研究[けんきゅう]しています。
\\	スミスさんは日本へ留学することについていろいろ私に尋ねた。	
\\	スミスさんは 日本[にほん]へ 留学[りゅうがく]することについていろいろ 私[わたし]に 尋[たず]ねた。
\\	御家族について話して下さい。	
\\	御[ご] 家族[かぞく]について 話[はな]して 下[くだ]さい。
\\	何について調べているんですか。	
\\	何[なに]について 調[しら]べているんですか。
\\	毎日の生活について聞いてもいいですか。	
\\	毎日[まいにち]の 生活[せいかつ]について 聞[き]いてもいいですか。
\\	日本に来る前に、日本についてどんなイメージを持っていらっしゃいましたか。	
\\	日本[にほん]に 来[く]る 前[まえ]に、 日本[にほん]についてどんなイメージを 持[も]っていらっしゃいましたか。
\\	先生は自分の国について作文を書くようにおっしゃいました。	
\\	先生[せんせい]は 自分[じぶん]の 国[くに]について 作文[さくぶん]を 書[か]くようにおっしゃいました。
\\	私はアメリカ人と結婚することについて両親と相談した。	
\\	私[わたし]はアメリカ 人[じん]と 結婚[けっこん]することについて 両親[りょうしん]と 相談[そうだん]した。
\\	日本人の宗教についての考え方は西洋人のとはかなり違う。	
\\	日本人[にほんじん]の 宗教[しゅうきょう]についての 考え方[かんがえかた]は 西洋[せいよう] 人[じん]のとはかなり 違[ちが]う。
\\	課長は私が会議に出なかったことについて何か言っていましたか。	
\\	課長[かちょう]は 私[わたし]が 会議[かいぎ]に 出[で]なかったことについて 何[なに]か 言[い]っていましたか。
\\	新聞社が外国人雇用問題についての意見を求めてきた。	
\\	新聞[しんぶん] 社[しゃ]が 外国[がいこく] 人[じん] 雇用[こよう] 問題[もんだい]についての 意見[いけん]を 求[もと]めてきた。
\\	日本が軍隊を平和維持のために海外に派遣することについての是非が盛んに討議された。	
\\	日本[にほん]が 軍隊[ぐんたい]を 平和[へいわ] 維持[いじ]のために 海外[かいがい]に 派遣[はけん]することについての 是非[ぜひ]が 盛[さか]んに 討議[とうぎ]された。
\\	ー田中さんを知っていますか。 ーええ、昔から。	
\\	ー 田中[たなか]さんを 知[し]っていますか。 ーええ、 昔[むかし]から。
\\	ー田中さんについて知っていますか。 ーええ、本で読んで、知っています。	
\\	ー 田中[たなか]さんについて 知[し]っていますか。 ーええ、 本[ほん]で 読[よ]んで、 知[し]っています。
\\	ー田中さんのことを知っていますか。 ー何かあったんですか。	
\\	ー 田中[たなか]さんのことを 知[し]っていますか。 ー 何[なに]かあったんですか。
\\	この仕事は一時間につき六ドルもらえる。	
\\	この 仕事[しごと]は 一時間[いちじかん]につき六[ろく]ドルもらえる。
\\	一冊につき百五十円の送料が要ります。	
\\	一冊[いっさつ]につき 百五十円[ひゃくごじゅうえん]の 送料[そうりょう]が 要[い]ります。
\\	五千円につき四十円の手数料をいただきます。	
\\	五千[ごせん] 円[えん]につき四十[よんじゅう] 円[えん]の 手数料[てすうりょう]をいただきます。
\\	米一カップにつき、しょう油大さじ二杯の割合にします。	
\\	米[こめ]一[いち]カップにつき、しょう 油[ゆ] 大[おお]さじ 二杯[にはい]の 割合[わりあい]にします。
\\	切符は一人につき三枚までにして下さい。	
\\	切符[きっぷ]は 一人[ひとり]につき 三枚[さんまい]までにして 下[くだ]さい。
\\	間違い一つにつき一点減点します。	
\\	間違[まちが]い 一[ひと]つにつき 一点[いってん] 減点[げんてん]します。
\\	この仕事は三時間につき十ドルもらえる。	
\\	この 仕事[しごと]は 三時間[さんじかん]につき 十[じゅう]ドルもらえる。
\\	日本語が上達するにつれて、日本人の友達が増えた。	
\\	日本語[にほんご]が 上達[じょうたつ]するにつれて、 日本人[にほんじん]の 友達[ともだち]が 増[ふ]えた。
\\	時代の変化につれて、文化も変わっていく。	
\\	時代[じだい]の 変化[へんか]につれて、 文化[ぶんか]も 変[か]わっていく。
\\	病気が治ってくるにつれて、食欲が出てきた。	
\\	病気[びょうき]が 治[なお]ってくるにつれて、 食欲[しょくよく]が 出[で]てきた。
\\	年を取るにつれて、体力がなくなる。	
\\	年[とし]を 取[と]るにつれて、 体力[たいりょく]がなくなる。
\\	日本の生活が長くなるにつれて、日本のよさ、悪さがよく分かってきた。	
\\	日本[にほん]の 生活[せいかつ]が 長[なが]くなるにつれて、 日本[にほん]のよさ、 悪[わる]さがよく 分[わ]かってきた。
\\	日本の経済力が伸びるにつれて、日本語の学生が増えてきた。	
\\	日本[にほん]の 経済[けいざい] 力[りょく]が 伸[の]びるにつれて、 日本語[にほんご]の 学生[がくせい]が 増[ふ]えてきた。
\\	秋が深くなるにつれて、紅葉がきれいになってきた。	
\\	秋[あき]が 深[ふか]くなるにつれて、 紅葉[こうよう]がきれいになってきた。
\\	子供は成長するにつれて、親から離れていく。	
\\	子供[こども]は 成長[せいちょう]するにつれて、 親[おや]から 離[はな]れていく。
\\	季節の変化につれて、風景も変わる。	
\\	季節[きせつ]の 変化[へんか]につれて、 風景[ふうけい]も 変[か]わる。
\\	産業の発展につれて、公害も増えていく。	
\\	産業[さんぎょう]の 発展[はってん]につれて、 公害[こうがい]も 増[ふ]えていく。
\\	日本語が上達すればするほど、日本人の友達が増えた。	
\\	日本語[にほんご]が 上達[じょうたつ]すればするほど、 日本人[にほんじん]の 友達[ともだち]が 増[ふ]えた。
\\	年を取れば取るほど、体力がなくなる。	
\\	年[とし]を 取[と]れば 取[と]るほど、 体力[たいりょく]がなくなる。
\\	日本語を教えれば教えるほど、教えることの難しさが分かった。	
\\	日本語[にほんご]を 教[おし]えれば 教[おし]えるほど、 教[おし]えることの 難[むずか]しさが 分[わ]かった。
\\	その小説は読めば読むほど味があった。	
\\	その 小説[しょうせつ]は 読[よ]めば 読[よ]むほど 味[あじ]があった。
\\	豊かな生活を送るには健康が第一だ。	
\\	豊[ゆた]かな 生活[せいかつ]を 送[おく]るには 健康[けんこう]が 第一[だいいち]だ。
\\	ハイヒールはハイキングには不向きだ。	
\\	ハイヒールはハイキングには 不向[ふむ]きだ。
\\	日本を知るには日本語を学ぶのが一番だ。	
\\	日本[にほん]を 知[し]るには 日本語[にほんご]を 学[まな]ぶのが 一番[いちばん]だ。
\\	人生を豊かにするには趣味をたくさん持つことが大事だ。	
\\	人生[じんせい]を 豊[ゆた]かにするには 趣味[しゅみ]をたくさん 持[も]つことが 大事[だいじ]だ。
\\	人の性格を見抜くには深い洞察力が必要だ。	
\\	人[ひと]の 性格[せいかく]を 見抜[みぬ]くには 深[ふか]い 洞察[どうさつ] 力[りょく]が 必要[ひつよう]だ。
\\	外国語の能力を伸ばすにはその国に行くのが一番だろう。	
\\	外国[がいこく] 語[ご]の 能力[のうりょく]を 伸[の]ばすにはその 国[くに]に 行[い]くのが 一番[いちばん]だろう。
\\	世界の平和を維持するにはエリートの交流より庶民のレベルの交流が肝心だ。	
\\	世界[せかい]の 平和[へいわ]を 維持[いじ]するにはエリートの 交流[こうりゅう]より 庶民[しょみん]のレベルの 交流[こうりゅう]が 肝心[かんじん]だ。
\\	期末レポートを書くにはワープロが欠かせない。	
\\	期末[きまつ]レポートを 書[か]くにはワープロが 欠[か]かせない。
\\	あの人を説得するには時間がかかる。	
\\	あの 人[ひと]を 説得[せっとく]するには 時間[じかん]がかかる。
\\	このクラブの会員になるには会員の推薦状が必要だった。	
\\	このクラブの 会員[かいいん]になるには 会員[かいいん]の 推薦[すいせん] 状[じょう]が 必要[ひつよう]だった。
\\	このかばんは長旅には便利だ。	
\\	このかばんは 長旅[ながたび]には 便利[べんり]だ。
\\	この研究にはかなりの時間とお金がかかる。	
\\	この 研究[けんきゅう]にはかなりの 時間[じかん]とお 金[かね]がかかる。
\\	週末には家でテレビを見ています。	
\\	週末[しゅうまつ]には 家[いえ]でテレビを 見[み]ています。
\\	母には時々電話をします。	
\\	母[はは]には 時々[ときどき] 電話[でんわ]をします。
\\	黒板には漢字が書いてあった。	
\\	黒板[こくばん]には 漢字[かんじ]が 書[か]いてあった。
\\	事務所には誰もいなかった。	
\\	事務所[じむしょ]には 誰[だれ]もいなかった。
\\	東京には一人で行きました。	
\\	東京[とうきょう]には 一人[ひとり]で 行[い]きました。
\\	私は韓国語の勉強に韓国へ行った。	
\\	私[わたし]は 韓国[かんこく] 語[ご]の 勉強[べんきょう]に 韓国[かんこく]へ 行[い]った。
\\	私は韓国語を勉強しに韓国へ行った。	
\\	私[わたし]は 韓国[かんこく] 語[ご]を 勉強[べんきょう]しに 韓国[かんこく]へ 行[い]った。
\\	私は韓国語を勉強するために韓国へ行った。	
\\	私[わたし]は 韓国[かんこく] 語[ご]を 勉強[べんきょう]するために 韓国[かんこく]へ 行[い]った。
\\	私は韓国語を勉強するのに韓国へ行った。	
\\	私[わたし]は 韓国[かんこく] 語[ご]を 勉強[べんきょう]するのに 韓国[かんこく]へ 行[い]った。
\\	漢和辞典によって知らない漢字を調べる。	
\\	漢和[かんわ] 辞典[じてん]によって 知[し]らない 漢字[かんじ]を 調[しら]べる。
\\	先生によって、もちろん、教え方が違います。	
\\	先生[せんせい]によって、もちろん、 教[おし]え 方[かた]が 違[ちが]います。
\\	戦争によって父を亡くした。	
\\	戦争[せんそう]によって 父[ちち]を 亡[な]くした。
\\	この研究所は文部省によって設立された。	
\\	この 研究所[けんきゅうじょ]は 文部省[もんぶしょう]によって 設立[せつりつ]された。
\\	奨学金がもらえるかどうかによって大学入学を決めます。	
\\	奨学[しょうがく] 金[きん]がもらえるかどうかによって 大学[だいがく] 入学[にゅうがく]を 決[き]めます。
\\	何を食べるかによって健康状態は変わる。	
\\	何[なに]を 食[た]べるかによって 健康[けんこう] 状態[じょうたい]は 変[か]わる。
\\	我々は本を読むことによって視野を広げることが出来る。	
\\	我々[われわれ]は 本[ほん]を 読[よ]むことによって 視野[しや]を 広[ひろ]げることが 出来[でき]る。
\\	このごろは宇宙中継によって世界のニュースをテレビで見ることが出来る。	
\\	このごろは 宇宙[うちゅう] 中継[ちゅうけい]によって 世界[せかい]のニュースをテレビで 見[み]ることが 出来[でき]る。
\\	その老人達はわずかの貯金によって生活している。	
\\	その 老人[ろうじん] 達[たち]はわずかの 貯金[ちょきん]によって 生活[せいかつ]している。
\\	その問題は話し合いによって解決出来るはずだ。	
\\	その 問題[もんだい]は 話し合[はなしあ]いによって 解決[かいけつ] 出来[でき]るはずだ。
\\	人によって、年のとり方が違う。	
\\	人[ひと]によって、 年[とし]のとり 方[かた]が 違[ちが]う。
\\	僕はその日の気分によって、違う音楽を聞きます。	
\\	僕[ぼく]はその 日[ひ]の 気分[きぶん]によって、 違[ちが]う 音楽[おんがく]を 聞[き]きます。
\\	定年は会社によって違う。	
\\	定年[ていねん]は 会社[かいしゃ]によって 違[ちが]う。
\\	その青年は麻薬によって青春時代を失ってしまった。	
\\	その 青年[せいねん]は 麻薬[まやく]によって 青春[せいしゅん] 時代[じだい]を 失[うしな]ってしまった。
\\	彼は難病によって再起不能になった。	
\\	彼[かれ]は 難病[なんびょう]によって 再起[さいき] 不能[ふのう]になった。
\\	この絵はピカソによって描かれた。	
\\	この 絵[え]はピカソによって 描[えが]かれた。
\\	アメリカ大陸はコロンブスによって発見された。	
\\	アメリカ大陸[あめりかたいりく]はコロンブスによって 発見[はっけん]された。
\\	日本へ行けるかどうかによって、来年の計画が全く変わってきます。	
\\	日本[にほん]へ 行[い]けるかどうかによって、 来年[らいねん]の 計画[けいかく]が 全[まった]く 変[か]わってきます。
\\	君と結婚出来るかどうかによって、僕の人生の幸、不幸が決まるんです。	
\\	君[きみ]と 結婚[けっこん] 出来[でき]るかどうかによって、 僕[ぼく]の 人生[じんせい]の 幸[こう]、 不幸[ふこう]が 決[き]まるんです。
\\	どの日本語の先生に習うかによって、学習者の進歩が違うはずだ。	
\\	どの 日本語[にほんご]の 先生[せんせい]に 習[なら]うかによって、 学習[がくしゅう] 者[しゃ]の 進歩[しんぽ]が 違[ちが]うはずだ。
\\	誰が演奏するかによって、同じ曲でも、印象が違う。	
\\	誰[だれ]が 演奏[えんそう]するかによって、 同[おな]じ 曲[きょく]でも、 印象[いんしょう]が 違[ちが]う。
\\	アメリカでは麻薬を厳しく取り締まることによって、犯罪を減らすことが出来るのではないだろうか。	
\\	アメリカでは 麻薬[まやく]を 厳[きび]しく 取り締[とりし]まることによって、 犯罪[はんざい]を 減[へ]らすことが 出来[でき]るのではないだろうか。
\\	自動車の事故による死者の数は毎年増えている。	
\\	自動車[じどうしゃ]の 事故[じこ]による 死者[ししゃ]の 数[かず]は 毎年[まいとし] 増[ふ]えている。
\\	誤診による手遅れはよくあることだ。	
\\	誤診[ごしん]による 手遅[ておく]れはよくあることだ。
\\	日本語の先生の教え方は先生によって違いますよ。	
\\	日本語[にほんご]の 先生[せんせい]の 教[おし]え 方[かた]は 先生[せんせい]によって 違[ちが]いますよ。
\\	日本語の教授法は教師により異なる。	
\\	日本語[にほんご]の 教授[きょうじゅ] 法[ほう]は 教師[きょうし]により 異[こと]なる。
\\	人文科学の領域でもコンピュータによる研究が盛んだ。	
\\	人文[じんぶん] 科学[かがく]の 領域[りょういき]でもコンピュータによる 研究[けんきゅう]が 盛[さか]んだ。
\\	ニューヨークでは麻薬常用者による殺人事件が多発している。	
\\	ニューヨークでは 麻薬[まやく] 常用[じょうよう] 者[しゃ]による 殺人[さつじん] 事件[じけん]が 多発[たはつ]している。
\\	僕はバスで会社に行っています。	
\\	僕[ぼく]はバスで 会社[かいしゃ]に 行[い]っています。
\\	都会のサラリーマンは公共の交通機関によって通勤している。	
\\	都会[とかい]のサラリーマンは 公共[こうきょう]の 交通[こうつう] 機関[きかん]によって 通勤[つうきん]している。
\\	私はワープロで手紙を書いています。	
\\	私[わたし]はワープロで 手紙[てがみ]を 書[か]いています。
\\	書類作成業務はワープロによって処理している。	
\\	書類作成[しょるいさくせい] 業務[ぎょうむ]はワープロによって 処理[しょり]している。
\\	すみませんが、原稿をファックスで送って下さいませんか。	
\\	すみませんが、 原稿[げんこう]をファックスで 送[おく]って 下[くだ]さいませんか。
\\	最近はファックスによって通信連絡するのが常識化されている。	
\\	最近[さいきん]はファックスによって 通信[つうしん] 連絡[れんらく]するのが 常識[じょうしき] 化[か]されている。
\\	アリスはジョンにだまされた。	
\\	アリスはジョンにだまされた。
\\	デモ隊は機動隊によって一掃された。	
\\	デモ 隊[たい]は 機動[きどう] 隊[たい]によって 一掃[いっそう]された。
\\	僕は手紙を両親に読まれた。	
\\	僕[ぼく]は 手紙[てがみ]を 両親[りょうしん]に 読[よ]まれた。
\\	我々は研究所の機密書類を外部の者によって読まれた。	
\\	我々[われわれ]は 研究所[けんきゅうじょ]の 機密[きみつ] 書類[しょるい]を 外部[がいぶ]の 者[もの]によって 読[よ]まれた。
\\	私は日本語の先生に発音を直された。	
\\	私[わたし]は 日本語[にほんご]の 先生[せんせい]に 発音[はつおん]を 直[なお]された。
\\	学生達は厳しい日本語教師によって発音矯正をされた。	
\\	学生[がくせい] 達[たち]は 厳[きび]しい 日本語[にほんご] 教師[きょうし]によって 発音[はつおん] 矯正[きょうせい]をされた。
\\	仕事でとても疲れた。	
\\	仕事[しごと]でとても 疲[つか]れた。
\\	病気のために会社を休んだ。	
\\	病気[びょうき]のために 会社[かいしゃ]を 休[やす]んだ。
\\	停電で大変困った。	
\\	停電[ていでん]で 大変[たいへん] 困[こま]った。
\\	日本の経済発展によって日本語学習者が増えた。	
\\	日本[にほん]の 経済[けいざい] 発展[はってん]によって 日本語[にほんご] 学習[がくしゅう] 者[しゃ]が 増[ふ]えた。
\\	金の問題で友達とけんかした。	
\\	金[きん]の 問題[もんだい]で 友達[ともだち]とけんかした。
\\	つまらないことで悩んでいるね。	
\\	つまらないことで 悩[なや]んでいるね。
\\	ホールさんは今仕事の関係で東京に行っています。	
\\	ホールさんは 今[いま] 仕事[しごと]の 関係[かんけい]で 東京[とうきょう]に 行[い]っています。
\\	鈴木は単位の関係で今年は卒業出来ないそうだ。	
\\	鈴木[すずき]は 単位[たんい]の 関係[かんけい]で 今年[ことし]は 卒業[そつぎょう] 出来[でき]ないそうだ。
\\	時間の関係で、残念ですが、詳しくお話出来ません。	
\\	時間[じかん]の 関係[かんけい]で、 残念[ざんねん]ですが、 詳[くわ]しくお 話[はなし] 出来[でき]ません。
\\	部屋の広さの関係で、四十人以上はお呼び出来ません。	
\\	部屋[へや]の 広[ひろ]さの 関係[かんけい]で、四十人[よんじゅうにん] 以上[いじょう]はお 呼[よ]び 出来[でき]ません。
\\	桃子は年齢の関係で、採用されなかったようだ。	
\\	桃子[ももこ]は 年齢[ねんれい]の 関係[かんけい]で、 採用[さいよう]されなかったようだ。
\\	レズリーは会社の関係でよく海外に出張するらしい。	
\\	レズリーは 会社[かいしゃ]の 関係[かんけい]でよく 海外[かいがい]に 出張[しゅっちょう]するらしい。
\\	みゆきはご主人の仕事の関係でロンドンに三年住むそうだ。	
\\	みゆきはご 主人[しゅじん]の 仕事[しごと]の 関係[かんけい]でロンドンに 三年[さんねん] 住[す]むそうだ。
\\	予算の関係で、そのプロジェクトは中止になった。	
\\	予算[よさん]の 関係[かんけい]で、そのプロジェクトは 中止[ちゅうし]になった。
\\	私は明日の講義で日本文化のことを話すつもりだ。	
\\	私[わたし]は 明日[あした]の 講義[こうぎ]で 日本[にほん] 文化[ぶんか]のことを 話[はな]すつもりだ。
\\	彼女のことはもう忘れた方がいいよ。	
\\	彼女[かのじょ]のことはもう 忘[わす]れた 方[ほう]がいいよ。
\\	日本留学のことでちょっとご相談があるんですが。	
\\	日本[にほん] 留学[りゅうがく]のことでちょっとご 相談[そうだん]があるんですが。
\\	実は、明日の会議のことなんですが。	
\\	実[じつ]は、 明日[あした]の 会議[かいぎ]のことなんですが。
\\	昨日荒井さんが君のことをいろいろ聞いていたよ。	
\\	昨日[きのう] 荒井[あらい]さんが 君[きみ]のことをいろいろ 聞[き]いていたよ。
\\	レポートは日本の大学生活のことを書こうと思っている。	
\\	レポートは 日本[にほん]の 大学[だいがく] 生活[せいかつ]のことを 書[か]こうと 思[おも]っている。
\\	子供のことは私が責任を持ちます。	
\\	子供[こども]のことは 私[わたし]が 責任[せきにん]を 持[も]ちます。
\\	今回の実習のことでは大変お世話になりました。	
\\	今回[こんかい]の 実習[じっしゅう]のことでは 大変[たいへん]お 世話[せわ]になりました。
\\	あのう、お電話したのは、来週のパーティーのことなんですが。	
\\	あのう、お 電話[でんわ]したのは、 来週[らいしゅう]のパーティーのことなんですが。
\\	ー先生、レポートのことなんですが。 ー何ですか。 ー締め切りを一日延ばしてもらえないでしょうか。	
\\	ー 先生[せんせい]、レポートのことなんですが。 ー 何[なん]ですか。 ー 締め切[しめき]りを 一日[いちにち] 延[の]ばしてもらえないでしょうか。
\\	中島のことだから、今日もまた遅れて来るだろう。	
\\	中島[なかじま]のことだから、 今日[きょう]もまた 遅[おく]れて 来[く]るだろう。
\\	傷みやすいとうふのことだから、冷蔵庫に入れておいた方がいいですよ。	
\\	傷[いた]みやすいとうふのことだから、 冷蔵庫[れいぞうこ]に 入[い]れておいた 方[ほう]がいいですよ。
\\	夏のことだから、食べ物には十分気を付けて下さい。	
\\	夏[なつ]のことだから、 食べ物[たべもの]には 十分[じゅうぶん] 気[き]を 付[つ]けて 下[くだ]さい。
\\	お金持ちの日本人のことだから、きっと高くても買いますよ。	
\\	お 金持[かねも]ちの 日本人[にほんじん]のことだから、きっと 高[たか]くても 買[か]いますよ。
\\	カラオケの好きな木下さんのことだから、歌い始めたらマイクを離さないんじゃありませんか。	
\\	カラオケの 好[す]きな 木下[きのした]さんのことだから、 歌[うた]い 始[はじ]めたらマイクを 離[はな]さないんじゃありませんか。
\\	酒好きの山田のことだから、この酒を見たら飲まずには帰らないだろう。	
\\	酒好[さけず]きの 山田[やまだ]のことだから、この 酒[さけ]を 見[み]たら 飲[の]まずには 帰[かえ]らないだろう。
\\	要点のみを話して下さい。	
\\	要点[ようてん]のみを 話[はな]して 下[くだ]さい。
\\	これは吉田先生にのみ話してあります。	
\\	これは 吉田[よしだ] 先生[せんせい]にのみ 話[はな]してあります。
\\	勝てるかどうか分からないが、ベストを尽くすのみだ。	
\\	勝[か]てるかどうか 分[わ]からないが、ベストを 尽[つ]くすのみだ。
\\	この答えは私のみ知っている。	
\\	この 答[こた]えは 私[わたし]のみ 知[し]っている。
\\	姓のみ記入のこと。	
\\	姓[せい]のみ 記入[きにゅう]のこと。
\\	私はただ言われたことをするのみだ。	
\\	私[わたし]はただ 言[い]われたことをするのみだ。
\\	ただ実行あるのみだ。	
\\	ただ 実行[じっこう]あるのみだ。
\\	外装が終わったのみでまだほかは何も手が着いていない。	
\\	外装[がいそう]が 終[お]わったのみでまだほかは 何[なに]も 手[て]が 着[つ]いていない。
\\	私は日本語でのみ説明出来る。	
\\	私[わたし]は 日本語[にほんご]でのみ 説明[せつめい] 出来[でき]る。
\\	私は日本語のみで説明出来る。	
\\	私[わたし]は 日本語[にほんご]のみで 説明[せつめい] 出来[でき]る。
\\	私は鈴木先生の指導の下で修士論文を書き上げた。	
\\	私[わたし]は 鈴木[すずき] 先生[せんせい]の 指導[しどう]の 下[もと]で 修士[しゅうし] 論文[ろんぶん]を 書き上[かきあ]げた。
\\	さんさんたる太陽の下で開会式が行われた。	
\\	さんさんたる 太陽[たいよう]の 下[もと]で 開会[かいかい] 式[しき]が 行[おこな]われた。
\\	彼はカラヤンの下で指導法を学んだ。	
\\	彼[かれ]はカラヤンの 下[もと]で 指導[しどう] 法[ほう]を 学[まな]んだ。
\\	弁護士の立ち会いの下で私達の離婚が成立した。	
\\	弁護士[べんごし]の 立ち会[たちあ]いの 下[もと]で 私[わたし] 達[たち]の 離婚[りこん]が 成立[せいりつ]した。
\\	囚人達は厳しい監視の下で強制労働をさせられた。	
\\	囚人[しゅうじん] 達[たち]は 厳[きび]しい 監視[かんし]の 下[もと]で 強制[きょうせい] 労働[ろうどう]をさせられた。
\\	このインフレの下では金を貯めても意味がない。	
\\	このインフレの 下[もと]では 金[きん]を 貯[た]めても 意味[いみ]がない。
\\	私はもうけは折半という約束の下にこの仕事をしている。	
\\	私[わたし]はもうけは 折半[せっぱん]という 約束[やくそく]の 下[もと]にこの 仕事[しごと]をしている。
\\	田口先生の許可の下にこの実験室を使っている。	
\\	田口[たぐち] 先生[せんせい]の 許可[きょか]の 下[もと]にこの 実験[じっけん] 室[しつ]を 使[つか]っている。
\\	この計画は書類の上では問題なさそうだ。	
\\	この 計画[けいかく]は 書類[しょるい]の 上[うえ]では 問題[もんだい]なさそうだ。
\\	計算の上では二百万円ぐらいもうかることになっている。	
\\	計算[けいさん]の 上[うえ]では 二百[にひゃく] 万[まん] 円[えん]ぐらいもうかることになっている。
\\	このプロジェクトは話の上ではうまく行きそうだが、何となく不安だ。	
\\	このプロジェクトは 話[はなし]の 上[うえ]ではうまく 行[い]きそうだが、 何[なん]となく 不安[ふあん]だ。
\\	数字の上では川野の方が倉田より強そうだ。	
\\	数字[すうじ]の 上[うえ]では 川野[かわの]の 方[ほう]が 倉田[くらた]より 強[つよ]そうだ。
\\	大野は仕事の上では完全主儀者だが、私生活はだらしがない。	
\\	大野[おおの]は 仕事[しごと]の 上[うえ]では 完全[かんぜん] 主[しゅ] 儀[ぎ] 者[しゃ]だが、 私生活[しせいかつ]はだらしがない。
\\	彼の行為は法律の上では罰しようがない。	
\\	彼[かれ]の 行為[こうい]は 法律[ほうりつ]の 上[うえ]では 罰[ばっ]しようがない。
\\	吉田さんと最後に会ったのは1985年の五月のことだ。	
\\	吉田[よしだ]さんと 最後[さいご]に 会[あ]ったのは1985 年[ねん]の 五月[ごがつ]のことだ。
\\	日本語の面白さが分かり始めたのはごく最近のことだ。	
\\	日本語[にほんご]の 面白[おもしろ]さが 分[わ]かり 始[はじ]めたのはごく 最近[さいきん]のことだ。
\\	本格的なロボットが現れたのは1960年代後半のことである。	
\\	本格[ほんかく] 的[てき]なロボットが 現[あらわ]れたのは1960 年代[ねんだい] 後半[こうはん]のことである。
\\	ゲラ刷りが出来るのは来月の終わりのことになるだろう。	
\\	ゲラ刷[げらず]りが 出来[でき]るのは 来月[らいげつ]の 終[お]わりのことになるだろう。
\\	日本製品の評判が悪かったのは戦後二十年くらいの間のことだ。	
\\	日本[にほん] 製品[せいひん]の 評判[ひょうばん]が 悪[わる]かったのは 戦後[せんご] 二十年[にじゅうねん]くらいの 間[あいだ]のことだ。
\\	健一と利子が仲良く歩いているところを見たのはつい一か月ほど前のことなのに彼らはもう別れたという話だ。	
\\	健一[けんいち]と 利子[としこ]が 仲良[なかよ]く 歩[ある]いているところを 見[み]たのはつい 一か月[いっかげつ]ほど 前[まえ]のことなのに 彼[かれ]らはもう 別[わか]れたという 話[はなし]だ。
\\	グローバルな時代に自国のことだけを考えるわけにはいかぬ。	
\\	グローバルな 時代[じだい]に 自国[じこく]のことだけを 考[かんが]えるわけにはいかぬ。
\\	知らぬことを知らぬと言うには勇気が要る。	
\\	知[し]らぬことを 知[し]らぬと 言[い]うには 勇気[ゆうき]が 要[い]る。
\\	彼は何を聞いても、知らぬとしか答えぬ。	
\\	彼[かれ]は 何[なに]を 聞[き]いても、 知[し]らぬとしか 答[こた]えぬ。
\\	その日本人は英語が分からぬようだった。	
\\	その 日本人[にほんじん]は 英語[えいご]が 分[わ]からぬようだった。
\\	最近は滅多に酒を飲まぬが、若い頃はよく飲んだものだ。	
\\	最近[さいきん]は 滅多[めった]に 酒[さけ]を 飲[の]まぬが、 若[わか]い 頃[ころ]はよく 飲[の]んだものだ。
\\	日本語が話せる者はこの辺にはおらぬ	
\\	日本語[にほんご]が 話[はな]せる 者[もの]はこの 辺[へん]にはおらぬ
\\	転ばぬ先の杖。	
\\	転[ころ]ばぬ 先[さき]の 杖[つえ]。
\\	知りぬが仏。	
\\	知[し]りぬが 仏[ほとけ]。
\\	言わぬが花。	
\\	言[い]わぬが 花[はな]。
\\	この本が面白くない。	
\\	この 本[ほん]が 面白[おもしろ]くない。
\\	朝ご飯を食べずに学校へ行った。	
\\	朝[あさ]ご 飯[はん]を 食[た]べずに 学校[がっこう]へ 行[い]った。
\\	昼ご飯を食べずに仕事をした。	
\\	昼[ひる]ご 飯[はん]を 食[た]べずに 仕事[しごと]をした。
\\	今日は風邪ぎみで、会社にも行かず、家で寝ていた。	
\\	今日[きょう]は 風邪[かぜ]ぎみで、 会社[かいしゃ]にも 行[い]かず、 家[いえ]で 寝[ね]ていた。
\\	そうは思わぬ。	
\\	そうは 思[おも]わぬ。
\\	知らぬことを知っていると言ってはいけない。	
\\	知[し]らぬことを 知[し]っていると 言[い]ってはいけない。
\\	お急ぎですか。	
\\	お 急[いそ]ぎですか。
\\	今お持ちのカードはもう使えません。	
\\	今[いま]お 持[も]ちのカードはもう 使[つか]えません。
\\	もう皆さんお集まりです。	
\\	もう 皆[みな]さんお 集[あつ]まりです。
\\	河野さん、課長がお呼びですよ。	
\\	河野[こうの]さん、 課長[かちょう]がお 呼[よ]びですよ。
\\	コピーはもうお済ですか。	
\\	コピーはもうお 済[すみ]ですか。
\\	どんな人をお望みですか。	
\\	どんな 人[ひと]をお 望[のぞ]みですか。
\\	お客様はまだお帰りじゃありません。	
\\	お 客様[きゃくさま]はまだお 帰[かえ]りじゃありません。
\\	お疲れでしたらこの部屋でしばらくお休み下さい。	
\\	お 疲[つか]れでしたらこの 部屋[へや]でしばらくお 休[やす]み 下[くだ]さい。
\\	今お使いのワープロはいつお買いになりましたか。	
\\	今[こん]お 使[つか]いのワープロはいつお 買[か]いになりましたか。
\\	先生は今何をご研究ですか。	
\\	先生[せんせい]は 今[いま] 何[なに]をご 研究[けんきゅう]ですか。
\\	天皇陛下がご訪問の国々はどこですか。	
\\	天皇陛下[てんのうへいか]がご 訪問[ほうもん]の 国々[くにぐに]はどこですか。
\\	お客様がもうすぐお帰りです。	
\\	お 客様[きゃくさま]がもうすぐお 帰[かえ]りです。
\\	今何をお読みですか。	
\\	今[いま] 何[なに]をお 読[よ]みですか。
\\	先生がお一人で部屋を掃除しておられる。	
\\	先生[せんせい]がお 一人[ひとり]で 部屋[へや]を 掃除[そうじ]しておられる。
\\	私のうちでは、父をはじめとして、家族全員スポーツが好きだ。	
\\	私[わたし]のうちでは、 父[ちち]をはじめとして、 家族[かぞく] 全員[ぜんいん]スポーツが 好[す]きだ。
\\	キャシーは、すしをはじめとして、日本食なら何でも食べる。	
\\	キャシーは、すしをはじめとして、 日本[にほん] 食[しょく]なら 何[なに]でも 食[た]べる。
\\	今年は大阪をはじめ、関西地方はどこも梅雨が短かった。	
\\	今年[ことし]は 大阪[おおさか]をはじめ、 関西[かんさい] 地方[ちほう]はどこも 梅雨[つゆ]が 短[みじか]かった。
\\	この会社は、社長をはじめとして、社員一同が同じ食堂で昼ご飯を食べている。	
\\	この 会社[かいしゃ]は、 社長[しゃちょう]をはじめとして、 社員[しゃいん] 一同[いちどう]が 同[おな]じ 食堂[しょくどう]で 昼[ひる]ご 飯[はん]を 食[た]べている。
\\	この事件には、警察をはじめ、近くの住民が皆強い興味を示した。	
\\	この 事件[じけん]には、 警察[けいさつ]をはじめ、 近[ちか]くの 住民[じゅうみん]が 皆[みな] 強[つよ]い 興味[きょうみ]を 示[しめ]した。
\\	少々お待ち下さい。	
\\	少々[しょうしょう]お 待[ま]ち 下[くだ]さい。
\\	電話でご注文下さい。	
\\	電話[でんわ]でご 注文[ちゅうもん] 下[くだ]さい。
\\	このペンをお使い下さい。	
\\	このペンをお 使[つか]い 下[くだ]さい。
\\	時間があまりありませんのでお急ぎ下さい。	
\\	時間[じかん]があまりありませんのでお 急[いそ]ぎ 下[くだ]さい。
\\	どうぞお気軽にご相談下さい。	
\\	どうぞお 気軽[きがる]にご 相談[そうだん] 下[くだ]さい。
\\	便利でお得な周遊券をご利用下さい。	
\\	便利[べんり]でお 得[とく]な 周遊[しゅうゆう] 券[けん]をご 利用[りよう] 下[くだ]さい。
\\	ご自分で掃除して下さい。	
\\	ご 自分[じぶん]で 掃除[そうじ]して 下[くだ]さい。
\\	ぬるま湯で洗濯なさって下さい。	
\\	ぬるま 湯[ゆ]で 洗濯[せんたく]なさって 下[くだ]さい。
\\	今晩お電話下さい。	
\\	今晩[こんばん]お 電話[でんわ] 下[くだ]さい。
\\	この部屋は使わないで下さい。	
\\	この 部屋[へや]は 使[つか]わないで 下[くだ]さい。
\\	見ていって下さい。	
\\	見[み]ていって 下[くだ]さい。
\\	食べてしまって下さい。	
\\	食[た]べてしまって 下[くだ]さい。
\\	来年の景気は今年よりもいいように思われる。	
\\	来年[らいねん]の 景気[けいき]は 今年[ことし]よりもいいように 思[おも]われる。
\\	この病気の治療は約一か月かかると思われる。	
\\	この 病気[びょうき]の 治療[ちりょう]は 約[やく] 一か月[いっかげつ]かかると 思[おも]われる。
\\	日本の輸出は今後あまり伸びないように思われる。	
\\	日本[にほん]の 輸出[ゆしゅつ]は 今後[こんご]あまり 伸[の]びないように 思[おも]われる。
\\	子供の非行を犯罪映画に結び付けるのは短絡的なように思われる。	
\\	子供[こども]の 非行[ひこう]を 犯罪[はんざい] 映画[えいが]に 結び付[むすびつ]けるのは 短絡[たんらく] 的[てき]なように 思[おも]われる。
\\	この大学の教育は理工系が特にいいように思われる。	
\\	この 大学[だいがく]の 教育[きょういく]は 理工[りこう] 系[けい]が 特[とく]にいいように 思[おも]われる。
\\	二十一世紀は情報化がもっと進んで、世界が一つの共同体になるだろうと思われる。	
\\	二十一世紀[にじゅういっせいき]は 情報[じょうほう] 化[か]がもっと 進[すす]んで、 世界[せかい]が 一[ひと]つの 共同[きょうどう] 体[たい]になるだろうと 思[おも]われる。
\\	日本語を習う時はローマ字を使わない方がよいと思われる。	
\\	日本語[にほんご]を 習[なら]う 時[とき]は ローマ字[ろーまじ]を 使[つか]わない 方[ほう]がよいと 思[おも]われる。
\\	健康管理には毎日三十分ぐらいの運動をするのが効果的だと思われる。	
\\	健康[けんこう] 管理[かんり]には 毎日[まいにち] 三十分[さんじゅっぷん]ぐらいの 運動[うんどう]をするのが 効果[こうか] 的[てき]だと 思[おも]われる。
\\	私には来年の景気は今年よりもいいように思われる。	
\\	私[わたし]には 来年[らいねん]の 景気[けいき]は 今年[ことし]よりもいいように 思[おも]われる。
\\	私にはこの病気の治療は約一か月かかると思われる。	
\\	私[わたし]にはこの 病気[びょうき]の 治療[ちりょう]は 約[やく] 一か月[いっかげつ]かかると 思[おも]われる。
\\	そんなことをしたら、人にばかだと思われますよ。	
\\	そんなことをしたら、 人[ひと]にばかだと 思[おも]われますよ。
\\	日本は外国人に不思議な国だと思われている。	
\\	日本[にほん]は 外国[がいこく] 人[じん]に 不思議[ふしぎ]な 国[くに]だと 思[おも]われている。
\\	山田には会社の仕事が単調に思われた。	
\\	山田[やまだ]には 会社[かいしゃ]の 仕事[しごと]が 単調[たんちょう]に 思[おも]われた。
\\	加代子には結婚はまだ夢のように思われた。	
\\	加代子[かよこ]には 結婚[けっこん]はまだ 夢[ゆめ]のように 思[おも]われた。
\\	この入口は従業員専用になっており、一般の人はここからは入れない。	
\\	この 入口[いりぐち]は 従業[じゅうぎょう] 員[いん] 専用[せんよう]になっており、 一般[いっぱん]の 人[ひと]はここからは 入[い]れない。
\\	彼は最近プラモデルに凝っており、暇さえあれば模型の飛行機を組み立てている。	
\\	彼[かれ]は 最近[さいきん]プラモデルに 凝[こ]っており、 暇[ひま]さえあれば 模型[もけい]の 飛行機[ひこうき]を 組み立[くみた]てている。
\\	日本では義務教育がよく行き届いており、識字率はほとんど100パーセントに近い。	
\\	日本[にほん]では 義務[ぎむ] 教育[きょういく]がよく 行き届[ゆきとど]いており、 識字[しきじ] 率[りつ]はほとんど100パーセントに 近[ちか]い。
\\	この高校は全寮制になっており、全学生が三つの寮で生活している。	
\\	この 高校[こうこう]は 全寮[ぜんりょう] 制[せい]になっており、 全[ぜん] 学生[がくせい]が 三[みっ]つの 寮[りょう]で 生活[せいかつ]している。
\\	私は親しい友達を通してそのピアニストと知り合いになった。	
\\	私[わたし]は 親[した]しい 友達[ともだち]を 通[とお]してそのピアニストと 知り合[しりあ]いになった。
\\	私は一年を通して五回ぐらい海外に行っている。	
\\	私[わたし]は 一年[いちねん]を 通[とお]して 五回[ごかい]ぐらい 海外[かいがい]に 行[い]っている。
\\	お書きになった御本を通して、先生のことは存じ上げておりました。	
\\	お 書[か]きになった 御[ご] 本[ほん]を 通[とお]して、 先生[せんせい]のことは 存じ上[ぞんじあ]げておりました。
\\	その事件のことは新聞の記事を通して知っていた。	
\\	その 事件[じけん]のことは 新聞[しんぶん]の 記事[きじ]を 通[とお]して 知[し]っていた。
\\	その政治家は、首相の側近を通して、首相に接近した。	
\\	その 政治[せいじ] 家[か]は、 首相[しゅしょう]の 側近[そっきん]を 通[とお]して、 首相[しゅしょう]に 接近[せっきん]した。
\\	二人は手紙のやり取りを通して親しくなっていた。	
\\	二人[ふたり]は 手紙[てがみ]のやり 取[と]りを 通[とお]して 親[した]しくなっていた。
\\	ラジオもテレビも全国の放送網を通して放送される。	
\\	ラジオもテレビも 全国[ぜんこく]の 放送[ほうそう] 網[もう]を 通[とお]して 放送[ほうそう]される。
\\	海外生活の経験を通していろいろなことを学んだ。	
\\	海外[かいがい] 生活[せいかつ]の 経験[けいけん]を 通[とお]していろいろなことを 学[まな]んだ。
\\	カーテンを通して室内の様子が見えた。	
\\	カーテンを 通[とお]して 室内[しつない]の 様子[ようす]が 見[み]えた。
\\	これらの写真を通して当時の人々の生活を偲ぶことが出来る。	
\\	これらの 写真[しゃしん]を 通[とお]して 当時[とうじ]の 人々[ひとびと]の 生活[せいかつ]を 偲[しの]ぶことが 出来[でき]る。
\\	現場に残された指紋を通して犯人が割れた。	
\\	現場[げんば]に 残[のこ]された 指紋[しもん]を 通[とお]して 犯人[はんにん]が 割[わ]れた。
\\	父は一生を通して、一度も東京を出たことがない。	
\\	父[ちち]は 一生[いっしょう]を 通[とお]して、 一度[いちど]も 東京[とうきょう]を 出[で]たことがない。
\\	新幹線が満員で東京から京都までずっと立ちっぱなしだった。	
\\	新幹線[しんかんせん]が 満員[まんいん]で 東京[とうきょう]から 京都[きょうと]までずっと 立[た]ちっぱなしだった。
\\	友達にまだお金を借りっぱなしだ。	
\\	友達[ともだち]にまだお 金[かね]を 借[か]りっぱなしだ。
\\	和彦はラジオをかけっぱなしでどこかへ行ってしまった。	
\\	和彦[かずひこ]はラジオをかけっぱなしでどこかへ 行[い]ってしまった。
\\	ブラジルのバレーボール・チームは今日まで勝ちっぱなしだ。	
\\	ブラジルのバレーボール・チームは 今日[きょう]まで 勝[か]ちっぱなしだ。
\\	今日は朝から電話が鳴りっぱなしだ。	
\\	今日[きょう]は 朝[あさ]から 電話[でんわ]が 鳴[な]りっぱなしだ。
\\	朝からしゃべりっぱなしで喉が痛い。	
\\	朝[あさ]からしゃべりっぱなしで 喉[のど]が 痛[いた]い。
\\	車のヘッドライトは一晩中つけっぱなしだったのでバッテリーが上がってしまった。	
\\	車[くるま]のヘッドライトは 一晩[ひとばん] 中[ちゅう]つけっぱなしだったのでバッテリーが 上[あ]がってしまった。
\\	光子は水を出しっぱなしで友達と電話で話している。	
\\	光子[みつこ]は 水[みず]を 出[だ]しっぱなしで 友達[ともだち]と 電話[でんわ]で 話[はな]している。
\\	茂はやかんをコンロにかけっぱなしにして出て行った。	
\\	茂[しげる]はやかんをコンロにかけっぱなしにして 出[で]て 行[い]った。
\\	開けっぱなしの窓から蚊がたくさん入ってきた。	
\\	開[あ]けっぱなしの 窓[まど]から 蚊[か]がたくさん 入[はい]ってきた。
\\	彼はまだたばこをやめたままだ。	
\\	彼[かれ]はまだたばこをやめたままだ。
\\	春子は帽子をかぶりっぱなしで頭を下げた。	
\\	春子[はるこ]は 帽子[ぼうし]をかぶりっぱなしで 頭[あたま]を 下[さ]げた。
\\	座ったままで話して下さい。	
\\	座[すわ]ったままで 話[はな]して 下[くだ]さい。
\\	長い間座りっぱなしでお尻が痛い。	
\\	長[なが]い 間[あいだ] 座[すわ]りっぱなしでお 尻[しり]が 痛[いた]い。
\\	私は苦しかったが走り続けた。	
\\	私[わたし]は 苦[くる]しかったが 走[はし]り 続[つづ]けた。
\\	サングラスをかけたやくざっぽい男が街角に立っていた。	
\\	サングラスをかけたやくざっぽい 男[おとこ]が 街角[まちかど]に 立[た]っていた。
\\	その女の人は確か黒っぽいセーターを着ていたと思います。	
\\	その 女[おんな]の 人[ひと]は 確[たし]か 黒[くろ]っぽいセーターを 着[き]ていたと 思[おも]います。
\\	彼はほれっぽい性格で、どの女の人も好きになってしまう。	
\\	彼[かれ]はほれっぽい 性格[せいかく]で、どの 女[おんな]の 人[ひと]も 好[す]きになってしまう。
\\	昼食時にはこのレストランはサラリーマンっぽい人達で一杯になる。	
\\	昼食[ちゅうしょく] 時[じ]にはこのレストランはサラリーマンっぽい 人達[ひとたち]で 一杯[いっぱい]になる。
\\	あの女優は実に女っぽいから好きだ。	
\\	あの 女優[じょゆう]は 実[じつ]に 女[おんな]っぽいから 好[す]きだ。
\\	こんな水っぽい酒、飲めないぞ。	
\\	こんな 水[みず]っぽい 酒[さけ]、 飲[の]めないぞ。
\\	あなた、こんな安っぽいソファを買ってきてどうするの?	
\\	あなた、こんな 安[やす]っぽいソファを 買[か]ってきてどうするの?
\\	公園のベンチに座って、きざっぽい男とあだっぽい女が話している。	
\\	公園[こうえん]のベンチに 座[すわ]って、きざっぽい 男[おとこ]とあだっぽい 女[おんな]が 話[はな]している。
\\	うちの息子は飽きっぽいんで、次から次へと新しいことをするんですが、一つとして、ものにならないんですよ。	
\\	うちの 息子[むすこ]は 飽[あ]きっぽいんで、 次[つぎ]から 次[つぎ]へと 新[あたら]しいことをするんですが、 一[ひと]つとして、ものにならないんですよ。
\\	うたぐりっぽい人って、嫌ですね。	
\\	うたぐりっぽい 人[ひと]って、 嫌[いや]ですね。
\\	じゅうたんは汚れっぽいから、私は使っていません。	
\\	じゅうたんは 汚[よご]れっぽいから、 私[わたし]は 使[つか]っていません。
\\	私は最近とても疲れっぽくなった。	
\\	私[わたし]は 最近[さいきん]とても 疲[つか]れっぽくなった。
\\	父は年をとってから怒りっぽくなった。	
\\	父[ちち]は 年[とし]をとってから 怒[おこ]りっぽくなった。
\\	トムは二十年来日本語の勉強を続けている。	
\\	トムは 二十年来[にじゅうねんらい] 日本語[にほんご]の 勉強[べんきょう]を 続[つづ]けている。
\\	今年の冬は五十年来の寒さだ。	
\\	今年[ことし]の 冬[ふゆ]は 五十年来[ごじゅうねんらい]の 寒[さむ]さだ。
\\	父が三十年来、同じ会社に勤めています。	
\\	父[ちち]が 三十年来[さんじゅうねんらい]、 同[おな]じ 会社[かいしゃ]に 勤[つと]めています。
\\	私はこの研究を十年来続けてきました。	
\\	私[わたし]はこの 研究[けんきゅう]を 十年来[じゅうねんらい] 続[つづ]けてきました。
\\	夜半来の雨が、ようやく上がったようだ。	
\\	夜半[やはん] 来[らい]の 雨[あめ]が、ようやく 上[あ]がったようだ。
\\	僕と岡田とは四十年来の友人だ。	
\\	僕[ぼく]と 岡田[おかだ]とは 四十年来[よんじゅうねんらい]の 友人[ゆうじん]だ。
\\	今年の夏は六十年来の暑さだそうだ。	
\\	今年[ことし]の 夏[なつ]は 六十年来[ろくじゅうねんらい]の 暑[あつ]さだそうだ。
\\	これは三十年来の大雪です。	
\\	これは 三十年来[さんじゅうねんらい]の 大雪[おおゆき]です。
\\	彼は二時間日本語を勉強している。	
\\	彼[かれ]は 二時間[にじかん] 日本語[にほんご]を 勉強[べんきょう]している。
\\	二日前から雪が降っている。	
\\	二日[ふつか] 前[まえ]から 雪[ゆき]が 降[ふ]っている。
\\	例のプロジェクトはどうなりましたか。	
\\	例[れい]のプロジェクトはどうなりましたか。
\\	今年の忘年会はまた例の中華料理屋だそうだ。	
\\	今年[ことし]の 忘年会[ぼうねんかい]はまた 例[れい]の 中華[ちゅうか] 料理[りょうり] 屋[や]だそうだ。
\\	例のレポートは出来ましたか。	
\\	例[れい]のレポートは 出来[でき]ましたか。
\\	例のコピーが出来ましたよ。	
\\	例[れい]のコピーが 出来[でき]ましたよ。
\\	例のダイアナ妃の本、さっそく買って読んだよ。	
\\	例[れい]のダイアナ 妃[ひ]の 本[ほん]、さっそく 買[か]って 読[よ]んだよ。
\\	ー一時に学生が来るんですよ。 ーああ、例のドイツから来た学生ですね。	
\\	ー 一時[いちじ]に 学生[がくせい]が 来[く]るんですよ。 ーああ、 例[れい]のドイツから 来[き]た 学生[がくせい]ですね。
\\	ー今晩、会える? ーええ、いいわよ。 ーじゃ、六時に例の喫茶店で。	ー
\\	ー 今晩[こんばん]、 会[あ]える? ーええ、いいわよ。 ーじゃ、六時[ろくじ]に 例[れい]の 喫茶店[きっさてん]で。
\\	吉田さんがまた例の冗談を言ったが誰も笑わなかった。	
\\	吉田[よしだ]さんがまた 例[れい]の 冗談[じょうだん]を 言[い]ったが 誰[だれ]も 笑[わら]わなかった。
\\	本は役に立つ。	
\\	本[ほん]は 役に立[やくにた]つ。
\\	例の本は役に立つ。	
\\	例[れい]の 本[ほん]は 役に立[やくにた]つ。
\\	頭がよくなる薬はない。	
\\	頭[あたま]がよくなる 薬[くすり]はない。
\\	日本語に強くなる本がありますか。	
\\	日本語[にほんご]に 強[つよ]くなる 本[ほん]がありますか。
\\	バロック音楽を心が静まる音楽だ。	
\\	バロック 音楽[おんがく]を 心[こころ]が 静[しず]まる 音楽[おんがく]だ。
\\	これは日本の経営法がよく分かる本です。	
\\	これは 日本[にほん]の 経営[けいえい] 法[ほう]がよく 分[わ]かる 本[ほん]です。
\\	体重が減る運動を教えて下さい。	
\\	体重[たいじゅう]が 減[へ]る 運動[うんどう]を 教[おし]えて 下[くだ]さい。
\\	気持ちが明るくなる話が聞きたいです。	
\\	気持[きも]ちが 明[あか]るくなる 話[はなし]が 聞[き]きたいです。
\\	これは太らないお菓子ですから、どうぞたくさん召し上がって下さい。	
\\	これは 太[ふと]らないお 菓子[かし]ですから、どうぞたくさん 召し上[めしあ]がって 下[くだ]さい。
\\	あの先生の講義はすぐ眠くなる講義だね。	
\\	あの 先生[せんせい]の 講義[こうぎ]はすぐ 眠[ねむ]くなる 講義[こうぎ]だね。
\\	その薬は頭がよくなる。	
\\	その 薬[くすり]は 頭[あたま]がよくなる。
\\	その薬を飲めば、頭がよくなる。	
\\	その 薬[くすり]を 飲[の]めば、 頭[あたま]がよくなる。
\\	飲めば頭がよくなる薬はない。	
\\	飲[の]めば 頭[あたま]がよくなる 薬[くすり]はない。
\\	読めば日本語に強くなる本がありますか。	
\\	読[よ]めば 日本語[にほんご]に 強[つよ]くなる 本[ほん]がありますか。
\\	バロック音楽は聞けば心が静まる音楽だ。	
\\	バロック 音楽[おんがく]は 聞[き]けば 心[こころ]が 静[しず]まる 音楽[おんがく]だ。
\\	そんな恥ずかしいことが出来ますか。	
\\	そんな 恥[は]ずかしいことが 出来[でき]ますか。
\\	誰があんな人にお金なんか貸すもんですか。	
\\	誰[だれ]があんな 人[ひと]にお 金[かね]なんか 貸[か]すもんですか。
\\	今さら何を隠すことがあるだろうか。	
\\	今[いま]さら 何[なに]を 隠[かく]すことがあるだろうか。
\\	彼女がこんな所へ来るわけがないじゃない。	
\\	彼女[かのじょ]がこんな 所[ところ]へ 来[く]るわけがないじゃない。
\\	何だ、テレビを見ているのか。	
\\	何[なん]だ、テレビを 見[み]ているのか。
\\	早くしないか。	
\\	早[はや]くしないか。
\\	こんなものがお客様に出せますか。	
\\	こんなものがお 客様[きゃくさま]に 出[だ]せますか。
\\	あんな男に何が出来るか。	
\\	あんな 男[おとこ]に 何[なに]が 出来[でき]るか。
\\	私の気持ちがあなたなんかに分かるもんですか。	
\\	私[わたし]の 気持[きも]ちがあなたなんかに 分[わ]かるもんですか。
\\	こんなばかなことってあるでしょうか。	
\\	こんなばかなことってあるでしょうか。
\\	どうしてそれが彼にわからないことがあろうか。	
\\	どうしてそれが 彼[かれ]にわからないことがあろうか。
\\	彼にそんな難しいことが分かるはずがないじゃない。	
\\	彼[かれ]にそんな 難[むずか]しいことが 分[わ]かるはずがないじゃない。
\\	何も泣くことはないじゃないか。	
\\	何[なに]も 泣[な]くことはないじゃないか。
\\	男は「開け、ゴマ!」と叫んだ。すると、なんと岩の扉が音もなく開いたではないか。	
\\	男[おとこ]は
\\	開[ひら]け、ゴマ!」と 叫[さけ]んだ。すると、なんと 岩[いわ]の 扉[とびら]が 音[おと]もなく 開[ひら]いたではないか。
\\	ああ、ここにいたの。	
\\	ああ、ここにいたの。
\\	さっさと歩かないか。	
\\	さっさと 歩[ある]かないか。
\\	そんな恥ずかしいことは出来ない。	
\\	そんな 恥[は]ずかしいことは 出来[でき]ない。
\\	誰もあんな人にお金なんか貸さない。	
\\	誰[だれ]もあんな 人[ひと]にお 金[かね]なんか 貸[か]さない。
\\	今さら何も隠すことはない。	
\\	今[いま]さら 何[なに]も 隠[かく]すことはない。
\\	飛行機の中ではろくに寝られなかった。	
\\	飛行機[ひこうき]の 中[なか]ではろくに 寝[ね]られなかった。
\\	ろくに勉強もしないで試験を受けたので、全然出来なかった。	
\\	ろくに 勉強[べんきょう]もしないで 試験[しけん]を 受[う]けたので、 全然[ぜんぜん] 出来[でき]なかった。
\\	短いパリ滞在だったので、彼女とはろくに会えなかった。	
\\	短[みじか]いパリ 滞在[たいざい]だったので、 彼女[かのじょ]とはろくに 会[あ]えなかった。
\\	彼はフランスのことなどろくに知らないのに、よく知っているような口を利く。	
\\	彼[かれ]はフランスのことなどろくに 知[し]らないのに、よく 知[し]っているような 口[くち]を 利[き]く。
\\	せっかくデパートまで買い物に出かけたのに、買い物中気分が悪くなってろくに買い物もせずに帰って来た。	
\\	せっかくデパートまで 買い物[かいもの]に 出[で]かけたのに、 買い物[かいもの] 中[ちゅう] 気分[きぶん]が 悪[わる]くなってろくに 買い物[かいもの]もせずに 帰[かえ]って 来[き]た。
\\	僕の犬はろくに芸もしない。	
\\	僕[ぼく]の 犬[いぬ]はろくに 芸[げい]もしない。
\\	あの男はろくなあいさつもしない。	
\\	あの 男[おとこ]はろくなあいさつもしない。
\\	ナンシーはろくな手紙も書けない。	
\\	ナンシーはろくな 手紙[てがみ]も 書[か]けない。
\\	あの人達はろくな食事をしていないらしい。	
\\	あの 人達[ひとたち]はろくな 食事[しょくじ]をしていないらしい。
\\	あの犬はろくなえさを食べていない。	
\\	あの 犬[いぬ]はろくなえさを 食[た]べていない。
\\	あの先生はろくな先生じゃない。	
\\	あの 先生[せんせい]はろくな 先生[せんせい]じゃない。
\\	あの男はろくにあいさつもしない。	
\\	あの 男[おとこ]はろくにあいさつもしない。
\\	ナンシーがろくに手紙も書けない。	
\\	ナンシーがろくに 手紙[てがみ]も 書[か]けない。
\\	あの人達はろくに食事もしていないらしい。	
\\	あの 人達[ひとたち]はろくに 食事[しょくじ]もしていないらしい。
\\	あの犬はろくにえさを食べていない。	
\\	あの 犬[いぬ]はろくにえさを 食[た]べていない。
\\	ー就職出来るかな。 ー出来るさ。	
\\	ー 就職[しゅうしょく] 出来[でき]るかな。 ー 出来[でき]るさ。
\\	ー君は日本語、読めないよね。 ーいや、読めるさ。	
\\	ー 君[きみ]は 日本語[にほんご]、 読[よ]めないよね。 ーいや、 読[よ]めるさ。
\\	人生は長い旅のようなものさ。	
\\	人生[じんせい]は 長[なが]い 旅[たび]のようなものさ。
\\	どうして酒を飲んじゃいけないのさ。	
\\	どうして 酒[さけ]を 飲[の]んじゃいけないのさ。
\\	ー今日の日本語のテストは難しいかな。 ーいや、難しくないさ。	
\\	ー 今日[きょう]の 日本語[にほんご]のテストは 難[むずか]しいかな。 ーいや、 難[むずか]しくないさ。
\\	ーこのスーツケース、持てるかな。 ー持てるさ。	
\\	ーこのスーツケース、 持[も]てるかな。 ー 持[も]てるさ。
\\	ーお前は車を買う金なんかないよな。 ーいや、あるさ。	
\\	ーお 前[まえ]は 車[くるま]を 買[か]う 金[きん]なんかないよな。 ーいや、あるさ。
\\	ーニューヨークの一人歩きは危ないだろう。 ーいや、大丈夫さ。	
\\	ーニューヨークの 一人[ひとり] 歩[ある]きは 危[あぶ]ないだろう。 ーいや、 大丈夫[だいじょうぶ]さ。
\\	ーこの問題、君には解けないと思うよ。 ーこんなもの、やさしいさ。	
\\	ーこの 問題[もんだい]、 君[くん]には 解[と]けないと 思[おも]うよ。 ーこんなもの、やさしいさ。
\\	人生とはこんなものさ。	
\\	人生[じんせい]とはこんなものさ。
\\	仕事に失敗はつきものさ。	
\\	仕事[しごと]に 失敗[しっぱい]はつきものさ。
\\	どうしてこんな簡単なことが分からないのさ。	
\\	どうしてこんな 簡単[かんたん]なことが 分[わ]からないのさ。
\\	ー僕は大学に行くのをあきらめたよ。 ー何を言っているのさ。	
\\	ー 僕[ぼく]は 大学[だいがく]に 行[い]くのをあきらめたよ。 ー 何[なに]を 言[い]っているのさ。
\\	昨日映画を見に行ったらね、雷で急に停電しちゃってね、ひどい目に遭ったんだ。	
\\	昨日[きのう] 映画[えいが]を 見[み]に 行[い]ったらね、 雷[かみなり]で 急[きゅう]に 停電[ていでん]しちゃってね、ひどい 目[め]に 遭[あ]ったんだ。
\\	彼女は現代日本語はもちろん古典さえ読める。	
\\	彼女[かのじょ]は 現代[げんだい] 日本語[にほんご]はもちろん 古典[こてん]さえ 読[よ]める。
\\	この研究所にはアメリカからさえ研究者が来る。	
\\	この 研究所[けんきゅうじょ]にはアメリカからさえ 研究[けんきゅう] 者[しゃ]が 来[く]る。
\\	その歌は子供でさえ知っている。	
\\	その 歌[うた]は 子供[こども]でさえ 知[し]っている。
\\	ここに置いておきさえすれば後はここの人が全部やってくれます。	
\\	ここに 置[お]いておきさえすれば 後[あと]はここの 人[ひと]が 全部[ぜんぶ]やってくれます。
\\	この仕事さえ片付いたら後は楽なんです。	
\\	この 仕事[しごと]さえ 片付[かたづ]いたら 後[あと]は 楽[らく]なんです。
\\	彼はもうおかゆさえのどを通らなかった。	
\\	彼[かれ]はもうおかゆさえのどを 通[とお]らなかった。
\\	私はその写真を大人にはもちろん子供にさえ見せた。	
\\	私[わたし]はその 写真[しゃしん]を 大人[おとな]にはもちろん 子供[こども]にさえ 見[み]せた。
\\	彼は私の言うことを聞こうとさえしない。	
\\	彼[かれ]は 私[わたし]の 言[い]うことを 聞[き]こうとさえしない。
\\	山田にさえ出来たんだから君にも出来るよ。	
\\	山田[やまだ]にさえ 出来[でき]たんだから 君[きみ]にも 出来[でき]るよ。
\\	先生でさえこの問題を解くのに二時間もかかった。	
\\	先生[せんせい]でさえこの 問題[もんだい]を 解[と]くのに 二時間[にじかん]もかかった。
\\	クレジットカードさえあればいつでも入会出来ます。	
\\	クレジットカードさえあればいつでも 入会[にゅうかい] 出来[でき]ます。
\\	君さえ承知してくれたら今すぐこのプロジェクトを始めるつもりだ。	
\\	君[きみ]さえ 承知[しょうち]してくれたら 今[いま]すぐこのプロジェクトを 始[はじ]めるつもりだ。
\\	まじめに勉強してさえいれば試験は大丈夫だ。	
\\	まじめに 勉強[べんきょう]してさえいれば 試験[しけん]は 大丈夫[だいじょうぶ]だ。
\\	手を上げさえすればすぐ警官が飛んでくる。	
\\	手[て]を 上[あ]げさえすればすぐ 警官[けいかん]が 飛[と]んでくる。
\\	高くさえなければ売れるはずだ。	
\\	高[たか]くさえなければ 売[う]れるはずだ。
\\	通勤に便利でさえあればどんなところでもいいんです。	
\\	通勤[つうきん]に 便利[べんり]でさえあればどんなところでもいいんです。
\\	アメリカ人でさえあれば雇ってもらえる。	
\\	アメリカ 人[じん]でさえあれば 雇[やと]ってもらえる。
\\	家族さえ無事なら後は何もいらない。	
\\	家族[かぞく]さえ 無事[ぶじ]なら 後[あと]は 何[なに]もいらない。
\\	詳しいことは八月に上京する際に、お話し致します。	
\\	詳[くわ]しいことは 八月[はちがつ]に 上京[じょうきょう]する 際[さい]に、お 話[はな]し 致[いた]します。
\\	これは大学卒業の際に父が買ってくれたワープロです。	
\\	これは 大学[だいがく] 卒業[そつぎょう]の 際[さい]に 父[ちち]が 買[か]ってくれたワープロです。
\\	私達の結婚に際して過分のお祝い物を頂き恐縮しております。	
\\	私[わたし] 達[たち]の 結婚[けっこん]に 際[さい]して 過分[かぶん]のお 祝[いわ]い 物[ぶつ]を 頂[いただ]き 恐縮[きょうしゅく]しております。
\\	大阪に転勤する際に、妻子は東京に残した。	
\\	大阪[おおさか]に 転勤[てんきん]する 際[さい]に、 妻子[さいし]は 東京[とうきょう]に 残[のこ]した。
\\	激しい運動をする際には準備運動が必要だ。	
\\	激[はげ]しい 運動[うんどう]をする 際[さい]には 準備[じゅんび] 運動[うんどう]が 必要[ひつよう]だ。
\\	首相はサミット会談に出席した際に国際収支に関する日本側の立場を説明した。	
\\	首相[しゅしょう]はサミット 会談[かいだん]に 出席[しゅっせき]した 際[さい]に 国際[こくさい] 収支[しゅうし]に 関[かん]する 日本[にほん] 側[がわ]の 立場[たちば]を 説明[せつめい]した。
\\	出発の際には、わざわざお見送りに来ていただき、ありがとうございました。	
\\	出発[しゅっぱつ]の 際[さい]には、わざわざお 見送[みおく]りに 来[き]ていただき、ありがとうございました。
\\	旅行の際には必ずカメラを持って行きます。	
\\	旅行[りょこう]の 際[さい]には 必[かなら]ずカメラを 持[も]って 行[い]きます。
\\	近くにおいでの際には、ぜひお立ち寄り下さい。	
\\	近[ちか]くにおいでの 際[さい]には、ぜひお 立ち寄[たちよ]り 下[くだ]さい。
\\	日本の近代化の研究に際して日本の国立国会図書館を利用した。	
\\	日本[にほん]の 近代[きんだい] 化[か]の 研究[けんきゅう]に 際[さい]して 日本[にほん]の 国立[こくりつ] 国会[こっかい] 図書館[としょかん]を 利用[りよう]した。
\\	この際、私が直接社長に話してみます。	
\\	この 際[さい]、 私[わたし]が 直接[ちょくせつ] 社長[しゃちょう]に 話[はな]してみます。
\\	これは帰国の際ホストファミリーのお父さんがくれた腕時計です。	
\\	これは 帰国[きこく]の 際[さい]ホストファミリーのお 父[とう]さんがくれた 腕時計[うでどけい]です。
\\	朝起きた時熱があった。	
\\	朝[あさ] 起[お]きた 時[とき] 熱[ねつ]があった。
\\	図書館に行った時本を三冊借りた。	
\\	図書館[としょかん]に 行[い]った 時[とき] 本[ほん]を 三冊[さんさつ] 借[か]りた。
\\	研究費をもらって国立国会図書館へ行った時、本を十二冊借り出した。	
\\	研究[けんきゅう] 費[ひ]をもらって 国立[こくりつ] 国会図書館[こっかいとしょかん]へ 行[い]った 時[とき]、 本[ほん]を 十二冊[じゅうにさつ] 借り出[かりだ]した。
\\	京都にいらっしゃる折にはご連絡下さい。	
\\	京都[きょうと]にいらっしゃる 折[おり]にはご 連絡[れんらく] 下[くだ]さい。
\\	ビルは結婚する際に両親の許可を得ていなかった。	
\\	ビルは 結婚[けっこん]する 際[さい]に 両親[りょうしん]の 許可[きょか]を 得[え]ていなかった。
\\	先生がいらっしゃる場合は私がお供致します。	
\\	先生[せんせい]がいらっしゃる 場合[ばあい]は 私[わたし]がお 供[とも] 致[いた]します。
\\	雨が降った時はハイキングは中止です。	
\\	雨[あめ]が 降[ふ]った 時[とき]はハイキングは 中止[ちゅうし]です。
\\	火災の場合はエレベーターを使わないで下さい。	
\\	火災[かさい]の 場合[ばあい]はエレベーターを 使[つか]わないで 下[くだ]さい。
\\	去年パリに行った時ミッシェルに会った。	
\\	去年[きょねん]パリに 行[い]った 時[とき]ミッシェルに 会[あ]った。
\\	僕が入院した際友達が見舞に来てくれた。	
\\	僕[ぼく]が 入院[にゅういん]した 際[さい] 友達[ともだち]が 見舞[みまい]に 来[き]てくれた。
\\	ネルソンさんは日本に三年留学していたので、さすがに日本語が上手です。	
\\	ネルソンさんは 日本[にほん]に 三年[さんねん] 留学[りゅうがく]していたので、さすがに 日本語[にほんご]が 上手[じょうず]です。
\\	さすがは広島のかきだ。実においしい。	
\\	さすがは 広島[ひろしま]のかきだ。 実[じつ]においしい。
\\	さすが元ボートの選手だけあって、体格がいい。	
\\	さすが 元[もと]ボートの 選手[せんしゅ]だけあって、 体格[たいかく]がいい。
\\	さすが、パリで十年暮らしただけに、彼は大変なフランス通だ。	
\\	さすが、パリで 十年[じゅうねん] 暮[く]らしただけに、 彼[かれ]は 大変[たいへん]なフランス 通[つう]だ。
\\	難しい質問に、さすがの大先生も、困ってしまった。	
\\	難[むずか]しい 質問[しつもん]に、さすがの 大[だい] 先生[せんせい]も、 困[こま]ってしまった。
\\	昨日は十時間も運転をしたので、さすがに疲れました。	
\\	昨日[きのう]は 十時間[じゅうじかん]も 運転[うんてん]をしたので、さすがに 疲[つか]れました。
\\	運動を何もしないで、食べてばかりいたから、さすがに太ってしまった。	
\\	運動[うんどう]を 何[なに]もしないで、 食[た]べてばかりいたから、さすがに 太[ふと]ってしまった。
\\	いつも、
\\	で音楽を聞いているんですが、生のコンサートはさすがにいいですね。	
\\	いつも、 
\\	で 音楽[おんがく]を 聞[き]いているんですが、 生[なま]のコンサートはさすがにいいですね。
\\	さすが本場のイタリア料理だ。日本のイタリア料理とは味が違う。	
\\	さすが 本場[ほんば]のイタリア 料理[りょうり]だ。 日本[にほん]のイタリア 料理[りょうり]とは 味[あじ]が 違[ちが]う。
\\	さすがは科学者だ。観察が鋭い。	
\\	さすがは 科学[かがく] 者[しゃ]だ。 観察[かんさつ]が 鋭[するど]い。
\\	さすが、若い頃山に登っていただけに、今でも足が強い。	
\\	さすが、 若[わか]い 頃[ころ] 山[やま]に 登[のぼ]っていただけに、 今[いま]でも 足[あし]が 強[つよ]い。
\\	さすがの父も、今度の入院はこたえたようだ。	
\\	さすがの 父[ちち]も、 今度[こんど]の 入院[にゅういん]はこたえたようだ。
\\	さすがの先生にも、解答が分からなかった。	
\\	さすがの 先生[せんせい]にも、 解答[かいとう]が 分[わ]からなかった。
\\	彼は肉が大好物のようで、大きなステーキを二枚も平らげたのはさすがだった。	
\\	彼[かれ]は 肉[にく]が 大[だい] 好物[こうぶつ]のようで、 大[おお]きなステーキを 二枚[にまい]も 平[たい]らげたのはさすがだった。
\\	ジョンはさすがだね。ビールを十本も飲んでしまったよ。	
\\	ジョンはさすがだね。ビールを 十本[じっぽん]も 飲[の]んでしまったよ。
\\	ー山田は司法試験に一度で受かったそうだよ。 ーさすが!	ー
\\	ー 山田[やまだ]は 司法[しほう] 試験[しけん]に 一度[いちど]で 受[う]かったそうだよ。 ーさすが!
\\	ネルソンさんは日本に三年留学していたので、やはり日本語が上手です。	
\\	ネルソンさんは 日本[にほん]に 三年[さんねん] 留学[りゅうがく]していたので、やはり 日本語[にほんご]が 上手[じょうず]です。
\\	やっぱり広島のかきだ。実においしい。	
\\	やっぱり 広島[ひろしま]のかきだ。 実[じつ]においしい。
\\	今年は不況のせいで車があまり売れない。	
\\	今年[ことし]は 不況[ふきょう]のせいで 車[くるま]があまり 売[う]れない。
\\	寝不足のせいか体に力が入らない。	
\\	寝不足[ねぶそく]のせいか 体[からだ]に 力[ちから]が 入[はい]らない。
\\	私が失敗したのは彼のせいだ。	
\\	私[わたし]が 失敗[しっぱい]したのは 彼[かれ]のせいだ。
\\	昨夜飲みすぎたせいで今日は頭がふらふらする。	
\\	昨夜[ゆうべ] 飲[の]みすぎたせいで 今日[きょう]は 頭[あたま]がふらふらする。
\\	最近運動不足のせいで体重が増えた。	
\\	最近[さいきん] 運動[うんどう] 不足[ふそく]のせいで 体重[たいじゅう]が 増[ふ]えた。
\\	雨が少なかったせいで今年は米が不作だそうだ。	
\\	雨[あめ]が 少[すく]なかったせいで 今年[ことし]は 米[こめ]が 不作[ふさく]だそうだ。
\\	年のせいか最近耳が聞こえにくくなった。	
\\	年[とし]のせいか 最近[さいきん] 耳[みみ]が 聞[き]こえにくくなった。
\\	新しい土地に来て水が変わったせいか近頃おなかの調子がよくない。	
\\	新[あたら]しい 土地[とち]に 来[き]て 水[みず]が 変[か]わったせいか 近頃[ちかごろ]おなかの 調子[ちょうし]がよくない。
\\	目が悪くなったのは一日中コンピュータを使って仕事をしているせいだ。	
\\	目[め]が 悪[わる]くなったのは 一日[いちにち] 中[じゅう]コンピュータを 使[つか]って 仕事[しごと]をしているせいだ。
\\	彼はよく自分の失敗を人のせいにする。	
\\	彼[かれ]はよく 自分[じぶん]の 失敗[しっぱい]を 人[ひと]のせいにする。
\\	これは誰のせいでもない。	
\\	これは 誰[だれ]のせいでもない。
\\	彼女がいるせいで仕事があまり出来ない。	
\\	彼女[かのじょ]がいるせいで 仕事[しごと]があまり 出来[でき]ない。
\\	彼女がいるから仕事がよく出来る。	
\\	彼女[かのじょ]がいるから 仕事[しごと]がよく 出来[でき]る。
\\	彼女がいるせいか仕事がよく出来る。	
\\	彼女[かのじょ]がいるせいか 仕事[しごと]がよく 出来[でき]る。
\\	排気ガス規制が効を奏したせいか近頃空気が前よりきれいになった。	
\\	排気[はいき]ガス 規制[きせい]が 効[こう]を 奏[そう]したせいか 近頃[ちかごろ] 空気[くうき]が 前[まえ]よりきれいになった。
\\	薬を飲んだら気のせいが少し気分がよくなった。	
\\	薬[くすり]を 飲[の]んだら 気[き]のせいが 少[すこ]し 気分[きぶん]がよくなった。
\\	漢字をあまり知らないので新聞が読めない。	
\\	漢字[かんじ]をあまり 知[し]らないので 新聞[しんぶん]が 読[よ]めない。
\\	宿題がたくさんあるためにパーティーには行けない。	
\\	宿題[しゅくだい]がたくさんあるためにパーティーには 行[い]けない。
\\	誰も手伝ってくれないから一人でやらなければならない。	
\\	誰[だれ]も 手伝[てつだ]ってくれないから 一人[ひとり]でやらなければならない。
\\	安全ベルトのおかげで命が助かった。	
\\	安全[あんぜん]ベルトのおかげで 命[いのち]が 助[たす]かった。
\\	彼女が手伝ってくれたおかげで仕事が早く片付いた。	
\\	彼女[かのじょ]が 手伝[てつだ]ってくれたおかげで 仕事[しごと]が 早[はや]く 片付[かたづ]いた。
\\	学生時代によく英語を勉強しておいたおかげで今アメリカにいてもあまり困らない。	
\\	学生[がくせい] 時代[じだい]によく 英語[えいご]を 勉強[べんきょう]しておいたおかげで 今[こん]アメリカにいてもあまり 困[こま]らない。
\\	君のおかげでひどい目に遭ったよ。	
\\	君[きみ]のおかげでひどい 目[め]に 遭[あ]ったよ。
\\	お前が余計なことをしてくれたおかげで計画が台無しになった。	
\\	お 前[まえ]が 余計[よけい]なことをしてくれたおかげで 計画[けいかく]が 台無[だいな]しになった。
\\	百点とは言いませんが、せめて八十点ぐらいは取りたいです。	
\\	百[ひゃく] 点[てん]とは 言[い]いませんが、せめて 八十点[はちじってん]ぐらいは 取[と]りたいです。
\\	せめてなべだけでもあれば、ご飯が炊けるんですが。	
\\	せめてなべだけでもあれば、ご 飯[はん]が 炊[た]けるんですが。
\\	せめて両親には分かってもらいたいんですが。	
\\	せめて 両親[りょうしん]には 分[わ]かってもらいたいんですが。
\\	せめて一年に十日ぐらい、休暇が取れるといいんですが。	
\\	せめて 一年[いちねん]に 十日[とおか]ぐらい、 休暇[きゅうか]が 取[と]れるといいんですが。
\\	日本にいる間に、せめて一度ぐらいは会いに来て下さい。	
\\	日本[にほん]にいる 間[あいだ]に、せめて 一度[いちど]ぐらいは 会[あ]いに 来[き]て 下[くだ]さい。
\\	美人でなくてもいいけど、せめて、可愛らしい女の子と結婚したい。	
\\	美人[びじん]でなくてもいいけど、せめて、 可愛[かわい]らしい 女の子[おんなのこ]と 結婚[けっこん]したい。
\\	日本に行くんだから、せめて、日本語だけが勉強して来ようと思っています。	
\\	日本[にほん]に 行[い]くんだから、せめて、 日本語[にほんご]だけが 勉強[べんきょう]して 来[こ]ようと 思[おも]っています。
\\	せめてもう後一日生きていたら、父の死に目に会えたのに。	
\\	せめてもう 後[ご] 一日[いちにち] 生[い]きていたら、 父[ちち]の 死に目[しにめ]に 会[あ]えたのに。
\\	あんなに広い家でなくてもいいけど、せめてもう一部屋あるといいのに。	
\\	あんなに 広[ひろ]い 家[いえ]でなくてもいいけど、せめてもう 一部屋[ひとへや]あるといいのに。
\\	一月もご厄介になるんですから、せめて、皿洗いぐらいはさせて下さい。	
\\	一月[ひとつき]もご 厄介[やっかい]になるんですから、せめて、 皿洗[さらあら]いぐらいはさせて 下[くだ]さい。
\\	せめて日本語で日常会話ぐらいは出来るようになりたいです。	
\\	せめて 日本語[にほんご]で 日常[にちじょう] 会話[かいわ]ぐらいは 出来[でき]るようになりたいです。
\\	この大学の学生数は少なくとも三万だろう。	
\\	この 大学[だいがく]の 学生[がくせい] 数[すう]は 少[すく]なくとも 三万[さんまん]だろう。
\\	この車は少なくとも三百万円はするだろう。	
\\	この 車[くるま]は 少[すく]なくとも 三百[さんびゃく] 万[まん] 円[えん]はするだろう。
\\	今学期漢字を少なくとも千字は覚えたはずだ。	
\\	今[こん] 学期[がっき] 漢字[かんじ]を 少[すく]なくとも 千[せん] 字[じ]は 覚[おぼ]えたはずだ。
\\	大阪に着き次第、お電話を差し上げます。	
\\	大阪[おおさか]に 着[つ]き 次第[しだい]、お 電話[でんわ]を 差し上[さしあ]げます。
\\	原稿は完成次第お送り致します。	
\\	原稿[げんこう]は 完成[かんせい] 次第[しだい]お 送[おく]り 致[いた]します。
\\	世の中で成功するかしないかはあなたの努力次第ですよ。	
\\	世の中[よのなか]で 成功[せいこう]するかしないかはあなたの 努力[どりょく] 次第[しだい]ですよ。
\\	夫は仕事が終わり次第、毎日まっすぐ帰宅します。	
\\	夫[おっと]は 仕事[しごと]が 終[お]わり 次第[しだい]、 毎日[まいにち]まっすぐ 帰宅[きたく]します。
\\	空港から電話があり次第、お迎えに参ります。	
\\	空港[くうこう]から 電話[でんわ]があり 次第[しだい]、お 迎[むか]えに 参[まい]ります。
\\	山岸先生がいらっしゃり次第、会議を始めたいと思います。	
\\	山岸[やまぎし] 先生[せんせい]がいらっしゃり 次第[しだい]、 会議[かいぎ]を 始[はじ]めたいと 思[おも]います。
\\	東大を卒業次第、京大の大学院に入学するつもりだ。	
\\	東大[とうだい]を 卒業[そつぎょう] 次第[しだい]、 京大[きょうだい]の 大学院[だいがくいん]に 入学[にゅうがく]するつもりだ。
\\	お尋ねの件は、判明次第、ご連絡申し上げます。	
\\	お 尋[たず]ねの 件[けん]は、 判明[はんめい] 次第[しだい]、ご 連絡[れんらく] 申し上[もうしあ]げます。
\\	レセプションが終了次第、晩餐会を始めます。	
\\	レセプションが 終了[しゅうりょう] 次第[しだい]、 晩餐[ばんさん] 会[かい]を 始[はじ]めます。
\\	就職はコネ次第ですよ。	
\\	就職[しゅうしょく]はコネ 次第[しだい]ですよ。
\\	「地獄の沙汰も金次第だ」ということわざを知っていますか。	
\\	地獄[じごく]の 沙汰[さた]も 金[きん] 次第[しだい]だ」ということわざを 知[し]っていますか。
\\	ご注文次第で、どんな物でも料理致します。	
\\	ご 注文[ちゅうもん] 次第[しだい]で、どんな 物[もの]でも 料理[りょうり] 致[いた]します。
\\	条件次第では、その仕事を引き受けたいと思います。	
\\	条件[じょうけん] 次第[しだい]では、その 仕事[しごと]を 引き受[ひきう]けたいと 思[おも]います。
\\	夫は仕事が終わるとすぐ、毎日まっすぐ帰宅しました。	
\\	夫[おっと]は 仕事[しごと]が 終[お]わるとすぐ、 毎日[まいにち]まっすぐ 帰宅[きたく]しました。
\\	空港から電話があってすぐ、迎えに行った。	
\\	空港[くうこう]から 電話[でんわ]があってすぐ、 迎[むか]えに 行[い]った。
\\	家に帰ったらまもなく、地震が起きた。	
\\	家[いえ]に 帰[かえ]ったらまもなく、 地震[じしん]が 起[お]きた。
\\	彼は日本に行ったらすぐ病気になった。	
\\	彼[かれ]は 日本[にほん]に 行[い]ったらすぐ 病気[びょうき]になった。
\\	その怪我は不注意による。	
\\	その 怪我[けが]は 不注意[ふちゅうい]による。
\\	彼は政治的手腕によって、世の中で成功した。	
\\	彼[かれ]は 政治[せいじ] 的[てき] 手腕[しゅわん]によって、 世の中[よのなか]で 成功[せいこう]した。
\\	ラジオによると明日は雪だそうだ。	
\\	ラジオによると 明日[あした]は 雪[ゆき]だそうだ。
\\	最近のテープレコーダーは小さくて軽く、しかも音がすばらしくよくなった。	
\\	最近[さいきん]のテープレコーダーは 小[ちい]さくて 軽[かる]く、しかも 音[おと]がすばらしくよくなった。
\\	彼は家を買った。しかも大きい庭付きのをだ。	
\\	彼[かれ]は 家[いえ]を 買[か]った。しかも 大[おお]きい 庭付[にわつ]きのをだ。
\\	彼はいつも夜遅くまで勉強し、しかも朝は誰よりも早く起きる。	
\\	彼[かれ]はいつも 夜[よる] 遅[おそ]くまで 勉強[べんきょう]し、しかも 朝[あさ]は 誰[だれ]よりも 早[はや]く 起[お]きる。
\\	彼女は美人でしかも頭がいい。男子学生が夢中になるのも無理はない。	
\\	彼女[かのじょ]は 美人[びじん]でしかも 頭[あたま]がいい。 男子[だんし] 学生[がくせい]が 夢中[むちゅう]になるのも 無理[むり]はない。
\\	日本語は語彙が英語と全然違うし文法も難しい。しかも、日本語には漢字がある。	
\\	日本語[にほんご]は 語彙[ごい]が 英語[えいご]と 全然[ぜんぜん] 違[ちが]うし 文法[ぶんぽう]も 難[むずか]しい。しかも、 日本語[にほんご]には 漢字[かんじ]がある。
\\	あのレストランの定食はおいしくて安い。しかも、ボリュームがあるから学生や若いサラリーマンに人気がある。	
\\	あのレストランの 定食[ていしょく]はおいしくて 安[やす]い。しかも、ボリュームがあるから 学生[がくせい]や 若[わか]いサラリーマンに 人気[にんき]がある。
\\	彼はこの大雨の中を出かけていった。しかも、傘も持たずに。	
\\	彼[かれ]はこの 大雨[おおあめ]の 中[なか]を 出[で]かけていった。しかも、 傘[かさ]も 持[も]たずに。
\\	彼は試験に遅刻してしまった。しかも、三十分も。	
\\	彼[かれ]は 試験[しけん]に 遅刻[ちこく]してしまった。しかも、三十分[さんじゅっぷん]も。
\\	彼女は英語のほかに四か国語を話す。しかも、すべて母国語話者のように。	
\\	彼女[かのじょ]は 英語[えいご]のほかに 四か国[よっかこく] 語[ご]を 話[はな]す。しかも、すべて 母国[ぼこく] 語[ご] 話者[わしゃ]のように。
\\	私は風邪を引いてしまった。しかも、大事な試験の前日にだ。	
\\	私[わたし]は 風邪[かぜ]を 引[ひ]いてしまった。しかも、 大事[だいじ]な 試験[しけん]の 前日[ぜんじつ]にだ。
\\	彼はどんなつらい仕事でも黙ってこなし、しかも、給料についてほかの者のように不平を言わなかった。	
\\	彼[かれ]はどんなつらい 仕事[しごと]でも 黙[だま]ってこなし、しかも、 給料[きゅうりょう]についてほかの 者[もの]のように 不平[ふへい]を 言[い]わなかった。
\\	彼女は何をやっても上手で、しかも、その才能をひけらかさない。	
\\	彼女[かのじょ]は 何[なに]をやっても 上手[じょうず]で、しかも、その 才能[さいのう]をひけらかさない。
\\	予算が不足している。したがってこの計画は実行できない。	
\\	予算[よさん]が 不足[ふそく]している。したがってこの 計画[けいかく]は 実行[じっこう]できない。
\\	今週は重役達は朝から晩まで会議で忙しい。したがって好きなゴルフをする暇もない。	
\\	今週[こんしゅう]は 重役[じゅうやく] 達[たち]は 朝[あさ]から 晩[ばん]まで 会議[かいぎ]で 忙[いそが]しい。したがって 好[す]きなゴルフをする 暇[ひま]もない。
\\	運動をすると、エネルギーを使う。したがって太らない。	
\\	運動[うんどう]をすると、エネルギーを 使[つか]う。したがって 太[ふと]らない。
\\	日本の車は故障が少ない。したがってよく売れる。	
\\	日本[にほん]の 車[くるま]は 故障[こしょう]が 少[すく]ない。したがってよく 売[う]れる。
\\	彼女は日本に五年も住んでいた。したがって日本のことをよく知っている。	
\\	彼女[かのじょ]は 日本[にほん]に 五年[ごねん]も 住[す]んでいた。したがって 日本[にほん]のことをよく 知[し]っている。
\\	日本では父親はあまり家にいない。したがって母親が子供の教育をすることになる。	
\\	日本[にほん]では 父親[ちちおや]はあまり 家[いえ]にいない。したがって 母親[ははおや]が 子供[こども]の 教育[きょういく]をすることになる。
\\	過半数の人が賛成した。だから私も賛成した。	
\\	過半数[かはんすう]の 人[ひと]が 賛成[さんせい]した。だから 私[わたし]も 賛成[さんせい]した。
\\	この本の書評はよかった。だから私は買って読んだ。	
\\	この 本[ほん]の 書評[しょひょう]はよかった。だから 私[わたし]は 買[か]って 読[よ]んだ。
\\	今日は天気がよかった。だから私はハイキングに出かけた。	
\\	今日[きょう]は 天気[てんき]がよかった。だから 私[わたし]はハイキングに 出[で]かけた。
\\	ー魚は体にいいそうですよ。 ーそうかと言って、毎日魚ばかりを食べているわけにもいかないでしょう。	
\\	ー 魚[さかな]は 体[からだ]にいいそうですよ。 ーそうかと 言[い]って、 毎日[まいにち] 魚[さかな]ばかりを 食[た]べているわけにもいかないでしょう。
\\	アメリカの方が土地も家も安いが、そうかと言って、簡単に移住するわけにもいかない。	
\\	アメリカの 方[ほう]が 土地[とち]も 家[いえ]も 安[やす]いが、そうかと 言[い]って、 簡単[かんたん]に 移住[いじゅう]するわけにもいかない。
\\	ー漫画は面白いですね。 ーそうかと言って、漫画ばかり読んでいるわけにもいきませんよ。	ー
\\	ー 漫画[まんが]は 面白[おもしろ]いですね。 ーそうかと 言[い]って、 漫画[まんが]ばかり 読[よ]んでいるわけにもいきませんよ。
\\	ー手ぶらで来て下さいと言っていましたよ。 ーそうかと言って、何も持っていかないわけにもいかないでしょうね。	
\\	ー 手[て]ぶらで 来[き]て 下[くだ]さいと 言[い]っていましたよ。 ーそうかと 言[い]って、 何[なに]も 持[も]っていかないわけにもいかないでしょうね。
\\	ー明日の試験はとてもやさしいそうですよ。 ーそうかと言って、何も勉強しないわけにもいかないでしょう。	
\\	ー 明日[あした]の 試験[しけん]はとてもやさしいそうですよ。 ーそうかと 言[い]って、 何[なに]も 勉強[べんきょう]しないわけにもいかないでしょう。
\\	ー来週は連休だから、遊べますね。 ーそうかと言って、ゴルフばかりしているわけにもいかないね。	
\\	ー 来週[らいしゅう]は 連休[れんきゅう]だから、 遊[あそ]べますね。 ーそうかと 言[い]って、ゴルフばかりしているわけにもいかないね。
\\	ーたばこは体に悪いですよ。 ーそうかと言って、たばこに代わる物が見つからないし。	
\\	ーたばこは 体[からだ]に 悪[わる]いですよ。 ーそうかと 言[い]って、たばこに 代[か]わる 物[もの]が 見[み]つからないし。
\\	ー毎日三十分くらい運動するといいそうですよ。 ーそうかと言って、それだけの時間は作れないしね。	
\\	ー 毎日[まいにち] 三十分[さんじゅっぷん]くらい 運動[うんどう]するといいそうですよ。 ーそうかと 言[い]って、それだけの 時間[じかん]は 作[つく]れないしね。
\\	漢字をもっと覚えなければならないのは分かっているが、そうかと言って、漢字ばかり覚えているわけにもいかない。	
\\	漢字[かんじ]をもっと 覚[おぼ]えなければならないのは 分[わ]かっているが、そうかと 言[い]って、 漢字[かんじ]ばかり 覚[おぼ]えているわけにもいかない。
\\	妻子と別れるのはつらいが、そうかと言って、この海外出張を断るわけにもいかない。	
\\	妻子[さいし]と 別[わか]れるのはつらいが、そうかと 言[い]って、この 海外[かいがい] 出張[しゅっちょう]を 断[ことわ]るわけにもいかない。
\\	この町は気に入らないんだが、そうかと言って、引っ越すには金がかかるし。	
\\	この 町[まち]は 気に入[きにい]らないんだが、そうかと 言[い]って、 引っ越[ひっこ]すには 金[きん]がかかるし。
\\	僕らは忙しいけど、そうかと言って、彼一人に任せておけるかい?	
\\	僕[ぼく]らは 忙[いそが]しいけど、そうかと 言[い]って、 彼[かれ] 一人[ひとり]に 任[まか]せておけるかい?
\\	ーこのアパートは広くていいね。 ーしかし、高いんだよ。	
\\	ーこのアパートは 広[ひろ]くていいね。 ーしかし、 高[たか]いんだよ。
\\	ー酒は体によくないよ。 ーけれど、適当に飲めば大丈夫だよ。	
\\	ー 酒[さけ]は 体[からだ]によくないよ。 ーけれど、 適当[てきとう]に 飲[の]めば 大丈夫[だいじょうぶ]だよ。
\\	漢字は覚えるのが大変だ。しかし、一度覚えると便利だ。	
\\	漢字[かんじ]は 覚[おぼ]えるのが 大変[たいへん]だ。しかし、 一度[いちど] 覚[おぼ]えると 便利[べんり]だ。
\\	友達が夕食は簡単なものでいいと言った。そこで、ピザを注文した。	
\\	友達[ともだち]が 夕食[ゆうしょく]は 簡単[かんたん]なものでいいと 言[い]った。そこで、ピザを 注文[ちゅうもん]した。
\\	会議室が使えなくなった。そこで、食堂で打ち合わせをすることにした。	
\\	会議[かいぎ] 室[しつ]が 使[つか]えなくなった。そこで、 食堂[しょくどう]で 打ち合[うちあ]わせをすることにした。
\\	荷物がたくさんあるのに雨が降り出した。そこで、タクシーで行くことにした。	
\\	荷物[にもつ]がたくさんあるのに 雨[あめ]が 降り出[ふりだ]した。そこで、タクシーで 行[い]くことにした。
\\	リーダーの山田さんが来られなくなった。そこで、ミーティングは延期せざるをえなくなった。	
\\	リーダーの 山田[やまだ]さんが 来[こ]られなくなった。そこで、ミーティングは 延期[えんき]せざるをえなくなった。
\\	試験の日に怪我をして学校へ行けなくなった。そこで、先生に電話して別の日に受けられるか聞いてみた。	
\\	試験[しけん]の 日[ひ]に 怪我[けが]をして 学校[がっこう]へ 行[い]けなくなった。そこで、 先生[せんせい]に 電話[でんわ]して 別[べつ]の 日[ひ]に 受[う]けられるか 聞[き]いてみた。
\\	彼はすべて私が悪かったような言い方をした。そこで、私も黙っていられなくなって、言い返した。	
\\	彼[かれ]はすべて 私[わたし]が 悪[わる]かったような 言い方[いいかた]をした。そこで、 私[わたし]も 黙[だま]っていられなくなって、 言い返[いいかえ]した。
\\	息子が来年大学なんですが英語が苦手で困っているんです。そこで、お願いがあるんですが、息子に英語を教えてやっていただけませんか。	
\\	息子[むすこ]が 来年[らいねん] 大学[だいがく]なんですが 英語[えいご]が 苦手[にがて]で 困[こま]っているんです。そこで、お 願[ねが]いがあるんですが、 息子[むすこ]に 英語[えいご]を 教[おし]えてやっていただけませんか。
\\	サンフランシスコで大地震があった。そのため建物がたくさん壊れた。	
\\	サンフランシスコで 大[だい] 地震[じしん]があった。そのため 建物[たてもの]がたくさん 壊[こわ]れた。
\\	昨夜安いウイスキーをたくさん飲まされた。それで今日は頭が痛くて起きられない。	
\\	昨夜[さくや] 安[やす]いウイスキーをたくさん 飲[の]まされた。それで 今日[きょう]は 頭[あたま]が 痛[いた]くて 起[お]きられない。
\\	明日森田さんのうちでパーティーがあって昔のクラスメートがたくさん来るという。そこで私も行くことにした。	
\\	明日[あす] 森田[もりた]さんのうちでパーティーがあって 昔[むかし]のクラスメートがたくさん 来[く]るという。そこで 私[わたし]も 行[い]くことにした。
\\	明日森田さんのうちでパーティーがあるという。それで、私も行くことにした。	
\\	明日[あす] 森田[もりた]さんのうちでパーティーがあるという。それで、 私[わたし]も 行[い]くことにした。
\\	その毛皮のコートはセールで半額だった。そこで、思いきって買った。	
\\	その 毛皮[けがわ]のコートはセールで 半額[はんがく]だった。そこで、 思[おも]いきって 買[か]った。
\\	その毛皮のコートはセールで半額だった。それで、私も買いたくなってきた。	
\\	その 毛皮[けがわ]のコートはセールで 半額[はんがく]だった。それで、 私[わたし]も 買[か]いたくなってきた。
\\	その毛皮のコートはセールで半額だった。それで、私にも買うことが出来た。	
\\	その 毛皮[けがわ]のコートはセールで 半額[はんがく]だった。それで、 私[わたし]にも 買[か]うことが 出来[でき]た。
\\	明日森田さんのうちでパーティーがあって昔のクラスメートがたくさん来ます。だからあなたもぜひ来て下さい。	
\\	明日[あした] 森田[もりた]さんのうちでパーティーがあって 昔[むかし]のクラスメートがたくさん 来[き]ます。だからあなたもぜひ 来[き]て 下[くだ]さい。
\\	サンフランシスコで大地震があった。それで建物がたくさん壊れた。	
\\	サンフランシスコで 大[だい] 地震[じしん]があった。それで 建物[たてもの]がたくさん 壊[こわ]れた。
\\	ー昨日はちょっと大阪で用事がありましたね。 ーああ、それでいらっしゃらなかったんですね。	ー
\\	ー 昨日[きのう]はちょっと 大阪[おおさか]で 用事[ようじ]がありましたね。 ーああ、それでいらっしゃらなかったんですね。
\\	私が手を振ったら、そこで拍手をして下さい。	
\\	私[わたし]が 手[て]を 振[ふ]ったら、そこで 拍手[はくしゅ]をして 下[くだ]さい。
\\	彼は飲み物を勧めるかもしれません。そこで遠慮をしてはいけません。	
\\	彼[かれ]は 飲み物[のみもの]を 勧[すす]めるかもしれません。そこで 遠慮[えんりょ]をしてはいけません。
\\	このランプがついたら、そこで充電を始めて下さい。	
\\	このランプがついたら、そこで 充電[じゅうでん]を 始[はじ]めて 下[くだ]さい。
\\	多分受付が身分証明書を要求するでしょう。そこでこのカードを見せなさい。	
\\	多分[たぶん] 受付[うけつけ]が 身分[みぶん] 証明[しょうめい] 書[しょ]を 要求[ようきゅう]するでしょう。そこでこのカードを 見[み]せなさい。
\\	一時間経ったら私が「やめ!」と言いますから、そこで書くのをやめて鉛筆を置いて下さい。	
\\	一時間[いちじかん] 経[た]ったら 私[わたし]が「やめ!」と 言[い]いますから、そこで 書[か]くのをやめて 鉛筆[えんぴつ]を 置[お]いて 下[くだ]さい。
\\	私が合図をしたら、そこで皆さんの立ち願います。	
\\	私[わたし]が 合図[あいず]をしたら、そこで 皆[みな]さんの 立[た]ち 願[ねが]います。
\\	酔ったなと思ったら、そこで飲むのをやめるべきです。	
\\	酔[よ]ったなと 思[おも]ったら、そこで 飲[の]むのをやめるべきです。
\\	小さい子供は時にとんでもないいたずらをする。そこで親は叱らないといけない。	
\\	小[ちい]さい 子供[こども]は 時[とき]にとんでもないいたずらをする。そこで 親[おや]は 叱[しか]らないといけない。
\\	矢野先生が学生を叱った。そこでその学生が素直に謝ればよかったのだが、彼は口答えをした。	
\\	矢野[やの] 先生[せんせい]が 学生[がくせい]を 叱[しか]った。そこでその 学生[がくせい]が 素直[すなお]に 謝[あやま]ればよかったのだが、 彼[かれ]は 口答[くちごた]えをした。
\\	トムはフットボールの選手としては小さいが、そこを脚力で補っている。	
\\	トムはフットボールの 選手[せんしゅ]としては 小[ちい]さいが、そこを 脚力[きゃくりょく]で 補[おぎな]っている。
\\	ーこれ以上金を貸すわけにはいかないよ。 ーそこを何とかして下さいませんか。	ー
\\	ーこれ 以上[いじょう] 金[きん]を 貸[か]すわけにはいかないよ。 ーそこを 何[なん]とかして 下[くだ]さいませんか。
\\	彼は頭はそれほどよくないが、そこを努力で何とかやっている。	
\\	彼[かれ]は 頭[あたま]はそれほどよくないが、そこを 努力[どりょく]で 何[なん]とかやっている。
\\	その女優はもう年齢を隠すことは出来ないが、そこを芸で何とか持たせている。	
\\	その 女優[じょゆう]はもう 年齢[ねんれい]を 隠[かく]すことは 出来[でき]ないが、そこを 芸[げい]で 何[なん]とか 持[も]たせている。
\\	お忙しいとは存じますが、そこを何とか出席していただけないでしょうか。	
\\	お 忙[いそが]しいとは 存[ぞん]じますが、そこを 何[なん]とか 出席[しゅっせき]していただけないでしょうか。
\\	ーこんな成績じゃ、君、卒業できないよ。 ーそこを、先生、どうにかしていただけないでしょうか。	
\\	ーこんな 成績[せいせき]じゃ、 君[きみ]、 卒業[そつぎょう]できないよ。 ーそこを、 先生[せんせい]、どうにかしていただけないでしょうか。
\\	ーこんな企画に予算を付けるわけにはいかないな。 ー部長、そこを何とか一つお願いします。	
\\	ーこんな 企画[きかく]に 予算[よさん]を 付[つ]けるわけにはいかないな。 ー 部長[ぶちょう]、そこを 何[なん]とか 一[ひと]つお 願[ねが]いします。
\\	冷たい雨の中を傘もささないで歩いていたら、風邪を引きそうになった。	
\\	冷[つめ]たい 雨[あめ]の 中[なか]を 傘[かさ]もささないで 歩[ある]いていたら、 風邪[かぜ]を 引[ひ]きそうになった。
\\	私が子供の時、プールで危うくおぼれそうになったことがある。	
\\	私[わたし]が 子供[こども]の 時[とき]、プールで 危[あや]うくおぼれそうになったことがある。
\\	昨日もうちょっとで車にひかれそうになった。	
\\	昨日[きのう]もうちょっとで 車[くるま]にひかれそうになった。
\\	地震で僕の家はつぶれそうになった。	
\\	地震[じしん]で 僕[ぼく]の 家[いえ]はつぶれそうになった。
\\	庭の古い松ノ木が台風で倒れそうになった。	
\\	庭[にわ]の 古[ふる]い 松ノ木[まつのき]が 台風[たいふう]で 倒[たお]れそうになった。
\\	今日混んだ電車の中で、財布をすられそうになった。	
\\	今日[きょう] 混[こ]んだ 電車[でんしゃ]の 中[なか]で、 財布[さいふ]をすられそうになった。
\\	数学の試験で最後の問題がもう少しで解けそうになった時に、時間切れになってしまった。	
\\	数学[すうがく]の 試験[しけん]で 最後[さいご]の 問題[もんだい]がもう 少[すこ]しで 解[と]けそうになった 時[とき]に、 時間切[じかんぎ]れになってしまった。
\\	やっと日本へ行けそうになった時に、母が病気になってしまった。	
\\	やっと 日本[にほん]へ 行[い]けそうになった 時[とき]に、 母[はは]が 病気[びょうき]になってしまった。
\\	私が昼ご飯を食べようとした時、友達から電話がかかってきた。	
\\	私[わたし]が 昼[ひる]ご 飯[はん]を 食[た]べようとした 時[とき]、 友達[ともだち]から 電話[でんわ]がかかってきた。
\\	僕が大学を出ようとした時、母が亡くなりました。	
\\	僕[ぼく]が 大学[だいがく]を 出[で]ようとした 時[とき]、 母[はは]が 亡[な]くなりました。
\\	私は危うくその毒まんじゅうを食べそうになった。	
\\	私[わたし]は 危[あや]うくその 毒[どく]まんじゅうを 食[た]べそうになった。
\\	僕はもうちょっとでその偽札で買い物をしそうになった。	
\\	僕[ぼく]はもうちょっとでその 偽札[にせさつ]で 買い物[かいもの]をしそうになった。
\\	小さい子供が酒を飲みそうになったので、母親は急いで取り上げた。	
\\	小[ちい]さい 子供[こども]が 酒[さけ]を 飲[の]みそうになったので、 母親[ははおや]は 急[いそ]いで 取り上[とりあ]げた。
\\	山田が会社を辞めそうになったので、同僚達は懸命に引き留めた。	
\\	山田[やまだ]が 会社[かいしゃ]を 辞[や]めそうになったので、 同僚[どうりょう] 達[たち]は 懸命[けんめい]に 引き留[ひきと]めた。
\\	長い論文をもう少しで書き上げるところだった。	
\\	長[なが]い 論文[ろんぶん]をもう 少[すこ]しで 書き上[かきあ]げるところだった。
\\	もうちょっとで家に帰り着くところだったが、あいにくと雨が降ってきてしまった。	
\\	もうちょっとで 家[いえ]に 帰[かえ]り 着[つ]くところだったが、あいにくと 雨[あめ]が 降[ふ]ってきてしまった。
\\	もう少しで出かけるところに友達がやって来た。	
\\	もう 少[すこ]しで 出[で]かけるところに 友達[ともだち]がやって 来[き]た。
\\	彼は一緒にアパートを探してくれて、その上、引っ越しまで手伝ってくれた。	
\\	彼[かれ]は 一緒[いっしょ]にアパートを 探[さが]してくれて、その 上[うえ]、 引っ越[ひっこ]しまで 手伝[てつだ]ってくれた。
\\	この仕事は給料がいいし、その上、時々外国に行く機会もある。	
\\	この 仕事[しごと]は 給料[きゅうりょう]がいいし、その 上[うえ]、 時々[ときどき] 外国[がいこく]に 行[い]く 機会[きかい]もある。
\\	バナナは安い。その上、栄養も豊富だ。	
\\	バナナは 安[やす]い。その 上[うえ]、 栄養[えいよう]も 豊富[ほうふ]だ。
\\	いつでも使えて、その上、無料だ。	
\\	いつでも 使[つか]えて、その 上[うえ]、 無料[むりょう]だ。
\\	不便で、その上、家賃が高い。	
\\	不便[ふべん]で、その 上[うえ]、 家賃[やちん]が 高[たか]い。
\\	純金製で、その上、ダイヤがはまっている。	
\\	純金[じゅんきん] 製[せい]で、その 上[うえ]、ダイヤがはまっている。
\\	ほめられ、その上、ほうびまでもらった。	
\\	ほめられ、その 上[うえ]、ほうびまでもらった。
\\	話が面白く、その上、読みやすい。	
\\	話[はなし]が 面白[おもしろ]く、その 上[うえ]、 読[よ]みやすい。
\\	その仕事はつまらないし、その上、給料が悪い。	
\\	その 仕事[しごと]はつまらないし、その 上[うえ]、 給料[きゅうりょう]が 悪[わる]い。
\\	その老人が体が不自由だ。その上、耳が遠い。	
\\	その 老人[ろうじん]が 体[からだ]が 不自由[ふじゆう]だ。その 上[うえ]、 耳[みみ]が 遠[とお]い。
\\	彼女は美人で、その上、気立てがいいから、みんなに好かれている。	
\\	彼女[かのじょ]は 美人[びじん]で、その 上[うえ]、 気立[きだ]てがいいから、みんなに 好[す]かれている。
\\	先生のうちに就職の相談に行ったら、いろいろアドバイスをいただいて、その上、夕食までごちそうになった。	
\\	先生[せんせい]のうちに 就職[しゅうしょく]の 相談[そうだん]に 行[い]ったら、いろいろアドバイスをいただいて、その 上[うえ]、 夕食[ゆうしょく]までごちそうになった。
\\	ジョンのうちに遊びに行ったら、家のペンキ塗りを手伝わされ、その上、芝刈りまでさせられた。	
\\	ジョンのうちに 遊[あそ]びに 行[い]ったら、 家[いえ]のペンキ 塗[ぬ]りを 手伝[てつだ]わされ、その 上[うえ]、 芝[しば] 刈[か]りまでさせられた。
\\	このアパートは通勤に便利で家賃も安く、その上、駐車場まである。	
\\	このアパートは 通勤[つうきん]に 便利[べんり]で 家賃[やちん]も 安[やす]く、その 上[うえ]、 駐車[ちゅうしゃ] 場[じょう]まである。
\\	この学校はいい先生が少なく設備もよくない。その上、生徒の非行も多い。	
\\	この 学校[がっこう]はいい 先生[せんせい]が 少[すく]なく 設備[せつび]もよくない。その 上[うえ]、 生徒[せいと]の 非行[ひこう]も 多[おお]い。
\\	両親は二人の結婚に反対している。それでも二人は結婚するつもりらしい。	
\\	両親[りょうしん]は 二人[ふたり]の 結婚[けっこん]に 反対[はんたい]している。それでも 二人[ふたり]は 結婚[けっこん]するつもりらしい。
\\	悪い天気だった。しかしそれでも私達は出かけた。	
\\	悪[わる]い 天気[てんき]だった。しかしそれでも 私[わたし] 達[たち]は 出[で]かけた。
\\	暑いので扇風機をつけたが、それでもまだ暑かった。	
\\	暑[あつ]いので 扇風機[せんぷうき]をつけたが、それでもまだ 暑[あつ]かった。
\\	父がひどい風邪を引いていた。しかしそれでも旅行に出かけた。	
\\	父[ちち]がひどい 風邪[かぜ]を 引[ひ]いていた。しかしそれでも 旅行[りょこう]に 出[で]かけた。
\\	この本はとても高いのに、それでもいい本だから、よく売れている。	
\\	この 本[ほん]はとても 高[たか]いのに、それでもいい 本[ほん]だから、よく 売[う]れている。
\\	彼はとても忙しいのに、それでもジョギングを一日も欠かさない。	
\\	彼[かれ]はとても 忙[いそが]しいのに、それでもジョギングを 一日[いちにち]も 欠[か]かさない。
\\	ーこのオーバー、高いねえ。 ーそれでも買うつもり?	ー
\\	ーこのオーバー、 高[たか]いねえ。 ーそれでも 買[か]うつもり?
\\	今日は仕事がある。でも明日は休みだ。	
\\	今日[きょう]は 仕事[しごと]がある。でも 明日[あした]は 休[やす]みだ。
\\	このアパートは広いです。けれどもとても高いんです。	
\\	このアパートは 広[ひろ]いです。けれどもとても 高[たか]いんです。
\\	僕はよく運動をする。しかし弟は全然しない。	
\\	僕[ぼく]はよく 運動[うんどう]をする。しかし 弟[おとうと]は 全然[ぜんぜん]しない。
\\	両親が二人の結婚に反対していても、二人は結婚するつもりらしい。	
\\	両親[りょうしん]が 二人[ふたり]の 結婚[けっこん]に 反対[はんたい]していても、二人[ふたり]は 結婚[けっこん]するつもりらしい。
\\	何を食べてもおいしい。	
\\	何[なに]を 食[た]べてもおいしい。
\\	彼は誰に対しても同じ態度を取る。	
\\	彼[かれ]は 誰[だれ]に 対[たい]しても 同[おな]じ 態度[たいど]を 取[と]る。
\\	スミスさんは漢字が書けない。それどころか、平仮名も満足に書けない。	
\\	スミスさんは 漢字[かんじ]が 書[か]けない。それどころか、 平仮名[ひらがな]も 満足[まんぞく]に 書[か]けない。
\\	ー今晩パーティーに行かない? ーそれどころか、親父が入院したんだ。	
\\	ー 今晩[こんばん]パーティーに 行[い]かない? ーそれどころか、 親父[おやじ]が 入院[にゅういん]したんだ。
\\	暑くない。それどころか寒い。	
\\	暑[あつ]くない。それどころか 寒[さむ]い。
\\	ー疲れたね。 ーそれどころじゃないよ。死にそうなんだよ。	
\\	ー 疲[つか]れたね。 ーそれどころじゃないよ。 死[し]にそうなんだよ。
\\	あの人が若くはありませんよ。それどころか、お孫さんが三人もいるんですよ。	
\\	あの 人[ひと]が 若[わか]くはありませんよ。それどころか、お 孫[まご]さんが 三人[さんにん]もいるんですよ。
\\	お金は一銭もありません。それどころか、借金で苦しんでいるんです。	
\\	お 金[かね]は 一銭[いっせん]もありません。それどころか、 借金[しゃっきん]で 苦[くる]しんでいるんです。
\\	ジムは日本語の会話が出来ない。それどころか、簡単なあいさつも日本語で出来ない。	
\\	ジムは 日本語[にほんご]の 会話[かいわ]が 出来[でき]ない。それどころか、 簡単[かんたん]なあいさつも 日本語[にほんご]で 出来[でき]ない。
\\	ー風邪を引いたんだって? ーそれどころか、肺炎になって、入院していたんだ。	
\\	ー 風邪[かぜ]を 引[ひ]いたんだって? ーそれどころか、 肺炎[はいえん]になって、 入院[にゅういん]していたんだ。
\\	ー映画を見に行こうか? ーそれどころじゃないよ。今晩は徹夜しなければならないほど忙しいんだ。	
\\	ー 映画[えいが]を 見[み]に 行[い]こうか? ーそれどころじゃないよ。 今晩[こんばん]は 徹夜[てつや]しなければならないほど 忙[いそが]しいんだ。
\\	ー和子、手伝ってくれる? ーそれどころじゃないのよ、今宿題で忙しくて。	
\\	ー 和子[かずこ]、 手伝[てつだ]ってくれる? ーそれどころじゃないのよ、 今[こん] 宿題[しゅくだい]で 忙[いそが]しくて。
\\	クラークさんは刺身が食べられる。それどころか納豆まで食べられる。	
\\	クラークさんは 刺身[さしみ]が 食[た]べられる。それどころか 納豆[なっとう]まで 食[た]べられる。
\\	ー奨学金はもらえそうですか。 ーいえ、それが難しそうなんです。	
\\	ー 奨学[しょうがく] 金[きん]はもらえそうですか。 ーいえ、それが 難[むずか]しそうなんです。
\\	ー試験、どうだった。 ーそれが、だめだったんだ。	
\\	ー 試験[しけん]、どうだった。 ーそれが、だめだったんだ。
\\	ー来年は日本で働くんですね。 ーいや、それが行けなくなってしまったんです。	ー
\\	ー 来年[らいねん]は 日本[にほん]で 働[はたら]くんですね。 ーいや、それが 行[い]けなくなってしまったんです。
\\	ー彼女に会って話したんだろう。 ーそれが、会ったことは会ったんだけど、顔を見たら何も言えなくなってしまって。	
\\	ー 彼女[かのじょ]に 会[あ]って 話[はな]したんだろう。 ーそれが、 会[あ]ったことは 会[あ]ったんだけど、 顔[かお]を 見[み]たら 何[なに]も 言[い]えなくなってしまって。
\\	ー試験、だめだったんだろう。 ーいや、それが、通ったんだよ。	
\\	ー 試験[しけん]、だめだったんだろう。 ーいや、それが、 通[とお]ったんだよ。
\\	ー彼女に会いたいでしょう。 ーそれは、会いたいですよ。	
\\	ー 彼女[かのじょ]に 会[あ]いたいでしょう。 ーそれは、 会[あ]いたいですよ。
\\	英語で、それも早口で話されたので全然分からなかった。	
\\	英語[えいご]で、それも 早口[はやくち]で 話[はな]されたので 全然[ぜんぜん] 分[わ]からなかった。
\\	彼女は銀座の、それも一流の店でばかり買い物をする。	
\\	彼女[かのじょ]は 銀座[ぎんざ]の、それも 一流[いちりゅう]の 店[みせ]でばかり 買い物[かいもの]をする。
\\	子供が五人もいて、それもいたずら盛りの男の子ばかりなので、野口さんの奥さんは毎日大変忙しいです。	
\\	子供[こども]が 五人[ごにん]もいて、それもいたずら 盛[ざか]りの 男の子[おとこのこ]ばかりなので、 野口[のぐち]さんの 奥[おく]さんは 毎日[まいにち] 大変[たいへん] 忙[いそが]しいです。
\\	日本語で、それもくずし字で書いてある。	
\\	日本語[にほんご]で、それもくずし 字[じ]で 書[か]いてある。
\\	ダイヤの、それも二カラットのダイヤの指輪だ。	
\\	ダイヤの、それも 二[に]カラットのダイヤの 指輪[ゆびわ]だ。
\\	新築で、それも駅から五分という便利な所にある。	
\\	新築[しんちく]で、それも 駅[えき]から五 分[ふん]という 便利[べんり]な 所[ところ]にある。
\\	彼はその場で、それも現金で払ってくれた。	
\\	彼[かれ]はその 場[ば]で、それも 現金[げんきん]で 払[はら]ってくれた。
\\	真冬に、それもはだしで走った。	
\\	真冬[まふゆ]に、それもはだしで 走[はし]った。
\\	その銀行は都心の、それも東京駅前の一当地に本社ビルを建てた。	
\\	その 銀行[ぎんこう]は 都心[としん]の、それも 東京[とうきょう] 駅前[えきまえ]の 一当地[いっとうち]に 本社[ほんしゃ]ビルを 建[た]てた。
\\	彼女は雑誌記者、それもゴシップ雑誌の記者に見つかってしまった。	
\\	彼女[かのじょ]は 雑誌[ざっし] 記者[きしゃ]、それもゴシップ 雑誌[ざっし]の 記者[きしゃ]に 見[み]つかってしまった。
\\	忙しい学期末に、それも論文の締め切り直前に手紙を訳してくれと頼まれた。	
\\	忙[いそが]しい 学期[がっき] 末[まつ]に、それも 論文[ろんぶん]の 締め切[しめき]り 直前[ちょくぜん]に 手紙[てがみ]を 訳[やく]してくれと 頼[たの]まれた。
\\	硬貨で十万円、それも十円玉ばかりだったからとても重かった。	
\\	硬貨[こうか]で 十万[じゅうまん] 円[えん]、それも 十円[じゅうえん] 玉[だま]ばかりだったからとても 重[おも]かった。
\\	ここのモーニングセットはコーヒーにトースト、それにゆでたまごが付いている。	
\\	ここのモーニングセットはコーヒーにトースト、それにゆでたまごが 付[つ]いている。
\\	車を買っても置く所がないし、それにあまり乗る機会もない。	
\\	車[くるま]を 買[か]っても 置[お]く 所[ところ]がないし、それにあまり 乗[の]る 機会[きかい]もない。
\\	私は家では仕事の話はしないことにしているんです。それに、今晩はとても疲れているので、その話は明日にしてもらえませんか。	
\\	私[わたし]は 家[いえ]では 仕事[しごと]の 話[はなし]はしないことにしているんです。それに、 今晩[こんばん]はとても 疲[つか]れているので、その 話[はなし]は 明日[あす]にしてもらえませんか。
\\	私のうちには私達夫婦と子供が二人、それにネコとイヌが一匹ずついる。	
\\	私[わたし]のうちには 私[わたし] 達[たち] 夫婦[ふうふ]と 子供[こども]が 二人[ふたり]、それにネコとイヌが 一匹[いっぴき]ずついる。
\\	いい仕事がないし、それに子供も日本の学校へ行きたがっているので、日本へ帰ろうと思っている。	
\\	いい 仕事[しごと]がないし、それに 子供[こども]も 日本[にほん]の 学校[がっこう]へ 行[い]きたがっているので、 日本[にほん]へ 帰[かえ]ろうと 思[おも]っている。
\\	ピアノなんか買っても誰も弾く者はないし、それにうちは狭いので置く所がありません。	
\\	ピアノなんか 買[か]っても 誰[だれ]も 弾[ひ]く 者[もの]はないし、それにうちは 狭[せま]いので 置[お]く 所[ところ]がありません。
\\	このアパートは明るくて清潔だ。それに家賃が安い。	
\\	このアパートは 明[あか]るくて 清潔[せいけつ]だ。それに 家賃[やちん]が 安[やす]い。
\\	昔と違って今は相手がどこにいても電話で簡単に話が出来る。それに、テレビ電話を使うと相手の顔を見ながら話すことも出来る。	
\\	昔[むかし]と 違[ちが]って 今[いま]は 相手[あいて]がどこにいても 電話[でんわ]で 簡単[かんたん]に 話[はなし]が 出来[でき]る。それに、テレビ 電話[でんわ]を 使[つか]うと 相手[あいて]の 顔[かお]を 見[み]ながら 話[はな]すことも 出来[でき]る。
\\	彼女は結婚していて、それに子供もある。	
\\	彼女[かのじょ]は 結婚[けっこん]していて、それに 子供[こども]もある。
\\	とうふは安くて、それに栄養がある。	
\\	とうふは 安[やす]くて、それに 栄養[えいよう]がある。
\\	このアパートは不便で、それに家賃が高い。	
\\	このアパートは 不便[ふべん]で、それに 家賃[やちん]が 高[たか]い。
\\	私は近視で、それに乱視も少し入っている。	
\\	私[わたし]は 近視[きんし]で、それに 乱視[らんし]も 少[すこ]し 入[はい]っている。
\\	彼は大学で日本語を三年間勉強し、それに日本で仕事をしたこともある。	
\\	彼[かれ]は 大学[だいがく]で 日本語[にほんご]を 三年間[さんねんかん] 勉強[べんきょう]し、それに 日本[にほん]で 仕事[しごと]をしたこともある。
\\	彼の書く小説はストーリーが面白く、それに読みやすい。	
\\	彼[かれ]の 書[か]く 小説[しょうせつ]はストーリーが 面白[おもしろ]く、それに 読[よ]みやすい。
\\	ここのモーニングセットはコーヒーにトースト、おまけにゆでたまごまで付いている。	
\\	ここのモーニングセットはコーヒーにトースト、おまけにゆでたまごまで 付[つ]いている。
\\	リンゴ三つとオレンジ二つ、それと、バナナを三本下さい。	
\\	リンゴ 三[みっ]つとオレンジ 二[ふた]つ、それと、バナナを 三本[さんぽん] 下[くだ]さい。
\\	スポーツはテニスとゴルフをします。それと、学生時代をアイスホッケーをしていました。	
\\	スポーツはテニスとゴルフをします。それと、 学生[がくせい] 時代[じだい]をアイスホッケーをしていました。
\\	音楽はコーラスをしています。それと、学生の頃フルートを少しやりました。	
\\	音楽[おんがく]はコーラスをしています。それと、 学生[がくせい]の 頃[ころ]フルートを 少[すこ]しやりました。
\\	山田さん、木下さん、上野さん、それと、高橋さんもお見えになります。	
\\	山田[やまだ]さん、 木下[きのした]さん、 上野[うえの]さん、それと、 高橋[たかはし]さんもお 見[み]えになります。
\\	外国語はドイツ語にフランス語、それと、日本語が少し話せます。	
\\	外国[がいこく] 語[ご]はドイツ 語[ご]に フランス語[ふらんすご]、それと、 日本語[にほんご]が 少[すこ]し 話[はな]せます。
\\	留守中に山村さんと浜田さんからお電話がありました。それと、吉田さんからお手紙が来ております。	
\\	留守[るす] 中[ちゅう]に 山村[やまむら]さんと 浜田[はまだ]さんからお 電話[でんわ]がありました。それと、 吉田[よしだ]さんからお 手紙[てがみ]が 来[き]ております。
\\	緊急の問題は先生をどうするかということです。それと、もう一つ、教室の問題も考えなければなりません。	
\\	緊急[きんきゅう]の 問題[もんだい]は 先生[せんせい]をどうするかということです。それと、もう 一[ひと]つ、 教室[きょうしつ]の 問題[もんだい]も 考[かんが]えなければなりません。
\\	この手紙を昼までにタイプしておいてくれる?あ、それと、昨日頼んでおいた飛行機の切符、取れた?	
\\	この 手紙[てがみ]を 昼[ひる]までにタイプしておいてくれる?あ、それと、 昨日[きのう] 頼[たの]んでおいた 飛行機[ひこうき]の 切符[きっぷ]、 取[と]れた?
\\	リンゴ三つとオレンジ二つ。あ、それと、バナナも三本下さい。	
\\	リンゴ 三[みっ]つとオレンジ 二[ふた]つ。あ、それと、バナナも 三本[さんぽん] 下[くだ]さい。
\\	ー勝ちたいでしょうね。 ーええ、それは勝ちたいですよ。	ー
\\	ー 勝[か]ちたいでしょうね。 ーええ、それは 勝[か]ちたいですよ。
\\	ー国へ帰りたいと思うことがありますか。 ーそれはありますよ。	
\\	ー 国[こく]へ 帰[かえ]りたいと 思[おも]うことがありますか。 ーそれはありますよ。
\\	ーこの値段だったらみんな買うでしょうか。 ーそれは買いますよ。そんな値段では絶対買えませんから。	
\\	ーこの 値段[ねだん]だったらみんな 買[か]うでしょうか。 ーそれは 買[か]いますよ。そんな 値段[ねだん]では 絶対[ぜったい] 買[か]えませんから。
\\	ーもっと給料が高い方がいいでしょう? ーそりゃ(あ)そうですが・・・	ー
\\	ーもっと 給料[きゅうりょう]が 高[たか]い 方[ほう]がいいでしょう? ーそりゃ(あ)そうですが・・・
\\	ー彼女は僕が嫌いになったのかなあ。 ーそれは違うよ。	
\\	ー 彼女[かのじょ]は 僕[ぼく]が 嫌[きら]いになったのかなあ。 ーそれは 違[ちが]うよ。
\\	ーこれ、つまらないものですが。 ーそれはご丁寧に。	
\\	ーこれ、つまらないものですが。 ーそれはご 丁寧[ていねい]に。
\\	山田さんと鈴木さんは、果物とお菓子をそれぞれ買った。	
\\	山田[やまだ]さんと 鈴木[すずき]さんは、 果物[くだもの]とお 菓子[かし]をそれぞれ 買[か]った。
\\	読んだ小説はそれぞれ面白かった。	
\\	読[よ]んだ 小説[しょうせつ]はそれぞれ 面白[おもしろ]かった。
\\	好子、美保子、信夫はそれぞれ十五歳、十三歳、九歳です。	
\\	好子[よしこ]、 美保子[みほこ]、 信夫[しのぶ]はそれぞれ 十五歳[じゅうごさい]、十三歳[じゅうさんさい]、九歳[きゅうさい]です。
\\	父は私と妹に、それぞれ、一万円のお小遣いをくれた。	
\\	父[ちち]は 私[わたし]と 妹[いもうと]に、それぞれ、一万[いちまん] 円[えん]のお 小遣[こづか]いをくれた。
\\	トムは寮の部屋で、僕は図書館で、それぞれ勉強する。	
\\	トムは 寮[りょう]の 部屋[へや]で、 僕[ぼく]は 図書館[としょかん]で、それぞれ 勉強[べんきょう]する。
\\	ボブはベスト、一雄はジェミート、それぞれ出かけた。	
\\	ボブはベスト、 一雄[かずお]はジェミート、それぞれ 出[で]かけた。
\\	私達は兄弟だが、趣味がそれぞれ違う。	
\\	私[わたし] 達[たち]は 兄弟[きょうだい]だが、 趣味[しゅみ]がそれぞれ 違[ちが]う。
\\	誰でもそれぞれ長所と短所がある。	
\\	誰[だれ]でもそれぞれ 長所[ちょうしょ]と 短所[たんしょ]がある。
\\	このキャンパスの建物はそれぞれ個性がある。	
\\	このキャンパスの 建物[たてもの]はそれぞれ 個性[こせい]がある。
\\	レストランではそれぞれ好きな物を食べました。	
\\	レストランではそれぞれ 好[す]きな 物[もの]を 食[た]べました。
\\	裁判官はそれぞれの席に着いた。	
\\	裁判官[さいばんかん]はそれぞれの 席[せき]に 着[つ]いた。
\\	山田さんは果物を買って、鈴木さんはお菓子を買った。	
\\	山田[やまだ]さんは 果物[くだもの]を 買[か]って、 鈴木[すずき]さんはお 菓子[かし]を 買[か]った。
\\	山田さんは果物を、鈴木さんをお菓子を買った。	
\\	山田[やまだ]さんは 果物[くだもの]を、 鈴木[すずき]さんをお 菓子[かし]を 買[か]った。
\\	山田さんは果物を、鈴木さんはお菓子を、それぞれ買った。	
\\	山田[やまだ]さんは 果物[くだもの]を、 鈴木[すずき]さんはお 菓子[かし]を、それぞれ 買[か]った。
\\	オーケストラはそれぞれの音色を持っている。	
\\	オーケストラはそれぞれの 音色[ねいろ]を 持[も]っている。
\\	学生はそれぞれの自転車で登校する。	
\\	学生[がくせい]はそれぞれの 自転車[じてんしゃ]で 登校[とうこう]する。
\\	宇宙の星はそれぞれが引力で引き合っている。	
\\	宇宙[うちゅう]の 星[ほし]はそれぞれが 引力[いんりょく]で 引き合[ひきあ]っている。
\\	このワインの中にはフランス産とドイツ産が混じっています。匂いをかいだだけで、それぞれを言い当てることが出来ますか。	
\\	このワインの 中[なか]にはフランス 産[さん]とドイツ 産[さん]が 混[ま]じっています。 匂[にお]いをかいだだけで、それぞれを 言い当[いいあ]てることが 出来[でき]ますか。
\\	家に帰ったらすぐ、寝てしまいました。	
\\	家[いえ]に 帰[かえ]ったらすぐ、 寝[ね]てしまいました。
\\	銀行は駅のすぐ前です。	
\\	銀行[ぎんこう]は 駅[えき]のすぐ 前[まえ]です。
\\	部長は会社に着くとすぐ、仕事を始めた。	
\\	部長[ぶちょう]は 会社[かいしゃ]に 着[つ]くとすぐ、 仕事[しごと]を 始[はじ]めた。
\\	日本に来たらすぐ、電話して下さい。	
\\	日本[にほん]に 来[き]たらすぐ、 電話[でんわ]して 下[くだ]さい。
\\	話したいから、すぐ来て下さい。	
\\	話[はな]したいから、すぐ 来[き]て 下[くだ]さい。
\\	安いテレビを買ったらすぐ、壊れてしまいました。	
\\	安[やす]いテレビを 買[か]ったらすぐ、 壊[こわ]れてしまいました。
\\	あの人はすぐ怒るから、嫌いだ。	
\\	あの 人[ひと]はすぐ 怒[おこ]るから、 嫌[きら]いだ。
\\	その数学の問題はすぐ解けた。	
\\	その 数学[すうがく]の 問題[もんだい]はすぐ 解[と]けた。
\\	私の家はすぐそこです。	
\\	私[わたし]の 家[いえ]はすぐそこです。
\\	郵便局はスーパーのすぐ隣です。	
\\	郵便[ゆうびん] 局[きょく]はスーパーのすぐ 隣[となり]です。
\\	そのエビを食べたら、すぐおなかが痛くなった。	
\\	そのエビを 食[た]べたら、すぐおなかが 痛[いた]くなった。
\\	スミスさんは電話したから、もうすぐ来るよ。	
\\	スミスさんは 電話[でんわ]したから、もうすぐ 来[く]るよ。
\\	日本へ行くたびに新しいことを学んで帰ります。	
\\	日本[にほん]へ 行[い]くたびに 新[あたら]しいことを 学[まな]んで 帰[かえ]ります。
\\	父は旅行のたびにおみやげを買ってきてくれる。	
\\	父[ちち]は 旅行[りょこう]のたびにおみやげを 買[か]ってきてくれる。
\\	あの人と話すたびに心が和みます。	
\\	あの 人[ひと]と 話[はな]すたびに 心[こころ]が 和[なご]みます。
\\	この本は読むたびに新しい発見がある。	
\\	この 本[ほん]は 読[よ]むたびに 新[あたら]しい 発見[はっけん]がある。
\\	僕は東京に行くたびに恩師のお宅を訪ねることにしている。	
\\	僕[ぼく]は 東京[とうきょう]に 行[い]くたびに 恩師[おんし]のお 宅[たく]を 訪[たず]ねることにしている。
\\	私の孫は会うたびに前より大きくなっている。	
\\	私[わたし]の 孫[まご]は 会[あ]うたびに 前[まえ]より 大[おお]きくなっている。
\\	モーツァルトの音楽を聞くたびに人生が豊かになった気がする。	
\\	モーツァルトの 音楽[おんがく]を 聞[き]くたびに 人生[じんせい]が 豊[ゆた]かになった 気[き]がする。
\\	誕生日のたびに夫はバラの花を買ってくれます。	
\\	誕生[たんじょう] 日[び]のたびに 夫[おっと]はバラの 花[はな]を 買[か]ってくれます。
\\	うちでは給料日のたびに銀座のレストランで食事をします。	
\\	うちでは 給料[きゅうりょう] 日[び]のたびに 銀座[ぎんざ]のレストランで 食事[しょくじ]をします。
\\	モーツァルトの音楽を聞く時にはいつでも人生が豊かになった気がする。	
\\	モーツァルトの 音楽[おんがく]を 聞[き]く 時[とき]にはいつでも 人生[じんせい]が 豊[ゆた]かになった 気[き]がする。
\\	モーツァルトの音楽を聞いた時にはいつでも人生が豊かになった気がする。	
\\	モーツァルトの 音楽[おんがく]を 聞[き]いた 時[とき]にはいつでも 人生[じんせい]が 豊[ゆた]かになった 気[き]がする。
\\	彼のただ一つの欠点は時間にルーズだということだ。	
\\	彼[かれ]のただ 一[ひと]つの 欠点[けってん]は 時間[じかん]にルーズだということだ。
\\	彼女はただの一度も手紙をくれなかった。	
\\	彼女[かのじょ]はただの 一度[いちど]も 手紙[てがみ]をくれなかった。
\\	彼はただ弁護士だけが頼りだ。	
\\	彼[かれ]はただ 弁護士[べんごし]だけが 頼[たよ]りだ。
\\	彼女はただお茶を飲んで話をしたというだけだ。	
\\	彼女[かのじょ]はただお 茶[ちゃ]を 飲[の]んで 話[はなし]をしたというだけだ。
\\	彼はただ一人でやってきた。	
\\	彼[かれ]はただ 一人[ひとり]でやってきた。
\\	私達が困っていた時、助けてくれた者はただの一人もいなかった。	
\\	私[わたし] 達[たち]が 困[こま]っていた 時[とき]、 助[たす]けてくれた 者[もの]はただの 一人[ひとり]もいなかった。
\\	それだけのことで彼を首にしたんですか。	
\\	それだけのことで 彼[かれ]を 首[くび]にしたんですか。
\\	ただ言われたことだけをやっていても進歩はない。	
\\	ただ 言[い]われたことだけをやっていても 進歩[しんぽ]はない。
\\	私はただ頼まれたことをしているだけだ。	
\\	私[わたし]はただ 頼[たの]まれたことをしているだけだ。
\\	あの男はただまじめだというだけだ。	
\\	あの 男[おとこ]はただまじめだというだけだ。
\\	別に大した意味はないんですが、ただちょっと聞いてみたかったんです。	
\\	別[べつ]に 大[たい]した 意味[いみ]はないんですが、ただちょっと 聞[き]いてみたかったんです。
\\	ただ人の言うことを鵜呑みにしていては面白い発想は生まれない。	
\\	ただ 人[ひと]の 言[い]うことを 鵜呑[うの]みにしていては 面白[おもしろ]い 発想[はっそう]は 生[う]まれない。
\\	彼の一つの欠点は時間にルーズだということだ。	
\\	彼[かれ]の 一[ひと]つの 欠点[けってん]は 時間[じかん]にルーズだということだ。
\\	私はこの夏たった二日しか休まなかった。	
\\	私[わたし]はこの 夏[なつ]たった 二日[ふつか]しか 休[やす]まなかった。
\\	ジョンはたった今帰ったばかりです。	
\\	ジョンはたった 今[いま] 帰[かえ]ったばかりです。
\\	私はただの平社員ですから、そんなことは知りません。	
\\	私[わたし]はただの 平[ひら] 社員[しゃいん]ですから、そんなことは 知[し]りません。
\\	ただの風邪だと思っていたら、肺炎だった。	
\\	ただの 風邪[かぜ]だと 思[おも]っていたら、 肺炎[はいえん]だった。
\\	あの人はただの学者ではないようですね。	
\\	あの 人[ひと]はただの 学者[がくしゃ]ではないようですね。
\\	ただの集まりだと思って、顔を出したら、偉い人達が出席していたのでびっくりした。	
\\	ただの 集[あつ]まりだと 思[おも]って、 顔[かお]を 出[だ]したら、 偉[えら]い 人達[ひとたち]が 出席[しゅっせき]していたのでびっくりした。
\\	ーお父様はお偉い方だったそうですね。 ーいいえ、とんでもありません。ただの大学教師でした。	
\\	ーお 父様[とうさま]はお 偉[えら]い 方[かた]だったそうですね。 ーいいえ、とんでもありません。ただの 大学[だいがく] 教師[きょうし]でした。
\\	ただの冗談ですよ。	
\\	ただの 冗談[じょうだん]ですよ。
\\	この車は確かに魅力的だが、値段が高すぎる。	
\\	この 車[くるま]は 確[たし]かに 魅力[みりょく] 的[てき]だが、 値段[ねだん]が 高[たか]すぎる。
\\	そのパーティーのことは確かに聞いたが、いつだったか思い出せない。	
\\	そのパーティーのことは 確[たし]かに 聞[き]いたが、いつだったか 思い出[おもいだ]せない。
\\	確かに約束はしたが、今すぐとは言わなかった。	
\\	確[たし]かに 約束[やくそく]はしたが、 今[いま]すぐとは 言[い]わなかった。
\\	彼は確かに第一印象がよくないが、本当はいい男なんだよ。	
\\	彼[かれ]は 確[たし]かに 第[だい]一印象[いちいんしょう]がよくないが、 本当[ほんとう]はいい 男[おとこ]なんだよ。
\\	この学校は確かにいい学校だが、僕には向いていない。	
\\	この 学校[がっこう]は 確[たし]かにいい 学校[がっこう]だが、 僕[ぼく]には 向[む]いていない。
\\	確かに多くの日本人が海外へ出かけるようになった。外国語を上手に話す人もたくさんいる。しかし、本当の国際人と呼べる人はまだ少ない。	
\\	確[たし]かに 多[おお]くの 日本人[にほんじん]が 海外[かいがい]へ 出[で]かけるようになった。 外国[がいこく] 語[ご]を 上手[じょうず]に 話[はな]す 人[ひと]もたくさんいる。しかし、 本当[ほんとう]の 国際[こくさい] 人[じん]と 呼[よ]べる 人[ひと]はまだ 少[すく]ない。
\\	どんなに低く見積もったところで、工費は五億円を超えるだろう。	
\\	どんなに 低[ひく]く 見積[みつ]もったところで、 工費[こうひ]は 五億[ごおく] 円[えん]を 超[こ]えるだろう。
\\	この薬を飲んだところで、風邪がすぐ治るわけじゃない。	
\\	この 薬[くすり]を 飲[の]んだところで、 風邪[かぜ]がすぐ 治[なお]るわけじゃない。
\\	この本を読んだところで、日本の経済の全体は分からないだろう。	
\\	この 本[ほん]を 読[よ]んだところで、 日本[にほん]の 経済[けいざい]の 全体[ぜんたい]は 分[わ]からないだろう。
\\	一週間に一度ぐらい運動したところで、あまり効果はないでしょう。	
\\	一週間[いっしゅうかん]に 一度[いちど]ぐらい 運動[うんどう]したところで、あまり 効果[こうか]はないでしょう。
\\	どんなに頑張ったところで、この科目では優が取れるはずがない。	
\\	どんなに 頑張[がんば]ったところで、この 科目[かもく]では 優[ゆう]が 取[と]れるはずがない。
\\	休みに天気が良くなったところで、どこかへ出かけるあてもない。	
\\	休[やす]みに 天気[てんき]が 良[よ]くなったところで、どこかへ 出[で]かけるあてもない。
\\	日本語が話せたところで、日本へ行く金もないので、何にもならない。	
\\	日本語[にほんご]が 話[はな]せたところで、 日本[にほん]へ 行[い]く 金[かね]もないので、 何[なに]にもならない。
\\	これ以上話し合ったところで、恐らく無駄でしょう。	
\\	これ 以上[いじょう] 話し合[はなしあ]ったところで、 恐[おそ]らく 無駄[むだ]でしょう。
\\	問題は起こったところで、君には迷惑はかからない。	
\\	問題[もんだい]は 起[お]こったところで、 君[きみ]には 迷惑[めいわく]はかからない。
\\	主人は私が何度注意してもたばこをやめない。	
\\	主人[しゅじん]は 私[わたし]が 何[なん] 度[ど] 注意[ちゅうい]してもたばこをやめない。
\\	僕は何を食べても太っちゃうんです。	
\\	僕[ぼく]は 何[なに]を 食[た]べても 太[ふと]っちゃうんです。
\\	この科目を取らなくても卒業は出来る。	
\\	この 科目[かもく]を 取[と]らなくても 卒業[そつぎょう]は 出来[でき]る。
\\	鉛筆で書いてもいいですか。	
\\	鉛筆[えんぴつ]で 書[か]いてもいいですか。
\\	ちょっと読んでもつまらない本だとすぐ分かる。	
\\	ちょっと 読[よ]んでもつまらない 本[ほん]だとすぐ 分[わ]かる。
\\	こちらにいらっしゃってもう何年になりますか。	
\\	こちらにいらっしゃってもう 何[なん] 年[ねん]になりますか。
\\	二人は結婚して二年後に離婚した。	
\\	二人[ふたり]は 結婚[けっこん]して 二年[にねん] 後[ご]に 離婚[りこん]した。
\\	会社を辞めてもう久しい。	
\\	会社[かいしゃ]を 辞[や]めてもう 久[ひさ]しい。
\\	この島を訪れてもう何年になるだろうか。	
\\	この 島[しま]を 訪[おとず]れてもう 何[なん] 年[ねん]になるだろうか。
\\	この会社で働き始めてこれで四年になります。	
\\	この 会社[かいしゃ]で 働[はたら]き 始[はじ]めてこれで 四年[よねん]になります。
\\	この町に来てまだ一週間目です。	
\\	この 町[まち]に 来[き]てまだ 一週間[いっしゅうかん] 目[め]です。
\\	二人は結婚する二年前にお見合いをした。	
\\	二人[ふたり]は 結婚[けっこん]する 二年[にねん] 前[まえ]にお 見合[みあ]いをした。
\\	アメリカへ来て始めて竜巻を見た。	
\\	アメリカへ 来[き]て 始[はじ]めて 竜巻[たつまき]を 見[み]た。
\\	日本で生活して始めて日本人のものの考え方が分かってきた。	
\\	日本[にほん]で 生活[せいかつ]して 始[はじ]めて 日本人[にほんじん]のものの 考え方[かんがえかた]が 分[わ]かってきた。
\\	日本語を勉強して初めて外国語を学ぶ面白さを知った。	
\\	日本語[にほんご]を 勉強[べんきょう]して 初[はじ]めて 外国[がいこく] 語[ご]を 学[まな]ぶ 面白[おもしろ]さを 知[し]った。
\\	考えは文字にしてみて初めてはっきりすることが多い。	
\\	考[かんが]えは 文字[もじ]にしてみて 初[はじ]めてはっきりすることが 多[おお]い。
\\	病気になって初めて健康の有り難さが分かる。	
\\	病気[びょうき]になって 初[はじ]めて 健康[けんこう]の 有り難[ありがた]さが 分[わ]かる。
\\	生まれて初めてさしみを食べた。	
\\	生[う]まれて 初[はじ]めてさしみを 食[た]べた。
\\	さしみは生まれて初めてです。	
\\	さしみは 生[う]まれて 初[はじ]めてです。
\\	彼はそういう点でまだ大人とは言えない。	
\\	彼[かれ]はそういう 点[てん]でまだ 大人[おとな]とは 言[い]えない。
\\	使いやすさの点ではこの辞書はネルソンの漢英辞典に及ばない。	
\\	使[つか]いやすさの 点[てん]ではこの 辞書[じしょ]はネルソンの 漢[かん] 英[えい] 辞典[じてん]に 及[およ]ばない。
\\	通勤の便利さという点では今のアパートは最高だ。	
\\	通勤[つうきん]の 便利[べんり]さという 点[てん]では 今[いま]のアパートは 最高[さいこう]だ。
\\	このテーブルは手作りである点で値打ちがある。	
\\	このテーブルは 手作[てづく]りである 点[てん]で 値打[ねう]ちがある。
\\	我が社の製品はすべての点で国外の類似製品より優れている。	
\\	我[わ]が 社[しゃ]の 製品[せいひん]はすべての 点[てん]で 国外[こくがい]の 類似[るいじ] 製品[せいひん]より 優[すぐ]れている。
\\	学力の点ではこの学生が一番だろう。	
\\	学力[がくりょく]の 点[てん]ではこの 学生[がくせい]が 一番[いちばん]だろう。
\\	住みやすさの点では私はこの町の方が前に住んだ町より気に入っている。	
\\	住[す]みやすさの 点[てん]では 私[わたし]はこの 町[まち]の 方[ほう]が 前[まえ]に 住[す]んだ 町[まち]より 気に入[きにい]っている。
\\	時間の正確さという点では日本の鉄道は世界一でしょう。	
\\	時間[じかん]の 正確[せいかく]さという 点[てん]では 日本[にほん]の 鉄道[てつどう]は 世界一[せかいいち]でしょう。
\\	イルカは言葉を話すという点で他の多くの水棲動物より高等であると言える。	
\\	イルカは 言葉[ことば]を 話[はな]すという 点[てん]で 他[た]の 多[おお]くの 水棲[すいせい] 動物[どうぶつ]より 高等[こうとう]であると 言[い]える。
\\	彼女と私はクラシック音楽が好きだという点で趣味が一致している。	
\\	彼女[かのじょ]と 私[わたし]はクラシック 音楽[おんがく]が 好[す]きだという 点[てん]で 趣味[しゅみ]が 一致[いっち]している。
\\	スミス氏は時代を先取り出来るという点で社長に適任だ。	
\\	スミス 氏[し]は 時代[じだい]を 先取[さきど]り 出来[でき]るという 点[てん]で 社長[しゃちょう]に 適任[てきにん]だ。
\\	彼女のそういう点が人に嫌われるのだ。	
\\	彼女[かのじょ]のそういう 点[てん]が 人[ひと]に 嫌[きら]われるのだ。
\\	私は彼の日本語がよく出来る点を評価したい。	
\\	私[わたし]は 彼[かれ]の 日本語[にほんご]がよく 出来[でき]る 点[てん]を 評価[ひょうか]したい。
\\	私はこの製品が未来を指向している点に引かれる。	
\\	私[わたし]はこの 製品[せいひん]が 未来[みらい]を 指向[しこう]している 点[てん]に 引[ひ]かれる。
\\	その点について何人かの人から質問を受けた。	
\\	その 点[てん]について 何[なん] 人[にん]かの 人[ひと]から 質問[しつもん]を 受[う]けた。
\\	あの会社は技術面に問題がある。	
\\	あの 会社[かいしゃ]は 技術[ぎじゅつ] 面[めん]に 問題[もんだい]がある。
\\	我々はその取引において値段の点で合意出来なかった。	
\\	我々[われわれ]はその 取引[とりひき]において 値段[ねだん]の 点[てん]で 合意[ごうい] 出来[でき]なかった。
\\	この車は性能が優れている点を高く評価された。	
\\	この 車[くるま]は 性能[せいのう]が 優[すぐ]れている 点[てん]を 高[たか]く 評価[ひょうか]された。
\\	そんなに勉強しては、体をこわしますよ。	
\\	そんなに 勉強[べんきょう]しては、 体[からだ]をこわしますよ。
\\	こんなに寒くては外出できません。	
\\	こんなに 寒[さむ]くては 外出[がいしゅつ]できません。
\\	仕事がそんなに嫌いでは困りますね。	
\\	仕事[しごと]がそんなに 嫌[きら]いでは 困[こま]りますね。
\\	そんなに遊んでいては、試験に失敗しますよ。	
\\	そんなに 遊[あそ]んでいては、 試験[しけん]に 失敗[しっぱい]しますよ。
\\	あの男が来ては邪魔になる。	
\\	あの 男[おとこ]が 来[き]ては 邪魔[じゃま]になる。
\\	こんなに働かされては、病気になってしまうよ。	
\\	こんなに 働[はたら]かされては、 病気[びょうき]になってしまうよ。
\\	こんなに部屋が汚くては、お客さんが見えた時に恥ずかしい。	
\\	こんなに 部屋[へや]が 汚[きたな]くては、お 客[きゃく]さんが 見[み]えた 時[とき]に 恥[は]ずかしい。
\\	人が皆個人主儀的では、国はやっていけない。	
\\	人[ひと]が 皆[みな] 個人[こじん] 主[しゅ] 儀[ぎ] 的[てき]では、 国[くに]はやっていけない。
\\	こんなに忙しくては、新聞も読めない。	
\\	こんなに 忙[いそが]しくては、 新聞[しんぶん]も 読[よ]めない。
\\	あんな女性がデートの相手では、彼は小さくなっているだろう。	
\\	あんな 女性[じょせい]がデートの 相手[あいて]では、 彼[かれ]は 小[ちい]さくなっているだろう。
\\	幸夫は来年は家が買えると大変喜んでいる。	
\\	幸夫[ゆきお]は 来年[らいねん]は 家[いえ]が 買[か]えると 大変[たいへん] 喜[よろこ]んでいる。
\\	博物館は開いているかなと、電話してみた。	
\\	博物館[はくぶつかん]は 開[あ]いているかなと、 電話[でんわ]してみた。
\\	テニスはしようと出かけたら、途中で雨が降ってきた。	
\\	テニスはしようと 出[で]かけたら、 途中[とちゅう]で 雨[あめ]が 降[ふ]ってきた。
\\	父はまだ元気なようだと安心していたが、最近急に弱ってきた。	
\\	父[ちち]はまだ 元気[げんき]なようだと 安心[あんしん]していたが、 最近[さいきん] 急[きゅう]に 弱[よわ]ってきた。
\\	外国語の能力は単語から文、文から段落、そして段落から福段落と、伸びていく。	
\\	外国[がいこく] 語[ご]の 能力[のうりょく]は 単語[たんご]から 文[ぶん]、 文[ぶん]から 段落[だんらく]、そして 段落[だんらく]から 福[ふく] 段落[だんらく]と、 伸[の]びていく。
\\	友達は長い間飼っていた猫が死んでしまったと、悲しがっていた。	
\\	友達[ともだち]は 長[なが]い 間[あいだ] 飼[か]っていた 猫[ねこ]が 死[し]んでしまったと、 悲[かな]しがっていた。
\\	彼はがんになったかもしれないと心配している。	
\\	彼[かれ]はがんになったかもしれないと 心配[しんぱい]している。
\\	その学生はもうちょっとで百点が取れたのにと、悔しがった。	
\\	その 学生[がくせい]はもうちょっとで 百[ひゃく] 点[てん]が 取[と]れたのにと、 悔[くや]しがった。
\\	彼女と会うのもこれが最後かと、寂しかったのです。	
\\	彼女[かのじょ]と 会[あ]うのもこれが 最後[さいご]かと、 寂[さび]しかったのです。
\\	変な音がしたなと、外に出てみたが、何でもなかった。	
\\	変[へん]な 音[おと]がしたなと、 外[そと]に 出[で]てみたが、 何[なに]でもなかった。
\\	僕は今晩は酒を飲むまいと、まっすぐ家に帰った。	
\\	僕[ぼく]は 今晩[こんばん]は 酒[さけ]を 飲[の]むまいと、まっすぐ 家[いえ]に 帰[かえ]った。
\\	年内に論文を仕上げようと、懸命に努力しているところです。	
\\	年内[ねんない]に 論文[ろんぶん]を 仕上[しあ]げようと、 懸命[けんめい]に 努力[どりょく]しているところです。
\\	彼にお礼を言おうとやって来たのに、ほかのことばかり話して、お礼を言うのを忘れてしまった。	
\\	彼[かれ]にお 礼[れい]を 言[い]おうとやって 来[き]たのに、ほかのことばかり 話[はな]して、お 礼[れい]を 言[い]うのを 忘[わす]れてしまった。
\\	少しは体にいいだろうと、最近は散歩をしています。	
\\	少[すこ]しは 体[からだ]にいいだろうと、 最近[さいきん]は 散歩[さんぽ]をしています。
\\	宿題を忘れたので、先生に叱られるかもしれないと、びくびくしていた。	
\\	宿題[しゅくだい]を 忘[わす]れたので、 先生[せんせい]に 叱[しか]られるかもしれないと、びくびくしていた。
\\	日本の四季は、春は三月から五月、夏は六月から八月、秋は九月から十一月、冬は十二月から二月までと、どの季節もほぼ同じ長さになっている。	
\\	日本[にほん]の 四季[しき]は、 春[はる]は 三月[さんがつ]から 五月[ごがつ]、 夏[なつ]は 六月[ろくがつ]から 八月[はちがつ]、 秋[あき]は 九月[くがつ]から 十一月[じゅういちがつ]、 冬[ふゆ]は 十二月[じゅうにがつ]から 二月[にがつ]までと、どの 季節[きせつ]もほぼ 同[おな]じ 長[なが]さになっている。
\\	彼は朝は和食、昼はめん類、夜は洋食と、食事のパターンが決まっている。	
\\	彼[かれ]は 朝[あさ]は 和食[わしょく]、 昼[ひる]はめん 類[るい]、 夜[よる]は 洋食[ようしょく]と、 食事[しょくじ]のパターンが 決[き]まっている。
\\	社長は先月はロンドン、今月はモスクワ、来月はソウルと、出張が多い。	
\\	社長[しゃちょう]は 先月[せんげつ]はロンドン、 今月[こんげつ]はモスクワ、 来月[らいげつ]はソウルと、 出張[しゅっちょう]が 多[おお]い。
\\	日本語を日本で勉強するのも面白いと思って、去年の夏日本で日本語を勉強した。	
\\	日本語[にほんご]を 日本[にほん]で 勉強[べんきょう]するのも 面白[おもしろ]いと 思[おも]って、 去年[きょねん]の 夏[なつ] 日本[にほん]で 日本語[にほんご]を 勉強[べんきょう]した。
\\	太郎は冬休みに海外旅行でもしようと思って、急に家の近くの旅行代理店に出かけた。	
\\	太郎[たろう]は 冬休[ふゆやす]みに 海外[かいがい] 旅行[りょこう]でもしようと 思[おも]って、 急[きゅう]に 家[いえ]の 近[ちか]くの 旅行[りょこう] 代理[だいり] 店[てん]に 出[で]かけた。
\\	お金を払おうとして、財布がないことに気が付いた。	
\\	お 金[かね]を 払[はら]おうとして、 財布[さいふ]がないことに 気が付[きがつ]いた。
\\	社長には問題はなかったと言っておいて下さい。	
\\	社長[しゃちょう]には 問題[もんだい]はなかったと 言[い]っておいて 下[くだ]さい。
\\	この物語は今の社会を風刺しているとは考えられませんか。	
\\	この 物語[ものがたり]は 今[いま]の 社会[しゃかい]を 風刺[ふうし]しているとは 考[かんが]えられませんか。
\\	彼はもう我慢が出来ないという風に荒々しく立ち上がった。	
\\	彼[かれ]はもう 我慢[がまん]が 出来[でき]ないという 風[ふう]に 荒々[あらあら]しく 立ち上[たちあ]がった。
\\	彼はどうしようもないというように首を振った。	
\\	彼[かれ]はどうしようもないというように 首[くび]を 振[ふ]った。
\\	彼は大学卒業と同時に銀行に就職した。	
\\	彼[かれ]は 大学[だいがく] 卒業[そつぎょう]と 同時[どうじ]に 銀行[ぎんこう]に 就職[しゅうしょく]した。
\\	部屋に入ると同時に電話がなった。	
\\	部屋[へや]に 入[はい]ると 同時[どうじ]に 電話[でんわ]がなった。
\\	彼はこの会社の社長であると同時に大株主でもある。	
\\	彼[かれ]はこの 会社[かいしゃ]の 社長[しゃちょう]であると 同時[どうじ]に 大[おお] 株主[かぶぬし]でもある。
\\	彼女は結婚と同時に会社を辞めた。	
\\	彼女[かのじょ]は 結婚[けっこん]と 同時[どうじ]に 会社[かいしゃ]を 辞[や]めた。
\\	銃声と同時に警官の一人が倒れた。	
\\	銃声[じゅうせい]と 同時[どうじ]に 警官[けいかん]の 一人[ひとり]が 倒[たお]れた。
\\	新しい単語の学習と同時に文法の復習も忘れてはならない。	
\\	新[あたら]しい 単語[たんご]の 学習[がくしゅう]と 同時[どうじ]に 文法[ぶんぽう]の 復習[ふくしゅう]も 忘[わす]れてはならない。
\\	私はルースと同時に日本語の勉強を始めた。	
\\	私[わたし]はルースと 同時[どうじ]に 日本語[にほんご]の 勉強[べんきょう]を 始[はじ]めた。
\\	電車が止まってドアが開くと同時に客がなだれ込んだ。	
\\	電車[でんしゃ]が 止[と]まってドアが 開[ひら]くと 同時[どうじ]に 客[きゃく]がなだれ 込[こ]んだ。
\\	新社長は就任すると同時に社名を変更した。	
\\	新[しん] 社長[しゃちょう]は 就任[しゅうにん]すると 同時[どうじ]に 社名[しゃめい]を 変更[へんこう]した。
\\	外国語を勉強する時は、文法を理解すると同時に文型の口頭練習もしなければならない。	
\\	外国[がいこく] 語[ご]を 勉強[べんきょう]する 時[とき]は、 文法[ぶんぽう]を 理解[りかい]すると 同時[どうじ]に 文型[ぶんけい]の 口頭[こうとう] 練習[れんしゅう]もしなければならない。
\\	この新しい装置は安全であると同時に効率がいい。	
\\	この 新[あたら]しい 装置[そうち]は 安全[あんぜん]であると 同時[どうじ]に 効率[こうりつ]がいい。
\\	森鴎外は医者であると同時に小説家でもあった。	
\\	森[もり] 鴎外[おうがい]は 医者[いしゃ]であると 同時[どうじ]に 小説[しょうせつ] 家[か]でもあった。
\\	この設備は危険であると同時に効率が悪い。	
\\	この 設備[せつび]は 危険[きけん]であると 同時[どうじ]に 効率[こうりつ]が 悪[わる]い。
\\	私には子供がいます。子供と言ってももう大学生ですが。	
\\	私[わたし]には 子供[こども]がいます。 子供[こども]と 言[い]ってももう 大学生[だいがくせい]ですが。
\\	ー毎日お忙しいんでしょうね。 ーいえ、忙しいと言っても週末は休んでいますから。	
\\	ー 毎日[まいにち]お 忙[いそが]しいんでしょうね。 ーいえ、 忙[いそが]しいと 言[い]っても 週末[しゅうまつ]は 休[やす]んでいますから。
\\	もう三月だ、しかし、三月と言ってもこの辺はまだ寒い。	
\\	もう 三月[さんがつ]だ、しかし、 三月[さんがつ]と 言[い]ってもこの 辺[あたり]はまだ 寒[さむ]い。
\\	山崎先生はまだお若いですよ。もちろん、若いと言っても五十代ですが。	
\\	山崎[やまざき] 先生[せんせい]はまだお 若[わか]いですよ。もちろん、 若[わか]いと 言[い]っても 五十代[ごじゅうだい]ですが。
\\	仕事が見つかりました。と言ってもパートなんですが。	
\\	仕事[しごと]が 見[み]つかりました。と 言[い]ってもパートなんですが。
\\	ー韓国語をご存知なんでしょう? ーいや、知っていると言っても二年勉強してだけですから大したことはありません。	
\\	ー 韓国[かんこく] 語[ご]をご 存知[ぞんじ]なんでしょう? ーいや、 知[し]っていると 言[い]っても 二年[にねん] 勉強[べんきょう]してだけですから 大[たい]したことはありません。
\\	ー辻村さんが手伝ってくれるそうですよ。 ーでも、手伝ってくれると言ってもせいぜい二、三時間でしょう。	ー
\\	ー 辻村[つじむら]さんが 手伝[てつだ]ってくれるそうですよ。 ーでも、 手伝[てつだ]ってくれると 言[い]ってもせいぜい 二[に]、三時間[さんじかん]でしょう。
\\	ーあの子は強いですよ。 ーいや、強いと言ってもたかが小学生ですよ。	ー
\\	ーあの 子[こ]は 強[つよ]いですよ。 ーいや、 強[つよ]いと 言[い]ってもたかが 小学生[しょうがくせい]ですよ。
\\	年をとったと言ってもビルは元プロ選手だ。私では勝てないだろう。	
\\	年[とし]をとったと 言[い]ってもビルは 元[もと]プロ 選手[せんしゅ]だ。 私[わたし]では 勝[か]てないだろう。
\\	年をとってもビルは元プロ選手だ。	
\\	年[とし]をとってもビルは 元[もと]プロ 選手[せんしゅ]だ。
\\	彼はどうしようもないという風に首を振った。	
\\	彼[かれ]はどうしようもないという 風[ふう]に 首[くび]を 振[ふ]った。
\\	私は河村さんは来られないという風に聞いている。	
\\	私[わたし]は 河村[かわむら]さんは 来[こ]られないという 風[ふう]に 聞[き]いている。
\\	浩司はうなぎ、友子はてんぷら、という風にめいめいが自分の好きなものを注文した。	
\\	浩司[ひろし]はうなぎ、 友子[ともこ]はてんぷら、という 風[ふう]にめいめいが 自分[じぶん]の 好[す]きなものを 注文[ちゅうもん]した。
\\	彼女はもうこれ以上聞きたくないという風に両手で耳を被った。	
\\	彼女[かのじょ]はもうこれ 以上[いじょう] 聞[き]きたくないという 風[ふう]に 両手[りょうて]で 耳[みみ]を 被[おお]った。
\\	社長には、特に問題はなかったという風に言っておいて下さい。	
\\	社長[しゃちょう]には、 特[とく]に 問題[もんだい]はなかったという 風[かぜ]に 言[い]っておいて 下[くだ]さい。
\\	そういうことは規則に反するから出来ないという風に説明しておけばいい。	
\\	そういうことは 規則[きそく]に 反[はん]するから 出来[でき]ないという 風[ふう]に 説明[せつめい]しておけばいい。
\\	去年はボストン、今年はニューヨークという風に、この協会の大会は毎年東部の大都市で開かれている。	
\\	去年[きょねん]はボストン、 今年[ことし]はニューヨークという 風[ふう]に、この 協会[きょうかい]の 大会[たいかい]は 毎年[まいねん] 東部[とうぶ]の 大都市[だいとし]で 開[ひら]かれている。
\\	ジョンはケーとと、マイクはシンディーと、という風にみんながパーティーを見つけてペアになった。	
\\	ジョンはケーとと、マイクはシンディーと、という 風[ふう]にみんながパーティーを 見[み]つけてペアになった。
\\	彼がこの集まりに来ないということは考えられない。	
\\	彼[かれ]がこの 集[あつ]まりに 来[こ]ないということは 考[かんが]えられない。
\\	手紙が戻って来たということは彼女はもうこの住所にはいないということだ。	
\\	手紙[てがみ]が 戻[もど]って 来[き]たということは 彼女[かのじょ]はもうこの 住所[じゅうしょ]にはいないということだ。
\\	彼が真面目に日本語を勉強していないということは本気で日本で仕事をする気がないのだ。	
\\	彼[かれ]が 真面目[まじめ]に 日本語[にほんご]を 勉強[べんきょう]していないということは 本気[ほんき]で 日本[にほん]で 仕事[しごと]をする 気[き]がないのだ。
\\	これだけ丁寧に説明すれば、分かれないということはあり得ない。	
\\	これだけ 丁寧[ていねい]に 説明[せつめい]すれば、 分[わ]かれないということはあり 得[え]ない。
\\	空がこんなに赤いということは何かよくないことが起こる前兆かもしれない。	
\\	空[そら]がこんなに 赤[あか]いということは 何[なに]かよくないことが 起[お]こる 前兆[ぜんちょう]かもしれない。
\\	彼が金を貸してくれたということは私は彼に信用されているということだ。	
\\	彼[かれ]が 金[きん]を 貸[か]してくれたということは 私[わたし]は 彼[かれ]に 信用[しんよう]されているということだ。
\\	言うことを聞かないということは体罰を与えるしかないということだ。	
\\	言[い]うことを 聞[き]かないということは 体罰[たいばつ]を 与[あた]えるしかないということだ。
\\	私がこれだけ言っても分からないということはどういうことなんでしょう。	
\\	私[わたし]がこれだけ 言[い]っても 分[わ]からないということはどういうことなんでしょう。
\\	彼女がパーティーをするということは試験に通ったのだ。	
\\	彼女[かのじょ]がパーティーをするということは 試験[しけん]に 通[とお]ったのだ。
\\	彼が毎日授業に来ているということは勉強を続けることにしたのだろう。	
\\	彼[かれ]が 毎日[まいにち] 授業[じゅぎょう]に 来[き]ているということは 勉強[べんきょう]を 続[つづ]けることにしたのだろう。
\\	彼がその集まりに来なかったことは事実だ。	
\\	彼[かれ]がその 集[あつ]まりに 来[こ]なかったことは 事実[じじつ]だ。
\\	彼が離婚したことが事実だ。	
\\	彼[かれ]が 離婚[りこん]したことが 事実[じじつ]だ。
\\	彼がこの集まりに来ないことは確かだ。	
\\	彼[かれ]がこの 集[あつ]まりに 来[こ]ないことは 確[たし]かだ。
\\	何かよくないことが起こりつつあることを感じる。	
\\	何[なに]かよくないことが 起[お]こりつつあることを 感[かん]じる。
\\	我々は社長が謝罪することを要求した。	
\\	我々[われわれ]は 社長[しゃちょう]が 謝罪[しゃざい]することを 要求[ようきゅう]した。
\\	彼がこの集まりに来ないということをみんなに伝えた。	
\\	彼[かれ]がこの 集[あつ]まりに 来[こ]ないということをみんなに 伝[つた]えた。
\\	彼がこの集まりに来ないということを聞いた。	
\\	彼[かれ]がこの 集[あつ]まりに 来[こ]ないということを 聞[き]いた。
\\	四月だというのに、まだ肌寒い。	
\\	四月[しがつ]だというのに、まだ 肌寒[はださむ]い。
\\	あさってはロンドンへ行かなければならないというのに、まだ何も準備していないんですよ。	
\\	あさってはロンドンへ 行[い]かなければならないというのに、まだ 何[なに]も 準備[じゅんび]していないんですよ。
\\	彼は母親が危篤だというのに、映画を見に行った。	
\\	彼[かれ]は 母親[ははおや]が 危篤[きとく]だというのに、 映画[えいが]を 見[み]に 行[い]った。
\\	年末でとても忙しいというのに、うちの人は一体どこへ行ったのだろう。	
\\	年末[ねんまつ]でとても 忙[いそが]しいというのに、うちの 人[ひと]は 一体[いったい]どこへ 行[い]ったのだろう。
\\	小学四年生だというのに、あの子はもう中学の数学をやっている。	
\\	小学[しょうがく] 四年生[よねんせい]だというのに、あの 子[こ]はもう 中学[ちゅうがく]の 数学[すうがく]をやっている。
\\	先生がわざわざ本を貸して下さったというのに、読んでいないの?	
\\	先生[せんせい]がわざわざ 本[ほん]を 貸[か]して 下[くだ]さったというのに、 読[よ]んでいないの?
\\	学生は貧乏だというのに、結構いい車を乗り回しているね。	
\\	学生[がくせい]は 貧乏[びんぼう]だというのに、 結構[けっこう]いい 車[くるま]を 乗り回[のりまわ]しているね。
\\	市役所は閉まっているというのに、妻はそんなことはないと言う。	
\\	市役所[しやくしょ]は 閉[し]まっているというのに、 妻[つま]はそんなことはないと 言[い]う。
\\	市役所は閉まっているというのに、その前に人が集まっている。	
\\	市役所[しやくしょ]は 閉[し]まっているというのに、その 前[まえ]に 人[ひと]が 集[あつ]まっている。
\\	彼はお金があるのになかなか出さない。	
\\	彼[かれ]はお 金[かね]があるのになかなか 出[だ]さない。
\\	彼は資産家の息子でお金が余るほどあるというのになかなか出さない。	
\\	彼[かれ]は 資産[しさん] 家[か]の 息子[むすこ]でお 金[かね]が 余[あま]るほどあるというのになかなか 出[だ]さない。
\\	パソコンというのはパーソナル・コンピュータのことだ。	パソコン 
\\	パソコンというのはパーソナル・コンピュータのことだ。
\\	過労死というのは働きすぎがもとで死ぬということだ。	過労死 
\\	過労[かろう] 死[し]というのは 働[はたら]きすぎがもとで 死[し]ぬということだ。
\\	「足が出る」というのは出費が予定していたより多くなることだ。	"""足が出る
\\	足[あし]が 出[で]る」というのは 出費[しゅっぴ]が 予定[よてい]していたより 多[おお]くなることだ。
\\	「猿も木から落ちる」というのは上手な人も時には失敗することがあるということだ。	"""猿も木から落ちる
\\	猿[さる]も 木[き]から 落[お]ちる」というのは 上手[じょうず]な 人[ひと]も 時[とき]には 失敗[しっぱい]することがあるということだ。
\\	マイカーというのは個人が持っている車のことだ。	マイカー 
\\	マイカーというのは 個人[こじん]が 持[も]っている 車[くるま]のことだ。
\\	裏口入学というのは試験を受けないで学校に入ることだ。	裏口入学 
\\	裏口[うらぐち] 入学[にゅうがく]というのは 試験[しけん]を 受[う]けないで 学校[がっこう]に 入[はい]ることだ。
\\	「口が軽い」というのは人に言ってはいけないことをすぐ言ってしまうことだ。	"""口が軽い
\\	口[くち]が 軽[かる]い」というのは 人[ひと]に 言[い]ってはいけないことをすぐ 言[い]ってしまうことだ。
\\	「頭を絞る」というのはいいアイディアを出そうとして一生懸命考えるということだ。	"""頭を絞る
\\	頭[あたま]を 絞[しぼ]る」というのはいいアイディアを 出[で]そうとして 一生懸命[いっしょうけんめい] 考[かんが]えるということだ。
\\	「猫の額のような」というのは場所がとても狭いことだ。	"""猫の額のような
\\	猫[ねこ]の 額[ひたい]のような」というのは 場所[ばしょ]がとても 狭[せま]いことだ。
\\	「花より団子」というのはきれいなものより食べられる団子の方がいい、外観より実質の方を取るということだ。	"""花より団子
\\	花[はな]より 団子[だんご]」というのはきれいなものより 食[た]べられる 団子[だんご]の 方[ほう]がいい、 外観[がいかん]より 実質[じっしつ]の 方[ほう]を 取[と]るということだ。
\\	カラオケというのは何ですか。	
\\	カラオケ 
\\	カラオケというのは 何[なに]ですか。
\\	「首を切る」というのはどういう意味ですか。	
\\	""首を切る
\\	首[くび]を 切[き]る」というのはどういう 意味[いみ]ですか。
\\	ーカラオケ好きですか。 ーカラオケって?	
\\	ーカラオケ 好[す]きですか。 ーカラオケって?
\\	サンフランシスコというと、あのケーブルカーを思い出す。	
\\	サンフランシスコというと、あのケーブルカーを 思い出[おもいだ]す。
\\	日本語を話すと言うと聞こえはいいんですが、実は簡単な挨拶が出来るだけなんです。	
\\	日本語[にほんご]を 話[はな]すと 言[い]うと 聞[き]こえはいいんですが、 実[じつ]は 簡単[かんたん]な 挨拶[あいさつ]が 出来[でき]るだけなんです。
\\	パリというと、誰でも真っ先にあのエッフェル塔を思い浮かべるだろう。	
\\	パリというと、 誰[だれ]でも 真っ先[まっさき]にあの エッフェル塔[えっふぇるとう]を 思い浮[おもいう]かべるだろう。
\\	柔道と言うと、毎日練習した学生時代を思い出す。	
\\	柔道[じゅうどう]と 言[い]うと、 毎日[まいにち] 練習[れんしゅう]した 学生[がくせい] 時代[じだい]を 思い出[おもいだ]す。
\\	漫画と言うと馬鹿にする人もあるが、あれはあれで面白いのだ。	
\\	漫画[まんが]と 言[い]うと 馬鹿[ばか]にする 人[ひと]もあるが、あれはあれで 面白[おもしろ]いのだ。
\\	東大卒と言うと偉そうに聞こえるが、私は勉強なんか全然しない劣等生だったんです。	
\\	東大[とうだい] 卒[そつ]と 言[い]うと 偉[えら]そうに 聞[き]こえるが、 私[わたし]は 勉強[べんきょう]なんか 全然[ぜんぜん]しない 劣等[れっとう] 生[せい]だったんです。
\\	会社を辞めたと言うと体裁がいいんですが、彼は本当は辞めさせられたんですよ。	
\\	会社[かいしゃ]を 辞[や]めたと 言[い]うと 体裁[ていさい]がいいんですが、 彼[かれ]は 本当[ほんとう]は 辞[や]めさせられたんですよ。
\\	ーカラオケは好きですか? ーカラオケと言いますと?	
\\	ーカラオケは 好[す]きですか? ーカラオケと 言[い]いますと?
\\	ー今、サンフランシスコから友達が来ているんです。 ーサンフランシスコと言えば、田中さんがサンフランシスコに転勤になるそうですよ。	
\\	ー 今[いま]、サンフランシスコから 友達[ともだち]が 来[き]ているんです。 ーサンフランシスコと 言[い]えば、 田中[たなか]さんがサンフランシスコに 転勤[てんきん]になるそうですよ。
\\	今年の夏は涼しいというよりは寒かった。	
\\	今年[ことし]の 夏[なつ]は 涼[すず]しいというよりは 寒[さむ]かった。
\\	カーラは臆病というよりはむしろ用心深い。	
\\	カーラは 臆病[おくびょう]というよりはむしろ 用心深[ようじんぶか]い。
\\	本田さんは教育者というよりむしろ学者に近い。	
\\	本田[ほんだ]さんは 教育[きょういく] 者[しゃ]というよりむしろ 学者[がくしゃ]に 近[ちか]い。
\\	あの人は酒を飲むというよりはむしろ酒に飲まれていると言った方がいい。	
\\	あの 人[ひと]は 酒[さけ]を 飲[の]むというよりはむしろ 酒[さけ]に 飲[の]まれていると 言[い]った 方[ほう]がいい。
\\	私にはこの音楽は楽しいというよりは、むしろやかましい。	
\\	私[わたし]にはこの 音楽[おんがく]は 楽[たの]しいというよりは、むしろやかましい。
\\	あの人が丁寧というよりは、いんぎん無礼だ。	
\\	あの 人[ひと]が 丁寧[ていねい]というよりは、いんぎん 無礼[ぶれい]だ。
\\	彼の説明は簡潔というよりは、むしろ不十分だと言うべきだ。	
\\	彼[かれ]の 説明[せつめい]は 簡潔[かんけつ]というよりは、むしろ 不十分[ふじゅうぶん]だと 言[い]うべきだ。
\\	父は父親というよりは、むしろいい友達という感じでした。	
\\	父[ちち]は 父親[ちちおや]というよりは、むしろいい 友達[ともだち]という 感[かん]じでした。
\\	彼は大学教授というよりは、ビジネスマンだ。	
\\	彼[かれ]は 大学[だいがく] 教授[きょうじゅ]というよりは、ビジネスマンだ。
\\	私は教師ですが、教えているというよりは、勉強させれもらっています。	
\\	私[わたし]は 教師[きょうし]ですが、 教[おし]えているというよりは、 勉強[べんきょう]させれもらっています。
\\	あの母親は子供を愛しているというよりは、むしろ、甘やかしている。	
\\	あの 母親[ははおや]は 子供[こども]を 愛[あい]しているというよりは、むしろ、 甘[あま]やかしている。
\\	彼と話していると、会話をしているというよりは、一方的に話しを聞かされているといった感じだ。	
\\	彼[かれ]と 話[はな]していると、 会話[かいわ]をしているというよりは、 一方[いっぽう] 的[てき]に 話[はな]しを 聞[き]かされているといった 感[かん]じだ。
\\	私が呼んだというより、むしろ彼女の方が押しかけて来たんです。	
\\	私[わたし]が 呼[よ]んだというより、むしろ 彼女[かのじょ]の 方[ほう]が 押[お]しかけて 来[き]たんです。
\\	高木さんは大阪に出張するとかでパーティーに来なかった。	
\\	高木[たかぎ]さんは 大阪[おおさか]に 出張[しゅっちょう]するとかでパーティーに 来[こ]なかった。
\\	友人の竹田は大学の授業が面白くないとかで大学を辞めてしまった。	
\\	友人[ゆうじん]の 竹田[たけだ]は 大学[だいがく]の 授業[じゅぎょう]が 面白[おもしろ]くないとかで 大学[だいがく]を 辞[や]めてしまった。
\\	ミリアムは日米関係に興味があるとかで、大学で国際関係論を専攻した。	
\\	ミリアムは 日[にち] 米[べい] 関係[かんけい]に 興味[きょうみ]があるとかで、 大学[だいがく]で 国際[こくさい] 関係[かんけい] 論[ろん]を 専攻[せんこう]した。
\\	エンジンに故障があったとかで、飛行機の出発が五時間も遅れた。	
\\	エンジンに 故障[こしょう]があったとかで、 飛行機[ひこうき]の 出発[しゅっぱつ]が五 時間[じかん]も 遅[おく]れた。
\\	受講生が少なかったとかで、四年生の日本語のクラスは取りやめになった。	
\\	受講[じゅこう] 生[せい]が 少[すく]なかったとかで、四年生[よねんせい]の 日本語[にほんご]のクラスは 取[と]りやめになった。
\\	田村は殺人現場に居合わせたとかで、警察に逮捕された。	
\\	田村[たむら]は 殺人[さつじん] 現場[げんば]に 居合[いあ]わせたとかで、 警察[けいさつ]に 逮捕[たいほ]された。
\\	ジョンは仕事があるとかで映画に行かない。	
\\	ジョンは 仕事[しごと]があるとかで 映画[えいが]に 行[い]かない。
\\	私は仕事があるから映画に行かない。	
\\	私[わたし]は 仕事[しごと]があるから 映画[えいが]に 行[い]かない。
\\	先生に相談したところ、ぜひ大学院に行くよう薦められた。	
\\	先生[せんせい]に 相談[そうだん]したところ、ぜひ 大学院[だいがくいん]に 行[い]くよう 薦[すす]められた。
\\	部長に頼んでみたところ、喜んで引き受けてくれた。	
\\	部長[ぶちょう]に 頼[たの]んでみたところ、 喜[よろこ]んで 引き受[ひきう]けてくれた。
\\	友達に話したところ、しばらく考えさせてほしいと言った。	
\\	友達[ともだち]に 話[はな]したところ、しばらく 考[かんが]えさせてほしいと 言[い]った。
\\	これまでに書いた論文を本にしたところ、意外によく売れた。	
\\	これまでに 書[か]いた 論文[ろんぶん]を 本[ほん]にしたところ、 意外[いがい]によく 売[う]れた。
\\	人に薦められてヨガを始めたところ、非常に効果があることが分かった。	
\\	人[ひと]に 薦[すす]められてヨガを 始[はじ]めたところ、 非常[ひじょう]に 効果[こうか]があることが 分[わ]かった。
\\	冗談のつもりで言ったところ、思わぬ結果になって驚いている。	
\\	冗談[じょうだん]のつもりで 言[い]ったところ、 思[おも]わぬ 結果[けっか]になって 驚[おどろ]いている。
\\	その仕事に興味のある旨、手紙を出したところ、すぐに面接をしたいとの返事があった。	
\\	その 仕事[しごと]に 興味[きょうみ]のある 旨[むね]、 手紙[てがみ]を 出[だ]したところ、すぐに 面接[めんせつ]をしたいとの 返事[へんじ]があった。
\\	日本式の部屋に泊まりたいと思って旅館を予約した。ところが、通された部屋は洋式だった。	
\\	日本[にほん] 式[しき]の 部屋[へや]に 泊[と]まりたいと 思[おも]って 旅館[りょかん]を 予約[よやく]した。ところが、 通[とお]された 部屋[へや]は 洋式[ようしき]だった。
\\	ー彼女にプロポースしたんだろう?どうだった? ーところが、だめだったんだ。	ー
\\	ー 彼女[かのじょ]にプロポースしたんだろう?どうだった? ーところが、だめだったんだ。
\\	私の娘はアメリカにいる時きれいな英語を話していた。ところが、日本へ帰って一年もたたないうちにきれいに忘れてしまった。	
\\	私[わたし]の 娘[むすめ]はアメリカにいる 時[とき]きれいな 英語[えいご]を 話[はな]していた。ところが、 日本[にほん]へ 帰[かえ]って 一年[いちねん]もたたないうちにきれいに 忘[わす]れてしまった。
\\	日本から来た有名な先生の講演があるというので行ってみた。ところが、その先生は英語が下手で何をいっているのか全然分からなかった。	
\\	日本[にほん]から 来[き]た 有名[ゆうめい]な 先生[せんせい]の 講演[こうえん]があるというので 行[い]ってみた。ところが、その 先生[せんせい]は 英語[えいご]が 下手[へた]で 何[なに]をいっているのか 全然[ぜんぜん] 分[わ]からなかった。
\\	先生は私達に遅刻してはいけないと何度も言った。ところが、次の日、時間通りに言ってみると先生はまだ来ていなかった。	
\\	先生[せんせい]は 私[わたし] 達[たち]に 遅刻[ちこく]してはいけないと 何[なん] 度[ど]も 言[い]った。ところが、 次[つぎ]の 日[ひ]、 時間[じかん] 通[どお]りに 言[い]ってみると 先生[せんせい]はまだ 来[き]ていなかった。
\\	アメリカへ来ればいくらでも仕事はあると言われた。ところが、来てみるとアメリカは不況でどこにも仕事はなかった。	
\\	アメリカへ 来[く]ればいくらでも 仕事[しごと]はあると 言[い]われた。ところが、 来[き]てみるとアメリカは 不況[ふきょう]でどこにも 仕事[しごと]はなかった。
\\	アメリカではのこぎりは押して切る。ところが、日本では引いて切る。	
\\	アメリカではのこぎりは 押[お]して 切[き]る。ところが、 日本[にほん]では 引[ひ]いて 切[き]る。
\\	ー今晩のパーティー、君も来てくれるね。
\\	ーところが、あいにく行けないの。
\\	ー 今晩[こんばん]のパーティー、 君[きみ]も 来[き]てくれるね。 ーところが、あいにく 行[い]けないの。
\\	ー彼女は結婚しているんでしょう?
\\	ーそう見えるでしょう?ところが、まだ独身なんですって。
\\	ー 彼女[かのじょ]は 結婚[けっこん]しているんでしょう? ーそう 見[み]えるでしょう?ところが、まだ 独身[どくしん]なんですって。
\\	この本は高い。だがいい本だ。	
\\	この 本[ほん]は 高[たか]い。だがいい 本[ほん]だ。
\\	結局負けてしまった。けれども我々はベストを尽くした。	
\\	結局[けっきょく] 負[ま]けてしまった。けれども 我々[われわれ]はベストを 尽[つ]くした。
\\	ここの冬は寒い。しかし夏は涼しい。	
\\	ここの 冬[ふゆ]は 寒[さむ]い。しかし 夏[なつ]は 涼[すず]しい。
\\	ー試験どうだった? ーそれがだめだったんだ。	
\\	ー 試験[しけん]どうだった? ーそれがだめだったんだ。
\\	よく勉強してから、試験にどんな問題が出ようとも大丈夫だ。	
\\	よく 勉強[べんきょう]してから、 試験[しけん]にどんな 問題[もんだい]が 出[で]ようとも 大丈夫[だいじょうぶ]だ。
\\	いかに家事が忙しくとも母は文句一つ言わなかった。	
\\	いかに 家事[かじ]が 忙[いそが]しくとも 母[はは]は 文句[もんく] 一[ひと]つ 言[い]わなかった。
\\	どんなに健康であろうとも、一年に一回は健康診断を受けるべきだ。	
\\	どんなに 健康[けんこう]であろうとも、一年[いちねん]に 一回[いっかい]は 健康[けんこう] 診断[しんだん]を 受[う]けるべきだ。
\\	このプロジェクトは遅くとも来年の四月には完了するだろう。	
\\	このプロジェクトは 遅[おそ]くとも 来年[らいねん]の 四月[しがつ]には 完了[かんりょう]するだろう。
\\	父は何が起きようともいつも平然としている。	
\\	父[ちち]は 何[なに]が 起[お]きようともいつも 平然[へいぜん]としている。
\\	私は世界のどこに住もうとも一向に構わない。	
\\	私[わたし]は 世界[せかい]のどこに 住[す]もうとも 一向[いっこう]に 構[かま]わない。
\\	万一失敗しようとも、簡単にあきらめてはいけない。	
\\	万一[まんいち] 失敗[しっぱい]しようとも、 簡単[かんたん]にあきらめてはいけない。
\\	いかに頭がよくとも、努力しなければいい仕事は出来ない。	
\\	いかに 頭[あたま]がよくとも、 努力[どりょく]しなければいい 仕事[しごと]は 出来[でき]ない。
\\	どんなに品物が安かろうとも、質が悪ければ金を捨てることになる。	
\\	どんなに 品物[しなもの]が 安[やす]かろうとも、 質[しつ]が 悪[わる]ければ 金[かね]を 捨[す]てることになる。
\\	金が十分なくとも、生活を楽しむことは出来るはずだ。	
\\	金[きむ]が 十分[じゅうぶん]なくとも、 生活[せいかつ]を 楽[たの]しむことは 出来[でき]るはずだ。
\\	どんなに運動が好きであろうとも、運動のやりすぎはかえって悪い。	
\\	どんなに 運動[うんどう]が 好[す]きであろうとも、 運動[うんどう]のやりすぎはかえって 悪[わる]い。
\\	この手紙はあなたがいくら読みたくとも読ませるわけにはいかない。	
\\	この 手紙[てがみ]はあなたがいくら 読[よ]みたくとも 読[よ]ませるわけにはいかない。
\\	僕は少なくとも一年に三回は海外出張をしています。	
\\	僕[ぼく]は 少[すく]なくとも 一年[いちねん]に 三回[さんかい]は 海外[かいがい] 出張[しゅっちょう]をしています。
\\	私は四時間歩かされても疲れなかった。	
\\	私[わたし]は 四時間[よじかん] 歩[ある]かされても 疲[つか]れなかった。
\\	何を食べさせられようとも喜んで食べる。	
\\	何[なに]を 食[た]べさせられようとも 喜[よろこ]んで 食[た]べる。
\\	吉田さんが初代の委員長となった。	
\\	吉田[よしだ]さんが 初代[しょだい]の 委員[いいん] 長[ちょう]となった。
\\	その土地は結局国の所有物となった。	
\\	その 土地[とち]は 結局[けっきょく] 国[こく]の 所有[しょゆう] 物[ぶつ]となった。
\\	戦後義務教育は九年間となった。	
\\	戦後[せんご] 義務[ぎむ] 教育[きょういく]は 九年間[きゅうねんかん]となった。
\\	彼らの結婚は悲劇的な結果となった。	
\\	彼[かれ]らの 結婚[けっこん]は 悲劇[ひげき] 的[てき]な 結果[けっか]となった。
\\	彼は証拠不十分で無罪となった。	
\\	彼[かれ]は 証拠[しょうこ] 不十分[ふじゅうぶん]で 無罪[むざい]となった。
\\	その試合は雨のため中止となった。	
\\	その 試合[しあい]は 雨[あめ]のため 中止[ちゅうし]となった。
\\	私は機械のこととなると何も分からない。	
\\	私[わたし]は 機械[きかい]のこととなると 何[なに]も 分[わ]からない。
\\	彼が手伝ってくれないとなると、誰かほかの人を頼むしかない。	
\\	彼[かれ]が 手伝[てつだ]ってくれないとなると、 誰[だれ]かほかの 人[ひと]を 頼[たの]むしかない。
\\	勉はギャンブルとなると人が変わってしまう。	
\\	勉[つとむ]はギャンブルとなると 人[ひと]が 変[か]わってしまう。
\\	日本語が出来る科学者となると数がかなり限られてくる。	
\\	日本語[にほんご]が 出来[でき]る 科学[かがく] 者[しゃ]となると 数[かず]がかなり 限[かぎ]られてくる。
\\	何でも仕事となると楽なものはない。	
\\	何[なに]でも 仕事[しごと]となると 楽[らく]なものはない。
\\	彼が参加できないとなるとこのプロジェクトは考え直さなければならない。	
\\	彼[かれ]が 参加[さんか]できないとなるとこのプロジェクトは 考え直[かんがえなお]さなければならない。
\\	秘密が漏れたとなるとこの計画は変更しなければならない。	
\\	秘密[ひみつ]が 漏[も]れたとなるとこの 計画[けいかく]は 変更[へんこう]しなければならない。
\\	彼が新しい社長になるとなると経営方針がかなり変わるだろう。	
\\	彼[かれ]が 新[あたら]しい 社長[しゃちょう]になるとなると 経営[けいえい] 方針[ほうしん]がかなり 変[か]わるだろう。
\\	この仕事は簡単そうに見えるが、一人でするとなると結構大変だ。	
\\	この 仕事[しごと]は 簡単[かんたん]そうに 見[み]えるが、一人[ひとり]でするとなると 結構[けっこう] 大変[たいへん]だ。
\\	今自分の家を建てるとするとどんな家を建てますか。	
\\	今[こん] 自分[じぶん]の 家[いえ]を 建[た]てるとするとどんな 家[いえ]を 建[た]てますか。
\\	説明書の通りにやったが、うまくいかなかった。	
\\	説明[せつめい] 書[しょ]の 通[とお]りにやったが、うまくいかなかった。
\\	私の言う通り体を動かして下さい。	
\\	私[わたし]の 言[い]う 通[とお]り 体[からだ]を 動[うご]かして 下[くだ]さい。
\\	思った通り彼が一番先に来ていた。	
\\	思[おも]った 通[とお]り 彼[かれ]が 一番[いちばん] 先[さき]に 来[き]ていた。
\\	結果はやはり予想した通りだった。	
\\	結果[けっか]はやはり 予想[よそう]した 通[とお]りだった。
\\	この地図の通りに行けば間違いなく行けます。	
\\	この 地図[ちず]の 通[とお]りに 行[い]けば 間違[まちが]いなく 行[い]けます。
\\	私はただ言われた通りやっているだけです。	
\\	私[わたし]はただ 言[い]われた 通[とお]りやっているだけです。
\\	ここに書いてある通りにすれば誰にでも出来ます。	
\\	ここに 書[か]いてある 通[とお]りにすれば 誰[だれ]にでも 出来[でき]ます。
\\	予想した通り誰も宿題をやってこなかった。	
\\	予想[よそう]した 通[とお]り 誰[だれ]も 宿題[しゅくだい]をやってこなかった。
\\	心配していた通りペンキが足りなくなった。	
\\	心配[しんぱい]していた 通[とお]りペンキが 足[た]りなくなった。
\\	内容はこの間説明した通りです。	
\\	内容[ないよう]はこの 間[あいだ] 説明[せつめい]した 通[とお]りです。
\\	これは私が考えていた通りのデザインです。	
\\	これは 私[わたし]が 考[かんが]えていた 通[とお]りのデザインです。
\\	ーこれは何語か分かりますか。 ーアラビア語です。 ーはい、その通りです。	
\\	ーこれは 何[なん] 語[ご]か 分[わ]かりますか。 ーアラビア 語[ご]です。 ーはい、その 通[とお]りです。
\\	思っている通りを書いて下さい。	
\\	思[おも]っている 通[とお]りを 書[か]いて 下[くだ]さい。
\\	指示通りに動いて下さい。	
\\	指示[しじ] 通[どお]りに 動[うご]いて 下[くだ]さい。
\\	これは予定通りの行動です。	
\\	これは 予定[よてい] 通[どお]りの 行動[こうどう]です。
\\	春になったように暖かい。	
\\	春[はる]になったように 暖[あたた]かい。
\\	彼女は日本人のように日本語を話す。	
\\	彼女[かのじょ]は 日本人[にほんじん]のように 日本語[にほんご]を 話[はな]す。
\\	温度は一定だとする。	
\\	温度[おんど]は 一定[いってい]だとする。
\\	一日に五百マイル運転できるとすれば、何日で着きますか。	
\\	一日[いちにち]に 五百[ごひゃく]マイル 運転[うんてん]できるとすれば、 何[なん] 日[にち]で 着[つ]きますか。
\\	温暖化現象がこのまま続いたとしよう。	
\\	温暖[おんだん] 化[か] 現象[げんしょう]がこのまま 続[つづ]いたとしよう。
\\	今、空気抵抗は無視できるものとする。	
\\	今[いま]、 空気[くうき] 抵抗[ていこう]は 無視[むし]できるものとする。
\\	男性の女性に対するこのような行為は性的いやがらせであるとする。	
\\	男性[だんせい]の 女性[じょせい]に 対[たい]するこのような 行為[こうい]は 性的[せいてき]いやがらせであるとする。
\\	西洋では日本のいわゆる私小説は小説ではないとされる。	
\\	西洋[せいよう]では 日本[にほん]のいわゆる 私小説[ししょうせつ]は 小説[しょうせつ]ではないとされる。
\\	私はアメリカを第二の故郷とすることに違和感を覚えない。	
\\	私[わたし]はアメリカを 第二[だいに]の 故郷[こきょう]とすることに 違和感[いわかん]を 覚[おぼ]えない。
\\	夏の九週間集中日本語講座の単位は六単位とする。	
\\	夏[なつ]の 九週間[きゅうしゅうかん] 集中[しゅうちゅう] 日本語[にほんご] 講座[こうざ]の 単位[たんい]は 六単位[ろくたんい]とする。
\\	利息は年3.5
\\	とする。	
\\	利息[りそく]は 年[ねん] 
\\	とする。
\\	日本経済がこのまま下降線をたどるとすると、日本語ブームもやがては終わるかもしれない。	
\\	日本[にほん] 経済[けいざい]がこのまま 下降[かこう] 線[せん]をたどるとすると、 日本語[にほんご]ブームもやがては 終[お]わるかもしれない。
\\	日本が集団主義的な国だとすれば、欧米は、やはり、個人主儀的な国と言えるだろう。	
\\	日本[にほん]が 集団[しゅうだん] 主義[しゅぎ] 的[てき]な 国[くに]だとすれば、 欧米[おうべい]は、やはり、 個人[こじん] 主儀[しゅぎ] 的[てき]な 国[くに]と 言[い]えるだろう。
\\	為替レートを一ドル百円とすれば、この家はいくらになりますか。	
\\	為替[かわせ]レートを 一[いち]ドル 百[ひゃく] 円[えん]とすれば、この 家[いえ]はいくらになりますか。
\\	で、
\\	が 
\\	だとすれば、
\\	は 
\\	になる。	
\\	20で、 
\\	が 8だとすれば、 
\\	は 12になる。
\\	一男を養子とすることには依存はない。	
\\	一男[かずお]を 養子[ようし]とすることには 依存[いぞん]はない。
\\	大陸の中国が完全に資本主義社会になったとしよう。中国は何年後に経済大国になるであろうか。	
\\	大陸[たいりく]の 中国[ちゅうごく]が 完全[かんぜん]に 資本[しほん] 主義[しゅぎ] 社会[しゃかい]になったとしよう。 中国[ちゅうごく]は 何[なん] 年[ねん] 後[ご]に 経済[けいざい] 大国[たいこく]になるであろうか。
\\	ニューヨークの物価指数を100としよう。	
\\	ニューヨークの 物価[ぶっか] 指数[しすう]を 100としよう。
\\	すぐ行く方がいいよ。	
\\	すぐ 行[い]く 方[ほう]がいいよ。
\\	その映画は面白いと思う。	
\\	その 映画[えいが]は 面白[おもしろ]いと 思[おも]う。
\\	その映画が面白いと思う。	
\\	その 映画[えいが]が 面白[おもしろ]いと 思[おも]う。
\\	その映画を面白いと思う。	
\\	その 映画[えいが]を 面白[おもしろ]いと 思[おも]う。
\\	温度は一定にする。	
\\	温度[おんど]は 一定[いってい]にする。
\\	一ドルは三百六十円とする。	
\\	一[いち]ドルは 三百六十[さんびゃくろくじゅう]円[えん]とする。
\\	一ドルは三百六十円にする。	
\\	一[いち]ドルは 三百六十[さんびゃくろくじゅう] 円[えん]にする。
\\	氷を水にする。	
\\	氷[こおり]を 水[みず]にする。
\\	魔法使いはカエルを王子様にする。	
\\	魔法使[まほうつか]いはカエルを 王子[おうじ] 様[さま]にする。
\\	一日に五百マイル運動できるとなれば、何日で着きますか。	
\\	一日[いちにち]に 五百[ごひゃく]マイル 運動[うんどう]できるとなれば、 何[なん] 日[にち]で 着[つ]きますか。
\\	日本経済がこのまま下降線をたどるとなると、日本語ブームもやがては終わるかもしれない。	
\\	日本[にほん] 経済[けいざい]がこのまま 下降[かこう] 線[せん]をたどるとなると、 日本語[にほんご]ブームもやがては 終[お]わるかもしれない。
\\	あの人はちょっとしたことでかっとする。	
\\	あの 人[ひと]はちょっとしたことでかっとする。
\\	あの男の人は目がギョロリとしている。	
\\	あの 男[おとこ]の 人[ひと]は 目[め]がギョロリとしている。
\\	姉は足がすらっとしている。	
\\	姉[あね]は 足[あし]がすらっとしている。
\\	今日の天気はからっとしていて、気持ちがいい。	
\\	今日[きょう]の 天気[てんき]はからっとしていて、 気持[きも]ちがいい。
\\	昔の恋人を東京駅で見かけて、はっとした。	
\\	昔[むかし]の 恋人[こいびと]を 東京[とうきょう] 駅[えき]で 見[み]かけて、はっとした。
\\	ジョンは遅刻して先生に叱られたのに、けろっとしていた。	
\\	ジョンは 遅刻[ちこく]して 先生[せんせい]に 叱[しか]られたのに、けろっとしていた。
\\	部長は部下に反対された時、むっとした。	
\\	部長[ぶちょう]は 部下[ぶか]に 反対[はんたい]された 時[とき]、むっとした。
\\	日本語の試験が思ったよりよく出来たのでほっとした。	
\\	日本語[にほんご]の 試験[しけん]が 思[おも]ったよりよく 出来[でき]たのでほっとした。
\\	難しいと思っていた日本語の試験がやさしかったのでほっとした。	
\\	難[むずか]しいと 思[おも]っていた 日本語[にほんご]の 試験[しけん]がやさしかったのでほっとした。
\\	休みの日は何もしないで、ぼうっとしているのが好きです。	
\\	休[やす]みの 日[ひ]は 何[なに]もしないで、ぼうっとしているのが 好[す]きです。
\\	今日は春のように、一日中ぽかぽかしていた。	
\\	今日[きょう]は 春[はる]のように、一日[いちにち] 中[じゅう]ぽかぽかしていた。
\\	あの女の人は目がきらきらしている。	
\\	あの 女[おんな]の 人[ひと]は 目[め]がきらきらしている。
\\	もうすぐ日本に行けると思うと、わくわくする。	
\\	もうすぐ 日本[にほん]に 行[い]けると 思[おも]うと、わくわくする。
\\	先生の顔を見た途端に宿題があったことを思い出した。	
\\	先生[せんせい]の 顔[かお]を 見[み]た 途端[とたん]に 宿題[しゅくだい]があったことを 思い出[おもいだ]した。
\\	ドアを開けたら、途端にカレーの匂いがした。	
\\	ドアを 開[あ]けたら、 途端[とたん]にカレーの 匂[にお]いがした。
\\	部屋に入った途端電話が鳴った。	
\\	部屋[へや]に 入[はい]った 途端[とたん] 電話[でんわ]が 鳴[な]った。
\\	その日本の写真を見た途端に家族に会いたくなった。	
\\	その 日本[にほん]の 写真[しゃしん]を 見[み]た 途端[とたん]に 家族[かぞく]に 会[あ]いたくなった。
\\	彼女は私の顔を見た途端泣き出した。	
\\	彼女[かのじょ]は 私[わたし]の 顔[かお]を 見[み]た 途端[とたん] 泣[な]き 出[だ]した。
\\	コンセントを差し込んだら、途端に家中の電気が消えた。	
\\	コンセントを 差し込[さしこ]んだら、 途端[とたん]に 家中[いえじゅう]の 電気[でんき]が 消[き]えた。
\\	その牛乳を飲んだら、途端に腹が痛み出した。	
\\	その 牛乳[ぎゅうにゅう]を 飲[の]んだら、 途端[とたん]に 腹[はら]が 痛[いた]み 出[だ]した。
\\	助かったと思った。その途端、力が抜けて気を失ってしまった。	
\\	助[たす]かったと 思[おも]った。その 途端[とたん]、 力[ちから]が 抜[ぬ]けて 気[き]を 失[うしな]ってしまった。
\\	洋子は先生の顔を見た途端、立ち上がった。	
\\	洋子[ようこ]は 先生[せんせい]の 顔[かお]を 見[み]た 途端[とたん]、 立ち上[たちあ]がった。
\\	私は先生の顔を見ると同時に立ち上がった。	
\\	私[わたし]は 先生[せんせい]の 顔[かお]を 見[み]ると 同時[どうじ]に 立ち上[たちあ]がった。
\\	群衆が遠くに見えると同時に太鼓の音も聞こえる。	
\\	群衆[ぐんしゅう]が 遠[とお]くに 見[み]えると 同時[どうじ]に 太鼓[たいこ]の 音[おと]も 聞[き]こえる。
\\	朝から降りそうだった雨がとうとう降り出した。	
\\	朝[あさ]から 降[ふ]りそうだった 雨[あめ]がとうとう 降り出[ふりだ]した。
\\	弟は医者の忠告を聞かないでたばこを吸い続け、とうとう肺がんになってしまった。	
\\	弟[おとうと]は 医者[いしゃ]の 忠告[ちゅうこく]を 聞[き]かないでたばこを 吸[す]い 続[つづ]け、とうとう 肺[はい]がんになってしまった。
\\	日本語の先生があまりにも厳しかったから、その学生はとうとう日本語が嫌いになってしまった。	
\\	日本語[にほんご]の 先生[せんせい]が 余[あま]りにも 厳[きび]しかったから、その 学生[がくせい]はとうとう 日本語[にほんご]が 嫌[きら]いになってしまった。
\\	その数学の問題は何時間かけても解けなかったので、とうとうあきらめてしまった。	
\\	その 数学[すうがく]の 問題[もんだい]は 何[なん] 時間[じかん]かけても 解[と]けなかったので、とうとうあきらめてしまった。
\\	ビルは妻との関係がだんだん悪化し、とうとう別れてしまった。	
\\	ビルは 妻[つま]との 関係[かんけい]がだんだん 悪化[あっか]し、とうとう 別[わか]れてしまった。
\\	この会社で働くのもとうとう今日が最後となった。	
\\	この 会社[かいしゃ]で 働[はたら]くのもとうとう 今日[きょう]が 最後[さいご]となった。
\\	十五年も飼っていた猫がとうとう老衰で死んでしまった。	
\\	十五年[じゅうごねん]も 飼[か]っていた 猫[ねこ]がとうとう 老衰[ろうすい]で 死[し]んでしまった。
\\	とうとう博士論文を書いてしまった。	
\\	とうとう 博士[はくし] 論文[ろんぶん]を 書[か]いてしまった。
\\	私達の結婚の日がとうとうやって来た。	
\\	私[わたし] 達[たち]の 結婚[けっこん]の 日[ひ]がとうとうやって 来[き]た。
\\	三時間も考えてやっと数学の問題が解けた。	
\\	三時間[さんじかん]も 考[かんが]えてやっと 数学[すうがく]の 問題[もんだい]が 解[と]けた。
\\	探していた本が神田の古本屋でやっと見つかった。	
\\	探[さが]していた 本[ほん]が 神田[かんだ]の 古本屋[ふるほんや]でやっと 見[み]つかった。
\\	冬がとうとうやって来た。	
\\	冬[ふゆ]がとうとうやって 来[き]た。
\\	その男は会社をやっと辞めた。	
\\	その 男[おとこ]は 会社[かいしゃ]をやっと 辞[や]めた。
\\	スミスさんはとうとうパーティーに来ませんでした。	
\\	スミスさんはとうとうパーティーに 来[き]ませんでした。
\\	なくした財布はとうとう見つかりませんでした。	
\\	なくした 財布[さいふ]はとうとう 見[み]つかりませんでした。
\\	待ちに待った夏休みはついにやって来た。	
\\	待[ま]ちに 待[ま]った 夏休[なつやす]みはついにやって 来[き]た。
\\	キャシーはとうとうパーテイーに現れなかった。	
\\	キャシーはとうとうパーテイーに 現[あらわ]れなかった。
\\	エドにはついに会うことが出来なかった。	
\\	エドにはついに 会[あ]うことが 出来[でき]なかった。
\\	スタインベックはヘミングウェイと共にアメリカの代表的作家の一人だ。	
\\	スタインベックはヘミングウェイと 共[とも]にアメリカの 代表[だいひょう] 的[てき] 作家[さっか]の 一人[ひとり]だ。
\\	私は職場の同僚達と共にそのデモに参加した。	
\\	私[わたし]は 職場[しょくば]の 同僚[どうりょう] 達[たち]と 共[とも]にそのデモに 参加[さんか]した。
\\	病気が回復すると共に食欲も出てきた。	
\\	病気[びょうき]が 回復[かいふく]すると 共[とも]に 食欲[しょくよく]も 出[で]てきた。
\\	彼は科学者であると共に哲学者でもあった。	
\\	彼[かれ]は 科学[かがく] 者[しゃ]であると 共[とも]に 哲学[てつがく] 者[しゃ]でもあった。
\\	原子力は石油と共に重要なエネルギー源だ。	
\\	原子力[げんしりょく]は 石油[せきゆ]と 共[とも]に 重要[じゅうよう]なエネルギー 源[げん]だ。
\\	轟音と共にジェット機が飛び去った。	
\\	轟音[ごうおん]と 共[とも]に ジェット機[じぇっとき]が 飛[と]び 去[さ]った。
\\	武士階級は封建制と共に滅びた。	
\\	武士[ぶし] 階級[かいきゅう]は 封建[ほうけん] 制[せい]と 共[とも]に 滅[ほろ]びた。
\\	終戦と共に激しいインフレが人々を襲った。	
\\	終戦[しゅうせん]と 共[とも]に 激[はげ]しいインフレが 人々[ひとびと]を 襲[おそ]った。
\\	景気の回復と共に失業率が下がってきた。	
\\	景気[けいき]の 回復[かいふく]と 共[とも]に 失業[しつぎょう] 率[りつ]が 下[さ]がってきた。
\\	年をとると共に耳が遠くなってきた。	
\\	年[とし]をとると 共[とも]に 耳[みみ]が 遠[とお]くなってきた。
\\	調査が進むと共に新しい事実が次々に明らかになっていった。	
\\	調査[ちょうさ]が 進[すす]むと 共[とも]に 新[あたら]しい 事実[じじつ]が 次々[つぎつぎ]に 明[あき]らかになっていった。
\\	文法を復習すると共に漢字の勉強も忘れてはならない。	
\\	文法[ぶんぽう]を 復習[ふくしゅう]すると 共[とも]に 漢字[かんじ]の 勉強[べんきょう]も 忘[わす]れてはならない。
\\	現在のエネルギー利用技術を改良すると共に新しいエネルギー源を探すことが急務である。	
\\	現在[げんざい]のエネルギー 利用[りよう] 技術[ぎじゅつ]を 改良[かいりょう]すると 共[とも]に 新[あたら]しいエネルギー 源[げん]を 探[さが]すことが 急務[きゅうむ]である。
\\	その提案は建設的であると共に革命的であった。	
\\	その 提案[ていあん]は 建設[けんせつ] 的[てき]であると 共[とも]に 革命[かくめい] 的[てき]であった。
\\	体の弱い人が早く死ぬとは限らない。	
\\	体[からだ]の 弱[よわ]い 人[ひと]が 早[はや]く 死[し]ぬとは 限[かぎ]らない。
\\	高い物がいつも安い物よりいいとは限らない。	
\\	高[たか]い 物[もの]がいつも 安[やす]い 物[もの]よりいいとは 限[かぎ]らない。
\\	日本に長く住んでいるから日本語が上手だとは限らない。	
\\	日本[にほん]に 長[なが]く 住[す]んでいるから 日本語[にほんご]が 上手[じょうず]だとは 限[かぎ]らない。
\\	日本に行っても日本語が上手になるとは限らない。	
\\	日本[にほん]に 行[い]っても 日本語[にほんご]が 上手[じょうず]になるとは 限[かぎ]らない。
\\	お金があれば幸福になれるとは限らない。	
\\	お 金[かね]があれば 幸福[こうふく]になれるとは 限[かぎ]らない。
\\	漫画が子供の教育にとって有害だとは限らない。	
\\	漫画[まんが]が 子供[こども]の 教育[きょういく]にとって 有害[ゆうがい]だとは 限[かぎ]らない。
\\	日本人の使う日本語がいつも正しいとは限らない。	
\\	日本人[にほんじん]の 使[つか]う 日本語[にほんご]がいつも 正[ただ]しいとは 限[かぎ]らない。
\\	スポーツ選手が必ずしも健康だとは限らない。	
\\	スポーツ 選手[せんしゅ]が 必[かなら]ずしも 健康[けんこう]だとは 限[かぎ]らない。
\\	技術の進歩が必ずしも我々に繁栄をもたらすとは限らない。	
\\	技術[ぎじゅつ]の 進歩[しんぽ]が 必[かなら]ずしも 我々[われわれ]に 繁栄[はんえい]をもたらすとは 限[かぎ]らない。
\\	先生がそう言ったからと言ってそれが正しいとは限らない。	
\\	先生[せんせい]がそう 言[い]ったからと 言[い]ってそれが 正[ただ]しいとは 限[かぎ]らない。
\\	薬を飲んだからと言ってよくなるとは限らない。	
\\	薬[くすり]を 飲[の]んだからと 言[い]ってよくなるとは 限[かぎ]らない。
\\	頭がよくても人生に成功するとは限らない。	
\\	頭[あたま]がよくても 人生[じんせい]に 成功[せいこう]するとは 限[かぎ]らない。
\\	実力があっても失業しないとは限らない。	
\\	実力[じつりょく]があっても 失業[しつぎょう]しないとは 限[かぎ]らない。
\\	悪い家庭環境に育てば必ず悪い子供になるとは限らない。	
\\	悪[わる]い 家庭[かてい] 環境[かんきょう]に 育[そだ]てば 必[かなら]ず 悪[わる]い 子供[こども]になるとは 限[かぎ]らない。
\\	時間がたくさんあればいい論文が書けるとは限らない。	
\\	時間[じかん]がたくさんあればいい 論文[ろんぶん]が 書[か]けるとは 限[かぎ]らない。
\\	ベルリンの壁が崩れた。これはつまり、民主化の嵐が始まったということだ。	
\\	ベルリンの 壁[かべ]が 崩[くず]れた。これはつまり、 民主[みんしゅ] 化[か]の 嵐[あらし]が 始[はじ]まったということだ。
\\	私の母の弟、つまり、叔父が最近離婚した。	
\\	私[わたし]の 母[はは]の 弟[おとうと]、つまり、 叔父[おじ]が 最近[さいきん] 離婚[りこん]した。
\\	アメリカに一年間一人で出張するんです。つまり、単身赴任ということですよ。	
\\	アメリカに 一年間[いちねんかん] 一人[ひとり]で 出張[しゅっちょう]するんです。つまり、 単身[たんしん] 赴任[ふにん]ということですよ。
\\	運動をした後は気分がよくなりますね、つまり、運動でストレスがなくなるということなんですよ。	
\\	運動[うんどう]をした 後[あと]は 気分[きぶん]がよくなりますね、つまり、 運動[うんどう]でストレスがなくなるということなんですよ。
\\	ー部長、このプロジェクトは出来ますか。 ーそうだね。いいプロジェクトだと思うけど、予算がどうもねえ。 ーつまり、出来ないということですね。	
\\	ー 部長[ぶちょう]、このプロジェクトは 出来[でき]ますか。 ーそうだね。いいプロジェクトだと 思[おも]うけど、 予算[よさん]がどうもねえ。 ーつまり、 出来[でき]ないということですね。
\\	日本の父親は子供がまだ寝ているときに家を出て、子供達が寝てから家に帰ってくる。つまり、父親不在ということだ。	
\\	日本[にほん]の 父親[ちちおや]は 子供[こども]がまだ 寝[ね]ているときに 家[いえ]を 出[で]て、 子供[こども] 達[たち]が 寝[ね]てから 家[いえ]に 帰[かえ]ってくる。つまり、 父親[ちちおや] 不在[ふざい]ということだ。
\\	ー彼女、僕が会いに行っても、あまり話してくれないんですよ。 ーそれは、つまり、君のことが嫌いなんだよ。	
\\	ー 彼女[かのじょ]、 僕[ぼく]が 会[あ]いに 行[い]っても、あまり 話[はな]してくれないんですよ。 ーそれは、つまり、 君[きみ]のことが 嫌[きら]いなんだよ。
\\	産業化のために自然が破壊されること、つまり、環境破壊は人類の問題だ。	
\\	産業[さんぎょう] 化[か]のために 自然[しぜん]が 破壊[はかい]されること、つまり、 環境[かんきょう] 破壊[はかい]は 人類[じんるい]の 問題[もんだい]だ。
\\	私の母の妹の娘、つまり、私のいとこが来週バンコクから来ます。	
\\	私[わたし]の 母[はは]の 妹[いもうと]の 娘[むすめ]、つまり、 私[わたし]のいとこが 来週[らいしゅう]バンコクから 来[き]ます。
\\	女性を女性だからといって差別すること、つまり、性差別の問題はどこへ行ってもある。	
\\	女性[じょせい]を 女性[じょせい]だからといって 差別[さべつ]すること、つまり、 性[せい] 差別[さべつ]の 問題[もんだい]はどこへ 行[い]ってもある。
\\	つまり、誰も悪くないということだよ。	
\\	つまり、 誰[だれ]も 悪[わる]くないということだよ。
\\	結局、誰も悪くないということだよ。	
\\	結局[けっきょく]、 誰[だれ]も 悪[わる]くないということだよ。
\\	結局お金を出したくないんだろう。	
\\	結局[けっきょく]お 金[かね]を 出[だ]したくないんだろう。
\\	台風が九州に接近しつつある。	
\\	台風[たいふう]が 九州[きゅうしゅう]に 接近[せっきん]しつつある。
\\	彼女は夫の無罪を信じつつ息を引き取った。	
\\	彼女[かのじょ]は 夫[おっと]の 無罪[むざい]を 信[しん]じつつ 息[いき]を 引き取[ひきと]った。
\\	たばこは体によくないと分かりつつなかなかやめられない。	
\\	たばこは 体[からだ]によくないと 分[わ]かりつつなかなかやめられない。
\\	新しい本社ビルが完成しつつある。	
\\	新[あたら]しい 本社[ほんしゃ]ビルが 完成[かんせい]しつつある。
\\	オリンピックの施設が出来つつある。	
\\	オリンピックの 施設[しせつ]が 出来[でき]つつある。
\\	この大学は現在キャンパスの移転を検討しつつある。	
\\	この 大学[だいがく]は 現在[げんざい]キャンパスの 移転[いてん]を 検討[けんとう]しつつある。
\\	甘い物を食べると太ると分かりつつ、つい手が出てします。	
\\	甘[あま]い 物[もの]を 食[た]べると 太[ふと]ると 分[わ]かりつつ、つい 手[て]が 出[で]てします。
\\	悪いことと知りつつ、友達にうそをついてしまった。	
\\	悪[わる]いことと 知[し]りつつ、 友達[ともだち]にうそをついてしまった。
\\	早く論文を書かなければいけないと思いつつ、今日もテレビを見てしまった。	
\\	早[はや]く 論文[ろんぶん]を 書[か]かなければいけないと 思[おも]いつつ、 今日[きょう]もテレビを 見[み]てしまった。
\\	私は父の無事を祈りつつ病院に急いだ。	
\\	私[わたし]は 父[ちち]の 無事[ぶじ]を 祈[いの]りつつ 病院[びょういん]に 急[いそ]いだ。
\\	彼は最後まで自分の学説が正しいことを主張しつつこの世を去った。	
\\	彼[かれ]は 最後[さいご]まで 自分[じぶん]の 学説[がくせつ]が 正[ただ]しいことを 主張[しゅちょう]しつつこの 世[よ]を 去[さ]った。
\\	ルームメートがテレビを見ている間、私は日本語を勉強した。	
\\	ルームメートがテレビを 見[み]ている 間[あいだ]、 私[わたし]は 日本語[にほんご]を 勉強[べんきょう]した。
\\	彼女は毎日あれだけ仕事をしつつ、一言も苦情を言わない。	
\\	彼女[かのじょ]は 毎日[まいにち]あれだけ 仕事[しごと]をしつつ、 一言[ひとこと]も 苦情[くじょう]を 言[い]わない。
\\	舞台の幕が開いている。	
\\	舞台[ぶたい]の 幕[まく]が 開[あ]いている。
\\	難民は自分達の国に帰っている。	
\\	難民[なんみん]は 自分[じぶん] 達[たち]の 国[くに]に 帰[かえ]っている。
\\	舞台の幕が開きつつある。	
\\	舞台[ぶたい]の 幕[まく]が 開[あ]きつつある。
\\	難民は自分達の国に帰りつつある。	
\\	難民[なんみん]は 自分[じぶん] 達[たち]の 国[くに]に 帰[かえ]りつつある。
\\	ジムは経験が浅いながらなかなかいい仕事をする。	
\\	ジムは 経験[けいけん]が 浅[あさ]いながらなかなかいい 仕事[しごと]をする。
\\	彼はこの事故の責任者でありながら責任を逃れようとしている。	
\\	彼[かれ]はこの 事故[じこ]の 責任[せきにん] 者[しゃ]でありながら 責任[せきにん]を 逃[のが]れようとしている。
\\	私は英語が分からないながらそのショーを十分楽しめた。	
\\	私[わたし]は 英語[えいご]が 分[わ]からないながらそのショーを 十分[じゅうぶん] 楽[たの]しめた。
\\	よく考えた上で御返事いたします。	
\\	よく 考[かんが]えた 上[うえ]で 御返事[ごへんじ]いたします。
\\	話し合いの上で結論を出したいと思います。	
\\	話し合[はなしあ]いの 上[うえ]で 結論[けつろん]を 出[だ]したいと 思[おも]います。
\\	よく問題を話し合った上で決めたらどうでしょうか。	
\\	よく 問題[もんだい]を 話し合[はなしあ]った 上[うえ]で 決[き]めたらどうでしょうか。
\\	実験をした上で、理論の正しさを証明しなければならない。	
\\	実験[じっけん]をした 上[うえ]で、 理論[りろん]の 正[ただ]しさを 証明[しょうめい]しなければならない。
\\	日本文学のスミス先生と相談した上で、修士論文の題目を決めました。	
\\	日本[にほん] 文学[ぶんがく]のスミス 先生[せんせい]と 相談[そうだん]した 上[うえ]で、 修士[しゅうし] 論文[ろんぶん]の 題目[だいもく]を 決[き]めました。
\\	彼女とはお見合いをして、一年間交際した上で、結婚に踏み切りました。	
\\	彼女[かのじょ]とはお 見合[みあ]いをして、一年間[いちねんかん] 交際[こうさい]した 上[うえ]で、 結婚[けっこん]に 踏み切[ふみき]りました。
\\	日本の生活は高くつくと知って上で、日本へ留学した。	
\\	日本[にほん]の 生活[せいかつ]は 高[たか]くつくと 知[し]って 上[うえ]で、 日本[にほん]へ 留学[りゅうがく]した。
\\	十冊ぐらい本を読んだ上で、学期末のレポートを書き上げた。	
\\	十冊[じっさつ]ぐらい 本[ほん]を 読[よ]んだ 上[うえ]で、 学期[がっき] 末[まつ]のレポートを 書き上[かきあ]げた。
\\	面談の上で、採否を決めます。	
\\	面談[めんだん]の 上[うえ]で、 採否[さいひ]を 決[き]めます。
\\	二人は協議の上で離婚した。	
\\	二人[ふたり]は 協議[きょうぎ]の 上[うえ]で 離婚[りこん]した。
\\	彼は新しい仕事を見つけた上で、彼女と結婚した。	
\\	彼[かれ]は 新[あたら]しい 仕事[しごと]を 見[み]つけた 上[うえ]で、 彼女[かのじょ]と 結婚[けっこん]した。
\\	彼は財産を全部整理した上で、自殺した。	
\\	彼[かれ]は 財産[ざいさん]を 全部[ぜんぶ] 整理[せいり]した 上[うえ]で、 自殺[じさつ]した。
\\	この問題はよく調査した上で、結論を出すべきだ。	
\\	この 問題[もんだい]はよく 調査[ちょうさ]した 上[うえ]で、 結論[けつろん]を 出[だ]すべきだ。
\\	ジョーンズさんは必ずシャワーを浴びてから寝る。	
\\	ジョーンズさんは 必[かなら]ずシャワーを 浴[あ]びてから 寝[ね]る。
\\	電話をしてから友達を訪ねた。	
\\	電話[でんわ]をしてから 友達[ともだち]を 訪[たず]ねた。
\\	彼は必ず翌日の予定表を調べてから床につく。	
\\	彼[かれ]は 必[かなら]ず 翌日[よくじつ]の 予定[よてい] 表[ひょう]を 調[しら]べてから 床[とこ]につく。
\\	この映画は話の筋が面白い上に配役がいい。	
\\	この 映画[えいが]は 話[はなし]の 筋[すじ]が 面白[おもしろ]い 上[うえ]に 配役[はいやく]がいい。
\\	彼は仕事を見つけてくれた上お金まで貸してくれた。	
\\	彼[かれ]は 仕事[しごと]を 見[み]つけてくれた 上[うえ]お 金[かね]まで 貸[か]してくれた。
\\	彼女はフランス語が専門の上にパリにしばらく住んでいたから、フランス語がぺらぺらだ。	
\\	彼女[かのじょ]は フランス語[ふらんすご]が 専門[せんもん]の 上[うえ]にパリにしばらく 住[す]んでいたから、 フランス語[ふらんすご]がぺらぺらだ。
\\	吉岡さんは奥さんに離婚された上仕事まで失った。	
\\	吉岡[よしおか]さんは 奥[おく]さんに 離婚[りこん]された 上[うえ] 仕事[しごと]まで 失[うしな]った。
\\	私は年をとっている上に特に技能もないので、なかなか仕事が見つからない。	
\\	私[わたし]は 年[とし]をとっている 上[うえ]に 特[とく]に 技能[ぎのう]もないので、なかなか 仕事[しごと]が 見[み]つからない。
\\	この車は故障が少ない上燃費がいいので、よく売れている。	
\\	この 車[くるま]は 故障[こしょう]が 少[すく]ない 上[うえ] 燃費[ねんぴ]がいいので、よく 売[う]れている。
\\	私のアパートは狭い上に家具がたくさんあるので、お客さんが来た時、寝てもらう場所がない。	
\\	私[わたし]のアパートは 狭[せま]い 上[うえ]に 家具[かぐ]がたくさんあるので、お 客[きゃく]さんが 来[き]た 時[とき]、 寝[ね]てもらう 場所[ばしょ]がない。
\\	私のアパートは通勤に便利な上家賃が安い。	
\\	私[わたし]のアパートは 通勤[つうきん]に 便利[べんり]な 上[うえ] 家賃[やちん]が 安[やす]い。
\\	田中先生は教え方が上手な上に学生の面倒をよく見るから、とても人気がある。	
\\	田中[たなか] 先生[せんせい]は 教[おし]え 方[かた]が 上手[じょうず]な 上[うえ]に 学生[がくせい]の 面倒[めんどう]をよく 見[み]るから、とても 人気[にんき]がある。
\\	彼はスピード違反で捕まった時、無免許の上酒に酔っていた。	
\\	彼[かれ]はスピード 違反[いはん]で 捕[つか]まった 時[とき]、 無[む] 免許[めんきょ]の 上[うえ] 酒[さけ]に 酔[よ]っていた。
\\	二十一世紀に日本語は国際語になり得ると思う。	
\\	二十一世紀[にじゅういっせいき]に 日本語[にほんご]は 国際[こくさい] 語[ご]になり 得[う/え]ると 思[おも]う。
\\	株の変動はいつでも起こり得る。	
\\	株[かぶ]の 変動[へんどう]はいつでも 起[お]こり 得[う/え]る。
\\	ほとんどのがんは予防し得る。	
\\	ほとんどのがんは 予防[よぼう]し 得[う/え]る。
\\	一度に記憶し得る単語の数は限られている。	
\\	一 度[ど]に 記憶[きおく]し 得[う/え]る 単語[たんご]の 数[かず]は 限[かぎ]られている。
\\	運動をすることによって健康を維持し得る。	
\\	運動[うんどう]をすることによって 健康[けんこう]を 維持[いじ]し 得[う/え]る。
\\	日本語の基礎はだいたい二年間で学び得る。	
\\	日本語[にほんご]の 基礎[きそ]はだいたい 二年間[にねんかん]で 学[まな]び 得[う/え]る。
\\	日本の経済が不況になることだって十分にあり得る。	
\\	日本[にほん]の 経済[けいざい]が 不況[ふきょう]になることだって 十分[じゅうぶん]にあり 得[う/え]る。
\\	彼は日本語を話せる。	
\\	彼[かれ]は 日本語[にほんご]を 話[はな]せる。
\\	ジョンは日本で生まれ、十歳まで日本の学校で勉強した。	
\\	ジョンは 日本[にほん]で 生[う]まれ、 十歳[じゅうさい]まで 日本[にほん]の 学校[がっこう]で 勉強[べんきょう]した。
\\	父は銀行に勤め、母はうちの近くでブティックを開いている。	
\\	父[ちち]は 銀行[ぎんこう]に 勤[つと]め、 母[はは]はうちの 近[ちか]くでブティックを 開[ひら]いている。
\\	みんながお金を出し合い、京子の出産祝いを買った。	
\\	みんながお 金[かね]を 出し合[だしあ]い、 京子[きょうこ]の 出産[しゅっさん] 祝[いわ]いを 買[か]った。
\\	エビは殻をむき、サラダ油で軽く炒めます。	
\\	エビは 殻[から]をむき、 サラダ油[さらだゆ]で 軽[かる]く 炒[いた]めます。
\\	日本では一週間前から梅雨に入り、毎日うっとうしい日が続いています。	
\\	日本[にほん]では 一週間[いっしゅうかん] 前[まえ]から 梅雨[つゆ]に 入[はい]り、 毎日[まいにち]うっとうしい 日[ひ]が 続[つづ]いています。
\\	本校の五十周年を記念し、祝賀パーティーを開きます。	
\\	本校[ほんこう]の 五十周年[ごじっしゅうねん]を 記念[きねん]し、 祝賀[しゅくが]パーティーを 開[ひら]きます。
\\	長男は医者になり、次男は弁護士になった。	
\\	長男[ちょうなん]は 医者[いしゃ]になり、 次男[じなん]は 弁護士[べんごし]になった。
\\	男が荷車の前の綱を引っ張り、女が後ろを押した。	
\\	男[おとこ]が 荷車[にぐるま]の 前[まえ]の 綱[つな]を 引っ張[ひっぱ]り、 女[おんな]が 後[うし]ろを 押[お]した。
\\	私は今日デパートへ行って買い物をしなければならない。	
\\	私[わたし]は 今日[きょう]デパートへ 行[い]って 買い物[かいもの]をしなければならない。
\\	実物を見て買うかどうか決めたい。	
\\	実物[じつぶつ]を 見[み]て 買[か]うかどうか 決[き]めたい。
\\	彼は働きすぎて体を壊してしまった。	
\\	彼[かれ]は 働[はたら]きすぎて 体[からだ]を 壊[こわ]してしまった。
\\	赤ん坊は私の顔を見て急に泣き出した。	
\\	赤ん坊[あかんぼう]は 私[わたし]の 顔[かお]を 見[み]て 急[きゅう]に 泣[な]き 出[だ]した。
\\	京子は毎日自転車に乗って学校へ行く。	
\\	京子[きょうこ]は 毎日[まいにち] 自転車[じてんしゃ]に 乗[の]って 学校[がっこう]へ 行[い]く。
\\	私は手袋をはめてその仕事をした。	
\\	私[わたし]は 手袋[てぶくろ]をはめてその 仕事[しごと]をした。
\\	幸男はよく働き、よく遊ぶ。	
\\	幸男[ゆきお]はよく 働[はたら]き、よく 遊[あそ]ぶ。
\\	みんなはその知らせに驚き、悲しんだ。	
\\	みんなはその 知[し]らせに 驚[おどろ]き、 悲[かな]しんだ。
\\	次郎はクラシック音楽を好み、演奏家もよく知っている。	
\\	次郎[じろう]はクラシック 音楽[おんがく]を 好[この]み、 演奏[えんそう] 家[か]もよく 知[し]っている。
\\	我々は平和を愛し、戦争を憎む。	
\\	我々[われわれ]は 平和[へいわ]を 愛[あい]し、 戦争[せんそう]を 憎[にく]む。
\\	エビは殻をむいて、炒めます。	
\\	エビは 殻[から]をむいて、 炒[いた]めます。
\\	これは炭火で焼いて食べます。	
\\	これは 炭火[すみび]で 焼[や]いて 食[た]べます。
\\	これは炭火で焼き、しょう油を付けて食べます。	
\\	これは 炭火[すみび]で 焼[や]き、しょう 油[ゆ]を 付[つ]けて 食[た]べます。
\\	勉がピザを焼いて、利子がサラダを作った。	
\\	勉[つとむ]がピザを 焼[や]いて、 利子[としこ]がサラダを 作[つく]った。
\\	彼はまだ若いから、考えが浅い。	
\\	彼[かれ]はまだ 若[わか]いから、 考[かんが]えが 浅[あさ]い。
\\	日本語は、話すのは比較的やさしいが、読み書きは難しい。	
\\	日本語[にほんご]は、 話[はな]すのは 比較的[ひかくてき]やさしいが、 読み書[よみか]きは 難[むずか]しい。
\\	終わりがよければ、全体がよくなる。	
\\	終[お]わりがよければ、 全体[ぜんたい]がよくなる。
\\	遊びのない生活はつまらない。	
\\	遊[あそ]びのない 生活[せいかつ]はつまらない。
\\	お金の貸し借りはしない方がいい。	
\\	お 金[かね]の 貸し借[かしか]りはしない 方[ほう]がいい。
\\	東京ディズニーランドに車で行って来たが、行き帰りに五時間以上もかかった。	
\\	東京[とうきょう]ディズニーランドに 車[くるま]で 行[い]って 来[き]たが、 行き帰[ゆきかえ]りに 五時間[ごじかん] 以上[いじょう]もかかった。
\\	人生は勝ち負けの連続だ。	
\\	人生[じんせい]は 勝ち負[かちま]けの 連続[れんぞく]だ。
\\	行きはしますが、何も手伝えませんよ。	
\\	行[い]きはしますが、 何[なに]も 手伝[てつだ]えませんよ。
\\	疲れてはいるが、まだ休みたくない。	
\\	疲[つか]れてはいるが、まだ 休[やす]みたくない。
\\	あの男には説明しても分かりはしない。	
\\	あの 男[おとこ]には 説明[せつめい]しても 分[わ]かりはしない。
\\	あのズボンは安くはあるが、ちょっと時代遅れだ。	
\\	あのズボンは 安[やす]くはあるが、ちょっと 時代遅[じだいおく]れだ。
\\	この本は内容を考えれば決して高くはない。	
\\	この 本[ほん]は 内容[ないよう]を 考[かんが]えれば 決[けっ]して 高[たか]くはない。
\\	酒は飲みはするが、それほど好きではない。	
\\	酒[さけ]は 飲[の]みはするが、それほど 好[す]きではない。
\\	彼は日本語を読めはするが、話せはしない。	
\\	彼[かれ]は 日本語[にほんご]を 読[よ]めはするが、 話[はな]せはしない。
\\	彼女はベッドに横になってはいるが、目は開いている。	
\\	彼女[かのじょ]はベッドに 横[よこ]になってはいるが、 目[め]は 開[ひら]いている。
\\	試験を受けてはみるが、通る自身は全然ない。	
\\	試験[しけん]を 受[う]けてはみるが、 通[とお]る 自身[じしん]は 全然[ぜんぜん]ない。
\\	彼は何を言っても聞きはしない。	
\\	彼[かれ]は 何[なに]を 言[い]っても 聞[き]きはしない。
\\	この花瓶は高くはあるが、それだけの値打ちはある。	
\\	この 花瓶[かびん]は 高[たか]くはあるが、それだけの 値打[ねう]ちはある。
\\	この問題はよく考えれば難しくはないはずだ。	
\\	この 問題[もんだい]はよく 考[かんが]えれば 難[むずか]しくはないはずだ。
\\	あの男は優秀ではあったが、不正なことをしたので首になった。	
\\	あの 男[おとこ]は 優秀[ゆうしゅう]ではあったが、 不正[ふせい]なことをしたので 首[くび]になった。
\\	彼は日本人ではあるが、あまり日本語を話せない。	
\\	彼[かれ]は 日本人[にほんじん]ではあるが、あまり 日本語[にほんご]を 話[はな]せない。
\\	大統領の職務は頭脳は言うまでもなく、体力も要求される。	
\\	大統領[だいとうりょう]の 職務[しょくむ]は 頭脳[ずのう]は 言[い]うまでもなく、 体力[たいりょく]も 要求[ようきゅう]される。
\\	日本は土地は言うまでもなく、家も非常に高い。	
\\	日本[にほん]は 土地[とち]は 言[い]うまでもなく、 家[いえ]も 非常[ひじょう]に 高[たか]い。
\\	この大学は教育内容は言うまでもなく、設備も抜群に優れている。	
\\	この 大学[だいがく]は 教育[きょういく] 内容[ないよう]は 言[い]うまでもなく、 設備[せつび]も 抜群[ばつぐん]に 優[すぐ]れている。
\\	彼は平日は言うまでもなく、週末や祭日も仕事をしている。	
\\	彼[かれ]は 平日[へいじつ]は 言[い]うまでもなく、 週末[しゅうまつ]や 祭日[さいじつ]も 仕事[しごと]をしている。
\\	ジュリアンは菜食主義者なので、牛肉に言うまでもなく、魚も鳥肉も食べない。	
\\	ジュリアンは 菜食[さいしょく] 主義[しゅぎ] 者[しゃ]なので、 牛肉[ぎゅうにく]に 言[い]うまでもなく、 魚[さかな]も 鳥[とり] 肉[にく]も 食[た]べない。
\\	あの学生は漢字は言うまでもなく、平仮名、片仮名さえ知らない。	
\\	あの 学生[がくせい]は 漢字[かんじ]は 言[い]うまでもなく、 平仮名[ひらがな]、 片仮名[かたかな]さえ 知[し]らない。
\\	読書が大事なことは言うまでもない。	
\\	読書[どくしょ]が 大事[だいじ]なことは 言[い]うまでもない。
\\	人生に浮き沈みがあるのは言うまでもない。	
\\	人生[じんせい]に 浮き沈[うきしず]みがあるのは 言[い]うまでもない。
\\	福田先生の死で私達はリーダーを失ってしまったわけですが、今後は私達自身でこの会を発展させていかなければなりません。	
\\	福田[ふくだ] 先生[せんせい]の 死[し]で 私[わたし] 達[たち]はリーダーを 失[うしな]ってしまったわけですが、 今後[こんご]は 私[わたし] 達[たち] 自身[じしん]でこの 会[かい]を 発展[はってん]させていかなければなりません。
\\	戦争は大きな犠牲を伴うわけで、我々は何としてもこれを避けなければならない。	
\\	戦争[せんそう]は 大[おお]きな 犠牲[ぎせい]を 伴[ともな]うわけで、 我々[われわれ]は 何[なん]としてもこれを 避[さ]けなければならない。
\\	今でこそ地球が丸いことは子供でも知っているが、コロンブス以前は誰もが地球は平らだと思っていたわけだ。	
\\	今[いま]でこそ 地球[ちきゅう]が 丸[まる]いことは 子供[こども]でも 知[し]っているが、コロンブス 以前[いぜん]は 誰[だれ]もが 地球[ちきゅう]は 平[たい]らだと 思[おも]っていたわけだ。
\\	残念ながら今回の交渉はこういう結果になってしまったわけですが、これからも交渉は続けていくつもりです。	
\\	残念[ざんねん]ながら 今回[こんかい]の 交渉[こうしょう]はこういう 結果[けっか]になってしまったわけですが、これからも 交渉[こうしょう]は 続[つづ]けていくつもりです。
\\	我々は多くの技術を日常生活に取り入れてきたわけだが、それらがすべて我々に幸福をもたらしたかどうかは疑問である。	
\\	我々[われわれ]は 多[おお]くの 技術[ぎじゅつ]を 日常[にちじょう] 生活[せいかつ]に 取り入[とりい]れてきたわけだが、それらがすべて 我々[われわれ]に 幸福[こうふく]をもたらしたかどうかは 疑問[ぎもん]である。
\\	私は動物のコミュニケーションを研究しているわけですが、この研究からいろいろ面白いことが分かります。	
\\	私[わたし]は 動物[どうぶつ]のコミュニケーションを 研究[けんきゅう]しているわけですが、この 研究[けんきゅう]からいろいろ 面白[おもしろ]いことが 分[わ]かります。
\\	何事をするにも基本がしっかり出来ているかどうかが進歩の鍵となるわけで、外国語学習についても同じことが言えるわけです。	
\\	何事[なにごと]をするにも 基本[きほん]がしっかり 出来[でき]ているかどうかが 進歩[しんぽ]の 鍵[かぎ]となるわけで、 外国[がいこく] 語[ご] 学習[がくしゅう]についても 同[おな]じことが 言[い]えるわけです。
\\	赤ん坊は自分ではものを言わないが、話しかけられている時に言葉を学習しているわけだ。	
\\	赤ん坊[あかんぼう]は 自分[じぶん]ではものを 言[い]わないが、 話[はな]しかけられている 時[とき]に 言葉[ことば]を 学習[がくしゅう]しているわけだ。
\\	我々は今コンピュータを取り上げられるとたちどころに困ってしまうけれども、今から十年程前はコンピュータなしでやっていたわけです。	
\\	我々[われわれ]は 今[いま]コンピュータを 取り上[とりあ]げられるとたちどころに 困[こま]ってしまうけれども、 今[いま]から 十年[じゅうねん] 程[ほど] 前[まえ]はコンピュータなしでやっていたわけです。
\\	私が帰ったらね、いとこの政ちゃんが来てたわけ。で、話してるうちにディスコへ行こうということになって、友達に電話したわけ。	
\\	私[わたし]が 帰[かえ]ったらね、いとこの 政[まさ]ちゃんが 来[き]てたわけ。で、 話[はな]してるうちにディスコへ 行[い]こうということになって、 友達[ともだち]に 電話[でんわ]したわけ。
\\	[先生:]ー私は来週一週間日本に行きます。 [学生:]ーじゃ、来週は授業がないわけですか。	
\\	先生[せんせい]:]ー 私[わたし]は 来週[らいしゅう] 一週間[いっしゅうかん] 日本[にほん]に 行[い]きます。 
\\	学生[がくせい]:]ーじゃ、 来週[らいしゅう]は 授業[じゅぎょう]がないわけですか。
\\	ジョーンズさんは日本に十年もいたんだそうだ。日本語が上手なわけだ。	
\\	ジョーンズさんは 日本[にほん]に 十年[じゅうねん]もいたんだそうだ。 日本語[にほんご]が 上手[じょうず]なわけだ。
\\	ジョーンズさんは日本語が上手なわけだ。日本に十年もいたんだそうだ。	
\\	ジョーンズさんは 日本語[にほんご]が 上手[じょうず]なわけだ。 日本[にほん]に 十年[じゅうねん]もいたんだそうだ。
\\	福田先生の死で私達はリーダーを失ってしまったのですが、今後は私達自身でこの会を発展させていかなければなりません。	
\\	福田[ふくだ] 先生[せんせい]の 死[し]で 私[わたし] 達[たち]はリーダーを 失[うしな]ってしまったのですが、 今後[こんご]は 私[わたし] 達[たち] 自身[じしん]でこの 会[かい]を 発展[はってん]させていかなければなりません。
\\	今でこそ地球が丸いことは子供でも知っているが、コロンブス以前は誰もが地球は平らだと思っていたのだ。	
\\	今[いま]でこそ 地球[ちきゅう]が 丸[まる]いことは 子供[こども]でも 知[し]っているが、コロンブス 以前[いぜん]は 誰[だれ]もが 地球[ちきゅう]は 平[たい]らだと 思[おも]っていたのだ。
\\	戦争は大きな犠牲を伴うので、我々は何としてもこれを避けなければならない。	
\\	戦争[せんそう]は 大[おお]きな 犠牲[ぎせい]を 伴[ともな]うので、 我々[われわれ]は 何[なん]としてもこれを 避[さ]けなければならない。
\\	今日のパーティーには行けません。宿題がたくさんあるんです。	
\\	今日[きょう]のパーティーには 行[い]けません。 宿題[しゅくだい]がたくさんあるんです。
\\	ー何をしているんですか。 ー日本語を勉強しているんです。	
\\	ー 何[なに]をしているんですか。 ー 日本語[にほんご]を 勉強[べんきょう]しているんです。
\\	この問題を解決するのは非常に難しい。しかし、不可能だというわけではない。	
\\	この 問題[もんだい]を 解決[かいけつ]するのは 非常[ひじょう]に 難[むずか]しい。しかし、 不可能[ふかのう]だというわけではない。
\\	君の考えに反対するわけではないが、そのやり方でうまくいくだろうか。	
\\	君[きみ]の 考[かんが]えに 反対[はんたい]するわけではないが、そのやり 方[かた]でうまくいくだろうか。
\\	結果さえよければいいというわけではない。	
\\	結果[けっか]さえよければいいというわけではない。
\\	私は最近あまりたくさん食べないが、食欲がないわけではない。	
\\	私[わたし]は 最近[さいきん]あまりたくさん 食[た]べないが、 食欲[しょくよく]がないわけではない。
\\	この二、三年本を書いていないが、スランプというわけではない。	
\\	この二[に]、三年[さんねん] 本[ほん]を 書[か]いていないが、スランプというわけではない。
\\	このプロジェクトは資金不足のため中止されることになった。しかし、今までしてきたことが全く無駄になるわけではない。	
\\	このプロジェクトは 資金[しきん] 不足[ぶそく]のため 中止[ちゅうし]されることになった。しかし、 今[いま]までしてきたことが 全[まった]く 無駄[むだ]になるわけではない。
\\	いくら日本語が出来ないと言っても全然話せないわけではない。	
\\	いくら 日本語[にほんご]が 出来[でき]ないと 言[い]っても 全然[ぜんぜん] 話[はな]せないわけではない。
\\	君の言うことが分からないわけではないが、やっぱりこの計画は止めた方がいいと思う。	
\\	君[きみ]の 言[い]うことが 分[わ]からないわけではないが、やっぱりこの 計画[けいかく]は 止[と]めた 方[ほう]がいいと 思[おも]う。
\\	全然見込みがないわけではないが、あまり期待はしない方がいいでしょう。	
\\	全然[ぜんぜん] 見込[みこ]みがないわけではないが、あまり 期待[きたい]はしない 方[ほう]がいいでしょう。
\\	あの人が嫌いだというわけではないんですが、結婚したいとは思いません。	
\\	あの 人[ひと]が 嫌[きら]いだというわけではないんですが、 結婚[けっこん]したいとは 思[おも]いません。
\\	言われたことだけしていたらそれでいいというわけではない。	
\\	言[い]われたことだけしていたらそれでいいというわけではない。
\\	ーこの病人は全く助からないんですか。 ーいや、そういうわけではありません。	
\\	ーこの 病人[びょうにん]は 全[まった]く 助[たす]からないんですか。 ーいや、そういうわけではありません。
\\	結果がいいことは大切だ。	
\\	結果[けっか]がいいことは 大切[たいせつ]だ。
\\	こんなやさしい仕事が君に出来ないわけがない。	
\\	こんなやさしい 仕事[しごと]が 君[きみ]に 出来[でき]ないわけがない。
\\	日本語がそんなに速くマスター出来るわけがない。	
\\	日本語[にほんご]がそんなに 速[はや]くマスター 出来[でき]るわけがない。
\\	そんなことをしていいわけがない。	
\\	そんなことをしていいわけがない。
\\	そんなことを言って人が喜ぶわけがない。	
\\	そんなことを 言[い]って 人[ひと]が 喜[よろこ]ぶわけがない。
\\	彼女がそんなひどいことを言うわけがない。	
\\	彼女[かのじょ]がそんなひどいことを 言[い]うわけがない。
\\	彼はあんなに酔っていて大丈夫なわけがない。	
\\	彼[かれ]はあんなに 酔[よ]っていて 大丈夫[だいじょうぶ]なわけがない。
\\	あの店のものがこんなに安いわけがない。	
\\	あの 店[みせ]のものがこんなに 安[やす]いわけがない。
\\	彼女がこの時間にこんな所にいるわけがない。	
\\	彼女[かのじょ]がこの 時間[じかん]にこんな 所[ところ]にいるわけがない。
\\	彼がこの学校の卒業生であるわけがない。	
\\	彼[かれ]がこの 学校[がっこう]の 卒業生[そつぎょうせい]であるわけがない。
\\	ーチャールズが一年で日本語をマスターしたそうだよ。 ー冗談だろう。日本語がそんなに速くマスター出来るわけがない。	
\\	ーチャールズが 一年[いちねん]で 日本語[にほんご]をマスターしたそうだよ。 ー 冗談[じょうだん]だろう。 日本語[にほんご]がそんなに 速[はや]くマスター 出来[でき]るわけがない。
\\	彼女がそんなことをしたはずがない。	
\\	彼女[かのじょ]がそんなことをしたはずがない。
\\	彼が優等生だったはずがない。	
\\	彼[かれ]が 優等生[ゆうとうせい]だったはずがない。
\\	来週は期末試験があるから、今週末は遊んでいるわけにはいかない。	
\\	来週[らいしゅう]は 期末[きまつ] 試験[しけん]があるから、 今週[こんしゅう] 末[まつ]は 遊[あそ]んでいるわけにはいかない。
\\	親友の結婚式だから、出席しないわけにはいかない。	
\\	親友[しんゆう]の 結婚式[けっこんしき]だから、 出席[しゅっせき]しないわけにはいかない。
\\	こんな高価な物を頂くわけにはいきません。	
\\	こんな 高価[こうか]な 物[もの]を 頂[いただ]くわけにはいきません。
\\	私のような若い者がそんな会議に出席するわけにはいきません。	
\\	私[わたし]のような 若[わか]い 者[もの]がそんな 会議[かいぎ]に 出席[しゅっせき]するわけにはいきません。
\\	まだ定職もないんだから、結婚するわけにはいかないんですよ。	
\\	まだ 定職[ていしょく]もないんだから、 結婚[けっこん]するわけにはいかないんですよ。
\\	これから運転しなければなりませんから、酒を飲むわけにはいきません。	
\\	これから 運転[うんてん]しなければなりませんから、 酒[さけ]を 飲[の]むわけにはいきません。
\\	まだ仕事が残っているので、帰るわけにはいきません。	
\\	まだ 仕事[しごと]が 残[のこ]っているので、 帰[かえ]るわけにはいきません。
\\	これは秘密なので、話すわけにはいきません。	
\\	これは 秘密[ひみつ]なので、 話[はな]すわけにはいきません。
\\	この論文は重要だから、読まないわけにはいかない。	
\\	この 論文[ろんぶん]は 重要[じゅうよう]だから、 読[よ]まないわけにはいかない。
\\	上司の命令なのだから、報告書を書かないわけにはいかない。	
\\	上司[じょうし]の 命令[めいれい]なのだから、 報告[ほうこく] 書[しょ]を 書[か]かないわけにはいかない。
\\	僕はピアノが弾けない。	
\\	僕[ぼく]はピアノが 弾[ひ]けない。
\\	僕はピアノを弾くわけにはいかない。	
\\	僕[ぼく]はピアノを 弾[ひ]くわけにはいかない。
\\	今日はいい天気だから、テニスが出来る。	
\\	今日[きょう]はいい 天気[てんき]だから、テニスが 出来[でき]る。
\\	貧乏な家庭は子供を大学にやれない。	
\\	貧乏[びんぼう]な 家庭[かてい]は 子供[こども]を 大学[だいがく]にやれない。
\\	僕は弟に自転車をやるつもりだ。	
\\	僕[ぼく]は 弟[おとうと]に 自転車[じてんしゃ]をやるつもりだ。
\\	午後、テニスをやりませんか。	
\\	午後[ごご]、テニスをやりませんか。
\\	今アメリカで日本語の教師をやっています。	
\\	今[いま]アメリカで 日本語[にほんご]の 教師[きょうし]をやっています。
\\	三船敏郎のやる役は正義感の強い男だ。	
\\	三船[みふね] 敏郎[としろう]のやる 役[やく]は 正義[せいぎ] 感[かん]の 強[つよ]い 男[おとこ]だ。
\\	先月までたばこをやっていました。	
\\	先月[せんげつ]までたばこをやっていました。
\\	息子を郵便局にやって、切手を買って来させた。	
\\	息子[むすこ]を 郵便[ゆうびん] 局[きょく]にやって、 切手[きって]を 買[か]って 来[こ]させた。
\\	猫にこのえさをやってね。	
\\	猫[ねこ]にこのえさをやってね。
\\	宿題をやりましたか。	
\\	宿題[しゅくだい]をやりましたか。
\\	チェスをやろうか。	
\\	チェスをやろうか。
\\	来月クラス会をやります。	
\\	来月[らいげつ]クラス 会[かい]をやります。
\\	ちょっと遅れるから、先に(食事を)やっていて下さい。	
\\	ちょっと 遅[おく]れるから、 先[さき]に
\\	食事[しょくじ]を)やっていて 下[くだ]さい。
\\	私は酒もたばこもやりません。	
\\	私[わたし]は 酒[さけ]もたばこもやりません。
\\	私は去年から生け花をやっています。	
\\	私[わたし]は 去年[きょねん]から 生け花[いけばな]をやっています。
\\	僕は小さい時いろいろな病気をやりました。	
\\	僕[ぼく]は 小[ちい]さい 時[とき]いろいろな 病気[びょうき]をやりました。
\\	友達がジャズ喫茶をやっている。	
\\	友達[ともだち]がジャズ 喫茶[きっさ]をやっている。
\\	昨日はおいしいすしを食べました。	
\\	昨日[きのう]はおいしいすしを 食[た]べました。
\\	ここではたばこを吸ってはいけません。	
\\	ここではたばこを 吸[す]ってはいけません。
\\	毎朝僕はオレンジジュースを一杯飲みます。	
\\	毎朝[まいあさ] 僕[ぼく]はオレンジジュースを 一杯[いっぱい] 飲[の]みます。
\\	今晩僕と一杯やりませんか。	
\\	今晩[こんばん] 僕[ぼく]と 一杯[いっぱい]やりませんか。
\\	どうぞ皿のものをやって下さい。	
\\	どうぞ 皿[さら]のものをやって 下[くだ]さい。
\\	先生はテストをやさしくした。	
\\	先生[せんせい]はテストをやさしくした。
\\	洋子は長い足をしている。	
\\	洋子[ようこ]は 長[なが]い 足[あし]をしている。
\\	子供の声がしました。	
\\	子供[こども]の 声[こえ]がしました。
\\	この時計は十万円する。	
\\	この 時計[とけい]は 十万[じゅうまん] 円[えん]する。
\\	私は正をからかってやった。	
\\	私[わたし]は 正[ただし]をからかってやった。
\\	ルームメートが部屋を全然掃除しないので、文句を言ってやった。	
\\	ルームメートが 部屋[へや]を 全然[ぜんぜん] 掃除[そうじ]しないので、 文句[もんく]を 言[い]ってやった。
\\	彼にあまり腹が立ったので、どなりつけてやった。	
\\	彼[かれ]にあまり 腹[はら]が 立[た]ったので、どなりつけてやった。
\\	会社の付けで飲んでやった。	
\\	会社[かいしゃ]の 付[つ]けで 飲[の]んでやった。
\\	浩はテレビゲームばかりしてちっとも勉強しないんですよ。一度叱ってやって下さい。	
\\	浩[ひろし]はテレビゲームばかりしてちっとも 勉強[べんきょう]しないんですよ。 一度[いちど] 叱[しか]ってやって 下[くだ]さい。
\\	言うことを聞かなかったら少し脅してやれ。	
\\	言[い]うことを 聞[き]かなかったら 少[すこ]し 脅[おど]してやれ。
\\	私は正をからかった。	
\\	私[わたし]は 正[ただし]をからかった。
\\	ジェリーはとんでもないことをしてくれた。	
\\	ジェリーはとんでもないことをしてくれた。
\\	よくも恥をかかせてくれたわね。	
\\	よくも 恥[はじ]をかかせてくれたわね。
\\	ケイトは私の魚を食べてくれた。	
\\	ケイトは 私[わたし]の 魚[さかな]を 食[た]べてくれた。
\\	テッドに風邪をもらった。	
\\	テッドに 風邪[かぜ]をもらった。
\\	課長に大変な仕事をもらってしまった。	
\\	課長[かちょう]に 大変[たいへん]な 仕事[しごと]をもらってしまった。
\\	日本史の期末レポートをやっと書き終えた。	
\\	日本[にほん] 史[し]の 期末[きまつ]レポートをやっと 書[か]き 終[お]えた。
\\	この道は車がやっと一台通れるくらいの狭さです。	
\\	この 道[みち]は 車[くるま]がやっと 一台[いちだい] 通[とお]れるくらいの 狭[せま]さです。
\\	私の日本語の力ではあいさつをするのがやっとです。	
\\	私[わたし]の 日本語[にほんご]の 力[ちから]ではあいさつをするのがやっとです。
\\	やっとのことで、家が買えた。	
\\	やっとのことで、 家[いえ]が 買[か]えた。
\\	半年の長い冬が終わって、やっと暖かい春になった。	
\\	半年[はんとし]の 長[なが]い 冬[ふゆ]が 終[お]わって、やっと 暖[あたた]かい 春[はる]になった。
\\	日本で一年間日本語を勉強したら、やっと、日本語が通じろようになった。	
\\	日本[にほん]で 一年間[いちねんかん] 日本語[にほんご]を 勉強[べんきょう]したら、やっと、 日本語[にほんご]が 通[つう]じろようになった。
\\	あの二人はずいぶん長い間付き合っていたが、やっと結婚したようだね。	
\\	あの 二人[ふたり]はずいぶん 長[なが]い 間[ま] 付き合[つきあ]っていたが、やっと 結婚[けっこん]したようだね。
\\	長い間の夢だった海外旅行にやっと行けそうだ。	
\\	長[なが]い 間[あいだ]の 夢[ゆめ]だった 海外[かいがい] 旅行[りょこう]にやっと 行[い]けそうだ。
\\	家から駅まで走って、やっと、七時半の電車に間に合った。	
\\	家[いえ]から 駅[えき]まで 走[はし]って、やっと、七時半[しちじはん]の 電車[でんしゃ]に 間に合[まにあ]った。
\\	家族四人がやっと住めるような小さいアパートに入った。	
\\	家族[かぞく] 四人[よにん]がやっと 住[す]めるような 小[ちい]さいアパートに 入[はい]った。
\\	父は腰を痛めているので、家のまわりを散歩するのがやっとです。	
\\	父[ちち]は 腰[こし]を 痛[いた]めているので、 家[いえ]のまわりを 散歩[さんぽ]するのがやっとです。
\\	やっとのことで、富士山の頂上に着いた。	
\\	やっとのことで、 富士山[ふじさん]の 頂上[ちょうじょう]に 着[つ]いた。
\\	彼は四年間癌の治療を受けていたが、とうとう死んでしまった。	
\\	彼[かれ]は 四年間[よねんかん] 癌[がん]の 治療[ちりょう]を 受[う]けていたが、とうとう 死[し]んでしまった。
\\	彼らは夫婦仲が前から悪かったが、ついに別れてしまった。	
\\	彼[かれ]らは 夫婦[ふうふ] 仲[なか]が 前[まえ]から 悪[わる]かったが、ついに 別[わか]れてしまった。
\\	ようやく蛍雪の功なって、ここに卒業の段となりましたことを心からお祝い申し上げます。	
\\	ようやく 蛍雪[けいせつ]の 功[こう]なって、ここに 卒業[そつぎょう]の 段[だん]となりましたことを 心[こころ]からお 祝[いわ]い 申し上[もうしあ]げます。
\\	十時になって、やっと事務所が開いた。	
\\	十時[じゅうじ]になって、やっと 事務所[じむしょ]が 開[ひら]いた。
\\	つまらない講演がやっと終わった。	
\\	つまらない 講演[こうえん]がやっと 終[お]わった。
\\	この作文は直しようがない。	
\\	この 作文[さくぶん]は 直[なお]しようがない。
\\	この成績じゃ救いようがない。	
\\	この 成績[せいせき]じゃ 救[すく]いようがない。
\\	こんなに壊れてしまっては直しようがない。	
\\	こんなに 壊[こわ]れてしまっては 直[なお]しようがない。
\\	みんなで同時に攻めて来たら防ぎようがない。	
\\	みんなで 同時[どうじ]に 攻[せ]めて 来[き]たら 防[ふせ]ぎようがない。
\\	これ以外に考えようがない。	
\\	これ 以外[いがい]に 考[かんが]えようがない。
\\	日本語があんなに下手じゃどうしようもない。	
\\	日本語[にほんご]があんなに 下手[へた]じゃどうしようもない。
\\	漢字が読めないんじゃ仕事のさせようがない。	
\\	漢字[かんじ]が 読[よ]めないんじゃ 仕事[しごと]のさせようがない。
\\	こんなに学生が多くては教えようがない。	
\\	こんなに 学生[がくせい]が 多[おお]くては 教[おし]えようがない。
\\	そんな聞き方をされたら答えようがない。	
\\	そんな 聞[き]き 方[かた]をされたら 答[こた]えようがない。
\\	私に謝ってもらってもしようがない。	
\\	私[わたし]に 謝[あやま]ってもらってもしようがない。
\\	ほかにやりようはないのだろうか。	
\\	ほかにやりようはないのだろうか。
\\	見つからないのは探しようが悪いんだよ。	
\\	見[み]つからないのは 探[さが]しようが 悪[わる]いんだよ。
\\	読みようによってはこの手紙は遺書ともとれる。	
\\	読[よ]みようによってはこの 手紙[てがみ]は 遺書[いしょ]ともとれる。
\\	すしの作り方を教えて下さい。	
\\	すしの 作り方[つくりかた]を 教[おし]えて 下[くだ]さい。
\\	漢字の勉強のし方が分からない。	
\\	漢字[かんじ]の 勉強[べんきょう]のし 方[かた]が 分[わ]からない。
\\	私は彼女の話し方が好きだ。	
\\	私[わたし]は 彼女[かのじょ]の 話し方[はなしかた]が 好[す]きだ。
\\	彼は面白い食べ方をする。	
\\	彼[かれ]は 面白[おもしろ]い 食[た]べ 方[かた]をする。
\\	この程度の文章なら日本語の一年生でも読めよう。	
\\	この 程度[ていど]の 文章[ぶんしょう]なら 日本語[にほんご]の 一年生[いちねんせい]でも 読[よ]めよう。
\\	こんな寒い日にオーバーも着ないで歩き回れば、風邪も引こう。	
\\	こんな 寒[さむ]い 日[ひ]にオーバーも 着[き]ないで 歩き回[あるきまわ]れば、 風邪[かぜ]も 引[ひ]こう。
\\	この政治的問題を解決するのは容易であろう。	
\\	この 政治[せいじ] 的[てき] 問題[もんだい]を 解決[かいけつ]するのは 容易[ようい]であろう。
\\	この問題についてはいろいろ政治的解決が考えられよう。	
\\	この 問題[もんだい]についてはいろいろ 政治[せいじ] 的[てき] 解決[かいけつ]が 考[かんが]えられよう。
\\	このピアノソナタだったら子供にも弾けよう。	
\\	このピアノソナタだったら 子供[こども]にも 弾[はじ]けよう。
\\	この報告書は来週の金曜日までには書けよう。	
\\	この 報告[ほうこく] 書[しょ]は 来週[らいしゅう]の 金曜日[きんようび]までには 書[か]けよう。
\\	あのタワーに上れば、町全体がよく見えよう。	
\\	あのタワーに 上[のぼ]れば、 町[まち] 全体[ぜんたい]がよく 見[み]えよう。
\\	米ロ関係の改善で、世界の平和は維持出来るよう。	
\\	米[べい]ロ 関係[かんけい]の 改善[かいぜん]で、 世界[せかい]の 平和[へいわ]は 維持[いじ] 出来[でき]るよう。
\\	あんなにひどいことをされたら、腹も立とう。	
\\	あんなにひどいことをされたら、 腹[はら]も 立[た]とう。
\\	あれだけむちゃくちゃに働けば、病気にもなろう。	
\\	あれだけむちゃくちゃに 働[はたら]けば、 病気[びょうき]にもなろう。
\\	この辺りでは夜の一人歩きは危険であろう。	
\\	この 辺[あた]りでは 夜[よる]の 一人[ひとり] 歩[ある]きは 危険[きけん]であろう。
\\	こんな難しい文章は小学生には分かるまい。	
\\	こんな 難[むずか]しい 文章[ぶんしょう]は 小学生[しょうがくせい]には 分[わ]かるまい。
\\	彼の講演は面白くあるまい。	
\\	彼[かれ]の 講演[こうえん]は 面白[おもしろ]くあるまい。
\\	経済の回復は不可能ではあるまい。	
\\	経済[けいざい]の 回復[かいふく]は 不可能[ふかのう]ではあるまい。
\\	私達はいつも、よりよいサービスの提供に努めています。	
\\	私[わたし] 達[たち]はいつも、よりよいサービスの 提供[ていきょう]に 努[つと]めています。
\\	外国語は新しい技術を利用すればより効果的に教えることが出来る。	
\\	外国[がいこく] 語[ご]は 新[あたら]しい 技術[ぎじゅつ]を 利用[りよう]すればより 効果[こうか] 的[てき]に 教[おし]えることが 出来[でき]る。
\\	子供達に自分で考えさせる方がより教育的だ。	
\\	子供[こども] 達[たち]に 自分[じぶん]で 考[かんが]えさせる 方[ほう]がより 教育[きょういく] 的[てき]だ。
\\	今後もより速い、より大容量の記憶装置の開発が続くだろう。	
\\	今後[こんご]もより 速[はや]い、より 大[だい] 容量[ようりょう]の 記憶[きおく] 装置[そうち]の 開発[かいはつ]が 続[つづ]くだろう。
\\	私達はより多くの人々にこの施設を利用してもらいたいと思っています。	
\\	私[わたし] 達[たち]はより 多[おお]くの 人々[ひとびと]にこの 施設[しせつ]を 利用[りよう]してもらいたいと 思[おも]っています。
\\	この車は車体をより軽くするためにアルミの合金を使っている。	
\\	この 車[くるま]は 車体[しゃたい]をより 軽[かる]くするためにアルミの 合金[ごうきん]を 使[つか]っている。
\\	「より軽く、より薄く、より小さく」が我が社のモットーだ。	
\\	「より 軽[かる]く、より 薄[うす]く、より 小[ちい]さく」が 我[わ]が 社[しゃ]のモットーだ。
\\	原色を使うと、より刺激的になる。	
\\	原色[げんしょく]を 使[つか]うと、より 刺激[しげき] 的[てき]になる。
\\	電話回線を利用すれば、より広い地域のユーザーにこのサービスが提供出来る。	
\\	電話[でんわ] 回線[かいせん]を 利用[りよう]すれば、より 広[ひろ]い 地域[ちいき]のユーザーにこのサービスが 提供[ていきょう] 出来[でき]る。
\\	集中管理システムの方がより効率的だろう。	
\\	集中[しゅうちゅう] 管理[かんり]システムの 方[ほう]がより 効率[こうりつ] 的[てき]だろう。
\\	バスで通勤する方がより経済的です。	
\\	バスで 通勤[つうきん]する 方[ほう]がより 経済[けいざい] 的[てき]です。
\\	私達はいつも今よりよいサービスの提供に努めています。	
\\	私[わたし] 達[たち]はいつも 今[いま]よりよいサービスの 提供[ていきょう]に 努[つと]めています。
\\	外国語は新しい技術を利用すれば、そうしないより効果的に教えることが出来る。	
\\	外国[がいこく] 語[ご]は 新[あたら]しい 技術[ぎじゅつ]を 利用[りよう]すれば、そうしないより 効果[こうか] 的[てき]に 教[おし]えることが 出来[でき]る。
\\	子供達に自分で教えさせる方が、そうしないより教育的だ。	
\\	子供[こども] 達[たち]に 自分[じぶん]で 教[おし]えさせる 方[ほう]が、そうしないより 教育[きょういく] 的[てき]だ。
\\	もっと大きいのはありませんか。	
\\	もっと 大[おお]きいのはありませんか。
\\	あの映画はもっと面白いと思っていたのだが、期待外れだった。	
\\	あの 映画[えいが]はもっと 面白[おもしろ]いと 思[おも]っていたのだが、 期待[きたい] 外[はず]れだった。
\\	もっと(たくさん)召し上がって下さい。	
\\	もっと(たくさん) 召し上[めしあ]がって 下[くだ]さい。
\\	バスで通勤する方がもっと経済的です。	
\\	バスで 通勤[つうきん]する 方[ほう]がもっと 経済[けいざい] 的[てき]です。
\\	それは高かったが、教科書だったから買わざるを得なかった。	
\\	それは 高[たか]かったが、 教科書[きょうかしょ]だったから 買[か]わざるを 得[え]なかった。
\\	日本へ行けば、日本の習慣に従わざるを得ない。	
\\	日本[にほん]へ 行[い]けば、 日本[にほん]の 習慣[しゅうかん]に 従[したが]わざるを 得[え]ない。
\\	上司に飲みに行こうと言われれば、部下は行かざるを得ない。	
\\	上司[じょうし]に 飲[の]みに 行[い]こうと 言[い]われれば、 部下[ぶか]は 行[い]かざるを 得[え]ない。
\\	これだけの人が拳銃で殺されているのだから、拳銃所持をより厳しくコントロールせざるを得ないと思う。	
\\	これだけの 人[ひと]が 拳銃[けんじゅう]で 殺[ころ]されているのだから、 拳銃[けんじゅう] 所持[しょじ]をより 厳[きび]しくコントロールせざるを 得[え]ないと 思[おも]う。
\\	ジョンは一年で日本語がすらすら話せて、新聞が楽に読めるようになったのだから、語学の天才と言わざるを得ない。	
\\	ジョンは 一年[いちねん]で 日本語[にほんご]がすらすら 話[はな]せて、 新聞[しんぶん]が 楽[らく]に 読[よ]めるようになったのだから、 語学[ごがく]の 天才[てんさい]と 言[い]わざるを 得[え]ない。
\\	日本は天然資源の乏しい国だから、経済発展のためには輸出に依存せざるを得ないのである。	
\\	日本[にほん]は 天然[てんねん] 資源[しげん]の 乏[とぼ]しい 国[くに]だから、 経済[けいざい] 発展[はってん]のためには 輸出[ゆしゅつ]に 依存[いぞん]せざるを 得[え]ないのである。
\\	映画に行く約束があったので、すぐ出かけなければならなかった。	
\\	映画[えいが]に 行[い]く 約束[やくそく]があったので、すぐ 出[で]かけなければならなかった。
\\	社長からの命令だったので、すぐ出かけざるを得なかった。。	
\\	社長[しゃちょう]からの 命令[めいれい]だったので、すぐ 出[で]かけざるを 得[え]なかった。。
\\	あれ、財布がどこかへ行ってしまったぞ。	
\\	あれ、 財布[さいふ]がどこかへ 行[い]ってしまったぞ。
\\	急がないとバスに乗り遅れるぞ。	
\\	急[いそ]がないとバスに 乗り遅[のりおく]れるぞ。
\\	今日はあの子とデートが出来るぞ。	
\\	今日[きょう]はあの 子[こ]とデートが 出来[でき]るぞ。
\\	変だぞ、女房の顔が猫の顔に見える。	
\\	変[へん]だぞ、 女房[にょうぼう]の 顔[かお]が 猫[ねこ]の 顔[かお]に 見[み]える。
\\	今度こそ文部省の奨学金をもらってやるぞ。	
\\	今度[こんど]こそ 文部省[もんぶしょう]の 奨学[しょうがく] 金[きん]をもらってやるぞ。
\\	今日の日本語の試験はうまくいったぞ。	
\\	今日[きょう]の 日本語[にほんご]の 試験[しけん]はうまくいったぞ。
\\	今日は寒いから、オーバーを着て行った方がいいぞ。	
\\	今日[きょう]は 寒[さむ]いから、オーバーを 着[き]て 行[い]った 方[ほう]がいいぞ。
\\	このケーキ、食べないなら、俺が食べちゃうぞ。	
\\	このケーキ、 食[た]べないなら、 俺[おれ]が 食[た]べちゃうぞ。
\\	同じ間違いをもう一度したら、許さないぞ。	
\\	同[おな]じ 間違[まちが]いをもう 一度[いちど]したら、 許[ゆる]さないぞ。
\\	おい、この酒、熱くないぞ。	
\\	おい、この 酒[さけ]、 熱[あつ]くないぞ。
\\	予告もなしに来るなんて失礼だぞ。	
\\	予告[よこく]もなしに 来[く]るなんて 失礼[しつれい]だぞ。
\\	最近日本の政治はずいぶん変わったよね。	
\\	最近[さいきん] 日本[にほん]の 政治[せいじ]はずいぶん 変[か]わったよね。
\\	日本人は世間体を気にするよね。	
\\	日本人[にほんじん]は 世間体[せけんてい]を 気[き]にするよね。
\\	酒を飲み過ぎると、病気になりますよ。	
\\	酒[さけ]を 飲[の]み 過[す]ぎると、 病気[びょうき]になりますよ。
\\	喜び合ってくれる人、悲しみ合ってくれる人が本当の友達だ。	
\\	喜[よろこ]び 合[あ]ってくれる 人[ひと]、 悲[かな]しみ 合[あ]ってくれる 人[ひと]が 本当[ほんとう]の 友達[ともだち]だ。
\\	二人は小さなことで言い合っていた。	
\\	二人[ふたり]は 小[ちい]さなことで 言い合[いいあ]っていた。
\\	この家庭では両親と子供達がよく話し合う。	
\\	この 家庭[かてい]では 両親[りょうしん]と 子供[こども] 達[たち]がよく 話し合[はなしあ]う。
\\	僕達はお金を出し合って、アパートを借りている。	
\\	僕達[ぼくたち]はお 金[かね]を 出し合[だしあ]って、アパートを 借[か]りている。
\\	男女が公園のベンチで抱き合っている。	
\\	男女[だんじょ]が 公園[こうえん]のベンチで 抱き合[だきあ]っている。
\\	男は立ち上がって、来客と握手をした。	
\\	男[おとこ]は 立ち上[たちあ]がって、 来客[らいきゃく]と 握手[あくしゅ]をした。
\\	私が知らないでしっぽを踏んだら猫は飛び上がった。	
\\	私[わたし]が 知[し]らないでしっぽを 踏[ふ]んだら 猫[ねこ]は 飛び上[とびあ]がった。
\\	私はお寺の階段を駆け上がった。	
\\	私[わたし]はお 寺[てら]の 階段[かいだん]を 駆[か]け 上[あ]がった。
\\	曇っていた空が急に晴れ上がった。	
\\	曇[くも]っていた 空[そら]が 急[きゅう]に 晴れ上[はれあ]がった。
\\	毎日すしを食べていたのに、すしは食べ飽きない。	
\\	毎日[まいにち]すしを 食[た]べていたのに、すしは 食[た]べ 飽[あ]きない。
\\	このワインはもう飲み飽きたよ。	
\\	このワインはもう 飲[の]み 飽[あ]きたよ。
\\	あの人の話はいつも同じなので、聞き飽きた。	
\\	あの 人[ひと]の 話[はなし]はいつも 同[おな]じなので、 聞[き]き 飽[あ]きた。
\\	このビデオは何度も見たので、見飽きたよ。	
\\	このビデオは 何[なん] 度[ど]も 見[み]たので、 見飽[みあ]きたよ。
\\	すみません、この箱を持ち上げて下さいませんか。	
\\	すみません、この 箱[はこ]を 持ち上[もちあ]げて 下[くだ]さいませんか。
\\	卒業論文をようやく書き上げた。	
\\	卒業[そつぎょう] 論文[ろんぶん]をようやく 書き上[かきあ]げた。
\\	先生のお名前は前々から存じ上げておりました。	
\\	先生[せんせい]のお 名前[なまえ]は 前々[まえまえ]から 存じ上[ぞんじあ]げておりました。
\\	「女」という漢字と「子」という漢字を組み合わせると、「好(き)」という漢字になる。	
\\	女 
\\	子 
\\	好 
\\	女[おんな]」という 漢字[かんじ]と
\\	子[こ]」という 漢字[かんじ]を 組み合[くみあ]わせると、
\\	好[す](き)」という 漢字[かんじ]になる。
\\	私はオートミールとヨーグルトとバナナを混ぜ合わせて、食べています。	
\\	私[わたし]はオートミールとヨーグルトとバナナを 混ぜ合[まぜあ]わせて、 食[た]べています。
\\	僕は彼女と駅前の喫茶店で待ち合わせた。	
\\	僕[ぼく]は 彼女[かのじょ]と 駅前[えきまえ]の 喫茶店[きっさてん]で 待ち合[まちあ]わせた。
\\	私達二人は同じ空港のロビーに居合わせた。	
\\	私[わたし] 達[たち] 二人[ふたり]は 同[おな]じ 空港[くうこう]のロビーに 居合[いあ]わせた。
\\	私達は偶然同じバスに乗り合わせた。	
\\	私[わたし] 達[たち]は 偶然[ぐうぜん] 同[おな]じバスに 乗り合[のりあ]わせた。
\\	ピアノを家に運び入れるのが大変だった。	
\\	ピアノを 家[いえ]に 運[はこ]び 入[い]れるのが 大変[たいへん]だった。
\\	ここに住所と電話番号を書き入れて下さい。	
\\	ここに 住所[じゅうしょ]と 電話[でんわ] 番号[ばんごう]を 書き入[かきい]れて 下[くだ]さい。
\\	地下鉄がやがてこの辺りにも乗り入れるはずだ。	
\\	地下鉄[ちかてつ]がやがてこの 辺[あた]りにも 乗り入[のりい]れるはずだ。
\\	明治時代に日本は西洋からいろいろな物を取り入れた。	
\\	明治[めいじ] 時代[じだい]に 日本[にほん]は 西洋[せいよう]からいろいろな 物[もの]を 取り入[とりい]れた。
\\	社長は私の辞職を聞き入れなかった。	
\\	社長[しゃちょう]は 私[わたし]の 辞職[じしょく]を 聞き入[ききい]れなかった。
\\	母に手紙を書き送った。	
\\	母[はは]に 手紙[てがみ]を 書き送[かきおく]った。
\\	友達を空港で見送った。	
\\	友達[ともだち]を 空港[くうこう]で 見送[みおく]った。
\\	先方にはその旨言い送りましょう。	
\\	先方[せんぽう]にはその 旨[むね] 言[い]い 送[おく]りましょう。
\\	コンクリートの天井が急に崩れ落ちて来た。	
\\	コンクリートの 天井[てんじょう]が 急[きゅう]に 崩れ落[くずれお]ちて 来[き]た。
\\	火事でデパートが焼け落ちた。	
\\	火事[かじ]でデパートが 焼け落[やけお]ちた。
\\	祖母が階段を転げ落ちた。	
\\	祖母[そぼ]が 階段[かいだん]を 転げ落[ころげお]ちた。
\\	地震で本が棚から滑り落ちた。	
\\	地震[じしん]で 本[ほん]が 棚[たな]から 滑り落[すべりお]ちた。
\\	僕は山を駆け降りた。	
\\	僕[ぼく]は 山[やま]を 駆[か]け 降[お]りた。
\\	地震の時、窓から飛び降りた。	
\\	地震[じしん]の 時[とき]、 窓[まど]から 飛び降[とびお]りた。
\\	木に登った猫を引き下ろした。	
\\	木[き]に 登[のぼ]った 猫[ねこ]を 引き下[ひきお]ろした。
\\	警官はデモ隊に棍棒を打ち下ろした。	
\\	警官[けいかん]はデモ 隊[たい]に 棍棒[こんぼう]を 打[う]ち 下[お]ろした。
\\	私の家は湖を見下ろす所に立っている。	
\\	私[わたし]の 家[いえ]は 湖[みずうみ]を 見下[みお]ろす 所[ところ]に 立[た]っている。
\\	山の方から冷たい風が吹き下ろして来る。	
\\	山[やま]の 方[ほう]から 冷[つめ]たい 風[かぜ]が 吹[ふ]き 下[お]ろして 来[く]る。
\\	小説家は新しい小説を書き下ろした。	
\\	小説[しょうせつ] 家[か]は 新[あたら]しい 小説[しょうせつ]を 書き下[かきお]ろした。
\\	博士論文をやっと書き終わった。	
\\	博士[はくし] 論文[ろんぶん]をやっと 書[か]き 終[お]わった。
\\	トルストイの「戦争と平和」を読み終えた。	
\\	トルストイの
\\	戦争[せんそう]と 平和[へいわ]」を 読[よ]み 終[お]えた。
\\	映画を見終わった時、もう夜中を過ぎていた。	
\\	映画[えいが]を 見[み] 終[お]わった 時[とき]、もう 夜中[よなか]を 過[す]ぎていた。
\\	この新聞、もう読み終わった?	
\\	この 新聞[しんぶん]、もう 読[よ]み 終[お]わった?
\\	昔読んだ漱石の小説を読み返して見た。	
\\	昔[むかし] 読[よ]んだ 漱石[そうせき]の 小説[しょうせつ]を 読み返[よみかえ]して 見[み]た。
\\	大事な物を持って行かれそうになったので、あわてて取り返した。	
\\	大事[だいじ]な 物[もの]を 持[も]って 行[い]かれそうになったので、あわてて 取り返[とりかえ]した。
\\	殴られたから、殴り返した。	
\\	殴[なぐ]られたから、 殴[なぐ]り 返[かえ]した。
\\	そいつがひどいことを言ったので、俺も言い返したんだ。	
\\	そいつがひどいことを 言[い]ったので、 俺[おれ]も 言い返[いいかえ]したんだ。
\\	あの男は同じことを何度も繰り返す。	
\\	あの 男[おとこ]は 同[おな]じことを 何[なん] 度[ど]も 繰り返[くりかえ]す。
\\	買ったばかりのテレビの調子が悪いので、取り換えてもらった。	
\\	買[か]ったばかりのテレビの 調子[ちょうし]が 悪[わる]いので、 取り換[とりか]えてもらった。
\\	ソファーの位置が変だったので、置き換えてみた。	
\\	ソファーの 位置[いち]が 変[へん]だったので、 置き換[おきか]えてみた。
\\	本棚の本を入れ換えた。	
\\	本棚[ほんだな]の 本[ほん]を 入れ換[いれか]えた。
\\	新宿で
\\	山手線に乗り換えた。	
\\	新宿[しんじゅく]で 
\\	山手[やまのて] 線[せん]に 乗り換[のりか]えた。
\\	彼女はきれいな洋服に着替えた。	
\\	彼女[かのじょ]はきれいな 洋服[ようふく]に 着替[きが]えた。
\\	父は若い時に、海でおぼれかかったそうだ。	
\\	父[ちち]は 若[わか]い 時[とき]に、 海[うみ]でおぼれかかったそうだ。
\\	ドアが開きかかったが、また閉まってしまった。	
\\	ドアが 開[ひら]きかかったが、また 閉[し]まってしまった。
\\	僕は車の事故で死にかかった。	
\\	僕[ぼく]は 車[くるま]の 事故[じこ]で 死[し]にかかった。
\\	森の上に月が出かかった。	
\\	森[もり]の 上[うえ]に 月[つき]が 出[で]かかった。
\\	月曜日に始めた仕事も金曜日には出来かかっていた。	
\\	月曜日[げつようび]に 始[はじ]めた 仕事[しごと]も 金曜日[きんようび]には 出来[でき]かかっていた。
\\	道を歩いていたら、男が僕に殴りかかってきた。	
\\	道[みち]を 歩[ある]いていたら、 男[おとこ]が 僕[ぼく]に 殴[なぐ]りかかってきた。
\\	花子は太郎の肩にもたれかかった。	
\\	花子[はなこ]は 太郎[たろう]の 肩[かた]にもたれかかった。
\\	犬が子供に飛びかかった。	
\\	犬[いぬ]が 子供[こども]に 飛[と]びかかった。
\\	ご飯を食べかけた時に、地震が起きた。	
\\	ご 飯[はん]を 食[た]べかけた 時[とき]に、 地震[じしん]が 起[お]きた。
\\	父は若い時に、海でおぼれかけたそうだ。	
\\	父[ちち]は 若[わか]い 時[とき]に、 海[うみ]でおぼれかけたそうだ。
\\	雨が降りかけたが、また天気になった。	
\\	雨[あめ]が 降[ふ]りかけたが、また 天気[てんき]になった。
\\	飲みかけたビールのコップがテーブルの上にあった。	
\\	飲[の]みかけたビールのコップがテーブルの 上[うえ]にあった。
\\	ドアが開きかけたが、また閉まってしまった。	
\\	ドアが 開[あ]きかけたが、また 閉[し]まってしまった。
\\	卒論を書きかけたんですが、まだ完成していません。	
\\	卒論[そつろん]を 書[か]きかけたんですが、まだ 完成[かんせい]していません。
\\	電車で、向こう側に座っていた美人が僕に笑いかけた。	
\\	電車[でんしゃ]で、 向[む]こう 側[がわ]に 座[すわ]っていた 美人[びじん]が 僕[ぼく]に 笑[わら]いかけた。
\\	彼は日本人に会うと日本語で話しかけることにしている。	
\\	彼[かれ]は 日本人[にほんじん]に 会[あ]うと 日本語[にほんご]で 話[はな]しかけることにしている。
\\	大統領候補が道行く人に呼びかけていた。	
\\	大統領[だいとうりょう] 候補[こうほ]が 道行[みちゆ]く 人[ひと]に 呼[よ]びかけていた。
\\	トルストイの「戦争と平和」を読みきった。	
\\	トルストイの
\\	戦争[せんそう]と 平和[へいわ]」を 読[よ]みきった。
\\	今月の予算を全部使いきってしまった。	
\\	今月[こんげつ]の 予算[よさん]を 全部[ぜんぶ] 使[つか]いきってしまった。
\\	これだけの漢字を覚えきれますか。	
\\	これだけの 漢字[かんじ]を 覚[おぼ]えきれますか。
\\	彼女は25マイルマラソンを走りきった。	
\\	彼女[かのじょ]は 25マイルマラソンを 走[はし]りきった。
\\	いろいろなレストランでカレーライスを食べ比べた。	
\\	いろいろなレストランでカレーライスを 食[た]べ 比[くら]べた。
\\	世界のビールを飲み比べてみた。	
\\	世界[せかい]のビールを 飲[の]み 比[くら]べてみた。
\\	子供はケーキの大きさをよく見比べてから、大きい方を取った。	
\\	子供[こども]はケーキの 大[おお]きさをよく 見比[みくら]べてから、 大[おお]きい 方[ほう]を 取[と]った。
\\	いろいろな新聞を読み比べてみたが、やっぱり、ニューヨークタイムズが一番いい。	
\\	いろいろな 新聞[しんぶん]を 読み比[よみくら]べてみたが、やっぱり、ニューヨークタイムズが 一番[いちばん]いい。
\\	風邪を引いたので、今日の出張を取り消したいんですが。	
\\	風邪[かぜ]を 引[ひ]いたので、 今日[きょう]の 出張[しゅっちょう]を 取り消[とりけ]したいんですが。
\\	山田氏は自分から醜聞にからんでいることを強く打ち消した。	
\\	山田[やまだ] 氏[し]は 自分[じぶん]から 醜聞[しゅうぶん]にからんでいることを 強[つよ]く 打ち消[うちけ]した。
\\	歩行者は吸っていたたばこを足でもみ消した。	
\\	歩行[ほこう] 者[しゃ]は 吸[す]っていたたばこを 足[あし]でもみ 消[け]した。
\\	知らない男が部屋に入り込んで来た。	
\\	知[し]らない 男[おとこ]が 部屋[へや]に 入り込[はいりこ]んで 来[き]た。
\\	昨日は風邪で寝込んでしまった。	
\\	昨日[きのう]は 風邪[かぜ]で 寝込[ねこ]んでしまった。
\\	友達と夜遅くまで話し込んだ。	
\\	友達[ともだち]と 夜[よる] 遅[おそ]くまで 話し込[はなしこ]んだ。
\\	私達は朝七時の新幹線に乗り込んだ。	
\\	私[わたし] 達[たち]は 朝[あさ] 七時[しちじ]の 新幹線[しんかんせん]に 乗り込[のりこ]んだ。
\\	子供がキャンディーを飲み込んだ。	
\\	子供[こども]がキャンディーを 飲み込[のみこ]んだ。
\\	妻はブラインドを引き下げた。	
\\	妻[つま]はブラインドを 引き下[ひきさ]げた。
\\	日本銀行が金利を引き下げた。	
\\	日本銀行[にほんぎんこう]が 金利[きんり]を 引き下[ひきさ]げた。
\\	原告は訴訟を取り下げた。	
\\	原告[げんこく]は 訴訟[そしょう]を 取り下[とりさ]げた。
\\	ビールを飲みさしたコップがテーブルの上にたくさんある。	
\\	ビールを 飲[の]みさしたコップがテーブルの 上[うえ]にたくさんある。
\\	僕は読みさした本を膝に置いて寝てしまった。	
\\	僕[ぼく]は 読[よ]みさした 本[ほん]を 膝[ひざ]に 置[お]いて 寝[ね]てしまった。
\\	私は小さな文法の間違いは見過ごす。	
\\	私[わたし]は 小[ちい]さな 文法[ぶんぽう]の 間違[まちが]いは 見過[みす]ごす。
\\	初めの電車がとても混んでいたので、やり過ごした。	
\\	初[はじ]めの 電車[でんしゃ]がとても 混[こ]んでいたので、やり 過[す]ごした。
\\	疲れてバスの中で寝ていたら、自分の停留所を乗り過ごしてしまった。	
\\	疲[つか]れてバスの 中[なか]で 寝[ね]ていたら、 自分[じぶん]の 停留所[ていりゅうじょ]を 乗り過[のりす]ごしてしまった。
\\	今朝は寝過ごしてしまった。	
\\	今朝[けさ]は 寝過[ねす]ごしてしまった。
\\	その本は面白いので、どんどん先を読み進んだ。	
\\	その 本[ほん]は 面白[おもしろ]いので、どんどん 先[さき]を 読[よ]み 進[すす]んだ。
\\	警官は犯人を追いかけて、林の中を突き進んだ。	
\\	警官[けいかん]は 犯人[はんにん]を 追[お]いかけて、 林[はやし]の 中[なか]を 突き進[つきすす]んだ。
\\	彼は小説を半分ぐらいまで書き進んできた時、病気になってしまった。	
\\	彼[かれ]は 小説[しょうせつ]を 半分[はんぶん]ぐらいまで 書[か]き 進[すす]んできた 時[とき]、 病気[びょうき]になってしまった。
\\	気が向くままに、小説を書き進めていった。	
\\	気[き]が 向[む]くままに、 小説[しょうせつ]を 書[か]き 進[すす]めていった。
\\	この海底トンネルを完成するために、深さ300メートルまで掘り進めていった。	
\\	この 海底[かいてい]トンネルを 完成[かんせい]するために、 深[ふか]さ 300メートルまで 掘[ほ]り 進[すす]めていった。
\\	この理論の枠組みで研究を推し進めていくつもりだ。	
\\	この 理論[りろん]の 枠組[わくぐ]みで 研究[けんきゅう]を 推し進[おしすす]めていくつもりだ。
\\	昨日は忙しくて、パーティーに行き損ないました。	
\\	昨日[きのう]は 忙[いそが]しくて、パーティーに 行[い]き 損[そこ]ないました。
\\	寝坊していつもの電車に乗り損なった。	
\\	寝坊[ねぼう]していつもの 電車[でんしゃ]に 乗[の]り 損[そこ]なった。
\\	ニューヨークフィルのコンサートを聞き損なった。	
\\	ニューヨークフィルのコンサートを 聞き損[ききそこ]なった。
\\	彼は交通事故で死に損なった。	
\\	彼[かれ]は 交通[こうつう] 事故[じこ]で 死に損[しにそこ]なった。
\\	ちょっと付け足したいことがあるんですが、いいですか。	
\\	ちょっと 付け足[つけた]したいことがあるんですが、いいですか。
\\	一度書いてから、大分書き足した。	
\\	一度[いちど] 書[か]いてから、 大分[たいぶ] 書[が]き 足[た]した。
\\	リンゴを六つ買ったが、あっと三つ買い足した。	
\\	リンゴを 六[むっ]つ 買[か]ったが、あっと 三[みっ]つ 買[か]い 足[た]した。
\\	男はポケットからナイフを取り出して、飛びかかってきた。	
\\	男[おとこ]はポケットからナイフを 取り出[とりだ]して、 飛[と]びかかってきた。
\\	スーパーに買い物に行きますから、買って来て欲しい物を書き出して下さい。	
\\	スーパーに 買い物[かいもの]に 行[い]きますから、 買[か]って 来[き]て 欲[ほ]しい 物[もの]を 書き出[かきだ]して 下[くだ]さい。
\\	人々は窓から身を乗り出してパレードを見ている。	
\\	人々[ひとびと]は 窓[まど]から 身[み]を 乗り出[のりだ]してパレードを 見[み]ている。
\\	このアイディアは誰が考え出したんですか。	
\\	このアイディアは 誰[だれ]が 考え出[かんがえだ]したんですか。
\\	彼女に最初に会ったのがどこだったか思い出せない。	
\\	彼女[かのじょ]に 最初[さいしょ]に 会[あ]ったのがどこだったか 思い出[おもいだ]せない。
\\	図書館で日本史の本を借り出した。	
\\	図書館[としょかん]で 日本[にほん] 史[し]の 本[ほん]を 借り出[かりだ]した。
\\	あの人の話を聞くと気持ちが奮い立つ。	
\\	あの 人[ひと]の 話[はなし]を 聞[き]くと 気持[きも]ちが 奮い立[ふるいた]つ。
\\	彼女は仕事でパリに向けて飛び立った。	
\\	彼女[かのじょ]は 仕事[しごと]でパリに 向[む]けて 飛び立[とびた]った。
\\	父は思い立つとすぐ実行する人だ。	
\\	父[ちち]は 思い立[おもいた]つとすぐ 実行[じっこう]する 人[ひと]だ。
\\	今日は妻と連れ立って、映画を見に行った。	
\\	今日[きょう]は 妻[つま]と 連れ立[つれだ]って、 映画[えいが]を 見[み]に 行[い]った。
\\	私の生活は満ち足りている。	
\\	私[わたし]の 生活[せいかつ]は 満ち足[みちた]りている。
\\	これだけでは食べ足りない。	
\\	これだけでは 食[た]べ 足[た]りない。
\\	僕は推理小説が大好きだから、一冊では読み足りない。	
\\	僕[ぼく]は 推理[すいり] 小説[しょうせつ]が 大好[だいす]きだから、一冊[いっさつ]では 読[よ]み 足[た]りない。
\\	私は七時間では寝足りない。	
\\	私[わたし]は 七時間[しちじかん]では 寝[ね] 足[た]りない。
\\	私は手紙の趣旨を取り違えていた。	
\\	私[わたし]は 手紙[てがみ]の 趣旨[しゅし]を 取り違[とりちが]えていた。
\\	住所を書き違えたために、手紙を着かなかった。	
\\	住所[じゅうしょ]を 書[か]き 違[ちが]えたために、 手紙[てがみ]を 着[つ]かなかった。
\\	彼女は見違えるほどきれいになった。	
\\	彼女[かのじょ]は 見違[みちが]えるほどきれいになった。
\\	そんなにどなり散らさない下さい。	
\\	そんなにどなり 散[ち]らさない 下[くだ]さい。
\\	父は機嫌の悪い時は誰にでも当り散らした。	
\\	父[ちち]は 機嫌[きげん]の 悪[わる]い 時[とき]は 誰[だれ]にでも 当り散[あたりち]らした。
\\	私の大事な書類に子供が漢字を書き散らした。	
\\	私[わたし]の 大事[だいじ]な 書類[しょるい]に 子供[こども]が 漢字[かんじ]を 書き散[かきち]らした。
\\	新宿で成田エクスプレスに乗り継いで、成田空港まで行った。	
\\	新宿[しんじゅく]で 成田[なりた]エクスプレスに 乗り継[のりつ]いで、 成田空港[なりたくうこう]まで 行[い]った。
\\	川端康成はエピソードをいくつか書き継いで「雪国」を完成したそうだ。	
\\	川端[かわばた] 康成[やすなり]はエピソードをいくつか 書[か]き 継[つ]いで
\\	雪国[ゆきぐに]」を 完成[かんせい]したそうだ。
\\	このタイヤを取り付けてくれる?	
\\	このタイヤを 取り付[とりつ]けてくれる?
\\	約束の日を手帳に書き付けた。	
\\	約束[やくそく]の 日[ひ]を 手帳[てちょう]に 書き付[かきつ]けた。
\\	あの町は行きつけているから、迷いません。	
\\	あの 町[まち]は 行[い]きつけているから、 迷[まよ]いません。
\\	食べつけないものは食べない方がいい。	
\\	食[た]べつけないものは 食[た]べない 方[ほう]がいい。
\\	このコンピュータは使いつけているから、問題ありません。	
\\	このコンピュータは 使[つか]いつけているから、 問題[もんだい]ありません。
\\	飛行機は乗りつけていないから、すぐ酔ってしまいます。	
\\	飛行機[ひこうき]は 乗[の]りつけていないから、すぐ 酔[よ]ってしまいます。
\\	私は一時間も泳ぎ続けたので、疲れてしまった。	
\\	私[わたし]は 一時間[いちじかん]も 泳[およ]ぎ 続[つづ]けたので、 疲[つか]れてしまった。
\\	今年の梅雨は長い間雨が降り続けた。	
\\	今年[ことし]の 梅雨[つゆ]は 長[なが]い 間[あいだ] 雨[あめ]が 降[ふ]り 続[つづ]けた。
\\	私はこれからも日本語を勉強し続けるつもりです。	
\\	私[わたし]はこれからも 日本語[にほんご]を 勉強[べんきょう]し 続[つづ]けるつもりです。
\\	ラッシュアワーの電車の中で私は押しつぶされそうになった。	
\\	ラッシュアワーの 電車[でんしゃ]の 中[なか]で 私[わたし]は 押[お]しつぶされそうになった。
\\	子供が手に持っていたバナナを握りつぶしてしまった。	
\\	子供[こども]が 手[て]に 持[も]っていたバナナを 握[にぎ]りつぶしてしまった。
\\	私は台所のゴキブリを踏みつぶした。	
\\	私[わたし]は 台所[だいどころ]のゴキブリを 踏[ふ]みつぶした。
\\	僕は今までに車を十台ぐらい乗りつぶした。	
\\	僕[ぼく]は 今[いま]までに 車[くるま]を 十台[じゅうだい]ぐらい 乗[の]りつぶした。
\\	私はジョギングを始めてから、靴を何足履きつぶしたか知らない。	
\\	私[わたし]はジョギングを 始[はじ]めてから、 靴[くつ]を 何[なん] 足[そく] 履[は]きつぶしたか 知[し]らない。
\\	この本は読み通すのに一週間かかってしまった。	
\\	この 本[ほん]は 読み通[よみとお]すのに 一週間[いっしゅうかん]かかってしまった。
\\	僕はついに30マイルを走り通した。	
\\	僕[ぼく]はついに 30マイルを 走[はし]り 通[とお]した。
\\	彼は大学の四年間アルバイトをしながら、頑張り通した。	
\\	彼[かれ]は 大学[だいがく]の 四年間[よねんかん]アルバイトをしながら、 頑張[がんば]り 通[とお]した。
\\	その政治家は自分の考えを押し通した。	
\\	その 政治[せいじ] 家[か]は 自分[じぶん]の 考[かんが]えを 押し通[おしとお]した。
\\	ピストルの弾が彼の心臓を突き通した。	
\\	ピストルの 弾[たま]が 彼[かれ]の 心臓[しんぞう]を 突き通[つきとお]した。
\\	作文を書き直しました。	
\\	作文[さくぶん]を 書き直[かきなお]しました。
\\	文を読み直してみたが、まだ意味がよく分からなかった。	
\\	文[ぶん]を 読み直[よみなお]してみたが、まだ 意味[いみ]がよく 分[わ]からなかった。
\\	計算が間違っていたようなので、計算をし直した。	
\\	計算[けいさん]が 間違[まちが]っていたようなので、 計算[けいさん]をし 直[なお]した。
\\	まだ早かったので、寝直した。	
\\	まだ 早[はや]かったので、 寝[ね] 直[なお]した。
\\	ジョギングの後の汗をシャワーで洗い流した。	
\\	ジョギングの 後[あと]の 汗[あせ]をシャワーで 洗い流[あらいなが]した。
\\	その子は母親の話を聞き流していた。	
\\	その 子[こ]は 母親[ははおや]の 話[はなし]を 聞き流[ききなが]していた。
\\	濁流が家屋を押し直した。	
\\	濁流[だくりゅう]が 家屋[かおく]を 押[お]し 直[なお]した。
\\	スミス先生は筆で日本語の手紙を書き直せる。	
\\	スミス 先生[せんせい]は 筆[ふで]で 日本語[にほんご]の 手紙[てがみ]を 書き直[かきなお]せる。
\\	住み慣れたシカゴを出て、テキサス州のダラスにやって来た。	
\\	住み慣[すみな]れたシカゴを 出[で]て、テキサス 州[しゅう]のダラスにやって 来[き]た。
\\	方言は聞き慣れるまで時間がかかる。	
\\	方言[ほうげん]は 聞[き]き 慣[な]れるまで 時間[じかん]がかかる。
\\	お母さん、見慣れない人達がうちの近くにいたよ。	
\\	お 母[かあ]さん、 見慣[みな]れない 人達[ひとたち]がうちの 近[ちか]くにいたよ。
\\	履き慣れない靴で旅行しない方がいい。	
\\	履[は]き 慣[な]れない 靴[くつ]で 旅行[りょこう]しない 方[ほう]がいい。
\\	使い慣れたワープロが壊れてしまった。	
\\	使い慣[つかいな]れたワープロが 壊[こわ]れてしまった。
\\	通い慣れた四キロの道を毎日会社まで自転車で行く。	
\\	通[かよ]い 慣[な]れた四キロの 道[みち]を 毎日[まいにち] 会社[かいしゃ]まで 自転車[じてんしゃ]で 行[い]く。
\\	ホームステイの小さい子供達とも話しなれてきた。	
\\	ホームステイの 小[ちい]さい 子供[こども] 達[たち]とも 話[はな]しなれてきた。
\\	大きな公園を走り抜けると、オフィス街だった。	
\\	大[おお]きな 公園[こうえん]を 走り抜[はしりぬ]けると、オフィス 街[がい]だった。
\\	正門を通り抜けると、正面に講堂があります。	
\\	正門[せいもん]を 通り抜[とおりぬ]けると、 正面[しょうめん]に 講堂[こうどう]があります。
\\	トンネルをくぐり抜けたら、海が見えてきた。	
\\	トンネルをくぐり 抜[ぬ]けたら、 海[うみ]が 見[み]えてきた。
\\	この困難を切り抜けるためにはかなりの努力が必要だ。	
\\	この 困難[こんなん]を 切り抜[きりぬ]けるためにはかなりの 努力[どりょく]が 必要[ひつよう]だ。
\\	ジョンはメアリーの嘘が見抜けなかった。	
\\	ジョンはメアリーの 嘘[うそ]が 見抜[みぬ]けなかった。
\\	食べ残してはいけません。	
\\	食[た]べ 残[のこ]してはいけません。
\\	書き残したことがあるから、明日また書きます。	
\\	書き残[かきのこ]したことがあるから、 明日[あす]また 書[か]きます。
\\	言い残したことが一つあります。	
\\	言い残[いいのこ]したことが 一[ひと]つあります。
\\	せめて新聞ぐらい読まないと、時代に取り残されてしまいますよ。	
\\	せめて 新聞[しんぶん]ぐらい 読[よ]まないと、 時代[じだい]に 取り残[とりのこ]されてしまいますよ。
\\	昨日ビデオで見残した部分を今晩見るつもりだ。	
\\	昨日[きのう]ビデオで 見[み] 残[のこ]した 部分[ぶぶん]を 今晩[こんばん] 見[み]るつもりだ。
\\	人生を十分楽しんだから、思い残すことは一つもない。	
\\	人生[じんせい]を 十分[じゅうぶん] 楽[たの]しんだから、 思い残[おもいのこ]すことは 一[ひと]つもない。
\\	私は会社に十一時ごろまで居残って仕事をした。	
\\	私[わたし]は 会社[かいしゃ]に 十一時[じゅういちじ]ごろまで 居残[いのこ]って 仕事[しごと]をした。
\\	その飛行機事故で生き残った人はたった三人だった。	
\\	その 飛行機[ひこうき] 事故[じこ]で 生き残[いきのこ]った 人[ひと]はたった 三人[さんにん]だった。
\\	九月だというのに、藤の花がまだ咲き残っている。	
\\	九月[くがつ]だというのに、 藤[ふじ]の 花[はな]がまだ 咲き残[さきのこ]っている。
\\	デパートでは売れ残った夏の衣服の安売りをしていた。	
\\	デパートでは 売れ残[うれのこ]った 夏[なつ]の 衣服[いふく]の 安売[やすう]りをしていた。
\\	私の母校は高校野球の試合で準決勝まで勝ち残った。	
\\	私[わたし]の 母校[ぼこう]は 高校[こうこう] 野球[やきゅう]の 試合[しあい]で 準決勝[じゅんけっしょう]まで 勝ち残[かちのこ]った。
\\	今日は忙しくて一日中都内を歩き回っていた。	
\\	今日[きょう]は 忙[いそが]しくて 一日[いちにち] 中[じゅう] 都内[とない]を 歩き回[あるきまわ]っていた。
\\	彼は証券会社に勤めていた、世界を飛び回っている。	
\\	彼[かれ]は 証券[しょうけん] 会社[がいしゃ]に 勤[つと]めていた、 世界[せかい]を 飛び回[とびまわ]っている。
\\	アメリカの大学のキャンパスでは、リスが走り回っている。	
\\	アメリカの 大学[だいがく]のキャンパスでは、リスが 走り回[はしりまわ]っている。
\\	よく動き回ることは健康にいい。	
\\	よく 動き回[うごきまわ]ることは 健康[けんこう]にいい。
\\	社長はニューヨークに出張中の部下を東京に呼び戻した。	
\\	社長[しゃちょう]はニューヨークに 出張[しゅっちょう] 中[ちゅう]の 部下[ぶか]を 東京[とうきょう]に 呼び戻[よびもど]した。
\\	一度買ったコンサートの切符は払い戻してくれませんよ。	
\\	一度[いちど] 買[か]ったコンサートの 切符[きっぷ]は 払い戻[はらいもど]してくれませんよ。
\\	友達に送った小包が住所不明で送り戻されて来た。	
\\	友達[ともだち]に 送[おく]った 小包[こづつみ]が 住所[じゅうしょ] 不明[ふめい]で 送[おく]り 戻[もど]されて 来[き]た。
\\	一度売った車を買い戻した。	
\\	一度[いちど] 売[う]った 車[くるま]を 買い戻[かいもど]した。
\\	雨が降り止んだ。	
\\	雨[あめ]が 降[ふ]り 止[や]んだ。
\\	赤ん坊が泣き止んだ。	
\\	赤ん坊[あかんぼう]が 泣[な]き 止[や]んだ。
\\	雷が鳴り止んだ。	
\\	雷[かみなり]が 鳴[な]り 止[や]んだ。
\\	男は女にしつこく言い寄った。	
\\	男[おとこ]は 女[おんな]にしつこく 言い寄[いいよ]った。
\\	犬がしっぽを振りながら走り寄って来た。	
\\	犬[いぬ]がしっぽを 振[ふ]りながら 走[はし]り 寄[よ]って 来[き]た。
\\	日米関係はお互いに歩み寄らなければ改善されない。	
\\	日[にち] 米[べい] 関係[かんけい]はお 互[たが]いに 歩み寄[あゆみよ]らなければ 改善[かいぜん]されない。
\\	お暇な時にお立ち寄り下さい。	
\\	お 暇[ひま]な 時[とき]にお 立ち寄[たちよ]り 下[くだ]さい。
\\	私は東京方言と大阪方言は聞き分けられる。	
\\	私[わたし]は 東京[とうきょう] 方言[ほうげん]と 大阪[おおさか] 方言[ほうげん]は 聞き分[ききわ]けられる。
\\	中国人、韓国人、日本人を見分けるのは難しい。	
\\	中国人[ちゅうごくじん]、 韓国[かんこく] 人[じん]、 日本人[にほんじん]を 見分[みわ]けるのは 難[むずか]しい。
\\	彼女は香水をかぎ分けられる。	
\\	彼女[かのじょ]は 香水[こうすい]をかぎ 分[わ]けられる。
\\	スミスさんはアメリカのいろいろな方言が使い分けられる。	
\\	スミスさんはアメリカのいろいろな 方言[ほうげん]が 使い分[つかいわ]けられる。
\\	いろいろなご馳走が盛り分けられて、出てきた。	
\\	いろいろなご 馳走[ちそう]が 盛り分[もりわ]けられて、 出[で]てきた。
\\	答案に名前を書き忘れた。	
\\	答案[とうあん]に 名前[なまえ]を 書[か]き 忘[わす]れた。
\\	言い忘れましたが、今晩うちでパーティーがありますから、どうぞいらして下さい。	
\\	言[い]い 忘[わす]れましたが、 今晩[こんばん]うちでパーティーがありますから、どうぞいらして 下[くだ]さい。
\\	ああそうだ。電話番号を聞き忘れた。	
\\	ああそうだ。 電話[でんわ] 番号[ばんごう]を 聞[き]き 忘[わす]れた。
\\	電気を消し忘れた。	
\\	電気[でんき]を 消[け]し 忘[わす]れた。
\\	鍵をかけ忘れた。	
\\	鍵[かぎ]をかけ 忘[わす]れた。
\\	電車の網棚にかばんを置き忘れた。	
\\	電車[でんしゃ]の 網棚[あみだな]にかばんを 置き忘[おきわす]れた。
\\	よく考えた上で御返事します。	
\\	よく 考[かんが]えた 上[うえ]で 御返事[ごへんじ]します。
\\	この辞書は日本語を勉強する上で大変役に立つ。	
\\	この 辞書[じしょ]は 日本語[にほんご]を 勉強[べんきょう]する 上[うえ]で 大変[たいへん] 役に立[やくにた]つ。
\\	努力次第で日本の小説が原文で読めるようになりますよ。	
\\	努力[どりょく] 次第[しだい]で 日本[にほん]の 小説[しょうせつ]が 原文[げんぶん]で 読[よ]めるようになりますよ。
\\	風が我々の進行方向と逆に吹いている。	
\\	風[かぜ]が 我々[われわれ]の 進行[しんこう] 方向[ほうこう]と 逆[ぎゃく]に 吹[ふ]いている。
\\	彼は私と逆の見方をした。	
\\	彼[かれ]は 私[わたし]と 逆[ぎゃく]の 見方[みかた]をした。
\\	プラスチックはガラスに代わる材料として広く利用されている。	
\\	プラスチックはガラスに 代[か]わる 材料[ざいりょう]として 広[ひろ]く 利用[りよう]されている。
\\	野口さんは私と違って手先が器用だ。	
\\	野口[のぐち]さんは 私[わたし]と 違[ちが]って 手先[てさき]が 器用[きよう]だ。
\\	ジョンは私と違った考えを持っている。	
\\	ジョンは 私[わたし]と 違[ちが]った 考[かんが]えを 持[も]っている。
\\	サッカーは今や野球と並んで日本の代表的スポーツだ。	
\\	サッカーは 今[いま]や 野球[やきゅう]と 並[なら]んで 日本[にほん]の 代表[だいひょう] 的[てき]スポーツだ。
\\	鷗外は漱石と並ぶ日本文学の巨砲だ。	
\\	鷗外[おうがい]は 漱石[そうせき]と 並[なら]ぶ 日本[にほん] 文学[ぶんがく]の 巨砲[きょほう]だ。
\\	ハイウェーが鉄道と並行して走っている。	
\\	ハイウェーが 鉄道[てつどう]と 並行[へいこう]して 走[はし]っている。
\\	この町では車なしでは生活できない。	
\\	この 町[まち]では 車[くるま]なしでは 生活[せいかつ]できない。
\\	政府の援助なしにはこの事業は進められない。	
\\	政府[せいふ]の 援助[えんじょ]なしにはこの 事業[じぎょう]は 進[すす]められない。
\\	ドイツ留学に当たって田中先生の助言を受けた。	
\\	ドイツ 留学[りゅうがく]に 当[あ]たって 田中[たなか] 先生[せんせい]の 助言[じょげん]を 受[う]けた。
\\	新しい事業を始めるに当たって、二千万円ほどの資金を確保しなければならない。	
\\	新[あたら]しい 事業[じぎょう]を 始[はじ]めるに 当[あ]たって、二千[にせん] 万[まん] 円[えん]ほどの 資金[しきん]を 確保[かくほ]しなければならない。
\\	新社長の就任に当たってのあいさつは社員に感銘を与えたようだ。	
\\	新[しん] 社長[しゃちょう]の 就任[しゅうにん]に 当[あ]たってのあいさつは 社員[しゃいん]に 感銘[かんめい]を 与[あた]えたようだ。
\\	緊急に際しての処置をよく習得しておいて下さい。	
\\	緊急[きんきゅう]に 際[さい]しての 処置[しょち]をよく 習得[しゅうとく]しておいて 下[くだ]さい。
\\	予算に合わせて家具を選んだ。	
\\	予算[よさん]に 合[あ]わせて 家具[かぐ]を 選[えら]んだ。
\\	私達は音楽に合せて踊った。	
\\	私[わたし] 達[たち]は 音楽[おんがく]に 合[あわ]せて 踊[おど]った。
\\	業績に応じて従業員全員にボーナスが支給された。	
\\	業績[ぎょうせき]に 応[おう]じて 従業[じゅうぎょう] 員[いん] 全員[ぜんいん]にボーナスが 支給[しきゅう]された。
\\	組合の要求に応じて標準就業時間が短縮されることになった。	
\\	組合[くみあい]の 要求[ようきゅう]に 応[おう]じて 標準[ひょうじゅん] 就業[しゅうぎょう] 時間[じかん]が 短縮[たんしゅく]されることになった。
\\	収入に応じた住居を選ぶべきです。	
\\	収入[しゅうにゅう]に 応[おう]じた 住居[じゅうきょ]を 選[えら]ぶべきです。
\\	この会社では性別にかかわらず、同様に昇進できる。	
\\	この 会社[かいしゃ]では 性別[せいべつ]にかかわらず、 同様[どうよう]に 昇進[しょうしん]できる。
\\	我々は助成金が下りるかどうかにかかわらず、このプロジェクトを進めるつもりだ。	
\\	我々[われわれ]は 助成[じょせい] 金[きん]が 下[お]りるかどうかにかかわらず、このプロジェクトを 進[すす]めるつもりだ。
\\	誰によらず、規則を犯した者は罰せられる。	
\\	誰[だれ]によらず、 規則[きそく]を 犯[おか]した 者[もの]は 罰[ばっ]せられる。
\\	これは生死にかかわる重大事だ。	
\\	これは 生死[せいし]にかかわる 重[じゅう] 大事[だいじ]だ。
\\	見合い結婚は日本に限らず、ほかの国でも行われている。	
\\	見合[みあ]い 結婚[けっこん]は 日本[にほん]に 限[かぎ]らず、ほかの 国[くに]でも 行[おこな]われている。
\\	彼に限ってそんなことはしない。	
\\	彼[かれ]に 限[かぎ]ってそんなことはしない。
\\	その日に限って妙子は留守だった。	
\\	その 日[ひ]に 限[かぎ]って 妙子[たえこ]は 留守[るす]だった。
\\	忙しい時に限ってよく電話がかかってくる。	
\\	忙[いそが]しい 時[とき]に 限[かぎ]ってよく 電話[でんわ]がかかってくる。
\\	今日、東海から関東にかけて、大雨が降った。	
\\	今日[きょう]、 東海[とうかい]から 関東[かんとう]にかけて、 大雨[おおあめ]が 降[ふ]った。
\\	週末にかけてワシントンに行きます。	
\\	週末[しゅうまつ]にかけてワシントンに 行[い]きます。
\\	金もうけにかけては、正男は天才だ。	
\\	金[かね]もうけにかけては、 正男[まさお]は 天才[てんさい]だ。
\\	彼女は美貌に加えて演技力がすばらしい。	
\\	彼女[かのじょ]は 美貌[びぼう]に 加[くわ]えて 演技[えんぎ] 力[りょく]がすばらしい。
\\	この実験に際して何人かのアメリカの専門家の協力を得た。	
\\	この 実験[じっけん]に 際[さい]して 何[なん] 人[にん]かのアメリカの 専門[せんもん] 家[か]の 協力[きょうりょく]を 得[え]た。
\\	日本を離れるに際し家財道具を全部処分した。	
\\	日本[にほん]を 離[はな]れるに 際[さい]し 家財道具[かざいどうぐ]を 全部[ぜんぶ] 処分[しょぶん]した。
\\	沢田は社長の命令に従ってただちにマニラに飛んだ。	
\\	沢田[さわだ]は 社長[しゃちょう]の 命令[めいれい]に 従[したが]ってただちにマニラに 飛[と]んだ。
\\	彼はアメリカの大学で学位を取ったにしては英語が下手だ。	
\\	彼[かれ]はアメリカの 大学[だいがく]で 学位[がくい]を 取[と]ったにしては 英語[えいご]が 下手[へた]だ。
\\	道路は海岸に沿って走っている。	
\\	道路[どうろ]は 海岸[かいがん]に 沿[そ]って 走[はし]っている。
\\	指導書に沿った教え方をして下さい。	
\\	指導[しどう] 書[しょ]に 沿[そ]った 教[おし]え 方[かた]をして 下[くだ]さい。
\\	人口の急増に伴って、いろいろの問題が出てきた。	
\\	人口[じんこう]の 急増[きゅうぞう]に 伴[ともな]って、いろいろの 問題[もんだい]が 出[で]てきた。
\\	景気が回復するに伴い失業率が下がってきた。	
\\	景気[けいき]が 回復[かいふく]するに 伴[ともな]い 失業[しつぎょう] 率[りつ]が 下[さ]がってきた。
\\	医者はその手術に伴う危険を患者の家族に説明した。	
\\	医者[いしゃ]はその 手術[しゅじゅつ]に 伴[ともな]う 危険[きけん]を 患者[かんじゃ]の 家族[かぞく]に 説明[せつめい]した。
\\	台風は時速三十キロの速さで北に向かって進んでいる。	
\\	台風[たいふう]は 時速[じそく] 三十キロ[さんじゅっきろ] の 速[はや]さで 北[きた]に 向[む]かって 進[すす]んでいる。
\\	和男は来年の大学入試に向けて猛勉強している。	
\\	和男[かずお]は 来年[らいねん]の 大学[だいがく] 入試[にゅうし]に 向[む]けて 猛[もう] 勉強[べんきょう]している。
\\	現在、石油に代わる新しいエネルギー開発に向けての研究が進められている。	
\\	現在[げんざい]、 石油[せきゆ]に 代[か]わる 新[あたら]しいエネルギー 開発[かいはつ]に 向[む]けての 研究[けんきゅう]が 進[すす]められている。
\\	先生によって教え方が違う。	
\\	先生[せんせい]によって 教[おし]え 方[かた]が 違[ちが]う。
\\	どこに泊まるかによって費用が大分変わる。	
\\	どこに 泊[と]まるかによって 費用[ひよう]が 大分[だいぶ] 変[か]わる。
\\	この研究所は日本政府によって設立された。	
\\	この 研究所[けんきゅうじょ]は 日本[にほん] 政府[せいふ]によって 設立[せつりつ]された。
\\	武力ではなく、話し合いによる解決が望まれる。	
\\	武力[ぶりょく]ではなく、 話し合[はなしあ]いによる 解決[かいけつ]が 望[のぞ]まれる。
\\	この会社は学歴によらず、実力さえあれば大きな仕事をさせてくれる。	
\\	この 会社[かいしゃ]は 学歴[がくれき]によらず、 実力[じつりょく]さえあれば 大[おお]きな 仕事[しごと]をさせてくれる。
\\	採否によらず結果は連絡します。	
\\	採否[さいひ]によらず 結果[けっか]は 連絡[れんらく]します。
\\	百メートルにわたって堤防が崩れた。	
\\	百[ひゃく]メートルにわたって 堤防[ていぼう]が 崩[くず]れた。
\\	そのドラマは一週間にわたり放映された。	
\\	そのドラマは 一週間[いっしゅうかん]にわたり 放映[ほうえい]された。
\\	二か月にわたるミュージカルの興行が先週終わった。	
\\	二か月[にかげつ]にわたるミュージカルの 興行[こうぎょう]が 先週[せんしゅう] 終[お]わった。
\\	来年研究のために日本へ行く予定です。	
\\	来年[らいねん] 研究[けんきゅう]のために 日本[にほん]へ 行[い]く 予定[よてい]です。
\\	州の財政難のために学校が閉鎖された。	
\\	州[しゅう]の 財政難[ざいせいなん]のために 学校[がっこう]が 閉鎖[へいさ]された。
\\	田中先生は日本へ行く学生のためにオリエンテーションをした。	
\\	田中[たなか] 先生[せんせい]は 日本[にほん]へ 行[い]く 学生[がくせい]のためにオリエンテーションをした。
\\	今外国人のための和英辞典を書いています。	
\\	今[いま] 外国[がいこく] 人[じん]のための 和英[わえい] 辞典[じてん]を 書[か]いています。
\\	日本語と英語のほかに何か話せますか。	
\\	日本語[にほんご]と 英語[えいご]のほかに 何[なに]か 話[はな]せますか。
\\	この仕事が出来る人はブラウンさんをおいてほかにない。	
\\	この 仕事[しごと]が 出来[でき]る 人[ひと]はブラウンさんをおいてほかにない。
\\	私達は弁護士を介してその問題を協議した。	
\\	私[わたし] 達[たち]は 弁護士[べんごし]を 介[かい]してその 問題[もんだい]を 協議[きょうぎ]した。
\\	私は友達を介してこの仕事を紹介してもらった。	
\\	私[わたし]は 友達[ともだち]を 介[かい]してこの 仕事[しごと]を 紹介[しょうかい]してもらった。
\\	通訳を介しての話し合いだったので、こちらの意図が十分相手に伝わったかどうか心もこない。	
\\	通訳[つうやく]を 介[かい]しての 話し合[はなしあ]いだったので、こちらの 意図[いと]が 十分[じゅうぶん] 相手[あいて]に 伝[つた]わったかどうか 心[こころ]もこない。
\\	地球は太陽を中心に回っている。	
\\	地球[ちきゅう]は 太陽[たいよう]を 中心[ちゅうしん]に 回[まわ]っている。
\\	今度の台風は関東を中心に大雨を降らせた。	
\\	今度[こんど]の 台風[たいふう]は 関東[かんとう]を 中心[ちゅうしん]に 大雨[おおあめ]を 降[ふ]らせた。
\\	しりあいを通じて木村氏に面会を申し込んだ。	
\\	しりあいを 通[つう]じて 木村[きむら] 氏[し]に 面会[めんかい]を 申し込[もうしこ]んだ。
\\	これは上司を通じての申し込みなので断るわけにはいかない。	
\\	これは 上司[じょうし]を 通[つう]じての 申し込[もうしこ]みなので 断[ことわ]るわけにはいかない。
\\	壁を通して隣の部屋の話し声が聞こえる。	
\\	壁[かべ]を 通[とお]して 隣[となり]の 部屋[へや]の 話し声[はなしごえ]が 聞[き]こえる。
\\	この島は一年を通じて気候が温暖だ。	
\\	この 島[しま]は 一年[いちねん]を 通[つう]じて 気候[きこう]が 温暖[おんだん]だ。
\\	私は除いてここには誰も日本語の分かる者がいない。	
\\	私[わたし]は 除[のぞ]いてここには 誰[だれ]も 日本語[にほんご]の 分[わ]かる 者[もの]がいない。
\\	ケンとスーザンを除いたクラスの全員が留学経験者だ。	
\\	ケンとスーザンを 除[のぞ]いたクラスの 全員[ぜんいん]が 留学[りゅうがく] 経験[けいけん] 者[しゃ]だ。
\\	その飛行機はロンドンを経てベルリンへ向かう。	
\\	その 飛行機[ひこうき]はロンドンを 経[へ]てベルリンへ 向[む]かう。
\\	クリントン氏はアーカンソー州知事を経て大統領になった。	
\\	クリントン 氏[し]はアーカンソー 州[しゅう] 知事[ちじ]を 経[へ]て 大統領[だいとうりょう]になった。
\\	健二は国立大学を目指して勉強している。	
\\	健二[けんじ]は 国立[こくりつ] 大学[だいがく]を 目指[めざ]して 勉強[べんきょう]している。
\\	我々はクリーンなエネルギーを目指した研究に取り組んでいる。	
\\	我々[われわれ]はクリーンなエネルギーを 目指[めざ]した 研究[けんきゅう]に 取り組[とりく]んでいる。
\\	この領収書をもって保証書に代えさせていただきます。	
\\	この 領収[りょうしゅう] 書[しょ]をもって 保証書[ほしょうしょ]に 代[か]えさせていただきます。
\\	これをもって本大会を閉会します。	
\\	これをもって 本[ほん] 大会[たいかい]を 閉会[へいかい]します。
\\	島田氏は一月一日をもって我が社の社長に就任されます。	
\\	島田[しまだ] 氏[し]は 一月[いちがつ] 一日[ついたち]をもって 我[わ]が 社[しゃ]の 社長[しゃちょう]に 就任[しゅうにん]されます。
\\	私は今朝ひどく頭が痛かった。だから、会社に行かなかった。	
\\	私[わたし]は 今朝[けさ]ひどく 頭[あたま]が 痛[いた]かった。だから、 会社[かいしゃ]に 行[い]かなかった。
\\	日本へ行って日本語を勉強したい。しかし、旅費が高くて行けない。	
\\	日本[にほん]へ 行[い]って 日本語[にほんご]を 勉強[べんきょう]したい。しかし、 旅費[りょひ]が 高[たか]くて 行[い]けない。
\\	私が別れようと言った。すると彼女は泣き出した。	
\\	私[わたし]が 別[わか]れようと 言[い]った。すると 彼女[かのじょ]は 泣[な]き 出[だ]した。
\\	不況が続いている。その結果、失業者の数が急激に増えている。	
\\	不況[ふきょう]が 続[つづ]いている。その 結果[けっか]、 失業[しつぎょう] 者[しゃ]の 数[かず]が 急激[きゅうげき]に 増[ふ]えている。
\\	去年の冬は例年より寒かったんです。そのために、暖房費が高かったんです。	
\\	去年[きょねん]の 冬[ふゆ]は 例年[れいねん]より 寒[さむ]かったんです。そのために、 暖房[だんぼう] 費[ひ]が 高[たか]かったんです。
\\	昨日は京都へ出張していました。それで、うちにいなかったんです。	
\\	昨日[きのう]は 京都[きょうと]へ 出張[しゅっちょう]していました。それで、うちにいなかったんです。
\\	日本は資源に乏しい。それ故に、日本の経済は底が浅い。	
\\	日本[にほん]は 資源[しげん]に 乏[とぼ]しい。それ 故[ゆえ]に、 日本[にほん]の 経済[けいざい]は 底[そこ]が 浅[あさ]い。
\\	日本語を勉強すると役に立つ。だから日本語を勉強している。	
\\	日本語[にほんご]を 勉強[べんきょう]すると 役に立[やくにた]つ。だから 日本語[にほんご]を 勉強[べんきょう]している。
\\	その日本語の文章には分からない言葉がたくさんありました。ですから和英辞典を使いました。	
\\	その 日本語[にほんご]の 文章[ぶんしょう]には 分[わ]からない 言葉[ことば]がたくさんありました。ですから 和英[わえい] 辞典[じてん]を 使[つか]いました。
\\	彼はノーベル賞をもらった学者だ。けれども、とても謙虚だ。	
\\	彼[かれ]はノーベル 賞[しょう]をもらった 学者[がくしゃ]だ。けれども、とても 謙虚[けんきょ]だ。
\\	したいことはたくさんある。しかし、人生は短い。	
\\	したいことはたくさんある。しかし、 人生[じんせい]は 短[みじか]い。
\\	彼は権利ばかり主張している。そのくせ、義務を果たさない。	
\\	彼[かれ]は 権利[けんり]ばかり 主張[しゅちょう]している。そのくせ、 義務[ぎむ]を 果[は]たさない。
\\	彼女は人の前ではいいことばっかり言っている。そのくせ、陰では悪口を言っているのだ。	
\\	彼女[かのじょ]は 人[ひと]の 前[まえ]ではいいことばっかり 言[い]っている。そのくせ、 陰[かげ]では 悪口[わるぐち]を 言[い]っているのだ。
\\	彼の生活は多忙を窮めている。それでも、一日一時間運動をしている。	
\\	彼[かれ]の 生活[せいかつ]は 多忙[たぼう]を 窮[きわ]めている。それでも、一日[いちにち] 一時間[いちじかん] 運動[うんどう]をしている。
\\	彼は日本経済の専門家だそうだ。それにしては、彼の講演はお粗末だったね。	
\\	彼[かれ]は 日本[にほん] 経済[けいざい]の 専門[せんもん] 家[か]だそうだ。それにしては、 彼[かれ]の 講演[こうえん]はお 粗末[そまつ]だったね。
\\	あなたは昨日まで入院していたんでしょ。それにしてはずいぶん元気ですね。	
\\	あなたは 昨日[きのう]まで 入院[にゅういん]していたんでしょ。それにしてはずいぶん 元気[げんき]ですね。
\\	あの人はいつも遅く来るんですよ。それにしても、今日はずいぶん遅いですねえ。	
\\	あの 人[ひと]はいつも 遅[おそ]く 来[く]るんですよ。それにしても、 今日[きょう]はずいぶん 遅[おそ]いですねえ。
\\	不況の時にレイオフがあるのは仕方がない。それにしても、従業員の一割が首になるとはひどい。	
\\	不況[ふきょう]の 時[とき]にレイオフがあるのは 仕方[しかた]がない。それにしても、 従業[じゅうぎょう] 員[いん]の 一割[いちわり]が 首[くび]になるとはひどい。
\\	エイズのワクチンはまだ見つかっていない。だが、遅かれ早かれ、見つかると思う。	
\\	エイズのワクチンはまだ 見[み]つかっていない。だが、 遅[おそ]かれ 早[はや]かれ、 見[み]つかると 思[おも]う。
\\	お酒、飲んでもいいわよ。ただし、飲み過ぎないでね。	
\\	お 酒[さけ]、 飲[の]んでもいいわよ。ただし、 飲[の]み 過[す]ぎないでね。
\\	漢字は確かに面白いです。ですが、大変覚えにくいです。	
\\	漢字[かんじ]は 確[たし]かに 面白[おもしろ]いです。ですが、 大変[たいへん] 覚[おぼ]えにくいです。
\\	日本語で簡単なことは言えます。ですけれども、難しい話はだめなんです。	
\\	日本語[にほんご]で 簡単[かんたん]なことは 言[い]えます。ですけれども、 難[むずか]しい 話[はなし]はだめなんです。
\\	姉は結婚しています。でも、兄はまだ独身なんです。	
\\	姉[あね]は 結婚[けっこん]しています。でも、 兄[あに]はまだ 独身[どくしん]なんです。
\\	アメリカ人はみんなハンバーガーが好きだと思っていた。ところが、ロバートさんは嫌いだった。	
\\	アメリカ 人[じん]はみんなハンバーガーが 好[す]きだと 思[おも]っていた。ところが、ロバートさんは 嫌[きら]いだった。
\\	去年から毎週五時間日本語を習っている。とは言うものの、少しも上手にならない。	
\\	去年[きょねん]から 毎週[まいしゅう] 五時間[ごじかん] 日本語[にほんご]を 習[なら]っている。とは 言[い]うものの、 少[すこ]しも 上手[じょうず]にならない。
\\	この万年筆はとても書きやすいですよ。もっとも値段もいいですが。	
\\	この 万年筆[まんねんひつ]はとても 書[か]きやすいですよ。もっとも 値段[ねだん]もいいですが。
\\	彼の現在の結婚生活は幸福なようです。もっとも、彼は前に三回離婚していますけど。	
\\	彼[かれ]の 現在[げんざい]の 結婚[けっこん] 生活[せいかつ]は 幸福[こうふく]なようです。もっとも、 彼[かれ]は 前[まえ]に 三回[さんかい] 離婚[りこん]していますけど。
\\	今日は明日からの出張の準備で忙しかったよ。おまけに、来客まであってさ、大変だったんだ。	
\\	今日[きょう]は 明日[あした]からの 出張[しゅっちょう]の 準備[じゅんび]で 忙[いそが]しかったよ。おまけに、 来客[らいきゃく]まであってさ、 大変[たいへん]だったんだ。
\\	彼女は結婚と同時に会社を止めた。	
\\	彼女[かのじょ]は 結婚[けっこん]と 同時[どうじ]に 会社[かいしゃ]を 止[と]めた。
\\	部屋に入ると同時に電話が鳴った。	
\\	部屋[へや]に 入[はい]ると 同時[どうじ]に 電話[でんわ]が 鳴[な]った。
\\	彼女は美人なんだよな。おまけに、頭がいいときている。	
\\	彼女[かのじょ]は 美人[びじん]なんだよな。おまけに、 頭[あたま]がいいときている。
\\	僕は駅前の喫茶店に行った。そしてそこで友達を待っていた。	
\\	僕[ぼく]は 駅前[えきまえ]の 喫茶店[きっさてん]に 行[い]った。そしてそこで 友達[ともだち]を 待[ま]っていた。
\\	八時まで日本語の宿題をした。それから、映画を見に出かけた。	
\\	八時[はちじ]まで 日本語[にほんご]の 宿題[しゅくだい]をした。それから、 映画[えいが]を 見[み]に 出[で]かけた。
\\	ホストファミリーは成田空港まで迎えに来てくれた。そればかりでなく毎日いろいろ面白い所に連れて行ってくれた。	
\\	ホストファミリーは 成田空港[なりたくうこう]まで 迎[むか]えに 来[き]てくれた。そればかりでなく 毎日[まいにち]いろいろ 面白[おもしろ]い 所[ところ]に 連[つ]れて 行[い]ってくれた。
\\	彼は研究を盛んにやっている。また、人とも親しく付き合って、なかなか社交的だ。	
\\	彼[かれ]は 研究[けんきゅう]を 盛[さか]んにやっている。また、 人[ひと]とも 親[した]しく 付き合[つきあ]って、なかなか 社交[しゃこう] 的[てき]だ。
\\	この絵は彼が描いたのでしょうか。あるいはどこかから買って来たのでしょうか。	
\\	この 絵[え]は 彼[かれ]が 描[えが]いたのでしょうか。あるいはどこかから 買[か]って 来[き]たのでしょうか。
\\	ビールにしますか。それともお酒にしますか。	
\\	ビールにしますか。それともお 酒[さけ]にしますか。
\\	これで、スポーツ関係のニュースを終わります。さて次は天気予報です。	
\\	これで、スポーツ 関係[かんけい]のニュースを 終[お]わります。さて 次[つぎ]は 天気[てんき] 予報[よほう]です。
\\	このところやたらと忙しくてねえ。ゴルフをする時間もないんだよ。ゴルフをしないと、頭がすっきりしないねえ。話は変わるけどアメリカ行きどうなった?	
\\	このところやたらと 忙[いそが]しくてねえ。ゴルフをする 時間[じかん]もないんだよ。ゴルフをしないと、 頭[あたま]がすっきりしないねえ。 話[はなし]は 変[か]わるけどアメリカ 行[い]きどうなった?
\\	先週まで暖かかったのに、今週は寒いですねえ。ときに、お宅のお坊ちゃん、来年は大学受験ですね。	
\\	先週[せんしゅう]まで 暖[あたた]かかったのに、 今週[こんしゅう]は 寒[さむ]いですねえ。ときに、お 宅[たく]のお 坊[ぼっ]ちゃん、 来年[らいねん]は 大学[だいがく] 受験[じゅけん]ですね。
\\	シンデレラは片方の靴を脱ぎ落として、急いでお城を出ました。話変わってこちらシンデレラの家にはがっかりした姉達が帰ってきました。	
\\	シンデレラは 片方[かたほう]の 靴[くつ]を 脱[ぬ]ぎ 落[お]として、 急[いそ]いでお 城[しろ]を 出[で]ました。 話[はなし] 変[か]わってこちらシンデレラの 家[いえ]にはがっかりした 姉達[あねたち]が 帰[かえ]ってきました。
\\	日本人にとって戦後は実に長かった。言い換えれば、戦争は1945年8月15日に本当に終わったのではなかったのだ。	
\\	日本人[にほんじん]にとって 戦後[せんご]は 実[じつ]に 長[なが]かった。 言い換[いいか]えれば、 戦争[せんそう]は 
\\	年[ねん] 
\\	月[がつ] 
\\	日[にち]に 本当[ほんとう]に 終[お]わったのではなかったのだ。
\\	大統領は税金を大幅に上げることを主張している。そのために、消費者の購買力が低下しても構わないという考えだ。結局、国民に犠牲を払え、と言っているのである。	
\\	大統領[だいとうりょう]は 税金[ぜいきん]を 大幅[おおはば]に 上[あ]げることを 主張[しゅちょう]している。そのために、 消費[しょうひ] 者[しゃ]の 購買[こうばい] 力[りょく]が 低下[ていか]しても 構[かま]わないという 考[かんが]えだ。 結局[けっきょく]、 国民[こくみん]に 犠牲[ぎせい]を 払[はら]え、と 言[い]っているのである。
\\	私は体の害になることはしません。例えば、たばこを吸うとか、カフェインを取り過ぎるとか、夜更かしをするとかはしません。	
\\	私[わたし]は 体[からだ]の 害[がい]になることはしません。 例[たと]えば、たばこを 吸[す]うとか、カフェインを 取[と]り 過[す]ぎるとか、 夜更[よふ]かしをするとかはしません。
\\	私はこのプロジェクトはやめた方がいいと思います。なぜかと言うと同じようなプロジェクトを
\\	社でもやっているからです。	
\\	私[わたし]はこのプロジェクトはやめた 方[ほう]がいいと 思[おも]います。なぜかと 言[い]うと 同[おな]じようなプロジェクトを 
\\	社[しゃ]でもやっているからです。
\\	出張旅行の時の領収書は全部取っておかなければなりません。と言うのは、税の申告の時に必用だからです。	
\\	出張[しゅっちょう] 旅行[りょこう]の 時[とき]の 領収[りょうしゅう] 書[しょ]は 全部[ぜんぶ] 取[と]っておかなければなりません。と 言[い]うのは、 税[ぜい]の 申告[しんこく]の 時[とき]に 必用[ひつよう]だからです。
\\	私はどちらかと言うと人間嫌いだ。一方、弟は誰とでも友達のように話すタイプで、私と全然違う。	
\\	私[わたし]はどちらかと 言[い]うと 人間[にんげん] 嫌[ぎら]いだ。 一方[いっぽう]、 弟[おとうと]は 誰[だれ]とでも 友達[ともだち]のように 話[はな]すタイプで、 私[わたし]と 全然[ぜんぜん] 違[ちが]う。
\\	日本語を習う時、学生ははじめローマ字から入った方がいいと考える人がいる。それに対してはじめから、平仮名で習うべきだと考えている人もいる。	
\\	日本語[にほんご]を 習[なら]う 時[とき]、 学生[がくせい]ははじめ ローマ字[ろーまじ]から 入[は]った 方[ほう]がいいと 考[かんが]える 人[ひと]がいる。それに 対[たい]してはじめから、 平仮名[ひらがな]で 習[なら]うべきだと 考[かんが]えている 人[ひと]もいる。
\\	ジョンソンさんは日本語スピーチコンテストで三位になった。	
\\	ジョンソンさんは 日本語[にほんご]スピーチコンテストで 三位[さんい]になった。
\\	昨日は雪が十インチも積もった。	
\\	昨日[きのう]は 雪[ゆき]が 十[じゅう]インチも 積[つ]もった。
\\	新宿は高田馬場から二駅です。	
\\	新宿[しんじゅく]は 高田馬場[たかだのばば]から 二駅[ふたえき]です。
\\	1ドルは今百円ぐらいです。	
\\	1ドルは 今[いま] 百[ひゃく] 円[えん]ぐらいです。
\\	今日はうちと郵便局の間を三往復した。	
\\	今日[きょう]はうちと 郵便[ゆうびん] 局[きょく]の 間[ま]を 三往復[さんおうふく]した。
\\	日本の人口は一億以上で、アメリカの人口は二億以上だ。	
\\	日本[にほん]の 人口[じんこう]は 一億[いちおく] 以上[いじょう]で、アメリカの 人口[じんこう]は 二億[におく] 以上[いじょう]だ。
\\	「聞(く)」と言う漢字は十四画の漢字です。	
\\	聞く 
\\	聞[き](く)」と 言[い]う 漢字[かんじ]は 十四画[じゅうよんかく]の 漢字[かんじ]です。
\\	この家には兄の家族と僕の家族の二家族が住んでいる。	
\\	この 家[いえ]には 兄[あに]の 家族[かぞく]と 僕[ぼく]の 家族[かぞく]の 二家族[ふたかぞく]が 住[す]んでいる。
\\	僕はこの大学でまだ二学期しか教えていない。	
\\	僕[ぼく]はこの 大学[だいがく]でまだ 二学期[にがっき]しか 教[おし]えていない。
\\	この車は一ガロンで40マイル走る。	
\\	この 車[くるま]は 一[いち]ガロンで 40マイル 走[はし]る。
\\	アメリカの大統領で三期務めた大統領はいますか。	
\\	アメリカの 大統領[だいとうりょう]で 三期[さんき] 務[つと]めた 大統領[だいとうりょう]はいますか。
\\	空港にはジャンボジェットが十機ぐらい止まっていた。	
\\	空港[くうこう]にはジャンボジェットが 十機[じっき]ぐらい 止[と]まっていた。
\\	この車は八気筒だから、燃費が悪い。	
\\	この 車[くるま]は 八気筒[はちきとう]だから、 燃費[ねんぴ]が 悪[わる]い。
\\	先生、26ページの五行目の「これ」は何を指しているんですか。	
\\	先生[せんせい]、 26ページの 五行[ごぎょう] 目[め]の「これ」は 何[なに]を 指[さ]しているんですか。
\\	彼女はベートーベンのピアノ曲を三曲弾いた。	
\\	彼女[かのじょ]はベートーベンのピアノ 曲[きょく]を 三曲[さんきょく] 弾[ひ]いた。
\\	家から会社までは八キロあります。	
\\	家[いえ]から 会社[かいしゃ]までは 八キロ[はちきろ/はっきろ]あります。
\\	私の体重は五十六キロです。	
\\	私[わたし]の 体重[たいじゅう]は 五十六キロ[ごじゅうろっきろ]です。
\\	僕は俳句を百句ぐらい作った。	
\\	僕[ぼく]は 俳句[はいく]を 百句[ひゃっく]ぐらい 作[つく]った。
\\	私は毎日繊維質を20グラムとっている。	
\\	私[わたし]は 毎日[まいにち] 繊維[せんい] 質[しつ]を 20グラム[にじゅうぐらむ]とっている。
\\	あの学生は二桁の数字の掛け算も出来ない。	
\\	あの 学生[がくせい]は 二桁[ふたけた]の 数字[すうじ]の 掛け算[かけざん]も 出来[でき]ない。
\\	この町には本屋が八軒もある。	
\\	この 町[まち]には 本屋[ほんや]が 八軒[はちけん/はっけん]もある。
\\	あの店は間口二間の小さい店だ。	
\\	あの 店[みせ]は 間口[まぐち] 二間[にけん]の 小[ちい]さい 店[みせ]だ。
\\	この新しい住宅地には家が四十戸ぐらい建っている。	
\\	この 新[あたら]しい 住宅[じゅうたく] 地[ち]には 家[いえ]が 四十戸[よんじっこ]ぐらい 建[た]っている。
\\	答えを百語以内の日本語で書きなさい。	
\\	答[こた]えを 百[ひゃく] 語[ご] 以内[いない]の 日本語[にほんご]で 書[か]きなさい。
\\	私は大学受験の時、三校に願書を出しました。	
\\	私[わたし]は 大学[だいがく] 受験[じゅけん]の 時[とき]、三校[さんこう]に 願書[がんしょ]を 出[だ]しました。
\\	私のアパートは五号の五階です。	
\\	私[わたし]のアパートは 五号[ごごう]の 五階[ごかい]です。
\\	食堂車は4号車でございます。	
\\	食堂[しょくどう] 車[しゃ]は 4号車[よんごうしゃ]でございます。
\\	これはフルコースでございますから、十皿以上出てまいります。	
\\	これはフルコースでございますから、十皿[じっさら] 以上[いじょう] 出[で]てまいります。
\\	ここの大学の日本語の一年生は漢字を三百字ぐらい勉強します。	
\\	ここの 大学[だいがく]の 日本語[にほんご]の 一年生[いちねんせい]は 漢字[かんじ]を 三百[さんびゃく] 字[じ]ぐらい 勉強[べんきょう]します。
\\	四社が同じ製品を作っているので、競争が激しい。	
\\	四社[よんしゃ]が 同[おな]じ 製品[せいひん]を 作[つく]っているので、 競争[きょうそう]が 激[はげ]しい。
\\	僕は毎日グラウンドを十周ぐらい走っている。	
\\	僕[ぼく]は 毎日[まいにち]グラウンドを 十周[じっしゅう]ぐらい 走[はし]っている。
\\	夏の日本語の講座は九週間だ。	
\\	夏[なつ]の 日本語[にほんご]の 講座[こうざ]は 九週間[きゅうしゅうかん]だ。
\\	父は十種類ぐらい薬を飲んでいた。	
\\	父[ちち]は 十種類[じっしゅるい]ぐらい 薬[くすり]を 飲[の]んでいた。
\\	日本語には二種(類)の形容詞がある。	
\\	日本語[にほんご]には 二種[にしゅ]
\\	類[るい])の 形容詞[けいようし]がある。
\\	博士論文は全部で八章だが、まだ二章しか書いていない。	
\\	博士[はくし] 論文[ろんぶん]は 全部[ぜんぶ]で 八章[はっしょう]だが、まだ 二章[にしょう]しか 書[か]いていない。
\\	彼は酒を一升わけもなく飲んでしまう。	
\\	彼[かれ]は 酒[さけ]を 一升[いっしょう]わけもなく 飲[の]んでしまう。
\\	私は忙しいので、一日三食食べる日は滅多にない。	
\\	私[わたし]は 忙[いそが]しいので、一日[いちにち] 三食[さんしょく] 食[た]べる 日[ひ]は 滅多[めった]にない。
\\	このクレヨンは十二色です。	
\\	このクレヨンは 十二色[じゅうにしょく]です。
\\	虹は七色です。	
\\	虹[にじ]は 七色[なないろ]です。
\\	二十一世紀の日米関係はどうなるだろうか。	
\\	二十一世紀[にじゅういっせいき]の 日[にち] 米[べい] 関係[かんけい]はどうなるだろうか。
\\	日本には二世帯住宅や三世帯住宅が多い。	
\\	日本[にほん]には 二世帯[にせたい] 住宅[じゅうたく]や 三世帯[さんせたい] 住宅[じゅうたく]が 多[おお]い。
\\	わが家には三世代が同居している。	
\\	わが 家[や]には 三世代[さんせだい]が 同居[どうきょ]している。
\\	成田空港から東京まで電車で確か三千五百円ぐらいでした。	
\\	成田空港[なりたくうこう]から 東京[とうきょう]まで 電車[でんしゃ]で 確[たし]か 三千[さんぜん] 五百[ごひゃく] 円[えん]ぐらいでした。
\\	彼は一銭も無駄遣いをしない。	
\\	彼[かれ]は 一銭[いっせん]も 無駄遣[むだづか]いをしない。
\\	私の身長は1メートル75センチです。	
\\	私[わたし]の 身長[しんちょう]は 1[いち]メートル 75[ななじゅうご]センチです。
\\	このセーターは99ドル99セントでした。	
\\	このセーターは 99[きゅうじゅうきゅう]ドル 99[きゅうじゅうきゅう]セントでした。
\\	すみませんが、赤鉛筆を一ダース買って来て下さいませんか。	
\\	すみませんが、 赤[あか] 鉛筆[えんぴつ]を 一[いち]ダース 買[か]って 来[き]て 下[くだ]さいませんか。
\\	数学のテストに問題が十題出た。	
\\	数学[すうがく]のテストに 問題[もんだい]が 十題[じゅうだい] 出[で]た。
\\	この部屋にはテレビが一台、ラジオが二台、
\\	プレーヤーが一台あります。	
\\	この 部屋[へや]にはテレビが 一台[いちだい]、ラジオが 二台[にだい]、 
\\	プレーヤーが 一台[いちだい]あります。
\\	このお寺の階段は全部で三百段あります。	
\\	このお 寺[てら]の 階段[かいだん]は 全部[ぜんぶ]で 三百[さんびゃく] 段[だん]あります。
\\	このエッセイは五段落から出来ている。	
\\	このエッセイは 五段落[ごだんらく]から 出来[でき]ている。
\\	彼は背広を三十着ぐらい持っている。	
\\	彼[かれ]は 背広[せびろ]を 三十着[さんじゅっちゃく]ぐらい 持[も]っている。
\\	このお豆腐、三丁下さい。	
\\	このお 豆腐[とうふ]、三丁[さんちょう] 下[くだ]さい。
\\	男はピストルを三丁も持っていた。	
\\	男[おとこ]はピストルを 三丁[さんちょう]も 持[も]っていた。
\\	この会社の資本金は約三兆円だ。	
\\	この 会社[かいしゃ]の 資本[しほん] 金[きん]は 約[やく] 三兆[さんちょう] 円[えん]だ。
\\	銀座四丁目の辺りはにぎやかですね。	
\\	銀座[ぎんざ] 四丁目[よんちょうめ]の 辺[あた]りはにぎやかですね。
\\	韓国語を三月韓国で勉強しました。	
\\	韓国[かんこく] 語[ご]を 三月[みつき] 韓国[かんこく]で 勉強[べんきょう]しました。
\\	私の家は百坪ぐらいです。	
\\	私[わたし]の 家[いえ]は 百[ひゃく] 坪[つぼ]ぐらいです。
\\	雨が一滴、二滴と、降り始めた。	
\\	雨[あめ]が 一滴[いってき]、二滴[にてき]と、 降[ふ]り 始[はじ]めた。
\\	日本語の期末試験は九十八点だった。	
\\	日本語[にほんご]の 期末[きまつ] 試験[しけん]は 九十八点[きゅうじゅうはってん]だった。
\\	この問題の解き方は三通りある。	
\\	この 問題[もんだい]の 解き方[ときかた]は 三通[さんとお]りある。
\\	この車は2万5,000ドルぐらいでした。	
\\	この 車[くるま]は 2[に] 万[まん] 
\\	000[ごせん]ドルぐらいでした。
\\	すしを五人前お願いします。	
\\	すしを 五人前[ごにんまえ]お 願[ねが]いします。
\\	カレンはスタンフォード大学の三年生です。	
\\	カレンは スタンフォード大[すたんふぉーどだい] 学[がく]の 三年生[さんねんせい]です。
\\	日本のキリスト教徒の数は人口の1パーセント以下である。	
\\	日本[にほん]の キリスト教徒[きりすときょうと]の 数[かず]は 人口[じんこう]の 1[いち]パーセント 以下[いか]である。
\\	北海道に一泊二日の旅行をします。	
\\	北海道[ほっかいどう]に 一泊[いっぱく] 二日[ふつか]の 旅行[りょこう]をします。
\\	たばこを毎日三箱吸っています。	
\\	たばこを 毎日[まいにち] 三箱[さんぱこ] 吸[す]っています。
\\	ピストルの弾は心臓に一発当たっていた。	
\\	ピストルの 弾[たま]は 心臓[しんぞう]に 一発[いっぱつ] 当[あ]たっていた。
\\	成田エクスプレスは3番線と4番線です。	
\\	成田[なりた]エクスプレスは 3[さん] 番線[ばんせん]と 4[よん] 番線[ばんせん]です。
\\	私の東京の住所は新宿区戸塚町3丁目25番地です。	
\\	私[わたし]の 東京[とうきょう]の 住所[じゅうしょ]は 新宿[しんじゅく] 区[く] 戸塚[とつか] 町[ちょう] 3[さん] 丁目[ちょうめ] 25[にじゅうご] 番地[ばんち]です。
\\	百の発音では、三百、六百と八百に気を付けて下さい。	
\\	'ひゃく,
\\	'さんびゃく,
\\	'ろっぴゃく
\\	はっぴゃく.
\\	百[ひゃく]の 発音[はつおん]では、三百[さんびゃく]、六百[ろっぴゃく]と 八百[はっぴゃく]に 気[き]を 付[つ]けて 下[くだ]さい。
\\	僕は百メートルを12秒で走ります。	
\\	僕[ぼく]は 百[ひゃく]メートルを 12[じゅうに] 秒[びょう]で 走[はし]ります。
\\	兄は背が6フィート10インチです。	
\\	兄[あに]は 背[せ]が 6[ろく]フィート 10[じゅう]インチです。
\\	この論文の内容を十文でまとめなさい。	
\\	この 論文[ろんぶん]の 内容[ないよう]を 十文[じゅうぶん]でまとめなさい。
\\	今日はメアリーに六遍電話をかけたが、いなかった。	
\\	今日[きょう]はメアリーに 六遍[ろっぺん] 電話[でんわ]をかけたが、いなかった。
\\	千里の道も一歩から。
\\	千里[せんり]の 道[みち]も 一歩[いっぽ]から。
\\	ここの電流は100ボルトです。	
\\	ここの 電流[でんりゅう]は 100[ひゃく]ボルトです。
\\	私達の家には祖父母用に二間とってあります。	
\\	私[わたし] 達[たち]の 家[いえ]には 祖父母[そふぼ] 用[よう]に 二間[ふたま]とってあります。
\\	百万円もらったら何をしますか。	
\\	百[ひゃく] 万[まん] 円[えん]もらったら 何[なに]をしますか。
\\	この日本語のクラスには女子学生が三名、男子学生が五名います。	
\\	この 日本語[にほんご]のクラスには 女子[じょし] 学生[がくせい]が 三名[さんめい]、 男子[だんし] 学生[がくせい]が 五名[ごめい]います。
\\	富士山の高さは三千メートル以上です。	
\\	富士山[ふじさん]の 高[たか]さは 三千[さんぜん]メートル 以上[いじょう]です。
\\	この新聞の三面と四面には国際関係の記事が出る。	
\\	この 新聞[しんぶん]の 三面[さんめん]と 四面[よんめん]には 国際[こくさい] 関係[かんけい]の 記事[きじ]が 出[で]る。
\\	数学の試験で五問中一問しか解けなかった。	
\\	数学[すうがく]の 試験[しけん]で 五問[ごもん] 中[ちゅう] 一問[いちもん]しか 解[と]けなかった。
\\	アメリカのプールはたいてい25ヤードだ。	
\\	アメリカのプールはたいてい25ヤードだ。
\\	このりんごは一山500円です。	
\\	このりんごは 一山[ひとやま] 500[ごひゃく] 円[えん]です。
\\	子供の時、小学校まで毎日一里歩かなければならなかった。	
\\	子供[こども]の 時[とき]、 小学校[しょうがっこう]まで 毎日[まいにち] 一里[いちり] 歩[ある]かなければならなかった。
\\	日本のガソリンは1リットルいくらですか。	
\\	日本[にほん]のガソリンは 1[いち]リットルいくらですか。
\\	この靴は三割引きですよ。	
\\	この 靴[くつ]は 三割引[さんわりび]きですよ。
\\	200ワットの電球が欲しいんですが。	
\\	200[にひゃく]ワットの 電球[でんきゅう]が 欲[ほ]しいんですが。
\\	漫画が教育上よくないとは、一概には言えない。	
\\	漫画[まんが]が 教育[きょういく] 上[じょう]よくないとは、 一概[いちがい]には 言[い]えない。
\\	この会社には一人として尊敬できる上司がいない。	
\\	この 会社[かいしゃ]には 一人[ひとり]として 尊敬[そんけい]できる 上司[じょうし]がいない。
\\	ここにあるコンピュータは一台としてまともに動かない。	
\\	ここにあるコンピュータは 一台[いちだい]としてまともに 動[うご]かない。
\\	先生は一人もパーティーに来なかった。	
\\	先生[せんせい]は 一人[ひとり]もパーティーに 来[こ]なかった。
\\	この図書館には日本語の本は一冊もない。	
\\	この 図書館[としょかん]には 日本語[にほんご]の 本[ほん]は 一冊[いっさつ]もない。
\\	彼の言っていることは一向に要領を得ない。	
\\	彼[かれ]の 言[い]っていることは 一向[いっこう]に 要領[ようりょう]を 得[え]ない。
\\	お金は必ずしも人を幸福にはしない。	
\\	お 金[かね]は 必[かなら]ずしも 人[ひと]を 幸福[こうふく]にはしない。
\\	彼は決してそんな人じゃない。	
\\	彼[かれ]は 決[けっ]してそんな 人[ひと]じゃない。
\\	友達がたくさんいるので、少しも寂しくない。	
\\	友達[ともだち]がたくさんいるので、 少[すこ]しも 寂[さび]しくない。
\\	この辞書は全然に役に立たない。	
\\	この 辞書[じしょ]は 全然[ぜんぜん]に 役に立[やくにた]たない。
\\	これくらいの翻訳なら大して時間はかからない。	
\\	これくらいの 翻訳[ほんやく]なら 大[たい]して 時間[じかん]はかからない。
\\	吉田先生の講義はちょっとも面白くない。	
\\	吉田[よしだ] 先生[せんせい]の 講義[こうぎ]はちょっとも 面白[おもしろ]くない。
\\	この問題はちょっとやそっとでは解けない。	
\\	この 問題[もんだい]はちょっとやそっとでは 解[と]けない。
\\	この仕事はとうてい今日中には出来ない。	
\\	この 仕事[しごと]はとうてい 今日[きょう] 中[じゅう]には 出来[でき]ない。
\\	この故障はどうにも直しようがない。	
\\	この 故障[こしょう]はどうにも 直[なお]しようがない。
\\	もう二度とこんな仕事はしたくない。	
\\	もう 二度[にど]とこんな 仕事[しごと]はしたくない。
\\	まさかこんな所で君に会うとは思わなかった。	
\\	まさかこんな 所[ところ]で 君[きみ]に 会[あ]うとは 思[おも]わなかった。
\\	私はめったにうちで勉強しない。	
\\	私[わたし]はめったにうちで 勉強[べんきょう]しない。
\\	忙しくてろくに新聞も読めない。	
\\	忙[いそが]しくてろくに 新聞[しんぶん]も 読[よ]めない。
\\	ここ二、三日ろくな食事をしていない。	
\\	ここ 二[に]、 三日[さんにち]ろくな 食事[しょくじ]をしていない。
\\	新しい商売を始めようにも資金がない。	
\\	新[あたら]しい 商売[しょうばい]を 始[はじ]めようにも 資金[しきん]がない。
\\	私の友達には誰一人英語を話せる人がいない。	
\\	私[わたし]の 友達[ともだち]には 誰[だれ] 一人[ひとり] 英語[えいご]を 話[はな]せる 人[ひと]がいない。
\\	今朝から何も食べていない。	
\\	今朝[けさ]から 何[なに]も 食[た]べていない。
\\	誰も手伝ってくれない。	
\\	誰[だれ]も 手伝[てつだ]ってくれない。
\\	私は肉はあまり食べません。	
\\	私[わたし]は 肉[にく]はあまり 食[た]べません。
\\	あまり暑かったので、パジャマを着ないで寝ました。	
\\	あまり 暑[あつ]かったので、パジャマを 着[き]ないで 寝[ね]ました。
\\	彼の言っていることは、私にはさっぱり分からない。	
\\	彼[かれ]の 言[い]っていることは、 私[わたし]にはさっぱり 分[わ]からない。
\\	彼女との結婚はさっぱりあきらめた。	
\\	彼女[かのじょ]との 結婚[けっこん]はさっぱりあきらめた。
\\	この問題はなかなか解けない。	
\\	この 問題[もんだい]はなかなか 解[と]けない。
\\	このすしはなかなかおいしい。	
\\	このすしはなかなかおいしい。
\\	今日は別に予定はありません。	
\\	今日[きょう]は 別[べつ]に 予定[よてい]はありません。
\\	これは別に送って下さい。	
\\	これは 別[べつ]に 送[おく]って 下[くだ]さい。
\\	飛行機の中ではほとんど寝られませんでした。	
\\	飛行機[ひこうき]の 中[なか]ではほとんど 寝[ね]られませんでした。
\\	日本語の映画でしたがほとんど分かりました。	
\\	日本語[にほんご]の 映画[えいが]でしたがほとんど 分[わ]かりました。
\\	会議はまだ始まっていません。	
\\	会議[かいぎ]はまだ 始[はじ]まっていません。
\\	ルームメートはまだ寝ています。	
\\	ルームメートはまだ 寝[ね]ています。
\\	川村は英語がまるで出来ない。	
\\	川村[かわむら]は 英語[えいご]がまるで 出来[でき]ない。
\\	サリーはまるで日本人のように日本語を話す。	
\\	サリーはまるで 日本人[にほんじん]のように 日本語[にほんご]を 話[はな]す。
\\	もう遅刻はしません。	
\\	もう 遅刻[ちこく]はしません。
\\	朝ご飯はもう食べました。	
\\	朝[あさ]ご 飯[はん]はもう 食[た]べました。
\\	私は田中さんほど上手に話せない。	
\\	私[わたし]は 田中[たなか]さんほど 上手[じょうず]に 話[はな]せない。
\\	今井さんほどの実力があれば、どこへ行っても仕事には困らないだろう。	
\\	今井[いまい]さんほどの 実力[じつりょく]があれば、どこへ 行[い]っても 仕事[しごと]には 困[こま]らないだろう。
\\	一体誰がそんなひどいことを言ったの?	
\\	一体[いったい] 誰[だれ]がそんなひどいことを 言[い]ったの?
\\	この計画ははたしてうまくいくだろうか。	
\\	この 計画[けいかく]ははたしてうまくいくだろうか。
\\	はたして彼はうちにいなかった。	
\\	はたして 彼[かれ]はうちにいなかった。
\\	この仕事は恐らく彼には出来ないだろう。	
\\	この 仕事[しごと]は 恐[おそ]らく 彼[かれ]には 出来[でき]ないだろう。
\\	私が今日本に来ていることは、恐らく誰も知るまい。	
\\	私[わたし]が 今[いま] 日本[にほん]に 来[き]ていることは、 恐[おそ]らく 誰[だれ]も 知[し]るまい。
\\	明日はきっと晴れるでしょう。	
\\	明日[あす]はきっと 晴[は]れるでしょう。
\\	小林君はきっとそのことを知っているに違いない。	
\\	小林[こばやし] 君[くん]はきっとそのことを 知[し]っているに 違[ちが]いない。
\\	母はこのことを話したらさぞ喜ぶことだろう。	
\\	母[はは]はこのことを 話[はな]したらさぞ 喜[よろこ]ぶことだろう。
\\	森さんは多分そのことを知っているだろう。	
\\	森[もり]さんは 多分[たぶん]そのことを 知[し]っているだろう。
\\	和田さんは多分来るまい。	
\\	和田[わだ]さんは 多分[たぶん] 来[く]るまい。
\\	ひょっとすると奨学金がもらえるかも知れない。	
\\	ひょっとすると 奨学[しょうがく] 金[きん]がもらえるかも 知[し]れない。
\\	まさか私がアメリカ人と結婚するとは誰も思うまい。	
\\	まさか 私[わたし]がアメリカ 人[じん]と 結婚[けっこん]するとは 誰[だれ]も 思[おも]うまい。
\\	まさかこの秘密を人に漏らしたんじゃないだろうね。	
\\	まさかこの 秘密[ひみつ]を 人[ひと]に 漏[も]らしたんじゃないだろうね。
\\	もしかすると私の勘違いかも知れない。	
\\	もしかすると 私[わたし]の 勘違[かんちが]いかも 知[し]れない。
\\	彼の話し方はとても論理的でいかにも科学者らしい。	
\\	彼[かれ]の 話し方[はなしかた]はとても 論理[ろんり] 的[てき]でいかにも 科学[かがく] 者[しゃ]らしい。
\\	幸男はいかにも気持ちよさそうに寝ている。	
\\	幸男[ゆきお]はいかにも 気持[きも]ちよさそうに 寝[ね]ている。
\\	清水氏はいかにも感心したように私の顔を見た。	
\\	清水[しみず] 氏[し]はいかにも 感心[かんしん]したように 私[わたし]の 顔[かお]を 見[み]た。
\\	彼はさも満足そうにうなずいた。	
\\	彼[かれ]はさも 満足[まんぞく]そうにうなずいた。
\\	田中は自分がさもその分野の専門家であるかのように話した。	
\\	田中[たなか]は 自分[じぶん]がさもその 分野[ぶんや]の 専門[せんもん] 家[か]であるかのように 話[はな]した。
\\	木村さんはまるで酒を飲んだみたいだ。	
\\	木村[きむら]さんはまるで 酒[さけ]を 飲[の]んだみたいだ。
\\	彼は一度約束したら、必ずそれを守る。	
\\	彼[かれ]は 一度[いちど/ひとたび] 約束[やくそく]したら、 必[かなら]ずそれを 守[まも]る。
\\	核戦争は一度起きると、地球全体の破滅につながる。	
\\	核[かく] 戦争[せんそう]は 一度[いちど/ひとたび] 起[お]きると、 地球[ちきゅう] 全体[ぜんたい]の 破滅[はめつ]につながる。
\\	仮に今度の失敗がなかったとしても、彼はやはり解雇されただろう。	
\\	仮[かり]に 今度[こんど]の 失敗[しっぱい]がなかったとしても、 彼[かれ]はやはり 解雇[かいこ]されただろう。
\\	仮に今の二倍働いたところで、この借金は返せない。	
\\	仮[かり]に 今[いま]の 二倍[にばい] 働[はたら]いたところで、この 借金[しゃっきん]は 返[かえ]せない。
\\	たとえ両親が反対しても、私は彼と結婚します。	
\\	たとえ 両親[りょうしん]が 反対[はんたい]しても、 私[わたし]は 彼[かれ]と 結婚[けっこん]します。
\\	たとえ忠告してやったところで、彼は人の言うことなど聞きはしない。	
\\	たとえ 忠告[ちゅうこく]してやったところで、 彼[かれ]は 人[ひと]の 言[い]うことなど 聞[き]きはしない。
\\	万一雨が降ったら、ピクニックは中止します。	
\\	万一[まんいち] 雨[あめ]が 降[ふ]ったら、ピクニックは 中止[ちゅうし]します。
\\	万一大地震が起きても、この建物は大丈夫です。	
\\	万一[まんいち] 大[だい] 地震[じしん]が 起[お]きても、この 建物[たてもの]は 大丈夫[だいじょうぶ]です。
\\	万一彼が参加出来なかったところで、プロジェクトには支障はない。	
\\	万一[まんいち] 彼[かれ]が 参加[さんか] 出来[でき]なかったところで、プロジェクトには 支障[ししょう]はない。
\\	もし村上さんから電話があったら、帰ったらすぐ電話すると言っておいて下さい。	
\\	もし 村上[むらかみ]さんから 電話[でんわ]があったら、 帰[かえ]ったらすぐ 電話[でんわ]すると 言[い]っておいて 下[くだ]さい。
\\	もし来られないようなら、知らせて下さい。	
\\	もし 来[こ]られないようなら、 知[し]らせて 下[くだ]さい。
\\	せっかく三年間も日本語を勉強したのに、あまり使う機会がない。	
\\	せっかく 三年間[さんねんかん]も 日本語[にほんご]を 勉強[べんきょう]したのに、あまり 使[つか]う 機会[きかい]がない。
\\	せっかくニューヨークまで来たのだから、ついでにワシントンにも行ってみたい。	
\\	せっかくニューヨークまで 来[き]たのだから、ついでにワシントンにも 行[い]ってみたい。
\\	先生がせっかく一生懸命教えても、学生が勉強しなければ何にもならない。	
\\	先生[せんせい]がせっかく 一生懸命[いっしょうけんめい] 教[おし]えても、 学生[がくせい]が 勉強[べんきょう]しなければ 何[なん]にもならない。
\\	ぜひ日本の大学に留学したいです。	
\\	ぜひ 日本[にほん]の 大学[だいがく]に 留学[りゅうがく]したいです。
\\	アメリカにいるうちに、ぜひ一度グランドキャニオンを見ようと思っています。	
\\	アメリカにいるうちに、ぜひ 一度[いちど]グランドキャニオンを 見[み]ようと 思[おも]っています。
\\	ぜひ私達のパーティーに来て下さい。	
\\	ぜひ 私[わたし] 達[たち]のパーティーに 来[き]て 下[くだ]さい。
\\	もうちょっとで電車に乗り遅れるところだった。	
\\	もうちょっとで 電車[でんしゃ]に 乗り遅[のりおく]れるところだった。
\\	これを見ろ。	
\\	これを 見[み]ろ。
\\	図1を見よ。	
\\	図[ず] 1を 見[み]よ。
\\	漢字で書きなさい。	
\\	漢字[かんじ]で 書[か]きなさい。
\\	今すぐ私の部屋に来るように。	
\\	今[いま]すぐ 私[わたし]の 部屋[へや]に 来[く]るように。
\\	漢字で書いて下さい。	
\\	漢字[かんじ]で 書[か]いて 下[くだ]さい。
\\	お座り下さい。	
\\	お 座[すわ]り 下[くだ]さい。
\\	お立ち願います。	
\\	お 立[た]ち 願[ねが]います。
\\	日本語で書くこと。	
\\	日本語[にほんご]で 書[か]くこと。
\\	5月1日までに提出のこと。	
\\	5月[ごがつ] 1日[ついたち]までに 提出[ていしゅつ]のこと。
\\	論文は英語で発表することになっています。	
\\	論文[ろんぶん]は 英語[えいご]で 発表[はっぴょう]することになっています。
\\	まずお湯を沸騰させます。	
\\	まずお 湯[ゆ]を 沸騰[ふっとう]させます。
\\	うるさい!	
\\	うるさい!
\\	邪魔だ!	
\\	邪魔[じゃま]だ!
\\	早く!	
\\	早[はや]く!
\\	静かに!	
\\	静[しず]かに!
\\	辞書を見てはいけません。	
\\	辞書[じしょ]を 見[み]てはいけません。
\\	コストが高くてはいけない。	
\\	コストが 高[たか]くてはいけない。
\\	日常生活が不規則ではいけない。	
\\	日常[にちじょう] 生活[せいかつ]が 不規則[ふきそく]ではいけない。
\\	激しい運動はいけません。	
\\	激[はげ]しい 運動[うんどう]はいけません。
\\	この部屋を使うことはなりません。	
\\	この 部屋[へや]を 使[つか]うことはなりません。
\\	外泊はなりません。	
\\	外泊[がいはく]はなりません。
\\	夜は一人で外出しないように。	
\\	夜[よる]は 一人[ひとり]で 外出[がいしゅつ]しないように。
\\	無断で入るべからず。	
\\	無断[むだん]で 入[はい]るべからず。
\\	この付近で行商することを禁ず。	
\\	この 付近[ふきん]で 行商[ぎょうしょう]することを 禁[きん]ず。
\\	室内での飲食を禁ず。	
\\	室内[しつない]での 飲食[いんしょく]を 禁[きん]ず。
\\	芝生内立ち入り禁止。	
\\	芝生[しばふ] 内[ない] 立ち入[たちい]り 禁止[きんし]。
\\	張り紙厳禁。	
\\	張り紙[はりがみ] 厳禁[げんきん]。
\\	手かぎ無用。	
\\	手[て]かぎ 無用[むよう]。
\\	十八歳未満(の人の入場)お断り。	
\\	十八歳[じゅうはっさい] 未満[みまん](の 人[ひと]の 入場[にゅうじょう])お 断[ことわ]り。
\\	教科書を見ないで下さい。	
\\	教科書[きょうかしょ]を 見[み]ないで 下[くだ]さい。
\\	この建物内ではたばこは吸えないことになっています。	
\\	この 建物[たてもの] 内[ない]ではたばこは 吸[す]えないことになっています。
\\	日本語を教えてくれませんか。	
\\	日本語[にほんご]を 教[おし]えてくれませんか。
\\	日本語を教えて下さいませんか。	
\\	日本語[にほんご]を 教[おし]えて 下[くだ]さいませんか。
\\	ペンを貸してもらえませんか。	
\\	ペンを 貸[か]してもらえませんか。
\\	ペンを貸していただけませんか。	
\\	ペンを 貸[か]していただけませんか。
\\	邪魔をしないでくれませんか。	
\\	邪魔[じゃま]をしないでくれませんか。
\\	これ、アメリカへ送りたいんですが。	
\\	これ、アメリカへ 送[おく]りたいんですが。
\\	この報告書、今日中に書いてもらいたいんですが。	
\\	この 報告[ほうこく] 書[しょ]、 今日[きょう] 中[じゅう]に 書[か]いてもらいたいんですが。
\\	ここに車を止めないでいただきたいんですが。	
\\	ここに 車[くるま]を 止[と]めないでいただきたいんですが。
\\	今すぐ来てほしいんですが。	
\\	今[いま]すぐ 来[き]てほしいんですが。
\\	勝手にこの部屋に入らないでほしいんですが。	
\\	勝手[かって]にこの 部屋[へや]に 入[はい]らないでほしいんですが。
\\	東京駅までお願いします。	
\\	東京[とうきょう] 駅[えき]までお 願[ねが]いします。
\\	これ、航空便でお願いします。	
\\	これ、 航空[こうくう] 便[びん]でお 願[ねが]いします。
\\	窓が開かないんですが。	
\\	窓[まど]が 開[ひら]かないんですが。
\\	日本で仕事をしたいんですが、何がいい仕事はありませんか。	
\\	日本[にほん]で 仕事[しごと]をしたいんですが、 何[なに]がいい 仕事[しごと]はありませんか。
\\	専門家の意見を聞くべきだ。	
\\	専門[せんもん] 家[か]の 意見[いけん]を 聞[き]くべきだ。
\\	今すぐ結論を出すべきではない。	
\\	今[いま]すぐ 結論[けつろん]を 出[だ]すべきではない。
\\	あまり心配しないことです。	
\\	あまり 心配[しんぱい]しないことです。
\\	人の意見は聞くものだ。	
\\	人[ひと]の 意見[いけん]は 聞[き]くものだ。
\\	人の悪口を言うもんじゃない。	
\\	人[ひと]の 悪口[わるぐち]を 言[い]うもんじゃない。
\\	病院へ行った方がいいですよ。	
\\	病院[びょういん]へ 行[い]った 方[ほう]がいいですよ。
\\	先生に相談する方がいいです。	
\\	先生[せんせい]に 相談[そうだん]する 方[ほう]がいいです。
\\	このことは誰にも言わない方がいいですよ。	
\\	このことは 誰[だれ]にも 言[い]わない 方[ほう]がいいですよ。
\\	タクシーで行くといいです。	
\\	タクシーで 行[い]くといいです。
\\	アスピリンを飲んだらどうですか。	
\\	アスピリンを 飲[の]んだらどうですか。
\\	日曜日はどうですか。	
\\	日曜日[にちようび]はどうですか。
\\	ーこの辞書、ちょっと借りてもいいですか。 ーええ、どうぞ。	ー
\\	ーこの 辞書[じしょ]、ちょっと 借[か]りてもいいですか。 ーええ、どうぞ。
\\	ーこの辞書、ちょっと借りてもいいですか。 ーあ、すみません。それはちょっと。	
\\	ーこの 辞書[じしょ]、ちょっと 借[か]りてもいいですか。 ーあ、すみません。それはちょっと。
\\	これ、ちょっといいですか。	
\\	これ、ちょっといいですか。
\\	このワープロ、使わせていただけませんか。	
\\	このワープロ、 使[つか]わせていただけませんか。
\\	ちょっと電話をかけたいんですが、いいですか。	
\\	ちょっと 電話[でんわ]をかけたいんですが、いいですか。
\\	ー今晩、映画に行きませんか。 ーいいですね。行きましょう。	
\\	ー 今晩[こんばん]、 映画[えいが]に 行[い]きませんか。 ーいいですね。 行[い]きましょう。
\\	ー今晩、映画に行きませんか。 ーすみませんが、今晩はちょっと予定があるんです。	
\\	ー 今晩[こんばん]、 映画[えいが]に 行[い]きませんか。 ーすみませんが、 今晩[こんばん]はちょっと 予定[よてい]があるんです。
\\	すしを食べましょうか。	
\\	すしを 食[た]べましょうか。
\\	すしを食べましょうよ。	
\\	すしを 食[た]べましょうよ。
\\	すしを食べようか。	
\\	すしを 食[た]べようか。
\\	すしを食べよう。	
\\	すしを 食[た]べよう。
\\	今晩食事でもどうですか。	
\\	今晩[こんばん] 食事[しょくじ]でもどうですか。
\\	コーヒーはいかがですか。	
\\	コーヒーはいかがですか。
\\	このハンカチ、どうぞ。	
\\	このハンカチ、どうぞ。
\\	そのかばん、持ちましょう。	
\\	そのかばん、 持[も]ちましょう。
\\	そのかばん、持ちましょうか。	
\\	そのかばん、 持[も]ちましょうか。
\\	明日井上さんに会わなければならない。	
\\	明日[あす] 井上[いのうえ]さんに 会[あ]わなければならない。
\\	学校の成績がよくなければならない。	
\\	学校[がっこう]の 成績[せいせき]がよくなければならない。
\\	体が丈夫でなければならない。	
\\	体[からだ]が 丈夫[じょうぶ]でなければならない。
\\	応募者は日本人でなければならない。	
\\	応募[おうぼ] 者[しゃ]は 日本人[にほんじん]でなければならない。
\\	大切な会議なので休むわけにはいかない。	
\\	大切[たいせつ]な 会議[かいぎ]なので 休[やす]むわけにはいかない。
\\	これは規則だから従わざるを得ない。	
\\	これは 規則[きそく]だから 従[したが]わざるを 得[え]ない。
\\	この会議には君も出席すべきだ。	
\\	この 会議[かいぎ]には 君[きみ]も 出席[しゅっせき]すべきだ。
\\	専門家に相談する必要がある。	
\\	専門[せんもん] 家[か]に 相談[そうだん]する 必要[ひつよう]がある。
\\	この手紙は日本語で書かなくてもいい。	
\\	この 手紙[てがみ]は 日本語[にほんご]で 書[か]かなくてもいい。
\\	背が高くなくてもいい。	
\\	背[せ]が 高[たか]くなくてもいい。
\\	日本語が上手でなくてもいい。	
\\	日本語[にほんご]が 上手[じょうず]でなくてもいい。
\\	材料は金属でなくてもいい。	
\\	材料[ざいりょう]は 金属[きんぞく]でなくてもいい。
\\	この書類は保存する必要はない。	
\\	この 書類[しょるい]は 保存[ほぞん]する 必要[ひつよう]はない。
\\	構造は柔軟である必要はない。	
\\	構造[こうぞう]は 柔軟[じゅうなん]である 必要[ひつよう]はない。
\\	場所は大都市である必用はない。	
\\	場所[ばしょ]は 大都市[だいとし]である 必用[ひつよう]はない。
\\	来年日本へ行くつもりだ。	
\\	来年[らいねん] 日本[にほん]へ 行[い]くつもりだ。
\\	家を買おうと思っています。	
\\	家[いえ]を 買[か]おうと 思[おも]っています。
\\	ー誰か吉田さんに電話してくれませんか。 ー私がしましょう。	
\\	ー 誰[だれ]か 吉田[よしだ]さんに 電話[でんわ]してくれませんか。 ー 私[わたし]がしましょう。
\\	日本の大学へ行くことにしました。	
\\	日本[にほん]の 大学[だいがく]へ 行[い]くことにしました。
\\	私は天ぷらにします。	
\\	私[わたし]は 天[てん]ぷらにします。
\\	その会議にはスミスさんが行くことになった。	
\\	その 会議[かいぎ]にはスミスさんが 行[い]くことになった。
\\	次期社長は上田氏に決まった。	
\\	次期[じき] 社長[しゃちょう]は 上田[うえだ] 氏[し]に 決[き]まった。
\\	僕はメアリーに歌って欲しい。	
\\	僕[ぼく]はメアリーに 歌[うた]って 欲[ほ]しい。
\\	もっと時間があったらいいと思います。	
\\	もっと 時間[じかん]があったらいいと 思[おも]います。
\\	ブラウンさんが結婚するそうだ。	
\\	ブラウンさんが 結婚[けっこん]するそうだ。
\\	河野さんがうちを買ったらしい。	
\\	河野[こうの]さんがうちを 買[か]ったらしい。
\\	パーティーは取り止めになったとかいうことだ。	
\\	パーティーは 取り止[とりや]めになったとかいうことだ。
\\	永田さんは少し遅れるとのことです。	
\\	永田[ながた]さんは 少[すこ]し 遅[おく]れるとのことです。
\\	吉田さんは神戸に転勤になったと聞いています。	
\\	吉田[よしだ]さんは 神戸[こうべ]に 転勤[てんきん]になったと 聞[き]いています。
\\	来週東京にいらっしゃる由、楽しみにしています。	
\\	来週[らいしゅう] 東京[とうきょう]にいらっしゃる 由[よし]、 楽[たの]しみにしています。
\\	お元気の由、何よりです。	
\\	お 元気[げんき]の 由[よし]、 何[なに]よりです。
\\	テレビのニュースによると、昨日アラスカで大きな地震があったそうだ。	
\\	テレビのニュースによると、 昨日[きのう]アラスカで 大[おお]きな 地震[じしん]があったそうだ。
\\	ジョージとアリスも来るはずだ。	
\\	ジョージとアリスも 来[く]るはずだ。
\\	好子がみんなにしゃべったに違いない。	
\\	好子[よしこ]がみんなにしゃべったに 違[ちが]いない。
\\	和男はきっと来る。	
\\	和男[かずお]はきっと 来[く]る。
\\	ジョンはこのことを知るまい。	
\\	ジョンはこのことを 知[し]るまい。
\\	この日本語なら一年生の学生でも読めよう。	
\\	この 日本語[にほんご]なら 一年生[いちねんせい]の 学生[がくせい]でも 読[よ]めよう。
\\	このプロジェクトは難しかろう。	
\\	このプロジェクトは 難[むずか]しかろう。
\\	この問題は学生には難しすぎるようだ。	
\\	この 問題[もんだい]は 学生[がくせい]には 難[むずか]しすぎるようだ。
\\	前田さんは来られないかも知れない。	
\\	前田[まえだ]さんは 来[こ]られないかも 知[し]れない。
\\	この報告が間違っていることもあり得る。	
\\	この 報告[ほうこく]が 間違[まちが]っていることもあり 得[う]る。
\\	そんなことは起こり得ない。	
\\	そんなことは 起[お]こり 得[え]ない。
\\	あの男ならそれぐらいのことはやりかねない。	
\\	あの 男[おとこ]ならそれぐらいのことはやりかねない。
\\	その証人の言っていることが間違っていないとも限らない。	
\\	その 証人[しょうにん]の 言[い]っていることが 間違[まちが]っていないとも 限[かぎ]らない。
\\	彼にこの問題が解けるわけがない。	
\\	彼[かれ]にこの 問題[もんだい]が 解[と]けるわけがない。
\\	それは私には答えかねます。	
\\	それは 私[わたし]には 答[こた]えかねます。
\\	家賃が安くて助かった。	
\\	家賃[やちん]が 安[やす]くて 助[たす]かった。
\\	あの飛行機に乗らなくて助かった。	
\\	あの 飛行機[ひこうき]に 乗[の]らなくて 助[たす]かった。
\\	長い間お借りしてすみませんでした。	
\\	長[なが]い 間[あいだ]お 借[か]りしてすみませんでした。
\\	送っていただいて助かりました。	
\\	送[おく]っていただいて 助[たす]かりました。
\\	ー遅くなってすみません。 ーいいえ。	
\\	ー 遅[おそ]くなってすみません。 ーいいえ。
\\	部長はあえて社長に反対意見を述べた。	
\\	部長[ぶちょう]はあえて 社長[しゃちょう]に 反対[はんたい] 意見[いけん]を 述[の]べた。
\\	私はあえて大がかりな研究計画を立てたいと思う。	
\\	私[わたし]はあえて 大[おお]がかりな 研究[けんきゅう] 計画[けいかく]を 立[た]てたいと 思[おも]う。
\\	あえて先生の理論に異議を唱えてみようと思う。	
\\	あえて 先生[せんせい]の 理論[りろん]に 異議[いぎ]を 唱[とな]えてみようと 思[おも]う。
\\	あえて言えばあなたの考えは机上の空論だ。	
\\	あえて 言[い]えばあなたの 考[かんが]えは 机上[きじょう]の 空論[くうろん]だ。
\\	君の言うことにあえて反対はしないよ。	
\\	君[きみ]の 言[い]うことにあえて 反対[はんたい]はしないよ。
\\	アメリカでは黙っていると損をするので、私は日本人としてあえて自己主張をすることにしている。	
\\	アメリカでは 黙[だま]っていると 損[そん]をするので、 私[わたし]は 日本人[にほんじん]としてあえて 自己[じこ] 主張[しゅちょう]をすることにしている。
\\	委員会の賛同を得るのは難しいとは知りつつも、あえて新機軸の計画を提案した。	
\\	委員[いいん] 会[かい]の 賛同[さんどう]を 得[え]るのは 難[むずか]しいとは 知[し]りつつも、あえて 新[しん] 機軸[きじく]の 計画[けいかく]を 提案[ていあん]した。
\\	彼の両親があえて住居を変えたのは彼の教育を考えた上であった。	
\\	彼[かれ]の 両親[りょうしん]があえて 住居[じゅうきょ]を 変[か]えたのは 彼[かれ]の 教育[きょういく]を 考[かんが]えた 上[うえ]であった。
\\	悪いとは思いつつも、あえて先輩の非をとがめたら、やっぱりそれ以後、その先輩は私と口をかなくなった。	
\\	悪[わる]いとは 思[おも]いつつも、あえて 先輩[せんぱい]の 非[ひ]をとがめたら、やっぱりそれ 以後[いご]、その 先輩[せんぱい]は 私[わたし]と 口[くち]をかなくなった。
\\	私はあえて人とは違った道を進みたいと思っている。	
\\	私[わたし]はあえて 人[ひと]とは 違[ちが]った 道[みち]を 進[すす]みたいと 思[おも]っている。
\\	時代の流れにあえて抗おうとしたが、徒労に終わった。	
\\	時代[じだい]の 流[なが]れにあえて 抗[あらが]おうとしたが、 徒労[とろう]に 終[お]わった。
\\	あの男は能力がないけれど、あえて長所を拾えば、人のよさだろう。	
\\	あの 男[おとこ]は 能力[のうりょく]がないけれど、あえて 長所[ちょうしょ]を 拾[ひろ]えば、 人[ひと]のよさだろう。
\\	あえて日本の政治を一言で批判すると、派閥の行動ということになるだろう。	
\\	あえて 日本[にほん]の 政治[せいじ]を 一言[ひとこと]で 批判[ひはん]すると、 派閥[はばつ]の 行動[こうどう]ということになるだろう。
\\	でもいいことばかりでもなかったです。あえて今言いませんが、いやなこともたくさんありました。	
\\	でもいいことばかりでもなかったです。あえて 今[いま] 言[い]いませんが、いやなこともたくさんありました。
\\	私は赤信号だったのにあえて道を渡った。	
\\	私[わたし]は 赤信号[あかしんごう]だったのにあえて 道[みち]を 渡[わた]った。
\\	思い切ってアメリカ人と英語で話してみたら、通じたのでとても嬉しかった。	
\\	思い切[おもいき]ってアメリカ 人[じん]と 英語[えいご]で 話[はな]してみたら、 通[つう]じたのでとても 嬉[うれ]しかった。
\\	思い切って彼女に声をかけたら、にっこりと笑ってくれたんだ。	
\\	思い切[おもいき]って 彼女[かのじょ]に 声[こえ]をかけたら、にっこりと 笑[わら]ってくれたんだ。
\\	どちらの絵の方がいいかは言いにくいけれど、強いて言えば、このセザンヌの方だ。	
\\	どちらの 絵[え]の 方[ほう]がいいかは 言[い]いにくいけれど、 強[し]いて 言[い]えば、このセザンヌの 方[ほう]だ。
\\	これと言って趣味はないんですが、強いて言えば、音楽鑑賞ですね。	
\\	これと 言[い]って 趣味[しゅみ]はないんですが、 強[し]いて 言[い]えば、 音楽[おんがく] 鑑賞[かんしょう]ですね。
\\	彼はあえて危険を冒してイラクの戦場に出かけて行った。	
\\	彼[かれ]はあえて 危険[きけん]を 冒[おか]してイラクの 戦場[せんじょう]に 出[で]かけて 行[い]った。
\\	彼女はあえて両親の反対を押し切って彼と結婚した。	
\\	彼女[かのじょ]はあえて 両親[りょうしん]の 反対[はんたい]を 押し切[おしき]って 彼[かれ]と 結婚[けっこん]した。
\\	さんざン考えたあげくに大学院へ進学することにした。	
\\	さんざン 考[かんが]えたあげくに 大学院[だいがくいん]へ 進学[しんがく]することにした。
\\	彼女の離婚は数年悩んだあげくの決断だった。	
\\	彼女[かのじょ]の 離婚[りこん]は 数[すう] 年[ねん] 悩[なや]んだあげくの 決断[けつだん]だった。
\\	私はさんざん悩んだあげくに最初に勤めていた会社を辞めた。	
\\	私[わたし]はさんざん 悩[なや]んだあげくに 最初[さいしょ]に 勤[つと]めていた 会社[かいしゃ]を 辞[や]めた。
\\	彼はアジア、ヨーロッパ、北米などいろいろな国の女性とつき合ったあげくに、結局は生涯独身で過ごした。	
\\	彼[かれ]はアジア、ヨーロッパ、 北米[ほくべい]などいろいろな 国[くに]の 女性[じょせい]とつき 合[あ]ったあげくに、 結局[けっきょく]は 生涯[しょうがい] 独身[どくしん]で 過[す]ごした。
\\	妻はあの靴がいい、この靴がいいと、いろいろと履いてみたあげくに、どれも買わずに店を出た。	
\\	妻[つま]はあの 靴[くつ]がいい、この 靴[くつ]がいいと、いろいろと 履[は]いてみたあげくに、どれも 買[か]わずに 店[みせ]を 出[で]た。
\\	山田は私の車を一か月も使ったあげく、返す時お礼の一言も言わなかった。	
\\	山田[やまだ]は 私[わたし]の 車[くるま]を 一か月[いっかげつ]も 使[つか]ったあげく、 返[かえ]す 時[とき]お 礼[れい]の 一言[ひとこと]も 言[い]わなかった。
\\	さんざん迷ったあげく、一年休職してアメリカに留学することにした。	
\\	さんざん 迷[まよ]ったあげく、一年[いちねん] 休職[きゅうしょく]してアメリカに 留学[りゅうがく]することにした。
\\	彼女は既婚の男性との恋に溺れたあげくに、自分の人生を台無しにしてしまった。	
\\	彼女[かのじょ]は 既婚[きこん]の 男性[だんせい]との 恋[こい]に 溺[おぼ]れたあげくに、 自分[じぶん]の 人生[じんせい]を 台無[だいな]しにしてしまった。
\\	どのテレビを買おうかと迷ったあげく、何も買わずに帰ってしまった。	
\\	どのテレビを 買[か]おうかと 迷[まよ]ったあげく、 何[なに]も 買[か]わずに 帰[かえ]ってしまった。
\\	彼が自殺したのはよくよく悩みぬいたあげくのことだったのであろう。	
\\	彼[かれ]が 自殺[じさつ]したのはよくよく 悩[なや]みぬいたあげくのことだったのであろう。
\\	無料のネットサービスには、登録時に住所、氏名、電話番号、生年月日、あげくの果てには学歴、年収、家族構成など、様々な個人情報を要求してくれものがある。	
\\	無料[むりょう]のネットサービスには、 登録[とうろく] 時[じ]に 住所[じゅうしょ]、 氏名[しめい]、 電話[でんわ] 番号[ばんごう]、 生年月日[せいねんがっぴ]、あげくの 果[は]てには 学歴[がくれき]、 年収[ねんしゅう]、 家族[かぞく] 構成[こうせい]など、 様々[さまざま]な 個人[こじん] 情報[じょうほう]を 要求[ようきゅう]してくれものがある。
\\	彼は3時間ぐらい友達とテニスを楽しんでから、家に帰った。	
\\	彼[かれ]は 3時間[さんじかん]ぐらい 友達[ともだち]とテニスを 楽[たの]しんでから、 家[いえ]に 帰[かえ]った。
\\	一週間よく考えた後で、その仕事を引き受けることにした。	
\\	一週間[いっしゅうかん]よく 考[かんが]えた 後[あと]で、その 仕事[しごと]を 引き受[ひきう]けることにした。
\\	父は長年自分の車を自分で直し続けたが、あげくの果てには、とうとう自分で車を作ってしまった。	
\\	父[ちち]は 長年[ながねん] 自分[じぶん]の 車[くるま]を 自分[じぶん]で 直[なお]し 続[つづ]けたが、あげくのはてには、とうとう 自分[じぶん]で 車[くるま]を 作[つく]ってしまった。
\\	キャシーは京都の舞妓の生活に強い興味を待ち、あげくの果てには一年間祇園で舞妓と寝食をともにして、その生活を観察した。	
\\	キャシーは 京都[きょうと]の 舞妓[まいこ]の 生活[せいかつ]に 強[つよ]い 興味[きょうみ]を 待[ま]ち、あげくの 果[は]てには一 年間[ねんかん] 祇園[ぎおん]で 舞妓[まいこ]と 寝食[しんしょく]をともにして、その 生活[せいかつ]を 観察[かんさつ]した。
\\	よく考えた上で行動に移しなさい。	
\\	よく 考[かんが]えた 上[うえ]で 行動[こうどう]に 移[うつ]しなさい。
\\	一郎は両親とよく相談をした結果アメリカの大学に入学することにした。	
\\	一郎[いちろう]は 両親[りょうしん]とよく 相談[そうだん]をした 結果[けっか]アメリカの 大学[だいがく]に 入学[にゅうがく]することにした。
\\	電車は車を引きずって約200メートル走った末にようやく止まった。	
\\	電車[でんしゃ]は 車[くるま]を 引[ひ]きずって 約[やく]200メートル 走[はし]った 末[すえ]にようやく 止[と]まった。
\\	江本氏はあくまでも自説を変えなかった。	
\\	江本[えもと] 氏[し]はあくまでも 自説[じせつ]を 変[か]えなかった。
\\	これはあくまでも小説であって実話ではない。	
\\	これはあくまでも 小説[しょうせつ]であって 実話[じつわ]ではない。
\\	彼女はあくまでも自分に正直だった。	
\\	彼女[かのじょ]はあくまでも 自分[じぶん]に 正直[しょうじき]だった。
\\	私はあくまでも彼の無実を信じる。	
\\	私[わたし]はあくまでも 彼[かれ]の 無実[むじつ]を 信[しん]じる。
\\	彼はあくまでも日本の参戦に反対を唱え続けた。	
\\	彼[かれ]はあくまでも 日本[にほん]の 参戦[さんせん]に 反対[はんたい]を 唱[とな]え 続[つづ]けた。
\\	あくまで一人でやると言うのならあえて手助けはしない。	
\\	あくまで 一人[ひとり]でやると 言[い]うのならあえて 手助[てだす]けはしない。
\\	反乱軍はあくまでも抵抗を続けた。	
\\	反乱[はんらん] 軍[ぐん]はあくまでも 抵抗[ていこう]を 続[つづ]けた。
\\	君がそういう風にあくまで話し合いを拒否するのならこちらにも考えがある。	
\\	君[きみ]がそういう 風[ふう]にあくまで 話し合[はなしあ]いを 拒否[きょひ]するのならこちらにも 考[かんが]えがある。
\\	ケイトはあくまでも自分の外見にこだわった。	
\\	ケイトはあくまでも 自分[じぶん]の 外見[がいけん]にこだわった。
\\	理論はあくまで理論であって、現実がその通りになるという保証はどこにもない。	
\\	理論[りろん]はあくまで 理論[りろん]であって、 現実[げんじつ]がその 通[とお]りになるという 保証[ほしょう]はどこにもない。
\\	ここに書かれた費用はあくまでも目安であり、サービスによっては異なる場合があります。	
\\	ここに 書[か]かれた 費用[ひよう]はあくまでも 目安[めやす]であり、サービスによっては 異[こと]なる 場合[ばあい]があります。
\\	これはあくまでも私の直感なんですが、山田さんは近々この会社を辞めるんじゃないかと思います。	
\\	これはあくまでも 私[わたし]の 直感[ちょっかん]なんですが、 山田[やまだ]さんは 近々[ちかぢか]この 会社[かいしゃ]を 辞[や]めるんじゃないかと 思[おも]います。
\\	少しお金を貸していただきたいのですが、家族にはあくまで内緒にしていただきたいのです。	
\\	少[すこ]しお 金[かね]を 貸[か]していただきたいのですが、 家族[かぞく]にはあくまで 内緒[ないしょ]にしていただきたいのです。
\\	空はあくまでも青く澄み切っていた。	
\\	空[そら]はあくまでも 青[あお]く 澄み切[すみき]っていた。
\\	僕は明日のマラソンで絶対最後まで走り抜くつもりだ。	
\\	僕[ぼく]は 明日[あした]のマラソンで 絶対[ぜったい] 最後[さいご]まで 走り抜[はしりぬ]くつもりだ。
\\	地球上に最後まで生き残る生物は何だろう。	
\\	地球[ちきゅう] 上[じょう]に 最後[さいご]まで 生き残[いきのこ]る 生物[せいぶつ]は 何[なん]だろう。
\\	彼の推理もあながち間違っていないのかもしれない。	
\\	彼[かれ]の 推理[すいり]もあながち 間違[まちが]っていないのかもしれない。
\\	バーチャル指揮者ソフトが売れているところを見ると、一度は指揮者になってみたいというのは、あながち僕だけの願望ではないらしい。	
\\	バーチャル 指揮[しき] 者[しゃ]ソフトが 売[う]れているところを 見[み]ると、一度[いちど]は 指揮[しき] 者[しゃ]になってみたいというのは、あながち 僕[ぼく]だけの 願望[がんぼう]ではないらしい。
\\	最後の一章はやや蛇足とも思えるが、そのエピソードは読んでいて気分がいい。ということは、あながち蛇足とは言えないということか。	
\\	最後[さいご]の 一章[いっしょう]はやや 蛇足[だそく]とも 思[おも]えるが、そのエピソードは 読[よ]んでいて 気分[きぶん]がいい。ということは、あながち 蛇足[だそく]とは 言[い]えないということか。
\\	皮膚の
\\	が体調によっても変わるとすれば、「気分次第で色が変わる口紅」というのもあながち嘘ではない。	
\\	皮膚[ひふ]の 
\\	が 体調[たいちょう]によっても 変[か]わるとすれば、
\\	気分[きぶん] 次第[しだい]で 色[いろ]が 変[か]わる 口紅[くちべに]」というのもあながち 嘘[うそ]ではない。
\\	完全に男女平等な社会が来るのは困るというのも、あながち偽った気持ちではない。	
\\	完全[かんぜん]に 男女[だんじょ] 平等[びょうどう]な 社会[しゃかい]が 来[く]るのは 困[こま]るというのも、あながち 偽[いつわ]った 気持[きも]ちではない。
\\	この俳優の演技はうまいが、時々違和感を覚えるのは、あながち演出のせいとばかりは言えない。	
\\	この 俳優[はいゆう]の 演技[えんぎ]はうまいが、 時々[ときどき] 違和感[いわかん]を 覚[おぼ]えるのは、あながち 演出[えんしゅつ]のせいとばかりは 言[い]えない。
\\	生死の境をさまよっていた人が手術中の会話を鮮明に覚えていたりすることから、幽体離脱現象もあながち否定できない。	
\\	生死[せいし]の 境[さかい]をさまよっていた 人[ひと]が 手術[しゅじゅつ] 中[ちゅう]の 会話[かいわ]を 鮮明[せんめい]に 覚[おぼ]えていたりすることから、 幽[ゆう] 体[たい] 離脱[りだつ] 現象[げんしょう]もあながち 否定[ひてい]できない。
\\	光るもの、必ずしも金ではない。	
\\	光[ひか]るもの、 必[かなら]ずしも 金[きん]ではない。
\\	金持ちは必ずしも傲慢ではない。	
\\	金持[かねも]ちは 必[かなら]ずしも 傲慢[ごうまん]ではない。
\\	日本人は必ずしも丁寧ではない。	
\\	日本人[にほんじん]は 必[かなら]ずしも 丁寧[ていねい]ではない。
\\	彼の話は必ずしも真実ではない。	
\\	彼[かれ]の 話[はなし]は 必[かなら]ずしも 真実[しんじつ]ではない。
\\	あの男は傲慢だと言われているけど、僕の観察では時々非常に優しい面を見せる。あながち傲慢ではないのだ。	
\\	あの 男[おとこ]は 傲慢[ごうまん]だと 言[い]われているけど、 僕[ぼく]の 観察[かんさつ]では 時々[ときどき] 非常[ひじょう]に 優[やさ]しい 面[めん]を 見[み]せる。あながち 傲慢[ごうまん]ではないのだ。
\\	は全く根拠のないことだというのが常識だが、
\\	の証拠写真を見ると、必ず真実ではないとは言い切れないようだ。	
\\	[ユーフォー]は 全[まった]く 根拠[こんきょ]のないことだというのが 常識[じょうしき]だが、 
\\	[ユーフォー]の 証拠[しょうこ] 写真[しゃしん]を 見[み]ると、 必[かなら]ず 真実[しんじつ]ではないとは 言[い]い 切[き]れないようだ。
\\	ボールぺンあるいは万年筆でお書き下さい。	
\\	ボールぺンあるいは 万年筆[まんねんひつ]でお 書[か]き 下[くだ]さい。
\\	日本のような少子化社会では、大学の数を少なくするか、あるいは外国人の学生を増やすかしなければなるまい。	
\\	日本[にほん]のような 少子化[しょうしか] 社会[しゃかい]では、 大学[だいがく]の 数[かず]を 少[すく]なくするか、あるいは 外国[がいこく] 人[じん]の 学生[がくせい]を 増[ふ]やすかしなければなるまい。
\\	性格が明るいか、あるいは暗いかによって、人生は非常に変わってくる。	
\\	性格[せいかく]が 明[あか]るいか、あるいは 暗[くら]いかによって、 人生[じんせい]は 非常[ひじょう]に 変[か]わってくる。
\\	すべての宗教は、あるいはアニミズムから来ているのかもしれない。	
\\	すべての 宗教[しゅうきょう]は、あるいはアニミズムから 来[き]ているのかもしれない。
\\	社長は朝ご飯にはトーストかあるいはオートミールを召し上がる。	
\\	社長[しゃちょう]は 朝[あさ]ご 飯[はん]にはトーストかあるいはオートミールを 召し上[めしあ]がる。
\\	政治学あるいは経済学を専攻した学生を採用します。	
\\	政治[せいじ] 学[がく]あるいは 経済[けいざい] 学[がく]を 専攻[せんこう]した 学生[がくせい]を 採用[さいよう]します。
\\	ご注文は電話か、あるいはインターネットでお願いします。	
\\	ご 注文[ちゅうもん]は 電話[でんわ]か、あるいはインターネットでお 願[ねが]いします。
\\	この会社は大規模なリストラをするか、あるいは、倒産するかしかない。	
\\	この 会社[かいしゃ]は 大[だい] 規模[きぼ]なリストラをするか、あるいは、 倒産[とうさん]するかしかない。
\\	定年になったら毎日川で釣りをするか、あるいはゴルフをしようと思っています。	
\\	定年[ていねん]になったら 毎日[まいにち] 川[かわ]で 釣[つ]りをするか、あるいはゴルフをしようと 思[おも]っています。
\\	人々の大統領の評価は好きか、あるいは、嫌いかのどちらかだ。	
\\	人々[ひとびと]の 大統領[だいとうりょう]の 評価[ひょうか]は 好[す]きか、あるいは、 嫌[きら]いかのどちらかだ。
\\	現在日本の政党は乱立しているが、将来あるいはアメリカのように二大政党になるかもしれない。	
\\	現在[げんざい] 日本[にほん]の 政党[せいとう]は 乱立[らんりつ]しているが、 将来[しょうらい]あるいはアメリカのように 二大[にだい] 政党[せいとう]になるかもしれない。
\\	冬休みはここにいますか。あるいは、旅行かなんかしますか。	
\\	冬休[ふゆやす]みはここにいますか。あるいは、 旅行[りょこう]かなんかしますか。
\\	彼はあたかも責任者のように振る舞っている。	
\\	彼[かれ]はあたかも 責任[せきにん] 者[しゃ]のように 振る舞[ふるま]っている。
\\	リニアモーターカーはあたかも氷の上をすべるかのように走る。	
\\	リニアモーターカーはあたかも 氷[こおり]の 上[うえ]をすべるかのように 走[はし]る。
\\	この装置を使って、あたかも現場にいるような感覚で現場にいるロボットを操縦できる。	
\\	この 装置[そうち]を 使[つか]って、あたかも 現場[げんば]にいるような 感覚[かんかく]で 現場[げんば]にいるロボットを 操縦[そうじゅう]できる。
\\	この磁石はあたかも永久磁石のように動作する。	
\\	この 磁石[じしゃく]はあたかも 永久[えいきゅう] 磁石[じしゃく]のように 動作[どうさ]する。
\\	リモートコントロールソフトは遠くの
\\	のデスクトップ画面をそのまま手元の
\\	に映し出し、あたかも自分のデスクトップのように操作することができる。	
\\	リモートコントロールソフトは 遠[とお]くの 
\\	のデスクトップ 画面[がめん]をそのまま 手元[てもと]の 
\\	に 映し出[うつしだ]し、あたかも 自分[じぶん]のデスクトップのように 操作[そうさ]することができる。
\\	彼女はあたかも自分のうちにいるように振る舞っている。	
\\	彼女[かのじょ]はあたかも 自分[じぶん]のうちにいるように 振る舞[ふるま]っている。
\\	上手な説明文は、あたかも目の前にその光景が広がっているように鮮明にイメージできるものです。	
\\	上手[じょうず]な 説明[せつめい] 文[ぶん]は、あたかも 目[め]の 前[まえ]にその 光景[こうけい]が 広[ひろ]がっているように 鮮明[せんめい]にイメージできるものです。
\\	彼はあたかも自分が上司であるかのように私に仕事を言いつける。	
\\	彼[かれ]はあたかも 自分[じぶん]が 上司[じょうし]であるかのように 私[わたし]に 仕事[しごと]を 言[い]いつける。
\\	連結納税制度とは、企業グループがあたかも一つの法人であるかのように捉えて、納税させる仕組みです。	
\\	連結[れんけつ] 納税[のうぜい] 制度[せいど]とは、 企業[きぎょう]グループがあたかも 一[ひと]つの 法人[ほうじん]であるかのように 捉[とら]えて、 納税[のうぜい]させる 仕組[しく]みです。
\\	この映画はコンピュータ・グラフィックスによって、あたかも人間が実際に人の体の中に入って活動しているような効果を生み出している。	
\\	この 映画[えいが]はコンピュータ・グラフィックスによって、あたかも 人間[にんげん]が 実際[じっさい]に 人[ひと]の 体[からだ]の 中[なか]に 入[はい]って 活動[かつどう]しているような 効果[こうか]を 生み出[うみだ]している。
\\	私はあたかも空を飛んでいるかのような錯覚を覚えた。	
\\	私[わたし]はあたかも 空[そら]を 飛[と]んでいるかのような 錯覚[さっかく]を 覚[おぼ]えた。
\\	この文章はあたかも我々が不正を行っているかのごとき印象を与える。	
\\	この 文章[ぶんしょう]はあたかも 我々[われわれ]が 不正[ふせい]を 行[い]っているかのごとき 印象[いんしょう]を 与[あた]える。
\\	この夜空の色はあたかも火事を思わせる。	
\\	この 夜空[よぞら]の 色[いろ]はあたかも 火事[かじ]を 思[おも]わせる。
\\	製品の特定部分の品質、性能等が著しく強調され、あたかも全体が優良であると誤解させるような表示をしてはならない。	
\\	製品[せいひん]の 特定[とくてい] 部分[ぶぶん]の 品質[ひんしつ]、 性能[せいのう] 等[など]が 著[いちじる]しく 強調[きょうちょう]され、あたかも 全体[ぜんたい]が 優良[ゆうりょう]であると 誤解[ごかい]させるような 表示[ひょうじ]をしてはならない。
\\	時はあたかも就職シーズンだ。	
\\	時[とき]はあたかも 就職[しゅうしょく]シーズンだ。
\\	私が日本に来たのは、時あたかも桜が満開の時だった。	
\\	私[わたし]が 日本[にほん]に 来[き]たのは、 時[とき]あたかも 桜[さくら]が 満開[まんかい]の 時[とき]だった。
\\	僕が一言変なことをいったばかりに彼女との関係は悪くなってしまった。	
\\	僕[ぼく]が 一言[ひとこと] 変[へん]なことをいったばかりに 彼女[かのじょ]との 関係[かんけい]は 悪[わる]くなってしまった。
\\	あの子は、体が弱いばかりにクラスメートにいじめられている。	
\\	あの 子[こ]は、 体[からだ]が 弱[よわ]いばかりにクラスメートにいじめられている。
\\	あの日たまたま休んでいたばかりに、大事な連絡を聞きそびれた。	
\\	あの 日[ひ]たまたま 休[やす]んでいたばかりに、 大事[だいじ]な 連絡[れんらく]を 聞[き]きそびれた。
\\	彼女が会いたいと言ってきたときに会いに行かなかったばかりに、それが最後のチャンスになってしまった。	
\\	彼女[かのじょ]が 会[あ]いたいと 言[い]ってきたときに 会[あ]いに 行[い]かなかったばかりに、それが 最後[さいご]のチャンスになってしまった。
\\	遺言がないばかりに相続人の間で遺産争いに発展するということがよくある。	
\\	遺言[ゆいごん]がないばかりに 相続[そうぞく] 人[にん]の 間[あいだ]で 遺産[いさん] 争[あらそ]いに 発展[はってん]するということがよくある。
\\	お金がなかったばかりに、映画もろくに見られなかった。	
\\	お 金[かね]がなかったばかりに、 映画[えいが]もろくに 見[み]られなかった。
\\	医者の処置が不適切であったばかりに、助かる患者を死なせてしまった。	
\\	医者[いしゃ]の 処置[しょち]が 不適切[ふてきせつ]であったばかりに、 助[たす]かる 患者[かんじゃ]を 死[し]なせてしまった。
\\	彼女は黒人女性であったばかりに人種差別と性差別を受けた。	
\\	彼女[かのじょ]は 黒人[こくじん] 女性[じょせい]であったばかりに 人種[じんしゅ] 差別[さべつ]と 性[せい] 差別[さべつ]を 受[う]けた。
\\	姑との関係が悪いばかりに嫁はストレスが溜まり、よく病気になった。	
\\	姑[しゅうと]との 関係[かんけい]が 悪[わる]いばかりに 嫁[よめ]はストレスが 溜[た]まり、よく 病気[びょうき]になった。
\\	両親が甘かったばかりに子供たちは自立心がない。	
\\	両親[りょうしん]が 甘[あま]かったばかりに 子供[こども]たちは 自立[じりつ] 心[しん]がない。
\\	高校時代の教育がよくなかったばかりに、大学に入ってから苦労している。	
\\	高校[こうこう] 時代[じだい]の 教育[きょういく]がよくなかったばかりに、 大学[だいがく]に 入[はい]ってから 苦労[くろう]している。
\\	あいつは目立ちたいばかりに似合わない派手な背広を着ている。	
\\	あいつは 目立[めだ]ちたいばかりに 似合[にあ]わない 派手[はで]な 背広[せびろ]を 着[き]ている。
\\	ヤンキースの松井秀樹を見たいばかりに、わざわざ日本からニューヨークにやってきました。	
\\	ヤンキースの 松井[まつい] 秀樹[ひでき]を 見[み]たいばかりに、わざわざ 日本[にほん]からニューヨークにやってきました。
\\	彼は今がチャンスとばかりに彼女に言い寄った。	
\\	彼[かれ]は 今[いま]がチャンスとばかりに 彼女[かのじょ]に 言い寄[いいよ]った。
\\	環境保護を支持していればこそ植林に力を入れているのだ。	
\\	環境[かんきょう] 保護[ほご]を 支持[しじ]していればこそ 植林[しょくりん]に 力[ちから]を 入[い]れているのだ。
\\	「愛」があればこそ生きる喜びがある。	
\\	愛[あい]」があればこそ 生[い]きる 喜[よろこ]びがある。
\\	膨大な電力の需要があればこそ原子力発電を開発せざるを得ないのだ。	
\\	膨大[ぼうだい]な 電力[でんりょく]の 需要[じゅよう]があればこそ 原子力[げんしりょく] 発電[はつでん]を 開発[かいはつ]せざるを 得[え]ないのだ。
\\	中近東の和平を熱望すればこそ、国連の役割を強化しなければならないと思う。	
\\	中近東[ちゅうきんとう]の 和平[わへい]を 熱望[ねつぼう]すればこそ、 国連[こくれん]の 役割[やくわり]を 強化[きょうか]しなければならないと 思[おも]う。
\\	親友であればこそ君にこんなにずけずけとものが言えるんだよ。	
\\	親友[しんゆう]であればこそ 君[きみ]にこんなにずけずけとものが 言[い]えるんだよ。
\\	ビジョンを持った社長であればこそ社員は張り切って仕事をしているのだ。	
\\	ビジョンを 持[も]った 社長[しゃちょう]であればこそ 社員[しゃいん]は 張り切[はりき]って 仕事[しごと]をしているのだ。
\\	人は一人でいるのが寂しければこそペットを飼うのだ。	
\\	人[ひと]は 一人[ひとり]でいるのが 寂[さび]しければこそペットを 飼[か]うのだ。
\\	言論が自由であればこそ民主主義は育っていくのである。	
\\	言論[げんろん]が 自由[じゆう]であればこそ 民主[みんしゅ] 主義[しゅぎ]は 育[そだ]っていくのである。
\\	僕がこんなことを言うのも君を心配すればこそだ。	
\\	僕[ぼく]がこんなことを 言[い]うのも 君[きみ]を 心配[しんぱい]すればこそだ。
\\	英語が上手に話せないからこそ人より余計に練習しなければならない。	
\\	英語[えいご]が 上手[じょうず]に 話[はな]せないからこそ 人[ひと]より 余計[よけい]に 練習[れんしゅう]しなければならない。
\\	他の授業がそれほど面白くないからこそこの授業は面白く感じられる。	
\\	他[ほか]の 授業[じゅぎょう]がそれほど 面白[おもしろ]くないからこそこの 授業[じゅぎょう]は 面白[おもしろ]く 感[かん]じられる。
\\	芝生に入るべからず。	
\\	芝生[しばふ]に 入[はい]るべからず。
\\	原子力はエネルギー源として欠くべからざるものとなっている。	
\\	原子力[げんしりょく]はエネルギー 源[げん]として 欠[か]くべからざるものとなっている。
\\	電車の中で携帯電話を使うべからず。	
\\	電車[でんしゃ]の 中[なか]で 携帯[けいたい] 電話[でんわ]を 使[つか]うべからず。
\\	ここで犬に糞をさせるべからず。	
\\	ここで 犬[いぬ]に 糞[ふん]をさせるべからず。
\\	授業中に隣の者と私語をするべからず。	
\\	授業[じゅぎょう] 中[ちゅう]に 隣[となり]の 者[もの]と 私語[しご]をするべからず。
\\	無用の者、入るべからず。	
\\	無用[むよう]の 者[もの]、 入[はい]るべからず。
\\	虚言を吐くべからず。	
\\	虚言[きょげん]を 吐[は]くべからず。
\\	インターネットを使う時の「べからず集」からいくつか引用してみよう。プライバシーを公開するべからず。怪しげなファイルに触れるべからず。インターネットの向こうに人間がいることを忘れるべからず。	
\\	インターネットを 使[つか]う 時[とき]の「べからず 集[しゅう]」からいくつか 引用[いんよう]してみよう。プライバシーを 公開[こうかい]するべからず。 怪[あや]しげなファイルに 触[ふ]れるべからず。インターネットの 向[む]こうに 人間[にんげん]がいることを 忘[わす]れるべからず。
\\	彼は学生として許すべからざる行為を行ったとして大学を退学させられた。	
\\	彼[かれ]は 学生[がくせい]として 許[ゆる]すべからざる 行為[こうい]を 行[おこな]ったとして 大学[だいがく]を 退学[たいがく]させられた。
\\	この世の中には批判すべからざる学説はほとんどないと言ってよいだろう。	
\\	この 世の中[よのなか]には 批判[ひはん]すべからざる 学説[がくせつ]はほとんどないと 言[い]ってよいだろう。
\\	奇跡とは一言にしていえば、起こり得べからざる事実だ起こる場合のことである。	
\\	奇跡[きせき]とは 一言[ひとこと]にしていえば、 起[お]こり 得[う]べからざる 事実[じじつ]だ 起[お]こる 場合[ばあい]のことである。
\\	それは総理大臣としてすべからざる発言だった。	
\\	それは 総理[そうり] 大臣[だいじん]としてすべからざる 発言[はつげん]だった。
\\	児童虐待などは許すべきではない行為だ。	
\\	児童[じどう] 虐待[ぎゃくたい]などは 許[ゆる]すべきではない 行為[こうい]だ。
\\	この大学では、2学期制に変えるべく今準備が進められている。	
\\	この 大学[だいがく]では、 2学期[にがっき] 制[せい]に 変[か]えるべく 今[いま] 準備[じゅんび]が 進[すす]められている。
\\	社は人件費を削減すべく大規模なリストラを行うよていだ。	
\\	社[しゃ]は 人件[じんけん] 費[ひ]を 削減[さくげん]すべく 大[だい] 規模[きぼ]なリストラを 行[おこな]うよていだ。
\\	当社はいち早くお客さまの多様なニーズに対応すべく、最新の設備を導入してまいりました。	
\\	当社[とうしゃ]はいち 早[はや]くお 客[きゃく]さまの 多様[たよう]なニーズに 対応[たいおう]すべく、 最新[さいしん]の 設備[せつび]を 導入[どうにゅう]してまいりました。
\\	二児の母親ですが、現在フットセラピストとして独立すべく勉強中です。	
\\	二児[にじ]の 母親[ははおや]ですが、 現在[げんざい]フットセラピストとして 独立[どくりつ]すべく 勉強[べんきょう] 中[ちゅう]です。
\\	本学は、経済的な理由で大学教育を受けられない海外の優秀な学生にその機会を与えるべく、奨学金を供与している。	
\\	本学[ほんがく]は、 経済[けいざい] 的[てき]な 理由[りゆう]で 大学[だいがく] 教育[きょういく]を 受[う]けられない 海外[かいがい]の 優秀[ゆうしゅう]な 学生[がくせい]にその 機会[きかい]を 与[あた]えるべく、 奨学[しょうがく] 金[きん]を 供与[きょうよ]している。
\\	重大な事故が発生した場合、お客様の安全確保を最優先として迅速な対応を行うべく、社内システムの見直しを行きました。	
\\	重大[じゅうだい]な 事故[じこ]が 発生[はっせい]した 場合[ばあい]、お 客様[きゃくさま]の 安全[あんぜん] 確保[かくほ]を 最[さい] 優先[ゆうせん]として 迅速[じんそく]な 対応[たいおう]を 行[おこな]うべく、 社内[しゃない]システムの 見直[みなお]しを 行[い]きました。
\\	より進んだ経営革新を行うためには、マネジメントシステム運用時の無駄を省くべく、システム間の重複をなくすことが重要である。	
\\	より 進[すす]んだ 経営[けいえい] 革新[かくしん]を 行[おこな]うためには、マネジメントシステム 運用[うんよう] 時[じ]の 無駄[むだ]を 省[はぶ]くべく、システム 間[かん]の 重複[ちょうふく]をなくすことが 重要[じゅうよう]である。
\\	私が一億円の家を買うなど、望むべくもない。	
\\	私[わたし]が 一億[いちおく] 円[えん]の 家[いえ]を 買[か]うなど、 望[のぞ]むべくもない。
\\	10年前に別れた女性の居どころなど知るべくもない。	
\\	10年[じゅうねん] 前[まえ]に 別[わか]れた 女性[じょせい]の 居[い]どころなど 知[し]るべくもない。
\\	終戦後は海外旅行など望むべくもなかった。	
\\	終戦[しゅうせん] 後[ご]は 海外[かいがい] 旅行[りょこう]など 望[のぞ]むべくもなかった。
\\	初日は雨で山頂は望むべくもなかったが、翌日はすっかり晴れ、白銀に輝く山は神々しい美しさだった。	
\\	初日[しょにち]は 雨[あめ]で 山頂[さんちょう]は 望[のぞ]むべくもなかったが、 翌日[よくじつ]はすっかり 晴[は]れ、 白銀[はくぎん]に 輝[かがや]く 山[やま]は 神々[こうごう]しい 美[うつく]しさだった。
\\	我々のような弱いチームが優勝候補の
\\	チームに勝つべくもない。	
\\	我々[われわれ]のような 弱[よわ]いチームが 優勝[ゆうしょう] 候補[こうほ]の 
\\	チームに 勝[か]つべくもない。
\\	その事件が起きた時私は海外旅行中で知れべくもなかった。	
\\	その 事件[じけん]が 起[お]きた 時[とき] 私[わたし]は 海外[かいがい] 旅行[りょこう] 中[なか]で 知[し]れべくもなかった。
\\	最近は多忙で本を読む時間など期待すべくもない。	
\\	最近[さいきん]は 多忙[たぼう]で 本[ほん]を 読[よ]む 時間[じかん]など 期待[きたい]すべくもない。
\\	こんな難しい日本語が読めるわけがない。	
\\	こんな 難[むずか]しい 日本語[にほんご]が 読[よ]めるわけがない。
\\	僕に詩が書けるはずがない。	
\\	僕[ぼく]に 詩[し]が 書[か]けるはずがない。
\\	第二次世界大戦は起こるべくして起こった。	
\\	第[だい] 二次[にじ] 世界[せかい] 大戦[たいせん]は 起[お]こるべくして 起[お]こった。
\\	彼は大統領になるべくしてなった。	
\\	彼[かれ]は 大統領[だいとうりょう]になるべくしてなった。
\\	真知子は高校の時から成績が抜群によく、東大に入るべくして入ったという感じである。	
\\	真知子[まちこ]は 高校[こうこう]の 時[とき]から 成績[せいせき]が 抜群[ばつぐん]によく、 東大[とうだい]に 入[はい]るべくして 入[はい]ったという 感[かん]じである。
\\	二人は性格、趣味、学歴がすべて一致している。二人は結婚するべくして結婚したのだ。	
\\	二人[ふたり]は 性格[せいかく]、 趣味[しゅみ]、 学歴[がくれき]がすべて 一致[いっち]している。二人[ふたり]は 結婚[けっこん]するべくして 結婚[けっこん]したのだ。
\\	ゆみの運転はいつも荒っぽいし、エンジンもブレーキも調子が悪かった。事故は起こるべくして起こった。	
\\	ゆみの 運転[うんてん]はいつも 荒[あら]っぽいし、エンジンもブレーキも 調子[ちょうし]が 悪[わる]かった。 事故[じこ]は 起[お]こるべくして 起[お]こった。
\\	あの会社は経営がずさんだったから倒産するべくして倒産したのだ。	
\\	あの 会社[かいしゃ]は 経営[けいえい]がずさんだったから 倒産[とうさん]するべくして 倒産[とうさん]したのだ。
\\	私は自宅が海のすぐ近くのため、釣り好きになるべくしてなったと言っていいでしょう。	
\\	私[わたし]は 自宅[じたく]が 海[うみ]のすぐ 近[ちか]くのため、 釣[つ]り 好[ず]きになるべくしてなったと 言[い]っていいでしょう。
\\	日本のおすしはやはりおいしい。	
\\	日本[にほん]のおすしはやはりおいしい。
\\	外見は日本人に見えるロイ・ヤマダさんもやっぱりアメリカ人だ。	
\\	外見[がいけん]は 日本人[にほんじん]に 見[み]えるロイ・ヤマダさんもやっぱりアメリカ 人[じん]だ。
\\	ベイリーさんはやはり来なかった。	
\\	ベイリーさんはやはり 来[こ]なかった。
\\	彼は嫌われるべくして嫌われている。	
\\	彼[かれ]は 嫌[きら]われるべくして 嫌[きら]われている。
\\	実社会は、基本的に自分以外はすべて敵だから、勝つやつは勝つべくして勝つし、負けるやつは負けるべくして負ける。	
\\	実社会[じっしゃかい]は、 基本[きほん] 的[てき]に 自分[じぶん] 以外[いがい]はすべて 敵[てき]だから、 勝[か]つやつは 勝[か]つべくして 勝[か]つし、 負[ま]けるやつは 負[ま]けるべくして 負[ま]ける。
\\	社会通念は変わるべくして変わるのだから、地域社会の通念を行政的に改革しようとしても無理だ。	
\\	社会[しゃかい] 通念[つうねん]は 変[か]わるべくして 変[か]わるのだから、 地域[ちいき] 社会[しゃかい]の 通念[つうねん]を 行政[ぎょうせい] 的[てき]に 改革[かいかく]しようとしても 無理[むり]だ。
\\	彼の小説はポストモダン的で大変気に入っている。解釈が読み手にゆだねられているところがよい。ちなみに、去年の作品はベストセラー第1位だった。	
\\	彼[かれ]の 小説[しょうせつ]はポストモダン 的[てき]で 大変[たいへん] 気に入[きにい]っている。 解釈[かいしゃく]が 読み手[よみて]にゆだねられているところがよい。ちなみに、 去年[きょねん]の 作品[さくひん]はベストセラー 第[だい] 1位[いちい]だった。
\\	先週から上野の美術館で印象派の絵画展が開かれており、ルノアール、モネ、マネなどの代表作が出品されている。ちなみに、入場料は大人3,500円、子供1,000円である。	
\\	先週[せんしゅう]から 上野[うえの]の 美術館[びじゅつかん]で 印象派[いんしょうは]の 絵画[かいが] 展[てん]が 開[ひら]かれており、ルノアール、モネ、マネなどの 代表[だいひょう] 作[さく]が 出品[しゅっぴん]されている。ちなみに、 入場[にゅうじょう] 料[りょう]は 大人[おとな]3,500 円[えん]、 子供[こども]1,000 円[えん]である。
\\	先月の26日と27日に東京で国際建築学会が開かれた。建築の人間化というのがテーマだった。ちなみに参加者数は約1,500人だった。	
\\	先月[せんげつ]の 
\\	日[にち]と 
\\	日[にち]に 東京[とうきょう]で 国際[こくさい] 建築[けんちく] 学会[がっかい]が 開[ひら]かれた。 建築[けんちく]の 人間[にんげん] 化[か]というのがテーマだった。ちなみに 参加[さんか] 者[しゃ] 数[すう]は 約[やく] 
\\	人[にん]だった。
\\	アメリカの大学教授は7年に一度、研究休暇をとることができる。その期間中に研究を集中的にやるわけである。ちなみにその間の給料は、休暇が1年間なら半額、半年ならば全額が支給される。	
\\	アメリカの 大学[だいがく] 教授[きょうじゅ]は 
\\	年[ねん]に 一度[いちど]、 研究[けんきゅう] 休暇[きゅうか]をとることができる。その 期間[きかん] 中[ちゅう]に 研究[けんきゅう]を 集中[しゅうちゅう] 的[てき]にやるわけである。ちなみにその 間[あいだ]の 給料[きゅうりょう]は、 休暇[きゅうか]が 
\\	年間[ねんかん]なら 半額[はんがく]、 半年[はんとし]ならば 全額[ぜんがく]が 支給[しきゅう]される。
\\	若いうちに本を読めるだけ読んでおいた方がよい。	
\\	若[わか]いうちに 本[ほん]を 読[よ]めるだけ 読[よ]んでおいた 方[ほう]がよい。
\\	食べるだけ食べると、礼も言わずに帰ってしまった。	
\\	食[た]べるだけ 食[た]べると、 礼[れい]も 言[い]わずに 帰[かえ]ってしまった。
\\	食べたいだけ食べるのが僕の主義です。	
\\	食[た]べたいだけ 食[た]べるのが 僕[ぼく]の 主義[しゅぎ]です。
\\	うちの子は背が伸びるだけ伸びた感じです。	
\\	うちの 子[こ]は 背[せ]が 伸[の]びるだけ 伸[の]びた 感[かん]じです。
\\	修士論文の内容についてできるだけ詳しく話して下さい。	
\\	修士[しゅうし] 論文[ろんぶん]の 内容[ないよう]についてできるだけ 詳[くわ]しく 話[はな]して 下[くだ]さい。
\\	きのうはすし屋ですしを食べられるだけ食べた。	
\\	きのうはすし 屋[や]ですしを 食[た]べられるだけ 食[た]べた。
\\	学生のうちに海外に行けるだけ行ってみたいと思います。	
\\	学生[がくせい]のうちに 海外[かいがい]に 行[い]けるだけ 行[い]ってみたいと 思[おも]います。
\\	運転手さん、飛行機に遅れてしまうので、飛ばせるだけ飛ばして下さい。	
\\	運転[うんてん] 手[しゅ]さん、 飛行機[ひこうき]に 遅[おく]れてしまうので、 飛[と]ばせるだけ 飛[と]ばして 下[くだ]さい。
\\	前から抱いていた不服を言うだけ言ったらすうっとした。	
\\	前[まえ]から 抱[いだ]いていた 不服[ふふく]を 言[い]うだけ 言[い]ったらすうっとした。
\\	お金は使うだけ使っても故障の少ない車が買いたい。	
\\	お 金[かね]は 使[つか]うだけ 使[つか]っても 故障[こしょう]の 少[すく]ない 車[くるま]が 買[か]いたい。
\\	きのうは寝たいだけ寝たので、心も休みもすっきりした。	
\\	きのうは 寝[ね]たいだけ 寝[ね]たので、 心[こころ]も 休[やす]みもすっきりした。
\\	通風になった時、私の足の親指は腫れるだけ腫れてしまいました。	
\\	通風[つうふう]になった 時[とき]、 私[わたし]の 足[あし]の 親指[おやゆび]は 腫[は]れるだけ 腫[は]れてしまいました。
\\	今まで都会の真ん中でとてもうるさかったので、今度はできるだけ静かなところに住みたいです。	
\\	今[いま]まで 都会[とかい]の 真ん中[まんなか]でとてもうるさかったので、 今度[こんど]はできるだけ 静[しず]かなところに 住[す]みたいです。
\\	この会社に入りたい理由、また入社したらやりたいことは何ですか。できるだけ具体的にお書き下さい。	
\\	この 会社[かいしゃ]に 入[はい]りたい 理由[りゆう]、また 入社[にゅうしゃ]したらやりたいことは 何[なん]ですか。できるだけ 具体[ぐたい] 的[てき]にお 書[か]き 下[くだ]さい。
\\	その学会にはできるだけ出席するようにします。	
\\	その 学会[がっかい]にはできるだけ 出席[しゅっせき]するようにします。
\\	洗濯したら、シャツが縮むだけ縮んでしまった。	
\\	洗濯[せんたく]したら、シャツが 縮[ちぢ]むだけ 縮[ちぢ]んでしまった。
\\	飲み会は盛り上がるだけ盛り上がった。	
\\	飲[の]み 会[かい]は 盛り上[もりあ]がるだけ 盛り上[もりあ]がった。
\\	彼女は人気女優だけあってすごいうちに住んでいる。	
\\	彼女[かのじょ]は 人気[にんき] 女優[じょゆう]だけあってすごいうちに 住[す]んでいる。
\\	聡子は小さい時からバレエをしているだけあって体が柔らかい。	
\\	聡子[さとこ]は 小[ちい]さい 時[とき]からバレエをしているだけあって 体[からだ]が 柔[やわ]らかい。
\\	彼が自慢するだけあって、彼のオフィスからの眺めはちょっとほかでは見られないものだった。	
\\	彼[かれ]が 自慢[じまん]するだけあって、 彼[かれ]のオフィスからの 眺[なが]めはちょっとほかでは 見[み]られないものだった。
\\	日本は火山国だけあって年中地震が絶えない。	
\\	日本[にほん]は 火山[かざん] 国[こく]だけあって 年中[ねんじゅう] 地震[じしん]が 絶[た]えない。
\\	この壷はさすが名工の作品だけあって色や形に気品がある。	
\\	この 壷[つぼ]はさすが 名工[めいこう]の 作品[さくひん]だけあって 色[いろ]や 形[かたち]に 気品[きひん]がある。
\\	オーストラリア南端の地だけあって8月のメルボルンは実に寒い。	
\\	オーストラリア 南端[なんたん]の 地[ち]だけあって 8月[はちがつ]のメルボルンは 実[じつ]に 寒[さむ]い。
\\	さすが日本一の花火大会と言われるだけあって、規模も芸術性もすばらしいものだった。	
\\	さすが 日本一[にほんいち]の 花火[はなび] 大会[たいかい]と 言[い]われるだけあって、 規模[きぼ]も 芸術[げいじゅつ] 性[せい]もすばらしいものだった。
\\	あのレストランは笠原さんのお薦めだけあっておいしかったです。	
\\	あのレストランは 笠原[かさはら]さんのお 薦[すす]めだけあっておいしかったです。
\\	この鞄は高いだけあって材料もいいし作りもしっかりしている。	
\\	この 鞄[かばん]は 高[たか]いだけあって 材料[ざいりょう]もいいし 作[つく]りもしっかりしている。
\\	あのうるさい社長が信頼するだけあって、彼の秘書は実に有能だ。	
\\	あのうるさい 社長[しゃちょう]が 信頼[しんらい]するだけあって、 彼[かれ]の 秘書[ひしょ]は 実[じつ]に 有能[ゆうのう]だ。
\\	ビルはもとフットボール選手だけに足が速い。	
\\	ビルはもとフットボール 選手[せんしゅ]だけに 足[あし]が 速[はや]い。
\\	ホワイトさんは長い間日本に住んでいただけに日本のことをよく知っている。	
\\	ホワイトさんは 長[なが]い 間[あいだ] 日本[にほん]に 住[す]んでいただけに 日本[にほん]のことをよく 知[し]っている。
\\	彼はもとアナウンサーだけに声がよく通る。	
\\	彼[かれ]はもとアナウンサーだけに 声[こえ]がよく 通[とお]る。
\\	山内先生は英語の先生だっただけに英語の文法をよく知っている。	
\\	山内[やまうち] 先生[せんせい]は 英語[えいご]の 先生[せんせい]だっただけに 英語[えいご]の 文法[ぶんぽう]をよく 知[し]っている。
\\	雅人は一人っ子だけについ甘やかしてしまう。	
\\	雅人[まさと]は 一人っ子[ひとりっこ]だけについ 甘[あま]やかしてしまう。
\\	孝夫は学生時代に相撲をやっていただけに体が大きい。	
\\	孝夫[たかお]は 学生[がくせい] 時代[じだい]に 相撲[すもう]をやっていただけに 体[からだ]が 大[おお]きい。
\\	チェンさんはアメリカの大学を卒業しただけに英語が上手だ。	
\\	チェンさんはアメリカの 大学[だいがく]を 卒業[そつぎょう]しただけに 英語[えいご]が 上手[じょうず]だ。
\\	ハリスさんは日本の大学で長い間教えていただけに日本の大学の問題をよく知っている。	
\\	ハリスさんは 日本[にほん]の 大学[だいがく]で 長[なが]い 間[あいだ] 教[おし]えていただけに 日本[にほん]の 大学[だいがく]の 問題[もんだい]をよく 知[し]っている。
\\	春樹はジャズが何よりも好きなだけにジャズ・ミュージシャンの名前をよく知っている。	
\\	春樹[はるき]はジャズが 何[なに]よりも 好[す]きなだけにジャズ・ミュージシャンの 名前[なまえ]をよく 知[し]っている。
\\	彼女がとても忙しいことをよく知っているだけに、彼女にはこの仕事を頼みにくい。	
\\	彼女[かのじょ]がとても 忙[いそが]しいことをよく 知[し]っているだけに、 彼女[かのじょ]にはこの 仕事[しごと]を 頼[たの]みにくい。
\\	ジョージは仕事の覚えが遅いだけに昇進も遅かった。	
\\	ジョージは 仕事[しごと]の 覚[おぼ]えが 遅[おそ]いだけに 昇進[しょうしん]も 遅[おそ]かった。
\\	ハリーは仕事の覚えが早いだけあって昇進も早かった。	
\\	ハリーは 仕事[しごと]の 覚[おぼ]えが 早[はや]いだけあって 昇進[しょうしん]も 早[はや]かった。
\\	吉岡さんが薦めるだけあってここの料理はちょっとそこらでは食べられない。	
\\	吉岡[よしおか]さんが 薦[すす]めるだけあってここの 料理[りょうり]はちょっとそこらでは 食[た]べられない。
\\	彼が自慢するだけあって彼のオフィスからの眺めはちょっとほかでは見られないものだった。	
\\	彼[かれ]が 自慢[じまん]するだけあって 彼[かれ]のオフィスからの 眺[なが]めはちょっとほかでは 見[み]られないものだった。
\\	明子は韓国語がとても上手だ。さすがに2年間ソウル大学に留学していただけのことはある。	
\\	明子[あきこ]は 韓国[かんこく] 語[ご]がとても 上手[じょうず]だ。さすがに2 年間[ねんかん]ソウル 大学[だいがく]に 留学[りゅうがく]していただけのことはある。
\\	毎日1キロ以上水泳をしてきただけのことはあって、彼は90歳の今もかくしゃくとしている。	
\\	毎日[まいにち]1キロ 以上[いじょう] 水泳[すいえい]をしてきただけのことはあって、 彼[かれ]は90 歳[さい]の 今[いま]もかくしゃくとしている。
\\	山田先生は日本語の教え方が上手だ。アメリカの大学で20年近く日本語を教えていらしただけのことはある。	
\\	山田[やまだ] 先生[せんせい]は 日本語[にほんご]の 教[おし]え 方[かた]が 上手[じょうず]だ。アメリカの 大学[だいがく]で 
\\	年[ねん] 近[ちか]く 日本語[にほんご]を 教[おし]えていらしただけのことはある。
\\	チャンさんは世界の経済状況をよくつかんでいる。経済コンサルタントとして世界を飛び回っているだけのことはある。	
\\	チャンさんは 世界[せかい]の 経済[けいざい] 状況[じょうきょう]をよくつかんでいる。 経済[けいざい]コンサルタントとして 世界[せかい]を 飛び回[とびまわ]っているだけのことはある。
\\	隣の家の子供たちは実に行儀がいい。親のしつけが厳しいだけのことはある。	
\\	隣[となり]の 家[いえ]の 子供[こども]たちは 実[じつ]に 行儀[ぎょうぎ]がいい。 親[おや]のしつけが 厳[きび]しいだけのことはある。
\\	田中はさして才能もないのに昇進が早い。上司への気配りが巧みなだけのことはある。	
\\	田中[たなか]はさして 才能[さいのう]もないのに 昇進[しょうしん]が 早[はや]い。 上司[じょうし]への 気配[きくば]りが 巧[たく]みなだけのことはある。
\\	アインシュタインの理論は21世紀の現在でも死んでいない。さすがに20世紀最大の物理学者と言われるだけのことはある。	
\\	アインシュタインの 理論[りろん]は 
\\	世紀[せいき]の 現在[げんざい]でも 死[し]んでいない。さすがに 
\\	世紀[せいき] 最大[さいだい]の 物理[ぶつり] 学者[がくしゃ]と 言[い]われるだけのことはある。
\\	幸い日本で日本語を勉強して、かなり話せるようになりました。日本に行っただけのことはありました。	
\\	幸[さいわ]い 日本[にほん]で 日本語[にほんご]を 勉強[べんきょう]して、かなり 話[はな]せるようになりました。 日本[にほん]に 行[い]っただけのことはありました。
\\	彼は若いころから俳句を作ってきただけのことはあって、さすがに自然に対する観察眼が鋭い。	
\\	彼[かれ]は 若[わか]いころから 俳句[はいく]を 作[つく]ってきただけのことはあって、さすがに 自然[しぜん]に 対[たい]する 観察[かんさつ] 眼[がん]が 鋭[するど]い。
\\	あの時の大きな失敗は思い出すだに恥ずかしい。	
\\	あの 時[とき]の 大[おお]きな 失敗[しっぱい]は 思い出[おもいだ]すだに 恥[は]ずかしい。
\\	イチローが、84年間続いたジョージ・シスラーの最多安打記録を破るとは想像だにしなかった。	
\\	イチローが、 84年間[はちじゅうよねんかん] 続[つづ]いたジョージ・シスラーの 最多[さいた] 安打[あんだ] 記録[きろく]を 破[やぶ]るとは 想像[そうぞう]だにしなかった。
\\	私の体脂肪率は今どれくらいあるんだろうか。考えるだに、恐ろしい。	
\\	私[わたし]の 体[からだ] 脂肪[しぼう] 率[りつ]は 今[いま]どれくらいあるんだろうか。 考[かんが]えるだに、 恐[おそ]ろしい。
\\	私が父の真似をしていたら、どんなことになった。想像するだに心細いものがあります。	
\\	私[わたし]が 父[ちち]の 真似[まね]をしていたら、どんなことになった。 想像[そうぞう]するだに 心細[こころぼそ]いものがあります。
\\	この猛暑の中、友人は避暑地で休日を楽しんでいるとは、思うだにうらやましい。	
\\	この 猛暑[もうしょ]の 中[なか]、 友人[ゆうじん]は 避暑[ひしょ] 地[ち]で 休日[きゅうじつ]を 楽[たの]しんでいるとは、 思[おも]うだにうらやましい。
\\	アメリカの指導者たちの大多数は9・11のようなテロが発生する可能性を考慮だにしなかったらしい。	
\\	アメリカの 指導[しどう] 者[しゃ]たちの 大[だい] 多数[たすう]は 9・ 11のようなテロが 発生[はっせい]する 可能[かのう] 性[せい]を 考慮[こうりょ]だにしなかったらしい。
\\	若者たちは毎晩スケートボードで駅前のロータリーを走り回っているのに、警官は注意だにしない。	
\\	若者[わかもの]たちは 毎晩[まいばん]スケートボードで 駅前[えきまえ]のロータリーを 走り回[はしりまわ]っているのに、 警官[けいかん]は 注意[ちゅうい]だにしない。
\\	発表当時、彼の文献学的方法論は否定的評価を受けたが、彼の信念は微動だにしなかった。	
\\	発表[はっぴょう] 当時[とうじ]、 彼[かれ]の 文献[ぶんけん] 学[がく] 的[てき] 方法[ほうほう] 論[ろん]は 否定[ひてい] 的[てき] 評価[ひょうか]を 受[う]けたが、 彼[かれ]の 信念[しんねん]は 微動[びどう]だにしなかった。
\\	小林は何か心配事があるのか、青い顔をして、こちらを一瞥だにしなかった。	
\\	小林[こばやし]は 何[なに]か 心配[しんぱい] 事[ごと]があるのか、 青[あお]い 顔[かお]をして、こちらを 一瞥[いちべつ]だにしなかった。
\\	人間は他の種を絶滅させるような環境を作り出しても一顧だにしない。	
\\	人間[にんげん]は 他[ほか]の 種[しゅ]を 絶滅[ぜつめつ]させるような 環境[かんきょう]を 作り出[つくりだ]しても 一顧[いっこ]だにしない。
\\	最近は新幹線で東京から京都まで行くだけでも疲れてしまう。	
\\	最近[さいきん]は 新幹線[しんかんせん]で 東京[とうきょう]から 京都[きょうと]まで 行[い]くだけでも 疲[つか]れてしまう。
\\	僕は彼女のそばにいるだけでも気持ちが安らぐ。	
\\	僕[ぼく]は 彼女[かのじょ]のそばにいるだけでも 気持[きも]ちが 安[やす]らぐ。
\\	父は散歩さえしない。	
\\	父[ちち]は 散歩[さんぽ]さえしない。
\\	彼はひらがなすら読めない。	
\\	彼[かれ]はひらがなすら 読[よ]めない。
\\	庭にはすみれだのチューリップだのつつじだの、花がたくさん咲いていた。	
\\	庭[にわ]にはすみれだのチューリップだのつつじだの、 花[はな]がたくさん 咲[さ]いていた。
\\	学生たちは食堂のご飯がまずいだの、授業がつまらないだのと、いつも不平を言っている。	
\\	学生[がくせい]たちは 食堂[しょくどう]のご 飯[はん]がまずいだの、 授業[じゅぎょう]がつまらないだのと、いつも 不平[ふへい]を 言[い]っている。
\\	僕はそばだの、うどんだの、ラーメンだのといった麺類が大好きだ。	
\\	僕[ぼく]はそばだの、うどんだの、ラーメンだのといった 麺類[めんるい]が 大好[だいす]きだ。
\\	今の高校生は携帯電話だのコンピュータゲームだのにはまっている。	
\\	今[いま]の 高校生[こうこうせい]は 携帯[けいたい] 電話[でんわ]だのコンピュータゲームだのにはまっている。
\\	彼は僕が飲みに行こうかと誘っても、いつも頭が痛いだの、約束があるだのと言い訳をする。	
\\	彼[かれ]は 僕[ぼく]が 飲[の]みに 行[い]こうかと 誘[さそ]っても、いつも 頭[あたま]が 痛[いた]いだの、 約束[やくそく]があるだのと 言い訳[いいわけ]をする。
\\	彼女は仕事が好きだから結婚しても仕事を続けるだの、家庭生活と職業を両立できるだのと主張していた。	
\\	彼女[かのじょ]は 仕事[しごと]が 好[す]きだから 結婚[けっこん]しても 仕事[しごと]を 続[つづ]けるだの、 家庭[かてい] 生活[せいかつ]と 職業[しょくぎょう]を 両立[りょうりつ]できるだのと 主張[しゅちょう]していた。
\\	彼女は自分がテニスが上手だの、ピアノがうまいだのと自慢がかりしている。	
\\	彼女[かのじょ]は 自分[じぶん]がテニスが 上手[じょうず]だの、ピアノがうまいだのと 自慢[じまん]がかりしている。
\\	受験生たちは塾だのテストだのと忙しい毎日を送っている。	
\\	受験生[じゅけんせい]たちは 塾[じゅく]だのテストだのと 忙[いそが]しい 毎日[まいにち]を 送[おく]っている。
\\	ーどうして会ってくれないの? ーだって、僕、今日忙しいんだ。	ー
\\	ーどうして 会[あ]ってくれないの? ーだって、 僕[ぼく]、 今日[きょう] 忙[いそが]しいんだ。
\\	[母:]ーさや、どうしてもっとご飯食べないの。 [さや:]ーだっておなか、空いていないもん。	
\\	ーさや、どうしてもってご 飯[はん] 食[た]べないの。 ーだっておなか、 空[す]いていないもん。
\\	[妻:]ー健太さん、今日はどうしてゴルフに行かないの。 [夫:]ーだって、今日は客と銀座で昼食をとらなきゃならないんだ。	ー
\\	ー 健太[けんた]さん、 今日[きょう]はどうしてゴルフに 行[い]かないの。 ーだって、 今日[きょう]は 客[きゃく]と 銀座[ぎんざ]で 昼食[ちゅうしょく]をとらなきゃならないんだ。
\\	ー路子、笑い事じゃないよ。 ーだって二人の話しているのを聞いていると面白いんだもん。	
\\	ー 路子[みちこ]、 笑い事[わらいごと]じゃないよ。 ーだって 二人[ふたり]の 話[はな]しているのを 聞[き]いていると 面白[おもしろ]いんだもん。
\\	もう35度じゃ驚かないね。だって連日気温は35度、36度、37度が当たり前の状態なんだもの。	
\\	もう 
\\	度[ど]じゃ 驚[おどろ]かないね。だって 連日[れんじつ] 気温[きおん]は 
\\	度[ど]、 
\\	度[ど]、 
\\	度[ど]が 当たり前[あたりまえ]の 状態[じょうたい]なんだもの。
\\	ーなんで、僕だけ烏龍茶なんですか? ーだって、お前はまだ未成年だろう?	
\\	ーなんで、 僕[ぼく]だけ 烏龍茶[うーろんちゃ]なんですか? ーだって、お 前[まえ]はまだ 未成年[みせいねん]だろう?
\\	彼がいるはずの店にやってきたが、入るべきかどうか迷っていた。だって、ここに来るとろくなことがない。	
\\	彼[かれ]がいるはずの 店[みせ]にやってきたが、 入[はい]るべきかどうか 迷[まよ]っていた。だって、ここに 来[く]るとろくなことがない。
\\	一般的に偏頭痛の方がつらいらしいけれど、僕にはこの緊張型の方がつらい。だって「頭が痛い」というのがすでにストレスの一部に荷担しているから。	
\\	一般[いっぱん] 的[てき]に 偏頭痛[へんずつう]の 方[ほう]がつらいらしいけれど、 僕[ぼく]にはこの 緊張[きんちょう] 型[がた]の 方[ほう]がつらい。だって
\\	頭[あたま]が 痛[いた]い」というのがすでにストレスの 一部[いちぶ]に 荷担[かたん]しているから。
\\	どんなにいい先生でも時々間違います。	
\\	どんなにいい 先生[せんせい]でも 時々[ときどき] 間違[まちが]います。
\\	幸恵、来ないんだって。	
\\	幸恵[ゆきえ]、 来[こ]ないんだって。
\\	明日は雪だってさ。	
\\	明日[あす]は 雪[ゆき]だってさ。
\\	この翻訳なら私にだってできます。	
\\	この 翻訳[ほんやく]なら 私[わたし]にだってできます。
\\	数多いお客さんの中には難しい人だっている。	
\\	数多[かずおお]いお 客[きゃく]さんの 中[なか]には 難[むずか]しい 人[ひと]だっている。
\\	今からだって、遅くないから、彼に電話してみたら?	
\\	今[いま]からだって、 遅[おそ]くないから、 彼[かれ]に 電話[でんわ]してみたら?
\\	そんないい仕事だったら誰だってやりたい。	
\\	そんないい 仕事[しごと]だったら 誰[だれ]だってやりたい。
\\	君だって子供のことは考えているだろう。	
\\	君[きみ]だって 子供[こども]のことは 考[かんが]えているだろう。
\\	それは部長だって知らないことだ。	
\\	それは 部長[ぶちょう]だって 知[し]らないことだ。
\\	最近は忙しいので日曜だって休めないことが多い。	
\\	最近[さいきん]は 忙[いそが]しいので 日曜[にちよう]だって 休[やす]めないことが 多[おお]い。
\\	もういい加減に許してやれよ。彼だって悪気があってやったわけじゃないんだから。	
\\	もういい 加減[かげん]に 許[ゆる]してやれよ。 彼[かれ]だって 悪気[わるぎ]があってやったわけじゃないんだから。
\\	これからだって彼にはいろいろ頼まないといけないんだから、そんなに彼のことを悪く言うもんじゃないよ。	
\\	これからだって 彼[かれ]にはいろいろ 頼[たの]まないといけないんだから、そんなに 彼[かれ]のことを 悪[わる]く 言[い]うもんじゃないよ。
\\	この間だってそうだ。僕のノートを借りたまま言われるまで返さないんだ。	
\\	この 間[あいだ]だってそうだ。 僕[ぼく]のノートを 借[か]りたまま 言[い]われるまで 返[かえ]さないんだ。
\\	一度だって僕が君に嘘をついたころがあるかい。	
\\	一度[いちど]だって 僕[ぼく]が 君[きみ]に 嘘[うそ]をついたころがあるかい。
\\	それだけ日本語を知っていたら一人だってできるからやってごらん。	
\\	それだけ 日本語[にほんご]を 知[し]っていたら 一人[ひとり]だってできるからやってごらん。
\\	私は子供がほしがるものは何だって買い与えた。今から思えばそれは間違いだったかもしれない。	
\\	私[わたし]は 子供[こども]がほしがるものは 何[なん]だって 買い与[かいあた]えた。 今[いま]から 思[おも]えばそれは 間違[まちが]いだったかもしれない。
\\	どっちだって同じだろう。早く決めちゃえよ。	
\\	どっちだって 同[おな]じだろう。 早[はや]く 決[き]めちゃえよ。
\\	いつだってあなたのやりたいことを優先してきたじゃないの。	
\\	いつだってあなたのやりたいことを 優先[ゆうせん]してきたじゃないの。
\\	何だってそんな馬鹿なことをしたんだ。	
\\	何[なん]だってそんな 馬鹿[ばか]なことをしたんだ。
\\	その集まりには500人も集まった。	
\\	その 集[あつ]まりには 
\\	人[にん]も 集[あつ]まった。
\\	パーティーに先生が一人も来なかった。	
\\	パーティーに 先生[せんせい]が 一人[ひとり]も 来[こ]なかった。
\\	そんな簡単なこともできないのか。	
\\	そんな 簡単[かんたん]なこともできないのか。
\\	課長は私のレポートを見もしなかった。	
\\	課長[かちょう]は 私[わたし]のレポートを 見[み]もしなかった。
\\	たとえ社長であれ、そのような行為は許されない。	
\\	たとえ 社長[しゃちょう]であれ、そのような 行為[こうい]は 許[ゆる]されない。
\\	どんな問題であれ、必ず私に報告して下さい。	
\\	どんな 問題[もんだい]であれ、 必[かなら]ず 私[わたし]に 報告[ほうこく]して 下[くだ]さい。
\\	通勤がいかに不便であれ、今の家を変わるつもりはない。	
\\	通勤[つうきん]がいかに 不便[ふべん]であれ、 今[いま]の 家[いえ]を 変[か]わるつもりはない。
\\	いかに難しい問題であれ彼ならきっと解決できるはずだ。	
\\	いかに 難[むずか]しい 問題[もんだい]であれ 彼[かれ]ならきっと 解決[かいけつ]できるはずだ。
\\	この練習は、たとえ数分間であれ、毎日することが大切です。	
\\	この 練習[れんしゅう]は、たとえ 数[すう] 分間[ふんかん]であれ、 毎日[まいにち]することが 大切[たいせつ]です。
\\	彼の提言は、たとえそれが実行不可能であれ、聞くに値する。	
\\	彼[かれ]の 提言[ていげん]は、たとえそれが 実行[じっこう] 不可能[ふかのう]であれ、 聞[き]くに 値[あたい]する。
\\	暗号技術については、たとえ短期間の使用であれ、その輸出入を規制している国があります。	
\\	暗号[あんごう] 技術[ぎじゅつ]については、たとえ 短期間[たんきかん]の 使用[しよう]であれ、その 輸出入[ゆしゅつにゅう]を 規制[きせい]している 国[くに]があります。
\\	何事であれ、その仕事で生活するとなれば楽なものはない。	
\\	何事[なにごと]であれ、その 仕事[しごと]で 生活[せいかつ]するとなれば 楽[らく]なものはない。
\\	どのような理由であれ、著作物を複製する時は、著作権の侵害に注意しなければならない。	
\\	どのような 理由[りゆう]であれ、 著作[ちょさく] 物[ぶつ]を 複製[ふくせい]する 時[とき]は、 著作[ちょさく] 権[けん]の 侵害[しんがい]に 注意[ちゅうい]しなければならない。
\\	当時の軍隊では上官の命令がいかに理不尽であれ、下級兵はそれに従うしかなかったのだ。	
\\	当時[とうじ]の 軍隊[ぐんたい]では 上官[じょうかん]の 命令[めいれい]がいかに 理不尽[りふじん]であれ、 下級[かきゅう] 兵[へい]はそれに 従[したが]うしかなかったのだ。
\\	男性であれ、女性であれ、人間としての権利は同じはずだ。	
\\	男性[だんせい]であれ、 女性[じょせい]であれ、 人間[にんげん]としての 権利[けんり]は 同[おな]じはずだ。
\\	有名であれ、無名であれ、人生は短い。	
\\	有名[ゆうめい]であれ、 無名[むめい]であれ、 人生[じんせい]は 短[みじか]い。
\\	ここは春であれ秋であれ美しいから観光に向いている。	
\\	ここは 春[はる]であれ 秋[あき]であれ 美[うつく]しいから 観光[かんこう]に 向[む]いている。
\\	イスラエル人であれ、パレスチナ人であれ、平和を望んでいることでは同じはずだ。	
\\	イスラエル 人[じん]であれ、パレスチナ 人[じん]であれ、 平和[へいわ]を 望[のぞ]んでいることでは 同[おな]じはずだ。
\\	彼は和食であれ、洋食であれ、何でも食べてしまう。	
\\	彼[かれ]は 和食[わしょく]であれ、 洋食[ようしょく]であれ、 何[なに]でも 食[た]べてしまう。
\\	あの学者は哲学であれ、言語学であれ、人類学であれ、同じように精通している。	
\\	あの 学者[がくしゃ]は 哲学[てつがく]であれ、 言語[げんご] 学[がく]であれ、 人類[じんるい] 学[がく]であれ、 同[おな]じように 精通[せいつう]している。
\\	平日であれ、週末であれ、健太は仕事に追われている。	
\\	平日[へいじつ]であれ、 週末[しゅうまつ]であれ、 健太[けんた]は 仕事[しごと]に 追[お]われている。
\\	写真であれ、コンピュータグラフィックスであれ、求められたイメージを表現するのはたいへん難しい。	
\\	写真[しゃしん]であれ、コンピュータグラフィックスであれ、 求[もと]められたイメージを 表現[ひょうげん]するのはたいへん 難[むずか]しい。
\\	場所が便利であれ、不便であれ、マイホームが持ちたい。	
\\	場所[ばしょ]が 便利[べんり]であれ、 不便[ふべん]であれ、マイホームが 持[も]ちたい。
\\	遅かれ早かれ、この国でも
\\	革命が起こるだろう。	
\\	遅[おそ]かれ 早[はや]かれ、この 国[くに]でも 
\\	革命[かくめい]が 起[お]こるだろう。
\\	インターネットの使用で国境の意識が薄れるから、良かれ悪しかれ、国の概念も薄らいでいくだろう。	
\\	インターネットの 使用[しよう]で 国境[こっきょう]の 意識[いしき]が 薄[うす]れるから、 良[よ]かれ 悪[あ]しかれ、 国[くに]の 概念[がいねん]も 薄[うす]らいでいくだろう。
\\	言語学者は多かれ少なかれチョムスキーの影響を受けている。	
\\	言語[げんご] 学者[がくしゃ]は 多[おお]かれ 少[すく]なかれチョムスキーの 影響[えいきょう]を 受[う]けている。
\\	彼はプロジェクトが難しかろうと易しかろうと、同じような情熱でそれに挑む。	
\\	彼[かれ]はプロジェクトが 難[むずか]しかろうと 易[やさ]しかろうと、 同[おな]じような 情熱[じょうねつ]でそれに 挑[いど]む。
\\	暑かろうと寒かろうと、彼女は一年中泳ぐ。	
\\	暑[あつ]かろうと 寒[さむ]かろうと、 彼女[かのじょ]は 一年中[いちねんじゅう] 泳[およ]ぐ。
\\	値段が高かろうと高くなかろうとおいしいものが食べたい。	
\\	値段[ねだん]が 高[たか]かろうと 高[たか]くなかろうとおいしいものが 食[た]べたい。
\\	日本は戦後アメリカから良かれ悪しかれ、いろいろな影響を受けてきた。	
\\	日本[にほん]は 戦後[せんご]アメリカから 良[よ]かれ 悪[あ]しかれ、いろいろな 影響[えいきょう]を 受[う]けてきた。
\\	結果が良かろうと悪かろうと、別にかまわない。	
\\	結果[けっか]が 良[よ]かろうと 悪[わる]かろうと、 別[べつ]にかまわない。
\\	私たちは多かれ少なかれ弱点を持っている。	
\\	私[わたし]たちは 多[おお]かれ 少[すく]なかれ 弱点[じゃくてん]を 持[も]っている。
\\	給料が多かろうと少なかろうと、その人の価値に違いはない。	
\\	給料[きゅうりょう]が 多[おお]かろうと 少[すく]なかろうと、その 人[ひと]の 価値[かち]に 違[ちが]いはない。
\\	男性であろうと、なかろうと、人間としての権利は同じはずだ。	
\\	男性[だんせい]であろうと、なかろうと、 人間[にんげん]としての 権利[けんり]は 同[おな]じはずだ。
\\	有名であれ、なかろうと人生は短い。	
\\	有名[ゆうめい]であれ、なかろうと 人生[じんせい]は 短[みじか]い。
\\	幼稚園児でもあるまいし一人で学校に行きなさい。	
\\	幼稚園[ようちえん] 児[じ]でもあるまいし 一人[ひとり]で 学校[がっこう]に 行[い]きなさい。
\\	フランスに行けるわけじゃあるまいし、どうしてそんなにフランス語を勉強しているの。	
\\	フランスに 行[い]けるわけじゃあるまいし、どうしてそんなに フランス語[ふらんすご]を 勉強[べんきょう]しているの。
\\	学生じゃあるまいし、仕事でも探したらどうだ。	
\\	学生[がくせい]じゃあるまいし、 仕事[しごと]でも 探[さが]したらどうだ。
\\	政治家でもあるまいし、嘘をつかないで下さい。	
\\	政治[せいじ] 家[か]でもあるまいし、 嘘[うそ]をつかないで 下[くだ]さい。
\\	犬じゃあるまいし、そんなに人にじゃれるようなことはやめなさい。	
\\	犬[いぬ]じゃあるまいし、そんなに 人[ひと]にじゃれるようなことはやめなさい。
\\	お葬式でもあるまいし、そんなにしゅんとしないで下さい。	
\\	お 葬式[そうしき]でもあるまいし、そんなにしゅんとしないで 下[くだ]さい。
\\	現地の英語学校に入ったら、日本人の女の子が英語で「分からないことがあったら聞いて」と言ってきた。授業中じゃあるまいし、日本語で言えばいいのに。	
\\	現地[げんち]の 英語[えいご] 学校[がっこう]に 入[はい]ったら、 日本人[にほんじん]の 女の子[おんなのこ]が 英語[えいご]で
\\	分[わ]からないことがあったら 聞[き]いて」と 言[い]ってきた。 授業[じゅぎょう] 中[ちゅう]じゃあるまいし、 日本語[にほんご]で 言[い]えばいいのに。
\\	このうどん屋では注文してから、うどんがくるまで長く待ったされ、いらいらする。フランス料理でもあるまいし……。	
\\	このうどん 屋[や]では 注文[ちゅうもん]してから、うどんがくるまで 長[なが]く 待[ま]ったされ、いらいらする。フランス 料理[りょうり]でもあるまいし……。
\\	息子は音楽を専攻しているんじゃあるまいし、ギターばかり弾いていて卒業できるんだろうか。	
\\	息子[むすこ]は 音楽[おんがく]を 専攻[せんこう]しているんじゃあるまいし、ギターばかり 弾[ひ]いていて 卒業[そつぎょう]できるんだろうか。
\\	そんなことは子供でも誰でも知っているよ。	
\\	そんなことは 子供[こども]でも 誰[だれ]でも 知[し]っているよ。
\\	私ってマンガでも何でも読み出すと止まらないタイプなの。	
\\	私[わたし]ってマンガでも 何[なん]でも 読み出[よみだ]すと 止[と]まらないタイプなの。
\\	シャツの大きさは
\\	でも
\\	でも、どちらでもかまいません。	
\\	シャツ[てぃーしゃつ]の 大[おお]きさは 
\\	でも 
\\	でも、どちらでもかまいません。
\\	欧米では、バスでも地下鉄でも、車椅子でどこへでも行ける。どうしても助けが必要な時は誰かに頼めばたとえ日本人にでも誰にでも手を貸してくれる。	
\\	欧米[おうべい]では、バスでも 地下鉄[ちかてつ]でも、 車椅子[くるまいす]でどこへでも 行[い]ける。どうしても 助[たす]けが 必要[ひつよう]な 時[とき]は 誰[だれ]かに 頼[たの]めばたとえ 日本人[にほんじん]にでも 誰[だれ]にでも 手[て]を 貸[か]してくれる。
\\	最近のパーティーは留学生でも日本人学生でも誰でも参加自由で、来る人は飲み物、音楽、自分の作った料理を持ってくるというようにしています。	
\\	最近[さいきん]のパーティーは 留学生[りゅうがくせい]でも 日本人[にほんじん] 学生[がくせい]でも 誰[だれ]でも 参加[さんか] 自由[じゆう]で、 来[く]る 人[ひと]は 飲み物[のみもの]、 音楽[おんがく]、 自分[じぶん]の 作[つく]った 料理[りょうり]を 持[も]ってくるというようにしています。
\\	ーすみませんが、来週、検査の結果が出たら、もう一度来ていただけますか。 ー分かりました。来週でもいつでも伺います。	
\\	ーすみませんが、 来週[らいしゅう]、 検査[けんさ]の 結果[けっか]が 出[で]たら、もう 一度[いちど] 来[き]ていただけますか。 ー 分[わ]かりました。 来週[らいしゅう]でもいつでも 伺[うかが]います。
\\	この野球選手は小学生にでも、誰にでも人気がある。	
\\	この 野球[やきゅう] 選手[せんしゅ]は 小学生[しょうがくせい]にでも、 誰[だれ]にでも 人気[にんき]がある。
\\	彼は若い人でも、年取った人とでも、誰とでもうまく話せる。	
\\	彼[かれ]は 若[わか]い 人[ひと]でも、 年取[としと]った 人[ひと]とでも、 誰[だれ]とでもうまく 話[はな]せる。
\\	このプロジェクトは今月からでも、来月からでも、いつからでも始められる。	
\\	このプロジェクトは 今月[こんげつ]からでも、 来月[らいげつ]からでも、いつからでも 始[はじ]められる。
\\	彼はどちらかと言うと教育者というより研究者だ。	
\\	彼[かれ]はどちらかと 言[い]うと 教育[きょういく] 者[しゃ]というより 研究[けんきゅう] 者[しゃ]だ。
\\	私はどちらかと言うと外向的というよりは内向的だ。	
\\	私[わたし]はどちらかと 言[い]うと 外向[がいこう] 的[てき]というよりは 内向[ないこう] 的[てき]だ。
\\	母はどちらかと言うと、おとなしい方です。	
\\	母[はは]はどちらかと 言[い]うと、おとなしい 方[ほう]です。
\\	洋食も悪くないが、どちらかと言うと、和食の方がいい。	
\\	洋食[ようしょく]も 悪[わる]くないが、どちらかと 言[い]うと、 和食[わしょく]の 方[ほう]がいい。
\\	山本は決して呑気な男ではない。どちらかと言うと神経質な男だ。	
\\	山本[やまもと]は 決[けっ]して 呑気[のんき]な 男[おとこ]ではない。どちらかと 言[い]うと 神経質[しんけいしつ]な 男[おとこ]だ。
\\	あの政治家のやり方はどちらかと言うと賢いというよりはずる賢い。	
\\	あの 政治[せいじ] 家[か]のやり 方[かた]はどちらかと 言[い]うと 賢[かしこ]いというよりはずる 賢[かしこ]い。
\\	君はどちらかと言うと楽観的というよりは悲観的だね。	
\\	君[きみ]はどちらかと 言[い]うと 楽観[らっかん] 的[てき]というよりは 悲観[ひかん] 的[てき]だね。
\\	彼女はどちらかと言うと、仕事に生きるタイプで、精神的にも強い人間だと思う。	
\\	彼女[かのじょ]はどちらかと 言[い]うと、 仕事[しごと]に 生[い]きるタイプで、 精神[せいしん] 的[てき]にも 強[つよ]い 人間[にんげん]だと 思[おも]う。
\\	あの大学はどちらかと言うと入学がやさしい方だ。	
\\	あの 大学[だいがく]はどちらかと 言[い]うと 入学[にゅうがく]がやさしい 方[ほう]だ。
\\	私はパーティーなどの集まりでは、どちらかと言うと、話役ではなく、聞き役の方だ。	
\\	私[わたし]はパーティーなどの 集[あつ]まりでは、どちらかと 言[い]うと、 話[はなし] 役[やく]ではなく、 聞き役[ききやく]の 方[ほう]だ。
\\	どちらかと言うと、夏目漱石よりは芥川龍之介の方が私の趣味に合っている。	
\\	どちらかと 言[い]うと、 夏目[なつめ] 漱石[そうせき]よりは 芥川[あくたがわ] 龍之介[りゅうのすけ]の 方[ほう]が 私[わたし]の 趣味[しゅみ]に 合[あ]っている。
\\	この大学の日本語の学生は、どちらかと言うと目的を持って勉強しているというよりは趣味として日本語を勉強している。	
\\	この 大学[だいがく]の 日本語[にほんご]の 学生[がくせい]は、どちらかと 言[い]うと 目的[もくてき]を 持[も]って 勉強[べんきょう]しているというよりは 趣味[しゅみ]として 日本語[にほんご]を 勉強[べんきょう]している。
\\	彼はどちらかと言うと話せより書く方が得意だ。	
\\	彼[かれ]はどちらかと 言[い]うと 話[はな]せより 書[か]く 方[ほう]が 得意[とくい]だ。
\\	彼はどちらかと言うと頭脳明晰だというよりは要領がいい。	
\\	彼[かれ]はどちらかと 言[い]うと 頭脳[ずのう] 明晰[めいせき]だというよりは 要領[ようりょう]がいい。
\\	彼女はどちらかと言うと人に優しいというよりはおせっかいだ。	
\\	彼女[かのじょ]はどちらかと 言[い]うと 人[ひと]に 優[やさ]しいというよりはおせっかいだ。
\\	この大学の日本語の学生は増えるどころか、むしろ減ってきている。	
\\	この 大学[だいがく]の 日本語[にほんご]の 学生[がくせい]は 増[ふ]えるどころか、むしろ 減[へ]ってきている。
\\	あの人と結婚するぐらいなら、むしろ死んだ方がいい。	
\\	あの 人[ひと]と 結婚[けっこん]するぐらいなら、むしろ 死[し]んだ 方[ほう]がいい。
\\	どうか私の言い分を聞いて下さい。	
\\	どうか 私[わたし]の 言い分[いいぶん]を 聞[き]いて 下[くだ]さい。
\\	どうか私を見捨てないで下さい。	
\\	どうか 私[わたし]を 見捨[みす]てないで 下[くだ]さい。
\\	どうかお許し下さい。	
\\	どうかお 許[ゆる]し 下[くだ]さい。
\\	どうかこれに懲りずにまたいらして下さい。	
\\	どうかこれに 懲[こ]りずにまたいらして 下[くだ]さい。
\\	どうか本当のことを話して下さい。	
\\	どうか 本当[ほんとう]のことを 話[はな]して 下[くだ]さい。
\\	どうかお体を大切に。	
\\	どうかお 体[からだ]を 大切[たいせつ]に。
\\	どうかお気を使わないで下さい。	
\\	どうかお 気[き]を 使[つか]わないで 下[くだ]さい。
\\	向寒の折から、どうかお体にお気をつけ下さい。	
\\	向寒[こうかん]の 折[おり]から、どうかお 体[からだ]にお 気[き]をつけ 下[くだ]さい。
\\	どうか命をお助け下さい。	
\\	どうか 命[いのち]をお 助[たす]け 下[くだ]さい。
\\	どうか今後ともよろしくお願いいたします。	
\\	どうか 今後[こんご]ともよろしくお 願[ねが]いいたします。
\\	はじめまして、どうぞよろしく。	
\\	はじめまして、どうぞよろしく。
\\	どうぞお先にいらして下さい。	
\\	どうぞお 先[さき]にいらして 下[くだ]さい。
\\	どうぞ週末に遊びにいらして下さい。	
\\	どうぞ 週末[しゅうまつ]に 遊[あそ]びにいらして 下[くだ]さい。
\\	ーもう食べてもよろしいですか。 ーええ、どうぞ。	
\\	ーもう 食[た]べてもよろしいですか。 ーええ、どうぞ。
\\	先生、どうか母の病気を治して下さい!	
\\	先生[せんせい]、どうか 母[はは]の 病気[びょうき]を 治[なお]して 下[くだ]さい!
\\	裁判長、どうか息子を死刑にしないで下さい。お願いです。	
\\	裁判[さいばん] 長[ちょう]、どうか 息子[むすこ]を 死刑[しけい]にしないで 下[くだ]さい。お 願[ねが]いです。
\\	自分のことで精一杯で、人のことなど考えるどころではない。	
\\	自分[じぶん]のことで 精一杯[せいいっぱい]で、 人[ひと]のことなど 考[かんが]えるどころではない。
\\	年末なのに多忙で、海外旅行どころではなかった。	
\\	年末[ねんまつ]なのに 多忙[たぼう]で、 海外[かいがい] 旅行[りょこう]どころではなかった。
\\	津波で愛する娘を亡くしたので、悲しいどころではない。死にたい気持ちだ。	
\\	津波[つなみ]で 愛[あい]する 娘[むすめ]を 亡[な]くしたので、 悲[かな]しいどころではない。 死[し]にたい 気持[きも]ちだ。
\\	定年になっても暇どころではない。毎日野菜作りに精を出している。	
\\	定年[ていねん]になっても 暇[ひま]どころではない。 毎日[まいにち] 野菜[やさい] 作[づく]りに 精[せい]を 出[だ]している。
\\	毎日の生活に汲々としていて、株を買うどころではありません。	
\\	毎日[まいにち]の 生活[せいかつ]に 汲々[きゅうきゅう]としていて、 株[かぶ]を 買[か]うどころではありません。
\\	私はテニスが大好きだが、最近大怪我をしてテニスどころではない。	
\\	私[わたし]はテニスが 大好[だいす]きだが、 最近[さいきん] 大[だい] 怪我[けが]をしてテニスどころではない。
\\	ーおい、今晩、飲みに行こうか。 ーそれどころじゃないんだ。妻が入院したんだ。	
\\	ーおい、 今晩[こんばん]、 飲[の]みに 行[い]こうか。 ーそれどころじゃないんだ。 妻[つま]が 入院[にゅういん]したんだ。
\\	目下寝たきりの母の介護で、外出どころではないんです。	
\\	目下[もっか] 寝[ね]たきりの 母[はは]の 介護[かいご]で、 外出[がいしゅつ]どころではないんです。
\\	このところ学生の修論指導で研究どころじゃないんだよ。	
\\	このところ 学生[がくせい]の 修論[しゅうろん] 指導[しどう]で 研究[けんきゅう]どころじゃないんだよ。
\\	[妻:]-あなた、この週末に熱海の温泉にでも行かない? [夫:]-それどころじゃないんだよ。原稿の締め切りが来週の月曜日なんだ。	
\\	ーあなた、この 週末[しゅうまつ]に 熱海[あたみ]の 温泉[おんせん]にでも 行[い]かない? ーそれどころじゃないんだよ。 原稿[げんこう]の 締め切[しめき]りが 来週[らいしゅう]の 月曜日[げつようび]なんだ。
\\	スイカマラソンは、レース中やレース後にスイカが食べられる。でも、走り終わったら、へとへとで、とてもスイカを食べるどころではないそうだ。	
\\	スイカマラソンは、レース 中[ちゅう]やレース 後[ご]にスイカが 食[た]べられる。でも、 走[はし]り 終[お]わったら、へとへとで、とてもスイカを 食[た]べるどころではないそうだ。
\\	今日は非常に暑かった。もう暖かいどころではない。	
\\	今日[きょう]は 非常[ひじょう]に 暑[あつ]かった。もう 暖[あたた]かいどころではない。
\\	ものすごい排気ガスを吸っているから、健康どころではない。	
\\	ものすごい 排気[はいき]ガスを 吸[す]っているから、 健康[けんこう]どころではない。
\\	ー彼女、日本語、話せるの? ーいや、話せるどころじゃない。一度も勉強したことがないんだ。	
\\	ー 彼女[かのじょ]、 日本語[にほんご]、 話[はな]せるの? ーいや、 話[はな]せるどころじゃない。一度[いちど]も 勉強[べんきょう]したことがないんだ。
\\	この二つの文章は言い方を違えども、言おうとしていることは同じだ。	
\\	この 二[ふた]つの 文章[ぶんしょう]は 言い方[いいかた]を 違[ちが]えども、 言[い]おうとしていることは 同[おな]じだ。
\\	行けども行けども人家が見えてこない。	
\\	行[い]けども 行[い]けども 人家[じんか]が 見[み]えてこない。
\\	私は知識はあれども学才がなく、今後学問の世界でやっていけるとは思えない。	
\\	私[わたし]は 知識[ちしき]はあれども 学才[がくさい]がなく、 今後[こんご] 学問[がくもん]の 世界[せかい]でやっていけるとは 思[おも]えない。
\\	汲めども尽きぬ泉のように、彼は次々と美しい曲を書き続けた。	
\\	汲[く]めども 尽[つ]きぬ 泉[いずみ]のように、 彼[かれ]は 次々[つぎつぎ]と 美[うつく]しい 曲[きょく]を 書[か]き 続[つづ]けた。
\\	見慣れているものというのは、視れども見えずで、毎日見ていてもその問題に気がつかないものである。	
\\	見慣[みな]れているものというのは、 視[み]れども 見[み]えずで、 毎日[まいにち] 見[み]ていてもその 問題[もんだい]に 気[き]がつかないものである。
\\	親といえども子供に対してそんな横暴は許されない。	
\\	親[おや]といえども 子供[こども]に 対[たい]してそんな 横暴[おうぼう]は 許[ゆる]されない。
\\	老いたりといえども、伊藤先生はまだ動きも速く、この剣道場の人は誰も先生に勝つことができない。	
\\	老[お]いたりといえども、 伊藤[いとう] 先生[せんせい]はまだ 動[うご]きも 速[はや]く、この 剣道[けんどう] 場[じょう]の 人[ひと]は 誰[だれ]も 先生[せんせい]に 勝[か]つことができない。
\\	走れども走れども雪原が続いているだけだった。	
\\	走[はし]れども 走[はし]れども 雪原[せつげん]が 続[つづ]いているだけだった。
\\	「心ここに在らざれば、視れども見えず、聴けども聞こえず、食らえども其の味わいを知らず。」―孔子	
\\	心[こころ]ここに 在[あ]らざれば、 視[み]れども 見[み]えず、 聴[き]けども 聞[き]こえず、 食[く]らえども 其[そ]の 味[あじ]わいを 知[し]らず。」― 孔子[こうし]
\\	王は君臨すれども統治せず。	
\\	王[おう]は 君臨[くんりん]すれども 統治[とうち]せず。
\\	声はすれども姿は見えず。	
\\	声[こえ]はすれども 姿[すがた]は 見[み]えず。
\\	笛吹けど踊らず。	
\\	笛吹[ふえふ]けど 踊[おど]らず。
\\	待てど暮らせど彼女から返事が来なかった。	
\\	待[ま]てど 暮[く]らせど 彼女[かのじょ]から 返事[へんじ]が 来[こ]なかった。
\\	何が起ころうとも驚かない。	
\\	何[なに]が 起[お]ころうとも 驚[おどろ]かない。
\\	お金がないので海外旅行をしたくともできない。	
\\	お 金[かね]がないので 海外[かいがい] 旅行[りょこう]をしたくともできない。
\\	私には小説などはどうにも書けない。	
\\	私[わたし]には 小説[しょうせつ]などはどうにも 書[か]けない。
\\	この作文は間違いがひどくてどうにも直しようがない。	
\\	この 作文[さくぶん]は 間違[まちが]いがひどくてどうにも 直[なお]しようがない。
\\	最近は携帯がないとどうにも仕事にならない。	
\\	最近[さいきん]は 携帯[けいたい]がないとどうにも 仕事[しごと]にならない。
\\	この書類は印刷が悪くてどうにも読めない。	
\\	この 書類[しょるい]は 印刷[いんさつ]が 悪[わる]くてどうにも 読[よ]めない。
\\	あの人はあまりにも内気でどうにも好きになれない。	
\\	あの 人[ひと]はあまりにも 内気[うちき]でどうにも 好[す]きになれない。
\\	人の悪口をかげで言うなどということはどうにも許せない。	
\\	人[ひと]の 悪口[わるぐち]をかげで 言[い]うなどということはどうにも 許[ゆる]せない。
\\	会社経営で破綻寸前まで来るとどうにも救いようがない。	
\\	会社[かいしゃ] 経営[けいえい]で 破綻[はたん] 寸前[すんぜん]まで 来[く]るとどうにも 救[すく]いようがない。
\\	これだけストレスが溜まると、どうにも解消しようがない。	
\\	これだけストレスが 溜[た]まると、どうにも 解消[かいしょう]しようがない。
\\	バブル経済崩壊後の日本の経済の低迷ぶりはひどいものだけれど、どうにもならない。	
\\	バブル 経済[けいざい] 崩壊[ほうかい] 後[ご]の 日本[にほん]の 経済[けいざい]の 低迷[ていめい]ぶりはひどいものだけれど、どうにもならない。
\\	いろいろお礼の言葉を書こうと思っていたのですが、どうにも言葉になりません。今まで本当にありがとうございました。	
\\	いろいろお 礼[れい]の 言葉[ことば]を 書[か]こうと 思[おも]っていたのですが、どうにも 言葉[ことば]になりません。 今[いま]まで 本当[ほんとう]にありがとうございました。
\\	磁石の
\\	極(または
\\	極)同士は反発する。	
\\	磁石[じしゃく]の 
\\	極[きょく](または 
\\	極[きょく]) 同士[どうし]は 反発[はんぱつ]する。
\\	高校時代の友達同士で海外旅行に行くことになった。	
\\	高校[こうこう] 時代[じだい]の 友達[ともだち] 同士[どうし]で 海外[かいがい] 旅行[りょこう]に 行[い]くことになった。
\\	決勝戦はアルゼンチン選手同士の対決になった。	
\\	決勝[けっしょう] 戦[せん]はアルゼンチン 選手[せんしゅ] 同士[どうし]の 対決[たいけつ]になった。
\\	このケーブルを使うと、パソコン同士を直接つないで高速ファイル転送をすることができる。	
\\	このケーブルを 使[つか]うと、パソコン 同士[どうし]を 直接[ちょくせつ]つないで 高速[こうそく]ファイル 転送[てんそう]をすることができる。
\\	たばこを吸う者同士は、何となく相手に親しみを感じやすい。	
\\	たばこを 吸[す]う 者[もの] 同士[どうし]は、 何[なん]となく 相手[あいて]に 親[した]しみを 感[かん]じやすい。
\\	面白いことに、隣り合った国同士は仲が悪いことが多い。	
\\	面白[おもしろ]いことに、 隣[とな]り 合[あ]った 国[くに] 同士[どうし]は 仲[なか]が 悪[わる]いことが 多[おお]い。
\\	茨城沖で貨物船同士が衝突、2人死亡4人行方不明。	
\\	茨城[いばらき] 沖[おき]で 貨物[かもつ] 船[せん] 同士[どうし]が 衝突[しょうとつ]、 2人[ふたり] 死亡[しぼう] 4人[よにん] 行方[ゆくえ] 不明[ふめい]。
\\	これは分子同士が衝突して起こる現象だ。	
\\	これは 分子[ぶんし] 同士[どうし]が 衝突[しょうとつ]して 起[お]こる 現象[げんしょう]だ。
\\	男同士で旅行に行ってもつまらない。	
\\	男[おとこ] 同士[どうし]で 旅行[りょこう]に 行[い]ってもつまらない。
\\	韓国では1997年に法律が変わり同姓同士でも自由に結婚できるようになった。	
\\	韓国[かんこく]では 
\\	年[ねん]に 法律[ほうりつ]が 変[か]わり 同姓[どうせい] 同士[どうし]でも 自由[じゆう]に 結婚[けっこん]できるようになった。
\\	この国ではいとこ同士の結婚は法的に認められている。	
\\	この 国[くに]ではいとこ 同士[どうし]の 結婚[けっこん]は 法的[ほうてき]に 認[みと]められている。
\\	何も知らない者同士が話し合ってもいい答えは出てこない。	
\\	何[なに]も 知[し]らない 者[もの] 同士[どうし]が 話し合[はなしあ]ってもいい 答[こた]えは 出[で]てこない。
\\	日本ではバレンタインデーに女の子が男の子にチョコレートを渡して愛を告白するのが一般的になっていますが、アメリカでは夫婦や恋人同士はもちろんのこと、友達や家族同士でもこの日、愛を言葉にします。	
\\	日本[にほん]ではバレンタインデーに 女の子[おんなのこ]が 男の子[おとこのこ]にチョコレートを 渡[わた]して 愛[あい]を 告白[こくはく]するのが 一般[いっぱん] 的[てき]になっていますが、アメリカでは 夫婦[ふうふ]や 恋人[こいびと] 同士[どうし]はもちろんのこと、 友達[ともだち]や 家族[かぞく] 同士[どうし]でもこの 日[ひ]、 愛[あい]を 言葉[ことば]にします。
\\	このタンパク質は細胞同士を結びつけるセメントのような役割をする。	
\\	この タンパク質[たんぱくしつ]は 細胞[さいぼう] 同士[どうし]を 結[むす]びつけるセメントのような 役割[やくわり]をする。
\\	モンスター同士を戦わせるコンピュータゲームが間もなく発売される。	
\\	モンスター 同士[どうし]を 戦[たたか]わせるコンピュータゲームが 間[ま]もなく 発売[はつばい]される。
\\	学生が立ち上がった。	
\\	学生[がくせい]が 立ち上[たちあ]がった。
\\	機動隊が学生に襲いかかった。	
\\	機動[きどう] 隊[たい]が 学生[がくせい]に 襲[おそ]いかかった。
\\	ヘリコプターが飛び立った。	
\\	ヘリコプターが 飛び立[とびた]った。
\\	3台の車が衝突した。	
\\	台[だい]の 車[くるま]が 衝突[しょうとつ]した。
\\	5人の友達で映画に行った。	
\\	人[にん]の 友達[ともだち]で 映画[えいが]に 行[い]った。
\\	友達同士5人で映画に行った。	
\\	友達[ともだち] 同士[どうし] 
\\	人[にん]で 映画[えいが]に 行[い]った。
\\	絵里子はよく旅行する。先月も友達同士で九州に行った。	
\\	絵里子[えりこ]はよく 旅行[りょこう]する。 先月[せんげつ]も 友達[ともだち] 同士[どうし]で 九州[きゅうしゅう]に 行[い]った。
\\	男と一緒に旅行に行ってもつまらない。	
\\	男[おとこ]と 一緒[いっしょ]に 旅行[りょこう]に 行[い]ってもつまらない。
\\	彼は朝起きるが早いか、シャワーを浴びた。	
\\	彼[かれ]は 朝[あさ] 起[お]きるが 早[はや]いか、シャワーを 浴[あ]びた。
\\	彼女は日本に着くが早いか友人に電話をかけた。	
\\	彼女[かのじょ]は 日本[にほん]に 着[つ]くが 早[はや]いか 友人[ゆうじん]に 電話[でんわ]をかけた。
\\	彼は私がいるのを見るが早いか、部屋を出て行ってしまった。	
\\	彼[かれ]は 私[わたし]がいるのを 見[み]るが 早[はや]いか、 部屋[へや]を 出[で]て 行[い]ってしまった。
\\	猫はねずみを捕まえるが早いか、食べ始めた。	
\\	猫[ねこ]はねずみを 捕[つか]まえるが 早[はや]いか、 食[た]べ 始[はじ]めた。
\\	電車はジョンが乗るが早いか、動き始めた。	
\\	電車[でんしゃ]はジョンが 乗[の]るが 早[はや]いか、 動[うご]き 始[はじ]めた。
\\	ディーンはきれいな女の子を見かけるが早いか、近寄って行って話しかけた。	
\\	ディーンはきれいな 女の子[おんなのこ]を 見[み]かけるが 早[はや]いか、 近寄[ちかよ]って 行[い]って 話[はな]しかけた。
\\	子供たちはパンが焼けるが早いか、全部食べてしまった。	
\\	子供[こども]たちはパンが 焼[や]けるが 早[はや]いか、 全部[ぜんぶ] 食[た]べてしまった。
\\	ひったくりは金を奪うが早いか、自転車で逃げた。	
\\	ひったくりは 金[かね]を 奪[うば]うが 早[はや]いか、 自転車[じてんしゃ]で 逃[に]げた。
\\	彼は家に帰るが早いかビールを飲んだ。	
\\	彼[かれ]は 家[いえ]に 帰[かえ]るが 早[はや]いかビールを 飲[の]んだ。
\\	会社が会社ならもう少しましな給料を出してくれるのだろうが。	
\\	会社[かいしゃ]が 会社[かいしゃ]ならもう 少[すこ]しましな 給料[きゅうりょう]を 出[だ]してくれるのだろうが。
\\	大学が大学ならもっといい教育が受けられて、いい就職もできたのだろう。	
\\	大学[だいがく]が 大学[だいがく]ならもっといい 教育[きょういく]が 受[う]けられて、いい 就職[しゅうしょく]もできたのだろう。
\\	友人が友人なら、役に立つ助言をしてくれたかもしれません。	
\\	友人[ゆうじん]が 友人[ゆうじん]なら、 役に立[やくにた]つ 助言[じょげん]をしてくれたかもしれません。
\\	季節が季節なら、ここはすばらしい観光地なんですが、今が寒すぎて観光には向いていないんですよ。	
\\	季節[きせつ]が 季節[きせつ]なら、ここはすばらしい 観光[かんこう] 地[ち]なんですが、 今[いま]が 寒[さむ]すぎて 観光[かんこう]には 向[む]いていないんですよ。
\\	時代が時代なら、そんな服装も悪くはないのかもしれないが、今ではおよそ古臭い。	
\\	時代[じだい]が 時代[じだい]なら、そんな 服装[ふくそう]も 悪[わる]くはないのかもしれないが、 今[いま]ではおよそ 古臭[ふるくさ]い。
\\	世が世なら、こんな情けない生活はしなくても済んだのに。	
\\	世[よ]が 世[よ]なら、こんな 情[なさ]けない 生活[せいかつ]はしなくても 済[す]んだのに。
\\	親が親なら、子供の子供だ。	
\\	親[おや]が 親[おや]なら、 子供[こども]の 子供[こども]だ。
\\	医者が医者なら、看護師も看護師だ。	
\\	医者[いしゃ]が 医者[いしゃ]なら、 看護[かんご] 師[し]も 看護[かんご] 師[し]だ。
\\	人生は旅のごとし。	
\\	人生[じんせい]は 旅[たび]のごとし。
\\	その無謀な行為は火中に飛び込むがごとし。	
\\	その 無謀[むぼう]な 行為[こうい]は 火中[かちゅう]に 飛び込[とびこ]むがごとし。
\\	彼女はモナリザのごとくいつも微笑んでいる。	
\\	彼女[かのじょ]はモナリザのごとくいつも 微笑[ほほえ]んでいる。
\\	氏は超人のごとき人だ。	
\\	氏[し]は 超人[ちょうじん]のごとき 人[ひと]だ。
\\	彼女は僕のことを全然覚えていないかのごとき様子だった。	
\\	彼女[かのじょ]は 僕[ぼく]のことを 全然[ぜんぜん] 覚[おぼ]えていないかのごとき 様子[ようす]だった。
\\	光陰矢のごとし。	
\\	光陰[こういん] 矢[や]のごとし。
\\	人生は航海のごとし。	
\\	人生[じんせい]は 航海[こうかい]のごとし。
\\	東京人は万人走っているかのごとし。	
\\	東京[とうきょう] 人[じん]は 万[ばん] 人[にん] 走[はし]っているかのごとし。
\\	日本の経済は、上述のごとく、今後しばらくは回復しない見込みである。	
\\	日本[にほん]の 経済[けいざい]は、 上述[じょうじゅつ]のごとく、 今後[こんご]しばらくは 回復[かいふく]しない 見込[みこ]みである。
\\	彼女は女王のごとく芸能界に君臨している。	
\\	彼女[かのじょ]は 女王[じょおう]のごとく 芸能[げいのう] 界[かい]に 君臨[くんりん]している。
\\	課長は当然のごとく女性の職員にお茶くみをさせている。	
\\	課長[かちょう]は 当然[とうぜん]のごとく 女性[じょせい]の 職員[しょくいん]にお 茶[ちゃ]くみをさせている。
\\	夢のごとき新婚生活の後に悲劇が訪れた。	
\\	夢[ゆめ]のごとき 新婚[しんこん] 生活[せいかつ]の 後[あと]に 悲劇[ひげき]が 訪[おとず]れた。
\\	彼は平社員なのに社長であるかのごとき言動をするのでみんなに嫌われている。	
\\	彼[かれ]は 平[ひら] 社員[しゃいん]なのに 社長[しゃちょう]であるかのごとき 言動[げんどう]をするのでみんなに 嫌[きら]われている。
\\	皮膚の色だけで差別するごとき行為は絶対にあってはならぬ。	
\\	皮膚[ひふ]の 色[いろ]だけで 差別[さべつ]するごとき 行為[こうい]は 絶対[ぜったい]にあってはならぬ。
\\	子供が大人のような犯罪を犯すがごとき風潮は何が原因なのだろうか。	
\\	子供[こども]が 大人[おとな]のような 犯罪[はんざい]を 犯[おか]すがごとき 風潮[ふうちょう]は 何[なに]が 原因[げんいん]なのだろうか。
\\	携帯電話は便利な反面、わずらわしいこともある。	
\\	携帯[けいたい] 電話[でんわ]は 便利[べんり]な 反面[はんめん]、わずらわしいこともある。
\\	この国は個人の意思が尊重される反面、グループとしてのまとまりが弱い。	
\\	この 国[くに]は 個人[こじん]の 意思[いし]が 尊重[そんちょう]される 反面[はんめん]、グループとしてのまとまりが 弱[よわ]い。
\\	この仕事は面白くて収入もいいが、その反面、忙しくて自分の時間がなかなか持てないという問題もある。	
\\	この 仕事[しごと]は 面白[おもしろ]くて 収入[しゅうにゅう]もいいが、その 反面[はんめん]、 忙[いそが]しくて 自分[じぶん]の 時間[じかん]がなかなか 持[も]てないという 問題[もんだい]もある。
\\	ここは海産物が安く、また種類も多くて本土では食べられない魚もあります。反面、野菜や果物は高いです。	
\\	ここは 海産物[かいさんぶつ]が 安[やす]く、また 種類[しゅるい]も 多[おお]くて 本土[ほんど]では 食[た]べられない 魚[さかな]もあります。 反面[はんめん]、 野菜[やさい]や 果物[くだもの]は 高[たか]いです。
\\	この仕事は厳しい反面、勉強になることも多い。	
\\	この 仕事[しごと]は 厳[きび]しい 反面[はんめん]、 勉強[べんきょう]になることも 多[おお]い。
\\	外食は手軽な反面、栄養が偏る恐れがある。	
\\	外食[がいしょく]は 手軽[てがる]な 反面[はんめん]、 栄養[えいよう]が 偏[かたよ]る 恐[おそ]れがある。
\\	この地域は商業や医療、文化施設などの利用が便利な反面、自然環境はあまり望めない。	
\\	この 地域[ちいき]は 商業[しょうぎょう]や 医療[いりょう]、 文化[ぶんか] 施設[しせつ]などの 利用[りよう]が 便利[べんり]な 反面[はんめん]、 自然[しぜん] 環境[かんきょう]はあまり 望[のぞ]めない。
\\	アメリカンタイプのホテルは建物も比較的新しく部屋が機能的な反面、雰囲気に欠けるところがあります。	
\\	アメリカンタイプのホテルは 建物[たてもの]も 比較的[ひかくてき] 新[あたら]しく 部屋[へや]が 機能[きのう] 的[てき]な 反面[はんめん]、 雰囲気[ふんいき]に 欠[か]けるところがあります。
\\	ポリエステルのカーテンは縮みにくく、熱にも強いが、反面、静電気が発生して汚れがつきやすい。	
\\	ポリエステルのカーテンは 縮[ちぢ]みにくく、 熱[ねつ]にも 強[つよ]いが、 反面[はんめん]、 静電気[せいでんき]が 発生[はっせい]して 汚[よご]れがつきやすい。
\\	この磁石は優れた磁気特性を持ち、機会強度が大きい。反面、錆びやすく、高温での使用には適さない。	
\\	この 磁石[じしゃく]は 優[すぐ]れた 磁気[じき] 特性[とくせい]を 持[も]ち、 機会[きかい] 強度[きょうど]が 大[おお]きい。 反面[はんめん]、 錆[さ]びやすく、 高温[こうおん]での 使用[しよう]には 適[てき]さない。
\\	理想主義はある面ではすばらしいが、その反面、それが行きすぎていろいろな問題が起こることもある。	
\\	理想[りそう] 主義[しゅぎ]はある 面[めん]ではすばらしいが、その 反面[はんめん]、それが 行[い]きすぎていろいろな 問題[もんだい]が 起[お]こることもある。
\\	インターネットの恋人探しサービスを使う場合、出合いという観点から見ると、こちらから積極的にアプローチをする方法はかなり効果的です。その反面、個人情報の流出という危険性もあります。	
\\	インターネットの 恋人[こいびと] 探[さが]しサービスを 使[つか]う 場合[ばあい]、 出合[であ]いという 観点[かんてん]から 見[み]ると、こちらから 積極[せっきょく] 的[てき]にアプローチをする 方法[ほうほう]はかなり 効果[こうか] 的[てき]です。その 反面[はんめん]、 個人[こじん] 情報[じょうほう]の 流出[りゅうしゅつ]という 危険[きけん] 性[せい]もあります。
\\	一般に、日本の学生は大学に入る前によく勉強するのに対して、アメリカの学生は大学に入ってからよく勉強するようだ。	
\\	一般[いっぱん]に、 日本[にほん]の 学生[がくせい]は 大学[だいがく]に 入[はい]る 前[まえ]によく 勉強[べんきょう]するのに 対[たい]して、アメリカの 学生[がくせい]は 大学[だいがく]に 入[はい]ってからよく 勉強[べんきょう]するようだ。
\\	我が社はテレビの売れ行きが伸びている反面、オーディオ製品の売れ行きが下っている。	
\\	我[わ]が 社[しゃ]はテレビの 売れ行[うれゆ]きが 伸[の]びている 反面[はんめん]、オーディオ 製品[せいひん]の 売れ行[うれゆ]きが 下[くだ]っている。
\\	今年の冬も暖冬だと言われていたが、はたして暖冬だった。	
\\	今年[ことし]の 冬[ふゆ]も 暖冬[だんとう]だと 言[い]われていたが、はたして 暖冬[だんとう]だった。
\\	はたして日本の教育は改革されるのだろうか。	
\\	はたして 日本[にほん]の 教育[きょういく]は 改革[かいかく]されるのだろうか。
\\	はたして女性の地位が改善されるとしても時間がかかるだろう。	
\\	はたして 女性[じょせい]の 地位[ちい]が 改善[かいぜん]されるとしても 時間[じかん]がかかるだろう。
\\	東西ドイツ統一はいずれは起こると思っていたが、はたして予想通り起きた。	
\\	東西ドイツ[とうざいどいつ] 統一[とういつ]はいずれは 起[お]こると 思[おも]っていたが、はたして 予想[よそう] 通[どお]り 起[お]きた。
\\	二人の仲は周知の事実で、いつかはゴールインすると思っていたら、はたしてその通りになった。	
\\	二人[ふたり]の 仲[なか]は 周知[しゅうち]の 事実[じじつ]で、いつかはゴールインすると 思[おも]っていたら、はたしてその 通[とお]りになった。
\\	そこのレストランは魚料理で有名だったが、はたして思った通りのおいしい料理が出てきた。	
\\	そこのレストランは 魚[さかな] 料理[りょうり]で 有名[ゆうめい]だったが、はたして 思[おも]った 通[とお]りのおいしい 料理[りょうり]が 出[で]てきた。
\\	イスラエルとパレスチナ両国の間にはたして真の平和が訪れるのだろうか。	
\\	イスラエルとパレスチナ 両国[りょうこく]の 間[あいだ]にはたして 真[しん]の 平和[へいわ]が 訪[おとず]れるのだろうか。
\\	はたして日本経済は近い将来に復興するだろうか。	
\\	はたして 日本[にほん] 経済[けいざい]は 近[ちか]い 将来[しょうらい]に 復興[ふっこう]するだろうか。
\\	21世紀ははたしてどの国の世紀になるのだろうか。	
\\	世紀[せいき]ははたしてどの 国[くに]の 世紀[せいき]になるのだろうか。
\\	はたして政治家たちの言うことがうわべだけのことであるとしても、多少の真実はあると考えたい。	
\\	はたして 政治[せいじ] 家[か]たちの 言[い]うことがうわべだけのことであるとしても、 多少[たしょう]の 真実[しんじつ]はあると 考[かんが]えたい。
\\	彼は来ないと思ったけど、はたして現れなかった。	
\\	彼[かれ]は 来[こ]ないと 思[おも]ったけど、はたして 現[あらわ]れなかった。
\\	車の調子が悪いなと思っていたら、案の定、会社に行く途中でエンストを起こしてしまった。	
\\	車[くるま]の 調子[ちょうし]が 悪[わる]いなと 思[おも]っていたら、 案の定[あんのじょう]、 会社[かいしゃ]に 行[い]く 途中[とちゅう]でエンストを 起[お]こしてしまった。
\\	イラクへの攻撃はアメリカの孤立を招き、ひいてはイスラム社会全体との戦争に発展していく恐れがある。	
\\	イラクへの 攻撃[こうげき]はアメリカの 孤立[こりつ]を 招[まね]き、ひいてはイスラム 社会[しゃかい] 全体[ぜんたい]との 戦争[せんそう]に 発展[はってん]していく 恐[おそ]れがある。
\\	彼は同僚にライバル意識、ひいては殺意すら抱いていた。	
\\	彼[かれ]は 同僚[どうりょう]にライバル 意識[いしき]、ひいては 殺意[さつい]すら 抱[いだ]いていた。
\\	東京の産業構造を適切に転換していくことが、東京の産業を活性化させ、ひいては日本の経済力の強化につながっていく。	
\\	東京[とうきょう]の 産業[さんぎょう] 構造[こうぞう]を 適切[てきせつ]に 転換[てんかん]していくことが、 東京[とうきょう]の 産業[さんぎょう]を 活性[かっせい] 化[か]させ、ひいては 日本[にほん]の 経済[けいざい] 力[りょく]の 強化[きょうか]につながっていく。
\\	「誰もが自分自身の能力を最大限に活かして、自ら積極的に新しいことにチャレンジして、よりよく生きていく」という考えを、日本国内、ひいては世界に波及させたいと思っています。	
\\	誰[だれ]もが 自分[じぶん] 自身[じしん]の 能力[のうりょく]を 最大限[さいだいげん]に 活[い]かして、 自[みずか]ら 積極[せっきょく] 的[てき]に 新[あたら]しいことにチャレンジして、よりよく 生[い]きていく」という 考[かんが]えを、 日本[にほん] 国内[こくない]、ひいては 世界[せかい]に 波及[はきゅう]させたいと 思[おも]っています。
\\	世間では、企業経営から、教育、ひいては犯罪まで、話題はインターネットが独占しているという感じです。	
\\	世間[せけん]では、 企業[きぎょう] 経営[けいえい]から、 教育[きょういく]、ひいては 犯罪[はんざい]まで、 話題[わだい]はインターネットが 独占[どくせん]しているという 感[かん]じです。
\\	道を歩いていて、奇妙な姿の人が向こうから歩いてくると、不安な気持ちになる。これは人間、ひいては動物の防衛機能の一つだろう。	
\\	道[みち]を 歩[ある]いていて、 奇妙[きみょう]な 姿[すがた]の 人[ひと]が 向[む]こうから 歩[ある]いてくると、 不安[ふあん]な 気持[きも]ちになる。これは 人間[にんげん]、ひいては 動物[どうぶつ]の 防衛[ぼうえい] 機能[きのう]の 一[ひと]つだろう。
\\	栄養価が高い旬のものをおいしく食べることは、ひいては病気を予防する。	
\\	栄養[えいよう] 価[か]が 高[たか]い 旬[しゅん]のものをおいしく 食[た]べることは、ひいては 病気[びょうき]を 予防[よぼう]する。
\\	マスメディアが個人の名誉やプライバシーを不当に侵害する状況が続けば、ひいては報道に対して権力が介入する口実を与えることになる。	
\\	マスメディアが 個人[こじん]の 名誉[めいよ]やプライバシーを 不当[ふとう]に 侵害[しんがい]する 状況[じょうきょう]が 続[つづ]けば、ひいては 報道[ほうどう]に 対[たい]して 権力[けんりょく]が 介入[かいにゅう]する 口実[こうじつ]を 与[あた]えることになる。
\\	会員は37名だったが、さらに12名加わって49名になった。	
\\	会員[かいいん]は 
\\	名[めい]だったが、さらに 
\\	名[めい] 加[くわ]わって 
\\	名[めい]になった。
\\	二人はすでにビールを十本も飲んでいたが、さらにウイスキーも二瓶空けた。	
\\	二人[ふたり]はすでにビールを 十本[じっぽん]も 飲[の]んでいたが、さらにウイスキーも 二瓶[ふたびん] 空[あ]けた。
\\	うちの課には一人として英語を話せる者がいない。	
\\	うちの 課[か]には 一人[ひとり]として 英語[えいご]を 話[はな]せる 者[もの]がいない。
\\	この資料室には一冊とし役に立つ本がない。	
\\	この 資料[しりょう] 室[しつ]には 一冊[いっさつ]とし 役に立[やくにた]つ 本[ほん]がない。
\\	私が困っていた時、一人として私を助けてくれる者がいなかった。	
\\	私[わたし]が 困[こま]っていた 時[とき]、一人[ひとり]として 私[わたし]を 助[たす]けてくれる 者[もの]がいなかった。
\\	この宗教は、この世の中には一つとして自分のものはないと教えている。	
\\	この 宗教[しゅうきょう]は、この 世の中[よのなか]には 一[ひと]つとして 自分[じぶん]のものはないと 教[おし]えている。
\\	剣道の審判は一秒として気を抜くことは許されない。	
\\	剣道[けんどう]の 審判[しんぱん]は 一秒[いちびょう]として 気[き]を 抜[ぬ]くことは 許[ゆる]されない。
\\	単身赴任中、一日として家族のことを思わない日はなかった。	
\\	単身[たんしん] 赴任[ふにん] 中[ちゅう]、一日[いちにち]として 家族[かぞく]のことを 思[おも]わない 日[ひ]はなかった。
\\	彼女の演奏は一音としておろそかに弾かれることはなく、どの曲も聞く者に深い感銘を与えないではおかない。	
\\	彼女[かのじょ]の 演奏[えんそう]は 一音[いちおん]としておろそかに 弾[ひ]かれることはなく、どの 曲[きょく]も 聞[き]く 者[もの]に 深[ふか]い 感銘[かんめい]を 与[あた]えないではおかない。
\\	一部のイスラム世界では、女性は何人たりとも髪の毛を他人に見せることは許されない。	
\\	一部[いちぶ]のイスラム 世界[せかい]では、 女性[じょせい]は 何[なん] 人[ぴと]たりとも 髪の毛[かみのけ]を 他人[たにん]に 見[み]せることは 許[ゆる]されない。
\\	ひとつ日本の小説の英訳をやってみよう。	
\\	ひとつ 日本[にほん]の 小説[しょうせつ]の 英訳[えいやく]をやってみよう。
\\	ひとつゴルフでもやったらどうですか。	
\\	ひとつゴルフでもやったらどうですか。
\\	ひとつ相談にのって下さい。	
\\	ひとつ 相談[そうだん]にのって 下[くだ]さい。
\\	ひとつインターネットでもやってみよう。	
\\	ひとつインターネットでもやってみよう。
\\	ひとつ東大をうけて見たいと思います。	
\\	ひとつ 東大[とうだい]をうけて 見[み]たいと 思[おも]います。
\\	ダンス?」という面白い映画を見たので、ひとつ社交ダンスでもやってみようかという気になった。	
\\	ダンス?」という 面白[おもしろ]い 映画[えいが]を 見[み]たので、ひとつ 社交[しゃこう]ダンスでもやってみようかという 気[き]になった。
\\	ひとつ映画でも見に行きませんか。	
\\	ひとつ 映画[えいが]でも 見[み]に 行[い]きませんか。
\\	ひとつ株でもやってみませんか。	
\\	ひとつ 株[かぶ]でもやってみませんか。
\\	ー君の言うことなど聞きたくないよ。 ーまあ、そう言わないで、ひとつ聞いて下さいよ。	
\\	ー 君[きみ]の 言[い]うことなど 聞[き]きたくないよ。 ーまあ、そう 言[い]わないで、ひとつ 聞[き]いて 下[くだ]さいよ。
\\	そこのところを、ひとつよろしくお願いいたします。	
\\	そこのところを、ひとつよろしくお 願[ねが]いいたします。
\\	じゃあ、この仕事をひとつやってもらいましょうか。	
\\	じゃあ、この 仕事[しごと]をひとつやってもらいましょうか。
\\	彼女は手紙ひとつ満足に書けない。	
\\	彼女[かのじょ]は 手紙[てがみ]ひとつ 満足[まんぞく]に 書[か]けない。
\\	息子はもう大人なのに、挨拶ひとつできない。	
\\	息子[むすこ]はもう 大人[おとな]なのに、 挨拶[あいさつ]ひとつできない。
\\	このごろの若い子のファッションなんかひとつもいいとは思わない。	
\\	このごろの 若[わか]い 子[こ]のファッションなんかひとつもいいとは 思[おも]わない。
\\	このマンガ、ひとつも面白くないね。	
\\	このマンガ、ひとつも 面白[おもしろ]くないね。
\\	彼は何ひとつ取り柄がない。	
\\	彼[かれ]は 何[なに]ひとつ 取り柄[とりえ]がない。
\\	彼女は美貌ひとつで昇進していった。	
\\	彼女[かのじょ]は 美貌[びぼう]ひとつで 昇進[しょうしん]していった。
\\	母は女手ひとつで子供五人を育てた。	
\\	母[はは]は 女手[おんなで]ひとつで 子供[こども] 五人[ごにん]を 育[そだ]てた。
\\	大学教育の目的は、一つには幅広い教養と専門的な知識・能力を授けること、一つには社会に貢献する指導者を育成することだ。	
\\	大学[だいがく] 教育[きょういく]の 目的[もくてき]は、 一[ひと]つには 幅広[はばひろ]い 教養[きょうよう]と 専門[せんもん] 的[てき]な 知識[ちしき]・ 能力[のうりょく]を 授[さず]けること、 一[ひと]つには 社会[しゃかい]に 貢献[こうけん]する 指導[しどう] 者[しゃ]を 育成[いくせい]することだ。
\\	彼が外国語を好んで勉強しているのは、一つには、異文化学習が好きだからだ。	
\\	彼[かれ]が 外国[がいこく] 語[ご]を 好[この]んで 勉強[べんきょう]しているのは、 一[ひと]つには、 異[い] 文化[ぶんか] 学習[がくしゅう]が 好[す]きだからだ。
\\	彼がよく家族と旅行に出かけるのは、一つには自分のため、一つには家族のためだ。	
\\	彼[かれ]がよく 家族[かぞく]と 旅行[りょこう]に 出[で]かけるのは、 一[ひと]つには 自分[じぶん]のため、 一[ひと]つには 家族[かぞく]のためだ。
\\	日本人が大人まで漫画を読んでいるのは、一つには、大人が十分味わえるような漫画があるからだ。	
\\	日本人[にほんじん]が 大人[おとな]まで 漫画[まんが]を 読[よ]んでいるのは、 一[ひと]つには、 大人[おとな]が 十分[じゅうぶん] 味[あじ]わえるような 漫画[まんが]があるからだ。
\\	彼がその子を好きな理由はいろいろとあるが、一つには彼女の話し方がいつも生き生きとしているためだ。	
\\	彼[かれ]がその 子[こ]を 好[す]きな 理由[りゆう]はいろいろとあるが、 一[ひと]つには 彼女[かのじょ]の 話し方[はなしかた]がいつも 生き生[いきい]きとしているためだ。
\\	私がアメリカに来た理由は、一つには、そこに学問の自由があると思ったからだ。	
\\	私[わたし]がアメリカに 来[き]た 理由[りゆう]は、 一[ひと]つには、そこに 学問[がくもん]の 自由[じゆう]があると 思[おも]ったからだ。
\\	政治家が嫌われる理由は、一つには、選挙の時には魅力的な公約を言うのに、当選するとそれを実行しないからだ。	
\\	政治[せいじ] 家[か]が 嫌[きら]われる 理由[りゆう]は、 一[ひと]つには、 選挙[せんきょ]の 時[とき]には 魅力[みりょく] 的[てき]な 公約[こうやく]を 言[い]うのに、 当選[とうせん]するとそれを 実行[じっこう]しないからだ。
\\	日本に来た外国人が体験するカルチャーショックにはいろいろあると思うが、一つには日本人が彼らをガイジン、つまりソトの人と呼ぶということがある。	
\\	日本[にほん]に 来[き]た 外国[がいこく] 人[じん]が 体験[たいけん]するカルチャーショックにはいろいろあると 思[おも]うが、 一[ひと]つには 日本人[にほんじん]が 彼[かれ]らをガイジン、つまりソトの 人[ひと]と 呼[よ]ぶということがある。
\\	「10年日記」をやってみたいとよく思うが、まだ始めていない。それは一つには、前の年の方がいい年だったらいやだからだ。	
\\	年[ねん] 日記[にっき]」をやってみたいとよく 思[おも]うが、まだ 始[はじ]めていない。それは 一[ひと]つには、 前[まえ]の 年[とし]の 方[ほう]がいい 年[とし]だったらいやだからだ。
\\	アパートに決めたのは一つには学生寮に比べて居心地がいいということもあります。	
\\	アパートに 決[き]めたのは 一[ひと]つには 学生[がくせい] 寮[りょう]に 比[くら]べて 居心地[いごこち]がいいということもあります。
\\	この国で別の大学への転学が容易に認められる理由の一つには、厳しい大学入試がないという事情がある。	
\\	この 国[くに]で 別[べつ]の 大学[だいがく]への 転学[てんがく]が 容易[ようい]に 認[みと]められる 理由[りゆう]の 一[ひと]つには、 厳[きび]しい 大学[だいがく] 入試[にゅうし]がないという 事情[じじょう]がある。
\\	その日本語の学生は、いろいろ問題があるのだが、一つにはまだ語彙が不足しているということがある。	
\\	その 日本語[にほんご]の 学生[がくせい]は、いろいろ 問題[もんだい]があるのだが、 一[ひと]つにはまだ 語彙[ごい]が 不足[ふそく]しているということがある。
\\	会議の日取りを一応来週の水曜日と決めておきましょう。	
\\	会議[かいぎ]の 日取[ひど]りを 一応[いちおう] 来週[らいしゅう]の 水曜日[すいようび]と 決[き]めておきましょう。
\\	私は一応トムにその仕事を担当させることにした。	
\\	私[わたし]は 一応[いちおう]トムにその 仕事[しごと]を 担当[たんとう]させることにした。
\\	この問題についてはまだ議論の余地はあるとは思うが、一応ここでけりをつけたいと思う。	
\\	この 問題[もんだい]についてはまだ 議論[ぎろん]の 余地[よち]はあるとは 思[おも]うが、 一応[いちおう]ここでけりをつけたいと 思[おも]う。
\\	一応大学を出ておいた方が都合がいいと思い、大学に進学したんです。	
\\	一応[いちおう] 大学[だいがく]を 出[で]ておいた 方[ほう]が 都合[つごう]がいいと 思[おも]い、 大学[だいがく]に 進学[しんがく]したんです。
\\	ーコンピュータを使っていらっしゃいますか。 ーはあ、まあ、一応は使っていますが。	ー
\\	ーコンピュータを 使[つか]っていらっしゃいますか。 ーはあ、まあ、 一応[いちおう]は 使[つか]っていますが。
\\	ーあなたは情報技術の研究者ですか。 ーまあ、一応そういうことです。	
\\	ーあなたは 情報[じょうほう] 技術[ぎじゅつ]の 研究[けんきゅう] 者[しゃ]ですか。 ーまあ、 一応[いちおう]そういうことです。
\\	ーこのお金どうするの? ー俺が一応銀行に入れておくよ。	
\\	ーこのお 金[かね]どうするの? ー 俺[おれ]が 一応[いちおう] 銀行[ぎんこう]に 入[い]れておくよ。
\\	二人は兄弟と言うだけに顔はまあ、一応似ている。しかし気性はかなり違う。	
\\	二人[ふたり]は 兄弟[きょうだい]と 言[い]うだけに 顔[かお]はまあ、 一応[いちおう] 似[に]ている。しかし 気性[きしょう]はかなり 違[ちが]う。
\\	すみません。これでも一応大学教師なんです。	
\\	すみません。これでも 一応[いちおう] 大学[だいがく] 教師[きょうし]なんです。
\\	滅多に買わない店だけど、名前くらいは一応覚えてもらっているから安くしてくれるかもしれない。	
\\	滅多[めった]に 買[か]わない 店[みせ]だけど、 名前[なまえ]くらいは 一応[いちおう] 覚[おぼ]えてもらっているから 安[やす]くしてくれるかもしれない。
\\	ー何を飲もうか。 ー取りあえずビールで始めようよ。	
\\	ー 何[なに]を 飲[の]もうか。 ー 取[と]りあえずビールで 始[はじ]めようよ。
\\	私がこの会社に採用されるかどうかは、面接の結果いかんだ。	
\\	私[わたし]がこの 会社[かいしゃ]に 採用[さいよう]されるかどうかは、 面接[めんせつ]の 結果[けっか]いかんだ。
\\	今年の新車の売れ行きいかんでこの会社の将来が決まりそうだ。	
\\	今年[ことし]の 新車[しんしゃ]の 売れ行[うれゆ]きいかんでこの 会社[かいしゃ]の 将来[しょうらい]が 決[き]まりそうだ。
\\	このテストの結果いかんによっては今年卒業できないかもしれない。	
\\	このテストの 結果[けっか]いかんによっては 今年[ことし] 卒業[そつぎょう]できないかもしれない。
\\	納入した学費は、理由のいかんに関わらず返却しません。	
\\	納入[のうにゅう]した 学費[がくひ]は、 理由[りゆう]のいかんに 関[かか]わらず 返却[へんきゃく]しません。
\\	燃料電池自動車が普及するかどうかは価格とパワーいかんである。	
\\	燃料[ねんりょう] 電池[でんち] 自動車[じどうしゃ]が 普及[ふきゅう]するかどうかは 価格[かかく]とパワーいかんである。
\\	子供の失敗をやる気に結びつけ成長への糧にできるかどうかは、その時の親の態度いかんだ。	
\\	子供[こども]の 失敗[しっぱい]をやる 気[き]に 結[むす]びつけ 成長[せいちょう]への 糧[かて]にできるかどうかは、その 時[とき]の 親[おや]の 態度[たいど]いかんだ。
\\	国や組織の将来はリーダーの知的能力いかんで決まると言える。	
\\	国[くに]や 組織[そしき]の 将来[しょうらい]はリーダーの 知的[ちてき] 能力[のうりょく]いかんで 決[き]まると 言[い]える。
\\	実験の結果いかんによっては、大発見になる可能性もある。	
\\	実験[じっけん]の 結果[けっか]いかんによっては、 大発見[だいはっけん]になる 可能[かのう] 性[せい]もある。
\\	生活排水などの汚水は、その処理方法いかんによっては、周辺環境の悪化を招くこととなります。	
\\	生活[せいかつ] 排水[はいすい]などの 汚水[おすい]は、その 処理[しょり] 方法[ほうほう]いかんによっては、 周辺[しゅうへん] 環境[かんきょう]の 悪化[あっか]を 招[まね]くこととなります。
\\	賞味期限のある食品類は開封・未開封のいかんに関わらず返品対象外となります。	
\\	賞味[しょうみ] 期限[きげん]のある 食品[しょくひん] 類[るい]は 開封[かいふう]・ 未[み] 開封[かいふう]のいかんに 関[かか]わらず 返品[へんぴん] 対象[たいしょう] 外[がい]となります。
\\	このコンテンツの複製は、方法、媒体のいかんを問わず禁止されている。	
\\	このコンテンツの 複製[ふくせい]は、 方法[ほうほう]、 媒体[ばいたい]のいかんを 問[と]わず 禁止[きんし]されている。
\\	事態がここまで悪化しては、いかんともしがたい。	
\\	事態[じたい]がここまで 悪化[あっか]しては、いかんともしがたい。
\\	生物社会学とはいかなる学問分野か。	
\\	生物[せいぶつ] 社会[しゃかい] 学[がく]とはいかなる 学問[がくもん] 分野[ぶんや]か。
\\	彼はいかなる忠告にも耳を貸さなかった。	
\\	彼[かれ]はいかなる 忠告[ちゅうこく]にも 耳[みみ]を 貸[か]さなかった。
\\	いかなる理論でも穴はあるものだ。	
\\	いかなる 理論[りろん]でも 穴[あな]はあるものだ。
\\	父はいかなる事態になっても常に沈着だった。	
\\	父[ちち]はいかなる 事態[じたい]になっても 常[つね]に 沈着[ちんちゃく]だった。
\\	重要なことは、いかなる結果が出ようと自分で決めたい。	
\\	重要[じゅうよう]なことは、いかなる 結果[けっか]が 出[で]ようと 自分[じぶん]で 決[き]めたい。
\\	日本語運用能力はいかなる基準で測るべきか。	
\\	日本語[にほんご] 運用[うんよう] 能力[のうりょく]はいかなる 基準[きじゅん]で 測[はか]るべきか。
\\	厳しい父はいかなる口実も許してくれなかった。	
\\	厳[きび]しい 父[ちち]はいかなる 口実[こうじつ]も 許[ゆる]してくれなかった。
\\	いかなる文化にも宗教は存在する。	
\\	いかなる 文化[ぶんか]にも 宗教[しゅうきょう]は 存在[そんざい]する。
\\	あの男はいかなる失敗にも負けず、不死鳥のように蘇る。	
\\	あの 男[おとこ]はいかなる 失敗[しっぱい]にも 負[ま]けず、 不死鳥[ふしちょう]のように 蘇[よみがえ]る。
\\	いかなる外国語学習でもその国の文化が好きになってくるものだ。	
\\	いかなる 外国[がいこく] 語[ご] 学習[がくしゅう]でもその 国[くに]の 文化[ぶんか]が 好[す]きになってくるものだ。
\\	スポーツはいかなるスポーツであれ、肉体だけではなく精神をも鍛えてくれる。	
\\	スポーツはいかなるスポーツであれ、 肉体[にくたい]だけではなく 精神[せいしん]をも 鍛[きた]えてくれる。
\\	いかなるアイディアでも、みんなで討議する価値がある。	
\\	いかなるアイディアでも、みんなで 討議[とうぎ]する 価値[かち]がある。
\\	いかなる社会を見ても女性の地位は徐々に向上している。	
\\	いかなる 社会[しゃかい]を 見[み]ても 女性[じょせい]の 地位[ちい]は 徐々[じょじょ]に 向上[こうじょう]している。
\\	私の研究がいかなる賞を得ようとも、その賞が私個人のものではなく、プロジェクトチームのものだ。	
\\	私[わたし]の 研究[けんきゅう]がいかなる 賞[しょう]を 得[え]ようとも、その 賞[しょう]が 私[わたし] 個人[こじん]のものではなく、プロジェクトチームのものだ。
\\	オリンピック・エリアにおいては、いかなる種類のデモも、いかなる種類の政治的、宗教的もしくは人種的な宣伝活動も認められない。	
\\	オリンピック・エリアにおいては、いかなる 種類[しゅるい]のデモも、いかなる 種類[しゅるい]の 政治[せいじ] 的[てき]、 宗教[しゅうきょう] 的[てき]もしくは 人種[じんしゅ] 的[てき]な 宣伝[せんでん] 活動[かつどう]も 認[みと]められない。
\\	いかに日本語が難しいかは学習者の母語による。	
\\	いかに 日本語[にほんご]が 難[むずか]しいかは 学習[がくしゅう] 者[しゃ]の 母語[ぼご]による。
\\	いかに運動しても長生きできるとは限らない。	
\\	いかに 運動[うんどう]しても 長生[ながい]きできるとは 限[かぎ]らない。
\\	いかに一生懸命働こうとも一戸建ての家は買えない。	
\\	いかに 一生懸命[いっしょうけんめい] 働[はたら]こうとも 一戸建[いっこだ]ての 家[いえ]は 買[か]えない。
\\	人生はいかに短いことか。	
\\	人生[じんせい]はいかに 短[みじか]いことか。
\\	いかに死ぬかはいかに生きるかと同じように大事だ。	
\\	いかに 死[し]ぬかはいかに 生[い]きるかと 同[おな]じように 大事[だいじ]だ。
\\	ヒトゲノムの解読が病気の根本的な治療にいかに必要かが分かってきた。	
\\	ヒトゲノムの 解読[かいどく]が 病気[びょうき]の 根本[こんぽん] 的[てき]な 治療[ちりょう]にいかに 必要[ひつよう]かが 分[わ]かってきた。
\\	クローン人間がいかに創られるのかは私には理解できないが、その倫理性についての論争には強い興味を覚える。	
\\	クローン 人間[にんげん]がいかに 創[つく]られるのかは 私[わたし]には 理解[りかい]できないが、その 倫理[りんり] 性[せい]についての 論争[ろんそう]には 強[つよ]い 興味[きょうみ]を 覚[おぼ]える。
\\	食べた物がいかにおいしかったかを言葉で適確に説明することは至難のわざだ。	
\\	食[た]べた 物[もの]がいかにおいしかったかを 言葉[ことば]で 適確[てきかく]に 説明[せつめい]することは 至難[しなん]のわざだ。
\\	いかに努力しても自分の才能のない領域だったら限界があるだろう。	
\\	いかに 努力[どりょく]しても 自分[じぶん]の 才能[さいのう]のない 領域[りょういき]だったら 限界[げんかい]があるだろう。
\\	いかに環境問題の重要性を強調しようとも強調しすぎることはない。	
\\	いかに 環境[かんきょう] 問題[もんだい]の 重要[じゅうよう] 性[せい]を 強調[きょうちょう]しようとも 強調[きょうちょう]しすぎることはない。
\\	いかに天才でも、運が悪ければいい仕事はできない。	
\\	いかに 天才[てんさい]でも、 運[うん]が 悪[わる]ければいい 仕事[しごと]はできない。
\\	日本の経済はいかに脆いことか!	
\\	日本[にほん]の 経済[けいざい]はいかに 脆[もろ]いことか!
\\	日本語がいかに勉強を続けたい。	
\\	日本語[にほんご]がいかに 勉強[べんきょう]を 続[つづ]けたい。
\\	東京の3
\\	のマンションはいくらぐらいですか。	
\\	東京[とうきょう]の 
\\	のマンションはいくらぐらいですか。
\\	いくら働いてもお金が貯まらない。	
\\	いくら 働[はたら]いてもお 金[かね]が 貯[た]まらない。
\\	夏休みだから時間はいくらでもある。	
\\	夏休[なつやす]みだから 時間[じかん]はいくらでもある。
\\	ミルクはもういくらも残っていない。	
\\	ミルクはもういくらも 残[のこ]っていない。
\\	今日はいくらか気分がいい。	
\\	今日[きょう]はいくらか 気分[きぶん]がいい。
\\	その車はいくらでしたか。	
\\	その 車[くるま]はいくらでしたか。
\\	ー寄付はいくらぐらいすればいいでしょうか。 ーいくらでも結構です。	
\\	ー 寄付[きふ]はいくらぐらいすればいいでしょうか。 ーいくらでも 結構[けっこう]です。
\\	いくら時間とお金をかけてもそのプロジェクトは終わるまい。	
\\	いくら 時間[じかん]とお 金[かね]をかけてもそのプロジェクトは 終[お]わるまい。
\\	会議はいくら長くても一時間ぐらいだろう。	
\\	会議[かいぎ]はいくら 長[なが]くても一 時間[じかん]ぐらいだろう。
\\	いくら頭脳明晰でもよき指導者になれるとは限らない。	
\\	いくら 頭脳[ずのう] 明晰[めいせき]でもよき 指導[しどう] 者[しゃ]になれるとは 限[かぎ]らない。
\\	お酒はいくらでもありますから、どうぞたくさん飲んで下さい。	
\\	お 酒[さけ]はいくらでもありますから、どうぞたくさん 飲[の]んで 下[くだ]さい。
\\	私はまだ経験が浅いので報酬はいくらでもいいです。	
\\	私[わたし]はまだ 経験[けいけん]が 浅[あさ]いので 報酬[ほうしゅう]はいくらでもいいです。
\\	寄付はいくらでも大丈夫です。	
\\	寄付[きふ]はいくらでも 大丈夫[だいじょうぶ]です。
\\	僕みたいに毎日泳いでいる人間はいくらもいない。	
\\	僕[ぼく]みたいに 毎日[まいにち] 泳[およ]いでいる 人間[にんげん]はいくらもいない。
\\	我々のように、いくらも休みがない人間だと、どんなに山登りをしたくても時間がないのである。	
\\	我々[われわれ]のように、いくらも 休[やす]みがない 人間[にんげん]だと、どんなに 山登[やまのぼ]りをしたくても 時間[じかん]がないのである。
\\	いくらか日本語の面白さが分かってきた。	
\\	いくらか 日本語[にほんご]の 面白[おもしろ]さが 分[わ]かってきた。
\\	危険なものはほかにいくらもある。	
\\	危険[きけん]なものはほかにいくらもある。
\\	こんな仕事をしたい人はいくらでもいる。	
\\	こんな 仕事[しごと]をしたい 人[ひと]はいくらでもいる。
\\	この話は二年も前に話し合ったじゃないか。なんで、いまさら同じ話を持ち出すんだ。	
\\	この 話[はなし]は二 年[ねん]も 前[まえ]に 話し合[はなしあ]ったじゃないか。なんで、いまさら 同[おな]じ 話[はなし]を 持ち出[もちだ]すんだ。
\\	いまさら契約を解消するわけにはいかない。	
\\	いまさら 契約[けいやく]を 解消[かいしょう]するわけにはいかない。
\\	いまさら後悔しても仕方がない。	
\\	いまさら 後悔[こうかい]しても 仕方[しかた]がない。
\\	この問題をいまさら議論したところで無意味でしょう。	
\\	この 問題[もんだい]をいまさら 議論[ぎろん]したところで 無意味[むいみ]でしょう。
\\	あなたはあの男とはきっぱり別れたはずなのに、いまさらなんで合おうとしているの。	
\\	あなたはあの 男[おとこ]とはきっぱり 別[わか]れたはずなのに、いまさらなんで 合[あ]おうとしているの。
\\	いまさらですが、「明けましておめでとうございます」。	
\\	いまさらですが、
\\	明[あ]けましておめでとうございます」。
\\	今日の講師は皆さんよくご存じの方ですから、いまさらご紹介する必要はないとは思いますが……。	
\\	今日[きょう]の 講師[こうし]は 皆[みな]さんよくご 存[ぞん]じの 方[かた]ですから、いまさらご 紹介[しょうかい]する 必要[ひつよう]はないとは 思[おも]いますが……。
\\	会議の日程はすでに全員に電子メールで通知を出したのだから、いまさら変更することはできない。	
\\	会議[かいぎ]の 日程[にってい]はすでに 全員[ぜんいん]に 電子[でんし]メールで 通知[つうち]を 出[だ]したのだから、いまさら 変更[へんこう]することはできない。
\\	喜んでプロジェクトに加わりたいと言った手前、いまさら断れないよ。	
\\	喜[よろこ]んでプロジェクトに 加[くわ]わりたいと 言[い]った 手前[てまえ]、いまさら 断[ことわ]れないよ。
\\	がんがここまで進むと、いまさら手術をしても治らないと思います。	
\\	がんがここまで 進[すす]むと、いまさら 手術[しゅじゅつ]をしても 治[なお]らないと 思[おも]います。
\\	子供時代に過ごした村を訪ねたいと思っていますが、いまさら行ったところで、古いものは何も残っていないでしょうね。	
\\	子供[こども] 時代[じだい]に 過[す]ごした 村[むら]を 訪[たず]ねたいと 思[おも]っていますが、いまさら 行[い]ったところで、 古[ふる]いものは 何[なに]も 残[のこ]っていないでしょうね。
\\	母親を失って10年、自分も親になってみて、いまさらながら、親不孝だったことを反省しています。	
\\	母親[ははおや]を 失[うしな]って 
\\	年[ねん]、 自分[じぶん]も 親[おや]になってみて、いまさらながら、 親不孝[おやふこう]だったことを 反省[はんせい]しています。
\\	もうその電車には乗れない。	
\\	もうその 電車[でんしゃ]には 乗[の]れない。
\\	いまさら考えてみても、どうしようもない。	
\\	いまさら 考[かんが]えてみても、どうしようもない。
\\	今になってやろうと思っても、無駄だよ。	
\\	今[いま]になってやろうと 思[おも]っても、 無駄[むだ]だよ。
\\	何をいまさらしようと言うのかね。	
\\	何[なに]をいまさらしようと 言[い]うのかね。
\\	いまさらどうしようもない。	
\\	いまさらどうしようもない。
\\	いまさら無駄だよ。	
\\	いまさら 無駄[むだ]だよ。
\\	何をいまさら。	
\\	何[なに]をいまさら。
\\	物価は上がる一方だ。	
\\	物価[ぶっか]は 上[あ]がる 一方[いっぽう]だ。
\\	久子は私にものを頼む一方で、何もしてくれたことがない。	
\\	久子[ひさこ]は 私[わたし]にものを 頼[たの]む 一方[いっぽう]で、 何[なに]もしてくれたことがない。
\\	電話や電子メールを利用した詐欺は巧妙化する一方だ。	
\\	電話[でんわ]や 電子[でんし]メールを 利用[りよう]した 詐欺[さぎ]は 巧妙[こうみょう] 化[か]する 一方[いっぽう]だ。
\\	この市は急増した人口にインフラ整備が追いつかず、犯罪も増える一方だ。	
\\	この 市[し]は 急増[きゅうぞう]した 人口[じんこう]にインフラ 整備[せいび]が 追[お]いつかず、 犯罪[はんざい]も 増[ふ]える 一方[いっぽう]だ。
\\	ビジネスのグローバル化が進む今日、英語によるコミュニケーション能力の重要性は増す一方だ。	
\\	ビジネスのグローバル 化[か]が 進[すす]む 今日[こんにち]、 英語[えいご]によるコミュニケーション 能力[のうりょく]の 重要[じゅうよう] 性[せい]は 増[ま]す 一方[いっぽう]だ。
\\	サッカー熱は高まる一方だが、少年サッカーでひざの故障を起こすケースも増えている。	
\\	サッカー 熱[ねつ]は 高[たか]まる 一方[いっぽう]だが、 少年[しょうねん]サッカーでひざの 故障[こしょう]を 起[お]こすケースも 増[ふ]えている。
\\	深刻化する一方の医師不足に早急に何らかの対策が必要だ。	
\\	深刻[しんこく] 化[か]する 一方[いっぽう]の 医師[いし] 不足[ぶそく]に 早急[そうきゅう]に 何[なん]らかの 対策[たいさく]が 必要[ひつよう]だ。
\\	昨日のホッケーでは我々のチームは攻められる一方だった。	
\\	昨日[きのう]のホッケーでは 我々[われわれ]のチームは 攻[せ]められる 一方[いっぽう]だった。
\\	私たちはいつも子供や犬のことでお隣に迷惑をかける一方なので、申し訳なく思う。	
\\	私[わたし]たちはいつも 子供[こども]や 犬[いぬ]のことでお 隣[となり]に 迷惑[めいわく]をかける 一方[いっぽう]なので、 申し訳[もうしわけ]なく 思[おも]う。
\\	あとはレポートを書くばかりだ。	
\\	あとはレポートを 書[か]くばかりだ。
\\	稔は笑うばかりで何も説明してくれない。	
\\	稔[みのる]は 笑[わら]うばかりで 何[なに]も 説明[せつめい]してくれない。
\\	物価はますます上がっている。	
\\	物価[ぶっか]はますます 上[あ]がっている。
\\	電話や電子メールを利用した詐欺はますます巧妙化している。	
\\	電話[でんわ]や 電子[でんし]メールを 利用[りよう]した 詐欺[さぎ]はますます 巧妙[こうみょう] 化[か]している。
\\	片方が何かをしてもらうだけの関係は長続きしないと思う。	
\\	片方[かたほう]が 何[なに]かをしてもらうだけの 関係[かんけい]は 長続[ながつづ]きしないと 思[おも]う。
\\	物価は上がる一方で、みんな困っている。	
\\	物価[ぶっか]は 上[あ]がる 一方[いっぽう]で、みんな 困[こま]っている。
\\	我々は化石燃料による環境汚染の問題に対応する一方で、新しいエネルギーの開発に力を入れなければならない。	
\\	我々[われわれ]は 化石[かせき] 燃料[ねんりょう]による 環境[かんきょう] 汚染[おせん]の 問題[もんだい]に 対応[たいおう]する 一方[いっぽう]で、 新[あたら]しいエネルギーの 開発[かいはつ]に 力[ちから]を 入[い]れなければならない。
\\	企業は
\\	投資には熱心な一方、セキュリティ意識はまだ低いところが多い。	
\\	企業[きぎょう]は 
\\	投資[とうし]には 熱心[ねっしん]な 一方[いっぽう]、セキュリティ 意識[いしき]はまだ 低[ひく]いところが 多[おお]い。
\\	軍事予算拡大の一方、教育や社会福祉関連は大きく切り捨てられた。	
\\	軍事[ぐんじ] 予算[よさん] 拡大[かくだい]の 一方[いっぽう]、 教育[きょういく]や 社会[しゃかい] 福祉[ふくし] 関連[かんれん]は 大[おお]きく 切り捨[きりす]てられた。
\\	日本は工作機械の分野では依然優位に立っている。しかしその一方で、最終製品の生産でシェアを下げつつある。	
\\	日本[にほん]は 工作[こうさく] 機械[きかい]の 分野[ぶんや]では 依然[いぜん] 優位[ゆうい]に 立[た]っている。しかしその 一方[いっぽう]で、 最終[さいしゅう] 製品[せいひん]の 生産[せいさん]でシェアを 下[さ]げつつある。
\\	テレビはソファに深々ともたれ「後傾姿勢で受動的」に見る。一方、インターネットは椅子に浅く腰を掛け「前傾姿勢で能動的」にやる。	
\\	テレビはソファに 深々[ふかぶか]ともたれ
\\	後傾[こうけい] 姿勢[しせい]で 受動[じゅどう] 的[てき]」に 見[み]る。 一方[いっぽう]、インターネットは 椅子[いす]に 浅[あさ]く 腰[こし]を 掛[か]け
\\	前傾[ぜんけい] 姿勢[しせい]で 能動[のうどう] 的[てき]」にやる。
\\	我が社は新製品を開発する一方で、多岐にわたる現在のモデルを少数に統合する計画だ。	
\\	我[わ]が 社[しゃ]は 新[しん] 製品[せいひん]を 開発[かいはつ]する 一方[いっぽう]で、 多岐[たき]にわたる 現在[げんざい]のモデルを 少数[しょうすう]に 統合[とうごう]する 計画[けいかく]だ。
\\	政府は各種の予算を削減している一方、無駄遣いも目につく。	
\\	政府[せいふ]は 各種[かくしゅ]の 予算[よさん]を 削減[さくげん]している 一方[いっぽう]、 無駄遣[むだづか]いも 目[め]につく。
\\	就職できない若者がたくさんいる一方で、国民の生活に欠かせない仕事の分野で人手が不足しているそうだ。	
\\	就職[しゅうしょく]できない 若者[わかもの]がたくさんいる 一方[いっぽう]で、 国民[こくみん]の 生活[せいかつ]に 欠[か]かせない 仕事[しごと]の 分野[ぶんや]で 人手[ひとで]が 不足[ふそく]しているそうだ。
\\	花粉症は完治が難しい一方、予防はしやすい。	
\\	花粉[かふん] 症[しょう]は 完治[かんち]が 難[むずか]しい 一方[いっぽう]、 予防[よぼう]はしやすい。
\\	外国の食品がどこでも手に入るということは便利な一方で、世界で食材不足を起こす危険性もある。	
\\	外国[がいこく]の 食品[しょくひん]がどこでも 手[て]に 入[はい]るということは 便利[べんり]な 一方[いっぽう]で、 世界[せかい]で 食材[しょくざい] 不足[ぶそく]を 起[お]こす 危険[きけん] 性[せい]もある。
\\	日本では高齢人口が増加する一方、年少人口の減少が進んでいる。	
\\	日本[にほん]では 高齢[こうれい] 人口[じんこう]が 増加[ぞうか]する 一方[いっぽう]、 年少[ねんしょう] 人口[じんこう]の 減少[げんしょう]が 進[すす]んでいる。
\\	今日では、海外に出かける日本人の数は、日本に来る外国人観光客の数の3倍もあります。しかしその一方で、海外から日本への観光客も、1970年以降、確実に増加しています。	
\\	今日[こんにち]では、 海外[かいがい]に 出[で]かける 日本人[にほんじん]の 数[かず]は、 日本[にほん]に 来[く]る 外国[がいこく] 人[じん] 観光[かんこう] 客[きゃく]の 数[かず]の 
\\	倍[ばい]もあります。しかしその 一方[いっぽう]で、 海外[かいがい]から 日本[にほん]への 観光[かんこう] 客[きゃく]も、 
\\	年[ねん] 以降[いこう]、 確実[かくじつ]に 増加[ぞうか]しています。
\\	情報というものは簡単に複製し共有することができる。一方、物理的なモノの場合は、完全に複製するのは容易ではない。	
\\	情報[じょうほう]というものは 簡単[かんたん]に 複製[ふくせい]し 共有[きょうゆう]することができる。 一方[いっぽう]、 物理[ぶつり] 的[てき]なモノの 場合[ばあい]は、 完全[かんぜん]に 複製[ふくせい]するのは 容易[ようい]ではない。
\\	大統領の支持率が主任以来最低となった。	
\\	大統領[だいとうりょう]の 支持[しじ] 率[りつ]が 主任[しゅにん] 以来[いらい] 最低[さいてい]となった。
\\	社が創業以来始めてホームページのデザインを変更した。	
\\	社[しゃ]が 創業[そうぎょう] 以来[いらい] 始[はじ]めてホームページのデザインを 変更[へんこう]した。
\\	肝臓を患って以来、酒をやめている。	
\\	肝臓[かんぞう]を 患[わずら]って 以来[いらい]、 酒[さけ]をやめている。
\\	上村に会うのは10年前の高校のクラス会以来だ。	
\\	上村[うえむら]に 会[あ]うのは 
\\	年[ねん] 前[まえ]の 高校[こうこう]のクラス 会[かい] 以来[いらい]だ。
\\	二年ほど前に重いスーツケースを持ち上げようとして腰を痛めてしまった。それ以来、重いものは持てなくなった。	
\\	二 年[ねん]ほど 前[まえ]に 重[おも]いスーツケースを 持ち上[もちあ]げようとして 腰[こし]を 痛[いた]めてしまった。それ 以来[いらい]、 重[おも]いものは 持[も]てなくなった。
\\	国の教育ローンの利用者が制度発足以来初めて300万人を超えた。	
\\	国[くに]の 教育[きょういく]ローンの 利用[りよう] 者[しゃ]が 制度[せいど] 発足[ほっそく] 以来[いらい] 初[はじ]めて 
\\	万[まん] 人[にん]を 超[こ]えた。
\\	当ホテルは、明治42年創業以来90年の歴史を持っています。	
\\	当[とう]ホテルは、 明治[めいじ] 
\\	年[ねん] 創業[そうぎょう] 以来[いらい] 
\\	年[ねん]の 歴史[れきし]を 持[も]っています。
\\	琵琶湖の水位が今月6日、プラス31センチとなり、昨年7月21日以来、8か月ぶりに基準値に戻った。	
\\	琵琶湖[びわこ]の 水位[すいい]が 今月[こんげつ] 6日[むいか]、プラス 31センチとなり、 昨年[さくねん] 
\\	月[がつ] 
\\	日[にち] 以来[いらい]、 8か月[はっかげつ]ぶりに 基準[きじゅん] 値[ち]に 戻[もど]った。
\\	半導体市場は95年以来の二桁成長が予測されている。	
\\	半導体[はんどうたい] 市場[しじょう]は 
\\	年[ねん] 以来[いらい]の 二桁[ふたけた] 成長[せいちょう]が 予測[よそく]されている。
\\	ジョンソン氏が社長に就任して以来、この会社は次々に大きな改革を行った。	
\\	ジョンソン 氏[し]が 社長[しゃちょう]に 就任[しゅうにん]して 以来[いらい]、この 会社[かいしゃ]は 次々[つぎつぎ]に 大[おお]きな 改革[かいかく]を 行[おこな]った。
\\	生の舞台を見て以来、すっかり歌舞伎のファンになってしまった。	
\\	生[なま]の 舞台[ぶたい]を 見[み]て 以来[いらい]、すっかり 歌舞伎[かぶき]のファンになってしまった。
\\	この空港始まって以来の大事故となった。	
\\	この 空港[くうこう] 始[はじ]まって 以来[いらい]の 大[だい] 事故[じこ]となった。
\\	先週清美と大げんかをした。それ以来、彼女とは口をきいていない。	
\\	先週[せんしゅう] 清美[きよみ]と 大[おお]げんかをした。それ 以来[いらい]、 彼女[かのじょ]とは 口[くち]をきいていない。
\\	植村には10年前に高校のクラス会で会った。彼に会うのはそれ以来だ。	
\\	植村[うえむら]には 
\\	年[ねん] 前[まえ]に 高校[こうこう]のクラス 会[かい]で 会[あ]った。 彼[かれ]に 会[あ]うのはそれ 以来[いらい]だ。
\\	植村には10年前に高校のクラス会で会った、今回はそれ以来の再会だ。	
\\	植村[うえむら]には 
\\	年[ねん] 前[まえ]に 高校[こうこう]のクラス 会[かい]で 会[あ]った、 今回[こんかい]はそれ 以来[いらい]の 再会[さいかい]だ。
\\	今朝から頭痛がする。	
\\	今朝[けさ]から 頭痛[ずつう]がする。
\\	1時からずっとここで仕事をしていた。	
\\	時[じ]からずっとここで 仕事[しごと]をしていた。
\\	昨日から何も食べていない。	
\\	昨日[きのう]から 何[なに]も 食[た]べていない。
\\	訪問は来月以降にして下さい。	
\\	訪問[ほうもん]は 来月[らいげつ] 以降[いこう]にして 下[くだ]さい。
\\	5月以降三度雨が降った。	
\\	月[がつ] 以降[いこう]三 度[ど] 雨[あめ]が 降[ふ]った。
\\	6月以降はほとんど毎週出張があった。	
\\	月[がつ] 以降[いこう]はほとんど 毎週[まいしゅう] 出張[しゅっちょう]があった。
\\	6月1日以来雨が降っていない。	
\\	月[がつ] 1日[ついたち] 以来[いらい] 雨[あめ]が 降[ふ]っていない。
\\	6月1日以降雨が降っていない。	
\\	月[がつ] 1日[ついたち] 以降[いこう] 雨[あめ]が 降[ふ]っていない。
\\	第二次大戦以来日本人の戦争観は変わった。	
\\	第[だい] 二次[にじ] 大戦[たいせん] 以来[いらい] 日本人[にほんじん]の 戦争[せんそう] 観[かん]は 変[か]わった。
\\	第二次大戦以降日本人の戦争観は変わった。	
\\	第[だい] 二次[にじ] 大戦[たいせん] 以降[いこう] 日本人[にほんじん]の 戦争[せんそう] 観[かん]は 変[か]わった。
\\	以後注意するように。	
\\	以後[いご] 注意[ちゅうい]するように。
\\	知らずに女性言葉を使って恥をかいた。以後同じ間違いはしなくなった。	
\\	知[し]らずに 女性[じょせい] 言葉[ことば]を 使[つか]って 恥[はじ]をかいた。 以後[いご] 同[おな]じ 間違[まちが]いはしなくなった。
\\	飛行機事故の原因は一切分からない。	
\\	飛行機[ひこうき] 事故[じこ]の 原因[げんいん]は 一切[いっさい] 分[わ]からない。
\\	この製品には遺伝子組み換え大豆は一切使用していません。	
\\	この 製品[せいひん]には 遺伝子[いでんし] 組み換[くみか]え 大豆[だいず]は 一切[いっさい] 使用[しよう]していません。
\\	盗難に関しては当社は一切責任を負いません。	
\\	盗難[とうなん]に 関[かん]しては 当社[とうしゃ]は 一切[いっさい] 責任[せきにん]を 負[お]いません。
\\	私はその汚職事件とは一切関係ない。	
\\	私[わたし]はその 汚職[おしょく] 事件[じけん]とは 一切[いっさい] 関係[かんけい]ない。
\\	謝金は一切いただくわけにはまいりません。	
\\	謝金[しゃきん]は 一切[いっさい]いただくわけにはまいりません。
\\	私は有害物質を一切含んでいない洗剤を使うことにしている。	
\\	私[わたし]は 有害[ゆうがい] 物質[ぶっしつ]を 一切[いっさい] 含[ふく]んでいない 洗剤[せんざい]を 使[つか]うことにしている。
\\	その教授の講義は全然面白くなかった。	
\\	その 教授[きょうじゅ]の 講義[こうぎ]は 全然[ぜんぜん] 面白[おもしろ]くなかった。
\\	僕の大学のキャンパスは全くきれいじゃない。	
\\	僕[ぼく]の 大学[だいがく]のキャンパスは 全[まった]くきれいじゃない。
\\	僕はあなたの意見に全く賛成です。	
\\	僕[ぼく]はあなたの 意見[いけん]に 全[まった]く 賛成[さんせい]です。
\\	父は全く健康だ。	
\\	父[ちち]は 全[まった]く 健康[けんこう]だ。
\\	初めてよしもとばななの小説を読んだよ。もっと難しいと思ってたけど、全然面白い。	
\\	初[はじ]めてよしもとばななの 小説[しょうせつ]を 読[よ]んだよ。もっと 難[むずか]しいと 思[おも]ってたけど、 全然[ぜんぜん] 面白[おもしろ]い。
\\	両国は一切外交関係を断ち切った。	
\\	両国[りょうこく]は 一切[いっさい] 外交[がいこう] 関係[かんけい]を 断ち切[たちき]った。
\\	全くの初心者も大歓迎です。	
\\	全[まった]くの 初心者[しょしんしゃ]も 大[だい] 歓迎[かんげい]です。
\\	それは全くの誤解だと思います。	
\\	それは 全[まった]くの 誤解[ごかい]だと 思[おも]います。
\\	この小説はいったん読み始めたらやめられない。	
\\	この 小説[しょうせつ]はいったん 読[よ]み 始[はじ]めたらやめられない。
\\	邦彦はいったやめたたばこをまた吸い出した。	
\\	邦彦[くにひこ]はいったやめたたばこをまた 吸い出[すいだ]した。
\\	開演までまだだいぶ時間があるので、いったんホテルに帰ってから出直そう。	
\\	開演[かいえん]までまだだいぶ 時間[じかん]があるので、いったんホテルに 帰[かえ]ってから 出直[でなお]そう。
\\	この大学は、いったん入学してしまえば、そんなに勉強しなくても必ず卒業できる。	
\\	この 大学[だいがく]は、いったん 入学[にゅうがく]してしまえば、そんなに 勉強[べんきょう]しなくても 必[かなら]ず 卒業[そつぎょう]できる。
\\	この電子掲示板はいったん書き込むと消せないので、自分のメッセージを書く時に注意しないといけない。	
\\	この 電子[でんし] 掲示板[けいじばん]はいったん 書き込[かきこ]むと 消[け]せないので、 自分[じぶん]のメッセージを 書[か]く 時[とき]に 注意[ちゅうい]しないといけない。
\\	石田は、いったん約束した以上は必ず守る男だ。	
\\	石田[いしだ]は、いったん 約束[やくそく]した 以上[いじょう]は 必[かなら]ず 守[まも]る 男[おとこ]だ。
\\	いったんこじれた我々の関係は、容易には元に戻らなかった。	
\\	いったんこじれた 我々[われわれ]の 関係[かんけい]は、 容易[ようい]には 元[もと]に 戻[もど]らなかった。
\\	予算は厳しくなったので、無料健康相談サービスをいったん中止します。	
\\	予算[よさん]は 厳[きび]しくなったので、 無料[むりょう] 健康[けんこう] 相談[そうだん]サービスをいったん 中止[ちゅうし]します。
\\	新型ウイルスの被害はいったん終息したように見えた。	
\\	新型[しんがた]ウイルスの 被害[ひがい]はいったん 終息[しゅうそく]したように 見[み]えた。
\\	ひょっとすると違う原因かもしれないので、今飲んでいる薬をいったんやめてしばらく様子を見てみましょう。	
\\	ひょっとすると 違[ちが]う 原因[げんいん]かもしれないので、 今[いま] 飲[の]んでいる 薬[くすり]をいったんやめてしばらく 様子[ようす]を 見[み]てみましょう。
\\	今使っているプログラムをいったん終了して、再起動してみて下さい。	
\\	今[いま] 使[つか]っているプログラムをいったん 終了[しゅうりょう]して、 再[さい] 起動[きどう]してみて 下[くだ]さい。
\\	いったん日本へ帰って、親と相談してからアメリカで就職するかどうかを決めたい。	
\\	いったん 日本[にほん]へ 帰[かえ]って、 親[おや]と 相談[そうだん]してからアメリカで 就職[しゅうしょく]するかどうかを 決[き]めたい。
\\	私は韓国へ一度行ったことがある。	
\\	私[わたし]は 韓国[かんこく]へ一 度[ど] 行[おこな]ったことがある。
\\	一度医者に見てもらったらどうですか。	
\\	一度[いちど] 医者[いしゃ]に 見[み]てもらったらどうですか。
\\	一度私のうちにも遊びに来て下さい。	
\\	一度[いちど] 私[わたし]のうちにも 遊[あそ]びに 来[き]て 下[くだ]さい。
\\	言ってみれば、人生は航海だ。	
\\	言[い]ってみれば、 人生[じんせい]は 航海[こうかい]だ。
\\	彼女の家は、言ってみれば、お城のように作られている。	
\\	彼女[かのじょ]の 家[いえ]は、 言[い]ってみれば、お 城[しろ]のように 作[つく]られている。
\\	日本の文化には厳しいしきたりがあって、言ってみれば、それは宗教のようなものだ。	
\\	日本[にほん]の 文化[ぶんか]には 厳[きび]しいしきたりがあって、 言[い]ってみれば、それは 宗教[しゅうきょう]のようなものだ。
\\	結婚は言ってみれば人生の墓場だ。	
\\	結婚[けっこん]は 言[い]ってみれば 人生[じんせい]の 墓場[はかば]だ。
\\	医者と弁護士は、言ってみれば、神様のように振る舞う。	
\\	医者[いしゃ]と 弁護士[べんごし]は、 言[い]ってみれば、 神様[かみさま]のように 振る舞[ふるま]う。
\\	学生たちが学んだ多くの知識は、頭の中に放置されたままで、言ってみれば、死んだ状態になっているのである。	
\\	学生[がくせい]たちが 学[まな]んだ 多[おお]くの 知識[ちしき]は、 頭[あたま]の 中[なか]に 放置[ほうち]されたままで、 言[い]ってみれば、 死[し]んだ 状態[じょうたい]になっているのである。
\\	シリコンバレーも、結局は日本の、言ってみれば商店街なり地域作りの話と本質的には同じなんだと思いました。	
\\	シリコンバレーも、 結局[けっきょく]は 日本[にほん]の、 言[い]ってみれば 商店[しょうてん] 街[がい]なり 地域[ちいき] 作[づく]りの 話[はなし]と 本質[ほんしつ] 的[てき]には 同[おな]じなんだと 思[おも]いました。
\\	著作権とは、言ってみれば、創造者の権利を守る法律だ。	
\\	著作[ちょさく] 権[けん]とは、 言[い]ってみれば、 創造[そうぞう] 者[しゃ]の 権利[けんり]を 守[まも]る 法律[ほうりつ]だ。
\\	このブログは言ってみれば僕の絵日記のようなものだ。	
\\	このブログは 言[い]ってみれば 僕[ぼく]の 絵日記[えにっき]のようなものだ。
\\	すしが日本の代表的な料理であることは言うまでもない。	
\\	すしが 日本[にほん]の 代表[だいひょう] 的[てき]な 料理[りょうり]であることは 言[い]うまでもない。
\\	言うまでもなく、日本の社会は世界一の高齢化社会になっている。	
\\	言[い]うまでもなく、 日本[にほん]の 社会[しゃかい]は 世界一[せかいいち]の 高齢[こうれい] 化[か] 社会[しゃかい]になっている。
\\	少子化が日本の社会を変えていくことは言うまでもない。	
\\	少子化[しょうしか]が 日本[にほん]の 社会[しゃかい]を 変[か]えていくことは 言[い]うまでもない。
\\	「アジアの人々は貧しい」「貧しいのは可哀想」「開発援助しなくては」という固定観念が有害なのは言うまでもありません。	
\\	「アジアの 人々[ひとびと]は 貧[まず]しい」
\\	貧[まず]しいのは 可哀想[かわいそう]」
\\	開発[かいはつ] 援助[えんじょ]しなくては」という 固定[こてい] 観念[かんねん]が 有害[ゆうがい]なのは 言[い]うまでもありません。
\\	二人はパーティーで会って、すぐ意気投合しました。そのあと毎日のようにデートをしたのは言うまでもありません。	
\\	二人[ふたり]はパーティーで 会[あ]って、すぐ 意気投合[いきとうごう]しました。そのあと 毎日[まいにち]のようにデートをしたのは 言[い]うまでもありません。
\\	オープンネットワークコンピューティング環境では、日々増大するネットワークデータをいかに敏速にかつ効率よくバックアップするかが重要であることは言うまでもありません。	
\\	オープンネットワークコンピューティング 環境[かんきょう]では、 日々[ひび] 増大[ぞうだい]するネットワークデータをいかに 敏速[びんそく]にかつ 効率[こうりつ]よくバックアップするかが 重要[じゅうよう]であることは 言[い]うまでもありません。
\\	言うまでもありませんが、大学の成績の優秀な学生が社会で成功するとは限りません。	
\\	言[い]うまでもありませんが、 大学[だいがく]の 成績[せいせき]の 優秀[ゆうしゅう]な 学生[がくせい]が 社会[しゃかい]で 成功[せいこう]するとは 限[かぎ]りません。
\\	いまさら言うまでもなく、地球の温暖化は生態系に悪影響をもたらしている。	
\\	いまさら 言[い]うまでもなく、 地球[ちきゅう]の 温暖[おんだん] 化[か]は 生態[せいたい] 系[けい]に 悪影響[あくえいきょう]をもたらしている。
\\	菜穂子はとびきりの美人でもなかったが、誰とでもにこやかに話し、いつも溌剌としていた。言うまでもないことだが、彼女の回りには人が自然に集まり、職場のみんなにも好かれていた。	
\\	菜穂子[なほこ]はとびきりの 美人[びじん]でもなかったが、 誰[だれ]とでもにこやかに 話[はな]し、いつも 溌剌[はつらつ]としていた。 言[い]うまでもないことだが、 彼女[かのじょ]の 回[まわ]りには 人[ひと]が 自然[しぜん]に 集[あつ]まり、 職場[しょくば]のみんなにも 好[す]かれていた。
\\	麻薬は言うまでもなく、肉体と精神を滅ぼす。	
\\	麻薬[まやく]は 言[い]うまでもなく、 肉体[にくたい]と 精神[せいしん]を 滅[ほろ]ぼす。
\\	英語は言うまでもなく、国際語だ。	
\\	英語[えいご]は 言[い]うまでもなく、 国際[こくさい] 語[ご]だ。
\\	この本は、言わば私のバイブルだ。	
\\	この 本[ほん]は、 言[い]わば 私[わたし]のバイブルだ。
\\	この試験は言わばアメリカの大学に留学する際の
\\	に当たる。	
\\	この 試験[しけん]は 言[い]わばアメリカの 大学[だいがく]に 留学[りゅうがく]する 際[さい]の 
\\	に 当[あ]たる。
\\	ここまでは、言わばウォーミングアップのようなもので、ここからが本番だ。	
\\	ここまでは、 言[い]わばウォーミングアップのようなもので、ここからが 本番[ほんばん]だ。
\\	その判決は言わばこの国の良心を代弁したものだった。	
\\	その 判決[はんけつ]は 言[い]わばこの 国[くに]の 良心[りょうしん]を 代弁[だいべん]したものだった。
\\	言わば趣味で始めたことがいつの間にか大きなビジネスになってしまった。	
\\	言[い]わば 趣味[しゅみ]で 始[はじ]めたことがいつの 間[あいだ]にか 大[おお]きなビジネスになってしまった。
\\	当地区には、言わば幽霊人口である非合法居住者が7-10万人いますが、この人たちは補助金対象外となります。	
\\	当[とう] 地区[ちく]には、 言[い]わば 幽霊[ゆうれい] 人口[じんこう]である 非合法[ひごうほう] 居住[きょじゅう] 者[しゃ]が 
\\	万[まん] 人[にん]いますが、この 人[ひと]たちは 補助[ほじょ] 金[きん] 対象[たいしょう] 外[がい]となります。
\\	こうして見てみると、かつてはサダム・フセインも、ソ連やイランと戦ってきた、言わばアメリカの盟友だったわけである。	
\\	こうして 見[み]てみると、かつてはサダム・フセインも、 ソ連[それん]やイランと 戦[たたか]ってきた、 言[い]わばアメリカの 盟友[めいゆう]だったわけである。
\\	ファッション写真は、1930年代までは、言わば「新参者」の分野と見られていたらしい。	
\\	ファッション 写真[しゃしん]は、 
\\	年代[ねんだい]までは、 言[い]わば
\\	新参[しんざん] 者[しゃ]」の 分野[ぶんや]と 見[み]られていたらしい。
\\	この酵素は、正常状態では必要な時だけ活性化して細胞を増殖させるのだが、がん細胞の場合は常に活性化し、言わばアクセルを踏みっぱなしの状態で、無秩序に細胞を増殖させてしまうのだ。	
\\	この 酵素[こうそ]は、 正常[せいじょう] 状態[じょうたい]では 必要[ひつよう]な 時[とき]だけ 活性[かっせい] 化[か]して 細胞[さいぼう]を 増殖[ぞうしょく]させるのだが、がん 細胞[さいぼう]の 場合[ばあい]は 常[つね]に 活性[かっせい] 化[か]し、 言[い]わばアクセルを 踏[ふ]みっぱなしの 状態[じょうたい]で、 無秩序[むちつじょ]に 細胞[さいぼう]を 増殖[ぞうしょく]させてしまうのだ。
\\	伊藤さんはいわゆる「エリート」の大学を卒業した。	
\\	伊藤[いとう]さんはいわゆる「エリート」の 大学[だいがく]を 卒業[そつぎょう]した。
\\	いわゆるリストラのために失業した人は非常に多い。	
\\	いわゆるリストラのために 失業[しつぎょう]した 人[ひと]は 非常[ひじょう]に 多[おお]い。
\\	1986年から90年代初めにかけてのいわゆる「バブル経済」の時代には土地や株が高騰した。	
\\	年[ねん]から 
\\	年代[ねんだい] 初[はじ]めにかけてのいわゆる「バブル 経済[けいざい]」の 時代[じだい]には 土地[とち]や 株[かぶ]が 高騰[こうとう]した。
\\	21世紀には多くの国でいわゆる安楽死が合法化されていくだろう。	
\\	世紀[せいき]には 多[おお]くの 国[くに]でいわゆる 安楽[あんらく] 死[し]が 合法[ごうほう] 化[か]されていくだろう。
\\	いわゆる「いじめ」の問題は、自分と異なる人間を認めるメンタリティーが日本に根づかないと容易にはなくならないであろう。	
\\	いわゆる「いじめ」の 問題[もんだい]は、 自分[じぶん]と 異[こと]なる 人間[にんげん]を 認[みと]めるメンタリティーが 日本[にほん]に 根[ね]づかないと 容易[ようい]にはなくならないであろう。
\\	いわゆる
\\	すなわち情報技術の開発に、日本政府は本腰を入れ始めた。	
\\	いわゆる 
\\	すなわち 情報[じょうほう] 技術[ぎじゅつ]の 開発[かいはつ]に、 日本[にほん] 政府[せいふ]は 本腰[ほんごし]を 入[い]れ 始[はじ]めた。
\\	内分泌撹乱化学物質、いわゆる環境ホルモンは、人間と野生生物へ悪影響を与えると言われている。	
\\	内分泌[ないぶんぴつ] 撹乱[かくらん] 化学[かがく] 物質[ぶっしつ]、いわゆる 環境[かんきょう]ホルモンは、 人間[にんげん]と 野生[やせい] 生物[せいぶつ]へ 悪影響[あくえいきょう]を 与[あた]えると 言[い]われている。
\\	ジョンは自称占い師に騙された。	
\\	ジョンは 自称[じしょう] 占い師[うらないし]に 騙[だま]された。
\\	インターネット自体は情報の通路であって、内容は持たない。	
\\	インターネット 自体[じたい]は 情報[じょうほう]の 通路[つうろ]であって、 内容[ないよう]は 持[も]たない。
\\	私にとってはこの試合に出られること自体が光栄なことです。	
\\	私[わたし]にとってはこの 試合[しあい]に 出[で]られること 自体[じたい]が 光栄[こうえい]なことです。
\\	ケータイ自体には問題はない。問題は使う人のマナーや気配りだ。	
\\	ケータイ 自体[じたい]には 問題[もんだい]はない。 問題[もんだい]は 使[つか]う 人[ひと]のマナーや 気配[きくば]りだ。
\\	いまの仕事自体には満足していますが、職場の人間関係がうまくいかず悩んでいます。	
\\	いまの 仕事[しごと] 自体[じたい]には 満足[まんぞく]していますが、 職場[しょくば]の 人間[にんげん] 関係[かんけい]がうまくいかず 悩[なや]んでいます。
\\	音声データ自体に異状があるので、どのプレーヤーで再生してもノイズが入ってしまう。	
\\	音声[おんせい]データ 自体[じたい]に 異状[いじょう]があるので、どのプレーヤーで 再生[さいせい]してもノイズが 入[はい]ってしまう。
\\	世の中が大きく変わった今、この組織の存在自体が疑われている。	
\\	世の中[よのなか]が 大[おお]きく 変[か]わった 今[いま]、この 組織[そしき]の 存在[そんざい] 自体[じたい]が 疑[うたが]われている。
\\	この広告は、販売より広告それ自体が目的だ。	
\\	この 広告[こうこく]は、 販売[はんばい]より 広告[こうこく]それ 自体[じたい]が 目的[もくてき]だ。
\\	裁判所はこれらのファイル交換ソフト自体の違法性は認められないという判決を下した。	
\\	裁判所[さいばんしょ]はこれらのファイル 交換[こうかん]ソフト 自体[じたい]の 違法[いほう] 性[せい]は 認[みと]められないという 判決[はんけつ]を 下[くだ]した。
\\	そういうところから金を借りること自体が間違っている。	
\\	そういうところから 金[きん]を 借[か]りること 自体[じたい]が 間違[まちが]っている。
\\	問題はここの差別表現自体ではなく、そういう表現が出てくる背景なのだ。	
\\	問題[もんだい]はここの 差別[さべつ] 表現[ひょうげん] 自体[じたい]ではなく、そういう 表現[ひょうげん]が 出[で]てくる 背景[はいけい]なのだ。
\\	音楽のデジタル化やインターネット配信が音楽業界の主流になってきたが、私はそのこと自体は悪いことではないと思う。問題は、それによって不正コピーや大量配布がしやすくなったことだ。	
\\	音楽[おんがく]のデジタル 化[か]やインターネット 配信[はいしん]が 音楽[おんがく] 業界[ぎょうかい]の 主流[しゅりゅう]になってきたが、 私[わたし]はそのこと 自体[じたい]は 悪[わる]いことではないと 思[おも]う。 問題[もんだい]は、それによって 不正[ふせい]コピーや 大量[たいりょう] 配布[はいふ]がしやすくなったことだ。
\\	あなた自身はこの事件についてどう思いますか。	
\\	あなた 自身[じしん]はこの 事件[じけん]についてどう 思[おも]いますか。
\\	私自身は海外生活の経験がないので、異文化の問題はよく分かりません。	
\\	私[わたし] 自身[じしん]は 海外[かいがい] 生活[せいかつ]の 経験[けいけん]がないので、 異[い] 文化[ぶんか]の 問題[もんだい]はよく 分[わ]かりません。
\\	自分自身に挑戦してみよう。	
\\	自分[じぶん] 自身[じしん]に 挑戦[ちょうせん]してみよう。
\\	うちの子自身はアメリカには行きたがっていません。	
\\	うちの 子[こ] 自身[じしん]はアメリカには 行[い]きたがっていません。
\\	いろいろなところに行っておられるんですね。うらやましい限りです。	
\\	いろいろなところに 行[い]っておられるんですね。うらやましい 限[かぎ]りです。
\\	電子メールでご連絡いただければ、嬉しい限りです。	
\\	電子[でんし]メールでご 連絡[れんらく]いただければ、 嬉[うれ]しい 限[かぎ]りです。
\\	私の講演が皆様の知性を刺激できれば喜ばしい限りです。	
\\	私[わたし]の 講演[こうえん]が 皆様[みなさま]の 知性[ちせい]を 刺激[しげき]できれば 喜[よろこ]ばしい 限[かぎ]りです。
\\	あなたの明るい顔が見られなくなって、寂しい限りです。	
\\	あなたの 明[あか]るい 顔[かお]が 見[み]られなくなって、 寂[さび]しい 限[かぎ]りです。
\\	私がかいた絵をお見せするのはお恥ずかしい限りです。	
\\	私[わたし]がかいた 絵[え]をお 見[み]せするのはお 恥[は]ずかしい 限[かぎ]りです。
\\	その頃は大阪に単身赴任で行っていたが、家族から離れて夜一人でアパートで食事をするのはわびしい限りだった。	
\\	その 頃[ころ]は 大阪[おおさか]に 単身[たんしん] 赴任[ふにん]で 行[い]っていたが、 家族[かぞく]から 離[はな]れて 夜[よる] 一人[ひとり]でアパートで 食事[しょくじ]をするのはわびしい 限[かぎ]りだった。
\\	試合中止とは残念な限りです。	
\\	試合[しあい] 中止[ちゅうし]とは 残念[ざんねん]な 限[かぎ]りです。
\\	木下さんはお母さんが交通事故に遭ったとか。お気の毒な限りです。	
\\	木下[きのした]さんはお 母[かあ]さんが 交通[こうつう] 事故[じこ]に 遭[あ]ったとか。お 気の毒[きのどく]な 限[かぎ]りです。
\\	私の講演はお粗末な限りで、申し訳ございませんでした。	
\\	私[わたし]の 講演[こうえん]はお 粗末[そまつ]な 限[かぎ]りで、 申し訳[もうしわけ]ございませんでした。
\\	この大学の建物はとてもきれいです。	
\\	この 大学[だいがく]の 建物[たてもの]はとてもきれいです。
\\	あの人はとても健康だ。	
\\	あの 人[ひと]はとても 健康[けんこう]だ。
\\	本プロジェクトを皆様のご協力により完成させることができましたことは誠に感謝の限りです。	
\\	本[ほん]プロジェクトを 皆様[みなさま]のご 協力[きょうりょく]により 完成[かんせい]させることができましたことは 誠[まこと]に 感謝[かんしゃ]の 限[かぎ]りです。
\\	つたない料理をお褒めいただき恐縮の限りです。	
\\	つたない 料理[りょうり]をお 褒[ほ]めいただき 恐縮[きょうしゅく]の 限[かぎ]りです。
\\	日本語をよく勉強したかいがあって、日本関係の会社に就職することができた。	
\\	日本語[にほんご]をよく 勉強[べんきょう]したかいがあって、 日本[にほん] 関係[かんけい]の 会社[かいしゃ]に 就職[しゅうしょく]することができた。
\\	社員一同の努力のかいがあって一時低調だった会社も上向きになった。	
\\	社員[しゃいん] 一同[いちどう]の 努力[どりょく]のかいがあって 一時[いちじ] 低調[ていちょう]だった 会社[かいしゃ]も 上向[うわむ]きになった。
\\	あなたにとって生きがいとは何ですか。	
\\	あなたにとって 生[い]きがいとは 何[なに]ですか。
\\	彼は東南アジア諸国の旅行中、毎日詳しく日記をつけておいた。そのかいがあって、日本に戻って旅行記を書く時にとても書きやすかった。	
\\	彼[かれ]は 東南アジア[とうなんあじあ] 諸国[しょこく]の 旅行[りょこう] 中[ちゅう]、 毎日[まいにち] 詳[くわ]しく 日記[にっき]をつけておいた。そのかいがあって、 日本[にほん]に 戻[もど]って 旅行[りょこう] 記[き]を 書[か]く 時[とき]にとても 書[か]きやすかった。
\\	一生懸命やったかいがあって、マラソン大会ごとに順位が上がってきました。	
\\	一生懸命[いっしょうけんめい]やったかいがあって、マラソン 大会[たいかい]ごとに 順位[じゅんい]が 上[あ]がってきました。
\\	寒い中待っていたかいがあって、見事な日の出を見ることができた。	
\\	寒[さむ]い 中[なか] 待[ま]っていたかいがあって、 見事[みごと]な 日の出[ひので]を 見[み]ることができた。
\\	せっかく富士山に登ったのに頂上からは下が全然見えず、登ったかいがなかった。	
\\	せっかく 富士山[ふじさん]に 登[のぼ]ったのに 頂上[ちょうじょう]からは 下[した]が 全然[ぜんぜん] 見[み]えず、 登[のぼ]ったかいがなかった。
\\	弟は看病のかいもなく5歳の短い生涯を閉じた。	
\\	弟[おとうと]は 看病[かんびょう]のかいもなく 
\\	歳[さい]の 短[みじか]い 生涯[しょうがい]を 閉[と]じた。
\\	辛抱のかいがあって、運が開けた。	
\\	辛抱[しんぼう]のかいがあって、 運[うん]が 開[あ]けた。
\\	この大学の学生はよく勉強するから、本当に教えがいがある。	
\\	この 大学[だいがく]の 学生[がくせい]はよく 勉強[べんきょう]するから、 本当[ほんとう]に 教[おし]えがいがある。
\\	私の料理をおいしく食べてくれる人がいるから、作りがいがあるのです。	
\\	私[わたし]の 料理[りょうり]をおいしく 食[た]べてくれる 人[ひと]がいるから、 作[つく]りがいがあるのです。
\\	交通事故で脚に大怪我をして、毎日リハビリに通うことになった。そのかいがあって、今では元どおり歩けるようになった。	
\\	交通[こうつう] 事故[じこ]で 脚[あし]に 大[おお] 怪我[けが]をして、 毎日[まいにち]リハビリに 通[かよ]うことになった。そのかいがあって、 今[いま]では 元[もと]どおり 歩[ある]けるようになった。
\\	植物人間になった場合、もはや人間として機能していないのだから、その人間に生きる価値はないという立場もあろう。	
\\	植物[しょくぶつ] 人間[にんげん]になった 場合[ばあい]、もはや 人間[にんげん]として 機能[きのう]していないのだから、その 人間[にんげん]に 生[い]きる 価値[かち]はないという 立場[たちば]もあろう。
\\	このソフトが役に立つか否かは保証の限りではありません。	
\\	このソフトが 役に立[やくにた]つか 否[いな]かは 保証[ほしょう]の 限[かぎ]りではありません。
\\	選挙の時に、有権者は候補者が倫理観のある政治家か否かを見抜かなければならない。	
\\	選挙[せんきょ]の 時[とき]に、 有権者[ゆうけんしゃ]は 候補[こうほ] 者[しゃ]が 倫理[りんり] 観[かん]のある 政治[せいじ] 家[か]か 否[いな]かを 見抜[みぬ]かなければならない。
\\	子供に携帯電話を持たせるべきか否かは議論が分かれる。	
\\	子供[こども]に 携帯[けいたい] 電話[でんわ]を 持[も]たせるべきか 否[いな]かは 議論[ぎろん]が 分[わ]かれる。
\\	アメリカの大学では教授が外国人であるか否かは問題にならないが、日本ではそれがよく問題になるようだ。	
\\	アメリカの 大学[だいがく]では 教授[きょうじゅ]が 外国[がいこく] 人[じん]であるか 否[いな]かは 問題[もんだい]にならないが、 日本[にほん]ではそれがよく 問題[もんだい]になるようだ。
\\	家を探す場合、その地域が安全であるか否かが一番大事なポイントになる。	
\\	家[いえ]を 探[さが]す 場合[ばあい]、その 地域[ちいき]が 安全[あんぜん]であるか 否[いな]かが 一番[いちばん] 大事[だいじ]なポイントになる。
\\	クラシック音楽を聞く時、私は旋律が美しいか否かよりその音楽が生命を感じさせるか否かに注意を払う。	
\\	クラシック 音楽[おんがく]を 聞[き]く 時[とき]、 私[わたし]は 旋律[せんりつ]が 美[うつく]しいか 否[いな]かよりその 音楽[おんがく]が 生命[せいめい]を 感[かん]じさせるか 否[いな]かに 注意[ちゅうい]を 払[はら]う。
\\	大学の入試要綱をよく読んでから受験するか否かを決めるべきだ。	
\\	大学[だいがく]の 入試[にゅうし] 要綱[ようこう]をよく 読[よ]んでから 受験[じゅけん]するか 否[いな]かを 決[き]めるべきだ。
\\	愛されたら愛しかえすべきか否かという疑問を愚問です。	
\\	愛[あい]されたら 愛[あい]しかえすべきか 否[いな]かという 疑問[ぎもん]を 愚問[ぐもん]です。
\\	容疑者に遺体を見せると、真犯人か否かが分かることがある。	
\\	容疑[ようぎ] 者[しゃ]に 遺体[いたい]を 見[み]せると、 真犯人[しんはんにん]か 否[いな]かが 分[わ]かることがある。
\\	大学を出てすぐ就職するのがいいか悪いかは一概に言えない。	
\\	大学[だいがく]を 出[で]てすぐ 就職[しゅうしょく]するのがいいか 悪[わる]いかは 一概[いちがい]に 言[い]えない。
\\	あいつがやってくるかこないか分からないじゃん。	
\\	あいつがやってくるかこないか 分[わ]からないじゃん。
\\	彼、一見若いのかどうかはっきりしないね。	
\\	彼[かれ]、 一見[いっけん] 若[わか]いのかどうかはっきりしないね。
\\	悲しいかな、年を取って往年の記憶力がなくなってしまった。	
\\	悲[かな]しいかな、 年[とし]を 取[と]って 往年[おうねん]の 記憶[きおく] 力[りょく]がなくなってしまった。
\\	悲しいかな、自分の能力を過大評価する人間がいる。	
\\	悲[かな]しいかな、 自分[じぶん]の 能力[のうりょく]を 過大[かだい] 評価[ひょうか]する 人間[にんげん]がいる。
\\	インターネットを使うといろいろ面白いことが分かるそうですが、惜しいかな、私はコンピュータを買うお金がないんです。	
\\	インターネットを 使[つか]うといろいろ 面白[おもしろ]いことが 分[わ]かるそうですが、 惜[お]しいかな、 私[わたし]はコンピュータを 買[か]うお 金[かね]がないんです。
\\	画面では絵も動きます。チャットもできます。惜しいかな、全部英語ですが。	
\\	画面[がめん]では 絵[え]も 動[うご]きます。チャットもできます。 惜[お]しいかな、 全部[ぜんぶ] 英語[えいご]ですが。
\\	秋の訪れを知ると共に、大空は広大なるかなと驚く。	
\\	秋[あき]の 訪[おとず]れを 知[し]ると 共[とも]に、 大空[おおぞら]は 広大[こうだい]なるかなと 驚[おどろ]く。
\\	みんな退職後の生活の甘いイメージを抱いている。しかし、残念なるかな、現実には、熟年離婚や体の不調にさいなまれている。	
\\	みんな 退職[たいしょく] 後[ご]の 生活[せいかつ]の 甘[あま]いイメージを 抱[だ]いている。しかし、 残念[ざんねん]なるかな、 現実[げんじつ]には、 熟年[じゅくねん] 離婚[りこん]や 体[からだ]の 不調[ふちょう]にさいなまれている。
\\	途中に教会があったので期待して入ってみると、果たせるかな、クリスマス礼拝で荘厳な雰囲気であった。	
\\	途中[とちゅう]に 教会[きょうかい]があったので 期待[きたい]して 入[はい]ってみると、 果[は]たせるかな、クリスマス 礼拝[れいはい]で 荘厳[そうごん]な 雰囲気[ふんいき]であった。
\\	せっかく富士山に登ったのに残念なるかな雲で下が見えなかった。	
\\	せっかく 富士山[ふじさん]に 登[のぼ]ったのに 残念[ざんねん]なるかな 雲[くも]で 下[した]が 見[み]えなかった。
\\	面白いことに、日本人はやたらと英語の単語を使いたがる。	
\\	面白[おもしろ]いことに、 日本人[にほんじん]はやたらと 英語[えいご]の 単語[たんご]を 使[つか]いたがる。
\\	恥ずかしいことに、私は日本人でありながら日本の歴史をほとんど何も知りません。	
\\	恥[は]ずかしいことに、 私[わたし]は 日本人[にほんじん]でありながら 日本[にほん]の 歴史[れきし]をほとんど 何[なに]も 知[し]りません。
\\	わびしいことに、私は今一人暮らしなんです。	
\\	わびしいことに、 私[わたし]は 今[こん] 一人暮[ひとりぐ]らしなんです。
\\	彼女はまるで整形手術をしたかのようにきれいになった。	
\\	彼女[かのじょ]はまるで 整形[せいけい] 手術[しゅじゅつ]をしたかのようにきれいになった。
\\	彼女の病気は年内に治るかのように見えた。	
\\	彼女[かのじょ]の 病気[びょうき]は 年内[ねんない]に 治[なお]るかのように 見[み]えた。
\\	あたかも日本の製造業全体の競争力は強いかのように思われてきた。	
\\	あたかも 日本[にほん]の 製造[せいぞう] 業[ぎょう] 全体[ぜんたい]の 競争[きょうそう] 力[りょく]は 強[つよ]いかのように 思[おも]われてきた。
\\	彼は有能な政治学者であるかのように言われているが、実は無能極まりない。	
\\	彼[かれ]は 有能[ゆうのう]な 政治[せいじ] 学者[がくしゃ]であるかのように 言[い]われているが、 実[じつ]は 無能[むのう] 極[きわ]まりない。
\\	あの男は不確かなことをあたかも正確であるかのように断言する。	
\\	あの 男[おとこ]は 不確[ふたし]かなことをあたかも 正確[せいかく]であるかのように 断言[だんげん]する。
\\	その教師は私がまるでカンニングをしたかのように私を問いただした。	
\\	その 教師[きょうし]は 私[わたし]がまるでカンニングをしたかのように 私[わたし]を 問[と]いただした。
\\	彼は心理学のことは何も知らないのにあたかもよく知っているかのように心理学のことを話したがる。	
\\	彼[かれ]は 心理[しんり] 学[がく]のことは 何[なに]も 知[し]らないのにあたかもよく 知[し]っているかのように 心理[しんり] 学[がく]のことを 話[はな]したがる。
\\	あたかも僕の心を解かすかのように、白くて透明な雪が降っている。	
\\	あたかも 僕[ぼく]の 心[こころ]を 解[と]かすかのように、 白[しろ]くて 透明[とうめい]な 雪[ゆき]が 降[ふ]っている。
\\	まるで私の結婚を祝ってくれているかのように空は完璧に晴れていた。	
\\	まるで 私[わたし]の 結婚[けっこん]を 祝[いわ]ってくれているかのように 空[そら]は 完璧[かんぺき]に 晴[は]れていた。
\\	あの教師は学生が何を考えているかまるで意に介していないかのように見える。	
\\	あの 教師[きょうし]は 学生[がくせい]が 何[なに]を 考[かんが]えているかまるで 意[い]に 介[かい]していないかのように 見[み]える。
\\	平凡でつまらないかのようにみえる彼らの人生とて起伏がないわけではない。	
\\	平凡[へいぼん]でつまらないかのようにみえる 彼[かれ]らの 人生[じんせい]とて 起伏[きふく]がないわけではない。
\\	空港での持ち物検査の時、うっかりはさみをポケットに入れていたので、犯罪者であるかのように扱われた。	
\\	空港[くうこう]での 持ち物[もちもの] 検査[けんさ]の 時[とき]、うっかりはさみをポケットに 入[い]れていたので、 犯罪[はんざい] 者[しゃ]であるかのように 扱[あつか]われた。
\\	イスラム系組織のテロがあるからと言って、イスラム系の人があたかもすべて危険であるかのように考えるのは単純すぎる。	
\\	イスラム 系[けい] 組織[そしき]のテロがあるからと 言[い]って、イスラム 系[けい]の 人[ひと]があたかもすべて 危険[きけん]であるかのように 考[かんが]えるのは 単純[たんじゅん]すぎる。
\\	まるで別世界に来たかのようだった。	
\\	まるで 別世界[べっせかい]に 来[き]たかのようだった。
\\	私は金ですべてが解決するかのような考え方にはついていけない。	
\\	私[わたし]は 金[かね]ですべてが 解決[かいけつ]するかのような 考え方[かんがえかた]にはついていけない。
\\	燃費から言ってこの車は非常に経済的だ。	
\\	燃費[ねんぴ]から 言[い]ってこの 車[くるま]は 非常[ひじょう]に 経済[けいざい] 的[てき]だ。
\\	学生の評価から言って鈴木先生はいい先生に違いない。	
\\	学生[がくせい]の 評価[ひょうか]から 言[い]って 鈴木[すずき] 先生[せんせい]はいい 先生[せんせい]に 違[ちが]いない。
\\	性能から言って、このエンジンをしのぐものは少ない。	
\\	性能[せいのう]から 言[い]って、このエンジンをしのぐものは 少[すく]ない。
\\	この小説は内容から言って子供向きではない。	
\\	この 小説[しょうせつ]は 内容[ないよう]から 言[い]って 子供[こども] 向[む]きではない。
\\	仕事の効率から言って週休二日の方がいい。	
\\	仕事[しごと]の 効率[こうりつ]から 言[い]って 週休[しゅうきゅう] 二日[ににち]の 方[ほう]がいい。
\\	鑑定から言って、彼が犯人であることは確実だ。	
\\	鑑定[かんてい]から 言[い]って、 彼[かれ]が 犯人[はんにん]であることは 確実[かくじつ]だ。
\\	顔の色つやから言って、父は間違いなく健康だ。	
\\	顔[かお]の 色[いろ]つやから 言[い]って、 父[ちち]は 間違[まちが]いなく 健康[けんこう]だ。
\\	日本語能力試験の結果から言って、ジョンの日本語はすばらしいに違いない。	
\\	日本語[にほんご] 能力[のうりょく] 試験[しけん]の 結果[けっか]から 言[い]って、ジョンの 日本語[にほんご]はすばらしいに 違[ちが]いない。
\\	英語の“
\\	はラテン語の“
\\	(再び)と“
\\	(緩む)とからなっている。	
\\	英語[えいご]の“ 
\\	は ラテン語[らてんご]の“ 
\\	再[ふたた]び)と“ 
\\	緩[ゆる]む)とからなっている。
\\	タンパク質はアミノ酸からなっている。	
\\	タンパク質[たんぱくしつ]は アミノ酸[あみのさん]からなっている。
\\	夫婦と子供からなる世帯は今後減るのだろうか。	
\\	夫婦[ふうふ]と 子供[こども]からなる 世帯[せたい]は 今後[こんご] 減[へ]るのだろうか。
\\	日本の神社は拝殿と本田とからなっている。	
\\	日本[にほん]の 神社[じんじゃ]は 拝殿[はいでん]と 本田[ほんだ]とからなっている。
\\	この大学の東洋学科は言語、文学と歴史からなっている。	
\\	この 大学[だいがく]の 東洋[とうよう] 学科[がっか]は 言語[げんご]、 文学[ぶんがく]と 歴史[れきし]からなっている。
\\	日本は島国で、本州と北海道と四国と九州と沖縄とからなっている。	
\\	日本[にほん]は 島国[しまぐに]で、 本州[ほんしゅう]と 北海道[ほっかいどう]と 四国[しこく]と 九州[きゅうしゅう]と 沖縄[おきなわ]とからなっている。
\\	地球の内部構造は、何重もの階層構造からなっている。	
\\	地球[ちきゅう]の 内部[ないぶ] 構造[こうぞう]は、 何[なん] 重[じゅう]もの 階層[かいそう] 構造[こうぞう]からなっている。
\\	ブラムスのレクイエムは演奏に約1時間半を要する大作で、全7楽章からなっている。	
\\	ブラムスのレクイエムは 演奏[えんそう]に 約[やく] 
\\	時間[じかん] 半[はん]を 要[よう]する 大作[たいさく]で、 全[ぜん] 
\\	楽章[がくしょう]からなっている。
\\	フィリピンは多くの島々からなっている。	
\\	フィリピンは 多[おお]くの 島々[しまじま]からなっている。
\\	このボランティア団体は約150名の会員からなる団体です。	
\\	このボランティア 団体[だんたい]は 約[やく] 
\\	名[めい]の 会員[かいいん]からなる 団体[だんたい]です。
\\	本誌の掲載原稿は、登校原稿と依頼原稿とからなる。	
\\	本誌[ほんし]の 掲載[けいさい] 原稿[げんこう]は、 登校[とうこう] 原稿[げんこう]と 依頼[いらい] 原稿[げんこう]とからなる。
\\	1個のモーラは、普通、1個の子音と1個の母音とからなっている。	
\\	個[こ]のモーラは、 普通[ふつう]、 
\\	個[こ]の 子音[しいん]と 
\\	個[こ]の 母音[ぼいん]とからなっている。
\\	米国議会は上院と下院よりなる。	
\\	米国[べいこく] 議会[ぎかい]は 上院[じょういん]と 下院[かいん]よりなる。
\\	いったん引き受けたからには最後まで責任をもってやります。	
\\	いったん 引き受[ひきう]けたからには 最後[さいご]まで 責任[せきにん]をもってやります。
\\	裁判と言うからには弁護人が不可欠だ。	
\\	裁判[さいばん]と 言[い]うからには 弁護[べんご] 人[にん]が 不可欠[ふかけつ]だ。
\\	生まれてきたからには長生きしたい。	
\\	生[う]まれてきたからには 長生[ながい]きしたい。
\\	行くって言ったからにはちゃんと行かなきゃ駄目だよ。	
\\	行[い]くって 言[い]ったからにはちゃんと 行[い]かなきゃ 駄目[だめ]だよ。
\\	ネット上で文章を書いたからには、必ずそれを読む人が存在するのです。	
\\	ネット 上[うえ]で 文章[ぶんしょう]を 書[か]いたからには、 必[かなら]ずそれを 読[よ]む 人[ひと]が 存在[そんざい]するのです。
\\	写真をカラーで撮るからには、色にこだわりたい。	
\\	写真[しゃしん]をカラーで 撮[と]るからには、 色[いろ]にこだわりたい。
\\	マラソンに出るからには、絶対に勝ちたい。	
\\	マラソンに 出[で]るからには、 絶対[ぜったい]に 勝[か]ちたい。
\\	アメリカに住んでいるからには郷に入っては郷に従えで、批判ばっかりしていたって仕方がない。	
\\	アメリカに 住[す]んでいるからには 郷[ごう]に 入[い]っては 郷[ごう]に 従[したが]えで、 批判[ひはん]ばっかりしていたって 仕方[しかた]がない。
\\	「ブレーンストーミング」と言うからには、発言に制限を設けてはならない。	
\\	「ブレーンストーミング」と 言[い]うからには、 発言[はつげん]に 制限[せいげん]を 設[もう]けてはならない。
\\	世界一週と言うからには、経度の基準点であるグリニッジ天文台から出発しようと決め込んだ。	
\\	世界[せかい] 一週[いっしゅう]と 言[い]うからには、 経度[けいど]の 基準[きじゅん] 点[てん]であるグリニッジ 天文台[てんもんだい]から 出発[しゅっぱつ]しようと 決め込[きめこ]んだ。
\\	個性、能力を重視すると言うからには、教育改革が必要だ。	
\\	個性[こせい]、 能力[のうりょく]を 重視[じゅうし]すると 言[い]うからには、 教育[きょういく] 改革[かいかく]が 必要[ひつよう]だ。
\\	一度すると言ったからにはちゃんとやってくれなければ困る。	
\\	一度[いちど]すると 言[い]ったからにはちゃんとやってくれなければ 困[こま]る。
\\	ビールの本場と言うからにはさぞおいしいだろうね。	
\\	ビールの 本場[ほんば]と 言[い]うからにはさぞおいしいだろうね。
\\	函館に来たからは……まずはやっぱ、函館ラーメン!	
\\	函館[はこだて]に 来[き]たからは……まずはやっぱ、 函館[はこだて]ラーメン!
\\	やるからは一生懸命にやりたい。	
\\	やるからは 一生懸命[いっしょうけんめい]にやりたい。
\\	始めたからは死ぬまで続けてね。	
\\	始[はじ]めたからは 死[し]ぬまで 続[つづ]けてね。
\\	社長からして会社再建の意欲がないんだから、社員の志気が上げらないのは当たり前だ。	
\\	社長[しゃちょう]からして 会社[かいしゃ] 再建[さいけん]の 意欲[いよく]がないんだから、 社員[しゃいん]の 志気[しき]が 上[あ]げらないのは 当たり前[あたりまえ]だ。
\\	あの日本語の話し方からして彼は日本に行ったことがあるに違いない。	
\\	あの 日本語[にほんご]の 話し方[はなしかた]からして 彼[かれ]は 日本[にほん]に 行[い]ったことがあるに 違[ちが]いない。
\\	あの男は歩き方からして他の人と違う。	
\\	あの 男[おとこ]は 歩[ある]き 方[かた]からして 他[た]の 人[ひと]と 違[ちが]う。
\\	プロの銀行員からして、初歩的な経済の分析に必要な数学に弱い。	
\\	プロの 銀行[ぎんこう] 員[いん]からして、 初歩[しょほ] 的[てき]な 経済[けいざい]の 分析[ぶんせき]に 必要[ひつよう]な 数学[すうがく]に 弱[よわ]い。
\\	週末の彼女とのデートでは、待ち合わせの場所からして間違えてしまった。	
\\	週末[しゅうまつ]の 彼女[かのじょ]とのデートでは、 待ち合[まちあ]わせの 場所[ばしょ]からして 間違[まちが]えてしまった。
\\	沈没した原子力潜水艦を引き揚げるのは、その大きさからして技術的な限界に近い。	
\\	沈没[ちんぼつ]した 原子力[げんしりょく] 潜水[せんすい] 艦[かん]を 引き揚[ひきあ]げるのは、その 大[おお]きさからして 技術[ぎじゅつ] 的[てき]な 限界[げんかい]に 近[ちか]い。
\\	「車は走る凶器」「飲んだら乗るな」の掛け声も、事故の実態からして死語同然となっている。	
\\	車[くるま]は 走[はし]る 凶器[きょうき]」
\\	飲[の]んだら 乗[の]るな」の 掛け声[かけごえ]も、 事故[じこ]の 実態[じったい]からして 死語[しご] 同然[どうぜん]となっている。
\\	僕は彼女の話し方からして気に食わないんだ。	
\\	僕[ぼく]は 彼女[かのじょ]の 話し方[はなしかた]からして 気[き]に 食[く]わないんだ。
\\	彼女はその話し方からして明晰な頭脳を持っていると思う。	
\\	彼女[かのじょ]はその 話し方[はなしかた]からして 明晰[めいせき]な 頭脳[ずのう]を 持[も]っていると 思[おも]う。
\\	あの男は自分の子供ですら殴りつける。	
\\	あの 男[おとこ]は 自分[じぶん]の 子供[こども]ですら 殴[なぐ]りつける。
\\	彼女は両親にさえ本当のことを言わない。	
\\	彼女[かのじょ]は 両親[りょうしん]にさえ 本当[ほんとう]のことを 言[い]わない。
\\	仮に1億円もらうとしたら何に使いますか。	
\\	仮[かり]に1 億[おく] 円[えん]もらうとしたら 何[なに]に 使[つか]いますか。
\\	仮に月収が30万円あったとして、どのようなアパートを借りることができるだろうか。	
\\	仮[かり]に 月収[げっしゅう]が 
\\	万[まん] 円[えん]あったとして、どのようなアパートを 借[か]りることができるだろうか。
\\	仮にこれが日本の教育の現状だったら、構造改革の余地は大いにある。	
\\	仮[かり]にこれが 日本[にほん]の 教育[きょういく]の 現状[げんじょう]だったら、 構造[こうぞう] 改革[かいかく]の 余地[よち]は 大[おお]いにある。
\\	仮に手術が成功するとしても、高齢だから父の体が元に戻ることはないだろう。	
\\	仮[かり]に 手術[しゅじゅつ]が 成功[せいこう]するとしても、 高齢[こうれい]だから 父[ちち]の 体[からだ]が 元[もと]に 戻[もど]ることはないだろう。
\\	仮にあなたがベンチャー企業の経営権を持っていたら、どのようなアプローチを取りますか。	
\\	仮[かり]にあなたがベンチャー 企業[きぎょう]の 経営[けいえい] 権[けん]を 持[も]っていたら、どのようなアプローチを 取[と]りますか。
\\	よくある質問ですが、仮にあなたが今小さい島で一人で暮らすとして、一冊だけ本を持って行けるとしたら、どの本を持って行きますか。	
\\	よくある 質問[しつもん]ですが、 仮[かり]にあなたが 今[いま] 小[ちい]さい 島[しま]で 一人[ひとり]で 暮[く]らすとして、一冊[いっさつ]だけ 本[ほん]を 持[も]って 行[い]けるとしたら、どの 本[ほん]を 持[も]って 行[い]きますか。
\\	仮に今の調子で老齢化が進むとすれば、21世紀後半には日本の人口の5割以上が65歳以上になる可能性がある。	
\\	仮[かり]に 今[いま]の 調子[ちょうし]で 老齢[ろうれい] 化[か]が 進[すす]むとすれば、 
\\	世紀[せいき] 後半[こうはん]には 日本[にほん]の 人口[じんこう]の 
\\	割[わり] 以上[いじょう]が 
\\	歳[さい] 以上[いじょう]になる 可能[かのう] 性[せい]がある。
\\	仮に日本に1年ぐらい行って日本語を勉強すれば、君たちの日本語はすごく上達するよ。	
\\	仮[かり]に 日本[にほん]に 
\\	年[ねん]ぐらい 行[い]って 日本語[にほんご]を 勉強[べんきょう]すれば、 君[きみ]たちの 日本語[にほんご]はすごく 上達[じょうたつ]するよ。
\\	信号を待っている時、仮に前の車が青になったのに進まなかったとしても、クラクションを鳴らさない方がよいでしょう。	
\\	信号[しんごう]を 待[ま]っている 時[とき]、 仮[かり]に 前[まえ]の 車[くるま]が 青[あお]になったのに 進[すす]まなかったとしても、クラクションを 鳴[な]らさない 方[ほう]がよいでしょう。
\\	仮に話を経済学の「理論」に限定するとしても、今日までのあらゆる経済学理論が結局においてはすべてゲーム理論だと言っていいだろう。	
\\	仮[かり]に 話[はなし]を 経済[けいざい] 学[がく]の
\\	理論[りろん]」に 限定[げんてい]するとしても、 今日[こんにち]までのあらゆる 経済[けいざい] 学理[がくり] 論[ろん]が 結局[けっきょく]においてはすべてゲーム 理論[りろん]だと 言[い]っていいだろう。
\\	仮に仲介者を通して売る場合、利益が少なくなるのは当然だ。	
\\	仮[かり]に 仲介[ちゅうかい] 者[しゃ]を 通[とお]して 売[う]る 場合[ばあい]、 利益[りえき]が 少[すく]なくなるのは 当然[とうぜん]だ。
\\	仮に日本の国立大学の大学院に入る場合、どのぐらいお金がかかりますか。	
\\	仮[かり]に 日本[にほん]の 国立[こくりつ] 大学[だいがく]の 大学院[だいがくいん]に 入[はい]る 場合[ばあい]、どのぐらいお 金[かね]がかかりますか。
\\	最近は、肌寒くなったかと思うと、翌日には暖かくなったりする。	
\\	最近[さいきん]は、 肌寒[はだざむ]くなったかと 思[おも]うと、 翌日[よくじつ]には 暖[あたた]かくなったりする。
\\	私の書いた小説が出版されるかと思うと、とても嬉しい。	
\\	私[わたし]の 書[か]いた 小説[しょうせつ]が 出版[しゅっぱん]されるかと 思[おも]うと、とても 嬉[うれ]しい。
\\	ハワイでは、雨が降ってきたかと思うと、すぐやんでしまう。	
\\	ハワイでは、 雨[あめ]が 降[ふ]ってきたかと 思[おも]うと、すぐやんでしまう。
\\	花火は、あがったかと思うとすぐ消えてしまうからいいのだ。	
\\	花火[はなび]は、あがったかと 思[おも]うとすぐ 消[き]えてしまうからいいのだ。
\\	うちの小学生の子供は、テレビを見終わったかと思うと、コンピュータゲームをやり始める。	
\\	うちの 小学生[しょうがくせい]の 子供[こども]は、テレビを 見[み] 終[お]わったかと 思[おも]うと、コンピュータゲームをやり 始[はじ]める。
\\	妻がおいしい夕食を作ってくれているかと思うと、家に帰る足も速まる。	
\\	妻[つま]がおいしい 夕食[ゆうしょく]を 作[つく]ってくれているかと 思[おも]うと、 家[いえ]に 帰[かえ]る 足[あし]も 速[はや]まる。
\\	会社のリストラで職を失うのではないかと思うと、気が重いです。	
\\	会社[かいしゃ]のリストラで 職[しょく]を 失[うしな]うのではないかと 思[おも]うと、 気[き]が 重[おも]いです。
\\	子供たちも大学を出て一人立ちしたので、もう学費もかからないかと思うと、ほっとします。	
\\	子供[こども]たちも 大学[だいがく]を 出[で]て 一人[ひとり] 立[た]ちしたので、もう 学費[がくひ]もかからないかと 思[おも]うと、ほっとします。
\\	先月は、肌寒くなったかと思うと、翌日には暖かくなったりした。	
\\	先月[せんげつ]は、 肌寒[はださむ]くなったかと 思[おも]うと、 翌日[よくじつ]には 暖[あたた]かくなったりした。
\\	私の書いた小説が出版されるかと思うと、とても嬉しかった。	
\\	私[わたし]の 書[か]いた 小説[しょうせつ]が 出版[しゅっぱん]されるかと 思[おも]うと、とても 嬉[うれ]しかった。
\\	スミス氏は流暢な日本語を話し、かつ日本の歴史や文化にも明るい。	
\\	スミス 氏[し]は 流暢[りゅうちょう]な 日本語[にほんご]を 話[はな]し、かつ 日本[にほん]の 歴史[れきし]や 文化[ぶんか]にも 明[あか]るい。
\\	この本は、分かりやすく、かつ味わい深い言葉で書かれている。	
\\	この 本[ほん]は、 分[わ]かりやすく、かつ 味わい深[あじわいぶか]い 言葉[ことば]で 書[か]かれている。
\\	彼女は一流のピアニストで、かつ優れた随筆家だ。	
\\	彼女[かのじょ]は 一流[いちりゅう]のピアニストで、かつ 優[すぐ]れた 随筆[ずいひつ] 家[か]だ。
\\	外国語は、楽しく、かつ効果的に勉強したい。	
\\	外国[がいこく] 語[ご]は、 楽[たの]しく、かつ 効果[こうか] 的[てき]に 勉強[べんきょう]したい。
\\	英語ができること、かつウェブサイトの作成経験があることがこの仕事の要件だ。	
\\	英語[えいご]ができること、かつウェブサイトの 作成[さくせい] 経験[けいけん]があることがこの 仕事[しごと]の 要件[ようけん]だ。
\\	パソコンさえあれば自宅ででき、かつ高収入の仕事をご紹介します。	
\\	パソコンさえあれば 自宅[じたく]ででき、かつ 高[こう] 収入[しゅうにゅう]の 仕事[しごと]をご 紹介[しょうかい]します。
\\	この図は、自動車を個人で利用できず、かつ家族とも共有できない65歳以上の高齢者の外出頻度を示している。	
\\	この 図[ず]は、 自動車[じどうしゃ]を 個人[こじん]で 利用[りよう]できず、かつ 家族[かぞく]とも 共有[きょうゆう]できない 
\\	歳[さい] 以上[いじょう]の 高齢[こうれい] 者[しゃ]の 外出[がいしゅつ] 頻度[ひんど]を 示[しめ]している。
\\	和太鼓による若々しく、かつたくましい演奏をご堪能下さい。	
\\	和[わ] 太鼓[たいこ]による 若々[わかわか]しく、かつたくましい 演奏[えんそう]をご 堪能[たんのう] 下[くだ]さい。
\\	当店では、スタイリッシュで、かつシンプルなトレーニングウエアを各種取り揃えております。	
\\	当店[とうてん]では、スタイリッシュで、かつシンプルなトレーニングウエアを 各種[かくしゅ] 取り揃[とりそろ]えております。
\\	ご注文のキャンセルについては、注文から24時間以内で、かつ注文されたご本人によりメールあるいはファックスでご連絡いただいた場合のみ認められます。	
\\	ご 注文[ちゅうもん]のキャンセルについては、 注文[ちゅうもん]から 
\\	時間[じかん] 以内[いない]で、かつ 注文[ちゅうもん]されたご 本人[ほんにん]によりメールあるいはファックスでご 連絡[れんらく]いただいた 場合[ばあい]のみ 認[みと]められます。
\\	この仕事では大量の文献を速く且つ正確に読むことが要求される。	
\\	この 仕事[しごと]では 大量[たいりょう]の 文献[ぶんけん]を 速[はや]く 且[か]つ 正確[せいかく]に 読[よ]むことが 要求[ようきゅう]される。
\\	当社は、個人情報を取り扱うにあたり、個人情報保護法はじめ適用される諸法令、規範遵守を徹底し、適切且つ慎重に管理いたします。	
\\	当社[とうしゃ]は、 個人[こじん] 情報[じょうほう]を 取り扱[とりあつか]うにあたり、 個人[こじん] 情報[じょうほう] 保護[ほご] 法[ほう]はじめ 適用[てきよう]される 諸法[しょほう] 令[れい]、 規範[きはん] 遵守[じゅんしゅ]を 徹底[てってい]し、 適切[てきせつ] 且[か]つ 慎重[しんちょう]に 管理[かんり]いたします。
\\	売主は、製品には欠陥またはきずのある材料の使用がないこと、且つ当該製品が売主の発行した仕様に合致するものであることを保証する。	
\\	売主[うりぬし]は、 製品[せいひん]には 欠陥[けっかん]またはきずのある 材料[ざいりょう]の 使用[しよう]がないこと、 且[か]つ 当該[とうがい] 製品[せいひん]が 売主[うりぬし]の 発行[はっこう]した 仕様[しよう]に 合致[がっち]するものであることを 保証[ほしょう]する。
\\	中国と韓国がその会議に参加した。	
\\	中国[ちゅうごく]と 韓国[かんこく]がその 会議[かいぎ]に 参加[さんか]した。
\\	マリアはスペイン語とポルトガル語を話す。	
\\	マリアはスペイン 語[ご]とポルトガル 語[ご]を 話[はな]す。
\\	大統領はインドやパキスタンを訪問した。	
\\	大統領[だいとうりょう]はインドやパキスタンを 訪問[ほうもん]した。
\\	石田氏はカナダやオーストラリアに不動産を持っている。	
\\	石田[いしだ] 氏[し]はカナダやオーストラリアに 不動産[ふどうさん]を 持[も]っている。
\\	彼は物事を単純に考えるきらいがある。	
\\	彼[かれ]は 物事[ものごと]を 単純[たんじゅん]に 考[かんが]えるきらいがある。
\\	彼は独断のきらいがある。	
\\	彼[かれ]は 独断[どくだん]のきらいがある。
\\	我々はとかく自分に都合がいい意見だけに耳を傾けるきらいがある。	
\\	我々[われわれ]はとかく 自分[じぶん]に 都合[つごう]がいい 意見[いけん]だけに 耳[みみ]を 傾[かたむ]けるきらいがある。
\\	今の子供たちは、夜遅くまでコンピュータゲームなどをして寝ないきらいがある。	
\\	今[いま]の 子供[こども]たちは、 夜[よる] 遅[おそ]くまでコンピュータゲームなどをして 寝[ね]ないきらいがある。
\\	主婦は毎日の雑事に追われて、自分の体調管理がおろそかになるきらいがある。	
\\	主婦[しゅふ]は 毎日[まいにち]の 雑事[ざつじ]に 追[お]われて、 自分[じぶん]の 体調[たいちょう] 管理[かんり]がおろそかになるきらいがある。
\\	彼女は親切な人ですが、知らずに人を傷つけることを言うきらいがあります。	
\\	彼女[かのじょ]は 親切[しんせつ]な 人[ひと]ですが、 知[し]らずに 人[ひと]を 傷[きず]つけることを 言[い]うきらいがあります。
\\	私たちは、あまりに自分のからだを医者任せにしてきたきらいがあるのではないでしょうか。	
\\	私[わたし]たちは、あまりに 自分[じぶん]のからだを 医者[いしゃ] 任[まか]せにしてきたきらいがあるのではないでしょうか。
\\	これまでの日本の高等教育は、アジアとの共生を重視しなかったきらいがある。	
\\	これまでの 日本[にほん]の 高等[こうとう] 教育[きょういく]は、アジアとの 共生[きょうせい]を 重視[じゅうし]しなかったきらいがある。
\\	現在の日本の教育は、子供を社会から保護しすぎるきらいがある。	
\\	現在[げんざい]の 日本[にほん]の 教育[きょういく]は、 子供[こども]を 社会[しゃかい]から 保護[ほご]しすぎるきらいがある。
\\	彼は家庭環境が複雑なせいか、人間不信のきらいがある。	
\\	彼[かれ]は 家庭[かてい] 環境[かんきょう]が 複雑[ふくざつ]なせいか、 人間[にんげん] 不信[ふしん]のきらいがある。
\\	彼は頭もよくて、仕事もばりばりやるけど、八方美人のきらいがあるね。	
\\	彼[かれ]は 頭[あたま]もよくて、 仕事[しごと]もばりばりやるけど、 八方美人[はっぽうびじん]のきらいがあるね。
\\	健康診断で白内障のきらいがあると医者に言われました。	
\\	健康[けんこう] 診断[しんだん]で 白内障[はくないしょう]のきらいがあると 医者[いしゃ]に 言[い]われました。
\\	糖尿病のきらいがあるので、甘いものは控えています。	
\\	糖尿[とうにょう] 病[びょう]のきらいがあるので、 甘[あま]いものは 控[ひか]えています。
\\	これまでの学力試験は、記憶力を試す問題が多すぎたきらいがある。	
\\	これまでの 学力[がくりょく] 試験[しけん]は、 記憶[きおく] 力[りょく]を 試[ため]す 問題[もんだい]が 多[おお]すぎたきらいがある。
\\	これまで我が社は社員研修を軽視してきたきらいがある。	
\\	これまで 我[わ]が 社[しゃ]は 社員[しゃいん] 研修[けんしゅう]を 軽視[けいし]してきたきらいがある。
\\	うちの娘は病気がちで、学校をよく休みます。	
\\	うちの 娘[むすめ]は 病気[びょうき]がちで、 学校[がっこう]をよく 休[やす]みます。
\\	先月は出張が多くて、留守がちだった。	
\\	先月[せんげつ]は 出張[しゅっちょう]が 多[おお]くて、 留守[るす]がちだった。
\\	彼は小さい時から遠慮がちな性格だった。	
\\	彼[かれ]は 小[ちい]さい 時[とき]から 遠慮[えんりょ]がちな 性格[せいかく]だった。
\\	研究的使用のための十全なデータベースが構築されてきた傾向がある。	
\\	研究[けんきゅう] 的[てき] 使用[しよう]のための 十全[じゅうぜん]なデータベースが 構築[こうちく]されてきた 傾向[けいこう]がある。
\\	日本では箸は使い捨てや個人専用など、使用者が箸を共用しない傾向がある。	
\\	日本[にほん]では 箸[はし]は 使い捨[つかいす]てや 個人[こじん] 専用[せんよう]など、 使用[しよう] 者[しゃ]が 箸[はし]を 共用[きょうよう]しない 傾向[けいこう]がある。
\\	山下氏は1964年に日本を出たっきり日本には一度も帰っていない。	
\\	山下[やました] 氏[し]は 
\\	年[ねん]に 日本[にほん]を 出[で]たっきり 日本[にほん]には一 度[ど]も 帰[かえ]っていない。
\\	ソウルにいるのも今夜きりです。	
\\	ソウルにいるのも 今夜[こんや]きりです。
\\	彼女とは10年以上も前にパリで会ったきりで、その後どうしているのか、全然知らない。	
\\	彼女[かのじょ]とは 
\\	年[ねん] 以上[いじょう]も 前[まえ]にパリで 会[あ]ったきりで、その 後[ご]どうしているのか、 全然[ぜんぜん] 知[し]らない。
\\	小学校6年の時に別れたっきりの同級生に気づかれるなんて、ひょっとして私って小学校の時から顔が変わってないんだろうか。	
\\	小学校[しょうがっこう] 
\\	年[ねん]の 時[とき]に 別[わか]れたっきりの 同級生[どうきゅうせい]に 気[き]づかれるなんて、ひょっとして 私[わたし]って 小学校[しょうがっこう]の 時[とき]から 顔[かお]が 変[か]わってないんだろうか。
\\	フロイトの本はずいぶん前に読んだっきりなので、内容のほとんどは忘れてしまいました。	
\\	フロイトの 本[ほん]はずいぶん 前[まえ]に 読[よ]んだっきりなので、 内容[ないよう]のほとんどは 忘[わす]れてしまいました。
\\	父は雪の日に倒れて、腰の骨を折ってから、寝たきりになってしまった。	
\\	父[ちち]は 雪[ゆき]の 日[ひ]に 倒[たお]れて、 腰[こし]の 骨[ほね]を 折[お]ってから、 寝[ね]たきりになってしまった。
\\	これっきりなの?もっとたくさんちょうだい。	
\\	これっきりなの?もっとたくさんちょうだい。
\\	もう会うのはこれっきりにしましょう。	
\\	もう 会[あ]うのはこれっきりにしましょう。
\\	10年前に別れて二人はそれきりだった。	
\\	年[ねん] 前[まえ]に 別[わか]れて 二人[ふたり]はそれきりだった。
\\	彼からはあれっきり何も連絡がないの?	
\\	彼[かれ]からはあれっきり 何[なに]も 連絡[れんらく]がないの?
\\	夏休みも今日一日きりでおわっちゃうね。	
\\	夏休[なつやす]みも 今日[きょう] 一日[いちにち]きりでおわっちゃうね。
\\	私は母と二人っきりで暮らしています。	
\\	私[わたし]は 母[はは]と 二人[ふたり]っきりで 暮[く]らしています。
\\	領収書の整理や計算など、一日中、確定申告書の作成にかかりっきりだった。	
\\	領収[りょうしゅう] 書[しょ]の 整理[せいり]や 計算[けいさん]など、一 日[にち] 中[じゅう]、 確定[かくてい] 申告[しんこく] 書[しょ]の 作成[さくせい]にかかりっきりだった。
\\	一昨日、敏男が倒れてから、心配でたまらず、ずっとつききりで看病していたのだ。	
\\	一昨日[おととい]、 敏男[としお]が 倒[たお]れてから、 心配[しんぱい]でたまらず、ずっとつききりで 看病[かんびょう]していたのだ。
\\	その短編集に載った彼の短編は四つきりだった。	
\\	その 短[たん] 編集[ぺんしゅう]に 載[の]った 彼[かれ]の 短編[たんぺん]は 四[よっ]つきりだった。
\\	僕が読んだ村上春樹の小説は「ノルウェイの森」一作きりです。	
\\	僕[ぼく]が 読[よ]んだ 村上[むらかみ] 春樹[はるき]の 小説[しょうせつ]は「ノルウェイの 森[もり]」一作[いっさく]きりです。
\\	老後は妻と二人っきりの生活を楽しもうと思う。	
\\	老後[ろうご]は 妻[つま]と 二人[ふたり]っきりの 生活[せいかつ]を 楽[たの]しもうと 思[おも]う。
\\	今日はいつもと違ってビールを2本きりしか飲まなかった。	
\\	今日[きょう]はいつもと 違[ちが]ってビールを 
\\	本[ほん]きりしか 飲[の]まなかった。
\\	彼女はお金を借りたまま返しません。	
\\	彼女[かのじょ]はお 金[かね]を 借[か]りたまま 返[かえ]しません。
\\	靴をはいたまま家に入ってはいけません。	
\\	靴[くつ]をはいたまま 家[いえ]に 入[はい]ってはいけません。
\\	電気をつけたまま寝てしまった。	
\\	電気[でんき]をつけたまま 寝[ね]てしまった。
\\	鈴木さんは英語が話せるんだっけ。	
\\	鈴木[すずき]さんは 英語[えいご]が 話[はな]せるんだっけ。
\\	鈴木さんは英語が話せたっけ。	
\\	鈴木[すずき]さんは 英語[えいご]が 話[はな]せたっけ。
\\	今日は木曜日だったっけ。	
\\	今日[きょう]は 木曜日[もくようび]だったっけ。
\\	ジェーンさんはピアノが弾けたっけ。	
\\	ジェーンさんはピアノが 弾[ひ]けたっけ。
\\	パーティーは何時に始まるんだっけ。	
\\	パーティーは 何[なん] 時[じ]に 始[はじ]まるんだっけ。
\\	スミスさんはワシントン大学の学生だったっけ。	
\\	スミスさんはワシントン 大学[だいがく]の 学生[がくせい]だったっけ。
\\	僕が子供の頃はこの辺を林だったっけ。	
\\	僕[ぼく]が 子供[こども]の 頃[ころ]はこの 辺[へん]を 林[はやし]だったっけ。
\\	この映画、面白いんだっけ。	
\\	この 映画[えいが]、 面白[おもしろ]いんだっけ。
\\	あの人って、正直なんだっけ。	
\\	あの 人[ひと]って、 正直[しょうじき]なんだっけ。
\\	子供の時は父とよくこの川に釣りに来たなあ。	
\\	子供[こども]の 時[とき]は 父[ちち]とよくこの 川[かわ]に 釣[つ]りに 来[き]たなあ。
\\	ここは昔、小学校だったなあ。	
\\	ここは 昔[むかし]、 小学校[しょうがっこう]だったなあ。
\\	生まれたばかりの赤ん坊を車に残して買い物に行くとは非常識この上ない。	
\\	生[う]まれたばかりの 赤ん坊[あかんぼう]を 車[くるま]に 残[のこ]して 買い物[かいもの]に 行[い]くとは 非常識[ひじょうしき]この 上[うえ]ない。
\\	この部屋はクーラーも扇風機もなく、風も全然通らないので、暑いことこの上ない。	
\\	この 部屋[へや]はクーラーも 扇風機[せんぷうき]もなく、 風[かぜ]も 全然[ぜんぜん] 通[とお]らないので、 暑[あつ]いことこの 上[うえ]ない。
\\	大任を無事に果たしてこの上ない充実感に浸っている。	
\\	大任[たいにん]を 無事[ぶじ]に 果[は]たしてこの 上[うえ]ない 充実[じゅうじつ] 感[かん]に 浸[ひた]っている。
\\	最近よく食事時にセールスの電話がかかってくるが、迷惑この上ない。	
\\	最近[さいきん]よく 食事[しょくじ] 時[とき]にセールスの 電話[でんわ]がかかってくるが、 迷惑[めいわく]この 上[うえ]ない。
\\	一人住まいの高齢者を狙って金を騙し取るなど卑劣この上ない。	
\\	一 人[にん] 住[ず]まいの 高齢[こうれい] 者[しゃ]を 狙[ねら]って 金[きん]を 騙[だま]し 取[と]るなど 卑劣[ひれつ]この 上[うえ]ない。
\\	こんな大事な仕事をアルバイトにさせるなんて無責任この上ない。	
\\	こんな 大事[だいじ]な 仕事[しごと]をアルバイトにさせるなんて 無責任[むせきにん]この 上[うえ]ない。
\\	また選挙シーズンが来た。いつものことだが、選挙の車が連日うちの前を通り、マイクで候補者の名前を連呼する。うるさいことこの上ない。	
\\	また 選挙[せんきょ]シーズンが 来[き]た。いつものことだが、 選挙[せんきょ]の 車[くるま]が 連日[れんじつ]うちの 前[まえ]を 通[とお]り、マイクで 候補[こうほ] 者[しゃ]の 名前[なまえ]を 連呼[れんこ]する。うるさいことこの 上[うえ]ない。
\\	このコンピュータは日本語入力の時、いちいち入力方式を指定しないといけないので、面倒なことこの上ない。	
\\	このコンピュータは 日本語[にほんご] 入力[にゅうりょく]の 時[とき]、いちいち 入力[にゅうりょく] 方式[ほうしき]を 指定[してい]しないといけないので、 面倒[めんどう]なことこの 上[うえ]ない。
\\	新しく入ったアパートはキッチンの流しが高くて、私のように背の低いものは使づらいことこの上ない。	
\\	新[あたら]しく 入[はい]ったアパートはキッチンの 流[なが]しが 高[たか]くて、 私[わたし]のように 背[せ]の 低[ひく]いものは 使[し]づらいことこの 上[うえ]ない。
\\	一人でハワイに来て、この上ない解放感に満たされている。	
\\	一人[ひとり]でハワイに 来[き]て、この 上[うえ]ない 解放[かいほう] 感[かん]に 満[み]たされている。
\\	ご出席いただいた皆様からこの上ないお褒めの言葉をいただきました。	
\\	ご 出席[しゅっせき]いただいた 皆様[みなさま]からこの 上[うえ]ないお 褒[ほ]めの 言葉[ことば]をいただきました。
\\	彼とやっと一緒の生活を始めることができ、この上ない幸福感に浸っている。	
\\	彼[かれ]とやっと 一緒[いっしょ]の 生活[せいかつ]を 始[はじ]めることができ、この 上[うえ]ない 幸福[こうふく] 感[かん]に 浸[ひた]っている。
\\	この国は、他の国の批判に耳を貸すこともなく、この上ないごう慢さをもって自国のイデオロギーを主張し続けている。	
\\	この 国[くに]は、 他[ほか]の 国[くに]の 批判[ひはん]に 耳[みみ]を 貸[か]すこともなく、この 上[うえ]ないごう 慢[まん]さをもって 自国[じこく]のイデオロギーを 主張[しゅちょう]し 続[つづ]けている。
\\	日本の選挙カーはうるさいことこの上ない。	
\\	日本[にほん]の 選挙[せんきょ]カーはうるさいことこの 上[うえ]ない。
\\	旧友と酒を飲みながら話すのは愉快この上ない。	
\\	旧友[きゅうゆう]と 酒[さけ]を 飲[の]みながら 話[はな]すのは 愉快[ゆかい]この 上[うえ]ない。
\\	皇室に待望の男の子が生まれて喜ばしいことこの上ない。	
\\	皇室[こうしつ]に 待望[たいぼう]の 男の子[おとこのこ]が 生[う]まれて 喜[よろこ]ばしいことこの 上[うえ]ない。
\\	ここはバスが1時間に1本しか来ないので不便極まりない。	
\\	ここはバスが 
\\	時間[じかん]に 
\\	本[ほん]しか 来[こ]ないので 不便[ふべん] 極[きわ]まりない。
\\	銃を子供が勝手に持ち出せるようなところに置いておくのは危険極まりない。	
\\	銃[じゅう]を 子供[こども]が 勝手[かって]に 持ち出[もちだ]せるようなところに 置[お]いておくのは 危険[きけん] 極[きわ]まりない。
\\	カルフォルニアは何と太陽が明るいことか。	
\\	カルフォルニアは 何[なん]と 太陽[たいよう]が 明[あか]るいことか。
\\	私は今までに何度オーストラリアを訪ねたことか。	
\\	私[わたし]は 今[いま]までに 何[なん] 度[ど]オーストラリアを 訪[たず]ねたことか。
\\	アメリカの大学に比べると日本の大学は何と貧弱に見えることか。	
\\	アメリカの 大学[だいがく]に 比[くら]べると 日本[にほん]の 大学[だいがく]は 何[なに]と 貧弱[ひんじゃく]に 見[み]えることか。
\\	ペダルを踏み替える時に膝がハンドルにぶつかる車がどんなに多いことか。	
\\	ペダルを 踏[ふ]み 替[か]える 時[とき]に 膝[ひざ]がハンドルにぶつかる 車[くるま]がどんなに 多[おお]いことか。
\\	日本語がよく読めるようになるためにどれぐらい勉強したことか。	
\\	日本語[にほんご]がよく 読[よ]めるようになるためにどれぐらい 勉強[べんきょう]したことか。
\\	新鮮な魚を探して何軒の魚屋を見て歩いたことか。	
\\	新鮮[しんせん]な 魚[さかな]を 探[さが]して 何[なん] 軒[けん]の 魚屋[さかなや]を 見[み]て 歩[ある]いたことか。
\\	この数学の問題を解くのに何時間使ったことか。	
\\	この 数学[すうがく]の 問題[もんだい]を 解[と]くのに 何[なん] 時間[じかん] 使[つか]ったことか。
\\	日本人はいかにしばしばうちと外の行動を区別することか。	
\\	日本人[にほんじん]はいかにしばしばうちと 外[そと]の 行動[こうどう]を 区別[くべつ]することか。
\\	二人はどれだけ強く愛し合っていたことか。	
\\	二人[ふたり]はどれだけ 強[つよ]く 愛し合[あいしあ]っていたことか。
\\	彼はアリバイが成立しなかったことから犯行の嫌疑がかかった。	
\\	彼[かれ]はアリバイが 成立[せいりつ]しなかったことから 犯行[はんこう]の 嫌疑[けんぎ]がかかった。
\\	妙なことから二人の関係は悪化した。	
\\	妙[みょう]なことから 二人[ふたり]の 関係[かんけい]は 悪化[あっか]した。
\\	その教授は学生に高圧的だったことから、学生にひどく嫌われていた。	
\\	その 教授[きょうじゅ]は 学生[がくせい]に 高圧[こうあつ] 的[てき]だったことから、 学生[がくせい]にひどく 嫌[きら]われていた。
\\	今年は大学創立から250年に当たることから、盛大な記念祝典が開かれた。	
\\	今年[ことし]は 大学[だいがく] 創立[そうりつ]から 
\\	年[ねん]に 当[あ]たることから、 盛大[せいだい]な 記念[きねん] 祝典[しゅくてん]が 開[ひら]かれた。
\\	私が余計なことを言ったことから議論が紛糾した。	
\\	私[わたし]が 余計[よけい]なことを 言[い]ったことから 議論[ぎろん]が 紛糾[ふんきゅう]した。
\\	魚を水槽内で飼って行う研究は、長期的な観察が可能であることから、汎用性のあるデータが取得しやすい。	
\\	魚[さかな]を 水槽[すいそう] 内[ない]で 飼[か]って 行[おこな]う 研究[けんきゅう]は、 長期[ちょうき] 的[てき]な 観察[かんさつ]が 可能[かのう]であることから、 汎用[はんよう] 性[せい]のあるデータが 取得[しゅとく]しやすい。
\\	首相は不用意な言動が多かったことから、不信任案が議会で可決された。	
\\	首相[しゅしょう]は 不用意[ふようい]な 言動[げんどう]が 多[おお]かったことから、 不信任[ふしんにん] 案[あん]が 議会[ぎかい]で 可決[かけつ]された。
\\	本当にささいなことから会談は中止になってしまった。	
\\	本当[ほんとう]にささいなことから 会談[かいだん]は 中止[ちゅうし]になってしまった。
\\	彼が一つのアイディアについて話したことからプロジェクトが始まった。	
\\	彼[かれ]が 一[ひと]つのアイディアについて 話[はな]したことからプロジェクトが 始[はじ]まった。
\\	彼が話したことから、プロジェクトが始まった。	
\\	彼[かれ]が 話[はな]したことから、プロジェクトが 始[はじ]まった。
\\	嬉しいことに、今学期は祭日が3日もある。	
\\	嬉[うれ]しいことに、 今[こん] 学期[がっき]は 祭日[さいじつ]が 
\\	日[にち]もある。
\\	残念なことに、この本は絶版で、普通の本屋では手に入らない。	
\\	残念[ざんねん]なことに、この 本[ほん]は 絶版[ぜっぱん]で、 普通[ふつう]の 本屋[ほんや]では 手[て]に 入[はい]らない。
\\	驚いたことに、村田さんはあんなに気に入っていた会社を半年で辞めてしまった。	
\\	驚[おどろ]いたことに、 村田[むらた]さんはあんなに 気に入[きにい]っていた 会社[かいしゃ]を 半年[はんとし]で 辞[や]めてしまった。
\\	恥ずかしいことに、私は英文学専攻でありながら英語が話せない。	
\\	恥[は]ずかしいことに、 私[わたし]は 英[えい] 文学[ぶんがく] 専攻[せんこう]でありながら 英語[えいご]が 話[はな]せない。
\\	面白いことに、ハワイの日系人の方が日本人より日本の古い伝統をよく守っているという。	
\\	面白[おもしろ]いことに、ハワイの 日系[にっけい] 人[じん]の 方[ほう]が 日本人[にほんじん]より 日本[にほん]の 古[ふる]い 伝統[でんとう]をよく 守[まも]っているという。
\\	悪いことに、オーディションの前の日に風邪を引いてしまった。	
\\	悪[わる]いことに、オーディションの 前[まえ]の 日[ひ]に 風邪[かぜ]を 引[ひ]いてしまった。
\\	悔しいことに、私はロジャーに負けた。	
\\	悔[くや]しいことに、 私[わたし]はロジャーに 負[ま]けた。
\\	吉岡さんのころの悦子ちゃんは、感心なことに、お母さんが仕事で遅くなる時は自分で晩ご飯を作るんだそうです。	
\\	吉岡[よしおか]さんのころの 悦子[えつこ]ちゃんは、 感心[かんしん]なことに、お 母[かあ]さんが 仕事[しごと]で 遅[おそ]くなる 時[とき]は 自分[じぶん]で 晩[ばん]ご 飯[はん]を 作[つく]るんだそうです。
\\	浄水器は濾過の過程で塩素をすべて取り去ってしまうため、濾過した瞬間から水は腐敗の危険にさらされる。さらに心配なことに、濾過材に残った水から大量に細菌が繁殖するといった問題が報道された。	
\\	浄水[じょうすい] 器[き]は 濾過[ろか]の 過程[かてい]で 塩素[えんそ]をすべて 取り去[とりさ]ってしまうため、 濾過[ろか]した 瞬間[しゅんかん]から 水[みず]は 腐敗[ふはい]の 危険[きけん]にさらされる。さらに 心配[しんぱい]なことに、 濾過[ろか] 材[ざい]に 残[のこ]った 水[みず]から 大量[たいりょう]に 細菌[さいきん]が 繁殖[はんしょく]するといった 問題[もんだい]が 報道[ほうどう]された。
\\	厄介なことに、この種の詐欺は被害者が世間体のために警察に届けを出さないことが多く、これが犯人検挙を難しくしている。	
\\	厄介[やっかい]なことに、この 種[しゅ]の 詐欺[さぎ]は 被害[ひがい] 者[しゃ]が 世間体[せけんてい]のために 警察[けいさつ]に 届[とど]けを 出[だ]さないことが 多[おお]く、これが 犯人[はんにん] 検挙[けんきょ]を 難[むずか]しくしている。
\\	困ったことに、日本の
\\	はアメリカのプレーヤーでは見られないのです。	
\\	困[こま]ったことに、 日本[にほん]の 
\\	はアメリカのプレーヤーでは 見[み]られないのです。
\\	腹の立つことに、私のフライトは5時間も待たされたあげく、キャンセルされた。	
\\	腹[はら]の 立[た]つことに、 私[わたし]のフライトは 
\\	時間[じかん]も 待[ま]たされたあげく、キャンセルされた。
\\	悲しいことに、彼は私を全然覚えていなかった。	
\\	悲[かな]しいことに、 彼[かれ]は 私[わたし]を 全然[ぜんぜん] 覚[おぼ]えていなかった。
\\	幸いだったことに、私はその時携帯電話を持っていた。	
\\	幸[さいわ]いだったことに、 私[わたし]はその 時[とき] 携帯[けいたい] 電話[でんわ]を 持[も]っていた。
\\	恥ずかしいことに、ジョンは泳げなかった。	
\\	恥[は]ずかしいことに、ジョンは 泳[およ]げなかった。
\\	面白いことに、ケンは自分が日本人であることを最近まで知らなかった。	
\\	面白[おもしろ]いことに、ケンは 自分[じぶん]が 日本人[にほんじん]であることを 最近[さいきん]まで 知[し]らなかった。
\\	久保先生はその本が絶版で普通の本屋では手に入らないことを残念に思っている。	
\\	久保[くぼ] 先生[せんせい]はその 本[ほん]が 絶版[ぜっぱん]で 普通[ふつう]の 本屋[ほんや]では 手[て]に 入[はい]らないことを 残念[ざんねん]に 思[おも]っている。
\\	渡辺さんは、村田さんがあんなに気に入っていた会社を半年で辞めてしまったので驚いた。	
\\	渡辺[わたなべ]さんは、 村田[むらた]さんがあんなに 気に入[きにい]っていた 会社[かいしゃ]を 半年[はんとし]で 辞[や]めてしまったので 驚[おどろ]いた。
\\	アンディーはロジャーに負けて悔しがった。	
\\	アンディーはロジャーに 負[ま]けて 悔[くや]しがった。
\\	彼は、5時間も待たされたあげくフライトがキャンセルされて、腹を立てている。	
\\	彼[かれ]は、 
\\	時間[じかん]も 待[ま]たされたあげくフライトがキャンセルされて、 腹[はら]を 立[た]てている。
\\	幸いにも、私はその時携帯電話を持っていた。	
\\	幸[さいわ]いにも、 私[わたし]はその 時[とき] 携帯[けいたい] 電話[でんわ]を 持[も]っていた。
\\	感心にも、吉岡さんのところの悦子ちゃんは、お母さんが仕事で遅くなる時は自分で晩ご飯を作るんだそうです。	
\\	感心[かんしん]にも、 吉岡[よしおか]さんのところの 悦子[えつこ]ちゃんは、お 母[かあ]さんが 仕事[しごと]で 遅[おそ]くなる 時[とき]は 自分[じぶん]で 晩[ばん]ご 飯[はん]を 作[つく]るんだそうです。
\\	この冬のアメリカ東部は雪、雨、加えて強風の被害が大きかった。	
\\	この 冬[ふゆ]のアメリカ 東部[とうぶ]は 雪[ゆき]、 雨[あめ]、 加[くわ]えて 強風[きょうふう]の 被害[ひがい]が 大[おお]きかった。
\\	この辺は物価が安い。加えて自然環境もすばらしい。	
\\	この 辺[へん]は 物価[ぶっか]が 安[やす]い。 加[くわ]えて 自然[しぜん] 環境[かんきょう]もすばらしい。
\\	この作家の小説は想像力の豊かさ、鮮明な文体、加えて、極めて音楽的なリズム感がある。	
\\	この 作家[さっか]の 小説[しょうせつ]は 想像[そうぞう] 力[りょく]の 豊[ゆた]かさ、 鮮明[せんめい]な 文体[ぶんたい]、 加[くわ]えて、 極[きわ]めて 音楽[おんがく] 的[てき]なリズム 感[かん]がある。
\\	金融業の倒産、失業率の上昇、加えて、不安定な政治状況によって、この国の経済は沈滞している。	
\\	金融[きんゆう] 業[ぎょう]の 倒産[とうさん]、 失業[しつぎょう] 率[りつ]の 上昇[じょうしょう]、 加[くわ]えて、 不安定[ふあんてい]な 政治[せいじ] 状況[じょうきょう]によって、この 国[くに]の 経済[けいざい]は 沈滞[ちんたい]している。
\\	健康維持には正しい食事、適度な運動、加えて人との暖かいコミュニケーションが必要であろう。	
\\	健康[けんこう] 維持[いじ]には 正[ただ]しい 食事[しょくじ]、 適度[てきど]な 運動[うんどう]、 加[くわ]えて 人[ひと]との 暖[あたた]かいコミュニケーションが 必要[ひつよう]であろう。
\\	彼女はおしゃべりが好きで、とても明るい。それに加えて、なかなかユニークなキャラクターだ。	
\\	彼女[かのじょ]はおしゃべりが 好[す]きで、とても 明[あか]るい。それに 加[くわ]えて、なかなかユニークなキャラクターだ。
\\	北海道の秋は紅葉がきれいだ。加えて、空気がからっと乾燥していて気持ちがいい。	
\\	北海道[ほっかいどう]の 秋[あき]は 紅葉[こうよう]がきれいだ。 加[くわ]えて、 空気[くうき]がからっと 乾燥[かんそう]していて 気持[きも]ちがいい。
\\	京都は神社仏閣がたくさんある。加えて、魅力的な日本庭園も多い。	
\\	京都[きょうと]は 神社[じんじゃ] 仏閣[ぶっかく]がたくさんある。 加[くわ]えて、 魅力[みりょく] 的[てき]な 日本[にほん] 庭園[ていえん]も 多[おお]い。
\\	もし
\\	の点が悪かったらもう一度受けるまでだ。	
\\	もし 
\\	の 点[てん]が 悪[わる]かったらもう 一度[いちど] 受[う]けるまでだ。
\\	日本語は面白そうだったので、取ったまでのことだ。	
\\	日本語[にほんご]は 面白[おもしろ]そうだったので、 取[と]ったまでのことだ。
\\	結婚できなかったら、一生独身で仕事をするまでのことだ。	
\\	結婚[けっこん]できなかったら、 一生[いっしょう] 独身[どくしん]で 仕事[しごと]をするまでのことだ。
\\	就職できなかったら、フリーターになるまでのことだ。	
\\	就職[しゅうしょく]できなかったら、フリーターになるまでのことだ。
\\	バスが来ないんなら歩いて帰るまでだ。	
\\	バスが 来[こ]ないんなら 歩[ある]いて 帰[かえ]るまでだ。
\\	夕食にどうぞと言われたから彼女のアパートに行ったまでだ。とやかく言われる筋合いはない。	
\\	夕食[ゆうしょく]にどうぞと 言[い]われたから 彼女[かのじょ]のアパートに 行[い]ったまでだ。とやかく 言[い]われる 筋合[すじあ]いはない。
\\	小説みたいなものを書いたが、別に小説家になりたいと思ったのではない。小説を書くということがどんなことか知りたかったまでのことだ。	
\\	小説[しょうせつ]みたいなものを 書[か]いたが、 別[べつ]に 小説[しょうせつ] 家[か]になりたいと 思[おも]ったのではない。 小説[しょうせつ]を 書[か]くということがどんなことか 知[し]りたかったまでのことだ。
\\	あなたに私の個人的なことを話したのはあなたの将来に多少役立つかと思ったまでのことです。	
\\	あなたに 私[わたし]の 個人[こじん] 的[てき]なことを 話[はな]したのはあなたの 将来[しょうらい]に 多少[たしょう] 役立[やくだ]つかと 思[おも]ったまでのことです。
\\	これは簡単な書類です。ここに名前と住所を書くだけのことです。	
\\	これは 簡単[かんたん]な 書類[しょるい]です。ここに 名前[なまえ]と 住所[じゅうしょ]を 書[か]くだけのことです。
\\	健康になるには規則的な生活をするだけのことなんです。	
\\	健康[けんこう]になるには 規則[きそく] 的[てき]な 生活[せいかつ]をするだけのことなんです。
\\	アパートの家賃を払うのが精一杯なのに、ましてや家を買うなんて夢物語だ。	
\\	アパートの 家賃[やちん]を 払[はら]うのが 精一杯[せいいっぱい]なのに、ましてや 家[いえ]を 買[か]うなんて 夢物語[ゆめものがたり]だ。
\\	彼女は日本語で講演もできる。まして日常会話など問題ではない。	
\\	彼女[かのじょ]は 日本語[にほんご]で 講演[こうえん]もできる。まして 日常[にちじょう] 会話[かいわ]など 問題[もんだい]ではない。
\\	この小説は大人にも読みにくいのに、まして小学生に読めるはずがない。	
\\	この 小説[しょうせつ]は 大人[おとな]にも 読[よ]みにくいのに、まして 小学生[しょうがくせい]に 読[よ]めるはずがない。
\\	彼は簡単な日本語すら話せない。ましてや、書くことなど思いもよらない。	
\\	彼[かれ]は 簡単[かんたん]な 日本語[にほんご]すら 話[はな]せない。ましてや、 書[か]くことなど 思[おも]いもよらない。
\\	僕は自転車さえ買えない。まして、車を買うなんて不可能だ。	
\\	僕[ぼく]は 自転車[じてんしゃ]さえ 買[か]えない。まして、 車[くるま]を 買[か]うなんて 不可能[ふかのう]だ。
\\	国内での単身赴任でもつらい。ましてや、外国での単身赴任となるともってつらい。	
\\	国内[こくない]での 単身[たんしん] 赴任[ふにん]でもつらい。ましてや、 外国[がいこく]での 単身[たんしん] 赴任[ふにん]となるともってつらい。
\\	この数学の問題は教授でも解けないのだから、まして、学生の僕に解けるはずがない。	
\\	この 数学[すうがく]の 問題[もんだい]は 教授[きょうじゅ]でも 解[と]けないのだから、まして、 学生[がくせい]の 僕[ぼく]に 解[と]けるはずがない。
\\	65歳の人が毎日1キロ泳いでいるんです。まして、32歳のあなたが同じことができないわけがないですよ。	
\\	歳[さい]の 人[ひと]が 毎日[まいにち]1キロ 泳[およ]いでいるんです。まして、32 歳[さい]のあなたが 同[おな]じことができないわけがないですよ。
\\	ニューヨークのような大都市でも犯罪率を下げることに成功している。ましてや、中小都市なら十分可能だろう。	
\\	ニューヨークのような 大都市[だいとし]でも 犯罪[はんざい] 率[りつ]を 下[さ]げることに 成功[せいこう]している。ましてや、 中小[ちゅうしょう] 都市[とし]なら 十分[じゅうぶん] 可能[かのう]だろう。
\\	このちっぽけな蟻だって協力し合っているんです。まして、頭脳の発達した人間にそれができないわけはないでしょう。	
\\	このちっぽけな 蟻[あり]だって 協力[きょうりょく]し 合[あ]っているんです。まして、 頭脳[ずのう]の 発達[はったつ]した 人間[にんげん]にそれができないわけはないでしょう。
\\	家を買うどころか、アパートの家賃を払うのが精一杯だ。	
\\	家[いえ]を 買[か]うどころか、アパートの 家賃[やちん]を 払[はら]うのが 精一杯[せいいっぱい]だ。
\\	この小説は小学生に読めるどころか、大人にも読みにくい。	
\\	この 小説[しょうせつ]は 小学生[しょうがくせい]に 読[よ]めるどころか、 大人[おとな]にも 読[よ]みにくい。
\\	彼女は日常会話など問題ではないどころか、日本語で講演もできる。	
\\	彼女[かのじょ]は 日常[にちじょう] 会話[かいわ]など 問題[もんだい]ではないどころか、 日本語[にほんご]で 講演[こうえん]もできる。
\\	来年の春、またお会いします。	
\\	来年[らいねん]の 春[はる]、またお 会[あ]いします。
\\	山田も東京出身だが、林もまた東京出身だ。	
\\	山田[やまだ]も 東京[とうきょう] 出身[しゅっしん]だが、 林[はやし]もまた 東京[とうきょう] 出身[しゅっしん]だ。
\\	机に座ってものを考えるのもいいが、歩きながら考えるのもまたいい。	
\\	机[つくえ]に 座[すわ]ってものを 考[かんが]えるのもいいが、 歩[ある]きながら 考[かんが]えるのもまたいい。
\\	この日本語の教科書は内容も面白く、また文法の記述が正確だ。	
\\	この 日本語[にほんご]の 教科書[きょうかしょ]は 内容[ないよう]も 面白[おもしろ]く、また 文法[ぶんぽう]の 記述[きじゅつ]が 正確[せいかく]だ。
\\	発表は英語でしてもよい。また、日本語でしてもよい。	
\\	発表[はっぴょう]は 英語[えいご]でしてもよい。また、 日本語[にほんご]でしてもよい。
\\	発表は英語でしてもよいし、また、日本語でしてもよい。	
\\	発表[はっぴょう]は 英語[えいご]でしてもよいし、また、 日本語[にほんご]でしてもよい。
\\	今日もまた雪が降った。	
\\	今日[きょう]もまた 雪[ゆき]が 降[ふ]った。
\\	また風邪をひいちゃった。	
\\	また 風邪[かぜ]をひいちゃった。
\\	大統領もまた人間だ。過ちを犯すこともあるだろう。	
\\	大統領[だいとうりょう]もまた 人間[にんげん]だ。 過[あやま]ちを 犯[おか]すこともあるだろう。
\\	大学で勉強をするのはもちろん大切だが、生涯の友達を作るのもまた大切だ。	
\\	大学[だいがく]で 勉強[べんきょう]をするのはもちろん 大切[たいせつ]だが、 生涯[しょうがい]の 友達[ともだち]を 作[つく]るのもまた 大切[たいせつ]だ。
\\	彼は優れた研究者であり、また、優れた教育者でもある。	
\\	彼[かれ]は 優[すぐ]れた 研究[けんきゅう] 者[しゃ]であり、また、 優[すぐ]れた 教育[きょういく] 者[しゃ]でもある。
\\	日本は失業率が高くなってきている。また、株価も急落している。	
\\	日本[にほん]は 失業[しつぎょう] 率[りつ]が 高[たか]くなってきている。また、 株価[かぶか]も 急落[きゅうらく]している。
\\	この会議には出席してもよい、また、出席しなくてもよい。	
\\	この 会議[かいぎ]には 出席[しゅっせき]してもよい、また、 出席[しゅっせき]しなくてもよい。
\\	日中間の相互理解を図る上で、文化交流は大変重要だし、また大変有益だ。	
\\	日[にち] 中間[ちゅうかん]の 相互[そうご] 理解[りかい]を 図[はか]る 上[うえ]で、 文化[ぶんか] 交流[こうりゅう]は 大変[たいへん] 重要[じゅうよう]だし、また 大変[たいへん] 有益[ゆうえき]だ。
\\	山田か田中が代表で会議に出る。	
\\	山田[やまだ]か 田中[たなか]が 代表[だいひょう]で 会議[かいぎ]に 出[で]る。
\\	電子メールか、またはファックスで送って下さい。	
\\	電子[でんし]メールか、またはファックスで 送[おく]って 下[くだ]さい。
\\	試験を受けるかレポートを提出しなければならない。	
\\	試験[しけん]を 受[う]けるかレポートを 提出[ていしゅつ]しなければならない。
\\	この辺りでは地震は滅多に起きない。	
\\	この 辺[あた]りでは 地震[じしん]は 滅多[めった]に 起[お]きない。
\\	日本の父親は忙しいので、週末でも滅多にうちにいない。	
\\	日本[にほん]の 父親[ちちおや]は 忙[いそが]しいので、 週末[しゅうまつ]でも 滅多[めった]にうちにいない。
\\	大学時代の同級生とは滅多に会えない。	
\\	大学[だいがく] 時代[じだい]の 同級生[どうきゅうせい]とは 滅多[めった]に 会[あ]えない。
\\	日本の車は滅多に故障しないので、海外で人気がある。	
\\	日本[にほん]の 車[くるま]は 滅多[めった]に 故障[こしょう]しないので、 海外[かいがい]で 人気[にんき]がある。
\\	汚染のためにこの川では滅多に魚が泳いでいるのを見られなくなった。	
\\	汚染[おせん]のためにこの 川[かわ]では 滅多[めった]に 魚[さかな]が 泳[およ]いでいるのを 見[み]られなくなった。
\\	ジョンは滅多に日本語を勉強しません。	
\\	ジョンは 滅多[めった]に 日本語[にほんご]を 勉強[べんきょう]しません。
\\	ジョンはたまには日本語を勉強します。	
\\	ジョンはたまには 日本語[にほんご]を 勉強[べんきょう]します。
\\	最近彼はあまりゴルフをやっていない。	
\\	最近[さいきん] 彼[かれ]はあまりゴルフをやっていない。
\\	この辞書はあまり役に立たない。	
\\	この 辞書[じしょ]はあまり 役に立[やくにた]たない。
\\	忙しくてほとんど映画を見ていない。	
\\	忙[いそが]しくてほとんど 映画[えいが]を 見[み]ていない。
\\	中国のことはほとんど何も知らない。	
\\	中国[ちゅうごく]のことはほとんど 何[なに]も 知[し]らない。
\\	僕は全然運動をしていない。	
\\	僕[ぼく]は 全然[ぜんぜん] 運動[うんどう]をしていない。
\\	彼女は政治には全く興味がない。	
\\	彼女[かのじょ]は 政治[せいじ]には 全[まった]く 興味[きょうみ]がない。
\\	太陽の光をたっぷり浴びたみかんは、見るからにおいしそうだ。	
\\	太陽[たいよう]の 光[ひかり]をたっぷり 浴[あ]びたみかんは、 見[み]るからにおいしそうだ。
\\	彼女が身につけているものは見るからに高級品だった。	
\\	彼女[かのじょ]が 身[み]につけているものは 見[み]るからに 高級[こうきゅう] 品[ひん]だった。
\\	このパンは見るからに手作りという感じですね。	
\\	このパンは 見[み]るからに 手作[てづく]りという 感[かん]じですね。
\\	その青年は見るからに聡明そうだった。	
\\	その 青年[せいねん]は 見[み]るからに 聡明[そうめい]そうだった。
\\	老人が見るからに重そうな買い物袋をかけて歩いている。	
\\	老人[ろうじん]が 見[み]るからに 重[おも]そうな 買い物[かいもの] 袋[ぶくろ]をかけて 歩[ある]いている。
\\	その写真には見るからに寂しそうな孤児たちが写っていた。	
\\	その 写真[しゃしん]には 見[み]るからに 寂[さび]しそうな 孤児[こじ]たちが 写[うつ]っていた。
\\	阪神・淡路大震災の時は見るからに新聞記者らしい人たちが大勢写真を撮っていた。	
\\	阪神[はんしん]・ 淡路[あわじ] 大震災[だいしんさい]の 時[とき]は 見[み]るからに 新聞[しんぶん] 記者[きしゃ]らしい 人[ひと]たちが 大勢[おおぜい] 写真[しゃしん]を 撮[と]っていた。
\\	彼の顔は青黒くむくみ、見るからに健康を害していた。	
\\	彼[かれ]の 顔[かお]は 青黒[あおぐろ]くむくみ、 見[み]るからに 健康[けんこう]を 害[がい]していた。
\\	見るからにジューシーなお肉にかぶりつくと、うまみが口の中に広がていった。	
\\	見[み]るからにジューシーなお 肉[にく]にかぶりつくと、うまみが 口[くち]の 中[なか]に 広[ひろ]がていった。
\\	このスープは野菜たっぷりで、見るからに体によさそう!	
\\	このスープは 野菜[やさい]たっぷりで、 見[み]るからに 体[からだ]によさそう!
\\	この男の人、見るからに誠実で優しそうね。	
\\	この 男[おとこ]の 人[ひと]、 見[み]るからに 誠実[せいじつ]で 優[やさ]しそうね。
\\	ゆったりした音楽を聞きながらたっぷりマッサージを受けるなんて、聞くからにリラックスできるそうです。	
\\	ゆったりした 音楽[おんがく]を 聞[き]きながらたっぷりマッサージを 受[う]けるなんて、 聞[き]くからにリラックスできるそうです。
\\	「地獄谷」というのは、聞くからに身の毛がよだつ名前ですね。	
\\	地獄谷[じごくだに]」というのは、 聞[き]くからに 身の毛[みのけ]がよだつ 名前[なまえ]ですね。
\\	ー失礼ですが、千鶴子さんのお父さまですか。 ーいかにも私は千鶴子の父親ですが、それがどうかしましたか。	
\\	ー 失礼[しつれい]ですが、 千鶴子[ちずこ]さんのお 父[とう]さまですか。 ーいかにも 私[わたし]は 千鶴子[ちずこ]の 父親[ちちおや]ですが、それがどうかしましたか。
\\	父は感情的な人ではないが、時に怒りもすれば泣きもする。	
\\	父[ちち]は 感情[かんじょう] 的[てき]な 人[ひと]ではないが、 時[とき]に 怒[おこ]りもすれば 泣[な]きもする。
\\	私だってロックも聞けばダンスもする。	
\\	私[わたし]だってロックも 聞[き]けばダンスもする。
\\	日本全国が担当地域なので北海道へも行けば沖縄へも行く。	
\\	日本[にほん] 全国[ぜんこく]が 担当[たんとう] 地域[ちいき]なので 北海道[ほっかいどう]へも 行[い]けば 沖縄[おきなわ]へも 行[い]く。
\\	交通事情によって30分で行けることもあれば1時間以上かかることもある。	
\\	交通[こうつう] 事情[じじょう]によって 30分[さんじゅっぷん]で 行[い]けることもあれば 
\\	時間[じかん] 以上[いじょう]かかることもある。
\\	円満な夫婦は口論もすれば不満も言い合う。	
\\	円満[えんまん]な 夫婦[ふうふ]は 口論[こうろん]もすれば 不満[ふまん]も 言い合[いいあ]う。
\\	私は子供をよく可愛いがるが、悪いことをしたら怒りもすれば叱りもする。	
\\	私[わたし]は 子供[こども]をよく 可愛[かわい]いがるが、 悪[わる]いことをしたら 怒[おこ]りもすれば 叱[しか]りもする。
\\	この学会では英語で発表する人もいれば日本語で発表する人もいる。	
\\	この 学会[がっかい]では 英語[えいご]で 発表[はっぴょう]する 人[ひと]もいれば 日本語[にほんご]で 発表[はっぴょう]する 人[ひと]もいる。
\\	人生は晴れの日もあれば雨の日もある。	
\\	人生[じんせい]は 晴[は]れの 日[ひ]もあれば 雨[あめ]の 日[ひ]もある。
\\	うちの会社は服装の規則がないので、スーツを着て行く時もあればセーターで行く時もある。	
\\	うちの 会社[かいしゃ]は 服装[ふくそう]の 規則[きそく]がないので、スーツを 着[き]て 行[い]く 時[とき]もあればセーターで 行[い]く 時[とき]もある。
\\	担当の先生が来られない時は、代わりの先生を頼むこともあれば、やむを得ず休講にすることもある。	
\\	担当[たんとう]の 先生[せんせい]が 来[こ]られない 時[とき]は、 代[か]わりの 先生[せんせい]を 頼[たの]むこともあれば、やむを 得[え]ず 休講[きゅうこう]にすることもある。
\\	アイディアが突然湧き上がってくる時もあれば、全く出てこない時もある。	
\\	アイディアが 突然[とつぜん] 湧[わ]き 上[あ]がってくる 時[とき]もあれば、 全[まった]く 出[で]てこない 時[とき]もある。
\\	外国語は勉強の仕方次第で、面白くもなればつまらなくもなる。	
\\	外国[がいこく] 語[ご]は 勉強[べんきょう]の 仕方[しかた] 次第[しだい]で、 面白[おもしろ]くもなればつまらなくもなる。
\\	父は感情的な人ではないが、時に怒ったり泣いたりする。	
\\	父[ちち]は 感情[かんじょう] 的[てき]な 人[ひと]ではないが、 時[とき]に 怒[おこ]ったり 泣[な]いたりする。
\\	私だってロックを聞いたりダンスをしたりする。	
\\	私[わたし]だってロックを 聞[き]いたりダンスをしたりする。
\\	日本全国が担当地域なので北海道へ行ったり沖縄へ行ったりする。	
\\	日本[にほん] 全国[ぜんこく]が 担当[たんとう] 地域[ちいき]なので 北海道[ほっかいどう]へ 行[い]ったり 沖縄[おきなわ]へ 行[い]ったりする。
\\	交通事情によって30分で行けたり1時間以上かかったりする。	
\\	交通[こうつう] 事情[じじょう]によって 30分[さんじゅっぷん]で 行[い]けたり 
\\	時間[じかん] 以上[いじょう]かかったりする。
\\	円満な夫婦は時には口論したり不満を言い合ったりする。	
\\	円満[えんまん]な 夫婦[ふうふ]は 時[とき]には 口論[こうろん]したり 不満[ふまん]を 言い合[いいあ]ったりする。
\\	夜遅く人に電話をかけるものではない。	
\\	夜[よる] 遅[おそ]く 人[ひと]に 電話[でんわ]をかけるものではない。
\\	彼の演奏は聴けたものではない。	
\\	彼[かれ]の 演奏[えんそう]は 聴[き]けたものではない。
\\	そんな時間のかかる仕事は引き受けられるものではない。	
\\	そんな 時間[じかん]のかかる 仕事[しごと]は 引き受[ひきう]けられるものではない。
\\	日本では先生と話す時に帽子をかぶったまま話すものではない。	
\\	日本[にほん]では 先生[せんせい]と 話[はな]す 時[とき]に 帽子[ぼうし]をかぶったまま 話[はな]すものではない。
\\	日本では人のうちに尋ねて行く時には手ぶらで行くものではありません。	
\\	日本[にほん]では 人[ひと]のうちに 尋[たず]ねて 行[い]く 時[とき]には 手[て]ぶらで 行[い]くものではありません。
\\	未成年者は酒を飲むもんじゃない。	
\\	未成年[みせいねん] 者[しゃ]は 酒[さけ]を 飲[の]むもんじゃない。
\\	女性に年齢を聞くものではない。	
\\	女性[じょせい]に 年齢[ねんれい]を 聞[き]くものではない。
\\	こんなつまらない小説、読めたもんじゃない。	
\\	こんなつまらない 小説[しょうせつ]、 読[よ]めたもんじゃない。
\\	私のかいた絵なんか、あなたに見せられたものではありません。	
\\	私[わたし]のかいた 絵[え]なんか、あなたに 見[み]せられたものではありません。
\\	辞書などはそう簡単に書けるものではない。	
\\	辞書[じしょ]などはそう 簡単[かんたん]に 書[か]けるものではない。
\\	現在の補助金の額で十分な研究ができるものではない。	
\\	現在[げんざい]の 補助[ほじょ] 金[きん]の 額[がく]で 十分[じゅうぶん]な 研究[けんきゅう]ができるものではない。
\\	彼の演奏はおそまつで聴けたものではない。	
\\	彼[かれ]の 演奏[えんそう]はおそまつで 聴[き]けたものではない。
\\	彼の演奏は滅多に聴けるものではない。	
\\	彼[かれ]の 演奏[えんそう]は 滅多[めった]に 聴[き]けるものではない。
\\	あんなやつとは二度と会うものか。	
\\	あんなやつとは 二度[にど]と 会[あ]うものか。
\\	ー彼は5年後には自分でコンピュータの会社をつくるって言っているよ。 ーそんなこと彼にできるもんか。	
\\	ー 彼[かれ]は 
\\	年[ねん] 後[ご]には 自分[じぶん]でコンピュータの 会社[かいしゃ]をつくるって 言[い]っているよ。 ーそんなこと 彼[かれ]にできるもんか。
\\	あんなケチな男とはもうつき合うものか。	
\\	あんなケチな 男[おとこ]とはもうつき 合[あ]うものか。
\\	ー鈴木、
\\	が650点だったってよ。 ーまさか!そんなことがあるもんか。	
\\	ー 鈴木[すずき]、 
\\	が 
\\	点[てん]だったってよ。 ーまさか!そんなことがあるもんか。
\\	世話をしてもお礼の一言も言わないの。あんな人、もう世話をするもんですか。	
\\	世話[せわ]をしてもお 礼[れい]の 一言[ひとこと]も 言[い]わないの。あんな 人[ひと]、もう 世話[せわ]をするもんですか。
\\	ー池田先生、親切な生成? ー親切なもんか。	
\\	ー 池田[いけだ] 先生[せんせい]、 親切[しんせつ]な 生成[せいせい]? ー 親切[しんせつ]なもんか。
\\	ーそのパズル、やさしいんだろう。 ーやさしいもんか。	
\\	ーそのパズル、やさしいんだろう。 ーやさしいもんか。
\\	ーあいつは自分のことをいっぱしの政治家だと思っているようだな。 ーあいつは政治家なもんか。政治屋じゃないか。	
\\	ーあいつは 自分[じぶん]のことをいっぱしの 政治[せいじ] 家[か]だと 思[おも]っているようだな。 ーあいつは 政治[せいじ] 家[か]なもんか。 政治[せいじ] 屋[や]じゃないか。
\\	ー黙って帰ったら彼に悪いだろう。 ーかまうもんか。	ー
\\	ー 黙[だま]って 帰[かえ]ったら 彼[かれ]に 悪[わる]いだろう。 ーかまうもんか。
\\	ーみんな、あの政治家は信頼できるって言っているよ。 ー分かるもんか。	
\\	ーみんな、あの 政治[せいじ] 家[か]は 信頼[しんらい]できるって 言[い]っているよ。 ー 分[わ]かるもんか。
\\	あんなやつとは二度と会うつもりはない。	
\\	あんなやつとは 二度[にど]と 会[あ]うつもりはない。
\\	絶対に親切じゃない。	
\\	絶対[ぜったい]に 親切[しんせつ]じゃない。
\\	もう少し広い家に住めないものだろうか。	
\\	もう 少[すこ]し 広[ひろ]い 家[いえ]に 住[す]めないものだろうか。
\\	修士論文の題目をどのように決めたものか迷っています。	
\\	修士[しゅうし] 論文[ろんぶん]の 題目[だいもく]をどのように 決[き]めたものか 迷[まよ]っています。
\\	もう少し真面目にこの問題に取り組んでくれないものか。	
\\	もう 少[すこ]し 真面目[まじめ]にこの 問題[もんだい]に 取り組[とりく]んでくれないものか。
\\	何とかしてこの原稿を月曜日までに書き終われないものか。	
\\	何[なん]とかしてこの 原稿[げんこう]を 月曜日[げつようび]までに 書[か]き 終[お]われないものか。
\\	もう少し自由に休暇が取れないものだろうか。	
\\	もう 少[すこ]し 自由[じゆう]に 休暇[きゅうか]が 取[と]れないものだろうか。
\\	東京の夏はやたらと暑い。もう少し涼しくならないものだろうか。	
\\	東京[とうきょう]の 夏[なつ]はやたらと 暑[あつ]い。もう 少[すこ]し 涼[すず]しくならないものだろうか。
\\	日本語を勉強したいんですが、どの大学を選んだものかと考えているところです。	
\\	日本語[にほんご]を 勉強[べんきょう]したいんですが、どの 大学[だいがく]を 選[えら]んだものかと 考[かんが]えているところです。
\\	今年の夏は海外旅行をしようと思っていますが、どの国に行ったものかと思案中です。	
\\	今年[ことし]の 夏[なつ]は 海外[かいがい] 旅行[りょこう]をしようと 思[おも]っていますが、どの 国[くに]に 行[い]ったものかと 思案[しあん] 中[ちゅう]です。
\\	子供が反抗期に入っている。その期間を問題なく乗り越えるにはどうしたものかと思っている。	
\\	子供[こども]が 反抗[はんこう] 期[き]に 入[はい]っている。その 期間[きかん]を 問題[もんだい]なく 乗り越[のりこ]えるにはどうしたものかと 思[おも]っている。
\\	このところからだの調子が悪いので、医者に診てもらったものかと考えているところです。	
\\	このところからだの 調子[ちょうし]が 悪[わる]いので、 医者[いしゃ]に 診[み]てもらったものかと 考[かんが]えているところです。
\\	日本の狭い社会から飛び出して海外移住でもしたものかと思っています。	
\\	日本[にほん]の 狭[せま]い 社会[しゃかい]から 飛び出[とびだ]して 海外[かいがい] 移住[いじゅう]でもしたものかと 思[おも]っています。
\\	人の性格は変わらないものだろうか。	
\\	人[ひと]の 性格[せいかく]は 変[か]わらないものだろうか。
\\	私も食事に注意しているのだから、少しはやせないものだろうか。	
\\	私[わたし]も 食事[しょくじ]に 注意[ちゅうい]しているのだから、 少[すこ]しはやせないものだろうか。
\\	もうちょっと雨が降らないものだろうか。	
\\	もうちょっと 雨[あめ]が 降[ふ]らないものだろうか。
\\	もう少し広い家に住めたらいいのに。	
\\	もう 少[すこ]し 広[ひろ]い 家[いえ]に 住[す]めたらいいのに。
\\	何とかしてこの原稿を月曜日までに書き終われたらいいのだが。	
\\	何[なん]とかしてこの 原稿[げんこう]を 月曜日[げつようび]までに 書[か]き 終[お]われたらいいのだが。
\\	もう少し休暇が自由取れたらいいのに。	
\\	もう 少[すこ]し 休暇[きゅうか]が 自由[じゆう] 取[と]れたらいいのに。
\\	もし一戸建ての家が買えるものなら買いたい。	
\\	もし 一戸建[いっこだ]ての 家[いえ]が 買[か]えるものなら 買[か]いたい。
\\	一人でやれるものならやってみろ!	
\\	一人[ひとり]でやれるものならやってみろ!
\\	近い将来における日本語の存在意義など、書けるものなら書いてみたい。	
\\	近[ちか]い 将来[しょうらい]における 日本語[にほんご]の 存在[そんざい] 意義[いぎ]など、 書[か]けるものなら 書[か]いてみたい。
\\	喫煙所の中でたばこを吸っている人たちを外から見ればまるで動物園……。もしやめられるものならやめてみようかなあと思い始めたこの頃です。	
\\	喫煙[きつえん] 所[しょ]の 中[なか]でたばこを 吸[す]っている 人[ひと]たちを 外[そと]から 見[み]ればまるで 動物[どうぶつ] 園[えん]……。もしやめられるものならやめてみようかなあと 思[おも]い 始[はじ]めたこの 頃[ころ]です。
\\	この新刊、もし町の図書館で借りられるものなら買わずに済むのに、まだ入っていないようだ。	
\\	この 新刊[しんかん]、もし 町[まち]の 図書館[としょかん]で 借[か]りられるものなら 買[か]わずに 済[す]むのに、まだ 入[はい]っていないようだ。
\\	南極旅行は、気持ち的には、借金してでも行けるものなら行きたいんだが。	
\\	南極[なんきょく] 旅行[りょこう]は、 気持[きも]ち 的[てき]には、 借金[しゃっきん]してでも 行[い]けるものなら 行[い]きたいんだが。
\\	もしできるものなら、もう一度生まれ変わりたい。	
\\	もしできるものなら、もう 一度[いちど] 生まれ変[うまれか]わりたい。
\\	西行じゃないけれど、願いがかなうものなら、満開の桜の下で春に死にたい。	
\\	西行[さいぎょう]じゃないけれど、 願[ねが]いがかなうものなら、 満開[まんかい]の 桜[さくら]の 下[した]で 春[はる]に 死[し]にたい。
\\	この曲がやさしいって言うんですか?何も分かっていないからそんなことを言うんですよ。弾けるものなら弾いてみなさい。	
\\	この 曲[きょく]がやさしいって 言[い]うんですか? 何[なに]も 分[わ]かっていないからそんなことを 言[い]うんですよ。 弾[ひ]けるものなら 弾[ひ]いてみなさい。
\\	訴えるなんて言ってもちっとも怖くなんかありませんよ。訴えられるものなら訴えてごらんなさい。	
\\	訴[うった]えるなんて 言[い]ってもちっとも 怖[こわ]くなんかありませんよ。 訴[うった]えられるものなら 訴[うった]えてごらんなさい。
\\	カリスマのある哲学者が大学に来て講演をしようものなら、1万人を超える聴衆が集まる。	
\\	カリスマのある 哲学[てつがく] 者[しゃ]が 大学[だいがく]に 来[き]て 講演[こうえん]をしようものなら、 
\\	万[まん] 人[にん]を 超[こ]える 聴衆[ちょうしゅう]が 集[あつ]まる。
\\	一戸建ての家が買えるものなら買いたかったのだが。	
\\	一戸建[いっこだ]ての 家[いえ]が 買[か]えるものなら 買[か]いたかったのだが。
\\	この新刊、町の図書館で借りられるものなら買わずに済んだのに。	
\\	この 新刊[しんかん]、 町[まち]の 図書館[としょかん]で 借[か]りられるものなら 買[か]わずに 済[す]んだのに。
\\	通訳の仕事を引き受けたものの、うまくやれる自信がない。	
\\	通訳[つうやく]の 仕事[しごと]を 引き受[ひきう]けたものの、うまくやれる 自信[じしん]がない。
\\	一人でいる方が気楽だとは言うものの、やはり話し相手が欲しいと思うこともある。	
\\	一人[ひとり]でいる 方[ほう]が 気楽[きらく]だとは 言[い]うものの、やはり 話し相手[はなしあいて]が 欲[ほ]しいと 思[おも]うこともある。
\\	大学を卒業はしたものの、不況でなかなかいい仕事が見つからない。	
\\	大学[だいがく]を 卒業[そつぎょう]はしたものの、 不況[ふきょう]でなかなかいい 仕事[しごと]が 見[み]つからない。
\\	マネージャーになって給料は上がったものの、急にいろんな仕事が増えてとても忙しくなった。	
\\	マネージャーになって 給料[きゅうりょう]は 上[あ]がったものの、 急[きゅう]にいろんな 仕事[しごと]が 増[ふ]えてとても 忙[いそが]しくなった。
\\	市内を抜け出すまでは時間がかかったものの、高速に入ってからは渋滞もなく、4時前にホテルにチェックインすることができた。	
\\	市内[しない]を 抜け出[ぬけだ]すまでは 時間[じかん]がかかったものの、 高速[こうそく]に 入[はい]ってからは 渋滞[じゅうたい]もなく、 
\\	時[じ] 前[まえ]にホテルにチェックインすることができた。
\\	韓国産のハマグリは形はふくらみが少なく身がややかたいものの、うまみが強く、寿司ネタにも使われている。	
\\	韓国[かんこく] 産[さん]のハマグリは 形[かたち]はふくらみが 少[すく]なく 身[み]がややかたいものの、うまみが 強[つよ]く、 寿司[すし]ネタにも 使[つか]われている。
\\	現地の子供たちは最初こそよそよそしかったものの、時間がたつにつれて慣れてきてとても親切にしてくれました。	
\\	現地[げんち]の 子供[こども]たちは 最初[さいしょ]こそよそよそしかったものの、 時間[じかん]がたつにつれて 慣[な]れてきてとても 親切[しんせつ]にしてくれました。
\\	小林氏は身のこなしが軽快で、髪こそ真っ白なものの、「若い」という印象を与える。	
\\	小林[こばやし] 氏[し]は 身[み]のこなしが 軽快[けいかい]で、 髪[かみ]こそ 真っ白[まっしろ]なものの、
\\	若[わか]い」という 印象[いんしょう]を 与[あた]える。
\\	我がチームは出だしは好調だったものの、夏場で失速し、シーズンが終わってみれば、勝率5割を切るという不本意な成績に終わった。	
\\	我[わ]がチームは 出[で]だしは 好調[こうちょう]だったものの、 夏場[なつば]で 失速[しっそく]し、シーズンが 終[お]わってみれば、 勝率[しょうりつ] 
\\	割[わり]を 切[き]るという 不[ふ] 本意[ほんい]な 成績[せいせき]に 終[お]わった。
\\	彼は敬虔なイスラム教徒だったものの、ドイツ国内のイスラム主義団体との交流はなかった。	
\\	彼[かれ]は 敬虔[けいけん]なイスラム 教徒[きょうと]だったものの、ドイツ 国内[こくない]のイスラム 主義[しゅぎ] 団体[だんたい]との 交流[こうりゅう]はなかった。
\\	暦の上では秋とは言うものの、まだ真夏のような暑さが続いている。	
\\	暦[こよみ]の 上[うえ]では 秋[あき]とは 言[い]うものの、まだ 真夏[まなつ]のような 暑[あつ]さが 続[つづ]いている。
\\	同居していた息子の家族が県外に引っ越したので、やっと静かな生活に戻った。とは言うものの、孫の顔がいつも見られないのはちょっと寂しい。	
\\	同居[どうきょ]していた 息子[むすこ]の 家族[かぞく]が 県[けん] 外[がい]に 引っ越[ひっこ]したので、やっと 静[しず]かな 生活[せいかつ]に 戻[もど]った。とは 言[い]うものの、 孫[まご]の 顔[かお]がいつも 見[み]られないのはちょっと 寂[さび]しい。
\\	ハリーはパーティーに行くが、メアリーは行かない。	
\\	ハリーはパーティーに 行[い]くが、メアリーは 行[い]かない。
\\	ジョンは民主党だけれども、ジョージは共和党だ。	
\\	ジョンは 民主党[みんしゅとう]だけれども、ジョージは 共和党[きょうわとう]だ。
\\	彼は日本語を続けて勉強すればいいものを、1年間やっただけでやめてしまった。	
\\	彼[かれ]は 日本語[にほんご]を 続[つづ]けて 勉強[べんきょう]すればいいものを、 
\\	年間[ねんかん]やっただけでやめてしまった。
\\	すぐ医者に診てもらったら簡単に治っていたものを手遅れになってしまった。	
\\	すぐ 医者[いしゃ]に 診[み]てもらったら 簡単[かんたん]に 治[なお]っていたものを 手遅[ておく]れになってしまった。
\\	ゴルフかなんかして少しリラックスすればいいものを、父は週末でも会社の仕事を家でやっている。	
\\	ゴルフかなんかして 少[すこ]しリラックスすればいいものを、 父[ちち]は 週末[しゅうまつ]でも 会社[かいしゃ]の 仕事[しごと]を 家[いえ]でやっている。
\\	あのくだらない小説はちょっと読んだだけでやめればよかったものを、なんと最後まで読んでしまった。全くの時間の浪費だった。	
\\	あのくだらない 小説[しょうせつ]はちょっと 読[よ]んだだけでやめればよかったものを、なんと 最後[さいご]まで 読[よ]んでしまった。 全[まった]くの 時間[じかん]の 浪費[ろうひ]だった。
\\	居留守を使えばいいものを、彼女は気が弱いのでそうもできず、いやいや人に会うことになってしまうのだ。	
\\	居留守[いるす]を 使[つか]えばいいものを、 彼女[かのじょ]は 気[き]が 弱[よわ]いのでそうもできず、いやいや 人[ひと]に 会[あ]うことになってしまうのだ。
\\	私のことが嫌いなら嫌いだともっと早くはっきり言ってくれたらよかったものを。	
\\	私[わたし]のことが 嫌[きら]いなら 嫌[きら]いだともっと 早[はや]くはっきり 言[い]ってくれたらよかったものを。
\\	時間があったらいろいろな外国旅行ができたものを、いつも忙しく働いてきたので、外国はどこにも行っていない。	
\\	時間[じかん]があったらいろいろな 外国[がいこく] 旅行[りょこう]ができたものを、いつも 忙[いそが]しく 働[はたら]いてきたので、 外国[がいこく]はどこにも 行[い]っていない。
\\	父があと1年生きていてくれたら私の結婚式に出てもらえたものを、私がぐずぐずしていたばかりに、父に私の晴れ姿を見てもらえず、とても残念です。	
\\	父[ちち]があと 
\\	年[ねん] 生[い]きていてくれたら 私[わたし]の 結婚式[けっこんしき]に 出[で]てもらえたものを、 私[わたし]がぐずぐずしていたばかりに、 父[ちち]に 私[わたし]の 晴れ姿[はれすがた]を 見[み]てもらえず、とても 残念[ざんねん]です。
\\	父があと1年生きていてくれたら私の結婚式に出てもらえただろう。	
\\	父[ちち]があと 
\\	年[ねん] 生[い]きていてくれたら 私[わたし]の 結婚式[けっこんしき]に 出[で]てもらえただろう。
\\	私のことが嫌いなら嫌いだともっと早くはっきり言ってくれたらよかったものをなぜ言ってくれなかったのか。	
\\	私[わたし]のことが 嫌[きら]いなら 嫌[きら]いだともっと 早[はや]くはっきり 言[い]ってくれたらよかったものをなぜ 言[い]ってくれなかったのか。
\\	毎日漢字を勉強しているのによく覚えられない。	
\\	毎日[まいにち] 漢字[かんじ]を 勉強[べんきょう]しているのによく 覚[おぼ]えられない。
\\	漢字が面白かったら毎日漢字を勉強しているものを、ちょっとも面白くないのでよく覚えられない。	
\\	漢字[かんじ]が 面白[おもしろ]かったら 毎日[まいにち] 漢字[かんじ]を 勉強[べんきょう]しているものを、ちょっとも 面白[おもしろ]くないのでよく 覚[おぼ]えられない。
\\	中学と高校で6年間も英語を勉強したのにまだ英語が話せません。	
\\	中学[ちゅうがく]と 高校[こうこう]で 
\\	年間[ねんかん]も 英語[えいご]を 勉強[べんきょう]したのにまだ 英語[えいご]が 話[はな]せません。
\\	英語に興味があったらもっと一生懸命勉強したものを、興味もなかったのであまり勉強せず、そのためまだ英語が話せません。	
\\	英語[えいご]に 興味[きょうみ]があったらもっと 一生懸命[いっしょうけんめい] 勉強[べんきょう]したものを、 興味[きょうみ]もなかったのであまり 勉強[べんきょう]せず、そのためまだ 英語[えいご]が 話[はな]せません。
\\	この図書館は蔵書の多さもさることながらサービスが実にすばらしい。	
\\	この 図書館[としょかん]は 蔵書[ぞうしょ]の 多[おお]さもさることながらサービスが 実[じつ]にすばらしい。
\\	指圧やマッサージといった「タッチセラピー」が注目されている。身体への直接的な効果もさることながら、安らぎ感や癒やしといった心理的な効果も大きいようだ。	
\\	指圧[しあつ]やマッサージといった「タッチセラピー」が 注目[ちゅうもく]されている。 身体[しんたい]への 直接的[ちょくせつてき]な 効果[こうか]もさることながら、 安[やす]らぎ 感[かん]や 癒[い]やしといった 心理[しんり] 的[てき]な 効果[こうか]も 大[おお]きいようだ。
\\	高齢化が進むにつれ、認知症老人の介護が問題になっているが、これからは、介護のこともさることながらぼけをいかに予防するかがさらに重要だ。	
\\	高齢[こうれい] 化[か]が 進[すす]むにつれ、 認知[にんち] 症[しょう] 老人[ろうじん]の 介護[かいご]が 問題[もんだい]になっているが、これからは、 介護[かいご]のこともさることながらぼけをいかに 予防[よぼう]するかがさらに 重要[じゅうよう]だ。
\\	障害者向け製品の価格が高いのは、技術的な困難もさることながら市場の小ささが大きな原因になっている。	
\\	障害[しょうがい] 者[しゃ] 向[む]け 製品[せいひん]の 価格[かかく]が 高[たか]いのは、 技術[ぎじゅつ] 的[てき]な 困難[こんなん]もさることながら 市場[しじょう]の 小[ちい]ささが 大[おお]きな 原因[げんいん]になっている。
\\	下から眺めるテレビ塔やイルミネーションの美しさもさることながら、展望台に登って眺める市の夜景もまた格別です。	
\\	下[した]から 眺[なが]める テレビ塔[てれびとう]やイルミネーションの 美[うつく]しさもさることながら、 展望[てんぼう] 台[だい]に 登[のぼ]って 眺[なが]める 市[いち]の 夜景[やけい]もまた 格別[かくべつ]です。
\\	この合作映画は、作品の完成度の高さもさることながら、文化の異なる二国が協力して一つの映画を完成させたという事実が非常に意義深い。	
\\	この 合作[がっさく] 映画[えいが]は、 作品[さくひん]の 完成[かんせい] 度[ど]の 高[たか]さもさることながら、 文化[ぶんか]の 異[こと]なる二 国[こく]が 協力[きょうりょく]して 一[ひと]つの 映画[えいが]を 完成[かんせい]させたという 事実[じじつ]が 非常[ひじょう]に 意義[いぎ] 深[ぶか]い。
\\	この申込書は本人か、もしくは保証人が記入しなければならない。	
\\	この 申込[もうしこみ] 書[しょ]は 本人[ほんにん]か、もしくは 保証[ほしょう] 人[にん]が 記入[きにゅう]しなければならない。
\\	ワークショップに出席なさりたい方はオンラインで登録するか、もしくは、メールでお申し込み下さい。	
\\	ワークショップに 出席[しゅっせき]なさりたい 方[かた]はオンラインで 登録[とうろく]するか、もしくは、メールでお 申し込[もうしこ]み 下[くだ]さい。
\\	基調講演者は日本研究もしくは関連領域の研究について話すことになっている。	
\\	基調[きちょう] 講演[こうえん] 者[しゃ]は 日本[にほん] 研究[けんきゅう]もしくは 関連[かんれん] 領域[りょういき]の 研究[けんきゅう]について 話[はな]すことになっている。
\\	この予算は研究費、もしくは研究補助費にのみ使用できる。	
\\	この 予算[よさん]は 研究[けんきゅう] 費[ひ]、もしくは 研究[けんきゅう] 補助[ほじょ] 費[ひ]にのみ 使用[しよう]できる。
\\	大学のプールは学生か、もしくは使用料を支払った大学関係者のみ利用できる。	
\\	大学[だいがく]のプールは 学生[がくせい]か、もしくは 使用[しよう] 料[りょう]を 支払[しはら]った 大学[だいがく] 関係[かんけい] 者[しゃ]のみ 利用[りよう]できる。
\\	夏季のドライブシーズンを迎える頃には、ガソリンの在庫が昨年とほぼ同水準まで下がるか、もしくはそれ以下に減少する見通しである。	
\\	夏季[かき]のドライブシーズンを 迎[むか]える 頃[ころ]には、ガソリンの 在庫[ざいこ]が 昨年[さくねん]とほぼ 同[どう] 水準[すいじゅん]まで 下[さ]がるか、もしくはそれ 以下[いか]に 減少[げんしょう]する 見通[みとお]しである。
\\	日本の経済力が弱まってきている現在、日本が世界で生き残る道はテクノロジーで時代の先取りをするか、もしくは、モノとしての日本の文化をより積極的に海外に輸出することであろう。	
\\	日本[にほん]の 経済[けいざい] 力[りょく]が 弱[よわ]まってきている 現在[げんざい]、 日本[にほん]が 世界[せかい]で 生き残[いきのこ]る 道[みち]はテクノロジーで 時代[じだい]の 先取[さきど]りをするか、もしくは、モノとしての 日本[にほん]の 文化[ぶんか]をより 積極[せっきょく] 的[てき]に 海外[かいがい]に 輸出[ゆしゅつ]することであろう。
\\	ご近所の、大きいか、もしくは、有名な病院に行くことをお勧めします。	
\\	ご 近所[きんじょ]の、 大[おお]きいか、もしくは、 有名[ゆうめい]な 病院[びょういん]に 行[い]くことをお 勧[すす]めします。
\\	大学のプールの使用者は学生か、もしくは、大学関係者だ。	
\\	大学[だいがく]のプールの 使用[しよう] 者[しゃ]は 学生[がくせい]か、もしくは、 大学[だいがく] 関係[かんけい] 者[しゃ]だ。
\\	彼女は仕事を探すか、もしくは大学院に進学するか、迷っている。	
\\	彼女[かのじょ]は 仕事[しごと]を 探[さが]すか、もしくは 大学院[だいがくいん]に 進学[しんがく]するか、 迷[まよ]っている。
\\	ステーキにしますか。それとも魚にしますか。	
\\	ステーキにしますか。それとも 魚[さかな]にしますか。
\\	運動は体にいいと言われている。もっとも、やりすぎると逆効果だ。	
\\	運動[うんどう]は 体[からだ]にいいと 言[い]われている。もっとも、やりすぎると 逆[ぎゃく] 効果[こうか]だ。
\\	息子はよく病気をする。もっとも大病をしたことはないが。	
\\	息子[むすこ]はよく 病気[びょうき]をする。もっとも 大病[たいびょう]をしたことはないが。
\\	あの政治家は保守的だと言われている。もっとも、時々革新的なことを言うこともあるが。	
\\	あの 政治[せいじ] 家[か]は 保守[ほしゅ] 的[てき]だと 言[い]われている。もっとも、 時々[ときどき] 革新[かくしん] 的[てき]なことを 言[い]うこともあるが。
\\	彼は女性にとても親切だ。もっとも、それには下心がある場合が多いようだ。	
\\	彼[かれ]は 女性[じょせい]にとても 親切[しんせつ]だ。もっとも、それには 下心[したごころ]がある 場合[ばあい]が 多[おお]いようだ。
\\	今日は魚がよく釣れた。もっとも、いつもこのように幸運だとは限らない。	
\\	今日[きょう]は 魚[さかな]がよく 釣[つ]れた。もっとも、いつもこのように 幸運[こううん]だとは 限[かぎ]らない。
\\	このドイツの車の方があの日本の車よりいいんじゃない。もっとも、僕はドイツの車を運転したことはないけどさ。	
\\	このドイツの 車[くるま]の 方[ほう]があの 日本[にほん]の 車[くるま]よりいいんじゃない。もっとも、 僕[ぼく]はドイツの 車[くるま]を 運転[うんてん]したことはないけどさ。
\\	顔写真の入った名詞というのも、強い印象を与えるからいいかもしれない。もっとも私はそこまでやりたいとは思わないけれども。	
\\	顔[かお] 写真[しゃしん]の 入[はい]った 名詞[めいし]というのも、 強[つよ]い 印象[いんしょう]を 与[あた]えるからいいかもしれない。もっとも 私[わたし]はそこまでやりたいとは 思[おも]わないけれども。
\\	若い頃は漠然としていたものが歳を重ねるごとにハッキリと見えるようになり。目標を見据えてよりポジティブに生きるようにしています。もっとも私はまだ、その途上にありますが。	
\\	若[わか]い 頃[ころ]は 漠然[ばくぜん]としていたものが 歳[とし]を 重[かさ]ねるごとにハッキリと 見[み]えるようになり。 目標[もくひょう]を 見据[みす]えてよりポジティブに 生[い]きるようにしています。もっとも 私[わたし]はまだ、その 途上[とじょう]にありますが。
\\	僕は憲法改正論者だ。しかし友人の
\\	は憲法擁護論者だ。	
\\	僕[ぼく]は 憲法[けんぽう] 改正[かいせい] 論[ろん] 者[しゃ]だ。しかし 友人[ゆうじん]の 
\\	は 憲法[けんぽう] 擁護[ようご] 論[ろん] 者[しゃ]だ。
\\	日本は島国だ。しかし韓国は大陸とつながった半島だ。	
\\	日本[にほん]は 島国[しまぐに]だ。しかし 韓国[かんこく]は 大陸[たいりく]とつながった 半島[はんとう]だ。
\\	春とは言うもののまるで初夏のようだ。	
\\	春[はる]とは 言[い]うもののまるで 初夏[しょか]のようだ。
\\	日本語の先生とは言うものの日本語のことを何も知らない。	
\\	日本語[にほんご]の 先生[せんせい]とは 言[い]うものの 日本語[にほんご]のことを 何[なに]も 知[し]らない。
\\	今は暦の上では春だ。とは言うもののまるでここ数日は初夏のようだが。	
\\	今[いま]は 暦[こよみ]の 上[うえ]では 春[はる]だ。とは 言[い]うもののまるでここ 数[すう] 日[じつ]は 初夏[しょか]のようだが。
\\	彼は日本語の先生だ。もっとも日本語のことはまるで知らないが。	
\\	彼[かれ]は 日本語[にほんご]の 先生[せんせい]だ。もっとも 日本語[にほんご]のことはまるで 知[し]らないが。
\\	このギター曲集は初心者向けだ。	
\\	このギター 曲[きょく] 集[しゅう]は 初心者[しょしんしゃ] 向[む]けだ。
\\	この秋はアジア向け産業用口ボットの販売が好調だ。	
\\	この 秋[あき]はアジア 向[む]け 産業[さんぎょう] 用[よう] 口[ぐち]ボットの 販売[はんばい]が 好調[こうちょう]だ。
\\	社は携帯機器向けに小型燃料電池を開発している。	
\\	社[しゃ]は 携帯[けいたい] 機器[きき] 向[む]けに 小型[こがた] 燃料[ねんりょう] 電池[でんち]を 開発[かいはつ]している。
\\	このセミナーはこれから株を始める人向けだ。	
\\	このセミナーはこれから 株[かぶ]を 始[はじ]める 人[ひと] 向[む]けだ。
\\	輸出はここ数年、中国向けが急速に伸びている。	
\\	輸出[ゆしゅつ]はここ 数[すう] 年[ねん]、 中国[ちゅうごく] 向[む]けが 急速[きゅうそく]に 伸[の]びている。
\\	このサービスは、基本的にはコンシューマー向けだが、法人向けも用意する。	
\\	このサービスは、 基本[きほん] 的[てき]にはコンシューマー 向[む]けだが、 法人[ほうじん] 向[む]けも 用意[ようい]する。
\\	助成金をもらって先生向けの情報教育サイトを作ることにした。	
\\	助成[じょせい] 金[きん]をもらって 先生[せんせい] 向[む]けの 情報[じょうほう] 教育[きょういく]サイトを 作[つく]ることにした。
\\	このショールームでは住宅向けのタイル・建材を展示しております。	
\\	このショールームでは 住宅[じゅうたく] 向[む]けのタイル・ 建材[けんざい]を 展示[てんじ]しております。
\\	アメリカ進出を考える場合は、アメリカ向けの読みやすい商品説明書と価格表を用意することが必須です。	
\\	アメリカ 進出[しんしゅつ]を 考[かんが]える 場合[ばあい]は、アメリカ 向[む]けの 読[よ]みやすい 商品[しょうひん] 説明[せつめい] 書[しょ]と 価格[かかく] 表[ひょう]を 用意[ようい]することが 必須[ひっす]です。
\\	者は近々場図サービスと地域情報検索を国内向けに開始すると発表した。	
\\	者[しゃ]は 近々[ちかぢか] 場[ば] 図[ず]サービスと 地域[ちいき] 情報[じょうほう] 検索[けんさく]を 国内[こくない] 向[む]けに 開始[かいし]すると 発表[はっぴょう]した。
\\	代表的な曲を小さなお子様向けにアレンジしました。	
\\	代表[だいひょう] 的[てき]な 曲[きょく]を 小[ちい]さなお 子様[こさま] 向[む]けにアレンジしました。
\\	このオペレーティングシステムは、巨大なアプリケーション向けに設計されている。	
\\	このオペレーティングシステムは、 巨大[きょだい]なアプリケーション 向[む]けに 設計[せっけい]されている。
\\	私は南向きの部屋が欲しい。	
\\	私[わたし]は 南[みなみ] 向[む]きの 部屋[へや]が 欲[ほ]しい。
\\	彼の考え方は常に前向きだ。	
\\	彼[かれ]の 考え方[かんがえかた]は 常[つね]に 前向[まえむ]きだ。
\\	景気がやっと上向きになってきた。	
\\	景気[けいき]がやっと 上向[うわむ]きになってきた。
\\	この辺りは若者向きの店が多い。	
\\	この 辺[あた]りは 若者[わかもの] 向[む]きの 店[みせ]が 多[おお]い。
\\	この話は映画よりむしろアニメ向きだ。	
\\	この 話[はなし]は 映画[えいが]よりむしろアニメ 向[む]きだ。
\\	本日は夏向きに冷たいお料理をいくつかご紹介します。	
\\	本日[ほんじつ]は 夏向[なつむ]きに 冷[つめ]たいお 料理[りょうり]をいくつかご 紹介[しょうかい]します。
\\	夏休みはアルバイトをするよりもむしろ海外旅行がしたい。	
\\	夏休[なつやす]みはアルバイトをするよりもむしろ 海外[かいがい] 旅行[りょこう]がしたい。
\\	僕はすしよりむしろさしみの方が好きだ。	
\\	僕[ぼく]はすしよりむしろさしみの 方[ほう]が 好[す]きだ。
\\	彼は秀才というよりはむしろ努力家だ。	
\\	彼[かれ]は 秀才[しゅうさい]というよりはむしろ 努力[どりょく] 家[か]だ。
\\	無為の生活をするぐらいなら、むしろ死んだ方がいい。	
\\	無為[むい]の 生活[せいかつ]をするぐらいなら、むしろ 死[し]んだ 方[ほう]がいい。
\\	英語は話すよりもむしろ読むのが好きだ。	
\\	英語[えいご]は 話[はな]すよりもむしろ 読[よ]むのが 好[す]きだ。
\\	時間があるときは、観光バスに乗るよりはむしろ歩いた方がずっとよく観光できる。	
\\	時間[じかん]があるときは、 観光[かんこう]バスに 乗[の]るよりはむしろ 歩[ある]いた 方[ほう]がずっとよく 観光[かんこう]できる。
\\	休みの日は、人の多い行楽地に行くよりは、むしろ、家でテレビでも見ている方がましだ。	
\\	休[やす]みの 日[ひ]は、 人[ひと]の 多[おお]い 行楽[こうらく] 地[ち]に 行[い]くよりは、むしろ、 家[いえ]でテレビでも 見[み]ている 方[ほう]がましだ。
\\	静脈を見つけ、正確に注射するという作業は、人間の手よりもむしろコンピュータを使ってロボットが行った方が、ミスが少ないと思う。	
\\	静脈[じょうみゃく]を 見[み]つけ、 正確[せいかく]に 注射[ちゅうしゃ]するという 作業[さぎょう]は、 人間[にんげん]の 手[て]よりもむしろコンピュータを 使[つか]ってロボットが 行[おこな]った 方[ほう]が、ミスが 少[すく]ないと 思[おも]う。
\\	私は、健康上、肉よりむしろ魚の方を選ぶ。	
\\	私[わたし]は、 健康[けんこう] 上[じょう]、 肉[にく]よりむしろ 魚[さかな]の 方[ほう]を 選[えら]ぶ。
\\	今年は2月よりもむしろ3月の方が寒かった。	
\\	今年[ことし]は 
\\	月[がつ]よりもむしろ 
\\	月[がつ]の 方[ほう]が 寒[さむ]かった。
\\	彼らが突如大金持ちの仲間入りを果たすことができたのは、彼らの才能というよりもむしろ幸運が味方したからではないだろうか。	
\\	彼[かれ]らが 突如[とつじょ] 大金持[おおがねも]ちの 仲間入[なかまい]りを 果[は]たすことができたのは、 彼[かれ]らの 才能[さいのう]というよりもむしろ 幸運[こううん]が 味方[みかた]したからではないだろうか。
\\	シアトルはアメリカなのですが、他の大都市とは違って、アメリカというよりはむしろカナダのような雰囲気の小ぢんまりとした街でした。	
\\	シアトルはアメリカなのですが、 他[ほか]の 大都市[だいとし]とは 違[ちが]って、アメリカというよりはむしろカナダのような 雰囲気[ふんいき]の 小[こ]ぢんまりとした 街[まち]でした。
\\	今年の北海道の夏は涼しかったというよりはむしろ寒かった。	
\\	今年[ことし]の 北海道[ほっかいどう]の 夏[なつ]は 涼[すず]しかったというよりはむしろ 寒[さむ]かった。
\\	前の週の豪雨のため、ドナウ川の水は青いというよりはむしろ茶色に近い色だった。	
\\	前[まえ]の 週[しゅう]の 豪雨[ごうう]のため、ドナウ 川[がわ]の 水[みず]は 青[あお]いというよりはむしろ 茶色[ちゃいろ]に 近[ちか]い 色[いろ]だった。
\\	こんなつまらない講義に出るぐらいならむしろ寮で本でも読んでいた方がましだ。	
\\	こんなつまらない 講義[こうぎ]に 出[で]るぐらいならむしろ 寮[りょう]で 本[ほん]でも 読[よ]んでいた 方[ほう]がましだ。
\\	優しくしてから突き放すくらいなら、むしろはじめから優しくするなと思う。	
\\	優[やさ]しくしてから 突き放[つきはな]すくらいなら、むしろはじめから 優[やさ]しくするなと 思[おも]う。
\\	日本人よりも、むしろ、外国人の方が日本文化を深く理解している場合がある。	
\\	日本人[にほんじん]よりも、むしろ、 外国[がいこく] 人[じん]の 方[ほう]が 日本[にほん] 文化[ぶんか]を 深[ふか]く 理解[りかい]している 場合[ばあい]がある。
\\	身内よりも、かえって、他人の方が親切なことも多い。	
\\	身内[みうち]よりも、かえって、 他人[たにん]の 方[ほう]が 親切[しんせつ]なことも 多[おお]い。
\\	今が私の人生で一番幸せな時期だという気がしないでもない。	
\\	今[いま]が 私[わたし]の 人生[じんせい]で 一番[いちばん] 幸[しあわ]せな 時期[じき]だという 気[き]がしないでもない。
\\	あの時大学を辞めたことを、今では少し早まったなと思わないでもない。	
\\	あの 時[とき] 大学[だいがく]を 辞[や]めたことを、 今[いま]では 少[すこ]し 早[はや]まったなと 思[おも]わないでもない。
\\	そういえば、そういう話、どこかで聞いたことがあるような気がしないでもないです。	
\\	そういえば、そういう 話[はなし]、どこかで 聞[き]いたことがあるような 気[き]がしないでもないです。
\\	私は「結婚しない主義」でもないし、一度結婚してみたいという気もないでもないんですが、なかなかいい相手が見つかりません。	
\\	私[わたし]は
\\	結婚[けっこん]しない 主義[しゅぎ]」でもないし、 一度[いちど] 結婚[けっこん]してみたいという 気[き]もないでもないんですが、なかなかいい 相手[あいて]が 見[み]つかりません。
\\	日本人として、アメリカ企業におけるリストラを見ると非常さを感じないでもない。	
\\	日本人[にほんじん]として、アメリカ 企業[きぎょう]におけるリストラを 見[み]ると 非常[ひじょう]さを 感[かん]じないでもない。
\\	あなたの苦しい気持ちが分からないでもない。	
\\	あなたの 苦[くる]しい 気持[きも]ちが 分[わ]からないでもない。
\\	白衣に眼鏡なら、一見普通の医者に見えないでもない。	
\\	白衣[はくい]に 眼鏡[めがね]なら、 一見[いっけん] 普通[ふつう]の 医者[いしゃ]に 見[み]えないでもない。
\\	心底謝れば許してやらないでもないけど。	
\\	心底[しんそこ] 謝[あやま]れば 許[ゆる]してやらないでもないけど。
\\	そんなに日本語を勉強したいのなら教えないでもない。	
\\	そんなに 日本語[にほんご]を 勉強[べんきょう]したいのなら 教[おし]えないでもない。
\\	2年間でこれを出版することはできないでもない。	
\\	年間[ねんかん]でこれを 出版[しゅっぱん]することはできないでもない。
\\	この小説は面白くないこともない。	
\\	この 小説[しょうせつ]は 面白[おもしろ]くないこともない。
\\	あの人は変じゃなくもない。	
\\	あの 人[ひと]は 変[へん]じゃなくもない。
\\	彼女は病気じゃないこともない。	
\\	彼女[かのじょ]は 病気[びょうき]じゃないこともない。
\\	彼は学者じゃなくもない。	
\\	彼[かれ]は 学者[がくしゃ]じゃなくもない。
\\	彼だってアメリカのことに詳しくないこともない。	
\\	彼[かれ]だってアメリカのことに 詳[くわ]しくないこともない。
\\	この問題は重要ではなくもない。	
\\	この 問題[もんだい]は 重要[じゅうよう]ではなくもない。
\\	この仕事が終わらないことには、家族と旅行に行くこともできない。	
\\	この 仕事[しごと]が 終[お]わらないことには、 家族[かぞく]と 旅行[りょこう]に 行[い]くこともできない。
\\	追加予算が下りないことには、このプロジェクトの遂行は無理だ。	
\\	追加[ついか] 予算[よさん]が 下[お]りないことには、このプロジェクトの 遂行[すいこう]は 無理[むり]だ。
\\	書き始めないことには論文ができるはずはない。	
\\	書[か]き 始[はじ]めないことには 論文[ろんぶん]ができるはずはない。
\\	コンピュータが作動しないことには仕事は始まらない。	
\\	コンピュータが 作動[さどう]しないことには 仕事[しごと]は 始[はじ]まらない。
\\	面接を受けないことには就職はできない。	
\\	面接[めんせつ]を 受[う]けないことには 就職[しゅうしょく]はできない。
\\	為替相場が安定しないことには、経済の回復もおぼつかない。	
\\	為替[かわせ] 相場[そうば]が 安定[あんてい]しないことには、 経済[けいざい]の 回復[かいふく]もおぼつかない。
\\	日本語学習者は、日本に行かないことには日本語を自然に話せるようになるのは難しいだろう。	
\\	日本語[にほんご] 学習[がくしゅう] 者[しゃ]は、 日本[にほん]に 行[い]かないことには 日本語[にほんご]を 自然[しぜん]に 話[はな]せるようになるのは 難[むずか]しいだろう。
\\	際立った才能がないことには、一流の音楽大学を出てもソリストになることは夢物語だ。	
\\	際立[きわだ]った 才能[さいのう]がないことには、 一流[いちりゅう]の 音楽[おんがく] 大学[だいがく]を 出[で]てもソリストになることは 夢物語[ゆめものがたり]だ。
\\	情報を教えていただけないことには検討のしようがありません。	
\\	情報[じょうほう]を 教[おし]えていただけないことには 検討[けんとう]のしようがありません。
\\	この仕事が今年中に終わらなかったら、完成は来年の半ばぐらいまで延ばします。	
\\	この 仕事[しごと]が 今年[ことし] 中[ちゅう]に 終[お]わらなかったら、 完成[かんせい]は 来年[らいねん]の 半[なか]ばぐらいまで 延[の]ばします。
\\	情報を教えていただけなければ自分たちで集めますから大丈夫です。	
\\	情報[じょうほう]を 教[おし]えていただけなければ 自分[じぶん]たちで 集[あつ]めますから 大丈夫[だいじょうぶ]です。
\\	この公演は大成功とは言えないまでも、それなりの成果をもたらした。	
\\	この 公演[こうえん]は 大[だい] 成功[せいこう]とは 言[い]えないまでも、それなりの 成果[せいか]をもたらした。
\\	彼の論文は決して多くないまでも、その質は非常に高い。	
\\	彼[かれ]の 論文[ろんぶん]は 決[けっ]して 多[おお]くないまでも、その 質[しつ]は 非常[ひじょう]に 高[たか]い。
\\	「生命とはこういうものです」とは言えないまでも、「少なくともこうではありません」とは言えるようになってくるはずだ。	
\\	生命[せいめい]とはこういうものです」とは 言[い]えないまでも、
\\	少[すく]なくともこうではありません」とは 言[い]えるようになってくるはずだ。
\\	その彫刻は完璧とは言えないまでも、非常に可能性を感じさせる作品だった。	
\\	その 彫刻[ちょうこく]は 完璧[かんぺき]とは 言[い]えないまでも、 非常[ひじょう]に 可能[かのう] 性[せい]を 感[かん]じさせる 作品[さくひん]だった。
\\	虚偽だとは言わないまでも、かなり紛らわしい表現だ。	
\\	虚偽[きょぎ]だとは 言[い]わないまでも、かなり 紛[まぎ]らわしい 表現[ひょうげん]だ。
\\	中国語は漢字だから、読めないまでも意味はなんとなく分かるのだが、韓国語はまったく見当がつかない。	
\\	中国[ちゅうごく] 語[ご]は 漢字[かんじ]だから、 読[よ]めないまでも 意味[いみ]はなんとなく 分[わ]かるのだが、 韓国[かんこく] 語[ご]はまったく 見当[けんとう]がつかない。
\\	インターネットは完璧ではないまでも、生涯学習もしくは生涯教育のメディアとして、いろいろな可能性をもっています。	
\\	インターネットは 完璧[かんぺき]ではないまでも、 生涯[しょうがい] 学習[がくしゅう]もしくは 生涯[しょうがい] 教育[きょういく]のメディアとして、いろいろな 可能[かのう] 性[せい]をもっています。
\\	運命には逆らえないまでも、それを味方にすることはできる。	
\\	運命[うんめい]には 逆[さか]らえないまでも、それを 味方[みかた]にすることはできる。
\\	大方の日本人なら、「源氏物語」の原文に目を通したとまではいかないまでも、どんな話なのかは知っているだろう。	
\\	大方[おおかた]の 日本人[にほんじん]なら、
\\	源氏物語[げんじものがたり]」の 原文[げんぶん]に 目[め]を 通[とお]したとまではいかないまでも、どんな 話[はなし]なのかは 知[し]っているだろう。
\\	明日は天気がよくないまでも、気温はあまり低くなりません。	
\\	明日[あす]は 天気[てんき]がよくないまでも、 気温[きおん]はあまり 低[ひく]くなりません。
\\	彼は正直者じゃないまでも、嘘つきじゃない。	
\\	彼[かれ]は 正直[しょうじき] 者[もの]じゃないまでも、 嘘[うそ]つきじゃない。
\\	このクラスの学生のほとんどは3年ないし4年の日本語学習歴を持っている。	
\\	このクラスの 学生[がくせい]のほとんどは 
\\	年[ねん]ないし 4年[よねん]の 日本語[にほんご] 学習[がくしゅう] 歴[れき]を 持[も]っている。
\\	この論文集の論文はすげて日本語かないしは英語で書かれている。	
\\	この 論文[ろんぶん] 集[しゅう]の 論文[ろんぶん]はすげて 日本語[にほんご]かないしは 英語[えいご]で 書[か]かれている。
\\	生産量を20
\\	減らす、ないしは新しい市場を急遽開拓するしかない。	
\\	生産[せいさん] 量[りょう]を 
\\	減[へ]らす、ないしは 新[あたら]しい 市場[しじょう]を 急遽[きゅうきょ] 開拓[かいたく]するしかない。
\\	細菌の大きさは1ミクロンないし数ミクロンだ。	
\\	細菌[さいきん]の 大[おお]きさは 1ミクロンないし 数[すう]ミクロンだ。
\\	電子メールは、忙しい人でも、1日に1回ないしは数回チェックしているはずだ。	
\\	電子[でんし]メールは、 忙[いそが]しい 人[ひと]でも、 
\\	日[にち]に 
\\	回[かい]ないしは 数[すう] 回[かい]チェックしているはずだ。
\\	サン・モリッツでは5つ星ないしは3つ星クラスのホテルに滞在し、4日間、スイスの冬を満喫していただきます。	
\\	サン・モリッツでは 5つ[いつつ] 星[ぼし]ないしは 3つ[みっつ] 星[ぼし]クラスのホテルに 滞在[たいざい]し、 4日間[よっかかん]、スイスの 冬[ふゆ]を 満喫[まんきつ]していただきます。
\\	本冊子の購入希望者は、氏名、住所、購入希望冊数を明記の上、ファックス、ないしは電子メールで協会事務局までお申し込み下さい。	
\\	本[ほん] 冊子[さっし]の 購入[こうにゅう] 希望[きぼう] 者[しゃ]は、 氏名[しめい]、 住所[じゅうしょ]、 購入[こうにゅう] 希望[きぼう] 冊数[さっすう]を 明記[めいき]の 上[うえ]、ファックス、ないしは 電子[でんし]メールで 協会[きょうかい] 事務[じむ] 局[きょく]までお 申し込[もうしこ]み 下[くだ]さい。
\\	本アンケートでは、2004年鉛全廃について、「必要」ないしは「やむを得ない」とするものが74
\\	を占めた。	
\\	本[ほん]アンケートでは、 
\\	年[ねん] 鉛[なまり] 全廃[ぜんぱい]について、
\\	必要[ひつよう]」ないしは「やむを 得[え]ない」とするものが 
\\	を 占[し]めた。
\\	本品は1回50
\\	(2錠)を1日1回就寝前、ないしは1日2回、朝及び就寝前に経口服用して下さい。	
\\	本[ほん] 品[ぴん]は 
\\	回[かい] 
\\	錠[じょう])を 
\\	日[にち] 
\\	回[かい] 就寝[しゅうしん] 前[まえ]、ないしは 
\\	日[にち] 
\\	回[かい]、 朝[あさ] 及[およ]び 就寝[しゅうしん] 前[まえ]に 経口[けいこう] 服用[ふくよう]して 下[くだ]さい。
\\	当社は、本ウェブサイトの利用者からの問い合わせにお答えするため、ないしは、あらかじめ利用者に明示させていただいた目的の範囲で、利用者の個人情報を収集します。	
\\	当社[とうしゃ]は、 本[ほん]ウェブサイトの 利用[りよう] 者[しゃ]からの 問い合[といあ]わせにお 答[こた]えするため、ないしは、あらかじめ 利用[りよう] 者[しゃ]に 明示[めいじ]させていただいた 目的[もくてき]の 範囲[はんい]で、 利用[りよう] 者[しゃ]の 個人[こじん] 情報[じょうほう]を 収集[しゅうしゅう]します。
\\	当社は、犯罪活動の証拠となる、ないしはその一部となる顧客情報を公開することを、法律によって義務づけられることがあります。	
\\	当社[とうしゃ]は、 犯罪[はんざい] 活動[かつどう]の 証拠[しょうこ]となる、ないしはその 一部[いちぶ]となる 顧客[こきゃく] 情報[じょうほう]を 公開[こうかい]することを、 法律[ほうりつ]によって 義務[ぎむ]づけられることがあります。
\\	あの人たちの国籍は中国、ないしは韓国だ。	
\\	あの 人[ひと]たちの 国籍[こくせき]は 中国[ちゅうごく]、ないしは 韓国[かんこく]だ。
\\	右端かその隣が田中さんだ。	
\\	右端[みぎはし]かその 隣[となり]が 田中[たなか]さんだ。
\\	右端、またはその隣が田中さんだ。	
\\	右端[みぎはし]、またはその 隣[となり]が 田中[たなか]さんだ。
\\	冷たい風が吹く中を10キロも歩いた。	
\\	冷[つめ]たい 風[かぜ]が 吹[ふ]く 中[なか]を 10キロも 歩[ある]いた。
\\	お忙しい中をいらしていただき、申し訳ございません。	
\\	お 忙[いそが]しい 中[ちゅう]をいらしていただき、 申し訳[もうしわけ]ございません。
\\	赤十字の人たちがハリケーン被災者たちの中を歩き回って食料を配った。	
\\	赤十字[せきじゅうじ]の 人[ひと]たちがハリケーン 被災[ひさい] 者[しゃ]たちの 中[なか]を 歩き回[あるきまわ]って 食料[しょくりょう]を 配[くば]った。
\\	子供たちが遊んでいる中を子犬がちょろちょろ歩り回っていた。	
\\	子供[こども]たちが 遊[あそ]んでいる 中[なか]を 子犬[こいぬ]がちょろちょろ 歩[ふ]り 回[まわ]っていた。
\\	公園では、人が集まっている中を、アイスクリーム屋が歩いていた。	
\\	公園[こうえん]では、 人[ひと]が 集[あつ]まっている 中[なか]を、アイスクリーム 屋[や]が 歩[ある]いていた。
\\	雪の降る中を頑張って歩っている人を見ていたら、自分が軟弱に思えてきました。	
\\	雪[ゆき]の 降[ふ]る 中[なか]を 頑張[がんば]って 歩[ふ]っている 人[ひと]を 見[み]ていたら、 自分[じぶん]が 軟弱[なんじゃく]に 思[おも]えてきました。
\\	私たちが呆然として見ている中を、二人は一つの傘の下に入って行ってしまいました。	
\\	私[わたし]たちが 呆然[ぼうぜん]として 見[み]ている 中[なか]を、二人[ふたり]は 一[ひと]つの 傘[かさ]の 下[した]に 入[はい]って 行[い]ってしまいました。
\\	本日は、雨で足元が大変悪い中を、このようにたくさんの方にご参加いただきまして、誠にありがとうございました。	
\\	本日[ほんじつ]は、 雨[あめ]で 足元[あしもと]が 大変[たいへん] 悪[わる]い 中[なか]を、このようにたくさんの 方[かた]にご 参加[さんか]いただきまして、 誠[まこと]にありがとうございました。
\\	私だけ門限があるのでパーティーから先に帰ってきたが、楽しい中を抜けて帰るのはとてもつらかった。	
\\	私[わたし]だけ 門限[もんげん]があるのでパーティーから 先[さき]に 帰[かえ]ってきたが、 楽[たの]しい 中[なか]を 抜[ぬ]けて 帰[かえ]るのはとてもつらかった。
\\	大人たちは戦後のあの苦しい中を、気力をとり戻して復興に精を出したのだ。	
\\	大人[おとな]たちは 戦後[せんご]のあの 苦[くる]しい 中[なか]を、 気力[きりょく]をとり 戻[もど]して 復興[ふっこう]に 精[せい]を 出[だ]したのだ。
\\	我々は地震で道路が崩壊する危険な中を車で10時間ぐらい歩った。	
\\	我々[われわれ]は 地震[じしん]で 道路[どうろ]が 崩壊[ほうかい]する 危険[きけん]な 中[なか]を 車[くるま]で 
\\	時間[じかん]ぐらい 歩[ふ]った。
\\	飛行機は嵐の中を随分と長い間飛んでいるらしく、雲に突っ込むたびに大きく揺れている。	
\\	飛行機[ひこうき]は 嵐[あらし]の 中[なか]を 随分[ずいぶん]と 長[なが]い 間[あいだ] 飛[と]んでいるらしく、 雲[くも]に 突っ込[つっこ]むたびに 大[おお]きく 揺[ゆ]れている。
\\	会議中に友人から電話がかかってきた。	
\\	会議[かいぎ] 中[ちゅう]に 友人[ゆうじん]から 電話[でんわ]がかかってきた。
\\	みんな忙しい中を無理して出てきたんだ。	
\\	みんな 忙[いそが]しい 中[なか]を 無理[むり]して 出[で]てきたんだ。
\\	みんな忙しいところをすまないね。	
\\	みんな 忙[いそが]しいところをすまないね。
\\	足元が悪い中を頑張って、やっとの思いで会場にたどり着いた。	
\\	足元[あしもと]が 悪[わる]い 中[なか]を 頑張[がんば]って、やっとの 思[おも]いで 会場[かいじょう]にたどり 着[つ]いた。
\\	努力なくしてはこの事業はできない。	
\\	努力[どりょく]なくしてはこの 事業[じぎょう]はできない。
\\	山川教授なくしては今の私は存在しなかったであろう。	
\\	山川[やまかわ] 教授[きょうじゅ]なくしては 今[いま]の 私[わたし]は 存在[そんざい]しなかったであろう。
\\	コンピュータなくして現在の生活は不可能だ。	
\\	コンピュータなくして 現在[げんざい]の 生活[せいかつ]は 不可能[ふかのう]だ。
\\	の活用なくして企業の成長は望めません。	
\\	の 活用[かつよう]なくして 企業[きぎょう]の 成長[せいちょう]は 望[のぞ]めません。
\\	日本の平和と繁栄は、世界の平和と安定なくしてはあり得ない。	
\\	日本[にほん]の 平和[へいわ]と 繁栄[はんえい]は、 世界[せかい]の 平和[へいわ]と 安定[あんてい]なくしてはあり 得[え]ない。
\\	あらゆる生物は水なくして生きていけない。	
\\	あらゆる 生物[せいぶつ]は 水[みず]なくして 生[い]きていけない。
\\	コミュニケーションなくしては、人間が関与する一切の活動は成立し得ない。	
\\	コミュニケーションなくしては、 人間[にんげん]が 関与[かんよ]する 一切[いっさい]の 活動[かつどう]は 成立[せいりつ]し 得[え]ない。
\\	関係者の密接な共同作業なくしてはこの計画の実現は不可能であったろう。	
\\	関係[かんけい] 者[しゃ]の 密接[みっせつ]な 共同[きょうどう] 作業[さぎょう]なくしてはこの 計画[けいかく]の 実現[じつげん]は 不可能[ふかのう]であったろう。
\\	沿道の人たちの声援なくしてが、このマラソンは完走できなかったでしょう。	
\\	沿道[えんどう]の 人[ひと]たちの 声援[せいえん]なくしてが、このマラソンは 完走[かんそう]できなかったでしょう。
\\	彼女は別れの挨拶もなしに外国に行ってしまった。	
\\	彼女[かのじょ]は 別[わか]れの 挨拶[あいさつ]もなしに 外国[がいこく]に 行[い]ってしまった。
\\	ホームページの内容を事前の告知なしに変更することがございます。	
\\	ホームページの 内容[ないよう]を 事前[じぜん]の 告知[こくち]なしに 変更[へんこう]することがございます。
\\	いきなり男が断りなしに家に入ってきた。	
\\	いきなり 男[おとこ]が 断[ことわ]りなしに 家[いえ]に 入[はい]ってきた。
\\	僕はなまじっか英語ができるばかりに、会社で重宝がられて困っています。	
\\	僕[ぼく]はなまじっか 英語[えいご]ができるばかりに、 会社[かいしゃ]で 重宝[ちょうほう]がられて 困[こま]っています。
\\	なまじの学問は、ひけらかすと人に軽蔑される。	
\\	なまじの 学問[がくもん]は、ひけらかすと 人[ひと]に 軽蔑[けいべつ]される。
\\	なまじっか政治に手を出すのはやめた方がよい。	
\\	なまじっか 政治[せいじ]に 手[て]を 出[だ]すのはやめた 方[ほう]がよい。
\\	なまじっか手伝ってもらうより、一人でやったほうが早くできそうです。	
\\	なまじっか 手伝[てつだ]ってもらうより、一人[ひとり]でやったほうが 早[はや]くできそうです。
\\	彼はなまじっか自信があるために、失敗した時にひどい挫折感を味わう。	
\\	彼[かれ]はなまじっか 自信[じしん]があるために、 失敗[しっぱい]した 時[とき]にひどい 挫折[ざせつ] 感[かん]を 味[あじ]わう。
\\	なまじっか財産など持っていたら、最低の人間になってしまうのではないかと思う。	
\\	なまじっか 財産[ざいさん]など 持[も]っていたら、 最低[さいてい]の 人間[にんげん]になってしまうのではないかと 思[おも]う。
\\	なまじ見る目が肥えてくると、アニメだったら何でもいいてわけにはいかなくなる。	
\\	なまじ 見[み]る 目[め]が 肥[こ]えてくると、アニメだったら 何[なん]でもいいてわけにはいかなくなる。
\\	なまじっか物を知ってる人よりは全然知らない人の方が、「知らないことを知る楽しみ」を感じられると思う。	
\\	なまじっか 物[もの]を 知[し]ってる 人[ひと]よりは 全然[ぜんぜん] 知[し]らない 人[ひと]の 方[ほう]が、
\\	知[し]らないことを 知[し]る 楽[たの]しみ」を 感[かん]じられると 思[おも]う。
\\	なまじ頭がいいと組織の中で協力して働くということが難しくなる。	
\\	なまじ 頭[あたま]がいいと 組織[そしき]の 中[なか]で 協力[きょうりょく]して 働[はたら]くということが 難[むずか]しくなる。
\\	なまじっか元気だと、無理をしやすい。	
\\	なまじっか 元気[げんき]だと、 無理[むり]をしやすい。
\\	身体が健全に発達していなければ、なまじの美貌はかえって空しい。	
\\	身体[しんたい]が 健全[けんぜん]に 発達[はったつ]していなければ、なまじの 美貌[びぼう]はかえって 空[むな]しい。
\\	電池式ブザーはなまじの武器よりも襲われた時に効果が高い上に、コストパフォーマンスに優れています。	
\\	電池[でんち] 式[しき]ブザーはなまじの 武器[ぶき]よりも 襲[おそ]われた 時[とき]に 効果[こうか]が 高[たか]い 上[うえ]に、コストパフォーマンスに 優[すぐ]れています。
\\	体験者が語る真実は、なまじな小説よりも深く人の心を打つ。	
\\	体験[たいけん] 者[しゃ]が 語[かた]る 真実[しんじつ]は、なまじな 小説[しょうせつ]よりも 深[ふか]く 人[ひと]の 心[こころ]を 打[う]つ。
\\	なまじ運動しているのがよくなかったんです。それを言い訳にして、以前よりもたくさん食べるようになってしまいました。	
\\	なまじ 運動[うんどう]しているのがよくなかったんです。それを 言い訳[いいわけ]にして、 以前[いぜん]よりもたくさん 食[た]べるようになってしまいました。
\\	この機械は下手にいじると壊れるよ。	
\\	この 機械[きかい]は 下手[へた]にいじると 壊[こわ]れるよ。
\\	下手に上司につっかからない方がいい。	
\\	下手[へた]に 上司[じょうし]につっかからない 方[ほう]がいい。
\\	マイクさんのゴルフはプロ並みだ。	
\\	マイクさんのゴルフはプロ 並[な]みだ。
\\	吉田さんは植物については専門家並みの知識を持っている。	
\\	吉田[よしだ]さんは 植物[しょくぶつ]については 専門[せんもん] 家[か] 並[な]みの 知識[ちしき]を 持[も]っている。
\\	日本人も欧米人並みに肉や乳製品を食べるようになった。	
\\	日本人[にほんじん]も 欧米人[おうべいじん] 並[な]みに 肉[にく]や 乳製品[にゅうせいひん]を 食[た]べるようになった。
\\	今月の降水量は平年並みだそうだ。	
\\	今月[こんげつ]の 降水[こうすい] 量[りょう]は 平年[へいねん] 並[な]みだそうだ。
\\	当社に登録されている通訳の英語力はネイティブ並みです。	
\\	当社[とうしゃ]に 登録[とうろく]されている 通訳[つうやく]の 英語[えいご] 力[りょく]はネイティブ 並[な]みです。
\\	今日はまだ5月だというのに8月並みの暑さだ。	
\\	今日[きょう]はまだ 
\\	月[がつ]だというのに 
\\	月並[がつな]みの 暑[あつ]さだ。
\\	ダイエット用品、スポーツ器具、美容雑貨など、アメリカのヒット商品を現地並みの価格でお届けします。	
\\	ダイエット 用品[ようひん]、スポーツ 器具[きぐ]、 美容[びよう] 雑貨[ざっか]など、アメリカのヒット 商品[しょうひん]を 現地[げんち] 並[な]みの 価格[かかく]でお 届[とど]けします。
\\	今日はリハーサルだったが本番並みの緊張感を味わった。	
\\	今日[きょう]はリハーサルだったが 本番[ほんばん] 並[な]みの 緊張[きんちょう] 感[かん]を 味[あじ]わった。
\\	やっと仕事も見つかって何とか人並みに暮らしています。	
\\	やっと 仕事[しごと]も 見[み]つかって 何[なん]とか 人並[ひとな]みに 暮[く]らしています。
\\	私はパートなのに、正社員並みに仕事をさせられている。	
\\	私[わたし]はパートなのに、 正社員[せいしゃいん] 並[な]みに 仕事[しごと]をさせられている。
\\	電気自動車の価格をガソリン自動車並みに下げるのは極めれ難しい。	
\\	電気[でんき] 自動車[じどうしゃ]の 価格[かかく]をガソリン 自動車[じどうしゃ] 並[な]みに 下[さ]げるのは 極[きわ]めれ 難[むずか]しい。
\\	ここの寿司屋は回転寿司並みに安い。	
\\	ここの 寿司[すし] 屋[や]は 回転[かいてん] 寿司[ずし] 並[な]みに 安[やす]い。
\\	並みの日本語力ではこの仕事はできない。	
\\	並[な]みの 日本語[にほんご] 力[りょく]ではこの 仕事[しごと]はできない。
\\	東京では並みのホテルに泊まったが、それでも高かった。	
\\	東京[とうきょう]では 並[な]みのホテルに 泊[と]まったが、それでも 高[たか]かった。
\\	長谷川さんは並の人間ではない。	
\\	長谷川[はせがわ]さんは 並[なみ]の 人間[にんげん]ではない。
\\	私の妻は特に美人じゃありませんが、十人並み以上だと思います。	
\\	私[わたし]の 妻[つま]は 特[とく]に 美人[びじん]じゃありませんが、十人並[じゅうにんな]み 以上[いじょう]だと 思[おも]います。
\\	何でも十人並みにできるというのは自信にはつながらない。だから、どんな小さなことでも「これだけは誰にも負けない」と思えるものを持つことが大事だ。	
\\	何[なに]でも 十人[じゅうにん] 並[な]みにできるというのは 自信[じしん]にはつながらない。だから、どんな 小[ちい]さなことでも「これだけは 誰[だれ]にも 負[ま]けない」と 思[おも]えるものを 持[も]つことが 大事[だいじ]だ。
\\	私は日本へは何回行ったことがあるが、韓国と中国はまだだ。	
\\	私[わたし]は 日本[にほん]へは 何[なん] 回[かい] 行[い]ったことがあるが、 韓国[かんこく]と 中国[ちゅうごく]はまだだ。
\\	私は日本語の辞書を何十冊も持っている。	
\\	私[わたし]は 日本語[にほんご]の 辞書[じしょ]を 何[なん] 十冊[じっさつ]も 持[も]っている。
\\	私は今までに何百人もの認知症患者を診察した。	
\\	私[わたし]は 今[いま]までに 何[なん] 百[びゃく] 人[にん]もの 認知[にんち] 症[しょう] 患者[かんじゃ]を 診察[しんさつ]した。
\\	マンガ図書館に何万冊も在庫があっても、人気が高いマンガだと、誰かに借り出されている可能性が高い。	
\\	マンガ 図書館[としょかん]に 何[なん] 万[まん] 冊[さつ]も 在庫[ざいこ]があっても、 人気[にんき]が 高[たか]いマンガだと、 誰[だれ]かに 借り出[かりだ]されている 可能[かのう] 性[せい]が 高[たか]い。
\\	毎日のように我が家の上空をコウモリが何十匹も乱舞する。	
\\	毎日[まいにち]のように 我が家[わがや]の 上空[じょうくう]をコウモリが 何[なん] 十匹[じっぴき]も 乱舞[らんぶ]する。
\\	ヨーロッパには「農家民宿」が何万軒もあり、都会人や観光客を受け入れている。	
\\	ヨーロッパには
\\	農家[のうか] 民宿[みんしゅく]」が 何[なん] 万[まん] 軒[けん]もあり、 都会人[とかいじん]や 観光[かんこう] 客[きゃく]を 受け入[うけい]れている。
\\	人類は、何億年も積み重ねてきた生物としての経験を完全に捨てることなく、動物から人間へと進化した。	
\\	人類[じんるい]は、 何[なん] 億[おく] 年[ねん]も 積み重[つみかさ]ねてきた 生物[せいぶつ]としての 経験[けいけん]を 完全[かんぜん]に 捨[す]てることなく、 動物[どうぶつ]から 人間[にんげん]へと 進化[しんか]した。
\\	我が社のパイロット陣は
\\	の承認を得ており、何千時間もの飛行経験を持っています。	
\\	我[わ]が 社[しゃ]のパイロット 陣[じん]は 
\\	の 承認[しょうにん]を 得[え]ており、 何[なん] 千[せん] 時間[じかん]もの 飛行[ひこう] 経験[けいけん]を 持[も]っています。
\\	阪神・淡路大震災では、住宅倒壊により何千人もの命が一瞬のうちに奪われた。	
\\	阪神[はんしん]・ 淡路[あわじ] 大震災[だいしんさい]では、 住宅[じゅうたく] 倒壊[とうかい]により 何[なん] 千[せん] 人[にん]もの 命[いのち]が 一瞬[いっしゅん]のうちに 奪[うば]われた。
\\	このサイトを見ると、興味深い情報がいくつも見つかると思います。	
\\	このサイトを 見[み]ると、 興味深[きょうみぶか]い 情報[じょうほう]がいくつも 見[み]つかると 思[おも]います。
\\	彼女は何不自由ない裕福な家庭で育った。	
\\	彼女[かのじょ]は 何[なに] 不自由[ふじゆう]ない 裕福[ゆうふく]な 家庭[かてい]で 育[そだ]った。
\\	マイケルは何問題なく日本での留学生活を続けている。	
\\	マイケルは 何[なに] 問題[もんだい]なく 日本[にほん]での 留学[りゅうがく] 生活[せいかつ]を 続[つづ]けている。
\\	あの人は欠点が何一つない。	
\\	あの 人[ひと]は 欠点[けってん]が 何一[なにひと]つない。
\\	二人は結婚して、二人だけの蜜月を楽しみ、何不測ない甘い生活を送った。	
\\	二人[ふたり]は 結婚[けっこん]して、二人[ふたり]だけの 蜜月[みつげつ]を 楽[たの]しみ、 何[なに] 不測[ふそく]ない 甘[あま]い 生活[せいかつ]を 送[おく]った。
\\	この会社は、課長をはじめ若い社員が多く、何気兼ねたく話ができる雰囲気だ。	
\\	この 会社[かいしゃ]は、 課長[かちょう]をはじめ 若[わか]い 社員[しゃいん]が 多[おお]く、 何[なに] 気兼[きが]ねたく 話[はなし]ができる 雰囲気[ふんいき]だ。
\\	あれだけの財産を手に入れたからには、一生何不都合なく暮らせると彼は思ったのであろうが、それは間違っていたのだ。	
\\	あれだけの 財産[ざいさん]を 手[て]に 入[い]れたからには、 一生[いっしょう] 何[なん] 不都合[ふつごう]なく 暮[く]らせると 彼[かれ]は 思[おも]ったのであろうが、それは 間違[まちが]っていたのだ。
\\	日本人のホストファミリーと私は日本語で何不便なく意思疎通ができた。	
\\	日本人[にほんじん]のホストファミリーと 私[わたし]は 日本語[にほんご]で 何[なに] 不便[ふべん]なく 意思[いし] 疎通[そつう]ができた。
\\	自然界はすばらしいエコシステムになっていて、そこには無駄な物が何一つないようだ。	
\\	自然[しぜん] 界[かい]はすばらしいエコシステムになっていて、そこには 無駄[むだ]な 物[もの]が 何一[なにひと]つないようだ。
\\	今は銀行でさえ破綻する時代だ。100
\\	安全なものを実は何一つない。	
\\	今[いま]は 銀行[ぎんこう]でさえ 破綻[はたん]する 時代[じだい]だ。 
\\	安全[あんぜん]なものを 実[じつ]は 何一[なにひと]つない。
\\	そのドイツの片田舎の駅の待合室には誰一人いなかった。	
\\	そのドイツの 片田舎[かたいなか]の 駅[えき]の 待合室[まちあいしつ]には 誰[だれ] 一人[ひとり]いなかった。
\\	車は一台もなかった。	
\\	車[くるま]は一 台[だい]もなかった。
\\	リンゴは一個もない。	
\\	リンゴは 一個[いっこ]もない。
\\	何もそんなに怒ることはないよ。	
\\	何[なに]もそんなに 怒[おこ]ることはないよ。
\\	良性の腫瘍だから、何もそんなに心配する必要はない。	
\\	良性[りょうせい]の 腫瘍[しゅよう]だから、 何[なに]もそんなに 心配[しんぱい]する 必要[ひつよう]はない。
\\	何もいまさらみんながやっている研究をやる必要なんかないと思うだろう?それが間違いなんだ。	
\\	何[なに]もいまさらみんながやっている 研究[けんきゅう]をやる 必要[ひつよう]なんかないと 思[おも]うだろう?それが 間違[まちが]いなんだ。
\\	突然携帯電話が鳴ったのでびっくりした。友人にはすでに番号を知らせてあったから、何もそんなに驚くこともないのだが、悪いことに会議中だったのだ。	
\\	突然[とつぜん] 携帯[けいたい] 電話[でんわ]が 鳴[な]ったのでびっくりした。 友人[ゆうじん]にはすでに 番号[ばんごう]を 知[し]らせてあったから、 何[なに]もそんなに 驚[おどろ]くこともないのだが、 悪[わる]いことに 会議[かいぎ] 中[ちゅう]だったのだ。
\\	まあ落ち着いて下さい。私たちは何もあなたが犯人だと決めてかかっているわけじゃありませんから。	
\\	まあ 落ち着[おちつ]いて 下[くだ]さい。 私[わたし]たちは 何[なに]もあなたが 犯人[はんにん]だと 決[き]めてかかっているわけじゃありませんから。
\\	甘えるのは何も日本人に限ったことではない。	
\\	甘[あま]えるのは 何[なに]も 日本人[にほんじん]に 限[かぎ]ったことではない。
\\	彼女の性格を変えようとしているようだけど、何もそこまでしなくたっていいんじゃないか。	
\\	彼女[かのじょ]の 性格[せいかく]を 変[か]えようとしているようだけど、 何[なに]もそこまでしなくたっていいんじゃないか。
\\	元々十分に安いのだから、何もそんなに値切らなくてもと思うんですが、それがゲームになってしまうんです。	
\\	元々[もともと] 十分[じゅうぶん]に 安[やす]いのだから、 何[なに]もそんなに 値切[ねぎ]らなくてもと 思[おも]うんですが、それがゲームになってしまうんです。
\\	何らかの原因で走行中の電車が急停車した。	
\\	何[なん]らかの 原因[げんいん]で 走行[そうこう] 中[ちゅう]の 電車[でんしゃ]が 急停車[きゅうていしゃ]した。
\\	私たちが患う病気は、たいてい、何らかの意味でゲノムと関係があります。	
\\	私[わたし]たちが 患[わずら]う 病気[びょうき]は、たいてい、 何[なん]らかの 意味[いみ]でゲノムと 関係[かんけい]があります。
\\	世界の総人口のおよそ半数の人々が、何らかの形で米を食べている。	
\\	世界[せかい]の 総[そう] 人口[じんこう]のおよそ 半数[はんすう]の 人々[ひとびと]が、 何[なん]らかの 形[かたち]で 米[こめ]を 食[た]べている。
\\	何らかの理由で会議に参加できなくなった場合は、学会事務局へご連絡下さるようお願い申し上げます。	
\\	何[なん]らかの 理由[りゆう]で 会議[かいぎ]に 参加[さんか]できなくなった 場合[ばあい]は、 学会[がっかい] 事務[じむ] 局[きょく]へご 連絡[れんらく] 下[くだ]さるようお 願[ねが]い 申し上[もうしあ]げます。
\\	食品添加物が我々の身体に何らかの悪影響を与えると考える人は多い。	
\\	食品[しょくひん] 添加[てんか] 物[ぶつ]が 我々[われわれ]の 身体[しんたい]に 何[なん]らかの 悪影響[あくえいきょう]を 与[あた]えると 考[かんが]える 人[ひと]は 多[おお]い。
\\	火星表面の「運河」と呼ばれる溝は、何らかの液体によって作られたと考えられていた。	
\\	火星[かせい] 表面[ひょうめん]の
\\	運河[うんが]」と 呼[よ]ばれる 溝[みぞ]は、 何[なん]らかの 液体[えきたい]によって 作[つく]られたと 考[かんが]えられていた。
\\	我々はそれぞれ何らかの人生観、社会観を持って生きています。各人の生涯の生活設計に関しても、漠然とした形ではあっても、何らかのデザインをしていると思います。	
\\	我々[われわれ]はそれぞれ 何[なん]らかの 人生[じんせい] 観[かん]、 社会[しゃかい] 観[かん]を 持[も]って 生[い]きています。 各人[かくじん]の 生涯[しょうがい]の 生活[せいかつ] 設計[せっけい]に 関[かん]しても、 漠然[ばくぜん]とした 形[かたち]ではあっても、 何[なん]らかのデザインをしていると 思[おも]います。
\\	日本の大学教育の現状は何ら改善されていない。	
\\	日本[にほん]の 大学[だいがく] 教育[きょういく]の 現状[げんじょう]は 何[なん]ら 改善[かいぜん]されていない。
\\	会社側は労働組合に対して何らの誠意ある回答もしなかった。	
\\	会社[かいしゃ] 側[がわ]は 労働[ろうどう] 組合[くみあい]に 対[たい]して 何[なん]らの 誠意[せいい]ある 回答[かいとう]もしなかった。
\\	私は彼女とは何ら面識もないので、彼女の能力を知る由もありません。	
\\	私[わたし]は 彼女[かのじょ]とは 何[なん]ら 面識[めんしき]もないので、 彼女[かのじょ]の 能力[のうりょく]を 知[し]る 由[よし]もありません。
\\	ソウルの町は、人の話さえ聞かなければ東京にいるのと何ら変わりがありません。	
\\	ソウルの 町[まち]は、 人[ひと]の 話[はなし]さえ 聞[き]かなければ 東京[とうきょう]にいるのと 何[なん]ら 変[か]わりがありません。
\\	このドラマに登場する人物は実在する個人とは何ら関わりがありません。	
\\	このドラマに 登場[とうじょう]する 人物[じんぶつ]は 実在[じつざい]する 個人[こじん]とは 何[なん]ら 関[かか]わりがありません。
\\	1990年の入管法の改正で日系人はその在留が合法化され、日本での就労についても何ら制限を受けないことになった。	
\\	年[ねん]の 入管[にゅうかん] 法[ほう]の 改正[かいせい]で 日系[にっけい] 人[じん]はその 在留[ざいりゅう]が 合法[ごうほう] 化[か]され、 日本[にほん]での 就労[しゅうろう]についても 何[なん]ら 制限[せいげん]を 受[う]けないことになった。
\\	当ホテルはホテル内での紛失に関しては何らの責任も負いません。	
\\	当[とう]ホテルはホテル 内[ない]での 紛失[ふんしつ]に 関[かん]しては 何[なん]らの 責任[せきにん]も 負[お]いません。
\\	彼女は生まれつき楽観的で、自分の将来についても何らの不安も感じていなかった。	
\\	彼女[かのじょ]は 生[う]まれつき 楽観[らっかん] 的[てき]で、 自分[じぶん]の 将来[しょうらい]についても 何[なん]らの 不安[ふあん]も 感[かん]じていなかった。
\\	この町は災害に対して何らの備えもない。	
\\	この 町[まち]は 災害[さいがい]に 対[たい]して 何[なん]らの 備[そな]えもない。
\\	うちの社長は厳しくて、何らの失敗も許してくれない。	
\\	うちの 社長[しゃちょう]は 厳[きび]しくて、 何[なん]らの 失敗[しっぱい]も 許[ゆる]してくれない。
\\	レポートのこと、社長はなんて言っていましたか。	
\\	レポートのこと、 社長[しゃちょう]はなんて 言[い]っていましたか。
\\	きのう見た映画、なんていう映画だったっけ?	
\\	きのう 見[み]た 映画[えいが]、なんていう 映画[えいが]だったっけ?
\\	これぐらいの宿題、なんてことはないよ。	
\\	これぐらいの 宿題[しゅくだい]、なんてことはないよ。
\\	うちにホームステイしている留学生がテレビを観ていると「今なんで言った?」「英語だとなんて言うの?」と質問の連続だ。	
\\	うちにホームステイしている 留学生[りゅうがくせい]がテレビを 観[み]ていると
\\	今[いま]なんで 言[い]った?」
\\	英語[えいご]だとなんて 言[い]うの?」と 質問[しつもん]の 連続[れんぞく]だ。
\\	推薦状を書くのを頼まれると、時々なんて書いたらいいのか分からないことがある。	
\\	推薦[すいせん] 状[じょう]を 書[か]くのを 頼[たの]まれると、 時々[ときどき]なんて 書[か]いたらいいのか 分[わ]からないことがある。
\\	なんていう人か名前を忘れたけれど、なかなか魅力的な女性に出会った。	
\\	なんていう 人[ひと]か 名前[なまえ]を 忘[わす]れたけれど、なかなか 魅力[みりょく] 的[てき]な 女性[じょせい]に 出会[であ]った。
\\	彼はつらい目にたくさん遭っているのに、「なんていうことはない」という感じでひょうひょうと生きている。	
\\	彼[かれ]はつらい 目[め]にたくさん 遭[あ]っているのに、「なんていうことはない」という 感[かん]じでひょうひょうと 生[い]きている。
\\	うさぎをじーっと見てると、なんてかわいいんだろう、なんてかわいい仕草をするんだろうって思った。	
\\	うさぎをじーっと 見[み]てると、なんてかわいいんだろう、なんてかわいい 仕草[しぐさ]をするんだろうって 思[おも]った。
\\	自分のことを書くのって、なんて難しいんだろう。	
\\	自分[じぶん]のことを 書[か]くのって、なんて 難[むずか]しいんだろう。
\\	小学校からのいじめが中学まで続くなんてひどい。	
\\	小学校[しょうがっこう]からのいじめが 中学[ちゅうがく]まで 続[つづ]くなんてひどい。
\\	最近は電子メールばかり使っていて、手紙なんて滅多に書かない。	
\\	最近[さいきん]は 電子[でんし]メールばかり 使[つか]っていて、 手紙[てがみ]なんて 滅多[めった]に 書[か]かない。
\\	生演奏で踊れるなんて、うらやましい。	
\\	生[なま] 演奏[えんそう]で 踊[おど]れるなんて、うらやましい。
\\	大勢の皆さんに家まで来ていただけるなんて、とても嬉しいです。	
\\	大勢[おおぜい]の 皆[みな]さんに 家[いえ]まで 来[き]ていただけるなんて、とても 嬉[うれ]しいです。
\\	夏にエアコンをつけすぎて寒いなんて、電力浪費も甚だしい。	
\\	夏[なつ]にエアコンをつけすぎて 寒[さむ]いなんて、 電力[でんりょく] 浪費[ろうひ]も 甚[はなは]だしい。
\\	あの人がそんなに有名だなんてちょっとも知らなかった。	
\\	あの 人[ひと]がそんなに 有名[ゆうめい]だなんてちょっとも 知[し]らなかった。
\\	あなたなんて最低よ。	
\\	あなたなんて 最低[さいてい]よ。
\\	納豆なんて嫌いだ。	
\\	納豆[なっとう]なんて 嫌[きら]いだ。
\\	我々は今の日本の政治状況を何とか変えようと努力しています。	
\\	我々[われわれ]は 今[いま]の 日本[にほん]の 政治[せいじ] 状況[じょうきょう]を 何[なん]とか 変[か]えようと 努力[どりょく]しています。
\\	一見無理なことも、やってみれば何とか形になる。	
\\	一見[いっけん] 無理[むり]なことも、やってみれば 何[なん]とか 形[かたち]になる。
\\	この夏は友達と一緒に何とか富士山に登ってきました。	
\\	この 夏[なつ]は 友達[ともだち]と 一緒[いっしょ]に 何[なん]とか 富士山[ふじさん]に 登[のぼ]ってきました。
\\	夜型の生活を何とかしたいと思いながら、今でもそれを続けている。	
\\	夜[よる] 型[がた]の 生活[せいかつ]を 何[なん]とかしたいと 思[おも]いながら、 今[いま]でもそれを 続[つづ]けている。
\\	ボスにせかされているから、この書類を何とか明日までに仕上げなければならないんだ。	
\\	ボスにせかされているから、この 書類[しょるい]を 何[なん]とか 明日[あす]までに 仕上[しあ]げなければならないんだ。
\\	[患者:]ーおでこから血が出ることがあるんです。 [医者:]ーおでこから血が?うーん。そんな症例は聞いたことがないなあ……。 [患者:]ーほんとなんです。お願いします、先生。何とかして下さい。	
\\	患者[かんじゃ]:]ーおでこから 血[ち]が 出[で]ることがあるんです。 
\\	医者[いしゃ]:]ーおでこから 血[ち]が?うーん。そんな 症例[しょうれい]は 聞[き]いたことがないなあ……。 
\\	患者[かんじゃ]:]ーほんとなんです。お 願[ねが]いします、 先生[せんせい]。 何[なん]とかして 下[くだ]さい。
\\	[母親:]-あら、有香、もう勉強は終わったの。 [娘:]-勉強なんかやってらんないわよ。 [母親:]-まったくしょうがない子ねえ。お父さん、何とか言って下さいよ。	
\\	母親[ははおや]:]あら、 有香[ゆか]、もう 勉強[べんきょう]は 終[お]わったの。 
\\	娘[むすめ]:] 勉強[べんきょう]なんかやってらんないわよ。 
\\	母親[ははおや]:]まったくしょうがない 子[こ]ねえ。お 父[とう]さん、 何[なん]とか 言[い]って 下[くだ]さいよ。
\\	何とかホームページができたのでみんなに見せたい。	
\\	何[なん]とかホームページができたのでみんなに 見[み]せたい。
\\	アメリカへ行けば何とかなるなんていう考えでアメリカに来たって、自分で何とかしなきゃ何ともならないんだ!	
\\	アメリカへ 行[い]けば 何[なん]とかなるなんていう 考[かんが]えでアメリカに 来[き]たって、 自分[じぶん]で 何[なん]とかしなきゃ 何[なに]ともならないんだ!
\\	いろんな形で支援して下さった皆様のおかげで、このプロジェクトは何とか無事に終了しました。	
\\	いろんな 形[かたち]で 支援[しえん]して 下[くだ]さった 皆様[みなさま]のおかげで、このプロジェクトは 何[なん]とか 無事[ぶじ]に 終了[しゅうりょう]しました。
\\	宮崎駿のアニメは、20年30年たっても、なお古典アニメとして残っているだろう。	
\\	宮崎[みやざき] 駿[はやお]のアニメは、 
\\	年[ねん] 
\\	年[ねん]たっても、なお 古典[こてん]アニメとして 残[のこ]っているだろう。
\\	この小説はパリを知っている人が読めばなお面白いだろう。	
\\	この 小説[しょうせつ]はパリを 知[し]っている 人[ひと]が 読[よ]めばなお 面白[おもしろ]いだろう。
\\	日本へ行く前から日本が好きだったが、日本へ行ったらなお好きになった。	
\\	日本[にほん]へ 行[い]く 前[まえ]から 日本[にほん]が 好[す]きだったが、 日本[にほん]へ 行[い]ったらなお 好[す]きになった。
\\	来週の金曜日午後6時から拙宅で夕食会をしますので、お知らせいたします。なお、ご都合のつかない方はお電話下さい。	
\\	来週[らいしゅう]の 金曜日[きんようび] 午後[ごご] 
\\	時[じ]から 拙宅[せったく]で 夕食[ゆうしょく] 会[かい]をしますので、お 知[し]らせいたします。なお、ご 都合[つごう]のつかない 方[かた]はお 電話[でんわ] 下[くだ]さい。
\\	会は毎週月曜日です。なお詳しい情報は会のサイトをご覧下さい。	
\\	会[かい]は 毎週[まいしゅう] 月曜日[げつようび]です。なお 詳[くわ]しい 情報[じょうほう]は 会[かい]のサイトをご 覧[らん] 下[くだ]さい。
\\	欧米やアジアでは景気は拡大し、総じて良好な状態がなお続いている。	
\\	欧米[おうべい]やアジアでは 景気[けいき]は 拡大[かくだい]し、 総[そう]じて 良好[りょうこう]な 状態[じょうたい]がなお 続[つづ]いている。
\\	テレビで「三億円強奪事件」というドラマを見たが、面白かった。実話をもとにしているから、なお面白い。	
\\	テレビで「三 億[おく] 円[えん] 強奪[ごうだつ] 事件[じけん]」というドラマを 見[み]たが、 面白[おもしろ]かった。 実話[じつわ]をもとにしているから、なお 面白[おもしろ]い。
\\	考えていることを書くのは勉強になる。その文章を先生が読んで感想を書いてくれるから、なお勉強になる。	
\\	考[かんが]えていることを 書[か]くのは 勉強[べんきょう]になる。その 文章[ぶんしょう]を 先生[せんせい]が 読[よ]んで 感想[かんそう]を 書[か]いてくれるから、なお 勉強[べんきょう]になる。
\\	日本語で話すことは難しいが、書くことはなお難しい。	
\\	日本語[にほんご]で 話[はな]すことは 難[むずか]しいが、 書[か]くことはなお 難[むずか]しい。
\\	この論文を完成するまでには、なお2か月は必要だろう。	
\\	この 論文[ろんぶん]を 完成[かんせい]するまでには、なお 2か月[にかげつ]は 必要[ひつよう]だろう。
\\	金沢での夏季日本語講座参加希望者は、今月末までに願書を提出して下さい。なお、奨学金応募希望者は、至急用紙を取りに学部事務所に来て下さい。	
\\	金沢[かなざわ]での 夏季[かき] 日本語[にほんご] 講座[こうざ] 参加[さんか] 希望[きぼう] 者[しゃ]は、 今月[こんげつ] 末[まつ]までに 願書[がんしょ]を 提出[ていしゅつ]して 下[くだ]さい。なお、 奨学[しょうがく] 金[きん] 応募[おうぼ] 希望[きぼう] 者[しゃ]は、 至急[しきゅう] 用紙[ようし]を 取[と]りに 学部[がくぶ] 事務所[じむしょ]に 来[き]て 下[くだ]さい。
\\	私は月曜日と水曜日の午後2時から5時までは研究室にいますから、質問のある学生は自由に来て下さい。なお、それ以外の日に研究室に来たい学生は、あらかじめアポを取って下さい。	
\\	私[わたし]は 月曜日[げつようび]と 水曜日[すいようび]の 午後[ごご] 
\\	時[じ]から 
\\	時[じ]までは 研究[けんきゅう] 室[しつ]にいますから、 質問[しつもん]のある 学生[がくせい]は 自由[じゆう]に 来[き]て 下[くだ]さい。なお、それ 以外[いがい]の 日[ひ]に 研究[けんきゅう] 室[しつ]に 来[き]たい 学生[がくせい]は、あらかじめアポを 取[と]って 下[くだ]さい。
\\	前からタイに行きたいと思っていたが、すでに行ったことのある友人が強く勧めたので、なおさら行きたくなった。	
\\	前[まえ]からタイに 行[い]きたいと 思[おも]っていたが、すでに 行[い]ったことのある 友人[ゆうじん]が 強[つよ]く 勧[すす]めたので、なおさら 行[い]きたくなった。
\\	悪人さえも往生するのなら、まして善人はなおさらだ。	
\\	悪人[あくにん]さえも 往生[おうじょう]するのなら、まして 善人[ぜんにん]はなおさらだ。
\\	貧乏を弁解すると、なおさら貧乏くさく聞こえる。	
\\	貧乏[びんぼう]を 弁解[べんかい]すると、なおさら 貧乏[びんぼう]くさく 聞[き]こえる。
\\	このチーズは搾りたての新鮮な牛乳で作るから、なおさらおいしい。	
\\	このチーズは 搾[しぼ]りたての 新鮮[しんせん]な 牛乳[ぎゅうにゅう]で 作[つく]るから、なおさらおいしい。
\\	旅で足を踏み入れるだけで、沖縄はいまだに戦争を引きずっていることを感じる。まして、現地の人ではなおさらであろう。	
\\	旅[たび]で 足[あし]を 踏み入[ふみい]れるだけで、 沖縄[おきなわ]はいまだに 戦争[せんそう]を 引[ひ]きずっていることを 感[かん]じる。まして、 現地[げんち]の 人[ひと]ではなおさらであろう。
\\	英語を書くのは難しいが、正確さが要求される科学論文を書くのはなおさらだ。	
\\	英語[えいご]を 書[か]くのは 難[むずか]しいが、 正確[せいかく]さが 要求[ようきゅう]される 科学[かがく] 論文[ろんぶん]を 書[か]くのはなおさらだ。
\\	アジア諸国の学者や政府が、この教科書を詳細かつ適確に批判し、発酵しないよう強く求めたにもにもに関わらず検定を通ったということは、たおさら重大です。	
\\	アジア 諸国[しょこく]の 学者[がくしゃ]や 政府[せいふ]が、この 教科書[きょうかしょ]を 詳細[しょうさい]かつ 適確[てきかく]に 批判[ひはん]し、 発酵[はっこう]しないよう 強[つよ]く 求[もと]めたにもにもに 関[かか]わらず 検定[けんてい]を 通[とお]ったということは、たおさら 重大[じゅうだい]です。
\\	前からニューヨークは危ないところだと思っていたが、あのテロ事件以来、なおさら、危険なところだと思うようになった。	
\\	前[まえ]からニューヨークは 危[あぶ]ないところだと 思[おも]っていたが、あのテロ 事件[じけん] 以来[いらい]、なおさら、 危険[きけん]なところだと 思[おも]うようになった。
\\	丸山博士の講演は、一流の科学者ならではの示唆に富む内容だった。	
\\	丸山[まるやま] 博士[はかせ]の 講演[こうえん]は、 一流[いちりゅう]の 科学[かがく] 者[しゃ]ならではの 示唆[しさ]に 富[と]む 内容[ないよう]だった。
\\	四季通じて美しい景色が楽しめるのはこの場ならではだ。	
\\	四季[しき] 通[つう]じて 美[うつく]しい 景色[けしき]が 楽[たの]しめるのはこの 場[ば]ならではだ。
\\	女性ならではできない仕事や役割もあるはずだ。	
\\	女性[じょせい]ならではできない 仕事[しごと]や 役割[やくわり]もあるはずだ。
\\	この大学の法律図書館には、この大学ならではの充実した資料が揃っている。	
\\	この 大学[だいがく]の 法律[ほうりつ] 図書館[としょかん]には、この 大学[だいがく]ならではの 充実[じゅうじつ]した 資料[しりょう]が 揃[そろ]っている。
\\	プロショップならではの品揃え。人気商品を特別価格でお求めいただけます。	
\\	プロショップならではの 品[しな] 揃[ぞろ]え。 人気[にんき] 商品[しょうひん]を 特別[とくべつ] 価格[かかく]でお 求[もと]めいただけます。
\\	このマンションは都心の住居ならではの便利さと快適さをお約束します。	
\\	このマンションは 都心[としん]の 住居[じゅうきょ]ならではの 便利[べんり]さと 快適[かいてき]さをお 約束[やくそく]します。
\\	興味がわいたら、その分野を徹底して勉強してみる時間があるというのは学生ならではだ。	
\\	興味[きょうみ]がわいたら、その 分野[ぶんや]を 徹底[てってい]して 勉強[べんきょう]してみる 時間[じかん]があるというのは 学生[がくせい]ならではだ。
\\	近郊にはゴルフコースが数多く、仕事帰りにハーフを回れるのは、緯度が高く日没の遅いカナダならではだ。	
\\	近郊[きんこう]にはゴルフコースが 数多[かずおお]く、 仕事[しごと] 帰[がえ]りにハーフを 回[まわ]れるのは、 緯度[いど]が 高[たか]く 日没[にちぼつ]の 遅[おそ]いカナダならではだ。
\\	産地直送ならでは味わえない本場のおいしさをお届けいたします。	
\\	産地[さんち] 直送[ちょくそう]ならでは 味[あじ]わえない 本場[ほんば]のおいしさをお 届[とど]けいたします。
\\	日本最高峰の富士山は、毎年夏の登山シーズンになると約20万人が山頂を目指す。登山者が長蛇の列をつくる光景は富士山ならでは見られないものである。	
\\	日本[にほん] 最高峰[さいこうほう]の 富士山[ふじさん]は、 毎年[まいとし] 夏[なつ]の 登山[とざん]シーズンになると 約[やく] 
\\	万[まん] 人[にん]が 山頂[さんちょう]を 目指[めざ]す。 登山[とざん] 者[しゃ]が 長蛇[ちょうだ]の 列[れつ]をつくる 光景[こうけい]は 富士山[ふじさん]ならでは 見[み]られないものである。
\\	房子は彼女ならではの趣味の悪い服を着ていた。	
\\	房子[ふさこ]は 彼女[かのじょ]ならではの 趣味[しゅみ]の 悪[わる]い 服[ふく]を 着[き]ていた。
\\	そういうくだらない考えは、ジョージならでは思いつかないことだ。	
\\	そういうくだらない 考[かんが]えは、ジョージならでは 思[おも]いつかないことだ。
\\	国連総会には首相並びに外務大臣が出席した。	
\\	国連[こくれん] 総会[そうかい]には 首相[しゅしょう] 並[なら]びに 外務[がいむ] 大臣[だいじん]が 出席[しゅっせき]した。
\\	県並びに県内各市町村では、地域イントラネットを構築して、行政や福祉情報の円滑な提供に努めている。	
\\	県[けん] 並[なら]びに 県内[けんない] 各[かく] 市町村[しちょうそん]では、 地域[ちいき]イントラネットを 構築[こうちく]して、 行政[ぎょうせい]や 福祉[ふくし] 情報[じょうほう]の 円滑[えんかつ]な 提供[ていきょう]に 努[つと]めている。
\\	以下は犬及びねこの引き取り並びに負傷動物の収容に関する措置要領である。	
\\	以下[いか]は 犬[いぬ] 及[およ]びねこの 引き取[ひきと]り 並[なら]びに 負傷[ふしょう] 動物[どうぶつ]の 収容[しゅうよう]に 関[かん]する 措置[そち] 要領[ようりょう]である。
\\	その政党は、夫婦間における暴力の防止並びに被害者の保護に関する法律案を発表した。	
\\	その 政党[せいとう]は、 夫婦[ふうふ] 間[かん]における 暴力[ぼうりょく]の 防止[ぼうし] 並[なら]びに 被害[ひがい] 者[しゃ]の 保護[ほご]に 関[かん]する 法律[ほうりつ] 案[あん]を 発表[はっぴょう]した。
\\	今日は、大気や河川の汚染並びに酸性雨が環境に及ぼす影響についてお話いたします。	
\\	今日[きょう]は、 大気[たいき]や 河川[かせん]の 汚染[おせん] 並[なら]びに 酸性[さんせい] 雨[う]が 環境[かんきょう]に 及[およ]ぼす 影響[えいきょう]についてお 話[はなし]いたします。
\\	このホームページ並びにメールマガジンのスタッフをご紹介します。	
\\	このホームページ 並[なら]びにメールマガジンのスタッフをご 紹介[しょうかい]します。
\\	本学学長の選挙資格者は、選挙公示の日における学長並びに専任の教授、准教授及び講師とする。	
\\	本学[ほんがく] 学長[がくちょう]の 選挙[せんきょ] 資格[しかく] 者[しゃ]は、 選挙[せんきょ] 公示[こうじ]の 日[ひ]における 学長[がくちょう] 並[なら]びに 専任[せんにん]の 教授[きょうじゅ]、 准[じゅん] 教授[きょうじゅ] 及[およ]び 講師[こうし]とする。
\\	本日ここに、大統領、並びに国務長官ほか、御一行の歓迎の宴を開くことができましたことは、私の最も欣快とするところであります。	
\\	本日[ほんじつ]ここに、 大統領[だいとうりょう]、 並[なら]びに 国務[こくむ] 長官[ちょうかん]ほか、 御[ご] 一行[いっこう]の 歓迎[かんげい]の 宴[うたげ]を 開[ひら]くことができましたことは、 私[わたし]の 最[もっと]も 欣快[きんかい]とするところであります。
\\	糖尿病患者は、糖分並びに炭水化物を摂取しすぎない方がよい。	
\\	糖尿[とうにょう] 病[びょう] 患者[かんじゃ]は、 糖分[とうぶん] 並[なら]びに 炭水化物[たんすいかぶつ]を 摂取[せっしゅ]しすぎない 方[ほう]がよい。
\\	日本語には、動詞及び形容詞のように活用する品詞、並びに名詞、接続詞、及び助詞のように活用しない品詞がある。	
\\	日本語[にほんご]には、 動詞[どうし] 及[およ]び 形容詞[けいようし]のように 活用[かつよう]する 品詞[ひんし]、 並[なら]びに 名詞[めいし]、 接続詞[せつぞくし]、 及[およ]び 助詞[じょし]のように 活用[かつよう]しない 品詞[ひんし]がある。
\\	妻は私の顔を見るなり泣き出した。	
\\	妻[つま]は 私[わたし]の 顔[かお]を 見[み]るなり 泣[な]き 出[だ]した。
\\	首相は会議場を出るなり記者団の質問攻めにあった。	
\\	首相[しゅしょう]は 会議[かいぎ] 場[じょう]を 出[で]るなり 記者[きしゃ] 団[だん]の 質問[しつもん] 攻[ぜ]めにあった。
\\	部屋に入るなりルームサービスの人がお茶とお菓子を持ってきてくれた。	
\\	部屋[へや]に 入[はい]るなりルームサービスの 人[ひと]がお 茶[ちゃ]とお 菓子[かし]を 持[も]ってきてくれた。
\\	編集長は私と顔を合わせるなり私をにらみつけ、「話がある」と言った。	
\\	編集[へんしゅう] 長[ちょう]は 私[わたし]と 顔[かお]を 合[あ]わせるなり 私[わたし]をにらみつけ、
\\	話[はなし]がある」と 言[い]った。
\\	家に帰るまで何とかもつかと思ったが、バスを降りるなり激しい雨が降り出した。	
\\	家[いえ]に 帰[かえ]るまで 何[なん]とかもつかと 思[おも]ったが、バスを 降[お]りるなり 激[はげ]しい 雨[あめ]が 降り出[ふりだ]した。
\\	会議中たばこが吸えなくて禁断症状だった伊藤さんは、部屋を出るなりたばこに火をつけた。	
\\	会議[かいぎ] 中[ちゅう]たばこが 吸[す]えなくて 禁断症状[きんだんしょうじょう]だった 伊藤[いとう]さんは、 部屋[へや]を 出[で]るなりたばこに 火[ひ]をつけた。
\\	そのニュースが報道されるなり、放送局に問い合われの電話が殺到した。	
\\	そのニュースが 報道[ほうどう]されるなり、 放送[ほうそう] 局[きょく]に 問[と]い 合[あ]われの 電話[でんわ]が 殺到[さっとう]した。
\\	所属レコード会社が宣伝に力を入れたため、そのアルバムは発売されるなりヒットチャート1位になった。	
\\	所属[しょぞく]レコード 会社[かいしゃ]が 宣伝[せんでん]に 力[ちから]を 入[い]れたため、そのアルバムは 発売[はつばい]されるなりヒットチャート 
\\	位[い]になった。
\\	オフィスに着いたらすぐ電話を下さい。	
\\	オフィスに 着[つ]いたらすぐ 電話[でんわ]を 下[くだ]さい。
\\	空港からお電話をいただいたらすぐ課の者がお迎えに参ります。	
\\	空港[くうこう]からお 電話[でんわ]をいただいたらすぐ 課[か]の 者[もの]がお 迎[むか]えに 参[まい]ります。
\\	彼らは社長の許可が出たらすぐこのプロジェクトを始めるつもりだ。	
\\	彼[かれ]らは 社長[しゃちょう]の 許可[きょか]が 出[で]たらすぐこのプロジェクトを 始[はじ]めるつもりだ。
\\	アンはいつもオフィスに着くなりすぐメールをチェックする。	
\\	アンはいつもオフィスに 着[つ]くなりすぐメールをチェックする。
\\	挨拶するなり自分の部屋に入ってしまうというのはお客さまに失礼だよ。	
\\	挨拶[あいさつ]するなり 自分[じぶん]の 部屋[へや]に 入[い]ってしまうというのはお 客[きゃく]さまに 失礼[しつれい]だよ。
\\	私は社長が部屋に入ってくるなり無意識に立ち上がった。	
\\	私[わたし]は 社長[しゃちょう]が 部屋[へや]に 入[はい]ってくるなり 無意識[むいしき]に 立ち上[たちあ]がった。
\\	私は家に帰るなり目眩がして玄関にしゃがみこんでしまった。	
\\	私[わたし]は 家[いえ]に 帰[かえ]るなり 目眩[めまい]がして 玄関[げんかん]にしゃがみこんでしまった。
\\	藤原さんは社長が部屋に入ってくるなり立ち上がって挨拶した。	
\\	藤原[ふじわら]さんは 社長[しゃちょう]が 部屋[へや]に 入[はい]ってくるなり 立ち上[たちあ]がって 挨拶[あいさつ]した。
\\	ジョンは家に帰るなり冷蔵庫からビールを取り出して一息に飲み干した。	
\\	ジョンは 家[いえ]に 帰[かえ]るなり 冷蔵庫[れいぞうこ]からビールを 取り出[とりだ]して 一息[ひといき]に 飲み干[のみほ]した。
\\	君は先生の顔を見るなり何か話しに行ったけど、どうかしたの。	
\\	君[きみ]は 先生[せんせい]の 顔[かお]を 見[み]るなり 何[なに]か 話[はな]しに 行[い]ったけど、どうかしたの。
\\	私は社長が部屋に入ってくるとすぐ立ち上がって挨拶した。	
\\	私[わたし]は 社長[しゃちょう]が 部屋[へや]に 入[はい]ってくるとすぐ 立ち上[たちあ]がって 挨拶[あいさつ]した。
\\	私は家に帰ると同時に冷蔵庫からビールを取り出して一息に飲み干した。	
\\	私[わたし]は 家[いえ]に 帰[かえ]ると 同時[どうじ]に 冷蔵庫[れいぞうこ]からビールを 取り出[とりだ]して 一息[ひといき]に 飲み干[のみほ]した。
\\	パリはきれいだと聞いていたが、なるほどその芸術性には舌を巻いた。	
\\	パリはきれいだと 聞[き]いていたが、なるほどその 芸術[げいじゅつ] 性[せい]には 舌[した]を 巻[ま]いた。
\\	なるほど、日本の官僚はいろいろと批判されているが、戦後の日本の復興は彼らに負うところが多い。	
\\	なるほど、 日本[にほん]の 官僚[かんりょう]はいろいろと 批判[ひはん]されているが、 戦後[せんご]の 日本[にほん]の 復興[ふっこう]は 彼[かれ]らに 負[お]うところが 多[おお]い。
\\	ー文化って、宗教のように倫理的な決まりが多いですね。 ーなるほどね。	ー
\\	ー 文化[ぶんか]って、 宗教[しゅうきょう]のように 倫理[りんり] 的[てき]な 決[き]まりが 多[おお]いですね。 ーなるほどね。
\\	この大学のことは大学案内で読んではいたが、実際に来てみると、なるほど、学生たちはすばらしい教育を受けているようだ。	
\\	この 大学[だいがく]のことは 大学[だいがく] 案内[あんない]で 読[よ]んではいたが、 実際[じっさい]に 来[き]てみると、なるほど、 学生[がくせい]たちはすばらしい 教育[きょういく]を 受[う]けているようだ。
\\	この辺りでは菜の花畑が4月中旬ごろ特にきれいだと聞いていたが、なるほど見事である。	
\\	この 辺[あた]りでは 菜の花[なのはな] 畑[ばたけ]が 
\\	月[がつ] 中旬[ちゅうじゅん]ごろ 特[とく]にきれいだと 聞[き]いていたが、なるほど 見事[みごと]である。
\\	本日、川上弘美「椰子・椰子」を読了。なるほど面白い本だ。	
\\	本日[ほんじつ]、 川上[かわかみ] 弘美[ひろみ]
\\	椰子[やし]・ 椰子[やし]」を 読了[どくりょう]。なるほど 面白[おもしろ]い 本[ほん]だ。
\\	なるほど、これだけの規模のテロが起きると人々は恐怖に捕われるけれど、自分でコントロールできないことをくよくよ心配しても仕方がないだろう。	
\\	なるほど、これだけの 規模[きぼ]のテロが 起[お]きると 人々[ひとびと]は 恐怖[きょうふ]に 捕[とら]われるけれど、 自分[じぶん]でコントロールできないことをくよくよ 心配[しんぱい]しても 仕方[しかた]がないだろう。
\\	なるほど日本の文化とドイツの文化は違っている点があるが、実は共痛点の方が多いようだ。	
\\	なるほど 日本[にほん]の 文化[ぶんか]とドイツの 文化[ぶんか]は 違[ちが]っている 点[てん]があるが、 実[じつ]は 共[きょう] 痛点[つうてん]の 方[ほう]が 多[おお]いようだ。
\\	ー実は僕には好きな人がいるんですが、勇気がなくて告白できないんです。 ーなるほど。でも、それは自分に対する自信が足リないからじゃないの?	ー
\\	ー 実[じつ]は 僕[ぼく]には 好[す]きな 人[ひと]がいるんですが、 勇気[ゆうき]がなくて 告白[こくはく]できないんです。 ーなるほど。でも、それは 自分[じぶん]に 対[たい]する 自信[じしん]が 足[あし]リないからじゃないの?
\\	ーこの使い捨てカメラって、写真を撮りたい時簡単に買えて、撮ったらそのままカメラ屋に現像に出せるのが魅力なんだ。面倒なフィルムの出し入れもないし。 ーなるほど。	ー
\\	ーこの 使い捨[つかいす]てカメラって、 写真[しゃしん]を 撮[と]りたい 時[とき] 簡単[かんたん]に 買[か]えて、 撮[と]ったらそのままカメラ 屋[や]に 現像[げんぞう]に 出[だ]せるのが 魅力[みりょく]なんだ。 面倒[めんどう]なフィルムの 出し入[だしい]れもないし。 ーなるほど。
\\	僕の自転車、たしかにここに置いといたんだけど、ない。どうしたんだろう。	
\\	僕[ぼく]の 自転車[じてんしゃ]、たしかにここに 置[お]いといたんだけど、ない。どうしたんだろう。
\\	彼女、さっきまでたしかにここにいたんだけど、どこに行ってしまったんだろう。	
\\	彼女[かのじょ]、さっきまでたしかにここにいたんだけど、どこに 行[い]ってしまったんだろう。
\\	日本の生活費はたしかに高いね。	
\\	日本[にほん]の 生活[せいかつ] 費[ひ]はたしかに 高[たか]いね。
\\	料金は予告することなしに変更する場合がございます。	
\\	料金[りょうきん]は 予告[よこく]することなしに 変更[へんこう]する 場合[ばあい]がございます。
\\	友達が何の連絡もなしにいきなり訪ねてきた。	
\\	友達[ともだち]が 何[なに]の 連絡[れんらく]もなしにいきなり 訪[たず]ねてきた。
\\	日本語の授業は手続きなしには聴講できません。	
\\	日本語[にほんご]の 授業[じゅぎょう]は 手続[てつづ]きなしには 聴講[ちょうこう]できません。
\\	ソフトウエアの不正コピーには、許諾なしに複製されたソフトウエアや、偽造されたソフトウエアが含まれている。	
\\	ソフトウエアの 不正[ふせい]コピーには、 許諾[きょだく]なしに 複製[ふくせい]されたソフトウエアや、 偽造[ぎぞう]されたソフトウエアが 含[ふく]まれている。
\\	お客様の同意なしに、第三者に個人情報を転送することはありません。	
\\	お 客様[きゃくさま]の 同意[どうい]なしに、 第三者[だいさんしゃ]に 個人[こじん] 情報[じょうほう]を 転送[てんそう]することはありません。
\\	男たちは何の許可もなしに事務所のコンピュータを勝手に使っていた。	
\\	男[おとこ]たちは 何[なに]の 許可[きょか]もなしに 事務所[じむしょ]のコンピュータを 勝手[かって]に 使[つか]っていた。
\\	キリスト教を理解することなしにヨーロッパの文化や社会を理解することは難しいと言われています。	
\\	キリスト 教[きょう]を 理解[りかい]することなしにヨーロッパの 文化[ぶんか]や 社会[しゃかい]を 理解[りかい]することは 難[むずか]しいと 言[い]われています。
\\	この実験の最初の10年は何の成果も得ることなしに過ぎてしまった。	
\\	この 実験[じっけん]の 最初[さいしょ]の 
\\	年[ねん]は 何[なに]の 成果[せいか]も 得[え]ることなしに 過[す]ぎてしまった。
\\	コンピュータと通信技術が飛躍的に発展するにつれ、社会生活が大きく変化し、いまやコンピュータなしには我々の生活が成立しなくなりつつある。	
\\	コンピュータと 通信[つうしん] 技術[ぎじゅつ]が 飛躍[ひやく] 的[てき]に 発展[はってん]するにつれ、 社会[しゃかい] 生活[せいかつ]が 大[おお]きく 変化[へんか]し、いまやコンピュータなしには 我々[われわれ]の 生活[せいかつ]が 成立[せいりつ]しなくなりつつある。
\\	天気予報もテレビ中継も国際電話も、人工衛星の助けなしにはやっていけない時代になっている。	
\\	天気[てんき] 予報[よほう]もテレビ 中継[ちゅうけい]も 国際[こくさい] 電話[でんわ]も、 人工[じんこう] 衛星[えいせい]の 助[たす]けなしにはやっていけない 時代[じだい]になっている。
\\	この掲示板では、営業目的の書き込みや著しく長い書き込みについては、断りなしに削除させていただくことがあります。	
\\	この 掲示板[けいじばん]では、 営業[えいぎょう] 目的[もくてき]の 書き込[かきこ]みや 著[いちじる]しく 長[なが]い 書き込[かきこ]みについては、 断[ことわ]りなしに 削除[さくじょ]させていただくことがあります。
\\	大事なことを何の話し合いもなしに決められて、私としては納得ができない。	
\\	大事[だいじ]なことを 何[なに]の 話し合[はなしあ]いもなしに 決[き]められて、 私[わたし]としては 納得[なっとく]ができない。
\\	一切の政治権力なしに人々の自由を実現することは、はたして可能なのだろうか。	
\\	一切[いっさい]の 政治[せいじ] 権力[けんりょく]なしに 人々[ひとびと]の 自由[じゆう]を 実現[じつげん]することは、はたして 可能[かのう]なのだろうか。
\\	人一倍照れ屋だった父は、酒なしには人と話しもできなかった。	
\\	人一倍[ひといちばい] 照[て]れ 屋[や]だった 父[ちち]は、 酒[さけ]なしには 人[ひと]と 話[はな]しもできなかった。
\\	災いを転じて福となす。	
\\	災[わざわ]いを 転[てん]じて 福[ふく]となす。
\\	比喩に関しては現在多くの研究がなされている。	
\\	比喩[ひゆ]に 関[かん]しては 現在[げんざい] 多[おお]くの 研究[けんきゅう]がなされている。
\\	人間というものは、悪をなす存在である。	
\\	人間[にんげん]というものは、 悪[あく]をなす 存在[そんざい]である。
\\	民主主義はアメリカ合衆国の基盤をなす思想である。	
\\	民主[みんしゅ] 主義[しゅぎ]は アメリカ合衆国[あめりかがっしゅうこく]の 基盤[きばん]をなす 思想[しそう]である。
\\	顧客創造のためには自社製品の潜在利益を利益となす努力を払わなければならない。	
\\	顧客[こきゃく] 創造[そうぞう]のためには 自社[じしゃ] 製品[せいひん]の 潜在[せんざい] 利益[りえき]を 利益[りえき]となす 努力[どりょく]を 払[はら]わなければならない。
\\	患者が入院時に病名や病状について理解していない場合は、必要に応じて適切な説明がなされるべきである。	
\\	患者[かんじゃ]が 入院[にゅういん] 時[とき]に 病名[びょうめい]や 病状[びょうじょう]について 理解[りかい]していない 場合[ばあい]は、 必要[ひつよう]に 応[おう]じて 適切[てきせつ]な 説明[せつめい]がなされるべきである。
\\	インターネットに関してこのような質問がなされる可能性は高い。	
\\	インターネットに 関[かん]してこのような 質問[しつもん]がなされる 可能[かのう] 性[せい]は 高[たか]い。
\\	善をなすのを急げ。悪から心を退けよ。	
\\	善[ぜん]をなすのを 急[いそ]げ。 悪[あく]から 心[こころ]を 退[しりぞ]けよ。
\\	権利の保護には最善の努力をなすことを目標にしている。	
\\	権利[けんり]の 保護[ほご]には 最善[さいぜん]の 努力[どりょく]をなすことを 目標[もくひょう]にしている。
\\	コンピュータサイエンスの基本をなす考え方は、様々な科学の分野で重要になってくると思われる。	
\\	コンピュータサイエンスの 基本[きほん]をなす 考え方[かんがえかた]は、 様々[さまざま]な 科学[かがく]の 分野[ぶんや]で 重要[じゅうよう]になってくると 思[おも]われる。
\\	情報リスクマネジメントという視点で、ビジネスの根幹をなす情報システムを評価・管理することが必須だ。	
\\	情報[じょうほう]リスクマネジメントという 視点[してん]で、ビジネスの 根幹[こんかん]をなす 情報[じょうほう]システムを 評価[ひょうか]・ 管理[かんり]することが 必須[ひっす]だ。
\\	多くの学者が比喩の研究をしている。	
\\	多[おお]くの 学者[がくしゃ]が 比喩[ひゆ]の 研究[けんきゅう]をしている。
\\	ここを根拠地となす。	
\\	ここを 根拠地[こんきょち]となす。
\\	環境保護の活動が盛んになされている。	
\\	環境[かんきょう] 保護[ほご]の 活動[かつどう]が 盛[さか]んになされている。
\\	その問題に関して様々な議論が行われている。	
\\	その 問題[もんだい]に 関[かん]して 様々[さまざま]な 議論[ぎろん]が 行[おこな]われている。
\\	この町では最近高層ビルの建設が盛んに行われている。	
\\	この 町[まち]では 最近[さいきん] 高層[こうそう]ビルの 建設[けんせつ]が 盛[さか]んに 行[おこな]われている。
\\	このところなぜか体の調子が悪い。	
\\	このところなぜか 体[からだ]の 調子[ちょうし]が 悪[わる]い。
\\	ヨーロッパやアジアには受け入れられなかった
\\	が、なぜか日本には根づいた。	
\\	ヨーロッパやアジアには 受け入[うけい]れられなかった 
\\	が、なぜか 日本[にほん]には 根[ね]づいた。
\\	なぜか僕は日本語の難しさにはまってしまった。	
\\	なぜか 僕[ぼく]は 日本語[にほんご]の 難[むずか]しさにはまってしまった。
\\	最近私が住んでいる町ではなぜか古本屋が増えている。	
\\	最近[さいきん] 私[わたし]が 住[す]んでいる 町[まち]ではなぜか 古本屋[ふるほんや]が 増[ふ]えている。
\\	ログインの手順を教えて下さい。なぜかできなくなってしまったんです。	
\\	ログインの 手順[てじゅん]を 教[おし]えて 下[くだ]さい。なぜかできなくなってしまったんです。
\\	自分の文化のリズムには、なぜか体が自然と反応し、踊りたくなる。	
\\	自分[じぶん]の 文化[ぶんか]のリズムには、なぜか 体[からだ]が 自然[しぜん]と 反応[はんのう]し、 踊[おど]りたくなる。
\\	彼はみんなに好かれている。なぜならば、誰にでも親切だからだ。	
\\	彼[かれ]はみんなに 好[す]かれている。なぜならば、 誰[だれ]にでも 親切[しんせつ]だからだ。
\\	私は自分が好きです。なぜならば、今を自由に生きているからです。	
\\	私[わたし]は 自分[じぶん]が 好[す]きです。なぜならば、 今[いま]を 自由[じゆう]に 生[い]きているからです。
\\	ゲームにおいて最も重要な要素は、目的です。なぜならば、ゲームは本能ではなく理性で行われるものだからです。	
\\	ゲームにおいて 最[もっと]も 重要[じゅうよう]な 要素[ようそ]は、 目的[もくてき]です。なぜならば、ゲームは 本能[ほんのう]ではなく 理性[りせい]で 行[おこな]われるものだからです。
\\	外国語学習者は物事を考える時、自分の母語で考える。なぜならば、目標言語では考えがまとめられないからだ。	
\\	外国[がいこく] 語[ご] 学習[がくしゅう] 者[しゃ]は 物事[ものごと]を 考[かんが]える 時[とき]、 自分[じぶん]の 母語[ぼご]で 考[かんが]える。なぜならば、 目標[もくひょう] 言語[げんご]では 考[かんが]えがまとめられないからだ。
\\	変化を遂げない経営は当然効率が悪くなります。なぜならば顧客も競合他社も常に変化しているからです。	
\\	変化[へんか]を 遂[と]げない 経営[けいえい]は 当然[とうぜん] 効率[こうりつ]が 悪[わる]くなります。なぜならば 顧客[こきゃく]も 競合[きょうごう] 他社[たしゃ]も 常[つね]に 変化[へんか]しているからです。
\\	近現代の日本が本当に正しい道を歩んできたのかどうかは吟味に値する。なぜならば、その過程で日本が失ったものも多いからである。	
\\	近[きん] 現代[げんだい]の 日本[にほん]が 本当[ほんとう]に 正[ただ]しい 道[みち]を 歩[あゆ]んできたのかどうかは 吟味[ぎんみ]に 値[あたい]する。なぜならば、その 過程[かてい]で 日本[にほん]が 失[うしな]ったものも 多[おお]いからである。
\\	店内を改装するので来月は休業いたします。	
\\	店内[てんない]を 改装[かいそう]するので 来月[らいげつ]は 休業[きゅうぎょう]いたします。
\\	今日はね、頭ががんがん痛いから、学校は休むよ。	
\\	今日[きょう]はね、 頭[あたま]ががんがん 痛[いた]いから、 学校[がっこう]は 休[やす]むよ。
\\	あたし、彼、とても優しいんで、大好きなの。	
\\	あたし、 彼[かれ]、とても 優[やさ]しいんで、 大好[だいす]きなの。
\\	犬は噛みつかんばかりに僕に向かってほえた。	
\\	犬[いぬ]は 噛[か]みつかんばかりに 僕[ぼく]に 向[む]かってほえた。
\\	どの演奏も若さと情熱があふれんばかりに満ちていた。	
\\	どの 演奏[えんそう]も 若[わか]さと 情熱[じょうねつ]があふれんばかりに 満[み]ちていた。
\\	時折、木の枝も折れんばかりの強風が吹いていた。	
\\	時折[ときおり]、 木[き]の 枝[えだ]も 折[お]れんばかりの 強風[きょうふう]が 吹[ふ]いていた。
\\	驚いて腰を抜かさんばかりだった。	
\\	驚[おどろ]いて 腰[こし]を 抜[ぬ]かさんばかりだった。
\\	その選手がシュートを決めた時、夜空に鳴り響く雷鳴をかき消さんばかりに、大歓声が競技場にとどろいた。	
\\	その 選手[せんしゅ]がシュートを 決[き]めた 時[とき]、 夜空[よぞら]に 鳴り響[なりひび]く 雷鳴[らいめい]をかき 消[け]さんばかりに、 大[だい] 歓声[かんせい]が 競技[きょうぎ] 場[じょう]にとどろいた。
\\	樹高10メートルの梢から地面に届かんばかりに垂れ下がっている様子は、まさに流れ落ちる桜花の滝にふさわしい。	
\\	樹[じゅ] 高[こう] 10メートルの 梢[こずえ]から 地面[じめん]に 届[とど]かんばかりに 垂れ下[たれさ]がっている 様子[ようす]は、まさに 流れ落[ながれお]ちる 桜花[おうか]の 滝[たき]にふさわしい。
\\	「風の谷のナウシカ」を娘さんに薦められて見たお父さんは、涙を流さんばかりに感激した。	
\\	風[かぜ]の 谷[たに]のナウシカ」を 娘[むすめ]さんに 薦[すす]められて 見[み]たお 父[とう]さんは、 涙[なみだ]を 流[なが]さんばかりに 感激[かんげき]した。
\\	演奏が終わるか終わらないかのうちに、この曲を知っているのは自分だと言わんばかりに拍手をする人もいます。	
\\	演奏[えんそう]が 終[お]わるか 終[お]わらないかのうちに、この 曲[きょく]を 知[し]っているのは 自分[じぶん]だと 言[い]わんばかりに 拍手[はくしゅ]をする 人[ひと]もいます。
\\	彼に別れたいと言われた時には、私は胸が張り裂けんばかりに苦しかった。	
\\	彼[かれ]に 別[わか]れたいと 言[い]われた 時[とき]には、 私[わたし]は 胸[むね]が 張り裂[はりさ]けんばかりに 苦[くる]しかった。
\\	彼の死を聞いた時、彼女は気も狂わぬばかりに泣き叫んだ。	
\\	彼[かれ]の 死[し]を 聞[き]いた 時[とき]、 彼女[かのじょ]は 気[き]も 狂[くる]わぬばかりに 泣き叫[なきさけ]んだ。
\\	上司から叱りつけられた山崎は土下座せぬばかりに平身低頭して謝った。	
\\	上司[じょうし]から 叱[しか]りつけられた 山崎[やまざき]は 土下座[どげざ]せぬばかりに 平身低頭[へいしんていとう]して 謝[あやま]った。
\\	彼女はいつもこぼれんばかりの笑みをたたえている。	
\\	彼女[かのじょ]はいつもこぼれんばかりの 笑[え]みをたたえている。
\\	引っ越しの時あまり重いものを持ったので、腕が肩から抜けんばかりだった。	
\\	引っ越[ひっこ]しの 時[とき]あまり 重[おも]いものを 持[も]ったので、 腕[うで]が 肩[かた]から 抜[ぬ]けんばかりだった。
\\	初めに、これだけは必ずお読み願います。	
\\	初[はじ]めに、これだけは 必[かなら]ずお 読[よ]み 願[ねが]います。
\\	禁煙にご協力願います。	
\\	禁煙[きんえん]にご 協力[きょうりょく] 願[ねが]います。
\\	ご相談・ご意見・ご質問などがありましたら、下記へお知らせ願います。	
\\	ご 相談[そうだん]・ご 意見[いけん]・ご 質問[しつもん]などがありましたら、 下記[かき]へお 知[し]らせ 願[ねが]います。
\\	当オンラインショップで商品をご注文いただく場合は、まず最初にカタログページからご希望の商品をお選び願います。	
\\	当[とう]オンラインショップで 商品[しょうひん]をご 注文[ちゅうもん]いただく 場合[ばあい]は、まず 最初[さいしょ]にカタログページからご 希望[きぼう]の 商品[しょうひん]をお 選[えら]び 願[ねが]います。
\\	この喫煙所は、入院しておられる患者さんの施設です。外来の方々は利用できませんのでご了承願います。	
\\	この 喫煙[きつえん] 所[しょ]は、 入院[にゅういん]しておられる 患者[かんじゃ]さんの 施設[しせつ]です。 外来[がいらい]の 方々[かたがた]は 利用[りよう]できませんのでご 了承[りょうしょう] 願[ねが]います。
\\	応募に際しては、以下の書類を
\\	メールまたは郵便にてご送付願います。	
\\	応募[おうぼ]に 際[さい]しては、 以下[いか]の 書類[しょるい]を 
\\	メールまたは 郵便[ゆうびん]にてご 送付[そうふ] 願[ねが]います。
\\	電車とホームの間が広く空いています。足元にご注意願います。	
\\	電車[でんしゃ]とホームの 間[あいだ]が 広[ひろ]く 空[あ]いています。 足元[あしもと]にご 注意[ちゅうい] 願[ねが]います。
\\	学術研究を目的とする方意外の当研究所図書館の利用はご遠慮願います。	
\\	学術[がくじゅつ] 研究[けんきゅう]を 目的[もくてき]とする 方[ほう] 意外[いがい]の 当[とう] 研究所[けんきゅうじょ] 図書館[としょかん]の 利用[りよう]はご 遠慮[えんりょ] 願[ねが]います。
\\	ご協力願うこともあろうかと思いますので、その節にはどうぞよろしく。	
\\	ご 協力[きょうりょく] 願[ねが]うこともあろうかと 思[おも]いますので、その 節[せつ]にはどうぞよろしく。
\\	お支払い願う時はあらかじめお知らせいたします。	
\\	お 支払[しはら]い 願[ねが]う 時[とき]はあらかじめお 知[し]らせいたします。
\\	学会の挨拶のため学長にご出席願った。	
\\	学会[がっかい]の 挨拶[あいさつ]のため 学長[がくちょう]にご 出席[しゅっせき] 願[ねが]った。
\\	経済界の大御所にご助力を願った。	
\\	経済[けいざい] 界[かい]の 大御所[おおごしょ]にご 助力[じょりょく]を 願[ねが]った。
\\	この書類にサインをなさってください。	
\\	この 書類[しょるい]にサインをなさってください。
\\	この手紙をご覧下さい。	
\\	この 手紙[てがみ]をご 覧[らん] 下[くだ]さい。
\\	またお出で下さい。	
\\	またお 出[い]で 下[くだ]さい。
\\	几帳面な父にひきかえ、息子はだらしなかった。	
\\	几帳面[きちょうめん]な 父[ちち]にひきかえ、 息子[むすこ]はだらしなかった。
\\	この大学は教育の質がとてもいいのにひきかえ、学内の設備が劣る。	
\\	この 大学[だいがく]は 教育[きょういく]の 質[しつ]がとてもいいのにひきかえ、 学内[がくない]の 設備[せつび]が 劣[おと]る。
\\	昨日にひきかえ今朝は快晴です。	
\\	昨日[きのう]にひきかえ 今朝[けさ]は 快晴[かいせい]です。
\\	姉は何事にも堅実なタイプなのにひきかえ、妹は奔放なタイプだ。	
\\	姉[あね]は 何事[なにごと]にも 堅実[けんじつ]なタイプなのにひきかえ、 妹[いもうと]は 奔放[ほんぽう]なタイプだ。
\\	昨年楽しい年だったのにひきかえ、今年はいろいろな意味で試練の年だった。	
\\	昨年[さくねん] 楽[たの]しい 年[とし]だったのにひきかえ、 今年[ことし]はいろいろな 意味[いみ]で 試練[しれん]の 年[とし]だった。
\\	16世紀のフランスでは、貴族の食事は贅沢だったのにひきかえ、市民の食事はとても貧しいものだった。	
\\	世紀[せいき]のフランスでは、 貴族[きぞく]の 食事[しょくじ]は 贅沢[ぜいたく]だったのにひきかえ、 市民[しみん]の 食事[しょくじ]はとても 貧[まず]しいものだった。
\\	我々は
\\	という国は、こんなにすばらしいことをしている。それにひきかえ日本は全く駄目だ」という話をずっと聞かされ続けたものだ。	
\\	我々[われわれ]は
\\	という 国[くに]は、こんなにすばらしいことをしている。それにひきかえ 日本[にほん]は 全[まった]く 駄目[だめ]だ」という 話[はなし]をずっと 聞[き]かされ 続[つづ]けたものだ。
\\	ハーバードや
\\	の有名は日本で広く知られているのにひきかえ、残念ながら私が教えているプリンストンの有名はほとんど知られていない。	
\\	ハーバードや 
\\	の 有名[ゆうめい]は 日本[にほん]で 広[ひろ]く 知[し]られているのにひきかえ、 残念[ざんねん]ながら 私[わたし]が 教[おし]えているプリンストンの 有名[ゆうめい]はほとんど 知[し]られていない。
\\	については、経済問題はよく議論されてきたのにひきかえ、文化面の議論があまりなされてこなかった。	
\\	については、 経済[けいざい] 問題[もんだい]はよく 議論[ぎろん]されてきたのにひきかえ、 文化[ぶんか] 面[めん]の 議論[ぎろん]があまりなされてこなかった。
\\	このアパートは駅にとても近くて便利なのにひきかえ、部屋は狭く、家賃も高い。	
\\	このアパートは 駅[えき]にとても 近[ちか]くて 便利[べんり]なのにひきかえ、 部屋[へや]は 狭[せま]く、 家賃[やちん]も 高[たか]い。
\\	日本の大学は、入るのが難しいのにひきかえ、出るのは簡単だと言われている。	
\\	日本[にほん]の 大学[だいがく]は、 入[はい]るのが 難[むずか]しいのにひきかえ、 出[で]るのは 簡単[かんたん]だと 言[い]われている。
\\	人間の体は60
\\	が水、魚は75
\\	、くらげにいたっては96
\\	が水だ。	
\\	人間[にんげん]の 体[からだ]は 
\\	が 水[みず]、 魚[さかな]は 
\\	、くらげにいたっては 
\\	が 水[みず]だ。
\\	私自身は社会人になってからコンピュータを使い始めたが、新入社員は中学生の頃から、私の子供たちにいたっては、生まれたときから家にコンピュータがある世代だ。	
\\	私[わたし] 自身[じしん]は 社会[しゃかい] 人[じん]になってからコンピュータを 使[つか]い 始[はじ]めたが、 新入[しんにゅう] 社員[しゃいん]は 中学生[ちゅうがくせい]の 頃[ころ]から、 私[わたし]の 子供[こども]たちにいたっては、 生[う]まれたときから 家[いえ]にコンピュータがある 世代[せだい]だ。
\\	一年間に日本中の廃棄物は5,160万トン、しかもこれは家庭や飲食店などから出る一般廃棄物で、産業廃棄物にいたってはその8倍近くの4億8百万トンも出ているのです。	
\\	一 年間[ねんかん]に 日本[にほん] 中[ちゅう]の 廃棄[はいき] 物[ぶつ]は 
\\	万[まん]トン、しかもこれは 家庭[かてい]や 飲食[いんしょく] 店[てん]などから 出[で]る 一般[いっぱん] 廃棄[はいき] 物[ぶつ]で、 産業[さんぎょう] 廃棄[はいき] 物[ぶつ]にいたってはその 
\\	倍[ばい] 近[ちか]くの 
\\	億[おく] 
\\	百[ひゃく] 万[まん]トンも 出[で]ているのです。
\\	少林寺拳法では女性挙士の数が増加傾向にあり、高校や大学の部活動にいたっては、部員の半数近くが女性だ。	
\\	少林寺拳法[しょうりんじけんぽう]では 女性[じょせい] 挙士[けんし]の 数[かず]が 増加[ぞうか] 傾向[けいこう]にあり、 高校[こうこう]や 大学[だいがく]の 部[ぶ] 活動[かつどう]にいたっては、 部員[ぶいん]の 半数[はんすう] 近[ちか]くが 女性[じょせい]だ。
\\	今私たちの生活には多くの椅子が使われている。ダイニングにいたっては、ダイニングテーブルと椅子は、もはや一般的である。	
\\	今[いま] 私[わたし]たちの 生活[せいかつ]には 多[おお]くの 椅子[いす]が 使[つか]われている。ダイニングにいたっては、ダイニングテーブルと 椅子[いす]は、もはや 一般[いっぱん] 的[てき]である。
\\	日本には、一般紙やスポーツ紙、専門的な業界紙など様々な新聞があるが雑誌にいたっては、どんな分野のものでもある。	
\\	日本[にほん]には、 一般[いっぱん] 紙[し]やスポーツ 紙[し]、 専門[せんもん] 的[てき]な 業界[ぎょうかい] 紙[し]など 様々[さまざま]な 新聞[しんぶん]があるが 雑誌[ざっし]にいたっては、どんな 分野[ぶんや]のものでもある。
\\	アジア太平洋地域以外にも、旅行者の注目を集めている地域がある。中央アメリカは昨年、旅行者数が23
\\	増、中東も17.5
\\	増となった。スペインも9
\\	増加したし、モロッコにいたっては、22
\\	も増えた。	
\\	アジア 太平洋[たいへいよう] 地域[ちいき] 以外[いがい]にも、 旅行[りょこう] 者[しゃ]の 注目[ちゅうもく]を 集[あつ]めている 地域[ちいき]がある。 中央[ちゅうおう]アメリカは 昨年[さくねん]、 旅行[りょこう] 者[しゃ] 数[すう]が 
\\	増[ぞう]、 中東[ちゅうとう]も 
\\	増[ぞう]となった。スペインも 
\\	増加[ぞうか]したし、モロッコにいたっては、 
\\	も 増[ふ]えた。
\\	私の主人にいたっては仕事だけで、家のことは何もしない。	
\\	私[わたし]の 主人[しゅじん]にいたっては 仕事[しごと]だけで、 家[いえ]のことは 何[なに]もしない。
\\	衣食住の住ときたら、普通のサラリーマンには手が出ない。	
\\	衣食住[いしょくじゅう]の 住[じゅう]ときたら、 普通[ふつう]のサラリーマンには 手[て]が 出[で]ない。
\\	当社は性別に関わらず同様に昇進できます。	
\\	当社[とうしゃ]は 性別[せいべつ]に 関[かか]わらず 同様[どうよう]に 昇進[しょうしん]できます。
\\	出席か欠席かに関わらず同封の葉書でお知らせ下さい。	
\\	出席[しゅっせき]か 欠席[けっせき]かに 関[かか]わらず 同封[どうふう]の 葉書[はがき]でお 知[し]らせ 下[くだ]さい。
\\	実際に使う使わないに関わらずレンタル料を払わなければならない。	
\\	実際[じっさい]に 使[つか]う 使[つか]わないに 関[かか]わらずレンタル 料[りょう]を 払[はら]わなければならない。
\\	商品化が可能かどうかに関わらず我々はこの製品の研究を進める予定だ。	
\\	商品[しょうひん] 化[か]が 可能[かのう]かどうかに 関[かか]わらず 我々[われわれ]はこの 製品[せいひん]の 研究[けんきゅう]を 進[すす]める 予定[よてい]だ。
\\	会員であるなしに関わらず商品カタログをお送りします。	
\\	会員[かいいん]であるなしに 関[かか]わらず 商品[しょうひん]カタログをお 送[おく]りします。
\\	自覚症状のあるなしに関わらず医師に相談することを勧めます。	
\\	自覚[じかく] 症状[しょうじょう]のあるなしに 関[かか]わらず 医師[いし]に 相談[そうだん]することを 勧[すす]めます。
\\	者は市場調査の結果のいかんに関わらず日本でのマーケティングを開始することを決めた。	
\\	者[しゃ]は 市場[しじょう] 調査[ちょうさ]の 結果[けっか]のいかんに 関[かか]わらず 日本[にほん]でのマーケティングを 開始[かいし]することを 決[き]めた。
\\	これはどの言語を学習するかに関わらず役に立つ情報だと思います。	
\\	これはどの 言語[げんご]を 学習[がくしゅう]するかに 関[かか]わらず 役に立[やくにた]つ 情報[じょうほう]だと 思[おも]います。
\\	このオンラインゲームは、性別。年齢に関わらず誰でも楽しむことができます。	
\\	このオンラインゲームは、 性別[せいべつ]。 年齢[ねんれい]に 関[かか]わらず 誰[だれ]でも 楽[たの]しむことができます。
\\	日本で外国人が日本語を話すと、実際に上手か下手かに関わらず、「日本語が上手ですね」とほめられます。	
\\	日本[にほん]で 外国[がいこく] 人[じん]が 日本語[にほんご]を 話[はな]すと、 実際[じっさい]に 上手[じょうず]か 下手[へた]かに 関[かか]わらず、
\\	日本語[にほんご]が 上手[じょうず]ですね」とほめられます。
\\	実際にオーストラリアに行く行かないに関わらず休暇は取ろうと思っている。	
\\	実際[じっさい]にオーストラリアに 行[い]く 行[い]かないに 関[かか]わらず 休暇[きゅうか]は 取[と]ろうと 思[おも]っている。
\\	我々は、助成金が下りるかどうかに関わらず、このプロジェクトを続ける。	
\\	我々[われわれ]は、 助成[じょせい] 金[きん]が 下[お]りるかどうかに 関[かか]わらず、このプロジェクトを 続[つづ]ける。
\\	大学の図書館は、その大学の学生・教職員であるなしに関わらず閲覧できるのが普通だ。	
\\	大学[だいがく]の 図書館[としょかん]は、その 大学[だいがく]の 学生[がくせい]・ 教職員[きょうしょくいん]であるなしに 関[かか]わらず 閲覧[えつらん]できるのが 普通[ふつう]だ。
\\	条例のあるなしに関わらず、公共のスペースでの喫煙は明らかなマナー違反であることを認識してほしい。	
\\	条例[じょうれい]のあるなしに 関[かか]わらず、 公共[こうきょう]のスペースでの 喫煙[きつえん]は 明[あき]らかなマナー 違反[いはん]であることを 認識[にんしき]してほしい。
\\	いかなる団体・個人についても、その意図や理由のいかんに関わらず、ここにある画像の転載・再配布等は許可しません。	
\\	いかなる 団体[だんたい]・ 個人[こじん]についても、その 意図[いと]や 理由[りゆう]のいかんに 関[かか]わらず、ここにある 画像[がぞう]の 転載[てんさい]・ 再[さい] 配布[はいふ] 等[とう]は 許可[きょか]しません。
\\	この製品の著作者及び、製造、配布に関わるいかなる者も、本ソフトウエアの使用、または使用不能によって発生する損害に対する責任は、それが直接的であるか間接的であるか、必然的であるか偶発的であるかに関わらず、負わないものとします。	
\\	この 製品[せいひん]の 著作[ちょさく] 者[しゃ] 及[およ]び、 製造[せいぞう]、 配布[はいふ]に 関[かか]わるいかなる 者[もの]も、 本[ほん]ソフトウエアの 使用[しよう]、または 使用[しよう] 不能[ふのう]によって 発生[はっせい]する 損害[そんがい]に 対[たい]する 責任[せきにん]は、それが 直接的[ちょくせつてき]であるか 間接[かんせつ] 的[てき]であるか、 必然[ひつぜん] 的[てき]であるか 偶発[ぐうはつ] 的[てき]であるかに 関[かか]わらず、 負[お]わないものとします。
\\	経済学のノーベル賞を受賞したジョン・ナシュは数学にかけては自分の右に出る者はいないおと思っていた。	
\\	経済[けいざい] 学[がく]のノーベル 賞[しょう]を 受賞[じゅしょう]したジョン・ナシュは 数学[すうがく]にかけては 自分[じぶん]の 右[みぎ]に 出[で]る 者[もの]はいないおと 思[おも]っていた。
\\	ジャッキー・チェンは、アクションにかけては右に出る者がいない俳優である。	
\\	ジャッキー・チェンは、アクションにかけては 右[みぎ]に 出[で]る 者[もの]がいない 俳優[はいゆう]である。
\\	山崎氏はこの分野の政策立案にかけては政界随一の見識と経験を持っている。	
\\	山崎[やまざき] 氏[し]はこの 分野[ぶんや]の 政策[せいさく] 立案[りつあん]にかけては 政界[せいかい] 随一[ずいいち]の 見識[けんしき]と 経験[けいけん]を 持[も]っている。
\\	ベートーベンは耳の障害にも関わらず、最後まですばらしい作品を書き続けた。	
\\	ベートーベンは 耳[みみ]の 障害[しょうがい]にも 関[かか]わらず、 最後[さいご]まですばらしい 作品[さくひん]を 書[か]き 続[つづ]けた。
\\	写真というものは絵画に比べて、その客観的再現力にかけては圧倒的にまさっている。	
\\	写真[しゃしん]というものは 絵画[かいが]に 比[くら]べて、その 客観[きゃっかん] 的[てき] 再現[さいげん] 力[りょく]にかけては 圧倒的[あっとうてき]にまさっている。
\\	このユースホステルは年齢を問わず誰でも利用でき、快適さにかけては世界中でも最高クラスだ。	
\\	このユースホステルは 年齢[ねんれい]を 問[と]わず 誰[だれ]でも 利用[りよう]でき、 快適[かいてき]さにかけては 世界中[せかいじゅう]でも 最高[さいこう]クラスだ。
\\	この工場は特殊鋼の製造にかけては世界屈指の技術を有している。	
\\	この 工場[こうじょう]は 特殊[とくしゅ] 鋼[こう]の 製造[せいぞう]にかけては 世界[せかい] 屈指[くっし]の 技術[ぎじゅつ]を 有[ゆう]している。
\\	彼はその問題については目下検討中だ。	
\\	彼[かれ]はその 問題[もんだい]については 目下[もっか] 検討[けんとう] 中[ちゅう]だ。
\\	地球温暖化に関してはかなりの研究が進んでいる。	
\\	地球[ちきゅう] 温暖[おんだん] 化[か]に 関[かん]してはかなりの 研究[けんきゅう]が 進[すす]んでいる。
\\	この問題では異論があった。	
\\	この 問題[もんだい]では 異論[いろん]があった。
\\	コンピュータサイエンスの領域ではさして進展はなかった。	
\\	コンピュータサイエンスの 領域[りょういき]ではさして 進展[しんてん]はなかった。
\\	給料が低いために教師の質が低下したであろうことは想像にかたくない。	
\\	給料[きゅうりょう]が 低[ひく]いために 教師[きょうし]の 質[しつ]が 低下[ていか]したであろうことは 想像[そうぞう]にかたくない。
\\	幼い時に両親を失った彼女の生活がいかに悲惨だったかは推測にかたくない。	
\\	幼[おさな]い 時[とき]に 両親[りょうしん]を 失[うしな]った 彼女[かのじょ]の 生活[せいかつ]がいかに 悲惨[ひさん]だったかは 推測[すいそく]にかたくない。
\\	この新宅には相当お金をかけたであろうことは想像にかたくない。	
\\	この 新宅[しんたく]には 相当[そうとう]お 金[かね]をかけたであろうことは 想像[そうぞう]にかたくない。
\\	日本語教育のめざましい進歩の裏には、日本語教育者・研究者の並々ならぬ努力はあったことは、想像にかたくありません。	
\\	日本語[にほんご] 教育[きょういく]のめざましい 進歩[しんぽ]の 裏[うら]には、 日本語[にほんご] 教育[きょういく] 者[しゃ]・ 研究[けんきゅう] 者[しゃ]の 並々[なみなみ]ならぬ 努力[どりょく]はあったことは、 想像[そうぞう]にかたくありません。
\\	留学生にとって日本の伝統文化を見つけることがいかに難しくなっているかは想像にかたくないことでしょう。	
\\	留学生[りゅうがくせい]にとって 日本[にほん]の 伝統[でんとう] 文化[ぶんか]を 見[み]つけることがいかに 難[むずか]しくなっているかは 想像[そうぞう]にかたくないことでしょう。
\\	好きな文芸の道と経営者としての責任の間で、多才な彼が苦悩していたことは想像にかたくない。	
\\	好[す]きな 文芸[ぶんげい]の 道[みち]と 経営[けいえい] 者[しゃ]としての 責任[せきにん]の 間[あいだ]で、 多才[たさい]な 彼[かれ]が 苦悩[くのう]していたことは 想像[そうぞう]にかたくない。
\\	その青年が非常に心優しく生真面目かつ勉学熱心な人間だったことは推測にかたくない。	
\\	その 青年[せいねん]が 非常[ひじょう]に 心[こころ] 優[やさ]しく 生真面目[きまじめ]かつ 勉学[べんがく] 熱心[ねっしん]な 人間[にんげん]だったことは 推測[すいそく]にかたくない。
\\	海外に在住して母国の軍事政権への反対活動を行っている人が帰国すれば、拘禁や拷問にさらされるであろうことは予測にかたくない。	
\\	海外[かいがい]に 在住[ざいじゅう]して 母国[ぼこく]の 軍事[ぐんじ] 政権[せいけん]への 反対[はんたい] 活動[かつどう]を 行[い]っている 人[ひと]が 帰国[きこく]すれば、 拘禁[こうきん]や 拷問[ごうもん]にさらされるであろうことは 予測[よそく]にかたくない。
\\	性格は、いい方がいいに決まっている。	
\\	性格[せいかく]は、いい 方[ほう]がいいに 決[き]まっている。
\\	お客さまだって安く買いたいに決まっているじゃないか。	
\\	お 客[きゃく]さまだって 安[やす]く 買[か]いたいに 決[き]まっているじゃないか。
\\	ただただ漢字を書くなんて、つまらないに決まっている。	
\\	ただただ 漢字[かんじ]を 書[か]くなんて、つまらないに 決[き]まっている。
\\	すぐ角のたばこ屋に行く程度の用事なら、車より自転車の方が便利に決まっている。	
\\	すぐ 角[かど]のたばこ 屋[や]に 行[い]く 程度[ていど]の 用事[ようじ]なら、 車[くるま]より 自転車[じてんしゃ]の 方[ほう]が 便利[べんり]に 決[き]まっている。
\\	立派な
\\	染色体と小さいな
\\	染色体をつくづく見ていると、女性の方が丈夫に決まっていると思う。	
\\	立派[りっぱ]な 
\\	染色[せんしょく] 体[たい]と 小[ちい]さいな 
\\	染色[せんしょく] 体[たい]をつくづく 見[み]ていると、 女性[じょせい]の 方[ほう]が 丈夫[じょうぶ]に 決[き]まっていると 思[おも]う。
\\	同じ商品なら気持ちのいい店で買うに決まっている。	
\\	同[おな]じ 商品[しょうひん]なら 気持[きも]ちのいい 店[みせ]で 買[か]うに 決[き]まっている。
\\	子供は、大人が作っている社会の中で育つのだから、子供が変なのは大人のせいに決まっている。	
\\	子供[こども]は、 大人[おとな]が 作[つく]っている 社会[しゃかい]の 中[なか]で 育[そだ]つのだから、 子供[こども]が 変[へん]なのは 大人[おとな]のせいに 決[き]まっている。
\\	何しに来たって?飲みに来たに決まっているじゃないか。	
\\	何[なに]しに 来[き]たって? 飲[の]みに 来[き]たに 決[き]まっているじゃないか。
\\	アルコールに強い・弱いは、遺伝的に決まっている。	
\\	アルコールに 強[つよ]い・ 弱[よわ]いは、 遺伝[いでん] 的[てき]に 決[き]まっている。
\\	どの目が出るかは、サイコロを投げた瞬間に決まっているはずだ。	
\\	どの 目[め]が 出[で]るかは、サイコロを 投[な]げた 瞬間[しゅんかん]に 決[き]まっているはずだ。
\\	仲間は多いに越したことはない。	
\\	仲間[なかま]は 多[おお]いに 越[こ]したことはない。
\\	元気であるに越したことはない。	
\\	元気[げんき]であるに 越[こ]したことはない。
\\	虫歯は予防するに越したことはない。	
\\	虫歯[むしば]は 予防[よぼう]するに 越[こ]したことはない。
\\	温度は一定であるに越したことはない。	
\\	温度[おんど]は 一定[いってい]であるに 越[こ]したことはない。
\\	環境はよいに越したことはない。	
\\	環境[かんきょう]はよいに 越[こ]したことはない。
\\	働く時間は短いに越したことはないんじゃないの?	
\\	働[はたら]く 時間[じかん]は 短[みじか]いに 越[こ]したことはないんじゃないの?
\\	部屋は清潔に越したことはない。	
\\	部屋[へや]は 清潔[せいけつ]に 越[こ]したことはない。
\\	うちの息子はわんぱくで手に余る。まあ、元気に越したことはないけれど。	
\\	うちの 息子[むすこ]はわんぱくで 手[て]に 余[あま]る。まあ、 元気[げんき]に 越[こ]したことはないけれど。
\\	今週の土曜日から来週水曜日までの5日間、僕も夏休みに入ります。特別な予定もないけど仕事より休みに越したことはありません。	
\\	今週[こんしゅう]の 土曜日[どようび]から 来週[らいしゅう] 水曜日[すいようび]までの 5日間[いつかかん]、 僕[ぼく]も 夏休[なつやす]みに 入[はい]ります。 特別[とくべつ]な 予定[よてい]もないけど 仕事[しごと]より 休[やす]みに 越[こ]したことはありません。
\\	痛みは避けるべきものであり、ないに越したことはない。	
\\	痛[いた]みは 避[さ]けるべきものであり、ないに 越[こ]したことはない。
\\	クリスマスに「私にとって最高の贈り物はあなたです」なんて言葉が言えるに越したことはない。	
\\	クリスマスに
\\	私[わたし]にとって 最高[さいこう]の 贈り物[おくりもの]はあなたです」なんて 言葉[ことば]が 言[い]えるに 越[こ]したことはない。
\\	ネットワークがどれだけ発達しても、本当のコミュニケーションには、人と人とが向かい合って話すのに越したことはない。	
\\	ネットワークがどれだけ 発達[はったつ]しても、 本当[ほんとう]のコミュニケーションには、 人[ひと]と 人[ひと]とが 向かい合[むかいあ]って 話[はな]すのに 越[こ]したことはない。
\\	桜の季節にもなるとこの公園は花見客でにぎわう。	
\\	桜[さくら]の 季節[きせつ]にもなるとこの 公園[こうえん]は 花見[はなみ] 客[きゃく]でにぎわう。
\\	外国へ行くともなると、その国の言語や文化を勉強したくなるはずだ。	
\\	外国[がいこく]へ 行[い]くともなると、その 国[くに]の 言語[げんご]や 文化[ぶんか]を 勉強[べんきょう]したくなるはずだ。
\\	正月ともなると、友人同士、お互いの家を行き来して、正月酒を楽しむ。	
\\	正月[しょうがつ]ともなると、 友人[ゆうじん] 同士[どうし]、お 互[たが]いの 家[いえ]を 行き来[いきき]して、 正月[しょうがつ] 酒[ざけ]を 楽[たの]しむ。
\\	ここは紅葉が始まると訪れる人が増え始め、11月半ばともなると大勢の人でごったがえす。	
\\	ここは 紅葉[こうよう]が 始[はじ]まると 訪[おとず]れる 人[ひと]が 増[ふ]え 始[はじ]め、 
\\	月[がつ] 半[なか]ばともなると 大勢[おおぜい]の 人[ひと]でごったがえす。
\\	社会人にもなると、先輩や友人の結婚式に招かれる機会が出てきます。	
\\	社会[しゃかい] 人[じん]にもなると、 先輩[せんぱい]や 友人[ゆうじん]の 結婚式[けっこんしき]に 招[まね]かれる 機会[きかい]が 出[で]てきます。
\\	高校生にもなると、自分のお小遣いくらいは、バイトして稼ぐようになる。	
\\	高校生[こうこうせい]にもなると、 自分[じぶん]のお 小遣[こづか]いくらいは、バイトして 稼[かせ]ぐようになる。
\\	大きなホテルともなると、一日に数千人もの利用者が集まってくる。	
\\	大[おお]きなホテルともなると、一日[いちにち]に 数[すう] 千[せん] 人[にん]もの 利用[りよう] 者[しゃ]が 集[あつ]まってくる。
\\	「美女と野獣」ともなると人気の作品だから、当日券なんてなかなか買えない。	
\\	美女と野獣[びじょとやじゅう]」ともなると 人気[にんき]の 作品[さくひん]だから、 当日[とうじつ] 券[けん]なんてなかなか 買[か]えない。
\\	大学へ進学するともなると、学問、バイト、サークルなど、人それぞれのライフスタイルができてくる。	
\\	大学[だいがく]へ 進学[しんがく]するともなると、 学問[がくもん]、バイト、サークルなど、 人[ひと]それぞれのライフスタイルができてくる。
\\	今年から修学旅行先が北海道になり、飛行機で行くともなると、胸を躍らさずにはいられなかった。	
\\	今年[ことし]から 修学旅行[しゅうがくりょこう] 先[さき]が 北海道[ほっかいどう]になり、 飛行機[ひこうき]で 行[い]くともなると、 胸[むね]を 躍[おど]らさずにはいられなかった。
\\	夜も更けていた。	
\\	夜[よる]も 更[ふ]けていた。
\\	夏も終わってしまった。	
\\	夏[なつ]も 終[お]わってしまった。
\\	報道班は次の目的地に向けて出発した。	
\\	報道[ほうどう] 班[はん]は 次[つぎ]の 目的[もくてき] 地[ち]に 向[む]けて 出発[しゅっぱつ]した。
\\	このテレビコマーシャルは20,30代の独身女性に向けて発信されている。	
\\	このテレビコマーシャルは 
\\	代[だい]の 独身[どくしん] 女性[じょせい]に 向[む]けて 発信[はっしん]されている。
\\	今、来年の入学試験に向けて頑張っています。	
\\	今[いま]、 来年[らいねん]の 入学[にゅうがく] 試験[しけん]に 向[む]けて 頑張[がんば]っています。
\\	新しいエネルギーの開発に向けて研究が進んでいる。	
\\	新[あたら]しいエネルギーの 開発[かいはつ]に 向[む]けて 研究[けんきゅう]が 進[すす]んでいる。
\\	地球温暖化防止に向けての動きが活発化している。	
\\	地球[ちきゅう] 温暖[おんだん] 化[か] 防止[ぼうし]に 向[む]けての 動[うご]きが 活発[かっぱつ] 化[か]している。
\\	救援物資を積んだヘリコプターが被災地に向けて飛び立った。	
\\	救援[きゅうえん] 物資[ぶっし]を 積[つ]んだヘリコプターが 被災[ひさい] 地[ち]に 向[む]けて 飛び立[とびた]った。
\\	機動隊はデモ隊に向けて催涙弾を発射した。	
\\	機動[きどう] 隊[たい]はデモ 隊[たい]に 向[む]けて 催涙[さいるい] 弾[だん]を 発射[はっしゃ]した。
\\	者は1月から九州で生産した高級車を中国に向けて輸出する。	
\\	者[しゃ]は 
\\	月[がつ]から 九州[きゅうしゅう]で 生産[せいさん]した 高級[こうきゅう] 車[しゃ]を 中国[ちゅうごく]に 向[む]けて 輸出[ゆしゅつ]する。
\\	このプロジェクトでは、世界各国のアニメーターが世界中の子供たちに向けて作品を作り、それはそれぞれの国の言葉に翻訳します。	
\\	このプロジェクトでは、 世界[せかい] 各国[かっこく]のアニメーターが 世界中[せかいじゅう]の 子供[こども]たちに 向[む]けて 作品[さくひん]を 作[つく]り、それはそれぞれの 国[くに]の 言葉[ことば]に 翻訳[ほんやく]します。
\\	バレーボール協会は次のオリンピックに向けて選手の強化を急いでいる。	
\\	バレーボール 協会[きょうかい]は 次[つぎ]のオリンピックに 向[む]けて 選手[せんしゅ]の 強化[きょうか]を 急[いそ]いでいる。
\\	異なる特許制度の世界的統一に向けて交渉が開始された。	
\\	異[こと]なる 特許[とっきょ] 制度[せいど]の 世界[せかい] 的[てき] 統一[とういつ]に 向[む]けて 交渉[こうしょう]が 開始[かいし]された。
\\	者は新しく開発された電池の事業化に向けて新会社を設立した。	
\\	者[しゃ]は 新[あたら]しく 開発[かいはつ]された 電池[でんち]の 事業[じぎょう] 化[か]に 向[む]けて 新[しん] 会社[かいしゃ]を 設立[せつりつ]した。
\\	我が社は効率的な社内コミュニケーションに向けての新しいコミュニケーションシステムの導入を進めている。	
\\	我[わ]が 社[しゃ]は 効率[こうりつ] 的[てき]な 社内[しゃない]コミュニケーションに 向[む]けての 新[あたら]しいコミュニケーションシステムの 導入[どうにゅう]を 進[すす]めている。
\\	氏の提言は、新たな世紀に向けた日本外交の指針を考えていくにあたり、非常に示唆に富むものであった。	
\\	氏[し]の 提言[ていげん]は、 新[あら]たな 世紀[せいき]に 向[む]けた 日本[にほん] 外交[がいこう]の 指針[ししん]を 考[かんが]えていくにあたり、 非常[ひじょう]に 示唆[しさ]に 富[と]むものであった。
\\	風が北に向かって吹いている。	
\\	風[かぜ]が 北[きた]に 向[む]かって 吹[ふ]いている。
\\	後任者が決まらないので辞めるに辞められない。	
\\	後任[こうにん] 者[しゃ]が 決[き]まらないので 辞[や]めるに 辞[や]められない。
\\	大事なお客さんからの電話がいつまで待ってもかかってこないので、帰るに帰れない。	
\\	大事[だいじ]なお 客[きゃく]さんからの 電話[でんわ]がいつまで 待[ま]ってもかかってこないので、 帰[かえ]るに 帰[かえ]れない。
\\	みんなの前で大見えを切ってできると言ってしまったので引くに引けない。	
\\	みんなの 前[まえ]で 大見[おおみ]えを 切[き]ってできると 言[い]ってしまったので 引[ひ]くに 引[ひ]けない。
\\	同窓会などで会った旧友が、何かの団体に入っていたり、何かの商品を売っていたりして、勧誘を断るに断れないケースがよくある。	
\\	同窓[どうそう] 会[かい]などで 会[あ]った 旧友[きゅうゆう]が、 何[なに]かの 団体[だんたい]に 入[はい]っていたり、 何[なに]かの 商品[しょうひん]を 売[う]っていたりして、 勧誘[かんゆう]を 断[ことわ]るに 断[ことわ]れないケースがよくある。
\\	録画予約をして帰ってみたら、停電で全然録画されておらず泣くに泣けなかったという経験はありませんか。	
\\	録画[ろくが] 予約[よやく]をして 帰[かえ]ってみたら、 停電[ていでん]で 全然[ぜんぜん] 録画[ろくが]されておらず 泣[な]くに 泣[な]けなかったという 経験[けいけん]はありませんか。
\\	急に今日中に終わらせなければならない仕事が入った。今日は子供の誕生日で早く帰ると約束したのに、周りの同僚も自分の仕事に必死で、頼むに頼めない。	
\\	急[きゅう]に 今日[きょう] 中[じゅう]に 終[お]わらせなければならない 仕事[しごと]が 入[はい]った。 今日[きょう]は 子供[こども]の 誕生[たんじょう] 日[び]で 早[はや]く 帰[かえ]ると 約束[やくそく]したのに、 周[まわ]りの 同僚[どうりょう]も 自分[じぶん]の 仕事[しごと]に 必死[ひっし]で、 頼[たの]むに 頼[たの]めない。
\\	通訳の失敗談は、話として聞いている分にはおかしいものが多いが、中には深刻すぎて笑うに笑えないものもある。	
\\	通訳[つうやく]の 失敗談[しっぱいだん]は、 話[はなし]として 聞[き]いている 分[ぶん]にはおかしいものが 多[おお]いが、 中[なか]には 深刻[しんこく]すぎて 笑[わら]うに 笑[わら]えないものもある。
\\	長い間、畳に正座していて痺れがきれ、立つに立てない。	
\\	長[なが]い 間[あいだ]、 畳[たたみ]に 正座[せいざ]していて 痺[しび]れがきれ、 立[た]つに 立[た]てない。
\\	お尻に出来物ができているので座るに座れない。	
\\	お 尻[しり]に 出来物[できもの]ができているので 座[すわ]るに 座[すわ]れない。
\\	後任者が決まらないので辞めたくても辞められない。	
\\	後任[こうにん] 者[しゃ]が 決[き]まらないので 辞[や]めたくても 辞[や]められない。
\\	大事なお客さんからの電話がいつまで待ってもかかってこないので、帰りたくとも帰れない。	
\\	大事[だいじ]なお 客[きゃく]さんからの 電話[でんわ]がいつまで 待[ま]ってもかかってこないので、 帰[かえ]りたくとも 帰[かえ]れない。
\\	彼女は今日はいつにもなく陽気だった。	
\\	彼女[かのじょ]は 今日[きょう]はいつにもなく 陽気[ようき]だった。
\\	宏は「ただいまー」といつになく明るい声で帰ってきた。	
\\	宏[ひろし]は「ただいまー」といつになく 明[あか]るい 声[こえ]で 帰[かえ]ってきた。
\\	普段はたいていビールや日本酒だが、たまに柄にもなくバーにカクテルを飲みに行ったりする。	
\\	普段[ふだん]はたいていビールや 日本[にほん] 酒[しゅ]だが、たまに 柄[がら]にもなくバーにカクテルを 飲[の]みに 行[い]ったりする。
\\	今年は例年になく猛暑で、いつもは省エネを心がけている私が、いつの間にかクーラーをかけてしまっている。	
\\	今年[ことし]は 例年[れいねん]になく 猛暑[もうしょ]で、いつもは 省エネ[しょうえね]を 心[こころ]がけている 私[わたし]が、いつの 間[あいだ]にかクーラーをかけてしまっている。
\\	気になっていた彼女が隣に座ったので、僕の心は我にもなく弾んだ。	
\\	気[き]になっていた 彼女[かのじょ]が 隣[となり]に 座[すわ]ったので、 僕[ぼく]の 心[こころ]は 我[わが]にもなく 弾[はず]んだ。
\\	天候や道路状況に応じた適切な運転を心がけましょう。	
\\	天候[てんこう]や 道路[どうろ] 状況[じょうきょう]に 応[おう]じた 適切[てきせつ]な 運転[うんてん]を 心[こころ]がけましょう。
\\	当社では要請に応じて講師を学校に派遣することもできます。	
\\	当社[とうしゃ]では 要請[ようせい]に 応[おう]じて 講師[こうし]を 学校[がっこう]に 派遣[はけん]することもできます。
\\	病院でもらう薬は、医師が患者の体質や症状に応じて処方します。	
\\	病院[びょういん]でもらう 薬[くすり]は、 医師[いし]が 患者[かんじゃ]の 体質[たいしつ]や 症状[しょうじょう]に 応[おう]じて 処方[しょほう]します。
\\	湯沸かし器の設定温度は、目的に応じてその都度調節するのが賢い使い方だ	
\\	湯沸かし器[ゆわかしき]の 設定[せってい] 温度[おんど]は、 目的[もくてき]に 応[おう]じてその 都度[つど] 調節[ちょうせつ]するのが 賢[かしこ]い 使い方[つかいかた]だ
\\	今年行われた「今後の大学教育のあり方に関する世論調査」によると、開国語教育の効果的だと思われる授業のあり方に関して、33.7
\\	が「学生個人の能力に応じてクラスを分けて授業を行く」と答えた。	
\\	今年[ことし] 行[おこな]われた
\\	今後[こんご]の 大学[だいがく] 教育[きょういく]のあり 方[かた]に 関[かん]する 世論[よろん] 調査[ちょうさ]」によると、 開国[かいこく] 語[ご] 教育[きょういく]の 効果[こうか] 的[てき]だと 思[おも]われる 授業[じゅぎょう]のあり 方[かた]に 関[かん]して、 
\\	が
\\	学生[がくせい] 個人[こじん]の 能力[のうりょく]に 応[おう]じてクラスを 分[わ]けて 授業[じゅぎょう]を 行[い]く」と 答[こた]えた。
\\	髪のケアには欠かせないシャンプーには症状に応じて様々なタイプがある。	
\\	髪[かみ]のケアには 欠[か]かせないシャンプーには 症状[しょうじょう]に 応[おう]じて 様々[さまざま]なタイプがある。
\\	一般に供給は需要に応じて変動する。しかし、場合によって需要が供給に応じて変動するのも事実である。	
\\	一般[いっぱん]に 供給[きょうきゅう]は 需要[じゅよう]に 応[おう]じて 変動[へんどう]する。しかし、 場合[ばあい]によって 需要[じゅよう]が 供給[きょうきゅう]に 応[おう]じて 変動[へんどう]するのも 事実[じじつ]である。
\\	家を購入する場合には収入に応じた物件を選ぶべきです。	
\\	家[いえ]を 購入[こうにゅう]する 場合[ばあい]には 収入[しゅうにゅう]に 応[おう]じた 物件[ぶっけん]を 選[えら]ぶべきです。
\\	我が社はお客さまのご予算に応じた費用対効果の高いホームページ製作を支援します。	
\\	我[わ]が 社[しゃ]はお 客[きゃく]さまのご 予算[よさん]に 応[おう]じた 費用[ひよう] 対[たい] 効果[こうか]の 高[たか]いホームページ 製作[せいさく]を 支援[しえん]します。
\\	平成15年に公布された「個人情報の保護に関する法律」は、個人情報取扱事業者に対し、本人の求めに応じて保有個人データの開示や訂正を行うこと、また本人の求めに応じてその利用を停止することなどを義務づけている。	
\\	平成[へいせい] 
\\	年[ねん]に 公布[こうふ]された
\\	個人[こじん] 情報[じょうほう]の 保護[ほご]に 関[かん]する 法律[ほうりつ]」は、 個人[こじん] 情報[じょうほう] 取扱[とりあつかい] 事業[じぎょう] 者[しゃ]に 対[たい]し、 本人[ほんにん]の 求[もと]めに 応[おう]じて 保有[ほゆう] 個人[こじん]データの 開示[かいじ]や 訂正[ていせい]を 行[おこな]うこと、また 本人[ほんにん]の 求[もと]めに 応[おう]じてその 利用[りよう]を 停止[ていし]することなどを 義務[ぎむ]づけている。
\\	冷凍食品の種類に応じた解凍方法をご紹介します。	
\\	冷凍[れいとう] 食品[しょくひん]の 種類[しゅるい]に 応[おう]じた 解凍[かいとう] 方法[ほうほう]をご 紹介[しょうかい]します。
\\	地域における多様な需要に応じた公的賃貸住宅の整備が必要だ。	
\\	地域[ちいき]における 多様[たよう]な 需要[じゅよう]に 応[おう]じた 公的[こうてき] 賃貸[ちんたい] 住宅[じゅうたく]の 整備[せいび]が 必要[ひつよう]だ。
\\	このカメラは水中でレンズが交換できるため、被写体に応じた多彩な撮影が楽しめます。	
\\	このカメラは 水中[すいちゅう]でレンズが 交換[こうかん]できるため、 被写体[ひしゃたい]に 応[おう]じた 多彩[たさい]な 撮影[さつえい]が 楽[たの]しめます。
\\	バッターに応じて守備位置を変える。	
\\	バッターに 応[おう]じて 守備[しゅび] 位置[いち]を 変[か]える。
\\	症状にによって異なる治療を施す。	
\\	症状[しょうじょう]にによって 異[こと]なる 治療[ちりょう]を 施[ほどこ]す。
\\	業績に応じてボーナスを決める。	
\\	業績[ぎょうせき]に 応[おう]じてボーナスを 決[き]める。
\\	この本は読み方によって毒にも楽にもなる。	
\\	この 本[ほん]は 読み方[よみかた]によって 毒[どく]にも 楽[らく]にもなる。
\\	言い方によっては彼女を傷つけるかもしれない。	
\\	言い方[いいかた]によっては 彼女[かのじょ]を 傷[きず]つけるかもしれない。
\\	この検査の結果によって手術をしない可能性もある。	
\\	この 検査[けんさ]の 結果[けっか]によって 手術[しゅじゅつ]をしない 可能[かのう] 性[せい]もある。
\\	この検査の結果に応じて手術をするか薬餌療法にするかを決めたい。	
\\	この 検査[けんさ]の 結果[けっか]に 応[おう]じて 手術[しゅじゅつ]をするか 薬餌[やくじ] 療法[りょうほう]にするかを 決[き]めたい。
\\	この検査の結果によって治療法を決めたい。	
\\	この 検査[けんさ]の 結果[けっか]によって 治療[ちりょう] 法[ほう]を 決[き]めたい。
\\	使う言葉が英語か日本語かによって発表の準備の仕方が変わる。	
\\	使[つか]う 言葉[ことば]が 英語[えいご]か 日本語[にほんご]かによって 発表[はっぴょう]の 準備[じゅんび]の 仕方[しかた]が 変[か]わる。
\\	誰が聞きに来るかによって話の内容を変えるつもりだ。	
\\	誰[だれ]が 聞[き]きに 来[く]るかによって 話[はなし]の 内容[ないよう]を 変[か]えるつもりだ。
\\	業績次第で従業員全員にボーナスが支給される。	
\\	業績[ぎょうせき] 次第[しだい]で 従業[じゅうぎょう] 員[いん] 全員[ぜんいん]にボーナスが 支給[しきゅう]される。
\\	実験の結果次第で、大発見になる可能性もある。	
\\	実験[じっけん]の 結果[けっか] 次第[しだい]で、 大発見[だいはっけん]になる 可能[かのう] 性[せい]もある。
\\	生活排水などの汚水は、その処理方法いかんで、周辺の水環境に悪影響を与えます。	
\\	生活[せいかつ] 排水[はいすい]などの 汚水[おすい]は、その 処理[しょり] 方法[ほうほう]いかんで、 周辺[しゅうへん]の 水[みず] 環境[かんきょう]に 悪影響[あくえいきょう]を 与[あた]えます。
\\	時間があまりなかったにしろ、この仕事はひどすぎる。	
\\	時間[じかん]があまりなかったにしろ、この 仕事[しごと]はひどすぎる。
\\	誰が書くにせよ、絶対間違いがあってはいけない。	
\\	誰[だれ]が 書[か]くにせよ、 絶対[ぜったい] 間違[まちが]いがあってはいけない。
\\	行くにしろ、行かないにしろ、後で電話を下さい。	
\\	行[い]くにしろ、 行[い]かないにしろ、 後[あと]で 電話[でんわ]を 下[くだ]さい。
\\	何にせよ、このままでは問題は解決しない。	
\\	何[なに]にせよ、このままでは 問題[もんだい]は 解決[かいけつ]しない。
\\	たとえ練習にしろ、真面目にやらなければならない。	
\\	たとえ 練習[れんしゅう]にしろ、 真面目[まじめ]にやらなければならない。
\\	程度の差はあるにせよ、性差別のない国はないだろう。	
\\	程度[ていど]の 差[さ]はあるにせよ、 性[せい] 差別[さべつ]のない 国[くに]はないだろう。
\\	薬を飲むことですべての問題が解決はしないにしろ、あなたの場合は薬で症状を軽減させることが先決です。	
\\	薬[くすり]を 飲[の]むことですべての 問題[もんだい]が 解決[かいけつ]はしないにしろ、あなたの 場合[ばあい]は 薬[くすり]で 症状[しょうじょう]を 軽減[けいげん]させることが 先決[せんけつ]です。
\\	誰かに書いてもらうにしろ、内容は我々が考えなければならない。	
\\	誰[だれ]かに 書[か]いてもらうにしろ、 内容[ないよう]は 我々[われわれ]が 考[かんが]えなければならない。
\\	どのソフトを使うにしろ、早く使い方に慣れてもらわないといけない。	
\\	どのソフトを 使[つか]うにしろ、 早[はや]く 使い方[つかいかた]に 慣[な]れてもらわないといけない。
\\	誰がそれを説明するにせよ、相手の気持ちを傷つけないように細心の注意が必要だ。	
\\	誰[だれ]がそれを 説明[せつめい]するにせよ、 相手[あいて]の 気持[きも]ちを 傷[きず]つけないように 細心[さいしん]の 注意[ちゅうい]が 必要[ひつよう]だ。
\\	どんな事情があったにせよ、何も言わずに仕事を辞めるのは無責任すぎる。	
\\	どんな 事情[じじょう]があったにせよ、 何[なに]も 言[い]わずに 仕事[しごと]を 辞[や]めるのは 無責任[むせきにん]すぎる。
\\	ホームページの制作に関する書籍は書店の棚にあふれている状態で、自分が読むにせよ、他人に勧めるにせよ、大いに迷う。	
\\	ホームページの 制作[せいさく]に 関[かん]する 書籍[しょせき]は 書店[しょてん]の 棚[たな]にあふれている 状態[じょうたい]で、 自分[じぶん]が 読[よ]むにせよ、 他人[たにん]に 勧[すす]めるにせよ、 大[おお]いに 迷[まよ]う。
\\	現政権を支持するにせよ批判するにせよ、まずはじめに問題に対する正確な理解と冷静な分析が不可欠である。	
\\	現[げん] 政権[せいけん]を 支持[しじ]するにせよ 批判[ひはん]するにせよ、まずはじめに 問題[もんだい]に 対[たい]する 正確[せいかく]な 理解[りかい]と 冷静[れいせい]な 分析[ぶんせき]が 不可欠[ふかけつ]である。
\\	野球にせよサッカーにせよ、プロでやっていくには並の努力では続かない。	
\\	野球[やきゅう]にせよサッカーにせよ、プロでやっていくには 並[なみ]の 努力[どりょく]では 続[つづ]かない。
\\	普通の人は、意識的にせよ無意識的にせよ、不快なことを避けようとする。	
\\	普通[ふつう]の 人[ひと]は、 意識[いしき] 的[てき]にせよ 無[む] 意識[いしき] 的[てき]にせよ、 不快[ふかい]なことを 避[さ]けようとする。
\\	一旦ネットに発信してしまった情報は、それが真実にせよそうでないにせよ、取り消しができない。	
\\	一旦[いったん]ネットに 発信[はっしん]してしまった 情報[じょうほう]は、それが 真実[しんじつ]にせよそうでないにせよ、 取り消[とりけ]しができない。
\\	いずれにせよ早急に対策を考えなければならない。	
\\	いずれにせよ 早急[そうきゅう]に 対策[たいさく]を 考[かんが]えなければならない。
\\	何にせよこの問題は我々だけで考えていても解決策は出てこない。	
\\	何[なに]にせよこの 問題[もんだい]は 我々[われわれ]だけで 考[かんが]えていても 解決[かいけつ] 策[さく]は 出[で]てこない。
\\	一時的にせよ国内全空港で、航空機が約20分間にわたり離陸できなくなったことは深刻な問題だ。	
\\	一時[いちじ] 的[てき]にせよ 国内[こくない] 全[ぜん] 空港[くうこう]で、 航空機[こうくうき]が 約[やく] 20分間[にじゅっぷんかん]にわたり 離陸[りりく]できなくなったことは 深刻[しんこく]な 問題[もんだい]だ。
\\	たとえ知らずにやったことであるにしろ、彼は責任を免れないだろう。	
\\	たとえ 知[し]らずにやったことであるにしろ、 彼[かれ]は 責任[せきにん]を 免[まぬか]れないだろう。
\\	ヒューマン・インターフェースにせよネットワークにせよ、この数年間でさらに大きく変わるだろう。	
\\	ヒューマン・インターフェースにせよネットワークにせよ、この 数[すう] 年間[ねんかん]でさらに 大[おお]きく 変[か]わるだろう。
\\	会社を辞めるにせよ留まるにせよ、早く結論を出さなければならない。	
\\	会社[かいしゃ]を 辞[や]めるにせよ 留[とど]まるにせよ、 早[はや]く 結論[けつろん]を 出[だ]さなければならない。
\\	高いにせよ安いにせよ、必要なら買うしかない。	
\\	高[たか]いにせよ 安[やす]いにせよ、 必要[ひつよう]なら 買[か]うしかない。
\\	たまににしろしょっちゅうにしろ、私用で電話をするのは勤務時間外にして下さい。	
\\	たまににしろしょっちゅうにしろ、 私用[しよう]で 電話[でんわ]をするのは 勤務[きんむ] 時間[じかん] 外[がい]にして 下[くだ]さい。
\\	たとえそれが練習であるにしろ、真面目にやらなければならない。	
\\	たとえそれが 練習[れんしゅう]であるにしろ、 真面目[まじめ]にやらなければならない。
\\	どんな理由からにしろ何も言わずに仕事を辞めるのは無責任すぎる。	
\\	どんな 理由[りゆう]からにしろ 何[なに]も 言[い]わずに 仕事[しごと]を 辞[や]めるのは 無責任[むせきにん]すぎる。
\\	彼は日本語を20年も勉強しているが、その前にしたところで、まだ分からない文法に時々出くわすそうだ。	
\\	彼[かれ]は 日本語[にほんご]を 
\\	年[ねん]も 勉強[べんきょう]しているが、その 前[まえ]にしたところで、まだ 分[わ]からない 文法[ぶんぽう]に 時々[ときどき] 出[で]くわすそうだ。
\\	子供はいつか手の届かないところへ行ってしまうものだ。自分の子だけはそうでないと思うかもしれないが、あなたの子にしたところで同じだ。	
\\	子供[こども]はいつか 手[て]の 届[とど]かないところへ 行[い]ってしまうものだ。 自分[じぶん]の 子[こ]だけはそうでないと 思[おも]うかもしれないが、あなたの 子[こ]にしたところで 同[おな]じだ。
\\	昔の浮気が発覚した夫は「10年たってるんだからもう時効だ」と開き直りの態度です。私にしたところで二人の子供のことを考えると離婚にはなかなか踏み切れません。	
\\	昔[むかし]の 浮気[うわき]が 発覚[はっかく]した 夫[おっと]は
\\	年[ねん]たってるんだからもう 時効[じこう]だ」と 開き直[ひらきなお]りの 態度[たいど]です。 私[わたし]にしたところで 二人[ふたり]の 子供[こども]のことを 考[かんが]えると 離婚[りこん]にはなかなか 踏み切[ふみき]れません。
\\	最近は万年筆で字を書く人というのを、ほとんど見なくなって残念だ。もっとも、そういう私にしたところで、普段使っているのはボールペンだ。	
\\	最近[さいきん]は 万年筆[まんねんひつ]で 字[じ]を 書[か]く 人[ひと]というのを、ほとんど 見[み]なくなって 残念[ざんねん]だ。もっとも、そういう 私[わたし]にしたところで、 普段[ふだん] 使[つか]っているのはボールペンだ。
\\	三島由紀夫を産湯に浸かった記憶があると書いているが、彼にしたところで、死ぬまでのすべての記憶があるわけではない。	
\\	三島[みしま] 由紀夫[ゆきお]を 産湯[うぶゆ]に 浸[つ]かった 記憶[きおく]があると 書[か]いているが、 彼[かれ]にしたところで、 死[し]ぬまでのすべての 記憶[きおく]があるわけではない。
\\	彼はひらがなでも読めない。	
\\	彼[かれ]はひらがなでも 読[よ]めない。
\\	キャロルは納豆も食べられる。	
\\	キャロルは 納豆[なっとう]も 食[た]べられる。
\\	そんなことは子供でも知っている。	
\\	そんなことは 子供[こども]でも 知[し]っている。
\\	そんな数学の問題、天才も解けないだろうな。	
\\	そんな 数学[すうがく]の 問題[もんだい]、 天才[てんさい]も 解[と]けないだろうな。
\\	国粋主義者でもその戦争に反対している人は多い。	
\\	国粋[こくすい] 主義[しゅぎ] 者[しゃ]でもその 戦争[せんそう]に 反対[はんたい]している 人[ひと]は 多[おお]い。
\\	過度の恐怖や心労は一夜にして白髪をつくる。	
\\	過度[かど]の 恐怖[きょうふ]や 心労[しんろう]は 一夜[いちや]にして 白髪[しらが]をつくる。
\\	彼女は45歳にして初めて子供ができた。	
\\	彼女[かのじょ]は 
\\	歳[さい]にして 初[はじ]めて 子供[こども]ができた。
\\	外国に居ながらにして日本からの生のテレビ放送が見られるようになった。	
\\	外国[がいこく]に 居[い]ながらにして 日本[にほん]からの 生[なま]のテレビ 放送[ほうそう]が 見[み]られるようになった。
\\	空襲で東京の町は一夜にして焼け野原になってしまった。	
\\	空襲[くうしゅう]で 東京[とうきょう]の 町[まち]は 一夜[いちや]にして 焼け野原[やけのはら]になってしまった。
\\	この本は発売1週間にして売れ行きが目に見えて落ち込み始めた。	
\\	この 本[ほん]は 発売[はつばい] 
\\	週間[しゅうかん]にして 売れ行[うれゆ]きが 目[め]に 見[み]えて 落ち込[おちこ]み 始[はじ]めた。
\\	美男美女のカップルの子にふさわしく、その女の子は生後1か月にして目鼻立ちがはっきりしていた。	
\\	美男[びなん] 美女[びじょ]のカップルの 子[こ]にふさわしく、その 女の子[おんなのこ]は 生後[せいご] 1か月[いっかげつ]にして 目鼻立[めはなだ]ちがはっきりしていた。
\\	ケプラーは若年にして地位と収入の安定を獲得し、生涯天文学の研究に打ち込む機会をも与えられたのだった。	
\\	ケプラーは 若年[じゃくねん]にして 地位[ちい]と 収入[しゅうにゅう]の 安定[あんてい]を 獲得[かくとく]し、 生涯[しょうがい] 天文学[てんもんがく]の 研究[けんきゅう]に 打ち込[うちこ]む 機会[きかい]をも 与[あた]えられたのだった。
\\	彼はなんと気の早いことに、高校1年にして大学受験日記をつけ始めた。	
\\	彼[かれ]はなんと 気[き]の 早[はや]いことに、 高校[こうこう] 
\\	年[ねん]にして 大学[だいがく] 受験[じゅけん] 日記[にっき]をつけ 始[はじ]めた。
\\	今年受験2回目にして日本語能力試験1級に合格した。	
\\	今年[ことし] 受験[じゅけん] 
\\	回[かい] 目[め]にして 日本語[にほんご] 能力[のうりょく] 試験[しけん] 
\\	級[きゅう]に 合格[ごうかく]した。
\\	人間は生まれながらにして健康に生きていこうとする力を持っているそうだ。	
\\	人間[にんげん]は 生[う]まれながらにして 健康[けんこう]に 生[い]きていこうとする 力[ちから]を 持[も]っているそうだ。
\\	ローマは一日にして成らず。	
\\	ローマは一 日[にち]にして 成[な]らず。
\\	四十にして惑わず。五十にして天命を知る。(『論語』)	
\\	四十[しじゅう] にして 惑[まど]わず。五十[ごじゅう] にして 天命[てんめい]を 知[し]る。
\\	論語[ろんご]』)
\\	春学期は5月10日で終わる。	
\\	春[はる] 学期[がっき]は 
\\	月[がつ] 10日[とうか]で 終[お]わる。
\\	彼女は頭脳明晰にして端麗な秘書である。	
\\	彼女[かのじょ]は 頭脳[ずのう] 明晰[めいせき]にして 端麗[たんれい]な 秘書[ひしょ]である。
\\	彼は偉大な学者にして優れた教育者でもあった。	
\\	彼[かれ]は 偉大[いだい]な 学者[がくしゃ]にして 優[すぐ]れた 教育[きょういく] 者[しゃ]でもあった。
\\	あの男の性格は粗野にして猛々しい。	
\\	あの 男[おとこ]の 性格[せいかく]は 粗野[そや]にして 猛々[たけだけ]しい。
\\	首相の容体は不安定にして、予断を許さない。	
\\	首相[しゅしょう]の 容体[ようだい]は 不安定[ふあんてい]にして、 予断[よだん]を 許[ゆる]さない。
\\	その政治家は志操堅固にして、高潔清廉な人だった。	
\\	その 政治[せいじ] 家[か]は 志操[しそう] 堅固[けんご]にして、 高潔[こうけつ] 清廉[せいれん]な 人[ひと]だった。
\\	これほど巧妙にして完璧な犯罪はないでしょう。	
\\	これほど 巧妙[こうみょう]にして 完璧[かんぺき]な 犯罪[はんざい]はないでしょう。
\\	彼女の絵は緻密にして軽やかな筆づかいで圧倒的な支持を集めた。	
\\	彼女[かのじょ]の 絵[え]は 緻密[ちみつ]にして 軽[かろ]やかな 筆[ふで]づかいで 圧倒的[あっとうてき]な 支持[しじ]を 集[あつ]めた。
\\	あの文法辞典は詳細にして明解な説明が施されている。	
\\	あの 文法[ぶんぽう] 辞典[じてん]は 詳細[しょうさい]にして 明解[めいかい]な 説明[せつめい]が 施[ほどこ]されている。
\\	これは堅牢にして素朴な民芸品であります。	
\\	これは 堅牢[けんろう]にして 素朴[そぼく]な 民芸[みんげい] 品[ひん]であります。
\\	この監督の映画はすべて陰鬱にして物悲しい。	
\\	この 監督[かんとく]の 映画[えいが]はすべて 陰鬱[いんうつ]にして 物悲[ものがな]しい。
\\	あの政治家の言動は非常識にして、絶えず批判されている。	
\\	あの 政治[せいじ] 家[か]の 言動[げんどう]は 非常識[ひじょうしき]にして、 絶[た]えず 批判[ひはん]されている。
\\	栄太は内心頑固にして決して人に使われる男ではない。	
\\	栄太[えいた]は 内心[ないしん] 頑固[がんこ]にして 決[けっ]して 人[ひと]に 使[つか]われる 男[おとこ]ではない。
\\	彼は優秀な研究者にして熱烈な教育者でもある。	
\\	彼[かれ]は 優秀[ゆうしゅう]な 研究[けんきゅう] 者[しゃ]にして 熱烈[ねつれつ]な 教育[きょういく] 者[しゃ]でもある。
\\	あの子はかわいくて、頭がいい。	
\\	あの 子[こ]はかわいくて、 頭[あたま]がいい。
\\	その秘書は頭脳明晰にして端麗である。	
\\	その 秘書[ひしょ]は 頭脳[ずのう] 明晰[めいせき]にして 端麗[たんれい]である。
\\	不幸にして彼は自分の欠点が見えていない。	
\\	不幸[ふこう]にして 彼[かれ]は 自分[じぶん]の 欠点[けってん]が 見[み]えていない。
\\	ワールドカップの時はサッカーファンでもない私にしてからが興奮状態でした。	
\\	ワールドカップの 時[とき]はサッカーファンでもない 私[わたし]にしてからが 興奮[こうふん] 状態[じょうたい]でした。
\\	警察にしてからが、いざ事件が起きないと市民を暴力から守ってくれない。	
\\	警察[けいさつ]にしてからが、いざ 事件[じけん]が 起[お]きないと 市民[しみん]を 暴力[ぼうりょく]から 守[まも]ってくれない。
\\	ギリシャやローマの彫刻や壁画などに、痩せた女性は皆無だ。美の女神、ヴィーナスにしてからが、かなり太めだ。	
\\	ギリシャやローマの 彫刻[ちょうこく]や 壁画[へきが]などに、 痩[や]せた 女性[じょせい]は 皆無[かいむ]だ。 美[び]の 女神[めがみ]、ヴィーナスにしてからが、かなり 太[ふと]めだ。
\\	当代随一の知性人である彼にしてからが、恋愛にたぶらかされている。	
\\	当代[とうだい] 随一[ずいいち]の 知性[ちせい] 人[じん]である 彼[かれ]にしてからが、 恋愛[れんあい]にたぶらかされている。
\\	息子は父から受け継いだ作風で作品を描いていたが、すでにその父にしてからが、古くさいと言われていた。	
\\	息子[むすこ]は 父[ちち]から 受け継[うけつ]いだ 作風[さくふう]で 作品[さくひん]を 描[えが]いていたが、すでにその 父[ちち]にしてからが、 古[ふる]くさいと 言[い]われていた。
\\	首相の構造改革案にしてからが、官僚の強い抵抗にあっている。	
\\	首相[しゅしょう]の 構造[こうぞう] 改革[かいかく] 案[あん]にしてからが、 官僚[かんりょう]の 強[つよ]い 抵抗[ていこう]にあっている。
\\	夫は簡単な家事でさえしてくれない。	
\\	夫[おっと]は 簡単[かんたん]な 家事[かじ]でさえしてくれない。
\\	彼女は自分の町のニュースですら興味を示さない。	
\\	彼女[かのじょ]は 自分[じぶん]の 町[まち]のニュースですら 興味[きょうみ]を 示[しめ]さない。
\\	今までの証拠からして彼が犯人であることは確実だ。	
\\	今[いま]までの 証拠[しょうこ]からして 彼[かれ]が 犯人[はんにん]であることは 確実[かくじつ]だ。
\\	外見からして彼はビジネスマンだろう。	
\\	外見[がいけん]からして 彼[かれ]はビジネスマンだろう。
\\	大学院生にしたところで、学者になるとは限らない。	
\\	大学院生[だいがくいんせい]にしたところで、 学者[がくしゃ]になるとは 限[かぎ]らない。
\\	三島由紀夫は産湯に浸かった記憶があると書いているが、彼にしたところで、死ぬまでのすべての記憶があるわけではない。	
\\	三島[みしま] 由紀夫[ゆきお]は 産湯[うぶゆ]に 浸[つ]かった 記憶[きおく]があると 書[か]いているが、 彼[かれ]にしたところで、 死[し]ぬまでのすべての 記憶[きおく]があるわけではない。
\\	ピアノを買うにしても、こんな狭いアパートでは置くところがない。	
\\	ピアノを 買[か]うにしても、こんな 狭[せま]いアパートでは 置[お]くところがない。
\\	美紀の言ったことは大がさだったにしても、すべてがウソというわけではないだろう。	
\\	美紀[みき]の 言[い]ったことは 大[おお]がさだったにしても、すべてがウソというわけではないだろう。
\\	誰がこの仕事をやるにしても失敗は許されない。	
\\	誰[だれ]がこの 仕事[しごと]をやるにしても 失敗[しっぱい]は 許[ゆる]されない。
\\	君が行くにしても他の誰かが行くにしても今回の会議はかなりの準備が必要だ。	
\\	君[きみ]が 行[い]くにしても 他[ほか]の 誰[だれ]かが 行[い]くにしても 今回[こんかい]の 会議[かいぎ]はかなりの 準備[じゅんび]が 必要[ひつよう]だ。
\\	敬語の使い方にしても近頃は間違った言い方が目につく。	
\\	敬語[けいご]の 使い方[つかいかた]にしても 近頃[ちかごろ]は 間違[まちが]った 言い方[いいかた]が 目[め]につく。
\\	今すぐ帰るにしても、飛行機の切符が取れるだろうか。	
\\	今[いま]すぐ 帰[かえ]るにしても、 飛行機[ひこうき]の 切符[きっぷ]が 取[と]れるだろうか。
\\	京都は見るところがたくさんあるので、市内だけ回るにしても3日くらいとておいた方がいいでしょう。	
\\	京都[きょうと]は 見[み]るところがたくさんあるので、 市内[しない]だけ 回[まわ]るにしても 
\\	日[にち]くらいとておいた 方[ほう]がいいでしょう。
\\	この作品は中国大陸の影響が濃厚であるにしても、日本の画工の高い水準を示している。	
\\	この 作品[さくひん]は 中国[ちゅうごく] 大陸[たいりく]の 影響[えいきょう]が 濃厚[のうこう]であるにしても、 日本[にほん]の 画工[がこう]の 高[たか]い 水準[すいじゅん]を 示[しめ]している。
\\	今度の台風では最悪の事態は避けられたにしても、被害は甚大だ。	
\\	今度[こんど]の 台風[たいふう]では 最悪[さいあく]の 事態[じたい]は 避[さ]けられたにしても、 被害[ひがい]は 甚大[じんだい]だ。
\\	ニューヨークほどではないにしても、この市でもかなり質の高い音楽が聴けます。	
\\	ニューヨークほどではないにしても、この 市[いち]でもかなり 質[しつ]の 高[たか]い 音楽[おんがく]が 聴[き]けます。
\\	何をするにしても英語はしっかり勉強しておいた方がいい。	
\\	何[なに]をするにしても 英語[えいご]はしっかり 勉強[べんきょう]しておいた 方[ほう]がいい。
\\	行くにしてもやめるにしても早く決めてもらえませんか。	
\\	行[い]くにしてもやめるにしても 早[はや]く 決[き]めてもらえませんか。
\\	達也にしても悪気があってあんなことを言ったわけじゃない。	
\\	達也[たつや]にしても 悪気[わるぎ]があってあんなことを 言[い]ったわけじゃない。
\\	輸入キャンピングカーの車内器具はほとんどが120
\\	の規格で作られているので、100
\\	電源の日本のキャンプ場ではレンジにしてもエアコンにしても、今一つ出力が弱い。	
\\	輸入[ゆにゅう]キャンピングカーの 車内[しゃない] 器具[きぐ]はほとんどが 
\\	の 規格[きかく]で 作[つく]られているので、 
\\	電源[でんげん]の 日本[にほん]のキャンプ 場[じょう]ではレンジにしてもエアコンにしても、 今一[いまひと]つ 出力[しゅつりょく]が 弱[よわ]い。
\\	散歩に行くにしても、一言そう言っておいてくれたら心配しなかったのに。	
\\	散歩[さんぽ]に 行[い]くにしても、 一言[ひとこと]そう 言[い]っておいてくれたら 心配[しんぱい]しなかったのに。
\\	部下を叱るにしても、もう少しほかに言い方がなかったのだろうか。	
\\	部下[ぶか]を 叱[しか]るにしても、もう 少[すこ]しほかに 言い方[いいかた]がなかったのだろうか。
\\	長い人生いろいろあるのは当たり前だけど、それにしてもいやなことがあまりに多すぎる。	
\\	長[なが]い 人生[じんせい]いろいろあるのは 当たり前[あたりまえ]だけど、それにしてもいやなことがあまりに 多[おお]すぎる。
\\	ー慶子、遅いね。渋滞にあってるのかな。 ーそれにしても、遅すぎると思わない?	
\\	ー 慶子[けいこ]、 遅[おそ]いね。 渋滞[じゅうたい]にあってるのかな。 ーそれにしても、 遅[おそ]すぎると 思[おも]わない?
\\	それにしても、なぜこんなに多くのフォーマットがあるのですか。	
\\	それにしても、なぜこんなに 多[おお]くのフォーマットがあるのですか。
\\	それにしても、名古屋は本当に久しぶりだ。	
\\	それにしても、 名古屋[なごや]は 本当[ほんとう]に 久[ひさ]しぶりだ。
\\	何にしても早急に結論を出さないといけない。	
\\	何[なに]にしても 早急[そうきゅう]に 結論[けつろん]を 出[だ]さないといけない。
\\	彼が社長だったとしてもこの経営危機は救えないだろう。	
\\	彼[かれ]が 社長[しゃちょう]だったとしてもこの 経営[けいえい] 危機[きき]は 救[すく]えないだろう。
\\	90年代のバブル経済の崩壊がなかったとしても、日本の高度経済成長はいつか止まっていたはずだ。	
\\	年代[ねんだい]のバブル 経済[けいざい]の 崩壊[ほうかい]がなかったとしても、 日本[にほん]の 高度[こうど] 経済[けいざい] 成長[せいちょう]はいつか 止[と]まっていたはずだ。
\\	ピアノを買うとしても、こんな狭いアパートでは置くところがない。	
\\	ピアノを 買[か]うとしても、こんな 狭[せま]いアパートでは 置[お]くところがない。
\\	美紀の言ったことは大げせだったとしても、すべてがウソというわけではないだろう。	
\\	美紀[みき]の 言[い]ったことは 大[おお]げせだったとしても、すべてがウソというわけではないだろう。
\\	彼の研究範囲は哲学だけにとどまるず、認知心理学の領域まで広がっている。	
\\	彼[かれ]の 研究[けんきゅう] 範囲[はんい]は 哲学[てつがく]だけにとどまるず、 認知[にんち] 心理[しんり] 学[がく]の 領域[りょういき]まで 広[ひろ]がっている。
\\	彼女は小説を広く読むのとどまるず、自分でもいくつか書いて、すでに雑誌に発表している。	
\\	彼女[かのじょ]は 小説[しょうせつ]を 広[ひろ]く 読[よ]むのとどまるず、 自分[じぶん]でもいくつか 書[か]いて、すでに 雑誌[ざっし]に 発表[はっぴょう]している。
\\	博士は専門の数学にとどまらず、コンピュータサイエンスでも優れた研究をしている。	
\\	博士[はかせ]は 専門[せんもん]の 数学[すうがく]にとどまらず、コンピュータサイエンスでも 優[すぐ]れた 研究[けんきゅう]をしている。
\\	本研究室での研究は、単なる理論にとどまらず、その応用も重視している。	
\\	本[ほん] 研究[けんきゅう] 室[しつ]での 研究[けんきゅう]は、 単[たん]なる 理論[りろん]にとどまらず、その 応用[おうよう]も 重視[じゅうし]している。
\\	先輩方には単に学生時代にとどまらず、卒業後も人生のよき先輩として貴重なアドバイスをいただいてまいりました。	
\\	先輩[せんぱい] 方[がた]には 単[たん]に 学生[がくせい] 時代[じだい]にとどまらず、 卒業[そつぎょう] 後[ご]も 人生[じんせい]のよき 先輩[せんぱい]として 貴重[きちょう]なアドバイスをいただいてまいりました。
\\	本医学部は医学の専門知識だけにとどまらず、人間愛、使命感、責任感、といった人間教育にも力を注いでいます。	
\\	本[ほん] 医学部[いがくぶ]は 医学[いがく]の 専門[せんもん] 知識[ちしき]だけにとどまらず、 人間[にんげん] 愛[あい]、 使命[しめい] 感[かん]、 責任[せきにん] 感[かん]、といった 人間[にんげん] 教育[きょういく]にも 力[ちから]を 注[そそ]いでいます。
\\	室内楽では一人一人の奏者が自分のパートを理解するだけにとどまらず、他のすべてのパートも理解しなければならない。	
\\	室内楽[しつないがく]では 一人[ひとり] 一人[ひとり]の 奏者[そうしゃ]が 自分[じぶん]のパートを 理解[りかい]するだけにとどまらず、 他[た]のすべてのパートも 理解[りかい]しなければならない。
\\	この薬は髪が薄くなるのを防ぐだけにとどまらず、白髪を黒い髪に戻します。	
\\	この 薬[くすり]は 髪[かみ]が 薄[うす]くなるのを 防[ふせ]ぐだけにとどまらず、 白髪[しらが]を 黒[くろ]い 髪[かみ]に 戻[もど]します。
\\	本学の博士号は専門的知識における卓越性を証明するものであるにとどまらず、取得者が人間的力量の持ち主であることを宣言するものです。	
\\	本学[ほんがく]の 博士[はかし] 号[ごう]は 専門[せんもん] 的[てき] 知識[ちしき]における 卓越[たくえつ] 性[せい]を 証明[しょうめい]するものであるにとどまらず、 取得[しゅとく] 者[しゃ]が 人間[にんげん] 的[てき] 力量[りきりょう]の 持ち主[もちぬし]であることを 宣言[せんげん]するものです。
\\	力学という学問は、物質世界を統制する科学技術の体系として重要であるにとどまらず、世界を認識するための体系として重要である。	
\\	力学[りきがく]という 学問[がくもん]は、 物質[ぶっしつ] 世界[せかい]を 統制[とうせい]する 科学[かがく] 技術[ぎじゅつ]の 体系[たいけい]として 重要[じゅうよう]であるにとどまらず、 世界[せかい]を 認識[にんしき]するための 体系[たいけい]として 重要[じゅうよう]である。
\\	彼女の講演はただ面白いだけにとどまらず、必ず深い哲学が含まれている。	
\\	彼女[かのじょ]の 講演[こうえん]はただ 面白[おもしろ]いだけにとどまらず、 必[かなら]ず 深[ふか]い 哲学[てつがく]が 含[ふく]まれている。
\\	ナポリで撮った両親の写真を見るにつけ、あの時の楽しかった旅行を思い出す。	
\\	ナポリで 撮[と]った 両親[りょうしん]の 写真[しゃしん]を 見[み]るにつけ、あの 時[とき]の 楽[たの]しかった 旅行[りょこう]を 思い出[おもいだ]す。
\\	新緑の香りを嗅ぐにつけ、「ああ、春なんだなあ」と思わずのほほんとしてしまいます。	
\\	新緑[しんりょく]の 香[かお]りを 嗅[か]ぐにつけ、「ああ、 春[はる]なんだなあ」と 思[おも]わずのほほんとしてしまいます。
\\	そのピアニストの深みのある音楽を聴き、温かい人柄に触れるにつけ、「音楽は人たり」の思いを強くしております。	
\\	そのピアニストの 深[ふか]みのある 音楽[おんがく]を 聴[き]き、 温[あたた]かい 人柄[ひとがら]に 触[ふ]れるにつけ、
\\	音楽[おんがく]は 人[ひと]たり」の 思[おも]いを 強[つよ]くしております。
\\	深刻化する年金問題や経済問題に対する政府の無策ぶりを見るにつけ、自分のことは自分で守るしかないと思う。	
\\	深刻[しんこく] 化[か]する 年金[ねんきん] 問題[もんだい]や 経済[けいざい] 問題[もんだい]に 対[たい]する 政府[せいふ]の 無策[むさく]ぶりを 見[み]るにつけ、 自分[じぶん]のことは 自分[じぶん]で 守[まも]るしかないと 思[おも]う。
\\	二十歳をはるかに過ぎても自立への見通しのつかない若者を見るにつけ、聞くにつけ、やるせない思いがあります。	
\\	二十歳[はたち]をはるかに 過[す]ぎても 自立[じりつ]への 見通[みとお]しのつかない 若者[わかもの]を 見[み]るにつけ、 聞[き]くにつけ、やるせない 思[おも]いがあります。
\\	いじめや虐待、陰惨な事件など、子供を取り巻く不幸な状況を考えるにつけ、この国の行く末を案じずにはいられない。	
\\	いじめや 虐待[ぎゃくたい]、 陰惨[いんさん]な 事件[じけん]など、 子供[こども]を 取り巻[とりま]く 不幸[ふこう]な 状況[じょうきょう]を 考[かんが]えるにつけ、この 国[くに]の 行く末[ゆくすえ]を 案[あん]じずにはいられない。
\\	世界各地での戦いの様子を見るにつけ胸が痛む。	
\\	世界[せかい] 各地[かくち]での 戦[たたか]いの 様子[ようす]を 見[み]るにつけ 胸[むね]が 痛[いた]む。
\\	最近頻発している医療ミスの記事を読むにつけ、私も被害者になる可能性があったのではと戦慄を覚えてしまう。	
\\	最近[さいきん] 頻発[ひんぱつ]している 医療[いりょう]ミスの 記事[きじ]を 読[よ]むにつけ、 私[わたし]も 被害[ひがい] 者[しゃ]になる 可能[かのう] 性[せい]があったのではと 戦慄[せんりつ]を 覚[おぼ]えてしまう。
\\	年末が一気に駆け足でやってくるのを感じるにつけ、自分の年齢を思い知らされます。	
\\	年末[ねんまつ]が 一気[いっき]に 駆け足[かけあし]でやってくるのを 感[かん]じるにつけ、 自分[じぶん]の 年齢[ねんれい]を 思い知[おもいし]らされます。
\\	携帯電話を持つようになって、何かにつけ便利になった。	
\\	携帯[けいたい] 電話[でんわ]を 持[も]つようになって、 何[なに]かにつけ 便利[べんり]になった。
\\	何事につけても熱中すると他のことが見えなくなる。	
\\	何事[なにごと]につけても 熱中[ねっちゅう]すると 他[ほか]のことが 見[み]えなくなる。
\\	彼は飲むたびに陽気になる。	
\\	彼[かれ]は 飲[の]むたびに 陽気[ようき]になる。
\\	僕は会うたびに彼女のことが好きになった。	
\\	僕[ぼく]は 会[あ]うたびに 彼女[かのじょ]のことが 好[す]きになった。
\\	オリンピックが16日間にわたって開かれた。	
\\	オリンピックが 
\\	日間[にちかん]にわたって 開[ひら]かれた。
\\	干ばつのため広範囲にわたって作物が被害を受けた。	
\\	干[かん]ばつのため 広範囲[こうはんい]にわたって 作物[さくもつ]が 被害[ひがい]を 受[う]けた。
\\	配管工事のため、明日午後1時から午後9時ごろにわたって断水する。	
\\	配管[はいかん] 工事[こうじ]のため、 明日[あす] 午後[ごご] 
\\	時[じ]から 午後[ごご] 
\\	時[じ]ごろにわたって 断水[だんすい]する。
\\	その記事は8回にわたって新聞に連載された。	
\\	その 記事[きじ]は 
\\	回[かい]にわたって 新聞[しんぶん]に 連載[れんさい]された。
\\	約40年にわたる冷戦がついに終わった。	
\\	約[やく] 
\\	年[ねん]にわたる 冷戦[れいせん]がついに 終[お]わった。
\\	者の携帯電話サービスで、関東の一部で昨夜午後9時ごろから約6時間にわたって接続障害があった。	
\\	者[しゃ]の 携帯[けいたい] 電話[でんわ]サービスで、 関東[かんとう]の 一部[いちぶ]で 昨夜[さくや] 午後[ごご] 
\\	時[じ]ごろから 約[やく] 
\\	時間[じかん]にわたって 接続[せつぞく] 障害[しょうがい]があった。
\\	コミック『ピーナッツ』は、1950年に誕生して以来半世紀以上にわたって世界中のファンを魅了してきた。	
\\	コミック『ピーナッツ』は、 
\\	年[ねん]に 誕生[たんじょう]して 以来[いらい] 半[はん] 世紀[せいき] 以上[いじょう]にわたって 世界中[せかいじゅう]のファンを 魅了[みりょう]してきた。
\\	その川の両側には20エーカーにわたって森が広がっている。	
\\	その 川[かわ]の 両側[りょうがわ]には 20エーカーにわたって 森[もり]が 広[ひろ]がっている。
\\	厚い雨雲が関東から九州にわたって広がっている。	
\\	厚[あつ]い 雨雲[あまぐも]が 関東[かんとう]から 九州[きゅうしゅう]にわたって 広[ひろ]がっている。
\\	昨年1年間に両者の間で5回にわたって賄賂の授受があったことが明らかになった。	
\\	昨年[さくねん] 
\\	年間[ねんかん]に 両者[りょうしゃ]の 間[あいだ]で 
\\	回[かい]にわたって 賄賂[わいろ]の 授受[じゅじゅ]があったことが 明[あき]らかになった。
\\	殺人現場は再三にわたって捜査が行われた。	
\\	殺人[さつじん] 現場[げんば]は 再三[さいさん]にわたって 捜査[そうさ]が 行[おこな]われた。
\\	当社では、中古車の電気装置を点火装置やバッテリーを含む5項目にわたってチェックいたします。	
\\	当社[とうしゃ]では、 中古[ちゅうこ] 車[しゃ]の 電気[でんき] 装置[そうち]を 点火[てんか] 装置[そうち]やバッテリーを 含[ふく]む 
\\	項目[こうもく]にわたってチェックいたします。
\\	実験の前に詳細にわたって器具の使い方の指示があった。	
\\	実験[じっけん]の 前[まえ]に 詳細[しょうさい]にわたって 器具[きぐ]の 使い方[つかいかた]の 指示[しじ]があった。
\\	今回の会議では、環境保護の多岐にわたる問題について、議論が交わされた。	
\\	今回[こんかい]の 会議[かいぎ]では、 環境[かんきょう] 保護[ほご]の 多岐[たき]にわたる 問題[もんだい]について、 議論[ぎろん]が 交[か]わされた。
\\	生産段階から消費段階にわたる食品の安全性を確保することが急務だ。	
\\	生産[せいさん] 段階[だんかい]から 消費[しょうひ] 段階[だんかい]にわたる 食品[しょくひん]の 安全[あんぜん] 性[せい]を 確保[かくほ]することが 急務[きゅうむ]だ。
\\	全身にわたって発疹が出た。	
\\	全身[ぜんしん]にわたって 発疹[はっしん]が 出[で]た。
\\	西日本全体にわたって大雨が降った。	
\\	西日本[にしにほん] 全体[ぜんたい]にわたって 大雨[おおあめ]が 降[ふ]った。
\\	地震のため橋全身にわたって亀裂が生じた。	
\\	地震[じしん]のため 橋[はし] 全身[ぜんしん]にわたって 亀裂[きれつ]が 生[しょう]じた。
\\	全560ページにわたる詳細な報告が委員会から提出された。	
\\	全[ぜん] 560ページにわたる 詳細[しょうさい]な 報告[ほうこく]が 委員[いいん] 会[かい]から 提出[ていしゅつ]された。
\\	メルボルンで長さ100キロにわたる巨大なありの巣が見つかった。	
\\	メルボルンで 長[なが]さ 100キロにわたる 巨大[きょだい]なありの 巣[す]が 見[み]つかった。
\\	このプログラムは複数行にわたるひらがな文字列を一括漢字変換できる。	
\\	このプログラムは 複数[ふくすう] 行[ぎょう]にわたるひらがな 文字[もじ] 列[れつ]を 一括[いっかつ] 漢字[かんじ] 変換[へんかん]できる。
\\	4回にわたる手術の末、彼の心臓はほぼ完全に機能を回復した。	
\\	回[かい]にわたる 手術[しゅじゅつ]の 末[すえ]、 彼[かれ]の 心臓[しんぞう]はほぼ 完全[かんぜん]に 機能[きのう]を 回復[かいふく]した。
\\	西宮インターから芦屋インターにわたって交通が渋滞した。	
\\	西宮[にしのみや]インターから 芦屋[あしや]インターにわたって 交通[こうつう]が 渋滞[じゅうたい]した。
\\	左腰から左膝にかけてまだ痺れが残っている。	
\\	左[ひだり] 腰[ごし]から 左[ひだり] 膝[ひざ]にかけてまだ 痺[しび]れが 残[のこ]っている。
\\	私は長年にわたってその研究をしている。	
\\	私[わたし]は 長年[ながねん]にわたってその 研究[けんきゅう]をしている。
\\	5キロにわたって交通が渋滞した。	
\\	5キロにわたって 交通[こうつう]が 渋滞[じゅうたい]した。
\\	リストラによらずにこの会社を救う方法はない。	
\\	リストラによらずにこの 会社[かいしゃ]を 救[すく]う 方法[ほうほう]はない。
\\	この会社では学歴によらず、実力があればどんどん大きい仕事をまかされる。	
\\	この 会社[かいしゃ]では 学歴[がくれき]によらず、 実力[じつりょく]があればどんどん 大[おお]きい 仕事[しごと]をまかされる。
\\	何事によらず困ったことがあったらいつでも相談して下さい。	
\\	何事[なにごと]によらず 困[こま]ったことがあったらいつでも 相談[そうだん]して 下[くだ]さい。
\\	国産か輸入かによらず牛肉は食べないことにしている。	
\\	国産[こくさん]か 輸入[ゆにゅう]かによらず 牛肉[ぎゅうにく]は 食[た]べないことにしている。
\\	この会社は経験のあるなしによらず面接してくれる。	
\\	この 会社[かいしゃ]は 経験[けいけん]のあるなしによらず 面接[めんせつ]してくれる。
\\	天候のいかんによらず試合は予定通り行われます。	
\\	天候[てんこう]のいかんによらず 試合[しあい]は 予定[よてい] 通[どお]り 行[おこな]われます。
\\	誰がそのプロジェクトを担当するかによらず担当者には特別手当が支給される。	
\\	誰[だれ]がそのプロジェクトを 担当[たんとう]するかによらず 担当[たんとう] 者[しゃ]には 特別[とくべつ] 手当[てあて]が 支給[しきゅう]される。
\\	理想は武力によらず紛争を解決することだ。	
\\	理想[りそう]は 武力[ぶりょく]によらず 紛争[ふんそう]を 解決[かいけつ]することだ。
\\	吉川さんは勤務先の斡旋によらず別の会社に再就職した。	
\\	吉川[よしかわ]さんは 勤務[きんむ] 先[さき]の 斡旋[あっせん]によらず 別[べつ]の 会社[かいしゃ]に 再[さい] 就職[しゅうしょく]した。
\\	本契約が両者の合意によらずに変更されることはない。	
\\	本[ほん] 契約[けいやく]が 両者[りょうしゃ]の 合意[ごうい]によらずに 変更[へんこう]されることはない。
\\	当ダンス教育は、いつからでも始められ、経験によらずどなたでもご参加いただけます。	
\\	当[とう]ダンス 教育[きょういく]は、いつからでも 始[はじ]められ、 経験[けいけん]によらずどなたでもご 参加[さんか]いただけます。
\\	当社のセールスマンの給与は勤務年数や年齢によらず、常に販売実績を基に決定される。	
\\	当社[とうしゃ]のセールスマンの 給与[きゅうよ]は 勤務[きんむ] 年数[ねんすう]や 年齢[ねんれい]によらず、 常[つね]に 販売[はんばい] 実績[じっせき]を 基[もと]に 決定[けってい]される。
\\	の長さは
\\	の位置によらず常に一定である。	
\\	の 長[なが]さは 
\\	の 位置[いち]によらず 常[つね]に 一定[いってい]である。
\\	この標準テストは在籍する大学や学部によらず、また大学生であるか社会人であるかによらず、誰でも受けられます。	
\\	この 標準[ひょうじゅん]テストは 在籍[ざいせき]する 大学[だいがく]や 学部[がくぶ]によらず、また 大学生[だいがくせい]であるか 社会[しゃかい] 人[じん]であるかによらず、 誰[だれ]でも 受[う]けられます。
\\	どんな仕事によらず、いつも誠実に務めなければいけない。	
\\	どんな 仕事[しごと]によらず、いつも 誠実[せいじつ]に 務[つと]めなければいけない。
\\	持ち物が多いか少ないかによらず、きちんと整理ができている部屋は気持ちいい。	
\\	持ち物[もちもの]が 多[おお]いか 少[すく]ないかによらず、きちんと 整理[せいり]ができている 部屋[へや]は 気持[きも]ちいい。
\\	発表者がコンピュータを使うか使わないかによらず、プロジェクターとスクリーンは会場に準備しておきます。	
\\	発表[はっぴょう] 者[しゃ]がコンピュータを 使[つか]うか 使[つか]わないかによらず、プロジェクターとスクリーンは 会場[かいじょう]に 準備[じゅんび]しておきます。
\\	入門コースは本会会員であるか否かによらず無料です。	
\\	入門[にゅうもん]コースは 本会[ほんかい] 会員[かいいん]であるか 否[いな]かによらず 無料[むりょう]です。
\\	ストレスはその多少によらずいつも健康と関係する。	
\\	ストレスはその 多少[たしょう]によらずいつも 健康[けんこう]と 関係[かんけい]する。
\\	世の中には、潜在的な生命力のあるなしによらず、生き長らえられる人とそうでない人がいる。	
\\	世の中[よのなか]には、 潜在[せんざい] 的[てき]な 生命[せいめい] 力[りょく]のあるなしによらず、 生[い]き 長[なが]らえられる 人[ひと]とそうでない 人[ひと]がいる。
\\	実験データは、どの機器を使ったかによらず1か所に集めて管理すべきだ。	
\\	実験[じっけん]データは、どの 機器[きき]を 使[つか]ったかによらず 1か所[いっかしょ] に 集[あつ]めて 管理[かんり]すべきだ。
\\	性別によって会費が異なる。	
\\	性別[せいべつ]によって 会費[かいひ]が 異[こと]なる。
\\	奨学金を貰えるかどうかによりどの大学に行くかを決める。	
\\	奨学[しょうがく] 金[きん]を 貰[もら]えるかどうかによりどの 大学[だいがく]に 行[い]くかを 決[き]める。
\\	その温泉は旅行雑誌によると神経痛にいいそうだ。	
\\	その 温泉[おんせん]は 旅行[りょこう] 雑誌[ざっし]によると 神経[しんけい] 痛[つう]にいいそうだ。
\\	日本の法律によるとおおむね12歳未満の者は逮捕の対象にならない。	
\\	日本[にほん]の 法律[ほうりつ]によるとおおむね 
\\	歳[さい] 未満[みまん]の 者[もの]は 逮捕[たいほ]の 対象[たいしょう]にならない。
\\	聞いたところによると来年は授業料が上がるらしい。	
\\	聞[き]いたところによると 来年[らいねん]は 授業[じゅぎょう] 料[りょう]が 上[あ]がるらしい。
\\	新聞の世論調査によると、内閣支持率は22
\\	にまで急落したそうだ。	
\\	新聞[しんぶん]の 世論[せろん] 調査[ちょうさ]によると、 内閣[ないかく] 支持[しじ] 率[りつ]は 
\\	にまで 急落[きゅうらく]したそうだ。
\\	友達に聞いたところによると、フィンランドのメーデーは学生も集まって春を祝う祭りの日だそうだ。	
\\	友達[ともだち]に 聞[き]いたところによると、フィンランドのメーデーは 学生[がくせい]も 集[あつ]まって 春[はる]を 祝[いわ]う 祭[まつ]りの 日[ひ]だそうだ。
\\	博士によると、犬は人間が歩く際に発する弱い電磁波を感じ取っている可能性があるという。	
\\	博士[はかせ]によると、 犬[いぬ]は 人間[にんげん]が 歩[ある]く 際[さい]に 発[はっ]する 弱[よわ]い 電磁波[でんじは]を 感じ取[かんじと]っている 可能[かのう] 性[せい]があるという。
\\	政府筋によると、政府はバイオテロリズムへの対策を研究する審議会を設置する予定であるという。	
\\	政府[せいふ] 筋[すじ]によると、 政府[せいふ]はバイオテロリズムへの 対策[たいさく]を 研究[けんきゅう]する 審議[しんぎ] 会[かい]を 設置[せっち]する 予定[よてい]であるという。
\\	法律によると、介護保険は適切な医療および福祉サービスが受けられるようにすることを目的としています。	
\\	法律[ほうりつ]によると、 介護[かいご] 保険[ほけん]は 適切[てきせつ]な 医療[いりょう]および 福祉[ふくし]サービスが 受[う]けられるようにすることを 目的[もくてき]としています。
\\	辞書によると、「さびしい」には「寂しい」と「淋しい」の二つの漢字が使われている。	
\\	寂 
\\	淋.	辞書[じしょ]によると、「さびしい」には
\\	寂[さび]しい」と
\\	淋[さび]しい」の 二[ふた]つの 漢字[かんじ]が 使[つか]われている。
\\	調査によると、高齢者に対する暴力と虐待の加害者は、家族、友人あるいは知人であることが最も多い。	
\\	調査[ちょうさ]によると、 高齢[こうれい] 者[しゃ]に 対[たい]する 暴力[ぼうりょく]と 虐待[ぎゃくたい]の 加害[かがい] 者[しゃ]は、 家族[かぞく]、 友人[ゆうじん]あるいは 知人[ちじん]であることが 最[もっと]も 多[おお]い。
\\	懲戒処分基準によると、教職員が酒酔い運転で死亡・重傷事故を起こした場合は免職になる。	
\\	懲戒[ちょうかい] 処分[しょぶん] 基準[きじゅん]によると、 教職員[きょうしょくいん]が 酒[さけ] 酔[よ]い 運転[うんてん]で 死亡[しぼう]・ 重傷[じゅうしょう] 事故[じこ]を 起[お]こした 場合[ばあい]は 免職[めんしょく]になる。
\\	韓国人の友人によると、中国では韓国語学習者が増えているらしい。	
\\	韓国[かんこく] 人[じん]の 友人[ゆうじん]によると、 中国[ちゅうごく]では 韓国[かんこく] 語[ご] 学習[がくしゅう] 者[しゃ]が 増[ふ]えているらしい。
\\	我が社は日本国内のみならず、アメリカやヨーロッパなどでもビジネス活動を行っております。	
\\	我[わ]が 社[しゃ]は 日本[にほん] 国内[こくない]のみならず、アメリカやヨーロッパなどでもビジネス 活動[かつどう]を 行[い]っております。
\\	これからの企業は単に利益を追求するのみならず、環境に配慮した生産活動を行わなければならない。	
\\	これからの 企業[きぎょう]は 単[たん]に 利益[りえき]を 追求[ついきゅう]するのみならず、 環境[かんきょう]に 配慮[はいりょ]した 生産[せいさん] 活動[かつどう]を 行[おこな]わなければならない。
\\	彼のしたことは人間として恥ずべきことである。のみならず、それは明らかに犯罪行為だ。	
\\	彼[かれ]のしたことは 人間[にんげん]として 恥[は]ずべきことである。のみならず、それは 明[あき]らかに 犯罪[はんざい] 行為[こうい]だ。
\\	料金体系のみならず、サービス体制も改善する必要がある。	
\\	料金[りょうきん] 体系[たいけい]のみならず、サービス 体制[たいせい]も 改善[かいぜん]する 必要[ひつよう]がある。
\\	大統領の決定には国内のみならず、海外からも強い批判の声が上がっている。	
\\	大統領[だいとうりょう]の 決定[けってい]には 国内[こくない]のみならず、 海外[かいがい]からも 強[つよ]い 批判[ひはん]の 声[こえ]が 上[あ]がっている。
\\	この会社は炭酸飲料メーカーとして世界的に有名であるのみならず、アメリカ合衆国を象徴する存在の一つでさえある。	
\\	この 会社[かいしゃ]は 炭酸[たんさん] 飲料[いんりょう]メーカーとして 世界[せかい] 的[てき]に 有名[ゆうめい]であるのみならず、 アメリカ合衆国[あめりかがっしゅうこく]を 象徴[しょうちょう]する 存在[そんざい]の 一[ひと]つでさえある。
\\	この種の犯罪は、老後の生活資金を根こそぎ奪うなど、市民への経済的な影響が甚大であるのみならず、精神的打撃も多大である。	
\\	この 種[しゅ]の 犯罪[はんざい]は、 老後[ろうご]の 生活[せいかつ] 資金[しきん]を 根[ね]こそぎ 奪[うば]うなど、 市民[しみん]への 経済[けいざい] 的[てき]な 影響[えいきょう]が 甚大[じんだい]であるのみならず、 精神[せいしん] 的[てき] 打撃[だげき]も 多大[ただい]である。
\\	技術者は単に技術の進歩の推進者であるのみならず、人類・社会に及ぼす技術の影響についても強い責任感を持つ自律的な行動者であるべきである。	
\\	技術[ぎじゅつ] 者[しゃ]は 単[たん]に 技術[ぎじゅつ]の 進歩[しんぽ]の 推進[すいしん] 者[しゃ]であるのみならず、 人類[じんるい]・ 社会[しゃかい]に 及[およ]ぼす 技術[ぎじゅつ]の 影響[えいきょう]についても 強[つよ]い 責任[せきにん] 感[かん]を 持[も]つ 自律[じりつ] 的[てき]な 行動[こうどう] 者[しゃ]であるべきである。
\\	不正軽油の製造・販売・使用は、極めれ悪質な脱税行為である。のみならず、ディーゼル車の排気ガスは大気中の有害物質を増加させるなど、環境汚染の原因にもなっている。	
\\	不正[ふせい] 軽油[けいゆ]の 製造[せいぞう]・ 販売[はんばい]・ 使用[しよう]は、 極[きわ]めれ 悪質[あくしつ]な 脱税[だつぜい] 行為[こうい]である。のみならず、ディーゼル 車[しゃ]の 排気[はいき]ガスは 大気[たいき] 中[ちゅう]の 有害[ゆうがい] 物質[ぶっしつ]を 増加[ぞうか]させるなど、 環境[かんきょう] 汚染[おせん]の 原因[げんいん]にもなっている。
\\	私はすっかり疲れていた。肩や頸の凝るのはもちろん、不眠症もかなり甚しかった。のみならず偶々眠ったと思うと、いろいろの夢を見勝ちだった。(芥川龍之介『夢』)	
\\	私[わたし]はすっかり 疲[つか]れていた。 肩[かた]や頸[くび] の 凝[こ]るのはもちろん、 不眠症[ふみんしょう]もかなり 甚[はなはだ]しかった。のみならず偶々[たまたま] 眠[ねむ]ったと 思[おも]うと、いろいろの 夢[ゆめ]を 見[み] 勝[が]ちだった。
\\	芥川[あくたがわ] 龍之介[りゅうのすけ]
\\	夢[ゆめ]』)
\\	今すぐ謝罪すべきであるのみならず、現職を辞すべきである。	
\\	今[いま]すぐ 謝罪[しゃざい]すべきであるのみならず、 現職[げんしょく]を 辞[じ]すべきである。
\\	甘いのなんのって私は一口食べて全部残してしまいました。	
\\	甘[あま]いのなんのって 私[わたし]は 一口[ひとくち] 食[た]べて 全部[ぜんぶ] 残[のこ]してしまいました。
\\	飲んだのなんのって彼らは5本の酒を全部空けてしまった。	
\\	飲[の]んだのなんのって 彼[かれ]らは 
\\	本[ほん]の 酒[さけ]を 全部[ぜんぶ] 空[あ]けてしまった。
\\	かゆいのなんのって一晩中寝られなかった。	
\\	かゆいのなんのって 一晩[ひとばん] 中[じゅう] 寝[ね]られなかった。
\\	重いのなんのって二人でも持ち上がりませんでした。	
\\	重[おも]いのなんのって 二人[ふたり]でも 持ち上[もちあ]がりませんでした。
\\	海の醍醐味を味わうなら、ホエールウォッチングだ。実際に見るクジラは、大きいのなんのって、すごい迫力だよ。	
\\	海[うみ]の 醍醐味[だいごみ]を 味[あじ]わうなら、ホエールウォッチングだ。 実際[じっさい]に 見[み]るクジラは、 大[おお]きいのなんのって、すごい 迫力[はくりょく]だよ。
\\	もう少しでつれたのに糸が切れて魚は海の中。結構大きかったので悔しかったのなんのって!	
\\	もう 少[すこ]しでつれたのに 糸[いと]が 切[き]れて 魚[さかな]は 海[うみ]の 中[なか]。 結構[けっこう] 大[おお]きかったので 悔[くや]しかったのなんのって!
\\	ー試合はどうだった? ーいやもう、ひどいのなんのって、話したくもないよ。	
\\	ー 試合[しあい]はどうだった? ーいやもう、ひどいのなんのって、 話[はな]したくもないよ。
\\	不便なのなんのって一番近くのスーパーに行くのに1時間もかかるんです。	
\\	不便[ふべん]なのなんのって 一番[いちばん] 近[ちか]くのスーパーに 行[い]くのに 
\\	時間[じかん]もかかるんです。
\\	朝日岳の山々はちょうど見事に色づいていてきれいなのなんのって。思わず時間を忘れて、しばらく見入ってしまいました。	
\\	朝日[あさひ] 岳[だけ]の 山々[やまやま]はちょうど 見事[みごと]に 色[いろ]づいていてきれいなのなんのって。 思[おも]わず 時間[じかん]を 忘[わす]れて、しばらく 見入[みい]ってしまいました。
\\	驚いたのなんのって、しばらくは言葉も出なかった。	
\\	驚[おどろ]いたのなんのって、しばらくは 言葉[ことば]も 出[で]なかった。
\\	この望遠鏡、よく見るのなんのって土星の輪まではっきり見えたよ。	
\\	この 望遠鏡[ぼうえんきょう]、よく 見[み]るのなんのって 土星[どせい]の 輪[わ]まではっきり 見[み]えたよ。
\\	読めないのなんのって、ひらがなもまともに読めないんです。	
\\	読[よ]めないのなんのって、ひらがなもまともに 読[よ]めないんです。
\\	ウエイターの感じが悪い上に、出テきた料理がこれまたおいしくないのなんのって。	
\\	ウエイターの 感[かん]じが 悪[わる]い 上[うえ]に、 出[で]テきた 料理[りょうり]がこれまたおいしくないのなんのって。
\\	人は照美のことを冷たいの冷たくないのっていろいろ言うが、私は彼女が温かい人であることを知っている。	
\\	人[ひと]は 照美[てるみ]のことを 冷[つめ]たいの 冷[つめ]たくないのっていろいろ 言[い]うが、 私[わたし]は 彼女[かのじょ]が 温[あたた]かい 人[ひと]であることを 知[し]っている。
\\	アンディのピアノは下手なんてもんじゃない。	
\\	アンディのピアノは 下手[へた]なんてもんじゃない。
\\	彼は飲むなどというものではない。	
\\	彼[かれ]は 飲[の]むなどというものではない。
\\	私は彼の常識のなさにあきれている。	
\\	私[わたし]は 彼[かれ]の 常識[じょうしき]のなさにあきれている。
\\	跳びたいと思っても跳べない。体力、筋力のなさを痛感しました。	
\\	跳[と]びたいと 思[おも]っても 跳[と]べない。 体力[たいりょく]、 筋力[きんりょく]のなさを 痛感[つうかん]しました。
\\	今回の一連の貴社の対応は、私たちには全く納得できるものではなく、貴社の誠意のなさに大変な憤りを感じます。	
\\	今回[こんかい]の 一連[いちれん]の 貴社[きしゃ]の 対応[たいおう]は、 私[わたし]たちには 全[まった]く 納得[なっとく]できるものではなく、 貴社[きしゃ]の 誠意[せいい]のなさに 大変[たいへん]な 憤[いきどお]りを 感[かん]じます。
\\	辞書の使用を許可してもらったものの、時間不足で思ったように答えられない問題があり、自分の英語力のなさに落ち込みました。	
\\	辞書[じしょ]の 使用[しよう]を 許可[きょか]してもらったものの、 時間[じかん] 不足[ぶそく]で 思[おも]ったように 答[こた]えられない 問題[もんだい]があり、 自分[じぶん]の 英語[えいご] 力[りょく]のなさに 落ち込[おちこ]みました。
\\	親しみやすさと、温かさと、飾り気のなさが、さやの絶対的な魅力である。	
\\	親[した]しみやすさと、 温[あたた]かさと、 飾り気[かざりけ]のなさが、さやの 絶対[ぜったい] 的[てき]な 魅力[みりょく]である。
\\	最近の映画やドラマは、その内容のなさにがっかりさせられる。	
\\	最近[さいきん]の 映画[えいが]やドラマは、その 内容[ないよう]のなさにがっかりさせられる。
\\	調査によると日本の子供の自信のなさは世界一だそうである。	
\\	調査[ちょうさ]によると 日本[にほん]の 子供[こども]の 自信[じしん]のなさは 世界一[せかいいち]だそうである。
\\	このマンガは設定は面白かったが、物語が進むにつれて話が大きくなり、リアリティのなさについていけなくなってしまった。	
\\	このマンガは 設定[せってい]は 面白[おもしろ]かったが、 物語[ものがたり]が 進[すす]むにつれて 話[はなし]が 大[おお]きくなり、リアリティのなさについていけなくなってしまった。
\\	やったのやらなかったのといつまで言っていても仕方がない。	
\\	やったのやらなかったのといつまで 言[い]っていても 仕方[しかた]がない。
\\	いつまでも好きだの嫌いだのと言っていないで、早く結婚相手を決めなさい。	
\\	いつまでも 好[す]きだの 嫌[きら]いだのと 言[い]っていないで、 早[はや]く 結婚[けっこん] 相手[あいて]を 決[き]めなさい。
\\	彼女はアパートの部屋が狭いの場所が不便だのと文句ばかり言っている。	
\\	彼女[かのじょ]はアパートの 部屋[へや]が 狭[せま]いの 場所[ばしょ]が 不便[ふべん]だのと 文句[もんく]ばかり 言[い]っている。
\\	彼女は一緒に行くの行かぬのといつまでも煮え切らない。	
\\	彼女[かのじょ]は 一緒[いっしょ]に 行[い]くの 行[い]かぬのといつまでも 煮[に]え 切[き]らない。
\\	彼らは、誰がそのことをしゃべったのしゃべらなかったのと言い争っている。	
\\	彼[かれ]らは、 誰[だれ]がそのことをしゃべったのしゃべらなかったのと 言い争[いいあらそ]っている。
\\	条件がいいの悪いのといつまで言っていても仕方がない。	
\\	条件[じょうけん]がいいの 悪[わる]いのといつまで 言[い]っていても 仕方[しかた]がない。
\\	彼女はやっと就職できたのに、やれ場所が遠いの給料が悪いのと文句ばかり言っている。	
\\	彼女[かのじょ]はやっと 就職[しゅうしょく]できたのに、やれ 場所[ばしょ]が 遠[とお]いの 給料[きゅうりょう]が 悪[わる]いのと 文句[もんく]ばかり 言[い]っている。
\\	コーチにフォームが悪いの力が入りすぎているのとさんざん直された。	
\\	コーチにフォームが 悪[わる]いの 力[ちから]が 入[はい]りすぎているのとさんざん 直[なお]された。
\\	真理は婚約者の給料が安いの背が低いのと勝手なことを言っている。	
\\	真理[まり]は 婚約[こんやく] 者[しゃ]の 給料[きゅうりょう]が 安[やす]いの 背[せ]が 低[ひく]いのと 勝手[かって]なことを 言[い]っている。
\\	隆志は給料がどうの待遇がどうのと不平が多い。	
\\	隆志[たかし]は 給料[きゅうりょう]がどうの 待遇[たいぐう]がどうのと 不平[ふへい]が 多[おお]い。
\\	頼子は夫がどうしたの子供がこうしたのと文句を言わない日はない。	
\\	頼子[よりこ]は 夫[おっと]がどうしたの 子供[こども]がこうしたのと 文句[もんく]を 言[い]わない 日[ひ]はない。
\\	私はバッハとかモーツアルトとかが好きだ。	
\\	私[わたし]はバッハとかモーツアルトとかが 好[す]きだ。
\\	教授の言わんとすることが学生に伝わらないのは、たいてい教授の方が悪いのだ。	
\\	教授[きょうじゅ]の 言[い]わんとすることが 学生[がくせい]に 伝[つた]わらないのは、たいてい 教授[きょうじゅ]の 方[ほう]が 悪[わる]いのだ。
\\	太陽はまさに山の稜線に沈まんとしていた。	
\\	太陽[たいよう]はまさに 山[やま]の 稜線[りょうせん]に 沈[しず]まんとしていた。
\\	あの政治家は言わんとしているところだよく分からない。	
\\	あの 政治[せいじ] 家[か]は 言[い]わんとしているところだよく 分[わ]からない。
\\	善意で何事かを行わんとすると、必ずそれに水を差す人間がいる。	
\\	善意[ぜんい]で 何事[なにごと]かを 行[おこな]わんとすると、 必[かなら]ずそれに 水[みず]を 差[さ]す 人間[にんげん]がいる。
\\	専門家でない者を排除せんとする科学者の過剰防衛には辟易させられる。	
\\	専門[せんもん] 家[か]でない 者[もの]を 排除[はいじょ]せんとする 科学[かがく] 者[しゃ]の 過剰[かじょう] 防衛[ぼうえい]には 辟易[へきえき]させられる。
\\	死なんとする者は本音を吐くものだ。	
\\	死[し]なんとする 者[もの]は 本音[ほんね]を 吐[は]くものだ。
\\	暴風のために木々が倒れんとしている。	
\\	暴風[ぼうふう]のために 木々[きぎ]が 倒[たお]れんとしている。
\\	まさに輝かんとする月が雲で隠れた。	
\\	まさに 輝[かがや]かんとする 月[つき]が 雲[くも]で 隠[かく]れた。
\\	今にも崩れんとする天守閣には、城と命運をともにする城主の姿があった。	
\\	今[いま]にも 崩[くず]れんとする 天守閣[てんしゅかく]には、 城[しろ]と 命運[めいうん]をともにする 城主[じょうしゅ]の 姿[すがた]があった。
\\	東京を中心に大雨が降った。	
\\	東京[とうきょう]を 中心[ちゅうしん]に 大雨[おおあめ]が 降[ふ]った。
\\	東南アジアを中心に鳥インフルエンザが急速に広がっている。	
\\	東南アジア[とうなんあじあ]を 中心[ちゅうしん]に 鳥[とり]インフルエンザが 急速[きゅうそく]に 広[ひろ]がっている。
\\	この研究は中島教授を中心に進められている。	
\\	この 研究[けんきゅう]は 中島[なかじま] 教授[きょうじゅ]を 中心[ちゅうしん]に 進[すす]められている。
\\	最近は小説よりもビジネス書を中心に読んでいる。	
\\	最近[さいきん]は 小説[しょうせつ]よりもビジネス 書[しょ]を 中心[ちゅうしん]に 読[よ]んでいる。
\\	ファーストフードの
\\	者は近々神戸を中心に新たに25店舗を聞く予定だ。	
\\	ファーストフードの 
\\	者[しゃ]は 近々[ちかぢか] 神戸[こうべ]を 中心[ちゅうしん]に 新[あら]たに 
\\	店舗[てんぽ]を 聞[き]く 予定[よてい]だ。
\\	現在、文部科学省を中心に、義務教育改革に向けた検討が進められている。	
\\	現在[げんざい]、 文部[もんぶ] 科学[かがく] 省[しょう]を 中心[ちゅうしん]に、 義務[ぎむ] 教育[きょういく] 改革[かいかく]に 向[む]けた 検討[けんとう]が 進[すす]められている。
\\	最近の調査で、アメリカでのオンラインビデオの視聴が若い層を中心に伸びていることが分かった。	
\\	最近[さいきん]の 調査[ちょうさ]で、アメリカでのオンラインビデオの 視聴[しちょう]が 若[わか]い 層[そう]を 中心[ちゅうしん]に 伸[の]びていることが 分[わ]かった。
\\	コンピュータメーカーの
\\	者はデスクトップモデルを中心に20
\\	以上の大幅値下げを行うと発表した。	
\\	コンピュータメーカーの 
\\	者[しゃ]はデスクトップモデルを 中心[ちゅうしん]に 
\\	以上[いじょう]の 大幅[おおはば] 値下[ねさ]げを 行[おこな]うと 発表[はっぴょう]した。
\\	このサイトは熊本のニュースを中心に、温泉やグルメ、イベント情報などもご紹介します。	
\\	このサイトは 熊本[くまもと]のニュースを 中心[ちゅうしん]に、 温泉[おんせん]やグルメ、イベント 情報[じょうほう]などもご 紹介[しょうかい]します。
\\	本プロジェクトでは、言語による情報伝達と、韻律や動作、表情など、言語以外の要素との関係を中心に研究を行った。	
\\	本[ほん]プロジェクトでは、 言語[げんご]による 情報[じょうほう] 伝達[でんたつ]と、 韻律[いんりつ]や 動作[どうさ]、 表情[ひょうじょう]など、 言語[げんご] 以外[いがい]の 要素[ようそ]との 関係[かんけい]を 中心[ちゅうしん]に 研究[けんきゅう]を 行[い]った。
\\	この展示会はある著名な日本の建築家の作品を中心に、東京に建てられたモダンな建築について紹介している。	
\\	この 展示[てんじ] 会[かい]はある 著名[ちょめい]な 日本[にほん]の 建築[けんちく] 家[か]の 作品[さくひん]を 中心[ちゅうしん]に、 東京[とうきょう]に 建[た]てられたモダンな 建築[けんちく]について 紹介[しょうかい]している。
\\	ここでは会話を中心とした授業が行われている。	
\\	ここでは 会話[かいわ]を 中心[ちゅうしん]とした 授業[じゅぎょう]が 行[おこな]われている。
\\	この辺りはトウモロコシ栽培を中心にした農業が盛んだ。	
\\	この 辺[あた]りはトウモロコシ 栽培[さいばい]を 中心[ちゅうしん]にした 農業[のうぎょう]が 盛[さか]んだ。
\\	今、現地の代理人を介して製造元と価格の交渉をしている。	
\\	今[いま]、 現地[げんち]の 代理人[だいりにん]を 介[かい]して 製造元[せいぞうもと]と 価格[かかく]の 交渉[こうしょう]をしている。
\\	インターネットを介して音楽や映画を配信できるようになった。	
\\	インターネットを 介[かい]して 音楽[おんがく]や 映画[えいが]を 配信[はいしん]できるようになった。
\\	我々はボランティア活動を介して地域の交流を図ろうとしている。	
\\	我々[われわれ]はボランティア 活動[かつどう]を 介[かい]して 地域[ちいき]の 交流[こうりゅう]を 図[はか]ろうとしている。
\\	通訳を介しての商談は効率が悪い。	
\\	通訳[つうやく]を 介[かい]しての 商談[しょうだん]は 効率[こうりつ]が 悪[わる]い。
\\	日本にいる友達を介してこの情報を手に入れた。	
\\	日本[にほん]にいる 友達[ともだち]を 介[かい]してこの 情報[じょうほう]を 手[て]に 入[い]れた。
\\	このウイルスはそれに感染して鳥から蚊を介して人に感染する。人から人や人から蚊を介しての感染はない。	
\\	このウイルスはそれに 感染[かんせん]して 鳥[とり]から 蚊[か]を 介[かい]して 人[ひと]に 感染[かんせん]する。 人[ひと]から 人[ひと]や 人[ひと]から 蚊[か]を 介[かい]しての 感染[かんせん]はない。
\\	自殺系サイトを介して知り合った男女が、初対面でいきなり一緒に自殺するという事件が続き、社会問題となった。	
\\	自殺[じさつ] 系[けい]サイトを 介[かい]して 知り合[しりあ]った 男女[だんじょ]が、 初対面[しょたいめん]でいきなり 一緒[いっしょ]に 自殺[じさつ]するという 事件[じけん]が 続[つづ]き、 社会[しゃかい] 問題[もんだい]となった。
\\	このテロ組織は海外の金融機関を介して資金の調達と仕送りをしているらしい。	
\\	このテロ 組織[そしき]は 海外[かいがい]の 金融[きんゆう] 機関[きかん]を 介[かい]して 資金[しきん]の 調達[ちょうたつ]と 仕送[しおく]りをしているらしい。
\\	記憶は再生時に脳の「海馬」という部分を介して再構成される。	
\\	記憶[きおく]は 再生[さいせい] 時[じ]に 脳[のう]の
\\	海馬[かいば]」という 部分[ぶぶん]を 介[かい]して 再[さい] 構成[こうせい]される。
\\	新しいビジネスを生むためのアイデアは、勤務外のインフォーマルなコミュニケーションを介して、より多く得られると言われる。	
\\	新[あたら]しいビジネスを 生[う]むためのアイデアは、 勤務[きんむ] 外[がい]のインフォーマルなコミュニケーションを 介[かい]して、より 多[おお]く 得[え]られると 言[い]われる。
\\	このグループは、同じような障害で悩んでいる人々が、お互いの交流を介して問題をよりよく理解し、助け合っていけるよう、様々な支援を行っている。	
\\	このグループは、 同[おな]じような 障害[しょうがい]で 悩[なや]んでいる 人々[ひとびと]が、お 互[たが]いの 交流[こうりゅう]を 介[かい]して 問題[もんだい]をよりよく 理解[りかい]し、 助け合[たすけあ]っていけるよう、 様々[さまざま]な 支援[しえん]を 行[い]っている。
\\	代理人を介しての交渉は失敗に終わった。	
\\	代理人[だいりにん]を 介[かい]しての 交渉[こうしょう]は 失敗[しっぱい]に 終[お]わった。
\\	これからはインターネットを介した音楽配信が主流になるだろう。	
\\	これからはインターネットを 介[かい]した 音楽[おんがく] 配信[はいしん]が 主流[しゅりゅう]になるだろう。
\\	壁を通して隣の部屋の話が聞こえてくる。	
\\	壁[かべ]を 通[とお]して 隣[となり]の 部屋[へや]の 話[はなし]が 聞[き]こえてくる。
\\	カーテンを通して家の中の様子が見えた。	
\\	カーテンを 通[とお]して 家[いえ]の 中[なか]の 様子[ようす]が 見[み]えた。
\\	この辺りは四季を通して気候が温暖だ。	
\\	この 辺[あた]りは 四季[しき]を 通[とお]して 気候[きこう]が 温暖[おんだん]だ。
\\	このような無責任な行いに対して怒りを禁じ得ない。	
\\	このような 無責任[むせきにん]な 行[おこな]いに 対[たい]して 怒[いか]りを 禁[きん]じ 得[え]ない。
\\	その感動的な光景を見て、涙がこみ上げてくるのを禁じ得なかった。	
\\	その 感動[かんどう] 的[てき]な 光景[こうけい]を 見[み]て、 涙[なみだ]がこみ 上[あ]げてくるのを 禁[きん]じ 得[え]なかった。
\\	24年の苦悩と空白の後、ついに実現した家族との再会に、周囲の者も涙を禁じ得なかった。	
\\	年[ねん]の 苦悩[くのう]と 空白[くうはく]の 後[のち]、ついに 実現[じつげん]した 家族[かぞく]との 再会[さいかい]に、 周囲[しゅうい]の 者[もの]も 涙[なみだ]を 禁[きん]じ 得[え]なかった。
\\	その美しい早春の風景画に作者の温かい郷土愛を感じ、深い感動を禁じ得なかった。	
\\	その 美[うつく]しい 早春[そうしゅん]の 風景[ふうけい] 画[が]に 作者[さくしゃ]の 温[あたた]かい 郷土[きょうど] 愛[あい]を 感[かん]じ、 深[ふか]い 感動[かんどう]を 禁[きん]じ 得[え]なかった。
\\	出版社による無神経な個人写真の扱い方に、驚きと憤りを禁じ得ない。	
\\	出版[しゅっぱん] 社[しゃ]による 無神経[むしんけい]な 個人[こじん] 写真[しゃしん]の 扱[あつか]い 方[かた]に、 驚[おどろ]きと 憤[いきどお]りを 禁[きん]じ 得[え]ない。
\\	大統領の声明は、せっかく第一歩を踏み出した両国の平和統一の動きに水を差すのではないかという疑問を禁じ得ない。	
\\	大統領[だいとうりょう]の 声明[せいめい]は、せっかく 第一歩[だいいっぽ]を 踏み出[ふみだ]した 両国[りょうこく]の 平和[へいわ] 統一[とういつ]の 動[うご]きに 水[みず]を 差[さ]すのではないかという 疑問[ぎもん]を 禁[きん]じ 得[え]ない。
\\	長い間憧れていた人に会う瞬間が近づいて、次第に動悸が速まるのを禁じ得なかった。	
\\	長[なが]い 間[あいだ] 憧[あこが]れていた 人[ひと]に 会[あ]う 瞬間[しゅんかん]が 近[ちか]づいて、 次第[しだい]に 動悸[どうき]が 速[はや]まるのを 禁[きん]じ 得[え]なかった。
\\	この業界の救いようのない保守性に、深いため息が出るのを禁じ得なかった。	
\\	この 業界[ぎょうかい]の 救[すく]いようのない 保守[ほしゅ] 性[せい]に、 深[ふか]いため 息[いき]が 出[で]るのを 禁[きん]じ 得[え]なかった。
\\	モノレール建設をめぐって市民の意見が対立している。	
\\	モノレール 建設[けんせつ]をめぐって 市民[しみん]の 意見[いけん]が 対立[たいりつ]している。
\\	無駄な公共事業をめぐる批判の声が強まっている。	
\\	無駄[むだ]な 公共[こうきょう] 事業[じぎょう]をめぐる 批判[ひはん]の 声[こえ]が 強[つよ]まっている。
\\	大学運営は今後どうあるべきかをめぐって活発に意見が交わされた。	
\\	大学[だいがく] 運営[うんえい]は 今後[こんご]どうあるべきかをめぐって 活発[かっぱつ]に 意見[いけん]が 交[か]わされた。
\\	進攻か撤退かをめぐって激しい議論が戦わされた。	
\\	進攻[しんこう]か 撤退[てったい]かをめぐって 激[はげ]しい 議論[ぎろん]が 戦[たたか]わされた。
\\	新制度に即時切り替えるべきか来年度まで待つべきかをめぐって委員の間で意見が分かれた。	
\\	新[しん] 制度[せいど]に 即時[そくじ] 切り替[きりか]えるべきか 来年度[らいねんど]まで 待[ま]つべきかをめぐって 委員[いいん]の 間[あいだ]で 意見[いけん]が 分[わ]かれた。
\\	新しい歴史教科書をめぐって激しい議論が続いている。	
\\	新[あたら]しい 歴史[れきし] 教科書[きょうかしょ]をめぐって 激[はげ]しい 議論[ぎろん]が 続[つづ]いている。
\\	審判の判定をめぐって選手が乱闘し試合が約1時間中断した。	
\\	審判[しんぱん]の 判定[はんてい]をめぐって 選手[せんしゅ]が 乱闘[らんとう]し 試合[しあい]が 約[やく] 
\\	時間[じかん] 中断[ちゅうだん]した。
\\	自衛隊派遣の是非をめぐって激しい議論が交わされた。	
\\	自衛隊[じえいたい] 派遣[はけん]の 是非[ぜひ]をめぐって 激[はげ]しい 議論[ぎろん]が 交[か]わされた。
\\	老化のメカニズムをめぐる研究が注目を浴びている。	
\\	老化[ろうか]のメカニズムをめぐる 研究[けんきゅう]が 注目[ちゅうもく]を 浴[あ]びている。
\\	我が県でも
\\	をめぐる動きが加速しそうです。	
\\	我[わ]が 県[けん]でも 
\\	をめぐる 動[うご]きが 加速[かそく]しそうです。
\\	本章では、原子力発電をめぐる最近の動向を紹介すると共に、将来の有力なエネルギー源について概観する。	
\\	本章[ほんしょう]では、 原子力[げんしりょく] 発電[はつでん]をめぐる 最近[さいきん]の 動向[どうこう]を 紹介[しょうかい]すると 共[とも]に、 将来[しょうらい]の 有力[ゆうりょく]なエネルギー 源[げん]について 概観[がいかん]する。
\\	他の国でもいつ軍隊を引き上げるべきかをめぐって議論が戦わされている。	
\\	他[た]の 国[くに]でもいつ 軍隊[ぐんたい]を 引き上[ひきあ]げるべきかをめぐって 議論[ぎろん]が 戦[たたか]わされている。
\\	深刻化する少年犯罪が連日のように報道されている中、厳罰主義化更生重視かをめぐって社会の対応が揺れている。	
\\	深刻[しんこく] 化[か]する 少年[しょうねん] 犯罪[はんざい]が 連日[れんじつ]のように 報道[ほうどう]されている 中[なか]、 厳罰[げんばつ] 主義[しゅぎ] 化[か] 更生[こうせい] 重視[じゅうし]かをめぐって 社会[しゃかい]の 対応[たいおう]が 揺[ゆ]れている。
\\	調停案を受け入れるかこのまま交渉を続けるかをめぐって我々の間で意見が分かれている。	
\\	調停[ちょうてい] 案[あん]を 受け入[うけい]れるかこのまま 交渉[こうしょう]を 続[つづ]けるかをめぐって 我々[われわれ]の 間[あいだ]で 意見[いけん]が 分[わ]かれている。
\\	青木教授がヒトゲノムに関して論文を書いた。	
\\	青木[あおき] 教授[きょうじゅ]がヒトゲノムに 関[かん]して 論文[ろんぶん]を 書[か]いた。
\\	その政策について吉村氏が意見を述べた。	
\\	その 政策[せいさく]について 吉村[よしむら] 氏[し]が 意見[いけん]を 述[の]べた。
\\	その件につきましては、まだ報告できる段階ではありません。	
\\	その 件[けん]につきましては、まだ 報告[ほうこく]できる 段階[だんかい]ではありません。
\\	先月ハイテク農業に関する学会に出席した。	
\\	先月[せんげつ]ハイテク 農業[のうぎょう]に 関[かん]する 学会[がっかい]に 出席[しゅっせき]した。
\\	このページは陶芸についてのイベントのニュースを掲載しています。	
\\	このページは 陶芸[とうげい]についてのイベントのニュースを 掲載[けいさい]しています。
\\	来週、国際交流に関する集まりがある。	
\\	来週[らいしゅう]、 国際[こくさい] 交流[こうりゅう]に 関[かん]する 集[あつ]まりがある。
\\	思えば、大学時代はとても楽しかった。	
\\	思[おも]えば、 大学[だいがく] 時代[じだい]はとても 楽[たの]しかった。
\\	思えば、あの頃私は彼と政治問題について毎日のように議論したものだ。	
\\	思[おも]えば、あの 頃[ころ] 私[わたし]は 彼[かれ]と 政治[せいじ] 問題[もんだい]について 毎日[まいにち]のように 議論[ぎろん]したものだ。
\\	思えば、終戦後数年はほとんどの日本人が野菜を庭で育てていた。	
\\	思[おも]えば、 終戦[しゅうせん] 後[ご] 数[すう] 年[ねん]はほとんどの 日本人[にほんじん]が 野菜[やさい]を 庭[にわ]で 育[そだ]てていた。
\\	今思えばどうしてこんな
\\	を買ったのだろう。	
\\	今[いま] 思[おも]えばどうしてこんな 
\\	を 買[か]ったのだろう。
\\	思えば心筋梗塞の前兆はあった。	
\\	思[おも]えば 心筋梗塞[しんきんこうそく]の 前兆[ぜんちょう]はあった。
\\	小学校3年生の時、祖母と銀座で映画を見た。今にして思えば、あれが祖母と見た最初で最後の映画だった。	
\\	小学校[しょうがっこう] 
\\	年生[ねんせい]の 時[とき]、 祖母[そぼ]と 銀座[ぎんざ]で 映画[えいが]を 見[み]た。 今[いま]にして 思[おも]えば、あれが 祖母[そぼ]と 見[み]た 最初[さいしょ]で 最後[さいご]の 映画[えいが]だった。
\\	今から思えば、私が同僚の男性と結婚すると言った時、父は猛然と反対した。あの時父の忠告を聞いておけばと悔やまれる。	
\\	今[いま]から 思[おも]えば、 私[わたし]が 同僚[どうりょう]の 男性[だんせい]と 結婚[けっこん]すると 言[い]った 時[とき]、 父[ちち]は 猛然[もうぜん]と 反対[はんたい]した。あの 時[とき] 父[ちち]の 忠告[ちゅうこく]を 聞[き]いておけばと 悔[く]やまれる。
\\	自転車で出かけようとした途端、チェーンが外れてしまった。今から思えば、あれがケチのつき始めだった。	
\\	自転車[じてんしゃ]で 出[で]かけようとした 途端[とたん]、チェーンが 外[はず]れてしまった。 今[いま]から 思[おも]えば、あれがケチのつき 始[はじ]めだった。
\\	彼女は重度の身体障害をものともせず、自分の人生をつかみった。	
\\	彼女[かのじょ]は 重度[じゅうど]の 身体[しんたい] 障害[しょうがい]をものともせず、 自分[じぶん]の 人生[じんせい]をつかみった。
\\	彼は日本語がしゃべれないのをものともせず、身振り手振りで日本を旅行して歩いた。	
\\	彼[かれ]は 日本語[にほんご]がしゃべれないのをものともせず、 身振[みぶ]り 手振[てぶ]りで 日本[にほん]を 旅行[りょこう]して 歩[ある]いた。
\\	友人は高波をものともせず4キロを泳ぎ切った。	
\\	友人[ゆうじん]は 高波[たかなみ]をものともせず 4キロを 泳[およ]ぎ 切[き]った。
\\	子供は時差をものともせず、早速水着に着替えてワイキキビーチへ出かけた。	
\\	子供[こども]は 時差[じさ]をものともせず、 早速[さっそく] 水着[みずぎ]に 着替[きが]えてワイキキビーチへ 出[で]かけた。
\\	高層ビル火災の際、消防士たちは危険をものともせずその建物の階段を駆け登っていた。	
\\	高層[こうそう]ビル 火災[かさい]の 際[さい]、 消防[しょうぼう] 士[し]たちは 危険[きけん]をものともせずその 建物[たてもの]の 階段[かいだん]を 駆[か]け 登[のぼ]っていた。
\\	民衆は軍事政権の弾圧をものともせず、デモ行進を続けた。	
\\	民衆[みんしゅう]は 軍事[ぐんじ] 政権[せいけん]の 弾圧[だんあつ]をものともせず、デモ 行進[こうしん]を 続[つづ]けた。
\\	彼女は周囲の人の批判をものともせず、自分の信じる道を進んだ。	
\\	彼女[かのじょ]は 周囲[しゅうい]の 人[ひと]の 批判[ひはん]をものともせず、 自分[じぶん]の 信[しん]じる 道[みち]を 進[すす]んだ。
\\	高校生たちは雨が激しく降っているのをものともせず、熱心に野球の練習をしていた。	
\\	高校生[こうこうせい]たちは 雨[あめ]が 激[はげ]しく 降[ふ]っているのをものともせず、 熱心[ねっしん]に 野球[やきゅう]の 練習[れんしゅう]をしていた。
\\	美佐子は片道所要時間が2時間を越えるのをものともせず、東京の証券会社に通勤した。	
\\	美佐子[みさこ]は 片道[かたみち] 所要[しょよう] 時間[じかん]が 
\\	時間[じかん]を 越[こ]えるのをものともせず、 東京[とうきょう]の 証券[しょうけん] 会社[がいしゃ]に 通勤[つうきん]した。
\\	懸命の努力にも関わらず健一は大学入試に失敗した。	
\\	懸命[けんめい]の 努力[どりょく]にも 関[かか]わらず 健一[けんいち]は 大学[だいがく] 入試[にゅうし]に 失敗[しっぱい]した。
\\	あの人はよく運動するのにも関わらず太っている。	
\\	あの 人[ひと]はよく 運動[うんどう]するのにも 関[かか]わらず 太[ふと]っている。
\\	彼は東大に入学きたにも関わらずすぐやめてしまった。	
\\	彼[かれ]は 東大[とうだい]に 入学[にゅうがく]きたにも 関[かか]わらずすぐやめてしまった。
\\	私が思うに、文化は宗教の一種だ。	
\\	私[わたし]が 思[おも]うに、 文化[ぶんか]は 宗教[しゅうきょう]の 一種[いっしゅ]だ。
\\	医学界の体質を思うに、当然医者と製薬会社の癒着はあると思う。	
\\	医学[いがく] 界[かい]の 体質[たいしつ]を 思[おも]うに、 当然[とうぜん] 医者[いしゃ]と 製薬[せいやく] 会社[かいしゃ]の 癒着[ゆちゃく]はあると 思[おも]う。
\\	思うに、情報を発信するマスコミと受信する我々国民は、完全に分化され固定化されてしまっているのが現状だ。	
\\	思[おも]うに、 情報[じょうほう]を 発信[はっしん]するマスコミと 受信[じゅしん]する 我々[われわれ] 国民[こくみん]は、 完全[かんぜん]に 分化[ぶんか]され 固定[こてい] 化[か]されてしまっているのが 現状[げんじょう]だ。
\\	「思うに、快楽に耽る人生ほど快楽から遠いものはない。」―ジョン・
\\	・ロックフェラー2世	
\\	思[おも]うに、 快楽[かいらく]に 耽[ふけ]る 人生[じんせい]ほど 快楽[かいらく]から 遠[とお]いものはない。」―ジョン・ 
\\	・ロックフェラー 
\\	世[せい]
\\	私が思うに、大学時代の部活はお薦めだ。きっと一生の宝物になるだろう。	
\\	私[わたし]が 思[おも]うに、 大学[だいがく] 時代[じだい]の 部活[ぶかつ]はお 薦[すす]めだ。きっと 一生[いっしょう]の 宝物[たからもの]になるだろう。
\\	思うに、優れた画家であっても、生命感あふれる作品を生み出せるのはほんの一時期ではないだろうか。	
\\	思[おも]うに、 優[すぐ]れた 画家[がか]であっても、 生命[せいめい] 感[かん]あふれる 作品[さくひん]を 生み出[うみだ]せるのはほんの 一時期[いちじき]ではないだろうか。
\\	思うに、「人を使う」ことが上手な人間というのは魅力のない人間である。	
\\	思[おも]うに、
\\	人[ひと]を 使[つか]う」ことが 上手[じょうず]な 人間[にんげん]というのは 魅力[みりょく]のない 人間[にんげん]である。
\\	自分の来し方を思うに、常に幅広い好奇心を持ち続けてきたことが、研究を続ける原動力ではなかっただろうか。	
\\	自分[じぶん]の 来し方[こしかた]を 思[おも]うに、 常[つね]に 幅広[はばひろ]い 好奇[こうき] 心[しん]を 持[も]ち 続[つづ]けてきたことが、 研究[けんきゅう]を 続[つづ]ける 原動力[げんどうりょく]ではなかっただろうか。
\\	大地震の被災者の人たちの状況を思うに、何もできずにいることが本当につらいです。	
\\	大[おお] 地震[じしん]の 被災[ひさい] 者[しゃ]の 人[ひと]たちの 状況[じょうきょう]を 思[おも]うに、 何[なに]もできずにいることが 本当[ほんとう]につらいです。
\\	つらつら思うに、カラオケは平気で歌える人と絶対歌えない人にはっきり分かれるものらしい。	
\\	つらつら 思[おも]うに、カラオケは 平気[へいき]で 歌[うた]える 人[ひと]と 絶対[ぜったい] 歌[うた]えない 人[ひと]にはっきり 分[わ]かれるものらしい。
\\	現代人は物質的には満ち足りているが精神的には貧しいとよく言われる。つらつら思うに、想像するということがなくなったからではなかろうか。	
\\	現代[げんだい] 人[じん]は 物質[ぶっしつ] 的[てき]には 満ち足[みちた]りているが 精神[せいしん] 的[てき]には 貧[まず]しいとよく 言[い]われる。つらつら 思[おも]うに、 想像[そうぞう]するということがなくなったからではなかろうか。
\\	文章から想像するに、著者は哲学に造詣が深いに違いない。	
\\	文章[ぶんしょう]から 想像[そうぞう]するに、 著者[ちょしゃ]は 哲学[てつがく]に 造詣[ぞうけい]が 深[ふか]いに 違[ちが]いない。
\\	私が考えるに、人間にとって自然が究極の教師だ。	
\\	私[わたし]が 考[かんが]えるに、 人間[にんげん]にとって 自然[しぜん]が 究極[きゅうきょく]の 教師[きょうし]だ。
\\	どうして日本人はアニメが好きなのか。思うに、日本人は空想的な映像化が好きなのも一つの理由ではないだろうか。	
\\	どうして 日本人[にほんじん]はアニメが 好[す]きなのか。 思[おも]うに、 日本人[にほんじん]は 空想[くうそう] 的[てき]な 映像[えいぞう] 化[か]が 好[す]きなのも 一[ひと]つの 理由[りゆう]ではないだろうか。
\\	日本語を短期間で習得したいのならこの学校をおいてほかにはない。	
\\	日本語[にほんご]を 短期間[たんきかん]で 習得[しゅうとく]したいのならこの 学校[がっこう]をおいてほかにはない。
\\	これほど多くの作品を手がけた監督を
\\	氏をおいてほかにいない。	
\\	これほど 多[おお]くの 作品[さくひん]を 手[て]がけた 監督[かんとく]を 
\\	氏[し]をおいてほかにいない。
\\	あなたを救うのはあなた自身をおいてほかにない。	
\\	あなたを 救[すく]うのはあなた 自身[じしん]をおいてほかにない。
\\	これほどのプロフェッショナリズムと豊かな経験を提供できる会社は、日本中で当社をおいてほかにはありません。	
\\	これほどのプロフェッショナリズムと 豊[ゆた]かな 経験[けいけん]を 提供[ていきょう]できる 会社[かいしゃ]は、 日本[にほん] 中[ちゅう]で 当社[とうしゃ]をおいてほかにはありません。
\\	契約に対してこれだけ無神経な国はこの国をおいてほかに思いつかない。	
\\	契約[けいやく]に 対[たい]してこれだけ 無神経[むしんけい]な 国[くに]はこの 国[くに]をおいてほかに 思[おも]いつかない。
\\	この役割をこなせるのはジェーンをおいてほかに考えられない。	
\\	この 役割[やくわり]をこなせるのはジェーンをおいてほかに 考[かんが]えられない。
\\	味のよさと形の美しさ、この二つの要件を満たす桃は、山梨県産をおいてほかにない。	
\\	味[あじ]のよさと 形[かたち]の 美[うつく]しさ、この 二[ふた]つの 要件[ようけん]を 満[み]たす 桃[もも]は、 山梨[やまなし] 県[けん] 産[さん]をおいてほかにない。
\\	非母国語話者の間のコミュニケーションにおいてこれほど広範囲に用いられている言語は、英語をおいてほかには存在しない。	
\\	非[ひ] 母国[ぼこく] 語[ご] 話者[わしゃ]の 間[あいだ]のコミュニケーションにおいてこれほど 広範囲[こうはんい]に 用[もち]いられている 言語[げんご]は、 英語[えいご]をおいてほかには 存在[そんざい]しない。
\\	日本語を短期間で習得したいのならこの学校が一番だ。	
\\	日本語[にほんご]を 短期間[たんきかん]で 習得[しゅうとく]したいのならこの 学校[がっこう]が 一番[いちばん]だ。
\\	最も多くの作品を手がけた監督は
\\	氏だ。	
\\	最[もっと]も 多[おお]くの 作品[さくひん]を 手[て]がけた 監督[かんとく]は 
\\	氏[し]だ。
\\	非母国語話者の間のコミュニケーションにおいて一番広範囲に用いられている言語は英語だ。	
\\	非[ひ] 母国[ぼこく] 語[ご] 話者[わしゃ]の 間[あいだ]のコミュニケーションにおいて 一番[いちばん] 広範囲[こうはんい]に 用[もち]いられている 言語[げんご]は 英語[えいご]だ。
\\	最も上手にこの役割をこなせるのはジェーンだ。	
\\	最[もっと]も 上手[じょうず]にこの 役割[やくわり]をこなせるのはジェーンだ。
\\	上京の折にはぜひお立ち寄り下さい。	
\\	上京[じょうきょう]の 折[おり]にはぜひお 立ち寄[たちよ]り 下[くだ]さい。
\\	この前日本に帰った折高校時代の友達に会った。	
\\	この 前[まえ] 日本[にほん]に 帰[かえ]った 折[おり] 高校[こうこう] 時代[じだい]の 友達[ともだち]に 会[あ]った。
\\	援助が必要な折には早めに連絡して下さい。	
\\	援助[えんじょ]が 必要[ひつよう]な 折[おり]には 早[はや]めに 連絡[れんらく]して 下[くだ]さい。
\\	近くにお越しの折には、ぜひ当店をご利用下さい。	
\\	近[ちか]くにお 越[こ]しの 折[おり]には、ぜひ 当店[とうてん]をご 利用[りよう] 下[くだ]さい。
\\	ロスへ出張した折に息子の様子を見るために大学の寮に寄ってみた。	
\\	ロスへ 出張[しゅっちょう]した 折[おり]に 息子[むすこ]の 様子[ようす]を 見[み]るために 大学[だいがく]の 寮[りょう]に 寄[よ]ってみた。
\\	都心に出た折に、古本屋街に足を延ばした。	
\\	都心[としん]に 出[で]た 折[おり]に、 古本屋[ふるほんや] 街[がい]に 足[あし]を 延[の]ばした。
\\	こういうスーツを持っていれば冠婚葬祭の折に着るものを心配しなくてもいい。	
\\	こういうスーツを 持[も]っていれば 冠婚葬祭[かんこんそうさい]の 折[おり]に 着[き]るものを 心配[しんぱい]しなくてもいい。
\\	卒業の折に親に買ってもらった時計をなくひてしまった。	
\\	卒業[そつぎょう]の 折[おり]に 親[おや]に 買[か]ってもらった 時計[とけい]をなくひてしまった。
\\	私は大きな借金をかかえて途方にくれていました。そんな折に彼から電話があったのです。	
\\	私[わたし]は 大[おお]きな 借金[しゃっきん]をかかえて 途方[とほう]にくれていました。そんな 折[おり]に 彼[かれ]から 電話[でんわ]があったのです。
\\	日本へ行った折に、山内さんに会うつもりだ。	
\\	日本[にほん]へ 行[い]った 折[おり]に、 山内[やまうち]さんに 会[あ]うつもりだ。
\\	日本へ行く折に、山内さんに会うつもりだ。	
\\	日本[にほん]へ 行[い]く 折[おり]に、 山内[やまうち]さんに 会[あ]うつもりだ。
\\	ニューヨークに行った折にメトロポリタン美術館に行った。	
\\	ニューヨークに 行[い]った 折[おり]にメトロポリタン 美術館[びじゅつかん]に 行[い]った。
\\	ニューヨークに行った時にハンドバッグを盗られた。	
\\	ニューヨークに 行[い]った 時[とき]にハンドバッグを 盗[と]られた。
\\	二人きりになった折にガールフレンドに自分の気持ちを打ち明けた。	
\\	二人[ふたり]きりになった 折[おり]にガールフレンドに 自分[じぶん]の 気持[きも]ちを 打ち明[うちあ]けた。
\\	二人きりになった時にガールフレンドにさんざんなじられた。	
\\	二人[ふたり]きりになった 時[とき]にガールフレンドにさんざんなじられた。
\\	毎朝顔を洗う時に、ひげも剃る。	
\\	毎[まい] 朝顔[あさかお]を 洗[あら]う 時[とき]に、ひげも 剃[そ]る。
\\	私のうちでは、夕食の時にみんなでその日にあったことを話し合う。	
\\	私[わたし]のうちでは、 夕食[ゆうしょく]の 時[とき]にみんなでその 日[ひ]にあったことを 話し合[はなしあ]う。
\\	私がアメリカで勉強いしていた時、スミスさんは日本で仕事をしていた。	
\\	円
\\	私[わたし]がアメリカで 勉強[べんきょう]いしていた 時[とき]、スミスさんは 日本[にほん]で 仕事[しごと]をしていた。
\\	上司から電話があった時、私はちょうどシャワーを浴びていた。	
\\	上司[じょうし]から 電話[でんわ]があった 時[とき]、 私[わたし]はちょうどシャワーを 浴[あ]びていた。
\\	やっと仕事が終わった時、もう11時を過ぎていた。	
\\	やっと 仕事[しごと]が 終[お]わった 時[とき]、もう 
\\	時[じ]を 過[す]ぎていた。
\\	刑事は男が横を向いた瞬間に飛びついた。	
\\	刑事[けいじ]は 男[おとこ]が 横[よこ]を 向[む]いた 瞬間[しゅんかん]に 飛[と]びついた。
\\	私はトラックが止まった時に荷台から飛び下りた。	
\\	私[わたし]はトラックが 止[と]まった 時[とき]に 荷台[にだい]から 飛び下[とびお]りた。
\\	寒さの厳しい折から、お体には十分お気をつけ下さい。	
\\	寒[さむ]さの 厳[きび]しい 折[おり]から、お 体[からだ]には 十分[じゅうぶん]お 気[き]をつけ 下[くだ]さい。
\\	国際交流の盛んな折から、外国からの来客を接待する機会が増えてきた。	
\\	国際[こくさい] 交流[こうりゅう]の 盛[さか]んな 折[おり]から、 外国[がいこく]からの 来客[らいきゃく]を 接待[せったい]する 機会[きかい]が 増[ふ]えてきた。
\\	これは本題とずれるので、また別な折に書きたい。	
\\	これは 本題[ほんだい]とずれるので、また 別[べつ]な 折[おり]に 書[か]きたい。
\\	機会を見て彼に一言注意しておこう。	
\\	機会[きかい]を 見[み]て 彼[かれ]に 一言[ひとこと] 注意[ちゅうい]しておこう。
\\	彼とはなかなか落ち着いて話をする折がない。	
\\	彼[かれ]とはなかなか 落ち着[おちつ]いて 話[はなし]をする 折[おり]がない。
\\	これはまたとない機会だ。	
\\	これはまたとない 機会[きかい]だ。
\\	スーツケースを持ち上げた際に、腰を痛めたらしい。	
\\	スーツケースを 持ち上[もちあ]げた 際[さい]に、 腰[こし]を 痛[いた]めたらしい。
\\	テストを提出する際に名前を忘れないように。	
\\	テストを 提出[ていしゅつ]する 際[さい]に 名前[なまえ]を 忘[わす]れないように。
\\	近くにお越しの節はお電話下さい。	
\\	近[ちか]くにお 越[こ]しの 節[せつ]はお 電話[でんわ] 下[くだ]さい。
\\	その節はどうもお世話になりました。	
\\	その 節[ふし]はどうもお 世話[せわ]になりました。
\\	この前の日本への帰国の折に高校時代の友達に会った。	
\\	この 前[まえ]の 日本[にほん]への 帰国[きこく]の 折[おり]に 高校[こうこう] 時代[じだい]の 友達[ともだち]に 会[あ]った。
\\	ロスへ出張の折に息子の様子を見るために大学の寮に寄ってみた。	
\\	ロスへ 出張[しゅっちょう]の 折[おり]に 息子[むすこ]の 様子[ようす]を 見[み]るために 大学[だいがく]の 寮[りょう]に 寄[よ]ってみた。
\\	東京にいらっしゃる時はお電話下さい。	
\\	東京[とうきょう]にいらっしゃる 時[とき]はお 電話[でんわ] 下[くだ]さい。
\\	台風の影響で、西日本は大雨の恐れがある。	
\\	台風[たいふう]の 影響[えいきょう]で、 西日本[にしにほん]は 大雨[おおあめ]の 恐[おそ]れがある。
\\	この戦争は第二のベトナム戦争になる恐れがある。	
\\	この 戦争[せんそう]は 第[だい]二のベトナム 戦争[せんそう]になる 恐[おそ]れがある。
\\	テロリストたちが入国した恐れがある。	
\\	テロリストたちが 入国[にゅうこく]した 恐[おそ]れがある。
\\	測定誤差が大きい恐れがある。	
\\	測定[そくてい] 誤差[ごさ]が 大[おお]きい 恐[おそ]れがある。
\\	腫瘍が悪性である恐れがある。	
\\	腫瘍[しゅよう]が 悪性[あくせい]である 恐[おそ]れがある。
\\	この地域の野生動物は絶滅の恐れがある。	
\\	この 地域[ちいき]の 野生[やせい] 動物[どうぶつ]は 絶滅[ぜつめつ]の 恐[おそ]れがある。
\\	者は自社の携帯電話のいくつかの機種に発熱の恐れがあると発表した。	
\\	者[しゃ]は 自社[じしゃ]の 携帯[けいたい] 電話[でんわ]のいくつかの 機種[きしゅ]に 発熱[はつねつ]の 恐[おそ]れがあると 発表[はっぴょう]した。
\\	この薬品が目に入ると失明の恐れがあるので、保護眼鏡を着用して下さい。	
\\	この 薬品[やくひん]が 目[め]に 入[はい]ると 失明[しつめい]の 恐[おそ]れがあるので、 保護[ほご] 眼鏡[めがね]を 着用[ちゃくよう]して 下[くだ]さい。
\\	県は近く、がけ崩れなどの恐れがある場所で地形や土地利用状況などの調査を始める。	
\\	県[けん]は 近[ちか]く、がけ 崩[くず]れなどの 恐[おそ]れがある 場所[ばしょ]で 地形[ちけい]や 土地[とち] 利用[りよう] 状況[じょうきょう]などの 調査[ちょうさ]を 始[はじ]める。
\\	添付書類を不用意に開けるとウイルスに感染する恐れがある。	
\\	添付[てんぷ] 書類[しょるい]を 不用意[ふようい]に 開[あ]けるとウイルスに 感染[かんせん]する 恐[おそ]れがある。
\\	この薬は副作用がないと言われているが、8週間以上続けて服用すると肝臓に害を及ぼす恐れがある。	
\\	この 薬[くすり]は 副作用[ふくさよう]がないと 言[い]われているが、 
\\	週間[しゅうかん] 以上[いじょう] 続[つづ]けて 服用[ふくよう]すると 肝臓[かんぞう]に 害[がい]を 及[およ]ぼす 恐[おそ]れがある。
\\	選手たちが試合の中止を知らないまま、まだ練習している恐れがある。	
\\	選手[せんしゅ]たちが 試合[しあい]の 中止[ちゅうし]を 知[し]らないまま、まだ 練習[れんしゅう]している 恐[おそ]れがある。
\\	警視庁は、殺人容疑者の
\\	が海外へ逃亡した恐れがあると発表した。	
\\	警視庁[けいしちょう]は、 殺人[さつじん] 容疑[ようぎ] 者[しゃ]の 
\\	が 海外[かいがい]へ 逃亡[とうぼう]した 恐[おそ]れがあると 発表[はっぴょう]した。
\\	ここのカフェテリアは食材が古い恐れがある。	
\\	ここのカフェテリアは 食材[しょくざい]が 古[ふる]い 恐[おそ]れがある。
\\	この薬は胎児に有害である恐れがある。	
\\	この 薬[くすり]は 胎児[たいじ]に 有害[ゆうがい]である 恐[おそ]れがある。
\\	取引相手が暴力団である恐れがある。	
\\	取引[とりひき] 相手[あいて]が 暴力団[ぼうりょくだん]である 恐[おそ]れがある。
\\	他人の特許権を侵害する恐れがある。	
\\	他人[たにん]の 特許[とっきょ] 権[けん]を 侵害[しんがい]する 恐[おそ]れがある。
\\	他人の特許権を侵害している恐れがある。	
\\	他人[たにん]の 特許[とっきょ] 権[けん]を 侵害[しんがい]している 恐[おそ]れがある。
\\	他人の特許権を侵害した恐れがある。	
\\	他人[たにん]の 特許[とっきょ] 権[けん]を 侵害[しんがい]した 恐[おそ]れがある。
\\	他人の特許権を侵害する恐れはない。	
\\	他人[たにん]の 特許[とっきょ] 権[けん]を 侵害[しんがい]する 恐[おそ]れはない。
\\	震度5以上の地震では全壊の恐れがある。	
\\	震度[しんど] 
\\	以上[いじょう]の 地震[じしん]では 全壊[ぜんかい]の 恐[おそ]れがある。
\\	この紛争は戦争に発展する危険性がある。	
\\	この 紛争[ふんそう]は 戦争[せんそう]に 発展[はってん]する 危険[きけん] 性[せい]がある。
\\	友達が道を間違う心配がある。	
\\	友達[ともだち]が 道[みち]を 間違[まちが]う 心配[しんぱい]がある。
\\	選手たちが試合の中止を知らないまま、まだ練習している心配がある。	
\\	選手[せんしゅ]たちが 試合[しあい]の 中止[ちゅうし]を 知[し]らないまま、まだ 練習[れんしゅう]している 心配[しんぱい]がある。
\\	彼が私の作ったものを食べられない心配がある。	
\\	彼[かれ]が 私[わたし]の 作[つく]ったものを 食[た]べられない 心配[しんぱい]がある。
\\	助手の作ったテストの問題が難しすぎる心配がある。	
\\	助手[じょしゅ]の 作[つく]ったテストの 問題[もんだい]が 難[むずか]しすぎる 心配[しんぱい]がある。
\\	他人の特許権を侵害する可能性がある。	
\\	他人[たにん]の 特許[とっきょ] 権[けん]を 侵害[しんがい]する 可能[かのう] 性[せい]がある。
\\	奨学金がもらえるかもしれない。	
\\	奨学[しょうがく] 金[きん]がもらえるかもしれない。
\\	この辺りは洪水の恐れがある。	
\\	この 辺[あた]りは 洪水[こうずい]の 恐[おそ]れがある。
\\	彼は精神状態がまだ不安定な可能性がある。	
\\	彼[かれ]は 精神[せいしん] 状態[じょうたい]がまだ 不安定[ふあんてい]な 可能[かのう] 性[せい]がある。
\\	彼は精神状態がまだ不安定かもしれない。	
\\	彼[かれ]は 精神[せいしん] 状態[じょうたい]がまだ 不安定[ふあんてい]かもしれない。
\\	この辺りは洪水かもしれない。	
\\	この 辺[あた]りは 洪水[こうずい]かもしれない。
\\	この辺りは洪水の可能性がある。	
\\	この 辺[あた]りは 洪水[こうずい]の 可能[かのう] 性[せい]がある。
\\	この会社は、経験があれば学歴を問わず採用してくれる。	
\\	この 会社[かいしゃ]は、 経験[けいけん]があれば 学歴[がくれき]を 問[と]わず 採用[さいよう]してくれる。
\\	卒業後就職するか進学するかを問わず、担任との個人面談を受けなければならない。	
\\	卒業[そつぎょう] 後[ご] 就職[しゅうしょく]するか 進学[しんがく]するかを 問[と]わず、 担任[たんにん]との 個人[こじん] 面談[めんだん]を 受[う]けなければならない。
\\	本ソフトウエアの複製は全部か否かを問わず違法行為になる。	
\\	本[ほん]ソフトウエアの 複製[ふくせい]は 全部[ぜんぶ]か 否[いな]かを 問[と]わず 違法[いほう] 行為[こうい]になる。
\\	不要の電化製品、中古新品を問わず買い取ります。	
\\	不要[ふよう]の 電化[でんか] 製品[せいひん]、 中古[ちゅうこ] 新品[しんぴん]を 問[と]わず 買い取[かいと]ります。
\\	当協会は、信者であるなしを問わず挙式をお引き受けいたします。	
\\	当[とう] 協会[きょうかい]は、 信者[しんじゃ]であるなしを 問[と]わず 挙式[きょしき]をお 引き受[ひきう]けいたします。
\\	経験のあるなしを問わず誰でもこのクラブに参加できます。	
\\	経験[けいけん]のあるなしを 問[と]わず 誰[だれ]でもこのクラブに 参加[さんか]できます。
\\	者は、市場調査の結果のいかんを問わず、日本でのマーケティングを開始する。	
\\	者[しゃ]は、 市場[しじょう] 調査[ちょうさ]の 結果[けっか]のいかんを 問[と]わず、 日本[にほん]でのマーケティングを 開始[かいし]する。
\\	どのような利用であるかを問わず学内の施設利用は事前に大学の承認が必要だ。	
\\	どのような 利用[りよう]であるかを 問[と]わず 学内[がくない]の 施設[しせつ] 利用[りよう]は 事前[じぜん]に 大学[だいがく]の 承認[しょうにん]が 必要[ひつよう]だ。
\\	このリゾート地は、登山、ハイキング、スキー等、四季を問わず一年中楽しめる。	
\\	このリゾート 地[ち]は、 登山[とざん]、ハイキング、スキー 等[など]、 四季[しき]を 問[と]わず 一年中[いちねんじゅう] 楽[たの]しめる。
\\	無線
\\	のおかげで、場所や時間を問わずネットへのアクセスが可能になった。	
\\	無線[むせん] 
\\	[らん] のおかげで、 場所[ばしょ]や 時間[じかん]を 問[と]わずネットへのアクセスが 可能[かのう]になった。
\\	古くなったパソコン、メーカーを問わず引き取ります。	
\\	古[ふる]くなったパソコン、メーカーを 問[と]わず 引き取[ひきと]ります。
\\	この本は、イスラム世界に興味があるか否かを問わず、ぜひ読んでおくべき一冊である。	
\\	この 本[ほん]は、イスラム 世界[せかい]に 興味[きょうみ]があるか 否[いな]かを 問[と]わず、ぜひ 読[よ]んでおくべき 一冊[いっさつ]である。
\\	選挙期間中であるか否かを問わず、選挙運動またはこれに類する行為にコンピュータ、掲示板等学内施設を利用することを禁ずる。	
\\	選挙[せんきょ] 期間[きかん] 中[ちゅう]であるか 否[いな]かを 問[と]わず、 選挙[せんきょ] 運動[うんどう]またはこれに 類[るい]する 行為[こうい]にコンピュータ、 掲示板[けいじばん] 等[など] 学内[がくない] 施設[しせつ]を 利用[りよう]することを 禁[きん]ずる。
\\	インストラクター募集。(男女を問わず)	
\\	インストラクター 募集[ぼしゅう]。
\\	男女[だんじょ]を 問[と]わず)
\\	この
\\	は映画を見た見ないを問わず、ぜひ持っておくことを勧めます。	
\\	この 
\\	は 映画[えいが]を 見[み]た 見[み]ないを 問[と]わず、ぜひ 持[も]っておくことを 勧[すす]めます。
\\	このソフトはパソコンに精通しているいないを問わず、誰でも簡単に使える。	
\\	このソフトはパソコンに 精通[せいつう]しているいないを 問[と]わず、 誰[だれ]でも 簡単[かんたん]に 使[つか]える。
\\	クリスチャンであるないを問わず、この運動に賛同して下さる方を広く募集しております。	
\\	クリスチャンであるないを 問[と]わず、この 運動[うんどう]に 賛同[さんどう]して 下[くだ]さる 方[ほう]を 広[ひろ]く 募集[ぼしゅう]しております。
\\	この競技にはプロアマを問わず参加が可能だ。	
\\	この 競技[きょうぎ]にはプロアマを 問[と]わず 参加[さんか]が 可能[かのう]だ。
\\	年内は平日、休日を問わず、全点朝9時から夜8時まで営業しております。	
\\	年内[ねんない]は 平日[へいじつ]、 休日[きゅうじつ]を 問[と]わず、 全[ぜん] 点[てん] 朝[あさ] 
\\	時[じ]から 夜[よる] 
\\	時[じ]まで 営業[えいぎょう]しております。
\\	アメリカには、自国民であるなしを問わず、優れた才能をもつ人に機会と資金を与える懐の深さがある。	
\\	アメリカには、 自[じ] 国民[こくみん]であるなしを 問[と]わず、 優[すぐ]れた 才能[さいのう]をもつ 人[ひと]に 機会[きかい]と 資金[しきん]を 与[あた]える 懐[ふところ]の 深[ふか]さがある。
\\	当行とのお取引のあるなしを問わず、国際業務に関するどのようなご質問・ご相談でもお気軽にお電話して下さい。	
\\	当行[とうこう]とのお 取引[とりひき]のあるなしを 問[と]わず、 国際[こくさい] 業務[ぎょうむ]に 関[かん]するどのようなご 質問[しつもん]・ご 相談[そうだん]でもお 気軽[きがる]にお 電話[でんわ]して 下[くだ]さい。
\\	理由のいかんを問わず、次回の会合の欠席者は本委員会のメンバーから除外する。	
\\	理由[りゆう]のいかんを 問[と]わず、 次回[じかい]の 会合[かいごう]の 欠席[けっせき] 者[しゃ]は 本[ほん] 委員[いいん] 会[かい]のメンバーから 除外[じょがい]する。
\\	いかなる国における利用であるかを問わず、本ソフトエアの利用に当たっては当社と利用合意書を交わすことが必要です。	
\\	いかなる 国[くに]における 利用[りよう]であるかを 問[と]わず、 本[ほん]ソフトエアの 利用[りよう]に 当[あ]たっては 当社[とうしゃ]と 利用[りよう] 合意[ごうい] 書[しょ]を 交[か]わすことが 必要[ひつよう]です。
\\	買い取り品は新品中古を問いません。	
\\	買い取[かいと]り 品[ひん]は 新品[しんぴん] 中古[ちゅうこ]を 問[と]いません。
\\	この奨学金は応募者の国籍を問わない。	
\\	この 奨学[しょうがく] 金[きん]は 応募[おうぼ] 者[しゃ]の 国籍[こくせき]を 問[と]わない。
\\	どんな仕事によらず、それで生活するとなると楽なものはない。	
\\	どんな 仕事[しごと]によらず、それで 生活[せいかつ]するとなると 楽[らく]なものはない。
\\	リストラによらずこの会社を救う方法はない。	
\\	リストラによらずこの 会社[かいしゃ]を 救[すく]う 方法[ほうほう]はない。
\\	使用して機種及びそふトウエアを表1に示す。	
\\	使用[しよう]して 機種[きしゅ] 及[およ]びそふトウエアを 表[ひょう] 1に 示[しめ]す。
\\	本奨学金への応募資格は、日本国籍を有すること、及び国内の四年制大学に在籍する学生であることである。	
\\	本[ほん] 奨学[しょうがく] 金[きん]への 応募[おうぼ] 資格[しかく]は、 日本[にほん] 国籍[こくせき]を 有[ゆう]すること、 及[およ]び 国内[こくない]の 四年[よねん] 制[せい] 大学[だいがく]に 在籍[ざいせき]する 学生[がくせい]であることである。
\\	室内及び廊下での喫煙は禁止されています。	
\\	室内[しつない] 及[およ]び 廊下[ろうか]での 喫煙[きつえん]は 禁止[きんし]されています。
\\	今月号では広島及び近郊他県の観光名所を紹介します。	
\\	今月[こんげつ] 号[ごう]では 広島[ひろしま] 及[およ]び 近郊[きんこう] 他[た] 県[けん]の 観光[かんこう] 名所[めいしょ]を 紹介[しょうかい]します。
\\	本店移転及び商号変更についてお知らせいたします。	
\\	本店[ほんてん] 移転[いてん] 及[およ]び 商号[しょうごう] 変更[へんこう]についてお 知[し]らせいたします。
\\	者は2002年及びそれ以前の製品のサポートの打ち切りを発表した。	
\\	者[しゃ]は 
\\	年[ねん] 及[およ]びそれ 以前[いぜん]の 製品[せいひん]のサポートの 打ち切[うちき]りを 発表[はっぴょう]した。
\\	これは昨年度における全国の一般廃棄物の排出及び処理状況に関する報告である。	
\\	これは 昨年度[さくねんど]における 全国[ぜんこく]の 一般[いっぱん] 廃棄[はいき] 物[ぶつ]の 排出[はいしゅつ] 及[およ]び 処理[しょり] 状況[じょうきょう]に 関[かん]する 報告[ほうこく]である。
\\	記者会見の前に、情報が正確であること、及びそれが公表してもいい情報であることを確認しておく必要がある。	
\\	記者[きしゃ] 会見[かいけん]の 前[まえ]に、 情報[じょうほう]が 正確[せいかく]であること、 及[およ]びそれが 公表[こうひょう]してもいい 情報[じょうほう]であることを 確認[かくにん]しておく 必要[ひつよう]がある。
\\	装置には著しい破損のないこと、及びメンテナンスが規定どおり行われていることを確認した。	
\\	装置[そうち]には 著[いちじる]しい 破損[はそん]のないこと、 及[およ]びメンテナンスが 規定[きてい]どおり 行[おこな]われていることを 確認[かくにん]した。
\\	外国人研修生を受け入れる事業者の評価機関は、5人以上の委員からなる委員会を持つこと、及びその委員の半数以上が外国人の研究について専門知識を持つことが義務づけられた。	
\\	外国[がいこく] 人[じん] 研修生[けんしゅうせい]を 受け入[うけい]れる 事業[じぎょう] 者[しゃ]の 評価[ひょうか] 機関[きかん]は、 
\\	人[にん] 以上[いじょう]の 委員[いいん]からなる 委員[いいん] 会[かい]を 持[も]つこと、 及[およ]びその 委員[いいん]の 半数[はんすう] 以上[いじょう]が 外国[がいこく] 人[じん]の 研究[けんきゅう]について 専門[せんもん] 知識[ちしき]を 持[も]つことが 義務[ぎむ]づけられた。
\\	午前9時から及び午後3時からの2回、面接を行います。	
\\	午前[ごぜん] 
\\	時[じ]から 及[およ]び 午後[ごご] 
\\	時[じ]からの 
\\	回[かい]、 面接[めんせつ]を 行[おこな]います。
\\	朝6時から9時まで、及び夕方5時から7時までの計5時間、検査のため装置の運転を停止する。	
\\	朝[あさ] 
\\	時[じ]から 
\\	時[じ]まで、 及[およ]び 夕方[ゆうがた] 
\\	時[じ]から 
\\	時[じ]までの 計[けい] 
\\	時間[じかん]、 検査[けんさ]のため 装置[そうち]の 運転[うんてん]を 停止[ていし]する。
\\	川辺博士にはがん治療について、及び患者の精神療法についての著述がある。	
\\	川辺[かわべ] 博士[はかせ]にはがん 治療[ちりょう]について、 及[およ]び 患者[かんじゃ]の 精神療法[せいしんりょうほう]についての 著述[ちょじゅつ]がある。
\\	集団として、及び個人として、今我々が直面している問題をしっかり考える必要がある。	
\\	集団[しゅうだん]として、 及[およ]び 個人[こじん]として、 今[いま] 我々[われわれ]が 直面[ちょくめん]している 問題[もんだい]をしっかり 考[かんが]える 必要[ひつよう]がある。
\\	浅野君、そして裕子さん、ご結婚おめでとうございます。	
\\	浅野[あさの] 君[くん]、そして 裕子[ゆうこ]さん、ご 結婚[けっこん]おめでとうございます。
\\	学生諸君、及び教職員の皆様に一言ご挨拶申し上げます。	
\\	学生[がくせい] 諸君[しょくん]、 及[およ]び 教職員[きょうしょくいん]の 皆様[みなさま]に 一言[ひとこと]ご 挨拶[あいさつ] 申し上[もうしあ]げます。
\\	日本へ行って、そして昔の友人たちに会った。	
\\	日本[にほん]へ 行[い]って、そして 昔[むかし]の 友人[ゆうじん]たちに 会[あ]った
\\	この本は面白くて、そして為になる。	
\\	この 本[ほん]は 面白[おもしろ]くて、そして 為[ため]になる。
\\	この辞書は便利で、そして安価だ。	
\\	この 辞書[じしょ]は 便利[べんり]で、そして 安価[あんか]だ。
\\	彼はビジネスマンであり、そして学者でもある。	
\\	彼[かれ]はビジネスマンであり、そして 学者[がくしゃ]でもある。
\\	私はおよそ名作と言われている映画はすげて見ているつもりだ。	
\\	私[わたし]はおよそ 名作[めいさく]と 言[い]われている 映画[えいが]はすげて 見[み]ているつもりだ。
\\	彼女はこういう根気のいる仕事にはおよそ向いていない。	
\\	彼女[かのじょ]はこういう 根気[こんき]のいる 仕事[しごと]にはおよそ 向[む]いていない。
\\	彼の考えていることはおよそ想像がつく。	
\\	彼[かれ]の 考[かんが]えていることはおよそ 想像[そうぞう]がつく。
\\	その学会にはおよそ1500人の会員がいる。	
\\	その 学会[がっかい]にはおよそ 
\\	人[にん]の 会員[かいいん]がいる。
\\	およそ楽をして得られる技能で身につくものはない。	
\\	およそ 楽[らく]をして 得[え]られる 技能[ぎのう]で 身[み]につくものはない。
\\	およそ天才と言われる人は誰でも人には知れない重荷を背負っているものだ。	
\\	およそ 天才[てんさい]と 言[い]われる 人[ひと]は 誰[だれ]でも 人[ひと]には 知[し]れない 重荷[おもに]を 背負[せお]っているものだ。
\\	我々はおよそ不合理なことの制度に長い間縛られている。	
\\	我々[われわれ]はおよそ 不合理[ふごうり]なことの 制度[せいど]に 長[なが]い 間[あいだ] 縛[しば]られている。
\\	田宮先生は大学の政治的なことにはおよそ縁遠い人だった。	
\\	田宮[たみや] 先生[せんせい]は 大学[だいがく]の 政治[せいじ] 的[てき]なことにはおよそ 縁遠[えんどお]い 人[ひと]だった。
\\	彼には気の毒だが、彼の努力はおよそ意味のないものだった。	
\\	彼[かれ]には 気の毒[きのどく]だが、 彼[かれ]の 努力[どりょく]はおよそ 意味[いみ]のないものだった。
\\	それは彼の普段の言動からはおよそ想像できない行動であった。	
\\	それは 彼[かれ]の 普段[ふだん]の 言動[げんどう]からはおよそ 想像[そうぞう]できない 行動[こうどう]であった。
\\	仕事は午前中でおよそ片づいたので今日は早く帰れそうだ。	
\\	仕事[しごと]は 午前[ごぜん] 中[ちゅう]でおよそ 片[かた]づいたので 今日[きょう]は 早[はや]く 帰[かえ]れそうだ。
\\	その大学のおよその情報はインターネットで得られる。	
\\	その 大学[だいがく]のおよその 情報[じょうほう]はインターネットで 得[え]られる。
\\	この会社の年間取引額はおよそどのくらいですか。	
\\	この 会社[かいしゃ]の 年間[ねんかん] 取引[とりひき] 額[がく]はおよそどのくらいですか。
\\	現在中国に生息するジャイアントパンダの数はおよそ1千頭ということだ。	
\\	現在[げんざい] 中国[ちゅうごく]に 生息[せいそく]するジャイアントパンダの 数[かず]はおよそ 1千[いっせん] 頭[とう]ということだ。
\\	全くすばらしい話だ。	
\\	全[まった]くすばらしい 話[はなし]だ。
\\	この制度は全く合理的だ。	
\\	この 制度[せいど]は 全[まった]く 合理[ごうり] 的[てき]だ。
\\	彼の言っていることは全く正しい。	
\\	彼[かれ]の 言[い]っていることは 全[まった]く 正[ただ]しい。
\\	彼女はおよそ間違ったことは言わない。	
\\	彼女[かのじょ]はおよそ 間違[まちが]ったことは 言[い]わない。
\\	私の父はおよそ不正なことのできない性格だった。	
\\	私[わたし]の 父[ちち]はおよそ 不正[ふせい]なことのできない 性格[せいかく]だった。
\\	首相は国民の批判をよそにイラク戦争支持を表明した。	
\\	首相[しゅしょう]は 国民[こくみん]の 批判[ひはん]をよそにイラク 戦争[せんそう] 支持[しじ]を 表明[ひょうめい]した。
\\	青春時代、友達がデートに明け暮れるのをよそに、僕は家に一人こもって
\\	を読んでいた。	
\\	青春[せいしゅん] 時代[じだい]、 友達[ともだち]がデートに 明け暮[あけく]れるのをよそに、 僕[ぼく]は 家[いえ]に 一人[ひとり]こもって 
\\	を 読[よ]んでいた。
\\	今回のイベントは、我々の心配をよそに多くの参加者を集め、大成功だった。	
\\	今回[こんかい]のイベントは、 我々[われわれ]の 心配[しんぱい]をよそに 多[おお]くの 参加[さんか] 者[しゃ]を 集[あつ]め、 大[だい] 成功[せいこう]だった。
\\	たび重なる妻の忠告をよそに、彼は酒もたばこもやめなかった。	
\\	たび 重[かさ]なる 妻[つま]の 忠告[ちゅうこく]をよそに、 彼[かれ]は 酒[さけ]もたばこもやめなかった。
\\	彼はその恋を自身の最後の恋と感じて、周囲の警告をよそにのめり込んでいった。	
\\	彼[かれ]はその 恋[こい]を 自身[じしん]の 最後[さいご]の 恋[こい]と 感[かん]じて、 周囲[しゅうい]の 警告[けいこく]をよそにのめり 込[こ]んでいった。
\\	2006年12月、様々な反対の声をよそに、教育基本法改正案が国会の十分な審議を経ることなく可決された。	
\\	年[ねん] 
\\	月[がつ]、 様々[さまざま]な 反対[はんたい]の 声[こえ]をよそに、 教育[きょういく] 基本[きほん] 法[ほう] 改正[かいせい] 案[あん]が 国会[こっかい]の 十分[じゅうぶん]な 審議[しんぎ]を 経[へ]ることなく 可決[かけつ]された。
\\	平和を希求する市民による核兵器廃絶への切なる願いをよそに、核兵器開発、核軍拡競争は、今なおとどまるところを知りません。	
\\	平和[へいわ]を 希求[ききゅう]する 市民[しみん]による 核兵器[かくへいき] 廃絶[はいぜつ]への 切[せつ]なる 願[ねが]いをよそに、 核兵器[かくへいき] 開発[かいはつ]、 核[かく] 軍拡[ぐんかく] 競争[きょうそう]は、 今[いま]なおとどまるところを 知[し]りません。
\\	多くの国々が経済発展ばかりに躍起になっているのをよそに、この国は個人の豊かな生活を基盤にした幸福を追求している。	
\\	多[おお]くの 国々[くにぐに]が 経済[けいざい] 発展[はってん]ばかりに 躍起[やっき]になっているのをよそに、この 国[くに]は 個人[こじん]の 豊[ゆた]かな 生活[せいかつ]を 基盤[きばん]にした 幸福[こうふく]を 追求[ついきゅう]している。
\\	浩二は冷めたコーヒーをさもまずそうに読んだ。	
\\	浩二[こうじ]は 冷[さ]めたコーヒーをさもまずそうに 読[よ]んだ。
\\	古田はさも自分一人が苦労したように言っているが、実は彼は文句を言うばかりで何もしなかったのだ。	
\\	古田[ふるた]はさも 自分[じぶん] 一人[ひとり]が 苦労[くろう]したように 言[い]っているが、 実[じつ]は 彼[かれ]は 文句[もんく]を 言[い]うばかりで 何[なに]もしなかったのだ。
\\	今すぐ病院に連れて行こう。さもないと手遅れになる。	
\\	今[いま]すぐ 病院[びょういん]に 連[つ]れて 行[い]こう。さもないと 手遅[ておく]れになる。
\\	係員はさも面倒くさそうに私の苦情を聞いていた。	
\\	係員[かかりいん]はさも 面倒[めんどう]くさそうに 私[わたし]の 苦情[くじょう]を 聞[き]いていた。
\\	彼女はさも忙しそうにしているが、本当は大した仕事なんかないのだ。	
\\	彼女[かのじょ]はさも 忙[いそが]しそうにしているが、 本当[ほんとう]は 大[たい]した 仕事[しごと]なんかないのだ。
\\	子供たちはさも退屈そうな顔をして授業を聞いている。	
\\	子供[こども]たちはさも 退屈[たいくつ]そうな 顔[かお]をして 授業[じゅぎょう]を 聞[き]いている。
\\	川村はいつもその話をさも自分自身の体験みたいに話しているが、本当はそれは私が彼に話したことなのだ。	
\\	川村[かわむら]はいつもその 話[はなし]をさも 自分[じぶん] 自身[じしん]の 体験[たいけん]みたいに 話[はな]しているが、 本当[ほんとう]はそれは 私[わたし]が 彼[かれ]に 話[はな]したことなのだ。
\\	部長はさも当然のごとく私に日曜出勤を命じた。	
\\	部長[ぶちょう]はさも 当然[とうぜん]のごとく 私[わたし]に 日曜[にちよう] 出勤[しゅっきん]を 命[めい]じた。
\\	そのプログラムは、捜査説明書を見るとさも簡単に使えそうに書いてあるが、本当にそうなのだろうか。	
\\	そのプログラムは、 捜査[そうさ] 説明[せつめい] 書[しょ]を 見[み]るとさも 簡単[かんたん]に 使[つか]えそうに 書[か]いてあるが、 本当[ほんとう]にそうなのだろうか。
\\	男はさも親しげに私に近寄ってきた。	
\\	男[おとこ]はさも 親[した]しげに 私[わたし]に 近寄[ちかよ]ってきた。
\\	あの「自分はさもワインの通ですよ」って感じの戸田の話し方はまったく嫌みだ。	
\\	あの
\\	自分[じぶん]はさもワインの 通[つう]ですよ」って 感[かん]じの 戸田[とだ]の 話し方[はなしかた]はまったく 嫌[いや]みだ。
\\	彼はさもおいしそうに食べる。	
\\	彼[かれ]はさもおいしそうに 食[た]べる。
\\	彼はさもおいしそうな食べ方をする。	
\\	彼[かれ]はさもおいしそうな 食[た]べ 方[かた]をする。
\\	彼の食べ方は本当においしそうだ。	
\\	彼[かれ]の 食[た]べ 方[かた]は 本当[ほんとう]においしそうだ。
\\	彼はさも自分一人でやったように話す。	
\\	彼[かれ]はさも 自分[じぶん] 一人[ひとり]でやったように 話[はな]す。
\\	彼はさも自分一人でやったような話し方をする。	
\\	彼[かれ]はさも 自分[じぶん] 一人[ひとり]でやったような 話し方[はなしかた]をする。
\\	彼の話し方はまるで自分一人でやったようだ。	
\\	彼[かれ]の 話し方[はなしかた]はまるで 自分[じぶん] 一人[ひとり]でやったようだ。
\\	彼はさも革新派らしく振る舞っているが、本当は相当保守的だ。	
\\	彼[かれ]はさも 革新[かくしん] 派[は]らしく 振る舞[ふるま]っているが、 本当[ほんとう]は 相当[そうとう] 保守[ほしゅ] 的[てき]だ。
\\	もっと真面目に勉強にしなさい。さもないと卒業できませんよ。	
\\	もっと 真面目[まじめ]に 勉強[べんきょう]にしなさい。さもないと 卒業[そつぎょう]できませんよ。
\\	この仕事を取るか、さもなくば日本へ帰るしかない。	
\\	この 仕事[しごと]を 取[と]るか、さもなくば 日本[にほん]へ 帰[かえ]るしかない。
\\	米田は一週間で会社を辞めたそうだが、さもあろう。彼にはとても会社勤めは向かない。	
\\	米田[よねだ]は 一週間[いっしゅうかん]で 会社[かいしゃ]を 辞[や]めたそうだが、さもあろう。 彼[かれ]にはとても 会社[かいしゃ] 勤[つと]めは 向[む]かない。
\\	このイチゴはいかにもおいしそうな色をしている。	
\\	このイチゴはいかにもおいしそうな 色[いろ]をしている。
\\	いかにも美術館らしい建物が見えてきた。	
\\	いかにも 美術館[びじゅつかん]らしい 建物[たてもの]が 見[み]えてきた。
\\	その部屋はいかにも訪問者の気持ちを和ませるような装飾が施されていた。	
\\	その 部屋[へや]はいかにも 訪問[ほうもん] 者[しゃ]の 気持[きも]ちを 和[なご]ませるような 装飾[そうしょく]が 施[ほどこ]されていた。
\\	あの二人はいかにもよく似ている。	
\\	あの 二人[ふたり]はいかにもよく 似[に]ている。
\\	あの反論の仕方はいかにも彼らしい。	
\\	あの 反論[はんろん]の 仕方[しかた]はいかにも 彼[かれ]らしい。
\\	前の翻訳も難しかったが今度のはさらに難しい。	
\\	前[まえ]の 翻訳[ほんやく]も 難[むずか]しかったが 今度[こんど]のはさらに 難[むずか]しい。
\\	グローバル時代に向けて企業の国際化をさらに強力に押し進める必要がある。	
\\	グローバル 時代[じだい]に 向[む]けて 企業[きぎょう]の 国際[こくさい] 化[か]をさらに 強力[きょうりょく]に 押し進[おしすす]める 必要[ひつよう]がある。
\\	さらに500人の人員が整理される見通した。	
\\	さらに 
\\	人[にん]の 人員[じんいん]が 整理[せいり]される 見通[みとお]した。
\\	我が国の経済はさらに悪化しそうだ。	
\\	我が国[わがくに]の 経済[けいざい]はさらに 悪化[あっか]しそうだ。
\\	さらに大きいハリケーンがテキサス南部を襲った。	
\\	さらに 大[おお]きいハリケーンがテキサス 南部[なんぶ]を 襲[おそ]った。
\\	さらに大型のジャンボジェットが開発されることになった。	
\\	さらに 大型[おおがた]のジャンボジェットが 開発[かいはつ]されることになった。
\\	者のノート型コンピュータがさらに薄くなった。	
\\	者[しゃ]のノート 型[がた]コンピュータがさらに 薄[うす]くなった。
\\	電化製品をディスカウントショップよりさらにお安く提供しております。	
\\	電化[でんか] 製品[せいひん]をディスカウントショップよりさらにお 安[やす]く 提供[ていきょう]しております。
\\	メモリがさらに10パーセント値下げされた。	
\\	メモリがさらに 10パーセント 値下[ねさ]げされた。
\\	今年は主要銀行の合併がさらに進んだ。	
\\	今年[ことし]は 主要[しゅよう] 銀行[ぎんこう]の 合併[がっぺい]がさらに 進[すす]んだ。
\\	鳥インフルエンザの被害がさらに広がる恐れがある。	
\\	鳥[とり]インフルエンザの 被害[ひがい]がさらに 広[ひろ]がる 恐[おそ]れがある。
\\	センサーでデータを集め、これをさらに同一チップ上の増幅器で増幅する。	
\\	センサーでデータを 集[あつ]め、これをさらに 同一[どういつ]チップ 上[うえ]の 増幅器[ぞうふくき]で 増幅[ぞうふく]する。
\\	開店日の来場者にはもれなく記念品を差し上げます。さらに、抽選で50名の方に当店の商品券が当たります。	
\\	開店[かいてん] 日[び]の 来場[らいじょう] 者[しゃ]にはもれなく 記念[きねん] 品[ひん]を 差し上[さしあ]げます。さらに、 抽選[ちゅうせん]で 
\\	名[めい]の 方[かた]に 当店[とうてん]の 商品[しょうひん] 券[けん]が 当[あ]たります。
\\	今度の翻訳は一層難しい。	
\\	今度[こんど]の 翻訳[ほんやく]は 一層[いっそう] 難[むずか]しい。
\\	彼は一緒にアパートを探してくれて、さらに引っ越しまで手伝ってくれた。	
\\	彼[かれ]は 一緒[いっしょ]にアパートを 探[さが]してくれて、さらに 引っ越[ひっこ]しまで 手伝[てつだ]ってくれた。
\\	さて、話は変わりますが、日本では今ちょっとした韓国ブームです。	
\\	さて、 話[はなし]は 変[か]わりますが、 日本[にほん]では 今[いま]ちょっとした 韓国[かんこく]ブームです。
\\	さて、それでは本題に入りましょう。	
\\	さて、それでは 本題[ほんだい]に 入[はい]りましょう。
\\	あて、ここで今までお話したポイントを復習してみましょう。	
\\	あて、ここで 今[いま]までお 話[はなし]したポイントを 復習[ふくしゅう]してみましょう。
\\	ワールドカップ開幕!さて、今回はどこが勝つでしょうか。	
\\	ワールドカップ 開幕[かいまく]!さて、 今回[こんかい]はどこが 勝[か]つでしょうか。
\\	さて、私はそろそろ失礼します。	
\\	さて、 私[わたし]はそろそろ 失礼[しつれい]します。
\\	さて、これからどうしようか。	
\\	さて、これからどうしようか。
\\	さて、先日お願いしました契約変更の件ですが、課内部でご検討いただけましたでしょうか。	
\\	さて、 先日[せんじつ]お 願[ねが]いしました 契約[けいやく] 変更[へんこう]の 件[けん]ですが、 課内[かない] 部[ぶ]でご 検討[けんとう]いただけましたでしょうか。
\\	あて、前回の講義の続きですが、覚えていますか。	
\\	あて、 前回[ぜんかい]の 講義[こうぎ]の 続[つづ]きですが、 覚[おぼ]えていますか。
\\	さて、次はスポーツです。	
\\	さて、 次[つぎ]はスポーツです。
\\	ここに3枚のポートレートがあります。さて、吉永さんはどれでしょう。	
\\	ここに 
\\	枚[まい]のポートレートがあります。さて、 吉永[よしなが]さんはどれでしょう。
\\	さて、もう寝ようかな。	
\\	さて、もう 寝[ね]ようかな。
\\	さて、困った。	
\\	さて、 困[こま]った。
\\	さて、何から始めようか。	
\\	さて、 何[なに]から 始[はじ]めようか。
\\	青森への単身赴任はさぞかし大変でしょう。	
\\	青森[あおもり]への 単身[たんしん] 赴任[ふにん]はさぞかし 大変[たいへん]でしょう。
\\	シカゴの冬はさぞ寒いに違いない。	
\\	シカゴの 冬[ふゆ]はさぞ 寒[さむ]いに 違[ちが]いない。
\\	誕生日を忘れてしまったんだから、アリスはさぞ怒っているだろう。	
\\	誕生[たんじょう] 日[び]を 忘[わす]れてしまったんだから、アリスはさぞ 怒[おこ]っているだろう。
\\	彼女は大学合格の知らせを聞いてさぞ喜んだだろう。	
\\	彼女[かのじょ]は 大学[だいがく] 合格[ごうかく]の 知[し]らせを 聞[き]いてさぞ 喜[よろこ]んだだろう。
\\	被災地の人々はさぞかし困っているだろうと思う。	
\\	被災[ひさい] 地[ち]の 人々[ひとびと]はさぞかし 困[こま]っているだろうと 思[おも]う。
\\	谷本先生は長旅でさぞかしお疲れだろうと思います。	
\\	谷本[たにもと] 先生[せんせい]は 長旅[ながたび]でさぞかしお 疲[つか]れだろうと 思[おも]います。
\\	もうすぐクリスマス。子供たちはさぞ楽しみなことでしょう。	
\\	もうすぐクリスマス。 子供[こども]たちはさぞ 楽[たの]しみなことでしょう。
\\	孝史はあの音が聞こえなかったのだから、さぞかしよく寝ていたに違いない。	
\\	孝史[たかし]はあの 音[おと]が 聞[き]こえなかったのだから、さぞかしよく 寝[ね]ていたに 違[ちが]いない。
\\	彼はもう勝った後のことを話しているから、さぞかし自信があるんだろうね。	
\\	彼[かれ]はもう 勝[か]った 後[あと]のことを 話[はな]しているから、さぞかし 自信[じしん]があるんだろうね。
\\	私どもの手違い続きでさぞお腹立ちのことでしょう。	
\\	私[わたし]どもの 手違[てちが]い 続[つづ]きでさぞお 腹[はら] 立[だ]ちのことでしょう。
\\	子供を失って純子はさぞ悲しんでいるだろう。	
\\	子供[こども]を 失[うしな]って 純子[じゅんこ]はさぞ 悲[かな]しんでいるだろう。
\\	青森への単身赴任はきっと大変でしょう。	
\\	青森[あおもり]への 単身[たんしん] 赴任[ふにん]はきっと 大変[たいへん]でしょう。
\\	シカゴの冬はきっと寒いに違いない。	
\\	シカゴの 冬[ふゆ]はきっと 寒[さむ]いに 違[ちが]いない。
\\	青森への単身赴任はきっと大変だ。	
\\	青森[あおもり]への 単身[たんしん] 赴任[ふにん]はきっと 大変[たいへん]だ。
\\	シカゴの冬はきっと寒い。	
\\	シカゴの 冬[ふゆ]はきっと 寒[さむ]い。
\\	誕生日を忘れてしまったんだから、アリスはきっと怒っている。	
\\	誕生[たんじょう] 日[び]を 忘[わす]れてしまったんだから、アリスはきっと 怒[おこ]っている。
\\	彼はきっと写真を撮るだろう。	
\\	彼[かれ]はきっと 写真[しゃしん]を 撮[と]るだろう。
\\	彼はさぞたくさん写真を撮るだろう。	
\\	彼[かれ]はさぞたくさん 写真[しゃしん]を 撮[と]るだろう。
\\	そもそもこんなところにスーパーを造ったことが問題だ。	
\\	そもそもこんなところにスーパーを 造[つく]ったことが 問題[もんだい]だ。
\\	そもそもどうしてこんなばかばかしい企画が通ったのでしょうか。	
\\	そもそもどうしてこんなばかばかしい 企画[きかく]が 通[とお]ったのでしょうか。
\\	今回の事件のそもそもの起こりは、一人の社員が会社のパソコンをうちに持ち帰ったことにあった。	
\\	今回[こんかい]の 事件[じけん]のそもそもの 起[お]こりは、一人[ひとり]の 社員[しゃいん]が 会社[かいしゃ]のパソコンをうちに 持ち帰[もちかえ]ったことにあった。
\\	そもそも人間は一人で生きているのではない。	
\\	そもそも 人間[にんげん]は 一人[ひとり]で 生[い]きているのではない。
\\	私はそもそも、作品のいいところをコメントしないで欠点ばかり指摘する審査のやり方に賛成できない。	
\\	私[わたし]はそもそも、 作品[さくひん]のいいところをコメントしないで 欠点[けってん]ばかり 指摘[してき]する 審査[しんさ]のやり 方[かた]に 賛成[さんせい]できない。
\\	そもそも電波って何なんでしょう。	
\\	そもそも 電波[でんぱ]って 何[なに]なんでしょう。
\\	そもそも一体何で君がこのパーティーにいるんだ?	
\\	そもそも 一体[いったい] 何[なん]で 君[きみ]がこのパーティーにいるんだ?
\\	彼に期待したことがそもそもの間違いだった。	
\\	彼[かれ]に 期待[きたい]したことがそもそもの 間違[まちが]いだった。
\\	契約書に誤訳があったのがそもそもの問題だ。	
\\	契約[けいやく] 書[しょ]に 誤訳[ごやく]があったのがそもそもの 問題[もんだい]だ。
\\	そもそもの目的は会員を増やすことだった。	
\\	そもそもの 目的[もくてき]は 会員[かいいん]を 増[ふ]やすことだった。
\\	企画そのものは悪くない。	
\\	企画[きかく]そのものは 悪[わる]くない。
\\	この国は独裁国家そのものだ。	
\\	この 国[くに]は 独裁[どくさい] 国家[こっか]そのものだ。
\\	わいせつ画像は、それそのものが問題なのではなく、それを誰が見るかが問題だと思う。	
\\	わいせつ 画像[がぞう]は、それそのものが 問題[もんだい]なのではなく、それを 誰[だれ]が 見[み]るかが 問題[もんだい]だと 思[おも]う。
\\	武田さんは今年80歳だが、健康そのものだ。	
\\	武田[たけだ]さんは 今年[ことし] 
\\	歳[さい]だが、 健康[けんこう]そのものだ。
\\	マイナス思考ばかりしていると、人生そのものが駄目になる。	
\\	マイナス 思考[しこう]ばかりしていると、 人生[じんせい]そのものが 駄目[だめ]になる。
\\	私はこの団体の存在そのものを疑っている。	
\\	私[わたし]はこの 団体[だんたい]の 存在[そんざい]そのものを 疑[うたが]っている。
\\	我々は外務省改革にとどまらず、現在の外交そのものの転換を望む。	
\\	我々[われわれ]は 外務省[がいむしょう] 改革[かいかく]にとどまらず、 現在[げんざい]の 外交[がいこう]そのものの 転換[てんかん]を 望[のぞ]む。
\\	人間はその人の考えそのものである。	
\\	人間[にんげん]はその 人[ひと]の 考[かんが]えそのものである。
\\	私にとってバレーボールは人生そのものでした。	
\\	私[わたし]にとってバレーボールは 人生[じんせい]そのものでした。
\\	土地は、それそのものは富を生み出さないが、その上で生産、販売などの経済活動を行うことで富を生み出す。	
\\	土地[とち]は、それそのものは 富[とみ]を 生み出[うみだ]さないが、その 上[うえ]で 生産[せいさん]、 販売[はんばい]などの 経済[けいざい] 活動[かつどう]を 行[おこな]うことで 富[とみ]を 生み出[うみだ]す。
\\	お年寄りを狙い、強引な方法で家屋の修理契約を結ぶ手口は卑劣そのものだ。	
\\	お 年寄[としよ]りを 狙[ねら]い、 強引[ごういん]な 方法[ほうほう]で 家屋[かおく]の 修理[しゅうり] 契約[けいやく]を 結[むす]ぶ 手口[てぐち]は 卑劣[ひれつ]そのものだ。
\\	そのマニュアルは説明も図も明解そのもので、画像処理の知識がない者にもよく分かる。	
\\	そのマニュアルは 説明[せつめい]も 図[ず]も 明解[めいかい]そのもので、 画像[がぞう] 処理[しょり]の 知識[ちしき]がない 者[もの]にもよく 分[わ]かる。
\\	米倉さんは誠実そのものの人柄で知られている。	
\\	米倉[よねくら]さんは 誠実[せいじつ]そのものの 人柄[ひとがら]で 知[し]られている。
\\	私自身はこの合併案に特に問題を感じない。	
\\	私[わたし] 自身[じしん]はこの 合併[がっぺい] 案[あん]に 特[とく]に 問題[もんだい]を 感[かん]じない。
\\	田中さん自身、自分の選択が間違っていたかもしれないと思い始めている。	
\\	田中[たなか]さん 自身[じしん]、 自分[じぶん]の 選択[せんたく]が 間違[まちが]っていたかもしれないと 思[おも]い 始[はじ]めている。
\\	待遇をよくすることも必要だが、技術者そのものを増やすことが急務だ。	
\\	待遇[たいぐう]をよくすることも 必要[ひつよう]だが、 技術[ぎじゅつ] 者[しゃ]そのものを 増[ふ]やすことが 急務[きゅうむ]だ。
\\	海外では、日本人の作った製品は評価されるが、日本人そのものはあまり評価されていないような印象を受ける。	
\\	海外[かいがい]では、 日本人[にほんじん]の 作[つく]った 製品[せいひん]は 評価[ひょうか]されるが、 日本人[にほんじん]そのものはあまり 評価[ひょうか]されていないような 印象[いんしょう]を 受[う]ける。
\\	このトイレは不潔そのものだ。	
\\	このトイレは 不潔[ふけつ]そのものだ。
\\	彼らは貧困そのものだった。	
\\	彼[かれ]らは 貧困[ひんこん]そのものだった。
\\	この仕事は苦労も多いが、それだけやりがいもある。	
\\	この 仕事[しごと]は 苦労[くろう]も 多[おお]いが、それだけやりがいもある。
\\	彼らは同じ宗教内の派閥だから、それだけ憎しみも大きいのかもしれない。	
\\	彼[かれ]らは 同[おな]じ 宗教[しゅうきょう] 内[ない]の 派閥[はばつ]だから、それだけ 憎[にく]しみも 大[おお]きいのかもしれない。
\\	期待が大きいとそれだけ失望も大きくなってしまう。	
\\	期待[きたい]が 大[おお]きいとそれだけ 失望[しつぼう]も 大[おお]きくなってしまう。
\\	大きな仕事をまかされたということは、それだけ期待されているということだろう。	
\\	大[おお]きな 仕事[しごと]をまかされたということは、それだけ 期待[きたい]されているということだろう。
\\	株式投資は当たった時の利益は大きいが、それだけリスクも大きい。	
\\	株式[かぶしき] 投資[とうし]は 当[あ]たった 時[とき]の 利益[りえき]は 大[おお]きいが、それだけリスクも 大[おお]きい。
\\	たくさんのオプションがあるのはいいのだが、それだけ選択に悩むことにもなる。	
\\	たくさんのオプションがあるのはいいのだが、それだけ 選択[せんたく]に 悩[なや]むことにもなる。
\\	花粉症の薬は、眠くなるからそれだけ効くというわけではありません。	
\\	花粉[かふん] 症[しょう]の 薬[くすり]は、 眠[ねむ]くなるからそれだけ 効[き]くというわけではありません。
\\	できるだけ多く条件を指定すると、それだけ理想に近い人が検索されるはずです。	
\\	できるだけ 多[おお]く 条件[じょうけん]を 指定[してい]すると、それだけ 理想[りそう]に 近[ちか]い 人[ひと]が 検索[けんさく]されるはずです。
\\	情報の内容が二つのソースで一致するのなら、それだけ信用できるということではないだろうか。	
\\	情報[じょうほう]の 内容[ないよう]が 二[ふた]つのソースで 一致[いっち]するのなら、それだけ 信用[しんよう]できるということではないだろうか。
\\	子供にお金をかければそれだけ愛情を注いだことになる、という間違った思い込みがしばしばあるようだ。	
\\	子供[こども]にお 金[かね]をかければそれだけ 愛情[あいじょう]を 注[そそ]いだことになる、という 間違[まちが]った 思い込[おもいこ]みがしばしばあるようだ。
\\	これに関する読者からの投書が最近増えているということは、それだけこの問題が注目されてきたということでしょう。	
\\	これに 関[かん]する 読者[どくしゃ]からの 投書[とうしょ]が 最近[さいきん] 増[ふ]えているということは、それだけこの 問題[もんだい]が 注目[ちゅうもく]されてきたということでしょう。
\\	販売経験が豊富な営業マンは、それだけ顧客の信頼も多く勝ち取るはずだ。	
\\	販売[はんばい] 経験[けいけん]が 豊富[ほうふ]な 営業[えいぎょう]マンは、それだけ 顧客[こきゃく]の 信頼[しんらい]も 多[おお]く 勝ち取[かちと]るはずだ。
\\	クラシックもジャズもロックも、聞いてみればみなそれなりに面白い。	
\\	クラシックもジャズもロックも、 聞[き]いてみればみなそれなりに 面白[おもしろ]い。
\\	外国で仕事をするのは面白いが、それなりの苦労は覚悟しなければならない。	
\\	外国[がいこく]で 仕事[しごと]をするのは 面白[おもしろ]いが、それなりの 苦労[くろう]は 覚悟[かくご]しなければならない。
\\	買った参考書はどれもみなそれなりに役に立っている。	
\\	買[か]った 参考[さんこう] 書[しょ]はどれもみなそれなりに 役に立[やくにた]っている。
\\	選手たちはみんなそれなりに精一杯やっている。ただ結果が出ないだけだ。	
\\	選手[せんしゅ]たちはみんなそれなりに 精一杯[せいいっぱい]やっている。ただ 結果[けっか]が 出[で]ないだけだ。
\\	天才と言われる人にはそれなりの苦しみがあると思う。	
\\	天才[てんさい]と 言[い]われる 人[ひと]にはそれなりの 苦[くる]しみがあると 思[おも]う。
\\	お金がなくてもそれなりに余暇を楽しむ方法はある。	
\\	お 金[かね]がなくてもそれなりに 余暇[よか]を 楽[たの]しむ 方法[ほうほう]はある。
\\	ほとんどの人は、悩みや問題があってもそれなりに暮らしているのだと思います。	
\\	ほとんどの 人[ひと]は、 悩[なや]みや 問題[もんだい]があってもそれなりに 暮[く]らしているのだと 思[おも]います。
\\	上の地位にいればそれなりの責任がかかってくる。	
\\	上[うえ]の 地位[ちい]にいればそれなりの 責任[せきにん]がかかってくる。
\\	きつい仕事にはそれなりの報酬を払うべきだ。	
\\	きつい 仕事[しごと]にはそれなりの 報酬[ほうしゅう]を 払[はら]うべきだ。
\\	今年それなりの成績を上げることができれば、来年支店長に昇進できそうだ。	
\\	今年[ことし]それなりの 成績[せいせき]を 上[あ]げることができれば、 来年[らいねん] 支店[してん] 長[ちょう]に 昇進[しょうしん]できそうだ。
\\	それなりの収入があるのなら、結婚相手はすぐ見つかるはずです。	
\\	それなりの 収入[しゅうにゅう]があるのなら、 結婚[けっこん] 相手[あいて]はすぐ 見[み]つかるはずです。
\\	彼が準決勝まで残ったということは、それなりの実力があるということだ。	
\\	彼[かれ]が 準決勝[じゅんけっしょう]まで 残[のこ]ったということは、それなりの 実力[じつりょく]があるということだ。
\\	彼女がみんなに好かれていないのはそれなりの理由がある。	
\\	彼女[かのじょ]がみんなに 好[す]かれていないのはそれなりの 理由[りゆう]がある。
\\	彼はそれなりに努力しているのだ。	
\\	彼[かれ]はそれなりに 努力[どりょく]しているのだ。
\\	洋子には洋子なりの考えかあるようだ。	
\\	洋子[ようこ]には 洋子[ようこ]なりの 考[かんが]えかあるようだ。
\\	抽象画はそれなりに面白い。	
\\	抽象[ちゅうしょう] 画[が]はそれなりに 面白[おもしろ]い。
\\	試行錯誤の末に、やっと車に使える燃料電池ができた。	
\\	試行錯誤[しこうさくご]の 末[すえ]に、やっと 車[くるま]に 使[つか]える 燃料[ねんりょう] 電池[でんち]ができた。
\\	一週間悩んだ末、家族をおいて単身赴任することにした。	
\\	一週間[いっしゅうかん] 悩[なや]んだ 末[すえ]、 家族[かぞく]をおいて 単身[たんしん] 赴任[ふにん]することにした。
\\	両社は8時間にわたる交渉の末に、ついに合意に達した。	
\\	両社[りょうしゃ]は 
\\	時間[じかん]にわたる 交渉[こうしょう]の 末[すえ]に、ついに 合意[ごうい]に 達[たっ]した。
\\	5時間を越える熱戦の末、タイガースが勝った。	
\\	時間[じかん]を 越[こ]える 熱戦[ねっせん]の 末[すえ]、タイガースが 勝[か]った。
\\	さんざん迷った末に、友達の結婚祝いはコーヒーカップのセットを贈ることにした。	
\\	さんざん 迷[まよ]った 末[すえ]に、 友達[ともだち]の 結婚[けっこん] 祝[いわ]いはコーヒーカップのセットを 贈[おく]ることにした。
\\	銀行と
\\	銀行の合併交渉は1年以上も難航した末、やっと成立した。	
\\	銀行[ぎんこう]と 
\\	銀行[ぎんこう]の 合併[がっぺい] 交渉[こうしょう]は 
\\	年[ねん] 以上[いじょう]も 難航[なんこう]した 末[すえ]、やっと 成立[せいりつ]した。
\\	下書きを20枚近く失敗した末に、やっとなんとか気に入った絵ができた。	
\\	下書[したが]きを 
\\	枚[まい] 近[ちか]く 失敗[しっぱい]した 末[すえ]に、やっとなんとか 気に入[きにい]った 絵[え]ができた。
\\	再就職は、苦労した末、過去のコンピュータ経験を評価してくれた
\\	関係の会社に決まりました。	
\\	再[さい] 就職[しゅうしょく]は、 苦労[くろう]した 末[すえ]、 過去[かこ]のコンピュータ 経験[けいけん]を 評価[ひょうか]してくれた 
\\	関係[かんけい]の 会社[かいしゃ]に 決[き]まりました。
\\	男は走って逃げ、最後には暴れ回って抵抗した末に逮捕された。	
\\	男[おとこ]は 走[はし]って 逃[に]げ、 最後[さいご]には 暴[あば]れ 回[まわ]って 抵抗[ていこう]した 末[すえ]に 逮捕[たいほ]された。
\\	この案は、各国の意見が衝突した末の「妥協の産物」だった。	
\\	この 案[あん]は、 各国[かっこく]の 意見[いけん]が 衝突[しょうとつ]した 末[すえ]の
\\	妥協[だきょう]の 産物[さんぶつ]」だった。
\\	私は宿題をした後で友達と食事に出かけた。	
\\	私[わたし]は 宿題[しゅくだい]をした 後[あと]で 友達[ともだち]と 食事[しょくじ]に 出[で]かけた。
\\	家を出てから、今日は図書館が休みであることを思い出した。	
\\	家[いえ]を 出[で]てから、 今日[きょう]は 図書館[としょかん]が 休[やす]みであることを 思い出[おもいだ]した。
\\	少し考えてからパーティーには行かないことにした。	
\\	少[すこ]し 考[かんが]えてからパーティーには 行[い]かないことにした。
\\	ご注文の品は5月の末に入ります。	
\\	ご 注文[ちゅうもん]の 品[しな]は 
\\	月[がつ]の 末[すえ]に 入[はい]ります。
\\	来週の末に郊外に引っ越します。	
\\	来週[らいしゅう]の 末[すえ]に 郊外[こうがい]に 引っ越[ひっこ]します。
\\	去年の末に子供が生まれた。	
\\	去年[きょねん]の 末[すえ]に 子供[こども]が 生[う]まれた。
\\	ジミーとは今度の連休の終わりに会う予定だ。	
\\	ジミーとは 今度[こんど]の 連休[れんきゅう]の 終[お]わりに 会[あ]う 予定[よてい]だ。
\\	面接は20分で済んだ。	
\\	面接[めんせつ]は 
\\	分[ふん]で 済[す]んだ。
\\	あまり難しい日本語ではなかったので、辞書なしで済んだ。	
\\	あまり 難[むずか]しい 日本語[にほんご]ではなかったので、 辞書[じしょ]なしで 済[す]んだ。
\\	旅行先ではどこでもクレジットカードが使えたので、現金がなくても済んだ。	
\\	旅行[りょこう] 先[さき]ではどこでもクレジットカードが 使[つか]えたので、 現金[げんきん]がなくても 済[す]んだ。
\\	これは黙っていて済む問題ではない。	
\\	これは 黙[だま]っていて 済[す]む 問題[もんだい]ではない。
\\	ちょっと顔を見せるだけで済むと思う。	
\\	ちょっと 顔[かお]を 見[み]せるだけで 済[す]むと 思[おも]う。
\\	奨学金をもらったので、お金を借りずに済むと思う。	
\\	奨学[しょうがく] 金[きん]をもらったので、お 金[かね]を 借[か]りずに 済[す]むと 思[おも]う。
\\	解約したい時は、入会金だけ損をすれば済む。	
\\	解約[かいやく]したい 時[とき]は、 入会[にゅうかい] 金[きん]だけ 損[そん]をすれば 済[す]む。
\\	この程度の翻訳なら3日で済むだろう。	
\\	この 程度[ていど]の 翻訳[ほんやく]なら 3日[みっか]で 済[す]むだろう。
\\	この方法だと今のシステムの大部分がそのまま使えるので、新たな投資が最小で澄む。	
\\	この 方法[ほうほう]だと 今[いま]のシステムの 大[だい] 部分[ぶぶん]がそのまま 使[つか]えるので、 新[あら]たな 投資[とうし]が 最小[さいしょう]で 澄[す]む。
\\	徒歩あるいは自転車で済むところへ車で行く無駄をしていませんか。	
\\	徒歩[とほ]あるいは 自転車[じてんしゃ]で 済[す]むところへ 車[くるま]で 行[い]く 無駄[むだ]をしていませんか。
\\	15キロオーバーだったが、幸い警告だけで済んだ。	
\\	15キロオーバーだったが、 幸[さいわ]い 警告[けいこく]だけで 済[す]んだ。
\\	こんな大きな失敗をして「ごめん」「知らなかった」では済まない。	
\\	こんな 大[おお]きな 失敗[しっぱい]をして「ごめん」
\\	知[し]らなかった」では 済[す]まない。
\\	電動自転車はモーターの力だけで動くことはないから、法律上は「自転車」とされ、運転免許もヘルメットもなしで済む。	
\\	電動[でんどう] 自転車[じてんしゃ]はモーターの 力[ちから]だけで 動[うご]くことはないから、 法律[ほうりつ] 上[じょう]は
\\	自転車[じてんしゃ]」とされ、 運転[うんてん] 免許[めんきょ]もヘルメットもなしで 済[す]む。
\\	ここからは地下鉄でどこへでも行けるから車がなくても済む。	
\\	ここからは 地下鉄[ちかてつ]でどこへでも 行[い]けるから 車[くるま]がなくても 済[す]む。
\\	謝って済むことと済まないことがある。	
\\	謝[あやま]って 済[す]むことと 済[す]まないことがある。
\\	ヘルメットをかぶっていたので、怪我が軽くて済んだ。	
\\	ヘルメットをかぶっていたので、 怪我[けが]が 軽[かる]くて 済[す]んだ。
\\	このソフトは画面上の指示に従って捜査するだけで済むので、コンピュータに弱い人でも簡単に使える。	
\\	このソフトは 画面[がめん] 上[じょう]の 指示[しじ]に 従[したが]って 捜査[そうさ]するだけで 済[す]むので、コンピュータに 弱[よわ]い 人[ひと]でも 簡単[かんたん]に 使[つか]える。
\\	途中で雨が降り出したが、幸い傘を持っていたのでぬれずに済んだ。	
\\	途中[とちゅう]で 雨[あめ]が 降り出[ふりだ]したが、 幸[さいわ]い 傘[かさ]を 持[も]っていたのでぬれずに 済[す]んだ。
\\	もっと早くこのことを知っていればこんなに悩まなくて済んだのに。	
\\	もっと 早[はや]くこのことを 知[し]っていればこんなに 悩[なや]まなくて 済[す]んだのに。
\\	この電子掲示板は自分のメールアドレスを公開しないで済むので安心だ。	
\\	この 電子[でんし] 掲示板[けいじばん]は 自分[じぶん]のメールアドレスを 公開[こうかい]しないで 済[す]むので 安心[あんしん]だ。
\\	彼が一言謝ればそれで済むことだ。	
\\	彼[かれ]が 一言[ひとこと] 謝[あやま]ればそれで 済[す]むことだ。
\\	これは単にソフトウエアを書き換えたら済むという問題じゃない。	
\\	これは 単[たん]にソフトウエアを 書き換[かきか]えたら 済[す]むという 問題[もんだい]じゃない。
\\	ジョンはサンドイッチで昼食を済ませた。	
\\	ジョンはサンドイッチで 昼食[ちゅうしょく]を 済[す]ませた。
\\	私は後任者への挨拶を電話で済ませた。	
\\	私[わたし]は 後任[こうにん] 者[しゃ]への 挨拶[あいさつ]を 電話[でんわ]で 済[す]ませた。
\\	彼女は1時間でその仕事を済ませた。	
\\	彼女[かのじょ]は 
\\	時間[じかん]でその 仕事[しごと]を 済[す]ませた。
\\	私は今回の引っ越しを20万円で済ませた。	
\\	私[わたし]は 今回[こんかい]の 引っ越[ひっこ]しを 
\\	万[まん] 円[えん]で 済[す]ませた。
\\	とてもいいコンピュータだと思います。ただ値段が問題ですね。	
\\	とてもいいコンピュータだと 思[おも]います。ただ 値段[ねだん]が 問題[もんだい]ですね。
\\	彼はいい人だが、ただちょっと考えが浅いところが気になる。	
\\	彼[かれ]はいい 人[ひと]だが、ただちょっと 考[かんが]えが 浅[あさ]いところが 気[き]になる。
\\	この故障は直せないことはないと思います。ただ時間がかかると思います。	
\\	この 故障[こしょう]は 直[なお]せないことはないと 思[おも]います。ただ 時間[じかん]がかかると 思[おも]います。
\\	とても静かでいいうちだ。ただ近くに店がないので買い物は不便だ。	
\\	とても 静[しず]かでいいうちだ。ただ 近[ちか]くに 店[みせ]がないので 買い物[かいもの]は 不便[ふべん]だ。
\\	この辺りは駅にもスーパーにも近くて便利だ。ただ環境があまりよくない。	
\\	この 辺[あた]りは 駅[えき]にもスーパーにも 近[ちか]くて 便利[べんり]だ。ただ 環境[かんきょう]があまりよくない。
\\	このデジカメは優れた機能をたくさん持っている。ただユーザーインターフェースが悪くて使いにくいのが難点だ。	
\\	このデジカメは 優[すぐ]れた 機能[きのう]をたくさん 持[も]っている。ただユーザーインターフェースが 悪[わる]くて 使[つか]いにくいのが 難点[なんてん]だ。
\\	たいていのことは私がいなくても他の者が代わりにやってくれるが、ただこの仕事は他の者に頼むわけには行かない。	
\\	たいていのことは 私[わたし]がいなくても 他[ほか]の 者[もの]が 代[か]わりにやってくれるが、ただこの 仕事[しごと]は 他[ほか]の 者[もの]に 頼[たの]むわけには 行[い]かない。
\\	これが彼のただ一つの欠点だ。	
\\	これが 彼[かれ]のただ 一[ひと]つの 欠点[けってん]だ。
\\	彼女はただの一日も休んだことがない。	
\\	彼女[かのじょ]はただの一 日[にち]も 休[やす]んだことがない。
\\	私はただ言われたことをやっているだけだ。	
\\	私[わたし]はただ 言[い]われたことをやっているだけだ。
\\	ーこんなにしていただいてどうもありがとう。 ーいや、ただこれは人には言わないでほしいんだ。	
\\	ーこんなにしていただいてどうもありがとう。 ーいや、ただこれは 人[ひと]には 言[い]わないでほしいんだ。
\\	これを待ちまして第5回国際環境学会を閉会いたします。皆様、ご参加ありがとうございました。なお、予稿集10月に発刊される予定です。	
\\	これを 待[ま]ちまして 第[だい] 
\\	回[かい] 国際[こくさい] 環境[かんきょう] 学会[がっかい]を 閉会[へいかい]いたします。 皆様[みなさま]、ご 参加[さんか]ありがとうございました。なお、 予稿[よこう] 集[しゅう] 
\\	月[がつ]に 発刊[はっかん]される 予定[よてい]です。
\\	会員には学会誌(年2回)と会報(年4回)が送られます。なお、来月お届けする学会誌は環境問題の特集号です。	
\\	会員[かいいん]には 学会[がっかい] 誌[し]
\\	年[とし] 
\\	回[かい])と 会報[かいほう]
\\	年[とし] 
\\	回[かい])が 送[おく]られます。なお、 来月[らいげつ]お 届[とど]けする 学会[がっかい] 誌[し]は 環境[かんきょう] 問題[もんだい]の 特集[とくしゅう] 号[ごう]です。
\\	彼はいい人だ。もっともちょっと考えが浅いところが気になる。	
\\	彼[かれ]はいい 人[ひと]だ。もっともちょっと 考[かんが]えが 浅[あさ]いところが 気[き]になる。
\\	この部屋にある雑誌や本はどれを見てもいいです。ただし、部屋からは持ち出さないで下さい。	
\\	この 部屋[へや]にある 雑誌[ざっし]や 本[ほん]はどれを 見[み]てもいいです。ただし、 部屋[へや]からは 持ち出[もちだ]さないで 下[くだ]さい。
\\	当館の開館時間は午前9時から午後6時まで。ただし、特別行事がある場合はこの限りではない。	
\\	当館[とうかん]の 開館[かいかん] 時間[じかん]は 午前[ごぜん] 
\\	時[じ]から 午後[ごご] 
\\	時[じ]まで。ただし、 特別[とくべつ] 行事[ぎょうじ]がある 場合[ばあい]はこの 限[かぎ]りではない。
\\	試験の時、辞書を見てもいいです。ただし、和英辞典だけです。	
\\	試験[しけん]の 時[とき]、 辞書[じしょ]を 見[み]てもいいです。ただし、 和英[かずひで] 辞典[じてん]だけです。
\\	この製品は定価の3割引で販売しています。ただし、返品はできません。	
\\	この 製品[せいひん]は 定価[ていか]の 
\\	割引[わりびき]で 販売[はんばい]しています。ただし、 返品[へんぴん]はできません。
\\	問題があったらいつでも相談して下さい。ただし、お金の問題は相談に乗れません。	
\\	問題[もんだい]があったらいつでも 相談[そうだん]して 下[くだ]さい。ただし、お 金[かね]の 問題[もんだい]は 相談[そうだん]に 乗[の]れません。
\\	発表のトピックは何でもかまいません。ただし、発表時間は20分以内です。	
\\	発表[はっぴょう]のトピックは 何[なに]でもかまいません。ただし、 発表[はっぴょう] 時間[じかん]は 
\\	分[ふん] 以内[いない]です。
\\	実習時間は午前8時20分から午後5時30分まで。但し、12時から1時までは休憩時間とする。	
\\	実習[じっしゅう] 時間[じかん]は 午前[ごぜん] 
\\	時[じ] 
\\	分[ふん]から 午後[ごご] 
\\	時[じ] 
\\	分[ふん]まで。 但[ただ]し、 
\\	時[じ]から 
\\	時[じ]までは 休憩[きゅうけい] 時間[じかん]とする。
\\	総会の議決は会員の過半数を以って決定される。但し、規定改定はこの限りではない。	
\\	総会[そうかい]の 議決[ぎけつ]は 会員[かいいん]の 過半数[かはんすう]を 以[も]って 決定[けってい]される。 但[ただ]し、 規定[きてい] 改定[かいてい]はこの 限[かぎ]りではない。
\\	第7条(出張の経路等)出張の経路とその利用交通機関は、経済性を重視して選ぶことを原則とする。但し、特別の理由がある場合はこの限りではない。	
\\	第[だい] 
\\	条[じょう]
\\	出張[しゅっちょう]の 経路[けいろ] 等[など]) 出張[しゅっちょう]の 経路[けいろ]とその 利用[りよう] 交通[こうつう] 機関[きかん]は、 経済[けいざい] 性[せい]を 重視[じゅうし]して 選[えら]ぶことを 原則[げんそく]とする。 但[ただ]し、 特別[とくべつ]の 理由[りゆう]がある 場合[ばあい]はこの 限[かぎ]りではない。
\\	ただし、
\\	は
\\	間の電位差、
\\	は電流、
\\	は抵抗。	
\\	ただし、 
\\	は 
\\	間[かん]の 電位差[でんいさ]、 
\\	は 電流[でんりゅう]、 
\\	は 抵抗[ていこう]。
\\	ソフトウエアのアップグレードは無料です。ただし、
\\	をご希望の場合は、実費と送料(5ドル)が必要です。	
\\	ソフトウエアのアップグレードは 無料[むりょう]です。ただし、 
\\	をご 希望[きぼう]の 場合[ばあい]は、 実費[じっぴ]と 送料[そうりょう]
\\	5ドル)が 必要[ひつよう]です。
\\	この故障は直せないことはないと思います。ただ、時間がかかると思います。	
\\	この 故障[こしょう]は 直[なお]せないことはないと 思[おも]います。ただ、 時間[じかん]がかかると 思[おも]います。
\\	この辺りは駅にもスーパーにも近くて便利だ。ただし、環境があまりよくない。	
\\	この 辺[あた]りは 駅[えき]にもスーパーにも 近[ちか]くて 便利[べんり]だ。ただし、 環境[かんきょう]があまりよくない。
\\	彼はいい人だ。ただ、私個人はあまり好きではない。	
\\	彼[かれ]はいい 人[ひと]だ。ただ、 私[わたし] 個人[こじん]はあまり 好[す]きではない。
\\	ここはカニがとてもおいしいそうだ。ただ、私はアレルギーのためカニは食べられない。	
\\	ここはカニがとてもおいしいそうだ。ただ、 私[わたし]はアレルギーのためカニは 食[た]べられない。
\\	たかが子供の言ったことだ。そんなに深刻に考えることはない。	
\\	たかが 子供[こども]の 言[い]ったことだ。そんなに 深刻[しんこく]に 考[かんが]えることはない。
\\	たかが漫画くらいと馬鹿にしてはいけない。	
\\	たかが 漫画[まんが]くらいと 馬鹿[ばか]にしてはいけない。
\\	たかが車の故障でそんなに大騒ぎする必要はない。	
\\	たかが 車[くるま]の 故障[こしょう]でそんなに 大騒[おおさわ]ぎする 必要[ひつよう]はない。
\\	たかが無断で自転車を借りたくらいのことでそんなに怒らなくてもいいでしょう。	
\\	たかが 無断[むだん]で 自転車[じてんしゃ]を 借[か]りたくらいのことでそんなに 怒[おこ]らなくてもいいでしょう。
\\	人間一人の知恵などたかが知れている。	
\\	人間[にんげん] 一人[ひとり]の 知恵[ちえ]などたかが 知[し]れている。
\\	たかが喫煙じゃないか。どうしてそんなに犯罪者のように言われなきゃならないんだ。	
\\	たかが 喫煙[きつえん]じゃないか。どうしてそんなに 犯罪[はんざい] 者[しゃ]のように 言[い]われなきゃならないんだ。
\\	たかがノートくらいと言われるかもしれませんが、僕にとってはとても大事なものなんです。	
\\	たかがノートくらいと 言[い]われるかもしれませんが、 僕[ぼく]にとってはとても 大事[だいじ]なものなんです。
\\	たかが10万円で何ができるというのですか。	
\\	たかが 
\\	万[まん] 円[えん]で 何[なに]ができるというのですか。
\\	たかがアパート探しくらいで君に面倒をかけたくない。	
\\	たかがアパート 探[さが]しくらいで 君[きみ]に 面倒[めんどう]をかけたくない。
\\	たかが転んだくらいで救急車を呼ぶなんて大げさだ。	
\\	たかが 転[ころ]んだくらいで 救急[きゅうきゅう] 車[しゃ]を 呼[よ]ぶなんて 大[おお]げさだ。
\\	公共事業によって景気が回復する領域などたかが知れている。	
\\	公共[こうきょう] 事業[じぎょう]によって 景気[けいき]が 回復[かいふく]する 領域[りょういき]などたかが 知[し]れている。
\\	たかが椅子。しかし、これによって仕事の能率が大きく左右されるし健康にも影響する。	
\\	たかが 椅子[いす]。しかし、これによって 仕事[しごと]の 能率[のうりつ]が 大[おお]きく 左右[さゆう]されるし 健康[けんこう]にも 影響[えいきょう]する。
\\	たかが風邪、されど風邪。	
\\	たかが 風邪[かぜ]、されど 風邪[かぜ]。
\\	この車はただの50万円で買えた。	
\\	この 車[くるま]はただの 
\\	万[まん] 円[えん]で 買[か]えた。
\\	彼は椅子に腰掛けたなり、ずっと黙っていた。	
\\	彼[かれ]は 椅子[いす]に 腰掛[こしか]けたなり、ずっと 黙[だま]っていた。
\\	片足を上げたなりの格好で、数秒間目をつぶるように言われた。	
\\	片足[かたあし]を 上[あ]げたなりの 格好[かっこう]で、 数[すう] 秒間[びょうかん] 目[め]をつぶるように 言[い]われた。
\\	純一は17歳の時にフランスに行ったなりで、日本には帰らず画家になって、今ではモンマルトルに住みついている。	
\\	純一[じゅんいち]は 
\\	歳[さい]の 時[とき]にフランスに 行[い]ったなりで、 日本[にほん]には 帰[かえ]らず 画家[がか]になって、 今[いま]ではモンマルトルに 住[す]みついている。
\\	エリサは『源氏物語』を買ったなり、一ページも読まず、本棚に置きっぱなしだった。	
\\	エリサは
\\	源氏物語[げんじものがたり]』を 買[か]ったなり、一ページも 読[よ]まず、 本棚[ほんだな]に 置[お]きっぱなしだった。
\\	彼は私の部屋の入り口に突っ立ったなりでニヤニヤ笑っていた。	
\\	彼[かれ]は 私[わたし]の 部屋[へや]の 入り口[いりくち]に 突[つ]っ 立[た]ったなりでニヤニヤ 笑[わら]っていた。
\\	妻がドアを開けてくれたんですけど、私はもう倒れた鳴りで、体が動きませんでした。	
\\	妻[つま]がドアを 開[あ]けてくれたんですけど、 私[わたし]はもう 倒[たお]れた 鳴[な]りで、 体[からだ]が 動[うご]きませんでした。
\\	腰をよじったなりの姿勢で長時間座っていたので、立てないほど腰が痛くなった。	
\\	腰[こし]をよじったなりの 姿勢[しせい]で 長時間[ちょうじかん] 座[すわ]っていたので、 立[た]てないほど 腰[こし]が 痛[いた]くなった。
\\	人形がほこりをかぶったなりで棚に乗っかっている。	
\\	人形[にんぎょう]がほこりをかぶったなりで 棚[たな]に 乗[の]っかっている。
\\	50を過ぎたら、50を過ぎたなりの頑張り方がある。	
\\	50を 過[す]ぎたら、 50を 過[す]ぎたなりの 頑張[がんば]り 方[かた]がある。
\\	勉強しなかったらしなかったなりの結果が出る。	
\\	勉強[べんきょう]しなかったらしなかったなりの 結果[けっか]が 出[で]る。
\\	ドアを閉めないままで出かけてしまった。	
\\	ドアを 閉[し]めないままで 出[で]かけてしまった。
\\	彼女はさよならの挨拶をしないまま行ってしまった。	
\\	彼女[かのじょ]はさよならの 挨拶[あいさつ]をしないまま 行[おこな]ってしまった。
\\	松の木が暴風で倒れたままだ。	
\\	松の木[まつのき]が 暴風[ぼうふう]で 倒[たお]れたままだ。
\\	彼は1964年にアメリカに行ったままだった。	
\\	彼[かれ]は 
\\	年[ねん]にアメリカに 行[い]ったままだった。
\\	この
\\	は千円単位でお金を引き出すことができる。	
\\	この 
\\	は 千[せん] 円[えん] 単位[たんい]でお 金[かね]を 引き出[ひきだ]すことができる。
\\	この社内調査では課単位でリサイクル実施率を調べる。	
\\	この 社内[しゃない] 調査[ちょうさ]では 課[か] 単位[たんい]でリサイクル 実施[じっし] 率[りつ]を 調[しら]べる。
\\	日本では湯飲みは5個単位で売られることが多い。	
\\	日本[にほん]では 湯飲[ゆの]みは 
\\	個[こ] 単位[たんい]で 売[う]られることが 多[おお]い。
\\	者の新しいデータ通信サービスは、最低契約量である150時間分のデータ通信を1分単位で利用できる。	
\\	者[しゃ]の 新[あたら]しいデータ 通信[つうしん]サービスは、 最低[さいてい] 契約[けいやく] 量[りょう]である 
\\	時間[じかん] 分[ぶん]のデータ 通信[つうしん]を 
\\	分[ふん] 単位[たんい]で 利用[りよう]できる。
\\	我が社は今回のアジアからの労働者派遣が成功した場合には、次回から規模を大幅に拡大し、千人単位で派遣する予定だ。	
\\	我[わ]が 社[しゃ]は 今回[こんかい]のアジアからの 労働[ろうどう] 者[しゃ] 派遣[はけん]が 成功[せいこう]した 場合[ばあい]には、 次回[じかい]から 規模[きぼ]を 大幅[おおはば]に 拡大[かくだい]し、 千[せん] 人[にん] 単位[たんい]で 派遣[はけん]する 予定[よてい]だ。
\\	オンライン書店の
\\	者は、本をページまたは章単位で販売するそうだ。	
\\	オンライン 書店[しょてん]の 
\\	者[しゃ]は、 本[ほん]をページまたは 章[しょう] 単位[たんい]で 販売[はんばい]するそうだ。
\\	このプロジェクト管理ソフトの新バージョンでは、これまで日単位で行われていた計画と管理を時間・分単位で行えるようになった。	
\\	このプロジェクト 管理[かんり]ソフトの 新[しん]バージョンでは、これまで 日[ひ] 単位[たんい]で 行[おこな]われていた 計画[けいかく]と 管理[かんり]を 時間[じかん]・ 分[ぶん] 単位[たんい]で 行[おこな]えるようになった。
\\	研究グループ単位に特別端末が1台設置された。	
\\	研究[けんきゅう]グループ 単位[たんい]に 特別[とくべつ] 端末[たんまつ]が 
\\	台[だい] 設置[せっち]された。
\\	この教科書は章単位に練習問題がつけてある。	
\\	この 教科書[きょうかしょ]は 章[しょう] 単位[たんい]に 練習[れんしゅう] 問題[もんだい]がつけてある。
\\	この
\\	は千円ずつ引き出すことができる。	
\\	この 
\\	は 千[せん] 円[えん]ずつ 引き出[ひきだ]すことができる。
\\	私は毎日漢字を5つずつ覚える。	
\\	私[わたし]は 毎日[まいにち] 漢字[かんじ]を 5つずつ 覚[おぼ]える。
\\	私たちは奨学金応募者を2人ずつ面接した。	
\\	私[わたし]たちは 奨学[しょうがく] 金[きん] 応募[おうぼ] 者[しゃ]を 
\\	人[にん]ずつ 面接[めんせつ]した。
\\	子供たちにキャンディーを5つずつあげた。	
\\	子供[こども]たちにキャンディーを 5つずつあげた。
\\	そのバスは20分ごとに来る。	
\\	そのバスは 
\\	分[ふん]ごとに 来[く]る。
\\	お正月なので家ごとに国旗が出ている。	
\\	お 正月[しょうがつ]なので 家[いえ]ごとに 国旗[こっき]が 出[で]ている。
\\	これは単に文化の違いの問題だ。	
\\	これは 単[たん]に 文化[ぶんか]の 違[ちが]いの 問題[もんだい]だ。
\\	この車は単にボディーのデザインを変えただけだ。	
\\	この 車[くるま]は 単[たん]にボディーのデザインを 変[か]えただけだ。
\\	仕事が難しいのではなく、単に私が不器用なだけです。	
\\	仕事[しごと]が 難[むずか]しいのではなく、 単[たん]に 私[わたし]が 不器用[ぶきよう]なだけです。
\\	円高の影響は単に輸出だけにとどまらない。	
\\	円[えん] 高[だか]の 影響[えいきょう]は 単[たん]に 輸出[ゆしゅつ]だけにとどまらない。
\\	これは単に私個人の意見に過ぎない。	
\\	これは 単[たん]に 私[わたし] 個人[こじん]の 意見[いけん]に 過[す]ぎない。
\\	それは単に解釈の問題だ。	
\\	それは 単[たん]に 解釈[かいしゃく]の 問題[もんだい]だ。
\\	これは単にあなた一人の問題ではない。	
\\	これは 単[たん]にあなた 一人[ひとり]の 問題[もんだい]ではない。
\\	この文章は単に原文を直訳しただけだ。	
\\	この 文章[ぶんしょう]は 単[たん]に 原文[げんぶん]を 直訳[ちょくやく]しただけだ。
\\	当ホテルは単に安いだけでなく、行き届いたサービスで快適にお過ごしいただけます。	
\\	当[とう]ホテルは 単[たん]に 安[やす]いだけでなく、 行き届[いきとど]いたサービスで 快適[かいてき]にお 過[す]ごしいただけます。
\\	今は単に多くの利益を上げているだけでは一流企業とは言えない。	
\\	今[いま]は 単[たん]に 多[おお]くの 利益[りえき]を 上[あ]げているだけでは 一流[いちりゅう] 企業[きぎょう]とは 言[い]えない。
\\	このカメラは、単に被写体にカメラを向けてシャッターを押すだけで高画質の写真を撮ることができる。	
\\	このカメラは、 単[たん]に 被写体[ひしゃたい]にカメラを 向[む]けてシャッターを 押[お]すだけで 高画質[こうがしつ]の 写真[しゃしん]を 撮[と]ることができる。
\\	彼は考えが一貫しているのではなく、単に頑固なだけだ。	
\\	彼[かれ]は 考[かんが]えが 一貫[いっかん]しているのではなく、 単[たん]に 頑固[がんこ]なだけだ。
\\	私は単に生まれがドイツだっただけで、ドイツ語はもちろんドイツのことは何も知らない。	
\\	私[わたし]は 単[たん]に 生[う]まれがドイツだっただけで、ドイツ 語[ご]はもちろんドイツのことは 何[なに]も 知[し]らない。
\\	雅夫は単に数学だけができないのではない。	
\\	雅夫[まさお]は 単[たん]に 数学[すうがく]だけができないのではない。
\\	この場合、単にアグネスだけを非難するのは間違っている。	
\\	この 場合[ばあい]、 単[たん]にアグネスだけを 非難[ひなん]するのは 間違[まちが]っている。
\\	それは単に君の憶測に過ぎない。	
\\	それは 単[たん]に 君[きみ]の 憶測[おくそく]に 過[す]ぎない。
\\	ただ単に飲むだけの集まりなら参加しても意味がない。	
\\	ただ 単[たん]に 飲[の]むだけの 集[あつ]まりなら 参加[さんか]しても 意味[いみ]がない。
\\	ただ単に犯行推定時間の前に現場近くにいたという事実だけで彼を犯人だと決めつけることはできない。	
\\	ただ 単[たん]に 犯行[はんこう] 推定[すいてい] 時間[じかん]の 前[まえ]に 現場[げんば] 近[ちか]くにいたという 事実[じじつ]だけで 彼[かれ]を 犯人[はんにん]だと 決[き]めつけることはできない。
\\	彼の問題は単に技術的な問題だ。	
\\	彼[かれ]の 問題[もんだい]は 単[たん]に 技術[ぎじゅつ] 的[てき]な 問題[もんだい]だ。
\\	彼の問題は単に技術的な問題だけだ。	
\\	彼[かれ]の 問題[もんだい]は 単[たん]に 技術[ぎじゅつ] 的[てき]な 問題[もんだい]だけだ。
\\	彼の問題は単に技術的な問題ではない。	
\\	彼[かれ]の 問題[もんだい]は 単[たん]に 技術[ぎじゅつ] 的[てき]な 問題[もんだい]ではない。
\\	彼の問題は単に技術的な問題だけではない。	
\\	彼[かれ]の 問題[もんだい]は 単[たん]に 技術[ぎじゅつ] 的[てき]な 問題[もんだい]だけではない。
\\	それは単なる彼の誤解だ。	
\\	それは 単[たん]なる 彼[かれ]の 誤解[ごかい]だ。
\\	このアパートは単に家賃が安いだけだ。	
\\	このアパートは 単[たん]に 家賃[やちん]が 安[やす]いだけだ。
\\	このアパートはただ会社へ行くのに便利なだけだ。	
\\	このアパートはただ 会社[かいしゃ]へ 行[い]くのに 便利[べんり]なだけだ。
\\	私はただ弁護士だけが頼りだ。	
\\	私[わたし]はただ 弁護士[べんごし]だけが 頼[たよ]りだ。
\\	ただこれだけは分かってもらいたい。	
\\	ただこれだけは 分[わ]かってもらいたい。
\\	ただ一人の肉親に死なれた。	
\\	ただ 一人[ひとり]の 肉親[にくしん]に 死[し]なれた。
\\	ただの一日も休んだことがない。	
\\	ただの一 日[にち]も 休[やす]んだことがない。
\\	この湖にはここにしかいないという生物がおり、その数たるや50種類を越える。	
\\	この 湖[みずうみ]にはここにしかいないという 生物[せいぶつ]がおり、その 数[かず]たるや 
\\	種類[しゅるい]を 越[こ]える。
\\	ノーベル文学賞はどうやって選考するのだろう。1年間に世界で発表される文学作品の数たるや膨大なものだと思うのだが。	
\\	ノーベル 文学[ぶんがく] 賞[しょう]はどうやって 選考[せんこう]するのだろう。 
\\	年間[ねんかん]に 世界[せかい]で 発表[はっぴょう]される 文学[ぶんがく] 作品[さくひん]の 数[かず]たるや 膨大[ぼうだい]なものだと 思[おも]うのだが。
\\	1987年春、彼はついにデビューを果たした。しかし、その初戦の結果たるや散々なものだった。	
\\	年[ねん] 春[はる]、 彼[かれ]はついにデビューを 果[は]たした。しかし、その 初戦[しょせん]の 結果[けっか]たるや 散々[さんざん]なものだった。
\\	彼のバンジョーは最高だ。特にその速弾きテクニックたるやたぶん世界でも彼の右に出る者はいないのではないだろうか。	
\\	彼[かれ]のバンジョーは 最高[さいこう]だ。 特[とく]にその 速[はや] 弾[び]きテクニックたるやたぶん 世界[せかい]でも 彼[かれ]の 右[みぎ]に 出[で]る 者[もの]はいないのではないだろうか。
\\	このダイエット教室の参加費は決して安くはないが、その効果たるや絶大である。	
\\	このダイエット 教室[きょうしつ]の 参加[さんか] 費[ひ]は 決[けっ]して 安[やす]くはないが、その 効果[こうか]たるや 絶大[ぜつだい]である。
\\	ロボカップは、サッカーをするロボットの世界一を競う大会である。1997年に名古屋で第1回大会が開かれて以来、国際大会が毎年開かれてきたが、その間の「選手」の技量の向上たるや著しい。	
\\	ロボカップは、サッカーをするロボットの 世界一[せかいいち]を 競[きそ]う 大会[たいかい]である。 
\\	年[ねん]に 名古屋[なごや]で 第[だい] 
\\	回[かい] 大会[たいかい]が 開[ひら]かれて 以来[いらい]、 国際[こくさい] 大会[たいかい]が 毎年[まいとし] 開[ひら]かれてきたが、その 間[あいだ]の
\\	選手[せんしゅ]」の 技量[ぎりょう]の 向上[こうじょう]たるや 著[いちじる]しい。
\\	一人で旅行に行ったって面白くない。	
\\	一人[ひとり]で 旅行[りょこう]に 行[い]ったって 面白[おもしろ]くない。
\\	私は少しくらしい高くたって、質のいいものを買うことにしている。	
\\	私[わたし]は 少[すこ]しくらしい 高[たか]くたって、 質[しつ]のいいものを 買[か]うことにしている。
\\	いくら日本語が上手だって専門知識がなかったらこの翻訳はできないいよ。	
\\	いくら 日本語[にほんご]が 上手[じょうず]だって 専門[せんもん] 知識[ちしき]がなかったらこの 翻訳[ほんやく]はできないいよ。
\\	プロだって失敗することもある。	
\\	プロだって 失敗[しっぱい]することもある。
\\	仕事を変えたって問題は解決しない。	
\\	仕事[しごと]を 変[か]えたって 問題[もんだい]は 解決[かいけつ]しない。
\\	苦しくったって諦めちゃいけない。	
\\	苦[くる]しくったって 諦[あきら]めちゃいけない。
\\	古くたって新しくたって、よいものはよいのだ。	
\\	古[ふる]くたって 新[あたら]しくたって、よいものはよいのだ。
\\	いくら通勤に便利だってこの家賃は高すぎる。	
\\	いくら 通勤[つうきん]に 便利[べんり]だってこの 家賃[やちん]は 高[たか]すぎる。
\\	誰だってそんなことを言われたら腹が立つよ。	
\\	誰[だれ]だってそんなことを 言[い]われたら 腹[はら]が 立[た]つよ。
\\	コンピュータがなくったってこの仕事はできる。	
\\	コンピュータがなくったってこの 仕事[しごと]はできる。
\\	自分で料理しなくたって、お金さえ出さば食べたいものが簡単に手に入るようになった。	
\\	自分[じぶん]で 料理[りょうり]しなくたって、お 金[かね]さえ 出[だ]さば 食[た]べたいものが 簡単[かんたん]に 手[て]に 入[はい]るようになった。
\\	見たって言ったでしょ。	
\\	見[み]たって 言[い]ったでしょ。
\\	聞けばよかったって思った。	
\\	聞[き]けばよかったって 思[おも]った。
\\	古田さんは英語の先生だったって聞いた。	
\\	古田[ふるた]さんは 英語[えいご]の 先生[せんせい]だったって 聞[き]いた。
\\	一人でしようたって、それは無理だ。	
\\	一人[ひとり]でしようたって、それは 無理[むり]だ。
\\	今すぐ買いにいくったって、もう店はみんな閉まっている。	
\\	今[いま]すぐ 買[か]いにいくったって、もう 店[みせ]はみんな 閉[し]まっている。
\\	高いたって、100万円はしないだろう。	
\\	高[たか]いたって、 
\\	万[まん] 円[えん]はしないだろう。
\\	日本語教育の経験者だったって、彼女はまだ日本以外で教えたことはない。	
\\	日本語[にほんご] 教育[きょういく]の 経験[けいけん] 者[しゃ]だったって、 彼女[かのじょ]はまだ 日本[にほん] 以外[いがい]で 教[おし]えたことはない。
\\	あの家は君が買おうったって買えるようなものじゃない。	
\\	あの 家[いえ]は 君[きみ]が 買[か]おうったって 買[か]えるようなものじゃない。
\\	僕を騙そうったってそうはいかない。	
\\	僕[ぼく]を 騙[だま]そうったってそうはいかない。
\\	彼が何を言おうったってそれをとめようとは思わない。	
\\	彼[かれ]が 何[なに]を 言[い]おうったってそれをとめようとは 思[おも]わない。
\\	今から帰るったってもう電車はないよ。	
\\	今[いま]から 帰[かえ]るったってもう 電車[でんしゃ]はないよ。
\\	調べるったって、こんなたくさんある資料を一人で調べるのは無理だ。	
\\	調[しら]べるったって、こんなたくさんある 資料[しりょう]を 一人[ひとり]で 調[しら]べるのは 無理[むり]だ。
\\	近いったってとても歩いていける距離じゃない。	
\\	近[ちか]いったってとても 歩[ある]いていける 距離[きょり]じゃない。
\\	いくら簡単ったって、やっぱり最低限のパソコン知識がなくては使えない。	
\\	いくら 簡単[かんたん]ったって、やっぱり 最低限[さいていげん]のパソコン 知識[ちしき]がなくては 使[つか]えない。
\\	「断食」ったって、全く何も口に入れないんじゃない。1日に2回、野菜ジュースを飲むことはできる。	
\\	断食[だんじき]」ったって、 全[まった]く 何[なに]も 口[くち]に 入[い]れないんじゃない。 
\\	日[にち]に 
\\	回[かい]、 野菜[やさい]ジュースを 飲[の]むことはできる。
\\	過去の業績に安住してばかりはいられない。	
\\	過去[かこ]の 業績[ぎょうせき]に 安住[あんじゅう]してばかりはいられない。
\\	確かにこれは深刻な事態だが、議論ばかりしてはいられない。	
\\	確[たし]かにこれは 深刻[しんこく]な 事態[じたい]だが、 議論[ぎろん]ばかりしてはいられない。
\\	周りの人が助けてくれるからと言って、いつまでも人にばかり頼ってはいられない。	
\\	周[まわ]りの 人[ひと]が 助[たす]けてくれるからと 言[い]って、いつまでも 人[ひと]にばかり 頼[たよ]ってはいられない。
\\	やっと長年の夢であったマイホームを手に入れたが、喜んでばかりはいられない。これから大きいなローンを返していかなければならないのだ。	
\\	やっと 長年の夢[ながねんのゆめ]であったマイホームを 手[て]に 入[い]れたが、 喜[よろこ]んでばかりはいられない。これから 大[おお]きいなローンを 返[かえ]していかなければならないのだ。
\\	リストラのショックからまだ完全に立ち直ったわけじゃないけど、いつまでも落ち込んでばかりもいられない。すぐに仕事探しを始めなければ。	
\\	リストラのショックからまだ 完全[かんぜん]に 立ち直[たちなお]ったわけじゃないけど、いつまでも 落ち込[おちこ]んでばかりもいられない。すぐに 仕事[しごと] 探[さが]しを 始[はじ]めなければ。
\\	希望の学校には入れなかったからと言って、くじけてばかりはいられない。	
\\	希望[きぼう]の 学校[がっこう]には 入[い]れなかったからと 言[い]って、くじけてばかりはいられない。
\\	人の成功に感心ばかりしてはいられない。我々ももっと頑張らなければ。	
\\	人[ひと]の 成功[せいこう]に 感心[かんしん]ばかりしてはいられない。 我々[われわれ]ももっと 頑張[がんば]らなければ。
\\	障害の原因ばかり考えてもいられない。早急に失われたデータを回復する必要がある。	
\\	障害[しょうがい]の 原因[げんいん]ばかり 考[かんが]えてもいられない。 早急[そうきゅう]に 失[うしな]われたデータを 回復[かいふく]する 必要[ひつよう]がある。
\\	退職して悠々自適の生活ができるようになったが、家にばかりこもってもいられないのでボランティアの仕事を始めた。	
\\	退職[たいしょく]して 悠々自適[ゆうゆうじてき]の 生活[せいかつ]ができるようになったが、 家[いえ]にばかりこもってもいられないのでボランティアの 仕事[しごと]を 始[はじ]めた。
\\	後悔ばかりしてはいられない。	
\\	後悔[こうかい]ばかりしてはいられない。
\\	カップラーメンばかり食べてはいられない。	
\\	カップラーメンばかり 食[た]べてはいられない。
\\	日本語にばかり時間をかけてはいられない。	
\\	日本語[にほんご]にばかり 時間[じかん]をかけてはいられない。
\\	親に頼ってばかりはいられない。	
\\	親[おや]に 頼[たよ]ってばかりはいられない。
\\	親にばかり頼ってはいられない。	
\\	親[おや]にばかり 頼[たよ]ってはいられない。
\\	考えても考えても思い出せない。	
\\	考[かんが]えても 考[かんが]えても 思い出[おもいだ]せない。
\\	働いても働いてもお金が貯まらない。	
\\	働[はたら]いても 働[はたら]いてもお 金[かね]が 貯[た]まらない。
\\	掘っても掘っても水は出てこなかった。	
\\	掘[ほ]っても 掘[ほ]っても 水[みず]は 出[で]てこなかった。
\\	歩いても歩いても町が見えてこない。	
\\	歩[ある]いても 歩[ある]いても 町[まち]が 見[み]えてこない。
\\	練習しても練習してもゴルフがうまくならない。	
\\	練習[れんしゅう]しても 練習[れんしゅう]してもゴルフがうまくならない。
\\	暑い上に湿度が高いので、拭っても拭っても汗が吹き出してくる。	
\\	暑[あつ]い 上[うえ]に 湿度[しつど]が 高[たか]いので、 拭[ぬぐ]っても 拭[ぬぐ]っても 汗[あせ]が 吹き出[ふきだ]してくる。
\\	このクラスの学生は教えても教えても覚えないので、教えがいがない。	
\\	このクラスの 学生[がくせい]は 教[おし]えても 教[おし]えても 覚[おぼ]えないので、 教[おし]えがいがない。
\\	彼は断られても断られてもあきらめない。	
\\	彼[かれ]は 断[ことわ]られても 断[ことわ]られてもあきらめない。
\\	勉強しても勉強しても成績が上がらない。	
\\	勉強[べんきょう]しても 勉強[べんきょう]しても 成績[せいせき]が 上[あ]がらない。
\\	いくら炊いても駄目だ。	
\\	いくら 炊[た]いても 駄目[だめ]だ。
\\	いくら頼まれてもこれは引き受けられない。	
\\	いくら 頼[たの]まれてもこれは 引き受[ひきう]けられない。
\\	いくら断っても駄目だ。	
\\	いくら 断[ことわ]っても 駄目[だめ]だ。
\\	彼女がいくら頼んでも駄目だ。	
\\	彼女[かのじょ]がいくら 頼[たの]んでも 駄目[だめ]だ。
\\	会社の理不尽なリストラに腹が立って仕方がない。	
\\	会社[かいしゃ]の 理不尽[りふじん]なリストラに 腹[はら]が 立[た]って 仕方[しかた]がない。
\\	久しぶりに妻や子供たちに会えるので嬉しくて仕方がない。	
\\	久[ひさ]しぶりに 妻[つま]や 子供[こども]たちに 会[あ]えるので 嬉[うれ]しくて 仕方[しかた]がない。
\\	ここは何もすることがないので退屈で仕方がない。	
\\	ここは 何[なに]もすることがないので 退屈[たいくつ]で 仕方[しかた]がない。
\\	アンディーがアメリカに帰ってまだ二日しか経っていないのに、寂しくて仕方がない。	
\\	アンディーがアメリカに 帰[かえ]ってまだ二 日[にち]しか 経[た]っていないのに、 寂[さび]しくて 仕方[しかた]がない。
\\	こんなすばらしい建物が取り壊されるのかと思うと残念で仕方がない。	
\\	こんなすばらしい 建物[たてもの]が 取り壊[とりこわ]されるのかと 思[おも]うと 残念[ざんねん]で 仕方[しかた]がない。
\\	ボーイフレンドが一年の日本留学を終えて来週帰ってくる。早く会いたくて仕方がない。	
\\	ボーイフレンドが一 年[ねん]の 日本[にほん] 留学[りゅうがく]を 終[お]えて 来週[らいしゅう] 帰[かえ]ってくる。 早[はや]く 会[あ]いたくて 仕方[しかた]がない。
\\	悪い風邪を引いたらしい。咳が出て仕方がない。	
\\	悪[わる]い 風邪[かぜ]を 引[ひ]いたらしい。 咳[せき]が 出[で]て 仕方[しかた]がない。
\\	ここは空港のすぐ近くなので飛行機の音がうるさくて仕方がない。	
\\	ここは 空港[くうこう]のすぐ 近[ちか]くなので 飛行機[ひこうき]の 音[おと]がうるさくて 仕方[しかた]がない。
\\	この参考書は索引が不完全なので使いにくくて仕方がない。	
\\	この 参考[さんこう] 書[しょ]は 索引[さくいん]が 不完全[ふかんぜん]なので 使[つか]いにくくて 仕方[しかた]がない。
\\	私のうちは近くにスーパーもコンビにもないので不便で仕方がない。	
\\	私[わたし]のうちは 近[ちか]くにスーパーもコンビにもないので 不便[ふべん]で 仕方[しかた]がない。
\\	先週、学会であった女性が気になって仕方がない。	
\\	先週[せんしゅう]、 学会[がっかい]であった 女性[じょせい]が 気[き]になって 仕方[しかた]がない。
\\	母親になった途端に自分の子供がよその子供より可愛く思えて仕方がない。	
\\	母親[ははおや]になった 途端[とたん]に 自分[じぶん]の 子供[こども]がよその 子供[こども]より 可愛[かわい]く 思[おも]えて 仕方[しかた]がない。
\\	浅田さんは久しぶりに妻や子供たちに会えるので嬉しくて仕方がないようだ。	
\\	浅田[あさだ]さんは 久[ひさ]しぶりに 妻[つま]や 子供[こども]たちに 会[あ]えるので 嬉[うれ]しくて 仕方[しかた]がないようだ。
\\	ジョージは何もすることがないので退屈で仕方がないらしい。	
\\	ジョージは 何[なに]もすることがないので 退屈[たいくつ]で 仕方[しかた]がないらしい。
\\	どういうわけか、喉が渇いてならない。	
\\	どういうわけか、 喉[のど]が 渇[かわ]いてならない。
\\	最近、人命や地球環境など、とても大切なものがビジネスの道具に使われているような気がして仕方がない。	
\\	最近[さいきん]、 人命[じんめい]や 地球[ちきゅう] 環境[かんきょう]など、とても 大切[たいせつ]なものがビジネスの 道具[どうぐ]に 使[つか]われているような 気[き]がして 仕方[しかた]がない。
\\	将来のことを考えると、いつまでもこんな仕事をしてはいられない。	
\\	将来[しょうらい]のことを 考[かんが]えると、いつまでもこんな 仕事[しごと]をしてはいられない。
\\	彼女があまりにも自分勝手だったので、一言言わないではいられなかった。	
\\	彼女[かのじょ]があまりにも 自分勝手[じぶんかって]だったので、 一言[ひとこと] 言[い]わないではいられなかった。
\\	彼が好きだと気づいた時から、もうただの友達ではいられなくなった。	
\\	彼[かれ]が 好[す]きだと 気[き]づいた 時[とき]から、もうただの 友達[ともだち]ではいられなくなった。
\\	もうすぐ飛行機が出るのでぐずぐずしてはいられない。	
\\	もうすぐ 飛行機[ひこうき]が 出[で]るのでぐずぐずしてはいられない。
\\	やることがたくさんあるのでパーティーなんかに行ってはいられない。	
\\	やることがたくさんあるのでパーティーなんかに 行[い]ってはいられない。
\\	自分にも同じ口癖があるので人のことを笑ってはいられない。	
\\	自分[じぶん]にも 同[おな]じ 口癖[くちぐせ]があるので 人[ひと]のことを 笑[わら]ってはいられない。
\\	あの教え方を見ていると時代錯誤を感じないではいられない。	
\\	あの 教[おし]え 方[かた]を 見[み]ていると 時代[じだい] 錯誤[さくご]を 感[かん]じないではいられない。
\\	忙しい時代に生まれ、時間に追われがちな今の子供ナチを案じないではいられない。	
\\	忙[いそが]しい 時代[じだい]に 生[う]まれ、 時間[じかん]に 追[お]われがちな 今[いま]の 子供[こども]ナチを 案[あん]じないではいられない。
\\	私は一日も本なしではいられない。	
\\	私[わたし]は一 日[にち]も 本[ほん]なしではいられない。
\\	よその国の事件とは言え、同じようなことはこの国でも起きる可能性がある。とても無関心ではいられない。	
\\	よその 国[くに]の 事件[じけん]とは 言[い]え、 同[おな]じようなことはこの 国[くに]でも 起[お]きる 可能[かのう] 性[せい]がある。とても 無[む] 関心[かんしん]ではいられない。
\\	彼女は日本中のみんなに期待されているので、もう普通の選手ではいられなくなった。	
\\	彼女[かのじょ]は 日本[にほん] 中[ちゅう]のみんなに 期待[きたい]されているので、もう 普通[ふつう]の 選手[せんしゅ]ではいられなくなった。
\\	地球環境は急速に悪化している。いつまでもこのままではいられない。	
\\	地球[ちきゅう] 環境[かんきょう]は 急速[きゅうそく]に 悪化[あっか]している。いつまでもこのままではいられない。
\\	長年勤めてきた同僚が最近リストラされた。私もうかうかしてはいられない。	
\\	長年[ながねん] 勤[つと]めてきた 同僚[どうりょう]が 最近[さいきん]リストラされた。 私[わたし]もうかうかしてはいられない。
\\	論文の締め切りまでまだ4か月あるとは言え、ほかにもすることがたくさんあるので、あまりのんびりしてはいられない。	
\\	論文[ろんぶん]の 締め切[しめき]りまでまだ 4か 月[げつ]あるとは 言[い]え、ほかにもすることがたくさんあるので、あまりのんびりしてはいられない。
\\	この本を読まないわけにはいかない。	
\\	この 本[ほん]を 読[よ]まないわけにはいかない。
\\	この本を読めないではいられない。	
\\	この 本[ほん]を 読[よ]めないではいられない。
\\	ボーナス後の連休とあって、観光地はどこも人でいっぱいだ。	
\\	ボーナス 後[ご]の 連休[れんきゅう]とあって、 観光[かんこう] 地[ち]はどこも 人[ひと]でいっぱいだ。
\\	夏休みが始まったとあって、子供たちはみんな嬉しそうだ。	
\\	夏休[なつやす]みが 始[はじ]まったとあって、 子供[こども]たちはみんな 嬉[うれ]しそうだ。
\\	優勝を争うチーム同士の対決とあって、スタジアム超満員だ。	
\\	優勝[ゆうしょう]を 争[あらそ]うチーム 同士[どうし]の 対決[たいけつ]とあって、スタジアム 超[ちょう] 満員[まんいん]だ。
\\	人気作家のサイン会とあって、本屋にはサインを求める人たちが長い列を作った。	
\\	人気[にんき] 作家[さっか]のサイン 会[かい]とあって、 本屋[ほんや]にはサインを 求[もと]める 人[ひと]たちが 長[なが]い 列[れつ]を 作[つく]った。
\\	レポートの締め切りがあさってとあって、学生たちはみんな焦っている。	
\\	レポートの 締め切[しめき]りがあさってとあって、 学生[がくせい]たちはみんな 焦[あせ]っている。
\\	掃除が終わったらお菓子がもらえるとあって、子供たちは一生懸命手伝っている。	
\\	掃除[そうじ]が 終[お]わったらお 菓子[かし]がもらえるとあって、 子供[こども]たちは 一生懸命[いっしょうけんめい] 手伝[てつだ]っている。
\\	中途半端な訓練では実社会で通用しないとあって、この語学学校の指導は非常に厳しい。	
\\	中途半端[ちゅうとはんぱ]な 訓練[くんれん]では 実社会[じっしゃかい]で 通用[つうよう]しないとあって、この 語学[ごがく] 学校[がっこう]の 指導[しどう]は 非常[ひじょう]に 厳[きび]しい。
\\	この社会人とんプログラムは、科目を低額で聴講できるだけでなく、自習室や図書館も自由に利用できるとあって、かなりの応募があります。	
\\	この 社会[しゃかい] 人[じん]とんプログラムは、 科目[かもく]を 低額[ていがく]で 聴講[ちょうこう]できるだけでなく、 自習[じしゅう] 室[しつ]や 図書館[としょかん]も 自由[じゆう]に 利用[りよう]できるとあって、かなりの 応募[おうぼ]があります。
\\	自動車が壊れたのでバスで学校に行った。	
\\	自動車[じどうしゃ]が 壊[こわ]れたのでバスで 学校[がっこう]に 行[い]った。
\\	社長の頼みとあっては断るわけには行かない。	
\\	社長[しゃちょう]の 頼[たの]みとあっては 断[ことわ]るわけには 行[い]かない。
\\	全商品が3割引きで買えるとあっては、遠方からわざわざ人が買いに来るのもうなずける。	
\\	全[ぜん] 商品[しょうひん]が 
\\	割[わり] 引[び]きで 買[か]えるとあっては、 遠方[えんぽう]からわざわざ 人[ひと]が 買[か]いに 来[く]るのもうなずける。
\\	この大学が他の大学に統合されるのは残念だが、国の方針とあっては致し方ない。	
\\	この 大学[だいがく]が 他[ほか]の 大学[だいがく]に 統合[とうごう]されるのは 残念[ざんねん]だが、 国[くに]の 方針[ほうしん]とあっては 致し方[いたしかた]ない。
\\	ブラジル・ドイツ戦のすぐ後とあっては国内チーム同士の試合が物足りなく感じられたのも仕方がない。	
\\	ブラジル・ドイツ 戦[せん]のすぐ 後[あと]とあっては 国内[こくない]チーム 同士[どうし]の 試合[しあい]が 物足[ものた]りなく 感[かん]じられたのも 仕方[しかた]がない。
\\	皆が多忙な身とあっては30人以上の家族が集まれるのはクリスマスの休みしかない。	
\\	皆[みな]が 多忙[たぼう]な 身[み]とあっては 
\\	人[にん] 以上[いじょう]の 家族[かぞく]が 集[あつ]まれるのはクリスマスの 休[やす]みしかない。
\\	スカイダイビングがここサイパンで楽しめるとあっては、挑戦してみない手はない。	
\\	スカイダイビングがここサイパンで 楽[たの]しめるとあっては、 挑戦[ちょうせん]してみない 手[て]はない。
\\	どんなに短い作文でも、他の人に読まれ、コメントがつくとあっては、書く時の真剣味が増す。	
\\	どんなに 短[みじか]い 作文[さくぶん]でも、 他[ほか]の 人[ひと]に 読[よ]まれ、コメントがつくとあっては、 書[か]く 時[とき]の 真剣[しんけん] 味[み]が 増[ま]す。
\\	役人の飲食費に税金が使われたとあっては、誰もが腹立たしく思うのは当たり前だ。しかも、そのことが明るみに出そうになると、今度は圧力をかけて「口封じ」をやったとあっては、もう救いようがない。	
\\	役人[やくにん]の 飲食[いんしょく] 費[ひ]に 税金[ぜいきん]が 使[つか]われたとあっては、 誰[だれ]もが 腹立[はらだ]たしく 思[おも]うのは 当たり前[あたりまえ]だ。しかも、そのことが 明[あか]るみに 出[で]そうになると、 今度[こんど]は 圧力[あつりょく]をかけて
\\	口[くち] 封[ふう]じ」をやったとあっては、もう 救[すく]いようがない。
\\	皆が多忙な身であれば30人以上の家族が集まれるのはクリスマスの休みしかない。	
\\	皆[みな]が 多忙[たぼう]な 身[み]であれば 
\\	人[にん] 以上[いじょう]の 家族[かぞく]が 集[あつ]まれるのはクリスマスの 休[やす]みしかない。
\\	悟は「黙ってついてこい」とばかりに先に立って歩き出した。	
\\	悟[さとる]は
\\	黙[だま]ってついてこい」とばかりに 先[さき]に 立[た]って 歩[ある]き 出[だ]した。
\\	課長は余計な質問はするなとばかりに私をにらみつけた。	
\\	課長[かちょう]は 余計[よけい]な 質問[しつもん]はするなとばかりに 私[わたし]をにらみつけた。
\\	霧が晴れたので、私は今がチャンスとばかりに写真を撮りまくった。	
\\	霧[きり]が 晴[は]れたので、 私[わたし]は 今[いま]がチャンスとばかりに 写真[しゃしん]を 撮[と]りまくった。
\\	竹本はこの時とばかりに日頃の不満を吐き出した。	
\\	竹本[たけもと]はこの 時[とき]とばかりに 日頃[ひごろ]の 不満[ふまん]を 吐き出[はきだ]した。
\\	彼女は「えい!」とばかりに侵入者を投げ飛ばした。	
\\	彼女[かのじょ]は「えい!」とばかりに 侵入[しんにゅう] 者[しゃ]を 投げ飛[なげと]ばした。
\\	庭園の花たちは早く撮ってとばかりに美しく咲いていました。	
\\	庭園[ていえん]の 花[はな]たちは 早[はや]く 撮[と]ってとばかりに 美[うつく]しく 咲[さ]いていました。
\\	思い立ったが吉日とばかりにエアロビックスを始めました。	
\\	思い立[おもいた]ったが 吉日[きちじつ]とばかりにエアロビックスを 始[はじ]めました。
\\	新鮮な魚を食べられるのは今日が最後とばかりに、さしみを食べられるだけ食べた。	
\\	新鮮[しんせん]な 魚[さかな]を 食[た]べられるのは 今日[きょう]が 最後[さいご]とばかりに、さしみを 食[た]べられるだけ 食[た]べた。
\\	守は心の洗濯とばかりに一週間の旅行に出かけた。	
\\	守[まもり]は 心[こころ]の 洗濯[せんたく]とばかりに 一週間[いっしゅうかん]の 旅行[りょこう]に 出[で]かけた。
\\	彼らはここぞとばかりに自分たちの給料の安さを訴えた。	
\\	彼[かれ]らはここぞとばかりに 自分[じぶん]たちの 給料[きゅうりょう]の 安[やす]さを 訴[うった]えた。
\\	悟は「黙ってついてこい」と先に立って歩き出した。	
\\	悟[さとる]は
\\	黙[だま]ってついてこい」と 先[さき]に 立[た]って 歩[ある]き 出[だ]した。
\\	霧が晴れたので、私は今がチャンスと写真を撮りまくった。	
\\	霧[きり]が 晴[は]れたので、 私[わたし]は 今[いま]がチャンスと 写真[しゃしん]を 撮[と]りまくった。
\\	あれは運命とでも言うべき出来事だった。	
\\	あれは 運命[うんめい]とでも 言[い]うべき 出来事[できごと]だった。
\\	これはエッセイ的なサイトとでも言うべきでしょう。	
\\	これはエッセイ 的[てき]なサイトとでも 言[い]うべきでしょう。
\\	近いうちに試用版とでも言うべきものをお送りします。	
\\	近[ちか]いうちに 試用[しよう] 版[ばん]とでも 言[い]うべきものをお 送[おく]りします。
\\	ミュンヘンのシンボルとでも言うべきこの市役所は1867年から1908年にかけて建てられたものです。	
\\	ミュンヘンのシンボルとでも 言[い]うべきこの 市役所[しやくしょ]は 
\\	年[ねん]から 
\\	年[ねん]にかけて 建[た]てられたものです。
\\	我々は20世紀の技術の賜物とでも言うべき明石海峡大橋を渡った。	
\\	我々[われわれ]は 
\\	世紀[せいき]の 技術[ぎじゅつ]の 賜物[たまもの]とでも 言[い]うべき 明石海峡大橋[あかしかいきょうおおはし]を 渡[わた]った。
\\	マイホーム計画最大のポイントとでも言うべきローンの組み方には、大きく分けると公的融資と民間融資があります。	
\\	マイホーム 計画[けいかく] 最大[さいだい]のポイントとでも 言[い]うべきローンの 組み方[くみかた]には、 大[おお]きく 分[わ]けると 公的[こうてき] 融資[ゆうし]と 民間[みんかん] 融資[ゆうし]があります。
\\	彼を一言で表現するとしたら「ネットワークの旅人」とでも言うべきだろうか。	
\\	彼[かれ]を 一言[ひとこと]で 表現[ひょうげん]するとしたら「ネットワークの 旅人[たびびと]」とでも 言[い]うべきだろうか。
\\	医者といえども時に病気になることもある。	
\\	医者[いしゃ]といえども 時[とき]に 病気[びょうき]になることもある。
\\	いかなる名人といえども時には失敗することもある。	
\\	いかなる 名人[めいじん]といえども 時[とき]には 失敗[しっぱい]することもある。
\\	東京に住んでいるといえども知らない都内の名所は沢山ある。	
\\	東京[とうきょう]に 住[す]んでいるといえども 知[し]らない 都内[とない]の 名所[めいしょ]は 沢山[たくさん]ある。
\\	同時通訳は一瞬といえども気が抜けない。	
\\	同時[どうじ] 通訳[つうやく]は 一瞬[いっしゅん]といえども 気[き]が 抜[ぬ]けない。
\\	科学技術の分野では、たとえグループ研究といえども個人の新しい発想によるところが非常に大きい。	
\\	科学[かがく] 技術[ぎじゅつ]の 分野[ぶんや]では、たとえグループ 研究[けんきゅう]といえども 個人[こじん]の 新[あたら]しい 発想[はっそう]によるところが 非常[ひじょう]に 大[おお]きい。
\\	いかなる自動翻訳機といえども、完全に機械だけで自然言語を翻訳するのは難しい。	
\\	いかなる 自動[じどう] 翻訳[ほんやく] 機[き]といえども、 完全[かんぜん]に 機械[きかい]だけで 自然[しぜん] 言語[げんご]を 翻訳[ほんやく]するのは 難[むずか]しい。
\\	当たらずといえども遠からず。	
\\	当[あ]たらずといえども 遠[とお]からず。
\\	世界広しといえども、東京の神田ほど多くの古本屋が一か所に集中しているところはほかにないだろう。	
\\	世界[せかい] 広[ひろ]しといえども、 東京[とうきょう]の 神田[かんだ]ほど 多[おお]くの 古本屋[ふるほんや]が 一か所[いっかしょ]に 集中[しゅうちゅう]しているところはほかにないだろう。
\\	筋力は少し衰えたといえども、彼はまだまだ若い選手に負けない集中力とスタミナを保っている。	
\\	筋力[きんりょく]は 少[すこ]し 衰[おとろ]えたといえども、 彼[かれ]はまだまだ 若[わか]い 選手[せんしゅ]に 負[ま]けない 集中[しゅうちゅう] 力[りょく]とスタミナを 保[たも]っている。
\\	いかに日本での生活が長いといえども、言葉の問題には常に遭遇する。	
\\	いかに 日本[にほん]での 生活[せいかつ]が 長[なが]いといえども、 言葉[ことば]の 問題[もんだい]には 常[つね]に 遭遇[そうぐう]する。
\\	天性の才能を持ち合わせているといえども、絶え間ない努力と訓練なくして今日の彼女はなかった。	
\\	天性[てんせい]の 才能[さいのう]を 持ち合[もちあ]わせているといえども、 絶え間[たえま]ない 努力[どりょく]と 訓練[くんれん]なくして 今日[こんにち]の 彼女[かのじょ]はなかった。
\\	このコーヒーは味といい香りといい、申し分ありません。	
\\	このコーヒーは 味[あじ]といい 香[かお]りといい、 申し分[もうしぶん]ありません。
\\	内容といい話し方といい、見事なスピーチでした。	
\\	内容[ないよう]といい 話し方[はなしかた]といい、 見事[みごと]なスピーチでした。
\\	声といい演技力といい、彼女は最高だ。	
\\	声[こえ]といい 演技[えんぎ] 力[りょく]といい、 彼女[かのじょ]は 最高[さいこう]だ。
\\	人柄といい才能といい、米倉さんに勝る人はいない。	
\\	人柄[ひとがら]といい 才能[さいのう]といい、 米倉[よねくら]さんに 勝[まさ]る 人[ひと]はいない。
\\	音楽の情感といい音の豊かさといい、こんな感動的な演奏を聴いたことがない。	
\\	音楽[おんがく]の 情感[じょうかん]といい 音[おと]の 豊[ゆた]かさといい、こんな 感動[かんどう] 的[てき]な 演奏[えんそう]を 聴[き]いたことがない。
\\	家柄といい学歴といい、彼は非の打ちどころがない。	
\\	家柄[いえがら]といい 学歴[がくれき]といい、 彼[かれ]は 非[ひ]の 打[う]ちどころがない。
\\	スピードといいパワーといい、今の彼に勝てる選手はいない。	
\\	スピードといいパワーといい、 今[いま]の 彼[かれ]に 勝[か]てる 選手[せんしゅ]はいない。
\\	速報性といい機能の多彩さといい情報の種類といい、インターネットニュースにかなう情報メディアはない。	
\\	速報[そくほう] 性[せい]といい 機能[きのう]の 多彩[たさい]さといい 情報[じょうほう]の 種類[しゅるい]といい、インターネットニュースにかなう 情報[じょうほう]メディアはない。
\\	ドイツ、フランス、ロシアといった国々が戦争に反対した。	
\\	ドイツ、フランス、ロシアといった 国々[くにぐに]が 戦争[せんそう]に 反対[はんたい]した。
\\	このエンジンで一番困るのはパワーが弱い、製造コストが高いといった問題だ。	
\\	このエンジンで 一番[いちばん] 困[こま]るのはパワーが 弱[よわ]い、 製造[せいぞう]コストが 高[たか]いといった 問題[もんだい]だ。
\\	終身雇用、年功序列といった日本の伝統的雇用形態が崩れつつある。	
\\	終身[しゅうしん] 雇用[こよう]、 年功序列[ねんこうじょれつ]といった 日本[にほん]の 伝統[でんとう] 的[てき] 雇用[こよう] 形態[けいたい]が 崩[くず]れつつある。
\\	ここでは
\\	やベンツといったドイツの高級車が人気があるようだ。	
\\	ここでは 
\\	やベンツといったドイツの 高級[こうきゅう] 車[しゃ]が 人気[にんき]があるようだ。
\\	学校選択制、そして基礎学力の向上といったテーマが今、全国で議論されている。	
\\	学校[がっこう] 選択[せんたく] 制[せい]、そして 基礎[きそ] 学力[がくりょく]の 向上[こうじょう]といったテーマが 今[いま]、 全国[ぜんこく]で 議論[ぎろん]されている。
\\	こんな機能があるといい、こんな情報が欲しいといったご意見・ご要望がありましたら、下記のアドレスに電子メールでお知らせ下さい。	
\\	こんな 機能[きのう]があるといい、こんな 情報[じょうほう]が 欲[ほ]しいといったご 意見[いけん]・ご 要望[ようぼう]がありましたら、 下記[かき]のアドレスに 電子[でんし]メールでお 知[し]らせ 下[くだ]さい。
\\	一流大学から一流企業に就職すれば一生安泰の人生が送れるといった考え方はもう通用しない。	
\\	一流[いちりゅう] 大学[だいがく]から 一流[いちりゅう] 企業[きぎょう]に 就職[しゅうしょく]すれば 一生[いっしょう] 安泰[あんたい]の 人生[じんせい]が 送[おく]れるといった 考え方[かんがえかた]はもう 通用[つうよう]しない。
\\	ゴミを燃やせばダイオキシンや二酸化炭素といった有害物質が排出される。またゴミの埋立地もあとわずかで満杯になる、しかも新たな埋立地がない、といった問題がある。	
\\	ゴミを 燃[も]やせばダイオキシンや 二酸化炭素[にさんかたんそ]といった 有害[ゆうがい] 物質[ぶっしつ]が 排出[はいしゅつ]される。またゴミの 埋立[うめたて] 地[ち]もあとわずかで 満杯[まんぱい]になる、しかも 新[あら]たな 埋立[うめたて] 地[ち]がない、といった 問題[もんだい]がある。
\\	9月になって猛暑も一段落といたところです。	
\\	月[がつ]になって 猛暑[もうしょ]も 一段落[いちだんらく]といたところです。
\\	ここは山というよりは、小高い丘といたところだ。	
\\	ここは 山[やま]というよりは、 小高[こだか]い 丘[おか]といたところだ。
\\	ホテルの部屋の広さはまあまあといったところでした。	
\\	ホテルの 部屋[へや]の 広[ひろ]さはまあまあといったところでした。
\\	この大学の学部生と院生は合わせて6,000人といったところだ。	
\\	この 大学[だいがく]の 学部[がくぶ] 生[せい]と 院生[いんせい]は 合[あ]わせて 
\\	人[にん]といったところだ。
\\	ヤンキースとレッドソックスはほぼ互角といったところだ。	
\\	ヤンキースとレッドソックスはほぼ 互角[ごかく]といったところだ。
\\	たった1年間の経験だけでは、図書館の「達人」と呼ばれるには、まだまだといったところです。	
\\	たった 
\\	年間[ねんかん]の 経験[けいけん]だけでは、 図書館[としょかん]の
\\	達人[たつじん]」と 呼[よ]ばれるには、まだまだといったところです。
\\	英語の“
\\	は日本語の「すみません」よりも「謝罪」の意味合いが強い。いわゆる「ごめんなさい」「申し訳ありません」といったところだ。	
\\	英語[えいご]の“ 
\\	は 日本語[にほんご]の「すみません」よりも
\\	謝罪[しゃざい]」の 意味合[いみあ]いが 強[つよ]い。いわゆる「ごめんなさい」
\\	申し訳[もうしわけ]ありません」といったところだ。
\\	この小説の主人公の男の性格を一言で表すとすれば、「自分には厳しいけれど、人にはとても優しい」といったところです。	
\\	この 小説[しょうせつ]の 主人公[しゅじんこう]の 男[おとこ]の 性格[せいかく]を 一言[ひとこと]で 表[あらわ]すとすれば、
\\	自分[じぶん]には 厳[きび]しいけれど、 人[ひと]にはとても 優[やさ]しい」といったところです。
\\	あと1000万円あればなんとか欲しい家が買える。と言って、そんな大金を借りられるあてはない。	
\\	あと 
\\	万[まん] 円[えん]あればなんとか 欲[ほ]しい 家[いえ]が 買[か]える。と 言[い]って、そんな 大金[たいきん]を 借[か]りられるあてはない。
\\	外食は飽きたし自分で料理するのも億劫だ。と言って、何も食べずにいるというわけにもいかない。	
\\	外食[がいしょく]は 飽[あ]きたし 自分[じぶん]で 料理[りょうり]するのも 億劫[おっくう]だ。と 言[い]って、 何[なに]も 食[た]べずにいるというわけにもいかない。
\\	いくら考えても名案は浮かばない。と言って、何もしなければ事態は悪化する一方だ。	
\\	いくら 考[かんが]えても 名案[めいあん]は 浮[う]かばない。と 言[い]って、 何[なに]もしなければ 事態[じたい]は 悪化[あっか]する 一方[いっぽう]だ。
\\	彼女と別れるのは絶対いやだ。と言って、妻と離婚する勇気もない。	
\\	彼女[かのじょ]と 別[わか]れるのは 絶対[ぜったい]いやだ。と 言[い]って、 妻[つま]と 離婚[りこん]する 勇気[ゆうき]もない。
\\	最近は少し運動するとすぐ疲れてしまう。と言って、何もしないと体が衰えるばかりなので何かしなければならない。	
\\	最近[さいきん]は 少[すこ]し 運動[うんどう]するとすぐ 疲[つか]れてしまう。と 言[い]って、 何[なに]もしないと 体[からだ]が 衰[おとろ]えるばかりなので 何[なに]かしなければならない。
\\	アメリカで自分の能力をためすのも面白いかもしれない。と言って、英語ができなければどうしようもないが。	
\\	アメリカで 自分[じぶん]の 能力[のうりょく]をためすのも 面白[おもしろ]いかもしれない。と 言[い]って、 英語[えいご]ができなければどうしようもないが。
\\	自分の勘違いと言うか、不注意でした。	
\\	自分[じぶん]の 勘違[かんちが]いと 言[い]うか、 不注意[ふちゅうい]でした。
\\	彼のことは諦めたと言うか、もうどうでもよくなった。	
\\	彼[かれ]のことは 諦[あきら]めたと 言[い]うか、もうどうでもよくなった。
\\	我が社の話しは何もできません。というか、何も話すことはありません。	
\\	我[わ]が 社[しゃ]の 話[はな]しは 何[なに]もできません。というか、 何[なに]も 話[はな]すことはありません。
\\	彼は議論好きと言うか、いつも人の意見に反対したがります。	
\\	彼[かれ]は 議論[ぎろん] 好[ず]きと 言[い]うか、いつも 人[ひと]の 意見[いけん]に 反対[はんたい]したがります。
\\	この事件については、私たちも驚いていると言うか、困惑しています。	
\\	この 事件[じけん]については、 私[わたし]たちも 驚[おどろ]いていると 言[い]うか、 困惑[こんわく]しています。
\\	今日来た人は、商魂たくましいと言うか、本当に厚かましい人でした。	
\\	今日[きょう] 来[き]た 人[ひと]は、 商魂[しょうこん]たくましいと 言[い]うか、 本当[ほんとう]に 厚[あつ]かましい 人[ひと]でした。
\\	何度裏切られても涙を見ると許してしまう自分に腹が立つと言うか、愛想が尽きる。	
\\	何[なん] 度[ど] 裏切[うらぎ]られても 涙[なみだ]を 見[み]ると 許[ゆる]してしまう 自分[じぶん]に 腹[はら]が 立[た]つと 言[い]うか、 愛想[あいそ]が 尽[つ]きる。
\\	これは何と言うか、まあ、一種の妥協でしょうね。	
\\	これは 何[なに]と 言[い]うか、まあ、 一種[いっしゅ]の 妥協[だきょう]でしょうね。
\\	これは日記と言うか覚え書きと言うか……。	
\\	これは 日記[にっき]と 言[い]うか 覚え書[おぼえが]きと 言[い]うか……。
\\	ーあの人は奥さんですか。 ーうーん、と言うか、恋人みたいなもんです。	
\\	ーあの 人[ひと]は 奥[おく]さんですか。 ーうーん、と 言[い]うか、 恋人[こいびと]みたいなもんです。
\\	自分の勘違いと言うより不注意でした。	
\\	自分[じぶん]の 勘違[かんちが]いと 言[い]うより 不注意[ふちゅうい]でした。
\\	彼のことは諦めたと言うより、もうどうでもよくなった。	
\\	彼[かれ]のことは 諦[あきら]めたと 言[い]うより、もうどうでもよくなった。
\\	来月の旅行は取りやめます。と言うのは、同じ頃に大事なお客が来ることになったんです。	
\\	来月[らいげつ]の 旅行[りょこう]は 取[と]りやめます。と 言[い]うのは、 同[おな]じ 頃[ころ]に 大事[だいじ]なお 客[きゃく]が 来[く]ることになったんです。
\\	今まで読めていた画像ファイルがほとんど読めなくなってしまいました。ほとんどと言うのは、少しですが読めるファイルもあるのです。	
\\	今[いま]まで 読[よ]めていた 画像[がぞう]ファイルがほとんど 読[よ]めなくなってしまいました。ほとんどと 言[い]うのは、 少[すこ]しですが 読[よ]めるファイルもあるのです。
\\	社長に直接話してみたらどうですか。と言うのは、実は社長も以前そういうアイディアについて話していたことがあるんです。	
\\	社長[しゃちょう]に 直接[ちょくせつ] 話[はな]してみたらどうですか。と 言[い]うのは、 実[じつ]は 社長[しゃちょう]も 以前[いぜん]そういうアイディアについて 話[はな]していたことがあるんです。
\\	もう少し詳しく話してもらえませんか。と言うのは、私も少し前、同じような男から同じような被害に遭ったんです。	
\\	もう 少[すこ]し 詳[くわ]しく 話[はな]してもらえませんか。と 言[い]うのは、 私[わたし]も 少[すこ]し 前[まえ]、 同[おな]じような 男[おとこ]から 同[おな]じような 被害[ひがい]に 遭[あ]ったんです。
\\	それは一度専門家に調べてもらった方がいいでしょう。と言うのは、素人が勝手に直すと、知らないうちに部品を傷めてしまう可能性があるからです。	
\\	それは 一度[いちど] 専門[せんもん] 家[か]に 調[しら]べてもらった 方[ほう]がいいでしょう。と 言[い]うのは、 素人[しろうと]が 勝手[かって]に 直[なお]すと、 知[し]らないうちに 部品[ぶひん]を 傷[いた]めてしまう 可能[かのう] 性[せい]があるからです。
\\	最近のパソコン用
\\	の普及で視覚障害者が困っている。と言うのは、これらの
\\	は画像中心で文学情報が少なく、文学を音声で読ませる方法があまり使えないからだ。	
\\	最近[さいきん]のパソコン 用[よう] 
\\	の 普及[ふきゅう]で 視覚[しかく] 障害[しょうがい] 者[しゃ]が 困[こま]っている。と 言[い]うのは、これらの 
\\	は 画像[がぞう] 中心[ちゅうしん]で 文学[ぶんがく] 情報[じょうほう]が 少[すく]なく、 文学[ぶんがく]を 音声[おんせい]で 読[よ]ませる 方法[ほうほう]があまり 使[つか]えないからだ。
\\	これらのゲームは、より正確に言えば、「一人用ゲーム」というより「一人でもできるゲーム」と言うべきでしょう。と言いますのは、これらのゲームはやり方によっては二人以上でもできるからです。	
\\	これらのゲームは、より 正確[せいかく]に 言[い]えば、
\\	一人[ひとり] 用[よう]ゲーム」というより
\\	一人[ひとり]でもできるゲーム」と 言[い]うべきでしょう。と 言[い]いますのは、これらのゲームはやり 方[かた]によっては 二人[ふたり] 以上[いじょう]でもできるからです。
\\	あの提案書はたぶん通ると思う。たぶんと言うのは、審査委員の中に一人だけ反対するかもしれないのがいるんだ。	
\\	あの 提案[ていあん] 書[しょ]はたぶん 通[とお]ると 思[おも]う。たぶんと 言[い]うのは、 審査[しんさ] 委員[いいん]の 中[なか]に 一人[ひとり]だけ 反対[はんたい]するかもしれないのがいるんだ。
\\	友達の引っ越しを手伝ったら、翌日肩と言わず腰と言わず体のあちこちが痛んだ。	
\\	友達[ともだち]の 引っ越[ひっこ]しを 手伝[てつだ]ったら、 翌日[よくじつ] 肩[かた]と 言[い]わず 腰[こし]と 言[い]わず 体[からだ]のあちこちが 痛[いた]んだ。
\\	武史の部屋は、壁と言わず天井と言わず、いたるところにサッカーの写真が貼ってあった。	
\\	武史[たけし]の 部屋[へや]は、 壁[かべ]と 言[い]わず 天井[てんじょう]と 言[い]わず、いたるところにサッカーの 写真[しゃしん]が 貼[は]ってあった。
\\	しばらくすると、発信が首と言わず手足と言わず、全身に広がった。	
\\	しばらくすると、 発信[はっしん]が 首[くび]と 言[い]わず 手足[てあし]と 言[い]わず、 全身[ぜんしん]に 広[ひろ]がった。
\\	イベント会場は廊下と言わず階段と言わず、参加者で溢れていた。	
\\	イベント 会場[かいじょう]は 廊下[ろうか]と 言[い]わず 階段[かいだん]と 言[い]わず、 参加[さんか] 者[しゃ]で 溢[あふ]れていた。
\\	彼女が身につけているものは、靴と言わず、時計と言わず、スカーフと言わず、すべてブランドものだ。	
\\	彼女[かのじょ]が 身[み]につけているものは、 靴[くつ]と 言[い]わず、 時計[とけい]と 言[い]わず、スカーフと 言[い]わず、すべてブランドものだ。
\\	彼は仕事中と言わず休み時間と言わず、一時もコンピュータから離れることがない。	
\\	彼[かれ]は 仕事[しごと] 中[ちゅう]と 言[い]わず 休[やす]み 時間[じかん]と 言[い]わず、 一時[いっとき]もコンピュータから 離[はな]れることがない。
\\	姉のうちは昼と言わず夜と言わず、一日中テレビをつけっぱなしにしている。	
\\	姉[あね]のうちは 昼[ひる]と 言[い]わず 夜[よる]と 言[い]わず、一日[いちにち] 中[じゅう]テレビをつけっぱなしにしている。
\\	男と言わず女と言わず、誰もがこの物語のヒロインには共感するはずだ。	
\\	男[おとこ]と 言[い]わず 女[おんな]と 言[い]わず、 誰[だれ]もがこの 物語[ものがたり]のヒロインには 共感[きょうかん]するはずだ。
\\	保守と言わず革新と言わず、どんな政治家もみな選挙資金集めのために奔走する。	
\\	保守[ほしゅ]と 言[い]わず 革新[かくしん]と 言[い]わず、どんな 政治[せいじ] 家[か]もみな 選挙[せんきょ] 資金[しきん] 集[あつ]めのために 奔走[ほんそう]する。
\\	この宝石は猫の目に似ているところから「猫目石」と名づけられている。	
\\	この 宝石[ほうせき]は 猫[ねこ]の 目[め]に 似[に]ているところから
\\	猫[ねこ] 目[め] 石[いし]」と 名[な]づけられている。
\\	私が両者にそのことを十分説明しなかったところから誤解が生じた。	
\\	私[わたし]が 両者[りょうしゃ]にそのことを 十分[じゅうぶん] 説明[せつめい]しなかったところから 誤解[ごかい]が 生[しょう]じた。
\\	田中先生はいつも蝶ネクタイをしていたところから「蝶ネクタイ先生」と呼ばれていた。	
\\	田中[たなか] 先生[せんせい]はいつも 蝶[ちょう]ネクタイをしていたところから
\\	蝶[ちょう]ネクタイ 先生[せんせい]」と 呼[よ]ばれていた。
\\	日本は憲法が戦力の保持を禁止しているところから、「軍隊」ではなく「自衛隊」を持っている。	
\\	日本[にほん]は 憲法[けんぽう]が 戦力[せんりょく]の 保持[ほじ]を 禁止[きんし]しているところから、
\\	軍隊[ぐんたい]」ではなく
\\	自衛隊[じえいたい]」を 持[も]っている。
\\	この説は、両者が生物的に多くの共通点を持っているところから来ている。	
\\	この 説[せつ]は、 両者[りょうしゃ]が 生物[せいぶつ] 的[てき]に 多[おお]くの 共通[きょうつう] 点[てん]を 持[も]っているところから 来[き]ている。
\\	論語に「何もしないでいるよりは碁を打つ方がましだ」という教えがあるところから、碁は孔子以前にかなり普及していたとみられる。	
\\	論語[ろんご]に
\\	何[なに]もしないでいるよりは 碁[ご]を 打[う]つ 方[ほう]がましだ」という 教[おし]えがあるところから、 碁[ご]は 孔子[こうし] 以前[いぜん]にかなり 普及[ふきゅう]していたとみられる。
\\	そのイギリス人とは家が近くだったところから、よく遊びに行って英語や英語の歌を教えてもらったりした。	
\\	そのイギリス 人[じん]とは 家[いえ]が 近[ちか]くだったところから、よく 遊[あそ]びに 行[い]って 英語[えいご]や 英語[えいご]の 歌[うた]を 教[おし]えてもらったりした。
\\	この雑誌は最初、ある有名月刊誌の別冊として刊行されたのが好評だったところから、後に単独の雑誌として独立したものだ。	
\\	この 雑誌[ざっし]は 最初[さいしょ]、ある 有名[ゆうめい] 月刊[げっかん] 誌[し]の 別冊[べっさつ]として 刊行[かんこう]されたのが 好評[こうひょう]だったところから、 後[のち]に 単独[たんどく]の 雑誌[ざっし]として 独立[どくりつ]したものだ。
\\	今音楽で広く使われている
\\	は、実はベートーベンの第九交響曲の演奏時間が74分間だったところからその容量が決められたということだ。	
\\	今[いま] 音楽[おんがく]で 広[ひろ]く 使[つか]われている 
\\	は、 実[じつ]はベートーベンの 第[だい] 九交響[くこうきょう] 曲[きょく]の 演奏[えんそう] 時間[じかん]が74 分間[ふんかん]だったところからその 容量[ようりょう]が 決[き]められたということだ。
\\	散歩をしているうちに、来るともなく駅まで来てしまった。	
\\	散歩[さんぽ]をしているうちに、 来[く]るともなく 駅[えき]まで 来[き]てしまった。
\\	我が家ではいつからともなく正月にスキーに行く習慣が始まった。	
\\	我が家[わがや]ではいつからともなく 正月[しょうがつ]にスキーに 行[い]く 習慣[しゅうかん]が 始[はじ]まった。
\\	聴くともなく聴いていたら、だんだんその演奏に引き込まれていった。	
\\	聴[き]くともなく 聴[き]いていたら、だんだんその 演奏[えんそう]に 引き込[ひきこ]まれていった。
\\	大学時代のことを考えるともなく考えていると、次々にクラスメートの顔が浮かんできた。	
\\	大学[だいがく] 時代[じだい]のことを 考[かんが]えるともなく 考[かんが]えていると、 次々[つぎつぎ]にクラスメートの 顔[かお]が 浮[う]かんできた。
\\	目撃者の話を聞くともなく後ろで聞いていたが、突然その話の中に事件解決のための重要な情報が含まれていることに気がついた。	
\\	目撃[もくげき] 者[しゃ]の 話[はなし]を 聞[き]くともなく 後[うし]ろで 聞[き]いていたが、 突然[とつぜん]その 話[はなし]の 中[なか]に 事件[じけん] 解決[かいけつ]のための 重要[じゅうよう]な 情報[じょうほう]が 含[ふく]まれていることに 気[き]がついた。
\\	どこからともなくいい匂いが漂ってきた。	
\\	どこからともなくいい 匂[にお]いが 漂[ただよ]ってきた。
\\	「やっと終わった」。彼は誰にともなくつぶやいた。	
\\	「やっと 終[お]わった」。 彼[かれ]は 誰[だれ]にともなくつぶやいた。
\\	二人はどちらからともなく顔を見合わせて微笑んだ。	
\\	二人[ふたり]はどちらからともなく 顔[かお]を 見合[みあ]わせて 微笑[ほほえ]んだ。
\\	いつも母がその歌をロシア語で歌うのを聞いていたので、習うともなく自然に覚えてしまった。	
\\	いつも 母[はは]がその 歌[うた]をロシア 語[ご]で 歌[うた]うのを 聞[き]いていたので、 習[なら]うともなく 自然[しぜん]に 覚[おぼ]えてしまった。
\\	その習慣はいつからともなく始まった。	
\\	その 習慣[しゅうかん]はいつからともなく 始[はじ]まった。
\\	どこからともなく鐘の音が聞こえてくる。	
\\	どこからともなく 鐘[かね]の 音[ね]が 聞[き]こえてくる。
\\	この文章はどことなく変だ。	
\\	この 文章[ぶんしょう]はどことなく 変[へん]だ。
\\	哲也のことが何となく気になる。	
\\	哲也[てつや]のことが 何[なん]となく 気[き]になる。
\\	ゆりは誰かれとなくよく人の面倒をみる。	
\\	ゆりは 誰[だれ]かれとなくよく 人[ひと]の 面倒[めんどう]をみる。
\\	それとなく飯田に聞いてみたらどう。	
\\	それとなく 飯田[いいだ]に 聞[き]いてみたらどう。
\\	切れ味のよい議論はともすると言葉の暴力となる。	
\\	切れ味[きれあじ]のよい 議論[ぎろん]はともすると 言葉[ことば]の 暴力[ぼうりょく]となる。
\\	親は自分の子供にはともすると点数が甘くなりがちだ。	
\\	親[おや]は 自分[じぶん]の 子供[こども]にはともすると 点数[てんすう]が 甘[あま]くなりがちだ。
\\	大きな組織はともするとセクショナリズムに陥る危険性がある。	
\\	大[おお]きな 組織[そしき]はともするとセクショナリズムに 陥[おちい]る 危険[きけん] 性[せい]がある。
\\	商品づくりにおける標準化活動はともすると社内設計部門から嫌われる。	
\\	商品[しょうひん]づくりにおける 標準[ひょうじゅん] 化[か] 活動[かつどう]はともすると 社内[しゃない] 設計[せっけい] 部門[ぶもん]から 嫌[きら]われる。
\\	日本ではともすると議論が感情的になってしまい、知的なディベートの場が育ちにくいと言われる。	
\\	日本[にほん]ではともすると 議論[ぎろん]が 感情[かんじょう] 的[てき]になってしまい、 知的[ちてき]なディベートの 場[ば]が 育[そだ]ちにくいと 言[い]われる。
\\	スポーツ選手がいい記録を出すと、世間の期待が高くなり、ともするとその重圧に押しつぶされるようなケースも出てくる。	
\\	スポーツ 選手[せんしゅ]がいい 記録[きろく]を 出[だ]すと、 世間[せけん]の 期待[きたい]が 高[たか]くなり、ともするとその 重圧[じゅうあつ]に 押[お]しつぶされるようなケースも 出[で]てくる。
\\	日本語が論理的な表現に不適だとは思わないが、日常会話的な表現を安易に使うと、ともすると論理が曖昧になりやすい。	
\\	日本語[にほんご]が 論理[ろんり] 的[てき]な 表現[ひょうげん]に 不適[ふてき]だとは 思[おも]わないが、 日常[にちじょう] 会話[かいわ] 的[てき]な 表現[ひょうげん]を 安易[あんい]に 使[つか]うと、ともすると 論理[ろんり]が 曖昧[あいまい]になりやすい。
\\	国際協力と言うと、ともすると富める国が貧しい国に援助を施すというように受け取られやすい。	
\\	国際[こくさい] 協力[きょうりょく]と 言[い]うと、ともすると 富[と]める 国[くに]が 貧[まず]しい 国[くに]に 援助[えんじょ]を 施[ほどこ]すというように 受け取[うけと]られやすい。
\\	従来の音楽教育はともすると学校の中だけの活動に終始しがちであった。	
\\	従来[じゅうらい]の 音楽[おんがく] 教育[きょういく]はともすると 学校[がっこう]の 中[なか]だけの 活動[かつどう]に 終始[しゅうし]しがちであった。
\\	長期不況のもとでは、ともすると様々な議論が極端な悲観論や危機論に傾きがちである。	
\\	長期[ちょうき] 不況[ふきょう]のもとでは、ともすると 様々[さまざま]な 議論[ぎろん]が 極端[きょくたん]な 悲観[ひかん] 論[ろん]や 危機[きき] 論[ろん]に 傾[かたむ]きがちである。
\\	現在の状況では、ともすると経済的な視点からの開発が先行し、貴重な資源を浪費したり、生活環境を損ないかねない。	
\\	現在[げんざい]の 状況[じょうきょう]では、ともすると 経済[けいざい] 的[てき]な 視点[してん]からの 開発[かいはつ]が 先行[せんこう]し、 貴重[きちょう]な 資源[しげん]を 浪費[ろうひ]したり、 生活[せいかつ] 環境[かんきょう]を 損[そこ]ないかねない。
\\	この国ではワインが安い。普段飲むものなどどうかするとミネラルウォーターより安い。	
\\	この 国[くに]ではワインが 安[やす]い。 普段[ふだん] 飲[の]むものなどどうかするとミネラルウォーターより 安[やす]い。
\\	東京の都心への通勤時間は通勤距離に比例しない。どうかすると他県から通う方が都内から通うよりずっと便利なこともある。	
\\	東京[とうきょう]の 都心[としん]への 通勤[つうきん] 時間[じかん]は 通勤[つうきん] 距離[きょり]に 比例[ひれい]しない。どうかすると 他[た] 県[けん]から 通[かよ]う 方[ほう]が 都内[とない]から 通[かよ]うよりずっと 便利[べんり]なこともある。
\\	経済情勢は悪化しているとの見方が優勢だ。	
\\	経済[けいざい] 情勢[じょうせい]は 悪化[あっか]しているとの 見方[みかた]が 優勢[ゆうせい]だ。
\\	邦画の入場者数がハリウッド映画を上回って1位になったのは初めてだとのことだ。	
\\	邦画[ほうが]の 入場[にゅうじょう] 者[しゃ] 数[すう]がハリウッド 映画[えいが]を 上回[うわまわ]って 
\\	位[い]になったのは 初[はじ]めてだとのことだ。
\\	伊豆半島付近で地震があったとの知らせを受けた。	
\\	伊豆半島[いずはんとう] 付近[ふきん]で 地震[じしん]があったとの 知[し]らせを 受[う]けた。
\\	探検隊が日本時間の13時30分にベースキャンプに到着したとの報告がもたらされた。	
\\	探検[たんけん] 隊[たい]が 日本[にほん] 時間[じかん]の 
\\	時[じ] 
\\	分[ぷん]にベースキャンプに 到着[とうちゃく]したとの 報告[ほうこく]がもたらされた。
\\	この研究テーマは社会のニーズに合っていないとの指摘がなされた。	
\\	この 研究[けんきゅう]テーマは 社会[しゃかい]のニーズに 合[あ]っていないとの 指摘[してき]がなされた。
\\	副大臣の数をあまり多くすると行政を混乱させるとの意見がある。	
\\	副[ふく] 大臣[だいじん]の 数[かず]をあまり 多[おお]くすると 行政[ぎょうせい]を 混乱[こんらん]させるとの 意見[いけん]がある。
\\	無断リンクは著作権侵害との見解が発表された。	
\\	無断[むだん]リンクは 著作[ちょさく] 権[けん] 侵害[しんがい]との 見解[けんかい]が 発表[はっぴょう]された。
\\	当地からは遠隔で不便との理由で患者さん自身が受診を希望しない場合が多い。	
\\	当地[とうち]からは 遠隔[えんかく]で 不便[ふべん]との 理由[りゆう]で 患者[かんじゃ]さん 自身[じしん]が 受診[じゅしん]を 希望[きぼう]しない 場合[ばあい]が 多[おお]い。
\\	新システムの評価はどのように行うのかとの質問に対し、ユーザーへのアンケートを行うとの回答があった。	
\\	新[しん]システムの 評価[ひょうか]はどのように 行[おこな]うのかとの 質問[しつもん]に 対[たい]し、ユーザーへのアンケートを 行[おこな]うとの 回答[かいとう]があった。
\\	銀行が中小企業の手数料を優遇との報道がありました。	
\\	銀行[ぎんこう]が 中小[ちゅうしょう] 企業[きぎょう]の 手数料[てすうりょう]を 優遇[ゆうぐう]との 報道[ほうどう]がありました。
\\	ローマ法王ヨハネ・パウロ2世が死去したとのニュースが今朝流れた。	
\\	ローマ法王[ろーまほうおう]ヨハネ・パウロ 
\\	世[せい]が 死去[しきょ]したとのニュースが 今朝[けさ] 流[なが]れた。
\\	ジェーンが日本へ行くという話は本当ですか。	
\\	ジェーンが 日本[にほん]へ 行[い]くという 話[はなし]は 本当[ほんとう]ですか。
\\	森下さんが「すぐお電話下さい」とのことです。	
\\	森下[もりした]さんが「すぐお 電話[でんわ] 下[くだ]さい」とのことです。
\\	この仕事はとても私にはできない。	
\\	この 仕事[しごと]はとても 私[わたし]にはできない。
\\	吉岡さんはとても私の勝てる相手ではない。	
\\	吉岡[よしおか]さんはとても 私[わたし]の 勝[か]てる 相手[あいて]ではない。
\\	この病気はとても治る見込みはない。	
\\	この 病気[びょうき]はとても 治[なお]る 見込[みこ]みはない。
\\	私は数学ではとても彼にかなわない。	
\\	私[わたし]は 数学[すうがく]ではとても 彼[かれ]にかなわない。
\\	こんな予算ではとてもこの学校を維持していけない。	
\\	こんな 予算[よさん]ではとてもこの 学校[がっこう]を 維持[いじ]していけない。
\\	今の大統領はとてもこの国を統治する能力はない。	
\\	今[いま]の 大統領[だいとうりょう]はとてもこの 国[くに]を 統治[とうち]する 能力[のうりょく]はない。
\\	この病人はとても助からない。	
\\	この 病人[びょうにん]はとても 助[たす]からない。
\\	この地方に雪が降ることはとてもあり得ない。	
\\	この 地方[ちほう]に 雪[ゆき]が 降[ふ]ることはとてもあり 得[え]ない。
\\	とても我々に勝ち目はない。	
\\	とても 我々[われわれ]に 勝[か]ち 目[め]はない。
\\	経済はとてもすぐには回復しそうにない。	
\\	経済[けいざい]はとてもすぐには 回復[かいふく]しそうにない。
\\	山岸さんはとても80歳には見えない。	
\\	山岸[やまぎし]さんはとても 
\\	歳[さい]には 見[み]えない。
\\	そんな危ないところにはとても子供を行かせられない。	
\\	そんな 危[あぶ]ないところにはとても 子供[こども]を 行[い]かせられない。
\\	その体ではとても旅行は無理だ。	
\\	その 体[からだ]ではとても 旅行[りょこう]は 無理[むり]だ。
\\	今年中の完成なんてとても不可能だ。	
\\	今年[ことし] 中[ちゅう]の 完成[かんせい]なんてとても 不可能[ふかのう]だ。
\\	私なんかっても駄目です。	
\\	私[わたし]なんかっても 駄目[だめ]です。
\\	彼が自分のベンチャー会社を立ち上げるとは思いもしなかった。	
\\	彼[かれ]が 自分[じぶん]のベンチャー 会社[かいしゃ]を 立[た]ち 上[あ]げるとは 思[おも]いもしなかった。
\\	あんないい店がなくなったとは残念だ。	
\\	あんないい 店[みせ]がなくなったとは 残念[ざんねん]だ。
\\	子供が生まれた途端にがんの宣告を受けるとは何という悲劇だ。	
\\	子供[こども]が 生[う]まれた 途端[とたん]にがんの 宣告[せんこく]を 受[う]けるとは 何[なん]という 悲劇[ひげき]だ。
\\	こんな便利なサイトがあったとは!	
\\	こんな 便利[べんり]なサイトがあったとは!
\\	お父さんがそんなにお悪いとは知りませんでした。	
\\	お 父[とう]さんがそんなにお 悪[わる]いとは 知[し]りませんでした。
\\	こんなに早く自分の家が持てるとは思いもしなかった。	
\\	こんなに 早[はや]く 自分[じぶん]の 家[いえ]が 持[も]てるとは 思[おも]いもしなかった。
\\	ハリケーンのためにガソリンがこんなに上がるとは想像すらしなかった。	
\\	ハリケーンのためにガソリンがこんなに 上[あ]がるとは 想像[そうぞう]すらしなかった。
\\	一つのテレビドラマが日韓交流にこれほど影響を与えるとは誰が予測し得たであろう。	
\\	一[ひと]つのテレビドラマが 日[にっ] 韓[かん] 交流[こうりゅう]にこれほど 影響[えいきょう]を 与[あた]えるとは 誰[だれ]が 予測[よそく]し 得[え]たであろう。
\\	蒸したてのお芋がこんなにおいしいとは驚きました。	
\\	蒸[む]したてのお 芋[いも]がこんなにおいしいとは 驚[おどろ]きました。
\\	まだ知り合って一週間にもならない私にお金を借りに来るとは驚きだ。	
\\	まだ 知り合[しりあ]って 一週間[いっしゅうかん]にもならない 私[わたし]にお 金[かね]を 借[か]りに 来[く]るとは 驚[おどろ]きだ。
\\	こんなにおいしくて、ボリュームがあって、たったの500円とはありがたい。	
\\	こんなにおいしくて、ボリュームがあって、たったの 
\\	円[えん]とはありがたい。
\\	大勢の人が家をなくして困っているというのに、それを利用して金をもうけるとはけしからん。	
\\	大勢[おおぜい]の 人[ひと]が 家[いえ]をなくして 困[こま]っているというのに、それを 利用[りよう]して 金[かね]をもうけるとはけしからん。
\\	バッハの最高傑作の演奏を
\\	で見られるとは何という幸せでしょう!	
\\	バッハの 最高[さいこう] 傑作[けっさく]の 演奏[えんそう]を 
\\	で 見[み]られるとは 何[なん]という 幸[しあわ]せでしょう!
\\	自分たちで嘘を広めておいて、それを既成事実にしようとは何たることだ!	
\\	自分[じぶん]たちで 嘘[うそ]を 広[ひろ]めておいて、それを 既成[きせい] 事実[じじつ]にしようとは 何[なん]たることだ!
\\	私は君に帰ってもいいとは言わなかった。	
\\	私[わたし]は 君[きみ]に 帰[かえ]ってもいいとは 言[い]わなかった。
\\	核兵器が戦争で使われるとは思わない。	
\\	核兵器[かくへいき]が 戦争[せんそう]で 使[つか]われるとは 思[おも]わない。
\\	あんな素敵なカップルが離婚だとは!	
\\	あんな 素敵[すてき]なカップルが 離婚[りこん]だとは!
\\	申し込み手続きがこんなに面倒とは思わなかった。	
\\	申し込[もうしこ]み 手続[てつづ]きがこんなに 面倒[めんどう]とは 思[おも]わなかった。
\\	彼がサウジアラビアへ転勤するとはねえ。	
\\	彼[かれ]がサウジアラビアへ 転勤[てんきん]するとはねえ。
\\	ケータイとは携帯電話のことだ。	
\\	ケータイとは 携帯[けいたい] 電話[でんわ]のことだ。
\\	夏とは言え、ここはまだちょっと肌寒いくらいの涼しさだ。	
\\	夏[なつ]とは 言[い]え、ここはまだちょっと 肌寒[はださむ]いくらいの 涼[すず]しさだ。
\\	時間がなかったとは言え、ジェーンにこれくらいのレポートが書けないはずはない。	
\\	時間[じかん]がなかったとは 言[い]え、ジェーンにこれくらいのレポートが 書[か]けないはずはない。
\\	今度の学会が楽しみだ。とは言え、まだ発表準備は全然できていない。	
\\	今度[こんど]の 学会[がっかい]が 楽[たの]しみだ。とは 言[い]え、まだ 発表[はっぴょう] 準備[じゅんび]は 全然[ぜんぜん]できていない。
\\	子供がやったこととは言え、これは立派な犯罪だ。	
\\	子供[こども]がやったこととは 言[い]え、これは 立派[りっぱ]な 犯罪[はんざい]だ。
\\	予想していた返答だったとは言え、やはりそれを聞いた時はショックだった。	
\\	予想[よそう]していた 返答[へんとう]だったとは 言[い]え、やはりそれを 聞[き]いた 時[とき]はショックだった。
\\	ノートブック型とは言え、ディスプレイは17インチ、プロセッサーは2.4
\\	、内臓ドライブは250
\\	もあり、
\\	コンボドライブ付きという優れものだ。	
\\	ノートブック 型[がた]とは 言[い]え、ディスプレイは 17インチ、プロセッサーは 
\\	、 内臓[ないぞう]ドライブは 
\\	もあり、 
\\	コンボドライブ 付[つ]きという 優[すぐ]れものだ。
\\	安いとは言え、都会の高級マンションだ。普通の者が簡単に買える代物ではない。	
\\	安[やす]いとは 言[い]え、 都会[とかい]の 高級[こうきゅう]マンションだ。 普通[ふつう]の 者[もの]が 簡単[かんたん]に 買[か]える 代物[しろもの]ではない。
\\	いかに暫定的処置とは言え、これはずさんすぎる。	
\\	いかに 暫定[ざんてい] 的[てき] 処置[しょち]とは 言[い]え、これはずさんすぎる。
\\	怪我をして動けないとは言え、ものを考えることはできる。	
\\	怪我[けが]をして 動[うご]けないとは 言[い]え、ものを 考[かんが]えることはできる。
\\	毎日走っているとは言え、1キロほどジョギングするだけだから大した運動にはならない。	
\\	毎日[まいにち] 走[はし]っているとは 言[い]え、 1キロほどジョギングするだけだから 大[たい]した 運動[うんどう]にはならない。
\\	退職したとは言え、毎日やることがたくさんあるので忙しい。	
\\	退職[たいしょく]したとは 言[い]え、 毎日[まいにち]やることがたくさんあるので 忙[いそが]しい。
\\	燃料電池は将来化石燃料に代わる主要エネルギーになるだろう。とは言え、これにはまだ解決すべき技術的な問題が多く残っている。	
\\	燃料[ねんりょう] 電池[でんち]は 将来[しょうらい] 化石[かせき] 燃料[ねんりょう]に 代[か]わる 主要[しゅよう]エネルギーになるだろう。とは 言[い]え、これにはまだ 解決[かいけつ]すべき 技術[ぎじゅつ] 的[てき]な 問題[もんだい]が 多[おお]く 残[のこ]っている。
\\	楽しかったのでつい飲みすぎてしまった。	
\\	楽[たの]しかったのでつい 飲[の]みすぎてしまった。
\\	ついさっきまで山口さんが来ていたんです。	
\\	ついさっきまで 山口[やまぐち]さんが 来[き]ていたんです。
\\	人のお金を使うのは悪いとは知りながら、つい使ってしまった。	
\\	人[ひと]のお 金[かね]を 使[つか]うのは 悪[わる]いとは 知[し]りながら、つい 使[つか]ってしまった。
\\	安かったのでつい買ってしまった。	
\\	安[やす]かったのでつい 買[か]ってしまった。
\\	恵美につい亜紀の秘密を話してしまった。	
\\	恵美[えみ]につい 亜紀[あき]の 秘密[ひみつ]を 話[はな]してしまった。
\\	彼の仕事を見ているとじれったいのでつい手を貸してしまう。	
\\	彼[かれ]の 仕事[しごと]を 見[み]ているとじれったいのでつい 手[て]を 貸[か]してしまう。
\\	あの人は面白いのでついからかいたくなる。	
\\	あの 人[ひと]は 面白[おもしろ]いのでついからかいたくなる。
\\	この店は感じがいいのでつい入ってみたくなる。	
\\	この 店[みせ]は 感[かん]じがいいのでつい 入[い]ってみたくなる。
\\	ついこの間、新年を祝ったと思ったら今日はもうバレンタインデーだ。	
\\	ついこの 間[あいだ]、 新年[しんねん]を 祝[いわ]ったと 思[おも]ったら 今日[きょう]はもうバレンタインデーだ。
\\	彼女からは、つい2,3日前にメールをもらった。	
\\	彼女[かのじょ]からは、つい 
\\	日[にち] 前[まえ]にメールをもらった。
\\	ついそこに喫茶店がありますからそこで話しましょう。	
\\	ついそこに 喫茶店[きっさてん]がありますからそこで 話[はな]しましょう。
\\	その家の見張りをしていないといけないのにうっかり居眠りをしてしまった。	
\\	その 家[いえ]の 見張[みは]りをしていないといけないのにうっかり 居眠[いねむ]りをしてしまった。
\\	うっかり違う物を買ってしまった。	
\\	うっかり 違[ちが]う 物[もの]を 買[か]ってしまった。
\\	うっかり自分の駅を乗り過ごした。	
\\	うっかり 自分[じぶん]の 駅[えき]を 乗り過[のりす]ごした。
\\	うっかりアポイントメントを忘れてしまった。	
\\	うっかりアポイントメントを 忘[わす]れてしまった。
\\	応援している力士が負けそうになって、思わず力が入った。	
\\	応援[おうえん]している 力士[りきし]が 負[ま]けそうになって、 思[おも]わず 力[ちから]が 入[い]った。
\\	目の前に子供が飛び出したので思わずクラクションを鳴らした。	
\\	目[め]の 前[まえ]に 子供[こども]が 飛び出[とびだ]したので 思[おも]わずクラクションを 鳴[な]らした。
\\	歯医者に痛い歯を触られて思わず悲鳴を上げた。	
\\	歯医者[はいしゃ]に 痛[いた]い 歯[は]を 触[さわ]られて 思[おも]わず 悲鳴[ひめい]を 上[あ]げた。
\\	京都へ来たついでに龍安寺の石庭を見に行こうと思っている。	
\\	京都[きょうと]へ 来[き]たついでに 龍安寺[りょうあんじ]の 石庭[せきてい]を 見[み]に 行[い]こうと 思[おも]っている。
\\	頼みついでにもう一つお願いしてもいいですか。	
\\	頼[たの]みついでにもう 一[ひと]つお 願[ねが]いしてもいいですか。
\\	この仕事は大変だが、大変ついでに来週の分もいま片づけておけば後が楽だ。	
\\	この 仕事[しごと]は 大変[たいへん]だが、 大変[たいへん]ついでに 来週[らいしゅう]の 分[ぶん]もいま 片[かた]づけておけば 後[ご]が 楽[らく]だ。
\\	生活費を抑えるために電気や水を節約し始めたが、節約ついでにエアコンも止めることにした。	
\\	生活[せいかつ] 費[ひ]を 抑[おさ]えるために 電気[でんき]や 水[みず]を 節約[せつやく]し 始[はじ]めたが、 節約[せつやく]ついでにエアコンも 止[や]めることにした。
\\	話のついでにこのことも言っておきます。	
\\	話[はなし]のついでにこのことも 言[い]っておきます。
\\	コピーしに行くの?じゃ、これもついでにお願いできる?	
\\	コピーしに 行[い]くの?じゃ、これもついでにお 願[ねが]いできる?
\\	バンクーバーへ行くついでにシアトルの友達のところに寄る予定だ。	
\\	バンクーバーへ 行[い]くついでにシアトルの 友達[ともだち]のところに 寄[よ]る 予定[よてい]だ。
\\	車を洗ったついでに中も掃除しておいた。	
\\	車[くるま]を 洗[あら]ったついでに 中[なか]も 掃除[そうじ]しておいた。
\\	叱られついでに、もう一つの失敗のことも話してしまいます。	
\\	叱[しか]られついでに、もう 一[ひと]つの 失敗[しっぱい]のことも 話[はな]してしまいます。
\\	恥かきついでにもう一つお聞きしますが、アメリカも大学入試はあるんですか。	
\\	恥[はじ]かきついでにもう 一[ひと]つお 聞[き]きしますが、アメリカも 大学[だいがく] 入試[にゅうし]はあるんですか。
\\	今回は高級ホテルに泊まっているので、贅沢ついでにホテルのレストランで食事をすることにした。	
\\	今回[こんかい]は 高級[こうきゅう]ホテルに 泊[と]まっているので、 贅沢[ぜいたく]ついでにホテルのレストランで 食事[しょくじ]をすることにした。
\\	ご苦労ついでにこの翻訳も手伝ってもらえませんか。	
\\	ご 苦労[くろう]ついでにこの 翻訳[ほんやく]も 手伝[てつだ]ってもらえませんか。
\\	厚かましいついでにお願いがもう一つあります。	
\\	厚[あつ]かましいついでにお 願[ねが]いがもう 一[ひと]つあります。
\\	新しい電気自動車の取材ついでに自分も少し運転させてもらった。	
\\	新[あたら]しい 電気[でんき] 自動車[じどうしゃ]の 取材[しゅざい]ついでに 自分[じぶん]も 少[すこ]し 運転[うんてん]させてもらった。
\\	買い物のついでに銀行に寄って少し現金を下ろしていきます。	
\\	買い物[かいもの]のついでに 銀行[ぎんこう]に 寄[よ]って 少[すこ]し 現金[げんきん]を 下[お]ろしていきます。
\\	以上が今週の予定ですが、ついでに来週の予定も言っておきます。	
\\	以上[いじょう]が 今週[こんしゅう]の 予定[よてい]ですが、ついでに 来週[らいしゅう]の 予定[よてい]も 言[い]っておきます。
\\	明日用事で銀座に行くので、そのついでにデパートに寄って少し買い物をしようと思う。	
\\	明日[あした] 用事[ようじ]で 銀座[ぎんざ]に 行[い]くので、そのついでにデパートに 寄[よ]って 少[すこ]し 買い物[かいもの]をしようと 思[おも]う。
\\	お金はついでの時に返してもらえば結構です。	
\\	お 金[かね]はついでの 時[とき]に 返[かえ]してもらえば 結構[けっこう]です。
\\	渡したいものがあるので、ついでの折でいいですから私のオフィスに寄ってもらえますか。	
\\	渡[わた]したいものがあるので、ついでの 折[おり]でいいですから 私[わたし]のオフィスに 寄[よ]ってもらえますか。
\\	コンピュータウイルスが全国的に広がっています。ついては、不振なメールが届いた場合は、添付ファイルを聞くことなく当該メールを削除して下さい。	
\\	コンピュータウイルスが 全国[ぜんこく] 的[てき]に 広[ひろ]がっています。ついては、 不振[ふしん]なメールが 届[とど]いた 場合[ばあい]は、 添付[てんぷ]ファイルを 聞[き]くことなく 当該[とうがい]メールを 削除[さくじょ]して 下[くだ]さい。
\\	この書類は契約分の日本語抄訳です。ついては、本訳をご参照の上、ウェブの英文契約書にご同意いただく必要があります。	
\\	この 書類[しょるい]は 契約[けいやく] 分[ぶん]の 日本語[にほんご] 抄訳[しょうやく]です。ついては、 本[ほん] 訳[やく]をご 参照[さんしょう]の 上[うえ]、ウェブの 英文[えいぶん] 契約[けいやく] 書[しょ]にご 同意[どうい]いただく 必要[ひつよう]があります。
\\	11月分授業料は10月15日に指定預金口座から引き落とさせていただきます。ついては、前日までに預金残高の確認をお願いします。	
\\	月[がつ] 分[ぶん] 授業[じゅぎょう] 料[りょう]は 
\\	月[がつ] 
\\	日[か]に 指定[してい] 預金[よきん] 口座[こうざ]から 引き落[ひきお]とさせていただきます。ついては、 前日[ぜんじつ]までに 預金[よきん] 残高[ざんだか]の 確認[かくにん]をお 願[ねが]いします。
\\	夏の省エネ対策について本社から通知がありました。ついては、別紙を参照の上、夏季のエネルギー節約にご協力をお願いいたします。	
\\	夏[なつ]の 省エネ[しょうえね] 対策[たいさく]について 本社[ほんしゃ]から 通知[つうち]がありました。ついては、 別紙[べっし]を 参照[さんしょう]の 上[うえ]、 夏季[かき]のエネルギー 節約[せつやく]にご 協力[きょうりょく]をお 願[ねが]いいたします。
\\	深刻な雇用問題に対しては、再就職や雇用流動性を促進させる必要がある。ついては、以下の施策を積極的に展開されたい。	
\\	深刻[しんこく]な 雇用[こよう] 問題[もんだい]に 対[たい]しては、 再[さい] 就職[しゅうしょく]や 雇用[こよう] 流動[りゅうどう] 性[せい]を 促進[そくしん]させる 必要[ひつよう]がある。ついては、 以下[いか]の 施策[しさく]を 積極[せっきょく] 的[てき]に 展開[てんかい]されたい。
\\	新しくインターネット販売の事業を始める計画です。つきましては、その資金を融資していただきたいのです。	
\\	新[あたら]しくインターネット 販売[はんばい]の 事業[じぎょう]を 始[はじ]める 計画[けいかく]です。つきましては、その 資金[しきん]を 融資[ゆうし]していただきたいのです。
\\	今月は「リサイクル強化月間」になっています。つきましては、皆様方のご理解とご協力をお願いいたします。	
\\	今月[こんげつ]は「リサイクル 強化[きょうか] 月間[げっかん]」になっています。つきましては、 皆様[みなさま] 方[がた]のご 理解[りかい]とご 協力[きょうりょく]をお 願[ねが]いいたします。
\\	長い文は耳で聞いた場合、理解するのが難しい。そのためテレビやラジオのニュースでは短い文が使われる。	
\\	長[なが]い 文[ふみ]は 耳[みみ]で 聞[き]いた 場合[ばあい]、 理解[りかい]するのが 難[むずか]しい。そのためテレビやラジオのニュースでは 短[みじか]い 文[ふみ]が 使[つか]われる。
\\	私の父は私が3つの時に交通事故で亡くなった。そのため、私は父の記憶がほとんどない。	
\\	私[わたし]の 父[ちち]は 私[わたし]が 3つの 時[とき]に 交通[こうつう] 事故[じこ]で 亡[な]くなった。そのため、 私[わたし]は 父[ちち]の 記憶[きおく]がほとんどない。
\\	つまらないミスで試合に負けて悔しいと言ったらない。	
\\	つまらないミスで 試合[しあい]に 負[ま]けて 悔[くや]しいと 言[い]ったらない。
\\	老眼が進んで眼鏡なしでは近くの字が読めないので、不便ったらない。	
\\	老眼[ろうがん]が 進[すす]んで 眼鏡[めがね]なしでは 近[ちか]くの 字[じ]が 読[よ]めないので、 不便[ふべん]ったらない。
\\	大きな書類を保存するのを忘れてコンピュータがフリーズした時の情けなさと言ったらない。	
\\	大[おお]きな 書類[しょるい]を 保存[ほぞん]するのを 忘[わす]れてコンピュータがフリーズした 時[とき]の 情[なさ]けなさと 言[い]ったらない。
\\	この辞書は使いにくいことったらない。	
\\	この 辞書[じしょ]は 使[つか]いにくいことったらない。
\\	5分ごとにジェット機が離陸するのでうるさいと言ったらない。	
\\	分[ふん]ごとに ジェット機[じぇっとき]が 離陸[りりく]するのでうるさいと 言[い]ったらない。
\\	いつもよくしゃべる俊子が慎一の前で赤くなって何も言えないので、おかしいったらなかった。	
\\	いつもよくしゃべる 俊子[としこ]が 慎一[しんいち]の 前[まえ]で 赤[あか]くなって 何[なに]も 言[い]えないので、おかしいったらなかった。
\\	ティムにデートをすっぷかされて頭に来るったらない。	
\\	ティムにデートをすっぷかされて 頭[あたま]に 来[く]るったらない。
\\	この仕事は一日中細かい字を見ているので目が疲れると言ったらない。	
\\	この 仕事[しごと]は一 日[にち] 中[じゅう] 細[こま]かい 字[じ]を 見[み]ているので 目[め]が 疲[つか]れると 言[い]ったらない。
\\	出張したところはインターネットも携帯も使えず、不便と言ったらなかった。	
\\	出張[しゅっちょう]したところはインターネットも 携帯[けいたい]も 使[つか]えず、 不便[ふべん]と 言[い]ったらなかった。
\\	プロジェクトに変更があるたびに関係者全員に知らせた上で承認を得なければならないので、面倒と言ったらない。	
\\	プロジェクトに 変更[へんこう]があるたびに 関係[かんけい] 者[しゃ] 全員[ぜんいん]に 知[し]らせた 上[うえ]で 承認[しょうにん]を 得[え]なければならないので、 面倒[めんどう]と 言[い]ったらない。
\\	夏、暑い外から帰ってきた後の冷えたビールのうまさったらない。	
\\	夏[なつ]、 暑[あつ]い 外[そと]から 帰[かえ]ってきた 後[あと]の 冷[ひ]えたビールのうまさったらない。
\\	この国ではインターネットが自由に使えないので、その不便さったらない。	
\\	この 国[くに]ではインターネットが 自由[じゆう]に 使[つか]えないので、その 不便[ふべん]さったらない。
\\	ハイヒールで砂利道を歩かされたので歩きにくいことと言ったらなかった。	
\\	ハイヒールで 砂利[じゃり] 道[みち]を 歩[ある]かされたので 歩[ある]きにくいことと 言[い]ったらなかった。
\\	雪に覆われた山々が早朝の朝日を受けて宝石のように輝く姿の美しいことと言ったらない。	
\\	雪[ゆき]に 覆[おお]われた 山々[やまやま]が 早朝[そうちょう]の 朝日[あさひ]を 受[う]けて 宝石[ほうせき]のように 輝[かがや]く 姿[すがた]の 美[うつく]しいことと 言[い]ったらない。
\\	昔、痛風にかかったことがあるが、その痛みと言ったらなかった。	
\\	昔[むかし]、 痛風[つうふう]にかかったことがあるが、その 痛[いた]みと 言[い]ったらなかった。
\\	学生がいいのは、学期が終わったら次の学期の始まりまでは自由なことだ。試験やレポートが全部終わった時の開放感と言ったらない。	
\\	学生[がくせい]がいいのは、 学期[がっき]が 終[お]わったら 次[つぎ]の 学期[がっき]の 始[はじ]まりまでは 自由[じゆう]なことだ。 試験[しけん]やレポートが 全部[ぜんぶ] 終[お]わった 時[とき]の 開放[かいほう] 感[かん]と 言[い]ったらない。
\\	この辞書は使いにくいことと言ったら言いようがない。	
\\	この 辞書[じしょ]は 使[つか]いにくいことと 言[い]ったら 言[い]いようがない。
\\	良夫の歌い方ったらない。	
\\	良夫[よしお]の 歌[うた]い 方[かた]ったらない。
\\	彼がその知らせを聞いた時の顔と言ったらなかった。	
\\	彼[かれ]がその 知[し]らせを 聞[き]いた 時[とき]の 顔[かお]と 言[い]ったらなかった。
\\	つまらないミスで試合に負けて本当に悔しい。	
\\	つまらないミスで 試合[しあい]に 負[ま]けて 本当[ほんとう]に 悔[くや]しい。
\\	老眼が進んで眼鏡なしでは近くの字が読めないので、本当に不便だ。	
\\	老眼[ろうがん]が 進[すす]んで 眼鏡[めがね]なしでは 近[ちか]くの 字[じ]が 読[よ]めないので、 本当[ほんとう]に 不便[ふべん]だ。
\\	大きな書類を保存するのを忘れてコンピュータがフリーズした時に本当に情けない。	
\\	大[おお]きな 書類[しょるい]を 保存[ほぞん]するのを 忘[わす]れてコンピュータがフリーズした 時[とき]に 本当[ほんとう]に 情[なさ]けない。
\\	この辞書は本当に使いにくい。	
\\	この 辞書[じしょ]は 本当[ほんとう]に 使[つか]いにくい。
\\	その発表は衝撃的と言ったらなかった。	
\\	その 発表[はっぴょう]は 衝撃[しょうげき] 的[てき]と 言[い]ったらなかった。
\\	インターフェースに若干の違いはあれ、この二つのプログラムは機能的には全く変わらない。	
\\	インターフェースに 若干[じゃっかん]の 違[ちが]いはあれ、この 二[ふた]つのプログラムは 機能[きのう] 的[てき]には 全[まった]く 変[か]わらない。
\\	歴史の違いはあれ、米国には120万を超す非営利団体
\\	が存在します。	
\\	歴史[れきし]の 違[ちが]いはあれ、 米国[べいこく]には 
\\	万[まん]を 超[こ]す 非[ひ] 営利[えいり] 団体[だんたい]
\\	が 存在[そんざい]します。
\\	ほとんどの者が、程度の差はあれ、マスメディアの影響を受けていると言える。	
\\	ほとんどの 者[もの]が、 程度[ていど]の 差[さ]はあれ、マスメディアの 影響[えいきょう]を 受[う]けていると 言[い]える。
\\	レベルに違いはあれ、この課にいる者は全員英語を話せる。	
\\	レベルに 違[ちが]いはあれ、この 課[か]にいる 者[もの]は 全員[ぜんいん] 英語[えいご]を 話[はな]せる。
\\	海外旅行に興味はあれ、先立つ費用がない。	
\\	海外[かいがい] 旅行[りょこう]に 興味[きょうみ]はあれ、 先立[さきだ]つ 費用[ひよう]がない。
\\	若干の不満はあれ、今の職場を離れるつもりはない。	
\\	若干[じゃっかん]の 不満[ふまん]はあれ、 今[いま]の 職場[しょくば]を 離[はな]れるつもりはない。
\\	今まで自分を使ってくれた上司に対して感謝する気持ちはあれ、非難する気持ちは全くありません。	
\\	今[いま]まで 自分[じぶん]を 使[つか]ってくれた 上司[じょうし]に 対[たい]して 感謝[かんしゃ]する 気持[きも]ちはあれ、 非難[ひなん]する 気持[きも]ちは 全[まった]くありません。
\\	アンダーソン氏は海外生活の経験はあれど、アジアを経験しておらず、今回の中国赴任に若干お不安を残す。	
\\	アンダーソン 氏[し]は 海外[かいがい] 生活[せいかつ]の 経験[けいけん]はあれど、アジアを 経験[けいけん]しておらず、 今回[こんかい]の 中国[ちゅうごく] 赴任[ふにん]に 若干[じゃっかん]お 不安[ふあん]を 残[のこ]す。
\\	ゴマには白ゴマと黒ゴマのほか、黄ゴマという風味豊かなゴマもある。見た目の色の違えはあれど、栄養的な違いは微々たるものだ。	
\\	ゴマには 白[しろ]ゴマと 黒[くろ]ゴマのほか、 黄[き]ゴマという 風味[ふうみ] 豊[ゆた]かなゴマもある。 見た目[みため]の 色[いろ]の 違[ちが]えはあれど、 栄養[えいよう] 的[てき]な 違[ちが]いは 微々[びび]たるものだ。
\\	インターフェースに若干の違いこそあれ、この二つのプログラムは機能的には全く変わらない。	
\\	インターフェースに 若干[じゃっかん]の 違[ちが]いこそあれ、この 二[ふた]つのプログラムは 機能[きのう] 的[てき]には 全[まった]く 変[か]わらない。
\\	今まで自分を使ってくれた上司に対して感謝する気持ちこそあれ、非難する気持ちは全くありません。	
\\	今[いま]まで 自分[じぶん]を 使[つか]ってくれた 上司[じょうし]に 対[たい]して 感謝[かんしゃ]する 気持[きも]ちこそあれ、 非難[ひなん]する 気持[きも]ちは 全[まった]くありません。
\\	特殊な場合は別として、手書きのレポートを書くことは少なくなった。	
\\	特殊[とくしゅ]な 場合[ばあい]は 別[べつ]として、 手書[てが]きのレポートを 書[か]くことは 少[すく]なくなった。
\\	実施できるかどうかは別として、まず可能な解決策をすべてリストアップしてみよう。	
\\	実施[じっし]できるかどうかは 別[べつ]として、まず 可能[かのう]な 解決[かいけつ] 策[さく]をすべてリストアップしてみよう。
\\	上手か下手かは別としてその人にまず翻訳してもらおう。	
\\	上手[じょうず]か 下手[へた]かは 別[べつ]としてその 人[ひと]にまず 翻訳[ほんやく]してもらおう。
\\	面白い面白くないは別としてこの授業はきっと将来役に立つ。	
\\	面白[おもしろ]い 面白[おもしろ]くないは 別[べつ]としてこの 授業[じゅぎょう]はきっと 将来[しょうらい] 役に立[やくにた]つ。
\\	医師により薬として使用される場合は別として、麻薬は本質的に反社会的なものとされている。	
\\	医師[いし]により 薬[くすり]として 使用[しよう]される 場合[ばあい]は 別[べつ]として、 麻薬[まやく]は 本質[ほんしつ] 的[てき]に 反[はん] 社会[しゃかい] 的[てき]なものとされている。
\\	一部の過激派は別として、一般大衆は民主化を望んでいる。	
\\	一部[いちぶ]の 過激[かげき] 派[は]は 別[べつ]として、 一般[いっぱん] 大衆[たいしゅう]は 民主[みんしゅ] 化[か]を 望[のぞ]んでいる。
\\	その映画は、ストーリーは別として、音楽やアクションは楽しめた。	
\\	その 映画[えいが]は、ストーリーは 別[べつ]として、 音楽[おんがく]やアクションは 楽[たの]しめた。
\\	賛成していただけるかどうかは別として、私の率直な意見を述べさせていただきます。	
\\	賛成[さんせい]していただけるかどうかは 別[べつ]として、 私[わたし]の 率直[そっちょく]な 意見[いけん]を 述[の]べさせていただきます。
\\	そういうところが実際にあるかどうかは別として、自分の理想とする会社はどんなところかを書いて下さい。	
\\	そういうところが 実際[じっさい]にあるかどうかは 別[べつ]として、 自分[じぶん]の 理想[りそう]とする 会社[かいしゃ]はどんなところかを 書[か]いて 下[くだ]さい。
\\	ケーブル接続かワイヤレスかは別として、今時インターネットに接続されていないコンピュータはほとんどないと思う。	
\\	ケーブル 接続[せつぞく]かワイヤレスかは 別[べつ]として、 今時[いまどき]インターネットに 接続[せつぞく]されていないコンピュータはほとんどないと 思[おも]う。
\\	いい悪いは別として、この国では患者本人に病状の事実を伝えるのが普通だ。	
\\	いい 悪[わる]いは 別[べつ]として、この 国[くに]では 患者[かんじゃ] 本人[ほんにん]に 病状[びょうじょう]の 事実[じじつ]を 伝[つた]えるのが 普通[ふつう]だ。
\\	この賞は、生活者の目に触れるか触れないかは別として、人々の生活を豊かにする技術や発明に対して与えられる。	
\\	この 賞[しょう]は、 生活[せいかつ] 者[しゃ]の 目[め]に 触[ふ]れるか 触[ふ]れないかは 別[べつ]として、 人々[ひとびと]の 生活[せいかつ]を 豊[ゆた]かにする 技術[ぎじゅつ]や 発明[はつめい]に 対[たい]して 与[あた]えられる。
\\	単身赴任など仕事上の理由がある場合は別として、夫婦が5年間も別居しているのは尋常ではない。	
\\	単身[たんしん] 赴任[ふにん]など 仕事[しごと] 上[じょう]の 理由[りゆう]がある 場合[ばあい]は 別[べつ]として、 夫婦[ふうふ]が 
\\	年間[ねんかん]も 別居[べっきょ]しているのは 尋常[じんじょう]ではない。
\\	掲示板やチャンとで発言する際は、ごく親しい仲間内の場合は別として、丁寧な言葉遣いを心がけるべきだ。	
\\	掲示板[けいじばん]やチャンとで 発言[はつげん]する 際[さい]は、ごく 親[した]しい 仲間[なかま] 内[うち]の 場合[ばあい]は 別[べつ]として、 丁寧[ていねい]な 言葉[ことば] 遣[づか]いを 心[こころ]がけるべきだ。
\\	明日は午前中を除いて塞がっています。	
\\	明日[あした]は 午前[ごぜん] 中[じゅう]を 除[のぞ]いて 塞[ふさ]がっています。
\\	この店は月曜日以外は週六日間営業しています。	
\\	この 店[みせ]は 月曜日[げつようび] 以外[いがい]は 週[しゅう] 六日間[むいかかん] 営業[えいぎょう]しています。
\\	友達を助けようという動機はいいとしてもやり方が間違っている。	
\\	友達[ともだち]を 助[たす]けようという 動機[どうき]はいいとしてもやり 方[かた]が 間違[まちが]っている。
\\	私はいいとしても他の人がこの部屋では満足しないだろう。	
\\	私[わたし]はいいとしても 他[ほか]の 人[ひと]がこの 部屋[へや]では 満足[まんぞく]しないだろう。
\\	部屋が狭いのはいいとしても、エアコンがないのは困る。	
\\	部屋[へや]が 狭[せま]いのはいいとしても、エアコンがないのは 困[こま]る。
\\	著作権のあるものは、個人用のコピーはいいとしても、それを他人に配布することは違法になります。	
\\	著作[ちょさく] 権[けん]のあるものは、 個人[こじん] 用[よう]のコピーはいいとしても、それを 他人[たにん]に 配布[はいふ]することは 違法[いほう]になります。
\\	街頭はいいとしても、電車やレストランなどの公共の場所で声高に携帯電話で話されるのは周りのものにとって極めて迷惑だ。	
\\	街頭[がいとう]はいいとしても、 電車[でんしゃ]やレストランなどの 公共[こうきょう]の 場所[ばしょ]で 声高[こわだか]に 携帯[けいたい] 電話[でんわ]で 話[はな]されるのは 周[まわ]りのものにとって 極[きわ]めて 迷惑[めいわく]だ。
\\	このような結婚式では、結婚する本人たちはいいとしても両親が承知しないだろう。	
\\	このような 結婚式[けっこんしき]では、 結婚[けっこん]する 本人[ほんにん]たちはいいとしても 両親[りょうしん]が 承知[しょうち]しないだろう。
\\	この店は安いから味が少しくらい悪いのはまあいいとしても、古い材料を使うのはやめてもらいたい。	
\\	この 店[みせ]は 安[やす]いから 味[あじ]が 少[すこ]しくらい 悪[わる]いのはまあいいとしても、 古[ふる]い 材料[ざいりょう]を 使[つか]うのはやめてもらいたい。
\\	今の社宅は一般の賃貸に比べて格段に安いのであまり文句は言えないのだが、建物が古いのはいいとしても、傷んだところをなかなか補修してもらえないのは問題だ。	
\\	今[いま]の 社宅[しゃたく]は 一般[いっぱん]の 賃貸[ちんたい]に 比[くら]べて 格段[かくだん]に 安[やす]いのであまり 文句[もんく]は 言[い]えないのだが、 建物[たてもの]が 古[ふる]いのはいいとしても、 傷[いた]んだところをなかなか 補修[ほしゅう]してもらえないのは 問題[もんだい]だ。
\\	ウイルスのために自分自身のコンピュータが使えなくなるのはいいとしても、何も対処しないで他の人にウイルスをまき散らすことは許されない。	
\\	ウイルスのために 自分[じぶん] 自身[じしん]のコンピュータが 使[つか]えなくなるのはいいとしても、 何[なに]も 対処[たいしょ]しないで 他[ほか]の 人[ひと]にウイルスをまき 散[ち]らすことは 許[ゆる]されない。
\\	部屋が狭いのはいいとして、エアコンがないのは困る。	
\\	部屋[へや]が 狭[せま]いのはいいとして、エアコンがないのは 困[こま]る。
\\	部屋が狭いのは仕方がないとしても、エアコンがないのは困る。	
\\	部屋[へや]が 狭[せま]いのは 仕方[しかた]がないとしても、エアコンがないのは 困[こま]る。
\\	インターネットはおろか、パソコンも触ったことがないのだが、講習会についていけるだろうか。	
\\	インターネットはおろか、パソコンも 触[さわ]ったことがないのだが、 講習[こうしゅう] 会[かい]についていけるだろうか。
\\	最近は大学生はおろか、中学生でもかなりの者が海外旅行を経験している。	
\\	最近[さいきん]は 大学生[だいがくせい]はおろか、 中学生[ちゅうがくせい]でもかなりの 者[もの]が 海外[かいがい] 旅行[りょこう]を 経験[けいけん]している。
\\	アメリカでは、隣町まで100マイル、途中ガソリンスタンドはおろか、民家さえないという場所がたくさんある。	
\\	アメリカでは、 隣[となり] 町[まち]まで100マイル、 途中[とちゅう]ガソリンスタンドはおろか、 民家[みんか]さえないという 場所[ばしょ]がたくさんある。
\\	ハリス氏は日本の現代文学はおろか、古典にまで通じている。	
\\	ハリス 氏[し]は 日本[にほん]の 現代[げんだい] 文学[ぶんがく]はおろか、 古典[こてん]にまで 通[つう]じている。
\\	「近くて遠い」。この表現は日本との国交がない北朝鮮はおろか、韓国との関係を指すのにもよく使われてきた。	
\\	近[ちか]くて 遠[とお]い」。この 表現[ひょうげん]は 日本[にほん]との 国交[こっこう]がない 北朝鮮[きたちょうせん]はおろか、 韓国[かんこく]との 関係[かんけい]を 指[さ]すのにもよく 使[つか]われてきた。
\\	英語を話すのが好きなだけで、英米の文学はおろか日本の文学もろくに読んでいない学生が英米文学専攻に入ってくると悲惨だ。	
\\	英語[えいご]を 話[はな]すのが 好[す]きなだけで、 英[えい] 米[べい]の 文学[ぶんがく]はおろか 日本[にほん]の 文学[ぶんがく]もろくに 読[よ]んでいない 学生[がくせい]が 英[えい] 米[べい] 文学[ぶんがく] 専攻[せんこう]に 入[はい]ってくると 悲惨[ひさん]だ。
\\	企業は今、セキュリティのための支出を増やさなければ、セキュリティの強化はおろか、維持さえも難しくなっているという。	
\\	企業[きぎょう]は 今[いま]、セキュリティのための 支出[ししゅつ]を 増[ふ]やさなければ、セキュリティの 強化[きょうか]はおろか、 維持[いじ]さえも 難[むずか]しくなっているという。
\\	このサイトは検索はおろか、本の内容紹介までしてくれるので非常に便利だ。	
\\	このサイトは 検索[けんさく]はおろか、 本[ほん]の 内容[ないよう] 紹介[しょうかい]までしてくれるので 非常[ひじょう]に 便利[べんり]だ。
\\	戦争はおろか食糧難すら経験したことがない者が、口先だけで戦争を議論しているのは笑止だ。	
\\	戦争[せんそう]はおろか 食糧難[しょくりょうなん]すら 経験[けいけん]したことがない 者[もの]が、 口先[くちさき]だけで 戦争[せんそう]を 議論[ぎろん]しているのは 笑止[しょうし]だ。
\\	私の上司は自信よく仕事をするだけでなく部下の面倒もよくみてくれる。	
\\	私[わたし]の 上司[じょうし]は 自信[じしん]よく 仕事[しごと]をするだけでなく 部下[ぶか]の 面倒[めんどう]もよくみてくれる。
\\	このレストランはおいしいのみならず値段も安い。	
\\	このレストランはおいしいのみならず 値段[ねだん]も 安[やす]い。
\\	この辺りは不便なだけでなく環境も悪い。	
\\	この 辺[あた]りは 不便[ふべん]なだけでなく 環境[かんきょう]も 悪[わる]い。
\\	この店のサービスランチは値段の割にはおいしくてボリュームがある。	
\\	この 店[みせ]のサービスランチは 値段[ねだん]の 割[わり]にはおいしくてボリュームがある。
\\	ビルはよく勉強している割に成績が伸びない。	
\\	ビルはよく 勉強[べんきょう]している 割[わり]に 成績[せいせき]が 伸[の]びない。
\\	このチームは強い割には人気がない。	
\\	このチームは 強[つよ]い 割[わり]には 人気[にんき]がない。
\\	彼は大学で秀才だった割に実社会であまり成功していない。	
\\	彼[かれ]は 大学[だいがく]で 秀才[しゅうさい]だった 割[わり]に 実社会[じっしゃかい]であまり 成功[せいこう]していない。
\\	彼女はすばらしい研究をしているが、その割には認められていない。	
\\	彼女[かのじょ]はすばらしい 研究[けんきゅう]をしているが、その 割[わり]には 認[みと]められていない。
\\	清水さんは年の割には若く見える。	
\\	清水[しみず]さんは 年[とし]の 割[わり]には 若[わか]く 見[み]える。
\\	この辺りはいつも海からの風があるので、強い日射しの割には過ごしやすい。	
\\	この 辺[あた]りはいつも 海[うみ]からの 風[かぜ]があるので、 強[つよ]い 日射[ひざ]しの 割[わり]には 過[す]ごしやすい。
\\	このソフトは安い割には機能がよく、強い安いです。	
\\	このソフトは 安[やす]い 割[わり]には 機能[きのう]がよく、 強[つよ]い 安[やす]いです。
\\	この図書館は建物が立派な割には中の設備がよくない。	
\\	この 図書館[としょかん]は 建物[たてもの]が 立派[りっぱ]な 割[わり]には 中[なか]の 設備[せつび]がよくない。
\\	多くの人が「有機野菜」表示を気にする割には、その意味を知っている人は少ない。	
\\	多[おお]くの 人[ひと]が
\\	有機[ゆうき] 野菜[やさい]」 表示[ひょうじ]を 気[き]にする 割[わり]には、その 意味[いみ]を 知[し]っている 人[ひと]は 少[すく]ない。
\\	シアトルなどの都市がある北部太平洋岸は比較的暖かく、緯度が北海道と変わらない割には冬の寒さも厳しくない。	
\\	シアトルなどの 都市[とし]がある 北部[ほくぶ] 太平洋[たいへいよう] 岸[がん]は 比較的[ひかくてき] 暖[あたた]かく、 緯度[いど]が 北海道[ほっかいどう]と 変[か]わらない 割[わり]には 冬[ふゆ]の 寒[さむ]さも 厳[きび]しくない。
\\	記念すべきイベントの割には参加者が少なく物足りない気がする。	
\\	記念[きねん]すべきイベントの 割[わり]には 参加[さんか] 者[しゃ]が 少[すく]なく 物足[ものた]りない 気[き]がする。
\\	彼は自国の大学で英語を教えていると言っているが、その割には英語が下手だ。	
\\	彼[かれ]は 自国[じこく]の 大学[だいがく]で 英語[えいご]を 教[おし]えていると 言[い]っているが、その 割[わり]には 英語[えいご]が 下手[へた]だ。
\\	彼女はすばらしい研究をしているが、それにしては認められていない。	
\\	彼女[かのじょ]はすばらしい 研究[けんきゅう]をしているが、それにしては 認[みと]められていない。
\\	直樹は子供にしては漢字をよく知っている。	
\\	直樹[なおき]は 子供[こども]にしては 漢字[かんじ]をよく 知[し]っている。
\\	飛行機は12時間も遅れるわ荷物はなくなるわで、今度の旅行は最悪だった。	
\\	飛行機[ひこうき]は12 時間[じかん]も 遅[おく]れるわ 荷物[にもつ]はなくなるわで、 今度[こんど]の 旅行[りょこう]は 最悪[さいあく]だった。
\\	客が来るわ来るわ、またたく間に小さい店はいっぱいになった。	
\\	客[きゃく]が 来[く]るわ 来[く]るわ、またたく 間[ま]に 小[ちい]さい 店[みせ]はいっぱいになった。
\\	仕事は増えるわ給料は下がるわ、今の状況は最悪だ。	
\\	仕事[しごと]は 増[ふ]えるわ 給料[きゅうりょう]は 下[さ]がるわ、 今[いま]の 状況[じょうきょう]は 最悪[さいあく]だ。
\\	クレジットカードをなくすわ怪我をするわで、今度のスキー旅行ではひどい目に遭った。	
\\	クレジットカードをなくすわ 怪我[けが]をするわで、 今度[こんど]のスキー 旅行[りょこう]ではひどい 目[め]に 遭[あ]った。
\\	新しく雇った学生アルバイトは、仕事は遅いわ文句は多いわで、とても強いものにならない。	
\\	新[あたら]しく 雇[やと]った 学生[がくせい]アルバイトは、 仕事[しごと]は 遅[おそ]いわ 文句[もんく]は 多[おお]いわで、とても 強[つよ]いものにならない。
\\	この辺りは買い物には不便だわ環境は悪いわ、何一ついいところがない。	
\\	この 辺[あた]りは 買い物[かいもの]には 不便[ふべん]だわ 環境[かんきょう]は 悪[わる]いわ、 何一[なにひと]ついいところがない。
\\	台所は何一つ手伝おうとしないわ、一緒に外食してもお金を払わないわ、もうあんな客は来てほしくない。	
\\	台所[だいどころ]は 何一[なにひと]つ 手伝[てつだ]おうとしないわ、 一緒[いっしょ]に 外食[がいしょく]してもお 金[かね]を 払[はら]わないわ、もうあんな 客[きゃく]は 来[き]てほしくない。
\\	面白い仕事はもらえるわ給料は上げてもらえるわ、新井君は最近いいことずくめだ。	
\\	面白[おもしろ]い 仕事[しごと]はもらえるわ 給料[きゅうりょう]は 上[あ]げてもらえるわ、 新井[あらい] 君[くん]は 最近[さいきん]いいことずくめだ。
\\	辞書で「コウ」という発音の漢字を調べてみたら、あるわあるわ、全部で174も出ていた。	
\\	辞書[じしょ]で「コウ」という 発音[はつおん]の 漢字[かんじ]を 調[しら]べてみたら、あるわあるわ、 全部[ぜんぶ]で174も 出[で]ていた。
\\	余程おなかが空いていたのだろう。食べるわ食べるわ、またたく間に多いな茶わんでご飯を5杯も平らげた。	
\\	余程[よほど]おなかが 空[す]いていたのだろう。 食[た]べるわ 食[た]べるわ、またたく 間[ま]に 多[おお]いな 茶[ちゃ]わんでご 飯[はん]を5 杯[はい]も 平[たい]らげた。
\\	新しく雇った学生アルバイトは、仕事を怠けるやらよく文句を言うやらで、とても使いものにならない。	
\\	新[あたら]しく 雇[やと]った 学生[がくせい]アルバイトは、 仕事[しごと]を 怠[なま]けるやらよく 文句[もんく]を 言[い]うやらで、とても 使[つか]いものにならない。
\\	会議やら講習会やらで最近出張が多い。	
\\	会議[かいぎ]やら 講習[こうしゅう] 会[かい]やらで 最近[さいきん] 出張[しゅっちょう]が 多[おお]い。
\\	久子は来るやら来ないやら分からない。	
\\	久子[ひさこ]は 来[く]るやら 来[こ]ないやら 分[わ]からない。
\\	その小説は発売されるやいなやミリオンセラーになった。	
\\	その 小説[しょうせつ]は 発売[はつばい]されるやいなやミリオンセラーになった。
\\	彼女は車から降りるやいなや報道陣やファンに取り囲まれた。	
\\	彼女[かのじょ]は 車[くるま]から 降[お]りるやいなや 報道陣[ほうどうじん]やファンに 取り囲[とりかこ]まれた。
\\	スタジオに着くやいなやリハーサルが始まった。	
\\	スタジオに 着[つ]くやいなやリハーサルが 始[はじ]まった。
\\	秀夫はアパートに帰るやいなやベッドに倒れ込んだ。	
\\	秀夫[ひでお]はアパートに 帰[かえ]るやいなやベッドに 倒れ込[たおれこ]んだ。
\\	母の顔を見るやいなや涙が込み上げてきた。	
\\	母[はは]の 顔[かお]を 見[み]るやいなや 涙[なみだ]が 込み上[こみあ]げてきた。
\\	小池氏はメディアから批判されるや否や、手のひらを返すように音見を変えた。	
\\	小池[こいけ] 氏[し]はメディアから 批判[ひはん]されるや 否[いな]や、 手[て]のひらを 返[かえ]すように 音[おと] 見[み]を 変[か]えた。
\\	ジョージは大学を卒業するや否やゲームソフトのビジネスを始めた。	
\\	ジョージは 大学[だいがく]を 卒業[そつぎょう]するや 否[いな]やゲームソフトのビジネスを 始[はじ]めた。
\\	安くておいしいと評判のそのレストランは、朝11時に開店するやいなや、常連客でいっぱいになった。	
\\	安[やす]くておいしいと 評判[ひょうばん]のそのレストランは、 朝[あさ]11 時[じ]に 開店[かいてん]するやいなや、 常連[じょうれん] 客[きゃく]でいっぱいになった。
\\	そのテレビドラマは1回目が放映されるや否や、国中に大きな韓国ブームを引き起こした。	
\\	そのテレビドラマは 1回[いっかい] 目[め]が 放映[ほうえい]されるや 否[いな]や、 国[くに] 中[ちゅう]に 大[おお]きな 韓国[かんこく]ブームを 引き起[ひきお]こした。
\\	ホテルに着いたらすぐ電話しなさい。	
\\	ホテルに 着[つ]いたらすぐ 電話[でんわ]しなさい。
\\	ホテルに着いたらすぐに電話しよう。	
\\	ホテルに 着[つ]いたらすぐに 電話[でんわ]しよう。
\\	啓次はいつも家に帰ったらすぐシャワーを浴びる。	
\\	啓[けい] 次[じ]はいつも 家[いえ]に 帰[かえ]ったらすぐシャワーを 浴[あ]びる。
\\	友美は和男に会ったらすぐにそのことを話すだろう。	
\\	友美[ゆみ]は 和男[かずお]に 会[あ]ったらすぐにそのことを 話[はな]すだろう。
\\	私は先生の顔を見るやいなや無意識に立ち上がっていた。	
\\	私[わたし]は 先生[せんせい]の 顔[かお]を 見[み]るやいなや 無意識[むいしき]に 立ち上[たちあ]がっていた。
\\	教室に入るやいなや宿題を忘れたことを思い出した。	
\\	教室[きょうしつ]に 入[はい]るやいなや 宿題[しゅくだい]を 忘[わす]れたことを 思い出[おもいだ]した。
\\	ジェーンはその写真を見た途端に気絶した。	
\\	ジェーンはその 写真[しゃしん]を 見[み]た 途端[とたん]に 気絶[きぜつ]した。
\\	啓次は家に帰るやいなやシャワーを浴びた。	
\\	啓[けい] 次[じ]は 家[いえ]に 帰[かえ]るやいなやシャワーを 浴[あ]びた。
\\	その店はバーゲンセールの日はいつも入り口のドアが開いた途端客が走り込む。	
\\	その 店[みせ]はバーゲンセールの 日[ひ]はいつも 入り口[いりくち]のドアが 開[ひら]いた 途端[とたん] 客[きゃく]が 走り込[はしりこ]む。
\\	私は先生の顔を見た途端無意識に立ち上がっていた。	
\\	私[わたし]は 先生[せんせい]の 顔[かお]を 見[み]た 途端[とたん] 無意識[むいしき]に 立ち上[たちあ]がっていた。
\\	教室に入った途端に宿題を忘れたことを思い出した。	
\\	教室[きょうしつ]に 入[はい]った 途端[とたん]に 宿題[しゅくだい]を 忘[わす]れたことを 思い出[おもいだ]した。
\\	減量のためにダイエットをすると同時に運動もしている。	
\\	減量[げんりょう]のためにダイエットをすると 同時[どうじ]に 運動[うんどう]もしている。
\\	群集が遠くに見えると同時に太鼓の音も聞こえる。	
\\	群集[ぐんしゅう]が 遠[とお]くに 見[み]えると 同時[どうじ]に 太鼓[たいこ]の 音[おと]も 聞[き]こえる。
\\	その小説は発売されるとすぐにミリオンセラーになった。	
\\	その 小説[しょうせつ]は 発売[はつばい]されるとすぐにミリオンセラーになった。
\\	彼女は車から隣りるとすぐ報道陣やファンに取り囲まれた。	
\\	彼女[かのじょ]は 車[くるま]から 隣[お]りるとすぐ 報道陣[ほうどうじん]やファンに 取り囲[とりかこ]まれた。
\\	スタジオに着くとすぐにリハーサルが始まった。	
\\	スタジオに 着[つ]くとすぐにリハーサルが 始[はじ]まった。
\\	ホテルに着いたらすぐに電話して下さい。	
\\	ホテルに 着[つ]いたらすぐに 電話[でんわ]して 下[くだ]さい。
\\	ホテルに着いたらすぐ電話しよう。	
\\	ホテルに 着[つ]いたらすぐ 電話[でんわ]しよう。
\\	彼女が部屋から出てきた瞬間にカメラのシャッターを切ろう。	
\\	彼女[かのじょ]が 部屋[へや]から 出[で]てきた 瞬間[しゅんかん]にカメラのシャッターを 切[き]ろう。
\\	カメラマンたちは、彼女が部屋から出てきた瞬間にカメラのシャッターを切るだろう。	
\\	カメラマンたちは、 彼女[かのじょ]が 部屋[へや]から 出[で]てきた 瞬間[しゅんかん]にカメラのシャッターを 切[き]るだろう。
\\	彼女が部屋から出てきた瞬間にカメラのシャッターを切る。	
\\	彼女[かのじょ]が 部屋[へや]から 出[で]てきた 瞬間[しゅんかん]にカメラのシャッターを 切[き]る。
\\	私は彼女が部屋から出てきた瞬間にカメラのシャッターを切った。	
\\	私[わたし]は 彼女[かのじょ]が 部屋[へや]から 出[で]てきた 瞬間[しゅんかん]にカメラのシャッターを 切[き]った。
\\	私は犯人が後ろを向いた瞬間に飛びかかった。	
\\	私[わたし]は 犯人[はんにん]が 後[うし]ろを 向[む]いた 瞬間[しゅんかん]に 飛[と]びかかった。
\\	ラケットがボールに当たる瞬間にシャッターを切って下さい。	
\\	ラケットがボールに 当[あ]たる 瞬間[しゅんかん]にシャッターを 切[き]って 下[くだ]さい。
\\	猫は鳥が飛び立つ瞬間に飛びついた。	
\\	猫[ねこ]は 鳥[とり]が 飛び立[とびた]つ 瞬間[しゅんかん]に 飛[と]びついた。
\\	彼は優勝の瞬間に右の拳を天に突き上げた。	
\\	彼[かれ]は 優勝[ゆうしょう]の 瞬間[しゅんかん]に 右[みぎ]の 拳[こぶし]を 天[あま]に 突き上[つきあ]げた。
\\	離陸の瞬間にめまいを覚えた。	
\\	離陸[りりく]の 瞬間[しゅんかん]にめまいを 覚[おぼ]えた。
\\	このビルが完成するのはいつになるのやら。	
\\	このビルが 完成[かんせい]するのはいつになるのやら。
\\	これから先、うちの会社は一体どうなることやら。	
\\	これから 先[さき]、うちの 会社[かいしゃ]は 一体[いったい]どうなることやら。
\\	こんな映画のどこがいいのやらさっぱり分からない。	
\\	こんな 映画[えいが]のどこがいいのやらさっぱり 分[わ]からない。
\\	学生たちが何やら面白そうな計画をしている。	
\\	学生[がくせい]たちが 何[なに]やら 面白[おもしろ]そうな 計画[けいかく]をしている。
\\	競技場ははたしたオリンピック開会までに完成するのやら。	
\\	競技[きょうぎ] 場[じょう]ははたしたオリンピック 開会[かいかい]までに 完成[かんせい]するのやら。
\\	一体いつになったら自分の家が持てることやら。	
\\	一体[いったい]いつになったら 自分[じぶん]の 家[いえ]が 持[も]てることやら。
\\	さっきまでの元気はどこへ行ったのやら。	
\\	さっきまでの 元気[げんき]はどこへ 行[い]ったのやら。
\\	どこが入り口やら分からずビルの周りを何度も回った。	
\\	どこが 入り口[いりくち]やら 分[わ]からずビルの 周[まわ]りを 何[なん] 度[ど]も 回[まわ]った。
\\	誰も名札をつけていないので誰が誰やらさっぱり分からない。	
\\	誰[だれ]も 名札[なふだ]をつけていないので 誰[だれ]が 誰[だれ]やらさっぱり 分[わ]からない。
\\	今日はどんなご馳走を食べさせてもらえるのやら、とても楽しみだ。	
\\	今日[きょう]はどんなご 馳走[ちそう]を 食[た]べさせてもらえるのやら、とても 楽[たの]しみだ。
\\	いつのまにやら外は暗くなっていた。	
\\	いつのまにやら 外[そと]は 暗[くら]くなっていた。
\\	あんな男のどこがいいのやら。	
\\	あんな 男[おとこ]のどこがいいのやら。
\\	この本はどこにしまえばいいのやら。	
\\	この 本[ほん]はどこにしまえばいいのやら。
\\	あの人にいつ会ったのやら全然覚えていない。	
\\	あの 人[ひと]にいつ 会[あ]ったのやら 全然[ぜんぜん] 覚[おぼ]えていない。
\\	彼には何度裏切られたことやら。	
\\	彼[かれ]には 何[なん] 度[ど] 裏切[うらぎ]られたことやら。
\\	夫ははたして無事に戦争から帰ってきてくれることやら。	
\\	夫[おっと]ははたして 無事[ぶじ]に 戦争[せんそう]から 帰[かえ]ってきてくれることやら。
\\	この不況は一体いつまで続くのやら。	
\\	この 不況[ふきょう]は 一体[いったい]いつまで 続[つづ]くのやら。
\\	今日はどんなものを食べさせられることやら。	
\\	今日[きょう]はどんなものを 食[た]べさせられることやら。
\\	このホールの完成はいつになるのやら。	
\\	このホールの 完成[かんせい]はいつになるのやら。
\\	誰が手伝ってくれるのやら全然知らない。	
\\	誰[だれ]が 手伝[てつだ]ってくれるのやら 全然[ぜんぜん] 知[し]らない。
\\	この不況が一体いつまで続くのやら専門家にも全く予測がつかない。	
\\	この 不況[ふきょう]が 一体[いったい]いつまで 続[つづ]くのやら 専門[せんもん] 家[か]にも 全[まった]く 予測[よそく]がつかない。
\\	今日はどんなご馳走を食べさせてもらえるのやら分からないが、とても楽しみだ。	
\\	今日[きょう]はどんなご 馳走[ちそう]を 食[た]べさせてもらえるのやら 分[わ]からないが、とても 楽[たの]しみだ。
\\	エレベーターで誰やら知らない人に話しかけられた。	
\\	エレベーターで 誰[だれ]やら 知[し]らない 人[ひと]に 話[はな]しかけられた。
\\	どこかで大きな地震があったらしい。	
\\	どこかで 大[おお]きな 地震[じしん]があったらしい。
\\	いつやらそういう話しを友達から聞いたことがある。	
\\	いつやらそういう 話[はな]しを 友達[ともだち]から 聞[き]いたことがある。
\\	どうやら電車に間にあった。	
\\	どうやら 電車[でんしゃ]に 間[あいだ]にあった。
\\	友美はパーティーに来られるだろうか。	
\\	友美[ゆみ]はパーティーに 来[きた]られるだろうか。
\\	友美がパーティーに来られるのやら。	
\\	友美[ゆみ]がパーティーに 来[きた]られるのやら。
\\	この論文はいつ書き上げられることやら。	
\\	この 論文[ろんぶん]はいつ 書き上[かきあ]げられることやら。
\\	明日のパーティーには誰が来るやら分からない。	
\\	明日[あした]のパーティーには 誰[だれ]が 来[く]るやら 分[わ]からない。
\\	私は会議やら講習会やらでこのところ出張が続いている。	
\\	私[わたし]は 会議[かいぎ]やら 講習[こうしゅう] 会[かい]やらでこのところ 出張[しゅっちょう]が 続[つづ]いている。
\\	みんなに自分の写真をほめられて嬉しいやら恥ずかしいやら変な気持ちでした。	
\\	みんなに 自分[じぶん]の 写真[しゃしん]をほめられて 嬉[うれ]しいやら 恥[は]ずかしいやら 変[へん]な 気持[きも]ちでした。
\\	今日は、上司に叱られるやら財布を落とすやらでさんざんな一日だった。	
\\	今日[きょう]は、 上司[じょうし]に 叱[しか]られるやら 財布[さいふ]を 落[お]とすやらでさんざんな 一日[いちにち]だった。
\\	久子は来るやら来ないやら分からないから、もう行こう。	
\\	久子[ひさこ]は 来[く]るやら 来[こ]ないやら 分[わ]からないから、もう 行[い]こう。
\\	美術品のことは何も知らないので、この壷が安いのやら高いのやら全然見等がつかない。	
\\	美術[びじゅつ] 品[ひん]のことは 何[なに]も 知[し]らないので、この 壷[つぼ]が 安[やす]いのやら 高[たか]いのやら 全然[ぜんぜん] 見[けん] 等[とう]がつかない。
\\	着替えやらお土産やらでスーツケースはもういっぱいになってしまった。	
\\	着替[きが]えやらお 土産[みやげ]やらでスーツケースはもういっぱいになってしまった。
\\	趣味のことやら家族のことやらを時々ブログに書いている。	
\\	趣味[しゅみ]のことやら 家族[かぞく]のことやらを 時々[ときどき]ブログに 書[か]いている。
\\	音を立ててスープを飲むやら口にものを入れたまま話すやら、直樹のテーブルマナーはひどかった。	
\\	音[おと]を 立[た]ててスープを 飲[の]むやら 口[くち]にものを 入[い]れたまま 話[はな]すやら、 直樹[なおき]のテーブルマナーはひどかった。
\\	アンディと楽しそうにおしゃべりしている陽子が憎らしいやらねたましいやらで、人と話していても上の空だった。	
\\	アンディと 楽[たの]しそうにおしゃべりしている 陽子[ようし]が 憎[にく]らしいやらねたましいやらで、 人[ひと]と 話[はな]していても 上の空[うわのそら]だった。
\\	彼女は日本語を話すそうだが、まだ話すのを聞いたことがないので、上手やら下手やら全く見当がつかない。	
\\	彼女[かのじょ]は 日本語[にほんご]を 話[はな]すそうだが、まだ 話[はな]すのを 聞[き]いたことがないので、 上手[じょうず]やら 下手[へた]やら 全[まった]く 見当[けんとう]がつかない。
\\	この仕事は自分にできるやらできないやら、やってみなければ分からない。	
\\	この 仕事[しごと]は 自分[じぶん]にできるやらできないやら、やってみなければ 分[わ]からない。
\\	彼はアメリカ生まれの生粋のアメリカ人だが、日本人なのやらアメリカ人なのやら分からないくらい、考え方が日本人的だ。	
\\	彼[かれ]はアメリカ 生[う]まれの 生粋[きっすい]のアメリカ 人[じん]だが、 日本人[にほんじん]なのやらアメリカ 人[じん]なのやら 分[わ]からないくらい、 考え方[かんがえかた]が 日本人[にほんじん] 的[てき]だ。
\\	貸してもらえるのやらもらえないのやら分からないお金を当てにしているのはよくない。	
\\	貸[か]してもらえるのやらもらえないのやら 分[わ]からないお 金[かね]を 当[あ]てにしているのはよくない。
\\	私は電車かバスかで行くつもりだ。	
\\	私[わたし]は 電車[でんしゃ]かバスかで 行[い]くつもりだ。
\\	知らない単語が出てくるたびに辞書を引いているようではいつまでたっても速く読めない。	
\\	知[し]らない 単語[たんご]が 出[で]てくるたびに 辞書[じしょ]を 引[ひ]いているようではいつまでたっても 速[はや]く 読[よ]めない。
\\	アパートがそんなに高いようでは私たちは東京には住めない。	
\\	アパートがそんなに 高[たか]いようでは 私[わたし]たちは 東京[とうきょう]には 住[す]めない。
\\	成績が今までのようでは困ります。	
\\	成績[せいせき]が 今[いま]までのようでは 困[こま]ります。
\\	今からそんなに文句を言っているようでは、この仕事をやり遂げるのは難しいだろう。	
\\	今[いま]からそんなに 文句[もんく]を 言[い]っているようでは、この 仕事[しごと]をやり 遂[と]げるのは 難[むずか]しいだろう。
\\	困った時に助けてくれないようでは友達とは言えない。	
\\	困[こま]った 時[とき]に 助[たす]けてくれないようでは 友達[ともだち]とは 言[い]えない。
\\	朝が弱いようではこの仕事は無理でしょう。	
\\	朝[あさ]が 弱[よわ]いようではこの 仕事[しごと]は 無理[むり]でしょう。
\\	千円程度の旅費払い戻しに4枚も5枚も書類が必要なようでは問題だ。	
\\	千[せん] 円[えん] 程度[ていど]の 旅費[りょひ] 払い戻[はらいもど]しに4 枚[まい]も5 枚[まい]も 書類[しょるい]が 必要[ひつよう]なようでは 問題[もんだい]だ。
\\	今頃そんなことをしているようでは、とても締め切りに間に合わない。	
\\	今頃[いまごろ]そんなことをしているようでは、とても 締め切[しめき]りに 間に合[あいだにあ]わない。
\\	部下の面倒が見られないようでは上司失格だ。	
\\	部下[ぶか]の 面倒[めんどう]が 見[み]られないようでは 上司[じょうし] 失格[しっかく]だ。
\\	大学の施設が現状のようでは、せっかく研究資金を増やしても有効に活用できない。	
\\	大学[だいがく]の 施設[しせつ]が 現状[げんじょう]のようでは、せっかく 研究[けんきゅう] 資金[しきん]を 増[ふ]やしても 有効[ゆうこう]に 活用[かつよう]できない。
\\	次の試合も今日の試合のようでは、とても決勝には進めない。	
\\	次[つぎ]の 試合[しあい]も 今日[きょう]の 試合[しあい]のようでは、とても 決勝[けっしょう]には 進[すす]めない。
\\	委員長が病気では会議は開けない。	
\\	委員[いいん] 長[ちょう]が 病気[びょうき]では 会議[かいぎ]は 開[あ]けない。
\\	今回の実験が失敗ではこのプロジェクトを続けるわけにはいかない。	
\\	今回[こんかい]の 実験[じっけん]が 失敗[しっぱい]ではこのプロジェクトを 続[つづ]けるわけにはいかない。
\\	山野氏は非常に時間に厳しい人で、面会時間に5分も遅れようものなら以後二度と会ってはもらえない。	
\\	山野[やまの] 氏[し]は 非常[ひじょう]に 時間[じかん]に 厳[きび]しい 人[ひと]で、 面会[めんかい] 時間[じかん]に5 分[ふん]も 遅[おく]れようものなら 以後[いご]二 度[ど]と 会[あ]ってはもらえない。
\\	私はたばこの煙に特に敏感で、近くで座れようものならもう何事にも集中できなくなってしまう。	
\\	私[わたし]はたばこの 煙[けむり]に 特[とく]に 敏感[びんかん]で、 近[ちか]くで 座[すわ]れようものならもう 何事[なにごと]にも 集中[しゅうちゅう]できなくなってしまう。
\\	ニューヨークでは信号が青に変わって一秒でも発進が遅れようものなら、すかさず後ろからクラクションを鳴らされる。	
\\	ニューヨークでは 信号[しんごう]が 青[あお]に 変[か]わって 一秒[いちびょう]でも 発進[はっしん]が 遅[おく]れようものなら、すかさず 後[うし]ろからクラクションを 鳴[な]らされる。
\\	この島は、晴れていれば散歩や海水浴などが楽しめるが、雨に降られようものなら何もやることがなくなる。	
\\	この 島[しま]は、 晴[は]れていれば 散歩[さんぽ]や 海水浴[かいすいよく]などが 楽[たの]しめるが、 雨[あめ]に 降[ふ]られようものなら 何[なに]もやることがなくなる。
\\	この飛行機は前の座席との間隔が極端に狭いので、私のように大きい人間は、前の人に椅子を倒されようものなら、文字通り身動きできなくなってしまう。	
\\	この 飛行機[ひこうき]は 前[まえ]の 座席[ざせき]との 間隔[かんかく]が 極端[きょくたん]に 狭[せま]いので、 私[わたし]のように 大[おお]きい 人間[にんげん]は、 前[まえ]の 人[ひと]に 椅子[いす]を 倒[たお]されようものなら、 文字通[もじどお]り 身動[みうご]きできなくなってしまう。
\\	最近頻繁にやってくる訪問販売は頭痛の種です。断っても簡単には帰りませんし、うっかり家の中に入れようものなら買うまで帰りません。	
\\	最近[さいきん] 頻繁[ひんぱん]にやってくる 訪問[ほうもん] 販売[はんばい]は 頭痛[ずつう]の 種[たね]です。 断[ことわ]っても 簡単[かんたん]には 帰[かえ]りませんし、うっかり 家[いえ]の 中[なか]に 入[い]れようものなら 買[か]うまで 帰[かえ]りません。
\\	花粉症の季節にうっかりバイクで出かけようものなら、走行中にくしゃみ連発でフルフェイスのヘルメットの中は悲惨な状態になる。	
\\	花粉[かふん] 症[しょう]の 季節[きせつ]にうっかりバイクで 出[で]かけようものなら、 走行[そうこう] 中[ちゅう]にくしゃみ 連発[れんぱつ]でフルフェイスのヘルメットの 中[なか]は 悲惨[ひさん]な 状態[じょうたい]になる。
\\	この魚は猛毒を持っていて、間違ってそのトゲに刺されようものなら、患部が大きく腫れ上がってしまう。	
\\	この 魚[さかな]は 猛毒[もうどく]を 持[も]っていて、 間違[まちが]ってそのトゲに 刺[さ]されようものなら、 患部[かんぶ]が 大[おお]きく 腫れ上[はれあ]がってしまう。
\\	田村さんは最近漢方医学に凝っていて、うっかりそのことを聞こうものなら、たっぷり一時間はその効用を聞くはめになる。	
\\	田村[たむら]さんは 最近[さいきん] 漢方[かんぽう] 医学[いがく]に 凝[こ]っていて、うっかりそのことを 聞[き]こうものなら、たっぷり一 時間[じかん]はその 効用[こうよう]を 聞[き]くはめになる。
\\	テストの成績が悪かったりしたら、今月の小遣いをもらえないかもしれない。	
\\	テストの 成績[せいせき]が 悪[わる]かったりしたら、 今月[こんげつ]の 小遣[こづか]いをもらえないかもしれない。
\\	客が外国人だったりしたら、私は緊張してしまった何も話せなくなる。	
\\	客[きゃく]が 外国[がいこく] 人[じん]だったりしたら、 私[わたし]は 緊張[きんちょう]してしまった 何[なに]も 話[はな]せなくなる。
\\	徹に連絡しようにも電話番号もメールアドレスも知らない。	
\\	徹[とおる]に 連絡[れんらく]しようにも 電話[でんわ] 番号[ばんごう]もメールアドレスも 知[し]らない。
\\	ひどく頭が痛くて起きようにも起きられない。	
\\	ひどく 頭[あたま]が 痛[いた]くて 起[お]きようにも 起[お]きられない。
\\	記事を書こうにも書く材料がない。	
\\	記事[きじ]を 書[か]こうにも 書[か]く 材料[ざいりょう]がない。
\\	すしを作ろうにも材料が手に入らない。	
\\	すしを 作[つく]ろうにも 材料[ざいりょう]が 手[て]に 入[はい]らない。
\\	苦情を言おうにも誰に言っていいか分からなかった。	
\\	苦情[くじょう]を 言[い]おうにも 誰[だれ]に 言[い]っていいか 分[わ]からなかった。
\\	体を動かそうにもあちこち痛くて腕も上げられない。	
\\	体[からだ]を 動[うご]かそうにもあちこち 痛[いた]くて 腕[うで]も 上[あ]げられない。
\\	母は僕の日本人のガールフレンドと話そうにも日本語ができなかった。	
\\	母[はは]は 僕[ぼく]の 日本人[にほんじん]のガールフレンドと 話[はなし]そうにも 日本語[にほんご]ができなかった。
\\	気の毒に彼女は悩み事を相談しようにも相談できる人が近くにいない。	
\\	気の毒[きのどく]に 彼女[かのじょ]は 悩[なや]み 事[ごと]を 相談[そうだん]しようにも 相談[そうだん]できる 人[ひと]が 近[ちか]くにいない。
\\	カウンセリングに行こうにも、その質や学派も様々でついつい懐疑的になってしまいます。	
\\	カウンセリングに 行[い]こうにも、その 質[しつ]や 学派[がくは]も 様々[さまざま]でついつい 懐疑[かいぎ] 的[てき]になってしまいます。
\\	その本は買おうにも絶版だった。	
\\	その 本[ほん]は 買[か]おうにも 絶版[ぜっぱん]だった。
\\	携帯電話の電池が切れたので、電話をかけようにもかけられない。	
\\	携帯[けいたい] 電話[でんわ]の 電池[でんち]が 切[き]れたので、 電話[でんわ]をかけようにもかけられない。
\\	記事を書こうにも書く材料がないだろう?	
\\	記事[きじ]を 書[か]こうにも 書[か]く 材料[ざいりょう]がないだろう?
\\	その仕事を一人でしようったってそれは無理だ。	
\\	その 仕事[しごと]を 一人[ひとり]でしようったってそれは 無理[むり]だ。
\\	そんなことを信じさせようったって誰が信じるものか。	
\\	そんなことを 信[しん]じさせようったって 誰[だれ]が 信[しん]じるものか。
\\	僕にお金を借りようったって無駄だよ。	
\\	僕[ぼく]にお 金[かね]を 借[か]りようったって 無駄[むだ]だよ。
\\	徹に連絡しようったって電話番号もメールアドレスも知らない。	
\\	徹[とおる]に 連絡[れんらく]しようったって 電話[でんわ] 番号[ばんごう]もメールアドレスも 知[し]らない。
\\	公共の交通機関がないので、自分の車で行くよりほかにはない。	
\\	公共[こうきょう]の 交通[こうつう] 機関[きかん]がないので、 自分[じぶん]の 車[くるま]で 行[い]くよりほかにはない。
\\	新聞はニューヨークタイムズのほかは読まない。	
\\	新聞[しんぶん]はニューヨークタイムズのほかは 読[よ]まない。
\\	アルバイトの仕事が駄目になったので、両親にお金を借りるよりほかになかった。	
\\	アルバイトの 仕事[しごと]が 駄目[だめ]になったので、 両親[りょうしん]にお 金[かね]を 借[か]りるよりほかになかった。
\\	友人がどうして気が狂ってしまったのかは推測するよりほかにはない。	
\\	友人[ゆうじん]がどうして 気[き]が 狂[くる]ってしまったのかは 推測[すいそく]するよりほかにはない。
\\	信号機の故障で電車が止まってしまった以上、待っているよりほかない。	
\\	信号[しんごう] 機[き]の 故障[こしょう]で 電車[でんしゃ]が 止[と]まってしまった 以上[いじょう]、 待[ま]っているよりほかない。
\\	国際的に活躍したければ英語を学ぶよりほかはないだろう。	
\\	国際[こくさい] 的[てき]に 活躍[かつやく]したければ 英語[えいご]を 学[まな]ぶよりほかはないだろう。
\\	秩序を回復するには権威の力を直接見せ付けた容赦ない手段をとるよりほかはない。	
\\	秩序[ちつじょ]を 回復[かいふく]するには 権威[けんい]の 力[ちから]を 直接[ちょくせつ] 見せ付[みせつ]けた 容赦[ようしゃ]ない 手段[しゅだん]をとるよりほかはない。
\\	モーツアルトが35年の生涯に600曲以上の作品を書いたというのには、ただただ驚くよりほかはない。	
\\	モーツアルトが35 年[ねん]の 生涯[しょうがい]に600 曲[きょく] 以上[いじょう]の 作品[さくひん]を 書[か]いたというのには、ただただ 驚[おどろ]くよりほかはない。
\\	典型的な科学者のイメージに合う人はアインシュタインのほかにはいない。	
\\	典型[てんけい] 的[てき]な 科学[かがく] 者[しゃ]のイメージに 合[あ]う 人[ひと]はアインシュタインのほかにはいない。
\\	学会に出席している学者は数人のほかは面識がなかった。	
\\	学会[がっかい]に 出席[しゅっせき]している 学者[がくしゃ]は 数[すう] 人[にん]のほかは 面識[めんしき]がなかった。
\\	この度は本社営業部長にご栄転の由、誠におめでとう存じます。	
\\	この 度[たび]は 本社[ほんしゃ] 営業[えいぎょう] 部長[ぶちょう]にご 栄転[えいてん]の 由[よし]、 誠[まこと]におめでとう 存[ぞん]じます。
\\	地球環境保護の国際会議が来年当市で開催される由である。	
\\	地球[ちきゅう] 環境[かんきょう] 保護[ほご]の 国際[こくさい] 会議[かいぎ]が 来年[らいねん] 当市[とうし]で 開催[かいさい]される 由[よし]である。
\\	野沢先生は来年一年ドイツにいらっしゃる由、伺っております。	
\\	野沢[のざわ] 先生[せんせい]は 来年[らいねん]一 年[ねん]ドイツにいらっしゃる 由[よし]、 伺[うかが]っております。
\\	第一子御出産の由、心からお祝い申し上げます。	
\\	第[だい] 一子[いっし] 御[ご] 出産[しゅっさん]の 由[よし]、 心[こころ]からお 祝[いわ]い 申し上[もうしあ]げます。
\\	当ホテルでのご滞在が快適でなかった由、誠に申し訳なく存じます。	
\\	当[とう]ホテルでのご 滞在[たいざい]が 快適[かいてき]でなかった 由[よし]、 誠[まこと]に 申し訳[もうしわけ]なく 存[ぞん]じます。
\\	交通事故のこと伺いました。お怪我などなかった由、ほっといたしております。	
\\	交通[こうつう] 事故[じこ]のこと 伺[うかが]いました。お 怪我[けが]などなかった 由[ゆかり]、ほっといたしております。
\\	石垣島は運よく台風の通過コースからはずれたため被害は少なかった由。	
\\	石垣島[いしがきじま]は 運[うん]よく 台風[たいふう]の 通過[つうか]コースからはずれたため 被害[ひがい]は 少[すく]なかった 由[よし]。
\\	日本消費者連盟は、一部ウイスキーの不正表示に関して、公正取引委員会に申し入れる由である。	
\\	日本[にほん] 消費[しょうひ] 者[しゃ] 連盟[れんめい]は、 一部[いちぶ]ウイスキーの 不正[ふせい] 表示[ひょうじ]に 関[かん]して、 公正[こうせい] 取引[とりひき] 委員[いいん] 会[かい]に 申し入[もうしい]れる 由[よし]である。
\\	懇親会にご出席の由、吉本君から連絡を受けました。	
\\	懇親[こんしん] 会[かい]にご 出席[しゅっせき]の 由[よし]、 吉本[よしもと] 君[くん]から 連絡[れんらく]を 受[う]けました。
\\	平井さんが3時にいらっしゃるとのことです。	
\\	平井[ひらい]さんが3 時[じ]にいらっしゃるとのことです。
\\	人が何と言おうと、私は自分の子供を信じている。	
\\	人[ひと]が 何[なん]と 言[い]おうと、 私[わたし]は 自分[じぶん]の 子供[こども]を 信[しん]じている。
\\	頑固と言われようが馬鹿と言われようが、私は自分の信じることをやります。	
\\	頑固[がんこ]と 言[い]われようが 馬鹿[ばか]と 言[い]われようが、 私[わたし]は 自分[じぶん]の 信[しん]じることをやります。
\\	あの教授は学生が分かろうと分かるまいと、かまわず授業を進めていく。	
\\	あの 教授[きょうじゅ]は 学生[がくせい]が 分[わ]かろうと 分[わ]かるまいと、かまわず 授業[じゅぎょう]を 進[すす]めていく。
\\	報酬が多かろうが少なかろうが、私はやるべきことをやるだけだ。	
\\	報酬[ほうしゅう]が 多[おお]かろうが 少[すく]なかろうが、 私[わたし]はやるべきことをやるだけだ。
\\	英語が上手だろうと下手だろうとそういうことは関係ない。	
\\	英語[えいご]が 上手[じょうず]だろうと 下手[へた]だろうとそういうことは 関係[かんけい]ない。
\\	プロの作品だろうが素人の作品だろうがいいものは採用する。	
\\	プロの 作品[さくひん]だろうが 素人[しろうと]の 作品[さくひん]だろうがいいものは 採用[さいよう]する。
\\	人が何をしようと大きなお世話だ。	
\\	人[ひと]が 何[なに]をしようと 大[おお]きなお 世話[せわ]だ。
\\	費用がいくらかかろうとかまわない。何としてもこの訴訟に勝たなければならない。	
\\	費用[ひよう]がいくらかかろうとかまわない。 何[なん]としてもこの 訴訟[そしょう]に 勝[か]たなければならない。
\\	彼はみんなが聞いていようがおかまいなしで、上司の悪口を言っている。	
\\	彼[かれ]はみんなが 聞[き]いていようがおかまいなしで、 上司[じょうし]の 悪口[わるぐち]を 言[い]っている。
\\	自分の子供たちが走り回ろうが飲み物をこぼそうが、親たちは叱りもしないで自分達の話に夢中になっている。	
\\	自分[じぶん]の 子供[こども]たちが 走り回[はしりまわ]ろうが 飲み物[のみもの]をこぼそうが、 親[おや]たちは 叱[しか]りもしないで 自分[じぶん] 達[たち]の 話[はなし]に 夢中[むちゅう]になっている。
\\	彼女が結婚しようがするまいが、そんなことは私には関係ないことだ。	
\\	彼女[かのじょ]が 結婚[けっこん]しようがするまいが、そんなことは 私[わたし]には 関係[かんけい]ないことだ。
\\	高かろうが安かろうが、必要なものなら買えばいい。	
\\	高[たか]かろうが 安[やす]かろうが、 必要[ひつよう]なものなら 買[か]えばいい。
\\	参加者がたとえ一人であろうとツアーは予定通り行います。	
\\	参加[さんか] 者[しゃ]がたとえ 一人[ひとり]であろうとツアーは 予定[よてい] 通[どお]り 行[おこな]います。
\\	日本人だろうがアメリカ人だろうが、そういうことは関係ない。好きになったら結婚する。	
\\	日本人[にほんじん]だろうがアメリカ 人[じん]だろうが、そういうことは 関係[かんけい]ない。 好[す]きになったら 結婚[けっこん]する。
\\	彼女はその美貌の故に、悩みも多い。	
\\	彼女[かのじょ]はその 美貌[びぼう]の 故[ゆえ]に、 悩[なや]みも 多[おお]い。
\\	結果にこだわりすぎる故、小さなことに心を奪われることになる。	
\\	結果[けっか]にこだわりすぎる 故[ゆえ]、 小[ちい]さなことに 心[こころ]を 奪[うば]われることになる。
\\	これは極めてデリケートな問題だ。それが故に、慎重に取り組む必要がある。	
\\	これは 極[きわ]めてデリケートな 問題[もんだい]だ。それが 故[ゆえ]に、 慎重[しんちょう]に 取り組[とりく]む 必要[ひつよう]がある。
\\	彼の精神は大病の故に強靭になった。	
\\	彼[かれ]の 精神[せいしん]は 大病[たいびょう]の 故[ゆえ]に 強靭[きょうじん]になった。
\\	電子テキストは改変と公開が容易であるが、この特徴ゆえに公開されている電子テキストは元データを追跡することが難しい。	
\\	電子[でんし]テキストは 改変[かいへん]と 公開[こうかい]が 容易[ようい]であるが、この 特徴[とくちょう]ゆえに 公開[こうかい]されている 電子[でんし]テキストは 元[もと]データを 追跡[ついせき]することが 難[むずか]しい。
\\	携帯電話が便利ゆえに、使い方を誤らないようにしなければならない。	
\\	携帯[けいたい] 電話[でんわ]が 便利[べんり]ゆえに、 使い方[つかいかた]を 誤[あやま]らないようにしなければならない。
\\	山は高いが故に尊からず。	
\\	山[やま]は 高[たか]いが 故[ゆえ]に 尊[とうと]からず。
\\	日本列島は南北に伸びているが故に、温度差が大きい。	
\\	日本[にほん] 列島[れっとう]は 南北[なんぼく]に 伸[の]びているが 故[ゆえ]に、 温度[おんど] 差[さ]が 大[おお]きい。
\\	アメリカでは今でも黒人であるが故に差別されるということがある。	
\\	アメリカでは 今[いま]でも 黒人[こくじん]であるが 故[ゆえ]に 差別[さべつ]されるということがある。
\\	「我思う、故に我あり。」―デカルト	
\\	我[われ] 思[おも]う、 故[ゆえ]に 我[われ]あり。」―デカルト
\\	人はきっとミスをする。故に、それを前提とした対策を考える必要がある。	
\\	人[ひと]はきっとミスをする。 故[ゆえ]に、それを 前提[ぜんてい]とした 対策[たいさく]を 考[かんが]える 必要[ひつよう]がある。
\\	俳句は言葉が最小限に抑えられている。それ故に、表現されていないことを想像しなければならない。	
\\	俳句[はいく]は 言葉[ことば]が 最小限[さいしょうげん]に 抑[おさ]えられている。それ 故[ゆえ]に、 表現[ひょうげん]されていないことを 想像[そうぞう]しなければならない。
\\	地震のためにビルが壊れた。	
\\	地震[じしん]のためにビルが 壊[こわ]れた。
\\	その事故は飲酒運転のために起こった。	
\\	その 事故[じこ]は 飲酒[いんしゅ] 運転[うんてん]のために 起[お]こった。
\\	そういう考え方しかできないのは、視野が狭いゆえである。	
\\	そういう 考え方[かんがえかた]しかできないのは、 視野[しや]が 狭[せま]いゆえである。
\\	この小説がこんなに広く読まれているのは、一つにはそのテーマの普遍性ゆえだろう。	
\\	この 小説[しょうせつ]がこんなに 広[ひろ]く 読[よ]まれているのは、 一[ひと]つにはそのテーマの 普遍[ふへん] 性[せい]ゆえだろう。
\\	技術の発達は社会環境の変化をもたらさずにはおかない。	
\\	技術[ぎじゅつ]の 発達[はったつ]は 社会[しゃかい] 環境[かんきょう]の 変化[へんか]をもたらさずにはおかない。
\\	コンピュータの普及とインターネット利用の拡大は大きな教育の変革を促さずにはおかない。	
\\	コンピュータの 普及[ふきゅう]とインターネット 利用[りよう]の 拡大[かくだい]は 大[おお]きな 教育[きょういく]の 変革[へんかく]を 促[うなが]さずにはおかない。
\\	この映画は見る者に感動を与えずにはおかない。	
\\	この 映画[えいが]は 見[み]る 者[もの]に 感動[かんどう]を 与[あた]えずにはおかない。
\\	この小説は我々に人間とは何かを考えさせずにはおかない。	
\\	この 小説[しょうせつ]は 我々[われわれ]に 人間[にんげん]とは 何[なに]かを 考[かんが]えさせずにはおかない。
\\	この事故は日本とアメリカの関係に影響を及ぼさずにはおかないだろう。	
\\	この 事故[じこ]は 日本[にほん]とアメリカの 関係[かんけい]に 影響[えいきょう]を 及[およ]ぼさずにはおかないだろう。
\\	彼女は聴く者を魅了せずにはおかない音楽性とそれを表現する卓越した技術を持っている。	
\\	彼女[かのじょ]は 聴[き]く 者[もの]を 魅了[みりょう]せずにはおかない 音楽[おんがく] 性[せい]とそれを 表現[ひょうげん]する 卓越[たくえつ]した 技術[ぎじゅつ]を 持[も]っている。
\\	環境や用途に合わせて動物たちがつくりあげる構造物は、彼らの知恵と工夫がいたるところに見られ、我々の好奇心を惹起せずにはおかない。	
\\	環境[かんきょう]や 用途[ようと]に 合[あ]わせて 動物[どうぶつ]たちがつくりあげる 構造[こうぞう] 物[ぶつ]は、 彼[かれ]らの 知恵[ちえ]と 工夫[くふう]がいたるところに 見[み]られ、 我々[われわれ]の 好奇[こうき] 心[しん]を 惹起[じゃっき]せずにはおかない。
\\	彼は必ず来る。	
\\	彼[かれ]は 必[かなら]ず 来[く]る。
\\	彼は間違いなく来る。	
\\	彼[かれ]は 間違[まちが]いなく 来[く]る。
\\	彼は絶対に来る。	
\\	彼[かれ]は 絶対[ぜったい]に 来[く]る。
\\	この車は近いうちに必ず壊れる。	
\\	この 車[くるま]は 近[ちか]いうちに 必[かなら]ず 壊[こわ]れる。
\\	彼は毎晩寝る前に間違いなくワインを飲む。	
\\	彼[かれ]は 毎晩[まいばん] 寝[ね]る 前[まえ]に 間違[まちが]いなくワインを 飲[の]む。
\\	彼は絶対正しい。	
\\	彼[かれ]は 絶対[ぜったい] 正[ただ]しい。
\\	この状況は間違いなく日本チームに有利だ。	
\\	この 状況[じょうきょう]は 間違[まちが]いなく 日本[にほん]チームに 有利[ゆうり]だ。
\\	彼女は絶対に日系アメリカ人です。	
\\	彼女[かのじょ]は 絶対[ぜったい]に 日系[にっけい]アメリカ 人[じん]です。
\\	彼女は周りにいる者を魅了せずにはおかない。	
\\	彼女[かのじょ]は 周[まわ]りにいる 者[もの]を 魅了[みりょう]せずにはおかない。
\\	周りの者は彼女に魅了されずにはいられない。	
\\	周[まわ]りの 者[もの]は 彼女[かのじょ]に 魅了[みりょう]されずにはいられない。
\\	技術の発達によって社会環境の変化が絶対にもたらされる。	
\\	技術[ぎじゅつ]の 発達[はったつ]によって 社会[しゃかい] 環境[かんきょう]の 変化[へんか]が 絶対[ぜったい]にもたらされる。
\\	インカ帝国を知らずして南米大陸を語ることはできない。	
\\	インカ 帝国[ていこく]を 知[し]らずして 南米[なんべい] 大陸[たいりく]を 語[かた]ることはできない。
\\	我々のチームは最終日の最終戦を待たずして優勝が決定した。	
\\	我々[われわれ]のチームは 最終[さいしゅう] 日[び]の 最終[さいしゅう] 戦[せん]を 待[ま]たずして 優勝[ゆうしょう]が 決定[けってい]した。
\\	自分の義務を果たさずして人の責任を問うべきではない。	
\\	自分[じぶん]の 義務[ぎむ]を 果[は]たさずして 人[ひと]の 責任[せきにん]を 問[と]うべきではない。
\\	文雄は宝くじが当たって労せずして大金を得た。	
\\	文雄[ふみお]は 宝[たから]くじが 当[あ]たって 労[ろう]せずして 大金[たいきん]を 得[え]た。
\\	新しい議長の選出で期せずしてみんなの意見が一致した。	
\\	新[あたら]しい 議長[ぎちょう]の 選出[せんしゅつ]で 期[き]せずしてみんなの 意見[いけん]が 一致[いっち]した。
\\	今勉強せずしていつするのか?	
\\	今[いま] 勉強[べんきょう]せずしていつするのか?
\\	荒木氏は自分の設計した建物の完成を見ずして亡くなってしまった。	
\\	荒木[あらき] 氏[し]は 自分[じぶん]の 設計[せっけい]した 建物[たてもの]の 完成[かんせい]を 見[み]ずして 亡[な]くなってしまった。
\\	ウイルスに感染した電子メールを社外の個人、団体に送信してしまい、意図せずして加害者になってしまった。	
\\	ウイルスに 感染[かんせん]した 電子[でんし]メールを 社外[しゃがい]の 個人[こじん]、 団体[だんたい]に 送信[そうしん]してしまい、 意図[いと]せずして 加害[かがい] 者[しゃ]になってしまった。
\\	敵を知らずしては戦えない。	
\\	敵[てき]を 知[し]らずしては 戦[たたか]えない。
\\	あまり勉強しないでテストを受けたのでひどい点を取ってしまった。	
\\	あまり 勉強[べんきょう]しないでテストを 受[う]けたのでひどい 点[てん]を 取[と]ってしまった。
\\	今年の夏は涼しいので、エアコンを使わずとも過ごせそうだ。	
\\	今年[ことし]の 夏[なつ]は 涼[すず]しいので、エアコンを 使[つか]わずとも 過[す]ごせそうだ。
\\	シカゴには親しい友人が何人かいるので、ホテルに泊まらずともよい。	
\\	シカゴには 親[した]しい 友人[ゆうじん]が 何[なん] 人[にん]かいるので、ホテルに 泊[と]まらずともよい。
\\	こんなに安くヨーロッパへ行けるのなら、旅行好きならずとも行ってみたいと思うだろう。	
\\	こんなに 安[やす]くヨーロッパへ 行[い]けるのなら、 旅行[りょこう] 好[す]きならずとも 行[い]ってみたいと 思[おも]うだろう。
\\	東京で会えずとも、大阪では必ず会える。	
\\	東京[とうきょう]で 会[あ]えずとも、 大阪[おおさか]では 必[かなら]ず 会[あ]える。
\\	彼がどのくらいできるかはテストをせずとも見当がつく。	
\\	彼[かれ]がどのくらいできるかはテストをせずとも 見当[けんとう]がつく。
\\	この程度の故障ならサービスセンターに送らずともこの店で修理できます。	
\\	この 程度[ていど]の 故障[こしょう]ならサービスセンターに 送[おく]らずともこの 店[みせ]で 修理[しゅうり]できます。
\\	この携帯電話は音声による操作が可能で、数字キーを押さずとも声だけで電話をかけられる。	
\\	この 携帯[けいたい] 電話[でんわ]は 音声[おんせい]による 操作[そうさ]が 可能[かのう]で、 数字[すうじ]キーを 押[お]さずとも 声[こえ]だけで 電話[でんわ]をかけられる。
\\	これからは特別な訓練を受けずとも使えるようなユーザーインターフェースが必要となるだろう。	
\\	これからは 特別[とくべつ]な 訓練[くんれん]を 受[う]けずとも 使[つか]えるようなユーザーインターフェースが 必要[ひつよう]となるだろう。
\\	この理論は難解なので、今すぐ分からずともよい。	
\\	この 理論[りろん]は 難解[なんかい]なので、 今[いま]すぐ 分[わ]からずともよい。
\\	今回の野球ストは、これからのプロスポーツのあり方を考えさせる出来事として、野球ファンならずとも注目する人が多かった。	
\\	今回[こんかい]の 野球[やきゅう]ストは、これからのプロスポーツのあり 方[かた]を 考[かんが]えさせる 出来事[できごと]として、 野球[やきゅう]ファンならずとも 注目[ちゅうもく]する 人[ひと]が 多[おお]かった。
\\	将来のことは心配せずとも、私がいい仕事を見つけてあげる。	
\\	将来[しょうらい]のことは 心配[しんぱい]せずとも、 私[わたし]がいい 仕事[しごと]を 見[み]つけてあげる。
\\	安くなくても買うつもりだ。	
\\	安[やす]くなくても 買[か]うつもりだ。
\\	便利でなくても住んでみたい。	
\\	便利[べんり]でなくても 住[す]んでみたい。
\end{CJK}
\end{document}