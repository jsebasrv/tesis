\documentclass[8pt]{extreport} 
\usepackage{hyperref}
\usepackage{CJKutf8}
\begin{document}
\begin{CJK}{UTF8}{min}
\\	愛	あい	
\\	親子・兄弟などがいつくしみ合う気持ち。 また、生あるものをかわいがり大事にする気持ち。 「―を注ぐ」 
\\	異性をいとしいと思う心。 男女間の、相手を慕う情。 恋。 「―が芽生える」 
\\	ある物事を好み、大切に思う気持ち。 「芸術に対する―」 
\\	個人的な感情を超越した、幸せを願う深く温かい心。 「人類への―」 
\\	キリスト教で、神が人類をいつくしみ、幸福を与えること。 また、他者を自分と同じようにいつくしむこと。 →アガペー 
\\	仏教で、主として貪愛(とんあい)のこと。 自我の欲望に根ざし解脱(げだつ)を妨げるもの。 [用法]愛・愛情――「親と子の愛(愛情)」「夫の妻に対する愛(愛情)」などでは、相通じて用いられる。 
\\	「愛」は、「国家への愛」など、広く抽象的な対象にも向けられる。 
\\	「愛情」は、主に肉親・親しい異性に対して用いられ、「幼なじみにあわい愛情を抱きはじめた」などという。 
\\	類似の語に「情愛」がある。 「情愛」は「愛情」と同じく肉親・異性間の感情を表すが、「絶ちがたい母子の情愛」のように、「愛情」よりも思いやる心が具体的である。	▲死ぬまであなたを愛するでしょう。 ▲私たちは、愛がどんなものかわからないほど若くはありません。
\\	愛情	あいじょう	
\\	深く愛し、いつくしむ心。 「―を注ぐ」 
\\	異性を恋い慕う心。 「ひそかな―をいだく」→愛[用法]	▲彼女は愛情のかけらもない。 ▲彼女はできる限りの愛情を子供達に注いだ。
\\	合図・相図	あいず	[名]スル身ぶりなどで知らせること。 前もって取り決めた方法で物事を知らせること。 また、その方法や信号。 「―を送る」「手を振って―する」	▲彼は私たちに始めるように合図した。 ▲彼は私が送った合図に反応した。
\\	アイスクリーム	アイスクリーム	牛乳・砂糖・卵黄に香料を加えて凍らせた氷菓子。 《季 夏》	▲私はほんとうにアイスクリームが好きです。 ▲私はこんなに甘いアイスクリームを食べるのをやめなければならない。
\\	愛する	あいする	[動サ変][文]あい・す[サ変] 
\\	かわいがり、いつくしむ。 愛情を注ぐ。 「我が子を―・する」 
\\	異性を慕う。 恋する。 「―・する人と結ばれる」 
\\	とりわけ好み、それに親しむ。 「音楽を―・する」 
\\	かけがえのないものとして、それを心から大切にする。 「祖国を―・する」 
\\	機嫌をとる。 あやす。 「よしよし、しばし―・せよ」 
\\	気に入って執着する。 「六趣に輪廻(りんゑ)することは、ただ一塵のたくはへをむさぼり―・するゆゑなり」 [類語]
\\	慈しむ・いとおしむ・かわいがる・寵愛(ちようあい)する/
\\	慕う・恋する・愛慕する・思慕する・恋慕する/
\\	好む・好(す)く・愛(め)でる・愛好する	▲彼が一番愛しているのは彼女の長女です。 ▲彼がそうするのは、君を嫌っているからではなく君を愛しているからだ。
\\	相手	あいて	
\\	一緒になって物事をする、一方の人。 「孫の―をする」「話し―」 
\\	その行為の対象となるもの。 「交渉の―」「―の出方を見る」「子供―の商売」 
\\	対抗して勝負を争う人。 「―にとって不足はない」「―にならない」「手ごわい―」 [類語]
\\	相棒・相方(あいかた)・仲間・パートナー	▲あなたのダンスの相手はだれですか。 ▲あなたの議論は良く相手に伝わった。
\\	生憎	あいにく	《「あやにく」の音変化》 ㊀[形動][文][ナリ]期待や目的にそぐわないさま。 都合の悪いさま。 「―な空模様」「―ですが、もう売り切れました」 ㊁[副]折あしく。 ぐあい悪く。 「彼を訪ねたが、―留守だった」	▲あいにく祖母が家にいなかった。 ▲あいにく私は彼の演説に間に合わなかった。
\\	アイロン	アイロン	《鉄の意》 
\\	布や衣服に押し当てて熱を伝え、しわを伸ばし、形を整える金属製のこて。 現在は、電気アイロンが普通。 
\\	整髪用のこて。	▲そのアイロンは過熱のために故障した。 ▲そのシャツはアイロンが必要です。
\\	アウト	アウト	
\\	スポーツ用語。 ㋐野球で、打者が打席に、走者が塁にいる権利を失うこと。 ↔セーフ。 ㋑テニス・卓球などで、ボールが規定の線より外に出ること。 ↔イン。 ㋒ゴルフで、一八ホールで構成されているコースの前半九ホール。 アウトコース。 ↔イン。 
\\	ぐあいが悪いこと。 不成功なこと。 「こんなに暑くちゃ―だよ」「今回の試験は完全に―だ」 
\\	他の外来語の上に付いて、外の、外側の、などの意を表す。 「―コース」「―ドア」	▲打球がインかアウトかを判定するのは難しいことが多い。 ▲打者はアウトになった。
\\	明かり	あかり	
\\	光。 明るさ。 「―がさす」 
\\	ともしび。 灯火。 「―を消す」 
\\	潔白であることの証明。 疑いを晴らす証拠。 あかし。 「なに、―を立てねば帰られぬ」 
\\	その時期が過ぎること。 あけ。 「諒闇(りやうあん)今朝御―なり」 [下接語]薄明かり・川明かり・月明かり・面(つら)明かり・時明かり・西明かり・初明かり・花明かり・星明かり・榾(ほた)明かり・窓明かり・夕明かり・雪明かり	▲この明かりで字を読もうとすれば目が悪くなるよ。 ▲はるか遠くに明かりが見えた。
\\	空き・明き	あき	
\\	物が詰まっていないこと。 すきま。 空間。 余地。 余白。 「本棚を置く―を作る」「行間の―を大きくとる」 
\\	欠員があること。 「定員に二名の―がある」 
\\	ひま。 「―の時間を利用して本を読む」 
\\	使っていないこと。 「傘の―があったら貸してください」「―部屋」	▲字と字の間の空きをもっと広くしなさい。 ▲彼の辞職で官僚の席に空きができた。
\\	明らか	あきらか	[形動][文][ナリ] 
\\	光が満ちて、明るく物を照らしているさま。 曇りなく明るいさま。 「水の中に―な光線がさし透って」「夜深き月の―にさし出でて」 
\\	はっきりとしていて疑う余地のないさま。 明白なさま。 「火を見るよりも―だ」「失敗は―に彼の責任だ」「論点を―にする」 
\\	道理に通じているさま。 賢明である。 「まして―ならん人の、まどへる我等を見んこと」 
\\	心が晴れやかなさま。 ほがらかである。 「むつかしくものおぼし乱れず、―にもてなし給ひて」	▲明らかに、彼らは彼の富と地位を妬んでいる。 ▲明らかに、彼は嘘をついている。
\\	諦める	あきらめる	[動マ下一][文]あきら・む[マ下二]もう希望や見込みがないと思ってやめる。 断念する。 「助からぬものと―・めている」「どしゃ降りで、外出を―・めた」 [用法]あきらめる・おもいきる――「進学をあきらめる(思い切る)」「あの人のことはなかなかあきらめられない(思い切れない)」のような場合は、相通じて用いられる。 
\\	「あきらめる」は「優勝はあきらめる」「あきらめてすごすご帰る」のように、望んでもかなわないことがわかって、望むのをやめる意。 これらの場合、「思い切る」は用いない。 
\\	「思い切る」は、「思い切って発言する」「思い切ったデザイン」のように、積極的に行う、覚悟して行うの意がある。 また、名詞形「思い切り」の形で「四十代半ばで会社をやめるとは思い切りがいい」のようにも用いる。 これらの場合に「あきらめる」は用いない。 
\\	類似の語に「断念する」がある。 「法案の提出を断念する」のように、周囲の状況が悪くなったりして実行に移すのをやめる意で用いる。	▲つきが、ずっと廻ってこないと、あきらめない賭博師は危機をおかして、大金を狙わざるをえなくなる。 ▲ついに彼らはその少年を死んだものとあきらめた。
\\	飽きる・厭きる・倦きる	あきる	[動カ上一]《動詞「あ(飽)く」(四段)の上一段化。 近世後期、江戸で使われはじめた語》 
\\	多すぎたり、同じことが長く続いたりして、いやになる。 「勉強に―・きた」「彼の長話に―・きた」 
\\	十分に味わったり経験したりして、それ以上欲しくなくなる。 「牛肉を―・きるほど食べたい」 
\\	動詞の連用形に付いて、いやになるほど十分に…するの意を表す。 「見―・きる」「聞き―・きる」→飽く [類語]
\\	倦(う)む・飽き飽きする・うんざりする・食傷する・退屈する・倦怠(けんたい)する・鼻につく	▲もう甘いものは食べ飽きた。 ▲よくも飽きずに毎日同じ事ができるね。
\\	握手	あくしゅ	[名]スル 
\\	互いに手を握り合うこと。 あいさつや、親愛の情、喜びの表現として行う。 「初対面の―を交わす」 
\\	仲直りをすること。 また、協力すること。 「新薬開発のため両社が―する」	▲私たちは旅の終わりに握手をして別れた。 ▲私はジェーンと握手をした。
\\	悪魔	あくま	
\\	残虐非道で、人に災いをもたらし、悪に誘い込む悪霊。 また、そのような人間。 
\\	仏道修行を妨げる悪神の総称。 魔。 魔羅。 
\\	キリスト教で、神の創造した世界に対する破壊的で攪乱(かくらん)的な要素。 悪への誘惑者。 地獄に落ちた天使という解釈もある。 サタン。	▲悪魔がやってくるかもしれないよ。 ▲悪魔にも当然与えるべきものは与えよ。
\\	預ける	あずける	[動カ下一][文]あづ・く[カ下二] 
\\	金品や身柄を人に頼んで、その保管や世話を頼む。 「荷物を―・ける」「銀行に金を―・ける」 
\\	物事の処理を人にゆだねる。 「店を―・ける」「帳場を―・ける」 
\\	からだをもたせかける。 「上体を―・ける」 
\\	紛争や勝負の決着を第三者に一任する。 「勝負を―・ける」 
\\	茶の湯の点前(てまえ)で、茶道具を仮置きする。 
\\	関係させる。 「花の賀に召し―・けられたりけるに」 [類語]
\\	託する・委(ゆだ)ねる・頼む・任せる・寄託する・預託する・信託する・委託する・委任する・付託する/
\\	もたせかける・もたせる・寄せかける	▲私は書類を彼に預けた。 ▲私は子どもを彼女に預けて買い物に行った。
\\	汗	あせ	
\\	皮膚の汗腺(かんせん)から分泌される液。 水と、微量の食塩・尿素などからなり、皮膚の乾燥を防ぎ、また、体温の調節をする。 興奮・恐怖などの精神的影響からも手のひらや足の裏などに分泌する。 「―が吹き出す」「―にまみれる」「―をぬぐう」「―が引く」「手に―を握る激しいレース」《季 夏》 
\\	物の表面に、内部からにじみ出たり、空中の水蒸気が凝結したりしてつく水滴。 [下接語]脂汗・大汗・玉の汗・血の汗・寝汗・鼻汗・一汗・冷や汗	▲汗がダラダラです。 ▲汗が額を滴り落ちるのを感じた。
\\	価・値	あたい	
\\	価格。 値段。 また、代価。 「商品に―をつける」「質屋の使の…尋ね来(こ)ん折には―を取らすべきに」 
\\	物の値打ち。 価値。 「一顧の―もない」 
\\	(値)数学で、文字や式・関数などがとる数値。 「比の―」	▲そんな言い方をするとあなたの値が下がりますよ。 ▲カウンターの値が指定の「キリ番」になったら記念メッセージを表示させます。
\\	与える	あたえる	[動ア下一][文]あた・ふ[ハ下二] 
\\	自分の所有物を他の人に渡して、その人の物とする。 現在ではやや改まった言い方で、恩恵的な意味で目下の者に授ける場合に多く用いる。 「子供におやつを―・える」「賞を―・える」 
\\	相手のためになるものを提供する。 「援助を―・える」「注意を―・える」 
\\	ある人の判断で人に何かをさせる。 ㋐相手に何かができるようにしてやる。 配慮して利用することを認める。 「発言の自由を―・える」「口実を―・える」 ㋑割り当てる。 課する。 「宿題を―・える」「役割を―・える」 
\\	影響を及ぼす。 ㋐相手に、ある気持ち・感じなどをもたせる。 「感銘を―・える」「いい印象を―・える」「苦痛を―・える」 ㋑こうむらせる。 「損害を―・える」 ◆室町時代以降はヤ行にも活用した。 →与ゆ [類語]
\\	遣(や)る・遣(つか)わす・取らせる・授ける・贈る・施す・恵む・くれる・授与する・賜与する・付与する・譲与する・贈与する・供与する・提供する・供給する・給付する・支給する・給する(尊敬)賜る・下さる・下される/
\\	㋑)あてがう・課する/
\\	及ぼす・もたらす	▲とにかく冷凍エビを与えよう。 ▲敵に120ダメージを与えた!
\\	暖かい・温かい	あたたかい	[形][文]あたたか・し[ク]《形容動詞「あたたか」の形容詞化》 
\\	(暖かい)寒すぎもせず、暑すぎもせず、程よい気温である。 あったかい。 「―・い部屋」「―・い地方」《季 春》「―・きドアの出入となりにけり/万太郎」 
\\	(温かい)物が冷たくなく、また熱すぎもせず、程よい状態である。 「―・い御飯」 
\\	(温かい)思いやりがある。 いたわりの心がある。 「―・くもてなす」↔冷たい。 
\\	(暖かい)金銭が十分にある。 「今日は懐が―・い」↔寒い。 
\\	(暖かい)色感がやわらかく、冷たい感じがしない。 「―・い色調の壁紙」 ◆気温のようにからだ全体で感じるあたたかさに、「寒い」に対して「暖かい」、部分で感じたり心で感じたりするあたたかさに、「冷たい」に対して「温かい」と書くのが普通。 [類語]
\\	あたたか・あったか・温暖・温和・ほかほか・ほやほや・温(ぬる)い・ぬくい/
\\	優しい・情け深い・手厚い・細やか・懇(ねんご)ろ・親切・懇篤(こんとく)・温厚	▲暖かい冬は好きではない。 ▲暖かい天候の時は、発汗作用が体温の調節をする上で役立つ。
\\	辺り	あたり	
\\	ある地点の周囲。 ある範囲の場所。 付近。 周り。 「―に気を配る」「この―は静かな住宅地だ」「―かまわず泣きだす」「―一面が火の海だ」 
\\	場所・時・人・事柄・数量などをはっきり示さずに、婉曲に言い表す語。 多く、名詞の下に付いて接尾語的に用いる。 ㋐そのへん。 一帯。 近所。 「六本木―で遊ぶ」 ㋑そのころ。 その時分。 「あした―行ってみよう」 ㋒たとえば…など。 「部長にかみつく―、けっこう気が強い」「山田君―に代わってもらおう」 ㋓その程度。 「県代表―までなれるだろう」「千円―の品物」 [類語]
\\	周辺・近辺・四辺・周囲・まわり・近く・付近・界隈(かいわい)・近傍・一帯	▲水から上がり、入江のほとりの砂の上を、よちよち歩き出しました。 ▲聖火のあたりもひどい状態だと言うことも聞いたわ。
\\	当たる・中る	あたる	[動ラ五(四)] 
\\	物事や人が直面、接触する。 ㋐動いて来たものがぶつかる。 また、動きのあるものが触れる。 「ボールが顔に―・る」「雨がフロントガラスに―・る」 ㋑断続的に触れる。 さわる。 「堅いカラーが首筋に―・る」 ㋒光・熱・風などを受ける。 「日がよく―・る部屋」「ストーブに―・る」「冷たい風に―・る」 ㋓人に接する。 人を待遇する。 現在では、ひどく扱う場合に用いる。 「つらく―・る」「家族に―・る」 ㋔対抗する。 対応する。 「強敵に―・る」「勢い―・るべからずだ」 
\\	物事がその状態である。 相当する。 ㋐そのような関係にある。 「伯父に―・る人」 ㋑その方角にある。 「東の方角に―・る家」 ㋒他と比べて、それに当てはまる。 「人の手に―・る部分」 ㋓結果としてそういうことになる。 「今日は結婚記念日に―・る」「失礼に―・る」 
\\	物事がふさわしい状態になる。 ねらいや希望などに当てはまる。 ㋐ねらいや予想のとおりになる。 的中する。 「天気予報が―・る」「山が―・る」 ㋑催しや企画などが成功する。 「商売が―・る」 ㋒くじなどで選ばれる。 当籤(とうせん)する。 「賞品としてテレビが―・る」 ㋓適合する。 合っている。 「彼の批評は―・っている」 
\\	物事に探りを入れる。 ようすを見る。 確かめてみる。 「原本に―・る」「他の店を―・ってみよう」 
\\	受けとめる。 担当する。 ㋐身に引き受ける。 従事する。 「あえて難局に―・る」 ㋑割り当てられる。 指名される。 「当番に―・る」 
\\	身体などにぐあいの悪い触れ方をする。 ㋐よくないことが身に及ぶ。 「罰が―・る」 ㋑からだに害を受ける。 「暑さに―・る」「河豚(ふぐ)に―・る」 ㋒果物などが傷む。 「この桃はところどころ―・っている」 
\\	(「…にあたらない」などの形で)…するに及ばない。 「驚くに―・らない」「腹をたてるには―・りません」 
\\	(多く「…にあたり」「…にあたって」の形で)何かを行う時・場合になる。 「新年を迎えるに―・り」「友達を選ぶに―・っては」 
\\	野球で、打者がよくヒットを打つ。 「あのバッターはよく―・っている」 
\\	釣りで、釣り針のえさに魚が食いついた感触がある。 「四投目のキャスティングで―・る」 
\\	《「する」が失う意に通じるところから、それを忌み嫌っていう》ひげなどを、する。 そる。 「顔を―・る」 [可能]あたれる [下接句]犬も歩けば棒に当たる・肯綮(こうけい)に中(あた)る・事に当たる・図に当たる・時に当たる・下手な鉄砲も数打てば当たる・耳に当たる・胸に当たる	▲弾丸は警官の脚に当たった。 ▲他人を指さすのは失礼にあたる。
\\	彼方此方	あちこち	㊀[代]指示代名詞。 いろいろの場所や方向をさす。 あちらこちら。 あっちこっち。 「―から寄付が集まる」「―歩き回る」 ㊁[形動][文][ナリ]物事の順序や位置が逆になっているさま。 あべこべ。 「話が―になる」「靴下を―にはく」	▲毎週火曜日には、先生のころころした小さな指がピアノの鍵盤をあちこちと鳥のように飛びはねていました。 ▲明日おひまならば、京都のあちこちをご案内できます。
\\	あっ	あっ	[感] 
\\	驚いたり感動したりしたときなどに思わず発する語。 「―、忘れた」「―、雪だ」 
\\	承諾したことを示す応答の語。 はい。 「少し存ずる旨(むね)あれば、急に―とも申されず」	▲「あっ。あたし生卵!」「一個でいいですか?」「うん。足りなくなったらまた注ぎ足すから」。 ▲「ちょっと、寮生相手にアンケートでもとってみる?」「あっ、いわゆるひとつのマーケティングリサーチだな」。
\\	扱う	あつかう	[動ワ五(ハ四)] 
\\	道具・機械などを、使ったり操作したりする。 取り扱う。 「壊れやすいので丁寧に―・う」「旋盤を―・う」 
\\	物事をとりさばく。 仕事として処理する。 「事務を―・う」「輸入品を―・う店」 
\\	人をもてなす。 世話をする。 「大ぜいの客を―・う」 
\\	ある身分・役割・状態にあるものとして遇する。 「大人として―・う」「欠席として―・う」 
\\	特に取り上げて問題にする。 「環境問題を―・った番組」「新聞で大きく―・われる」 
\\	調停する。 仲裁する。 「けんかを―・う」 
\\	看護する。 「病者のことを思う給へ―・ひ侍るほどに」 
\\	持て余す。 処置に困る。 「皆この事を―・ひて議するに」 
\\	うわさをする。 「人々も、思ひの外なることかなと―・ふめるを」 [可能]あつかえる [類語]
\\	使う・操る・使用する・操作する/
\\	受け持つ・手がける・取り仕切る・捌(さば)く・処理する・担当する・管掌する/
\\	遇する・処遇する・待遇する・あしらう	▲あの国では、私は外国人だったのでそれなりに扱われた。 ▲あの時あの少女をもっと親切に扱ってやればよかったのに。
\\	悪口	あっこう	[名]スル人を悪く言うこと。 悪態をつくこと。 また、その言葉。 わるくち。 「―を浴びせる」「ゆるりと磔柱(はりき)にかかって、休まるる体じゃと―し」→あっく(悪口)	▲彼は決して人の悪口を言う人ではない。 ▲彼は決して他人の悪口を言わないようにしている。
\\	集まり	あつまり	
\\	集まること。 また、集まったもの。 「客の―が悪い」 
\\	共通の目的で人が大ぜい寄り合うこと。 集会。 会合。 寄り合い。 「身内の―」	▲病気のために、私はその集まりに出席出来なかった。 ▲彼女の服は儀式的な集まりのなかで場違いです。
\\	当てる・充てる・宛てる	あてる	[動タ下一][文]あ・つ[タ下二] 
\\	あるものを他のものに触れるようにする。 直面させる。 ㋐ある物にぶつける。 「ボールを頭に―・てる」「的に―・てる」 ㋑光・熱・風などに触れさせる。 さらす。 「日に―・てて布団を乾かす」「鉢植えを夜露に―・てないようにする」 ㋒密着させる。 あてがう。 「額に手を―・てる」「継ぎを―・てる」「座布団を―・ててください」 ㋓対抗させる。 「練習試合で強豪に―・てて実力を試す」 
\\	期待やねらいどおりの状態にする。 ㋐くじなどで賞を得る。 「一等賞を―・てる」 ㋑催しや企画などが成功する。 「株で一山―・てる」「芝居で―・てる」 ㋒正しく推測する。 「彼の年齢を―・てる」「小説の途中で犯人を―・てる」 
\\	他のものに合わせる。 うまく振り分ける。 ㋐対応させてつける。 「外来語に漢字を―・てる」 ㋑仮にあてはめる。 「わが身に―・てて考える」 ㋒(充てる)全体の一部をそのために使う。 「余暇を読書に―・てる」「ボーナスを旅費に―・てる」 ㋓指名してやらせる。 「先生に―・てられる」 ㋔仕事や役などを割り振る。 「重要なポストに新人を―・てる」 ㋕(宛てる)相手に向ける。 「母に―・てて手紙を書く」→当てられる	▲かろうじて車に当てられずにすんだ。 ▲先生に急に当てられて、シドロモドロになってしまった。
\\	跡・迹	あと	《「足(あ)所(と)」の意》 
\\	何かが通っていったしるし。 「靴の―」「船の通った―」「頬(ほお)を伝う涙の―」「犯人の―を追う」 
\\	以前に何かが行われたしるし。 痕跡。 形跡。 「消しゴムで消した―」「手術の―」「苦心の―が見受けられる」「水茎(みずくき)の―」 
\\	以前に何かが存在したしるし。 「太古の海の―」「寺院の―」 
\\	家の跡目。 家督。 「父の―を継ぐ」 
\\	先人の手本。 先例。 「古人の―にならう」 
\\	足のあたり。 足もと。 「太神宮の御方を、御―にせさせ給ふこと、いかが」 
\\	で傷には「痕」とも、 
\\	で建造物には「址」とも書く。 [下接語]足跡・雨跡・家跡・窯(かま)跡・刈り跡・傷跡・靴跡・城跡・剃(そ)り跡・爪痕(つめあと)・鳥の跡・波跡・食(は)み跡・人跡・筆の跡・船(ふな)跡・水茎の跡・焼け跡	▲台風は破壊の跡を残して行った。 ▲城の跡は今は公園になっている。
\\	穴・孔	あな	
\\	反対側まで突き抜けている空間。 「針の―」 
\\	深くえぐりとられた所。 くぼんだ所。 「道に―があく」「耳の―」 
\\	㋐金銭の損失。 欠損。 「帳簿に―があく」 ㋑必要な物や人が抜けて空白になった所。 「人員に―があく」 ㋒不完全な所。 欠点。 弱点。 「下位打線が―だ」「彼の論理は―だらけだ」 
\\	他人が気づかない、よい場所や得になる事柄。 穴場。 
\\	競馬・競輪などで、番狂わせの勝負。 配当金が多い。 「―をねらう」 
\\	世間の裏面。 うら。 「世間の―を能く知って堺町とは気づいたり」 [下接語]蟻(あり)穴・息衝(つ)き穴・岩穴・埋め穴・鰓孔(えらあな)・大穴・落とし穴・鍵(かぎ)穴・隠れ穴・風穴・気抜き穴・切り穴・錐(きり)穴・毛穴・獣(しし)穴・縦穴・塚穴・抜け穴・螺子(ねじ)穴・鼠(ねずみ)穴・覗(のぞ)き穴・墓穴・人穴・一つ穴・節穴・?(ほぞ)穴・洞(ほら)穴・焼け穴・雪穴・横穴	▲彼らは穴を掘っている。 ▲彼らは穴をあけて石油を掘り当てようとした。
\\	油・脂・膏	あぶら	
\\	水に溶けず、水よりも軽い可燃性物質の総称。 動物性・植物性・鉱物性があり、食用・灯火用・燃料用・化学工業の原料など用途が広い。 ㋐動物の肉についている脂肪分。 脂身(あぶらみ)。 「―の多い切り身」 ㋑皮膚から分泌する脂肪。 「汗と―の結晶」 ㋒植物の種子などからとれる液体。 菜種油・ごま油など。 「―で揚げる」 ㋓原油を精製したもの。 重油・軽油・灯油など。 ㋔髪油。 ポマードやチック類もいう。 「―でなでつける」 
\\	活力のみなもと。 特に酒をさすことが多い。 「疲れたから―を補給しよう」 
\\	《火に油を注ぐとよく燃えるところから》おせじ。 へつらい。 うれしがらせ。 「えらい―言ひなます」 ◆一般に、常温で液体のものを「油」、固体のものを「脂」、肉のあぶらを「膏」と書き分ける。 [下接語]揚げ油・荏(え)の油・牡蠣(かき)油・固(かた)油・樺(かば)の油・蝦蟇(がま)の膏(あぶら)・髪油・榧(かや)の油・機械油・木の実油・桐(きり)油・胡桃(くるみ)油・黒油・芥子(けし)油・漉(こ)し油・胡麻(ごま)油・米油・差し油・白油・梳(す)き油・種油・椿(つばき)油・灯(とぼし)油・菜種(なたね)油・匂(にお)い油・鯡(にしん)油・糠(ぬか)油・鼻脂(はなあぶら)・鬢(びん)付け油・松脂(まつやに)油・豆油・水油・密陀(みつだ)の油・綿油	▲怒りっぽい人は10まで数えるようにすべきだ。そうすれば油が風波をしずめるように、むしゃくしゃした気持ちはおさまる。 ▲水は油よりも重い。
\\	誤り・謬り	あやまり	
\\	正しくないこと。 まちがい。 「記憶の―」「―を正す」 
\\	やりそこない。 失敗。 失策。 「計算の―」「書き―」 
\\	正しくない行為。 まちがった行為。 特に、男女間の不倫。 「いささかの事の―もあらば、軽々(かろがろ)しきそしりをや負はむ」 
\\	心が異常な状態になること。 「御心地の―にこそはありけれ」	▲私たちはしばしば誤りをおかす。 ▲弘法も筆の誤り。
\\	粗い	あらい	[形][文]あら・し[ク]《「荒い」と同語源》 
\\	すきまが大きい。 また、粒が大きくざらざらしている。 細かでない。 「目の―・い網」「粉のひき方が―・い」↔細かい。 
\\	手触りがなめらかでない。 すべすべしていない。 「きめの―・い肌」 
\\	粗雑である。 大ざっぱである。 大まかである。 「表現が―・い」「経費を―・く見積もる」 [派生]あらさ[名]	▲その奇妙な物体の表面はかなり粗い。 ▲その物体の表面はかなり粗い。
\\	嵐	あらし	
\\	荒く激しく吹く風。 雨・雪・雷を伴う場合にもいう。 暴風。 暴風雨。 「花に―」 
\\	激しく乱すもの。 また、事態や社会を揺るがす重大事。 「拍手の―」「革新の―」「不況の―が吹き荒れる」 [下接語]青嵐・朝嵐・小夜(さよ)嵐・地嵐・磁気嵐・砂嵐・初嵐・花嵐・鼻嵐・春嵐・山嵐・夕嵐・雪嵐・夜(よ)嵐	▲ラジオによると、北海で嵐が起こるとのことだ。 ▲一日中待った後、研究者達はまだ嵐がやむのを待っていたので、調査を再開することができた。
\\	新た	あらた	[形動][文][ナリ] 
\\	新しいさま。 今までなかったさま。 「―な局面を迎える」「―な感動を呼ぶ」「―な力がわく」「冬過ぎて春の来(きた)れば年月(としつき)は―なれども人は古(ふ)り行く」 
\\	(「あらたに」の形で)改めて行うさま。 「認識を―にする」	▲私の娘は最近口答えをしなくなった。気持ちを新たにして生活を一新したに違いない。 ▲私の息子は最近口答えをしなくなった。気持ちを新たにして生活を一新したに違いない。
\\	有らゆる	あらゆる	⦅連体⦆ (アリに奈良時代の助動詞ユの連体形ユルの付いたもの)あるかぎりの。 すべての。 ありとあらゆる。 地蔵十輪経元慶点「所在アラユル悩害に随ひて」。 「―手段をつくす」	▲外国人を悩ますもう一つの、多くの日本人のもつ傾向は、「すべての」「あらゆる」というような言葉を使ったり、仄めかしたりして、あまりにも一般的であり、あまりにも広がりのある表現をする点にある。 ▲学習のあらゆる機会を利用すべきだ。
\\	表す・現す・顕す・著す	あらわす	⦅他五⦆ 
\\	形・ようすなどを表に出して示す。 特に、神仏が霊験などを示す。 万葉集18「遠き世にかかりしことをわが御世に―・してあれば」。 「姿を―・す」「本性を―・す」 
\\	考え・意思・感情などをことばなどで表現する。 万葉集5「玉島のこの川上に家はあれど君を恥やさしみ―・さずありき」。 平家物語1「娑羅双樹の花の色、盛者じょうしゃ必衰のことわりを―・す」。 「承諾の意を―・す」「怒りを全身で―・す」「赤色は情熱を―・す」 
\\	《著》書物を書いて世に出す。 浮世風呂2「さきに―・す男湯の浮世風呂、一篇入つた大入り」。 「数々の名作を―・す」 
\\	《顕》広く世界に知らせる。 知れわたらせる。 「世に名を―・す」 
\\	「表」は内面にあるものを外に示したり、事物を象徴したりする場合、「現」は隠れていたものが姿を見せる場合に使うことが多い。	
\\	表れ・現れ	あらわれ	あらわれること。 あらわれたもの。 「好意の―」	
\\	現れる・顕れる・表れる	あらわれる	⦅自下一⦆[文]あらは・る(下二) 
\\	隠れていたものごとや今までなかったものが、はっきり表面に出る。 特に、神仏が示現する。 万葉集7「埋木の―・るましじきことにあらなくに」。 宇津保物語俊蔭「俊蔭が仕うまつる本尊―・れ給へ」。 大鏡道長「極楽浄土のあらたに―・れいで給ふべきために」。 「遅れて―・れる」「真価が―・れる」 
\\	隠していたものごとが人に知れる。 発覚する。 露顕する。 持統紀「謀反みかどかたぶけむとして―・れぬ」。 「隠すより―・れるがはやい」「悪事が―・れる」 
\\	考え・意思などがはっきりと出る。 「闘志が全身に―・れる」「喜びが文面に―・れる」	▲これよりよい物が現れるだろうか。 ▲禁煙による健康上の利点は、顕著であり、すぐに現れ、着実に増加していくのである。
\\	有り難う	ありがとう	[感]《形容詞「ありがたい」の連用形「ありがたく」のウ音便》感謝したり、礼を言ったりするときに用いる言葉。 ありがと。 「おみやげ―」 ◆丁寧にいうときは「ございます」を付ける。 関西地方では「おおきに」。	▲とにかく有難う。 ▲パーティーに招待してくれてありがとう。
\\	或いは	あるいは	《動詞「あり」の連体形+副助詞「い」+係助詞「は」から。 本来は、「ある人は」「ある場合は」などの意の主格表現となる連語》 ㊀[副] 
\\	同類の事柄を列挙していろいろな場合のあることを表す。 一方では。 「―歌をうたい、―笛を吹く」 
\\	ある事態が起こる可能性があるさま。 ひょっとしたら。 「―私がまちがっていたかもしれない」「明日は―雨かもしれない」 ㊁[接]同類の物事の中のどれか一つであることを表す。 または。 もしくは。 「みりん、―酒を加える」 ◆歴史的仮名遣いで「あるひは」と書く習慣は誤り。 [用法]あるいは・または――「多くの主婦が、外で働き、あるいは(または)学習に励んでいる」「明日は雨あるいは(または)雪になるでしょう」のように、二つのうちのどちらかということを表す場合は、「あるいは」「または」の両方が使える。 
\\	「会議は五時終了の予定だが、あるいは、三〇分ほど延びるかもしれない」のような「もしかすると」の意の副詞用法では、「または」は使えない。 
\\	類似の語に「それとも」がある。 「それとも」は「進学するか、それとも就職するか、まだ決めていない」のように疑問の形の文をつなぐときに用いる。 この場合、「あるいは」も「または」も使えるが、「それとも」が最も話し言葉的である。	
\\	アルバム	アルバム	
\\	写真・切手などを整理・保存する帳面。 写真帳・切手帳・サイン帳など。 
\\	写真などを印刷・製本したもの。 
\\	一連のレコードをブック型ケースに収めたもの。 
\\	複数の曲を収めたLPレコードやコンパクトディスク。	▲このアルバムを見ると私は楽しかった学生時代を思い出す。 ▲これは、私のアルバムのすべての写真の中で、もっとも美しいものです。
\\	泡・沫	あわ	
\\	液体が空気を包んでできた小さい玉。 あぶく。 「―が立つ」 
\\	口の端に吹き出る唾液(だえき)のあぶく。 「―を吹く」「口角(こうかく)―を飛ばす」 
\\	すぐ消えるところから、はかないことのたとえ。 「多年の苦労も水の―となる」	▲彼の努力はすべて水の泡に帰した。 ▲泡はみるみる無くなった。
\\	合わせる	あわせる	[動サ下一][文]あは・す[サ下二]《合うようにする、一致させる、が原義》 
\\	(「併せる」とも書く)二つ以上のものを一つにする。 ㋐二つ以上のものをつけて一つにする。 「仏前に手を―・せる」「周辺の町村を―・せて市にする」 ㋑心や力などをまとめて一つにする。 一致させる。 「心を―・せて事に当たる」「力を―・せて頑張る」 ㋒付け加える。 合計する。 「三と四とを―・せると七」「人口は両村を―・せても三〇〇〇人」「今までの業績も―・せて考慮する」 ㋓薬や食品などをまぜる。 調合する。 「二種の薬を―・せる」 
\\	二つのものを釣り合うようにする。 ㋐食い違いのないように、他のものに一致させる。 また、一致するように物事を行う。 「音楽に―・せて歌う」「彼の予定に―・せる」「歩調を―・せる」「口裏を―・せてごまかす」 ㋑釣り合うようにする。 相応するようにする。 調和させる。 「環境に―・せた建築物」「洋服に靴を―・せる」 ㋒異なる種類の楽器をいっしょに鳴らす。 合奏する。 「琴に尺八を―・せる」 ㋓正しいかどうか、他と比べて調べてみる。 照らし合わせる。 「答えを―・せる」「原文と―・せる」 
\\	武器を互いに打ち合わせる。 転じて、戦う。 「チャンピオンとグローブを―・せる」 
\\	対抗させる。 戦わせる。 「練習試合で昨年の優勝校と―・せる」 
\\	夢と事実との合致を判断する。 夢判断をする。 「さま異なる夢を見給ひて、―・するものを召して問はせ給へば」 
\\	夫婦にする。 めあわす。 「伊勢の守もろみちのむすめを正明の中将の君に―・せたりける時に」 
\\	比べて優劣を争う。 「詩に歌を―・せられしにも」 [下接句]顔が合わせられない・顔を合わせる・口を合わせる・口裏(くちうら)を合わせる・心を合わせる・力を合わせる・調子を合わせる・帳尻(ちようじり)を合わせる・手を合わせる・鬨(とき)を合わせる・肌を合わせる・腹を合わせる・額(ひたい)を合わせる・間(ま)を合わせる。 掌(たなごころ)を合わす・鞭(むち)鐙(あぶみ)を合わす・夢を合わす	▲マユコは音楽に合わせておどっている。 ▲三人の少年は合わせて2ドルしかもっていなかった。
\\	哀れ	あわれ	㊀[名]しみじみ心に染みる感動、また、そのような感情を表す。 
\\	(「憐れ」とも書く)強い心の動き。 特に悲哀・哀憐の感情。 不憫(ふびん)と思う気持ち。 「人々の―を誘った」「―をかける」「そぞろ―を催す」 
\\	かわいそうな状態。 無惨な姿。 「―をとどめる」 
\\	底知れないような趣。 情趣。 ものがなしさ。 「心なき身にも―は知られけり鴫(しぎ)立つ沢の秋の夕暮れ」 
\\	どうすることもできないような心の動き。 感慨。 「―進みぬれば、やがて尼になりぬかし」 
\\	しみじみとした情愛・人情。 慈愛の気持ち。 「子ゆゑにこそ、万(よろづ)の―は思ひ知らるれ」 ㊁[形動][文][ナリ]感動を起こさせる状況、しみじみ心を打つもののさまを広く表す。 現在では、多く悲哀・哀憐の感情に限定される。 
\\	(「憐れ」とも書く)かわいそうに思われるさま。 気の毒だ。 惨めだ。 「その姿はいかにも―であった」 
\\	しみじみともの悲しく感じるさま。 はかなく、また、さびしく思われるさま。 「夕暮れは、なんとなく―に思われてしかたがない」 
\\	しみじみと心を打つ風情があるさま。 趣があるさま。 「滝の音、水の声―に聞こゆる所なり」 
\\	しみじみと心に染みて愛着を感じるさま。 いとしいさま。 かわいいさま。 「なま心なく若やかなるけはひも―なれば」 
\\	しみじみとした愛情があるさま。 優しいさま。 「見る人も、いと―に忘るまじきさまにのみ語らふめれど」 
\\	感服させられるさま。 感心だ。 殊勝だ。 「―なるもの、孝(けう)ある人の子」 
\\	尊く、ありがたいさま。 「霊山(りやうぜん)は釈迦仏の御すみかなるが―なるなり」 [派生]あわれがる[動ラ五]あわれげ[形動]あわれさ[名] ㊂[感] 
\\	ものに感動したときに発する語。 感嘆賞美の場合にも哀傷の場合にも用いる。 ああ。 「―、あなおもしろ」「―あれをはしたなく言ひそむこそ、いとほしけれ」 
\\	願望の気持ちを表す。 ぜひとも。 「―、よい所もあれかし」 
\\	囃子詞(はやしことば)として用いる。 「いで我が駒早く行きこせ待乳山(まつちやま)―待乳山」 ◆本来、自然に発する感動の声に基づく感動詞として上代から用いられているが、平安時代以後、感動の声を発せさせられるような状況をいう形容動詞用法や、さらに、そのような状況のときの感情、心のありさまを表す名詞用法が生じて広く用いられた。 近世以後は主として悲哀・哀憐の感情を表すのに限定される。 なお、中世ごろ「あっぱれ」を派生している。 [類語] 
\\	哀感・悲哀・哀愁・哀憐・憐情・哀れみ/ 
\\	哀切・可哀相(かわいそう)・気の毒・不憫(ふびん)・いじらしい・痛ましい・惨(みじ)め・悲惨・情けない・見るに忍びない	▲彼女は哀れを誘う有様だった。 ▲彼は事業に失敗して今や全く哀れなものだ。
\\	案	あん	
\\	考え。 計画。 「―を練る」 
\\	予想。 推量。 
\\	文書の下書き。 草案。 「―を提出する」 
\\	物を載せる台。 机。 「此の経の―の前に立ちて」	▲私はやっと御兄弟を説得して私の案を受け入れさせた。 ▲専門家の委員達がその案を討論した。
\\	暗記・諳記	あんき	[名]スル文字・数字などを、書いたものを見ないでもすらすらと言えるように、よく覚えること。 「英単語を―する」「丸―」「―力」	▲私はやっとゲティスバーグの演説を暗記した。 ▲私はその単語を暗記中だ。
\\	安定	あんてい	[名]スル 
\\	物事が落ち着いていて、激しい変動のないこと。 「心の―を保つ」「物価が―する」 
\\	平衡状態に微小な変化を与えても、もとの状態とのずれがわずかの範囲にとどまること。 「―のいい花瓶」 
\\	物質が容易に分解・反応・崩壊しないこと。 「この元素は―している」	▲兄は、うまい、安定したレイアップができます。 ▲卸売物価は基本的に安定している。
\\	位	い	㊀〔接尾〕助数詞。 
\\	物事の順位・等級・位階などを表す。 「第三―」「従五―」 
\\	死者の霊を数えるのに用いる。 「百―の英霊」 
\\	計算の位取(くらいど)りを表す。 「百―の数」「小数点以下三―」 ㊁[名]くらい。 位階。 「一品以下。 初位(そゐ)以上を―と曰ふ」	▲チューリッヒはロンドンに次ぐ世界第二位の金市場である。 ▲結果は次の通りでした。1位日本、2位スペイン、3位イタリア。
\\	委員	いいん	国家・公共団体その他の団体において、選挙または指名を受け、特定の事項の調査や処理に当たる人。	▲彼は委員会の委員だ。 ▲調査委員会の新委員を任命しなければならない。
\\	意外	いがい	[名・形動]考えていた状態と非常に違っていること。 また、そのさま。 「事件は―な展開を見せた」「―に背が高い」「ベランメーに接近した彼の口の利き方にも―を呼んだ」→案外[用法] ◆現在では「意外に」と同様、「意外と知られていない事実」のように「意外と」の形も用いられる。 [派生]いがいさ[名]	▲仕事をする時仲間が多いと、コンセンサスをとるのが意外と大変だ。 ▲昨日のけんかを気にしてるの?意外にナイーブなのね。
\\	息	いき	
\\	口・鼻から空気を吸ったり吐いたりすること。 また、吸う空気や吐く空気。 「大きく―をする」「―が荒い」 
\\	二人以上で何かをする場合の、相互の気持ちのかねあい。 調子。 呼吸。 「二人の―がぴったりだ」 
\\	芸事の要領・こつ。 「名人の―を盗む」 
\\	ゆげ。 蒸気。 「飯も焚きたての―の立つやつで」 
\\	音声学で、声帯の振動を伴わない呼気。 ごくまれには吸気も含む。 
\\	いのち。 「あずの上に駒をつなぎて危(あや)ほかど人妻児ろを―に我がする」 [下接語]青息吐息・大息・風の息・片息・酒(さか)息・死に息・溜(た)め息・吐息・寝息・鼻息・一息・太息・虫の息	▲寒い天気に息を吐くと、息が見える。 ▲君は息が臭い。
\\	勢い	いきおい	㊀[名] 
\\	他を圧倒する力。 活気。 気勢。 「―を増す」「破竹の―」 
\\	社会を支配する力。 権力。 権勢。 「武力を背景に―を振るう」 
\\	自然の活動力。 「水の―で流される」「火の―が強い」 
\\	盛んな意気。 元気。 「一杯飲んで―をつける」 
\\	物事が動くときに加わる速さや強さ。 「下り坂で―がつく」「―余って土俵を飛び出す」「筆の―」 
\\	余勢。 もののはずみ。 なりゆき。 調子。 「酔った―で言う」「時の―に乗じる」 ㊁[副]その時のなりゆきで。 必然的に。 「その場の雰囲気から―そう答えざるをえなかった」 [下接句]騎虎(きこ)の勢い・旭日(きよくじつ)昇天の勢い・飛ぶ鳥を落とす勢い・破竹(はちく)の勢い・日の出の勢い	▲台風が勢いを増した。 ▲しばらくプータローしていて、迷ってたんです。勢いで辞表出しちゃったけど、本当は我慢して続けるべきだったのかな、って。
\\	生き物	いきもの	
\\	生きているもの。 特に、動物。 生物(せいぶつ)。 「―をかわいがる」 
\\	生命があるかのように、生き生きとして、絶えず変化するもの。 「言葉は―だ」	▲全ての生き物は、生き延びるための本能的衝動を持っている。 ▲人間は感情の生き物である。
\\	医師	いし	
\\	医術を仕事にする人。 医師法の適用を受けて、病気の診察・治療に当たる人。 医者。 古くは「くすし」「くすりし」といった。 
\\	律令制で、典薬寮(てんやくりよう)の職員。 治療と医生(いしよう)の教授をつかさどった者。 
\\	中世、公家や僧侶で医術の知識を持って施療した者。 
\\	江戸幕府の職名。 僧体で医療をつかさどった。 これに対し、民間では士分・公卿の服装をした古方(こほう)派があった。 →奥医師	▲医師が患者をよく診てくれたので回復した。 ▲医師たちは病気と闘っている。
\\	意志	いし	
\\	あることを行いたい、または行いたくないという考え。 意向。 「参加する―がある」「こちらの―が通じる」 
\\	目的や計画を選択し、それを実現しようとする精神の働き。 知識・感情に対立するものと考えられ、合わせて「知情意」という。 「―を貫く」「―強固」 
\\	哲学で、個人あるいは集団の行動を意識的に決定する能力。 広義には、欲望も含まれる。 倫理学的には、道徳的判断の主体あるいは原因となるものをいい、衝動と対立する。 [用法]意志・意思――「意志」は「意志を貫く」「意志の強い人」「意志薄弱」など、何かをしよう、したいという気持ちを表す場合に用いられる。 哲学・心理学用語としては「意志」を用いることが多い。 
\\	「意思」は、「双方の意思を汲(く)む」「家族の意思を尊重する」など、思い・考えの意味に重点を置いた場合に用いられる。 法律用語としては「意思」を用いることが多い。 
\\	「意志(意思)の疎通を欠く」「意志(意思)表示」などは、話し手の意識によって使い分けられることもある。	▲相手は自分の意志で選びたいと思っている。 ▲制御さえなければ、これらの力は危険と破壊をもたらすかもしれないが、ひとたび完全に支配されたならば、それらは人間の意志と欲望に従わせることができる。
\\	意思	いし	
\\	何かをしようとするときの元となる心持ち。 「本人の―に任せる」 
\\	法律用語。 ㋐民法上、身体の動作の直接の原因となる心理作用や、ある事実に対する意欲をさす。 ㋑刑法上、自分の行為に対する認識をさし、時には犯意と同じ意味をもつ。 「犯行の―」→意志[用法]	▲テレビはしばしば家庭内の意思の疎通を妨げるという事実はすでによく知られている。 ▲にやりと笑って賛成の意思を示した。
\\	維持	いじ	[名]スル物事の状態をそのまま保ちつづけること。 「健康を―する」「現状―」	▲警察は法と秩序の維持に対して責任を持つ。 ▲官僚達は大企業との強固な関係を維持している。
\\	意識	いしき	[名]スル 
\\	心が知覚を有しているときの状態。 「―を取り戻す」 
\\	物事や状態に気づくこと。 はっきり知ること。 また、気にかけること。 「勝ちを―して硬くなる」「彼女の存在を―する」 
\\	政治的、社会的関心や態度、また自覚。 「―が高い」「罪の―」 
\\	心理学・哲学の用語。 ㋐自分自身の精神状態の直観。 ㋑自分の精神のうちに起こることの知覚。 ㋒知覚・判断・感情・欲求など、すべての志向的な体験。 
\\	《梵
\\	の訳》仏語。 六識・八識の一。 目や耳などの感覚器官が、色や声など、それぞれ別々に認識するのに対し、対象を総括して判断し分別する心の働き。 第六識。	▲この論評の最初のセクションでは、脳のプロセスがどのように我々の意識経験を引き起こすかという問題を提起する。 ▲その英語学者は自分の意識不足を認識していない。
\\	異常	いじょう	[名・形動]普通と違っていること。 正常でないこと。 また、そのさま。 「この夏は―に暑かった」「―な執着心」「害虫の―発生」↔正常。 [派生]いじょうさ[名]	▲彼女は金に対して異常なほど欲望をもていた。 ▲飛行機が離陸しようとしたとき、異常な音を聞いた。
\\	泉	いずみ	
\\	《「出水(いずみ)」の意》地下水が自然に地表にわき出る所。 また、そのわき出た水。 湧泉(ゆうせん)。 《季 夏》「―への道後(おく)れゆく安けさよ/波郷」 
\\	物事が出てくるもと。 源泉。 「希望の―」「知識の―」	▲彼らはその泉で喉の渇きをいやした。 ▲彼はその泉の水を飲んだ。
\\	何れ・孰れ	いずれ	㊀[代]不定称の指示代名詞。 どれ。 どちら。 どっち。 「―の物も名品ぞろいだ」「合否―の場合も通知します」 ㊁[副] 
\\	いろいろな過程を経たうえでの結果をいう。 いずれにしても。 結局。 「その場はごまかせても―ばれるに決まっている」 
\\	あまり遠くない将来をいう。 そのうちに。 近々。 「―改めて伺います」	▲どれが君の本ですか。 ▲テーブルのケーキはどれを食べてもいい。
\\	以前・已前	いぜん	
\\	その時よりも前。 「一二時―に到着する」↔以後。 
\\	今より前の時点。 現在から見て近い過去。 副詞的にも用いる。 「―と違って今では」「―会ったことがある」 
\\	ある状態に達する前の段階。 「結婚―の住所」「能力―の問題だ」 ◆「以」は基準となる数値を含むのが普通であるが、例えば「明治以前」というときに、明治時代を除いて、その前をさす場合もある。	▲私はずっと以前にパリに訪れた。 ▲私はあの少女と以前会ったことがある。
\\	板	いた	
\\	材木を薄く平たく切ったもの。 「床に―を張る」 
\\	金属・石または合成樹脂などを薄く平たくしたもの。 「―ガラス」「ブリキ―」 
\\	まな板。 
\\	「板場(いたば)」「板前(いたまえ)」の略。 「―さん」 
\\	「板付き蒲鉾(かまぼこ)」の略。 「―わさ」 
\\	芝居の舞台。 「新作を―に掛ける(=上演スル)」 
\\	版木(はんぎ)。 
\\	「板の物」の略。 
\\	板敷き。 板縁。 「つややかなる―のはし近う、鮮やかなる畳一枚(ひとひら)うちしきて」	▲私はその板をインチで計った。 ▲壊れた窓は板でふさがれた。
\\	偉大	いだい	[形動][文][ナリ]すぐれて大きいさま。 りっぱであるさま。 「―な業績」「―な人物」 [派生]いだいさ[名]	▲サー・ウィンストン・チャーチルは偉大な政治家であっただけでなく、偉大な作家でもあった。 ▲さすがに偉大な学者だけあって、彼はその問いに容易に答えた。
\\	抱く・懐く	いだく	[動カ五(四)] 
\\	腕でかかえ持つ。 だく。 「ひしと―・く」「母親の胸に―・かれる」 
\\	かかえるように包み込む。 「村々を―・く山塊」「大自然の懐に―・かれる」 
\\	ある考えや感情をもつ。 「疑問を―・く」「青年よ大志を―・け」 
\\	しっかり守る。 擁護する。 「任那(みまな)を―・き守ること、おこたることなきなり」 [可能]いだける→抱(かか)える[用法]	▲赤ん坊はやさしく抱きなさい。 ▲心に抱いた意志とともに。
\\	悪戯	いたずら	[名・形動]スル《「徒(いたずら)」から》 
\\	人の迷惑になることをすること。 また、そのさま。 悪ふざけ。 「―が過ぎる」「―な子」 
\\	いたずら小僧。 いたずらっこ。 「弁当箱をポンと抛(ほう)り上げてはチョイと受けて行く―がある」 
\\	もてあそんではならない物をいじったりおもちゃにしたりすること。 「子供がマッチを―する」「―半分」 
\\	自分のすることを謙遜していう語。 芸事・習い事などにいう。 「歌つくりはほんの―です」 
\\	性的にみだらなふるまいをすること。 「旦那が鳥渡(ちよつと)―をしたくなるのも…無理もねえて」	▲男のこはいたずらが好きである。 ▲男の子のいたずらはしょうがない。
\\	痛み・傷み	いたみ	
\\	病気や傷などによる肉体的な苦しみ。 「腰に―が走る」「傷の―」 
\\	精神的な苦しみ。 悩み。 悲しみ。 「胸の―をいやす」 
\\	(傷み)器物などの損傷。 破損。 「家の―がひどい」 
\\	(傷み)食物、特に果実などの腐敗。 「―が早い果物」	▲私は頭に少し痛みを感じた。 ▲私は突然胃に鋭い痛みを感じた。
\\	至る・到る	いたる	[動ラ五(四)] 
\\	ある目的地・場所に行き着く。 到達する。 「峠を経て山頂に―・る」 
\\	ある時間・時点になる。 「今に―・るも連絡がない」「交渉が深夜に―・る」 
\\	ある段階・状態になる。 結果が…となる。 「大事に―・る」「倒産するに―・る」「事ここに―・ってはやむをえない」 
\\	㋐広い範囲に及ぶ。 行きわたる。 「関東全域に―・る」「恩沢―・らざる所なし」 ㋑細かいところまで行き届く。 「注意が―・らない」「―・らない看護」 
\\	自分の方へやって来る。 到来する。 「好機―・る」「悲喜こもごも―・る」 
\\	㋐(「…から…にいたるまで」の形で)ある範囲の両端の事柄を例示して、その範囲のものすべて、の意を表す。 「頭の先から足の先に―・るまで」 ㋑(「…にいたっては」の形で)中でもそれが極端であることを表す。 「腕力に訴えるに―・っては許しがたい」 
\\	極限に達する。 きわまる。 「徳の―・れりけるにや」 [可能]いたれる	▲昔から今に至るまで存在する、あらゆる社会の歴史は階級闘争の歴史である。 ▲彼らは、カルカッタからニューヨーク市に至るまで、世界中に支部を持っている。
\\	位置	いち	[名]スル 
\\	ものがある所。 ものがあるべき場所。 また、ある場所を占めること。 「―がずれる」「所定の―につく」「青森県は本州の最北端に―する」 
\\	㋐物事が全体の中で占める場所。 「この問題は重要な―を占める」 ㋑人が置かれている状態。 境遇。 立場。 「次期会長と目される―にある」	▲その天文台はよい位置にある。 ▲プエルトリコの位置を地図の上に示しなさい。
\\	市	いち	
\\	毎日、または一定の日に物を持ち寄り売買・交換すること。 また、その場所。 市場。 「―が立つ」「朝顔―」 
\\	多くの人が集まる所。 原始社会や古代社会では、歌垣(うたがき)・祭祀・会合・物品交換などに用いられた場所。 
\\	市街。 町。 「野を越え山越え、…シラクスの―にやって来た」	▲市は月曜ごとに立つ。 ▲女3人と鵞鳥一羽で市ができる。
\\	一時	いちじ	
\\	時刻の名。 →時(じ) 
\\	ある短い時間。 しばらくの間。 天気予報では、連続した三時間以内をいう。 「都合により―休業する」「曇り、―雨」 
\\	過去のある短い時間・期間。 かつて。 「彼は―大臣を務めたことがある」「―はどうなるかと心配した」 
\\	その時だけ。 その場かぎり。 臨時。 「―の間に合わせ」 
\\	一回。 「―払い」	▲彼女の急死で私は一時何も考えられなかった。 ▲彼女は一時に二事をなす能力をもっている。
\\	一度に	いちどに	[副]物事が同時に行われるさま。 いちじに。 いちどきに。 「―入れるのは無理だ」「―花が咲く」	▲バースデーケーキのろうそくを一度に吹き消しなさい。 ▲一度に15以上ご注文いただけるという条件で、STL#3456の特別値引きをいたします。
\\	市場・市庭	いちば	
\\	一定の商品を大量に卸売りする所。 「魚―」「青物―」 
\\	小売店が集まって常設の設備の中で、食料品や日用品を売る所。 マーケット。 「公設―」	▲私たちは市場で野菜と魚を買った。 ▲市場は今日は閑散としていた。
\\	一家	いっか	
\\	一つの所帯。 一つの家族。 「結婚して―を構える」「―の柱」 
\\	家族全体。 家じゅう。 「―をあげて移住する」 
\\	学芸・技術などの一つの流派。 また、独自の権威を認められた存在。 「歌道で―を立てる」 
\\	博徒など、親分子分の関係で結ばれた集まり。 「国定(くにさだ)―」	▲一家はトムを除いて全員、黙ってテレビを見ていた。 ▲一家は1830年頃故国のドイツからシカゴに移住した。
\\	何時か	いつか	[副] 
\\	未来の不定の時を表す。 そのうちに。 「―お会いしたい」「あの国には―行ってみたい」 
\\	過去の不定の時を表す。 いつぞや。 以前。 「―来た道」「―読んだ本」 
\\	時がたつのに気がつかないさま。 いつのまにか。 「―日が暮れていた」 
\\	過去・未来の事柄について、それがいつであったかという疑問、または反語の意を表す。 いつ…したであろうか。 「―若やかなる人など、さはしたりし」 [類語]
\\	そのうち・いずれ・今に・やがて・遅かれ早かれ・遠からず・早晩・他日・後日/
\\	前に・ある時・いつぞや・かつて・以前	▲私はいつかエジプトに行きたい。 ▲私はいつかいかなければならない。
\\	一種	いっしゅ	
\\	㋐一つの種類。 ひといろ。 ㋑同類の中で、少し異なるもの。 「イルカはクジラの―である」 
\\	ある意味で、ほぼ同類と思われるもの。 「彼は―の天才である」 
\\	(副詞的に用いて)どことなく、また、ちょっと他と異なっているさま。 「―独特の書体」「―異様な雰囲気」	▲私は、友人の息子が約6か月間一種の農場研修生として、日中この農場までやってくるのを許可するのに同意した。 ▲厳密に言えば、竹は草の一種である。
\\	一瞬	いっしゅん	一度またたきをするほどの、きわめてわずかな時間。 刹那(せつな)。 副詞的にも用いる。 「―の出来事」「―目を疑った」	▲それは私にとってはらはらする一瞬だった。 ▲どうぞ、皆様も最後の一瞬まで粘り抜いてください。
\\	一生	いっしょう	
\\	生まれてから死ぬまでの間。 終生(しゆうせい)。 生涯。 「幸せな―を送る」「事業に―を捧げる」「―を棒にふる」「―忘れられない出来事」 
\\	やっと生き延びること。 一命。 「九死に―を得る」 
\\	(「一生の…」の形で)生きている間に一度しかないようなこと。 生涯にかかわる重大なこと。 「―の願い」「―の不覚」 [類語]
\\	生涯・人生・終生・畢生(ひつせい)・終身・一生涯・一代・一世(いつせい・いつせ)・一期(いちご)・今生(こんじよう)・ライフ	▲ご親切は一生忘れません。 ▲聞くは一時の恥、聞かぬは一生の恥。
\\	一層	いっそう	㊀[名] 
\\	ひとかさね。 
\\	数層の建物のいちばん下。 ㊁[副] 
\\	程度がいちだんと進むさま。 ひときわ。 ますます。 「寒さが―厳しくなる」「末っ子をより―かわいがる」 
\\	むしろ。 かえって。 いっそのこと。 「何を見てもつまらなく、―消えて了(しま)いたい」	▲彼は転地したためにかえっていっそう悪くなった。 ▲ケンは1人息子なので、父親は一層可愛がった。
\\	一体	いったい	㊀[名] 
\\	一つのからだ。 また、同一のからだのようになっていること。 同体。 「―を成す」「夫婦は―」「三位(さんみ)―」 
\\	一つにまとまっていること。 「クラスが―となる」 
\\	一つの体裁(ていさい)・様式。 「書の―」 
\\	仏像・彫像などの一つ。 「菩薩像―」 
\\	(多く「一体に」の形で副詞的に用いて)全体にならしていうさま。 総じて。 概して。 「―に今年は雪が多い」「日本人は―に表情に乏しい」 ㊁[副] 
\\	強い疑問や、とがめる意を表す。 そもそも。 「―君は何者だ」 
\\	もともと。 元来。 「―生徒が全然悪いです」	▲その時あなたたちはいったい何をしていたのですか。 ▲彼は取りあえず借金は返済したらしいが、一体あんな大金を誰が都合したのだろうか。
\\	一致	いっち	[名]スル 
\\	二つ以上のものがぴったり一つになること。 くいちがいなく同じであること。 合致。 「意見の―をみる」「指紋が―する」「満場―」 
\\	ごく普通の道理。 「気遣ひいたすも―なれば」	▲彼は言行が一致しない。 ▲彼女の言行は一致している。
\\	何時でも	いつでも	どんな時でも。 常時。 「―よいから電話を下さい」「―眠そうだ」	▲暇なときはいつでも遊びに来て下さい。 ▲学者によれば、大きな地震はもういつでも起きておかしくないそうだ。
\\	一般	いっぱん	[名・形動] 
\\	広く全体に共通して認められ、行き渡っていること。 また、そのさま。 全般。 「―の傾向」「―に景気が悪い」 
\\	㋐ありふれていること。 あたりまえ。 普通。 「―の会社」「―市民」 ㋑多くの普通の人々。 世間。 「―に公開する」 
\\	特に違いが認められないこと。 また、そのさま。 同一。 同様。 「私は彼女と同じい罪を犯したも―だ」 [類語]
\\	(「一般に」の形で副詞的に用いる場合)全般に・総じて・概して・多く・おしなべて・おおむね・大概・普通・通例・通常・一体に・総体・およそ・広く/
\\	㋐)普通・通常・尋常・つね/
\\	㋑)世間・世人(せじん)・世俗・万人(ばんにん)・公衆	▲少年犯罪が目立つが、だからといって一般犯罪件数が減少したわけではない。 ▲世界一般のホテル並みに言えばよい朝食会だった。
\\	一方	いっぽう	㊀[名] 
\\	一つの方面。 一つの方向。 「―が海に面する町」「―交通」 
\\	二つあるうちの一つ。 片方。 「―の足に体重をかける」「―の意見だけでは決められない」 
\\	(副助詞的に用いて)もっぱらその方向・方面にかたよること。 …するばかり。 だけ。 「太る―」「蓄える―の人」 
\\	(接続助詞的に用いて)…する反面。 …と同時に。 「根気強い―、短気なところもある」 ㊁[接]関連するもう一つのほうについて言うと。 話かわって。 「君は渡したという。 ―、彼は受け取っていないという」→片方[用法]	▲企業社会が彼女らの活躍の場をどう用意できるか、も大きな課題だが、一方でこれからの日本社会が必要とする創造性豊かな人材を、教育産業がどう育成し、提供していけるかも重要だ。 ▲一方で彼は私の報告書を賞賛したが、他方ではそれを批判した。
\\	移動	いどう	[名]スルある場所から他の場所へ移ること。 「次の会場へ―する」「―図書館」	▲白鳥は当地からフロリダに移動する。 ▲田舎から都会へこの移動はここ2百年以上も続いてきたことである。
\\	従兄弟・従姉妹	いとこ	父または母の兄弟姉妹の子。 おじ・おばの子。 ◆自分との年齢の上下関係や性別によって「従兄」「従弟」「従姉」「従妹」などとも書く。	▲実を言えば彼女は僕のいとこなのです。 ▲私は母方にいとこが三人いる。
\\	稲	いね	
\\	イネ科の一年草。 実が米で、広く主食とされ、水田や畑で栽培し、畑に作るものは陸稲(おかぼ)・(りくとう)とよばれる。 インドまたは東南アジアの原産。 日本では先史時代から栽培。 高さ約一メートル。 春、種を苗代にまいて、梅雨のころ苗を本田に移し植え、秋に収穫。 ふつう飯に炊く粳(うるち)と、餅(もち)にする糯(もち)とがあり、栽培品種は多い。 また収穫の時期により、早稲(わせ)・中稲(なかて)・晩稲(おくて)と区別する。 《季 秋》「道暮れて―の盛りぞちからなる/暁台」 
\\	紋所の名。 ◆イネ科の単子葉植物は約七〇〇属一万種がある。 多くは草本、茎は中空で節があり、葉は細長い。 花はふつう両性花で、穂状につく。 麦・トウモロコシなど主要な穀物が含まれる。	▲農夫達は稲を植えていた。 ▲新種の稲によっては年に2、3回収穫できる物もある。
\\	居眠り	いねむり	[名]スル座ったり腰かけたりしたままで眠ること。 「授業中に―する」「―運転」	▲講演があまりにつまらなかったので、彼は居眠りしてしまった。 ▲昨夜、私はテレビの前で居眠りをしてしまった。
\\	命	いのち	
\\	生物が生きていくためのもとの力となるもの。 生命。 「―にかかわる病気」「―をとりとめる」「―ある限り」 
\\	生きている間。 生涯。 一生。 「短い―を終える」 
\\	寿命。 「―が延びる」 
\\	最も大切なもの。 唯一のよりどころ。 そのものの真髄。 「―と頼む」「商売は信用が―だ」 
\\	運命。 天命。 「年ごとにあひ見ることは―にて老いの数そふ秋の夜の月」 
\\	近世、遊里などで、相愛の男女が互いの二の腕に「命」の一字、または「誰々命」と入れ墨をすること。 また、その文字。 [類語]
\\	生(せい・しよう)・生命・人命・一命・身命(しんめい)・露命・命脈・息の根・息の緒(お)・玉の緒	▲私たちは命を失う危険が有った。 ▲私の命は危なかった。
\\	違反	いはん	[名]スル法規・協定・契約などにそむくこと。 違背。 「ルールに―する」「選挙―」	▲相次ぐ不祥事にも関わらず、警察はオメオメ違反切符を切っている。 ▲彼は我が国の法律に違反した。
\\	衣服	いふく	からだにまとうもの。 着物。 衣装。	▲衣服からその迷子の身元が確認された。 ▲衣服を使い古したらそれをどうしますか。
\\	居間	いま	家族がふだんいる部屋。 居室(きよしつ)。	▲彼らの居間は我が家の2倍の広さがある。 ▲彼はその訪問者を居間に通した。
\\	今に	いまに	[副] 
\\	近い将来に関する推量または決意を表す。 そのうち。 いずれ必ず。 「この空では―雨が降る」「―やっつけてやる」 
\\	(多くあとに打消しの語を伴って用いる)以前の事柄が今に至るまで変わらないさま。 今もなお。 いまだに。 「考え続けて居るが―少しも解決の手掛が出来ぬ」	▲いまに誰も相手にしてくれなくなるよ。
\\	今にも	いまにも	[副]目前に何かが起こりそうなさま。 すぐにも。 今まさに。 「―笑いだしそうな表情」「―壊れるかもしれない」	▲彼女は道化師の仕草を見ていまにも笑い出しそうだった。 ▲彼は今にもここに来るでしょう。
\\	以来・已来	いらい	
\\	その時よりこのかた。 それより引き続き。 「正月―ずっと禁煙している」 
\\	こののち。 今後。 以後。 「―慎みます」	▲彼は妻が死んで以来酒を飲む癖がついた。 ▲彼は去年の2月以来私達に便りをしてこない。
\\	依頼	いらい	[名]スル 
\\	人に用件を頼むこと。 「―を引き受ける」「執筆を―する」 
\\	他人を当てにすること。 頼み。 「―心が強い」	▲出荷済みでしたら、お礼を申し上げるとともに、この依頼を無視してくださるようお願いします。 ▲上司は自分が留守の間事務所の管理を、ブラウン氏に依頼した。
\\	苛苛	いらいら	
\\	刺とげなどが皮膚や粘膜を刺激する不快な感触。 ちくちく。 日葡辞書「テニイライラトサワル」。 「喉が―する」 
\\	ものごとが思うようにならず腹立たしいさま。 「席があくのを―と待つ」「―が高じて不眠症になる」 
\\	イラクサの異称。	▲静かにならないかなぁ。いらいらするよ。 ▲人にいらいらさせられても、すぐに反応しないのが一番です。
\\	いらっしゃい	いらっしゃい	《「いらっしゃる」の命令形》 
\\	おいでなさい。 「こっちへ―」「まだ寝て―」 
\\	歓迎の心持ちを表すあいさつの言葉。 「いらっしゃいまし」の略ともいう。 「やあ―。 どうぞお上がりください」	▲いつでも遊びにいらっしゃい。 ▲何が起こってもいいように用心をしていらっしゃい。
\\	医療	いりょう	医術・医薬で病気やけがを治すこと。 治療。 療治。	▲赤十字は被災者に食料と医療を分配した。 ▲提示された4つの抑制策のうち最も効果があると思われるのが、「予防医療・健康増進活動の大充実」であろう。
\\	岩・巌・磐	いわ	
\\	地殻を形づくっている堅い物質。 
\\	石の大きなもの。 岩石。 いわお。 「一念―をも通す」 
\\	(「錘」「沈子」とも書く) ㋐漁網を沈めるためにつけるおもり。 ㋑船のいかり。 「―下ろすかたこそなけれ伊勢の海のしほせにかかる蜑(あま)の釣り舟」	▲波が岩に激しくぶつかった。 ▲波が岩に打ち寄せた。
\\	祝い・斎	いわい	
\\	めでたいとして喜ぶこと。 祝賀。 「米寿の―」 
\\	祝う気持ちを示す言葉や金品。 「お―を述べる」 
\\	(斎)心身を清らかにして神を祭ること。 「朕みづからうつし―をなさむ」 
\\	(斎)神を祭る所。 また、人。 「是の皇女(ひめみこ)、伊勢の大神(おほむかみ)の―に侍り」 [下接語]内祝い・産(うぶ)祝い・産衣(うぶぎ)の祝い・快気祝い・賀の祝い・喜の字の祝い・心祝い・名付け祝い・八十八の祝い・褌(へこ)祝い・間(ま)祝い・前祝い・身祝い・水祝い・米(よね)の祝い	▲昨日私たちは結婚10周年のお祝いをした。 ▲父は誕生日の祝いに時計をくれた。
\\	祝う・斎う	いわう	[動ワ五(ハ四)] 
\\	めでたい物事を喜ぶ。 ことほぐ。 「新年を―・う」「全線開通を―・う」 
\\	将来の幸運を祈る。 祝福する。 「二人の前途を―・って乾杯する」 
\\	㋐祝福して贈り物をする。 「結婚する友人に時計を―・う」 ㋑祝福して飲食する。 「正月に屠蘇(とそ)を―・う」 
\\	(斎う)身を慎み、けがれを避けて神を祭る。 「祝部(はふり)らが―・ふ社のもみち葉も標縄(しめなは)越えて散るといふものを」 
\\	(斎う)神の力を借りて守る。 「家にして恋ひつつあらずは汝(な)が佩(は)ける太刀になりても―・ひてしかも」 
\\	大切にする。 かしずく。 「わたつみのかざしにさすと―・ふ藻も君がためには惜しまざりけり」 [可能]いわえる	▲私たちは母の45歳の誕生日を祝った。 ▲私たちをパーティーをして新年を祝った。
\\	言わば	いわば	[副]《動詞「い(言)う」の未然形+接続助詞「ば」から》言ってみれば。 たとえて言えば。 「彼は、―業界の救世主だ」	▲その画家は、いわば永遠の少年だ。 ▲良きコーチはいわば選手の親のようなものだ。
\\	所謂	いわゆる	[連体]《動詞「い(言)う」の未然形+上代の受身の助動詞「ゆ」の連体形から》世間一般に言われる。 俗に言う。 よく言う。 「―独身貴族」「これこそ、―瓢箪(ひようたん)から駒というものだ」	▲彼はいわゆるりっぱな紳士だ。 ▲彼はいわゆる音楽的天才である。
\\	インク	インク	筆記や印刷などに用いる有色の液体。 ペン・万年筆用のブルーブラックインクは硫酸鉄 Ⅱ・タンニン酸・没食子酸などの混合液。 明治から第二次大戦前までは「インキ」と書かれるほうが多かった。	▲この万年筆はインクが切れた。 ▲その本のカバーにはインクのしみがいくつかあった。
\\	印刷	いんさつ	[名]スル原稿に従って印刷版を作り、その版面にインクなどをつけて文字・図形を多数の紙や布などに刷りうつすこと。 また、その技術。 印刷版の種類により凸版印刷・平版印刷・凹版印刷などがある。 「ポスターを―する」「―所」「―物」	▲印刷ミスはすぐに指摘されなければならない。 ▲印刷の誤りがたくさん見つかった。
\\	印象	いんしょう	[名]スル 
\\	人間の心に対象が与える直接的な感じ。 また、強く感じて忘れられないこと。 「鮮やかな―を与える」「―が薄い」「第一―」「静かに物象を眺め、自然を―するほどの余裕もなかった」 
\\	美学で、対象が人間の精神に直接与える感覚的あるいは情熱的な影響。 [類語]
\\	感じ・感・観・心象・感銘・直感・感触・心証・イメージ・インプレッション	▲私の印象に残ったのはそらの青さです。 ▲私の仕事で重要なのは具体的な事実や数字であってあいまいな印象ではない。
\\	引退	いんたい	[名]スル役職や地位から身を退くこと。 スポーツなどで現役から退くこと。 「スター選手が―する」「―興行」	▲彼は技師を引退した。 ▲彼は引退したが、いまだに事実上指導者である。
\\	引用	いんよう	[名]スル人の言葉や文章を、自分の話や文の中に引いて用いること。 「古詩を―する」	▲彼はその文法書から多くの用例を引用している。 ▲彼はしばしばシェイクスピアから引用する。
\\	ウィスキー	ウィスキー	大麦・ライ麦・トウモロコシなどを麦芽で糖化し、酵母を加えて発酵させ、蒸留した酒。 オーク樽に貯蔵して熟成する。 国木田独歩、馬上の友「ボーイは―を持て来た」	
\\	嗽	うがい	[名]スル水や薬液などを口に含んで、口やのどをすすぐこと。 含嗽(がんそう)。 「食塩水で―する」「―薬」	
\\	受け取る・請け取る	うけとる	[動ラ五(四)] 
\\	受けて取る。 渡されたものを受け収める。 「手紙を―・る」 
\\	人の言葉や行動などを自分なりの判断で解釈する。 また、納得する。 「話を額面どおりに―・る」「善意に―・る」 
\\	責任をもって引き受ける。 担当する。 「帝隠れさせ給ひしかば、神功皇后御世を―・らせ給ひ」	▲彼は彼女の贈り物を受け取った。 ▲彼女は彼の言葉を同意を意味していると受け取った。
\\	動かす	うごかす	[動サ五(四)] 
\\	物を他の位置に移したり、占めていた位置を変えたりする。 また、配置・地位などを変える。 「箪笥(たんす)を―・す」「人事部から経理部へ―・す」 
\\	もとが固定しているものの一部を揺らす。 震動させる。 「風が梢を―・す」「首を左右に―・す」 
\\	機械などを作動させる。 「モーターを―・す」 
\\	物事のようす・状態・内容を変える。 「市民運動が社会を―・す」「―・しがたい証拠」 
\\	人の心に訴えて感動させる。 気持ちをゆすぶる。 「名演説に心が―・される」 
\\	㋐自分の目的にかなうよう人を行動させる。 「思いのままに人を―・す」 ㋑ものを有効に機能させる。 運用する。 「裏で金を―・して工作する」 [可能]うごかせる	▲重病のために、彼は、たいていの人のようには体を動かすことができない。 ▲学生達は扇動者のアピールに動かされた。
\\	兎	うさぎ	ウサギ目ウサギ科の哺乳類の総称。 ノウサギ類と、飼いウサギの原種であるアナウサギ類とに分けられる。 体長四〇〜六〇センチのものが多く、一般に耳が長く、前肢は短く、後肢は長い。 上唇は縦に裂け、上あごの門歯は二対ある。 飼いウサギの品種は多く、肉は食用、毛皮は襟巻きなどにし、医学実験用・愛玩(あいがん)用ともする。 ウサギ目にはナキウサギ科も含まれる。 《季 冬》	▲「あなたの願い事はなに?」と小さい白いウサギが聞きました。 ▲「あなたは、いつも何を考えているの?」と小さい白いウサギが聞きました。
\\	牛	うし	
\\	偶蹄(ぐうてい)目ウシ科の哺乳類で、家畜化されたもの。 大形で、雌雄ともに二本の頭角をもち、四肢は短い。 すでに滅びたオーロックスより改良されたもので、乳用のホルスタイン・ジャージー、肉用のデボン・但馬(たじま)牛、役用の黄牛・朝鮮牛などがある。 昔からきわめて有益な動物として信仰の対象になったこともある。 広義には、ウシ亜科ウシ族のバイソン・ガウア・バンテン・ヤク・スイギュウなどの総称。 
\\	牛肉。 ぎゅう。 
\\	「牛梁(うしばり)」の略。 
\\	竹や木を家の棟木のように組んで立てたもの。 物を立てかける台にする。	▲それらの牛には焼き印がついている。 ▲ニュージーランドは羊と牛の国です。
\\	失う	うしなう	[動ワ五(ハ四)] 
\\	今まで持っていたもの、備わっていたものをなくす。 「職を―・う」「友情を―・う」 
\\	普通の状態でなくなる。 安定した状態でなくなる。 「気を―・う」「バランスを―・う」 
\\	かけがえのない人をなくす。 死に別れる。 「戦争で父を―・う」 
\\	手に入れかけて、のがしてしまう。 取り逃がす。 「機会を―・う」 
\\	競技・ゲームなどで、相手に得点を入れられる。 「守りのミスから三点を―・った」 
\\	手段や方法などをなくす。 どうしたらよいかわからなくなる。 「人生の指針を―・う」「解決の道を―・う」 
\\	(「…たるを失わない」の形で)…であることができる、十分にその資格がある、の意を表す。 「富岡先生もその一人たるを―・わない」 
\\	消滅させる。 「深き心もなき人さへ罪を―・ひつべし」 
\\	こわしてなくす。 ほろぼす。 「寝殿を―・ひて、異(こと)ざまにも造りかへむ」 
\\	殺す。 「召し出だして―・はん」 [類語]
\\	なくす・なくする・なくなす・喪失する・亡失する・紛失する/
\\	失(しつ)する・逃(のが)す・逃がす・取り逃がす・逸する・ふいにする	▲失われるべき時間はない。 ▲首相は国民の支持を失った。
\\	疑う	うたがう	[動ワ五(ハ四)] 
\\	本当かどうか怪しいと思う。 不審に思う。 うたぐる。 「―・う余地がない」「自分の目を―・う」 
\\	事柄・事態を推測する。 うたぐる。 「にせ札ではないかと―・われる」 
\\	本当かどうか不安に思う。 危ぶむ。 「効果を―・う」 [可能]うたがえる	▲彼の行動には疑う余地がない(ほど立派だ)。 ▲彼の誠実さを疑うのなら彼に援助を求めるな。
\\	宇宙	うちゅう	
\\	あらゆる存在物を包容する無限の空間と時間の広がり。 ㋐哲学では、秩序ある統一体と考えられる世界。 コスモス。 ㋑物理学的には、存在し得る限りのすべての物質および放射を含む空間。 ㋒天文学では、あらゆる天体の存在する空間。 銀河系外星雲を小宇宙、それらを包含する空間として大宇宙ということもある。 
\\	宇宙空間、特に太陽系空間のこと。 「―旅行」 ◆「淮南子(えなんじ)」斉俗訓によれば、「宇」は天地四方、「宙」は古往今来の意で、空間と時間の広がりを意味する。	▲多くの天文学者は、宇宙は永遠に膨張し続けると思っている。 ▲多くの天文学者は、宇宙は永遠に膨張してゆくものだと考えている。
\\	撃つ・射つ	うつ	[動タ五(四)]《「打つ」と同語源》弾丸・矢などを発射する。 「拳銃で―・つ」「標的を―・つ」 [可能]うてる	▲私はその猿を撃たないように警官に説き伏せた。 ▲撃ち合い場面の多い西部劇ですよ。
\\	移す・遷す	うつす	[動サ五(四)] 
\\	位置や地位を変える。 他の所へ持っていく。 また、中身を別のものに入れ替える。 「住まいを―・す」「首都を―・す」「庶務課に―・す」「小皿に―・す」 
\\	目の向きや関心の対象を変える。 「視線を―・す」「別の相手に心を―・す」 
\\	時を過ごす。 時間を経る。 「時を―・さず決行する」 
\\	伝染させる。 「風邪を―・される」 
\\	色や香りを他の物にすりつけて染み込ませる。 「花をすって布地に色を―・す」 
\\	物事を別の段階に進める。 「計画を実行に―・す」 
\\	(遷す)神仏の座所を動かす。 また、分けて他の所に祭る。 「伏見稲荷を―・して守護神とする」 
\\	物の怪(け)を寄坐(よりまし)にのりうつらせる。 「物の怪にいたう悩めば、―・すべき人とて」 
\\	高貴の人を流罪にする。 「いかでか我が山の貫首をば、他国へは―・さるべき」 [可能]うつせる	▲会社は税金上の目的で本社所在地を香港に移した。 ▲信念を行動に移した。
\\	訴える	うったえる	[動ア下一][文]うった・ふ[ハ下二]《「うるた(訴)う」の音変化》 
\\	物事の善悪、正邪の判定を求めて裁判所などの機関に申し出る。 申し立てる。 告訴する。 「警察に―・える」 
\\	有識者などに物事の是非の判断を求めて、申し出る。 「同級生の乱暴を先生に―・える」 
\\	他人の理解・同情・救いなどを強く期待して不満・不平・苦しみなどを言い知らせる。 「腹痛を―・える」「空腹を―・える」 
\\	強い手段を用いて事を解決しようとする。 「腕力に―・える」 
\\	感覚や感情に働きかける。 「良識に―・える」 [類語]
\\	申し立てる・提訴する・告訴する・告発する/
\\	直訴する・直願する・嘆願する・哀訴する・哀願する・泣訴する・愁訴する・泣き付く・掻(か)き口説(くど)く(人々に広く訴える)呼び掛ける・アピールする	▲その中年の男は暴行のかどで訴えられた。 ▲場合によっては腕力に訴えてもよい。
\\	唸る	うなる	[動ラ五(四)]《「う」は擬声語》 
\\	力を入れたり苦しんだりするときに、長く引いた低い声を出す。 うめく。 「痛くてうんうん―・る」 
\\	獣が低く力の入った声を出す。 「犬が―・る」 
\\	鈍く低い音を長く響かせながら出す。 「モーターが―・る」「風が―・る」 
\\	謡曲・浄瑠璃などを、のどをしぼるように低音でうたったり語ったりする。 「義太夫を―・る」 
\\	感嘆のあまり、思わず、 
\\	のような声を出す。 ひどく感心する。 「満員の観衆を―・らせる」 
\\	内に満ちている力が、あふれ出るばかりになる。 「腕が―・る」「金が―・るほどある」	▲うなるほど金を持っている。 ▲そのうなり声はだんだん大きくなった。
\\	奪う	うばう	[動ワ五(ハ四)] 
\\	他人の所有するものを無理に取り上げる。 「金品を―・う」「自由を―・う」「地位を―・う」 
\\	取り去る。 取り除く。 「命を―・う」「地表の熱を―・う」「雪で通勤の足が―・われる」 
\\	注意・関心などを強く引きつける。 夢中にさせる。 「観客の目を―・う華麗な演技」「あまりの美しさに心を―・われる」 
\\	競技などで得点する。 また、獲得する。 「三振を―・う」「タイトルを―・う」 [可能]うばえる [類語]
\\	取る・取り上げる・分捕(ぶんど)る・掠(かす)め取る・もぎ取る・引ったくる・ぶったくる・ふんだくる・攫(さら)う・掻(か)っ攫う・横取りする・強奪する・奪取する・略取する・略奪する・収奪する・簒奪(さんだつ)する・剥奪(はくだつ)する	▲その家は我々から光を奪った。 ▲この法律は我々の基本的な権利を奪うだろう。
\\	馬	うま	《「馬」の字音「マ」から変化したものという。 平安時代以降「むま」と表記されることが多い》 
\\	奇蹄(きてい)目ウマ科の哺乳類。 体は一般に大形で、顔が長く、たてがみがあり、長い毛の尾がある。 力強く、走ることが速い。 古くから家畜とされ、農耕・運送・乗用・競馬などに用いられ、肉は食用。 東洋種の蒙古馬(もうこうま)・朝鮮馬、日本在来種の木曾馬・北海道和種、西洋種のアラビア馬・サラブレッド・ペルシュロンなどがある。 こま。 
\\	踏み台。 脚立(きやたつ)。 
\\	競馬。 
\\	将棋で、桂馬または成角(なりかく)の略称。 
\\	《「付け馬」の略》料理屋などで、勘定の未払いを取り立てるために客の家までついて行く者。 
\\	動植物などの同類の中で、大きなものの意を表す語。 「―すげ」「―ぜみ」	▲君は馬に乗ることはできますか。 ▲君は馬に乗る事が出来ますか。
\\	生まれ	うまれ	
\\	生まれること。 出生(しゆつしよう)。 誕生。 「昭和の―」 
\\	生まれた所。 出生地。 「―は東京」 
\\	その人の生まれた家柄。 素性。 「医者の家の―」 
\\	生まれたときからの性質。 天性。 「お常は、亡兄全次郎には似もつかず、心やさしき―なれば」	▲彼は生まれも育ちも東京です。 ▲彼は名門の生まれなのだ。
\\	梅	うめ	《「梅」の字音「メ」から変化したものという。 平安時代以降「むめ」と表記されることが多い》 
\\	バラ科の落葉高木。 葉は卵形で縁に細かいぎざぎざがある。 早春、葉より先に、白・淡紅・紅色などの香りの強い花を開く。 実は球形で、六月ごろ黄熟し、酸味がある。 未熟なものは漢方で烏梅(うばい)といい薬用に、また梅干し・梅酒などに用いる。 中国の原産で、古くから庭木などにし、品種は三〇〇以上もあり、野梅(やばい)系・紅梅系・豊後(ぶんご)系などに分けられる。 松や竹とともにめでたい植物とされ、花兄(かけい)・風待ち草・風見草・好文木(こうぶんぼく)・春告げ草・匂い草などの別名がある。 《季 春》「二もとの―に遅速を愛すかな/蕪村」 
\\	梅の実。 実梅。 《季 夏》 
\\	襲(かさね)の色目の名。 表は白、裏は蘇芳(すおう)。 うらうめ。 
\\	紋所の名。 梅の花・裏梅・向こう梅など。	▲梅の花は3月にさく。 ▲梅の花は今週が見所です。
\\	裏切る	うらぎる	[動ラ五(四)] 
\\	味方に背いて敵方につく。 「同志を―・る」 
\\	約束・信義・期待などに反する。 「信頼を―・る」「おおかたの予想を―・る」	▲ジョンはあなたを裏切るような人ではない。 ▲その結果は私の期待を裏切った。
\\	得る	える	[動ア下一][文]う[ア下二] 
\\	努力して自分のものにする。 手に入れる。 獲得する。 「利益をえる」「信頼をえる」「承認をえる」 
\\	納得する。 理解できる。 悟る。 「要領をえない質問」「よくその意をえない」 
\\	好ましくないものを身に受ける。 「罪をえる」「病をえる」 
\\	(多く、活用語の連体形に助詞「を」を添えた形に付いて)可能である、の意を表す。 …できる。 「そうせざるをえない」「ようやく監視の目を逃れることをえた」 
\\	(動詞の連用形に付いて) ㋐…できる。 「微笑を禁じえない」 ㋑そのようになる可能性がある。 「交渉決裂もありえる」「起こりえない事故」 
\\	得意とする。 すぐれている。 「これかれえたる所えぬ所、互ひになむある」 ◆鳥や獣のえものを手に入れる場合には「獲る」とも書く。 また、終止するときは文語形の「うる」となることがあり、特に 
\\	の終止形・連体形は「うる」を用いることが多い。 →う(得)る [下接句]簡にして要を得る・九死に一生を得る・御意(ぎよい)を得る・志(こころざし)を得る・事なきを得る・力を得る・時を得る・所を得る・名を得る・万死(ばんし)の中(うち)に一生を得る・要領を得ない・我が意を得る。 蛟竜(こうりよう)雲雨(うんう)を得(う)・虎穴(こけつ)に入(い)らずんば虎子(こじ)を得ず・二兎(にと)を追う者は一兎をも得ず	▲あなたとお話しする機会を得てうれしいです。 ▲そう言う事故は時折起こり得る事だ。
\\	売れる	うれる	[動ラ下一][文]う・る[ラ下二] 
\\	商品などが買われる。 「高値で―・れる」「よく―・れる新製品」 
\\	世間に知られる。 有名になる。 「業界で顔が―・れている」 
\\	人気があって、もてはやされる。 「最近―・れている小説家」	▲苦肉の策で企画したんですが、その本はよく売れました。 ▲最近の日本のレコードで一番売れているのは何だと思いますか。
\\	噂	うわさ	[名]スル 
\\	そこにいない人を話題にしてあれこれ話すこと。 また、その話。 「同僚の交遊関係を―する」 
\\	世間で言いふらされている明確でない話。 風評。 「変な―が立つ」 [類語]
\\	風聞・風説・風評・風声(ふうせい)・風の便り・評判・世評・取り沙汰(ざた)・下馬評・巷説(こうせつ)・浮説・流説・流言・飛語・ゴシップ	▲彼女は噂をふりまく妖精だ。 ▲彼女は噂の真相を調べてみようとした。
\\	運	うん	
\\	人の身の上にめぐりくる幸・不幸を支配する、人間の意志を超越したはたらき。 天命。 運命。 「―が悪い」 
\\	よいめぐりあわせ。 幸運。 「―が向いてくる」「―がない」	▲私は彼の運の良さを羨ましく思った。 ▲私は子供運に恵まれている。
\\	柄	え	
\\	手で握りやすいように、道具類につけた棒状の部分。 取っ手。 「ひしゃくの―」 
\\	キノコの、傘を支える部分。 また、葉柄や花柄。	▲われわれはそんなりっぱな家に住む柄ではない。 ▲この赤色で柄全体がだいなしだ。
\\	永遠	えいえん	[名・形動] 
\\	いつまでも果てしなく続くこと。 時間を超えて存在すること。 また、そのさま。 「―に残る名曲」「―のスター」「―に語り伝える」 
\\	哲学で、それ自身時間の内にありながら、無限に持続すると考えられるもの。 また、数学的真理のように、時間の内に知られても時間とかかわりなく妥当すると考えられるもの。	▲永遠にあなたと一緒にいます。 ▲永遠にあなたを愛します。
\\	永久	えいきゅう	[名・形動]いつまでも限りなく続くこと。 また、そのさま。 「―に平和を守る」「―不変」 [類語]とこしえ・とこしなえ・とわ・ときわ・永遠・恒久・永劫(えいごう)・永世・永代・悠久・久遠(くおん)・無限・無窮・不朽・不変・万代不易・万世不易・万古不易・千古不易	▲彼は永久に日本を離れた。 ▲私は永久に生きたいのです。
\\	影響	えいきょう	[名]スル《影が形に従い、響きが音に応じるの意から》 
\\	物事の力や作用が他のものにまで及ぶこと。 また、その結果。 「環境に―を及ばす」「大勢に―しない」「―力」 
\\	影と響き。 また、物事の関係が密接なこと。 「夫(そ)れ感化の速かなる事、―の如し」	▲天気の変わりやすさは、英国人の正確にはっきりした影響を与えている。 ▲天候の影響をわずかしか受けなければ受けないほど、それはよい時計だ。
\\	営業	えいぎょう	[名]スル 
\\	利益を得る目的で、継続的に事業を営むこと。 また、その営み。 特に、企業の販売活動をいう。 「年中無休で―する」「―マン」 
\\	法律で、継続的に同種の営利行為を行うこと。 また、その活動のために供される土地・建物などの財産をいう。 [類語]
\\	営利事業・経営・商業・商売・商行為・業務・ビジネス:外商・外交・セールス	▲マンハッタンの金融街にある証券取引所の中には、火災と停電のため、営業を早めに切り上げてしまったところもあります。 ▲営業員をこの新しい取引先に送ってください。
\\	衛星	えいせい	
\\	惑星の周りを楕円軌道を描いて公転している天体。 地球に対する月など。 「人工―」 
\\	あるものを中心として、その周辺にあって従属関係にあるもの。	▲衛星は今軌道に乗っている。 ▲木星の周辺を回っているもっと小さな惑星を見てケプラーは、外出中の王の回りを取り囲む護衛を思いだしたので、それを衛星と名付けたのだ。
\\	栄養・営養	えいよう	
\\	生物体が体外から物質を取り入れ、成長や活動に役立たせること。 無機物のみを取り入れる独立栄養と、有機物も取り入れる従属栄養に分けられる。 
\\	栄養となる個々の物質。 栄養素。 また、それを含む食物。 「―を取る」	▲ハンバーガーばかり食べていると、栄養が偏るよ。 ▲栄養があるからといって食べられるものではない。
\\	笑顔	えがお	にこにこと笑った顔。 笑い顔。	▲彼は笑顔で私に挨拶した。 ▲暖かい笑顔で、彼を迎えて。
\\	描く・画く	えがく	[動カ五(四)] 
\\	物の形を絵や図にかき表す。 「田園の風景を―・く」 
\\	物事のありさまを文章や音楽などで写し出す。 描写する。 表現する。 「下町の生活を―・いた小説」 
\\	物事のありさまを心に思い浮かべる。 「夢に―・く」 
\\	物の動いた跡が、ある形をとる。 「波紋を―・く」 [可能]えがける	▲マネがこの絵を描くまで、女性の裸像は女神に限られていました。 ▲やっと絵を描き終えた。
\\	餌	えさ	
\\	動物を飼育したり、誘い寄せたりするための食物。 え。 「金魚に―をやる」 
\\	生き物の食物。 また、食事をわざと下品にいう語。 「スズメが―をあさる」「―にありつく」 
\\	人をだましたり、誘ったりするために提供される金品や利益。 「景品を―に客を集める」	▲羊に餌を与える時間だ。 ▲ヒヨコがえさを探して地面をつついている。
\\	エネルギー	エネルギー	
\\	物事をなしとげる気力・活力。 精力。 「仕事で―を消耗する」「若い―」 
\\	物体が物理的な仕事をすることのできる能力。 力学的エネルギー(運動エネルギーと位置エネルギー)のほか、化学・電磁気・熱・光・原子などの各エネルギーがある。 さらに相対性理論によれば、質量そのものもエネルギーの一形態である。 
\\	「エネルギー資源」の略。	▲怒りはエネルギー。 ▲東アジアの経済はエネルギー価格の上昇で大きな打撃を受けた。
\\	延期	えんき	[名]スル期日や期限を延ばすこと。 「開催を来月に―する」「公開をさらに一週間―する」	▲パーティーは一週間延期された。 ▲フットボールの試合は、悪天候のために、延期された。
\\	演技	えんぎ	[名]スル 
\\	見物人の前で芝居・曲芸・歌舞や、体操などの技を行って見せること。 また、その技。 「模範―」 
\\	本心を隠して見せかけの態度をとること。 「ことさらに仲のよさを―する」	▲僕はそのグループの演技にうっとりしたよ。 ▲彼女は自分の演技が批判されることを懸念していた。
\\	援助	えんじょ	[名]スル困っている人に力を貸すこと。 「資金を―する」「国際―」	▲私は君の援助を確信している。 ▲私は喜んであなたの援助をします。
\\	エンジン	エンジン	機械的エネルギーを継続的に発生させる装置。 内燃機関と外燃機関がある。 自動車のガソリンエンジンなど。 発動機。 原動機。	▲ジョンは車のエンジンをかけた。 ▲そのエンジンはどこか故障している。
\\	演説・演舌	えんぜつ	[名]スル 
\\	大勢の前で自分の意見や主張を述べること。 「税制について―する」「選挙―」 
\\	道理や意義を説き明かすこと。 「宿老畏まって一々に是を―す」	▲知事の演説は記者団に向けて行われた。 ▲聴衆は彼の演説が終わると大きな拍手を送った。
\\	演奏	えんそう	[名]スル音楽を奏すること。 「タンゴを―する」	▲今日のABC交響楽団の演奏は期待はずれだった。 ▲管弦楽団は郷愁に満ちた音楽を演奏し始めた。
\\	老い	おい	
\\	年老いること。 老年。 「―を感じさせない身のこなし」 
\\	年とった人。 老人。 「―も若きも」 
\\	名詞の上に付いて複合語をつくり、年とった、の意を表す。 「―松」「―武者」	▲少年老い易く学成り難し。
\\	追う・逐う	おう	[動ワ五(ハ四)] 
\\	㋐先に進むものに行き着こうとして急ぐ。 あとをついて行く。 追いかける。 「母親のあとを―・う幼な子」「逃走者を―・う」「機影をレーダーで―・う」 ㋑目標となるものに至り着こうとする。 また、あるものを得ようとする。 追い求める。 「理想を―・う」「世の流行を―・う」「掘り出し物を―・って古本屋をまわる」 
\\	㋐順序に従って進む。 「順を―・って話す」「話の筋を―・ってみる」 ㋑時間が経過するのに従って変化する。 「日を―・って忙しくなる」 
\\	無理にその場所・地位などを去らせる。 追い払う。 追い立てる。 「地位を―・われる」「子犬が―・っても―・ってもついてくる」「蠅(はえ)を―・う」 
\\	(「…に追われる」の形で)せきたてられて余裕のない状態である。 「生活に―・われる」「仕事に―・われる」 
\\	せきたてて先に進ませる。 「牛を―・う生活」 
\\	目的の場所を目ざして進む。 「和泉の灘(なだ)より小津のとまりを―・ふ」 
\\	貴人の行列の先払いをする。 先追う。 「容儀をいつくしく整へ、御さきに―・ひて」 [可能]おえる [下接句]顎(あご)で蠅(はえ)を追う・頭の上の蠅を追う・跡を追う・頤(おとがい)で蠅を追う・巻(かん)を追う・先を追う・去る者は追わず・鹿(しか)を逐(お)う・中原(ちゆうげん)に鹿を逐う・二兎(にと)を追う・日を追って	▲パトロール・カーは、違反のスピードを出して走っていたスポーツ・カーを追ったが、結局無駄だった。 ▲私は仕事に追われています。
\\	王	おう	
\\	国などを治める人。 ㋐一国の最高主権者。 君主。 国王。 中国では、始皇帝以後「帝」より一級下の称号。 ㋑儒教で、道徳をもって天下を治める者。 王者。 
\\	皇族で、親王宣下(せんげ)のない男子。 皇室典範では、天皇の三世(旧制では五世)以下の皇族男子。 
\\	同類中、またその道で最もすぐれているもの。 「百獣の―」「発明―」 
\\	将棋の駒の王将。	▲彼は王その人にはならなかった。 ▲この国の王は人ではなく、はるか天空に居られるという三対の翼を持つ神獣なんだ。
\\	王様	おうさま	
\\	王を尊敬、または敬愛していう語。 
\\	㋐同類の中で第一のもの、最高のもののたとえ。 「ドリアンは果物の―だ」 ㋑絶対的な権力・勢力をもつもののたとえ。 「消費者は―だ」	▲ハンプダンプティは塀の上;ハンプダンプティは落っこった;王様の馬と王様の家来を連れてくる;ハンプダンプティを元通りにできなかった。 ▲ライオンはジャングルの王様です。
\\	王子	おうじ	
\\	王の息子。 ↔王女。 
\\	親王宣下のない皇族の男子。	▲王子はその貧しい娘と結婚することを国中に知らせた。 ▲王子は長い旅にでました。
\\	応じる	おうじる	[動ザ上一]「おうずる」(サ変)の上一段化。	▲我々は彼らの要求に応じなければならなかった。 ▲この例文は、書き方のサンプルなので必要に応じて内容を追加削除をしてからお使いください。
\\	横断	おうだん	[名]スル 
\\	横に断ち切ること。 ↔縦断。 
\\	横切ること。 「道路を―する」 
\\	大陸や大洋を東西の方向に通り過ぎていくこと。 「アメリカ大陸を―する」↔縦断。	▲彼らは大西洋を横断した。 ▲大陸横断飛行はまだ大胆な冒険的な事業だった。
\\	終える	おえる	[動ア下一][文]を・ふ[ハ下二] 
\\	続けてきたことを済ませる。 終わらせる。 「学業を―・える」「収穫を―・える」↔始める。 
\\	続いてきたことが終わる。 「仕事が―・えたら、つきあうよ」	▲「君はそれを終えたか」「とんでもない、始めたばかりだ」。 ▲「論文できましたか」「いや、残念ですが、まだ書き終えていません」。
\\	大いに	おおいに	[副]《形容動詞「大いなり」の連用形から》非常に。 はなはだ。 たくさん。 「―感謝している」「今夜は―飲もう」	▲ジェーンは日本語が大いに進歩した。 ▲しかしながら、これらのデータの解釈は大いに議論の対象となっている。
\\	覆う・被う・蔽う・蓋う・掩う	おおう	[動ワ五(ハ四)] 
\\	あるものが一面に広がりかぶさってその下のものを隠す。 「雲が山の頂を―・う」「落ち葉に―・われた道」 
\\	表面にある物を広げて、その物を外界からさえぎられた状態にする。 「ベールで顔を―・う」「目を―・うばかりの惨状」 
\\	すみずみまで行き渡って、いっぱいに満たす。 「あたりは闇(やみ)に―・われた」「場内を―・う熱気」 
\\	本当のことがわからないように、つつみ隠す。 「お師匠様の名によって、おのれの非を―・おうとするのは」 
\\	全体をつつみ含む。 「これを、ひと言で―・えば…」 
\\	広く行き渡らせる。 「威をあまねく海内(かいだい)に―・ひしかども」 [可能]おおえる [用法]おおう・かくす――「おおう」は、表面に何かを広げて、中の物を隠したり、保護したりする動作。 「布団をシーツでおおう」
\\	「隠す」は他人の目に触れないようにすることに重点があり、「両手で顔を隠す」は、両手で顔をおおって顔が見えないようにすること。 
\\	見つからないようにしまい込んだりするのも「隠す」。 「おおう」という方法で「隠す」のが「おおいかくす」で、この場合「かくしおおう」とはならない。 
\\	類似の語に「かぶせる」がある。 「かぶせる」は「帽子をかぶせる」「ふとんをかぶせる」のように上に何かをのせて、下の物を見えないようにしたり、保護したりすること。	▲山の上は、雪で覆われていた。 ▲山の頂上は新雪でおおわれている。
\\	大家・大屋	おおや	
\\	貸家の持ち主。 家主。 ↔店子(たなこ)。 
\\	母屋(おもや)。 
\\	本家(ほんけ)。	
\\	丘・岡	おか	
\\	小高くなった土地。 山よりも低く、傾斜もなだらかなもの。 丘陵。 
\\	(名詞の上に付いて、複合語をつくり)かたわら、局外からの見方や立場のものである意を表す。 「―目」「―ぼれ」「―焼き」	▲新雪が丘に積もって美しく見える。 ▲正午までには丘の頂上に着くだろう。
\\	沖・澳	おき	
\\	海または湖などで、岸から遠く離れた所。 「―に出る」 
\\	広々とした田畑や野原の遠い所。	▲海岸から約1マイル沖に漁船が見えた。 ▲暖流が四国の沖を流れている。
\\	奥	おく	㊀ 
\\	入り口・表から中のほうへ深く入った所。 「洞窟の―」「引き出しの―を探す」 
\\	㋐家屋の、入り口から内へ深く入った所。 家族が起居する部屋。 また、奥座敷。 「主人は―にいます」「客を―へ通す」 ㋑江戸時代、将軍・大名などの城館で、妻妾(さいしよう)の住む所。 「大(おお)―」 
\\	㋐表面に現れない深い所。 内部。 「言葉の―に隠された本音」 ㋑心の底。 内奥(ないおう)。 「心の―を明かす」 ㋒容易には知りえない深い意味。 物事の神髄までの距離。 「―が深い研究」 ㋓芸や学問などの極致として会得されるもの。 奥義。 秘奥。 「茶道の―を極める」 
\\	行く末。 将来。 「伊香保ろの沿ひの榛原(はりはら)ねもころに―をなかねそまさかし良かば」 
\\	物事の終わりのほう。 特に、書物・手紙・巻物などの末尾。 「―より端へ読み、端より―へ読みけれども」 
\\	㋑から》身分の高い人が自分の妻をいう語。 また、貴人の妻の敬称。 奥方。 夫人。 「この―の姿を見るに」→奥さん →奥様 ㊁《「道の奥」の意》奥州。 みちのく。 「風流の初(はじ)めや―の田植うた」	▲あお向けに寝ると、舌やノドチンコがノドの奥に下がるため、上気道が塞がりやすくなります。 ▲部屋の奥の壁の中央に大きな窓がある。
\\	贈る	おくる	[動ラ五(四)]《「送る」と同語源》 
\\	感謝や祝福などの気持ちを込めて、人に金品などを与える。 贈り物をする。 「記念品を―・る」「はなむけの言葉を―・る」 
\\	官位や称号などを与える。 「位階を―・る」 [可能]おくれる	▲彼は、彼女に人形を贈った。 ▲彼はその政治家に多額の賄賂を贈った。
\\	起こる	おこる	[動ラ五(四)] 
\\	今までなかったものが新たに生じる。 おきる。 「静電気が―・る」「さざ波が―・る」 
\\	㋐自然が働きや動きを示す。 おきる。 「地震が―・る」「洪水が―・る」 ㋑平常と異なる状態や、好ましくない事態が生じる。 おきる。 「事件が―・る」「戦争が―・る」 
\\	ある感情・欲望が生じる。 また、からだの働きがある状態を示す。 おきる。 「疑いが―・る」「仏ごころが―・る」「発作が―・る」 
\\	大ぜいの人が出てくる。 大挙する。 「大衆(だいしゆ)―・って西坂本より皆おっかへす」 [類語]
\\	起きる・生ずる・生まれる・兆(きざ)す・発する・生起する・発生する/
\\	㋑)起きる・持ち上がる・出来(しゆつたい)する・勃発(ぼつぱつ)する・突発する・偶発する	▲もし戦争が起こったら、我々はどうなるでしょう。 ▲もし戦争が起こったら君はどうするか。
\\	幼い	おさない	[形][文]をさな・し[ク]《「長(をさ)無し」の意》 
\\	年齢が若い。 幼少である。 いとけない。 「息子はまだ―・い」 
\\	幼稚である。 子供っぽい。 「考え方が―・い」 [派生]おさなげ[形動]おさなさ[名]	▲彼はあまりにも幼いので学校に行けなかった。 ▲聴衆は大部分、幼い子供たちからなっていた。
\\	収める・納める	おさめる	[動マ下一][文]をさ・む[マ下二]《「治める」と同語源》 
\\	一定の範囲の中にきちんと入れる。 収納する。 きまった所にしまう。 「製品を倉庫に―・める」「刀を鞘(さや)に―・める」「カメラに―・める」「胸に―・めておく」 
\\	金や物などを受け取って自分のものとする。 手に入れる。 受納する。 獲得する。 「薄志ですが、―・めてください」「勝利を―・める」「手中に―・める」 
\\	渡すべき金や物を受け取る側に渡す。 納入する。 「授業料を―・める」「注文の品を―・める」「お宮にお札を―・める」 
\\	乱れているものを、落ち着いて穏やかな状態にする。 争いや動揺をしずめる。 治める。 「紛争を―・める」「怒りを―・める」 
\\	物事をそれで終わりにする。 「今日で今年の仕事を―・める」「歌い―・める」 
\\	死骸を葬る。 「骸(から)は、けうとき山の中に―・めて」 [類語]
\\	入れる・仕舞う・仕舞い込む・蔵(ぞう)する・収蔵する・収納する・格納する(書物などに収める)収録する・収載する/
\\	納入する・納付する・上納する・納金する・入金する・払い込む	▲ついに年貢の納め時が来たか。 ▲彼はその美しい風景をカメラに収めた。
\\	汚染	おせん	[名]スル汚れること。 特に、細菌・ガス・放射能などの有毒成分やちりなどで汚れること。 また、汚すこと。 「工場廃液が河川を―する」「大気―」	▲汚染公害は地域の生態環境に壊滅的な影響を与える。 ▲我が国の海岸の汚染はひどく深刻な状態である。
\\	恐らく	おそらく	[副]《「恐らくは」の略》 
\\	確度の高い推量を表す語。 きっと。 「明日は―雨だろう」 
\\	はばかりながら。 「身共は一人ぢゃと思うて、あなどっておぢゃるが、―、いづれも大勢なれども、負くる太郎ではおりないぞ」→多分(たぶん)[用法]	▲つまり、おそらく、初めに調査をした者や技術者には知られていなかった物理的な障害が出てきて、それを克服するためにさまざまな変更が必要になる。 ▲彼はおそらく来るだろう。
\\	恐れる・怖れる・畏れる・懼れる	おそれる	[動ラ下一][文]おそ・る[ラ下二] 
\\	危険を感じて不安になる。 恐怖心を抱く。 「報復を―・れる」「死を―・れる」「社会から―・れられている病気」 
\\	よくないことが起こるのではないかと心配する。 危ぶむ。 「失敗を―・れるな」「トキの絶滅を―・れる」 
\\	近づきがたいものとしてかしこまり敬う。 畏敬(いけい)する。 「神をも―・れぬしわざ」 
\\	《近世江戸語》「恐れ入る 
\\	㋑」に同じ。 「栄螺(さざえ)の壺へ赤辛螺(あかにし)を入れて出すから―・れらあ」 [類語]
\\	こわがる・臆(おく)する・おびえる・びくつく・びくびくする・おどおどする・おじる・おじける・恐怖する・恐れをなす/
\\	危惧(きぐ)する・危懼(きく)する・懸念する・憂える・気づかう/
\\	憚(はばか)る・畏怖(いふ)する	▲彼女は再び病気になるのではないかと恐れている。 ▲彼女は犬をたいへん恐れている。
\\	恐ろしい	おそろしい	[形][文]おそろ・し[シク]《動詞「恐る」の形容詞化》 
\\	危険を感じて、不安である。 こわい。 「―・い目にあう」「戦争になるのが―・い」「ほめるだけほめて後が―・い」 
\\	程度がはなはだしい。 ㋐驚くほどすぐれている。 はかりしれない。 「―・く頭の回転が速い」「人の一念とは―・いものだ」 ㋑驚きあきれるほどである。 ひどい。 「―・く寒い」「こんなことも知らないとは―・い」 [派生]おそろしがる[動ラ五(四)]おそろしげ[形動]おそろしさ[名] [用法]おそろしい・こわい――「草原で恐ろしい毒蛇にあい、怖かった」「彼の恐ろしい考えを知って、怖くなった」のように用いる。 それぞれの「恐ろしい」「怖い」を入れ換えるのは不自然である。 
\\	「恐ろしい」は、「怖い」に比べて、より客観的に対象の危険性を表す。 「怖い」は主観的な恐怖感を示す。 「草原で恐ろしい蛇にあって」も「怖い」とは感じない場合もあるわけである。 
\\	「恐ろしい」は、「日曜の行楽地は恐ろしいばかりの人出だ」「習慣とは恐ろしいものだ」のように、程度がはなはだしいとか、驚くほどだ、ということを示す場合もある。 この場合、「怖い」とはふつう言わない。 「怖いほどの人出」と言えば、自分に危険が及びそうな、という主観的表現となる。	▲原子爆弾は恐ろしい武器だ。 ▲彼があんな恐ろしいことをしたのは間違いない。
\\	お互い	おたがい	「互い」を丁寧にいう語。 自分と相手。 また,双方。 「―の立場を尊重する」「―年をとったね」	▲ほとんどの人々がお互いをよく知らない。 ▲マサシとタカコは、どこで休暇を過ごすかでおたがいけんかをしていた。
\\	穏やか	おだやか	[形動][文][ナリ]《形容動詞「おだ(穏)い」から派生した「おだいか」の音変化》 
\\	静かでのどかなさま。 安らか。 「―な天気」「世の中が―だ」 
\\	気持ちが落ち着いていて物静かなさま。 「―な人柄」「―に話し合う」「心中―でない」 
\\	極端でなく、人に受け入れられやすいさま。 穏当。 「新制度へ―に移行する」「こう言っては―でないかもしれないが」 [派生]おだやかさ[名]	▲母は穏やかではないようだった。 ▲彼女は穏やかな口調で話した。
\\	劣る	おとる	[動ラ五(四)] 
\\	価値・能力・質・数量などが、他に比べて程度の低い状態にある。 引けを取る。 「技量は数段―・る」↔勝る。 
\\	(「…におとらず」の形で)…と同じように。 「今日も昨日に―・らず暑い」 
\\	身分・階級などが下である。 「―・りたる人の、ゐずまひもかしこまりたるけしきにて」 
\\	年齢が下である。 年月が後である。 「年、我より少し―・りたるをば弟の如く哀れび」 
\\	減る。 損をする。 「益(まさ)る所無くして、―・り費(つひ)ゆる極めて甚し」	▲彼は英語が苦手だが、数学では誰にも劣らない。 ▲彼はほとんどまったく君に劣っていない。
\\	鬼	おに	《「おん(隠)」の音変化で、隠れて見えないものの意とも》 ㊀[名] 
\\	仏教、陰陽道(おんようどう)に基づく想像上の怪物。 人間の形をして、頭には角を生やし、口は横に裂けて鋭い牙(きば)をもち、裸で腰にトラの皮のふんどしを締める。 性質は荒く、手に金棒を握る。 地獄には赤鬼・青鬼が住むという。 
\\	のような人の意から》 ㋐勇猛な人。 「―の弁慶」 ㋑冷酷で無慈悲な人。 「渡る世間に―はない」「心を―にする」 ㋒借金取り。 債鬼。 ㋓あるひとつの事に精魂を傾ける人。 「仕事の―」「土俵の―」 
\\	鬼ごっこや隠れんぼうで、人を捕まえる役。 「―さん、こちら」 
\\	紋所の名。 鬼の形をかたどったもの。 
\\	目に見えない、超自然の存在。 ㋐死人の霊魂。 精霊。 「異域の―となる」 ㋑人にたたりをする化け物。 もののけ。 「南殿(なんでん)の―の、なにがしの大臣(おとど)脅かしけるたとひ」 
\\	飲食物の毒味役。 「鬼一口の毒の酒、是より毒の試みを―とは名付けそめつらん」→鬼食(おにく)い →鬼飲(おにの)み ㊁〔接頭〕名詞に付く。 
\\	荒々しく勇猛である意を表す。 「―将軍」 
\\	残酷・無慈悲・非情の意を表す。 「―婆(ばば)」「―検事」 
\\	外見が魁偉(かいい)・異形であるさま、また大形であるさまを表す。 「―歯」「―やんま」 [下接語]異郷の鬼・牛鬼・屈(かが)み鬼・隠れ鬼・心の鬼・人鬼・向かい鬼・雪鬼	▲来年のことを言えば鬼が笑う。 ▲寺の隣に鬼が住む。
\\	帯	おび	《身に帯びるものの意》 
\\	和服を着るとき、腰の辺りに巻いて結ぶ細長い布。 「―を結ぶ」「―を緩める」 
\\	岩田帯。 また、それを巻くこと。 
\\	に似た形のもの。 帯紙の類。 
\\	「帯番組」の略。 ラジオやテレビで、毎日または毎週同時刻に設けられている番組。 
\\	太刀。 「二つなき宝にめで給ふ―あり」	▲彼女の着物と帯の取り合わせはおつだね。 ▲帯に短し、襷に長し。
\\	オフィス	オフィス	事務所。 会社。 官公庁。 「―街」	▲彼は大急ぎでオフィスを去った。 ▲彼は新しいパソコンを自分のオフィスに入れてもらうことを期待して、いつも社長にごまをすっている。
\\	溺れる	おぼれる	[動ラ下一][文]おぼ・る[ラ下二]《「おぼ(溺)ほる」の音変化》 
\\	泳げないで死にそうになる。 また、水中に落ちて死ぬ。 「川で―・れる」 
\\	理性を失うほど夢中になる。 心を奪われる。 ふける。 「酒色に―・れる」	▲さみしさにおぼれ深く落ちてゆく。 ▲彼はコカインに溺れている。
\\	思い出	おもいで	
\\	過去に自分がであった事柄を思い出すこと。 また、その事柄。 「―にひたる」 
\\	あることを思い出すよすがになるもの。 「旅の―に写真を撮る」	▲その写真を見て、子供の頃楽しかった思い出がよみがえった。 ▲よい思い出になりました。
\\	面	おもて	《「おも(面)」に、方向・方面を表す「て」の付いたもの。 正面のほう、の意》 
\\	顔面。 顔。 「―を赤らめる」「―を伏せる」 
\\	仮面。 めん。 主として能・狂言に用いるものにいう。 
\\	物の表面。 外面。 「池の―」 
\\	面目。 体面。 「いづこを―にかはまたも見え奉らむ」	▲その事件の説明は、紙面が足りないため割愛された。
\\	主に	おもに	[副]主として。 大部分。 ほとんど。 「大学では―物理を学んだ」→主(おも)	▲米国において「リベート」は、主にメーカーが消費者に直接提供するインセンティブ手段として広く認識されている。 ▲特に、調和平均の実際の使用例としては、「平均速度」が主に取り上げられ、説明がそこで終わってしまうのが通例である。
\\	思わず	おもわず	㊀[副] 
\\	そのつもりではないのに。 考えもなく。 無意識に。 「―かっとなる」「うれしくて―跳び上がる」 
\\	思いがけず。 意外なことに。 「四鳥の塒(ねぐら)に親と子の、―帰り逢ひながら」 ㊁[形動ナリ] 
\\	思いのほか。 意外。 案外。 「かかる世を見るより外に、―なる事は何事にか」 
\\	心外である。 期待がはずれて気に入らない。 「いかが、世の中さびしく、―なる事あるとも忍び過ぐし給へ」	▲その光景を見て、私は思わず吹き出してしまった。 ▲海外に行くと、気が大きくなって思わず使いすぎちゃうんだよね。
\\	泳ぎ・游ぎ	およぎ	泳ぐこと。 また、泳ぎ方。 水泳。 「―を習う」「―がうまい」《季 夏》	▲彼は泳ぎがうまいのを自慢している。 ▲直子さんは泳ぎ手です。
\\	凡そ	およそ	《「おおよそ」の音変化》 ㊀[名・形動] 
\\	物事のだいたいのところ。 大要。 あらまし。 「計画の―は承知している」「―の見積もりを立てる」 
\\	いいかげんなさま。 ぞんざいなさま。 「かやうに大事の謡ひを―にしては叶(かな)ふまじい」 ㊁[副] 
\\	大まかに言って。 だいたい。 約。 「―二キロ離れている」「被害は―どのくらいか」 
\\	そもそも。 総じて。 一般に。 話を切りだすときに用いる。 「―日本人は働きすぎるきらいがある」 
\\	(否定的な表現を伴って用いる)全く。 全然。 「これは―おもしろくない本だ」 [類語] 
\\	おおよそ・あらまし・概略・大略・大約・大要・大体・一通り/ 
\\	ほぼ・ざっと・大体・約・大約・大略・無慮(むりよ)・かれこれ	▲地球はおよそ50億年前に生まれた。 ▲彼は、収入のおよそ十分の一を貧しい人に与えた。
\\	及ぼす	およぼす	[動サ五(四)]及んだ状態にする。 ある作用・影響などが達するようにする。 「感化を―・す」「甚大な被害を―・す」 [可能]およぼせる	▲あなたに被害を及ぼそうなどとは夢にも思いませんでした。 ▲彼は自分の生命の危険を及ぼすほどの馬鹿ではない。
\\	下ろす・降ろす	おろす	[動サ五(四)] 
\\	上から下に移動させる。 ㋐高い所から低い方へ移す。 「屋根の雪を―・す」「腰を―・す」↔上げる。 ㋑操作によって物がおりた状態にする。 下に垂らす。 「ブラインドを―・す」↔上げる。 ㋒陸から水面に移す。 「ボートを―・す」↔上げる。 
\\	掲げたものを取り外す。 「旗を―・す」「看板を―・す」 
\\	㋐乗り物などから外へ出す。 「駅前で客を―・す」「積み荷を―・す」 ㋑(「堕ろす」とも書く)体外へ出す。 堕胎する。 「子を―・す」 
\\	神仏・貴人・客などに供した物を下げる。 また、お下がりをもらう。 「膳を―・す」「供物(くもつ)を―・す」 
\\	生えているものを切ったり、そったりして落とす。 「枝を―・す」「髪を―・す」 
\\	役職・地位を下げたり、辞めさせたりする。 「主役を―・される」 
\\	深部へしっかりと伸ばす。 「木が根を―・す」 
\\	㋐料理のために魚・獣の肉を切り分ける。 「アジを三枚に―・す」 ㋑(「卸す」とも書く)下ろし金ですり砕く。 「わさびを―・す」 
\\	納めてある物や、あらかじめ作ってある物を取り出す。 ㋐引き出す。 「貯金を―・す」 ㋑衣類・道具などの、新品を初めて使う。 「新調の服を―・す」 ㋒別の用途に当てる。 「古タオルを雑巾に―・す」 
\\	扉をしめる。 かぎをかける。 とざす。 「鎧戸(よろいど)を―・す」 
\\	製版・印刷に回す。 下版する。 
\\	神降ろしをする。 「この類の小さな神を招き―・す方式となっていたものであろう」 
\\	貴人の前から退出させる。 「みな下屋(しもや)に―・しはべりぬるを」 
\\	悪く言う。 けなす。 「ここにても、また―・しののしる者どもありて」 
\\	高い所から風が吹く。 吹き下ろす。 「三室山(みむろやま)―・す嵐の寂しきに妻よぶ鹿の声たぐふなり」 [可能]おろせる [下接句]錨(いかり)を下ろす・重荷を下ろす・飾りを下ろす・頭(かしら)を下ろす・肩揚げを下ろす・髪を下ろす・看板を下ろす・錠を下ろす・根を下ろす・暖簾(のれん)を下ろす・箸(はし)を下ろす・筆を下ろす	
\\	恩	おん	人から受ける、感謝すべき行為。 恵み。 情け。 「―を施す」	▲命がある限りあなたのご恩は忘れません。 ▲彼は私の命を助けてくれたので非常に恩がある。
\\	温暖	おんだん	[形動][文][ナリ]気候があたたかなさま。 「―な地方」↔寒冷。	▲一般的に言えば、日本の気候は温暖です。 ▲より温暖な気候の中でゴルフやテニスといったスポーツを楽しもうと陽光地帯(サンベルト)へ引っ越す退職者も多い。
\\	温度	おんど	物体のあたたかさ・冷たさを示す尺度。 熱力学的には物体中の分子や原子の平均運動エネルギーに比例した量を示す。 普通の温度計ではセ氏温度やカ氏温度による目盛りがつけられ、熱力学では絶対温度が用いられる。	▲体は温度の変化にすばやく順応する。 ▲他の条件が等しいなら、温度がこの実験でもっとも影響を与える要素であるに違いない。
\\	可	か	
\\	良い悪いの二段階評価で合格を示す。 「栄養―」 
\\	《「可能」の略》よいとして許すこと。 「分売も―」 
\\	成績などの段階を示す語。 優、良の次。 学校の成績評価では、及第を認められるものの最下位。	▲学生に限り入場可。 ▲数学は何とか可を取った。
\\	課	か	
\\	事務機構の小区分。 多く局・部の下にあり、係の上にある。 「人事異動で―が変わる」「資材―」 
\\	教科書などの内容のひと区切り。 単元より小さい単位。 「前の―を復習する」「第一―」	▲この課で何か質問がありますか。 ▲その大学には学生のための就職課がある。
\\	カー	カー	
\\	自動車。 「マイ―」「―ラジオ」 
\\	列車の車両。 「ロマンス―」	▲ハーツ社とエイビィス社はカーレンタルの業界でしのぎを削っている。 ▲彼らはGTカーが大好きだった。
\\	カード	カード	
\\	小さな四角い紙。 特に、ある規格に従ってそろえたものをいう。 「単語―」「蔵書―」「クリスマス―」 
\\	トランプ。 また、その札。 「―を切る」 
\\	野球などで、試合の組み合わせ。 「好―」 
\\	「クレジットカード」「キャッシュカード」「テレホンカード」などの略。	▲私のカードにつけておいてくれ。 ▲支払にカードはつかえますか。
\\	害	がい	悪い結果や影響を及ぼす物事。 「健康に―がある」「農作物に―を及ぼす」↔益。	▲運動不足が健康に害を及ぼすかもしれない。 ▲過度の飲酒は健康に害となることがある。
\\	会員	かいいん	ある会に加わっている個人または法人。 「―制クラブ」「名誉―」	▲この会の会員ですか。 ▲このクラブの会員は50名です。
\\	絵画	かいが	造形美術の一。 線や色彩で、物の形・姿を平面上に描き出したもの。 絵。 画。	▲この絵画を描いたのは叔父です。 ▲これは絵画だ。
\\	海外	かいがい	海の向こうの国。 外国。 「―に進出する」「―旅行」	▲彼女は海外生活をしている。 ▲父は海外にいる間、手紙や電話でわれわれと接触し続けた。
\\	会計	かいけい	
\\	代金の支払い。 勘定。 「―をすませて店を出る」 
\\	金銭の収支や物品・不動産の増減など財産の変動、または損益の発生を貨幣単位によって記録・計算・整理し、管理および報告する行為。 また、これに関する制度。 [類語]
\\	支払い・勘定・精算・お愛想(あいそ)・レジ/
\\	計理・経理・出納(すいとう)・簿記・帳付け	▲会計の窓口はどこですか。 ▲会計はお帰りのレジでおねがいします。
\\	解決	かいけつ	[名]スル 
\\	問題のある事柄や、ごたごたした事件などを、うまく処理すること。 また、かたづくこと。 「紛争を―する」 
\\	疑問のあるところを解きほぐして、納得のいくようにすること。 また、納得のいくようになること。 「疑問が―する」	▲このようにして、問題が解決されたので、皆は非常に安心した。 ▲このようにして私はその問題を解決した。
\\	会合	かいごう	[名]スル 
\\	相談・討議などのために人が寄り集まること。 また、その集まり。 寄り合い。 
\\	⇒合(ごう) 
\\	同種の分子またはイオンが集まって、水素結合や分子間力などの比較的弱い結び付きにより、一つの分子またはイオンのように動くこと。	▲これはとても大切な会合だ。 ▲例によって、マイクは今日の午後会合に遅刻した。
\\	外交	がいこう	
\\	外国との交渉・交際。 国家相互の関係。 
\\	外部との交渉・交際。 特に会社・商店などで、外部に出て勧誘・受注などの仕事をすること。 また、その人。 「―販売」	▲上手い外交官は、人に秘密を漏らさせる手をいつも使う人である。 ▲彼はアメリカ大使館の外交官である。
\\	開始	かいし	[名]スル始めること。 また、始まること。 「交渉を―する」「試合―」↔終了。	▲彼らは敵に激しい攻撃を開始した。 ▲搭乗開始は何時からですか。
\\	解釈	かいしゃく	[名]スル 
\\	言葉や文章の意味・内容を解きほぐして明らかにすること。 また、その説明。 「徒然草を―する」「英文―」 
\\	物事や人の言動などについて、自分なりに考え理解すること。 「善意に―する」 [類語]
\\	釈義・講釈・評釈・解義・義解(ぎげ)・読解(―する)釈する・説き明かす/
\\	理解・判断(―する)解する・取る・受け取る・とらえる	▲その問題には一つの解釈しかない。 ▲我々の解釈では、表2に示された出力データは表1のデータの容認できる変異形と言える。
\\	外出	がいしゅつ	[名]スル 
\\	自宅や勤め先などから、よそへ出かけること。 「急用で―する」 
\\	物が外部に出ていくこと。 「正金の次第に―するは当然のことなり」	▲激しく雨が降っていたので、ナンシーは外出するのをためらった。 ▲今は外出しないで。もうすぐ昼ご飯にするとこだから。
\\	改善	かいぜん	[名]スル悪いところを改めてよくすること。 「生活を―する」↔改悪。	▲2国間の貿易上のアンバランスを改善しなければならない。 ▲事態はまだ改善可能だ。
\\	快適	かいてき	[名・形動]心身に不快に感じられるところがなく気持ちがいいこと。 ぐあいがよくてこころよいこと。 また、そのさま。 「―な生活」「―な一日を過ごす」 [派生]かいてきさ[名]	▲あなたの親切のおかげで快適な旅ができ、とても感謝しています。 ▲エコロジーのために堪え忍ぶのではなく、自然と調和した住環境の快適性が必要である。
\\	回復・恢復	かいふく	[名]スル 
\\	悪い状態になったものが、もとの状態に戻ること。 また、もとの状態に戻すこと。 「健康が―する」「ダイヤの乱れが―する」 
\\	一度失ったものを取り返すこと。 「名誉を―する」「信用―」 [類語]
\\	復旧・復元・復興・復調・復活・蘇生(そせい)(―する)復する・戻る・蘇(よみがえ)る・立ち直る・持ち直す (健康状態について)快復・平復・平癒・治癒・快癒・本復・回春・全快・快気・快方/
\\	挽回(ばんかい)(―する)取り戻す・取り返す・盛り返す	▲残念な事に私の父は長煩いから回復できなかった。 ▲私がほっとしたことに、彼は病気から回復した。
\\	飼う	かう	[動ワ五(ハ四)] 
\\	食べ物を与えて動物を養い育てる。 「猫を―・う」 
\\	動物に食べ物・水を与える。 「鷹に―・はんとて、生きたる犬の足を斬り侍りつるを」 
\\	毒や薬などを飲ませる。 盛る。 「馬銭(まちん)といへる毒を―・うて殺したるらん」 [可能]かえる	▲彼の叔母は猫を3匹飼っている。 ▲彼が飼っている犬は時々見知らぬ人に吠えた。
\\	換える・替える・代える	かえる	[動ア下一][文]か・ふ[ハ下二] 
\\	(換える・替える)相手に与える代わりに、相手のものを自分のものとする。 等しいもの、同種のものを他とやりとりする。 交換する。 「円をドルに―・える」「現金に―・える」「小銭に―・える」 
\\	(代える)あるものに他のものと同じ役目・働きをさせる。 「書面をもってあいさつに―・えさせていただきます」 
\\	(換える・替える)今まで使っていたものを別のものにする。 古くなったものを新しいものにする。 「畳の表を―・える」「かみそりの刃を―・える」 
\\	(代える)飲食物のおかわりをする。 「御飯を―・えてください」 
\\	(「…に替える」の形で)…を犠牲にする。 …と引きかえにする。 「命に―・えて子供を守る」 
\\	(動詞の連用形に付いて)今までしていたのをやめて、新たに同じことを行う。 「乗り―・える」「着―・える」「張り―・える」 ◆室町時代以降はヤ行にも活用した。 →換ゆ [下接句]命に替える・色を替え品を替える・裘葛(きゆうかつ)を易(か)える・背に腹はかえられない・手を替え品を替え	
\\	香り・薫り・馨り	かおり	
\\	よいにおい。 香気。 
\\	顔などのにおいたつような美しさ。 「―をかしき顔ざまなり」→匂(にお)い[用法]	▲このバラは実に甘い香りがする。 ▲このワインは、香りにおいて、あのワインに劣る。
\\	画家	がか	絵をかくことを職業とする人。 絵かき。 「日曜―」	▲日本の画家が空間によって生み出したようなすぐれた対象の効果。 ▲私は長い間、画家になりたいと思っている。
\\	抱える	かかえる	[動ア下一][文]かか・ふ[ハ下二] 
\\	物を囲むように腕を回して持つ。 胸にだくようにして持つ。 「ひざを―・えて座る」「包みを小脇に―・える」 
\\	自分の負担になるものをもつ。 厄介なもの、世話をしなければならないものを自分の身に引き受ける。 「多くの負債を―・えて倒産する」「妻子を―・えて路頭に迷う」 
\\	人を雇う。 雇って使う。 「運転手を―・える」 
\\	その範囲の内にもつ。 また、まわりを囲む。 「湾を―・えた地勢」 
\\	かばう。 保護する。 「流罪せられよと、公家に申ししかども、君―・へ仰せられしを」 
\\	今の状態を保ちつづける。 維持する。 「今一両日は―・へて見申し候ふべし」 ◆室町時代以降はヤ行にも活用した。 →抱(かか)ゆ [用法]かかえる・だく――「人形をかかえる(だく)」などでは相通じて用いられる。 
\\	「かかえる」は荷物などを腕で囲んで、胸や脇に持つこと。 「かばんを小脇にかかえる」「大きなふろしき包みを両手でかかえる」のように用いる。 また、抽象的に「三人の遺児をかかえて途方に暮れる」「借金をかかえる」などとも用いる。 
\\	「だく」は赤ん坊や恋人などを胸のところで支え持つ意。 「幼子キリストを胸にだくマリア」「病児をしっかりとだいている母親」「強くだいて!」「鳥が卵をだく」などと一般に用いられる。 この場合「かかえる」では置き換えられない。 
\\	「だく」の類似の語に「いだく」がある。 「いだく」は、やや古い語であるとともに、心の中にある感情・考えをもつ意味もあり、「おそれをいだく」「大志をいだく」などと用いられる。	▲彼が忘れていった書類の束を抱えて彼の後を追いかけた。 ▲言葉は国際結婚がかかえている基本的な問題である。
\\	価格	かかく	商品の価値を貨幣で表したもの。 値段。	▲この車の価格はとても高い。 ▲この中古車の価格は手ごろだ。
\\	化学	かがく	
\\	物質を構成する原子・分子に着目し、その構造や性質、その構成の変化すなわち化学反応などを取り扱う自然科学の一部門。 対象や研究方向により、無機化学・有機化学・物理化学・生化学・地球化学・核化学などに分けられる。	▲彼の化学の講義は苦痛以外のなにものでもなかった。 ▲化学には少し知識があります。
\\	輝く・耀く・赫く	かがやく	[動カ五(四)]《古くは「かかやく」》 
\\	まばゆいほどきらめく。 きらきら光る。 光を放つ。 「ネオンが―・く」 
\\	生き生きとして明るさがあふれる。 「希望に―・く未来」 
\\	名誉や名声を得て華々しい状態にある。 名が上がる。 威光が現れる。 「栄冠に―・く」 
\\	恥ずかしがる。 まばゆく思う。 「見苦しげなる人々も、―・き隠れぬるほどに」	▲彼女は目を輝かせて私の言うことを聞いた。 ▲彼女は目を輝かせて走ってきた。
\\	係り	かかり	〔動詞「かかる(係)」の連用形から〕 
\\	特定の仕事・役目を受け持つこと。 また,その人。 「―の者を呼んで来ます」 〔「受付―」「会計―」のように名詞の下に付くときは,多く「がかり」の形になり,「係」と書く。 ただし,官庁や鉄道などの場合は多く「掛」と書く〕 
\\	〔文法〕 係り結びで,呼応する文末の活用語に定まった活用形をとらせる助詞。 
\\	関係。 かかわり。 「今は仏がゆかり―の者ども,はじめて楽しみ栄えた/天草本平家 
\\	▲申込書に記入した後で、登録係から手数料が8ドルだといわれた。 ▲販売係の職員は全部一週間昼夜ぶっ通しで働いた。
\\	限る	かぎる	[動ラ五(四)] 
\\	時間・空間・数量・資格などに境をし、範囲を定める。 「期限を十日と―・る」「申し込みは一人一枚に―・ります」「荒いが此の風、五十鈴川で―・られて、宇治橋の向うまでは吹くまいが」 
\\	㋐特にそれと限定する。 「その件に関する―・り、問題はない」「うちの子に―・り、そんなことをするはずはない」→限って ㋑(「…に限る」の形で)他に、これに勝るものがない。 最上である。 「夏はビールに―・る」「こういうときにはその手に―・る」 ㋒(打消しの語を伴って)そうと断定できない意を表す。 「酒がからだに毒だとは―・らない」 [可能]かぎれる	▲教育は若い時代に限られてはならず、われわれの全生涯を通じて継続して行われるものでなければならない。 ▲金持ちが必ずしも幸福であると限らない。
\\	家具	かぐ	家に備えつけ、日常使用する道具類。 たんす・机・いすなど。	▲彼らの家具はデザインよりむしろ実用性の観点から選ばれていた。 ▲彼は彼らと家具の売買契約をした。
\\	額	がく	
\\	数量。 特に、金銭の量。 「賠償金の―」「―を上積みする」 
\\	物の量。 「生産の―」 
\\	書画を枠に入れて室内の壁などに掛けておくもの。 また、その枠。 額縁(がくぶち)。 
\\	紋所の名。 
\\	を図案化したもの。 [アクセント] 
\\	はガク、 
\\	はガク。	▲その絵は装飾の施された額に入れられた。 ▲統計的に予測した歳入見込み額ならお知らせできます。
\\	学	がく	
\\	学ぶこと。 学問。 「―にいそしむ」 
\\	学識。 知識。 「―がある」	▲彼は非常に学がある。 ▲スミス教授は英語学では、最高級の学者の一人とみとめられている。
\\	覚悟	かくご	[名]スル 
\\	危険なこと、不利なこと、困難なことを予想して、それを受けとめる心構えをすること。 「苦労は―のうえだ」「断られるのは―している」 
\\	仏語。 迷いを脱し、真理を悟ること。 
\\	きたるべきつらい事態を避けられないものとして、あきらめること。 観念すること。 「もうこれまでだ、と―する」 
\\	覚えること。 記憶すること。 「時にあたりて本歌を―す」 
\\	知ること。 存知。 「郎従小庭に伺候の由、全く―仕らず」	▲自由を守るためにどんな事でもする覚悟です。 ▲私は日本に骨を埋める覚悟でやってきた。
\\	確実	かくじつ	[名・形動]たしかで、まちがいのないこと。 また、そのさま。 「―な情報」「―に成功する」「当選―」 [派生]かくじつさ[名]	▲常に戦争の用意が出来ていることは戦争を避ける最も確実な道であるとメントールは言っている。 ▲女性の立場は多年にわたって確実に好転している。
\\	学者	がくしゃ	
\\	学問の研究を仕事としている人。 
\\	学問のある人。 豊富な知識のある人。	▲彼は学者というよりもむしろテレビタレントである。 ▲彼は学者というよりもむしろ詩人である。
\\	学習	がくしゅう	[名]スル 
\\	学問・技術などをまなびならうこと。 「―の手引」「―会」 
\\	学校で系統的・計画的にまなぶこと。 「英語を―する」 
\\	人間も含めて動物が、生後に経験を通じて知識や環境に適応する態度・行動などを身につけていくこと。 不安や嫌悪など好ましくないものの体得も含まれる。	▲これは生物の学習法としてはあまりよくないと思う。 ▲第13週:絶対運動と相対運動について学習する。
\\	隠す	かくす	[動サ五(四)] 
\\	人の目に触れないようにする。 物で覆ったり、しまい込んだりする。 「姿を―・す」「両手で顔を―・す」「押し入れに―・す」 
\\	物事を人に知られないようにする。 秘密にする。 「身分を―・す」「―・さずに事実を話す」「当惑の表情を―・さない」 
\\	死者を葬る。 「畝傍山の東北(うしとら)のすみの陵(みささき)に―・しまつる」→覆(おお)う[用法] [可能]かくせる [下接句]頭隠して尻(しり)隠さず・跡を隠す・色の白いは七難隠す・髪の長きは七難隠す・上手(じようず)の猫が爪(つめ)を隠す・爪を隠す・能ある鷹(たか)は爪を隠す	▲必要な場合があれば彼は自分の感情をかくすことができる。 ▲彼女は悲しみを隠して晩年を送った。
\\	拡大	かくだい	[名]スル広げて大きくすること。 また、広がって大きくなること。 郭大。 「勢力の―を図る」「写真を―する」↔縮小。	▲彼らの事業は拡大している。 ▲彼は研究の対象を拡大した。
\\	確認	かくにん	[名]スル 
\\	はっきり認めること。 また、そうであることをはっきりたしかめること。 「安全を―する」「生存者はまだ―できない」 
\\	特定の事実や法律関係の存否について争いや疑いのあるとき、これを判断・認定する行為。 当選者の決定など。	▲まずはオシロでタイミングを確認してみて下さい。 ▲予約は確認されています。
\\	学問	がくもん	[名]スル 
\\	学び習うこと。 学校へ通ったり、先生についたり、本を読んだりして、新しい知識を学習すること。 また、身につけた知識。 「―のある人」「―する楽しさ」 
\\	理論に基づいて体系づけられた知識と研究方法の総称。 学。 ◆中世から近世にかけて「学文」と書くのが一般であり、また、「学門」と書くこともあった。 [類語]
\\	学業・勉学・勉強・研鑽(けんさん)・研究・学究・学事・学び(学問から得た知識)学識・学殖・学(がく)・蘊蓄(うんちく)・教養/
\\	学(がく)・学術・学芸・学理・科学・学知	▲書籍が学問に従うべく、学問が書籍に従うべからず。 ▲少しばかりの学問は危険なもの。
\\	隠れる	かくれる	[動ラ下一][文]かく・る[ラ下二] 
\\	物の陰になったり、さえぎられたりして見えなくなる。 「月が雲間に―・れる」 
\\	身を人目につかないようにする。 「物陰に―・れる」「親に―・れてたばこを吸う」 
\\	表面・外部から見えないところに存在する。 ひそむ。 「事件の裏に―・れた謎がある」 
\\	(ふつう、「隠れた」の形で)世間に名前や力が知られないでいる。 また、官職につかないで民間にいる。 「―・れた人材を発掘する」「―・れた功績」 
\\	世をのがれて、ひそむ。 隠遁(いんとん)する。 「山に―・れる」 
\\	高貴な人が死ぬ。 現代ではふつう「お隠れになる」の形を使う。 「後二条関白殿…御年三十八にて遂に―・れさせ給ひぬ」→おかくれ [類語]
\\	紛(まぎ)れる・没する/
\\	潜(ひそ)む・忍(しの)ぶ・伏(ふ)す・潜(もぐ)る・紛(まぎ)れる・紛れ込む・逃げ込む・潜伏(せんぷく)する・隠伏する・韜晦(とうかい)する・身を隠す・身を潜(ひそ)める・人目を盗む/
\\	潜(ひそ)む・伏在する・潜在する	▲雲に隠れて月が見えない。 ▲雲の陰に隠れて月は見えません。
\\	影・景	かげ	《「陰」と同語源》 
\\	日・月・星・灯火などの光。 「月の―」「木陰にまたたく灯火(ともしび)の―」 
\\	光が反射して水や鏡などの表面に映った、物の形や色。 「湖面に雲の―を落とす」 
\\	目に見える物の姿や形。 「どこへ行ったのか子供たちの―も見えない」 
\\	物が光を遮って、光源と反対側にできる、そのものの黒い像。 影法師。 投影。 「夕日に二人の―が長く伸びた」 
\\	心に思い浮かべる、人の顔や姿。 おもかげ。 「かすかに昔日の―を残す」 
\\	ある現象や状態の存在を印象づける感じ。 不吉な兆候。 「忍び寄る死の―」「社会に暗い―を落とす事件」 
\\	心に思い描く実体のないもの。 幻影。 まぼろし。 「そのころの幸福は現在の幸福ではなくて、未来の幸福の―を楽しむ幸福で」 
\\	つきまとって離れないもの。 「寄るべなみ身をこそ遠く隔てつれ心は君が―となりにき」 
\\	やせ細った姿のこと。 「恋すれば我身は―となりにけりさりとて人に添はぬものゆゑ」 
\\	死者の霊魂。 「亡き―やいかが見るらむよそへつつ眺むる月も雲隠れぬる」 
\\	よく似せて作ったもの。 模造品。 「誠の小水竜は、蔵に納め―を作りて持ったる故」 
\\	江戸時代、上方の遊里で揚げ代二匁の下級女郎。 [下接語]朝日影・後ろ影・面影・島影・透(す)き影・月影・鳥影・初日影・春日影・日影・人影・船(ふな)影・火(ほ)影・帆影・星影・御(み)影・水影・物影・山影・夕影・夕日影	▲その少女は自分の影におびえていた。 ▲彼はこのごろ影を潜めてる。
\\	陰・蔭・翳	かげ	《「影」と同語源》 
\\	物に遮られて、日光や風雨の当たらない所。 「木の―で休む」 
\\	物の後ろや裏など、遮られて見えない所。 裏側。 「戸の―に隠れる」「月が雲の―にかくれる」 
\\	その人のいない所。 目の届かない所。 「―で悪口を言う」「―で支える」 
\\	物事の表面にあらわれない部分。 裏面。 背後。 「事件の―に女あり」「―の取引をする」 
\\	(翳)表にはっきり現れない、人の性質や雰囲気の陰気な感じ。 「どことなく―のある人」 
\\	他の助け。 庇護(ひご)。 恩恵。 現代では、ふつう「おかげ」の形で用いる。 「元はといえばかの西内氏のお―である」 
\\	正式なものに対する略式。 「―の祭り」 [下接語]磯(いそ)陰・岩陰・片陰・草陰・草葉の陰・小陰・木(こ)陰・木(こ)の下陰・下陰・島陰・谷陰・軒陰・葉陰・花陰・日陰・目(ま)陰・物陰・森陰・柳陰・藪(やぶ)陰・山陰	▲あの木の陰に腰をおろしましょう。 ▲陰で彼の悪口を言うな。
\\	欠ける・缺ける・闕ける	かける	[動カ下一][文]か・く[カ下二] 
\\	かたい物の一部分が壊れてとれる。 「コップが―・ける」「歯が―・ける」 
\\	そろうべきものの一部分、または必要なものが抜けている。 脱落する。 「メンバーが一人―・ける」「百科事典が一巻―・けている」 
\\	あって当然のものが必要なだけない。 足りない。 不足する。 「千円に一円―・ける」「常識に―・ける」「協調性に―・ける」 
\\	なすべきことをしていない。 おろそかになる。 「礼儀に―・ける」「義理に―・ける」 
\\	(「虧ける」とも書く)満月が次第に円形でなくなる。 「月が―・ける」↔満ちる。	▲奥歯が欠けました。 ▲彼女は美的感覚に欠けている。
\\	加減	かげん	㊀[名]スル 
\\	加えることと減らすこと。 数学で、加法と減法。 
\\	適度に調節すること。 程よくすること。 「暖房を―する」「―して採点する」 
\\	物事の状態・程度。 物のぐあい。 「風の―で波の音が聞こえる」「湯の―をみる」 
\\	からだのぐあい。 健康状態。 「お―はいかが」「―が悪い」 ㊁〔接尾〕動詞の連用形や状態を表す名詞に付く。 
\\	ぐあい・程度の意を表す。 「焼き―」「塩―」 
\\	そのような傾向、そのような気味である、の意を表す。 「うつむき―」 
\\	ちょうどよい程度である、の意を表す。 「ちょうど食べ―のメロン」「入り―の湯」	▲煮え加減がちょうどよい。 ▲母の加減は少しもよくなっていなかった。むしろ悪くなっているようだった。
\\	過去	かこ	
\\	現在より以前の時。 過ぎ去った時。 昔。 「―を振り返る」 
\\	好ましくない経歴・前歴。 「―を清算する」 
\\	仏語。 三世(さんぜ)の一。 この世に生まれる前の世。 前世。 過去世。 
\\	文法で、ある時点(一般には現在)よりも以前の動作・状態を表す言い方。 動詞の連用形に、文語では助動詞「き」「けり」、口語では助動詞「た」などを付けて言い表す。 なお、英語などでは動詞の時制の一とする。	▲彼は過去の苦しい生活を黙想した。 ▲彼は過去の失敗の事をくよくよ思っている。
\\	籠	かご	竹・籐(とう)・柳、または針金などを編んで作った入れ物。 「買い物―」「―の鳥」	▲彼は花でいっぱいの大きな籠を持っています。 ▲彼女はかごの中のぶどうを丹念に選んでいる。
\\	囲む	かこむ	[動マ五(四)]《古くは「かごむ」とも》 
\\	人や物を中にして、その周囲にぐるりと位置する。 また、何かを周囲にぐるりと位置させて、中のものが占め得る場所を限る。 まわりを取り巻く。 「恩師を―・む」「山に―・まれた村里」「記事を罫(けい)で―・む」 
\\	《盤・卓などを囲むところから》囲碁やマージャンなどをする。 「一局―・む」 [可能]かこめる	▲赤丸で囲む。 ▲先生は学生たちに囲まれた。
\\	火災	かさい	火による災難。 火事。 「―に遭う」	▲今週は火災予防週間です。 ▲今朝のテレビで、あなたの地域で大火災があったことを知り、驚いています。
\\	菓子	かし	食事のほかに食べる嗜好品(しこうひん)。 ふつう米・小麦・豆などを主材料とし、砂糖・乳製品・鶏卵・油脂・香料などを加えて作る。 和菓子と洋菓子、また生菓子と干菓子などに分けられる。 古くは果物をさしていい、今も果物を水菓子とよぶ。	▲菓子類は歯に悪いとよく言われる。 ▲子供は一般には菓子が好きだ。
\\	家事	かじ	
\\	家庭内の事情や事柄。 「―の都合により休暇」 
\\	掃除・洗濯・食事の支度・育児など、家庭生活に欠かせない仕事。 「―手伝い」「―に追われる」	▲私達は、家事を分担することで合意した。 ▲家事をしながら働きに出る女性がたくさんいる。
\\	一時	いっとき	
\\	わずかな時間。 しばらくの間。 暫時。 「花の盛りも―」「―小雨になった」 
\\	(「一時に」の形で副詞的に用いて)同時に。 一度に。 「客が―に押しかける」 
\\	過去の、ある時。 あるひととき。 「―は、どうなるかと思った」 
\\	昔の時間の単位で、今の二時間。 ひととき。	
\\	一時	ひととき	
\\	しばらくの間。 いっとき。 「憩いの―」 
\\	過去の、ある時。 いちじ。 いっとき。 「―は栄華を誇った」「―はやった歌」 
\\	昔の時間の単位。 今の約二時間。 いっとき。	▲私達は海岸ですばらしいひとときを過ごした。 ▲素敵な楽しいひと時でした。
\\	一家	いっけ	
\\	一軒の家。 いっか。 
\\	家族全体。 一家族。 いっか。 「―四人のものがふだんのように膳に向かって」 
\\	同じ家系の者全体。 または、それに家来や雇い人をも含めた全体。 同族。 同門。 一族。 一門。 「親しき―の一類はらから集めて」	▲田中さん一家が私を食事に招いてくれた。 ▲男性が一家の主と言うことはアメリカ社会に当てはまる。
\\	得る	うる	[動ア下二]《本来は下二段動詞「う」の連体形》 
\\	「え(得)る」に同じ。 「うるところが多い」「承認をうる」 
\\	動詞の連用形に付いて、…することができる、可能である、の意を表す。 「できうるかぎりの努力」「ストライキは回避しうる」→える ◆ふつう連体修飾語として用いるが、改まった表現や古めかしい表現には終止法としても用いられる。	▲あなたが教えてくれたことからおおいに得るところがあった。 ▲攻撃的な行動に出やすい人は、危険な人間になり得る。
\\	カード	カード	乳汁が酵素や酸の作用で凝固したもの。 牛乳を飲んだあと胃内で、カゼインなどが胃酸によって凝固してできる。 また、チーズやヨーグルト製造時にも生じる。	▲ブライアンがカードを書くのに数時間かかった。 ▲あなたの搭乗カードを見てみましょう。
\\	いけない	いけない	〔連語〕《動詞「い(行)ける」の未然形+打消しの助動詞「ない」》 
\\	「悪い」の遠回しな言い方。 ㋐人のしたことなどに対して非難するさま。 感心できない。 よくない。 「いたずらばかりして、―ない子だ」「定刻に遅れたのが―ない」 ㋑(「…にいけない」の形で)悪い結果を招くさま。 よくない。 「塩分のとりすぎはからだに―ない」 ㋒ぐあいが悪い。 まずい。 「―ない、寝坊しちゃった」 ㋓困ったことだと同情するさま。 「病気だって。 それは―ない」 
\\	よくなる見込みがないさま。 だめだ。 「手を尽くしたが、もう―ないようだ」「ついに店も―なくなる」 
\\	酒が飲めないさま。 「あまり―ない口でして」 
\\	(「…ていけない」などの形で)その点が気に入らない、好ましくないという気持ちを表す。 嫌だ。 「気が散って―ない」「北向きの部屋は寒くて―ない」「いい人だが、おしゃべりで―ない」 
\\	(「…てはいけない」などの形で)その行為や状態が規則などで許されていないことを表す。 「芝生に入っては―ない」「話し方は速すぎても―ない」「お返しの品はあまりりっぱでは―ない」 
\\	(「…といけない」の形で)そういう状態になると困るという気持ちを表す。 「遅刻すると―ないと思って早めに出かける」「腐ると―ないから冷蔵庫に入れておく」「寒いと―ないので厚着をしてきた」 
\\	(「…なければいけない」などの形で)そうする義務や必要があるという気持ちを表す。 「平和憲法はどんなことがあっても守らなければ―ない」「明日までにレポートを提出しなくては―ない」	▲あなたの顔を忘れるといけないので写真を下さい。 ▲あのベンチに座ってはいけません。
\\	何時迄も	いつまでも	[副] 
\\	ある事柄が終わるときの限度がないさま。 末長く。 「―お幸せに」 
\\	どこまでも。 あくまでも。 「―辞退仕りまする」	▲私はいつまでも彼の名を覚えている。 ▲例のスキャンダルはそういつまでも臭いものにフタというわけにはいくまい。いずれ人は嗅ぎつけてしまうさ。
\\	追い付く・追い着く	おいつく	[動カ五(四)] 
\\	追いかけて先に出たものに行き着く。 「先発隊に―・く」 
\\	能力・技術などが目標とするものに達する。 「先進国の技術に―・く」 
\\	(多く「おいつかない」の形で用いる)間に合う。 取り返しがつく。 埋め合わせがつく。 おっつく。 「今さら後悔しても―・かない」	▲彼はすぐにトムに追いつくでしょう。 ▲彼はクラスの他のみんなに追いつくために一生懸命に勉強した。
\\	御喋り	おしゃべり	㊀[名]スル人と雑談すること。 「電話で―する」 ㊁[名・形動]口数の多いこと。 口が軽いこと。 また、そのさまや、その人。 「秘密を守れない―な人」	▲彼女の欠点はおしゃべりをしすぎるところだ。 ▲彼女はおしゃべりが一番好きです。
\\	御昼	おひる	
\\	「昼」の丁寧語。 ㋐正午。 ㋑昼食。 「―の支度をする」 
\\	起きることをいう尊敬語。 お目覚めになること。 「―より先にと急ぎ参りたれば」→御昼(おひる)成る	▲お昼をおごってくれたのを覚えてますか。 ▲母はお昼に私が食べたいものを出してくれた。
\\	御前	おまえ	㊀[名]《「おおまえ(大前)」の音変化で、神仏・貴人の前を敬っていう。 転じて、間接的に人物を表し、貴人の敬称となる》 
\\	神仏・貴人のおん前。 おそば近く。 みまえ。 ごぜん。 「主君の―へ進み出る」 
\\	貴人を間接にさして敬意を表す言い方。 「…のおまえ」の形でも用いる。 「かけまくもかしこき―をはじめ奉りて」「宮の―の、うち笑ませ給へる、いとをかし」 ㊁[代]《古くは目上の人に対して用いたが、近世末期からしだいに同輩以下に用いるようになった》二人称の人代名詞。 
\\	親しい相手に対して、または同輩以下をやや見下して呼ぶ語。 「―とおれの仲じゃあないか」 
\\	近世前期まで男女ともに目上の人に用いた敬称。 あなたさま。 「私がせがれにちゃうど―ほどながござれども」	▲おまえ、変わってるな。
\\	御目出度う	おめでとう	(オメデタクの音便。下の「ございます」「存じます」の略された形。「御目出度」「御芽出度」は当て字)慶事・祝事・新年などを祝う挨拶の言葉。「明けまして―ございます」	▲ご結婚おめでとう。 ▲ご結婚おめでとうございます。(女性)。
\\	賢い・畏い・恐い	かしこい	[形][文]かしこ・し[ク] 
\\	(賢い)頭の働きが鋭く、知能にすぐれている。 利口だ。 賢明だ。 「―・くて聞き分けのいい子供」 
\\	(賢い)抜け目がない。 要領がいい。 「あまり―・いやり方とはいえない」「もっと―・く立ち回れよ」 
\\	恐れ多く、もったいない。 「おことばはまことに―・くて、なんとお答えいたしていいか、とみにことばもいでませぬ」 
\\	神や自然などの超越的なものに対して、畏怖の念を覚えるさま。 恐ろしい。 恐るべきだ。 「わたつみの―・き道を安けくもなく悩み来て」 
\\	尊い。 ありがたい。 「―・き御蔭をば頼みきこえながら」 
\\	すばらしい。 結構だ。 りっぱだ。 「おのが―・きよしなど」 
\\	都合がよい。 運がいい。 幸いだ。 「―・く京の程は雨も降らざりしぞかし」 
\\	(連用形を用いて副詞的に)程度のはなはだしいさま。 非常に。 盛大に。 「いと―・く遊ぶ」 [派生]かしこげ[形動]かしこさ[名] [類語]
\\	聡(さと)い・賢(さか)しい・利口・利発・発明・聡明(そうめい)・怜悧(れいり)・慧敏(けいびん)・明敏・鋭敏・穎悟(えいご)・英明・賢明・頭がいい/
\\	うまい・巧(たく)み・巧妙・クレバー	▲賢い者もあるし、そうでもない者もある。 ▲賢い子供は人生や現実について知りたがる。
\\	歌手	かしゅ	歌をうたうことを職業とする人。 うたいて。 「オペラ―」「流行―」	▲この歌手はあまりにももてはやされている。 ▲これが彼女がポップス歌手として成功した理由です。
\\	数	かず	㊀[名] 
\\	物の順序を示す語。 また、その記号。 数字。 「二けたの―」 
\\	個々の事物が、全体または一定の範囲で、いくつ(何回)あるかということを表すもの。 数量。 「参加者の―を数える」「―多い候補者から選ぶ」「―が合わない」「はしたの―」 
\\	数量や回数が多いこと。 多数。 「―ある作品の中から選ばれる」「―をこなさないと間に合わない」「―で押し切る」「―を頼む」 
\\	価値あるものとして取り立てて認められる範囲。 また、その範囲に入るものとして価値を認められるもの。 「こんな苦労は物の―に入らない」 
\\	同類として数えたてられる範囲。 仲間。 「亡き―に入る」「正選手の―に加える」「子供は―に入らない」 
\\	(多く「の」を伴って)種類などの多いこと。 いろいろ。 「―の仏を見奉りつ」 ㊁〔接頭〕名詞に付いて、粗末な、ありふれた、安価な、などの意を表す。 「―扇」「―雪駄」 [下接語]頭数・稲(いな)数・忌み数・色数・御(お)数・数々・句数・口数・鞍(くら)数・言葉数・字数・品数・手数・亡き数・場数・番数・日数・人数・間(ま)数・物数・物の数・矢数・家(や)数	▲交差点の信号は町の交通事故の数を増えないようにした。 ▲雇用者の数という見地から見ると、すべての工場のうちここは最も大規模なところだ。
\\	稼ぐ	かせぐ	[動ガ五(四)] 
\\	生計を立てるために、一生懸命に働く。 「骨身を惜しまず―・ぐ」 
\\	働いてお金を得る。 「学費を―・ぐ」 
\\	試合などで、得点をあげる。 「打点を―・ぐ」「下位力士を相手に星を―・ぐ」 
\\	(「点をかせぐ」「点数をかせぐ」の形で)自分の立場が有利になるように行動をとる。 評価を高める。 「母の手助けをして点を―・ぐ」 
\\	(「時をかせぐ」「時間をかせぐ」の形で)都合のよい状態になるまで何かをして時間を経過させる。 「支度ができるまで司会者が時間を―・ぐ」 
\\	探し求める。 「繁華の都へ出て奉公を―・ぎ」 [可能]かせげる [類語]
\\	働く・労働する・精を出す・額に汗する/
\\	得る・儲(もう)ける・一稼ぎする・商売する	▲彼は毎月お金をいくら稼ぎますか。 ▲彼は毎月50万円稼ぐ。
\\	数える	かぞえる	[動ア下一][文]かぞ・ふ[ハ下二] 
\\	数量や順番を調べる。 勘定する。 「人数を―・える」「指折り―・える」 
\\	一つ一つ挙げる。 列挙する。 「理由は種々―・えられる」 
\\	数がそれだけのものになる。 「蔵書は五万冊を―・える」 
\\	その中の一つに加える。 数に入れる。 「候補者の一人に―・えられる」 
\\	拍子をとって歌う。 「別れの白拍子をぞ―・へける」 [下接句]死んだ子の年を数える・隣の宝を数える・鼻毛を数える。 星を数うる如(ごと)し	▲うちの息子は百まで数えられる。 ▲ここを出発する前に頭数を数えておきましょう。
\\	肩	かた	
\\	人の腕が胴体に接続する部分の上部、および、そこから首の付け根にかけての部分。 「―をもむ」「―を組む」 
\\	動物の前肢や翼が胴体に接続する部分の上部。 
\\	衣服の、 
\\	に相当する部分。 「―にパッドを入れる」 
\\	物の上部のかどの部分。 「―書き」「各句の―に番号を付ける」 
\\	地形・物の形などの、 
\\	に相当する部分。 「道路の―」「壺の―」 
\\	山頂から少し下った所にある平らな所。 「―の小屋」 
\\	球などを投げる力。 「―が弱い」 
\\	物をかつぐ力。 「足をくじいた友人に―を貸す」 
\\	背負った責任。 「乗客の安全は運転士の―にかかっている」 
\\	《肩に倶生神(くしようじん)が宿っていて、運命を支配するという俗信から》運。 めぐりあわせ。 「此方等のやうな―の悪い夫婦なれば」 [下接語]後(あと)肩・襟肩・五十肩・先肩・四十肩・半肩・一(ひと)肩・路肩(がた)怒り肩・地(じ)肩・撫(な)で肩	▲彼にはあなたがよりかかれる丈夫な肩がある。 ▲髪の毛を肩のところで切りました。
\\	形・型	かた	
\\	(形)物の姿や格好。 形状。 かたち。 「洋服の―が崩れる」「髪の―をととのえる」 
\\	(形)証拠に残すしるし。 保証のしるし。 抵当。 「カメラを借金の―に置く」 
\\	(型)ある物のかたちを作り出すためのもの。 鋳型、型紙などの類。 「石膏(せつこう)で義歯の―を取る」 
\\	(型)芸能や武道などで、規範となる動作・方式。 「能楽の―」「投げの―」 
\\	(型)きまったやり方。 伝統的なしきたり。 慣例。 「―を破る」「―どおりのあいさつ」 
\\	(型)事物を類別するとき、その個々に共通した特徴を表している形式、形態。 タイプ。 「血液の―」「古い―の人間」 
\\	(型)きまった大きさ。 サイズ。 「靴の―が大きすぎる」 
\\	物に似せて作った絵・図・像など。 「馬の―かきたる障子(さうじ)」 
\\	図柄。 模様。 「着る物の―にてばし侍るか」 
\\	もと何かがあったことのしるし。 あとかた。 形跡。 「―もなく荒れたる家の」 
\\	占いに現れたしるし。 うらかた。 「生ふ?(しもと)この本山のましばにも告(の)らぬ妹が名―に出でむかも」 
\\	(「がた」の形で) ㋐(形)名詞の下に付いて、その物に似たかたちをしていることを表す。 「卵―」「ピラミッド―」 ㋑(型)名詞と形容詞の語幹の下に付いて、ある性質・特徴・形式をもっていることを表す。 「最新―」「ハムレット―」「うるさ―」→形(かたち)[用法] [類語]
\\	形(かたち)・形状・外形・格好/
\\	抵当・担保・引き当て・質(しち)/
\\	定式・形式・定型・定法(じようほう)・定例(じようれい)・通例・常道・作法・定石・パターン/
\\	類型・型式(けいしき・かたしき)・様式・タイプ・モデル・パターン	▲この型のセーターはすべて今在庫切れです。 ▲あなたのスキーを逆V字型にしなさい。
\\	方方	かたがた	
\\	⦅名⦆ 
\\	「人々」の敬意を含んだ言い方。 「こちらの―にはこれを差し上げます」 
\\	あちらこちら。 ほうぼう。 
\\	⦅代⦆ (二人称複数)おのおのがた。 あなたがた。	▲わからなかったら方々の人に聞いた方がいいよ。 ▲私どもの新製品は、耳にされた全ての方々から、たいへん注目されています。
\\	刀	かたな	《「かた」は片、「な」は刃の古語》 
\\	武器として使った片刃の刃物。 
\\	江戸時代、武士が脇差(わきざし)とともに差した大刀。 
\\	太刀の小さいもの。 「我は元より太刀も―も持たず」 
\\	小さい刃物。 きれもの。 「紙をあまた押し重ねて、いと鈍き―して切るさまは」 [下接語](がたな)菖蒲(あやめ)刀・一本刀・打ち刀・押っ取り刀・返し刀・小刀・腰刀・提げ刀・錆(さび)刀・反り刀・竹刀・小さ刀・血刀・手刀・鈍(なまくら)刀・腹切り刀・懐(ふところ)刀・包丁刀・枕(まくら)刀・守り刀・山刀	▲「刀の投擲の練習?」「すっぽ抜けただけです」。 ▲彼は鋼を鍛えて刀を作った。
\\	語る	かたる	[動ラ五(四)] 
\\	話す。 特に、まとまった内容を順序だてて話して聞かせる。 「目撃者の―・るところによれば」「決意の程を―・る」 
\\	語り物を節をつけて朗読する。 「浪曲を―・る」 
\\	ある事実がある意味・真実・事情などをおのずから示す。 物語る。 「この惨状が台風のすさまじさを―・っている」 
\\	親しくまじわる。 「其の里の人、年ごろ別して―・り、殊更(ことさら)内縁のよしみなりけるが」 [可能]かたれる	▲アメリカのニュースでは、だれが話題になっているときでも、いつも「日本人全体」として語られる。 ▲あらゆる真実がいつでも語られるとは限らない。
\\	価値	かち	
\\	その事物がどのくらい役に立つかの度合い。 値打ち。 「読む―のある本」「―のある一勝」 
\\	経済学で、商品が持つ交換価値の本質とされるもの。 →価値学説 
\\	哲学で、あらゆる個人・社会を通じて常に承認されるべき絶対性をもった性質。 真・善・美など。 [類語]
\\	値打ち・価(あたい)・意義・真価(しんか)・有用性・バリュー・メリット	▲もちろん万能ではないからといって価値がないわけではないから、GTDはダメということにはならない。 ▲精液は瓶詰めにする価値はあるよ。
\\	勝ち	かち	勝つこと。 勝利。 「―を得る」↔負け。	▲負けるが勝ち。 ▲表が出たら僕の勝ち、裏が出たら君の勝ち。
\\	がっかり	がっかり	[副]スル 
\\	望みがなくなったり、当てが外れたりして、気力をなくすさま。 「遠足が中止になって―する」 
\\	ひどく疲労するさま。 がっくり。 「一日起ち続けで日が暮れると―して座睡(いねむり)が出てくる」	▲私は彼の死を聞いてがっかりした。 ▲私は息子にはがっかりしています。
\\	活気	かっき	生き生きとした気分。 生気。 「―のある教室」「―にあふれる」	▲活気が出るからいいことだと思うよ。 ▲活気も変化もない家庭内のきまりきった生活に少年は退屈した。
\\	学期	がっき	学校で、一学年を区分した一定の期間。 小学校・中学・高校では一般に三学期制、大学では二学期制が行われている。 「―末試験」	▲君の成績は今学期は平均よりだいぶ下だった。 ▲今学期、2つの試験がある。
\\	活動	かつどう	[名]スル 
\\	活発に動くこと。 ある動きや働きをすること。 「暖かくなって虫が―し始めた」「―範囲が広い」「火山―」 
\\	「活動写真」の略。映画の旧称。主に無声映画の時代に使われた。 「ちょいと―でも見るつもりが」 [類語]
\\	動き・運動・行動・生動・蠢動(しゆんどう)・躍動・活躍・奔走 (―する)動く・動き回る	▲我が社の販売活動は大いに成功しています。 ▲我々の組合では一握りの活動家だけがうるさく言っている。
\\	活用	かつよう	[名]スル 
\\	物や人の機能・能力を十分に生かして用いること。 効果的に利用すること。 「学んだ知識を―する」「資料を―する」 
\\	文法で、語がその用法の違いによって体系的に語形変化をすること。 また、その変化の体系。 日本語では用言(動詞・形容詞・形容動詞)および助動詞に活用がある。	▲そのような時は、財政政策の抑制が活用されなければならない。 ▲せいぜい自分の能力を活用しなさい。
\\	悲しむ・哀しむ・愛しむ	かなしむ	[動マ五(四)] 
\\	心が痛む思いだ。 悲しく思う。 また、なげかわしく思う。 「別れを―・む」「道徳心の低下を―・む」↔喜ぶ。 
\\	(愛しむ)いとしいと思う。 愛する。 「端正(たんじやう)美麗なる男子を産めば、父母これを―・み愛して」 
\\	(愛しむ)深く感動する。 「国王、これを見給ひて、―・み貴びて」 
\\	嘆願する。 「手をすり―・めども」 [可能]かなしめる	▲なんでそんなに悲しむの。 ▲もし彼が死んだらさぞ悲しむだろう。
\\	必ずしも	かならずしも	[副]《「し」は副助詞、「も」は係助詞》打消しの語を伴って、必ず…というわけではない、…とは限らない、という気持ちを表す。 「金持ちが―幸福ではない」「―そうとはいえない」	▲近頃では、結婚の動機は必ずしも純粋とは限らない。 ▲近代戦の戦術は兵士が戦闘員として効果的な働きをするために必ずしも十分に武装することを必要とはしていない。
\\	可成り・可也	かなり	《許し認める意の「可なり」から》 ㊀[形動][文][ナリ]相当の程度まで行っているさま。 また、相当の程度以上に達しているさま。 「―な収入がある」「―な数にのぼる」 ㊁[副]極端ではないが、並の程度を超えているさま。 思ったより以上に。 相当。 「今日は―人が出ている」「―酔っているようだ」 [用法]かなり・だいぶ――「台風の影響で海はかなり(だいぶ)荒れている」「病気はかなり(だいぶ)重い」などでは相通じて用いられ、物事の程度が「非常に」「とても」「大変」ほどではないが、平均以上であることを表す。 
\\	「かなり」は「かなりな金額」「かなり出来る」のように、平均を超えたある程度を示すが、「だいぶ」は「夜になって、気温がだいぶ下がってきた」「だいぶ片付いた」のように、進行している事柄でさらに程度が進む可能性のある場合にぴったりする。 
\\	類似の語に「相当」がある。 「相当」は「かなり」「だいぶ」と同じように用いるが、より程度が上回っている感じがあり、「相当金のかかる大学だ」「相当な切手マニアだ」などでは、驚きあきれる感情がこもっている。 
\\	また、類似の語「随分」の用法には「かなり」「だいぶ」と相通じるものがある。	▲果実がなる樹木は、成長するための空間がかなり必要だ。 ▲火事は鎮火するまでかなりの間燃え続けていた。
\\	金	かね	
\\	金属の総称。 特に、金・銀・鉄・銅など。 
\\	貨幣。 金銭。 おかね。 「―に困る」「―がかかる」「裏で―が動く」「―がたまる」 [下接語]唐金・切り金・銭金(がね)遊び金・粗金・有り金・生き金・板金・打ち金・腕金・裏金・大金・帯金・下ろし金・隠し金・掛け金・烏(からす)金・腐れ金・口金・小金・黄(こ)金・座金・差し金・地金・下金・死に金・締め金・筋金・捨て金・包み金・壺(つぼ)金・胴金・綴(と)じ金・留め金・偽(にせ)金・延べ金・端(はし)金・端(はした)金・針金・火打ち金・日金・引き金・肘(ひじ)金・日済(ひな)し金・臍繰(へそく)り金・真金・見せ金・耳金・無駄金・目腐れ金・持ち金・焼き金・渡し金	▲二人の反目の原因は金だ。 ▲小切手は人にお金を払う一つの方法です。
\\	可能	かのう	[名・形動]《「能(あた)う可(べ)き」の音読》 
\\	ある物事ができる見込みがあること。 ありうること。 また、そのさま。 「現在―な方法は限られている」「実現―な(の)計画」 
\\	文法で、そうすることができるということを表す言い方。 動詞の未然形に、文語では助動詞「る」「らる」(古くは「ゆ」「らゆ」)、口語では助動詞「れる」「られる」などを付けて言い表す。	▲男の子にしつけは可能である。 ▲玄関にも取り付け可能な暗証タイプです。
\\	株	かぶ	㊀[名] 
\\	切り倒した木や、刈り取った稲などの、あとに残った根元の部分。 切り株や刈り株。 くいぜ。 
\\	草木の、何本にも分かれた根元。 「菊の―を分ける」 
\\	株式。 株券。 
\\	㋐特定の身分・地位または職業上・営業上の権利・資格・格式。 「相撲の年寄―」「このまま家(=芸者屋)の―をそっくり譲ってやりたいと」 ㋑江戸時代、株仲間の一員として持つ特権。 また、御家人(ごけにん)・名主(なぬし)などの身分・地位を世襲・継続する特権。 売買の対象ともなった。 
\\	その仲間・社会で評価を得ていること。 また、その評価。 「日本の―が上がる」 
\\	その人特有の癖。 得意なわざ。 現代では「おかぶ」の形で用いる。 「このばあさまは…泣きごとばかりいふが―なり」→御株(おかぶ) ㊁〔接尾〕 
\\	助数詞。 ㋐根のついた草木を数えるのに用いる。 「カンナ三―」 ㋑株式・株券を数えるのに用いる。 
\\	名詞に付いて、その地位・資格を持つ者の意を表す。 「兄貴―」「番頭―」	▲株に手を出すなんて彼女はなんと愚かなのだろう。 ▲彼は前回の好景気の間に財産を株取り引きのあてた。
\\	我慢	がまん	[名・形動]スル 
\\	耐え忍ぶこと。 こらえること。 辛抱。 「彼の仕打ちには―がならない」「ここが―のしどころだ」「痛みを―する」 
\\	我意を張ること。 また、そのさま。 強情。 「―な彼は…外(うわべ)では強いて勝手にしろという風を装った」 
\\	仏語。 我に執着し、我をよりどころとする心から、自分を偉いと思っておごり、他を侮ること。 高慢。 「汝仏性を見んとおもはば、先づすべからく―を除くべし」 [用法]我慢・辛抱――「文句ばかり言ってないで、もっと我慢(辛抱)することを覚えなさい」のように、一般的に、こらえるの意味では相通じて用いられる。 
\\	「仕事はきつかったが、どうにか我慢(辛抱)した」「食糧不足によるひもじさを我慢(辛抱)した」など、苦しさ・つらさ・痛さ・寒さ・くやしさなどをこらえる場合、「我慢」「辛抱」の用法にほとんど違いはない。 
\\	「我慢」は、そのほかに「笑いだしたいのを我慢する」など、より幅広く使える。 「辛抱のよい人」は「辛抱」独特の使い方である。 
\\	類似の語に「忍耐」がある。 「忍耐」は文章語的で、「忍耐を要する仕事」「忍耐力に富む」などのように用いられる。	▲私はその騒音にはもはやがまんできない。 ▲私はその寒さにがまんできない。
\\	神	かみ	
\\	信仰の対象として尊崇・畏怖(いふ)されるもの。 人知を超越した絶対的能力をもち、人間に禍福や賞罰を与える存在。 キリスト教やイスラム教では、宇宙・万物の創造主であり、唯一にして絶対的存在。 「―を信じる」「合格を―に祈る」「―のみぞ真実を知る」 
\\	神話や伝説に人格化されて登場する語りつがれる存在。 「火の―」「縁結びの―」 
\\	㋐偉大な存在である天皇をたとえていう語。 また、天皇の尊称。 「現人―(あらひとがみ)」「大君は―にしませば赤駒の腹ばふ田居を都となしつ」 ㋑神社にまつられる死者の霊魂。 
\\	助けられたり、恩恵を受けたりする、非常にありがたい人やもの。 「救いの―が現れた」 
\\	㋐人間に危害を加える恐ろしいもの。 「虎(とら)といふ―を生け取りに」 ㋑雷。 なるかみ。 「―は落ちかかるやうにひらめく」 [下接語]天つ神・出雲(いずも)の神・縁結びの神・大神・大御(おおみ)神・風の神・竈(かま)の神・河の神・国魂(くにたま)の神・国つ神・塞(さい)の神・救いの神・皇(すめ)神・田の神・鳴る神・火の神・福の神・禍(まが)神・道の神・結びの神・産霊(むすび)の神・八百万(やおよろず)の神・山の神(がみ)商い神・現人(あらひと)神・生き神・軍(いくさ)神・石神・市神・犬神・氏神・産(うぶ)神・産土(うぶすな)神・枝神・男(お)神・臆病(おくびよう)神・おなり神・竈(かまど)神・猿神・地神・式神・死に神・鎮守神・年神・天一(なか)神・霹靂(はたた)神・ひだる神・貧乏神・蛇神・箒(ほうき)神・枕(まくら)神・守り神・迷わし神・女(め)神・巡り神・疫病神・留守神	▲神の存在を信じる事それ自体は悪ではない。 ▲神の存在を信じる人もいれば、信じない人もいる。
\\	上	かみ	
\\	ひと続きのものの初め。 また、いくつかに区分したものの初め。 ㋐川の上流。 また、その流域。 川上。 「―へ船で上る」「川沿いを―に一キロほど行く」↔下(しも)。 ㋑時間的に初めと考えられるほう。 昔。 いにしえ。 「―は奈良時代から下(しも)は今日まで」↔下(しも)。 ㋒ある期間を二つに分けた場合の前のほう。 「―の半期」↔下(しも)。 ㋓月の上旬。 「寄席の―席を聴きに行く」 ㋔物事の初めの部分。 前の部分。 「―に申したごとく」「―二桁(けた)の数字」「―の巻」↔下(しも)。 ㋕和歌の前半の三句。 「―の句」↔下(しも)。 
\\	位置の高い所。 ㋐上方に位置する所。 上部。 「山の―にある村」「几帳(きちやう)の―よりさしのぞかせ給へり」↔下(しも)。 ㋑からだの腰から上の部分。 「―半身」↔下(しも)。 ㋒上位の座席。 上座。 上席。 「主賓が―に座る」↔下(しも)。 ㋓台所などに対して、客間・座敷や奥向きをさす語。 ↔下(しも)。 ㋔舞台の、客席から見て右のほう。 上手(かみて)。 「主役が―から登場する」↔下(しも)。 
\\	地位・身分の高い人。 ㋐天皇の敬称。 陛下。 「―御一人」 ㋑高位・上位にある人。 「―は皇帝から下(しも)は庶民に至るまで」↔下(しも)。 ㋒朝廷・政府・官庁などの機関。 また、為政者。 「お―からのお達し」→御上(おかみ) ㋓他人の妻、また、料理屋の女主人などを軽い敬意を含んでいう語。 「隣家のお―さん」「料亭のお―」→御上(おかみ) 
\\	皇居のある地。 ㋐都。 京都。 また、その周辺。 「―へのぼる」「―方(かみがた)」 ㋑京都で、御所のある北の方角・地域。 転じて一般に、北の方の意で地名などに用いる。 「河原町通りを―へ向かう」「―京(かみぎよう)」「―井草(かみいぐさ)」↔下(しも)。 ㋒他の地域で、より京都に近いほう。 昔の国名などで、ある国を二分したとき、都から見て近いほう。 「―諏訪(かみすわ)」「―つけ(=上野(こうずけ))」↔下(しも)。 
\\	格や価値が優れているほう。 「人丸は赤人が―に立たむこと難(かた)く」 
\\	㋐年長の人。 「七つより―のは、みな殿上せさせ給ふ」 ㋑主人。 かしら。 「―へ申しませう」	
\\	雷	かみなり	《「神鳴り」の意》 
\\	電気を帯びた雲と雲との間、あるいは雲と地表との間に起こる放電現象。 電光が見え、雷鳴が聞こえる。 一般に強い風と雨を伴う。 いかずち。 なるかみ。 「―が鳴る」「―に打たれる」《季 夏》「―に小屋は焼かれて瓜の花/蕪村」 
\\	雲の上にいて、雷を起こすという神。 鬼の姿をしていて、虎の皮の褌(ふんどし)を締め、太鼓を背負って、これを打ち鳴らし、また、人間のへそを好むとされる。 雷神。 はたた神。 かみなりさま。 
\\	頭ごなしにどなりつけること。 腹を立ててがみがみと叱責(しつせき)すること。 「―を落とす」 [類語]
\\	雷(いかずち)・鳴る神・雷(らい)・雷鳴・雷電・天雷・急雷・疾雷(しつらい)・迅雷(じんらい)・霹靂(へきれき)・雷公・遠雷・春雷・界雷・熱雷・落雷・稲妻(いなずま)・稲光(いなびかり)・電光・紫電(しでん)	▲さらに悪いことに、激しく雷が鳴り始めた。 ▲その日は雨が降っており、さらに悪いことには、雷も鳴っていた。
\\	髪の毛	かみのけ	頭の毛。 頭髪。 かみ。	▲君は髪の毛を刈ってもらったほうがいい。 ▲光ファイバーケーブルは人間の髪の毛ほどの細さの小さなガラスでできている。
\\	科目	かもく	
\\	いくつかに分けたそれぞれの項目。 「予算―」 
\\	学問の区分。 特に、学校で教科を分野別に分類したもの。 課目。 「選択―」 
\\	古く、中国での官吏登用試験の類別。 →科挙	▲どの科目が一番好きですか。 ▲普通科目の他に機械についての基礎的な事柄を学習し、 各種の機械の使用法や技術を身につける実習などを行います。
\\	かも知れない	かもしれない	〔連語〕《「か」は副助詞、「も」は係助詞》断定はできないが、その可能性があることを表す。 「あの建物は学校―ない」「君の言うとおりなの―ない」→かも〔連語〕	▲これは、結婚に対する人々の見方が変わったことと、1日24時間あいているファーストフード・ストアやコンビニエンス・ストアが急速に増加して、若い人たちがもっと気楽に暮らせるようになったためかもしれない。 ▲これはあなたの口に合わないかもしれない。
\\	火曜	かよう	週の第三日。 月曜の次の日。 火曜日。	▲火曜の朝までにすべて終えなければならないのです。 ▲来週の月曜と火曜は連休だ。
\\	空・虚	から	《「殻」と同語源》 ㊀[名]内部に物のないこと。 からっぽ。 「―の箱」「家を―にする」 ㊁〔接頭〕名詞に付く。 
\\	何も持たないこと。 何も伴っていないこと。 「―馬」「―手」 
\\	実質的なものが伴わないこと。 うわべや形だけで役に立っていないこと。 「―元気」「―回り」「―手形」「―世辞」	▲財布をあらためて見たら中は空だった。 ▲私はその箱が空だと分かった。
\\	柄	がら	㊀[名] 
\\	からだつき。 体格。 「―が大きい」 
\\	その人に本来そなわっている品位・性格。 「人のことを言える―ではない」「―が悪い」「―に合わない」 
\\	布・織物などの模様。 「はでな―」 ㊁〔接尾〕名詞に付く。 
\\	そのものの品位・性質の意を表す。 「土地―がうかがわれる」「家―」「作―」 
\\	それに相応して、の意を表す。 「時節―御自愛ください」「仕事―こういうことは詳しい」 [下接語]間柄・家柄・歌柄・大柄・男柄・女柄・木柄・句柄・国柄・声柄・子柄・小柄・心柄・骨柄・事柄・作柄・品柄・縞(しま)柄・新柄・図柄・総柄・染め柄・珍柄・続き柄・角柄・手柄・所柄・花柄・日柄・人柄・身柄・銘柄・紋柄・矢柄・役柄・訳(わけ)柄	▲われわれはそんなりっぱな家に住む柄ではない。 ▲この赤色で柄全体がだいなしだ。
\\	刈る・苅る	かる	[動ラ五(四)] 
\\	伸び茂っているものを根元を残して切り払ったり、切り取ったりする。 「草を―・る」「髪を―・る」 
\\	演劇で、俳優や時間その他の都合により、上演中の脚本の一部分を省略して演じる。 [可能]かれる	▲ある暑い夏の午後、ジョンとダンヌは長くなった牧草を刈っていました。 ▲この芝は刈らなければならない。
\\	皮	かわ	
\\	動植物の肉・身を包んでいる外側の膜。 表皮。 「みかんの―をむく」「鮫(さめ)の―」「魚の―をはぐ」 
\\	物の表面にあって、中身を覆ったり包んだりしているもの。 「饅頭(まんじゆう)の―」 
\\	物事の表面にあって、本質を覆っているもの。 「欲の―」「化けの―をはがす」 [下接語]厚皮・甘皮・粗皮・薄皮・嘘(うそ)の皮・姥(うば)皮・上(うわ)皮・鬼皮・帯皮・辛皮・唐皮・栗(くり)皮・黒皮・渋皮・尻(しり)皮・白皮・杉皮・竹の皮・爪(つま)皮・面(つら)の皮・生皮・化けの皮・撥(ばち)皮・腹の皮・一皮・糸瓜(へちま)の皮・松皮・的皮・身の皮・桃皮 (がわ)裏皮・黄皮・毛皮・鮫(さめ)皮・鹿(しか)皮・敷皮・鰐(わに)皮	▲あのサッカーボールは本物の皮でできている。 ▲皮は乾くにつれて堅くなった。
\\	革	かわ	《「皮」と同語源》獣類の表皮の毛を取り去り、なめしたもの。	▲あれは革のベルトです。 ▲その靴は革に似た何か柔らかい材料で出来ていた。
\\	可哀相・可哀想	かわいそう	[形動][文][ナリ]同情の気持ちが起こるさま。 ふびんに思えるさま。 「―な境遇」「彼ばかり責めては―だ」「お―に」 ◆「可哀相」「可哀想」は当て字。	▲かわいそうにその少女は死にかかっていた。 ▲かわいそうにその少年はどうしたらよいのか途方にくれた。
\\	可愛らしい	かわいらしい	[形][文]かはいら・し[シク]《「かわゆらしい」の音変化》 
\\	子供らしい無邪気さや見た目の好ましさで、人をほほえませるようなさま。 「―・い口もと」 
\\	小さくて愛らしい。 「指先ほどの―・い魚」 [派生]かわいらしげ[形動]かわいらしさ[名]	▲彼女は、ほんとうにかわいらしい少女だ。 ▲彼は大変かわいらしい女性と結婚した。
\\	管	かん	㊀[名]内部がからで筒状のもの。 くだ。 「ガスの―」 ㊁〔接尾〕助数詞。 笛・筆など、くだ状の物を数えるのに用いる。 「一―(いつかん)の笛」	▲管から水が吹き出した。
\\	缶・罐・鑵	かん	《英
\\	または、(オランダ)
\\	から、「缶」「罐」「鑵」は当て字》 
\\	金属の薄い板で作った容器。 特に、ブリキ製のものをいう。 「石油―」 
\\	「缶詰」の略。 「鮭(さけ)―」	▲道に缶をすてるな。 ▲彼は道で空き缶をひろった。
\\	勘	かん	
\\	物事の意味やよしあしを直感的に感じとり、判断する能力。 「―が働く」 
\\	古文書で、内容の了解を示す符号や点。	▲その新聞記者はニュースを嗅ぎつける鋭い勘を持っている。
\\	館	かん	
\\	大きな建物。 「万博のアメリカ―」 
\\	図書館・博物館など、「館」と名のつく建物・施設。 「―の運営」「―の財産」	▲タッソーろう人形館へ行く道を教えて頂けませんか。 ▲この方が、「真砂館」のおかみさんの染葉洋子さん。
\\	巻	かん	㊀[名] 
\\	巻物。 巻物にした書物。 巻子本(かんすぼん)。 「―を開く」 
\\	書物。 書籍。 
\\	何冊か合わせてひとまとまりとなる書籍の、その一つ一つ。 ㊁〔接尾〕助数詞。 
\\	書籍の冊数をかぞえるのに用いる。 「全三―の書物」 
\\	巻物やテープ、フィルムなどの数をかぞえるのに用いる。 「巻物三―」「フィルム五―」	▲この全集の5巻目が見当たらない。 ▲その辞書は全2巻です。
\\	感	かん	
\\	深く心が動くこと。 感動。 「―に入る」 
\\	物事に接して生ずる心の動き。 感じ。 「今さらの―は否めない」「隔世の―」	▲デザインも、アーチ型のロゴデザインにより現代的で「登場感」「躍動感」あるものに仕上げました。 ▲暴力団の頭目は団員全部に規則を守らせ、恐怖感によって、団員を掌握した。
\\	間	かん	㊀[名] 
\\	物と物、場所と場所とを隔てる空間的な広がり。 また、その距離。 「天地の―」「その―約八キロ」「目睫(もくしよう)の―に迫る」 
\\	ある時点とある時点とのあいだ。 あるひと続きの時間。 「その―の事情はわからない」「ボールが外野を転々とする―に」 
\\	すきま。 間隙(かんげき)。 「多忙の―を縫って出席する」「―に乗じる」 
\\	心の隔たり。 「―を生じる」 ㊁〔接尾〕名詞に付いて、ある物事・時間・場所と他の物事・時間・場所とのあいだ、人と人との関係などの意を表す。 「五日―」「東京、大阪―」「学校―の連絡」「夫婦―のもめごと」	▲お支払いの期日を二ヶ月間延長していただけませんでしょうか。 ▲この列車はニューヨークとボストン間を走っている。
\\	考え	かんがえ	考えること。 また、考えて得た結論・判断・予測・決意など。 「―をまとめる」「―を示す」「甘い―」「そういうことならこちらにも―がある」 [類語]思考・思案・思索・考察・考慮・思慮・念慮・念頭・想念・想(そう)・構想・案・意向・意図・意思・意・意見・見解・所見・所懐・所思・所存・存念・存意・料簡(りようけん)・思惑(おもわく)・魂胆・腹(はら)・腹づもり(尊敬)貴慮・尊慮・賢慮・御意(ぎよい)・貴意・尊意・思(おぼ)し召し(謙譲)愚考・愚案	▲老人が考えを変えるのは難しい。 ▲御社のお考えを先におっしゃってください。
\\	感覚	かんかく	
\\	外界からの光・音・におい・味・寒温・触などの刺激を感じる働きと、それによって起こる意識。 視覚・聴覚・嗅覚(きゆうかく)・味覚・触覚や、温覚・冷覚・痛覚など。 「寒さで指の―がなくなる」 
\\	美醜やよしあし、相違などを感じとる心の働き。 センス。 感受性。 「日本人の―では理解しにくい」「―が鋭い」「新―のデザイン」 [類語]
\\	知覚・官能・五感・体感・肉感・感触・感じ/
\\	感性・感受性・美感・美意識・神経・センス・センシビリティー・フィーリング	▲暗い森の中で彼は方向感覚を失った。 ▲鞍馬は、力よりもバランス感覚が必要です。
\\	観客	かんきゃく	映画・演劇・スポーツなどの見物人。 かんかく。 「―席」	▲予想以上に多くの観客が来ていました。 ▲野茂が打者をおさえると観客がわきあがります。
\\	環境	かんきょう	まわりを取り巻く周囲の状態や世界。 人間あるいは生物を取り囲み、相互に関係し合って直接・間接に影響を与える外界。	▲私たちに環境を守るために私たちができることはいくつもある。 ▲私たちの学校は健全な環境に囲まれている。
\\	歓迎	かんげい	[名]スル喜んでむかえること。 喜んで受け入れること。 「ご来場を―する」「建設的な批判は―する」「―会」	▲いつおいでくださっても歓迎いたします。 ▲いつ来ても歓迎します。
\\	観光	かんこう	[名]スル他の国や地方の風景・史跡・風物などを見物すること。 「各地を―してまわる」「―シーズン」「―名所」	▲この町を観光できるツアーがありますか。 ▲ところで、そこに滞在中、観光に行く暇はありましたか。
\\	観察	かんさつ	[名]スル 
\\	物事の状態や変化を客観的に注意深く見ること。 「動物の生態を―する」「―力」 
\\	《「かんざつ」とも》仏語。 智慧によって対象を正しく見極めること。	▲彼は人を観察するのが趣味だ。 ▲彼は鏡をとって舌をよく観察した。
\\	感じ	かんじ	
\\	感覚器官に受ける刺激によって生じる反応。 感覚。 「指先の―がなくなる」「舌をさすような―がある」 
\\	物事を見聞したり、人に接したりしたときに受ける気持ち。 印象や感想。 「―のいい人」「春らしい―の日ざし」 
\\	刺激に対する反応。 「―が鈍い」 
\\	その物事に特有の雰囲気。 「母親役らしい―が出る」「ピッチングの―をつかむ」	▲肌にシェーバーを滑らす度に確かに滑らか〜な感じがします。 ▲「順路→」といったかんじの看板を設置したいと思うんですけど、これを英語で作るとどうなるでしょうか?
\\	感謝	かんしゃ	[名]スルありがたいと思う気持ちを表すこと。 また、その気持ち。 「―の心」「深く―する」	▲ご援助いただき、あなたに感謝しています。 ▲ご援助を感謝します。
\\	患者	かんじゃ	病気やけがの治療を受ける人。 医師の側からいう語。 クランケ。 「入院―」	▲このタイプのセラピーを開始する前に患者自信の希望を注意深く考慮に入れなければならない。 ▲この患者がもう一度良くなるまでにはだいぶん時間がかかるだろう。
\\	感情	かんじょう	物事に感じて起こる気持ち。 外界の刺激の感覚や観念によって引き起こされる、ある対象に対する態度や価値づけ。 快・不快、好き・嫌い、恐怖、怒りなど。 「―をむきだしにする」「―に訴える」「―を抑える」「国民―を刺激する」 [類語]情(じよう)・情感・心情・情緒(じようしよ・じようちよ)・情調・情操・情念・情動・喜怒哀楽・気分・気(き)・気色(きしよく)・機嫌(きげん)・気持ち・感じ・エモーション	▲利己的な人は自分の感情しか考えない。 ▲必要な場合があれば彼は自分の感情をかくすことができる。
\\	勘定	かんじょう	[名]スル 
\\	物の数量、または金銭を数えること。 「売上金の―が合わない」「人員を―する」 
\\	代金を支払うこと。 また、その代金。 「―を済まして店を出る」 
\\	他から受ける作用や、先々生じるかもしれない事態などを、あらかじめ見積もっておくこと。 「列車の待ち時間を―に入れて行動する」 
\\	いろいろ考え合わせて出た結論。 「うまくいけばみんなが得をする―だ」 
\\	簿記で、資産・負債・資本の増減、収益・費用の発生を記録・計算するために設ける形式。 [類語]
\\	計算・計数・算用・カウント/
\\	代金・お代・払い・支払い・会計・お愛想(あいそ)・精算・清算・決済/
\\	予測・予想・予定・目算・計算・考慮	▲勘定をお願いします。 ▲勘定を頼むよ。
\\	感じる	かんじる	[動ザ上一]「かん(感)ずる」(サ変)の上一段化。 「寒さを―・じる」	▲彼は冒険の誘惑を感じた。 ▲彼は背中に痛みを感じた。
\\	関心	かんしん	ある物事に特に心を引かれ、注意を向けること。 「政治に―がある」「幼児教育に―が高まる」「周囲の―の的」	▲私は歴史に関心を持っています。 ▲私は歴史に関心がある。
\\	感心	かんしん	㊀[名]スル  
\\	りっぱな行為や、すぐれた技量に心を動かされること。 心に深く感じること。 感服。 「うまいことを言うものだと―する」「あまり―できないやり方だ」 
\\	(逆説的に用いて)あきれること。 びっくりすること。 「ばかさかげんに―する」 ㊁[形動][文][ナリ]りっぱであるとして褒められるべきさま。 「親思いの―な少年」	▲彼らはその丘から素晴らしい眺めに感心した。 ▲彼は服装がきちんとしているので、私はいつも感心していた。
\\	関する	かんする	[動サ変][文]くゎん・す[サ変]関係がある。 かかわる。 「将来に―・する問題」「映画に―・しては、ちょっとうるさい」「我―・せず」	▲君の関する事柄ではない。 ▲これに対して霊的なことがらに関する熱心な会話は、霊的な進歩に大いなる助けとなります。
\\	完成	かんせい	[名]スル完全に出来上がること。 すっかり仕上げること。 「―を見る」「ビルが―する」「大作を―する」	▲それを完成するのに休日のほとんどを費やしてしまった。 ▲外国、特に西洋の諸国では、学生は個人として自己を表明したり完成させるようにし向けられる。
\\	完全	かんぜん	[名・形動]スル 
\\	欠けたところや足りないところがまったくないこと。 必要な条件がすべてそろっていること。 また、そのさま。 「―を期す」「―な形で保存する」「―に失敗だ」 
\\	欠点などのないようにすること。 「その人と為(なり)を―するに於て」 [類語]
\\	完璧(かんぺき)・万全・十全・両全・満点・金甌(きんおう)無欠・百パーセント・パーフェクト(形動用法で)全(まつた)い・文句なし・間然(かんぜん)する所がない	▲宿題を全部やってしまってから、月曜まで完全に自由だ。 ▲宿題は完全に終えたのですか。
\\	感動	かんどう	[名]スルある物事に深い感銘を受けて強く心を動かされること。 「深い―を覚える」「名曲に―する」	▲彼の涙に感動した。 ▲彼の雄弁は素晴らしいものだったので、誰もが感動して涙を流した。
\\	監督	かんとく	[名]スル 
\\	取り締まったり、指図をしたりすること。 また、その人や機関。 「工事現場を―する」「試験―」 
\\	映画・舞台・スポーツ競技などで、グループやチームをまとめ、指揮・指導する役の人。 「撮影―」 
\\	日本の聖公会やメソジスト教会における第二次大戦前の職制名。 主教、司教にあたる。 
\\	法律で、人または機関の行為が、その守るべき義務に違反していないか、その目的達成のために適当か否かを監視し、必要なときには指示・命令などを出すこと。	▲代表取締役は取締役の職務の執行を監督するものだ。 ▲彼は監督に挨拶さえしなかった。
\\	冠	かんむり	《「こうぶり」の音変化》 
\\	頭にかぶるもの。 特に、許されて直衣(のうし)を着て参内する束帯・衣冠などのときにかぶるもの。 黒の羅(うすもの)で作る。 頂にあたる所を甲(こう)、前額部を額(ひたい)という。 後方の高い壺は髻(もとどり)を入れる巾子(こじ)で、その後ろに長方形の纓(えい)(俗に燕尾(えんび)という)二枚を重ねて垂れる。 有文(うもん)と無文の冠の区別があり、時代によって形式の変化がみられる。 こうむり。 かむり。 かぶり。 かんぶり。 
\\	漢字の構成部位の一。 上下の組み合わせからなる漢字の上側の部分。 「安」の「宀(ウかんむり)」、「茶」の
\\	(草かんむり)」など。	▲先生は生徒の作った花の冠をかぶった。 ▲絵の中の少女は黄金ではなくて花のかんむりをかぶっています。
\\	管理	かんり	[名]スル 
\\	ある規準などから外れないよう、全体を統制すること。 「品質を―する」「健康―」「―教育」 
\\	事が円滑に運ぶよう、事務を処理し、設備などを保存維持していくこと。 「―の行き届いたマンション」「生産―」 
\\	法律上、財産や施設などの現状を維持し、また、その目的にそった範囲内で利用・改良などをはかること。 [類語]
\\	監理・監督・統轄・総轄・管轄・管掌・掌理・主管・所管・取り締まり・マネージメント(設備、財物などについて)保全・保守・維持・保管・管財・差配	▲その公園は市に管理されている。 ▲その事故の責任は管理人の怠惰にある。
\\	完了	かんりょう	[名]スル 
\\	物事が完全に終わること。 また、完全に終えること。 「開店の準備が―する」「予定の仕事を―する」 
\\	文法で、動作・状態がすでに終了していること、また、その結果が現在まで実現している状態にあることなどを表す言い方。 動詞に、文語では助動詞「つ」「ぬ」「たり」「り」、口語では助動詞「た」などを付けて表す。 ヨーロッパ諸語では、基準となる時の違いに応じて、現在完了・過去完了・未来完了などの区別のあるものがある。	▲君たちいつ準備は完了するつもりだい。 ▲アメリカ軍はイラクでの戦闘任務を完了することを発表した。
\\	関連・関聯	かんれん	[名]スルある事柄と他の事柄との間につながりがあること。 連関。 「―が深い」「―する事柄」「―性」「―質問」	▲自分の才能や興味に関連して仕事を選ぶべきだ。 ▲警察はその強盗に関連のある容疑者を逮捕した。
\\	議員	ぎいん	国会や地方議会など合議体の機関を構成し、議決権を有する者。	▲報道機関は現職議員についてはいつもマル秘情報をつかんでいます。 ▲彼は議員に自分から進んで立候補した。
\\	記憶	きおく	[名]スル 
\\	過去に体験したことや覚えたことを、忘れずに心にとめておくこと。 また、その内容。 「―に新しい出来事」「少年時代のことを今でも―している」「―力」 
\\	心理学で、生物体に過去の影響が残ること。 また、過去の経験を保持し、これを再生・再認する機能の総称。 
\\	コンピューターに必要なデータを蓄えておくこと。 [類語]
\\	覚え・物覚え・メモリー(昔のことについての記憶)思い出・追憶(――する)覚える・銘ずる・銘記する・牢記(ろうき)する・暗記する	▲彼女は記憶を新たにするためにその写真を見た。 ▲スペースシャトルの爆発はまだ私の記憶に新しい。
\\	気温	きおん	大気の温度。 ふつう地上一・五メートルの高さの通風のよい日陰で計った温度をいい、百葉箱に入れた温度計を用いる。	▲昨日は気温が零下5度に下がった。 ▲再び今日はかなり冷え込んでいるが、明日は気温が高くなると予想されます。
\\	機械・器械	きかい	
\\	動力を受けて、目的に応じた一定の運動・仕事をするもの。 
\\	実験・測定・運動競技などに使う装置・道具。 
\\	自分の意思を失ったように、指令どおりに動いたり、物事を繰り返したりすること。 ◆「工作機械」「包装機械」のように、動力を用いて操作する装置(マシーン)を「機械」、「測定器械」「光学器械」のように、人間が直接動かし、比較的小型で小規模な装置や道具(インストルメント)を「器械」と使い分けることが多い。 [類語]
\\	機器・機具・器具・利器・装置・機関・からくり・仕掛け・マシン・メカニズム	▲彼女はその機械を動かし始めた。 ▲あの機械の使い方を私に教えてくれたのはスミス氏だった。
\\	議会	ぎかい	公選された議員で組織され、選挙民の意思を代表して法律などを決定することを目的とする合議制の機関。 国会・都道府県議会・市区町村議会など。	▲彼が平等を勝ち取るための運動に献身していた期間に、いくつかの新しい法律が議会を通過していた。 ▲大統領はその法案に対して拒否権を行使しましたが、議会が再度それを無効にしました。
\\	期間	きかん	ある期日または日時から、他の期日または日時に至るまでの間。 「―を延長する」	▲彼らの研究では患者たちがセラピーの期間中にこのシンドロームを克服したかどうかが明確に述べられていない。 ▲彼は長い期間ここに滞在するだろう。
\\	機関	きかん	
\\	火力・水力・電力などのエネルギーを機械的エネルギーに変える装置。 
\\	活動のしかけのあるもの。 からくり。 「ただ一槌を受くるのみにて全体の―これが為に廃して」 
\\	法人や団体などの意思を決定したり、代表したりする者、または組織。 「行政―」「国家―」 
\\	ある目的を達成する手段として設けた組織や機構。 「報道―」	▲調査機関がその効果を調べた。 ▲サンプルは世界200の医療機関から収集された。
\\	企業	きぎょう	営利を目的として、継続的に生産・販売・サービスなどの経済活動を営む組織体。 また、その事業。 資本主義経済のもとでは、ふつう、私企業をさす。	▲企業の好業績発表を受けて株価は活発な取引の中で値を上げた。 ▲企業の政治団体は厳しい検査の対象になっています。
\\	利く・効く	きく	[動カ五(四)] 
\\	効果や働きなどが現れる。 期待どおりのよい結果が実現する。 効き目がある。 「てきめんに―・く薬」「宣伝が―・いて大評判だ」「腹部へのパンチが―・く」 
\\	本来の機能を十分に発揮する。 機敏に、また、さかんに活動する。 「鼻が―・く」「麻痺(まひ)して手足が―・かない」 
\\	それをすることが可能である。 できる。 「洗濯の―・く生地」「無理の―・かないからだ」「学割が―・く」 
\\	(多く「口を利く」の形で) ㋐言葉を発する。 物を言う。 「生意気な口を―・く」「口も―・かない仲」 ㋑間に入って、うまくいくように世話をしてやる。 まとまるように話をつける。 「取引先に口を―・いてやる」 
\\	技能がすぐれている。 腕が立つ。 「日頃―・いたる口三味線、太鼓持ちとなれり」 ◆ふつう、 
\\	は「効く」、 
\\	は「利く」と書く。 [下接句]大きな口を利く・押さえが利く・押しが利く・顔が利く・気が利く・小口を利く・潰(つぶ)しがきく・睨(にら)みが利く・鼻が利く・幅が利く・目が利く・目先が利く・目端(めはし)が利く・山葵(わさび)が利く	▲この薬は頭痛に効く。 ▲よくも私にそんな口を利けるものだな。
\\	機嫌・譏嫌	きげん	㊀[名] 
\\	表情や態度にあらわれる気分のよしあし。 快・不快などの感情。 気分。 「―がよい」「―を損ねる」 
\\	人の意向や思わく。 また、安否やようす。 「―をうかがう」 
\\	そしりきらうこと。 嫌悪すること。 「時人の―をかへりみず、誓願の一志不退なれば」 
\\	時機。 しおどき。 「病をうけ、子産み、死ぬることのみ―を計らず」 ㊁[形動][文][ナリ](多く「御機嫌」の形で)気分がよいさま。 愉快なさま。 「だいぶお酒が入ってご―なようす」→御機嫌(ごきげん) ◆もと「譏嫌」と書き、そしりきらうの意。 仏教で、他人の「譏嫌」を受けないようにする戒律「息世譏嫌戒」から出た語。 のちに「機」が、気持ちに通じる意を生じてから用いられるようになった。 [類語] 
\\	感情・気分・気(き)・気色(きしよく)・気持ち・顔色(かおいろ)・風向き/
\\	安否・様子	▲僕の行為でかれはきげんを損じた。 ▲彼は他人の機嫌を損ねないように気をつけている。
\\	気候	きこう	ある土地で、一年を周期として繰り返される大気の総合状態。 現在は気温・降水量・風などの三〇年間の平均値を気候値とする。 「―の変化が激しい」「温暖な―」「―のよい土地」 [類語]気象・季候・時候・陽気・寒暖・寒暑・天候・天気	▲気候の違いのため、同種の穀物が国の北部、東部においては収穫されていない。 ▲気候が性格に影響すると思いますか。
\\	岸	きし	
\\	陸地の、海・川・湖などの水に接している所。 みずぎわ。 「―に打ち寄せる波」 
\\	土地の切り立った所。 がけ。 「あしひきの山かも高き巻向(まきむく)の―の小松にみ雪降り来る」	▲風と潮の流れがその舟を岸に押しやった。 ▲彼らは船を岸に引き上げた。
\\	記事	きじ	
\\	事実を書くこと。 また、その文章。 
\\	新聞・雑誌などで伝える事柄。 また、その文章。 「事件を―にする」「三面―」 
\\	「記事文」の略。事実、事物を主として叙述する文。	▲私はその記事を旅行専門誌に採用してもらった。 ▲私は昨日、酸性雨についての記事を読んだ。
\\	生地・素地	きじ	
\\	手を加えていない、もともとの性質。 「―が出る」 
\\	化粧しないままの素肌。 素顔。 「―のままできれいな人」 
\\	布・織物などの地質。 また、染色や仕立てなどの加工をするための布・織物。 「―のいい背広」 
\\	陶磁器の、まだ釉(うわぐすり)を塗っていないもの。 
\\	パン・麺(めん)やパイ皮にするために、粉をこねあげたもの。	▲この生地はアイロンがよくきく。 ▲この生地で洋服を作ってください。
\\	技師	ぎし	
\\	機械・土木建築などの専門技術をもち、職業とする人。 エンジニア。 
\\	技官(ぎかん)の旧称。	▲彼は技師になりたく思っています。 ▲彼は技師だった、それで技師として扱われた。
\\	記者	きしゃ	
\\	新聞・雑誌や放送などで、記事の取材・執筆、また編集に携わる人。 
\\	文章を書く人。 文書を起草する人。 筆者。	▲記者が話をするゴリラのココについてパターソン博士にインタビューしている。 ▲記者たちは彼の私生活に関心を持っている。
\\	傷・疵・瑕	きず	
\\	切る、打つ、突くなどして、皮膚や筋肉が裂けたり破れたりした部分。 「深い―を負う」 
\\	物の表面の裂け目や、欠けたりした部分。 「レンズに―がつく」 
\\	人の行為・性質・容貌(ようぼう)などや物事の不完全な部分。 好ましくない点。 欠点。 「怒りやすいのが玉に―」 
\\	不名誉なこと。 恥ずべきこと。 汚点。 「経歴に―がつく」 
\\	心などに受けた痛手。 「失恋の―をいやす」 [下接語]後ろ傷・打ち傷・掠(かす)り傷・刀傷・咬(か)み傷・切り傷・刺し傷・擦(す)り傷・弾(たま)傷・突き傷・手傷・生(なま)傷・古傷・無傷・向こう傷・矢傷・山傷・槍(やり)傷	▲これは私の血ではない。すべて相手の返り血だ。私の身体には傷ひとつない。 ▲彼女の美貌もその傷で台無しになった。
\\	期待	きたい	[名]スルあることが実現するだろうと望みをかけて待ち受けること。 当てにして心待ちにすること。 「―に添うよう努力する」「活躍を―している」「―薄」	▲我々はほのかな期待を待って待った。 ▲我々は期待を胸に旅立った。
\\	帰宅	きたく	[名]スル自分の家に帰ること。 「夜中に―する」	▲彼は6時に帰宅するでしょうか。 ▲彼は6時に帰宅しますか。
\\	貴重	きちょう	㊀[名]スルとうとびおもんじること。 非常に大切にすること。 「主人の尤(もつと)も―する命が」 ㊁[形動][文][ナリ]非常に大切なさま。 得がたいものであるさま。 「―な時間を割く」「―な体験」	▲時間は他の何よりも貴重だ。 ▲時間は最も貴重なものだ。
\\	議長	ぎちょう	
\\	会議の席で、議事を進行させ採決を行う人。 また、機関としての会議を代表し、その活動の中心となる人。 「―をつとめる」「共闘会議―」 
\\	国会両議院や地方公共団体の議会で、議員中から選挙され、議事整理、議場の秩序維持、事務監督に当たり、議会を代表する人。	▲彼は議長に発言許可を求めた。 ▲彼は議長に任命された。
\\	きちんと	きちんと	[副]スル 
\\	よく整っていて、乱れたところのないさま。 「洋服を―着る」「部屋が―している」 
\\	正確な、また規則正しいさま。 「集会時間に―集まる」「家賃を―払う」 
\\	すきまや過不足のないさま。 ぴったり。 「帳尻が―合う」	▲メアリーは宿題を時間通りにきちんとやるべきです。 ▲メリーゴーランドの管理をしている男は、すべてがきちんと作動しているか確かめることに決めた。
\\	きつい	きつい	[形][文]きつ・し[ク] 
\\	物事の程度がはなはだしい。 「―・い勾配(こうばい)」「日ざしが―・い」 
\\	鼻や舌などへの刺激が強い。 「―・いにおい」「―・い酒」 
\\	力の入れ方・加わり方が強い。 「洗濯物を―・く絞る」「目を―・く閉じる」 
\\	ゆとりがなく、窮屈である。 「帯が―・い」「去年の服が―・くなった」 
\\	規律・要求などが厳しい。 「―・いおきて」「―・く戒める」 
\\	ある事柄をこなしたり、それに耐えたりするのが容易でない。 「仕事が―・い」「からだが―・い」 
\\	気性が激しい。 気が強い。 「―・いところのある子」「目つきが―・い」 [派生]きつがる[動ラ五]きつさ[名]	▲きつい生活が彼の健康にこたえてきている。 ▲この靴はきつすぎてはけない。
\\	気付く	きづく	[動カ五(四)] 
\\	それまで気にとめていなかったところに注意が向いて、物事の存在や状態を知る。 気がつく。 「誤りに―・く」「忘れ物に―・く」 
\\	意識を取り戻す。 正気に戻る。 気がつく。 「―・いたらベッドの上だった」	▲約束を果たさずに1ヶ月が過ぎてしまったのに気づいた。 ▲妙な話だが、我々は誰もその間違いに気付かなかった。
\\	キッチン	キッチン	《「キチン」とも》台所。 調理場「ダイニング―」	▲この2日間皿洗いする時間もないよ。キッチンの流しに山積みしているよ。 ▲キッチンよ。
\\	気に入る	きにいる	自分の好みや望みに合う。好きになる。	▲私に対する彼の口のきき方が気に入りません。 ▲私のやり方がきにいらないなら、自分の好きなようにしなさい。
\\	記入	きにゅう	[名]スル所定の用紙などに書き入れること。 「必要事項を―する」	▲ボールペンで申込書に記入しなさい。 ▲以下の空欄部分にご記入頂くだけで結構です。
\\	記念	きねん	[名]スル 
\\	思い出となるように残しておくこと。 また、そのもの。 「卒業を―して写真を撮る」「―品」 
\\	過去の出来事・人物などを思い起こし、心を新たにすること。 「創立五〇周年を―する式典」 ◆「紀念」とも書いた。	▲行事を記念してパレードが行われた。 ▲今日の式典はわが校の100周年を記念するものです。
\\	機能	きのう	[名]スルある物が本来備えている働き。 全体を構成する個々の部分が果たしている固有の役割。 また、そうした働きをなすこと。 「心臓の―」「言語の―」「正常に―する」	▲こんな余計な機能なんでつけたんだろう。 ▲システムのこの予測されなかった機能不全は不適切な配線系統によって引き起こされた。
\\	気の毒	きのどく	[名・形動]スル《もと、自分の気持ちにとって毒になることの意で、「気の薬(くすり)」に対する語》 
\\	他人の不幸や苦痛などに同情して心を痛めること。 また、そのさま。 「お―に存じます」「―な境遇」 
\\	他人に迷惑をかけて申し訳なく思うこと。 また、そのさま。 「彼には―なことをした」 
\\	気にかかること。 不快に思うこと。 また、そのさま。 「思ひもつかねえことを言はれると、おいらも腹は立たねえが―だ」 
\\	困ってしまうこと。 きまりが悪いこと。 また、そのさま。 「親方の手前―のおもはくにて、顔を真赤にしてゐる」 [派生]きのどくがる[動ラ五]きのどくげ[形動]きのどくさ[名] [類語]
\\	可哀相(かわいそう)・哀れ・不憫(ふびん)・痛痛しい・痛ましい・労(いたわ)しい/
\\	心苦しい・済まない・申し訳ない	▲我々は彼の間違いを気の毒に思う。 ▲気の毒な光景は私たちの涙をさそった。
\\	寄付・寄附	きふ	[名]スル公共事業や社寺などに、金品を贈ること。 「―を募る」「被災者に衣類を―する」「―金」	▲彼はその大学に多額のきふをした。 ▲彼はその病院に多額の寄付をした。
\\	希望・冀望	きぼう	[名]スル 
\\	あることの実現をのぞみ願うこと。 また、その願い。 「みんなの―を入れる」「入社を―する」 
\\	将来に対する期待。 また、明るい見通し。 「―に燃える」「―を見失う」 
\\	文法で、 
\\	の意を表す言い方。 動詞に、文語では助動詞「たし」「まほし」、口語では助動詞「たい」などを付けて言い表す。 [類語]
\\	望み・願い・夢・願望・志望・素志(人に何かを望むこと)要望・所望(しよもう)・注文(―する)望む・欲する/
\\	望み・期待・光明(こうみよう)・曙光(しよこう)・光・ホープ	▲彼は希望に満ち溢れていた。 ▲彼は君に彼の希望を伝えてくれと私に頼んだ。
\\	基本	きほん	判断・行動・方法などのよりどころとなる大もと。 基礎。 「―の型」「―を身につける」「―に忠実な演技」→基礎[用法] [類語]大本(おおもと)・基礎・根本(こんぽん)・根幹・中心・基軸・基調・基底・根底・基(もとい)・土台・下地・初歩・いろは・ABC	▲基本ルールを学んでしまえば、そのゲームは簡単です。 ▲基本計画には、レクリエーションだけでなく職を供給する計画も含まれている。
\\	決まり・極まり	きまり	
\\	物事が決まること。 問題になっていたり面倒だったりした物事の終わり。 決着。 おさまり。 「これで話は―だ」「仕事に―を付ける」 
\\	よりどころとして定められている事柄。 規則。 通則。 「―を破る」「―に従う」 
\\	一定していること。 いつものこと。 定例。 「散歩が朝の―だ」 
\\	(多く「おきまり」の形で)言動がいつも同じで新鮮味がないこと。 また、きまり文句。 「お―の自慢話」 
\\	面目。 体裁。 「文三はお勢よりは―を悪がって口数をきかず」 
\\	万事首尾よくいっていること。 明和・安永(一七六四〜一七八一)ころの江戸の流行語。 「『朧月(おぼろづき)と五色丹前を買って参りやした』『おお―』」 
\\	遊里で、客と遊女が恋仲になること。 また、その相手や間柄。 「今夜のぬしの客衆はとんだ―だの」	▲母親は息子の行儀の悪さにきまりの悪い思いをした。 ▲毎日のきまり仕事にはつくづく飽きた。
\\	気味	きみ	
\\	ある事態や物事から受ける感じ。 また、その感じた気持ち。 きび。 「―が悪い」「総て―のよい、きらびやかな、うつくしい、月は」 
\\	いくらかその傾向にあること。 「かぜの―がある」 
\\	香りと味。 「喉(のど)渇き口損じて、―も皆忘れにけり」 
\\	物事の趣。 味わい。 「閑居の―もまた同じ」→気味(ぎみ)	▲最近気分が萎え気味だ。 ▲その事件には何となく気味の悪いところがあった。
\\	奇妙	きみょう	[名・形動] 
\\	珍しく、不思議なこと。 また、そのさま。 「科学では説明できない―な現象」 
\\	風変わりなこと。 また、そのさま。 「―な格好」 
\\	非常に趣・おもしろみ・うまみなどがあること。 また、そのさま。 「むむ、それは―だ。 世話でもそれを煮てくんな」 [派生]きみょうさ[名]	▲彼は、奇妙な音を聞いてベッドから飛びおきた。 ▲彼の話は奇妙に聞こえる。
\\	義務	ぎむ	
\\	人がそれぞれの立場に応じて当然しなければならない務め。 「―を果たす」↔権利。 
\\	倫理学で、人が道徳上、普遍的・必然的になすべきこと。 
\\	法律によって人に課せられる拘束。 法的義務はつねに権利に対応して存在する。 「納税の―」↔権利。	▲義務を果たすように努力しなさい。 ▲義務を怠ってはならない。
\\	疑問	ぎもん	
\\	うたがい問うこと。 「―を発する」 
\\	本当かどうか、正しいかどうか、疑わしいこと。 また、その事柄。 「学説に―をいだく」「本物であるかどうかは―だ」 [類語]
\\	質疑・質問・クエスチョン/
\\	疑い・不審・疑義・疑念・疑団・懐疑・疑点	▲弁護士は彼の無実に疑問を持った。 ▲僕の先輩たちは、純粋な好奇心を抱いて自分たちの疑問を自然に問いかけ、自然が答えるのを待った。
\\	逆	ぎゃく	[名・形動] 
\\	物事の順序・方向などが反対であること。 また、そのさま。 さかさま。 「立場が―になる」「―コース」↔順。 
\\	論理学で、ある命題の主語と述語を換位して得られる命題。 
\\	ならばqである」に対して
\\	ならばpである」という形式の命題。 最初の命題が真でも、逆命題は必ずしも真ではない。 
\\	柔道で、関節技のこと。 逆手(ぎやくて)。 
\\	道理や道徳に反すること。 また、そのさま。 「朝廷の御為(おんため)には…―に与(くみ)する条理なし」	▲ベストを裏返さなくちゃ。表裏逆ですよ。 ▲テニスやピンポンのバックハンドでは手は球を打つ際逆向きになる。
\\	キャプテン	キャプテン	集団の統率者。 船長・艦長・機長、チームの主将など。	▲前のキャプテンは、現在のキャプテンより優れていました。 ▲赤木キャプテンは練習中に足を捻挫したので、試合前にテーピングでガチガチに固めた。
\\	キャンプ	キャンプ	[名]スル 
\\	テントを張って野営すること。 「高原で―する」《季 夏》「白樺の雨に来て張る―あり/たかし」 
\\	軍隊の駐屯地。 「米軍―」 
\\	スポーツ選手などの合宿練習。 「プロ野球のスプリング―」 
\\	捕虜や難民の収容所。	▲水がない所ではキャンプはできません。 ▲私は高熱があった。さもなければキャンプに行くことができただろう。
\\	球	きゅう	
\\	丸いもの。 たま。 
\\	空間の一定点から一定の距離にある点の軌跡。 その定点を球の中心、一定距離を球の半径という。	
\\	級	きゅう	㊀[名] 
\\	物事を上下の地位・段階に分ける区切り。 階級。 等級。 「柔道の―が上がる」 
\\	学校で、同一の学年。 また、学級。 組。 クラス。 「彼はぼくより一つ―が上だ」 
\\	写真植字の文字の大きさの単位。 一級は四分の一ミリ。 ㊁〔接尾〕 
\\	名詞に付いて、その程度であることを表す。 「国宝―の重要文化財」 
\\	珠算や柔・剣道など、技能の段階に応じて免許状を発行するようなものについて、その程度・段階などを表すのに用いる。 「珠算三―」「二―整備士」 
\\	助数詞。 ㋐学校で学級を数えるのに用いる。 「一学年を五―に分ける」 ㋑階段の一つ一つを数えるのに用いる。 「入口の石段を二三―上ると」	▲海外の子会社は最高級品を生産しています。 ▲私の日本の級友達は、いつも一緒にいるので、お互いに大変よく知り合うことになりました。
\\	旧	きゅう	
\\	古いこと。 古い物事。 「―を捨て新につく」↔新。 
\\	昔。 以前。 もと。 「―に倍するお引き立て」「―に復する」 
\\	「旧暦」の略。古い暦。昔、用いられた暦。 「―の正月」↔新。	▲官僚は旧法規の改正に関して、成り行きを見守っています。 ▲旧システム用に書かれたソフトとの上位互換性を保つのは大切です。
\\	休暇	きゅうか	会社・官庁・学校などで認められた、休日以外の休み。 「―をとる」「夏季―」	▲我々は休暇の大部分を田舎で過ごした。 ▲海辺で休暇を過ごしている。
\\	休憩	きゅうけい	[名]スル仕事や運動などを一時やめて、休むこと。 休息。 「ゆっくり―をとる」	▲旅行者の便宜をはかって高速道路沿いに多くの休憩場所がある。 ▲彼らはみんなその休憩を待ち焦がれた。
\\	急激・急劇	きゅうげき	[形動][文][ナリ]物事の変化や動きなどが急で、はげしいさま。 「気温の―な変化」	▲温度の急激な変化に順応するのは困難である。 ▲株価の急激な下落があった。
\\	吸収	きゅうしゅう	[名]スル 
\\	吸い取ること。 「汗を―する」 
\\	外から内に取り入れて自分のものにすること。 「知識を―する」「大資本に―される」 
\\	音や光・粒子線などが物質を通過するとき、そのエネルギーや粒子が物質中に取り込まれて失われること。 また、気体が液体や固体の内部に取り込まれること。 
\\	生物体が生体膜を通して物質を内部に取り入れること。 特に、栄養素を消化管壁の細胞膜を通して血管・リンパ管中に取り入れることをいい、主に小腸で行われる。 植物では根から水分などを吸い入れることをいう。	▲バンパーが衝撃をいくらか吸収してくれた。 ▲黒い紙は、光を吸収する。
\\	救助	きゅうじょ	[名]スル危険な状態から救い助けること。 被災者・遭難者などを救うこと。 「沈没船の乗組員を―する」「人命―」	▲彼らは通りかかった船に救助された。 ▲彼らは直ちに私たちに救助に来た。
\\	急速	きゅうそく	[名・形動]物事の起こり方や進み方が非常に速いこと。 またそのさま。 「―な時代の変化」「―に親しくなる」	▲比較研究が今や急速に進歩した。 ▲彼女は急速に英語力が伸びた。
\\	給料	きゅうりょう	労働者・使用人などに対して、雇い主が支払う報酬。 俸給。	▲彼はいい給料をもらっている。 ▲彼の第一の趣味であるヨットに彼の給料の大部分が費やされる。
\\	器用	きよう	[名・形動] 
\\	からだを思うように動かして、芸事・工作などをうまくこなすこと。 また、そのさま。 「手先が―だ」「―に箸(はし)を使う」 
\\	要領よく、いろいろな物事を処理すること。 また、そのさま。 「何事も―にこなす」 
\\	抜けめなく立ち回ること。 また、そのさま。 「世渡りが―だ」 
\\	不平不満なく、受け入れること。 いさぎよいこと。 また、そのさま。 「なんにも言わずに、―に買っときなさい」「気遣ひしやるな、逃げはせぬと、もっとも―な白状」 
\\	すぐれた才能のあること。 また、その人。 「武家の棟梁(とうりやう)と成りぬべき―の仁(じん)」 [派生]きようさ[名]	▲ライザは大変器用なので、自分でネジや同じような小物を作ることさえできる。 ▲君は手先が器用だね。
\\	教科書	きょうかしょ	教科の主たる教材として用いられる図書。 教科用図書。	▲彼は日本の民話を教科書用に纏めた。 ▲彼は教科書型の人間だ。
\\	競技	きょうぎ	[名]スル一定の規則に従って、技術や運動能力の優劣を互いにきそうこと。 「陸上―」「珠算―」	▲その体の弱い少年はその厳しい競技でふるい落とされた。 ▲ついに1314年には、この競技はとても乱暴で危険なものとなっていたので、エドワード二世は法律を制定したのです。
\\	行儀	ぎょうぎ	
\\	礼儀の面からみた立ち居振る舞い。 また、その作法・規則。 「―よく座る」「―が悪い」 
\\	しわざ。 行状。 「悉皆(しつかい)盗人の―か」 
\\	仏語。 行事の儀式のやり方。	▲もうこれ以上彼の行儀の悪さには我慢できない。 ▲行儀の悪さは彼の良識を疑わせるものだ。
\\	供給	きょうきゅう	[名]スル 
\\	必要に応じて、物を与えること。 「被災者に物資を―する」 
\\	販売のために、商品を市場に出すこと。 また、その数量。 ↔需要。	▲彼等は兵士達に水と食料を十分供給した。 ▲彼等は、戦争の被災者達に食料を供給した。
\\	教師	きょうし	
\\	学校などで、学業・技芸を教える人。 先生。 教員。 「数学の―」「家庭―」 
\\	宗教上の教化を行う人。	▲ベイカー先生は教師というよりはむしろ学者だ。 ▲委員会は教師と親から成り立っている。
\\	教授	きょうじゅ	[名]スル 
\\	学問や技芸を教え授けること。 「書道を―する」 
\\	児童・生徒・学生に知識・技能を授け、その心意作用の発達を助けること。 
\\	大学や高等専門学校・旧制高等学校などで、研究・教育職階の最高位。 また、その人。 「大学―」	▲彼はその教授の話し方を真似してからかった。 ▲彼はその教授を大いに尊敬している。
\\	強調	きょうちょう	[名]スル 
\\	ある事柄を特に強く主張すること。 「事の重大さを―する」 
\\	絵画・音楽などで、ある一部分を特に目立つように表現すること。 
\\	取引相場が上がろうとしている状態。	▲彼女はそれを自分でやった事を強調した。 ▲彼は平和の大切さを強調した。
\\	共通	きょうつう	[名・形動]スル二つまたはそれ以上のものの、どれにもあること。 どれにもあてはまること。 また、そのさま。 「―の理解」「国民に―な意見」「両者に―する特徴」 [類語]通有・普遍・同一・一律・一つ(―する)通ずる・通う・通底する・軌(き)を一(いつ)にする	▲私と彼は共通することが多い。 ▲私の理解ではその二つの実験には共通の因子はない。
\\	共同	きょうどう	[名]スル 
\\	複数の人や団体が、同じ目的のために一緒に事を行ったり、同じ条件・資格でかかわったりすること。 「―で経営する」「―で利用する」「三社が―する事業」 
\\	「協同」に同じ。 [類語]
\\	合同・協同・連携・提携・連名・共有・共用・催合(もや)い・タイアップ	▲彼女はアパートを、友達と共同で使っていた。 ▲彼らは南ローデシアに着くと、共同給水設備のある泥でできた簡易住宅からなる移民キャンプか、ホテルを選択しなければならなかった。そこで彼らは資産家として知られていたのでホテルを選んだ。
\\	恐怖	きょうふ	[名]スルおそれること。 こわいと思うこと。 また、その気持ち。 「―にかられる」「人心を―せしめる事件」「―心」	▲多くの子どもにとって暗闇は恐怖のまとである。 ▲全国民は恐怖からこの独裁者の前にひれ伏した。
\\	協力	きょうりょく	[名]スル力を合わせて事にあたること。 「―を仰ぐ」「事業に―する」	▲私は互いに協力せざるをえなかった。 ▲私は姉と協力して部屋を掃除した。
\\	強力	きょうりょく	[名・形動]力や作用が強いこと。 また、そのさま。 ごうりき。 「―な味方」「運動を―に推進する」 [派生]きょうりょくさ[名]	▲ヨーロッパは強力な雑種民族大陸である。 ▲我々のチームは強力なライバルと競った。
\\	許可	きょか	[名]スル 
\\	願いを聞き届け、ある行為・行動を許すこと。 「外出の―が下りる」「使用を―する」 
\\	ある行為が一般に禁止されているとき、特定の場合にそれを解除し、適法にその行為ができるようにする行政行為。 警察許可・財政許可・統制許可などがある。 [類語]
\\	認可・許諾・承認・認許・允許(いんきよ)・允可(いんか)・容認・許容・聴許・裁許・免許・公許・官許・許し・オーケー・ライセンス	▲いまや大統領がなくなったので、新しい政策は政府の許可をまたなければならない。 ▲ここの駐車許可をとりましたか。
\\	局	きょく	㊀[名] 
\\	官庁などで、業務分担の大きな区分。 また、それを扱う部署。 部・課などの上にある。 
\\	「郵便局」「放送局」などの略。 
\\	当面の事件・仕事・職務。 「その―に当たる」 
\\	囲碁・将棋などの盤。 また、その盤を使ってする勝負。 「―に向かう」 ㊁〔接尾〕助数詞。 囲碁・将棋などで、対局数を数えるのに用いる。 「碁を三―打つ」「名人戦第五―」	▲この地区では三局受信できます。 ▲疫病管理局の報告では、罹患率は10パーセントだった。
\\	巨大	きょだい	[名・形動]非常に大きいこと。 また、そのさま。 「―な船体」 [派生]きょだいさ[名]	▲北部には巨大な山々がある。 ▲飛行機から見るとその島は巨大なクモのように見える。
\\	嫌う	きらう	[動ワ五(ハ四)] 
\\	いやがって、その対象とかかわりたくないと思う。 好ましくないものとして、避ける。 「世間から―・われる」「不誠実な人を―・う」 
\\	はばかって、それをしないようにする。 また、そうすることをいやだと思う。 「葬式は友引の日を―・う」「相手の差し手を―・う」 
\\	(人になぞらえた言い方で)それがあるとそこなわれやすいので避けるべきである。 「塩は湿気を―・う」 
\\	(「きらわず」の形で用いる)区別する。 わけへだてをする。 「相手―・わず論争を挑む」「所―・わずつばをはく」 
\\	連歌・連句で、句の配列上、同類の言葉を付けたり、ある特定の語を特定の場所に使ったりすることを忌み避ける。 
\\	よくないものとして退ける。 「穢(きたな)き奴(やつこ)どもを―・ひ賜ひ棄て賜ふに依りて」 [用法]きらう・いやがる――「ゴキブリを嫌う(いやがる)」では相通じて用いられるが、「ごきぶりを嫌って、見るのもいやがった」では「嫌う」「いやがる」を入れ換えると不自然になる。 
\\	「嫌う」はいやだと思う気持ちを示すだけだが、「いやがる」は嫌う気持ちを態度や言葉に表すことである。 「母と別れるのをいやがって泣いた」「いやがる相手を交渉の場に引き出す」などでは「嫌う」は使わない。 
\\	類似の語に「いとう」がある。 「いとう」はやや古い言葉で、「世をいとう」のように、できれば避けたいものだという気持ちが強い。 
\\	また、「おからだ、十分においといください」は、悪い要素を避けておいたわりくださいの意で、「いとう」独特の用法。	▲彼女は彼を嫌っていた。 ▲彼女は時間を守らない人を嫌っている。
\\	霧	きり	《動詞「き(霧)る」の連用形から》 
\\	地表や海面付近で大気中の水蒸気が凝結し、無数の微小な水滴となって浮遊する現象。 古くは四季を通じていったが、平安時代以降、秋のものをさし、春に立つものを霞(かすみ)とよび分けた。 気象観測では、視程一キロ未満のものをいい、これ以上のものを靄(もや)とよぶ。 《季 秋》「―しばし旧里に似たるけしき有り/几董」 
\\	液体を細かい水滴にして空中に飛ばしたもの。 「―を吹いてアイロンをかける」 [下接語]雲霧・黒い霧(ぎり)秋霧・朝霧・薄霧・川霧・狭(さ)霧・初秋(はつあき)霧・山霧・夕霧・夜霧	▲彼は霧のせいであまり遠くまで見えなかった。 ▲飛行機は霧の為離陸できなかった。
\\	切れ	きれ	㊀[名] 
\\	㋐物の切れ端。 「板の―」「布―」 ㋑(「布」「裂」とも書く)織物を切ったもの。 また、織物。 布(ぬの)。 「木綿の―」「余り―(ぎれ)」 ㋒書画などの、古人の筆跡の断片。 断簡。 「高野―(ぎれ)」「古筆―(ぎれ)」 
\\	刃物の切れぐあい。 切れ味。 「包丁の―がにぶる」 
\\	㋐頭脳や技術の働きの鋭さ。 「頭の―のいい人」「技に―がない」 ㋑投げた球の曲がりぐあいの鋭さ。 「カーブの―がいい」 ㋒さらっとして後に残らない口あたり。 「―のいいウイスキー」 
\\	㋐水気などがなくなること。 また、そのぐあい。 「油の―がよくないフライ」 ㋑付着していたものや残っていたものがなくなること。 また、そのぐあい。 「泡の―のよい洗剤」「痰(たん)の―をよくする薬」 
\\	目じりの切れ込みのぐあい。 「―の長い目」 
\\	石材の体積の単位。 一切れは一尺立方で、約〇・〇二八立方メートル。 
\\	(「ぎれ」の形で)名詞の下に付き、そのものを使い切っている意を表す。 「期限―」「在庫―」 
\\	同類の中の末端の一人。 はしくれ。 「望んで軍(いくさ)に立ってこそ男の―ともいふべけれ」 ㊁〔接尾〕助数詞。 
\\	切ったものを数えるのに用いる。 「たくあん一―」「ようかん二―」 
\\	江戸時代、一分金を数えるのに用いる。 「白銀五百匁二包み、小判二十五両一歩合わせて四十―」 [下接語]板切れ・紙切れ・半切れ・一切れ・棒切れ・襤褸(ぼろ)切れ(ぎれ)当て切れ・有り切れ・歌切れ・裏切れ・恵比須(えびす)切れ・木切れ・錦(きん)切れ・小切れ・古(こ)切れ・古代切れ・古筆切れ・細(こま)切れ・時代切れ・竹切れ・裁ち切れ・継ぎ切れ・出切れ・共切れ・布切れ・端(は)切れ・古(ふる)切れ・名物切れ・寄せ切れ	▲私はその犬に肉を2切れやった。 ▲彼はその袋を作るのに一片の大きな紙切れを使いました。
\\	切れる	きれる	[動ラ下一][文]き・る[ラ下二] 
\\	力が加わって、ひと続きのもの、つながっているものなどが分かれる。 「ひもが―・れる」「緊張の糸が―・れる」 
\\	傷ついたり、裂け目ができたりする。 「ひびが―・れる」「手の―・れるような札(さつ)」 
\\	㋐破れ崩れる。 「堤防が―・れる」 ㋑こすれてへる。 すりきれる。 「着物の裾が―・れる」「靴下が―・れる」 
\\	つながっていた関係がなくなる。 「縁が―・れる」「彼女とはもう―・れた」 
\\	㋐今まで続いていたものが、途中やあるところでなくなる。 「音信が―・れる」「電話が―・れる」「―・れた雲間から光がさす」 ㋑並び続いているものがとだえる。 「町並みが―・れる」「人通りが―・れる」 
\\	㋐売れたり使いきったりして今まであった物がなくなる。 「品物が―・れている」「油が―・れる」 ㋑それが有効である一定の時間・期間が終わりになる。 「期限が―・れる」「車検が―・れる」「麻酔が―・れる」 ㋒ある基準以下になる。 不足する。 「元値が―・れる」「目方が―・れる」 
\\	振り落としたり、したたらせたりした結果、水分がなくなる。 「洗った野菜の水気が―・れる」 
\\	進む方向が横へそれる。 左右いずれかへ曲がる。 「道が左へ―・れる」「打球が―・れる」 
\\	トランプ・カルタなどの札(ふだ)がよくまじり合う。 
\\	囲碁で、石がつながらない状態になる。 
\\	㋐刃物の切れ味が鋭い。 「よく―・れる刀」 ㋑見事な効果を示す。 「技が―・れる」 ㋒頭の働きが速く、物事をてきぱきと処理する能力にすぐれる。 「頭が―・れる」「―・れる男」 
\\	(「息がきれる」の形で)激しい運動などのあと、せわしく苦しそうに呼吸する。 
\\	(「しびれがきれる」の形で) ㋐足がしびれる。 ㋑待ちくたびれる。 
\\	(動詞の連用形に付いて、多く打消しの語を伴って用いる) ㋐最後までしとおすことができる。 「こらえ―・れずに泣き出す」「逃げ―・れないと観念する」 ㋑すっかり…することができる。 「食べ―・れないほどのごちそう」 
\\	物事のきまりがつく。 「よき事もわろき事も其時ことは―・るるなり」 
\\	勢力をもつ。 幅が利く。 「わしゃあこではえらう―・れるがな」 
\\	気前よく振る舞う。 金離れがよい。 「是でもはづむ所では随分―・れて見せるよ」 [下接句]油が切れる・息が切れる・堪忍袋の緒(お)が切れる・痺(しび)れが切れる・手が切れる・元が切れる	▲ナイフが切れなくなった。 ▲「どーした、もじもじして」「あーいや、何かパンツのゴム切れちゃったみたいで」。
\\	記録	きろく	[名]スル 
\\	将来のために物事を書きしるしておくこと。 また、その書いたもの。 「―に残す」「実験の―」「議事を―する」 
\\	競技などで、数値として表された成績や結果。 また、その最高数値。 レコード。 「―を更新する」 
\\	歴史学・古文書学で、史料としての日記や書類。 [類語]
\\	筆録・採録・詳録・登録・記載(―する)録する・書き留(とど)める(事実を書きしるしたもの)実録・実記・記事・ドキュメント	▲その火事は記録に残っている。 ▲この冬は降雪量の記録を更新した。
\\	議論	ぎろん	[名]スル互いの意見を述べて論じ合うこと。 また、その内容。 「―を戦わす」「―を尽くす」「仲間と―する」	▲議論は白熱していた。 ▲議論は相互の尊敬の念に基づいている。
\\	銀	ぎん	
\\	銅族元素の一。 金と並び称される貴金属。 単体は白色で金属光沢がある。 電気・熱の伝導性は金属中最大で、展延性は金に次いで大きく、厚さ〇・一五マイクロメートルの箔(はく)にすることが可能。 硝酸および熱硫酸に溶け、硫黄や硫化水素で黒変する。 自然銀・輝銀鉱などとして産出。 装飾品・貨幣・感光材料などに使用。 記号
\\	原子番号四七。 原子量一〇七・九。 しろがね。 
\\	銀貨。 また、貨幣。 「―一〇枚」「小判走れば―が飛ぶ」 
\\	ぎんいろ。 「―の世界」 
\\	将棋の駒で、銀将。 
\\	銀メダル。 
\\	銀ギセルなど 
\\	で作ったものの略称。 「藁(わら)つ火へ―を突っ込む田舎道」	▲昔は、書物の価値は同じ重さの金とまでは行かないにしても、銀ぐらいの価値があった。 ▲世界中の銀のティースプーンのコレクションがある。
\\	禁煙・禁烟	きんえん	[名]スル 
\\	タバコを吸う習慣を断つこと。 「健康のため―する」 
\\	タバコを吸うのを禁止すること。 「―車」	▲映画館内は禁煙です。 ▲釈迦に説法とは存じますが、医者も禁煙されたほうがよろしいのではないでしょうか。
\\	金額	きんがく	具体的な数字で表される金銭の量。 きんだか。 「莫大(ばくだい)な―」	▲私はその金額の2倍払った。 ▲購入金額を返金してくれるのかどうか、教えてください。
\\	金庫	きんこ	
\\	金銭・財宝を保管するための倉庫。 かねぐら。 
\\	現金・重要書類・貴重品などを盗難や火災から守り安全にしまっておくための鋼鉄製などの箱や室。 
\\	国または地方公共団体の現金出納機関。 
\\	特別法によって設立された特殊金融機関の名称。 農林中央金庫・商工組合中央金庫の二つがある。 信用金庫・労働金庫は一般の金融機関。	▲金庫にカギをかけてください。 ▲金庫に金を入れる。
\\	禁止	きんし	[名]スル《古くは「きんじ」とも》ある行為を行わないように命令すること。 「通行を―する」「外出―」 [類語]禁・禁制・禁断・禁令・禁遏(きんあつ)・禁圧・厳禁・無用・法度(はつと)・差し止め・駄目(だめ)・禁忌	▲父は私が夜外出するのを禁止している。 ▲武器の輸出が禁止された。
\\	金銭	きんせん	
\\	貨幣の総称。 かね。 ぜに。 「―に細かい人」「―感覚」 
\\	金で鋳造した銭。 金貨。	▲若い人によくあることだが、彼も金銭に無頓着であった。 ▲私の父は私の金銭的問題を助けてくれた。
\\	金属	きんぞく	一般に、金属光沢をもち、熱や電気をよく伝え、強度が大きくて折れにくく、展性・延性をもち、常温で固体の物質の総称。 重金属と軽金属、貴金属と卑金属、遷移金属と非遷移金属などに分類される。	▲酸は金属を含む多くのものに作用する。 ▲酸は金属を含むものに作用する。
\\	近代	きんだい	
\\	現代に近い時代。 また、現代。 「―都市」 
\\	歴史の時代区分の一。 広義には「近世」と同義であるが、一般には封建制社会のあとの資本主義の社会をいう。 日本史では明治維新から太平洋戦争の終結まで、西洋史では市民革命・産業革命からロシア革命までの時代。	▲近代医学の進歩は長い道程を歩んだ。 ▲近代技術は多くの物を与えてくれる。
\\	緊張	きんちょう	[名]スル 
\\	心やからだが引き締まること。 慣れない物事などに直面して、心が張りつめてからだがかたくなること。 「―をほぐす」「―した面もち」 
\\	相互の関係が悪くなり、争いの起こりそうな状態であること。 「―が高まる」「―する国際情勢」 
\\	生理学で、筋肉や腱(けん)が一定の収縮状態を持続していること。 
\\	心理学で、ある行動への準備や、これから起こる現象・状況などを待ち受ける心の状態。	▲あまりやるべき仕事があると緊張し落ち着かない。 ▲いつも緊張しています。
\\	筋肉	きんにく	収縮性をもつ動物特有の運動器官。 原生・中生・海綿動物を除くすべてに存在。 脊椎動物では量が多く、たんぱく質に富む。 骨格に付着する骨格筋と、心臓壁をなす心筋は横紋筋からなり、胃腸などの壁をなす内臓筋は平滑筋からなる。 骨格筋は多数の筋繊維が束状に集まり、紡錘状などの形となっている。 筋。	▲水泳は体のいろいろな筋肉を発達させる。 ▲体中の筋肉が痛いです。
\\	金融	きんゆう	金銭の融通。 特に、資金の借り手と貸し手のあいだで行われる貨幣の信用取引。	▲彼の会社は日本で最もうまく経営されている消費者金融だ。 ▲日本は金融市場を開放するようにアメリカの圧力を受けた。
\\	金曜	きんよう	週の第六日。 木曜の次の日。 金曜日。	▲雪は月曜から金曜まで降った。 ▲彼らは普通月曜から金曜まで学校に行きます。
\\	句	く	㊀[名] 
\\	文中の言葉のひと区切り。 
\\	詩歌の構成上の単位。 ㋐和歌・俳句などで、韻律上、五音または七音からなるひと区切り。 また、その組み合わせでひとまとまりとなったもの。 「上(かみ)の―」 ㋑漢詩で、四字・五字・七字などからなるひと区切り。 
\\	連歌・連句の発句(ほつく)。 また、俳句。 「―を詠む」 
\\	慣用句やことわざ。 
\\	言語単位の一。 ㋐単語が連続して一つのまとまった意味を表し、文を形成するもの。 また、それが文の一部分をなすもの。 フレーズ。 ㋑二つ以上の単語が連なって、あるまとまった意味を表し、一つの単語と似たような働きをなすもの。 「副詞―」 ㋒文の構成要素の一つで、一つの自立語、または、それに付属語のついたもの。 文節。 ㊁〔接尾〕助数詞。 連歌の各句や俳句などを数えるのに用いる。 「一―浮かんだ」	▲私は単語や句をたくさん覚えなければならない。 ▲この句はどう解釈したらよいのだろうか。
\\	食う・喰う	くう	[動ワ五(ハ四)] 
\\	食物をかんでのみ込む。 食べる。 「飯を―・う」 
\\	生活をする。 暮らしを立てる。 「こんな薄給では―・っていけない」 
\\	口で物をしっかり捕らえる。 食いつく。 「えさを替えたら魚がよく―・う」 
\\	虫などがかじって物を傷める。 また、虫などがからだを刺す。 「衣魚(しみ)の―・った書籍」「蚊に―・われる」 
\\	しっかりと間に挟む。 また、縄状のものが物にめり込む。 「ファスナーに布地が―・われる」 
\\	金銭・時間などがかかる。 費やす。 「この車はガソリンを―・う」「手間ひま―・う仕事」 
\\	(「年をくう」の形で)かなりの年齢になる。 「いたずらに年を―・うばかりだ」 
\\	他の勢力範囲・領域に入り込む。 侵す。 「縄張りを―・う」 
\\	スポーツなどで、強い相手を負かす。 「強敵を―・う」 
\\	演劇・映画などで、ある俳優の演技が勝っていて共演者をしのぐ。 「脇役に―・われる」 
\\	他から、ある行為、特に望ましくない行為を受ける。 こうむる。 「門前払いを―・う」「お目玉を―・う」「肩すかしを―・う」 
\\	(「人をくう」の形で)ばかにする。 侮る。 「人を―・った態度」 
\\	自分の利益のために、だまして人を利用する。 食い物にする。 「タレント志望の少女たちを―・う芸能プロダクション」 
\\	演劇で、上演台本の一部を省略する。 カットする。 
\\	口で軽く挟んで物を支える。 くわえる。 ついばむ。 「春霞流るるなへに青柳の枝―・ひ持ちてうぐひす鳴くも」 
\\	かみつく。 歯をたてる。 「指(および)ひとつを引き寄せて―・ひてはべりしを」 
\\	薬などを飲む。 「つとめて―・ふ薬といふもの」 ◆現代語では、食する意では「食う」がぞんざいで俗語的とされ、一般に「食べる」を用いる。 しかし、複合語・慣用句では「食う」が用いられ、「食べる」とは言い換えができないものもある。 「たべる(たぶ)」はもともと謙譲・丁寧な言い方であったが、敬意がしだいに失われ通常語となった。 [可能]くえる [下接句]泡を食う・一杯食う・犬も食わぬ・同じ釜(かま)の飯を食う・鬼を酢にして食う・糟(かす)を食う・霞(かすみ)を食う・気に食わない・臭い飯を食う・背負(しよ)い投げを食う・粋(すい)が身を食う・すかを食う・側杖(そばづえ)を食う・他人の飯を食う・年を食う・栃麺棒(とちめんぼう)を食う・取って食う・煮て食おうと焼いて食おうと・煮ても焼いても食えない・塗り箸(ばし)で素麺(そうめん)を食う・弾みを食う・人を食う・冷や飯を食う・道草を食う・無駄飯を食う・割を食う	▲そんなの忘れてちょっくら晩飯でも食いにこいや。 ▲このエンジンが一番油を食う。
\\	偶然	ぐうぜん	㊀[名・形動]何の因果関係もなく、予期しないことが起こること。 また、そのさま。 「―の一致」「―に見つける」↔必然。 [派生]ぐうぜんさ[名] ㊁[副]思いがけないことが起こるさま。 たまたま。 「―旧友に出あう」	▲私は昨日空港で旧友に偶然出会った。 ▲私は昨日パーティーで偶然旧友にであった。
\\	臭い	くさい	㊀[形][文]くさ・し[ク] 
\\	不快なにおいを感じる。 いやなにおいがする。 「ごみ捨て場が―・くにおう」「息が―・い」 
\\	疑わしいようすである。 怪しい。 「あのそぶりはどうも―・い」「―・い仲」 
\\	演劇などで、せりふの言い方や動作が大げさすぎてわざとらしい。 「―・い芝居」 ㊁〔接尾〕《形容詞型活用[文]くさ・し(ク活)》名詞またはそれに準じるものに付く。 
\\	…のようなにおいがする意を表す。 「汗―・い」「こげ―・い」 
\\	…のようなようすであるの意を表す。 「年寄り―・い」「インテリ―・い」 
\\	上にくる語の意を強める。 「けち―・い」「てれ―・い」 [派生]くさがる[動ガ五]くさげ[形動]くささ[名]くさみ[名] [下接語]青臭い・汗臭い・阿呆(あほ)臭い・磯(いそ)臭い・田舎臭い・陰気臭い・胡散(うさん)臭い・白粉(おしろい)臭い・男臭い・女臭い・金(かな)臭い・黴(かび)臭い・きな臭い・吝嗇(けち)臭い・焦げ臭い・酒臭い・七面倒臭い・邪魔臭い・洒落(しやら)臭い・熟柿(じゆくし)臭い・小便臭い・素人臭い・辛気(しんき)臭い・饐(す)え臭い・乳臭い・土臭い・照れ臭い・泥臭い・鈍(どん)臭い・糠味噌(ぬかみそ)臭い・鈍(のろ)臭い・馬鹿(ばか)臭い・バタ臭い・半可臭い・人臭い・日向(ひなた)臭い・貧乏臭い・古臭い・分別臭い・仏臭い・抹香臭い・水臭い・面倒臭い・野暮(やぼ)臭い(ぐさい)生(なま)臭い・寝臭い・物臭い	▲私は私の鼻で臭いを嗅ぐ。 ▲彼の言うことはくさい。
\\	鎖・鏈	くさり	《動詞「くさ(鏈)る」の連用形から》 
\\	金属製の輪を数多くつなぎ合わせて、ひもや綱のようにしたもの。 かなぐさり。 「犬を―でつなぐ」「懐中時計の―」 
\\	物と物とを結びつけているもの。 また、つなぎ合わせること。 絆(きずな)。 「因習の―を断ち切る」 
\\	「鎖帷子(かたびら)」の略。 
\\	関節。 また、「笛のくさり」の形で、のどぼとけの軟骨をいう。 「骨の―」「音骨(おとぼね)立てるな女めと、笛の―をぐっと刺す」	▲ジョージはその犬に鎖をつけた。 ▲囚人たちは鎖につながれている。
\\	腐る	くさる	㊀[動ラ五(四)] 
\\	細菌の作用で植物性・動物性のものが分解して変質する。 食物などがいたむ。 腐敗する。 「魚が―・る」 
\\	からだの組織が破れ崩れる。 うみただれる。 「凍傷で指先が―・る」 
\\	木・繊維・金属などが風化したり酸化したりしてぼろぼろになる。 朽ち崩れる。 腐敗・腐食する。 「柱が―・る」「さびて―・ったナイフ」 
\\	物が変質して、嫌なにおいがついたり汚れたりして使えなくなる。 「金魚鉢の水が―・る」 
\\	純な心が失われてだめになる。 精神が救いようがなく堕落する。 「性根が―・っている」 
\\	思いどおりに事が運ばないため、やる気をなくしてしまう。 いや気がさす。 めいる。 「気が―・る」「原稿が没になって―・っている」 
\\	(他の動詞の連用形に付いて)その動作をする人に対する軽蔑・ののしりの気持ちを表す。 「いばり―・る」「つまらんことを言い―・る」 
\\	賭(か)け事で負ける。 「夕べ胴が―・ってありたけ取られ」 
\\	すっかり濡れる。 びしょ濡れになる。 「―・った着物はしぼって引きさげ」 ㊁[動ラ下二]「くされる」の文語形。 [類語] 
\\	傷(いた)む・饐(す)える・?(あざ)れる・腐敗する・酸敗する・腐乱する/
\\	朽ちる・腐食する・腐朽する	▲暑いとミルクが腐る。 ▲暑いと牛乳はすぐに腐る。
\\	癖	くせ	《「曲(くせ)と同語源》 
\\	無意識に出てしまうような、偏った好みや傾向。 習慣化している、あまり好ましくない言行。 「爪をかむ―」「なくて七―」「怠け―(ぐせ)がつく」 
\\	習慣。 ならわし。 「早起きの―をつける」 
\\	一般的でない、そのもの特有の性質・傾向。 「―のある味」「―のある文章」 
\\	折れ曲がったりしわになったりしたまま、元に戻りにくくなっていること。 「髪の―をとる」「着物の畳み―(ぐせ)」→癖に →その癖 [下接語]足癖・髪癖・噛(か)み癖・酒癖・尻(しり)癖・其(そ)の癖・手癖・難癖・一癖・筆癖・読み癖(ぐせ)歌癖・着癖・口癖・抱き癖・出癖・寝癖・話し癖	▲ペリーは独り言を言う癖がついた。 ▲悪い癖はいったんつくとなかなか取れないものだ。
\\	具体	ぐたい	物事が、直接に知覚され認識されうる形や内容を備えていること。 「―案」↔抽象。	
\\	下り・降り	くだり	
\\	上から下へ、高いところから低いところへ移動すること。 また、その道や流れ。 「急な―が続く」「川―」↔上(のぼ)り。 
\\	鉄道の路線や道路で、各線区ごとの起点から終点への方向。 また、その方向に走る列車・バス。 ↔上(のぼ)り。 
\\	都から地方へ行くこと。 「東(あずま)―」「海道―」↔上(のぼ)り。 
\\	(土地の名に付けて接尾語的に用いて)遠く隔った場所の意を表す。 くんだり。 「わざわざ鎌倉―まで出掛けて」 
\\	時間が移ってある刻限の終わり近くになること。 また、その時。 「申(さる)の―になり候ひにたり」 
\\	《北に内裏があったところから》京都内で北から南に行くこと。 「大宮を―に二条を東へざざめいて引きければ」↔上(のぼ)り。 [下接語]東(あずま)下り・天(あま)下り・御(お)下り・オランダ下り・海道下り・川下り・京下り・上り下り・腹下り	▲ここの下りも手掛かりがないので危険だ。
\\	苦痛	くつう	からだや心に感じる苦しみや痛み。 「―に顔がゆがむ」「精神的―を与える」	▲その男は苦痛でわめいた。 ▲その薬は彼の苦痛を和らげた。
\\	ぐっすり	ぐっすり	[副] 
\\	深く眠っているさま。 熟睡するさま。 「―(と)眠っている」 
\\	物を突きさす音、また、そのさまを表す語。 「泥濘(ぬかるみ)へ―片足を踏み込み」 
\\	十分にするさま。 「雪を掻いて祝儀を貰ひ、晩には―暖まらう」	▲疲れた子供はぐっすり寝ている。 ▲彼女は学習参考書を読みながらぐっすり寝入ってしまった。
\\	区別	くべつ	[名]スルあるものと他のものとが違っていると判断して分けること。 また、その違い。 「善悪の―」「公私を―する」	▲我々は彼女と彼女の妹を区別できない。 ▲ボブは酷く取り乱していて、現実と虚構の区別がほとんど出来なかった。
\\	組み	くみ	
\\	一
\\	(名) 〔複合語を作る場合には「…ぐみ」と濁る〕 
\\	同じ類のいくつかのものが集まって一そろいになっているもの。 (ア)同じような形・用途・特徴などをもった一そろいのもの。 そろい。 セット。 「このコーヒー茶碗は六個で―になっています」(イ)同じような性格や特徴をもつ人の集まり。 「バスで行く―はこちらに集まれ」「学生時代に遊んでばかりいた―でね」 
\\	同じ教室で学習するように編成した,生徒の集まり。 クラス。 学級。 「―で一番背が高い」 
\\	いくつかのものを取り合わせて,一つにまとめて扱うこと。 また,そのもの。 つい。 「テキストとカセット-テープが―になっている」 
\\	原稿どおりに活字を並べて,印刷するための版を作ること。 また,そのようにした版。 組版。 「―がきれいな辞書」 
\\	同じ目的で行動をともにする人の集まり。 (ア)結社・団体・仲間などの構成単位。 それら組織の名の下に付けても用いる。 「―の若い者」「白柄(シラツカ)―」「新撰―」(イ)近世,幕府・大名の職名の中で,一定の職能をもった集団の名の下に付けて用いる。 「鉄砲―」「徒(カチ)―」(ウ)近世,地域社会で生活の必要上結ばれた組織。 ゆい組・祭組など。 (エ)近世,領主が民衆支配のためにつくった組織。 五人組・十人組など。 (オ)株仲間のこと。 
\\	「組糸(クミイト)」の略。 
\\	「組歌(クミウタ)」の略。 「古今―」 
\\	「組屋敷(クミヤシキ)」の略。 
\\	二
\\	(接尾) 助数詞。 ひとそろいあるいは一群となったものを数えるのに用いる。 「コーヒー-セット一―」「三―にわかれて頂上をめざす登山隊」	▲我々の組みは50人のおとこ生徒から成り立っている。 ▲もうひと組の観光客が到着した。
\\	組合	くみあい	
\\	(組み合い)互いに組みついて争うこと。 組み打ち。 「取っ―」 
\\	共通の目的のために何人かが寄り合って仲間を作ること。 また、その人々。 組。 「まさか一人じゃあるまい。 ―か」 
\\	民法上、二人以上の者が出資し合って共同の事業を営むことを約束する契約によって成立する団体。 法人となる資格がないもの。 
\\	特別法によって、特定の共同目的を果たすために、一定の資格のある者で組織することを認められている団体。 協同組合・共済組合など。 
\\	「労働組合」の略。労働者が労働条件の維持・改善および経済的社会的地位の向上をはかるために作る団体。企業別・産業別・職業別などの形態があり、単一組合と連合体との別がある。略称、労組。 「―運動」	▲組合は無期限ストに入った。 ▲組合は経営者側と賃金交渉をした。
\\	組む	くむ	[動マ五(四)] 
\\	㋐がっちりと、互いのからだに取りつき合う。 取り組む。 「四つに―・む」 ㋑同じ目的で何かをするために仲間になる。 組になる。 「彼と―・んで事業を始める」 
\\	㋐ものを互い違いに交差させたりからみ合わせたりする。 「腕を―・む」「足を―・む」 ㋑材料・部分を順序に従って合わせたり結んだりして、まとまりのある全体を作り上げる。 「ひもを―・む」「足場を―・む」 ㋒統一あるものにまとめ上げる。 組織したり編成したりする。 「徒党を―・む」「時間割を―・む」 ㋓活字を、指定に従って原稿どおりに並べる。 「版を―・む」 [可能]くめる [下接句]鬼とも組む・座を組む・手を組む・徒党を組む・膝(ひざ)を組む・四つに組む	▲その男は腕を組んだりほどいたりしていた。 ▲その老人は脚を組んでそこに座っていた。
\\	位	くらい	《「座(くら)」に「居る」意から》 
\\	定められた序列の中での位置。 地位。 ㋐皇帝・国王などの地位。 皇位。 王位。 帝位。 「―に即(つ)く」 「―を譲る」 ㋑官職などにおける身分の段階。 等級。 「三位(さんみ)の―」→位階 
\\	地位・身分の上下関係。 階級。 
\\	出来のよしあし、品格などからみた、優劣の段階。 ㋐物事の等級。 ㋑連歌・俳諧・能楽などで、作品や所作の品位。 
\\	十進法での数の段階。 また、その位置の名。 「十の位」「百の位」などという。 表は位の名の一例であるが、恒河沙から無量大数までを八桁とびにする説もある。 
\\	将棋で、敵陣を制圧する位置。 特に、盤面の中央をいう。 
\\	芸道などで、実力の程度。 到達した境地。 芸位。 「我が―のほどを能々(よくよく)心得ぬれば」	▲ハワイ旅行は200ドルくらいかかるでしょう。 ▲恥をかくぐらいなら死んだほうがましだ。
\\	暮らし	くらし	
\\	暮らすこと。 一日一日を過ごしていくこと。 「都会での―に慣れる」 
\\	日々の生活。 生計。 「豊かな―」「―の足しにする」	▲金持ちだが貧しい暮らしをしている。 ▲君はこのまま収入不相応な暮らしを続ければ金に困って身動きがとれなくなるだろう。
\\	クラシック	クラシック	㊀[名] 
\\	古代ギリシア‐ローマ時代の文学や美術などの古典作品。 
\\	文学・芸術において、時代を超えて認められる名作。 古典。 
\\	《「クラシック音楽」の略》ジャズ・流行歌などのポピュラーミュージックに対して、西洋の伝統的な芸術音楽。 
\\	「クラシックレース」の略。 ㊁[形動] 
\\	古典的。 古典主義的。 模範的。 「―な作風」 
\\	古風なさま。 古雅なさま。 クラシカル。 「―な髪型」「―な雰囲気」	▲私はあらゆる音楽が好きですが、クラシックが一番好きです。 ▲私はクラシックが好きです。
\\	暮らす	くらす	[動サ五(四)] 
\\	日が暮れるまで時間を過ごす。 時を過ごす。 「一日を読書で―・す」 
\\	日々を送る。 月日を過ごす。 「余生は郷里で―・したい」 
\\	生活する。 また、生計を立てる。 「少ない収入でなんとか―・している」 
\\	(他の動詞の連用形に付いて)一日中その事をし続ける意を表す。 「遊び―・す」「泣き―・す」 [可能]くらせる	▲彼女のお母さんは田舎でまったく一人ぼっちで暮らしている。 ▲ある朝、食事のときに、私たち子どもは、もうこれからはしたい放題のことをして暮らすことは許されなくなると知らされて、すっかり落胆した。
\\	グラス	グラス	
\\	ガラス製のコップ。 
\\	ガラス。 「ステンド―」 
\\	眼鏡。 「サン―」「オペラ―」	▲グラスをいっぱいにしなさい。 ▲グラスを割った。
\\	グランド	グランド	〔語素〕他の外来語の上に付いて、大きな、りっぱな、などの意を表す。 「―ホール」「―セール」	▲何百人もの少年達がグランドで遊んでいる。 ▲雨のため、少年たちはグランドで野球ができなかった。
\\	クリーム	クリーム	
\\	牛乳からとれる黄白色のどろっとした脂肪質。 成分は乳脂肪・水分・たんぱく質・乳糖などで、乳脂肪分一八・〇パーセント以上のもの。 バター・菓子の製造や料理・コーヒーなどに用いる。 乳脂。 生クリーム。 クレーム。 
\\	カスタードクリーム。 
\\	凝乳状の基礎化粧料。 油性と水性の二タイプがある。 皮膚の保護、髪の手入れなどに用いる。 「ヘア―」 
\\	靴墨。 靴クリーム。 
\\	クリーム色。 
\\	「アイスクリーム」の略。牛乳・砂糖・卵黄に香料を加えて凍らせた氷菓子。	▲砂糖とクリームを少し入れてください。 ▲サブロンは肌に潤いを与えるクリームです。
\\	繰り返す	くりかえす	[動サ五(四)] 
\\	同じことをもう一度、あるいは何度もする。 反復する。 「失敗を―・す」「歴史は―・す」 
\\	本のページなどをめくる。 「戸棚から暦を出して―・して見ると」	▲彼女は自分の名前をゆっくりと繰り返した。 ▲彼女は何度も自分は無実だと繰り返した。
\\	クリスマス	クリスマス	イエス=キリストの誕生を祝う祭り。 一二月二五日に行われる。 多くの民族の間にみられた、太陽の再生を祝う冬至の祭りと融合したものといわれる。 聖誕祭。 降誕祭。 《季 冬》「ほんものの樅(もみ)は嵐や―/正雄」	▲クリスマスがじきにまためぐってくる。 ▲クリスマスが近くなってきた。
\\	狂う	くるう	[動ワ五(ハ四)] 
\\	精神の正常な調和がとれなくなる。 気が違う。 気がふれる。 「気が―・う」「―・ったようにわめく」 
\\	物事・機械の働きや状態が正常でなくなる。 「時計が―・う」「音程の―・った歌声」「歯車が―・う」 
\\	ねらい・見込みなどが外れる。 予測・計画通りにならない。 「手元が―・う」「見通しが―・う」 
\\	物事に異常に熱中して見さかいがつかなくなる。 おぼれる。 「かけ事に―・う」「女に―・う」 
\\	(他の動詞の下に付いて)普通の程度を越えて激しく動き回る。 ひどく…する。 「踊り―・う」「荒れ―・う」 
\\	神霊・もののけが取りついて、普通ではない行動をする。 神がかりになる。 「こは物に―・はせ給ふか」 
\\	激しく動き回ったり、舞い踊ったりする。 「ひとへに死なんとぞ―・ひける」 
\\	ふざける。 じゃれつく。 「あれ御亭(ごて)さん、―・ひなんすな」	▲そんなことを言うとは気でも狂ったのか。 ▲そんなことを言うなんて気でも狂ったのか。
\\	グループ	グループ	
\\	仲間。 集団。 「―旅行」 
\\	共通の性質で分類した、人や物の一団。 群。 
\\	同系列に属する組織。 「企業―」	▲囲い込みに関する要点は、エスニック・グループの構成を明確に記述しなければならないということである。 ▲警察はそのグループの動静を常に監視する。
\\	苦しい	くるしい	[形][文]くる・し[シク] 
\\	痛みや圧迫感で、肉体的にがまんができない。 「激しいせきこみで息が―・い」「満員の乗客に押されて胸が―・い」 
\\	悩み・せつなさ・悲しさ・後悔などで、心が痛んでつらい。 「―・い胸のうち」「―・い試練に耐える」 
\\	物や金銭のやりくりが思うようにならない。 「―・い家計」 
\\	無理を承知で、ある事をするさま。 こじつけるさま。 「ずいぶん―・い言い訳だ」 
\\	(多く否定の形で)差し支えがある。 都合が悪い。 「切捨てても―・くない奴だ」 
\\	(多く「ぐるしい」の形で、動詞の連用形に付いて)快くない、しにくいなどの意を表す。 「見―・い振る舞い」「聞き―・い話」 
\\	不愉快になりおもしろくない。 見ぐるしい。 聞きぐるしい。 「前栽の草木まで心のままならず作りなせるは、見る目も―・く」 [派生]くるしがる[動ラ五]くるしげ[形動]くるしさ[名] [用法]くるしい・つらい――「せきがひどく、息をするのも苦しい(つらい)」「仕事が苦しい(つらい)」のように相通じて用いられる。 
\\	「くるしい」は一般的な苦痛の状況を、「つらい」は精神的苦痛について用いることが多い。 「苦しい立場」は動くに動けない状況を、「つらい立場」は困っている精神状況を表すのに重点がある。 「家計が苦しい」とはいうが、「家計がつらい」とは普通は言わない。 
\\	類似の語に「せつない」がある。 「せつない」は、悲しさや恋しさなどのために、胸がしめつけられるような思いを表し、「せつないほどの愛情」「せつない胸の内」のように用いる。 
\\	「病に苦しむ人の姿を見るのはつらい」は「苦しい」とはいわないが、「せつない」と置き換えることはできる。	▲踊り子の優雅さは懸命な練習、汗と苦しさから生まれるのです。 ▲良薬は口に苦し。
\\	苦しむ	くるしむ	㊀[動マ五(四)] 
\\	からだに痛みや苦しみを感じる。 「病気に―・む」 
\\	心につらく思う。 思い悩む。 「恋に―・む」「貧乏で生活に―・む」 
\\	思うような処理方法が見つからず困る。 窮する。 「理解に―・む」「判断に―・む」 
\\	力を尽くして物事を行う。 骨折る。 苦労する。 「―・んだ甲斐(かい)がある」 ㊁[動マ下二]「くるしめる」の文語形。	▲彼らはスモッグで苦しんだ。 ▲彼は頭痛で苦しんでいる。
\\	暮れ	くれ	
\\	太陽が沈むころ。 夕暮れ。 また、日の暮れること。 「日の―が早まる」↔明け。 
\\	ある期間、特に季節の終わり。 「秋の―」 
\\	年の終わり。 年末。 歳末。 「―も押し詰まりまして」《季 冬》	
\\	苦労	くろう	[名]スル 
\\	精神的、肉体的に力を尽くし、苦しい思いをすること。 「―が絶えない」「―を共にする」「―の種」「―して育てた子供」 
\\	(多く「ごくろう」の形で)人に世話をかけたり、厄介になったりすること。 「ご―をかける」「ご―さま」→御苦労(ごくろう) [類語]
\\	骨折り・労(ろう)・労苦・苦心・腐心・辛苦・辛労・心労・煩労・艱苦(かんく)・艱難(かんなん)・苦難・辛酸(―する)難儀する・骨折る・てこずる・労する・心を砕く	▲君のマンションを探すのには苦労したよ。 ▲君の父さんは君のためにあんなにも苦労したのだ。
\\	加える	くわえる	[動ア下一][文]くは・ふ[ハ下二] 
\\	今まであるものに、さらに他のものを添えて合わせる。 現在あるものの上に付け足す。 また、そのようにして数量や度合いを増す。 「だし汁を―・える」「規約に一項を―・える」「列車が速度を―・える」 
\\	同じことをする人の集まりに含める。 仲間に入れる。 「一行に―・える」「役員に―・える」 
\\	ある作用を他におよぼす。 影響を与える。 「危害を―・える」「一撃を―・える」「手心を―・える」「説明を―・える」 
\\	あるものを付ける。 載せる。 「それに判を―・へよ」 ◆室町時代以降はヤ行にも活用した。 →加ゆ [用法]くわえる・そえる――「もう一品加える(添える)」「言葉を加える(添える)」などでは相通じて用いる。 
\\	「加える」はあるものに他のものを入れて一つにしたり、何らかの作用を他に与えたりすること。 「2に3を加える」「仲間に加える」「攻撃を加える」などと用いる。 
\\	「添える」はすでに満たされ、完成しているものに、さらに何かを付け加える意。 「贈り物に手紙を添える」「舞台に花を添える」
\\	類似の語に「足す」がある。 「足す」は、必要量が満たされるように足りないものを補い加える意。 「煮物が焦げないように水を足す」などと用いる。	▲彼は必要な変更を加える。 ▲分子Dの運動の観察に基づき、可能な出口が3つ存在するようシステムに変更を加える。
\\	詳しい・委しい・精しい	くわしい	[形][文]くは・し[シク]《「くわ(細)し」と同語源》 
\\	細かいところまで注意や調査などがよく行き渡っている。 詳細である。 「―・い地図」「事情を―・く説明する」 
\\	細部までよく知っている。 精通している。 「魚には特に―・い人」「事情を―・く知っている男」↔疎い。 [派生]くわしげ[形動]くわしさ[名] [類語]
\\	細かい・詳細・詳密・精細・明細・克明・つまびらか・事細か(連用修飾語として)子細に・具(つぶさ)に・逐一・細大漏らさず/
\\	明るい・造詣(ぞうけい)が深い・精通している・通暁している	▲彼はコンピューターに詳しい。 ▲彼はその事故のことをくわしく述べた。
\\	加わる	くわわる	[動ラ五(四)] 
\\	あるものに、さらに他のものが添えられてその数・量・程度が増す。 「会員が新しく―・る」「要素が―・る」「貫禄が―・る」 
\\	ある事に参加する。 仲間に入る。 「一行に―・る」「話に―・る」 
\\	度合いが強くなる。 その程度や状態が増す。 「暑さが―・る」「スピードが―・る」 
\\	ある作用が他に及ぶ。 行き渡る。 「圧力が―・る」 [可能]くわわれる	▲彼が新しく教授陣に加わった人です。 ▲彼が仲間に加わるのは当然だと思ってました。
\\	訓	くん	漢字の意味に基づいて、それに当てた日本語による読み。 「山」を「やま」「川」を「かわ」と読む類。 和訓。 ↔音(おん)。	
\\	軍	ぐん	
\\	軍隊。 「―を率いる」 
\\	陸軍・海軍・空軍の総称。 
\\	数個軍団または師団によって構成される軍隊の編制単位。 「方面―」	▲我が軍はその王国を不意打ちにした。 ▲我が軍は絶えずゲリラの攻撃を受けた。
\\	軍隊	ぐんたい	一定の秩序をもって編制された軍人の集団。	▲軍隊は城を何日間も包囲した。 ▲軍隊は国家の中の国家である、現代の諸悪のひとつである。
\\	訓練	くんれん	[名]スル 
\\	あることを教え、継続的に練習させ、体得させること。 「きびしい―にたえる」「―して生徒を鍛える」 
\\	能力・技能を体得させるための組織的な教育活動のこと。 「職業―」 [用法]訓練・練習――「訓練(練習)を積む」などでは相通じて用いられる。 
\\	「訓練」はある技術について教え込み、身につけさせることだが、「練習」は自らが繰り返したり、工夫したりして技術の向上をはかることをいう。 「教官は練習生に対して訓練を開始した」「よく訓練された盲導犬」「短距離走のスタートの練習をしている」などの「訓練」「練習」は互いに置き換えることはできない。 
\\	類似の語に「稽古(けいこ)」がある。 「稽古する」は「練習する」の意、「稽古をつける」は「訓練する」の意で、主に武術、芸事などに関して、「寒稽古に励む」「お茶の稽古に通う」などと用いる。	▲少し訓練すれば、その仕事は楽になるだろう。 ▲生涯教育は絶え間ない再訓練を意味する。
\\	下	げ	
\\	程度・価値・等級・序列などが低いこと。 標準より劣っていること。 下等。 した。 「中の―の成績」↔上(じよう)。 
\\	書物や文の章段などで、二つまたは三つに分けたものの最後のもの。 「―の巻」↔上(じよう)。	
\\	敬意	けいい	尊敬する気持ち。 「―を表する」「―を払う」「―をこめる」	▲来訪中の作家に敬意を表してパーティーが開かれた。 ▲彼らはその著名な科学者に敬意を表して宴会を催した。
\\	経営	けいえい	[名]スル 
\\	事業目的を達成するために、継続的・計画的に意思決定を行って実行に移し、事業を管理・遂行すること。 また、そのための組織体。 「会社を―する」 
\\	政治や公的な行事などについて、その運営を計画し実行すること。 「国家の―」 
\\	測量して、建物をつくること。 「十兵衛が辛苦―むなしからで、感応寺生雲塔いよいよ物の見事に出来上り」 
\\	物事の準備や人の接待などにつとめはげむこと。 けいめい。 「湯を沸(わ)かすやら、粥(かゆ)を煮るやら、いろいろ―してくれたそうでございます」 
\\	急ぎあわてること。 けいめい。 「早朝告げあり。 ―参入す」	▲彼の母親は15年間薬局を経営している。 ▲彼のお母さんは15年間前から薬局を経営している。
\\	景気	けいき	
\\	売買や取引などに現れる経済活動の状況。 特に、経済活動が活気を帯びていること。 好景気。 「―が上向く」「―が回復する」「―のいい店」 
\\	活気があること。 威勢がよいこと。 「一杯飲んで―をつける」「―よく太鼓を打ち鳴らす」 
\\	人気。 評判。 「牛肉も伝染病のふうぶんで大きに―を落としましたが」 
\\	物事のようす。 ありさま。 また、情景。 景色。 「幾分厳(いか)めしい―を夜陰に添えたまでで」 
\\	和歌・連歌・俳諧で、景色や情景のおもしろさを主として詠むこと。 景曲。 「―の句」 [類語]
\\	景況・市況・商況・商状・気配・売れ行き・金回り/
\\	元気・活気・意気・威勢・気勢	▲商売の景気が向上している。 ▲私の考えでは、景気は上向きになります。
\\	傾向	けいこう	
\\	物事の大勢や態度が特定の方向にかたむくこと、または、かたむきがちであること。 「最近の消費者の―」「彼は大げさに言う―がある」 
\\	思想的にある特定の方向にかたよること。 特に、左翼的思想にかたよること。 「―小説」 
\\	心理学で、一定の刺激に対して、一定の反応を示す生活体の素質。 [類語]
\\	傾き・気味・性向(全体の動きや流れ)趨勢(すうせい)・趨向(すうこう)・動向・流れ・大勢(たいせい)・トレンド	▲経営陣は収益の短期的改善に気を取られすぎて、長期的な将来計画に気が回らない傾向があった。 ▲傾向を発見するため、1950年から1970年までのXYZ年鑑を調べた。
\\	警告	けいこく	[名]スル 
\\	よくない事態が生じそうなので気をつけるよう、告げ知らせること。 「再三の―を無視する」「事前に―する」 
\\	柔道の反則で、「技あり」と同等となるもの。 禁止事項を犯したり、「注意」が二度目の場合に主審が宣告する。	▲彼の警告を聞かないなんてなんて不注意だったんだ。 ▲彼は、我々の警告を無視した。
\\	計算	けいさん	[名]スル 
\\	物の数量をはかり数えること。 勘定。 「―が合う」 
\\	加減乗除など、数式に従って処理し数値を引き出すこと。 演算。 「損失額はざっと―しても一億円」 
\\	結果や成り行きをある程度予測し、それを予定の一部に入れて考えること。 「多少の失敗は―に入れてある」「―された演技」「―外」 [類語]
\\	勘定・算用・計数/
\\	運算・演算・加減乗除・算術 (計算して数字を出すこと)算出・算定・概算・試算・見積もり/
\\	予測・見込み・読み・勘定・目算・成算・胸算用・打算(―する)算盤(そろばん)を弾く	▲今度は指をポケットの中に入れて計算してみよう。 ▲結果、計算ミスが多い。
\\	刑事	けいじ	
\\	刑法の適用を受け、それによって処理される事柄。 「―責任を問われる」↔民事。 
\\	犯罪の捜査を主任務とする警察官の通称。 私服で勤務することが多く、階級では巡査または巡査部長。	▲男は刑事に銃を向けた。 ▲刑事は彼の腕をつかんだ。
\\	掲示	けいじ	[名]スル人に伝えるべき事柄を、紙に書くなどしてかかげ示すこと。 また、その文書など。 「合格者の氏名を―する」	▲掲示の前に何人かの人が立っていました。 ▲掲示には「芝生に入らないでください」と書いてあります。
\\	芸術	げいじゅつ	
\\	特定の材料・様式などによって美を追求・表現しようとする人間の活動。 および、その所産。 絵画・彫刻・建築などの空間芸術、音楽・文学などの時間芸術、演劇・映画・舞踊・オペラなどの総合芸術など。 「―の秋」「―品」 
\\	学芸と技術。	▲専門の図書館が芸術に関する文献を収集している。 ▲誰でも多かれ少なかれ芸術に関心を抱いている。
\\	契約	けいやく	[名]スル 
\\	二人以上の当事者の意思表示の合致によって成立する法律行為。 売買・交換・贈与・貸借・雇用・請負・委任・寄託など。 「―を結ぶ」「三年間の貸借を―する」→単独行為 →合同行為 
\\	約束を取り交わすこと。 また、その約束。 「日来の―をたがへず、まゐりたるこそ神妙なれ」 
\\	ユダヤ教・キリスト教に特徴的な思想で、救いの恩恵に関して神と人間との間で交わされた約束。 モーセを仲介者としてイスラエル民族に与えられたものを旧約、キリストの十字架上の犠牲を通じてなされたものを新約という。	▲やむを得ず契約に署名させられた。 ▲むりやりサインをさせられたのなら、その契約は無効です。
\\	経由	けいゆ	[名]スル 
\\	目的地へ行く途中、ある地点を通ること。 けいゆう。 「京都―で奈良へ行く」 
\\	物事が中間のある機関を経ること。 けいゆう。 「部長を―して提案する」	▲アメリカ先住民の祖先は、ベーリング海峡を経由してアジアから大陸へ渡った。 ▲シカゴ経由でニューヨークからセントルイスへ飛んだ。
\\	ケース	ケース	
\\	容器。 入れ物。 「タバコの―」「スーツ―」 
\\	個々の事例。 場合。 「特殊な―」 
\\	文法用語で、格のこと。	▲そのケースは冷静に対処する必要がある。 ▲そのケースを運ばせてください。
\\	ゲーム	ゲーム	
\\	遊びごと。 遊戯。 「―コーナー」 
\\	競技。 試合。 勝負。 「白熱した―」 
\\	テニスで、セットを構成する一試合。 「先に二―とったほうが勝ち」 
\\	「ゲームセット」の略。球技の試合で、勝負がつくこと。試合終了。	▲ゲームについてのご感想は? ▲ゲームで一段と盛り上がった。
\\	劇	げき	脚本中の役を動作とせりふで演じながら筋書きに従って場面を進行させていくもの。 演劇。 芝居。 ドラマ。 「人形―」	▲劇はちょうど定刻に始まった。 ▲その劇で彼女は女中と女店員との二役を努めた。
\\	劇場	げきじょう	演劇・映画・舞踊などを観客に見せるための建物。 「円形―」「国立―」	▲8月の蒸し暑い夜に新しい劇場がオープンしました。 ▲あの劇場ではPを上演している。
\\	化粧・仮粧	けしょう	[名]スル 
\\	紅(べに)やおしろいなどを使って、顔を美しく見えるようにすること。 つくり。 けそう。 「念入りに―する」 
\\	物の表面を美しく飾ること。 「壁を白いペンキで―する」「雪―」 
\\	うわべだけのこと。 虚飾。 「差いた刀は、―か、伊達(だて)か」 [類語]
\\	作り・お作り・美容・粉黛(ふんたい)・脂粉・粉粧・メーキャップ・メーク/
\\	装飾・デコレーション	▲彼女は化粧が濃い。 ▲彼女は少しお化粧をすればもっと美しく見えるだろうに。
\\	けち	けち	[名・形動] 
\\	(「吝嗇」とも書く)むやみに金品を惜しむこと。 また、そういう人や、そのさま。 吝嗇(りんしよく)。 「何事につけても―な男だ」 
\\	粗末なこと。 価値がないこと。 また、そのさま。 貧弱。 「―な賞品をもらった」 
\\	気持ちや考えが卑しいこと。 心が狭いこと。 また、そのさま。 「―な振る舞いをするな」「―な料簡」「―な根性」 
\\	縁起の悪いこと。 不吉なこと。 また、難癖(なんくせ)。 
\\	景気が悪いこと。 また、そのさま。 不景気。 「あんまり―な此の時節」 [派生]けちさ[名] [類語]
\\	吝嗇・しみったれ・しわい・渋い・しょっぱい・細かい・みみっちい(けちな人)けちん坊・しわん坊・握り屋・締まり屋・吝嗇漢(りんしよくかん)・守銭奴(しゆせんど)・倹約家・始末屋/
\\	ちゃち・ちっぽけ・安手(やすで)・安っぽい・つまらない・くだらない・取るに足りない/
\\	みみっちい・いじましい・せせこましい・狡辛(こすから)い・さもしい・卑しい・せこい・陋劣(ろうれつ)・低劣・卑怯(ひきよう)・姑息(こそく)・狭量・小量・けつの穴が小さい	
\\	結果	けっか	[名]スル 
\\	ある原因や行為から生じた、結末や状態。 また、そのような状態が生じること。 「よい―をもたらす」「―した肺尖カタルや神経衰弱がいけないのではない」↔原因。 
\\	副詞的に用いて、ある事態の生じるもととなる結末状態を表す。 「猛勉強をした―、合格した」 
\\	植物が実を結ぶこと。 結実。 「例年より―する時期が遅い」 [類語]
\\	結末・帰結・帰趨(きすう)・首尾・成り行き・仕儀(しぎ) (よい結果)成果/
\\	末(すえ)・あげく・果て	▲試験の結果が発表されたら知らせてあげよう。 ▲試験の結果なんて怖くて聞けないよ。
\\	欠陥	けっかん	欠けて足りないこと。 不備な点。 「論理上の―を衝(つ)く」「―商品」	▲初期のジェット旅客機の墜落事故は機体とエンジンの金属疲労のような技術的欠陥が原因で起こることが多かった。 ▲世界の旅客機の半数以上を製造しているボーイング社は、もっともなことだが、機体の他に欠陥を起こす可能性のあるものに、注意を引こうと躍起になっている。
\\	結局	けっきょく	㊀[名] 
\\	囲碁で、一局を打ちおえること。 終局。 
\\	いろいろの経過を経て落ち着いた最後。 結末。 「話は随分長かったが、要するに覚束(おぼつか)ない―に陥ったのである」 ㊁[副]いろいろなことがあったうえで、最後に落ち着くさまを表す。 最終的には。 つまるところ。 結句。 「あれこれやってみたが、―だめだった」「二人は―元のさやにおさまった」 [類語] ❷結句・遂(つい)に・畢竟(ひつきよう)・とどの詰まり・詰まるところ・帰するところ・詮(せん)ずるところ・要するに・どの道・いずれ・所詮(しよせん)・どうせ	▲結局すべてが旨く行った。 ▲結局その計画は失敗だった。
\\	決心	けっしん	[名]スル心を決めること。 考えを決めること。 「やっと―がつく」「こうと―した以上は必ずやる」 [用法]決心・決意――「すべてを忘れて再出発することを決心(決意)した」「社長はA社との合併を決心(決意)した」のように、心(気持ち)を定めるの意では相通じて用いられる。 
\\	「決心」は考えを決めるというところに重点があり、「決意」は意志を固めるというところに重点がある。 したがって「なかなか決心がつかない」場合に「決意」を用いることを普通はしない。 
\\	「今年中に結婚する決心をした」「優勝を目指してがんばる決意です」の場合、「決意」と「決心」は交換できないこともないが、それぞれを置き換えることによって、表現される内容に原文と微妙な違いを生じる。	▲ついにおとうさんから聞いたひどい話を、おっかさんに伝える決心をした。 ▲どうかきっぱりと決心してください。
\\	欠席・闕席	けっせき	[名]スル出席すべき会合などに出ないこと。 また、生徒や学生が学校を休むこと。 「同窓会に―する」「授業を―する」↔出席。	▲きょうの英語の授業には欠席します。 ▲ケイトは会合を欠席した。
\\	決定	けってい	[名]スル 
\\	物事をはっきりと決めること。 物事がはっきりと決まること。 また、その内容。 「会議の日取りを―する」「―権」 
\\	裁判所が行う判決以外の裁判。 口頭弁論を経ない点で判決と異なり、個々の裁判官がなす命令と区別される。 「公訴棄却の―」 [類語]
\\	決まり・本決まり・確定・画定(評議による決定)議決・決議・論決・評決・議定・取り決め(ある事柄について下す決定)断(だん)・断案・決(けつ)・裁決・裁定	▲決定を延ばさせてください。 ▲決定を下すのは彼の権限だ。
\\	欠点	けってん	
\\	不十分なところ。 足りないところ。 短所。 あら。 「―を補う」「―をつく」 
\\	学校の成績で、必要な点数に足りないこと。 落第点。 [用法]欠点・弱点――「燃料を食うのが、この車の欠点(弱点)だ」「相手の欠点(弱点)をつく」「欠点(弱点)を補う」など、相通じて使われることが多い。 
\\	「欠点」は難ずるべきところ、改めるべきところという意味合いが強く、「弱点」は十分でない弱い部分という意味合いで用いられる。 「他人に対する思いやりに欠けるのが彼の欠点だ」を「弱点」で置き換えることには無理がある。 「彼の弱点をつかむ」を「欠点」に置き換えることはできない。 
\\	類似の語に「短所」がある。 「短所」は、「気が短く怒りっぽいのが彼の短所だ」のように「欠点」と相通じて用いられるが、性格的なことについて用いられることが多い。	▲欠点はあるけれどもやはり彼が好きです。 ▲欠点ゆえにそれだけいっそう彼が好きだ。
\\	月曜	げつよう	週の第二日。 日曜の次の日。 月曜日。	▲月曜から木曜までここにおります。 ▲月曜までにこの仕事を終えてください。
\\	結論	けつろん	[名]スル 
\\	考えたり論じたりして最終的な判断をまとめること。 また、その内容。 「調査の―を出す」 
\\	論理学で、推論において前提から導き出された判断。 終結。 断案。 ↔前提。	▲論文は結論を残し、あとは仕上がっている。 ▲そしてもう一つは疑わしい結論を出した。
\\	煙・烟	けむり	《「けぶり」の音変化》 
\\	物が燃えるときに立ちのぼるもの。 有機物が不完全燃焼するときに出る気体で、固体の微粒子が浮遊している状態をいうが、液体の微粒子が含まれている場合もある。 「―がたちこめる」 
\\	霞(かすみ)・水蒸気など、 
\\	のように空中にたちこめるもの。 「湯の―がただよう町」 
\\	かまどから立ちのぼるもの。 炊煙。 また、暮らし。 生計。 「細いながら―絶えせず安らかに日は送れど」 [下接語]黒煙・潮煙・砂煙・血煙・朝夕(ちようせき)の煙・土煙・野辺の煙・火煙・水煙・夕煙・雪煙・湯煙	▲煙が風に吹き流された。 ▲煙が空へ昇っている。
\\	券	けん	
\\	入場券・乗車券・食券など、特定の資格や条件などを表示した紙片。 切符。 チケット。 「映画の―」「パーティー―」 
\\	荘園・田地・邸宅などの所有を証明する手形。 割り符。 「家の―奉り給へり」	▲建雄はポケットに手を入れて券を探した。 ▲券の手配しとくよ。
\\	県	けん	
\\	都・道・府とともに、市町村を包括する広域の地方公共団体。 議決機関として議会、執行機関として知事・教育委員会・公安委員会などを置き、条例の制定、地方税の賦課・徴収などの権能をもつ。 現在、四三県。 
\\	明治初年、藩に対して、朝廷の直轄地の称。 
\\	中国の地方行政区画の一。 周代に郡と並んで設置されたが、秦代に郡県制の実施により郡の下に置かれ、以後、州・府・道・省などに属した。 現在は省の下に位置する。	▲高松塚古墳は、奈良県明日香村に存在する古墳。 ▲日本には、県がいくつありますか。
\\	見解	けんかい	物事に対する考え方や価値判断。 「―の相違」「―を明らかにする」	▲綿密に言うと、彼の見解は私のとはいくらか異なる。 ▲本当はその声明は彼の個人的見解にすぎない。
\\	限界	げんかい	物事の、これ以上あるいはこれより外には出られないというぎりぎりの範囲、境。 限り。 「広葉樹分布の北の―」「能力の―を知る」「体力の―に挑戦する」 [用法]限界・限度――「疲労が限界(限度)に達している」「限界(限度)を超える」などでは、相通じて用いられる。 
\\	「限界」は、それ以上進めなくなるところという意が強く、「限度」は、あらかじめそこまでと限られたところという意が強い。 
\\	「体力(能力)の限界を感じた」は「限度」に置き換えられないし、「有給休暇は二週間を限度とする」は「限界」に置き換えられない。 
\\	類似の語に「極限」がある。 「極限」はぎりぎりのところという意が強く、「能力の極限に挑む」などのほか、「極限状態」のような熟語も生む。	▲もう我慢の限界だ! ▲限界まで泳ぎ続けろ。
\\	現金	げんきん	㊀[名] 
\\	手持ちのかね。 その場で受け渡しをすることのできる金銭。 また、金銭をその場で受け渡しすること。 キャッシュ。 「―で支払う」「―の持ち合わせがない」 
\\	通用の貨幣。 小切手・手形・為替などに対していう。 キャッシュ。 通貨。 「―に換える」 
\\	簿記上、通貨およびいつでも通貨に換えられる小切手・送金為替手形・郵便為替証書など。 
\\	まとまった額の金銭。 げんなま。 「―が渡る」「―を積まれる」 ㊁[形動]目先の利害損得によってすぐ態度や主張を変えるさま。 「―なやつだ」 [派生]げんきんさ[名]	▲銀行に行けばその50ドルの小切手を現金に換えてくれる。 ▲経済学者の中には、主婦の労働は、現金に換算すると、GNPの約25パーセントにのぼると見積もる人もいる。
\\	言語	げんご	音声や文字によって、人の意志・思想・感情などの情報を表現・伝達する、または受け入れ、理解するための約束・規則。 また、その記号の体系。 音声を媒介とするものを音声言語(話し言葉)、文字を媒介とするものを文字言語(書き言葉)、コンピューターなど機械を媒介とするものを機械言語・アセンブリ言語などという。 ことば。 ごんご。 げんぎょ。	▲しかし、言語の場合は、私達が遺伝的に受け継ぐのは、話したり、理解したりする能力だけである。私達が話す特定の1つないし複数の言語は、遺伝ではなく、文化的な伝達によって私達に伝えられるのである。 ▲しかしながら、伝達の手段として言語を持っているのは、人間だけである。
\\	健康	けんこう	[名・形動] 
\\	異状があるかないかという面からみた、からだの状態。 「―がすぐれない」「―優良児」 
\\	からだに悪いところがなく、丈夫なこと。 また、そのさま。 「―を保つ」「―な肉体」 
\\	精神の働きやものの考え方が正常なこと。 また、そのさま。 健全。 「―な考え方」「―な笑い」 [派生]けんこうさ[名] [類語]
\\	体調・具合(ぐあい)・塩梅(あんばい)/
\\	無病息災・無事・健勝・清勝(形動用法で)健(すこ)やか・壮健・健全・丈夫(じようぶ)・達者・元気・まめ・つつがない	▲健康を害してはじめてその価値がわかるものだ。 ▲健康を維持するためには散歩するのが一番良い。
\\	検査	けんさ	[名]スルある基準をもとに、異状の有無、適不適などを調べること。 「所持品を―する」「適性―」 [類語]点検・吟味・検閲・検分・臨検・巡検・検定・監査・検診・チェック (―する)検する・調べる・あらためる	▲検査に手落ちがあったのではないだろうか。 ▲検査の結果が出るまで待ってください。
\\	現在	げんざい	[名]スル 
\\	過去と未来の間。 過去から未来へと移り行く、今。 また、近い過去や未来を含む、今。 副詞的にも用いる。 「数千年の時を経て―に至る」「―のところ見通しは立っていない」「―、出張中です」 
\\	(時間を表す語の下に付き、接尾語的に用いて)変化する物事の状態をある時点で区切って示すときの、その時点。 「八月末日―の応募者数」 
\\	現に存在すること。 目の前にあること。 「―する最重要課題」 
\\	(「現在の」の形で血縁関係などを表す語を修飾して)正真正銘の。 まぎれもない。 「いはんや彼らは―の孫なり。 しかも嫡孫なり」 
\\	(副詞的に用いて)明白な事実であるさま。 また、近い将来そのことが実現するのが確実であるさま。 「そなたは―奥様になることぢゃ」 
\\	仏語。 三世(さんぜ)の一。 今、生をうけているこの世。 現世。 
\\	文法の時制の一。 今の時点での動作・状態など表すときに用いる形。 現在形。 [類語]
\\	今(いま)・只今(ただいま)・目下(もつか)・刻下・現下・現時点・現時・今日(こんにち)・方今・当今・現今	▲以前はよく釣りにいったのですが現在はめったに行きません。 ▲育児休暇や老人介護のための休みも、現在日本で起きている人口構成の変動に対応するために必要となっているのである。
\\	現実	げんじつ	いま目の前に事実として現れている事柄や状態。 「夢と―」「―を直視する」「―に起きてしまった事故」↔理想。 [類語]実際・実地・実情・実態・実相・現状・事実・実在・実(まこと)・現(うつつ)	▲現実を直視すべきだ。 ▲現実を直視するべきだ。
\\	現象	げんしょう	
\\	人間が知覚することのできるすべての物事。 自然界や人間界に形をとって現れるもの。 「不思議な―が起こる」「一時的な―」「自然―」 
\\	哲学で、 ㋐本体・本質が外的に発現したもの。 ㋑カント哲学で、主観によって感性的に受容された内容が、時間・空間およびカテゴリーなどの主観にそなわる認識形式によって、総合的に統一されたもの。 その背後にある物自体は不可知とされる。 ㋒ワッサールの現象学で、意識に現前し、直接的に自らを現している事実。 本体などの背後根拠との相関は想定しない。	▲その現象は今の時代に特有のものだ。 ▲ヨーロッパの主要都市の多くはドーナツ化現象に悩まされている。
\\	現状	げんじょう	現在の状態、ありさま。 「―を打破する」「―に甘んじる」「―分析」「―維持」	▲現状のままにして置く。 ▲現状では倒産はさけられない。
\\	建設	けんせつ	[名]スル 
\\	建物・施設・道路などを、新たに造ること。 「ダムを―する」 
\\	新しい機構や組織を作り上げること。 「平和な社会を―する」	▲新しい道路が次から次へと建設された。 ▲新しい橋の建設が進行中だ。
\\	現代	げんだい	
\\	現在の時代。 今の世。 当世。 「―社会」 
\\	歴史上の時代区分の一。 ふつう、日本史では第二次大戦後の時代、世界史では第一次大戦後の時代をさす。 [類語]
\\	当世・当代・近代・今日(こんにち)・現今・同時代・今の世(よ)・モダン・コンテンポラリー	▲現代美術は19世紀の慣習とはすっかり変わっている。 ▲現代哲学は19世紀に始まる。
\\	建築	けんちく	[名]スル家屋などの建物を、土台からつくり上げること。 また、その建物やその技術・技法。 「注文―」「木造―」「耐震―」 [類語]建設・建造・築造・営造・造営・建立(こんりゆう)・普請(ふしん)・作事(さくじ)・造作(ぞうさく)・新築・改築・増築・移築	▲その教会はゴシック風の建築である。 ▲その遊園地を建築するのに10年かかった。
\\	検討	けんとう	[名]スルよく調べ考えること。 種々の面から調べて、良いか悪いかを考えること。 「―を重ねる」「―の余地がある」「問題点を―する」	▲来週の火曜日に役員会がその提案を検討することになっている。 ▲私たちはその件を詳細に検討した。
\\	見当	けんとう	
\\	大体の方向・方角。 「駅はこの―です」 
\\	はっきりしていない事柄について大体の予想をすること。 見込み。 「犯人の―はついている」「―をつける」 
\\	版画や印刷で、刷る紙の位置を決めるための目印。 その形からトンボともいう。 
\\	(接尾語的に用いて)数量を表す語に付いて、その程度の数量であることを表す。 …ぐらい。 「五〇人―」	▲私は彼女の年を40歳と見当をつけた。 ▲彼女の心配は、全くの見当はずれだとわかった。
\\	現場	げんば	
\\	事件や事故が実際に起こった場所。 また、現にそれが起こっている場所。 げんじょう。 「事故の―を目撃する」 
\\	実際に作業が行われている場所。 また、企業などで、管理部門に対する実務部門をいう。 「工事―」「―の意見を採用する」	▲局長は、現地の朝日の記者を現場に行かせようとしていた。 ▲警官がその事故の現場に居合わせた。
\\	憲法	けんぽう	《近世まで「けんぼう」》 ㊀[名] 
\\	基本となるきまり。 おきて。 
\\	国家の統治権・統治作用に関する根本原則を定める基礎法。 他の法律や命令で変更することのできない国の最高法規。 近代諸国では多く成文法の形をとる。 →日本国憲法 →大日本帝国憲法 ㊁[名・形動ナリ]正しいこと。 公正であること。 また、そのさま。 「主人ガ―ナレバ国ガヨウ治リ」	▲学生たちは、憲法を暗記するようにいわれました。 ▲憲法を改正したいと考えている人たちがいる。
\\	権利	けんり	
\\	ある物事を自分の意志によって自由に行ったり、他人に要求したりすることのできる資格・能力。 「邪魔する―は誰にもない」「当然の―」「―を主張する」↔義務。 
\\	一定の利益を自分のために主張し、また、これを享受することができる法律上の能力。 私権と公権とに分かれる。 「店の―を譲る」↔義務。 
\\	権勢と利益。 [類語]
\\	資格・権限・権能・権益・特権・特典/
\\	私権・公権・人権	▲権利ばかり主張する人が多い。 ▲君は私に要求する権利はない。
\\	湖	みずうみ	《「水海」の意》周囲を陸地で囲まれたくぼ地に水をたたえる水域。 池や沼よりは大きく、沿岸植物の侵入できない深さのもので、ふつう最深部が五メートル以上をいう。	▲よく湖へスケートに行ったものだ。 ▲よく湖へつりに行ったものだ。
\\	小	こ	〔接頭〕 
\\	名詞に付いて、小さい、細かい、などの意を表す。 「―馬」「―石」 
\\	名詞に付いて、わずかな、少しの、などの意を表す。 「―雨」「―降り」 
\\	数量を表す名詞や数詞に付いて、ほぼ、だいたい、約、などの意を表す。 「―一時間」「―半時(はんとき)」 
\\	動詞・形容詞・形容動詞などに付いて、すこし、なんとなく、などの意を表す。 「―ざっぱりしたなり」「―高い」「―ぎれい」 
\\	名詞や用言などに付いて、軽んじたり、ややばかにしたりするような意を表す。 「―せがれ」「―利口」「―ざかしい」	▲大か、小か。
\\	後	ご	ある事件よりものちの日、または、時。 あと。 「その―」「数分―」	▲船は出航したが、2日後に難破した。 ▲車は10分後にガソリンが切れた。
\\	恋	こい	
\\	特定の異性に強くひかれること。 また、切ないまでに深く思いを寄せること。 恋愛。 「―に落ちる」「―に破れる」 
\\	土地・植物・季節などに思いを寄せること。 「明日香川川淀さらず立つ霧の思ひ過ぐべき―にあらなくに」 [類語]
\\	恋愛・愛恋(あいれん)・愛・恋情(れんじよう)・恋慕(れんぼ)・思慕(しぼ)・眷恋(けんれん)・色恋(いろこい)・ラブ・アムール	▲彼女は恋をしたことがありますか。 ▲彼女を一目見るやいなや、彼は恋してしまった。
\\	濃い	こい	[形][文]こ・し[ク] 
\\	色合いが強い。 「墨が―・い」「―・い藍(あい)染め」↔薄い/淡い。 
\\	においや味などが強い。 「塩味が―・い」「百合の―・い香り」↔薄い/淡い。 
\\	液体の中に溶けている物質の割合が高い。 「―・くいれたコーヒー」↔薄い。 
\\	密度が高い。 生え方などが密である。 「霧が―・い」「―・い眉」↔薄い。 
\\	度合いが強い。 「化粧が―・い」「並木が―・い影を落とす」↔薄い。 
\\	可能性などの程度が高い。 「敗色が―・い」「犯人である疑いが―・い」↔薄い。 
\\	関係が密接である。 「血のつながりが―・い」 [派生]こさ[名] [類語]
\\	深い・濃(こま)やか・濃密・濃厚	▲垣根の向こう側の芝生はいつも緑が濃い。 ▲コーヒーは濃いのが好きだ。
\\	恋人	こいびと	恋しく思う相手。 普通、相思相愛の間柄にいう。 「―にあう」「―ができる」 [用法]恋人・愛人――「恋人」は恋しいと思っている異性で、多く相思相愛の間柄についていうが、片思いの場合にも使うことがある。 「スクリーンの恋人」
\\	「愛人」は、かつては「恋人」の漢語的表現として同義に用いたが、現在では多く配偶者以外の恋愛関係にある異性をいい、一般に肉体関係があることを意味する。 「情婦」「情夫」の婉曲的な言葉として使われる傾向もある。	▲僕だって彼女を恋人にできてうれしいんだ。 ▲恋人たちはお互いに腕を組んで歩いていた。
\\	幸運・好運	こううん	[名・形動]運がよいこと。 めぐりあわせがよいさま。 しあわせ。 「―を祈る」「―の女神」「―な人」↔非運/不運。	▲努力が私に幸運をもたらした。 ▲遅かれ早かれ彼の幸運も終わるだろう。
\\	講演	こうえん	[名]スル 
\\	(カウ―)大ぜいの人に向かって、ある題目に従って話をすること。 また、その話。 「政治問題について―する」 
\\	(コウ―)経典を講じ仏法を説くこと。 説法。	▲彼はクラブの講演を免じてもらった。 ▲彼は講演するためホールに行った。
\\	効果	こうか	
\\	ある働きかけによって現れる、望ましい結果。 ききめ。 しるし。 「薬の―が現れる」「宣伝―」「―覿面(てきめん)」 
\\	演劇・映画などで、その場面に情趣を加える技術および方法。 雨音・風音・煙・雪など。 エフェクト。 「音響―」 [類語]
\\	効(き)き目・徴(しるし)・成果・効(こう)・実効・効験・効能・効力・効用・甲斐(かい)	▲カーペットには埃を吸収するダストポケット効果があり、埃の飛散を防ぐ特長があるのだが、それが仇になった結果といえる。 ▲花乃、嘘泣きはたまーにやるから効果あるんだぞ。
\\	金	きん	㊀[名] 
\\	銅族元素の一。 単体は黄金色で光沢がある。 金属中最も展延性に富み、厚さ〇・一マイクロメートルの箔(はく)にすることが可能。 化学的に安定で、酸化されにくく錆(さ)びず、また、王水には溶けるが、普通の酸やアルカリにはおかされない。 自然金の形で主に石英鉱脈中から産出し、母岩が風化したあと川に沈積した砂金としても得られる。 貴金属として貨幣・装飾品や歯科医療材料などに使用。 比重一九・三。 記号
\\	原子番号七九。 原子量一九七・〇。 こがね。 黄金(おうごん)。 
\\	値打ちのあるもののたとえ。 「―の卵」「沈黙は―」 
\\	㋐金貨。 また、金銭。 「―一封」「手切れ―」 ㋑金額を記すときに、上に付けて用いる語。 「―五万円」 
\\	きんいろ。 こがねいろ。 「―ラメのスカーフ」 
\\	将棋の駒で、金将。 
\\	金メダル。 「日本選手が―・銀・銅を独占する」 
\\	睾丸(こうがん)のこと。 きんたま。 
\\	金曜日。 
\\	五行の第四位。 方位では西、季節では秋、五星では金星、十干では庚(かのえ)・辛(かのと)に配する。 ㊁〔接尾〕数を示す語に付いて、金の純度を表すのに用いる。 二四金が純金。 カラット。 「一八―のペン先」	▲日本シンクロ界の悲願である金には、あと一歩で届かなかった。 ▲金は鉄より重い。
\\	管	くだ	
\\	細長い円筒形で中が空洞になっているもの。 「―を通して水を送る」 
\\	機(はた)の横糸を巻いて杼(ひ)に入れる道具。 
\\	糸繰り車の紡錘(つむ)に差して糸を巻きつける軸。 
\\	「管狐(くだぎつね)」の略。 想像上の小さなキツネ。竹管の中で飼われ、飼い主の問いに応答したり、予言をしたりする通力をもつ。 
\\	「管の笛」に同じ。 管状の小さい笛。戦場で大角(はらのふえ)と共に用いた。くだぶえ。「吹き響(な)せる―の音も」	▲管から水が吹き出した。
\\	急	きゅう	㊀[名] 
\\	切迫した事態。 また、突然の変事。 「風雲―を告げる」「国家の―を救う」「―を知らせる」 
\\	急いですること。 「―を要する仕事」 
\\	舞楽や能などで、一曲全体または一曲中の舞などを序・破・急の三つに分けた場合、その最後の部分。 →序破急 ㊁[形動][文][ナリ] 
\\	切迫したさま。 急いで対処しなければならないさま。 「―な事態」 
\\	物事が前触れなく突然に起こるさま。 にわか。 だしぬけ。 「―に雨が降りだす」「―な腹痛」 
\\	気短なさま。 性急。 「新奇を求めるのに―なあまり」 
\\	手を緩めずきびしいさま。 「追撃がはなはだ―だ」 
\\	速度・調子が速いさま。 「脈拍が―になる」「―な流れ」 
\\	傾斜などが大きいさま。 「―な坂」 [類語] 
\\	緊急・火急・緊切/
\\	にわか・出し抜け・突然・急遽(きゆうきよ)・唐突・短兵急・不意(連用修飾語として)突如・いきなり・不意に・ふと・矢庭に/
\\	急速・急激・激しい・慌ただしい・目まぐるしい	▲事態は急を要するので一刻も待てないと彼は言った。 ▲機は急角度で上昇し、それから海岸を離れるにつれて水平飛行に移った。
\\	高価	こうか	[名・形動]値段が高いこと。 価値が高いこと。 また、そのさま。 「―な品物」「―な犠牲を払う」↔安価/廉価。	▲あのソファーはこのテーブルほど高価ではない。 ▲あのペンはこのペンよりも高価だ。
\\	硬貨	こうか	
\\	金属で鋳造した貨幣。 紙幣と区別してよばれる。 コイン。 
\\	国際金融上、金または米ドルなどの外貨と交換可能な通貨。 →軟貨	▲彼女は少なかったが財布の中にあるすべての硬貨を少年にあげた。 ▲これらの硬貨はほとんど価値がない。
\\	豪華	ごうか	[名・形動]ぜいたくで、はでなこと。 また、そのさま。 「―な舞台衣装」「絢爛(けんらん)―」 [派生]ごうかさ[名]	▲彼らは私を豪華な中華料理でもてなした。 ▲超年老いた夫婦が、結婚75周年を祝して豪華な夕食を食べていた。
\\	合格	ごうかく	[名]スル 
\\	特定の資格や条件に適合すること。 「規格に―した製品」 
\\	試験や検査などに及第すること。 「入学試験に―する」	▲数学の試験に合格するために私は一生懸命勉強しました。 ▲先生はトムが試験に合格するだろうと結論付けた。
\\	交換	こうかん	[名]スル 
\\	取りかえること。 また、互いにやり取りすること。 「現金と―に品物を渡す」「意見を―する」 
\\	電話の交換手、または交換台のこと。 「―を通して電話をかける」 
\\	民法上、当事者が互いに金銭以外の財産権を移転することを約する契約。 [類語]
\\	互換(ごかん)・取り換え・付け替え・入れ替え・チェンジ・引き換え(―する)引き換える・交(か)わす・取り交わす	▲どのような集まりでも贈り物を交換することの喜びの半分は、他の人たちが持ってきたものを見、そしてそれについて語り合うことの中。 ▲ドルの交換レートは、いまいくらですか。
\\	航空	こうくう	航空機などに乗って空中を飛行すること。 「民間―」	▲ユナイテッド航空の受付カウンターはどこですか。 ▲航空管制官はすごく集中力がいる仕事だ。
\\	光景	こうけい	
\\	目前に広がる景色。 眺め。 「白銀にかがやく峰々の―」 
\\	ある場面の具体的なありさま。 情景。 「惨憺(さんたん)たる―」 
\\	日のひかり。	▲その光景は目も当てられなかった。 ▲その光景は描写できないほど美しかった。
\\	合計	ごうけい	[名]スル二つ以上の数値を合わせまとめること。 また、そのようにして出した数。 「三教科の得点を―する」「―金額」	▲くしくも、レジでの合計額は777円であった。 ▲保険、税金を含めて、合計金額はいくらになりますか。
\\	攻撃	こうげき	[名]スル 
\\	進んで敵を攻め撃つこと。 「総勢を挙げて―する」「奇襲―」 
\\	議論などで、相手を責めなじること。 非難すること。 「団交で―の的になる」「個人―」 
\\	スポーツの試合などで、相手を攻めること。 「九回裏の―」↔守備。 [類語]
\\	襲撃・急襲・強襲・突撃・進撃・進攻・侵攻・攻勢・狙(ねら)い撃ち・征伐(せいばつ)(―する)襲(おそ)う・襲いかかる・攻める・攻めかかる・攻め立てる/
\\	非難・指弾・糾弾(きゆうだん)・弾劾(だんがい)・論難・弁難・駁撃(ばくげき)(―する)責め立てる・槍玉(やりだま)にあげる	▲彼らは敵を攻撃した。 ▲兵士たちは敵の攻撃に抵抗した。
\\	貢献	こうけん	[名]スル 
\\	ある物事や社会のために役立つように尽力すること。 「学界の発展に―する」「―度」 
\\	貢ぎ物を奉ること。 また、その品物。	▲彼は社会の福利のために貢献してくれた。 ▲彼は経済学に相当な貢献をした。
\\	広告	こうこく	[名]スル 
\\	広く世間一般に告げ知らせること。 
\\	商業上の目的で、商品やサービス、事業などの情報を積極的に世間に広く宣伝すること。 また、そのための文書や放送など。 「―を載せる」「新製品を―する」「募集―」 [類語]
\\	広報・公告・布告・告示/
\\	宣伝・PR(ピーアール)・アドバタイジング・コマーシャル・CM(シーエム)	▲彼は新聞に広告を載せた。 ▲彼は新聞から広告を切り抜いた。
\\	交際	こうさい	[名]スル人と人とが互いに付き合うこと。 まじわり。 「―を結ぶ」「グループで―する」「男女―」 [用法]交際・つきあい――「友人との交際(付き合い)」「交際(付き合い)が広い」のように、他人と関係を持つという意では相通じて用いられる。 
\\	「付き合い」は、「家族も同然の付き合い」のように私的に親密な関係から、「付き合いで飲みに行く」「あの会社とは付き合いがない」のように、義理や業務上の関係にまで広く使われる。 
\\	「交際」は、義理や世間体の関係ではなく、表立った付き合いの意で用いて、「交際費」のような複合語をつくる。 
\\	類義語「まじわり」は多く文章語として用いる。	▲ボブとルーシーは交際をやめたということだ。 ▲悪い奴らと交際するな。
\\	後者	こうしゃ	
\\	二つ挙げたうちのあとのもの。 ↔前者。 
\\	あとに続く者。 後世の人。	▲前者の選択肢を支持している人が多いが、私は後者のほうが好きだ。 ▲前者が解決すれば後者も解決するであろう。
\\	構成	こうせい	[名]スル 
\\	いくつかの要素を一つのまとまりのあるものに組み立てること。 また、組み立てたもの。 「国会は衆議院と参議院とで―されている」「家族―」 
\\	文芸・音楽・造形芸術などで、表現上の諸要素を独自の手法で組み立てて作品にすること。 「番組を―する」	▲大気は生物が反応する環境の主要な部分を構成しており、大きいな天然資源の持つ特徴を高度に備えている。 ▲水は水素と酸素で構成されている。
\\	高速	こうそく	
\\	速度の速いこと。 高速度。 「―運転」↔低速。 
\\	「高速道路」の略。自動車が高速度で走るための専用道路。 「東名―」	▲昼も夜もたくさんの車がこのハイウェーを高速でつうかする。 ▲普通の人は、3Dだとか高速でうんたらだとか必要としないですからね。
\\	行動	こうどう	[名]スル 
\\	あることを目的として、実際に何かをすること。 行い。 「具体的な―を起こす」「―を共にする」「自分で考えて―する」「―力」 
\\	心理学で、外部から観察可能な人間や動物の反応をいう。 →行為[用法] [類語]
\\	行(おこな)い・振る舞い・行為・挙(きよ)・活動・動き・所行(しよぎよう)・言動・言行(げんこう)・行状(ぎようじよう)・行跡(ぎようせき)	▲その運動は女性の行動に大きな影響を及ぼした。 ▲その自分勝手な男は、一緒に行動している人達に軽蔑された。
\\	強盗	ごうとう	《古くは「ごうどう」とも》暴力や脅迫などの手段で他人の金品を奪うこと。 また、その者。 「―を働く」	▲父は強盗と取っ組み合いをした。 ▲脱走した強盗はまだつかまらない。
\\	幸福	こうふく	[名・形動]満ち足りていること。 不平や不満がなく、たのしいこと。 また、そのさま。 しあわせ。 「―を祈る」「―な人生」「―に暮らす」 [派生]こうふくさ[名] [類語]幸せ・幸(さいわ)い・幸(さち)・福・果報・冥利(みようり)・多幸・多祥(たしよう)・万福(ばんぷく)・至福・浄福・清福・ハッピー	▲私はとても幸福です。 ▲私は幸福すぎて眠れなかった。
\\	公平	こうへい	[名・形動]すべてのものを同じように扱うこと。 判断や処理などが、かたよっていないこと。 また、そのさま。 「―を期する」「―な判定」 [派生]こうへいさ[名] [用法]公平・公正――「商売の利益を公平(公正)に分配する」「評価の公平(公正)を心がける」のように、平等に扱うの意では相通じて用いられる。 
\\	「公平」は「おやつのお菓子を公平に分ける」「公平無私」など、物事を偏らないようにすることに重点があるのに対し、「公正」は「公正な商取引を目指す」のように、不正・ごまかしがないことを主にいう。 
\\	「受験のチャンスは公平に与えられる」では「公平」が適切であり、「行政は常に公正でなくてはならない」では「公正」が適切である。 
\\	類似の語「公明」は「公明選挙」など、多く公的な立場について用いられる。	▲公平に評すれば、彼はお人好しだ。 ▲公平に評すれば、彼はいい奴だ。
\\	候補	こうほ	ある地位や資格などを得るのにふさわしいと、他から認められている人。 また、ある選択の対象としてあげられている人や物。 「―に上る」「優勝―」	▲スミス氏は市長候補である。 ▲彼はその選挙で対立候補を破った。
\\	考慮	こうりょ	[名]スル物事を、いろいろの要素を含めてよく考えること。 「―に入れる」「―の余地がない」「相手の事情を―する」	▲あなたの提案は考慮するに値する。 ▲あなたは彼女の病気を考慮すべきだ。
\\	越える・超える	こえる	[動ア下一][文]こ・ゆ[ヤ下二] 
\\	(越える)物の上・間・境界などを通り過ぎて、向こうへ行く。 「打球がフェンスを―・える」「山を―・え、また谷を―・える」「海を―・えてきた便り」「国境を―・える」  
\\	(越える)区切りとなるある日時が過ぎる。 時を経過する。 「―・えて翌年の春を迎える」「齢(よわい)八〇を―・える」 
\\	ある基準・数量を上回る。 超過する。 「四万人を―・える観衆」「危険水位を―・える」 
\\	地位・段階などで、順序をとばして先になる。 飛びこす。 「先輩を―・えて重役になる」 
\\	他のものよりすぐれる。 ぬきんでる。 「力量が衆を―・えている」 
\\	ある考えや主義にとらわれず先に進む。 また、ある基準・範囲の外まで出る。 超越する。 「互いに立場を―・えて手を結ぶ」「想像を―・える」「常識を―・える」 
\\	規則やきまりに外れる。 「矩(のり)を―・えず」 [類語]
\\	越す・過ぎる・渡る・通り越す・またぐ・越境する/
\\	越す・上回る・食(は)み出す・超過する・突破する・オーバーする/
\\	抜く・凌(しの)ぐ・凌駕(りようが)する	▲彼は国境を越えて行った。 ▲彼はおじいさんは90歳を超えていると言った。
\\	コーチ	コーチ	[名]スル運動競技の技術などについて指導・助言すること。 また、その人。 「ピッチングを―する」「バッティング―」	▲そのコーチは彼によい助言をした。 ▲そのチームは誰がコーチしているのですか。
\\	コード	コード	
\\	規定。 規則。 特に、新聞社・放送局内部で設けた、準拠すべき倫理規定。 「プレス―」「放送―」 
\\	略号。 符号。 電信符号。 暗号。 「―ブック」 
\\	情報を表現するための記号や符号の体系。 特に、コンピューターのデータ・命令などを符号で表現したもの。 「漢字のJIS―」	
\\	といっても、Mac 
\\	のコードをバージョンアップした訳ではない。 ▲彼は機械にコードをつないだ。
\\	氷・凍り	こおり	
\\	水が固体状態になったもの。 一気圧のもとではセ氏0度以下で固体化する。 比重〇・九一七。 《季 冬》「歯豁(あらは)に筆の―を噛む夜哉/蕪村」 
\\	冷たいものや鋭いもののたとえ。 「―の刃(やいば)」「―のような心」 
\\	「氷水(こおりすい)」の略。 「―いちご」 
\\	「氷襲(こおりがさね)」の略。 [類語]
\\	氷塊・氷片・氷柱(ひようちゆう・つらら)・氷層・堅氷(けんぴよう)・薄氷(はくひよう)・薄ら氷(ひ)・流氷・氷雪・氷霜(ひようそう)・アイス	▲氷は日にあたると溶ける。 ▲氷は日なたで溶けてしまった。
\\	凍る・氷る	こおる	[動ラ五(四)] 
\\	液体、特に水が低温のため凝結して固体の状態になる。 「池が一面に―・る」《季 冬》 
\\	外気などがひどく冷たく感じられる。 「冬の朝の―・った空気」 
\\	寒さや恐ろしさのために、からだがこわばり自由に動かなくなる。 「恐怖で血も―・るオカルト映画」 [類語]
\\	氷結する・結氷する・凍結する・凍(い)てる・凍(い)て付く・凝(こご)る・凍(し)みる・しばれる	▲そのニュースをラジオで聴いたとき背筋の凍る思いがした。 ▲その光景を見て血が凍った。
\\	ゴール	ゴール	[名]スル 
\\	競技の決勝線。 決勝点。 「―直前のデッドヒート」「―に達する」 
\\	ラグビー・サッカー・ホッケーなどで、ボールを入れると得点になる枠。 また、そこにボールを入れて得点すること。 「ペナルティー―」 
\\	努力などの目標点。 最終目的。 「入学という―を目ざしてがんばる」	▲マラソン競技に初めには、何十人もの選手が出発するが、ゴールまでくるのはごく一部の選手であり、優勝するのはたった一人である。 ▲君は人生におけるゴールを見失ってはいけない。
\\	誤解	ごかい	[名]スルある事実について、まちがった理解や解釈をすること。 相手の言葉などの意味を取り違えること。 思い違い。 「―を招く」「―を解く」「人から―されるような行動」	▲誤解を正させてください。 ▲誤解を正してくださってありがとう。
\\	語学	ごがく	
\\	言語を対象とする学問。 言語学。 
\\	外国語の学習。 また、その学科。 外国語を使う能力についてもいう。 「―に堪能だ」	▲プレインイングリッシュは語学習得の近道です。 ▲概して女の子の方が男の子より語学がうまい。
\\	呼吸	こきゅう	[名]スル 
\\	息を吸ったり吐いたりすること。 「―を整える」「荒々しく―する」 
\\	共に動作をするときの互いの調子。 息。 「二人の―が合う」「阿吽(あうん)の―」 
\\	物事をうまく行う微妙な調子。 こつ、また、ころあい。 「―を覚える」「―をのみこむ」「―をはかる」 
\\	短い時間。 間(ま)。 「ひと―おいて再び話しはじめる」 
\\	生物が生命維持に必要なエネルギーを得るために、酸素を取り入れて養分を分解し、その際に生じた二酸化炭素を排出する現象。 体外とガス交換を行う外呼吸と、それにより運ばれた酸素による細胞内での内呼吸とがあり、一般には外呼吸をさす。 また、酸素を必要としない無気呼吸もある。 [類語]
\\	息(いき)・気息(きそく)・息衝(づ)き・息遣(づか)い・息差し・息吹(いぶき)・息の根/
\\	息・気合い・加減・按排(あんばい)・こつ・要領・タイミング	▲それは私たちが呼吸する空気のようなものです。 ▲先ずは憧れの作家の文章の呼吸をつかむためにひたすら筆写、丸写しをする。
\\	国語	こくご	
\\	一国の主体をなす民族が、共有し、広く使用している言語。 その国の公用語・共通語。 
\\	日本の言語。 日本語。 
\\	「国語科」の略。学校の教科の一。国語の理解・表現などの学習を目的とする。 「―の先生」 
\\	外来語・漢語に対して、日本固有の言葉。 和語。 大和言葉。	▲国語の先生は私たちにとても親切だ。 ▲国語の問題ができないというケースには三つあり、一つは速読力がないということです。
\\	黒板	こくばん	チョークで文字や図が書けるように黒または緑の塗料を塗った板。	▲あのー、先生?黒板に書いてあるの、指数関数じゃなくて三角関数ですけど・・・。 ▲彼は黒板に正方形を二つ書いた。
\\	克服	こくふく	[名]スル努力して困難にうちかつこと。 「病を―する」	▲その困難を克服しなければならない。 ▲その混乱をどうやって克服したか話して下さい。
\\	国民	こくみん	国家を構成し、その国の国籍を有する者。 国政に参与する地位では公民または市民ともよばれる。 [類語]人民・公民・市民・万民(ばんみん)・四民・臣民・同胞・国人(くにびと)・国民(くにたみ)・民(たみ)・民草(たみくさ)・億兆(おくちよう)・蒼生(そうせい)・蒼氓(そうぼう)	▲国民は自由を味わった。 ▲国民は重税に苦しんだ。
\\	穀物	こくもつ	人間がその種子などを常食とする農作物。 米・麦・粟(あわ)・稗(ひえ)・豆・黍(きび)の類。 穀類。	▲彼らは大量の穀物を蓄えている。 ▲彼は新式の穀物生産方法を学びにアメリカへ渡った。
\\	腰	こし	㊀[名] 
\\	人体で、骨盤のある部分。 脊椎が骨盤とつながっている部分で、上半身を屈曲・回転できるところ。 腰部(ようぶ)。 「―が曲がる」「―をおろす」 
\\	裳(も)や袴(はかま)などの 
\\	にあたる部分。 また、そのあたりに結ぶひも。 
\\	物の 
\\	に相当する部分。 中ほどより少し下部。 ㋐壁や建具の下部。 「―の高い障子」 ㋑器具の下部。 また、器具を支える台や脚。 ㋒山の中腹より下の方。 「山の―を巡る道」 ㋓兜(かぶと)の鉢の下部につける帯状の金具。 ㋔和歌の第三句。 「―の折れた歌」 
\\	餅(もち)・粉などの粘り・弾力。 
\\	紙・布などのしなやかで破れにくい性質。 
\\	(他の語の下に付き、「…ごし」と濁って)何かをする際の姿勢・構え。 「けんか―」「及び―」 ㊁〔接尾〕助数詞。 
\\	刀・袴など腰につけるものを数えるのに用いる。 「刀ひと―」「袴ひと―」 
\\	矢を盛った箙(えびら)を数えるのに用いる。 「矢ひと―」 [下接語]足腰・襟腰・尻(しつ)腰・尻(しり)腰(ごし)居合い腰・浮き腰・受け腰・後ろ腰・裏腰・海老(えび)腰・大腰・及び腰・ぎっくり腰・喧嘩(けんか)腰・小腰・高腰・中腰・強腰・釣り込み腰・逃げ腰・二枚腰・粘り腰・袴(はかま)腰・跳ね腰・払い腰・二重(ふたえ)腰・屁(へ)っ放(ぴ)り腰・細腰・本腰・前腰・丸腰・物腰・柳腰・弓腰・弱腰	▲席を立ちかけて腰を浮かした。 ▲祖母は老齢で腰がくの字に曲がっている。
\\	個人	こじん	
\\	国家や社会、また、ある集団に対して、それを構成する個々の人。 一個人。 「―の意思を尊重する」 
\\	所属する団体や地位などとは無関係な立場に立った人間としての一人。 私人。 「私―としての意見」 ◆「一個+人」の「一個人」が「一+個人」に誤解された語。	▲社会は個人に大きな影響を与える。 ▲社会は個人からなりたっている。
\\	越す・超す	こす	[動サ五(四)] 
\\	(越す)ある物の上を通り過ぎて一方から他方へ行く。 また、難所や障害となるものを通って、その先へ行く。 「塀を―・す」「難関を―・す」「峠を―・す」 
\\	数量・程度がある基準以上になる。 「一万人を―・す応募者」「気温が三〇度を―・す」 
\\	(越す)ある時期・期間を過ごす。 「年を―・す」「還暦を―・す」 
\\	(越す)追い抜く。 「先を―・される」 
\\	(「…にこしたことはない」のように打消しの表現を伴って)…するのがいちばんよい。 「早いに―・したことはない」 
\\	(越す) ㋐別の所へ移って住む。 引っ越す。 「新居へ―・す」 ㋑(「おこし」の形で)「行く」「来る」の意の尊敬語。 「どちらへお―・しですか」「またお―・しください」 [可能]こせる [下接句]先を越す・先(せん)を越す・峠を越す・年を越す・一山(ひとやま)越す	▲船は今夜赤道を越すだろう。 ▲彼は年を越すことができなかった。
\\	国家	こっか	
\\	くに。 
\\	一定の領土とそこに居住する人々からなり、統治組織をもつ政治的共同体。 または、その組織・制度。 主権・領土・人民がその三要素とされる。	▲国家の文化とアイデンティティという枢要な問題から切り離された時点で、比較文化はその方向性を失う。 ▲国家の名誉は最高の価値ある国家的財産である。
\\	国会	こっかい	
\\	日本国憲法の定める国の議会。 国権の最高機関で、国の唯一の立法機関。 衆議院と参議院の両議院で構成され、主権者である全国民を代表する議員で組織される。 
\\	明治憲法下における帝国議会の俗称。	▲国会は来週月曜に開かれる。 ▲国会は来週休会となる。
\\	国境	こっきょう	隣接する国と国との境目。 国家主権の及ぶ限界。 河川・山脈などによる自然的なものと、協定などによって人為的に決定するものとがある。 くにざかい。 「―を固める」「芸術に―はない」	▲使節団は空路を利用して国境までの最短距離を行った。 ▲国境問題での小競り合いから重大な国際紛争へと広がりました。
\\	骨折	こっせつ	[名]スル骨が折れること。 また、骨にひびが入ったり、その一部または全部が折れたりすること。 傷口が開いていない場合を閉鎖性骨折・単純骨折、傷口が開いている場合を開放性骨折・複雑骨折とよぶ。	▲私は腕を骨折した。 ▲私は足を骨折した。
\\	小包	こづつみ	
\\	小さな包み。 
\\	「小包郵便物」の略。通常郵便物以外の、小形の物品を内容とする郵便物。	▲彼はその小包を縛った。 ▲彼に小包を送った。
\\	琴・箏	こと	
\\	箏(そう)のこと。 箏が流行弦楽器となった江戸時代以後、特にいわれる。 近年は「琴」の字を当てて、箏を表すことも多い。 
\\	日本で、弦楽器の総称。 「琴」の字を当てたが、のち「箏」の字も用いた。 琴(きん)・箏(そう)・和琴(わごん)・百済琴(くだらごと)・新羅琴(しらぎごと)などのすべてをさす。	▲彼女は琴を弾くことがとても好きだ。
\\	異なる	ことなる	[動ラ五(四)]《形容動詞「異なり」の動詞化》違いがある。 別である。 「兄弟でも性格は―・る」「習慣が―・る」→違う[用法]	▲これは、通常の値引きとは異なります。 ▲これらの2つの分子が異なったスピードで動いていることを示すだけでは不十分である。
\\	諺	ことわざ	古くから言い伝えられてきた、教訓または風刺の意味を含んだ短い言葉。 生活体験からきた社会常識を示すものが多い。 「情けは人のためならず」「まかぬ種は生えぬ」の類。	▲そのことわざを私達はよく知っている。 ▲その諺はフランクリンの言葉から引用した。
\\	断る・判る	ことわる	⦅他五⦆ (「事割る」の意) 
\\	物事の筋道をはっきりさせる。 理非曲直を判断する。 また、筋道を立てて説明する。 三蔵法師伝承徳点「願はくは自ら裁コトハリ抑おさへむ」。 徒然草「にぎはひ豊かなれば、人には頼まるるぞかしと―・られ侍りし」 
\\	道理ありと認める。 源氏物語竹河「いとやむごとなくて、久しくなり給へる御方にのみ―・りて、…この御方ざまをよからず取りなしなどするを」 
\\	物事の内容を理解する。 思い知る。 源氏物語蛍「つらしと思ひしをりをりいかで人にも―・らせ奉らむと」 
\\	あらかじめ了解を求める。 予告する。 歌舞伎、東海道四谷怪談「必ず後で恨まつしやるな。 ―・りましたぞ―・りましたぞ」。 「誰に―・ってそんな事をしたのか」「一言―・っておく」 
\\	(理由をのべて)言いわけをする。 釈明する。 今鏡「いと頼むかひなくと仰せられければ、―・り申す限りなくて」 
\\	《断》(理由を言って)辞退する。 承諾しない。 拒絶する。 「援助の申出を―・る」「頼みを―・る」	▲私は病気のためその招待を断らなければならなかった。 ▲彼は断りなく私の部屋に入ってきた。
\\	粉	こな	
\\	砕けて細かくなったもの。 粉末。 「木炭の―」 
\\	米や麦、ソバなどをひいて細かくしたもの。 特に、小麦粉。 「―をまぶす」	▲私たちは小麦をひいて粉にする。 ▲水を加え、練り粉がどろどろしない程度に混ぜなさい。
\\	好み	このみ	
\\	好むこと。 好きなものの傾向。 嗜好。 「―のタイプ」「人によって―が違う」「―にあう」 
\\	特に望むこと。 物を選ぶときの希望や注文。 「いかようにもお―に合わせます」 
\\	歌舞伎で、大道具・小道具・衣装などについて役者が特に工夫したり注文したりすること。 
\\	(「…ごのみ」の形で)名詞の下に付いて、複合語をつくる。 ㋐好きなものの傾向。 「はで―」 ㋑ある時代、または、ある特定の人に好まれた様式。 「元禄―」「利休―」	▲彼女は彼に好みに合ったネクタイをプレゼントした。 ▲彼女は食べ物の好みがうるさい。
\\	好む	このむ	[動マ五(四)] 
\\	多くのものの中から特にそれを好きだと感じる。 気に入って味わい楽しむ。 「甘いものを―・む」「推理小説を―・む」 
\\	特にそれを望む。 欲する。 「組み打ちはこっちの―・むところ」 
\\	趣向をこらす。 風流にする。 「この男の家には前栽―・みて造りければ」 
\\	注文する。 あつらえる。 「はばかりながら文章を―・まん」	▲彼は毎日好んで独りで過ごす時間を持つ。 ▲彼は飛行機での海外旅行を好む。
\\	小麦	こむぎ	イネ科の越年草。 高さ約一メートル。 茎は節のある円筒状で、葉は細長く、基部は茎を抱く。 五月ごろに穂状の花をつけ、実は楕円形で溝があり、熟すと褐色になる。 西アジアの原産で、重要な穀物として世界中で栽培。 多くの品種があり、春小麦と冬小麦とに大別される。 実を主に粉にひいてパンなどの原料にする。 まむぎ。 パン小麦。 《季 夏》	▲それぞれの畑はどのくらいの小麦を産出しますか。 ▲その農民は小麦を無駄にしたことを後悔した。
\\	小屋	こや	
\\	小さくて粗末な建物。 仮に建てた簡単な造りの小さな建物。 「掘っ建て―」 
\\	雑物や家畜を入れておく簡単な建物。 「物置―」「うさぎ―」 
\\	《もと仮設の粗末な建物であったところから》芝居・見世物などの興行をするための建物。 「―を掛ける」「見せ物―」 
\\	平安時代、京都の大路に設けられた、衛府の役人などが夜回りにあたるときの詰め所。 
\\	江戸時代、城中または藩士の藩邸にあった下級藩士の住居。 
\\	家の天井と屋根の間の部分。	▲彼は十年以上もその小屋に一人で住んでいる。 ▲彼がその中に住んでいた小屋。
\\	殺す	ころす	[動サ五(四)] 
\\	㋐他人や生き物の生命を絶つ。 命を取る。 「首を絞めて―・す」「虫も―・さぬ顔」 ㋑自分ではどうすることもできないで、死に至らせる。 死なせる。 「惜しい人を―・したものだ」 
\\	㋐活動や動作をおさえとどめる。 「息を―・して潜む」「感情を―・す」「声を―・して笑う」 ㋑勢いを弱める。 「球速を―・したカーブ」 
\\	その人・物がもつ能力・素質・長所などを発揮できない状態にする。 特性・持ち味などをだめにする。 「せっかくの才能を―・してしまう」「濃い味付けで素材のうまみを―・している」 
\\	相手を悩殺する。 惑わせる。 「男を―・すまなざし」 
\\	競技やゲームなどで、何かの方法によって相手方が活動できないようにする。 ㋐野球で、アウトにする。 「牽制球で走者を―・す」 ㋑相撲で、相手の差し手をつかみ、または押しつけて動きを封じてしまう。 ㋒囲碁で、相手の石を攻めて、目が二つ以上できない状態にする。 「端の石を―・す」 
\\	動詞の連用形に付いて、いやになるほどその動作をする意を表す。 「ほめ―・す」 
\\	質に入れる。 「脇差曲げ、夜着を―・して」 [可能]ころせる [下接句]息を殺す・窮鳥懐に入れば猟師も殺さず・薬(くすり)人を殺さず薬師(くすし)人を殺す・声を殺す・大の虫を生かして小の虫を殺す・角を矯(た)めて牛を殺す・天道人を殺さず・二桃(にとう)三士を殺す・虫も殺さない・虫を殺す	▲彼は剣で殺された。 ▲彼は一発の弾丸で殺された。
\\	転ぶ	ころぶ	[動バ五(四)] 
\\	ころころと回転しながら進む。 ころがる。 「子犬が―・ぶように駆けてくる」 
\\	からだのバランスを失って倒れる。 転倒する。 「ぬかるみで滑って―・んだ」 
\\	物事の成り行きが他の方向に変わる。 事態がある方向へ向かう。 「どちらへ―・んでも大勢は変わらない」 
\\	㋐江戸時代、キリシタンが幕府に弾圧されて仏教に改宗する。 ㋑権力や誘惑に負けて、今までの主義・主張などを変える。 転向する。 「金に―・んで言いなりになる」 
\\	芸者などがひそかに売春をする。 「二貫も御祝儀を遣りゃすぐ―・ぶっていうんで」 [可能]ころべる	▲そのピエロはわざと転んだ。 ▲その子供は、つまずいて転んで膝を付いた。
\\	今回	こんかい	このたび。 「―はじめて出席した」→今度[用法]	▲もっとしっかり仕事をしてよ。今回は物見遊山の旅じゃないんだから。 ▲今回お手数をかけたことについて、いつかお返しをしたいと思っております。
\\	今後	こんご	今からのち。 こののち。 以後。 「―もよろしく願います」「―いっさい関知しない」	▲今回の事は公にすると今後の売り上げに影響が出るから、今回だけは許してやる。 ▲今後、あなたの仕事を手伝うようにしましょう。
\\	混雑	こんざつ	[名]スル 
\\	物事が無秩序に入りまじること。 「かかる―せる原理を以て」 
\\	たくさんの人が集まって込み合うこと。 「人と車で―する観光地」 
\\	もめごとがあること。 ごたごたすること。 また、いざこざ。 「教授会議や何ぞで、何か問題が―して来て」	▲お盆期間中は駅はとても混雑する。 ▲この時間帯は道路が混雑する。
\\	困難	こんなん	[名・形動]スル 
\\	物事をするのが非常にむずかしいこと。 また、そのさま。 難儀。 「―に立ち向かう」「予期しない―な問題にぶつかる」 
\\	苦しみ悩むこと。 苦労すること。 「道の上はぬかるみで―した」	▲農場生活から都市の生活への移行は困難なことが多い。 ▲日本人が英語をしゃべる場合、パーティーとか少人数の気さくな集まりなどでの形式張らない会話に対応するのが困難であることがよくある。
\\	婚約	こんやく	[名]スル結婚の約束を交わすこと。 また、その約束。 エンゲージ。 「―したばかりのカップル」「―者」「―指輪」→婚姻予約	▲ご婚約おめでとうございます。 ▲その女優は銀行家と婚約したといった。
\\	混乱	こんらん	[名]スル物事が入り乱れて秩序をなくすこと。 いろいろなものが入りまじって、整理がつかなくなること。 「交通機関に―を来たす」「経済の―をまねく」「頭が―して考えられない」	▲田中前外相の更迭に続く政治混乱がその象徴である。 ▲その混乱実に名状すべからず。
\\	差	さ	
\\	物事と物事の間の性質・状態・程度などの違い。 へだたり。 「大きな―をつける」「大した―はない」「世代間の―を感じる」 
\\	ある数や式から他の数や式を引いて得られた結果の数や式。 「一点の―で敗れる」↔和。	▲私たちの年齢の差は重要ではない。 ▲私は30秒の差で電車に乗り遅れた。
\\	サービス	サービス	[名]スル 
\\	人のために力を尽くすこと。 奉仕。 「休日は家族に―する」 
\\	商売で、客をもてなすこと。 また、顧客のためになされる種々の奉仕。 「―のよい店」「アフター―」 
\\	商売で、値引きしたり、おまけをつけたりすること。 「買ってくだされば―しますよ」 
\\	運輸・通信・商業など、物質的財貨を生産する過程以外で機能する労働。 用役。 役務。 
\\	⇒サーブ	▲日本には良いサービスに対してチップを払う習慣はない。 ▲現在、通常のサービスの早急な復旧に努めています。
\\	際	さい	
\\	とき。 場合。 機会。 「有事の―」「この―だから言っておこう」 
\\	物と物との接するところ。 「天地の―」→頃(ころ)[用法]	▲自宅を建て増し改築する際、この部屋にだけ防音設備と内鍵を付けて貰った。 ▲また辞去する際に決して忘れないようにすること。
\\	最高	さいこう	[名・形動] 
\\	地位や高さなどがいちばんたかいこと。 「世界―の山」「史上―の競争率」「―幹部」↔最低。 
\\	物事の程度が特にいちじるしいこと。 また、そのさま。 「―におもしろい映画」「今月は―に忙しかった」「―傑作」 
\\	物事が最も望ましい状態にあること。 この上なくすばらしいこと。 また、そのさま。 「―な(の)気分」「今日の試合は―だった」↔最低。 [類語]
\\	至高・最上・至上・無上・一番・最上級(数値、数量について)最大・最多・最大限・マキシマム・レコード/
\\	絶好・最上・ベスト・この上ない・すばらしい	▲質の点で彼のレポートが最高だ。 ▲乗鞍で温泉に入るのは最高だね。
\\	財産	ざいさん	
\\	個人や団体などの所有している、金銭・有価証券や土地・家屋・物品などの金銭的な価値のあるものの総称。 資産。 「一代で―を築きあげる」「子供に―を残す」 
\\	ある主体に属する積極財産(資産)と消極財産(負債)の総体。 「営業―」 
\\	あるものにとって、価値あるもの。 金銭的価値にも精神的価値にもいう。 「健康が私の―です」「自然は人類共有の―だ」 [類語]
\\	財・産・資産・資財・財貨・貨財・私産・私財・家産・家財・富・身代(しんだい)・身上(しんしよう)	▲彼はその妻に莫大な財産を残した。 ▲彼はこの2、3年でばく大な財産を手に入れた。
\\	最終	さいしゅう	
\\	いちばん終わり。 「いよいよ―の局面を迎える」「―目標」↔最初。 
\\	その日、最後に運行されるバス・電車・汽車・飛行機など。 「―に間に合う」	▲我々は最終決定を彼にまかせた。 ▲急げ、さもないと最終電車に乗り遅れるぞ。
\\	最中	さいちゅう	㊀[名] 
\\	動作・状態などが、いちばん盛んな状態にあるとき。 進行中のとき。 まっさかり。 さなか。 「今が暑い―だ」「食事の―」 
\\	まんなか。 
\\	いちばん盛りの状態にある人。 「渡辺党の―なり」 ㊁[副]しきりに。 「三皿目のシチウを今三人で―食っている」	▲火事のさなかに靴を片一方なくしてしまいました。 ▲私がジョンと電話で話している最中に、交換手が割り込んできた。
\\	最低	さいてい	[名・形動] 
\\	高さ・位置・程度などがいちばんひくいこと。 「一日の―の気温」↔最高。 
\\	物事の状態などがもっとも望ましくないこと。 きわめて質の劣ること。 また、そのさま。 「今度の試験は―だった」「あいつは―な男だ」↔最高。	▲あの映画、最低! ▲あんたは男としてってゆうか人間として最低だ〜!!
\\	才能	さいのう	物事を巧みになしうる生まれつきの能力。 才知の働き。 「音楽の―に恵まれる」「―を伸ばす」「豊かな―がある」「―教育」	▲彼の数学の才能はずば抜けている。 ▲彼の成功は才能というよりむしろ努力によるものだ。
\\	裁判	さいばん	[名]スル 
\\	物事の正・不正を判定すること。 「公正に―する」「宗教―」 
\\	裁判所が法的紛争を解決する目的で行う公権的な判断。 その形式には判決・決定・命令の三種がある。 「―に訴える」「―を受ける」 [類語]
\\	裁き・審判・裁定・裁決/
\\	訴訟・公判・審判・審理	▲裁判を中断するのは不可能だ。 ▲宗教がらみの裁判で、野心的な弁護士は教団の指導者の代理をする。
\\	材料	ざいりょう	
\\	ものを作るとき、そのもとにするもの。 「料理の―を用意する」「―費」 
\\	研究や調査、または判断などを裏づける証拠とするもの。 「結論を出すには―が不足だ」 
\\	芸術的表現の対象になるもの。 題材・素材。 「説話に―を求めた作品」 
\\	相場を動かすような要因。 「―待ち」 [類語]
\\	素材・材(ざい)・料(りよう)・資材・原料・マテリアル・マチエール(料理で)種(たね)・具(ぐ)/
\\	資料・データ/
\\	題材・素材	▲最良の材料のみを使用するよう十分な注意を払っております。 ▲穀物を作るのに必要な材料はありますか。
\\	幸い	さいわい	《「さきわ(幸)い」の音変化》 ㊀[名・形動] 
\\	その人にとって望ましく、ありがたいこと。 また、そのさま。 しあわせ。 幸福。 「不幸中の―」「君たちの未来に―あれと祈る」「御笑納いただければ―です」 
\\	運のいいさま。 また、都合のいいさま。 「―なことに明日は休みだ」 
\\	(「さいわいに」の形で)そうしていただければしあわせだと人に頼む気持ちを表す。 どうぞ。 なにとぞ。 「読者―に恕せよ」 ㊁[副]運よく。 都合よく。 幸せにも。 「―命だけは助かった」「―と事はうまく運んだ」	▲幸いにも誰もぬれなかった。 ▲幸いにも天気は良かった。
\\	サイン	サイン	[名]スル 
\\	㋐署名すること。 署名。 「小切手に―する」 ㋑スポーツ選手や芸能人などがファンのために自分の名前を色紙などに書くこと。 英語では
\\	という。 
\\	符号。 信号。 また、合図すること。 「目くばせで―を送る」 
\\	スポーツで、示し合わせた合図によるプレーの指示。 シグナル。	▲あなたのサインをいただけませんか。 ▲いつ禁煙のサインを消すのでしょうね。
\\	境・界	さかい	
\\	土地と土地との区切り。 境界。 「隣との―」「県―」 
\\	ものとものとが接する所。 また、ある状態と他の状態との分かれ目。 区切り目。 境目。 「空と海との―」「生死の―をさまよう」→境する 
\\	ある範囲の内。 地域。 場所。 また、境遇。 「身体を安逸の―に置くという事を文明人の特権のように考えている彼は」 
\\	心境。 境地。 「―に入りはてたる人の句は、此の風情のみなるべし」 [類語]
\\	境界(きようかい)・境界線・区画・仕切り・境目(さかいめ)・際(きわ)・分かれ目・分界・臨界・閾(いき)・ボーダーライン	▲彼は忘我の境をさまよっている。 ▲隣の家との境を示す柵がある。
\\	逆らう	さからう	[動ワ五(ハ四)] 
\\	物事の自然の勢いに従わないで、その逆の方向に進もうとする。 「風に―・って進む」「運命に―・って生きる」「時流に―・う」 
\\	目上の人の意見などに従わないで、反抗する。 はむかう。 たてつく。 「親に―・って進学する」「命令に―・う」「神の意思に―・う」→背(そむ)く[用法] [可能]さからえる	▲上司に逆らうのは賢明ではありません。 ▲忠言耳に逆らう。
\\	盛り	さかり	
\\	物事の勢いが頂点に達していること。 また、その時期。 「暑さが―を過ぎる」「食べたい―の子供達」「みかんの―」 
\\	人の一生のうちで心身ともに最も充実した状態・時期。 「人生の―を越す」 
\\	動物が一年の一定の時期に発情すること。 「―のついた猫」 ◆他の語の下に付いて複合語をつくるときは一般に「ざかり」となる。 「伸びざかり」「花ざかり」「働きざかり」など。 [下接語]出盛り・真っ盛り(ざかり)幼気(いたいけ)盛り・悪戯(いたずら)盛り・色盛り・男盛り・女盛り・血気盛り・育ち盛り・食べ盛り・年盛り・伸び盛り・働き盛り・花盛り・日盛り・分別(ふんべつ)盛り・娘盛り・若盛り	▲彼はいまを盛りにと言う時に倒れた。 ▲彼女は今が若い盛りだ。
\\	作業	さぎょう	[名]スル仕事。 また、仕事をすること。 特に、一定の目的と計画のもとに、身体または知能を使ってする仕事。 「修復―にとりかかる」「徹夜で―する」「―能率」「農―」 [類語]仕事・労働・労作・労務・役務(えきむ)・労役(ろうえき)・操業・業務	▲入りたての新人はまず出世の第一歩として、人が嫌がるつまらない単純作業を何でもやることだ。 ▲私たちは彼女の指示に従って作業を完了した。
\\	作品	さくひん	製作したもの。 特に、芸術活動による製作物。 「文学―」「工芸―」	▲私は両方の彼の作品を見たわけではない。 ▲私は今までシェークスピアの作品を3編読んだ。
\\	作物	さくもつ	
\\	田畑につくる植物。 穀類や野菜など。 農作物。 さくもの。 「園芸―」「救荒―」 
\\	「さくぶつ」に同じ。	▲日照りが作物に大損害を与えた。 ▲農家の人々は温室で作物を育てざるを得ないのです。
\\	桜	さくら	
\\	バラ科サクラ属の落葉高木の総称。 日本の代表的な花として、古来、広く親しまれている。 ヤマザクラ・サトザクラ・オオシマザクラなど種類は多く、園芸品種も多い。 現在多く植えられているのはソメイヨシノ。 花は春に咲き、淡紅色・白色など。 古くから和歌の題材とされ、単に花といえば桜をさし、かざしぐさ・あだなぐさ・たむけぐさなどともよばれた。 花は塩漬けにして桜湯に、葉は塩漬けにして桜餅に用いられ、またミザクラの実は食用。 樹皮は漢方で薬用。 木材は家具・建築用。 《季 花=春 実=夏》「宵浅くふりいでし雨の―かな/万太郎」 
\\	「桜色」の略。 
\\	「桜襲(さくらがさね)」の略。 
\\	芝居などで、ただで見物するかわりに、頼まれて役者に声をかける者。 転じて、露店商などの仲間で、客のふりをし、品物を褒めたり買ったりして客に買い気を起こさせる者。 
\\	《色が桜の花に似ているところから》馬肉の俗称。 桜肉。 
\\	紋所の名。 桜の花を図案化したもの。	▲上野の桜は今が見ごろだ。 ▲私達は川沿いの桜の花を見にでかけた。
\\	酒	さけ	
\\	エチルアルコールを含んだ飲料の総称。 製造法から、醸造酒・蒸留酒・混成酒に大別され、また、原料の違いによって世界中に多くの種類がある。 
\\	清酒の通称。 外国でも
\\	で通用する。 「辛口の―を好む」 
\\	酒を飲むこと。 飲む度合いや飲み方についていう。 「―が強い」「あの人はいい―だ」 
\\	酒盛り。 酒宴。 「―の席」	▲あまり酒を飲み過ぎるのは、危険だ。 ▲アメリカでは酒を買うには自分を証明しなければならない。
\\	叫ぶ	さけぶ	[動バ五(四)] 
\\	大声を発する。 大声で言う。 「助けを求めて―・ぶ」「万歳を―・ぶ」 
\\	世間に対して強く訴える。 強く主張する。 「無実を―・ぶ」「政治改革が―・ばれる」「核兵器廃絶を―・び続ける」 [可能]さけべる	▲彼は「出て行け!」と叫んだ。 ▲彼は助けてくれと叫んだ。
\\	避ける	さける	[動カ下一][文]さ・く[カ下二] 
\\	それとかかわることで不都合や不利益が生じると予測される人や事物から離れるようにする。 また、そのような人や事物に近づかないようにする。 「あの人は―・けたほうがよい」「反抗期で父親を―・ける」「都会の騒音を―・けて暮らす」「―・けて通れない問題」「ラッシュ時を―・けて出勤する」「人目を―・けて会う」 
\\	不都合や不利益をもたらすような言動をしないようにする。 差し控える。 「コメントを―・ける」「どぎつい表現は―・けたほうがよい」「武力衝突は―・けたい」 
\\	よける。 「飛び出してきた自転車を―・けきれずに衝突する」 [用法]さける・よける――「走ってくる車をさけよう(よけよう)としてころんだ」「水たまりをさけて(よけて)歩く」など、自分に害を与えるものや好ましくないものから意識的に離れることを表す場合、相通じて用いられる。 
\\	「さける」は「人目をさけて暮らす」「視線をさける」「明言をさける」など、抽象的なものが対象の場合にも用いる。 この場合「よける」は使わない。 
\\	「よける」は「落石をよけて事なきを得た」「相手のパンチをよけそこなう」のように、わきへ寄る、身をかわすなどの具体的な動作に重点がある。	▲家のない人たちは冷たいにわか雨をよける場所を探した。 ▲その地域の工業化は環境破壊を避けるため慎重に進められなければならない。
\\	支える	ささえる	[動ア下一][文]ささ・ふ[ハ下二] 
\\	倒れたり落ちたりしないように、何かをあてがっておさえる。 「太い柱で梁(はり)を―・える」「―・えられてよろよろ歩く」 
\\	ある状態が崩れないように、もちこたえる。 維持する。 「一家の暮らしを―・える」 
\\	精神的・経済的に支援する。 「地元の人達の声援に―・えられて選挙戦を勝ち抜く」 
\\	防ぎとめる。 くいとめる。 「敵の攻撃をかろうじて―・える」 ◆室町時代以降はヤ行にも活用した。 →支ゆ	▲彼はつえで体を支えた。 ▲彼の給料は安すぎて一家を支えていけない。
\\	差す・指す	さす	[動サ五(四)] 
\\	(差す) ㋐(「射す」とも書く)まっすぐに光が照り入る。 光が当たる。 「西日が―・す」 ㋑潮が満ちてくる。 また、水が増して入り込む。 しみ込む。 「潮が―・す」「氾濫した川の水が床下まで―・してきた」「井戸に廃水が―・す」 ㋒何かのしるし・気配などが自然と外に現れる。 「ほおに血の気が―・す」「景気にかげりが―・す」 ㋓ある種の気分・気持ちが生じる。 きざしてくる。 「眠けが―・す」「魔が―・す」「気が―・す(=気ガトガメル)」 ㋔平熱より高くなる。 熱が出る。 「熱が―・す」 ㋕枝や根が伸び広がる。 草木が伸びて出る。 「枝葉が―・す」 
\\	(指す・差す) ㋐指などで目標とする物や場所・方向を示す。 指さす。 「指で―・して教える」「後ろ指を―・される」「時計の針が七時を―・している」 ㋑人や物をそれと決めて示す。 指名する。 また、密告する。 「文中のそれは何を―・しますか」「生徒を―・して答えさせる」「犯人を警察に―・す」 ㋒その方向へ向かう。 目ざす。 「南を―・して飛ぶ」 ㋓物差しで寸法を測る。 「縦横の寸法を―・してみた」 ㋔指物を作る。 ㋕将棋で、駒を動かす。 また、対局する。 「将棋を―・す」「一局―・す」 ㋖物を手で持って上げる。 両手で高く上げる。 「米俵を―・す」 ㋗傘などをかざす。 ㋘肩に担ぐ。 になう。 「駕籠(かご)を―・す」 ㋙舞で、手を前方に伸ばす。 「―・す手引く手」 ㋚相撲で、相手の脇の下に手を入れる。 「右を―・す」 ㋛競馬などで、ゴールの直前で先行するものを追い抜く。 「―・して首の差で勝つ」 
\\	㋐雲などが、立ちのぼる。 「八雲―・す出雲の児らが黒髪は吉野の川の沖になづさふ」 ㋑さしつかえる。 「ちとお寺に―・す事ある」 [可能]させる ◆「指す」「差す」「射す」「刺す」「注す」「点す」「挿す」「鎖す」などと、いろいろに漢字が当てられるが、本来は同一の語。 [下接句]嫌気が差す・影が射す・気が差す・図星を指す・掌(たなごころ)を指す・鳥影(とりかげ)が射す・魔が差す・指一本も差させない・指を差す	▲他人を指差すのは失礼なことです。 ▲あなたのことをさして言っているのではない。
\\	座席	ざせき	すわる場所。 すわる席。 席。 「―指定車」	▲座席を予約しました。 ▲前もって座席の予約をすることが絶対に必要である。
\\	誘う	さそう	[動ワ五(ハ四)] 
\\	一緒に行動するようにすすめる。 また、連れ出す。 「ボランティア活動に―・う」「ドライブに―・う」 
\\	そのことが原因となって、ある気持ちを引き起こさせる。 促す。 「涙を―・うドラマ」「いい陽気に―・われて行楽地に繰り出す」 
\\	好ましくない状況などに引き入れる。 誘惑する。 「悪の道に―・う」 [可能]さそえる	▲彼女を映画に誘ったのだ。 ▲太陽の日差しに誘われて人々が外出した。
\\	札	さつ	㊀[名]紙幣。 ㊁〔接尾〕助数詞。 書状・証文などを数えるのに用いる。 「証文を一―入れる」	▲この10ドル札をくずしてもらえませんか。 ▲この100ドル札を20ドル札4枚と1ドル札20枚にくずしてください。
\\	作家	さっか	芸術作品の制作をする人。 また、それを職業とする人。 特に、小説家。 「―志望」	▲書物の選択に際して、過去の偉大な作家たちは最も注目されるべきだ。 ▲あなたの言葉から判断すると、彼は偉大な作家に違いない。
\\	作曲	さっきょく	[名]スル楽曲を創作すること。 また、詩歌・戯曲などに節や旋律をつけること。 「交響曲を―する」「ゲーテの詩に―してみる」	▲この音楽はバッハによって作曲された。 ▲美しい教会を建てたり、テニスをしたり、冗談を言ったり歌を作曲したり、月旅行をするような動物はいない。
\\	ざっと	ざっと	[副] 
\\	細部を問題にせず、おおまかに物事を行うさま。 ひととおり。 「書類に―目を通す」 
\\	全体の数量や内容などについておおまかな見当をつけるさま。 だいたい。 およそ。 「出席者は―一〇〇〇人だ」 
\\	水や雨が勢いよく落ちかかるさま。 「―水をかける」 
\\	動作がすばやく行われるさま。 あっという間に。 「敵を―けちからかして」	▲自分の名前があるかと彼女はリストにざっと目を通した。 ▲若い夫婦はその部屋をざっと眺めた。
\\	さっぱり	さっぱり	㊀[副]スル  
\\	不快感やわだかまりなどが消えて気持ちのよいさま。 すっきり。 「入浴して―(と)する」「思う存分泣いたので―した」 
\\	いやみのないさま。 また、しつこくないさま。 あっさり。 「―(と)した味」 
\\	あとに何も残らないさま。 すっかり。 「約束を―(と)忘れていた」「出世などとうの昔に―(と)あきらめている」 
\\	(あとに否定を表す語を伴って)全然。 まったく。 「―見えない」「―だめだ」 ㊁[形動]物事の状態が、非常に好ましくないさま。 「頑張ったのだが、成績のほうは―だ」	▲一風呂浴びてさっぱりした。 ▲彼女は高価ではないがさっぱりとした服装をしていた。
\\	扨・扠・偖	さて	
\\	一
\\	(接続) 
\\	それまでの話をきりあげ,別な話題に移る意を表す語。 ところで。 
\\	次に討論に入ります」 
\\	これまでの話を受けて,次の話に続けていく語。 そうして。 それから。 
\\	舟に乗った桃太郎はいよいよ鬼が島に着きました」「渠(カレ)は…地理書とを書箱(ホンバコ)から出して,―静かに昨日の続きの筆を執(ト)り始めた/蒲団(花袋)」 
\\	二
\\	(感) 
\\	感心したり驚いたりしたときに発する語。 
\\	ここはどこだろう」 
\\	次の行動に移ろうとするときに発する語。 
\\	ぼちぼち行くか」
\\	困った」 
\\	文末に用いて感動を表す語。 …よ。 「はて,そなたが待たば,愚僧も待たうは―/狂言・宗論(虎寛本)」 
\\	三
\\	(副) 
\\	その状態で。 そのままで。 「さらに,―過ぐしてむと思されず/源氏(夕顔)」 
\\	(「さての」の形で)そのほかの。 それ以外の。 「―の人々は,みな臆しがちに鼻じろめる/源氏(花宴)」	
\\	砂漠・沙漠	さばく	雨量が極端に少ないため植物がほとんど育たず、岩石や砂礫(されき)からなる地域。 サハラ・カラハリ・ゴビなど。	▲中国の砂漠は日本より多くの人間を養っている。 ▲地球の表面の3分の1は砂漠である。
\\	差別	さべつ	[名]スル 
\\	あるものと別のあるものとの間に認められる違い。 また、それに従って区別すること。 「両者の―を明らかにする」 
\\	取り扱いに差をつけること。 特に、他よりも不当に低く取り扱うこと。 「性別によって―しない」「人種―」 
\\	⇒しゃべつ(差別)	▲その週間は差別とは関係がなかった。 ▲論争の多い話題:積極的差別是正措置とバス通学。
\\	作法	さほう	
\\	ハフ) ㋐物事を行う方法。 きまったやり方。 きまり。 しきたり。 「婚儀は旧来の―にのっとる」 ㋑起居・動作の正しい法式。 「礼儀―」 ㋒詩歌・小説などのきまった作り方。 さくほう。 「小説―」 
\\	ホフ)仏事を行う法式。 葬礼・授戒などの法式。 「例の―にをさめ奉るを」→礼儀(れいぎ)[用法]	▲彼らは君の悪い作法に腹をたてている。 ▲彼の貴族的な作法には感心する。
\\	様様	さまざま	あれこれ異なっているさま。 いろいろ。 しゅじゅ。 「―の服装」「―な角度」	▲この問題は様々なやり方で解決できるかもしれない。 ▲これには、イギリスの生活のさまざまな面がそうであるように、もっともな歴史的理由があります。
\\	覚ます・醒ます	さます	[動サ五(四)] 
\\	眠っている状態から意識のはっきりした状態に戻す。 「目を―・す」「眠気を―・す」 
\\	酒の酔いをなくす。 「風にあたって酔いを―・す」 
\\	迷っている心を正常な状態にする。 迷いを解く。 「青少年の迷いを―・す」 [可能]さませる	▲その物音が私を眠りから覚まさせた。 ▲彼を眠りからさますな。
\\	覚める・醒める	さめる	[動マ下一][文]さ・む[マ下二] 
\\	眠っている状態から、意識のはっきりした状態に戻る。 「朝早く目が―・める」「麻酔が―・める」 
\\	眠けや酒の酔いが消える。 「酔いがいっぺんに―・める」 
\\	心をとらえていた迷いがなくなる。 正気をとりもどす。 冷静になる。 「悪い夢から―・める」「―・めた目で見る」	▲長い夢からさめた。 ▲君は何で目が覚めたの?
\\	左右	さゆう	[名]スル 
\\	ひだりとみぎ。 「―を確認する」「―の手」 
\\	かたわら。 そば。 まわり。 「―に従える」 
\\	そば近く仕える者。 側近。 「―に問う」 
\\	年齢などが、それに近いこと。 前後。 「六〇―の人」 
\\	立場や態度をあいまいにすること。 「言を―にする」 
\\	左か右かを決定すること。 どちらかに決めること。 
\\	思うままに支配すること。 決定的な影響を与えること。 「一生を―するような出来事」「作物の生育は天候に―されやすい」 
\\	能や狂言の舞の型の一。 左手をやや高く出し、左斜め前へ左足を出して右足を引きつけ、右手をやや高く出し、右斜め前へ右足を出して左足を引きつける。	▲彼は左右を見回した。 ▲人間はいわば感情に左右される生き物である。
\\	皿・盤	さら	㊀[名] 
\\	食物を盛る、浅くて平たい容器。 陶製・ガラス製・金属製などがある。 
\\	供応の膳(ぜん)などで、 
\\	に盛って出す料理。 
\\	に似た形のもの。 「ひざの―」「はかりの―」「灰―」 
\\	漢字の脚(あし)の一。 「盆」「益」「盛」「監」などの「皿」の部分の称。 ㊁〔接尾〕助数詞。 皿に盛った食物や料理などの数を数えるのに用いる。 「カレーライス二―」「炒(いた)め物三―」 [下接語]頭の皿・膝(ひざ)の皿 (ざら)油皿・石皿・受け皿・絵皿・大皿・角(かく)皿・菊皿・木皿・口取り皿・小皿・蒸発皿・中(ちゆう)皿・壺(つぼ)皿・手塩皿・時計皿・取り皿・灰皿・秤(はかり)皿・膝(ひざ)皿・火皿・平(ひら)皿・銘銘皿・薬味皿	▲誰かがこの皿を割りました。 ▲食卓にたくさんの皿がのっている。
\\	更に	さらに	[副] 
\\	同じことが重なったり加わったりするさま。 重ねて。 加えて。 その上に。 「―一年の月日が過ぎた」「―こういう問題もある」 
\\	今までよりも程度が増すさま。 前にも増して。 いっそう。 ますます。 「―きれいになった」「事態は―悪くなった」 
\\	(あとに打消しの語を伴って)いっこうに。 まったく。 少しも。 「―覚えがない」「反省するようすは―なく」 
\\	事新しく。 今さら。 「―何事をかは疑ひ侍らむ」 [類語]
\\	また・その上・おまけに・剰(あまつさ)え・加うるに・糅(か)てて加えて・のみならず/
\\	もっと・より・なお・なおさら・ますます・一層・一段と・余計・弥(いや)が上に	▲経験を積むにつれて更に知恵が身につく。 ▲景気は悪く来年の見通しはさらに悪い。
\\	猿	さる	
\\	霊長目のうち、ヒト科を除いた哺乳類の総称。 原始的な原猿、中南米の広鼻猿、アジア・アフリカの狭鼻猿、類人猿の四つに大別される。 ほとんどがオーストラリア以外の熱帯地方にすみ、ニホンザルはその北限の種。 日本では、ふつうこれをさす。 ましら。 →日本猿 
\\	㋐ずるがしこい者や、物まねのじょうずな者などをあざけっていう語。 ㋑野暮(やぼ)な人やまぬけな者をあざけっていう語。 
\\	雨戸などの上下の桟に取り付け、鴨居(かもい)・敷居の穴に差し込んで戸締まりをする用具。 
\\	自在鉤(じざいかぎ)をつるす竹に取り付けて、自在鉤を上にあげて留めておく器具。 
\\	小さな紙片の四隅を折って括猿(くくりざる)のような形を作り、中央に穴をあけて凧(たこ)の糸に通し、凧の糸目の所まで上って行かせる仕掛けの玩具。 
\\	ミカンの実の一袋を髪の毛などでくくって、括猿の形をこしらえる遊び。 
\\	《浴客の垢(あか)をかく動作を猿が爪(つめ)で物をかくのになぞらえていう》江戸で、湯女(ゆな)のこと。 風呂屋者。 
\\	江戸時代、上方で、岡っ引き・目明かしのこと。 [下接語]心の猿・竪(たて)猿・真(ま)猿・横猿 (ざる)赤毛猿・言わ猿・送り猿・尾長猿・尾巻猿・蟹(かに)食猿・瓦(かわら)猿・聞か猿・狐(きつね)猿・括(くくり)猿・蜘蛛(くも)猿・虚仮(こけ)猿・小猿・木の葉猿・米搗(つ)き猿・鹿(しか)猿・千疋(せんびき)猿・台湾猿・手長猿・天狗(てんぐ)猿・日本猿・幟(のぼり)猿・人似(ひとに)猿・日避(ひよけ)猿・豚尾猿・吠(ほえ)猿・見猿・眼鏡猿・山猿・栗鼠(りす)猿	▲安全日だからといって、サルのように生ではしません。 しっかり避妊するのが愛のセックスの義務ではないでしょうか? ▲あれはさる年に因んだ猿の絵です。
\\	去る	さる	[動ラ五(四)] 
\\	《本来は移動する意で、古くは、遠ざかる意にも近づく意にもいう》 ㋐ある場所から離れる。 そこを離れてどこかへ行ってしまう。 遠ざかる。 「故郷を―・る」「この世を―・る」「片時も念頭を―・らない」 ㋑地位・職業などを退く。 「王位を―・る」「舞台を―・る」 ㋒時が過ぎる。 ある季節・時期が遠のく。 「冬が―・る」「青春は―・った」 ㋓時間的、空間的に隔たる。 離れている。 「今を―・る七年前」「東京を―・ること二〇〇キロ」 ㋔今まであった状態が薄らいだり、なくなったりする。 消える。 「痛みが―・る」「危険が―・る」 
\\	㋐離して遠くへやる。 遠ざける。 離縁する。 「妻を―・る」 ㋑除いてなくす。 消す。 「雑念を―・る」「虚飾を―・る」 
\\	(動詞の連用形に付いて)すっかり…する、…しつくすの意を表す。 「忘れ―・る」「葬り―・る」 
\\	時・季節などが近づく。 巡ってくる。 「秋―・らば黄葉(もみち)の時に」 [可能]される [類語]
\\	㋐)遠ざかる・遠のく・離れる・立ち去る・引き払う・引き上げる・辞去する・退去する・退散する・失(う)せる・後(あと)にする/
\\	㋑)退(しりぞ)く・退(ひ)く・辞(や)める/
\\	㋒)過ぎる・過ぎ去る・過ぎ行く	▲その翌日、彼は去った。 ▲古きよき時代は去り、二度と戻らない。
\\	騒ぎ	さわぎ	《上代は「さわき」》 
\\	騒ぐこと。 また、騒がしいこと。 やかましさ。 「教室の―が静まる」 
\\	人々が騒ぐような出来事。 ごたごた。 騒動。 「―を起こす」「―になる」 
\\	(「…どころのさわぎ」の形で、あとに打消しの語を伴って用いる)そのような程度の事柄。 「見物どころの―ではない」 
\\	大変なこと。 めんどうなこと。 「こんなことが上司に知れたら―だ」 
\\	酒席などで、にぎやかにたわむれること。 遊興。 「何時なりとも―の節、きっと参上申すべく候」 
\\	「騒ぎ歌」の略。	▲まるで革命前夜のような騒ぎだった。 ▲学生は一晩中騒ぎ回った。
\\	参加	さんか	[名]スル 
\\	ある目的をもつ集まりに一員として加わり、行動をともにすること。 「討論に―する」「―者」 
\\	法律上の関係または訴訟に当事者以外の者が加わること。 「訴訟―」 [類語]
\\	加入・加盟・仲間入り・参入・参画・参与・参会・参列(―する)加わる・列する・連なる・名を連ねる	▲あなたはもちろん参加するものと思った。 ▲あなた方は年齢に関係なく、その話し合いに参加できる。
\\	参考	さんこう	[名]スル何かをしようとするときに、他人の意見や他の事例・資料などを引き合わせてみて、自分の考えを決める手がかりにすること。 また、そのための材料。 「研究の上で―になる」「内外の判例を―する」 [類語]参照・参看・参酌(さんしやく)・レファレンス	▲辞書は優れた参考本だ。 ▲米国の研修医制度については、田中まゆみ氏の著書「ハーバードの医師づくり」(医学書院)を一部参考とさせていただきました。
\\	賛成	さんせい	[名]スル 
\\	人の意見や行動をよいと認めて、それに同意すること。 「原案に―する」↔反対。 
\\	助力すること。 賛助。 「造化の功用を補弼し、万物の化育を―する」	▲その新しい計画に関してはあなたに賛成できない。 ▲その大臣は私が最近話しかけた人であるが、彼は私に賛成している。
\\	酸素	さんそ	酸素族元素の一。 単体は二原子分子からなる無色無臭の気体。 地球上で最も多量に存在する元素で、空気中には体積で約二一パーセント含まれる。 生物の呼吸や燃料の燃焼に不可欠。 反応性に富み、ほとんどの元素と化合して酸化物をつくる。 その際に熱と光とを伴うことが多い。 元素記号O。 原子番号八。 原子量一六・〇〇。	▲水分子は、2個の水素原子と1個の酸素原子からなる。 ▲水素と酸素が結合して水になる。
\\	詩	し	
\\	文学の様式の一。 自然や人事などから受ける感興・感動を、リズムをもつ言語形式で表現したもの。 押韻・韻律・字数などに規定のある定型詩と、それのない自由詩・散文詩とがあり、また、内容から叙情詩・叙事詩・劇詩などに分ける。 
\\	漢詩。 [類語]
\\	うた・詩歌・韻文・詩賦(しふ)・賦(ふ)・吟詠・ポエム・バース	▲その歌はどんな歌詩ですか。 ▲その詩はみながよく知っている。
\\	氏	し	㊀[名] 
\\	同一血族の系統。 うじ。 
\\	話し手・相手以外の第三者。 代名詞的に用いる。 「―は静養中」 ㊁〔接尾〕 
\\	氏名に付けて敬意を表す。 主として男子に用いる。 「佐藤―は欠席」 
\\	氏族の姓氏に付けて、その氏族の出身であることを表す。 「藤原―」 
\\	助数詞。 敬意をこめて人数を表すのに用いる。 「三―の御執筆」	▲ホワイト氏は理性的な人だ。 ▲ホワイト氏は彼らを助けてあげたいと思いました。
\\	幸せ・仕合わせ・倖せ	しあわせ	《動詞「しあ(為合)わす」の連用形から》[名・形動] 
\\	運がよいこと。 また、そのさま。 幸福。 幸運。 「思わぬ―が舞い込む」「―な家庭」「末永くお―にお暮らしください」 
\\	めぐり合わせ。 運命。 「―が悪い」「道がわかんねえで困ってると、―よく水車番に会ったから」 
\\	運がよくなること。 うまい具合にいくこと。 「―したとの便りもなく」 
\\	物事のやり方。 また、事の次第。 「その科(とが)のがれず、終(つひ)には捕へられて此の―」 [類語]
\\	幸福・幸(さち)・幸(さいわ)い・福・幸運・福運・果報・冥加(みようが)・多幸・多祥・万福	▲由美はしあわせですね。 ▲友達が多いという点で、ナンシーは幸せだ。
\\	ジーンズ	ジーンズ	
\\	細綾織りの丈夫な綿布。 スポーツウエア・作業衣などに広く使われる。 
\\	⇒ジーパン	▲1853年に青いジーンズが初めて出現した。 ▲私に合うサイズのジーンズはありますか。
\\	ジェット機	ジェットき	ジェットエンジンを推進装置として用いる航空機の総称。	▲上空をジェット機がキーンという音を立てて飛んでいった。 ▲現在の飛行機墜落は2週間ごとにほぼ1件の割合で、あらゆる重大事故があらゆるタイプの輸送用ジェット機に起こっている。
\\	直に	じかに	[副]間にほかのものを入れないで直接にするさま。 「ワイシャツを肌に―着る」	▲あなたとじかに会ってじっくり話したいことがあるのです。 ▲英国大使は大統領とじかに会見することを要求した。
\\	然も・而も	しかも	《副詞「しか」+係助詞「も」から》 ㊀[接] 
\\	前述の事柄を受けて、さらに別の事柄を加えるときに用いる。 その上。 「あの方は私の恩師で、―命の恩人だ」 
\\	前述の事柄を受けて、それに反する帰結を付け加えるときに用いる。 それなのに。 それでも。 「あれだけ練習して、―勝てなかった」 ㊁[副]そんなにまでも。 「三輪山を―隠すか雲だにも心あらなも隠さふべしや」 [類語] 
\\	その上・かつ・かつまた・なおかつ・おまけに・加うるに・のみならず・しかのみならず・そればかりか・糅(か)てて加えて・同時に・更に・あまつさえ/
\\	それでも・なおかつ	▲君は彼女に謝らなければならぬ、しかもいますぐに。 ▲君は彼女を助けなければいけない。しかもすぐに。
\\	時期	じき	
\\	ある幅をもった時。 期間。 「入学の―」「―が重なる」 
\\	その時。 そのおり。 「―が来ればわかる」	▲一年のこの時期には雪がたくさん降る。 ▲郷愁にふける時、私達は子供時代をこの上なく幸福な時期と考える傾向があるかもしれない。
\\	直	じき	㊀[名・形動] 
\\	間に人や物を置かずにすること。 また、そのさま。 直接。 じか。 「―の兄」「―にお奉行様に差し出したい」 
\\	まっすぐであること。 また、そのさま。 一直線。 「両の耳は竹を剥いで―に天を指し」 
\\	「直取引(じきとりひき)」の略。 →直(す)ぐ[用法] ㊁[副]時間的、距離的に近いさま。 すぐ。 「もう―春だ」「学校は―そばだ」 [類語] 
\\	じか・直直(じきじき)・直接/ ❷すぐ・間(ま)もなく・程なく・幾許(いくばく)もなく(述語として)間近い・程近い	▲彼女はじきよくなるだろう。
\\	支給	しきゅう	[名]スル金銭・物品を、給与・給付として払い渡すこと。 「扶養手当を―する」	▲彼は彼らに食べ物と金を支給した。 ▲従ってそれ以来、製造業者たちは本物の現金を支給しなければなりませんでした。
\\	頻りに	しきりに	
\\	しばしば。 ひっきりなしに。 しきりと。 「雪が―降っている」「―誘われる」 
\\	むやみに。 無性に。 ひどく。 しきりと。 「―家が恋しい」「―恐縮している」	▲彼女はプレゼントをしきりに望んでいる。 ▲トムはしきりに新車を買いたがっている。
\\	刺激・刺戟	しげき	[名]スル 
\\	生体に作用してなんらかの現象や反応を起こさせること。 特に、知覚や感覚に作用して反応を起こさせること。 また、その原因となるもの。 「学習意欲を―する」「都会は―が強い」 
\\	物事の動きを活発にさせるきっかけとして、外から作用すること。 また、そのもの。 「景気を―する」	▲彼女の元気な性格に刺激を受けた。 ▲彼が毎日日記を付けているのに刺激されて、私も英語で日記を付けることにしました。
\\	資源	しげん	
\\	自然から得る原材料で、産業のもととなる有用物。 土地・水・埋蔵鉱物・森林・水産生物など。 天然資源。 「海洋―」「地下―」 
\\	広く、産業上、利用しうる物資や人材。 「人的―」「観光―」	▲戦争が国の資源と労働力を使い果たしてしまった。 ▲世界的コミュニケーション産業の資源へのアクセス。
\\	事件	じけん	
\\	世間が話題にするような出来事。 問題となる出来事。 「奇妙な―が起こる」 
\\	「訴訟事件」の略。裁判所に訴えが提起されている事件。	▲私立探偵たちがその変わった事件を調査するために雇われた。 ▲私達はその事件を再調査する必要がある。
\\	時刻・時剋	じこく	
\\	時の流れにおける、ある瞬間。 連続する時間の中のある一点。 「約束の―」「列車の着く―」 
\\	ちょうどよい時。 時機。	▲もう寝る時刻ですよ。 ▲ゆうべこの町に火事があったが、燃え出した正確な時刻はわからない。
\\	自殺	じさつ	[名]スル自分で自分の命を絶つこと。 ↔他殺。	▲彼女は30歳の時に自殺した。 ▲彼女が自殺したというのは本当か。
\\	事実	じじつ	㊀[名] 
\\	実際に起こった事柄。 現実に存在する事柄。 「意外な―が判明する」「供述を―に照らす」「―に反する」「―を曲げて話す」「歴史的―」 
\\	哲学で、ある時、ある所に経験的所与として見いだされる存在または出来事。 論理的必然性をもたず、他のあり方にもなりうるものとして規定される。 ㊁[副]本当に。 実際に。 「―一度もその人には会っていない」 [類語] 
\\	真実・真相・現実・実情・実態・実際・本当・有りのまま・有り様(よう)・実(まこと)	▲私はその事実を今まで知らなかった。 ▲私はその事実を十分よく知っている。
\\	支出	ししゅつ	[名]スルある目的のために自分の所有する金銭や物品を支払うこと。 また、その金品。 「経費を―する」「―を抑える」「過年度―」↔収入。 [類語]出金・出費・出銭(でせん)・失費・掛かり・費(つい)え・物入り・支払い・歳出	▲支出は合計20万円になった。 ▲最近会社はこの様な支出を正当化できる。
\\	事情	じじょう	物事がある状態に至るまでの理由や状態。 また、その結果。 事の次第。 「やむをえぬ―があって遅れる」「―が許す限り協力する」「業界の―に通じる」「中南米諸国の―に明るい」「住宅―」「交通―」	▲事情があって私たちは会うのを取り止めねばならなかった。 ▲事情があって私には、それ以上は言えません。
\\	詩人	しじん	
\\	詩を作る人。 詩作を業とする人。 
\\	詩的な感受性を強く持っている人。 「彼はなかなかの―だ」	▲シェイクスピアは、イギリスが生んだ最大の詩人です。 ▲シェイクスピアは偉大な詩人です。
\\	自身	じしん	
\\	自分みずから。 自分。 「私が―でしたことだ」「自分―」 
\\	他の何ものでもなくそれみずからの意で、他の語に付けてそれを強調する語。 そのもの。 自体。 「彼―の問題だ」「それ―の重さ」	▲その手紙は女王自身の筆跡で書かれていた。 ▲その任務が困難であることを彼女自身が知っていた。
\\	沈む	しずむ	㊀[動マ五(四)] 
\\	水面上にあったものが水中に没する。 水底へ下降する。 また、水底につく。 「ボートが―・む」「島が―・む」↔浮かぶ/浮く。 
\\	周囲より低くなる。 「地盤が―・む」「床が―・む」 
\\	下の物にめり込む。 「柱が―・む」「頭が枕に―・む」 
\\	太陽・月などが、地平線・水平線より下へ移動する。 「夕日が水平線に―・む」 
\\	飛行する物体・投球などが急にその位置を下げる。 「機体ががくんと―・む」「ボールが打者の手元で―・む」 
\\	㋐望ましくない境遇・状態に陥る。 「不運に―・む」「病に―・む」 ㋑その心境になりきる。 特に、暗い気持ちになる。 落ち込む。 「物思いに―・む」「悲しみに―・む」「―・んだ声」 
\\	地味で落ち着いた感じになる。 「―・んだ鐘の音」「全体に―・んだ色調の絵」 
\\	色や模様などが浮き立たなくなる。 また、存在が目立たなくなる。 「黒っぽい背広にグレーのネクタイではネクタイが―・んでしまう」 
\\	ボクシングで、打ち倒されて立てなくなる。 「マットに―・む」 
\\	ゲームなどで、最初の持ち点以下になる。 「麻雀で一万点―・む」 [可能]しずめる ㊁[動マ下二]「しず(沈)める」の文語形。 [下接句]石が流れて木の葉が沈む・浮きつ沈みつ・浮きぬ沈みぬ・思案に沈む・涙に沈む	▲鼠は沈みかかった船を見捨てる。 ▲太陽がしずんでいく。
\\	自然	しぜん	㊀[名] 
\\	山や川、草、木など、人間と人間の手の加わったものを除いた、この世のあらゆるもの。 「―に親しむ」「郊外には―がまだ残っている」 
\\	人間を含めての天地間の万物。 宇宙。 「―の営み」 
\\	人間の手の加わらない、そのもの本来のありのままの状態。 天然。 「野菜には―の甘みがある」 
\\	そのものに本来備わっている性質。 天性。 本性。 「人間の―の欲求」 
\\	哲学で、 ㋐他の力に依存せず、自らの内に生成・変化・消滅の原理を育するもの。 ㋑精神とは区別された物質的世界。 もしくは自由を原理とする本体の世界に対し、因果的必然的法則の下にある現象的世界。 経験の対象となる一切の現象。 ㊁[形動][文][ナリ] 
\\	言動にわざとらしさや無理のないさま。 「気どらない―な態度」「―に振る舞う」 
\\	物事が本来あるとおりであるさま。 当然。 「こうなるのも―な成り行きだ」 
\\	ひとりでにそうなるさま。 「―にドアが閉まる」 [派生]しぜんさ[名] ㊂[副] 
\\	ことさら意識したり、手を加えたりせずに事態が進むさま。 また、当然の結果としてそうなるさま。 おのずから。 ひとりでに。 「無口だから―(と)友だちも少ない」「大人になれば―(と)わかる」 
\\	《「自然の事」の略》もしかして。 万一。 「都へ上らばやと思ひしが、―舟なくてはいかがあるべきとて」 
\\	たまたま。 偶然。 「礫(つぶて)打ちかけしに、―と当り所悪しくそのままむなしくなりぬ」 [類語] 
\\	天然(てんねん)・森羅万象(しんらばんしよう)・天工・造化(ぞうか)・天造・原始/
\\	天地(てんち)・あめつち・山河(さんが)・山水(さんすい)・山川草木(さんせんそうもく)・生態系・ネーチュア/ 
\\	無為・素朴・有るがまま・ナチュラル/
\\	(連用修飾語として)自(おの)ずから・自(おの)ずと・ひとりでに	▲私たちの周囲の美しい自然を守ろう。 ▲私たちは自然と調和して生活することを学ばなければならない。
\\	思想	しそう	[名]スル 
\\	心に思い浮かべること。 考えること。 考え。 「新しい―が浮かぶ」「普天下の人をして自由に―し」 
\\	人生や社会についての一つのまとまった考え・意見。 特に、政治的、社会的な見解をいうことが多い。 「反体制―を弾圧する」「末法(まつぽう)―」「危険―」 
\\	哲学で、考えることによって得られた、体系的にまとまっている意識の内容をいう。 [類語]
\\	想念・思念・観念・考え/
\\	主義・理念・信条・信念・哲学・人生観・世界観・思潮・イズム・イデオロギー	▲思想は世界中をアッという間に巡る。 ▲思想は言葉によって表現されている。
\\	舌	した	
\\	口腔底から突出している筋肉性の器官。 粘膜に覆われ、非常によく動き、食物の攪拌(かくはん)・嚥下(えんげ)を助け、味覚・発音をつかさどる。 べろ。 
\\	話すこと。 言葉遣い。 弁舌。 「―を振るう」 
\\	雅楽器の篳篥(ひちりき)のリード。 二寸(約六センチ)ほどに切った蘆(あし)の一端をつぶして吹き口とし、他の一端に和紙を巻いて管に差し込んだもの。 蘆舌(ろぜつ)。	▲コーラを飲んだら舌がぴりぴりした。 ▲このからしは舌が痺れるほど辛い。
\\	次第	しだい	㊀[名] 
\\	物事が行われる際の一定の順序。 「式の―を書き出す」 
\\	今まで経過してきた状態。 なりゆき。 「事の―を話す」 
\\	物事の、そうなるに至った理由。 わけ。 事情。 「そんな―で明日は伺えない」 
\\	能や狂言の構成部分の一。 七・五、返句、七・四の三句からなる拍子に合った謡。 シテ・ワキなどの登場第一声として、また曲舞(くせまい)や乱拍子の序歌としても謡われる。 
\\	能や狂言で、シテ・ワキなどの登場に用いる囃子事(はやしごと)。 大鼓・小鼓に笛があしらい、続いて 
\\	が謡われる。 
\\	歌舞伎囃子(ばやし)の一。 
\\	を取り入れたもので、能がかりの登場音楽として用いるほか、「関の扉(と)」などの幕開きにも奏する。 ㊁〔接尾〕 
\\	名詞に付いて、その人の意向、またはその事物の事情のいかんによるという意を表す。 「あなた―でどうともなる」「この世はすべて金―」 
\\	動詞の連用形に付いて、その動作が行われるままにという意を表す。 「手当たり―に投げつける」「望み―に買い与える」 
\\	動詞の連用形または動作性の名詞に付いて、その動作がすむと直ちにという意を表す。 「満員になり―締め切る」「本が到着―送金する」	▲あなた次第です。 ▲7月10日以降なら、いつでも請求次第に支払います。
\\	事態・事体	じたい	物事の状態、成り行き。 「容易ならない―を収拾する」「緊急―」	▲君はこの重大な事態の責任を免除されていないよ。 ▲君は最悪の事態を覚悟しなくてはならない。
\\	従う・随う・順う・遵う	したがう	㊀[動ワ五(ハ四)] 
\\	後ろについて行く。 あとに続く。 「案内人に―・う」「前を行く人に―・って歩く」 
\\	沿う。 たどる。 「川の流れに―・って下る」「標識に―・って進む」 
\\	他からの働きに順応する。 ㋐法律・慣習・意見などに逆らわないでそのとおりにする。 意のままになる。 服従する。 「法の定めるところに―・う」「指示に―・って行動する」「意向に―・う」 ㋑ほかの力のままに動く。 任せる。 「風に―・う葦(あし)」 
\\	㋐応じる。 また、順応する。 「問いに―・って答える」「条件に―・って賃金が異なる」 ㋑ならう。 まねる。 「見本に―・って作る」 
\\	たずさわる。 従事する。 「業務に―・う」 
\\	(「…にしたがい」「…にしたがって」の形で)…につれて。 …とともに。 「年をとるに―・い円満さを増してきた」「登るに―・って気温が下がる」 [可能]したがえる ㊁[動ハ下二]「したがえる」の文語形。 [類語] 
\\	付く・くっつく・随行する・随伴する・随従する・追随する/
\\	服する・応ずる・則(のつと)る・則(そく)する・準ずる・倣(なら)う・拠(よ)る・準拠する・依拠する	▲その命令に嫌々従う兵もいた。 ▲その命令に従うより他に仕方ない。
\\	従って	したがって	[接]《動詞「したがう」の連用形+接続助詞「て」から》前の条件によって順当にあとの事柄が起こることを表す。 だから。 それゆえ。 「この品は手作りだ。 ―値が高い」	▲私たちは彼女の指示に従って作業を完了した。 ▲私に従ってきなさい。
\\	親しい	したしい	[形][文]した・し[シク] 
\\	互いに打ちとけて仲がよい。 懇意だ。 「家族ぐるみで―・くしている」「―・い友達」 
\\	血筋が近い。 「―・い縁者」 
\\	いつも接していて、なじみ深い。 「子供のころから耳に―・いおとぎ話」→親しく [派生]したしげ[形動]したしさ[名] [類語]
\\	近しい・心安い・気安い・睦(むつ)まじい・親密・懇意・昵懇(じつこん)・懇親・別懇・懇(ねんご)ろ・仲が良い・気が置けない	▲彼とは親しい間柄だ。 ▲彼とはあまり親しくするなよ。
\\	質	しつ	
\\	そのものの良否・粗密・傾向などを決めることになる性質。 実際の内容。 「量より―」「―が落ちる」 
\\	生まれながらに持っている性格や才能。 素質。 資質。 「天賦の―に恵まれる」「蒲柳(ほりゆう)の―」 
\\	論理学で、判断が肯定判断か否定判断かということ。 
\\	物の本体。 根本。 本質。 「結合せるを―とし、流動するを気とす」 
\\	飾りけのないこと。 素朴なこと。 「古今集の歌よりは―なり」	▲考慮すべきもう一つのことは、カーペットの素材、織り方、染料の質である。 ▲現代は質に関係なく、量を求める。
\\	失業	しつぎょう	[名]スル 
\\	職を失うこと。 失職。 「会社が倒産して―する」 
\\	労働者が労働の能力と意欲とを持ちながら、労働の機会を得ることのできない状態。 「潜在―」 [類語]失職・食い上げ・お払い箱(―する)食いはぐれる・あぶれる	▲警察は失業中の者を何名か集めた。 ▲何故彼は失業したと思いますか。
\\	実験	じっけん	[名]スル 
\\	事柄の当否などを確かめるために、実際にやってみること。 また、ある理論や仮説で考えられていることが、正しいかどうかなどを実際にためしてみること。 「化学の―」「―を繰り返す」「新製品の効能を―する」 
\\	実際に経験すること。 「自家(おのれ)が―せざる事実は、決して穿(うが)ちがたきものとや思える」 [類語]
\\	試験・試行・テスト・エクスペリメント	▲その実験は惨めな失敗に終わった。 ▲その実験は結局成功した。
\\	実現	じつげん	[名]スル計画・期待などが現実のものになること。 また、現実のものとすること。 「―不可能な要求」「夢が―する」「要求を―する」	▲彼女の望みは実現したようだ。 ▲彼女の夢の実現を助けたのはジョンだった。
\\	実行	じっこう	[名]スル 
\\	実際に行うこと。 「計画を―に移す」「予定どおりに―する」 
\\	刑法で、故意をもって犯罪を構成する要件にあたる行為を行うこと。 →実践(じつせん)[用法] [類語]
\\	実践・行動・躬行(きゆうこう)・励行・履行・実施・施行(しこう)・執行・決行・敢行・断行・遂行	▲私達はみな、その計画を実行するのは難しいと思った。 ▲私達はそれを実行不可能と思ったことはない。
\\	実際	じっさい	㊀[名] 
\\	物事のあるがままの状態。 「老人医療の―に目を向ける」「―は経営が苦しい」 
\\	想像や理論でなく、実地の場合。 「―に応用する」「―にあった話」「―問題」 
\\	仏語。 真如、または無余涅槃(むよねはん)のこと。 存在の究極的な姿。 ㊁[副]ほんとに。 実に。 確かに。 「あの時は―だめだと思った」「―やってみると難しい」 [類語] 
\\	実地・現実・本当・実情・実相・実態・実像・内実・事実	▲原則として、客車に駐車場がなくてはならないが、実際にはあり得ない。 ▲包装によって実際にある種の無駄を防ぐことが出来る。
\\	実施	じっし	[名]スル法律・計画などを実際に行うこと。 「試験を―する」「―要綱」→実践(じつせん)[用法]	▲乙が受託業務の実施により得た成果は、甲乙双方に帰属するものとする。 ▲金融引き締め政策が実施されている。
\\	実は	じつは	[副]事実を言えば。 本当のところは。 打ち明けて言うと。 「―私が企てた事なのです」	▲実はこれで4度目の質問になります。 ▲実は、大量の資料を持ってくるのは、反論させないための姑息な手段である。
\\	失望	しつぼう	[名]スル期待がはずれてがっかりすること。 また、その結果、希望を持てなくなること。 「―の色を隠せない」「前途に―する」	▲私たちは多かれ少なかれ選挙の結果に失望した。 ▲私たちは彼に失望した。
\\	支店	してん	本店とは別の場所にあって、その指揮・命令を受ける営業所。 ↔本店。	▲支店は全国にまたがっている。 ▲支店を出すなどという考えを彼はどこで思い付いたのだろうか。
\\	指導	しどう	[名]スルある目的・方向に向かって教え導くこと。 「演技の―にあたる」「―を受ける」「人を―する立場」「行政―」 [類語]導き・教え・手引き・指南・教授・教育・訓育・教導・補導(ほどう)・善導・誘掖(ゆうえき)・鞭撻(べんたつ)	▲その節はよろしくご指導のほどお願いいたします。 ▲彼女は慈善伝導団と呼ばれる修道女達の集団を指導していた。
\\	自動・自働	じどう	
\\	他の力によらず、みずからの力で動くこと。 特に、機械・装置についていう。 「―ドア」 
\\	特別に手を加えなくても自然にそうなること。 「条約の―延長」「―消滅」 
\\	「自動詞」の略。	▲ほとんどのエレベーターは自動で動く。 ▲自動ドアが開き、トムは乗り込んだ。
\\	品・科・階	しな	
\\	(品)何かに使用する、形のあるもの。 品物。 「見舞いの―」「結構なお―」 
\\	(品)商品。 「良い―を安く売る」「―が豊富な店」「―ぞろえ」 
\\	物の品質。 「―が落ちる」 
\\	材料や品質の良し悪しによって分けた種類。 等級。 「―分け」 
\\	(科)ちょっとした媚(こび)を含んだ身ぶりやしぐさ。 特に、女が男に見せるようす・態度についていう。 「―をする」 
\\	地位。 身分。 家柄。 「人の―高く生まれぬれば」 
\\	人の品格。 人柄。 品位。 「さぶらふ中に―心すぐれたる限りを選(え)りて」 
\\	きざはし。 階段。 「御階(みはし)の中の―のほどに居給ひぬ」 
\\	物事の事情、立場。 「徳様も死なねばならぬ―なるが」	▲海外の子会社は最高級品を生産しています。 ▲彼女にはどことなく品がある。
\\	支配	しはい	[名]スル 
\\	ある地域や組織に勢力・権力を及ぼして、自分の意のままに動かせる状態に置くこと。 「異民族の―から脱する」「諸国を―する」 
\\	ある要因が人や物事に影響を及ぼして、その考えや行動を束縛すること。 「先入観に―される」「物体は引力に―されている」 
\\	仕事を配分したり監督・指揮したりして、部下に仕事をさせること。 「宇治の大臣(おとど)、成佐が弟子どもに―して、一日に三尺の地蔵菩薩の像を図絵し」 [類語]
\\	統治・君臨・制覇・制圧・征服・圧伏・管理・管轄・統轄・統御・統率・宰領(さいりよう)・監督・統制・取り締まり(―する)統(す)べる・制する・領する・握る・牛耳(ぎゆうじ)る/
\\	束縛・拘束・規制・制約(―する)左右する・縛る	▲その王様は何年もその国を支配した。 ▲その男は50年にわたってその国を支配した。
\\	芝居	しばい	
\\	などに由来》歌舞伎などの興行物。 しばや。 「―好き」「―通」 
\\	役者などが演技をすること。 また、その演技。 「いい―をする」 
\\	計画的に人をだますためのこしらえごと。 狂言。 「ひと―打つ」 
\\	芝生に席を設けて座ること。 また、芝生。 「搦手(からめて)は―の長酒盛(さかもり)にてさてやみぬ」 
\\	勧進の猿楽・曲舞(くせまい)・田楽などで、舞台と桟敷との間の芝生に設けた庶民の見物席。 
\\	歌舞伎など有料の興行物の見物席。 特に桟敷に対して、大衆の見物席をいう。 「舞の―で同じ莚(むしろ)に居たる人」 [下接語]操り芝居・田舎芝居・大芝居・御伽(おとぎ)芝居・戯(おど)け芝居・女芝居・陰芝居・歌舞伎芝居・紙芝居・絡繰(からく)り芝居・草芝居・首掛け芝居・小(こ)芝居・子供芝居・薦(こも)張り芝居・猿芝居・地(じ)芝居・書生芝居・素人芝居・壮士芝居・旅芝居・辻(つじ)芝居・道化芝居・緞帳(どんちよう)芝居・人形芝居・初芝居・一人芝居・宮芝居・村芝居	▲芝居をよく見に行きますか。 ▲芝居は面白かったですか。
\\	屡・屡々	しばしば	[副]同じ事が何度も重なって行われるさま。 たびたび。 「この種の事件は―起こる」→度度(たびたび)[用法]	▲彼はその時より前に、しばしばヨーロッパを訪れていた。 ▲彼はしばしば恋に落ちる。
\\	芝生	しばふ	芝が一面に生えている所。 また、庭などで芝が植えてある所。 「―に寝そべる」	▲春の日差しを浴びて芝生に座っているのはとてもすてきでした。 ▲芝生立ち入り禁止。
\\	支払い	しはらい	
\\	品物やサービスなどに対して、金銭を払い渡すこと。 「カードで―をする」「―を済ませる」 
\\	金銭債務の履行として金銭を渡すこと。 ◆国庫からの支払いにはもと「仕払い」の字を用いた。	▲お支払いは60日以内という条件だったと思いますが。 ▲お支払いはどのようにしますか。
\\	支払う	しはらう	[動ワ五(ハ四)]《「し」は動詞「す」の連用形。 「支」は当て字》代金・料金を払い渡す。 金銭の支払いをする。 「ガス代を―・う」 [可能]しはらえる	▲私の父が先月死んだとき、私が借金を支払う金しか残さなかった。 ▲私は1個につき彼らに千円支払った。
\\	死亡	しぼう	[名]スル人が死ぬこと。 死去。 「事故で―する」「―届け」	▲その墜落事故で乗っていた乗客は全員死亡した。 ▲その墜落事故で乗客は全員死亡した。
\\	資本	しほん	
\\	商売や事業をするのに必要な基金。 もとで。 
\\	生産の三要素(労働・土地・資本)の一。 新たな生産のために投入される、過去の生産活動が生みだした生産物のストック。 
\\	資本制生産で、剰余価値を生むことによって自己増殖を行う価値の運動体。 
\\	簿記で、企業の資産総額から負債総額を差し引いた純資産。 自己資本。	▲資本、土地、労働は生産の三大要素である。 ▲健康が彼の資本だ。
\\	姉妹	しまい	
\\	姉と妹。 女のきょうだい。 「三人―」 
\\	同じ系統に属し、互いに類似点または共通点をもっている二つ以上のもの。 「―品」「―校」	▲彼女の4人の姉妹のうち1人は他界したが、ほかは健在だ。 ▲彼女には姉妹が3人あり、1人は看護婦で残る2人は先生です。
\\	仕舞った	しまった	[感]《動詞「しまう」の連用形+完了の助動詞「た」から》失敗したときに思わず発する語。 「―、間に合わなかった」	
\\	自慢	じまん	[名]スル自分で、自分に関係の深い物事を褒めて、他人に誇ること。 「―ののど」「成績を―する」 [派生]じまんげ[形動]	▲彼女は1等をとったことを自慢した。 ▲彼女はアメリカ大統領と握手したことを自慢していた。
\\	事務	じむ	役所・会社などで、書類・帳簿の作成・処理など、主として机の上でする仕事。 「会社で―を執る」「―を引き継ぐ」	▲辞任劇は来るべき徴兵事務にはさして影響はないであろう。 ▲退職する前に彼は事務を渡した。
\\	示す	しめす	[動サ五(四)] 
\\	相手によくわかるように、出して見せたり、自分で何かをして見せたりする。 「定期券を―・す」「身をもって―・す」 
\\	指などでさして教える。 「地図を出して―・す」 
\\	計器・時計などが、ある目盛りを指す。 「寒暖計が三〇度を―・す」 
\\	考え・気持ち・反応などが相手に伝わるように、何かの方法で表して見せる。 「誠意を―・す」「態度で―・す」「格段の進歩を―・す」 
\\	ある現象が、物事の状態・傾向を表す。 「気圧配置が今年の暖冬を―・している」 
\\	さとし戒める。 「宿の男などとの事は末の名の立つをひそかに―・し」 [可能]しめせる [類語]
\\	見せる・呈示する・提示する・開示する・明示する・表示する・掲げる/
\\	指す・指し示す・指示する/
\\	表す・見せる/
\\	見せる・表す	▲ボーイング社の分析は過去10年間のあらゆる事故の60%以上が乗務員の行動が主要な原因だったことを示している。 ▲まずは自ら見本を示す。そういった率先垂範の気持ちがなければ誰も君には付いてこないよ。
\\	占める	しめる	[動マ下一][文]し・む[マ下二] 
\\	あるもの・場所・位置・地位などを自分のものとする。 占有する。 「三賞を一人で―・める」「国際経済の中で重要な役割を―・める」「業界トップの座を―・める」「連番で座席を―・める」 
\\	全体の中である割合をもつ。 「賛成が過半数を―・める」「ビルの八割をテナントが―・める」 
\\	(「味をしめる」の形で)体験して、うまみを知る。 良さを知って、次を期待する。 「一度味を―・めたらやめられない」 
\\	《自分のものにするところから》食べる。 「すき焼きを―・めたあとで、葱(ねぎ)の湯どおしをあがってごろうじろ」 
\\	ある才能・性質などを備える。 「いとあはれと人の思ひぬべきさまを―・め給へる人柄なり」	▲水は地球の表面の大部分を占めている。 ▲選挙の結果その党は政権の一角を占めた。
\\	霜	しも	
\\	氷点下に冷却した地面や地上の物体に、空気中の水蒸気が触れて昇華してできる氷の結晶。 風の弱い、晴れた夜にできやすい。 《季 冬》「―のふる夜を菅笠のゆくへ哉/竜之介」 
\\	使用中の電気冷蔵庫の内側に付着する細かい氷。 
\\	白髪をたとえていう語。 「頭髪に―を交える」	▲冬の朝には車の窓に霜が沢山つく。 ▲夕べ霜が降りた。
\\	下	しも	
\\	ひと続きのものの末。 また、いくつかに区別したものの終わりの部分。 ㋐川の下流。 また、その流域。 川下。 「―へ漕ぎ下る」「―で釣る」↔上(かみ)。 ㋑時間的にあとと考えられるほう。 現在に近いほう。 後世。 「上は太古の昔から―は現在ただ今まで」↔上(かみ)。 ㋒ある期間を二つに分けた場合のあとのほう。 「―の半期」↔上(かみ)。 ㋓月の下旬。 「寄席の―に出演する」 ㋔物事の終わりの部分。 末の部分。 「詳しくは―に記す」「―二桁(けた)は切り捨て」「―の巻」↔上(かみ)。 ㋕和歌の後半の二句。 「―の句」↔上(かみ)。 
\\	位置の低い所。 また、低いと考えられる所。 ㋐下方に位置する所。 下部。 「―の田に水を落とす」「外(と)のかたを見いだしたれば、堂は高くて―は谷と見えたり」↔上(かみ)。 ㋑からだの腰から下の部分。 また、特に陰部や尻をさすことが多く、それを話題にする下品さや、大小便に関する事柄をもいう。 「―の病気」「話が―へ下る」「―の世話をする」「―半身」↔上(かみ)。 ㋒下位の座席。 下座。 末座。 末席。 「幹事役が―に控える」↔上(かみ)。 ㋓客間・座敷などに対して、台所・勝手などをさす語。 ↔上(かみ)。 ㋔舞台の、客席から見て左のほう。 下手(しもて)。 「斬られた役者が―に引っ込む」↔上(かみ)。 
\\	地位・身分の低い人。 君主に対して、臣下・人民。 雇い主に対して、使用人・召し使い。 「―の者をいたわる」「夫を待(あつか)う塩梅(あんばい)、他(ひと)に対するから―に臨む調子」「上(かみ)は―に助けられ、―は上になびきて」 
\\	中心から離れた地。 ㋐都から離れた地。 特に、京都から離れた地方。 ↔上(かみ)。 ㋑京都で、御所から離れた南の方角・地域。 転じて一般に、南の方の意で地名などに用いる。 「寺町通りの―にある家」「―京(しもぎよう)」↔上(かみ)。 ㋒他の地域で、より京都に遠いほう。 昔の国名などで、ある国を二分したとき、都から見て遠いほう。 「―関(しものせき)」「―つふさ(=下総(しもうさ))」↔上(かみ)。 ㋓京都から見て、中国・四国・九州などの西国地方。 特に、キリシタン関係書では九州をさす。 
\\	格や価値が劣っているほう。 「上(かみ)中(なか)―の人」 
\\	㋐宮中や貴人の家で、女房が詰めている局(つぼね)。 「腹を病みて、いとわりなければ、―に侍りつるを」 ㋑《下半身につけるところから》袴(はかま)。 「―ばかり着せてやらう」	
\\	借金	しゃっきん	[名]スル金銭を借りること。 また、借りた金銭。 借財。 借銭。 「―で首が回らない」「―を踏み倒す」「―して車を買う」	▲私の親父が先月死んだ時、私が借金を払う金しか残さなかった。 ▲私の父が先月死んだとき、私が借金を支払う金しか残さなかった。
\\	シャベル	シャベル	土砂・石炭・雪などをすくったり掘り起こしたりするのに用いる、さじ状の道具。 ショベル。 →スコップ	▲彼らは歩道の雪をシャベルで片付けていた。 ▲彼等はシャベルで除雪していた。
\\	喋る	しゃべる	[動ラ五(四)] 
\\	物を言う。 話す。 「一言も―・らない」「君のことをうっかり―・ってしまった」 
\\	口数多く話す。 口に任せてぺらぺら話す。 「よく―・る人だ」 [可能]しゃべれる	▲彼女は魅力的だがしゃべりすぎる。 ▲彼女は彼のそういうしゃべりかたにスリルを感じたんですよ。
\\	州・洲	しゅう	㊀[名] 
\\	アメリカ・オーストラリアなどの連邦国家を構成する行政区画。 「―の法律」「―政府」 
\\	日本で古くから用いた地域単位としての国(くに)。 「甲州」「上州」など。 
\\	古代中国の行政区画の一。 漢の武帝が郡県の上に一三州を置いたのに始まる。 のち、しだいに細分化されて郡との差がなくなり、近代になって廃止。 
\\	地球上の地を大陸で区分していう称。 「五大―」「大洋―」 ㊁〔接尾〕近世、人名などに付いて、親愛の意を表す。 「何、野(や)―には手管なしとや」	▲若きマーテインは、ジョージア州アトランタで、比較的平穏な子供時代を過ごした。 ▲私達の先生は、私達の学校が州で一番だというが、ある意味でそれは本当のことだ。
\\	週	しゅう	日曜日から土曜日までの七日を一期とした時間の単位。	▲彼は週に2回残業をする。 ▲彼は週に2、3日働いて少しお金をもらう。
\\	周	しゅう	㊀[名]数学で、図形を囲む閉じた曲線または折れ線。 また、その長さ。 円の場合は円周という。 ㊁〔接尾〕助数詞。 あるもののまわりをまわる回数を数えるのに用いる。 「トラックを三―する」	▲飛行機は離陸後に空港を二周した。 ▲その衛星は地球の軌道を10周した。
\\	銃	じゅう	弾丸を発射する装置をもつ小型の武器。 砲に対して、口径の小さい拳銃・小銃・機関銃などの総称。 また、それに似た形・用途のもの。 「―を構える」「水中―」	▲気をつけろ、その男は銃を持っている。 ▲学校での銃乱射事件はこの半年で3回目だ。
\\	周囲	しゅうい	
\\	もののまわり。 ぐるり。 また、周辺。 「―を木でかこまれた家」 
\\	まわりの長さ。 外周。 「―五キロの島」 
\\	円周の長さ。 
\\	まわりの人や事物。 「子供は―の影響を受けやすい」	▲周囲に敵影ありません。 ▲その湖は周囲が5キロメートルある。
\\	収穫	しゅうかく	[名]スル 
\\	農作物をとりいれること。 また、とりいれたもの。 「―が多い」「―の秋」「米を―する」 
\\	何かをすることで得られた成果。 「たいした―もなく取材から帰る」	▲好天のおかげで作物すべてを一日で収穫できた。 ▲今年のトウモロコシの収穫は素晴らしい。
\\	宗教	しゅうきょう	
\\	神・仏などの超越的存在や、聖なるものにかかわる人間の営み。 古代から現代に至るまで、世界各地にさまざまな形態のものがみられる。 →原始宗教 →民族宗教 →世界宗教 [類語]宗門・宗旨・信教・信仰・信心・敬神	▲宗教がらみの裁判で、野心的な弁護士は教団の指導者の代理をする。 ▲宗教と政治について論じ合うことは避けたほうがよい。
\\	重視	じゅうし	[名]スル重要なものとして注目すること。 「実績を―する」↔軽視。	▲彼らは私の意見を重視しなかった。 ▲彼を重視しているのですか。
\\	就職	しゅうしょく	[名]スル職業につくこと。 新しく職を得て勤めること。 「地元の企業に―する」「―試験」	▲彼は就職を丁重に断った。 ▲彼は就職のチャンスに飛びついた。
\\	ジュース	ジュース	果物や野菜の絞り汁。 果汁。 また、それを薄めて砂糖などを加えた清涼飲料水。 食品の表示基準では果汁一〇〇パーセントのものをいう。	▲ウェートレスはジュースを私の前に置いた。 ▲だいたい何でこんな真夜中にジュース買う為にパシらされなきゃなんないんだか・・・。
\\	修正	しゅうせい	[名]スル不十分・不適当と思われるところを改め直すこと。 「文章の誤りを―する」「―案」「軌道―」	▲・テキストデータの誤字脱字を修正。 ▲とある映画を文庫化した―いや、映画の為に書かれたシナリオを小説として加筆修正し、日本語にローカライズしたものだ。
\\	渋滞	じゅうたい	[名]スル物事がとどこおってすらすらと進まないこと。 つかえて流れないこと。 「事務が―する」「交通―」	▲交通事故があって、1インチも動かない渋滞になってしまいました。 ▲高速道路が渋滞している。
\\	重大	じゅうだい	[名・形動] 
\\	事柄が普通でなく、大変な結果や影響をもたらすような状態であること。 また、そのさま。 重要。 「事の―を感じる」「―な局面」「―発表」 
\\	軽々しく扱えない、大切な事柄であること。 また、そのさま。 「―な役割」 
\\	重く大きいさま。 「―なる舟車を自由に進退す可し」 [派生]じゅうだいさ[名]	▲重大な罪については、指導者に告白しなければ赦しを受けることができないとも教えています。 ▲万一重大な危機が生じたら、政府はすばやく行動しなければならないだろう。
\\	住宅	じゅうたく	
\\	人が住むための家。 住居。 すまい。 すみか。 「―地」「公営―」 
\\	すみかとすること。 住みつくこと。 「六条の三筋町に―しけり」	▲今私の家の付近に住宅が続々建っている。 ▲私は住宅ローンで苦しんだ。
\\	集団	しゅうだん	
\\	人や動物、また、ものが集まってひとかたまりになること。 また、その集まり。 群れ。 「―で登校する」「野生馬が―をつくる」「―生活」 
\\	なんらかの相互関係によって結ばれている人々の集まり。 「政治―」「演劇―」	▲混戦模様となったレースだが、四宮は集団をラップしたこともあり、終盤、確実にタイミングよくポイントを重ね優勝した。 ▲霊長類の毛づくろいは集団の結合を強める。
\\	集中	しゅうちゅう	[名]スル 
\\	一か所に集めること。 また、集まること。 「精神を―する」「質問が―する」 
\\	ある作品集や文集の中。 「この描写は―の圧巻だ」	▲私はその講義に集中した。 ▲私はその講義に注意を集中した。
\\	収入	しゅうにゅう	金銭や物品を他から収め入れて自分の所有とすること。 また、その金品。 「安定した―を得る」「臨時―」↔支出。 [類語]所得・入金・収益・実入り・入(い)り・稼ぎ・実収・現収・月収・年収・歳入・インカム	▲著述から収入を受けている。 ▲第3四半期は収入減が見込まれている。
\\	住民	じゅうみん	ある一定の地域内に居住している人。	▲政府が住民に問う一般投票を実施しました。 ▲一部の住民が様子見の態度を取る一方で他の者は大洪水に備えた。
\\	重要	じゅうよう	[名・形動]物事の根本・本質・成否などに大きくかかわること。 きわめて大切であること。 また、そのさま。 「戦略上―な地域」「―性」 [派生]じゅうようさ[名]	▲特許権は重要な財産権である。 ▲特に重要なことは伝統的価値観を厳守することである。
\\	修理	しゅうり	[名]スル壊れたり傷んだりした部分に手を加えて、再び使用できるようにすること。 修繕。 「時計を―に出す」「車を―する」	▲私のカメラは修理の必要はない。 ▲私に修理させて下さい。
\\	主義	しゅぎ	
\\	持ちつづけている考え・方針・態度など。 「それが僕の―だ」「完全―」「菜食―」 
\\	思想・学説・芸術理論などにおける一定の立場。 イズム。 「実存―」「自然―文学」 
\\	特定の原理に基づく社会体制・制度など。 「資本―」	▲アメリカの修正論主義者は日本との関係について強硬な態度をとっています。 ▲うそをつくことは私の主義に反する。
\\	宿泊	しゅくはく	[名]スル自宅以外の所に泊まること。 「親類の家に―する」「―所」	▲このホテルの宿泊料金はいくらですか。 ▲この重要な7月のDCA会議にご出席いただき、さらにご宿泊中に東京の多様な魅力もお楽しみくださるよう希望しています。
\\	手術	しゅじゅつ	[名]スル 
\\	医者がメスなどを用い、患部を切開したり切断・摘出したりして回復させる治療法。 オペ。 
\\	物事を大幅に改めること。 「旧来の機構に大―を施す」 
\\	手段。 方法。 「何卒して―を用い」	▲半年前に右目の白内障の手術をしました。 ▲胆のうの手術でしたら、この廊下をずっと行って右に曲がってください。
\\	首相	しゅしょう	《内閣の首席の大臣の意》内閣総理大臣の通称。	▲彼は首相を辞めざるを得なかった。 ▲彼は名目上では首相だが、実際はそうではない。
\\	手段	しゅだん	ある事を実現させるためにとる方法。 てだて。 「―を講じる」「目的のためには―を選ばない」「強硬―」「生産―」 [用法]手段・方法――「患者の生命を救うための手段(方法)を考える」「相手に自分の意志を伝える有効な手段(方法)」など、目的を実現するためのやり方の意では相通じて用いられる。 
\\	「生産の手段」というと、原料・道具・建物などをさし、それらを使って物を生産するやり方が「方法」となる。 「強行手段に訴える」といえば、交渉を一方的に打ち切ったり、武力を用いたりすることで、これを「強硬方法に訴える」とは普通はいわない。 
\\	手段は具体的な行為・方策をさし、「方法」は一つ一つの手段を総合して効果的に動かすやり方をさすといえる。 
\\	類似の語「手だて」は、「もはやほかに手だてはない」のように、やり方の意で用いられる。	▲言語は人間の思想を伝達手段である。 ▲言語は人々が他人と伝達し合うのに使う手段である。
\\	主張	しゅちょう	[名]スル自分の意見や持論を他に認めさせようとして、強く言い張ること。 また、その意見や持論。 「―を通す」「自説を―する」	▲彼は自分の意見を強固に主張した。 ▲彼は自分の意見を頑固に主張した。
\\	出身	しゅっしん	
\\	その土地・身分などの生まれであること。 その学校・団体などから出ていること。 「九州の―」「民間―の閣僚」「―校」 
\\	官に挙げ用いられること。 出世すること。 「名利を志し―を宗として」	▲どこの出身であっても問題ではない。 ▲私の妻はスミス家の出身でした。
\\	出版	しゅっぱん	[名]スル印刷その他の方法により、書籍・雑誌などを製作して販売または頒布すること。 「児童書を―する」「自費―」「―社」 [類語]上梓(じようし)・上木(じようぼく)・版行・刊行・発刊・公刊・印行・発行・発兌(はつだ)・刊	▲彼は物理学の著書を出版した。 ▲彼女の詩集が出版されたところだ。
\\	首都	しゅと	その国の中央政府のある都市。 首府。	▲京都は以前日本の首都でした。 ▲京都はかつて日本の首都でした。
\\	主婦	しゅふ	一家の家事の切り盛りをする女性。 「専業―」	▲日本の主婦の中には主人に構わずにおいて満足している人もいる。 ▲物価が高いと不平を言う主婦が多い。
\\	主要	しゅよう	[名・形動]いろいろある中で特に大切なこと。 また、そのさま。 「―なメンバー」「―な事項」「世界の―都市」	▲これは主要な決定要素が存在していない興味深い例である。 ▲シカゴは、米国中西部の主要な都市である。
\\	需要	じゅよう	
\\	もとめること。 いりよう。 「人々の―に応じる」 
\\	家計・企業などの経済主体が市場において購入しようとする欲求。 購買力に裏づけられたものをいう。 ↔供給。	▲私に関する限り、その問題は需要でない。 ▲需要が増すにつれて、値段が上がる。
\\	種類	しゅるい	ある基準でみて性質・形態などが共通するものを分類し、それぞれのまとまりとしたもの。 「どういう―の本ですか」「蝶は非常に―が多い」	▲素晴らしく富んだ気候のお陰で合衆国はほとんどのあらゆる種類のスポーツの天国になっている。 ▲全ての書物は二種類に分類できると言ってよい。
\\	順	じゅん	[名・形動] 
\\	ある基準に従った、物事の配列。 順序。 順番。 「―を追って話す」 
\\	物事の行われる段取りが正当・順当であること。 理にかなっていること。 また、そのさま。 「お年寄りには席を譲るのが―だ」↔逆。 
\\	素直でおとなしいこと。 また、そのさま。 「―な若者」「女人は―を以て道とす」	▲我々は大きさの順に並べた。 ▲我々は大きさの順に本を並べた。
\\	瞬間	しゅんかん	《またたきをする間の意》きわめて短い時間。 またたく間。 また、何かをした、そのとたん。 「事故は―の出来事だった」「声を聞いた―子どものころを思い出した」「決定的―」	▲彼は入院した瞬間から、いつ家に戻れるか教えてほしいと主治医に尋ね、困らせ続けた。 ▲彼らの最良の瞬間は、最後の舞台である。
\\	順調	じゅんちょう	[名・形動]物事が調子よく運ぶこと。 とどこおりなくはかどること。 また、そのさま。 順潮。 「―な売れゆき」「経過は―だ」 [派生]じゅんちょうさ[名]	▲これまでのところすべてが順調だ。 ▲これまでのところすべて順調だ。
\\	順番	じゅんばん	順序に従って代わる代わるそのことに当たること。 また、その順序。 「―を待つ」	▲順番に書きましょう。 ▲順番をお待ち下さい。
\\	章	しょう	
\\	文章や楽曲などの全体の構成の中で、大きく分けた区分。 「―を改める」 
\\	ひとまとまりの文章。 
\\	しるし。 記章。 「会員の―」 
\\	古代中国の文体の名。 上奏文の一様式。 
\\	古代中国の暦で、一九年のこと。	▲あなたはその章の終わりまでにその意味を推測してしまっているでしょう。 ▲この章ではその惑星の謎に焦点をあてます。
\\	賞	しょう	功績をあげた者に与える褒美。 また、そのしるしの金品。 「―を受ける」「ノーベル―」↔罰。	▲私達はお互いにその賞を競った。 ▲私の娘のケイトは歌のコンテストで賞を取りました。私は彼女を誇りに思います。
\\	小	しょう	
\\	小さいこと。 重要さの程度の少ないこと。 また、そのもの。 「大は―を兼ねる」「―宇宙」「―企業」↔大。 
\\	一か月の日数が、陰暦で三〇日、陽暦では三一日に満たない月。 「―の月」↔大。 
\\	田畑の面積の単位。 太閤検地以前は一二〇歩(約四アール)、以後は一〇〇歩(約三・三アール)。 
\\	「小学校」の略。 「―・中・高・大」 
\\	名詞の上に付く。 ㋐似ているが規模の小さいものである意を表す。 「―京都」 ㋑同名の父子のうち、息子のほうを表す。 「―デュマ」↔大。	▲大か、小か。
\\	使用	しよう	[名]スル人や物を使うこと。 「会議室を―するには許可がいる」「ストロボの―は御遠慮ください」 [類語]利用・活用・運用・行使・使役(しえき)	▲コンピューターの使用は急速に増加しつつある。 ▲会議室は現在使用中です。
\\	上	じょう	㊀[名] 
\\	質の程度・価値・等級・序列などが高いこと。 標準よりすぐれていること。 また、その記号にも使う。 「中の―の生活」「握りずしの―をたのむ」↔下。 
\\	本を二冊または三冊に分けたときの、第一冊。 上巻。 「―の巻」↔下。 
\\	「上声(じようしよう)」に同じ。 
\\	進物などの包み紙に書く語。 「奉る」「差し上げます」の意。 ㊁〔接尾〕名詞に付いて、…に関して、…の面で、…の上で、などの意を表す。 「一身―の都合」「経済―の理由」「行きがかり―そうせざるを得なかった」	▲われわれの芸術上の好みは一致する。 ▲私どもの製品についての詳細な情報は、インターネット上のhttp://www
\\	でご覧になれます。
\\	障害・障碍・障礙	しょうがい	[名]スル 
\\	さまたげること。 また、あることをするのに、さまたげとなるものや状況。 しょうげ。 「旧弊が改革の―になる」「―を乗り越える」「電波―」「立憲公議の美政を組織せんと欲せば、之を―し是を非難し」 
\\	身体上の機能が十分に働かないこと。 「胃腸―」「言語―」 
\\	「障害競走」の略。馬術競技・競馬で、障害物を設けた場所で行う競走。障害レース。	▲彼は予期せぬ障害に出会った。 ▲文字を大きくし、文字間や行間に余裕をもたせ、高齢者の方や、視力に障害のある方が読み易いように注意いしました。
\\	奨学金	しょうがくきん	
\\	すぐれた学術研究を助けるため、研究者に与えられる金。 
\\	奨学制度で、貸与または給付される学資金。	▲彼はその奨学金に応募した。 ▲彼は教育奨学金でやって行くのは困難なことがわかった。
\\	乗客	じょうきゃく	船舶・航空機・列車などに乗る客。 また、乗っている客。 じょうかく。 「―名簿」	▲飛行機には150名の乗客が乗っていた。 ▲彼女は白人のすぐ後に座り、彼女の後から乗ってきた白人の乗客に自分の席を譲るのを拒否した。
\\	状況・情況	じょうきょう	移り変わる物事の、その時々のありさま。 「―を見きわめる」「周囲の―」	▲状況しだいですね。でも、たいてい週に3回です。 ▲状況がさらに悪化するのではないかと心配しています。
\\	条件	じょうけん	
\\	約束や決定をする際に、その内容に関しての前提や制約となる事柄。 「―をのむ」「―をつける」「一日だけという―で借りる」「―のいい会社」 
\\	ある物事が成立・実現するために必要な、または充分な事柄。 「いつ倒産してもおかしくない―がそろっている」「一定の―を満たす物件」「―が整う」 
\\	法律行為の効力の発生または消滅を、発生するかどうか不確定な将来の事実にかからせる付款(ふかん)。 また、その事実。 「入学したら学費を出す」などがこれにあたる。 [類語]
\\	箇条・条項・限定・制約/
\\	前提・与件・要件・要素	▲日本のODAは返済期間30年、利率2%前後という条件の緩い円借款が大部分を占める。 ▲独裁者が部族に対しその降伏条件に無理矢理同意させた。
\\	正午	しょうご	昼の一二時。 この時刻に太陽が子午線を通過する。	▲私たちは正午前にそこへ着いた。 ▲私たちは毎日正午に昼食を食べる。
\\	正直	しょうじき	㊀[名・形動]正しくて、うそや偽りのないこと。 また、そのさま。 「―なところ自信がない」「―に非を認める」「―者」 ㊁[名] 
\\	おもりを糸で垂らして柱などの傾きを調べる道具。 
\\	桶屋(おけや)の職人などが用いる一メートルくらいの大きな鉋(かんな)。 木のほうをのせて削る。 ㊂[副]見せかけやごまかしではないさま。 率直なさま。 本当のところ。 「その計画は―不可能だ」 [派生]しょうじきさ[名] [類語] 
\\	実直(じつちよく)・実体(じつてい)・誠実・真率(しんそつ)・善良・朴直・律儀(りちぎ)・真(ま)っ直(す)ぐ	▲正直は結局損にならない。 ▲指示代名詞が多すぎて、正直、わかりづらいことこの上ない。
\\	常識	じょうしき	一般の社会人が共通にもつ、またもつべき普通の知識・意見や判断力。 「―がない人」「―で考えればわかる」「―に欠けた振る舞い」「―外れ」 
\\	の訳語として明治時代から普及。 [類語]通念・良識・思慮・分別(ふんべつ)・知識・教養・心得(こころえ)・コモンセンス	▲アジア諸国などから出稼ぎにきた外国人をメイドとして使うのが常識のようになっている。 ▲これは世界の常識であり、資源管理の大原則だ。
\\	少女	しょうじょ	
\\	年少の女子。 ふつう七歳前後から一八歳前後までの、成年に達しない女子をさす。 おとめ。 「多感な―時代」「文学―」 
\\	律令制で、一七歳以上、二〇歳(のち二一歳)以下の女子の称。 [類語]
\\	女の子・娘・小娘・童女(どうじよ)・乙女(おとめ)・乙女子(おとめご)・女子(おなご)・ガール・ギャル	▲少女は悲しみにうちひしがれた。 ▲少女は彼に本当のことを言いたかったのだが、言えなかった。
\\	少少・小小	しょうしょう	
\\	すくないさま。 わずかなこと。 大鏡序「翁こそ―のことはおぼえ侍らめ」。 「―の犠牲はやむを得ない」 
\\	なみなみであること。 普通。 大方。 源氏物語蛍「―の殿上人に劣るまじ」 
\\	(副詞的に)少し。 ちょっと。 日葡辞書「ショウショウコレヲモウソウズ」。 「―ごめん下さい」	▲彼女は待たされて少々お冠です。 ▲晴耕雨読の人生にも少々は憧れるけれど、僕にはそんな生き方、3日ももたないだろうな。
\\	症状	しょうじょう	病気やけがの状態。 病気などによる肉体的、精神的な異状。 「自覚―」	▲薬を飲み始めると、すぐに痛みなどの症状はとれてきますが、すぐに潰瘍がなおるわけではありません。 ▲症状は良くなってきていますか。それとも悪くなってきていますか。
\\	生じる	しょうじる	[動ザ上一]「しょう(生)ずる」(サ変)の上一段化。 「義務が―・じる」	▲離婚の増大の結果、夫婦間、親子間に大きな不安を生じさせることは間違いない。 ▲両国の間では貿易摩擦がいつ生じてもおかしくない。
\\	状態・情態	じょうたい	人や物事の、ある時点でのありさま。 「危険な―」「昏睡(こんすい)―」「健康―」 [類語]有り様(さま)・様子(ようす)・動静・様相・模様・態様・様態・具合(ぐあい)・状況・概況・情勢・形勢・容体	▲私たちは自然をよい状態にしておかなければなりません。 ▲最高の状態ですね。
\\	上達	じょうたつ	[名]スル 
\\	《古くは「しょうたつ」》技芸・技術などがよく身につき、進歩すること。 「英会話が―する」 
\\	下の者の意見などが君主や上位の官に知られること。 「下意―」↔下達(かたつ)。	▲間違いを恐れるような人は英会話は上達しないだろう。 ▲君の英語はとても上達したと思う。
\\	冗談	じょうだん	[名・形動] 
\\	遊びでいう言葉。 ふざけた内容の話。 「―を交わす」「―を真に受ける」 
\\	たわむれにすること。 また、そのさま。 いたずら。 「―が過ぎる」「―な女どもだ。 みんな着物をかぶってくるは」	▲彼はすべて悪い冗談だと思った。 ▲彼はうまい冗談を言う。
\\	上等	じょうとう	[名・形動] 
\\	物の品質や出来ばえなどが、すぐれてよいこと。 また、そのさま。 優秀。 「―な品」「マラソンで一〇着に入れば―だ」↔下等。 
\\	等級が上であること。 等級が上のもの。 「船賃は―にて十円か十五円」 [派生]じょうとうさ[名]	▲どうしてこんな上等のぶどう酒を今まで取って置いたのか。 ▲この机はあの机と同じくらい上等です。
\\	衝突	しょうとつ	[名]スル 
\\	突き当たること。 ぶつかること。 「電柱に―する」「―事故」 
\\	相反する立場・利害などがぶつかって争いとなること。 「意見の―がみられる」「国境で軍隊が―する」	▲その問題で意見の衝突が起きた。 ▲トラックが自動車に衝突した。
\\	承認	しょうにん	[名]スル 
\\	そのことが正当または事実であると認めること。 「相手の所有権を―する」 
\\	よしとして、認め許すこと。 聞き入れること。 「知事の―を得て認可される」 
\\	国家・政府・交戦団体などの国際法上の地位を認めること。 「国連に―された国」	▲私その計画を承認することができません。 ▲私たちの決めることは何でも、委員会に承認してもらわなければならない。
\\	商人	しょうにん	
\\	商業を営む人。 あきんど。 「御用―」 
\\	商法上、自己の名をもって商行為をなすことを業とする者。	▲その結果として、彼は偉大な商人になった。 ▲その少年は頭がよかったので、商人の取引に役立った。
\\	少年	しょうねん	
\\	年が若い人。 特に、年少の男子。 ふつう、七、八歳くらいから一五、六歳くらいまでをいう。 「―の心」「―時代」 
\\	少年法などでは満二〇歳に満たない者。 児童福祉法では小学校就学から満一八歳に達するまでの者。 いずれも男子と女子を含んでいう。 [類語]
\\	男の子・童子(どうじ)・少童(しようどう)・小僧・小童(こわつぱ)・坊や・ボーイ/
\\	未成年・年少者・子供・ティーンエージャー・ティーン	▲少年たちは「万難を排して決行しましょう」と言った。 ▲毎日その少年はやってきたものでした。
\\	商売	しょうばい	[名]スル 
\\	利益をあげる目的で物を売り買いすること。 あきない。 「数軒の支店を持って―している」「客―」「―繁盛」 
\\	生活の基盤になっている仕事。 職業。 「本を書くのが―だ」「因果な―だ」 
\\	芸者・遊女などの仕事。 水商売。 [類語]
\\	商(あきな)い・小商い・営業・売買・取引・商業・商事・ビジネス/
\\	職業・仕事・稼業・生業(せいぎよう)・なりわい・飯の種	▲あの会社は先行き不安な商売をしています。 ▲あの人達はなにを商売にしているの。
\\	消費	しょうひ	[名]スル 
\\	使ってなくすこと。 金銭・物質・エネルギー・時間などについていう。 「ガスを―する」「―電力」 
\\	人が欲望を満たすために、財貨・サービスを使うこと。 「個人―」 [類語]
\\	費消・消耗・消尽・消却(―する)費やす・使う	▲消費は心持ち伸びる程度でしょう。 ▲収入が増えれば増えるほど、消費もいっそう多くなる。
\\	商品	しょうひん	売るための品物。 販売を目的とする財およびサービス。 「―を陳列する」「目玉―」「キャラクター―」 [類語]品(しな)・品物(しなもの)・物品(ぶつぴん)・売品(ばいひん)・売り物・商い物・製品・物件・グッズ	▲小娘は商品を素早く受け取ると、彼女の隣に立っていた背の小さな年寄りを指差してにっこり笑ってこう言った。 ▲商品は値段が高いためにかえってよく売れるということがしばしばある。
\\	消防	しょうぼう	
\\	火事を消し、延焼を防ぎ、また、火災・水害の警戒・予防などをすること。 
\\	「消防士」「消防団」などの略。	▲彼が言うには、消防の仕事というのは大忙しか全く暇かのいずれかだそうだ。
\\	情報	じょうほう	
\\	ある物事の内容や事情についての知らせ。 インフォメーション。 「事件についての―を得る」「―を流す」「―を交換する」「―がもれる」「極秘―」 
\\	文字・数字などの記号やシンボルの媒体によって伝達され、受け手に状況に対する知識や適切な判断を生じさせるもの。 「―時代」 
\\	生体系が働くための指令や信号。 神経系の神経情報、内分泌系のホルモン情報、遺伝情報など。 [類語]
\\	報(ほう)・報知・消息・ニュース・インフォメーション/
\\	知識・ノウハウ・データ	▲この情報は正しいか。 ▲「情報スーパーハイウェイ」の真のインパクトは、情報インフラの構築により経済が従来のハードやモノづくり中心の実体経済から知識、情報、ソフトを主体とした経済に移行し、そこから生まれる新しい産業や経済活動にある。
\\	証明	しょうめい	[名]スル 
\\	ある物事や判断の真偽を、証拠を挙げて明らかにすること。 「身の潔白を―する」「本人であることを―する書類」「身分―」「印鑑―」 
\\	数学および論理学で、真であると認められているいくつかの命題(公理)から、ある命題が正しいことを論理的に導くこと。 論証。 
\\	訴訟法上、当事者が事実の存否について、裁判官に確信を抱かせること。 または、これに基づき裁判官が確信を得た状態。 →疎明(そめい) [類語]
\\	立証・実証・例証・論証・検証・挙証・証言・証(あかし)・裏付け・裏書き(―する)証(しよう)する・裏付ける・明かす・証拠立てる	▲品質証明書を添付してください。 ▲彼は事態が正しいものだと証明しようとした。
\\	女王	じょおう	
\\	女性の王。 「ビクトリア―」 
\\	王の后(きさき)。 
\\	内親王の宣下のない皇族の女性。 
\\	皇族で、三世以下の嫡男系嫡出の女性。 旧皇室典範では五世以下の皇族の女性。 
\\	その分野で最も実力または人気のある女性。 「社交界の―」「銀幕の―」	▲女王の役は彼女に似合わない。 ▲女王はバッキンガム宮殿に住んでいる。
\\	職	しょく	
\\	担当する務め。 また、その地位。 職務。 「会長の―につく」「管理―」 
\\	生活を支えるための仕事。 職業。 「―を探す」「―を失う」 
\\	身についた技術。 技能。 「手に―をつける」→職として	▲彼は彼女をほめてその職に就かせた。 ▲彼は父親のコネのおかげで職を得た。
\\	職業	しょくぎょう	生計を維持するために、人が日常従事する仕事。 生業。 職。 「教師を―とする」「―につく」「家の―を継ぐ」「―に貴賤(きせん)なし」 [類語]職・仕事・稼業・生業(せいぎよう)・業(ぎよう)・なりわい・商売・渡世・家業・現職	▲すべての職業が女性に開かれるべきだ、というのは今や完全に許容されている考え。 ▲すべての職業は女性に門戸を開かねばならないと彼は主張した。
\\	食卓	しょくたく	食事用のテーブル。 食台。 「―につく」「―を囲む」	▲彼女は朝食のため食卓の用意をする。 ▲彼女は食卓から皿を片づけた。
\\	食品	しょくひん	人が食用にする品物の総称。 直接料理の材料としたり、そのまま食べたりすることができる食用の品。 飲食品。 「生鮮―」「―売り場」	▲この食品は密閉された容器に入れておけば一週間もつ。 ▲その食品の中からコレラ菌が検出された。
\\	植物	しょくぶつ	生物を大きく二大別した場合の、動物に対する一群。 木や草、藻類など。 一般に、一か所に固定して暮らし、細胞壁をもち、光合成を行って主に空気や水から養分をとって生きている生物。 種子植物・シダ植物・コケ植物・緑藻植物・紅藻植物などに分類される。 [類語]草木(そうもく)・くさき・本草(ほんぞう)・樹木(じゆもく)・緑(みどり)・プラント	▲植物の生長と生産性は、気温と湿度が作り出す入り組んだ関係に敏感に反応する。 ▲植物はみな水と光を必要とします。
\\	食物	しょくもつ	食べ物。 生物が食べてからだの栄養とするもの。	▲我々は水害の被害者に食物と衣類を支給した。 ▲家主に家賃を払えば、食物を買う金がなくなる。進退きわまったというところだ。
\\	食欲・食慾	しょくよく	何かを食べたいと思う欲望。 食い気。 「―がわく」「―をそそる」「―不振」	▲運動して食欲を増進させよう。 ▲運動不足で食欲が余りない。
\\	コード	コード	細い銅線を束にし、ゴム・糸・ビニールなどで覆って絶縁した電線。 屋内配線や電気器具の接続などに用いる。	
\\	といっても、Mac 
\\	のコードをバージョンアップした訳ではない。 ▲彼は機械にコードをつないだ。
\\	サイン	サイン	三角比・三角関数の一。 直角三角形で、一つの鋭角について、斜辺に対する対辺の比。 また、これを一般角に拡張して得られる関数。 記号
\\	正弦。 正弦関数。	▲ここにサインをお願いできますか。 ▲サインが消えるまでおタバコはご遠慮下さい。
\\	境	きょう	
\\	場所。 地域。 土地。 「無人の―」 
\\	心の状態。 境地。 「無我の―に入る」 
\\	環境。 境遇。 「誰しも―には転ぜらるる習いなり」 
\\	仏語。 五官および心の働きにより認識される対象。 六根の対象の、色・声・香・味・触・法の六境をいう。 境界(きようがい)。	▲我々の国はいくつかの国々と境を接している。 ▲彼は忘我の境をさまよっている。
\\	避ける・除ける	よける	[動カ下一][文]よ・く[カ下二] 
\\	触れたり出あったりしないようにわきに寄る。 また、身をかわしてさける。 「日なたを―・けて歩く」「車を―・ける」 
\\	前もって被害を防ぐ。 災いなどからのがれようとする。 「囲いをして風を―・ける」 
\\	別にしておく。 除外する。 「自分で食べる分は―・けておく」→避(さ)ける[用法]	▲そのスピードを出した車は道路に飛び出した子供を、間一髪で、避けることができた。 ▲彼はすぐにかんしゃくをおこすので皆が彼を避ける。
\\	誘う	いざなう	[動ワ五(ハ四)]《「いざ」は勧誘する意の感動詞。 「なう」は接尾語》さそう。 勧める。 「旅に―・う」「源氏物語の世界へ―・う」	▲彼女は一度もデートに誘われたことがない。 ▲太陽の日差しに誘われて人々が外出した。
\\	此れ等・是等	これら	[代]近称の指示代名詞。 
\\	「これ」の複数形。 ㋐話し手の側に属する事物についていう。 「今ここに―の問題がある」「―は私の収集品です」 ㋑この人たち。 「―には劣りなる白銀(しろがね)の箔を」 
\\	このあたり。 このへん。 「山ならねども―にも」	▲これらの茶碗はみんな壊れている。 ▲これらの地域の人々は年々飢えてきている。
\\	じっと	じっと	[副]スル 
\\	動かないで、そのままの状態を保つさま。 「家で―している」 
\\	視線や心などを集中して、よく見たり考えたりするさま。 つくづく。 「相手の顔を―見つめる」「―考え込む」 
\\	心を抑えてがまんするさま。 「非難に―耐える」 
\\	力をこめて、押したり引いたり踏んだりするさま。 ぎゅっと。 「陳平が高祖の足を―踏んだぞ」	▲彼女は赤ん坊をおこさないようにじっと座っていた。 ▲彼女は彼をじっと見つめた。
\\	こんな	こんな	[形動] 
\\	話し手、または、そのそばにいる人が当面している事態や、現に置かれている状況がこのようであるさま。 このような。 「世の中に―ひどい事があっていいのか」「―に親切にしてもらったのは初めてだ」「―仕事はもうごめんだ」 
\\	話し手のそばにある、または、手に持っている物のようすがこのようであるさま。 このような。 「―形の服がほしい」「―におもしろい本は読んだことがない」 ◆連体形に「こんな」「こんなな」の二形がある。 連体形として一般には「こんな」の形が用いられるが、接続助詞「ので」「のに」などに続くときは「こんなな」の形が用いられる。 「交通事情がこんななのに、よく来られたものだ」	▲こんなことをするとは彼は馬鹿にちがいない。 ▲こんなことを今までにお聞きになったことがありますか。
\\	あんな	あんな	[形動] 
\\	話し手も聞き手もともに知っている人や事物の状態があのようであるさま。 あれほど。 あれくらい。 「―にひどい被害とは思わなかった」「彼はなぜいつも―なのだろう」 
\\	話し手にも聞き手にも見えている人や事物のようす・状態があのようであるさま。 「あそこに立っている、―人が僕の好みだ」 ◆連体形に「あんな」「あんなな」の二形がある。 連体形として一般には「あんな」の形が用いられるが、接続助詞「ので」「のに」などに続くときは「あんなな」の形が用いられる。 「事態があんななので、どうすることもできない」	▲あんな男を信頼したのが私の間違いだった。 ▲あんな男を相手にしては駄目だ。全然信用できない奴なんだ。
\\	食料	しょくりょう	
\\	食用にする物。 食べ物。 
\\	食事の費用。 また、生活費。 「宿銭―の借り越しこそあれ」	▲食料雑貨店の主人は自分の誠実さをお客に何とか説得した。 ▲赤十字は被災者に食料と医療を分配した。
\\	食糧	しょくりょう	食用とする物。 食物。 糧食。 特に、米・麦などの主食物をさす。 「三日分の―」「―援助」	▲私たちは大量の食糧を輸入している。 ▲食糧が尽きた。
\\	書斎	しょさい	個人の家で、読書や書き物をするための部屋。 書室。	▲彼の書斎には何百冊という本が有る。 ▲彼の書斎には少なくとも1000冊の本がある。
\\	女子	じょし	
\\	おんなのこ。 むすめ。 ↔男子。 
\\	女性。 おんな。 「―学生」↔男子。	▲私たちの学校は男子より女子の方が多い。 ▲私たちのクラスには男子25名、女子20名いる。
\\	助手	じょしゅ	
\\	仕事の手助けをする人。 
\\	大学で、教授・助教授の職務を助ける職。 また、その人。	▲助手を求めています。なるべくならば経験のある人を望む。 ▲彼の現在の助手は野上さんです。
\\	徐徐	じょじょ	(多く「―に」の形で)ゆるやかに進むさま。 少しずつ変化するさま。 ゆっくり。 だんだん。 「―に歩を進める」「―に水位があがる」	▲単語は徐々に増えますので、こまめにチェックしてください。 ▲彼の言ったことの真意が徐々にわたしにわかり始めた。
\\	署名	しょめい	[名]スル本人が自分の名を書類などに書くこと。 また、その書かれたもの。 「契約書に―する」	▲私は、無理にその用紙に署名させられた。 ▲私は書類に署名した。
\\	書物	しょもつ	本。 書籍。	▲彼は書物や音楽にはほとんど興味を示さなかった。 ▲書物の選択に際して、過去の偉大な作家は最も注意されるべきだ。
\\	女優	じょゆう	女性の俳優。 女役者。 ↔男優。	▲その女優は銀行家と婚約したといった。 ▲その女優は雑誌を名誉き損で訴えた。
\\	処理	しょり	[名]スル物事を取りさばいて始末をつけること。 「事務を手早く―する」「事後―」「熱―」→処分(しよぶん)[用法]	▲近い将来、ゴミ処理費用が容積基準で有料化される可能性が高い。 ▲苦情は出来るだけ迅速に処理されるよう取り計らいなさい。
\\	書類	しょるい	文書・書き付けなどの総称。 特に、事務や記録などに関する書き付け。 「重要―」	▲こんなに書類があったらどうやって仕事ができるんだい。 ▲そのスパイは書類を燃やした。
\\	知らせ・報せ	しらせ	
\\	知らせること。 また、その内容。 通知。 「合格の―を待つ」「悪い―が届く」 
\\	何か事が起こるような兆し。 前兆。 「不吉な―」「虫の―」 
\\	歌舞伎で、幕開きや舞台転換などのとき、その合図に打つ拍子木。	▲知らせを聞いて彼は顔を曇らせた。 ▲知らせを聞いて興奮した。
\\	尻・臀・後	しり	㊀[名] 
\\	人や動物の胴体の後部で、肛門の付近の肉づきの豊かなところ。 けつ。 おいど。 臀部(でんぶ)。 
\\	動く人や物の後ろ。 あと。 後方。 「行進の―について歩く」 
\\	物事の一番あと。 終わりの部分。 しまい。 最後。 「ことばの―」「―から二番目の成績」 
\\	物の、最も後ろの部分。 最後部。 「―の切れたわらぞうり」 
\\	本・末のある長い物の、末の部分。 末端。 「縄の―を持つ」 
\\	容器の外側の底の部分。 また、果物の底部。 「鍋の―」 
\\	着物の裾。 
\\	物事や行為の結果。 結果として生じた事態。 また、事件の余波。 とばっちり。 「その責任を持ち込んで来る―はなかった」 ◆他の語の下に付いて複合語をつくるときは一般に「じり」となる。 ㊁〔接尾〕助数詞。 矢羽に用いる鳥の羽を数えるのに用いる。 尾羽を用いるところからの語。 大鷲(おおわし)は一四枚、小鷲は一二枚、鷹(たか)は一〇枚で一尻という。 「鷲の羽百―、よき馬三疋」 [下接語](じり)脂尻・糸尻・押っ立て尻・織り尻・仮名尻・川尻・為替尻・勘定尻・木尻・口尻・鞍(くら)尻・交換尻・湖尻・言葉尻・賽(さい)尻・財布尻・鞘(さや)尻・地(じ)尻・瀬尻・台尻・檀(だん)尻・帳尻・月尻・出尻・どん尻・長尻・鍋(なべ)尻・沼尻・半尻・貿易尻・幕尻・眉(まゆ)尻・目尻・矢尻	▲頭隠して尻隠さず。 ▲大きな集団の尻につくより頭になれ。
\\	印・標・証	しるし	
\\	他と紛れないための心覚えや、他人に合図するために、形や色などで表したもの。 目じるし。 「非常口の―」「持ち物に―をつける」 
\\	抽象的なものを表すための具体的な形。 ㋐ある概念を象徴するもの。 「平和の―の鳩」「純潔の―の白い衣装」 ㋑(証)ある事実を証明するもの。 証拠になるもの。 「見学した―にスタンプを押す」 ㋒(証)気持ちを形に表したもの。 「感謝の―に記念品を贈る」「お近付きの―におひとついかがですか」 
\\	所属・家柄などを表すもの。 記章・旗・紋所など。 「会員の―」 
\\	(「璽」とも書く) ㋐官印。 印綬。 押し手。 「未だ―及び公財を動かさしめず」 ㋑三種の神器の一である、八尺瓊勾玉(やさかにのまがたま)。 神璽(しんじ)。 「今天皇のみ―を上(たてまつ)るべし」 
\\	結納(ゆいのう)。 「―を厚く調へて送り納(い)れ、良き日をとりて婚儀(ことぶき)を催しけり」 [下接語](じるし)合い印・家印・糸印・馬印・笠標(かさじるし)・風標(かざじるし)・木印・袖標(そでじるし)・爪(つま)印・苗標(なえじるし)・荷印・墓標(はかじるし)・旗印・船(ふな)印・星印・無印・目印・矢印・槍(やり)印	▲先生は彼の名前に欠席の印をつけた。 ▲これは私たちから感謝の印のプレゼントです。
\\	城	しろ	
\\	敵襲を防ぐための軍事施設。 古代には朝鮮・蝦夷(えぞ)対策のために築かれ、中世には自然の要害を利用した山城が発達したが、このころのものは堀・土塁・柵(さく)などを巡らした簡単な施設であった。 戦国時代以降、政治・経済の中心地として平野に臨む小高い丘や平地に築かれて城下町が形成され、施設も天守を中心とした堅固なものとなった。 き。 じょう。 「―を明け渡す」 
\\	他人の入って来られない自分だけの領域。 「自分の―に閉じこもる」	▲城の跡は今は公園になっている。 ▲私は今古い城にいます。
\\	進学	しんがく	[名]スル 
\\	上級の学校に進むこと。 「大学に―する」《季 春》 
\\	学問の道に進み励むこと。	▲せいぜい40%の高校生しか大学に進学しない。 
\\	もの生徒が大学に進学する。
\\	神経	しんけい	《(オランダ)
\\	の訳語で、杉田玄白ほか訳「解体新書」に現れる語。 神気の経脈の意》 
\\	からだの機能を統率し、刺激を伝える組織。 ふつう神経細胞と、神経繊維である軸索および樹状突起からなるニューロン、あるいは、それらの集合である神経組織のこと。 
\\	物事に感じ、それに反応する心の働き。 また、特に過敏な心の働き。 感受性。 「―が細い」「―が高ぶる」「―をすり減らす」「無―」 [類語]
\\	中枢神経・末梢(まつしよう)神経/
\\	気(き)・心・精神・感受性・感性・感覚・センシビリティー	▲いつも神経が高ぶっています。 ▲サクラの話し方は私の神経に障る。
\\	真剣	しんけん	㊀[名]本物の刀剣。 木刀や竹刀(しない)に対していう。 ㊁[形動][文][ナリ]まじめに物事に対するさま。 本気で物事に取り組むさま。 「将来を―に考える」「―なまなざし」 [派生]しんけんさ[名]しんけんみ[名]	▲試験に受かるように真剣に勉強した。 ▲私達は真剣に話し合った。
\\	信仰	しんこう	[名]スル《古くは「しんごう」》 
\\	神仏などを信じてあがめること。 また、ある宗教を信じて、その教えを自分のよりどころとすること。 「―が厚い」「守護神として―する」 
\\	特定の対象を絶対のものと信じて疑わないこと。 「古典的理論への―」「ブランド―」 [類語]
\\	信心・敬神・崇拝・尊信・渇仰(かつごう)・帰依(きえ)・信教・入信	▲牧師は彼女のキリスト信仰の告白を聞いた。 ▲母はキリスト教を信仰している。
\\	信号	しんごう	[名]スル 
\\	色・音・光・形・電波など、言語に代わる一定の符号を使って、隔たった二地点間で意思を伝達すること。 また、それに用いる符号。 サイン。 「―を送る」「危険―」「わたり鳥へ―してるんです」 
\\	道路・鉄道線路などで進行の可否を知らせる機械。 信号機。 シグナル。 「―無視」「赤―」	▲信号は赤に変わった。 ▲信号は全部赤だった。
\\	人工	じんこう	自然の事物や現象に人間が手を加えること。 また、人間の手で自然と同じようなものを作り出したり、自然と同じような現象を起こさせたりすること。 「―の湖」「―着色」↔天然。	▲柔らかいウールの方が粗いウールより高価で、そのどちらともナイロン製の人工繊維より上等である。 ▲人工地球衛星の打ち上げは、普通宇宙探求のために行われるものと見なされている。
\\	深刻	しんこく	[名・形動] 
\\	事態が容易ならないところまできていること。 また、そのさま。 「住宅問題が―になる」 
\\	容易ならない事態と受けとめて、深く思いわずらうこと。 また、そのさま。 「―に考え込む」「―な表情」 
\\	考え・表現などが深いところにまで達していて重々しいこと。 また、そのさま。 「悲壮な熱情と―な思想とは」 
\\	無慈悲で厳しいこと。 むごいこと。 また、そのさま。 過酷。 「是程―な復讎(かたき)を取られる程」 [派生]しんこくさ[名]	▲ボブ・ジョンソンはアフリカの現状の深刻さを人々に気付かせようとした。 ▲ロビンソンさんの病気は深刻なものだが、彼は上機嫌だ。
\\	診察	しんさつ	[名]スル病気の有無や病状などを判断するために、医師が患者のからだを調べたり質問したりすること。 「患者を―する」「―室」	▲私はジョンを説得して医者の診察を受けさせた。 ▲救急診察を受けるにはどこへ行けばいいですか。
\\	人種	じんしゅ	
\\	人類を骨格・皮膚・毛髪などの形質的特徴によって分けた区分。 一般的には皮膚の色により、コーカソイド(白色人種)・モンゴロイド(黄色人種)・ニグロイド(黒色人種)に大別するが、この三大別に入らない集団も多い。 
\\	人をその社会的地位・生活習慣・職業や気質などによって分類していう言い方。 「仕事を生きがいとする―」	▲彼はいろいろな人種の人と接触している。 
\\	の低さの原因を人種に求めるという議論を論駁するどころか、リンのデータはそれを事実上補強することになっている。
\\	信じる	しんじる	[動ザ上一]「しん(信)ずる」(サ変)の上一段化。 「無罪を―・じる」	▲彼は正直は最良の策を信じなかった。 ▲彼は神の存在を信じない。
\\	人生・人世	じんせい	
\\	人がこの世で生きていくこと。 また、その生活。 「第二の―を送る」「―を左右する出来事」「―経験」 
\\	人の、この世に生きている間。 人の一生。 生涯。 「芸術は長く―は短い」 [類語]
\\	生(せい)・生活・日常・現世・生き方/
\\	生涯・一生・一代・終生・畢生(ひつせい)・ライフ	▲背が高くないということは、人生ではひどい欠点にはならない。 ▲波乱万丈の人生か。テレビでみるのはいいけど、わが身に置きかえれば結構きついね。
\\	親戚	しんせき	血縁や婚姻によって結びつきのある人。 親類。 →親類[用法]	▲クニ子は長井さんと親戚です。 ▲「あ、は、はい・・・ごめん、玲姉」「コラ。幾ら親戚とはいえ、私は先輩医師よ?院内ではちゃんとケジメをつけなさい」。
\\	新鮮	しんせん	[名・形動] 
\\	魚・肉・野菜などが、新しくて生き生きとしていること。 また、そのさま。 「―なくだもの」 
\\	汚れがなく、澄みきっていること。 また、そのさま。 「山の―な空気を吸う」 
\\	物事に今までにない新しさが感じられるさま。 「―な感覚の絵」 [派生]しんせんさ[名]しんせんみ[名]	▲彼女は新鮮な野菜を買いに市場へ行った。 ▲彼女は新鮮な空気を一息吸いに表へ出た。
\\	心臓	しんぞう	
\\	血液循環の原動力となる器官。 収縮と拡張を交互に繰り返し、静脈から戻ってくる血液を動脈に押し出し、全身に送るポンプの働きをする。 ヒトでは握りこぶし大で、胸腔内の横隔膜のすぐ上、やや左側にあり、三層の膜に包まれ、内腔は隔壁・弁膜によって左右の心房・心室の四部分に分かれる。 
\\	物事の中心部のたとえ。 「都市の―部」 
\\	《「心臓が強い」から》厚かましいこと、ずうずうしいこと、押しが強いことなどの意の、俗な言い方。 「あの人に借金を頼むなんて、君も―だね」	▲医師は患者の心臓の鼓動と血圧をモニターで監視した。 ▲運動すると心臓の鼓動が速くなる。
\\	慎重	しんちょう	[名・形動]注意深くて、軽々しく行動しないこと。 また、そのさま。 「―を期する」「万事に―な人」「―に検討を重ねる」↔軽率。 [派生]しんちょうさ[名]	▲この計画を君は慎重に調べなければならない。 ▲この事態は慎重な取り扱いを要する。
\\	身長	しんちょう	からだの高さ。 背丈。 身の丈。 「―が伸びる」「―がある」	▲身長順に男の子を並べる。 ▲若者は非常に身長の伸びを見せた。
\\	審判	しんぱん	[名]スル《「しんばん」とも》 
\\	物事の是非・適否・優劣などを判定すること。 「国民の―を受ける」 
\\	ある事件を審理し、その正否の判断・裁決をすること。 ㋐訴訟における審理と裁判。 ㋑家庭裁判所が家事事件または少年事件について行う手続き。 ㋒行政機関が前審として事件を審理・裁定すること。 公正取引委員会の審判手続きや海難審判・特許審判など。 
\\	運動競技などで、技の優劣、反則の有無、勝敗などを判定すること。 また、その役。 
\\	キリスト教で、神がこの世を裁くこと。 「最後の―」	▲勇敢に難局に立ち向かい、その結果は神の審判にまつほかは、包囲から逃れる方法はない。 ▲魔王は言った。「俺が負けるわっきゃねーべ。審判は皆地獄にいるのだ」。
\\	人物	じんぶつ	
\\	ひと。 人間。 「偉大な―」「登場―」 
\\	人柄。 ひととなり。 「面接試験では主として―を見る」「―は確かだ」 
\\	人格・才能などのすぐれた人。 人材。 「肚のすわった、なかなかの―だ」 
\\	描画の対象である人間の姿・形。 「―のデッサン」	▲その慈善団体には、およそ20億円の寄付をした人物の名前が付けられている。 ▲なるほど彼はまだ若いが、実に信頼できる人物だ。
\\	進歩	しんぽ	[名]スル 
\\	物事がしだいによりよいほうや望ましいほうへ進んでいくこと。 「―が早い」「長足の―を遂げる」「技術が―する」↔退歩。 
\\	歩を進めること。 前進。 「いっそ橋を越えて…、向う両国へ―」 [類語]
\\	発達・発展・向上・上達・進化・進展・飛躍・進境・日進月歩	▲彼は英語がほとんど進歩しなかった。 ▲彼は懸命に勉強したが。たいして進歩しなかった。
\\	親友	しんゆう	互いに心を許し合っている友。 特に親しい友。 「無二の―」	▲私は彼を親友と見なしている。 ▲私は彼を親友の一人と考えています。
\\	信用	しんよう	[名]スル 
\\	確かなものと信じて受け入れること。 「相手の言葉を―する」 
\\	それまでの行為・業績などから、信頼できると判断すること。 また、世間が与える、そのような評価。 「―を得る」「―を失う」「―の置けない人物」「店の―に傷がつく」 
\\	現在の給付に対して、後日にその反対給付を行うことを認めること。 当事者間に設定される債権・債務の関係。 「―貸付」 [類語]
\\	信憑(しんぴよう)・信認(―する)信ずる・真(ま)に受ける/
\\	信(しん)・信頼・信任・信望・人望・名(な)・定評・評判・暖簾(のれん)	▲彼女はある程度信用できる。 ▲彼女の言うことは信用できない。
\\	信頼	しんらい	[名]スル信じて頼りにすること。 頼りになると信じること。 また、その気持ち。 「―できる人物」「両親の―にこたえる」「医学を―する」	▲彼の言うことを信頼できないのは、風を頼りにできないのと同じだ。 ▲彼の性格についてもっとよく知っていたなら、彼を信頼しなかっただろう。
\\	心理	しんり	
\\	心の働きやありさま。 精神の状態。 「複雑な―」「深層―」 
\\	「心理学」の略。生物体の意識の内面的な動きの過程や、経験的具体的な事実としての意識と行動とを研究する学問。古くは形而上学の内に含まれたが、一九世紀以後実験科学として考えられている。領域は発達心理、個人心理、集団心理、応用心理など多岐にわたっている。	▲私は彼の心理が分からない。 ▲彼女は夫の心理を心得ている。
\\	人類	じんるい	人間。 ひと。 動物学上は、脊椎動物門哺乳綱霊長目ヒト科に分類される。 →人(ひと)	▲初期の人類は世界のあらゆる所に移住した。 ▲宗教は人類のアヘンである。
\\	巣・栖	す	
\\	鳥・獣・虫などのすむ所。 「ネズミの―」「小鳥の―」 
\\	人の住む所。 すみか。 「愛の―」 
\\	よくない仲間が寄り集まる場所。 「悪党の―」 
\\	クモが獲物を捕まえるために張る網。	▲あの鳥たちは夏に巣を作り、冬に南へ渡る。 ▲次には、これらのやの上に絹の糸をさらに張り、巣の中央に滑らかで、粘りのない部分を残す。
\\	図	ず	
\\	物の形や状態を描いたもの。 絵図・地図・図面など。 「掛け―」「見取り―」 
\\	点・線・面が集まって一つの形を構成しているもの。 図形。 
\\	物事のようす。 状態。 「見られた―ではない」 
\\	考えどおり。 思うつぼ。 「謀(はかりごと)の―を外させ」 
\\	くふう。 計画。 「何にてもあたらしい思ひつき、今迄ない―を案ずるに」 
\\	十二律の各音階の正しい調子を書き表したもの。 「当寺の楽は、よく―を調べ合はせて」	▲各種の性格類型が概略図の形で図1に示されている。 ▲出生数1000に対する死亡率が図13
\\	1に示されている。
\\	水準	すいじゅん	
\\	事物の一定の標準。 また、価値・能力などを定めるときの標準となる程度。 レベル。 「技術が―に達する」「―を上回る成績」「生活―が高い」 
\\	土地・建物などの高低・水平の度合いを測ること。 また、その道具。 水盛(みずもり)。 
\\	線路の曲線部における、左右のレールの高低差。 →カント	▲彼は教育の水準の低下についてくどくどしゃべり続ける。 ▲彼は水準を満たしていなかった。
\\	推薦	すいせん	[名]スル 
\\	人をその地位・名誉に適している者として他人にすすめること。 推挙。 「委員長に―する」「―状」 
\\	よいものとして人にすすめること。 「―図書」	▲ゲイのクラブを推薦してください。 ▲このレストランは推薦できます。
\\	スイッチ	スイッチ	[名]スル 
\\	電気回路の開閉を行う装置。 開閉器。 点滅器。 「―を入れる」「―を切る」 
\\	鉄道の転轍(てんてつ)器。 ポイント。 
\\	考え・方法などを切り替えること。 「左のピンチヒッターに―する」	▲炊飯器のスイッチを入れてね。 ▲絶対にそのスイッチに触ってはいけない。
\\	睡眠	すいみん	
\\	ねむること。 ねむり。 周期的に繰り返す、意識を喪失する生理的な状態。 「―をとる」「―が足りる」「―不足」 
\\	活動を休止していること。 「―状態」 [類語]
\\	眠り・寝(ね)・就眠・睡臥(すいが)・安眠・熟睡・熟眠・昏睡(こんすい)・居眠り/
\\	休眠・冬眠	▲十分に睡眠を取りなさいといわれました。 ▲人は十分な睡眠が必要だ。
\\	数	すう	
\\	もののかず。 ものの多少を表す概念。 「一定の―に満たない」 
\\	数をかぞえること。 計数。 「―に明るい」 
\\	物事の成り行き。 情勢。 また、めぐりあわせ。 運命。 「美術の次第に衰うるは天の―なり」 
\\	自然数およびこれを順次拡張した、整数・有理数・実数・複素数などの総称。 
\\	インド‐ヨーロッパ語で、名詞・代名詞・形容詞・冠詞・動詞の語形によって表される文法範疇(はんちゆう)。 一つのものには単数、二つ以上のものには複数を区別する。 その他、言語によっては双数・三数・四数もある。 日本語には、文法範疇としては存在しない。 
\\	数をかぞえる語の上に付いて、二、三か五、六ぐらいの数量を漠然と表す。 「―組」「―ページ」「―メートル」→数名 [類語]
\\	数(かず)・数値・数量・量・分量・数字	▲彼女は、数ペニーを節約するために半時間を浪費するなと、彼を説得できなかった。 ▲お庭の方にも縁台を並べれば、かなり座席数を確保できそうだなーと思いました。
\\	数字	すうじ	
\\	数を表すのに用いる記号や文字。 アラビア数字(「1、2」など)・ローマ数字(「Ⅰ、Ⅱ」など)・漢数字(「一、二」など)の類。 
\\	統計・成績・計算など、数字によって表される事柄。 「―に強い」「―がものを言う」 
\\	数個の文字。 [アクセント] 
\\	はスージ、 
\\	はスー	▲メーターの数字を読んでください。 ▲それらの数字は、合計があわない。
\\	スープ	スープ	西洋料理の汁物。 スープストックを塩・香辛料などで味つけしたり、ルーで濃度をつけたりしたもの。 コンソメとポタージュとに大別される。 ソップ。	▲ガブリエルは熱いスープとシェリー酒を少し飲んだだけだった。 ▲このインスタントスープは1つ1つ包みの中に入っている。
\\	末	すえ	
\\	(本(もと)に対して)続いているものの先端の方。 末端。 「毛の―」 
\\	川下(かわしも)。 下流。 「山中の渓流の―である河は」 
\\	中央から離れた端の所。 場末・野ずえ・末席など。 「―の座」 
\\	本筋から隔たった物事。 つまらないこと。 「そんな細かいことは―の―だ」 
\\	物事の行われたのち。 あげく。 「ごたごたの―落ち着く」「苦心の―完成した」 
\\	ある期間の終わりのほう。 「今月の―」 
\\	一生の最後の時期。 晩年。 「人一代の―」 
\\	今からのち。 行く末。 将来。 「―が思いやられる」 
\\	子孫。 「源氏の―」 
\\	一番あとに生まれた子。 末っ子。 「―は女です」 
\\	仏教がおとろえ人心がすさみ、道徳も秩序も乱れ衰えた時代。 末世(まつせ)。 「世も―となる」 
\\	短歌の下(しも)の句。 
\\	(本(もと)に対して)後編。 
\\	神楽歌を奏するのに、神座に向かって右方の座席。 また、そこにすわる奏者。 
\\	草木の伸びている先。 こずえ、枝先など。 「うぐひすの…紅梅の―にうち鳴きたるを」 
\\	山頂。 山のいただき。 「高山、短山(ひきやま)の―より」 
\\	江戸時代、将軍・大名などに仕えた女中。 おすえ。 
\\	身分の低いもの。 下等。 下級。 「―の傾城四人まゐりて」 [下接語]来(こ)し方行く末・末の末・場末・本(もと)末・行く末(ずえ)末末・月末・野末・葉末・穂末	▲よく考えた末、私は家にいることに決めた。 ▲夏の末と秋には紅葉が見られる。
\\	姿	すがた	
\\	人のからだの格好。 衣服をつけた外見のようす。 「顔もいいし―もいい」「鏡に―を映す」「後ろ―」 
\\	身なり。 風采。 「―をやつす」「うらぶれた―」 
\\	目に見える、人の形。 存在するものとしての人。 「―をくらます」「あれっきり彼は―を見せない」 
\\	物の、それ自体の形。 「山が雲間から―を現す」 
\\	物事のありさまや状態。 事の内容を示す様相。 「移り行く世の―」「主人公の成長する―を描く」 
\\	和歌や俳句の、一首・一句に表れる趣や格調などの全体的な風体。 特に、歌体。 
\\	美しい顔形の人。 美人。 「―の関守、京の四条は生きた花見あり」 [下接語]艶(あで)姿・後ろ姿・絵姿・幼姿・男姿・帯解き姿・女姿・死に姿・立ち姿・伊達(だて)姿・旅姿・道中姿・夏姿・似姿・寝姿・初姿・晴れ姿・遍路姿・優(やさ)姿・窶(やつ)れ姿・童(わらわ)姿	▲船の姿は水平線の向こうへと見えなくなった。 ▲船は港を出港して、二度と姿を見られることはなかった。
\\	スキー	スキー	
\\	雪上を滑走したり移動したりするために、両足につける細長い板状の用具。 
\\	雪上で 
\\	を用いて行うスポーツや競技。 また、それをはいて雪上を滑ること。 《季 冬》	▲うちは家族で毎年冬スキーに行きます。 ▲当地では誰でもスキーをします。
\\	救う・済う	すくう	[動ワ五(ハ四)]《「掬う」と同語源》 
\\	危機的な状況や苦しい境遇、悪い環境などにある人に力を貸したり金品を与えたりして、そこから抜け出させる。 助ける。 救助する。 救済する。 「人命を―・う」「地球環境を―・う」 
\\	神・仏などの力によって平安な心的状態に導いたり、迷いや悩みを取り除いたりする。 「信仰に―・われる」 
\\	好ましくない状態からのがれ出させ、良いほうに進むように導く。 「堕落の道から生徒を―・う」 
\\	(多く「救われる」の形で)悪い条件を相殺する。 不安・不満などが一応解消する。 「重病だが本人が明るいので―・われる」「仕事がきつい上に給料が安いのでは―・われない」 [可能]すくえる[ア下一]	▲その医師は内科医として、瀕死の子供の生命を救うために、できることはなんでもしたばかりか、危機を切り抜けるために自分の血を提供までした。 ▲その医者は事故で負傷した4人を救った。
\\	優れる・勝れる	すぐれる	[動ラ下一][文]すぐ・る[ラ下二] 
\\	能力・容姿・価値などが他よりまさる。 他よりぬきんでる。 「語学に―・れる」「人並み―・れて足が速い」「―・れた作品」 
\\	(多く打消しを伴って用いる)よい状態である。 「健康が―・れない」「天候が―・れない」	▲彼は優れた記憶力を発揮してわたしの質問に答えた。 ▲彼は優れた記憶力の持ち主だ。
\\	スケート	スケート	
\\	氷上を滑走するための用具。 靴の底に金属製のブレード(板)を取り付けたもので、スピードスケート用・フィギュアスケート用・アイスホッケー用の三種がある。 アイススケート。 
\\	氷上で 
\\	を用いて行うスポーツ。 《季 冬》 
\\	「ローラースケート」に同じ。底に四個の小輪の付いた靴で、床板の上を滑るスポーツ。また、その靴。	▲いっしょにスケートをしたのが、まるで昨日のことみたいです。 ▲この川でスケートをした事がありますか。
\\	少しも	すこしも	[副] 
\\	(下に打消しの語を伴って)全然。 まったく。 ちっとも。 「―寒くない」 
\\	少しでも。 わずかながら。 「―益の増さらんことを営みて」	▲私は少しもお金を持っていません。 ▲私は君のお情けなど少しもいらない。
\\	過ごす	すごす	[動サ五(四)]《「すぐす」の音変化》 
\\	何かをして時間を費やす。 「休日は子供と遊んで―・した」 
\\	月日を送る。 暮らす。 「その後は、いかがお―・しですか」 
\\	そのままの状態にしておく。 「お前はどうかして一寸それをぼかして―・そうと云うのだ」 
\\	物事の程度を越す。 特に、酒を飲み過ぎる。 または、酒を飲む。 「冗談も度を―・すと不愉快だ」「友達にすすめられて、ついつい―・してしまった」「どうぞ一つ、お―・しください」 
\\	生活の面倒を見る。 養う。 「二十三にもなって親を―・す所か」 
\\	物事を終わらせる。 済ませる。 すぐす。 「えさらず思ふべき産屋の事もあるを、これ―・すべしと思ひて」 
\\	年をとる。 老いる。 すぐす。 「久我の少将通宣、いたく―・したる程にて」 
\\	(動詞の連用形に付いて) ㋐適当な程度を越して事をする。 「遊び―・す」「寝―・す」 ㋑そのままにしておく。 「見―・す」「やり―・す」 ◆上代には「すぐす」が普通で「すごす」はほとんど見られない。 平安時代には両者が併用され、中世末期ごろから「すごす」が普通になった。 [可能]すごせる [類語]
\\	送る・費やす・暮らす・明かし暮らす・明け暮れる・消光する	▲とても楽しい時を過ごした。 ▲とても楽しく過ごしていた、するとそのとき悲しい知らせが届いた。
\\	筋・条	すじ	㊀[名] 
\\	筋肉。 また、その繊維。 「肩の―が凝る」 
\\	筋肉を骨に付着させている組織。 腱(けん)。 「足の―を切る」 
\\	皮膚の表面に浮き上がってみえる血管。 「―の浮き出た手」「額に―を立てて怒る」 
\\	植物などの繊維。 「―のかたい野菜」 
\\	細長く、ひと続きになっているもの。 線。 「まっすぐに―を引く」 
\\	縞模様。 「赤い―のある布地」 
\\	家系。 家柄。 「貴族の―を引く」 
\\	学問や芸術の流儀。 流派。 「彼の絵は狩野派の―だ」 
\\	素質。 たち。 「芸の―がいい」 
\\	物事の道理。 すじみち。 「―の通った話」 
\\	小説や演劇などの、大体の内容。 梗概(こうがい)。 「芝居の―」 
\\	そのことに関係のある方面。 「確かな―からの情報」「消息―」 
\\	依頼したい事柄。 おもむき。 「お願いの―があって参上いたしました」 
\\	道路や川に沿った所。 「街道―」「利根川―」 
\\	囲碁・将棋で、本筋とされている打ち方・指し方。 
\\	将棋で、盤面の縦九列のそれぞれをいう。 
\\	身分。 地位。 「めでたきにても、ただ人の―は、何の珍しさにか思ひ給へかけむ」 
\\	「すじかまぼこ」の略。 ㊁〔接尾〕助数詞。 
\\	細長いものを数えるのに用いる。 「帯をひと―」「ふた―の道」 
\\	江戸時代、銭(ぜに)一〇〇文を数えるのに用いる。 「銭さし一―」 [下接語]青筋・粗(あら)筋・家筋・粋(いき)筋・糸筋・大筋・大手筋・思惑筋・海道筋・街道筋・川筋・癇癪(かんしやく)筋・官辺筋・客筋・金筋・銀筋・首筋・玄人(くろうと)筋・毛筋・権威筋・主(しゆう)筋・消息筋・素人(しろうと)筋・親類筋・政府筋・背筋・疝気(せんき)筋・千(せん)筋・其(そ)の筋・太刀筋・球筋・千(ち)筋・血筋・強気筋・手筋・手の筋・鼻筋・腹筋・一(ひと)筋・二(ふた)筋・本筋・升掛け筋・町筋・万筋・澪(みお)筋・三(み)筋・水筋・道筋・目先筋・矢筋・横筋・弱気筋	▲物語の筋はある島を舞台に展開する。 ▲私はそのニュースを確かな筋から得た。
\\	勧める・奨める	すすめる	[動マ下一][文]すす・む[マ下二]《「進める」と同語源》 
\\	人がその事を行うように誘いかける。 勧誘する。 「辞任を―・める」「加入を―・める」 
\\	物を供して、飲食または使用してもらおうとする。 「茶菓を―・める」「風呂を―・める」 
\\	積極的に実行するようにたすけ励ます。 奨励する。 「資源の有効利用を―・める行政」	▲彼女はその客にブルーのネクタイを勧めた。 ▲彼女の先生は彼女にもっと小説を読むように勧めた。
\\	進める	すすめる	[動マ下一][文]すす・む[マ下二] 
\\	前の方へ動かして位置を移す。 前進させる。 「馬を―・める」「歩(ほ)を―・める」「ひざを―・める」 
\\	予定の手順に従って、物事を進行させる。 はかどらせる。 「工事を―・める」「話を―・める」 
\\	物事の内容・程度をさらに高める。 「一歩―・めて考える」「合理化を―・める」 
\\	上の地位・段階に移す。 位を高くする。 「一階級―・める」 
\\	(刺激して)盛んにする。 うながす。 「食欲を―・める酒」 
\\	時計の針を正しい時刻より先の時刻を示すようにする。 「念のため五分―・めておく」	▲時間をむだにしないで、この仕事を進めよう。 ▲人混みのために私は一歩も進めなかった。
\\	スター	スター	
\\	星。 星の形をしたもの。 また、星印。 
\\	人気のある芸能人や運動選手。 花形。 また、ある分野で際だった人気者。	▲彼女は人気スターとしては無器量で肉付きがよい。 ▲スターの野球選手はよくサインを求められる。
\\	スタイル	スタイル	
\\	からだつき。 姿。 格好。 「すらりとして―がいい」 
\\	服飾・頭髪などの型。 「最新流行の―」「ヘア―」 
\\	建築・美術・音楽などの様式。 型。 「前衛的―のビル」「演奏―」 
\\	文章や文学作品の表現形式。 特に、文体。 「独自の―をもつ作家」 
\\	個人や集団などに固有の、考え方や行動のしかた。 「ライフ―」	▲ランダーが当初からエメットのスタイルを変更する意図を持っていたかどうかは明らかでない。 ▲彼女はスタイルがいいから、何を着てもよく似合う。
\\	スタンド	スタンド	
\\	競技場・野球場などの階段式の観覧席。 「―を埋める観衆」「メーン―」 
\\	屋台式の売店。 「駅の―」 
\\	カウンターで飲食させる店。 また、カウンターに沿って並べた席。 「コーヒー―」 
\\	㋐物をのせたり、立てたりする台。 「エッグ―」 ㋑駐車中の自転車・オートバイが倒れないための支えとする部品。 
\\	電気スタンドのこと。 
\\	ガソリンスタンドのこと。	▲次のスタンドまでどのくらいありますか。 ▲私は大きな電気スタンドは嫌いだ。
\\	頭痛	ずつう	
\\	頭が痛むこと。 とうつう。 「―持ち」 
\\	気にかかること。 心労。 心配。 「―の種」	▲頭痛を軽くみてはいけません。 ▲頭痛はするし、せきでも苦しんでいます。
\\	素敵・素的	すてき	[形動][文][ナリ]《「すばらしい」の「す」に、接尾語「てき」の付いたものという。 「素敵」「素的」は当て字》 
\\	自分の気持ちに合っていて、心を引かれるさま。 非常にすぐれているさま。 「―な服装」 
\\	程度がはなはだしいさま。 「―に堅そうな首を、…濶(ひろ)い肩の上にしっかりすげ込んだようにして」	▲彼はハンサムですてきなのだけど上品とはいえないわ。 ▲彼は幸運にもすてきな女の子と結婚した。
\\	既に・已に	すでに	[副] 
\\	ある動作が過去に行われていたことを表す。 以前に。 前に。 「―述べた事柄」 
\\	その時点ではもうその状態になっていることを表す。 もはや。 とっくに。 「彼は―おとなだ」「手術はしたものの―手後れだった」 
\\	動かしがたい事実であることを表す。 どう見ても。 現に。 「この事が―権威の失遂を物語っている」 
\\	すっかり。 まったく。 「天の下―覆ひて降る雪の光を見れば貴くもあるか」 
\\	ある事態が近づいていることを表す。 もう少しで。 今にも。 「仏御前はすげなう言はれ奉って―出でんとしけるを」	▲ピーターが起きたときには、ジーンはすでに家を出ていた。 ▲メアリーがバス停に着いたときは最終バスはすでに出ていた。
\\	即ち・則ち・乃ち	すなわち	㊀[接] 
\\	前に述べた事を別の言葉で説明しなおすときに用いる。 言いかえれば。 つまり。 「日本の首都―東京」 
\\	前に述べた事と次に述べる事とが、まったく同じであることを表す。 とりもなおさず。 まさしく。 「生きることは―戦いである」 
\\	(「…ば」の形を受けて)前件の事実によって、後件の事実が自然に成り立つことを表す。 その時は。 そうすれば。 「戦えば―勝つ」「信じれば―救われる」 ㊁[名] 
\\	(連体修飾語に続けて)その時。 「綱絶ゆる―に八島の鼎(かなへ)の上に、のけざまに落ち給へり」 
\\	むかし。 あのころ。 当時。 「若宮は、―より寝殿に通る渡殿におはしまさせて」 ㊂[副] 
\\	すぐに。 たちまち。 「立て籠(こ)めたる所の戸、―ただ開きに開きぬ」 
\\	もう。 すでに。 「頗る出精せしが、今は、―亡(な)し」 ◆この語の語源は、いわゆる「時を表す名詞」の一種であり、平安時代以後、「即・則・乃・便」などの字の訓読から接続詞として用いられるようにもなったと考えられ、現在ではその用法に限られるといってよい。 [類語] 
\\	つまり・言い換えれば・要するに/
\\	取りも直さず・まさしく・ほかでもない・即(そく)	▲通貨、すなわち、当時だれもが用いていた言い方に従えば、通常の王国法貨を与える代わりに、雇い主は従業員に代用貨幣をあたえていました。そして、この代用貨幣は金属だったり、木だったり、厚紙だったりしました。 ▲調査担当者はどのルートが一番容易に、すなわち一番安上がりに作れるかも、鉄道の推進者にアドバイスしうるであろう。
\\	スピーチ	スピーチ	談話。 演説。 「テーブル―」	▲3人の生徒が短いスピーチを言い、自己紹介をしたり自分の国について話した。 ▲あなたのスピーチはあの場にふさわしいものだった。
\\	凡て・総て・全て	すべて	《動詞「す(統)ぶ」の連用形+接続助詞「て」から》 ㊀[名]ある物や、ある事の全部。 いっさい。 「―を知る」「見るもの―が珍しい」「金が―の世の中」 ㊁[副] 
\\	ことごとく。 残らず。 「財産を―投げ出す」「見たことを―話す」 
\\	おおよそ。 大体。 総じて。 「―、きむぢ、いとくちをし」 
\\	(打消しの語を伴って用いる)全然。 まるっきり。 「―音もせで五六日になりぬ」 [用法]すべて・全部・みな――「植木がすべて(全部・みな)枯れた」「株で失敗して財産をすべて(全部・みな)失った」「島民はすべて(全部・みな)避難した」など、相通じて用いられる。 
\\	「すべて」「全部」は、「在庫はすべて(全部)売り切れた」「会員すべて(全部)が反対だ」のように、物についても人についても使う。 
\\	「みな」は「みな、出かけようか」という代名詞としての用法があるように、特に人について多く使われる。 「みなで協力しよう」「みな帰ってしまった」などの文脈では「すべて」「全部」は不適当である。 
\\	「すべて」は文章語的で、「みな」「全部」は口語的である。 [類語] ❶一切(いつさい)・全部・全体・全般・万般・万端・万事/ 
\\	何もかも・ことごとく・なべて・皆(みな・みんな)・悉皆(しつかい)・残らず・余す所なく・漏れなく・逐一(ちくいち)・すっかり・そっくり・洗い浚(ざら)い・一から十まで(「すべての」の形で用いる場合)有る限りの・有りっ丈(たけ)の・あらゆる・有りとあらゆる	▲私は服はすべて注文で作らせる。 ▲私は本棚の本をすべて読んだわけではない。
\\	済ませる	すませる	[動サ下一]「済ます」に同じ。 「食事を急いで―・せる」 ①すむようにする。なしとげる。はたす。平家物語10「…と言ふ白拍子を、誠に面白くかぞへ―・したりければ」。槐記「拾芥抄を見られたるや、ことごとく―・されたるや」。「仕事を―・す」「外で食事を―・す」 ②借りを返す。返金する。返済する。日葡辞書「ヲイモノ(負物)ヲスマス」。浮世草子、御前義経記「―・さねば一分立たず」。「借金を―・す」 ③決着をつける。片付ける。「金銭で―・される事柄ではない」 ④それでよいとしておく。間に合わせる。梅暦「マア辛抱してこの場を―・して、あとで恨みをおはらしなんし」。「当分の間これで―・しておいて下さい」 すま・す【澄ます・清ます】 他五 ①洗いきよめる。洗いすすぐ。落窪物語1「まづ水とて、御足―・さす」。源氏物語若菜上「御髪―・し、ひきつくろひておはする」 ②濁りを除いて透き通った状態にする。また、曇りを取り去り冴さえた状態にする。源氏物語少女「泉の水遠く―・しやり水の音まさるべき岩を立て並べ」。源氏物語明石「今の世に聞えぬ筋引きつけて、手づかひいといたう唐めき、揺ゆのね深う―・したり」。「水を―・す」「刀をとぎ―・す」 ③心のけがれを清める。宇津保物語俊蔭「夏は清く涼しきかげに眺めて、花紅葉の下に心を―・しつつ」 ④心を落ちつかす。しずめる。源氏物語夕霧「今すこし思ひしづめ、心―・してこそともかうも」 ⑤鎮定する。治める。平家物語12「一天を鎮め、四海を―・す」 ⑥人影などがなくなる状態にする。今昔物語集28「しばしかくてあらむ。さて大路を―・して歩かちより行くべきなり」 ⑦理非を調べてはっきりさせる。また、道理をはっきりさせる。日葡辞書「クジ(公事)ヲスマス」「リヲスマス」 ⑧注意を集中する。平家物語1「諸人目を―・しけり」。日葡辞書「キキスマス」。「耳を―・す」 ⑨(他の動詞の下に付いて)その事を完全にしおおせる意を表す。平家物語9「暁立んとての夜、舎人に心をあはせて、さしも御秘蔵候いけずきを盗み―・いて上りさうはいかに」。「警官になり―・す」 ⑩(自動詞的に)何事もなかった顔付きをする。きどる。平気な様子をする。浮世草子、御前義経記「拙者は博奕嫌ひなれども何れもめさるに―・したらしくもならず」。「つんと―・している」「さんざ迷惑をかけておきながら―・している」	▲繁雑な手続きをすませ、船が定時に出港できるためには、特別の上にも特別なはからいが、どうしても必要である。 ▲買い物を済ませてしまったら電話します。
\\	刷る・摺る	する	[動ラ五(四)] 
\\	活版・版木などの面にインク・絵の具などをつけて、紙を当てて文字や絵を写し取る。 印刷する。 「紙幣を―・る」 
\\	布に木型を押し当てて、彩色したり、模様を染め出したりする。 「月草に衣は―・らむ朝露に濡れてののちは移ろひぬとも」 [可能]すれる	▲私達は招待状を50通刷りました。
\\	鋭い	するどい	[形][文]するど・し[ク]《形容動詞「するど」の形容詞化》 
\\	物の先が細くてとがっている。 また、刃物の切れ味がよい。 「―・い牙(きば)」「―・いナイフ」↔鈍い。 
\\	感覚が鋭敏である。 反応が速い。 また、判断力がすぐれている。 「嗅覚が―・い」「―・い洞察力」「勘が―・い」「目の付け方が―・い」↔鈍い。 
\\	㋐物に向かっていく勢いが激しくて強い。 「―・いパンチをかます」「―・い攻撃」 ㋑勢いが激しくて、人の心を突き刺すようである。 きびしい。 「語調が―・い」「―・い目付きだ」「―・い批判」 ㋒人の感覚を刺激する力が強い。 「―・い叫び声をあげる」「―・い痛みが走る」「―・い光が目を射る」↔鈍い。 [派生]するどさ[名] [類語]
\\	先鋭・鋭利・シャープ/
\\	聡(さと)い・鋭利・鋭敏・敏感・明敏・慧敏(けいびん)・犀利(さいり)/
\\	激しい・きびしい・きつい・強烈・痛烈・峻烈(しゆんれつ)	▲その大型機は鋭い音を立ててほとんど直角に降下した。 ▲以前私達がパンを切っていたナイフは鋭かった。
\\	性	せい	㊀[名] 
\\	人が本来そなえている性質。 うまれつき。 たち。 「人の―は善である」 
\\	同種の生物の、生殖に関して分化した特徴。 雄性と雌性。 雄(おす)と雌(めす)、男と女の区別。 また、その区別があることによって引き起こされる本能の働き。 セックス。 「―に目覚める」 
\\	インド‐ヨーロッパ語・セム語などにみられる、名詞・代名詞・形容詞・冠詞などの語形変化によって表される文法範疇(はんちゆう)の一。 男性・女性・中性などの区別がある。 日本語には、文法範疇としての性の区別はない。 英語でも代名詞にみられるだけで、それ以外の品詞では消滅している。 ㊁〔接尾〕名詞の下に付いて、物事の性質・傾向を表す。 「安全―」「アルカリ―」「向日―」「人間―」	▲下手に独自の方向性を出すと、攻撃されてしまうから、安全パイの報道しかしない。 ▲エコロジーのために堪え忍ぶのではなく、自然と調和した住環境の快適性が必要である。
\\	生	せい	㊀[名] 
\\	生きていること。 「―と死の分かれ目」↔死。 
\\	生命。 いのち。 「この世に―をうける」「―なきもの」 
\\	毎日の暮らし。 生活。 「充実した―を送る」 ㊁[代]一人称の人代名詞。 男性が自分をへりくだっていう語。 わたくし。 小生。 「妻より君へあてたる手紙、ふとしたることより―の目に触れ」 ㊂〔接尾〕人名に付いて、へりくだった意を添える。 手紙文などで、差し出し人の姓または姓名の下に付けて用いる。 「山田―」	▲生を感じる。 ▲死は生の正反対である。
\\	税	ぜい	国費・公費をまかなうため、国・地方公共団体が国民・地域住民・消費者などから強制的に徴収する金銭。 租税。 税金。	▲政府はワインに新しく税を課した。 ▲新しい税の導入は経済全体に影響を与えるものと見られる。
\\	性格	せいかく	
\\	行動のしかたに現れる、その人に固有の感情・意志の傾向。 「ほがらかな―」「夫婦の―が合わない」 
\\	特定の事物にきわだってみられる傾向。 「二つの問題は―が異なる」「趣味的―の濃い団体」→性質[用法] [類語]
\\	性質・性向・性情・気質・質(たち)・性(しよう)・性分(しようぶん)・気性(きしよう)・気立て・人柄・心柄(こころがら)・心根(こころね)・心性(しんせい)・品性・資性・資質・個性・人格・キャラクター・パーソナリティー/
\\	性質・特質・特性・特徴・特色・本質	▲問題となっているのは、彼の能力ではなく性格だ。 ▲夢は我々の性格の試金石である。
\\	正確	せいかく	[名・形動]正しく確かなこと。 事実と合っていて少しもまちがいのないこと。 また、そのさま。 「―を期する」「―な時刻を報ずる」「事実を―に記録する」 [派生]せいかくさ[名]	▲ボブ叔父さんが時計を直してくれて、今は時間が正確だ。 ▲もっと正確な時間をおしえてください。
\\	世紀	せいき	
\\	西暦で、一〇〇年を単位とする年代の数え方。 キリスト生誕の年を基点として数える。 二〇世紀は一九〇一年から二〇〇〇年まで。 
\\	ある、ひと続きの年月。 時代。 「科学の―」 
\\	(「世紀の」の形で)一世紀に一度しかないほどまれなこと。 「―の大事件」「―のロマンス」	▲この風習の起源は12世紀にさかのぼる。 ▲この油絵は17世紀のものだ。
\\	請求	せいきゅう	[名]スル 
\\	ある行為をするように相手方に求めること。 また特に、金銭の支払い、物品の受け渡しなどを求めること。 「情報開示の―」「代金を―する」 
\\	民事訴訟法上、原告が訴えによってその趣旨および事実関係の当否について裁判所の審理・判決を求めること。	▲彼が私の芝を刈り始めた時の金額のざっと4倍を請求してくるであろう。 ▲彼はそれに五ドルを請求した。
\\	税金	ぜいきん	国または地方公共団体に租税として納付する金銭。 「―がかかる」「―を納める」	▲浪費する人を守るために倹約する人に税金をかけるのは、賢明なことではないし、結局はためにならないのである。 ▲利益は税金抜きですか。
\\	清潔	せいけつ	[名・形動] 
\\	汚れがないこと。 衛生的であること。 また、そのさま。 「からだを―に保つ」「―な下着」↔不潔。 
\\	人柄や行いが清らかで、うそやごまかしなどがないこと。 また、そのさま。 「―な選挙」 [派生]せいけつさ[名]	▲スキャンダルは彼の清潔なイメージを酷く傷つけた。 ▲医者の器具は常に完全に清潔でなければならない。
\\	制限	せいげん	[名]スル物事にある限界を設けること。 また、その限界。 「入会資格に―を加える」「医者に飲酒を―される」 [用法]制限・制約――「制限(制約)を無視して勝手にふるまう」「年齢に制限(制約)がある」など、限度を設ける意では相通じて用いられる。 
\\	「制限」はある枠の中におさえる意。 「酒の量は一合に制限する」「速度制限」「制限時間」
\\	「制約」は条件をつけて行動をおさえる意。 「時間の制約があって、十分に話せなかった」「制約をはねのける」
\\	類似の語に「規制」がある。 「規制」は規則に従って、一定の限度におさえたり、禁止したりすること。 「輸入量を規制する」「夜間の飛行を規制する」	▲彼はスピード制限を無視して、とても速く走った。 ▲彼はこの二ヶ月間食事制限をしている。
\\	成功	せいこう	[名]スル 
\\	物事を目的どおりに成し遂げること。 「失敗は―の母」「新規事業が―をおさめる」「実験に―する」 
\\	物事をうまく成し遂げて、社会的地位や名声などを得ること。 「写真家として―する」 
\\	⇒じょうごう(成功)	▲もう少し努力をしていたら、彼は成功していただろう。 ▲もう少し努力すれば、あなたは成功するだろう。
\\	正式	せいしき	[名・形動]定められた正しい方式や、簡略化しない本来の形式に従っていること。 また、そのさま。 「―な(の)要請」「―な(の)名称」	▲彼は息子を正式の相続人と認めた。 ▲日本の国会は、第52第首相に橋本龍太郎氏を正式に選出した。
\\	精神	せいしん	
\\	人間のこころ。 また、その知的な働き。 「健全な―」 
\\	物質に対し、人間を含む生命一般の原理とみなされた霊魂。 たましい。 
\\	物事をなしとげようとする心の働き。 気力。 「―を鍛える」「―統一」 
\\	物事の基本的な意義・理念。 「憲法の―」 
\\	ある歴史的過程や共同体などを特徴づける意識形態。 「時代―」「民族―」 [類語]
\\	心(こころ)・知情意・心神・内心・心情・心魂・マインド・ハート・スピリット・エスプリ/
\\	精魂・気魄(きはく)・神気・気概・気力・意力・意志・気構え・気持ち/
\\	理念・本義・本旨(ほんし)・真義・真意・神髄(しんずい)/
\\	思想・気風・心性・メンタリティー	▲彼女は大量の本を読んで精神を陶冶した。 ▲不偏不党の精神で、などとかっこ付けているけど、結局のところ自分の意見を持っていないだけじゃないの。
\\	成人	せいじん	[名]スル 
\\	心身が発達して一人前になった人。 成年に達した人間。 おとな。 現在一般的には、満二〇歳以上の者をいう。 
\\	子供が成長して大人になること。 「娘はもう―して働きに出ている」	▲君はもう成人に達したのだから、もっと分別をもたなければならない。 ▲彼女はアメリカ生まれ、日本で成人した。
\\	精精	せいぜい	
\\	⦅名⦆ つとめはげむこと。 力の限りを尽くすこと。 幸若舞曲、大臣「―を尽して作りたつる弓の長さは八尺五寸」 
\\	⦅副⦆ 
\\	力の及ぶ限り。 精一杯。 「―努力します」 
\\	十分に多く見積もっても。 たかだか。 「―三日もあれば出来る」	▲桜の花は数日、せいぜい一週間しか持たない。 ▲私がそれに思いきって金が使えるのはせいぜいそれぐらいだ。
\\	成績	せいせき	
\\	成し遂げた仕事などの結果。 また、その結果の評価。 「営業―」 
\\	学業の評価や試験の結果。 「国語の―があがる」	▲一生懸命に勉強しなければ決して良い成績はとれない。 ▲営業成績はまさに鰻上り、というところかね。
\\	製造	せいぞう	[名]スル原料に手を加えて製品にすること。 「菓子を―する」「―販売」	▲はい。テレビを製造している工場を訪問します。 ▲テレビを製造している工場を訪問します。
\\	贅沢	ぜいたく	[名・形動]スル 
\\	必要な程度をこえて、物事に金銭や物などを使うこと。 金銭や物などを惜しまないこと。 また、そのさま。 「―を尽くす」「―な暮らし」「布地を―に使った服」「たまには―したい」 
\\	限度や、ふさわしい程度をこえること。 また、そのさま。 「―を言えばきりがない」「―な望み」 [派生]ぜいたくさ[名] [類語]
\\	豪奢(ごうしや)・豪勢・奢侈(しやし)・華奢(かしや)・驕奢(きようしや)・驕侈(きようし)・贅(ぜい)・奢(おご)り/
\\	分(ぶん)不相応・身の程知らず	▲彼は贅沢な暮らしをしている。 ▲彼は贅沢をし好き勝手なことをして暮らした。
\\	成長	せいちょう	[名]スル 
\\	人や動植物が育って大きくなること。 おとなになること。 「子供が―する」「ひなが―する」「経験が人を―させる」 
\\	物事の規模が大きくなること。 拡大。 「事業が―する」「経済の高度―」 [類語]
\\	生長・成育・生育・発育・発達・成熟(―する)育つ・生い立つ・長ずる/
\\	伸長・伸張・伸展・発展・拡大・膨脹(ぼうちよう)	▲雑草は成長が早い。 ▲四半期1
\\	の成長は年率4
\\	の成長率を意味する。
\\	制度	せいど	社会における人間の行動や関係を規制するために確立されているきまり。 また、国家・団体などを統治・運営するために定められたきまり。 「封建―」「貨幣―」 [類語]制(せい)・機構・体制・法制・仕組み・決まり・定め・掟(おきて)・システム	▲酷使されている不法外国人労働者は制度の隙間にこぼれ落ちてしまうことが多いのです。 ▲社会福祉制度は抜本的な改革が必要です。
\\	青年	せいねん	青春期の男女。 一〇代後半から二〇代の、特に男子をいうことが多い。 わかもの。 わこうど。 「―実業家」 [類語]若者・若人(わこうど)・若い者・若い衆(しゆ)・若造・青少年・ヤングジェネレーション・ヤング	▲登山は困難と冒険と伴うが故に、特に青年にとって魅力がある。 ▲田舎の青年の中には都会の生活にあこがれるものが多い。
\\	製品	せいひん	販売するためにつくった品物。 ある原料からつくった品物。 「プラスチック―」	▲海外製品に不公平な関税が課せられている。 ▲会議場内で自社の製品の展示場を設けたいとお考えでしたら、早急にご連絡下さい。
\\	政府	せいふ	政治を行う所。 立法・司法・行政のすべての作用を包含する、国家の統治機構の総称。 日本では、内閣および内閣の統轄する行政機構をさす。 [類語]行政府・政庁・政権・内閣・台閣・官府・官庁・官衙(かんが)・官(かん)・国(くに)・公(おおやけ)・お上(かみ)	▲我々はその国の新政府との友好関係を樹立した。 ▲我々は政府の経済政策を検討した。
\\	生物	せいぶつ	動物・植物・微生物など生命をもつものの総称。 細胞という単位からなり、自己増殖・刺激反応・成長・物質交代などの生命活動を行うもの。 いきもの。	▲海に住む大半の生物は汚染による悪影響を受けている。 ▲空気がなければ生物は生きていけないだろう。
\\	生命	せいめい	
\\	生物が生物でありつづける根源。 いのち。 「―の危険を冒す」「尊い―を犠牲にする」 
\\	ある方面で活躍しつづけることができる根源。 「政治―」 
\\	人や物事がよりどころとするもの、また、それなしには価値がなくなるもの。 いのち。 「車の―はエンジンにある」	▲水は生命に不可欠です。 ▲水は生命にとって不可欠である。
\\	整理	せいり	[名]スル 
\\	乱れた状態にあるものを整えて、きちんとすること。 「資料を―する」「気持ちの―がつく」「交通―」 
\\	無駄なもの、不要なものを処分すること。 「人員を―する」 
\\	株式会社が支払不能・債務超過に陥るおそれまたはその疑いがあるとき、再建を目的として裁判所の監督の下に行われる手続き。 商法に規定がある。 [用法]整理・整頓――「部屋の中を整理(整頓)しなさい」「書棚をきちんと整理(整頓)する」など、整えるの意では相通じて用いられる。 
\\	「整理」は、「交通整理」「感情の整理がつく」のように、混乱しているものをきちんとした状態にする意。 また、無駄なもの、余分なものを除く意もある。 「人員整理」「蔵書を整理する」など。 
\\	「整頓」は乱れているものの位置を元にもどし、整えること。 「教室の机を整頓する」「乱れた資料の順序を整頓する」	▲部屋の中をきちんと整理しなさい。 ▲何がなんだか分からない。頭がパニックに陥って整理がつかない。
\\	咳	せき	《「堰」と同語源》のど・気管の粘膜が刺激されたとき、反射的に呼吸を止め、短く強く吐き出す息。 また、その音。 しわぶき。 《季 冬》「―をする母を見あげてゐる子かな/汀女」	▲老人の話は何度か咳で中断された。 ▲風邪が胸にきて、咳と頭痛がする。
\\	石炭	せきたん	地中に堆積(たいせき)した過去の植物が、埋没後長い年月の間に分解・炭化した可燃性の岩石。 炭化の程度により泥炭・亜炭・褐炭・瀝青(れきせい)炭・無煙炭に分けるが、普通は瀝青炭をさす。 色は黒く緻密(ちみつ)で塊状。 多くは古生代石炭紀の植物を起源とするが、日本では古第三紀のものが主。 燃料・化学工業用原料にする。 《季 冬》「―を投じたる火の沈みけり/虚子」	▲石炭を石油の代わりに使いました。 ▲彼らは石炭をもやした。
\\	責任	せきにん	
\\	立場上当然負わなければならない任務や義務。 「引率者としての―がある」「―を果たす」 
\\	自分のした事の結果について責めを負うこと。 特に、失敗や損失による責めを負うこと。 「事故の―をとる」「―転嫁」 
\\	法律上の不利益または制裁を負わされること。 特に、違法な行為をした者が法律上の制裁を受ける負担。 主要なものに民事責任と刑事責任とがある。 [類語]
\\	責務・義務・任務・本務・使命・職責・重責・責め・務め	▲君はこの重大な事態の責任を免除されていないよ。 ▲君はその結果に責任がある。
\\	石油	せきゆ	
\\	種々の炭化水素の混合物を主成分とする液状の物質。 海底に堆積(たいせき)した生物遺体がバクテリアの作用や熱・圧力で分解してできたとされる。 天然のままのものを原油とよび、蒸留・精製してガソリン・灯油・軽油・ピッチなどを得る。 燃料・化学工業用原料として重要。 
\\	特に、灯油の俗称。	▲投機師のエドワードは石油事業で大金をえた。 ▲中東は日本が消費する石油のかなりの部分を供給する。
\\	世間	せけん	
\\	が原義》 
\\	人が集まり、生活している場。 自分がそこで日常生活を送っている社会。 世の中。 また、そこにいる人々。 「―を騒がした事件」「―がうるさい」「―を渡る」 
\\	人々との交わり。 また、その交わりの範囲。 「―を広げる」 
\\	仏語。 生きもの(衆生(しゆじよう)世間)と、それを住まわせる山河大地(器(き)世間)、および、生きものと山河大地を構成する要素(五陰(ごおん)世間)の総称。 
\\	人の住む空間の広がり。 天地の間。 あたり一面。 「俄(にはか)に霧立ち、―もかいくらがりて」 
\\	僧に対する一般の人。 俗人。 「ある律僧、―になりて子息あまたありけるうち」 
\\	社会に対する体面やそれに要する経費。 「―うちばに構へ、又ある時は、ならぬ事をもするなり」 
\\	この世の生活。 財産。 暮らし。 境涯。 「武州に―ゆたかなる、所の地頭あり」 [類語]
\\	世(よ)・世の中・天下・江湖(こうこ)・社会・実社会・世上・世俗・俗世・人世(じんせい)・人間(じんかん)・俗間・民間・巷間(こうかん)・巷(ちまた)・市井(しせい)・浮き世・娑婆(しやば)・塵界(じんかい)・世界	▲噂が世間に広まっている。 ▲ニクソンは世間から忘れられた。
\\	説	せつ	
\\	ある物事に対する主義、主張。 「新しい―をたてる」「御―御もっともです」 
\\	うわさ。 風説。 「彼女が結婚したという―がある」 
\\	漢文の一体。 道理を解釈したもの。 また、自分の意見を述べたもの。 韓愈の「師説」、柳宗元の「捕蛇者説」など。	▲厳密に言うと、その説は正しくない。 ▲私はこの説を信じることは出来ない。
\\	積極的	せっきょくてき	[形動]物事を進んでするさま。 「―に仕事に取り組む」↔消極的。	▲彼女は婦人解放運動で積極的な役割をした。 ▲彼女は仕事に対してとても積極的な態度を示している。
\\	設計	せっけい	[名]スル 
\\	建造物の工事、機械の製造などに際し、対象物の構造・材料・製作法などの計画を図面に表すこと。 「ビルを―する」 
\\	一般に、計画を立てること。 また、その計画。 「老後の生活を―する」	▲ご自分で設計なさったのですか。 ▲そのスペースシャトルは、宇宙ステーションに行くために設計された。
\\	絶対	ぜったい	[名・形動] 
\\	他に比較するものや対立するものがないこと。 また、そのさま。 「―の真理」「―な(の)存在」「―君主」 
\\	他の何ものにも制約・制限されないこと。 また、そのさま。 「―な(の)権力」 
\\	⇒絶対者 
\\	(副詞的に用いる) ㋐どうしても。 何がどうあっても。 「―に行く」「―合格する」 ㋑(あとに打消しの語を伴って)決して。 「―に負けない」「―許さない」「―反対」 [類語]
\\	至高・至上・唯一無二/
\\	不可侵・完全無欠/
\\	㋐)何が何でも・是が非でも・どうしても・必ず・誓って/
\\	㋑)決して・断じて・金輪際(こんりんざい)・ゆめ・ゆめゆめ	▲絶対トイレの蓋を開けたままにするなよ。 ▲神は絶対の存在である。
\\	セット	セット	[名]スル 
\\	組み合わせて一そろいにすること。 また、そのようなもの。 「食器を―で買う」「百科事典一―」 
\\	物を配置すること。 「テーブルを―する」 
\\	道具・機械などを組み立てて使えるようにすること。 装置すること。 「目ざまし時計を七時に―する」 
\\	映画やテレビで、撮影用に設ける建物・街路などの装置。 また、演劇の舞台装置。 
\\	テニス・卓球・バレーボールなどで、一試合中の一勝負。 ふつう、三セットあるいは五セットで一試合とする。 
\\	髪形を整えること。 「美容院で―する」 
\\	ラジオの受信機、テレビの受像機。	▲もし眠るといけないので目覚し時計をセットしなさい。 ▲両親がクリスマスに化学実験セットをくれた時、僕は10歳近い年齢だった。
\\	設備	せつび	[名]スル必要な建物・機器などを備えつけること。 また、備えつけたもの。 「下水の―が整う」「情報機器を―する」	▲その工場では、設備はすべて最新式のものだった。 ▲その実験室には最新式の設備がある。
\\	絶滅	ぜつめつ	[名]スル 
\\	生物の種などが滅びて絶えること。 「乱獲により―する」 
\\	残らず絶やすこと。 なくすること。 「交通事故を―する」	▲多くの弱い生物種が絶滅の危機に瀕している。 ▲絶滅の危機に瀕した海洋生物を保護する為に募金が設立された。
\\	節約	せつやく	[名]スルむだ遣いをやめて切りつめること。 「電気の―」「交際費を―する」	▲時間を節約するためにコンピューターを使った。 ▲私達は時間を節約するためにコンピューターを使った。
\\	責める	せめる	[動マ下一][文]せ・む[マ下二]《「攻める」と同語源》 
\\	過失・怠慢・違約などを取り上げて非難する。 とがめる。 なじる。 「失敗を―・める」「無責任な行為を―・める」 
\\	厳しく催促する。 せきたてる。 また、しつこく求める。 せがむ。 「早く返答しろと―・める」「子供に―・められて買い与える」 
\\	㋐苦しめる。 悩ませる。 「みずからを―・める」 ㋑苦痛を与えていじめ苦しめる。 いためつける。 「むちで―・めて白状を強いる」 
\\	目的を果たすために、積極的な働きかけをする。 「泣き落とし戦術で―・める」 
\\	一心に努力する。 真剣に追い求める。 「侘びの精神を―・める」「―・むる者は、その地に足を据ゑがたく」 
\\	馬を乗りならす。 調教をする。 「早朝から馬場で―・める」「生食(いけずき)・摺墨(するすみ)という名馬を―・めさせられしに」 [類語]
\\	咎(とが)める・詰(なじ)る・難ずる・嘖(さいな)む・吊(つる)し上げる・非難する・難詰する・面詰する・面責する・問責する・詰責する・叱責(しつせき)する・譴責(けんせき)する・弁難する・論難する・指弾する	▲彼の年齢を考えれば彼の行動は責められない。 ▲彼はその事故の責任は私にあると責めた。
\\	善	ぜん	よいこと。 道義にかなっていること。 また、そのような行為。 「―を積み、功を重ねる」「一日一―」↔悪。	▲美しいものは必ずしも善ではない。 ▲彼は善と悪の区別がわからない。
\\	全員	ぜんいん	その団体などに属するすべての人員。 総員。	▲彼の努力のおかげで、乗組員全員が救助された。 ▲彼の冗談はクラス全員を爆笑させた。
\\	専攻	せんこう	[名]スルある一つのことを専門に研究すること。 「社会学を―する」	▲ケイトはドイツ語を専攻している。 ▲スーザンはアメリカ史を専攻しています。
\\	全国	ぜんこく	
\\	その国全体。 
\\	すべての国々。	▲リンカーンは、全国の奴隷を解放せよと命令した。 ▲悪性の風邪が全国で流行っている。
\\	先日	せんじつ	近い過去のある日。 このあいだ。 過日。 「―お会いしましたね」	▲先日私達のところに、奥さんに赤ん坊が生まれかけている男の人から電話がかかってきました。 ▲先日私は通りで彼にあった。
\\	前者	ぜんしゃ	二つ示したもののうち、前のもの。 ↔後者。	▲私は能楽が歌舞伎より好きだが、それは前者が後者よりも優雅に思えるからである。 ▲私は犬の方が猫より好きです。なぜなら前者の方が後者より忠実だからです。
\\	選手	せんしゅ	
\\	競技会・試合などに選ばれて出場する人。 「オリンピック―」 
\\	スポーツを職業にする人。 「有名な野球―」	▲彼の兄さんは有名なサッカーの選手です。 ▲彼のお兄さんは有名なサッカーの選手です。
\\	前進	ぜんしん	[名]スル 
\\	前へ進むこと。 「―してフライを捕る」↔後進/後退。 
\\	物事がよいほうへ動くこと。 「学力に―が見られる」「解決に向かって一歩―する」	▲私には前進する勇気がなかった。 ▲仕官は兵士達に前進を命じた。
\\	センター	センター	
\\	中心。 中央。 
\\	その分野・部門の中心的役割をする機関や施設。 「ビジネス―」「文化―」 
\\	球技で、中央のポジション。 野球では外野の中央部、また、中堅手。 アメリカンフットボールではスクリメージラインの中央に位置する攻撃側の選手。	▲センターの選手はウイニングボールをガッチリとった。 ▲そのセンターの目的は、ある特定の期間内に他の国々からの若者を訓練することである。
\\	全体	ぜんたい	㊀[名] 
\\	からだのすべての部分。 全身。 
\\	あるひとまとまりの物事のすべての部分。 「組織の―にかかわる問題」「―の構造を把握する」「画用紙の―を使って描く」「―像」 ㊁[副] 
\\	もともと。 もとより。 「―自分が悪いのだ」 
\\	(あとに疑問を表す語を伴って)いったい。 いったいぜんたい。 「―これはどういうことか」 [類語] 
\\	総体・全部・全般・全面・全豹(ぜんぴよう)・全容・全貌(ぜんぼう)・すべて	▲今のところ、全体の意見の一致には至っていない。 ▲私たちは畑全体に種をまいた。
\\	選択・撰択	せんたく	[名]スル 
\\	多くのものの中から、よいもの、目的にかなうものなどを選ぶこと。 「―を誤る」「テーマを―する」「取捨―」 
\\	「選択科目」の略。必修科目以外に、学生・生徒が各人の個性と必要に応じて自由に選択履修することができる科目。一定のわく内で選択する選択必修科目と、まったく自由に選択できる自由選択科目とがある。一般に中等教育以上の段階で採用されている制度。	▲何が正しくて何が間違いか選択するのは難しいが、選択しなければならない。 ▲目標は授業設計をするときの、学生の思考を触発するメディア教材の選択および活用方法について理解することである。
\\	艘	そう	〔接尾〕助数詞。 比較的小さい船を数えるのに用いる。 「屋形船一―」	▲むろん、舟は何百艘も水面を動きまわっているが、どれでもいいというわけにはいかない。 ▲これらの人々が3艘の船で彼の国へやってきました。
\\	象	ぞう	長鼻目ゾウ科の哺乳類の総称。 陸上動物では最大。 頭部が巨大で、鼻は上唇とともに長く伸び、人間の手と同様の働きをする。 上あごの門歯が伸びて牙(きば)となり、臼歯(きゆうし)は後ろから前へずれながら生え替わる。 現生種はアフリカゾウ・アジアゾウに大別され、化石種にはマンモス・ナウマンゾウなどがある。	▲象は陸上の動物の中で最も大きい。 ▲象は猟師に殺された。
\\	騒音	そうおん	騒がしく、不快感を起こさせる音。 また、ある目的に対して障害になる音。 計量的には八〇デシベル以上の大きな音。 「―防止」	▲その騒音に慣れるのに長い時間かかった。 ▲その騒音に我慢できない。
\\	増加	ぞうか	[名]スル物の数量がふえること。 また、ふやすこと。 「人口が―する」↔減少。	▲人口はどんどん増加ししていた。 ▲人口は増加しつつある。
\\	操作	そうさ	[名]スル 
\\	機械などをあやつって動かすこと。 「ハンドルを―する」「遠隔―」 
\\	自分の都合のよいように手を加えること。 「株価を―する」「帳簿を―する」	▲この機械の操作は私には難しすぎる。 ▲この機械はどうやって操作するのですか。
\\	想像	そうぞう	[名]スル実際には経験していない事柄などを推し量ること。 また、現実には存在しない事柄を心の中に思い描くこと。 「―をたくましくする」「―上の動物」「縄文人の生活を―する」「―したとおりの結果になる」 [類語]推測・臆測(おくそく)・仮想・想見・空想・夢想・幻想・連想・イマジネーション(―する)思い描く	▲私がその光景を見たときの恐ろしさをちょっと想像して下さい。 ▲残念ながら君の想像は見当違いだ。
\\	相続	そうぞく	[名]スル 
\\	家督・地位などを受け継ぐこと。 跡目を継ぐこと。 「宗家を―する」 
\\	法律で、人が死亡した場合に、その者と一定の親族関係にある者が財産上の権利・義務を承継すること。 現行民法では財産相続だけを認め、共同相続を原則とする。	▲長男がすべての財産を相続した。 ▲私は彼の財産を相続するだろう。
\\	装置	そうち	[名]スル 
\\	ある目的のために、機械・器具などをそなえつけること。 また、その設備。 「濾過(ろか)器を―する」「安全―」 
\\	舞台装置のこと。	▲例えば、小鳥は特別な防御装置を備えている。 ▲有効な触媒がないので、その装置を改良することは困難であろう。
\\	相当	そうとう	㊀[名・形動]スル 
\\	価値や働きなどが、その物事とほぼ等しいこと。 それに対応すること。 「五〇〇円―の贈り物」「ハイスクールは日本の高校に―する」 
\\	程度がその物事にふさわしいこと。 また、そのさま。 「能力―の地位」「それ―な(の)覚悟がいる」「収入に―した生活」 
\\	かなりの程度であること。 また、そのさま。 「―な(の)成果をおさめる」 ㊁[副]物事の程度が普通よりはなはだしいさま。 かなり。 「―勉強したらしい」	▲それを発明した教授は大学から相当の対価を受ける権利がある。 ▲被害額は相当なものになるだろう。
\\	速度	そくど	
\\	物事の進む速さ。 スピード。 「一定の―で歩く」「制限―」 
\\	単位時間に進んだ距離に方向を合わせたベクトル量。 運動する物体の単位時間当たりの位置の変化を表す。 単位にはメートル毎秒のほか、ノットが用いられる。 [用法]速度・速力・速さ――「速度(速力・速さ)を一定に保つ」など、進む程度を表す語としては相通じて用いられる。 
\\	「速度」は自動車など物体が単位時間内に移動する距離の大小についていうほかに、機械などが単位時間内にする仕事の量の大小についても用いる。 「高速道路を一〇〇キロの速度で走る」「この機械は一分間に百部の速度で製本する」
\\	「速力」は一般に、移動する距離の大小について表す。 「この列車は現在時速二二〇キロの速力で走っております」
\\	「速さ」は最も普通に使われ、移動の距離についても仕事の量についても用いる。 「飛ぶような速さで通り過ぎた」	▲科学技術は我々の生活のあらゆる側面に浸透するようになり、その結果として社会は全く前例のない速度で変化しつつある。 ▲我々は流感の広がる速度を鈍らすことができるだろうか。
\\	底	そこ	
\\	物のいちばん下。 ㋐容器その他くぼみのある物の、いちばん下の平らな部分。 「コップの―」「箱の―が抜ける」 ㋑地面・水面から離れたいちばん下の所。 「地の―」「海の―」 ㋒重なりのいちばん下。 「積荷の―」 
\\	物事の極まるところ。 はて。 極限。 際限。 「―の知れない実力」 
\\	奥深い所。 「腹の―から笑う」 
\\	相場が下落して、いちばん安くなったところ。 ↔天井。 
\\	そのものがもつ真の力量。 実力。 「義経が乗たる大鹿毛(おほかげ)は…薄墨にも―はまさりてこそあるらめ」 [下接語]奥底・心の底・心(しん)底・手(たな)底・谷底・奈落の底・水(みな)底・胸(むな)底(ぞこ)上げ底・石底・糸底・大底・織底・川底・靴底・どん底・鍋(なべ)底・二重底・平(ひら)底・船(ふな)底	▲彼は心の底でその知らせを喜んだ。 ▲彼は心の底から笑った。
\\	其処で	そこで	[接] 
\\	前述の事柄を受けて、次の事柄を導く。 それで。 そんなわけで。 「いろいろ意見された。 ―考えた」 
\\	話題をかえたり、話題をもとにもどしたりすることを示す。 さて。 「―一つお願いがあります」	▲彼は公園に行って、そこで一休みした。 ▲彼は私に前日そこで彼女に会ったと言った。
\\	組織	そしき	[名]スル 
\\	組み立てること。 組み立てられたもの。 「奥座敷は一種の宿屋見た様な―に出来ている」 
\\	一定の共通目標を達成するために、成員間の役割や機能が分化・統合されている集団。 また、それを組み立てること。 「組合を―する」「全国―」 
\\	生物体を構成する単位の一で、同一の形態・機能をもつ細胞の集まり。 さらに集まって器官を構成する。 動物では上皮組織・結合組織・筋肉組織・神経組織、植物では分裂組織・永久組織などに分けられる。 
\\	岩石を構成する鉱物の結晶度・大きさ・形・配列などのようす。 石理。 
\\	織物で、縦糸と横糸とを組み合わせること。 織り方。 「ラシャは―が密だ」 [類語]
\\	構成・編成・編制・組成・構造・機構・体制・体系・組立て・仕組み・システム/
\\	団体・結社・集団	▲自然環境の回復を宣伝する組織がリサイクルに力を入れて、植林に貢献しないのは何故か。 ▲組織培養の視点からは、この実験の環境はもっと厳密に規定されるべきである。
\\	注ぐ・灌ぐ	そそぐ	[動ガ五(四)]《室町時代ごろまで「そそく」》 
\\	㋐流れ入る。 流れ込む。 「淀川は大阪湾に―・ぐ」 ㋑雨や雪などがとぎれなく降りかかる。 「雨が―・ぐ」「降り―・ぐ」 
\\	㋐流し入れる。 また、容器に水などをつぐ。 「田に水を―・ぐ」「茶碗に酒を―・ぐ」 ㋑涙を流す。 「熱い涙を―・ぐ」 ㋒水などを上からかける。 ふりかける。 「盆栽に水を―・ぐ」 ㋓もっぱら、その方へ向ける。 一つことに集中する。 「心血を―・ぐ」「視線を―・ぐ」「全力を―・ぐ」 [可能]そそげる [下接句]油を注ぐ・意を注ぐ・心血を注ぐ・火に油を注ぐ・満面朱を濺(そそ)ぐ・目を注ぐ	▲彼女はブランデーをグラスに注いだ。 ▲彼女の視線は赤ん坊に注がれた。
\\	育つ	そだつ	㊀[動タ五(四)] 
\\	生まれた生物が時間がたつにつれてしだいに大きくなり、成熟に向かう。 成長する。 生長する。 「母乳で―・つ」「麦が―・つ」 
\\	鍛えられ、力を身につけて一人前になる。 「若手の投手が―・つ」「研究者が―・つ」 
\\	㋐小さな規模で始めた物事が順調に発展する。 「民主政治が―・つ」「会社が―・つ」 ㋑ある考え方・気持ちなどが着実に伸びる。 「自立心が―・つ」「愛情が―・つ」 ㊁[動タ下二]「そだてる」の文語形。	▲ローラ・インガルスは、大草原で育った。 ▲雨が降った後は植物がすくすく育つ。
\\	そっくり	そっくり	㊀[形動]非常によく似ているさま。 「父親に―な顔」 ㊁[副]欠けることのないさま。 そのまま。 残らず。 全部。 「昔の建物が―残る」	▲彼女の肖像画は本物そっくりだ。 ▲彼女は何から何まで母親そっくりだ。
\\	そっと	そっと	[副]スル 
\\	音を立てないように物事をするさま。 静かに。 「障子を―閉める」 
\\	他人に気づかれないように物事をするさま。 こっそり。 ひそかに。 「秘密を―打ち明ける」 
\\	干渉しないで、静かにしておくさま。 「しばらく―しておこう」 
\\	少し。 ちょっと。 「沢山にあらば、皆々を申し入れうが、―した樽を下された程に」	▲その母親は赤ちゃんをそっとベッドに横たえた。 ▲少しの間そっとしてやったほうがいいですね。
\\	袖	そで	
\\	衣服の身頃(みごろ)について、両腕を覆うもの。 和服ではたもとの部分を含めていう。 「―をたくしあげる」 
\\	建造物・工作物などの本体の両わき、または片方にあるもの。 門のわきの小さな門、机のわきの引き出しなど。 
\\	舞台の左右の端。 「―で出を待つ」 
\\	文書の初め、右端の余白。 
\\	鎧(よろい)の付属具。 肩からひじの部分を覆い、矢や刀を防ぐもの。 
\\	牛車(ぎつしや)や輿(こし)などで、出入り口の左右の張り出した部分。 [下接語]大袖・角袖・片袖・元禄(げんろく)袖・小袖・籠手(こて)袖・袞竜(こんりよう)の袖・七分袖・削(そぎ)袖・誰(た)が袖・筒袖・壺(つぼ)袖・詰め袖・鉄砲袖・留袖・長袖・薙(なぎ)袖・名残の袖・花の袖・半袖・平袖・広袖・振袖・巻き袖・丸袖・諸(もろ)袖・両袖	▲セーターの袖がほぐれ始めた。 ▲ない袖は振れぬ。
\\	備える・具える	そなえる	[動ア下一][文]そな・ふ[ハ下二] 
\\	ある事態が起こったときにうろたえないように、また、これから先に起こる事態に対応できるように準備しておく。 心構えをしておく。 「万一に―・える」「地震に―・える」「試験に―・えて夜遅くまで勉強する」 
\\	必要なときにいつでも使えるように、前もって整えておく。 設備や装置を用意しておく。 「各室に空調設備が―・えてある」「応接セットを―・える」 
\\	必要なものを、どこも足りないところがないように持っている。 具備する。 「資格を―・える」「あらゆる条件を―・えている」 
\\	生まれたときから自分のものとしてもっている。 身につけている。 「人徳を―・えている」 [類語]
\\	準備する・用意する/
\\	備え付ける・しつらえる・設ける・設備する・設置する・装備する・装置する・完備する/
\\	有する・具備する・具有する	▲その部屋は寝台が2台備えられている。 ▲我々は将来に備えねばならない。
\\	園・苑	その	
\\	果樹・花・野菜などを植えた一区画の土地。 庭園。 「梅の―」 
\\	ある物事が行われる場所。 また、ある特定の場所。 「学びの―」「女の―」	▲彼らは私の果実園を荒らした。 ▲彼らは私の果実園からりんごを盗んだ。
\\	其の儘	そのまま	㊀[名] 
\\	その状態のとおりで変化のないこと。 もとのまま。 今のまま。 あるがまま。 「―の姿勢でいる」「見てきた―を話す」 
\\	そのものに完全に似ていること。 そっくり。 「実戦―の演習」 ㊁[副]前の動作から、すぐ次の動作に移るさま。 すぐに。 「横になるなり、―寝てしまった」	
\\	蕎麦	そば	
\\	タデ科の一年草。 高さ四〇〜七〇センチ。 茎は赤みを帯び、葉は心臓形で先がとがる。 夏または秋、白色や淡紅色の花びら状の萼(がく)をもつ小花を総状につける。 実は三角卵形で緑白色、乾くと黒褐色になり、ひいてそば粉を作る。 中央アジアの原産で、古く渡来し栽培される。 そばむぎ。 くろむぎ。 《季 花=秋》「山畠や―の白さもぞっとする/一茶」 
\\	「蕎麦切(そばき)り」の略。そば粉につなぎを加え、水でこねて薄く延ばし、細く切ったもの。ゆでてつけ汁につけたり、汁をかけたりして食べる。	▲そばはそば粉から、うどんやきしめんは普通の小麦粉からできてるの。 ▲長野にいる友人を訪問した際、おいしいそばをごちそうになった。
\\	ソファー	ソファー	背もたれがあり、クッションの利いた長椅子ながいす。 ソーファ。 夏目漱石、それから「櫛は長椅子ソーフアの足の所にあつた」	▲少年はソファーに横になっていた。 ▲私がその部屋に入っていくと、そこでは子供たちはソファーに座っていた。
\\	粗末・麁末	そまつ	[名・形動] 
\\	作り方などが、大ざっぱなこと。 品質などが上等でないこと。 また、そのさま。 「―な家」「―な食事」→御粗末(おそまつ) 
\\	いいかげんに扱うこと。 ないがしろにすること。 また、そのさま。 「食べ物を―にするな」「―な扱いを受ける」	▲彼の食べるものは粗末だった。 ▲食べ物を粗末にしたくないので彼は全部食べた。
\\	剃る	そる	[動ラ五(四)]毛髪やひげなどを、かみそりなどで根元からきれいに切り落とす。 「ひげを―・る」 ◆なまって「する」ともいう。 [可能]それる	▲ひげをそる前に石けんをつける。 ▲ひげを剃って下さい。
\\	各・各々	おのおの	《「己(おの)己(おの)」の意》 ㊀[名]多くのもののそれぞれ。 めいめい。 副詞的にも用いる。 「学生―の自覚にまつ」「入選作は―すぐれている」→其(そ)れ其(ぞ)れ[用法] ㊁[代]二人称の人代名詞。 皆さん。 「是御覧ぜよ、―」	
\\	損	そん	[名・形動] 
\\	利益を失うこと。 また、そのさま。 不利益。 「―を出す」「―な取引」↔得。 
\\	努力をしても報われないこと。 また、そのさま。 「正直者が―をする」「―な性分」↔得。 
\\	そこなうこと。 こわすこと。 「一命を―にすべきなり」 [類語]
\\	不利益・損失・損害・損亡(そんもう)・欠損・実損・差損・赤字・出血・持ち出し・採算割れ/
\\	不利・不利益・不為(ふため)・不得策	▲我々はその仕事で損をした。 ▲月々給料の一部を貯金しておけば損はないよ。
\\	損害	そんがい	[名]スルそこない、傷つけること。 利益を失わせることや、失うこと。 また、事故などで受けた不利益。 損失。 「取引で―をこうむる」「吾人の安全幸福を―するは必然なり」	▲台風がその町を直撃し、ひどい損害を与えた。 ▲損害は百万ドルにのぼる。
\\	尊敬	そんけい	[名]スル 
\\	その人の人格をとうといものと認めてうやまうこと。 その人の行為・業績などをすぐれたものと認めて、その人をうやまうこと。 「互いに―の念を抱く」「―する人物」 
\\	文法で、聞き手や話題の主、また、その動作・状態などを高めて待遇する言い方。 →尊敬語 [類語]
\\	敬愛・敬慕(けいぼ)・敬仰(けいぎよう)・景仰(けいこう)・畏敬(いけい)・崇敬(すうけい)・崇拝・私淑(ししゆく)・傾倒・心酔・心服・敬服(―する)敬(うやま)う・敬(けい)する・尊(とうと・たつと)ぶ・崇(あが)める・仰(あお)ぐ・慕(した)う	▲私は先生をとても尊敬している。 ▲私は人間的には彼が好きだが医者としては尊敬していない。
\\	存在	そんざい	[名]スル 
\\	人間や事物が、あること。 また、その人間や事物。 「神の―を信じる」「歴史上に―する人物」「クラスの中でも目立つ―」 
\\	(ドイツ)
\\	哲学で、あること。 あるもの。 有。 ㋐実体・基体など他のものに依存することなく、それ自体としてあるもの。 ㋑ものの本質としてあるもの。 ㋒現実存在としてあることやあるもの。 特に、人間の実存。 ㋓現象として主観に現れているものや経験に与えられているもの。 ㋔判断において、主語と述語とを結合する繋辞(けいじ)。 
\\	はpである」の「ある」。 [類語]
\\	実在・実存・現存・現在・厳存(げんそん)・存立(―する)存(そん)する・在(あ)る・居(い)る	▲獲物がいなければ、猟師は存在することができない。 ▲会社にパンタロンとアロハシャツを着てくる彼は、かなりユニークな存在だ。
\\	尊重	そんちょう	[名]スル価値あるもの、尊いものとして大切に扱うこと。 「人権を―する」	▲彼女は他人の意見を尊重する。 ▲法の尊重が我々の社会の基本だ。
\\	他	た	
\\	示されたもの以外のもの。 ほか。 「―は推して知るべし」「―チーム」 
\\	自分以外の人。 ほかの人。 他人。 「―の迷惑を顧みない」 
\\	ほかの場所。 よそ。 「住所を―に移す」	▲彼女は彼を他の誰かとまちがえた。
\\	田	た	耕して稲などを栽培する土地。 ふつうは水を引き水稲を栽培する水田をさす。 畑に対していう。 たんぼ。 「―を打つ」 [下接語]青田・荒(あら)田・新(あら)田・荒れ田・植え田・門(かど)田・刈り田・黒田・塩田・代(しろ)田・白田・泥田・沼田・冬田・古田・水田(だ)浅田・稲(いな)田・陸(おか)田・小(お)田・牡蠣(かき)田・隠し田・草田・棚田・築(つき)田・苗代(なわしろ)田・野田・蓮(はす)田・?(ひつじ)田・深田・外持(ほま)ち田・谷(やち)田・病(やまい)田・山田・早稲(わせ)田	
\\	対	たい	
\\	対照をなすこと。 また、反対の関係にあること。 「男の―は女」 
\\	互いに相手関係にあること。 「巨人―中日戦」 
\\	数を表す語の間に入れて、数量の比例・割合を表す語。 「三―二の割でまぜる」 
\\	同等の力量・資格であること。 「―で碁を打つ」 
\\	二つで一組みとなるもの。 つい。 「ある時宣宗が一句を得て―を挙人(きよじん)中に求めると」 
\\	「対の屋」の略。 「ひんがしの―の西の廂」 
\\	名詞などの上に付けて、比較・交渉・戦いなどの相手であることを表す。 「―前年比」「―欧州貿易」	▲3対0で彼には負けています。 ▲3対1で負けた。
\\	題	だい	㊀[名] 
\\	書物、または文章・芸術作品などの趣旨・内容を簡潔に、総括的に示す見出しの語句。 標題。 タイトル。 「―をつける」 
\\	詩歌などに詠み込む事柄。 「秋の風物を―に歌を詠む」 
\\	解答を求めて出す質問。 問題。 ㊁〔接尾〕助数詞。 試験などの問題を数えるのに用いる。 「五―できた」 [類語] 
\\	題名・題目・題号・標題・表題・外題(げだい)・内題・名題(なだい)・作品名・書名・書目・編目・演題・画題・タイトル	▲「秘めた恋」という題の作文で、メアリーは賞をもらった。 ▲合わせて600題はきつい。
\\	体育	たいいく	知育・徳育に対して、適切な運動の実践を通して身体の健全な発達を促し、運動能力や健康な生活を営む態度などを養うことを目的とする教育。 また、その教科。	▲体育の授業は全員が必修です。 ▲僕は体育の授業中に怪我をした。
\\	体温	たいおん	動物体の温度。 体内の物質代謝の反応によって生じ、定温動物ではほぼ一定、変温動物では外界の温度とともに変化する。 人間ではセ氏三六・五〜三七・〇度が普通。	▲眠ると体の機能がゆっくりになり、体温が下がる。 ▲六時間おきに体温を測った。
\\	大会	たいかい	
\\	大規模な集まり。 大きな会合。 「弁論―」 
\\	ある組織や会の全体的な会合。 「党―」「組合―」	▲高校になってからは、クロスカントリースキー、ノルディック複合競技の大阪大会および近畿大会で幾度となく優勝。 ▲毎年、浜松で凧揚げ大会が開かれる。
\\	大気	たいき	㊀[名]天体の表面を取り巻いている気体の層。 普通は地球の空気をさし、地上一〇〇〇キロまで存在して太陽の強烈な紫外線やX線を遮るとともに保温の役割などを果たす。 その九五パーセントは地上二〇キロ以下にあって、上層ほど希薄。 ㊁[形動][文][ナリ]心が広く、こせこせしないさま。 「―な人で…札(さつ)を撒いて歩いたという話を」	▲空気の場合、大気中には常にある程度の湿気があるが、その量が大幅に増えると光の波に影響してくる。 ▲高く昇るにつれて大気は薄くなる。
\\	代金	だいきん	品物の買い手が売り手に支払う金。 代価。	▲彼は車の代金を翌月まで払わなくてもよいように取り決めた。 ▲彼は私たちに代金を払ってくれるように要求した。
\\	退屈	たいくつ	[名・形動]スル 
\\	することがなくて、時間をもてあますこと。 また、そのさま。 「散歩をして―をまぎらす」「読む本がなくて―する」 
\\	飽き飽きして嫌けがさすこと。 また、そのさま。 「―な話」「―な人」 
\\	疲れて嫌になること。 「呪咀(じゆそ)することの及ばぬとて、かの后たち―し給ふ」 
\\	困難にぶつかってしりごみすること。 「聞きしにもなほ過ぎたれば、官軍御方(みかた)を顧りみて、―してぞ覚えける」 
\\	仏語。 修行の苦難に負け、精進の気をなくすこと。 [派生]たいくつさ[名]	▲彼のスピーチは私を退屈にさせた。 ▲彼のジョークはいつも退屈だ。
\\	滞在	たいざい	[名]スルよそに行って、ある期間そこにとどまること。 逗留(とうりゆう)。 「ホテルに―する」	▲いつまでご滞在ですか。 ▲いつまでの滞在予定ですか。
\\	大使	たいし	
\\	外交使節の最上位のもの。 臨時的な特派大使と常駐の特命全権大使とがあるが、普通には後者をさす。 
\\	朝廷・幕府などの中央政府機関の命を受けて公式に派遣される使者。 
\\	遣唐使の正使。	▲駐日英国大使はだれだか知っていますか。 ▲大使は今夜日本をたつ予定です。
\\	大した	たいした	[連体] 
\\	程度がはなはだしいさまをいう語。 非常な。 たいへんな。 度はずれた。 「―ものだ」「―数にのぼる」 
\\	あとに打消しの語を伴って、特に取り立てて言うほどのことではないという気持ちを表す。 それほどの。 「―ことはない」「―用事ではない」	▲彼らは大したお金もなしにどうにか暮らしている。 ▲彼は大した人物にはなれないわね。
\\	対象	たいしょう	
\\	行為の目標となるもの。 めあて。 「幼児を―とする絵本」「調査の―」 
\\	哲学で、主観・意識に対してあり、その認識や意志などの作用が向けられるもの。	▲顕微鏡を使って、さらに一層、物質の核心近くまで迫ることはあるだろうが、微生物学でさえ、客観的なものであり、対象と観察者との間にスペースを置くことによって知識を拡大していくのである。 ▲この本は子供を対象とした本です。
\\	大臣	だいじん	
\\	国務大臣または各省大臣の称。 
\\	律令制で、太政官の長官。 太政大臣・左大臣・右大臣・内大臣をいう。 おおいもうちぎみ。 おとど。 おおおみ。	▲社長は大臣にそでの下を使った。 ▲新大臣は月曜日に職務の引継をした。
\\	対する	たいする	[動サ変][文]たい・す[サ変] 
\\	二つのものが向かい合う。 また、あるものに向き合う。 「机を挟んで二人が―・する」「海に―・して建つ家」 
\\	ある対象に向かう。 「家族に―・する思いやり」「将来に―・する期待」 
\\	人に接する。 応対する。 「客に愛想よく―・する」 
\\	立ち向かう。 相手にして争う。 「最強チームに―・する」 
\\	対応する。 応じる。 「質問に―・する答え」「記事に―・して反響があった」 
\\	比較する。 比べる。 「プロに―・しても見劣りしない」 
\\	対(つい)になる。 「善に―・する悪」	
\\	大戦	たいせん	参加兵員の多い激しい戦争。 また、広範囲にわたって行われる大規模な戦争。 特に、第一次または第二次大戦をいう。	▲あの家の繁栄は大戦中からのことだ。
\\	態度	たいど	
\\	物事に対したときに感じたり考えたりしたことが、言葉・表情・動作などに現れたもの。 「落ち着いた―を見せる」「―がこわばる」 
\\	事に臨むときの構え方。 その立場などに基づく心構えや身構え。 「慎重な―を示す」「反対の―を貫く」「人生に対する―」 
\\	心理学で、ある特定の対象または状況に対する行動の準備状態。 また、ある対象に対する感情的傾向。	▲加盟国中数カ国は、G7の協定に従うことにあいまいな態度を見せています。 ▲医者は威厳ある態度で患者を診察した。
\\	大統領	だいとうりょう	
\\	共和制国家の元首。 国民の直接選挙または議会での選挙などによって選出される。 米国などのように行政府の首長として強い権限をもつ国もあれば、形式的存在の国もある。 
\\	芝居に出演している役者などに対して、親しみを込めて呼び掛けるときに用いる語。 「待ってました、―」	▲今日遅くに大統領は記者会見を開く。 ▲今や大統領はレイム・ダックにすぎません。
\\	大半	たいはん	全体の半数を超えていること。 半分以上。 過半。 大部分。 副詞的にも用いる。 「出席者の―は初心者だ」「構想は―でき上がっている」	▲法律用語の大半は素人にはわかりにくい。 ▲父のわずかの蔵書は主に論争神学の本から成り立っていたが、その大半を読んでいた。
\\	代表	だいひょう	[名]スル 
\\	その中の一部であるものが全体をよく表していること。 また、そのもの。 「日本を―する風景」「若者を―する意見」 
\\	法人・団体や多数の人に代わって、その意思を他に表示すること。 また、その人。 「卒業生を―して答辞を述べる」「世界会議の日本―」 
\\	その技術や能力が特にすぐれているということで、ある集団の中から選ばれた人。 「―選手」	▲ボブはクラスを代表してお礼のことばを述べた。 ▲もっとも優秀な学生がクラスを代表して感謝の意をあらわした。
\\	大部分	だいぶぶん	ほとんどの部分。 副詞的にも用いる。 大半。 おおかた。 「出席者の―が賛成する」「仕事は―終わった」	▲大部分の人々は都市部に住んでいる。 ▲大部分の日本車は信頼できる。
\\	タイプライター	タイプライター	指で鍵盤(けんばん)をたたいて、文字や記号を紙面に印字する機械。 欧文のものは米国のC=L=ショールズが試作しレミントン父子社が一八七三年に実用機を販売。 仮名文字のものは明治三一年(一八九八)黒沢貞次郎が創案。 和文のものは杉本京太が大正五年(一九一六)に発売。 印字機。	▲彼女はタイプライターの経験はないし、そういう技術もない。 ▲彼女はこのタイプライターを使ってもよろしい。
\\	逮捕	たいほ	[名]スル 
\\	人の身体に直接力を加えて身柄を拘束すること。 
\\	検察官などの捜査機関が裁判官の発する令状(逮捕状)で被疑者を引致し、一定期間抑留するための強制手段。 現行犯は、だれでも逮捕状なしに逮捕できる。 「真犯人を―する」	▲少年は暴動に関連したために逮捕された。 ▲私は彼の逃亡助けたので逮捕された。
\\	ダイヤ	ダイヤ	《「ダイヤモンド」の略》 
\\	金剛石。 
\\	トランプで赤い?の模様。 また、その模様のついた札。	▲彼女は左手の薬指にダイヤの指輪をはめていた。 ▲列車はダイヤどおりに動いている。
\\	太陽	たいよう	
\\	太陽系の中心にある恒星。 地球からの距離は約一・五億キロ。 直接見える部分を光球といい、外側には彩層やコロナがある。 光球の半径は地球の一〇九倍、質量は三三万倍、平均密度は一・四。 表面温度はセ氏約六〇〇〇度。 恒星としては大きさも明るさもふつうの星で、エネルギーは中心における水素の核融合反応によってまかなわれている。 地球上の万物を育てる光と熱の源(みなもと)となっている。 
\\	物事の中心となるもの、人に希望を与えるもの、輝かしいものなどのたとえ。 「心の―」 [類語]日(ひ)・天日(てんじつ)・日輪(にちりん)・火輪(かりん)・金烏(きんう)・日天子(につてんし)・白日(はくじつ)・赤日(せきじつ)・烈日(れつじつ)・お日様・お天道(てんと)様・今日(こんにち)様・サン・ソレイユ (太陽の光)陽光・日光・日色(につしよく)・日差し・日影	▲太陽がなかったら、地上に生物は存在できないだろう。 ▲太陽がなければ、あらゆる生き物は死ぬだろう。
\\	平ら	たいら	㊀[形動][文][ナリ] 
\\	高低や起伏のないさま。 でこぼこでないさま。 「―な地形」「運動場を―にならす」 
\\	おだやかで安定しているさま。 「心を―にする」「世が―に治まる」 
\\	(多く「おたいらに」の形で用いる)かしこまってすわったりしないで、楽な姿勢でいるさま。 「皆―に、趺坐(あぐら)をかき給え」 ㊁[名](地名の下に付け、多く「だいら」の形で)山に囲まれた、広い平地。 「松本―」「日本―」	▲地球は平らだと信じられていた。 ▲人々はかつて世界は平らだと信じていた。
\\	代理	だいり	[名]スル 
\\	本人に代わって事を処理すること。 また、その人。 「父の―を務める」「交渉に―を立てる」「課長―」 
\\	ある人が、本人のために第三者に対して意思表示を行い、または第三者から意思表示を受けることによって、その法律効果が直接本人について生じること。 法定代理・任意代理などがある。	▲彼はその会に社長の代理で出席した。 ▲彼は兄の代理として集会に参加した。
\\	大陸	たいりく	
\\	海面上に現れている広大な陸地。 ユーラシア・アフリカ・北アメリカ・南アメリカ・オーストラリア・南極の六大陸がある。 
\\	日本からアジア大陸、特に中国をさしていう語。 
\\	英国からヨーロッパ大陸をさしていう語。	▲翌朝になって初めて大陸が見えた。 ▲アフリカは大陸であるがグリーンランドはそうではない。
\\	倒す	たおす	[動サ五(四)] 
\\	力を加えて、立っている状態のものを横にする。 横にねかす。 また、ころばす。 「木を―・す」「からだを―・す」「足をかけて―・す」 
\\	(「殪す」「斃す」「仆す」とも書く)殺す。 「銃で―・す」「一刀のもとに―・す」 
\\	勝負で相手を打ち負かす。 打ち破る。 「優勝候補のチームを―・す」 
\\	政体・国家などを存続できなくする。 くつがえす。 滅ぼす。 転覆する。 「内閣を―・す」「幕府を―・す」 
\\	借りた金を返さず、相手に損害を与える。 ふみたおす。 「飲食代を―・す」 [可能]たおせる [類語]
\\	ひっくり返す・覆(くつがえ)す・転がす・転ばす・倒(こか)す (横にする)寝かす・横たえる/
\\	破る・打ち破る・打ち負かす・打ち取る・下す・屠(ほふ)る・やっつける・打倒する・ノックアウトする	▲シート横にあるノブを前に動かし、シートを倒します。 ▲そのボクサーは倒されて10分後にようやく意識が回復した。
\\	タオル	タオル	
\\	布面に小さな糸の輪を織り出した綿織物。 柔軟で吸水性に富む。 タオル地。 
\\	タオル地製の西洋手ぬぐい。 「バス―」	▲ぬれたタオルをちょうだい。 ▲ぬれたタオルからしずくがたれている。
\\	だが	だが	[接]前に述べた事柄と反対・対立の関係の内容を述べるのに用いる語。 そうではあるが。 けれど。 だけど。 「失敗した。 ―有意義な経験だった」	▲だが、徐々に生活水準が高まるようになるにつれて、ますます大勢の人が我が家に浴室を持つようになった。 ▲子供には多くのものが必要だが、まず第一に愛が要る。
\\	互い	たがい	相対する関係にある二者。 双方、または、そのひとつひとつ。 「お互い」の形でも用いる。 「―の意思を尊重する」「―が譲り合う」	▲彼らはおたがい握手をした。 ▲人は互いの家を詳しく調べるのが大好きだ。
\\	宝・財・貨	たから	
\\	世の中に数少なく、特に貴重なもの。 宝物。 財宝。 「家に伝わる―」 
\\	財産。 金銭。 「悖(さか)って来る―の悖って出るに任せ」 
\\	ほかのものと取り替えることのできない、特に大切なもの。 また、かけがえのない人。 「国の―ともいうべき人材」「子―」→御宝(おたから)	▲子に過ぎたる宝なし。 ▲賢明で助けになってくれる友人ほど貴重な価値をもつ宝はほとんどありません。
\\	宅	たく	
\\	住居。 住まい。 「先生のお―」 
\\	自分の住居。 我が家。 「明日―へおいでください」 
\\	妻が、他人に対して自分の夫をいう語。 「―がそのように申しました」→御宅(おたく)	▲ワシントン市の友人宅に泊まります。 ▲貴殿宅の水道水は硬水過ぎます。軟水を使用しましょう。
\\	だけど	だけど	[接]「だけれど」のくだけた言い方。	▲結婚式のことだけどさあ、君の一世一代の晴れの舞台なんだから、地味婚などといわないで、パーッと派手にしたらどうなのさ。 ▲残念だけどコーヒーがなくなりました。
\\	胼胝・胝	たこ	繰り返し圧迫を受けた皮膚の部分が角質化し厚くなったもの。 骨の出っぱったところにできやすい。 ペンだこや座りだこ、肘(ひじ)だこなど。 べんち。 「耳に―ができるほど聞かされた」	▲長時間歩くと踵にたこができる。
\\	確かめる・慥かめる	たしかめる	[動マ下一][文]たしか・む[マ下二]調べたり人に聞いたりして、あいまいな物事をはっきりさせる。 確認する。 「真相を―・める」「点呼をして人数を―・める」	▲私は手紙ですべて適切なことを述べているか確かめるため。 ▲私は時計で時刻を確かめた。
\\	多少	たしょう	㊀[名] 
\\	数量の多いことと少ないこと。 多いか少ないかの程度。 「―にかかわらず、ご注文に応じます」 
\\	《「少」は助字》多いこと。 たくさん。 「人の家に―の男子を生ぜるは此れを以て家の栄えとす」 ㊁[副]数量のあまり多くないさま。 程度のあまり大きくないさま。 いくらか。 少し。 「―難点がある」「―遅れるかもしれない」 [用法]多少・若干――「その意見に対しては多少(若干)疑問がある」「原案に多少(若干)手を加えた」など、数量・程度が少なく、はっきり示せない場合には、相通じて用いられる。 
\\	「多少のばらつきがあってもかまわない」「朝晩は多少冷えるでしょう」「病状も多少落ち着いた」のように、大目に見たり全体からすればほんの少しという気持ちで言う場合は、「多少」のほうがふさわしい。 「若干」は文語的で語感が硬い。 「募集人数は若干名」「若干の問題を話し合う」などでは、「若干」だけが用いられる。	▲彼の声に多少怒りの響きがあった。 ▲彼は状況を理解するのに多少の時間がかかった。
\\	助ける・扶ける・援ける・佐ける	たすける	[動カ下一][文]たす・く[カ下二] 
\\	力を貸して、危険な状態から逃れさせる。 救助する。 「おぼれている子を―・ける」「命を―・ける」 
\\	経済的に困っている人などに金品を与えて苦しみ・負担を軽くする。 救済する。 「被災者を―・ける」 
\\	(「佐ける・輔ける・佑ける」とも書く)不十分なところを補い、物事がうまく運ぶように手助けする。 助力する。 補佐する。 「仕事を―・ける」「家業を―・ける」 
\\	ある働きがより好ましい状態になるようにする。 促進させる。 促す。 「成長を―・けるホルモン」「消化を―・ける」 
\\	倒れたり傾きそうになるのを支える。 「子どもに―・けられて駅の階段をのぼる」 [類語]
\\	救う・救い出す・救助する・救出する・救護する/
\\	救援する・救済する・救恤(きゆうじゆつ)する・援護する・援助する・扶助する/
\\	助(す)ける・手伝う・手助けする・助力する・幇助(ほうじよ)する・助勢する・加勢する・助太刀(すけだち)する・力添えする・協力する・援助する・応援する・支援する・後押しする・守(も)り立てる・バックアップする・フォローする・力を貸す・手を貸す・肩を貸す・促進する・助長する	▲誰かが助けてくれと叫んでいる。 ▲誰が私を助けてくれるの。
\\	唯・只・但	ただ	《「直(ただ)」と同語源》 ㊀[副] 
\\	そのことだけをするさま。 それよりほかにないと限定するさま。 ひたすら。 もっぱら。 「―時間ばかりかかる」「―無事だけを祈る」 
\\	数量・程度などがごく少ないさま。 わずかに。 たった。 「正解は―の三人だった」「―一度しか休まない」 
\\	(「ただ」+動詞の連用形+「に」+動詞の形で)そのことだけが行われるさま。 ひたすら。 「―泣きに泣く」 ㊁[接]前述の事柄に対して、条件をつけたりその一部を保留したりするときに用いる。 ただし。 「出かけていい。 ―、昼までには帰るように」 [類語] 
\\	ひたすら・専ら・単(たん)に・啻(ただ)に・偏(ひとえ)に・一(いつ)に/
\\	たった・わずか・僅僅(きんきん)・たかだか/ ❷但し・尤(もつと)も・とは言え・だが	
\\	戦い・闘い	たたかい	
\\	戦争。 戦闘。 「ゲリラとの―」 
\\	競争。 試合。 勝負。 「ライバルとの―」 
\\	抗争。 闘争。 「貧困との―」「労使の―」	▲そのインディアンの群れは、ほんのちょっとでも怒らすと戦いを挑もうとした。 ▲その戦いは生死を賭けた戦いのようであった。
\\	戦う・闘う	たたかう	[動ワ五(ハ四)]《動詞「たた(叩)く」の未然形に反復継続の助動詞「ふ」の付いたものからとも、「叩き合ふ」の音変化とも》 
\\	武力を用いて互いに争う。 戦争する。 「反乱軍と―・う」 
\\	互いに技量などを競い、勝負を争う。 競争する。 試合する。 「強豪と―・う」「優勝をかけて―・う」 
\\	思想や利害の対立する者どうしが自分の利益や要求の獲得のために争う。 「労使が―・う」「賃上げのために―・う」 
\\	苦痛や障害を乗りきろうとする。 打ち勝とうと努力する。 「困難と―・う」「病気と―・う」 
\\	互いにたたき合う。 「うつ浪に満ちくる潮の―・ふを楯が崎とはいふにそありける」 [可能]たたかえる [類語]
\\	争う・渡り合う・切り結ぶ・交戦する・合戦(かつせん)する・会戦する・衝突する・激突する・戦闘する・一戦を交える・砲火を交える・兵刃(へいじん)を交える・干戈(かんか)を交える・奮戦する・奮闘する/
\\	勝負する・対戦する・対する/
\\	争う・闘争する・対決する・抗争する/
\\	立ち向かう・抗する・あらがう・抵抗する・格闘する	▲若者たちは祖国を守るために戦った。 ▲手もとにあるどのような武器を用いてもエイズと闘う必要がある。
\\	叩く・敲く	たたく	[動カ五(四)] 
\\	㋐手や道具を用いて打つ。 また、続けて、あるいは何度も打つ。 「ハエを―・く」「肩を―・く」 ㋑打って音を出す。 「手を―・いて呼ぶ」「太鼓を―・く」 ㋒強く打つ。 なぐる。 ぶつ。 「棒で―・く」「尻を―・く」 ㋓さかんに当たる。 雨・風が打ちつける。 「窓を―・く雨」 
\\	魚肉を包丁で打つようにして細かく切ったり柔らかくしたりする。 「アジを―・く」 
\\	㋐攻撃を加えて相手を負かす。 やっつける。 「敵の精鋭を―・く」「出はなを―・く」 ㋑厳しく仕込む。 鍛える。 「新弟子のうちに―・いておく」 ㋒相手の言論・文章などを徹底的に批判する。 強く非難する。 「新聞に―・かれる」 
\\	相手の考えを聞いたり、ようすを探ったりする。 打診する。 「先方の意向を―・く」 
\\	値段をまけさせる。 値切る。 買いたたく。 「二束三文に―・いて買う」 
\\	すっかり使ってしまう。 はたく。 「財布の底を―・く」 
\\	(多く「…口をたたく」の形で)さかんに、またいろいろに言う。 「むだ口を―・く」「陰口を―・く」 
\\	(「門をたたく」などの形で)教えを請うためにたずねる。 「師の門を―・く」 
\\	《扇子などで演台をたたくところから》講談を演じる。 「一席―・く」 
\\	将棋で、歩(ふ)を打ち捨てる。 
\\	《鳴き声が戸をたたく音に似ているところから》クイナが鳴く。 「早苗とるころ、水鶏(くひな)の―・くなど」→打つ[用法] [可能]たたける [下接句]頤(おとがい)を叩く・口を叩く・尻(しり)を叩く・底を叩く・太鼓を叩く・出端(ではな)を叩く・門を叩く	▲しかし今日では、計算機は学校の試験では自由に使うことが出来るし、数学の試験の時に聞こえる音といえば、子供たちが計算機を叩く音しかしない、という学校も多い。 ▲その演奏者はドラムを強くたたいた。
\\	直ちに	ただちに	[副] 
\\	間に何も置かないで接しているさま。 直接。 じかに。 「窓は―通りに面している」「その方法が―成功につながるとは限らない」 
\\	時間を置かずに行動を起こすさま。 すぐ。 「通報を受ければ―出動する」	▲君は直ちにクラブを脱退したほうがよい。 ▲君は直ちに医者に行く必要がある。
\\	立ち上がる	たちあがる	[動ラ五(四)] 
\\	座ったりかがんだりしている姿勢から身を起こして立つ。 「いすから―・る」 
\\	よくない状態に陥ったものが再び勢いを取り戻す。 「地震の痛手から―・る」 
\\	行動を起こす。 「反対運動に―・る」 
\\	上の方に立つ。 立ちのぼる。 「砂ぼこりが―・る」 
\\	相撲で、仕切りから身を起こし、勝負を始める。 「制限時間の前に―・る」 
\\	機械が動き始める。 また、コンピューターのプログラムが起動する。 「このパソコンは―・るのが速い」	▲彼は突然椅子から立ちあがった。 ▲彼は突然立ち上がり、その部屋から歩き去りました。
\\	立場	たちば	
\\	人の立つ場所。 立っている所。 
\\	その人の置かれている地位や境遇。 また、面目。 「苦しい―に追い込まれる」「負けたら―がない」 
\\	その状況から生じる考え方。 観点。 立脚点。 「医者の―からの発言」「賛成の―をとる」「第三者の―」	▲もし私が君の立場なら、その計画に反対するだろう。 ▲もし私が君の立場だったら、そうはしないだろう。
\\	経つ	たつ	[動タ五(四)]《「立つ」と同語源》 
\\	時が過ぎる。 「日が―・つ」「いつまで―・っても帰ってこない」 
\\	ろうそくや油などが燃え尽きる。 「線香は、まだ半分も―・って居ない」	▲時のたつのってはやいものですね。 ▲時が経てば全ては美しい思い出に・・・。
\\	達する	たっする	[動サ変][文]たっ・す[サ変] 
\\	㋐ある場所・目的地に行きつく。 至る。 「登山隊が山頂に―・する」 ㋑物事の内容が伝わり届く。 「噂が教師の耳にも―・する」 ㋒ある数値になる。 一定の数値に届く。 「被害は二億円にも―・した」「募金が目標額に―・する」 ㋓ある状態・程度になる。 「世界的水準に―・する」 
\\	学問・技芸に深く通じる。 熟達する。 「一芸に―・した人」 
\\	物事をなしとげる。 はたす。 達成する。 「望みを―・する」 
\\	告げ知らせる。 伝える。 わからせる。 「命令を―・する」	▲努力したおかげで彼は目的を達した。 ▲討論は結論に達した。
\\	たった	たった	[副]《「ただ(唯)」の音変化》 
\\	数量が少ないことを強調するさま。 わずか。 ほんの。 「出発まで―一時間しかない」「―の一〇〇円だ」 
\\	ごく近い過去を強調するさま。 「―こないだのように思っていたけれど」 
\\	ひたすら。 いちずに。 「―説かせませ」	▲この学校の教師のたった16%が女性です。 ▲この机はたった30000円だった。
\\	だって	だって	㊀[接助]《接続助詞「たって」が、ガ・ナ・バ・マ行の五段活用動詞の連用形に付く場合の形》「たって」に同じ。 「ここなら泳い―かまわない」 ㊁[係助]《断定の助動詞「だ」に係助詞「とて」の付いた「だとて」の音変化という》名詞・副詞、一部の助詞に付く。 「でも」に似るが、語調がより強い。 
\\	ある事柄を例示し、それが他と同類、または、同様であるという意を表す。 …もやはり。 …でも。 「鯨―人間の仲間だ」「ここから―見える」 
\\	いくつかの事柄を並べて例示し、すべてが同類であるという意を表す。 「水銀―鉛―公害のもとだ」「野球―テニス―うまい」 
\\	疑問・不定を表す語、または、数量・程度を表す語に付いて、例外なくそうである意を表す。 …でも。 …も。 「だれ―知っている」「一度―姿を見せない」 ㊂[終助]《係助詞「だって」の文末用法から》引用句に付く。 相手の言葉に対して、非難・驚きの気持ちを込めて強調する意を表す。 「欲しいくせに、いらない―さ」「なぜ休んだか―。 病気だよ」	
\\	たっぷり	たっぷり	㊀[形動][文][ナリ] 
\\	満ちあふれるほど十分にあるさま。 「―な水で麺(めん)をゆでる」 
\\	名詞の下に付いて、ある要素が普通よりも多めであることを表す。 「皮肉―」「自信―」 ㊁[副] 
\\	に同じ。 「コップに―(と)牛乳を注ぐ」 
\\	分量などに余裕のあるさま。 「―(と)した仕立ての服」「時間は―(と)ある」 
\\	少なめに見積もっても、それだけの数量は十分にあるさま。 「駅まで―二〇分はかかる」	▲私はたっぷり10マイルは歩いた。 ▲私たちは食糧をたっぷり持っている。
\\	例え・譬え・喩え	たとえ	
\\	たとえること。 また、たとえられた語句や事柄。 比喩。 「―に引く」 
\\	同じような例。 「世間の―にもれない」	▲たとえ火の中水の中あなたにならば、どこまでもついていきます。 ▲たとえ外はどんなに寒くとも、室内は気持ちよく暖められている。
\\	仮令・縦令・縦え	たとえ	[副] 
\\	(あとに逆接条件を表す「ても」「でも」「とも」などを伴って)仮にある事柄を想定しながら、結果はそれに影響されないことを表す。 もし…だとしても。 仮に。 よしんば。 「―失敗しようとも悔いはない」 
\\	(多く、あとに「ば」を伴って)もし。 仮に。 「―汝此の国を治(し)らば、必ずそこなひやぶる所多けむとおもふ」 「―親友でも許せない」「―むだになってもやってみよう」	
\\	谷・渓・谿	たに	
\\	地表にできた狭くて細長いくぼ地。 浸食作用でできた河谷・氷食谷が一般的であるが、断層運動や褶曲(しゆうきよく)運動でできた断層谷や褶曲谷もある。 「千尋の―」 
\\	細長くくぼんでいる部分。 また、波の山と山との間の低い所。 「気圧の―」「景気の―」 
\\	二つの屋根の斜面が交わる、くぼんだ所。	▲一歩誤るとせんじんの谷だ。 ▲丘の向こう側に美しい谷がある。
\\	他人	たにん	
\\	自分以外の人。 ほかの人。 「―まかせの態度」「―のことはわからない」 
\\	血のつながりのない人。 親族でない人。 「赤の―」「遠くの親類より近くの―」 
\\	その事柄に関係のない人。 第三者。 「内部の問題に―を巻き込む」「―の出る幕ではない」 [類語]
\\	人(ひと)・他(た)・他者・余人(よじん)・人様(ひとさま)/
\\	よそ様・よその人/
\\	第三者・局外者	▲各人が個性的であればあるほど、それだけ他人の英知に寄与する。 ▲許可なしで他人の私的な手紙を読むべきではない。
\\	束・把	たば	㊀[名]物をひとまとめにしてくくったもの。 「薪を―にする」 ㊁〔接尾〕助数詞。 たばねたものを数えるのに用いる。 「花二―」「ねぎ一―」	▲新聞をたばにするのを手伝ってくれ。 ▲彼が忘れていった書類の束を抱えて彼の後を追いかけた。
\\	旅	たび	
\\	住んでいる所を離れて、よその土地を訪ねること。 旅行。 「かわいい子には―をさせよ」「日々―にして―を栖(すみか)とす」 
\\	自宅を離れて臨時に他所にいること。 「あるやうありて、しばし、―なる所にあるに」 [下接語]御(お)旅・帰らぬ旅・神の旅・死出の旅・長旅・俄(にわか)旅・一人旅・船旅・股(また)旅・宿無し旅	▲私は日本を旅していた。 ▲私は飛行機の旅が好きではない。
\\	度	たび	㊀[名] 
\\	何かの行われる、または行われた、その時。 おり。 「この―はお世話になります」 
\\	回数。 度数。 「―を重ねるうちに上達する」 
\\	その時はいつも。 その時ごとに。 「会う―に大きくなっている」 ㊁〔接尾〕助数詞。 度数を表すのに用いる。 「三(み)―当選する」「いく―か繰り返す」	▲私が訪ねるたびあなたは留守だ。 ▲最近の凶悪事件をニュースで見るたび、バーチャルな世界と生きる世界の区別がなくなった若者が急増していることに気付かされる。
\\	度度	たびたび	同じことが何度も繰り返して行われるさま。 毎度。 毎回。 しばしば。 「―注意を受ける」「―の訪問」	▲私はたびたび彼を訪ねます。 ▲私たちの先生は、たびたび名簿にある彼の名前を見落とした。
\\	玉・球・珠	たま	㊀[名] 
\\	球体・楕円体、またはそれに類した形のもの。 ㋐球形をなすもの。 「―の汗」「露の―」「目の―」 ㋑丸くまとめられたひとかたまり。 「毛糸の―」「うどんの―」 ㋒レンズ。 「眼鏡の―をぬぐう」「長い―で撮る」 ㋓球技などに用いるボール。 まり。 また、投球などの種類。 「遅い―」「―を打つ」「―をとる」 ㋔玉突きの球。 転じてビリヤードや、そのゲームをいう。 「友人と―を突く」 ㋕電球。 「切れた―を取り替える」 ㋖そろばんで、はじく丸い粒。 そろばんだま。 「帳簿を開いて―を置く」 ㋗(「弾」「弾丸」とも書く)銃砲の弾丸(だんがん)。 「―が飛びかう」「―を込める」 ㋘鶏卵。 玉子。 「掻(か)き―」 
\\	㋐丸い形の美しい石の総称。 宝石や真珠など。 「―を磨く」「―で飾る」 ㋑きわめて大切に思う貴重なもの。 「掌中の―」 ㋒張りがあって美しく、清らかなもの。 「―の肌」 
\\	人を丸め込むために策略の手段として使う品物・現金。 「ゴルフ会員権を贈賄の―に使う」 
\\	美しい女性。 また、転じて芸者・遊女。 「上―」 
\\	あざけりの気持ちで、人をその程度の人物であるときめつける語。 やつ。 「あいつもたいした―だよ」 
\\	《「金玉(きんたま)」の略》睾丸(こうがん)。 
\\	紋所の名。 
\\	㋐を図案化したもの。 ㊁〔接頭〕名詞に付く。 
\\	神事や高貴な物事に付いて、それを褒めたたえる意を添える。 「―垣」「―襷(だすき)」 
\\	玉のように美しいもの、玉をちりばめたものなどの意を添える。 「―藻」「―櫛笥(くしげ)」 [下接句]傷無き玉・傷に玉・衣(ころも)の裏の珠・掌中の珠・掌(たなごころ)の玉・手の内の珠・驪竜(りりよう)頷下(がんか)の珠・連城の璧(たま)	▲地球は球の形をしている。 ▲球はわずかにカーブした。
\\	堪らない	たまら
\\	ない	〔動詞「堪る」に打ち消しの助動詞「ない」の付いたもの。 「たまらぬ」の形でも用いられる〕 
\\	持ちこたえられない。 だめになってしまう。 「いくら丈夫な身体でも無理がつづいては―
\\	ない」 
\\	ある感情・感覚をおさえきれない。 がまんできない。 「現在の生活が―
\\	なくいやになる」「寒くて―
\\	ない」「男性には―
\\	ない魅力を感じさせるらしい」 
\\	…されることに耐えられない。 とても困る。 「毎朝五時に起こされたのでは―
\\	ない」 
\\	程度がはなはだしい。 じっとしていられないほどである。 
\\	なくかわいい」
\\	なく好きだ」	▲このにおいはたまらなく嫌だ。 ▲このうれしい知らせを家族に知らせたくてたまらない。
\\	黙る	だまる	[動ラ五(四)] 
\\	ものを言うことをやめる。 無言になる。 また、泣くのをやめる。 「―・って人の話を聞く」「急に―・ってしまった」「泣く子も―・る」 
\\	自分の意見・主張などを言わない。 また、積極的に働きかけない。 「―・って引きさがる」「―・って休む」「―・って見ているだけだ」	▲日本にきたら黙っていることと同意することは同じ物だと考えてはいけません。 ▲先生はメアリーに黙ってなさいといった。
\\	試し・験し	ためし	ためすこと。 こころみ。 多く「ためしに」の形で副詞的にも用いる。 「ものは―だ」「―に使ってみる」	▲あの少年は試しに重いソファーを動かした。 ▲このワインがお口にあうかどうか試しに飲んでみてください。
\\	試す	ためす	[動サ五(四)]物事の良否・真偽や能力の程度などを実際に調べ確かめる。 こころみる。 「機械が動くかどうか―・してみる」「体力の限界を―・す」→試(こころ)みる[用法] [可能]ためせる	▲その事件によって彼の勇敢さが試された。 ▲まずぼくたちは君の計画を試してみなければならない。
\\	頼り・便り	たより	
\\	(頼り)何かをするためのよりどころとして、たよっているもの。 頼み。 「地図を―に家を探す」「兄を―にする」 
\\	(便り)何かについての情報。 手紙。 知らせ。 「―が届く」「風の―に聞く」 
\\	縁故。 てづる。 「―を求めて上京する」 
\\	都合のよいこと。 便利なこと。 「凄涼たる夜色…落ち行くには―よしと」 
\\	あることをするきっかけ、手がかり。 「彼の幽玄なる仏道をも窺い見るべき―となる」 
\\	つくりぐあい。 配置。 「簀子(すのこ)、透垣(すいかい)の―をかしく」	▲彼は毎週欠かさず母親に便りをする。 ▲彼は頼りになる男だ。
\\	頼る・便る	たよる	[動ラ五(四)]《「手(た 
\\	寄る」の意》 
\\	たのみとする。 つてを求めて近づく。 「友人を―・って上京する」 
\\	助けとして用いる。 依存する。 「つえに―・って歩く」「自然の恵みに―・る」 [可能]たよれる	▲彼女は3年間、学費を叔父に頼った。 ▲彼女の親に頼りたくない。
\\	段	だん	㊀[名] 
\\	上方へ高くのぼるように重なっている台状のもの。 また、その一つ一つ。 段々。 「石の―を上る」「―を踏み外す」 
\\	上下に区切ったものや順に重なったものの一つ一つ。 「寝台車の上の―」 
\\	㋐長く続く文章のひとくぎり。 段落。 「文を三つの―に分ける」 ㋑浄瑠璃など、語り物のひとくぎり。 「『義経千本桜』の鮨屋の―」 ㋒掛け算の九九(くく)で被乗数を同じくするもの。 「二の―を唱える」 ㋓五十音図で、行(ぎよう)に対し、「あ」「い」「う」などの列。 「た行う―」 
\\	武道や囲碁・将棋などで、技量によって与えられる等級。 ふつう、初段から一〇段まである。 「―を取る」 
\\	ある事柄をそれとさす語。 「無礼の―お許しください」 
\\	物事の一局面。 そういう場合。 「いよいよという―になって逃げだす」 
\\	否定や疑問の語を伴って、それどころではないという気持ちを表す語。 そういう程度。 それほどの程度。 「痛かったのなんのという―じゃない」 ㊁〔接尾〕 
\\	助数詞。 階段状、または層をなしたものを数える。 「階段を二―ずつ駆け上がる」「三―組みのページ」 
\\	武道や囲碁・将棋などの技量の程度を表す。 「柔道三―の腕前」	▲彼は階段を一度に三段ずつかけあがった。 ▲私は一度に2段駆け上がった。
\\	単位	たんい	
\\	ものの量をはかるための基準として定められた量。 ヤード・ポンド・尺・貫・円・ドル・メートル・グラム・アールなど。 「―符号」 
\\	物事を数値で表す際に、計算の基になるもの。 「億―の金が動く」「―面積当たりの収穫」 
\\	一定の組織を構成している要素。 「家族は社会構成の―をなす」「グループ―で行動する」 
\\	高校・大学で、進級・卒業の資格を認定するために用いられる学課履修計算の基準。 一般に学習時間により決定される。 「―が足りず留年する」 
\\	禅宗で、僧堂での座位。 座席の上に名札がはってある。	▲家族は社会の基本的単位である。 ▲生理学は3単位です。
\\	単語	たんご	文法上、意味・職能をもった最小の言語単位。 例えば「鳥が鳴く」という文は、「鳥」「が」「鳴く」の三つの単語からなる。 日本語では自立語・付属語に大別される。	▲先生は生徒たちにそれらの単語をノートに控えておくよう指示した。 ▲先生は私たちにその単語を繰り返して言わせた。
\\	男子	だんし	
\\	男の子。 男児。 ↔女子。 
\\	男性。 おとこ。 また、一人前のおとこ。 「―の本懐」↔女子。	▲男子は妹を、見下しがちだ。 ▲第1次世界大戦で英国の成年男子の大半が応召した。
\\	単純	たんじゅん	[名・形動] 
\\	そのものだけで、まじりけがないこと。 他の要素などが混入していないこと。 また、そのさま。 「言葉どおりの―な意味」 
\\	機能・構造・形式などがこみいっていないこと。 また、そのさま。 「―な計算ミス」「―な機械」↔複雑。 
\\	考え方やとらえ方が素直であること。 とらえ方などが一面的で浅いこと。 また、そのさま。 「その見方は―すぎる」「彼は意外に―なところがある」「―明快な論理」↔複雑。 
\\	条件・制限などがないこと。 また、そのさま。 「―に権利を承継する」 [派生]たんじゅんさ[名]	▲ちょっと待ってよ、そんなに単純に、終わりよければ総て良し、ってわけにはいかないよ。 ▲ラッセルは、ずば抜けた哲学者であったが、日常的なことは、ごく単純なことさえ全然出来なかった。
\\	誕生	たんじょう	[名]スル 
\\	人が生まれること。 出生。 生誕。 「長男が―する」 
\\	生まれて一回目の誕生日。 「―を過ぎて歩きはじめた」 
\\	物事や状態が新しくできること。 「文化センターの―を祝う」「新政権が―する」 [類語]
\\	出生(しゆつしよう・しゆつせい)・生誕(せいたん)・降誕/
\\	発生・生成・成立・発足・出現・登場・旗上げ	▲彼らは最初の子供の誕生を喜んだ。 ▲彼は最初の子の誕生の知らせを聞いて、飛び上がって喜んだ。
\\	ダンス	ダンス	舞踊。 特に社交ダンスをいうこともある。 「フォーク―」「ジャズ―」	▲ジュディさんはダンスがとても上手です。 ▲その2人の娘はダンスパーティーに同じ洋服を着ていった。
\\	団体	だんたい	
\\	ある目的のために、人々が集まって一つのまとまりとなったもの。 「―で見学する」「―旅行」「―割引」 
\\	二人以上の者が共同の目的を達成するために結合した集団。 法人・政党など。 「政治―」「宗教―」 [類語]
\\	集団・一団・一行(いつこう)・グループ・サークル・パーティー・チーム/
\\	組織・結社・法人・組合・連盟・協会・ユニオン・ソサエティー・アソシエーション	▲その団体は全部で50名の学生から成っている。 ▲その団体は政府への支持をとりやめた。
\\	担当	たんとう	[名]スル一定の事柄を受け持つこと。 「営業部門を―する」「―者」	▲エンジニアリングの仕事は日本の会社が担当することになっています。 ▲きみはこの部屋の赤ちゃんたちを担当してもらいます。
\\	単なる	たんなる	[連体]それだけで、ほかに何も含まないさま。 ただの。 「―うわさにすぎない」	▲医者に診てもらうほうがよい。単なる風邪ではないかもしれない。 ▲医者に見てもらったほうがいい。単なる風邪ではないかもしれない。
\\	単に	たんに	[副](あとに「だけ」「のみ」などの語を伴って)その事柄だけに限られるさま。 ただ。 ただに。 「―個人のみの問題にとどまらない」	▲彼は単に読むふりをしていたとわかった。 ▲彼は単に冗談としてそれを言った。
\\	地	ち	
\\	天に対して、地上。 人間が生活し、動植物が生息・繁茂する大地。 「天と―」 
\\	地面。 地上。 「枯れ葉が―に落ちる」 
\\	海に対して、陸地。 おか。 「―の果て」 
\\	場所。 ところ。 「安住の―」「思い出の―」 
\\	領土。 「隣国と―を接する」 
\\	荷物・掛け物・本などの下の部分。 「天―無用」↔天。 
\\	「天・地・人」と三段階に分けたときの、二番目。 三つ一組のものの中位。	▲彼女はあきらめてその地に住むことにした。 ▲これらの国々はヨーロッパ文明発祥の地である。
\\	地位	ちい	
\\	社会やある組織の中で、人や物の占めている位置。 身分や立場など。 「高い―に就く」「生産物中、重要な―を占める」 
\\	存在している場所。 位置。 「ありゃ、いい―にあるが、誰の家(うち)なんですか」	▲新しい地位でご活躍することを確信しています。 ▲新しい地位を捨てることで妻をがっかりさせたくもなかった。
\\	地域	ちいき	区画された土地の区域。 一定の範囲の土地。 「―の代表」「防火―」 [類語]区域・地区・地方・方面・一円・一帯・地帯・界隈(かいわい)・土地・地(ち)・境域・境(さかい)・領域・エリア・ゾーン	▲樹木から水の供給がなければ、降雨量はやがて減少し、その地域は乾燥し続ける。 ▲暑く乾燥した地域はますます暑く乾燥するであろう。
\\	チーズ	チーズ	牛などの乳を乳酸菌や酵素を加えて凝固させ、微生物の作用によって発酵・熟成させた食品。 ナチュラルチーズとプロセスチーズに大別される。 乾酪。	▲このチーズはピリッとした味がする。 ▲チーズはミルクから作られる。
\\	チーム	チーム	ある目的のために協力して行動するグループ。 組。 スポーツや共同作業についていわれる。 「―を組む」「野球―」	▲私たちは彼女をチームの主将に選んだ。 ▲私と一緒に、スミス博士を我がチームに歓迎してください。
\\	知恵・智慧	ちえ	
\\	物事の道理を判断し処理していく心の働き。 物事の筋道を立て、計画し、正しく処理していく能力。 「―を借りる」「生活の―」 
\\	(智慧)仏語。 相対世界に向かう働きの智と、悟りを導く精神作用の慧。 物事をありのままに把握し、真理を見極める認識力。	▲経験を積むにつれて更に知恵が身につく。 ▲諺は知恵について満ちている。
\\	地下	ちか	
\\	地面の下。 土の下。 地中。 「―二階」↔地上。 
\\	死後の世界。 冥土(めいど)。 泉下。 「―に眠る」 
\\	表面に表れない所。 社会の表面から隠れて行う社会運動や政治運動などの非合法面をさす。 「―に潜行する」	▲洗濯室は地下です。 ▲四天王も倒した、クソ長い地下迷宮もクリアした。
\\	違い	ちがい	
\\	違うこと。 異なること。 「趣味の―」「実力の―」「三つ―の姉」 
\\	誤ること。 まちがい。 「文字の―を正す」 [下接語]板違い・従兄弟(いとこ)違い・入れ違い・色違い・打ち違い・大違い・御門(おかど)違い・思い違い・思惑違い・掛け違い・門(かど)違い・考え違い・勘違い・気違い・食い違い・計算違い・桁(けた)違い・見当違い・心得違い・作(さく)違い・筋違い・擦れ違い・互い違い・種違い・段違い・手違い・畑違い・場違い・腹違い・引き違い・一足違い・人違い・打(ぶ)っ違い・間違い・見込み違い・眼鏡違い・目利き違い・目違い・行き違い・了見違い	▲気候の違いのため、同種の穀物が国の北部、東部においては収穫されていない。 ▲音楽家は音の小さな違いが分かる。
\\	違いない	ちがいない	〔連語〕 
\\	応答の言葉として、肯定の返事を表す。 そのとおりだ。 「『お互いにもう年だね』『―・い』」 
\\	(「…にちがいない」の形で)きっと…である。 …にきまっている。 「雨が降るに―・い」「犯人は男に―・い」	
\\	近頃	ちかごろ	㊀[名]このごろ。 最近。 近来。 副詞的にも用いる。 「―の若い者は」「―手に入れた品」 ㊁[副]《「近ごろになく」の意から》はなはだ。 非常に。 「―かたじけなし」 ㊂[形動ナリ] 
\\	大変結構だ。 「それは―にて候」 
\\	もってのほかだ。 「やい石見守、―なる雑言」→最近[用法]	▲近頃の学生といえば、何を考えているのか私にはわからない。 ▲近頃どんなスポーツをやっていますか。
\\	地球	ちきゅう	太陽系の三番目の惑星で、人類をはじめ各種生物が住む天体。 太陽からの平均距離は約一・五億キロで、自転周期は二三時五六分四秒、公転周期は三六五・二五六四日。 形はほぼ回転楕円体で、赤道半径六三七八キロ、極半径六三五七キロ。 地殻・マントル・核からなり、平均密度は一立方センチ当たり五・五二グラム。 年齢は約四六億年。 表面は窒素と酸素とを主成分とする大気に囲まれ、水がある。 衛星を一個もち、月とよぶ。	▲月は月に1回地球の周りを回る。 ▲月は地球から遠い。
\\	地区	ちく	
\\	ある限られた範囲内の土地。 地域。 
\\	行政上、ある目的のために特に指定された地域。 「風致―」	▲現地の人達はその地区に立ち入ることを許されなかった。 ▲ビル・ピアソンは、建設会社で15年働いた後、地区支配人という責任ある地位を与えられた。
\\	遅刻	ちこく	[名]スル決められた時刻に遅れること。 「待ち合わせに―する」	▲彼は昨日遅刻した。 ▲彼は遅刻したので、われわれは彼をおいて出発した。
\\	知事	ちじ	
\\	各都道府県を統轄し、代表する首長。 都道府県の事務およびその権限に属する国や他の公共団体の事務を管理執行する。 任期は四年。 明治憲法下では官選。 昭和二二年(一九四七)から住民による直接選挙となる。 
\\	中国の官名。 州・県の長官。 
\\	寺院の雑事や庶務に当たる役職。 →六知事	▲大統領や知事が立派に合法的に行動した場合には、米国民は彼らを再選し、彼らが属する政党に指示を送ることによって承認を表明する。 ▲州知事は黒い金に手をつけました。
\\	知識・智識	ちしき	[名]スル 
\\	知ること。 認識・理解すること。 また、ある事柄などについて、知っている内容。 「日々新しい―を得る」「―をひけらかす」「予備―」「幸福とは何かと云う事を明細に―して了っている」 
\\	考える働き。 知恵。 「―が発達する」 
\\	(多く「智識」と書く)仏語。 ㋐仏法を説いて導く指導者。 善知識。 ㋑堂塔や仏像などの建立に金品を寄進すること。 また、その人や金品。 知識物。 ㋒対象を外界に実在すると認める心の働き。 
\\	(ドイツ)
\\	哲学で、確実な根拠に基づく認識。 客観的認識。 [類語]
\\	知見・学(がく)・学識・学殖・造詣(ぞうけい)・蘊蓄(うんちく)・教養・素養・理解・認識・知(ち)	▲彼はその問題に関して、表面的な知識しか持っていない。 ▲彼はその事柄について表面的な知識しかない。
\\	ダイヤグラム	ダイヤグラム	
\\	線図。 図表。 
\\	列車など、交通機関の運行状況を一枚の図に表示したもの。 また、その運行予定。 ダイヤ。	
\\	ダイヤモンド	ダイヤモンド	《「ダイアモンド」とも》 
\\	炭素の同素体の一。 数ミリから数センチの結晶または破片の形で産出し、ふつう正八面体。 純粋なものは無色透明。 硬度は最高。 宝石のほか、工業用として研磨材・切削工具などに利用。 人工的に合成もされる。 金剛石。 ダイヤ。 
\\	野球で、本塁と三つの塁に囲まれた部分。 内野。	
\\	徒・只	ただ	《「直(ただ)」と同語源》 ㊀[名] 
\\	取り立てて値打ちや意味がないこと。 普通。 「―の人」「―のからだではない」 
\\	何事もなく、そのままであること。 無事。 「見つかったら―では済まない」 
\\	(只)代金のいらないこと。 また、報酬のないこと。 無料。 「―でくれる」「―で働く」 ㊁[形動ナリ] 
\\	ありきたり。 また、ありのまま。 「―なる絹綾など取り具し給ふ」 
\\	何事もないさま。 「朝露のおくる思ひに比ぶれば―に帰らむ宵はまされり」 
\\	は「只」の字を「ロ」と「ハ」とに分け、俗に「ろは」という。	
\\	其の内	そのうち	[副]それが実現するまでに、それほど時日を要しないことを表す語。 近いうち。 「また―お伺いします」「―雨が降るだろう」	▲そのうちお宅にお伺いしてもいいでしょうか。 ▲そのうち、小さな村落の真上を高架線が行ったり来たりするようになりました。
\\	それでも	それでも	[接]そうであっても。 「失敗の可能性は大きい。 ―やめるわけにはいかない」	▲彼には欠点があったが、それでも私は彼を愛していた。 ▲誰もが反対したが、それでもサリーとボブは結婚した。
\\	それとも	それとも	[接]あるいは。 または。 もしくは。 「コーヒーか、―紅茶か」	▲何分か経ったのか、それとも何時間か経ったのか、沼田先生が家に入ってきて、禎子の額に手を当てた。 ▲学校へは歩いていけますか、それともバスを使わなければいけませんか。
\\	偶・適・会	たまたま	[副] 
\\	時おり。 時たま。 たまに。 「春とはいえ―寒い日がある」 
\\	偶然に。 ちょうどその時。 「―駅で旧友にあった」	▲私はそのとき、たまたまロンドンにいなかった。 ▲私のポケットにはたまたま穴があいていた。
\\	父親	ちちおや	男親。 父。 「―参観日」↔母親。	▲戦争が始まったと聞いたが、父親が軍に徴兵されるまで長いこと、彼はそれを十分に実感していなかった。 ▲その少年は父親に口答えをした。
\\	知能・智能	ちのう	
\\	物事を理解したり判断したりする力。 「―の高い動物」 
\\	心理学で、環境に適応し、問題解決をめざして思考を行うなどの知的機能。	▲人間はみな動物よりも知能が高い。 ▲言い換えれば、教育は知能に自然の法則を教え込むことである。
\\	地平線	ちへいせん	
\\	視野の開けた広野で、大地と天との境にほぼ水平に見える線。 
\\	観測者を通る鉛直線に垂直な平面が天球と交わる大円。	▲太陽はゆっくりと地平線の下に沈んでいった。 ▲太陽は今し方地平線に沈んだ。
\\	地方	ちほう	
\\	ある国の中のある地域。 「この―独特の風習」「関東―」 
\\	首都などの大都市に対してそれ以外の土地。 「―へ転勤になる」「―の出身」↔中央。 
\\	旧軍隊で、軍以外の一般社会をさす語。	▲彼らの結婚式はその地方の習慣に従って行われた。 ▲彼、ボスの逆鱗に触れ、地方支店に追いやられたよ。
\\	茶	ちゃ	㊀[名] 
\\	ツバキ科の常緑低木。 暖地に自生。 葉は長楕円形で厚くつやがある。 秋、白い五弁花を開く。 原産地は中国の四川・雲南・貴州など霧の多い山岳地方。 若葉を緑茶などとするため広く栽培され、延暦二四年(八〇五)に最澄が中国から種子を持ち帰り栽培したのが始まりという。 日本では五月ごろから八、九月ごろまで三、四回摘む。 ちゃのき。 めざましぐさ。 《季 花=冬》「―の花に今夕空の青さかな/万太郎」 
\\	の若葉・若芽を摘み、飲料用に製したもの。 葉茶。 また、その飲料。 製法により玉露・煎茶・番茶など種類が多い。 一般に日本茶(緑茶)をさすが、発酵させた紅茶・中国茶もある。 「―をいれる」 
\\	抹茶をたてること。 点茶。 また、その作法。 茶の湯。 茶道。 「お―を習う」「―をたしなむ」 
\\	「茶色」の略。 「―のセーター」 ㊁[名・形動]ちゃかすこと。 からかうこと。 また、その言葉や、そのさま。 「―を言う」「そんな地口のやうな―な事ではなく」	▲茶が一杯ほしい。 ▲わたしはコーヒーよりも茶をこのむ。
\\	チャンス	チャンス	物事をするのによい機会。 好機。 「絶好の―を生かす」「今一度―を与える」	▲こんなよいチャンスは二度と来ないよ。 ▲そういう人々がチャンスをつかむのだ。
\\	ちゃんと	ちゃんと	[副]スル 
\\	少しも乱れがなく、よく整っているさま。 「部屋の中を―かたづける」「いつも―した身なりをしている」 
\\	確実で間違いのないさま。 「言われたことは―やる」「―した職業につく」 
\\	結果が十分であるさま。 「朝食は―食べてくる」 
\\	すばやく動作をするさま。 さっと。 「鉦(かね)と撞木(しゆもく)のやうに―だまんな」	▲自分が約束したことはちゃんと実行するように最善を尽くすべきだ。 ▲人はちゃんと自分の役割を果たしていたもの。
\\	注・註	ちゅう	本文の意味を詳しく説明したり補足したりするために、本文の間に書き込んだり、別の箇所に記したりする文句。 その位置によって頭注・割り注・脚注などという。 「―を付す」「―を加える」「訳者―」	▲注・かなり昔に描いてるのでクオリティは低いです。 ▲(注)画像がリンク切れになっている場合があります。
\\	中央	ちゅうおう	
\\	ある物や場所などのまんなかの位置。 「町の―にある公園」 
\\	ある組織や機関の中で、最も重要な機能をになっているところ。 中枢。 「―に意見を具申する」「党の―」 
\\	中央官庁の置かれている土地。 首都。 「行政機関が―に集中する」↔地方。	▲主任整備員はテニスコートの敷地のちょうど中央に小さな家を持っている。 ▲図書館は市の中央にある。
\\	中学	ちゅうがく	「中学校」の略。 
\\	旧学制下の男子の中等教育機関。小学校六年卒業以上の学力ある男子を収容し、修業年限は五年で、小学校と、高等学校または専門学校の中間にあたる学校。現在の高等学校と同じく高等普通教育を授けることを目的としていた。 
\\	新学制下で、小学校六年の課程を終えた者に、さらに中等普通教育を授ける三年制の義務教育の学校。新制中学校。	▲入学して1か月・・・まだ一人も友達がいないってのはやばすぎる。このままじゃ中学の二の舞だ!! ▲「そんなに性格がいいかな〜?」「ありえない。今回もだけど、中学ん時は女子のパンツを売りさばいて儲けてたし。」。
\\	中古	ちゅうこ	
\\	使って、やや古くなっていること。 また、その品物。 ちゅうぶる。 セコハン。 「―のカメラ」 
\\	主として日本文学史の時代区分で、平安時代のこと。 「―の物語文学」 
\\	その時代からある程度隔たった昔。 なかむかし。 中世。 「上古、―、当世」 [アクセント] 
\\	はチューコ、 
\\	はチューコ。	▲あのカーデイラーはこの中古のトヨタが調子がいいなどと、まんまと一杯くわせやがった。 ▲なに、中古だよ。
\\	中止	ちゅうし	[名]スル中途でやめること。 また、計画を取りやめにすること。 「雨で試合が―になる」「取引を―する」「発売―」 [類語]休止・停止・中断・中絶・ストップ・取り止(や)め・沙汰止(さたや)み・お流れ・立ち消え	▲計画を中止するよりほかなかった。 ▲交渉は中止になった。
\\	駐車	ちゅうしゃ	[名]スル自動車などをとめておくこと。 道路交通法では、車両等が継続的に停止すること、または、運転者が、車両等を離れてすぐには運転できない状態にあること。 「路上に―する」「―禁止」「違法―」	▲ここに駐車してもよろしいですか。 ▲ここに駐車できますよ。
\\	昼食	ちゅうしょく	昼の食事。 ひるめし。 ちゅうじき。 「―をとる」「―会」	▲私は体重を減らすために昼食を抜きはじめた。 ▲私は正午に昼食を食べます。
\\	中心	ちゅうしん	
\\	まんなか。 中央。 「町の―に公民館がある」「地域の―」 
\\	物事の集中する場所。 また、最も重要な位置にある物や人。 また、その位置。 「―となって組織をとりしきる」「政治経済の―をなす」 
\\	心のなか。 心中。 胸中。 「我―に満足を与えんも定かならず」 
\\	重心のこと。 「うまく―をとって歩く」 
\\	㋐円周上・球面上のすべての点から等距離にある点。 ㋑図形が点対称であるときの、その点。 [類語]
\\	真ん中・真ん真ん中・中央・正中(せいちゆう)・中点・心(しん)・センター/
\\	主(しゆ)・軸(じく)・要(かなめ)・柱(はしら)・中軸・枢軸・主軸・主体・主力・基幹・根幹・中枢・中核・核・コア・核心・焦点	▲彼は最初の一発で的の中心を撃ち抜いた。 ▲彼らは円の中心に棒を立てた。
\\	注目	ちゅうもく	[名]スル 
\\	注意して見つめること。 「目の前の舞台を―する」 
\\	関心をもって見守ること。 「―に値する意見」「―される作品」 
\\	旧軍隊などで、姿勢を正し相手に目をそそいで敬意を表すこと。 また、それを命じる語。	▲その広告はたいへん注目を集めていた。 ▲その少年は注目されたくて髪を染めた。
\\	注文・註文	ちゅうもん	[名]スル 
\\	種類・寸法・数量・価格などを示して、その物品の製造や配達・購入などを依頼すること。 また、その依頼。 「酒の―をとる」「新刊書を―する」 
\\	人に依頼したり、自分が希望したりするときにつける条件。 「原作者の―にこたえた演出」 
\\	「注進状」に同じ。 
\\	文書。 かきつけ。 「各々聞き書きの―に子細(しさい)を載せられたり」 [類語]
\\	発注・用命・予約・申し込み・オーダー/
\\	要望・希望・要求・リクエスト	▲私はそれらの本をドイツに注文した。 ▲私はその本を電話で注文した。
\\	長期	ちょうき	長い期間。 長期間。 「―に及ぶ出張」↔短期。	▲戦後の日本はいくつかの力強い長期繁栄を経験し、その中には神武景気や岩戸景気がある。 ▲全員が戦後、連合軍によって投獄され、その後戦犯として死刑か長期の刑期を宣告された。
\\	調査	ちょうさ	[名]スル物事の実態・動向などを明確にするために調べること。 「都市の言語生活を―する」「国勢―」「市場―」「信用―」	▲資料不足のため調査は中止された。 ▲事故の原因を調査中である。
\\	調子	ちょうし	
\\	音の高低のぐあい。 また、音の速さのぐあい。 リズム。 拍子。 「カラオケの―が合わない」「足で―をとる」 
\\	言葉の表現のぐあい。 音声の強弱や、文章などの言い回し。 口調。 語調。 「意気込んだ―で話す」「激しい―で非難する」 
\\	動作や進行の勢い。 「―が出る」「―を落とす」 
\\	活動するものの状態・ぐあい。 「からだの―をくずす」「エンジンの―を見る」 
\\	音楽で、主音の高さによって決まる音階の種類。 雅楽の壱越(いちこつ)調・盤渉(ばんしき)調など。 
\\	弦楽器の調弦法。 箏(そう)の平調子、三味線の本調子など。 
\\	雅楽で、舞楽の一種の前奏曲。 各楽器が順次演奏に加わり、同一旋律を追いかけて奏する。 →具合(ぐあい)[用法] [類語]
\\	音調・音律・調性・音階・音程・音高・トーン・拍子・拍(はく)・律動・乗り・リズム・テンポ/
\\	具合・塩梅(あんばい)・加減・状態・体調・コンディション	▲この車は調子がいい。 ▲この調子ではいえは買えそうにない。
\\	頂上	ちょうじょう	
\\	山などのいちばん高いところ。 いただき。 「富士山の―に立つ」 
\\	最高の状態に達していること。 絶頂にあること。 「景気が―に達する」 
\\	最高の地位、また、その人。 「財界の―会談」	▲頂上が雪で覆われたその山がみえますか。 ▲頂上に着くと皆で「ヤッホー」と叫んだ。
\\	頂戴	ちょうだい	[名]スル 
\\	もらうこと、また、もらって飲食することをへりくだっていう語。 「結構な品を―いたしました」「お叱りを―する」「もう十分―しました」 
\\	(多く、女性・子供の用いる語) ㋐物を与えてくれ、また、売ってくれという命令の意を、親しみの気持ちをこめて促すようにいう語。 ください。 「それを―」「牛肉五〇〇グラム―」 ㋑「…てちょうだい」の形で補助動詞の命令形のように用いて、相手に何かをしてもらうのを促す気持ちを、親しみをこめていう語。 「その新聞を取って―」 
\\	顔の上にささげ持つこと。 「黄衣の神人神宝を―して、次々に順(したが)ふ」 [アクセント] 
\\	はチョーダイ、 
\\	はチョーダイ。	▲いすで床をこすらないでちょうだい。 ▲連絡ちょうだい!
\\	貯金	ちょきん	[名]スル金銭をためること。 また、その金。 「毎月決まった額を―する」「郵便―」	▲私の貯金はとても少ないので、あまり長くもたないだろう。 ▲私は50万円の貯金が銀行にある。
\\	直接	ちょくせつ	[名・形動]スル間に他のものをはさまないで接すること。 間に何も置かないで関係したり、働きかけたりすること。 また、そのさま。 副詞的にも用いる。 「事故の―の原因」「相手国政府への―な(の)交渉」「会って―話す」「是等は唯書生の一身に―して然るのみ」↔間接。 [類語]直(じき)・無媒介・ダイレクト・直直(じきじき)・直(じか)に	▲私はその情報を直接手に入れた。 ▲私はそのことについて隣の人から直接聞きました。
\\	著者	ちょしゃ	書物を書き著した人。 著作者。	▲その詩は無名の著者が詠んだ。 ▲その書物の出版は著者の誕生日に合わせられた。
\\	対	つい	㊀[名] 
\\	㋐二つそろって一組みとなること。 また、そのもの。 「―になっている置き物」 ㋑素材や模様・形などを同じに作って、そろえること。 また、そのもの。 「―の着物」 
\\	「対句(ついく)」に同じ。 ㊁〔接尾〕助数詞。 
\\	二つで一組みになっているものを数えるのに用いる。 「一―の夫婦茶碗(めおとぢやわん)」 
\\	衣服・調度など、ひとそろいのものを数えるのに用いる。 「三幅(さんぷく)―」「竜虎梅竹唐絵一―」	▲この国の王は人ではなく、はるか天空に居られるという三対の翼を持つ神獣なんだ。
\\	終に・遂に・竟に	ついに	[副] 
\\	長い時間ののちに、最終的にある結果に達するさま。 とうとう。 しまいに。 「―優勝を果たした」「―完成した」「疲れ果てて―倒れた」 
\\	(多く、打消しの語を伴って用いる)ある状態が最後まで続くさま。 とうとう。 「―現れなかった」「作品は―日の目を見なかった」 
\\	「終(つい)ぞ」に同じ。 「其の後は―ない存外の御無沙汰をいたしました」 [用法]ついに・とうとう――「苦心の末、ついに(とうとう)完成の日を迎えた」「海外旅行の夢がついに(とうとう)実現した」など、結果が現れることを表す意では、相通じて用いられる。 
\\	「とうとう」が長い時間を要してある結果が生じるという意味合いを持つのに対して、「ついに」には長い時間の後、最終的な時点で新しい何かが実現した、またはしなかったという意味合いがある。 また、口頭語としては「とうとう」が多く用いられ、「ついに」は文語的である。 
\\	類似の語に「結局」がある。 「結局」には、いろいな経過があったが、という意味合いがある。 「ずいぶん頑張ったが、結局成功しなかった」	▲ついに彼は目標を達した。 ▲ついに彼も強い反対に折れた。
\\	通過	つうか	[名]スル 
\\	ある場所・時点・状態などを通り過ぎること。 「台風が―する」「―駅」 
\\	試験・検査などを無事に通ること。 「審査を―する」 
\\	議会などで議案が可決されること。 「法案が国会を―する」	▲与党は強引に税制法案を通過させた。 ▲その法案はついにつうかした。
\\	通学	つうがく	[名]スル学生・生徒・児童が学校に通うこと。 「電車で―する」	▲彼は以前自転車通学だったが、今はバス通学です。 ▲彼はバス通学をしていますか。
\\	通行	つうこう	[名]スル 
\\	ある所を通って行くこと。 また、ある方向へ向かって行くこと。 「車が道の左側を―する」「―止め」「―人」 
\\	広く世間一般に行われること。 「現代―している暦」	▲ここは大型車の通行は禁止されている。 ▲この先通行禁止。
\\	通じる	つうじる	[動ザ上一]「つうずる」(サ変)の上一段化。 「電話が―・じる」	▲私は英語で話が通じた。 ▲私は話を通じさせることができなかった。
\\	通信	つうしん	[名]スル 
\\	手紙などで自分の意思やようすなどを他人に伝えること。 また、そのたより。 しらせ。 「―文」 
\\	郵便・電信・電話などによって情報を伝達すること。 「無線で漁船と―する」	▲彼は、無電で通信を送った。 ▲通信業界はいとも簡単に手中におさめることができます。
\\	捕まる・掴まる・捉まる	つかまる	[動ラ五(四)] 
\\	取り押さえられて、逃げることができなくなる。 とらえられる。 つかまえられる。 「どろぼうが―・る」「スピード違反で―・る」「先発投手が相手の打線に―・る」 
\\	目的のものを探し当てたり、呼びとめたりすることができる。 見つかる。 「夜討ち朝駆けでも担当者が―・らない」「タクシーが―・る」 
\\	呼ばれてその場に無理にひきとめられる。 「記者団に―・る」 
\\	(掴まる・捉まる)からだを支えるために手でしっかりと何かにとりすがる。 「つり革に―・る」「手すりに―・る」	▲手すりにつかまっていなさい。 ▲私は仕方なく、つり革につかまった。
\\	掴む・攫む	つかむ	[動マ五(四)] 
\\	手でしっかりと握り持つ。 強くとらえて離すまいとする。 「腕を―・む」「まわしを―・む」 
\\	自分のものとする。 手に入れる。 「思いがけない大金を―・む」「幸運を―・む」 
\\	人の気持ちなどを自分に引きつけて離さないようにする。 「大衆の心を―・む」「固定客を―・む」 
\\	物事の要点などを確実にとらえる。 「事件解決の糸口を―・む」「こつを―・む」 
\\	遊女を呼んで遊興する。 揚げる。 「天神、鹿恋(かこひ)七人―・みて」 
\\	遊女を身請けする。 「早駕籠の大臣と申せし人の―・んで」 [可能]つかめる [用法]つかむ・にぎる――「私はそこにあった棒をつかむと、ぎゅっと握った」では「つかむ」と「握る」を置き換えることはできない。 「つかむ」はその物を手で捕らえる動作が主であり、「握る」は手の中に入れたまま締め付けるようにして離さずにいる持続的な動きである。 
\\	「情報をつかむ」「大金をつかむ」は、それを手に入れること。 「情報を握っている」「大金を握る」は、それを持ち続ける状態を言う。 
\\	「拳(こぶし)を握る」とはいうが「つかむ」とは普通はいわない。 もしいうならば、自分の拳を他方の手で取るか、または他者の拳を取ることである。	▲私は彼のそでをつかんだ。 ▲私は彼の腕をつかんだ。
\\	疲れ	つかれ	
\\	疲れること。 くたびれること。 疲労。 「昨日の―がどっと出る」 
\\	⇒疲労	▲疲れが彼女の顔に見えた。 ▲疲れやら空腹やらで、彼は死んだように倒れた。
\\	付き合い	つきあい	
\\	人と交際すること。 「彼とは長い―だ」 
\\	義理や社交上の必要から人と交わること。 「―が悪い人」「―酒」→交際[用法]	▲今後とも、おつきあいのほど、よろしくお願いします。 ▲今後とも、貴社と緊密なおつきあいをいただけますよう希望しております。
\\	継ぎ継ぎ・次次	つぎつぎ	
\\	第2以下に列する程度の身分や地位。 また、その人。 源氏物語若菜下「その―をなむことみこたちには御処分どもありける」 
\\	子孫。 源氏物語橋姫「いよいよかの御―になり果てぬる世にて」	▲彼らは部屋を次々と出て行った。 ▲彼らは次々に立って出ていった。
\\	就く	つく	㊀[動カ五(四)]《「付く」と同語源》 
\\	㋐(「即く」とも書く)ある地位に身を置く。 特に、即位する。 「王座に―・く」 ㋑ある役職に身を置く。 また、就職する。 「管理職に―・く」「販売の仕事に―・く」 
\\	みずからある動作を始める。 「帰路に―・く」「眠りに―・く」 
\\	選んで、それに従う。 「易(やす)きに―・く」「師に―・く」 
\\	(「…につき」「…につきて」「…について」の形で用いる) ㋐ある物事に関して。 …にちなんで。 「会社設立に―・いて会合を開く」「この件に―・き御意見を」 ㋑(「…につき」の形で)…であるから。 …のために。 「喪中に―・き年賀を御遠慮します」 [可能]つける ㊁「つ(就)ける」の文語形。 [下接句]華を去り実に就く・官途に就く・位に即(つ)く・緒(しよ・ちよ)に就く・小を捨てて大に就く・小異を捨てて大同に就く・途に就く・床(とこ)に就く・鳥屋(とや)に就く・縛(ばく)に就く・水の低きに就くが如(ごと)し	▲彼は世襲によって王位についた。 ▲私の友人は1年間に3つの仕事についた。彼は何ごとも長くはやり続けられない。
\\	付く・附く・着く	つく	㊀[動カ五(四)] 
\\	あるものと他のものが離れない状態になる。 ㋐あるものが表面に密着する。 くっつく。 付着する。 接着する。 「ほこりが―・く」「飯粒が―・く」「この糊(のり)はよく―・く」 ㋑主となるものに、さらに添え加わる。 「おまけが―・く」「条件が―・く」 ㋒消えないであとが残る。 印される。 「歯形が―・く」「傷が―・く」「しみが―・く」 ㋓ある性質・能力などがそなわる。 「知恵が―・く」「身に―・いたしぐさ」 ㋔ある物事・状態・作用などが新たに生じたり、増し加わったりする。 「電話が―・く」「味が―・く」「はずみが―・く」 
\\	(着く)あるものが他のものや他の所まで達する。 ㋐到着する。 「手紙が昨日―・いた」「目的地に―・く」「定刻に―・く」 ㋑一部分がある所に届いて触れる。 「船底が海底に―・く」「枝が伸びて塀に―・く」 ㋒ある場所を占める。 位置する。 「守備位置に―・く」 
\\	あるもののそばに寄ってそい従う。 ㋐一緒になって行く。 あとにつづき従う。 「先頭ランナーに―・く」「彼のやり方には―・いていけない」 ㋑その立場に心を寄せて行動を共にする。 味方になる。 「有利な側に―・く」「何かというと母親に―・く」 ㋒そばを離れずにいる。 いつもそばにいて世話をする。 「護衛が―・く」「看護婦が―・く」 ㋓頼りになる、または助けてくれる者が背後にいる。 「スポンサーが―・く」 ㋔寄り集まる。 たかる。 「果物に虫が―・く」 ㋕あるものに沿う。 「急いで堤(どて)に―・いて左の方へ道を折れた」 
\\	ある働きが活動を始める。 ㋐働きが盛んになる。 「食欲が―・く」「精力が―・く」 ㋑(「点く」とも書く)燃えはじめる。 また、あかりがともる。 「火が―・く」「電灯が―・く」 ㋒病気にかかる。 感染する。 「ばい菌が―・く」 ㋓(「憑く」とも書く)乗り移る。 とりつく。 「ものに―・かれたような目」「狐(きつね)が―・く」 ㋔五感にとらえられる。 意識・知覚がはたらく。 「よく気の―・く人」「目に―・く広告」「高慢さが鼻に―・く」 ㋕植物が根をおろす。 根づく。 「挿し木が―・く」 
\\	ある定まった状態がつくられる。 ㋐解決する。 まとまる。 落着する。 「話が―・く」「勝負が―・く」 ㋑ある名前や値段になる。 「愛称が―・く」「高値が―・く」 ㋒あることの前提・結果のもとでは、その値段になる。 「一つ百円に―・く」「結局は高く―・いた」 ㋓意志が固まる。 「決心が―・く」 
\\	偶然などがうまく味方して、都合よく事が運ぶ。 運が向く。 「今日は―・いている」 
\\	ある物事に付随して、別の物事が起こる。 「風に―・きてさと匂ふがなつかしく」 
\\	病気にかかる。 感染する。 「月ごろ御やまひも―・かせ給て」 
\\	(動詞の連用形について)動作・状態の激しい意を表す。 「しがみ―・く」「食い―・く」 [可能]つける ㊁[動カ下二]「つ(付)ける」の文語形。 [下接句]悪銭身につかず・足が地に着かない・足が付く・足元に火が付く・油紙に火が付いたよう・板に付く・生まれもつかない・海の物とも山の物ともつかない・襟に付く・襟元に付く・御釈迦(おしやか)様でも気がつくまい・尾鰭(おひれ)が付く・及びもつかない・方が付く・格好がつく・気が付く・金箔(きんぱく)が付く・愚にも付かぬ・けちが付く・けりが付く・時代が付く・示しがつかない・尻(しり)に付く・尻に火が付く・土がつく・手が付く・手に付かない・箔(はく)が付く・鼻につく・話がつく・火が付く・火の付いたよう・引っ込みがつかない・人垢(ひとあか)は身につかぬ・人心地が付く・人目に付く・頬返(ほおがえ)しが付かない・眉(まゆ)に火が付く・身に付く・耳に付く・虫が付く・目に付く・目処(めど)が付く・目鼻が付く・焼け木杭(ぼつくい)に火が付く・理屈と膏薬(こうやく)はどこへでも付く	▲川越の山車は、いわゆる鉾山車と呼ばれる形で、車輪が3つ、もしくは4つ付いています。 ▲かのじょはビールを飲むくせがついた。
\\	注ぐ	つぐ	[動ガ五(四)]《「継ぐ」と同語源》容器に物を満たす。 特に、液体を容器にそそぎ入れる。 「御飯を―・ぐ」「お茶を―・ぐ」 [可能]つげる	▲彼女はお客たちにお茶をついだ。 ▲「コーヒーのお替わりいる?」「うん、ぼくのは少なめ、弟のには多めについでね」。
\\	付ける・附ける・着ける	つける	[動カ下一][文]つ・く[カ下二] 
\\	あるものが他のものから離れない状態にする。 ㋐表面に密着させる。 くっつける。 付着させる。 「おしろいを―・ける」「マニキュアを―・ける」「扉に金具を―・ける」 ㋑主となるものに他のものを加える。 何かに添えたり、付属させたりする。 「利息を―・ける」「振り仮名を―・ける」「部屋にクーラーを―・ける」 ㋒あとを残す。 あとに残るように書きとめる。 しるす。 印する。 「しみを―・ける」「日記を―・ける」 ㋓ある性質・能力などがそなわるようにする。 「悪知恵を―・ける」「技術を身に―・ける」 ㋔ある物事・状態・作用などを新たに生じさせたり、増し加えたりする。 「雪をかいて道を―・ける」「丸みを―・ける」 
\\	(着ける) ㋐からだにまとわせたり、帯びたりする。 衣服などを着る。 着用する。 「はかまを―・ける」「犬に首輪を―・ける」 ㋑乗り物をある場所に寄せ止める。 「船を岸に―・ける」「車寄せに―・ける」 ㋒からだのある部分を何かに届かせる。 近寄せて触れさせる。 「頭を地に―・ける」「頬と頬を―・ける」 ㋓ある場所に位置させる。 命じて一定の所にいさせる。 「走者をスタートラインに―・ける」 
\\	㋐あとに続き従わせる。 「好位置に―・けている」 ㋑ある立場に心を寄せさせて従わせる。 「味方に―・ける」 ㋒人をそばに置く。 そばにいさせて世話をさせる。 「ボディーガードを―・ける」「付き添いを―・ける」 ㋓あとを追う。 尾行する。 「少し離れて―・けて行く」 
\\	ある働きを発動させる。 活動を開始させる。 ㋐働きを盛んにする。 「食欲を―・ける」「元気を―・ける」 ㋑(点ける)燃えるようにする。 また、あかりをともす。 スイッチなどを入れて器具を作動させる。 「枯れ草に火を―・ける」「電灯を―・ける」 ㋒五感でとらえる。 感覚器官を働かせる。 注意を向ける。 「気を―・ける」「目を―・ける」 
\\	ある定まった状態をつくり出す。 ㋐解決させる。 落着させる。 まとめる。 「話を―・ける」「けりを―・ける」 ㋑ある名前や値段にする。 「題名を―・ける」「時価で値を―・ける」 ㋒意志を固める。 「決心を―・ける」「死ぬ覚悟を―・ける」 
\\	連歌・俳諧で、前の句にうまくつながらせて、あとに句をつづける。 
\\	器に盛ったり燗(かん)をしたりして、飲食物を用意する。 「御飯を―・ける」「一本―・けてくれ」 
\\	(「…につけ」「…につけて」の形で)…に関連して。 …に伴って。 …の場合も。 「何事に―・け相談してください」「よきに―・け悪しきに―・け」「暑さ寒さに―・けて故郷を思う」 
\\	(動詞の連用形に付いて) ㋐それをすることが習慣になっている、しなれている意を表す。 「履(は)き―・けている靴」「行き―・けない場所」 ㋑相手に対してなされる行為の勢いが激しい意を表す。 「たたき―・ける」「しかり―・ける」 ㋒その行為が、ある対象に向けられる意を表す。 「言い―・ける」 ㋓到着する、または来させる意を表す。 「駆け―・ける」「呼び―・ける」 ㋔しっかりととどめる意を表す。 「心に刻み―・ける」 ㋕鼻や目で感じとって、何かを探り当てる意を表す。 「嗅(か)ぎ―・ける」 [下接句]跡をつける・糸目を付けない・色を付ける・因縁をつける・後ろを付ける・尾に尾を付ける・尾鰭(おひれ)を付ける・折り紙を付ける・方を付ける・金に糸目は付けぬ・眼(がん)を付ける・気を付ける・けちを付ける・けりを付ける・黒白(こくびやく)を付ける・腰に付ける・先鞭(せんべん)をつける・田にも畦(あぜ)にも腥物(なまぐさもの)つけて・知恵を付ける・注文を付ける・提灯(ちようちん)を付ける・唾(つば)を付ける・手が付けられない・手を付ける・取って付けたよう・難を付ける・難癖を付ける・猫の首に鈴をつける・熨斗(のし)をつける・箔(はく)を付ける・箸(はし)をつける・火を付ける・眉(まゆ)に唾をつける・身に付ける・見切りを付ける・味噌(みそ)をつける・道を付ける・目を付ける・目処(めど)を付ける・目鼻を付ける・目星を付ける・目安を付ける・勿体(もつたい)を付ける・文句を付ける・楊枝(ようじ)に目鼻を付けたよう・渡りを付ける	▲おまえの上着は私の勘定につけておきなさい。 ▲その女優は美しい衣装をつけていた。
\\	土・地	つち	
\\	岩石が分解して粗い粉末になったもの。 土壌。 「花壇の―を入れ替える」 
\\	地球の陸地の表面。 地面。 大地。 「故国の―を踏む」 
\\	「天」に対し、地上のこと。 「空から―へひと息にポーンと降り立つ雨の脚」 
\\	鳥の子紙の一。 紙の原料となる植物繊維に泥土をまぜて製した下等な和紙。 
\\	(「犯土」「槌」「椎」とも書く)陰陽道(おんようどう)で、土公神(どくじん)のいる方角を犯して工事などをすることを忌むこと。 また、その期間。 暦の庚午から甲申までの一五日間をいう。 つちび。 
\\	人の容貌(ようぼう)の醜いことをたとえていう語。 「御前なる人は、まことに―などの心地ぞするを」 
\\	地下(じげ)のこと。 「―の帯刀(たちはき)の、歳二十ばかり、長(たけ)は一寸ばかりなり」 [下接語]赤土・荒(あら)土・合わせ土・上(うわ)土・置き土・鹿沼(かぬま)土・壁土・黒土・肥え土・白(しら)土・底土・叩(たた)き土・作り土・床(とこ)土・苦(にが)土・粘(ねば)土・練り土・埴(はに)土・粘(へな)土・惚(ほ)け土・真(ま)土・盛り土・焼き土・焼け土・痩(や)せ土・用心土	▲この土は最近の雨のために湿っている。 ▲だって私達は結局冷たい土に倒れるのだから。
\\	続き	つづき	
\\	あるものの延長。 つながっていて、次にあるもの。 「ドラマの―」「この話には―がある」「―の部屋」 
\\	続いていくぐあい。 つながり方。 「文章の―が悪い」 
\\	名詞に付いて、前からの状態などが変わらないことを表す。 「お天気―」「好運―」 [下接語]家続き・縁続き・国続き・地続き・血続き・手続き・長続き・庭続き・日照り続き・一続き・道続き・峰続き・棟続き・山続き・陸続き・廊下続き	▲三日続きの雨だった。 ▲収穫不良は日照り続きのせいである。
\\	包み	つつみ	㊀[名] 
\\	紙や布などで包むこと。 また、そのもの。 「―をほどく」「小物をまとめて―にする」 
\\	物を包むのに使う物。 風呂敷の類。 「よき―、袋などに、衣(きぬ)ども包みて」 ㊁〔接尾〕助数詞。 包んであるものを数えるのに用いる。 「薬を毎食後一―ずつ飲む」 [下接語](づつみ)上包み・紙包み・香包み・茣座(ござ)包み・小包み・薦(こも)包み・根包み・袱紗(ふくさ)包み・風呂敷包み・藁(わら)包み	▲この包みは彼によってここに置かれた。 ▲これらの包みをほどくのを手伝って下さい。
\\	勤め・務め	つとめ	
\\	当然果たさなければならない事柄。 任務。 義務。 「税金を納めることは国民の―だ」 
\\	官公庁・会社などに雇われて、働くこと。 勤務。 「一日の―を終える」 
\\	仏道の修行。 また、僧侶が日課として行う勤行(ごんぎよう)。 「朝夕の―を欠かさない」 
\\	遊女などが稼業として客の相手をすること。 「あの娼妓は、あなたにゃあ―をはなれた、仕うちでげすぜ」 
\\	遊女の揚げ代。 また一般に、支払うべき金銭。 勘定。 「四十ばかりの女、…―をとりにきたり」 [類語]
\\	任(にん)・任務・義務・責任・責務・本務・使命・役目・役(やく)・役儀・分(ぶん)・本分・職分・職責・責め/
\\	勤務・仕事・労働・職務/
\\	お勤め・勤行(ごんぎよう)・看経(かんきん)・読経(どきよう)・礼拝(らいはい)	▲彼がどうして突然勤めを辞めてしまったのかさっぱり分からない。 ▲自分の国の人々の福祉に努力するのを自分の努めだと彼はみなしていた。
\\	繋ぐ	つなぐ	《名詞「綱」の動詞化》 ㊀[動ガ五(四)] 
\\	㋐ひも・綱などで物を結びとめて、そこから離れたり、逃げたりしないようにする。 「犬を―・ぐ」「鎖で―・ぐ」 ㋑相手の気持ちなどが離れていかないようにする。 「彼女の心を―・ぐために一芝居打つ」 
\\	一定の所に留め置いて外へ出さないようにする。 拘禁したり、拘束したりする。 「獄に―・がれる」 
\\	結びつけてひと続きのものにする。 「車両を―・ぐ」 
\\	離れているもの、切れているものを続け合わせて一つにする。 また、そのようにして通じるようにする。 「手を―・ぐ」「電話を―・ぐ」 
\\	なんとか長く、切れないようにたもたせる。 たえないようにする。 「望みを―・ぐ」「命を―・ぐ」「座を―・ぐ」 
\\	足跡などをたどって行方を追い求める。 「射ゆ鹿(しし)を―・ぐ川辺のにこ草の」 [可能]つなげる ㊁[動ガ下二]「つなげる」の文語形。 [類語] 
\\	㋐)繋ぎ止める・縛る・結び付ける・繋(か)ける・舫(もや)う/
\\	繋(つな)げる・繋ぎ合わせる・結び合わせる・継ぎ合わせる・継ぐ・接続する・連結する/
\\	結ぶ・連絡する・中継する	▲馬をあの木につなぎなさい。 ▲馬車を馬の前につなぐな。
\\	常に	つねに	[副] 
\\	どんな時でも。 いつも。 絶えず。 「―微笑を絶やさない」 
\\	変わることなく。 そのままに。 「世の中や―ありける」→何時(いつ)も[用法]	▲彼はつねに黒眼鏡をかけている。 ▲彼は自分の妻には常に誠実だったと言っている。
\\	角	つの	
\\	動物の頭部に突き出た、堅い骨質や角質のもの。 「牡鹿(おじか)の―」 
\\	物の表面などに突き出ているもの。 とがったもの。 「かたつむりの―」 
\\	その形相が角を生やした鬼に似るとして、女性の嫉妬心(しつとしん)や怒りなどを、角のある状態にたとえる語。 
\\	紋所の名。 
\\	の形を図案化したもの。	▲水牛は大きな角をもっている。 ▲カタツムリがすっと角を出した。
\\	翼	つばさ	
\\	鳥類の空中を飛ぶための器官。 前肢が変形したもので、先端から初列風切り羽が一〇枚ほど、次列風切り羽が六〜三〇枚並び、その上面に雨覆い羽が並ぶ。 
\\	飛行機の左右に突き出た翼(よく)。 また、飛行機。	▲もし私に空を飛ぶ翼があったら、彼女を助けに行ったのに。 ▲もし私に翼があれば、君のところへ飛んでいくのだが。
\\	詰まり	つまり	㊀[名] 
\\	物が詰まること。 また、詰まっている度合い。 「排水溝の―」「袖丈(そでたけ)の―ぐあい」 
\\	いろいろと経過して行きつく最後のところ。 事の結末。 果て。 終わり。 「とどの―」「身の―」「我分別で―を付ねば」 
\\	追いつめられること。 困窮すること。 行きづまり。 「作料を取らせねば、諸職人の―となる」 
\\	行きづまった所。 すみ。 「ここの―に追っつめてはちゃうど斬(き)る」 ㊁[副] 
\\	話の落ち着くところは。 要するに。 結局。 「今までいろいろ述べたが、―それはこういうことになる」 
\\	別の語に置きかえれば。 言い換えると。 すなわち。 「地図の上方、―北方は山岳地帯である」 [下接語](づまり)織り詰まり・金(かね)詰まり・気詰まり・金(きん)詰まり・寸詰まり・手詰まり・どん詰まり・鼻詰まり・糞(ふん)詰まり・目詰まり・行き詰まり	▲つまり、ある人が民主主義の価値を受け入れる場合には、その人は民主主義の責任も同時に受け入れなければならない。 ▲つまり、英語はもはや、イギリスの人々だけの言語ではないということです。
\\	罪	つみ	㊀[名] 
\\	道徳・法律などの社会規範に反する行為。 「―を犯す」 
\\	罰。 
\\	を犯したために受ける制裁。 「―に服する」「―に問われる」 
\\	よくない結果に対する責任。 「―を他人にかぶせる」 
\\	宗教上の教義に背く行為。 ㋐仏教で、仏法や戒律に背く行為。 罪業(ざいごう)。 ㋑キリスト教で、神の意志や愛に対する背反。 ㊁[形動]無慈悲なさま。 残酷なさま。 「―なことをする」「―な人」 [類語] 
\\	咎(とが)・過ち・罪悪・罪科・罪過・犯罪・罪障・罪業・悪徳・背徳・不徳・不仁・不義・不倫・破倫・悪(あく)・悪行(あくぎよう)・悪事・違犯/ ❷非道・没義道(もぎどう)・殺生・無慈悲・無情・罪作り・ひどい・むごい	▲罪に比例して罰するべきだ。 ▲罪の意識が彼の顔にはっきり現れている。
\\	詰める	つめる	[動マ下一][文]つ・む[マ下二] 
\\	容器などに物を入れていっぱいにする。 ぎっしり入れてすきまがないようにする。 「衣装を―・めた鞄(かばん)」「料理を重箱に―・める」 
\\	穴やすきまに物を入れてふさぐ。 「虫歯を―・める」 
\\	長さを短くする。 寸法や間隔を縮める。 「着物の丈(たけ)を―・める」「細かい字で―・めて書く」「席を―・めて座る」 
\\	節約する。 きりつめる。 「生活費を―・める」「経費を―・める」 
\\	最後の所まで行く。 「沢筋を―・める」 
\\	十分に検討し尽くして物事の決着がつくようにする。 煮つめる。 「話を―・める」「議論を―・める」 
\\	将棋などで、王将の逃げ場がないようにする。 「王手王手で敵玉を―・める」 
\\	たゆまずその事を続けてする。 かかりきりになる。 「―・めて仕事をする」「根(こん)を―・める」 
\\	(「息をつめる」の形で)呼吸を止める。 「息を―・めて成り行きを見守る」 
\\	(「指をつめる」の形で)謝罪などの意志を表すために指を切り落とす。 関西地方では、ドアなどに指をはさむことをいう。 「指を―・めてわびを入れる」 
\\	決まった場所に出向き、用事に備えて待機する。 出仕して控えている。 「首相官邸に―・める」「持ち場に―・める」 
\\	(動詞の連用形に付いて) ㋐身動きできないような状況に追いこむ。 行きづまらせる。 「問い―・める」「追い―・める」 ㋑最後・限度まで…する。 また、休みなく続けて…する。 「のぼり―・める」「通い―・める」 ㋒一面に…する。 「敷き―・める」「タイルを張り―・める」 [下接句]息を詰める・石で手を詰める・根(こん)を詰める・道理を詰める・指を詰める	▲私がスーツケースに荷物を詰めるのを手伝ってくれませんか。 ▲丈を少し詰めていただけますか。
\\	積もる	つもる	[動ラ五(四)] 
\\	物が次々に重なって高くなる。 一面に多くたまる。 「雪が―・る」「ほこりが―・る」 
\\	物事が少しずつたまって多くなる。 次々と加わってふえる。 「不平が―・る」「―・る思い」「―・る話」 
\\	時や日が重なる。 時間が経過する。 「日数が―・る」 
\\	あらかじめ計算をして見当をつける。 値段・数量などを概算する。 見積もる。 「工事費を―・ってみる」「安く―・っても一万円をくだらない品」 
\\	推測する。 おしはかる。 「人の心を―・る」 
\\	酒宴で、この酌で終わりにする。 おつもりにする。 「盃の手もとへよるの雪の酒―・る―・ると言ひながら飲む」 
\\	見抜いてだます。 見くびってばかにする。 「さりとは憎いと言はうか、―・られたと申さうか」	▲雪が深くつもっていた。 ▲雪が1
\\	5メートル積もっていた。
\\	梅雨・黴雨	つゆ	六月ころの長雨の時節。 また、その時期に降る長雨。 暦の上では入梅・出梅の日が決められているが、実際には必ずしも一定していない。 北海道を除く日本、中国の揚子江流域、朝鮮半島南部に特有の現象。 五月雨(さみだれ)。 ばいう。 《季 夏》「―ふかし猪口にうきたる泡一つ/万太郎」	▲梅雨が始まった。 ▲梅雨です。
\\	辛い	つらい	[形][文]つら・し[ク] 
\\	他人に対して冷酷である。 非情である。 むごい。 「―・いしうち」「―・く当たる」 
\\	精神的にも肉体的にも、がまんできないくらい苦しい。 苦しさで耐えがたい。 「―・い別れ」「いじめられて―・い目にあう」「練習が―・い」 
\\	対処が難しい。 困難である。 「―・い立場にいる」「その話をされると―・い」 
\\	人の気持ちを考えない。 つれない。 「からころも君が心の―・ければ袂はかくぞそぼちつつのみ」 
\\	冷たい態度が恨めしい。 しゃくにさわる。 「―・しともまた恋しとも様々に思ふ事こそひまなかりけれ」→苦しい[用法] [派生]つらがる[動ラ五]つらげ[形動]つらさ[名] [類語]
\\	苦しい・憂(う)い・耐えがたい・しんどい・苦痛である・切ない・やりきれない・遣(や)る瀬ない	▲彼女は笑顔でさよならと言ったが、心の中はとても辛かった。 ▲彼女は私だけにつらく当たる。
\\	釣り	つり	《「吊り」と同語源》 
\\	魚を釣ること。 また、その方法。 さかなつり。 「―の名人」「鮒(ふな)―」 
\\	「釣り銭(せん)」の略。代価より高額の貨幣で支払ったときに戻される差額の金銭。 「―を渡す」	▲子供のころよく父と釣りに行きました。 ▲私は休日によく何時間もつりをしたものだった。
\\	連れ	つれ	
\\	一諸に伴って行くこと。 一緒に行動すること。 また、その人。 同伴者。 「大阪まで車中の―ができる」「―があるので失礼します」「お―さま」 
\\	(ふつう「ツレ」と書く)能の役柄で、シテまたはワキに従属し、その演技を助ける者。 シテ方に属するシテヅレ(ツレ)とワキ方に属するワキヅレとがある。 
\\	東宮坊の帯刀(たちはき)の一。 脇の次に位する。 
\\	多く「その」「あの」「この」などの下に付いて、種類、程度、また、そのようなもの、などの意を表す。 「その―な事言うたらばこの宿には置くまいぞ」 
\\	(接頭語的に用いて)一緒に物事をする意を表す。 「―三味線」	▲私はあなたをジェンキンズさんに会いにお連れしようと思っています。 ▲彼とその連れはいっしょに来ないかと私を誘った。
\\	出合い・出会い	であい	
\\	(「出逢い」とも書く)であうこと。 思いがけなくあうこと。 めぐりあい。 「師との運命的な―」「一冊の本との―」 
\\	(出合い)川や沢などの合流する所。 「本流との―」 
\\	(出合い)男女が密会すること。 あいびき。 
\\	(出合)取引で、売り手と買い手の値段・数が一致して、売買が成立すること。 「―がつく」 
\\	連歌・俳諧で、一座の者が順によらず付句のできた者から付けていくこと。 
\\	知り合うこと。 交際。 つきあい。 「博奕(ばくち)の―は相対づく」	▲それは偶然の出会いだった。 ▲それは偶然の出会いであった。
\\	出合う・出会う	であう	[動ワ五(ハ四)] 
\\	(「出逢う」とも書く)人・事件などに偶然に行きあう。 「街角で旧友と―・う」「帰宅途中に事故に―・う」 
\\	(出合う)ある場所でいっしょになる。 「本流と支流が―・う地点」 
\\	出て立ち向かう。 「曲者だ、皆の者―・え」 
\\	男女が密会する。 「中二階に上がれば樽屋―・ひ、末々約束の盃事して」	▲昨日映画を見に行ったら、偶然昔の友人に出会った。 ▲昨日駅で偶然彼にであった。
\\	提案	ていあん	[名]スル議案や意見を提出すること。 また、その議案や意見。 「具体策を―する」「―者」	▲会合を終わりにしようと私は提案した。 ▲関税引き上げの提案国はお互いに反目しています。
\\	定期	ていき	
\\	あることが行われる時期が定まっていること。 また、一定の期間や期限。 「―演奏」「―点検」 
\\	「定期乗車券」の略。 通学・通勤などのために、電車・汽車などの一定区間を、ある期間内往復できる割引乗車券。定期券。定期。パス。「―入れ」 
\\	「定期預金」の略。銀行や郵便局が一定の期間を定めて預かる預金。	▲定期雑誌類は閲覧室より帯出禁止。 ▲汽車の中で眠っている間に、財布と定期を盗まれてしまった。
\\	抵抗	ていこう	[名]スル 
\\	外部から加わる力に対して、はむかうこと。 さからうこと。 「権力に―する」「大手資本の進出に地元の商店会が―する」 
\\	すなおに受け入れがたい気持ち。 反発する気持ち。 「相手の態度に―を感じる」「一人で入るには―がある」 
\\	流体中を運動する物体が流れから受ける、運動方向と逆向きの力。 
\\	「電気抵抗」の略。 [用法]抵抗・反抗――「むやみに抵抗(反抗)したって仕方がない」など、手向かう意では相通じて用いられる。 
\\	「抵抗」は他からの力に張り合い、それを退けようとすることに重点がある。 「誘惑に抵抗する」「市民の激しい抵抗にあう」「病魔への抵抗にも限界があった」
\\	「反抗」は外からの力、特に権威・権力にさからおうとすることに重点がある。 「反抗」が否定的に評価されることがあるのはそのためである。 「理由なき反抗」「親に反抗して家を出る」
\\	類似の語に「反発」がある。 他の人の意見・考え方にさからう点で「抵抗」と相通じて用いられる。 「友人の反発を招く」「上司の発言に反発する」	▲その軍勢は敵の攻撃に対して勇敢に抵抗した。 ▲その人々はひどい支配者に抵抗した。
\\	提出	ていしゅつ	[名]スル書類・資料などを、ある場所、特に公(おおやけ)の場に差し出すこと。 「議案の―」「レポートを―する」「辞表を―する」	▲この案をボスに提出する前に書き直しておこう。 ▲裁判所はその弁護士に証拠の提出を求めた。
\\	程度	ていど	
\\	物事の性質や価値を相対的にみたときの、その物事の置かれる位置。 他の物事と比べた際の、高低・強弱・大小・多少・優劣などの度合い。 ほどあい。 「文化の―が高い」「傷の―は大したことない」「―の差はあれ、誰もが損をした」 
\\	許容される限度。 適当と考えられる度合い。 ほどあい。 「大きいにも―がある」「いたずらといっても―を超えている」 
\\	他の語の下に付き、それにちょうど見合った度合い、それくらいの度合いの、の意を表す。 「高校―の学力」「二時間―の遅れ」「申しわけ―の金額」	▲どの程度までそのそのうわさは本当だったか。 ▲どの程度まで彼らを信じてよいのかわからない。
\\	停留所	ていりゅうじょ	バスや路面電車が客の乗り降りのためにとまる一定の場所。 停留場。	▲バスの停留所まで走りましょう。 ▲バスの停留所はどこですか。
\\	デート	デート	[名]スル 
\\	日付。 
\\	男女が日時を定めて会うこと。 「恋人と―する」	▲彼は彼女が弟とデートしていることに腹を立てた。 ▲彼はメアリーと今日の午後デートをする。
\\	敵	てき	
\\	戦い・競争・試合の相手。 「大国を―に回して戦う」「―の意表をつく」「―をつくりやすい言動」↔味方。 
\\	害を与えるもの。 あるものにとってよくないもの。 「民衆の―」「社会の―」「ぜいたくは―だ」 
\\	比較の対象になる相手。 「―のほうがもてる」「弁舌にかけては彼の―ではない」 
\\	遊里で、客と遊女とが互いに相手をさしていう語。 相方。 おてき。 「―もをかしき奴(やつ)にて」 
\\	(「的」とも書く。 代名詞的に用いて)多少軽蔑して、第三者をさしていう語。 やつ。 やつら。 「―めもえらい痴呆(へげたれ)めぢゃ」 [用法]敵(てき)・かたき――自分にとって害をなすもの、滅ぼすべき相手の意では「敵」も「かたき」も相通じて用いられるが、普通は「敵」を使う。 「かたき」はやや古風ないい方。 
\\	「敵」は戦争・競争・試合の相手全般について使う。 「敵を負かす」「敵に屈する」「敵が多い」
\\	争いなどの相手の意で使う「かたき」は、「恋がたき」「商売がたき」「碁(ご)がたき」のように複合語として用いられることが多い。 
\\	深い恨みを抱き、滅ぼしたいと思う相手の意では「かたき」を使う。 「親のかたきを討つ」「父のかたきを取る」「目のかたきにする」など。 
\\	類似の語に「あだ」がある。 「かたき」と同じように使われ、「あだ(かたき)討ち」などという。 ただし「恩をあだで返す」は「かたき」で置き換えられない。	▲神を信じ、敢然と敵に向かった、そのクリスチャンの剣闘士は、たくさんの敵を倒した。 ▲敵船は火曜日に神戸港に寄港するでしょう。
\\	的	てき	〔接尾〕 
\\	名詞に付いて、形容動詞の語幹をつくる。 ㋐そのような性質をもったものの意を表す。 「文学―表現」「詩―発想」 ㋑それについての、その方面にかかわる、などの意を表す。 「教育―見地」「政治―発言」「科学―方法」 ㋒そのようなようすの、それらしい、などの意を表す。 「大陸―風土」「平和―解決」「徹底―追求」 
\\	人名や人を表す語(また、その一部)に付いて、親しみや軽蔑(けいべつ)の気持ちを込めて、その人を呼ぶのに用いる。 「取―(=下級の力士)」「泥―(=泥棒)」「幸―(=幸次郎)」	
\\	標準的コーディング技法/Stephan 
\\	著;クイープ訳。 ▲老人と海はとても感動的な本だ。
\\	出来事	できごと	社会や身のまわりに起こる事柄。 また、ふいに起こった事件・事故。 「一瞬の―」	▲彼の死は思いがけない出来事だった。 ▲彼の演説中に不思議なできごとが起こった。
\\	適する	てきする	[動サ変][文]てき・す[サ変] 
\\	ある対象・目的などによく合う。 「子供に―・した映画」 
\\	条件などにうまくあてはまる。 適合する。 「寒冷地に―・した作物」 
\\	それにふさわしい素質や能力がある。 「教師に―・した人物」	▲この水は飲料に適している。 ▲これらの靴は深い雪の中を歩くのに適している。
\\	適切	てきせつ	[名・形動]状況・目的などにぴったり当てはまること。 その場や物事にふさわしいこと。 また、そのさま。 「―に判断する」「―な表現」 [派生]てきせつさ[名] [用法]適切・適当――「適切な(適当な)処置をとる」「この心情を伝えるのに適切な(適当な)言葉を知らない」など、ふさわしい、あてはまるの意では、相通じて用いられる。 
\\	「適切」は過不足なく、よくあてはまる場合に用いる。 「適切な指導を行う」「適切な忠告」
\\	「適当」は「適切」より幅のあるふさわしさ、また、ほどよいことをいう。 「結婚したいが、適当な相手がいない」「適当なところで切り上げる」など。 
\\	また、「適当」には、いいかげんな、あいまいな、の意を表す用法もある。 「よくわからないから適当に選んでおいた」「適当にごまかす」
\\	類似の語に「適宜」がある。 状況にかなっているさまを表し、「適当」と同じように使う。 「適宜に取りはからう」	▲その問題に関する彼の意見はたいへん適切だ。 ▲それから、それが適切であることを確認するために再度見直した。
\\	適度	てきど	[名・形動]程度がほどよいこと。 また、そのさま。 「―な(の)運動」「―な(の)湿り気」	▲祖父は毎朝適度の運動をしているので丈夫です。 ▲適度な運動することは体によい。
\\	適用	てきよう	[名]スル法律・規則などを、事例にあてはめて用いること。 「会社更生法を―する」	▲我々の計画をこの新しい事態に適用させねばならない。 ▲その法律は人種宗教肌の色に関わらずすべての人に適用される。
\\	手品	てじな	
\\	巧みな手さばきで、人の目をくらまし、不思議なことをしてみせる芸。 奇術。 「―の種あかし」 
\\	手並み。 腕前。 「ただ君と我とがおのおの―を知らんとなり」 
\\	手つき。 手振り。 「拳(けん)の―の手もたゆく」	▲私の手品はせいぜいこんなもんです。 ▲私はこの手品のトリックに気がついた。
\\	ですから	ですから	[接]《断定の助動詞「です」+接続助詞「から」から》「だから」を丁寧にいう語。 「九時に出社いたします。 ―九時以降なら何時でも結構です」	▲彼は何の準備もしてないのではないかと、ちょっと心配しています。私は彼が資料を準備してプレゼンテーションをしたのを見たことがないものですから。 ▲彼はドライですからね。
\\	鉄	てつ	
\\	鉄族元素の一。 地球上でアルミニウムに次いで多く、赤鉄鉱・磁鉄鉱・褐鉄鉱・砂鉄などとして産出。 純鉄は銀白色で光沢があり、延性・展性に富み、強磁性をもつ。 空気中ではさびやすい。 少量の炭素その他を含む鋳鉄や鋼鉄にして利用。 植物では微量養素の一で、クロロフィルの合成などに必要。 動物ではヘモグロビン・チトクロム・ミオグロビンなどの成分として重要。 元素記号
\\	原子番号二六。 原子量五五・八五。 くろがね。 
\\	かたくて強いもののたとえ。 「―の規律」	▲鉄のカーテンがヨーロッパ大陸におりた。 ▲鉄の原子番号は26です。
\\	哲学	てつがく	
\\	の訳語。 ギリシア語の
\\	に由来し、
\\	(智)を
\\	(愛する)」という意。 西周(にしあまね)が賢哲を愛し希求する意味で「希哲学」の訳語を造語したが、のち「哲学」に改めた》 
\\	世界・人生などの根本原理を追求する学問。 古代ギリシアでは学問一般として自然を含む多くの対象を包括していたが、のち諸学が分化・独立することによって、その対象領域が限定されていった。 しかし、知識の体系としての諸学の根底をなすという性格は常に失われない。 認識論・論理学・存在論・倫理学・美学などの領域を含む。 
\\	各人の経験に基づく人生観や世界観。 また、物事を統一的に把握する理念。 「仕事に対しての―をもつ」「人生―」	▲哲学はあなたが想像するほど難しい学科ではない。 ▲哲学は難しいと見なされることが多い。
\\	手伝い	てつだい	手伝うこと。 また、その人。 「引っ越しの―」「もう一人―を頼む」	▲4年間は彼は最初の値段で芝を刈ってくれたが、その年の終わりに彼はしょっちゅう手伝いを連れていることに気がついた。 ▲彼は疲れていたが、それでも彼らの手伝いに行った。
\\	徹底	てってい	[名]スル《底までつらぬき通ることの意》 
\\	中途半端でなく一貫していること。 「―した利己主義者」 
\\	すみずみまで行き届くこと。 「会の趣旨を―させる」「命令が―しない」	▲彼は徹底した利己主義者だ。 ▲弁護士というものは、苦境を切り抜けるために、ささいな点についても徹底して調べ、同じことを反覆して調べてみることが大切である。
\\	鉄道	てつどう	レールを敷き、その上に電車・列車などを走らせ、人や貨物を運ぶ陸上交通機関。 日本では明治五年(一八七二)の新橋・横浜間の開業を最初とする。	▲当時は日本には鉄道はなかった。 ▲当時、日本には鉄道が無かった。
\\	徹夜	てつや	[名]スル夜どおし寝ないで過ごすこと。 「―で看病する」「―して論文を仕上げる」	▲彼は徹夜で勉強しようとつとめたが、だめだった。 ▲彼は徹夜を何とも思わない。
\\	出端	では	
\\	《「でば」とも》外へ出るきっかけ。 「跳ね返されて、―を失って、ごうと吼(ほ)えている」 
\\	《「でば」とも》外出する手段。 交通の便。 「わざわざこんな―の悪い処へ引込んで」 
\\	能で、神・鬼・精・霊などの後ジテやツレが登場するときに用いる囃子(はやし)。 
\\	歌舞伎で、主役などの登場。 また、その際の所作や下座音楽。 
\\	舞踊的な芸能で、登場するとき、また、退場するときの舞踊・音楽など。 出羽。 ↔入端(いりは)。 
\\	《「でば」とも》ちょうど出るおり。 出しな。 出ばな。 「あれあれ、いま月の―ぢゃ」	
\\	手間	てま	
\\	そのことをするのに費やされる時間や労力。 「―を省く」「―がかかる」 
\\	「手間賃」の略。手間仕事の報酬。手間代。 「―を払う」 
\\	手間賃を取ってする仕事。 手間仕事。 また、その仕事をする人。 「―を雇う」	▲これは主婦の手間を省く便利な器具です。 ▲これで手間がだいぶ省けるだろう。
\\	典型	てんけい	
\\	規範となる型。 基準となるもの。 「此等の人の遺せる標準(のり)―に由て観るときは」 
\\	同類ないし同種のものの中で、それらの特性を端的に示しているもの。 代表例となるもの。 「現代の若者の―」「アールヌーボーの―とされる作品」	▲彼女はあらゆる婦徳の典型である。 ▲浅学な人ほど自分が正しいと思っている典型ですね。
\\	天候	てんこう	比較的短い期間の天気の総合的状態。 また、天気のぐあい。 空模様。 天気と気候との中間の概念。 「大会は―に恵まれた」「不順な―」「悪―」→天気[用法]	▲明日になればきっと天候もよくなるだろう。 ▲比較的活動していない状態で、風にさらされていなければ、熊は寒い天候においても余分なエネルギーを消費することはない。
\\	電子	でんし	原子内で、原子核の周りに分布して負の電荷をもつ素粒子。 電子数は原子番号に一致する。 質量は陽子の約一八〇〇分の一で9.109×10-31キログラム、電荷は-1.602×10-19クーロン。 記号
\\	エレクトロン。	▲日本のカメラ、自動車、ハイファイ装置などは海外で広く使われているし、日本で開発された先端電子なしにやっていける先進国はほとんどないほどになっている。 ▲テルミン:一九二〇年、ロシアの物理学者レフ・セルゲイヴィッチ・テルミンが作った世界初の電子楽器。
\\	テント	テント	支柱および布製の覆いを組み立ててつくった簡易な家屋。 野営のときに用いる小型のもの、サーカスや芝居の掛け小屋として用いる大型のものなどいろいろある。 天幕。	▲彼女は店でテントを手に入れようとしたが、欲しいと思うテントを見つけることができなかった。 ▲彼女はテントを手に入れようとしたが、欲しいと思うテントを見つけることができなかった。
\\	伝統	でんとう	ある民族・社会・集団の中で、思想・風俗・習慣・様式・技術・しきたりなど、規範的なものとして古くから受け継がれてきた事柄。 また、それらを受け伝えること。 「歌舞伎の―を守る」「―芸能」	▲この伝統は代々受け継がれている。 ▲その昔からの伝統はすたれてしまった。
\\	天然	てんねん	[名・形動] 
\\	人為が加わっていないこと。 自然のままであること。 また、そのさま。 「―の良港」「栄養不足で―に立枯になった朴の木の様なもので」↔人工。 
\\	うまれつき。 天性。 「―の美声」	▲我々は天然の資源を保存するように勤めなければならない。 ▲天然自然が荒らされていくのは残念なことだ。
\\	問い	とい	
\\	問うこと。 質問。 「―を発する」 
\\	試験などの問題。 設問。 「左記の―に答えよ」	▲圭はその問いにギクリとさせられたが、頭を何でもないといいたそうに横に振る。 ▲彼は10分で全部の問いの答えを出した。
\\	党	とう	
\\	利害や目的などの共通性によって結びついた集団。 仲間。 「―をなす」 
\\	政治的な主張を一にする人々の団体。 政党。 「―の方針」 
\\	中世における武士の集団。 平安後期以降、血縁的武士団が発達し、のち、地域的な連合に移行した。 武蔵七党や松浦(まつら)党など。	▲彼は党を内部から改革しようとした。 ▲彼は党の代表に立てられた。
\\	塔	とう	
\\	《「卒塔婆(そとうば)」の略》仏教建築における仏塔。 仏舎利を安置し、あるいは供養・報恩などのために設ける多層の建造物。 
\\	高くそびえる建造物。 「教会の―」「テレビ―」	▲雷がその塔に落ちた。 ▲教会の塔の時計が9時を打った。
\\	答案	とうあん	出された問題に対して書いた答え。 また、それの書かれた用紙。	▲答案を提出せよ。 ▲答案用紙は、月曜日までに提出するように。
\\	同一	どういつ	[名・形動]同じであること。 一つのものであること。 差のないこと。 また、そのさま。 「成人と―に扱う」「―犯人」	▲これは私が先日なくしたのと同一の鉛筆である。 ▲同一差出人から同一受取人に宛てて郵袋という、文字通り袋に印刷物を入れて郵送します。
\\	銅貨	どうか	銅を主原料とした貨幣。 銅銭。	▲ここの数枚の銅貨がある。
\\	当時	とうじ	
\\	過去のある時点、ある時期。 その時。 そのころ。 「―を思い出す」「―はやった曲」「終戦―」 
\\	現在。 いま。 「―人心の未だ一致せざるを匡済するに」	▲その国は当時無政府状態だった。 ▲その当時、アメリカは英国から独立していなかった。
\\	動詞	どうし	国語の品詞の一。 事物の動作・作用・状態・存在などを表す語で、形容詞・形容動詞とともに用言に属する。 活用のある自立語で、文中において単独で述語になりうる。 その言い切りの形は、一般にウ段の音で終わるが、文語のラ行変格活用の語に限り、「り」とイ段の音で終わる。 口語の動詞には、五段・上一段・下一段・カ行変格・サ行変格の五種類の活用形式があるが、文語の動詞には、四段・上二段・下二段・上一段・下一段・カ行変格・サ行変格・ナ行変格・ラ行変格の九種類の活用形式がある。	▲動詞は述語動詞のことです。述語動詞は、主語や表す時によって形を変えます。 ▲仮定法過去の場合be動詞はすべてwereになるんだよ?
\\	同時	どうじ	
\\	時を同じにすること。 同じ時に行われること。 「到着はほとんど―だ」「―進行」 
\\	同じ時代。 同じ年代。	▲彼らは皆同時に話そうとした。
\\	如何しても	どうしても	
\\	どんな手段を用いてでも。 ぜひとも。 絶対に。 「―実現させなければならない」 
\\	(打ち消しの語を伴って)どんな手段を用いても…(ない)。 絶対に。 「―勝てない相手」「―解けない問題」 
\\	そうなりがちであることを表す。 とかく。 つい。 「体が悪いと―気分がめいる」	▲彼らは次の試合にどうしても勝ちたいと思っている。 ▲彼は闘牛を見たかったが、父はどうしても彼を行かせようとはしなかった。
\\	到着	とうちゃく	[名]スル目的地などに行きつくこと。 到達。 「取材先に―する」「―時刻」	▲彼女がそこに到着する時までに、ほとんど暗くなっているだろう。 ▲彼女がお化粧をしたとたんに彼が到着した。
\\	道徳	どうとく	
\\	人々が、善悪をわきまえて正しい行為をなすために、守り従わねばならない規範の総体。 外面的・物理的強制を伴う法律と異なり、自発的に正しい行為へと促す内面的原理として働く。 
\\	小・中学校で行われる指導の領域の一。 昭和三三年(一九五八)教育課程に設けられた。 
\\	《道と徳を説くところから》老子の学。 [類語]
\\	倫理・道義・徳義・人倫・人道・世道・公道・公徳・正義・規範・大義・仁義・徳・道(みち)・モラル・モラリティー	▲彼には道徳観念が欠けている。 ▲動物に道徳感はない。
\\	投票	とうひょう	[名]スル 
\\	選挙や採決のとき、各人の意思表示のため、氏名や賛否などを規定の用紙に記し、一定の場所に提出すること。 「支持政党に―する」「不在者―」 
\\	競馬・競輪などで、馬券・車券を買い求めること。 「勝馬―券」	▲委員の大多数はその議案に反対投票をした。 ▲我々はその議案に反対投票をした。
\\	同様	どうよう	[名・形動]同じであること。 ほとんど同じであること。 また、そのさま。 「前の事件と―な(の)手口」「兄弟―に育てられる」	▲かなづち同様ボブは泳げない。 ▲このことは犬と同様猫にも当てはまる。
\\	同僚	どうりょう	職場が同じである人。 また、地位・役目が同じである人。	▲彼は同僚から批判を受けやすい。 ▲彼は同僚がミスをするとああでもないこうでもないとうるさく言う癖があるみたいだ。
\\	道路	どうろ	人や車などの通行するみち。 往来。 [類語]道(みち)・通り・往来・車道・街路・舗道(ほどう)・街道・往還・通路・路上・路面・ロード・ルート	▲道路の凍結状態の結果多くの事故が発生した。 ▲道路の横断中は、どんなに注意してもし過ぎることはない。
\\	通す・徹す・透す	とおす	[動サ五(四)] 
\\	㋐一方から他方へ突き抜けさせる。 「針に糸を―・す」 ㋑まんべんなくゆきわたらせる。 「中まで十分に火を―・す」 ㋒二点間を結ぶ道筋をつくる。 「バイパスを―・す」 ㋓正しい筋目をつける。 「話の筋を―・す」 
\\	㋐ある点を過ぎて行かせる。 「車を止めて人を―・す」 ㋑(人を)屋内や室内に導き入れる。 「客間に―・す」 ㋒人を仲立ちとして、また、物を隔ててそのことをする。 「先生を―・して頼む」「レンズを―・して見る」 ㋓料理店などで、客の注文を帳場に取り次ぐ。 
\\	㋐審査して成り立たせる。 「法案を―・す」 ㋑無理やりに受け入れさせる。 「我を―・す」 
\\	㋐最後までその状態を続ける。 「独身で―・す」 ㋑全期間、また、全体にわたってする。 「夜を―・して話す」「書類に目を―・す」 
\\	(動詞の連用形について)最後まで…続ける。 「遂にやり―・した」 [可能]とおせる [下接句]一念岩をも通す・牛は願いから鼻を通す・思う念力岩をも通す・女の一念岩をも通す・我を通す・気を通す・錐(きり)嚢(ふくろ)を通す・袖(そで)を通す・手を通す・念力岩を徹(とお)す・火を通す・目を通す	▲どうぞ通してください。 ▲ボスの経費チェックは厳しいね。まさに、眼光紙背に徹す、だよ。
\\	通り過ぎる	とおりすぎる	[動ガ上一][文]とほりす・ぐ[ガ上二]ある所を通って向こうへ行く。 通りこす。 「足早に―・ぎる」「夕立が―・ぎる」	▲彼女は気づかずに私の前を通りすぎた。 ▲彼女はみすぼらしい小さいおうちを見ましたが急いで通り過ぎてしまいませんでした。
\\	都会	とかい	
\\	人が多数住み、行政府があったり、商工業や文化が発達していたりする土地。 都市。 みやこ。 
\\	「都議会」の略。(東京都議会の略)東京都の議決機関。任期四年の都議会議員で組織される。 「―議員」	▲彼は都会生活の便利な面を強調した。 ▲彼女は都会で仕事を得る為に田舎を出た。
\\	解く	とく	㊀[動カ五(四)] 
\\	結んだりしばったりしてあるものをゆるめて分け離す。 ほどく。 「帯を―・く」「包みを―・く」 
\\	縫い合わせてあるものの糸を抜き取って離す。 また、編んであるものをほどく。 「着物を―・く」「セーターを―・く」 
\\	㋐もつれたものをもとに戻す。 ほぐす。 「からまった釣り糸を―・く」 ㋑(「梳く」とも書く)もつれた髪に櫛(くし)を入れて整える。 とかす。 すく。 「乱れた髪を―・く」 
\\	着ていたもの、身に取り付けたものをはずす。 「旅装を―・く」 
\\	命令・束縛・制約などから解放する。 「鎖を―・く」「統制を―・く」「契約を―・く」 
\\	任務・職をやめさせる。 免じる。 「任を―・く」 
\\	拘束していた態勢を崩してもとの状態に戻す。 「警備を―・く」「武装を―・く」 
\\	心のわだかまりや緊張状態をほぐす。 ふさがっていた気持ちをすっきりさせる。 「怒りを―・く」「疑いを―・く」「誤解を―・く」 
\\	筋道をたどって解答を出す。 「問題を―・く」「なぞを―・く」 [可能]とける ㊁[動カ下二]「と(解)ける」の文語形。 [用法]とく・ほどく――「帯を解く(ほどく)」「もつれた糸を解く(ほどく)」「着物を解く(ほどく)」など、ばらばらにする意では相通じて用いられるが、「ほどく」の方が口語的である。 
\\	「問題を解く」「謎を解く」「誤解を解く」「禁止を解く」「武装を解く」「警戒を解く」「緊張を解く」など、結び目、縫い目をばらばらにする意以外の場合は「解く」だけを用いる。 
\\	類似の語に「ほぐす」がある。 解けた状態、ほどけた状態にする意で、「からんだ糸をほぐす」のほか、「緊張をほぐす」「肩の凝(こ)りをほぐす」などと用いる。 [下接句]印綬(いんじゆ)を解く・頤(おとがい)を解く・帯を解く・帯紐(おびひも)を解く・褐(かつ)を釈(と)く・産(さん)の紐を解く・綬(じゆ)を釈(と)く・刃(じん)を迎えて解く・謎(なぞ)を解く	▲彼はその問題をすべて簡単に解いた。 ▲外出する前に髪を櫛でときなさい。
\\	毒	どく	
\\	健康や生命を害するもの。 特に、毒薬。 「夜ふかしはからだに―だ」「―を仰ぐ」 
\\	ためにならないもの。 わざわいになるもの。 害悪。 「目の―」「青少年には―となる雑誌」 
\\	人の心を傷つけるもの。 悪意。 「―のある言い方」 
\\	「毒口(どくぐち)」の略。どくどくしく皮肉や悪口などを言うこと。また、そのことば。あくたれぐち。 「やいのっそりめと頭から―を浴びせて呉れましたに」	▲ある人の食べ物が別の人には毒。 ▲この透明な液体には毒が含まれている。
\\	得意	とくい	[名・形動] 
\\	自分の思いどおりになって満足していること。 「―の絶頂」↔失意。 
\\	誇らしげなこと。 また、そのさま。 「―な顔」「―になる」 
\\	最も手なれていて自信があり、じょうずであること。 また、そのさま。 得手(えて)。 「―な競技種目」「―中の―」 
\\	いつも商品を買ってもらったり取引したりする相手。 顧客(こかく)。 お得意。 
\\	親しい友。 「東山の辺にぞ―はある。 いでさらば文をやらう」 [派生]とくいがる[動ラ五]とくいげ[形動]とくいさ[名] [類語]
\\	鼻高高・誇らか・誇らしい・鼻が高い・肩身が広い・得得(とくとく)・揚揚・時を得顔・したり顔・自慢顔・自慢げ・自慢たらしい/
\\	得手(えて)・達者・堪能・上手(じようず)・巧者・得手物(えてもの)・特技・おはこ・十八番・お株	▲わたしは得意です。 ▲料理が得意ですか。
\\	読書	どくしょ	[名]スル《古くは「とくしょ」》本を読むこと。 「日がな一日―する」「―家」	▲読書は余暇を過ごすための楽しい方法です。 ▲読書は彼にとって大きな楽しみです。
\\	独身	どくしん	
\\	配偶者のいないこと。 ひとりもの。 ひとりみ。 
\\	ただ一人であること。 単独。 単身。 「にはかに―の遠行を企つ」	▲結婚には多くの苦悩があるが、独身には何の喜びもない。 ▲結婚して不幸になるより、独身で平穏無事に暮らした方がいい。
\\	特徴	とくちょう	他と比べて特に目立つ点。 きわだったしるし。 「―のある声」→特長[用法]	▲ここの地理的特徴は私達の県のそれと似ている。 ▲この計画の主要な特徴はまだ曖昧です。
\\	独特・独得	どくとく	[名・形動] 
\\	そのものだけが特別にもっていること。 また、そのさま。 「―な(の)雰囲気」 
\\	(独得)その人だけが会得していて、他の人には及ぶことができないこと。 また、そのさま。 「自家―の曲乗のままで」	▲その絵には独特の魅力がある。 ▲その習慣は日本独特のものだ。
\\	独立	どくりつ	[名]スル 
\\	他のものから離れて別になっていること。 「母屋から―した離れ」 
\\	他からの束縛や支配を受けないで、自分の意志で行動すること。 「―の精神」「―した一個の人間」 
\\	自分の力で生計を営むこと。 また、自分で事業を営むこと。 「親から―して一家を構える」「―して自分の店をもつ」 
\\	他からの干渉・拘束を受けずに、単独にその権限を行使できること。 「司法の―」「政府から―した機関」 
\\	一国または一団体が完全にその主権を行使できる状態になること。 「―を宣言する」「―したての若い国」「―国家」 [類語]
\\	分離・分立・別立て/
\\	自立・自律・自決・自主・自助・一本立ち・独り立ち/
\\	自立・自活・独歩・独行・特立	▲彼は両親から経済的に独立している。 ▲彼は独立して商売を始めた。
\\	何処か	どこか	〔連語〕《「か」は副助詞》(副詞的にも用いる) 
\\	はっきりと指示できない場所を示す。 「―で聞いた文句だ」「―遠くへ行きたい」 
\\	はっきりとは示せないが、そのようであるという気持ちを表す。 何となく。 どこやら。 「―変だ」	▲彼は公園のどこかにいる。 ▲彼は確かにハンサムで頭もいいかもしれないけど、どこか虫の好かないところがあるの。
\\	所が	ところが	
\\	⦅助詞⦆ 
\\	「したところ(が)」の形で、後のことが続くことを示す。 順接にも逆接にもなる。 …すると。 …たけれども。 浮世床2「直に返すと言つた―若い者が脇差を二階へあげることはなりませぬ」。 「応募した―すぐ採用された」 
\\	仮定の逆接を表す。 たとえ…しても。 「考えた―分かるはずもない」 
\\	⦅接続⦆ しかるに。 そうであるのに。	▲彼女は大丈夫だと言った。ところが実際はひどいけがをしていた。 ▲ところが実はどうすることもできないのです。
\\	所で	ところで	
\\	⦅助詞⦆ (名詞「ところ」に助詞「で」の付いたもの) 
\\	…によって。 …ので。 狂言、鹿狩「終に持た事が御ざらぬ―持ちやうを存ぜぬほどに」 
\\	(「…たところで」の形で)仮定の事態を述べ、後にそれに反する事態が続くことを述べる語。 もし…としても。 たとえ…でも。 …したからといって。 「私が意見した―、彼は耳をかすまい」 
\\	⦅接続⦆ 
\\	そうすると。 それで。 蒙求抄1「其のくじに一くじが出たぞ。 ―臣下共が今年ばかり代を御もちあらうかと云ふ心にみたぞ」 
\\	しかるに。 蒙求抄1「こなたへはまゐり候まいと云ぞ。 ―三度まで行んたぞ」 
\\	(別な話題をもち出す時に使う)時に。 それはそれとして。 「―お父さんはお元気ですか」	▲ところで、何歳ですか。 ▲ところで、何時にそこへ行くつもりですか。
\\	登山	とざん	[名]スル 
\\	山に登ること。 山登り。 「家族連れで―する」「―家」《季 夏》↔下山。 
\\	山上の寺社に参詣すること。	▲私にはたくさん趣味がある。例えば魚釣り、登山です。 ▲私はアルプス登山に行った。
\\	都市	とし	多数の人口が比較的狭い区域に集中し、その地方の政治・経済・文化の中心となっている地域。 「商業―」「学園―」	▲その都市は1664年にイギリス人に占領された。 ▲その都市はいっそうの活気を持って生き返った。
\\	年月	としつき	
\\	年と月。 ねんげつ。 
\\	長い歳月。 つきひ。 ねんげつ。 「―を経る」「―を重ねる」 
\\	年来。 としごろ。 副詞的にも用いる。 「―願いつづけていたことがかなう」	▲年月が経った。 ▲彼らの親密さは年月とともに深まった。
\\	図書	としょ	書籍。 書物。 本。 ずしょ。 「―を購入する」	▲私の図書は自由にお使い下さい。 ▲あなたはもう推薦図書を読み終えましたか。
\\	年寄り	としより	
\\	年をとった人。 高齢の人。 老人。 
\\	武家時代、政務に参与した重臣。 室町幕府の評定衆・引付衆、江戸幕府の老中、大名家の家老など。 
\\	江戸幕府の、大奥の取り締まりをつかさどった女中の重職。 
\\	江戸時代、町村の行政にあたった指導的立場の人。 
\\	大相撲の関取以上の力士で、引退して年寄名跡を襲名・継承した者。 日本相撲協会の運営や各部屋の力士養成に当たる。 →老人[用法]	▲両親は私に年寄りを敬うように言った。 ▲彼女はお年寄りに親切です。
\\	閉じる	とじる	[動ザ上一][文]と・づ[ダ上二] 
\\	㋐あけてあったもの、あいていたものがしまる。 両端を合わせた状態になる。 ふさがる。 「水門が―・じる」「ドアが―・じる」「貝のふたが―・じる」 ㋑続いていた物事が終わりになる。 「会議が―・じる」 
\\	㋐あいていたもの・部分をふさいでしまう。 「本を―・じる」「まぶたを―・じる」「心を―・じる」 ㋑今まで続いたものを終わりにする。 「店を―・じる」「会を―・じる」 
\\	とじこめる。 こもらせる。 「葎(むぐら)の門に、思ひのほかにらうたげならむ人の―・ぢられたらむこそ」 [用法]とじる・しめる――「門を閉じる(閉める)」「店を閉じる(閉める)」「ふたを閉じる(閉める)」など、開いていたものの空間を埋める意では相通じて用いられる。 
\\	「戸が閉じる」「貝のふたが閉じる」のように「〜が閉じる」の形では「閉(しめ)る」は使えない。 「〜が閉まる」の形になる。 
\\	「閉じる」と「閉める」の使い分けは慣用による。 目・口や本・傘などは「閉じる」、引き出し・門などは「閉める」を使うことが多い。 
\\	「店を閉じる」は廃業する意で多く使うが、「店を閉める」は、一日の営業を終る、または廃業するのどちらにも使う。	▲私たちは本を閉じていた。 ▲傘を閉じました。
\\	途端	とたん	あることが行われた、その瞬間。 そのすぐあと。 多く副詞的に用い、「に」を伴うこともある。 「よそ見をした―転んだ」「飲むと―に人が変わる」	▲なるべく考えまいとは思っていたのだが、自覚をしてしまうと途端に侘しいような気持ちにもなってくる。 ▲由香は新任の英語の先生に紹介されたとたん、恋におちた。
\\	土地	とち	
\\	陸地。 大地。 地。 「人跡未踏の―」 
\\	植物・作物などが育つ土壌。 土。 「肥沃な―」「―を耕す」 
\\	耕地や宅地など、さまざまに利用する地面。 地所。 「―を買う」「―を開発する」 
\\	その地域。 地方。 「―の習慣」「―の人」「風光明媚(めいび)な―」 
\\	領土。 「―割譲」 [類語]
\\	土(つち)・大地・地(ち)・土壌・壌土/
\\	地所・地面・用地・敷地・宅地・料地・農地・耕地・田地(でんち)・田畑(たはた・でんばた)・沃土(よくど)・沃地・沃野/
\\	地(ち)・地方・当地・御当地・当所・現地・地元・地(じ)	▲必要なら、政府は、不動産業者に土地の価格を落とすよう強制するだろう。 ▲不動産市場が低迷しているため、東京の土地所有者たちは恐慌をきたしています。
\\	突然	とつぜん	㊀[ト・タル][文][形動タリ]予期しないことが急に起こるさま。 だしぬけであるさま。 突如。 「―として平野次郎が大変ありと言出(いいいず)るにぞ」 ㊁[副] ❶に同じ。 「―大声を出す」「―訪問する」 [用法]突然・不意に――「突然(不意に)電車が急停車した」「笛の音が突然(不意に)やんだ」など、急に事が起こる意では相通じて用いられる。 
\\	「突然」は、前触れなしに急に何かが起こるさまに意味の重点がある。 「突然大音響が聞こえた」「突然のお話で戸惑っております」
\\	「不意に」は、思ってもいなかったことが起きて驚き、当惑する気持ちに意味の重点がある。 「不意にけいれんが起こった」「不意に飛び出してきた」
\\	類似の語に「いきなり」がある。 「いきなり殴るとはひどい」「初出場でいきなり優勝した」のように、一足飛びに何かをするさまの意で、右の二例では「突然」「不意に」とは置き換えられない。	▲くすぶっていた薪が突然燃え上がった。 ▲ケーキ?僕、突然またお腹が空いちゃった!!
\\	トップ	トップ	
\\	順序の最初。 一番目。 「本日―の議題」 
\\	首位。 一位。 「前半を―で折り返す」 
\\	あるまとまりの中での最上層。 また、そこに位置する人や物。 「世界の―ランナー」 
\\	最上部。 てっぺん。 頂上。 「バックスイングの―の位置を直す」 
\\	表面。 上部。 また、屋根。 「デスク―型のパソコン」「ハード―のスポーツカー」 
\\	新聞で、紙面の最上段の右にあたるところ。 また、雑誌の巻頭。 「第一面の―を飾るスクープ」 
\\	自動車などの変速機で、最高速度を出すときに用いるギア。 トップギア。 「加速して―に入れる」 
\\	上半身に着るもの。 特に、上下に分かれている洋服の上着。 ↔ボトム。 
\\	ゴルフで、ボールの上部を打ってしまうミスショット。	▲その新米のセールスマンが、トップの販売実績をあげたやり方は指導員の目を見はらせた。 ▲彼女は試験にトップで合格した。
\\	届く	とどく	㊀[動カ五(四)] 
\\	ある所にまで至りつく。 達する。 及ぶ。 「四十に手が―・く」「遠くまで―・く声」 
\\	送った品物や郵便物が相手の所に着く。 「母から便りが―・く」「贈り物が―・く」 
\\	注意などが十分に行きわたる。 行き届く。 「親の目が―・く」「細かいところまで神経が―・く」 
\\	願い事がかなう。 気持ちが通じる。 「祈りが―・く」「誠意が―・く」 ㊁[動カ下二]「とどける」の文語形。	▲その品は届くまで2週間ほどかかります。 ▲丈の長いコートは、ほとんどくるぶしまで届いていた。
\\	飛ばす	とばす	[動サ五(四)] 
\\	空中を飛ぶようにする。 「模型飛行機を―・す」「風船を―・す」 
\\	何かをめがけて、手元から勢いよく放つ。 「ボールを―・す」「矢を―・す」 
\\	強く吹いて、空中に舞い上がらせる。 「台風で屋根が―・される」「帽子を―・す」 
\\	空中に散るようにする。 はね散らす。 飛散させる。 「自動車が泥水を―・して通る」 
\\	手や足で攻撃やわざをすばやくしかける。 「内掛けを―・す」 
\\	乗り物を速く走らせる。 また、スピードを出して速く進む。 「時速一〇〇キロで―・す」「前半をハイペースで―・す」 
\\	勢いよく、また、遠慮なく言う。 「やじを―・す」「皮肉を―・す」 
\\	命令などが早く伝わるようにする。 うわさなどが広まるようにする。 「檄(げき)を―・す」「デマを―・す」 
\\	急いで派遣する。 「急便を―・す」「記者を現場に―・す」 
\\	途中を抜かして先へ進む。 間をぬく。 「―・して読む」 
\\	中央から地方へ、または都会から辺地へ転任させる。 左遷する。 「地方の出張所へ―・された」 
\\	あったものを一時になくする。 「強火で鍋のアルコール分を―・す」 
\\	(動詞の連用形に付いて)その動詞の意味を強める。 勢いよく…する。 乱暴に…する。 「しかり―・す」「売り―・す」「突き―・す」 
\\	手の届かない所へ飛び去らせる。 死なせる。 「手に持てる我(あ)が子―・しつ」 [可能]とばせる [下接句]羽觴(うしよう)を飛ばす・檄(げき)を飛ばす・口角沫(あわ)を飛ばす・錫(しやく)を飛ばす・魂を天外に飛ばす・与太を飛ばす	▲凧を飛ばすのは危険になることがある。 ▲自分の知らない単語を飛ばし読みする読者は多い。
\\	飛び出す	とびだす	[動サ五(四)] 
\\	急に勢いよく飛んで出る。 勢いよく外や前へ出る。 また、勢いよく他を抜いて前へ出る。 「箱を開けると人形が―・す」「ゴール直前に―・す」 
\\	思いがけないものがその場へ突然現れる。 「車の前に急に―・す」「隠し芸が―・す」 
\\	外の方へ突き出る。 「目玉が―・す」 
\\	そこからあわただしく出て行く。 「遅刻しそうになって家を―・す」 
\\	よくない事情などが生じて、職場や本拠とする所から急に外部へ出る。 「会社を―・す」「親とけんかして家を―・す」 
\\	飛び始める。 「今日から新空港で飛行機が―・した」	▲箱を開けると、様々な夢が中から飛び出した。見えなくなるインクで書かれた秘密とか、ものすごい臭いといったものについての夢が飛び出したのだった。 ▲椎間板ヘルニアは背骨の間にある椎間板という軟骨が飛び出すものです。
\\	友・朋	とも	《「共」と同語源》 
\\	いつも親しく交わっている相手。 友人。 朋友(ほうゆう)。 ともだち。 「良き―に恵まれる」 
\\	志や目的を同じくする人。 仲間。 同志。 「学問の―」「類は―を呼ぶ」 
\\	ふだん好んで親しんでいるもの。 「音楽を―とする」	▲私と友に眠る。 ▲新たなる狙撃が別の友を殺す。
\\	共に・倶に	ともに	〔連語〕 
\\	一緒にあることをするさま。 また、そろって同じ状態であるさま。 「父と―行く」「私も兄も―健康だ」 
\\	あることに伴って、別のことが同時に起こるさま。 「雪解けと―草木が芽吹く」	▲水と石油は共に液体である。 ▲世界の同胞諸君、アメリカ合衆国が諸君のために何をしてくれるかを求めず、人類の自由のためにともに何ができるかを求めよう。
\\	土曜	どよう	週の第七日。 金曜の次の日。 一週の最終の日。 土曜日。	▲彼はよく土曜の夜に外食します。 ▲土曜以外は一日に5時間授業がある。
\\	虎	とら	
\\	ネコ科の哺乳類。 ライオンと並ぶ大形の猛獣。 体長約二メートル。 全身黄褐色で黒い横縞がある。 シカ・イノシシなどを捕食。 沿海州から朝鮮半島・中国を経てインド・ジャワ・バリ島まで広くアジアに分布し、主に密林に単独またはつがいで暮らす。 シベリアトラ・ベンガルトラなどの亜種に分けられる。 乱獲により数が激減。 
\\	酒に酔って言動が荒くなった人。 酔っぱらい。 「―箱」	▲猫はトラと近縁である。 ▲彼は虎が好き。
\\	ドライブ	ドライブ	[名]スル 
\\	自動車を運転すること。 また、自動車で遠出すること。 「半島を―する」 
\\	テニス・卓球などで、順回転するようにボールを打つこと。 また、その打球。 「サーブに―をかける」 
\\	機械などを駆動すること。 また、その装置。 コンピューターでは、磁気ディスクの駆動装置。 ディスクドライブ。	▲我々は今度の日曜日にドライブに出かけるつもりです。 ▲海までドライブしましょう。
\\	トラック	トラック	
\\	陸上競技場で、競走用の走路。 「―を一周する」 
\\	「トラック競技」の略。 陸上競技のうち、トラックで行う競技の総称。競走・障害物競走・リレーなど。 
\\	㋐磁気テープや映画フィルムなどの録音する部分。 また、レコード盤の溝。 ㋑コンピューターの記憶装置の、データを記録する領域。 磁気ディスクなどでは同心円状に配置される。 →セクター →クラスター	▲このトラックは修理が必要である。 ▲彼女はかつてトラック競技のスターだった。
\\	ドラマ	ドラマ	
\\	演劇。 芝居。 「テレビ―」 
\\	戯曲。 脚本。 「―を書く」 
\\	劇的な出来事。 劇的事件。 「旅先で―がある」	▲彼は人生のドラマに満ちていた。 ▲どのドラマが一番好きですか。
\\	トランプ	トランプ	《切り札の意》カード式の室内遊戯具の一。 スペード・ハート・クラブ・ダイヤの四種のマークの札が各一三枚と、ジョーカー一枚、計五三枚を一組とする。 西洋カルタ。 また、それを使用するゲーム。	▲彼はトランプがうまい。 ▲彼がみんなとトランプをしているところをみました。
\\	取り上げる	とりあげる	[動ガ下一][文]とりあ・ぐ[ガ下二] 
\\	置かれているものを手に取って持ち上げる。 手に取る。 「受話器を―・げる」 
\\	申し出や意見を受け入れる。 採用する。 「緊急動議を―・げる」 
\\	相手のもっているものを無理に奪う。 ㋐財産・地位などを奪い取る。 没収する。 「田畑を―・げる」「免許を―・げる」 ㋑相手の持ち物を一方的に奪う。 おさえ取る。 「賊の凶器を―・げる」 
\\	税金などを取り立てる。 徴収する。 「会費の滞納分を―・げる」 
\\	産婦を助けて子を産ませる。 「産院で―・げてもらう」 
\\	問題として扱う。 「訴えを―・げる」 
\\	髪を結いあげる。 「内儀さんの背後へまわって髪を―・げてやったりした」	▲この本に入っている話は若者が直面する多くの問題のいくつかを取り上げている。 ▲これが雑誌で大きく取り上げられている歌手達の写真です。
\\	努力	どりょく	[名]スルある目的のために力を尽くして励むこと。 「―が実る」「たゆまず―する」「―家」	▲努力は成功の基本要素である。 ▲努力は必ずしも報われるものではない。
\\	ドレス	ドレス	衣服・衣装・服装などの総称。 特に、ワンピース型の婦人服。 正装や礼服の場合に用いられる。 「ウエディング―」	▲この帽子は茶色のドレスにぴったりだ。 ▲これは私が先週作ったドレスです。
\\	取れる・捕れる・採れる・撮れる	とれる	[動ラ下一][文]と・る[ラ下二] 
\\	ついていたものが離れ落ちる。 はなれる。 「ボタンが―・れる」「表紙が―・れる」 
\\	今まであった好ましくない状態が消え去る。 「疲れが―・れる」「痛みが―・れる」 
\\	(「穫れる」「獲れる」とも書く)収穫物・捕獲物や資源が得られる。 「米が―・れる」「近海で―・れた魚」「良質の鉄鉱石が―・れる」 
\\	そのように解釈できる。 理解される。 「どちらにも―・れる説明」「皮肉に―・れる」 
\\	調和した状態になる。 「釣り合いが―・れる」「栄養のバランスの―・れた食事」 
\\	(撮れる)写真に写る。 「よく―・れた写真」 
\\	(「録れる」と書く)録音される。 「鳥の声がよく―・れる」 [下接句]肩上げが取れる・角(かど)が取れる・圭角(けいかく)が取れる・採算が取れる	▲ドアを開けようとしたら、ドアの握りがとれた。 ▲すぐにいたみはとれます。
\\	泥	どろ	
\\	水がまじってやわらかくなった土。 粒子は砂よりも細かく、大きさによりシルトと粘土とに分ける。 
\\	「泥棒」の略。 「こそー」	▲彼女は足踏みをして靴の泥を落とした。 ▲母さんは彼の靴から泥を落とした。
\\	屯	たむろ	
\\	仲間や同じ職業の人々などが寄り集まっていること。 また、その集団・場所。 
\\	明治時代、特に巡査の詰めている所。 駐在所。 「―へ訴えて出るがいい」	▲そのトラックの積み荷は三トンを超えていた。 ▲今年の鉄鋼生産は1億トンに達するものと見積もられている。
\\	豚	とん	ぶた。 ぶた肉。 「―カツ」「―汁」	▲彼らは銃でねらわれ豚のように逃げ回る。 ▲豚は太ってきている。
\\	トンネル	トンネル	[名]スル 
\\	山腹や地下などを掘り貫いた通路。 鉄道・自動車道・人道や水路用。 隧道(ずいどう)。 
\\	野球で、野手がゴロを捕りそこない、球をまたの間を通過させて、後ろに逃がすこと。	▲汽車はトンネルを通り抜けた。 ▲そのトンネルは先日の地震で崩れ落ちた。
\\	名	な	
\\	ある事物を他の事物と区別するために、それに対応するものとして与える、言語による記号。 名前。 ㋐一般に、その事物の呼び方。 「人と―の付く生き物」「花の―」 ㋑ただ一つしか存在しないものとしての、その事物の固有の呼び方。 「富士という―の山」「―もない島」 ㋒その人の、固有の呼び方。 氏名。 姓名。 また、姓に対して、家の中でその人を区別する呼び方。 「初対面で―を名乗る」「子に―を付ける」 ㋓その集団・組織などの呼び方。 「学校の―」 
\\	集団・組織などを代表するものとして、表向きに示される呼び方。 名義。 「会社の―で登記する」 
\\	㋐評判。 うわさ。 「好き者の―が広がる」 ㋑名声。 名誉。 「世に―の聞こえた人物」「家の―を傷つける」 ㋒守るべき分際。 名分。 →名を正す 
\\	㋐うわべの形式。 体裁。 「会社とは―ばかりの個人経営」 ㋑表向きの理由。 名目。 「福祉事業の―で営利をむさぼる」 [下接語]徒(あだ)名・渾(あだ)名・宛(あて)名・家名・一名・浮き名・氏(うじ)名・烏帽子(えぼし)名・大(おお)名・贈り名・幼(おさな)名・男名・替え名・隠し名・仮(か)名・唐(から)名・国名・源氏名・小路(こうじ)名・小(こ)名・醜(しこ)名・通り名・殿名・暖簾(のれん)名・又の名・真(ま)名・物の名・大和(やまと)名・呼び名・童(わらわ)名	▲誰かが人込みの中で私の名を呼ぶのが聞こえた。 ▲池田が姓で和子が名です。
\\	内容	ないよう	
\\	容器や包みなどの、中に入っているもの。 なかみ。 「手荷物の―を申告する」 
\\	物事を成り立たせているなかみ。 実体。 実質。 「試合の―に不満が残る」 
\\	文章や話などの中で伝えようとしている事柄。 意味。 「手紙の―」「―のない番組」 
\\	哲学で、事物や事象を成立させ、また、表面に現れている実質・意味。 ↔形式。 [類語]
\\	中身(なかみ)・正味・内訳・品目・コンテンツ/
\\	中身・実質・実体・内実・実(じつ)/
\\	趣(おもむき)・旨(むね)・趣旨(しゆし)・趣意・要旨	▲私たちは簡単な内容を伝えるのにしばしば身振りを用いる。 ▲私は彼の話の内容が分からなかった。
\\	猶・尚	なお	㊀[副] 
\\	以前の状態がそのまま続いているさま。 相変わらず。 やはり。 まだ。 「今も―健在だ」「昼―暗い」 
\\	状態や程度がいちだんと進むさま。 さらに。 もっと。 いっそう。 「君が来てくれれば―都合がいい」「会えば別れが―つらい」 
\\	現にある物事に付け加えるべきものがあるさま。 「―検討の余地がある」「―一〇日の猶予がほしい」 
\\	(あとに「ごとし」を伴って)ちょうど。 あたかも。 「過ぎたるは―及ばざるがごとし」「御首は敷皮の上に落ちて、質(むくろ)は―坐せるが如し」 ㊁[接]ある話の終わったあとで、さらに別のことを言い添えるのに用いる語。 「―詳しくは後便にて申し上げます」	▲彼は約束を破るが、それでもなお私は彼が好きだ。 ▲彼は背が高いが、兄はなお高い。
\\	仲	なか	《「中」と同語源》人と人との間柄。 「―のよい友達」「気楽な―」「犬猿の―」	▲私達の仲も終わりね。 ▲仲の良い友達といると、時間が経つのが早い。
\\	流す	ながす	[動サ五(四)] 
\\	液体が流れるようにする。 ㋐水などを流れさせる。 「汚水をどぶに―・す」「トイレの水を―・す」 ㋑血・汗・涙などをしたたらせる。 「脂汗を―・す」「よだれを―・す」 ㋒水流に乗せて他の物を運ばせる。 「いかだを―・す」「台風で橋が―・される」 ㋓付着物を水や湯などで洗い落とす。 「背中を―・す」「シャワーで汗を―・す」 
\\	物を動かして移らせる。 ㋐空中に漂わせる。 「異臭を―・す」 ㋑流罪に処する。 配流する。 「罪人を離れ島に―・す」 ㋒次々にめぐらせる。 「電流を―・す」「ムード音楽を―・す」 ㋓広く伝わらせる。 「デマを―・す」「情報を―・す」 ㋔ひそかに売ったり渡したりする。 「物資を闇に―・す」 
\\	物事が成立しないようにする。 ㋐とりやめにする。 中止する。 「総会を―・す」「ゲームを―・す」 ㋑わきにそらす。 その事にこだわらないようにする。 「論敵の攻撃を軽く―・す」「柳に風と聞き―・す」 ㋒流産させる。 「おなかの子を―・す」 ㋓一定の期間が過ぎて質物の所有権を失う。 「入質したカメラを―・す」 ㋔力まないで気楽にする。 「ウオーミングアップに一〇〇メートルを軽く―・す」 ㋕野球で、流し打ちをする。 「ライトに―・す」↔引つ張る。 
\\	客を求めてあちこち移り動く。 「タクシーが盛り場を―・す」「酒場をギターを弾いて―・す」 
\\	遊郭で居つづける。 流連する。 「長く居なさんな、息子株ぢゃああるめえし、―・せば迚(とつて)も程があらあな」 [可能]ながせる [下接句]汗を流す・汗水を流す・油を流したよう・車軸を流す・智(ち)に働けば角(かど)が立つ、情(じよう)に棹(さお)させば流される・水に流す・涎(よだれ)を流す	▲あんな恐ろしい男のために流す涙はない。 ▲津波で流されてしまったのです。
\\	半ば	なかば	㊀[名] 
\\	全体を二つに分けた、その一方。 半分。 「敷地の―を人手に渡す」 
\\	一定の距離・期間などの中間のあたり。 「枝を―から切り落とす」「五月の―」「人生の―」 
\\	ある物事の途中。 ある物事をしている最中。 「式典の―で退席する」「志―で挫折する」 ㊁[副] 
\\	半分ほど、ある状態になっているさま。 「―あきれ、―感心する」 
\\	完全にではないが、かなりの程度。 ほとんど。 「―観念している」	▲伝記を書くことが難しいのは、それが半ば記録であり、半ば芸術であるからだ。 ▲東京では、11月半ばに寒い季節が始まります。
\\	仲間	なかま	
\\	一緒に物事をする間柄。 また、その人。 「趣味を同じにする―に加わる」「飲み―」 
\\	地位・職業などの同じ人々。 「文士の―」 
\\	同じ種類のもの。 同類。 「オオカミは犬の―だ」 
\\	近世、商工業者の同業組合。 官許を得たものを株仲間といった。 [類語]
\\	同輩・朋輩(ほうばい)・同僚・同志・同人・友・メート/
\\	同類・一類・一党・徒輩(とはい)・徒(と)・ともがら・やから・たぐい	▲水夫は仲間の水夫が力尽きて沈むのを目撃した。 ▲息子が悪友仲間に入った。
\\	眺め	ながめ	
\\	見渡すこと。 遠くまで見ること。 また、その風景。 眺望。 「―のきく場所」「―が良い」 
\\	物思いにふけりながら見ること。 和歌では多く「長雨(ながめ)」と掛けて用いる。 「花の色は移りにけりないたづらに我身世にふる―せしまに」	▲誰もがここからの眺めを美しいと言う。 ▲大きな柱が湖水の眺めを遮っている。
\\	眺める	ながめる	[動マ下一][文]なが・む[マ下二] 
\\	視野に入ってくるもの全体を見る。 のんびりと遠くを見る。 広く見渡す。 「星を―・める」「田園風景を―・める」 
\\	じっと見つめる。 感情をこめて、つくづくと見る。 「しげしげと人の顔を―・める」 
\\	かたわらで成り行きを見る。 静観する。 「状況を―・める」 
\\	物思いにふけりながら、見るともなくぼんやり見る。 「夕月夜のをかしきほどに出だし立てさせ給ひて、やがて―・めおはします」	▲子供はその動物を面白がって眺めた。 ▲犬は不安そうに主人をながめた。
\\	流れ	ながれ	㊀[名] 
\\	液体や気体が流れること。 また、その状態や、そのもの。 「潮の―が速い」「空気の―が悪い」「川の―をせき止める」 
\\	流れるように連なって動くもの。 また、その動き。 「人の―に逆らって歩く」「車の―がとどこおる」 
\\	目上の人などから、杯を順にめぐらせること。 また、杯の飲み残しのしずく。 「お―を頂戴する」「―の御かはらけ給はらばや」→お流れ 
\\	時間の経過や物事の移り変わり。 「時代の―に乗る」「試合の―を読む」 
\\	血筋。 また、流派。 系譜。 「菅家の―」「印象派の―」 
\\	集会などから人が一斉に移動すること。 「卒業式の―」 
\\	予定・計画などが中止になること。 →お流れ 
\\	質物を受け出す期限が過ぎて、所有権がなくなること。 また、その質物。 「質―」 
\\	屋根などの傾斜。 また、その度合い。 「片―」 
\\	人が当て所もなく歩くこと。 また、定めのない境遇。 遊女の身の上などにいう。 「―の者」「つらさ果てなき―の苦しみ」 ㊁〔接尾〕助数詞。 旗・幟(のぼり)など、細長いものを数えるのに用いる。 「白旗二十余―」 [下接語]枝流れ・御(お)流れ・片流れ・川流れ・里流れ・質流れ・四方流れ・注文流れ・抵当直(じき)流れ・抵当流れ・手付(てつけ)流れ・横流れ	▲昨夜の雨で川の水の流れが非常に早い。 ▲私たちは川の強い流れにさからってボートをこぐことができなかった。
\\	流れる	ながれる	[動ラ下一][文]なが・る[ラ下二] 
\\	㋐液体がある方向へ道筋をなすように移動する。 「川が―・れる」「潮が―・れる」 ㋑水滴などが筋となって伝わり落ちる。 「汗が―・れる」「涙が―・れる」 ㋒液体の移動とともに動く。 川の水などに運ばれて動く。 「洪水で橋が―・れる」「氷山が―・れる」 
\\	川の水などが移動するように、連続してものが動く。 ㋐空中を移動する。 「霧が―・れる」「星が―・れる」 ㋑経路を伝って移動する。 「電流が―・れる」「渋滞で車が―・れない」 ㋒伝わり広がる。 「世間にうわさが―・れる」「怪情報が―・れる」 
\\	時間が経過する。 「歳月が―・れる」 
\\	人が定まりなく移動する。 「職を求めて土地から土地へ―・れる」 
\\	㋐本来の経路などから外れて動く。 思いがけない方向へ行く。 「他店へ客が―・れる」「砲弾が―・れる」 ㋑押さえがきかないで思わず動いてしまう。 「腰が―・れる」「筆が―・れる」 ㋒人の態度などが、望ましくない方へ傾く。 「怠惰に―・れる」「奢侈(しやし)に―・れる」 
\\	㋐予定されていた行事などが中止になる。 物事が実現しないまま終わる。 「会議が―・れる」「企画が―・れる」 ㋑流産する。 「おなかの子が―・れる」 
\\	一定の期限が過ぎて、質物の所有権がなくなる。 「質草が―・れる」 
\\	テレビの映像が乱れる。 「画面が上下に―・れる」 [類語]
\\	㋐)流動する・貫流する・流通する・流出する・捌(は)ける/
\\	㋑)滴る・零(こぼ)れる・零れ落ちる・伝う/
\\	㋒)漂う・浮流する・漂流する	▲その川は両国の間を流れている。 ▲ボールが川を流れている。
\\	無し	なし	
\\	無いこと。 無(む)。 「抜け駆けは―にしよう」 
\\	名詞などに付いて複合語をつくり、…のないこと、…ない人、の意を表す。 「芸―」「人で―」「待った―」	▲彼なら相手にとって不足なし。 ▲彼の援助なしでもやっていける。
\\	何故なら	なぜなら	[接]「なぜならば」に同じ。 事柄の原因・理由の説明をみちびくのに用いる。なぜかというと。なぜなれば。なぜなら。「今は何とも言えない。―まだ協議中だから」「もうがまんできない。 ―彼の態度はあまりにもひどい」	▲私は犬の方が猫より好きです。なぜなら前者の方が後者より忠実だからです。 ▲私は山に登る、なぜならそれがそこにあるからだ。
\\	謎	なぞ	(「何ぞ」の意) ①なぞなぞ。〈運歩色葉集〉 ②遠回しにそれとさとらせるように言いかけること。 ③正体がはっきりしないこと。不思議・不可解なこと。「―に包まれる」「―の人物」「宇宙の―を解く」	▲彼がそれを書いたかどうかは、いつまでもなぞだろう。 ▲謎を未解決のままにするな。
\\	納得	なっとく	[名]スル他人の考えや行動などを十分に理解して得心すること。 「―のいかない話」「説明を聞いて―する」	▲恵子さんが良くてもみんなが納得しないんです。後で俺がドヤされるんだから。 ▲「そっか」ウィリーはようやく納得した。
\\	何か	なにか	〔連語〕 ㊀《代名詞「なに」+助詞「か」》 
\\	《「か」は副助詞》感覚・願望などの内容がはっきりしない事物をさす。 「―がありそうだ」「お茶か―飲みたい」 
\\	《「か」は係助詞。 感動詞的に用いる》 ㋐相手の言葉・気持ちを確認しようとする意を表す。 「それなら―、君のほうが正しいというのだな」 ㋑今まで述べてきたことや相手の言葉などを否定して、それとは反対の趣旨を述べるときに用いる。 いやいや。 とんでもない。 「―。 この歌よみ侍らじとなむ思ひ侍るを」 ㊁《副詞「なに」+助詞「か」》 
\\	《「か」は副助詞》はっきりした訳もなく、ある感情が起こるさま。 どことなく。 なんだか。 「―気味が悪い」 
\\	《「か」は係助詞》 ㋐疑問の意を表す。 なぜ…か。 どうして…か。 「あしひきの山も近きをほととぎす月立つまでに―来鳴かぬ」 ㋑反語の意を表す。 どうして…か、いやそんなことはない。 「命だに心にかなふものならば―別れの悲しからまし」	▲もし私の身に何か起こったら、ここを調べてみて。 ▲もし私の部屋に泥棒が入って来たら、何かを投げつけてやります。
\\	何も	なにも	〔連語〕 
\\	(「…も何も」の形で)同類の事物を一括して示す。 どんなものでも。 「仕事も―忘れて休養する」 
\\	全面的に否定する気持ちを表す。 まったく。 「―知らない」 
\\	取り立ててそう限定する必要もないという気持ちを表す。 別段。 「―今日でなくともいいのに」	▲その箱には何もありませんでした。 ▲我々は本当に何も予言できはしない。
\\	鍋	なべ	《肴(な)を煮る瓮(へ)の意》 
\\	食物を煮る、揚げる、ゆでる、蒸すなどの加熱調理をする器。 金属製・陶器製などで、ふた・つる・取っ手などが付き、用途別に多くの種類がある。 
\\	鍋料理のこと。 鍋物。 「―をつつく」「牛―」 [下接語]胡坐(あぐら)鍋・揚げ鍋・圧力鍋・石狩鍋・煎(い)り鍋・御(お)鍋・親子鍋・牡蠣(かき)鍋・燗(かん)鍋・牛(ぎゆう)鍋・薬鍋・小鍋・桜鍋・猪(しし)鍋・慈善鍋・社会鍋・塩汁(しよつつる)鍋・成吉思(ジンギス)汗鍋・鋤(すき)鍋・楽しみ鍋・中華鍋・ちり鍋・弦(つる)鍋・手鍋・泥鰌(どじよう)鍋・土手鍋・土鍋・鳥鍋・生臭鍋・肉鍋・蛤(はま)鍋・平鍋・牡丹(ぼたん)鍋・蒸し鍋・焼き鍋・柳川(やながわ)鍋・行平(ゆきひら)鍋・寄せ鍋・割れ鍋	▲鍋にごま油を中火で熱し、にんにく、鶏肉を入れて炒め、色が変わったら中華スープと白菜を入れて煮る。 ▲ここに手のないナベがある。
\\	生	なま	㊀[名・形動] 
\\	食物などを煮たり焼いたりしていないこと。 加熱・殺菌などの処理をしていないこと。 また、そのさま。 「魚を―で食う」「しぼりたての―の牛乳」 
\\	作為がなく、ありのままであること。 また、そのさま。 「国民の―の声」「―な身をもってしたおのれの純粋体験から」 
\\	㋐演技・演奏などを直接その場で見たり聞いたりすること。 「―の舞台」 ㋑録音・録画などによらないで直接その場から放送すること。 「―の番組」 
\\	技術・経験などが未熟であること。 また、そのさま。 「石鹸(しやぼん)なんぞを、つけて、剃るなあ、腕が―なんだが」 
\\	「生意気」の略。 「―を言う」「お―な子」 
\\	「生ビール」の略。 「ビールは―がうまい」 ㊁[副]なんとなく。 中途半端に。 「この男も―頭(かしら)痛くなりて」 ㊂〔接頭〕 
\\	名詞に付いて、いいかげんな、中途半端な、などの意を表す。 「―返事」「―あくび」「―煮え」 
\\	形容詞・形容動詞に付いて、少しばかり、何となく、などの意を表す。 「―ぬるい」「―暖かい」 
\\	人を表す名詞に付いて、年功が足りない、世慣れていない、年が若いなどの意を表す。 「―女房」「―侍」	▲君はその魚を生で食べるべきではなかった。 ▲安全日だからといって、サルのように生ではしません。 しっかり避妊するのが愛のセックスの義務ではないでしょうか?
\\	怠ける・懶ける	なまける	[動カ下一][文]なま・く[カ下二] 
\\	なすべきことをしない。 働かない。 ずるける。 「仕事を―・ける」 
\\	元気がなくなる。 力がなくなる。 「旅人を乗せたる馬士(まご)―・けたる声にて」 
\\	なまやさしくなる。 鈍くなる。 「京の詞は―・けて悪い」	▲怠けていて母にしかられた。 ▲青春時代を怠けて過ごすな。さもないと後で後悔するぞ。
\\	波・浪・濤	なみ	
\\	風や震動によって起こる海や川の水面の高低運動。 波浪。 「―が寄せてくる」「―が砕ける」「逆巻く―」 
\\	空間や物体の一部における振動や変化が、周囲の部分に次々に伝わっていく現象。 波動。 「光の―」「音の―」 
\\	押し寄せるように揺れ動くものの動き。 「人の―」 
\\	個人ではどうしようもない変化が、かわるがわる生じること。 「歴史の―」「国際化の―」 
\\	形状や有様などが、波の形や動きを思わせるもの。 「いらかの―」「雲の―」「稲穂の―が揺れる」 
\\	物事の動向にでこぼこ・高低・出来不出来などがあって、一定しないこと。 むら。 「調子に―がある」「成績に―がある」「感情の―が激しい」 
\\	老いて皮膚にできるしわ。 「老いの―」「額の―」 
\\	海水の流れ。 うしお・潮流。 「はやき―ありて、はなはだはやきに会ひぬ」 
\\	世の乱れ。 騒ぎ。 ごたごた。 「四海の―も静かにて」 
\\	はかないもの、消えやすいものをたとえていう語。 「―と消えにし跡なれや」 
\\	紋所の名。 
\\	を図案化したもの。 [下接語]徒(あだ)波・荒波・磯(いそ)波・浮き世の波・卯(う)波・大波・男(お)波・風(かざ)波・風(かぜ)波・川波・黄金(こがね)の波・小波・逆(さか)波・細(さざ)波・細(ささら)波・細(さざれ)波・三角波・白(しら)波・高波・縦波・津波・年波・土用波・早波・人波・藤(ふじ)波・穂波・女(め)波・夕波・横波	▲波が浜に打ち寄せている。 ▲波が高い。
\\	涙・涕・泪	なみだ	
\\	涙腺(るいせん)から分泌される液体。 眼球をうるおし、異物を洗い流す作用がある。 刺激や感動で分泌が盛んになる。 「―がこぼれる」「―を浮かべる」 
\\	涙を流すこと。 泣くこと。 「聞くも―、語るも―」「―をこらえる」「―をさそう話」 
\\	人間らしい思いやり。 人情。 情愛。 同情心。 「血も―もない」 
\\	(接頭語的に)名詞に冠して、ほんの少しの意を表す。 「―金」 [下接語]有り難(がた)涙・嬉(うれ)し涙・おろおろ涙・忝(かたじけ)涙・悔し涙・忍び涙・空(そら)涙・溜(た)め涙・血の涙・共涙・泣きの涙・貰(もら)い涙	▲彼女は目に涙を浮かべてその事故について語った。 ▲涙は子供の武器である。
\\	悩む	なやむ	㊀[動マ五(四)] 
\\	決めかねたり解決の方法が見いだせなかったりして、心を痛める。 思いわずらう。 「進学か就職かで―・む」「恋に―・む若者」「人生に―・む」 
\\	対応や処理がむずかしくて苦しむ。 困る。 「騒音に―・む」「人材不足に―・む企業」 
\\	からだの痛みなどに苦しむ。 また、病気になる。 「頭痛に―・む」「持病のぜんそくに―・む」 
\\	とやかく言う。 非難する。 「御子もおはせぬ女御の后にゐ給ひぬる事、安からぬ事に世の人―・み申して」 
\\	(動詞の連用形に付いて)その動作が思うようにはかどらない意を表す。 「伸び―・む」「行き―・む」 ㊁[動マ下二]「なやめる」の文語形。 [類語] 
\\	苦しむ・煩(わずら)う・悶(もだ)える・思い煩う・思い迷う・思い乱れる・苦悩する・懊悩(おうのう)する・煩悶(はんもん)する・憂悶(ゆうもん)する・苦悶(くもん)する・苦慮する・頭を痛める・頭を悩ます	▲彼はひどい頭痛に悩んでいる。 ▲彼は何か悩んでいるようだ。
\\	縄	なわ	
\\	麻・わらなどの植物繊維や化学繊維をより合わせて作った、細長いひも。 物を縛ったりつないだりするのに用いる。 「―をなう」 
\\	罪人を縛るための縄。 捕縄(ほじよう)。 とりなわ。 
\\	田畑の面積を間縄(けんなわ)で測ること。 また、それに用いるもの。 縄入れ。 →綱(つな)[用法] [下接語]麻縄・荒縄・鵜(う)縄・鉤(かぎ)縄・掛け縄・飾り縄・口取り縄・口縄・朽ち縄・首縄・化粧縄・間(けん)縄・極楽縄・腰縄・下げ縄・差し縄・緡(さし)縄・三寸縄・渋縄・注連(しめ)縄・棕櫚(しゆろ)縄・墨縄・高縄・手(た)縄・釣り縄・手縄・胴縄・年縄・捕り縄・泥縄・投げ縄・鳴子縄・荷縄・延(はえ)縄・早縄・引き縄・左縄・一筋縄・火縄・振り縄・水(みず)縄・水(み)縄・黐(もち)縄・櫓(ろ)縄・藁(わら)縄	▲彼は縄で木に縛り付けられた。 ▲禍福は糾える縄の如し。
\\	なんか	なんか	[副助]《代名詞「なに」に副助詞「か」の付いた「なにか」の音変化から》名詞、名詞に準じる語、活用語の連用形、一部の助詞などに付く。 
\\	一例を挙げて示す。 …など。 「この着物―お似合いです」「映画―よく行く」 
\\	ある事物を例示し、それを軽んじていう意を表す。 …など。 「彼の言うこと―聞くな」「君に―わからない」	▲身なりなんか気にしていません。 ▲冗談じゃないよ。あの人は私同様医者なんかじゃないよ。
\\	何で	なんで	[副] 
\\	どういうわけで。 どうして。 なぜ。 「―けんかなんかしたのだ」 
\\	反語表現に用いて、強く否定する意を表す。 どうして。 「―泣かずにいられようか」	▲なんでもっと早くここに来なかったのだ。 ▲ちょっとぉ、冗談でしょッ! なんでそんな辺鄙なところに行くわけッ?
\\	何でも	なんでも	《代名詞「なに」に副助詞「でも」が付いた「なにでも」の音変化》 ㊀[副] 
\\	よくはわからないが。 どうやら。 「―近く結婚するらしい」 
\\	どうしても。 ぜひ。 「何が―やりぬこう」「あれは世間に重宝する三光とやらいふ鳥であらう。 ―刺いてくれう」 ㊁〔連語〕どういうものでも。 どういうことでも。 「生活用品なら―ある」「頼まれたことは―する」	▲健康なら何でもできる。 ▲見込のありそうなことは何でも見失わないことだ。
\\	何とか	なんとか	㊀[副]スル 
\\	あれこれ工夫や努力をするさま。 どうにか。 「そこを―頼む」「―しよう」 
\\	完全・十分とはいえないが、条件・要求などに一応かなうさま。 かろうじて。 どうにか。 「―暮らしていける」「―間に合う」 ㊁〔連語〕 
\\	名称・内容などのはっきりしないものをさす。 「銀座の―という店」「―言ってみろよ」 
\\	あるものの名称などをはっきり言わないで、それをさす。 「―の一つおぼえ」 
\\	(「…とか何とか」の形で)それを含めて、あれこれ。 「病気だとか―言って休めばよい」	▲傷から流れる血をなんとかしなさい。 ▲何とか手配してあげましょう。
\\	似合う	にあう	[動ワ五(ハ四)]ちょうどよくつりあう。 調和する。 「着物がよく―・う人」「年に―・わず行動的だ」	▲この背広に似合うネクタイを選ぶのを手伝ってください。 ▲この背広に似合うネクタイが欲しいのですが。
\\	苦手	にがて	[名・形動] 
\\	扱いにくく、いやな相手。 なかなか勝てなくて、いやな相手。 また、そのようなさま。 「あいつはどうも―だ」 
\\	得意でないこと。 また、そのさま。 不得手。 「数学の―な人」 
\\	不思議な力をもつ手。 その手で押さえると、腹の痛みはおさまり、蛇は動けなくなって捕らえられるなどという。 「天性―といふものにて、小児の虫つかへをさするに妙を得て」	▲人前で歌うのは苦手です。 ▲数学が苦手なので家庭教師をつけてもらいたい。
\\	握る	にぎる	[動ラ五(四)] 
\\	手の指全部を内側へ曲げる。 また、そのようにして、物をつかんだり、持ったりする。 「こぶしを―・る」「ペンを―・る」「車のハンドルを―・る」 
\\	物事をとらえて自分のものとする。 手中に収める。 「実権を―・る」「政権を―・る」「大金を―・る」 
\\	重要な事柄を確実につかむ。 「相手の弱みを―・る」「秘密を―・る」「証拠を―・る」 
\\	握り飯や握りずしを作る。 「好みのねたを―・ってもらう」→掴(つか)む[用法] [可能]にぎれる [下接句]キャスチングボートを握る・采柄(さいづか)を握る・財布の紐(ひも)を握る・手に汗を握る・手を握る [類語]
\\	掴(つか)む・持つ・執(と)る・把持する・把握する/
\\	押さえる・制する・掌握する・確保する・保持する・独占する・占有する・支配する・手中に収める・我が物にする	▲彼が私の運命を握っている。 ▲彼はこの問題の鍵を握っている。
\\	日常	にちじょう	つねひごろ。 ふだん。 平生。 「―用いる道具」「―会話」「―性」	▲私たちは新聞で日常の出来事を知ります。 ▲日常の運動は体に良い。
\\	日曜	にちよう	週の第一日。 土曜の次の日。 日曜日。 キリスト教の安息日に由来し、官公庁・学校・一般企業で休日とする。	▲日曜の午後はたいてい買い物に出かけます。 ▲日曜に配達していますか。
\\	日光	にっこう	日の光。 太陽の光線。	▲動物も植物も、一般に考えられている以上に日光を必要とする。 ▲動物に食物と飲み物が大切なように、植物には雨と日光が大切である。
\\	にっこり	にっこり	[副]スル「にこり」に同じ。 嬉しそうに笑みをふくむさま。「―ともしない」「愛想よく―(と)笑う」「うれしさに思わず―(と)する」	▲彼女は手を振りながらにっこり笑った。 ▲彼女はにっこり笑って僕に挨拶した。
\\	日中	にっちゅう	
\\	日がのぼっている間。 ひるま。 「―留守にすることが多い」 
\\	六時の一。 まひる。 正午。 また、その時に行う勤行(ごんぎよう)。	▲私は、友人の息子が約6か月間一種の農場研修生として、日中この農場までやってくるのを許可するのに同意した。 ▲教科書問題や歴史認識、靖国神社への首相の参拝などで、日中関係に波風が立っている。
\\	日本	にっぽん	㊀わが国の呼び名。 「ヒノマルノハタハ―ノシルシデアリマス」→にほん(日本) ◆「日本」が「ニホン」か「ニッポン」かについては決定的な説はない。 「日」は漢音ジツ、呉音ニチで、ニチホンがニッポンに音変化し、発音の柔らかさを好むところからさらにニホンが生じたものか。 ジパング・ジャパンなどはジツホンに基づくものであろう。 国の呼称としては、昭和九年(一九三四)に臨時国語調査会(国語審議会の前身)が国号呼称統一案としてニッポンを決議したが、政府採択には至っていない。 日本放送協会は昭和二六年に、正式の国号としてはニッポン、その他の場合はニホンといってもよいとした。 日本銀行券(紙幣)や国際運動競技のユニホームのローマ字表記が
\\	なのは、右の事情による。 外務省では、英語による名称はジャパン
\\	を用いている。 なお本辞典では、両様に通用する語については、便宜上「にほん」の見出しのもとに集めた。 ㊁[名・形動]《安永・天明(一七七二〜一七八九)ごろの江戸での流行語》日本一であること。 すばらしいこと。 また、そのさま。 「この不自由なところが―だとうれしがりけり」	▲京都は以前日本の首都でした。 ▲教授は研究休暇で日本にいる。
\\	入場	にゅうじょう	[名]スル会場・競技場・式場などにはいること。 「選手団が―する」「―無料」↔退場。	▲子供は入場できません。 ▲子どもは入場お断りです。
\\	人気	にんき	
\\	人々の気受け。 世間一般の評判。 「―が上がる」「―をさらう」 
\\	その土地の人々の気風。 じんき。 「―の荒い土地柄」 
\\	人間の意気。 じんき。 「天道―に空しからず」	▲彼女は学生の間でたいへん人気がある。 ▲彼女は学識を鼻にかけていたので人気がなかった。
\\	人間	にんげん	
\\	ひと。 人類。 「―の歴史」 
\\	人柄。 また、人格。 人物。 「―がいい」「―ができている」 
\\	人の住む世界。 人間界。 世の中。 じんかん。 「―五十年下天のうちをくらぶれば」 [類語]
\\	人(ひと)・人類・人倫・万物の霊長・考える葦(あし)・米の虫・ホモサピエンス・人物・人士・仁(じん)・者(もの)/
\\	人(ひと)・人柄・為人(ひととなり)・人物・人格・器量・器(うつわ)・人(にん)	▲まるで化石みたいな人間だわ。 ▲私は昭和生まれの人間です。
\\	抜く	ぬく	㊀[動カ五(四)] 
\\	中にはいっているもの、はまっているもの、刺さっているものを引っ張って取る。 「刀を―・く」「歯を―・く」「とげを―・く」 
\\	中に満ちていたり含まれていたりするものを外へ出す。 「浮き袋の空気を―・く」「プールの水を―・く」「力を―・いて楽にする」 
\\	中にはいっている金品をこっそり盗み取る。 「車中で財布を―・かれた」「積み荷を―・かれる」 
\\	多くのものの中から必要なものを選び取る。 全体から一部分を取り出す。 「書棚から読みたい本を―・く」「秀歌を―・いて詞華集を編む」 
\\	今まであったもの、付いていたものを除き去る。 不要のものとして取り除く。 「染みを―・く」「籍を―・く」「不良品を―・く」「さびを―・いたにぎりずし」 
\\	手順などを省く。 また、それなしで済ませる。 省略する。 「仕事の手を―・く」「朝食を―・く」 
\\	前にいる者や上位の者に追いつき、さらにその先に出たり、その上位になったりする。 「先頭の走者を一気に―・く」「すでに師匠の芸を―・いている」 
\\	新聞報道などで、スクープする。 すっぱ抜く。 
\\	力などが他よりすぐれている。 基準よりも上である。 「実力が群を―・いている」 
\\	(「貫く」とも書く)突き通して向こう側へ出るようにする。 一方から他方へ通じさせる。 つらぬく。 「山を―・いてトンネルをつくる」「一、二塁間を―・くヒット」 
\\	型にはめて、ある形として取り出す。 また、ある部分だけ残して他の部分を染める。 「ハート形に―・く」「紫紺の地に白く―・いた紋」 
\\	攻め落とす。 「城を―・く」「堅塁を―・く」 
\\	和服の着方で、抜き衣紋にする。 「襟を―・く」 
\\	囲碁で、相手の死んだ石を取る。 
\\	(動詞の連用形に付いて)そのことを最後までする。 しとおす。 また、すっかり…する。 しきる。 「難工事をやり―・く」「がんばり―・く」「ほとほと困り―・く」 [可能]ぬける ㊁[動カ下二]「ぬける」の文語形。 [下接句]足を抜く・荒肝(あらぎも)を抜く・息を抜く・生き馬の目を抜く・生き肝を抜く・一頭地を抜く・肩を抜く・気を抜く・群を抜く・尻毛(しりげ)を抜く・筋骨(すじぼね)を抜かれたよう・月夜に釜(かま)を抜かれる・手を抜く・度肝を抜く・毒気(どつけ)を抜かれる・鼻毛を抜く・山を抜く	▲庭の雑草を抜くのは彼の仕事だ。 ▲山を掘りぬいて、トンネルを作った。
\\	抜ける	ぬける	[動カ下一][文]ぬ・く[カ下二] 
\\	中にはまっていたものや、ついていたものが離れて取れる。 「歯が―・ける」「栓が―・ける」「髪の毛が―・ける」 
\\	中に満ちていたり含まれたりしていたものが外へ出る。 「タイヤの空気が―・ける」「気の―・けたビール」「臭みが―・ける」 
\\	ある傾向・習慣・くせや力などがなくなる。 「怠けぐせが―・けない」「疲れが―・ける」「腰が―・ける」 
\\	本来あるべきもの、必要なものが漏れたり欠けたりしている。 「名簿から名前が―・けている」「主語が―・けている」 
\\	(「脱ける」とも書く) ㋐それまでいた場所や、属していた組織・仲間から離れる。 「座敷を―・ける」「組合を―・ける」 ㋑しばらくの間だけ自分の部署を離れる。 「仕事を―・けて人に会いに行く」 ㋒ある場所・状況から逃れ出る。 脱する。 「危ないところを無事に―・ける」「最悪の状況から―・ける」 ㋓言いつくろって責任を免れる。 言い逃れる。 「左様(そう)―・けてはいけぬ、真実の処を話して聞かせよ」 
\\	その所を通って向こう側へ出る。 一方の側から他方の側へ通って出る。 通り抜ける。 「打球が右中間を―・ける」「トンネルを―・ける」 
\\	(多く「ぬけた」「ぬけている」の形で)知恵が十分に働かない。 気がきかずぼんやりしている。 足りない。 「あの人はどこか―・けている」「間(ま)の―・けた話」 
\\	(「ぬけるような」「ぬけるように」の形で)隔てがなくなり、どこまでも続いている。 透き通っている。 「―・けるような青空」「―・けるように白い肌」 
\\	他が及ばないほどすぐれている。 ひいでる。 抜きんでる。 「官(つかさ)位高くのぼり、世に―・けぬる人の」 [用法]ぬける・おちる――「名簿に君の名が抜けて(落ちて)いたよ」「この記事には大切な部分が抜けて(落ちて)いる」など、あるべきものが欠けている意では相通じて用いられる。 
\\	「抜ける」は、中にあるものがなくなる、外に出る意に重点がある。 「くさみが抜ける」「魂が抜けたよう」「気の抜けたサイダー」「歯が抜ける」など。 
\\	「落ちる」は付いていたものが取れる意に重点がある。 「洗うと色が落ちる」「憑(つ)き物が落ちる」「がんこな油のしみがきれいに落ちた」などと使う。	▲風邪が抜けない。 ▲風邪がなかなか抜けません。
\\	布	ぬの	
\\	織物の総称。 古くは、絹に対して、麻・葛(くず)などの植物繊維で織ったものをいい、のち、木綿も含めた。 
\\	建築で、横・平ら・水平・平行などの意を表す語。 「―羽目(はめ)」	▲この布は変色しません。 ▲この布は肌触りが良い。
\\	根	ね	
\\	維管束植物の基本器官の一。 普通は地中にあって、植物体を支え、水・養分を吸収する。 先端に根冠に包まれた生長点があり、根毛をもつ。 「植木の―がつく」「竹が―を張る」 
\\	立ったり生えたりしているものの下の部分。 「歯の―」「髷(まげ)の―」 
\\	物事の基礎・土台。 根本。 「息の―を止める」「思想の―」 
\\	物事の起こるもと。 根本原因。 「悪の―を絶つ」「両国の対立の―は深い」 
\\	はれ物などの中心になっている堅い部分。 「できものの―」「魚の目の―」 
\\	本来の性質。 生まれつきの性質。 「―は心のやさしい人だ」 
\\	釣りで、海底にある岩礁帯。 「―魚」 
\\	名詞の下に付いて、複合語をつくる。 ㋐地に根ざしている、立っている意を表す。 「垣―」「岩―」 ㋑語調を整えるために用いる。 「杵(き)―」「島―」 [下接語]息の根・岩根・枝根・海老(えび)根・大(おお)根・尾根・主(おも)根・垣根・固根・草の根・首根・心根・舌の根・性(しよう)根・白(しら)根・白(しろ)根・付け根・這(は)い根・蓮(はす)根・羽根・歯の根・鬚(ひげ)根・菱(ひし)根・棒根・細根・眉(まゆ)根・屋根・矢の根・横根・若根	▲彼は根はやさしい男だった。 ▲彼は根は親切な男なのだ。
\\	願い	ねがい	
\\	願うこと。 また、その事柄。 「―を聞き入れる」「―が届く」 
\\	手続きを踏んで願い出ること。 また、その文書。 「―を出す」「退職―」	▲彼女の願いはいつか外国に留学することだ。 ▲彼女の願いを聞いてやるように言われていたが、彼は完全に無視してしまった。
\\	願う	ねがう	[動ワ五(ハ四)]《「ね(祈)ぐ」の未然形+反復継続の助動詞「ふ」から》 
\\	神仏に、希望の実現することを祈る。 祈請する。 願をかける。 「神前で合格を―・う」 
\\	望みがかなうように請い求める。 望み求める。 「成功を心から―・う」 
\\	こうしてほしいことを人に頼む。 助力や配慮を求める。 「手伝いを―・う」「ご援助を―・う」 
\\	公の機関に希望することを申請する。 請願する。 「国有林の払い下げを―・う」 
\\	商店で、品物を客に買ってもらう。 「お安く―・っております」 
\\	動詞の連用形や動作性のある漢語の名詞などに付いて、「…していただく」などの意を表す。 「ぜひともお越し―・います」「ぜひ出席―・いたい」 [可能]ねがえる [類語]
\\	祈る・祈願する・祈念する・誓願する・立願する・発願(ほつがん)する・願を掛ける・願を立てる/
\\	希(こいねが)う・望む・念ずる・念願する・願望する・希求する・希望する・庶幾する・切望する・切願する・熱望する・熱願する/
\\	請う・仰ぐ・頼む・懇願する・懇請する・懇望する/
\\	願い出る・請願する・出願する・申請する・陳情する	▲彼は私たちの幸福を願ってくれている。 ▲彼は裁判官に寛大な処置を願った。
\\	鼠	ねずみ	
\\	齧歯(げつし)目ネズミ科の哺乳類の総称。 一般に小形で、体毛は灰色・黒褐色で尾は細長い。 犬歯はなく、一対の門歯が発達し一生伸び続ける。 繁殖力は旺盛だが、寿命は短い。 農作物・貯蔵穀物などに甚大な損害を与え、また病気を媒介する。 
\\	「ねずみ色」の略。 
\\	ひそかに害をなす者。 
\\	つまらない者。 「ただの―ではない」 [下接語]藍(あい)鼠・赤鼠・油鼠・家鼠・薄鼠・萱(かや)鼠・川鼠・関節鼠・絹毛鼠・木鼠・銀鼠・熊(くま)鼠・黒鼠・毛長鼠・小鼠・濃(こ)鼠・独楽(こま)鼠・子守鼠・地鼠・麝香(じやこう)鼠・白鼠・大黒鼠・田鼠・旅鼠・月の鼠・天竺(てんじく)鼠・尖(とがり)鼠・棘(とげ)鼠・跳(とび)鼠・溝(どぶ)鼠・南京(ナンキン)鼠・濡(ぬ)れ鼠・野鼠・畑(はた)鼠・二十日鼠・針鼠・火鼠・姫鼠・袋鼠・葡萄(ぶどう)鼠・舞鼠・谷地(やち)鼠・鎧(よろい)鼠・利休鼠	▲ネズミはどんどん繁殖する。 ▲ねずみはペスト菌を運ぶ。
\\	熱帯	ねったい	赤道を中心として南北両回帰線に挟まれた地帯。 気候的には年平均気温がセ氏二〇度以上、または最寒月の平均気温がセ氏一八度以上の地域をいい、回帰線よりやや広い。	▲あの島は熱帯性気候です。 ▲熱帯の太陽は容赦なくぎらぎら照り付けた。
\\	熱中	ねっちゅう	[名]スル一つの物事に深く心を傾けること。 夢中になること。 「勝負事に―する」	▲彼は漫画雑誌に熱中している。 ▲彼は勉強に熱中していて私の呼ぶ声が聞こえなかった。
\\	年間	ねんかん	
\\	一年のあいだ。 「―計画」「―所得」 
\\	ある年代の間。 「元禄―」	▲私たちが訪問したときは、小説に半年間も取り組んでいたんですから。 ▲私たちは4年間ルームメイトだった。
\\	年中	ねんじゅう	
\\	《「ねんちゅう」とも》一年の間。 年間。 「―無休」 
\\	ある年代の間。 年間。 「宝暦―」 
\\	(副詞的に用いて)いつも。 始終。 「―働いてばかりいる」	▲ハワイでは1年中海水浴が楽しめる。 ▲沖縄は1年中よい気候だ。
\\	年代	ねんだい	
\\	経過してきた年月。 「―を経た樹木」 
\\	時の流れをあるまとまりで区切った期間。 「一九五〇―」 
\\	紀元から順に数えた年数。 「―順に整理する」 
\\	年齢層。 世代。 「同じ―の人」「―の差を感じる」	▲1960年代に日本の大学生は政府に対して示威運動を起こした。 ▲1960年代に日本では大学生による、政府に対するデモが起きた。
\\	年齢	ねんれい	生まれてから経過した年数。 とし。 よわい。 年歯。 「満(まん)―」「精神―」	▲彼は大体あなたぐらいの年齢です。 ▲彼は善悪の区別がつく年齢になっている。
\\	野	の	
\\	自然のままの広い平らな地。 のはら。 「―に咲く花」「―にも山にも若葉が茂る」 
\\	広々とした田畑。 のら。 「朝早くから―に出て働く」 
\\	動植物を表す名詞の上に付いて、そのものが野生のものであることを表す。 「―うさぎ」「―ばら」 
\\	人を表す名詞の上に付いて、粗野であるという意で卑しめる気持ちを表す。 「―幇間(だいこ)」「―育ち」 [下接語]荒(あら)野・荒れ野・枯れ野・裾野・夏野・花野・原野・春野・広野・冬野・焼け野	▲私の趣味は野の花の写真を撮ることです。 ▲恋人達は野苺を探して野原をあちこちと歩き回った。
\\	能	のう	
\\	ある物事をなしとげる力。 はたらき。 能力。 「人を動かす―にたける」 
\\	ききめ。 効能。 「薬の―書き」 
\\	技能。 また、誇ったり取り立てていったりするのにふさわしい事柄。 「机に向かうだけが―ではない」 
\\	日本の古典芸能の一。 中世に猿楽から発展した歌舞劇。 能は歌舞劇の一般名称で、田楽・延年などの能もあったが、猿楽の能がもっぱら盛行したため、それを単に能と称した。 室町時代に観阿弥・世阿弥父子が大成、江戸中期にほぼ現在の様式となった。 役に扮する立方(たちかた)と声楽をうたう地謡方(じうたいかた)、器楽を奏する囃子方(はやしかた)があり、立方はシテ方・ワキ方・狂言方、地謡方はシテ方、囃子方は笛方・小鼓方・大鼓方・太鼓方がつとめる。 現在、その流派はシテ方に五流、ワキ方に三流、狂言方に二流、囃子方に一四流がある。 能の詞章を謡曲といい、ふつう脇能物・修羅(しゆら)物・鬘(かずら)物・雑物・切能(きりのう)物の五つに分類し、現在約二四〇曲が上演可能である。	▲私は能に興味を持つアメリカ人に会いました。 ▲能は伝統的な日本の芸術である。
\\	農家	のうか	
\\	農業により生計を立てている世帯。 また、その家屋。 「専業―」 
\\	中国、戦国時代における諸子百家の一。 農耕につとめ、衣食を充足することを主張した。	▲農家の人々は温室で作物を育てざるを得ないのです。 ▲農家の人は春に小麦の種まきをする。
\\	農業	のうぎょう	土地を利用して作物を栽培し、また家畜を飼育して衣食住に必要な資材を生産する産業。 広義には農産加工・林業も含む。	▲農業は穀物を育てるための土壌の耕作と定義される。 ▲農業は多量の水を消費する。
\\	農民	のうみん	農業生産に従事する人。	▲農民たちは、なにをするのかわからなかった。 ▲農民が穀物の種を蒔いた。
\\	能力	のうりょく	
\\	物事を成し遂げることのできる力。 「―を備える」「―を発揮する」「予知―」 
\\	法律上、一定の事柄について要求される人の資格。 権利能力・行為能力・責任能力など。	▲彼の能力を全面的に信頼している。 ▲彼はアメリカにいる間に英語能力を伸ばした。
\\	ノー	ノー	㊀[名] 
\\	否定。 拒否。 不賛成。 「イエスか―か」 
\\	外来語の上に付いて、ないこと、しないこと、また、禁止することの意を表す。 「―コメント」「―スモーキング」 ㊁[感]拒否や不承知の意を表す語。 いいえ。 いや。 だめ。	▲君の言っている事は結局ノーという事ですね。 ▲私はあなたにノーと言わなければならない。
\\	軒・宇・簷・檐	のき	
\\	屋根の下端で、建物の壁面より外に突出している部分。 
\\	庇(ひさし)。	▲彼女は鳥かごを軒からつるした。
\\	残す・遺す	のこす	[動サ五(四)] 
\\	あとにとどめておく。 残るようにする。 「放課後まで生徒を―・す」「メモを―・して帰る」 
\\	もとのままにしておく。 「昔の面影を―・す」「武蔵野の自然を―・す地区」 
\\	全体のうちの一部などに手をつけないでおく。 「食べきれずに―・す」「電車賃だけは―・す」 
\\	消さないでそのままにしておく。 「証拠を―・す」「誤解を―・す」 
\\	後世に伝える。 死後にとどめる。 「偉大な足跡を―・す」「名を―・す」 
\\	ためこむ。 「小金(こがね)を―・す」 
\\	相撲で、相手の攻めに対して踏みこらえる。 「土俵際で―・す」 [可能]のこせる [類語]
\\	留(とど)める・残置する/
\\	保存する・温存する・存置する/
\\	余す・浮かす	▲彼女は息子に多額のお金を残して亡くなった。 ▲彼女は家族を後に残してニューヨークに行った。
\\	残り	のこり	残ること。 また、残ったもの。 「―の仕事をかたづける」「金の―を数える」「生き―」「売れ―」 [類語]余り・残余・残部・残物・剰余・余剰・余分・余裕	▲彼は残りの仕事を免除された。 ▲彼は財産の残りを処分した。
\\	乗せる・載せる	のせる	[動サ下一][文]の・す[サ下二] 
\\	物の上に置く。 「荷物を網棚に―・せる」「子供をひざに―・せてあやす」 
\\	乗り物の上、または中に人や物を置く。 「客を―・せたタクシー」「トラックに引っ越し荷物を―・せる」 
\\	音や調子に合わせる。 「ピアノの調べに―・せて歌う」 
\\	物事が順調にいくようにする。 勢いにまかせて進める。 「計画を軌道に―・せる」 
\\	仲間として加える。 参画する。 「その仕事に一口―・せて下さい」 
\\	思惑どおりに相手を動かす。 計略にかける。 だます。 「口車に―・せる」「まんまと―・せられる」 
\\	物事を、ある手段や経路によって運ぶ。 「ニュースを電波に―・せる」「午後の便に―・せる」「販売ルートに―・せる」 
\\	(載せる)新聞・雑誌などの刊行物に掲載する。 また、帳簿などに記載する。 「広告を―・せる」「名簿に―・せる」 [下接句]口に乗せる・口車に乗せる・俎上(そじよう)に載せる・手車に乗せる・俎(まないた)に載せる	▲彼らは彼の名前を名簿に載せた。 ▲彼を病院まで乗せていっていただけないでしょうか。
\\	除く	のぞく	[動カ五(四)] 
\\	取ってなくする。 取りのける。 除去する。 「障害を―・く」「不良品を―・く」「花壇の雑草を―・く」 
\\	その範囲に加えないようにする。 除外する。 「未経験者は―・く」「母を―・いてみんな賛成した」 
\\	邪魔者などを殺す。 「奸臣(かんしん)を―・く」 [可能]のぞける [用法]のぞく・のける――「障害物を除く(のける)」のように、じゃまな物を取りのける意では相通じて用いられる。 
\\	「除く」はその場からなくすという意に重点がある。 「不安を除く」「障害を除く」など、抽象的な事柄については「のける」は使わない。 
\\	「のける」はその場所から他の位置に移すこと。 「ちょっと、そこにある椅子(いす)をのけてください」を「除いて」とは言わない。 
\\	類似の語に「どける」がある。 「どける」は「のける」とほぼ同じだが、「のける」のほうが古くからの言い方で、話し言葉としては「どける」が一般化している。 ただし「ずばりと言ってのける」「見事にやってのける」のような補助動詞としては「どける」は使わない。	▲医者は彼の苦痛を除いてやった。 ▲彼女は日曜を除く毎日、入院中のその老人を訪ねた。
\\	望み	のぞみ	
\\	そうなればよい、そうしたいと思うこと。 願い。 希望。 「―が大きい」「長年の―がかなう」 
\\	望ましい結果を得る可能性。 よいほうに進みそうな見込み。 「助かる―はない」「一縷(いちる)の―がある」 
\\	人望。 名望。 「江湖の―を一身に集める」 
\\	ながめ。 眺望。 「青波に―は絶えぬ」	▲お望みなら教えてあげましょう。 ▲再び彼に会う望みはない。
\\	望む	のぞむ	[動マ五(四)]《「臨む」と同語源》 
\\	はるかに隔てて見る。 遠くを眺めやる。 「富士を―・む展望台」 
\\	㋐物事がこうであればいい、自分としてはこうしたい、こうなりたい、また、なんとか得られないものかなどと、心に思う。 「栄達を―・む」「名声など―・まない」「誰でも幸福を―・んでいる」 ㋑特定の相手に対して、こうあってほしい、こうしてもらいたいと思う。 注文する。 「自重を―・む」「一層の努力が―・まれる」 
\\	自分の所に来てくれるように働きかける。 欲しがる。 「後妻にと―・まれる」 
\\	したう。 仰ぐ。 「徳を―・む」 [可能]のぞめる	▲現在は彼らが望むのは頭の上の屋根(家)だけである。 ▲幸福を望まない人がいようか。
\\	後	のち	
\\	その時のあと。 その事のあと。 「晴れ―曇り」「協議の―結論を出す」 
\\	これから先。 未来。 将来。 「―の時代を担う人」 
\\	死後。 なきあと。 「―の世」 
\\	子孫。 後胤(こういん)。 「元輔(もとすけ)が―といはるる君しもや今宵の歌に外れてはをる」	▲彼が大音楽家になるだろうという母親の予言はのちに現実になった。 ▲晴れ後曇りでした。
\\	ノック	ノック	[名]スル 
\\	打つこと。 たたくこと。 
\\	訪問や入室を知らせる合図に、戸を軽くたたくこと。 「扉を軽く―する」 
\\	野球で、守備練習のため、野手に向けてボールを打つこと。 「シート―」	▲彼はドアをノックして、それから入ってきた。 ▲彼はドアをノックしたが、だれも答えなかった。
\\	伸ばす・延ばす	のばす	[動サ五(四)] 
\\	空間的に長さを増したり、広さを大きくしたりする。 ㋐ある物に続けたり、時間がたったりして長くなる。 「路線を―・す」「ひげを―・す」 ㋑(伸ばす)ゴム・スプリングなどを、ひっぱって長くする。 「ゴムひもを―・す」 ㋒(伸ばす)まがったり、ちぢんだりしているものを、まっすぐにする。 「曲がった釘を―・す」 ㋓(伸ばす)厚みのある物を、押し広げて薄く平たくする。 「餅を―・す」 ㋔(伸ばす)しわなどをなくして、平らにする。 「しわを―・す」 ㋕(伸ばす)他のものを加えて、濃度をうすめる。 「糊(のり)を水で―・す」 
\\	(延ばす)時間や日時をおくらせる。 長びくようにする。 「会期を―・す」「返事を―・す」 
\\	(伸ばす)勢力や能力などを大きくしたり、豊かにしたりする。 「輸出を―・す」「個性を―・す」 
\\	(伸ばす)相手をなぐり倒す。 のす。 「一発で―・されてしまった」 
\\	遠くへ逃がす。 落ちのびさせる。 「いくばくも―・さずして、捕へたる所に」 [可能]のばせる [下接句]足を延ばす・猿臂(えんぴ)を伸ばす・驥足(きそく)を伸ばす・腰を伸ばす・触手を伸ばす・手を延ばす・手足を伸ばす・鼻毛を伸ばす・羽を伸ばす	▲トムは口髭を伸ばしている。 ▲それから、脚を伸ばして、座席にゆったりともたれかかった。
\\	伸びる・延びる	のびる	[動バ上一][文]の・ぶ[バ上二] 
\\	ものの長さ・高さ・広がりが増す。 ㋐(伸びる)生長して、長くなったり、高くなったりする。 「背が―・びる」「すらりと―・びた脚」「枝が―・びる」 ㋑(延びる)もとからある物につなげられて長くなる。 「バス路線が郊外まで―・びる」 ㋒(伸びる)ある段階・範囲にまでおよぶ。 達する。 「司直の手が―・びる」「つい菓子に手が―・びてしまう」 ㋓(伸びる)まがったり、ちぢんだりしていたものがまっすぐになったり、ひろがったりする。 「腰が―・びる」「しわが―・びる」 ㋔(伸びる)全体にうすく、均質にひろがる。 「よく―・びる塗料」 ㋕(伸びる)時間がたって、長い状態になったり、しまりなくやわらかくなったりして弾力がなくなる。 「そばが―・びる」 ㋖(伸びる)さんざんなぐられて動けなくなってしまう。 ひどく疲れてぐったりする。 グロッキーになる。 「過労で―・びる」 
\\	(伸びる)勢いや力が増す。 ㋐大きく盛んになる。 発展する。 「輸出が―・びる」「会社は順調に―・びている」 ㋑能力などがつき、向上する。 「記録が―・びる」「学力が―・びる」 
\\	(延びる) ㋐時間的に、長くひろがる。 「日脚(ひあし)が―・びる」 ㋑命などが長く保つ。 長生きする。 「平均寿命が―・びる」 ㋒期日・時刻が、決められたその時よりおくれる。 「事故で出発が―・びる」 
\\	危機をのがれて、かろうじて命を長らえる。 「今まで御命の―・びさせ給ひて候ふこそ不思議に覚え候へ」 
\\	緊張がとける。 のんびりとする。 くつろぐ。 「空もうららかにて人の心も―・び」 
\\	女に夢中になって本心を失う。 「ひったり抱き寄せしみじみ囁く…(与兵衛ハ)忝いと―・びた顔付き」 [類語]
\\	㋐)伸長する・生長する・成長する/
\\	伸張する・伸展する・発展する・躍進する・拡大する・増大する・向上する/
\\	㋒)延引する・遅延する・遷延する・順延する・繰り下がる	▲この道路は海岸まで伸びている。 ▲貯蔵庫のドアのそばの刈り残した伸びた芝はどうなんだい。
\\	述べる・宣べる・陳べる	のべる	[動バ下一][文]の・ぶ[バ下二]《「伸べる」と同語源》 
\\	考え・意見などを口に出して言う。 「所信を―・べる」「礼を―・べる」 
\\	文章で表す。 「前章に―・べたごとく」	▲マイクはほのめかすつもりで2、3の言葉を述べた。 ▲まず第一にここにお招きいただいてとても嬉しく思っている事を述べます。
\\	のんびり	のんびり	[副]スルゆったりとしてくつろいださま。 のびのび。 「―(と)湯につかる」「―した性格」→ゆっくり[用法]	▲私は働くよりものんびりするほうが好きだ。 ▲今日はビーチでのんびりしよう。
\\	場	ば	㊀[名] 
\\	物や身を置く所。 場所。 「足の踏み―がない」 
\\	ある事が行われる所。 「仕事の―」「―を外す」「その―に居合わせる」 
\\	ある事が行われている所の状況。 また、その雰囲気。 「その―でとっさに答える」「―が白ける」「―をもたせる」 
\\	機会。 折り。 「話し合いの―をもつ」「―を踏んでなれている」 
\\	芝居や映画などの場面。 特に舞台で、一幕のうち、舞台情景を変化させず、同じ場面で終始する一区切りの部分。 「源氏店(げんじだな)の―」 
\\	花札やトランプなどのゲームで、札を積んだり並べたりしてゲームを進めていく所。 また、マージャンで、東西南北の局面。 
\\	取引所内の売買をする所。 立会場。 「―が立つ」 
\\	ゲシュタルト心理学で、行動や反応のしかたに直接影響し関係する環境や条件。 「―の心理学」 
\\	物理学で、そのものの力が周囲に及んでいると考えられる空間。 電磁場・重力場など。 ㊁〔接尾〕助数詞。 演劇で、一幕のうちで一区切りのつく場面を数えるのに用いる。 「二幕三―」	▲しかし今日では、公共の場での喫煙は禁止されたり、きびしく制限されている。 ▲この場をお借りして一言挨拶を申し上げます。
\\	はあ	はあ	[感] 
\\	ややかしこまって応答するとき、また、相手の言葉に同調するときに用いる語。 「―、わかりました」「―、それは大変でしたね」 
\\	返答に困ったとき、また、相手の言葉に同調できかねるときに用いる語。 「―、それはそうですが」「―、そうおっしゃられても」 
\\	ややかしこまって、聞き返すときに用いる語。 は。 「―、なんでしょうか」 
\\	大声で笑う声。 あはは。 「あまた声して―と笑ひて」 
\\	失敗などをして困っているときに発する声。 しまった。 「―、ひきさいた」	▲・・・はぁ、どう状況を受け止めればいいのかしら。
\\	パーセント	パーセント	一〇〇分の幾つであるかを表す語。 一パーセントは一〇〇分の一。 記号%。 百分率。 プロセント。	▲天才は1パーセントが霊感であり、99パーセントは精進である。 ▲南部の売り上げは47パーセント増加した。
\\	灰	はい	物が燃え尽きたあとに残る粉末状のもの。 「かまどの―」「死の―」	▲暖炉の灰を掃除して下さい。 ▲大火事のために町全体が灰になってしまった。
\\	バイオリン	バイオリン	擦弦楽器の一。 全長約六〇センチで、四本の弦を張り、五度間隔に調弦して、馬の尾毛を張った弓でこすって奏する。 提琴。 ビオロン。	
\\	ハイキング	ハイキング	自然を楽しみながら野山などを歩くこと。 ハイク。 「―コース」	▲私達は来週ハイキングに行くのを楽しみにしています。 ▲私はハイキングに参加した。
\\	配達	はいたつ	[名]スル郵便物や商品などを指定された宛先へ届けること。 「市内は無料で―する」「新聞―」	▲ここでお買い上げの品は無料で配達します。 ▲ここへは手紙は正午頃配達される。
\\	パイプ	パイプ	
\\	液体・気体などを通すための管。 
\\	から転じて》二者の間をとりもつ人や組織。 「交渉の―役」 
\\	西洋風の喫煙具。 キセル状の刻みタバコ用と、巻きタバコの吸い口用とがある。 
\\	管楽器。 また、その管。	▲私が覚えている限りでは、彼はいつもパイプを吹かしていた。 ▲パイプが見つからないよ。
\\	俳優	はいゆう	舞台に立って、また演劇・映画・テレビなどで、演技することを職業としている人。 役者。 「歌舞伎―」「性格―」	▲好きな俳優は誰ですか。 ▲高齢の守衛、トム・スケレトンは、よれよれのアームチェアーに座りこんで、俳優たちが楽屋からの石段を上ってくるのを聞いていた。
\\	パイロット	パイロット	
\\	水先人(みずさきにん)。 水先案内人。 
\\	航空機の操縦士。 
\\	ガスの種火。 
\\	「パイロットランプ」の略。 
\\	試験的に行うもの。 先行するもの。 「―フィルム」「―版」	▲私の父は国内線のパイロットです。 ▲私の父は子供のころ、パイロットになりたいと思っていた。
\\	墓	はか	遺体・遺骨を埋葬した場所。 また、そこに記念のために建てられた建造物。 塚。 「―に詣でる」	▲あのお墓には誰が入っているのだろう。 ▲「ロメオはどこだ」「こちらへ。お墓へ。」。
\\	馬鹿・莫迦	ばか	[名・形動]《梵
\\	の音写。 無知の意》 
\\	知能が劣り愚かなこと。 また、その人や、そのさま。 人をののしっていうときにも用いる。 あほう。 「―なやつ」↔利口。 
\\	社会的な常識にひどく欠けていること。 また、その人。 「役者―」「親―」 
\\	つまらないこと。 無益なこと。 また、そのさま。 「―を言う」「―なまねはよせ」 
\\	度が過ぎること。 程度が並はずれていること。 また、そのさま。 「―に風が強い」「―騒ぎ」「―正直」 
\\	用をなさないこと。 機能が失われること。 また、そのさま。 「蛇口が―で水が漏れる」→馬鹿になる 
\\	◆「馬鹿」は当て字。 [派生]ばかさ[名] [用法]馬鹿・阿呆(あほう)――「馬鹿」のほうが広い地域で使われ、意味の範囲も広い。 「馬鹿に暑い」「馬鹿正直」「このネジは馬鹿になっている」のような意は「阿呆」にはない。 
\\	「馬鹿」は関東で、「阿呆」は関西で多く使われる。 [類語]
\\	阿呆(あほう)・魯鈍(ろどん)・愚鈍・無知・蒙昧(もうまい)・愚昧(ぐまい)・愚蒙(ぐもう)・暗愚・頑愚・愚か・薄のろ・盆暗(ぼんくら)・まぬけ・とんま/
\\	愚か・愚かしい・馬鹿らしい・馬鹿馬鹿しい・阿呆らしい・下らない・愚劣・無思慮(ぶしりよ)・無考え・浅はか・浅薄(せんぱく)・軽はずみ・軽率	▲その少年は馬鹿なまねをした。 ▲その申し出を断るなんて君はばかだ。
\\	博士	はかせ	
\\	学問やその道の知識にくわしい人。 「お天気―」「物知り―」 
\\	学位の「博士(はくし)」の俗称。 「―号」「文学―」 
\\	律令制の官名。 大学寮に明経(みようぎよう)・紀伝(のちに文章(もんじよう))・明法(みようぼう)・算・音・書、陰陽(おんよう)寮に陰陽・暦・天文・漏刻、典薬寮に医・呪禁(じゆごん)などの各博士が置かれ、それぞれ専門の学業を教授した。 
\\	(「墨譜」とも書く)声明(しようみよう)の経文の傍らに記して、音の高低・長短を示す記号。 上がり下がりなど音の動きを線で視覚的に表した目安(めやす)博士と、音階(宮・商・角・徴(ち)・羽)を文字または記号で表した五音(ごいん)博士とに大別される。 節博士(ふしはかせ)。	▲私たちは、ブラウン博士を合衆国で最高の心臓病の専門医とみなしている。 ▲今朝ホーキング博士が選んでいた単語はプリントアウトされ、彼ほんの出版社に送られるだろう。
\\	計る・量る・測る・図る・謀る・諮る	はかる	[動ラ五(四)]《「はか」に活用語尾の付いたもの》 
\\	(計る・測る・量る) ㋐ある基準をもとにして物の度合いを調べる。 特に、はかり・ゲージなどの計測機器で測定する。 「体温を―・る」「距離を―・る」 ㋑推しはかって見当をつける。 おもんぱかる。 忖度(そんたく)する。 「己をもって人を―・る」「真意を―・りかねる」 
\\	(図る・謀る) ㋐物事を考え合わせて判断する。 見はからう。 「時期を―・る」「敵情を―・る」 ㋑企てる。 もくろむ。 「自殺を―・る」 
\\	(謀る) ㋐はかりごとをする。 たくらむ。 「悪事を―・る」 ㋑あざむく。 たばかる。 「さては―・られたか」 
\\	(図る) ㋐くふうして努力する。 「再起を―・る」「利益を―・る」 ㋑うまく処理する。 とりはからう。 「便宜を―・る」 
\\	(諮る)相談する。 「会議に―・って決める」 [可能]はかれる [類語]
\\	㋐)測定する・計測する・計量する・秤量(しようりよう)・(ひようりよう)する・実測する・計時する/
\\	㋐)もくろむ・企てる・策する・企図する	▲看護婦が私の体温を測った。 ▲看護婦が彼の体温を計った。
\\	拍手	はくしゅ	[名]スル手を打ち合わせて音を出すこと。 神を拝んだり賞賛・賛成などの気持ちを表して、手をたたくこと。 「―して賛意を表す」	▲その公演は万雷の拍手を受けた。 ▲彼らはわたしたちに拍手をしませんでした。
\\	麦酒	ばくしゅ	麦を原料に醸造した酒。 特に、ビール。	▲ビールはあまりたくさん飲みません。 ▲ビールは麦芽から醸造される。
\\	莫大	ばくだい	[形動][文][ナリ]《これより大なるは莫(な)しの意。 古くは「ばくたい」》程度や数量がきわめて大きいさま。 「被害は―だ」「―な財産」 [派生]ばくだいさ[名] [用法]莫大・多大――「莫大な(多大な)損害をこうむった」のように相通じて用いられる。 
\\	「莫大」は「額」「量」「財産」など数量としてとらえられるものについて言うことが多い。 「莫大な遺産を残す」
\\	「多大」は「多大な金額」のように用いるほか、「成果」「努力」「困難」などはっきり数量化できないことにも使われる。 「多大な(の)恩恵を受けた」「多大な(の)影響を与える」
\\	「莫大」は連体形が「莫大な」となるが、「多大」は「多大の」の形でも使う。 
\\	類似の語に「甚大」がある。 「甚大」はふつう好ましくないことに用いる。 「被害甚大」
\\	三語とも「莫大(多大・甚大)な死者が出た」のような、人数についての使い方はない。	▲彼は息子に莫大な財産を残した。 ▲彼は莫大な遺産を受け継いだ。
\\	爆発	ばくはつ	[名]スル 
\\	物質が急激な化学変化または物理変化を起こし、体積が一瞬に著しく増大して、音や破壊作用を伴う現象。 ガス・粉塵・火薬などの化学的爆発は発熱反応が激しく行われたことにより、ボイラー・火山などの物理的爆発は圧力の激しい発生・解放により起こる。 また核分裂による核爆発がある。 
\\	抑圧されていた感情が急激に外に現れること。 「不満が―する」 [類語]
\\	爆裂・炸裂(さくれつ)・破裂・起爆・誘爆/
\\	激発・暴発・噴出	▲セントヘレンズ山が爆発したのは、ほとんど10年前、つまり1980年の5月8日のことであった。 ▲数回の爆発の衝撃は何マイルにもわたって感じられた。
\\	博物館	はくぶつかん	歴史・芸術・民俗・産業・自然科学などに関する資料を収集・展示して一般公衆の利用に供し、教養に資する事業を行うとともに資料に関する調査・研究を行う施設。	▲その博物館は月曜から金曜まで開いています。 ▲その博物館は今は閉鎖している。
\\	激しい・烈しい・劇しい	はげしい	[形][文]はげ・し[シク] 
\\	勢いがたいへん強い。 「―・い風雨」「気性の―・い人」「―・い反対に遭う」 
\\	程度が度を過ぎてはなはだしい。 ひどい。 「―・い痛みに悩まされる」「暑さが―・い」 
\\	行われる回数が驚くほど多い。 非常に頻繁である。 「変化の―・い社会」「人の出入りが―・い」 
\\	けわしい。 「―・しき道に打ち越えて」 [派生]はげしさ[名] [類語]
\\	凄(すさ)まじい・すごい・ものすごい・荒荒しい・強烈・猛烈・激烈・熾烈(しれつ)・苛烈(かれつ)・激甚・急激・峻烈(しゆんれつ)・激越・矯激(きようげき)・ドラスティック・ファナティック	▲この政治的問題は激しい議論を巻き起こした。 ▲この男たちははげしい仕事に慣れている。
\\	鋏・剪刀	はさみ	《「挟み」と同語源》 
\\	物を二枚の刃で挟んで切る道具。 「裁ち―」「花―」 
\\	(鋏)切符などに穴をあけたり切り込みを入れたりする道具。 パンチ。 
\\	(「螯」 「鉗」とも書く)カニ・エビなどの節足動物の脚の、物をはさむような形に発達した部分。 これをもつ脚を鉗脚(かんきやく)という。 
\\	じゃんけんで、ちょき。 [下接語](ばさみ)金(かな)鋏・紙鋏・空(から)鋏・刈り込み鋏・木鋏・裁ち鋏・爪(つめ)切り鋏・握り鋏・花鋏	▲このはさみは良く切れない。 ▲この紙を切るのにはさみが必要です。
\\	破産	はさん	[名]スル 
\\	財産をすべて失うこと。 「投機に失敗して―する」 
\\	債務者が債務を完済することができなくなった場合に、債務者の総財産をすべての債権者に公平に弁済することを目的とする裁判上の手続き。	▲彼の破産を引き起こしたのは賭事のためでしたか。 ▲彼は一文なしだ(彼は破産した)。
\\	端	はし	《「はじ」とも》 
\\	中央や中心からいちばん離れた部分。 ㋐細長いものの先端や末端。 「棒の―」「行列の―」 ㋑ある場所や空間内の、周辺に近い部分。 「手紙の―に書き添える」「机を部屋の―に寄せる」 
\\	物を切り離したうちの、小さい方。 切れはし。 「革の―」 
\\	物事の、核心から遠い部分。 重要でない部分。 また、全体の中の一部分。 「言葉の―をとらえる」 
\\	物事の初めの部分。 順序の一番目。 「問題を―から解く」「聞いた―から忘れる」 
\\	建築物で、外に面した所。 表に面した部屋・出入り口や廂(ひさし)・縁側・簀子(すのこ)など。 「―に出でゐて、…頭かい梳(けづ)りなどしてをり」 
\\	書物などの冒頭。 また、序文。 「―に書くべきことを奥に書き」 
\\	物事のおこりはじめ。 端緒。 また、とっかかり。 「逢ひ見むと言ひ渡りしは行く末の物思ふことの―にぞありける」 
\\	中途半端の意で、自分の地位・身分などを謙遜していう語。 はしくれ。 「木にもあらず草にもあらぬ竹のよの―に我が身はなりぬべらなり」 
\\	「端女郎(はしじよろう)」の略。 「あの太夫の―へ下りるは間のないこと」 
\\	(「端に」の形で接続助詞的に用いて)…する間に、一方では。 …するかたわらで。 「うち嘆き萎(しな)えうらぶれ偲(しの)ひつつ争ふ―に木の暗(くれ)の四月し立てば」 [類語]
\\	端(はな)・端(はし)っこ・末(すえ)・先(さき)・先っぽ・突先(とつさき)・突端(とつぱな)・先端(せんたん)・突端(とつたん)・末端(まつたん)・一端(いつたん)・両端(りようたん)・尻尾(しつぽ)/
\\	縁(へり)・縁(ふち)・耳・外(はず)れ・隅(すみ)・際(きわ)・端(はた)/
\\	末節(まつせつ)・枝葉(しよう)・細部・一部・一端(いつたん)	▲机の端にヒジ先をぶつけちゃった。 ▲妻は部屋のむこう端から私に合図した。
\\	始まり・初まり	はじまり	
\\	はじまること。 また、はじまった時期。 「授業の―を知らせるベル」「事件の―は一年前だ」 
\\	物事の起こり。 起源。 「近代医学の―」	▲始まりがうまければ半ばうまくいったもおなじ。 ▲熱狂の宴が始まりを告げる。
\\	中古	ちゅうぶる	すでに使用され、少し古くなっていること。 また、その品物。 まだ使用できる物にいう。 セコハン。 ちゅうこ。 「―の車」	▲あのカーデイラーはこの中古のトヨタが調子がいいなどと、まんまと一杯くわせやがった。 ▲なに、中古だよ。
\\	梅雨・黴雨	ばいう	六月上旬から七月上・中旬にかけて、本州以南から朝鮮半島、揚子江流域に顕著に現れる季節的な雨。 梅雨前線上を小低気圧が次々に東進して雨を降らせるもの。 入梅の前に走り梅雨(づゆ)の見られることが多く、中休みには五月晴(さつきば)れとなることもあり、梅雨明けは雷を伴うことが多い。 つゆ。 さみだれ。 《季 夏》「草の戸の開きしままなる―かな/虚子」 ◆梅の実が熟するころに降る雨の意、または、物に黴(かび)が生じるころに降る雨の意か。	▲梅雨に入りましたね。 ▲梅雨に入ってしまったね。
\\	トラック	トラック	貨物運搬用の大型荷台をもつ自動車。 貨物自動車。	▲私の家はトラックが通過する時いつも揺れる。 ▲彼女はかつてトラック競技のスターだった。
\\	トン	トン	【屯・×噸・×瓲】 
\\	質量の単位。 記号t。 ㋐メートル法で、一トンは一〇〇〇キログラム。 仏トン。 グラムトン。 メートルトン。 ㋑英トン。 ヤード‐ポンド法で、一トンは二二四〇ポンド、すなわち約一〇一六キログラム。 ロングトン。 大トン。 ㋒米トン。 ヤード‐ポンド法で、一トンは二〇〇〇ポンド、すなわち九〇七・一八キログラム。 ショートトン。 
\\	船舶の大きさを質量や容積で表す単位。 排水トン数・載貨重量トン数・総トン数・容積トン数・純トン数など。 ◆「瓲」は国字。	
\\	とんとん	とんとん	㊀[副] 
\\	物を続けざまに軽くたたく音を表す語。 「ドアを―(と)ノックする」「はしご段を―(と)上る」 
\\	物事がとどこおることなく進むさま。 「縁談が―(と)まとまる」 ㊁[形動]ふたつのものがほとんど同じで差がないさま。 また、損得のないさま。 「実力は―だ」「収支が―になる」[アクセント] ❶はトントン、 ❷はトントン。	▲彼は窓をとんとんたたいた。 ▲彼は賭けをやってとんとんに終わった。
\\	軒	けん	〔接尾〕 
\\	助数詞。 家屋の数をかぞえるのに用いる。 「三―」「数千―」 
\\	雅号・屋号などの末尾に用いる語。 「桃中―」「精養―」	▲私の家からほど近い場所にある何軒かのホテルの名前と料金設定を同封しました。 ▲医者は午後6軒往診した。
\\	とにかく	とにかく	[副] 
\\	他の事柄は別問題としてという気持ちを表す。 何はともあれ。 いずれにしても。 ともかく。 「―話すだけ話してみよう」「間に合うかどうか、―行ってみよう」 
\\	(「…はとにかく」の形で)上の事柄にはかかわらないという気持ちを表す。 さておき。 ともかく。 「結果は―、努力が大切だ」 ◆「兎に角」とも当てて書く。 [用法]とにかく・なにしろ――「彼はとにかく(なにしろ)まじめな人だから」「このごろ、とにかく(なにしろ)忙しくってね」のように、取り上げた事柄をまず強調しようとする意では相通じて用いられる。 
\\	「時間だから、とにかく出発しよう」「とにかく現場を見てください」のように、細かいことはさて置いて、まず行動をという場合は、「とにかく」しか使えない。 「私はとにかく、あなたまで行くことはない」のような「…は別として」の意の用法も、「とにかく」に限られる。 
\\	「なにしろ」は「なにしろあの人の言うことだから」「なにしろ暑いので」のように「から」「ので」と結び付いて、その事柄を理由・原因として強調する用法が多い。	▲とにかくあなたは10時までにここに来なければなりません。 ▲とにかくあなたは一生懸命勉強しなければならない。
\\	とんでもない	とんでもない	[形]《「とでもない」の音変化》 
\\	思いもかけない。 意外である。 「―・い人にばったり出会う」「―・い発明」 
\\	もってのほかである。 「―・い悪さをする」 
\\	まったくそうではない。 滅相もない。 相手の言葉を強く否定していう。 「―・い、私は無関係だ」	▲私は料理をしますが、とんでもなく下手くそです。 ▲あいつが品行方正だって。とんでもない。裏で何をやってるか知らないからそんなこと言えるんだよ。
\\	どんな	どんな	[形動] 
\\	はっきりしないそのものの状態・性質・程度などを想像しようとするさま。 「―人が来るのだろう」「―にかうれしかったろう」「―ものをお探しですか」 
\\	物事の状態・性質・程度などに左右されないさま。 「―物でも買い取る」「―に悲しくとも泣かない」 ◆連体形に「どんな」「どんなな」の二形がある。 連体形として一般には「どんな」の形が用いられるが、助詞「の」に続くときなどは「どんなな」の形が用いられる。 「通信事情が悪いので、今の状況がどんななのか、よくわからない」	▲他人の意見を押さえつけることはできても、自分の発言がどんな結果に結びつくかを想像できていない。 ▲貸借対照表にはどんなものが含まれるのでしょうか。
\\	パス	パス	[名]スル 
\\	通過すること。 特に、試験や審査などに合格すること。 「試験に―する」 
\\	通行許可証。 出入国許可証や無料入場券・乗車券など。 「顔―」 
\\	サッカー・バスケットボールなどの球技で、ボールを味方に渡すこと。 送球。 「前衛に―する」「チェスト―」 
\\	トランプ遊びで、自分の番を飛ばして次へ回すこと。 転じて、一般に、自分の順番を回避すること。 
\\	フェンシングで、突きのこと。	▲彼がもっと熱心に勉強していたら試験にパスしただろう。 ▲彼はきっと試験にパスすると思う。
\\	外す	はずす	[動サ五(四)] 
\\	取り付けたり、掛けたりしていたものを取って離す。 「錠を―・す」「受話器を―・す」 
\\	身につけていたものをとる。 「ネクタイを―・す」「胸当てを―・す」 
\\	取り逃がす。 機会などを失う。 逸する。 「絶好のチャンスを―・す」 
\\	一時的に、また途中である場所から退いたり離れたりする。 「席を―・す」「バッターボックスを―・す」 
\\	ある集団・任務や予定などから除く。 「メンバーから―・す」「担当を―・される」 
\\	相手のものをそらして避ける。 かわす。 「攻撃を―・す」「質問を巧みに―・す」「タイミングを―・したスローボール」 
\\	ねらいをそらす。 「コースを―・す」「的を―・す」 [可能]はずせる	▲彼は眼鏡を外した。 ▲的を外してしまった。
\\	パスポート	パスポート	旅券。	▲パスポートがなければ、出国など論外だ。 ▲パスポートが見つからなかった時、とてもあせった。
\\	旗・幡・旌	はた	
\\	布や紙などで作り、高く掲げて標識や装飾にするもの。 大きさ・形・色・図案は種々で、ふつう一端をさおの端や綱などに結びつける。 古くは、朝廷の儀式や祭礼の飾り、また、軍陣の標識として用いた。 近世は、布帛(ふはく)の側面に乳(ち)をつけてさおに通した幟(のぼり)がある。 
\\	(幡)⇒ばん(幡) [下接語]赤旗・白旗・錦(にしき)の御(み)旗・一(ひと)旗(ばた)小旗・指し小旗・背旗・大漁旗・手旗・幟(のぼり)旗・筵(むしろ)旗	▲その棒のてっぺんには旗がついていた。 ▲ネッドは旗をまっすぐに立てて持ってきた。
\\	肌・膚	はだ	
\\	人のからだを覆う表皮。 皮膚。 はだえ。 「―を刺す寒気」 
\\	物の表皮。 「木の―」「山―」 
\\	気質。 気性。 「研究者―」 [下接語]青肌・赤肌・荒れ肌・勇み肌・岩肌・絵肌・鏡肌・片肌・競(きお)い肌・木肌・黄(き)肌・鮫(さめ)肌・地肌・白(しら)肌・素肌・玉の肌・鉄火肌・伝法肌・鳥肌・新(にい)肌・如来(によらい)肌・羽二重肌・一(ひと)肌・人肌・美(び)肌・槙(まい)肌・饅頭(まんじゆう)肌・餅(もち)肌・諸(もろ)肌・山肌・柔(やわ)肌・雪肌	▲私の肌は日焼けしやすい。 ▲私は浜辺で肌を焼いた。
\\	裸	はだか	
\\	身に衣類を着けていないこと。 また、そのからだ。 「上半身―になる」《季 夏》 
\\	覆いや飾りがなく、むき出しであること。 「心付けを―で渡す」 
\\	包み隠しのないこと。 「―のつきあい」 
\\	財産・所持品などが全くないこと。 無一物。 無一文。 「事業に失敗して―になる」 [類語]
\\	裸体(らたい)・裸身(らしん)・裸形(らぎよう)・全裸(ぜんら)・赤裸(せきら)・真っ裸・素っ裸・ヌード/
\\	剥(む)き出し・丸出し・露出・裸出(らしゆつ)/
\\	無一文(むいちもん)・無一物(むいちもつ)・身(み)一つ・すってんてん	▲子供たちは裸で泳いでいた。 ▲上半身裸になってください。
\\	畑・畠	はたけ	
\\	野菜や穀類などを作る農耕地。 水田に対して、水を入れない耕地をいう。 はた。 
\\	専門とする領域・分野。 「法律―」 
\\	母親の腹。 また、出生地。 「―の違う兄弟」 ◆「畑」「畠」は国字。 [下接語](ばたけ)梅畑・お花畑・桑畑・段段畑・茶畑・花畑・麦畑	▲春になると畑をほりかえして種をまきます。 ▲私たちは畑全体に種をまいた。
\\	畑地	はたち	畑として利用されている土地。	
\\	働き	はたらき	
\\	仕事をすること。 労働すること。 「―に出る」 
\\	活躍すること。 また、その実績。 仕事の成果。 「―に応じた給料を支払う」「メンバーの―が功を奏する」「―が認められて昇進する」 
\\	稼ぎ。 収入。 また、それを得る才覚。 「―が悪い」「今月は―が少ない」「―のある人」 
\\	物事に備わっている機能。 また、その及ぼす作用。 「頭の―が鈍る」「遠心力の―を応用する」「ビタミンの―」 
\\	能・狂言の用語。 ㋐「働き事」の略。 ㋑「舞働(まいばたら)き」に同じ。 
\\	日本語文法での活用。 [下接語](ばたらき)荒(あら)働き・一時働き・気働き・下働き・節句働き・只(ただ)働き・共働き・仲働き・糠(ぬか)働き・舞働き・夜(よ)働き	▲きっと働きすぎですよ。 ▲ジョン君は働きすぎだよ。座ってしばらくはのん気にかまえなさい。
\\	バッグ	バッグ	物を入れて持ち歩く袋やかばんの総称。 「ハンド―」「ショルダー―」「ショッピング―」	▲このバッグは素晴らしくて、しかも安い。 ▲このバッグは皮でできています。
\\	発見	はっけん	[名]スルまだ知られていなかったものを見つけ出すこと。 また、わからなかった存在を見いだすこと。 「新大陸の―」「犯人のアジトを―する」	▲彼女は連続して医学的発見をした。 ▲彼らは新しい方法を発見した。
\\	発行	はっこう	[名]スル 
\\	図書・新聞・雑誌などを印刷して、世に出すこと。 「週刊誌を―する」 
\\	紙幣・債券・証明書・定期券・入場券などを作って、通用させること。 「旅券を―する」 
\\	(「発向」とも書く)流行すること。 「その時分は、塩浜が―しやした」 [類語]
\\	刊行・発刊・公刊・刊(かん)・出版・版行(はんこう)・印行・発兌(はつだ)/
\\	発券・発給・交付	▲その雑誌は月に二度発行されている。 ▲教育のメルマガを発行したい方はこちら!
\\	発車	はっしゃ	[名]スル汽車・電車・バスなどが出発すること。 「定時に―する」	▲バスは五分後に発車です。 ▲列車は今日の午後1時半に発車します。
\\	罰する	ばっする	[動サ変][文]ばっ・す[サ変]罰を与える。 処罰する。 「違反者を―・する」	▲その強盗は罰せられることを免れた。 ▲その生徒はタバコを吸ったために罰せられた。
\\	発達	はったつ	[名]スル 
\\	からだ・精神などが成長して、より完全な形態や機能をもつようになること。 「心身が―する」 
\\	そのものの機能がより高度に発揮されるようになること。 「文明が―する」「交通機関が―する」 
\\	そのものの規模がしだいに大きくなること。 「―した低気圧」 [用法]発達・発展――「最近急速に発達(発展)した都市」「文化の発達(発展)」など、規模が大きくなったり、高い程度に進んだりする意では相通じて用いられる。 
\\	「発達」は成長してより完成した状態に近づく意。 「心身の発達」「嗅覚(きゆうかく)の発達した犬」「発達した低気圧」など、生物の身体や器官の機能、自然現象については「発達」を使う。 
\\	「発展」は物事の勢いや力などが増し広がっていく意。 「会社の発展に尽力する」「御発展を祈ります」「事件は意外な方向に発展した」などでは「発展」を使う。 
\\	類似の語「進歩」は、すぐれた段階、状態になること。 「学業に進歩の跡がある」「めざましい科学の進歩」のように質に重点があり、「進歩発展する」と重ねて使うこともある。	▲今の子供は、発達中の未来の大人である。 ▲今では交通機関が発達したため、歩く人が少ないのは遺憾である。
\\	ばったり	ばったり	[副] 
\\	勢いよく倒れたり落ちたりするさま。 ばたり。 「疲れきって―(と)倒れる」 
\\	思いがけなく人に出会うさま。 「旧友と―(と)顔を合わせる」 
\\	続いていた物事が急に途絶えるさま。 「人通りが―(と)とぎれる」	▲昨夜劇場である友人にばったり会った。 ▲私は劇場で先生にばったり会った。
\\	発展	はってん	[名]スル 
\\	物事の勢いなどが伸び広がって盛んになること。 物事が、より進んだ段階に移っていくこと。 「経済が―する」「大事件に―する」 
\\	異性関係や遊蕩について、活動の範囲を広めること。 「だいぶご―のようですね」→発達[用法] [類語]
\\	発達・伸展・伸張・成長・興隆・隆盛・躍進・飛躍・展開・進展・拡大	▲これら最後の英語はそれぞれの話し手の必要に応じて独自の路線で発展しつづけるだろう。 ▲コンピューター産業の発展は非常に急速である。
\\	発表	はっぴょう	[名]スル世間一般に知らせること。 表向きに知らせること。 また、その知らせ。 「新製品を―する」「小説を―する」「研究―」	▲ある航空会社が運賃値下げ計画を発表すると、他社もすぐに追従した。 ▲ジェーンはオフィーリアを演じると発表された。
\\	発明	はつめい	㊀[名]スル 
\\	今までなかったものを新たに考え出すこと。 特に、新しい器具・機械・装置、また技術・方法などを考案すること。 「必要は―の母」「蒸気機関を―する」 
\\	物事の道理や意味を明らかにすること。 明らかに悟ること。 「文明の進歩は…其働の趣を詮索して真実を―するに在り」 ㊁[名・形動]賢いこと。 また、そのさま。 利発。 「息子たちのなかで際立って―なのと」 [類語] 
\\	考案・案出・創案・発案・新案・創造・独創/ ❷利口・利発・聡明(そうめい)	▲コンピューターは最近の発明です。 ▲そのようにして彼はその機械を発明したのです。
\\	話し合う	はなしあう	[動ワ五(ハ四)] 
\\	互いに話す。 打ち解けて話す。 語り合う。 「楽しく―・う」 
\\	問題を解決するために、立場・考えなどを述べ合う。 「とことんまで―・って決める」	▲私は上司とその問題について話し合います。 ▲私は彼とその新しい計画について話し合わなければならない。
\\	離す	はなす	[動サ五(四)] 
\\	くっついているものを解き分ける。 「付箋(ふせん)を―・す」「魚の身を骨から―・す」 
\\	他のものとの間を隔てる。 その位置から遠ざける。 「席を―・す」「少し―・して植える」「二人の仲を―・す」 
\\	二つのものの間に隔たりをつくる。 間に距離を置く。 「二位以下を大きく―・す」「業績で同期生に―・される」 
\\	(「目をはなす」の形で)視線を別の所に移す。 「いたずらっ子なので目が―・せない」 
\\	除く。 はずす。 「コレヲ―・シテワホカニナイ」 [可能]はなせる [類語]
\\	外す・分ける・分かつ・切り離す・分離する/
\\	隔てる・遠ざける・隔離する・離隔する	▲油を火からはなしておきなさい。 ▲木は3メートル離して植えられている。
\\	放す	はなす	[動サ五(四)]《「離す」と同語源》 
\\	捕らえられたりつながれたりしている動物などを自由にしてやる。 「魚を川に―・す」「猟犬を―・す」 
\\	握ったりつかんだりしていたのをやめる。 「母の手を―・す」「ハンドルから手を―・す」 
\\	手元から遠くにやる。 手放す。 「刀を―・したことがない」 
\\	矢や弾丸を発射する。 はなつ。 「近くで―・しただけに狙いも狂わず」 
\\	料理で、水や汁などに入れて散らす。 「ナスを水に―・してあくを抜く」 
\\	(他の動詞の連用形に付いて) ㋐あることをしたままほうっておく。 「見―・す」 ㋑ある状態を続ける。 前に促音が挿入されて「ぱなす」の形をとることが多い。 「勝ちっ―・す」 [可能]はなせる	▲放してくれ、息が詰まる。 ▲僕がいいと言うまでロープを放さないで。
\\	離れる	はなれる	[動ラ下一][文]はな・る[ラ下二] 
\\	くっついているものの一方が動いて別々になる。 「足が地を―・れる」「雀が電線から―・れる」 
\\	あるものとの間に隔たりができる。 その位置から遠ざかる。 「一メートル―・れて立つ」「少し―・れて絵を鑑賞する」「持ち場を―・れる」「故郷を―・れる」 
\\	㋐二つのものが隔たって存在する。 間にかなりの距離がある。 「家と学校とはだいぶ―・れている」「人里―・れた一軒家」 ㋑二つの数値・地位などに隔たりがある。 「年の―・れた弟」「トップと大きく―・れる」 ㋒夫婦・親子など、一つであるものが別々になる。 別れる。 「家族が―・れて暮らす」 
\\	㋐関係がなくなる。 縁が切れる。 「俗世を―・れる」「損得を―・れて面倒をみる」「話が本筋から―・れる」「―・れられない仲」 ㋑信頼や情愛を失う。 「人心が―・れる」「恋人から気持ちが―・れる」 ㋒ある事柄に対する思いがなくなる。 「仕事のことが頭から―・れない」 
\\	職務や仕事をやめる。 「戦列を―・れる」「会長職を―・れる」 
\\	戸などが開いた状態になる。 あく。 「格子を探り給へば、―・れたる所もありけり」 
\\	除かれる。 除外する。 「琴の音を―・れては、何事をか物をととのへ知るしるべとはせむ」 [類語]
\\	外(はず)れる・分かれる・分離する・離脱する/
\\	隔たる・遠ざかる・遠のく・離隔する・隔絶する・遊離する・乖離(かいり)する・去る・出る・空(あ)ける・外す・後(あと)にする/
\\	㋒)別れる・離別する/
\\	離反する・離背する・絶縁する・訣別(けつべつ)する・おさらばする・袂(たもと)を分かつ/
\\	退(しりぞ)く・退(の)く・辞(や)める・抜ける・離脱する・脱退する・引退する・辞任する・離任する・手を引く	▲その町は海岸から2マイル離れたところにある。 ▲その島は海岸から約2マイル離れた沖にあります。
\\	羽・羽根	はね	
\\	鳥の全身を覆う羽毛。 
\\	㋐鳥が空を飛ぶための器官。 翼。 「―を畳む」 ㋑(「翅」とも書く)昆虫の飛ぶための器官。 ㋒翼状のもの。 特に、器具・機械に取り付けた翼状のもの。 「飛行機の―」「外套の―を後ろに反(は)ねて」 
\\	矢につけた羽毛。 やばね。 
\\	(羽根)ムクロジの実に穴をあけ、数本の羽をさしたもの。 羽子板でこれをついて遊ぶ。 はご。 「正月に―をつく」《季 新年》「大空に―の白妙とどまれり/虚子」 
\\	バドミントンで用いるシャトルコック。 
\\	水車・タービンなどで回転体の周囲に取り付けた金属片。 「プロペラの―」「扇風機の―」 
\\	紋所の名。 
\\	の形を組み合わせて図案化したもの。 [下接語]赤い羽根・緑の羽根 (ばね)案内羽根・追い羽根・大羽・尾羽・風(かざ)切り羽・切り羽子(ばね)・固定羽根・小羽根・衝(つく)羽根・手羽・撚翅(ねじればね)・矢羽根・遣(や)り羽根・綿羽	▲くじゃくは羽は美しいが足は汚い。 ▲そこに羽の生えたばかりの小鳥がありました。
\\	幅・巾	はば	
\\	物の横の端から端までの距離。 また、長方形の短いほうの辺の長さ。 差し渡し。 「―の広い肩」「机の―」「道路の―」 
\\	声・価格などの高低の隔たり。 「―のある声」「値上げの―」 
\\	制約の中で自由にできる、ゆとり。 「規則に―をもたせる」 
\\	心の広さ。 ゆとり。 「―のある人間」 
\\	その領域で実力があり、発言力が大きいこと。 はぶり。 威勢。 「主が時の―に任せて、佐々木殿の御意を聞かず」 [下接語]後ろ幅・大幅・肩幅・川幅・小幅・シングル幅・背幅・袖(そで)幅・ダブル幅・中(ちゆう)幅・並幅・値幅・半幅・広幅・歩幅・前幅・丸幅・道幅・身幅・木綿幅・ヤール幅・横幅・利幅	▲その川は幅35メートルです。 ▲その川は幅が50メートルある。
\\	母親	ははおや	母である親。 女親。 はは。 ↔父親。	▲彼女は私には母親みたいなものです。 ▲彼女は顔立ちが母親と似ている。
\\	省く	はぶく	[動カ五(四)] 
\\	不要のものとして取り除く。 「説明を―・く」 
\\	全体から一部を取り除く。 減らす。 また、節約する。 「手間を―・く」「時間を―・く」 
\\	分け与える。 「貧しき民に財を―・き」 [可能]はぶける	▲ワープロを使えば、ずいぶん手間がはぶけますよ。 ▲この機会のおかげで私たちは大いに手間暇が省ける。
\\	場面	ばめん	
\\	変化する状況の、ある部分。 物事が行われているその場のようす。 「苦しい―に直面する」 
\\	演劇・映画などの一情景。 シーン。 「―が変わる」「出会いの―」 
\\	市場の状況。 場況。	▲そのショーの他の場面は特にきわだったものではなかった。 ▲スローモーションでその場面を見せた。
\\	流行	りゅうこう	[名]スル 
\\	世間に広く行われ、用いられること。 服装・言葉・思想など、ある様式や風俗が一時的にもてはやされ、世間に広まること。 はやり。 「ミニスカートが―する」「―を追う」「―遅れ」 
\\	病気などが、急速な勢いで世の中に広がること。 「はしかが―する」 
\\	蕉風俳諧で、句の姿が、その時々を反映して変化していくもの。 →不易流行(ふえきりゆうこう) [類語]
\\	はやり・時好・好尚・時流・風潮・トレンド・モード・ファッション・ブーム/
\\	蔓延(まんえん)・猖獗(しようけつ)	▲流行に付いて行くことはやめた。 ▲あの種の服が今流行だ。
\\	腹・肚	はら	㊀[名] 
\\	動物の、胸部と尾部との間の部分。 胴の後半部。 また、背に対して、地に面する側。 人間では、胸から腰の間で中央にへそがある前面の部分。 横隔膜と骨盤の間で、胃腸のある部分。 腹部。 「魚の―を割く」「中年になって―が出てきた」「―の底から声を出す」 
\\	胃腸。 「食べ過ぎて―にもたれる」 
\\	(「胎」とも書く)母親が子を宿すところ。 母の胎内。 また、そこから生まれること。 「子が―にある」「同じ―から生まれる」 
\\	(「胆」とも書く) ㋐考えていること。 心中。 本心。 また、心づもり。 「口は悪いが、―はそれほどでもない」「自分一人の―に納めておく」「折をみて逃げ出す―らしい」 ㋑胆力。 気力。 また、度量。 「―の大きい、なかなかの人物」「少しくらいのミスを許す―がなくては勤まらない」 
\\	感情。 気持ち。 「これでは―が収まらない」 
\\	物の中ほどの広い部分。 また、ふくらんだ部分。 「徳利(とつくり)の―」「転覆した船が―を見せる」 
\\	背に対して、物の内側の部分。 「親指の―でつぼを押す」 
\\	定常波で振幅が最大となるところ。 ↔節(ふし)。 ㊁〔接尾〕助数詞。 魚の卵巣、特に食用のはららごを数えるのに用いる。 「たらこ二(ふた)―」 [下接語]赤腹・朝腹・裏腹・片腹・業(ごう)腹・下腹・白腹・空き腹 (ばら)後(あと)腹・追い腹・扇腹・男腹・女腹・陰腹・亀(かめ)腹・粥(かゆ)腹・下り腹・小腹・先腹・里腹・地腹・自腹・渋り腹・蛇(じや)腹・皺(しわ)腹・太鼓腹・茶腹・詰め腹・冷え腹・脾(ひ)腹・太(ふと)腹・船(ふな)腹・布袋(ほてい)腹・負け腹・水腹・虫腹・むしゃくしゃ腹・無駄腹・餅(もち)腹・自棄(やけ)腹・雪腹・湯腹・横腹・脇(わき)腹 (ぱら)朝っ腹・金(きん)腹・銀腹・空きっ腹・中(ちゆう)っ腹・土手っ腹・太っ腹・向かっ腹・自棄(やけ)っ腹	▲ジョージは彼の腹を殴った。 ▲もう私の腹は決まっている。
\\	原	はら	草などが生えた、平らで広い土地。 野原。 原っぱ。	▲私達がよく野球をして遊んだ原は今すっかり家が建て込んでしまった。
\\	針	はり	
\\	布などを縫う、細くて先のとがった金属製の道具。 一方のはしに糸を通す穴(めど)がある。 縫い針。 また、布を刺して留めるための穴のない留め針・待ち針もある。 
\\	形や用途が 
\\	に似ているもの。 ㋐サソリ・ハチなどのもつ、他の動物に刺して毒を注入する器官。 ㋑注射器の先端につけ、皮膚などに刺して薬剤を注入する器具。 注射針。 ㋒レコードの盤面の溝をなぞり、振動をひろい伝えるもの。 レコード針。 ㋓編み物に用いる棒針の類。 ㋔書類などをとじるための金具。 「ホッチキスの―」 ㋕植物のとげ。 「枳殻(からたち)の生垣のすき間もなく―を立てて」 
\\	時計・計器の目盛りを指し示すもの。 「磁石の―」 
\\	裁縫。 おはり。 「―を習う」 
\\	感情を刺激すること。 害意。 「言葉に―を含む」 
\\	助数詞的に用いて、針で縫った目数を数えるのに用いる。 「傷口を五―縫う」 [下接語]御(お)針・縫い針 (ばり)網針・編み針・置き針・返し針・鉤(かぎ)針・掛け針・蚊針・革針・擬餌(ぎじ)針・絹針・絎(くけ)針・毛針・小町針・仕付け針・千人針・空(そら)針・畳針・釣り針・綴(と)じ針・留め針・縫い針・刃針・平針・棒針・待ち針・メリケン針・木綿針	▲彼は針で指を刺した。 ▲何回か当たりはあったが針にはかからなかった。
\\	バン	バン	後部に荷物を積めるようにした屋根付きの箱型の自動車。 「ライト―」	▲二台のバンが激突してめり込んだ。
\\	範囲	はんい	ある一定の限られた広がり。 ある区域。 「広い―に渡る」「できる―で協力する」	▲君は自分の収入の範囲内で生活するように、余分な支出は切りつめなければならない。 ▲我々の保険の範囲は多様な損害に及びます。
\\	反抗	はんこう	[名]スルさからうこと。 長上や権威・権力などに従わないこと。 「親に―する」「体制に―する」→抵抗[用法]	▲このごろ彼は親に反抗する。 ▲その学生達は政府に対して反抗した。
\\	犯罪	はんざい	
\\	罪をおかすこと。 また、おかした罪。 「―を防ぐ」「完全―」 
\\	刑法その他の刑罰法規に規定する犯罪構成要件に該当する有責かつ違法な行為。	▲彼は犯罪に関して無罪だった。 ▲彼は重大な犯罪を犯した。
\\	ハンサム	ハンサム	[形動]男性の顔だちや風采(ふうさい)のよいさま。 「―な少年」	▲父は、私がハンサムなのを自慢にしている。 ▲父は、私が背が高くてハンサムなのを自慢している。
\\	判断	はんだん	[名]スル 
\\	物事の真偽・善悪などを見極め、それについて自分の考えを定めること。 「適切な―を下す」「なかなか―がつかない」「君の―にまかせる」「状況を―する」 
\\	吉凶を見分けること。 占い。 「姓名―」 
\\	(ドイツ)
\\	論理学で、ある対象について何事かを断定する思考作用。 また、その言語表現。 普通は
\\	はpである」
\\	はpでない」という形式をとる。 [類語]
\\	判定・判別・推断・論断・断定・明断・結論・考え・見立て (―する)判ずる・見極める・見定める・見る	▲これから判断すると。 ▲ご判断を任せます。
\\	犯人	はんにん	罪を犯した人。 犯罪人。	▲彼はまるで犯人を知っているかのような話しぶりだ。 ▲犯人を弁護してた弁護士が実は真犯人だったなんて、前代未聞だ。
\\	販売	はんばい	[名]スル商品を売ること。 「輸入雑貨を―する」「通信―」	▲我が社の販売活動は大いに成功しています。 ▲会社はニューヨークにあるアメリカ販売子会社を閉鎖する計画だ。
\\	被害	ひがい	損害や危害を受けること。 また、受けた損害や危害。 「台風の―をまぬがれる」「―にあう」↔加害。 [類語]損害・損亡(そんもう)・損失・実損・不利益・実害・害・禍害・惨害・惨禍・災禍・災害・難(なん)・被災	▲大雨は洪水となって被害を与えた。 ▲台風は作物に大きな被害を与えた。
\\	比較	ひかく	[名]スル 
\\	二つ以上のものを互いにくらべ合わせること。 「優等生の兄といつも―される」 
\\	(「比較にならない」の形で)くらべるに価する対象。 「まるで―にならない得票差」	▲私たちはよく日本と英国を比較する。 ▲私のコレクションは彼のものと比較すれば取るに足らない。
\\	轢く	ひく	〔「引く」と同源〕 人や物などを車輪の下に踏みつけて通り過ぎる。 「車が歩行者を―・いた」	▲そのかわいそうなネコはトラックにひかれた。 ▲そのトラックが少年をひいた。
\\	ピクニック	ピクニック	野山に出かけて遊んだり食事をしたりすること。 野遊び。 遠足。	▲どこにピクニックに行くかを私達が話し合った時、森に行きたがる人もいれば、海に行きたがる人もいた。 ▲とてもいい天気だったので、私たちはピクニックに行くことに決めた。
\\	悲劇	ひげき	
\\	主人公が運命や社会の圧力、人間関係などによって困難な状況や立場に追い込まれ、不幸な結末に至る劇。 トラジェディー。 ↔喜劇。 
\\	人生や社会の痛ましい出来事。 「貧困がもたらした―」↔喜劇。	▲悲劇は突然起こった。 ▲悲劇の本質は、短編小説のそれと同じように、その葛藤である。
\\	飛行	ひこう	[名]スル 
\\	空中を飛んでいくこと。 「山岳地帯の上空を―する」「太平洋横断―」 
\\	⇒ひぎょう(飛行)	▲ボーイング社の安全担当の専門家は航空産業の他の専門家と一緒になって制御飛行中の墜落
\\	として知られている墜落事故をなくそうと国際的な対策委員会を組織している。 ▲リンドバーグが幸運に恵まれず、飛行機の操縦に明るくなかったならば、かれは大西洋横断飛行に成功することはできなかっただろう。
\\	膝	ひざ	
\\	ももとすねとの境の関節部の前面。 ひざがしら。 「―をすりむく」「―まで水につかる」 
\\	座ったときの、ももの上側にあたる部分。 「荷物を―にのせる」 [下接語]片膝・小膝・擦(す)り膝・立て膝・突き膝・回り膝・諸(もろ)膝・両膝・割り膝	▲子供が母のひざのうえで眠っていた。 ▲私は膝の長さのスカートは好きです。
\\	非常	ひじょう	㊀[名]普通でない差し迫った状態。 また、思いがけない変事。 緊急事態。 「―を告げる電話の声」「―持ち出しの荷物」 ㊁[形動][文][ナリ] 
\\	並の程度でないさま。 はなはだしいさま。 「―に悲しい」「―な才能」 
\\	行動やようすが異常であるさま。 「―な事だと思わないで―なことをするから奇人だろう」 [類語] ❶緊急・急難・異変・事変・変事/ 
\\	大変・大層・異常・極度・桁(けた)外れ・桁違い・並み外れ・格段・著しい・甚だしい・すごい・ものすごい・計り知れない・恐ろしい・ひどい・えらい・途方もない・途轍(とてつ)もない・この上ない・筆舌(ひつぜつ)に尽くしがたい・言語に絶する・並並(なみなみ)ならぬ(連用修飾語として)極めて・至って・甚だ・極(ごく)・至極(しごく)・滅法(めつぽう)・すこぶる・いとも・とても・大いに・実に・まことに・一方(ひとかた)ならず	▲非常に高い収入を得るチャンスがあるビジネス、月額100万円可能!! ▲私は美術に非常な関心を抱いています。
\\	美人	びじん	
\\	容姿の美しい女性。 美女。 
\\	容姿の美しい男子。 「玉のやうなる―…聟にいたします」 [類語]
\\	別嬪(べつぴん)・美女・麗人・佳人(かじん)・美形・美姫(びき)・尤物(ゆうぶつ)・名花・解語の花・シャン	▲彼女は母親と同じぐらい美人だ。 ▲彼女は母とまったく同じほど美人だ。
\\	額	ひたい	
\\	顔の上部の、髪の生えぎわと眉(まゆ)との間の部分。 おでこ。 
\\	冠・烏帽子(えぼし)などの前額に当たる部分。 厚額(あつびたい)・薄額(うすびたい)・透き額などがある。 
\\	(「蔽髪」と書く)平安時代、女官が礼装のときに用いた髪の飾り。 頭髪の前につける。 平額(ひらびたい)。 
\\	童舞(わらわまい)の冠のこと。 
\\	「額髪(ひたいがみ)」の略。 
\\	「額付(ひたいつ)き」の略。 
\\	(多く「岸の額」の形で)物の突き出ている部分。 「岸の―のかた土くゎっと崩れて」	▲彼女を胸に抱き寄せて、額に口付けをした。 ▲彼女は額に皺を寄せた。
\\	日付	ひづけ	
\\	文書などに、その作成・提出などの年月日を記すこと。 また、その年月日。 「領収書に―を入れる」 
\\	暦に記された、年月日を表す数字。 「午前零時を過ぎて―が変わる」	▲彼は私に日付を確かめるように言った。 ▲日付を暗記することは容易ではない。
\\	必死	ひっし	[名・形動] 
\\	必ず死ぬこと。 
\\	死ぬ覚悟で全力を尽くすこと。 また、そのさま。 死にものぐるい。 「―の形相」「―に逃げる」 
\\	(「必至」とも書く)将棋で、放っておくと、次に王将が詰んでしまう状態。 また、その差し手。 「―をかける」	▲彼女は必死になって走った。 ▲営業マンだって、リストラ予備軍に配属されないように必死な訳です。
\\	ぴったり	ぴったり	①接合部に隙間やずれがなく密着しているさま。「戸を―閉める」 ②二つの物事が完全に合致しているさま。「収支が―合う」 ③物事がふさわしかったり的中したりしているさま。「若い人に―の本」「天気予報が―当たった」 ④続いていた状態が急に完全に停止するさま。「タバコを―やめる」	▲濡れた服はからだにぴったりひっつく。 ▲彼の靴に泥がぴったりくっついていた。
\\	引っ張る	ひっぱる	[動ラ五(四)]《「ひきはる」の音変化》 
\\	引いて、ぴんと張った状態にする。 「綱を―・る」 
\\	電線などを長く張りわたす。 また、導管・路線などをある場所まで延長する。 「電話線を―・る」「給水管を家まで―・る」 
\\	物の一部を持って自分の方へ強く引く。 また、引き寄せる。 「缶のタブを―・って開ける」「袖を―・って合図をする」 
\\	(車両などを)強く引いて前へ進める。 牽引(けんいん)する。 「三〇両もの貨車を―・って走る」 
\\	先に立って人を自分のめざす方へ導く。 「リーダーとなって仲間を―・ってゆく」 
\\	人をある場所へ連れて行く。 また、連行する。 「方々―・って歩く」「警察に―・られる」 
\\	仲間になるように強く誘う。 誘い入れる。 「優秀な技術者を―・る」「野球部に―・る」 
\\	長引かせたり遅らせたりする。 引き延ばす。 「たまった借金を―・っておく」「策を弄して審議を―・る」 
\\	発音を長くのばす。 「語尾を―・る」 
\\	引用する。 「例を―・ってきて説明する」 
\\	野球で、投手の球を思いきり打って、右打者はレフト方向、左打者はライト方向へ飛ばす。 「外角球を強引に―・る」↔流す。 
\\	着る。 ひっかける。 「着物らしい着物を―・っていたこともなく」 
\\	磔(はりつけ)にする。 「木の空(=磔柱)に―・らるるは今のこと」 [可能]ひっぱれる [類語]
\\	引く・引き寄せる・引き絞る・手繰(たぐ)る・手繰り寄せる・手繰り込む/
\\	引く・牽引(けんいん)する・曳航(えいこう)する/
\\	率いる・先導する・嚮導(きようどう)する・誘導する・主導する・リードする	▲彼は私の手をつかんで二階へ引っ張って行った。 ▲彼は私の襟首を引っ張った。
\\	否定	ひてい	[名]スル 
\\	そうではないと打ち消すこと。 また、非として認めないこと。 「うわさを―する」「暴力を―する」↔肯定。 
\\	論理学で、ある命題の主語と述語の関係が成立しないこと。 また、その関係を承認しないこと。 ↔肯定。 
\\	ヘーゲル弁証法で、発展の契機の一。 →否定の否定 
\\	文法で、打ち消しの語法のこと。	▲彼はそのニュースを否定した。 ▲彼はその言いがかりを否定した。
\\	ビデオ	ビデオ	
\\	映像。 特にテレビで、オーディオ(音声)に対する画像。 
\\	映像信号を取り扱う装置。 「―カセット」 
\\	「ビデオテープ」「ビデオテープレコーダー」「ビデオディスク」などの略。	▲10人中3人はビデオを持っている。 ▲カラオケ、ゲーム、ビデオに冷蔵庫・・・今のラブホってなんでもあるのねー。
\\	一言	ひとこと	
\\	一つの言葉。 一語。 「―のあいさつもない」 
\\	ちょっとした言葉。 短い言葉。 「とても―では言い尽くせない」	▲彼は一言も言わないで部屋から出ていった。 ▲彼は一言も言わなかった。
\\	等しい・均しい・斉しい	ひとしい	[形][文]ひと・し[シク] 
\\	二つ以上の物事の間で、その数量・程度・形状などに相違がない。 同じである。 「二辺の長さが―・い」「三人に―・く分ける」 
\\	二つ以上の物事の間に、性質・状況の上で同一性がある。 よく似ている。 「薬効が無いに―・い」「詐欺に―・い行為」 
\\	(「ひとしく」の形で)大勢が同じ行動をするさま。 一斉に。 こぞって。 「皆―・く非難する」 
\\	(「…とひとしく」の形で)時間的に同じであるさま。 と同時に。 …するやいなや。 「此の言葉を聞くと―・く、…身体がぐたりとなった」→同じ[用法] [派生]ひとしさ[名] [類語]
\\	同じ・同一・等価・同等・均等・等し並み・一律・一様(いちよう)・イコール(力量が)互角・五分(ごぶ)/
\\	同然・同断・同様	▲すべての読み物が等しく読む価値があるわけではない。 ▲その2人の先生には等しい数の生徒がいた。
\\	一人一人	ひとりひとり	《「ひとりびとり」とも》 
\\	多くの中のそれぞれの人。 めいめい。 各人。 副詞的にも用いる。 「―の自覚が大切だ」「―診察する」 
\\	どちらかひとり。 だれかひとり。 「思ひ定めて―に逢ひ奉り給ひね」	▲私たちは独特で成熟したひとりひとりの人間になるべきだ。 ▲私たちの一人一人が運転をする時には気を付けなければならない。
\\	批判	ひはん	[名]スル 
\\	物事に検討を加えて、判定・評価すること。 「事の適否を―する」「―力を養う」 
\\	人の言動・仕事などの誤りや欠点を指摘し、正すべきであるとして論じること。 「周囲の―を受ける」「政府を―する」 
\\	哲学で、認識・学説の基盤を原理的に研究し、その成立する条件などを明らかにすること。 →批評[用法] [類語]
\\	批評・講評・評価/
\\	論難・弁難・批難・批正・酷評・否定・駁論(ばくろん)・反論・反対(―する)難ずる・論(あげつら)う・駁(ばく)する・非を打つ	▲彼の批判は日本政府に向けられたものだった。 ▲彼の批判は場違いであった。
\\	批評	ひひょう	[名]スル物事の是非・善悪・正邪などを指摘して、自分の評価を述べること。 「論文を―する」「印象―」 [用法]批評・批判――「映画の批評(批判)をする」のように、事物の価値を判断し論じることでは、両語とも用いられる。 
\\	「批評」は良い点も悪い点も同じように指摘し、客観的に論じること。 「習作を友人に批評してもらう」「文芸批評」「批評眼」
\\	「批判」は本来、検討してよしあしを判定することで「識者の批判を仰ぎたい」のように用いるが、現在では、よくないと思う点をとりあげて否定的な評価をする際に使われることが多い。 「徹底的に批判し、追及する」「批判の的となる」「自己批判」	▲あなたの批評はいつも私には有益でした。 ▲これについて、自由に批評して下さい。
\\	秘密	ひみつ	[名・形動] 
\\	他人に知られないようにすること。 隠して人に見せたり教えたりしないこと。 また、そのようなさまやそのような事柄。 「―をもらす」「―にする」「云わば―な悪事でも唆(そその)かすように」 
\\	一般に知られていないこと。 また、公開されていないこと。 「古代史の―」 
\\	人に知らせない奥の手。 秘訣(ひけつ)。 「成功の―」 
\\	仏語。 ㋐深遠・微妙で容易に知りがたい奥義。 ㋑ある隠された意味をもって説くこと。 密教で大日如来の説法などをいう。 ㋒真言宗の教義。 密教。 [類語]
\\	内密・内証(ないしよ)・内内(ないない)・隠密(おんみつ)・極秘(ごくひ)・厳秘(げんぴ)・丸秘(まるひ)・機密・枢密(すうみつ)・天機・機事・密事・秘事・暗部・隠し事・秘め事・密(みそ)か事・内証(ないしよ)事・秘中の秘 (形動用法で)秘(ひそ)か・密(みそ)か	▲私にその秘密を知らせてくれませんか。 ▲私には秘密を打ち明けて相談できる人がいない。
\\	微妙	びみょう	[名・形動] 
\\	趣深く、何ともいえない美しさや味わいがあること。 また、そのさま。 みみょう。 「此―な叙景の筆の力が」 
\\	一言では言い表せないほど細かく、複雑なさま。 また、きわどくてどちらとも言い切れないさま。 「気持ちが―に変化する」「―な判定」 [派生]びみょうさ[名]	▲その二つの事の間には微妙な違いがある。 ▲非常に微妙な状況だった。
\\	紐	ひも	
\\	物をしばったり束ねたりするのに用いる細長いもの。 ふつう、糸より太く、綱より細いものにいう。 布・麻・化学繊維・紙・革などで作る。 ひぼ。 「羽織の―を結ぶ」「小包の―を解く」 
\\	物事を背後から支配すること。 引き替えの条件。 「―のついた援助」 
\\	女性を働かせて金をみつがせる情夫。 「あの女には―がついている」 
\\	ホタテガイ・アカガイなどの外套膜(がいとうまく)の部分。 →綱(つな)[用法] [下接語]顎(あご)紐・後ろ紐・打ち紐・負ぶい紐・掛け紐・飾り紐・革紐・括(くく)り紐・絎(くけ)紐・靴紐・組み紐・腰紐・ゴム紐・真田(さなだ)紐・しで紐・付け紐・綴じ紐・平(ひら)紐・結び紐・胸(むな)紐	▲彼女は洗濯物を紐にかけた。 ▲彼女は財布のひもをしっかり締めていた。
\\	表	ひょう	
\\	複雑な事柄を、見やすいように整理分類して、一目でわかるように書き表したもの。 「人口動態を―にする」 
\\	臣下から君主に差し出す文書。 「出師(すいし)の―」	▲この表には過去の会員も含まれている。 ▲この表の数値は千単位で示されている。
\\	費用	ひよう	[名]スル 
\\	ある事をするのに必要な金銭。 また、ある事のために金銭を使うこと。 「―がかさむ」「―を捻出(ねんしゆつ)する」「往昔十字軍の為に前後幾百万の資本を―せしや」 
\\	企業が収益を挙げるために費やした経済価値。 →経費[用法] [類語]
\\	掛かり・費(つい)え・入(い)り・入(い)り目・入(い)り用・入用(にゆうよう)・入費(にゆうひ)・出費・用度・経費・実費	▲この計画は実行するのに多額の費用がかかる。 ▲この春休みには、海外費用の足しにするため、レストランでアルバイトをした。
\\	秒	びょう	国際単位系
\\	の基本単位の一。 
\\	時間の単位。 一秒は一時間の三六〇〇分の一、一分の六〇分の一。 一九六七年の国際度量衡総会で、セシウム原子一三三の固有振動数が九一億九二六三万一七七〇ヘルツの時間間隔を一秒と定義し、それ以前に採用されていた暦表時と量的に等しくなるように決められた。 記号
\\	セコンド。 
\\	角度・緯度・経度の単位。 一秒は一度の三六〇〇分の一、一分の六〇分の一。 記号?	▲私は30秒の差で電車に乗り遅れた。 ▲私は2、3秒のきわどいところで終バスに間に合った。
\\	評価	ひょうか	[名]スル 
\\	品物の価格を決めること。 また、その価格。 ねぶみ。 「―額」 
\\	事物や人物の、善悪・美醜などの価値を判断して決めること。 「外見で人を―する」 
\\	ある事物や人物について、その意義・価値を認めること。 「―できる内容」「仕事ぶりを―する」 
\\	「教育評価」の略。児童・生徒の学習や行動の発達を、教育の目標に照らして測り、判定すること。試験、考査だけでなく、日常の観察による判定を含まれる。	▲若者は自分の能力を正しく評価できないことが多い。 ▲人はしばしば付き合う友達によって評価される。
\\	表現	ひょうげん	[名]スル心理的、感情的、精神的などの内面的なものを、外面的、感性的形象として客観化すること。 また、その客観的形象としての、表情・身振り・言語・記号・造形物など。 「情感を―する」「全身で―する」 
\\	および
\\	の訳語。 [類語]表出・表白・発現・描出・形象化・言い表す・書き表す・名状する・形容する	▲我々は言葉によって思想を表現する。 ▲外国人を悩ますもう一つの、多くの日本人のもつ傾向は、「すべての」「あらゆる」というような言葉を使ったり、仄めかしたりして、あまりにも一般的であり、あまりにも広がりのある表現をする点にある。
\\	表情	ひょうじょう	
\\	感情や情緒を顔つきや身振りに表すこと。 また、その顔つきや身振り。 「悲しげな―」「―がくもる」「自分に向って何だか―しているような可憐な花」 
\\	一般に、状況・ようす。 「全国各地の歳末の―」「被災地の―」 [類語]
\\	面(おも)持ち・顔(かお)・顔つき・顔色(かおいろ)・色(いろ)・血相・形相(ぎようそう)・相好(そうごう)・気色(きしよく)・神色(しんしよく)	▲彼女は高慢な表情で私を見て、歩き去った。 ▲彼女は悲しそうな表情をしていた。
\\	平等	びょうどう	[名・形動]かたよりや差別がなく、みな等しいこと。 また、そのさま。 「利益を―に分配する」「男女―」	▲法のもとではすべての人は平等だ。 ▲平等は憲法で保障されている。
\\	評判	ひょうばん	[名・形動]スル 
\\	世間の人が批評して是非を判定すること。 また、その判定。 「―の高い作品」「―を落とす」 
\\	世間でうわさをすること。 また、そのうわさ。 「―が立つ」「人々がさまざまに―する事件」 
\\	世間の関心の的になっていること。 名高いこと。 また、そのさま。 「今年―になった映画」「―な(の)働き者」 [類語]
\\	世評・評価・人気・受け・人受け・気受け・聞こえ・名(な)・声聞(せいぶん)・声名(せいめい)・名声・盛名/
\\	噂(うわさ)・取り沙汰(ざた)・風評・風説・風聞(ふうぶん)・外聞・人聞き・下馬評/
\\	人気・高名・著名・有名・名代(なだい)・名うて・名高い	▲恥の文化は評判に関心を持つ。 ▲人と会う約束をしたときは時間を守りなさい。そうしないと評判を落としますよ。
\\	表面	ひょうめん	
\\	物の外側をなす面。 おもて。 「月の―」「液体の―」↔裏面。 
\\	物事の、外から見える部分。 表立つところ。 うわべ。 「―を飾る」「事が―に出る」「―的な見方」↔裏面。 [類語]
\\	表(おもて)・面(おも・おもて)・上面(うわつら)・上側(うわかわ)・上面(じようめん)・界面・表層/
\\	表(おもて)・外(そと)・外面(がいめん)・上辺(うわべ)・上面(うわつら)・上皮(うわかわ)・外観・外見(がいけん)・見掛け・見た目・皮相・表層・現象	▲人々は必ずしも表面に表われた通りではない。 ▲水は地球の表面の大部分を占めている。
\\	広がる・拡がる	ひろがる	[動ラ五(四)] 
\\	空間・面積・幅が大きくなる。 「改築して家が―・る」「川幅が―・る」 
\\	範囲・規模が大きくなる。 「視野が―・る」「販路が―・る」「汚染が―・る」 
\\	畳んだり閉じたりしてある物などが開く。 また、先の方に向かって幅が大きくなる。 「傘が―・る」「裾が―・ったスカート」「花火が―・る」 
\\	大きく展開する。 「眼前に大海原が―・る」 [可能]ひろがれる [用法]ひろがる・ひろまる――「大地震のうわさが広がる(広まる)」では、相通じて用いられる。 
\\	「広がる」は自然現象として、また人の営みの結果として、面積や範囲が大きくなる意。 「眼下に広がる大平原」「火事が広がる」「事業が広がる」
\\	「広まる」は自然にという意は少なく、人が大きくのばそうと努めた結果、行きわたるの意が強い。 
\\	機器の利用が広まる」「教育が全国民に広まる」 [類語]
\\	拡大する・拡張する・伸張する・膨張する/
\\	広まる・行き渡る・流布(るふ)する・伝播(でんぱ)する・浸透する・波及する・瀰漫(びまん)する・蔓延(まんえん)する・伸展する・発展する	▲彼が現れると気まずい沈黙が広がった。 ▲電気ストーブで部屋中に熱が広がった。
\\	品	ひん	㊀[名]人や物にそなわっている、好ましい品格・品質。 「―がよい」「―がない」 ㊁〔接尾〕助数詞。 料理などの品数を数えるのに用いる。 上にくる語によっては「ぴん」となる。 「二―注文する」	▲注文の品が届いた。 ▲彼女にはどことなく品がある。
\\	瓶・壜・罎	びん	《「びん(瓶)」は唐音》液体などを入れる、ガラス製や陶製の容器。	▲薬を服用するときは、ビンに書いてある用法に注意深く従いなさい。 ▲葡萄酒を1瓶ください。
\\	便	びん	
\\	人や荷物・手紙などをある場所まで運ぶこと。 また、その手段。 「飛行機の―がある」「午後の―で届く」「宅配―」「速達―」 
\\	都合のよい機会。 よい方法。 ついで。 「侍のをのこども仕まつるもののうちに、―ある所をなむ僧坊にしける」	▲ユナイテッドの111便に乗り換えるのですが。 ▲ユナイテッド航空111便の搭乗ゲートはどこですか。
\\	ピン	ピン	
\\	つき刺したりはさんだりして、物を留める道具。 虫ピン・ネクタイピン・ヘアピン・安全ピンなど。 
\\	部材を接合するために端にある穴に挿入する細い丸鋼。 
\\	ボウリングで、瓶の形をした標的。 
\\	ゴルフで、ホールにさす目印の旗、または旗ざお。 
\\	登山で、ザイルを使って確保するときに支点とするハーケンなどのこと。	▲写真はピンでとめられていた。 ▲彼女が彼の名前を言ったのでピンと来た。
\\	不	ふ	〔接頭〕名詞または形容動詞の語幹に付いて、それを打ち消し、否定する意を表す。 
\\	…でない、…しない、などの意を添える。 「―必要」「―一致」「―確か」「―行き届き」 
\\	…がない、…がわるい、…がよくない、などの意を添える。 「―人情」「―景気」「―出来」「―手際」	
\\	分	ぶん	
\\	分けられた部分。 分けまえ。 「これは私の―です」 
\\	ある範囲の分量。 区別されたもの。 「あまった―をわける」 
\\	その人の持っている身分や能力。 身の程。 分際。 「―をわきまえる」「―を守る」「―に安んずる」「―に過ぎる」 
\\	当然なすべきつとめ。 本分。 「学生は学問をその―とする」「己(おのれ)の―を尽くす」 
\\	物事の状態・様子・程度。 「この―なら計画の実行は大丈夫だ」 
\\	仮にそうであるとする状態。 「行く―には差し支えあるまい」 
\\	しな。 ほう。 「よい―のふだん着に着換えている」 
\\	それだけのこと。 だけ。 「跣足(はだし)になります―のこと」 
\\	他の名詞につけて用いる。 ㋐それに相当するもの、それに見合うものの意を表す。 「増加―」「苦労―」「五日―」 ㋑その身分に準じる意を表す。 「私の兄貴―」 ㋒成分の意を表す。 「アルコール―」 ㋓区切られた時間の意を表す。 「夏―の水飴の様に、だらしがないが」	▲部屋代は半年分支払い済みだ。 ▲彼女は30分前だと思いますが。
\\	無	ぶ	〔接頭〕名詞または形容動詞の語幹に付いて、それを打ち消し、否定する意を表す。 不(ぶ)。 
\\	…でない、…しない、などの意を添える。 「―風流」「―遠慮」 
\\	…がわるい、…がよくない、などの意を添える。 「―愛想」「―作法」「―細工」 ◆「不…」「無…」の使い分けについては、概して「不」は状態を表す語に付き、「無」は体言に付くとはいえるが、古来、「不(無)気味」「不(無)作法」など両様に用いられる語も少なくない。 また、「不」字は呉音フ、漢音フウであって、ブは「無」字の漢音ブに影響されて生じた慣用音と思われる。	▲無から有は生じ得ない。 ▲無回答を拒否する。
\\	不安	ふあん	[名・形動]気がかりで落ち着かないこと。 心配なこと。 また、そのさま。 「―を抱く」「―に襲われる」「―な毎日」「夜道は―だ」 [派生]ふあんがる[動ラ五]ふあんげ[形動] [類語]心配・懸念・危惧(きぐ)・危懼(きく)・疑懼(ぎく)・恐れ・胸騒ぎ・気がかり・心がかり・不安心・心細い・心許(こころもと)ない	▲その手紙を読んだとき、私は不安を感じなかった。 ▲その知らせで私は不安になった。
\\	風景	ふうけい	
\\	目に映る広い範囲のながめ。 景色。 風光。 「山岳―」 
\\	ある場面の情景・ありさま。 「ほほえましい親子の―」「新春―」	▲彼は丘の上に立って風景を見渡した。 ▲風景に気を取られて運転者は道路から目をそらした。
\\	夫婦	ふうふ	婚姻関係にある男女の一組。 夫と妻。 めおと。 「似た者―」 [類語]夫婦(めおと・みようと)・夫妻・妹背(いもせ)・連れ合い・配偶者・配偶・匹偶(ひつぐう)・伴侶(はんりよ)・カップル	▲年輩の夫婦は贈り物より現金を好むことが多いが、それは、そうした贈り物は必要でもなければ、置く場所もないからである。 ▲超年老いた夫婦が、結婚75周年を祝して豪華な夕食を食べていた。
\\	笛	ふえ	
\\	管楽器のうち、らっぱ類を除いたものの一般的呼称。 フルート・篠笛(しのぶえ)などの横笛と、リコーダー・尺八・篳篥(ひちりき)などの縦笛に分けられる。 また、口笛・草笛など。 
\\	特に、横笛のこと。 
\\	呼び子・ホイッスルなど、合図に吹き鳴らすもの。 「集合の―が鳴る」 
\\	汽笛。 
\\	(「吭」とも書く)のどぶえ。 「横に―を切ったが、それでは死に切れなかったので」 [下接語]?(う)の笛・笙(しよう)の笛・早笛 (ぶえ)葦(あし)笛・鶯(うぐいす)笛・神楽(かぐら)笛・烏(からす)笛・雉(きじ)笛・草笛・口笛・駒(こま)笛・高麗(こま)笛・鹿(しか)笛・篠(しの)笛・柴(しば)笛・蝉(せみ)笛・竹笛・縦笛・調子笛・角(つの)笛・唐人笛・鳥笛・喉(のど)笛・鳩(はと)笛・鼻笛・雲雀(ひばり)笛・牧(まき)笛・麦笛・虫笛・虎落(もがり)笛・指笛・横笛	▲警官はその車に停車せよと笛で合図した。 ▲警官は笛を吹いて車に止まるよう合図した。
\\	不可	ふか	
\\	よくないこと。 いけないこと。 「可もなく―もなし」 
\\	成績などの等級の最下位。 優・良・可に次ぐもので、不合格。	▲それは可でもなく不可でもなしというところだ。 ▲全体として、彼の作品は可もなし不可もなしです。
\\	武器	ぶき	
\\	戦いに用いる種々の道具や器具。 刀や銃などの、敵を攻撃したり自分を守ったりするための兵器や武具。 
\\	何かをするための有力な手段となるもの。 「弁舌を―にする」	▲彼らの中には武器を作る才能のある者がいた。 ▲彼らはいつも武器を持っていた。
\\	服装	ふくそう	衣服とその装身具。 また、それをつけたときのようす。 身なり。 「質素な―」	▲この服装様式はパリに始まった。 ▲その泥棒の人相とか服装を簡単に話してくれますか。
\\	含む	ふくむ	㊀[動マ五(四)] 
\\	かんだり、飲みこんだりせず、物を口の中に入れたままの状態を保つ。 また、口でくわえる。 「水を口に―・む」「マウスピースを―・む」 
\\	成分・内容としてうちに包みもつ。 また、ある範囲の中に要素として入っている。 包含する。 「この温泉は硫黄分を―・んでいる」「予備費の―・まれた予算案」 
\\	思いや感情などを心の中におさめてもつ。 「非難を―・んだ言い方」 
\\	事情をよく理解して心にとめておく。 「その点を―・んでおいてください」 
\\	ある感情を表情や態度に表す。 ようすを帯びる。 「悲しみを―・んだ目つき」 
\\	中に包み持つような形になる。 ふくらむ。 「指貫(さしぬき)の裾つかた、少し―・みて」 [可能]ふくめる ㊁[動マ下二]「ふくめる」の文語形。 [類語]
\\	含有する・包含する・内含する・内包する・包括する・包蔵する・包摂する・含める	▲これには、様々な種類の団体、人々、考え方が含まれている。 ▲全角文字を含むファイル名の場合、一部のOSでは文字化けが生じることがありますので、ダウンロードの際に適宜ファイル名を変更してください。
\\	袋・嚢	ふくろ	
\\	布・紙・革・ビニールなどで、中に物を入れて口を閉じるように作ったもの。 「―に詰める」「給料―」 
\\	ミカン・ホオズキなどの果肉を包む薄い皮。 「ミカンを―ごと食べる」 
\\	体内にある、物を入れるような形の器官。 「胃―」「子―」 
\\	あいている方向が一つしかないもの。 行き止まりの場所。 「―地」「抜裏と間違えて―の口へ這入り込んだ結果」 
\\	水に囲まれた土地。 
\\	おでん種の一。 開いた油揚げの中に野菜やしらたき・豚肉を入れ、かんぴょうでしばったもの。 
\\	巾着(きんちやく)。 また、所持金。 「―をかたぶけて酒飯の設(まうけ)をす」 [下接語](ぶくろ)胃袋・慰問袋・浮き袋・歌袋・腕袋・大入り袋・大津袋・合切(がつさい)袋・紙袋・革袋・堪忍袋・救助袋・氷袋・乞食(こじき)袋・漉(こ)し袋・子袋・小袋・米袋・酒(さか)袋・散財袋・地袋・祝儀袋・状袋・信玄袋・頭陀(ずだ)袋・砂袋・墨袋・段袋・知恵袋・茶袋・柄(つか)袋・手袋・天袋・戸袋・南京(ナンキン)袋・匂(にお)い袋・糠(ぬか)袋・寝袋・熨斗(のし)袋・火袋・封じ袋・福袋・不祝儀袋・文(ふみ)袋・頬(ほお)袋・蛍袋・ぽち袋・ポリ袋・守り袋・水袋・耳袋	▲少女がビニール袋に物を詰めて運んでいる。 ▲犯人は袋の中の鼠だ。
\\	不幸	ふこう	[名・形動] 
\\	幸福でないこと。 また、そのさま。 ふしあわせ。 「―な境遇」 
\\	身内の人などに死なれること。	▲詩人は不幸にも若死にした。 ▲自伝の中で彼はくりかえし不幸な少年時代に言及している。
\\	節	ふし	
\\	棒状の物の盛り上がった部分。 ㋐竹・葦(あし)などの茎にあるふくれた区切り。 ㋑幹や茎から枝が出るところ。 また、木材に残る枝の出たあと。 「―のある板」 ㋒骨のつなぎ目。 関節。 「指の―」 ㋓糸や縄のこぶ状になった所。 「―の多い糸」 
\\	区切りとなる箇所。 段落。 せつ。 「これを人生の―としよう」 
\\	心のとまるところ。 …と思われる点。 「疑わしい―が二、三ある」 
\\	機会。 おり。 おりふし。 「何かの―に思い出す」 
\\	㋐歌などの旋律。 また、旋律のひとくぎり。 曲節。 「―をつけて歌う」「出だしの―を口ずさむ」 ㋑文章を音読するときの抑揚。 「―をつけて朗読する」 ㋒(ふつう「フシ」と書く)浄瑠璃や謡曲などの語り物で、詞(ことば)に対する旋律的な部分。 
\\	その人独特の語り口。 演説や講演にいう。 
\\	「鰹(かつお)節」「鯖(さば)節」などの略。 
\\	定常波で、ほとんど振動していない部分。 振幅が最小の点。 ↔腹。 
\\	なんくせ。 言いがかり。 「喧嘩に―はなくてめでたし」 [下接語]憂(う)き節・折節・七(なな)節・一(ひと)節(ぶし)一中節・田舎節・入れ節・歌沢節・腕節・愁い節・荻江(おぎえ)節・雄節・鰹(かつお)節・河東(かとう)節・亀(かめ)節・義太夫(ぎだゆう)節・清元節・削り節・小節・鯖(さば)節・鮪(しび)節・新内節・説経節・背節・薗八(そのはち)節・連れ節・常磐津(ときわず)節・常(とこ)節・富本(とみもと)節・浪花(なにわ)節・生(なま)節・生(なま)り節・節節・骨節・本節・都節・雌節(ぷし)腕っ節・骨っ節	▲彼には加古川の人を軽蔑しているふしがある。
\\	無事	ぶじ	[名・形動] 
\\	普段と変わりないこと。 また、そのさま。 「日々を―に送る」「平穏―」 
\\	過失や事故のないこと。 また、そのさま。 「航海の―を祈る」「―に任務を果たす」「手術が―終了する」 
\\	健康で元気なこと。 つつがないこと。 また、そのさま。 「―を知らせる便り」「父母の―な顔を見て喜ぶ」 
\\	なすべき事がないこと。 ひまな状態。 「われ―に苦みて、外に出でて遊ばんことを請い」 
\\	何もしないこと。 「只道士の術を学んで、無為を業とし、―を事とす」 [類語]
\\	平安・平穏・安穏(あんのん)・安全・安泰・安楽・事無く・恙(つつが)無く・無難に/
\\	元気・達者・息災(そくさい)・壮健・丈夫	▲好天が続いたので、私達は無事収穫できた。 ▲行方不明だった漁船が無事帰港した。
\\	不思議	ふしぎ	[名・形動]《「不可思議」の略》 
\\	どうしてなのか、普通では考えも想像もできないこと。 説明のつかないこと。 また、そのさま。 「―な出来事」「成功も―でない」 
\\	仏語。 人間の認識・理解を越えていること。 人知の遠く及ばないこと。 
\\	非常識なこと。 とっぴなこと。 また、そのさま。 「花山院とあらがひごと申させ給へりしはとよ。 いと―なりしことぞかし」 
\\	怪しいこと。 不審に思うこと。 また、そのさま。 「明くればうるはしき女?に―を立て、いかなる御方ぞ、と尋ね給ふに」 [派生]ふしぎがる[動ラ五]ふしぎさ[名] [類語]
\\	不可思議・不可解・不審・奇妙・面妖(めんよう)・妙(みよう)・変(へん)・異(い)・謎(なぞ)・怪(かい)・奇(き)・奇異・奇怪・幻怪・怪奇・怪異・神秘・霊妙・霊異・玄妙・あやかし・ミステリー・ミステリアス	▲不思議な話だが、彼の予言は当たった。 ▲不思議な話だが、先生は叱らなかった。
\\	不自由	ふじゆう	[名・形動]スル思うようにならないこと。 不足や欠けた点があって困ること。 不便なこと。 また、そのさま。 「何かと―な暮らし」「小遣いにも―する」 [派生]ふじゆうさ[名]	▲妻がいないと何かと不自由だ。 ▲私が生きている間は君には何一つ不自由させません。
\\	夫人	ふじん	
\\	貴人の妻。 また、他人の妻を敬っていう語。 「―同伴」「令―」「社長―」 
\\	律令制で、皇后・妃の次に位する後宮(こうきゆう)の女性。 三位以上の女性から選んだ。 ぶにん。 
\\	昔、中国で、天子の后(きさき)や諸侯の妻などの称。 ぶにん。	▲私は家にあるだけのバターをジョーンズ夫人に貸してあげた。 ▲はじめまして、ジョーンズ夫人。
\\	婦人	ふじん	成人した女性。 相応の年齢に達している一人前の女性。 「―服」「職業―」→女性[用法]	▲彼女は成長して美しい婦人になった。 ▲彼女は親切にもその婦人を家まで送ってあげた。
\\	不正	ふせい	[名・形動]正しくないこと。 また、その行為や、そのさま。 「―をはたらく」「―な取引」	▲彼は不正との戦いに一生をささげた。 ▲彼は商売で不正な手段を用いた。
\\	防ぐ・禦ぐ・拒ぐ	ふせぐ	[動ガ五(四)]《古くは「ふせく」》 
\\	㋐敵の攻撃を抑える。 敵に侵害されないようにする。 「本土への侵攻を―・ぐ」 ㋑好ましくないものを、さえぎって中へ入れないようにする。 「窓を二重にして寒さを―・ぐ」 
\\	好ましくない事態が生じないようにする。 「二次感染を―・ぐ」「混乱を―・ぐ」 [可能]ふせげる	▲花粉よりも小さな黄砂をマスクでどれだけ防ぐことが出来るのか?花粉よりもずっと厄介者のように思います。 ▲壊滅的な被害を防ぐために世界的な警戒を強化していく重要性を確認した。
\\	不足	ふそく	[名・形動]スル 
\\	足りないこと。 十分でないこと。 また、その箇所や、そのさま。 「―を補う」「学力が―だ」「料金が―する」 
\\	満足でないこと。 また、そのさま。 不満。 「対戦相手として―はない」「思いどおりにならず―な顔をする」	▲この地域はでは水が不足している。 ▲コンピューター・プログラマーの労働力が不足している。
\\	舞台	ぶたい	
\\	演劇・舞踊・音楽などを行うために設けられた場所。 ステージ。 「―に上がる」「能―」「回り―」 
\\	の上で行われる演技や演奏。 「晴れの―をつとめる」「一人―」 
\\	腕前を見せる場所。 活躍の場所。 「世界を―に飛び回る」	▲これらの柱が舞台をささえている。 ▲『カムイの剣』は、1868年の徳川将軍時代の崩壊と、明治天皇下での日本の復興という変革期を舞台にした、一種の侍/忍者物語だ。
\\	双子・二子	ふたご	同じ母親から一度の出産で生まれた二人の子。 双生児。	▲彼の双子の妹たちを区別することができない。 ▲彼の妻は双子の男の子を産んだ。
\\	再び・二度	ふたたび	
\\	同じ動作や状態を繰り返すこと。 副詞的にも用いる。 「―の来訪」「―過ちを犯す」 
\\	二番目。 二度目。 「―の御祓(はら)へのいそぎ」	▲彼らは別れて二度と再び会う事はなかった。 ▲彼らは10マイル歩いて10分間休み、また再び歩いた。
\\	普段	ふだん	⇒ふだん(不断)
\\	二
\\	▲彼らは普段自転車で登校します。 ▲正子はふだん歩いて学校へ行く。
\\	縁	ふち	
\\	物の端の部分。 また、物の周りの、ある幅をもった部分。 へり。 「がけの―」「―が欠ける」「帽子の―」 
\\	刀の柄口(つかぐち)の金具。 [用法]ふち・へり――「机のふち(へり)に手をつく」「茶碗のふち(へり)」「崖のふち(へり)」のように、物のまわりやまぎわの部分の意では、相通じて用いられる。 
\\	「ふち」には「目のふちを赤くする」とか、「眼鏡のふち」「額(がく)ぶち」のような、回りの枠をいう使い方もあり、この場合は「へり」は用いない。 
\\	「へり」は、「船べり」「川べり」のように平らなものの周辺部をいうことが多く、さらに周辺部につける飾り物などの意まで広がる。 「リボンでへりをつける」「畳のへりがすり切れる」	▲その後、カードを引き抜いた(空気はまったく入っていないので水は漏れない。グラスの縁はぴったりテーブルに接しているから)。 ▲彼らは水瓶を縁まで一杯にした。
\\	打つ・撃つ・撲つ	ぶつ	[動タ五(四)]《「うつ」の音変化》 
\\	たたく。 なぐる。 また、ぶつける。 「子供のおしりを―・つ」「転んでひざを―・つ」 
\\	演説する、語る意などを強めていう語。 「一席―・つ」 
\\	博打(ばくち)をする。 「飲むも可(よ)し、―・つも可し、買うも可しだが」 [可能]ぶてる	▲やめなさい、お尻をぶちますよ。 ▲もしケーキをひとつでも食べたら、ぶつからね。
\\	物価	ぶっか	品物の値段。 種々の財・サービスの平均的な価格。	▲賃金と物価の悪循環を断ち切ることは困難だ。 ▲当時、物価は毎週変化していた。
\\	物質	ぶっしつ	
\\	もの。 品物。 生命や精神に対立する存在としての物。 「―の世界」 
\\	物理学で、物体を形づくり、任意に変化させることのできない性質をもつ存在。 空間の一部を占め、有限の質量をもつもの。 素粒子の集まり。 相対性理論ではエネルギーの一形態、量子論では場とされる。 
\\	哲学で、感覚によってその存在が認められるもの。 人間の意識に反映するが、意識からは独立して存在するもの。	▲塩は有用な物質だ。 ▲もう一つの面白いエネルギー源は、放射能の廃棄物質から取り出せる熱である。
\\	物理	ぶつり	
\\	物の道理。 物の理法。 「その倫に由て―を害する勿れ」 
\\	「物理学」の略。自然科学の一部門。天然現象や実験室内で制御された条件下で生ずる自然現象を観察したり数量的に測定したりして、その結果から現象を支配する法則を帰納し、その法則を類似現象に演繹し、また、その法則をより基礎的な法則から理論的に説明する学問。力学、熱学、光学、電磁気学、統計力学、量子力学、原子物理学、物性物理学、生物物理学などに大別される。	▲山田先生は物理の先生ですか、化学の先生ですか。 ▲私の物理の知識は貧弱です。
\\	筆	ふで	㊀[名] 
\\	竹や木の柄の先に獣毛をたばねてつけ、これに墨や絵の具などをふくませて字や絵をかく道具。 毛筆。 また、筆記具の総称。 「―の運び」 
\\	書くこと。 また、書いたもの。 「定家の―になる」 
\\	文章を書くこと。 また、その文章。 「―で飯を食う」 ㊁〔接尾〕助数詞。 文字や絵を書くとき、筆に墨や絵の具などをつける回数、または筆や鉛筆を紙にあてて動かす回数を数えるのに用いる。 「一―で書く」 [下接語]絵筆・大筆・掠(かす)り筆・鉄漿(かね)付け筆・隈(くま)取り筆・毛描(が)き筆・小筆・彩色筆・戯(ざ)れ筆・椎(しい)の実筆・朱筆・初筆・添え筆・禿(ち)び筆・椽大(てんだい)の筆・留め筆・中筆・偽(にせ)筆・一(ひと)筆・平筆・紅(べに)筆・坊主筆・巻き筆・面相筆・焼き筆	▲弘法も筆の誤り。 ▲弘法筆を選ばず。
\\	部分	ぶぶん	全体をいくつかに分けたものの一つ。 「シャツの胸の―に汚れがある」 [類語]箇所(かしよ)・ところ・部位・一部・一部分・局部・局所・細部・断片・一端(いつたん)・一斑(いつぱん)・一節(いつせつ)・件(くだり)・パート・セクション	▲私は自分の演説の最も重要な部分を落としてしまったと気がついたが、遅すぎた。 ▲私は問題のこの部分は詳しくない。
\\	不平	ふへい	[名・形動]納得できず不満であること。 また、そのさま。 「しかられて―な顔をする」「―たらたら」 [用法]不平・不満・不服――「交渉の結果に不平(不満・不服)をもらす」では三語とも用いられる。 
\\	「不平」は要求が満たされなくて不愉快な思いが態度やことばに出る場合に多く用いる。 「『弟ばかりかわいがって』と兄が不平を言う」「政治に対して不平を並べる」
\\	「不満」は要求の水準に達していないので物足りない、気に入らないと思う意が強い。 「与えられた役を不満に思う」「欲求不満」「この成績では不満だ」など「不平」「不服」では置き換えられない。 
\\	「不服」は相手の申し出・条件などに従えない場合に多く用いる。 「判決に不服を申し立てる」「一方的な押しつけには不服だ」	▲彼らはいつも不平ばかり言っている。 ▲お母さんはいつも私の不平ばかり言っている。
\\	不満	ふまん	[名・形動]もの足りなく、満足しないこと。 また、そのさまやそう思う気持ち。 不満足。 「―を口にする」「―な点が残る」「欲求―」→不平[用法] [派生]ふまんがる[動ラ五]ふまんげ[形動]	▲彼は彼等の自分に対する扱いに不満を言った。 ▲彼は上司についていつも不満を言っている。
\\	プラス	プラス	[名]スル 
\\	加えること。 加算。 「本給に手当てを―する」↔マイナス。 
\\	㋐加えることを表す符号。 加号。 
\\	。 ↔マイナス。 ㋑正数を表す符号。 正号。 
\\	。 ↔マイナス。 
\\	よいこと。 また、よい面。 「―に考える」「―に評価する」↔マイナス。 
\\	ためになること。 有益。 「体験が―になる」↔マイナス。 
\\	黒字。 利益。 ↔マイナス。 
\\	陽電気。 また、その記号。 
\\	。 「―極」↔マイナス。 
\\	陽性。 陽性反応。 ↔マイナス。	▲結婚について言えば、身を固めることは女性より男性にプラス面が大きい。 ▲こうしたら君のプラスになる。
\\	プラン	プラン	
\\	計画。 構想。 案。 「作戦の―を立てる」「マスター―」 
\\	図面。 設計図。 平面図。	▲とても現実味のあるプランとは言えなかったな。 ▲これは、お前の体躯、護衛能力を考慮した上でのプランなのだ。是が非でもやってもらう。
\\	不利	ふり	[名・形動]利益にならないこと。 条件・形勢などがよくないこと。 また、そのさま。 「―な取引」「―な立場」↔有利。	▲背が低いことはバレーボールの選手にとって不利である。 ▲背が高くないことは不利ではない。
\\	振る	ふる	[動ラ五(四)] 
\\	からだの一部を、また物の一方の端をもって上下・左右・前後に何度も繰り返すようにして動かす。 「ハンカチを―・る」「腕を―・って歩く」「犬がしっぽを―・る」 
\\	手を動かして握ったものを下方に投げる。 また、勢いをつけて振りまく。 「さいころを―・る」「塩を―・る」 
\\	割り当てる。 「大役を―・られる」 
\\	文字のわきに記号・読みがななどをつける。 「ルビを―・る」 
\\	相手の求めを退ける。 はねつけて相手にしない。 「女に―・られる」 
\\	得た地位・立場などをあっさり捨てる。 また、しようとする意志を捨てる。 むだにする。 「重役の地位を―・る」「一生を棒に―・る」 
\\	動かして方向を少しずらせる。 進む向きをある方向に変える。 「舵を右に―・る」 
\\	煎じて出す。 「薬を―・る」 
\\	勢いよく担ぎ動かす。 「みこしを―・る」 
\\	本題に入るきっかけとして話す。 話を導き出そうとする。 「落語家がまくらを―・る」「司会者が話題を―・る」 
\\	為替・手形などを発行する。 「為替を―・る」 
\\	神体を移す。 「三笠山に―・り奉りて、春日明神と名づけ奉りて」 
\\	入れかえる。 置きかえる。 「行く春は行く歳にも―・るべしといへり」 [可能]ふれる [下接句]命を棒に振る・尾を振る・大手(おおで)を振る・顔を振る・頭(かしら)を縦に振る・頭(かしら)を横に振る・頭(かぶり)を振る・首を縦に振る・首を横に振る・采配(さいはい)を振る・先棒を振る・尻尾(しつぽ)を振る・身代を棒に振る・無い袖(そで)は振られぬ・棒に振る・脇目(わきめ)も振らず	▲このビンは開ける前に振りなさい。 ▲彼女は昨日、彼をふった。
\\	震える	ふるえる	[動ア下一][文]ふる・ふ[ハ下二] 
\\	細かく揺れ動く。 震動する。 「地震で窓ガラスが―・える」 
\\	寒さや激しい感情のために、からだが小刻みに動く。 「恐ろしさに膝が―・える」「雨にずぶぬれになって―・える」	▲こう見えても意外に人前で話すの苦手で、手なんか震えるし、口ごもちゃって、自分でも何言っているかわからない時がある。 ▲しゃべり出した時彼の両手はぶるぶる震えた。
\\	古里・故里・故郷	ふるさと	
\\	自分の生まれ育った土地。 故郷。 郷里。 「―に帰る」 
\\	荒れ果てた古い土地。 特に、都などがあったが今は衰えている土地。 「君により言の繁きを―の明日香(あすか)の川にみそぎしに行く」 
\\	以前住んでいた、また、前に行ったことのある土地。 「ひとはいさ心もしらず―は花ぞ昔の香に匂ひける」 
\\	宮仕え先や旅先に対して、自分の家。 自宅。 「見どころもなき―の木立を見るにも」	▲彼は故郷の町に引退し、そこで静かな生活を送った。 ▲彼は故郷の夢を見た。
\\	ブレーキ	ブレーキ	
\\	機械の運動を停止させたり減速させたりする装置。 制動機。 「―がよくきく車」「サイド―」 
\\	事物の進行・進展などを妨げるもの。 歯止め。 「チャンスに―となったバッター」	▲ブレーキが効かなかった。 ▲ブレーキがよくきかない。
\\	触れる	ふれる	[動ラ下一][文]ふ・る[ラ下二] 
\\	㋐ある物が他の物に、瞬間的に、または軽くくっつく。 ちょっとさわる。 「肩に―・れる」「機雷に―・れる」「外の空気に―・れる」 ㋑脈が反応する。 脈拍を指先に感じる。 「脈が―・れなくなる」 ㋒(「耳(目)にふれる」の形で)ちょっと耳にしたり見たりする。 「人の目に―・れる」「耳に―・れるうわさの数々」 ㋓あることを話題にする。 言及する。 「食料問題に―・れる」「核心に―・れる」 ㋔ある時期や物事に出あう。 「折に―・れて訪れる」「事に―・れてからかわれる」 ㋕規則・法律などに反する。 抵触する。 「学則に―・れる」「法に―・れる」 ㋖怒りなどの感情を身に受ける。 「勘気に―・れる」「怒りに―・れる」 ㋗感動・感銘を受ける。 「心の琴線に―・れる」「心に―・れる話」 
\\	㋐物に軽くくっつくようにする。 「髪の毛に手を―・れる」「花に手を―・れる」 ㋑広く人々に知らせる。 「隣近所に―・れて回る」 
\\	食べ物にちょっと手を付ける。 「朝餉(あさがれひ)のけしきばかり―・れさせ給ひて」→触(さわ)る[用法] [類語]
\\	触(さわ)る・接する・着く・当たる・擦(かす)る・接触する・触接する・タッチする	▲彼女は何かが首に触れるのを感じた。 ▲少女の鋭い感性に触れている。
\\	プロ	プロ	
\\	《「プロフェッショナル」の略》ある物事を職業として行い、それで生計を立てている人。 本職。 くろうと。 「その道の―」「―顔負けの腕前」「―ゴルファー」↔アマ。 
\\	「プログラム」「プロダクション」「プロレタリア」「プロパガンダ」などの略。	▲あのプロのスキーヤーは離れ業をしながら山をおりるのが好きだった。 ▲楽器・機材の質は完全にプロ仕様!ですが値段は何処よりも安い!
\\	文	ぶん	
\\	文字で書かれたまとまった一連の言葉。 文章。 また、詩文。 「巧みな―」「―をつづる」 
\\	文法上の言語単位の一。 一語またはそれ以上の語からなり、ひと区切りのまとまりある考えを示すもの。 文字で書くときは、ふつう
\\	(句点)でその終わりを示す。 センテンス。 
\\	学芸。 学問。 文事。 「―武両道」↔武。 
\\	外見を美しくするための飾り。 あや。 もよう。 「すなほなるを賤しくし、―を尊ぶ故に、人、―を学でいつはり多し」	▲文の最初の語はすべて大文字ではじめなければならない。 ▲文の初めには大文字が用いられる。
\\	雰囲気	ふんいき	
\\	天体、特に地球をとりまく空気。 大気。 
\\	その場やそこにいる人たちが自然に作り出している気分。 また、ある人が周囲に感じさせる特別な気分。 ムード。 「家庭的な―の店」「職場の―を壊す」「―のある俳優」	▲オフィスにはなごやかな雰囲気がある。 ▲お坊さんが突然大声で笑い出し、厳粛な雰囲気を台無しにした。
\\	分析	ぶんせき	[名]スル 
\\	複雑な事柄を一つ一つの要素や成分に分け、その構成などを明らかにすること。 「情勢の―があまい」「事故の原因を―する」 
\\	哲学で、複雑な現象・概念などを、それを構成している要素に分けて解明すること。 ↔総合。 
\\	物質の組成を調べ、その成分の種類や量の割合を明らかにすること。	▲あらゆる事態を個別に分析する必要がある。 ▲この分析では次の結果が出ている。
\\	文明	ぶんめい	人知が進んで世の中が開け、精神的、物質的に生活が豊かになった状態。 特に、宗教・道徳・学問・芸術などの精神的な文化に対して、技術・機械の発達や社会制度の整備などによる経済的・物質的文化をさす。 →文化[用法] [類語]文化・文物・文華・人文・人知・文運・開化・シビリゼーション	▲石油は文明の発達において重要な役割を果たしてきた。 ▲想像力は、すべての文明の根本である。
\\	分野	ぶんや	人間の活動における、分化した一つの領域。 物事のある方面・範囲。 「新しい―の研究」	▲この熱源は安心して使えるし、将来の予測もたてられるが、この分野での研究はさらに必要である。 ▲この分野では多くの科学者たちが研究している。
\\	塀・屏	へい	用心や目隠しのため、家や敷地の境界に建てた板・土・ブロックなどの障壁。	▲球はバウンドして塀を越えた。 ▲今、私が彼にやってもらいたいことは、塀のペンキ塗りです。
\\	平均	へいきん	[名]スル《古くは「へいぎん」とも》 
\\	大小・多少などの差が少なく、そろっていること。 また、そうすること。 ならすこと。 「年間を通じて売り上げが―している」 
\\	いくつかの数や量の中間的な値を求めること。 また、その数値。 それらの和をその個数で割る相加平均をいうことが多いが、ほかに相乗平均・調和平均などがある。 「―を上回る」「一日―乗降客数」「年―気温」 
\\	ほどよくつりあうこと。 均衡。 平衡。 バランス。 「―のとれたからだ」「―を保つ」 
\\	平定すること。 統一すること。 「大明、韃靼を―し」 [類語]
\\	均等・均分・等分・平準・標準・アベレージ(―する)均(なら)す・押し均す/
\\	平衡・均衡・衡平・つりあい・バランス	▲彼女は1週間に平均3、4冊の本を読む。 ▲彼は平均して7マイル歩く。
\\	平和	へいわ	[名・形動] 
\\	戦争や紛争がなく、世の中がおだやかな状態にあること。 また、そのさま。 「世界の―を守る」 
\\	心配やもめごとがなく、おだやかなこと。 また、そのさま。 「―な暮らし」 [類語]
\\	平安・平穏・安穏(あんのん)・平静・静穏/
\\	太平・昌平(しようへい)・安寧・静寧・和平	▲原子力エネルギーを平和のために利用することができる。 ▲原子力は平和目的に利用されることが望ましい。
\\	別に	べつに	[副](多く下に打消しの語を伴う)取り立てて言うほどではないさま。 これと言って特別に。 別段。 「―たいした用事ではない」	▲これは別に間違ったことではない。 ▲2、3の欠点を別にすれば、彼は信頼できるパートナーだ。
\\	減らす	へらす	[動サ五(四)] 
\\	物の数・量を少なくする。 減じる。 「負担を―・す」「体重を―・す」↔増やす/増す。 
\\	人をけなす。 人をへこませる。 「人ヲ―・ス」 [可能]へらせる	▲そのボクサーは試合のために体重を減らした。 ▲その国は輸入を減らそうとしている。
\\	減る	へる	[動ラ五(四)] 
\\	数・量・程度などが少なくなる。 とぼしくなる。 「収入が―・る」「水かさが―・る」↔増える/増す。 
\\	すりへる。 摩耗する。 「タイヤが―・る」「靴が―・る」 
\\	(「腹が減る」の形で)空腹である。 「腹が―・っては戦(いくさ)はできぬ」 
\\	(打消しの語を伴って用いる)気おくれする。 ひるむ。 臆する。 「口の―・らない若僧だ」「祐慶は少しも―・らず」	▲本に書かれていた全てのダイエットを試したが、まだ全然体重が減っていない。 ▲米の生産高は減ってきた。
\\	ベルト	ベルト	
\\	胴部に締める帯。 帯革。 バンド。 「安全―」 
\\	二つの車に掛け渡して、回転を伝えたり物をのせて移動させたりするための帯状のもの。 調べ革。 調べ帯。 「ファン―」 
\\	帯状をしている地域。 ベルト地帯。 「グリーン―」「太平洋―地帯」	▲彼女は腰にベルトを締めた。 ▲彼女は腰に革のベルトをしている。
\\	変化	へんか	[名]スル 
\\	ある状態や性質などが他の状態や性質に変わること。 「時代の―についていけない」「―に富む生活」「気温が急激に―する」 
\\	文法で、単語の語形が人称・数・格などに応じて変わること。 「動詞の語尾が―する」 [類語]
\\	変動・変転・変移・変遷・推移・転化・転変・変性・変質	▲この15年から20年の間に英国の家族生活には大きな変化があった。 ▲もっと仕事に変化があったならばなあ。
\\	ペンキ	ペンキ	《(オランダ)
\\	から》ペイント。 特に油ペイント。	▲その家は職人の手でペンキを塗られている。 ▲その家は父によってペンキが塗られていた。
\\	変更	へんこう	[名]スル決められた物事などを変えること。 「計画を―する」	▲その計画は変更の余地がない。 ▲その計画は現在では全く変更の余地がない。
\\	ベンチ	ベンチ	
\\	簡単な造りの長い腰掛け。 「公園の―」 
\\	野球場などで、監督・コーチ・選手などの控え席。 転じて、監督・コーチのこと。 「―から指示が出る」→ダッグアウト	▲彼は私たちに机とベンチを二つ作ってくれた。 ▲彼は公園のベンチに腰をかけた。
\\	弁当	べんとう	
\\	外出先で食べるために持っていく食べ物。 「手―」 
\\	料理店などで出す、主食と副食を箱などに詰めたもの。 「幕の内―」	▲私達は正午に弁当を食べた。 ▲自分の弁当を学校へ持ってきてもよろしい。
\\	法	ほう	Ⅰ(ハフ) 
\\	現象や事象などがそれに従って生起し、進展するきまり。 法則。 「自然には自然の―がある」 
\\	社会秩序を維持するために、その社会の構成員の行為の基準として存立している規範の体系。 裁判において適用され、国家の強制力を伴う。 法律。 「―のもとの平等」「民事訴訟―」 
\\	集団生活において、その秩序を維持するために必要とされる規範。 おきて。 しきたり。 道徳や慣習など。 「―にはずれたやり方」 
\\	物事をする定まったやり方。 正しいしかた・方法。 「―にかなった筆使い」「そんなばかな―はない」 
\\	珠算で、乗数。 または、除数。 →実 
\\	インド‐ヨーロッパ語で、文の内容に対する話し手の心的態度の相違が、動詞の語形変化の上に現れたもの。 直説法・接続法・希求法・命令法・条件法など。 叙法。 Ⅱ(ホフ)《梵
\\	の訳。 達磨・曇摩と音写。 保持するものの意》仏語。 
\\	永遠普遍の真理。 
\\	法則。 規準。 
\\	有形・無形の一切の存在。 また、その本体。 
\\	仏の教え。 仏法。 また、それを記した経典。 
\\	祈祷(きとう)。 また、その儀式。 「―を修する」	▲そんな法に従わなくてもよい。 ▲英国の測量法では4クオートは1ガロンだ。
\\	棒	ぼう	
\\	まっすぐで細長い木・竹や金属製のものなど。 「―でたたく」「天秤(てんびん)―」 
\\	棒術。 また、棒術に使う長さ六尺(約一・八メートル)ほどの丸いカシの木。 「―の使い手」 
\\	音楽の指揮棒。 「―を振る」 
\\	まっすぐ引いた太い線。 「不要な字句を―で消す」 
\\	疲れなどで足の筋肉や関節の自由がきかなくなること。 「足が―になる」	▲針ほどのことを棒ほどにいうのはどうかと思います。 ▲私は夏は凍った棒アイスクリームをしゃぶるのが好きです。
\\	冒険	ぼうけん	[名]スル危険な状態になることを承知の上で、あえて行うこと。 成功するかどうか成否が確かでないことを、あえてやってみること。 「前途に多くの―が待ち受ける」「―してみる価値がある」「―者」「―心」	▲彼らは場所から場所へと動き回り、よく職業を変え、より多く離婚し、危険と思える経済的、社会的冒険を冒す。 ▲彼は冒険好きだ。
\\	方向	ほうこう	
\\	物が向いたり、進んだりする方。 向き。 方角。 「北の―を指す」「右の―から風が吹く」「進行―」 
\\	気持ちや行動の向かうところ。 めざすところ。 方針。 「将来の―を決める」「妥協の―で話し合いがまとまる」 [用法]方向・方角――「県庁の方向(方角)に火の手が上がった」のように、向きの意では相通じて用いる。 
\\	「方向」は上下左右などに向いたり進んだりする向きで、特に東西南北を基準にはしない。 「右の方向へハンドルを切れ」「方向転換」
\\	軸の負の方向」
\\	「方角」はある地点を基準にした東西南北の向き。 「艮(うしとら)の方角を鬼門という」
\\	類似の語の「方位」は東西南北を基準として位置づけた向きの意で、専門用語として多く用いる。 「羅針盤で方位を確かめる」「方位計」	▲テニスは難しい。ボールがすぐあさっての方向に飛んでいってしまう。 ▲ハイカーは森の中でも方向が分かるようにコンパスを携帯する必要がある。
\\	報告	ほうこく	[名]スル告げ知らせること。 特に、ある任務を与えられた者が、その経過や結果などを述べること。 また、その内容。 「出張の―」「事件の顛末(てんまつ)を―する」「研究の中間―」	▲ある研究の報告によれば、間接喫煙の結果53、000のアメリカ人が毎年死亡しているそうだ。 ▲この報告はその事実を重視した。
\\	宝石	ほうせき	産出量が少なく、硬度が高くて、美しい光彩をもち、装飾用としての価値が高い非金属鉱物。 ダイヤモンド・エメラルド・サファイア・ルビーなど。	▲調べてみるとその宝石はイミテーションだと分かった。 ▲泥棒が入って、私の宝石類をみんな持っていってしまった。
\\	豊富	ほうふ	[名・形動]豊かであること。 ふんだんにあること。 また、そのさま。 「―な天然資源」「―な知識」 [派生]ほうふさ[名]	▲彼は人生経験が豊富である。 ▲彼は私たちのうちでもっとも経験豊富な議長の1人だ。
\\	方法	ほうほう	
\\	目標に達するための手段。 目的を遂げるためのやり方。 てだて。 
\\	哲学で、真理に到達するための考えの進め方。 →手段[用法] [類語]
\\	仕方・やり方・行き方・方途・方策・策(さく)・手段・手立て・手(て)・術(すべ)・法(ほう)・手法・技法・手順・方式・メソッド	▲我々はいろいろな方法でお互いにコミュニケーションする。 ▲我々は教育の方法を時代に呼応させなくてはならない。
\\	方方	ほうぼう	
\\	あちらこちら。 諸方。 諸所。 「―を捜す」「―旅行する」 
\\	さんざん。 今昔物語集26「―さる堪へ難き目を見て命を生きたる」	▲観光名所をほうぼう訪ね歩いたので、すっかり疲れ果ててしまった。
\\	訪問	ほうもん	[名]スル人をたずねること。 他人の家などをおとずれること。 「友人宅を―する」「会社―」	▲中川君を訪問したとき彼は出かけようとしていた。 ▲大臣は来週メキシコ訪問の予定です。
\\	吠える・吼える	ほえる	[動ア下一][文]ほ・ゆ[ヤ下二] 
\\	獣などが大声で鳴く。 「虎が―・える」 
\\	風・波などが、荒れて大きな音を立てる。 「荒海が―・える」 
\\	わめく。 どなる。 「壇上で―・える」	▲時々、私の犬は夜中の間に吠えます。 ▲隣の犬はほえてばかりいます。
\\	頬	ほお	顔の両面、耳と鼻・口との間の柔らかい部分。 ほほ。 「―がこける」「―を赤らめる」	▲恥ずかしさで彼の頬は真っ赤になっていた。 ▲恥ずかしさで彼のほおはかっと燃えていた。
\\	ボーイ	ボーイ	
\\	男の子。 少年。 年若い男。 ↔ガール。 
\\	食堂やホテルなどで、飲食物の給仕や客の世話などをする男性。 給仕。 ウエーター。	▲私は鞄をボーイに運んでもらった。 ▲彼は海辺のレストランのボーイである。
\\	ボート	ボート	
\\	オールで漕いで進む洋式の小舟。 端艇(たんてい)。 《季 夏》 
\\	船。 汽船。 「モーター―」「フェリー―」	▲大波で彼らのボートは転覆した。 ▲大波がその男をボートからさらっていった。
\\	ホーム	ホーム	
\\	家庭。 家。 「―バー」「マイ―」 
\\	故郷。 本国。 また、本拠地。 
\\	人々を収容する施設。 収容所・療養所など。 「老人―」 
\\	「ホームベース」の略。	▲彼女はもう少しでホームから落ちそうだった。 ▲彼は列車を待つ間、ホームを行ったり戻ったりした。
\\	ボール	ボール	
\\	球形のもの。 たま。 「ミラー―」「メンチ―」 
\\	球技に使う球。 「サッカー―」 
\\	野球で、投手の打者に対する投球でストライクと判定されなかったもの。 「ツーストライク、ワン―」	▲彼は場外にボールを飛ばした。 ▲彼女はミルクをボールに注いだ。
\\	誇り	ほこり	誇ること。 名誉に感じること。 また、その心。 「一家の―」「―を傷つけられる」	▲私の坊やたちをそれは誇りに思っています。 ▲私の娘のケイトは歌のコンテストで賞を取りました。私は彼女を誇りに思います。
\\	埃	ほこり	
\\	粉のような細かいちり。 「―がたまる」「―だらけの車」「砂―」 
\\	数量や金銭のはした。 余り。 「二千貫目足らずの商ひに九貫目の―を取り」	▲彼の部屋はほこりでいっぱいだった。 ▲道路はほこりで灰色になっていた。
\\	保証	ほしょう	[名]スル 
\\	間違いがない、大丈夫であると認め、責任をもつこと。 「品質を―する」「彼の人柄については―する」 
\\	債務者が債務を履行しない場合に、代わって債権者に債務を履行する義務を負うこと。 「―責任」	▲日本人は出来るだけ自分と同じような結婚相手を選んだり、安定と、ゆっくりではあるが着実な昇進とを保証する職業を探したり、銀行に貯金したりすることを好むように見える。 ▲日本は開発途上国に対し、20億円の包括援助を保証しました。
\\	保存	ほぞん	[名]スルそのままの状態に保っておくこと。 「文化財を―する」「永久―」	▲冷蔵庫は食べ物を保存するのに役立つ。 ▲彼らは果物を保存するために缶詰めにした。
\\	歩道	ほどう	人が歩くように車道と区別して設けた道。 人道。 「横断―」→車道	▲歩道にお金が落ちていた。 ▲歩道を歩きなさい。
\\	仏	ほとけ	《「ぶつ(仏)」の音変化した「ほと」に、目に見える形の意の「け」の付いた語で、仏の形、仏像が原義かという》 
\\	仏語。 悟りを得た者。 仏陀(ぶつだ)。 特に、釈迦(しやか)のこと。 「―の慈悲にすがる」 
\\	仏像。 また、仏の画像。 「―を刻む」 
\\	死者。 また、その霊。 「―になる」「―が浮かばれない」 
\\	温厚で慈悲心の深い人をたとえていう語。 
\\	仏法。 「―の御験(みしるし)はかやうにこそと」 
\\	仏事を営むこと。 「明後日(あさて)、―にいと善き日なり」 [下接語](ぼとけ)新(あら)仏・生き仏・石仏・板仏・懸け仏・木仏・外法(げほう)仏・骨(こつ)仏・小仏・新(しん)仏・杉仏・摺(す)り仏・立ち仏・土仏・流れ仏・撫(な)で仏・新(にい)仏・濡(ぬ)れ仏・寝仏・喉(のど)仏・野仏・星仏・守り仏・無縁仏・雪仏・笑い仏	▲仏の顔も三度。 ▲知らぬが仏。
\\	骨	ほね	㊀[名] 
\\	脊椎動物の内骨格を構成する構造物。 膠質(こうしつ)および石灰質を成分とし、骨組織・骨髄・軟骨組織・骨膜からなり、体の支持・運動や内臓の保護、骨髄での血球生成などの働きをする。 ふつうは硬骨をさし、膠質だけのものを軟骨という。 こつ。 
\\	建造物・器物などの形体を形づくって全体を支える材料。 「傘の―」「障子の―」 
\\	組織や物事などの中心となるもの。 また、人。 中核。 核心。 「会の―になる人」 
\\	何事にも屈しない強い気力。 気骨。 「―のある人」 
\\	遺骨。 また、死ぬこと。 ㊁[名・形動]困難であること。 骨が折れること。 また、そのさま。 「最後まで読むのはなかなか―だ」 [下接語]馬の骨・河(こう)骨・塗り骨・一(ひと)骨 (ぼね)肋(あばら)骨・鰓(えら)骨・大骨・親骨・貝殻骨・傘骨・蕪(かぶら)骨・気骨・首骨・腰骨・子骨・小骨・根性骨・繁(しげ)骨・筋骨・背骨・土性(どしよう)骨・喉(のど)骨・膝(ひざ)骨・平(ひら)骨・頬(ほお)骨・無駄骨・屋台骨	▲犬はその魚を骨も尾も全部食べた。 ▲犬はよく骨を地面に埋める。
\\	炎・焔	ほのお	《「火(ほ)の穂」の意》 
\\	気体が燃焼したときの、熱と光を発している部分。 液体・固体では、燃焼によって一部が気化し、反応している。 ふつう最下部の炎心、輝きの強い内炎、その外にあり完全燃焼している外炎の三つに分けられ、温度は外炎内側で最も高い。 火炎。 「真っ赤な―が上がる」 
\\	ねたみ・怒り・恋情など、心中に燃え立つ激しい感情をたとえていう語。 ほむら。 「嫉妬(しつと)の―に狂う」 [類語]
\\	火・炎(ほむら)・火炎(かえん)・光炎(こうえん)・紅炎(こうえん)・火柱(ひばしら)・火先(ほさき)	▲火山が炎と煙を噴き出している。 ▲火山は炎と溶岩を吹き出す。
\\	略・粗	ほぼ	[副]全部あるいは完全にではないが、それに近い状態であるさま。 だいたい。 おおよそ。 「物価が―二倍になる」「―満点の出来」	▲彼はもうほぼ父と同じ身長だ。 ▲彼はほぼ私の年です。
\\	微笑む・頬笑む	ほほえむ	[動マ五(四)] 
\\	声をたてずににっこり笑う。 微笑する。 ほおえむ。 「―・んだ顔」 
\\	花のつぼみが少し開く。 ほころびる。 ほおえむ。 「梅の花が―・みはじめる」	▲彼は微笑んで、別れを告げた。 ▲彼は立ち上がって彼女に微笑んだ。
\\	堀・濠・壕	ほり	
\\	土地を掘って水を通した所。 掘り割り。 
\\	敵の侵入を防ぐために、城の周囲を掘って水をたたえた所。	
\\	本人	ほんにん	
\\	その事に直接関係のある人。 当事者。 当人。 「―に確かめる」「―次第」 
\\	首領。 張本人。 「城の―平野将監入道」	▲できるだけ君本人が行って調べたほうがいいよ。 ▲本人は唯我独尊を決め込んでいるようだけども、周りから見れば単なるわがままだよね。
\\	本物	ほんもの	
\\	にせものや作りものでない、本当のもの。 また、本当のこと。 「―の真珠」「―の情報」 
\\	見せかけでなく実質を備えていること。 本格的であること。 「彼の技量は―だ」	▲私はそれは本物のピカソの画だと信じる。 ▲従ってそれ以来、製造業者たちは本物の現金を支給しなければなりませんでした。
\\	ぼんやり	ぼんやり	㊀[名]気持ちが集中せず間が抜けていること。 また、その人。 ㊁[副]スル 
\\	物の形や色などがはっきりせず、ぼやけて見えるさま。 「島影が―(と)見える」 
\\	事柄の内容などがはっきりしないさま。 「記憶が―(と)している」 
\\	元気がなく、気持ちが集中しないさま。 「終日―(と)過ごす」 
\\	気がきかず、間が抜けているさま。 「―(と)して手伝おうともしない」 [類語] 
\\	ぼうっと・茫(ぼう)と・ぼやっと・もやもや・おぼろ・おぼろげ・不鮮明・不明瞭・朦朧(もうろう)・模糊(もこ)・茫漠(ぼうばく)・茫茫(ぼうぼう)・漠(ばく)・漠然(ばくぜん)/
\\	ぼうっと・ぽうっと・茫(ぼう)と・ぽかんと・ぼけっと・ぼやっと・ぼやぼや・ぼさっと・ぼさぼさ・ぼそっと	▲遠くにぼんやりした明かりが見えた。 ▲警戒していた警備員が遠くのぼんやりとした影に気づいた。
\\	間	ま	㊀[名] 
\\	物が並んでいるときの空間。 あいだ。 あい。 すきま。 「車と車との―を置く」 
\\	家のひと区切りをなしている部屋。 「次の―に控える」 
\\	畳の大きさを表す名称。 「京―」「江戸―」 
\\	連続している事と事のあいだの時間。 ひま。 いとま。 「食事をする―もない」 
\\	話の中に適当にとる無言の時間。 「話は―が大切だ」 
\\	邦楽・舞踊・演劇などで、拍と拍、動作と動作、せりふとせりふなどのあいだの時間的間隔。 転じて、リズムやテンポの意に用いる。 「―をとる」「―を外す」 
\\	ちょうどよい折。 しおどき。 ころあい。 機会。 「―を見計らう」 
\\	その場のようす。 その場のぐあい。 
\\	家などの柱と柱との間。 けん。 「我は南の隅の―より格子叩きののしりて入りぬ」 ㊁〔接尾〕助数詞。 
\\	部屋の数を数えるのに用いる。 「六畳と四畳半の二(ふた)―」 
\\	柱と柱のあいだを単位として数えるのに用いる。 「勢多の橋をひと―ばかりこぼちて」 
\\	建物や部屋の広さをいうのに用いる。 
\\	をもとにして、縦一間(ひとま)・横一間の広さを一間(ひとま)とする。 「六―の客殿へ跳り出で」 
\\	障子の桟(さん)で囲まれた一区切りなど、一定の区切られた空間を数えるのに用いる。 「明かり障子の破ればかりを…なほ一―づつ張られけるを」 [下接語]間(あい)の間・合間・空き間・雨(あま)間・生け間・伊勢(いせ)間・板の間・田舎間・居間・岩間・畝(うね)間・江戸間・応接間・大間・奥の間・落ち間・鏡の間・額の間・陰間・風(かざ)間・貸し間・株間・上(かみ)の間・客間・京間・切れ間・雲間・下段の間・格(ごう)間・木(こ)の間・小間・作間・狭(さ)間・鞘(さや)の間・三の間・潮間・借間・上段の間・透き間・絶え間・谷間・近間・茶の間・中京間・中(ちゆう)間・ちょんの間・束(つか)の間・次の間・露の間・手間・殿上の間・胴の間・時の間・床の間・土間・中の間・仲間・波間・日本間・寝間・狭(はざ)間・階隠(はしがく)しの間・梁(はり)間・晴れ間・半間・庇(ひさし)の間・一(ひと)間・昼間・広間・深間・仏間・不(ぶ)間・別間・本間・瞬く間・間間・雪間・洋間・欄間	▲鬼の居ぬ間に洗濯。
\\	まあ	まあ	㊀[副] 
\\	とりあえずするように勧めるさま。 何はともあれ。 まず。 「話はあとにして、―一杯どうぞ」「―お掛けください」 
\\	結果に自信を持てないが、一応してみるさま。 とにかく。 「ちょっと厄介だが―やってみるか」 
\\	多少のためらいをもちながら、意見を述べるさま。 「―やめたほうがいい」「―彼が勝つだろう」 
\\	十分ではないが、一応はがまんできる程度であるさま。 「―よくできたほうだ」 ㊁[感]驚きや意外な気持ちを表す語。 あら。 おや。 「―、お久しぶり」「―、そうでしたか」「―、失礼ね」	▲まあ、そう言ってくださって本当にありがとう。 ▲まぁ、そんなに格好つけるのやめとけよ。
\\	マーケット	マーケット	
\\	食料品や日用品などを商う店が集まっている場所や建物。 市場(いちば)。 「スーパー―」 
\\	市場(しじよう)。 「―を広げる」	▲母はマーケットへ買い物に行きました。 ▲コミックマーケット67にてコピー本販売。無事完売したので製本しました。
\\	迷子	まいご	《「まよいご」の音変化》道がわからなくなったり、連れにはぐれたりすること。 また、その子供やその人。 まよいご。	▲私はデパートで迷子になった少年を助けた。 ▲私はどこへいっても迷子になる。
\\	任せる・委せる	まかせる	[動サ下一][文]まか・す[サ下二] 
\\	仕事などを他にゆだね、その自由にさせる。 「経営を―・せる」 
\\	相手の好きなようにさせる。 「御想像に―・せます」「身を―・せる」 
\\	そのままにしておく。 ほうっておく。 「髪の毛が乱れるに―・せる」「成り行きに―・せる」 
\\	(「…にまかせて」の形で)自然の勢いのままにする。 「口に―・せてでたらめをいう」「足に―・せて歩く」 
\\	従う。 「然ればすなはち先例に―・せて」	▲後は任せた! ▲彼はいっさいを運に任せた。
\\	幕	まく	㊀[名] 
\\	布を縫い合わせるなどして作り、仕切りや隔て、また装飾として垂らしたりめぐらしたりするもの。 「―を張る」「紅白の―」 
\\	劇場などで、舞台の前面に垂らし、舞台と客席とを仕切る布。 「―を下ろす」 
\\	演劇で、幕 
\\	を開けてから閉じるまでのひとまとまりの場面。 脚本全体の大きな段落。 幕の開閉を伴わない小段落を場というのに対する。 「最初の―の通行人役を演じる」 
\\	演劇で、幕 
\\	を引いてある場面を終わりにすること。 「見得を切って―にする」 
\\	場面。 場合。 「君などの出る―ではない」 
\\	物事の終わり。 「騒動もこれで―となる」 
\\	相撲で、幕内。 「―下」 ㊁〔接尾〕助数詞。 演劇の一段落を数えるのに用いる。 「二―三場」「一―物」	▲万雷のうちに幕が下りた。 ▲その物語は彼の死で幕を閉じる。
\\	負け	まけ	
\\	まけること。 敗北。 ↔勝ち。 
\\	値段を安くすること。 また、その代わりのもの。 →お負け 
\\	名詞に付いて、その事柄において他に圧倒される、それに値しない、などの意を表す。 「気力―」「根―」「位(くらい)―」「名前―」	▲君の負けだ、降参しろ。 ▲今の負けは御愛嬌さ。
\\	孫	まご	《「うまご」の音変化》 
\\	子の子。 
\\	もとのものから間を一つ隔てること。 また、そのような関係。 「―弟子」「―引き」	▲彼は孫達に囲まれて座っていました。 ▲彼は孫への愛におぼれている。
\\	誠に・真に・実に	まことに	㊀[副]まちがいなくある事態であるさま。 じつに。 本当に。 「―彼女は美しい」「―ありがとうございます」 ㊁[感]「まこと 
\\	に同じ。 「―雪少しうち散りて、折節とり集めて、さることやは候ひしとよ」	
\\	正に	まさに	[副] 
\\	ある事が確かな事実であるさま。 まちがいなく。 本当に。 「事実は―予言のとおりだった」 
\\	実現・継続の時点を強調するさま。 ちょうど。 あたかも。 「彼は―車から降りた瞬間、凶弾に倒れた」 
\\	《漢文訓読から起こった用法》 ㋐(「当に」とも書く。 「まさに…べし」などの形で)当然あることをしなければならないさま。 ぜひとも。 「学生たる者―学問に励むべきだ」 ㋑(「将に」とも書く)(「まさに…せんとする」などの形で)ある事が実現しそうだという気持ちを表す語。 今にも。 「飛行機が―飛び立とうとしている」 
\\	(主に、あとに反語表現を伴って)どうして…しようか。 「あやしかりつるほどのあやまりを、―人の思ひとがめじや」	▲まさに臨機応変の対応。見事というべきだね。 ▲太陽がまさに沈むところだった。
\\	増す・益す	ます	[動サ五(四)] 
\\	数量や程度が大きくなる。 ↔減る。 ㋐多くなる。 ふえる。 「体重が―・す」 ㋑高まる。 進む。 「秋になると食欲が―・す」「不安が―・す」 ㋒(「…にもまして」の形で)あるものよりも、もっと程度が上であることを表す。 「前にも―・して元気になる」 
\\	数量や程度を大きくする。 ↔減らす。 ㋐加える。 また、加えて大きくする。 ふやす。 「人員を―・す」「紅葉が渓谷の景観を―・す」 ㋑高める。 伸ばす。 進める。 「興味を―・す」「親しみを―・す」 ㋒すぐれるようにする。 まさらせる。 「待てと言ふに散らでしとまるものならば何を桜に思ひ―・さまし」 [可能]ませる [用法]ます・ふやす――「権力が増す」「人気が増す」「水かさが増す」のように、「増す」が「が」をともなう場合は、物の量・程度が多くなる意で用いる。 
\\	「速度を増す」「明るさを増す」「人手を増す」と「増す」が「を」をとる場合、物の量・程度を多くする意で用いる。 
\\	「ふやす」は「貯金をふやす」「文庫の本をふやす」のように「を」をとる用法だけで、物の数・量を多くする意に使う。 
\\	類似の語の「ふえる」は、「町の人口がふえた」「体重が五キロふえた」と「が」をともない、具体的な物の数・量が多くなる意に用いる。	▲知識を増す者は悲哀を増す。 ▲台風が勢いを増した。
\\	貧しい	まずしい	[形][文]まづ・し[シク] 
\\	財産や金銭がとぼしく、生活が苦しい。 貧乏である。 「暮らしが―・い」「―・い家に生まれる」 
\\	量・質ともに劣っている。 粗末である。 貧弱である。 乏しい。 「―・い食卓」「想像力の―・い人」「―・い経験」 
\\	満たされていない。 「心の―・い人」 [派生]まずしげ[形動]まずしさ[名]	▲彼は貧しい人に援助の手を差し伸べた。 ▲彼は貧しい人のために多くのことをしてきた。
\\	マスター	マスター	[名]スル 
\\	師匠。 親方。 
\\	集団の責任者。 長。 「バンド―」「コンサート―」 
\\	経営者。 主人。 特に、バー・喫茶店などの主人、または支配人。 
\\	学位の一。 修士。 
\\	物事に熟達すること。 習得すること。 「中国語を―する」 
\\	多く複合語の形で用い、元になるもの、基本となるもの、の意を表す。 「―テープ」	▲英語をマスターする事は簡単でない。 ▲英語を短期間にマスターすることは不可能です。
\\	間違い	まちがい	
\\	真実と違うこと。 誤り。 まちがえ。 「―を正す」 
\\	しくじり。 過失。 あやまち。 まちがえ。 「―がないか確かめる」 
\\	異常な出来事。 事故。 まちがえ。 「途中で何か―でもなければよいが」 
\\	情事、特に分別のない男女の関係についていう。 まちがえ。 「若い男女が―を起こす」	▲僕は妻を選ぶのに大変な間違いをした。 ▲万一間違いを見つけたら、直ちに知らせてください。
\\	松	まつ	
\\	マツ科マツ属の常緑高木の総称。 明るく乾燥した地に生え、樹皮はひび割れするものが多い。 葉は針状で、ふつうアカマツ・クロマツなどでは二本、ゴヨウマツ・チョウセンゴヨウ・ハイマツなどでは五本が束になって出る。 春、球状の雌花と雄花とがつき、黄色い花粉が風に飛ぶ。 果実は松かさとよばれ、多数の硬い鱗片(りんぺん)からなる。 種子は食用。 材は薪炭・松明(たいまつ)・建築・パルプなどに広く用いられ、また松脂(まつやに)をとる。 竹・梅あるいは鶴とともにめでたい取り合わせとされ、正月の門松にする。 翁草・千代見草・常盤草(ときわぐさ)など異称も多い。 《季 花=春 落葉=夏》「線香の灰やこぼれて―の花/蕪村」 
\\	門松(かどまつ)。 また、門松を飾っている期間。 「―が取れる」「―の内」 
\\	松明(たいまつ)。 「月のない晩だったから、私は―などお持たせするように言いつけた」 
\\	紋所の名。 松の幹・枝・葉または松かさを図案化したもの。 
\\	遊女の最高の位。 松の位。 「この子は―に極めて」 
\\	マツタケをいう女房詞。 ◆和歌では「待つ」と掛けて用いられる。 「立ち別れいなばの山の峰におふるまつとし聞かば今かへりこむ」 [下接語]相生(あいおい)の松・赤松・アメリカ松・磯(いそ)松・一の松・市松・美し松・海松・蝦夷(えぞ)松・老い松・拝み松・雄(お)松・鏡の松・笠(かさ)松・飾り松・門(かど)松・傘(からかさ)松・唐松・ぐい松・黒松・腰掛け松・小松・五葉松・下がり松・曝(さ)れ松・三蓋(さんがい)松・三の松・霜降り松・磯馴(そな)れ松・朝鮮松・椴(とど)松・鳥総(とぶさ)松・二の松・子(ね)の日の松・這(はい)松・柱松・姫松・米(べい)松・見越しの松・夫婦(めおと)松・雌(め)松・琉球(りゆうきゆう)松・若松	
\\	真っ赤	まっか	[名・形動] 
\\	非常に赤いこと。 充血して赤いこと。 また、そのさま。 「―な花」「西の空が―だ」「―になって怒る」「目を―にする」 
\\	全くそうであるさま。 まるっきり。 「―なにせ物」	▲彼は非常に怒って、顔を真っ赤にしていた。 ▲彼は真っ赤になって怒っていた。
\\	全く	まったく	[副]《形容詞「まったい」の連用形から》 
\\	完全にその状態になっているさま。 すっかり。 「―新しい企画」「回復の希望は―絶たれた」 
\\	打消しの語を伴って、完全な否定の意を表す。 決して。 全然。 「彼は事件とは―関係がない」「―話にならない」 
\\	ある事実・判断を強調する気持ちを表す。 本当に。 実に。 「今日は―寒い」「―けしからん話だ」「―君の言う通りだよ」 [類語]
\\	すっかり・完全に・全面的に・百パーセント/
\\	全然・一向・まるきり・まるで・皆目(かいもく)・からきし・さっぱり・とんと・ちっとも・少しも・いささかも・何ら・毫(ごう)も・微塵(みじん)も・毛頭(もうとう)・露(つゆ)・更更(さらさら)/
\\	実に・本当に・まことに・何とも・実以(もつ)て	▲あと一台パソコンを購入する君の案はまったく問題にならない。 ▲彼女の勤勉はまったく賞賛に値する。
\\	祭り	まつり	①まつること。祭祀。祭礼。俳諧では特に夏祭をいう。 ②特に、京都賀茂神社の祭の称。葵祭あおいまつり。蜻蛉日記上「このごろは四月、―見にいでたれば」 ③近世、江戸の二大祭。日吉山王神社の祭と神田明神の祭。 ④記念・祝賀・宣伝などのために催す集団的行事。祭典。「梅―」「港―」	
\\	学ぶ	まなぶ	《「まねぶ」と同語源》 ㊀[動バ五(四)] 
\\	勉強する。 学問をする。 「大学で心理学を―・ぶ」「同じ学校で―・んだ仲間」 
\\	教えを受けたり見習ったりして、知識や技芸を身につける。 習得する。 「よく―・びよく遊べ」 
\\	経験することによって知る。 「苦労して人間のすばらしさを―・んだ」 
\\	まねをする。 「五月に雨の声を―・ぶらむもあはれなり」 [可能]まなべる ㊁[動バ上二]《「まねぶ」が和文体に多いのに対して、漢文訓読体に多く見られる》 ❶に同じ。 「僧多かれど、―・ぶる所少し」 [類語] 
\\	勉強する・勉学する・学習する・学修する・専修する・修業(しゆうぎよう・しゆぎよう)する・修める・習う・習得する・教わる/
\\	知る・理解する・体得する・会得する・習得する・覚える・身に付ける	▲先月、生け花を学び始めたばかりですので、まだ、初心者です。 ▲青年がまず第一に学ぶべき重要な教訓は、自分は何も知らないということである。
\\	真似	まね	[名]スル 
\\	まねること。 また、形だけ似た動作をすること。 模倣。 「ボールを投げる―をする」「アメリカ映画の―をする」 
\\	行動。 ふるまい。 「ばかな―はよせ」	▲子供は両親よりもむしろ友達の真似をする。 ▲子供みたいな真似をするのはやめなさい。
\\	招く	まねく	[動カ五(四)] 
\\	合図をして人を呼び寄せる。 「手を振って―・く」 
\\	客として来るように誘う。 招待する。 「歓迎会に―・かれる」 
\\	ある目的のために、礼をつくして来てもらう。 また、しかるべき地位を用意して、人に来てもらう。 招聘(しようへい)する。 「作家を―・いて講演会を開く」「ゲストに―・く」「教授として―・く」 
\\	好ましくない事態を引き起こす。 もたらす。 「惨事を―・く」「誤解を―・く」 [可能]まねける [類語]
\\	呼ぶ・呼び寄せる・差し招く・手招きする/
\\	招待する・呼ぶ・招(しよう)ずる・請(しよう)ずる・迎える・誘(さそ)う・誘(いざな)う/
\\	招聘(しようへい)する・招致する・招請する・聘(へい)する/
\\	もたらす・持ち来(きた)す・来す・引き起こす・生む・将来する・招来する・誘発する・惹起(じやつき)する	▲戦争は必ず不幸を招く。 ▲戦争は不足と欠乏の時代を招いた。
\\	ママ	ママ	
\\	母親。 おかあさん。 また、子供などが母親を呼ぶ語。 ↔パパ。 
\\	酒場などの女主人。 マダム。	▲ママのミルクはもう飲んだでちょ?泣かないでネンネしてくだちゃ〜い。 ▲台所でママのお手伝いをしたの。
\\	豆・荳・菽	まめ	㊀[名] 
\\	マメ科植物の種子。 特にそのうち、食用にするものの総称。 大豆(だいず)・小豆(あずき)・ササゲ・エンドウ・ソラマメ・インゲンマメ・ラッカセイなど。 マメ科の双子葉植物は約一万三〇〇〇種が寒帯から熱帯まで広く分布し、草本または木本。 葉は複葉で、花は蝶形花が多い。 果実は豆果で、莢(さや)の中に種子がある。 種子は胚乳が発達せず、子葉が発達して大部分を占め、でんぷんや脂肪を蓄える。 
\\	特に、大豆。 「―かす」「―細工」 
\\	女性の陰部。 特に陰核をいう。 
\\	料理に使う、豚・牛などの腎臓。 ㊁〔接頭〕名詞に付く。 
\\	形や規模などが小さい意を表す。 「―電球」「―台風」 
\\	子供である意を表す。 「―記者」 [下接語]青豆・畦(あぜ)豆・煎(い)り豆・隠元豆・鶯(うぐいす)豆・鶉(うずら)豆・打ち豆・枝豆・阿多福(おたふく)豆・籬(かき)豆・カラバル豆・金時豆・黒豆・源氏豆・コーヒー豆・五目豆・砂糖豆・莢(さや)豆・三度豆・塩豆・白豆・底豆・空豆・狸(たぬき)豆・痰(たん)切り豆・血豆・蔓(つる)豆・年の豆・鉈(なた)豆・夏豆・南京(ナンキン)豆・煮豆・羽団扇(はうちわ)豆・弾け豆・八升豆・雛(ひよこ)豆・富貴(ふき)豆・福豆・藤(ふじ)豆・味噌(みそ)豆・蜜(みつ)豆	▲まめのかきあげが好きです。 ▲右の鼻に豆が入ってしまいました。
\\	守る・護る	まもる	[動ラ五(四)]《「目(ま)守(も)る」の意》 
\\	侵されたり、害が及ばないように防ぐ。 「犯罪から青少年を―・ろう」「身を―・る術」 
\\	決めたことや規則に従う。 「約束を―・る」「教えを―・る」 
\\	相手の攻撃に備え、守備する。 「ゴールを―・る」「外野を―・る」 
\\	目を離さずに見る。 みまもる。 「庄兵衛は喜助の顔を―・りつつ」 
\\	様子を見定める。 「近江(あふみ)の海波恐(かしこ)みと風―・り年はや経なむ漕ぐとはなしに」 [可能]まもれる [類語]
\\	庇(かば)う・保護する・擁護する・庇護(ひご)する・守護する・防護する・ガードする・警護する・警衛する・護衛する/
\\	遵守(じゆんしゆ)する・厳守する・遵奉する・護持する・堅持する・固守する/
\\	防衛する・防御する・自衛する・守備する・固守する・死守する	▲ゴーグルは砂ぼこりから目を守ってくれる。 ▲私はいつも締め切りを守っている。
\\	丸・円	まる	《「まろ」の音変化》 ㊀[名] 
\\	まるい形。 円形。 また、球形。 「該当する項目を―で囲む」 
\\	答案などに正解または合格・優良の評価の意味でつけるまるい印。 「正しい答えに―をつける」「図画で三重―をもらう」 
\\	㋐句点。 文の終わりにつける「。 
\\	の符号。 ㋑半濁点。 「ぱ」「ピ」などの半濁点。 
\\	数字の零を読み上げるときにいう語。 「一(いち)―三(さん)―(=一〇時三〇分)に到着」 
\\	金銭のこと。 会話で親指と人差し指とで輪をつくって示すこともある。 
\\	㋐《甲羅が円形であるところから》主に関西で、スッポンのこと。 「鯛と烏賊のつくり合せや、―の吸物に」 ㋑料理に使う骨付きのドジョウ。 また関西で、ウナギのこと。 
\\	城郭の内部。 近世の城郭で内郭・外郭の外周をいい、その位置から本丸、二の丸、三の丸などと称する。 「―の内」 
\\	円形の紋所の名。 円形単独のもののほか、薄(すすき)の丸、鶴の丸など他の模様と組み合わせたものもある。 
\\	完全で、欠けたところのないこと。 また、全部を包含していること。 まるごと。 「リンゴを―のままかじる」「―に一夜さ添ひ果てず」 
\\	㋐重さの単位。 一丸は五〇斤で、約三〇キロ。 ㋑和紙を数える単位。 一丸は、半紙では六締め、奉書紙では一〇束。 
\\	江戸の吉原遊郭で、遊女の揚代が倍額になる日。 正月や節句の日など。 丸の日。 ㊁〔接頭〕 
\\	数詞に付いて、その数が欠けることなく満ちている意を表す。 「―一日」「―一月(ひとつき)」 
\\	名詞に付いて、完全にその状態である、の意を表す。 全体。 そっくり。 「―もうけ」「―焼け」 ㊂〔接尾〕《「まろ(麻呂)」から転じて》 
\\	人名、特に稚児に用いる。 「石童―」「牛若―」 
\\	船の名に用いる。 「海神―」 
\\	刀・楽器その他の器物の名に用いる。 「蜘蛛切(くもきり)―」 
\\	犬や馬などの名に用いる。 「常陸(ひたち)―」「木下―」 [下接語]角(かど)丸・北の丸・黒丸・三の丸・樽(たる)丸・手丸・出丸・問(とい)丸・唐丸・胴丸・西の丸・二重丸・二の丸・日の丸・本丸・丸丸・真ん丸	▲赤丸で囲む。 ▲今日から丸1日お願いします。
\\	丸で	まるで	[副] 
\\	違いがわからないほどあるものやある状態に類似しているさま。 あたかも。 さながら。 「この惨状は―地獄だ」「―夢のよう」 
\\	(下に否定的な意味の語を伴って)まさしくその状態であるさま。 すっかり。 まったく。 「―だめだ」「兄弟だが―違う」	▲いっしょにスケートをしたのが、まるで昨日のことみたいです。 ▲お二人はまるで兄弟のようによく似ています。
\\	回す・廻す	まわす	[動サ五(四)] 
\\	軸を中心にして、円を描くように動かす。 回転させる。 「腕を―・す」「プロペラを―・す」 
\\	周囲を取り巻くようにする。 めぐらす。 「敷地に柵を―・す」 
\\	順に送り渡す。 「杯を―・す」「通知を―・す」 
\\	人や物を必要とする場所へ移す。 「総務から営業へ人員を―・す」「出先に車を―・す」「預金の一部を学費に―・す」 
\\	その立場に置く。 「敵に―・す」 
\\	配慮などを行き渡らせる。 「気を―・す」「手を―・す」 
\\	利益を得るように金銭を運用する。 「高利で―・す」 
\\	㋐自分の意のままに人を使う。 「あの女に―・さるる女郎いとしや」 ㋑遊里で、客が遊女や幇間(ほうかん)を思うままに従わせる。 「太鼓持ちは、ある知恵を隠して、我より鈍い客に―・さるるがよし」 
\\	動詞の連用形に付いて、全体に…する、あちこち…する、また、さんざん…する、の意を表す。 「いじくり―・す」「追っかけ―・す」「引っ張り―・す」 [可能]まわせる [下接句]気を回す・手を回す・向こうに回す・目を回す・悪気(わるぎ)を回す	▲いや、それじゃだめだ。逆になるようにまわしてごらん。 ▲書類を回してくださいませんか。
\\	万一	まんいち	㊀[名]万の中に一つ。 めったにないが、ごくまれにあること。 まんがいち。 「―に備える」 ㊁[副]めったにないことが起こるのを予測するさま。 もしも。 まんがいち。 「―火事になったら、これを持って逃げろ」	▲万一雨が降ったら試合は中止になるだろう。 ▲万一雨が降ったら旅は楽しくないだろう。
\\	満足	まんぞく	[名・形動]スル 
\\	心にかなって不平不満のないこと。 心が満ち足りること。 また、そのさま。 「―な(の)ようす」「今の生活に―している」 
\\	十分であること。 申し分のないこと。 また、そのさま。 「―な答え」「料理も―にできない」 
\\	数学で、ある条件を満たしていること。 [派生]まんぞくげ[形動]まんぞくさ[名] [類語]
\\	満悦・充足・飽満・自足・自得・会心・充足感・充実感(―する)満ち足りる・心行く・堪能(たんのう)する・満喫する・安んずる・甘んずる	▲その説明は決して満足すべきものではない。 ▲その説明は決して満足するものではない。
\\	身	み	《「実」と同語源》 ㊀[名] 
\\	生きている人間のからだ。 身体。 「茂みに―を隠す」「装飾品を―につける」 
\\	わが身。 自分自身。 「―を犠牲にする」「だまっている方が―のためだぞ」「―の危険を感じる」 
\\	自分が何かをやろうとする心。 誠心。 「勉強に―が入らない」 
\\	地位。 身分。 立場。 「―のほどをわきまえる」「家族を扶養する―」「他人の―になって考える」 
\\	皮や骨に対し、食べられる部分。 肉。 「魚の―をむしる」「―の小さな蛤(はまぐり)」 
\\	容器の、ふたに対して物を入れる部分。 また、ふた付きの鏡などで、ふたに対して、本体のほう。 
\\	衣服の袖・襟・衽(おくみ)などを除き、胴体を覆う部分。 身頃(みごろ)。 
\\	刀の、鞘(さや)に収まっている刃の部分。 刀身。 
\\	木の、皮に包まれた部分。 
\\	身ぶり。 「少し案じる―ありてうなづき」 ㊁[代] 
\\	一人称の人代名詞。 わたし。 それがし。 中世・近世で用いられた上品でやや尊大な言い方。 「―が申すやうは」 
\\	二人称の人代名詞。 「お」「おん」に続けて用いられる。 →御身(おみ)・(おんみ) [下接語]赤身・当たり身・当て身・脂身・新(あら)身・生き身・入り身・浮き身・憂き身・受け身・後ろ身・打ち身・現(うつ)し身・上(うわ)身・大身・御(お)身・御(おん)身・影身・片身・肩身・空(から)身・変わり身・黄身・切り身・黒身・笹(ささ)身・刺し身・差し身・下身・死に身・白身・親身(しんみ)・剥(す)き身・捨て身・擂(す)り身・総身・反り身・立ち身・作り身・中身・長身・生(なま)身・抜き身・裸(はだか)身・肌身・半身・一つ身・人身・独り身・不死身・古身・細身・骨身・本身・前身・三つ身・剥(む)き身・四つ身・寄り身・我が身	▲冷たい風が彼のコートを通して身にしみた。 ▲まあちょっと私の身になってくださいよ。
\\	見送り	みおくり	
\\	見送ること。 また、その人。 「―を受ける」 
\\	実行をさしひかえてようすを見ること。 「昇格は―となる」 
\\	野球で、打者が投球に対してバットを振らずにいること。 「―の三振」 
\\	相場のようすを眺めるだけで売買を手控えること。	▲彼は駅まで彼女を見送りにいってきたところだ。 ▲彼は駅へ祖父を見送りに言った。
\\	味方・御方・身方	みかた	[名]スル《「かた」の敬称「御方」の意。 「味方」「身方」は当て字》 
\\	対立するものの中で、自分が属しているほう。 また、自分を支持・応援してくれる人。 「心強い―」↔敵。 
\\	対立するものの一方を支持したり、応援したりすること。 「いつも女性に―する」 
\\	天皇の軍勢。 官軍。 「―の軍おぢおそれ三たび退き返る」	▲彼の味方になる。 ▲当時彼と私は味方同士だった。
\\	見事・美事	みごと	
\\	❸が原義。 「美事」は当て字》 ㊀[形動][文][ナリ] 
\\	すばらしいさま。 りっぱなさま。 「バラが―に咲く」「―な床柱」 
\\	巧みなさま。 あざやか。 「予想が―に的中した」「―な腕前」 
\\	完全であるさま。 すっかり。 「ものの―に失敗した」「―な負けぶりだ」 [派生]みごとさ[名] ㊁[副] ❶に同じ。 「―やり通した」 ㊂[名]みるべきこと。 みるべき価値のあるもの。 「―いとおそし。 そのほどは桟敷不用なり」	▲彼はそれをひとりで、しかも見事にやってのけた。 ▲彼はその役を見事に演じた。
\\	ミス	ミス	
\\	未婚女性の姓または姓名の前につける敬称。 嬢(じよう)。 
\\	未婚女性。 独身の女性。 「ハイ―」 
\\	名詞の上に付けて、それを代表する美人として選ばれた未婚女性。 美人コンテストなどの優勝者。 「―ワールド」	▲新しい調査では、65歳以上の病院患者の診察記録には誤りが多く、重大な診察ミスにつながりかねない、ということです。 ▲熟練したドライバーでもミスをすることがある。
\\	満ちる・充ちる	みちる	[動タ上一][文]み・つ[タ上二] 
\\	一定の枠、空間や限界を越えそうになるほどいっぱいになる。 あふれる。 「会場は熱気に―・ちている」「香りが部屋に―・ちる」「危険に―・ちた戦線」 
\\	ある感情・気持ちなどがいっぱいにゆきわたる。 「希望に―・ちた青春」「敵意に―・ちる」 
\\	整って欠けたところがなくなる。 特に、満月になる。 「月が―・ちる」↔欠ける。 
\\	海面の水位が最も高くなる。 満潮になる。 「潮が―・ちる」 
\\	決められた期間・期限に達する。 一定の期間が終わる。 「任期が―・ちる」「月が―・ちて子供が生まれる」	▲彼女の言葉は憂鬱に満ちていた。 ▲潮が満ち始めている。
\\	認める	みとめる	[動マ下一][文]みと・む[マ下二] 
\\	目にとめる。 存在を知覚する。 気づく。 「人影を―・めた」「どこにも異常は―・められない」 
\\	見て、また考えて確かにそうだと判断する。 「有罪と―・める」「頭がよいと―・める」 
\\	正しいとして、また、かまわないとして受け入れる。 「自分の非を―・める」「試験に教科書の持ち込みを―・める」 
\\	能力があると判断する。 「世に―・められる」 
\\	気をつけて見る。 じっと見る。 「五百の仏を心静かに―・めしに」 [類語]
\\	見る・目撃する・確認する・認知する・看取する・見て取る・見取る/
\\	見なす・判断する・判定する・認定する/
\\	承認する・同意する・肯定する・うべなう・うけがう・是認する・容認する・認容する・許容する・許可する・認許する・認可する・公認する・許す/
\\	買う・評価する・一目置く	▲我々は彼の才能を認めている。 ▲彼は女性の喫煙をいいものだと認めない。
\\	見舞い	みまい	
\\	病人や災難にあった人などを訪れて慰めたり、書面などで安否をたずねたりすること。 また、その手紙や贈り物。 「病人の―に行く」「暑中―」「火事―」「陣中―」 
\\	見回ること。 巡視。 巡回。 「今日は畠へ―に参らばやとぞんずる」 
\\	訪ねること。 訪問。 「久しう―にもまゐらぬ程に、―に参らうと思うて出てをりやらします」	▲彼らは私の見舞いに病院へ来た。 ▲彼は病気の友を毎日見舞いに来る。
\\	土産	みやげ	
\\	外出先や旅先で求め、家などに持ち帰る品物。 
\\	他人の家を訪問するときに持っていく贈り物。 手みやげ。 「―に酒を持参する」 
\\	迷惑なもらい物を冗談めかしていう語。 「伝染病という、とんだ―をもって帰国した」	▲彼は私たちを訪ねるたびにちょっとした土産を持ってきた。 ▲旅行から帰るときの土産のことを考えておきなさい。
\\	都	みやこ	《「宮(みや)処(こ)」の意》 
\\	皇居のある土地。 「―を定める」「京の―」 
\\	その国の中央政府の所在地。 首都。 首府。 また一般に、人口が多く、政治・経済・文化などの中心となる繁華な土地。 都会。 「住めば―」 
\\	何かを特徴としたり、何かが盛んに行われることで人が集まったりする都会。 「音楽の―ウィーン」「水の―ベニス」 
\\	天皇が仮の住居とする行宮(あんぐう)。 「秋の野のみ草刈り葺(ふ)き宿れりし宇治の―の仮廬(かりいほ)し思ほゆ」	▲彼は銀の木がある黄金の都についてすばらしい話を聞いたことがあった。 ▲日本では8世紀には既に立派な都がつくられていました。
\\	妙	みょう	[名・形動] 
\\	いうにいわれぬほどすぐれていること。 きわめてよいこと。 また、そのさま。 「演技の―」「自然の―」「言い得て―だ」 
\\	不思議なこと。 奇妙なこと。 また、そのさま。 「―な事件」「夜中に―な音がする」「―に憎めない人」 
\\	《「妙」の字を分解すると「少女」となるところから》寺の囲い女。 僧侶たちの間で用いた語。 「庫裡(くり)から―が粗忽(そこつ)に出でて言ひけるは」	▲あなたのは初めて耳にする妙なお話です。 ▲彼が解雇されるとは妙な話だ。
\\	未来	みらい	
\\	現在のあとに来る時。 これから来る時。 将来。 「―に向けて羽ばたく」「―都市」 
\\	仏語。 三世(さんぜ)の一。 死後の世。 来世。 後世(ごせ)。 未来世。 
\\	主として西欧語の文法で、時制の一。 過去・現在に対して、これから実現するものとして述べる場合の語法。 動詞の語形変化で示される。 →将来[用法]	▲彼の未来はばら色だ。 ▲彼の本には未来の世界への理想が込められている。
\\	魅力	みりょく	人の心をひきつけて夢中にさせる力。 「―のある人柄」「―的な笑顔」	▲彼女の魅力は言葉では表現できない。 ▲彼女の魅力はその美しさだけにあるのではない。
\\	ミルク	ミルク	
\\	乳。 特に、牛乳。 
\\	練乳・粉乳などの、牛乳の加工品。	▲どこかのお利口さんが一晩中ミルクを冷蔵庫から出しっぱなしにしておいたな。 ▲はい。ここに私たちが牛のミルクをしぼった牧場の写真がありますよ。
\\	向かい	むかい	
\\	向き合っていること。 正対すること。 また、正面。 「駅の―にある喫茶店」 
\\	自分の家の正面にある家。 また、その家の人。 「お―の坊や」	▲店は銀行の向かいにある。 ▲向かいのパン屋へ行ってパンを買ってきて。
\\	迎え	むかえ	来る人を迎えること。 迎えに行くこと。 また、その人。 むかい。 「―の車が来る」「―をやる」	▲お昼にお迎えに参ります。 ▲お迎えが来た。
\\	向く	むく	㊀[動カ五(四)] 
\\	その方向に正面が位置するようにする。 対する。 面する。 「上を―・く」「東に―・く」 
\\	その方向を指し示す。 「磁石の針は北を―・く」 
\\	その方向・状態にかたむく。 「気が―・く」「心は故郷に―・いている」 
\\	適する。 ふさわしい。 「若者に―・いた仕事」 [可能]むける ㊁[動カ下二]「む(向)ける」の文語形。	▲彼は海の方を向いた。 ▲左を向いてください。
\\	向ける	むける	[動カ下一][文]む・く[カ下二] 
\\	その方向に正面が位置するようにする。 ある方向を向かせる。 「視線を―・ける」「背を―・ける」「マイクを―・ける」「怒りを他人に―・ける」 
\\	その方向をめざす。 「現地へ―・けて出発する」「母校に足を―・ける」 
\\	ある目的・用途にそれをあてる。 ふりあてる。 ふりむける。 「寄付金を人件費に―・ける」 
\\	使いとして行かせる。 さしむける。 「使者を―・ける」 
\\	たむける。 ささげる。 「幣(ぬさ)取り―・けてはや帰り来(こ)ね」 
\\	従わせる。 服従させる。 「韓国(からくに)を―・け平(たひ)らげて」	▲彼は、東京に向けて出発した。 ▲彼は、10時に東京に向けて出発する。
\\	無視	むし	[名]スル存在価値を認めないこと。 また、あるものをないがごとくみなすこと。 「人の気持ちを―する」「信号―」	▲この点は無視したほうがいい。 ▲ジェニーは両親が安全を願う気持ちを無視するわけにはいかなかった。
\\	虫歯・齲歯	むしば	歯の硬い組織が、口腔内の細菌の作用による食べかすの発酵で溶解し、破壊される状態。 また、その歯。 虫食い歯。	▲甘いものを食べ過ぎると虫歯になるよ。 ▲今は虫歯だらけだ。
\\	寧ろ	むしろ	[副]二つを比べて、あれよりもこれを選ぶ、また、これのほうがよりよいという気持ちを表す。 どちらかといえば。 「休日は遊びに行くより―家で寝ていたい」→却(かえ)って[用法]	▲私はむしろここにいたい。 ▲私はむしろビールを注文したい。
\\	結ぶ	むすぶ	[動バ五(四)] 
\\	ひもなど、細長いものを組んでつなぐ。 また、結び目をつくる。 「髪を―・ぶ」「包帯を―・ぶ」 
\\	手の指をからませるなどして形をつくる。 ㋐(ふつう「掬ぶ」と書く)手のひらを組み合わせて水をすくう。 「花を採り水を―・んでは」 ㋑仏教で、手の指でさまざまの形をつくる。 「印(いん)を―・ぶ」 
\\	開いたものを閉じる。 「口をへの字に―・ぶ」「手を―・んだり開いたりする」 
\\	互いに関係をつくる。 ㋐交わりを緊密にする。 「親交を―・ぶ」「縁を―・ぶ」 ㋑同じ考えの者どうしが一緒になる。 組む。 「強い勢力と―・ぶ」「同盟を―・ぶ」 ㋒互いに約束する。 「条約を―・ぶ」「契りを―・ぶ」 
\\	二つの地点をつなぐ。 連絡する。 「本州と四国とを―・ぶ橋」「東京と北京とを四時間で―・ぶ空路」 
\\	まとめて形にする。 また、まとまって形になる。 ㋐植物が実をつくる。 結実する。 みのる。 「ぶどうがたわわに―・ぶ」 ㋑空中の水分などが固まる。 結露する。 「ハスの葉に露が―・ぶ」 ㋒結果が出る。 「努力が実を―・ぶ」 
\\	建物を構える。 「庵(いおり)を―・ぶ」 
\\	文章などを終わりにする。 締めくくる。 「話を―・ぶ」 
\\	係り結びで、文末の活用語を、上の係助詞に応じた活用形とする。 「こそ」を受けて已然形とする類。 
\\	誓いや願いを込めて、草や木の枝の端などをつなぎ合わせる。 「君が代も我が代も知るや磐代(いはしろ)の岡の草根をいざ―・びてな」 [可能]むすべる [下接句]縁を結ぶ・同じ流れを掬(むす)ぶ・局(きよく)を結ぶ・草を結ぶ・口を結ぶ・綬(じゆ)を結ぶ・契りを結ぶ・手を結ぶ・実を結ぶ・夢を結ぶ	▲この道路は二つの市を結んでいる。 ▲この高速度道路は東京と名古屋を結んでいる。
\\	無駄・徒	むだ	[名・形動] 
\\	役に立たないこと。 それをしただけのかいがないこと。 また、そのさま。 無益。 「―な金を使う」「時間を―にする」 
\\	「むだぐち」に同じ。 役に立たないおしゃべり。むだごと。「しゃれも―もいっかう言はず」 ◆「無駄」は当て字。	▲彼に話しかけても無駄だ。 ▲彼に助けを求めても無駄だ。
\\	夢中	むちゅう	[名・形動] 
\\	物事に熱中して我を忘れること。 また、そのさま。 「―で本を読む」「競馬に―になる」「無我―」 
\\	夢を見ている間。 夢の中。 「多数は猶安眠の―にあり」 
\\	正気を失うこと。 また、そのさま。 「余りの嬉しさに…―な程でした」	▲彼は探偵小説を読むのに夢中になっている。 ▲彼は成長しようと夢中になっている。
\\	胸	むね	
\\	首と腹との間の部分。 哺乳類では横隔膜により腹部と仕切られ、肋骨(ろつこつ)に囲まれて肺・心臓などが収まる。 「―を張って歩く」 
\\	心臓。 「―をときめかす」 
\\	肺。 「排気ガスで―をやられる」 
\\	胃。 「―焼け」 
\\	乳房。 「―の豊かな女性」 
\\	こころ。 思い。 心の中。 「―のうちを明かす」「―をはずませる」「自分の―一つにおさめる」 
\\	衣服の 
\\	にあたる部分。 「―ポケット」 [類語]
\\	胸部・胸腔(きようこう)・胸郭(きようかく)・胸板(むないた)・胸間(きようかん)・胸元(むなもと)・胸先(むなさき)・胸倉(むなぐら)・懐(ふところ)・バスト・チェスト/
\\	胸裏・胸中・胸間・胸底・胸奥(きようおう)・胸臆(きようおく)・肺腑(はいふ)・心(こころ)・心中・心裏・方寸・と胸	▲彼女の胸は喜びに躍った。 ▲彼女の胸は早鐘を打つようだ。
\\	無料	むりょう	
\\	料金を払わなくてよいこと。 無代(むだい)。 ただ。 「入場―」「―券」↔有料。 
\\	人のために何かしてもお金を受け取らないこと。 「―奉仕」	▲無料なのですか。 ▲無料で見本を配っています。
\\	芽	め	
\\	植物の種子から最初にもえ出す茎・葉。 また、茎・葉などが未発達の状態にあるもの。 生じる位置により定芽・不定芽に、展開後の器官により葉芽・花芽・混合芽に、形成時期などにより夏芽・冬芽などに分けられる。 
\\	卵の黄身の上にあり、将来ひなになる部分。 
\\	新たに生じ、これから成長しようとするもの。 「才能の―を伸ばす」 [下接語]赤芽・独活(うど)芽・木の芽・木(こ)の芽・挿し芽・篠(しの)芽・新芽・土用芽・冬芽・若芽	▲寒い天候のために植物は芽を出せないかもしれない。 ▲今日新しい葉が芽を出した。
\\	明確	めいかく	[名・形動]はっきりしていてまちがいのないこと。 また、そのさま。 「―な指示を与える」「立場を―にする」 [派生]めいかくさ[名]	▲相補的分布に関する重要な点は、個々の音が起こる環境を明確に述べることである。 ▲著者たちはごく少数の種の雌も歌うと非常に明確に述べている。
\\	命じる	めいじる	[動ザ上一]「めい(命)ずる」(サ変)の上一段化。 「本社勤務を―・じる」	▲将軍はその都市の攻撃を命じた。 ▲将軍は彼に司令部へ出頭しろと命じた。
\\	名人	めいじん	
\\	技芸にすぐれている人。 また、その分野で評判の高い人。 
\\	江戸時代、囲碁・将棋で九段の技量を持つ最高位者に与えられた称。 現在は、それぞれタイトルの名称。 「―戦」	▲彼はうそつきの名人だ。 ▲彼はその種の問題を解決する名人だ。
\\	命令	めいれい	[名]スル 
\\	上位の者が下位の者に対して、あることを行うように言いつけること。 また、その内容。 「―を下す」「―に従う」「部下に―する」「―一下」 
\\	国の行政機関が制定する法の形式、および、その法の総称。 法律を実施するため、または法律の委任によって制定される。 政令・総理府令・省令など。 「執行―」 
\\	行政庁が特定の人に対して一定の義務を課する具体的な処分。 
\\	訴訟法上、裁判官がその権限に属する事項について行う裁判。 「略式―」 
\\	コンピューターで、コマンドのこと。 [類語]
\\	言い付け・命(めい)・令(れい)・指令・下命・指示・指図(さしず)・号令・発令・沙汰(さた)・主命・君命・上意・達し・威令・厳令・厳命 (尊敬)仰せ・尊命・懇命(―する)命ずる・言い付ける・申し付ける・仰せ付ける	▲君の提案は命令同然だ。 ▲君の命令にしたがって私はボートを売ろう。
\\	迷惑	めいわく	[名・形動]スル 
\\	ある行為がもとで、他の人が不利益を受けたり、不快を感じたりすること。 また、そのさま。 「人に―をかける」「―な話」「一人のために全員が―する」 
\\	どうしてよいか迷うこと。 とまどうこと。 「一生の間煩悩の林に―し」 [派生]めいわくがる[動ラ五]めいわくげ[形動]めいわくさ[名]	▲ご迷惑をおかけしまして申し訳ありません。 ▲約1年半で約22億通の迷惑メールを送りました。
\\	飯	めし	《召し上がる物の意から》 
\\	米・麦などを炊いたもの。 ごはん。 いい。 「―を炊く」「米の―」 
\\	食事。 ごはん。 「三度の―」「朝―」	▲とにかくメシ・・・といきたいところだが、その前に用を足すことにした。 ▲テレビを点け、彼女がブラウン管の前で遊弋している。 「おい、飯だぞ?」。
\\	滅多に	めったに	(下に打ち消しの語を伴って) 
\\	まれにしかないさま。 ほとんど。 「映画館には―行かない」「負けじ魂から―には屈服せず/浮雲(四迷)」 
\\	思慮なく行うさま。 うかつに。 「気むずかしくて―話しかけられない」	▲彼女はまずめったに11時前に寝ることはない。 ▲彼女はとても注意深いのでめったに間違いをしない。
\\	メモ	メモ	[名]スル忘れないように要点を書き留めること。 また、書き留めたもの。 覚え書き。 「会議の―をとる」「談話を―する」「―用紙」	▲彼は私にメモをそっと渡した。 ▲彼女のいうことには必ずメモを取ってください。
\\	綿	めん	もめん。 もめんわた。 また、綿糸・綿織物のこと。 「―の肌着」	▲上質の綿でできています。 ▲彼はウールと綿の区別がつかない。
\\	面	めん	㊀[名] 
\\	顔。 「―のいいのを鼻にかける」 
\\	顔につけるかぶりもの。 多くは人物・動物などの顔をかたどったもので、神楽・舞楽・能・狂言や、子供のおもちゃなどに使われる。 仮面。 面形(おもてがた)。 おもて。 
\\	顔面または頭部を保護するためにつける防具。 剣道の面頬(めんぽお)、野球の捕手がつけるマスクなど。 
\\	剣道の技の一。 頭部を打つこと。 
\\	物の外側の、平らな広がり。 表面。 
\\	数学で、線が運動したときにできる、広がりはあるが厚さのない図形。 平面と曲面がある。 「―に垂直な直線」 
\\	建築で、角材の稜角(りようかく)を削り落としてできる部分。 切り面・几帳面(きちようめん)など。 
\\	方面。 「資金の―で援助する」 ㊁〔接尾〕助数詞。 鏡・琵琶(びわ)・硯(すずり)・能面・仮面・碁盤など、平たいものを数えるのに用いる。 「琴一―」「三―のテニスコート」	▲二人の男が面と向かい合った。 ▲風の水の面が波だった。
\\	免許	めんきょ	[名]スル 
\\	ある特定の事を行うのを官公庁が許すこと。 また、法令によって、一般には禁止されている行為を、特定の場合、特定の人だけに許す行政処分。 「―を取得する」「―がおりる」「運転―」「幕府時代に―した敷設の権利を」 
\\	師から弟子にその道の奥義を伝授すること。 また、その証書。 ゆるし。 「師範の―を与える」	▲車の免許を取りに行く。 ▲車の免許は18歳から取ることが出来る。
\\	面倒	めんどう	[名・形動] 
\\	手間がかかったり、解決が容易でなかったりして、わずらわしいこと。 また、そのさま。 「―な手続き」「―なことにならなければよいが」「断るのも―だ」「―を起こす」 
\\	世話。 「この子の―をお願いします」 
\\	体裁がわるいこと。 見苦しいこと。 また、そのさま。 「此の君の御供申し、不足なく見する物は―なり」 ◆「目(め)どうな」の音変化。 「どうな」は、むだになることで、見るだけむだなものが原義。 「な」が形容動詞連体形語尾の「な」のように意識され、「めどう」「めんどう」となり、「面倒」と当て字されるようになった。 [派生]めんどうがる[動ラ五]めんどうさ[名] [用法]面倒・厄介――「面倒な(厄介な)問題をかかえこんだ」「入国には面倒な(厄介な)手続きが必要だ」「面倒(厄介)をおかけしてすみません」など、わずらわしいの意、また人をわずらわすの意では相通じて用いられる。 
\\	「面倒」は気分としてわずらわしいという意が強いのに対し、「厄介」は事柄そのものが手間がかかってむずかしいというときに多く用いられる。 「ごはんをたくのが面倒だから店屋物にしよう」「面倒がらずに辞典を引こう」では、ふつう「厄介」は使わない。 
\\	「後輩の面倒を見る」は、世話をするの意。 「知人の家に厄介になる」は、世話になるの意。 それぞれ置き換えはできない。 
\\	「面倒」「厄介」よりも文章語的な言い方として「煩雑」がある。 「煩雑に入り組んだ人間関係」「事後処理の煩雑さに音を上げる」などと用いる。	▲君は病気のお母さんの面倒をもっと見るべきだ。 ▲私たちの忠告どおりにしていたら、面倒なことにならなかったのに。
\\	メンバー	メンバー	
\\	集団を構成する人。 構成員。 一員。 「クラブの―」 
\\	そこに集まる面々。 顔ぶれ。 「いつもの―がそろう」	▲ボーリングに行くなら私をメンバーからはずしておいて。 ▲例えば、社会科の授業では、先生がメンバーの一人になって、議論がされることがしばしばあります。
\\	も	も	㊀[係助]種々の語に付く。 
\\	ある事柄を挙げ、同様の事柄が他にある意を表す。 …もまた。 「国語―好きだ」「ぼく―知らない」「み吉野の山のあらしの寒けくにはたや今夜(こよひ)―我(あ)がひとり寝む」 
\\	同類の事柄を並列・列挙する意を表す。 「木―草―枯れる」「右―左―わからない」「銀(しろかね)―金(くがね)―玉―何せむに優(まさ)れる宝子にしかめやも」 
\\	全面的であることを表す。 ㋐不定称の指示語に付き、全面的否定、または全面的肯定を表す。 「疑わしいことは何―ない」「どこ―いっぱいだ」「だれ―が知っている」「何―何―、小さきものは皆うつくし」 ㋑動詞の連用形や動作性名詞に付き、打消しの語と呼応して、強い否定の意を表す。 「思い―よらぬ話」「返事―しない」 
\\	おおよその程度を表す。 …ぐらい。 …ほど。 「一週間―あればできる」「今なら一万円―しようかね」 
\\	驚き・感動の意を表す。 「この本、三千円―するんだって」「限りなく遠く―来にけるかなとわびあへるに」 
\\	ある事柄を示し、その中のある一部分に限定する意を表す。 …といっても。 …のうちの。 「中世―鎌倉のころ」「東京―西のはずれ」→もこそ →もぞ →もや ㊁[接助]形容詞・形容詞型活用語の連用形、動詞・動詞型活用語の連体形に付く。 逆接の意を表す。 …とも。 …ても。 …けれども。 「見たく―見られない」「努力する―報われなかった」「いつしかと涼しき程待ち出(い)でたる―、なほ、はればれしからぬは、見苦しきわざかな」「身一つ、からうじて逃るる―、資財を取り出(い)づるに及ばず」 ㊂[終助]文末で、活用語の終止形、助詞、接尾語「く」に付く。 感動・詠嘆を表す。 …ことよ。 …なあ。 「春の野に霞たなびきうら悲しこの夕影にうぐひす鳴く―」→かも →ぞも →はも →やも ◆主に上代の用法で、その後は「かな」に代わった。 係助詞の終助詞的用法ともいう。	▲もすぐ夜が明ける。 ▲携帯電話から国際電話をかけても、モビラなら「1分あたり20円」でかけられます。
\\	申し込む	もうしこむ	[動マ五(四)] 
\\	意志・希望・要求などを相手方に伝える。 「抗議を―・む」「結婚を―・む」「試合を―・む」 
\\	募集などに応じて手続きをとる。 「予約を―・む」	▲彼はその乗馬クラブへ入会を申しこんだ。 ▲彼はその若い婦人に結婚を申し込んだ。
\\	申し訳・申し分け	もうしわけ	
\\	申し開き。 言いわけ。 弁解。 「―が立つ」「―がない」「出席できず―ありません」 
\\	なんとか形だけつけること。 体裁だけであること。 「―に並べただけのもの」「―程度の謝礼」	▲申し訳ございませんが、私自身が会合に出席することはできません。 ▲直ちにご注文に応じられずまことに申し訳ございません。
\\	毛布	もうふ	寝具などに用いる、厚地で縮絨(しゆくじゆう)・起毛を施した毛織物。 混紡糸・化学繊維などを用いたものもある。 ブランケット。 ケット。 《季 冬》「いと古りし―なれども手離さず/たかし」	▲明け方に寒かったので毛布をもう1枚掛けた。 ▲枕と毛布を取って下さい。
\\	燃える	もえる	[動ア下一][文]も・ゆ[ヤ下二] 
\\	火がついて炎が立つ。 燃焼する。 「紙が―・える」「ストーブの火が―・える」 
\\	激しく気持ちが高まる。 情熱が盛んに起こる。 「愛国心に―・える」「怒りに―・える」 
\\	炎のような光を放つ。 光る。 陽炎(かげろう)や蛍の光、夏の厳しい陽光などにいう。 「夕日に赤く―・える空」《季 夏》「―・ゆる海わんわんと児が泣き喚き/誓子」	▲プラスチックは燃えにくい。 ▲ますます多くの家がコンクリートで作られるようになり、コンクリートの家は木造の家屋ほど簡単には燃えないため、火事は今は以前ほど恐ろしいものではなくなっている。
\\	不	ぶ	〔接頭〕名詞または形容動詞の語幹に付いて、それを打ち消し、否定する意を表す。 無(ぶ)。 
\\	…でない、…しない、などの意を添える。 「―調法」 
\\	…がわるい、…がよくない、などの意を添える。 「―気味」「―器量」	
\\	分	ぶ	
\\	どちらに傾くかの度合い。 自分のほうに有利になる度合い。 「対戦成績では―が悪い」 
\\	利益の度合い。 「―のいい商売」 
\\	平らなものの厚さの度合い。 「―の厚い本」 
\\	音楽で、全音符の長さを等分に分けること。 「四―音符」 
\\	全体を一〇等分したもの。 一〇分の一相当の量。 「工事は九―どおり完成した」「三―咲きの桜」 
\\	単位の名。 ㋐割合・利率で、一割の一〇分の一。 全体の一〇〇分の一。 「打率二割三―」 ㋑尺貫法で、一寸の一〇分の一。 ㋒尺貫法で、一匁の一〇分の一。 ㋓温度で、一度の一〇分の一。 体温にいう。 「七度八―の熱が出た」 ㋔江戸時代の通貨で、銭一文の一〇分の一、または金一両の四分の一。 ㋕足袋などの大きさで、一文(もん)の一〇分の一。 「一〇文三―」	▲昨年度のコンピューターからの利益は、今年度分よりも10%近く多かった。 ▲彼のほうに分がある。
\\	無・无	む	㊀[名] 
\\	何もないこと。 存在しないこと。 「―から有を生ずる」↔有。 
\\	哲学の用語。 ㋐存在の否定・欠如。 特定の存在がないこと。 また、存在そのものがないこと。 ㋑一切の有無の対立を超え、それらの存立の基盤となる絶対的な無。 
\\	禅宗で、経験・知識を得る以前の純粋な意識。 「―の境地」 ㊁〔接頭〕名詞に付いて、そのものが存在しないこと、その状態がないことの意を表す。 「―感覚」「―資格」「―届け」「―免許」	▲我々は、無から有を作り出すことができるだろうか。 ▲彼は高校3年間無遅刻無欠席だった。
\\	プラットホーム	プラットホーム	
\\	電車・列車への乗客の乗り降り、貨物の積み下ろしのため、線路に沿って築いた駅の施設。 ホーム。 
\\	大型の無人観測衛星。	
\\	マイクロホン	マイクロホン	《「マイクロフォン」とも》音声を電気信号に変換する装置。 電気信号にして拡声器で再生したり送信したりする。 マイク。	
\\	人込み・人混み	ひとごみ	たくさんの人がいて込み合っていること。 また、その場所。 雑踏。 「―にまぎれる」「―を避ける」	▲彼は人込みの中に姿を消した。 ▲彼は人込みの中を押し分けてすすんだ。
\\	ふと	ふと	[副] 
\\	はっきりした理由や意識もないままに事が起こるさま。 思いがけず。 不意に。 ふっと。 「―立ち止まる」「夜中に―目がさめた」 
\\	素早く容易に行われるさま。 すぐに。 即座に。 「竜あらば、―射殺して」 
\\	動作の敏速なさま。 つっと。 「猫また、あやまたず足もとへ―寄り来て」 ◆「不図」「不斗」などと当てても書く。	▲それは秘密にしておくべきではないと私はふと思った。 ▲ふと街で彼に会った。
\\	まさか	まさか	㊀[名] 
\\	今まさに物事が目の前に迫っていること。 予期しない緊急の事態にあること。 「―の場合に役立てる」 
\\	目前のとき。 さしあたっての今。 「梓弓(あづさゆみ)末は寄り寝む―こそ人目を多み汝(な)を端に置けれ」 ㊁[副] 
\\	(あとに打消しや反語の表現を伴って) ㋐打消しの推量を強める。 よもや。 「―彼が来るとは思わなかった」「この難問を解ける者は―あるまい」 ㋑ある事がとうてい不可能だという気持ちを表す。 とても。 どうしても。 「病気の彼に出て来いとは―言えない」 
\\	その状態であることを肯定して強調するさま。 まさしく。 ほんとうに。 「―影口が耳に入ると厭なものサ」 ◆「真逆」とも当てて書く。 [用法]まさか・よもや――「まさか(よもや)オリンピックに出られるとは思わなかった」「まさか(よもや)私を疑っているわけではないだろうね」のように、両語ともに、そんなことはあるはずがないという気持ちを強める表現で、打消しを伴って用いられる。 
\\	「まさか」は「まさかの時に備えて貯金する」のように名詞としても使うが、「よもや」に名詞用法はない。 また、容易に信じられない気持ちを感動詞的に表す用法もある。 「『この辞書を五十円で売ろうか』『まさか』」、この場合には「よもや」は使えない。 
\\	「よもや」は「まさか」より古風な言い方で、改まった感じの語。 「あの約束をよもやお忘れではないでしょう」「この家がよもや地震で倒壊することはあるまい」	▲まさかあれは私たちの探している家じゃないだろう。 ▲まさかここで君に会うなんて思ってもいなかった。
\\	益・益益	ますます	[副]《動詞「ま(増)す」を重ねた語》程度が一層はなはだしくなるさま。 いよいよ。 「風雨は―激しくなる」「老いて―盛んだ」	▲高く登れば登るほどますます気温は下がる。 ▲今ではますます多くの外国人を見かける。
\\	目的	もくてき	
\\	実現しようとしてめざす事柄。 行動のねらい。 めあて。 「当初の―を達成する」「―にかなう」「旅行の―」 
\\	倫理学で、理性ないし意志が、行為に先だって行為を規定し、方向づけるもの。 [用法]目的・目標――「目的(目標)に向かって着実に進む」のように、めざすものの意では相通じて用いられる。 
\\	「目的」は、「目標」に比べ抽象的で長期にわたる目あてであり、内容に重点を置いて使う。 「人生の目的を立身出世に置く」
\\	「目標」は、目ざす地点・数値・数量などに重点があり、「目標は前方三〇〇〇メートルの丘の上」「今週の売り上げ目標」のようにより具体的である。	▲彼女は入賞の目的を達成した。 ▲彼女は叔母にあう目的でパリへ行った。
\\	目標	もくひょう	
\\	そこに行き着くように、またそこから外れないように目印とするもの。 「島を―にして東へ進む」 
\\	射撃・攻撃などの対象。 まと。 「砲撃の―になる」 
\\	行動を進めるにあたって、実現・達成をめざす水準。 「―を達成する」「月産五千台を―とする」「―額」→目的[用法]	▲法の目標は正義である。 ▲募金はまだ目標額に達しない。
\\	木曜	もくよう	週の第五日。 水曜の次の日。 木曜日。	▲月曜から木曜までここにおります。 ▲会合は来週木曜に開かれるはずです。
\\	文字	もじ	《「もんじ」の撥音の無表記から》 
\\	言葉を表記するために社会習慣として用いられる記号。 個々の字の性質から表意文字・表音文字、また表語文字(単語文字)・音節文字・単音文字などに分けられる。 もんじ。 
\\	文章。 また、読み書きや学問のこと。 「―を見る眼は中々慥にして」 
\\	言葉。 文言(もんごん)。 「ただ―一つにあやしう」 
\\	字の数。 音節。 「―のかずも定まらず」 
\\	(近世、関西地方で)字の記された銭の面。 
\\	語の後半を省き、その語の頭音または前半部分を表す仮名の下に付いて、品よく言い表したり、婉曲に言い表したりする語。 →文字言葉 [類語]
\\	文字(もんじ)・字・鳥跡(ちようせき)・鳥の跡・用字・表記・点画(てんかく)・レター	▲辞書のその文字をご覧。 ▲文字はそれが半ば商売、半ば芸術であるとき最高に栄える。
\\	若しも	もしも	[副]「もし」を強めた語。 「―負けたらどうしよう」「―のとき」	▲もしもクレオパトラの鼻がもっと低かったら、世界の歴史は変わっていただろう。 ▲もしもご質問に全部お答えしていないのでしたら、ご連絡ください。
\\	持ち上げる	もちあげる	[動ガ下一][文]もちあ・ぐ[ガ下二] 
\\	手で持ったり、下から支えたりして、物を上の方へ上げる。 「バーベルを―・げる」「ジャッキで―・げる」 
\\	頭などを上の方へ起こす。 もたげる。 「読んでいた本から頭を―・げる」 
\\	ほめておだて上げる。 「―・げられていい気になる」	▲私はそれが見えるように息子を持ち上げた。 ▲女性ではあるが、彼女はこのバーベルを持ち上げられる。
\\	用いる	もちいる	[動ア上一][文][ワ上一]《「持ち率(い)る」の意》 
\\	用にあてて使う。 使用する。 「調味料に―・いる」「新しい方法を―・いる」 
\\	よいとして取り上げる。 採用する。 「人の意見を―・いない」 
\\	見込んで職に就かせる。 任用する。 「人材を選んで―・いる」 
\\	心を十分働かせる。 心を労する。 「供応に意を―・いる」 
\\	(多く否定の形をとる)必要とする。 「その産地を問うことを―・いず」 ◆ワ行上一段の「用ゐる」が、「用ふ」とハ行上二段に活用するようになり、さらに「用ゆ」とヤ行上二段にも活用するなど、平安時代以降のいろいろの音韻変化の影響で複雑な活用をとげた。 →用う →用ゆ [用法]もちいる・つかう――「コンピューターを用いて(使って)収支計算をする」のように、ある用に役立てる意では相通じて用いられる。 
\\	「用いる」は文章語的で「部下の提案を用いる(=採用する)」「有能と見て重く用いる(=登用する)」などの用法があるように、特にそれを取り上げて使用する意が強い。 
\\	「使う」の方が口頭語的で、意味の範囲も広い。 「頭を使う」「神経を使う」の形には普通は「用いる」を使わない。 また、「意を用いる」「心を用いる」の形には「使う」を使用しない。 
\\	人について、「新人を使う(用いる)」のように起用するの意では両語とも使えるが、「店員を三人使っている」「人に使われる身」のように働かせるの意では「用いる」を使うことはない。 
\\	「類似の語に「使用」がある。 「使用する」「使う」は相通じて用いられる。	▲この語は現在用いられていない。 ▲この構成において、三角形の変わりに長方形を用いても類似の困難が生ずる。
\\	尤も	もっとも	㊀[名・形動]道理にかなっていること。 なるほどその通りだと思われること。 また、そのさま。 当然。 「―な言い分」「いやがるのも―なことだ」 ㊁[接続]前の事柄を肯定しつつ、例外あるいは一部相反する内容を補足するときに用いる。 とはいうものの。 なるほどそうだが。 ただし。 「旅行にはみんな参加する。 ―行かない人も二、三いるが」 ㊂[副] 
\\	いかにもなるほどと思われるさま。 本当に。 まったく。 当然。 「事すでに重畳せり。 罪科―逃れがたし」 
\\	(あとに打消しの語を伴って)少しも。 全然。 「ふっつり心残らねば―足も踏み込まじ」 [類語] ❶当然・自然・至当・無理からぬ/ ❷ただし・ただ・とは言え・とは言うものの・さはあれ・しかし	▲彼女が自分の古い服を恥ずかしがるのももっともだ。 ▲彼女が自分の才能を自慢するのももっともだ。
\\	本・元	もと	㊀[名] 
\\	物事の起こり。 始まり。 「事件の―をさぐる」「うわさの―をただす」 
\\	(「基」とも書く)物事の根本をなすところ。 基本。 「生活の―を正す」「悪の―を断つ」 
\\	(「基」とも書く)基礎。 根拠。 土台。 「何を―に私を疑うのか」「事実を―にして書かれた小説」 
\\	(「因」とも書く)原因。 「酒が―でけんかする」「風邪は万病の―」 
\\	もとで。 資金。 また、原価。 仕入れ値。 「―がかからない商売」「―をとる」 
\\	(「素」とも書く)原料。 材料。 たね。 「たれの―」「料理の―を仕込む」 
\\	それを出したところ。 それが出てくるところ。 「火の―」「製造―」「販売―」 
\\	ねもと。 付け根。 「―が枯れる」「葉柄の―」 
\\	箸(はし)や筆の、手に持つ部分。 
\\	短歌の上の句。 「歌どもの―を仰せられて」 ㊁〔接尾〕(本)助数詞。 
\\	草や木を数えるのに用いる。 「一(ひと)―の松」 
\\	鷹(たか)狩りに使う鷹を数えるのに用いる。 「いづくよりとなく大鷹一―それて来たり」 [下接句]孝は百行(ひやつこう)の本・失敗は成功のもと・短気は未練の元・釣り合わぬは不縁の基・生兵法は大怪我(おおけが)の基・油断は怪我(けが)の基	
\\	戻す	もどす	[動サ五(四)] 
\\	㋐もとの状態や、もとあった場所などへ返す。 「本を棚へ―・す」「話をもとに―・す」「白紙に―・す」「よりを―・す」 ㋑水に浸したり、解凍したりして、加工する前の状態にする。 「ヒジキを水で―・す」 
\\	逆の方向へ返す。 「時計の針を一〇分―・す」 
\\	飲食したものを吐く。 嘔吐(おうと)する。 「食べ物を―・す」 
\\	下がっていた相場が回復する。 「下向きだった株価が―・す」→返(かえ)す[用法] [可能]もどせる	▲元のところへ戻しておきなさい。 ▲済んだら戻してください。
\\	基づく	もとづく	[動カ五(四)] 
\\	それが基となって起こる。 起因する。 また、それを根拠・基盤とする。 「政治の介入に―・く相場の変動」「規則に―・く処理」 
\\	近づく。 到達する。 「この舟に―・きしかひもなく、帰れと仰せ候ふことのあさましさよ」	▲彼の今度の小説は自分自身の体験に基づいている。 ▲彼の今度の小説は自分の体験に基づいていると言われている。
\\	求める	もとめる	[動マ下一][文]もと・む[マ下二] 
\\	欲しいと望む。 ほしがる。 「平和を―・める」「権力を―・める」 
\\	相手に要求する。 「賠償を―・める」「援助を―・める」「退陣を―・める」 
\\	得ようとしてさがす。 「職を―・める」「優秀な人材を―・める」 
\\	買って手に入れる。 購入する。 「古書を―・める」 [類語]
\\	欲する・望む・希(こいねが)う・希求する・追求する・渇望する・切望する・熱望する/
\\	頼む・請う・仰ぐ・懇請する・懇望する・要望する・所望する・要請する・要求する・請求する・徴(ちよう)する・催告する・迫る・せがむ・せびる・ねだる/
\\	探す・尋ねる・探し求める・物色する	▲彼は握手を求めててを差し出した。 ▲彼は握手を求めて私に手をだした。
\\	者	もの	《「物」と同語源》人。 多く、他の語句による修飾を受ける。 卑下・軽視する場合や、改まった場合に用いられる。 「店の―に言いつけてください」「土地の―に任せる」「持てる―の悩み」 [下接語]愛嬌(あいきよう)者・暴れ者・荒くれ者・慌て者・悪戯(いたずら)者・一刻者・一徹者・田舎者・浮かれ者・うっかり者・空(うつ)け者・浮気者・偉(えら)者・おいそれ者・御(お)尋ね者・御店(おたな)者・御(お)調子者・戯(おど)け者・思い者・愚か者・囲い者・果報者・変わり者・利け者・気紛(きまぐ)れ者・切れ者・曲(くせ)者・食わせ者・剛の者・極道(ごくどう)者・困り者・小者・晒(さら)し者・然(さ)る者・仕合わせ者・強(したた)か者・確(しつか)り者・忍びの者・邪魔者・洒落(しやれ)者・小心者・小身者・痴(し)れ者・好き者・拗(す)ね者・粗忽(そこつ)者・只(ただ)者・立て者・戯(たわ)け者・手の者・道化者・道楽者・流れ者・亡き者・慰み者・何者・怠け者・ならず者・成り上がり者・偽者・似た者・人気者・除(の)け者・のら者・馬鹿(ばか)者・働き者・日陰者・引かれ者・独り者・捻(ひね)くれ者・無精(ぶしよう)者・不束(ふつつか)者・回し者・昔者・無宿者・無法者・やくざ者・厄介者・余計者・余所(よそ)者・与太者・利口者・律義者・若い者・若者・渡り者・笑われ者・悪者	▲盗みを働く者は罰せられて当然だ。 ▲頭の空っぽな者ほど良くしゃべる。
\\	物音	ものおと	何かの物が立てる音。 「―がやむ」「―一つしない」	▲音楽会場では物音1つ聞こえなかった。 ▲我々は確かにその物音を聞いた。
\\	物語	ものがたり	[名]スル 
\\	さまざまの事柄について話すこと。 語り合うこと。 また、その内容。 「世にも恐ろしい―」 
\\	特定の事柄の一部始終や古くから語り伝えられた話をすること。 また、その話。 「湖にまつわる―」 
\\	文学形態の一。 作者の見聞や想像をもとに、人物・事件について語る形式で叙述した散文の文学作品。 狭義には、平安時代の「竹取物語」「宇津保物語」などの作り物語、「伊勢物語」「大和物語」などの歌物語を経て、「源氏物語」へと展開し、鎌倉時代における擬古物語に至るまでのものをいう。 広義には歴史物語・説話物語・軍記物語を含む。 ものがたりぶみ。 
\\	歌舞伎・人形浄瑠璃の演出の一。 また、その局面。 時代物で、立ち役が過去の思い出や述懐を身振りを交えて語るもの。 [類語]
\\	話(はなし)・叙事・ストーリー・お話・作り話・虚構・フィクション・説話・口碑(こうひ)・伝え話・昔話・民話・伝説・言い伝え	▲彼は自分の書く物語の材料はたいてい子供時代の体験に頼っている。 ▲彼は試しに短い物語を書いてみた。
\\	物事	ものごと	物と事。 もろもろの物や事柄。 「―の加減を知る」「―の順序をわきまえる」	▲彼は物事を真剣に考え過ぎる傾向にある。 ▲彼も大人になって、物事を総合的な視野で見られるようになった。
\\	模様	もよう	
\\	織物・染め物・工芸品などに装飾として施す種々の絵や形。 また、ものの表面にあらわれた図形。 文(あや)。 文様。 「美しい―の木目」「幾何学的な―」 
\\	物事のありさま。 ようすや経過。 「現場から事件の―をお伝えします」 
\\	現時点で推測される状況。 「列車は遅れる―だ」 
\\	手本。 模範。 「そもそも禅宗の―とするところは宋朝の行儀」 
\\	身ぶり。 所作。 「若衆に茶のたてやうを教ゆべしと、自ら茶をたつる―をなして」 
\\	名詞の下に付いて、それらしいようすであることを表す。 「雨―」「荒れ―」 [類語]
\\	文様(もんよう)・紋(もん)・文(あや)・文目(あやめ)・地紋(じもん)・柄(がら)・紋柄(もんがら)・図柄・絵柄・図様・図案・意匠・パターン・デザイン/
\\	様子・様相・状況・実況・現況・情勢・成り行き	▲雨もようだ。 ▲海軍のジェット機はとんでもない方向に飛行して、味方の軍隊を誤爆した模様だ。
\\	文句	もんく	
\\	文章中の語句。 文言。 「気のきいた―」「殺し―」 
\\	歌謡などで、メロディーに対して歌詞をいう。 「歌の―」 
\\	相手に対する言い分や苦情。 不服。 「弁償してもらえるのなら―はない」	▲いくつかの素晴らしい文句が詩人の心に浮かんだ。 ▲どうも君は未だ胸に一物もっているような気がする。文句があればはっきり言ってよ。
\\	軈て・頓て	やがて	[副] 
\\	あまり時間や日数がたたないうちに、ある事が起こるさま、また、ある事態になるさま。 そのうちに。 まもなく。 じきに。 「―日が暮れる」「東京へ出てから、―三年になる」 
\\	それにほかならない。 まさに。 とりもなおさず。 「自尊の念は―人間を支持しているもので」 
\\	そのまま。 引き続いて。 「山の仕事をして、―食べる弁当が」「(道真ガ大宰府ニ流サレテ)―かしこにてうせ給へる」 
\\	時を移さず。 ただちに。 すぐさま。 「―具して宮に帰りて后に立てむ」 ◆「軈」は国字。 [類語]
\\	間もなく・程なく・もうすぐ・もうじき・そろそろ・追っつけ・そのうち・今に・追い追い・遠からず・遅かれ早かれ・早晩・いずれ・いつか・行く行く	▲彼はやがて平静に戻った。 ▲彼は来る途中ですから、やがて到着するでしょう。
\\	役	やく	
\\	㋐受け持ちの任務。 役目。 「仲裁の―を買って出る」 ㋑組織の中で、責任のある地位・職務。 「―に就く」 
\\	演劇などで、俳優が扮(ふん)する人物。 配役。 「せりふのある―がつく」「―になりきる」 
\\	花札・マージャン・トランプなどで、ある条件がそろって特定の点数などを加える権利が生じること。 「高い―で上がる」 
\\	もっぱらのつとめ。 唯一の仕事。 「そこはかなきことを思ひつづくるを―にて」 
\\	官が人民に課す労役。 公役(くやく)。 夫役(ぶやく)。 「かやうの―に催し給ふはいかなることぞ」 
\\	物品に課す税。 「山中の関にて―をせよといふ」	▲彼はリヤ王の役を演じた。 ▲彼は議長の役を務めた。
\\	約	やく	㊀[名] 
\\	約束。 取り決め。 「―を交わす」「―を守る」 
\\	短く簡単にすること。 また、そのもの。 「長大な文章の―」 
\\	「約音」に同じ。ことばの相連続する二個の音節が、一個の音節に縮約する現象。「あらうみ」が「あるみ」となるように、前の音節の母音が消滅するものと、「おほまします」が「おはします」となるように、前の母音とともに後の音節の子音が消滅するものなどがある。約言。約訓。 ㊁[副]数量を大まかに数えるさま。 おおよそ。 だいたい。 「―一週間」「―一〇万円」	▲その山は海抜約3000メートルだ。 ▲その市の人口は約10万である。
\\	訳	やく	
\\	訳すこと。 また、その文章や語句。 翻訳。 「英文に日本語の―をつける」「源氏物語の現代語―」 
\\	漢字の訓。	▲私は学校で日本文学の英語訳、特に漱石の『吾輩は猫である』や『心』、芥川の『鼻』や『河童』を楽しく読んだ。 
\\	標準的コーディング技法/Stephan 
\\	著;クイープ訳。
\\	役割	やくわり	
\\	役目を割り当てること。 また、割り当てられた役目。 「大切な―をになう」「自分の―を確実に果たす」 
\\	社会生活において、その人の地位や職務に応じて期待され、あるいは遂行しているはたらきや役目。 →役目[用法]	▲日本は国際社会でますます大きな役割を演ずる事が予想される。 ▲日本は、世界経済の中で主な役割を果たしている。
\\	家賃	やちん	家や部屋の借り賃。 たなちん。	▲家主は彼が家賃を払っていなかったので出ていくように言った。 ▲家賃に関して私は彼と折り合いがついた。
\\	厄介	やっかい	[名・形動] 
\\	めんどうなこと。 扱いに手数がかかり、わずらわしいこと。 また、そのさま。 「―なことに巻き込まれる」 
\\	めんどうをみること。 また、世話になること。 「親の―になる」 
\\	他家に寄食すること。 居候(いそうろう)。 →面倒(めんどう)[用法] [派生]やっかいさ[名]	▲僕は君に厄介になるんじゃないかな。 ▲厄介な事だ。
\\	宿・屋戸	やど	《「屋の処(と)」の意か。 または「屋の戸」「屋の外(と)」の意か》 
\\	家。 すみか。 「埴生(はにゆう)の―」 
\\	《「やどり」との混同から》旅先で一時的に泊まる家。 また、宿屋。 「今日の―を決める」 
\\	妻が他人に対して夫のことをいう語。 主人。 宅。 「私が申しますと―が立腹致しますから」 
\\	奉公人の親元や請け人。 また、その家。 「―へ下がる」 
\\	ある目的のもとに、人々が集まる所。 若者宿・娘宿など。 
\\	揚屋(あげや)。 置屋。 また、その主人。 「―を頼んで田舎客の談合破らせ」 
\\	家の入り口。 戸口。 「夕さらば―開け設(ま)けて我待たむ夢(いめ)に相見に来むといふ人を」 
\\	家の庭先。 「秋は来ぬ紅葉は―に降り敷きぬ道ふみわけてとふ人はなし」	▲私達は山のふもとの宿に泊まった。 ▲宿が見つからなかったら、野宿しかないね。
\\	雇う・傭う	やとう	[動ワ五(ハ四)] 
\\	賃金を払って人を使う。 また、料金を払って乗り物などを使う。 「人を―・う」「ハイヤーを―・う」 
\\	借りて使う。 借用する。 「舌根を―・ひて、不請の阿弥陀仏両三遍申して止みぬ」 [可能]やとえる	▲彼はまだその会社に雇われている。 ▲彼は近頃雇っている人たちとうまくやっている。
\\	屋根・家根	やね	
\\	雨・風・日射などを防ぐために建物の最上部にあるおおい。 「―を葺(ふ)く」「藁葺(わらぶ)き―」「―瓦(がわら)」 
\\	物の上部をおおうもの。 また、最上部にあるもの。 「自動車の―」「ヨーロッパの―といわれるアルプス」 [下接語]板屋根・大屋根・瓦(かわら)屋根・切妻屋根・草屋根・腰折れ屋根・越し屋根・小屋根・晒(さら)し屋根・取り葺(ぶ)き屋根・鋸(のこぎり)屋根・丸屋根・マンサード屋根・陸(ろく)屋根・藁(わら)屋根	▲彼は屋根から落ちて、さらに悪い事に足を折った。 ▲彼は屋根からまっさかさまに落ちた。
\\	破る	やぶる	㊀[動ラ五(四)] 
\\	引き裂いたり、傷をつけたり、穴をあけたりして、もとの形をこわす。 「障子を―・る」「書類を―・る」 
\\	相手の守りなどを突き抜ける。 突破する。 「警戒網を―・る」 
\\	今まで続いてきた状態をそこなう。 かきみだす。 「太平の夢を―・る」 
\\	従来のものに代わって新しくする。 記録などを更新する。 「世界記録を―・る」 
\\	相手を打ち負かす。 「強敵を―・る」 
\\	守るべき事柄にそむく。 きまりや約束などを無視する。 「約束を―・る」「校則を―・る」 
\\	傷つける。 害する。 「身体髪膚を―・らずして」 [可能]やぶれる ㊁[動ラ下二]「やぶれる」の文語形。 [下接句]雨(あめ)塊(つちくれ)を破らず・竈(かまど)を破る・産を破る・叢蘭(そうらん)茂らんと欲し秋風(しゆうふう)之(これ)を敗(やぶ)る・操(みさお)を破る・横紙を破る	▲たとえ太陽が西から出るようなことがあっても、決して約束は破りません。 ▲だれかがこの本から2ページ破り取った。
\\	辞める・罷める	やめる	[動マ下一][文]や・む[マ下二]《「止(や)める」と同語源》職や地位から離れる。 退く。 「会社を―・める」「教師を―・める」	▲今の仕事やめたいんだ。 ▲今は仕事を辞めたいとは思わない。
\\	稍・漸	やや	[副] 
\\	いくらかその傾向を帯びているさま。 少しばかり。 「今年は昨年より―暑い」 
\\	少しの間。 しばらく。 「―間をおいて話し始める」 
\\	状況が少しずつ進むさま。 しだいに。 「風冷やかにうち吹きて、―更けゆくほどに」	▲身振り言語はしかしながら、暗い所や、やや離れた所では使えないので重大な限界があった。 ▲私に返事をするのに彼女はややせっかちであった。
\\	唯一	ゆいいつ	ただ一つであること。 それ以外にはないこと。 ゆいいち。 ゆいつ。 「世界で―の逸品」「―の趣味」	▲健康が私の唯一の資本です。 ▲経験は賢明な人の唯一の予言である。
\\	勇気	ゆうき	いさましい意気。 困難や危険を恐れない心。 「―がわく」「―を出す」「―凜々(りんりん)」	▲彼は危険に直面しても勇気があった。 ▲彼は危険をかえりみず勇気を示した。
\\	有効	ゆうこう	[名・形動] 
\\	ききめのあること。 効力をもっていること。 また、そのさま。 「資金を―に使う」「―な手段」↔無効。 
\\	法律上の効力をもっていること。 ↔無効。 
\\	柔道の試合で、「技あり」に近い技のときに下す判定。 抑え込みでは、二〇秒以上二五秒未満経過した場合。	▲その態度はその状況においては有効だ。 ▲その法律はまだ有効である。
\\	優秀	ゆうしゅう	[名・形動]非常にすぐれていること。 また、そのさま。 「―な人材」「成績―」 [派生]ゆうしゅうさ[名]	▲ビルは優秀な科学者になる素質を持っている。 ▲もっとも優秀な学生がクラスを代表して感謝の意をあらわした。
\\	優勝	ゆうしょう	[名]スル 
\\	競技などで第一位になること。 「全国大会で―する」 
\\	力のまさるものが勝つこと。 「実(げ)に―の時世とて、才名早くたつか弓」	▲彼はチェスのトーナメントで優勝を勝ち取った。 ▲彼が優勝しそうだ。
\\	友情	ゆうじょう	友達の間の情愛。 友人としてのよしみ。 「―が芽生える」「―に厚い」	▲我々の古き友情のためにご援助いたしましょう。 ▲我々の友情は依然として揺るがなかった。
\\	友人	ゆうじん	ともだち。 朋友。 とも。	▲健のお母さんと友人の両方ともまもなく空港に着くでしょう。 ▲賢明で助けになってくれる友人ほど貴重な価値をもつ宝はほとんどありません。
\\	有能	ゆうのう	[名・形動]才能のあること。 また、そのさま。 「―な人材」「―の士」↔無能。 [派生]ゆうのうさ[名]	▲有能な連邦捜査局員はちゅうちょすることなく、自分の義務を実行する。 ▲有能な探偵がその悲劇の原因を調査する任務に当てられた。
\\	郵便	ゆうびん	
\\	信書の書状・はがきや小包などを宛先の人に送り届ける通信事業。 各国とも国営で行われる。 日本の郵便制度は前島密(まえじまひそか)により、明治四年(一八七一)発足。 郵政大臣が管理する。 「荷物を―で送る」「速達―」「航空―」 
\\	「郵便物(ゆうびんぶつ)」の略。郵便の目的となる物件。その物件の種類によって通常郵便物と小包郵便物に分け、その取り扱いの方法によって普通取扱郵便物と特殊取扱郵便物、料金の有無によって有料郵便物と無料郵便物に分ける。通常郵便物は、第一種・第二種・第三種・第四種に細別される。郵品。 「―が届く」「―を出す」	▲郵便配達人は一軒づつ郵便を配る。 ▲強盗が郵便列車をめちゃくちゃにした。
\\	ユーモア	ユーモア	人の心を和ませるようなおかしみ。 上品で、笑いを誘うしゃれ。 諧謔(かいぎやく)。 「―に富んだ会話」「―の通じない人」「ブラック―」	▲私たちの先生にはすばらしいユーモアのセンスがある。 ▲君はユーモアのセンスがある。
\\	有利	ゆうり	[名・形動]利益のあること。 利益を望めること。 他よりも条件や状態がよいこと。 また、そのさま。 「―な取引」「相手方に―な情報」「戦局が―に展開する」↔不利。 [派生]ゆうりさ[名]	▲地方の党員たちは自党に有利な形の選挙区割りをもくろんでいます。 ▲地上で空費される時間が飛行機の速さという有利な店を帳消しにしてしまう。
\\	床・牀	ゆか	
\\	建物の内で、根太(ねだ)を立て、地面より高く板を張った部分。 そのままで、また畳や敷物などを敷いて生活する。 また、広く建物の内で、人の立ったり歩いたりする底面。 
\\	劇場で、義太夫節の太夫と三味線弾きが座る所。 舞台上手(かみて)に常設または仮設される。 歌舞伎ではチョボ床ともいう。 
\\	京都の鴨川沿いの茶屋で、座敷から川原へ張り出してつくった納涼用の桟敷。 川床。 《季 夏》 
\\	家の中で、一段高く作った所。 寝所などにする。 「―の下(しも)に二人ばかりぞ臥したる」	▲床はとても汚れているので洗う必要がある。 ▲床はほこりをかぶっていた。
\\	愉快	ゆかい	[名・形動]楽しく気持ちのよいこと。 おもしろく、心が浮きたつこと。 また、そのさま。 「―な話」「―に遊ぶ」 [派生]ゆかいげ[形動]ゆかいさ[名]	▲私たちのクラスではいつも愉快なことが起こっている。 ▲控え目にいっても、彼は愉快なやつではない。
\\	行き・往き	ゆき	
\\	目的地に向かって行くこと。 また、その時や、その道筋。 いき。 「―は飛行機にする」「―は雨に降られた」↔帰り。 
\\	地名のあとに付けて、そこが乗り物の進む目的地であることを表す。 いき。 「大阪―」 
\\	旅。 旅行。 「君が―日(け)長くなりぬ奈良路なる山斎(しま)の木立も神(かむ)さびにけり」 [下接語]売れ行き・奥行き・唐行き・柄(がら)行き・雲行き・桁(けた)行き・先行き・成り行き・梁(はり)行き・道行き・余所(よそ)行き	▲彼はボストン行きの往復切符を買った。 ▲ついに、その残酷な男に刑務所行きの判決がくだされた。
\\	譲る	ゆずる	[動ラ五(四)] 
\\	自分の物・地位・権利などを他人に与える。 譲渡する。 「財産を―・る」「後進に道を―・る」 
\\	欲しい人に売る。 「安値で―・る」 
\\	自分を後にし他人を先にする。 「席を―・る」「順番を―・る」 
\\	自分の主張を抑えて他人の主張を通させる。 譲歩する。 「自説に固執して―・らない」 
\\	他の機会にする。 「会見は後日に―・ろう」 [可能]ゆずれる	▲その仕事を彼にゆずった。 ▲ビルは絶対に彼女が彼に従うべきだと譲らなかった。
\\	豊か	ゆたか	[形動][文][ナリ] 
\\	満ち足りて不足のないさま。 十分にあるさま。 「黒髪の―な女性」「緑―な森」「才能の―な画家」「国際色―なマラソン大会」 
\\	経済的に恵まれていてゆとりのあるさま。 「―な家に育つ」「―な生活」「給料日後で懐(ふところ)が―だ」 
\\	心や態度に余裕があって、落ち着いているさま。 「―な心を育む音楽」「心―に余生を過ごす」 
\\	量感のあるさま。 「―な花房」「腰の―な丸み」 
\\	他の語に付いて、基準・限度を超えているさまを表す。 「六尺―な大男」 [派生]ゆたかさ[名]	▲なるほど彼は若いが年の割には経験が豊かだ。 ▲わたし達は皆、この方の満ち満ちた豊かさの中から、恵みの上に更に恵みを受けたのである。
\\	許す	ゆるす	[動サ五(四)]《「緩(ゆる)」「緩(ゆる)い」と同語源》 
\\	(「聴す」とも書く)不都合なことがないとして、そうすることを認める。 希望や要求などを聞き入れる。 「外出が―・される」「営業を―・す」 
\\	(「赦す」とも書く)過失や失敗などを責めないでおく。 とがめないことにする。 「あやまちを―・す」 
\\	(「赦す」とも書く)義務や負担などを引き受けなくて済むようにする。 免除する。 「税を―・す」「兵役を―・す」 
\\	相手がしたいようにさせる。 まかせる。 「追加点を―・す」「わがままを―・す」「肌を―・す」 
\\	そうするだけの自由を認める。 ある物事を可能にする。 「楽観は―・されない」「時間が―・せば」「事情の―・すかぎり」 
\\	警戒や緊張状態などをゆるめる。 うちとける。 「気を―・す」「心を―・した友」 
\\	高い評価を与える。 世間が認める。 「自他ともに―・すその道の大家」 
\\	引き張ったものをゆるめる。 「猫の綱―・しつれば」 
\\	捕らえたものを逃がす。 「つととらへて、さらに―・し聞こえず」 [可能]ゆるせる [下接句]気を許す・葷酒(くんしゆ)山門に入るを許さず・心を許す・自他共に許す・仙籍を許す・肌を許す	▲3人の中国人留学生がその大学に入学が許された。 ▲過ちは人の常、許すは神の業。
\\	夜明け	よあけ	
\\	夜が明けること。 また、その時分。 明け方。 あかつき。 「―に出発する」 
\\	日の出前、太陽の中心が地平線下の七度二一分四〇秒に来た時刻。 明け六つ。 →日暮れ 
\\	新しい時代や文化、芸術などの始まり。 「近代文学の―」	▲敵の攻撃は夜明けにやすんだ。 ▲敵の攻撃は夜明けに止んだ。
\\	酔う	よう	[動ワ五(ハ四)]《「え(酔)う」の音変化》 
\\	飲んだ酒のアルコール分が体中にまわり、正常な判断や行動がとれなくなったりする。 「美酒に―・う」「―・った勢いでけんかを売る」 
\\	乗り物に揺られたり、人込みの熱気に当てられたりして気分が悪くなる。 「船に―・う」「人に―・う」 
\\	そのことに心を奪われてうっとりする。 また、自制心を失う。 「成功に―・う」「妙技に―・う」「太平に―・う」 [可能]よえる [類語]
\\	酔っ払う・出来上がる・酩酊(めいてい)する・沈酔する・大酔(たいすい)する・泥酔する・乱酔する・飲まれる・へべれけになる・虎(とら)になる・酒気を帯びる・微醺(びくん)を帯びる/
\\	酔い痴(し)れる・浸(ひた)る・陶酔する・うっとりする・恍惚(こうこつ)となる	▲彼女は幸福に酔っている。 ▲彼女は酔っていた。
\\	容易	ようい	[名・形動]たやすいこと。 やさしいこと。 また、そのさま。 「―なことでは到達できない」「―に解ける問題」 [派生]よういさ[名]	▲その島は船で容易に行ける。 ▲その病気が急速に広がるのを防ぐのは容易な事ではなかった。
\\	容器	ようき	物を入れるうつわ。 入れ物。	▲これらの容器は気密になっている。 ▲容器を見ずに中身を見よ。
\\	陽気	ようき	[名・形動] 
\\	気候。 時候。 「春らしい―になる」 
\\	万物生成の根本となる二気の一。 万物が今まさに生まれ出て、活動しようとする気。 陽の気。 ↔陰気。 
\\	気分。 雰囲気などがはればれしていること。 にぎやかで明るいこと。 また、そのさま。 「―を装う」「天性―な人」「―にはしゃぐ」↔陰気。 [派生]ようきさ[名]	▲彼女の魅力は陽気さと親切さにある。 ▲陽気が寒くなるにつれて彼の具合がますます悪くなった。
\\	要求	ようきゅう	[名]スル 
\\	必要または当然なこととして相手に強く求めること。 「待遇改善を―する」「―を飲む」 
\\	必要とすること。 「からだが水分を―している」「時代の―」 
\\	⇒欲求 
\\	▲英語を上手に話す技能がその地位を志望する者に要求される。 ▲英国大使は大統領とじかに会見することを要求した。
\\	用心	ようじん	[名]スル 
\\	心をくばること。 気をつけること。 「風邪をひかないように―する」 
\\	万一に備えて注意・警戒を怠らないこと。 「火の―」「―の悪い家」 [類語]
\\	注意・警戒・戒心・配慮・用意・心掛け・気配り・気遣い	▲かぜをひかないよう用心しなければなりません。 ▲ここでは、すりに御用心ください。
\\	様子	ようす	《「す(子)」は唐音》 
\\	外から見てわかる物事のありさま。 状況。 状態。 「当時の―を知る人」「室内の―をうかがう」 
\\	身なり。 なりふり。 「―のいい人」 
\\	態度。 そぶり。 「悲しそうな―をする」「手持ち無沙汰な―でいる」 
\\	物事の起きそうなけはい。 兆候。 「帰る―もない」 
\\	しさい。 わけ。 事情。 「何か―がありそうだ」「―ありげな顔つき」 
\\	もったいぶること。 思わせぶり。 「どうも見て居られぬ程に―を売る男で有ッた」 [用法]様子・ありさま――「町の様子(ありさま)は変わってしまった」「被災地の悲惨な様子(ありさま)」など、状況の意では相通じて用いられる。 
\\	「様子」の方が一般的に用いられ、意味の範囲も広い。 「病人の様子がおかしい」「何か隠している様子だ」「交渉はまとまりそうな様子だ」など、外見だけでなく、そこから受ける印象も「様子」には含まれる。 この用法は「ありさま」にはない。 
\\	「ありさま」は外から見える状況の意が中心になる。 「ちょっと目を離すと、このありさまだ」のように、結果として生じた状況を表す用法は「様子」にはない。	▲その子は餓死しかかっているような様子をしていた。 ▲ディックは私たちを喜んで助けようとする様子を見せた。
\\	要するに	ようするに	[副]今まで述べてきたことをまとめれば。 かいつまんで言えば。 つまり。 「―勉強をしろということだ」「―君は何を言いたいのかね」	▲要するに、彼が不注意だったのだ。 ▲要するに、花がはちみつを作るのだ。
\\	要素	ようそ	
\\	あるものごとを成り立たせている基本的な内容や条件。 「危険な―を含む」「犯罪を構成する―」 
\\	物を分析したとき、その中に見出されるそれ以上簡単にならない成分。 「色の三―」 
\\	法律行為または意思表示の内容において、その表意者に重要な意味をもつ部分。	▲人生の最も重要な要素は驚きだ。 ▲他の条件が等しいなら、温度がこの実験でもっとも影響を与える要素であるに違いない。
\\	要点	ようてん	物事の中心となるところ。 重要な点。 「話の―をつかむ」	▲彼の話の要点が理解できなかった。 ▲彼の評論は簡潔で要点を押さえたものだった。
\\	曜日	ようび	曜の名で表した、一週間のそれぞれの日。 すなわち、日・月・火・水・木・金・土のこと。	
\\	ヨーロッパ	ヨーロッパ	(ギリシア語の
\\	から)六大州の一つ。 ユーラシア大陸の西部をなす半島状の部分と、それに付属する諸島とから成り、面積約1050万平方キロメートル。 人口約7億2600万
\\	。 北は北極海、西は大西洋に臨み、南は地中海を距ててアフリカ大陸に対し、アジアとは東はウラル山脈、南東はカフカス山脈・黒海・カスピ海で境を接する。 ギリシア・ローマの高度古代文明を経て、中世の約千年間キリスト教的統一文明圏を形成。 イギリス・ドイツ・フランス・イタリア・ロシアなど約40の独立国に分かれる。 エウロパ。 欧州。 ヨーロッパの国々 
\\	加盟国 ヨーロッパの主な山・川・湖	▲アジアはヨーロッパのほぼ4倍の大きさである。 ▲あなたはヨーロッパに行ったことが一度もないのですね。
\\	予期	よき	[名]スル前もって期待すること。 「―に反する」「―した以上の成果」→予想[用法]	▲わたしはその事業がうまくやれると予期しております。 ▲映画は、私が予期したように面白かった。
\\	横切る	よこぎる	㊀[動ラ五(四)] 
\\	横の方向に通りすぎる。 一方の側から他方の側へ渡る。 横断する。 「車道を―・る」 
\\	ふと現れて消える。 「失望の色が顔を―・る」 ㊁[動ラ下二]横に切れていく。 横にさえぎる。 「澄みのぼる月の光に―・れてわたるあきさの音の寒けさ」	▲交通の激しい通りを横切る時には、注意しなければいけません。 ▲公園を横切って歩いた。
\\	予算	よさん	[名]スル 
\\	ある計画のために、あらかじめ必要な費用を見積もること。 また、その金額。 「―を立てる」「建築―」 
\\	国または地方公共団体の一会計年度における歳入・歳出の見積もり。 議会の議決を経て成立する。 
\\	あらかじめ見積もること。 あらかじめ考えておくこと。 「細君はちゃんと主人の寿命を―して居る」	▲彼らは軍事予算を増大させようとした。 ▲彼らは苦労して1997会計年度の予算を作成した。
\\	止す	よす	[動サ五(四)]やめる。 中止する。 「いたずらは―・しなさい」「行くのは―・そう」	▲おい、馬鹿な真似はよせ。 ▲お父さんを困らせるのはおよしなさい。
\\	予測	よそく	[名]スル事の成り行きや結果を前もっておしはかること。 また、その内容。 「一〇年後の人口を―する」	▲システムのこの予測されなかった機能不全は不適切な配線系統によって引き起こされた。 ▲その時計会社は年間100万個以上の新しい時計を製造すると予測されている。
\\	ヨット	ヨット	巡航または競走に用いる小型の帆走船。 発動機などの推進機関をもつものもある。 オリンピック競技ではフィン級・四七〇級など七種目がある。 《季 夏》 ◆英語では大型のものをさし、日本で普通にいう小型のものは
\\	とよぶ。	▲私はヨット部に入っている。 ▲私は昨年ヨットの操縦を始めた。
\\	夜中	よなか	夜のなかば。 夜ふけ。 夜半。 「―まで起きている」	▲ジョンは夜中まで起きている習慣である。 ▲たとえ夜中すぎまで起きていなければならなくてもあなたを待っています。
\\	世の中	よのなか	
\\	人々が互いにかかわり合って生きて暮らしていく場。 世間。 社会。 「―が騒がしくなる」「暮らしにくい―になる」 
\\	世間の人々の間。 また、社会の人間関係。 「―はもちつもたれつだ」「親も友達もないんです。 つまり―がないんですね」 
\\	世間のならい。 「病気が出るほど嫌な人でも、―にゃ勝たれないから」 
\\	当世。 その時分。 「入道殿をはじめ参らせて―におはしある人、参らぬはなかりけり」 
\\	統治者の在任期間。 「―かはりて後、よろづ物うくおぼされ」 
\\	世間的な人望。 「父殿うせ給ひにしかば、―おとろへなどして」 
\\	男女の関係。 男女間の情愛。 「歌はよまざりけれど、―を思ひ知りたりけり」 
\\	人の一生。 寿命。 「―の今日か明日かに覚え侍りし程に」 
\\	外界のようす。 あたりの自然。 「秋待ちつけて、―すこし涼しくなりては」 
\\	作物のできばえ。 「播磨路の―が悪うて」	▲君が居なくてはどんなにさびしい世の中になることだろう。 ▲厳しい世の中だなあ。
\\	余分	よぶん	[名・形動] 
\\	余った分。 残り。 余り。 「―が出る」 
\\	必要や予定より多いこと。 また、その数量や、そのさま。 余計。 「―に仕入れる」 
\\	必要以外のこと。 また、そのさま。 余計。 「―なことは考えないほうがいい」「―な口出しはするな」	▲余分の時間がたくさんある。 ▲比較的活動していない状態で、風にさらされていなければ、熊は寒い天候においても余分なエネルギーを消費することはない。
\\	予報	よほう	[名]スル 
\\	前もって知らせること。 また、その知らせ。 「将来の社益を慮て此に―す」 
\\	「天気予報」の略。将来のある期間におけるある地域の天気を予報すること。期間によって短期予報・週間予報・長期予報など、目的によって航空気象予報・船舶気象予報・農業気象予報などがある。 「―によれば明日は雨だ」	▲天気の予報は過去のデータに基づいて行われる。 ▲天気は科学的に予報される。
\\	予防	よぼう	[名]スル悪い事態の起こらないように前もってふせぐこと。 「病気の蔓延を―する」	▲今週は火災予防週間です。 ▲迅速な行動をとれば、将来起こる問題の予防になる。
\\	読み	よみ	
\\	文字・文章を読むこと。 読む方法。 「―、書き、そろばん」「斜め―」 
\\	(「訓み」とも書く)漢字・漢文に国語の読み方をあてること。 訓。 
\\	人の心中や物事の成り行きを深く見通すこと。 「―が深い」「―が外れる」 
\\	碁・将棋で打つ手順をさきざきまで見通すこと。 
\\	「読みガルタ」の略。	▲それを理解するには、この本を読みさえすればいい。 ▲もう2、3ページ読みさえすればいい。
\\	嫁・娵	よめ	
\\	結婚して夫の家族の一員となった女性。 「―に行く」↔婿。 
\\	息子の妻となる女性。 「長男の―を探す」↔婿。 
\\	妻。 また、他人の妻をいう語。 「彼の―さんは働き者だ」	▲私の夫には、兄が2人います。(そう、私は三男の嫁です)。 ▲嫁と姑の中は大嵐。
\\	余裕	よゆう	
\\	必要分以上に余りがあること。 また、限度いっぱいまでには余りがあること。 「金に―がある」「時間の―がない」「まだ席に―がある」 
\\	ゆったりと落ち着いていること。 心にゆとりがあること。 「―の話し振り」「周りを見る―もない」	▲旅行は私には余裕のない贅沢である。 ▲洋服を定期的に買う余裕はありません。
\\	因る・拠る・由る・依る	よる	[動ラ五(四)]《「寄る」と同語源》 
\\	(因る・由る)それを原因とする。 起因する。 「濃霧に―・る欠航」「成功は市民の協力に―・る」 
\\	(依る)物事の性質や内容などに関係する。 応じる。 従う。 「時と場合に―・る」「人に―・って感想が違う」「成功は努力いかんに―・る」 
\\	(依る)動作の主体をだれと指し示す。 「市民楽団に―・る演奏」 
\\	(依る)それと限る。 「だれに―・らず文句を言う」「何事に―・らず好き嫌いが多い」 
\\	(依る)手段とする。 「機械に―・る生産」「挙手に―・る採決」 
\\	(依る)頼る。 依存する。 「信仰に―・って生きる」「年金に―・って生計を立てる」 
\\	(拠る)根拠とする。 「実験に―・る結論」「天気予報に―・ると大雨らしい」 
\\	(拠る)よりどころとする。 のっとる。 「法律の定めるところに―・る」 
\\	(拠る)根拠地とする。 たてこもる。 「城塞に―・って戦う」 [可能]よれる [類語]
\\	基づく・由来する・因由する・原因する・起因する・根差す/
\\	基づく・従う・則(のつと)る・準ずる・準拠する・依拠する・立脚する・憑依(ひようい)する・徴する	▲造られたもので、この方によらずできたものは一つもない。 ▲人前で話すようなことは何によらず彼はいつも敬遠する。
\\	喜び・悦び・歓び・慶び	よろこび	
\\	よろこぶこと。 うれしく思うこと。 「苦労の分だけ―も大きい」「受賞の―をかみしめる」↔悲しみ。 
\\	祝いごと。 おめでた。 慶事。 「結婚、栄転と―が続く」 
\\	祝うこと。 祝いの言葉。 「新年の―を寿(ことほ)ぐ」 
\\	与えられた慶事や好意に対するお礼。 また、その謝辞。 「紀介殿ただ貸し給へかし、―は思ひ当らん」	▲健康は富に優る、後者は前者ほどに喜びを与えない。 ▲結婚には多くの苦悩があるが、独身には何の喜びもない。
\\	来	らい	㊀[連体](日付・年月などで)この次の。 きたる。 「―場所」「―シーズン」 ㊁〔接尾〕時などを表す語に付いて、その時から現在まで続いている意を表す。 以来。 このかた。 「数日―」「昨年―」「一別―」	▲来年末までで、ここで働き始めてどれくらいの期間になりますか。 ▲彼女は数ヶ月来はじめて悲しく思った。
\\	ライター	ライター	点火する器具。 特に、タバコに火をつける小形の器具。 「ガス―」	▲お客様テーブルにライターがおわすれですよ。 ▲彼はポケットに手を入れてライターをさがした。
\\	楽	らく	㊀[名・形動] 
\\	心身に苦痛などがなく、快く安らかなこと。 また、そのさま。 「気が―になる」「―な姿勢」「どうぞお―に」 
\\	生計が豊かなこと。 また、そのさま。 「不動産収入で―な暮らしをする」 
\\	たやすいこと。 簡単なこと。 また、そのさま。 「―な計算問題」「―に勝てる相手」 ㊁[名] 
\\	「千秋楽(せんしゆうらく)」の略。 「今日で―を迎える」 
\\	「楽焼き」の略。 [アクセント] ❶はラク、 ❷はラク。 [類語] 
\\	快適・安楽・安逸・気楽/
\\	容易・簡単・容易(たやす)い・訳(わけ)無い・与(くみ)し易(やす)い・楽楽・易易(やすやす)・軽く・悠悠・難無く・苦も無く	▲その仕事は楽にできるだろう。 ▲その少年は連立方程式を楽に解いた。
\\	ラケット	ラケット	テニス・卓球・バドミントンなどで、ボールやシャトルコックを打つ用具。	▲ラケット1本とテニスシューズを送ってください。 ▲マイクは良いラケットを持っている。
\\	利益	りえき	[名]スル 
\\	事業などをして得るもうけ。 利潤。 「莫大(ばくだい)な―を上げる」↔損失。 
\\	得になること。 益になること。 「遥に労働者を―するに足るだろう」 [類語]
\\	儲(もう)け・得(とく)・利・利得・利潤・利沢・黒字・得分(とくぶん)・益・実益・収益・益金・利金・純利・純益・差益・利鞘(りざや)・マージン・ゲイン・プロフィット/
\\	益・得・為(ため)・裨益(ひえき)・便益・実利・メリット・得る所	▲私は自分の車を売って大きな利益を得た。 ▲私は物質的な利益に関心がない。
\\	理解	りかい	[名]スル 
\\	物事の道理や筋道が正しくわかること。 意味・内容をのみこむこと。 「―が早い」 
\\	他人の気持ちや立場を察すること。 「彼の苦境を―する」 
\\	「了解」に同じ。 →了解(りようかい)[用法]	▲その先生は自分の考えを学生達に理解させるのは困難だとわかった。 ▲その説明は私には理解できなかった。
\\	陸	りく	陸地。 おか。 くが。 「―に上がる」↔海。	▲陸と水で地球の表面は出来ている。 ▲陸が見えてきた。
\\	利口・悧口	りこう	[名・形動] 
\\	頭がよいこと。 賢いこと。 また、そのさま。 利発。 「―な犬」「―そうな人」「―ぶる」↔馬鹿(ばか)。 
\\	要領よく抜け目のないこと。 また、そのさま。 「―に立ちまわる」 
\\	(多く「お利口」の形で)子供などがおとなしく聞きわけのよいこと。 また、そのさま。 「お―にしている」「お―さん」 
\\	巧みにものを言うこと。 口先のうまいこと。 また、そのさま。 「興言―は放遊境を得るの時談話に虚言を成し」 
\\	軽口を言うこと。 冗談。 「『またお上人様の―仰せらるる』『いや―ではない。 有様(ありやう)ぢゃ』」 [派生]りこうさ[名] [類語]
\\	利発・怜悧(れいり)・聡明(そうめい)・発明・慧敏(けいびん)・明敏・才気煥発(かんぱつ)・穎悟(えいご)・利根・賢明・賢い・聡(さと)い/
\\	賢(さか)しい・小賢(こざか)しい・うまい・クレバー・抜け目がない	▲僕は君くらい利口だといいのだが。 ▲黙っていた方が利口だと彼は考えた。
\\	離婚	りこん	[名]スル夫婦が生存中に法律上の婚姻関係を解消すること。 日本では、協議離婚・調停離婚・審判離婚・裁判離婚の四種がある。	▲彼は妻に離婚しないでくれと説得した。 ▲彼は妻と離婚したという事実を隠した。
\\	理想	りそう	
\\	人が心に描き求め続ける、それ以上望むところのない完全なもの。 そうあってほしいと思う最高の状態。 「―を高く掲げる」↔現実。 
\\	理性によって考えうる最も完全な状態。 また、実現したいと願う最善の目標あるいは状態。	▲彼は自分の理想を具体化したいと思っている。 ▲彼女は僕の理想の人です。
\\	率	りつ	
\\	割合。 比率。 歩合(ぶあい)。 「成功する―の高い実験」「正解―」 
\\	かけた手間や時間に対する効果の程度。 「―の悪い仕事」	▲賭け率2対1でレッズが勝つだろう。 ▲私たちは割引率についての意見が一致した。
\\	留学	りゅうがく	[名]スル他の土地、特に外国に在留して学ぶこと。 「イギリスへ―する」「内地―」	▲彼は留学できるように一生懸命勉強している。 ▲彼は留学するチャンスがほしいと強く望んでいる。
\\	量	りょう	
\\	測定の対象となり、大小の比較が可能なもの。 質量・長さ・時間・個数など。 また、測定して得られる数値や限度。 「―が多い」「―より質」 
\\	論理学で、判断が全称判断か特称判断かということ。 
\\	インド哲学漢訳術語で、知識一般のこと。 直接知覚による認識を現量、それを超える対象の論証を比量という。	▲アメリカは世界の1/4の二酸化炭素を排出しており、一人当たりの排出量も世界で最も多いのです。 ▲暖房と同様に、冷房時の消費電力量を測定し、旧型と省エネ型(2001年製)のエアコンを比較しました。
\\	両替	りょうがえ	[名]スル 
\\	ある種の貨幣をそれと等しい額の他の種類の貨幣と交換すること。 「千円札を―する」「円をドルに―する」 
\\	有価証券や物品などを現金と交換すること。 「当たり馬券を―する」	▲私は銀行で円を少しドルに両替した。 ▲今日の両替のレートはいくらですか。
\\	料金	りょうきん	何かを使用または利用したことに対して支払う金銭。 運輸機関では、運賃とは別に支払われる、グリーン車・寝台などの使用代金をいう。 「電話―」「特急―」	▲新サービスの料金設定はマーケティング部が行う。 ▲裁判所はその料金を支払うように命じた。
\\	礼	れい	
\\	社会秩序を保ち、人間関係を円滑に維持するために守るべき、社会生活上の規範。 礼儀作法・制度など。 「―にかなったやり方」「―を失する」「―を尽くす」 
\\	敬意を表すために頭を下げること。 おじぎ。 「先生に―をする」 
\\	謝意を表すこと。 また、その言葉。 また、謝礼のために贈る金品。 「本を借りた―を言う」「世話になった人に―をする」 
\\	儀式。 「即位の―」 [アクセント] 
\\	はレイ、 
\\	はレイ。 [類語]
\\	礼儀・礼節・儀礼・礼式・礼法・作法(さほう)・風儀・マナー・エチケット/
\\	お辞儀・一礼・敬礼・最敬礼・黙礼・拝礼・低頭/
\\	謝礼・返礼・報礼・謝儀・志(こころざし)・礼物・礼金・謝金・報謝・報酬・薄謝・薄志・謝辞	▲彼は目で礼を言った。 ▲彼女の親切な助力に対して彼は礼を述べた。
\\	例	れい	㊀[名] 
\\	以前からのやり方。 しきたり。 ならわし。 慣習。 「長年の―にならう」 
\\	過去または現在の事実で、典拠・標準とするに足る事柄。 「古今に―を見ない」 
\\	他を説明するために、同類の中から引いて示す事柄。 「―を挙げて説明する」「その―に漏れない」 
\\	いつものとおりであること。 「―によって話が大きい」 ㊁[副]いつも。 つねづね。 「―ある所にはなくて」 [類語]
\\	習い・習わし・仕来(しきた)り・慣行・慣例・常例・定例・通例/
\\	例(ためし)・先例・前例・先蹤(せんしよう)・事例・類例・類(るい)/
\\	実例・一例・具体例・例証・たとえ・引き合い	▲春ごとに恋は例のいたずらを始める。 ▲例のスキャンダルはそういつまでも臭いものにフタというわけにはいくまい。いずれ人は嗅ぎつけてしまうさ。
\\	礼儀	れいぎ	
\\	人間関係や社会生活の秩序を維持するために人が守るべき行動様式。 特に、敬意を表す作法。 「―にかなう」「―正しい人」「親しき中にも―あり」「―作法」 
\\	謝礼。 報謝。 「それは―いかほど入り候はんや」 [用法]礼儀・作法――「物を食べながら人に会うのは礼儀(作法)に反する」のように、人と接する時の態度の意では相通じて用いられる。 
\\	「礼儀」は対人関係での気配りや敬意、慎しみの気持ちにもとづく行動の規範である。 「作法」は対人関係に限らず、礼儀にかなった一定の行動のしかたを言う。 
\\	「年賀状をもらったら返事を出すのが礼儀だ」は「作法」では言えないし、「お茶の作法を覚える」を「礼儀」とは言えない。 
\\	「作法を知らない」と言えば単にその知識がないだけである場合も多いが、「礼儀を知らない」では、敬意や慎しみの気持ちがなく、常識に欠けることを非難する意が含まれてくる。 
\\	類似の語に「行儀」がある。 「行儀の悪い子」「行儀よくすわっている」のように、礼儀にかなった立ち居のしかたの意で使う。	▲彼が礼儀正しいので、私たちは尊敬している。 ▲堅苦しい礼儀は一切抜きにしましょう。
\\	冷静	れいせい	[名・形動]感情に左右されず、落ち着いていること。 また、そのさま。 「―な判断」「―に処理する」 [派生]れいせいさ[名]	▲冷静になろうとしたのだが、とうとうかっとなった。 ▲冷静な判断を必要とする状況である。
\\	列	れつ	㊀[名] 
\\	順に長く並ぶこと。 連なること。 また、そのもの。 ならび。 行列。 「―になる」「―を作る」「―を離れる」 
\\	仲間。 「大臣の―に連なる」「大国の―にはいる」 
\\	数学で、行列または行列式での縦の並び。 ㊁〔接尾〕助数詞。 つらなっているものを数えるのに用いる。 「二―に並ぶ」	▲延々と続く車の列があった。 ▲英国人は列を作って並ぶのに慣れている。
\\	列車	れっしゃ	旅客や貨物を運ぶために線路上を走る連結した車両。 「長距離―」	▲その列車は2時間前に出発した。 ▲その列車は8時にパリに到着する予定だった。
\\	連想・聯想	れんそう	[名]スル 
\\	ある事柄から、それと関連のある事柄を思い浮かべること。 また、その想念。 「雲を見て綿菓子を―する」 
\\	心理学で、ある観念の意味内容・発音・外形の類似などにつれて、他の観念が起きてくること。 観念連合。 →連合 
\\	▲私はイチゴと言うとショートケーキを連想する。 ▲私はこの歌を聞くと彼の名を連想する。
\\	連続	れんぞく	[名]スル 
\\	切れ目なく続くこと。 また、続けること。 「不祥事が―する」「三日―」 
\\	数学で、関数
\\	で定義域内の点aにxが近づくときの極限値が存在し、
\\	に等しいとき、
\\	は
\\	で連続であるという。	▲彼の生涯は長い失敗の連続だった。 ▲日本経済は連続60ヶ月以上の拡大を記録した。
\\	老人	ろうじん	年をとった人。 年寄り。 老人福祉法では、六五歳以上をいう。 「―医療」 [用法]老人・としより――「老人」は、文章やあらたまった話の中では最も一般的に使われる語。 特に「老人福祉」「老人ホーム」のように複合語を作る場合、「年寄り」は使わないのが普通。 
\\	「年寄り」は「老人」よりややくだけた親しみのある感じで使われる。 前後関係によって軽蔑の感じが強く出ることもある。 「お年寄りを大切にしよう」「年寄りの冷や水」など。 
\\	類似の語に「老体」がある。 「老体」は「御老体を煩わせてすみません」というような形で尊敬をこめて言う場合にも用いる。 
\\	「老人」が個人を指す場合は男であることが多い。 女性については「老婦人」「老女」「老婆」などを用いることが多い。 「年寄り」「老体」にはこのような使い分けはない。	▲私は老人に遠慮して少し離れていた。 ▲私は老人がその席に着けるように立ち上がった。
\\	労働	ろうどう	[名]スル 
\\	からだを使って働くこと。 特に、収入を得る目的で、からだや知能を使って働くこと。 「工場で―する」「時間外―」「頭脳―」 
\\	経済学で、生産に向けられる人間の努力ないし活動。 自然に働きかけてこれを変化させ、生産手段や生活手段をつくりだす人間の活動。 労働力の使用・消費。 [類語]
\\	仕事・勤労・作業・労作・労務・労役・実働・稼働・働き	▲むかしのような激しい日雇い労働はできやしない。 ▲その結果、同一労働に対し同一賃金を得ている女性が増えつつある。
\\	ロケット	ロケット	推進剤を燃焼させ、噴出するガスの反動によって前進する装置。 また、それで推進される飛行体。	▲数年前の母の日に、義母にロケットをプレゼントしました。 ▲ロケットは通信衛星を軌道に乗せた。
\\	論じる	ろんじる	[動ザ上一]「ろん(論)ずる」(サ変)の上一段化。 「事の是非を―・じる」	▲しかし、この主張は、デネットが論じているものとは違うということを強調しておきたい。 ▲スミスはこの事例にはどのような国際法も適用できないと論じている。
\\	論争	ろんそう	[名]スル互いに言い争うこと。 「人類の起源について―する」	▲彼らは幾年間も、その土地の所有権について論争した。 ▲彼女は友達と論争して全く疲れきっていた。
\\	論文	ろんぶん	
\\	論議する文。 筋道を立てて述べた文。 
\\	学術的な研究の結果などを述べた文章。 「博士―」	▲その論文は注意深く研究するのに値する。 ▲その論文を翻訳するには少なくとも3日は必要です。
\\	輪・環	わ	
\\	曲げて円形にしたもの。 また、円い輪郭。 環(かん)。 「鳥が―を描いて飛ぶ」「指―」「花―」 
\\	軸について回転し、車を進めるための円形の具。 車輪。 「荷車の―が外れる」 
\\	桶(おけ)などのたが。 「桶の―がゆるむ」 
\\	人のつながりを 
\\	に見立てていう語。 「友情の―を広げる」 
\\	紋所の名。 円形を図案化したもの。 [下接語]浮き輪・渦輪・内輪・腕輪・襟輪・面(おも)輪・貝輪・金(かな)輪・唐(から)輪・口輪・首輪・曲(くる)輪・ゴム輪・後(しず)輪・外輪・台輪・知恵の輪・竹(ちく)輪・稚児(ちご)輪・茅(ち)の輪・月の輪・吊(つ)り輪・弦(つる)輪・泣き輪・喉(のど)輪・花輪・鼻輪・埴(はに)輪・吹き輪・前輪・三つ輪・耳輪・指輪・両輪	▲英語の授業では、時々輪になって座り、読んでいる本について話し合いをすることもあります。 ▲鎖の強さはその環の一番弱いところに左右される。
\\	ワイン	ワイン	
\\	ぶどう酒。 
\\	酒。 酒類。 「ピーチ―」	▲彼はケイトにワインを飲ませない。 ▲彼はブドウからワインを作る。
\\	我が儘	わがまま	㊀[名・形動]自分の思いどおりに振る舞うこと。 また、そのさま。 気まま。 ほしいまま。 自分勝手。 「―を通す」「―な人」 ㊁〔連語〕《代名詞「わ」+助詞「が」+名詞「まま」》自分の思いのまま。 「―に誇りならひたる乳母の」	▲その子供はあくまでわがままを通そうとする。 ▲わがままな子供を満足させることはできない。
\\	別れ	わかれ	
\\	別れること。 互いに離れて別々になること。 別離。 「友との―を惜しむ」「―の日を迎える」「―の杯」 
\\	別離のあいさつ。 いとまごい。 「故郷に―を告げる」「―の言葉」 
\\	死に別れ。 死別。 「永(なが)の―」「世の中にさらぬ―のなくもがな千世もと祈る人の子のため」 
\\	立ち去るにあたって、心付けとして与える金銭。 「―に七百くだんせ」 [下接語]暁の別れ・生き別れ・扇の別れ・更衣(きさらぎ)の別れ・食い別れ・喧嘩(けんか)別れ・子別れ・逆さ別れ・四鳥(しちよう)の別れ・死に別れ・終(つい)の別れ・永の別れ・泣き別れ・夫婦別れ・物別れ・行き別れ・別れ別れ	▲彼は我々に丁寧に別れを告げた。 ▲彼は学生に別れを告げた。
\\	脇・腋・掖	わき	
\\	両腕の付け根のすぐ下の所。 また、体側とひじとの間。 わきのした。 「本を―に抱える」 
\\	衣服で、 
\\	にあたる部分。 「洋服の―を詰める」 
\\	(「傍」「側」とも書く)すぐそば。 かたわら。 「門の―に車をとめる」 
\\	目ざすものからずれた方向。 よそ。 横。 「話題が―にそれる」「―を見る」 
\\	「脇句」の略。 「―をつける」 
\\	平安時代、相撲人(すまいびと)のうちで最手(ほて)に次ぐ地位の者。 今の関脇にあたる。 ほてわき。 
\\	(ふつう「ワキ」と書く)能で、シテの相手役。 また、その演者。 原則として現実の男性の役で、面はつけない。 
\\	邦楽で、首席奏者(タテ)に次ぐ奏者。 また、その地位。 演奏するものによって、脇唄・脇三味線・脇鼓などという。 →側(そば)[用法]	▲車のわきの男の子をごらんなさい。 ▲彼女は白い本を脇に抱えていた。
\\	分ける・別ける	わける	[動カ下一][文]わ・く[カ下二] 
\\	一つにまとまっているものをいくつかの部分にする。 分割する。 「ドラマを前半と後半に―・ける」「五回に―・けて支払う」「髪を七三(しちさん)に―・ける」 
\\	種類によって区分する。 分類する。 「子供と大人に―・ける」「大きさによって―・ける」 
\\	幾つかに割って与える。 分配する。 また、一部分を人に与える。 「財産を三人の息子に―・ける」「いただき物をお隣に―・ける」 
\\	「売る」を婉曲にいう語。 「この絵を―・けて下さいませんか」 
\\	物を左右に押し開く。 「人波を―・けて前に出る」「草の根―・けて捜し出す」 
\\	勝負事で、勝負がつかないとして、やめさせる。 引き分けにする。 「勝負を―・ける」「星を―・ける」 
\\	仲裁して、やめさせる。 「けんかを―・ける」 [類語]
\\	分かつ・仕切る・区切る・区分けする・区分する・分節する・分割する・等分する・均分する/
\\	分類する・類別する・区別する・峻別(しゆんべつ)する・区分する・分別する・仕分けする・色分けする・品分けする/
\\	分かつ・配分する・分配する・分与する・案分する・折半する・山分けする [下接句]馬の背を分ける・事を分ける・血を分ける・暖簾(のれん)を分ける・夕立は馬の背を分ける・理を分ける	▲母は子供たちに金をわけてやった。 ▲霧が深かったので、歩いている人たちの姿を似分けるのは困難だった。
\\	技	わざ	
\\	ある物事を行うための一定の方法や手段。 技術。 技芸。 「―を磨く」「―を競う」 
\\	相撲・柔道などで、相手を負かすために仕掛ける一定の型に基づいた動作。 「―がきまる」「寝―」	▲柔道では力より技のほうが大切である。 ▲一子相伝の技と言うわりには、彼の蹴りは大した事はないね。
\\	態と	わざと	[副]《名詞「わざ(業)」+格助詞「と」から》 
\\	意識して、また、意図的に何かをするさま。 ことさら。 故意に。 わざわざ。 「―負ける」 
\\	とりわけ目立つさま。 格別に。 「―深き御敵と聞こゆるもなし」 
\\	正式であるさま。 本格的に。 「―の御学問はさるものにて」 
\\	事新しく行うさま。 「―かう立ち寄り給へること」 
\\	ほんのちょっと。 少しばかり。 「ではござりませうが、―一口」	▲私はわざとその花瓶を割った。 ▲私はわざと彼女の気持ちを傷付けた。
\\	僅か・纔か	わずか	[形動][文][ナリ] 
\\	数量・程度・価値・時間などがほんのすこしであるさま。 副詞的にも用いる。 「―な金の事でいがみ合う」「―な食料しかない」「―に制限重量をオーバーする」「ここから―一〇分の距離」 
\\	(多く「わずかに」の形で用いて)そうするのがやっとであるさま。 かろうじて。 「―に記憶している」「―に難を逃れた」 
\\	ささやかで粗末なさま。 「―なる腰折文つくることなど習ひ侍りしかば」 [用法]わずか・かすか――「わずかな(かすかな)痛み」「わずかに(かすかに)息をしている」など、程度が少ない意では相通じて用いられる。 
\\	「わずか」は数量・価値・時間など具体的な事柄について多く使う。 「わずかなすき間」「わずかなことが原因で対立する」「乗り換え時間はわずか五分しかない」
\\	「かすか」は色・音・匂いなど感覚的な事柄について使うことが多い。 「かすかに助けを呼ぶ声がする」「ほおにかすかな赤みがさす」「かすかに香水が匂う」
\\	「わずかな記憶」は断片的に残る記憶であり、「かすかな記憶」は、あいまいなはっきりしない記憶である。	▲わずか36ヶ月後に、ココは184語にあたる手振りを使うことができた。 ▲ほんのわずかの人しか彼の話を聞かなかった。
\\	綿・棉・草綿	わた	
\\	アオイ科ワタ属の植物の総称。 古くから重要な繊維作物として栽培され、アジア綿(めん)・エジプト綿・海島綿・陸地綿などがある。 日本では江戸時代から盛んになった。 栽培されるのはインドワタの変種で、一年草。 高さ約一メートル。 葉は手のひら状に三〜五つに裂ける。 夏から秋に、黄や紅色の五弁花が咲く。 果実は卵形で、褐色に熟すと裂開し、中の多数の種子に生じた白く長い毛が露出する。 この実綿(みわた)を摘み取り、毛(綿花)と種子(綿実(めんじつ))とに分けて利用する。 《季 花=夏 実=秋》「―の実を摘みゐてうたふこともなし/楸邨」 
\\	木綿綿(もめんわた)・真綿・絹綿・化学繊維綿などの総称。 古くは蚕の繭からとった真綿をさしたが、木綿が普及してからは主に木綿綿をさすようになった。 綿織物などの紡績用や布団綿・中入れ綿・脱脂綿などに利用。 《季 冬》「―を干す寂光院を垣間(かいま)見ぬ/虚子」 [下接語]厚綿・石綿・入れ綿・薄綿・打ち綿・青梅(おうめ)綿・置き綿・菊の被(きせ)綿・着せ綿・絹綿・木綿・繰り綿・小袖(こそで)綿・裾(すそ)綿・種綿・血綿・摘み綿・唐(とう)綿・中綿・抜き綿・引き綿・含み綿・布団綿・穂綿・丸綿・真綿・木綿(もめん)綿・結(ゆい)綿	▲我々はその畑に今年は綿を植え付けるつもりだ。 ▲綿は水を吸収する。
\\	話題	わだい	話の題目。 談話・文章などの中心的な材料。 話の種。 「―が尽きない」「―にのぼる」	▲高度に専門的な話題で、多分に誤った情報が含まれる可能性があることはあらかじめお断りしておきます。 ▲「きれいな宝石ですね」、適当な話題かどうかわからないが、とりあえずそう水を向けてみた。
\\	笑い	わらい	
\\	笑うこと。 また、その声。 えみ。 「儲(もう)かりすぎて―がとまらない」 
\\	(「嗤い」とも書く)あざけり笑うこと。 嘲笑(ちようしよう)。 「聴衆の―をかう」 
\\	性に関係するもの、春画・淫具などの総称。 
\\	石を積むとき、間にモルタルなどを詰めず、少しあけておくこと。 また、そのあけた所。 [下接語]愛嬌(あいきよう)笑い・愛想(あいそ)笑い・薄ら笑い・薄笑い・大笑い・思い出し笑い・豪傑笑い・忍び笑い・せせら笑い・空笑い・高笑い・千葉笑い・追従(ついしよう)笑い・作り笑い・泣き笑い・苦笑い・盗み笑い・馬鹿(ばか)笑い・初笑い・独り笑い・含み笑い・福笑い・物笑い・貰(もら)い笑い	▲赤ん坊は敵意のある笑いができるほど年をとってはいない。 ▲私は笑いをこらえることができなかった。
\\	割る	わる	㊀[動ラ五(四)] 
\\	強い力を加え固体の物をいくつかに分けて離す。 「茶碗を―・る」「クルミを―・る」「まきを―・る」 
\\	ある物をいくつかの部分に分ける。 「土地を三つに―・る」「部屋を―・って使う」 
\\	まとまっているもの、組織などを分裂させる。 「党を―・る」 
\\	押し分けて間を離す。 「両者の間に―・ってはいる」 
\\	割り算をする。 除する。 「六を二で―・る」 
\\	分けて与える。 配分する。 割り当てる。 「それぞれに役を―・る」「頭数で―・る」 
\\	他の液体にまぜて濃度を薄める。 「ブランデーを水で―・る」 
\\	心のうちを隠さずにすっかり出す。 うちあける。 また、白状する。 「腹を―・って話す」「口を―・る」 
\\	一定数に達しないで下回る。 ある水準以下になる。 「志願者が定員を―・る」「仕入れ値を―・る」 
\\	きまった範囲の外に出る。 「土俵を―・る」 
\\	(サッカーなどで)ラインをこえる。 「ボールがタッチラインを―・る」 
\\	突き当たったり切ったりして傷をつける。 できた傷の部分を開いた状態にする。 「激しい申し合いで額を―・る」 
\\	(相撲で「腰をわる」の形で)足を開き膝を曲げ、体をまっすぐにした姿勢で腰を低くする。 「腰を―・って寄る」 
\\	追い求めて捜し出す。 つきとめる。 「ほしを―・る」 
\\	手形を割り引く。 「手形を―・る」 
\\	わけを細かく説明する。 「―・っつ砕いつ叱れども」 [可能]われる ㊁[動ラ下二]「われる」の文語形。 [類語] 
\\	打ち割る・叩(たた)き割る・叩き壊す・壊す・砕く・欠く・打(ぶ)っ欠く・破砕する/
\\	除する・等分する・均分する/
\\	割り込む・下回る・切る・下(くだ)る [下接句]川口で船を破(わ)る・口を割る・尻(けつ)を割る・腰を割る・尻(しり)を割る・底を割る・竹を割ったよう・土俵を割る・腹を割る・枕(まくら)を割る・水を割る	▲彼はその花瓶を割ったと白状した。 ▲彼はバスを待っている人の列に割ってはいた。
\\	我我	われわれ	
\\	⦅代⦆ 
\\	わたくしたち。 われら。 
\\	自分の謙称。 わたくしのようなもの。 昨日は今日の物語「立ち聞きしたる男…―はちとよそへ参り候ふほどに」 
\\	⦅名⦆ 自分自分。 めいめい。 各自。 論語抄「官々職々の者―の事をすべし」	▲我々の計画には他にも多くの利点がある。 ▲我々の計画には多くの利点がある。
\\	湾	わん	海が陸地に大きく入り込んでいる海面。	▲丘の上に立っているので、そのホテルは湾の見晴らしがすばらしい。 ▲私たちの家からは下に湾が見える。
\\	訳	わけ	《「分け」と同語源》 
\\	物事の道理。 すじみち。 「―のわからない人」「―を説明する」 
\\	言葉などの表す内容、意味。 「言うことの―がわからない」 
\\	理由。 事情。 いきさつ。 「これには深い―がある」「どうした―かきげんが悪い」 
\\	男女間のいきさつ。 また、情事。 「―のありそうな二人」 
\\	(「わけにはいかない」の形で)そうすることはできない。 筋道ではない。 「やらない―にはいかない」 
\\	(「わけではない」の形で)否定・断定をやわらげた言い方。 「だからといって君が憎い―ではない」「別に反対する―ではない」 
\\	結果として、それが当然であること。 「やっと今度、その宿望がかなった―です」 
\\	遊里のしきたりや作法。 「色道の―を会得せねば」 [下接語]言い訳・入り訳・内訳・事訳・諸(しよ)訳・申し訳	▲その二人の学生は両方ともがテストに合格したわけではなかった。 ▲その件について自分の意見を述べないわけには行かない。
\\	ライター	ライター	文章を書くことを職業とする人。 著作家。 「ルポ―」「コピー―」	▲お客様テーブルにライターがおわすれですよ。 ▲ライター持ってる?
\\	ロケット	ロケット	写真などを入れて身につける小型の容器。 鎖に通して首から下げることが多い。	▲数年前の母の日に、義母にロケットをプレゼントしました。 ▲ロケットは月を回る軌道に乗っている。
\\	本・元	もと	㊀[名] 
\\	物事の起こり。 始まり。 「事件の―をさぐる」「うわさの―をただす」 
\\	(「基」とも書く)物事の根本をなすところ。 基本。 「生活の―を正す」「悪の―を断つ」 
\\	(「基」とも書く)基礎。 根拠。 土台。 「何を―に私を疑うのか」「事実を―にして書かれた小説」 
\\	(「因」とも書く)原因。 「酒が―でけんかする」「風邪は万病の―」 
\\	もとで。 資金。 また、原価。 仕入れ値。 「―がかからない商売」「―をとる」 
\\	(「素」とも書く)原料。 材料。 たね。 「たれの―」「料理の―を仕込む」 
\\	それを出したところ。 それが出てくるところ。 「火の―」「製造―」「販売―」 
\\	ねもと。 付け根。 「―が枯れる」「葉柄の―」 
\\	箸(はし)や筆の、手に持つ部分。 
\\	短歌の上の句。 「歌どもの―を仰せられて」 ㊁〔接尾〕(本)助数詞。 
\\	草や木を数えるのに用いる。 「一(ひと)―の松」 
\\	鷹(たか)狩りに使う鷹を数えるのに用いる。 「いづくよりとなく大鷹一―それて来たり」 [下接句]孝は百行(ひやつこう)の本・失敗は成功のもと・短気は未練の元・釣り合わぬは不縁の基・生兵法は大怪我(おおけが)の基・油断は怪我(けが)の基	▲それを元の所へ戻しなさい。 ▲おもしろい話しのほとんどがこっけいな場面を基にしている。
\\	曜日	ようび	曜の名で表した、一週間のそれぞれの日。 すなわち、日・月・火・水・木・金・土のこと。	
\end{CJK}
\end{document}