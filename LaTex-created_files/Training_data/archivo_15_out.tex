\documentclass[8pt]{extreport} 
\usepackage{hyperref}
\usepackage{CJKutf8}
\begin{document}
\begin{CJK}{UTF8}{min}
\\	メアリーさん、週末はたいてい何をしますか。	メアリーさん、しゅうまつはたいていなにをしますか 
\\	たいていうちで勉強します。でも、ときどき映画を見ます。	たいていうちでべんきょうします。でも、ときどきえいがをみます。 
\\	土曜日に映画を見ませんか。	どようびにえいがをみませんか 
\\	今日は京都に行きます。	きょうはきょうとにいきます 
\\	図書館で本を読みます。	としょかんでほんをよみます 
\\	私は今日学校に行きません。	わたしはきょうがっこうにいきません 
\\	十一時に寝ます。	じゅういちじにねます 
\\	十時四十二分に起きます。	じゅうじよんじゅうにふんにおきます 
\\	晩ごはんは?	ばんごはんは? 
\\	何時に昼ごはんを食べますか。	なんじにひるごはんをたべますか 
\\	三じに図書館で日本語を勉強します。	さんじにとしょかんでにほんごをべんきょうします 
\\	マクドナルドはあのデパトーの前ですよ。	マクドナルドはあのデパトのまえですよ 
\\	お寺で写真をたくさん撮りました。	おてらでしゃしんをたくさんとりました 
\\	メアリーさん、さっき電話がありましたよ。	メアリーさん、さっきでんわがありましたよ 
\\	あしたは日本語のクラスがありません。	あしたはにほんごのクラスがありません 
\\	レストランはデパトと病院の間です。	レストランはデパトとびょういんのあいだです 
\\	私はハーゲンダッツの前でメアリーさんを待ちました。	わたしはハーゲンダッツのまえでメアイリーさんをまちました 
\\	あなたの家に猫がいますか。	あなたのうちにねこがいますか 
\\	きれいな海。泳ぎましょう。	きれいなうみ。およぎましょう 
\\	七十円切手を二枚お願いします。それから、五十円切手を一枚ください。	ななじゅうえんきってをにまいおねがいします。それから、ごじゅうえんきってをいちまいください。 
\\	はがきありがとう。旅行は楽しかったですか。	はがきありがとう。りょこうはたのしかったですか 
\\	この部屋はちょっと暑いです。	このへやはちょっとあついです 
\\	ここに名前と電話番号をお願いします。	ここになまえとでんわばんごうをおねがいします 
\\	ロバートさん、起きてください。クラスで寝てはいけませんよ。	ロバートさん、おきてください。クラスでねてはいけませんよ 
\\	喫茶店でコーヒーを飲みましょうか。	きっさてんでコーヒーをのみましょうか 
\\	バスに乗って、会社に行きます。	バスにのって、かいしゃにいきます 
\\	週末には、十時ごら起きて、遅い朝ごはんを食べます。	しゅうまつには、じゅうじごろおきて、おそいあさごはんをたべます 
\\	きのう、遅く寝ました。	きのう、あそくねました。 
\\	映画の切符が二枚あります。今晩一緒に行きませんか。	えいがのきっぷがにまいあります。こんばんいっしょにいきませんか 
\\	リーさんは切手を三枚買いました。	リーさんはきってをさんまいかいました。 
\\	何日ぐらいかかりますか。	なんにちぐらいかかりますか 
\\	今日は家に帰って、勉強します。	きょはいえにかえって、べんきょうします 
\\	教科書のテープを聞いてください。	きょうかしょのテープをきいてください 
\\	教科書を見てもいいですか。	きょうかしょをみてもいいですか 
\\	ノートを借りて、コピーします。	ノートをかりて、コピーします 
\\	食堂に行って、昼ごはんを食べましょう。	しょくどうにいって、ひるごはんをたべましょう 
\\	このバスは市民病院へ行きますか。	このバスはしみんびょういんへいきますか 
\\	荷物を持ちましょうか。	にもつをもちましょうか 
\\	今晩勉強します。あしたテストがありますから。	こんばんべんきょうします。あしたテストがありますから。 
\\	バスに乗りましょう。タクシーは高いですから。	バスにのりましょう。タクシーはたかいですから 
\\	郵便局はどこですか。	ゆうびんきょくはどこですか 
\\	まっすぐ行って、三つ目の角を右に曲がってください。郵便局は道の右側にありますよ。	まっすぐいって、みっつめのかどをみぎにまがってください。ゆうびんきょくはみちのみぎがわにありますよう 
\\	一つ目の信号を右に曲がる。	ひとつめのしんごうをみぎにまがる 
\\	二つ目の角を左に曲がる。	ふたつめのかどひだりにまがる 
\\	これはスーさんの家族の写真ですか。	これはスーさんのかぞくのしゃしんですか 
\\	背が高くて、ハンサムですね。	せがたかくて、ハンサムですね 
\\	これはお姉さんですか。	これはおねえさんですか 
\\	姉は結婚していです。今ソウルに住んでいます。子供が一人います。三歳です。	あねはけっこんしています。いまソウルにすんでいます。こどもがひとりいます。さんさいです。 
\\	猫がいますね。でも、ちょっと太っていますね。	ねこがいますね。でも、ちょっとふとっていますね。 
\\	山下先生は結婚しています。	やましたせんせいはけっこんしています 
\\	みちこさんは窓のそばに座っています。	みちこさんはまどのそばにすわっています 
\\	スーさんはお金をたくさん持っています。	スーさんはおかねをたくさんもっています 
\\	山下先生は英語を知っています。	やましたせんせいはえいごをしっています 
\\	トムさんはちょっと太っています。	トムさんはちょっとふとっています 
\\	私の弟はとてもやせいています。	わたしのおとうとはとてもやせています 
\\	メアリーさんは丁シャツを着ています。	メアリーさんはテイシャツをきています 
\\	お父さんは起きています。	おとうさんはおきています 
\\	父と母は東京に住んでいます	ちちとはははとうきょうにすんでいます 
\\	私の姉は日本の会社に勤めています。	わたしのあねはにほんおかいしゃにつとめています 
\\	トムさんは髪が長いです。	トムさんはかみがながいです 
\\	あの店の食べ物は安くて、おいしいです。	あのみせのたべものはやすくて、おいしいです 
\\	あの人はいつも元気で、おもしろいです。	あのひとはいつもげんきで、おもしろいです 
\\	デパートにかばんを買いに行きました。	デパートにかばんをかいにいきました 
\\	メアリーさんは日本に日本語を勉強しに来ました。	メアリーさんはにほんにほんごをべんきょうしにきました 
\\	私のクラスにスウエーデン人の学生が一人います。	わたしのクラスにスウエーデンじんのがぐせいがひとりいます 
\\	テレビを見てもいいですか。	テレビをみてもいいですか 
\\	ゆっくり話してください。	ゆっくりはなしてください 
\\	コーヒーを飲んでください。	コーヒーをのんでください 
\\	写真を撮ってください。	しゃしんをとってください 
\\	窓を開けてもいいですか。	まどをあけてもいいですか 
\\	英語を話してもいいですか。	えいごをはなしてもいいですか 
\\	電話をかけてもいいですか。	でんわをかけてもいいですか 
\\	お風呂に入ってもいいですか。	おふろにはいってもいいですか 
\\	次のページを読んでください。	つぎのページをよんでください 
\\	昼ごはんを食べて、トイレに行きます。	ひるごはんをたべて、トイレにいきます 
\\	新聞を読んで、お茶を飲みます。	しんばんをよんで、おちゃをのみます 
\\	バスに乗ります。時間がありませんから。	バスにのります。じかんがありませんから 
\\	先週は大変でした。テストがありましたから。	せんしゅうはたいへんでした、テストがありましたから 
\\	あの映画を見ません。時間がありませんから。	あのえいがをみません。じかんがありませんから 
\\	あのレストランに行きました。食べ物は安いですから。	あのレストランに行きました。たべものはやすいですから 
\\	みちこが大好です。とてもおもしろくて、元気ですから。	みちこ画題すきです。とてもおもしろくて、げんきですから 
\\	私は朝ごはんを食べません。あまりお金がありませんから。	わたしはあさごはんをたべません。あまりおかねがありませんから 
\\	映画を見ません。お金がぜんぜんありませんから。	えいがをみません。おかねがぜんぜんあませんから 
\\	自転車を買います。電車はとても高いですから。	じてんしゃをかいます。でんしゃはとてもたかいですから 
\\	お金を借りてもいいですか。	おかねをかりてもいいですか 
\\	明日は友だちの誕生日です。	あしたはともだちのたんじょうびです 
\\	でも、私もお金がありません。あした、旅行に行きますから。	でも、わたしもおかねがありません。あした、りょこうにいきますから 
\\	あなたの車を借りてもいいですか。今晩デートがありますから。	あなたのくるまをかりてもいいですか。こんばんデートがありますから 
\\	子供の時、よく友だちと遊びました。	こどものとき、よくともだちとあそびました 
\\	先週の週末は東京に遊びに行きました。	せんしゅうのしゅまつはとうきょうにあそびにいきました 
\\	私の家に遊びに来てください。	わたしのうちにあそびにきてください 
\\	お姉さんは結婚していますか。	おねえさんはけっこんしていますか 
\\	お父さんは何歳ですか。	おとうさんはなんさいですか 
\\	弟さんはアメリカに住んでいますか。	おとうとさんはアメリカにすんでいますか 
\\	山田さんはやせていますか。	やまださんはやせていますか 
\\	水野さんは帽子をかぶっています。	みずのさんはぼうしをかぶっています 
\\	髪が短いです。ティーシャツを着ています。ジーンスをはいてません。	かみがみじかいです。ティーシャツをきています。ジーンスをはいてません 
\\	隣の人は背が高くて、親切でした。	となりのひとはせがたかくて、しんせつでした 
\\	私のルームメートは親切ですが、つまらないです。	わたしのルームメートはしんせつですが、つまらないです 
\\	京都に歌舞伎を見に行きます。	きょうとにかぶきをみにいきます 
\\	この部屋に女の人が何人いますか。	このへやにおんなのひとがなんにんいますか 
\\	すみませんが、この漢字の読み方を教えてください。	すみませんが、このかんじのよみかたをおしえてください 
\\	この部屋に元気な人が何人いますか。	このへやにげんきなひとがなんにんいますか 
\\	上手ですね。ロバートさんは料理するのが好きですか。	じょうずですね。ロバートさんはりょうりするのがすきですか 
\\	スーさんは、あした試験があると言っていました。	スーさんは、あしたしけんがあるといっていました 
\\	たけしさんは、メアリーさんが好きだと思います。	たけしさんはメアリーさんがすきだとおもいます 
\\	ここで写真を撮らないでください。	ここでしゃしんをとらないでください 
\\	日本語を勉強するのが好きです。	にほんごをべんきょうするのがすきです 
\\	部屋を掃除するのが嫌いです	へやをそうじするのがきらいです 
\\	ロバートさんは料理を作るのが上手です。	ロバートさんはりょうりをつくるじょうずです 
\\	今日、勉強する?	きょう、べんきょうする? 
\\	よく、お茶を飲む?	よく、おちゃをのむ? 
\\	毎日、日本語を話す?	まいにち、にほんごをはなす? 
\\	今日、友だちに会う?	きょう、ともだちにあう? 
\\	よく、電車に乗る?	よく、でんしゃにのる? 
\\	あした、大学に来る?	あした、だいがくにくる? 
\\	自転車を持っている?	じてんしゃをもっている? 
\\	毎週、部屋を掃除する?	まいしゅう、へやをそうじする? 
\\	メアリーさんは何と言ってましたか。	メアリーさんはなんといってましたか 
\\	今月は忙しいと言っていました。	こんげつはいそがしいといっていました 
\\	メアリーさんは明日買い物をすると言っていました。	メアリさんはあしたかいものをするといっていました 
\\	お弁当を二つください。	おべんとうをふたつください 
\\	薬を飲む。	くすりをのむ 
\\	髪が長い人。	かみがながいひと 
\\	眼鏡をかけている人。	めがねをかけているひと 
\\	猫が好きな人。	ねこがすきなひと 
\\	あそこで写真を撮っている人はだれですか。	あそこでしゃしんをとっているひとはだれですか 
\\	毎日運動をする人は元気です。	まいにちうんどうをするひとはげんきです 
\\	たばこを吸わない人が好きです。	たばこをすわないひとがすきです 
\\	去年結婚した友だちから手紙がきました。	きょねんけっこんしたともだちからてがみがきました 
\\	きのう宿題をしました。	きのうしゅくだいをしました 
\\	もう宿題をしました。	もうしゅくだいをしました 
\\	きのう宿題をしませんでした。	きのうしゅくだいをしませんでした 
\\	まだ宿題をしていません。	まだしゅくだいをしていません 
\\	スーさんはまだ起きていません。	スーさんはまだおきていません 
\\	まだ昼ごはんを食べていません。	まだひるごはんをたべていません 
\\	朝ごはんを食べませんでした。忙しかったですから。	あさごはんをたべませんでした。いそがしかったですから 
\\	明日試験あるから、私は今晩勉強します。	あしたしけんがあるから、わたしはこんばんべんきょうします 
\\	寒かったから、出かけませんでした。	さむかったから、でかけませんでした 
\\	歌舞伎の切符がありますから、一緒に見に行きましょう。	かぶきのきっぷがありますから、いっしょにみにいきましょう 
\\	赤が一番好きです。	あかがいちばんすきです 
\\	顔が青いですね。	かおがあおいですね 
\\	白黒の写真。	しろくろのしゃしん 
\\	メアリーさんは金髪です。	メアリーさんはきんぱつです 
\\	去年の誕生日に何がもらいましたか。	きょねんのたんじょうびになにがもらいましたか 
\\	ピアノを弾きますか。	ピアノをひきますか 
\\	今日、クラスは何時に始まりましたか。何時に終わりますか。	こう、クラスはなんじにはじまりましたか。なんじにおわりますか 
\\	再来週。	さらいしゅう 
\\	リックとキャロルは二ヶ月前に別れた。	リックとキャロルはにかげつまえにわかれた 
\\	寒くなりましたね。	さむくなりましたね 
\\	冬休みはどうしますか。	ふゆやすみはどうしますか 
\\	韓国か台湾に行くつもりですが、まだ決めていません。	かんこくかたいわんにいくつもりですが、まだきめていません 
\\	台湾のほうが暖かいと思います。でも、スーさんは韓国の食べ物はおいしいと言っていましたよ。	たいわんのほうがあたたかいとおもいます。でも、スーさんはかんこくのたべものはおいしいといっていましたよ 
\\	大阪からソウルまで飛行機の予約をお願いします。	おおさかからソウルまでひこうきのよやくをおねがいします 
\\	午前と午後の便があります。	ごぜんとごごのびんがあります 
\\	クレジットカードで払ってもいいですか。	クレジットカードではらってもいいです 
\\	ソウルまでどのぐらいかかりますか。	ソウルまでどのぐらいかかりますか 
\\	エルビスプレスリーのほうがフランクシナトラよりかっこいいです。	エルビスプレスリーのほうがフランクシナトラよりかっこいいです 
\\	バスと電車とどっちのほうが安いですか。	バスとでんしゃとどっちのほうがやすいですか 
\\	パバトッテイとカレーラスとドミンゴの中で、だれがいちばん歌が上手だと思いますか。	パバロッチとカレーラスとドミンゴのなかで、だれがいちばんうたがじょうずだとおもいますか 
\\	もちろん、パバロッテイが一番歌が上手です。	もちろん、パバロッテイがいちばんうたがじょうずです 
\\	黒いセーターを持っています。赤いのも持っています。	くろいセーターをもっています。あかいのももっています 
\\	安い辞書を買いに行きました。でもいいのがありませんでした。	やすいじしょをかいにいきました。でもいいにがありませんでした 
\\	週末にたけしさんとテニスをするつもりです。	しゅまつにたけしさんとテニスをするつもりです 
\\	山下先生はあした大学に来ないつもりです。	やましたせんせいはあしただいがくにこないつもりです 
\\	お寺を見に行くつもりでしたけど、天気がよくなかったから、行きませんでした。	おてらをみにいくつもりでしたけど、てんきがよくなかったから、いきませんでした 
\\	久しぶりですね。休みはどうでしたか。	ひさしぶりですね。やすみはどうでしたか 
\\	一日だけドライブに行きましたが、毎日アルバイトをしていました。	いちにちだけドライブにいきましたが、まいにちアルバイトをしていました 
\\	出身はどこですか。	しゅっしんはどこですか 
\\	山や海があって、きれいな所ですよ。	やまやうみがあって、きれいなところですよ 
\\	まもなく発車します。	まもなくはっしゃします 
\\	ドアが閉まります。ご注意ください。	ドアがしまります。ごちゅういください 
\\	次は新宿に止まります。	つぎはしんじゅくにとまります 
\\	電車が参ります。	でんしゃがまいります 
\\	終電は何時ですか。	しゅうでんはなんじですか 
\\	東京まで指定席の一枚お願いします。	とうきょうまでしていせきのいちまいおねがいします 
\\	静かになる。	しずかになる 
\\	会社員になる。	かいしゃいんになる 
\\	日本語の勉強が楽しくなりました。	にほんごのべんきょうがたのしくなりました 
\\	日本語の勉強が好きになりました。	にほんごのべんきょうがすきになりました 
\\	メアリーさんは前より日本語が上手になりました。	メアリーさんはまえよりにほんごがじょうずになりました 
\\	どこかへ行きましたか。	どこかへいきましたか 
\\	いいえ、どこへも行きませんでした。	いいえ、どこへもいきませんでした 
\\	だれかに会いましたか。	だれかにあいましたか 
\\	いいえ、だれにも会いませんでした。	いいえ、だれにもあいませんでした 
\\	何かしましたか。	なにかしましたか 
\\	いいえ、何もしませんでした。	いいえ、なにもしませんでした 
\\	はしでごはんを食べます。	はしでごはんをたべます 
\\	バスで駅まで行きました。	バスでえきまでいきました 
\\	テレビで映画を見ました。	テレビでえいがをみました 
\\	今度の週末は映画を見たいです。	こんどのしゅまつえいがをみたいです 
\\	いつか中国に行きたいです。	いつかちゅうごくにいきたいです 
\\	同じです。	おなじです 
\\	だいたい同じです。	だいたいおなじです 
\\	ちょっと違います。	ちょっとちがいます 
\\	使えません。	つかえません 
\\	間違ってます。	まちがってます 
\\	手を上げてください。	てをあげてください 
\\	宿題を出してください。	しゅくだいをだしてください 
\\	教科書を閉じてください。	きょうかしょをとじてください 
\\	となりの人に聞いてください。	となりのひとにきいてください 
\\	今日はこれで終わります。	きょうはこれでおわります 
\\	あの人にはもう会いたくありません。	あのひとにはもうあいたくありません 
\\	セーターが買いたかったから、デパートに行きました。	セーターがかいたかったから、デパートにいきました 
\\	メアリーさんはトイレに行きたいと言っていました。	メアリーさんはトイレにいきたいといっていました 
\\	ちょっとおなかが痛いんです。	ちょっとおなかがいたいんです 
\\	どうしたんですか。	どうしたんですか 
\\	きのう友だちと晩ごはんを食べに行ったんです。たぶん食べすぎたんだと思います。	きのうともだちとばんごはんをたべにいったんです。たぶんたべすぎたんだとおもいます 
\\	大丈夫ですか。	だいじょうぶですか 
\\	心配しないでください。	しんぱいしないでください 
\\	病院に行ったほうがいいですよ。	びょういんにいたほうがいいですよ 
\\	のどが痛いんです。きのうはおなかが痛かったんです。	のどがいたいんです。きのうはおなかがいたかったんです 
\\	熱もありますね。風邪ですね。	ねつもありますね。かぜですね 
\\	二三日、運動しないほうがいいでしょう。	にさんにち、うんどうしないほうがいいでしょう 
\\	今日は薬を飲んで、早く寝てください。	きょうはくすりをのんで、はやくねてください 
\\	お大事に。	おだいじに 
\\	バスが来なかったんです。	バスがこなかったんです 
\\	あしたテストがあるんです。	あしたテストがあるんです 
\\	トイレに行きたいんです。	トイレにいきたいんです 
\\	試験が終わったんです。	しけんがおわったんです 
\\	どうして彼と別れたんですか。	どうしてかれとわかれたんですか 
\\	彼、ぜんぜんお風呂に入らないんです。	かれ、ぜんぜんおふろにはいらないんです 
\\	猫がしんだんです。	ねこがしんだんです 
\\	とてもいい教科書ですね。	とてもいいきょうかしょですね 
\\	私の大学の先生が書いたんです。	わたしのだいがくのせんせいがかいたんです 
\\	食べすぎてはいけません。	たべすぎてはいけません 
\\	早く起きすぎました。	はやくおきすぎました 
\\	この本は高すぎます。	このほんはたかすぎます 
\\	あの人は親切すぎます。	あのひとはしんせつすぎます 
\\	もっと野菜を食べたほうがいいですよ。	もっとやさいをたべたほうがいいですよ 
\\	授業を休まないほうがいいですよ。	じゅぎょうをやすまないほうがいいですよ 
\\	いつも日本語で話すので、日本語が上手になりました。	いつもにほんごではなすので、にほんごがじょうずになりました 
\\	宿題がたくさんあったので、きのうの夜、寝ませんでした	しゅくだいがたくさんあったので、きのうのよる、ねませんでした 
\\	その人は意地悪なので、嫌いです。	そのひとはいじわるなので、きらいです 
\\	今日は日曜日なので、銀行は休みです。	きょうはにちようびなので、ぎんこうはやすみです 
\\	来週テストがありますから、たくさん勉強しなくちゃいけません。	らいしゅうテストがありますから、たくさんべんきょうしなくちゃいけません 
\\	けさは、六時に起きなくちゃいけませんでした。	けさは、ろくじにおきなくちゃいけませんでした 
\\	毎日、練習しなくちゃいけないんです。	まいにち、れんしゅうしなくちゃいけないんです 
\\	あしたは雨が降るでしょう。	あしたはあめがふるでしょう 
\\	明日は雨が降らないでしょう。	あしたはあめがふらないでしょう 
\\	北海道は寒いでしょう。	ほっかいどうはさむいでしょう 
\\	山下先生は魚が好きでしょう。	やましたせんせいはさかながすきでしょう 
\\	スザンさんは魚が好きじゃないでしょう。	スザンさんはさかながすきじゃないでしょう 
\\	あの人はオーストラリア人でしょう。	あの人はオーストラリア人でしょう 
\\	日本語と韓国語と、どっちのほうが難しいでしょうか。	にほんごとかんこくごと、どっちのほうがむずかしいでしょうか 
\\	たけしさんは興味があるだろうと思います。	たけしさんはきょうみがあるだろうとおもいます 
\\	ジョン、中国語がわかるでしょ? これ、読んで。	ジョン、ちゅごくごがわかるでしょ? これ、よんで 
\\	すみません、初めてなんですが。	すみません、はじめてなんですが 
\\	保険証を見せてください。	ほけんしょうをみせてください 
\\	この紙に名前と住所を書いてください。	このかみになまえとじゅうじょをかいてください 
\\	これは何の薬ですか。	これはなんのくすりですか 
\\	痛み止めです。食後に飲んでください。	いたみどめです。しょくごにのんでください 
\\	駅で新聞を買います。	えきでしんぶんをかいます 
\\	山田さんは木村さんに花をあげました。	やまださんはきむらさんにはなをあげました 
\\	イーさんに本を貸しました	イーさんにほんをかしました 
\\	太郎君に英語教えます。	たろうさんにえいごをおしえます 
\\	会社へ電話をかけます。	かいしゃへでんわをかけます 
\\	木村さんは山田さんに花をもらいます。	きむらさんはやまださんにはなをもらいます 
\\	ワンさんに中国語を習います。	ワンさんにちゅごくごをならいます 
\\	銀行からお金を借りました。	ぎんこうからおかねをかりました 
\\	もう荷物を送りましたか。	もうにもつをおくりましたか 
\\	琵琶湖の水はきれいですか。	びわこのみずはきれいですか 
\\	これはとても有名な映画です。	これはとてもゆうめいなえいがです 
\\	さくら大学はあまり有名な大学じゃありません。	さくらだいがくはあまりゆうめいなだいがくじゃありません 
\\	奈良はどんな町ですか。	ならはどんなまちですか 
\\	日本の食べ物はおいしいですが、高いです。	にほんおたべものはおいしいですが、たかいです 
\\	ミラーさんの傘はどれですか。	ミラーさんのかさですか 
\\	お金が全然ありません。	おかねがぜんぜんありません 
\\	受付にだれがいますか。	うけつけにだれがいますか 
\\	机の上に写真があります。	つくえのうえにしゃしんがあります 
\\	箱の中に手紙や写真があります。	はこのなかにてがみやしゃしんがあります 
\\	りんごを四つ買いました。	りんごをよっつかいました 
\\	外国人の学生が二人います。	がいこくじんのがくせいがふたりいます 
\\	みかんをいくつ買いましたか。	みかんをいくつかいましたか 
\\	この会社に外国人が何人いますか。	このかいしゃにがいこくじんがなんにんいますか 
\\	毎晩、何時間日本語を勉強しますか。	まいばん、なんじかんにほんごをべんきょうしますか 
\\	大阪から東京までどのぐらい掛かりますか。	おおさかからとうきょうまでどのぐらいかかりますか 
\\	新幹線で二時間半掛かります。	しんかんせんでにじかんはんかかります 
\\	学校に先生が三十人ぐらいいます。	がっこうにせんせいがさんじゅうにんぐらいいます 
\\	パワー電気に外国人の会社員が一人だけいます。	パワーでんき苦いこくじんおかいしゃいんがひとりだけいます 
\\	休みは日曜日だけです。	やすみはにちようびだけです 
\\	きのうの試験は簡単じゃありません。	きのうのしけんはかんたんじゃありません 
\\	きのうは暑かったです。	きのうはあつかったです 
\\	きのうのパーティはあまり楽しくなかったです。	きのうのパーティはあまりたのしくなかったです 
\\	サッカーと野球とどちらがおもしろいですか。	サッカーとやきゅうとどちらがおもしろいですか 
\\	ミラーさんとサントスとどちらがテニスが上手ですか。	ミラーさんとサントスさんがテニスがじょうずですか 
\\	北海道と大阪とどちらが涼しいですか。	ほっかいどうとおおさかとどちらがすずしいですか 
\\	春と秋とどちらが好きですか。	はるとあきどちらがすきですか 
\\	日本料理の中で何が一番おいしいですか。	にほんりょうりのなかでなにがいちばんおいしいですか 
\\	ユーロッパでどこが一番よかったですか。	ユーロッパでどこがいちばんよかったですか 
\\	家族でだれが一番背が高いですか。	かぞくでだれがいちばんせがたかいですか 
\\	弟が一番背が短いです。	おとうとがいちばんせがみじかいです 
\\	今何が一番欲しいですか。	いまなにがいちばんほしいですか 
\\	てんぷらを食べたいです。	てんぷらをたべたいです 
\\	靴を買いたいです。	くつをかいたいです 
\\	おなかが痛いですから、何も食べたくないです。	おなかがいたいですから、なにもたべたくないです 
\\	神戸へインド料理を食べに行きます。	こうべへインドりょうりをたべにいきます 
\\	神戸へ買い物に行きます。	こうべへかいものにいきます 
\\	日本へ美術の勉強に来ました。	にほんへびじゅつのべんきょうにきました 
\\	あした京都のお祭りに行きます。	あしたきょうとのおまつりにいきます 
\\	あの喫茶店に入りましょう。	あのきっさてんにはいりましょう 
\\	七時に家を出ます。	しちじにうちをでます 
\\	のどがかわきましたから、何かを飲みたいです。	のどがかわきましたから、なにかをのみたいです 
\\	ご注文は?	ごちゅうもんは? 
\\	ミラーさんは今電話をかけています。	ミラーさんはいまでんわをかけています 
\\	今雨が降っていますか。	いまあめがふっていますか 
\\	傘を貸しましょうか。	かさをかしましょうか 
\\	すみませんが、塩を取ってください。	すみませんが、しおをとってください 
\\	たばこを吸ってもいいですか。	たばこをすってもいいですか 
\\	カメラを持っています。	カメラをもっています 
\\	はコンピューターソフトを作っています。	
\\	はコンピューターソフトをつくっています 
\\	妹は大学で勉強しています。	いもうとはだいがくでべんきょうしています 
\\	市役所の電話番号を知っていますか。	しやくしょうのでんわばんごうをしっていますか 
\\	いいえ、電話番号は知りません。	いいえ、でんわばんごうはしりません 
\\	朝ジョギングをして、シャワーを浴びて、会社へ行きます。	あさジョギングをして、シャワーをあびて、かいしゃへいきます 
\\	神戸に行って、映画を見て、お茶を飲みました。	こうべにいって、えいがをみて、おちゃをのみました 
\\	ミラーさんは若くて、元気です。	ミラーさんはわかくて、げんきです 
\\	きのうは天気がよくて、暑かったです。	きのうはてんきがこくて、あつかったです 
\\	ミラーさんはハンサムで、親切です。	ミラーさんはハンサムで、しんせつです 
\\	カリナさんは学生で、マリアさんは主婦です。	カリナさんはがくせいで、マリアはしゅふです 
\\	この部屋は狭いですが、きれいです。	このへやはせまいですが、きれいです 
\\	コンサートが終わってから、レストランで食事しました。	コンサートがおわってから、レストランでしょくじしました 
\\	大阪は食べ物がおいしいです。	おおさかはたべものがおいしいです 
\\	サントスさんはどの人ですか。 あの背が高くて、髪が黒い人です。	さんとすさんはどのひとですか あのせががたかくて、かみがくろいひとです 
\\	信号を右へ曲がってください。	しんごうをみぎへまがってください 
\\	公園を散歩します。	こうえんをしんぽします 
\\	お引き出しですか。	おひきだしですか 
\\	富士山に登りたいです。	ふじさんにのぼりたいです 
\\	今晩、暇ですか。	こんばん、ひまですか 
\\	そこにはさみがある。	
\\	ミラーさんはこのニュースを知っていますか。 いいえ、多分知らないと思います。	ミラーさんはニュースをしっていますか。 いいえ、たぶんしらないとおもいます 
\\	日本の物価が高いと思います。	にほんのぶっかがたかいとおもいます 
\\	新しい空港についてどう思いますか。 きれいですが、ちょっと交通が不便がと思います。	あたらしいくうこうについてどうおもいますか。 きれいですが、ちょっとこうつうがふべんとおもいます 
\\	ファクスは便利ですね。	ファクスはべんりですね 
\\	寝る前に「お休みなさい」と言います。	ねるまえに「おやすみなさい」といいます 
\\	ミラーさんは「来週東京へ出張します」と言いました。	"ミラーさんは「らいしゅうとうきょうへしゅっちょうします」といいました 
\\	ミラーさんは来週東京へ出張すると言いました。	ミラーさんはらいしゅとうきょうへしゅっちょうするといいました 
\\	あしたパーティーに行くでしょう。	あしたパーティーにいくでしょう 
\\	北海道は寒かったでしょう。 いいえ、そんないに寒くなかったです。	ほっかいどはさむかったでしょう いいえ、そんないにさむくなかったです 
\\	東京で日本とブラジルのサッカーの試合があります。	とうきょうでにほんとブラジルのサッカーのしあいがあります 
\\	会議で何か意見を言いましたか。	かいぎでなにかいけんをいいましたか 
\\	ちょっとビールでも飲みませんか。	ちょっとビールでものみませんか 
\\	ミラーさんが住んでいた家に猫がいました。	ミラーさんがすんでいたうちにねこがいました 
\\	これはミラーさんが作ったケーキです。	これはミラーさんがつくったケーキです 
\\	私はカリナさんがかいた絵が好きです。	わたしわカリナさんがかいたえがすきです 
\\	私は朝ごはんを食べる時間がありません。	わたしはあさごはんをたべるじかんがありません 
\\	私は友達と映画を見る約束があります。	わたしはともだちとえいがをみるやくそくがあります 
\\	今日は市役所へ行く用事があります。	きょうはしやくしょへいくようじがあります 
\\	図書館で本を借りるとき、カードが要ります。	としょかんでほんをかりるとき、カードがいります 
\\	使い方が、わからないとき、わたしに聞いてください。	つかいかたが、わからないとき、わたしにきいてください 
\\	体の調子が悪いとき「元気茶」を飲みます。	"からだのちょうしがわるいとき、「げんきちゃ」をのみます 
\\	暇なとき、家へ遊びに来ませんか。	ひまなとき、うちへあそびにきませんか 
\\	妻が病気のとき、会社を休みます。	つまがびょうきのとき、かいしゃをやすみます 
\\	若いとき、あまり勉強しませんでした。	わかいとき、あまりべんきょうしませんでした 
\\	子どものとき、よく川で泳ぎました。	こどものとき、よくかわでおよぎました 
\\	国へ帰るとき、かばんを買いました。	くにへかえるとき、かばんをかいました 
\\	このボタンを押すと、お釣りが出ます。	このボタンをおすと、おつりがでます 
\\	これを回すと、音が大きくなります。	これをまわすと、おとがおおきくなります 
\\	右へ曲がると、郵便局があります。	みぎへまがると、ゆうびんきょくがあります 
\\	音が小さいです。	おとがちいさいです 
\\	電気が明るくなりました。	でんきがあかるくなりました 
\\	道を渡ります。	みちをわたります 
\\	交差点を左へ曲がります。	こうさてんをひだりへまがります 
\\	食事の前に、手を洗います。	しょくじのまえに、てをあらいます 
\\	田中さんは一時間前に、出かけました。	たなかさんはいちじかんまえに、でかけました 
\\	日本ではなかなか馬を見ることができません。	にほんではなかなかうまをみることができません 
\\	ここに荷物を置かないでください。	こにににもつをおかないでください 
\\	荷物はここに置かないでください。	にもつはここにおかないでください 
\\	会社の食堂で昼ごはんを食べました。	かいしゃのしょくどうでひるごはんをたべました 
\\	佐藤さんは妹にお菓子をくれました。	さとうさんはいもうとにおかしをくれました 
\\	タクシーを呼びましょうか。	タクシーをよびましょうか 
\\	手伝いましょうか。	てつだいましょうか 
\\	私は山田さんに図書館の電話番号を教えてもらいました。	わたしはやまださんにとしょかんのでんわばんごうをおしえてもらいました 
\\	だれが手伝いに行きますか。	だれがてつだいにいきますか 
\\	お金があったら、旅行します。	おかねがあったら、りょこうします 
\\	時間がなかったら、テレビを見ません。	じかんがなかったら、テレビをみません 
\\	安かったら、パソコンを買いたいです。	やすかったら、パソコンをかいたいです 
\\	暇だったら、手伝ってください。	ひまだったら、てつだってください 
\\	いい天気だったら、散歩しませんか。	いいてんきだったら、さんぽしませんか 
\\	十時になったら、出かけましょう。	じゅうじになったら、でかけましょう 
\\	家へ帰ったら、すぐシャワーを浴びます。	うちへかえったら、すぐシャワーをあびます 
\\	雨が降っても、洗濯します。	あめがふっても、せんたくします 
\\	安くても、私はグループ旅行が嫌いです。	やすくても、わたしはグループりょこうがきらいです 
\\	便利でも、パソコンを使いません。	べんりでも、パソコンをつかいません 
\\	日曜日でも、働きます。	にちようびべも、はたらきます 
\\	もし一億円あったら、いろいろな国を旅行したいです。	もしいちおくえんあったら、いろいろなくにをりょこうしたいです 
\\	幾ら考えても、わかりません。	いくらかんがえても、わかりません 
\\	幾ら高くても、買います。	いくらたかくても、かいます 
\\	友達が来る前に、部屋を掃除します。	ともだちがくろまえに、へやをそうじします 
\\	友達が約束の時間に来なかったら、どうしますか。	ともだちがやくそくのじかんにこなかったら、どうしますか 
\\	渡辺さんは時々大阪弁を使いますね。大阪に住んでいたんですか。	わたなべさんはときどきおおさかべんをつかいますね。おおさかにすんでいたんですか 
\\	十五歳まで大阪に住んでいました。	じゅうごさいまでおおさかにすんでいました 
\\	おもしろいデザインの靴ですね。どこで買ったんですか。	おもしろいデザインのくつですね。どこでかったんですか 
\\	どうして遅れたんですか。	どうしておくれたんですか 
\\	毎朝、新聞を読みますか。	まいあさ、しんぶんをよみますか 
\\	日本語で手紙を書いたんですが、ちょっと見ていただきませんか。	にほんごでてがみをかいたんですが、ちょっとみていただきませんか 
\\	お湯が、出ないんですが。	おゆが、でないんですが 
\\	いい先生を紹介していただきませんか。	いいせんせいをしょうかいしていただきませんか 
\\	どこでカメラを買ったらいいですか。	どこでカメラをかったらいいですか 
\\	細かいお金がないんですが、どうしたらいいですか。	こまかいおかねがないんですが、どうしたらいいですか 
\\	困ったなあ。	こまったなあ 
\\	運動会に参加しますか いいえ。スポーツはあまり好きじゃないんです。	うんどうかいにさんかしますか いいえ。スポーツはすきじゃないんです 
\\	趣味は異なる。	しゅみはことなる 
\\	この銀行でドルが換えられます。	このぎんこうでドルがかえられます 
\\	日本語が話せます。	にほんごがはなせます 
\\	新宿で今黒沢の映画が見られます。	しんじゅくでくろさわのえいががみられます 
\\	田中さんに会えませんでした。	たなかさんにあえませんでした 
\\	新幹線で富士山が見えます。	しんかんせんでふじさんがみえます 
\\	電話に天気予報が聞けます。	でんわにてんきよほうがきけます 
\\	駅の前に大きいスーパーができました。	えきのまえにおおきスーパーができました 
\\	時計の修理はいつできますか。	とけいのしゅうりはいつできますか 
\\	橋は修理中です。	はしはしゅうりちゅうです 
\\	私の学校では中国語が習えます。	わたしのがっこうではちゅうごくごがならえます 
\\	昨日は山が見えましたが、今日は見えません。	きのうはやまがみえましたが、きょうはみえません 
\\	ワインは飲みますが、ビールは飲みません。	ワインはのみますが、ビールはのみません 
\\	京都へは行きますが、大阪へは行きません。	きょうとへはいきますが、おおさかへはいきません 
\\	クララさんは英語が話せます。フランス語も話せます。	クララさんはえいごごはなせます。 フランスごもはなせます 
\\	私の部屋から海が見えます。弟の部屋からも見えます。	わたしのへやからうみがみえます。おとうとのへやからもみえます 
\\	ローマ字しか書けません。	ローマじしかかけません 
\\	ローマ字だけ書けます。	ローマじだけかけます 
\\	音楽を聞きながら、食事します。	おんがくをききながら、しょくじします 
\\	女の髪は長い;舌はもっと長い。	かのじょかみはながい;ちはもっとながい 
\\	最近、寝不足だ。	さいきん、ねぶそくだ 
\\	雪でなければ、父は帰宅します。	ゆきでなければ、ちちはきたくします 
\\	ピアノを弾くことができますか。	ピアノをひくことができますか 
\\	危ないですから、押さないでください。	あぶないですから、おさないでください 
\\	傘を忘れないでください。	かさをわすれないでください 
\\	体にいいですか。	からだにいいですか 
\\	去年北海道で馬に乗りました。	きょねんほっかいどうでうまにのりました 
\\	元気になります。	げんきになります 
\\	日毎にだんだん寒くなっています。	ひごとにだんだんさむくなっています 
\\	二十五歳になります。	にじゅうごさいになります 
\\	暑くなりましたね。	あつくなりましたね 
\\	ここに塵箱を置かないでください。	ここにゴミばこをおかないでください 
\\	ここに塵を捨てないでください。	ここにゴミをすてないでください 
\\	足を踏まないでください。	あしでふまないでください 
\\	笑わないでください。	わらわないでください 
\\	怒らないでください。	おこらないでください 
\\	泣かないでください。	なかないでください 
\\	晩御飯を作ります。	ばんごはんをつくります 
\\	今お金を払わなくてもいいです。	いまおかねをはらわなくてもいいです 
\\	今日は子供の誕生日ですから、早く帰ります。	きょうはこどものたんじょうびですからはやくかえります 
\\	カリナさんは絵が上手ですか。	カリナさんはえがじょうずですか 
\\	全然わかりません。	ぜんぜんわかりません 
\\	朝に時間がありませんから、新聞を読みません。	あさにじかんがありませんから、しんぶんをよみません 
\\	どうしてきのう早く帰りましたか。 用事がありましたから。	どうしてきのうはやくかえりましたか ようじがありましたから 
\\	カラオケが好きですが、あまり行きません。 歌が下手ですから。	カラオケがすきですが、あまりいきません うたがへたですから 
\\	残念ですね。	ざんねんですね 
\\	週末はテレサちゃんの学校とサントスさんの会社は休みです。	しゅうまつはテレサちゃんのがっこうとサントスさんのかいしゃはやすみです 
\\	猫は何もしません。	ねこはなにもしません 
\\	わたしは朝から晩まで忙しいです。	わたしはあさからばんまでいそがしいです 
\\	休みが全然ありません。	やすみがぜんぜんありません 
\\	もう少し大きいのはありませんか。	もうすこしおおきいのはありませんか 
\\	窓を開けましょうか。	まどをあけましょうか 
\\	駅まで迎えに行きましょうか。	えきまでむかえにいきましょうか 
\\	佐藤さんは今会議室で松本さんと話しています。	さとうさんはいまかいぎしつでまつもとさんとはなしています 
\\	カリナさんは花をかいています。	カリナさんははなをかいています 
\\	立たないでください。	たたないでください 
\\	砂糖を取りましょうか。	さとうをとりましょうか 
\\	今、三菱で働いています。	いまみつびしではたらいています 
\\	想像してください。	そうぞうしてください 
\\	いろいろな国の人がいます。	いろいろなくにのひとがいます 
\\	後ろに立ってください。	うしろにたってください 
\\	昔はよくした。	むかしはよくした 
\\	グラスゴーの人口は二百万人です。	グラスゴーのじんこうはにひゃくまんにんです 
\\	用事がありますか。	ようじがありますか 
\\	泳いではいけません。	およいではいけません 
\\	私は英語を喋ることができる。	わたしはえいごをしゃべることができる 
\\	趣味は映画を見ることです	しゅみはえいがをみることです 
\\	音楽の趣味がいいね。	おんがくのしゅみがいいね 
\\	寝る前に、日記を書きます。	ねるまえに、にっきをかきます 
\\	パソコンを使うことができますか。	パソコンをつかうことができますか 
\\	カードで払うことができますか。	カードではらうことができますか 
\\	趣味は何ですか。 古い時計を集めることです。	しゅみはなんですか ふるいとけいをあつめることです 
\\	日本の子供は学校に入る前に、漢字を覚えなければなりませんか。 いいえ覚えなくてもいいです。	にほんのこどもはがっこうにはいるまえに、かんじをおぼえなければなりませんか いいえおぼえなくてもいいです 
\\	三年前に、結婚しました。	さんねんまえに、けっこんしました 
\\	特に馬が好きです。	とくにうまがすきです 
\\	日本では中々馬を見ることができません。	にほんではなかなかうまをみることができません 
\\	パソコンを欲しいです。	パソコンをほしいです 
\\	子供と神戸へ船を見に行きます。	こどもとこうべへふねをみにいきます 
\\	冬休みはどこか行きましたか。	ふゆやすみはどこかいきましたか 
\\	別々にお願いします。	べつべつにおねがいします 
\\	私は外国で働きたいです。	わたしはがいこくではたらきたいです 
\\	私は結婚したくないです。	わたしはけっこんしたくないです 
\\	小さいカメラが欲しいです。	ちいさいカメラがほしいです 
\\	黒い靴が欲しいです。	くろいくつがほしいです 
\\	次の交差点を右に曲がりなさい。	つぎのこうさてんをみぎにまがりなさい 
\\	冷蔵庫の中にいろいろな物があります。	れいぞうこのなかにいろいろなものがあります 
\\	彼女はいい奥さんになるだろう。	かのじょはいいおくさんになるだろう 
\\	もっとたくさん欲しい。	もっとたくさんほしい 
\\	もっと静かにして下さい。	もっとしずかにしてください 
\\	もっとゆっくり歩きなさい。	もっとゆっくりあるきなさい 
\\	京都は秋がもっとも美しい。	きょうとはあきがもっともうつくしい 
\\	仕事は大変ですか。	しごとはたいへんですか 
\\	彼は大変眠いらしい。	かれはたいへんねむいらしい 
\\	雨になるらしいよ。	あめになるらしいよ 
\\	本通りは大変広い。	ほんどおりはたいへんひろい 
\\	神田さんは大変速く走る。	かんださんはたいへんはやくはしる 
\\	庭にある花は大変美しい。	にわにあるはなはたいへんうつくしい 
\\	黒い服の女を見た。	くろいふくのおんなをみた 
\\	何かお捜しですか。	なにかおさがしですか 
\\	セーターを捜しています。	セーターをさがしています 
\\	試着して見たいのですが。	しちゃくしてみたいのですが 
\\	今夜はよく混んでいますね。	こんやはよくこんでいますね 
\\	軽い風邪です。	かるいかぜです 
\\	私は軽い昼食をとった。	わたしはかるいちゅうしょくをとった 
\\	彼に家を売るつもりですか。	かれにうちをうるつもりですか 
\\	彼はその車を売る決心をした。	かれはそのくるまをうるけっしんをした 
\\	ジェットコースターに乗りました。	ジェットコースターにのりました 
\\	川崎には工場が多い。	かわさきにはこうじょうがおおい 
\\	彼は工場で働いている。	かれはこうじょうではたらいている 
\\	サッカーは野球より人気です。	サッカーはやきゅうよりにんきです 
\\	その人気歌手は自殺した。	そのにんきかしゅはじさつした 
\\	朝食を八時にとる。	ちょうしょくをはちじにとる 
\\	リンゴの種を取る。	りんごのたねをとる 
\\	ボタンを押してください。	ボタンをおしてください 
\\	このパソコンは軽くて、便利です。	このパソコンはかるくて、べんりです 
\\	図書館へ行って、本を借りて、それから友達に会いました。	としょかんへいって、ほんをかりて、それからともだちにあいました 
\\	国へ帰ってから、父の会社で働きます。	くにへかえってから、ちちのかいしゃではたらきます 
\\	私は歴史に興味がある。	わたしはれきしにきょうみはある 
\\	政治に興味がありますか。	せいじにきょうみがありますか 
\\	起きるには早過ぎる。	おきるにははやすぎる 
\\	もっと運動をしないと太るよ。	もっとうんどうをしないとふとるよ 
\\	外は暗い。	そとはくらい 
\\	新しい上着を着ている。	あたらしうわぎをきている 
\\	次の駅で降りるつもりだ。	つぎのえきでおりるつもりだ 
\\	この薄い本は私のです。	このうすいほんはわたしのです 
\\	赤鉛筆を持っていますか。	あかえんぴつをもっていますか 
\\	私はドイツで生まれた。	わたしはドイツでうまれた 
\\	何時に駅に着くの。	なんじにえきにつくの 
\\	彼の声は大きい。	かれのこえはおおきい 
\\	そんな大声で話すな!	そんなおおごえではなすな 
\\	父はいつも大声で話す。	ちちはいつもおおごえではなす 
\\	彼は階段を上がった。	かれはかいだんをあがった 
\\	酔っ払いが階段から落ちた。	よっぱらいがかいだんからおちた 
\\	すみませんが、ちょっと使い方を教えてください。	すみませんが、ちょっとつかいかたをおしえてください 
\\	私はその使い方を知りません。	わたしはそのつかいかたをしりません 
\\	この機械の使い方が思い出せない。	このきかいのつかいかたがおもいだせない 
\\	雪を見ると故郷を思い出す。	ゆきをみるとこきょうをおもいだす 
\\	誰かが玄関にいるよ。	だれかがげんかんにいるよ 
\\	空には雲がない。	そらにはくもがない 
\\	空は青い。	そらはあおい 
\\	今日は風が強い。	きょうはかぜがつよい 
\\	彼女は気が強い。	かのじょのきがつよい 
\\	彼の名前を覚えるのがとても難しい。	かれのなまえをおぼえるのがとてもむずかしい 
\\	意味がわからなかった。	いみがわからなかった 
\\	生きる意味を教えてくれ。	いきるいみをおしえてくれ 
\\	日曜日には洗濯をする。	にちようびにはせんたくをする 
\\	月は地球から遠い。	つきはちきゅうからとおい 
\\	彼は一人でこの寂しい場所に住んでいます。	かれはひとりでこのさみしいばしょにすんでいます 
\\	銀行は9時に開く。	ぎんこうは9じにあく 
\\	今朝は風が強いですね。	けさはかぜがつよいですね 
\\	いろいろなお菓子があった。	いろいろおかしがあった 
\\	お母さんだけがこのお菓子を作れます。	おかさんだけがこのおかしをつくれます 
\\	彼のお兄さんは先月亡くなった。	かれのおにいさんはせんげつなくなった 
\\	私はあなたのお兄さんより年上です。	わたしはあなたのおにいさんよりとしうえです 
\\	どの季節が一番好きですか。	どのきせつがいちばんすきですか 
\\	一番暑い季節はいつですか。	いちばんあついきせつはいつですか 
\\	一年には四つの季節がある。	いちねんにはよつのきせつがある 
\\	この季節は卵が安い。	このきせつはたまごがやすい 
\\	彼は研究に夢中だ。	かれはけんきゅうにむちゅうだ 
\\	サッカーに夢中です。	サッカーにむちゅうです 
\\	昨夜は楽しい夢を見た。	さくやはたのしいゆめをみた 
\\	よく悪夢を見ます。	よくあくむをみます 
\\	調子はどう。	ちょうしはどう 
\\	彼は本当に調子がいい。	かれはほんとうにちょうしがいい 
\\	時計の調子が悪い。	とけいのちょうしがわるい 
\\	今晩お客さんを招待している。	こんばんおきゃくさんをしょうたいしている 
\\	招待したの?	しょうたいしたの? 
\\	彼は汽車を降りた。	かれはきしゃをおりた 
\\	あなたは安全な場所にいる。	あなたはあんぜんなばしょにいる 
\\	看護婦は私に注射した。	かんごふはわたしにちゅうしゃした 
\\	会議に出席した。	かいぎにしゅっせきした 
\\	針金は電気を伝える。	はりがねはでんきをつたえる 
\\	事故の原因は不明だ。	じこのげんいんはふめいだ 
\\	この悲劇の本当の原因は何ですか。	このひげきのほんとうのげんいんはなんですか 
\\	出発は何時ですか。	しゅっぱつはなんじですか 
\\	冬は火事が多い。	ふゆはかじがおおい 
\\	その火事で家は灰になった。	そのかじでうちははいになった 
\\	灰皿を下さい。	はいざらをください 
\\	頭の色は灰色だった。	あたまのいろははいいろだった 
\\	私の父は青と灰色のネクタイを持っている。	わたしのちちはあおとはいいろのネクタイをもっている 
\\	夕方私は犬と散歩する。	ゆうがたわたしはいぬとさんぽする 
\\	私は幸せに感じる。	わたしはしあわせにかんじる 
\\	背中に痛みを感じる。	せなかにいたみをかんじる 
\\	私は手が震えるのを感じた。	わたしはてがふるえるのをかんじた 
\\	今朝は寒く感じる。	けさはさむくかんじる 
\\	大変寒くなった。	たいへんさむくなった 
\\	のどが渇いた感じです。	のどがかわいたかんじです 
\\	彼を少し気の毒に感じます。	かれをすこしきのどくにかんじます 
\\	この壁はとても冷たい感じがする。	このかべはとてもつめたいかんじがする 
\\	私は地震が怖い。	わたしはじしんがこわい 
\\	壁が地震で崩れた。	かべがじしんでくずれた 
\\	昨夜、大地震があった。	さくや、おおじしんがあった 
\\	私は熊が怖い。	わたしはくまがこわい 
\\	今朝の地震は感じましたか。	けさのじしんはかんじましたか 
\\	近頃牛肉が高い。	ちかごろぎゅにくがたかい 
\\	彼の歌を聞いた事がある?	かれのうたをきいたことがある 
\\	馬に乗ったことがありますか。	うまにのったことがありますか 
\\	外国に行ったことがありますか。	がいこくにいったことがありますか 
\\	富士山に登ったことがありますか。	ふじさんにのぼったことがありますか 
\\	彼の噂を聞いたことがありますか。	かれのうわさをきいたことがありますか 
\\	あなたは今まで鯨を見たことがありますか。	あなたはいままでくじらをみたことがありますか 
\\	そのような話を聞いたことがありますか。	そのようなはなしをきいたことがありますか 
\\	本を書いたことがありますか。	ほんをかいたことがありますか 
\\	これまでに男の人を愛したことがありますか。	これまでにおとこのひとをあいしたことがありますか 
\\	テニソンの詩を何か読んだことがありますか。	テニソンのしをなにかよんだことがありますか 
\\	こんな美しい日の入りを見たことがありますか。	こんなうつくしいひのいりをみたことがありますか 
\\	私はそのように考えた。	わたしはそのようにかんがえた 
\\	あなたは恥じるべきだ。	あなたははじるべきだ 
\\	公園にはそのような小鳥がたくさんいます。	こうえんにはそのようなことりがたくさんいます 
\\	私は結局失敗した。	はたしはけっきょくしっぱいした 
\\	ピーターは結局来なかった。	ピーターはけっきょくこなかった 
\\	彼女は噂話が大好きだ。	かのじょはうわさはなしがだいすきだ 
\\	彼は見聞の広い人だ。	かれはみききのひろいひとだ 
\\	その市には広い道が多い。	そのしにはひろいみちがおおい 
\\	日本では広い庭はなかなかありません。	にほんではひろいにわはなかなかありません 
\\	彼はレースに参加した。	かれはレースにさんかした 
\\	彼は元気に試合に参加した。	かれはげんきにしあいにさんかした 
\\	両親には従うべきだ。	りょうしんにはしたがうべきだ 
\\	法律には、従うべきだ。	ほうりつには、したがうべきだ 
\\	彼女は壁を白く塗った。	かのじょはかべをしろくぬった 
\\	ねずみが壁に穴を開けた。	ねずみがかべにあなをあけた 
\\	深い穴を掘る。	ふかいあなをほる 
\\	靴下に穴が開いているよ。	くつしたにあながあけていろよ 
\\	その犬は穴を掘っていた。	そのいぬはあなをほっていた 
\\	船は底に沈んだ。	ふねはそこにしずんだ 
\\	グラスの底に少しワインが残っている。	グラスのそこにすこしわいんがのこっている 
\\	今日は残業をしないつもりです。	きょうはざんぎょうをしないつもりです 
\\	彼は週に2回残業をする。	かれはしゅうに2かいざんぎょうをする 
\\	私は昨日残業しなければならなかった。	わたしはきのうざんぎょうしなければならなかった 
\\	鍵を捜すのを手伝ってくれませんか。	かぎをさがすのをてつだってくれませんか 
\\	私の祖父は長生きした	わたしのそふはながいきした 
\\	祖父は大阪の出身です。	そふはおおさかのしゅっしんです 
\\	祖父は戦争で負傷した。	そふはせんそうでふしょうした 
\\	祖父はいつもこの椅子に座る	そふはいつもこのいすにすわる 
\\	いい上司で羨ましいですね。	いいじょうしでうらやましいですね 
\\	あなたがとても羨ましい。	あなたがとてもうらやましい 
\\	彼女は何でも両親に相談する。	かのじょはなんでもりょうしんにそうだんする 
\\	相談に来てください。	そうだんにきてください 
\\	私は姉に相談した。	わたしはあねにそうだんした 
\\	彼女は弁護士と相談した。	かのじょはべんごしとそうだんした 
\\	私はその結果を心配している。	わたしはそのけっかをしんぱいしている 
\\	町は霧に包まれた。	まちはきりにつつまれた 
\\	彼女は本を紙で包んだ。	かのじょはほんをかみでつつんだ 
\\	頭をスカーフで包みなさい。	あたまをスカーフでつつみなさい 
\\	試験に失敗した。	しけんにしっぱいした 
\\	試験に合格した。	しけんにごうかくした 
\\	一生懸命走った。	いっしょうけんめいはしった 
\\	彼は一生懸命飲んだ。	かれはいっしょうけんめいのんだ 
\\	彼は大変一生懸命働きました。	かれはたいへんいっしょうけんめいはたらきました 
\\	彼は一生懸命やったが、失敗した。	かれはいっしょうけんめいやったが、しっぱいした 
\\	次の朝はみんなひどい二日酔いした。	つぎのあさはみんなひどいふつかよいした 
\\	翌日は足が痛かった。	よくじつはあしがいたかった 
\\	彼は翌日に家に帰ると言った。	かれはよくじつにうちにかえるといった 
\\	トムは月曜日に来て翌日帰った。	トムはげつようびにきてよくじつかえった 
\\	彼女はその翌年に女優になった。	かのじょはぞのよくとしにじょゆうになった 
\\	結婚した翌年に女の子が生まれた。	けっこんしたよくねんにおんんあのこがうまれた 
\\	楽しみのない生活。	たのしみのないせいかつ 
\\	生活費を切りつめた。	せいかつひをきりつめた 
\\	彼は新生活を始めた。	かれはしんせいかつをはじめた 
\\	この騒音は何だ?	このそうおんはなんだ? 
\\	彼は友達のノートを写すのに忙しかった。	かれはともだちのノートをうつすのにいそがしかった 
\\	その男の人は森で道に迷いました。	そのおとこのひとはもりでみちにまよいました 
\\	彼女は道に迷い、そのうえ雨が降り出した。	かのじょはみちにまよい、そのうえあめがふりだした 
\\	道に迷うと行けないから、地図を持っていきなさい。	みちにまようといけないから、ちずをもっていなさい 
\\	心配するな。よくある間違いだから。	しんぱいするな。よくあるまちがいだから 
\\	よくあることだが、彼女は傘を忘れた。	よくあることだが、かのじょはかさをわすれた 
\\	彼は丸一日外出することがよくある。	かれはまるいちにちそがいしゅつすることがよくある 
\\	誰でもたまには間違いをする。	だれでもたまにはまちがいをする 
\\	たまには頭を使えよ。	たまにはあたまをつかえよ 
\\	偉い人は身なりを気にしない。	えらいひとはみなりをきにしない 
\\	彼女は服装をたいへん気にする。	かのじょはふくそうをたいへんきにする 
\\	彼は怒った。	かれはいかった 
\\	彼女はいつも快活でにこにこしている。	かのじょはいつもかいかつでにこにこしている 
\\	乾いた木材はよく燃える。	かわいたもくざいはよくもえる 
\\	その家は燃えていた。	そのうちはもえていた 
\\	彼女は自殺を企てた。	かのじょはじさつをくわだてた 
\\	彼は今朝銃で自殺した。	かれはけさじゅうでじさつした。 
\\	彼は日に何度か祈る。	かれはにちになんどかいのる 
\\	どこで安い電気製品を売っていますか。	どこでやすいでんきせいひんをうっていますか 
\\	何か楽器が演奏できますか。	なにかがっきがえんそうできますか 
\\	その会社は電気製品を製造している。	そのかいしゃはでんきせいひんをせいぞうしている 
\\	その老人たちは紳士服を製造します。	そのろうじんたちはしんしふくをせいぞうします 
\\	老人に親切にしなさい。	ろうじんにしんせつにしなさい 
\\	日本で20歳からたばこを吸ってもいいです。	にほんで20さいたばこをすってもいいです 
\\	エレベーターで遊んではいけません	エレベーターであそんではいけません 
\\	その老人は地面に倒れた。	そのろうじんはじめんにたおれた 
\\	地面はまだ濡れている。	じめんはまだぬれている 
\\	地下鉄は地面の下を走る。	ちかてつはじめんのしたをはしる 
\\	リンゴが一個地面に落ちた。	リンゴがいっこじめんにおちた 
\\	秋には木の葉が地面に落ちる。	あきにはきのはがじめんにおちる 
\\	時刻表を見せてください。	じこくひょうをみせてください 
\\	3時に歯医者の予約がある。	3じにはいしゃのよやくがある 
\\	私は将来歯医者になりたい。	わたしはしょうらいはいしゃになりたい 
\\	床屋さんがあなたの髪をとても短く切りましたね。	とこやさんがあなたのかみをとてもみじかくきりましたね 
\\	昨日床屋で髪を切ってもらったんだ。	きのうとこやでかみをきってもらったんだ 
\\	最近あまり会わない	さいきんあまりあわない 
\\	おタバコはご遠慮下さい。	おタバコはごえんりょください 
\\	どうぞ遠慮なく質問してください。	どうぞえんりょなくしつもんしてください 
\\	遠慮なく私の辞書を使ってください。	えんりょなくわたしのじしょをつかってください 
\\	彼の計画に賛成です。	かれのけいかくにさんせいです 
\\	何か計画がありますか。	なにかけいかくがありますか 
\\	あなたの意見に賛成です。	あなたのいけんにさんせいです 
\\	賛成の人は皆手を挙げた。	さんせいのひとはみなてをあげた 
\\	彼は魚を三匹釣った。	かれはさかなをさんびきつった 
\\	二匹の犬は眠っている。	にひきのいぬがねむっている 
\\	羊が2匹狼に殺されました。	ひつじがにひきおおかみにころされました 
\\	羊は何て鳴くの。	ひつじはなんでなくの 
\\	彼は羊とヤギの区別が付かない。	かれはひつじとヤギのくべつがつかない 
\\	このテープはよく付く。	このテープはよくつく 
\\	彼女の水着は目に付く。	かのじょのみずぎはめにつく 
\\	それは秘密です。	それはひみつです 
\\	部屋は兎小屋みたいだけど	へやはうさぎごやみたいだけど 
\\	雪の中で、その白兎の姿は見えなかった。	ゆきのなかで、そのしろうさぎのすがたはみえなかった 
\\	秘密を守れるか。	ひみつをまもれるか 
\\	彼は秘密を知っている。	かれはひみつをしっている 
\\	私は男の姿を見た。	わたしはおとこのすがたをみた 
\\	その山の姿は美しい。	そのやまのすがたはうつくしい 
\\	彼女は暗闇の中に姿を消した。	かのじょはくらやみのなかにすがたをけした 
\\	飛行機の姿は見えなくなった	ひこうきのすがたはみえなくなった 
\\	値段が高すぎです。	ねだんがたかすぎです 
\\	その店は値段を全部上げた。	そのみせはねだんをぜんぶあげた 
\\	真っ暗闇だ。	まっくらやみだ 
\\	猫は暗闇でも見える。	ねこはくらやみでもみえる 
\\	彼女は暗闇を恐れる。	かのじょはくらやみをおそれる 
\\	暗闇の中で小さな物が動いた。	くらやみのなかでちいさなものがうごいた 
\\	彼は動く事ができなかった。	かれはうごくことができなかった 
\\	嘘は決してつくな!	うそはけっしてつくな! 
\\	この部屋は決して狭くない。	このへやはけっしてせまくない 
\\	その事故で彼らの幸せを奪った。	そのじこでかれらのしあわせをうばった 
\\	彼は金持ちだが幸せではない。	かれはかねもちだがしあわせではない 
\\	試験が終わったとき私たちは幸せに感じた。	しけんがおわったときわたしたちはしあわせにかんじた 
\\	彼は金持ちだが不親切だ。	かれはかねもちだがふしんせつだ 
\\	金に不足している。	かねがふそくしている 
\\	事故は不注意から生じる。	じこはふちゅいからしょうじる 
\\	ちっとも構いませんよ。	ちょっともかまいませんよ 
\\	今日はちっとも風がない。	きょうはちょっともかぜがない 
\\	犬なんかちっとも恐くない。	いぬなんかちょっともこわくない 
\\	彼は科学者として有名だ。	かれはかがくしゃとしてゆうめいだ 
\\	絹は手触りが柔らかい。	きぬはてざわりがやわらかい 
\\	此処に鉛筆が二本ある。一本は堅く、もう一本は柔らかい。	ここにえんぴつにほんがある。いっぽんはかたく、もういっぽんはやわらかい 
\\	ニッケルは硬い銀白色の金属です。	ニッケルはかたいぎんはくしょくのきんぞくです 
\\	彼は床に本を落とした。	かれはゆかにほんをおとした 
\\	鳥は枝に止まった。	とりはえだにとまった 
\\	猫は枝の間に隠れた。	ねこはえだのあいだにかくれた 
\\	その少年はドアの陰に隠れた。	そのしょうねんはドアのかげにかくれた 
\\	彼はナイフでその木の小枝を切り取った。	かれはナイフでそのきのこえだをきりとった 
\\	彼は道具として使われた。	かれはどうぐとしてつかわれる 
\\	暖房が壊れています。	だんぼうがこわれています 
\\	暖房を消しましたか。	だんぼうをけしましたか 
\\	経済はやや不景気だ。	けいざいはややふけいきだ 
\\	彼は経済の専門家だ。	かれはけいざいのせんもんかだ 
\\	彼は両親から経済的に独立している。	かれはりょうしんからけいざいてきにどくりつしている 
\\	貧困が彼に独立することを教えた。	ひんこんがかれにどくりつすることをおしえた 
\\	第一に名前を決めなくちゃ。	だいいちになまえをきめなくちゃ 
\\	もう行かなくちゃ。じゃ、またね。	もういかなくちゃ。じゃ、またね 
\\	第一に、彼女は若すぎる。	だいいちに、かのじょはわかすぎる 
\\	あの店は広い範囲の品物を売っている。	あのみせはひろいはんいのしなものをうるっている 
\\	私は新しい服を買わなくちゃ。	わたしはあたらしいふくをかわなくちゃ 
\\	ところで今夜暇ですか。	ところでこんやひまですか 
\\	ところで、予備の電池はあるの?	ところで、よびのでんちはあるの? 
\\	貧乏暇なしですよ。	びんぼうひまなしですよ 
\\	彼は確かに貧乏だが、幸せだ。	かれはたしかにびんぼうだが、しあわせだ 
\\	一緒に踊りませんか。	いっしょにおどりませんか 
\\	ネコはネズミを捕まえる。	ネコはネズミをつかまえる 
\\	彼はすべて悪い冗談だと思った。	かれはすべてわるいじょうだんだとおもった 
\\	冗談でしょう。	じょうだんでしょう 
\\	私はその冗談の意味がわからなかった。	わたしはそのじょうだんのいみがわからなかった 
\\	ギターを弾くことだと言いました。	ギターをひくことだといいました 
\\	彼の言ったことは全て正しかった。	かれのいったことはすべてただしかった 
\\	彼はイタリアに行くと私に言った。	かれはイタリアにいくとわたしにいった 
\\	トムは頭が痛いと私たちに言った。	トムはあたまがいたいとわたしたちにいった 
\\	食べ過ぎると太りますよ。	たべすぎるとふとりますよ 
\\	もっと運動しないと、あなたは太りますよ。	もっとうんどうしないと、あなたはふとりますよ 
\\	彼女は寿司が好きだと言いました	かのじょはすしがすきだといいました 
\\	サーカーを見ることだと言いました	サーカーをみることだといいました 
\\	空が晴れた。	そらがはれた 
\\	その女優は若者に人気がある。	そのじょゆうはわかものににんきががある 
\\	彼の音楽は若者に受ける。	かれのおんがくはわかものにうける 
\\	彼らの大多数は若者です。	かれらのだいたすうはわかものです 
\\	若者は冒険を愛する。	わかものはぼうけんをあいする 
\\	この本は2ページ足りない。	このほんは2ページたりない 
\\	彼はお金が足りないと言った。	かれはおかねがたりないといった 
\\	大多数の若者は戦争の恐怖を知らない。	だいたすうのわかものはせんそうのきょうふをしらない 
\\	突然彼は死んだ。	とつぜんかれはしんだ 
\\	寝る時間が足りないだと言った。	ねるじかんがたりないだといった 
\\	この会議は時間の無駄だったと言いました。	このかいぎはじかんのむだだったといいました 
\\	胸がムカムカします。	むねがムカムカします 
\\	微熱があります。	びねつがあります 
\\	指にやけどをしました。	ゆびにやけどをしました 
\\	熱いスープで彼女の舌がやけどした。	あついスープでかのじょのしたがやけどした 
\\	彼は通りから栗を取り除いた。	かれはとおりからくりをとりのぞいた 
\\	悪い習慣を取り除くことはとても難しい。	わるいしゅうかんをとりのぞくにことはとてもむずかしい 
\\	最悪を覚悟している。	さいあくをかくごしている 
\\	最悪の事態はもう終わった。	さいあくのじたいはもうおわった 
\\	寒いからコートを着るべきだ。	さむいからコートをきるべきだ 
\\	彼は厚い眼鏡をかけている。	かれはあついめがねをかけている 
\\	その少年は運動靴を履くと、外へ走って出て行った。	そのしょうねんはうんどうくつをはくと、そとへ走ってでていった 
\\	これは万里の長城で撮った写真です。	これはばんりのちょうじょうでとったしゃしんです 
\\	あの着物を着ている人はだれですか。	あのきものをきているひとはだれですか 
\\	あなたがとても恋しい。	あなたがとてもこいしい 
\\	彼女は色白金髪だ。	かのじょはいろじろきんぱつだ 
\\	あのグリーンの服を着た金髪の背の高い女の子は誰だか知っている?	あのグリーンのふくをきたきんぱつのせのたかいおんなのこはだれだかしっている? 
\\	物価は2年前の2倍である。	ぶっかは2ねんまえの2ばいである 
\\	彼と議論しても無駄だ。	かれとぎろんしてもむだだ 
\\	彼はよく時間を無駄に過ごす。	かれはよくじかんをむだにすごす 
\\	ジュデイーは鏡を見て多くの時間を過ごす。	ジュデイーはかがみをみておおくのじかんをすごす 
\\	愛を告白する。	あいをこくはくする 
\\	彼は私を好きになったと告白した。	かれはわたしをすきになったとこくはくした 
\\	彼はその箱を壊した。	かれはそのはこをこわした 
\\	その事が彼のすべての夢を壊した。	そのことがかれのすべてのゆめをこわした 
\\	ジャックは母の大切な花瓶を壊した	ジャックはははのたいせつなかびんをこわした 
\\	彼女は展覧会で一等賞を得た。	かのじょはてんらんかいでいっとうしょうをえた 
\\	昨日悲しい出来事があった。	きのうかなしいできごとがあった 
\\	この小説は退屈だ。	このしょうせつはたいくつだ 
\\	七面鳥は鶏より少し大きい。	しちめんちょうはとりよりすこしおおきい 
\\	彼は20年前、大学を卒業するとすぐに公務員になった。	かれは20ねんまえ、だいがくをそつぎょうするとすぐにこうむいんなった 
\\	いつ高校を卒業したの。	いつこうこうをそつぎょうしたの 
\\	大学を卒業したあとはどうしたいのですか。	だいがくをそつぎょうしたあとはどうしたいのですか 
\\	私は高校を卒業したらロンドンに行こうと思う。	わたしはこうこうをそつぎょうしたらロンドンにいこうとおもう 
\\	大学生の時、有理子は愛欲に身を委ねた。	だいがくせいのとき、ゆりこはあいよくにみをゆだねた 
\\	下着までビショビショです。	したぎまでビショビショです 
\\	兵士たちは故国のために死ぬ覚悟が出来ている。	へいしたちはここくのためにしぬかくごができている 
\\	昨日日記を書かなかった。	きのうにっきをかかなかった 
\\	市内をぐるっと案内しましょう。	しないをぐるっとあんないしましょう 
\\	私はシカゴの案内書が欲しい。	わたしはシカゴのあんないしょがほしい 
\\	いつ都合がいいですか。	いつつごうがいいですか 
\\	厳しい世の中だなあ。	きびしいよのなかだなあ 
\\	その先生は生徒に厳しい。	そのせんせいはせいとにきびしい 
\\	概して、カナダは厳しい気候である。	がいして、カナダはきびしいきこうである 
\\	気候は快適です。	きこうはかいてきです 
\\	タバコは悪い習慣です。	タバコはわるいしゅうかんです 
\\	間食はよくない習慣だ。	かんしょくはよくないしゅうかんだ 
\\	彼は習慣をなかなか変えない。	かれはしゅうかんをなかなかかえない 
\\	これは日本固有の習慣だ。	これはにほんこゆうのしゅうかんだ 
\\	これは日本特有の習慣だ。	これはにほんとくゆうのしゅうかんだ 
\\	彼女は早起きの習慣がついた。	かのじょははやおきのしゅうかんがついた 
\\	彼は経験不足だ。	かれはけいけんふそくだ 
\\	私たちは戦争を3回経験した。	わたしたちはせんそうを3かいけいけんした 
\\	金不足になった。	かねふそくになった 
\\	その犬は茶色で小さくて、やせています。	そのいぬはちゃいろでちいさくて、やせています 
\\	昨夜、屋上から星を観察した。	さくや、おくじょうからほしをかんさつした 
\\	彼女は屋上でよくヴァイオリンの練習をしていた	かのじょはおくじょうでよくヴァイオリンのれんしゅうをしていた 
\\	その孤独な男は蟻を観察することに楽しみを感じる。	そのこどくなおとこはありをかんさつすることにたのしみをかんじる 
\\	読書は精神を育てる。	どくしょはせいしんをそだてる 
\\	その動物を育てることは可能ですか。	そのどうぶつをそだてることはかのうですか 
\\	子供を育てるには忍耐が必要です。	こどもをそだてるにはにんたいがひつようです 
\\	その目標は達成不可能だ。	そのもくひょうはたっせいふかのうだ 
\\	彼に勝つことは不可能だ。	かれはかつことはふかのうだ 
\\	不可能なことはできない。	ふかのうなことはできない 
\\	彼はびっくりして裸足で外に飛び出した。	かれはびっくりしてはだしでそとにとびだした 
\\	朝食後すぐに飛び出した。	ちょうしょくごすぐにとびだした 
\\	彼の演技は期待に添わなかった。	かれのえんぎはきたいにそわなかった 
\\	乗客は皆乗りましたか。	じょうきゃくはみんなのりましたか 
\\	バスの乗客はほとんど寝ていた。	バスのじょうきゃくはほとんどねていた 
\\	元気を出せ、私たちはきっと助かる。	げんきをだす、わたしたちはきっとたすかる 
\\	大変助かりました。	たいへんたすかりました 
\\	彼は綱をつかんで助かった。	かれはつなをつかんでたすかった 
\\	羽目を外して飲む。	はめをはずしてのむ 
\\	鳥が2羽3羽飛んで来た。	とりが2わ3わとんできた 
\\	湖には何百羽もの鳥がいた。	こにはなんひゃくわものとりがいた 
\\	「好色と倒錯の違いとは?」「好色なら羽を使うが、倒錯なら鶏をまるごと使う」	"「こうしょくととうさくのちがいとは?」「こうしょくならはねをつかうが、とうさくならとりをまるごと使う」 
\\	最善を期待しよう。	さいぜんをきたいしょう 
\\	彼女は私達の期待にこたえた。	かのじょはわたしたちのきたいにこたえた 
\\	私は事の真相を確かめるつもりだ。	わたしはことのしんそうをたしかめるつもりだ 
\\	電車は待つことが退屈だ。	でんしゃはまつことがたいくつだ 
\\	その芝居の後半は少し退屈だった。	そのしばいのこうはんはすこしたいくつだった 
\\	法律を犯す者は罰せられる。	ほうりつをおかすものはばっせられる 
\\	政府は法律を施行しなければならない。	せいふはほうりつをしこうしなければなりません 
\\	コンピューターは現在では絶対的な必需品である。	コンピューターはげんざいでぜったいてきなひつじゅひんである 
\\	数人の人が畑で働いている。	すうにんのひとがはたけではたらいている 
\\	私は芸術畑で働いています。	わたしはげいじゅつはたけではたらいています 
\\	政府は教育を改革している。	せいふはきょういくをかいかくしている 
\\	台風が関東地方に上陸した。	たいふうがかんとうちほうにじょうりくした 
\\	東北地方は旅行する価値があります。	とうほくちほうはりょこうするかちがあります 
\\	勝ち負けの確率は五分五分だ。	かちまけのかくりつはごぶごぶだ 
\\	私たちのチームがその試合に勝てるかどうかは五分五分だ。	わたしたちのチームがそのしあいにかてるかどうかはごぶごぶだ 
\\	君は食べ物を買ったのだから僕がワインを買えば五分五分になる。	きみはたべものをかったのだからぼくがワインをかえばごぶごぶになる 
\\	私は窓から風景を眺めていた。	わたしはまどからふうけいをながめていた 
\\	その風景の美しさは筆舌に尽くしがたかった。	そのふうけいのうつくしさはひつぜつにつくしがたかった   
\\	その建物は周りの風景と調和しない。	そのたてものはまわりのふうけいとちょうわしない 
\\	その命令に従うより他に仕方ない。	そのめいれいにしたがうよりほかにしかたない 
\\	君の相変わらずの愚痴には、むかつくよ。	きみのあいかわらずのぐちには、むかつくよ 
\\	彼は私にあうといつも自分の奥さんの愚痴を言う。	かれはわたしにあうといつもじぶんのおくさんのぐちをいう 
\\	地平線に愛が現れる。	ちへいせんにあいがあらわれる 
\\	太陽が地平線の下に沈み暗くなった。	たいようがちへいせんのしたにしずみくらくなった 
\\	新しい場所でも友達を作るのは簡単だ。	あたらしいばしょでもともだちをつくるのはかんたんだ 
\\	この試合には負けたくない。	このしあいにはまけたくない 
\\	彼女の親に頼りたくない。	かのじょのおやにたよりたくない 
\\	私は一人では買い物にいきたくない。	わたしはひとりではかいものにいきたくない 
\\	朝早く起きてせっせと働きたくない。	あさはやくおきてせっせとはたらきたくない 
\\	海外旅行に誘われたけど、行きたくない。	かいがいりょこうにさそわれたけど、いきたくない 
\\	彼は、その事故で左足を怪我した。	かれは、そのじこでひだりあしをけがした 
\\	少し間違うと大怪我につながる大変危険なスポーツです。	すこしまちがうとおおけがにつながるたいへんきけんなスポーツです 
\\	この映画の指定席はありますか。	このえいがのしていせきはありますか 
\\	此処は交通量が多い。	ここはこうつうりょうがおおい 
\\	父は交通騒音について不平をこぼした。	ちちはこうつうそうおんについてふへいをこぼした 
\\	彼女はいつも何か不平を言っている。	かのじょはいつもなにかふへいをいっている 
\\	ニックは東京の物価は高くて大変だと私に不平を言った。	ニックはとうきょうのぶっかはたかくてたいへんだとわたしにふへいをいった 
\\	彼女はいつも夫について不平ばかり言っていた。	かのじょはいつもおっとについてふへいばかりいっていた 
\\	私は今着いたばかりだ。	わたしはいまついたばかりだ 
\\	銀行を捜してるんですが、近くにありますか。	ぎんこうをさがしてるんですが、ちかくにありますか 
\\	泥棒たちはお金を捜して机の引き出しを全部開けた。	どろぼうたちはおかねをさがしてつくえのひきだしをぜんぶあけた 
\\	彼の家を探して一時間以上も歩き回った。	かれのいえをさがしていちじかんいじょうもあるきまわった 
\\	列を上に下に歩き回る時、「ホット・ドッグはいかが。ホット・ドッグはいかが。」と叫ぶ。	"れつをうえにしたにあるきまわるとき、「ほっと・ドッグはいかが。ホット・ドッグはいかが。」とさけぶ。 
\\	私たちは誰かが叫ぶのを聞いた。	わたしたちはだれかさけぶのをきいた 
\\	もの凄い叫び声に彼はぞっとした。	ものすごいさけびこえにかれはぞっとした 
\\	突然、猫の鋭い叫び声が私たちに聞こえた。	とつぜん、ねこのするどいさけびこえがわたしたちにきこえた 
\\	その秘書は美人の上に英語が得意だ。	そのひしょはびじんのうえにえいごがとくいだ 
\\	概して、日本の人々は外国語が不得意だ。	がいして、にほんのひとびとはがいこくごがふとくいだ 
\\	道路を渡る時には車に注意しなさい。	どうろをわたるときにはくるまにちゅういしなさい 
\\	彼は複雑な数学の問題を解くことが得意だ。	かれはふくざつなすうがくのもんだいをとくことがとくいだ 
\\	雨にもかかわらず彼は出かけた。	あめにもかかわらずかれはでかけた 
\\	彼は多くの困難にもかかわらず、成功した。	かれはおおくのこんなんにもかかわらず、せいこうした 
\\	姉は我々の反対にもかかわらず彼と結婚した。	あねはわれわれのはんたいにもかかわらずかれとけっこんした 
\\	彼の反対は思ったより激しかった。	かれのはんたいはおもったよりはげしかった 
\\	家賃は月幾らですか。	やちんはがついくらですか 
\\	今月の家賃を払うのを忘れた。	こんげつのやちんをはらうのをわすれた 
\\	やっとややこしい迷路の外に出られた。	やっとややこしいめいろのそとにでられた 
\\	来月に引っ越す予定です。	らいげつにひっこすよていです 
\\	もっとお金があれば、もっと大きな家に引っ越すでしょう。	もっとおかねがあれば、もっとおおきなうちにひっこすでしょう 
\\	田舎に引っ越したい。	いなかにひっこしたい 
\\	彼はビールを一杯頼んだ。	かれはビールをいっぱいたのんだ 
\\	禁煙席を頼んだのですが。	きんえんせきをたのんだのですが 
\\	誤解しないでください。	ごかいしないでください 
\\	その人たちは契約に満足している。	そのひとたちはけいやくにまんぞくしている 
\\	彼女は結果に満足した。	かのじょはけっかにまんぞくした 
\\	私は自分の英語力に満足していない。	わたしはじぶんのえいごちからにまんぞくしていない 
\\	わがままな子供を満足させることはできない。	わがままなこどもをまんぞくさせることできない 
\\	飛べないのと同じように私は未来を予言できない。	とべないのとおなじようにわたしはみらいをよげんできない 
\\	未来のことをどうしてそんなに楽観できるんですか。	みらいのことをどうしてそんなにらっかんできるんですか 
\\	彼の病状は好転した。	かれのびょうじょうはこうてんした 
\\	彼は悲観する傾向がある。	かれはひかんするけいこうがある 
\\	私は医者を完全に信頼している。	わたしはいしゃをかんぜんにしんらいしている 
\\	彼が信頼できることは君に保証できる。	かれがしんらいできることはきみにほしょうできる 
\\	今日は寒くてしょうがない。	きょうはさむくてしょうがない 
\\	予定より進んでいる。	よていよりすすんでいる 
\\	その船は強い風に逆らってゆっくりと進んだ。	そのふねはつよいかぜにさからってゆくりとすすんだ 
\\	あなたは競技に参加しましたか。	あんたはきょうぎにさんかしましたか 
\\	彼は慎重な選手だ。	かれはしんちょうなせんしゅだ 
\\	あのフットボール選手は非常に大きい。	あのフットボールせんしゅはひじょうにおおきい 
\\	サッカー選手として彼は誰にも劣らない。	さっかーせんしゅとしてかれはだれにもおとらない 
\\	彼は怪我をした選手の代りをつとめた。	かれはけがをしたせんしゅのかわりをつとめる 
\\	彼はボブより賢明で慎重だ。	かれはボブよりけんめいでしんちょうだ 
\\	この点で私は彼に劣る。	このてんでわたしはかれにおとる 
\\	彼は私より学識が劣っている。	かれはわたしよりがくしきがおとっている 
\\	先生は彼に帰宅を許した。	せんせいはかれにきたくをゆるした 
\\	彼は善悪の区別が付かない。	かれはぜんあくのくべつがつかない 
\\	たいていのヨーロッパ人は日本人と中国人の区別が付かない。	たいていのユーロッパじんはにほんじんとちゅうごくじんのくべつがつかない 
\\	悲しくて彼女は突然泣き出した。	かなしくてかのじょはとつぜんなきでました 
\\	泣いている赤ん坊は手に負えない。	ないているあかんぼうはてにおえない 
\\	その子供は空腹で泣いていた。	そのこどもはくうふくでないていた 
\\	空腹は最上のソースである。	くうふくはさいじょうのソースである 
\\	ジョンは関東地区、そして太郎は関西地区を担当している。	ジョンはかんとうちく、そしてたろうはかんさいちくをたんとうしている 
\\	スミス先生が学校を辞めたら、誰が私たちのクラスを担当するのだろう。	スミス先生はがっこうをやめたら、だれがわたしたちのクラスをたんとうするのだろう 
\\	私たちは難しい選択に直面している。	わたしたちはむずかしいせんたくにちょくめんしている 
\\	涙は子供の武器である。	なみだはこどものぶきである 
\\	彼らの中には武器を作る才能のある者がいた。	かれらのなかにはぶきをつくるさいのうのあるものがいた 
\\	多分私はこの本が気に入るでしょう。	たぶんわたしはこのほんがきにいるでしょう 
\\	ジョージがこの考えを気に入るかどうか確かでない。	ジョージがこのかんがえをきにいるかどうかたしかでない 
\\	あのバーは彼が良く顔を出すお気に入りの場所である。	あのバーはかれがよくかおをだすおきにいりのばしょである 
\\	軽率さが彼の主な特徴である。	けいそつさがかれのおもなとくちょうである 
\\	ざくざくと雪を踏んで進む。	ざくざくとゆきをふんですすむ 
\\	あなたの態度は良くない。	あなたのたいどはよくない 
\\	彼女の横柄な態度は腹に据えかねた。	かのじょのおうへいなたいどははらにすえかねた 
\\	意外な結果が発表された。	いがいなけっかがはっぴょうされた 
\\	初めて彼に会ったとき、意外な質問にびっくりしました。	はじめてかれにあったとき、いがいなしつもんにびっくりしました 
\\	何か集めていますか。	なにかあつめていますか 
\\	彼女は子供たちを傍に呼び集めた。	かのじょはこどもたちをそばによびあつめた 
\\	私は辞書をつくる目的で用例を集めた。	わたしはじしょをつくるもくてきでようれいをあつめた 
\\	奇術師は子供たちの注目を集めていた。	きじゅつしはこどもたちのちゅうもくをあつめていた 
\\	母にプレゼントを捜しています。何か特にお考えですか。	ははにプレゼントをさがしています。なにかとくにおかんがえですか 
\\	少年と少女は知り合いらしい。	しょうねんとしょうじょはしりあいらしい 
\\	彼はその学生が気に入っているらしい。	かれはそのがくせいがきにいっているらしい 
\\	トムは明日の授業の予習をしているらしい。	トムはあしたのじゅぎょうのよしゅうをしているらしい 
\\	明らかにこれが最も重要な点です。	あきらかにこれがもっともじゅうよう なてんです 
\\	彼女はピンクのシャツを選んで、私に試着してみよと言った。	かのじょはピンクのシャツをえらんで、わたしはしちゃくしてみよといった 
\\	東京は混んだ電車に慣れなければなりません。	とうきょうはこんだでんしゃになれなければなりません 
\\	すぐにこの新しい生活に慣れるでしょう。	すぐにこのあたらしいせいかつになれるでしょう 
\\	私の娘は新しい学校に慣れるのがたやすいとは思わないだろう。	わたしのむすめはあたらしいがっこうになれるのがたやすいとはおもわないだろう 
\\	月の上にいると想像しなさい。	つきのうえにいるとそうぞうしなさい 
\\	お金のない世界を想像することが出来ますか。	おかねのないせかいをそうぞうすることができますか 
\\	私は2年ぶりで故郷の村に帰った。	わたしは2ねんぶりでこきょうのむらにかえった 
\\	故郷の最新情報を教えてあげましょう。	こきょうのさいしんじょうほうをおしえてあげましょう 
\\	彼は15歳で故郷を出て以来2度と帰らなかった。	かれは15さいでこきょうをでていらい2どとかえらなかった 
\\	此処に来て以来、生活は単調そのものだ。	ここにきていらい、せいかつはたんちょうそのものだ 
\\	前月仲たがいして以来カレンには会っていない。	ぜんげつなかたがいしていらいカレンにあっていない 
\\	宗教は信じない。	しゅうきょうはしんじない 
\\	あなたは何か宗教を信じていますか。	あなたはなにかしゅうきょうをしんじていますか 
\\	私たちは政治と宗教を分けなければならない。	わたしたちはせいじとしゅうきょうをわけなければならない 
\\	皆が不法外国人に対して敵対的な態度を取った。	みながふほうがいこくじんにたいしててきたいてきなたいどをとった 
\\	彼女は体重が増えるといけないからダイエットをしている。	かのじょはたいじゅうがふえるといけないからダイエットをしている 
\\	彼は決心を変えた。	かれはけっしんをかえた 
\\	世界の人口は毎年平均2パーセントの割合で増えている。	せかいのじんこうはまいとしへいきん2パーセントのわりあいでふえている 
\\	彼女は秘書になる決心をした。	かのじょはひしょになるけっしんをした 
\\	君は息が臭い。	きみはいきがくさい 
\\	「服汚れるでしょ」「それはノープロブレム。もともとあんまり綺麗じゃないし」	"「ふくよごれるでしょ」「それはノープロブレム。もともとあんまりきれいじゃない」 
\\	彼女は食事の前に汚れた手を洗った。	かのじょはしょくじのまえによごれたてをあらった 
\\	私の本に汚れた手で触れてはいけない。	わたしのほんによごれたてでふれてはいけない 
\\	石鹸は汚れを落とす特性がある。	せっけんはよごれをおとすとくせいがある 
\\	彼は現代生活の複雑性について長々と話した。	かれはげんだいせいかつのふくざつせいについてながながとはなした 
\\	彼女は外国の経験について長々と話した。	かのじょはがいこくのけいけんについてながながとはなした 
\\	彼を墓に埋めた。	かれをはかにうめた 
\\	そのトラは檻の真ん中に寝そべっていた。	そのトラはおりのまんなかにねそべっている 
\\	道の真ん中に故障した車がとまっていた。	みちんまんなかにこしょうしたくるまがとまっていた 
\\	救急車はけが人を最寄りの病院に運んだ。	きゅうきゅうしゃはけがにんをもよりのびょういんにはこんだ 
\\	マラリアは蚊が運ぶ病気です。	マラリアはかがはこぶびょうきです 
\\	私はその若者に荷物を運んでもらった。	わたしはそのわかものににもつをはこんでもらった 
\\	彼らは我々の計画を邪魔した。	かれらはわれわれのけいかくをじゃました 
\\	彼はほとんど天才です。	かれはほとんどてんさいです 
\\	外国語を短期間に習得することはほとんど不可能だ。	がいこくごをたんきかんにしゅうとくすることはほとんどふかのうだ 
\\	この講座は正確な発音を習得するための手助けになります。	このこうざはせいかくなはつおんをしゅうとくするためのてだすけになります 
\\	彼女は京都出身だった、そのことは彼女の発音から分かった。	かのじょはきょうとしゅっしんだった、そのことはかのじょのはつおんからわかった 
\\	発音に関しては、彼は全生徒のうちで一番だった。	はつおんにかんしては、かれはぜんせいとのうちでいちばんだった 
\\	負傷者は多かったが、行方不明の人はほとんどいなかった。	ふしょうしゃはおおかったが、ゆくえふめいのひとはほとんどいなかった 
\\	彼女は幸せな幼年時代を過ごした。	かのじょはしあわせなようねんじだいをすごした 
\\	自分だけの幸せを追求して生きるべきではない。	じぶんだけのしあわせをついきゅうでいきるべきではない 
\\	私は、友人の父が急死したのを気の毒に思った。	わたしは、ゆうじんのちちがきゅうししたのをきのどくにおもった 
\\	試験に落ちた学生の名前は掲示板に張り出された。	しけんにおちたがくせいのなまえはけいじばんにはりだされた 
\\	掲示板には「危険!1万ボルト」と書いてある。	"けいじばんには「きけん!1まんボルト」とかいてある 
\\	彼女はノートに何かを書き込みました。	かのじょはノートになにかをかきこみました 
\\	先ず、この書類に必要事項を書き込んでください。	まず、このしょるいにひつようじこうをかきこんでください 
\\	あなたは小切手に書き込む時に、日付を間違えた。	あなたはこぎってにかきこむときに、ひづけはまちがえだ 
\\	私の父は非常に年寄りなので耳が遠い。	わたしのちちはひじょうにとしよりなのでみみがとおい 
\\	祖母は耳が遠い、つまり、耳が少し不自由なのだ。	そぼはみみがとおい、つまり、みみがふじゆうなのだ 
\\	彼は片方の目が不自由だ。	かれはかたほうのめがふじゆうだ 
\\	その方法は両方とも危険だ。	そのほうほうはりょうほうともきけんだ 
\\	典型的な英国の食事は何ですか。	てんけいてきなえいこくのしょくじはなんですか 
\\	その仕事は最小限10日はかかるだろう。	そのしごとはさいしょうげんとおかはかかるだろう 
\\	別の方法を考え出そう。	べつのほうほうをかんがえだそう 
\\	橋は強い流れに耐えられず崩壊した。	はしはつよいながれにたえられずほうかいした 
\\	一万円札、崩してくれますか。	いちまんえんさつ、くずしてくれますか 
\\	家族もいっしょに連れてきてもいいですか。	かぞくもいっしょにつれてきてもいいですか 
\\	いつかそのうちに動物園に連れてあげるよ。	いつかそのうちにどうぶつえんにつれてあげるよ 
\\	助けてくれたお返しに、彼を夕食に連れて行った。	たすけてくれたおかえしに、かれをゆうしょくにつれていった 
\\	彼女は親切にも道を案内してくれた。	かのじょはしんせつにもみちをあんないしてくれた 
\\	彼女はそれまでニューヨークを見学したことがなかったので、彼女に案内して回りましょうと申し出た。	かのじょはおれまでニューヨークをかんがくしたことなかったので、かのじょにあんないしてまわりましょうともうしでた 
\\	その申し出を断った。	そのもうしでをことわった 
\\	不器用なその男は彼女の並外れた才能を羨ましく思った。	ぶきようなそのおとこはかのじょのなみはずれたさいのうをうらやましくおもった 
\\	買い物客たちが歩道を急いでいた。	かいものきゃくたちがほどうをいそいでいた 
\\	見知らぬ男が歩道を行ったり来たりしていた。	みしらぬおとこがほどうをいったりきたりしている 
\\	親友になるのに、期間は関係ないと思う。	しんゆうになるのに、きかんはかんけいないとおもう 
\\	海と魚の関係は空と鳥の関係と同じだ。	うみとさかなのかんけいはそらととりのかんけいとおなじだ 
\\	彼はその事件と関係がないと公言した。	かれはそのじけんとかんけいがないとこうげんした 
\\	私は思いもよらぬ結果に驚いた。	わたしはおもいもよらぬけっかにおどろいた 
\\	女の子のお喋りは止められない。	おんなのこのおしゃべりはとめられない 
\\	口いっぱいに食物をいれたままで喋ってはいけません。	くちいっぱいにたべものをいれたままでしゃべってはいけません 
\\	彼は色々な種類の人と接触する。	かれはいろいろなしゅるいのひととせっしょくする 
\\	こういう種類の絵は私には興味がない。	こういうしゅるいのえはわたしはきょうみがない 
\\	こういう実行不可能な提案には往生する。	こういうじっこうふかのうなていあんにはおうじょうする 
\\	彼らはその計画を実行した。	かれらはそのけいかくをじっこうした 
\\	看護婦が彼の体温を計った。	かんごふがはれのたいおんをはかった 
\\	浴室の計りで体重を計った。	よくしつのはかりでたいじゅうをはかった 
\\	私は犯人と疑われた。	わたしははんにんとうたがわれた 
\\	犯人は足跡を残していた。	はんにんはあしあとをのこしていた 
\\	その双子はまったく似ている。	そのふたごはまったくにている 
\\	彼ら双子だが共通の興味がほとんどない。	かれらふたごだがきょうつうのきょうみがほとんどない 
\\	当時の気候は英国と非常に似ている。	とうじのきこうはえいこくとひじょうににている 
\\	私はあなたの妻だから、あなたに喜んでもらうためにきれいな身なりをしていたい。	わたしはあなたのつまだから、あなたによろこんでもらうためにきれいなみなりをしていたい 
\\	彼の身なりは紳士だが、言葉や行いはいなか者だ。	かれのみなりはしんしだが、ことばやおこないはいなかものだ 
\\	紳士なら誰もそんな言葉は使わないでしょう。	しんしならだれもそんなことばはつかわないでしょう 
\\	殺人や強盗は犯罪行為である。	さつじんやごうとうははんざいこういである 
\\	暴力犯罪は郊外にも広がった。	ぼうりょくはんざいはこうがいにもひろがった 
\\	あなたが去る前には火を確実に消しなさい。	あなたがさるまえにはひをかくじつにけしなさい 
\\	兎は亀に追い越された。	うさぎはかめにおいこされた 
\\	彼には真実を話す勇気がない。	かれにはしんじつをはなすゆうきがない 
\\	彼女は勇気を持って行動した。	かのじょはゆうきをもってこうどうした 
\\	約束を破る人は信用されない。	やくそくをやぶるひとはしんようされない 
\\	このビデオは正しく機能しない。	このビデオはただしくきのうしない 
\\	こんな余計な機能なんでつけたんだろう。	こんなよけいなきのうなんでつけたんだろう 
\\	私が黙っていたので彼女は余計に腹を立てた。	わたしはだまっていたのでかのじょはよけいにはらをたてた 
\\	嘘吐きだと言われたのに腹を立てていた。	うそつきだといわれたのにはらを立てていた 
\\	水族館へ行く道を教えてくれますか。	すいぞくかんへいくみちをおしえてくれますか 
\\	彼は多少編集の知識がある。	かれはたしょうへんしゅうのちしきがある 
\\	うっかりして本を持ってくるのを忘れた。	うっかりしてほんをまってくるのをわすれた 
\\	私たちは彼の勇気にとても感心した。	わたしたちはかれのゆうきにとてもかんしんした 
\\	彼女の事で私が一番感心するのは無邪気な事です。	かのじょのことでわたしがいちいばんかんしんするはむじゃきなことです 
\\	彼女が会いにくるだろうと思ったのは彼の勘違いだった。	かのじょがあいにくるだろうとおもったのはかれのかんちがいだった 
\\	彼は目を閉じて自分の仕事のことを考えていた。	かれはめをとじてじぶんのしごとのことをかんがえていた 
\\	「僕は、僕の願い事について考えていたのだ」小さい黒いウサギが言いました。	"「ぼくは、ぼくのねがいことについてかんがえていたのだ」ちいさいくろいウサギがいいました 
\\	その船は昨日赤道を越えた。	そのふねはきのうせきどうをこえた 
\\	エクアドルのキトは赤道のすぐ南にある。	エクアドルのキトはせきどうのすぐみなみにある 
\\	彼が言ったことの真意が徐々にわかり始めた。	かれがいったことのしんいがじょじょにわかりはじめた 
\\	私達はすぐに晴れるかどうか疑わしいと思った。	わたしたちはすぐにはれるかどうかうたがわしいとおもった 
\\	この資料が信頼できるかどうか疑わしい。	このしりょうがしんらいできるかどうかうたがわしい 
\\	厄介者が、模範市民になることはあまりない。	やっかいものが、もはんしみんになることはあまりない 
\\	全市民が出てきて彼を歓迎した。	ぜんしみんがでてきてかれをかんげいした 
\\	彼は正直の模範だ。	かれはしょうじきのもはんだ 
\\	その竹はしなったが折れなかった。	そのたけはしなったがおれなかった 
\\	私たちは竹かごの作り方を教わった。	わたしたちはたけかごのつくりかたをおそわった 
\\	この画家は美しい絵画を創作する。	このがかはうつくしいかいがをそうさくする 
\\	その画家は独特なスタイルを持っている。	そのがかはどくとくなスタイルをもっている 
\\	その画家は勉強するためにパリへ行った。	そのがかはべんきょうするためにパリへいった 
\\	私はその画家に肖像画を描いてもらった。	わたしはそのがかにしょうぞうがをかいてもらった 
\\	独特のアイディアのおかげで、彼は高い収入を得た。	どくとくのアイディアのおかげで、かれはたかいしゅうにゅうをえた 
\\	私は卒業してから初めて勉強の重要さが解った。	わたしはそつぎょうしたからはじめてべんきょうのじゅうようさがわかった 
\\	その外科医はその患者を手術した。	そのげかいはそのかんじゃをしゅじゅつした 
\\	その手術には全く危険はありません。	そのしゅじゅつはまったくきけんはありません 
\\	患者の体内で再び脈打ち始める。	かんじゃのたいないでふたたびみゃくうちはじめる 
\\	私の患者の大半は郊外から来ています。	わたしのかんじゃのたいはんはこうがいからきています 
\\	その日、先生は50人以上の患者を診察した。	そのにち、せんせいは50にんいじょうのかんじゃをしんさつした 
\\	私は自分の財産を弁護士に委ねた。	わたしはじぶんのざいさんをべんごしにゆだねた 
\\	私は何かが足に触れるのを感じた。	わたしはなにかあしにふれるのをかんじた 
\\	私の持ち物に触れないでください。	わたしのもちものにふれないでください 
\\	彼女は去年初めて日本文化に触れた。	かのじょはきょねんはじめてにほんぶんかにふれた 
\\	東京に住んでいる外国のビジネスマンたちは、輸入欧米食料品の高価格にしばしば文句を言う。	とうきょうにすんでいるがいこくのビジネスマンたちは、ゆにゅうおうべいしょくりょうひんのたかかかくにしばしばもんくをいう 
\\	人参はその食料品店で売っています。	にんじんはそのしょくりょうひんてんでうっています 
\\	彼は誰にも見られずにこっそり家を出た。	かれはだれにもみられずにこっそりうちをでた 
\\	皆が私のうっかりした間違いを責めるのです。	みんながわたしのうっかりしたまちがいをせめるのです 
\\	私達は両親を愛情がないと責めた。	わたしたちはりょうしんをあいじょうがないとせめた 
\\	彼女は彼が車を盗んだと言って責めた。	かのじょはかれがくるまをぬすんだといってせめた 
\\	愛情と憎しみは正反対の感情だ。	あいじょうとにくしみはせいはんたいのかんじょうだ 
\\	彼の表情は憎しみに近いものだった。	かれのひょうじょうはにくしみにちかいものだった 
\\	ピン1本でさえ盗むのは罪である。	ピン1ぽんでさえぬすむのはつみである 
\\	顕微鏡と望遠鏡の違いがわかりますか。	けんびきょうとぼうえんきょうのちがいがわかりますか 
\\	主婦は節約に努めるべきである。	しゅふはせつやくにつとめるべきである 
\\	中国語で10まで数えられますか。	ちゅごくごで10までかぞえられますか 
\\	空には星がたくさんでていてとても数え切れない。	そらにはほしがたくさんてていてもかぞえきれない 
\\	彼らは目標を達成した。	かれらはもくひょうをたっせいした 
\\	アメリカのお年寄りは比較的裕福である。	アメリカのおとしよりはひかくてきゆうふくである 
\\	平日は比較的混んでいないようです。	へいじつはひかくてきこんでいないようです 
\\	画家はその婦人の魅力をうまく捕らえた。	がかはそのふじんのみりょくをうまくとらえた 
\\	ケイトはお姉さんと同様に魅力的です。	ケイトはおねえさんとどうようにみりょくてきです 
\\	彼の額から汗がどっと流れ出ていた。	かれのひたいからあせがどっとながれでていた 
\\	田中さんの教えるのは歴史の流れに逆らう。	たなかさんのおしえるのはれきしのながれにさからう 
\\	講師はジョークを言って演説を終えた。	こうしはジョークをいってえんぜつをおえた 
\\	皆様がお揃いになったので、送別会を始められます。	みんなさまがおそろいになったので、そうべつかいをはじめられます 
\\	その答えは完全に間違っている。	そのこたえはかんぜんにまちがっている 
\\	その切り傷は2、3に治したら完全に治るでしょう。	そのきりきずは2、3になおしたらかんぜんになおるでしょう 
\\	デビーに挨拶したが彼女は完全に私の事を無視した。	でビーにあいさつしたがかのじょはかんぜんにわたしのことをむしした 
\\	彼は厚かましくも私の助言を無視した。	かれはあつかましくもわたしのじょげんをむしした 
\\	彼は貯金の目標を300万に決めた。	かれはちょきんのもくひょうを300まんにきめた 
\\	彼女は祖母からバスケットを編む技術を学んだ。	かのじょはそぼかれバスケットをあむぎじゅつをまなんだ 
\\	良子は編み物に大変熱心です。	よしこはあみものにたいへんねっしんです 
\\	昔々ある所に美しいお姫様が住んでいました。	むかしむかしあるところにうつくしいがすんでいました 
\\	予備の枕をください。	よびのまくらをください 
\\	彼女は枕を二つ使ってねむる	かのじょはまくらをふたつつかってねむる 
\\	彼は妻を嫉妬した。	かれはつまをしっとした 
\\	彼がそんなに若く死んだのは気の毒だ。	かれはそんなにわかくしんだのはきのどくだ 
\\	この基本単語は必ず暗記しなさい。	このきほんたんごはかならずあんきしなさい 
\\	基本ルールを学んでしまえば、そのゲームは簡単です。	きほんルールをまなんでしまえば、そのゲームはかんたんです 
\\	中国語で10まで数えられますか	ちゅうごくごで10までかぞえられますか 
\\	インコを飼うために必要な物を揃えましょう。	インコをかうためにひつようなものをそろえましょう 
\\	私は彼のお母さんが彼を叱るのを見た。	わたしはかれのおかあさんがかれをしかるのみた 
\\	何度も私は彼を叱った。	なんどもわたしはかれをしかった 
\\	彼は私を、怠慢だと叱った。	かれはわたしを、たいまんだとしかった 
\\	叱られた少年はすすり泣いていた。	しかられたしょうねんはすすりないていた 
\\	彼は叱られているとき口をつぐんでいた。	かれはしかられているときくちをつぐんでいた 
\\	失敗は無知から生じることがよくある。	しっぱいはむちからしょうじることがよくある 
\\	社会における変化は個人から生じる。	しゃかいにおけるへんかはこじんからしょうじる 
\\	彼は社会主義から転向した。	かれはしゃかいしゅぎからてんこうした 
\\	彼は自分の無知を恥じていた。	かれはじぶんのむちをはじていた 
\\	恐怖は常に無知から生まれる。	きょうふはつねにむちからうまれる 
\\	個人はそれぞれ異なっている。	こじんはそれぞれことなっている 
\\	その制度に対する個人的な敵意はない。	そのせいどにたいするこじんてきなてきいはない 
\\	彼は私に敵意のある態度をとった。	かれはわたしにてきいのあるたいどをとった 
\\	その学生は社会学を勉強している。	そのがくせいはしゃかいがくをべんきょうしている 
\\	民主主義国家に住む利点の一つは、自分が考えていることを何でも言うことが許されることである。	みんしゅしゅぎこっかにすむりてんのひとつは、じぶんがかんがえていることをなんでもいうことがゆるされることである 
\\	そんな事しても利点がない。	そんなことしてもりてんがない 
\\	大都市に住むことには多くの利点がある。	だいとしにすむことにはおおくのりてんがある 
\\	ギリシャの哲学者達は民主主義を高く評価した。	ギリシャのてつがくしゃたちはみんしゅしゅぎをたかくひょうかした 
\\	私は彼を高く評価している。	わたしはかれをたかくひょうかしている 
\\	私はよい音楽を正当に評価する。	わたしはよいおんがくをせいとうにひょうかする 
\\	トムは紅茶の中へ砂糖を入れすぎます。	トムはこうちゃのなかへさとうをいれすぎます 
\\	男は刑事に銃を向けた。	おとこはけいじにじゅうをむけた 
\\	彼はその絵に目を向けた。	かれはそのえにめをむけた 
\\	彼は仰向けに寝転んだ。	かれはあおむけにねころんだ 
\\	彼はその仕事で私が面接した最初の人だった。	かれはそのしごとでわたしがめんせつしたさいしょのひとだった 
\\	私は試験を受ける前に面接を受けなければならない。	わたしはしけんをうけるまえにめんせつをうけなければならない 
\\	日光で看板の文字があせた。	にっこうでかんばんのもじがあせた 
\\	私は彼に五万円の借金がある。	わたしはかれにごまんえんのしゃっきんがある 
\\	借金を返さなければならない。	しゃっきんをかえさなければならない 
\\	私の皮膚は日焼けしやすい。	わたしのひふはひやけしやすい 
\\	例えば、トカゲは皮膚の色を変え、回りの木や葉っぱにとけ込むことが出来る。	たとえば、トカゲはひふのいろをかえ、まわりのきやはっぱにとけこむことができる 
\\	彼の黒のコートが暗闇に溶け込んで見えなくなってしまった。	かれのくろいコートがくらやみにとけこんでみえなくなってしまった 
\\	私の我慢も限界だ。	わたしのがまんもげんかいだ 
\\	自分の限界を知る事は重要である。	じぶんのげんかいをしることはじゅうようである 
\\	限界まで泳ぎ続けろ。	げんかいまでおよぎつづける 
\\	一生懸命し続けることはいいことだ。	いっしょうけんめいしつづけることはいいことだ 
\\	彼はその計画を続ける決心をした。	かれはそのけいかくをつづけるけっしんした 
\\	彼はいつまでも彼女を愛し続けるだろう。	かれはいつまでもかのじょをあいしつづけるだろう 
\\	成功するまで続けるようにしなければなりません。	せいこうするまでつづけるようにしなければなりません 
\\	これ以上この暑さには我慢できない。	これいじょうこのあつさにはがまんできない 
\\	私はあの騒音にはもう我慢できない。	わたしはあのそうおんにはもうがまんできない 
\\	私はもう彼の短気には我慢できない。	わたしはもうかれのたんきにはがまんできない 
\\	私は彼の傲慢な態度が我慢できない。	わたしはかれのごうまんのたいどががまんできない 
\\	彼はいつも無表情だ。	かれはいつもむひょうじょうだ 
\\	彼は私の驚きの表情を見てとりました。	かれはわたしのおどろきのひょうじょうをみてとりました 
\\	彼の表情から判断すると、彼は機嫌が悪い。	かれのひょうじょうからはんだんすると、かれはきげんがわるい 
\\	日本は驚きでいっぱいだ。	にほんはおどろきでいっぱいだ 
\\	ニュースを聞いて、とても驚きました。	ニュースをきいて、とてもおどろきました 
\\	彼は相手チームに入った。	かれはあいてチームにはいった 
\\	老人には話し相手が必要だ。	ろうじんにははなしあいてがひつようだ 
\\	恋する相手を捜しています。	こいするあいてをさがしている 
\\	外見で人を判断するな。	そとみでひとをはんだんするな 
\\	訛りから判断して彼は大阪の人に違いない。	なまりからはんだんしてかれはおおさかのひとにちがいない 
\\	間違ったに違いない。	まちがったにちがいない 
\\	彼は正直な男に違いない。	かれはしょうじきなおとこにちがいない 
\\	冗談を言ってるに違いない。	じょうだんをいってるにちがいない 
\\	彼女は嘗て本当に美しかったに違いない。	かのじょはかつてほんとうにうつくしかったにちがいない 
\\	彼の訛りから考えれば、九州出身に違いない。	かれのなまりからかんがえれば、きゅうしゅうしゅっしんにちがいない 
\\	われわれは皆汗でびしょ濡れになった。	われわれはみんなあせでびしょぬれになった 
\\	私の髪はまだ洗ったばかりで濡れていた。	わたしのかみはまだあらったばかりでぬれていた 
\\	私は「おしぼり」という濡れたタオルを差し出す習慣が好きです。	わたしは「おしぼり」というぬれたタオルをさしだすしゅうかんがすきです 
\\	普通電車は急行ほど快適ではない。	ふつうでんしゃはきゅうこうほどかいてきではない 
\\	私たちの職場はエアコンがあってとても快適だ。	わたしたちのしょくばはエアコンがあってとてもかいてきだ 
\\	私たちはそのコンピューターを職場へ運んだ。	わたしたちはそのコンピューターをしょくばへはこんだ 
\\	あなたには関係ない。	あなたにはかんけいない 
\\	その会議室は現在使用中です。	そのかいぎしつはげんざいしようちゅうです 
\\	彼は絶対的な権力を持っている。	かれはぜったいなけんりょくをもっている 
\\	昨日給料をもらった。	きのうきゅうりょうをもらった 
\\	彼の給料は7年前の2倍です。	かれのきゅうりょうは7ねんまえの2ばいです 
\\	彼女は私の顎をひっぱたいた。	かのじょは私のあごをひっぱたいた 
\\	肉を冷蔵庫に入れなさい。さもないと腐るよ。	にくをれいぞうこにいれなさい。さもないとくさるよ 
\\	あの学生はとても積極的だ。	あのがくせいはとてもせっきょくてきだ 
\\	牛乳が腐った。	ぎゅにゅうがくさった 
\\	暑いとミルクが腐る。	あついとミルクがくさる 
\\	そのりんごの半分は腐っていた。	そのりんごのはんぶんはくさっていた 
\\	私はダンスに決して飽きることはありません。	わたしはダンスにけっしてあきることはありません 
\\	彼はすぐその仕事に飽きるだろう。	かれはすぐそのしごとにあきるだろう 
\\	もうこの番組には飽きた。	もうこのばんぐみにはあきた 
\\	彼は読書に飽きた。	かれはどくしょにあきた 
\\	私はテレビを見るのに飽きています。	わたしはテレビをみるのにあきています 
\\	変化のない毎日の生活に飽きていた。	へんかのないまいにちのせいかつにあきていた 
\\	僕はレストランの食事には飽き飽きしている。	ぼくはレストランのしょくじにはあきあきしている 
\\	退屈な仕事に飽き飽きだから、何か新しいことを始めなければ。	たいくつなしごとにあきあきだから、なにかあたらしいことをはじめなければ 
\\	首相は明日放送に出る。	しゅしょうはあしたほうそうにでる 
\\	私の人生の目標は首相になることだ。	わたしのじんせいのもくひょうはしゅしょうになることだ 
\\	大学に入っていることが私の人生の目標ではない。	だいがくにはいっていることがわたしのじんせいのもくひょうではない 
\\	彼は小説家兼画家である。	かれはしょうせつかけんがかである 
\\	彼は政治家というよりは小説家である。	かれはせいじかというよりはしょうせつかである 
\\	この国の経済状態は悪い。	このくにのけいざいじょうたいはわるい 
\\	その家はあまりいい状態ではない。	そのいえはあまりいいじょうたいではない 
\\	霧はロンドンで見慣れた光景だった。	きりはロンドンでみなれたこうけいだった 
\\	目覚めると見慣れない部屋にいた。	めざめるとみなれないへやにいた 
\\	愛こそが世界を支配する。	あいこそがせかいをしはいする 
\\	誰がこの国を支配していたか。	だれかこのくにをしはいしていたか 
\\	静けさが森を支配していた。	しずけさがもりをしはいしていた 
\\	嵐の前の静けさ。	あらしのまえのしずけさ 
\\	彼は静けさを好む。	かれはしずけさをこのむ 
\\	静けさの中、ナンシーは突然叫び声をあげた。	しずけさのなか、ナンシーはとつぜんさけびごえをあげた 
\\	この冬は嵐が多かった。	このふつはあらしがおおかった 
\\	彼は嵐のため欠席した。	かれはあらしのためけっせきした 
\\	10年振りの最悪の嵐だ。	10ねんぶりのさいあくのあらしだ 
\\	私たちは嵐のために遅れた。	わたしたちはあらしのためにおくれた 
\\	その王様は何年もその国を支配した。	そのおうさまはなんねんもそのくにをしはいした 
\\	象は鼻が長い。	ぞうははながながい 
\\	象は巨大な動物である。	ぞうはきょだいなどうぶつである 
\\	象は猟師に殺された。	ぞうはりょうしにころされた 
\\	遠くから見れば、その山は象のようだ。	とおくからみれば、そのやまはぞうのようだ 
\\	その小さい男の子は目を大きく見開いて巨大な象を見た。	そのちいさいおとこのこはめをおおきくみひらいてきょだいなぞうをみた 
\\	庭の手入れでもしようかな。	にわのていれでもしようかな 
\\	この家は手入れをしなければならない。	このいえはていれをしなければならない 
\\	その値段は税金を含みます。	そのねだんはぜいきんをふくみます 
\\	タバコには重い税金がかかっている。	タバコにはおもいぜいきんがかかっている 
\\	どんな品物に税金がかかりますか。	どんなしなものにぜいきんがかかりますか 
\\	空を飛んでいる鳥を撃つ事は難しい。	そらをとんでいるとりをうつことはむずかしい 
\\	彼は腕を3度撃たれた。	かれはうでを3どうたれた 
\\	その男は銃で3羽の鳥を撃った。	そのおとこはじゅうで3わのとりをうった 
\\	暇潰しにテレビゲームをしよう。	ひまつぶしにテレビゲームをしよう 
\\	汽車旅行には読書がよい暇潰しになる。	きしゃりょこうにどくしょがよいひまつぶしになる 
\\	それを巣に戻しなさい。	それをすにもどしなさい 
\\	彼女は巣の中の鳥を注意して見つめた。	かのじょはすのなかのとりをちゅういしてみつめた 
\\	飛んだり巣を造ったりするのは鳥の本性です。	とんだりすをつくったりするのはとりのほんしょうです 
\\	私は彼と知り合いです。	わたしはかれとしりあいです 
\\	彼女は友達というより知り合いの仲です。	かのじょはともだちというよりしりあいのなかです 
\\	あなたを見つめている男性は知り合いですか。	あなたをみつめているだんせいはしりあいですか 
\\	私は上着を裏返しに着た。	わたしはうわぎをうらかえしにきた 
\\	彼は、裏切り者となった。	かれは、うらぎりものとなった 
\\	月の裏側は見えません。	つきのうらがわはみえません 
\\	弟はシャツを裏返しに着ていた。	おとうとはシャツをうらかえしにきていた 
\\	彼の家の裏手には広い庭がある。	かれのうちのうらてにはひろいにわがある 
\\	掃除機を貸してもらえますか。	そうじきをかしてもらえますか 
\\	この掃除機は非常にうるさい音がする。	このそうじきはひじょうにうるさいおとがある 
\\	彼はわざわざ家まで送ってくれた。	かれはわざわざうちまでおくってくれた 
\\	誰もわざわざそんなことをしないだろう。	だれもわざわざそんなことをしないだろう。 
\\	彼は私を見送りにわざわざ駅まで来てくれた。	かれは私をみおくりにわざわざえきまできてくれた 
\\	この雨の中、わざわざ来てもらってありがとう。	このあめのなか、わざわざきてもらってありがとう 
\\	その少年は大人になって偉人になった。	そのしょうねんはおとなになっていじんになった 
\\	グリーンは赤と調和されない。	グリーンはあかとちょうわされない 
\\	彼らは木の皮を剥いだ。	かれらはきのかわをはいだ 
\\	誰もあなたの人権を奪うことはできない。	だれもあなたのじんけんをうばうことはできない 
\\	私はかばんを奪われた。	わたしはかばんをうばわれた 
\\	誰かが彼女のお金を奪った。	だれかがかのじょのおかねをうばった 
\\	怒りが彼から理性を奪った。	いかりがかれからりせいをうばった 
\\	彼女はベストセラーの小説を早速読んだ。	かのじょはベストセラーのしょうせつをさっそくよんだ 
\\	一方は赤で、また一方は白である。	いっぽうはあかで、またいっぽうはしろである 
\\	薬局は何時まで開いていますか。	やっきょくはなんじまであいていますか 
\\	彼の母親は15年間薬局を経営している。	かれのははおやは15ねんかんやっきょくをけいえいしている 
\\	彼はホテルの経営者です。	かれはホテルのけいえいしゃです 
\\	彼は靴屋を経営している。	かれはくつやをけいえいしている 
\\	その会社は外国人が経営している。	そのかいしゃはがいこくじんがけいえいしている 
\\	わざとらしい笑い。	わざとらしいわらい 
\\	彼は花瓶をわざと壊した。	かれはかびんをわざとこわした 
\\	その少年はわざと私の足を踏んだ。	そのしょうねんはわざとわたしのあしをふんだ 
\\	彼のわざとらしい話し方が嫌いだ。	かれのわざとらしいはなしかたがきらいだ 
\\	彼女はわざとドレスを見せびらかした。	かのじょはわざととドレスをみせびらかした 
\\	彼女はわざと通りで私のことを無視した。	かのじょはわざととおりでわたしのことをむしした 
\\	ジョンは私の助言を無視した。	ジョンはわたしのじょげんをむしした 
\\	彼女は私の警告をすべて無視した。	かなじょはわたしのけいこくをすべてむしした 
\\	電話がまだ鳴っても、無視するつもりだ。	でんわがまだなっても、むしするつもりだ 
\\	彼は厚かましくも先生の助言を無視した。	かれはあつかましくもせんせいのじょげんをむしした 
\\	君の厚かましいのには呆れたよ。	きみのあつかましいのにはあきれたよ 
\\	私達は彼が現れるのを待った。	わたしたちはかれがあらわれるのをまった 
\\	彼女が現れるかどうかはわからない。	かのじょがあらわれるかどうかはわからない 
\\	地図の赤丸は学校を示す。	ちずのあかまるはがっこうをしめす 
\\	矢印は東京へ行く道を示す。	やじるしはとうきょうへいくみちをしめす 
\\	赤信号は、「止まれ」を示す。	"あかしんごうは、「とまれ」をしめす 
\\	例を一つ示してください。	れいをひとつしめしてください 
\\	友達が示した方向へ行った。	ともだちがしめしたほうこうへいった 
\\	トムはその計画に興味を示した。	トムはそのけいかくにきょうみをしめした 
\\	その男の子は何の恐怖も示さなかった。	そのおとこのこはなんのきょうふもしめさなかった 
\\	是非そうしたいですね。	ぜひそうしたいですね 
\\	君は、是非ともその本を読むべきだ。	きみは、ぜひともそのほんをよむべきだ 
\\	なるほどこの車は小さいが力強い。	なるほどこのくるまはちいさいがちからづよい 
\\	なるほど私が間違ってるかもしれない。	なるほどわたしはまちがってるかもしれない 
\\	なるほど彼は頭はいいがあまり役に立たない。	なるほどかれはあたまがいいがあまりやくにたたない 
\\	彼の助言など役に立たない。	かれのじょげんなどやくたたない 
\\	この辞書はほとんど役に立たない。	このじしょはほとんどやくにたたない 
\\	君の計画は全然役に立たない。	きみのけいかくはぜんぜんやくにたたない 
\\	是非とも外国語を勉強するようにしなさい。	ぜひともがいこくごをべんきょうするようにしなさい 
\\	警察官は彼らの住所氏名を詰問した。	けいさつかんはかれらのじゅしょしめいをきつもんした 
\\	このバスに乗れば博物館に行けます。	このバスにのればはくぶつかんにいけます 
\\	もし時間が許せば博物館を訪れたい。	もしじかんがゆるせばはくぶつかんをおとずれたい 
\\	どこを訪れる予定ですか。	どこをおとずれるよていですか 
\\	その寺は訪れる価値がある。	そのてらはおとずれるかちがある 
\\	彼らは時々私たちを訪れる。	かれらはときどきわたしたちをおとずれる 
\\	彼は10年ぶりに故郷を訪れた。	かれは10ねんぶりにこきょうをおとずれた 
\\	学校への近道だ。	がっこうへちかみちだ 
\\	成功への近道はない。	せいこうへのちかみちはない 
\\	彼らは地図で近道を調べた。	かれらはちずでちかみちをしらべた 
\\	近所に火事が起こった。	きんじょにかじがおこった 
\\	隣近所の人を夕食に招いた。	となりきんじょのひとをゆうしょくにまねいた 
\\	近所の人たちは彼をばかにした。	きんじょのひとたちはかれをばかにした 
\\	昔は近所に美術館がありました。	むかしはきんじょにびじゅつかんがありました 
\\	ラグビーは屋外競技である。	ラグビーはおくがいきょうぎである 
\\	彼らは屋外スポーツに熱心である。	かれらはおくがいスポーツにねっしんである 
\\	ナンシーは屋内競技が好きです。	なんしーはおくないきょうぎがすきです 
\\	雨がひどく降ってきた、それで私たちは屋内で遊んだ。	あめがひどくふってきた、それでわたしたちはおくないであそんだ 
\\	長崎は、私の生まれたところで、美しい港町です。	ながさきは、わたしのうまれたところ、うつくしいみなとまちです 
\\	熱湯でやけどをしました。	ねっとうでやけどをしました 
\\	彼は耳が鋭い。	かれのみみがするどい 
\\	此処に鋭い痛みがあります。	ここにするどいいたみがあります 
\\	あなたは数学が得意ですか。	あなたはすうがくがとくいですか 
\\	此処で泳ぐのは危険だ。	ここでおよぐのはきけんだ 
\\	この山を登るのは危険だ。	このやまをのぼるのはきけんだ 
\\	彼は危険な手術を受けた。	はれはきけんなしゅじゅつをうけた 
\\	医者は危険の可能性を警告する。	いしゃはきけんのかのうせいをけいこくする 
\\	彼らは船に危険を警告した。	かれらはふねにきけんをけいこくした 
\\	そういう話は苦手だ。	そういうはなしはにがてだ 
\\	彼女は科学が最も苦手だ。	かのじょはかがくがもっともにがてだ 
\\	人に会うのは苦手だ。	ひとにあうのはにがてだ 
\\	数学は彼の最も得意な科目です。	すうがくはかれのもっともとくいなかもくです 
\\	私たちは高校で多くの科目を勉強する。	わたしたちはこうこうでおおくのかもくをべんきょうする 
\\	何故このような科目を選んだの?	なぜこのようなかもくをえらんだの? 
\\	友達を選ぶ時は冷静に。	ともだちをえらぶときはれいせいに 
\\	あなたはどちらの道を選びましたか。	あなたはどちらのみちをえらびましたか 
\\	息子さんのために面白い本を選びましたか。	むすこさんのためにおもしろいほんをえらびましたか 
\\	とうとう、彼女は別の子猫を選びました。	とうとう、かのじょはべつのこねこをえらびました 
\\	私はこの辞書を自分で選んだ。	わたしはこのじしょをじぶんでえらんだ 
\\	好きなものを選んでいいよ。	すきなものをえらんでいいよ 
\\	あなたの名字はどうつづるのですか。	あなたのみょうじはどうつづるのですか 
\\	私は彼女を高く評価した。	わたしはかのじょをたかくひょうかした 
\\	この本を私は大変評価している。	このほんをわたしはたいへんひょうかしている 
\\	彼女は歌手として高く評価されていますか。	かのじょはかしゅとしてたかくひょうかされていますか 
\\	私は日本を去る決心をした。	わたしはにほんをさるけっしんした 
\\	40年が過ぎ去った。	40ねんがすぎさった 
\\	彼は2、3日前にここを去った。	かれは2,3にちまえにここをさった 
\\	彼は警官に連れ去られた。	かれはけいかんにつれさられた 
\\	綱を離すと犬は走り去った。	つなをはなすといぬがはしりさった 
\\	すぐにここを立ち去りなさい。	すぐにここをたちさりなさい 
\\	父と母は飛行場まで私たちを迎えに来てくれました。	ちちとはははひこうじょうまでわたしたちをむかえにきてくれました 
\\	どうして会議に間に会わなかったんですか。 新幹線が遅れたんです。	どうしてかいぎにまにあわなかったんですか。 しんかんせんがおくれたんです 
\\	彼女の3台の車は1台が青で残りは白だ。	かのじょの3だいのくるまは1だいがあおでのこりはしろだ 
\\	私は家賃に週に50ドルしか払えない。	わたしはやちんにしゅうに50ドルしかはらえない 
\\	一冊の本に40ドルも支払えない。	いっさつのほんに40ドルもしはらえない 
\\	その問題を更に調査しよう。	そのもんだいをさらにちょうさしよう 
\\	警察は事故の原因を調査中である。	けいさつはじこをげんいんをちょうさちゅうである 
\\	5ポンド紙幣をお持ちですか。	5ポンドしへいをおもちですか 
\\	10ドル紙幣を5枚、残りは1ドル紙幣でお願いします。	10ドルしへいを5まい、のこりは1ドルしへいでおねがいします 
\\	この分の中で
\\	という単語はどういう意味ですか。	"このぶんのなかで
\\	というたんごはどういういみですか 
\\	大阪の方言は聞き取り難い。	おおさかのほうげんはききとりにくい 
\\	彼らは南部の方言で話していた。	かれらはなんぶのほうげんではなしていた 
\\	チーズは簡単に消化しない。	チーズはかんたんにしょうかしない 
\\	ワインは消化を助ける。	ワインはしょうかをたすける 
\\	食べているもので嫌いなものはよく消化しません。	たべているものできらいなものはよくしょうかしません 
\\	誰でも幸福を求める。	だれでもこうふくをもとめる 
\\	私は助けを求める呼び声を聞いた。	わたしはたをもとめるよびこえをきいた 
\\	幸運を祈るよ。	こううんをいのるよ 
\\	幸運にも天気がよかった。	こううんにもてんきがよかった 
\\	合格したのは幸運だった。	ごうかくしたのはこううんだった 
\\	幸運にも彼女は死ななかった。	こううんにもかのじょはしななかった 
\\	彼女は私の幸運をうらやんでいる。	かのじょはわたしのこううんをうらやんでいる 
\\	7は幸運の数字だと信じられている。	7はこううんのすうじだとしんじられている 
\\	彼は聡明だと信じる。	かれはそうめいだとしんじる 
\\	見ることは信じることである。	みることはしんじることである 
\\	私は彼の言う事なら何でも信じる。	わたしはかれのいうことならなんでもしんじる 
\\	信じてくれよ。	しんじてくれよ 
\\	私たちは神の存在を信じる。	わたしたちはかみのそんざいをしんじる 
\\	最近は誰も幽霊の存在など信じない。	さいきんはだれもゆうれいのそんざいなどしんじない 
\\	ほとんどの人がその噂を信じた。	ほとんどのひとがそのうわさをしんじた 
\\	鯨は魚であると信じられていた。	くじらはさかなであるとしんじられていた 
\\	彼はその話が本当だと信じている。	かれはそのはなしがほんとうだとしんじている 
\\	私は父の助けを求めた。	わたしはちちのたすけをもとめた 
\\	鍵を求めて机の中を捜した。	かぎをもとめてつくえのなかをさがした 
\\	私たちは平和を求めている。	わたしたちはへいわをもとめている 
\\	彼は求めなければ話さない。	かれはもとめなければはなさない 
\\	彼女は私の助言を求めている。	かのじょはわたしのじょげんをもとめている 
\\	私は自分の通帳を見たが、幸運にも50ドル余分に入っていた。	わたしはじぶんのつうちょうをみたがこううんにも50ドルよぶんにはいっている 
\\	余分な金はない。	よぶんなかねはない 
\\	いつも冬には余分な毛布が必要だ。	いつもふゆにはよぶんなもうふがひつようだ 
\\	ひょっとして余分な英語の辞書をもってませんか。	ひょっとしてよぶんなえいごのじしょをもってませんか 
\\	じゃ、諦めるんだね。	じゃ、あきらめるんだね 
\\	失敗しても諦めるな。	しっぱいしてもあきらめるな 
\\	才能がなくても諦めてはいけない。	さいのうがなくてもあきらめてはいけない 
\\	雨は激しく降った。	あめははげしくふった 
\\	我々は激しく戦った。	われわれははげしくたたかった 
\\	激しい風が吹いていた。	はげしいかぜがふいていた 
\\	彼らは激しくののしり始めた。	かれらははげしくののしりはじめた 
\\	スミス氏は定期券を持っていくのを忘れた。	スミスしはていきけんをもっていきのをわすれた 
\\	私の定期券は3月31日で期限が切れる。	わたしのていきけんはさんがつさんじゅういちにちできげんがきれる 
\\	彼女が私に微笑んだ。	かのじょがわたしにほほえんだ 
\\	彼女の顔に突然微笑が浮かんだ。	かのじょのかおにとつぜんびしょうがうかんだ 
\\	速達で送ってください。	そくたつでおくってください 
\\	早起きは健康によい。	はやおきはけんこうによい 
\\	早起きは三文の得。	はやおきはさんもんのとく 
\\	食べ過ぎは健康に悪い。	たべすぎるけんこうにわるい 
\\	健康はお金よりも大切だ。	けんこうはおかねよりもたいせつだ 
\\	新鮮な果物は健康に良い。	しんせんなくだものはけんこうによい 
\\	健康には新鮮な空気がいる。	けんこうにはしんせんなくうきがいる 
\\	尿量がとても多いです。	にょうりょうがとてもおおいです 
\\	突然ですが離婚しました。	とつぜんですがりこんしました 
\\	合計で幾らですか。	ごうけいでいくらですか 
\\	彼の借金は合計十万円になる。	かれのしゃっきんはごうけいじゅうまんえんになる 
\\	決定はあなたに任せる。	けっていはあなたにまかせる 
\\	私はジョンに私の車を任せることはできない。	わたしはジョンにわたしのくるまをまかせることができない 
\\	その作家は世界的に有名である。	そのさっかはせかいてきにゆうめいである 
\\	我々はその問題を世界的視野で見なければならない。	われわれはそのもんだいをせかいてきしやでみなければならない 
\\	世界的な危機がすぐそこまで迫っている。	せかいてきなききがすぐそこまでせまっている 
\\	人々が危機は去ったと言っています。	ひとびとがききはさったといっています 
\\	彼らは勇敢にその危機に立ち向かった。	かれらはゆうかんにそのききにたちむかった 
\\	勇気を持って逆境に立ち向かう。	ゆうきをもってぎゃっきょうにたちむかう 
\\	ベロベロの酔っぱらっちゃった。	べろべろのよっぱらっちゃった 
\\	この薬は頭痛に効く。	このくすりはずつうにきく 
\\	その薬はてきめんに効く。	そのくすりはてきめんにきく 
\\	ブレーキが効かなかった。	ブレーキがきかなかった 
\\	風邪に効く薬はありますか。	かぜにきくくすりはありますか 
\\	空が怪しい。雨が降るかな。	そらがあやしい。あめがふるかな 
\\	彼が時間どおりに来るかどうかは私は怪しいと思う。	かれがじかんどおりにくるかどうかはわたしはあやしいとおもう 
\\	タクシーを拾うのに苦労した。	タクシーをひろうのにくろうした 
\\	彼は灰皿を拾い上げた。	かれははいざらをひろいあげた 
\\	駅前でタクシーを拾った。	えきまえでタクシーをひろった 
\\	床からペンを拾って下さい。	ゆかからぺんをひろってください 
\\	彼女は身をかがめて小石を拾い上げた。	かのじょはみをかがめてこいしをひろいあげた 
\\	皆が尊敬しています。	みんながそんけいしています 
\\	尊敬する人はいますか。	そんけいするひとはいますか 
\\	勇者のみが尊敬に値する。	ゆうしゃのみがそんけいにあたいする 
\\	彼は両親を尊敬していない。	かれはりょうしんをそんけいしていない 
\\	彼は東洋芸術のかなりの専門家だ。	かれはとうようげいじゅつのかなりのせんもんかだ 
\\	彼は東洋の神秘に興味を持っていた。	かれはとうようのしんぴにきょうみをもっていた 
\\	彼は東洋の国々の文化を研究していた。	かれはとうようのくにぐにのぶんかをけんきゅうしていた 
\\	彼は苦労してこの歌を書いた。	かれはくろうしてこのうたをかいた 
\\	彼はその問題を解くのに苦労した。	かれはそのもんだいをとくのにくろうした 
\\	金持ちも貧しい人と同様に苦労がある。	かねもちもまずしいひととどうようにくろうがある 
\\	宇宙は神秘に満ちている。	うちゅうはしんぴにみちている 
\\	その夢は私には神秘だった。	そのゆめはわたしにはしんぴだった 
\\	この神秘を解いたものはいますか。	このしんぴをといたものはいますか 
\\	それは壁に神秘的な影を投げかける。	それはかべにしんぴてきなかげをなけかける 
\\	その木は長い影を投げかけた。	そのきはながいかげをなげかけた 
\\	私はカナヅチ同様に泳げない。	わたしはカナヅチどうようにおよげない 
\\	私たち教師も生徒と全く同様に人間だ。	わたしたちきょうしもせいととまったくどうようににんげんだ 
\\	彼らは帰宅の途中だ。	かれらはきたくのとちゅうだ 
\\	彼は意気揚々と帰宅した。	かれはいきようようときたくした 
\\	夜になったので、帰宅した。	よるになったので、きたくした 
\\	私たちが帰宅した途端に雨が降り出した。	わたしたちがきたくしたとたんにあめがふりだした 
\\	雨がやんだ途端に虹が現れた。	あめがやんだとたんににじがあらわれた 
\\	彼はそう言った途端に後悔した。	かれはそういったとたんにこうかいした 
\\	選手達は意気揚々と競技場を行進した。	せんしゅたちはいきようようときょうぎじょうをこうしんした 
\\	虹の中にいくつの色が見えますか。	にじのなかにいくつのいろがみえましゅか 
\\	彼は賢明に行動した。	かれはけんめいにこうどうした 
\\	私は黙っている方が賢明だと思った。	わたしはだまっているほうがけんめいだとおもった 
\\	彼女が早く家を出たのは賢明だった。	かのじょがはやくいえをでたのはけんめいだった 
\\	我々は慎重に行動しなければなりません。	われわれはしんちょうにこうどうしなければなりません 
\\	大切なのは言葉より行動だ。	たいせつなのはことばよりこうどうだ 
\\	歩きながら本を読んだ。	あるきながらほんをよんだ 
\\	少年はラジオを聞きながら横になっていた。	しょうねんはラジオをききながらよこになっていた 
\\	彼は陽気に口笛を吹きながら通りを歩いた。	かれはようきにくちぶえをふきながらとおりをあるいた 
\\	お茶を飲みながら話しましょう。	おちゃをのみながらはなしましょう 
\\	彼女は微笑みながら言いました。	かのじょはほほえみながらいいました 
\\	彼女は涙ながらに友達と別れた。	かのじょはなみだながらにともだちとわかれた 
\\	彼はいつも原稿を見ながら教える。	かれはいつもげんこうをみながらおしえる 
\\	彼女は涙を流しながら答えました。	かのじょはなみだをながしながらこたえました 
\\	女の子は踊りながら私のほうに来た。	おんあのこはおどりながらわたしのほうにきた 
\\	姉は怒った目で僕を睨み付けた。	あねおこっためでぼくをにらみつけた 
\\	騒いでいたのでバスの運転手はこちらを睨んだ。	さわいでいたのでバスのうんてんしゅはこちらをにらんだ 
\\	そんな凄い目で睨み付けないでください。	そんなすごいめでにらみつけないでください 
\\	彼は仕事に全力を尽くした。	かれはしごとにぜんりょくをつくした 
\\	彼は成功のために熱心に勉強する。	かれはせいこうのためにねっしんにべんきょうする 
\\	この本は彼の熱心な研究の成果である。	このほんはかれのねっしんなけんきゅうのせいかである 
\\	タイヤに空気を入れれば膨らむ。	タイヤにくうきをいれればふくらむ 
\\	彼女のポケットはくるみで膨らんでいた。	かのじょのポケットはくるみでふくらんでいた 
\\	期待に胸を膨らませていた。	きたいにむねをふくらませていた 
\\	その取っ手を右にねじると箱は開きます。	そのとってをみぎにねじるとはこはあきます 
\\	この問題は手に負えない。	このもんだいはてにおえない 
\\	その少年は手に負えなくなった。	そのしょうねんはてにおえなくなった 
\\	彼の怠けぐせは、もう手に負えない。	かれのなまけぐせは、もうてにおえない 
\\	怠けていて母に叱られた	なまけていてははにしかられる 
\\	彼の怠けぶりを許せない。	かれのなまけぶりをゆるせない 
\\	怠け者は決して合格しない。	なまけものはけっしてごうかくしない 
\\	我々は多くの困難に直面している。	われわれはおおくのこんなんにちょくめんしている 
\\	この町の主な産業は何ですか。	このまちのおもなさんぎょうはなんですか 
\\	昨年の主な出来事は何でしたか。	さくねんのおもなできごとはなんですか 
\\	この国の主な産物の1つはコーヒーだ。	このくにのおもなさんぶつのひとつはコーヒーだ 
\\	学校に行く主な理由は何ですか。	がっこうにいくおもなりゆうはなんですか 
\\	これが私の彼を嫌う理由だ。	これがわたしのかれをきらうりゆうだ 
\\	どういう理由で泣いたの?	どういうりゆうでないたの? 
\\	彼はその理由を詳しく説明した。	かれはそのりゆうをくわしくせつめいした 
\\	彼は病気と言う理由で辞職した。	かれはびょうきというりゆうでじしょくした 
\\	彼女は遅くなった理由を説明した。	かのじょはおそくなったりゆうをせつめいした 
\\	私はその金額の2倍払った。	わたしはそのきんがくの2ばいはらった 
\\	君が使った金額は全部で幾らですか。	きみがつかったきんがくはぜんぶでいくらですか 
\\	その合計金額のほかに、彼は私にまだ10ドルの借りがある。	そのごうけいきんがくのほかに、かれはわたしにまだ10ドルのかりがある 
\\	彼はひどく疲れた様子だった。	かれはひどくつかれたようすだった 
\\	彼女は関心がなさそうな様子だった。	かのじょはかんしんがなさそうなようすだった 
\\	時計を無くしても彼は気にしていない様子だった。	とけいをなくしてもかれはきにしていないようすだった 
\\	警察は、娼婦全員を一列に並べさせた。	けいさつは、しょうふぜんいんをいちれつにならべさせた 
\\	ジャックは疲れているけれども、元気そうな様子をしていた。	ジャックはつかれているけれども、げんきそうのようすをしていた 
\\	そのスーパーマーケットは月曜から土曜まで営業している。	そのスーパーマーケットはげつようからどようまでえいぎょうしている 
\\	彼は漫画を読んでばかりいる。	かれはまんがをよんでばかりいる 
\\	彼は授業中漫画本を読んでいるところを見つかった。	かれはじゅぎょうちゅうまんがほんをよんでいるところをみつかり 
\\	簡単に見つかると思う。	かんたんにみつかるとおもう 
\\	その本は図書館の歴史部門で見つかるよ。	そのほんはとしょかんのれきしぶもんでみつかるよ 
\\	私は営業部門の一員です。	わたしはえいぎょうぶもんのいちいんです 
\\	母はサラダに塩を加えるのを忘れた。	はははサラダにしおをくわえるのをわすれた 
\\	レモンを加えると、それはすっぱくなるだろう。	レモンをくわえると、それはすっぱくなるだろう 
\\	私の言ったことに何か付け加えることがありませんか。	わたしのいったことになにかつけくわえることがありませんか 
\\	スープに塩を加えよう。	スープにしおをくわえよう 
\\	塩を加えたら味が大いに良くなった。	しおをくわえたらあじがおおきいよくなった 
\\	彼女は小説や詩に加えてエッセイも書く。	かのじょはしょうせつやしにくわえてエッセイもかく 
\\	意識がありません。	いしきがありません 
\\	意識を失いました。	いしきをうしないました 
\\	お母さんは無意識に椅子をつかんだ。	おかあさんはむいしきにいすをつかんだ 
\\	彼は鋭い痛みを意識していた。	かれはするどいいたみをいしきしていた 
\\	母は事故の現場を見て意識を失った。	はははじこのげんばをみていしきをうしなった 
\\	それを盗んでいる現場を捕らえられた。	それをぬすんでいるげんばをとらえられた 
\\	昨日私は自転車を盗まれた。	さくやわたしはじてんしゃをぬすまれた 
\\	空腹が彼を盗みに駆り立てた。	くうふくがかれをぬすみにかりたてた 
\\	彼らは囚人の手足を縛った。	かれらはしゅうじんのてあしをしばった 
\\	彼らは彼女の両足を縛りつけた。	かれらはかのじょのりょうそくをしばりつけた 
\\	2人は婚約を発表した。	ふたりはこんやくをはっぴょうした 
\\	台風が九州に接近していると発表された。	たいふうがきゅうしゅうにせっきんしているとはっぴょうされた 
\\	私達は彼の発言に注目した。	わたしたちはかれのはつげんにちゅうもくした 
\\	その少年は注目されたくて髪を染めた。	そのしょうねんはちゅうもくされたくてかみをそめた 
\\	彼女は何年も前から髪を黒く染めています。	かのじょはなんねんもまえからかみをくろくそめている 
\\	私は何年もこの店と取り引きがある。	わたしはなんねんもこのみせととりひきがある 
\\	日本はアメリカと多くの取引をしている。	にほんはアメリカとおおくのとりひきをしている 
\\	あなたの発言は私たちの議論には的外れである。	あなとのはつげんはわたしたちのぎろんにまとはずれである 
\\	あなたもすぐ騒音には慣れるでしょう。	あなたもすぐそうおんになれるでしょう 
\\	子供は新しい環境に慣れるのが早い。	こどもはあたらしいかんきょうになれるのがはやい 
\\	私の姉は料理に慣れていない。	わたしのあねはりょうりになれていない 
\\	日本食にはもう慣れましたか。	にほんしょくにはもうなれましたか 
\\	強盗が彼の家に押し入った。	ごうとうがかれのいえにおしいった 
\\	その強盗は罰せられることを免れた。	そのごうとうはばせられることをまぬかれる 
\\	運転していた人は幸運にも死を免れた。	うんてんしていたひとはこううんにもしをまぬかれた 
\\	英語の副詞の中には形容詞の役目をするものがある。	えいごのふくしのなかにはけいようしのやくめをするものがある 
\\	彼はこの役目を大きなチャンスと考えている。	かれはこのやくめをおおきいなチャンスとかんがえている 
\\	万一明日梅雨に入ったらどうしよう。	まんいちあしたつゆにはいったらどうしよう 
\\	今年は梅雨が早く始まると思いますか。	ことしはつゆがはやくはじまるとおもいますか 
\\	彼を犯罪者と呼んではいけない。	かれをはんざいしゃとよんではいけない 
\\	という文字は何を表していますか。	
\\	というもじはなにをあらわしていますか 
\\	壁に書かれた文字を判読しようとした。	かべにかかれたもじをはんどくしようとした 
\\	アラビア文字は読めません。	アラビアもじはよめません 
\\	雪は冬の到来を表す。	ゆきはふゆのとうらいをあらわす 
\\	「ブタ」を表す英語は何ですか。	"「ブタ」をあらわすえいごはなんですか 
\\	その時のうれしさは書き表すことができない。	そのときのうれしさはかきあらわすことができない 
\\	彼が化学の天才であることに気づいた。	かれがかがくのてんさいであることにきづいた 
\\	もっと詳しい情報が欲しい。	もっとくわしいじょうほうがほしい 
\\	彼は日本の文化に詳しい。	はれはにほんのぶんかにくわしい 
\\	専門家だけあって彼はその分野に詳しい。	せんもんかだけあってかれはそのぶんやにくわしい 
\\	彼は詳しいことはそれ以上言わなかった。	かれはくわしいことはそれいじょういわなかった 
\\	彼らは1時間3マイルの割合で歩いた。	かれらは1じかん3マイルのわりあいであるいた 
\\	彼の英語の能力は平均以上だ。	かれのえいごののうりょくはへいきんいじょうだ 
\\	日本人の平均寿命は大いに伸びた。	にほんじんのへいきんじゅみょうはおおいにのびた 
\\	髪が伸びてきた。	かみがのびてきた 
\\	彼女は背が伸びてきている。	かのじょはせがのびてきている 
\\	髪が腰まで伸びている少女を見た。	かみがこしまでのびれいるしょうじょをみた 
\\	最寄りの警察署はどこですか。	もよりのけいさつしょはどこですか 
\\	言語習得には創造力が必要だ。	げんごしゅうとくにはそうぞうりょくがひつよう 
\\	彼女は護身術の夜間講座を取った。	かのじょはごしんじゅつのやかんこうざをとった 
\\	彼女は地域の大学の夜間授業2クラスに登録した。	かのじょはちいきのだいがくのやかんじゅぎょう2クラスにとうろくした 
\\	この地域は今や立て込んできた。	このちいきはいまやたてこんできた 
\\	その地域はめったに雨が降らない。	そのちいきはめったにあめがふらない 
\\	最後まで戦い抜こう。	さいごまでたたかいぬこう 
\\	論議は最後に喧嘩になった。	ろんぎはさいごにけんかになった 
\\	最後の部分を詳しく説明してくれない?	さいごのぶぶんをくわしくせつめいしてくれない? 
\\	責任は大部分私にあった。	せきにんはだいぶぶんわたしにあった 
\\	この時計は不正確だ。	このとけいはふせいかくだ 
\\	これが原本の正確な写しだ。	これがげんぽんのせいかくなうつしだ 
\\	私の仕事には手助けがいる。	わたしのしごとにはてたすけがいる 
\\	警察は行方不明の少年を捜した。	けいさつはゆくえふめいのしょうねんをさがした 
\\	負傷者現場から運ばれて行った。	ふしょうしゃげんばからはこばれていった 
\\	彼は富の追求だけに興味を持った。	かれはとみのついきゅうだけにきょうみをもった 
\\	私は富や名声は欲しくない。	わたしはとみやめいせいはほしくない 
\\	最初メグは、家が恋しかった。	さいしょメグは、うちがこいしかった 
\\	日付は自分で書き込みなさい。	ひづけはじぶんでかきこみなさい 
\\	彼は典型的な日本の住宅に住んでいた。	かれはてんけいてきなにほんのじゅうたくにすんでいた 
\\	その建物は三年前に崩壊した。	そのたてものはさんねんまえにほうかいした 
\\	それは役に立つ情報です。	それはやくにたつじょうほうです 
\\	もっと情報を集めねばならない。	もっとじょうほうをねばならない 
\\	その情報の正確さは、疑わしい。	そのじょうほうのせいかくさは、うたがわしい 
\\	疑わしいことが一つ残っている。	うたがわしいことがひとつのこっている 
\\	彼が合格するかどうかは疑わしい。	かれがごうかくするかどうかはうたがわしい 
\\	私たちの申し出を断るとは大胆だ。	わたしたちのもしでをことわるとだいたんだ 
\\	私達は彼の大胆な企てにびっくりした。	わたしたちはかれのだいたんなくわだてにびっくりした 
\\	私達は大胆で新しい考え方をする人が必要だ。	わたしたちはだいたんであたらしいかんがえかたをするひとがひつようだ 
\\	彼は不器用な手つきではしを使っていた。	かれはぶきようなてつきではしをつかっていた 
\\	私は皿を洗う時にはきわめて不器用です。	わたしはさらをあらうときにはきわめてぶきようです 
\\	彼女は話し振りが極めて凡俗だ。	かのじょははなしぶりがきわめてぼんぞくだ 
\\	彼は責任感が強い。	かれのせきにんかんがつよい 
\\	犬は見知らぬ人の匂いをかいだ。	いぬはみしらぬひとのにおいをかいだ 
\\	その花は非常にいい匂いがする。	そのはなはひじょうにいいにおいがする 
\\	その腐った肉はひどく不快な匂いがした。	そのくさったにくはひどくふかいなにおいがした 
\\	何かが隣の部屋で燃えている匂いがしませんか。	なにかがとなりのへやでもえているにおいがしませんか 
\\	私達そこで不快な経験をした。	わたしたちそこでふかいなけいけんをした 
\\	彼は不快な甲高い声の持ち主だ。	かれはふかいなかんだかいこえのもちぬしだ 
\\	私は映画に行こうと提案した。	わたしはえいがにいこうとていあんした 
\\	大統領は新計画を提案した。	だいとうりょうはしんけいかくをていあんした 
\\	遂に彼は自分の計画を実行した。	ついたかれはじぶんのけいかくをじっこうした 
\\	ココは普通のゴリラではない。	ココはふつうのゴリラではない 
\\	稲妻は普通、雷鳴の前に光る。	いなずまはふつう、らいめいのまえにひかる 
\\	ダイヤは明るく光った。	ダイヤはあかるくひかった 
\\	月は自分では光らない。	つきはじぶんではひからない 
\\	目は光りに敏感である。	めはひかりにびんかんである 
\\	彼は強い光に目が眩んだ。	かれはつよいひかりにめがくらんだ 
\\	霧のために、交通は一時不通になっている。	きりのために、こうつうはいちじふつうになっている 
\\	日本語は韓国語と共通点がある。	にほんごはかんこくごときょうつうてんがある 
\\	金銭欲は私達全てに共通のものだと思う。	きんせんよくはわたしたちすべてにきょうつうのものだとおもう 
\\	彼は猫をいじめて喜んだ。	かれはねこをいじめてよろこんだ 
\\	彼は君に会えば喜ぶでしょう。	かれはきみにあえばよろこぶでしょう 
\\	彼女の足で書ける能力は驚くべきことだ。	かのじょのあしでかけるのうりょくはおどろくべきことだ 
\\	彼は自分の能力を確信している。	かれはじぶんののうりょくをかくしんしている 
\\	人間には話す能力がある。	じんかんにははなすのうりょくがある 
\\	彼女は夫の能力を疑っていた。	かのじょはおっとののうりょくをうたがっていた 
\\	彼は組織立てる能力が弱い。	かれはそしきたてるのうりょくがよわい 
\\	ジェーンは経験から学ぶ能力がない。	ジェーンはけいけんからまなぶのうりょくがない 
\\	彼女には能力はないけれどその代り会社に忠実だ。	かのじょにはのうりょくはないけれどそのかわりかいしゃちゅうじつだ 
\\	犬というものは忠実な動物である。	いぬというものはちゅうじつなどうぶつである 
\\	君は自分の友人には忠実でなければならない。	きみはじぶんのゆうじんにはちゅうじつでなければならない 
\\	彼は自分の成功を確信している。	かれはじぶんのせいこうをかくしんしている 
\\	私は自分が正しいと確信している。	わたしはじぶんがただしいとかくしんしている 
\\	私は暴力を憎む。	わたしはぼうりょくをにくむ 
\\	彼には生物学の知識が多少ある。	かれはせいぶつがくのちしきがたしょうある 
\\	黙っているのが一番よいと思った。	だまっているのがいちばんよいとおもった 
\\	君の無知には感心するよ。	きみのむちにかんしんするよ 
\\	私は彼の話に深い感心を受けた。	わたしはかれのはなしにふかいかんしんをうける 
\\	私はそのピアニストのすばらしい技術に感心している。	わたしはそのピアニストのすばらしいぎじゅつにかんしんしている 
\\	私は外国の学生としばしば接触した。	わたしはがいこくのがくせいとしばしばせっしょくした 
\\	この問題はしばしば起こった。	このもんだいはしばしばおこった 
\\	この花は独特の香りがする。	このはなはどくとくのかおりがある 
\\	この喫茶店は、独特のレトロさが若者に受けている。	このきっさてんは、どくとくのレトロさがわかものにうけている 
\\	それは子供じみた行いです。	それはこどもじみたおこないです 
\\	彼はばかげた行いをして笑いものになった。	かれはばかげたおこないをしてわらいものになった 
\\	フレッドは成長して外科医になった。	フレッドはせいちょうしてげかいになった 
\\	森の中は再び静かになった。	もりのなかはふたたびしずかになった 
\\	我々は再び戦争をしてはいけない。	われわれはふたたびせんそうをしてはいけない 
\\	部屋に入ると私達は再び話を始めた。	へやにはいるとわたしたちはふたたびはなしをはじめた 
\\	彼は再び大統領に選ばれると思いますか。	かれはだいとうりょうにえらばれるとおもいますか 
\\	彼は息子が本好きなので喜んでいる。	かれはむすこがほんすきなのでよろこんでいる 
\\	彼女はあなたが送った花を喜んでいました。	かのじょはあなたがおくったはなをよろこんでいました 
\\	私は彼の財産を相続するだろう。	わたしはかれのざいさんをそうぞくするだろう 
\\	私の60歳の叔母は莫大な財産を相続した。	わたしの60さいのおばはばくだいなざいさんをそうぞくした 
\\	飛行機は、楽々と離陸した。	ひこうきは、らくらくとりりくした 
\\	霧で飛行機は離陸を妨げられた。	きりでひこうきはりりくをさまたげられた 
\\	食料品の値段はすぐに下がるでしょうね。	しょくりょうひんのねだんはすぐさがるでしょうね 
\\	気温が急に下がった。	きおんがきゅうにさがった 
\\	すべての母は子に愛情を持っている。	すべてのはははこどもにあいじょうをもっている 
\\	彼らの友情は深い愛情に発展した。	かれらのゆうじょうはふかいあいじょうにはってんした 
\\	本当の友情は金銭よりも価値がある。	ほんとうのゆうじょうはきんせんよりもかちがある 
\\	彼らは私の丁寧さと友情を誤解した。	かれらはわたしのていねいさとゆうじょうをごかいした 
\\	彼は自分の商売を発展させた。	かれはじぶんのしょうばいをはってんされた 
\\	その都市の急速な発展に私たちは驚いた。	そのとしのきゅうそくなはってんにわたしたちはおどろいた 
\\	そのニュースは急速に広まった。	そのニュースはきゅうそくにひろまった 
\\	ラクダの長い列が西に向かって移動していた。	ラクダのながいれつがにしにむかっていどうしている 
\\	ラクダは、いわば砂漠の船です。	ラクダは、いわばさばくのふねです 
\\	夕闇が砂漠を包んだ。	ゆうやみがさばくをつつんだ 
\\	これはその砂漠で見つけられた動物だ。	これはそのさばくでみつけられたどうぶつだ 
\\	日本にも同様の諺がありますか。	にほんにもどうようのことわざがありますか 
\\	私はたった今キツネが道路を横切るのを見た。	わたしはたったいまきつねげどうろをよこきるのをみた 
\\	狐は穴があいた木の中へ隠れた。	きつねはあながあいたきの中へかくれた 
\\	烏は石炭のように黒い。	からすはせきたんのようにくろい 
\\	彼女がそんな事を言うとは不思議だ。	かのじょがそんなことをいうとはふしぎだ 
\\	水というものは不思議なものだ。	みずというものはふしぎなものだ 
\\	彼が試合に負けたことは不思議だ。	かれがしあいにまけたことはふしぎだ 
\\	運転手は乗客の安全に責任がある。	うんてんしゅはじょうきゃくのあんぜんにせきにんがある 
\\	川は徐々に増水した。	かれはじょじょにぞうすいした 
\\	私は最初、緊張したが、徐々に落ち着いた。	わたしはさいしょ、きんちょうしたが、じょじょにおちついた 
\\	彼は落ち着いて眠れなかった。	かれはおちついてねむれなかった 
\\	皆さん、落ち着いてください。	みなさん、おちついてください 
\\	彼女は地震の時落ち着いています。	かのじょはじしんのときおちついています 
\\	船が港に着くと人々を落ち着かせない。	ふねがみなとにつくとひとびとをおちつかせない 
\\	初めてベティに会った時、とても緊張しました。	はじめてベティにあったとき、とてもきんちょうしました 
\\	自分で服を作ったら節約になりますよ。	じぶんでふくをつくったらせつやくになりますよ 
\\	彼女は泣かないように努めた。	かのじょはなかないようにつとめた 
\\	私は笑わないように努めた。	わたしはわらわないようにつとめた 
\\	余暇を読書に利用するように努めなさい。	よかをどくしょにりようするようにつとめなさい 
\\	余暇をうまく利用するように努めなさい。	よかをうまくりようするようにつとめなさい 
\\	彼女は余暇に人形を作って過ごす。	かのじょはよかににんぎょうをつくってすごす 
\\	私は彼女を説得できなかった。	わたしはかのじょをせっとくできなかった 
\\	その議論には、全く説得力がない。	そのぎろんには、まったくせっとくりょくがない 
\\	彼を説得しても無駄だと思う。	かれをせっとくしてもむだだとおもう 
\\	私はたばこを止めるように説得された。	わたしはたばこをとめるようにせっとくされた 
\\	私は彼を説得して医者の診察を受けさせた。	わたしはかれをせっとくしていしゃのしんさつをうけさせた 
\\	長い週末を利用しよう。	ながいしゅうまつをりようしよう 
\\	彼はあらゆる機会を利用した。	かれはあらゆるきかいをりようした 
\\	あらゆる発明は必要から生じる。	あらゆるはつめいはひつようからしょうじる 
\\	早寝早起きは健康で裕福で賢くする。	はやねはやおきはけんこうでゆうふくでかしこくする 
\\	彼が賢いことを確信している。	かれがかしこいことをかくしんしている 
\\	賢い買い物は綿密な計画を必要とする。	かしこいかいものはめんみつなけいかくをひつようとする 
\\	彼女は綿密な診察を受けた。	かのじょはめんみつなしんさつをうけた 
\\	彼は演説するのに慣れている。	かれはえんぜつするのになれている 
\\	大統領は国民に向けて演説した。	だいとうりょうはこくみんにむけてえんぜつした 
\\	聴衆は彼の演説にとても感動した。	ちょうしゅうはかれのえんぜつにとてもかんどうした 
\\	聴衆は大笑いした。	ちょうしゅうはおおわらいした 
\\	聴衆のほとんどは実業家だった。	ちょうしゅうのほとんどはじつぎょうかだった 
\\	彼女の話を聞いて感動して泣いた。	かのじょのはなしをきいてかんどうしてないた 
\\	風邪が治るように彼はアスピリンを二錠飲んだ。	かぜがなおるようにかれはアスピリンをにじょうのんだ 
\\	あの人にはまだ挨拶をしていない。	あのひとにはまだあいさつをしていない 
\\	丁度欲しかった物です。	ちょうどほしがったものです 
\\	少年は厄介者扱いをされて憤慨した。	しょうねんはやっかいものあつかいをされてふんがいした 
\\	その若者は臆病者扱いをされて憤慨した。	そのわかものはおくびょうものあつかいされてふんがいした 
\\	彼は臆病者に過ぎない	かれはおくびょうものにすぎない 
\\	人は経験から学ぶ。	ひとはけいけんからまなぶ 
\\	私たちは未来のために過去を学ぶ。	わたしたちはみらいのためにかこをまなぶ 
\\	過去の失敗をくよくよ考えるな。	かこのしっぱいをくよくよかんがえるな 
\\	彼女はそこで過ごした不幸な日々をくよくよと考えていた。	かのじょはそこですごしたふこうなひびをくよくよとかんがえていた 
\\	休暇中は何もしないで日々を過ごした。	きゅうかちゅうはなにもしないでひびをすごした 
\\	犬を飼うための基本的なルールは何ですか。	いぬをかうためのきほんてきなルールはなんですか 
\\	会えば必ず喧嘩する。	あえばかならずけんかする 
\\	遅かれ早かれ私達は必ず死ぬ。	おそかれはやくれわたしたちはかならずしぬ 
\\	その老人は見かけほど意地悪くなかった。	そのろうじんはみかけほどいじわるくなかった 
\\	水泳は大変役に立つ技術である。	すいえいはたいへんやくにたつぎじゅつである 
\\	この仕事は特別な技術を必要とする。	このしごとはとくべつなぎじゅつをひつようとする 
\\	母は嘗て水泳で優勝したことがある。	はははかつてすいえいでゆうしょうしたことがある 
\\	私は彼と優勝を争った。	わたしはかれとゆうしょうあらそった 
\\	私にはその人は不正直なように思われる。	わたしにはそのひとはふしょうじきなようにおもわれた 
\\	目的は手段を正当化する。	もくてきはしゅだんをせいとうかする 
\\	あらゆる可能な手段を選びました。	あらゆるかのうなしゅだんをえらびました 
\\	最も急いで旅行する手段は飛行機だ。	もっともいそいでりょこうするしゅだんはひこうきだ 
\\	彼はそれを不正な手段でした。	かれはそれをふせいなしゅだんでした 
\\	くしゃみが出て止まりません。	くしゃみがでてとまりません 
\\	咳、くしゃみ、あくびをする時は口を手で隠しなさい。	せき、くしゃみ、あくびをするときはくちをてでかくしなさい 
\\	彼女は旅行の準備に忙しい。	かのじょはりょこうのじゅんびにいそがしい 
\\	彼は雑誌の出版準備を担当していた。	かれはざっしのしゅっぱんじゅんびをたんとうしていた 
\\	彼の新しい小説は来月出版される。	かれのあたらしいしょうせつはらいげつしゅっぱんされる 
\\	編集者と出版者がそのパーティーに出席していた。	へんしゅしゃとしゅっぱんしゃがそのパーティーにしゅっせきしていた 
\\	彼女の夫は新しい月刊雑誌を出版するつもりだ。	かのじょのおっとはあたらしいげっかんざっしをしゅっぱんするつもりだ 
\\	人々は自分自身の責任でその海岸で泳ぐ。	ひとびとはじぶんじしんのせきにんでそのかいがんでおよぐ 
\\	彼の今度の小説は自分自身の体験に基づいている。	かれのこんどのしょうせつはじぶんじしんのたいけんにもとづいている 
\\	その若者は自分自身の国についてほとんど知らない。	そのわかものはじぶんじしんのくにについてほとんどしらない 
\\	偉大な俳優の息子は自分自身の力で良い俳優になった。	いだいなはいゆうのむすこはじぶんじしんのちからでよいはいゆうになった 
\\	私には有名な俳優を父に持つ友人がいます。	わたしはゆうめいなはいゆうをちちにもつゆうじんがいます 
\\	自分の気持ちを表現できない。	じぶんのきもちをひょうげんできない 
\\	彼女の魅力は言葉では表現できない。	かのじょのみりょくはことばではひょうげんできない 
\\	語るべき時と沈黙すべき時とがある。	かたるべきときとちんもくすべきときとがある 
\\	私の奇妙な夢を解釈して下さい。	わたしのきみょうなゆめをかいしゃくしてください 
\\	この句はどう解釈したらよいのだろうか。	このくはどうかいしゃくしたらよいのだろうか 
\\	彼の話は、奇妙だが、信じられる。	かれのはなしは、きみょうだが、しんじられる 
\\	奇妙な病気が町を襲った。	きみょうなびょうきがまちをおそった 
\\	奇妙なことが彼女の誕生日に起こった。	きみょうなことがかのじょのたんじょうびにおこった 
\\	私はその知らせに当惑した。	わたしはそのしらせにとうわくした 
\\	彼は当惑して顔をしかめた。	かれはとうわくしてかおをしかめた 
\\	私は彼の奇妙な行動を目にして当惑した。	わたしはかれのきみょうなこうどうをめにしてとうわくした 
\\	彼女は子供が人前でお行儀悪かったのでとても当惑した。	かのじょはこどもがひとまえでをぎょうぎわるかったのでとてもとうわくした 
\\	行儀よくしなさい。	ぎょうぎよくしなさい 
\\	行儀の悪さは彼の良識を疑わせるものだ。	ぎょうぎのわるさはかれのりょうしきをうたがわせるものだ 
\\	いったん酔っぱらうと彼は行儀がよくない。	いったんよっぱらうとかれはぎょうぎがよくない 
\\	今片思いの人がいるのですが、 片思いの人は今忙しくて、メールが返ってきません。	いまかたおもいのひとがいるのですが、かたおもいのひとはいそがしくて、メールがかえってきません 
\\	悪天候が式を台無しにした。	あくてんこうがしきをだいなしにした 
\\	彼は働きすぎで健康を台無しにした。	かれははたらきすぎでけんこうをだいなしにした 
\\	彼女の美貌もその傷で台無しになった。	かのじょのびぼうもそのきずでだいなしになった 
\\	そのホテルは贅沢な雰囲気がある。	そのホテルはぜいたくなふんいきがある 
\\	芸術は贅沢品ではない、必需品だ。	げいじゅつはぜいたくひんではない、ひつじゅひんだ 
\\	私はいつも祖父母の元気さに驚きます。	わたしはいつもそふぼのげんきさにおどろきます 
\\	子供がどんなに速く成長するかを見るのは驚きだった。	こどもはどんなにはやくせいちょうするかをみるのはおどろきだった 
\\	彼女は成長して美人になった。	かのじょはせいちょうしてびじんになった 
\\	見かけで判断するな。	みかけではんだんするな 
\\	それは明らかに判断の誤りだった。	それはあきらかにはんだんのあやまりだった 
\\	彼は状況を判断してからすぐ行動に移った。	かれはじょうきょうをはんだんしてからすぐこうどうにうつった 
\\	様子から判断すれば、彼は兵士かもしれない。	ようすからはんだんすれば、かれはへいしかもしれない 
\\	我々の判断は正しかったと私は十分確認している。	われわれのはんだんはただしかったとわたしはじゅうぶんかくにんしている 
\\	この作文には誤りが少しある。	このさくぶんにはあやまりがすこしある 
\\	しばらくして彼は自分の誤りに気づいた。	しばらくしてかれはじぶんのあやまりにきづいた 
\\	彼女は7行に7個の誤りをしたようだ。	かのじょは7ぎょうに7このあやまりをしたようだ 
\\	誤りをおかすことを恐れてはいけません。	あやまりをおかすことをおそれてはいけません 
\\	暗くなってきた。直に雨が降るかも知れない。	くらくなってきた。じきにあめがふるかもしれない 
\\	彼はまた曖昧な事を言うかも知れない。	かれはまたあいまいなことをいうかもしれない 
\\	その法律には曖昧な点が多い。	そのほうりつにはあいまいなてんがおおい 
\\	契約には曖昧な箇所があってはならない。	けいやくにはあいまいなかしょがあってはならない 
\\	これはその分析を始めるのに良い箇所だ。	これはそのぶんせきをはじめるのいよいかしょだ 
\\	この分析では次の結果が出ている。	このぶんせきでつぎのけっかがでている 
\\	彼は財布を失くしてしまって不機嫌だった。	かれはさいふをなくしてしまってふきげんだった 
\\	彼は私同様、中国語は読めない。	かれはわたしどうよう、ちゅごくごはよめない 
\\	彼は握手を求めて私に手を差し出した。	かれはあくしゅをもとめてわたしにてをさしだした 
\\	政治犯は警察権力に反抗した。	せいじはんはけいさつけんりょくにはんこうした 
\\	あなたはアメリカで日本料理が恋しくなるでしょう。	あなたはアメリカでにほんりょうりがこいしくなるでしょう 
\\	私は彼女のような女の子に恋してしまって後悔している。	わたしはかのじょのようおんなのこにこいしてしまってこうかいしている 
\\	途中で止めたら後悔するぜ。	とちゅうでとめたらこうかいするぜ 
\\	私は若い頃怠惰であったことを後悔している。	わたしはわかいごろたいだであったことをこうかいしている 
\\	彼は彼女からその本を借りたことを後悔した。	かれはかのじょからそのほんをかりたことをこうかいした 
\\	私は彼の忠告に従わなかったことを後悔している。	わたしはかれのちゅうこくにしたがわなかったをこうかいしている 
\\	黙っていなさい。さもないと叱られるよ。	だまっていなさい。さもないとしかられるよ 
\\	黙ってろ。さもないと命はないぞ。	だまってろ。さもないといのちはないぞ 
\\	セーターを着なさい。さもないと風邪をひきますよ。	セーターをきなさい。さもないとかぜをひきますよ 
\\	コップに熱いお湯を注ぐな。さもないとひびが入るよ。	コップにあついおゆをそそぐな。さもないとひびがいるよ 
\\	その光景を見て血が凍った。	そのこうけいをみてちがこおった 
\\	その光景が心に焼き付いて離れない。	そのこうけいがこころにやきついてはなれない 
\\	私の目はその光景に吸い付けられた。	わたしのめはそのこうけいにすいつけられた 
\\	その光景を見て彼の勇気はぐらつき始めた。	そのこうけいをみてかれのゆうきはぐらつきはじめた 
\\	水は凍ると固体になる。	みずはこおるとこたいになる 
\\	私は凍った歩道で滑って転んだ。	わたしはこおったほどうですべってころんだ 
\\	中に入らないと、耳が凍りそうだ。	なかにはいらないと、みみがこおりそうだ 
\\	道路が凍っていて多くの事故が起きた。	どうろがこおっていておおくのじこがおきた 
\\	冬は道路が凍るのでお年寄りがよく転びます。	ふゆはどうろがこおるのでおとしよりがよくころびます 
\\	私は滑って階段から転げ落ちた。	わたしはすべってかいだんからころげおちた 
\\	目覚ましを7時にセットしよう。	めざましを7じにセットしよう 
\\	時計は動いていた。しかし目覚ましのベルが鳴らなかった。	とけいははたらいていた。しかしめざましのベルがならなかった 
\\	空から見ると河は巨大な蛇のように見えた。	そらからみるとかわはきょだいなへびのようにみえた 
\\	今日、多くの人々は、巨大な現代社会においては、重要なことで個人にできることは何もないという気持ちをいだいているように思われる。	きょう、おおくのひとびと、きょだいなげんざいしゃかいにおいては、じゅうようなことてこじんにできることはなにもないというきもちをいだいているようにおもわれる 
\\	その会社は急激な変化に対処できなかった。	そのかいしゃはきゅうげきなへんかにたいしょできなかった 
\\	警察は群衆に上手く対処することが出来た。	けいさつはぐんしゅうにうまくたいしょすることができた 
\\	群衆がすぐに彼の周りに集まった。	ぐんしゅうがすぐにかれのまわりにあつまった 
\\	興奮した群衆が競技場から溢れ出てきた。	こうふんしたぐんしゅうがきょうぎじょうからあふれでてきた 
\\	彼女の目から突然涙が溢れ出た。	かのじょのめからとつぜんなみだがあふれでた 
\\	群衆の中に彼を捜そうとしても無駄だ。	ぐんしゅうのなかにかれをさがそうとしてもむだだ 
\\	私達はその試合を観戦して興奮した。	わたしたちはそのしあいをかんせんしてこうふんした 
\\	彼は興奮を抑える事ができなかった。	かれはこうふんをおさえることができなかった 
\\	彼はそれ以上怒りを押さえられなかった。	かれはそれいじょうおこりをおさえられなかった 
\\	興奮するにつれて、彼はますます早口になった。	こうふんするにつれて、かれはますますはやくちになった 
\\	あんなに興奮する野球の試合は見たことがなかったよ。	あんなにこうふんするやきゅうのしあいはみたことがなかったよ 
\\	彼はとても興奮したので、彼の言う事は全く意味を成さなかった。	かれはとてもこうふんしたので、かれのいうことはまったくいみをなさなかった 
\\	少女は教室の隅ですすり泣いていた。	しょうじょはきょうしつのすみですすりないていた 
\\	彼は店の奥の暗い隅に姿を消した。	かれはみせのおくのくらいすみにすがたをけした 
\\	私はその町の隅から隅まで知っている。	わたしはそのまちのすみからすみまでしっている 
\\	会議で使った書類の片づけを頼まれた。	かいぎでつかったしょるいのかたづけをたのまれた 
\\	そろそろテーブルを片づけて始めてよい時間だよ。	そろそろテーブルをかたづけてはじめてよいじかんだよ 
\\	私達は台所を片付けなければならない。	わたしたちはだいどころをかたづけなければならない 
\\	彼らは歩道の雪をシャベルで片付けていた。	かれらはほどうのゆきをシャベルでかたづけていた 
\\	棚を片付けなさい、そうすれば本をそこへ置けます。	たなをかたづけなさい、そうすればほんをそこへおけます 
\\	その課の復習をしましたか。	そのかのふくしゅうをしましたか 
\\	彼は千葉の事務所に転勤した。	はれはちばのじむしょにてんきんした 
\\	彼女は私の事務所で飛びぬけて良く働く。	かのじょはわたしのじむしょでとびぬけてよくはたらく 
\\	ポールは学校で飛び抜けて魅力のある生徒だ。	ポールはがっこうでとびぬけてみりょくのあるせいとだ 
\\	此処は、この地域では飛び抜けて最高のシーフードレストランだ。	ここは、このちいきではとびぬけてさいこうのシーフードレストランだ 
\\	これは今年出版された断然最高の小説です。	これはことししゅっぱんされただんぜんさいこうのしょうせつです 
\\	毎年10月が観光客の数が最高になるそうだ。	まいとし10がつがかんこうきゃくのかずがさいこうになるそうだ 
\\	質の点で彼のレポートが最高だ。	しつのてんでかれのレポートがさいこうだ 
\\	納豆の匂いは酷いけれど味は最高。	"なっとうのにおいはひどいけれどあじはさいこう 
\\	現代は質に関係なく、量を求める。	げんだいはしつにかんけいなく、りょうをもとめる 
\\	これが断然最高の方法です。	これがだんぜんさいこうのほうほうです 
\\	彼らの同僚は海外の支店に転勤になった。	かれらのどうりょうはかいがいのしてんにてんきんになった 
\\	地球を含む9個の惑星が太陽の回りを回っている。	ちきゅうをふくむ9このわくせいがたいようのまわりをまわっている 
\\	地球は水があるという点でほかの惑星と違う。	ちきゅうはみずがあるというてんでほかのわくせいとちがう 
\\	生命は他の惑星に存在しているのだろうか。	せいめいはほかのわくせいにそんざいしているのだろうか 
\\	何故君が行きたいのか理解しにくい。	なぜきみがいきたいのかりかいしにくい 
\\	抽象的な芸術には理解しにくいものがある。	ちゅうしょうてきなげいじゅつにはりかいしにくいものがある 
\\	彼の提案はいくつか分かりにくい点がある。	かれのていあんはいくつかわかりにくいてんがある 
\\	尿が出にくいのです。	にょうがでにくいのです 
\\	食べ物が噛みにくいのです。	たべものがかみにくいのです 
\\	噛むとこの歯が痛みます。	かむとこのはがいたみます 
\\	虫に噛まれたところを掻いてはいけない。	むしにかまれたところをかいてはいけない 
\\	彼の説明は私には全然理解できなかった。	かれのせつめいはわたしにはぜんぜんりかいできなかった 
\\	結婚は2人が互いを理解し会えばすばらしいものとなりうる。	けっこんはふたりがたがいをりかいしあえばすばらしいものとなりうる 
\\	彼らは互いに出し抜こうと懸命だった。	かれらはたがいにだしぬこうとけんめいだった 
\\	海と空の色がお互いに溶け合っている。	うみとそらのいろがおたがいにとけあっている 
\\	説明書をちゃんと読めば間違える事はないよ。	せつめいしょをちゃんとよめばまちがえることはないよ 
\\	彼女はほこりを拭うのに湿った布を使った。	かのじょはほこりをぬぐうのにしめったぬのをつかった 
\\	コーヒーの染みは拭い取るのが難しい。	コーヒーのしみはぬぐいとるのがむずかしい 
\\	冷えたビールでも飲みながら話しましょう。	ひえたビールでものみながらはなしましょう 
\\	泳いでいる人達は体が冷えて感覚がなくなっていた。	およいでいるひとたちはからだがひえてかんかくがなくなっていた 
\\	氷点下まで冷えさえしなければいいですよ。	ひょうてんかまでひえさえしなければいいですよ 
\\	光や音は波の形で伝わる。	ひかりやおとはなみのかたちでつたわる 
\\	君の時計は形も色も私のに似ている。	きみのとけいはかたちもいろもわたしのににている 
\\	眼鏡の形をしていることから、その橋を眼鏡橋と呼ぶ。	めがねのかたちをしていることから、そのはしをめがねばしとよぶ 
\\	爆発で通行人が何人か死んだ。	ばくはつでつうこうにんがなんにんかしんだ 
\\	昨夜花火工場で、爆発事故が起こった。	さくやはなびこうじょうで、ばくはつじこがおこった 
\\	彼らは自然と急に笑い出した。	かれらはしぜんときゅうにわらいだした 
\\	日本人は自然と調和して暮らす。	にほんじんはしぜんとちょうわしてくらす 
\\	その山火事は自然の原因で起こった。	そのやまかじはしぜんのげんいんでおこった 
\\	夫婦が愛し合うのは全く自然である。	ふうふがあいしあうのまったくしぜんである 
\\	自然食品が常に消化にいいとは限らない。	しぜんしょくひんがつねにしょうかにいいとはかぎらない 
\\	その陳述は全くの真実とは限らない。	そのちんじゅつはまったくのしんじつとはかぎらない 
\\	明瞭な陳述をしていただけませんか。	めいりょうなちんじゅつをしていただきませんか 
\\	この文の意味は不明瞭だ。	このぶんのいみはふめいりょうだ 
\\	彼女の陳述は結局偽りであることがわかった。	かのじょのちんじゅつはけっきょくいつわりであることがわかった 
\\	田舎の生活がいつも平穏であるとは限らない。	いなかのせいかつがいつもへいおんであるとはかぎらない 
\\	老人達が若者達よりいつも賢いとは限らない。	ろうじんたちはわかものたちよりいつもかしこいとはかぎらない 
\\	天気予報は必ずしも当てになるとは限らない。	てんきよほうはかならずしもあてになるとはかぎらない 
\\	美しい花が必ず良い香りがするとは限らない。	うつくしいはながかならずよいかおりがするとかぎらない 
\\	この法則はすべての場合に適用されるとは限らない。	このほうそくはすべてのばあいにてきようされるとはかぎらない 
\\	その規則はあらゆる場合に適用される。	そのきそくはあらゆるばあいにてきようされる 
\\	どんな規則にも例外がある。	どんなきそくにもれいがいれいがいがある 
\\	どのチームも例外なく打ち負かされた。	どのチームもれいがいなくうちまかされた 
\\	交通規則を守るべきだ。	こうつうきそくをまもるべきだ 
\\	規則正しい生活、食事が健康の秘訣です。	きそくただしいせいかつ、しょくじがけんこうのひけつです 
\\	成功の秘訣は失敗を考えないことだ。	せいこうのひけつはしっぱいをかんがえないことだ 
\\	結婚して不幸になるより、独身で平穏無事に暮らした方がいい。	けっこんしてふこうになるより、どくしんでへいおんぶじにくらしたほうがいい 
\\	それは極端な場合だ。	それはきょくたんなばあいだ 
\\	彼女は洋服の好みが極端だ。	かのじょのようふくのこのみがきょくたんだ 
\\	私は酒類を好みません。	わたしはしゅるいをこのみません 
\\	正直が割に合わない場合もある。	しょうじきがわりにあわないばあいもある 
\\	ジムは一生懸命働くが、彼の仕事はあまり割に合わない。	ジムはいっしょうけんめいはたらくが、かれのしごとはあまりわりにあわない 
\\	それで、弟は出発した。兄はそのまま残った。	それで、おとうとはしゅっぱつした、あにはそのままのこった 
\\	彼はその役を見事に演じた。	かれはそのやくをみごとにえんじた 
\\	僕はサーカスで見事な動物の芸を見た。	ぼくはサーカスでみごとなどうぶつのげいをみた 
\\	今夜のパーティーで彼は見事にホスト役を演じた。	さくやのパーティーでかれはみごとにホストやくをえんじた 
\\	このホテルは見事な海の景色が見渡せる。	このほてるはみごとなうみのけしきがみわたせる 
\\	人々は侵略を撃退することができた。	ひとびとはしんりゃくをげきたいすることができた 
\\	他国への侵略は恥ずべき行為である。	たこくへのしんりゃくははずべきこういである 
\\	彼らは戦車と銃器でその国を侵略した。	かれらはせんしゃとじゅうきでそのくにをしんりゃくした 
\\	彼は侵略者から国を守ったので英雄として扱われている。	かれはしんりゃくしゃからくにをまもったのでえいゆうとしてあつかわれている 
\\	英雄好色	えいゆうこうしょく 
\\	私たちは行いが悪いといって彼女を非難した。	わたしたちはおこないがわるいといってかのじょをひなんした 
\\	自分の責任において何でも行いなさい。	じぶんのせきにんにおいてなんでもおこないなさい 
\\	きみの言語道断な行いを恥ずかしいと思う。	きみのごんごどうだんなおこないをはずはしいとおもう 
\\	彼は脱税で非難された。	かれはだつぜいでひなんされた 
\\	彼は非難されても堂々としていた。	かれはひなんされてもどうどうとしていた 
\\	勝っても負けても、正々堂々プレイしなさい。	かってもまけても、せいせいどうどうプレイしなさい 
\\	彼ほどの偉人は古来いない。	かれほどのいじんはこらいいない 
\\	今日古来の慣習は急速にすたれてきている。	きょうこらいのかんしゅうはきゅうそくにすたれてきている 
\\	彼は欠席したことを弁解した。	かれはけっせきしたことをべんかいした 
\\	彼女は無断で学校を欠席した。	かのじょはむだんでがっこうをけっせきした 
\\	無断でそこに駐車してはいけません。	むだんでそこにちゅうしゃしてはいけません 
\\	駐車場にとめてある車が燃えてます。	ちゅうしゃじょうにとめてあるくるまがもえてます 
\\	私達は駐車場を捜すのに多くの時間を費やした。	わたしたちはちゅうしゃじょうをさがすのにおおくのじかんをついやした 
\\	彼女は本に収入の大部分を費やす。	かのじょはほんにしゅうにゅうのだいぶぶんをついやす 
\\	いい監督を探すのに3年費やしたが、見つからなかった。	いいかんとくをさがすのに3ねんついやしたが、みつからなかった 
\\	監督は私に悪魔の役をくれた。	かんとくはわたしにあくまのやくをくれた 
\\	監督と仲良くやれなかったので彼はチームをやめた。	かんとくとなかよくやれなかったのでかれはチームをやめた 
\\	私の犬と猫は仲良く暮らしている。	わたしのいぬとねこはなかよくくらしている 
\\	彼らが仲良くやっていけない予感がする。	かれらがなかよくやっていないよかんがする 
\\	吹き荒れる嵐の予感に、僕らはこぞって震えあがった。	ふきあれるあらしのよかんに、ぼくらはこぞってふるえあがった 
\\	彼らはこぞってその候補者を支援した。	かれらはこぞってそのこうほしゃをしえんした 
\\	引き続きのご支援を感謝いたします。	ひきつづきのごしえんをかんしゃいたします 
\\	あなたの親切に感謝します。	あなたのしんせつにかんしゃします 
\\	彼の収入の大部分は家賃で消える。	かれのしゅうにゅうのだいぶぶんはやちんできえる 
\\	彼は夕食前に仕事の大部分を終えた。	かれはゆうしょくまえにしごとのだいぶぶんをおえた 
\\	彼は自分の不作法を弁解した。	かれはじぶんのふさほうをべんかいした 
\\	子供に行儀作法を教えるのは親の義務だ。	こどもにぎょうぎさほうをおしえるのはおやのぎむだ 
\\	彼は義務を怠った。	かれはぎむをおこたった 
\\	警官は、彼が義務を怠ったことを責めた。	けいかんは、かれがぎむをおこたったことをせめた 
\\	彼は自分の時間を犠牲にして義務を果たした。	かれはじぶんのじかんをぎせいにしてぎむをはたした 
\\	みんなが犠牲者達に深く同情した。	みんながぎせいしゃたちにふかくどうじょうした 
\\	我々は戸口に立って来客を歓迎した。	われわれはとぐちにたってらいきゃくをかんげいした 
\\	長いことお待たせして済みませんでした。今まで来客で忙しかったのです。	ながいことおまたせしてすみませんでした。いままでらいきゃくでいそがしかったのです 
\\	主人は自分の珍しい切手を客に見せびらかした。	しゅ人はじぶんのめずらしいきってをいきゃくにみせびらかした 
\\	犬は不安そうに主人をながめた。	いぬはふあんそうにしゅじんをながめた 
\\	その劇は主人公の死で終わる。	そのげきはしゅじんこうのしでおわる 
\\	社会的不安を取り除くことが大切です。	しゃかいてきふあんをとりのぞくことがたいせつです 
\\	だれでも人前では多かれ少なかれ、見せびらかしたくなるものだ。	だれでもひとまえではおおかれすくなかれ、みせびらかしたくなるものだ 
\\	私たちは多かれ少なかれ利己的である。	わたしたちはおおかれすくなかれりこてきである 
\\	この空いたカップをどこに捨てたらよいのか。	このあいたカップをどこにすてたらよいのか 
\\	適当な語で空所を満たせ。	てきとうなごでくうしょをみたせ 
\\	その家は私の要求を満たしていない。	そのうちはわたしのようきゅうをみたしていない 
\\	これらの便利な商品は私達のお客様の需要を満たすだろう。	これらのべんりなしょうひんはわたしたちのきゃくさまのじゅようをみたすだろう 
\\	需要が増すに連れて物価は上昇する。	じゅようがますにつれてぶっかはじょうしょうする 
\\	飛行機が急上昇した。	ひこうきがきゅうじょうしょうした 
\\	不景気のため失業率は5%にまで上昇した。	ふけいきのためにしつぎょうりつは5%までじょうしょうした 
\\	犯罪率がこの国で上昇してきている。	はんざいりつがこのくにでじょうしょうしてきている 
\\	私は彼女に行くよう要求した。	わたしはかのじょにいくようようきゅうした 
\\	私は適当な答えを探し求めた。	わたしはてきとうなこたえさがしもとめた 
\\	その科学者は谷で恐竜の骨を探し求めた。	そのかがくしゃはたにできょうりゅうのほねをさがしもとめた 
\\	竜は想像上の生物である。	りゅうはそうぞうじょうのせいぶつである 
\\	夫は失業中で職を探しています。	おっとはしつぎょうちゅうでしょくをさがしています 
\\	この地域の失業は略ないに等しい。	このちいきのしつぎょうはほぼないにひとしい 
\\	1ドルは100セントに等しい。	1ドルは100セントにひとしい 
\\	葉と植物との関係は、肺と動物との関係に等しい。	はとしょくぶつとのかんけいは、はいとどうぶつのかんけいにひとしい 
\\	すぐに両親と連絡を取りなさい。	すぐにりょうしんとれんらくをとりなさい 
\\	品物が到着しましたらご連絡いたします。	しなものがとうちゃくしましたられんらくいたします 
\\	飛行機の到着予定時刻は?	ひこうきのとうちゃくよていじこくは? 
\\	出発時刻を確認したいのですが。	しゅっぱつじこくをかくにんしたいのですが 
\\	見しらぬ人とが近寄ってきて私に時刻を尋ねた。	みしらぬひとがちかよってきてわたしにじこくをたずねた 
\\	真夜中は幽霊がうろつく時刻だと考えられている。	まよなかはゆうれいがうるろつくじこくだとかんがえられている 
\\	私は幽霊の存在を本当に信じている。	わたしはゆうれいのそんざいをほんとうにしんじている 
\\	案の定、幽霊はバルコニーに現れた。	あんのじょう、ゆうれいはバルコニーにあらわれた 
\\	ジョンはその暗い部屋に幽霊の存在を感じた。	ジョンはそのくらいへやにゆうれいのそんざいをかんじた 
\\	こんな真夜中に一体全体彼女はどこへ行くつもりだと思いますか。	こんなまよなかにいったいぜんたいかのじょはどこへいくつもりだとおもいますか 
\\	一体全体どうして新築した家を売ってしまったのですか。	いったいぜんたいどうしてしんちくしたうちをうってしまったのですか 
\\	それらの新築の家はみな同じ高さである。	それらのしんちくのうちはみなおなじたかさである 
\\	彼は新築の家に火災保険をかけた。	かれはしんちくのうちにかさいほけんをかけた 
\\	彼らはその宝石は災害を齎すと信じていた。	かれらはそのほうせきはさいがいをもたらすとしんじた 
\\	警察はその盗まれた宝石を取り戻した。	けいさつはそのぬすまれたほうせきをとりもどした 
\\	私はそれを取り戻す為に多額の金を払った。	わたしはそれをとりもどすためにたがくのかねをはらった 
\\	父は私に多額の財産を残してくれた。	ちちはわたしにたがくのざいさんをのこしてくれた 
\\	ラジオで洪水の警告をしていた。	ラジオでこうずいのけいこくをしていた 
\\	たくさんの家が洪水で流された。	たくさんのうちがこうずいでながされた 
\\	豪雨の結果、洪水が起こった。	ごううのけっか、こうずいがおこった 
\\	多くの人がその歓迎会への招待を辞退した。	おおくのひとがそのかんげいかいへのしょうたいをじたいした 
\\	大統領は奴隷制度を廃止した。	だいとうりょうはどれいせいどをはいしした 
\\	多くの国は死刑を廃止した。	おおくのくにはしけいをはいしした 
\\	彼らはその学校の制服を廃止した。	かれらはそのがっこうのせいふくをはいしした 
\\	学校の制服は全く時代遅れだ。	がっこうのせいどはおおくじだいおくれだ 
\\	君の考えは完全な時代遅れだ。	きみのかんがえはかんぜんなじだいおくれだ 
\\	率直に言うと、君の考え方は時代遅れだ。	そっちょくにいうと、きみのかんがえかたはじだいおくれだ 
\\	彼女が列の先頭です。	かのじょはれつのせんとうです 
\\	現代の教育制度にはどんな欠陥が見られますか。	げんだいのきょういくせいどにどんなけっかんがみられますか 
\\	彼の理論にはまったく欠陥が見当たらない。	かれのりろんにはまったくけっかんがみあたらない 
\\	報告書がいくつか見当たらなかった。	ほうこくしょがいくつかみあたらなかった 
\\	月曜日までに報告書を私に提出しなさい。	げつようびまでにほうこくしょをわたしにていしゅつしなさい 
\\	急いで書いたので、その報告書はよくなかった。	いそいでかいたので、そのほうこくしょはよくなかった 
\\	どんなに努力しても、金曜日までに報告書を用意しておけないだろう。	どんなにどりょくしても、きにょうびまでにほうこくしょをよういしておけないだろう 
\\	私はあらゆることを前もって用意して床に就いた。	わたしはあらゆることをまえもってよういしてとこについた 
\\	不用意な言葉は大きな災いのもとになるであろう。	ふよういなことばはおおきなわざわいのもとになるであろう 
\\	彼らの課題は9月1日に提出された。	かれらのかだいは9げつついたちにていしゅつされた 
\\	その学生は自分の課題に専念した。	おのがくせいはじぶんのかだいにせんねんした 
\\	彼は頭痛のために授業に専念できなかった。	かれはずつうのためにじゅぎょうにせんねんできなかった 
\\	彼は妻を喜ばせようとあらゆる努力をした。	かれはつまをよろこばせようとあらゆるどりょくをした 
\\	毎朝早起きできるように努力するつもりだ。	まいあさはやおきできるようにどりょくするつもりだ 
\\	彼は新しい環境に適応しようと努力した。	かれはあたらしいかんきょうにてきおうしようとどりょくした 
\\	先生は絵を用いて彼の理論を説明した。	先生はえをもちいてかれのりろんをせつめいした 
\\	その理論は周到な研究に基づいている。	そのりろんはしゅうとうなけんきゅうにもとづいている 
\\	彼は大変な努力をして障害を乗り越えた。	かれはたいへんなどりょくをしてしょうがいをのりこえた 
\\	我々はこの困難を乗り越えなければなりません。	われわれはこのこんなんをのりこえなければなりません 
\\	彼の大胆な決意のおかげで危機を乗り越えることができた。	かれのだいたんなけついのおかげでききをのりこえることができた 
\\	会計係はこれらの数字を調べるだろう。	かいけいけいはこれらのすうじをしらべるだろう 
\\	辞書は知らない単語を調べるために使われる。	じしょはしらないたんごをしらべるためにつかわれる 
\\	人は互いの家を詳しく調べるのが大好きだ。	ひとはたがいのうちをくわしくしらべるのがだいすきだ 
\\	警官は私の札入れを調べた。	けいかんはわたしのさついれをしらべた 
\\	彼はなくした鍵を捜して部屋を調べた。	かれはなくしたかぎをさがしてへやをしらべた 
\\	どうして君の答えが僕のと違うのか調べてみよう。	どうしてきみのこたえがぼくのとちがうのかしらべてみよう 
\\	私たちを招くようにネオンサインが輝いていた。	わたしたちをまねくようにねおんさいんがかがやいていた 
\\	彼の無謀な運転が死を招いた。	かれのむぼうなうんてんがしをまねいた 
\\	猫の目は何故暗闇で輝くのですか。	ねこのめはなぜくらやみでかがやくのですか 
\\	空は花火で光り輝いていた。	そらははなびでひかりかがやいていた 
\\	今夜は月が明るく輝いている。	こにゃはつきがあかるくかがやいている 
\\	朝日で城が燃えるように輝いた。	あさひでしろがもえるようにかがやいた 
\\	イヌは鋭い嗅覚を持っている。	イヌはするどいきゅうかくをもっている 
\\	猟犬は鋭い嗅覚で獲物を追う。	りょうけんはするどいきゅうかくでえものをおう 
\\	彼が嗅覚を失ったのは、頭部のけがのためだった。	かれがきゅうかくをうしなったのは、とうぶのけがのためだった 
\\	彼らは自ら困難を招いた。	かれらはみずからこんなんをまねいた 
\\	彼は自ら記者たちに向かって発言した。	かれはみずからきしゃたちにむかってはつげんした 
\\	あれはどうも適切な発言ではなかった。	あれはどうもてきせつなはつげんではなかった 
\\	病人が適切な看護を受けられるようにしなさい。	びょうにんがてきせつなかんごをうけられるようにしなさい 
\\	その計画はいくつかの点で不適切なものであると思う。	そのけいかくはいくつかのてんでふてきせつなものであるとおもう 
\\	どんな地位にも適切な人を選ばなくてはならない。	どんなちいにもてきせつなひとをえらばなくてはならない 
\\	よい仕事をしたければ、適切な道具を使うべきだ。	よいしごとをしたければ、てきせつなどうぐをつかうべきだ 
\\	彼は大声で笑ったが、その振る舞いは不適切に思えた。	かれはおおごえでわらったが、そのふるまいはふてきせつにおもえた 
\\	私は彼のあんな振る舞いは許せない。	わたしはかれのあんなふるまいはゆるせない 
\\	彼のよい振る舞いに最も強い印象を受けた。	かれのよいふるまいにもっともつよいいんしょうをうけた 
\\	展示会は大変印象的だった。	てんじかいはたいへんいんしょうてきだった 
\\	私の全体的な印象ではそれは非常によい。	わたしのぜんたいてきないんしょうではそれはひじょうによい 
\\	私の彼についての第一印象は正しいことがわかった。	わたしのかれについてのだいいちいんしょうはただしいことがわかった 
\\	風邪で彼の味覚は鈍くなっていた。	かぜでかれのみかくはおそくなっていた 
\\	このヌードのポスターには若者の視覚に訴えるものがある。	このヌードのポスターにはわかものしかくにうったえるものがある 
\\	視覚は五感の1つである。	しかくはごかんのひとつである 
\\	彼女はカップに熱湯を注いだ。	かのじょはカップにねっとうをそそいだ 
\\	だれもがその囚人に情けをかけた。	だれもがそのしゅうじんになさけをかけた 
\\	その事務員は過労がもとで死んだ。	そのじむいんはかろうがもとでしんだ 
\\	過労や粗末な食事の為に、彼女は病気になった。	かろうやそまつなしょくじのために、かのじょはびょうきになった 
\\	マイクは、丸太から粗末なテーブルを作った。	マイクは、まるたからそまつなテーブルをつくった 
\\	彼は暖炉用に丸太を鋸で切った。	かれはだんろようにまるたをのこぎりできった 
\\	暖炉のおかげでこの部屋は居心地が良い。	だんろのおかげでこのへやはいごこちがよい 
\\	この会社はどうも居心地が悪い。	このかいしゃはどうもいごこちがわるい 
\\	トムは知らない人に混じって居心地が悪かった。	トムはしらないひとにまじっていごこちがわるかった 
\\	喜びは悲しみと混じり合った。	よろこびはかなしみとまじりあった 
\\	彼女は自分の不幸な運命にため息をついた。	かのじょはじぶんのふこうなうんめいにためいきをついた 
\\	孤独は人間共通の運命だ。	こどくはにんげんきょうつうのうんめいだ 
\\	彼の無事を知らされて、部長は安堵のため息をついた。	かれのぶじをしらされて、ぶちょうはあんどのためいきをついた 
\\	彼が無事に戻るという望みはない。	かれがぶじにもどるというのぞみはない 
\\	彼は望みを捨てようとしない。	かれはのぞみをすてようとしない 
\\	彼らは2度と祖国へ戻ることはなかった。	かれらは2どとそこくへもどることはなかった 
\\	彼女は病院で意識を取り戻した。	かのじょはびょういんでいしきをとりもどした 
\\	もとの場所へ戻しておきなさい。	もとのばしょへもどしておきなさい 
\\	サラは払い戻してもらう事を要求した。	サラははらいもどしてもらうことをようきゅうした 
\\	彼女は忘れ物を渡すために彼を呼び戻した。	彼女はわすれものをわたすためによびもどした 
\\	その少女は、我々が彼女の顔に水をかけると、意識を取り戻した。	そのしょうじょは、われわれがかのじょのかおにみずをかけると、いしきをとりもどした 
\\	彼は数日間意識不明であった。	かれはすうじつかんいしきふめいであった 
\\	罪の意識が彼の顔にはっきり現れている。	つみのいしきがかれのかおにはっきりあらわれている 
\\	麻薬中毒は現代社会の癌だ。	まやくちゅうどくはげんだいしゃかいのがんだ 
\\	アメリカの犯罪者の多くは麻薬中毒である。	アメリカのはんざいしゃのおおくはまやくちゅうどくである 
\\	麻薬はどんなものでも敬遠しておいたほうがいい。	まやくはどんなものでもけいえんしておいたほうがいい 
\\	トムはすぐにヒステリーを起こすので、みんなが敬遠する。	トムはすぐにヒステリーをおこすので、みんながけいえんする 
\\	人前で話すようなことは何によらず彼はいつも敬遠する。	ひとまえではなすようなことはなんによらずかれはいつもけいえんする 
\\	政府は公害を除去しようと努めている。	せいふはこうがいをじょきょしようとつとめている 
\\	砂糖は熱いコーヒーに入れると溶ける。	さとうはあついコーヒーにいれてととける 
\\	愛より尊い物はない。	あいよりとうといものはない 
\\	英国人は法と秩序を大いに尊重する。	えいこくじんはほうとちつじょをおおいにそんちょうする 
\\	彼女は他人の意見を尊重する。	かのじょはたにんのいけんをそんちょうする 
\\	他人に対して優越感を持ってはいけない。	たにんにたいしてゆうえつかんをもってはいけない 
\\	彼らは箱をこなごなに打ち壊した。	かれらははこをこなごなにぶちこわした 
\\	葬式には黒い服を着ていくのが習慣です。	そうしきにはくろいふくをきていくのがしゅうかんです 
\\	彼の葬式に行かなかった事を後悔している。	かれのそうしきにいかなかったことをこうかいしている 
\\	彼は敵に国を売ったことを後悔した。	かれはてきにうったことをこうかいした 
\\	こういう種類の娯楽は私にはまったくない。	こういうしゅるいのごらくはわたしにはまったくない 
\\	彼には大好きな自分の娯楽に夢中になる時間があった。	かれはだいすきなじぶんのごらくにむちゅうになるじかんがあった 
\\	わが国は海産物に恵まれている。	わがくにはかいさんぶつにめぐまれている 
\\	私の叔母は生涯健康に恵まれた。	わたしのおばはしょうがいけんこうにめぐまれた 
\\	知恵は経験なしには得られない。	ちえはけいけんなしにはえられない 
\\	彼は知恵の足りないのを力で補った。	かれはちえのたりないのをちからでおぎなった 
\\	勤勉さが経験不足を補うこともあり得る。	きんべんさがけいけんふそくをおぎなうこともありえる 
\\	私は失った時間を補うために一生懸命に仕事をしなければいけない。	わたしはうしなったじかんをおぎなうためにいっしょうけんめいにしごとをしなければいけない 
\\	彼はどうにか逃げる事が出来た。	かれはにげることができた 
\\	彼は警官を見て逃げた。	かれはけいかんをみてにげた 
\\	彼女は当惑して私から逃げた。	かのじょはとうわくしてわたしから逃げた 
\\	彼女が当惑したことに赤ん坊はとうとう泣きやまなかった。	かのじょがとうわくしたことにあかんぼうはとうとうなきやまなかった 
\\	なるべく安いほうがいいです。	なるべくやすいほうがいいです 
\\	われわれはなるべくたくさん本を読むべきである。	われわれはなるべくたくさんほんをよむべきである 
\\	なるべく早くご返事いただければ幸いです。	なるべくはやくごへんじいただければさいわいです 
\\	彼女は幸いにも奨学金を得た。	かのじょはさいわいにもしょうがくきんをえた 
\\	私はその奨学金を申し込むつもりだ。	わたしはそのしょうがくきんをもうしこむつもりだ 
\\	彼は無礼な返事をした。	かれはぶれいなへんじをした 
\\	彼女は彼の無礼にこれ以上我慢できなかった。	かのじょはかれのぶれいにこれいじょうがまんできなかった 
\\	今さら騒いでもどうにもならないよ。後の祭りだよ。	いまさらさわいでもどうにもならないよ。あとのまつりだよ 
\\	私の好みはごく当たり前さ。	わたしのこのみはごくあたりまえさ 
\\	彼は危険を前にしても冷静だった。	かれはきけんをまえにしてもれいせいだった 
\\	事態は、私たちの冷静な判断力を必要としている。	じたいは、わたしたちのれいせいなはんだんりょくをひつようをしている 
\\	薬缶が沸いている。	やかんがわいている 
\\	牛乳は水よりも高い温度で沸騰する。	ぎゅうにゅうはみずよりたかいおんどでふっとうする 
\\	私はお湯も沸かせない、まして七面鳥など焼くことができない。	わたしはおゆもわかせない、ましてしちめんちょうなだやくことができない 
\\	薬缶は沸騰しているに違いない。	やかんはふっとうしているにちがいない 
\\	日本は海に囲まれている。	にほんはうみにかこまれている 
\\	池の周囲は木で囲まれていた。	いけのしゅいはきでかこまれていた 
\\	老人は子供たちに囲まれて座っていた。	ろうじんはこどもたちにかこまれてすわっていた 
\\	彼はどこに行っても周囲に幸福を発散する。	かれはどこにいってもしゅういにこうふくをはっさんする 
\\	火は赤々と燃えて実に気持ちの良い暖かさを発散した。	ひはあかあかともえてじつにきもちのよいあたたかさをはっさんした 
\\	去年ブラウン先生がこのクラスを受け持った。	きょねんブラウン先生がこのクラスをうけもった 
\\	この病院では、各看護婦は5人の患者の看護を受け持っている。	このびょういんでは、かくかんごふはごにんのかんじゃのかんごをうけもっている 
\\	薬の効果は消えていた。	くすりのこうかはきえていた 
\\	最も効果の高いサンスクリーンはどれですか。	もっともこうかのたかいサンスクリーンはどれですか 
\\	その方法は粗雑なものであったが効果的だった。	そのほうほうはそざつなものであったがこうかてきだった 
\\	忍耐はもっとも効果的な武器になることがある。	にんたいはもっともこうかてきなぶきになることがある 
\\	彼女は両膝に肘をのせた。	かのじょはりょうひざにひじをのせた 
\\	その皮製の上着の両肘部分が擦り減って穴が空いた。	そのかわせいのうわぎのりょうひじぶぶんがすりへってあながあいた 
\\	いつ何時戦争が起こるかもしれない。	いつなんじせんそうがおこるかもしれない 
\\	この薬は彼の回復に効果があるかもしれない。	このくすりはかれのかいふくにこうかがあるこもしれない 
\\	彼はその公式を知らなかったのかもしれない。	かれはそのこうしきをしらなかったのかもしれない 
\\	彼が回復する見込みはあるでしょうか。	かれがかいふくするみこみはあるでしょうか 
\\	コーヒーを1杯飲んで私は元気を回復した。	コーヒーをいちばいのんでわたしはげんきをかいふくした 
\\	驚くべきことに、その老人は健康を回復した。	おどろくべきことに、そのろうじんはけんこうをかいふくした 
\\	彼の成功の見込みは十分ある。	かれのせいこうのみこみはじゅうぶんある 
\\	彼が選挙に勝つ見込みはありますか。	かれがせんきょにかつみこみはありますか 
\\	選挙の結果を予測するのは難しい。	せんきょのけっかをよそくするのはむずかしい 
\\	相手チームをやっつけるのは朝飯前だった。	あいてちーむをやっつけるのはあさめしまえだった 
\\	今度の事件で彼を見直した。	こんどのじけんでかれをみなおした 
\\	史上もっとも重要な事件。	しじょうもっともじゅうようなじけん 
\\	歴史上の名所を見物しました。	れきしじょうのめいしょをけんぶつしました 
\\	観光名所を方々訪ね歩いたので、すっかり疲れ果ててしまった。	かんこうめいしょをほうぼうたずねあるいたので、すっかりつかれはててしまった 
\\	長い距離を歩いて疲れ果てた。	ながいきょりをあるいてつかれはてた 
\\	長距離電話の請求書を見て彼は驚いた。	ちょうきょりでんわのせいきゅうしょをみてかれはおどろいた 
\\	その絵は距離を置いてみると良く見える。	そのえはきょりをおいてみるとよくみえる 
\\	彼は家に帰った時には疲れ果てていた。	かれはいえにかえったときにはつかれはてていた 
\\	長い経験が減らすことを教えた。	ながいけいけんがへらすことをおしえた 
\\	彼はダイエットしたけれども、まだ体重を減らすことが出来なかった。	かれはダイエットしたけれども、まだたいじゅうをへらすことができなかった 
\\	戦争がその国の富を減らした。	せんそうがそのくにのとみをへらした 
\\	彼女は長年の間に包丁をする減らした。	かのじょはながねんのかんにほうちょうをするへらした 
\\	父は医者からタバコの量を減らせといわれた。	ちちはいしゃからタバコのりょうをへらせといわれた 
\\	その物質は水に浮かぶほど軽い。	そのぶっしつはみずにうかぶほどかるい 
\\	花に浮かれて踊っている。	はなにうかれておどっている 
\\	良い考えが彼に浮かんだ。	よいかんがえがかれにうかんだ 
\\	彼女の目の青さが浅黒い肌に浮き出て見える。	かのじょのめがあおさがあさぐろいはだにうきでてみえる 
\\	男性の方が女性より肌が脂っぽいって本当ですか?	だんせいのほうがじょせいよりはだがあぶらっぽいってほんとうですか? 
\\	世界経済の回復はまだ視野に入ってこない。	せかいけいざいのかいふくはまだしやにはいってこない 
\\	過去の過ちが迫ってくる。	かこのあやまちがせまってくる 
\\	その事故は運転手の側の過ちから起こった。	そのじこはうんてんしゅのがわのあやまちからおこった 
\\	不注意な人間は過ちを犯しがちである。	ふちゅういなにんげんはあやまちをおかしがちである 
\\	逆境で人は成長する。	ぎゃっきょうでひとはせいちょうする 
\\	日本の経済は昨年4%成長した。	にほんのけいざいはさくねんよんぱーせんとせいちょうした 
\\	彼は成長して立派な紳士になった。	かれはせいちょうしてりっぱなしんしになった 
\\	都市から遠くまで立派な道路が伸びている。	としからとおくまでりっぱなどうろがのびている 
\\	彼女は急速に英語力が伸びた。	かのじょはきゅうそくにえいごりょくがのびた 
\\	彼は過労で伸びている。	かれはかろうでのびている 
\\	この廃虚は嘗て立派な宮殿であった。	このはいきょはかつてりっぱなきゅうでんであった 
\\	我々の家と比べると、彼の家は宮殿だ。	われわれのいえとくらべると、かれのいえはきゅうでんだ 
\\	父親と比べてみると彼は深みがない。	ちちおやとくらべてみるとかれはふかみがない 
\\	そのビルはニューヨークの摩天楼と比べると小さい。	そのビルはニューヨークのまてんろうとくらべるとちいさい 
\\	人生にはいろいろ耐えるべき苦労がある。	じんせいにはいろいろたえるべきべきくろうがある 
\\	私はまだ英語を通じさせるのに苦労します。	わたしはまだえいごをつうじさせるのにくろうします 
\\	趣味は日常生活の苦労を忘れさせてくれる。	しゅみはにちじょうせいかつのくろうをわすれさせてくれる 
\\	従業員達は辛い仕事の苦労を共にしている。	じゅうぎょういんたちはからいしごとのくろうをともにしている 
\\	我が家の庭は甘い香りの花で満ちている。	わがやのにわはあまいかおりのはなでみちている 
\\	彼の声には優しい気づかいが満ちていた。	かれのこえにはやさしいきづかいがみちていた 
\\	水夫は陽気な歌を歌った。	すいふはようきなうたをうたった 
\\	彼は陽気そうに見えるが、本当はそれに反して悲しんでいるのだ。	かれはようきそうにみえるが、ほんとうはそれにかえしてかなしんでいるのだ 
\\	その3人の中では、彼女が一番内気ではなさそうだ。	そのさんにんのなかでは、かのじょがいちばんうちきではなさそうだ 
\\	彼は人付き合いが悪いと言うより内気なのです。	かれはひとづきあいがわるいというよりうちきなのです 
\\	彼の発言で私の希望は失われた。	かれのはつげんでわたしのきぼうはうしなわれた 
\\	彼はその無人島の探検を希望している。	かれはそのむじんとうのたんけんをきぼうしている 
\\	彼はアフリカへ探検旅行に行くのが好きだ。	かれはアフリカへたんけんりょこうにいくのがすきだ 
\\	昔の探検家たちは航海するのに星を利用した。	むかしのたんけんかたちはこうかいするのにほしをりようした 
\\	その工場では新型の機械を製造しています。	そのこうじょうでしんがたのきかいをせいぞうしています 
\\	彼は新型の掃除機の実物を見せて説明した。	かれはしんがたのそうじきのじつぶつをみせてせつめいした 
\\	この肖像画は略実物大です。	このしょうぞうがはほぼじつぶつだいです 
\\	矢印が進むべき方向を指示する。	やじるしがすすむべきほうこうをしじする 
\\	彼の友達はみんな彼の案を指示した。	あなたのともだちはみんなかれのあんをしじした 
\\	審判はどちらの側も指示するべきではない。	しんぱんはどちらのがわもしじするべきではない 
\\	彼一人だけ余る、だから審判させよう。	かれはひとりだけあまる、だからしんぱんさせよう 
\\	大衆の注意は彼の審判に向けられた。	たいしゅうのちゅういはかれのしんぱんにむけられた 
\\	その歌は大衆に流行した。	そのうたはたいしゅうにはやりした 
\\	真の歴史を形成するのは大衆である。	しんのれきしのけいせいするのはたいしゅうである 
\\	先生は子どもの心を形成するのを助ける。	せんせいはこどものこころをけいせいするのをたすける 
\\	教育の目標は、富や地位ではなく人格の形成にある。	きょういくのもくひょうはとみやちいではなくじんかくのけいせいにある 
\\	財産と人格とはまったく別のものだ。	ざいさんとじんかくとはまったくべつのものだ 
\\	あの種の服が今流行だ。	あのたねのふくがいまはやりだ 
\\	長髪は今や流行遅れだ。	ちょうはつはいまやはやりおくれだ 
\\	警官は口笛を吹いて車に止まるよう合図した。	けいかんはくちぶえをふいてくるまにとまるようあいずした 
\\	彼らは出発の合図を待っていた。	かれらはしゅっぱつのあいずをまっていた 
\\	彼女に喫煙しないように合図した。	かのじょはきつえんしないようにあいずした 
\\	妻は部屋のむこう端から私に合図した。	つまはへやのむこうはしからわたしにあいずした 
\\	唇に手を当てて黙っていろと合図する。	くちびるにてをあててだまっていろとあいずする 
\\	彼は私の家の6軒先に住んでいる。	かれはわたしのいえの6のきさきにすんでいる 
\\	私たちは彼女の指示に従って作業を完了した。	わたしたちはかのじょのしじにしたがってさぎょうをかんりょうした 
\\	彼女は素敵な女性に成熟していた。	かのじょはすてきなじょせいにせいじゅくしていた 
\\	このオレンジはすぐに成熟します。	このオレンジはすぐにせいじゅくします 
\\	看護婦は患者の快適さに配慮しなくてはいけない。	かんごふはかんじゃのかいてきさにはいりょしなくていけない 
\\	彼女がまだ若いという点を配慮しなければいけない。	かのじょがまだわかいというてんをはいりょしなければいけない 
\\	決して他人の悪口を言うな。	けっしてたにんのわるぐちをいうな 
\\	他人のためでなく、自分のためにしなさい。	たにんのためでなく、じぶんのためにしなさい 
\\	彼は他人を楽しくさせる才能を持っている。	かれはたにんをたのしくさせるさいのうをもっている 
\\	彼はまったく他人の気持ちを尊重しないの。	かれはまったくたにんのきもちをそんちょうしないの 
\\	私たちは個人の権利を尊重しなければならない。	わたしたちはこじんのけんりをそんちょうしなければならない 
\\	訪ねるときは前もって連絡します。	たずねるときはまえもってれんらくします 
\\	必要な場合には、訪ねていらっしゃい。	ひつようなばあいには、たずねていらっしゃい 
\\	この場合この規則を適用できますか。	このばあいこのきそくをてきようできますか 
\\	彼は自分の理論をいくつかの場合に適用した。	かれはじぶんのりろんをいくつかのばあいにてきようした 
\\	どの規則にも例外がある。	どのきそくにもれいがいがある 
\\	私は大変気分が悪い。ゲロをはきたいです。	わたしはたいへんきぶんがわるい。ゲロをきたいです 
\\	誰かが逃げるのを見ましたか。	だれかがにげるのをみましたか 
\\	大きな動物が動物園から逃げた。	おおきいどうぶつがどうぶつえんからにげた 
\\	犬を見ると猫は逃げ出した。	いぬをみるとねこはにげだした 
\\	鹿は驚いて急いで逃げていった。	しかはおどろいていそいでにげていった 
\\	車の免許を取りに行く。	くるまのめんきょをとりにいく 
\\	運転免許証用の写真をとってもらいましたか。	うんてんめんきょしょうようのしゃしんをとってもらいましたか 
\\	君の出席は免除する。	きみのしゅっせきはめんじょする 
\\	彼は借金を免除してもらった。	かれはしゃっきんをめんじょしてもらった 
\\	彼は辛うじて災難を免れた。	かれはかろうじてさいなんをまぬかれた 
\\	彼らは災難に遭っても冷静であった。	かれらはさいなんにあってもれいせいであった 
\\	古い木箱がテーブルの役目をした。	ふるいきばこがテーブルのやくめをした 
\\	それを見てうれしくて背筋がぞくぞくした。	それをみてうれしくてせすじがぞくぞくした 
\\	醜いアヒルの子は優雅な白鳥となった。	みにくいアヒルのこはゆうがなはくちょうとなった 
\\	観客たちは彼女の優雅な演技に感動した。	かんきゃくたちはかのじょのゆうがなえんぎにかんどうした 
\\	居合わせた人々は皆感動の余りないた。	いあわせたひとびとはみんなかんどうのあまりないた 
\\	居合わせた人はその報告にがっかりした。	いあわせたひとはそのほうこくにがっかりした 
\\	その報告は残念ながら事実だ。	そのほうこくはざんえんながらじじつだ 
\\	残念ながらその予想は外れてしまった。	ざんえんながらそのよそうははずれてしまった 
\\	残念ながら彼女の成功を確信していません。	ざんねんながらかのじょのせいこうをかくしんしていません 
\\	僕は悲しいような、嬉しいような気持ちになった	ぼくはかなしいような、うれしいようなきもちになった 
\\	あいつとは今後関係がない。	あいつとはこんごかんけいがない 
\\	今後、あなたの仕事を手伝うようにしましょう。	こんご、あなたのしごとをてつだうようにしましょう 
\\	選ぶべき道は自由か死だ。	えらぶべきみちはじゆうかしだ 
\\	自由に召し上がって下さい。	じゆうにめしあがってください 
\\	遠慮なくケーキを召し上がって下さい。	えんりょなくケーキをめしあがってください 
\\	この部屋は自由に使っていいですか。	このへやはじゆうにつかっていいですか 
\\	中世は人間が自由でない時代だった。	ちゅうせいはじんかんがじゆうでないじだいだった 
\\	言論の自由は厳しく制限されていた。	げんろんのじゆうはきびしくせいげんされていた 
\\	ひどい頭痛に悩んでいる。	ひどいずつうになやんでいる 
\\	彼の考えでは、近い将来水不足に悩む時代が来る。	かれのかんがえでは、ちかいしょうらいみずふそくになやむじだいがくる 
\\	私には悩みを相談できる人がいないのです。	わたしはなやみをそうだんできるひとがいないのです 
\\	制限速度を超えていましたね。	せいげんそくどをこえていましたね 
\\	サリーは先月からずっと食事制限をしている。	サリーはせんげつからずっとしょくじせいげんをしている 
\\	彼の説明はまったく理屈に合わない。	かれのせつめいはまったくりくつにあわない 
\\	彼女の理屈には全く面食らった。	かのじょのりくつにはまったくめんくらった 
\\	彼の質問にひどく面食らってしまった。	かれのしつもんにひどくめんくらってしまった 
\\	その物語は映画用に脚色された。	そのものがたりはえいがようにきゃくしょくされた 
\\	残念ながら、あなたの予測は的外れでした。	ざんえんながら、あなたのよそくはまとはずれでした 
\\	牧童たちは牛の群れを駆り集めた。	ぼくどうたちはうしのむれをかりあつめた 
\\	蚊の群れが彼を追った。	かのむれがかれをおった 
\\	少年は池の鯉の群れをじっと見つめていた。	しょうねんはいけのこいのむれをじっとみつめていた 
\\	ヤンマがすいすいと水の上を進んでいた。	ヤンマがすいすいとみずのうえをすすんでいた 
\\	氷が溶けている。	こおりがとけている 
\\	彼のおもみで氷が割れた。	かれのおもみでこおりがわれた 
\\	水と氷は形は異なるが、同じ物質だ。	みずとこおりはかたちはことなるが、おなじぶっしつだ 
\\	この物質はそれ自体では有毒ではない。	このぶっしつはそれじたいではゆうどくではない 
\\	この家は住み心地が全然よくない。	このうちはすみごこちがぜんぜんよくない 
\\	この喫茶店は居心地がよい。	このきっさてんはいごこちがよい 
\\	彼は本当に心地よくその金を貸してくれた。	かれはほんとうにここちよくそのおかねをかしてくれた 
\\	島は幅が1マイル近くある。	しまははばが1マイルちかくある 
\\	彼は幅広い経験を積んでいる人だ。	かれははばひろいけいけんをつんでいるひとだ 
\\	インカ族は幅広い興味を持っていた。	インカぞくははばひろいきょうみをもっていた 
\\	その本の返却には一定の期限がある。	そのほんのへんきゃくにはいっていのきげんがある 
\\	彼は自転車を一定の速度で走らせた。	かれはじてんしゃをいっていのそくどではしらせた 
\\	この道路の制限速度は何キロですか。	このどうろのせいげんそくどはなんキロですか 
\\	列車は一時間に500マイルの速度で走った。	れっしゃはいちじかんに500マイルのそくどではしった 
\\	彼は環境に順応した。	かれはかんきょうにじゅんのうした 
\\	温度の急激な変化に順応するのは困難である。	おんどのきゅうげきなへんかにじゅんのうするのはこんなんである 
\\	交換のために返却したいと思います。	こうかんのためにへんきゃくしたいとおもいます 
\\	預けていた貴重品を返却してください。	あずけていたきちょうひんをへんきゃくしてください 
\\	貴重品は銀行に保管してある。	きちょうひんはぎんこうにほかんしてある 
\\	あれは貴重な経験だった。	あれはきちょうなけいけんだった 
\\	友人ほど貴重な宝はほとんどない。	ゆうじんほどきちょうなたからはほとんどない 
\\	我々は埋められた宝を探していた	われわれはうめられたたからをさがしていた 
\\	彼女は宝石を得意げに見せびらかした。	かのじょはほうせきをとくいげにみせびらかした 
\\	その町は産業共同体である。	そのまちはさんぎょうきょうどうたいである 
\\	私は書類を彼に預けた。	わたしはしょるいをかれにあずけた 
\\	そのチームの選手は各自めいめいのバットを持っている。	そのチームのせんしゅはかくじめいめいのバットをもっている 
\\	彼女は私に1枚の紙を手渡した。	かのじょはわたしに1まいのかみをてわたした 
\\	アンは妹のために子守唄を歌ってあげた。	アンはいもうとのためにこもりうたをうたってあげた 
\\	彼が戻ってきた時には、女は歩み去っていた。	かれがもどってきたときには、おんなはあゆみさっていた 
\\	360号室の合い鍵を貸していただきませんか。	360ごうしつのあいかぎをかしていただきませんか 
\\	418号室に行く途中、彼は次のように思い始めました。	418ごうしつにいくとちゅう、かれはつぎのようにおもいはじめました 
\\	彼に会う機会を見送った。	かれにあうきかいをみおくった 
\\	彼女は外国人と接触する機会がない。	かのじょはがいこくじんとせっしょくするきかいがない 
\\	これは失うにはあまりにも惜しい機会だ。	これはうしなうにはあまりにもおしいきかいだ 
\\	社長は我々の給料を少しあげることすら惜しんだ。	しゃちょうはわれわれのきゅうりょうをすこしあげることすらおしんだ 
\\	係員が頷いたので彼女は小切手を書き、それを手渡した。	かかりいんがうなずいたのでかのじょはこぎってをかき、それをてわたした 
\\	彼は不安げに椅子の上でもじもじした。	かれはふあんげにいすのうえでもじもじした 
\\	彼女は人前に出るともじもじする癖がある。	かのじょはひとまえにでるともじもじするくせがある 
\\	たいてい、飢饉になると疫病も発生する。	たいてい、ききんになるとえきびょうもはっせいする 
\\	その疫病で現在までに100人もの人が亡くなった。	そのえきびょうでげんざいまでに100にんものひとがなくなった 
\\	その長い干ばつの後に飢饉が起こった。	そのながいかんばつのあとにききんがおこった 
\\	これは本来無害です。	これはほんらいむがいです 
\\	競争は本来悪いものではない。	きょうそうはほんらいわるいものではない 
\\	彼は強力な競争相手を打ち破った。	かれはきょうりょくなきょうそうあいてをうちやぶった 
\\	飛行機はニューヨークに接近している。	ひこうきはニューヨークにせっきんしている 
\\	彼女の頬には涙が流れていた。	かのじょのほおにはなみだがながれていた 
\\	恥ずかしさで彼の頬は真っ赤になっていた。	はずかしさでかれのほおはまっかになっていた 
\\	彼の誉め言葉に彼女の頬が赤くなりだした。	かれのほめことばにかのじょのほおがあかくなりだした 
\\	面と向かってあなたを褒めるような人を信用してはいけない。	めんとむかってあなたをほめるようなひとをしんようしてはいけない 
\\	住宅事情はと言うと、日本はとても貧しい状態だ。	じゅうたくじじょうはというと、にほんはとてもまずしいじょうたいだ 
\\	彼は中国の事情に通じている。	かれはちゅごくのじじょうにつうじている 
\\	事情があって私には、それ以上は言えません。	じじょうがあってわたしには、それいじょうはいえません 
\\	彼は大きな箱を両腕に抱えていた。	かれはおおきなはこをりょううでにかかえていた 
\\	彼女は脇の下にバッグを抱えています。	かのじょはわきのしたにバッグをかかえています 
\\	この学校のカリキュラムは広く浅い。	このがっこうのカリキュラムはひろくあさい 
\\	太った男が浅い溝を飛び越えて、よろけた。	ふとったおとこがあさいみぞをとびこえて、よろけた 
\\	溝を掘り終えたら花を植えるのは簡単だよ。	みぞをほりおえたらはなをうえるのはかんたんだよ 
\\	彼女は春に植えるたくさんの種を備えている。	かのじょははるにうえるたくさんのたねをそなえている 
\\	公園には何らかの種類の木が植えてある。	こうえんにはなんらかのしゅるいのきがうえてある 
\\	農夫達は稲を植えていた。	のうふたちはいねをうえていた 
\\	相撲は日本の伝統的なスポーツです。	すもうはにほんのでんとうてきなスポーツです 
\\	結婚の贈り物として伝統的にお金を与える人種集団はたくさんある。	けっこんのおくりものとしてでんとうてきにおかねをあたえるじんしゅしゅうだんはたくさんある 
\\	日本人は集団で旅行するのが好きだ。	にほんじんはしゅうだんでりょこうするのがすきです 
\\	毎日同じ教室を2つの異なった生徒達の集団で使っているのです。	まいにちおなじきょうしつをふたつのことなったせいとたちのしゅうだんでつかっているのです 
\\	アメリカにはいくつかの人種が一緒に住んでいる。	アメリカにいくつかのじんしゅがいっしょにすんでいる 
\\	彼は人種的な偏見を強く持っている人ではない。	かれはじんしゅてきなへんけんをつよくもっているひとではない 
\\	彼らは偏見と戦った。	かれらはへんけんとたたかった 
\\	民族的少数派は偏見、貧困、抑圧と戦っている。	みんぞくてきしょうすうははへんけん、ひんこん、よくあつとたたかっている 
\\	政府はすべての野党を抑圧しようとした。	せいふはすべてのやとうをよくあつしようとした 
\\	政府の政策は野党から非難された。	せいふのせいさくはやとうからひなんされた 
\\	その男は私を無責任だと非難した。	そのおとこはわたしをむせきにんだとひなんした 
\\	彼の行動は人から非難を受けやすい。	かれのこうどうはひとからひなんをうけやすい 
\\	彼女はその贈り物を喜んだ。	かのじょはそのおくりものをよろこんだ 
\\	店員は贈り物を包んでくれた。	てんいんはおくりものをつつんでくれた 
\\	店員はその仕事にあまり骨を折らなかった。	てんいんはそのしごとにあまりほねをおらなかった 
\\	最後のわら一本がらくだの背骨を折る。	さいごのわらいっぽんがらくだのせぼねをおる 
\\	彼はジャズに偏見を持っている。	かれはジャズにへんけんをもっている 
\\	外国人労働者に偏見は持っていない。	がいこくじんろうどうしゃにへんけんはもっていない 
\\	労働者たちは腰まで裸だった。	ろうどうしゃたちはこしまではだかだった 
\\	この労働者達は道路を建設している。	このろうどうしゃたちはどうろをけんせつしている 
\\	新しい橋の建設が進行中だ。	あたらしいはしのけんせつがしんこうちゅうだ 
\\	頑固になればなるほど独立するよ。	がんこになればなるほどどくりつするよ 
\\	彼は自分の意見を頑固に主張した。	かれはじぶんのいけんをがんこにしゅちょうした 
\\	彼は自分が無実だと主張した。	かれはじぶんがむじつだとしゅちょうした 
\\	弁護士は依頼人の無罪を主張した。	べんごしはいらいじんのむざいをしゅちょうした 
\\	臆病なその男は恐怖に震えた。	おくびょうなそのおとこはきょうふにふるえた 
\\	臆病者は本当に死ぬ前に何度も死ぬ。	おくびょうしゃはほんとうにしぬまえになんどもしぬ 
\\	痩せたいなら間食は控えるべきだ。	やせたいならかんしょくはひかえるべきだ 
\\	私に返事をするのに彼女はややせっかちであった。	わたしにへんじをするのかのじょはややせっかちであった 
\\	せっかちな運転者が赤信号を無視して交差点を通りぬけた。	せっかちなうんてんしゃがあかしんごうをむししてこうさてんをとおりぬけた 
\\	できるだけ早い返事をお待ちしています。	できるだけはやいへんじをおまちしています 
\\	富があるにもかかわらず、彼はけちだ。	とみがあるにもかかわらず、かれはけちだ 
\\	具体的に話して欲しい。	ぐたいてきにはなしてほしい 
\\	彼に詳しい具体的な指示を与えてください。	かれにくわしいぐたいてきなしじをあたえてください 
\\	彼の考え方は具体的でも抽象的でもなかった。	かれのかんがえかたはぐたいてきでもちゅうしょうてきでもなかった 
\\	迷惑をおかけして申し訳ありません。	めいわくをおかけてもうしわけありません 
\\	ラジオをつけてもご迷惑ではないでしょうか。	ラジオをつけてもごめいわくではないでしょうか 
\\	もしご迷惑でなければ今晩お伺いしたいのですが。	もしごめいわくでわなければこんばんおうかがいしたいのですが 
\\	あの悲劇は私の心に刻み込まれた。	あのひげきはわたしのこころにきざみこまれた 
\\	その出来事は彼の記憶に刻み込まれた。	そのできごとはかれのきおくにきざみこまれた 
\\	彼は優れた記憶力の持ち主だ。	かれはすぐれたきおくりょくのもちぬしだ 
\\	私は記憶が不足している。	わたしはきおくがふそくしている 
\\	この機械の優れた点を言ってください。	このきかいのすぐれたてんをいってください 
\\	彼はアメリカの舞台で最も優れた俳優になった。	かれはアメリカのぶたいでもっともすぐれたはいゆうになった 
\\	中国語を話すことでは彼女は私より優れている。	ちゅごくごをはなすことではかのじょはわたしよりすぐれている 
\\	この服は品質においてあの服よりも優れていると思う。	このふくはひんしつにおいてあのふくよりもすぐれているとおもう 
\\	彼女は内気な性格の持ち主です。	かのじょはうちきなせいかくのもちぬしです 
\\	その家の持ち主は海外留学中だと思われている。	そのうちのもちぬしはかいがいりゅうがくちゅうとおもわれている 
\\	そのラジオのアナウンサーは男らしい声の持ち主だった。	そのラジオのアナウンサーはおとこらしいこえのもちぬしだった 
\\	彼の英語は誤りはあるにしてもきわめて少ない。	かれのえいごはあやまりはあるにしてもきわめてすくない 
\\	幸福の観念はきわめて抽象的だ。	こうふくのかんねんはきわめてちゅうしょうてきだ 
\\	彼には道徳観念が欠けている。	かれにはどうとくかんねんがかけている 
\\	わが国の政治家の道徳は腐敗した。	わがくにのせいじかのどうとくはふはいした 
\\	冷蔵庫は食品の腐敗を守る。	れいぞうこはしょくひんのふはいをまもる 
\\	今日私達は道徳の点からこの問題について話し合うつもりです。	きょうわたしたちはどうとくのてんからこのもんだいについてはなしあうつもりです 
\\	道徳的な人は嘘をついたり騙したり盗んだりしない。	どうとくてきなひとはうそをついたりだましたりぬすんだりしない 
\\	奥歯が欠けました。	おくばがかけました 
\\	少女は音楽的な才能に欠けていた。	かのじょはおんがくてきなさいのうにかけていた 
\\	彼は自らの野心の犠牲になった。	かれはみずからのやしんのぎせいになった 
\\	このダムは多くの命を犠牲にして造られた。	このダムはおおくのいのちをぎせいにしてつくられた 
\\	彼は政治的な野心を持っていない。	かれはせいふてきなやしんをもっていない 
\\	ほとんどの作家は批評に敏感である。	ほとんどのさっかはひひょうにびんかんである 
\\	あなたの批評はいつも私には有益でした。	あなたのひひょうはいつもわたしにはゆうえきでした 
\\	彼以外の皆はその批評家に率直な意見を感謝した。	かれいがいのみんなはそのひひょうかにそっちょくないけんをかんしゃした 
\\	彼は率直に私の欠点を指摘した。	かれはそっちょくにわたしのけってんをしてきした 
\\	先生は私の朗読にいくつかの誤りを指摘した。	せんせいはわたしのろうどくにいくつかのあやまりをしてきした 
\\	自分たちの欠点を自覚するべきです。	じぶんたちのけってんをじかくするべきです 
\\	彼は私たちな有益な知識を与えてくれた。	かれはわたしたちなゆうえきなちしきをあたえてくれた 
\\	その問題の重要性を徹底的に彼女に自覚させなければなりません。	そのもんだいのじゅうようせいをてっていてきにかのじょにじかくさせなければなりません 
\\	自分の欠点を認めようとする人はほとんどいない。	じぶんのけってんをみとめようとするひとはほとんどない 
\\	人生が不可解なものであることは私も認める。	じんせいがふかかいなものであることはわたしもみとめる 
\\	自分の過ちを認める政治家は、ほとんどいない。	じぶんのあやまちをみとめるせいじかは、ほとんどいない 
\\	我々は攻撃に備えた。	われわれはこうげきにそなえた 
\\	敵はその建物に対して激しい攻撃をした。	てきはそのたてものにたいしてはげしいこうげきをした 
\\	彼らは政党を組織した。	かれらはせいとうをそしきした 
\\	その組織の運営は自発的な寄付に依存している。	そのそしきのうんえいはじはつてきなきふにいぞんしている 
\\	日本は石油を外国に依存している。	にほんはせきゆをがいこくにいぞんしている 
\\	その産業は政府の資金援助に大きく依存している。	そのさんぎょうはせいふのしきんえんじょにおおきくいぞんしている 
\\	彼は他人の援助に頼った。	かれはたにんのえんじょにたよった 
\\	仕事を失っても、僕は貯金に頼ることができる。	しごとをうしなっても、ぼくはちょきんにたよることができる 
\\	経済をうまく運営できる政府は少ない。	けいざいをうまくうんえいできるせいふはすくない 
\\	彼女は自発的に皿洗いした。	かのじょはじはつてきにさらあらいした 
\\	学校で自発的にものを考えるのを教えるのは難しい。	がっこうでじはつてきにものをかんがえるのをおしえるのはむずかしい 
\\	彼はその病院に多額の寄付をした。	かれはそのびょういんにたがくのきふをした 
\\	君が寄付するお金は立派に使われるだろう。	きみがきふするおかねはりっぱにつかわれるだろう 
\\	車に撥ねられるところだった。	くるまにはねられるところだった 
\\	この情報をすぐに提供してくれ。	このじょうほうをすぐにていきょうしてくれ 
\\	彼らはしばらくの間私達にスタッフを提供してくれた。	かれらはしばらくのあいだわたしたちにスタフをていきょうしてくれた 
\\	父はめったに極端なことはしない。	ちちはめったにきょくたんなことはしない 
\\	一般的にイギリス人は飼っているペットを極端にかわいがっている。	いっぱんてきにイギリスじんはかっているペットをきょくたんにかわいがっている 
\\	一般的に言って日本車は海外で人気が高い。	いっぱんてきにいってにほんしゃはかいがいでにんきがたかい 
\\	一般的に言えば、西欧人は魚を生では食べない。	いっぱんてきにいえば、せいおうじんはさかなをなまでたべない 
\\	一般的にいえば、歴史は繰り返す。	いっぱんてきにいえば、れきしはくりかえす 
\\	繰り返すことが物事を覚えるのに役に立つ。	くりかえすことがものごとをおぼえるのにやくにたつ 
\\	生意気な人間は誰からも嫌われる。	なまいきなにんげんはだれからもきらわれる 
\\	私が彼女に求婚していたら、生意気な奴らが割り込んできた。	わたしはかのじょにきゅうこんしていたら、なまいきなやつらがわりこんできた 
\\	人の話に割り込むのは失礼だ。	ひとのはなしにわりこむのはしつれいだ 
\\	彼らは復讐として隣人の家に火をつけた。	かれらはふくしゅうとしてりんじんのいえにひをつけた 
\\	彼は父の殺されたのを復讐した。	かれはちちのころされたのをふくしゅうした 
\\	急に暗い空から大粒の雨が降り始めた。	きゅうにくらいそらからおおつぶのあめがふりはじめた 
\\	見渡す限り、砂以外何も見えない。	みわたすぶり、すないがいなにもみえない 
\\	乾いた砂は水を吸い込む。	かわいたすなはみずをすいこむ 
\\	私は庭の花の香りを胸一杯に吸い込んだ。	わたしはにわのはなのかおりをむねいっばいすいこんだ 
\\	その崖はほとんど垂直です。	そのがけはほとんどすいちょくです 
\\	その柱は垂直になっていない。	そのはしらはすいちょくになっていない 
\\	彼は紙のうえに垂直な線を何本か引いた。	かれはかみのうえにすいちょくなせんをなんほんかひいた 
\\	大きな家が全て住み心地がよいとは限らない。	おおきいないえがすべてすみここちよいとはかぎらない 
\\	ジョンに時計を直してもらうつもりだ。	ジョンにとけいをなおしてもらうつもりだ 
\\	彼女は気を取り直し、また話し始めた。	かのじょはきをとりなおし、またはなしはじめた 
\\	歯が抜けて隙間が空いていた。	ははぬけてすきまがあいていた 
\\	雨が屋根の隙間からぽたぽた落ちていた。	あめがやねのすきまからぽたぽたおちていた 
\\	屋根は爆発で吹き飛ばされた。	やねはばくはつでふきとばされた 
\\	道端には、白や黄色の花が咲いていました。	みちばたには、しろやきいろのはながさいていました 
\\	夏休みが延びたので子供達は喜んだ。	なつやすみがのびたのでこどもたちはよろこんだ 
\\	大きな岩が岸から川に突き出ていた。	おおきいいわがきしからかわにつきでていた 
\\	火山は炎と溶岩を吹き出す。	かざんはほのおとようがんをふきだす 
\\	私の自転車に乗っていて、その少年は大きな岩と衝突した。	わたしのじてんしゃにのっていて、そのしょうねんはおおきいないわとしょうとつした 
\\	2台の車は激しい音を立てて衝突した。	にだいのくるまははげしいおとをたててしょうとつした 
\\	算数では正確さが重要だ。	さんすうではせいかくさはじゅうようだ 
\\	江戸時代、武士は刀を2本刺していた。	えどじだい、ぶしはかたなをにほんさしていた 
\\	彼女は小麦粉と油を大量に買い込んだ。	かのじょはこむぎことあぶらをたいりょうにかいこんだ 
\\	漁師は釣り糸を水中に投げた。	りょうしはつりいとをすいちゅうになげた 
\\	この材質は弾力性に欠ける。	このざいしつはだんりょくせいにかける 
\\	天井に手が届く。	てんじょうにてがとどく 
\\	その装置は天井にしっかりと固定されている。	そのそうちはてんじょうにしっかりとこていされている 
\\	書棚は壁に固定したほうがいい。	しょだなはかべにこていしたほうがいい 
\\	その未亡人は胃ガンを痛んでいた。	そのみぼうじんはいがんをいたんでいた 
\\	何回も失敗したが、彼は再度やってみようとした。	なんかいもしっぱいしたが、かれはさいどやってみようとした 
\\	彼は約束を守らなかったことを私たちに詫びた。	かれはやくそくをまもらなかったことをわたしたちにおわびた 
\\	ご注文の品が破損していたとのことで、お詫び申し上げます。	ごちゅうもんのひんがはそんしていたのことで、おわびもうしあげます 
\\	その音が睡眠を妨げた。	そのおとがすいみんをさまたげた 
\\	風邪には睡眠が最良の薬です。	かぜにはすいみんがさいりょうのくすりです 
\\	彼は昼寝をして睡眠不足を補おうとした。	かれはひるねをしてすいみんふそくをおぎなおうとした 
\\	われわれの進歩を妨げる障害がやっと取り除かれた。	われわれのしんぽをさまたげるしょうがいがやっととりのぞかれた 
\\	彼の思想は進歩的だ。	かれのしそうはしんぽてきだ 
\\	それは昨年のと比べると格段の進歩だ。	それはさくねんのとくらべるとかくだんのしんぽだ 
\\	この蛇は触っても安全ですか。	このへびはさわってもあんぜんですか 
\\	彼らは新事実によって議論を発展させた。	かれらはしんじじつによってぎろんをはってんさせた 
\\	彼らの小さな抗議が大衆デモに発展した。	かれらのちいさなこうぎがたいしゅうデモにはってんした 
\\	群集は妊娠中絶に抗議した。	ぐんしゅうはにんしんちゅうぜつにこうぎした 
\\	これは妊娠中絶に関する社会学的研究である。	これはにんしんちゅうぜつにかんするしゃかいがくてきけんきゅうである 
\\	彼女は腰掛けて足を組んだ。	かのじょはこしかけてあしをくんだ 
\\	恋人たちはお互いに腕を組んで歩いていた。	こいびとたちはおたがいにうでをくんであるいていた 
\\	天気はどうやら回復しそうだ。	てんきはどうやらかいふくしそうだ 
\\	どうやら電車の中で傘を置き忘れてきたらしい。	どうやらでんしゃのなかでかさをおきわすれてきたらしい 
\\	吹雪で視界が効かなかった。	ふぶきでしかいがきかなかった 
\\	列車が視界から消えた。	れっしゃがしかいからきえた 
\\	高い松の木が湖の周囲を取り囲んでいる。	たかいまつのきがみずうみのしゅういをとりかこんでいる 
\\	谷は滝の音を反響する。	たにはたきのおとをはんきょうする 
\\	滝のような汗が顔から流れ始めた。	たきのようなあせがかおからながれはじめた 
\\	洞窟からうつろに反響する音を耳にした。	どうくつからうつろにはんきょうするおとをみみにした 
\\	車がぬかるみに填まり込んだ。	くるまはぬかるみにはまりこんだ 
\\	もう畳の上に座るのには慣れました。	もうたたみのうえにすわるのにはなれました 
\\	彼は乱暴な扱いを受けた。	かれはらんぼうなあつかいをうけた 
\\	この車は乱暴な使い方をされてきたに違いない。	このくるまはらんぼうなつかえかたをされたにちがいない 
\\	内緒の話だが僕は近々仕事を辞める予定だ。	ないしょのはなしだがぼくはちかぢかしごとをやめるよていだ 
\\	内緒の話ですが、彼は収賄のために免職になったのです。	ないしょのはなしですが、かれはしゅうわいのためにめんしょくになったのです 
\\	彼がその贈り物を受け取ったのは収賄とみなされた。	かれがそのおくりものをうけとったのはしゅうわいとみなされた 
\\	爪を噛むのはよしなさい。	つめをかむのはよしなさい 
\\	ケンは年末ごろには大きくなって兄さんの服が着られるでしょう。	ケンはねんまつごろにはおおきくなってにいさんのふくがきられるでしょう 
\\	私と仲直りしてくれませんか。	わたしとなかなおりしてくれませんか 
\\	夫婦は喧嘩をしたがすぐに仲直りをした。	ふうふはけんかをしたがすぐになかなおりをした 
\\	典型的なベッドタウンで、昼間においても人通りが少ない。	てんけいてきなベッドタウンで、ひるまにおいてもひとどおりがすくない 
\\	すぐそこでそばを立ち食いしてきた。	すぐそこでそばをたちぐいしてきた 
\\	この図書館では1度に3冊まで借りられます。	このとしょかんでいちどにさんさつまでかりられます 
\\	廊下は滑りやすいので、足元に気を付けなさい。	ろうかはすべりやすいので、あしもとにきをつけなさい 
\\	我はぬかるんだ斜面をずるずると滑り下りた。	われはぬかるんだしゃめんをずるずるとすべりおりた 
\\	偶然、廊下でいじめを目撃した。	ぐうぜん、ろうかでいじめをもくげきした 
\\	水夫は仲間の水夫が力尽きて沈むのを目撃した。	すいふはなかまのすいふがちからつきてしずむのをもくげきした 
\\	彼は険しい斜面をじっと見た。	かれはけわしいしゃめんをじっとみた 
\\	険しいその道が国境へ行く唯一の方法だ。	けわしいそのみちがこっきょうへいくゆいつのほうほうだ 
\\	徳の唯一の報酬は徳である。	とくのゆいつのほうしゅうはとくである 
\\	私のイヌを見つけた人には報酬が出ます。	わたしのイヌをみつけたひとにはほうしゅうがでます 
\\	彼女は昨晩私達に中華料理を作ってくれた。	かのじょはさくばんわたしたちにちゅうかりょうりをつくってくれた 
\\	ハトもダチョウも共に鳥だが、前者は飛べるし、後者は飛べない。	ハトもダチョウもともにとりだが、ぜんしゃはとべるし、こうしゃはとべない 
\\	これら二つの意見のうち前者よりも後者のほうがよい。	これらふたつのいけんのうちぜんしゃよりもこうしゃのほうがよい 
\\	いくつかの点で、前者は後者よりも劣っていると彼は指摘した。	いくつかのてんで、ぜんしゃはこうしゃよりもおとっているとかれはしてきした 
\\	海賊達には降参する以外に道はなかった。	かいぞくたちにはこうさんするいがいにみちはなかった 
\\	敵はもうそれ以上抵抗せずに降参した。	てきはもうそれいじょうていこうせずにこうさんした 
\\	その人々はひどい支配者に抵抗した。	そのひとびとはひどいしはいしゃにていこうした 
\\	誘惑に抵抗することは難しい。	ゆうわくにていこうすることはむずかしい 
\\	若い人たちは誘惑に陥りやすい。	わかいひとたちはゆうわくにおちいりやすい 
\\	絶対に誘惑には負けないぞ。	ぜったいにゆうわくにはまけないぞ 
\\	私は彼を許せない、なぜなら彼は私を人前で侮辱したのだから。	わたしはかれをゆるせない、なぜならかれはわたしをひとまえでぶじょくしたのだから 
\\	彼は本気でそう言ってるのでない、芝居をしているだけだ。	かれはほんきでそういってるのでない、しばいをしているだけだ 
\\	エドワードはキャシーの目をじっと見て本気でそう言っているのか尋ねた。	エドワードはキャシーのめをじっとみてほんきでそういっているのかたずねる 
\\	私は首尾よく山頂に到達できた。	わたしはしゅびよくさんちょうにとうたつできた 
\\	彼女は遂に北極に到達した。	かのじょはついにほっきょくにとうたつした 
\\	二人は同じ結論に到達した。	ふたりはおなじけつろんにとうたつした 
\\	彼のほめ言葉に彼女の頬が赤くなりだした。	かれのhめことばにかのじょのほほがあかくなりだした 
\\	彼は倉庫の警備人として仕事をしている。	かれはそうこのけいびにんとしてしごとしている 
\\	倉庫には家具が一つの他には何もなかった。	そうこにはかぐがひとつのほかにはなにもなかった 
\\	私は床と家具を磨いた。	わたしはゆかとかぐをみがいた 
\\	靴をブラシで磨く必要がある。泥で汚れているから。	くつをブラシでみがくひつようがある。どろでよごれているから 
\\	警戒していた警備員が遠くのぼんやりとした影に気づいた。	けいかいしていたけいびいんがとおくのほんやりとしたかげにきづいた 
\\	兵士たちは警戒しながら国境へ向かった。	へいしたちはけいかいしながらこっきょうへむかった 
\\	ストライキは石炭の価格に影響を与えますか。	ストライキはせきたんのかかくにえいきょうをあたえますか 
\\	我々は環境に影響される。	われわれはかんきょうにえいきょうされた 
\\	学生達は教師の影響を受けやすい。	がくせいたちはきょうしのえいきょうをうけやすい 
\\	彼の新しい本が彼の名声を増した。	かれのあたらしいほんがかれのめいせいをました 
\\	この国のいくつかの大学は非常に名声が高い。	このくにのいくつかのだいがくはひじょうにめいせいがたかい 
\\	私は彼の損なわれた名声を取り戻そうとした。	わたしはかれのそこなわれためいせいをとりもどそうとした 
\\	健康を取り戻すのに丸一年かかった。	けんこうをとりもどすのにまるいちねんかかった 
\\	梅雨の晴れ間に洗濯物を干すと気分はもう夏でした。	つゆのはれまにせんたくものをほすときぶんはもうなつでした 
\\	天気がよいときには寝具を干しなさい。	てんきがよいときにはしんぐをほしなさい 
\\	彼はビールの大ジョッキを飲み干した。	かれはビールのだいジョッキをのみほした 
\\	私はその車を10%の割引で買った。	わたしはそのくるまを10パーセントわりびきでかった 
\\	今では多くの店や会社で、割引が老人に与えられている。	いまでおおくのみせやかいしゃで、わりびきがろうじんにあたえられている 
\\	彼らは手の空いた時間の一部を病人の介護にあてている。	かれらはてのすいたじかんのいちぶをびょうにんのかいごにあてている 
\\	自動ドアが開き、トムは乗り込んだ。	じどうドアがあき、トムはのりこんだ 
\\	私が乗り込むやいなや、汽車は動き出した。	わたしはのりこむやいなや、きしゃはうごきだした 
\\	つり革につかまりなさい。電車がすぐ動き出すだろうから。	つりかわにつかまりなさい。でんしゃがすぐうごきだすだろうから 
\\	基本的な本能は消えることはない。	きほんてきなほんのうはきえることはない 
\\	彼は闘争本能が強い。	かれはとうそうほんのうがつよい 
\\	彼の無知が我々の進歩を妨げた。	かれのむちがわれわれのしんぽをさまたげた 
\\	その国の文明は進歩した。	そのくにのぶんめいはしんぽした 
\\	ぬかるみ道で新しい靴が台無しになった。	ぬかるみみちであたらしいくつがだいなしになった 
\\	彼は歴史的な航海に出かけた。	かれはれきしてきなこうかいにでかけた 
\\	風やら雨やらで旅行は台無しだった。	かぜやらあめやらでりょこうはだいなしだった 
\\	とても高価な陶器がめちゃめちゃに割れてしまった。	とてもこうかなとうきがめちゃめちゃにわれてしまった 
\\	箱の中の卵はみな割れていた。	はこのなかのたまごはみなわれていた 
\\	湖は凍っていたので氷の上を歩いて渡った。	みずうみはこおっていたのでこおりのうえをあるいてわたった 
\\	通りが凍っていたので、車を走らせることが出来なかった。	とおりがこおっていたので、くるまをはしらせることができなかった 
\\	メアリーには常識が欠けている。	メアリーはじょうしきがかけている 
\\	この机は足が一本欠けている。	このつくえはあしがいっぽんかけている 
\\	彼は判断力に欠けている。	かれははんだんりょくにかけている 
\\	常識を伴わない知識は何の役にも立たない。	じょうしきをともなわないちしきはなんのやくにもたたない 
\\	権力には責任が伴う。	けんりょくにはせきにんがともなう 
\\	経験を積めば君にも常識がわかるだろう。	けいけんをつめばきみにもじょうしきがわかるだろう 
\\	経験を積むにつれて更に知恵が身につく。	けいけんをつむにつれてさらにちえがみにつく 
\\	君は指示に従う事を身につけなければいけない。	きみはしじにしたがうことをみにつけなければいけない 
\\	会議の議題が配布された。	かいぎのぎだいがはいふされた 
\\	会議の議題に変更がありましたので、ご注意下さい。	かいぎのぎだいにへんこうがありますので、ごちゅいください 
\\	突然の計画変更に面食らった。	とつぜんのけいかくへんこうにめんくらった 
\\	その規則は絶対変更できない。	そのきそくはぜったいへんこうできない 
\\	規則に違反してはいけない。	きそくにいはんしてはいけない 
\\	私の友人はスピード違反で逮捕された。	わたしのゆうじんはスピードいはんでたいほされた 
\\	彼は脱税容疑で逮捕された。	かれはぜいきんようぎでたいほされた 
\\	二人の刑事が容疑者をつけた。	ふたりのけいじがようぎしゃをつけた 
\\	裁判は三時間続いた。	さいばんはさんじかんつづいた 
\\	裁判官には慎重さがなくてはならない。	さいばんかんにはしんちょうさがなくてはならない 
\\	彼は裁判官に慈悲を求めた。	かれはさいばんかんにじひをもとめた 
\\	彼女は慈悲を懇願した。	かのじょはじひをこんがんした 
\\	彼女は目に懇願の表情を浮かべていた。	かのじょのめにこんがんのひょうじょうをうかべていた 
\\	その文章は全ての部長に配布された。	そのぶんしょうはすべてのぶちょうにはいふされた 
\\	有名な建築家がこの家を建てた。	ゆうめいなけんちくかがこのいえをたてた 
\\	その遊園地を建築するのに10年かかった。	そのゆうえんちをけんちくするのに10ねんかかった 
\\	東京には醜悪な建築物がたくさん見られる。	とうきょうにはしゅうあくなけんちくぶつがたくさんみられる 
\\	市の地区の建物はみな醜悪だ。	しのちくのたてものはみなしゅうあくだ 
\\	遠方にほの白い灯台が立っていた。	えんぽうにほのしろいとうだいがたてえいた 
\\	はるか遠方に稲妻が走るのが見えた。	はるかえんぽうにいなずまがはしるのがみえた 
\\	彼は私よりも遥かに流暢に英語を話すことが出来る。	かれはわたしよりもはるかにりゅうちょうにえいごをはなすことができる 
\\	1年以内にあなた方が全員、流暢な英語を話しているようにしてあげます。	いちねんいないにあなたがたがぜんいん、りゅうちょうなえいごをはなしているようにしてあげます 
\\	彼は貿易で財産を築いた。	かれはぼうえきでざいさんをきずいた 
\\	健全な経済には国際貿易が必要である。	けんぜんなけいざいにはこくさいぼうえきがひつようである 
\\	麻薬汚染の問題は国際的である。	まやくおせんのもんだいはこくさいてきである 
\\	全体としてみればその国際会議は成功だった。	ぜんたいとしてみればそのこくさいかいぎはせいこうだった 
\\	言葉は国際結婚がかかえている基本的な問題である。	ことばはこくさいけっこんがかかえているきほんてきなもんだいである 
\\	これが戦後移民に関する中心的問題である。	これがせんごいみんにかんするちゅうしんてきもんだいである 
\\	日本は戦後繁栄を享受している。	にほんはせんごはんえいをきょうじゅしている 
\\	戦後は帽子をかぶる人が少なくなっている。	せんごはぼうしをかぶるひとがすくなくなっている 
\\	ドアにいちばん近い駐車場所は重役専用です。	ドアにいちばんちかいちゅうしゃばしょはじゅうやくせんようです 
\\	そう、専用浴室付きのシングルにしてください。	そう、せんようよくしつつきのシングルにしてください 
\\	新しい水路を作るために岩が爆破された。	あたらしいすいろをつくるためにいわがばくはされた 
\\	馬車は今では完全に時代遅れだ。	ばしゃはいまでかんぜんにじだいおくれだ 
\\	道具なしでは職人は何の役に立とうか。	どうぐなしではしょくにんはなんのやくにたとうか 
\\	その鉄道は今建設中だ。	そのてつどうはいまけんせつちゅうだ 
\\	当時、日本には鉄道が無かった。	とうじ、にほんにはてつどうがなかった 
\\	私はその高級なレストランで場違いな感じがした。	わたしはそのこうきゅうなレストランでばちがいなかんじがした 
\\	彼は高級車を持っているのを自慢している。	かれはこうきゅうしゃをもっているのをじまんしている 
\\	自由の女神はアメリカの象徴である。	じゆうのめがみはアメリカのしょうちょうである 
\\	彼女の服は儀式的な集まりのなかで場違いです。	かのじょのふくはぎしきてきなあつまりのなかでばちがいです 
\\	教会では儀式が執り行われた。	きょうかいではぎしきがとりおこなわれた 
\\	彼女は料理の腕を自慢している。	かのじょはりょうりのうでをじまんしている 
\\	彼は自慢するけれども、臆病者だ。	かれはじまんするけれども、おくびょうものだ 
\\	彼はその仕事を完成しようと努力した。	かれはそのしごとをかんせいしようとどりょくした 
\\	彼が望んでいたのは絵を完成させる時間だけだった。	かれはのぞんでいたのはえをかんせいさせるじかんだけだった 
\\	行いは言葉より影響が大きい。	おこないはことばよりえいきょうがおおきい 
\\	彼の意見には強い影響力がある。	かれのいけんにはつよいえいきょうりょくがある 
\\	社会は個人に大きな影響を与える。	しゃかいはこじんにおおきいえいきょうをあたえる 
\\	引力は宇宙のあらゆる物に影響を与える。	いんりょくはうちゅうのあらゆるものにえいきょうをあたえる 
\\	海に住む大半の生物は汚染による悪影響を受けている。	うみにすむたいはんのせいぶつはおせんによるあくえいきょうをうけている 
\\	霜は農作物に悪影響を及ぼしました。	しもはのうさくぶつにあくえいきょうをおよぼしました 
\\	彼は医者に禁酒するようにいわれた。	かれはいしゃにきんしゅするようにいわれた 
\\	私は禁酒を誓ったが、結局次の週からまた飲み始めた。	わたしはきんしゅをちかったがけっきょくつぎのしゅうからまたのみはじめた 
\\	医師の警告で禁酒の決意が固くなった。	いしのけいこくできんしゅのけついがかたくなった 
\\	父と話し合って、転職を決意した。	ちちとはなしあって、てんしょくをけついした 
\\	彼が引退を決意したことは私たちみんなを驚かせた。	かれがいんたいをけついしたことはわたしたちみんなをおどろかせた 
\\	彼は健康を害したので引退した。	かれはけんこうをがいしたのでいんたいした 
\\	私は彼への忠誠を誓った。	わたしはかれへのちゅうせいをちかった 
\\	国旗に向かって忠誠を誓う。	こっきにむかってちゅうせいをちかう 
\\	彼女は草を編んで籠を作った。	かのじょはくさをあんでかごをつくった 
\\	問題はどこにテントを張るのかだった。	もんだいはどこにテントをはるのだった 
\\	ロープをつかみなさい、引っ張り上げてあげるから。	ロープをつかみなさい、ひっぱりあげてあげるから 
\\	彼は度胸を据えて外人に話し掛けた。	かれはどきょうをすえてがいじんにはなしかけた 
\\	公園を散歩していた時、私は老夫婦に話し掛けられた。	こうえんをさんぽしていたとき、わたしはろうふうふにはなしかけられた 
\\	私は好みに合わせてアパートを飾るのが好きです。	わたしはこのみにあわせてアパートをかざるのがすきです 
\\	通りは旗で飾られていた。	とおりははたでかざられていた 
\\	彼女は宝石で身を飾った。	かのじょはほうせきでみをかざった 
\\	その木は30メートル間隔で植えられている。	そのきは30メートルかんかくでうえられている 
\\	通り沿いに木が植えられている。	とおりぞいにきがうえられている 
\\	その部族の民はその川沿いに住み着いた。	そのぶぞくのたみはそのかわぞいにすみついた 
\\	街道沿いに家が並んでいた。	かいどうぞいにいえがならんでいた 
\\	私は自分の考えをまとめて本にしてみるつもりだ。	わたしはじぶんのかんがえをまとめてほんにしてみるつもりだ 
\\	ラジオで地震の警告をしたので、私たちは荷物をまとめ始めた。	ラジオでじしんのけいこくをしたので、わたしはもちものをまとめはじめた 
\\	彼女は夕食後食卓の上を片付けた。	かのじょはゆうしょくごしょくたくのうえをかたづけた 
\\	彼女の誘惑に勝てない。	かのじょのゆうわくにかてない 
\\	遂に彼女は誘惑に負けてケーキを全部食べた。	ついにかのじょはゆうわくにまけてケーキをぜんぶたべた 
\\	彼は誘惑に屈し、麻薬に手を出してしまいました。	かれはゆうわくにくっし、まやくにてをだしてしまいました 
\\	彼は狂人のように振る舞った。	かれはきょうじんのようにふるまった 
\\	天才と狂人の差は紙一重。	てんさいはきょうじんのさはかみひとえ 
\\	哀れみと愛情は紙一重。	あわれみとあいじょうはかみひとえ 
\\	彼がしたことは狂気の沙汰としか言いようがなかった。	かれがしたことはきょうきのさたとしかいいようがなかった 
\\	廊下は大変混雑していたので歩けなかった。	ろうかはたいへんこんざつしていたのであるけなかった 
\\	混雑した電車の中ではスリにご用心ください。	こんざつしたでんしゃのなかではスリにようじんください 
\\	彼の希望は無残に砕かれた。	かれのきぼうはむざんにくだかれた 
\\	波は岩に勢いよく当たって砕けた。	なみはいわにいきおいよくあたってくだけた 
\\	布を斜めに裁ちなさい。	ぬのをななめにたちなさい 
\\	学生達は今休憩時間中だ。	がくせいたちはいまきゅうけいじかんちゅうだ 
\\	景色のいいところで車をとめて、休憩しよう。	けしきのいいところでくるまをとめて、きゅうけいしよう 
\\	明日から5連休だから、みんなルンルン気分だね。	あしたから5れんきゅうだから、みんなルンルンきぶんだね 
\\	心理学は人間の感情を扱う。	しんりがくはにんげんのかんじょうをあつかう 
\\	彼は情熱に押し流された。	かれはじょうねつにおしながされた 
\\	年とともに、情熱は弱まり、習慣は強くなる。	としとともに、じょうねつはよわまり、しゅうかんはつよくなる 
\\	彼は自分のばかげた行為を正当化しようとした。	かれはじぶんのばかげたこういをせいとうかしようとした 
\\	私が疲れるのは、暑さというよりはむしろ湿度のせいだ。	わたしがつかれるのは、あつさというよりむしろしつどのせいだ 
\\	その事件に関する情報には賞金が出されている。	そのじけんにかんするじょうほうにはしょうきんがだされている 
\\	その賞金で私は世界一周の航海をすることが出来た。	そのしょうきんでわたしはせかいいっしゅうのこうかいをすることができた 
\\	支配人が外出中だったので、私は彼の秘書に伝言を残した。	しはいにんががいしゅつちゅうだったので、わたしはかれのひしょにでんごんをのこした 
\\	あなたに緊急の伝言が入っています。	あなたにきんきゅうのでんごんがはいっています 
\\	フランクは暗号による伝言を残した。	フランクはあんごうによるでんごんをのこした 
\\	母は留守です。	はははるすです 
\\	留守番しててね。	るすばんしててね 
\\	午後は家を留守にします。	ごごはいえをるすにします 
\\	私の留守中に泥棒に入られた。	わたしのるすちゅうにどろぼうにはいられた 
\\	私が訪ねるたびあなたは留守だ。	わたしがたずねるたびあなたはるすだ 
\\	具合悪いの?	ぐあいわるいの? 
\\	パーティの進み具合は。	パーティのすすみぐあいは 
\\	私は歳を取って体の具合が良くない。	わたしはとしをとってからだのぐあいがよくない 
\\	私の時計は月に30秒進む。	わたしのとけいはげつに30びょうすすむ 
\\	私は30秒の差で電車に乗り遅れた。	わたしは30びょうのさででんしゃにのりおくれた 
\\	急がないと、電車に乗り遅れるでしょう。	いそがないと、でんしゃにのりおくれるでしょう 
\\	彼は知識の宝庫だ。	かれはちしきのほうこだ 
\\	百科事典は知識の宝庫だ。	ひゃっかじてんはちしきのほうこだ 
\\	彼はいわば歩く百科事典だ。	かれはいわばあるくひゃっかじてんだ 
\\	この百科事典は検索に便利である。	このひゃっかじてんはけんさくにべんりである 
\\	これはかさばるから宅配便で送ろう。	これはかさばるからたくはいびんでおくろう 
\\	兵隊が橋を見張っていた。	へいたいがはしをみはっていた 
\\	この木は深くまで根が張っている。	このきはふかくまでねがはっている 
\\	彼は英雄だったが、威張ってなかった。	かれはえいゆうだったが、いばってなかった 
\\	母親は息子を引っ張って立たせた。	ははおやはむすこをひっぱってたたせた 
\\	新しい仕事で彼は気が張っていた。	あたらしいしごとでかれはきがはっていた 
\\	あの人はあまり威張るから好きになれない。	あのひとはあまりいばるからすきになれない 
\\	白いワインは出す前に冷やす方がよい。	しろいワインはだすまえにひやすほうがよい 
\\	冷やしたのをください。	ひやしたのをください 
\\	病人の頭を氷で冷やした。	びょうにんのあたまをこおりでひやした 
\\	お行儀はどうしたの?	おぎょうぎはどうしたの? 
\\	彼は良識ある人です。	かれはりょうしきあるひとです 
\\	彼は贅沢な生活を送った。	かれはぜいたくなせいかつをおくった 
\\	雰囲気がいやだった。	ふんいきがいやだった 
\\	子供は家族の雰囲気を映し出す。	こどもはかぞくのふんいきをうつしだす 
\\	月が湖に映し出されていた。	つきがみずうみにうつしだされていた 
\\	私は骨の髄まで冷えた。	わたしはほねのずいまでひえた 
\\	お皿を流しに置いてもらえますか。	おさらをながしにおいてもらえますか 
\\	いつ私たちを襲うかもしれない。	いつわたしたちをおそうかもしれない 
\\	トラは空腹の時は人を襲うものだ。	トラはくうふくのときはひとをおそうものだ 
\\	秋にはいくつかの台風が日本を襲う。	あきにはいくつかのたいふうがにほんをおそう 
\\	強盗が昨夜銀行を襲った。	ごうとうがさくやぎんこうをおそった 
\\	蜂が容赦なく襲ってきた。	はちがようしゃなくおそってきた 
\\	熱帯の太陽は容赦なくぎらぎら照り付けた。	ねったいのたいようはようしゃなくぎらぎらてりつけた 
\\	首や背に太陽が暑く照りつけた。	くびやせにたいようがあつくてりつけた 
\\	彼は熱帯魚を飼っている。	かれはねったいぎょをかっている 
\\	熱帯雨林が心配の種である。	ねったいうりんがしんぱいのたねである 
\\	今言ったことをどうかご容赦ください。	いまいったことをどうかごようしゃください 
\\	さて次の話題に移るもとにしよう。	さてつぎのわだいにうつるもとにしよう 
\\	彼女は話題を変えた。	かのじょはわだいをかえた 
\\	彼は話題の豊富な人だ。	かれはわだいのほうふなひとだ 
\\	食べ物は豊富にある。	たべものはほうふにある 
\\	あの国は石油が豊富だ。	あのくにはせきゆがほうふだ 
\\	手元に豊富な資金がある。	てもとにほうふなしきんがある 
\\	この本を手元に置いてください。	このほんをてもとにおいてください 
\\	信念を行動に移した。	しんねんをこうどうにうつした 
\\	彼は机を右に移動した。	かれはつくえをみぎにいどうした 
\\	母は庭へ花を移植した。	はははにわへはなをいしょくした 
\\	アメリカは移民の国である。	アメリカはいみんにくにである 
\\	工場は北海道へ移転した。	こうじょうはほっかいどうへいてんした 
\\	新しいビルへの移転にわくわくしてます。	あたらしいびるへのいてんにわくわくしてます 
\\	この旅館は露天風呂がすばらしいのよ。	このりょかんはろてんぶろがすばらしいのよ 
\\	我々は彼を代表に指名した。	われわれはかれをだいひょうにしめいした 
\\	彼らは国連のインド代表だった。	かれらはこくれんのインドだいひょうだった 
\\	国連はその国の選挙を監視した。	こくれんはそのくにのせんきょをかんしした 
\\	国連は一つの国際的機能である。	こくれんはひとつのこくさいてききのうである 
\\	私は議長に指名された。	わたしはぎちょうにしめいされた 
\\	先生は私たちを順番に指名した。	せんせいはわたしたちをじゅんばんにしめいした 
\\	順番をお待ち下さい。	じゅんばんをおまちください 
\\	私達は順番に自己紹介をした。	わたしたちはじゅんばんにじこしょうかいをした 
\\	彼は順番を間違えてカードを出してしまった。	かれはじゅんばんをまちがえてカードをだしてしまった 
\\	彼は選挙で大勝した。	かれはせんきょでたいしょうした 
\\	我々は選挙で選ばれます。	われわれはせんきょでえらばれます 
\\	彼は選挙で苦戦している。	かれはせんきょでくせんしている 
\\	私は監視されているのに気づいていた。	わたしはかんしされているのにきづいていた 
\\	彼は贅沢な暮らしをしている。	かれはぜいたくなくらしをしている 
\\	彼は質素に暮らした。	かれはしっそにくらした 
\\	彼は安楽に暮らしている。	かれはあんらくにくらしている 
\\	彼らは質素な服装をしていた。	かれらはしっそなふくそうをしていた 
\\	私達は質素な食物を食べるのに慣れている。	わたしたちはしっそなたべものをたべるのになれている 
\\	この文の意味は曖昧だ。	このぶんのいみはあいまいだ 
\\	この湖はこの箇所が一番深い。	このみずうみはこのかしょがいちばんふかい 
\\	私はスキー中に2箇所も骨折した。	わたしはスキーちゅうに2かしょもこっせつした 
\\	彼は左腕を骨折した。	かれはひだりうでをこっせつした 
\\	私は全従業員に絶対的な忠誠を求めます。	わたしはぜんじゅうぎょういんにぜったいてきなちゅうせいをもとめます 
\\	状況をどう分析しますか。	じょうきょうをどうぶんせきしますか 
\\	彼は地位に満足だ。	かれはちいにまんぞくだ 
\\	彼は彼女より地位が低い。	かれはかのじょよりちいがひくい 
\\	高級車は地位の象徴である。	こうきゅうしゃはちいのしょうちょうである 
\\	鳩は平和を象徴する。	はとはへいわをしょうちょうする 
\\	ローマ人は彼らの女神であるジューノを崇拝した。	ローマじんはかれらのめがみであるジューノをすうはいした 
\\	私は彼女を心から崇拝している。	わたしはかのじょをこころからすうはいしている 
\\	彼らは彼を英雄として崇拝した。	かれらはかれをえいゆうとしてすうはいした 
\\	ギリシア人はいくつもの神を崇拝した。	ギリシアじんはいくつものかみをすうはいした 
\\	日本では看護婦さんは社会的地位が高いのですか。	にほんではかんごふさんはしゃかいてきちいがたかいのですか 
\\	この刀は不思議ないわれがある。	このかたなはふしぎないわれがある 
\\	彼は鋼を鍛えて刀を作った。	かれははがねをきたえてかたなをつくった 
\\	若いうちに体を鍛えなさい。	わかいうちにからだをきたえなさい 
\\	この鋼は錆びない。	このはがねはさびない 
\\	その問題に別な取り組み方をしてみよう。	そのもんだいにべつなとりくみかたしてみよう 
\\	彼女は拳でテーブルをドンと叩いた。	かのじょはこぶしでテーブルをドンとたたいた 
\\	握った拳はストレスを示すこともある。	にぎったこぶしはストレスをしめすこともある 
\\	戸を叩く音がした。	とをたたくおとがした 
\\	少年がその太鼓を叩いていた。	しょうねんがそのたいこをたたいていた 
\\	彼らは太鼓をずっとたたき続けた。	かれらはたいこをずっとたたきつづけた 
\\	私たちはその男を叩きのめした。	わたしたしはそのおとこをたたきのめした 
\\	相手は4人だ。叩きのめされるぞ。	あいてはよにんだ。たたきのめされるぞ 
\\	猫に石を投げるな。	ねこにいしをなげるな 
\\	海に縄を投げ入れた。	うみになわをなげいれた 
\\	私は新聞を投げ出した。	わたしはしんぶんをなげだした 
\\	彼は石を池に投げた。	かれはいしをいけになげた 
\\	彼はその大男を投げ倒した。	かれはそのおおおとこをなげたおした 
\\	斧で木を切り倒す。	おのできをきりたおす 
\\	ベルは電話を発明した。	ベルはでんわをはつめいした 
\\	商売を営む。	しょうばいはいとなむ 
\\	商売は下り坂である。	しょうばいはくだりざかである 
\\	道が急に下り坂になる。	みちはきゅうにくだりざかになる 
\\	午後から天気は下り坂です。	ごごからてんきはくだりざかです 
\\	道は緩い下り坂になっている。	みちはゆるいくだりざかになっている 
\\	この靴は少し緩い。	このくつはすこしゆるい 
\\	体重が随分減ったのでズボンがとても緩い。	たいじゅうがずいぶんへったのてズボンがとてもゆるい 
\\	遊びが増えれば勉強する時間が減ることになるだろう。	あそびがふえればべんきょうするじかんがへることになるだろう 
\\	タクシーの数がこんなに多くなければ、交通事故はもっと減るだろう。	タクシーのかずがこんなにおおくなければ、こうつうじこはもっとへるだろう 
\\	デニスには粗暴なところがない。	デニスにはそぼうなところがない 
\\	彼は粗暴に見えるが、根は大変優しい。	かれはそぼうにみえるが、ねはたいへんやさしい 
\\	彼は運転が荒い。	かれはうんてんがあらい 
\\	彼女は言葉使いが荒い。	かのじょはことばつかいがあらい 
\\	荒野を開拓する。	あれのをかいたくする 
\\	開拓者は多くの危険に出会った。	かいたくしゃはおおくのきけんにであった 
\\	何日間も誰にも会わずに荒野を歩き回ることができますか。	なんにちかんもだれにもあわずにあらのをあるきまわることができますか 
\\	その党は急激にのびた	そのとうはきゅうげきにのびた 
\\	株価の急激な下落があった。	かぶかのきゅうげきなげらくがあった 
\\	この街は急激な変化を遂げた。	このまちはきゅうげきなへんかをとげた 
\\	世界の人口は急激に増加している。	せかいのじんこうはきゅうげきにぞうかしている 
\\	19世紀には移民の数が急激に増大した。	19せいきにはいみんのかずがきゅうげきにぞうだいした 
\\	今日で5日連続の株価下落だ。	きょうでいつかれんぞくのかぶかげらくだ 
\\	3日連続して雨が降った。	みっかれんぞくしてあめがふった 
\\	チームは5年間連続して優勝した。	チームはごねんかんれんぞくしてゆうしょうした 
\\	一日かかってもやり遂げる。	いちにちかかってもやりとげる 
\\	努力をすれば何事も成し遂げることができる。	どりょくをすればなにごともなしとげることができる 
\\	素早く問題に対処する。	すばやくもんだいにたいしょする 
\\	彼女はストレスに対処できない。	かのじょはストレスにたいしょできない 
\\	そのケースは冷静に対処する必要がある。	そのケースはれいせいにたいしょするひつようがある 
\\	犯罪が増加している。	はんざいがぞうかしている 
\\	殺人事件が増加してきている。	さつじんじけんがぞうかしてきている 
\\	騒音の増加に気付かなかった。	そうおんのぞうかにきづかなかった 
\\	会社の収益は飛躍的に増加した。	かいしゃのしゅうえきはひやくてきにぞうかした 
\\	旅行会社の収益が急増した。	りょこうかいしゃのしゅうえきがきゅうぞうした 
\\	投資の収益は高いだろう。	とうしのしゅうえきはたかいだろう 
\\	その会社の収益性が改善した。	そのかいしゃのしゅうえきせいがかいぜんした 
\\	彼は金を株に投資した。	かれはかねをかぶにとうしした 
\\	教育は未来への投資である。	きょういくはみらいへのとうしである 
\\	事態はまだ改善可能だ。	じたいはまだかいぜんかのうだ 
\\	改善の余地は大いに残されている。	かいぜんのよちはおおいにのこされている 
\\	状況は彼女に不利だ。	じょうきょうはかのじょにふりだ 
\\	状況は改善の余地がある。	じょうきょうはかいぜんのよちがある 
\\	証拠は私に不利だった。	しょうこはわたしにふりだった 
\\	例外は原則のある証拠。	れいがいはげんそくのあるしょうこ 
\\	彼らはまだ証拠を探している。	かれらはまだしょうこをさがしている 
\\	弁護士は、新しい証拠を提出した。	べんごしは、あたらしいしょうこをていしゅつした 
\\	答案を提出せよ。	とうあんをていしゅつせよ 
\\	彼は市長さんと握手をした。	かれはしちょうさんとあくしゅをした 
\\	彼は私の顔をぼんやり眺めた。	かれはわたしのかおをぼにゃりながめた 
\\	遠くにぼんやりした明かりが見えた。	とおくにぼんやりしたあかりがみえた 
\\	利口な人でも時にはぼんやりすることがある。	りこうなひとでもときにはぼにゃりすることがある 
\\	彼は兄に劣らず利口だ。	かれはあににおとらずりこうだ 
\\	私は企てに失敗した。	わたしはくわだてにしっぱいした 
\\	彼らは逃亡を企てた。	かれらはとうぼうをくわだてた 
\\	彼の逃亡の試みはうまくいった。	かれのとうぼうのこころみはうまくいった 
\\	彼は種々の方法を試みた。	かれはしゅじゅのほうほうをこころみた 
\\	川を泳いで渡ろうと試みた。	かわをおよいでわたろうとこころみた 
\\	彼は必死に逃げた。	かれはひっしににげた 
\\	彼らは四方へ逃げた。	かれらはしほうへにげた 
\\	的を外してしまった。	まとをはずしてしまった 
\\	田中は今席を外しております。	たなかはいませきをはずしております 
\\	水道が出ています。	すいどうがだしています 
\\	水道管が破裂した。	すいどうかんがはれつした 
\\	人々は水道水の汚染に苦しんでいる。	ひとびとはすいどうすいのおせんにくるしんでいる 
\\	心臓が、どきどきして破裂しそう!	しんぞうが、どきどきしてはれつしそう 
\\	この川は汚染されていない。	このかわはおせんされていない 
\\	それは大気汚染と関係がある。	それはたいきおせんとかんけいがある 
\\	大都会の大気は汚染されている。	だいとかいのたいきはおせんされている 
\\	此処は立入禁止である。	ここはたちいりきんしである 
\\	急用で彼は出かけています。	きゅうようでかれはでかけています 
\\	彼は急用のために来られなかった。	かれはきゅうようのためにこられなかった 
\\	あなたの成功をお祝いします。	あなたのせいこうをおいわいします 
\\	ご成功をお祝い申し上げます。	ごせいこうをおいわいもうしあげます 
\\	私は彼に、男子出産のお祝いを述べた。	わたしはかれに、むすこしゅっさんのおいわいをのべた 
\\	言い分を簡潔に述べよ。	いいぶんをかんけつにのべよ 
\\	彼女は健康な赤ん坊を出産した。	彼女はけんこうなあかんぼうをしゅっさんした 
\\	簡潔な説明をして欲しい。	かんけつなせつめいをしてほしい 
\\	彼の演説は簡潔で要を得たものだった。	かれのえんぜつはかんけつでようをえたものだった。 
\\	私はサッカー観戦が大好きです。	わたしはサッカーかんせんがだいすきです 
\\	テレビ観戦者はカメラが捕らえるものしか見られない。	てれびかんせんしゃはカメラがとらえるものしかみられない 
\\	彼は感情を抑えた。	かれはかんじょうをおさえた 
\\	少女は懸命に涙を抑えた。	しょうじょはけんめいになみをおさえた 
\\	私は喜びを抑え切れなかった。	わたしはよろこびをおさえきれなかった 
\\	彼は興奮を抑えられなかった。	かれはこうふんをおさえられなかった 
\\	この文章は意味を成さない。	このぶんしょうはいみをなさない 
\\	彼は雑誌を開くと、たいていまず自分の星占いを読みます。	かれはざっしをあくと、たいていまずじぶんのほしうらないをよみます 
\\	恋愛と結婚は別だ。	れんあいとけっこんはべつだ 
\\	時は友情を深めるが、恋愛を弱める。	ときはゆうじょうをふかめるが、れんあいをよわめる 
\\	宝くじが当たった。	たからくじがあたった 
\\	彼女は宝くじで1千万円も手に入れた。	かのじょはたからくじでいっせんまんえんもてにいれた 
\\	罰金は現金で支払うべし。	ばっきんはげんきんでしはらうべし 
\\	裁判官は彼に罰金を言い渡した。	さいばんかんはかれにばっきんをいいわたした 
\\	彼は駐車違反で罰金をとられた。	かれはちゅうしゃいはんでばっきんをとられた 
\\	この中古車は売り物です。	このちゅうこしゃはうりものです 
\\	この中古車の価格は手ごろだ。	このちゅうこしゃのかかくはてごろだ 
\\	料金の手ごろなホテルを見つけて下さい。	りょうきんのてごろなホテルをみつけてください 
\\	冬には枯れる植物もある。	ふゆにはかれるしょくぶつもある 
\\	その植物は霜で被害を受けた。	そのしょくぶつはしもでひがいをうけた 
\\	これはこの国に特有の植物だ。	これはこのくににとくゆうのしょくぶつだ 
\\	雨不足で野山の植物が枯れた。	あめふそくでのやまのしょくぶつがかれた 
\\	彼は被害を大げさに言う。	かれはひがいをおおげさにいう 
\\	嵐の被害は何もなかった。	あらしのひがいはなにもなかった 
\\	あの古本はまったくの掘り出し物だ。	あのふるほんはまったくのほりだしものだ 
\\	彼女には欠点があるがやはり好きだ。	かのじょにはけってんがあるがやはりすきだ 
\\	奇妙に思えるが、それでもやはりそれは事実だ。	きみょうにおもえるが、それでもやはりそれはじじつだ 
\\	皆は反対したが、それでもやはり彼らは結婚した。	みんなははんたいしたが、それでもやはりかれらはけっこんした 
\\	怠惰が私の欠点である。	たいだがわたしのけってんである 
\\	これが若者特有の欠点だ。	これがわかものとくゆうのけってんだ 
\\	彼の計画には長所も欠点もある。	かれのけいかくにはちょうしょもけってんもある 
\\	誰にでも長所と短所がある。	だれにでもちょうしょとたんしょがある 
\\	あなたの計画は実際的だという長所がある。	あなたのけいかくはじっさいてきだというちょうしょがある 
\\	私は彼に短所があるから、かえっていっそう好きだ。	わたしはかれにたんしょがあるから、かえっていっそう好きだ 
\\	イギリス人は実際的な国民だ。	いぎりすじんはじっさいてきなこくみんだ 
\\	彼の考えはいつでもとても実際的です。	かれのかんがえはいtsでもとてもじっさいてきです 
\\	彼はその革命で積極的な役割をした。	かれはそのかくめいでせっきょくてきなやくわりした 
\\	私の役割は何ですか。	わたしのやくわりはなんですか 
\\	彼は重要な役割を演じた。	かれはじゅうようなやくわりをえんじた 
\\	彼女はこの企画で重要な役割を演じた。	かのじょはこのきかくでじゅうようなやくわりをえんじた 
\\	彼は企画部門に属している。	かれはきかくぶもんにぞくしている 
\\	彼は上流階級に属する。	かれはじょうりゅうかいきゅうにぞくする 
\\	彼は写真部に所属している。	かれはしゃしんぶにしょぞくしている 
\\	その国に革命が起こった。	そのくににかくめいがおこった 
\\	その後長い沈黙が続いた。	そのあとながいちんもくがつづいた 
\\	軽率に答えるな。	けいそつにこたえるな 
\\	この間違いは彼の軽率さが原因である。	このまちがいはかれのけいそつさがげんいんである 
\\	そんな軽率な計画で私達の意見は一致しないだろう。	そんなけいそつなけいかくでわたしたちのいけんはいっちしないだろう 
\\	その話は証拠と一致する。	そのはなしはしょうこといっちする 
\\	我々はお互いに意見が一致した。	われわれはおたがいにいけんがいっちした 
\\	いつその法律は施行されますか。	いつそのほうりつはしこうされますか 
\\	その問題は考慮に値しない。	そのもんだいはこうりょにあたいしない 
\\	彼の提案は考慮に値する。	かれのていあんはこうりょにあたいする 
\\	それは慎重な考慮を要する。	それはしんちょうなこうりょをようする 
\\	彼女のばかげた考えは注目にも値しない。	かのじょのばかげたかんがえはちゅうもくにもあたいしない 
\\	彼の振る舞いは批判にも値しない。	かれのふるまいはひはんにもあたいしない 
\\	彼の振る舞いに腹が立った。	かれのふるまいにはらがたった 
\\	彼は立派に振る舞った。	かれはりっぱにふるまった 
\\	彼女の振る舞いは荒々しい。	かのじょのふるまいはあらあらしい 
\\	そう批判的にならないで。	そうひはんてきにならないで 
\\	彼の批判は非常に厳しかった。	かれのひはんはひじょうにきびしかった 
\\	彼は同僚から批判を受けやすい。	かれはどうりょうからひはんをうけやすい 
\\	彼女は批判をとても気にする。	かのじょはひはんをとてもきにする 
\\	他人の仕事を批判するのは簡単だ。	たにんのしごとをひはんするのはかんたんだ 
\\	彼の好意は人から批判されやすい。	かれのこういはひとからひはんされやすい 
\\	彼は我々の計画に好意的だ。	かれはわれわれのけいかくにこういてきだ 
\\	お母さんの具合は?	おかあさんのぐあいは? 
\\	いい具合に彼に会った。	いいぐあいにかれにあった 
\\	靴下にも流行がある。	くつしたにもりゅうこうがある 
\\	今や留学は大流行だ。	いまやりゅうがくはだいりゅうこうだ 
\\	あの種の服が今流行だ。	あのたねのふくがりゅうこうだ 
\\	この懐中電灯は2個の電池が必要だ。	このかいちゅうでんとうは2このでんちがひつようだ 
\\	この懐中電灯は明かりが弱くなってきた。	このかいちゅうでんとうはあかりがよわくなってきた 
\\	懐中電灯を手探りで探した。	かいちゅうでんとうをてさぐりでさがした 
\\	彼らは手探りで進み続けた。	かれらはてさぐりですすみつづけた 
\\	彼は暗闇の中を手探りで進んだ。	かれはくらやみのなかをてさぐりですすんだ 
\\	地球は美しい惑星だ。	ちきゅうはうつくしいわくせいだ 
\\	引力が惑星を引きつける。	いんりょくがわくせいをひきつける 
\\	ニュートンは引力の法則を確立した。	ニュートンはいんりょくのほうそくをかくりつした 
\\	ルネサンスは人間の尊厳を確立した。	レネサンスはじんかんのそんげんをかくりつした 
\\	すべての人々の権利を尊厳すべきだ。	すべてのひとびとのけんりをそんげんすべきだ 
\\	権利を行使する。	けんりをこうしする 
\\	国民は彼の権利を奪った。	こくみんはかれのけんりをうばった 
\\	幸福を求める権利は誰にもある。	ゆうふくを求める権利はだれにもある 
\\	私はその陳述を真実と認める。	わたしはそのちんじゅつをしんじつとみとめる 
\\	彼の陳述は事実に基づいていた。	かれのちんじゅつはじじつにもとづいていた 
\\	明確な陳述をしていただけませんか。	めいかくなちんじゅつをしていただけませんか 
\\	明確な返事が欲しい。	めいかくなへんじがほしい 
\\	彼の報告は偽りであると分かった。	かれのほうこくはいつわりであるとわかった 
\\	棚からその箱を下ろしてください。	たなからそのはこをおろしてください 
\\	その本は貸し出し中です。	そのほんはかしだしちゅうです 
\\	父は今、留守です。	ちちはいまるすです 
\\	祖父は足の骨を折りました。	そふはあしのほねをおりました 
\\	その可哀相な子供たちは食べるものがない。	そのかわいそうなこどもたちはたべるものがない 
\\	おいしいジャムを作る秘訣を教えてください。	おいしいジャムをつくるひけつをおしえてください 
\\	正直は結局割に合うものだ。	しょうじきはけっきょくわりにあうものだ 
\\	彼は年の割にはよく走る。	かれはとしのわりにはよくはしる 
\\	彼は年の割には若く見える。	かれはとしのわりにわかくみえる 
\\	猿はたくさん芸を覚える。	さるはたくさんげいをおぼえる 
\\	彼女は率直にものを言う。	かのじょはそっちょくにものをいう 
\\	彼女は率直に罪を認めた。	かのじょはそっちょくにつみをみとめた 
\\	正直に言うと、私は孤独だった。	しょうじきにいうと、わたしはこどくだった 
\\	その丘から海が見渡せた。	そのおかからうみがみわたせた 
\\	この塔から町全体が見渡せます。	このとうからまちぜんたいがみわたせます 
\\	屋上からは、何マイルも見渡せる。	おくじょうからは、なんまいるもみわたせる 
\\	この城の塔から町の全景が見渡せる。	このしろのとうからまちのぜんけいがみわたせる 
\\	多額の金がその橋に費やされた。	たがくのかねがそのはしについやされた 
\\	この事業には多額の資金が要る。	このじぎょうにはたがくのしきんがいる 
\\	事業は旨く行っている。	じぎょうはうまくいっている 
\\	結局万事旨く行くだろう。	けっきょくばんじうまくいくだろう 
\\	万事用意しておけ。	ばんじよういしておけ 
\\	彼は資金集めが上手い。	かれはしきんあつめがうまい 
\\	皮は乾くにつれて堅くなった。	かわはかわくにつれてかたくなった 
\\	その田んぼは荒れ果てたままだ。	そのたんぼはあれはてたままだ 
\\	私達は疲れ果てるまで何時間も、その音楽に合わせて踊った。	わたしたちはつかれはてるまでなんじかんも、そのおんがくにあわせておどった 
\\	その古城は荒れ果てている。	そのこじょうはあれはてている 
\\	宇宙の果てへと私を誘う。	うちゅうのはてへとわたしをさそう 
\\	困り果てて言葉につまった。	こまりはててことばにつまった 
\\	メアリーは見知らぬ人に話し掛けられて、言葉に詰まってしまった。	メアリーはみしらぬひとにはなしかけられて、ことばにつまってしまった 
\\	消防士は火事を消した。	しょうぼうしはかじをけした 
\\	消防士たちはドアをぶち破った。	しょうぼうしたちはドアをぶちやぶった 
\\	消防署は警察署の隣にあります。	しょうぼうしょはけいさつしょのとなりにあります 
\\	消防車はほかの乗り物に優先する。	しょうぼうしゃはほかののりものにゆうせんする 
\\	あまり多くの乗り物に乗ると酔います。	あまりおおくののりものにのるとよいます 
\\	消防士が火の消し方を実演した。	しょうぼうしがひのけしかたをじつえんした 
\\	私たちは彼の手品の実演に感嘆した。	わたしたちはかれのてじなのじつえんにかんたんした 
\\	美しい日没に感嘆せざるをえなかった。	うつくしいにちぼつにかんたんせざるをえなかった 
\\	日没後彼らはその旅館に着いた。	にちぼつごかれらはそのりょかんについた 
\\	そうせざるを得ないね。	そうせざるをえないね 
\\	私はそう決断せざるを得ない。	わたしはけつだんせざるをえない 
\\	私は彼の提案に反対せざるを得ない。	わたしはかれのていあんにはんたいせざるをえない 
\\	何よりも安全を優先すべきだ。	なによりもあんぜんをゆうせんすべきだ 
\\	何よりも義務を優先すべきだ。	なによりもぎむをゆうせんすべきだ 
\\	昆虫の世界では、常に力が優先する。	こんちゅうのせかいでは、つねにちからがゆうせんする 
\\	昆虫を探して森へ行った。	こんちゅうをさがしてもりへいった 
\\	蜘蛛は蝿やその他の昆虫を食べる。	くもははえやそのほかのこんちゅうをたべる 
\\	その遊園地を建設するのに10年かかった。	そのゆうえんちをけんせつするのに10ねんかかった 
\\	その鉄道は今建設中だ。	そのてつどうはいまけんせつちゅうだ 
\\	私は友達に年賀状を出すのを忘れた。	わたしはともだちにねんがじょうをだすのをわすれた 
\\	もう年賀状を全部書いてしまったのですか。	もうねんがじょうをぜんぶかいてしまったのですか 
\\	私は帰宅の許可を得た。	わたしはきたくのきょかをえた 
\\	許可なしに部屋に入るな。	きょかなしにへやにはいるな 
\\	上司に許可を取ってきます。	じょうしにきょかをとってきます 
\\	先生は早退の許可をくれた。	せんせいはそうたいのきょかをくれた 
\\	昨日数人の男の子が早退しなければならなかった。	きのうすうにんのおとこのこがそうたいしなければならなかった 
\\	その操作に全く危険はありません。	そのそうさにまったくきけんはありません 
\\	この機械の操作は私には難しすぎる。	このきかいのそうさはわたしにはむずかしすぎる 
\\	彼はこの機械の操作が大変進歩する。	かれはこのきかいのそうさがたいへんしんぽする 
\\	そのホテルは設備がよい。	そのほてるはせつびがよい 
\\	この学校は暖房設備がない。	このがっこうはだんぼうせつびがない 
\\	ひもを引けば水が流れ出る。	ひもをひけばみずがながれでる 
\\	このふたは固くて取れない。	このふたはかたくてとれない 
\\	一曲聞くか。	いっきょくきくか 
\\	この曲を前に聞いたのは覚えている。	このきょくをまえにきいたのはおぼえている 
\\	贈り物は慎重に選びなさい。	おくりものはしんちょうにえらびなさい 
\\	贈り物としても最適です。	おくりものとしてもさいてきです 
\\	確かに彼はその仕事に最適の人物だ。	だれかにかれはそのしごとにさいてきのじんぶつだ 
\\	彼は次の世代を指導すべき人物だ。	かれはつぎのせだいをしどうすべきじんぶつだ 
\\	彼らは盲目的に指導者に従った。	かれらはもうもくてきにしどうしゃにしたがった 
\\	恋は本来盲目である。	あいはほんらいもうもくである 
\\	彼は学校の売店で文房具を買った。	かれはがっこうのばいてんでぶんぼうぐをかった 
\\	私の従兄弟は医者を見ただけで恐がった。	わたしのいとこはいしゃをみただけでこわがった 
\\	私のいとこは手品が得意です。	わたしのいとこはてじながとくいです 
\\	お二人の未来に乾杯しましょう。	おふたり の みらい に かんぱい しましょう 
\\	これは各駅停車です。	これはかくえきていしゃです 
\\	どのくらい停車しますか。	どのぐらいていしゃしますか 
\\	バスはどの村にも停車しました。	バスはどのむらにもていしゃしました 
\\	この電車は中野より先は各駅に停車する。	このでんしゃはなかのよりさきはかくえきにていしゃする 
\\	タクシーは信号のところで急停車した。	タクシーはしんごうのところできゅうていしゃした 
\\	急行列車は各駅停車より一時間も早い。	きゅうこうれっしゃはかくえきていしゃよりいちじかんもはやい 
\\	急行列車は渋谷駅から中目黒駅まで停車しません。	きゅうこうれっしゃはしぶやえきからなかめぐろえきまでていしゃしません 
\\	天井に手が届きますか。	てんじょうにてがとどきますか 
\\	彼は天井を青く塗った。	かれはてんじょうをあおくぬった 
\\	彼は天井に頭をぶつけた。	かれはてんじょうにあたまをぶつけた 
\\	ハエが天井にとまっている。	ハエがてんじょうにとまっている 
\\	荷物は届きましたか。	にもつはとどきましたか 
\\	伝言が届いております。	でんごんがとどいております 
\\	製品は6月にお届けできます。	せいひんはろくがつにおとどけできます 
\\	百合はアリスよりじょうずにタイプを打ちます。	ゆりはアリスよりじょうずにタイプをうちます 
\\	彼女の仕事はタイプを打つ事だ。	かのじょのしごとはタイプをうつことです 
\\	この手紙をタイプで打ってください。	このてがみをタイプうってください 
\\	携帯電話が欲しいな。	けいたいでんわがほしいな 
\\	小さな道具セットは旅をする時携帯に便利だ。	ちいさいなどうぐセットはたびをするときけいたいにべんりいだ 
\\	娘を女子校に入れた。	むすめをじょしこうにいれた 
\\	できるだけ早く走った。	できるだけはやくはしった 
\\	できるだけお助けします。	できるだけおたすけします 
\\	余暇をできるだけ利用しなさい。	よかをできるだけりようしなさい 
\\	現代哲学は19世紀に始まる。	げんだいてつがくは19せいきにはじまる 
\\	科学は21世紀に備える方法だ。	かがくは21せいきにそなえるほうほうだ 
\\	不慮の事故に備える。	ふりょのじこにそなえる 
\\	彼は老後に備えて貯金した。	かれはろうごにそなえてちょきんした 
\\	老後は楽に暮らしたい。	ろうごはらくにくらしたい 
\\	あなたは将来に備えたほうがよい。	あなたはじょうしょうらいにそなえたほうがよい 
\\	汽船と汽車ではどちらが速く走りますか。	きせんときしゃではどちらがはやくはしりますか 
\\	面と向かってあなたを褒めるような人を信用してはいけない。	めんとむかってあなたをほめるようなひとをしんようしてはいけない 
\\	彼は先生に褒められたのでますます熱心に勉強した。	かれはせんせいにほめられたのでますますねっしんにべんきょうした 
\\	明日の朝八時に誘いに行きます。	あしたのあさはちじにさそいにいきます 
\\	ブライアンの彼女はよく彼に贅沢なレストランに連れていってと頼みます。	ブライアンのかのじょはよくかれにぜいたくなレストランにつれていってとたのみます 
\\	午後に面接を行います。	ごごにめんせつをおこないます 
\\	あなたの予約の確認はこちらで行います。	あなたのよやくのかくにんはこちらでおこないます 
\\	武器の輸出が禁止された。	ぶきのゆしゅつがきんしされた 
\\	昨年は輸入が輸出を超えた。	さくねんはゆにゅうがゆしゅつをこえた 
\\	その発見には驚いた。	そのはっけんにはおどろいた 
\\	彼は重大な発見をした。	かれはじゅうだいなはっけんをした 
\\	彼は新しい星を発見した。	かれはあたらしいほしをはっけんした 
\\	遂に彼は真実を発見した。	ついにかれはしんじつをはっけんした 
\\	街路はよく設計されている。	がいろはよくせっけいされている 
\\	その庭はプロの手で設計されている。	そのにわはプロのてでせっけいされている 
\\	彼らは街路の雪を取り払った。	かれらはがいろのゆきをとりはらった 
\\	日本はいろんな原料に乏しい。	にほんはいるんなげんりょうにとぼしい 
\\	日本は原料を輸入に頼っている。	にほんはげんりょうをゆにゅうにたよっている 
\\	偶然に頼るな。	ぐうぜんにたよるな 
\\	彼女は豪華な絹の服を着ていた。	かのじょはごうかなきぬのふくをきていた 
\\	豪華客船が港に入った。	ごうかきゃくせんがみなとにはいった 
\\	その彫刻は非常に貴重である。	そのちょうこくはひじょうにきちょうである 
\\	私は現代彫刻はよく理解できない。	わたしはげんだいちょうこくはよくりかいできない 
\\	その彫刻には1920年と刻まれている。	そのちょうこくには1920ねんときざまれている 
\\	その石は刻まれて大きな像になった。	そのいしはきざまれておおきなぞうになった 
\\	彼は仏像を木で彫った。	かれはぶつぞうをきでほった 
\\	彼は私に木の人形を彫ってくれた。	はれはわたしにきのにんぎょうをほってくれた 
\\	彼らは大理石で像を彫っている。	かれらはだいりせきでぞうをほっている 
\\	彼女の手は冷たい大理石のような感じがした。	かのじょのてはつめたいだいりせきのようなかんじがした 
\\	彼と仲間になった。	かれとなかまになった 
\\	悪友仲間と付き合うな。	あくゆうなかまとつきあうな 
\\	田中さんは父の釣り仲間の一人だ。	たなかさんはちちのつりなかまのひとつだ 
\\	一行は無事に戻った。	いっこうはぶじにもどった 
\\	この電車は東京まで停車致しません。	このでんしゃはとうきょうまでていしゃいたしません 
\\	彼は非常に想像力にとんだ作家です。	かれはひじょうにそうぞうりょくにとんださっかです 
\\	この町は活気がない。	このまちはかっきがない 
\\	楽団は急に活気づいた。	がくだんはきゅうにかっきづいた 
\\	産業界が活気づいています。	さんぎょうかいがかっきづいています 
\\	商売はまったく活気がない。	しょうばいはまったくかっきがない 
\\	オーケストラの楽団員はみな成功を喜んだ。	オーケストラのがくだんいんはみなせいこうをよろこんだ 
\\	うちの母は活発です。	うちのはははかっぱつです 
\\	私たちは活発な討論をした。	わたしたちはかっぱつなとうろんをした 
\\	売れ行きはものすごく活発だ。	うれゆきはものすごくかっぱつだ 
\\	スミス夫人は活発にボランティアの仕事に従事している。	スミスふじんはかっぱつにボランティアのしごとにじゅうじしている 
\\	彼は農業に従事している。	かれはのうぎょうにじゅうじしている 
\\	彼は自動車産業に従事している。	かれはじどうしゃさんぎょうにじゅうじしている 
\\	彼はエイズの研究に従事している。	かれはえいずのけんきゅうにじゅうじしている 
\\	私には政治活動に従事する暇はない。	わたしはせいじかつどうにじゅうじするひまはない 
\\	アメリカ人は積極的な国民である。	アメリカじんはせっきょくてきこくみんである 
\\	彼はその革命で積極的な役割をした。	かれはそのかくめいでせっきょくてきなやくわりをした 
\\	現代社会での大学の役割は何ですか。	げんだいしゃかいでのだいがくのやくわりはなんですか 
\\	ビールの売れ行きは天候次第です。	ビールのうれゆきはてんこうしだいです 
\\	あなた次第です。	あなたしだいです 
\\	彼の答えは気分次第だ。	かれのこたえはきぶんしだいだ 
\\	駅に着き次第、電話します。	えきのつきしだい、でんわします 
\\	彼の成功は努力次第だ。	かれのせいこうはどりょくしだいだ 
\\	努力をしたが無駄だった。	どりょくをしたがむだだった 
\\	努力は良い結果を生み出す。	どりょくはよいけっかをうみだす 
\\	それはこういう風にやりなさい。	それはこういうふうにやりなさい 
\\	昨年は災害が頻々とあった。	さくねんはさいがいがひんぴんとあった 
\\	私は彼の夕食の誘いを辞退した。	わたしはかれのゆうしょくのさそいをじたいした 
\\	食事中に音を立てるのは不作法だ。	しょくじちゅうにおとをたてるのはふさほうだ 
\\	彼はとても不作法なのでみなが嫌っている。	かれはとてもふさほうなのでみながきらっている 
\\	彼はとても粗野だ。彼の不作法には我慢できない。	かれはとてもそやだ。かれのふさほうにはがまんできない 
\\	私は彼女の粗野な態度に腹が立った。	わたしはかのじょのそやなたいどにはらがたった 
\\	人を指差すのは無作法だ。	ひとをゆびさすのはぶさほうだ 
\\	挨拶は礼儀作法の根源である。	あいさつはれいぎさほうのこんげんである 
\\	彼女はとても礼儀正しい。	かのじょはとてもれいぎただしい 
\\	礼儀正しいのが彼の特徴であった。	れいぎただしのがかれのとくちょうであった 
\\	50分の模擬試験に挑戦して、試験であなたの実力がどれくらいかがわかります。	50ぷんのもぎしけんにちょうせんして、しけんであなたのじつりょくがどれくらいわかります 
\\	我々の稼ぎは実力に比例している。	われわれのかせぎはじつりょくにひれいしている 
\\	私は何度もその問題に挑戦した。	わたしはなんどそのもんだいにちょうせんした 
\\	私は一か八か彼の挑戦を受けてみた。	わたしはいちかばちかかれのちょうせんをうけてみた 
\\	一か八かやってみることで一財産できるだろう。	いちかばちかやってみることでいっざいさんできるだろう 
\\	彼は毎月お金を幾ら稼ぎますか。	かれはまいげつおかねをいくらかせぎますか 
\\	彼の僅かな稼ぎで暮らしていくのは困難だった。	かれのわずかなかせぎでくらしていくのはこんなんだった 
\\	その列車には乗客はほんの僅かしかいなかった。	そのれっしゃにはじょうきゃくはほんのわずかしかいなかった 
\\	供給は需要に比例する。	きょうきゅうはじゅようにひれいする 
\\	罪に比例して罰するべきだ。	つみにひれいしてばっするべきだ 
\\	川は町や村に水を供給する。	かわはまちやむらにみずをきょうきゅうする 
\\	その川は町に電力を供給する。	そのかわはまちにでんりょくをきょうきゅうする 
\\	ひつじは我々に羊毛を供給する。	ひつじはわれわれにようもうをきょうきゅうする 
\\	ブラウンさんは羊毛を扱う商人です。	ブラウンさんはようもうをあつかうしょうにんです 
\\	その坂道を下ろう。	そのさかみちをくだろう 
\\	袖を引っ張らないでください。	そでをひっぱらないでください 
\\	他人の足を引っ張るようなことはするな。	たにんのあしをひっぱるようなことはするな 
\\	この生地は引っ張るとすぐ伸びる。	このきじはひっぱるとすぐのびる 
\\	重力が物を地球の中心に引っ張っている。	じゅうりょくがものをちきゅうのちゅうしんにひっぱっている 
\\	此処を中心とする一帯が爆撃された。	ここをちゅうしんとするいったいがばくげきされた 
\\	首都は繰り返し繰り返し爆撃された。	しゅとはくりかえしくりかえしばくげきされた 
\\	彼女は同じ誤りを繰り返した。	かのじょはおなじあやまりをくりかえした 
\\	言葉の学習には繰り返しが必要です。	ことばのがくしゅうにはくりかえしがひつようです 
\\	彼女は入院退院を繰り返している。	かのじょはにゅういんたいいんをくりかえしている 
\\	少女は彼に抱きついた。	しょうねんはかれにだきついた 
\\	母親は赤ん坊を抱きしめた。	ははおやはあかんぼうをだきしめた 
\\	彼は彼らに対して深い憎しみを抱いた。	かれはかれらにたいしてふかいにくしみをいだいた 
\\	その裁判官は厳粛な顔をして、怖そうに見えた。	そのさいばんかんはげんしゅくなかおをして、こわそうにみえた 
\\	観客は盛り上がっているよ。	かんきゃくはもりあがっているよ 
\\	ゲームで一段と盛り上がった。	【いちだん】 比べると、かなりのちがいのあるさま。ひときわ。いっそう。ずっと。
\\	ケイトは先ず誕生日のお祝いをすることはない。	ケイトはまずたんじょうびのおいわいをすることはない 
\\	彼は孫達に囲まれて座っていました。	かれはまごたちにかこまれてすわっていました 
\\	子孫にきれいな、緑の地球を残したい。	しそんにきれいな、みどりのちきゅうをのこりたい 
\\	その学者は自分の祖先を崇拝している。	そのがくしゃはじぶんのそせんをすうはいしている 
\\	私たちの祖先はこの国に150年前にやってきた。	わたしたちのそせんはこのくにに150ねんまえにやってきた 
\\	その見事なケーキをみてよだれが出そうになったよ。	そのみごとなケーキをみてよだれがでそうになったよ 
\\	女は男の災いである。	おんなはおとこのわざわいである 
\\	再び彼に会う望みはない。	ふたたびかれにあうのぞみはない 
\\	自分の行動を思い出すと情けないよ。	じぶんのこうどうをおもいだすとなさけないよ 
\\	我々は暴力に訴えるべきでない。	われわれはぼうりょくにうったえるべきでない 
\\	寝坊しないように注意しなさい。	ねぼうしないようにちゅいしなさい 
\\	船は島の間を縫うように進んだ。	ふねはしまのまをぬうようにすすんだ 
\\	体力を増やすように努めなさい。	たいりょくをふやすようにつとめなさい 
\\	学生は遅刻しないように努めるべきだ。	がくせいはちこくしないようにつとめるべきだ 
\\	できるだけ高く飛び上がるようにしなさい。	できるだけたかくとびあがるようにしなさい 
\\	不幸な過去は忘れるようにしたほうがよい。	ふこうなかこはわすれるようにしたほうがよい 
\\	私はもっとよく聞こえるように近くへ移動した。	わたしはもっとよくきこえるようにちかくへいどうした 
\\	彼は転々と移動した。	かれはてんてんといどうした 
\\	警察は私が車を移動するように求めた。	けいさつはわたしはくるまをいどうするようにもとめた 
\\	彼は夜更かしをするようになった。	かれはよふかしをするようになった 
\\	彼には夜更かしの癖がある。	かれにはよふかしのくせがある 
\\	彼は頭をかく癖がある。	かれはあたまをかくくせがある 
\\	彼は最近酒を飲む癖が付いた。	かれはさいきんさけをのむくせがついた 
\\	彼の怠け癖は我慢の限界を越える。	かれのなまけくせはがまんのげんかいをこえる 
\\	教育のおかげで私は今日のようになった。	きょういくのおかげでわたしはきょうのようになった 
\\	私は自分の間違いを後悔するようになった。	わたしはじぶんのまちがいをこうかいするようになった 
\\	彼女は徐々に私に憎しみを示すようになった。	かのじょはじょじょにわたしににくしみをしめすようになった 
\\	私は、日本人の友人達と同じように勉強し遊ぶようになりました。	わたしは、にほんじんのゆうじんたちとおなじようにべんきょうしあそぶようになりました 
\\	彼は目が見えなくなった。	かれはめがみえなくなった 
\\	私は辛抱しきれなくなった。	わたしはしんぼうしきれなくなった 
\\	濃い霧でその建物は見えなくなった。	こいきりでそのたてものはみえなくなった 
\\	彼女は辛抱強く彼を待った。	かのじょはしんぼうつよくかれをまった 
\\	濃いコーヒーを飲んだので彼女は一晩中寝れなかった。	こいこーひーをのんだのでかのじょはひとばんじゅうねれなかった 
\\	一晩中寒くて不安でした。	ひとばんじゅうさむくてふあんでした 
\\	新しい仕事に適応できますか。	あたらしいしごとにてきおうできますか 
\\	人間の脳は新しい状況に適応することが出来る。	にんげんののうはあたらしいじょうきょうにてきおうすることができる 
\\	彼は優秀な脳外科医だ。	かれはゆうしゅうなのうげかいだ 
\\	君の試験結果は優秀だ。	きみのしけんけっかはゆうしゅうだ 
\\	彼はクラスでずば抜けて優秀だ。	かれはクラスでずばぬけてゆうしゅうだ 
\\	彼の数学の才能はずば抜けている。	かれのすうがくののうりょくはずばぬけている 
\\	豪雨は雷を伴った。	ごううはかみなりをともなった 
\\	少しずつ水が浅くなる。	すこしずつみずがあさくなる 
\\	利率は少しずつ上昇するだろう。	りりつはすこしずつじょうしょうするだろう 
\\	視力障害があります。	しりょくしょうがいがあります 
\\	私は多くの障害に直面した。	わたしはおおくのしょうがいにちょくめんした 
\\	彼は予期せぬ障害に出会った。	かれはよきせぬしょうがいにであった 
\\	予期せぬ困難が起こった。	よきせぬこんなんがおこった 
\\	この町の秩序を守るのは難しい。	このまちのちつじょをまもるのはむずかしい 
\\	案の定、彼は第一位になった。	あんおじょう、かれはだいいちいになった 
\\	この語は今では用いられない。	このごはいまではもちいられない 
\\	彼は商売で不正な手段を用いた。	かれはしょうばいでふせいなしゅだんをもちいた 
\\	これらの道具は一般に用いられている。	これらのどうぐはいっぱんにもちいられている 
\\	一般に日本人は勤勉だ。	いっぱんににほんじんはきんべんだ 
\\	音楽家は一般に批評に敏感である。	おんがくかはいっぱんにひひょうにびんかんである 
\\	その監督は批評を気にする。	そのかんとくはひひょうをきにする 
\\	その映画はとても好意的な批評を受けた。	そのえいがはとてもこういてきなひひょうをうけた 
\\	彼は取引銀行の店長に好意的な印象を与えた。	かれはとりひきぎんこうのてんちょうにこういてきないんしょうをあたえた 
\\	取引契約は先月で切れた。	とりひきけいやくはせんげつできれた 
\\	翻訳の質が良くなった。	ほんやくのしつがよくなった 
\\	翻訳を原文と比べてみよう。	ほんやくをげんぶんとくらべてみよう 
\\	彼はその一節を英語に翻訳した。	かれはそのいっせつをえいごにほんやくした 
\\	この本は何回か翻訳されている。	このほんはなんかいかほんやくされている 
\\	彼は秘書にその手紙を英語に翻訳させた。	かれはひしょにそのてがみをえいごにほんやくさせた 
\\	その本の一節をノートに書き写した。	そのほんのいっせつをノートにかきうつした 
\\	日本料理の用語はほかの言語に訳すのが難しい。	にほんりょうりのようごはほかのげんごにやくすのがむずかしい 
\\	これらの専門用語はギリシャ語に由来している。	これらのせんもにょうごはギリシャごにゆらいしている 
\\	英語の多くはラテン語に由来する。	えいごのおおくはラテンごにゆらいする 
\\	このパレードは古い儀式に由来している。	このパレードはふるいぎしきにゆらいしている 
\\	その儀式は明日催される。	そのぎしきはあしたもよおされる 
\\	彼女は来週パーティーを催す。	かのじょはらいしゅうパーティーをもよおす 
\\	その肉は冷凍されている。	そのにくはれいとうされている 
\\	彼らはもうじき結婚すると噂されている。	かれらはもうじきけっこんするとうわさされている 
\\	箱の内容はラベルに表示されている。	はこをないようはラベルにひょうじされている 
\\	製造年月日はふたに表示されている。	せいぞうねんがっぴはぶたにひょうじされている 
\\	生年月日を教えてください。	せいねんがっぴをおしえてください 
\\	彼の話は内容がない。	かれのはなしはないようがない 
\\	手紙の内容は秘密であった。	てがみのないようはひみつであった 
\\	ほとんど内容のない議論だった。	ほとんどないようのないぎろんだった 
\\	このビルは有名な建築家によって設計された。	このビルはゆうめいなけんちくかによってせっけいされた 
\\	人は言葉によって考えを表現する。	ひとはことばによってかんがえをひょうげんする 
\\	その村はひどい嵐によって孤立した。	そのむらはひどいあらしによってこりつした 
\\	食べ物によっては喉が渇くものがある。	たべものによってはのどがかわくものがある 
\\	多くの都市が爆弾によって破壊された。	おおくのとしがばくだんによってはかいされた 
\\	車が木に衝突した。	くるまがきにしょうとつした 
\\	自動車が正面衝突した。	じどうしゃはしょうめんしょうとつした 
\\	その問題で意見の衝突が起きた。	そのもんだいでいけんのしょうとつがおきた 
\\	正面のドアは鍵がかかったままだった。	しょうめんのドアはかぎがかかったままだった 
\\	私はエイリアンに誘拐されていた。	わたしはエイリアンにゆうかいされていた 
\\	その子は帰宅の途中で誘拐されたのかもしれない。	そのこはきたくのとちゅうでゆうかいされたかもしれない 
\\	その飛行機は突然墜落した。	そのひこうきはとつぜんついらくした 
\\	飛行機は墜落寸前に右に旋回した。	ひこうきはついらくすんぜんにみぎにせんかいした 
\\	彼女は気絶寸前だった。	かのじょはきぜつすんぜんだった 
\\	彼女は雪の中で凍死寸前だった。	かのじょはゆきのなかでとうしすんぜんだった 
\\	可哀相にその子供は餓死寸前だった。	かわいそうにそのこどもはがしすんぜんだった 
\\	鷹は空を旋回した。	たかはそらをせんかいした 
\\	ヘリコプターが上空を旋回した。	ヘリコプターがじょうくうをせんかいした 
\\	彼女はトラの姿を見て気絶した。	かのじょはとらのすがたをみてきぜつした 
\\	その戦争中に多くの人が餓死した。	そのせんそうちゅうにおおくのひとががしした 
\\	彼らは沈没する船を見捨てた。	かれらはちんぼつするふねをみすてた 
\\	誰でも友人を見捨てるべきではない。	だれでもゆうじんをみすてるべきではない 
\\	携帯電話の電源を切るべきである。	けいたいでんわのでんげんをきるべきである 
\\	回路を調べる前に、電源のスイッチを切りなさい。	かいろをしらべるまえに、でんげんのスイッチをきりなさい 
\\	彼はその手紙を全員に回覧した。	かれはそのてがみをぜんいんにかいらんした 
\\	彼の姉妹は二人とも美人です。	かれのしまいはふたりともびじんです 
\\	神戸はシアトルの姉妹都市です。	こうべはシアトルのしまいとしです 
\\	移住者は大陸から日本海を渡ってきた。	いじゅうしゃはたいりくからにほんかいをわたってきた 
\\	日本海は日本とアジア大陸を隔てている。	にほんかいはにほんとアジアたいりくをへだてている 
\\	二つの町は川で隔てられている。	ふたつのまちはかわでへだてられている 
\\	イギリス海峡がイギリスとフランスを隔てている。	イギリスかいきょうがイギリスとフランスをへだてている 
\\	2人の生徒の意見には大きな隔たりがある。	ふたりのせいとのいけんにはおおきなへだたりがある 
\\	彼女の一家はブラジルへ移住して行った。	かのじょのいっかはブラジルへいじゅうしていった 
\\	アメリカは移住者の国である。	アメリカはいじゅうしゃのくにである 
\\	この地点では海は狭くなって海峡となっている。	このちてんではうみはせまくなってかいきょうとなっている 
\\	彼はイギリス海峡を泳ぎ渡った唯一のアメリカ人だ。	かれはイギリスかいきょうをおよぎわたったゆいいつのアメリカじんだ 
\\	その湖はこの地点が深い。	そのみずうみはこのちてんがふかい 
\\	日光は東京の北約75マイルの地点にある。	にっこうはとうきょうのきたやく75マイルのちてんにある 
\\	それは唯一の関心事です。	それはゆいいつのかんしんじです 
\\	健康が私の唯一の資本です。	けんこうがわたしのゆいいつのしほんです 
\\	彼女の行動は私の最大関心事だ。	かのじょのこうどうはわたしのさいだいかんしんじだ 
\\	彼らは我々を問題児と呼ぶ。	かれらはわれわれをもんだいじとよぶ 
\\	よい辞書を手元に置いておきなさい。	よいじしょをてもとにおいておきなさい 
\\	きちんと座りなさい。	きちんとすわりなさい 
\\	きちんと戸を閉めなさい。	きちんととをしめなさい 
\\	貴方の仕事をきちんとしなさい。	あなたのしごとをきちんとしなさい 
\\	約束はきちんと果たすべきだ。	やくそくはきちんとはたすべきだ 
\\	その子は帽子をきちんとかぶり直した。	そのこはぼうしをきちんとかぶりなおした 
\\	荷物を整理したいのですが。	にもつをせいりしたいのですが 
\\	彼は部屋をきちんと整理した。	かれはへやをきちんとせいりした 
\\	いつも仕事場をきちんと整理しておきなさい。	いつもしごとばをきちんとせいりしておきなさい 
\\	この手作りのイタリア製チタン自転車は、恐ろしく軽い。	このてづくりのイタリアせいチタンじてんしゃは、おそろしくかるい 
\\	昨晩恐ろしい夢を見た。	さくばんおそろしいゆめをみた 
\\	爆弾は恐ろしい武器だ。	ばくだんはおそろしいぶきだ 
\\	彼女は料理が恐ろしく下手だ。	かのじょはりょうりがおそろしくへただ 
\\	みんなを満足させるのは難しい。	みんなをまんぞくさせるのはむずかしい 
\\	核戦争は人類を滅亡させるだろう。	かくせんそうはじんるいのめつぼうさせるだろう 
\\	もう一度戦争があったら、私達は滅亡するだろう。	もういちどせんそうがあったら、わたしたちはめつぼうするだろう 
\\	人類は遂に月に到着した。	じんるいはついにつきにとうちゃくした 
\\	宗教は人類のアヘンである。	しゅうきょうはじんるいのアヘンである 
\\	現代は原子力の時代だ。	げんだいはげんしりょくのじだいだ 
\\	科学が原子爆弾を生み出した。	かがくがげんしばくだんをうみだした 
\\	水泳は体のいろいろな筋肉を発達させる。	すいえいはからだのいろいろなきんにくをはったつさせる 
\\	私は話を通じさせることができなかった。	わたしははなしをつうじさせることができなかった 
\\	この花の香りは私の子供時代を思い出させます。	このはなのかおりはわたしのこどもじだいをおもいださせます 
\\	その子に無理に食べさせてはいけません。	そのこにむりにたべさせてはいけません 
\\	秘書は手紙を封筒の中に差し込んだ。	ひしょはてがみをふうとうのなかにさしこんだ 
\\	彼は封筒を開けてみたが失望しただけであった。	かれはふうとうをあけてみたがしつぼうしただけであった 
\\	新製品には失望した。	しんせいひんにはしつぼうした 
\\	彼の講義は私たちを失望させた。	かれのこうぎはわたしたちをしつぼうさせた 
\\	彼らは犬の世話をしない。	かれらはいぬのせわをしない 
\\	彼女は子供の世話で忙しい。	かのじょはこどものせわでいそがしい 
\\	留守中犬を世話して下さい。	るすちゅういぬをせわしてください 
\\	嘗て残忍な王様がいた。	かつてざんにんなおうさまがいた 
\\	囚人たちは恐ろしいほど残忍に扱われた。	しゅうじんたちはおそろしいほどざんにんにあつかわれた 
\\	この町は自然の残忍な力によって破壊された。	このまちはしぜんのざんにんなちからによってはかいされた 
\\	原爆は広島全体を破壊した。	げんばくはひろしまぜんたいをはかいした 
\\	今月の電話代見て、目が飛び出た。	こんげつのでんわだいみて、めがとびでた 
\\	良くなったら伺います。	よくなったらうかがいます 
\\	ご伝言を伺いましょうか。	ごでんごんをうかがいましょうか 
\\	明日お宅に伺います。	あしたおたくにうかがいます 
\\	車が木にぶつかった。	くるまがきにぶつかった 
\\	船は岩にぶつかって難破した。	ふねはいわにぶつかってなんぱした 
\\	新年会はとても楽しかった。	しんねんかいはとてもたのしかった 
\\	私の成績は平均以上だ。	わたしのせいせきはへいきんいじょうだ 
\\	成績が大幅に下がった。	せいせきはおおはばにさがった 
\\	大幅に昇給した。	おおはばにしょうきゅうした 
\\	彼女は昇給を要求した。	かのじょはしょうきゅうをようきゅうした 
\\	私たちの昇給の要求は拒絶された。	わたしたちのしょうきゅうのようきゅうはきょぜつされた 
\\	彼は断固として拒絶した。	かれはだんことしてきょぜつした 
\\	私は会議で提案を拒絶された。	わたしはかいぎでていあんをきょぜつされた 
\\	私は彼女の沈黙は拒絶と解釈した。	わたしはかのじょのちんもくはきょぜつとかいしゃくした 
\\	私は彼の沈黙を同意だと解釈した。	わたしはかれのちんもくをどういだとかいしゃくした 
\\	この政策の結果、物価が大幅に上昇した。	このせいさくのけっか、ぶっかがおおはばにじょうしょうした 
\\	彼の政策は時代に先んじていた。	かれのせいさくはじだいにさきんじていた 
\\	一隻の船に全部を積む冒険をするな。	いっせきのふねにぜんぶをつむぼうけんをするな 
\\	そのトラックは家具を積んでいる。	そのトラックはかぐをつんでいる 
\\	雑誌はテーブルの脇に積んである。	ざっしはテーブルのわきにつんでいる 
\\	直線を引きなさい。	ちょくせんをひきなさい 
\\	数学では直線は2つの点によって定義される。	すうがくではちょくせんはふたつのてんによってていぎされる 
\\	この語を明確に定義できますか。	このごをめいかくにていぎできますか 
\\	円錐形の定義を教えてくれませんか。	えんすいけいのていぎをおしえてくれませんか 
\\	日本人は円錐形の山、富士山を誇りにしている。	にほんじんはえんすいけいのやま、ふじさんをほこりにしている 
\\	新しい校舎は村の誇りだ。	あたらしいこうしゃはむらのほこりだ 
\\	全員が仕事に誇りを持っています。	ぜんいんがしごとにほこりをもっています 
\\	負けて彼らの誇りが傷ついた。	まけてかれらのほこりがきずついた 
\\	彼は傷ついて倒れていた。	かれはきずついてたおれていた 
\\	無地の白い紙で十分です。	むじのしろいかみでじゅうぶんです 
\\	私は週単位で支払われる。	わたしはしゅうたんいでしはらわれる 
\\	米はキロ単位で売られる。	こめはキロたんいでうられる 
\\	ポンドは重さの単位である。	ポンドはおもさのたんいである 
\\	ダース単位で卵を買う。	ダースたんいでたまごをかう 
\\	私たちは、公園でボートを時間単位で借りた。	わたしたちは、こうえんでボートをじかんたんいでかりた 
\\	その絵には複雑な模様が見える。	そのえにはふくざつなもようがみえる 
\\	この空模様ではよい天気になりそうだ。	このそらもようではよいてんきになりそうだ 
\\	空模様から判断すると、雨になりそうだ。	そらもようからはんだんすると、あめになりそうだ 
\\	彼女のスカートは黄色で水玉模様がついている。	かのじょのスカートはきいろでみずたまもようがついている 
\\	犬に餌やった?	いぬにえさやった? 
\\	動物に餌を与えないで下さい。	どうぶつにえさをあたえないでください 
\\	鯨はプランクトンと小魚を餌にしている。	くじらはプランクトンとこざかなをえさにしている 
\\	絵本は子供たちの頭を啓発する。	えほんはこどもたちのあたまをけいはつする 
\\	請求書は合計25ドル以上になった。	せいきゅうしょはごうけい25ドルいじょうになった 
\\	この劇は小説から脚色したものです。	このげきはしょうせつかれきゃくしょくしたものです 
\\	この物語はテレビ用に脚色できるかもしれない。	このものがたりはテレビようにきゃくしょくできるかもしれない 
\\	とにかくその問題を解いてみよう。	とにかくそのもんだいをといてみよう 
\\	とにかくその包みはどこかに置きなさい。	とにかくそのつつみはどこかにおきなさい 
\\	ボスは1ヶ月ずっと態度が横柄だった。	ボスはいっかげつずっとたいどがおうへいだった 
\\	少しは腰を据えてやれ。	すこしはこしをすえてやれ 
\\	彼は腰を据えて仕事に取り掛かった。	かれははらをすえてしごとにとりかかった 
\\	向こうで腰をかけたい。	むこうでこしをかけたい 
\\	芝生に腰を下ろしましょう。	しばふにこしをおろしましょう 
\\	彼は芝生の上に寝転がっている。	かれはしばふのうえにねころがっている 
\\	これは電気を作る装置だ。	これはでんきをつくるそうちだ 
\\	みんな、その装置にとても感心した。	みんな、そのそうちにとてもかんしんした 
\\	この機械には安全装置が付いていない。	このきかいにはあんぜんそうちがついていない 
\\	彼らは固く抱き合った。	かれらはかたくだきあった 
\\	彼女は優しく弟を抱きしめた。	かのじょはやさしくおとうとをだきしめた 
\\	彼は無理やり残業させられた。	かれはむりやりざんぎょうさせられた 
\\	彼は無理やりに私に演説させた。	かれはむりやりにわたしにえんぜつさせた 
\\	彼らは私に無理やり署名させた。	かれらはわたしにむりやりしょめいさせた 
\\	警察は彼女を無理やり白状させた。	けいさつはかのじょをむりやりはくじょうさせた 
\\	彼はその花瓶を割ったと白状した。	かれはそのかびんをわったとはくじょうした 
\\	彼女は署名してその金を娘に送った。	かのじょはしょめいしてそのかねをむすめにおくった 
\\	彼は首尾よくその大学に入学した。	かれはしゅびよくそのだいがくににゅうがくした 
\\	新車はテストを受けて首尾よく合格した。	しんしゃはテストをうけてしゅびよくごうかくした 
\\	彼の冗談は侮辱に近い。	かれのじょうだんはぶじょくにちかい 
\\	彼は公然と私を侮辱した。	かれはこうぜんとわたしをぶじょくした 
\\	彼は私をロバと呼んで侮辱した。	かれはわたしをロバとよんでぶじょくした 
\\	彼は理由もなく私を侮辱した。	かれはりゆうもなくわたしをぶじょくした 
\\	あらゆる侮辱が彼に加えられた。	あらゆるぶじょくがかれにくわえられた 
\\	彼女は侮辱されて憤慨した。	かのじょはぶじょくされてふんがいした 
\\	時を移さずそうした。	ときをうつさずそうした 
\\	さて、次の話題に移ろう。	さて、つぎのわだいにうつろう 
\\	彼の芝居は当たった。	かれのしばいはあたった 
\\	このことで彼女の名声は、大いに損なわれた。	このことでかのじょのめいせいは、おおいにそこなわれた 
\\	夜更かししていると健康を損なうよ。	よふかししているとけんこうをそこなうよ 
\\	注意しなくては駄目だよ、さもないとまたやり損なうよ。	ちゅういしなくてはだめだよ、さもないとまたやりそこなうよ 
\\	甥は夜更かしに慣れていた。	おいはよふかしになれていた 
\\	夜更かしなど平気だ。	よふかしなどへいきだ 
\\	彼の冗談が一同をどっと笑わせた。	かれのじょうだんがいちどうをどっとわらわせた 
\\	誰もそのことに気づかせてくれなかった。	だれもそのことにきづかせてくれなかった 
\\	彼女は母親を喜ばせたかった。	かのじょはははおやをよろこばせたかった 
\\	そのニュースは彼を喜ばせた。	そのニュースはかれをよろこばせた 
\\	サーカスは子供たちを驚かせ喜ばせた。	サーカスはこどもたちをおどろかせよろこばせた 
\\	彼女は私の贈り物を少しも喜ばなかった。	かのじょはわたしのおくりものをすこしもよろこばなかった 
\\	彼は情に厚い人だ。	かれはじょうにあついひとだ 
\\	彼は情熱に押し流された。	かれはじょうねつにおしながされた 
\\	その情熱は彼の心の中で燃え尽きた。	そのじょうねつはかれのこころのなかでもえつきた 
\\	それは町の繁栄を脅かすだろう。	それはまちのはんえいをおどかすだろう 
\\	太陽エネルギーは環境を脅かさない。	たいようエネルギーはかんきょうをおどかさない 
\\	大気汚染は我々の生存を脅かすものになるだろう。	たいきおせんはわれわれのせいぞんをおどかすものになるだろう 
\\	生存者の数は死者の数より少なかった。	せいぞんしゃのかずはししゃのかずよりすくなかった 
\\	私たちは四十年以上平和を享受しています。	わたしたちはよんじゅうねんいじょうへいわをきょうじゅしています 
\\	彼は圧力に屈した。	かれはあつりょくにくっした 
\\	彼はその試みに屈した。	かれはそのこころみにくっした 
\\	それは我々に危害を及ぼすだろう。	それはわれわれにきがいをおよぼすだろう 
\\	雨は農作物によい影響を及ぼした。	あめはのうさくぶつによいえいきょうをおよぼした 
\\	台風が勢いを増した。	たいふうがいきおいをました 
\\	私たちは翌日日光を訪れた。	わたしたちはよくじつにっこうをおとずれた 
\\	夕焼けがあると翌日は晴れることが多い。	ゆうやけがあるとよくじつははれることがおおい 
\\	私は時計を買って、その翌日になくしてしまった。	わたしはとけいをかって、そのよくじつになくしてしまった 
\\	私達はパーティーをして新年を祝った。	わたしたちはパーティーしてしんねんをいわった 
\\	ドアがパッと勢い良くあいた。	ドアがパッといきおいよくあいた 
\\	彼は帽子を斜めにかぶっていた。	かれはぼうしをななめにかぶっていた 
\\	私は部屋を花で飾るのが好きだ。	わたしはへやをはなでかざるのがすきだ 
\\	それは飾りに過ぎない	それはかざりにすぎない 
\\	彼らは晴れ着で着飾っている。	かれらははれぎできかざっている 
\\	その王冠は宝石で飾られていた。	そのおうかんはほうせきでかざられていた 
\\	百円じゃなくて、王冠でした。	ひゃくえんじゃなくて、おうかんでした 
\\	私はお酒を控えている。	わたしはおさけをひかえている 
\\	水分を控えてください。	すいぶんをひかえてください 
\\	太陽の照り返しが強い。	たいようのてりかえしがつよい 
\\	地下鉄の痴漢には堪えられない。	ちかてつのちかんはたえられない 
\\	この騒音は耐え難い音だ。	このそうおんはたえかたいおとだ 
\\	この家は地震に耐えますか。	このいえはじしんにたえますか 
\\	私の健康では航海に耐えられない。	わたしのけんこうではこうかいにたえられない 
\\	彼のおかげで就職できた。	かれのおかげでしゅうしょくできた 
\\	悪天候のおかげで試合は中止だ。	あくてんこうのおかげでしあいはちゅうしだ 
\\	就職の面接を受けた。	しゅうしょくのめんせつをうけた 
\\	兄は大企業に就職した。	あにはだいきぎょうにしゅうしょくした 
\\	大企業が業界を支配しています。	だいきぎょうがぎょうかいをしはいしています 
\\	交渉は中止になった。	こうしょうはちゅうしになった 
\\	もし雨なら遠足は中止です。	もしあめならえんそくはちゅうしです 
\\	全体として、遠足は楽しかった。	ぜんたいとして、えんそくはたのしかった 
\\	私たちの遠足は異常な降雪で台無しになった。	わたしたちのえんそくはいじょうなこうせつでだいなしになった 
\\	予報はさらに降雪があると伝えていた。	よほうはさらにこうせつがあるとつたえていた 
\\	我々の交渉は途切れた。	われわれのこうしょうはとぎれた 
\\	両国は平和交渉を開始した。	りょうこくはへいわこうしょうをかいしした 
\\	敵は我々に攻撃を開始した。	てきはわれわれにこうげきをかいしした 
\\	雨は途切れることなく一日中降り続いた。	あめはとぎれることなくいちにちちゅうふりつづいた 
\\	書道を習うのは楽しい。	しょどうをならうのはたのしい 
\\	その問題の解答を教えて下さい。	そのもんだいのかいとうをおしえてください 
\\	君の解答は完全と程遠い。	きみのかいとうはかんぜんとほどとおい 
\\	優勝には程遠い。	ゆうしょうにはほどとおい 
\\	彼の回答は満足なものとは程遠いものだった。	かれはかいとうはまんぞくなものとはほどとおいものだった 
\\	彼女は海辺に別荘を持っている。	かのじょはうみべにべっそうをもっている 
\\	何軒かの別荘が洪水で孤立した。	なんげんかのべっそうがこうずいでこりつした 
\\	その別荘は風景と調和がしていた。	そのべっそうはふうけいとちょうわしていた 
\\	私は熱気球に乗った。	わたしはねつききゅうにのった 
\\	気球は西の方へ漂っていった。	ききゅうはにしのほうへただよっていった 
\\	バラの香りが漂っている。	バラのかおりがただよっている 
\\	工場からの煙が町に漂っていた。	こうじょうからのけむりがまちにただよっていた 
\\	煙突は煙を出し始めた。	えんとつはけむりをだしはじめた 
\\	喫煙が彼の肺を冒した。	きつえんがかれのはいをおかした 
\\	僕は危険を冒すのは嫌いだ。	ぼくはきけんんをおかすのはきらいだ 
\\	私はそんな冒険を冒したくない。	わたしはそんなぼうけんをおかしたくない 
\\	その煙突はレンガ造りだ。	そのえんとつはレンガづくりだ 
\\	煙突は煙を暖炉から外へ煙を出す。	えんとつはけむりをだんろからそとへけむりをだす 
\\	緊急連絡先は何番ですか。	きんきゅうれんらくさきはなんばんですか 
\\	ポーラは緊急の用事で呼び出された。	ポーラはきんきゅうのようじでよべだされた 
\\	あまりしっかりうなぎを握ると、かえって逃げられる。	あまりしっかりうなぎをにぎると、かえってにげられる 
\\	彼はこの問題の鍵を握っている。	かれはこのもんだいのかぎをにぎっている 
\\	彼女はロープをしっかりと握った。	かのじょはロープをしっかりとにぎった 
\\	何事も諦めが肝心だ。	なにごともあきらめがかんじんだ 
\\	彼女は何事にも正直だ。	かのじょはなにごとにもしょうじきだ 
\\	喧嘩では何事も解決しない。	けんかではなにごともかいけつしない 
\\	時が来れば万事解決するだろう。	ときがくればばんじかいけつするだろう 
\\	彼はその問題を容易に解決した。	かれはそのもんだいをよういにかいけつした 
\\	落ち着きが肝心です。	おちつきがかんじんです 
\\	肝心なのは何を読むかではなくて、どう読むかだ。	かんじんなのはなにをよむかではなくて、どうよむかだ 
\\	彼の精神発達は遅かった。	かれのせいしんはったつはおそかった 
\\	彼は留まって父親の商売を営むことにした。	かれはとどまってちちおやのしょうばいをいとなむことにした 
\\	僕は事業を大規模に営んでいる。	ぼくはじぎょうをだいきぼにいとなんでいる 
\\	その工場は大規模で運営されている。	そのこうじょうはだいきぼでうんえいされている 
\\	その地方は大規模に開発されるだろう。	そのちほうはだいきぼにかいはつされるだろう 
\\	我が国は自国の天然資源を開発しなければならない。	わがくにはじこくてんねんしげんをかいはつしなければならない 
\\	日本は天然資源に乏しい。	にほんはてんねんしげんにとぼしい 
\\	中国は天然資源が豊富だ。	ちゅうごくはてんねんしげんがほうふだ 
\\	彼らは自国の権利を擁護した。	かれらはじこくのけんりをようごした 
\\	その審判は自国の肩をもった。	そのしんぱんはじこくのかたをもった 
\\	あの候補者は自由貿易の擁護者である。	あのこうほしゃはじゆうぼうえきのようごしゃである 
\\	私たちは民主主義擁護のために戦わねばならない。	わたしはみんしゅしゅぎようごのためにたたかわねばならない 
\\	アメリカは自国の問題を自力で解決するであろう。	アメリカはじこくのもんだいをじりきでかいけつするであろう 
\\	彼らは高度な技術のコンピューターを開発した。	かれらはこうどなぎじゅつのコンピューターをかいはつした 
\\	この仕事は高度の熟練を必要とする。	このしごとはこうどのじゅくれんをひつようとする 
\\	日本では稲作が高度に発達した。	にほんでいなさくがこうどにはったつした 
\\	彼は工作に熟練している。	かれはこうさくにじゅくれんしている 
\\	彼は星空を見上げた。	かれはほしぞらをみあげた 
\\	山小屋で一夜を明かした。	やまごやでいちやをあかした 
\\	山小屋は頂上の下の方に在る。	やまごやはちょうじょうのしたのほうにある 
\\	夕焼けで西の空は真っ赤に輝く。	ゆうやけでにしのそらはまっかにかがやく 
\\	山は夕焼け色で燃えるようだった。	やまはゆうやけいろでもえるようだった 
\\	まもなく初雪が降った。	まもなくはつゆきがふった 
\\	どうしてあなたにお詫びする理由があるのですか。	どうしてあなたにおわびするりゆうがあるのですか 
\\	屋根が破損した家は今では修理が完了している。	やねがはそんしたいえはいまではしゅうりかかんりょうしている 
\\	任務を完了する。	にんむをかんりょうする 
\\	彼は任務を成し遂げた。	かれはにんむをなしとげた 
\\	彼は特別な任務でヨーロッパに派遣された。	かれはとくべつなにんむでユーロッパにはけんされた 
\\	彼は特派員として海外に派遣された。	かれはとくはいんとしてかいがいにはけんされた 
\\	彼女は派遣社員です。	かのじょははけんしゃいんです 
\\	その特派員はモスクワから記事を送った。	そのとくはいんはモスクワからきじをおくった 
\\	新聞から記事を切り抜いた。	しんぶんからきじをきりぬいた 
\\	私たちの学校の記事が新聞に出た。	わたしたちのがっこうのきじがしんぶんにでた 
\\	ご無沙汰しています。	ごぶさたしています 
\\	陸が見えてきた。	りくがみえてきた 
\\	飛行機は無事着陸した。	ひこうきはぶじちゃくりくした 
\\	軍隊はギリシャに上陸した。	ぐんたいはギリシャにじょうりくした 
\\	あなたは好き放題に酒を飲んではならない。	あなたはすきほうだいにさけをのんではならない 
\\	さあ、みんなで食べ放題の焼き肉屋さんに行こうよ。	さあ、みんなでたべほうだいのやきにくやさんにいこうよ 
\\	部屋には本が散らかっていた。	へやにはほんがちらかっていた 
\\	部屋は散らかし放題だった。	へやはちらかしほうだいだった 
\\	台所をこんなに散らかしたのは誰だ。	だいどころをこんなにちらかしたのはだれだ 
\\	僕は散らかっている家は我慢できるが、不潔な家は嫌いだ。	ぼくはちらかっているいえはがまんできるが、ふけつないえはきらいだ 
\\	不潔は病気を生み出す。	ふけつはびょうきをうみだす 
\\	彼は大統領を辞任した。	かれはだいとうりょうをじにんした 
\\	彼は近く辞任するという噂だ。	かれはちかくじにんするといううわさだ 
\\	男の人が車に撥ねられて、運転手が車で走り去った。	おとこのひとがくるまにはねられて、うんてんしゅがくるまではしりさった 
\\	彼は通りを横切っていて危うく車に撥ねられそうになった。	かれはとおりをよこきっていてあやうくくるまにはねられそうになった 
\\	彼は危うく死を免れた。	かれはあやうくしをまぬかれた 
\\	彼は危うく溺死するところだった。	かれはあやうくできしするところだった 
\\	彼は船から転落し溺死した。	かれはふねからてんらくしできしした 
\\	彼は列車を崖から転落させた。	かれはれっしゃをがけからてんらくさせた 
\\	彼は徹底した利己主義者だ。	かれはてっていしたりこしゅぎしゃだ 
\\	私はそれを徹底的に調べた。	わたしはそれをてっていてきにしらべた 
\\	どうせ英語をやるのなら徹底的にやれ。	どうせえいごをやるのならてっていてきにやれ 
\\	赤と青を混ぜると紫になる。	あかとあおをまぜるとむらさきになる 
\\	小銭を混ぜてください。	こぜにをまぜてください 
\\	粉と卵2個を混ぜなさい。	こなとたまご2こをまぜなさい 
\\	彼女はバターと砂糖を混ぜ合わせた。	かのじょはバターとさとうをまぜあわせた 
\\	先ず卵を強くかき混ぜそれをスープに加えます。	まずたまごをつよくかきまぜそれをスープにくわえます 
\\	私は急いで計算をした。	わたしはいそいでけいさんをした 
\\	彼は光の速度を計算した。	かれはひかりのそくどをけいさんした 
\\	女性は計算に弱いと言う。	じょせいはけいさんによわいという 
\\	計算機はすばらしい発明品だ。	けいさんきはすばらしいはつめいひんだ 
\\	彼の論文は申し分ない。	かれのろんぶんはもしぶんない 
\\	彼の論文は小説のように読める。	かれのろんぶんはしょうせつのようによめる 
\\	彼の論文には決して満足出来ない。	かれのろんぶんにはけっしてまんぞくできない 
\\	彼の卒業論文は私のと関係がある。	かれのそつぎょうろんぶんはわたしとかんけいがある 
\\	メグはトマトの缶詰を買った。	メグはトマトのかんづめをかった 
\\	彼は缶詰をあけているうちに指を切った。	かれはかんづめをあけているうちにゆびをきった 
\\	彼らは果物を保存するために缶詰めにした。	かれらはくだものをほぞんするためにかんづめにした 
\\	よい伝統は保存されるべきだ。	よいでんとうはほぞんされるべきだ 
\\	私達は田舎の美しさを保存するべきだ。	わたしたちはいなかのうつくしさをほぞんするべきだ 
\\	最新のメールの保存場所を間違ってしまったようです。	さいしんのメールのほぞんばしょをまちがってしまったようです 
\\	看護婦は体温計で彼の体温を計った。	かんごふはたいおんけいでかれのたいおんをはかった 
\\	体温計を口にくわえて、体温を測りましたか。	たいおんけいをくちにくわえて、たいおんをはかりましたか 
\\	あなたの血液型は何ですか。	あなたのけつえきけいはなんですか 
\\	血液のしみはたいてい落ちない。	けつえきのしみはたいていおちない 
\\	きついバンドは血液の循環を妨げる。	きついバンドはけつえきのじゅんかんをさまたげる 
\\	適度の運動は血液の循環を活発にする。	てきどのうんどうはけつえきのじゅんかんをかっぱつにする 
\\	うちの母は活発です。	うちのはははかっぱつです 
\\	彼は適度な量のコーヒーを飲む。	かれはてきどなりょうのコーヒーをのむ 
\\	極端に走ってはいけない。適度であることは何事においても大切である。	きょくたんにはしってはいけない。てきどであることはなにごとにおいてもたいせつである 
\\	経済は今景気循環の頂点にある。	けいざいはいまけいきじゅんかんのちょうてんにある 
\\	自動車生産は頂点を超えた。	じどうしゃせいさんはちょうてんをこえた 
\\	生産が落ち始めている。	せいさんがおちはじめている 
\\	カナダは良質の小麦を生産する。	カナダはりょうしつのこむぎをせいさんする 
\\	世界的な農業生産高は伸びていた。	せかいてきなのうぎょうせいさんだかはのびていた 
\\	神戸は良質の牛肉で有名だ。	こうべはりょうしつのぎゅうにくでゆうめいだ 
\\	あの店では良質の食料品を売っている。	あのみせではりょうしつのしょくりょうひんをうっている 
\\	私は一つの会社に束縛されたくない。	わたしはひとつのかいしゃにそくばくされたくない 
\\	彼は人々を束縛から開放した。	かれはひとびとをそくばくからかいほうした 
\\	私は邪魔されたくない。	わたしはじゃまされたくない 
\\	私は騙すよりも寧ろ騙されたい。	わたしはだますよりもむしろだまされたい 
\\	先生というより寧ろ学者だ。	せんせいというよりむしろがくしゃだ 
\\	今日は寧ろ外出したくない。	きょうはむしろがいしゅつしたくない 
\\	同僚を騙すのは良くないよ。	どうりょうをだますのはよくないよ 
\\	彼は顧客から大金を騙し取った。	かれはこきゃくからたいきんをだましとった 
\\	彼女がうちの子供たちを騙したので腹が立った。	かのじょはうちのこどもたちをだましたのではらがたった 
\\	婦人に開放されている職業は多い。	ふじんにかいほうされているしょくぎょうはおおい 
\\	この庭は一般に開放されています。	このにわはいっぱんにかいほうされています 
\\	いよいよ彼女の番になった。	とうとう。ついに。
\\	雨はやむどころかいよいよひどくなった。	あめはやめどころかいよいよひどくなった 
\\	いよいよという時に言葉が出ない。	いよいよというときにことばがでない 
\\	彼は彼女のアイルランド訛りを真似るのが上手い。	かれはかのじょのアイランドなまりをまねるのがうまい 
\\	この鳥は人の声を真似できる。	このとりはひとのこえをまねできる 
\\	健康と快活さは美を生む。	けんこうとかいかつさはびをうむ 
\\	彼女は以前のような快活な女性ではない。	かのじょはいぜんのようなかいかつなじょせいではない 
\\	晴れれば、森の中に散歩に出かけよう。	はれれば、もりのなかにさんぽにでかけよう 
\\	東京のラッシュ時は、交通量が多い。	とうきょうのラッシュじは、こうつうりょうがおおい 
\\	彼はラッシュ時の通勤を避けられない。	かれはラッシュじのつうきんをさけられない 
\\	自転車で通勤しています。	じてんしゃでつうきんしています 
\\	私は遠い郊外からわざわざ通勤しなくてはいけない。	わたしはとおいこうがいからわざわざつうきんしなくてはいけない 
\\	悪友は避けるべきだ。	あくゆうはさけるべきだ 
\\	塩を過剰に使うのは避けるべきだ。	しおをかじょうにつかうはさけるべきだ 
\\	あなたは嫌な経験を避けるかもしれない。	あなたはいやなけいけんをさけるかもしれない 
\\	君にいくつか忠告させてくれ。	きみにいくつかちゅうこくさせてくれ 
\\	子供が猿にバナナを食べさせていた。	こどもがさるにバナナをたべさせていた 
\\	最後まで話をつづけさせてください。	さいごまではなしをつづけさせてください 
\\	あなた方に妻を紹介させてください。	あなたがたにつまをしょうかいさせてください 
\\	彼はその小包を縛った。	かれはそのこづつみをしばった 
\\	これを小包郵便で送るのですか。	これをこづつみゆうびんでおくるのですか 
\\	この小包を船便で送ってください。	このこづつみをふなびんでおくってください 
\\	小包は隣の窓口で取り扱っている。	こづつみはとなりのまどぐちでとりあつかっている 
\\	自信過剰にならないよう注意しなくてはいけない。	じしんかじょうにならないようちゅういしなくてはいけない 
\\	時々イタリア人の大道音楽家が町にやって来ました。	ときどきイタリアじんのだいどうおんがくかがまちにやってきました 
\\	彼は芸人として成功した。	かれはげいにんとしてせいこうした 
\\	彼は教師というよりは芸人です。	かれはきょうしというよりはげいにんです 
\\	バスは早めに出発した。	バスははやめにしゅっぱつした 
\\	私は会議に間に合うよう早めに家を出た。	わたしはかいぎにまにあうようはやめにいえをでた 
\\	よい席をとることを目当てに早めに劇場にいった。	よいせきをとることとをめあてにはやめにげきじょうにいった 
\\	劇場の裏に駐車場がある。	げきじょうのうらにちゅしゃじょうがある 
\\	劇場から続々と人が出てきた。	げきじょうからぞくぞくとひとがでてきた 
\\	常に空腹感があります。	つねにくうふくかんがあります 
\\	水割りにしてください。	みずわりにしてください 
\\	丘は紅葉が美しい。	おかはもみじがうつくしい 
\\	秋になると山全体が紅葉する。	あきになるとやまぜんたいがもみじする 
\\	蛇口から水が漏れていますよ。	じゃぐちからみずがもれていますよ 
\\	水が壊れた蛇口から吹き出した。	みずがこわれたじゃぐちからふきだした 
\\	水道の蛇口が壊れていて使えなかった。	すいどうのじゃぐちがこわれていてつかえなかった 
\\	この蛇口は使えません。故障しています。	このじゃぐちはつかえません。こしょうしています 
\\	その話が漏れると私は困ったことになる。	そのはなしがもれるとわたしはこまったことになる 
\\	秘密が漏れたらしい。	ひみつがもれたらしい 
\\	情報が外部に漏れたらしい。	じょうほうががいぶにもれたらしい 
\\	彼らはニュースが漏れないように努めた。	かれらはニュースがもれないようにつとめた 
\\	外部からの助言が必要かもしれません。	がいぶからのじょげんがひつようかもしれません 
\\	彼は党を内部から改革しようとした。	かれはとうをないぶからかいかくしようとした 
\\	その家の内部はとても魅力的だった。	そのいえのないぶはとてもみりょくてきだった 
\\	谷は赤や黄色の葉で色づいていた。	たにはあかやきいろのはでいろづいていた 
\\	栓抜きは瓶のふたを開けるのに使います。	せんぬきはびんのふたをあけるのにつかいます 
\\	彼は自分の身辺を整理した。	かれはじぶんのしんぺんをせいりした 
\\	これは私に思考の整理の仕方を教えることになった。	これはわたしのしこうのせいりのしかたをおしえることになった 
\\	この論文は私の思考に影響を及ぼすだろう。	このろんぶんはわたしのしこうにえいきょうをおよぼすだろう 
\\	痛みが少し減りました。	いたみがすこしへりました 
\\	会員は全員出席した。	かいいんはぜんいんしゅっせきした 
\\	彼女はこの団体の会員である。	かのじょはこのだんたいのかいいんである 
\\	我々はその団体に参加しよう。	われわれはそのだんたいにさんかしよう 
\\	僕は文学の団体に所属している。	ぼくはぶんがくのだんたいにしょぞくしている 
\\	彼は年齢を偽った。	かれはねんれいをいつわった 
\\	彼は年齢の割には背が高い。	かれはねんれいのわりにせがたかい 
\\	彼は車が運転できる年齢だ。	かれはくるまがうんてんできるねんれいだ 
\\	彼の頭脳の働きは活発だった。	かれのずのうのうごきはかっぱつだった 
\\	彼は我が国有数の頭脳の一人だ。	かれはわがくにゆうすうのずのうのひとりだ 
\\	日本は世界有数の経済大国である。	にほんはせかいゆうすうおけいざいたいこくである 
\\	この山は世界でも有数の高い山です。	このやまはせかいでもゆうすうのやまです 
\\	二人の頭脳は一人の頭脳に勝る。	ふたりのずのうはひとりのずのうにまさる 
\\	睡眠は薬に勝る。	すいみんはくすりにまさる 
\\	知恵は富に勝る。	ちえはとみにまさる 
\\	彼は約束に誠実である。	かれはやくそくにせいじつである 
\\	彼女は行動も言葉も誠実だ。	かのじょはこうどうもことばもせいじつだ 
\\	彼の誠実さを疑うひともいた。	かれのせいじつさをうたがうひともいた 
\\	彼は性質が頑固だ。	かれはせいしつががんこだ 
\\	彼は人懐っこい性質だ。	かれはひとなつっこいせいしつだ 
\\	ほとんどの犬は人懐っこいものだ。	ほとんどのいぬはひとなつっこいものだ 
\\	その2つの性質は相容れない。	そのふたつのせいしつはあいいれない 
\\	我々の利益は彼らの利益と相容れない。	われわれのりえきはかれらのりえきとあいいれない 
\\	私は気が短いし、口も軽い男だ。	わたしはきがみじかいし、くちもかるいおとこだ 
\\	彼は素直に私の欠点を指摘した。	かれはすなおにけってんをしてきした 
\\	素直に言うと、彼は信頼できない男だ。	すなおにいうと、かれはしんらいできないおとこだ 
\\	君は騒音に神経質すぎる。	きみはそうおんにしんけいしつすぎる 
\\	批評にそんなに神経質になるな。	ひひょうにそんなにしんけいしつになるな 
\\	神経質な人はこの仕事には向かない。	しんけいしつなひとはこのしごとにはむかない 
\\	その神経質の少女は鉛筆の端をかむ癖がある。	そのしんけいしつのしょうじょはえんぴつのはしをかむくせがある 
\\	彼女はテレビの調節を彼に頼んだ。	かのじょはテレビのちょうせつをかれにたのんだ 
\\	テレビの画面を調節してください。	テレビのがめんをちょうせつしてください 
\\	この机は子供たちに合わせて高さが調節できる。	このつくえはこどもたちにあわせてたかさがちょうせつできる 
\\	彼は望遠鏡を自分の目に合うように調節した。	かれはぼうえんきょうをじぶんのめにあうようにちょうせつした 
\\	画面がもっと見えるように近づきたい。	がめんがもっとみえるようにちかづきたい 
\\	私は洋食はあまり好きではない。	わたしはようしょくはあまりすきではない 
\\	タンスで探していた物が見つかった。	タンスでさがしていたものがみつかった 
\\	猫はぬれるのを嫌がる。	ねこはぬれるのをいやがる 
\\	彼女は名前を言うのを嫌がった。	かのじょはなまえをいうのをいやがった 
\\	誰も事件の順序を思い出すことができなかった。	だれもじけんのじゅんじょをおもいだすことができなかった 
\\	何をするにも順序を踏んでやりなさい。	なにをするにもじゅんじょをふんでやりなさい 
\\	僕は君の怠慢が気に入らないんだ。	ぼくはきみのたいまんがきにいらないんだ 
\\	彼は義務怠慢で非難された。	かれはぎむたいまんでひなんされた 
\\	その事故の責任は管理人の怠慢にある。	そのじこのせきにんはかんりにんのたいまんにある 
\\	彼は管理職に昇進した。	かれはかんりしょくにしょうしんした 
\\	彼に財産管理を任せた。	かれにざいさんかんりをまかせた 
\\	彼は病院の管理運営に責任がある。	かれはびょういんのかんりうんえいにせきにんがある 
\\	その公園は市に管理されている。	そのこうえんはしにかんりされている 
\\	経営管理に経験のある方を求めます。	けいえいかんりにけいけんのあるかたをもとめます 
\\	昇進を期待して一生懸命働いた。	そうしんをきたいしていっしょうけんめいはたらいた 
\\	私は昇進を切に願っている。	わたしはしょうしんをせつにねがっている 
\\	彼女は慈悲を願った。	かのじょはじひをねがった 
\\	彼は私たちの幸福を願ってくれている。	かれはわたしたちのこうふくをねがってくれている 
\\	彼は本屋に立ち寄った。	かれはほにゃにたちよった 
\\	私は東京へ行く途中大阪へ立ち寄った。	わたしはとうきょうへいくとちゅうおおさかへたちよった 
\\	こちらに立ち寄った際には電話をかけてね。	こちらにたちよったさいにはでんわをかけてね 
\\	実際よい案が浮かんだ。	じっさいよいあんがうかんだ 
\\	彼は村人たちと交際しない。	かれはむらびとたちとこうさいしない 
\\	何ヶ月か彼女と交際している。	なんかげつかかのじょとこうさいしている 
\\	彼との交際で得るところが多かった。	かれとこうさいでえるところがおおかった 
\\	そんな利己的な男と交際してはいけない。	そんなりこてきなおとことこうさいしてはいけない 
\\	あの人って本当に気分屋ね。私、ついていけない。	あのひとってほんとうにきぶんやね。わたし、ついていけない 
\\	翻訳は母国語をよりよく知るのに役立つ。	ほにゃくはぼこくごをよりよくしるのにやくだつ 
\\	その外国人はまるで母国語のように日本語を話した。	そのがいこくじんはまるでぼこくごのようににほんごをはなした 
\\	光が刺激となって花が咲く。	ひかりがしげきとなってはながさく 
\\	私の忠告は彼女に刺激となった。	わたしのちゅうこくはかのじょにしげきとなった 
\\	人々は強い刺激を求める傾向がある。	ひとびとはつよいしげきをもとめるけいこうがある 
\\	刺激的な町だよ。ニューヨークは。	しげきてきなまちだよ。ニューヨークは。 
\\	物価は上昇傾向にある。	ぶっかはじょうしょうけいこうにある 
\\	彼は傲慢になる傾向がある。	かれはごうまんになるけいこうがある 
\\	彼は嘘をつく傾向がある。	かれはうそをつくけいこうがある 
\\	彼女は早口で話す傾向が多い。	かのじょははやぐちではなすけいこうがおおい 
\\	彼は何でも大げさに言う傾向がある。	かれはなんでもおおげさにいうけいこうがある 
\\	彼の意見は保守的傾向を帯びている。	かれのいけんはほしゅてきけいこうをおびている 
\\	世界の人口は増加する傾向にある。	せかいのじんこうはぞうかするけいこうにある 
\\	彼は何でもやりすぎる傾向がある。	かれはなんでもやりすぎるけいこうがある 
\\	彼女は、学校に遅れる傾向がある。	かのじょは、がっこうにおくれるけいこうがある 
\\	彼女は人の悪口を言う傾向がある。	かのじょはひとのわるぐちをいうけいこうがある 
\\	概して人間は怠ける傾向にある。	がいしてにんげんはなまけるけいこうにある 
\\	彼の怠けぶりを許せない。	かれのなまけぶりをゆるせない 
\\	困ったことに彼は怠け者だ。	こまったことにかれはなまけものだ 
\\	服の選びかたは保守的になってきている。	ふうのえらびかたはほしゅてきになってきている 
\\	いかなる政党も本質的に保守的である。	いかなるせいとうもほんしつてきにほしゅてきである 
\\	私たちの存在は本質的に1つの奇跡だ。	わたしたちのそんざいはほんしつてきにひとつのきせきだ 
\\	私が癌を克服したのは奇跡だ。	わたしはがんをこくふくしたのはきせきだ 
\\	彼は多くの困難を克服した。	かれはおおくのこんなんをこくふくした 
\\	彼は妻の死をまだ克服していない。	かれはつまのしをまだこくふくしていない 
\\	病人は遂に病気を克服した。	びょうにんはついにびょうきをこくふくした 
\\	われわれの脱出は奇跡というほかなかった。	われわれのだっしゅつはきせきというほかなかった 
\\	私はやっとのことで沈んでいく船から脱出した。	わたしはやっとのことでしずんでいくふねからだっしゅつした 
\\	アンはどのように脱出したのかを説明してくれた。	アンはどのようにだっしゅつしたのかをせつめいしてくれた 
\\	東の空が赤みを帯びている。	ひがしのそらがあかみをおびている 
\\	その問題は新しい性格を帯び始めた。	そのもんだいはあたらしせいかくをおびはじめた 
\\	現代は本質的に悲劇的な時代だと言われるのをよく耳にする。	げんだいはほんしつてきにひげきてきなじだいだといわれるのをよくみみにする 
\\	彼は一目見て彼女に恋をした。	かれはひとめみてかのじょにこいをした 
\\	一目見て彼には少年が空腹なのが分かった。	ひとめみてかれにはしょうねんがくうふくなのがわかった 
\\	牧師は私に彼の祝福を与えた。	ぼくしはわたしにかれのしゅくふくをあたえた 
\\	彼は牧師に自分の罪を告白した。	かれはぼくしにじぶんのつみをこくはくした 
\\	その牧師は貧乏人のために懸命に働いた。	そのぼくしはびんぼうにんのためにけんめいにはたらいた 
\\	すべての牧師が新しい法律に反対である。	すべてのぼくしがあたらしいほうりつにはんたいである 
\\	私たちは彼の成功を祝福した。	わたしたちはかれのせいこうをしゅくふくした 
\\	大体君と同じ年頃の若い娘。	だいたいきみとおなじとしごろのわかいむすめ 
\\	君はもう、自活できる年頃だ。	きみはもう、じかつできるとしごろだ 
\\	もっと分別があってもいい年頃だよ。	もっとぶんべつがあってもいいとしごろだよ 
\\	君もそろそろ結婚してもいい年頃だね。	きみはそろそろけっこんしてもいいとしごろだね 
\\	彼女にはかなり分別があると思う。	かのじょにはかなりふんべつがあるとおもう 
\\	彼はそう言わないだけの分別は得ていた。	かれはそういわないだけのふんべつはえていた 
\\	彼女には申し出を断るだけの分別があった。	かのじょはもうしでをことわるだけのふんべつがあった 
\\	彼は自活するので精一杯だ。	かれはじかつするのでせいいっぱいだ 
\\	これが精一杯です。	これがせいいっぱいです 
\\	彼は僅かな収入を精一杯活かした。	かれはわずかなしゅにゅうをせいいっぱいいかした 
\\	あなたはこのチャンスを活かすべきだ。	あなたはこのチャンスをいかすべきだ 
\\	彼はどこからともなく現れた。	かれはどこからともなくあらわれた 
\\	私は待ちに待ったが、遂にジョンがやってきた。	わたしはまちにまったが、ついにジョンがやってきた 
\\	私は待ちに待ちました。	わたしはまちにまちました 
\\	余りにも多くの人が政治に無関心である。	あまりにもおおくのひとがせいじにむかんしんである 
\\	彼は食べ物には無関心である。	かれはたべものにはむかんしんである 
\\	青年にはよくあることだが彼はお金に無関心である。	せいねんにはよくあることだがかれはおかねにむかんしんである 
\\	青年は失恋した。	せいねんはしつれんした 
\\	彼は有望な青年です。	かれはゆうぼうなせいねんです 
\\	彼の自殺は失恋の結果であった。	かれのじさつはしつれんのけっかであった 
\\	もうこの失恋の痛みに堪えることができない。	もうこのしつれんのいたみにたえることができない 
\\	彼は将来有望な記者と思う。	かれはしょうらいゆうぼうなきしゃとおもう 
\\	彼は将来有望な若手事業家だ。	かれはしょうらいゆうぼうなわかてじぎょうかだ 
\\	時が彼女の悲しみを癒した。	ときはかのじょのかなしみをいやした 
\\	心の傷は癒せない。	こころのきずはいやせない 
\\	君は悪い前例を作ってしまった。	きみはわるいぜんれいをつくってしまった 
\\	日本経済は当時、前例のない好況にあった。	にほんけいざいはとうじ、ぜんれいのないこうきょうにあった 
\\	その業界はこれまでにも何度も好況と不況を繰り返してきた。	そのぎょうかいはこれまでにもなんどもこうきょうとふきょうをくりかえしてきた 
\\	ここ2、3年は不況が続いている。	ここ2,3ねんはふきょうがつづいている 
\\	大企業が業界を支配しています。	だいきぎょうがぎょうかいをしはいしています 
\\	来年は不況が避けられませんよ。	らいねんはふきょうがさけられませんよ 
\\	悪友は避けるべきだ。	あくゆうはさけるべきだ 
\\	私は出来るだけトンネルを避けるようにします。	わたしはできるだけトンネルをさけるようにします 
\\	夜10時以後に人に電話するには避けるべきだ。	よる10じいごにひとにでんわするにはさけるべきだ 
\\	互いに顔を合わせるのを避ける。	たがいにかおをあわせるのをさける 
\\	事故は避けられない。	じこはさけられない 
\\	私は争いを避けようとした。	わたしはあらそいをさけようとした 
\\	ポスト争いは厳しい。	ポストあらそいはきびしい 
\\	町の他の地域では争いが続いた。	まちのほかのちいきであらそいがつづいた 
\\	彼は彼らの争いを和解させるつもりだ。	かれはかれらのあらそいをわかいさせるつもりだ 
\\	経営者側と組合は和解した。	けいえいしゃがわとくみあいはわかいした 
\\	労働組合はストを宣言した。	ろうどうくみあいはストをせんげんした 
\\	組合はデモに参加しましたか。	くみあいはデモにさんかしましたか 
\\	彼らは組合の指導者たちと折り合った。	かれらはくみあいのしどうしゃたちとおりあった 
\\	家賃に関して私は彼と折り合いがついた。	やちんにかんしてわたしはかれをおりあいがついた 
\\	彼はなかなか折り合ってつきあいにくい人だ。	かれはおりあってつきあいにくいひとだ 
\\	その植民地は独立を宣言した。	そのしょくみんちはどくりつをせんげんした 
\\	市長は調査の結果を公表すると宣言した。	しちょうはちょうさのけっかをこうひょうするとせんげんした 
\\	アメリカ政府は非常事態宣言を行った。	アメリカせいふはひじょうじたいせんげんをおこなった 
\\	彼の死が公表された。	かれのしはこうひょうされた 
\\	彼は彼女との婚約を公表した。	かれはかのじょとのこんやくをこうひょうした 
\\	彼はそれを公表すると脅した。	かれはそれをこうひょうするとおどした 
\\	彼の要求は脅しに近かった。	かれのようきゅうはおどしにちかかった 
\\	英国は多くの植民地を設けた。	えいこくはおおくのしょくみんちをもうけた 
\\	脅かさないでよ。	おどかさないでよ 
\\	病気は人類にとって脅威である。	びょうきはじんるいにとってきょういである 
\\	この作戦に対する脅威は数多い。	このさくせんにたいするきょういはかずおおい 
\\	原爆は人類にとって重大な脅威だ。	げんばくはじんるいにとってじゅうだいなきょういだ 
\\	誰もその作戦に失敗しなかった。	だれでもそのさくせんにしっぱいしなかった 
\\	ホールは仕切りを設ける前は千人収容できた。	ホールはしきりをもうけるまえはせんにんしゅうようできた 
\\	弾丸は仕切り壁を貫いた。	だんがんはしきりかべをつらぬいた 
\\	弾丸は的をかすめた。	だんがんはまとをかすめた 
\\	彼は一発の弾丸で殺された。	かれはいっぱつのだんがんでころされた 
\\	この部屋は三百人収容できる。	このへやはさんびゃくにんしゅうようできる 
\\	このホテルは千人を収容する設備がある。	このホテルはせんにんをしゅうようするせつびがある 
\\	悲しみが彼の胸に貫いた。	かなしみがかれのむねにつらぬいた 
\\	大きな川がその市を貫いて流れている。	おおきなかわがそのしをつらぬいてながれている 
\\	彼らは私達の町の上流階級の人とみなされている。	かれらはわたしたちのまちのじょうりゅうかいきゅうのひととみなされている 
\\	彼はもう略父と同じ身長だ。	かれのしんちょうほぼちちとおなじしんちょうだ 
\\	この線があなたの身長を示します。	このせんがあなたのしんちょうをしめします 
\\	その選手達の平均身長はどのくらいですか。	そのせんしゅたちのへいきんしんちょうはどのぐらいですか 
\\	彼は骨の髄まで日本人だ。	かれはほねのずいまでにほんじんだ 
\\	彼は骨の髄まで腐りきっている。	かれはほねのずいまでくさりきっている 
\\	彼は昨日入院した。	かれはきのうにゅういんした 
\\	入院治療が必要です。	にゅういんちりょうがひつようです 
\\	彼は治療を断られた。	かれはちりょうをことわられた 
\\	医師は彼女の怪我の治療をした。	いしはかのじょのけがのちりょうをした 
\\	がんにも治療可能なものがある。	がんにもちりょうかのうなものがある 
\\	今週は火災予防週間です。	こんしゅうはかさいよぼうしゅうかんです 
\\	病気を予防することはできますか。	びょうきをよぼうすることはできますか 
\\	予防接種を受けていかなきゃいけない。	よぼうせっしゅをうけていかなきゃいけない 
\\	予防運転は事故を防ぎます。	よぼううんてんはじこをふせぎます 
\\	戦争を防ぐ最も確かな方法は戦争を恐れないことである。	せんそうをふせぐもっともたしかなほうほうはせんそうをおそれないことである 
\\	堤防が洪水を防いだ。	ていぼうがこうずいをふせいだ 
\\	堤防が都市を洪水から守ってくれた。	ていぼうがとしをこうずいからまもってくれた 
\\	私たちはその少年が窓を破るのを防いだ。	わたしたちはしょうねんがまどをやぶるのをふせいだ 
\\	車が通路を防いでいた。	くるまはつうろをふせいでいた 
\\	通路側には座れる。	つうろがわにはすわれる 
\\	我々はその建物への秘密の通路を発見した。	われわれはそのたてものへのひみつのつうろをはっけんした 
\\	通路に物を置くな。	つうろにものをおくな 
\\	子供たちは堤防を滑り降りた。	こどもたちはていぼうをすべりおりた 
\\	インフルエンザの予防接種をしました。	インフルエンザのよぼうせっしゅをしました 
\\	もし予防接種証明書があったらお待ちください。	もしよぼうせっしゅしょうめいしょがあったらおもちください 
\\	それを証明できますわ?	それをしょうめいできますわ? 
\\	身分証明をお持ちですか。	みぶんしょうめいをおもちですか 
\\	彼の無実を証明できますか。	かれのむじつをしょうめいできますか 
\\	市長は私に身分証明書をくれた。	しちょうはわたしにみぶんしょうめいしょをくれた 
\\	外人登録証をお持ちですか。	がいじんとうろくしょうをもちですか 
\\	彼女は1時間前に退院しました。	かのじょはいちじかんまえにたいいんしました 
\\	昨日彼のお母さんに会ったら、彼は1週間前に退院したと言うのです。	きのうかれのおかあさんにあったら、かれはいっしゅうかんまえにたいいんしたというのです 
\\	その家は私の要求にぴったりだ。	そのいえはわたしのようきゅうにぴったりだ 
\\	彼の靴に泥がぴったりくっついていた。	かれのくつにどろがぴったりくっついていた 
\\	その子供は母親にぴったりとくっついていた。	そのこどんもはははおやにぴったりくっついていた 
\\	このワンピースは、君の雰囲気にぴったりだね。	このワンピースは、きみのふんいきにぴったりだね 
\\	その言葉がそれを説明するにはぴったりだね。	そのことばがそれをせつめいするにはぴったりだね 
\\	彼は頭が切れる。	かれはあたまがきれる 
\\	騒音で頭が変になりそうだ。	そうおんであたまがへんになりそうだ 
\\	コーヒーを一杯飲むと頭が冴える。	コーヒーをいっばいのむとあたまがさえる 
\\	昨日ゆっくり休んだ分、今日は頭が冴えている。	きのうゆっくりやすんだぶん、きょうはあたまがさえている 
\\	この問題は難しくて私には歯がたたない。	このもんだいはむずかしくてわたしにははがたたない 
\\	この歯が不安定です。	このははふあんていです 
\\	日本人からが、アメリカ人は不安定で、自由奔放に見える。	にほんじんからが、アメリカじんはふあんてい、じゆうほんぽうにみえる 
\\	たまに彼女の自由奔放な態度は無礼に見えることがある。	たまにかのじょのじゆうほんぽうなたいどはぶれいにみえることがある 
\\	奔放に生きる。	ほんぽうにいきる 
\\	彼は安定を失って倒れた。	かれはあんていをうしなってたおれた 
\\	日本の円は安定した通貨だ。	にほんのえんはあんていしたつうかだ 
\\	このところ物価が安定している。	このところぶっかがあんていしている 
\\	その国の政府は今安定している。	そのくにのせいふはいまあんていしている 
\\	賃金よりも職の安定の方が重要である。	ちんぎんよおりもしょくのあんていのほうがじゅうようである 
\\	会社は従業員を低賃金で不当に利用した。	かいしゃはじゅうぎょういんをていちんぎんでふとうにりようした 
\\	彼らは低賃金に不満を言った。	かれらはていちんぎんにふまんをいった 
\\	われわれの会社は賃金が低い。	われわれのかいしゃはちんぎんがひくい 
\\	先生を不当に非難する。	せんせいをふとうにひなんする 
\\	あなたは彼の不当な要求に屈してはいけない。	あなたはかれのふとうなようきゅうにくっしてはいけない 
\\	その女性は不当に私を責めた。	そのじょせいはふとうにわたしをせめた 
\\	当地では日本の通貨が広く使われている。	とうちではにほんのつうかがひろくつかわれている 
\\	私は当地は不案内です。	わたしはとうちはふあんないです 
\\	当地の夏はひどく湿気が多い。	とうちのなつはひどくしっきがおおい 
\\	私は湿気が我慢できない。	わたしはしっきががまんできない 
\\	確かに暑いが、湿気がない。	たしかにあついが、しっきがない 
\\	私は君の保護者のつもりだ。	わたしはきみのほごしゃのつもりだ 
\\	彼女を保護する義務がある。	かのじょをほごするぎむがある 
\\	彼女は警察に保護を求めた。	かのじょはけいさつにほごをもとめた 
\\	鯨は保護されなければ、絶滅するだろう。	くじらはほごされなければ、ぜつめつするだろう 
\\	象は絶滅する危険がある。	ぞうはぜつめつするきけんがある 
\\	彼らの仕事はネズミを絶滅させる事である。	かれらのしごとはねずみをぜつめつさせることである 
\\	彼の新しいヘアスタイルについてみんな何か感想を述べた。	かれのあたらしいヘアスタイルについてみんななにかかんそうをのべた 
\\	ゲームについてのご感想は?	ゲームについてのごかんそうは? 
\\	彼はその詩の感想を述べた。	かれはそのしのかんそうを述べた 
\\	子供には幸福な家庭環境が必要だ。	こどもはこうふくなかていかんきょうがひつようだ 
\\	少年のころ、彼の家庭環境はよかった。	そうねんごろ、かれのかていかんきょうはよかった 
\\	慈悲は家庭に始まる。	じひはかていにはじまる 
\\	これは家庭の収入を増やした。	これはかていのしゅうにゅうをふやした 
\\	彼は裕福な家庭の息子だ。	かれはゆうふくなかていのむすこだ 
\\	僕は仕事より家庭の方が大事だ。	ぼくはしごとよりかていのほうがだいじだ 
\\	この店は家庭用品を備えている。	このみせはかていようひんをそなえている 
\\	日用品の値段が上がった。	にちようひんにねだんがあがった 
\\	因みに私は英語がからきし駄目なんです。	ちなみにわたしのえいごがからきしだめなんです 
\\	初めて天の川を見た夜のことを私は覚えている。	はじめてあまのがわをみたよるのことをわたしはおぼえている 
\\	彼はその場面を見て驚いた。	かれはそのばめんをみておどろいた 
\\	そんな場面を見るに忍びなかった。	そんなばめんをみるにしのびなかった 
\\	スローモーションでその場面を見せた。	スローモーションでそのばめんをみせた 
\\	劇の滑稽な場面はやり過ぎであった。	げきのこっけいのばめんはやりすぎであった 
\\	こっそり家に忍び込む。	こっそりいえにしのびこむ 
\\	狼が羊の群れに忍び寄った。	おおかみがひつじのむれにしのびよった 
\\	私は見知らぬ人が彼の家に忍び込むのを見た。	わたしはみしらぬひとがかれのいえにしのびこむのをみた 
\\	本日のランチの内容は何ですか。	にほんのランチのないようはなんですか 
\\	そのパラグラフは内容を重視しています。	そのパラグラフはないようを 
\\	彼の妹は流行を重視し過ぎる。	かれのいもうとははやりをじゅうししすぎる 
\\	この報告はその事実を重視した。	このほうこくはそのじじつをじゅうしした 
\\	私はその発見を大して重視しない。	わたしはそのはっけんをたいしてじゅうししない 
\\	彼らの事業は拡大している。	かれらのじじつはかくだいしている 
\\	最近共産主義は拡大した。	さいきんきょうさんしゅぎはかくだいした 
\\	拡大コピーを撮ってくるよ。	かくだいコピーをとってくるよ 
\\	若いとき彼はしばらくの間共産主義をもてあそんだ。	わかいときかれはしばらくのあいだきょうさんしゅぎをもてあそんだ 
\\	船は波にもてあそばれていた。	ふねはなみにもてあそばれていた 
\\	値段には消費税を含みます。	ねだんにはしょうひぜいはふくみます 
\\	私も含めて全員、バスに乗った。	わたしもふくめてぜんいん、バスにのった 
\\	費用には朝食も含まれている。	ひようにはちょうしょくもふくまれている 
\\	費用は最低一万円です。	ひようはさいていいちまんえんです 
\\	その費用をお知らせ下さい。	そのひようをおしらせください 
\\	私たちは1日に最低7時間は寝なければならない。	わたしたちはいちにちにさいていしちじかんはねなければならない 
\\	そんな嘘をつくなんて彼は最低だ。	そんなうそをつくなんてかれはさいていだ 
\\	私たちの多くは消費税に反対である。	わたしたちのおおくはしょうひぜいにはんたいだある 
\\	日本では、ほとんどの物やサービスに5%の消費税がかけられる。	にほんではほとんどのものやサービスにごパーセントのしょうひぜいかかけられる 
\\	農業は多量の水を消費する。	のうぎょうはたりょうのみずをしょうひする 
\\	多量の水が残っている。	たりょうのみずがのこっている 
\\	日本は多量の石油を輸入している。	にほんはたりょうのせきゆをゆにゅうしている 
\\	緊急時に備えて多量の食料を蓄えた。	きんきゅうときにそなえてたりょうのしょくりょうをたくわえる 
\\	冬に備えて食料を秋に蓄える動物もいる。	ふゆにそなえてしょくりょうをあきにたくわえるどうぶつもいる 
\\	彼はガスを蓄えている。	かれはガスをたくわえている 
\\	彼らは家に十分食べ物を蓄えている。	かれらはいえにじゅうぶんたべものをたくわえている 
\\	彼は次の試合のために精力を蓄えた。	かれはつぎのしあいのためにせいりょくをたくわえた 
\\	彼女はパーティーに精力を使う。	かのじょはパーティーにせいりょくをつかう 
\\	彼は精力的な政治家だ。	かれはせいりょくてきなせいじかだ 
\\	その精力的な男は様々な活動に加わっている。	そのせいりょくてきなおとこはさまざまなかつどうにくわわっている 
\\	会話に加われば。	かいわにくわわれば 
\\	彼女は子供たちの遊びに加わった。	かのじょはこどもたちのあそびにくわわった 
\\	私はその討論に加わった。	わたしはそのとうろんにくわわった 
\\	彼が私達に加わるのは当然だと思った。	かれはわたしたちにくわわるのはとうぜんだとおもった 
\\	眠いのは当然だ。	ねむいのはとうぜんだ 
\\	人気があるのも当然だ。	にんきがあるのはとうぜんだ 
\\	当然昇進させられるべきだ。	とうぜんしょうしんさせるべきだ 
\\	彼女が怒るのも極めて当然だ。	かのじょがいかるのもきわめてとうぜんだ 
\\	彼は学者でなくて活動家だ。	かれはがくしゃでなくてかつどうかだ 
\\	十分な食料があるか。	じゅうぶんなしょくりょうがあるか 
\\	彼はゆうべ自宅で静かに息を引き取った。	かれはゆうべじたくでしずかにいきをひきとった 
\\	祖母は昨日安らかに息を引き取った。	そぼはきのうやすらかにいきをひきとった 
\\	彼女の心は安らかだ。	かのじょのこころはやすらかだ 
\\	君の保証人になりましょう。	きみのほしょうにんになりましょう 
\\	彼が借金の保証人になってくれた。	かれはしゃっきんのほしょうにんになってくれた 
\\	思想の自由は憲法で保証されている。	しそうのじゆうはけんぽうでほしょうされている 
\\	我々は平和憲法を守らなければならない。	われわれはへいわけんぽうをまもらなければならない 
\\	平等は憲法で保証されている。	びょうどうはけんぽうでほしょうされている 
\\	人間は全て平等である。	にんげんはすべてびょうどうである 
\\	彼はすべて平等であることに賛成した。	かれはすべてびょうどうであることにさんせいした 
\\	女性は男性との機会の平等を要求している。	じょせいはだんせいとのきかいのびょうどうをようきゅうしている 
\\	彼らは人種の平等を目指して運動した。	かれらはじんしゅのびょうどうをめざしてうんどうした 
\\	私は作家を目指している。	わたしはさっかをめざしている 
\\	成功の機会のある北部を目指す。	せいこうのきかいのあるほくぶをめざす 
\\	彼はバスを目指して懸命に走った。	かれはバスをめざしてけんめいにはしった 
\\	経済の安定はすべての政府の目指すところだ。	けいざいのあんていはすべてのせいふのめざすところだ 
\\	領収書をいただけますか。	りょうしゅうしょをいただきますか 
\\	私は彼らが当然領収書をくれるものと思った。	わたしはかれらがとうぜんりょうしゅうしょをくれるものとおもった 
\\	交渉は中止になった。	こうしょうはちゅうしになった 
\\	交渉は新局面に入った。	こうしょうはしんきょくめんにはいった 
\\	交渉は大事な局面を迎えた。	こうしょうはだいじなきょくめんをむかえた 
\\	これで私の悩みが増える。	これでわたしのなやみがふえる 
\\	あなたと私の悩みは似通っている。	あなたとわたしのなやみはにかよっている 
\\	彼らは全ての点でお互いに似通っている。	かれらはすべてのてんでとたがいににかよっている 
\\	現代の世界の文化はいくぶん似通ってきている。	げんだいのせかいのぶんかはいくぶんにかよってきている 
\\	幾分は君に賛成です。	いくぶんはきみにさんせいです 
\\	私は幾分疲れを感じています。	わたしはいくぶんつかれをかんじています 
\\	私は幾分困惑した気持ちだった。	わたしいくぶんこんわくしたきもちだった 
\\	予想外の事態に人々は困惑した。	よそうがいのじたいにひとびとはこんわくした 
\\	ある程度は彼の困惑も理解できる。	あるていどはかれのこんわくもりかいできる 
\\	それは予想通りでした。	それはよそうとおりでした 
\\	予想した以上に悪い天候だ。	よそうしたいじょうにわるいてんこうだ 
\\	ある程度彼を信用できる。	あるていどかれをしんようできる 
\\	彼の意見はある程度正しい。	かれのいけんはあるていどただしい 
\\	市長はある程度妥協するだろう。	しちょうはあるていどだきょうするだろう 
\\	結局我々は妥協した。	けっきょくわれわれはだきょうした 
\\	妥協の可能性はないように思える。	だきょうのかのうせいはないようにおもえる 
\\	彼女は自分自身の服を全部自分で縫う。	かのじょはじぶんじしんのふくをぜんぶじぶんでぬう 
\\	これを手で縫ったんですか。	これはてでぬったんですか 
\\	彼女は上着にボタンを縫い付けた。	かのじょはうわぎにボタンをぬいつけた 
\\	孤独なその患者は縫い物をすることに楽しみを感じている。	こどくなそのかんじゃはぬいものをすることにたのしみをかんじている 
\\	その上手な運転手は車の列を縫うように車を走らせた。	そのじょうずなうんてんしゅはくるまのれつをぬうようにくるまをはしらせた 
\\	彼は訴えようとした。	かれはうったえたようにした 
\\	彼は我々の感情に訴えた。	かれはわれわれのかんじょうにうったえた 
\\	彼は暴力に訴えざるを得なかった。	かれはぼうりょくにうったえざるをえなかった 
\\	カンボジアは国連に援助を訴えた。	カンボジアはこくれんにえんじょをうったえた 
\\	君は朝寝坊に違いない。	きみはあさねぼうにちがいない 
\\	遅刻の理由を言って下さい。	ちこくのりゆうをいってください 
\\	上司は私の遅刻を許してくれた。	じょうしはわたしのちこくをゆるしてくれた 
\\	今出発しなければ遅刻しますよ。	いましゅっぱつしなければちこくしますよ 
\\	私たちは雪合戦をした。	わたしたちはゆきがっせんをした 
\\	みんなが驚いたことに、マイクはスピーチコンテストで一位を取った。	みんながおどろいたことは、マイクはスピチコンテストでいちいをとった 
\\	結果は次の通りでした。1位日本、2位スペイン、3位イタリア。	けっかはつぎのとおりでした。1いにほん、2いスパイン、3いイタリア 
\\	彼は深い眠りに落ちた。	かれはふかいねむりにおちた 
\\	車の騒音が私の眠りの邪魔になった。	くるまのそうおんがわたしのねむりのじゃまになった 
\\	夕食後、私はいつも眠りそうになるんです。	ゆうしょくご、わたしはいつもねむりそうになるんです 
\\	彼女が気を失ったので、私は彼女が倒れないように支えなければならなかった。	かのじょがきをうしなってのて、わたしはかのじょがたおれないようにささえなければならなかった 
\\	今の俺を支えるものは希望だけ。	いまのおれをささえるものはきぼうだけ 
\\	この橋を支えるには重い柱が必要だ。	このはしをささえるにはおもいはしらがひつようだ 
\\	彼女は一家を支えている。	かのじょはいっかをささえている 
\\	湖の氷は彼の重さを支えれなかった。	みずうみのこおりはかれのおもさをささえれなかった 
\\	赤ちゃんは母親に支えられて歩きました。	あかちゃんはははおやにささえられてあるきました 
\\	精神的な支えになってあげられると思う。	せいしんてきなささえになってあげれるとおもう 
\\	君はきっと精神的に疲れている。	きみはきっとせいしんてきにつかれている 
\\	私はそのとき精神的なショックを受けた。	わたしはそのときせいしんてきなショックをうけた 
\\	我々は彼らに精神的な援助を与えよう。	われわれはかれらにせいしんてきなえんじょをあたえよう 
\\	彼らは物質的にも精神的にも彼を支えた。	かれらはぶっしつてきにもせいしんてきにもかれをささえた 
\\	彼は肉体的、精神的苦痛に耐えてきた。	かれはにくたいてき、せいしんてきくつうにたえてきた 
\\	彼の顔は苦痛でいがんだ。	かれのかおはくつうでいがんだ 
\\	苦痛を長引かせる。	くつうをながびかせる 
\\	彼女は苦痛に苦しんでいた。	かのじょはくつうにくるしんでいた 
\\	息子の犯罪行為が彼に苦痛を与えた。	むすこのはんざいこういがかれにくつうをあたえた 
\\	彼は内気で、人前で話す事は苦痛だった。	かれはうちきで、ひとまえではなすことはくつうだった 
\\	食物と飲み物は肉体的に必要なものだ。	たべものとのみものはにくたいてきにひつようなものだ 
\\	スポーツは、肉体的にも精神的にも人を健康にする。	スポーツは、にくたいてきにもせいしんてきにもひとをけんこうにする 
\\	その経済学者は長引く不況を予期していた。	そのけいざいがくしゃはながびくふきょうをよきしていた 
\\	予期しない結果に達した。	よきしないけっかにたっした 
\\	映画は、私が予期したように面白かった。	えいがは、わたしがよきしたようにおもしろかった 
\\	費用は何千ドルにも達するだろう。	ひようはなんぜんドルにもたっするだろう 
\\	彼らはそれで合意に達するだろう。	かれらはそれでごういにたっするだろう 
\\	彼らは値段の点で合意した。	かれらはねだんのてんでごういした 
\\	彼らは敵と合意に達した。	かれらはてきとごういにたっした 
\\	この生地は丈夫です。	このきじはじょうぶです 
\\	この生地で洋服を作ってください。	このきじでようふくをつくってください 
\\	諺は知恵について満ちている。	ことわざはちえについてみちている 
\\	彼は有名な諺をいくつか聖書から引用した。	かれはゆうめいなことわざをいくつかせいしょからいんようした 
\\	聖書を読むのは初めてです。	せいしょをよむのははじめてです 
\\	彼女はいつも聖書を持ち歩いている。	かのじょはいつもせいしょをもちあるいている 
\\	世界で最も読まれている本は聖書である。	せかいでもっともよまれているほんはせいしょである 
\\	ハムレットから1行引用できますか。	ハムレットからいっこういんようできますか 
\\	「自分を愛するよりあなたの隣人を愛せよ」ということばは聖書からの引用です。	"「じぶんをあいするよりあなたのりんじんをあいせよ」ということばはせいしょからのいんようです 
\\	今では多くの家庭が共稼ぎで、夫と妻の両方が働いている。	いまではおおくのかていがともかせぎでおっととつまのりょうほうがはたらいている 
\\	あの旅館は家庭的だ。	あのりょかんはかていてきだ 
\\	彼は芸術的な家庭に育った。	かれはげいじゅつてきなかていでそだった 
\\	家庭用品は三階にあります。	かていようひんはさんかいにあります 
\\	庭園の芸術的な美しさ。	ていえんのげいじゅつてきなうつくしさ 
\\	その料理はいつもこのように芸術的に盛りつけられる。	そのりょうりはいつもこのようにげいじゅつてきにもりつけられる 
\\	庭園の周りに、ぐるりと高い塀が立っている。	ていえんのまわりに、ぐるりにたかいへいがたっている 
\\	塀に登るな。	へいにのぼるな 
\\	彼は塀を乗り越えた。	かれはへいをのりこえた 
\\	彼らは塀の一部を取り壊した。	かれらはへいのいちぶをとりこわした 
\\	その日本庭園には石が見事に配置されている。	そのにほんていえんにはいしがみごとにはいちされている 
\\	私は部屋の家具の配置を変えた。	わたしのへやのかぐのはいちをかえた 
\\	彼女の息子は西ドイツに配置されている。	かのじょのむすこはにしドイツにはいちされている 
\\	大統領のボディーガードは入り口に配置されている。	だいとうりょうをボディーガードはいりぐちにはいちされている 
\\	彼には音楽家の素質がある。	かれはおんがくかのそしつがある 
\\	彼女は生まれつき芸術的な素質を持っている。	かのじょはうまれつきげいじゅつてきなそしつをもっている 
\\	彼女は生まれつき弱い。	かのじょはうまれつきよわい 
\\	彼女は生まれつき内気だった。	かのじょはうまれつきうちきだった 
\\	彼は生まれつきとても無口です。	かれはうまれつきとてもむくちです 
\\	彼は昔無口な男でした。	かれはむかしむくちなおとこでした 
\\	大きくて、鈍くて、無口でした。	おおきくて、のろくて、むくちでした 
\\	彼は感覚が鈍い。	かれはかんかくがのろい 
\\	足の裏は暑さや寒さに対して鈍い。	あしのうらはあつさやさむさにたいしてのろい 
\\	彼は生まれつき口が利けない。	かれはうまれつきくちがきけない 
\\	彼は生まれつき寛大な人だ。	かれはうまれつきかんだいなひとだ 
\\	婦人はその男を寛大に扱った。	ふじんはそのおとこをかんだいにあつかった 
\\	若い人を寛大に考えなさい。	わかいひとをかんだいにかんがえなさい 
\\	寛大すぎることが彼の最大の欠点です。	かんだいすぎることがかれのさいだいのけってんです 
\\	最大限の努力をする。	さいだいげんのどりょくをする 
\\	琵琶湖は日本最大の湖です。	びわこはにほんさいだいのみずうみです 
\\	当面の最大問題は失業である。	とうめんのさいだいもんだいはしつぎょうである 
\\	君の質問は当面の話題とは関係がない。	きみのしつもんはとうめんのわだいとかんけいがない 
\\	現在我々が当面している問題は新しいものではない。	げんざいわれわれがとうめんしているもんだいはあたらしいものではない 
\\	彼は生まれつき冒険家だ。	かれはうまれつきぼうけんかだ 
\\	私は生まれつき楽天家だ。	わたしはうまれつきらくてんかだ 
\\	彼は生まれつき温和な気質だ。	かれはうまれつきおんわなきしつだ 
\\	たとえ困っていても、マックはいつも楽天的だ。	たとえこまっていても、マックはいつもらくてんてきだ 
\\	病人は悲観的になりがちだ。	びょうにんはひかんてきになりがちだ 
\\	悲観的な人生観を捨てよう。	ひかんてきなじんせいかんをすてよう 
\\	彼は非常に悲観的で希望を持っていない。	かあれはひじょうにひかんてきできぼうをもっていない 
\\	人々は非常に人生観が異なる。	ひとびとはひじょうにじんせいかんがことなる 
\\	君の人生観は僕のと違っているね。	きみのじんせいかんはぼくのとちがっているね 
\\	彼らの人生観は奇妙に思われるだろう。	かれらのじんせいかんはきみょうにおもわれるだろう 
\\	彼の人生観は長年の経験に基づいている。	かれのじんせいかんはながねんのけいけんにもとづいている 
\\	メアリーは社交的な気質を持っている。	メアリーはしゃこうてきなきしつをもっている 
\\	アメリカ人気質の二つの特質は寛大さと行動力だ。	アメリカじんきしつのふたつのとくしつはかんだいさとこうどうりょくだ 
\\	子供ではあるけれど、彼女はとても社交的です。	こどもではあるけれども、かのじょはとてもしゃこうてきです 
\\	六月はロンドンでは社交の季節だ。	ろくがつはロンドンではしゃこうのきせつだ 
\\	彼は社交的な性格だ。	かれはしゃこうてきなせいかくだ 
\\	その国は気候が温和だ。	そのくにはきこうがおんわだ 
\\	彼女は温和な婦人です。	かのじょはおんわなふじんです 
\\	日本は海に取り囲まれているので、気候が温和である。	にほんはうみにとりかこまれているので、きこうがおんわである 
\\	彼は一番大事な問題を敬遠したように思えた。	かれはいちばんだいじなもんだいをけいえんしたようにおもえた 
\\	その取引に彼は加えられた。	そのとりひきにかれはくわえられた 
\\	人工的な光は電力という手段によって作られた。	じんこうてきなひかりはでんりょくというしゅだんによってつくられた 
\\	この庭の美しさは自然より人工のおかげだ。	このにわのうつくしさはしぜんよりじんこうのおかげだ 
\\	我々は避難民に毛布を与えた。	われわれはひなんみんにもうふをあたえた 
\\	彼は雨を避ける避難場所を捜した。	かれはあめをさけたひなんばしょをさがした 
\\	船は乗組員が避難できるように救命ボートを備えている。	ふねはのりくみいんがひなんできるようにきゅうめいボートをそなえている 
\\	乗組員は全員救われた。	のりくみいんはぜんいんすくわれた 
\\	救命係はいつでもすぐ人を助ける用意をしている。	きゅうめいかかりはいつでもすぐひとをたすけるよういをしている 
\\	平和しか世界を救うことはできない。	へいわしかせかいをすくうことはできない 
\\	医者は丁度間に合ったので彼女を救うことができた。	いしゃはちょうどまにあったのでかのおをすくうことができた 
\\	何としても彼女を救わねばならない。	なんとしてもかのじょをすくわねばならない 
\\	今、彼女を救えるのは奇跡だけだ。	いま、かのじょをすくえるのはきせきだけだ 
\\	彼が健康を回復したのは奇跡だ。	かれがけんこうをかいふくしたのはきせきだ 
\\	彼は高い木から猫を救った。	かれはたききからねこをすくった 
\\	乗組員の何人かは溺れ死んだ。	のりくみいんのなんにんかはおぼれしんだ 
\\	乗組員はその船を放棄した。	のりくみいんはそのふねをほうきした 
\\	彼はその考えを放棄した。	かれはそのかんがえをほうきした 
\\	多くの障害にも関わらず、彼は放棄しなかった。	おおくのしょうがいにもかかわらず、かれはほうきしなかった 
\\	会社はその計画を放棄した。	かいしゃはそのけいかくをほうきした 
\\	彼女の抗議にも関わらず彼は行った。	かのじょのこうぎにもかかわらずかれはいった 
\\	彼の努力にも関わらず、事故は起こった。	かれのどりょくにもかかわらず、じこはおこった 
\\	雨にも関わらずゴルフをした。	あめにもかかわらずゴルフをした 
\\	失敗にも関わらず彼は幸せを感じている。	しっぱいにもかかわらずかれはしあわせをかんじている 
\\	欠点があるにも関わらず、彼は人気がある。	けってんがあるにもかかわらず、かれはにんきがある 
\\	彼は努力をしたにも関わらず、事業に失敗した。	かれはどりょくをしたにもかかわらず、じぎょうにしっぱいした 
\\	私の命令にも関わらず、彼らはやって来なかった。	わたしのめいれいにもかかわらず、かれらはやってこなかった 
\\	彼女は自分の子供達が溺れるのを、救った。	かのじょはじぶんのこどもたちがおぼれるのを、すくった 
\\	1人の少女が昨日池で溺れ死んだ。	ひとりのしょうじょがきのういけでおぼれしんだ 
\\	溺れかけている男は大声で助けを求めた。	おぼれかけているおとこはおおごえでたすけをもとめた 
\\	溺れかかっていた乗客はすべて救助された。	おぼれかかっていたじょうきゃくはすべてきゅうじょされた 
\\	彼らには臨時救助が必要だ。	かれらはりんじきゅうじょがひつようだ 
\\	彼等は直ちに私たちに救助に来た。	かれらはただちにわたしたちにきゅうじょにきた 
\\	私たちは彼の救助に出かけた。	わたしたちはかれのきゅうじょにでかけた 
\\	私はその会社で臨時の仕事を得た。	わたしはそのかいしゃでりんじのしごとをえた 
\\	彼女は直ちに車の方へ向かった。	かのじょはただちにくるまのほうへむかった 
\\	直ちに大阪へ行ってもらいたい。	ただちにおおさかへいってもらいたい 
\\	乗組員たちは小躍りして喜んだ。	のりくみいんたちはこおどりしてよろこんだ 
\\	彼女は喜んで小躍りをした。	かのじょはよろこんでこおどりをした 
\\	私たちは結婚式と新婚旅行の準備で忙しい。	わたしたちはけっこんしきとしんこんりょこうのじゅんびでいそがしい 
\\	新婚旅行には世界一周の船旅をした。	しんこんりょこうにはせかいいっしゅうのふなたびをした 
\\	船旅に耐えられない人もいる。	ふなたびにたえられないひともいる 
\\	長い船旅は私たちにとって試練であった。	ながいふなたびはわたしたちにとってしれんであった 
\\	私はこの試練に耐えた。	わたしはそのしれんにたえた 
\\	彼は多くの試練に耐えなければならなかった。	かれはおおくのしれんにたえなければならなかった 
\\	理論がそのような試練に耐えて生き残るのは難しい。	りろんがそのようなしれんにたえていきのこるのはむずかしい 
\\	新聞は生き残れるか。	しんぶんはいきのこれるか 
\\	私たちは全くの幸運で生き残った。	わたしたちはまったくのこううんでいきのこった 
\\	98人の乗客のうち3人だけが生き残った。	98にんのじょうきゃくのうち3にんだけがいきのこった 
\\	その物語は口から口へ伝えられて生き残った。	そのもんがたりはくちからくちへつたえられていきのこった 
\\	その墜落事故で生き残った乗客がいたとしても僅かであった。	そのついらくじこでいきのこったじょうきゃくがいたとしてもわずかであった 
\\	その墜落事故で乗客は全員死亡した。	そのついらくじこでじょうきゃくはぜんいんしぼうした 
\\	墜落した飛行機は急に燃え上がった。	ついらくしたひこうきはきゅうにもえあがった 
\\	飛行機は墜落する前に地面すれすれに飛んだ。	ひこうきはついらくするまえにじめんすれすれにとんだ 
\\	僕は終電にすれすれのところで間に合った。	ぼくはしゅうでんにすれすれのところでまにあった 
\\	その鳥は鷲の半分の大きさだった。	そのとりはわしのはんぶんのおおきさだった 
\\	鷲の羽は広げると1メーターにもなる。	わしのはねはひろげると1メーターにもなる 
\\	この布は滑らかな手触りです。	このぬのはなめらかなてざわりです 
\\	車は滑らかに止まった。	くるまはなめらかにとまった 
\\	彼女は滑らかに英語を話す。	かのじょはなめらかにえいごをはなす 
\\	このテーブルの表面は滑らかだ。	このテーブルのひょうめんはなめらかだ 
\\	月の表面は凸凹だ。	つきのひょうめんはでこぼこだ 
\\	バスは凸凹道を走りながらガタガタと音を立てた。	バスはでこぼこみちくぉはしりながらガタガタおとをたてた 
\\	頭上で電車がガタガタ音を立てていた。	ずじょうででんしゃがかたかたおとをたてていた 
\\	ここら辺は道がガタガタしているので運転するのが大変だ。	ここらへんはみちがガタガタしているのでうんてんするのがたいへんだ 
\\	不言実行。ガタガタ言わずにやればいいんだよ。	ふげんじっこう。がたがたいわすにやればいいんだよ 
\\	政府は認めようとしないけれども、今の経済政策はガタガタだ。	せいふはみとめようとしないけれども、いまのけいざいせいさくはガタガタだ 
\\	不言実行が俺のやり方。	ふげんじっこうがおれのやりかた 
\\	我々は政府の経済政策を検討した。	われわれはせいふのけいざいせいさくをけんとうした 
\\	ご検討頂く為に。	ごけんとういただくために 
\\	別の角度から問題を検討する。	べつのかくどからもんだいをけんとうする 
\\	その問題全体を検討しなさい。	そのもんだいぜんたいをけんとうしなさい 
\\	残念ながら君の想像は検討違いだ。	ざんねんながらきみのそうぞうはけんとうちがいだ 
\\	妊娠中ですか。	にんしんちゅうですか 
\\	彼は奥さんが妊娠したといわれてびっくりした。	かれはおくさんがにんしんしたといわれてびっくりした 
\\	彼女は妊娠したので、健康のことを考えてタバコをやめた。	かのじょはにんしんしたので、けんこうのことをかんがえてタバコをやめた 
\\	妊娠の検査をしましょう。	にんしんのけんさをしましょう 
\\	視力検査をします。	しりょくけんさをします 
\\	検査の結果は陰性だった。	けんさのけっかはいんせいだった 
\\	妖精を信じるか。	ようせいをしんじるか 
\\	私は妖精の役演じた。	わたしはようせいのやくえんじた 
\\	彼女は善意に満ちている。	かのじょはぜんいにみちている 
\\	君にはあの人たちの善意が分からないようだ。	きみにはあのひとたちのぜんいがわからないようだ 
\\	彼は2つの方法のどちらを選ぶかを検討した。	かれはふたつのほうほうのどちらをえらぶかをけんとうした 
\\	あらゆる角度から物事を見る。	あらゆるかくどからものごとをみる 
\\	90度の角度は直角と呼ばれている。	90どのかくどはちょっかくとよばれている 
\\	その直線は直角に交わっている。	そのちょくせんはちょっかくにまじわっている 
\\	2本の通りは直角に交差している。	にほんのとおりはちょっかくいこうさしている 
\\	その棒は交差して置かれた	そのぼうはこうさしておかれた 
\\	彼は人に対して公平である。	かれはひとにたいしてこうへいである 
\\	彼はそれは不公平だと言った。	かれはそれはふこうへいだといった 
\\	公平に評価すれば彼は怠惰ではない。	こうへいにひょうかすればかれはたいだではない 
\\	公平に言うと、彼は心は優しい人だ。	こうへいにいうと、かれはこころはやさしいひとだ 
\\	公平に言うと、彼は有能な男だ。	こうへいにいうと、かれはゆうのうなおとこだ 
\\	親は子供達を公平に扱うべきだ。	おやはこどもたちをこうへいにあつかうべきだ 
\\	法が常に公平であるとは限らない。	ほうはつねにこうへいであるとはかぎらない 
\\	彼を公平に評価すれば愚かではない。	かれはこうへいにひょうかすればおろかではない 
\\	彼は確かに愚か者だ。	かれはたしかにおろかものだ 
\\	彼は自分の愚かさを恥じた。	かれはじぶんのおろかさをはじた 
\\	私は愚かにもそれを信じた。	わたしはおろかにもそれをしんじた 
\\	彼はいわば賢い愚か者だ。	かれはいわばかしこいおろかものだ 
\\	そんな愚かなことを言うな。	そんなおろかなことをいうな 
\\	利益は税金抜きですか。	りえきはぜいきんぬきですか 
\\	骨折りなければ利益なし。	ほねおりなければりえきなし 
\\	彼は、自分の利益に敏感だ。	かれは、じぶんのりえきにびんかんだ 
\\	彼は利益の分け前を要求した。	かれはりえきのわけまえをようきゅうした 
\\	我々はその利益を分け合った。	われわれはそのりえきをわけあった 
\\	彼は友達と喜びを分け合うのが好きだ。	かれはともだちとよろこびをわけあうのがすきだ 
\\	彼に分け前を払うのは当然だ。	かれにわけまえをはらうのはとうぜんだ 
\\	私は父の財産の少ないほうの分け前を与えた。	わたしはちちのざいさんのすくないほうのわけまえをあたえた 
\\	その仕事は十分骨折ってするだけの価値がある。	そのしごとはじゅうぶんほねおってするだけのかちがある 
\\	彼は私の為に大変骨折ってくれた。	かれはわたしのためにたいへんほねおってくれた 
\\	頭上の空は濃い水色だった。	ずじょうのそらはこいみずいろだった 
\\	つばめが二羽頭上を飛んでいる。	つばめがにわずじょうをとんでいる 
\\	頭上には月と星が輝いていた。	ずじょうにはつきとほしがかがやいていた 
\\	太陽は頭上でぎらぎら輝いていた。	たいようはずじょうでぎらぎらかがやいていた 
\\	次第に多くのつばめを見ることができる。	しだいにおおくのつばめをみることができる 
\\	彼は次第に出世した。	かれはしだいにしゅっせした 
\\	彼女は次第に回復している。	かのじょはしだいにかいふくしている 
\\	空は次第に暗くなった。	だんだん。
\\	彼の呼吸は次第に弱くなった。	かれのこきゅうはしだいによわくなった 
\\	彼女は深呼吸をした。	かのじょはしんこきゅうをした 
\\	深呼吸をして楽にしなさい。	しんこきゅうをしてらくにしなさい 
\\	彼は上司のいる部屋に入る前に深呼吸をした。	かれはじょうしのいるへやにはいるまえにしんこきゅうをした 
\\	彼は自力で出世した。	かれはじりきでしゅっせした 
\\	出世の足掛かりを掴む。	しゅっせのあしがかりをつかむ 
\\	彼は健康のおかげで出世できたのだ。	かれはけんこうのおかげでしゅっせできたのだ 
\\	彼は大いに努力して出世した。	かれはおおいにどりょくしてしゅっせした 
\\	自力でそうしなさいと父は私に言った。	じりきでそうしなさいとちちはわたしにいった 
\\	地球の表面の70%は水である。	ちきゅうのひょうめんの70%はみずである 
\\	突然の風で池の表面が波立った。	とつぜんのかぜでいけのひょうめんがなみだった 
\\	それらの間には表面的な相違はない。	それらのあいだにはひょうめんなそういはない 
\\	我々には僅かな意見の相違があった。	われわれにはわずかないけんのそういがあった 
\\	両者の間には、あったとしても、相違はごく僅かである。	りょうしゃのあいだには、あったとしても、そういはごくわずかである 
\\	私たちの間には見解の相違があるようです。	わたしたちのあいだにはけんかいのそういがあるようです 
\\	あなたの見解は私とは正反対です。	あなたのけんかいはわたしとはせいはんたいです 
\\	二人の政治家の見解は激しく激突している。	ふたりのせいじかのけんかいははげしくげきとつしている 
\\	この週末には2つの最強のチームが激突する。	このしゅうまつにはふたつのさいきょうのチームがげきとつする 
\\	私個人の見解を述べさせて下さい。	わたしこじんのけんかいをのべさせてください 
\\	転職に関しては人それぞれ見解が分かれる。	てんしょくにかんしてはひとそれぞれけんかいがわかれる 
\\	両親を紹介させて下さい。	りょうしんをしょうかいさせてください 
\\	スペインは世界最強の国の一つだった。	スペインはせかいさいきょうのくにのひとつだった 
\\	彼は私に恨みを抱いた。	かれはわたしにうらみをいだいた 
\\	彼はなんとなく僕に恨みを持っているようだ。	かれはなんとなくぼくにうらみをもっているようだ 
\\	恨みは深いですよ。	うらみはふかいですよ 
\\	彼は父の死の恨みを晴らした。	かれはちちのしのうらみをはらした 
\\	私はその事に対する疑いを晴らす事ができなかった。	わたしはそのことにたいするうたがいをはらすことができなかった 
\\	疑いの余地はない。	うたがいのよちはない 
\\	あなたの良識を疑います。	あなたのりょうしきをうたがいます 
\\	私は疑い深い性格です。	わたしはうたがいぶかいせいかくです 
\\	この部屋からは町の見晴らしがよい。	このへやからはまちのみはらしがよい 
\\	その部屋は湖の見晴らしが素晴らしい。	そのへやはみずうみのみはらしがすばらしい 
\\	私の一番の気晴らしは海岸を散歩することです。	わたしのいちばんのきばらしはかいがんをさんぽすることです 
\\	ピアノを弾くことが彼女のお気に入りの気晴らしです。	ピアノをひくことがかのじょのおきにいりきばらしです 
\\	警察は秩序を保ちます。	けいさつはちつじょをたもちます 
\\	警察は法と秩序の維持に対して責任を持つ。	けいさつはほうとちつじょのいじにたいしてせきにんをもつ 
\\	名声を保つことは難しい。	めいせいをたもつことはむずかしい 
\\	この車の維持は高くつく。	このくるまのいじはたかくつく 
\\	こんな大きな家を維持するのは金がかかる。	こんなおおきいいえをいじするのはかねがかかる 
\\	私は家の伝統を維持していくつもりはなかった。	わたしはいえのでんとうをいじしていくつもりはなかった 
\\	体力を維持するにはちゃんと食べなければいけません。	たいりょくをいじするにはちゃんとたべなければいけない 
\\	親友といえども、その友情を維持する努力が必要である。	しんゆうといえども、そのゆうじょうをいじするどりょくがひつようである 
\\	豊かな社会では、大部分の人々が高い生活水準を維持している。	ゆたかなしゃかいでは、だいぶぶんのひとびとがたかいせいかつすいじゅんをいじしている 
\\	彼の仕事は水準に達した。	かれのしごとはせいじゅんにたっした 
\\	君の仕事は期待している水準に達していない。	きみのしごとはきたいしているすいじゅんにたっしていない 
\\	彼の作品は水準に達していない。	かれのさくひんはすいじゅんにたっしていない 
\\	私たちは豊かな生活水準を当然のことと思っています。	わたしたちはゆたかなせいかつすいじゅんをとうぜんのこととおもっています 
\\	彼は教育の水準の低下についてくどくど喋り続ける。	かれはきょういくのすいじゅんのていかについてくどくどしゃべりつづける 
\\	この国では出生率が急速に低下した。	このくにではしゅっせいりつがきゅうそくにていかした 
\\	最近、出生率は低下し続けている。	さいきん、しゅっせいりつはていかしつづけている 
\\	金利が低下したことが自動車の市場を刺激した。	きんりがていかしたことがじどうしゃのしじょうをしげきした 
\\	ローンの金利は現在高い。	ローンのきんりはげんざいたかい 
\\	金利が低下したので銀行貸し出しが増加している。	きんりがていかしたのでぎんこうかりだしがぞうかしている 
\\	本を貸し出すほかに、図書館は他のいろいろなサービスを提供する。	ほんをかしだすほかに、としょかんはほかのいろいろなサービスをていきょうする 
\\	光が刺激となって花が咲く。	ひかりがしげきとなってはがさく 
\\	市場は盛り返してきた。	しじょうはもりかえしてきた 
\\	市場は輸入品で溢れた。	しじょうはゆにゅうひんであふれた 
\\	母は毎日市場に買い物に行く。	はははまいにちしじょうにかいものにいく 
\\	贅沢品の市場は急速に成長している。	ぜいたくひんのしじょうはきゅうそくにせいちょうしている 
\\	体調を保つために何をしていますか。	たいちょうをたもつためになにをしていますか 
\\	鼻の上でボールのバランスを保つのは難しい。	はなのうえでボールのバランスをたもつのはむずかしい 
\\	彼と接触を保ってはいけない。	かれとせっしょくをたもってはいけない 
\\	日本は米国と友好関係を保っている。	にほんはべいこくとゆうこうかんけいをたもっている 
\\	米国にノウと言えるのか?	"べいこくにノウといえるのか? 
\\	米国では酒類に税金をかける。	べいこくではしゅるいにぜいきんをかける 
\\	彼女は米国史を専攻するだろう。	かのじょはべいこくしをせんこうするだろう 
\\	私は大学で化学を専攻した。	わたしはだいがくでかがくをせんこうした 
\\	彼が専攻している学問は経済学である。	かれはせんこうしているがくもんはけいざいがくである 
\\	彼の考えは学問的過ぎる。	かれのかんがえはがくもんてきすぎる 
\\	少しばかりの学問は危険なもの。	すこしいばかりのがくもんはきけんなもの 
\\	父は手術をしてから体調が良い。	ちちはしゅじゅつをしてからたいちょうがよい 
\\	体調を回復していれば、彼は来るだろう。	たいちょうをかいふくしていれば、かれはくるだろう 
\\	運動選手はよいコンディションを保たねばならない。	うんどうせんしゅはよいコンディションをたもたねばならない 
\\	平静を保てないと君は地位を失うことになるよ。	へいせいをたもてないときみはちいをうしなうことになるよ 
\\	彼はやがて平静に戻った。	かれはやがてへいせいにもどった 
\\	彼は徐々に平静を取り戻した。	かれはじょじょにへいせいをとりもどった 
\\	その怪我人はもう平静になった。	そのけがにんはもうへいせいになった 
\\	運動が筋肉を鍛える。	うんどうがきんにくをきたえる 
\\	彼が私の運命を握っている。	かれはわたしのうんめいをにぎっている 
\\	テニスのラケットを固く握った。	テニスのラケットをかたくにぎった 
\\	我が家では女房が財布の紐を握っている。	わがやではにょうぼうがさいふのひもをにぎっている 
\\	女房が突然泣き出した。	にょうぼうがとつぜんなきだした 
\\	彼は女房の尻に敷かれている。	かれはにょうぼうのしりにしかれている 
\\	彼女は夫を尻に敷いている。	かのじょはおっとをしりにしいている 
\\	トムってしっかり奥さんの尻に敷かれているのね。	トムってしっかりおくさんのしりにしかれているのね 
\\	彼は尻が重い。	かれはしりがおもい 
\\	スーはお尻が大きいが、気にしていない。	スーはおしりがおおきいが、きにしていない 
\\	尻尾のない猫もいる。	しっぽのないねこもいる 
\\	少年はねずみの尻尾をつかんでいた。	しょうねんはねずみのしっぽをつかんでいた 
\\	床にカーペットを敷いた。	ゆかにカーペットをしいた 
\\	母はベッドに綺麗なシーツを敷いた。	はははベッドにきれいなシーツをしいた 
\\	彼は医学に専念した。	かのじょはいがくにせんねんした 
\\	彼女は訴えるような目で私を見た。	かのじょはうったえるようなめでわたしをみた 
\\	彼女は脱税を嗅ぎ出すのが得意である。	かのじょはだつぜいをかぎだすのがとくいである 
\\	この犬は麻薬を嗅ぎ出すよう訓練されている。	このいぬはまやくをかぎだすようくんれんされている 
\\	彼に厳しい訓練を与えなさい。	かれはきびしいくんれんをあたえなさい 
\\	才能は訓練を必要とする。	さいのうはくんれんをひつようとする 
\\	彼女は歌手として訓練を受けた。	かのじょはかしゅとしてくんれんをうけた 
\\	わたしはその学校で訓練された。	わたしはそのがっこうでくんれんされた 
\\	彼は競馬のために馬を訓練している。	かれはけいばのためにうまをくんれんしている 
\\	彼は双眼鏡で競馬を見た。	かれはそうがんきょうでけいばをみた 
\\	まだ肉眼じゃ無理だよ。双眼鏡だと、ちょびっとだけ見えるかも・・・。	まだにくがんじゃむだだよ。そうがんきょうだと、ちょびっとだけみえるかも・・・。 
\\	肉眼ではほとんど見えない星もある。	にくがんではほとんどみえないほしもある 
\\	バクテリアは肉眼では見えない。	バクテリアはにくがんではみえない 
\\	彼は病気の筈がない。	かれはびょうきはずがない 
\\	彼はすぐ戻る筈です。	かれはすぐもどるはずです 
\\	試合に負けた筈がない。	しあいにまけたはずがない 
\\	彼が正直ものの筈がない。	かれはしょうじきもののはずがない 
\\	その話が本当の筈がない。	そのはなしがほんとうのはずがない 
\\	彼の演説は抽象的なので私には理解できない。	かれのえんぜつはちゅうしょうてきのでりかいできない 
\\	両者の間には格段の違いがある。	りょうしゃのあいだにはかくだんのちがいがある 
\\	彼は周知の人々より格段に優れているので、すぐには理解されないのである。	かれはしゅうちのひとびとよりかくだんにすぐれているので、すぐにはりかいされないのである 
\\	書き写したものを原文と比較せよ。	かきうつしたものをげんぶんとひかくせよ 
\\	神戸は比較的物価が安い。	こうべはひかくてきぶっかがやすい 
\\	彼女は、比較的早口だ。	かのじょは、ひかくてきははやぐちだ 
\\	日本車の価格は、比較的高い。	にほんのかかくは、ひかくてきたかい 
\\	自分の子を他人の子と比較するな。	じぶんのこをたにんのことひかくするな 
\\	我々は二つの意見を比較検討した。	われわれはふたつのいけんをひかくけんとうした 
\\	彼らはその変化に素早く順応した。	かれらはそのへんかにすばやくじゅんのうした 
\\	私達は喫茶店で素早い昼食を食べた。	わたしはきっさてんですばやいちゅうしょくをたべた 
\\	ご無沙汰しました。	ごぶさたしました 
\\	2年間のご無沙汰でした。	にねんかんのごぶさたでした 
\\	もし核戦争になったら、われわれの子孫はどうなるのでしょう。	もしかくせんそうになったら、われわれのしそんはどうなるのでしょう 
\\	彼らは核戦争の心配をしている。	かれらはかくせんそうのしんぱいをしている 
\\	問題は、いかに核戦争を避けるかである。	もんだいはいかにかくせんそうをさけるかである 
\\	文明は今や核戦争に脅かされている。	ぶんめいはいまやかくせんそうにおどかされている 
\\	核戦争は地球上の生命を終わらせるだろう。	かくせんそうはちきゅうじゅうのせいめいをおわらせるだろう 
\\	私は喧嘩を終わらせたい。	わたしはけんかをおわらせたい 
\\	彼が戻るまでに必ず終わらせます。	かれはもどるまでにかならずおわらせます 
\\	彼は昨夜休憩しないで働き続けた。	かれはさくやきゅうけいしないではたらきつづけた 
\\	休憩は短いから十分に活用しなさい。	きゅうけいはみじかいからじゅうぶんにかつようしなさい 
\\	時間は最大限に活用すべきだ。	じかんはさいだいげんにかつようすべきだ 
\\	せいぜい自分の能力を活用しなさい。	せいぜいじぶんののうりょくをかつようしなさい 
\\	機会は常に最大限に活用すべきだ。	きかいはつねにさいだいげんにかつようすべきだ 
\\	私たちは私たちのお金を有効に活用したい。	わたしたちはわたしたちのおかねをゆうこうにかつようしたい 
\\	切符は三日間有効だ。	きっぷはみっかかんゆうこうだ 
\\	その法律はまだ有効である。	そのほうりつはまだゆうこうである 
\\	あのとき私がした約束は有効だ。	あのときわたしがしたやくそくはゆうこうだ 
\\	私は、日本での時間を有効に使った。	わたしは、にほんのじかんをゆうこうにつかった 
\\	このパスポートは5年間有効です。	このパスポートはごねんかんゆうこうです 
\\	それは細菌感染に有効だ。	それはさいきんかんせんにゆうこうだ 
\\	細菌が病気を引き起こすことを知っていた。	さいきんがびょうきをひきおこすことをしっていた 
\\	細菌は顕微鏡の力を借りて初めて見られる。	さいきんはけんびきょうのちからをかりてはじめてみられる 
\\	これらの感染病はミルクの汚染が原因だった。	これらのかんせんびょうはミルクのおせんがげんいんだった 
\\	感染症の病気ですか。	かんせんしょうのびょうきですか 
\\	耳の感染症によくかかります。	みみのかんせんしょうによくかかります 
\\	オーストラリアへの旅で初めて赤道を超えた。	オーストラリアへのたびではじめてせきどうをこえた 
\\	そのような時逃げるしか考えていない日本国民は情けない。	そのようなときにげるしかかんがえていないにほんこくみんはなさけない 
\\	高校生の頃は夜更かしをした物でした。	こうこうせいのごろよふかしをしたものでした 
\\	彼と同様に君も勤勉だ。	かれとどうようにきみもきんべんだ 
\\	君と同様に私も責任がある。	きみとどうようにわたしもせきにんがある 
\\	国会は午後2時に開催された。	こっかいはごごにじにかいさいされた 
\\	その展覧会は今開催中です。	そのてんらんかいはいまかいさいちゅうです 
\\	その会議は毎年開催される。	このかいぎはまいとしかいさいされた 
\\	その都市はフェアを開催している。	そのとしはフェアをかいさいしている 
\\	結婚式は天候にかかわらず催されるだろう。	けっこんしきは天候Zにかかわらずもよおされるだろう 
\\	国会は解散された。	こっかいはかいさんされた 
\\	彼は国会議員に選出された。	かれはこっかいぎいんにせんしゅつされた 
\\	私たちは彼を市長に選出しました。	わたしたちはかれをしちょうにせんしゅつしました 
\\	マイクは議長に選出された。	マイクはぎちょうにせんしゅつされた 
\\	警察は、群衆を解散させた。	けいさつは、ぐんしゅうをかいさんされた 
\\	私たちが着く頃には、その会は解散しているだろう。	わたしたちがつくごろには、そのかいはかいさんしているだろう 
\\	その書類に関しては私が保管しています。	そのしょるいにかんしてはわたしがほかんしている 
\\	ジムはすぐに入院させなければならない。	ジムはすぐににゅういんさせなければならない 
\\	彼女は息子をそのクラブから脱退させなければならなかった。	かのじょはむすこをそもクラブからだったいさせなければならなかった 
\\	君は直ちにクラブを脱退したほうがよい。	きみはただちにクラブをだったいしたほうがよい 
\\	彼は無謀運転にスリルを感じる。	かれはむぼううんてんにスリルをかんじる 
\\	それで余計に彼は不幸になった。	それでよけいにかれはふこうになった 
\\	手書きで書かれていたので、その手紙はあまり読みやすくなかった。	てがきでかかれていたので、そのてがみはあまりよみやすくなかった 
\\	私は、議論が終わるのを辛抱強く待つことに決め込んだ。	わたしは、ぎろんがおわるのをしんぼうつよくもつことにきめこんだ 
\\	教授は研究休暇で日本にいる。	きょうじゅはけんきゅうきゅうかでにほんにいる 
\\	その教授は現代文学に詳しい。	そのきょうじゅはげんだいぶんがくにくわしい 
\\	彼はハーバードの生物学教授だ。	かれはハーバードのせいぶつがくきょうじゅだ 
\\	教授はにっこりと微笑みました。	きょうじゅはにっこりとほほえみました 
\\	まもなく彼は教授に任命された。	まもなくかれはきょうじゅににんめいされた 
\\	彼は議長に任命された。	かれはぎちょうににんめいされた 
\\	彼が任命される見込みはない。	かれがにんめいされるみこみはない 
\\	彼は責任ある地位に任命された。	かれはせきにんあるちいににんめいされた 
\\	今、有給休暇中だ。	いま、ゆうきゅうきゅうかちゅうだ 
\\	休暇も終わりに近付いた。	きゅうかもおわりにちかづいた 
\\	彼女は有給休暇を利用して、スキーに行った。	かのじょはゆうきゅうきゅうかをりようして、スキーに行った 
\\	昨日、有給休暇を取りました。	きのう、ゆうきゅうきゅうかをとりました 
\\	彼は大学時代の教授に励まされた。	かれはだいがくじだいのきょうじゅにはげまされた 
\\	彼女は彼に小説を書くように励ました。	かのじょはかれにしょうせつをかくようにはげました 
\\	悲しい時は友達が励ましてくれる。	かなしいときはともだちがはげましてくれる 
\\	あなたの激励の言葉に励まされました。	あなたのげきれいのことばにはげまされました 
\\	彼は息子にもっと勉強するようにと励ました。	かれはむすこにもっとべんきょうするようにとはげました 
\\	教授は私の研究を励ましてくれた。	きょうじゅはわたしのけんきゅうをはげましてくれた 
\\	何か大きなことをしろと彼は息子を激励した。	なにかおおきいなことをしろとかれはむすこをげきれいした 
\\	あなたの激励がなかったら、私はその計画を諦めたでしょう。	あなたのげきれいがなかったら、わたしはそのけいかくをあきらめたでしょう 
\\	彼の演説を録音しておかねばならない。	かれのえんぜつをろくおんしておかねばならない 
\\	その放送をテープに録音してくれ。	そのほうそうをテープにろくおんしてくれ 
\\	彼の死は世界中に放送された。	かれのしはせかいじゅうにほうそうされた 
\\	この条件では拒絶に等しい。	このじょうけんではきょぜつにひとしい 
\\	彼の要求を拒絶する。	かれのようきゅうをきょぜつする 
\\	彼は、私の提案を拒絶した。	かれは、わたしのていあんをきょぜつした 
\\	彼女は私の申し出を拒絶した。	かのじょはわたしのもうしでをきょぜつした 
\\	議長は彼のばかげた提案を拒絶した。	ぎちょうはかれのばかげたていあんをきょぜつした 
\\	無条件でその計画に同意した。	むじょうけんでそのけいかくにどういした 
\\	健康が幸福の第1条件です。	けんこうがこうふくのだいいちじょうけんです 
\\	彼は説いて同意させた。	かれはといてどういさせた 
\\	私はこの点で君に同意する。	わたしはこのてんできみにどういする 
\\	彼は同意を表す為に微笑んだ。	かれはどういをあらわすためにほほえんだ 
\\	彼女は条件が不公平だと言い張る。	かのじょはじょうけんがふこうへいだといいはる 
\\	そちらの条件を受け入れましょう。	そちらのじょうけんをうけいれましょう 
\\	国籍に関係なく誰でも受け入れる。	こくせきにかんけいなくだれでもうけいれる 
\\	彼には国籍が不利に働いた。	かれにはこくせきがふりにはたらいていた 
\\	国籍に関係なくすべての人に資格がある。	こくせきにかんけいなくすべてのひとしかくがある 
\\	彼は昇進の資格がある。	かれはしょうしんのしかくがある 
\\	彼は英語教師の資格がある。	かれはえいごきょうしにしかくがある 
\\	彼はその報酬を受ける資格がある。	かれはそのほうしゅうをうけるしかくがある 
\\	ジムは君の申し出を受け入れるでしょう。	ジムはきみのもうしでをうけいれるでしょう 
\\	君の計画を受け入れよう。	きみのけいかくをうけいれよう 
\\	彼は提案を受け入れて賢明だった。	かれはていあんをうけいれてけんめいだった 
\\	竜巻で村全体が破壊された。	たつまきでむらぜんたいがはかいされた 
\\	視界の存在を全て破壊しろ。	しかいのそんざいをすべてはかいしろ 
\\	村に竜巻が起こった。	むらにたつまきがおこった 
\\	その村は洪水で孤立した。	そのむらはこうずいでこりつした 
\\	二台の車は道路で正面衝突するところだった。	にだいのくるまはどうろでしょうめんしょうとつするところだった 
\\	生理は規則正しくあります。	せいりはきそくただしくあります 
\\	生理が5週間遅れています。	せいりがごしゅうかんおくれています 
\\	被告は無罪になった。	ひこくはむざいになった 
\\	判決は被告に有利だった。	はんけつはひこくにゆうりだった 
\\	被告は無実を主張した。	ひこくはむざいをしゅちょうした 
\\	被告人は死刑を宣告された。	ひこくにんはしけいをせんこくされた 
\\	彼は殺人罪を宣告された。	かれはさつじんざいをせんこくされた 
\\	被告は懲役10年の刑を宣告された。	ひこくはちょうえきじゅうねんのけいをせんこくされた 
\\	彼は有罪と宣告された。	かれはゆうざいとせんこくされた 
\\	懲役3年の判決を受けた。	ちょうえきさんねんのはんけつをうけた 
\\	判決は明日下される。	はんけつはあしたくだされる 
\\	彼は殺人罪の評決を下された。	かれはさつじんざいのひょうけつをくだされた 
\\	評決は公平な審議の証拠である。	ひょうけつはこうへいなしんぎのしょうこである 
\\	その問題は審議中です。	そのもんだいはしんぎちゅうです 
\\	事件を審議したのはどの裁判官ですか。	じけんをしんぎしたのはどのさいばんかんですか 
\\	彼女は自分の分析が正しいと言い張る。	かのじょはじぶんのぶんせきがただしいといいはる 
\\	彼女は私の誤りだと言い張った。	かのじょはわたしのあやまりだといいはった 
\\	彼らは地球は丸いと言い張った。	かれらはちきゅうはまるいといいはった 
\\	彼女は、私が勘定を支払うように言い張った。	かのじょは、わたしがかんじょうをしはらうようにいいはった 
\\	母は暗くなってから私が外出してはいけないと言い張る。	はははくらくなってからわたしががいしゅつしてはいけないといいはる 
\\	彼は私に食事の勘定を払わせなかった。	かれはわたしにしょくじのかんじょうをはらわせなかった 
\\	彼女は私が勘定を払うべきだと主張した。	かのじょはわたしがかんじょうをはらうべきだとしゅちょうした 
\\	万事は我々に有利だ。	ばんじはわれわれにゆうりだ 
\\	その証拠は彼に有利であった。	そのしょうこはかれにゆうりであった 
\\	都市生活にはいくつかの有利な点がある。	としせいかつにはいくつかのゆうりなてんがある 
\\	今日英語が堪能であることは有利な技能である。	きょうえいごがたんのうであることはゆうりなぎのうである 
\\	学生は読書の技能を磨くべきだ。	がくせいはどくしょのぎのうをみがくべきだ 
\\	英語を上手に話す技能がその地位を志望する者に要求される。	えいごをじょうずにはなすぎのうがそのちいをしぼうするものにようきゅうされる 
\\	あの大学は私の第1志望だった。	あのだいがくはわたしのだいいちしぼうだった 
\\	志望者全員が試験に合格できるわけではない。	しぼうしゃぜんいんがしけんにごうかくできるわけではない 
\\	全く悪いというわけではない。	まったくわるいというわけではない 
\\	全ての鳥が飛べるわけではない。	まったくのとりがとべるわけではない 
\\	両親とも生きているわけではない。	りょうしんともいきているわけではない 
\\	彼が英語に堪能ならば、私は彼を雇います。	かれはえいごにたんのうならば、わたしはかれをやといます 
\\	彼女はフランス語、スペイン語共に堪能だ。	かのじょはフランスご、スペインごともにたんのうだ 
\\	彼らは私をその会社に雇うと言った。	かれらはわたしをそのかいしゃにやとうといった 
\\	その会社は2人の新しい秘書を雇うことにした。	そのかいしゃはふたりのあたらしいひしょをやとうことにした 
\\	彼は10人の労働者を雇った。	かれはじゅうにんのろうどうしゃをやとった 
\\	あの店は8人の店員を雇っている。	あおみせははちにんのてんいんをやとっている 
\\	翌年、第一次世界大戦が始まりました。	よくねん、だいいちじせかいたいせんがはじまりました 
\\	彼は兵役を免除されている。	かれはへいえきをめんじょされている 
\\	しかし人々は政府に対して5年間の兵役を努めなければならなかった。	しかしひとびとはせいふにたいしてごねんかんのへいえきをつとめなければならなかった 
\\	彼は残りの仕事を免除された。	かれはのこりのしごとをめんじょされた 
\\	あの人の家は地下鉄の最寄りにある。	あのひとのいえはちかてつのもよりにある 
\\	最寄りの駅から遠く離れた所に住んでいる。	もよりのえきからとおくはなれたところにすんでいる 
\\	ケン、すりには警戒しろよ。	ケン、すりにはけいかいしろよ 
\\	今日は予定が詰まっている。	きょうはよていがつまっている 
\\	鼻が詰まった。	はながつまった 
\\	彼は商売に行き詰まった。	かれはしょうばいにいきつまった 
\\	放してくれ、息が詰まる。	はなしてくれ、いきがつまる 
\\	彼女の手を放すと彼の声は真剣になった。	かのじょのてをはなすとかれのこえはしんけんになった 
\\	私達は真剣に話し合った。	わたしたちはしんけんにはなしあった 
\\	彼女はいつも真剣な表情をしている。	かのじょはいつもしんけんなひょうじょうをしている 
\\	彼らは彼の言葉に真剣な注意を払った。	かれらはかれのことばにしんけんなちゅういをはらった 
\\	試験に受かるように真剣に勉強した。	しけんにうかるようにしんけんにべんきょうした 
\\	彼は物事を真剣に考え過ぎる傾向にある。	かれはものごとをしんけんにかんがえすぎるけいこうにある 
\\	この計画の資本金が用意された。	このけいかくのしほんきんがよういされた 
\\	外国資本家は現地の政情不安が理由で手を引きました。	がいこくしほんかはげんちのせいじょうふあんがりゆうでてをひきました 
\\	手を引いた方がいいよ。	てをひいたほうがいいよ 
\\	母は私が中東に行くことに反対した。	はははちゅうとうにいくことにはんたいした 
\\	教授は中東問題について講義をした。	きょうじゅはちゅうとうもんだいについてこうぎをした 
\\	中東からの石油の供給は混乱するかもしれない。	ちゅうとうからのせきゆのきょうきゅうはこんらんするかもしれない 
\\	混乱を求めて悲鳴を上げろ。	こんらんをもとめてひめいをあげた 
\\	彼の発言は混乱を引き起こした。	かれのはつげんはこんらんをひきおこした 
\\	火事で劇場の中は大混乱になった。	かじでげきじょうのなかだいこんらんになった 
\\	国会は混乱のうちに散会した。	こっかいはこんらんのうちにさんかいした 
\\	私の上司は辞職せざる得なかった。	わたしのじょうしはじしょくせざるえなかった 
\\	僕はすぐ辞職しようかと思っている。	ぼくはすぐじしょくしようかとおもっている 
\\	彼の失敗は辞職という結果になった。	かれのしっぱいはじしょくというけっかになった 
\\	彼女は辞職しようと固く決心していた。	かのじょはじしょくしようとかたくけっしんしていた 
\\	彼は毎日、日記をつける決心をした。	かれはまいにち、にっきをつけるけっしんをした 
\\	彼は出発を延期することを決心した。	かれはしゅっぱつをえんきすることをけっしんした 
\\	決定は延期された。	けっていはえんきされた 
\\	決定は君次第である。	けっていはきみしだいである 
\\	彼は我々の決定に圧力をかけた。	かれはわれわれのけっていにあつりょくをかけた 
\\	会合は延期になるだろう。	かいごうはえんきになるだろう 
\\	会合は8時に解散した。	かいごうははちじにかいさんした 
\\	会議は延期されると発表された。	かいぎはえんきされるとはっぴょうされた 
\\	それは必然的に延期しなければならない。	それはひつぜんてきにえんきしなければならない 
\\	特別講義は悪天候のため翌日に延期された。	とくべつこうぎはあくてんこうのためよくじつにえんきされた 
\\	文明が進むにつれて、詩は殆ど必然的に衰える。	ぶんめいがすすむにつれて、しはほとんどひつぜんてきにおとろえる 
\\	年をとるにつれて記憶力は段々衰える。	としをとるにつれてきおくりょくはだんだんおとろえる 
\\	彼は健康が衰えた。	かれはけんこうがおとろえた 
\\	視力が衰え始めた。	しりょくがおとろえはじめた 
\\	彼の人気は衰えていた。	かれのにんきはおとろえていた 
\\	彼の体力は徐々に衰えた。	かれのたいりょくはじょじょにおとろえた 
\\	彼の影響力は未だ衰えていない。	かれのえいきょうりょくはまだおとろえていない 
\\	貿易の活動は最近衰えてきている。	ぼうえきのかつどうはさいきんおとろえてきている 
\\	彼は決して政治家の圧力に屈しないだろう。	かれはけっしてせいじかのあつりょくにくっしないだろう 
\\	税制改革はの圧力が高まっている。	ぜいせいかいかくはのあつりょくがたかまっている 
\\	冷戦はソビエトの崩壊と共に終わった。	れいせんはソビエトのほうかいとともにおわった 
\\	彼の到着の知らせで我々の興奮は高まった。	かれのとうちゃくのしらせでわれわれのこうふんはたかまった 
\\	そのことに対する人々の関心が高まってきている。	そのことにたいするひとびとのかんしんがたかまってきている 
\\	彼の新しい本によって彼の名声が高まった。	かれのあたらしいほんによってかれのめいせいがたかまった 
\\	今後の税制改革では銀行業界に何の影響も与えないだろう。	こんごのぜいせいかいかくではぎんこうぎょうかいになんのえいきょうもあたえないだろう 
\\	彼は大学教育の改革を主張している。	かれはだいがくきょういくのかいかくをしゅちょうしている 
\\	しげみちゃんを保育園に迎えに行ってもらえる?	しげみちゃんをほいくえんにむかえにいってもらえる? 
\\	一番下の子は毎日午前中を保育園で過ごしました。	いちばんしたのこはまいにちごぜんちゅうをほいくえんですごしました 
\\	幼稚園児たちは手をつないで公園の中を歩いていた。	ようちえんじたちはてをつないでこうえんのなかをあるいていた 
\\	しげみちゃんを幼稚園に迎えに行けないの。	しげみちゃんをようちえんにむかえにいけないの 
\\	今日は野球をやる気がしない。	きょうはやきゅうをやるきがしない 
\\	ソニーは従業員のやる気を引き出していますよ。	ソニーはじゅぎょういんのやるきをひきだしていますよ 
\\	教師達は生徒にやる気を起こさせるように努力している。	きょうしたちはせいとにやるきをおこさせるようにどりょくしている 
\\	やる気があれば、不可能なことはない。	やるきがあれば、ふかのうなことはない 
\\	私は首尾よく盗まれた財布を取り戻した。	わたしはしゅびよくぬすまれたさいふをとりもどした 
\\	彼の平静さは本物というよりは見かけだけのものだ。	かれのへいせいさはほんものというよりはみかけだけのものだ 
\\	大きな危険に直面しても彼は平静さを失わなかった。	おおきいなきけんにちょくめんしてもかれはへいせいさをうしなわなかった 
\\	彼こそ本物の紳士だ。	かれこそほんもののしんしだ 
\\	彼こそ本当の天才だ。	かれこそほんとうのてんさいだ 
\\	彼女の肖像画は本物そっくりだ。	かのじょのしょうぞうがはほんものそっくりだ 
\\	私は彼の述べたことをそっくりそのまま返した。	わたしはかれののべたことをそっくりそのままかえした 
\\	魔女は哀れな少女を呪った。	まじょはあわれなしょうじょをのろった 
\\	最初に戦争を思いついた者に呪いあれ。	さいしょにせんそうをおもいつぃたものにのろいあれ 
\\	その家族には呪いがかけられているようだ。	そのかぞくにはのろいがかけられているようだ 
\\	悪い魔女は呪文をかけてその男を虫に変えてしまった。	わるいまじょはじゅもんをかけてそのおとこをむしにかえてしまった 
\\	呪文が解けて豚は人間になった。	じゅもんがとげてぶたはにんげんになった 
\\	折角弁護士の資格があるのにもったいない。	せっかくべんごしのしかくがあるのにもったいない 
\\	その哀れな子供に彼女は心を痛めた。	そのあわれなこどもにかのじょはこころをいためた 
\\	心を痛めないでこの写真を見ることはできない。	こころをいためないでこのしゃしんをみることはできない 
\\	休暇はあっという間に終わった。	きゅうかはあっというまにおわった 
\\	消防士はあっという間に火を消した。	しょうぼうしはあっというまにひをけした 
\\	彼はあっという間に寝入ってしまった。	かれはあっというまにねいってしまった 
\\	電子レンジはあっという間に食べ物を温める。	でんしレンジはあっというまにたべものをあたためる 
\\	我々は太陽を巡るすべての惑星を探検するだろう。	われわれはたいようをめぐるすべてのわくせいをたんけんするだろう 
\\	春が再び巡ってくることを楽しみにしています。	はるがふたたびめぐってくることをたのしみにしています 
\\	この夏九州巡りをするつもりだ。	このなつきゅうしゅうめぐりをするつもりだ 
\\	彼は美術館巡りで疲れていた。	かれはびじゅつかんめぐりでつかれていた 
\\	不思議な巡り合わせで私達はばったり再会した。	ふしぎなめぐりあわせでわたしたちはばったりさいかいした 
\\	昔の恋人に再会してみたい。	むかしのこいびとにさいかいしてみたい 
\\	私は君との再会を待ち望んでいる。	わたしはきみとのさいかいをまちのぞんでいる 
\\	彼は結婚して親戚になった。	かれはけっこんしてしんせきになった 
\\	あの金持ち一家と親戚だと彼は言っている。	あのかねもちいっかとしんせきだとかれはいっている 
\\	若者は仲間や親戚の人たちに別れを告げた。	わかものはなかまやしんせきのひとたちにわかれをつげた 
\\	最もよいのは真実を告げることです。	もっともよいのはしんじつをつげることです 
\\	私は自分の愛を彼女に告げる事に決めた。	わたしはじぶんのあいをかのじょにつげることにきめた 
\\	彼は勿論私の事を告げ口するだろう。	かれはもちろんわたしのことをつげぐちするだろう 
\\	彼は警察に偽りの名前と住所を告げた。	かれはけいさつにいつわりのなまえとじゅしょをつげた 
\\	議長は委員会の始まりを告げた。	ぎちょうはいいんかいのはじまりをつげた 
\\	彼は委員会の委員長だ。	かれはいいんかいのいいんちょうだ 
\\	委員会が直ちに召集された。	いいんかいがただちにしょうしゅうされた 
\\	委員会は2週間延期になった。	いいんかいはにしゅうかんえんきになった 
\\	マクベスは敵を襲撃するために軍隊を召集した。	マクベスはてきをしゅうげきするためにぐんたいをしょうしゅうした 
\\	彼は1942年8月に召集された。	かれはせんきゅうひゃくよんじゅうにねんはちがつにしょうしゅうされた 
\\	我々はその襲撃に対して準備ができていなかった。	われわれはそのしゅうげきにたいしてじゅんびができていなかった 
\\	日本人は1941年12月7日パールハーバーを襲撃した。	にほんじんはせんきゅうひゃくよんじゅういちねんじゅうにがつなのかパールハーバーをしゅうげきした 
\\	軍隊は行進して過ぎ去った。	ぐんたいはこうしんしてすぎさった 
\\	私の一生の大半は過ぎ去った。	わたしのいっしょうのたいはんはすぎさった 
\\	歳月はいつの間にか過ぎ去っていった。	さいげつはいつのまにかすぎさっていった 
\\	何時の間にか暗くなった。	いつのまにかくらくなった 
\\	昔はペチャパイだったのに、いつの間にかこんなに大きくなりやがって。	むかしはぺちゃぱいだったのに、いつのまにかこんあおおきくなりやかって 
\\	この仕事の遂行は多くの歳月を要した。	このしごとのすいこうはおおくのさいげつをようした 
\\	クーデターは慎重に遂行された。	クーデターはしんちょうにすいこうされた 
\\	彼は慎重に義務を遂行した。	かれはしんちょうにぎむをすいこうした 
\\	私は能力の限り職務を遂行します。	わたしはのうりょくのかぎりしょくむをすいこうします 
\\	彼は職務怠慢だった。	かれはしょくむたいまんだった 
\\	彼は職務を与えられた。	かれはしょくむをあたえられた 
\\	怠け者のその男はしばしば自分の職務を怠る。	なまけもののそのおとこはしばしばじぶんのしょくむをおこたる 
\\	警察官の重要な職務の1つは泥棒を捕らえることである。	けいさつかんのじゅうようなしょくむのひとつはどろぼうをとらえることである 
\\	私たちは資金不足のため、計画を遂行することができなかった。	わたしたちはしきんぶそくのために、けいかくをすいこうすることができなかった 
\\	求人にふさわしい職務遂行能力があるか?	きゅうじんに相応しいしょくむすいこうのうりょくがあるか? 
\\	そのような行為は紳士に相応しくない。	そのようなこういはしんしにふさわしくない 
\\	退職後は田舎でのんびり暮したい。	たいしょくごでいなかでのんびりくらしたい 
\\	まる1週間働いたので日曜日はのんびりしました。	まるいっしゅうかんはたらいたのでにちようびはのんびりしました 
\\	君は何でもかんでも手を出したがる。	きみはなんでもかんでもてをだしたがる 
\\	彼女はめったに涙を流さない。	かのじょはめったになみだをながさない 
\\	私には読書の時間がめったにない。	わたしはどくしょのじかんがめったにない 
\\	私はとても臆病者なのでめったに歯医者に行かない。	わたしはとてもおくびょうものなのでめったにはいしゃにいかない 
\\	私は旅に出かけたときはめったに孤独な感じがしなかった。	わたしはたびにでかけたときはめったにこどくなかんじがしなかった 
\\	大量の文書を入力しなきゃならない。	たいりょうのぶんしょをにゅうりょくしなきゃならない 
\\	そのデータはコンピューターに入力された。	そのデータはコンピューターににゅうりょくされた 
\\	彼女は化粧が濃い。	かのじょはけしょうがこい 
\\	母は出かける前に化粧をした。	はははでかけるまえにけしょうをした 
\\	あの店ではもう化粧品は売っていない。	あのみせはもうけしょうひんはうっていない 
\\	彼女は20分で化粧を済ませた。	かのじょはにじゅうぷんでけしょうをすませた 
\\	彼らは食事を済ませた。	かれらはしょくじをすませた 
\\	買い物を済ませてしまったら電話します。	かいものをすませてしまったらでんわします 
\\	食後私はコーヒー無しでは済ませられない。	しょくごわたしはコーヒーなしではすませられない 
\\	良い辞書無しで済ます事は出来ない。	よいじしょなしですますことはできない 
\\	私が帰宅したときには、弟は宿題を済ませていた。	わたしがきたくしたときには、おとうとはしゅくだいをすませていた 
\\	ある日のこと、警察が娼婦の集団を手入れした。少女もその集団の一員だった。	あるひのこと、けいさつがしょうふのしゅうだんをていれした。しょうねんもそのしゅうだんのいちいんだった 
\\	最近の漫画は暴力や性の描写が多すぎる。	さいきんのまんがはぼうりょくやせいのびょうしゃがおおすぎる 
\\	その美しさを言葉では描写できない。	そのうつくしさをことばではびょうしゃできない 
\\	彼はそこで起きたことを正確に描写した。	かれはそこでおきたことをせいかくにびょうしゃした 
\\	その作家は殺人事件を生々しく描写した。	そのさっかはさつじんじけんをなまなましくびょうしゃした 
\\	彼は地震の後の混乱を生々しく描写した。	かれはじしんのあとのこんらんをなまなましくびょうしゃした 
\\	その交通事故は彼の記憶に生々しい。	そのこうつうじこはかれのきおくになまなましい 
\\	平野さんは優秀な技術者として尊敬されている。	ひらのさんはゆうしゅうなぎじゅつしゃとしてそんけいされている 
\\	その委員会は科学者と技術者からなる。	そのいいんかいはかがくしゃとぎじゅつしゃかれなる 
\\	そのことは人間全般に言える。	そのことはにんげんぜんぱんにいえる 
\\	全般的な状況はわれわれに有利だ。	ぜんぱんてきなじょうきょうはわれわれにゆうりだ 
\\	全般的に彼女はとても信頼のおける人間だ。	ぜんぱんてきにかれはとてもしんらいのおかげにんげんだ 
\\	これらのすべての装置は信頼性に欠けている。	これらのすべてのそうちはしんらいせいにかけている 
\\	あの子を無理に塾に通わせるのは反対だね。	あのこをむりにじゅくにかよわせるのははんたいだね 
\\	競争は激しくなった。	きょうそうははげしくなった 
\\	数社が契約を取ろうと競争している。	すうしゃがけいやくをとろうときょうそうしている 
\\	彼はテニスでは私のよい競争相手だ。	かれはテニスでわたしのよいきょうそうあいてだ 
\\	ゆっくりで着実なのが競走に勝つ。	ゆっくりでちゃくじつなのがきょうそうにかつ 
\\	犯罪率は着実に増加している。	はんざいりつはちゃくじつにぞうかしている 
\\	患者は着実に回復に向かっている。	かんじゃはちゃくじつにかいふくにむかっている 
\\	食料の供給は着実な改善を示している。	しょくりょうのきょうきゅうはちゃくじつなかいぜんをしめしている 
\\	二国間の貿易は着実に増加している。	にこくかんのぼうえきはちゃくじつにぞうかしている 
\\	彼はクラスを代表して会場に出た。	かれはクラスをだいひょうしてかいじょうにでた 
\\	コンサート会場には大勢の聴衆がいた。	コンサートかいじょうにはおおぜいのちょうしゅうがいた 
\\	新入生は希望に胸を膨らませて会場に入った。	しんにゅうせいはきぼうにむねをふくらませてかいじょうにはいった 
\\	新入生歓迎会は楽しかったですか。	しんにゅうせいかんげいかいはたのしかったですか 
\\	新入生向けのオリエンテーションを行う。	しんにゅうせいむけのオリエンテーションをおこなう 
\\	ご協力に感謝致します。	ごきょうりょくにかんしゃいたします 
\\	彼は協力の可能性を調査した。	かれはきょうりょくのかのうせいをちょうさした 
\\	大統領は国民に協力を呼びかけた。	だいとうりょうはこくみんにきょうりょくをよびかけた 
\\	彼は同僚と協力してその計画を立てた。	かれはどうりょうときょうりょくしてそのけいかくをたてた 
\\	主任技師は助手と協力して研究した。	しゅにんぎしはじょしゅときょうりょくしてけんきゅうした 
\\	技師が電柱を上った。	ぎしがでんちゅうをのぼった 
\\	その若い技師は経験が不足していた。	そのわかいぎしはけいけんがふそくしていた 
\\	壊れかかっていたので、技師達はその橋を爆破した。	こわれかかっていたので、ぎしたちはそのはしをばくはした 
\\	その技師はその機械をどう使ったらよいか私達に教えてくれた。	そのぎしはそのきかいをどうつかったらよいかわたしたしたちにおしえてくれた 
\\	彼の助手は彼の靴を磨いた。	かれのじょしゅはかれのくつをみがいた 
\\	彼は助手を非常に信頼している。	かれはじょしゅをひじょうにしんらいしている 
\\	車は電柱にぶつかった。	くるまはでんちゅうにぶつかった 
\\	彼が車をバックさせようとして電柱にぶつけたとき、彼の車をひどく壊れた。	かれはくるまをバックさせようとしてでんちゅうにぶつけたとき、かれのくるまをひどくこわれた 
\\	そこで何が起こったのかを彼は見事に書き表した。	そこでなにがおこったのかをかれはみごとにかきあらわした 
\\	彼が家を飛び出したのは父親が厳しかったせいだ。	かれがいえをとびだしたのはちちおやがきびしかったせいだ 
\\	万一失敗しても、落胆するな。	まんいちしっぱいしても、らくたんするな 
\\	彼は落胆した気持ちを表した。	かれはらくたんしたきもちをあらわした 
\\	その候補者は選挙の結果に落胆した。	そのこうほしゃはせんきょのけっかにらくたんした 
\\	彼が落胆している事は誰の目にも明らかだった。	かれがらくたんしていることはだれのめにもあきらかだった 
\\	彼はたった一回の失敗で失意落胆するような人間ではない。	かれはたったいっかいのしっぱいでしついらくたんするようなにんげんではない 
\\	試験の成績のせいでとても落胆している。	しけんのせいせきのせいでとてもらくたんしている 
\\	ビルは失意のうちに死んだ。	ビルはしついのうちにしんだ 
\\	その知らせで彼は失意した。	そのしらせでかれはしついした 
\\	7は縁起のいい番号だ。	ななはえんぎのいいばんごうだ 
\\	2は英語では縁起の悪い数字だ。	にはえいごではえんぎのわるいすうじだ 
\\	縁起に塩をまいて清める。	えんぎにしおをまいてきよめる 
\\	心を清める。	こころをきよめる 
\\	彼女は食器を洗うのを嫌がらなかった。	かのじょはしょっきをあらうのをいやがらなかった 
\\	彼は誤りを認めるのを嫌がらない。	かれはあやまりをみとめるのをいやがらない 
\\	先生は問題を説明し直すことを嫌がらなかった。	せんせいはもんだいをせつめいしなおすことをいやがらなかった 
\\	ナンシーは外国で一人で暮らすのを嫌がらなかった。	ナンシーはがいこくでひとりでくらすのをいやがらなかった 
\\	食器を台所に運んでね。	しょっきをだいどころにはこんでね 
\\	食器用洗剤で手がかぶれました。	しょっきようせんざいでてがかぶれました 
\\	化粧品にかぶれました。	けしょうひんにかぶれました 
\\	誤って洗剤を飲んでしまいました。	あやまってせんざいをのんでしまいました 
\\	先ずご両親に話してみるのが順序というものだろう。	まずごりょうしんにはなしてみるのがじゅんじょというものだろう 
\\	彼女は手紙を読んだあとで、それを細かく破った。	かのじょはてがみをよんだあとで、それをこまかくやぶった 
\\	実験の結果は我々の期待に添わなかった。	じっけんのけっかはわれわれのきたいにそわなかった 
\\	私は動物に実験を行った。	わたしはどうぶつにじっけんをおこなった 
\\	彼らは昼も夜も実験を続けた。	かれらはひるもよるもじっけんをつづけた 
\\	その実験には綿密な観察が必要だ。	そのじっけんにはめんみつなかんさつがひつようだ 
\\	同一現象が観察された。	どういつげんしょうがかんさつされた 
\\	意見と事実を同一視してはいけない。	いけんとじじつをどういつししてはいけない 
\\	幸福とお金とが同一視されることがある。	こうふくとおかねとがどういつしされることがある 
\\	金を幸福と同一視するなんてばかげたことだ。	かねをこうふくとどういつしするなんてばかげたことだ 
\\	その2つの役は同一の女優によって演じられた。	そのふたつのやくはどういつのじょゆうによってえんじられた 
\\	私は自分を映画のやくざと同一視した。	わたしはじぶんをえいがのやくざとどういつしした 
\\	その現象は今の時代に特有のものだ。	そのげんしょうはいまのじだいにとくゆうのものだ 
\\	虹は最も美しい自然現象の1つだ。	にじはもっともうつくしいしぜんげんしょうのひとつだ 
\\	生物学者はその現象の観察に集中した。	せいぶつがくしゃはそのげんしょうのかんさつにしゅうちゅうした 
\\	私はその講義に集中した。	わたしはそのこうぎにしゅうちゅうした 
\\	彼の仕事の大半が都市部に集中している。	かれのしごとのたいはんがとしぶにしゅうちゅうしている 
\\	太郎は英単語を暗記するのに集中した。	たろうはえいたんごをあんきするのにしゅうちゅうした 
\\	嘘をついているぞと私が言うと、彼女は憤慨していた。	うそをついているぞとわたしがいうと、かのじょはふんがいした 
\\	我が子を叩く親には本当に憤慨させられる。	わがこをたたくおやにはほんとうにふんがいさせられる 
\\	玄関で扉を叩く音がした。	げんかんでとびらをたたくおとがした 
\\	我々はその建物に入る秘密の扉を発見した。	われわれはそのたてものにはいるひみつのとびらをはっけんした 
\\	劇場の扉の上方にかかっているその言葉は、高さ1メートルありました。	げきじょうのとびらのじょうほうにかかっているそのことばは、たかさいちメートルありました 
\\	彼の言うことには理論の飛躍がありすぎる。	かれのいうことにはりろんのひやくがありすぎる 
\\	そんなに急激に利率が下がるとは誰も予想しなかった。	そんなにきゅうげきにりりつがさがるとはだれもよそうしなかった 
\\	その問題の解決は予想以上に難しかった。	そのもんだいのかいけつはよそういじょうにむずかしかった 
\\	塩分の高い食事をとると、高血圧の原因になるかもしれない。	えんぶんのたかいしょくじをとると、こうけつあつのげんいんになるかもしれない 
\\	彼は高血圧に悩んでいた。	かれはこうけつあつになやんでいた 
\\	血圧は健康のバロメーターとして重要である。	けつあつはけんこうのバロメーターとしてじゅうようである 
\\	血圧を計りましょう。	けつあつをはかりましょう 
\\	君の場合は例外としよう。	きみのばあいはれいがいとしよう 
\\	緊急の場合は警察を呼びなさい。	きんきゅうのばあいはけいさつをよびなさい 
\\	彼が助けを求めようとする場合は殆どない。	かれはたすけをもとめようとするばあいはほとんどない 
\\	旅行を計画する場合は、家族全員の希望を考慮すべきだ。	りょこうをけいかくするばあいは、かぞくぜんいんのきぼうをこうりょすべきだ 
\\	経験というものは、大きな犠牲を払って得た場合は、決して忘れてしまうことはないものだ。	けいけんというものは、おおきなぎせいをはらってえたばあいはけっしてわすれてしまうことはないものだ 
\\	彼が不合格だったのには驚いた。	かれはふごうかくだったのにはおどろいた 
\\	今朝遅刻したのには理由がある。	けさちこくしたのにはりゆうがある 
\\	不景気なのに依然物価は高い。	ふけいきなのにいぜんぶっかはたかい 
\\	彼女は利口なのによく驚嘆する。	かのじょはりこうなのによくきょうたんする 
\\	みんな彼女の勇気に驚嘆した。	みんなかのじょのゆうきにきょうたんした 
\\	彼の進歩ぶりに驚嘆せざるをえない。	かれのしんぽぶりにきょうたんせざるをえない 
\\	結局のところ、人生で一番大切な要素は驚嘆の気持ちです。	けっきょくのところ、じんせいでいちばんたいせつなようそはきょうたんのきもちです 
\\	いつもワイドな視野を持って、仕事をしなさい。	いつもワイドなしやをもって、しごとをしなさい 
\\	これがお米の炊き方です。	これはおこめのたきかたです 
\\	とろ火で時間をかけて豆を炊いてください。	とろびでじかんをかけてまめをたいてください 
\\	炊飯器のスイッチを入れてね。	すいはんきのスイッチをいれてね 
\\	パパは、何か焦げる匂いがして、電話が不通だと言っていたわ。	パパは、なにかこげるにおいがして、でんわがふつうだといっていたわ 
\\	交通が全く不通になっている。	こうつうがまったくふつうなっている 
\\	魚が真っ黒に焦げた。	さかながまっくろにこげた 
\\	それは焦げている臭いとは全然違う。	それはこげているにおいとぜんぜんちがう 
\\	地震で鉄道の運行が不通になった。	じしんでてつどうのうんこうがふつうになった 
\\	我々の便が運行中止となった。	われわれのびんがうんこうちゅうしとなった 
\\	バスは20分間隔で運行されている。	バスはにじゅうぷんかんかくでうんこうされている 
\\	彼は地球が太陽の周りを運行すると言った。	かれはちきゅうがたいようのまわりをうんこうするといった 
\\	彗星は運行しながら後ろに光の尾を引く。	すいせいはうんこうしながらうしろにひかりのおをひく 
\\	彼は新しい彗星を発見したと主張した。	かれはあたらしいすいせいをはっけんしたとしゅちょうした 
\\	ハレー彗星は、2061年に戻ってくる。	ハレーすいせいは、にせんろくじゅういちねんにもどってくる 
\\	彼らは互いに敵意を抱いている。	かれらはたがいにてきいをいだいている 
\\	私たちは村人から敵意を持って迎えられた。	わたしたちはむらびとからてきいをもってむかえられた 
\\	最善を尽くすつもりだ。	さいぜんをつくすつもりだ 
\\	選手一人一人が最善を尽くした。	せんしゅひとりひとりがさいぜんをつくした 
\\	最善と思われる処置を取りなさい。	さいぜんとおもわれるしょちをとりなさい 
\\	早急な処置が必要だ。	さっきゅうなしょちがひつようだ 
\\	彼に応急処置をお願いします。	かれにおうきゅうしょちをおねがいします 
\\	応急処置のできる人はいませんか。	おうきゅうしょちのできるひとはいますか 
\\	この件について早急に調べてください。	このけんについてさっきゅうにしらべてください 
\\	壊れていたものを早急に交換していただけますか。	こわれていたものをさっきゅうにこうかんしていただきますか 
\\	ダイヤル直通ですか。	ダイヤルちょくつうですか 
\\	私達は大坂からロサンゼルスまで直通で飛んでいった。	わたしたちはおおさかからロサンゼルスまでちょくつうでとんでいった 
\\	最後の最後に断ってきた。	さいごのさいごにことわってきた 
\\	何故彼の昇給の要求を断ったのですか。	なぜかれのしょうきゅうのようきゅうをことわったのですか 
\\	その議論の根拠は何ですか。	そのぎろんのこんきょはなんですか 
\\	彼が正直だと信じる根拠は十分ある。	かれはしょうじきでとしんじるこんきょはじゅうぶんある 
\\	これらの主張には科学的な根拠がない。	これらはしゅちょうにはかがくてきなこんきょがない 
\\	その結論はしっかりした根拠に基づいている。	そのけつろんはしっかりしたこんきょにもとづいている 
\\	事故は新しい安全対策のための有力な根拠となった。	じこはあたらしいあんぜんたいさくのまめのゆうりょくなこんきょとなった 
\\	彼は政界の有力者だ。	かれはせいかいのゆうりょくしゃだ 
\\	彼は出版業界に有力なコネがある。	かれはしゅっぱんぎょうかいにゆうりょくなネコがある 
\\	彼はその職の最も有力な候補者だった。	かれはそのしょくのもっともゆうりょくなこうほしゃだった 
\\	弁護士は彼女が潔白だという有力な証拠を握っている。	べんごしはかのじょがけっぱくだというゆうりょくなしょうこをにぎっている 
\\	彼は自分の潔白を主張した。	かれはじぶんのけっぱくをしゅちょうした 
\\	彼女が潔白であると信じて疑いません。	かのじょがけっぱくであるとしんじてうたがいません 
\\	彼女は僕の潔白を確信しているんだ。	かのじょはぼくのけっぱくをかくしんしているんだ 
\\	これは有効な犯罪防止対策だ。	これはゆうりょくなはんざいぼうしたいさくだ 
\\	交通事故の防止対策を講じなければならない。	こうつうじこのぼうしたいさくをこうじなければならない 
\\	日本政府は不況対策を講じる事になるだろう。	にほんせいふはふきょうたいさくをこうじることになるだろう 
\\	非行者に対して強硬な対策を講じるべきだ。	ひこうものにたいしてきょうこうなたいさくをこうじるべきだ 
\\	私のウイルス対策用ソフトウエアは不良品でした。	わたしのウイルスたいさくようソフトウエアはふりょうひんでした 
\\	農民たちはよい林檎と不良品を選別する。	のうみんたちはよいりんごとふりょうひんをせんべつする 
\\	卵は大きさと重さによって選別された。	たまごはおおきさとおもさによってせんべつされた 
\\	そのビデオは全くの不良品だ。	そのビデオはまったくのふりょうひんだ 
\\	彼は自分の意見を強硬に主張した。	かれはじぶんのいけんをきょうこうにしゅちょうした 
\\	彼は自分のプランを強硬に推し進めた。	かれはじぶんのぷらんをきょうこうにおしすすめた 
\\	仕事をどんどん推し進めるようにしましょう。	しごとをどんどんおしすすめるようにしましょう 
\\	政府は農民に新税を課した。	せいふはのうみんにしんぜいをかした 
\\	農民は政府に対して反乱を起こした。	のうみんはせいふにたいしてはんらんをおこした 
\\	彼はインドの反乱を鎮めた。	かれはインドのはんらんをしずめた 
\\	反乱は鎮圧された。	はんらんはちんあつされた 
\\	軍隊は簡単に反乱を鎮圧した。	ぐんたいはかんたんにはんらんをちんあつした 
\\	その暴動はすぐに警察によって鎮圧された。	そのぼうどうはすぐにけいさつによってちんあつされた 
\\	暴動の噂が広まっていた。	ぼうどうのうわさがひろまっていた 
\\	暴動はすぐに鎮められた。	ぼうどうはすぐにしずめられた 
\\	暴動は手の付けられない状態だった。	ぼうどうはてのつけられないじょうたいだった 
\\	暴動を鎮圧するためにただちに軍隊が派遣された。	ぼうどうをちんあつするためにただちにぐんたいがはけんされた 
\\	社長はあなたの海外派遣を真剣に考えていました。	しゃちょうはあなたのかいがいはけんをしんけんにかんがえていました 
\\	王は人民に重税を課した。	おうはじんみんにじゅうぜいをかした 
\\	彼はとても大きな宿題を課したので、私たちは抗議した。	かれはとてもおおきなしゅくだいをかしたので、わたしたちはこうぎした 
\\	多くの小さな会社が倒産した。	おおくのちいさなかいしゃがとうさんした 
\\	会社は赤字経営で倒産した。	かいしゃはあかじけいえいでとうさんした 
\\	あの会社は赤字を脱した。	あのかいしゃはあかじをだっした 
\\	危機をどうやら脱したようだ。	ききをどうやらだっしたようだ 
\\	会社は資金不足のため倒産した。	かいしゃはしきんぶそくのためとうさんした 
\\	その企業が倒産するという噂が広まっている。	そのきぎょうがとうさんするといううわさがひろまっている 
\\	彼が辞職するという噂が広まっている。	かれはじしょくするといううわさがひろまっている 
\\	ジャズはバッハが使ったのと同じ音符を使う。	ジャズはバッハがつかったのとおなじおんぷをつかう 
\\	警報機が鳴っているときは絶対に踏み切りを渡らないでください。	けいほうきがなっているときはぜったいにふみきりをわたらないでください 
\\	火災警報が鳴った。	火災けいほうがなった 
\\	津波警報は解除された。	つなみけいほうはかいじょされた 
\\	大地震が起これば警報器がなるでしょう。	おおじしんがおこればけいほうきがなるでしょう 
\\	2階で朝早い時間に警報器が鳴っているのが聞こえた。	にかいであさはやいじかんにけいほうきがなっているのがきこえた 
\\	地震が起こり、更に津波が襲った。	じしんがおこり、さらにつなみがおそった 
\\	政府は物価の統制を解除した。	せいふはぶっかのとうせいをかいじょした 
\\	戦争中には物価は激しく統制されていた。	せんそうちゅうにはぶっかはきびしくとうせいされていた 
\\	群衆は統制が効かなくなってフェンスを突き破った。	ぐんしゅうはとうせいがきかなくなってフェンスをつきやぶった 
\\	春は冬と夏の間に訪れる。	はるはふゆとなつのあいだにおとずれる 
\\	彼は日曜日に農場を訪れるつもりだ。	かれはにちようびにのうじょうをおとずれるつもりだ 
\\	彼は農場を継いだのを後悔した。	かれはのうじょうをついだのをこうかいした 
\\	彼は王位を継ぐだろう。	かれはおういをつぐだろう 
\\	彼は叔父の財産を継いだ。	かれはおじのざいさんをついだ 
\\	彼はその商売を引き継いだ。	かれはそのしょうばいをひきついだ 
\\	彼は父の仕事を継ぎたい。	かれはちちのしごとをつぎたい 
\\	王子が王位を継承した。	おうじがおういをけいしょうした 
\\	どちらの王子が正当な王位継承者か。	どちらのおうじがせいとうなおういけいしょうしゃか 
\\	彼には自分を継承してくれる子供が一人もいない。	かれにはじぶんをけいしょうしてくれるこどもがひとりもいない 
\\	朝鮮半島を訪れたことがありますか。	ちょうせんはんとうをおとずれたことがありますか 
\\	湿気の多い気候はその半島の特色です。	しっけのおおいきこうはそのはんとうのとくしょくです 
\\	確かに暑いが、湿気はない。	たしかにあついが、しっけはない 
\\	日本は多くのはっきりした特色がある。	にほんはおおくのはっきりしたとくしょくがある 
\\	その地方の自然の特色を教えてくれませんか。	そのちほうのしぜんのとくしょくをおしえてくれませんか 
\\	日本語は朝鮮語と共通点がある。	にほんごはちょうせんごときょうつうてんがある 
\\	遠距離恋愛をしたことはありますか。	えんきょりれんあいをしたことはありますか 
\\	お母さんは彼らを野原で遊ばせた。	おかあさんはかれらをのはらであそばせた 
\\	野原には六頭の羊がいた。	のはらにはろくとうのひつじがいた 
\\	彼の専門的知識の欠陥が昇進を妨げた。	かれのせんもんてきちしきのけっかんがしょうしんをさまたげた 
\\	彼の実験は細部において多くの欠陥があった。	かれのじっけんはさいぶにおいておおくのけっかんがあった 
\\	彼は野原で犬を自由に走らせた。	かれはのはらでいぬをじゆうにはしらせた 
\\	細部まで気を配りなさい。	さいぶまできをくばりなさい 
\\	彼は行儀には気を配っている。	かれはぎょうぎにはきをくばっている 
\\	彼女は客の対応に気を配っている。	かのじょはきゃくをたいおうにきをくばっている 
\\	今後どんな細部にも注意を払いなさいよ。	こんごどんなさいぶにもちゅういをはらいなさいよ 
\\	その日本語に対応する英語はない。	そのにほんごにたいおうするえいごはない 
\\	私に対応してくれる店員が見当たらなかった。	わたしにたいおうしてくれるてんいんがみあたらなかった 
\\	私は敵意をもったまでも、冷ややかな対応を受けた。	わたしはてきいをもっったまでも、ひややかなたいおうをうけた 
\\	道路は様々な乗物で混雑していた。	どうろはさまざまなのりものでこんざつしていた 
\\	私は行動する前に様々な要因を考えた。	わたしはこうどうするまえにさまざまなよういんをかんがえた 
\\	健康は幸福の重要な要因だ。	けんこうはこうふくのじゅうようなよういんだ 
\\	彼の失敗の第一の要因は怠惰である事だ。	かれのしっぱいのだいいちのよういんはたいだであることだ 
\\	値段は決断をする際に非常に重要な要因となる。	ねだんはけつだんをするさいにひじょうにじゅうようなよういんとなる 
\\	これら二つの要因は互いに無関係である。	これらふたつのよういんはたがいにむかんけいである 
\\	友人を選ぶ際には気を付けるべきだ。	ゆうじんをえらぶさいにはきをつけるべきだ 
\\	金は実際に支払われたのですか。	かねはじっさいにしはらわれたのですか 
\\	実際に何が起こるか誰にも分からない。	じっさいになにがおこるかだれにもわからない 
\\	皮製の椅子を持つのが流行です。	かわせいのいすをもつのがはやりです 
\\	食物包装は腐敗を減らす。	たべものほうそうはふはいをへらす 
\\	これらの贈り物といっしょに包装してください。	これらのおくりものといしょにほうそうしてください 
\\	包装によって実際にある種の無駄を防ぐことが出来る。	ほうそうによってじっさいにあるしゅのむだをふせぐことができる 
\\	鉄道はある種の革命を引き起こした。	てつどうはあるしゅのかくめいをひきおこした 
\\	ある種の毒は、適当に使えば役に立つ。	あるしゅのどくは、てきとうにつかえばやくにたつ 
\\	ある種の植物は寒さに順応できない。	あるしゅのしょくぶつはさむさにじゅんのうできない 
\\	塩は魚が腐るのを防ぐ。	しおはさかながくさるのをふせぐ 
\\	彼の迅速な行動により伝染病を防ぐことができた。	かれのじんそくなこうどうによりでんせんびょうをふせぐことができた 
\\	警察は暴動に対処するため迅速な行動をとった。	けいさつはぼうどうにたいしょするためじんそくなこうどうをとった 
\\	彼はその事態に迅速に対応した。	かれはそのじたいにじんそくにたいおうっした 
\\	伝染病が発生した。	でんせんびょうがはっせいした 
\\	伝染病が不意にその町を襲った。	でんせんびょうがふいにそのまちをおそった 
\\	我々は敵の不意をついた。	われわれはてきのふいをついた 
\\	その車は不意に方向を変えた。	そのくるまはふいにほうこうをかえた 
\\	町の真ん中で火災が発生した。	まちのまんなかでかさいがはっせいした 
\\	山火事が発生して森林を焼いた。	やまかじがはっせいしてしんりんをやいた 
\\	広大な森林が山々を覆っている。	こうだいなしんりんがやまやまをおおっている 
\\	太平洋は非常に広大だ。	たいへいようはひじょうにこうだいだ 
\\	こんな広大な景色は初めて見ました。	こんなこうだいなけしきははじめてみました 
\\	日本は東は太平洋に面する。	にほんはひがしはたいへいようにめんする 
\\	海に面した部屋に替えて下さい。	うみにめんしたへやにかえてください 
\\	雪がバスを覆っている。	ゆきがバスをおおっている 
\\	彼女の新しい髪形は耳を覆っている。	かのじょのあたらしいかみがたはみみをおおっている 
\\	畑は雑草で覆われていた。	はたけはざっそうでおおわれていた 
\\	頂上は雪で覆われている。	ちょうじょうはゆきでおおわれている 
\\	歩道は落ち葉で覆われていた。	ほどうはおちばでおおわれていた 
\\	一面の落ち葉。	いちめんのおちば 
\\	1枚の落ち葉が水面に浮かんでいた。	いちまいのおちばがすいめんにうかんでいた 
\\	あの落ち葉をどうやって取り除けようか。	あのおちばをどうやってとりのぞけようか 
\\	花びらが水面に浮かんでいる。	はなびらがすいめんにうかんでいる 
\\	静かな水面は、鏡のように彼女の顔立ちを映し出していた。	しずかなすいめんは、かがみのようにかのじょのかおだちをうつしだしていた 
\\	広子は魅力的な顔立ちをしている。	ひろこはみりょくてきなかおだちをしている 
\\	将軍は私たちと握手をした。	しょうぐんはわたしたちとあくしゅをした 
\\	将軍はその都市の攻撃を命じた。	しょうぐんはそのとしのこうげきをめいじた 
\\	その色白の女の子は19歳でとおっている	そのいろじろのおんなのこは19さいでとおっている 
\\	教室の整頓が命じられた。	きょうしつのせいとんがめいじられた 
\\	しかし部屋を整頓するのは面倒臭いし。	しかしへやをせいとんするのはめんどうくさいし 
\\	彼はその書類を整頓しようとするのを諦めた。	かれはそのしょるいをせいとんしようとするのをあきらめた 
\\	彼女は息子の面倒をみた。	かのじょはむすこのめんどうをみた 
\\	私は将来両親の面倒を見ます。	わたしはしょうらいりょうしんのめんどうをみます 
\\	結局面倒なのでタクシーで帰宅しました。	けっきょくめんどうなのでタクシーできたくしました 
\\	勢いが弱まってきたね。	いきおいがよわまっていたね 
\\	日本は戦後繁栄を享受している。	にほんはせんごはんえいをきょうじゅしている 
\\	女性は何故男性と同じ市民権を享受することが許されないのか。	じょせいはなぜだんせいとおなじしみんけんをきょうじゅすることがゆるされないのか 
\\	この実験では不注意は許されない。	このじっけんではふちゅういはゆるされない 
\\	昔、この港は繁栄していた。	むかし、このみなとははんえいしていた 
\\	国家の繁栄は主としてその青年にかかっている。	こっかのはんえいはしゅとしてそのせいねんにかかっている 
\\	マヤ文明はどのくらいの間に繁栄したのだろうか。	マヤぶんめいはどのぐらいのあいだにはんえいしたのだろうか 
\\	彼は米国の市民権を得た。	かれはべいこくのしみんけんをえた 
\\	彼は市民権を奪われた。	かれはしみんけんをうばわれた 
\\	子供はほうれん草が嫌いなことが多い。	こどもはほうれんそうがきらいなことがおおい 
\\	その州は周辺にさまざまな原料を供給している。	そのしゅうはしゅうへんにさまざまなげんりょうをきょうきゅうしている 
\\	彼は南極周辺の地域を探検した。	かれはなんきょくしゅうへんのちいきをたんけんした 
\\	昨年南極で厳しい寒さを経験した。	さくねんなんきょくできびしいさむさをけいけんした 
\\	姿勢を正しなさい。	しせいをただしなさい 
\\	彼は楽な姿勢で横になった。	かれはらくなしせいでよこになった 
\\	食卓の上に身をかがめないで、姿勢を正しなさい。	しょくたくのうえにみをかがめないで、しせいをただしなさい 
\\	彼女は身をかがめてコインを拾った。	かのじょはみをかがめてコインをひろった 
\\	摩擦でマッチに火がつくようになる。	まさつでマッチにひがつくようになる 
\\	米国人とイギリス人との間の摩擦が高まった。	べいこくじんとイギリスじんのあいだのまさつがたかまった 
\\	私は彼女が貿易摩擦について話すと思う。	わたしはかのじょがぼうえきまさつについてはなすとおもう 
\\	両国の間では貿易摩擦がいつ生じてもおかしくない。	りょうこくのあいだではぼうえきまさつがいつしょうじてもおかしくない 
\\	服がぴったり合っていた。	すきまなく密着しているさま。
\\	この合計は私のとぴったりあう。	このごうけいはわたしのとぴったりあう 
\\	とにかく〔どんなことがあっても〕最善を尽くします。	とにかく〔どんなことがあっても〕さいぜんをつくします 
\\	というわけで、人材を啓発するためのセンターが日本に作られるべきであろう。	というわけで、じんざいをけいはつするためのセンターがにほんにつくられるべきであろう 
\\	彼は会社にとって重要な人材だ。	かれはかいしゃにとってじゅうようなじんざいだ 
\\	私が目覚めるともう日は高く昇っていた。	わたしがめざめるともうひはたかくのぼっていた 
\\	眠っているライオンを目覚めさせるな。	ねむっているライオンをめざめさせるな 
\\	人は溜まれば溜まるほど欲しくなるものだ。	ひとはたまればたまるほどほしくなるものだ 
\\	雨のあと、道路に水溜まりができた。	あめのあと、どうろにみずたまりがてきた 
\\	彼女は目に涙を溜めていた。	かのじょはめがなmだをためていた 
\\	彼は海外旅行のため金を溜めている。	かれはがいこくりょこうのためかねをためている 
\\	彼女は出来るだけお金を溜めようと努力している。	かのじょはできるだけおかねをためようとどりょくしている 
\\	何事も中途半端にするな。	なにごともちゅうとはんぱにするな 
\\	火事は中途半端なやり方では防げない。	かじはちゅうとはんぱなやりかたではふせげない 
\\	仕事を中途半端で辞めてはいけない。	しごとをちゅうとはんぱでやめてはいけない 
\\	彼がもう少し注意していれば、事故は防げたろうに。	かれがもうすこしちゅういしていれば、じこはふせげたろうに 
\\	彼女は庭の雑草を抜いた。	かのじょはにわのざっそうをぬいた 
\\	この雑草は取り除いた方がいい。	このざっそうはとりのぞいたほうがいい 
\\	森林の保護は世界中の重要な問題だ。	しんりんのほごはせかいじゅうのじゅうようなもんだいだ 
\\	ある団体が雨林を保護する運動を起こした。	あるだんたいがうりんをほごするうんどうをおこした 
\\	その丘の頂上は平らである。	そのおかのちょうじょうはたいらである 
\\	彼らは詩人で外交官だった。	かれらはしじんでがいこうかんだった 
\\	彼外交官として長年人生を送ってきた。	かれがいこうかんとしてながねんじんせいをおくってきた 
\\	その両国は外交関係がない。	そのりょうこくはがいこうかんけいがない 
\\	英語は外交や観光事業に有効である。	えいごはがいこうやかんこうじぎょうにゆうこうである 
\\	その国は米国との外交関係を断絶した。	そのくにはべいこくとのがいこうかんけいをだんぜつした 
\\	当時日本は、数多くの外交問題に直面していた。	とうじにほんは、かずおおくのがいこうもんだいにちょくめんしていた 
\\	日本は数多くの優れたカメラを製造している。	にほんはかずおおくのすぐれたカメラをせいぞうしている 
\\	人生には数多くの不可解なことが起こる。	じんせいにはかずおおくのふかかいなことがおこる 
\\	古いインドの伝説が数多くある。	ふるいインドのでんせつがかずおおくある 
\\	その物語は伝説に基づいている。	そのものがたりはでんせつにもとづいている 
\\	その歌は伝説に由来する物だった。	そのうたはでんせつにゆらいするものだった 
\\	伝説では、彼女は人魚だったと言われる。	でんせつでは、かのじょはにんぎょだたといわれる 
\\	この湖についての不思議な伝説が言い伝えられている。	このみずうみについてのふしぎなでんせつがいいつたえられている 
\\	その伝説は昔からこの地方に伝わっている。	そのでんせつはむかしからこのちほうにつたわっている 
\\	見通しは否定的だった。	みとおしはひていてきだった 
\\	彼の欲望は収入と釣り合わない。	かれのよくぼうはしゅうにゅうとつりあわない 
\\	仕事の量と賃金が釣り合っていない。	しごとのりょうとちんぎんがつりあっていない 
\\	人間の欲望は、持てば持つほどますます増大する。	にんげんのよくぼうは、もてばもつほどますますぞうだいする 
\\	彼らは軍事予算を増大させようとした。	かれらはぐんじよさんをぞうだいさせようとした 
\\	エイズで苦しんでいる人の数は増大した。	エイズでくるしんでいるひとのかずはぞうだいした 
\\	宣伝予算を立てる。	せんでんよさんをたてる 
\\	予算が限られています。	よさんがかぎられています 
\\	委員会は予算を承認した。	いいんかいはよさんをしょうにんした 
\\	彼は承認を保留した。	かれはしょうにんをほりゅうした 
\\	さしあたりこの問題は保留としよう。	さしあたりこのもんだいはほりゅうとしよう 
\\	入札が全部出そろうまで、決定を保留してはどうでしょう。	にゅうさつがぜんぶでそろうまで、けっていをほりゅうしてはどうでしょう 
\\	私は彼に対抗して入札した。	わたしはかれにたいこうしてにゅうさつした 
\\	橋の建設の入札が募られた。	はしのけんせつのにゅうさつがつのられた 
\\	仕事に応募する。	しごとにおうぼする 
\\	彼はその奨学金に応募した。	かれはそのしょうがくきんにおうぼした 
\\	通訳の仕事に応募したらどうですか。	つうやくのしごとにおうぼしたらどうですか 
\\	同時通訳が彼女の憧れだ。	どうじつうやくがかのじょのあこがれだ 
\\	彼女は通訳として雇われた。	かのじょはつうやくとしてやとわれた 
\\	その会合で私は通訳を務めた。	そのかいごうでわたしはつうやくをつとめた 
\\	あなたは彼を雇うことができる。	あなたはかれをやとうことができる 
\\	彼は日給で雇われた。	かれはにっきゅうでやとわれた 
\\	この仕事は日給1万円です。	このしごとはにっきゅういちまんえんです 
\\	彼を雇うことは誰も雇わないことに等しい。	かれをやとうことはだれもやとわないことにひとしい 
\\	彼は彼らが名声に憧れていると思った。	かれはかれらがめいせいにあこがれているとおもった 
\\	死を憧れる者は惨めであるが、死を恐れる者はさらに惨めである。	しをあこがれるものはみじめであるが、しをおそれるものはさらにみじめである 
\\	彼は若い頃惨めな生活を送った。	かれはわかいごろみじめなせいかつをおくった 
\\	彼女は惨めな生活を送るよう運命づけられていた。	かのじょはみじめなせいかつをおくるよううんめいづけられていた 
\\	図表2を見ると、これらの貿易サイクルのいくつかは非常に短期のものであることが判明する。	ずひょう2をみると、これらのぼうえきサイクルのいくつかはひじょうにたんきのものであることがはんめいする 
\\	被告は有罪と判明した。	ひこくはゆうざいとはんめいした 
\\	新発見は科学に有益なものであるのが判明した。	しんはっけんはかがくにゆうえきなものであるのがはんめいした 
\\	軍事訓練とは兵士たちが受ける訓練である。	ぐんじくんれんとはへいしたちがうけるくんれんである 
\\	そのうまい宣伝に乗るな。	そのうまいせんでんにのるな 
\\	その商品はテレビで宣伝されている。	そのしょうひんはテレビでせんでんされている 
\\	彼は自己宣伝に熱心だ。	かれはじこせんでんにねっしんだ 
\\	彼が武器を商っているのは公然の秘密である。	かれはぶきをあきなっているのはこうぜんのひみつである 
\\	僕はレースで君を吉田君に対抗させることを考えているんだ。	ぼきはレースできみをよしだにたいこうさせることをかんがえているんだ 
\\	同時に起立した。	どうじにきりつした 
\\	名前を呼ばれたら起立しなさい。	なまえをよばれたらきりつしない 
\\	生徒達は先生が入って来ると起立する。	せいとたちはせんせいがはいってくるときりつする 
\\	彼は私に起立するように命令した。	かれはわたしにきりつするようにめいれいした 
\\	彼は厳しいと同時に優しい。	かれはきびしいとどうじにやさしい 
\\	同時に2つの場所にいる事は出来ない。	どうじにふたつのばしょにいることはできない 
\\	私たちの結婚記念日はもうすぐです。	わたしたちのけっこんきねんびはもうすぐです 
\\	学生は創立記念日で休みだ。	がくせいはそうりつきねんびでやすみだ 
\\	同社の創立は1950年である。	どうしゃのそうりつはせんきゅうひゃくごじゅうねんである 
\\	大学構内の中央に、創立者の像が立っている。	だいがくこうないのちゅうおうに、そうりつしゃのぞうがたっている 
\\	その学校は構内での生徒たちの喫煙を禁止している。	そのがっこうはこうないでのせいとたちのきつえんをきんししている 
\\	スクールバスが学生を駅から大学構内まで運んでいる。	スクールバスががくせいをえきからだいがくこうないまではこんでいる 
\\	アルプス山脈はヨーロッパの中央にある。	アルプスさんみゃくはユーロッパのちゅうおうにある 
\\	彼は舞台の中央に1人残された。	かれはぶたいのちゅうおうにひとりのこされた 
\\	セールスマンはその機械の使用方法を実演してみせた。	セールスマンはそのきかいのしようほうほうをじつえんしてみせた 
\\	やむを得ず契約に署名させられた。	やむをえずけいやくにしょめいさせられた 
\\	彼女はやむを得ずその計画を諦めた。	かのじょはやむをえずそのけいかくをあきらめた 
\\	それはやむを得ず延期されなければならない。	それはやむをえずえんきされなければならない 
\\	彼は家庭の事情でやむを得ず教師になった。	かれはかていのじじょうでやむをえずきょうしになった 
\\	彼はアルプス山脈の夜明けの美しさをぼんやりとたたずんで見ていた。	かれはアルプスさんみゃくのよあけのうつくしさをぼにゃりとたたずんでみていた 
\\	敵の攻撃は夜明けに止んだ。	てきのこうげきはよあけにやんだ 
\\	登山者は夜明け前に目を覚ました。	とざんしゃはよあけまえにめをさました 
\\	熟練したドライバーでもミスをすることがある。	じゅくれんしたドライバーでもミスをすることがある 
\\	現在平均的熟練労働者は1年に一万ドル以上稼ぐ。	げんざいへいきんてきじゅくれんろうどうしゃはいちねんにいちまんドルいじょうかせぐ 
\\	彼は熟練した登山家だ。	かれはじゅくれんしたとざんかだ 
\\	登山家達は暗くなる前に頂上に着いた。	とざんかたちはくらくなるまえにちょうじょうについた 
\\	彼が家出したのは事実だ。	かれはいえでしたのはじじつだ 
\\	少女は家出をして、両親の目の届かないところに行ってしまった。	しょうじょはいえでをして、りょうしんのめのとどかないところに行ってしまった 
\\	こんな贅沢は私には手が届かない。	こんなぜいたくはわたしにはてがとどかない 
\\	風呂の支度が出来ました。	ふろにしたくができました 
\\	彼女は毎朝、母が台所で朝食の支度をするのを手伝います。	かのじょはまいあさ、ははがだいどころでちょうしょくのしたくをするのをするのをてつだいます 
\\	私のロボットは食事の支度、掃除、皿洗いその他の家事が出来るでしょう。	わたしのロボットはしょくじのしたく、そうじ、さらあらいそのほかのかじができるでしょう 
\\	母は家事を切り盛りするのに忙しい。	はははかじをきりもりするのにいそがしい 
\\	彼女は家事を切り盛りするだけでなく学校の先生もしている。	かのじょはかじをきりもりするだけでなくがっこうのせんせいもしている 
\\	やる気があれば方法は見つかるもの。	やるきがあればほうほうはみつかるもの 
\\	やる気があれば英語はもっと上達する。	やるきがあればえいごはもっとじょうたつする 
\\	彼は彼らのやる気の無さに激怒した。	かれはかれらのやるきのなさにげきどした 
\\	彼はやる気満々です。	かれはやるきまんまんです 
\\	彼女は激怒のあまり我を忘れていた。	かのじょはげきどのあまりわれをわすれていた 
\\	彼は喜びに我を忘れた。	かれはよろこびにわれをわすれた 
\\	彼女は有名な歌手に会って我を忘れた。	かれはゆうめいなかしゅにあってわれをわすれた 
\\	彼女は彼の裏切りに激怒した。	かのじょはかれのうらぎりにげきどした 
\\	彼は激怒して体を震わせていた。	かれはげきどしてからだをふるわせていた 
\\	彼は権力獲得に野心満々だ。	かれはけんりょくかくとくにやしんまんまんだ 
\\	彼は勤勉によって賞を獲得した。	かれはきんべんによってしょうをかくとくした 
\\	その建築家は世界的名声を獲得した。	そのけんちくかはせかいてきめいせいをかくとくした 
\\	彼の音楽は海外で大変な人気を獲得した。	かれのおんがくはかいがいでたいへんなにんきをかくとくした 
\\	遂に彼らは血の犠牲によって自由を獲得した。	ついにかれらはちのぎせいによってじゆうをかくとくした 
\\	同じ原因が同じ結果を引き起こすとは限らない。	おなじげんいんがおなじけっかをひきおこすとはかぎらない 
\\	その機関は1960年代後半に設立された。	そのきかんは1960ねんだいこうはんにせつりつされた 
\\	古代ローマ人はヨーロッパ中に植民地を設立した。	こだいローマじんはユーロッパちゅうにしょくみんちをせつりつした 
\\	その協会は設立以来、素人の入会を断ってきた。	そのきょうかいはせつりついらい、しろうとのにゅうかいをことわってきた 
\\	彼はその会へ入会が認められた。	かれはそのかいへにゅうかいがみとめられた 
\\	ローマは古代建築で有名だ。	ローマはこだいけんちくでゆうめいだ 
\\	彼は古代神話に基づく小説を書いた。	かれはこだいしんわにもとづくしょうせつをかいた 
\\	この習慣は古代から続いている。	このしゅうかんはこだいからつづいている 
\\	父は古代史に関心を持っている。	ちちはこだいしにかんしんをもっている 
\\	彼は古代史に精通している。	かれはこだいしにせいつうしている 
\\	彼は物理に精通している。	かれはぶつりにせいつうしている 
\\	物理は私の好きな科目だ。	ぶつりはわたしのすきなかもくだ 
\\	私の物理の知識は貧弱です。	わたしのぶつりのちしきはひんじゃくです 
\\	彼の美術の知識は貧弱だ。	かれのびじゅつのちしきはひんじゃくだ 
\\	彼は芸能界に精通している。	かれはげいのうかいにせいつうしている 
\\	彼は幾何学に精通している。	かれはきかがくにせいつうしている 
\\	頭の良いその少年は幾何学の概念を理解した。	あたまのよいそのしょうねんはきかがくのがいねんをりかいした 
\\	その土地についての僕の概念はあまりはっきりしない。	そのとちについてのぼくのがいねんはあまりはっきりしない 
\\	この章では幾何学の概念に焦点をあてます。	このしょうはきかがくのがいねんにしょうてんをあてます 
\\	彼女の顔に焦点があっている。	かのじょのかおにしょうてんがあっている 
\\	この章ではその惑星の謎に焦点をあてます。	このしょうではそのわくせいのなぞにしょうてんをあてます 
\\	委員会はもっと具体的な問題に焦点を当てるべきだ。	いいんかいはもっとぐたいてきなもんだいにしょうてんをあてるべきだ 
\\	話の焦点は内容に置かれている。	はなしのしょうてんはないようにおかれている 
\\	その謎を解きましたか。	そのなぞをときましたか 
\\	これが謎の全てを解く鍵だ。	これがなぞのすべてをとくかぎだ 
\\	何故彼が妻を殺したのかは謎だと思う。	なぜかれがつまをころしたのかはなぞだとおもう 
\\	謎を未解決のままにするな。	なぞをみかいけつのままにするな 
\\	その問題は未解決のままである。	そのもんだいはみかいけつのままである 
\\	その謎々は今も未解決である。	そのなぞなぞはいまもみかいけつである 
\\	ゼロという概念はヒンドゥー文化に由来している。	ゼロというがいねんはヒンドゥーぶんかにゆらいしている 
\\	彼は米文学に精通している。	かれはべいぶんがくにせいつうしている 
\\	それは素人考えだ。	それはしろうとかんがえだ 
\\	法律用語の大半は素人には分かり難い。	ほうりつようごはたいはんはしろうとにはわかりにくい 
\\	調査機関がその効果を調べた。	ちょうさきかんがそのこうかをしらべた 
\\	その町の交通機関は大変よい。	そのまちのこうつうきかんはたいへんよい 
\\	今では交通機関が発達したため、歩く人が少ないのは遺憾である。	いまではこうつうきかんがはったつしたため、あるくひとがすくないのはいかんである 
\\	このホテルは公共交通機関の点から見ると便利な所に位置している。	このホテルはこうきょうこうつうきかんのてんからみるとべんりなところにいちしている 
\\	日本は北半球に位置する。	にほんはきたはんきゅうにいちする 
\\	標識はエスカレーターの位置を示している。	ひょうしきはエスカレーターのいちをしめしている 
\\	地図は自分の位置を確かめるのに役立った。	ちずはじぶんのいちをたしかめるのにやくだった 
\\	標識には、出口と書かれている。	"ひょうしきには、でぐちとかかれている 
\\	その標識は進む方向を示している。	そのひょうしきはすすむほうこうをしめしている 
\\	停止標識を無視しませんでしたか。	ていしひょうしきをむししませんでしたか 
\\	大部分の標識は英語で書かれている。	だいぶぶんのひょうしきはえいごでかかれている 
\\	「芝生に入らないで下さい」という標識があった。	"「しばふにはいらないでください」というひょうしきがあった 
\\	それは遺憾ながら本当だ。	それはいかんながらほんとうだ 
\\	トムの報告書には遺憾な点が多い。	トムのほうこくしょにはいかんなてんがおおい 
\\	遺憾ながら、2月27日のお約束を守ることが出来ません。	いかんながら、にがつにじゅうしちにちのおやくそくをまもることができません 
\\	このコンピューター1年前に買ったばかりなのに、もうすっかり時代遅れだわ。	このコンピューターいちねんまえにかったばかりなのに、もうすっかりじだいおくれだわ 
\\	彼女の物語が真実の筈がない。	かのじょのものがたりがじじつのはずがない 
\\	空腹の筈がない。彼は昼食とったばかりだから。	くうふくのはずがない。かれはちゅうしょくとったばかりだから 
\\	兄は丁度今札幌から帰ったところだ。	あにはちょうどいまさっぽろからかえったところだ 
\\	お手伝いできません。実を言うと、丁度今とても忙しいのです。	おてつだいできません。じつをいうと、ちょうどいまとてもいそがしいのです 
\\	彼らはワープロにいつも手を焼いている。	かれらはワープロにいつもてをやいている 
\\	たった今明かりが消えた。	たったいまあかりがきえた 
\\	彼はたった今夢から覚めたばかりのようでした。	かれはたったいまゆめからさめたばかりのようでした 
\\	友人を見送りにたった今空港へ行って来た所です。	ゆうじんをみおくりにたったいまくうこうへいってきたところです 
\\	彼女の服は膝まで届いていない。	かのじょのふくはひざまでとどいていない 
\\	彼らは家族を貧困から守る。	かれらはかぞくをひんこんからまもる 
\\	人口の増加こそが貧困を招いた。	じんこうのぞうかこそがひんこんをまねいた 
\\	貧困は依然として犯罪の主要原因である。	ひんこんはいぜんとしてはんざいのしゅようげんいんである 
\\	その殺人事件は依然として謎である。	そのさつじんじけんはいぜんとしてなぞである 
\\	彼が有罪であるという事実は依然として残っている。	かれがゆうざいであるというじけんはいぜんとしてのこっている 
\\	日本の主要作物は米である。	にほんのしゅようさくぶつはこめである 
\\	この計画の主要な特徴はまだ曖昧です。	このけいかくのしゅようなとくちょうはまだあいまいです 
\\	シカゴは、米国中西部の主要な都市である。	シカゴは、べいこくちゅうせいぶのしゅようなとしである 
\\	勤勉さが彼の素晴らしい昇進の主要因だった。	きんべんさがかれのすばらしいしょうしんのしゅよういんだった 
\\	車がぬかるみに填まり込んだ。	くるまがぬかるみにはまりこんだ 
\\	危険を冒さなければ何も得られない。	きけんをおかさなければなにもえられない 
\\	彼の妻は彼に危険を冒さないでね、と懇願した。	かれのつまはかれにきけんをおかさないでね、とこんがんした 
\\	俺はあんたに話し掛けてるんじゃない、猿に話し掛けてるんだ。	おれはあんたにはなしかけてるんじゃない、さるにはなしかけてるんだ 
\\	あんたは空腹の筈がない。少し前に軽食をとったんだからな。	あんたはくうふくのはずがない。すこしまえにけいしょくをとったんだからな 
\\	彼女は私たちに軽食を用意してくれた。	かのじょはわたしたちにけいしょくをよういしてくれた 
\\	大体あんたこの寒いのにノースリーブで何いってんのよ。	だいたいあんたこのさむいのにノースリーブでなにいってんのよ 
\\	何あんた、またパン?相変わらず変わり映えのしない食生活送ってんのね。	なにあんた、またパン?あいかわらずかわりばえのしないしょくせいかつおくってんのね 
\\	私たちの食生活はとても変化に富んでいます。	わたしたちのしょくせいかつはとてもへんかにとんでいます 
\\	カナダは木材に富む。	カナダはもくざいにとむ 
\\	彼は機知に富んだ人だ。	かれはきちにとんだひとだ 
\\	米国は天然資源に富んでいる。	べいこくはてんえんしげんにとんでいる 
\\	ビルは独創的な考えに富んでいる。	ビルはどくそうてきなかんがえにとんでいる 
\\	彼の小論文は独創的な考えに満ち溢れていた。	かのじょのしょうろんぶんはどくそうてきなかんがえにみちあふれていた 
\\	彼のデザインは大変独創的だ。	かれのデザインはたいへんどくそうてきだ 
\\	彼は希望に満ち溢れていた。	かれはきぼうにみちあふれていた 
\\	彼女は冒険心に満ち溢れている。	かのじょはぼうけんしんにみちあふれている 
\\	彼は奨学金を得られそうだ。	かれはしょうがくきんをえられそうだ 
\\	部長カンカンになって怒ってたぞ。お前何をしたんだよ。	ぶちょうカンカンになっておこってたぞ。おまえなにをしたんだよ 
\\	お前はまだ一人で泳ぎに行くにはまだ幼すぎる。	おまえはまだひとりでおよぎにいくにはまだおさなすぎる 
\\	幼い少年は彼の犬を抱きしめた。	おさないしょうねんはかれのいぬをだきしめた 
\\	幼い子どもを優れた音楽に触れさせるべきだ。	おさないこどもをすぐれたおんがくにふれさせるべきだ 
\\	その男は幼い少女を殺したかどで裁判にかけられている。	そのおとこはおさないしょうねんをさつじんしたかどてさいばんにかけられている 
\\	散々小言をいってやる。	さんざんこごとをいってやる 
\\	彼のうるさい小言に彼女は悩まされた。	かれのうるさいこごとにかれはなやまされた 
\\	お父さんがよく遅くまで仕事をするので、お母さんが小言をたくさん言う。	おとうさんがよくおそくまでしごとをするので、おかあさんがこごとをたくさんいう 
\\	その男は私にお前は誰かと尋ねたが、その質問に対しては私は答える必要はないと思った。	そのおとこはわたしにおまえはだれかとたずねたが、そのしつもんにたいしてはわたしはこたえるひつようはないとおもった 
\\	散々考えた挙げ句その計画を実行にうつした。	さんざんかんがえたあげくそのけいかくをじっこうに移した 
\\	彼は考えを実行に移すのが遅い。	かれはかんがえをじっこうにうつすのがおそい 
\\	私たちは是非ともこの計画を実行に移さなければならない。	わたしたちはぜひともこのけいかくをじっこうにうつさなければならない 
\\	包丁とか鍋とか台所用品を持参すること。	ほうちょうとかなべとかだいどころようひんをじさんすること 
\\	私の妻は、包丁を使っているときに、指を切った。	わたしのつまは、ほうちょうをつかっているとき、ゆびをきった 
\\	ほとんどの観光客がカメラを持参している。	ほとんどのかんこうきゃくがカメラをじさんしている 
\\	今週は弁当を持参する。	こんしゅうはべんとうをじさんする 
\\	火山の噴火に続いて飢饉が訪れた。	かざんのふんかにつづいてききんがおとずれた 
\\	その火山は突然噴火し、多くの人が亡くなった。	そのかざんはとつぜんふんかしおおくのひとがなくなった 
\\	彼女の厚化粧が嫌らしい。	かのじょのあつげしょうがいやらしい 
\\	まったく・・・付き合いが長くなってくると、どいつもこいつもお前に毒される。	まったく・・・つきあいかながくなってくると、どいつもこいつもおまえにどくされる 
\\	電車でポケットの中にあるものをすられた。	でんしゃでポケットのなかにあるものをすられた 
\\	その町の子供たちは安全のため連れて行かれた。	そのまちのこどもたちはあんぜんのためつれていかれた 
\\	笑われるのには慣れている。	わらわれるのにはなれている 
\\	漫画の中で政治家は愛される存在というより、笑われる存在として描かれることが多い。	まんがのなかでせいじかはあいされるそんざいというより、わらわれるそんざいとしてかかれることがおおい 
\\	彼女は自分の肖像画を描かせた。	かのじょはじぶんのしょうぞうがをかかせた 
\\	キャンバスのオイルはとても微細な花びらを描けない。	キャンバスのオイルはとてもびさいなはなびらをかけない 
\\	バラの棘で刺されるよりは、いらくさの棘で刺された方がましだ。	バラのとげでさされるよりは、いらくさのとげでさされたほうがましだ 
\\	私は指に棘を刺した。	わたしはゆびにとげをさした 
\\	彼の言葉には棘があった。	かれのことばにはとげがあった 
\\	その腫れから判断すると、その人はきっと蛇に噛まれたに違いありません。	そのはれからはんだんすると、そのひとはきっとへびにかまれたにちがいありません 
\\	我々は愛国心を今日の青年の心に植え付けなければならない。	われわれはあいこくしんをきょうのせいねんのこころにうえつけなければならない 
\\	我々はその畑に今年は綿を植え付けるつもりだ。	われわれはそのはたけにことしはめんをうえつけるつもりだ 
\\	国家主義と愛国心とを混同してはならない。	こっかしゅぎとあいこくしんとをこんどうしてはならない 
\\	公私を混同してはいけない。	こうしをこんどうしてはいけない 
\\	彼女はときに空想と現実を混同することがある。	かれはときにくうそうとげんじつをこんどうすることがある 
\\	孤独と孤立を混同してはいけない。それぞれ異なったものである。	こどくとこりつをこんどうしてはいけない。それぞれことなったものである 
\\	公私のけじめを付けなければいけません。	こうしのけじめをつけなければいけない 
\\	空想の力で、宇宙旅行も出来る。	くうそうのちからで、うちゅうりょこうもできる 
\\	不活発な子供は空想の世界に閉じ込める傾向がずっと高い。	ふかっぱつなこどもはくうそうのせかいにとじこめるけいこうがずっとたかい 
\\	彼女は書斎に閉じこもって思い切り泣いたのである。	かのじょはしょさいにとじこもっておもいきりないたのである 
\\	家内の愚痴を聞くのはもううんざりしています。	かないのぐちをきくのはもううんざりしています 
\\	家内のパートの仕事で少々余分な金が入る。	かないのパートのしごとでしょうしょうよぶんなかねがはいる 
\\	私は彼の演説にうんざりした。	わたしはかれのえんぜつにうんざりした 
\\	毎日の単調な生活にはうんざりだ。	まいにちのたんちょうなせいかつはうんざりだ 
\\	芸術は我々の生活の単調さを破ってくれる。	げいじゅつはわれわれのせいかつのたんちょうさをやぶってくれる 
\\	不平不満を治す薬はない。	ふへいふまんをなおすくすりはない 
\\	私達は君たちの不平不満にはうんざりしている。	わたしたちはきみたちのふへいふまんにはうんざりしている 
\\	俺も後輩にアドバイスする歳になったか。	おれもこうはいにアドバイスするとしになったか 
\\	彼は私より3年後輩です。	かれはわたしよりさんねんこうはいです 
\\	彼女は先輩を追い越して昇進した。	かのじょはせんぱいをおいこしてしょうしんした 
\\	我々は彼をよき先輩として尊敬している。	われわれはかれをよきせんぱいとしてそんけいしている 
\\	一年先輩だからって、そんなに威張らなくていいじゃない。	いちねんせんぱいだからって、そんなにひっぱらなくていいじゃない 
\\	もうこれからは俺を粗末には扱えないぞ。	もうこれからはおれをそまつにはあつかえないぞ 
\\	社会保障は軽々しく扱える問題ではない。	しゃかいほしょうはかるがるしくあつかえるもんだいではない 
\\	戦争は軽々しくするものではないし、憲法を改正するのも「戦争したくてしょうがない」わけではない。	"せんそうはかるがるしくするものではないし、けんぽうをかいせいするのも「せんそうしたくてしようがない」わけではない 
\\	法律が改正された。	ほうりつがかいせいされた 
\\	彼は規則の改正を唱えている。	かれはきそくのかいせいをとなえている 
\\	彼らは税法の改正を支持している。	かれらはぜいほうのかいせいをしじしている 
\\	平等は憲法で守られている。	びょうどうはけんぽうでまもられている 
\\	平等は憲法で保障されている。	びょうどうはけんぽうでほしょうされている 
\\	憲法を犯してはならない。	けんぽうをおかしてはならない 
\\	彼の冗談はクラス全員を爆笑させた。	かれのじょうだんはクラスぜんいんをばくしょうさせた 
\\	非常に上手い洒落だったので、満場爆笑した。	ひじょうにうまいしゃれだったので、まんじょうばくしょうした 
\\	私たちはその教授が昔から言っている洒落にうんざりしている。	わたしたちはそのきょうじゅがむかしからいっているしゃれにうんざりしてうる 
\\	彼のばかげた提案が満場一致で承認された。	かれのばかげたていあんがまんじょういっちでしょうにんされた 
\\	僕は異常な物音を聞いた。	ぼくはいじょうなものおとをきいた 
\\	彼女の異常な行動が私たちの疑いを引き起こした。	かのじょのいじょうなぎょうぎがわたしたちのうたがいをひきおこした 
\\	彼の仕事は外国の買い手と交渉することだ。	かれのしごとはがいこくのかいてとこうしょうすることだ 
\\	有望な買い手は、契約内容をよく理解できませんでした。	ゆうぼうなかいては、けいやくないようをよくりかいできませんでした 
\\	彼は前途有望な青年だ。	かれはぜんとゆうぼうなせいねんだ 
\\	彼には洋々たる前途があった。	かれはようようたるぜんとがあった 
\\	彼の前途は君が思うほど有望ではない。	かれのぜんとはきみがおもうほどゆうぼうではない 
\\	彼は郵便局から三軒目に住んでいる。	かれはゆうびんきょくからさんげんめにすんでいる 
\\	彼らは男女が平等なのは当然の事だと思っている。	かれらはだんじょがびょうどうなのはとうぜんのことだとおもっている 
\\	今や男女の賃金を平等にするだけでなく、家事の責任も平等にすべき時である。	いまやだんじょのちんぎんをびょうどうにするだけでなく、かじのせきにんもびょうどうにすべきときである 
\\	我が国の青年男女は政治に無関心だ。	わがくにのせいねんだんじょはせいじにむかんしんだ 
\\	一体だれが私の名簿をめちゃめちゃにしたのだ。	いったいだれがわたしのめいぼをめちゃめちゃにしたのだ 
\\	彼の名前は名簿には載っていない。	かれのなまえはめいぼにはのっていない 
\\	彼らはなぜ乗客名簿を作らなかったのだろう。	かれらはなぜじょうきゃくめいぼをつくらなかったのだろう 
\\	彼はその決定に不服を唱えた。	かれはそのけっていにふふくをとなえた 
\\	その提案に口を揃えて反対を唱えた。	そのていあんにくちをそろえてはんたいをとなえた 
\\	彼は異議を唱える誰に対しても腹を立てた。	かれはいぎをとなえるだれにたいしてもはらをたてた 
\\	出席者の側には異議はなかった。	しゅっせきしゃのがわにはいぎはなかった 
\\	彼女の提案に異議を唱えるとは君も大胆だ。	かのじょのていあんにいぎをとなえるとはきみもだいたんだ 
\\	彼は民衆の支持を得ようと努めていた。	かれはみんしゅうのしじをえようとつとめてい 
\\	後者の見解を支持する人が日本には多い。	こうしゃのけんかいをしじするひとがにほんにはおおい 
\\	彼女は女性の権利擁護の熱心な支持者である。	かのじょはじょせいのけんりようごのねっしんなしじしゃである 
\\	綿密に言うと、彼の見解は私のとはいくらか異なる。	めんみつにいうと、かれのけんかいはわたしのとはいくらかことなる 
\\	彼の書斎は公園に面している。	かれのしょさいはこうえんにめんしている 
\\	その詩人は自分の書斎で自殺した。	そのしじんはじぶんのしょさいでじさつした 
\\	彼女は僕の右足を思い切り蹴りつけた。	かのじょはぼくのみぎあしをおもいきりけりつけた 
\\	彼らは彼を殺しはしなかった。ただ警告の意味で殴ったり蹴ったりした。	かれらはかれをころしはしなかった。ただけいこくのいみでなぐったりけったりした 
\\	君がとても失礼だから、殴ってやりたいよ。	きみがとてもしつれいだから、なぐってやりたいよ 
\\	海辺は子供たちが遊ぶのに理想的な場所だ。	うみべはこどもたちがあそぶのにりそうてきなばしょだ 
\\	ニュービジネスアイディアの開発は誰もが望む理想的なことだろう。	ニュービジネスアイディアのかいはつはだれもがのぞむりそうてきなことだろう 
\\	彼は理想的な夫であることが分かった。	かれはりそうてきなおっとであることがわかった 
\\	「邪魔をしてはいけないよ。彼女は今、仕事中なのだから。」と彼は小声で言った。	"「じゃまをしてはいけないよ。かのじょはいま、しごとちゅうなのだから。」とかれはこごえでいった。 
\\	俺たちみんな黄色い潜水艦で暮らしている。	おれたちみんなきいろせんすいかんでくらしている 
\\	俺には何のコネもないから、就職するのが大変だ。	おれにはなんのコネもないから、しゅうしょくするのがたいへんだ 
\\	俺を絶望のなかに置き去りにする。	おれをぜつぼうのなかにおきざりにする 
\\	私を置き去りにしていかないでくれ。	わたしをおきざりにしていかないでくれ 
\\	戦況は絶望的だ。	せんきょうはぜつぼうてきだ 
\\	戦況は我々に有利に展開している。	せんきょうはわれわれにゆうりにてんかいしている 
\\	物語の筋はある島を舞台に展開する。	ものがたりのすじはあるしまをぶたいにてんかいする 
\\	彼女は事実に基づいて議論を展開する。	かのじょはじじつにもとづいてぎろんをてんかいする 
\\	それは異常な展開を見せた出来事でした。	それはいじょうなてんかいをみせてできごとでした 
\\	彼は絶望して帰宅した。	かれはぜつぼうしてきたくした 
\\	母親の顔に絶望の色がありありと見えた。	ははおやのかおにぜつぼうのいろがありありとみえた 
\\	少年は春の日差しを浴びて仰向けに寝ていた。	しょうねんははるのひざしをあびてあおむけにねていた 
\\	診察台に仰向けになってください。	しんさつだいにあおむけになってください 
\\	強い日差しで地面が乾いた。	つよいひざしでじめんがかわいた 
\\	外に出ると強い日差しにカッと照らされた。	そとにでるとつよいひざしにカッとてらされた 
\\	彼女その瞬間にカッとなったんだね。	かのじょそのしゅんかんにカッとなったんだね 
\\	彼は勝利の瞬間を待ちわびた。	かれはしょうりのしゅんかんをまちわびた 
\\	私は春の到来を待ちわびている。	わたしははるのとうらいをまちわびている 
\\	広場はライトで赤々と照らされている。	ひろばはライトであかあかとてらされている 
\\	広場には数百人の人がいた。	ひろばにはすうひゃくにんのひとがいた 
\\	暗い部屋の中を懐中電灯で照らした。	くらいへやのなかをかいちゅうでんとうでてらした 
\\	檻の中で飼われると子どもを産まない動物もいる。	おりのなかでかわれるとおどもをうまないどうぶつもいる 
\\	その彫刻家は木で仏像を刻んだ。	そのちょうこくかはきでぶつぞうをきざんだ 
\\	旅行代理店に問い合わせてみよう。	りょこうだいりてんにといあわせてみよう 
\\	私は旅行代理店の人とチケットの値段を交渉した。	わたしはりょこうだいりてんのひととチケットのねだんをこうしょうした 
\\	緊急の時は私の代理人に連絡をとってください。	きんきゅうのときはわたしのだいりにんにれんらくをとってください 
\\	彼は父親の代理をした。	かれはちちおやのだいりをした 
\\	観光案内所に問い合わせてください。	かんこうあんないしょにといあわせてください 
\\	飛行機が定刻に到着するかどうか問い合わせた。	ひこうきがていこくにとうちゃくするかどうかといあわせた 
\\	ご兄弟のことはカンザス州から問い合わせを受けたばかりです。	ごきょうだいのことはカンザスしゅうからといあわせをうけたばかりです 
\\	私は小銭の持ち合わせがない。	わたしはこぜにのまちあわせがない 
\\	大規模な道路工事が始まった。	だいきぼなどうろこうじがはじまった 
\\	日本の漫画は大規模なブームを起こした。	にほんのまんがはだいきぼなブームをおこした 
\\	ジョンは大規模な住宅計画をやり遂げた。	ジョンはだいきぼなじゅうたくけいかくをやりとげた 
\\	この国は気候が温暖だ。	このくにはきこうがおんだんだ 
\\	私は地球の温暖化傾向を心配している。	わたしはちきゅうのおんだんかけいこうをしんぱいしている 
\\	地球温暖化は世界規模での天候の傾向を変えるであろう。	ちきゅうおんだんかはせかいきぼでのてんこうのけいこうをかえるであろう 
\\	地球温暖化のために、アラスカではそれが溶け始めている地域もある。	ちきゅうおんだんかのために、アラスカではそれはとけはじめているちいきもある 
\\	第二次世界大戦後われわれはアメリカ化されるようになった。	だいにじせかいたいせんごわれわれはアメリカかされるようになった 
\\	事態はますます悪化した。	じたいはますますあっかした 
\\	長旅で彼女の傷は悪化した。	ながたびでかのじょのきずはあっかした 
\\	情勢は日増しに悪化している。	じょうせいはひましにあっかしている 
\\	自然環境の悪化を阻止する。	しぜんかんきょうのあっかをそしする 
\\	日増しに涼しくなっていく。	ひましにすずしくなっていく 
\\	インフレを阻止しなければならない。	インフレをそししなければならない 
\\	クーデター計画はぎりぎりのところで阻止されました。	クーデターけいかくはぎりぎりのところでそしされました 
\\	ダーウィンの進化論を学びましたか。	ダーウィンのしんかろんをまなびましたか 
\\	コンピューターは急速な進化を遂げた。	コンピューターはきゅうそくなしんかをとげた 
\\	私は恐竜の進化にとても興味を持っている。	わたしはきょうりゅうのしんかにとてもきょうみをもっている 
\\	私達はダーウィンという名前を聞くと進化論を連想する。	わたしたちはダーウィンというなまえをきくとしんかろんをれんそうする 
\\	リンカーンと言えば自由を連想する。	リンカーンといえばじゆうをれんそうする 
\\	その言葉には不愉快な連想がある。	そのことばにはふゆかいなれんそうがある 
\\	彼らは大雨といえば洪水を連想した。	かれらはおおあめといえばこうずいをれんそうした 
\\	これらの食べ物は、さまざまな民族の集団を連想させる。	これらのたべものは、さまざまなみんぞくのしゅうだんをれんそうさせる 
\\	離婚と言うと悲観的なイメージを連想しがちである。	りこんというとひかんてきなイメージをれんそうしかちである 
\\	日本人の民族的特性は何だと思いますか。	にほんじんのみんぞくてきとくせいはなんだとおもいますか 
\\	我々は少数民族の権利を守らなければならない。	われわれはしょうすうみんぞくのけんりをまもらなければならない 
\\	少数意見を尊重すべきだ。	しょうすういけんをそんちょうすべきだ 
\\	タンカーには少数の乗組員しかいない。	タンカーにはしょうすうののりくみいんしかいない 
\\	約束を破ることほど人を不愉快にするものはない。	やくそくをやぶることほどひとをふゆかいにするものはない 
\\	彼は騒がしい子供を不愉快そうな顔でちらっと見た。	かれはさわがしいこどもをふゆかいそうなかおでちらっとみた 
\\	車で通りかかったときに、有名な女優の家をチラッと見ました。	くるまでとおりかかったときに、ゆうめいなじょゆうのいえをチラッとみました 
\\	新聞によるとXYZオイルは今日倒産したらしい。	しんぶんによるとXYZオイルはきょうとうさんしたらしい 
\\	彼の概算によると家の新築費用は3000万円です。	かれのがいさんによるといえのしんちくひようはさんぜんまんえんです 
\\	この国の人口は概算5千万に達する。	このくにのじんこうはがいさんごせんまんにたっする 
\\	彼の財産は1億ドルと概算された。	かれのざいさんはいちおくドルとがいさんされた 
\\	テレビのニュースによると、インドで飛行機の墜落事故があったそうだ。	テレビのニュースによると、インドでひこうきのついらくじこがあったそうだ 
\\	先月の雨の日に対する晴れの日の割合は4対1だった。	せんげつのあめのひにかいするはれのひのわりあいはよんたいいちだった 
\\	その少女たちは体育館で踊った。	そのしょうじょはたいいくかんでおどった 
\\	私たちの学校の体育館は今建設中です。	わたしたちのがっこうのたいいくかんはいまけんせつちゅうです 
\\	その学校は最新の体育器具を備えている。	そのがっこうはさいしんのたいいくきぐをそなえている 
\\	これは主婦の手間を省く便利な器具です。	これはしゅふのてまをはぶくべんりなきぐです 
\\	機械は多くの手間を省く。	きかいはおおくのてまをはぶく 
\\	役に立たないものは省きなさい。	やくにたたないものははぶきなさい 
\\	彼女の台所は手間を省ける装置が装備されている。	かのじょのだいどころはてまをはぶけるそうちがそうびされている 
\\	その船にはレーダーが装備されていた。	そのふねはレーダーがそうびされている 
\\	私たちはスキーの装備を整えた。	わたしたちはスキーのそうびをととのえた 
\\	紙面が足りなくてこの問題を省かなければならなかった。	しめんがたりなくてこのもんだいをはぶかなければならない 
\\	時間が短いので、プログラムからスピーチの一部を省かなければならない。	じかんがみじかいので、プログラムからスピーチのいちぶをはぶかなければならない 
\\	私は文を書く前に頭の中で整えることにしている。	わたしはぶんをかくまえにあたまのなかでととのえることにしている 
\\	その病院は24時間体制を整えている。	そのびょういんはにじゅうよんじかんたいせいをととのえている 
\\	墜落に向けて用意は整えられた。	ついらくにむけてよういはととのえられた 
\\	共産主義はソ連で実践された体制である。	きょうさんしゅぎはソれんでじっせんされたたいせいである 
\\	爆破予告以来、空港の警備体制は強化された。	ばくはよこくいらい、くうこうのけいびたいせいはきょうかされた 
\\	私は1999年に向けて、市場強化に専念します。	わたしはせんきゅうひゃくきゅうねんにむけて、しじょうきょうかにせんねんします 
\\	輸出市場での競争力強化が緊急の課題である。	ゆしゅつしじょうでのきょうそうりょくきょうかがきんきゅうのかだいである 
\\	我々は軍事力を強化すべきだと、大統領は言っている。	われわれはぐんじりょくをきょうかすべきだと、だいとうりょうはいっている 
\\	多く学生が課題を出し損なった。	おおくのがくせいがかだいをだしそこなった 
\\	ふははは!やっとこの課題も終ったぜ!さぁ次の課題でもやるべぇか。	ふははは!やっとこのかだいもおわったぜ!さぁつぎのかだいでもやるべぇか 
\\	"「みろ、お前のお陰でフラれまくりだぞ」「そう?日頃の行いのせいじゃない?」 
\\	"「みろ、おまえのおかげでフラれまくりだぞ」「そう?ひごろのおこないのせいじゃない?」 
\\	おい、クレオ。あんまりうろちょろするなー?まだ入園したばっかりなんだぞ。	おい、クレオ。あんまりうろちょろするなー?まだにゅうえんしたばっかりなんだぞ 
\\	電車の中を子供がウロチョロしていて不愉快だった。	でんしゃのなかをこどもがウロチョロしていれふゆかいだった 
\\	そろそろ散髪してもいい頃だぞ。	そろそろさんぱつしてもいいころだぞ 
\\	そろそろ現実を直視していい頃だ。	そろそろげんじつをちょくししていいころだ 
\\	これは私に現在の悩みを直視して立ち向かう事を可能にしてくれる。	これはわたしにげんざいのなやみをちょくししてたちむかうことをかのうにしている 
\\	現実と幻想を区別するのは難しい。	げんじつとげんそうをくべつするのはむずかしい 
\\	恒久的な平和など幻想に過ぎない。	こうきゅうてきなへいわなどげんそうにすぎない 
\\	真夜中の太陽は幻想的な自然現象の一つだ。	まよなかのたいようはげんそうてきしぜんげんしょうのひとつだ 
\\	彼には事実と虚構の区別がつかない。	かれにはじじつときょこうのくべつがつかない 
\\	事実と虚構を見分けなければならない。	じじつときょこうをみわけなければなれない 
\\	理論と実践は相伴うとは限らない。	りろんとじっせんはあいともなうとはかぎらない 
\\	実践的見地からすれば、彼の計画には欠点が多くある。	じっせんてきけんちからすれば、かれのけいかくにはけってんがおおくある 
\\	あなたが結婚し子供を持ったら、言葉より実践ということを悟るでしょう。	あなたはけっこんしこどもをもったら、ことばよりじっせんということをさとるでしょう 
\\	間違いを悟るのに彼はほんの数分かかっただけだ。	まちがいをさとるのにかれはほんのすうふんかっかただけだ 
\\	少し考えれば君は自分が間違っていることを悟るだろう。	すこしかんがえればきみはじぶんがまちがっていることをさとるだろう 
\\	電気は数分後にまたついた。	でんきはすうふんごにまたついた 
\\	彼女は私がそれとなく言った意味を悟って微笑んだ。	かのじょはわたしがそれとなくいったいみをさとってほほえんだ 
\\	わたしは彼に彼の誤りを悟らせる事ができなかった。	わたしはかれにかれのあやまりをさとらせることができなかった 
\\	私には葉っぱと林檎しかありません。	わたしははっぱとりんごしかありません 
\\	この時計喉から手が出るほど欲しかったんだ。	このとけいのどからてがでるほどほしかったんだ 
\\	商店街に入ると、陽菜はまるでおのぼりさんのようにキョロキョロ辺りを見回した。	しょうてんがいにはいると、はるなはまるでおのぼりさんのようにキョロキョロあたりをみまわした 
\\	僕はポストを探して辺りを見回した。	ぼくはポストをさがしてあたりをみまわした 
\\	メースン博士は仕事第一だった。	メースンはかせはしごとだいいちだった 
\\	ホワイト博士が我々の通訳をして下さった。	ホワイトはかせがわれわれのつうやくをしてくださった 
\\	彼らは私を博士という肩書きで呼んだ。	かれらはわたしをはかせというかたがきでよんだ 
\\	彼は心理学博士の学位を持っている。	かれはしんりがくはかせのがくいをもっている 
\\	彼女は3年前に修士の学位をとりました。	かのじょはさんねんまえにしゅうしのがくいをとりました 
\\	彼は法学修士の学位を得た。	かれはほうがくしゅうしのがくいをえた 
\\	女性は男性より長生きだと言われている。	じょせいはだんせいよりながいきだといわれている 
\\	言語は人間固有の性質である。	げんごはにんげんこゆうのせいしつである 
\\	生存本能はあらゆる生物に固有のものである。	せいぞんほんのうはあらゆるせいぶつにこゆうのものである 
\\	この瞬間は歴史に記録されるだろう。	このしゅんかんはれきしにきろくされるだろう 
\\	彼の記録は誰も破れない。	かれのきろくはだれもやぶれない 
\\	その記録は一般に公開されていない。	そのきろくはいっぱんにこうかいされていない 
\\	彼は今年水泳で3つの世界記録を立てた。	かれはことしすいえいでみっつのせかいきろくをたてた 
\\	先ごろあなたの人事記録に誤りを発見しました。	さきごろあなたのじんじきろくにあやまりをはっけんしました 
\\	岡村さんは人事課長です。	おかむらさんはじんじかちょうです 
\\	履歴書はこの封筒に入れて人事部に提出して下さい。	りれきしょはこのふうとうにいれてじんじぶにていしゅつしてください 
\\	父はフランス語の履歴書を日本語に翻訳した。	ちちはフランスごのりれきしょをにほんごにほんやくした 
\\	その瞬間、私は現実感を失った。	そのしゅんかん、わたしはげんじつかんをうしなった 
\\	私は転んだ瞬間に手首を折ったことが分かった。	わたしはころんだしゅんかんにてくびをおったことがわかった 
\\	肩書きが偉くても地位が高いとは限らない。	かたがきがえらくてもちいがたかいとはかぎらない 
\\	それは暴力に対する文明の勝利であった。	それはぼうりょくにたいするぶんめいのしょうりであった 
\\	あいつ、絶対ペーパータオル使わないんだぜ。地球に優しいやつってことだよな。	あいつ、ぜったいペーパータオルつかわないんだぜ。ちきゅうにやさしいやつってことだよな 
\\	彼女は少なくとも週一回美容院へ行く。	かのじょはすくなくともしゅういっかいびよういんへいく 
\\	彼女は歯並びが悪い。	かのじょははならびがわるい 
\\	彼女は綺麗な歯並びだ。	かのじょはきれいなはならびだ 
\\	歯並びをきちんと直していただきたいのですが。	はならびをきちんとなおしていただきたいのですが 
\\	その部族は年中砂漠に住んでいる。	そのぶぞくはねんじゅうさばくにすんでいる 
\\	彼女は今が青春の盛りだ。	かのじょはいませいしゅんのさかりだ 
\\	私達の会議中に彼は自分の青春について言及した。	わたしたちのかいわにかれはじぶんのせいしゅんについてげんきゅうした 
\\	彼は私の著書に言及した。	かれはわたしのちょしょにげんきゅうした 
\\	彼女は演説の中で第2次世界大戦に言及した。	かのじょはえんぜつのなかでだいにじせかいたいせんにげんきゅうした 
\\	スピーチの中で、彼は企業の強さについて言及した。	スピーチのなかで、かれはきぎょうのつよさについてげんきゅうした 
\\	彼は物理学の著書を出版した。	かれはぶつりがくのちょしょをしゅっぱんした 
\\	私たちは彼の著書の発行を期待している。	わたしたちはかれのちょしょのはっこうをきたいしている 
\\	この地方新聞は週に1回発行される。	このちほうしんぶんはしゅうにいっかいはっこうされる 
\\	この週刊誌は毎週1回発行される。	このしゅうかんしはまいしゅういっかいはっこうされる 
\\	彼はその週刊誌が、結局は旨く行くものと確信していた。	かれはそのしゅうかんしが、けっきょくはうまくいくものとかくしんしている 
\\	イーストはビールを発酵させる。	イーストはビールをはっこうさせる 
\\	私はこの辺の地理に弱い。	わたしはこのへんのちりによわい 
\\	彼は東京の地理に精通している。	かれはとうきょうのちりにせいつうしている 
\\	溶岩を地理学的に説明してくれませんか。	ようがんをちりがくてきにせつめいしてくれませんか 
\\	その男はある乾燥した国で水が飲めずに死んだ。	そのおとこはあるかんそうしたくにでみずがのめずにしんだ 
\\	その猟師は森の中深く入り込んだが、二度と帰らなかった。	そのりょうしはもりのなかふかくはいりこんだが、にどとかえらなかった 
\\	彼はいまを盛りにと言う時に倒れた。	かれはいまをさかりにというときにたおれた 
\\	言わずもがなだ。	いわずもがなだ 
\\	恋と戦争は手段を選ばず。	あいとせんそうはしゅだんをえらばず 
\\	一日ひとくそ、医者要らず。	いちにちひとくそ、いしゃいらず 
\\	1日にリンゴ1個で医者要らず。	いちにちにリンゴいっこでいしゃいらず 
\\	忘れずにその手紙を投稿しなさい。	わすれずにそのてがみをとうこうしなさい 
\\	忘れずに安全ベルトを閉めなさい。	わすれずにあんぜんベルトをしめなさい 
\\	彼女は死ぬまで初恋の事を忘れずにいた。	かのじょはしぬまではつこいのことをわすれずにいた 
\\	日本で車を運転するときには左側通行を忘れずに。	にほんでくるまをうんてんするときにはひだりがわつうこうをわすれずに 
\\	群集が通行を妨げた。	ぐんしゅうがつうこうをさまたげた 
\\	彼らは通行人にパンフレットを配った。	かれらはつうこうにんにパンフレットをくばった 
\\	道路は洪水のために通行禁止となった。	どうろはこうずいのためにつうこうきんしとなった 
\\	仕事には十分気を配りなさい。	しごとにはじゅうぶんきをくばりなさい 
\\	食べ物と毛布が難民に配られました。	たべものともうふがなんみんにくばられました 
\\	兵隊は暗闇で四方に目を配った。	へいたいはくらやみでしほうにめをくばった 
\\	彼らは難民救済の資金を集めている。	かれらはなんみんきゅうさいのしきんをあつめている 
\\	救助された難民は自由を求めていた。	きゅうじょされたなんみんはじゆうをもとめていた 
\\	彼らは難民たちのために日本語の授業を設立した。	かれらはなんみんたちのためににほごのじゅぎょうをせつりつした 
\\	その老人は貧民救済に多額の金を寄付した。	そのろうじんはひんみんきゅうさいにたがくのかねをきふした 
\\	難民達の苦しみを救済すべきだ。	なんみんたちのくるしみをきゅうさいすべきだ 
\\	新しい毛布が貧民に配られた。	あたらしいもうふがひんみんにくばられた 
\\	この間投稿した記事がやっと雑誌に載ったんだよ。頑張って出し続けた甲斐があったよ。	このあいだとうこうしたきじがやっとざっしにのったんだよ。がんばってだしつづけたかいがあったよ 
\\	あなたのお陰で私は生き甲斐を感じます。	あなたのおかげでわたしはいきがいをかんじます 
\\	この車は直しても甲斐が無い。	このくるまはなおしてもかいがない 
\\	長生きの秘訣は、生き甲斐を持つ事だそうだ。	ながいきのひけつは、いきがいをもつことだそうだ 
\\	甲斐甲斐しく働く。	かいがいしくはたらく 
\\	おもちゃコンサルタントの方々が甲斐甲斐しく働く姿がとても印象的でした。	おもちゃコンサルタントのかたがたがかいがいしくはたらくすがたがとてもいんしょうてきでした 
\\	彼は一生独身で通した。	かれはいっしょうどくしんでとおした 
\\	冬中暖房無しで通した。	ふゆちゅうだんぼうなしでとおした 
\\	彼は理想的な紳士である。	かれはりそうてきなしんしである 
\\	彼は自分の理想を具体化したいと思っている。	かれはじぶんのりそうをぐたいかしたいとおもっている 
\\	彼の本には未来の世界への理想が込められている。	かれはほんにはみらいのせかいへのりそうがこめられている 
\\	不満はありません。私に関する限り、万事理想的です。	ふまんはありません。わたしにかんするかぎり、ばんりりそうてきです 
\\	彼は声の限りに叫んだ。	かれはこえのかぎりにさけんだ 
\\	彼らは声の限りに罵りあった。	かれらはこえのかぎりにののしりあった 
\\	私の力が及ぶ限り援助します。	わたしのちからがおよぶかぎりえんじょします 
\\	人間の力は自然に及ばない。	にんげんのちからはしぜんにおよばない 
\\	外交の駆け引きでは、とても彼には及ばない。	がいこうのかけひきでは、とてもかれにはおよばない 
\\	流れ星が空を駆けた。	ながれぼしがそらをかけた 
\\	夏休みの計画が具体化してきた。	なつやすみのけいかくがぐたいかしてきた 
\\	夫の年収は10万ドルだ。	おっとのねんしゅうはじゅうまんドルだ 
\\	彼は少なくとも年収2000万円は稼いでる。	かれはすくなくともねんしゅうにせんまんえんはかせいでる 
\\	彼はホームラン3本を打ち、8点を稼いだ。	かれはホームランさんぼんをうち、はってんをかせいだ 
\\	占い師に手相を見せました。	うらないしにてそうをみせました 
\\	何事が起ころうとも、君を応援するよ。	なにごとがおころうとも、きみをおうえんするよ 
\\	彼らは皆その候補者を応援した。	かれらはみんなそのこうほしゃをおうえんした 
\\	応援しているサッカーチームが負け続けているので、苛々する。	おうえんしているサッカーチームがまけつづけているので、いらいらする 
\\	待てば待つほど、私たちは苛々してきた。	まてばまつほど、わたしはいらいらしてきた 
\\	今日の格好はとっても粋ですね。	きょうのかっこうはとてもいきですね 
\\	あの飛行機は大変不格好だ。	あのひこうきはたいへんふかっこうだ 
\\	冬場に暖かい格好をしていないと結局ひどい風邪をひく羽目になる。	ふゆばにあたたかいかっこうをしていないとけっきょくひどいかぜをひくはめになる 
\\	失業者は常にどん底におちる羽目になります。	しつぎょうしゃはつねにどんぞこにおちるはねになります 
\\	彼女は機嫌が悪いというのも、いつも地下鉄に乗り遅れ仕事場まで歩く羽目になったからだ。	かのじょはきげんがわるいというのも、いつもちかてつにのりおくれしごとばまであるくはめになったからだ 
\\	彼は不幸のどん底にあった。	かれはふこうのどんぞこにあった 
\\	彼女は幸福の絶頂から不幸のどん底へ突き落とされた。	かのじょはこうふくのぜっちょうからふこうのどんぞこへつきおとされた 
\\	彼女は絶望のどん底にあった。	かのじょはぜつぼうのどんぞこにあった 
\\	その俳優は人気絶頂の時に死んだ。	そのはいゆうはにんきぜっちょうのときしんだ 
\\	ニキタ・フルシチョフは、権力の絶頂期にあった。	ニキタ・フルシチョフは、けんりょくのぜっちょうきにあった 
\\	男は幸運の絶頂にあるときくらい試練の場に立たされている時はない。	おとこはこううんのぜっちょうにあるときぐらいしれんのばにたたされているときはない 
\\	やくざがひろしに試練を課している。	やくざがひろしにしれんをかしている 
\\	彼は厳しい試練を受けた。	かれはきびしいしれんをうけた 
\\	彼の作品は時の試練に耐えて後世に残るだろう。	かれのさくひんはときのしれんにたえてこうせいにのこるだろう 
\\	試練を受けることになったダレン。失敗すれば、死刑!	しれんをうけることになったダレン。しっぱいすれば、しけい! 
\\	我々は技術を後世に伝えなければならない。	われわれはぎじゅつをこうせいにつたえなければなれない 
\\	文化遺産は後世に伝えられる。	ぶんかいさんはこうせいにつたえられる 
\\	私は莫大な遺産を相続した。	わたしはばくだいないさんをそうぞくした 
\\	我々には豊かな歴史的遺産がある。	われわれはゆたかなれきしてきいさんがある 
\\	彼女、小粋な服装してるよね。	かのじょ、こいきなふくそうしてるよね 
\\	彼は経験豊かな教師だ。	かれはけいけんゆたかなきょうしだ 
\\	彼女は才能豊かな画家さ。	かのじょはさいのうゆたかがかさ 
\\	殺人を犯せば死刑です。	さつじんをおかせばしけいです 
\\	賢い人なら息子を遊ばせてはおかないだろう。	かしこいひとならむすこをあそばせてはおかないだろう 
\\	お詫びとして一言言わせて頂きたいと思います。	おわびとしてひとこといわせていただきたいとおもいます 
\\	彼らから感謝の一言もなかった。	かれらからかんしゃのひとこともなかった 
\\	一言で言えば彼は臆病者だ。	ひとことでいえばかれはおくびょうものだ 
\\	彼女は一言も発せられなかった。	かのじょはひとこともはっせられなかった 
\\	専門家に言わせると、登山もスキーも危険なものではないそうだ。	せんもんかにいわせると、とざんもスキーもきけんなものではないそうだ 
\\	先生は私たちにその単語を繰り返して言わせた。	せんせいはわたしたちにそのたんごをくりかえししていわせた 
\\	母は冷蔵庫を買って届けさせた。	はははれいぞうこをかってとどけさせた 
\\	彼はその改革を止めさせようと出来るだけ努力した。	かれはそのかいかくをとめさせようとできるだけどりょくした 
\\	政治活動はほとんどの職場で止めさせられる傾向にある。	せいじかつどうはほとんどのしょくばでとめさせられるけいこうにある 
\\	私たちは彼がこれ以上酒を飲むのを止めさせなければならない。	わたしたちはかれがこれいじょうさけをのむのをとめさせなければならない 
\\	その制服、前カノになんちゃって女子高生プレイさせるために買ってたもんなんてバレたら・・・。	そのせいふく、まえかのになんちゃってじょしこうせいプレイさせるためにかってたもんなんてばれたら・・・ 
\\	殺人はばれるもの。	さつじんはばれるもの 
\\	『ねえ』って、僕を見ながら言われても・・・もしかして、またダブルブッキングですか?	『ねえ』って、ぼくをみながらいわれても・・・もしかして、またダブルブッキングですか? 
\\	もしかしてツンツンしているのは、生理痛なだけとか?いや、まさかね。	もしかしてツンツンしているのは、せいりつうなだけとか?いや、まさかね 
\\	まさかのときには貯蓄すればいい。	まさかのときにはちょちくすればいい 
\\	まさかの時に備えておくのは賢明だ。	まさかのときにそなえておくのはけんめいだ 
\\	まさか彼に会えるとは思わなかった。	まさかかれにあえるとはおもわなかった 
\\	彼女は私たちにとって素晴らしいお手本でした。	かのじょはわたしたちにとってすばらしいおてほんでした 
\\	彼の行儀をお手本にしなさい。	かれのぎょうぎをてほんにしなさい 
\\	残念ながら、多くの子供たちが、そうした自己中心的な大人たちを手本として育っている。	ざんねんながら、おおくのこどもたちが、そうしたじこちゅうしんてきなおとなたちをてほんとしてそだっている 
\\	彼は自己中心的で欲が深い。	かれはじこちゅうしんてきでよくがふかい 
\\	彼は運動不足であまり食欲がない。	かれはうんどうぶそくであまりしょくよくがない 
\\	彼は私を騙してそれを受け取らせた。	かれはわたしをだましてそれをうけとらせた 
\\	彼は疲れた手足を休ませた。	かれはつかれたてあしをやすませた 
\\	蟹を縦に歩かせることはできない。	かにをたてにあるかせることはできない 
\\	1塁があいていたので打者を歩かせた。	いちるいがあいていたのでだしゃをあるかせた 
\\	審判は打者にアウトを宣した。	しんぱんはだしゃにアウトをせんした 
\\	彼は一塁へ投げた。	かれはいちるいへなげた 
\\	走者は三塁でアウトになった。	そうしゃはさんるいでアウトになった 
\\	その走者は堅い筋肉をしている。	そのそうしゃはかたいきんにくをしている 
\\	世界最速の走者でさえ、空腹だったら走れない。	せかいさいそくのそうしゃでさえ、くうふくだったらはしれない 
\\	青空を背景に木々を描く。	あおぞらをはいけいにきぎをかく 
\\	写真の背景にいる男は誰ですか。	しゃしんのはいけいにいるおとこはだれですか 
\\	それは彼の理論的背景を示している。	それはかれのりろんてきはいけいをしめしている 
\\	芸術を理解するのに文化的背景は必要ではない。	げいじゅつをりかいするのにぶんかてきはいけいはひつようではない 
\\	私たちは趣味・教育的背景など、いろいろと共通したものを持っている。	わたしたちはしゅみ・きょういくてきはいけいなど、いろいろときょうつうしたものをもっている 
\\	彼はテレビでその戦争の政治的背景を説明した。	かれはテレビでそのせんそうのせいじてきはいけいをせつめいした 
\\	プールを縦に二回泳いだ。	プールをたてににかいおよいだ 
\\	容積を計算するためには縦と横と深さをかければよい。	ようせきをけいさんするためにはたてとよことふかさをかければよい 
\\	僕の弱点を見つけさせないぞ。	ぼくのじゃくてんをみつけさせないぞ 
\\	彼の弱点が彼の長所を帳消しにしている。	かれのじゃくてんがかれのちょうしょをちょうけしにしている 
\\	負債は帳消しにするしかなかった。	ふさいはちょうけしにするしかなかった 
\\	政府はその負債を支払うと発表した。	せいふはそのふさいをしはらうとはっぴょうした 
\\	クレジットカードの負債は毎月返済しておいた方がいい。	クレジットカードのふさいはまいつきへんさいしておいたほうがいい 
\\	彼は返済能力がないと宣告された。	かれはへんさいのうりょくがないとせんこくされた 
\\	私は父の借金返済を免除された。	わたしはちちのしゃっきんへんさいをめんじょされた 
\\	我々の借金は我々の返済能力を超えている。	われわれのしゃっきんはわれわれのへんさいのうりょくをこえている 
\\	弱点を克服する。	じゃくてんをこくふくする 
\\	彼は誰にも彼の私事に干渉させなかった。	かれはだれにもかれのしじにかんしょうさせなかった 
\\	彼は他人の私事を妨害する。	かれはたにんのしじをぼうがいする 
\\	彼は彼女の私事に立ち入った、プライバシーを侵害した。	かれはかのじょのしじにたちいった、プライバシーをしんがいした 
\\	私は人権侵害に反対だ。	わたしはじんけんしんがいにはんたいだ 
\\	彼は私たちの会話を妨害した。	かれはわたしたちのかいわをぼうがいした 
\\	大国は小国に干渉すべきではない。	たいこくはしょうこくにかんしょうすべきではない 
\\	母はいつも私の私的な生活に干渉してばかりいる。	はははいつもわたしのしてきなせいかつにかんしょうしてばかりいる 
\\	許可なしで他人の私的な手紙を読むべきではない。	きょかなしでたにんのしてきなてがみをよむべきではない 
\\	彼は馬を急がせた。	かれはうまをいそがせた 
\\	来年一週間泊まりにいらっしゃいませんか。	らいねんいっしゅうかんとまりにいらっしゃいませんか 
\\	洒落たお帽子をお召しになっていらっしゃいますね。	おしゃれたおぼうしをおめしになっていらっしゃいますね 
\\	お客様の中にお医者様はいらっしゃいませんか。	おきゃくさまのなかにおいしゃさまはいらっしゃいませんか 
\\	君の都合のよいときに遊びにいらっしゃい。	きみのつごうのよいよきにあそびにいらっしゃい 
\\	冷蔵庫の中の物は、何でも御自由に召し上がってください。	れいぞうこのなかのものは、なんでもごじゆうにめしあがってください 
\\	昨日起こった事について何とおっしゃいましたか。	きのうおこったことについてなんとおっしゃいましたか 
\\	もしあなたが私の立場だったらどうおっしゃいますか。	もしあなたがわたしのたちばだったらどうおっしゃいますか 
\\	私は証言する立場ではない。	わたしはしょうげんするたちばではない 
\\	私はかなり微妙な立場にある。	わたしはかなりびみょうなたちばにある 
\\	私は彼より有利な立場にある。	わたしはかれよりゆうりなたちばにある 
\\	君の立場は十分に理解している。	きみのたちばはじゅうぶんにりかいしている 
\\	私は自分の立場にあまり自信がない。	わたしはじぶんのたちばにあまりじしんがない 
\\	二人の考えには微妙な違いがあった。	ふたりのかんがえにはびみょうなちがいがあった 
\\	交渉はとても微妙な段階に差し掛かっている。	こうしょうはとてもびみょうなだんかいにさしかかっている 
\\	私達が峠に差し掛かる頃に雨になった。	わたしたちがとうげにさしかかるころにあめになった 
\\	病人はもう峠を越した。	病人Zはもうとうげをこした 
\\	ブームは峠を越した。	ブームはとうげをこした 
\\	彼の証言は真相に近い。	かれのしょうげんはしんそうにちかい 
\\	私は彼の潔白を証言することが出来る。	わたしはかれのけっぱくをしょうげんすることができる 
\\	病気はまだ初期の段階です。	びょうきはまだしょきのだんかいです 
\\	通常の睡眠は2つの段階からなる。	つうじょうのすいみんはふたつのだんかいからなる 
\\	我々は戦争の新しい段階に突入しつつある。	われわれはせんそうのあたらしいだんかいにとつにゅうしつつある 
\\	警察がバーに突入した。	けいさつがバーにとつにゅうした 
\\	鉱山労働者が賃上げを要求してストに突入した。	こうざんろうどうしゃがちんあげをようきゅうしてストにとつにゅうした 
\\	組合は5%の賃上げを獲得した。	くみあいは5%のちんあげをかくとくした 
\\	我々は平和を獲得するべきだ。	われわれはへいわをかくとくするべきだ 
\\	鉱山を所有している人から銀を買いました。	こうざんをしょゆうしているひとからぎんをかいました 
\\	その公爵はたくさんの土地を所有している。	そのこうしゃくはたくさんとちをしょゆうしている 
\\	それの代金を払うことによって確立する所有権。	それのだいきんをはらうことによってかくりつするしょゆうけん 
\\	これは、あの殺人的な公爵に復讐しようという彼の願いを強めただけだ。	これは、あのさつじんてきなこうしゃくにふくしゅうしようというかれのおねがいをつよめただけだ 
\\	その鉱山は閉鎖している。	そのこうざんはへいさしている 
\\	空港は霧のために閉鎖された。	くうこうはきりのためにへいさされた 
\\	彼らはその学校の閉鎖について討論した。	かれらはそのがっこうのへいさについてとうろんした 
\\	博物館は資金不足のために閉鎖しなければならなかった。	はくぶつかんはしきんぶそくのためにへいさしなければならなかった 
\\	ガンは第一段階で発見すれば容易に治療できる。	ガンはだいいちだんかいではっけんすればよういにちりょうできる 
\\	初期の人類は世界のあらゆる所に移住した。	しょきのじんるいはせかいのあらゆるところにいじゅうした 
\\	アメリカの歴史の初期には黒人は奴隷として生きていた。	アメリカのれきしのしょきにはこくじんはどれいとしていきていた 
\\	子供の初期の教育は普通家庭で始まる。	こどものしょきのきょういくはふつうかていではじまる 
\\	現代の車は初期の車と多くの点で異なる。	げんだいのくるまはしょきのくるまとおおくのてんでことなる 
\\	このデザインは彼の初期の作品と類似している。	このデザインはかれのしょきのさくひんとるいじしている 
\\	彼らはそこで使われていた道具に類似した道具を使っていた。	かれらはそこでつかわれていたどうぐにるいじしたどうぐをつかっていた 
\\	その二つの実験は類似の結果を出した。	そのふたつのじっけんはるいじのけっかをだした 
\\	彼は通常の料金の2倍払った。	かれはつうじょうのりょうきんのにばいはらった 
\\	通常、私は8時40分に出動する。	つうじょう、わたしははちじよんじゅっぷんにしゅつどうする 
\\	これは、通常の値引きとは異なります。	これは、つうじょうのねびきとはことなります 
\\	ほかに何か追加注文がありますか。	ほかになにかついかちゅうもんがありますか 
\\	時間をおかないと追加・削除が反映されない。	じかんをおかないとついか・さくじょがはんえいされない 
\\	僅かな追加料金で配達サービスが利用できます。	わずかなついかりょうきんではいたつサービスがりようできます 
\\	追加したい項目、削除したい項目がありましたら、6月12日までにご連絡下さい。	ついかしたいこうもく、さくじょしたいこうもくがありましたら、ろくがつじゅうににちまでにごれんらくください 
\\	これら追加された証拠を考慮すると、第2の法則は修正されなければならない。	これらついかされたしょうこをこうりょすると、だいにのほうそくはしゅうせいされなければならない 
\\	普通、私達は大学で自分の考えを修正します。	ふつう、わたしたちはだいがくでじぶんのかんがえをしゅうせいします 
\\	失礼ですが御提案を修正させていただきます。	しつれいですがごていあんをしゅうせいさせていただきます 
\\	合うようにテーブルの高さを修正しなければならない。	あうようにテーブルのたかさをしゅうせいしなければならない 
\\	価格は需要を反映する。	かかくはじゅようをはんえいする 
\\	社会の価値観はその伝統の中に反映されている。	しゃかいのかちかんはそのでんとうのなかにはんえいされている 
\\	彼の演説は党の意向を反映したものではなかった。	かれのえんぜつをとうのいこうをはんえいしたものではなかった 
\\	テレビに映し出されるものは、いわば実社会の反映である。	テレビにうつしだされるものは、いわばじっしゃかいのはんえいである 
\\	価値観の違う人とうまくやっていくのは難しい。	かちかんのちがうひととうまくやっていくのはむずかしい 
\\	特に重要なことは伝統的価値観を厳守することである。	とくにじゅうようなことはでんとうてきかちかんをげんしゅすることである 
\\	彼は時間厳守を誇りに思っている。	かれはじかんげんしゅをほこりにおもっている 
\\	この命令は厳守すべき。	このめいれいはげんしゅすべき 
\\	彼は利口でそのうえ正直で時間を厳守する。	かれはりこうでそのうえしょうじきでじかんをげんしゅする 
\\	彼の意向に逆らうな。	かれのいこうにさからうな 
\\	彼は両親の意向を無視してその少女と結婚した。	かれはりょうしんのいこうをむししてそのしょうじょとけっこんした 
\\	君の論文から不必要な語を削除した方がよろしい。	きみのろんぶんからふひつようなごをさくじょしたほうがよろしい 
\\	パーティーの案内と申込書を間違って削除してしまいました。	パーティーのあんあいともうしこみしょをまちがってさくじょしてしまいました 
\\	この本は何の項目に入りますか。	このほんはなんのこうもくにはいりますか 
\\	このアンケート、項目が多すぎて記入する気になれないよ。	このアンケート、こうもくがおおすぎてきにゅうするきにならないよ 
\\	これらの点は同じ項目にまとめて扱うことができる。	これらのてんはおなじこうもくにまとめてあつかうことができる 
\\	トムはその申込み用紙に記入した。	トムはそのもうしこみようしにきにゅうした 
\\	必ず本人が登録用紙に記入して下さい。	かならずほんにんがとうろくようしにきにゅうしてください 
\\	答案用紙は、月曜日までに提出するように。	とうあん用紙は、げつようびまでにていしゅつするように 
\\	申し込み用紙に糊で写真を貼りなさい。	もうしこみようしにのりでしゃしんをはりなさい 
\\	アンケート用紙が適当に配られた。	アンケートようしがてきとうにくばられた 
\\	彼はメモをドアに糊で貼った。	かれはメモをドアにのりではった 
\\	アンケート用紙が無作為に配布された。	アンケートようしがむさくいにはいふされた 
\\	実験用の被験者は無作為に選ばれた。	じっけんようのひけんしゃはむさくいにえらばれた 
\\	その実験の結果をコントロールするのは被験者の態度である。	そのじっけんのけっかをコントロールするのはひけんしゃのたいどである 
\\	私はその女優本人に話しかけた。	わたしはそのじょゆうほんにんにはなしかけた 
\\	彼はゆっくり頷いて了解の意を示した。	かれはゆっくりうなずいてりょうかいのいをしめした 
\\	我々には互いに支持しあおうという暗黙の了解があった。	われわれはたがいにしじしあおうというあんもくのりょうかいがあった 
\\	二人の間には暗黙の了解が会ったに違いない。	ふたりのあいだにはあんもくのりょうかいがあったにちがいない 
\\	彼が僕に辞任して欲しいと思っているのを、僕は暗黙のうちに理解した。	かれがぼぼくにじにんしてほしいとおもっているのを、ぼくはあんもくのうちにりかいした 
\\	適度な運動をすれば、心身共に爽やかになれますよ。	てきどなうんどうをすれば、しんしんともにさわやかになれますよ 
\\	暑い天候のとき、コップ1杯の冷たい水はとても爽やかだ。	あついてんこうのとき、コップいっばいのつめたいみずはとてもさわやかだ 
\\	青年時代は心身の発達が著しい。	せいねんじだいはしんしんのはったつがいちじるしい 
\\	彼の英語力は著しく向上した。	かれのえいごりょくはいちじるしくこうじょうした 
\\	彼の英語は著しく上達した。	かれのえいごはいちじるしくじょうたつした 
\\	間違いを恐れるような人は英会話は上達しないだろう。	まちがいをおそれるようなひとはえいかいわはじょうたつしないだろう 
\\	改札口で切符をお見せください。	かいさつぐちできっぷをおみせください 
\\	彼女は私に別れを告げて改札口を通って行った。	かのじょはわたしにわかれをつげてかいさつぐちをとおっていった 
\\	彼は私に何をすべきか告げる立場ではない。	かれはわたしになにをすべきかつげるたちばではない 
\\	今私の家の付近に住宅が続々建っている。	いまわたしのふきんにじゅうたくがぞくぞくたっている 
\\	人々が浜辺付近で遊んでいる。	ひとびとがはまべふきんであそんでいる 
\\	ビキニスタイルの美少女は浜辺では目を見張らすものだった。	ビキニスタイルのびしょうじょははまべではめをみはらすものだった 
\\	暴力団を見張るのは危険な冒険だった。	ぼうりょくだんをみはるのはきけんなぼうけんだった 
\\	彼の演奏は目を見張るものであった。	かれのえんそうはめをみはるものであった 
\\	住民の運動で暴力団を町から追放した。	じゅうみんのこうどうでぼうりょくだんをまちからついほうした 
\\	彼は母国を追放された。	かれはぼこくをついほうされた 
\\	私達は国民から麻薬を追放しなければならない。	わたしたちはこくみんからまやくをついほうしなければならない 
\\	彼は原住民との友好関係を確立した。	かれはげんじゅうみんとのゆうこうかんけいをかくりつした 
\\	その島の住民は友好的だ。	そのしまのじゅうみんはゆうこうてきだ 
\\	その会社は競争の激化のあおりを受けて、倒産した。	そのかいしゃはきょうそうのげきかのあおりをうけて、とうさんした 
\\	此処で激戦が行われた。	ここでげきせんがおこなわれた 
\\	先進国では虫歯が激減し、自分の歯で一生食べられる人が増えています。	せんしんこくではむしばがげきげんし、じぶんのはでいっしょうたべられるひとがふえています 
\\	先進国は発展途上国を援助しなければならない。	せんしんこくははってんとじょうこくをえんじょしなければならない 
\\	日本はエレクトロニクスの分野では他の先進国より先んじている。	にほんはエレクトロニクスのぶんやではほかのせんしんこくよりさきんじている 
\\	発展途上国の人口増加は急速だ。	はってんとじょうこくのじんこうぞうかはきゅうそくだ 
\\	財政危機に直面している発展途上国もある。	ざいせいききにちょくめんしているはってんとじょうこくもある 
\\	発展途上国では優れた技術者が不足している。	はってんとじょうこくではすぐれたぎじゅつしゃがふそくしている 
\\	彼の車は会社への途上で、故障した。	かれのくるまはかいしゃへのとじょうで、こしょうした 
\\	米国の豊かさは発展途上国の貧しさと比較対照される。	べいこくのゆたかさははtってんとじょうこくのまずしさとひかくたいしょうされる 
\\	都会生活と田園生活との対照的な相違。	とかいせいかつとでんえんせいかつとのたいしょうてきなそうい 
\\	彼の本は田園生活の話から始まる。	かれのほんはでんえんせいかつのはなしからまじまる 
\\	都市化が田園生活をどんどん侵食しています。	としかがでんえんせいかつをどんどんしんしょくしている 
\\	錆が金属の部分を少しずつ侵食している。	さびがきんぞくのぶぶんをすこしずつしんしょくしている 
\\	金属は冷やされると縮小する。	きんぞくはひやされるとしゅくしょうする 
\\	経済は厳しい不況で縮小した。	けいざいはきびしいふきょうでしゅくしょうした 
\\	国内市場の縮小はインフレに依るものです。	こくないしじょうのしゅくしょうはインフレによるものです 
\\	彼は財政的に困っている。	かれはざいせいてきにこまっている 
\\	彼らはその画家を財政的に援助した。	かれらはがかをざいせいてきにえんじょした 
\\	和室が好きな英国人もいると言われます。	わしつがすきなえいこくじんもいるといわれます 
\\	その眠っている赤ちゃんを御覧なさい。	そのねむっているあかちゃんをごらんなさい 
\\	あなたの前にある看板をご覧なさい。	あなたのまえにあるかんばんをごらんなさい 
\\	残念ながら昨日はご覧の通りの結果となりました。	ざんねんながらきのうはごらんのとおりのけっかとなりました 
\\	私どもの製品についての詳細な情報は、インターネット上のhttp://www
\\	でご覧になれます。	わたしどもせいひんについてのしょうさいなじょうほうは、インターネットじょうのhttp://www
\\	でごらんになれます 
\\	元旦には近所の神社にお参りする人が多い。	がんたんにはきんじょのじんじゃにおまいりするひとがおおい 
\\	彼は自分でその会議に出られないので、代わりに私が参ります。	かれはじぶんでそのかいぎにでられないので、かわりにわたしがまいります 
\\	ご親切に恐縮しております。	ごしんせつにきょうしゅくしております 
\\	ご用は承っておりますか。	ごようはうけたまわっております 
\\	ご伝言を承ります。	ごでんごんをうけたまわります 
\\	展示会へのご招待、本日ありがたく承りました。	てんじかいへのごしょうたい、ほんじつありがたくうけたまわりました 
\\	どんなご意見でもありがたく承ります。	どんなごいけんでもありがたくうけたまわります 
\\	店主は私に「御用は承っておりますか」と尋ねた。	"てんしゅはわたしに「ごようはうけたまわっておりますか」とたずねた 
\\	店主は私にそれを買うようしきりに勧めた。	てんしゅはわたしにそれをかうようしきりにすすめた 
\\	医者は転地を勧めた。	いしゃはてんちをすすめた 
\\	米国の親の中には、息子を麻薬に近づけないためにフットボールを勧めるものが多い。	べいこくのおやのなかには、むすこをまやくにちかづけないためにフットボールをすすめるのがおおい 
\\	君たちのうち誰かこの方の御用を伺っているか。	きみたちのうちだれかこのかたのごようをうかがっているか 
\\	計画はそっくり承認された。	けいかくはそっくりしょうにんされた 
\\	ご依頼の件、承知しました。	ごいらいのけん、しょうちしました 
\\	弁護士はたくさんの依頼人を持っている。	べんごしはたくさんのいらいにんをもっている 
\\	弁護士は依頼人のために説得力のある発言をした。	べんごしはいらいにんのためにせっとくりょくのあるはつげんをした 
\\	御依頼により本をお送りしました。	ごいらいによりほんをおおくりしました 
\\	彼の依頼は命令に等しかった。	かれのいらいはめいれいにひとしかった 
\\	私は彼の要求を承諾するだろう。	わたしはかれのようきゅうをしょうだくするだろう 
\\	父は私が外国へ行くことを承知した。	ちちはわたしががいこくへいくことをしょうちした 
\\	ご承知のように、彼は野球が好きだ。	ごしょうちのように、かれはやきゅうがすきだ 
\\	沈黙は承諾を意味する事が多い。	ちんもくはしょうだくをいみすることがおおい 
\\	日本はその国の新しい政府を承認した。	にほんはそのくにの新しいせいふをしょうにんした 
\\	恐縮ですがお手伝い願えませんか。	きょうしゅくですがおてつだいねがえませんか 
\\	恐縮だが、残業してもらわないと。	きょうしゅくだが、ざんぎょうしてもらわないと 
\\	良い御返事をおまちしております。	よいごへんじをおまちしております 
\\	ご助力にとても感謝しております。	ごじょりょくにとてもかんしゃしております 
\\	私達が成功するには君の助力が不可欠だ。	わたしたちがせいこうするにはきみのじょりょくがふかけつだ 
\\	水は生命に不可欠です。	みずはせいめいにふかけつです 
\\	幸福には健康が不可欠です。	こうふくにはけんこうがふかけつです 
\\	政治家には洞察力が不可欠である。	せいじかにはどうさつりょくがふかけつである 
\\	教師にとって忍耐力は不可欠だ。	きょうしにとってにんたいりょくはふかけつだ 
\\	彼は自分の洞察力という利益を彼らに与えてやった。	かれはじぶんのどうさつりょくというりえきをかれらにあたえてやった 
\\	彼は人間の心理に対する深い洞察力を持っている。	かれはにんげんのしんりにたいするふかいどうさつりょくをもっている 
\\	あの国では個性が重視される。	あのくにではこせいがじゅうしされる 
\\	彼は口語英語を非常に重視した。	かれはこうごえいごをひじょうにじゅうしした 
\\	今度の上司?個性的どころか、ありがちなタイプね。	こんどのじょうし?こせいてきどころか、ありがちなタイプね 
\\	各人が個性的であればあるほど、それだけ他人の英知に寄与する。	かくじんがこせいてきであればあるほど、それだけたにんのえいちにきよする 
\\	首相の演説は平和のために寄与するところがなかった。	しゅしょうのえんぜつはへいわのためにきよするところがなかった 
\\	軍縮は必ずや平和に寄与する。	ぐんしゅくはかならずやへいわにきよする 
\\	我々は軍縮に関して彼らと意見が一致したいと望んでいる。	われわれはぐんしゅくにかんしてかれらといけんがいっちしたいとのぞんでいる 
\\	口語体で書かれた本。	こうごたいでかかれたほん 
\\	私は口語英語に一層興味があります。	わたしはこうごえいごにいっそうきょうみがあります 
\\	彼は年をとるにつれて、一層謙虚になった。	かれはとしをとるにつれて、いっそうけんきょになった 
\\	あなたは本を読むたびに一層向上するだろう。	あなたはほんをよむたびにいっそうどうじょうするだろう 
\\	その謙虚な男は近所の人達とうまくやっている。	そのけんきょなおとこはきんじょのひとたちとうまくやっている 
\\	彼は、社会的地位が上がるにつれて、ますます謙虚になった。	かれはしゃかいてきちいがあがるにつれて、ますますけんきょになった 
\\	クリスはとても魅力的だし、お金も持っています。でも、少しばかり謙虚さが足りません。	クリスはとてもみりょくてきだし、おかねもっています。でも、すこしばかりけんきょさがたりません 
\\	彼の甥は謙虚で思いやりのある人になるように育てられた。	かれのおいはけんきょでおもいやりのあるひとになるようにそだてられた 
\\	あいつは思いやりがない。	あいつはおもいやりがない 
\\	彼は弱者に深い思いやりがある。	かれはじゃくしゃにふかいおもいやりがある 
\\	彼は重大な使命を任された。	かれはじゅうだいなしめいをまかされた 
\\	彼は極秘の使命を帯びていた。	かれはごくひのしめいをおびていた 
\\	兵士たちの使命はその橋を破壊することだった。	へいしたちのしめいはそのはしをはかいすることだった 
\\	彼女は極秘で私にその話をした。	かのじょはごくひでわたしにそのはなしをした 
\\	「君はそれを終えたか」「とんでもない、始めたばかりだ」	"「きみはそれをおえたか」「とんでもない、はじめたばかりだ」 
\\	パターソン博士:とんでもない。ココがゴリラは利口で手話を覚えられることを私たちに教えてくれました。	パターソンはかせ:とんでもない。ココがゴリラはりこうでしゅわをおぼえられることをわたしにおしえてくれました 
\\	耳の不自由な人は手話で会話が出きる。	みみのふじゆうなひとはしゅわでかいわができる 
\\	ライトの実物を検討させて頂きたいと思います。	ライトのじつぶつをけんとうさせていただきたいとおもいます 
\\	お蔭様で助かりました。	おかげさまでたすかりました 
\\	既に手に負えない状態だ。	すでにてにおえないじょうたいだ 
\\	彼女は既にお金を受け取っていた様子だった。	かのじょはすでにおかねをとっていたようすだった 
\\	大丈夫、キミなら出来る!自分を信じて!キミはもう既に立派なスイマーなのよ!	だいじょうぶ、キミならできる!じぶんをしんじて!キミはもうすでにりっぱなスイマーなのよ! 
\\	既に知っていることを証明してくれるものしか受け入れたがらない傾向が私たちにはあるのである。	すでにしっていることをしょうめいしてくれるものしかうけいれたがらないけいこうがわたしたちにはあるのである 
\\	質問が詳細に討議された。	しつもんがしょうさいにとうぎされた 
\\	調査が詳細を明らかにするだろう。	ちょうさがしょうさいをあきらかにするだろう 
\\	専門家がその統計を詳細に分析した。	せんもんかがそのとうけいをしょうさいにぶんせきした 
\\	統計はすべてを物語るとは限らない。	とうけいはすべてをものがたるとはかぎらない 
\\	統計は我々の生活水準が向上したことを示している。	とうけいはわれわれのせいかつすいじゅんがこうじょうしたことをしめしている 
\\	この統計ではエジプトはアフリカの国に分類されている。	このとうけいではエジプトはアフリカのくににぶんるいされている 
\\	これらの標本はいくつかの種類に分類される。	これらのひょうほんはいくつかのしゅるいにぶんるいされる 
\\	あなたの祖先をご存知ですか。	あなたのそせんをごぞんじですか 
\\	ご存知のように、東京は世界の金融の中心地だ。	ごぞんじのように、とうきょうはせかいのきんゆうのちゅうしんちだ 
\\	国際金融で仕事を続けるつもりです。	こくさいきんゆうでしごとをつづけるつもりです 
\\	一般的に、消費者は質よりも量を選ぶ。	いっぱんてきに、しょうひしゃはしつよりもりょうをえらぶ 
\\	多くの消費者は消費税反対に立ち上がった。	おおくのしょうひしゃはしょうひぜいはんたいにたちあがった 
\\	彼は大量の酒を消費する。	かれはたいりょうのさけをしょうひする 
\\	山田さんと言えば、あの人の息子さんはどうなったか御存知ですか。	やまださんといえば、あのひとのむすこさんはどうなったかごぞんじですか 
\\	免許証を拝見できますか。	めんきょしょうをはいけんできますか 
\\	貧しい老父が王様に拝見を許された。	まずしいろうふがおうさまにはいけんをゆるされた 
\\	搭乗券を拝見します。	とうじょうけんをはいけんします 
\\	観光客たちは全員搭乗しましたか。	かんこうきゃくたちはぜんいんとうじょうしましたか 
\\	日本航空124便の搭乗ゲートはどこですか。	にっぽんこうくう124びんのとうじょうゲートはどこですか 
\\	航空便でお誕生日プレゼントを送ります。	こうくうびんでおたんじょうびプレゼントをおくります 
\\	その航空専門家は統計を詳細に分析した。	そのこうくうせんもんかはとうけいをしょうさいにぶんせきした 
\\	選手は試合の規則を断固守らなければならない。	せんしゅはしあいのきそくをだんこまもらなければならない 
\\	警察は違法駐車の取り締まりを始めた。	けいさつはいほうちゅうしゃのとりしまりをはじめた 
\\	暴力を取り締まるのは、政府の責任である。	ぼうりょくをとりしまりのは、せいふのせきにんである 
\\	彼は自分の違法行為を恥じていない。	かれはじぶんのいほうこういをはじていない 
\\	彼は飲酒運転で罰せられた。	かれはいんしゅうんてんでばっせられた 
\\	市は飲酒運転を非常に厳しく取り締まって、罰金を課している。	しはいんしゅうんてんをひじょうにきびしくとりしまって、ばっきんをかしている 
\\	知能の高い人で飲酒した後で運転するような人はいない。	ちのうのたかいひとでいんしゅしたあとでうんてんするようなひとはいない 
\\	人間は知能を発達させた。	にんげんはちのうをはったつさせた 
\\	時間を最大限に活用するようにしなさい。	じかんをさいだいげんにかつようするようにしなさい 
\\	蔵書があるということと、それを活用することとは別のことだ。	ぞうしょがあるということと、それをかつようすることとはべつのことだ 
\\	彼の蔵書量は僕の3倍だ。	かれのぞうしょりょはぼくのさんばいだ 
\\	彼女の蔵書は全部で3500冊で多くの初版が入ってる。	かのじょのぞうしょはぜんぶでさんぜんごひゃくさつでおおくのしょはんがはいっている 
\\	初版は10年前に出版された。	しょはんはじゅうねんまえにしゅっぱんされた 
\\	弊社の製品と業務内容についてご説明させていただきます。	へいしゃのせいひんとぎょうむないようについてごせつめいさせていただきます 
\\	弊社は機械パーツの輸入を行っています。	へいしゃはきかいパーツのゆにゅうをおこなっています 
\\	日曜日は郵便業務がありますか。	にちようびにゆうびんぎょうむがありますか 
\\	業務を拡張して食料品を少し売ることに決めた。	ぎょうむをかくちょうしてしょくりょうひんをすこしうることにきめた 
\\	学校の体育館が拡張された。	がっこうのたいいくかんがかくちょうされた 
\\	彼らは征服によって、領土を拡張した。	かれらはせいふくによって、りょうどをかくちょうした 
\\	軍は全領土を支配するのに成功した。	ぐんはぜんりょうどをしはいするのにせいこうした 
\\	彼らは帝国の領土を広げた。	かれらはていこくのりょうどをひろげた 
\\	人類は科学と技術で自然を征服したいと願っている。	じんるいはかがくとぎじゅつでしぜんをせいふくしたいとねがっている 
\\	町は征服されて、彼は追い出された。	まちはせいふくされて、かれはおいだされた 
\\	よい映画は人の視野を広げる。	よいえいがはひとのしやをひろげる 
\\	最も強大な帝国ですら崩壊する。	もっともきょうだいなていこくですらほうかいする 
\\	インカ帝国の政府はすべての物を管理していた。	インカていこくのせいふはすべてのものをかんりしていた 
\\	ご要望の通りにいたします。	ごようぼうのとおりにいたします 
\\	ご要望にお応えできずに申しわけありません。	ごようぼうにおこたえできずにもうしわけありません 
\\	あなたは御両親の期待に応えるべきだ。	あなたはごりょうしんのきたいにこたえるべきだ 
\\	地方の需要に応えるのに十分な製品の在庫がある。	ちほうのじゅうようにこたえるのにじゅうぶんなせいひんのざいこがある 
\\	その本は在庫切れになっております。	そのほんはざいこきれになっております 
\\	あいにくお尋ねの商品は現在、在庫がありません。	あいにくおたずねのしょうひんはげんざい、ざいこがありません 
\\	あいにくその映画を見逃してしまった。	あいにくそのえいがをみのがしてしまった 
\\	注意深く見ないと、ジェーンのお母さんを見逃してしまいます。	ちゅういぶかくみないと、ジェーンのおかあさんをみのがしてしまいます 
\\	今度は見逃してやるが、またお前を盗みの現行犯で捕らえたくないもんだね。	こんどはみのがしてやるが、またおまえをぬすみのげんこうはんでとらえたくないもんだね 
\\	警察はすりを現行犯で取り押さえた。	けいさつはすりをげんこうはんでとりおさえた 
\\	本格的な休暇は3年ぶりだ。	ほんかくてきなきゅうかはさんねんぶりだ 
\\	勿論学生時代に英語は勉強したけど、本格的に始めたのはここ2、3年くらい前です。	もちろんがくせいじだいにえいごはべんきょうしてけど、ほんかくてきにはじめたのはここに、さんねんくらいまえです 
\\	彼は毎日の単調な生活に塞ぎ込んでいるようだ。	かれはまいにちのたんちょうなせいかつにふさぎこんでいるようだ 
\\	彼女は急にふさぎ込む性癖がある。	かのじょはきゅうにふさぎこむせいへきがある 
\\	彼の変わった性癖が彼らを困らせた。	かれのかわったせいへきがかれらをこまらせた 
\\	桜の花が真っ盛りです。	さくらのはながまっさかりです 
\\	石には何か書かれていた。彼らは何が書いているかを判読しようとした。	いしにはなにかかかれていた。かれらはなにがかいているかをはんどくしようとした 
\\	収賄スキャンダルは海外で激しい反発を生みました。	しゅうわいスキャンダルはかいがいできびしいはんぱつをうみました 
\\	収賄事件は政府に疑惑を投げかけた。	しゅうわいじけんはせいふにぎわくをなげかけた 
\\	若い者達は本来の親の考えに対して反発するものだ。	わかいものたちはほんらいのおやのかんがえにたいしてはんぱつするものだ 
\\	彼は反発したいという気になったが、思い直してやめた。	かれははんぱつしたいというきになったが、おもいなおしてやめた 
\\	彼は疑惑に悩まされた。	かれはぎわくになやまされた 
\\	この種の事件はよく疑惑を生む。	このたねのじけんはよくぎわくをうむ 
\\	彼の言葉は疑惑を引き起こした。	かれのことばはぎわくをひきおこした 
\\	彼は軍職を免職になった。	かれはぐんしょくをめんしょくになった 
\\	その女優は舞台に3度登場した。	そのじょゆうはぶたいにさんどとうじょうした 
\\	彼の到着は歓声の声で迎えられた。	かれのとうちゃくはかんせいのこえでむかえられた 
\\	彼女の動作はすべて優美だ。	かのじょのどうさはすべてゆうびだ 
\\	彼女は私たちみんな驚かせるほど優美に踊った。	かのじょはわたしたちみんなおどろかせるほどゆうびにおどった 
\\	私達は、春の夜、月光を浴びた桜の優美さに心を打たれる。	わたしたちは、はるのよる、げっこうをあびたさくらのゆうびさにこころをうたれる 
\\	ベスは彼の全く新しい服装に心打たれ、満足しました。	べスはかれのまったくあたらしいふくそうにこころうたれ、まんぞくしました 
\\	不意を打たれて、私は返事に困った。	ふいをうたれて、わたしはへんじにこまった 
\\	雉も鳴かずば打たれまい。	きじもなかずばうたれまい 
\\	彼は動作が鈍い。	かれのどうさがにぶい  
\\	過去の動作については過去形を用います。	"かこのどうさについてはかこけいをもちいます 
\\	雀は動作が非常にすばしこい。	すずめはどうさがひじょうにすばしこい 
\\	私はその点を強調した。	わたしはそのてんをきょうちょうした 
\\	彼は平和の大切さを強調した。	かれはへいわのたいせつさをきょうちょうした 
\\	人種的な誇りを強調するガーヴェイ。	じんしゅてきなほこりをきょうちょうするガーヴェイ。 
\\	彼都市生活の面で便利な面を強調した。	かれとしせいかつのつらでべんりなつらをきょうちょうした 
\\	私たちは彼の協力の重要性を強調した。	わたしたちはかれのきょうりょくのじゅうようせいをきょうちょうした 
\\	奇抜な考えが心に浮かんだ。	きばつなかんがえがこころにうかんだ 
\\	彼は、奇抜なアイデアを出すため、もっと保守的な同僚と何度も揉め事を起こしている。	かれは、きばつなアイデアをだすため、もっとほしゅてきなどうりょうとなんどももめごとをおこしている 
\\	私は君をその揉め事に巻き込ませたくない。	わたしはきみをそのもめごとにまきこませたくない 
\\	交通渋滞に巻き込まれた。	こうつうじゅうたいにまきこまれた 
\\	彼らはその紛争に巻き込まれたくないと思った。	かれらはそのふんそうにまきこまれたくないとおもった 
\\	高速道路は何千もの車で渋滞した。	こうそくどうろはなんぜんものくるまでじゅうたいした 
\\	私は交通渋滞で遅れたと主人に説明した。	わたしはこうつうじゅうたいでおくれたとしゅじんにせつめいした 
\\	ああ、そうそう! 滅多にっていうか、ほとんど使用したことがないコンドームを使いました。	ああ、そうそう!めったにっていうか、ほとんどしよしたことがないコンドームをつかいました 
\\	彼の機嫌がいいことなど滅多にない。	かれはきげんがいいことなどめったにない 
\\	道路で有名な音楽家に偶然会うなんて滅多にないことだ。	どうろでゆうめいなおんがくかにぐうぜんあうなんてめったにないことだ 
\\	つまみ食いと言うか、完食していたように見えたが。	つまみぐいというか。かんしょくしていたようにみえたが 
\\	彼はケーキをつまみ食いするのを見られた。	かれはケーキをつまみぐいするのをみられた 
\\	なによ!出来ないの?この度胸なし!腰抜けッ!	なによ!できないの?このどきょうなし!こしぬけッ! 
\\	私この度一身上の都合でやめさせていただきます。	わたしこのたびいっしんじょうのつごうででやめさせていただきます 
\\	度胸のない自分に我ながらひどく腹が立った。	どきょうのないじぶんにわれながらひどくはらがたった 
\\	また同じ失敗をするなんて、我ながら愚かだと思う。	またおなじしっぱいをするなんて、われながらおろかだともう 
\\	彼は単なる愚か者でしかない。	かれはたんなるおろかものでしかない 
\\	私は愚かにも定期券を家に置き忘れた。	わたしはおろかにもていきけんをいえにおきわすれた 
\\	愚か者の金はすぐにその手を離れる。	おろかもののかねはすぐにそのてをはなれる 
\\	この取り決めは一時的なものでしかない。	このとりきめはいちじてきなものでしかない 
\\	それは部分的な成功でしかなかった。	それはぶぶんてきなせいこうでしかなかった 
\\	彼はまだ未成年でしかない。	れはまだみせいねんでしかない 
\\	それは単なる手落ちだ。	それはたんなるておちだ 
\\	この寛大な申し出は単なる見せかけかもしれない。	このかんだいなもうしではたんなるみせかけかもしれない 
\\	彼女の謝罪はただの見せかけだった。	かのじょのしゃざいはただのみせかけだった 
\\	彼は実力のある政治家に見せかけているが、実際にはずるがしこい政治屋だ。	あれはじつりょくのあるせいじかにみせかけているが、じっさいにはずるがしこいせいじやだ 
\\	政治屋は次の選挙のことを考えるが、政治家は次の世代のことを考える。	せいじやはつぎのせんきょのことをかんがえるが、せいじかはつぎのせだいのことをかんがえる 
\\	私の怒りを静めるのは心からの謝罪しかない。	わたしのいかりをしずめるのはこころからのしゃざいしかない 
\\	彼は私たちの信頼を裏切ったことを謝罪しなかった。	かれはわたしたちのしんらいをうらぎったことをしゃざいしなかった 
\\	未成年者はたばこを吸うのを禁じられている。	みせいねんしゃはたばこをすうのをきんじられている 
\\	その壁は部分的に蔦で覆われている。	そのかべはぶぶんてきにつたでおおわれている 
\\	その林檎は一時的に私の空腹を満たしてくれた。	そのリンゴはいちじてきにわたしのくうふくをみたしてくれた 
\\	その映画監督は自分の権力を使って、映画界に特別な場を設けました。	そのえいがかんとくはじぶんのけんりょくをつかって、えいがかいにとくべつなばをもうけました 
\\	彼女は彼を責めるどころか、彼に褒美をあげた。	かのじょはかれをせめるどころか、かれにほうびをあげた 
\\	自然界には褒美もなければ罰もない。	しぜんかいにはほうびもなければばちもない 
\\	私は、鳩たちがその箱の上に登った褒美に餌を与えた。	わたしは、はとたちがそのはこのうえにのぼったほうびにえさをあたえた 
\\	二度の延期の後、その神聖な儀式は執り行われた。	にどのえんきのあと、そのしんせいなぎしきはとりおこなわれた 
\\	神聖な儀式がその荘厳な寺院で執り行われた。	しんせいなぎしきがそのそうごんなじいんでとりおこなわれた 
\\	その寺院は丘の天辺にある。	そのじいんはおかのてっぺんにある 
\\	何かが神聖であるとするならば、人体こそ神聖である。	なにかがしんせいであるとするならば、じんたいこそしんせいである 
\\	人体は無数の細胞からなっている。	じんたいはむすうのさいぼうからなっている 
\\	この地域社会に住む一人一人は、健康な人間が持つ免疫機構の細胞のようなもだ。	このちいきしゃかいにすむひとりひとりは、けんこうなにんげんがもつめんえききこうのさいぼうのようなもだ 
\\	科学のクラスで私達は細胞の絵を書きます。	かがくのクラスでわたしたちはさいぼうのえをかきます 
\\	どこか機構が悪いに違いない。	どこかきこうがわるいにちがいない 
\\	これこそ事故現場に乗り捨ててあった単車だ。	これこそじこげんばにのりすててあったたんしゃだ 
\\	彼女には欠点があるからこそ僕は一層彼女が好きだ。	かのじょにはけってんがあるからこそぼくはいっそうかのじょがすきだ 
\\	私は天然痘に免疫になっている。	わたしはてんねんとうにめんえきになっている 
\\	夜食にインスタントラーメンを食べた。	やしょくにインスタントラーメンをたべた 
\\	子供は反抗期になる傾向がある。	こどもははんこうきになるけいこうがある 
\\	この店の店長は髭が生えています。	このみせのてんちょうはひげがはえています 
\\	長靴に黴が生えた。	ながぐつにかびがはえた 
\\	トムは口髭を伸ばしている。	トムはくちひげをのばしている 
\\	黴臭い味がする。	かびくさいあじがする 
\\	どうして髪を乾かしているの。	どうしてかみをかわかしているの 
\\	船は無事に目的地に着いた。	ふねはぶじにもくてきちについた 
\\	私は村人たちに別れを告げて次の目的地へと向かった。	わたしはむらびとたちにわかれをつげてつぎのもくてきちへむかった 
\\	往復切符の料金は?	おうふくきっぷのりょうきんは? 
\\	ある19歳のカナダ人が先月、イギリス海峡をノンストップで往復泳ぎ切って世界記録を破った。	あるじゅうきゅうさいのカナダじんがせんげつ、イギリスかいきょうをノンストップおうふくおよぎきってせかいきろくをやぶった 
\\	彼らは王室からの訪問者に贈り物を幾つか渡した。	かれらはおうしつからのほうもんしゃにおくりものをいくつかわたした 
\\	イギリスは、長い間私が訪問したいと思っていた国です。	イギリスは、ながいかんわたしがほうもんしたいとおもっていたくにです 
\\	私たちが訪問したときは、小説に半年間も取り組んでいたんですから。	わたしたちがほうもんしたときは、しょうせつにはんとしかんおとりくんでいたんですから 
\\	王室の家族は皇居に住んでいる。	おうしつんのかぞくはこうきょにすんでいる 
\\	北欧では冬の訪れが早い。	ほくおうではふゆのおとずれがはやい 
\\	率直に言えば、彼は愛国者というよりはむしろ偽善者だ。	そっちょくにいえば、かれはあいこくしゃというよりはむしろぎぜんしゃだ 
\\	彼らは彼の批判を偽善であるとして退けた。	かれらはかれのひはんをぎぜんであるとしてどけた 
\\	私たちは政治家というと偽善を連想しがちだ。	わたしたちはせいじかというとぎぜんをれんそうしがしだ 
\\	彼女の提案は退けられたようだ。	かのじょのていあんはどけられたようだ 
\\	私はその抗議を退けた。	わたしのここうぎをどけた 
\\	彼はその難しい行為をごく簡単にやって退けた。	かれはこのむずかしいこういをごくかんたんにやってどけた 
\\	おまえもいい加減ウジウジしてないで、決断しなさい!	おまえもいいかげんウジウジしてないで、けつだんしなさい! 
\\	煮え加減が丁度よい。	にえかげんがちょうどよい 
\\	スープが鍋でぐつぐつ煮えている。	スープがなべでぐつぐつにえている 
\\	住民たちは騒音に対して苦情を訴えた。	じゅうみんたちはそうおんにたいしてくじょうをうったえた 
\\	顧客からの苦情が増えるのは業績悪化の始まりかもしれない。	こきゃくからのくじょうがふえるのはぎょうせきあっかのはじまりかもしれない 
\\	日本の企業業績は改善した。	にほんのきぎょうぎょうせきはかいぜんした 
\\	彼女は自分の業績に謙虚である。	かのじょはじぶんのぎょうせきにけんきょである 
\\	俺は人事課の馬鹿野郎にあごで指図されるつもりはない。	おれはじんじかのばかやろうにあごでさしずされるつもりはない 
\\	時計が人間の行動を指図する。	とけいがにんげんのこうどうをさしずする 
\\	あなたほどの才能の持ち主が世間に知られずにいるのは惜しいことです。	あなたほどのさいのうのもちぬしがせけんにしられずにいるのはおしいことです 
\\	哲学者は世間とあまり交渉を持たない傾向がある。	てつがくしゃはせけんとあまりこうしょうをもたないけいこうがある 
\\	世間知らずのその男は、恥ずかしさで顔を赤らめた。	せけんしらずのそのおとこは、はずかしさでかおをあからめた 
\\	彼は通りで女の子達が彼に口笛を吹いた時顔を赤らめた。	かれはとおりでおんなのこたちがかれにくちぶえをふいたときかおをあからめた 
\\	流石、彼は期待を裏切らないね。	さすが、かれはきたいをうらきらないね 
\\	流石に偉大な学者だけあって、彼はその問いに容易に答えた。	さすがにいだいながくしゃだけあって、かれはそのといによういにこたえた 
\\	序文で著者は次のように述べている。	じょぶんでちょしゃはつぎのようにのべている 
\\	著者と出版社の名前を記載します。	ちょしゃとしゅっぱんしゃのなまえをきさいします 
\\	著者は自分の本に多くの挿し絵を入れた。	ちょしゃはじぶんのほんにおおくのさしえをいれた 
\\	この著者は美しい文体を持っている。	このちょしゃはうつくしいぶんたいをもっている 
\\	著者の許可なしに本をコピーすることは違法です。	ちょしゃのきょかなしにほんをコピーすることはいほうです 
\\	他に記載すべき情報があれば教えてください。	ほかにきさいすべきじょうほうがあればおしえてください 
\\	カナダには木を切るのは違法とされる地域が沢山ある。	カナダにはほんをきるのはいほうとされるちいきがたくさんある 
\\	この本は挿絵を除いて252ページある。	このほんはさしえをのぞいてにひゃくごじゅうにページある 
\\	それら三通の手紙の文体を比較しなさい。	それらさんつうのてがみのぶんたいをひかくしなさい 
\\	彼はその戯曲に素晴らしい序文を書いた。	かれはそのぎきょくにすばらしいじょぶんをかいた 
\\	「ハムレット」はこれまでで最もおもしろい戯曲だと言われている。	"「ハムレット」はこれまででもっともおもしろいぎきょくだといわれている 
\\	このドラマの登場人物はすべて架空のものです。	このドラマのとうじょうじんぶつはすべてかくうのものです 
\\	君はネッシーなんて架空の存在だと言うが、僕はいると思うよ。	きみはネッシーなんてかくうのそんざいだというが、ぼくはいるとおもうよ 
\\	あなたは本当に幽霊を見たのではない。それは架空のものでしかなかったのだ。	あなたはほんとうにゆうれいをみたのではない。それはかくうのものでしかなかったのだ 
\\	この物語の中で書かれているすべての出来事は架空のものです。	このものがたりのなかでかかれているすべてのできごとはかくうのものです 
\\	その書評者はその小説に鋭く批判的である。	そのしょひょうしゃはそのしょうせつにするどくひはんてきである 
\\	彼は出鱈目に発砲した。	かれはでたらめにはっぽうした 
\\	彼は本棚に本を出鱈目に入れた。	かれはほんだなにほんをでたらめにいれた 
\\	彼は出鱈目な約束をして彼女を誘惑した。	かれはでたらめなやくそくをしてかのじょをゆうわくした 
\\	住宅不足は深刻だ。	じゅうたくぶそくはしんこくだ 
\\	深刻な伝染病が北京で発生した。	しんこくなでんせんびょうがぺきんではっせいした 
\\	我が国の海岸の汚染はひどく深刻な状態である。	わがくにのかいがんのおせんはひどくしんこくなじょうたいである 
\\	円高で日本経済に対する影響が深刻になってきている。	えんだかでにほんけいざいにたいするえいきょうがしんこくになってきている 
\\	彼は雁に向かって発砲した。	かれはがんにむかってはっぽうした 
\\	敵がこちらに近づいたら発砲するんだぞ。	てきはこちらにちかづいたらはっぽうするだぞ 
\\	そもそも何故鳥は移動するのだろうか。	そもそもなぜとりはいどうするのだろうか 
\\	この本は若い読者にも理解できる。	このほんはわかいどくしゃにもりかいできる 
\\	ディッケンズの考えが、ロンドンの下町の生き生きした描写と共に読者には伝わる。	ディッケンズのかんがえが、ロンドンのしたまちのいきいきしたびょうしゃにはつたわる 
\\	ある本が読者の興味を引かないからといって、その原因がその本にあるということにはならない。	あるほんがどくしゃのきょうみをひかないからといって、そのげんいんがそのほんにあるということにはならない 
\\	その不動産屋は銀座で気前よく金を使った。	そのふどうさんやはぎんざできまえよくかねをつかった 
\\	不動産の価格が日本では異常なほどに高騰してきた。	ふどうさんのかかくがにほんではいじょうなほどにこうとうしてきた 
\\	市の中心部の地価が高騰している。	しおのちゅうしんぶのちかがこうとうしている 
\\	物価の高騰が家計を圧迫している。	ぶっかのこうとうがかけいをあっぱくしている 
\\	胃に圧迫感があります。	いにあっぱくかんがあります 
\\	その政府は国民を圧迫した。	そのせいふはこくみんをあっぱくした 
\\	私は家計を妻に任せた。	わたしはかけいをつまにまかせた 
\\	地価は依然として落ち着く気配を見せない。	ちかはいぜんとしておちつくけはいをみせない 
\\	さらに困ったことに、恐ろしい嵐の来る気配があった。	さらにこまったことと、おそろしいあらしのくるけはいがあった 
\\	気前の良いその歯科医はおよそ20億円を慈善事業に寄付した。	きまえのよいそのしかいはおよそにじゅうおくえんをじぜんじぎょううにきふした 
\\	彼は自分の金をすべて慈善施設に寄付した。	かれはじぶんのかねをすべてじぜんしせつにきふした 
\\	その施設は爆破によって廃虚になった。	そのしせつはばくはによってはいきょになった 
\\	会員はだれでもこれらの施設を利用できる。	かいいんはだれでもこれらのしせつをりようできる 
\\	この施設内で塵の投げ捨てをすると、最高500ドルの罰金を課せられることがあります。	このしせつないでゴミのなげすてをすると、さいこうごひゃくドルのばっきんをかせられることがあります 
\\	実に高額の所得には特別税が課せられている。	じつにこうかくのしょとくにはとくべつぜいがかせられている 
\\	此処からあがる所得は無税である。	ここからあがるしょとくはむぜいである 
\\	彼は自分の真意を漏らしてしまった。	かれはじぶんのしんいをもらしてしまった 
\\	彼は敵対者にさえ新しい経済計画に同意させた。	かれはてきたいしゃにさえあたらしいけいざいけいかくにどういさせた 
\\	初期の宗教指導者の中には敵対する人々に迫害されたものもいる。	しょきのしゅうきょうしどうしゃのなかはてきたいするひとびとにはくがいされたものもいる 
\\	ローマ人はキリスト教徒を迫害した。	ローマじんはキリストきょうとをはくがいした 
\\	牛はヒンズー教徒には神聖なものである。	うしはヒンズーきょうとにはしんせいなものである 
\\	これらの商品は無税です。	これらのしょうひんはむぜいです 
\\	彼が現れると気まずい沈黙が広がった。	かれがあらわれるときまずいちんもくがひろがった 
\\	彼は聴衆に静かな口調で演説した。	かれはちょうしゅうにしずかなくちょうでえんぜつした 
\\	彼女は穏やかな口調で話した。	かのじょはおだやかなくちょうではなした 
\\	彼は断固たる口調で「だめだ」と言った。	"かれはだんこたるくちょうで「だめだ」といった 
\\	あの飛行船はなんて巨大なんだろう。	あのひこうせんはなんてきょだいなんだろう 
\\	彼の行為は共同体意識からのことだった。	かれのこういはきょうどうたいいしきからのことだった 
\\	人は市や国というような生活共同体を作って生活する。	ひとはしやくにというようなせいかつきょうどうたいをつくってせいかつする 
\\	インフレで給料が高い生活費に追いつかない。	インフレできゅうりょうがたかいせいかつひにおいつかない 
\\	税務署は所得税の脱税に目を光らせています。	ぜいむしょはしょとくぜいのだつぜいにめをひからせています 
\\	税務署は控除を認めた。	ぜいむしょはこうじょをみとめた 
\\	50万円の個人基礎控除がある。	ごじゅうまんえんのこじんきそこうじょがある 
\\	数学はすべての科学の基礎である。	すうがくはすべてのかがくのきそである 
\\	君は君の理論の基礎をどこに置いているのか。	きみはきみのりろんのきそをどこにおいているのか 
\\	概して言えば、親子の関係は本質的には教えることを基礎としている。	がいしていえば、おやこのかんけいはほんしつてきにはおしえることをきそとしている 
\\	体制に従おうとする意識は、この単一民族社会の本質的な要素である。	たいせいにしたがおうとするいしきは、このたんいつみんぞくしゃかいのほんしつてきなようそである 
\\	人生の最も重要な要素は驚きだ。	じんせいのもっともじゅうようなようそはおどろきだ 
\\	新しい家を買う時の主な要素はお金です。	あたらしいいえをかうときのおもなようそはおかねです 
\\	この100年、ユダヤ人ほど苦難をなめてきた民族はいないだろう。	このひゃくねん、ユダヤじんほどくなんをなめてきたみんぞくはいないだろう 
\\	顔をべろべろ舐めるな。わはは。止めろよ。	かおをべろべろなめるな。わはは。やめろよ 
\\	彼女は勇敢で朗らかで、いつも自分の苦難などは問題にしなかった。	かのじょはゆうかんでほがらかで、いつもじぶんのくなんなどはもんだいにしなかった 
\\	彼女はどんな苦難にも耐えられる人だ。	かのじょはどんなくなんにもたえられるひとだ 
\\	前途に苦難があるという思いで彼の心は重かった。	ぜんとにくなんがあるというおもいでかれのこころはおもかった 
\\	この若さで国際大会で優勝するなんて、まさに前途洋洋ですね。	このわかさでこくさいたいかいでゆうしょうするなんて、まさにぜんとようようですね 
\\	少年はまさに湖に飛び込もうとした。	しょうねんはまさにみずうみにとびこもうとした 
\\	こういうわけで、まさに私は芸術に全く関心がないのです。	こういうわけで、まさにわたしはげいじゅつにまったくかんしんがないのです 
\\	訂正された文は彼がまさに言いたいと思っていたものであった。	ていせいされたぶんはかれがまさにいいたいとおもっていたものであった 
\\	母は、暢気で朗らかでお人好しです。	ははは、のんきでほがらかでおひとよしです 
\\	君はいつも物事を暢気に考えすぎる。	きみはいつもものごとをのんきにかんがえすぎる 
\\	公平に評すれば、彼はお人好しだ。	こうへいにひょうすれば、かれはおひとよしだ 
\\	公平に評すれば、彼は最善を尽くしたと言うべきだ。	こうへいにひょうすれば、かれはさいぜんをつくしたというべきだ 
\\	彼はその男を典型的な紳士と評した。	かれはそのおとこをてんけいてきなしんしとひょうした 
\\	医者は最善を尽くしたが患者の回復は思わしくなかった。	いしゃはさいぜんをつくしたがかんじゃのかいふくはおもわしくなかった 
\\	その報告は販売に関しては思わしくない。	そのほうこくははんばいにかんしてはおもわしくない 
\\	この自動販売機に500円硬貨は使えない。	このじどうはんばいきにごひゃくえんこうかはつかえない 
\\	彼の手の平に硬貨を置く。	かれはてのひらにこうかをおく 
\\	あまりに訂正されると、話すのをやめてしまうのである。	あまりにていせいされると、はなすのをやめてしまうのである 
\\	口数が少ないほど、訂正も早くできる。	くちかずがすくないほど、ていせいもはやくできる 
\\	彼は口数が少ない男だ。	かれはくちかずがすくないおとこだ 
\\	黙秘権を行使したいと思います。	もくひけんをこうししたいとおもいます 
\\	君は暴力行使を正当化することができますか。	きみはぼうりょくこうしをせいとうかすることができますか 
\\	その労働組合は保守党に対して支配的な影響力を行使する。	そのろうどうくみあいはほしゅとうにたいしてしはいてきなえいきょうりょくをこうしする 
\\	ニクソンはウォーターゲート事件に関し、黙秘権を行使して無視しようとしたが、結局明るみに出た。	ニクソンはウォーターゲートじけんにかんし、もくひけんをこうししてむししようとしたが、けっきょくあかるみにでた 
\\	古代中国についての新しい事実が最近明るみに出た。	こだいちゅうごくについてのあたらしいじじつがさいきんあかるみにでた 
\\	警察の調べによって彼らの秘密の生活が明るみに出た。	けいさつのしらべによってかれらのひみつのせいかつがあかるみに出た 
\\	彼らはその汚職と何か関係がある。	かれらはそのおしょくとなにかかんけいがある 
\\	そのプロジェクトの責任者であったジェイソンが、汚職に手を染めた理由で解任された。	そのプロジェクトのせきにんしゃであったジェイソンが、おしょくにてをそめたりゆうでかいにんされた 
\\	彼は、怠け者なのでその地位から解任された。	かれは、なまけものなのでそのちいからかいにんされた 
\\	汚職政治家を追放せよ。	おしょくせいじかをついほうせよ 
\\	悪書を追放する。	あくしょをついほうする 
\\	支配者は打倒され国外に追放された。	しはいしゃはだとうされこくがいについほうされた 
\\	反乱軍の兵士が政権を打倒する野望を隠していた。	はんらんぐんのへいしがせいけんをだとうするやぼうをかくしていた 
\\	野望のために彼は殺人を犯した。	やぼうのためにかれはさつじんをおかした 
\\	仲間達が私に野望を果たすよう励ましてくれた。	なかまたちがわたしにやぼうをはたすようはげましてくれた 
\\	彼の政権はきっと崩壊する。	かれのせいけんはきっとほうかいする 
\\	時代遅れのその政権は崩壊寸前だ。	じだいおくれのそのせいけんはほうかいすんぜんだ 
\\	彼は委員会の中で重要な役割を果たした。	かれはいいんかいのなかでじゅうようなやくわりをはたした 
\\	インターネットの接続が遮断された。	インターネットのせつぞくがしゃだんされた 
\\	空気を遮断して火を消した。	くうきをしゃだんしてひをきえた 
\\	値段はさて置き、そのネクタイは色が私に合わない。	ねだんはさておき、そのネクタイはいろがわたしにあわない 
\\	両親はさておき、誰もその容疑者を弁護しないであろう。	りょうしんはさておき、だれもそのようぎしゃをべんごしないであろう 
\\	国中がその報道に沸き立った。	くにじゅうがそのほうどうにわきたった 
\\	彼のホームランは観衆を沸かせた。	かれのホームランはかんしゅうをわかせた 
\\	この報道は疑問の余地がある。	このほうどうはぎもんのよちがある 
\\	この報道は何の前兆だろうか。	このほうどうはなんにぜんちょうだろうか 
\\	報道の自由は阻害されてはならない。	ほうどうのじゆうはそがいされてはならない 
\\	ケリーは報道部を取り仕切っている。	ケリーはほうどうぶをとりしきっている 
\\	多くの犯罪が報道されないままに終わる。	おおくのはんざいがほうどうされないままにおわる 
\\	疑問なのはマイクがその事実を知っていたかどうかだ。	ぎもんなのはマイクがそのじじつをしっていたかどうかだ 
\\	素朴な疑問なんだけど、・・・トラとライオンはどっちが強いの?	そぼくなぎもんなんだけど、・・・トラとライオンはどっちがつよいの? 
\\	彼が正直である事は認めるけど、能力には疑問を持っている。	かれがしょうじきであることはみとめるけど。のうりょくにはぎもんをもっている 
\\	政治的な懸念から多くの人がその予測を疑問視した。	せいじてきなけねんからおおくのひとがそのよそくをぎもんしした 
\\	彼らは事態の悪化を懸念した。	かれらはじたいのあっかをけねんした 
\\	多くの産業人が経済について懸念を表明している。	おおくのさんぎょうにんがけいざいについてけねんをひょうめいしている 
\\	彼女は自分の演技が批判されることを懸念していた。	かのじょはじぶんのえんぎがひはんされることをけねんしていた 
\\	株主は同社の急速な海外での事業展開を懸念した。	かぶぬしはどうしゃのきゅうそくなかいがいでのじぎょうてんかいをけねんした 
\\	株主総会が開かれた。	かぶぬしそうかいがひらかれた 
\\	10月に総会を開催すべきだという提案があった。	じゅうがつにそうかいをかいさいすべきだというていあんがあった 
\\	会議は天候に関係なく開かれるでしょう。	かいぎはてんこうにかんけいなくひらかれるでしょう 
\\	その教授に敬意を表してパーティーが開かれた。	そのきょうじゅにけいいをあらわしてパーティーがひらかれた 
\\	勇者に敬意を払いなさい。	ゆうしゃにけいいをはらいなさい 
\\	親に対して敬意を表さなければ行けない。	おやにたいしてけいいをあらわさなければいけない 
\\	労働者階級には敬意を払うべきである。	ろうどうしゃかいきゅうにはけいいをはらうべきである 
\\	人の考え方は、その人の教育や、性別、階級、年齢などによって決定されるものだ。	ひとのかんがえかたはそのひとのきょういくや、せいべつ、かいきゅう、ねんれいなどによってけっていされるものだ 
\\	性別で人を分け隔てすべきでない。	せいべつでひとをわけへだてすべきでない 
\\	国籍や性別または職業などで人を差別してはいけない。	こくせきやせいべつまたはしょくぎょうなどでひとをさべつしてはいけない 
\\	みんな手があり、足があり、頭があるんだし、みんな歩きもするし、話もするんだもの。でも、今や、これらの人たちを分け隔てようとする何かがあるわ。	みんなてがあり、あしがあり、あたまがあるんだし、みんなあるきもするし、はなしもするんだもの。でも、いまや、これらのひとたちをわけへだてようとするなにかがあるわ 
\\	彼は人種差別と戦った。	かれはじんしゅさべつとたたかった 
\\	彼は階級差別の廃止を主張した。	かれはかいきゅうさべつのはいしをしゅちょうした 
\\	大衆は差別問題に関して全く無知である。	たいしゅうはさべつもんだいにかんするまったくむちである 
\\	「性差別」という言葉が今、流行っている。	"「せいさべつ」ということばがいま、はやっている 
\\	彼女がそうしたのは全く妥当であった。	かのじょがそうしたのはまったくだとうであった 
\\	我々の観点から言うと、彼の提案は妥当なものだ。	われわれのかんてんからいうと、かれのていあんはだとうなものだ 
\\	彼の理論は妥当なものとして広く認められている。	かれのりろんはだとうなものとしてひろくみとめられている 
\\	法律の観点からすると、彼は自由だ。	ほうりつのかんてんからすると、かれはじゆうだ 
\\	いろいろな観点からその問題を検討できる。	いろいろなかんてんからそのもんだいをけんとうできる 
\\	厳密な科学的観点からは、歴史は科学とは言えない。	げんみつなかがくてきかんてんからは、れきしはかがくとはいえない 
\\	厳密に言えば、竹は草の一種である。	げんみつにいえば、たけはくさのいっしゅである 
\\	厳密に言えば、地球は真ん丸ではない。	げんみつにいえば、ちきゅうはまんまるではない 
\\	私たちは船長の命令を厳密に実行した。	わたしはせんちょうのめいれいをげんみつにじっこうした 
\\	厳密に言うと、彼はその職業に適していない。	げんみつにいうと、かれはそのしょくぎょうにてきしていない 
\\	船長は船のすべてを統制する。	せんちょうはふねのすべてをとうせいする 
\\	協定が調印されれば、輸入規制が解除できる。	きょうていがちょういんされれば、ゆにゅうきせいがかいじょできる 
\\	暗い雲は雨の前兆だ。	くらいくもはあめのぜんちょうだ 
\\	彼の怠惰は将来に対する悪い前兆だった。	かれのたいだはしょうらいにたいするわるいぜんちょうだった 
\\	我々の協定を破ったのは君のほうだ。	われわれのきょうていをやぶったのはきみのほうだ 
\\	協定が結ばれる可能性は極めて少ない。	きょうていがむすばれるかのうせいはきわめてすくない 
\\	努力は必ず実を結ぶでしょう。	どりょくはかならずみをむすぶでしょう 
\\	その子は靴のひもを結ぶのがやっとだった。	そのこはくつのひもをむすぶのがやっとだった 
\\	私達はみんな友情で結ばれている。	わたしたちはみんなゆうじょうでむすばれている 
\\	その2つの都市はこの幹線道路によって結ばれている。	そのふたつのとしはこのかんせんどうろによってむすばれている 
\\	君の努力が実を結べばいいね。	きみのどりょくがみをむすべばいいね 
\\	その都市へ通じる幹線道路にはもう落石はない。	そのとしへつうじるかんせんどうろにはもうらくせきはない 
\\	落石は登山者にとって危険である。	らくせきはとざんしゃにとってきけんである 
\\	その条件は日本、ドイツ、イギリス、アメリカの間で調印された。	おそのじょうけんはにほん、ドイツ、イギリス、アメリカのあいだでちょういんされた 
\\	この規制は行く通りにも解釈できる。	このきせいはいくとおりにもかいしゃくできる 
\\	市内での車の使用を規制する計画がある。	しないでのくるまのしようをきせいするけいかくがある 
\\	当事者双方に義務を負わせる協定。	とうじしゃそうほうにぎむをおわせるきょうてい 
\\	お前は当事者じゃないから、そんなのんきなこと言ってられるんだよ。	おまえはとうじしゃじゃないから、そんなのんきなこといっていられるんだよ 
\\	彼は両当事者の間を調停した。	かれはりょうとうじしゃのあいだをちょうていした 
\\	彼は双方の候補者から票を奪った。	かれはそうほうのこうほしゃからひょうをうばった 
\\	双方が互いに歩み寄らねばならなかった。	そうほうがたがいにあゆみよらねばならなかった 
\\	双方の言い分を聞かないと真相は分からない。	そうほうのいいぶんをきかないとしんそうはわからない 
\\	その争いの後、双方に多数の死者が出た。	そのあらそいのあと、そうほうにたすうのししゃがでた 
\\	お互い歩み寄って問題を解決した。	おたがいあゆみよってもんだいをかいけつした 
\\	投票で決める。	とうひょうできめる 
\\	投票ができる年齢ですか。	とうひょうができるねんれいですか 
\\	我々は投票で議長を選んだ。	われわれはとうひょうでぎちょうをえらんだ 
\\	大統領は投票の過半数をもって選ばれる。	だいとうりょうはとうひょうのかはんすうをもってえらばれる 
\\	合格者の過半数は大学出身者であった。	ごうかくしゃのかはんすうはだいがくしゅっしんしゃであった 
\\	彼らは国会で過半数を制した。	かれらはこっかいでかはんすうをせいした 
\\	自制するよう努めなさい。	じせいするようつとめなさい 
\\	彼女は大変腹を立てたので、自制心を失った。	かのじょはたいへんはらをたてたので、じせいしんをうしなった 
\\	彼は私に事故の責任を負わせた。	かれはわたしにじこのせきにんをおわせた 
\\	その計画は会社に多額の出費を負わせるでしょう。	そのけいかくはかいしゃにたがくのしゅっぴをおわせるでしょう 
\\	車を持っていることは相当な出費だ。	くるまをもっていることはそうとうなしゅっぴだ 
\\	彼女は出費に関わらず自分の計画を実行するだろう。	かのじょはしゅっぴにかかわらずじぶんのけいかくをじっこうするだろう 
\\	我々は出費と収入の間で均衡をはかるべきだ。	われわれはしゅっぴとしゅうにゅうのあいだできんこうをはかるべきだ 
\\	彼の出費は収入をはるかに上回っている。	かれのしゅっぴはしゅうにゅうをはるかにうわまわっている 
\\	僕の学校の成績は平均をかなり上回ってきた。	ぼくのがっこうのせいせきはへいきんをかなりうわまわってきた 
\\	上位2社で市場の50%を上回るシェアを占めている。	じょういにしゃでしじょうの50%をうわまわるシェアをしめている 
\\	類人猿は知的には犬より上位である。	るいじんえんはちてきにはいぬよりじょういである 
\\	何故類人猿は他の動物よりも進化したのか。	なぜるいじんえんはほかのどうぶつよりもしんかしたのか 
\\	類人猿は知能が高い。	るいじんえんはちのうがたかい 
\\	誰にも知的な願いがある。	だれにもちてきなねがいがある 
\\	知的好奇心を持つことは重要だ。	ちてきこうきしんをもつことはじゅうようだ 
\\	彼女は知能の程度が高い。	かのじょはちのうのていどがたかい 
\\	日本は、サービスがGNPの50%以上を占めるサービス経済である。	にほんはサービスがGNPの50%いじょうをしめるサービスけいざいである 
\\	慎重は勇気の大半を占める。	しんちょうはゆうきのたいはんをしめる 
\\	私は地球外の知的生命が地球を見ていると信じている。	わたしはちきゅうがいのちてきせいめいがちきゅうをみているとしんじている 
\\	彼らの結婚費用は相当なものだった。	かれらのけっこんひようはそうとうなものだった 
\\	その場所は今は相当建て込んでいる。	そのばしょはいまはそうとうたてこんでいる 
\\	氷は毎日相当な距離を漂うこともある。	こおりはまいにちそうとうなきょりをただようこともある 
\\	貿易不均衡が大きな問題であるように思える。	ぼうえきふきんこうがいおおきなもんだいであるようにおもえる 
\\	予算は均衡がとれていなければならない。	よさんはきんこうがとれていなければならない 
\\	いたる所で自然の均衡が破られようとしている。	いたるところしぜんのきんこうがやぶれようとしている 
\\	彼は至る所で歓迎された。	かれはいたるところでかんげいされた 
\\	その写真は至る所に貼ってある。	そのしゃしんはいたるところにはってある 
\\	公民権運動はある夢に至る。	こうみんけんうんどうはあるゆめにいたる 
\\	研究すればするほど、好奇心が強くなるでしょう。	けんきゅうすればするほど、こうきしんがつよくなるでしょう 
\\	彼は好奇心に駆られて質問した。	かれはこうきしんにかられてしつもんした 
\\	私は彼女の過去について好奇心でいっぱいだ。	わたしはかのじょのかこについてこうきしんでいっぱいだ 
\\	彼女は好奇心からというより見栄からテニスを始めた。	かのじょはこうきしんからというよりみえからテニスをはじめた 
\\	彼は大学を卒業して、いつも見栄を張っていた。	かれはだいがくをそつぎょうして、いつもみえをはっていた 
\\	ビバリーヒルズのような高級住宅地で見栄を張り合うのは高くつく。	ビバリーヒルズのようなこうきゅうじゅうたくちでみえをはりあうのはたかくつく 
\\	修理は高くつくでしょう。	しゅうりはたかくつくでしょう 
\\	天文学は恒星と惑星を扱う。	てんもんがくはこうせいとわくせいをあつかう 
\\	私がとても興味を抱いているのは天文学だ。	わたしがとてもきょうみをだいているのはてんもんがくだ 
\\	惑星は恒星の周りを回る。	わくせいはこうせいのまわりをまわる 
\\	酔っ払っておそく家に帰ったかどで、怒った女房は亭主に食ってかかり、箒で亭主をひっぱたいた。	よっぱらっておそくいえにかえったかどで、おこったにょうぼうはていしゅにくってかかり、ほうきでていしゅをひっぱたいた 
\\	亭主が女房の尻にしかれるのも当然だ。	ていしゅがにょうぼうのしりにしかれるのもとうぜんだ 
\\	講師は本題から脱線してしまった。	こうしはほんだいからだっせんしてしまった 
\\	私はもう具体的な本題に入ってもいい頃だと思う。	わたしはもうぐたいてきなほんだいにはいってもいいころだとおもう 
\\	汽車が脱線した。	きしゃがだっせんした 
\\	「私の名刺です。追加情報があったらいつでも連絡下さい」と記者は言った。	"「わたしのめいしです。ついかじょうほうがあったらいつでもれんらくください」ときしゃはいった 
\\	第一の難関はどうにか突破した。	だいいちのなんかんはどうにかとっぱした 
\\	中学生が英語を学ぶ際の最難関の一つが関係代名詞です。	ちゅうがくせいがえいごをまなぶさいのさいなんかんのひとつがかんけいだいめいしです 
\\	我が軍は敵の防御を突破した。	わがぐんはてきのぼうぎょをとっぱした 
\\	彼らは敵陣を突破しようと試みた。	かれらはてきじんをとっぱしようとこころみた 
\\	大将は敵陣に攻撃をかける決断を下した。	たいしょうはてきじんにこうげきをかけるけつだんをおろした 
\\	彼は大将に命令を取り消すように頼んだ。	かれはたいしょうにめいれいをとりけすようにたのんだ 
\\	予約を取り消してください。	よやくをとりけしてください 
\\	緊急の用事ができたので約束を取り消した。	きんきゅうのようじができたのでよやくをとりけした 
\\	攻撃は最大の防御なり。	こうげきはさいだいのぼうぎょなり 
\\	例えば、小鳥は特別な防御装置を備えている。	たとえば、ことりはとくべつなぼうぎょそうちをそなえている 
\\	脅威は変化し続ける、進化できない防御は意味がない。	きょういはへんかしつづける、しんかできないぼうぎょはいみがない 
\\	野生の虎はアフリカでは見られません。	やせいのとらはアフリカではみられません 
\\	地球温暖化は野生動物にも深刻な問題を引き起こしうる。	ちきゅうおんだんかはやせいどうぶつにもしんこくなもんだいをひきおこしうる 
\\	その地域は風景と野生動物で注目に値する。	そのちいきはふうけいとやせいどうぶつでちゅうもくにたいする 
\\	木の下にいくつかの野性の茸を見つけた。	きのしたにいくつかのやせいのきのこをみつけた 
\\	椎茸を乾燥させて保存します。	しいたけをかんそうさせてほぞんします 
\\	暑くて乾燥した地方で最も手に入りやすいエネルギー源は、風と日光である。	あつくてかんそうしたちほうでもっともてにいりやすいエネルギーげんは、かぜとにっこうである 
\\	落ち込むなよ、君を傷つけるつもりじゃなかったんだ。	おちこむなよ、きみをきずつけるつもりじゃなかったんだ 
\\	冬の冷たい風に何時間もさらされていたせいで肌がカサカサになった。	ふゆのつめたいかぜになんじかんもさらされていたせいではだがカサカサになった 
\\	なにが起ころうとも音楽創作をとめることはできない。	なにがおろうともおんがくそうさくをとめることはできない 
\\	フラフラ迷っていたんだ。	フラフラまよっていたんだ 
\\	夫の死後私はゾンビのようにふらふら歩き回っていた。	おっとのしごわたしはゾンビのようにふらふらあるきまわっていた 
\\	道路をふらふら横断する人は非常な危険に身を晒す。	どうろをふらふらおうだんするひとはひじょうなきけんにみをさらす 
\\	老婦人が道を横断している。	ろうふじんがみちをおうだんしている 
\\	彼女は船で太平洋横断に成功した。	かのじょはふねでたいへいようおうだんにせいこうした 
\\	彼らは徒歩でその広大な大陸を横断した。	かれらはとほでそのこうだいなたいりくをおうだんした 
\\	彼は小さなヨットで大西洋を横断した。	かれはちいさなヨットでたいせいようをおうだんした 
\\	彼は、徒歩で通学している。	かれはとほでつうがくしている 
\\	徒歩旅行者たちはボストンを出発し6カ月後にサンフランシスコに到着した。	とほりょこうしゃたちはボストンをしゅっぱつしろっかげつあとにサンフランシスコにとうちゃくした 
\\	彼は雨の日以外は毎日徒歩で出勤します。	かれはあめのひいがいはまいにちとほでしゅっきんします 
\\	学校の近くに住んでいるので、普段は歩いて通学する。	がっこうのちかくにすんでいるので、ふだんはあるいてつうがくする 
\\	通りを横断する際に彼は足を滑らせた。	とおりをおうだんするさいにかれはあしをすべらせた 
\\	彼らは普段自転車で登校します。	かれらはふだんじてんしゃでとうこうします 
\\	ラジオの放送で普段と同じように話をするのは容易ではない。	ラジオのほうそうでふだんとおなじようにはなすのはよういではない 
\\	学生服は普段着としても式服としても着られるので便利だ。	がくせいふくはふだんぎとしてもしきふくとしてもきられるものでべんりだ 
\\	今朝、登校の途中で外国人の一団に会いました。	けさ、とうこうのとちゅうでがいこくじんのいちだんにあいました 
\\	明日はいつもより1時間早く出勤することができますか。	あしたはいつもよりいちじかんはやくしゅっきんすることができますか 
\\	彼は危険に晒されていた。	かれはきけんにさらされていた 
\\	清水の舞台から飛び降りたつもりで脱サラした。	きよみずのぶたいからとびおりたつもりでだつサラした 
\\	パティは浜辺で背中を太陽に晒した。	パティははまべでせなかをたいようにさらした 
\\	清水の舞台から飛び降りるつもりで勝負にでるよ。	きよみずのぶたいからとびおりるつもりでしょうぶにでるよ 
\\	私は清水の舞台から飛び降りる気持ちで南米に渡ります。	わたしはきよみずのぶたいからとびおりるきもちでなんべいにわたります 
\\	当社の第一目標は南米市場を拡大することです。	とうしゃのだいいちもくひょうはなんべいしじょうをかくだいすることです 
\\	石油の特質の一つは水に浮くということである。	せきゆのとくしつのひとつはみずにうくよいうことである 
\\	笑いは人間だけの特質なのか。	わらいはにんげんだけのとくしつなのか 
\\	その子供は、つまずいて転んで膝を付いた。	そのこどもは、つまずいてころんでひざをついた 
\\	その痩せた男は膝を曲げて日陰で少し休んだ。	そのやせたおとこはひざをまげてひかげですこしやすんだ 
\\	クリスティーンは1日中日陰にいました。なぜなら彼女は日焼けしたくないからです。	クリスティーンはいちにちちゅうひかげにいました。なぜならかのじょはひやけしたくないからです 
\\	この町の気候は非常に温和で、真夏でも寒暖計が30度にあがることはめったにない。	このまちのきこうはひじょうにおんわで、まなつでもかんだんけいがさんじゅうどにあがることはめったにない 
\\	寒暖計は零下に下がった。	かんだんけいはれいかにさがった 
\\	今朝は零下3度以下だった。	けさはれいかさんどいかだった 
\\	君の要約は平均以下だね。	きみのようやくはへいきんいかだね 
\\	彼の仕事は標準以下だ。	かれのしごとはひょうじゅんいかだ 
\\	16歳以下の子供は劇場には入場できません。	じゅうろくさいいかのこどもはげきじょうにはにゅうじょうできません 
\\	この食品は10度以下で保存したほうがいい。	このしょくひんはじゅうどいかでほぞんしたほうがいい 
\\	開演中入場お断り。	かいえんちゅうにゅうじょうおことわり 
\\	入場料は総計2500ドルになった。	にゅうじょうりょうはそうけいにせんごひゃくドルになった 
\\	損害は総計1千万ドルになる。	そんがいはそうけいせんまんドルになる 
\\	胸をどきどきさせて開演を待った。	むねをどきどきさせてかいえんをまった 
\\	彼はいつも公開演説会で演説しているかのように話す。	かれはいつもこうかいえんぜつかいでえんぜつしているかのようにはなす 
\\	その報告を1ページに要約せよ。	そのほうこくをいちページにようやくせよ 
\\	あなたの要約は文句の付けようが無い。	あなたのようやくはもんくのつけようがない 
\\	「結論」の目的は論文の主要な論点を要約することだ。	「けつろん」のもくてきはろんぶんのしゅようなろんてんをようやくすることだ 
\\	私は標準英語を勉強したい。	わたしはひょうじゅんえいごをべんきょうしたい 
\\	時計を日本標準時に合わせよう。	とけいをにほんひょうじゅんじにあわせよう 
\\	彼らの作品を同じ標準では判断できない。	かれらのさくひんをおなじひょうじゅんでははんだんできない 
\\	いわゆる「標準英語」とは世界中で話されている数多い方言のうちの1つにすぎない。	いわゆる「ひょうじゅんえいご」とはせかいじゅうではなされているかずおおいほうげんのうちのひとつにすぎない 
\\	その赤ちゃんの発育はその年齢では標準です。	そのあかちゃんのはついくはそのねんれいではひょうじゅんです 
\\	冷たい天候が稲の発育を遅らせた。	つめたいてんこうがいなのはついくをおくらせた 
\\	赤ん坊は正常な発育を示した。	あかんぼうはせいじょうなはついくをしめした 
\\	君の脈は正常だ。	きにのみゃくはせいじょうだ 
\\	政府は事態を正常に戻す努力をしている。	せいふはじたいをせいじょうにもどすどりょくをしている 
\\	まもなく景気は正常なレベルまで回復するだろう。	まもなくけいきはせいじょうなレベルまでかいふくするだろう 
\\	人体にはそれ自体を健康で正常にする神秘的な力がある。	じんたいにはそれじたいをけんこうでせいじょうにするしんぴてきなちからがある 
\\	台風は作物に大きな損害を与えた。	たいふうはさくぶつにおおきなそんがいをあたえた 
\\	彼は損害を500万円と見積もった。	かれはそんがいをごひゃくまんえんとみつもった 
\\	収入の範囲内で暮らすべきだ。	しゅうにゅうのはんいないでくらすべきだ 
\\	これは私の想像の範囲を超えている。	これはわたしのそうぞうのはんいをこえている 
\\	今日では車の値段は広い範囲に渡っている。	きょうではくるまのねだんはひろいはんいにわたっている 
\\	最も良い価格で見積もって下さい。	もっともよいかかくでみつもってください 
\\	山田さんは旅行社から出された見積もりに渋い顔をした。	やまださんはりょこうしゃからだされたみつもりにしぶいかおをした 
\\	タイムをとって作戦を練ろう。	タイムをとってさくせんをねろう 
\\	君の話から考えて、我々の計画は練り直すべきだと思う。	きみのはなしからかんがえて、われわれのけいかくはねりなおすべきだとおもう 
\\	上司は練り直したスケジュールを見ると、ウンウンと頷いた。	じょうしはねりなおしたしたスケジュールをみると、ウンウンとうなずいた。 
\\	彼はいつも慎重に計画を練ってから、それを実行に移す。	かれはいつもしんちょうにけいかくをねってから、それをじっこうにうつす 
\\	ホテルはその夜、満員だったので、遅い客は何人か断らざるを得なかった。	ホテルはそのよる、まんいんだったので、おそいきゃくはなんにんかことわらざるをえなかった 
\\	満員電車の中は息が詰まりそうだった。	まんいんでんしゃのなかはいきがつまりそうだった 
\\	どの電車も満員で、持ち物も手から放しても落ちないほどです。	どのでんしゃもまんいんで、もちものもてからはなしてもおちないほどです 
\\	リンカーンは、全国の奴隷を解放せよと命令した。	リンカーンは、ぜんこくのどれいをかいほうせよとめいれいした 
\\	容疑者は遂に口を割った。	ようぎしゃはついにくちをわった 
\\	尋問した後警察は容疑者を自宅まで連れ戻した。	じんもんしたあとけいさつはようぎしゃをじたくまでつれもどした 
\\	至急電報が彼女を大急ぎで東京に連れ戻した。	しきゅうでんぽうがかのじょをおおいそぎでとうきょうにつれもどした 
\\	残りの5個を至急お送りください。	のこりのごこをしきゅうおおくりください 
\\	ガーデンフェンスを至急配達して下さい。	ガーデンフェンスをしきゅうはいたつしてください 
\\	尋問の間彼はずっと黙っていた。	じんもんのあいだかれはずっとだまっていた 
\\	刑事はその事件について文字どおり何千もの人たちに尋問した。	けいじはそのじけんについてもじどおりなんぜんものひとたちにじんもんした 
\\	軍隊の規律は文字通り厳しいものだ。	ぐんたいのきりつはもじどおりきびしいもんだ 
\\	かわいそうにその犬は文字通りライオンに引き裂かれてしまった。	かわいそうにそのいぬはもじどおりライオンにひきさかれてしまった 
\\	彼は文字どおりの馬鹿だ。	かれはもじどおりのばかだ 
\\	学校の規律が乱れている。	がっこうのきりつがみだれている 
\\	生徒の中には規律を守るのが大変なものもいる。	せいとのなかにはきりつをまもるのがたいへんなものもいる 
\\	庭には様々な花が咲き乱れていた。	にわはさまざまなはながさきみだれている 
\\	恋人の死の知らせに彼女の心は乱れた。	こいびとのしのしらせにかのじょのこころはみだれた 
\\	オレの醜い部分がオレを引き裂く。	オレのみにくいぶぶんがおれをひきさく 
\\	彼は彼女からの手紙を怒って粉々に引き裂いた。	かれはかのじょからのてがみをおこってこなごなにひきさいた 
\\	猫は捕らえたネズミを引き裂き始めた。	ねこはとらえたネズミをひきさきはじめた 
\\	花瓶は粉々に砕けた。	かびんはこなごなにくだけた 
\\	東洋の陶器に興味があります。	とうようのとうきにきょうみがあります 
\\	陶器は火で焼かれた。	とうきはひでやかれた 
\\	大人の一方的な観点からは、子供たちの態度はしばしば生意気に見える。	おとなのいっぽうてきなかんてんからは、こどもたちのたいどはしばしばなまいきにみえる 
\\	その双子はよく似ているので一方を他方と区別するのはほとんど不可能だ。	そのふたごはよくにているのでいっぽうをたほうとくべつするのはほとんどふかのうだ 
\\	一方で我々は大損害を被ったが、他方その経験から学んだものも大きかった。	いっぽうでわれわれはおおそんがいをかぶったが、たほうそのけいけんからまなんだものもおおきかった 
\\	ただ一度の不注意な間違いがもとで会社は何百万ドルもの損失を被ることになった。	ただいちどのふちゅういなまちがいがもとでかいしゃはなんびゃくまんドルのそんしつをかぶることになった 
\\	その会社は甚大な被害を被った。	そのかいしゃはじんだいなひがいをかぶった 
\\	長く続く干ばつが収穫に甚大な被害をもたらした。	ながくつづくかんばつがしゅうかくにじんだいなひがいをもたらした 
\\	その台風の被害は甚大であった。	そのたいふうのひがいはじんだいであった 
\\	戦争は傷跡を齎す。	せんそうはきずあとをもたらす 
\\	革命は新たな時代をもたらした。	かくめいはあらたなじだいをもたらした 
\\	新たな議論が提起された。	あらたなぎろんがていきされた 
\\	それは私には全く新たな状況だ。	それはわたしにはまったくあらたなじょうきょうだ 
\\	政府はワインに新たに税を貸した。	せいふはワインにあらたにぜいをかした 
\\	二つの国の戦争は両国の大きな損失で終わった。	ふたつのくにのせんそうはりょうごくのおおきなそんしつでおわった 
\\	彼は損失の責任をとって、その仕事を辞任した。	かれはそんしつのせきにんをとって、そのしごとをじにんした 
\\	損失の程度は計り知れない。	そんしつのていどははかりしれない 
\\	彼は、その事業での損失のやりくりをつけるため、兄弟から金を借りた。	かれは、そのじぎょうでのそんしつのやりくりをつけるため、きょうだいからかねをかりた 
\\	今年の収穫は昨年には及ばない。	ことしのしゅうかくはさくねんにはおよばない 
\\	我々は秋に夏の穀物を収穫します。	われわれはあきになつのこくもつをしゅうかくします 
\\	彼は穀物を商っている。	かれはこくもつをあきなっている 
\\	彼らは大量の穀物を蓄えている。	かれらはたいりょうのこくもつをたくわえている 
\\	あの店では台所用品を商っている。	あのみせではだいどころようひんをあきなっている 
\\	公然と悪態をつかないでよ。	こうぜんとあくたいをつかないでよ 
\\	彼女は大声で悪態をついた。	かのじょはおおごえであくたいをついた 
\\	彼は私に悪態を浴びせかけた。	かれはわたしにあくたいをあびせかけた 
\\	彼女は私に悪口を浴びせた。	かのじょはわたしにわるぐちをあびせた 
\\	記者は政治家に質問を浴びせた。	きしゃはせいじかにしつもんをあびせた 
\\	日本の資金力は今や世界の隅々に及ぶ。	にほんのしきんりょくはいまやせかいのすみずみにおよぶ 
\\	その島は隅々まで探索されている。	そのしまはすみずみまでたんさくされている 
\\	詩は説明し難いものへの探索である。	しはせつめいしにくいものへのたんさくである 
\\	彼らの埋蔵された宝物を求めて砂漠を探索した。	かれらのまいぞうされたほうもつをもとめてさばくをたんさくした 
\\	彼は偶然宝物を見つけた。	かれはぐうぜんほうもつをみつけた 
\\	彼らは宝物を求めてあちこち掘った。	かれらはほうもつをもとめてあちこちほった 
\\	試験は大きな問題を提起する。	しけんはおおきなもんだいをていきする 
\\	男の子の中には定期的に入浴するのが嫌いな子もいる。	おとこのこのなかにはていきてきてきににゅうよくするのがきらいなこもいる 
\\	洋服を定期的に買う余裕はありません。	ようふくをていきてきにかうよゆうはありません 
\\	賃金や給料は定期的に受け取る給与のことです。	ちんぎんやきゅうりょうはていきてきにうけとるきゅうよのことです 
\\	私には高価な車を買う余裕がない。	わたしはこうかなくるまをかうよゆうがない 
\\	旅行は私には余裕のない贅沢である。	りょこうはわたしにはよゆうのないぜいたくである 
\\	一介の学生にすぎないので、私は結婚する余裕がない。	いっかいのがくせいにすぎないので、わたしはけっこんするよゆうがない 
\\	私は中古車を買う余裕がない、まして新車は買えない。	わたしはちゅうこしゃをかうよゆうがない、ましてしんしゃはかえない 
\\	彼一介の詩人に過ぎない。	かれはいっかいのしじんにすぎない 
\\	彼には日常必需品すらない、まして贅沢品はあるわけがない。	かれはにちじょうひつじゅひんすらない、ましてぜいたくひんはあるわけがない 
\\	例によって私の入浴中に電話が鳴った。	れいによってわたしのにゅうよくちゅうにでんわがなった 
\\	水不足のために入浴できなかった。	みずぶそくのためににゅうよくできなかった 
\\	例によって彼女はまた時間通りには現われなかった。	れいによってかのじょはまたじかんどおりにはあらわれなかった 
\\	アメリカへの旅は彼女にとって2年間の給与に相当した。	アメリカへのたびはかのじょにとってにねんかんのきゅうよにそうとうした 
\\	その仕事はあまり面白くなかったが、その一方で給与はよかった。	そのしごとはあまりおもしろくなかったが、そのいっぽうできゅうよはよかった 
\\	私は健康診断を受けた。	わたしはけんこうしんだんをうけた 
\\	診断は医者に任せなければなりません。	しんだんはいしゃにまかせなければなりません 
\\	もし診断書があったらお待ちください。	もししんだんしょがあったらおもちください 
\\	その女性は一人一人を注意深く診断している。	そのじょせいはひとりひとりをちゅういぶかくしんだんしている 
\\	その老人は道路を注意深く横断した。	そのろうじんはどうろをちゅういぶかくおうだんした 
\\	傷跡ははっきり残っている。	きずあとははっきりのこっている 
\\	口内炎ができて食事する時、痛くて痛くて。	こうないえんができてしょくじするとき、いたくていたくて 
\\	もう勝負は決まった。男らしく罰を受けようではないか。	もうしょうぶはきまった。おとこらしくばつをうけようではないか 
\\	勝負の見込みは五分五分。	しょうぶのみこみはごぶごぶ 
\\	キャンセル待ちで乗れる確率はどれくらいですか。	キャンセルまちでのれるかくりつはどれくらいですか 
\\	彼は彼女の足元に崩れるように倒れた。	かれはかのじょのあしもとにくずれるようにたおれた 
\\	その建物は突然崩れ落ちた。	そのたてものはとつぜんくずれおちた 
\\	彼女はその知らせを来て泣き崩れた。	かのじょはそのしらせをきてなきくずれた 
\\	同時通訳によって言語の障壁が崩れた。	どうじつうやくによってげんごのしょうへきがくずれた 
\\	我々は社会的障壁を取り壊すために、懸命に努力しなければならない。	われわれはしゃかいてきしょうへきをとりこわすために、けんめいにどりょくしなければならない 
\\	最初にあげなければならない問題は、それらのグループの間に文化障壁が存在していたかどうかということである。	さいしょにあげなければならないもんだいは、それらのグループのあいだにぶんかしょうへきがそんざいしていたかどうかということである 
\\	その建物は火災の検査に通らなかった。	そのたてものはかさいのけんさにとおらなかった 
\\	社長は現実的なタイプの人だ。	しゃちょうはげんじつてきなタイプのひとだ 
\\	予算は不正確で、しかも非現実的に思える。	よさんはふせいかくで、しかもひげんじつてきにおもえる 
\\	心情的には賛成、けれど現実的に反対します。	しんじょうてきにはさんせい、けれどげんじつてきにはんたいします 
\\	我々が真理を知るのは、理知によるだけではなく、また心情によって知るのである。	われわれはしんりをしるのはりちによるだけではなく。またしんじょうによってしるのである 
\\	意見は究極的には感情によって決定されるのであって、理知によってではない。	いけんはきゅうきょくてきにはかんじょうによってけっていされるのであって、りちによってではない 
\\	科学の第一の目的は、心理を、新しい真理を発見することである。	かがくのだいいちのもくてきは、しんりを、あたらしいしんりをはっけんすることである 
\\	概して、真理が基礎的なものであればあるほど、その実用の可能性も大きくなる。	がいして、しんりがきそてきなものであればあるほど、そのじつようのかのうせいもおおきくなる 
\\	彼女は生まれながらに知性と美しさに恵まれていた。	かのじょはうまれながらにちせいとうつくしさにめぐまれていた 
\\	彼女は美人でしかも知性も備わっている。	かのじょはびじんでしかもちせいもそなわっている 
\\	知性を持っていることが我々と動物との異なる点である。	ちせいをもっていることがわれわれとどうぶつとのことなるてんである 
\\	彼には権力も能力も備わっている。	かれにはけんりょくものうりょくもそなわっている 
\\	彼は生まれながらにして偉大な才能に恵まれていた。	かれはうまれながらにしていだいなさいのうにめぐまれている 
\\	寛大さが生まれながら身につけている人もいる。	かんだいさがうまれながらみにつけているひともいる 
\\	貧乏は、ある意味で、天の恵みだ。	びんぼうは、あるいみで、てんのめぐみだ 
\\	おいおい、一枚看板のお前が来れないんじゃ、今日の合コン盛り上がらないよ。	おいおい、いちまいかんばんのおまえがこられないんじゃ、きょうのごうこんもりあがらないよ 
\\	火山活動で地面が盛り上がった。	かざんかつどうでじめんがもりあがった 
\\	彼等の家具は実用的と言うよりも美的である。	かれらのかぐはじつようてきというよりもびてきである 
\\	彼の案は現実離れしすぎていて、我々にとって実用的ではない。	かれのあんはげんじつばなれしすぎていて、われわれにとってじつようてきではない 
\\	彼の現実離れした提案はみんなを驚かせた。	かれのげんじつばなれしたていあんはみんなをおどろかせた 
\\	彼女は美的感覚がある。	かのじょはびてきかんかくがある 
\\	私の妻は部屋を飾るときに優れた美的感覚を発揮した。	わたしのつまはへやをかざるときにすぐれたびてきかんかくをはっきした 
\\	彼は経済学の話をすると本領を発揮する。	かれはけいざいがくのはなしをするとほんりょうをはっきする 
\\	小説を書く時、私達は想像力を発揮する。	しょうせつをかくとき、わたしたちはそうぞうりょくをはっきする 
\\	利益は効果を発揮してる。	りえきはこうかをはっきしてる 
\\	この作家が本領を発揮しているのは短編小説だ。	このさっかがほんりょうをはっきしているのはたんぺんしょうせつだ 
\\	悪路になると、この小型の車が本当にその本領を発揮する。	あくろになると、このこがたのくるまがほんとうにそのほんりょうをはっきする 
\\	小型車を借りたいのですが。	こがたしゃをかりたいのですが 
\\	彼の家族は小麦農家だった。	かれのかていはこむぎのうかだった 
\\	悲劇の本質は、短編小説のそれと同じように、その葛藤である。	ひげきのほんしつは、たんぺんしょうせつのそれとおなじように、そのかっとうである 
\\	父と私の葛藤です。	ちちとわたしのかっとうです 
\\	人類の究極的運命はどうなるであろうか。	じんるいきゅうきょくてきうんめいはどうなるであろうか 
\\	究極的には宇宙飛行は全人類に有益なものとなろう。	きゅうきょくてきにはうちゅうひこうをぜんんじんるいにゆうえきなものとなろう 
\\	我々の究極の目標は世界平和を樹立することである。	われわれのきゅうきょくのもくひょうはせかいへいわをじゅりつすることである 
\\	船やヘリコプターが宇宙飛行士を救助しに出発した。	ふねやヘリコプターがうちゅうひこうしをきゅうじょしにしゅっぱつした 
\\	彼らは自分たちの信仰に熱狂している。	かれらはじぶんたちのしんこうにねっきょうしている 
\\	我々は熱狂的な阪神タイガースファンである。	われわれはねっきょうてきなはんしんタイガースファンである。 
\\	熱狂的な観客が競技場に傾れ込んだ。	ねっきょうてきなかんきゃくがきょうぎじょうになだれこんだ 
\\	難民が国中から傾れ込んだ。	なんみんがこくじゅうからなだれこんだ 
\\	彼らは新政府を樹立した。	かれらはしんせいふをじゅりつした 
\\	我々はその国の新政府との友好関係を樹立した。	われわれはそのくにのしんせいふとのゆうこうかんけいをじゅりつした 
\\	手を付けずダラダラしているくせに、「うへー、今度こそ間に合わないかも!?」と心はビクビクしている。	"てをつけずダラダラしているくせに、「うへー、こんどこそまにあわないかも!?」とこころはビクビクしている。 
\\	一言批判されただけで彼女はびくびくする。	ひとことひはんされただけでかのじょはびくびくする 
\\	強引な販売員がすぐに契約書に署名するよう強く迫った。	ごういんなはんばいいんがすぐにけいやくしょにしょめいするようつよくせまった 
\\	政府はその法案を強引に議会を通過させた。	せいふはそのほうあんをごういんにぎかいをつうかさせた 
\\	与党は強引に税制法案を通過させた。	よとうはごういんにぜいせいほうあんをつうかさせた 
\\	与党は前の選挙で過半数を占めた。	よとうはまえのせんきょでかはんすうをしめた 
\\	今度の選挙では与党が勝ちそうだね。	こんどのせんきょではよとうがかちそうだね 
\\	彼等は貿易問題について臨時の会合を開いた。	かれらはぼうえきもんだいについてりんじのかいごうをひらいた 
\\	我々みんなに臨時ボーナスが出るって君は言ったけど、一杯食わせたのかい。	われわれみんなにりんじボーナスがでるってきみはいったけど、いっぱいくわせたのかい 
\\	あいつに一杯食わされた。	あいつにいっぱいくわされた 
\\	ビルは優秀な科学者になる素質を持っている。	ビルはゆうしゅうなかがくしゃになるそしつをもっている 
\\	データは、メインコンピューターから自分のものに転送できるし、またその逆もできる。	データは、メインコンピューターからじぶんのものにてんそうできるし、またそのぎゃくもできる 
\\	セミナーのご案内と共に、この情報を貴社内の担当の管理職にご転送ください。	セミナーのごあんないとともに、そのじょうほうをきしゃないたんとうのかんりしょくにごてんそうください 
\\	事情にあわせて、貴社の記録も調整してください。	じじょうにあわせて、きしゃのきろくもちょうせいしてください 
\\	ご注文を受け取り次第、製品を貴社にお送りします。	ごちゅうもんをうけとりしだい、せいひんをきしゃにおおくりします 
\\	彼は、望遠鏡を自分に合うように調整した。	かれは、ぼうえんきょうをじぶんにあうようにちょうせいした 
\\	落ち込みは季節調整すればそれほど大きくない。	おちこみはきせつちょうせいすればそれほどおおきくない 
\\	彼女の熱心な希望に逆らって彼は外国へ行った。	かのじょのねっしんなきぼうにさからってかれはがいこくへいった 
\\	彼らは彼の意志に逆らって彼にその契約に署名させた。	かれらはかれのいしにさからってかれにそのけいやくにしょめいさせた 
\\	意志があるところに道は開ける。	いしがあるところにみちはあける 
\\	彼の意志の力によって勝利を収めた。	かれはいしのちからによってしょうりをおさめた 
\\	私たちはなんとか意志の疎通ができた。	わたしたちはなんとかいしのそつうができた 
\\	私は彼を私の意志に従わせることができない。	わたしはかれをいしにしたがわせることができない 
\\	私たちは、言語という手段を使って意思の疎通をします。	わたしたちは、げんごというしゅだんをつかっていしのそつうをします 
\\	意思の強い人は堕落しない。	いしのつよいひとはだらくしない 
\\	人間は互いに言葉で意思を通じ合う。	にんげんはたがいにことばでいしをつうじあう 
\\	彼らはお互いに身振りで意思を伝え合った。	かれらはおたがいにみぶりでいしをつたえあった 
\\	彼女は身振りを交えながら演説した。	かのじょはみぶりをまじえながらえんぜつした 
\\	彼は静かにするよう身振りで指図した。	かれはしずかにするようみぶりでさしずした 
\\	猫が気持ちを伝える主要なやり方は身振りである。	ねこがきもちをつたえるしゅようなやりかたはみぶりである 
\\	私たちは簡単な内容を伝えるのにしばしば身振りを用いる。	わたしたちはかんたんなないようをつたえるのにしばしばみぶりをもちいる 
\\	彼は選挙で勝利を収めるのは明らかだ。	かれはせんきょでしょうりをおさめるのはあきらかだ 
\\	その若者はこの仕事で成功を収めるであろう。	そのわかものはこのしごとでせいこうをおさめるであろう 
\\	彼女はこの闘争で勝利を収めた。	かのじょはこのとうそうでしょうりをおさめた 
\\	彼らは多くの命を犠牲にしてその戦いに勝利を収めた。	かれらはおおくのいのちをぎせいにしてそのたたかいにしょうりをおさめた 
\\	彼は勝利の瞬間を待ちわびた。	かれはしょうりのしゅんかんをまちわびた 
\\	彼は道徳的に堕落した。	かれはどうとくてきにだらくした 
\\	麻薬中毒で多くの人が堕落した。	まやくちゅうどくでおおくのひとがだらくした 
\\	彼は市政の堕落を暴露した。	かれはしせいのだらくをばくろした 
\\	市長は市政を司る。	しちょうはしせいをつかさどる 
\\	彼らの秘密が暴露された。	かれらのひみつがばくろされた 
\\	彼は勇敢にもそのスキャンダルを暴露した。	かれはゆうかんにもそのスキャンダルをばくろした 
\\	蛙はどんどん腹を膨らませとうとう破裂してしまった。	かえるはどんどんはらをふくらませとうとうはれつしてしまった 
\\	彼女の願いがとうとう実現された。	かのじょのねがいがとうとうじつげんされた 
\\	とうとう私達は山の頂上に到着した。	とうとうわたしたちはやまのちょうじょうにとうちゃくした 
\\	王子は魔法で木に変えられた。	おうじはまほうできにかえられた 
\\	少女はまるで魔法のように消え失せた。	しょうじょはまるでまほうのようにきえうせた 
\\	意地の悪い魔女が男にとんでもない魔法をかけて、虫に変えてしまった。	いじのわるいまじょがおとこにとんでもないまほうをかけて、むしにかえてしまった 
\\	もう一つお願いがある。とっとと消え失せろ。	もうひとつおねがいがある。とっとときえうせろ 
\\	その子は算数ならどんな問題でも解ける。	そのこはさんすうならどんなもんだいでもとける 
\\	算数は数を取り扱う。	さんすうはかずをとりあつかう 
\\	男と女は同等だと切に感じる。	おとことおんなはどうとうだとせつにかんじる 
\\	1メートルは1ヤードとは完全に同等ではない。	いちメートルはいちヤードとかんぜんにどうとうではない 
\\	時計の針を、逆に回してはいけない。	とけいのはりを、ぎゃくにまわしてはいけない 
\\	レンズを通ると像は逆になった。	レンズをとおるとぞうはぎゃくになった 
\\	薬によっては、役に立つどころか逆に害になるものもある。	くすりによっては、やくにたつどころかぎゃくにがいになるものもある 
\\	針で親指を突いてしまった。	はりでおやゆびをついてしまった 
\\	その部屋は針仕事ができるほど明るくない。	そのへやははりしごとができるほどあかるくない 
\\	彼は新しい電灯を止めるのに針金を使った。	かれはあたらしいでんとうをとめるのにはりがねをつかった 
\\	針は中心を一時間に10回転の割合で回る。	はりはちゅうしんをいちじかんにじゅうかいてんのわりあいでまわる 
\\	姉は頭の回転がいい。	あねはあたまのかいてんがいい 
\\	退屈するどころか、結構楽しかった。	たいくつするどころか、けっこうたのしかった 
\\	コーヒー通の人達に結構評判がいいそうです。	こーひーつうのひとたちにけっこうひょうばんがいいそうです 
\\	彼は嘘吐きで評判だ。	かれはうそつきでひょうばんだ 
\\	今度の市長は市民の評判がよい。	こんどのしちょうはしみんのひょうばんがよい 
\\	彼は高潔なことで非常に評判が高い。	かれはこうけつなことでひじょうにひょうばんがたかい 
\\	評判とは外見であり、人格とは人の本質である。	ひょうばんとはそとみであり、じんかくとはひとのほんしつである 
\\	このシンボルは強さと高潔さを表す。	このシンボルはつよさとこうけつさをあらわす 
\\	われわれの社会には、高潔な人もいれば、詐欺師もいる。	われわれのしゃかいには、こうけつなひともいれば、さぎしもいる 
\\	その計画は実にうまく考えた詐欺だった。	そのけいかくはじつにうまくかんがえたさぎだった 
\\	ジョージは詐欺にかかってその土地を買わされた。	ジョージはさぎにかかってそのとちうぃかわされた 
\\	詐欺師はことば巧みにやすやすと、女性を信頼させることができる。	さぎしはことばたくみにやすやすよ、じょせいをしんらいさせることができる 
\\	奴の巧みな話に僕は簡単に騙されてしまった。	やつのたくみなはなしにぼくはかんたんにだまされてしまった  
\\	現実の問題よりも、巧みな言葉での発言の方に関心が置かれていました。	げんじつのもんだいよりも、たくみなことばでのはつげんのほうにかんしんがおかれていました。 
\\	気温は連続して何日も氷点下だった。	きおんはれんぞくしてなんにちもひょうてんかだった 
\\	僕は芝居の自分の役の台詞を覚えた。	ぼくはしばいのじぶんのやくのせりふをおぼえた 
\\	テレビでは俳優が台詞を思い出せるようにキューカードが主に使われる。	テレビでははいゆうがせりふをおもいだせるようにキューカードがおもにつかわれる 
\\	その市長は市の金を横領した。	そのしちょうはしのかねをおうりょうした 
\\	この証拠で彼が横領者だということが分かった。	このしょうこでかれがおうりょうしゃだということがわかった 
\\	詐欺、横領などの犯罪によりアメリカ人は以前より政府を信じなくなった。	さぎ、おうりょうなどのはんざいによりアメリカじんはいぜんよりせいふをしんじなくなった 
\\	このレコード視聴できますか。	このレコードしちょうできますか 
\\	テレビは視聴者に娯楽ばかりではなく、知識も与える。	テレビはしちょうしゃにごらくばかりではなく、ちしきもあたえる 
\\	この映画は教育と娯楽を兼ねている。	このえいがはきょういくとごらくをかねている 
\\	私は勉強と遊びを兼ねている仕事に就きたい。	わたしはべんきょうとあそびをかねているしごとにつきたい 
\\	機知とユーモアを兼ね備えている。	きちとユーモアをかねそなえている 
\\	彼の講義は彼の機知に富んだジョークで終わった。	かれのこうぎはかれのきちにとんだジョークでおわった 
\\	それは蝶ですか、それとも蛾ですか。	それはちょうですか、それともがですか 
\\	それは偶然だったのか、それとも故意にであったのか。	それはぐうぜんだったのか、それともこいであったのか 
\\	彼の失敗は無能によるものかそれとも不運によるものか。	かれのしっぱいはむのうによるものかそれともふうんによるものか 
\\	彼女は故意に間違いの住所を私に教えた。	かのじょはこいにまちがいのじゅうしょをわたしにおしえた 
\\	彼は気の毒な程不運だ。	かれはきのどくなほどふうんだ 
\\	彼は身の不運を嘆き悲しんだ。	かれはみのふうんをなげきかなしんだ 
\\	彼女は夫の死を嘆き悲しむばかりであった。	かのじょはおっとのしをなげきかなしむばかりであった 
\\	彼の無能ぶりに誰もが苛立ち始めた。	かれのむのうぶりにだれもがいらだちはじめた 
\\	あらゆる芸術評論家達は無能であり危険な存在である。	あらゆるげいじゅつひょうろんかはむのうでありきけんなそんざいである 
\\	彼は苛立たしい表現で答えた。	かれはいらだたしいひょうげんでこたえた 
\\	娘はいつも、約束を守らない母親に苛立っていた。	むすめはいつも、やくそくをまもらないははおやにいらだっていた 
\\	最初の議題は、教育委員会によって提出された動議です。	さいしょのぎだいは、きょういくいいんかいによってていしゅつされたどうぎです 
\\	討論を継続することを動議します。	とうろんをけいぞくすることをどうぎします 
\\	彼は打ち切って、ひとまず休息しようという動議を出した。	かれはうちきって、ひとまずきゅうそくしようというどうぎをだした 
\\	この交渉を打ち切りたい。	このこうしょうをうちきりたい 
\\	両国は友好関係を打ち切った。	りょうこくはゆうこうかんけいをうちきった 
\\	彼は短い休息の後、仕事を再開した。	かれはみじかいきゅうそくのあと、しごとをさいかいした 
\\	医者は彼に休息をとるように命じた。	いしゃはかれにきゅうそくをとるようにめいじた 
\\	彼のつかの間の休息は彼女がきて中断された。	かれのつかのまのきゅうそくはかのじょがきてちゅうだんされた 
\\	老人の話は何度か咳で中断された。	ろうじんのはなしはなんどかせきでちゅうだんされた 
\\	ちょっと中断したあとで討議が再び始まった。	ちょっとちゅうだんしたあとでとうぎがふたたびはじまった 
\\	外で突然大きな物音がして私の瞑想は中断した。	そとでとつぜんおおきなものおとがしてわたしのめいそうはちゅうだんした 
\\	ハエと蚊が彼の瞑想を邪魔した。	はえとかがかれのめいそうをじゃました 
\\	彼は昨晩瞑想した時のイメージを描いた。	かれはさくばんめいそうしたときイメージをかいた 
\\	裁判は10日間継続して行われた。	さいばんはとおかかんけいぞくしておこなわれた 
\\	交渉を継続することで合意しました。	こうしょうをけいぞくすることでごういしました 
\\	科学者はその研究を継続することを強く要求した。	かがくしゃはそのけんきゅうをけいぞくすることをつよくようきゅうした 
\\	彼女の心臓は恐怖で早鐘を打った。	かのじょのしんぞうはきょうふではやがねをうった 
\\	彼女の胸は早鐘を打つようだ。	かのじょのむねははやがねをうつようだ 
\\	彼は小説を読んで徹夜した。	かれはしょうせつをよんでてつやした 
\\	彼女は病気の子供を徹夜で看病した。	かのじょはびょうきのこどもをてつやでかんびょうした 
\\	徹夜したから、私は今日は睡眠不足だ。	てつやしたから、わたしはきょうはすいみんぶそくだ 
\\	10代の友だち同士が徹夜でおしゃべりした。	じゅうだいのともだちどうしがてつやでおしゃべりした 
\\	碁のよい相手同士だった。	ごのよいあいてどうしだった 
\\	碁を打つことが一番の気晴らしだ。	ごをうつことがいちばんのきばらしだ 
\\	日本と韓国は、隣国同士だ。	にほんとかんこくは、りんごくどうしだ 
\\	その軍隊は隣国に侵入した。	そのぐんたいはりんごくにしんにゅうした 
\\	泥棒は2階から侵入したらしい。	どろぼうはにかいからしんにゅうしたらしい 
\\	医者は病気の老人を看病して夜を明かした。	いしゃはびょうきのろうじんをかんびょうしてよるをあかした 
\\	その王国は敵に侵入された。	そのおうこくはてきにしんにゅうされた 
\\	この国は隣国の支配下にあった。	このくにはりんごくのしはいかにあった 
\\	当時彼と私は味方同士だった。	とうじかれとわたしはみかたどうしだった 
\\	彼はいわゆる正義の味方だ。	かれはいわゆるせいぎにみかただ 
\\	その論争で我々は彼に味方した。	そのろんそうでわれわれはかれにみかたした 
\\	ポーシャは正義を慈悲で和らげた。	ポーシャはせいぎをじひでやわらげた 
\\	彼等は正義のために戦った。	かれらはせいぎのためにたたかった 
\\	その論争は完全に片付けた。	そのろんそうはかんぜんにかたづけた 
\\	彼の後継者についての問題はまだ論争中だ。	かれのこうけいしゃについてのもんだいはまだろんそうちゅうだ 
\\	彼の大胆な計画は大きな論争を巻き起こした。	かれのだいたんなけいかくはおおきなろんそうをまきおこした 
\\	国王の長男は、王座の後継者である。	こくおうのちょうなんは、おうざのこうけいしゃである 
\\	その土地は長男に与えられた。	そのとちはちょうなんにあたえられた 
\\	現在、チャンピオン・ベルトを締めているのは彼であり、その彼から王座を奪える若者はいないだろう。	げんざい、チャンピオン・ベルトをしめているのはかれであり、そのかれからおうざをうばえるわかものはいないだろう 
\\	泣くことは悲しみを和らげる。	なくことはかなしみをやわらげる 
\\	番組の一部は家庭向きにすこし刺激を和らげられた。	ばんぐみのいちぶはかていむきにすこししげきをやわらげられた 
\\	空手は武器を用いない護身術である。	からてはぶきをもちいないごしんじゅつである 
\\	科学は今世紀になって急速な進歩を遂げた。	かがくはこんせいきになってきゅうそくなしんぽをとげた 
\\	彼は過激とは言わないまでも非常に進歩的だ。	かれはかげきとはいわないまでもひじょうにしんぽてきだ 
\\	彼は、無礼だとはいわないまでも、礼儀正しくはなかった。	かれは、ぶれいだといわないまでも、れいぎただしくはなかった 
\\	彼はけちと言わないまでもとても倹約家だ。	かれはけちといわないまでもとてもけんやくかだ 
\\	収入が低いと倹約せざるを得なくなる。	しゅうにゅうがひくいとけんやくせざるをえなくなる 
\\	生活費は高い、だから私たちは倹約しなければならない。	せいかつひはたかい、だからわたしたちはけんやくしなければならない 
\\	政府は過激派グループの活動を注意深く監視した。	せいふはかげきはグループのかつどうをちゅういぶかくかんしした 
\\	一部の黒人はより過激な解決策を求める。	いちぶのこくじんはよりかげきなかいけつさくをもとめる 
\\	彼の解決策は一時的なものだ。	かれのかいけつさくはいちじてきなものだ 
\\	解決策がないのは明らかだった。	かいけつさくがないのはあきらかだった 
\\	私の家は地震に堪えるように設計されている。	わたしのいえはじしんにたえるようにせっけいされている 
\\	この奇妙なビルは一体誰が設計したのか。	ののきみょうなビルはいったいだれがせっけいしたのか 
\\	常識のある人なら、一体誰がそんなことを信じようか。	じょうしきのあるひとなら、いったいだれがそんなことをしんじようか 
\\	当面は彼に調子を合わせておいたほうがいいぞ。	とうめんはかれにちょうしをあわせておいたほうがいいぞ 
\\	ただ、調子を合わせているんじゃない。本当に、私の気持ちを分かっていて同情している目だった。	ただ、ちょうしをあわせているんじゃない、ほんとうに、わたしのきもちをわかっていてどうじょうしているめだった 
\\	彼らは惨めな犯人に同情した。	じかれらはみじめなはんにんにどうじょうした 
\\	同情は人間特有の感情である。	どうじょうはにんげんとくゆうのかんじょうである 
\\	同情と愛情を決して混同しないように。	どうじょうとあいじょうをけっしてこんどうしないように 
\\	厳密に言えば、中国は何百万という方言から成り立っている。	げんみつにいえば、ちゅうごくはなんびゃくまんというほうげんからなりたっている 
\\	合衆国は北半球にある。	がっしゅうこくはきたはんきゅうにある 
\\	合衆国は50の州から成り立っている。	がっしゅうこくはごじゅうのしゅうからなりたっている 
\\	赤道は地球を二つの半球に分ける。	せきどうはちきゅうをふたつのはんきゅうにわける 
\\	上の実験で示されていることは、脳の右半球がまったく使われていないということである。	うえのじっけんでしめされていることは、のうのみぎはんきゅうがまったくつかわれていないということである 
\\	私は彼の招待に応じた。	かれのしょうたいにおうじた 
\\	招待された事は大変な名誉です。	しょうたいされたことはたいへんなめいよです 
\\	彼は不名誉な称号を得た。	かれはふめいよなしょうごうをえた 
\\	大学は彼に名誉学位を与えた。	だいがくはかれにめいよがくいをあたえた 
\\	彼はサーの称号を持っている。	かれはサーのしょうごうをまっている 
\\	不名誉の中で生きるより殺された方がましだ。	ふめいよのなかでいきるよりころされたほうがましだ 
\\	ご要望には応じられません。	ごようぼうにはおうじられません 
\\	君の賃金は仕事量に応じて支払われる。	きみのちんぎんはしごとりょうにおうじてしはらわれる 
\\	彼は一日休みたいという彼女の要請を拒否した。	かれはいちにちやすみたいというかのじょのようせいをきょひした 
\\	彼らの要請に応じて学校側が動き出した。	かれらのようせいにおうじてがっこうがわがうごきだした 
\\	私は彼に知らせるように要請した。	わたしはかれにしらせるようにようせいした 
\\	犯罪捜査への協力を要請された。	はんざいそうさへのきょうりょくをようせいされた 
\\	容疑者は捜査官に嘘を言った。	ようぎしゃはそうさかんにうそをいった 
\\	捜査を立ち消えにして欲しくない。	そうさをたちきえにしてほしくない 
\\	中途半端にやるくらいなら、やらない方がましだ。	ちゅうとはんぱにやるくらいなら、やらないほうがましだ 
\\	屈辱を受けて生きるくらいなら死んだ方がましだ。	くつじょくをうけて生きるくらいならしんだほうがましだ 
\\	降伏するより死んだほうがましだ。	こうふくするよりしんだほうがましだ 
\\	自分の信念を隠すくらいなら死んだほうがましだ。	じぶんのしんねんをかくすくらいならしんだほうがましだ 
\\	彼は自発的に降伏した。	かれはじはつてきにこうふくした 
\\	降伏に変わるものは戦いのみ。	こうふくにかわるものはたたかいのみ 
\\	仕事を中途半端にするな。	しごとをちゅうとはんぱにするな 
\\	彼は中途で引き返した。	かれはちゅうとでひきかえした 
\\	君の言葉はほとんど屈辱に等しい。	きみのことばはほとんどくつじょくにひとしい 
\\	彼は甘んじて屈辱に耐えた。	かれはあまんじてくつじょくにたえた 
\\	彼は自分の運命に甘んじた。	かれはじぶんのうんめいにあまんじた 
\\	私は今の境遇に甘んじてはいない。	わたしはいまのきょうぐうにあまんじていない 
\\	女性は余りにも長い間不公平な待遇に甘んじてきた。	じょせいはあまりにもながいかんふこうへいなたいぐうにあまんじてきた 
\\	自分の運命に甘んじるくらいなら自殺した方がましだ。	じぶんのうんめいにあまんじるくらいならじさつしたほうがましだ 
\\	ぼくはこの点を特に待遇したい。	ぼくはこのてんをとくにたいぐうしたい 
\\	彼はもっとよい待遇を受ける権利がある。	かれはもっとよいたいぐうをうけるけんりがある 
\\	政治犯たちは待遇改善を求めてハンストをしています。	せいじはんたちはたいぐうかいぜんをもとめてハンストをしています 
\\	彼は自分の振る舞いを新しい境遇に合わせるよう努めた。	かれはじぶんのふるまいをあたらしいきょうぐうにあわせるようつとめた 
\\	そのような境遇であったにも関わらず、彼は自分一人で道を切り開いた。	そのようなきょうぐうであったにもかかわらず、かれはじぶんひとりでみちをきりひらいた 
\\	彼らは密林に道を切り開いた。	かれはみつりんにみちをきりひらいた 
\\	我々はどうにかしてその密林を通り抜けよう。	われわれはどうにかしてそのみつりんをとおりぬけよう 
\\	彼らは沼地を通り抜けた。	かれらはぬまちをとおりぬけた 
\\	船はパナマ運河を通り抜けた。	ふねはパナマうんがをとおりぬけた 
\\	この船は大きすぎて運河を通れない。	このふねはおおきすぎてうんがをとおれない 
\\	スエズ運河とパナマ運河は国際管理化におくべきでしょうか。	スエズうんがとパナマうんがはこくさいかんりかにおくべきでしょうか 
\\	砂漠に水を引くために運河が建設されている。	さばくにみずをひくためにうんががけんせつされている 
\\	沼地に建物は建てられない。	ぬまちにたてものをたてられない 
\\	この規則は外国人にのみ適用されます。	このきそくはがいこくじんにのみてきようされます 
\\	この種の植物は熱帯地方にのみ育ちます。	このしゅのしょくぶつはねったいちほうにのみそだちます 
\\	物質は温度によって形を変える。	ぶっしつはおんどによってかたちをかえる 
\\	会計は部屋代につけてください。	かいけいはへやだいにつけてください 
\\	その未亡人は多くの苦難を経験しなければならなかった。	そのみぼうじんはおおくのくなんをけいけんしなければならなかった 
\\	その子は猫の尻尾を掴まえた。	そのこはねこのしっぽをつかまえた 
\\	幸福は掴めるときに掴まなければならない。	こうふくをつかめるときにつかまなければならない 
\\	この机は堅い材質の木でできている。	このつくえはかたいざいしつのきでできている 
\\	休暇の予め計画を立てている。	きゅうかのあらかじめけいかくをたてている 
\\	予め断りますが、今日のブログは面白くないよ~。	あらかじめことわりますが、きょうのぶろぐはおもしろくないよ~ 
\\	予め警告を受けているのは、予め武装しているに同じ。	あらかじめけいこくをうけているのは、あらかじめぶそうしているにおなじ 
\\	その国は公然と核武装すると脅している。	そのくにはこうぜんとかくぶそうするとおどしている 
\\	四人の武装した男が銀行を襲って、四百万ドルを奪って逃げた。	よにんのぶそうしたおとこがぎんこうをおそって、よんひゃくまnドルをうばってにげた 
\\	私は使い古した注射器を安全に捨てるために缶の中に入れました。	わたしはつかいふるしたちゅうしゃきをあんぜんにすてるためにかんのなかにいれました 
\\	医者に行って注射を打ってもらったほうがいいよ。	いしゃにいってちゅうしゃをうってもらったほうがいいよ 
\\	麻酔の注射をします。	ますいのちゅうしゃをします 
\\	怪我人は麻酔から覚めた後痛みで泣き喚いた。	けがにんはますいからさめたあといたみでなきわめいた 
\\	子供はおもちゃの車が欲しいと泣き喚いた。	こどもはおもちゃのくるまがほしいとなきわめいた 
\\	衣服を使い古したらそれをどうしますか。	いふくをつかいふるしたらそれをどうしますか 
\\	親は子供に適切な食事と衣服を与えなければならない。	おやはこどもにてきせつなしょくじといふくをあたえなければならない 
\\	私たちは彼らに食料と衣服を供給した。	わたしたちはかれらにしょくりょうといふくをきょうきゅうした 
\\	内線214に出る人なら誰でも構いません。	ないせんびいちよんにでるひとだれでもかまいません 
\\	喫煙は今では、全ての国内線のフライトで禁止されている。	きつえんはいまでは、すべてのこくないせんのフライトできんしされている 
\\	交換手です。内線番号をどうぞ。	こうかんしゅです。ないせんばんごうをどうぞ 
\\	五ドル札を崩せますか。	ごドルさつをくずせますか 
\\	彼女は落ち着きを保とうと努めた。	かのじょはおちつきをたもとうとつとめた 
\\	今度の秘書は有能とは思えない。	こんどのひしょはゆうのうとはおもえない 
\\	彼は予想以上に有能な弁護士だ。	かれはよそういじょうにゆうのうなべんごしだ 
\\	彼女は有能で頼りになる助手だ。	かのじょはゆうのうでたよりになるじょしゅだ 
\\	彼女は魅力的で頼りになる人です。	かのじょはみりょくてきでたよりになるひとです 
\\	予想以上の多くの観客が来ていました。	よそういじょうのおおくのかんきゃくがきていました 
\\	左肩の関節が痛みます。	ひだりかたのかんせつがいたみます 
\\	気温が下がると関節が痛むんだ。	きおんがさがるとかんせつがいたむんだ 
\\	レストランの不潔な皿に私達は吐き気を催させられた。	レストランのふけつなさらにわたしたちははきけをもよおさせられた 
\\	講義の途中で彼女は吐き気を催した。	こうぎのとちゅうでかのじょははきけをもよおした 
\\	大量の煙を吐き出している火山。	たいりょうのけむりをはきだしているかざん 
\\	突然の吐き気が彼は抑えきれなかった。	とつぜんのはきけがかれはおさえきれなかった 
\\	紳士なら、道に唾など吐かないだろう。	しんしなら、みちにつばなどはかないだろう 
\\	彼の唾の吐き方が我慢できない。	かれのつばのはきかたががまんできない 
\\	シンガポールでは道路に唾を吐くのは犯罪とされる。	シンガポールではどうろにつばをはくのははんざいとされる 
\\	彼には感謝しきれない。	かれはかんしゃしきれない 
\\	台風は計りきれないほどの損害を齎した。	たいふうははかりきれないほどのそんがいをもたらした
\\	商業は都市の発展を齎した。	しょうぎょうはとしのはってんをもたらした 
\\	春の接近は暖かい天候を齎す。	はるのせっきんはあたたかいてんこうをもたらす 
\\	コンピューターの仕事から金融、会計などの仕事に転向することは可能である。	コンピューターのしごとからきんゆう、かいけいなどのしごとにてんこうすることはかのうである 
\\	領収書は必ず保管してください。	りょうしゅうしょはかならずほかんしてください 
\\	私は預けたスーツケースを取りに手荷物保管所へと向かって行った。	わたしはあずけたスーツケースをとりにてにもつほかんしょへむかっていった 
\\	私は鞄と傘を一時預かり室に預けた。	わたしはかばんとかさをいちじあずかりしつにあずけた 
\\	その男性は彼女の鞄を奪い取った。	そのだんせいはかのじょのかばんをうばいとった 
\\	直美は鞄をもう一方の手に移した。	なおみはかばんをもういっぽうのてにうつした 
\\	病気が彼の体力を奪い取った。	びょうきがかれのたいりょくをうばいとった 
\\	それにしても、幼稚園というところは、季節の行事にとても敏感です。	それにしても、ようちえんというところは、きせつのぎょうじにとてもびんかんです 
\\	鮫は音だけでなく電気の刺激にも敏感である。	さめはおとだけでなくでんきのしげきにもびんかんである 
\\	行事を記念してパレードが行われた。	ぎょうじをきねんしてパレードがおこなわれた 
\\	その運動会は毎年行われる行事だ。	そのうんどうかいはまいとしおこなわれるぎょうじだ 
\\	あなたはいつも気取っているし、いつも警戒している。	あなたはいつもきどっているし、いつもけいかいしている 
\\	そんなに気取る必要はない。	そんなにきどるひつようはない 
\\	ヘリコプターは垂直に離陸し、着陸することができる。	ヘリコプターはすいちょくにりりくし、ちゃくりくすることができる 
\\	月着陸は記念すべき偉業であった。	つきちゃくりくはきねんすべきいぎょうであった 
\\	宇宙船は完璧な着陸をした。	うちゅうせんはかんぺきなちゃくりくをした 
\\	君の答えは完璧には程遠い。	きみのこたえはかんぺきにはほどとおい 
\\	マユコはいつも完璧を目指している。	マユコはいつもかんぺきをめざしている 
\\	ホノルル着陸のため降下を始めます。	ホノルルちゃくりくのためこうかをはじめます 
\\	その飛行機は鋭い音を立ててほとんど直角に降下した。	そのひこうきはするどいおとをたててほとんどちょっかくにこうかした 
\\	離陸するジェット機の音が私の神経にさわる。	りりくするジェットきのおとがわたしのしんけいにさわる 
\\	サクラの話し方は私の神経に障る。	サクラのはなしかたはわたしのしんけいにさわる 
\\	彼女の行動は本当に私の神経に障った。	かのじょのこうどうはほんとうにわたしのしんけいにさわった 
\\	神経は抜くのですか。	しんけいはぬくのですか 
\\	我々は彼の偉業を幾ら高く評価してもしすぎることはない。	われわれはかれのいぎょうをいくらたかくひょうかしてもしすぎることはない 
\\	彼らのエベレスト登頂は偉業であった。	かれらのイベレストとうちょうはいぎょうであった 
\\	その登山家たちは登頂を成し遂げたが無事に帰れなかった。	そのとざんかたちはとうちょうをなしとげたがぶじにかえれなかった 
\\	彼の足は鰐に噛まれた。	かれのあしはわににかまれた 
\\	小屋は50メートルの間隔で建てられた。	こやはごじゅうメートルのかんかくでたてられた 
\\	その島の住民に新しい法律が施行された。	そのしまのじゅうみんにあたらしいほうりつがしこうされた 
\\	彼がいないことは恐らくお気づきのことでしょう。	【おそらく】 確度の高い推量を表す語。きっと。
\\	恐らく彼は弁論大会に優勝するだろう。	おそらくかれはべんろんたいかいにゆうしょうするだろう 
\\	今度の日曜日に弁論大会が開催される。	こんどのにちようびにべんろんたいかいがかいさいされる 
\\	生徒たちは恐がって素直に意見を述べられない。	せいとたちはこわがってすなおにいけんをのべられない 
\\	そういう発言は誤解を招きやすい。	そういうはつげんはごかいをまねきやすい 
\\	彼は粗野な言葉づかいのせいで誤解されている。	かれはそやなことばづかいのせいでごかいされている 
\\	ビールは麦芽から醸造される。	ビールはばくがからじょうぞうされる 
\\	彼女は彼に醸造所を案内して見せてくれた。	かのじょはかれにじょうぞうしょをあんないしてみせてくれた 
\\	与党は腐敗している、しかし野党だって同じようなものだ。	よとうはふはいしている、しかしやとうだっておなじようなものだ 
\\	その社会学者は背中を掻く癖がある。	そのしゃかいがくしゃはせなかをかくくせがある 
\\	彼は背中を掻いたり、爪を噛んだりする癖がある。	かれはせなかをかいたり、つめをかんだりするくせがある 
\\	水虫が痒いのです。	みずむしがかゆいのです 
\\	そこらじゅう蚊に刺されて痒くてしょうがない。	そこらじゅうかにさされてかゆくてしょうがない 
\\	彼は将来指導者になる素質がある。	かれはしょうらいしどうしゃになるそしつがある 
\\	彼の葬式には大勢の人が参列した。	かれのそうしきにはおおぜいのひとがさんれつした 
\\	たくさんの人が新しい橋の開通式に参列した。	たくさんのひとがあたらしいはしのかいつうしきにさんれつした 
\\	不通区間の開通の見込みはまだない。	ふつうくかんのかいつうのみこみはまない 
\\	このバスは一区間90円です。	このバスはいっくかんきゅうじゅうえんです 
\\	国勢調査の結果は左派に自己の政策が誤っていたことを信じさせるに至った。	こくせいちょうさのけっかはさはにじこのせいさくがあやまっていたことをしんじさせるにいたった 
\\	その語は広い意味を持つに至った。	そのごはひろいいみをもつにいたった 
\\	今のところ、全体の意見の一致には至っていない。	いまのところ、ぜんたいのいけんのいっちにはいたっていない 
\\	合衆国では10年に一度国勢調査が行われる。	がっしゅうこくではじゅうねんにいちどこくせいちょうさがおこなわれる 
\\	委員会はまだ決議に至らない。	いいんかいはまだけつぎにいたらない 
\\	新しく道路を作るという決議が可決されました。	あたらしくどうろをつくるというけつぎがかけつされました 
\\	私は委員会の決議を支持した。	わたしはいいんかいのけつぎをしじした 
\\	その税法案は昨日可決された。	そのぜいほうあんはきのうかけつされた 
\\	法案はまず可決されないだろう。	ほうあんはまずかけつされないだろう 
\\	大事に至る前に火事は消し止められた。	だいじにいたるまえにかじはけしとめられた 
\\	70年もしくは80年が人間の普通の寿命期間である。	どちらか一方を選択するのに用いる語。あるいは。さもなければ。または。
\\	私はアメリカ滞在の期間を延長したい。	わたしはアメリカたいざいのきかんをえんちょうしたい 
\\	滞在目的は何ですか。	たいざいもくてきはなんですか 
\\	彼はロンドンに滞在中に、大英博物館を訪れた。	かれはロンドンにたいざいちゅうにだいえいはくぶつかんをおとずれた 
\\	彼は日本語を習得するという目的でここに滞在している。	かれはにほんごをしゅうとくするというもくてきでここにたいざいしている 
\\	降参するくらいなら死んだ方がましだ。	こうさんすrくらいならしんだほうがましだ 
\\	双方が降参しようとしなかったので、長い戦争となった。	そうほうがこうさんしようとしなかったので、ながいせんそうとなった 
\\	誘拐犯は降参する気配を見せなかった。	ゆうかいはんはこうさんするけはいをみせなかった 
\\	この件についてのあなたのお求めを了承します。	このけんについてのあなたのおもとめをりょうしょうします 
\\	明後日いらして下さい。	あさっていらしてください 
\\	魚の骨がつかえて息が詰まりそうだった。	さかなのほねがつかえていきがつまりそうだった 
\\	つまり彼は大変な才能の持ち主なのだ。	つまりかれはたいへんなさいのうのもちぬしなのだ 
\\	外国人の一団が江戸、つまり東京に到着した。	がいこくじんのいちだんがえど、つまりとうきょうにとうちゃくした 
\\	金と私とは他人同士だ。つまり、貧しいのだ。	かねとわたしはたにんどうしだ。つまり、まずしいのだ 
\\	かのローマの偉大な英雄、つまりシーザーは暗殺されたのです。	かのローマのいだいなえいゆう、つまりシーザーはあんさつされたのです 
\\	彼らは大統領の暗殺を企てた。	かれらはだいとうりょうのあんさつをくわだてた 
\\	年配の人はまだケネディー暗殺事件を覚えている。	ねんぱいのひとはまだケネディーあんさつじけんをおぼえている 
\\	彼女はもっと分別があってよい年配だ。	かのじょはもっとぶんべつがあってよいねんぱい 
\\	私にはそれだけの余裕がなかった。つまり、貧しくて買えなかったのだ。	わたしはそれだけのよゆうがなかった。つまり、まずしくてかえなかったのだ 
\\	日本では米が恒常的に供給過剰である。	にほんではこめがこうじょうてきにきょうきゅうかじょうである 
\\	イングランドとスコットランドの間の恒常的な国境戦争は終わった。	イングランドとスコットランドのあいだのこうじょうてきなこっきょうせんそうはおわった 
\\	輸出は国境を越えた商業活動のひとつである。	ゆしゅつはこっきょうをこえたしょうぎょうかつどうのひとつである 
\\	自己の方針をあくまで守れ。	じこのほうしんをあくまでまもれ 
\\	困難に対してあくまでも抵抗した。	こんなんにたいしてあくまでもていこうした 
\\	彼はあくまでも沈黙を守ろうとした。	かれはあくまでもちんもくをまもろうとした 
\\	彼らは新方針を採用した。	かれらはしんほうしんをさいようした 
\\	彼の議論はわれわれの方針と矛盾している。	かれのぎろんはわれわれのほうしんとむじゅんしている 
\\	当店の方針はお客様に御満足いただくことです。	とうてんのほうしんはおきゃくさまにごまんぞくいただくことです 
\\	彼は会社の方針に抗議して辞表を出した。	かれはかいしゃのほうしんにこうぎしてじひょうをだした 
\\	彼は上司に辞表を提出した。	かれはじょうしにじひょうをていしゅつした 
\\	彼はその新方法を採用した。	かれはそのしんほうほうをさいようした 
\\	その会社は社員を季節的に採用する。	そのかいしゃはしゃいんをきせつてきにさいようする 
\\	ドイツは1880年代に社会保障制度を採用した。	ドイツは1980ねんだいにしゃかいほしょうせいどをさいようした 
\\	お名前と社会保障番号をおっしゃって下さい。	おなまえとしゃかいほしょうばんごうをおっしゃってください 
\\	踊り手たちは舞台を横切って軽々と踊っていった。	おどりてたちはぶたいをよこぎってかるがるとおどっていった 
\\	彼の言動は常に矛盾していた。	かれのげんどうはつねにむじゅんしていた 
\\	結果は理論に矛盾しないだろう。	けっかはりろんにむじゅんしないだろう 
\\	彼の昨日の発言は先週の発言と矛盾していた。	かれのきのうのはつげんはせんしゅうのはつげんとむじゅんしていた 
\\	我々の外交と戦略はあきらかに矛盾していた。	われわれのがいこうとせんりゃくはあきらかにむじゅんしていた 
\\	その会社の急速な成長はその独特な戦略によるものだった。	そのかいしゃのきゅうそくなせいちょうはそのどくとくなせんりゃくによるのだった 
\\	彼は行動も言動も田舎もんそのものだよ。	かれはこうどうもげんどうもいなかもんそのものだよ 
\\	返却日を変更する事ができますか。	へんきゃくひをへんこうすることができますか 
\\	原始社会では、物々交換が行われた。	げんししゃかいではぶつぶつこうかんがおこなわれた 
\\	探検家は現地人たちと物々交換をして食料を手に入れた。	たんけんかはげんちじんたちとぶつぶつこうかんをしてしょくりょうをてにいれた 
\\	その人類学者は原始文化に関する講演をした。	そのじんるいがくしゃはげんしぶんかにかんするこうえんをした 
\\	講師は公害問題について講演した。	こうしはこうがいもんだいについてこうえんした 
\\	講演者の議論は的外れであった。	こうえんしゃのぎろんはまとはずれであった 
\\	騒音公害を減らす。	そうおんこうがいをへらす 
\\	彼は率先して公害と戦った。	かれはそっせんしてこうがいとたたかった 
\\	彼はかなり仕事が出来るが率先力にかける。	かれはかなりしごとができるがそっせんりょくにかける 
\\	彼は率先して計画を実行した。	かれはそっせんしてけいかくをじっこうした 
\\	子供たちはキャンプに行くと、原始的な生活を楽しむ。	こどもたちはキャンプにいくと、げんしてきなせいかつをたのしむ 
\\	一人で原始林へ乗り込もうとは君はなんて勇ましいんだ。	ひとりでげんしりんへのりこもうとはきみはなんていさましいんだ 
\\	彼女は勇ましかった。	かのじょはいさましかった 
\\	火星には生物の形跡はない。	かせいにはせいぶつのけいせきはない 
\\	火星には嘗て原始的な生物が存在していたと彼は力説した。	かせいにはかつてげんしてきなせいぶつがそんざいしていたとかれはりきせつした 
\\	彼女は教育の重要性を力説した。	かのじょはきょういくのじゅうようせいをりきせつした 
\\	教師は生徒達が勇気を持つべきだとしばしば力説する。	きょうしはせいとたちがゆうきをもつべきとしばしばりきせつする 
\\	その島に人がいる形跡はなかった。	そのしまにひとがいるけいせきはなかった 
\\	ビールは泡立ってコップから溢れた。	ビールはあわだってコップからあふれた 
\\	彼女はデザート用にクリームを泡立てた。	かのじょはデザートようにクリームをあわだてた 
\\	彼と議論するのは骨折り損だ。	かれとぎろんするのはほねおりぞんだ 
\\	彼らの努力は骨折り損に終わった。	かれらのどりょくはほねおりぞんにおわった 
\\	非常に微妙な状況だった。	ひじょうにびみょうなじょうきょうだった 
\\	僕は事故を処理した。	ぼくはじこをしょりした 
\\	その警察官はその混雑をうまく処理できた。	そのけいさつかんはそのこんざつをうまくしょりできた 
\\	塵処理は当局の主な頭痛の種となっている。	ゴミしょりはとうきょくのおもなずつうのしゅとなっている 
\\	大統領にはそれらの問題を処理する能力がある。	だいとうりょうにはそれらのもんだいをしょりするのうりょくがある 
\\	もっと経験のある弁護士なら、その件は違ったやり方で処理しただろう。	もっとけいけんのあるべんごしなら、そのけんはちがったやりかたでしょりしただろう 
\\	当局は自国の通貨を何とか安定させた。	とうきょくはじこくのつうかをなんとかあんていさせた 
\\	当局は事実を大衆から隠してきた。	とうきょくはじじつのたいしゅうからかくしてきた 
\\	当局は彼の死についての疑惑を解き明かそうとしている。	とうきょくはかれのしについてのぎわくをときあかそうとしている 
\\	教授は遂にその問題を解き明かした。	きょうじゅはついにそのもんだいをときあかした 
\\	彼の科学的な発見は多くの謎を解き明かした。	かれのかがくてきなはっけんはおおくのなぞをときあかした 
\\	殺人事件の背後に潜むミステリーを解き明かせ。	さつじんじけんのはいごにひそむミステリーをときあかせ 
\\	盗賊が暗い戸口に潜んでいた。	とうぞくがくらいとぐちにひそんでいた 
\\	彼女は声を潜めた。	かのじょはこえをひそめた 
\\	此処では盗賊にご用心。	ここではとうぞくにごようじん 
\\	盗賊の一団が一行に襲い掛かった。	とうぞくのいちだんがいっこうにおそいかかった 
\\	面と向かって誉める人は用心しなさい。	めんとむかってほめるひとはようじんしなさい 
\\	私はだれかが背後から肩をたたいているのを感じた。	わたしはだれかがはいごからかたをたたいているのをかんじた 
\\	彼女は背後から自分の名前を呼ばれてびっくりしているようであった。	かのじょははいごからじぶんのなまえをよばれてびっくりしているようであった 
\\	真っ直ぐ前方を見てごらん。	まっすぐぜんぽうをみてごらん 
\\	奥さんがいると仮定してごらん。	おくさんがいるとかていしてごらん 
\\	此処での事故の数が公式に記録されているものの2倍あると仮定してみよう。	ここでのじこのかずがこうしきにきろくされているもののにばいあるとかていしてみよう 
\\	この報道は公式のものだ。	このほうどうはこうしきのものだ 
\\	私は核実験に反対だと喜んで公式に言明する。	わたしはかくじっけんにはんたいだとよろこんでこうしきにげんめいする 
\\	首相は国民の合意がなければ新税は導入しないと言明した。	しゅしょうはこくみんのごういがなければしんぜいはどうにゅうしないとげんめいした 
\\	どの大手のデパートも売上が落ちてきた。	どのおおてのデパートもうりあげがおちてきた 
\\	売り上げを伸ばそうと焦らなくてもいい。	うりあげをのばすそうとあせらなくてもいい 
\\	売り上げが増えるにつれて収益も上がる。	うりあげがふえるにつれてしゅうえきもあがる 
\\	彼らは大手レコード会社と三年契約を結んだ。	かれらはおおてレコードかいしゃとさんねんけいやくをむすんだ 
\\	彼らの努力は実を結ばなかった。	かれらのどりょくはみをむすばなかった 
\\	今は焦らずに時を待つべきだ。	いまはあせらずにときをまつべきだ 
\\	その会社は経営に日本式のやり方を導入した。	そのかいしゃはけいえいににほんしきのやりかたをどうにゅうした 
\\	その庭園は日本式に設定されている。	そのていえんはにほんしきにせっていされている 
\\	大手銀行の大半がこの制度を導入している。	おおてぎんこうのたいはんがこのせいどをどうにゅうしている 
\\	その戦争こそが日本を変えた。	そのせんそうこそがにほんをかえた 
\\	彼こそクラスの中で一番の発明家です。	かれこそクラスのなかでいちばんのはつめいかです 
\\	もうじき我々の食糧は尽きてしまうでしょう。	もうじきわれわれのしょくりょうはつきてしまうでしょう 
\\	早晩彼の運は尽きるだろう。	そうばんかれのうんはつきるだろう 
\\	お前には全く愛想が尽きる。	おまえにはまったくあいそうがつきる 
\\	毎晩飲み歩いてばかりいたら、奥さんに愛想尽かされるぞ。	まいばんのみあるいてばかりいたら、おくさんにあいそうつかされるぞ 
\\	早晩私たちはその問題に本気で取組まざるを得ないだろう。	そうばんわたしたちはそのもんだいにほんきでとりくまざるをえないだろう 
\\	環境汚染にいかに取り組むかは深刻な問題である。	かんきょうおせんにいかにとりくむかはしんこくなもんだいである 
\\	彼はいかにも留学したような事を言う。	かれはいかにもりゅうがくしたようなことをいう 
\\	遅刻するのはいかにも彼らしい。	ちこくするのはいかにもかれらしい 
\\	社長が社員の主体的な取り組みを促した。	しゃちょうがかいしゃいんのしゅたいてきなとりくみをうながした 
\\	学生は自分の主体性を見失ってはならない。	がくせいはじぶんのしゅたいせいをみうしなってはならない 
\\	人々は次第に本来の目的を見失うだろう。	ひとびとはしだいにほんらいのもくてきをみうしなうだろう 
\\	その男は人込みの中で見失われた。	そのおとこはひとごみのなかでみうしなわれた 
\\	彼女はとても愛想のよい隣人だ。	かのじょはとてもあいそうのよいりんじんだ 
\\	食糧不足のため、家畜が餓死した。	しょくりょうぶそくのため、かちくががしした 
\\	中米では森林が家畜の牧場に取って代わられている。	ちゅうべいではしんりんがかちくのぼくじょうにとってかわられている 
\\	都会人は田舎の人よりも死亡率が高い。	とかいじんはいなかのひとよりもしぼうりつがたかい 
\\	バスが崖から転落し、乗っていた10名全員が死亡した。	バスががけからてんらくし、のっていたじゅうめいぜんいんがしぼうした 
\\	因みにホイールマウスのホイールにも対応しているんだ。	ちなみにホイールマウスのホイールにもたいおうしているんだ 
\\	その本を汚さないように扱ってくれるなら、貸してあげるよ。	そのほんをよごさないようにあつかってくれるなら、かしてあげるよ 
\\	総理大臣が昨日辞職した。	そうりだいじんがきのうじしょくした 
\\	彼は総理大臣になりたいという野心を抱いた。	かれはそうりだいじんになりたいというやしんをいだいた 
\\	総理大臣は、明日、声明を発表する予定です。	そうりだいじんは、あした、せいめいをはっぴょうするよていです 
\\	彼らは共同声明に同意した。	かれらはきょうどうせいめいにどういした 
\\	本当はその声明は彼の個人的見解にすぎない。	ほんとうはそのせいめいはかれにこじんてきけんかいにすぎない 
\\	その声明では改革が必要だといっていた。	そのせいめいではかいかくがひつようだといっていた 
\\	セールスマンは挽肉機の使い方を操作して説明した。	セールスマンはひきにくきのつかいかたをそうさしてせつめいした 
\\	彼は受話器を取り上げた。	かれはじゅわきをとりあげた 
\\	彼は受話器を下に置いた。	かれはじゅわきをしたにおいた 
\\	私は受話器を耳に当てた。	わたしはじゅわきをみみにあてた 
\\	寒波が日本の上空を通過したのです。	かんぱがにほんのじょうくうをつうかしたのです 
\\	我々は、39、000フィートの上空を航行中です。	われわれはさんまんきゅうせんフィートのじょうくうをこうこうちゅうです 
\\	風に逆らって航行する。	かぜにさからってこうこうする 
\\	船は海岸沿いに航行していた。	ふねはかいがんぞいにこうこうしていた 
\\	この地方を寒波が襲った。	このちほうをかんぱがおそった 
\\	彼らは兵士として前線に行った。	かれらはへいしとしてぜんせんにいった 
\\	ポールは潤いのない髪がある。	ポールはうるおいのないかみがある 
\\	サブロンは肌に潤いを与えるクリームです。	サブロンははだにうるおいをあたえるクリームです 
\\	犬は確かに家庭に何か潤いを与えているね。	いぬはたしかにかていになにかうるおいをあたえているね 
\\	お前の世界へ光を齎す。	おまえのせかいへひかりをもたらす 
\\	教育は進歩を齎す力だ。	きょういくはしんぽをもたらすちからだ 
\\	お金が幸福をもたらすとは限らない。	おかねがこうふくをもたらすとはかぎらない 
\\	戦争はその街に死と破壊を齎した。	せんそうはそのまちにしとはかいをもたらした 
\\	一生懸命することは君に勝利を齎す。	いっしょうけんめいすることはきみにしょうりをもたらす 
\\	彼が齎した知らせを聞いて私たちは大喜びをした。	かれがもたらしたしらせをきいてわたしたちはおおよろこびをした 
\\	自らの義務を果たすべきだ。	みずからのぎむをはたすべきだ 
\\	日本は自らの経済成長を発展させた。	にほんはみずからのけいざいせいちょうをはってんさせた 
\\	ここ2週間鼻声が抜けません。	ここにしゅうかんはなごえがぬけません 
\\	森を抜ける小道があります。	もりをぬけるこみちがあります 
\\	彼の策略を見抜けなかった。	かれのさくりゃくをみぬけなかった 
\\	電話は彼を家から出すための策略だった。	でんわはかれをいえからだすためのさくりゃくだった 
\\	ニューヨーク滞在中にはお世話になりました。	ニューヨークたいざいちゅうにはおせわになりました 
\\	その用件は明日まで待てますか。	そのようけんはあしたまでまてますか 
\\	挨拶抜きでいきなり用件を切り出す。	あいさつぬきでいきなりようけんをきりだす 
\\	彼女は親から経済的に自立している。	かのじょはおやからけいざいてきにじりつしている 
\\	諸君全員がそれを読まなければならない。	しょくんぜんいんがそれをよまなければならない 
\\	諸君、ちょっと御挨拶申し上げます。	しょくん、ちょっとごあいさつもうしあげます 
\\	あんなに美人なんだから、彼女も優越感を感じているんだろうな、きっと。	あんなにびじんなんだから、かのじょもゆうえつかんをかんじているんだろうな、きっと 
\\	窓の破損料を請求された。	まどのはそんりょうをせいきゅうされた 
\\	小包と一緒に請求書が送られてきた。	こづつみといっしょにせいきゅうしょがおくられてきた 
\\	彼は友達を作るコツを知っている。	かれはともだちをつくるコツをしっている 
\\	彼女は先端を行っています。	かのじょはせんたんをおこなっています 
\\	表面に現れているのは氷山の先端に過ぎない。	ひょうめんにあらわれているのはひょうざんのせんたんにすぎない 
\\	氷山が海岸に打ち上げられていた。	ひょうざんがかいがんにうちあげられていた 
\\	北風は明らかに氷山から吹き出す。	きたかぜはあきらかにひょうざんからふきだす 
\\	人工衛星が軌道に向けて打ち上げられた。	じんこうえいせいがきどうにむけてうちあげられた 
\\	宇宙船は月を回る軌道を外れている。	うちゅせんはつきをまわるきどうをはずれている 
\\	私の事業もようやく軌道に乗りました。	わたしのじぎょうもようやくきどうにのりました 
\\	ようやく彼らは和解した。	ようやくかれらはわかいした 
\\	彼らは明日、人工衛星を発射するつもりです。	かれらはあした、じんこうえいせいをはっしゃするつもりです 
\\	現在では我々が作った人工衛星が地球の周辺を回転している。	げんざいではわれわれがつくったじんこうえいせいがちきゅうのしゅうへんをかいてんしている 
\\	遠くで銃の発射される音が聞こえた。	とおくでじゅうのはっしゃされたおとがきこえた 
\\	ロケットの発射は予定通り行なわれた。	ロケットのはっしゃはよていどおりおこなわれた 
\\	ランドリーには洗剤がありますか。	ランドリーにはせんざいがありますか 
\\	工場は住宅地域には相応しくない。	こうじょうはじゅうたくちいきにはふさわしくない 
\\	彼はわがチームの主将に相応しい。	かれはわがチームのしゅしょうにふさわしい 
\\	彼は理事なので、それに相応しい扱いを受けるべきである。	かれはりじなので、それにふさわしいあつかいをうけるべきである 
\\	主将の入院中は私が代理を務めた。	しゅしょうのにゅういんちゅうはわたしがだいりをつとめた 
\\	彼らは主将のいないところで悪口を言った。	かれらはしゅしょうのいないところでわるぐちをいった 
\\	理事会を開催しなければならない。	りじかいをかいさいしなければならない 
\\	その人達は大胆な発想をすべきだ。	そのひとたちはだいたんなはっそうをすべきだ 
\\	彼が花嫁の父親です。	かれははなよめのちちおやです 
\\	嫁にやらなくてはならない娘が3人いる。	よめにやらなくてはならないむすめがさんにんいる 
\\	花嫁姿の彼女は一段と美しかった。	はなよめすがたのかのじょはいちだんとうつくしかった 
\\	ビザの延長をお願いします。	ビザのえんちょうをおねがいします 
\\	宿泊をもう一晩延長できますか。	しゅくはくをもうひとばんえんちょうできますか 
\\	彼は私の遠い親類です。	かれはわたしのとおいしんるいです 
\\	親類も足が遠のきはじめた。	しんるいもあしがとおのきはじめた 
\\	その一家の財産は親類の間で分けられた。	そのいっかのざいさんはしんるいのあいだでわけられた 
\\	君を永遠に愛します。	きみをえいえんにあいします 
\\	その画家は、いわば永遠の少年だ。	そのがかは、いわばえいえん おしょうねんだ 
\\	完璧な平和を。 永遠なる平和を祈ろう。	かんぺきなへいわを。えいえんなるへいわをいのろう 
\\	繁栄が永遠には続かないことを知っておかなければいけない。	はんえいがえいえんにはつづかないことをしっておかなければいけない 
\\	彼は1日ごとに日程を変える。	かれはいちにちごとににっていをかえる 
\\	我々は日程を考慮に入れるべきだった。	われわれはにっていをこうりょにいれるべきだった 
\\	あす雨の場合は別の日程を組まなくちゃ。	あすあめのばあいはべつのにっていをくまなくちゃ 
\\	部長の都合が悪くなってしまったので、飲み会の日程は仕切り直しだね。	ぶちょうのつごうがわるくなってしまったので、のみかいのにっていはしきりなおしだね 
\\	研修会は午後4時開始の予定。	けんしゅうかいはごごよじかいしのよてい 
\\	全ての研修生は辛い仕事の苦労を分かち合っている。	すべてのけんしゅうせいはつらいしごとのくろうをわかちあっている 
\\	従業員に対する新コンピューター・システム研修があなたの仕事になります。	じゅぎょういんにたいするあたらしいコンピューター・システムけんしゅうがあなたのしごとになります 
\\	飢饉に直面してもあなたは食べ物を他の人と分かち合うことができますか。	ききんにちょくめんしてもあなたはたべものをほかのひととわかちあうことができますか 
\\	私の勤めは常に念頭にある。	わたしのつとめはつねにねんとうにある 
\\	彼女はあまり夢中になっていて周囲のことなど念頭になかった。	かのじょはあまりむちゅうになっていてしゅういのことなどねんとうになかった 
\\	私は炊事が全然できない。	わたしはすいじがぜんぜんできない 
\\	それは教育に次いで大きな問題だ。	それはきょういくについでおおきなもんだいだ 
\\	選挙公約を果たしてくれない政治家がいる。	せんきょこうやくをはたしてくれないせいじかがいる 
\\	給与支給にはそのほうが良い、とポーラ・グレイソンさんは言っています。	きゅうよしきゅうにはそのほうがよい、とポーラ・グレイソンさんはいっています。 
\\	彼らは被災者に食糧を支給した。	かれらはひさいしゃにしょくりょうをしきゅうした 
\\	この学校は、生徒に教科書を支給します。	このがっこうは、せいとにきょうかしょをしきゅうします 
\\	水害被災者たちは数校に収容された。	すいがいひさいしゃたちはすうこうにしゅうようされた 
\\	彼女は、一昨年よりもはるかに暮らし向きがよい。	かのじょは、いっさくねんよりもはるかにくらしむきがよい 
\\	私は足し算は好きだけど引き算は嫌いだ。	わたしはたしざんはすきだけどひきざんはきらいだ 
\\	子供達は、足し算と引き算を習っている。	こどもたちは、たしざんとひきざんをならっている 
\\	足し算を非常に早くすることは可能だ。	たしざんをひじょうにはやくすることはかのうだ 
\\	この商品は免税品です。	このしょうひんはめんぜいひんです 
\\	外国人旅行者には免税の特権がある。	がいこくじんりょこうしゃにはめんぜいのとっけんがある 
\\	彼らは、独裁者と戦った。	かれらは、どくさいしゃとたたかった 
\\	ヒトラーは悪名高い独裁者の一人です。	ヒトラーはあくみょうたかいどくさいしゃのひとりです 
\\	大衆は独裁者に反抗して反乱を起こした。	たいしゅうはどくさいしゃにはんはんこうしてはんらんをおこした 
\\	外交官には様々な特権が与えられている。	がいこうかんにはさまざまなとっけんがあたえられている 
\\	彼は特権を乱用したに違いない。	かれはとっけんをらんようしたにちがいない 
\\	王は権力を乱用した。	おうはけんりょくをらんようした 
\\	ジムは職権乱用で免職になった。	ジムはしょっけんらんようでめんしょくになった 
\\	鮫はその血に飢えた残忍さで悪名高い。	さめはそのちにうえたざんにんさであくみょうたかい 
\\	その悪名高い犯罪者は昨日逮捕された。	そのあくみょうたかいはんざいしゃはきのうたいほされた 
\\	飢えとの戦い。	うえとのたたかい 
\\	子供達は愛情に飢えていた。	こどもたちはあいじょうにうえていた 
\\	労働者の多くは飢えで死んだ。	ろうどうしゃのおおくはうえでしんだ 
\\	疲労やら飢えやらで彼は目眩を感じた。	ひろうやらうえやらでかれはめまいをかんじた 
\\	私は空腹で目眩がした。	わたしはくうふくでめまいがした 
\\	けれども、急に目眩がして、再び古びた椅子に座り込んだ。	けれども、きゅうにめまいがして、ふたたびふるびたいすにすわりこんだ 
\\	ポルノ映画はたいてい町のみすぼらしい所にある古びた映画館で上映される。	ポルノえいがはたいていまちのみすぼらしいところにあるふるびたえいがかんでじょうえいされる 
\\	成功を収めた劇の上映。	せいこうをおさめたげきのじょうえい 
\\	彼女は自分も事業で偉大な成功を収めた。	かのじょはじぶんもじぎょうでいだいなせいこうをおさめた 
\\	よく激しい疲労感に襲われます。	よくはげしいひろうかんにおそわれます 
\\	夜眠れなくて彼は疲労して顔色が悪かった。	よるねむれなくてかれはひろうしてかおいろがわるかった 
\\	どうやら、そのみすぼらしいアパートは空き家のようだ。	どうやら、そのみすぼらしいアパートはあきやのようだ 
\\	可能性は中立か戦争かの二つに一つだ。	かのうせいはちゅうりつかせんそうかのふたつにひとつだ 
\\	彼は話し合いでは中立の立場をとった。	かれははなしあいではちゅうりつのたちばをとった 
\\	できることは抵抗か逃亡か二つに一つだった。	できることはていこうかとうぼうかふたつにひとつだった 
\\	勿論首相に面会するのは難しい。	もちろんしゅしょうにめんかいするのはむずかしい 
\\	それが事実であるとしても、私の関知しないことだ。	それがじじつであるとしても、わたしのかんちしないことだ 
\\	部屋には切迫した空気がみなぎっていた。	へやにはせっぱくしたくうきがみなぎっていた 
\\	事態はかなり切迫している。	じたいはかなりせっぱくしている 
\\	形勢は逆転した。	けいせいはぎゃくてんした 
\\	形勢は2対1で不利、勝ち目は半分。	けいせいはにたいいちでふり、かちめははんぶん 
\\	2人を比較すると彼の方が形勢不利だった。	ふたりをひかくするとかれのほうがけいせいふりだった 
\\	今度の商売は、損して得取れ、という戦略でやろうよ。	こんどのしょうばいは、そんしてとくとれ、というせんりゃくでやろうよ 
\\	戦術を変えてみたら。	せんじゅつをかえてみたら 
\\	彼らは戦術を急に変更して軍を後退させた。	かれらはせんじゅつをきゅうにへんこうしてぐんをこうたいさせた 
\\	当分の間は柔軟な戦術を取るべきだ。	とうぶんのあいだはじゅうなんなせんじゅつをとるべきだ 
\\	前進しないことは後退につながる。	ぜんしんしないことはこうたいにつながる 
\\	前進命令を待っていた。	ぜんしんめいれいをまっていた 
\\	私は当分の間、学校を休まねばならない。	わたしはとうぶんのあいだ、がっこうをやすまねばならない 
\\	彼女は柔軟な頭をしている。	かのじょはじゅうなんなあたまをしている 
\\	柔軟性の欠如は進歩の障害となる。	じゅうなんせいのけつじょはしんぽのしょうがいとなる 
\\	人生を退屈にするのは動機の欠如である。	じんせいをたいくつにするのはどうきのけつじょである 
\\	若者と老人の間にはコミュニケーションの欠如がある。	わかものとろうじんのあいだはコミュニケーションのけつじょがある 
\\	彼の殺人の動機は何だ。	かれのさつじんのどうきはなんだ 
\\	嫉妬がその殺人の動機だった。	しっとがそのさつじんのどうきだった 
\\	彼女は嫉妬の炎を燃やした。	かのじょはしっとのほのおをもやした 
\\	国内は防衛問題で沸騰した。	こくないはぼうえいもんだいでふっとうした 
\\	その男は自己防衛を口実にした。	そのおとこはじこぼうえいをこうじつにした 
\\	評論家はその防衛計画のあらゆる面を十分に検討した。	ひょうろんかはそのぼうえいけいかくのあらゆるめんをじゅうぶんにけんとうした 
\\	防衛者達は強い抵抗を見せた。	ぼうえいしゃたちはつよいていこうをみせた 
\\	彼は遅刻の口実をこしらえた。	かれはちこくのこうじつをこしらえた 
\\	それは怠ける口実にすぎない。	それはなまけるこうじつにすぎない 
\\	カメラを付属品付きで買った。	カメラをふぞくひんつきでかった 
\\	その会社には会社の付属病院が3つある。	そのかいしゃにはかいしゃのふぞくびょういんがみつある 
\\	彼は生来の詩人だ。	かれはせいらいしじんだ 
\\	彼は生来ユーモアの感覚に恵まれている。	ぁれはせいらいユーモアのかんかくにめぐまれている 
\\	ヘレンは生来楽天家だ。	ヘレンはせいらいらくてんかだ 
\\	可哀相にその子は花粉症に悩んでいる。	かわいそうにそのこはかふんしょうににやんでいる 
\\	肺炎が治るのに長い時間かかった。	はいえんがなおるのにながいじかんかかった 
\\	彼は糖尿病のどんな兆しにも注意していた。	かれはとうにょうびょうのどんなきざしにもちゅういしていた 
\\	まだ目にみえる春の兆しはなかった。	まだめにみえるはるのきざしはなかった 
\\	目が炎症を起こしているようですが。	めがえんしょうをおこしているようですが 
\\	消防士たちは炎を消すことができなかった。	しょうぼうしたちはほのおをけすことができなかった 
\\	水は水素と酸素を含む。	みずはすいそとさんそをふくむ 
\\	酸素がないと何も燃やせない。	さんそがないとなにももやせない 
\\	空気中の酸素は水に溶解する。	くうきちゅうのさんそはみずにようかいする 
\\	二酸化炭素はそれ自体は毒ではない。	にさんかたんそはそれじたいはどくではない 
\\	二酸化炭素の量は10%増加している。	にさんかたんそのりょうはじゅうパーセントぞうかしている 
\\	石炭は大部分が炭素から成っている。	せきたんはだいぶぶんがたんそからなっている 
\\	熱帯雨林は、酸素を作り、二酸化炭素を消費する。	ねったいうりんは、さんそをつくり、にさんかたんそをしょうひする 
\\	鉄や酸素は元素である。	てつやさんそはげんそである 
\\	元素記号Hは水素を表す。	げんそきごうHはすいそをあらわす 
\\	彼らはそれが新しい元素に違いないと信じました。	かれらはそれがあたらしいげんそにちがいないとしんじました 
\\	発音記号が読めますか。	はつおんきごうがよめますか 
\\	その記号は答えが正しいことを示す。	そのきごうはこたえがただしいことをしめす 
\\	この記号の意味が理解できない。	このきごうのいみがりかいできない 
\\	君の動機は立派だったが行動はそうではなかった。	きみのどうきはりっぱだったがこうどうはそうではなかった 
\\	近頃では、結婚の動機は必ずしも純粋とは限らない。	ちかごろでは、けっこんのどうきはかならずしもじゅんすいとはかぎらない 
\\	彼は純粋の貴族だ。	かれはじゅんすいのきぞくだ 
\\	その子は純粋な心を持っていた。	そのこはじゅんすいなこころをもっていた 
\\	彼の貴族的な作法には感心する。	かれのきぞくてきなさほうにはかんしんする 
\\	お会いできて光栄です。	おあいできてこうえいです 
\\	この賞をいただいき光栄に存じます。	このしょうをいただいきこうえいにぞんじます 
\\	その貴族は過去の光栄にしがみ付いている。	そのきぞくはかこのこうえいにしがみついている 
\\	彼女は私の手にしっかりとしがみ付いた。	かのじょはわたしのてにしっかりとしがみついた 
\\	スキーの時期は過ぎた。	スキーのじきはすぎた 
\\	春は木を植える時期です。	はるはきをうえるじきです 
\\	子供時代は、急速な成長の時期です。	こどもじだいは、きゅうそくなせいちょうのじきです 
\\	彼は夏の熱い時期にネクタイを締めるのを嫌がる。	かれはなつのあついじきにネクタイをしめるのをいやがる 
\\	浪費なければ欠乏なし。	ろうひなければけつぼうなし 
\\	戦争は不足と欠乏の時代を招いた。	せんそうはふそくとけつぼうのじだいをまねいた 
\\	知識は欠乏しており、知恵は更に乏しい。	ちしきはけつぼうしており、ちえはさらにとぼしい 
\\	彼は闘牛を見たかったが、父はどうしても彼を行かせようとはしなかった。	かれはとうぎゅうをみたかったが、ちちはどうしてもかれをいかせようとはしなかった 
\\	彼は骸骨のようにやせている。	かれはがいこつのようにやせている 
\\	探検家達は洞穴の中で骸骨を発見した。	たんけんかたちはどうけつのなかでがいこつをはっけんした 
\\	その洞穴はその少年たちによって発見されたのですか。	そのどうけつはそのしょうねんたちによってはっけんされたのですか 
\\	その洞穴はとても暗かったので、彼らは手探りで進まねばならなかった。	そのどうけつはとてもくらかったので、かれらはてさぐりですすまねばならない 
\\	私は十分考慮したあげく申し出に応じた。	わたしはじゅうぶんこうりょしたあげくもうしでにおうじた 
\\	彼は苦痛のあまり声を上げた。	かれはくつうのあまりこえをあげた 
\\	彼女は悲嘆のあまり死にそうだ。	かのじょはひたんのあまりしにそうだ 
\\	彼は恐怖のあまり立ちすくんだ。	とても~から
\\	とにかく原因を調べなければならない。	とにかくげんいんをしらべなければならない 
\\	とにかく望みのものが手に入らなかった。	他の事柄は別問題としてという気持ちを表す。何はともあれ。いずれにしても。ともかく。
\\	彼女の死で家族全員が悲嘆に暮れた。	かのじょのしでかぞくぜんいんがひたんにくれた 
\\	彼は悲嘆に暮れて、首吊り自殺をした。	かれはひたんにくれて、くびつりじさつをした 
\\	泥棒にロープを十分に与えれば、彼は首吊りをするであろう。	どろぼうにロープをじゅうぶんにあたえれば、かれはくびつりをするであろう 
\\	悲しみに暮れるその女性は友人たちに慰められた。	かなしみにくれるそのじょせいはゆうじんたちになぐさめられた 
\\	彼は何と言ってよいか途方に暮れた。	かれはなんといってよいかとほうにくれた 
\\	私は家の鍵を失って途方に暮れた。	わたしはいえのかぎをうしなってとほうにくれた 
\\	彼女は花に慰められた。	かのじょははなになぐさめられた 
\\	私たちは互いに慰め合った。	わたしたちはたがいになぐさめあった 
\\	公園の緑は私たちの目を慰めてくれる。	こうえんのみどりはわたしたちのめをなぐさめてくれる 
\\	一方では彼は親切だが、他方では怠け者だ。	いっぽうではかれはしんせつだが、たほうではなまけものだ 
\\	物価は上がる一方だ。	どんどん~なる  ますます~なる
\\	彼は知識だけでなく経験も豊かである。	かれはちしきだけでなくけいけんもゆたかである 
\\	そのヨットは順調に航海中だ。	そのヨットはじゅんちょうにこうかいちゅうだ 
\\	アメリカ経済は順調ですよ。	アメリカけいざいはじゅんちょうですよ 
\\	私に関する限りでは、すべて順調です。	わたしにかんするかぎりでは、すべてじゅんちょうです 
\\	ことは私たちが予想していたよりも順調に進んでいる。	ことはわたしたちがよそうしていたよりもじゅんちょうにすすんでいる 
\\	近いうちに選挙があるそうだ。	~あいだに
\\	老人はあちこち帽子を探し回った。	ろうじんはあちこちぼうしをさがしまわった 
\\	暗くならないうちに帰宅しなさい。	(~の状態が続いている)間に
\\	彼は慌てて飛び出していった。	かれはあわててとびだしていった 
\\	彼は列車に乗るために慌てている。	かれはれっしゃにのるためにあわてていた 
\\	すべての乗客は、慌てて飛行機から離れた。	すべてのじょうきゃくは、あわててひこうきからはなれた 
\\	慌てて着替えなくてもいいよ。急いでいるわけじゃないから。	あわててきかえなくてもいいよ。いそいでいるわけじゃないから 
\\	私は服を着替えるために家へ帰った。	わたしはふくをきかえるためにいえへかえった 
\\	この歌を歌おうじゃないか。	いっしょに~(し)よう
\\	彼にはこの秘密を隠しておこうじゃないか。	かれはこにひみつをかくしておこうじゃないか 
\\	それはあり得ることだ。	それはありえることだ 
\\	勤勉さが経験不足を補うこともあり得る。	きんべんさがけいけんぶそくをおぎなうこともありえる 
\\	何人も運命より賢明ではあり得ない。	なんにんもうんめいよりけんめいではありえない 
\\	肺がんの原因にもなり得る。	はいがんのげんいんにもなりえる 
\\	美の認識は倫理の検査となり得る。	びのにんしきはりんりのけんさとなりえる 
\\	攻撃的な行動に出やすい人は、危険な人間になり得る。	こうげきてきなこうどうにでやすいひとは、きけんなにんげんになりえる 
\\	そう言う事故は時折起こり得る事だ。	そういうじこはときおりおこりえることだ 
\\	太陽は時折顔を見せた。	たいようはときおりかおをみせた 
\\	彼は、時折海辺に行くことが好きです。	かれは、ときおりうみべにいくことがすきです 
\\	赤ん坊には善悪が認識出来ない。	あかんぼうにはぜんあくがにんしきできない 
\\	私たちは状況の重大さを十分に認識しています。	わたしたちはじょうきょうのじゅうだいだをじゅうぶんににんしきしています 
\\	その英語学者は自分の意識不足を認識していない。	そのえいごがくしゃはじぶんのいしきぶそくをにんしきしていない 
\\	倫理学というのは、行動の規範を意味する。	りんりがくというのは、こうどうのきはんをいみする 
\\	それは私たちの道徳的規範には受け入れられない。	それはわたしたちのどうとくてききはんにはうけいれられない 
\\	たくさんの人々が倫理の面から遺伝子治療に反対した。	たくさんのひとびとがりんりのうらからいでんしちりょうにはんたいした 
\\	子供は遺伝病を持っています。	こどもはいでんびょうをもっている 
\\	我々は環境と遺伝の両方の影響を受けている。	われわれはかんきょうといでんのりょうほうのえいきょうをうけている 
\\	この動物には何か遺伝的な問題があるようだ。	このどうぶつにはなにかいでんてきなもんだいがあるようだ 
\\	この犯罪は遺伝の犠牲者だ。	このはんざいはいでんのぎせいしゃだ 
\\	彼はいつも私に皮肉を言う。	かれはいつもわたしにひにくをいう 
\\	彼女は少し皮肉っぽく話した。	かのじょはすこしひにくっぽくはなした 
\\	彼は私を見て皮肉な微笑を浮かべた。	かれはわたしをみてひにくなびしょうをうかべた 
\\	私は疑念を表明せずにはおれない。	わたしはぎねんをひょうめいせずにはおれない 
\\	会社は何とか倒産せずにすんだ。	かいしゃはなんとかとうさんせずにすんだ 
\\	我々は彼の才能に感嘆せずにはおれない。	われわれはかれのさいのうにかんたんせずにはおれない 
\\	彼は一睡もしなかった。	かれはいっすいもしなかった 
\\	彼は一睡もせずに取引が失敗した原因を考えた。	かれはいっすいもせずにとりひきがしっぱいしたげんいんをかんがえた 
\\	学生達は身じろぎもせずに講義に聞き入っていた。	がくせいたちはみじろぎもせずにこうぎにききはいっていた 
\\	いとこは前もって知らせずにやってきて僕を驚かせた。	いとこはまえもってしらせずにやってきてぼくをおどろかせた 
\\	遠慮せずに、好きな時にいつでも休暇を取りなさい。	えんりょせずに、すきなときにいつでもきゅうかをとりなさい 
\\	市長は近く辞意を表明するだろう。	しちょうはちかくじいをひょうめいするだろう 
\\	市長は新計画に不満を表明した。	しちょうはしんけいかくにふまんをひょうめいした 
\\	案の定、彼は疑念を抱いていた。	あんのじょう、かれはぎねんをいだいていた 
\\	彼のそのような行動が彼女の両親の疑念を生んだ。	かれはそのようなこうどうがかのじょのりょうしんのぎねんをうんだ 
\\	そういう事故は再発するおそれがある。	そういうじこはさいはつするおそれがある 
\\	私たちは生命の危険を失うおそれがあった。	わたしたちはせいめいのきけんをうしなうおそれがあった 
\\	私の知る限りでは彼女は気難しい。	わたしのしるかぎりではかのじょはきむずかしい 
\\	私の知る限りでは、これが手にはいる唯一の翻訳書だ。	わたしのしるかぎりでは、これがてにはいるゆいつのほんやくしょだ 
\\	再発の可能性が少しあります。	さいはつのかのうせいがすこしあります 
\\	私は都会を出て自然を再発見したい。	わたしはとしをでてしぜんをさいはっけんしたい 
\\	書物がなければ、それぞれの世代は過去の真理を自分で再発見しなければならないだろう。	しょもつがなければ、それぞれのせだいはかこのしんりをじぶんでさいはっけんしなければならないだろう 
\\	彼は英文学の大家だ。	かれはえいぶんがくのたいかだ 
\\	人はみな誤りに陥りがちだ。	ひとはみなあやまりにおちいりがちだ 
\\	彼は感情を表わしがちだ。	かれはかんじょうをあらわしがちだ 
\\	若い人は誘惑に陥りがちである。	わかいひとはゆうわくにおちいりがちである 
\\	彼は父の忠告を軽視しがちである。	かれはちちのちゅうこくをけいししがちである 
\\	金持ちは人を軽蔑しがちである。	かねもちはひとをけいべつしがちである 
\\	他人を軽蔑するな。	たにんをけいべつするな 
\\	勤勉な人は怠惰を軽蔑する。	きんべんなひとはたいだをけいべつする 
\\	慣れると軽視するようになる。	なれるとけいしするようになる 
\\	誠実さはどちらかというと軽視されているように見える。	せいじつさはどちらかというとけいしされているようにみえる 
\\	どちらかと言うと行きたくない。	どちらかというといきたくない 
\\	1つが通り過ぎたかと思うと、すぐに次の台風が接近する。	ひとつがとおりすぎたかとおもうと、すぐにつぎのたいふうがせっきんする 
\\	彼女がその本を読み始めたと思ったら誰かがドアをノックした。	~(するの)と、ほとんど同時
\\	薄暗い光の中で彼の顔を見た。	うすぐらいひかりのなかでかれのかおをみた 
\\	薄暗い照明の中で、彼女の顔がはっきり見えなかった。	うすぐらいしょうめいのなかで、かのじょのかおがはっきりみえなかった 
\\	我々は明るさを抑えた照明の中でダンスをした。	われわれはあかるいおさえたしょうめいのなかでだんすをした 
\\	我が家ではクリスマスツリーを照明で飾りました。	わがやではクリスマスツリーをしょうめいでかざりました 
\\	家を出るか出ないかのうちに雨が降り出した。	~と、すぐ   ~と、ほとんど同時に
\\	彼は逃げるか逃げないかのうちにまた捕まった。	かれはにげるかにげないかのうちにまたつかまった 
\\	その教師は講堂に学生を集めた。	そのきょうしはこうどうにがくせいをあつめた 
\\	私が講堂に入るか入らないかのうちに式が始まった。	わたしはこうどうにはいるかはいらないかのうちにしきがはじまった 
\\	講堂が暑くなると、いつも私はファンを相撲に連れていく。	こうどうがあつくなると、いつもわたしはファンをすもうにつれていく 
\\	その女性は「助けて」と叫ぼうとしたが、言葉が喉につかえた。	"そのじょせいは「たすけて」とさけぼうとしたが、ことばがのどにつかえた 
\\	陽子は必要でなかったなら化学をとらなかったでしょう。	ようこはひつようでなかったらかがくをとられなかったでしょう 
\\	このお金はもしものときの備えです。	このおかねはもしものときのそなえです 
\\	彼にもしものことがあったら教えて下さい。	かれにもしものことがあったらおしえてください 
\\	もしものときのために、お金は多少蓄えておいたほうがいいよ。	もしものときのために、おかねはたしょうたくわえておいたほうがいいよ 
\\	どんなに年をとっていても学べないことはない。	どんなにとしをっていてもまなべないことはない 
\\	若いけれども彼はこれまでに例のないほどの偉大な数学者である。	わかいけれどもかれはこれまでにれいのないほどのいだいなすうがくしゃである 
\\	もしも脳が死んでいたら、その患者を死なせてあげるべきです。	もしものうがしんでいたら、そのかんじゃをしなせてあげるべきです 
\\	もしも十分に金があったなら、私はその鞄を買っただろうに。	もしもじゅうぶんにおかねがあったなら、わたしはそのかばんをかっただろうに 
\\	その報告書を一気に書き上げた。	そのほうこくしょをいっきにかきあげた 
\\	車は一気にスピードを上げてトラックを追い越した。	くるまはいっきにスピードをあげてトラックをおいこした 
\\	彼は不眠症にかかりやすい。	かれはふみんしょうにかかりやすい 
\\	彼女は不眠症から解放された。	かのじょはふみんしょうからかいほうされた 
\\	規則的に戸外で働く人は不眠症で苦しむことはない。	きそくてきにこがいではたらくひとはふみんしょうでくるしむことはない 
\\	どうしたら不眠症を治せるのか教えてください。	どうしたらふみんしょうをなおせるのかおしえてください 
\\	彼女の肌は長年戸外で働いたのできめが粗くなっている。	かのじょのはだはながねんこがいではたらいたのできめがあらくなっている 
\\	内戦中その国は無政府状態だった。	ないせんちゅうそのくにはむせいふじょうたいだった 
\\	その無政府主義者はかっとなりやすい。	そのむせいふしゅぎしゃはかっとなりやすい 
\\	彼はいわゆる自由主義者だ。	かれはいわゆるじゆうしゅぎしゃ 
\\	わたしは菜食主義者を辞めたの。	わたしはさいしょくしゅぎしゃをあきらめたの 
\\	君が共産主義者にならないように希望する。	きみがきょうさんしゅぎしゃにならないとうにきぼうする 
\\	その社会主義者は女性の通訳を同伴させていた。	そのしゃかいしゅぎしゃはじょせいのつうやくをどうはんさせていた 
\\	その老人にはいつも孫が同伴している。	そのろうじんにはいつもまごがどうはんしている 
\\	彼はカッとなる傾向がある。	かれはカッとなるけいこうがある 
\\	数十年の内戦の後に秩序が回復した。	すうじゅうねんのないせんのあとにちつじょがかいふくした 
\\	内戦がなかったら、彼らは今ごろ裕福なことだろう。	ないせんがなかったら、かれらはいまごろゆうふくなことだろう 
\\	その恐ろしい事故で数十人が負傷した。	そのおそろしいじこですうじゅうにんがふしょうした 
\\	それでは本末転倒だ。	それではほんまつてんとうだ 
\\	あなたの言っていることは本末転倒だと思わないかい?	あなたのいっていることはほんまつてんとうだとおもわないかい? 
\\	プロポーズもしないうちから、結婚式の計画をするのは、本末転倒だ。	プロポーズもしないうちから、けっこんしきのけいかくをするのは、ほんまつてんとうだ 
\\	転倒して彼は大怪我をした。	てんとうしてかれはおおけがをした 
\\	彼はスキーをしていて急斜面で転倒した。	かれはスキーをしていてきゅうしゃめんでてんとうした 
\\	彼は決して自分の学識を見せびらかせない。	かれはけっしてじぶんのがくしきをみせびらかせない 
\\	隣の子供が友達に、新品の自転車を見せびらかしていた。	となりのこどもがともだちにしんぴんのじてんしゃをみせびらかしていた 
\\	この車は新品同様だ。	このくるまはしんぴんどうようだ 
\\	市の北のはずれは路地の迷路である。	しのきたのはずれはろじのめいろである 
\\	忽ちその地域一帯に恐怖が広まった。	たちまちそのちいきいったいにきょうふがひろまった 
\\	悪い噂はたちまち伝わる。	非常に短い時間のうちに動作が行われるさま。すぐ。
\\	切符は忽ち売り切れた。	きっぷはたちまちうりきれた 
\\	その男の子は忽ち皿を空っぽにした。	そのおとこのこは忽ちさらをからっぽにした 
\\	箱が全部空っぽである事が分かりました。	はこがぜんぶからっぽであることがわかりました 
\\	その点は賛成しかねる。	(心理的な抵抗があって)できない、難しい
\\	彼女の態度は少々腹に据えかねる。	かのじょのたいどはしょうしょうはらにすえかねる 
\\	こういう事情ですから、残念ながら、折角のご招待をお受けいたしかねるのです。	こういうじじょうですから、ざんねんながら、せっかくのごしょうたいをおうけいたしかねるのです 
\\	あの男は裏切りもしかねない。	(する)かもしれない、可能性がある
\\	彼はどんな悪事でもやりかねない。	かれはどんなあくじでもやりかねない 
\\	この手の雑誌は若者に害を与えかねない。	このてのざっしはわかものにがいをあたえかねない 
\\	彼は何事もなかったかのように話し続けた。	かれはなにごともなかったかのようにはんしつづけた 
\\	彼は腰から膝にかけてびしょぬれになった。	かれはこしからひざにかけてびしょぬれになった 
\\	2チームは決勝戦で競った。	にチームはけっしょうせんできそった 
\\	決勝戦は明日まで延期された。	けっしょうせんはあしたまでえんきされた 
\\	決勝戦で負けるほど悔しいものはない。	けっしょうせんでまけるほどくやしいものはない 
\\	審判は彼を勝者と認めた。	しんぱんはかれをしょうしゃとみとめた 
\\	決勝戦の勝者に金のカップが贈られた。	けっしょうせんのしょうしゃにきんのカップがおくられた 
\\	万一第三次世界大戦が起こるようなことがあれば、勝者はあり得ないだろう。	まんいちだいさんじせかいたいせんがおこるようなことがあれば、しょうしゃはありえないだろう 
\\	校長先生は勝った人達に賞を贈るでしょう。	こうちょうせんせいはかったひとたちにしょうをおくるでしょう 
\\	彼は事業の失敗を悔しがった。	かれはじぎょうのしっぱいをくやしがった 
\\	彼らは賞を取ろうとお互いに競った。	かれらはしょうをとろうとおたがいにきそった 
\\	いくつかのチームがその賞を勝ち取ろうと競い合っています。	いくつかのチームがそのしょうをかちとろうときそいあっています 
\\	彼女のお見舞いに行こうよ。	かのじょのおみまいにいこうよ 
\\	私は乗馬を体験した。	わたしはじょうばをたいけんした 
\\	彼は自分の体験を述べた。	かれはじぶんのたいけんをのべた 
\\	初体験の相手は、年上の女性だった。	はつたいけんのあいては、としうえのじょせいだった 
\\	彼はその乗馬クラブへ入会を申し込んだ。	かれはそのじょうばクラブへにゅうかいをもうしこんだ 
\\	彼女が作ったケーキを試食した。	かのじょがつくったケーキをししょくした 
\\	彼はスポーツのおかげで劣等感が直った。	かれはスポーツのおかげでれっとうかんがなおった 
\\	あなたはだれにも劣等感を感じる理由はない。	あなたはだれにもれっとうかんをかんじるりゆうはない 
\\	日本は火山列島だ。	にほんはかざんれっとうだ 
\\	乱気流のために飛行機が揺れた。	らんきりゅうのためにひこうきがゆれた 
\\	強風が吹けば高層ビルは揺れるだろう。	きょうふうがふけばこうそうビルはゆれるだろう 
\\	屋根は強風に飛ばされた。	やねはきょうふうにとばされた 
\\	強風が断続的に吹いた。	きょうふうがだんぞくてきにふいた 
\\	その高層ビルの上から町がよく見える。	そのこうそうビルのうえからまちがよくみえる 
\\	断続的ですが、もう2、3ヶ月になります。	だんぞくてきですが、もう2、3かげつになります 
\\	あの古代の廃虚は、嘗ては神社だった。	あのこだいのはいきょは、かつてはじんじゃだった 
\\	彼女は赤十字のために自発的な労働をたくさんした。	かのじょはせきじゅうじのためにじはつてきなろうどうをたくさんした 
\\	赤十字は病院に血液を供給した。	せきじゅうじはびょういんにけつえきをきょうきゅうした 
\\	二人の口論は結局引き分けに終わった。	ふたりのこうろんはけっきょくひきわけにおわった 
\\	その試合は6対6で引き分けに終わった。	そのしあいはろくたいろくでひきわけにおわった 
\\	彼は友人と口論して、彼を殴った。	かれはともだちとこうろんして、かれをなぐった 
\\	口論の果て取っ組み合いを始めた。	こうろんのはてとっくみあいをはじめた 
\\	警官は取っ組み合いをしている二人の男を引き離した。	けいかんはとっくみあいをしているふたりのおとこをひきはなした 
\\	彼を誘惑から引き離すべきだ。	かれをゆうわくからひきはなすべきだ 
\\	彼らは幼い時から逆境と闘ってきたに違いない。	かれらはおさないときからぎゃっきょうとたたかってきたにちがいない 
\\	逆境にもかかわらず、その発明の才に富む男は世界的名声を手に入れた。	ぎゃっきょうにもかかわらず、そのはつめいのさいにとむおとこはせかいめいせいをてにいれた 
\\	逆境で人は成長する。	ぎゃっきょうでひとはせいちょうする 
\\	彼の言葉はあまりにもひどくて繰り返しに耐えない。	かれのことばはあまりにもひどくてくりかえしにたえない 
\\	この件に関する彼の解釈はあまりにも一方的だ。	このけんにかんするかれのかいしゃくはあまりにもいっぽうてきだ 
\\	犬が椅子の上へ飛び上がり、5分間動かないでいた。	いぬがいすのうえへとびあがり、ごふんかんうごかないでいた 
\\	中古車にしては、値段が幾分高い。	ちゅうこしゃにしては、ねだんがいくぶんたかい 
\\	彼が正しいコースを選んだかどうかについて、私達は幾分疑念がある。	かれはただしいコースをえらんだかどうかについて、わたしはいくぶんぎねんがある 
\\	新聞は世論を反映する。	しんぶんはせろんをはんえいする 
\\	世論はその政策に反対している。	せろんはそのせいさくにはんたいしている 
\\	世論のため彼は引退を余儀なくされた。	せろんのためかれはいんたいをよぎなくされた 
\\	その世論調査は無作為に選ばれた成人に基づいてなされた。	そのせろんちょうさはむさくいにえらばれたせいじんにもとづいてなされた 
\\	その映画は成人向きだ。	そのえいがはせいじんむきだ 
\\	君はもう成人したから投票する権利がある。	きみはもうせいじんしたからとうひょうするけんりがある 
\\	日本では、法的には20歳で成人になる。	にほんでは、ほうてきにはにじゅうさいでせいじんになる 
\\	私達は地震によって契約の破棄を余儀なくされた。	わたしたちはじしんによってけいやくのはきをよぎなくされた 
\\	彼は健康上の理由で辞任を余儀なくされた。	かれはけんこうじょうのりゆうでじにんをよぎなくされた 
\\	首相は内閣からの辞職を余儀なくされた。	しゅしょうはないかくからのじしょくをよぎなくされた 
\\	首相は彼らを内閣の主要ポストに任命した。	しゅしょうはかれらをないかくのしゅようポストににんめいした 
\\	内閣はその危機について討議するために日本会合を持つ。	ないかくはそのききについてとうぎするためににほんかいごうをもつ 
\\	焦らずに頑張ってね。	あせらずにがんばってね 
\\	女優は怒って契約を破棄した。	じょゆうばおこってけいやくをはきした 
\\	彼らは婚約を破棄した。	かれらはこにゃくをはきした 
\\	台風で本土との通信が絶えた。	たいふうでほんどとのつうしんがたえた 
\\	この会社は通信部門でよく知られている。	このかいしゃはつうしんぶもんでよくしられている 
\\	電波の発見により、無線通信が可能になった。	でんぱのはっけんにより、むせんつうしんがかのうになった 
\\	パイロットは無線で空港と情報を交換する。	パイロットはむせんでくうこうとじょうほうをこうかんする 
\\	彼らは敵の無線通信を受信した。	かれらはてきのむせんつうしんをじゅしんした 
\\	受信状態がよくない。	じゅしんじょうたいがよくない 
\\	受信異常があったのだと思います。	じゅしんいじょうがあったのだとおもいます 
\\	本土の姿が最もはっきりと見えるのは普通、島の住民なのである。	ほんどのすがたがもっともはっきりとみえるのはふつう、しまのじゅうみんなのである 
\\	彼女には不幸が絶えない。	かのじょにはふこうがたえない 
\\	この習慣は絶えて久しい。	このしゅうかんはたえてひさしい 
\\	彼は父親を絶えず恐れている。	かれはちちおやをたえずおそれている 
\\	蚯蚓を踏むと雨が降るという人がいる。	みみずをふむとあめがふるというひとがいる 
\\	蚯蚓も時には土壌に有益です。	みみずもときにはどじょうにゆうえきです 
\\	この土壌では何も育たないように思われる。	このどじょうではなにもそだたないようにおもわれる 
\\	その土壌は豊かさを保った。	そのどじょうはゆたかさをたもった 
\\	私たちはみんな恒久的な平和を願っている。	わたしたちはみんなこうきゅうてきなへいわをねがっている 
\\	社会情勢は前進というより後退している。	しゃかいじょうせいはぜんしんというよりこうたいしている 
\\	国際情勢は重大になりつつある。	こくさいじょうせいはじゅうだいになりつつある 
\\	彼の視力は衰えつつある。	かれのしりょくはおとろえつつある 
\\	古い伝統が消滅しつつある。	ふるいでんとうがしょうめつしつつある 
\\	もし太陽が消滅したら、生物は皆死ぬだろう。	もしたいようがしょうめつしたら、せいぶつはみんなしぬだろう 
\\	もし酸素がなかったら、すべての動物はとうに消滅していただろう。	もしさんそがなかったら、すべてのどうぶつはとうにしょうめつしていただろう 
\\	虎は消滅しかかっている種族である。	とらはしょうめつしかかっているしゅぞくである 
\\	その島にはまだ未開の種族がいる。	そのしまにはまだみかいのしゅぞくがいた 
\\	王はすべての種族を服従させた。	おうbはすべてのしゅぞくをふくじゅうさせた 
\\	彼女は目方が増えつつある。	かのじょはめかたがふえつつある 
\\	塩は目方で売られる。	しおはめかたでうられる 
\\	その当時、そこには未開民族が住んでいた。	そのとうじ、そこにはみかいみんぞくがすんでいた 
\\	フォークやはしを使う人々は、しばしばフォークやはしを使わない人々のことを未開だと考える。	フォークやはしをつかうひとびとは、しばしばフォークやはしをつかわないひとびとのことをみかいだとかんがえる 
\\	彼らはリーダーの命令に服従した。	かれらはリーダーのめいれいにふくじゅうした 
\\	奴らは弱者を服従させて喜んでいるが。	やつらはじゃくしゃをふくじゅうさせてよろこんでいるが 
\\	部下の不服従はどんなに小さなものでも我慢できなかった。	ぶかのふふくじゅうはどんなにちいさなものでもがまんできなかった 
\\	彼女は常に弱者に味方する。	かのじょはつねにじゃくしゃにみかたする 
\\	強者は生き残り、弱者は死ぬものだ。	きょうしゃはいきのこり、じゃくしゃはしぬものだ 
\\	強者は弱者の面倒を見るべきだ。	きょうしゃはじゃくしゃのめんどうをみるべきだ 
\\	彼は常に強者に対抗して弱者を味方にした。	かれはつねにきょうしゃにたいこうしてじゃくしゃをみかたにした 
\\	何が起きようと僕は君の味方です。	なにがおきようとぼくはきみのみかたです 
\\	幸運の女神は冒険好きの人の味方だ。	こううんのめがみはぼうけんすきのひとのみかただ 
\\	上司は部下にそれを素早く完成するように命じた。	じょうしはぶかにそれをすばやくかんせいするようにめいじた 
\\	隊長は部下に撃てと命令した。	たいちょうはぶかにうてとめいれいした 
\\	隊長以下の者がこの建物に住む。	たいちょういかのものがこのたてものにすむ 
\\	隊長は兵士たちを従えて行進した。	たいちょうはへいしたちをしたがえてこうしんした 
\\	上司は部下に仕事を振り分けた。	じょうしはぶかにしごとをふりわけた 
\\	新しい駅ビルが建築中で、まもなく完成する。	あたらしいえきビルがけんちくちゅうで、ももなくかんせいする 
\\	日本は中国から原料を輸入し完成品を輸出する。	にほんはちゅうごくからげんりょうをゆにゅうしかんせいひんをゆしゅつする 
\\	彼は学生並みから言えば勤勉です。	かれはがくせいなみからいえばきんべんです 
\\	彼は政治家並から言えば、演説が上手い。	かれはせいじかなみからいえば、えんぜつがうまい 
\\	実践的見地からすれば彼の計画は実行しにくい。	じっせんてきけんちからすればかれにけいかくはじっこうしにくい 
\\	この観点からすれば、彼は正しかったと言えよう。	このかんてんからすれば、かれはただしかったといえよう 
\\	間違ったからといって彼のことを笑うな。	「~から、~」とはいえない
\\	貧乏だからといって他人を軽蔑するな。	びんぼうだからといってたにんをけいべつするな 
\\	だからといって異議があるわけではない。	だからといっていぎがあるわけではない 
\\	山は高いからといって価値があるわけではない。	やまはたかいからといってかちがあるわけではない 
\\	彼は賢いからといって正直だという事にはならない。	かれはかしこいからといってしょうじきだということにならない 
\\	その機械は部品が足りない。	そのきかいはぶひんがたりない 
\\	修理費には部品代と手数料が含まれます。	しゅうりひにはぶひんだいとてすうりょうがふくまれます 
\\	仕事の難しい部分は彼がやる羽目になった。	しごとのむずかしいぶぶんはかれがやるはめになった 
\\	私は風邪気味です。	【ぎみ】 少し~だ
\\	急に笑い出さないでよ。不気味だから。	きゅうにわらいださないでよ。ぶきみだから 
\\	彼が終電に乗りそこなうなんていい気味だ。	かれがしゅうでんにのりそこなうなんていいきみだ 
\\	彼は幾分太り気味だ。	かれはいくぶんふとりぎみだ 
\\	喋るのもいい加減にしたら。	しゃべるのもいいかげんにしたら 
\\	この静かな生活にはいい加減飽きてしまった。	このしずかなせいかつにはいいかげんあきてしまった 
\\	電気をいい加減に扱うのは危険だ。	でんきをいいかげんにあつかうのはきけんだ 
\\	彼は一人っ子だったので、唯一の相続人だった。	かれはひとりっこだったので、ゆいつのそうぞくにんだった 
\\	手数料は3パーセントかかります。	てすうりょうはさんパーセントかかります 
\\	レスキュー隊は行方不明の乗客を捜査した。	レスキューたいはゆくえふめいのじょうきゃくをそうさした 
\\	登山者たちは救助隊に救助された。	とざんしゃたちはきゅうじょたいにきゅうじょされた 
\\	彼はそこに行ったきり2度と帰ってこなかった。	~たまま
\\	彼は病気でずっと床についたきりだ。	かれはびょうきでずっとゆかについたきりだ 
\\	彼女は5年前に家を出たきりで、その後何の消息もありません。	かのじょはごねんまえにいえをでたきりで、そのあとなんのしょうそくもありません 
\\	それ以来彼らの消息は不明だ。	それいらいかれらのしょうそくはふめいだ 
\\	年賀状のおかげで私達は友達や親戚の消息が分かる。	ねんがじょうのおかげでわたしたちはともだちやしんせきのしょうそくがわかる 
\\	彼は自分の子供達の教育の問題を妻に任せきりだった。	かれはじぶんのこどもたちのきょういくのもんだいをつまにまかせきりだった 
\\	その新製品は発売中だ。	そのしんせいひんははつばいちゅうだ 
\\	その新型車は五月に発売される。	そのしんがたくるまはごがつにはつばいされる 
\\	彼はまったく両親に頼りきっている。	完全に~する
\\	この柱では屋根を支えきれない。	このはしらではやねをささえきれない 
\\	彼は金持ちのくせに乞食のような生活をしている。	かれはかねもちのくせにこじきのようなせいかつをしている 
\\	私は乞食も同然だ。	わたしはこじきもどうぜんだ 
\\	乞食は空腹と疲労で目眩がした。	こじきはくうふくとひろうでめまいがした 
\\	あの占い師は嘘吐き同然だ。	あのうらないしはうそつきどうぜんだ 
\\	死にたいくらいだ。	~ほど
\\	その事件には何となく気味の悪いところがあった。	そのじけんにはなんとなくぎみのわるいとことがあった 
\\	少女達は楽しげに歌います。	~そう(様子)
\\	彼は憎らしげに彼女を睨んだ。	かれはにくらしげにかのじょをにらんだ 
\\	彼は、とても満足げに見える。	かれは、とてもまんぞくげにみえる 
\\	彼は自信ありげに見えたが、内心は全然違っていた。	かれはじしんありげにみえたが、ないしんはぜんぜんちがっていた 
\\	彼は彼女の声に不安げな様子を感じ取った。	かれはかのじょのこえにふあんげなようすをかんじとった 
\\	特別扱いしますが、なるべく短めにしてくださいね。	とくべつあつかいしますが、なるべくみじかめにしてくださいね 
\\	彼は赤の他人だよ。	ちのつながりがない。全く関係の無い人。
\\	オランダ語はドイツ語と密接なつながりがある。	オランダごはドイツごとみっせつなつながりがある 
\\	2つの間には重要なつながりがある。	ふたつのあいだにはじゅうようなつながりがある 
\\	両国はお互いに密接な関係がある。	りょうこくはおたがいにみっせつなかんけいがある 
\\	従業員の利害は会社の利害と密接な関係を持つ。	じゅぎょういんのりがいはかいしゃのりがいとみっせつなかんけいをもつ 
\\	私の将来は会社の経済状態と密接に関係している。	わたしのしょうらいはかいしゃのけいざいじょうたいとみっせつにかんけいしている 
\\	彼は自分の利害に敏感である。	かれはじぶんのりがいにびんかんである 
\\	此処では多様な民族的・経済的利害関係がみられる。	ここではたようなみんぞくてき・けいざいてきりがいかんけいがみられる 
\\	我々の保険の範囲は多様な損害に及びます。	われわれのほけんのはんいはたようなそんがいにおよびます 
\\	彼の才能は大変素晴らしくそして多様である。	かれのさいのうはたいへんすばらしくそしてたようである 
\\	病気は通常1つではなく、多様な原因によって起こる。	びょうきはつうじょうひとつではなく、たようなげんいんによっておこる 
\\	私達は多種多様な動物達の生活を不可能にする危険がある。	わたしたちはたしゅたようなどうぶつたちのせいかつをふかのうにするきけんがある 
\\	話題は多様多種だった。	わだいはたようたしゅだった 
\\	彼はそこで多種の生物を観察した。	かれはそこでたしゅのせいぶつをかんさつした 
\\	彼女は私に白い目で見た	冷たい、非難を持った目で見る
\\	結論を白紙に戻そう。	けつろんをはくしにもどそう 
\\	すべてを白紙に返そう。	すべてをはくしにかえそう 
\\	彼女は白紙答案を出した。	かのじょははくしとうあんをだした 
\\	毎月の収入の一部を貯蓄すれば損はない。	まいつきのしゅうにゅうのいちぶをちょちくすればそんはない 
\\	老後を安心して暮らしたかったら今から貯蓄を始めなさい。	ろうごをあんしんしてくらしたかったらいまからちょちくをはじめなさい 
\\	上質の物を買うと結局損はない。	じょうしつのものをかうとけっきょくそんはない 
\\	彼は帰国を命じられた。	かれはきこくをめいじられた 
\\	帰国者たちは日本の生活に慣れるのに苦労している。	きこくしゃたちはにほんのせいかつになれるのびくろうしている 
\\	私たちはみんな彼の10年ぶりの帰国を待ち望んでいた。	わたしたちはみんなかれのじゅうねんぶりのきこくをまちのぞんでいた 
\\	定規は直線を引くのに役立つ。	じょうぎはちょくせんをひくのにやくだつ 
\\	この定規にはミリメートルの目盛りがある。	このじょうぎにはミリメートルのめもりがある 
\\	日本では摂氏の目盛りが使われている。	にほんではせっしのめもりがつかわれている 
\\	水は摂氏100度で沸騰する。	みずはせっしひゃくどでふっとうする 
\\	今日、気温は摂氏30度の高さまでも上昇した。	きょう、きおんはせっしさんじゅうどのたかさまでもじょうしょうした 
\\	子供たちはどのようにして理解力を身に付けるのでしょうか。	こどもたちはどのようにしてりかいりょくをみにつけるのでしょうか 
\\	あなたは忍耐力を身につけるべきだ。	あなたはにんたいりょくをみにつけるべきだ 
\\	私たちは身につけている衣服で他人を判断しがちである。	わたしたちはみにつけているいふくでたにんをはんだんしがちである 
\\	姫は衣服を脱ぎ捨てた。	ひめはいふくをぬぎすてた 
\\	この本は初心者向きである。	このほんはしょしんしゃむきである 
\\	初心者は先ず口語英語を学ぶべきだ。	しょしんしゃはまずこうごえいごをまなぶべきだ 
\\	折角来てくれたのに留守をしていてごめんね。	せっかくきてくれたのにるすをしていてごめんね 
\\	せっかく階段を走って降りたのに、もう一歩のところで電車に乗れなかった。	せっかくかいだんをはしっておりたのに、もういっぽのところででんしゃにのれなかった 
\\	何と素晴らしい時を過ごしたことか。	とても、非常に~ ~ことだろう(「なんとうれしいことか」=「とてもうれしい」)
\\	彼の身長はどれだけですか。	どのくらい。
\\	どれだけ本に使ったかをざっと計算してみてください。	全体の数量や内容などについておおまかな見当をつけるさま。だいたい。およそ。
\\	ざっと見積もって、その仕事は二週間かかるだろう。	ざっとみつもって、そのしごとはにしゅうかんかかるだろう 
\\	大まかに言って、犬は猫より忠実だ。	おおまかにいって、いぬはねこよりちゅうじつだ 
\\	警察官はその少女にいなくなった犬の、大まかな絵を描くように求めた。	けいさつかんはそのしょうじょにいなくなったいぬの、おおまかなえをかくようにもとめた 
\\	人々は話すことなくしゃべる。	ひとびとははなすことなくしゃべる 
\\	入り口で靴を脱ぐことになっている。	~と決まっている
\\	私が父の事業を受け継ぐことになっている。	わたしはちちのじぎょうをうけつぐことになっている 
\\	女王は来年中国を訪問することになっている。	じょおうはらいねんちゅうごくをほうもんすることになっている 
\\	日本女性は手が器用だということになっている。	にほんじょせいはてがきようだということになっている 
\\	退社前に電灯や暖房器を消すことになっている。	たいしゃまえにでんとうやだんぼうきをけすことになっている 
\\	彼は怠惰で無責任だった。結局、彼は退社を命じられた。	かれはたいだでむせきにんだった。けっきょく、かれはたいしゃをめいじられた 
\\	最近の船は乗組員が少なくすむ。	物事が終わる。終了する。
\\	夕食がすむと、彼はその小説を読みはじめた。	ゆうしょくがすむと、かれはそのしょうをよみはじめた 
\\	彼は多くても100冊しか小説を持っていない。	かれはおおくてもひゃくさつしかしょうせつをもっていない 
\\	是非取引させて頂きたいと思います。	ぜひとりひきさせていただきたいとおもいます 
\\	原稿を少し変えたいと思います。	げんこうをすこしかえたいとおもいます 
\\	彼の成功は大部分幸運に依るものだった。	かれのせいこうはだいぶぶんこううんによるものだった 
\\	私たちの成功は、結局、彼の真面目な努力に依るものだ。	わたしたちのせいこうは、けっきょく、かれのまじめなどりょくによるものだ 
\\	警察は自動車事故を無謀運転に依るものだと考えた。	けいさつはじどうしゃじこをむぼううんてんによるものだとかんがえる 
\\	私どものカタログにあなたの論文の一部を引用させていただきたいと思っています。	わたしとものカタログにあなたのろんぶんのいちぶをいにょうさせていただきたいとおもっています 
\\	ご移転の際はお知らせ下さい。	ごいてんのさいはおしらせください 
\\	今後旅行の際は是非当社をご検討ください。	こんごりょこうのさいはぜひとうしゃをごけんとうください 
\\	スピーチの際に珍しい話題を出す必要はない。	スピーチのさいにめずらしいわだいをだすひつようはない 
\\	人前で大騒ぎするな。	ひとまえでおおさわぎするな 
\\	映画の最中に彼は震え始めた。	(ちょうど)~(し)ているとき
\\	性交の最中彼女はつまらなそうな顔をしていた。	せいこうのさいちゅうかのじょはつまらなそうなかおをしていた 
\\	性交時に出血があります。	せいこうときにしゅっけつがあります 
\\	歯茎から出血しますか。	はぐきからしゅっけつしますか 
\\	歯茎が腫れています。	はぐきがはれています 
\\	その問題は委員会によって討議されている最中だった。	そのもんだいはいいんかいに依ってとうぎされているさいちゅうだった 
\\	私が彼らのアパートを訪問したとき、夫婦は議論の真っ最中だった。	わたしがかれらのアパートをほうもんしたとき、ふうふはぎろんのまっさいちゅうだった 
\\	赤くさえあれば、どんな花でも結構です。	あかくさえあれば、そんなはなでもけっこうです 
\\	彼に会う機会さえあればなあ。	かれにあうきかいさえあればなあ 
\\	地図さえあれば、君に道を教えてあげられるのに。	ちずさえあれば、きみにみちをおしえてあげられるのに 
\\	君は指示に従ってさえいればいいのです。	きみはしじにしたがってさえいればいいのです 
\\	席を確保するには列に並びさえすればいい。	せきをかくほするにはれつにならびさえすればいい 
\\	緊急時の飲料水の確保は、大丈夫ですか?	きんきゅうときのいんりょうすいのかくほは、だいじょうぶですか 
\\	酸性雨は自然現象ではない。	さんせいうはしぜんげんしょうではない 
\\	私は昨日、酸性雨についての記事を読んだ。	わたしはきのう、さんせいうについてのきじをよんだ 
\\	その神社の腐食の原因の一つは酸性雨である。	そのじんじゃのふしょくのげんいんのひとつはさんせいうである 
\\	ほら、錆で金属がどんどん腐食しているよ。	ほら、さびできんぞくがどんどんふしょくしているよ 
\\	飲料水中の酸性雨は人間の健康に影響する。	いんりょうすいちゅうのさんせいうはにんげんのけんこうにえいきょうする 
\\	トラブルを避けても必ずしも安全が確保されているわけではない。	トラブルをさけてもかならずしもあんぜんがかくほされているわけではない 
\\	いわゆる英知は単に知識の断片ではないことを心にとめておくべきだ。	いわゆるえいちはたんにちしきのだんぺんではないことをこころにとめておくべきだ 
\\	事故の断片的ニュース。	じこのだんぺんてきニュース 
\\	このパズルには500の断片がある。	このパズルにはごひゃくのだんぺんがある 
\\	彼女の忠告を心に留めておきなさい。	かのじょのちゅうこくをこころにとめておきなさい 
\\	40にして人生はまだ半分残っていることを心に留めよ。	よんじゅうにしてじんせいはまだはんぶんのこっていることをこころにとめよ 
\\	彼は単に冗談としてそれを言った。	かれはたんにじょうだんとしてそれをいった 
\\	彼は単に好奇心からそれをしただけだ。	かれはたんにこうきしんからそれをしただけだ 
\\	私はこの演劇を単に悲劇として扱うつもりだ。	わたしはこのえんげきをたんにひげきとしてあつかうつもりだ 
\\	彼は大学で演劇を専攻した。	かれはだいがくでえんげきをせんこうした 
\\	彼女は特別の奨学金をもらって演劇を研究していた。	かのじょはとくべつのしょうがくきんをもらってえんげきをけんきゅうしていた 
\\	他のどんな職業よりも演劇が彼女の性に合っている。	ほかのどんなしょくぎょうよりもえんげきがかのじょのしょうにあっている 
\\	机でする仕事はどうも性に合わない。	つくえでするしごとはどうもしょうにあわない 
\\	僕は自転車で通勤を試みたが遂にこれは性に合わなかった。	ぼくはじてんしゃでつうきんをこころみたがついにこれはしょうにあわなかった 
\\	彼は雨の中運転するときに極度に注意を払う。	かれはあめのなかうんてんするときにきょくどにちゅういをはらう 
\\	収入が多いおかげで彼は安楽に過ごせた。	しゅうにゅうがおおいおかげでかれははあんらくにすごせた 
\\	納豆は粘々している。	なっとうはねばねばしている 
\\	彼は粘土で鉢をつくった。	かれはねんどではちをつくった 
\\	彼は粘土を火に入れて固めた。	かれはねんどをひに入れてかためた 
\\	そして粘土を指の間で滑らせるの。	そしてねんどをゆびのあいだですべせるの 
\\	冬の間は苗を鉢植えにする。	ふゆのあいだはなえをはちうえにする 
\\	ガラスの鉢は粉々になってしまった。	ガラスのはちはこなごなになってしまった 
\\	彼は大学の学費を親に頼っている。	かれはだいがくのがくひをおよにたよっている 
\\	彼女は収入の大部分を食費に使う。	かのじょはしゅうにゅうのだいぶぶんをしょくひにつかう 
\\	私は事例をいくつか出せます。	わたしはじれいをいくつかだせます 
\\	この種の事例は統計的処理が適応できる。	このしゅのじれいはとうけいてきしょりがてきおうできる 
\\	この規則はその事例に適用できない。	このきそくはそのじれいにてきようできない 
\\	彼女は支配人にサービスが悪いと苦情を言った。	かのじょはしはいじんにサービスがわるいとくじょうをいった 
\\	私は苦情を言ったが、店ではこのセーターを引き取るのを拒んだ。	わたしはくじょうをいったが、みせではこのセーターをひきとるのをこばんだ 
\\	彼は異なった考え方を拒まなかった。	かれはことなったかんがえかたをこばまなかった 
\\	彼女は彼が嫌いなので、彼の忠告に従うことを拒みそうだ。	かのじょはかれがきらいなので、かれのちゅうこくにしたがうことをこばみそうだ 
\\	君がその提案を拒むのも当然だ。	きみがそのていあんをこばむのもとうぜんだ 
\\	私は彼の説明で納得した。	わたしはかれのせつめいでなっとくした 
\\	彼は彼女が無罪であることを我々に納得させた。	かれはかのじょがむざいであることをわれわれになっとくさせた 
\\	私は両親に結婚を納得してもらうことが出来た。	わたしはりょうしんにけっこんをなっとくしてもらうことができた 
\\	栄養のある朝食をとった。	えいようのあるちょうしょくをとった 
\\	彼女は子供の栄養に気をつけている。	かのじょはこどものえいようにきをつけている 
\\	このビタミン剤は栄養分を豊富に含んでいる。	このビタミンざいはえいようぶんをほうふにふくんでいる 
\\	工場は生産を縮小せざるを得なかった。	こうじょうはせいさんをしゅくしょうせざるをえなかった 
\\	私は職場に着いた後、電話で発注した。	わたしはしょくばについたあと、でんわではっちゅうした 
\\	それの価格を5%値引きしていただけるのでしたら、発注しようと思います。	それをかかくをごパーセントねびきしていただけるのでしたら、はっちゅうしようとおもいます 
\\	値引き交渉に必要な条件を教えてください。	ねびきこうしょうにひつようなじょうけんをおしえてください 
\\	あの会社の社長は切り札を隠し持っています。	あのかいしゃのしゃちょうはきりふだをかくしもっています 
\\	きっと無罪になるだけの決定的切り札を隠し持っているに違いない。	きっとむざいになるだけのけっていてききりふだをかくしもっているにちがいない 
\\	金曜日締め切りのプロジェクトに追われています。	きにょうびしめきりのプロジェクトにおわれています 
\\	彼は警察に追われている。	かれはけいさつにおわれている 
\\	自分の頭の上の蝿を追え。	じぶんのあたまのうえのハエをおえ 
\\	狩猟者たちは山を越えて鹿の跡を追った。	しゅりょうしゃたちはやまをこえてしかのあとをおった 
\\	若者の間で狩猟用ブーツが流行った。	わかもののあいだでしゅりょうようブーツがはやった 
\\	この地域では狩猟は禁止されている。	このちいきではしゅりょうはきんしされている 
\\	どうやってこの書類の締め切りに間に合わせるんだ?	どうやってこのしょるいのしめきりにまにあわせるんだ? 
\\	締め切り期限を過ぎてから彼女はレポートを提出した。	しめきりきげんをすぎてからかのじょはレポートをていしゅつした 
\\	最低価格と一番早い納期を見積もってくださるようお願い致します。	さいていかかくといちばんはやいのうきをみつもってくださるようおねがいいたします 
\\	材料の納期が2ヶ月かかってしまったので、12月12日に遅れてしまいました。	ざいりょうののうきがにかげつかかってしまったので、じゅうにがつじゅうににちにおくれてしまいました 
\\	出席状況が最終の成績に響きます。	しゅっせきじょうきょうがさいしゅうのせいせきにひびきます 
\\	戦争は最終段階にはいっていった。	せんそうはさいしゅうだんかいにはいっていった 
\\	彼の声に多少怒りの響きがあった。	かれのこえにたしょうのいかりのひびきがあった 
\\	前のメールがきつく響かなかったことを願っています。	まえのメールがきつくひびかなかったことをねがっています 
\\	最終提案は来週中に発表されます。	さいしゅうていあんはらいしゅうちゅうにはっぴょうされます 
\\	なんかその日暮らしって感じで、羨ましい。	なんかそのひぐらしってかんじで、うらやましい 
\\	彼らはその紛争を終わらせた。	かれらはそのふんそうをおわらせた 
\\	労使紛争はいまだに困った問題だ。	ろうしふんそうはいままでにこまったもんだいだ 
\\	労使間の話し合いはうやむやに終わった。	ろうしかんのはなしあいはうやむやにおわった 
\\	先生、息子を連れていった方がよろしいでしょうか。	せんせい、むすこをつれていったほうがよろしいでしょうか 
\\	これらの言葉ですらいつか消えてしまいます。	これらのことばですらいつかきえてしまいます 
\\	虫ですら向かってくるものだ。	むしですらむかってくるものだ 
\\	誰も僕の意見など聞きたがらない。	だれもぼくのいけんなどききたがらない 
\\	アメリカの文化はヨーロッパから移植されたものだ。	アメリカのぶんかはユーロッパからいしょくされたものだ 
\\	外科医に説得されて、かれは臓器移植手術を受けることにした。	げかいにせっとくされて、かれはぞうきいしょくしゅじゅつをうけることにした 
\\	彼が手術するかどうかは、移植する臓器の提供次第だ。	かれはしゅじゅつするかどうかは、いしょくするぞうきのていきょうしだいだ 
\\	外科医は私を説得して、臓器の移植手術を受けることに同意させた。	げかいはわたしをせっとくして、ぞうきのいしょくしゅじゅつをうけることにどういさせた 
\\	万一私が死んだら、私の心臓を必要な人に提供して下さい。	まんいちわたしがしんだら、わたしのしんぞうをひつようなひとにていきょうしてください 
\\	彼は、両親の負担になった。	かれは、りょうしんのふたんになった 
\\	彼女の親切に感謝する一方負担にも感じる。	かのじょのしんせつにかんしゃするいっぽうふたんにもかんじた 
\\	彼女は自分の負担で本を出版した。	かのじょはじぶんのふたんをしゅっぱんした 
\\	宙返りするジェットコースターに乗ったら、気持ち悪くなっちゃった。	ちゅうがえりするジェットコースターにのったら、きもちわるくなっちゃった 
\\	私は試しに逆立ちしてみた。	わたしはためしにさかだちしてみた 
\\	彼女は塀の上を逆立ちして歩いた。	かのじょはへいのうえをさかだちしてあるいた 
\\	彼は試しにその問題を解いてみた。	かれはためしにそのもんだいをといてみた 
\\	私たちは全く新しい方法を試しています。	わたしたちはまったくあたらしいほうほうをためしています 
\\	その事件によって彼の勇敢さが試された。	そのじけんによってかれのゆうかんさがためされた 
\\	彼はあまりに用心深いため、新しいことは何も試せない。	かれはあまりにようじんぶかいため、あたらしいことはなにもためせない 
\\	人生で成功の道は勤勉と用心深さにある。	じんせいでせいこうのみちはきんべんとようじんぶかさにある 
\\	巨大な肉の塊が当たった。	きょだいなにくのかたまりがたった 
\\	為替相場は1ドル145円だ。	かわせそうばはいちどるひゃくよんじゅうごえんだ 
\\	外貨の為替レートは毎日変わる。	がいかのかわせレートはまいにちかわる 
\\	今日の為替相場は幾らですか。	きょうのかわせそうばはいくらですか 
\\	日本の最新動向について報告したいと思います。	にほんのさいしんどうこうについてほうこくしたいとおもいます 
\\	財政の専門家たちは最近の動向をどう判断してよいのかわからないのです。	ざいせいのせんもんかたちはさいきんのどうこうをどうはんだんしてよいのかわからないのです 
\\	彼はそれを見るや否や青くなった。	かれはそれをみるやいなやあおくなった 
\\	彼女は私を見るや否や、わっと泣き出した。	かのじょはをみるやいなや、わっとなきだした 
\\	その部屋に入るや否や私は、煙草の臭いのほかにガスの匂いがするのに気がついた。	そのへやにはいるやいなやわたしは、たばこのにおいのほかにガスのにおいがするにきがついた 
\\	彼は古代史の分野で突っ込んだ研究をしている。	かれはこだいしのぶんやでつっこんだけんきゅうをしている 
\\	他人のことに首を突っ込むな。	たにんのことにくびをつっこむな 
\\	人の問題に鼻を突っ込むのはよしてくれよ。	ひとのもんだいにはなをつっこむのはよしてくれよ 
\\	水溜りに足を突っ込んだので、靴がグチョグチョになった。	みずたまりにあしをつっこんだので、くつがグチョグチョになった。 
\\	思わず笑ってしまった。	おもわずわらってしまった 
\\	彼は法律を構わず無視する。	かれはほうりつをかまわずむしする 
\\	賃料が手頃なら、どんなマンションでも構いません。	ちんりょうがてごろなら、どんなマンションでもかまいません 
\\	庭を通っても構いませんか。	にわをとおってもかまいませんか 
\\	彼は一言も言わず何時間も座っていた。	かれはひとこともいわずなんじかんもさわっていた 
\\	くすくす笑わずにいられなかった。	くすくすわらわずにいられなかった 
\\	私は頭が痛くなると鎮痛剤を飲まずにいられない。	わたしはあたまがいたくなるとちんつうざいをのまずにいられない 
\\	息子はまだ生きていると考えずにいられない。	むすこはまだいきているとかんがえずにいられない 
\\	その知らせを聞いて、ため息をつかずにいられなかった。	そのしらせをきいて、まめいきをつかずにいられなかった 
\\	私は自分の愚かさを笑わずにいられない。	わたしはじぶんのおろかさをわらわずにいられない 
\\	彼は後悔のため息をついた。	かれはこうかいのためいきをついた 
\\	彼女はため息をついて両手をしっかり握り締めた。	かのじょはためいきをついてりょうてをしっかりにぎりしめた 
\\	鎮痛剤は痛いときだけ飲んでください。	ちんつうざいはいたいときだけのんでください 
\\	彼は70歳でなお活躍している。	かれはななじゅうさいでなおかつやくしている 
\\	新しい地位でご活躍することを確信しています。	あたらしいちいでごかつやくすることをかくしんしています 
\\	彼女は婦人解放運動で積極的に活躍した。	かのじょはふじんかいほううんどうでせっきょくてきにかつやくした 
\\	彼はそれをうまくやれる、君ならなおさらだ。	かれはそれをうまくやれる、きみならなおさらだ 
\\	それらの星は肉眼で見える、まして望遠鏡ならなおさらである。	それらのほしはにくがんでみえる、ましてぼうえんきょうならなおさらである 
\\	私は物理が好きです。まして数学はなおさら好きです。	わたしはぶつりがすきです。ましてすうがくはなおさらすきです 
\\	私は音楽を聞くのが好きだし、まして演奏するのはなおさら好きだ。	わたしはおんがくをきくのが好きだし、ましてえんそうするのはなおさらすきだ 
\\	動物も植物も生きる権利がある。人間はなおさらだ。	どうぶつもしょくぶつもいきるけんりがる。にんげんはなおさらだ 
\\	冗談はさておき、君は何が言いたいのだ。	じょうだんはさておき、きみはなんいがいいたいのだ 
\\	費用はさておきその建築にはかなりの時間がかかるだろう。	ひようはさておきそのけんちくにはかなりのじかんがかかるだろう 
\\	父の死のショックが後を引いていて、彼女は外出する気力がなかった。	ちちのしのショックがあとひいていて、かのじょはがいしゅつするきりょくがなかった 
\\	私は、彼が時間通りに来るかしらと思わざるをえない。	わたしは、かれがじかんとおりにくるかしらとおもわざるをえない 
\\	以下は大統領の演説の要旨だ。	いかはだいとうりょうのえんぜつのようしだ 
\\	そのいさかいは両家の間に深い溝を残した。	そのうさかいはりょうけのあいだにふかいみぞをのこした 
\\	君達はいさかいを止めて、過去のことは水に流してもいい頃じゃないのかい?	きみたちはいさかいをとめて、かこのことはみずにながしてもいいころじゃないのかい? 
\\	過去のことは水に流せ。	かこのことはみずにながせ 
\\	本当に自分勝手だ。	ほんとうにじぶんかってだ 
\\	適当に入って勝手にやって。	てきとうにはいってかってにやって 
\\	君は勝手に出かけていいよ。	きみはかってにでかけていいよ 
\\	勝手にそんなふうに想像するなよ。	かってにそんなふうにそうぞうするなよ 
\\	自分勝手なその男は同僚達から軽蔑された。	じぶんかってなそのおとこはどうりょうたちからけいべつされた 
\\	彼は酒で身を滅ぼした。	かれはさけでみをほろぼした 
\\	勝手にさせておけば泥棒は自然に身を滅ぼす。	かってにさせておけばどろぼうはしぜんにみをほろぼす 
\\	彼は自分の行いのために身を滅ぼした。	かれはじぶんのおこないのためにみをほろぼした 
\\	核兵器はわが地球を滅ぼすと思う。	かくへいきはわがちきゅうをほろぼすとおもう 
\\	人は核兵器に抗議している。	ひとはかくへいきにこうぎしている 
\\	彼が核兵器は平和への脅威であると論じた。	かれはかくへいきはへいわのきょういであるとろんじた 
\\	子供を好き勝手にさせておくな。	こどもをすきかってにさせておくな 
\\	講演者はその問題をきわめて簡潔に論じた。	こうえんしゃはそのもんだいをきわめてかんけつにろんじた 
\\	彼女は成長して獣医になった。	かのじょはせいちょうしてじゅういになった 
\\	私には、お父さんが獣医をしている友達がいる。	わたしは、おとうさんがじゅういをしているともだちがいる 
\\	その日に限って彼は欠席だった。	そのひにきりってかれはけっせきだった 
\\	急いでいる時に限ってバスが遅れる。	いそいでいるときにきりってバスがおくれる 
\\	一緒にいるのを見られたくない人と一緒にいるときに限って、知り合いに会うことが多い。	いっしょにいるのをみられたくないひとといっしょにいるときにきりって、しりあいにあうことふぁおおい 
\\	「飲み物は無料ですか」「ご婦人に限ってです」	"「のみものはむりょうですか」「ごふじんにきりってです」 
\\	支配人は臨時休業の掲示を貼り出した。	しはいにんはりんじきゅうぎょうのけいじをはりだした 
\\	掲示には「全席予約済み」とあった。	"けいじには「ぜんせきよやくずみ」とあった 
\\	座席は全部予約済みです。	ざせきはぜんぶよやくずみです 
\\	移転にともない5月30日、31日は休業いたします。	いてんにともないごがつさんじゅうにち、さんじゅういちにちはきゅうぎょういたします 
\\	電話を切らずにおいてください。	でんわをきらずにおいてください 
\\	科学技術において革命が起きた。	かがくぎじゅつにおいてかくめいがおきた 
\\	車の代金を現金で半額支払った。	くるむのだいきんをげんきんではんがくしはらった 
\\	鉄道の料金は子供は大人の半額だ。	てつどうのりょうきんはこどもはおとなのはんがくだ 
\\	本屋はその本の代金の10ドルを私に請求した。	ほんやはそのほんのだいきんのじゅうどるをわたしにせいきゅうした 
\\	乗っ取り犯人の要求には応じられない。	のっとりはんにんのようきゅうにはおうじられない 
\\	万一この飛行機が乗っ取られたら、君は一体どうするだろうか。	まんいちこのひこうきがのっとられたら、きみはいったいどうするだろうか 
\\	軍が政府を乗っ取った。	ぐんがせいふをのっとった 
\\	万一突然英語で話しかけられたら、逃げ出すかもしれない。	まんいちとつぜんえいごではなしかけられたら、にげだすかもしれない 
\\	彼は出来心で盗みをした。	かれはできごころでぬすみをした 
\\	彼はその少女を盗み見た。	かれはそのしょうじょをぬすみみた 
\\	入国手続きって緊張しちゃう。	にゅうこくてつづきってきんちょうしちゃう 
\\	ビザを入手する手続きはどうしたらいいですか。	ビザをにゅうしゅするてつづきはどうしたらいいですか 
\\	スーはロイヤルホテルで宿泊手続きをした。	スーはロイヤルホテルでしゅくはくてつづきをした 
\\	登録用紙は無料で入手できます。	とうろくようしはむりょうでにゅうしゅできます 
\\	この本はただ一軒の店でだけ入手できる。	このほんはただいっけんでだけにゅうしゅできます 
\\	私たちは何とか外国切手を数枚入手できた。	わたしたちはなんとかがいこくきってをすうまいにゅうしゅできた 
\\	大統領は訪日を延期しました。	だいとうりょうはほうにちをえんきしました 
\\	来月フランス大統領が訪日する予定だ。	らいげつフランスだいとうりょうがほうにちするよていだ 
\\	規則にうるさいレフェリーは試合を台無しにしかねない。	きそくにうるさいレフェリーはしあいをだいなしにしかねない 
\\	二回目の上映がまもなく始まります。	にかいめのじょうえいがまもなくはじまります 
\\	知事自らテレビに出演した。	ちじみずからテレビにしゅつえんした 
\\	我々は丸一ヶ月間、ブロードウェーで出演予定です。	われわれはまるいっかげつかん、ブロードウェーでしゅつえんよていです 
\\	原子力が発電に使われている。	げんしりょくがはつでんにつかわれている 
\\	私は知事に立候補した。	わたしはちじにりっこうほした 
\\	知事の演説がその雑誌で発表された。	ちじのえんぜつがそのざっしではっぴょうされた 
\\	知事は委員会の反応に驚いた。	ちじはいいんかいのはんのうにおどろいた 
\\	彼は私が送った合図に反応した。	かれはわたしがおくったあいずにはんのうした 
\\	反応が早ければ点数が上がります。	はんのうがはやければてんすうがあがります 
\\	我々の体は感情に反応する。	われわれのからだはかんじょうにはんのうする 
\\	答案の点数を見て彼女はわっと泣き出した。	とうあんのてんすうをみてかのじょはわっとなきだした 
\\	彼は大統領に立候補するだろう。	かれはだいとうりょうにりっこうほするだろう 
\\	彼は無所属で立候補した。	かれはむしょぞくでりっこうほした 
\\	彼は代議士に立候補しています。	かれはだいぎしにりっこうほしています 
\\	ところで、平河代議士は、この件にどの程度タッチしているんでしょうか。	ところで、ひらがわだいぎしは、このけんにどのていどタッチしているんでしょうか 
\\	今回の雨には放射能はない。	こんかいのあめにはほうしゃのうはない 
\\	放射能が原子力発電所から漏れた。	ほうしゃのうがげんしりょくはつでんしょからもれた 
\\	その幼児は放射線に晒されていた。	そのようじはほうしゃせんにさらされていた 
\\	彼は危険に身を晒した。	かれはきけんにみをさらした 
\\	彼は故意に彼女を危険に晒した。	かれはこいにかのじょをきけんにさらした 
\\	幼児は病気にかかりやすい。	ようじはびょうきにかかりやすい 
\\	その若い女は腕に幼児を抱いていた。	そのわかいおんなはうでにようじをだいていた 
\\	この発電所だけで数個の市に電力を供給している。	このはつでんしょだけですうこのしにでんりょくをきょうきゅうしている 
\\	テレビからの小さい音でさえ、私の集中力を妨げる。	テレビからのちいさいおとでさえ、わたしのしゅうちゅうりょくをさまたげる 
\\	あなたは推量が当たった。	あなたはすいりょうがあたった 
\\	あなたの推量は的外れだ。	あなたのすいりょうはまとはずれだ 
\\	その少年は男らしくして泣くまいと頑張った。	そのしょうねんはおとこらしくしてなくまいとがんばった 
\\	私は彼女にそれについて聞こうか聞くまいか思案した。	わたしはかのじょにそれについてきこうかきくまいかしあんした 
\\	彼は思案に暮れている。	かれはしあんにくれている 
\\	私たちは予約をキャンセルすべきかどうか思案した。	わたしたちはよやくをキャンセルすべきかどうかしあんした 
\\	気に入ろうが入るまいが、君は行かねばならない。	きにいろうがいるまいが、きみはいかねばならない 
\\	彼がそれを私に貸そうと貸すまいとかまいません。	かれがそれをわたしにかそうとかすまいとかまいません 
\\	彼は二度とそれを繰り返すまいとかたい決心をした。	かれはにどとそれをくりかえすまいとかたいけっしんをした 
\\	少々のウイスキーを飲んでも害にはなりますまい。	しょうしょうのウイスキーをのんでもがいにはにりますまい 
\\	彼らは計画全体を諦めるよりほかないと意見が一致している。	かれらはけいかくぜんたいをあきらめるよりほかないといけんがいっちしている 
\\	良かれ悪しかれ、困難を乗り切るにはこうするより他ない。	よかれあしかれ、こんなんをのりきるにはこうするよりほかない 
\\	こうする他に手はなかったんだ。	こうするほかにてはなかったんだ 
\\	彼にはその病気を乗り切るだけの力がある。	かれはそのびょうきをのりきるだけのちからがある 
\\	彼の決断力のおかげで、彼はその危機を乗り切ることが出来た。	かれのけつだんりょくのおかげで、かれはそのききをのりきることができた 
\\	良かれ悪しかれ、テレビは世の中を変えた。	よかれあしかれ、テレビはよのなかをかえた 
\\	彼の顔は怖そうに見える反面、彼の声は優しく穏やかだった。	かれのかおはこわそうにみえるはんめん、かれのこえはやさしくおだやかだった 
\\	その仕事は十分にお金になるが、その反面1日に12時間働かなくてはならない。	そのしごとはじゅうぶんにおかねになるが、そのはんめんいちにじゅうにじかんはたらかなくてはならない 
\\	使いやすさはもとより、画質・音質の実力も高い。	つかいやすさはもとより、がしつ・おんしつのじつりょくもたかい 
\\	私のステレオはあなたのより音質が悪い。	わたしのステレオはあなたのよりおんしつがわるい 
\\	日本は穏やかな気候だ。	にほんはおだやかなきこうだ 
\\	私の父は穏やかな調子で話す。	わたしのちちはおだやかなちょうしではなす 
\\	時は虚偽のみならず真実も明らかにする。	ときはきょぎのみならずしんじつもあきらかにする 
\\	日本人学生は極度に内気であるのみならず、時としてほとんど全く話したがらないように見える。	にほんじんがくせいはきょくどにうちきであるのみならず、ときとしてほとんどまったくはなしたがらないようにみえる 
\\	彼が日本のみならず、世界においても有名な物理学者である。	かれはにほんのみならず、せかいにおいてもゆうめいなぶつりがくしゃである 
\\	彼女のみならず彼女の息子達も幸せだった。	かのじょのみならずかのじょのむすこたちもしあわせだった 
\\	彼はだれでも喜ばせたがる。	かれはだれでもあそばせたがる 
\\	男性は男らしく見せたがる。	だんせいはおとこらしくみせたがる 
\\	金持ちは人を軽蔑したがる。	かねもちはひとをけいべつしたがる 
\\	しきりに君と一緒に行きたがる。	しきりにきみといっしょうにいきたがる 
\\	賢い子供は人生や現実について知りたがる。	かしこいこどもはじんせいやげんじつについてしりたがる 
\\	政府はともすればマスメディアを統制したがる。	せいふはともすればマスメディアをとうせいしたがる 
\\	男の子達はみんな放課後フットボールをしたがる。	おとこのこたちはみんなほうかごフットボールをしたがる。 
\\	彼はともすれば子どもに甘かった。	かれはともすればこどもにあまかった 
\\	雪がしきりに降っていた。	ゆきがしきりにふっていた 
\\	彼は間違いをしきりに謝っていた。	かれはまちがいをしかりにあやまっていた 
\\	週末には町から逃れたいとしきりに思った。	しゅうまつにはまちからのがれたいとしかりにおもった 
\\	誰もが死を逃れる事ができない。	だれもしをのがれることができない 
\\	彼にいい加減な返事で言い逃れた。	かれはいいかげんなへんじでいいのがれた 
\\	私たちは辛うじて爆発から逃れた。	わたしたちはかろうじてばくはつからのがれた 
\\	政治家達は責任を逃れようとしている。	せいじかたちはせきにんをのがれようとしている 
\\	夢は時として当たる。	ゆめはときとしてあたる 
\\	時として大きな魚が釣れることもある。	ときとしておきいなさかながつれることもある 
\\	運命は時として残酷である。	うんめいはときとしてざんこくである 
\\	私が彼が嫌いなのは彼が残酷だからだ。	わたしがきらいなのはかれがざんこくだからだ 
\\	残酷ということは彼の性質にはないことだ。	ざんこくということはかれのせいかくにはないことだ 
\\	彼女は美しいばかりか、心も優しく、しかも、聡明である。	かのじょはうつくしいばかりか、こころもやさしく、しかも、そうめいである 
\\	労働はただ単に必要なものであるばかりか、楽しみでもある。	ろうどうはただたんにひつようなものであるばかりか、たのしみでもある 
\\	彼は学識ばかりでなく経験もある。	かれはがくしきばかりでなくけいけんもある 
\\	その提案には短所ばかりでなく長所もある。	そのていあんにはたんしょばかりでなくちょうしょもある 
\\	お腹が空いていたばかりでなく、私達は喉の渇きにも苦しんでいた。	おなかがすいていたばかりでなく、わたしたちはのどのかわきにもくるしんでいた 
\\	電流は磁力を発生することができる。	でんりゅうはじりょくをはっせいすることができる 
\\	彼女には何か磁力のようなものがある。	かのじょはなんかじりょくのようなもんがある 
\\	他人の悪口を言うもんじゃない。	たにんのわるぐちをいうもんじゃない 
\\	彼は研究の対象を広げた。	かれはけんきゅうのたいしょうをひろげた 
\\	彼の著作は非難の対象となった。	かれのちょさくはひなんのたいしょうとなった 
\\	この辞書は高校生を対象としたものです。	このじしょはこうこうせいをたいしょうとしたものです 
\\	その実験は未婚の男性100人を対象に行われた。	そのじっけんはみこんのだんせいひゃくにんをたいしょうにおこなわれた 
\\	彼はこの著作に大変骨を折った。	かれはこのちょさくにたいへんほねをおった 
\\	彼の他の著作と違い、この本は科学者達のためのものではなかった。	かれのほかのちょさくとちがい、このほんはかがくしゃたちのためのものではなかった 
\\	私は絵のことはよくわからないが、この筆のタッチは凄いと思う。	わたしはえのことはよくわからないが、おのふでのタッチはすごいとおもう 
\\	非合理的な会話が続いた。	ひごうりてきなかいわがつづいた 
\\	彼の議論はちっとも合理的ではなかった。	かれのぎろんはちっともごうりてきではなかった 
\\	もっと合理的な消費者になるように努めなさい。	もっとごうりてきなしょうひしゃになるようにつとめなさい 
\\	間違いを一度も認めないというのは不合理である。	まちがいをいちどもみとめないというのはふごうりである 
\\	結果にちっとも満足しなかった。	かっかにちっともまんぞくしなかった 
\\	これらのデータはちっとも正確ではない。	これらのデータはちっともせいかくではない 
\\	彼が試験に落ちても当然だ。ちっとも勉強しないんだから。	かれはしけんにおちてもとうぜんだ。ちっともべんきょうしないんだから 
\\	またのご来店、お待ちしております!	またのごらいてん、おまちしております! 
\\	次回は奈良を訪ねたいと思います。	じかいならをたずねたいとおもいます 
\\	次回はもっと早く来ていただけませんか。	じかいはもっとはやくきていただけませんか 
\\	言うまでもなく、軍隊の規律は文字通り厳しい。	いうまでもなく、ぐんたいのきりつはもじどおりきびしい 
\\	健康は富に勝ることは言うまでもない。	けんこうはとみにまさることはいうまでもない 
\\	正直が最上の策であることは言うまでもない。	しょうじきがさいじょうのさくであることはいうまでもない 
\\	言うまでもなく、誰でも法律を守る義務がある。	いうまでもなく、だれでもほうりつをまもるぎむがある 
\\	時間を超えて古代の昆虫たちが琥珀の中で生き生きと踊る。	じかんをこえてこだいのこんちゅうたちがこはくのなかでいきいきとおどる 
\\	彼はそれまで怠けていたので試験に落第した。	かれはそれまでなまけていたのでしけんにらくだいした 
\\	彼は試験に落第して始めて自分に怠惰を後悔した。	かれはしけんにらくだいしてはじめてじぶんにたいだをこうかいした 
\\	川は美しいからといって尊いわけではない。	かわはうつくしいからといってとうといわけではない 
\\	ロシア語は大層学びにくい。	ロシアごはたいそうまなびにくい 
\\	私は、その新しい雑誌が大層面白いとわかった。	わたしは、そのあたらしいざっしがたいそうおもしろいとわかった 
\\	ご友情は私にはたいそう尊いものです。	ごゆうじょうはわたしにはたいそうとうといものです 
\\	彼女が美しいからといって彼女を好きなわけではない。	かのじょがうつくしいからといってかのじょをすきなわけではない 
\\	警察官だからといってみな勇敢だとは限らない。	けいさつかんだからといってみなゆうかんだとはかぎらない 
\\	遅刻したからといって、彼女を叱る気になれなかった。	ちこくしたからといって、かのじょをしかるきになれなかった 
\\	幾ら親しいからといってそんな事を彼に頼めません。	いくらしたしいからといってそんなことをからにたのめません 
\\	彼は彼女と親しい間柄にある。	かれはかのじょとしたしいあいだがらにある 
\\	私の知人が私を彼の親しい友人達に紹介した。	わたしのちじんがわたしをかれのしたしいゆうじんたちにしょうかいした 
\\	彼とは20年以上も親しい間柄である。	かれはにじゅうねんいじょうもしたしいあいだがらである 
\\	トムとは口をきき合う間柄だ。	トムとはくちをききあうあいだがらだ 
\\	挨拶は礼儀作法の根本である。	あいさつはれいぎさほうのこんぽんである 
\\	想像力は、すべての文明の根本である。	そうぞうりょくは、すべてのぶんめいのこんぽんである 
\\	読み方をしっかり習うことは最も根本的なことである。	よみかたをしっかりならうことはもっともこんぽんてきなことである 
\\	空気は食べ物と同様、人間が根本的に必要とするものだ。	くうきはたべものとどうよう、にんげんがこんぽんてきにひつようとするものだ 
\\	展示会は大変印象的だった。	てんじかいはたいへんいんしょうてきだった 
\\	彼の最新の作品がその広場に展示されている。	かれのさいしんのさくひんがひろばにてんじされている 
\\	花の展示会で、彼らは彼女に1等賞を与えた。	はなのてんじかいで、かれらはかのじょにいっとうしょうをあたえた 
\\	コンサートチケットは当所で発売中です。	コンサートチケットはとうしょではつばいちゅうです 
\\	透明度はどれくらいですか。	とうめいどはどんれくらいですか 
\\	この透明な液体には毒が含まれている。	このとうめいなえきたいにはどくがふくまれている 
\\	氷が溶けると液体になる。	こおりがとけるとえきたいになる 
\\	その液体は強烈な匂いを発した。	そのえきたいはきょうれつなにおいをはっした 
\\	スポンジは液体を吸い取る。	スポンジはえきたいをすいとる 
\\	その薬の効果は強烈だが短い。	そのくすりのこうかはきょうれつだかみじかい 
\\	彼の報復は鼻への強烈なパンチだった。	かれのほうふくははなへのきょうれつなパンチだった 
\\	彼女は身の回りのあらゆる物に強烈な興味を持っている。	かのじょはみのまわりのあらゆるものにきょうれつなきょうみをもっている 
\\	報復の脅かしが交渉を妨げています。	ほうふくのおどかしがこうしょうをさまたげています 
\\	私は彼に報復するつもりだ。	わたしはかれにほうふくするつもりだ 
\\	宇宙にはたくさんの銀河がある。	うちゅうにはたくさんのぎんががある 
\\	銀河系には無数の星がある。	ぎんがけいにはむすうのほしがある 
\\	投資家グループは企業買収を企てています。	とうしかグループはきぎょうばいしゅうをくわだてています 
\\	どんな人でも買収できるものだ。	どんなひとでもばいしゅうできるものだ 
\\	彼らはその証人を買収しようとしたが駄目だった。	かれらはそのしょうにんをばいしゅうしようとしたかだめだった 
\\	もうじき林檎の収穫期になる。	もうじきりんごのしゅうかくきになる 
\\	誰にとっても自分の性格を客観的に見ることは非常に困難なことである。	だれにとってもじぶんのせいかくをきゃっかんてきにみることはひじょうにこんなんなことである 
\\	客観的に見て、彼の主張は全く理にかなっていなかった。	きゃっかんてきにもて、かれのしゅちょうをまったくりにかなっていなかった 
\\	受け持ちの患者に客観的になる事は看護婦にとって困難だ。	うけもちのかんじゃにきゃっかんてきになることはかんごふにとってこんなんだ 
\\	彼の意見は道理にかなっている。	かれのいけんはどうりにかなっている 
\\	その計画は理にかなっていると我々全員一致した。	そのけいかくはりにかなっているとわれわれぜんいんいっちした 
\\	理論は理にかなっていたけれども、彼は納得しなかった。	りろんはりにかなっていたけれども、かれはなっとくしなかった 
\\	彼の文章はとても主観的だ。	かれのぶんしょうはとてもしゅかんてきだ 
\\	いろいろ考えてみると私の主観ですが。	いろいろかんがえてみるとわたしのしゅかんですが 
\\	彼はジャーナリズム界の大物です。	かれはジャーナリズムかいのおおものです 
\\	仲介役がしっかりしていると取り引きがスムーズに行く。	ちゅうかいやくがしっかりしているととりひきがスムーズにいく 
\\	この論文では交渉における仲介者の立場に関する困難点は何かという問題をとりあげる。	このろんぶんではこうしょうにおけるちゅうかいしゃのたちばにかんするこんなんてんはなにかというもんだいととりあげる 
\\	誰が最初にその問題を取り上げるのだろうか。	だれがさいしょにそのもんだいをとりあげるのだろうか 
\\	彼は急いで受話器を置いた。	かれはいそいでじゅわきをおいた 
\\	電話が鳴る。スーザンは受話器を取り上げる。	でんわがなる。スーザンはじゅわきをとりあげる 
\\	彼の願望は商売を始める事です。	かれのがんぼうはしょうばいをはじめることです 
\\	ほっそりとした姿に対する若い女性の願望は強い。	ほっそりとしたすがたにたいするわかいじょせいのがんぼうはつよい 
\\	医者になりたいという願望は病弱な弟の世話をしたことから芽生えた。	いしゃになりたいというがんぼうはびょうじゃくなおとうとのせわをしたことからめばえた 
\\	太郎と花子の間に愛が芽生えた。	たろうとはなこのあいだにあいがめばえた 
\\	二人の間に愛情が芽生えた。	ふたりのあいだにあいじょうがめばえた 
\\	彼の病弱が研究の妨げになった。	かれのびょうじゃくがけんきゅうのさまたげになった 
\\	私の弟はずっと病弱です。	わたしのおとうとはずっとびょうじゃくです 
\\	彼女はすぐにお世辞に乗りやすい。	かのじょはすぐにおせじにのりやすい 
\\	お世辞には気をつけよう。	おせじにはきをつけよう 
\\	彼らは彼の勤勉さをほめてお世辞を言った。	かれらはかれのきんべんさをほめておせじをいった 
\\	お前どこで油を売っていたんだ。	おまえどこであぶらをうっていたんだ 
\\	大いに充実した学生生活を送りたい。	おおいにじゅうじつしたがくせい生活をおくりたい 
\\	ここ品物は高いけど、その分アフターサービスが充実してるから。	ここしなものはたきけど、そのぶんアフターサーにスがじゅうじつしてるから 
\\	マンツーマンで指導して頂けたという点でも、非常に充実した実習になった。	マンツーマンでしどうしていただけたというてんでも、ひじょうにじゅうじつしたじっしゅうになった 
\\	彼は受験資格がない。	かれはじゅけんしかくがない 
\\	私は受験勉強に専念するつもりです。	わたしはじゅけんべんきょうにせんねんするつもりです 
\\	ジムは受験勉強のころは夜中まで勉強した。	ジムはじゅけんべんきょうのころはよなかまでべんきょうした 
\\	彼は原稿を繰り返し、繰り返し書き直している。	かれはげんこうをくりかえし、くりかえしかきなおしている 
\\	この件についてご協力いただけたら、ありがたいと思います。	このけんについてきょうりょくいただけたら、ありがたいとおもいます 
\\	はすっかりレコード取って代わった。	残るもののないさま。ことごとく。
\\	効率的な機械が肉体労働に取って代わった。	こうりつてきなきかいがにくたいろうどうにとってかわった 
\\	私は肉体労働には向いていない。	わたしはにくたいろうどうにはむいていない 
\\	故に、筋肉の50パーセントが脂肪に取って代わられる可能性がある。	ゆえに、きんにくのごじゅうパーセントがしぼうにとってかわられるかのうせいがある 
\\	彼は事務効率をよくする案を出した。	かれはじむこうりつをよくするあんをだした 
\\	このやり方は、カロリーを得がたい環境ではエネルギー効率がよい。	このやりかたは、カロリーをえがたいかんきょうではエネルギーこうりつがよい 
\\	彼は今書類を作成しています。	かれはいましょるいをさくせいしています 
\\	報告書は委員会によって作成されている。	ほうこくしょはいいんかいによってさくせいされている 
\\	その報告書は急いで作成されていたので、いくつかの綴りの間違いがあった。	そのほうこくしょはいそいでさくせいされたので、いくつかのつづりのまちがいがあった 
\\	彼は彼女に綴りの誤りを指摘した。	かれはかのじょびつづりのあやまりをしてきした 
\\	私に綴りのまちがいが全くないことなどは期待できません。	わたしにつづりのましがいがまったくないことなどはきたいできません 
\\	肉体は余分なカロリーを脂肪に変える。	にくたいはよぶんなカロリーをしぼうにかえる 
\\	脂肪分は減らしたほうがいいですね。	しぼうぶんをへらしたほうがいいですね 
\\	昭和10年は西暦1935年です。	しょうわじゅうねんはせいれきせんさんじゅうごねんです 
\\	船は嵐で左右に揺れた。	ふねはあらしでさゆうにゆれた 
\\	彼は左右交互に手を上げた。	かれはさゆうこうごにてをあげた 
\\	彼は怒って首を左右に振った。	かれはおこってくびをさゆうにふった 
\\	ぶどう酒の味は天候に大きく左右される。	ぶどうしゅのあじはてんこうにおおきくさゆうされた 
\\	我々は決定にあたって偏見に左右されない。	われわれはけっていにあたってへんけんにさゆうされない 
\\	成功不成功は気質に左右されることが多い。	せいこうふせいこうはきしつにさゆうされることがおおい 
\\	ビールの消費量は天気におおいに左右される。	ビールのしょうひりょうはてんきにおおいにさゆうされる 
\\	釘で彼の上着が裂けた。	くぎでかれのうわぎがさけた 
\\	その板に釘を打ってください。	そのいたにくぎをうってください 
\\	私は釘に引っ掛かって上着を破ってしまった。	わたしはくぎにひっかかってうわぎをやぶってしまった 
\\	出る釘は打たれる。	でるくぎはうたれる 
\\	父は袖に長い裂け目を作った。	ちちはそでにながいさけめをつくった 
\\	私は壁の裂け目から外を覗いた。	わたしはかべのさけめからそとをのぞいた 
\\	だれも彼らの仲を引き裂けない。	だれもかれらのなかをひきさけない 
\\	昼と夜が交互にくる。	ひるとよるがこうごにくる 
\\	好運と不幸は交互に起こる。	こううんとふこうはこうごにおこる 
\\	彼らは交互に車を運転した。	かれらはこうごにくるまをうんてんした 
\\	私は交互に楽観主義になったり悲観主義になる。	わたしはこうごにらっかんしゅぎになったりひかんしゅぎになる 
\\	奇数と偶数は交互に現れる。	きすうとぐうすうはこうごにあらわれる 
\\	偶数掛ける奇数は偶数、奇数掛ける奇数は奇数。	ぐうすうかけるきすうはぐうすう、きすうかけるきすうはきすう 
\\	戸棚を覗いた。	とだなをのぞいた 
\\	戸棚の中の金はみんな盗まれた。	とだなのなかのかねはみんなぬすまれた 
\\	私たちは2時間の労働と10分の休憩を交互にとった。	わたしたちはにじかんのろうどうとじゅっぷんのきゅうけいをこうごになった 
\\	板は地面に凍りついた。	いたはじめんにこおりついた 
\\	アンは黒板に何か書いた。	アンはこくばんになにかかいた 
\\	彼は黒板に正方形を二つ書いた。	かれはこくばんにせいほうけいをふたつかいた 
\\	正方形を2等分せよ。	せいほうけいをにとうぶんせよ 
\\	正方形は4つの同じ長さの辺をもつ。	せいほうけいはよっつのおなじながさのへんをもつ 
\\	この線を20等分せよ。	このせんをにじゅうとうぶんせよ 
\\	そのメロンをきって6等分しなさい。	そのメロンをきってろくとうぶんしなさい 
\\	親孝行したいときには親はなし。	おやこうこうしたいときにはおやはなし 
\\	親孝行な息子さんがいるから、老後の心配しなくていいわね。	おやこうこうなむすこさんがいるから、ろうごのしんぱいしなくてもいいわね 
\\	湖は直径3マイルある。	みずうみはちょっけいさんマイルある 
\\	その丸は半径100センチある。	そのまるははんけいひゃくセンチある 
\\	直径1メートル、深さ2メートルの穴を掘るのに、約2時間半かかりました。	ちょっけいいちメートル、ふかさにメートルのあなをほるのに、やくにじかんはんかかりました 
\\	二線は平行するとせよ。	にせんはへいこうするとせよ 
\\	2本の道路は平行に走っている。	にほんのどうろはへいこうにはしっている 
\\	平行なスキーに等しく体重をかけて滑りなさい。	へいこうなスキーにひとしくたいじゅうをかけてすべりなさい 
\\	すべての読み物が等しく読む価値があるわけではない。	すべてのよみものがひとしくよむかちがあるわけではない 
\\	命が縮む思いをした。	いのちがちぢむおもいをした 
\\	少女は立って鏡を覗き込んだ。	しょうじょはたってかがみをのぞきこんだ 
\\	地震のあと、人々は驚いて地面の深い穴をじっと覗き込んだ。	じしんのあと、ひとびとはおどろいてじめんのふかいあなをじっとのぞきこんだ
\\	楽観主義者は鏡を覗き込んでますます楽天的に、悲観論者はますます悲観的になる。	らっかんしぎしゃはかがみをのぞきこんでますますらくてんてきに、ひかんろんしゃはますますひかんてきになる 
\\	彼は自分が無神論者だと告白した。	かれはじぶんがむしんろんしゃだとこくはくした 
\\	セーターが洗濯で縮んだ。	セーターがせんたくでちぢんだ 
\\	乞食は選択者にはなれない。	こじきはせんたくしゃにはなれない 
\\	職業選択の際に先生が相談に乗ってくれた。	しょくぎょうせんたくのさいにせんせいがそうだんにのってくれた 
\\	それでも、早期の退職を選択する人は多い。	それでも、そうきのたいしょくをせんたくするひとはおおい 
\\	その事件の早期解決を期待する。	そのじけんのそうきかいけつをきたいする 
\\	日曜日は日曜と短縮する。	にちようびはにちようとたんしゅくする 
\\	括弧内の言葉を短縮形にしなさい。	【かっこ】 特定の文字・語句・文などを囲って他の部分と区別する記号。
\\	など。また、その記号をつけること。
\\	パーマをかけたので、髪の毛が縮れています。	パーマをかけたので、かみのけがちぢれています 
\\	この服は縮んで、おまけに色褪せてしまった。	このふくはちぢんで、おまけにいろあせてしまった 
\\	彼らは労働時間の短縮を要求している。	かれらはろうどうじかんをたんしゅくをようきゅうしている 
\\	彼女は色褪せた綿のスカートをはいていた。	かのじょはいろあせためんのスカートをはいていた 
\\	私は疲れていた、おまけに眠かった。	わたしはつかれていた、おまけにねむかった 
\\	もしも私が生まれ変わるなら、鳥になりたい。	もしもわたしがうまれかわるなら、とりになりたい 
\\	もし生まれ変われたら、金持ちの家の子になりたい。それで何不自由なく暮らしたい。	もしうまれかわれたら、かねもちのいえのこになりたい。それでなにふじゆうなくくらしたい 
\\	顔はともかく、気立てはとてもいい。	かおはともかく、きだてはとてもいい 
\\	冗談はともかく、君の頭脳は医者にみせるべきだ。	じょうだんはともかく、きみのずのうはいしゃにみせるべきだ 
\\	彼はその破片をくっつけ合わした。	かれはそのはへんをくっつけあわした 
\\	彼は壊れた花瓶の破片をくっつけようとした。	かれはこわれたかびんのはへんをくっつけようとした 
\\	返答は肯定的なものだった。	へんとうはこうていてきなものだった 
\\	彼は肯定的な答えを出した。	かれはこうていてきなこたえをだした 
\\	ディベートとは肯定側と否定側で交わされる知的ゲームである。	ディベートとははこうていがわとひていがわでかわせるちてきゲームである 
\\	事故が起きる可能性は否定できない。	じこがおきるかのうせいはひていできない 
\\	彼が利口だということは否定できない。	かれがりこうだということはひていできない 
\\	市長は賄賂を受け取ったことを否定した。	しちょうはわいろをうけとったことをひていした 
\\	その大臣は自分の言ったことを否定した。	そのだいじんはじぶんのいったことをひていした 
\\	商人はその政治家に賄賂を送った。	しょうにんはそのせいじかにわいろをおくった 
\\	彼は賄賂の受け取りを拒否した。	かれはわいろのうけとりをきょひした 
\\	私がその賄賂を拒否したのは非常に賢明な事だった。	わたしがそのわいろをきょひしたのはひじょうにけんめいなことだった 
\\	彼女の返答はいつも私の予想どおりだ。	かのじょのへんとうはいつもわたしのよそうどうりだ 
\\	彼の返答は簡単で要領を得ていた。	かれのへんとうはかんたんでようりょうをえていた 
\\	あなたの返答はほとんど脅迫に等しい。	あなたのへんとうはほとんどきょうはくにひとしい 
\\	私は彼の言葉を脅迫と解した。	わたしはかれのことばをきょうはくとかいした 
\\	市長の家族は一日中脅迫電話に悩まされた。	しちょうのかぞくはいちにちちゅうきょうはくでんわになやまされた
\\	彼の要領を得ない話しにうんざりした。	かれのようりょうをえないはなしにうんざりした 
\\	その批評家の言うことはいつも簡潔で要領を得ている。	そのひひょうかのことばはいつもかんけつでようりょうをえている 
\\	その臆病な兵士は恐ろしい悪夢に悩まされた。	そのおくびょうなへいしはおそろしあくむになやまされた 
\\	昨夜に見た生なましい悪夢がまだ頭から離れない。	さくやにみたなまなましいあくむがまだあたまからはなれない 
\\	此処は大型車の通行は禁止されている。	ここはおおがたしゃのつうこうはきんしされている 
\\	今日の新聞によれば、大型の台風が接近中のようだ。	きょうのしんぶんによれば、おおがたのたいふうがせっきんちゅうのようだ 
\\	新しい校長が学校を管理運営している。	あたらしいこうちょうががっこうをかんりうんえいしている 
\\	委員会はその計画をめぐって意見が分かれた。	いいんかいはそのけいかくをめぐっていけんがわかれた 
\\	先生たちはその問題をめぐって賛否が分かれた。	せんせいたちはそのもんだいをめぐってさんぴがわかれた 
\\	その法案には賛否の議論がたくさんあった。	そのほうあんびはさんぴのぎろんがたくさんあった 
\\	その映画には賛否両論が出た。	そのえいがにはさんぴりょうろんがでた 
\\	その土地をどうするかをめぐって反目が生じた。	そのとちをどうするかをめぐってはんもくがしょうじた 
\\	二人の反目の原因は金だ。	ふたりのはんもくのげんいんはかねだ 
\\	彼はその選挙で対立候補を破った。	かれはそのせんきょでたいりつこうほをやぶった 
\\	彼は上司と対立を避けようとした。	かれはじょうしとたいりつをさけようとした 
\\	その争いの根源は二国間の対立関係にある。	そのあらそいのこんげんにこくかんのたいりつかんけいにある 
\\	人を嫌うのは言うまでもなく、人を信頼できないことが、人間の苦しみの根源だ。	ひとをきらうのはいうまでもなく、ひとをしんらいできないことが、にんげんのにくしみのこんげんだ 
\\	彼は対立を引き起こそうと、わざとしつこくした。	かれはたいりつをひきおこそうと、わざとしつごくした 
\\	その林檎を採って半分に切りなさい。	そのリンゴをとってはんぶんにきりなさい 
\\	スペインはオレンジがたくさん採れる。	スペインはオレンジがたくさんとれる。 
\\	試合は晴雨を問わず行われます。	しあいはせいうをとわずおこなわれます 
\\	マラソンは晴雨に関わらず開かれます。	マラソンはせいうにかかわらずひらかれます 
\\	晴雨にかかわらず、開会式は9時に始まる予定です。	せいうにかかわらず、かいかいしきはくじにはじまるよていです 
\\	開会式は昨日催された。	かいかいしきはきのうもよおされた 
\\	試合にさきだち代々木競技場で開会式が行われた。	しあいにさきだちよよぎきょうぎじょうでかいかいしきがおこなわれた 
\\	年齢を問わず人々はこの歌が好きだ。	ねんれいをとわずひとびとはこのうたがすきだ 
\\	その試合には国籍のいかんを問わず誰でも参加出来る。	そのしあいにはこくせきのいかんをとわずだれでもさんかできる 
\\	肌の色のいかんを問わず、彼は万人の言論の自由を擁護した。	はだのいろのいかんをとわず、かれはばんじんのげんろんのじゆうをようごした 
\\	国会は現在閉会中である。	こっかいはげんざいへいかいちゅうである 
\\	誰もそれ以上言わなかったので、閉会した。	だれもそれいじょういわなかったので、へいかいした 
\\	彼は閉会式には沢山の客を招待することを計画するでしょう。	かれはへいかいしきにはたくさんのきゃくをしょうたいするをけいかくするでしょう 
\\	彼はひどく動揺していたので、善悪の区別が出来なかった。	かれはひどくどうようしていたので、ぜんあくのくべつができなかった 
\\	その交通事故の知らせに私は動揺した。	そのこうつうじこのしらせにわたしはどうようした 
\\	製造部門は新しい金融政策に動揺しています。	せいぞうぶもんはあたらしいきんゆうせいさくにどうようしている 
\\	完全な宗教の自由が万人に保証されている。	かんぜんなしゅうきょうのじゆうがばんじんにほしょうされている 
\\	トランプでインチキをするという良い発想を偶然思い付いた。	トランプでインチキをするというよいはっそうをぐうぜんおもいついた 
\\	インチキなセールスマンに騙されて、役立たずの機械を買わされたとジョンは主張した。	インチキなセールスマンにだまされて、やくだたずのきかいをかわされたとジョンはしゅちょうした 
\\	ピアノ・コンクールで私が第1位になるなんて夢想だにしなかった。	ピアノ・コンクールでわたしがだいいちいになるなんてむそうだにしなかった 
\\	彼女は一言でいえば夢想家だ。	かのじょはひとことでいえばむそうかだ 
\\	買い物ついでにでもお寄りください。	かいものついでにでもおよりください 
\\	ガソリンを満タンにして、ついでにオイルも見てくれる。	ガソリンをまんタンにして、ついでにオイルもみてください 
\\	先生はいろいろな種類の花を見せてくれたついでに、これらの花は自分の家の庭から持ってきた物だと言った。	せんせいはいろいろなしゅるいのはなをみせてくれたついでに、これらのはなはじぶんのいえのにわからもってきたものだといった 
\\	あなたは気が狂うだろう。	あなたはきがくるうだろう 
\\	あんな風に振る舞うなんて彼は気が狂ってるに違いない。	あんなふうにふるまうなんてかれはきがくるってるにちがいない 
\\	ラジオを消してくれないと気が狂いそうだ。	ラジオをけしてくれないときがくるいそうだ 
\\	彼は悲しみで気も狂わんばかりだった。	かれはかなしみできもくるわんばかりだった 
\\	私は恐怖で気も狂わんばかりだった。	わたしはきょうふできもくるわんばかりだった 
\\	彼女は琴を弾くことがとても好きだ。	かのじょはことをひくことがとてもすきだ 
\\	その農民は200エーカーの農園を耕した。	そののうみんはにひゃくエーカーののうえんをたがやした 
\\	農業は穀物を育てるための土壌の耕作と定義される。	のうぎょうはこくもつをそだてるためのどうじょうのこうさくとていぎされた 
\\	300年間、彼らは周囲の土地を耕作してきた。	さんびゃくねんかん、かれらはしゅういのとちをこうさくしてきた 
\\	農夫は土地を耕す。	のうふはとちをたがやす 
\\	農場主は一日中畑を耕した。	のうじょうしゅはいちにちちゅうはたけをたがやした 
\\	オフィスにはなごやかな雰囲気がある。	ものやわらかなさま。穏やかなさま。
\\	彼は人見知りする。	かれはひとみしりする 
\\	筆者はそうした風潮を好まない。	ひっしゃはそうしたふうちょうをこのまない 
\\	この記事の筆者は有名な批評家だ。	このきじのひっしゃはゆうめいなひひょうかだ 
\\	この本で筆者は日本とアメリカを対照させている。	このほんでひっしゃはにほんとアメリカをたいしょうさせている 
\\	彼は世の風潮に逆らう。	かれはよのふうちょうにさからう 
\\	だれしも世の風潮には抵抗しがたいものだ。	だれしもよのふうちょうにはていこうしがたいものだ 
\\	少しは世間の風潮に合わせるほうが賢明かもしれない。	すこしはせけんのふうちょうにあわせるほうがけんめいかもしれない 
\\	都会生活と田園生活との対照的な相違。	とかいせいかつとでんえんせいかつとのたいしょうてきなそうい 
\\	彼は農園を売って大金を握った。	かれはのうえんをうってたいきんをにぎった 
\\	彼女は赤ん坊で手いっぱいだ。	かのじょはあかんぼうでていっぱいだ 
\\	君の言っていることは机上の空論にすぎないよ。	きみのいっていることはきじょうのくうろんにすぎないよ 
\\	零細小売店などは新年度には利益を上げるでしょう。	れいさいこうりてんなどはしんねんどにはりえきをあげるでしょう 
\\	彼らは苦労して1997会計年度の予算を作成した。	かれらはくろうしてせんきゅうひゃくななかいけいねんどのよさんをさくせいした 
\\	日本の新会計年度の予算は通常12月に編成される。	にほんのしんかいけいねんどのよさんはつうじょうじゅうにがつにへんせいされる 
\\	この列車は七両編成です。	このれっしゃはななりょうへんせいです 
\\	営業活動を強化するために再編成する必要があります。	えいぎょうかつどうをきょうかするためにさいへんせいするひつようがあります 
\\	零細企業はインフレで苦境に陥っています。	れいさいきぎょうはインフレでくきょうにおちいっている 
\\	零細なパン屋はスーパーマーケットに圧倒された。	れいさいなパンやはスーパーマーケットにあっとうされた 
\\	彼女は夫が苦境にある時はいつも手助けをした。	かのじょはおっとがくきょうにあるときはいつもたすけをした 
\\	彼はその苦境を克服した。	かれはそのくきょうをこくふくした 
\\	彼は彼女の愛情の強さに圧倒された。	かれはかのじょのあいじょうのつよさにあっとうされた 
\\	私たちの行動の圧倒的な部分は学んで身についたものだ。	わたしたちのこうどうのあっとうてきなぶぶんはまなぶでみについたものだ 
\\	彼らは圧倒的に優勢な敵と戦っていた。	かれらはあっとうてきにゆうせいなてきとたたかっていた 
\\	議案は圧倒的多数で可決された。	ぎあんはあっとうてきたすうでかけつされた 
\\	わが国は危機に陥っている。	わがくにはききにおちいっている 
\\	難民に人道的援助を行いました。	なんみんにじんどうてきえんじょをおこないました 
\\	それは罪人に対する最も人道的な刑罰ではないか。	それはざいにんにたいするもっともじんどうてきなけいばつではないか 
\\	圧倒的多数がその残酷な刑罰を廃止することに票を投じた。	あっとうてきたすうがそのざんこくなけいばつをはいしすることにひょうをとうじた 
\\	彼は煙に巻かれて窒息した。	かれはけむりにまかれてちっそくした 
\\	お菓子で赤ちゃんが窒息するところだった。	おかしであかちゃんがちっそくするところだった 
\\	もちを食べている時にのどに詰まらせて窒息死する老人がたくさんいる。	もちをたべているときにのどにつまらせてちっそくしするろうじんがたくさんいる 
\\	彼女はパンを喉に詰まらせた。	かのじょはパンをのどにつまらせた 
\\	彼には嘘をでっち上げる癖があるのは本当だ。	かれはうそをでっちあげるくせがあるのはほんとうだ 
\\	被告人は法廷で事実と違う話をでっち上げた。	ひこくにんはほうていでじじつとちがうはなしをでっちあげた 
\\	被告は金曜日に法廷に現れる予定です。	ひこくはきにょうびにほうていにあらわれるよていです 
\\	その写真は法廷内の緊張を非常によくとらえている。	そのしゃしんはほうていないのきんちょうをひじょうによくとらえている 
\\	その遺言は法廷で無効と宣告された。	そのゆいごんはほうていでむこうとせんこくされた 
\\	彼は遺言状も作らずに死んだ。	かれはゆいごんじょうもつくらずにしんだ 
\\	彼は弟に遺言を実行してくれと頼んだ。	かれはおとうとにゆいごんをじっこうしてくれとたのんだ 
\\	法廷は遺言状が有効であるとの判決を下した。	ほうていはゆいごんじょうがゆうこうであるとのはんけつをくだした 
\\	この証書は完全に無効である。	このしょうしょはかんぜんにむこうである 
\\	脅迫のもとになされた約束は無効だ。	きょうはくのもとになされたやくそくはむこうだ 
\\	自伝の中で彼は繰り返し不幸な少年時代に言及している。	じでんのなかでかれはくりかえしふこうなしょうねんじだいにげんきゅうしている 
\\	伝記を書くことが難しいのは、それが半ば記録であり、半ば芸術であるからだ。	でんきをかくことがむずかしいのは、それがなかばきろくであり、なかばげいじゅつであるからだ 
\\	偉大な天才は最短の伝記を有する。	いだいなてんさいはさいたんのでんきをゆうする 
\\	彼の伝記は全くの事実に即して書かれたものだ。	かれのでんきはまったくのじじつにそくしてかかれたものだ 
\\	東京では、11月半ばに寒い季節が始まります。	全体を二つに分けた、その一方。半分。
\\	成功を望まない人は誰一人いない。	せいこうをのぞまないひとはだれひとりいない 
\\	誠実にして貧しいのは不正な手段で得た富より望ましい。	せいじつにしてまずしいのはふせいなしゅだんでえたとみよりのぞましい 
\\	その建物の骨組みは今や完成している。	そのたてもののほねぐみはいまやかんせいしている 
\\	新しい建物の骨組みが徐々に姿を見せてきている。	あたらしいたてもののほねぐみがじょじょにすがたをみせてきている 
\\	今年はコレラ患者が多発した。	ことしはコレラかんじゃがたはつした 
\\	暴力や脅しによって金品を奪い去る事件も多発している。	どうりょくやおどしによってきんぴんをうばいさるじけんもたはつしている 
\\	男はその老婦人から金品を強奪したことを白状した。	おとこはそのろうふじんからきんぴんをごうだつしたことをはくじょうした 
\\	途中で文書を強奪された。	とちゅうでぶんしょをごうだつされた 
\\	昨夜、姉は帰宅の途中でバッグを強奪された。	さくや、あねはきたくのとちゅうでバッグをごうだつされた 
\\	大統領は記者会見を行った。	だいとうりょうはきしゃかいけんをおこなった 
\\	大臣は記者団との会見を拒んだ。	だいじんはきしゃだんとのかいけんをこばんだ 
\\	彼は我々の記者会見の申し出を拒否した。	かれはわれわれのきしゃかいけんのもうしでをきょひした 
\\	テレビは暴力行為を見せて、それがとりわけ年少の者たちに影響を及ぼす。	テレビはぼうりょくこういをみせて、それがとりわけねんしょうのものたちにえいきょうをおよぼす 
\\	英語は世界で取り分け最も広く普及している言語である。	えいごはせかいでとりわけもっともひろくふきゅうしているげんごである 
\\	昔この習慣は日本中で普及していた。	むかしこのしゅうかんはにほんちゅうでふきゅうしていた 
\\	戦後日本では民主主義の理念が普及した。	せんごにほんではみんしゅしゅぎのりねんがふきゅうした 
\\	エイズウイルスの普及は恐るべき速さで進んでいる。	エイズウイルスのふきゅうはおそるべきはやさですすんでいる 
\\	テレビの普及によって我々の読書の時間がかなり奪われている。	テレビのふきゅうによってわれわれのどくしょのじかんがかなりうばわれている 
\\	これらの理念は憲法に具体化されている。	これらのりねんはけんぽうにぐたいかされている 
\\	とりわけ我々は利己主義になってはならない。	特に。ことに。とりわけて。
\\	とりわけ、論理学には正確な定義が要求される。	とりわけ、ろんりがくにはせいかくなていぎがようきゅうされる 
\\	とりわけ人目をひいたのは、彼女の卵型の顔立ちだった。	とりわけひとめをひいたのは、かのじょのたまごがたのかおだちだった 
\\	パーティーの間中エマの存在は特に人目を引いた。	パーティーのあいだじゅうエマのそんざいはとくにひとめをひいた 
\\	講演者は話の間中メモを参照しなかった。	こうえんしゃははなしのあいだじゅうメモをさんしょうしなかった 
\\	作家はよく辞書を参照する。	さっかはよくじしょをさんしょうする 
\\	質問があれば、このガイドブックを参照してください。	しつもんがあれば、このガイドブックをさんしょうしてください 
\\	父は家族の居間を広くした。	ちちはかぞくのいまをひろくした 
\\	私たちは居間の半分の場所を取るグランドピアノを買った。	わたしたちはいまのはんぶんのばしょをとるグランドピアノをかった 
\\	ちょっと勘弁して下さい。	ちょっとかんべんしてください 
\\	戦時中政府は肉を配給にした。	せんじちゅうせいふはにくをはいきゅうにした 
\\	彼らはパンと牛乳の追加配給を交渉で決めた。	かれらはパンとぎゅうにゅうのついかはいきゅうをこうしょうできめた 
\\	イギリスと日本とは、政治の仕組みにかなり共通点がある。	イギリスとにほんとは、せいじのしくみにかなりきょうつうてんがある 
\\	経済学は経済の仕組みを研究する学問である。	けいざいがくはけいざいのしくみをけんきゅうするがくもんである 
\\	私はこの仕組みを知りませんが担当者が説明するでしょう。	わたしはこのしくみをしりませんがたんとうしゃがせつめいするでしょう 
\\	私たちは結婚を前提として交際しています。	わたしはけっこんをぜんていとしてこうさいしています 
\\	その前提が妥当かどうかよく考えるべきだ。	そのぜんていがだとうかどうかよくかんがえるべきだ 
\\	私たちはその基礎となっている前提を理解すべきだった。	わたしたちはそのきそとなっているぜんていをりかいすべきだった 
\\	人生とは不十分な前提から十分な結論を引き出す技術である。	じんせいとはふじゅうぶんのぜんていからじゅうぶんなけつろんをひきだすぎじゅつである 
\\	僕は君を全面的に支持する。	ぼくはきみをぜんめんてきにしじする 
\\	だれもが全面的な改革を要求している。	だれもがぜんめんてきなかいかくをようきゅうしている 
\\	君は方向音痴だ。	きみはほうこうおんちだ 
\\	私は音痴だから歌いたくない。	わたしはおんちだからうたいたくない 
\\	私は方向音痴なのでいつも方位磁石を持ち歩いています。	わたしはほうこうおんちなのでいつもほういじしゃくをもちあるいている 
\\	磁石は鉄を引きつける。	じしゃくはてつをひきつける 
\\	頭の上の磁石がその原因だった。	あたまをうえのじしゃくがそのげんいんだった 
\\	私はまだ時差ぼけに苦しんでいます。	わたしはまだじさぼけにくるしんでいます 
\\	海外にいくとかならず、時差ぼけと下痢になやまされる。	かいがいにいくとかならず、じさぼけとげりになやまされる 
\\	私はこれらの統計数値を政府の教育白書から借りた。	わたしはこれらのとうけいすうちをせいふのきょういくはくしょからかりた 
\\	経済白書を読んでから、我が国の財政状態が私に正しく、わかってきた。	けいざいはくしょをよんでから、わがくにのざいせいじょうたいがわたしにただしく、わかってきた 
\\	血圧計の数値は?	けつあつけいのすうちは? 
\\	気圧計の数値が下がっている。雨になりそうだな。	きあつけいのすうちがさがっている。あめになりそうだな 
\\	大草原では草以外何も見えなかった。	だいそうげんではくさいがいなにもみえなかった 
\\	野生の動物たちが草原を歩き回っていた。	やせいなどうぶつたちがそうげんをあるきまわっている 
\\	そこに行く道がわかるように、近くの目印を教えて下さい。	そこにいくみちがわからるように、ちかくのめじるしをおしえてください 
\\	彼女は自分の傘に赤いリボンで目印を付けた。	かのじょはじぶんのかさにあかいリボンでめじるしをつけた 
\\	彼は海外留学をしただけのことはあった。	かれはかいがいりゅがくをしただけのことはあった 
\\	このうれしい知らせを家族に知らせたくてたまらない。	このうれしいしらせをかぞくにしらせたくてたまらない 
\\	彼女は故郷が恋しくてたまらなかった。	かのじょはこきょうがこいしくてたまらなかった 
\\	彼は先ごろの成功を自慢したくてたまらない。	かれはさきごろのせいこうをじまんしたくてたまらない 
\\	子供達は風船が欲しくてたまらない。	こどもたちはふうせんがほしくてたまらない 
\\	一生懸命指でまぶたを広げて目薬を差しました。	いっしょうけんめいゆびでまぶたをひろげてめぐすりをさしました 
\\	彼がベスに言ったことは、まさに彼女に対する侮辱だ。	かれがべスにいったことは、まさにかのじょにたいするぶじょくだ 
\\	爆弾が爆発したのは、まさにその瞬間であった。	ばくだんがばくはつしたのは、まさにそのしゅんかんであった 
\\	彼の言ったことは脅しにほかならなかった。	かれのいったことはおどしにほかならなかった 
\\	世界の歴史は自由意識の進歩に他ならない。	せかいのれきしはじゆういしきのしんぽにほかならない 
\\	その脱出は全く奇跡に他ならなかった。	そのだっしゅつはまったくきせきにほかならなかった 
\\	私はほかならぬ大統領に会った。	わたしはほかならぬだいとうりょうにあった 
\\	最悪の友と敵は死にほかならぬ。	さいあくのともとてきはしにほかならぬ 
\\	その会合で最初の演説をしたのはほかならぬかの有名なクリント=イーストウッドだった。	そのかいごうでさいしょのえんぜつをしたのはほかならぬかのゆうめいなクリント=イーストウッドだった。 
\\	彼の研究は広範囲にわたっている。	かれのけんきゅうはこうはんいにわたっている 
\\	その講演の内容は多岐にわたっていた。	そのこうえんのないようはたきにわたっていた 
\\	女性の立場は多年にわたって確実に好転している。	じょせいのたちばはたねんにわたってかくじつにこうてんしている 
\\	ここ数年にわたって彼らの製品の質は落ちてきた。	ここすうねんにわたってかれらのせいひんのしつはおちてきた 
\\	その地震で広範囲に及ぶ被害がでた。	そのじしんでこうはんいにおよぶひがいがでた 
\\	話は多岐にわたった。	はなしはたきにわたった 
\\	長期にわたる深夜勤務がたたって彼は健康をひどく害してしまった。	ちょうきにわたるしんやきんむがたたてかれはけんこうをひどくがいしてしまった 
\\	彼は深夜のテレビショッピングに騙されて高い買い物をした。	かれはしんやのテレビショッピングにだまされてたかいかいものをした 
\\	勤務時間内で煙草を吸ってはいけない。	きんむじかんないでたばこをすってはいけない 
\\	政府での私の職歴は数多くの海外勤務を含む。	せいふでのわたしのしょくれきはかずおおくのかいがいきんむをふくむ 
\\	バスの中に空席がなかったので、私は立ちっぱなしだった。	バスのなかにくうせきがなかったので、わたしはたちっぱなしだった 
\\	どこかのお利口さんが一晩中ミルクを冷蔵庫から出しっぱなしにしておいたな。	どこかのおりこうさんがひとばんじゅうミルクをれいぞくからだしっぱなしにしておいたな 
\\	彼は本を家のあちこちに散らかしっぱなしにした。	かれはほんをいえのあちこちにちらかしっぱなしにした 
\\	今日ごみを再利用する大きな運動が見られる。	きょうごみをさいりようするおおきいなうんどうがみられる 
\\	下線部を訳せ。	かせんぶをやくせ 
\\	長期の欠席について彼に弁明を求めた。	ちょうきのけっせきについてかれにべんめいをもとめた 
\\	君は自分の不始末をどう弁明するのか。	きみはじぶんのふしまつをどうべんめいするのか 
\\	彼は部下の不始末を見つけた。	かれはぶかのふしまつをみつけた 
\\	君はベルリンに長期に滞在するつもりですか。	きみはベルリンにちょうきにたいざいするつもりですか 
\\	彼は天文学、すなわち星の研究をしている。	かれはてんもんがく、すなわちほしのけんきゅうをしている 
\\	田森は、1945年すなわち第二次世界大戦が終わった年に生まれた。	たもりは、1945ねんすなわちだいにじせかいたいせんがおわったねんにうまれた 
\\	来週の今日、すなわち7月21日に君に返済いたします。	らいしゅうのきょう、すなわち7げつ21にちにきみにへんさいいたします 
\\	三角形の面積を求める。	さんかくけいのめんせきをもとめる 
\\	日本は4つの大きな島と、3、000以上の小さな島からなり、面積はカリフォルニアと略同じです。	にほんはよつのおおきなしまと、さんぜんいじょうのちいさなしまからなり、めんせきはカリフォルニアとほぼおなじです 
\\	縦横8フィートの部屋の面積は64平方フィートである。	たてよこはちフィートのへやのめんせきはろくじゅうよんへいほうフィートである 
\\	工場の面積は1000平方メートルだ。	こうじょうのめんせきはせんへいほうメートルだ 
\\	非常に残念なことに地球は一秒で1900平方メートルが砂漠化している。	ひじょうのざんねんなことにちきゅうはいちびょうでせんきゅうひゃくへいほうメートルがさばくかしている 
\\	そんな汚らわしい話をするのはやめて。	そんなけがらわしいはなしをするのはやめて 
\\	やけくそになった男はしばしばやけくそな事をする。	やけくそになったおとこはしばしばやけくそなことをする 
\\	薬の効き目を調べてみます。	くすりのききめをしらべてみます 
\\	あなたの忠告は彼らに何の効き目もないだろう。	あなたのちゅうこくはかれらになんのききめもないだろう 
\\	今は原子力時代だといっても過言ではない。	いまはげんしりょくじだいだといってもかごんではない 
\\	彼は一流の作家であると言っても過言ではない。	かれはいちりゅうのさっかであるといってもかごんではない 
\\	健康はあらゆる富に勝ると言っても過言ではない。	けんこうはあらゆるとみにまさるといってもかごんではない 
\\	一度失われた時間は決して取り戻せないと言っても過言ではない。	いちどうしなわれたじかんはけっしてとりもどせないといってもかごんではない 
\\	彼女は日本で屈指の優れたテニスの選手であると言っても過言ではない。	かのじょはにほんでくっしのすぐれたテニスのせんしゅであるとうってもかごんではない 
\\	これらの語は同じ語源から出ている。	これらのごはおなじごげんからでている 
\\	以下の空欄部分にご記入頂くだけで結構です。	いかのくうらんぶぶんにごきにゅういただくだけでけっこうです 
\\	必ず、事前に全ての空欄部分を記入しておいて下さい。	かならず、じぜんにすべてのくうらんぶぶんをきにゅうしておいてください 
\\	呼吸をする以外、なにをするにも、事前に妻の許可を必要とする人は女房の尻にしかれている夫だ。	こきゅうをするいがい、なにをするにも、じぜんにつまのきょかをひつようとするひとはにょうぼうのしりにしかれているおっとだ 
\\	その通知はひどい印刷だった。	そのつうちはひどいいんさつだった 
\\	採用と決定したらご通知いたします。	さいようとけっていしたらごつうちいたします 
\\	次の会議の通知がドアに掲示されていた。	【つうち】 告げ知らせること。また、その知らせ。
\\	労働者は事前通知なしに解雇されることはない。	ろうどうしゃはじぜんつうちなしにかいこされることはない 
\\	先ず第一に彼を解雇しなければならない。	まずだいいちにかれをかいこしなければならない 
\\	その店員は無作法が理由で解雇された。	そのてんいんはぶさほうがりゆうでかいこされた 
\\	短期契約社員達は予告なしに解雇された。	たんきけいやくしゃいんたちはよこくなしにかいこされた 
\\	私は予告なしに話すように言われた。	わたしはよこくなしにはなすようにいわれた 
\\	印刷ミスはすぐに指摘されなければならない。	いんさつミスはすぐにしてきされなければならない 
\\	最初の印刷機はグーテンベルグによって発明された。	さいしょのいんさつきはグーテンベルグによってはつめいされた 
\\	読書をする暇がないほど多忙な人はいない。	どくしょくぉするひまがないほどたぼうなひとはいない 
\\	彼は多忙を極めていたので、自分で行かないで息子を行かせた。	かれは多忙を極めていたので、じぶんでいかないでむすこをいかせた 
\\	彼は新素材の開発に従事している。	かれはしんざいのかいはつにじゅうじしている 
\\	この素材は決してすり減ったりしない。	このそざいはけっしてすりへったりしない 
\\	いずれにせよ、私は義務を果たした。	いずれにせよ、わたしはぎむをはたした 
\\	いずれにせよ、君は好むと好まないにかかわらず早く出発する必要がある。	いずれにせよ、きみはこのむとこのまないにかかわらずはやくしゅっぱつするひつようがある 
\\	お目にかかれて嬉しい。	おめにかかれてうれしい 
\\	最近は異常気象がよくある。	さいきんはいじょうきしょうがよくある 
\\	気象庁は今晩雨が降ると言っている。	きしょうちょうはこんばんあえがふるといっている 
\\	彼は官庁で相当な職についている。	かれはかんちょうでそうとうなしょくについている 
\\	東京都庁ビルの高さはどのくらいありますか。	とうきょうとちょうビルのたかさはどのくらいありますか 
\\	彼はこの春で30年教員生活を続けたことになる。	かれはこのはるさんじゅうねんきょういんせいかつをつづけたことになる 
\\	金があれば借金を返すのだが、現状では払えない。	かねがあればしゃっきんをかえすのだが、げんじょうではらえない 
\\	現状では、私たちは降参するしかない。	げんじょうでは、わたしたちはこうさんするしかない 
\\	彼は現状を大いに嘆いた。	かれはげんじょうをおおいになげいた 
\\	「財布を無くした」ジョンは嘆いた。	"「さいふをなくした」ジョンはなげいた 
\\	彼は未婚だけど子供がいた。	かれはみこんだけどこどもがいた 
\\	お年寄りに席を譲ることは確かに親切です。	おとしよりにせきをゆずることはだれかにしんせつです 
\\	電車の中で席を譲るのって、照れくさいよね。	でんしゃのなかでせきをゆずるのって、てれくさいよね 
\\	彼は事業を息子に譲った。	かれはじぎょうをむすこにゆずった 
\\	その会社は新しい人に譲られた。	そのかいしゃはあたらしひとにゆずられた 
\\	エミは体の不自由な人に席を譲った。	エミはからだのふじゆうなひとにせきをゆずった 
\\	歴史は、古い思想が新しい思想に道を譲りながら進行する。	れきしは、ふるいしそうがあたらしいしそうにみちをゆずりながらしんこうする 
\\	その2台の車は互いに道を譲ろうとしたのだ。	そのにだいのくるまはたがいにみちをゆずろうとしたのだ 
\\	父は若い人たちに道を譲って退職した。	ちちはわかいひとたちにみちをゆずってたいしょくした 
\\	一列に立ち並んだ家が新しいアパートに席を譲るために取り壊されつつある。	いちれつにたちならんだいえがあたらしいアパートにせきをゆずるためにとりこわされつつある 
\\	彼はその種の問題を解決する名人だ。	かれはそのしゅのもんだいをかいけつするめいじんだ 
\\	彼は医師であるばかりでなくピアノの名人だった。	かれはいしであるばかりでなくピアノのめいじんだった 
\\	彼はその反戦デモに参加した。	かれはそのはんせんデモにさんかした 
\\	彼を彼の弟と間違えっこない。	かれをかれのおとうととまちがえっこない 
\\	将来何が起こるかなんて、誰にもわかりっこない。	動詞の連用形に付いて、…はずがない、…わけがない、の意を表す。
\\	彼ならいざ知らず、私ではその試験には合格できっこない。	かれならいざしらず、わたしではそのしけんにはごうかくできっこない 
\\	昔ならいざ知らず、今はFAXも、メールもある。	…についてはよくわからないが。…はともかくとして。
\\	いざというとき決意がくじけた。	いざというときけついがくじけた 
\\	いざという時役に立つ。	いざというときやくにたつ 
\\	いざというときのために貯金する。	いざというときのためにちょきんする 
\\	いざとなったら、傘が武器の代用になる。	いざとなったら、かさがぶきのだいようになる 
\\	バターの代用品としてマーガリンを使います。	バターのだいようひんとしてマーガリンをつかいます 
\\	計画はまだ流動的である。	けいかくはまだりゅうどうてきである 
\\	流動食を取ってください。	りゅうどうしょくをとってください 
\\	特に言語は最も流動的な媒体である。	とくにげんごはもっともりゅうどうてきなばいたいである 
\\	空気は音の媒体だ。	くうきはおとのばいたいだ 
\\	テレビは情報を与えるための非常に重要な媒体である。	テレビはじょうほうをあたえるためのひじょうにじゅうようなばいたいである 
\\	不純な飲料水は病気の媒体となりうる。	ふじゅんないんりょうすいはびょうきのばいたいとなりうる 
\\	彼は、たった一回の失敗でくじけるような男ではない。	かれは、たったいっかいのしっぱいでくじけるようなおとこ ではない 
\\	現役から引退した人は足が遠のくものだ。	げんえきからいんたいしたひとはあしがとおのくものだ 
\\	先生はチラシを配布した。	せんせいはチラシをはいふした 
\\	その建物は老朽の兆しをみせている。	そのたてものはろうきゅうのきざしをみせている 
\\	叔父は膨大な財産を所持している。	おじはぼうだいなざいさんをしょじしている 
\\	米国全土では毎年膨大な数の若者が大学にはいる。	べいこくぜんどではまいとしぼうだいなかずのわかものがだいがくにはいる 
\\	辞書を編さんするには膨大な時間を要する。	じしょをへんさんするにはぼうだいなじかんをようする 
\\	所持金を全部あげよう。	しょじきんをぜんぶあげよう 
\\	彼の所持品は全部あの箱に入っている。	かれのしょじひんはぜんぶあのはこにはいっている 
\\	これがかれの生家だ。	これはかれのせいかだ 
\\	彼女の言う事は妙に聞こえる。	かのじょのいうことはみょうにきこえる 
\\	彼が言った事はそれと反対の趣旨だった。	かれがいったことはそれとはんたいのしゅしだった 
\\	彼は愛していると言う趣旨の手紙を彼女に書いた。	かれはあいしているというしゅしのてがみをかのじょにかいた 
\\	政府は税金を値上げするという趣旨で発表を行った。	せいふはぜいきんをねあげするというしゅしではっぴょうをおこなった 
\\	医者は私の父がまもなくよくなるだろうという趣旨のことを言った。	いしゃはわたしのちちがまもなくよくなるだろうというしゅしのことをいった 
\\	私の申し出に応じられないという趣旨の手紙を彼から受け取った。	わたしのもうしでにおうじられないというしゅしのてがみをかれからうけとった 
\\	学校は授業料の値上げを発表した。	がっこうはじゅぎょうりょうのねあげをはっぴょうした 
\\	君の仕事が軌道に乗ったら、値上げの話をしましょう。	きみのしごとがきどうにのったら、ねあげのはなしをしましょう 
\\	機知は会話に趣を添える。	きちはかいわにおもむきをそえる 
\\	それは、幾分ナンセンスな趣きの楽しさがあった。	それは、いくぶんナンセンスなおもむきのたのしさがあった 
\\	散歩して頭をすっきりさせてくるよ。	さんぽしてあたまをすっきりさせてくるよ 
\\	一杯のコーヒーを飲んだら頭がすっきりした。	わだかまりがなく、気持ちのよいさま。また、余計なものがないさま。さっぱり。
\\	我々の先祖は星の読み方を知っていた。	われわれのせんぞはほしのよみかたをしっていた 
\\	日本人は、お盆の間先祖が自分達のところにやってきていると信じている。	にほんじんは、おぼんのかんせんぞがじぶんたちのところにやってきているとしんじている 
\\	お盆期間中は駅はとても混雑する。	おぼんきかんちゅうはえきはとてもこんざつする 
\\	花婿が突然笑った。	はなむこがとつぜんわらった 
\\	スミス氏は娘を医者に嫁がせた。	スミスしはむすめをいしゃにとつがせた 
\\	花嫁は皆の視線を浴びながら部屋に入ってきた。	はなよめはみんなのしせんをあびながらへやにはいってきた 
\\	彼が花婿の父親です。	かれがはなむこのちちおやです 
\\	私はだれかの視線を感じた。	わたしはだれかのしせんをかんじた 
\\	握手をする時には、視線を合わすべきだ。	あくしゅをするときには、しせんをあわすべきだ 
\\	あなたがたの期待に添えるように努力します。	あなたがたのきたいにそえるようにどりょくします 
\\	私は姑と仲良く暮らしています。	わたしはしゅうとめとなかよくくらしています 
\\	古今東西、親の子供に対する愛情に変わりはない。	ここんとうざい、おやのこどmにたいするあいじょうにかわりはない 
\\	古今東西、嫁と姑の仲は上手くいかぬ例が多いと見える。	ここんとうざい、よめとしゅうとめのなかはうまくいかぬれいがおおいとみえる 
\\	彼女は旦那さんがお舅さんとうまくやっていけるかが心配だ。	かのじょはだんなさんがおしゅうとさんとうまくやっていけるかがしんぱいだ 
\\	旦那も子供の引き取り権を望んでいた。	だんなもこどものひきとりけんをのぞんでいた 
\\	旦那様をしっかり捕まえていなさい。	だんなさまをしっかりつかまえていなさい 
\\	彼女のスカートは明るい色合いの緑だった。	かのじょのスカートはあかるいいろあいのみどりだった 
\\	食物繊維はダイエットに効果的だ。	たべものせんいはダイエットにこうかてきだ 
\\	繊維産業をとりまく状況は変化した。	せんいさんぎょうをとりまくじょうきょうはへんかした 
\\	全国的に好景気に見舞われている。	ぜんこくてきにこうけいきにみまわれている 
\\	今年その地方は厳しい干ばつに見舞われた。	ことしそのちほうはきびしいかんばつにみまわれた 
\\	彼は私の申請を却下した。	かれはわたしのしんせいをきゃっかした 
\\	彼女には離婚申請をする十分な根拠があった。	かのじょにはりこんしんせいをするじゅうぶんなこんきょがあった 
\\	残念ながらあなたの申請は却下されたことをお伝えします。	ざんねんながらあなたのしんせいはきゃっかされたことをつたえします 
\\	請願が却下されたのは可笑しいと思った。	せいがんがきゃっかされたのはおかしいとおもった 
\\	私は市長に請願を出した。	わたしはしちょうにせいがんをだした 
\\	会社へ特許権を売る。	かいしゃへとっきょけんをうる 
\\	インテル社はその発明で膨大な特許料を得ている。	インテルしゃはそのはつめいでぼうだいなとっきょりょうをえている 
\\	部活の勧誘にももう慣れた。	ぶかつのかんゆうにももうなれた 
\\	彼は私たちに投票を勧誘した。	かれはわたしたちにとうひょうをかんゆうした 
\\	停電が場内の混乱の一因となった。	ていでんがじょうないのこんらんのいちいんとなった 
\\	大変な努力が彼の成功の一因であった。	たいへんなどりょくがかれのせいこうのいちいんであった 
\\	ろうそくを消して下さい。停電は終わりました。	ろうそくをけしてください。ていでんはおわりました 
\\	彼らは遅くまで首脳会談を続けた。	かれらはおそくまでしゅのうかいだんをつづけた 
\\	首脳会談は世界中で同時に放送される予定だ。	しゅのうかいだんはせかいじゅうでどうじにほうそうされるよていだ 
\\	十人に一人は近眼である。	じゅうにんにひとりはきんがんである 
\\	ってゆーか、こうやってぐじぐじ考えるのが情けないんじゃないのか?	ってゆーか、こうやってぐじぐじかんがえるのがなさけないんじゃないのか? 
\\	傷が化膿しました。	きずがかのうしました 
\\	彼は倒れた木の下敷きになって動けなかった。	かれはたおれたこのしたじきになってうごけなかった 
\\	地面は一面に落ち葉が敷き詰められたようだった。	じめんはいちめんにおちばがしきつめられたようだった 
\\	目が部屋の明かりを反射したときに彼女の目は輝いた。	めがへやのあかりをはんしゃしたときにかのじょのめはかがやいた 
\\	時間をかけて反射しろ。	じかんをかけてはんしゃしろ 
\\	学校の敷地はこの垣根まで続いている。	がっこうのしきちはこのかきねまでつづいている 
\\	私は垣根に沿った小道を歩いた。	わたしはかきねにそったこみちをあるいた 
\\	彼はその垣根を飛び越えた。	かれはそのかきねをとびこえた 
\\	その敷地は軍事上の目的で利用されている。	そのしきちはぐんじじょうのもくてきでりようされている 
\\	彼らは石を並べて家の敷地の境界にした。	かれらはいしをならべていえのしきちのきょうかいにした 
\\	ライン川はフランスとドイツの境界線である。	ラインがわはフランスとドイツのきょうかいせんである 
\\	その老人は生きる意欲をなくした。	そのろうじんはいきるいよくをなくした 
\\	しかし、そういうエリート的な女性だけが、就労意欲を持っているわけではない。	しかし、そういうエリートてきなじょせいだけが、しゅうろういよくをもっているわけではない 
\\	我々は就労時間について社長と交渉した。	われわれはしゅうろうじかんについてしゃちょうとこうしょうした 
\\	ストライキとは従業員の集団が一体となって就労を拒否することです。	ストライキとはじゅうぎょういんのしゅうだんがいったいとなってしゅうろうをきょひすることです 
\\	彼女は30歳過ぎだと推定する。	かのじょはさんじゅうさいすぎだとすいていする 
\\	彼の死は事故だとする君の推定は間違っているようだ。	かれのしはじこだとするきみのすいていはまちがっているようだ 
\\	その習慣を彼は植民地時代から始まると推定している。	そのしゅうかんをかれはしょくみんちじだいからはじまるとすいていしている 
\\	毎年、世界各地に異常な天気が起きています。	まいとし、せかいかくちにひじょうなてんきがおきています 
\\	私はイギリス各地を見物した。	わたしはイギリスかくちをけんぶつした 
\\	日本の戦後の復興は有名な話である。	にほんのせんごのふっこうはゆうめいなはなしである 
\\	それらはさらに一層の活力で復興した。	それらはさらにいっそうのかつりょくでふっこうした 
\\	船長は疲れている乗組員に新たな活力を吹き込んだ。	せんちょうはつかれているのりくみいんにあらたなかつりょくをふきこんだ 
\\	その新人はチームに新たな生気を吹き込んだ。	そのしんじんはチームにあらたなせいきをふきこんだ 
\\	列車は加速した。	れっしゃはかそくした 
\\	凍った斜面を滑り降りる時そりは加速した。	こおったしゃめんをすべりおりるときそりはかそくした 
\\	その政策は、インフレを加速させるだけだ。	そのせいさくは、インフレをかそくさせるだけだ 
\\	お前を告訴するぞ。	おまえをこくそするぞ 
\\	若さゆえに彼に対する告訴は取り下げられた。	わかさゆえにかれにたいするこくそをとりさげられた 
\\	驚いたことに、その人類学者は殺人罪で告訴された。	おどろいたことに、そのじんるいがくしゃはさつじんざいでこくそされた 
\\	夜空に光線が見えた。	よぞらにこうせんがみえた 
\\	光線はプリズムによって七色に分解される。	こうせんはプリズムによってなないろにぶんかいされる 
\\	ラジオを修理するために分解した。	ラジオをしゅうりするためにぶんかいした 
\\	この液体は3つの要素に分解できる。	このえきたいはみっつのようそにぶんかいできる 
\\	その本を秋田の書店で購入できますか。	そのほんをあきたのしょてんでこうにゅうできますか 
\\	その家を徹底的に調べてから購入した。	そのいえをてっていてきにしらべてからこうにゅうした 
\\	私は彼に土地の購入についてよい助言を与えた。	わたしはかれにとちのこうにゅうについてよいじょげんをあたえた 
\\	そのお金は学校図書館の本を購入するための特別な基金に入れられた。	そのおかねはがっこうとしょかんのほんをこうにゅうするためのとくべつなききんにいれられた 
\\	彼は難民救済基金に1万ドルを寄付した。	かれはなんみんきゅうさいききんにいちまんドルをきふした 
\\	絶滅の危機に瀕した海洋生物を保護する為に募金が設立された。	ぜつめつのききにひんしたかいようせいぶつをほごするためにききんがせつりつされた 
\\	1939年には、1914年と同様、世界は戦争の危機に瀕していた。	せんきゅうひゃくきゅうねんには、せんきゅうひゃくじゅうよねんとどうよう、せかいはせんそうのききにひんしていた 
\\	多くの種類の昆虫が絶滅の危機に瀕している。	おおくのしゅるいのこんちゅうがぜつめつのききにひんしている 
\\	この地域に暮らしている人は水不足のため死に瀕している。	このちいきにくらしているひとはみずぶそくのためしにひんしている 
\\	最近変わった海洋生物が発見された。	さいきんかわったかいようせいぶつがはっけんされた 
\\	一般に若者は形式を嫌う。	いっぱんにわかものはけいしきをきらう 
\\	彼はそれを小説の形式で表した。	かれはそれをしょうせつのけいしきであらわした 
\\	此処で使われている文法形式は未来進行形である。	ここでつかわれているぶんぽうけいしきはみらいしんこうけいである 
\\	映像はピントがあっていない。	えいぞうはピントがあっていない 
\\	母親の死は少女にとって打撃であった。	ははおやのしはしょうじょにとってだげきであった 
\\	円高はその会社にとって致命的な打撃だった。	えんだかはそのかいしゃにとってちめいてきなだげきだった 
\\	それは彼らの計画にとって壊滅的な打撃となった。	それはかれのけいかくにとってかいめつてきなだげきとなった 
\\	彼女はテニス選手としては致命的な事故に会った。	かのじょはテニスせんしゅとしてはちめいてきなじこにあった 
\\	致命的な誤りは不注意から起こる。	ちめいてきなあやまりはふちゅういからおこる 
\\	王は敵軍を壊滅させた。	おうはてきぐんをかいめつされた 
\\	汚染公害は地域の生態環境に壊滅的な影響を与える。	おせんこうがいはちいきのせいたいかんきょうにかいめつてきなえいきょうをあたえる 
\\	砂漠の生態学は新しい学問分野である。	さばくのせいたいがくはあたらしいがくもんぶんやである 
\\	石油は天の恵みであるだけでなく災いのもとでもある、とその生態学者は私たちに警告した。	せきゆはてんのめぐみであるだけでなくわざわいのもとでもある、とそのせいたいがくはわたしたちにけいこくした 
\\	卵の白身はゆでれば固まります。	たまごのしろみはゆでればかたまります 
\\	私の娘は卵の黄身が好きです。	わたしのむすめはたまごのきみがすきです 
\\	彼は苦い経験を味わった。	かれはにがいけいけんをあじわった 
\\	私たちはあまりにも速く走ったので美しい景色を味わう事も出来ないくらいだった。	わたしたちはあまりにもはやくはしったのでうつくしいけしきをあじわうこともできないくらいだった 
\\	人間は無限の潜在能力を持っている。	にんげんはむげんのせんざいのうりょくをもっている 
\\	教育は潜在する能力の開発を目指すものだ。	きょういくはせんざいするのうりょくのかいはつをめざすものだ 
\\	その潜在的な影響力は幾ら評価してもし過ぎることはない。	そのせんざいてきなえいきょうりょくはいくらひょうかしてもしすぎることはない 
\\	私たちの話す言葉は潜在的に曖昧である。	わたしたちのはなすことばはせんざいてきにあいまいである 
\\	石油の供給は無限ではない。	せきゆのきょうきゅうはむげんではない 
\\	熱はたいていの物を膨張させる。	ねつはたいていのものをぼうちょうさせる 
\\	宇宙は無限に膨張し続けるのか?	うちゅうはむげんにぼうちょうしつづけるのか? 
\\	多くの天文学者は、宇宙は永遠に膨張し続けると思っている。	おおくのてんもんがくしゃはうちゅうはえいえんにぼうちょうしつづけるとおもっている 
\\	悪は行うに易しくて、その形体も無限である。	あくはおこなうにやさしくて、そのけいたいもむげんである 
\\	あなたは芸術家としての無限の可能性を秘めている。	あなたはげいじゅつかとしてのむげんのかのうせいをひめている 
\\	彼女は悲しみを胸に秘めていた。	かのじょはかなしみをむねにひめていた 
\\	民主主義は政治形態の一つである。	みんしゅしゅぎはせいじけいたいのひとつである 
\\	何といっても、彼らの輸送形態は公害を全く引き起こさない。	なんといっても、かれらのゆそうけいたいはこうがいをまったくひきおこさない 
\\	彼の発見は輸送における革命を齎した。	かれのはつめいはゆそうにおけるかくめいをもたらした 
\\	今では多くの貨物が飛行機で輸送されている。	いまではおおくのかもつがひこうきでゆそうされている 
\\	公共の交通輸送機関は正確に動いています。	こうきょうのこうつうゆそうきかんはせいかくにうごいています 
\\	その航空会社は貨物のみを扱っている。	そのこうくうかいしゃはかもつのみをあつかっている 
\\	その貨物にはあらゆる危険に対する保険が掛けられた。	そのかもつにはあらゆるきけんにたいするほけんがかけられた 
\\	囚人は拷問で死んだ。	しゅうじんはごうもんでしんだ 
\\	拷問にかけられて彼は犯していない罪を認めた。	ごうもんにかけられてかれはおかしていないつみをみとめた 
\\	彼は提案に対するいかなる反論もただ排除した。	かれはていあんにたいするいかなるはんろんもただはいじょした 
\\	私たちはこれらの古い決まり事を排除しなければならない。	わたしたちはこれらのふるいきまりことをはいじょしなければならない 
\\	議長は意見が不一致に終る可能性を排除しなかった。	ぎちょうはいけんがふいっちにおわるかのうせいをはいじょしなかった 
\\	首脳たちは経済成長に障害となるものを排除しようとしています。	しゅのうたちはけいざいせいちょうにしょうがいとなるものをはいじょしようとしています 
\\	いかなる状況でも部屋を離れてはならない。	いかなるじょうきょうでもへやをはなれてはならない 
\\	いかなる行為もそれ自体は良くも悪くもない。	いかなるこういもそれじたいはよくもわるくもない 
\\	昔はいかなる王も国民に重税を課して苦しめた。	むかしはいかなるおうもこくみんにじゅうぜいをかしてくるしめた 
\\	彼はいかなる困難にであっても、気を落とすことはない。	かれはいかなるこんなんにであっても、きをおとすことはない 
\\	嘘をついたことが彼の良心を苦しめた。	うそをついたことがかれのりょうしんをくるしめた 
\\	良心を犠牲にして富を得るな。	りょうしんをぎせいにしてとみをえるな 
\\	賢明で良心的な人に助言を求めなさい。	けんめいでりょうしんてきなひとにじょげんをもとめなさい 
\\	彼は私の陳述に反論した。	かれはわたしのちんじゅつにはんろんした 
\\	水平線に漁船がいくつか見えます。	すいへいせんにぎょせんがいくつかみえます 
\\	樹木は酸素を排出し、二酸化炭素を吸収する。	じゅもくはさんそをはいしゅつし、にさんかたんそをきゅうしゅうする 
\\	炭素排出量の変動が、以下のグラフに描かれている。	たんそはいしゅつりょうのへんどうが、いかのグラフにかかれている 
\\	毎日大量の二酸化炭素が生成されている。	まいにちたいりょうのにさんかたんそがせいせいされている 
\\	人口は変動しないでいる。	じんこうはへんどうしないでいる 
\\	消費者物価指数は激しく変動しています。	しょうひしゃぶっかしすうははげしくへんどうしています 
\\	樹木が切り倒され土地が切り開かれている。	じゅもくがきりたおされとちがきりひらかれている 
\\	果実がなる樹木は、成長するための空間がかなり必要だ。	かじつがなるじゅもくは、せいちょうするためのくうかんがかなりひつようだ 
\\	アメリカの平均的な生活空間は日本の二倍広い。	アメリカのへいきんてきなせいかつくうかんはにほんのにばいひろい 
\\	光の波は空間や様々な種類の物質の中を通って進む。	ひかりのなみはくうかんやさまざまなしゅるいのぶっしつのなかをとおってすすむ 
\\	それは君の努力の果実だ。	それはきみのどりょくのかじつだ 
\\	と
\\	を反応させると
\\	が生成されます。	
\\	と
\\	をはんのうさせると
\\	がせいせいされます 
\\	株価指数は過去最高に上昇した。	かぶかしすうはかこさいこうにじょうしょうした 
\\	ココの知能指数を調べるのは容易ではない。	ココのちのうしすうをしらべるのはよういではない 
\\	バンパーが衝撃をいくらか吸収してくれた。	バンパーがしょうげきをいくらかきゅうしゅうしてくれた 
\\	生徒たちは教師の与える知識をすべて吸収した。	せいとたちはきょうしのあたえるちしきをすべてきゅうしゅうした 
\\	彼はできるだけ多くその地方の文化を吸収しようとしていた。	かれはできるだけおおくそのちほうのぶんかをきゅうしゅうしようとしていた 
\\	全世界の人が衝撃を受けた。	ぜんせかいのひとがしょうげきをうけた 
\\	数回の爆発の衝撃は何マイルにもわたって感じられた。	すうかいのばくはつのしょうげきはなんマイルにもわたってかんじられた 
\\	鳶が鷹を生む。	とんびがたかをうむ 
\\	古い方法に拘っても仕方がない。	ふるいほうほうにこだわってもしかたがない 
\\	日本人はあまり宗教には拘らない。	にほんじんはあまりしゅうきょうにはこだわらない 
\\	こちらの殿方にビールを差し上げてください。	こちらのとのがたにビールをさしあげてください 
\\	殿方が細かいことに拘るものではありません。	とのがたがこまかいことにこだわるものではありません 
\\	田中さん!殿方と共同生活してるのよ!ノーパンは慎みなさい!	たなかさん!とのがたときょうどうせいかつしてるのよ!ノーパンはつつしみなさい 
\\	あなたはそのサインが消えるまでタバコを慎むように求められている。	あなたはそのサインがきえるまでタバコをつつしむようにもとめられている 
\\	その人は慎み深いのか、それとも怠惰なのかと人は思うだろう。	そのひとはつつしみぶかいのか、それともたいだなのかとひとはおもうだろう 
\\	彼がおしゃべりを慎もうと言った。	かれがおしゃべりをつつしもうといった 
\\	あなたは問題を大袈裟に考えている。	あなたはもんだいをおおげさにかんがえている 
\\	そんなこと真に受けちゃだめだよ。彼は大袈裟に言う傾向があるから。	そんなことまことにうけちゃだめだよ。かれはおおげさにいうけいこうがあるから 
\\	彼の話は相当大袈裟だ。	かれのはなしはそうとうおおげさだ 
\\	将軍は捕虜全員の虐殺を命じた。	しょうぐんはほりょぜんいんのぎゃくさつをめいじた 
\\	捕虜を虐殺したのは残忍な行為だ。	ほりょをぎゃくさつしたのはざんにんなこういだ 
\\	捕虜は釈放された。	ほりょはしゃくほうされた 
\\	その囚人は昨日釈放された。	そのしゅうじんはきのうしゃくほうされた 
\\	彼は四年刑期のところを二年で釈放された。	かれはよねんけいきのところをにねんでしゃくほうされた 
\\	証拠不十分のため被告は釈放された。	しょうこふじゅうぶんのためひこくはしゃくほうされた 
\\	何故その政治家は大多数の意見を抹殺しようとするのか。	なぜそのせいじかはだいたすうのいけんをまっさつしようとするのか 
\\	その結果、5月下旬に自閉症と診断された。	さおのけっか、ごがつげじゅんにじへいしょうとしんだんされた 
\\	7月の上旬は、海に行くのは早いかな?	なながつのじょうじゅんは、うみにいくのははやいかな? 
\\	予定日は1月中旬頃です。	よていびはいちがつちゅうじゅんごろです 
\\	その山は海抜2千メートルだ。	そのやまはかいばつにせんメートルだ 
\\	クスコは海抜3500メーターにあり、そこで多くのインカ族に会うことができる。	クスコはかいばつさんぜんごひゃくメーターにあり、そこでおおくのインカぞくにあうことができる 
\\	海抜の低い土地は水浸しになるだろう。	かいばつのひくいとちはみずびたしになるだろう 
\\	洪水で水浸しになった街路と家々。	こうずいでみずびたしになったがいろといえいえ 
\\	彼はびんに水を浸した。	かれはびんにみずをひたした 
\\	彼女は足首をお湯に浸した。	かのじょはあしくびをおゆにひたした 
\\	首相と閣僚が辞任した。	しゅしょうとかくりょうがじにんした 
\\	彼はあの時閣僚になり損ねた。	かれはあのときかくりょうになりそこねた 
\\	靖国神社参拝に関しては閣僚の自主的な判断に任せられている。	やすくにじんじゃさんぱいにかんしてはかくりょうのじしゅてきなはんだんにまかせられている 
\\	彼に機嫌を損ねられて実に遺憾だった。	かれにきげんをそこねられてじつにいかんだった 
\\	その大きな醜い木がその家の美観を損ねている。	そのおおきなみにくいきがそのいえのびかんをそこねている 
\\	その美観は筆では書き表せない。	そのびかんはふでではかきあらわせない 
\\	元日に神社へ参拝する日本人は多い。	がんじつにじんじゃへさんぱいするにほんじんはおおい 
\\	山火事の後なので今年の花火大会は自主的に中止を決定しました。	やまかじのあとなのでことしのはなびたいかいはじしゅてきにちゅうしをけっていしました 
\\	その時閣議は殆ど終わっていた。	そのときかくぎはほとんどおわっていた 
\\	町の人たちは密輸業者がしようと試みていたことに驚くほど無知だった。	まちのひとたちはみつゆぎょうしゃがしようとこころみていたことにおどろくほどむちだった 
\\	全世界で密輸されているコカインの90%が海路で運ばれている。	ぜんせかいでみつゆされているコカインのきゅうじゅうパーセントがかいろで運ばれている 
\\	罠かもしれん、油断するな。	わなかもしれん、ゆだんするな 
\\	「罠に気をつけてっ!」と彼女は高い声で叫んだ。	"「わなにきをつけてっ!」とかのじょはたかいこえでさけんだ 
\\	毛皮のコートのために動物を罠で捕獲するのは残酷なことだ。	けがわのコートのためにどうぶつをわなでほかくするのはざんこくなことだ 
\\	私たちは、昨日、森に行き、2頭の鹿を捕獲した。	わたしたちは、きのう、もりにいき、にとうのしかをほかくした 
\\	彼らは銃と交換に毛皮を手に入れた。	かれらはじゅうとこうかんにけがわをてにいれた 
\\	逃亡犯人を捕まえようと油断なく見張った。	とうぼうはんにんをつかまえようとゆだんなくみはった 
\\	油断大敵。	ゆだんたいてき 
\\	休日には歴史書か古典を読んで時を過ごしたものだ。	きゅうじつにはれきししょかこてんをよんでときをすごしたものだ 
\\	もっと古典的な顔立ちなのかと思いきや、今の時代でも充分通用する美形です。	もっとこてんてきなかおだちなのかとおもいきや、いまのじだいでもじゅうぶんつうようするびけいです 
\\	あっさり断られると思いきや、彼女は承諾してくれました。	あっさりことわられるとおもいきや、かのじょはしょうだくしてくれました 
\\	彼はあっさり罪を白状した。	かれはあっさりつみをはくじょうした 
\\	彼女はあっさり自分の間違いを認める必要はなかったのに。	人の性質や事物の状態などがしつこくないさま。複雑でないさま。さっぱり。
\\	彼女は20歳といっても通用する。	かのじょはにじゅうさいといってもつうようする 
\\	成り上がり者は教養人として通用しようとした。	なりあがりものはきょうようひととしてつうようしようとした 
\\	教養の点では彼らは野蛮人と同じ程度だ。	きょうようのてんではかれらはやばんじんとおなじていどだ 
\\	彼らは教養を身につけるために息子をヨーロッパへやった。	かれらはきょうようをみにつけるためにむすこをヨーロッパへやった 
\\	教養のある人によくあることだが、彼はジャズより古典音楽が好きだ。	きょうようのあるひとによくことだが、かれはジャズよりこてんおんがくがすきだ 
\\	指を使うことは野蛮なのだろうか。	ゆびをつかうことはやばんなのだろうか 
\\	人間の野蛮性は決して根絶できない。	にんげんのやばんせいはけっしてこんぜつできない 
\\	野蛮な男がわたしの高価な宝石を奪って逃走した。	やばんなおとこがわたしのこうかなほうせきをうばってとうそうした 
\\	彼は森のほうへ逃走した。	かれはもりのほうへとうそうした 
\\	盗賊は彼を縛り上げて窓から逃走した。	とうぞくはかれをしばりあげてまどかれとうそうした 
\\	我々の最初の攻撃で敵は逃走した。	われわれのさいしょのこうげきでてきはとうそうした 
\\	警察は飲酒運転を根絶やしにしようとしている。	けいさつはあいんしゅうんてんをこんぜつやしにしようとしている 
\\	私の飲んだスープは、熱くて飲めやしなかった。	わたしののんだスープは、あつくてのめやしなかった 
\\	これじゃあ、僕は一生結婚なんて出来やしないや。	これじゃあ、ぼくはいっしょうけっこんなんてできやしないや 
\\	助けを求めて叫んでも無駄だぞ。誰にも聞こえやしない。	たすけをもとめてさけんでもまだだぞ。だれにもきこえやしない 
\\	どうやってあの契約を成立させたんですか。	どうやってあのけいやくをせいりつさせたんですか 
\\	彼は合併成立後に新会社を設立したいと思っていた。	かれはがっぺいせいりつごにしんかいしゃをせつりつしたいとおもっていた 
\\	東京銀行と三菱銀行が合併した。	とうきょうぎんこうとみつびしぎんこうががっぺいした 
\\	その二つの国が合併する見込みはない。	そのふたつのくにががっぺいするみこみはない 
\\	両社の合併の知らせがきのう突然伝わった。	りょうしゃのがっぺいのしらせがきのうとつぜんつたわった 
\\	くつろいで相手を気楽にさせてあげなさい。	【きらく】 心配や苦労がなく、のんびりとしていられること。
\\	椅子に深く座ってくつろいだ途端に、電話が鳴った。	いすにふかくすわってくつろいだとたんに、でんわがなった 
\\	私たちは三度試みたが、いずれも失敗した。	わたしはさんどこころみたが、いずれもしっぱいした 
\\	騙されやすい人は絶えず生れてくるといった人があるが、詐欺師は、いずれも、このことを承知している。	"だまされやすいひとはたえずうまれてくるといったひとがあるが、さぎしは、いずれも、このことをしょうちしている 
\\	彼らは顧客との関係を向上させようと絶えず努力している。	かれらはこきゃくとのかんけいをこうじょうさせようとたえずどりょくしている 
\\	彼は絶えず他人の悪口ばかり言っている。	かれはたえずたにんのわるぐちばかりいっている 
\\	男は帽子、眼鏡、仮面を脱ぎ始めた。	おとこはぼうし、めがね、かめんをぬぎはじめた 
\\	彼は正体を見破られないように仮面をつけていた。	かれはしょうたいをみやぶられないようにかめんをつけていた 
\\	実行する前に彼らの陰謀を見破った。	じっこうするまえにかれらのいんぼうをみやぶった 
\\	その学生の不正行為はすぐに見破られた。	そのがくせいのふせいこういはすぐにみやぶられた 
\\	猫かぶりしてもお前の正体は分かっているよ。	ねこかぶりしてもおまえのしょうたいはわかっているよ 
\\	彼はイギリス人のフリをしていたが、外国訛りからその正体がばれた。	かれはイギリスじんのフリをしていたが、がいこくなまりからそのしょうたいかばれた 
\\	彼はその陰謀の陰の指導者だった。	かれはそのいんぼうのかげのしどうしゃだった 
\\	火事を起こすという彼らの陰謀は警察に発見された。	かじをおこすというかれらのいんぼうはけいさつにはっけんされた 
\\	その過激派が政府打倒の陰謀を企てているという噂が立ってる。	そのかげきはがせいふだとうをいんぼうをくわだてているといううわさがたってる 
\\	彼らは彼を陰謀に巻き込んだ。	かれらはかれをいんぼうにまきこんだ 
\\	赤字を解消するのは大変な難問題です。	あかじをかいしょうするのはたいへんななんもんだいです 
\\	人種対立の解消と国家建設が彼の政府の目的だと語っている。	じんしゅたいりつのかいしょうとこっかけんせつがかれのせいふのもくてきだとかたっている 
\\	カラオケはストレス解消によい。	カラオケはストレスかいしょうによい 
\\	商業テレビは広告の効果的な手段である。	しょうぎょうテレビはこうこくのこうかてきなしゅだんである 
\\	彼は新聞の広告に応募して職を得た。	かれはしんぶんのこうこくにおうぼしてしょくをえた 
\\	非常に沢山の人々がその広告に騙された。	ひじょうにたくさんのひとびとがそのこうこくにだまされた 
\\	三歳未満の子供は入場無料。	さんさいみまんのこどもはにゅうじょうむりょう 
\\	輸入車は8%未満しか占めていない。	ゆにゅうしゃははちパーセントみまんしかしめていない 
\\	彼は見かけほど単純ではない。	かれはみかけほどたんじゅんではない 
\\	君には単純に見えるものでも、ぼくには複雑に思われる。	きみにはたんじゅんにみえるものでも、ぼくにはふくざつにおもわれる 
\\	彼の話は単純そのものであった。	かれのはなしはたんじゅんそのものであった 
\\	彼は奮発して新車を買った。	かれはふんぱつしてしんしゃをかった 
\\	祖母は漢方薬が一番いいと信じている。	そぼはかんぽうやくがいちばんいいとしんじている 
\\	彼らは多くの労働者たちをその時点で一時解雇した。	かれらはおおくのろうどうしゃたちをそのじてんでいちじかいこした 
\\	卵はどのように調理しましょうか。	たまごはどのようにちょうりしましょうか 
\\	この調理法は中国独特のものだ。	このちょうりほうはちゅうごくどくとくのものだ 
\\	口と心は裏腹なことが多々ある。	くちとこころはうらはらなことがたたある 
\\	スウィート博士の性格はその名とは裏腹であった。	スウィートはかせのせいかくはそのめいとはうらはらであった 
\\	一刻の猶予もならない。	いっこくのゆうよもならない 
\\	一刻も早くここを出なければならない。	いっこくもはやくここをでなければならない 
\\	事態は一刻の猶予も許さない。	じたいはいっこくのゆうよもゆるさない 
\\	私は特殊部隊に勤務する。	わたしはとくしゅぶたいにきんむする 
\\	持って生まれたる特殊の才能なき者は幸いなるかな。	もってうまれたるとくしゅのさいのうなきものはさいわいなるかな 
\\	私は特殊効果がすばらしいので「ターミネーター」が好きです。	わたしはとくしゅこうかがすばらしいので「ターミネーター」がすきです 
\\	部隊は陣地を守り続けた。	ぶたいはじんちをまもりつづけた 
\\	兵士たちは敵の陣地へ向かって進んでいた。	へいしたちはてきのじんちへむかってすすんでいた 
\\	苦虫を噛みつぶしたような顔で、綾乃ちゃんは舌打ちした。	にがむしをかみつぶしたようなかおで、あやのちゃんはしたうちした 
\\	彼の死は彼らの生活に空しさを齎した。	かれのしはかれらのせいかつにむなしさをもたらした 
\\	医者の努力も空しく、その男はまもなく死んでしまいました。	いしゃのどりょくもむなしく、そのおとこはまもなくしんでしまいました 
\\	成功しようと奮闘していて、時に人は空しい気持ちになる。	せいこうしようとふんとうしていて、ときにひとはむなしいきもちになる 
\\	彼の気取った話し方がとても滑稽におもえた。	かれのきどったはなしかたがとてもこっけいにおもえた 
\\	滑稽な話をしている真最中に彼に電話がかかってきたので話を中止した。	こっけいなはなしをしているまっさいちゅうにかれにでんわがかかってきたのではなしをちゅうしした 
\\	そのパーティーで彼の振る舞いはあまりに滑稽だったので、私は笑わずにはいられなかった。	そのパーティーでかれのふるまいはあまりにこっけいだったので、わたしはわらわずにはいられなかった 
\\	彼は海軍を退役してみると陸上の生活に適応するのが難しいことがわかった。	かれはかいぐんをたいえきしてみるとりくじょうのせいかつにてきおうするのがむずかしいことがわかった 
\\	彼らはパイロットに海軍のヘリコプターを操縦させた。	かれらはパイロットにかいぐんのヘリコプターをそうじゅうさせた 
\\	未来のパイロットは模擬操縦室で訓練される。	みらいのパイロットはもぎそうじゅうしつでくんれんされる 
\\	クレーンを操縦するのには、勿論、免許が必要です。	クレーンをそうじゅうするのには、もちろん、めんきょがひつようです 
\\	彼は操縦士としての自分の技術を大変誇りにしている。	かれはそうじゅうしとしてのじぶんのぎじゅつをたいへんほこりにしている 
\\	人々は貧困に向かって奮闘した。	ひとびとはひんこんにむかってふんとうした 
\\	奮闘が収まると、話し合いが再び始まった。	ふんとうがおさまると、はなしあいがふたたびはじまった 
\\	1日1個のりんごは医者を遠ざける。	いちにちいっこのりんごはいしゃをとおざける 
\\	彼は政治から遠ざかった。	かれはせいじからとおざかった 
\\	咳には塩水のうがいが効く。	水や薬液などを口に含んで、口やのどをすすぐこと。
\\	これは確か淡水魚だと思います。	これはたしかたんすいぎょだとおもいます 
\\	鯉や鱒のような魚は淡水に住んでいる。	こいやますのようなさかなはたんすいにすんでいる 
\\	彼は富を蓄積しようとした。	【ちくせき】 たくさんたくわえること。また、たまること。たくわえ。
\\	日本の学生は知識を蓄積する事は大変得意だ。	にほんのがくせいはちしきをちくせきすることはたいへんとくいだ 
\\	彼は高校卒業直後に入社した。	【ちょくご】 物事の起こったり行われたりしたすぐあと。
\\	私たちは夕食直後にそれを再開した。	わたしたちはゆうしょくちょくごににそれをさいかいした 
\\	数年前の母の日に、義母にロケットをプレゼントしました。	すうねんまえのははのひに、ぎぼにロケットをプレゼントしました 
\\	高原を散歩するのは楽しい。	こうげんをさんぽするのはたのしい 
\\	生みたての卵が欲しい。	うみたてのたまごがほしい 
\\	その島の経済は漁業に依存している。	そのしまのけいざいはぎょぎょうにいぞんしている 
\\	村人たちは主として漁業に従事している。	【しゅとして】 おもに。もっぱら。物事の重点・大勢を述べるときに用いる。
\\	科学者は主として物質の問題を取り扱う。	かがくしゃはしゅとしてぶっしつのもんだいをとりあつかう 
\\	彼は、主として、宇宙の起源と進化に興味を持っていた。	【きげん】 物事の起こり。始まり。みなもと。
\\	熱帯雨林は地球に多くの恩恵を与える。	【おんけい】 恵み。いつくしみ。
\\	この地方は新しい産業の恩恵を被ることになるだろう。	このちほうはあたらしいざんぎょうのおんけいをかぶることになるだろう 
\\	近頃私たちは自然の恩恵を忘れがちです。	ちかごろわたしたちはしぜんのおんけいをわすれがちです 
\\	言葉による意思伝達では、語彙よりも感情の方が重要です。	ことばによるいしでんたつでは、ごいよりもかんじょうのほうがじゅうようです 
\\	水は空気よりも音をよく伝達する。	みずくうきよりもおとをよくでんたつする 
\\	彼は蜂がお互いに意思伝達をする事が出来ると言う証拠を見つけた。	かれははちがおたがいにいしでんたつをすることができるというしょうこをみつけた 
\\	読み書きが出来ないその男は自分の語彙を増やそうと一生懸命だった。	よみかきができないそのおとこはじぶんのごいをふやそうといっしょうけんめいだった 
\\	彼は私たちが戸惑うと喜ぶ。	かれはわたしたちがとまどうとよろこぶ 
\\	ジョンは自分が褒められるのを聞いた時、ひどく戸惑った。	ジョンはじぶんがほめられるのをきいたとき、ひどくとまどった 
\\	私たちを戸惑わせたのは、彼が会議に出席しないといったことだ。	わたしたちをとまどわせたのは、かれはかいぎにしゅっせきしないといったことだ 
\\	試行錯誤は進歩に不可欠だ。	しこうさくごはしんぽにふかけつだ 
\\	解決策が功を奏したのは試行錯誤の結果だった。	かいけつさくがこうをそうしたのはしこうさくごのけっかだった 
\\	レーガン大統領の税政策はまだ功を奏していない。	レーガンだいとうりょうのぜいせいさくはまだこうをそうしていない 
\\	試行錯誤の末、彼はふと正しい答えを思いついた。	しこうさくごのまつ、かれはふとただしいこたえをおもいついた 
\\	すばらしい考えがふと心に浮かんだ。	すばらしいかんがえがふとこころにうかんだ 
\\	ふと思うのだが、私は君を誤解していたかもしれない。	ふとおもうのだが、わたしはきみをごかいしていたかもしれない 
\\	彼は肌着を裏返しに着た。	かれははだぎをうらがえしにきた 
\\	弾丸は彼の頭を貫通した。	だんがんはかれのあたまをかんつうした 
\\	一本の矢が鷹を貫通した。	いっぽんのやがたかをかんつうした 
\\	寒さがじきに彼のキルトの上着を貫通して体に伝わってきた。	さむさがじきにかれのキルトにうわぎをかんつうしてからだにつたわってきた 
\\	彼は口元をしっかり結び一言もしゃべらなかった。	かれはくちもとをしっかりむすびひとこともしゃべらなかった 
\\	今日の朝刊によればその死刑囚は自殺したようだ。	きょうのちょうかんによればそのしけいしゅうはじさつしたようだ 
\\	彼は夕刊にさっと目を通した。	かれはゆうかんにさっとめをとおした 
\\	いじめによる高校生の自殺が相次いでいました。	うじめによるこうこうせいのじさつがあいついでいました 
\\	あなたの考え方には共感できます。	あなたのかんがえかたにはきょうかんできます 
\\	自分が生きている世界との共感がなければ、作家であることは無益である。	じぶんがいきているせかいとのきょうかんがなければ、さっかであることはむえきである 
\\	私は使い慣れた家具を手放したくない。	わたしはつかいなれたかぐをてばなしたくない 
\\	私は有能な秘書を手放さなくてはならなかった。	わたしはゆうのうなひしょをてばなさなくてはならなかった 
\\	「めがねなくても大丈夫なの?」「あ、これ伊達めがねだから、頭よくなるかなと思って」	"「めがねなくてもだいじょうぶなの?」「あ、これだてめがねだから、あたまよくなるかなとおもって」 
\\	私は先月、運転免許証を更新してもらった。	わたしはせんげつ、うんどうめんきょしょうをこうしんしてもらった 
\\	私達は更新できるエネルギー源を開発しなければならない。	わたしたちはこうしんできるエネルギーげんをかいはつしなければならない 
\\	彼は走り高跳びの世界記録を更新した。	かれははしりたかとびのせかいきろくをこうしんした 
\\	いくつかの地方自治体では開発を禁止した。	いくつかのちほうじちたいではかいはつをきんしした 
\\	地方自治体の時代が来たといわれてから久しい。	ちほうじちたいのじだいがきたといわれてからひさしい 
\\	彼はその情況を把握していた。	【はあく】 しっかりと理解すること。
\\	そのスピーチの要点は把握できた。	そのスピーチのようてんははあくできた 
\\	彼女は担任のクラスをよく把握している。	かのじょはたんにんのクラスをよくはあくしている 
\\	地震では地面は上下、そして横に揺れる。	じしんではじめんはじょうげ、そしてよこにゆれる 
\\	彼女はその絵を上下逆さまにかえた。	かのじょはそのえをじょうげさかさまにかえた 
\\	事故のあと、その車は道に逆さまに横たわっていた。	じこのあと、そのくるまはみちにさかさまによこたわっていた 
\\	その人形は床に横たわっていた。	そのにんぎょうはゆかによこたわっていた 
\\	広大な砂漠が我々の前に横たわっていた。	こうだいなさばくがわれわれのまえによこたわっていた 
\\	正午に彼らは休憩するために森の中で横たわった。	しょうごにかれらはきゅうけいするためにもりのなかでよこたわった 
\\	僕は土曜日は正午に勤務からひける。	ぼくはどようびはしょうごにきんむからひける 
\\	宿の主人は召使い達にがなりたてた。	やどのしゅじんはめしつかいたちにがなりたてた 
\\	彼の召使いでさえも彼を軽蔑した。	かれのめしつかいでさえもかれをけいべつした 
\\	我々はその問題を立体的に調査した。	われわれはそのもんだいをりったいてきにちょうさした 
\\	俺は家族全員に見送られながら、旅支度を整えたザックを担ぐ。	おれはかぞくぜんいんにみおくられながら、たびしたくをととのえたザックをかつぐ 
\\	我々は、日常生活の中に多くの義務や責任を担っている。	われわれは、にちじょうせいかつのなかにおおくのぎむやせきにんをかつっている 
\\	男は袋を肩に担いだ。	おとこはふくろをかたにかついだ 
\\	これらの品物は見計らい品です。	これらのしなものはみはからいひんです 
\\	私は真っ先に母にそれを告げた。	わたしはまっさきにははにそれをつげた 
\\	彼の名前が成績表に真っ先に出ていた。	かれのなまえがせいせきひょうにまっさきにでていた 
\\	次の交差点まで行って左折しなさい。	つぎのこうさてんまでいってさせつしなさい 
\\	右折しなさい、そうすれば私の事務所が見つかります。	うせつしなさい、そうすればわたしのじむしょがみつかります 
\\	私は愚かな迷信など信じない。	わたしはおろかなめいしんなどしんじない 
\\	その迷信は彼らの間で今なお残っている。	そのめいしんはかれらのあいだでいまなおのこっている 
\\	彼は13が不吉な数だという迷信を信じている。	かれはじゅうさんがふきつなかずだというめいしんをしんじている 
\\	ビルのぶっきらぼうな態度は誤解を生む原因になっている。	ビルのぶっきらぼうなたいどはごかいをうむげんいんになっている 
\\	今日の幸せが幾度も巡ってきますように。	きょうのしあわせがいくどもめぐってきますように 
\\	彼は幾度となくためになる助言をしてくれた。	かれはいくどとなくためになるじょげんをしてくれた 
\\	これはシルクの感触だ。	これはシルクのかんしょくだ 
\\	彼は勉強を始めるとどうして眠気がするのだろうか。	かれはべんきょうをはじめるとどうしてねむけがするのだろうか 
\\	パパ、肩車して。	パパ、かたぐるまして 
\\	現代詩はしばしばきわめて難解だ。	げんだいしはしばしばきわめてなんかいだ 
\\	今日は憂鬱な気分だ。	きょうはゆううつなきぶんだ 
\\	彼女は憂鬱状態になることがある。	かのじょはゆううつじょうたいになることがある 
\\	妻に四六時中ぶつぶつ言われるのには憂鬱になってしまう。	つまにしろくじちゅうぶつぶついわれるのびはゆううつになってしまう 
\\	この随筆は一個の旅行かばんについてのものです。	このずいひつはいっこのりょこうかばんについてのものです 
\\	彼女は随筆を書き始めた。	かのじょはずいひつをかきはじめた 
\\	彼女は重大な過ちを後悔した、反省した。	かのじょはじゅうだいなあやまちをこうかいした、はんせいした 
\\	彼は自分自身の考えを反省した。	かれはじぶんじしんのかんがえをはんせいした 
\\	彼のことばの魔力が聴衆を魅了した。	かれのことばのまりょくがちょうしゅうをみりょうした 
\\	私たちみなを魅了したのは彼の冒険談であった。	わたしたちみなをみりょうしたのはかれのぼうけんだんである 
\\	彼はすべての悪から魔力で守られた。	かれはすべてのあくからまりょくでまもられた 
\\	自分の能力を試すため何か新しいことをやってみる。	じぶんののうりょくをためすためなにかあたらしいことをやってみる 
\\	彼女はお色気たっぷりだ。	かのじょはいろけたっぷりだ 
\\	壁は落書きだらけだ。	かべはらくがきだらけだ 
\\	繁華街を当てもなくウロウロした。	はんかがいをあてもなくウロウロした 
\\	私の思考は当てもなく曲がりくねる。	わたしのしこうはあてもなくまがりくる 
\\	人生は長い曲がりくねった道だ。	じんせいはながいまがりくねったみちだ 
\\	その小川は牧草地の中を曲がりくねって流れている。	そのおがわはぼくそうちのなかをまがりくねってながれている 
\\	羊が牧草地で草を食べている。	ひつじがぼくそうちでたべている 
\\	この池には鯉がうようよいる。	このいけにはこいがうようよいる 
\\	この牧草地には蛙がうようよいる。	このぼくそうちにはかえるがうようよいる 
\\	彼女がその賞を受賞したのは少しも不思議ではない。	かのじょがそのしょうをじゅしょうしたのはすこしもふしぎではない 
\\	その建築家は権威ある賞を受賞したことを自慢した。	そのけんちくかはけんいあるしょうをじゅしょうしたことをじまんした 
\\	私の兄は哲学の権威だ。	わたしのあにはてつがくのけんいだ 
\\	洗面所の栓を抜くと、水がガバガバ流れ出した。	せんめんしょのせんをぬくと、みずがガバガバながれだした 
\\	コルクの栓がどうしても抜けなかった。	コルクのせんがどうしてもぬけなかった 
\\	彼は放心した顔つきをしていた。	かれはほうしんしたかおつきをしていた 
\\	鮮やかな色を使っているのが彼の絵の特徴だ。	あざやかないろをつかっているのがかれのえのとくちょうだ 
\\	私が特に気に入っているのは、この鮮やかな色彩の肖像画です。	わたしがとくにきにいっているのは、このあざやかなしきさいのしょうぞうがです 
\\	文体が作家に持つ関係は、色彩が画家に対するのと同じである。	ぶんたいがさっかにもつかんけいは、しきさいががっかにたいするのとおなじである 
\\	もちろん品も、そして郵送料すらも返ってこなかった。	もちろんひんも、そしてゆうそうりょうすらもかえってこなかった 
\\	私は学園祭で彼らと親しくなった。	わたしはがくえんさいでかれらとしたしくなった 
\\	私は彼女とは子供時代から親しい間柄だ。	わたしはかのじょとはこどもじだいからしたしいあいだがらだ 
\\	私と彼女とは会えば会釈し合う程度の間柄です。	わたしはかのじょとはあえばえしゃくしあうていどのあいだがらです 
\\	折り返し商品を郵送するべし。	おりかえししょうひんをゆうそうするべし 
\\	後で、折り返し電話するわ。	あとで、おりかえしでんわするわ 
\\	王子様は白雪姫に会釈した。	おうじさまはしらゆきひめにえしゃくした 
\\	彼女は祖母の名をとって命名された。	かのじょはそぼのめいをとってめいめいされた 
\\	赤ちゃん、もう命名した?	あかちゃん、もうめいめいした? 
\\	窓ガラスを割ったいたずら坊主はだれだ?	まどガラスをわったいたずらぼうずはだれだ? 
\\	引き出しの鍵がいたずらされて、書類が一部紛失した。	ひきだしのかぎがいたずらされて、しょるいがいちぶふんしつした 
\\	禅坊主じゃあるまいし、毎日毎日、一汁一菜のダイエットメニューは勘弁してよ。	ぜんぼうずじゃなあるまいし、まいにちまいにち、いちじゅういっさいのダエットメニューはかんべんしてよ 
\\	彼は両手をポケットに深く突っ込んでいた。	かれはりょうてをポケットにふかくつっこんでいた 
\\	私たちはそのまま洞窟の口にボートで突っ込んで行きました。	わたしたちはそのままどうくつのくちにボートでつっこんでいきました 
\\	彼は古代史の分野で突っ込んだ研究をしている。	かれはこだいしのぶんやでつっこんだけんきゅうをしている 
\\	我々3人で数百エーカーの土地を共有している。	われわれさんにんですうひゃくエーカーとちをきょうゆうしている 
\\	彼は軸のような重要な役割を演じた。	かれはじくのようなじゅうようなやくわりをえんじた 
\\	その小説は南北戦争を軸にしている。	そのしょうせつはなんぼくせんそうをじくにしている 
\\	ああいう洗練された人々の中で、自分はまったく場違いな気がした。	ああいうせんれんされたひとびとのなか、じぶんはまったくばちがいなきがした 
\\	洗練され教養のある人を区別する標識はなんであろう。	せんれんされきょうようのあるひとをくべつするひょうしきはなんであろう 
\\	彼は生粋のパリっ子です。	かれはきっすいのパリっこです 
\\	日本人離れしたこの美しい顔立ちからもわかるように、優奈は実は生粋の日本人じゃない。	にほんじんばなれしたこのうつくしいかおだちからもわかるように、ゆうなはじつはきっすいのにほんじんじゃない 
\\	時間はあなたの好きなように過ごせばいい。所詮、あなたの時間なのだから。	じかんはあなたのすきなようにすごせばいい。しょせん、あなたのじかんなのだから 
\\	しかし、天才であるが故に一般ピーポーから理解され難いというのは宿命とも言えるわ。	しかし、てんさいであるがゆえにいっぱんピーポーからりかいされにくいというのはしゅくめいともいえるわ 
\\	洪水は嘗てはこの地方には宿命であった。	こうずいはかつてはこのちほうにはしゅくめいであった 
\\	私たちは両親を尊敬し、愛しているが故に、両親に従う。	わたしたちはりょうしんをそんけい、あいしているがゆえに、りょうしんにしたがう 
\\	建築物は、現在では昔よりずっと堅牢になっている。	けんちくぶつは、げんざいはむかしよりずっとけんろうになっている 
\\	彼は家の内装を変えた。	かれはいえのないそうをかえた 
\\	日本の家屋は木造で、火がつきやすい。	にほんのかおくはもくぞうで、ひがつきやすい 
\\	ジェット機は離陸する時に轟音を立てた。	ジェットきはりりくするときにごうおんをたてた 
\\	これがその問題の核心である。	これがそのもんだいのかくしんである 
\\	ものの核心にふれることは容易ではありません。	もののかくしんにふれることはよういではありません 
\\	君は相撲取りの隣にくるとちっぽけにみえる。	きみはすもうとりのとなりにくるとちっぽけにみえる 
\\	東京の地価はどんなにちっぽけなマンションでもべらぼうな値段である。	とうきょうのちかはどんなにちっぽけなマンションでもべらぼうなねだんである 
\\	その災害の被害はべらぼうに大きかった。	そのさいがいのひがいはべらぼうにおおきかった 
\\	湾岸諸国は常に戦争の脅威にさらされている。	わんがんしょこくはつねにせんそうのきょういにさらされている 
\\	富山湾にはときどき蜃気楼が現われます。	とやまわんにはときどきしんきろうがあらわれます 
\\	蜃気楼は幻影だと言われている。	しんきろうはげんえいだといわれている 
\\	トムはそのプロジェクトを随分宣伝しているよ。	トムはそのプロジェクトをずいぶんせんでんしているよ 
\\	そのワープロのお陰で私は随分時間が節約できる。	そのワープロのおかげでわたしはずいぶんじかんがせつやくできる 
\\	放射能が原子力発電所から漏れた。	ほうしゃのうがでんしりょくはつでんしょからもれた 
\\	その幼児は放射線に晒されていた。	そのようじはほうしゃせんにさらされていた 
\\	今回の雨には放射能はない。	こんかいのあめにはほうしゃのうはない 
\\	心臓の鼓動が速まった。	しんぞうのこどうがはやまった 
\\	医師は患者の心臓の鼓動と血圧をモニターで監視した。	いしはかんじゃのしんぞうのこどうとけつあつをモニターでかんしした 
\\	お茶の木は椿の一種です。	おちゃのきはつばきのいっしゅです 
\\	菊はよい香りがする。	きくはよいかおりがする 
\\	物語は結末に近づいた。	ものがたりはけつまつにちかづいた 
\\	私達が住んでいる現代に入ってから、私達は国際論争の結末として、2度の世界大戦を体験した。	わたしたちがすんでいるげんだいにはいってから、わたしたちはこくさいろんそうのけつまつとして、にどのせかいたいせんをたいけんした 
\\	彼は指導者の資質を全て持っていた。	かれはしどうしゃのししつをすべてもっていた 
\\	彼は凡俗を超越している。	かれはぼんぞくをちょうえつしている 
\\	わが国の国民は独立を渇望している。	わがくにのこくみんはどくりつをかつぼうしている 
\\	私たちは老人と楽しく対談した。	わたしたちはろうじんとたのしくたいだんした 
\\	彼は即座に承知した。	かれはそくざにしょうちした 
\\	警察は強盗犯を即座に逮捕しました。	けいさつはごうとうはんをそくざにたいほしました 
\\	事態が悪化しないように即座に手を打った。	じたいがあっかしないようにそくざにてをうった 
\\	彼は私に毒舌を浴びせた。	かれはわたしにどくぜつをあびせた 
\\	彼女は毒舌家だ。	かのじょはどくぜつかだ 
\\	彼女は天邪鬼だ。	かのじょはあまのじゃくだ 
\\	彼の言葉は何を暗示しているのか。	かれのことばはなにをあんじしているのか 
\\	その薬が成長の過程を早めた。	そのくすりがせいちょうのかていをはやめた 
\\	彼はボート作りの過程を説明した。	かれはボートつくりのかていをせつめいした 
\\	民主主義では、国民は直接政府の役人を選ぶ。	みんしゅしゅぎではこくみんはちょくせつせいふのやくにんをえらぶ 
\\	直接税に反対する意見が支配的だった。	ちょくせつぜいにはんたいするいけんがしはいてきだった 
\\	私たちは文通していただけで、直接会ったことはないのです。	わたしたちはぶんつうしていただけで、ちょくせつあったことはないのです 
\\	その役人は、自分自身でその苦情処理が出来なかった。	そのやくにんは、じぶんじしんでそのくじょうしょりができなかった 
\\	彼はひたすら成功を望んで懸命に働く。	かれhひたすらせいこうをのぞんでけねんにはたらく 
\\	彼は目をつむり、腕組みをしたまま、肘掛けいすに座っていた。	かれはめをつむり、うでぐみをしたまま、ひじかけいすにすわっていた 
\\	彼女は私を枕元にくるようにと頼んだ。	かのじょはわたしをまくらもとにくるようにとたのんだ 
\\	彼はその機械の輪郭をスケッチします。	かれはそのきかいのりんかくをスケッチします 
\\	霧のため山の輪郭がぼんやりしていた。	きりのためやまのりんかくがぼんやりしていた 
\\	彼は今まで見た中ではもっとも大きな眉毛をしている。	かれはいままでみたなかではもっともおおきなまゆげをしている 
\\	あなたの話は到底真実だと思えない。	あなたのはなしはとうていしんじつだとおもえない 
\\	彼の部屋は小奇麗だとは到底言えない。	かれのへやはこぎれいだとはとうていいえない 
\\	は飛行機とは到底思えないジグザグの動きで、東の山に飛んでいった。	
\\	はひこうきとはとうていおもえないジグザグのうごきで、ひがしのやまにとんでいった 
\\	彼に会う事などは到底思いもよらない。	かれにあうことなどはとうていおもいもよらない 
\\	「人は見かけによらない」とはよく言ったものですね。	"「ひとはみかけによらない」とはよくいったものですね 
\\	こんな雨降りにピクニックに出かけるなんて思いもよらない。	こんなあめふりにピクニックにでかけるなんておもいもよらない 
\\	その真っ黒な瞳の奥に、自分の姿が鮮やかに浮かんでいる。	そのまっくろなひとみのおくに、じぶんのすがたがあざやかにうかんでいる 
\\	私は門の傍に駐車した。	わたしはもんのそばにちゅしゃした 
\\	この真珠は本物そっくりだ。	このしんじゅはほんものそっくりだ 
\\	ハワイはよく「太平洋の真珠」だと言われる。	"ハワイはよく「たいへいようのしんじゅ」だといわれる 
\\	床にガラスの破片が落ちていた。	ゆかにガラスのはへんがおちていた 
\\	彼女は割れた皿の破片を集めた。	かのじょはわれたさらのはへんをあつめた 
\\	雲が低く垂れ込めた。	くもがひくくたれこめた 
\\	彼の服から滴が垂れていた。	かれのふくからしずくがたれていた 
\\	川岸で数人の男が糸を垂れている。	かわぎしですうにんのおとこがいとをたれている 
\\	汗が額を滴り落ちるのを感じた。	あせがひたいをしたたりおちるのをかんじた 
\\	一滴の涙が彼女の頬を伝って落ちた。	ひとしずくのなみだがかのじょのほおをつたっておちた 
\\	個人の善意は大海の一滴にすぎません。	こじんのぜんいはたいかいのひとしずくにすぎません 
\\	雨が屋根からゆっくりと滴り落ちた。	あめがやねからゆっくりとしたたりおちた 
\\	「転石苔を生ぜず」は諺である。	"「てんせきこけをしょうぜず」はことわざである 
\\	苔は、倒れた丸太や岩の上の繊細な緑の毛だと私は心の中で思う。	こけは、たおれたまるたやいわのうえのせんさいなみどりのけだとわたしはこころのなかでおもう 
\\	日本の女性は小さくて繊細に見える。	にほんのじょせいはちいさくてせんさいにみえる 
\\	彼は子供の頃から繊細だった。	かれはこどものころからせんさいだった 
\\	対等の条件で契約を結びたいと思った。	たいとうのじょうけんでけいやくをむすびたいとおもった 
\\	子供達は以前、親を敬っていたが、今では親を自分たちと対等とみなす傾向がある。	こどもたちはいぜん、おやをうやまっていたがいまではおやをじぶんたちとたいとうとみなすけいこうがある 
\\	両親は私に年寄りを敬うように言った。	りょうしんはわたしにとしよりをうやまうようにいった 
\\	彼らは彼女を恩人として敬った。	かれらはかのじょをおんじんとしてうやまった 
\\	あんな二股かける女なんか忘れたわ、ボケ。	あんたふたまたかけるおんあなttかわすれたわ、ボケ 
\\	しばらくすると二股になった。	しばらくするとふたまたになった 
\\	彼の前途には一筋の希望の光もなかった。	かれのぜんとにはひとすじのきぼうのひかりもなかった 
\\	涙が一筋頬を流れた。	なみだがひとすじほおをながれた 
\\	彼女が自分で自分を嘲笑うのがわたしには魅力的だ。	かのじょがじぶんでじぶんをあざわらうのがわたしにはみりょくてきだ 
\\	彼らの嘲笑いをふと耳にした。	かれらのあざわらいをfとみみにした 
\\	彼は私の努力を嘲った。	かれはわたしのどりょくをあざけった 
\\	無神論者が司教の説明を嘲った。	むしんろんしゃがしきょうのせつめいをあざけった 
\\	彼は自分が無神論者だと告白した。	かれはじぶんがむしんろんしゃだとこくはくした 
\\	司教達はその提案に賛成であった。	しきょうたちはそのていあんにさんせいであった 
\\	「ファンの皆さんあっての僕なんですから...」と、その歌手はファンの握手攻めに快く応じていた。	~がある(いる)から存在できる
\\	人をいろいろな依頼で攻め立てる。	ひとをいろいろないらいでせめたてる 
\\	学生はしばしば先生を質問攻めにする。	がくせいはしばしばせんせいをしつもんぜめにする 
\\	彼は来客攻めにあった。	かれはらいきゃくぜめにあった 
\\	彼女は快く願いを聞いてくれた。	かのじょはこころよくねがいをきいてくれた 
\\	有名なその外交官は快く委員会に参加してくれた。	ゆうめいなそのがいこうかんはこころよくいいんかいにさんかしてくれた 
\\	雄弁なその学者は快く討論に参加してくれた。	ゆうべんなそのがくしゃはこころよくとうろんにさんかしてくれた 
\\	その雄弁な候補者は選挙に楽勝した。	そのゆうべんなこうほしゃはせんきょにらくしょうした 
\\	成功は努力如何による。	せいこうはどうりょくいかんによる 
\\	彼の命は判決如何にかかっている。	かれのいのちははんけついかんにかかっている 
\\	天候の如何に関わらず駅へ迎えに参ります。	てんこうのいかんにかかわらずえきへむかえにまいります 
\\	肌の色の如何を問わず、彼は万人の言論の自由を擁護した。	はだのいろのいかんをとわず、かれはばんじんのげんろんのじゆうをようごした 
\\	幸福は環境如何によるものでなくて、むしろ自分の人生に対する見方如何による。	こうふくはかんきょういかんによるものでなくて、むしろじぶんのじんせいにたいするみかたいかんによる 
\\	テストの成績いかんによっては、進級できない学生もでる。	~がどうであるかによる
\\	オリンピックは、勝敗のいかんによらず、参加することに意義がある	オリンピックは、しょうはいのいかんによらず、さんかすることにいぎがある 
\\	試験の結果いかんにかかわらず、必ず私に連絡してください	~がどうであっても、それに関係なく
\\	あなたは、どんなに有能であっても、昇進しないだろう。	あなたは、どんなにゆうのうであっても、しょうしんしないだろう 
\\	どんな値段であっても私はそれを売りたくない。	どんなねだんであってもわたしはそれをうりたくない 
\\	我々のどちらも相手を映画に連れて行くだけの余裕がなかったので、割り勘にした。	われわれのどちらもあいてをえいがにつれていくだけのよゆうがなかったので、わりかんにした 
\\	君はそこへ行こうと行くまいと、私には関係のないことだ	たとえ~しても
\\	信じようと信じまいと、それは真実だ。	しんじようとしんじまいと、それはしんじつだ 
\\	君が賛成しようとしまいと変わりはない。	きみがさんせいしようとしんまいとかわりはない 
\\	彼が来ようが来まいが結果は同じだろう。	かれがこようがきまいがけっかはおなじだろう 
\\	ぼくはうっかり受話器をはずさないでダイヤルを回した。	ぼくはうっかりじゅわきをはずさないでダイヤルをまわした 
\\	うっかり降りるところを通り越した。	うっかりおりるところをとおりこした 
\\	彼らは慈善募金を始めた。	かれらはじぜんぼきんをはじめた 
\\	当局は通貨を安定させようと懸命だが、どうにもならない。	とうきょくはつうかをあんていさせようとけんめいだが、どうにもならない 
\\	倒れた木に妨害されて、車は通ろうにも通れない	~したくてもできない
\\	第2に、喫煙者は、何れにせよ喫煙は自分では全くどうにもならないと信じ込んでいるのである。	だいにに、きつえんしゃは、いずれにせよきつえんはじぶんではまったくどうにもならないとしんじこんでいるのである 
\\	過ぎてしまったこと今更どうにもならないだろう。	すぎてしまったこといまさらどうにもならないだろう 
\\	今更悔やんでも後の祭りだ。	いまさらくやんでもあとのまつりだ 
\\	その象は一歩も動こうとしない。	そのぞうはいっぽもうごこうとしない 
\\	このドア、いくら押しても開こうとしない。鍵がかかっているようだ。	絶対に~しない
\\	彼は自分が悪くても決して認めようとしない。	かれはじぶんがわるくてもけっしてみとめようとしない 
\\	万一彼が君の結婚のことを聞こうものなら、彼はものすごく怒るだろう。	まんいちかれがきみのけっこんのことをきこうものなら、かれはものすごくおこるだろう 
\\	私はアルコールに弱くて、ビール一杯でも飲もうものなら、顔が真っ赤になってしまう。	~したら(たいへんだ)
\\	学生時代の仲間がみんな集まるなんて、うれしいかぎりだ。	非常に~だと感じる
\\	彼に酒を飲ませないようにすることはできない。	かれにさけをのませないようにすることはできない 
\\	自然は一度破壊されたが最後、元の状態に戻すことはできない。	~したら
\\	彼は浜辺へ行き、海上はるか彼方の水平線を眺めた。	かれははまべへいき、かいじょうはるかかなたのすいへいせんをながめた 
\\	らしき物体はガクンと方向を変え、空の彼方に消えた。	
\\	らしきぶったいはガクンとほうこうをかえ、そらのかなたにきえた 
\\	車は草地に飛び込み、しばらくガクンガクンと走って止まったのです。	くるまはくさちにとびこみ、しばらくガクンガクンとはしってとまったのです 
\\	彼らは来春にそのプロジェクトを実施する。	かれらはらいしゅんにそのプロジェクトをじっしする 
\\	新しい貿易区域の計画はまだ実施の段階にありません。	あたらしいぼうえきくいきのけいかくはまだじっしもだんかいにありません 
\\	アメリカでは禁酒法が実施されている州がまだいくつかあった。	アメリカではきんしゅほうがじっしされているしゅうがまだいくつかあった 
\\	彼の店は町の賑やかな区域にある。	かれのみせはまちのにぎやかなくいきにある 
\\	管理区域 
\\	許可なくして立ち入りを禁ず。	かんりくいき 
\\	きょかなくしてたちいりをきんず 
\\	来春結婚する孫が、結婚の情報かたがた婚約者を連れてやって来た。	~のついでに
\\	彼は読書をし、その傍らで妻が手袋を編んでいた。	かれはどくしょをし、そのかたわらでつまがてぶくろをあんでいた 
\\	審判はコートの傍らにある高い椅子に座る。	しんぱんはコートのかたわらにあるたかいいすにすわる 
\\	妹は会社に勤めるかたわら、夜は専門学校で英語を勉強しています。	~しながら
\\	そんな物を売っても二束三文にしかならない。	そんなものをうってもにそくさんもんにしかならない 
\\	二足のわらじをはこうとして失敗した。	にそくのわらじをはこうとしてしっぱいした 
\\	彼が我が軍の指揮官です。	かれはわがぐんのしきかんです 
\\	このオーケストラの指揮者は優れた音楽家です。	このオーケストラのしきしゃはすぐれたおんがくかです 
\\	彼女たちは選挙権を勝ち取るための運動を指揮した。	かのじょたちはせんきょけんをかちとるためのうんどうをしきした 
\\	外見から判断して、彼が指揮者にちがいない。	そとみからはんだんして、かれがしきしゃにちがいない 
\\	そうだ、どうせなら散歩がてらに、林道に行ってプチ森林谷でも・・・。	そうだ、どうせならさんぽがてらに、りんどうにいってプチしんりんたにでも・・・ 
\\	花見がてら上野へ買い物に行ったが、あまりの人出に驚いて、桜もゆっくり見ずに帰ってきた。	~をするときに
\\	大勢の人出でしたよ。	おおぜいのひとででしたよ 
\\	地下鉄の駅までの道順を教えていただけませんか。	ちかてつのえきまでのみちじゅんをおしえていただけませんか 
\\	助手席に置いていた鞄がない!	じょしゅせきにおいていたかばんがない! 
\\	愛国心にかこつけて多くの殺人が行われてきた。	あいこくしんにかこつけておおくのさつじんがおこなわれてきた 
\\	彼女は叔父さんの病気見舞いにかこつけて、昨日から故郷の実家へ帰っている。	~を口実にして
\\	ボブは昆虫の視察に楽しみを感じている。	ボブはこんちゅうのしさつにたのしみをかんじている 
\\	彼はこの町の産業を視察するために近く当地へやって来ます。	かれはこのまちのさんぎょうをしさつするためにとうちへやってきます 
\\	その突発について不必要に心配する必要はない。	そのとっぱつについてふひつようにしんぱいするひつようはない 
\\	ほんの一握りの男性しか育児休暇を取りたがらない。	ほんのひとにぎりのだんせいしかいくじきゅうかをとりたがらない 
\\	日本語のむずかしさが、一握りの外国人を除いてすべての外国人が、原語で日本文学に近づくのを妨げている。	にほんごのむずかしさが、ひとにぎりのがいこくじんをのぞいてすべてのがいこくじんが、げんごでにほんぶんがくにちかづくのをさまたげている 
\\	彼はまたとない優れた思想家である。	かれはまたとないすぐれたしそうかである 
\\	またとないチャンスだ。このチャンスを逃してはならない。	二度とない  とてもよい
\\	そんなところで彼女と会うなんて、思ってもみなかった。	そんなところでかのじょとあうなんて、おもってもみなかった 
\\	株の値下がりは、とどまるところを知らない。これでもう3ヶ月も下がり続けている。	どんどん進んで止まらない
\\	警戒するに越したことはない。	けいかいするにこしたことはない 
\\	英語ができるに越したことはないが、できなくてもこの仕事をすることはできる。	どちらでもいいが、まあ、~するほうがいい
\\	核兵器は人類の破滅以外の何も齎さないだろう。	かくへいきはじんるいのはめついがいのなにももたらさないだろう 
\\	将軍は難局に敢然と立ち向かい、自軍を破滅から救った。	しょうぐんはなんきょくにかんぜんとたちむかい、じぐんをはめつからすくった 
\\	お世辞に乗せられると、身の破滅を招く。	おせじにのせられると、みのはめつをまねく 
\\	彼は聞こえないふりをしてその難局を切り抜けた。	かれはきこえないふりをしてそのなんきょくをきりぬけた 
\\	山の麓に日が昇るが早いかただ一人山を登り始めた。	やまのふもとにひがのぼるがはやいかただひとりやまをのぼりはじめた 
\\	ベルが鳴るがはやいか、学生は机の上の本やノートを鞄にしまう。	と、すぐ
\\	休暇中、姉と私は富士山の麓にある小さな村に滞在した。	きゅうかちゅう、あねとわたしはふじさんのふもとにあるちいさなむらにたいざいした 
\\	その俳優は5ページからある長い台詞をすぐに覚えてしまった。	~もある
\\	そのお相撲さんは大きいねえ。体重画100キロはあるだろう。	~以上  少なくとも~
\\	彼女、細いねえ。40キロもないんじゃない。	~以下  多くても~
\\	彼は慎重すぎて、決断が遅れるきらいがある。	~(の)傾向がある
\\	ジムくんは行き過ぎの嫌いがある。	ジムくんはいきすぎのきらいがある 
\\	その模様の色は実に平凡なものである。	そのもようのいろはじつにへいぼんなものである 
\\	彼は地位も名声ももたない平凡な人だ。	かれはちいもめいせいももたないへいぼんなひとだ 
\\	その小説を読んでみたが、平凡極まるストーリーで、がっかりした。	非常に~
\\	彼女は感極まって泣いた。	かのじょはかんきわまってないた 
\\	その机は、乱雑極まりない状態だ。	そのつくえは、らんざつきわまりないじょうたいだ 
\\	彼女はいつでも何でも一番上等のものしか買わない。	かのじょはいつでもなんでもいちばんじょうとうのものしかかわない 
\\	柔らかいウールの方が粗いウールより高価で、そのどちらともナイロン製の人工繊維より上等である。	やわらかいウールのほうがあらいウールよりこうかで、そのどちらともナイロンせいのじんこうせんいよりじょうとうである 
\\	今夜はお祝いだから、とびきり上等のワインを飲もう。	とても。
\\	彼は飛び切りやさしい男だった。	かれはとびきりやさしいおとこだった 
\\	利益追求に必死のブローカーは、必ずしもルールブックに則っているとは限らないのです。	りえきついきゅうにひっしのブローカーは。かならずしもルールブックにのっとっているとはかぎらないのです 
\\	日本の神道の儀式に則って、挙式をしたいという方がおられれば、この教会で出来ます。	にほんのしんとうのぎしきにのっとって、きょしきをしたいというかたがおられれば、このきょうかいでできます 
\\	いつものごとく、彼の考えはあまりに非現実的だった。	いつものごとく、かれのかんがえはあまりにひげんじつてきだった 
\\	あれは朝にラブホから出るのを目撃されるかのごとく気まずかった。	あれはあさにラブホテルからでるのをもくげきされるかのごとくきまずかった 
\\	鳥のごとく空を飛ぶこと、これは人間の夢である。	~(の)ような  ~(の)ように
\\	その事件については、かなり以前のこととて、ほとんど記憶しておりませんが。	~なので  ~だから
\\	教えられることなしにできるようになる。	~しないで  ~せずに
\\	人は生まれたときから他の人々を頼ることなしに生きていくこと。	ひとはうまれたときからほかのひとびとをたよることなしにいきていくこと 
\\	論文の締め切りが迫っていて、2晩も徹夜するしまつだ。もっと早く書き始めればよかったのだが。	~ということになってしまった
\\	彼女は哀れを誘う有様だった。	かのじょはあわれをさそうありさまだった 
\\	最近弟はお金に困っていて、有様に借金までする有様です。	【ありさま】 ~状態
\\	規則ずくめだが、静かにするという規則だけはない病院!?	きそくずくめだが、しずかにするというきそくだけはあいびょういん!? 
\\	あの黒ずくめの紳士は誰ですか。	あのくろずくめのしんしはだれですか 
\\	娘は結婚したし、息子は就職が決まったし、今年はいいことずくめだった。	~ばかりたくさんある
\\	心づくしの品を頂きありがとうございます。	こころづくしのひんをいただきありがとうございます 
\\	大衆は子供のようなので、中に何が入っているかを見る為には、何でもかでも粉砕せずにはおかぬ。	たいしゅうはこどものようなので、なかになにがはいっているかをみるためには、なんでもかでもふんさいせずにはおかぬ 
\\	何か重い兇器でやられたらしく、頭蓋骨は粉砕された。	なんかおもいきょうきでやられたらしく、ずがいこつはふんさいされた 
\\	調べを進めるうちに、頭蓋骨が、何か重い一撃を受けて打ち砕かれているのが明らかになった。	しらべすすめるうちに、ずがいこつが、なにかおもいいちげきをうけてうちくだかれているのがあきらかになった 
\\	警官は凶器を持った強盗の頭をうった。その強盗は即死も同然だった。	けいかんはきょうきをもったごうとうのあたまをうった。そのごうとうはそくしもとうぜんだった 
\\	彼はその事故に遭い、即死した。	かれはそのじこにあい、そくしした 
\\	会社の合理化の話は、社員を不安にさせずにはおかない。	絶対に~する  必ず~する
\\	彼のふしだらな行為は気付かれずにはすまなかった。	かれのふしだらなこういはきづかれずにはさまなかった 
\\	この件については、正直に上司に話さずにはすまないと思う。	必ず~しなければならない
\\	両親は子どもを叱らないわけにはいかなかった。	りょうしんはこどもをしからないわけにはいかなかった 
\\	私はこの女性労働者達の健康について心配しないわけには行けない。	わたしはこのじょせいろうどうしゃたちについてしんぱいしないわけにはいけない 
\\	止むを得ず訪問するのはいやだったが、やはりしないわけにはいかなかった。	やむをえずほうもんするのはいやだったが、やはりしないわけにはいかなかった 
\\	私たちは彼を勇敢な男だと認めないわけにはいかない。	わたしたちはかれをゆうかんなおとことみとめないわけにはいかない 
\\	彼らの努力には賞賛しないわけにはいけません。	かれらのどりょくにはしょうさんしないわけにはいけません 
\\	彼の賞賛の対象は、彼がなりたい種類の人間を示す。	かれのしょうさんのたいしょうは、かれがなりたいしゅるいのにんげんをしめす 
\\	少年は最善を尽くしたという限りにおいて、賞賛されるべきだ。	しょうねんはさいぜんをつくしたというかぎりにおいて、しょうさんされるべきだ 
\\	子供の溺れるところを救った彼の勇気には賞賛の言葉もない。	こどものおぼれるところをすくったかれのゆうきにはしょうさんのことばもない 
\\	手術後しばらくの間は水すら飲めず、辛かった。	~も  ~さえ  ~でさえ
\\	彼ですらそれが上手にできるなら、我々ならなおさらだ。	かれですらそれがじょうずにできるなら、われわれならなおさらだ 
\\	彼は、オペラはもちろんのこと、童謡すら歌えない。	かれは、オペラはもちろんのこと、どうようすらうたえない 
\\	あなたは童謡のレコードのセットをさがしているそうですね。	あなたはどうようのレコードのセットをさがしているそうですね 
\\	彼女は、彼のことをペテン師とすら言った。	かのじょは、かれのことをペテンしとすらいった 
\\	その歌手は売れっ子になるだろう。	そのかしゅはうれっこになるだろう 
\\	「近頃はとっても忘れっぽくなって、何でも聞いたそばから忘れてしまうの」とよくそぼが言っている。	(~すると、)そのすぐ後で、また
\\	練習は十分やった。あとはただ試合の日を待つのみだ。	~だけ  ~しかない
\\	ただ人間のみならず動物もストレスを感じるというのは、驚くべきことだ。	~だけでなく  ひとり~のみならず
\\	ここから駅まで15分はかかるから、どんなに急いだところで、9時の特急には間に合わない。	~ても  ~たとえ~としても
\\	運命に文句を言ってみたところで始まらない。	うんめいにもんくをいってみたところではじまらない 
\\	父は、定年までには、約30年働くことになる。	ちちは、ていねんまでには、やくさんじゅうねんはたらくことになる 
\\	株式市場の暴落で、定年退職者の多くが労働市場に戻らざるを得なかった。	かぶしきしじょうのぼうらくで、ていねんたいしょくしゃのおおくがろうどうしじょうにもどらざるをえなかった 
\\	企業業績の改善は株式市場の回復が背景にある。	きぎょうぎょうせきのかいぜんはかぶしきしじょうのかいふくがはいけいにある 
\\	投資銀行家たちは暴落で途方に暮れています。	とうしぎんこうかたちはぼうらくでとほうにくれています 
\\	あんたらの名前なんか興味ないね。どうせこの仕事が終わるとお別れだ。	あんたらのなまえなんかきょうみないね。どうせこのしごとがおわるとおわかれだ 
\\	どうせやるなら上手にやれ。	どうせやるならじょうずにやれ 
\\	どっちみち、関係ないよ。	どっちみち、かんけいないよ 
\\	彼は過密スケジュールを都合して、私の舞台を見に来てくれた。	かれはかみつスケジュールをつごうして、わたしのぶたいをみにきてくれた 
\\	この過密都市東京で、万一大地震が起こったら、どうなるだろう。考えるだに恐ろしいことだ。	~さえ  ~すら
\\	その子供は恐怖で身動きできなかった。	そのこどもはきょうふでみうごきできなかった 
\\	70歳にしては、彼はいまだに驚くほど元気である。	ななじゅうさいにしては、かれはいまだにおどろくほどげんきである 
\\	私の祖母は、口癖のように100歳まで生きると言っていましたが、85歳で亡くなりました。	わたしのそぼは、くちぐせのようにひゃくさいまでいきるといっていましたが、はちじゅうごさいでなくなりました 
\\	「倹約しろ。1円たりとも無駄には使うな」というのは父の口癖だ。	~も  ~たった~さえも
\\	判事は神経性の過労でくたくただった。	はんじはしんけいせいのかろうでくたくただった 
\\	判事は傍聴人に静かにするよう警告した。	はんじはぼうちょうにんにしずかにするようけいこくした 
\\	傍聴人の一人が大声を上げて議事進行を妨げた。	ぼうちょうにんのひとりがおおごえをあげてぎじしんこうをさまたげた 
\\	時々、政治家の一人がテレビの討論会に出て傍聴者の意見を押さえつけようとする場面をみる。	ときどき、せいじかのひとりがテレビのとうろんかいにでてぼうちょうしゃのいけんをおさえつけようとするばめんをみる 
\\	前回の議事録は承認されました。	ぜんかいのぎじろくはしょうにんされました 
\\	6月16日のミーティングの最終議事事項をお送りします。	ろくがつじゅうろくにちのミーティングのさいしゅうぎじじこうをおおくりします 
\\	彼は一日たりとも自分のしたことを後悔せずに過ごした日はなかった。	かれはいちにちたりともじぶんのしたことをこうかいせずにすごしたひはなかった 
\\	何人たりとも公共の利益を独占すべきではない。	なにびとたりともこうきょうのりえきをどくせんすべきではない 
\\	1銭たりとも無駄にしないのが彼の信条だ。	いっせんたりともむだにしないのがかれのしんじょうだ 
\\	彼は一銭ももらえずその日暮らしだった。	かれはいっせんももらえずそのひぐらしだった 
\\	彼は早起きを信条にしていた。	かれははやおきをしんじょうにしていた 
\\	理性的な人であれば、政治的信条がどうであれ、その計画に反対することはないだろう。	りせいてきなひとであれば、せいじてきしんじょうがどうであれ、そのけいかくにはんたいすることはないだろう 
\\	その愛国者は自分の道徳的な信条を曲げない。	そのあいこくしゃはじぶんのどうとくてきなしんじょうをまげない 
\\	同社は国のタバコ業を独占している。	どうしゃはくにのタバコぎょうをどくせんしている 
\\	女の子が父親の愛情を独占したいと思い、母親を競争者とみなしがちであった。	おんなのこがおやのあいじょうをどくせんしたいとおもい、ははおやをきょうそうしゃとみなしがちであった 
\\	子たる者すべからく親の命に従うべし。	こたるものすべきからくおやのいのちにしたがうべし 
\\	医者たる者は最新の医学の発達についていくべきだ。	いしゃたるものはさいしんのいがくのはったつについていくべきだ 
\\	警察官たる者、本来なら飲酒運転などするはずがないのだが。	~である  ~という立場にある
\\	この本は興味津々たるものがあって飽きない。	このほんはきょうみしんしんたるものがあってあきない 
\\	彼の決心は確固たるものだった。	かれはけっしんはかっこたるものだった 
\\	いずれの会社にも確固たる事業計画がある。	いずれのかいしゃにもかっこたるじぎょうけいかくがある 
\\	喫煙が健康に悪いと言う確固たる証拠がある。	きつえんがけんこうにわるいというかっこたるしょうこがある 
\\	確たる信念をもって生きているという人は少ないだろう。	【かくたる】 確かな~
\\	刑事はその男が有罪だという確たる証拠を握った。	けいじはそのおとこがゆうざいというかくたるしょうこをにぎった 
\\	その会社は確たる理由もなく、彼を不採用にしました。	そのかいしゃはかくたるりゆうもなく、かれをふさいようにしました 
\\	彼女は涙を辛うじて押さえた。	かのじょはなみだをかろうじておさえた 
\\	際どいところを辛うじて助かった。	きわどいところをかろうじてたすかった 
\\	際どいところで勝つ。	きわどいところでかつ 
\\	私は2、3秒のきわどいところで終バスに間に合った。	わたしはに、さんびょうのきわどいところでしゅうバスにまにあった 
\\	わかりにくい場所だった。地図を見ながら行きつ戻りつ30分も歩き回って、やっとその店を見つけることができた。	~たり~たりしながら
\\	彼は部屋の中を行きつ戻りつした。	かれはへやのなかいきつもどりつした 
\\	双子の兄弟にあった。ためつすがめつ見たけれど違いはわからなかった。	いろいろの方向から(よく見る)
\\	あ、いけない。図書館の本を借りっぱなしだ。返さなくちゃ。	~たままだ
\\	父が寝たきりなので私たちが交代で面倒をみています。	ちちがねたきりなのでわたしたちがこうたいでめんどうをみています 
\\	彼は彼の先生を誉めちぎった。	かれはかれのせんせいのほめちぎった 
\\	ライオンは死んだ麒麟の肉を食いちぎった。	ライオンは死んだきりんのにくをくいちぎった 
\\	寒くて、おまけに風が強かった。	それに加えて。その上。さらに。
\\	彼は裕福だし、おまけに名門の出だ。	かれはゆうふくだし、おまけにめいもんのでだ 
\\	事故はとかく起こりがちなもの。	ある状態になりやすいさま。または、ある傾向が強いさま。
\\	人はとかく自分は欠点がないと考えがちである。事実をありのままに述べなさい。	ひとはとかくじぶんはけってんがないとかんがえがちである。じじつをありのままにのべなさい 
\\	ありのままの事実を伝えることは難しい。	実際にあるとおり。偽りのない姿。
\\	この小説は、百年前の日本人のありのままの生活を描いている。	このしょうせつは、ひゃくねんまえのにほんじんのありのままのせいかつをかいている 
\\	交渉はほとんど進展しなかった。	【しんこう】 事態が進行して、新たな局面があらわれること。また、物事が進歩・発展すること
\\	軍縮については超大国間で意義深い進展があった。	ぐんしゅくについてはちょうたいこくかんでいぎぶかいしんてんがあった 
\\	超大国が激しい国境紛争を解決するために本格的に交渉した。	ちょうたいこくがはげしいこっきょうふんそうをかいけつsるためにほんかくてきにこうしょうした 
\\	化学物質に敏感な人々への配慮がない。	かがくぶっしつにびんかんなひとびとへのはいりょがない 
\\	私はムスリムではないので断食を守る義務はないのだが、同じアパートで暮らす以上、そうした慣習に配慮することは大事なことだ。	わたしはムスリムではないのでだんじきをまもるぎむはないのだが、おなじアパートでくらすいじょう、そうしたしゅうかんにはいりょすることはだいじなことだ 
\\	私は指定の日までに仕事を片づけなければならない。	【してい】 人・時・所・事物などを特にそれとさして決めること。
\\	新型インフルエンザの影響は、国内にとどまらず国外にも及んだ。	~で終わらないで。  さらに。
\\	蒸気ボイラーは爆発する可能性がある。	じょうきボイラーはばくはつするかのうせいがある 
\\	水は熱せられると蒸気になる。	みずはねっせられるとじょうきになる 
\\	蒸し暑いと心も体もだらける。	むしあついとこころもからだもだらける 
\\	そのコラムニストは古い醜聞を蒸し返した。	そのコラムニストはふるいしゅうぶんをむしかえした 
\\	醜聞のニュースのため、その政治家は体面を失った。	しゅうぶんのニュースのため、そのせいじかはたいめんをうしなった 
\\	あんな行動は彼の体面を汚すものだ。	あんなこうどうはかれのたいめんをよごすものだ 
\\	蒸し返すのはやめろ。	むしかえすのはやめろ 
\\	このスキー場は、週末は言うに及ばず平日も若者たちでにぎわっています。	~はもちろん  ~は言うまでもなく
\\	彼女はフランス語は言うに及ばず母国語すらろくに話せない。	かのじょはフランスごはいうにおよばずぼこくごすらろくにはなせない 
\\	多くの女性は自分の名前もろくに書けなかった。	おおくのじょせいはじぶんのなまえもろくにかけなかった 
\\	彼女はろくに食べないで席を立った。	かのじょはろくにたべないでせきをたった 
\\	この次の火曜日、すなわち9月10日に君に会いたい。	前に述べた事を別の言葉で説明しなおすときに用いる。言いかえれば。つまり。
\\	「失敗は成功の母」というのは、「失敗は、とりもなおさず成功の元である」という意味です。	(書き言葉)  すなわち  つまり  言葉を変えて言えば
\\	その事実は誰にでもわかりきったことだ。	すっかりわかっている。あたりまえの
\\	そんなことは言わずもがなだ。	今さら言う必要がない。  もちろん。  わかりきった。
\\	いまさらあの事故のことで彼を非難しても仕方ない。	もっと早ければともかく、今となっては遅すぎる、という意を表す。今ごろになって。
\\	株式市場は長い不振を続けている。	かぶしきしじょうはながいふしんをつづけている 
\\	今の経済の不振は深刻な状況にはならないだろう。	いまのけいざいのふしんはしんこくなじょうきょうにはならないだろう 
\\	都合により明日の会には出席できません。悪しからず、よろしくお願い申し上げます。	【あしからず】 (手紙や挨拶で使う) 悪いのですが  悪く思わないで
\\	私は私と同姓同名の女の人を知っている。	わたしはわたしとどうせいどうめいのおんなのひとをしっている 
\\	君の考えは当たらずとも遠からずだ。	【あたらずともとおからず】 全く正しいとは言えないが、大体当たっている
\\	地震の予知が出来る日が遠からずやってくるだろう。	じしんのよちができるひがとおからずやってくるだろう 
\\	彼は怪我にかまわず戦っていた。	かれはけがにかまわずたたかっていた 
\\	彼は他人の感情にかまわず思っていることを口に出す。	かれはたにんのかんじょうにかまわずおもっていることをくちにだす 
\\	若いころは生活が苦しかったので、なりふりかまわず、ひとすら働いた。	服装や振る舞いなどのことを考えずに必死で
\\	日本の主婦の中には主人に構わずにおいて満足している人もいる。	にほんのしゅふのなかにはしゅじんにかまわずにおいてまんぞくしているひともいる 
\\	成功率が10㌫以下であれ、成功する可能性があるなら、手術を受けようと思う。	~ても  ~でも  ~(の)場合でも
\\	彼女は奨学金のおかげで大学に進学することができた。	かのじょはしょうがくきんのおかげでだいがくにしんがくすることができた 
\\	偶然であれ故意であれ、彼がそれをしたのは本当だ。	ぐうぜんであれこいであれ。かれはそれをしたのはほんとうだ 
\\	選挙では、それが誰であれ、過半数を得た候補者が当選とされる。	せんきょでは、それがだれであれ、かはんすうをえたこうほしゃがとうせんとされる 
\\	あなたの職業がなんであれ、また、その職業にどんなに満足していても、何かほかの仕事を選べばよかったと思うときがあるものだ。	あなたのしょくぎょうがなんであれ、また、そのしょくぎょうにどんんだにまんぞくしていても、なにかほかのしごとをえらべばよかったとおもうときがあるものだ 
\\	どういう始まりであったにせよバレンタインデーには長いロマンチックな歴史がある。	どういうはじまりであったにせよバレンタインデーにはながいロマンチックなれきしがある 
\\	冗談にしろ、恐怖からにしろ、決してうそを言うな。	じょうだんにしろ、きょうふからにしろ、けっしてうそをいうな 
\\	いずれにしろ、明日は列車に乗りなさい。	いずれにしろ、あしたはれっしゃにのりなさい 
\\	新学期が始まってからというもの、とっても忙しいの。	しんがっきがはじまってからというもの、とってもいそがしいの 
\\	子供が生まれてからというもの、生活のすべてが子供中心になっている。	~てから、ずっと  ~た後、ずっと
\\	ミスを繰り返す、遅刻が多い、これが怠慢でなくてなんだろう。	まさに~だ  これこそ~だ
\\	ロイは秘密主義だがテッドはざっくばらんだ。	ロイはひみつしゅぎだがテッドはざっくばらんだ 
\\	出生率の低下は、育児に対する国民の不安感の表れにほかならない。	まさに~だ
\\	彼の横暴ぶりは目に余った。	かれのおうぼうぶりはめにあまった 
\\	彼の横暴な態度には友人たちはみな不愉快に思っている。	かれのおうぼうなたいどにはゆうじんたちはみなふゆかいにおもっている 
\\	グループ全員の意見を無視してことを進めるのは横暴以外の何ものでもない。	これこそ~だ
\\	不可能以外のなにものでもない。	ふかのういがいのなにものでもない 
\\	子供ではあるまいし、こんな簡単な漢字もかけないなんて、恥ずかしいことだ。	~ではないのだから
\\	彼女がいなくなったら世界が終わる訳じゃあるまいし。	かのじょがいなくなったらせかいがおわるわけじゃあるまいし 
\\	私達は金持ちではあるまいし。	わたしたちはかねもちではあるまいし 
\\	そんなことをするなんて馬鹿じゃあるまいか。	たぶん~だ  ~に違いない
\\	一体いつ地球上から戦争がなくなるのだろうか。平和を祈ってやまない。	(私は)いつまでも~ている  心から~ている
\\	素晴らしい料理とサービスのよさとが相まって、このレストランの評判は上がる一方だ。	【あいまって】 ~と一緒になって
\\	実用性と芸術性が相まって住みよい家ができる。	じつようせいとげいじゅつせいがあいまってすみよいいえができる 
\\	性能のよさとデザインの優美さが両々相まって、評判を高めてきた。	せいのうのよさとデザインのゆうびさがりょうりょうあいまって、ひょうばんをたかめてきた 
\\	当社の目的は工場や家庭に高性能のロボットを提供することです。	とうしゃのもくてきはこうじょうやかていにこうせいのうのロボットをていきょうすることです 
\\	技師がその高性能なシステムの操作方法を実演してくれた。	ぎしがそのこうせいのうなシステムのそうさほうほうをじつえんしてくれた 
\\	我々の利害は相反するようだ。	【あいはんする】 一致しない  食うちがう
\\	相反する理想を抱いた二つの強力な勢力が対決するのは、最終手段の時である。	あいはんするりそうをいだいたふたつのきょうりょくなせいりょくがたいけつするのは、さいしゅうしゅだんのときである 
\\	経営側と労働者側との対決があるでしょう。	けいえいがわとろうどうしゃがわとのたいけつがあるでしょう 
\\	失敗のさいの危険を考慮しながら、彼は相手との対決を要求した。	しっぱいのさいのきけんをこうりょしながら、かれはあいてとのたいけつをようきゅうした 
\\	首相が政敵の挑戦と真っ向から対決しました。	しゅしょうがせいてきのちょうせんとまっこうからたいけつしました 
\\	その勢力的な男は様々な活動に携わっている。	そのせいりょくてきなおとこはさまざまなかつどうにたずさわっている 
\\	我々は、誰であろうとテロ活動に携わるものに寛容でいるつもりはない。	われわれは、だれであろうテロかつどうにたずさわるものにかんようでいるつもりはない 
\\	戦闘に携わる兵士たちは戦闘が止んでいるときに好んで、平穏無事な時代を想い返す。	せんとうにたずさわるへいしたちはせんとうがやんでいるときにこのんで、へいおんぶじなじだいをおもいかえす 
\\	その戦闘機は爆弾を投下した。	そのせんとうきはばくだんをとうかした 
\\	1945年広島に原子爆弾が投下された。	せんきゅうひゃくよんじゅうごねんひろしまにげんしばくだんがとうかされた 
\\	うんと暖かくなった。	数量の多いさま。たくさん。どっさり。
\\	彼は金がうんとあって使えない。	かれはかねがうんとあってつかえない 
\\	食卓には果物がどっさりのせてあった。	数量の多いさま。たくさん。
\\	連休の最後の日とあって、遊園地には家族連れが多かった。	~だから  ~という状況だから
\\	上司の命令とあれば、長期出張も海外主張のせざるを得ない。	~なら
\\	君を助けるためなら私は何でも喜んでする。	きみをたすけるためならわたしはなんでもよろこんでする 
\\	味といい香りといい、今年のワインは最高のできだ。	~も~も
\\	悪いって言えば、悪いけど、みんな税金は払いたくないだろう。だから「脱税の知識」なんて本が売れるんだ。	~と言えるが、しかし
\\	穴をたくさん持っていて、しかも液体を保つのに適しているのはなんだ。	あなをたくさんもっていて、しかもえきたいをたもつのにてきしているのはなんだ 
\\	われわれは教育的見地から、その事柄について議論した。	われわれはきょういくてきけんちから、そのことがらについてぎろんした 
\\	私はその事柄について彼の父に話すつもりだったが、思い直してやめた。	わたしはそのことがらについてかれのちちにはなすつもりだったが、おもいなおしてやめた 
\\	彼らは、貴社とお互いに利益となる事柄について話し合うことを望んでいます。	かれらは、きしゃとおたがいにりえきとなることがらについてはなしあうことをのぞんでいます 
\\	あれらの修理工たちは時給一万円ももらっている。	あれらのしゅうりこうたちはじきゅういちまんえんももらっている 
\\	単純な仕事だから、自給は800円から900円というところだ。	だいたい~ぐらいだ。
\\	彼女は多く見ても20歳というところだ。	かのじょはおおくみてもにじゅうさいというところだ 
\\	あのプレイボーイと結婚するのはブロンドの女の子か、ブルーネットの女の子か五分五分といったところだ。	あのプレイボーイとけっこんするのはブロンドのおんあのこか、ブルーネットのおんなのこかごぶごぶといったところだ 
\\	ライオンズが勝つかタイガースが勝つか、五分と五分といったところ。どちらも、同じように強いから。	ライオンズがかつかタイガーズがかつか。ごぶごぶといったところ。おちらも、おなじようにつよいから 
\\	彼はいうところの「おたく」で、ほとんど外出せずに部屋でゲームばかりしているそうだ。	いわゆる
\\	君の狙いはなんなのだ。	きみのねらいはなんなのだ 
\\	その選手はボールを狙ってバットを振った。	そのせんしゅはボールをねらってバットをふった 
\\	兄は大学にいけるように奨学金を狙っている。	あにはだいがくにいけるようにしょうがくきんをねらっている 
\\	好きであろうとなかろうと、勉強は根気よく続けなければだめだ。	すきであろうとなかろうと、べんきょうはこんきよくつづけなければだめだ 
\\	英会話に堪能になりたかったから、根気よく続けてやらないと駄目だ。	えいかいわにたんのうになりたかったから、こんきよくつづけてやらないとだめだ 
\\	あなたが賛成であろうとなかろうと、私は彼を議長に推薦します。	あなたがさんせいであろうとなかろうと、わたしはかれをぎちょうにすいせんします 
\\	幸福が最高の価値であろうとなかろうと、人間はそれを切望する。	こうふくがさいこうのかちであろうとなかろうと、にんげんはそれをせつぼうする 
\\	素早い決定を切望していたので、議長は投票を要求した。	すばやいけっていをせつぼうしていたので、ぎちょうはとうひょうをようきゅうした 
\\	彼女が僕の申込を受諾してくれるように切望していた。	かのじょがぼくのもうしこみをじゅだくしてくれるようにせつぼうしていた 
\\	問題は私が受諾するか拒絶するかである。	もんだいはわたしがじゅだくするかきょぜつするかである 
\\	美しいということは、無視することがほとんど不可能な推薦状のようなものである。	うつくしいということは、むしすることがほとんどふかのうなすいせんじょうのようなものである 
\\	私は彼が推薦する人なら誰でも雇うつもりだ。	わたしはかれがすいせんするひとならだれでもやとうつもりだ 
\\	これはあれよりも長いように見えるが、錯覚だ。	これはあれよりもながいようにみえるが、さっかくだ 
\\	目の錯覚かと思った。	めのさっかくかとおもった 
\\	1時間経過すれば戻ってきます。	いちじかんけいかすればもどってきます 
\\	彼が出発して3時間半が経過した。	かれがしゅっぱつしてさんじかんはんがけいかした 
\\	ピカソのような画家は稀だ。	ピカソのようながかはまれだ 
\\	目が見えない人の聴力は鋭敏な場合が多い。	めがみえないひとのちょうりょくはえいびんなばあいがおおい 
\\	君の指摘、中らずといえども遠からずだね。	きみのしてき、あたらずといえどもとおからずだね 
\\	貧しいといえども彼女は幸せだ。	たとえ~でも  ~けれども
\\	水平線から昇る日の出の美しさといったらない。	本当に~だ
\\	おかしいったらない。あの人、靴をはくのを忘れて、スリッパのまま出かけちゃったんですって。	とても~
\\	彼女はまだ十代に見えたので、独身だと思いきや、実は子供が二ついるそうだ。	【とおもいきや】 ~と思ったのだが、そうではなく
\\	あいつときたら。酒を飲み出すと止まらないんだ。一緒に飲みに行ったら大変だよ。	~は  (「だめだ、あきれた」と不満や非難を言う)
\\	彼の子供の扱い方ときたら酷いものだ。	かれのこどものあつかいかたときたらひどいものだ 
\\	彼の母親は可哀相だ。あいつときたら母親泣かせもいいところだ。	かれのははおやはかわいそうだ。あいつときたらははおやなかせもいいところだ 
\\	あの喜劇役者のジョークときたら、どれもこれも古くて、以前に聞いたことのあるものばかりだ。	あのきげきやくしゃのジョークときたら、どれもこれもふるくて、いぜんにきいたことのあるものばかりだ 
\\	概して私は悲劇よりも喜劇が好きだ。	がいしてわたしはひげきよりもきげきがすきだ 
\\	ご多忙のところを恐縮ですが、ちょっとご相談したいことがありまして・・・。	~ときなのに  ~という状況なのに
\\	彼らは幼児の死亡率の低さは医学の進歩のゆえと考えた。	事の起こるわけ。理由。原因。
\\	彼女は私の提案に反対しているが、彼女にしたところで、よい案があるわけではない。	~も  ~でも  ~の場合も
\\	大企業にしたって、この不況の影響を受けないわけがないだろう。	~も   ~でも   ~の場合も
\\	彼はいわゆる秀才である。	かれはいわゆるしゅうさいである 
\\	携帯電話でテレビ放送まで見られるとは、技術の進歩は大したものだ。	~ということは  ~なんて
\\	あんなことを彼女に言うなんて非常識も甚だしい。	あんなことをかのじょにいうなんてひじょうしきもはなはだしい 
\\	彼が病気だという事実は、彼女にはそれほど大したことではなかった。	程度がはなはだしいさまをいう語。非常な。たいへんな。
\\	そんな発言はいかにも彼らしい。	程度・状態のはなはだしいことを表す。どう考えても。全く。実に。
\\	彼はいかにも「営業」って感じだね。	かれはいかにも「えいぎょう」ってかんじだね 
\\	悩みを笑い飛ばすとはいかにも彼らしい。	なやみをわらいとばすとはいかにもかれらしい 
\\	あんなふうに話すとは彼は阿呆に違いない。	あんなふうにはなすとはかれはあほにちがいない 
\\	土壇場で言葉が旨く言えなかった。	どたんばでことばがうまくいえなかった 
\\	土壇場で踏ん張ってその契約を勝ち取らない限り、我々は破産も同然だ。	どたんばでふんばってそのけいやくをかちとらないかぎり、われわれははさんもどうぜんだ 
\\	土壇場になる前にこのプロジェクトを終わらせたほうがいい。	どたんばになるまえにこのプロジェクトをおわらせたほうがいい 
\\	我々は財政的に困窮していた。要するに破産したのだ。	われわれはざいせいてきにこんきゅうしていた。ようするにはさんしたのだ 
\\	インサイダー取引スキャンダルによって多数の人が破産しました。	インサイダーとりひきスキャンダルによってたすうのひとがはさんしました 
\\	彼は仕事を首になってから困窮している。	かれはしごとをくびになってからこんきゅうしている 
\\	要するに、彼の新しい小説は期待はずれのつまらない作品と言える。	【ようするに】 今まで述べてきたことをまとめれば。かいつまんで言えば。つまり。
\\	彼女は快活で愛想がよく、親切でなおかつ思いやりがある。要するに立派な人です。	かのじょはかいかつであいそうがよく、しんせつでなおかつおもいやりがある。ようするにりっぱなひとです 
\\	要するに、借金を踏み倒して逃げちまった。	ようするに、しゃっきんをふみたおしてにげちゃった 
\\	彼は借金を踏み倒して姿を暗ました。	かれはしゃっきんをふみたおしてすがたをくらました 
\\	群衆に押されないよう力を入れて踏ん張った。	ぐんしゅうにおされないようちからをいれてふんばった 
\\	「ドタキャン」とは、土壇場でキャンセルすること、すなわち直前のキャンセルのことです。	説明・定義するときの用法
\\	薬を服用するときは、瓶に書いてある用法に注意深く従いなさい。	くすりをふくようするときは、びんにかいてあるようほうにちゅういぶかくしたがいなさい 
\\	だが定期的な服用が必需であり、一日でも服用しなければたちまち死に至る。	だがていきてきなふくようがひつじゅだあり、いちにちでもふくようしなければたちまちしにいたる 
\\	ポスターは即刻壁から撤去された。	ポスターはそっこくかべからてっきょされた 
\\	かんかんに怒った社員は、即刻会社を辞めた。	かんかんにおこったしゃいんは、そっこくかいしゃをやめた 
\\	われわれは即刻アメリカにむかって出発しなければならなかった。	【そっこく】 すぐその時。すぐさま。即時。
\\	歩道から自転車を撤去してくれ。	ほどうからじてんしゃをてっきょしてくれ 
\\	あの投手とはどうも相性がよくない。	あのとうしゅとはどうもあいしょうがよくない 
\\	貴社に伺うのに便利な場所の部屋を予約できればありがたいのですが。	きしゃにうかがうのにべんりなばしょのへやをよやくできればありがたいのですが 
\\	私は自分の意図を両親に知らせた。	わたしはじぶんのいとをりょうしんにしらせた 
\\	彼女は彼の真の意図を嗅ぎつけましたね。	かのじょはかれしんのいとをかぎつけましたね 
\\	首相の演説は野党を怒らせようという意図でなされたものだった。	しゅしょうのえんぜつはやとうをおこらせようといういとでなされたものだった 
\\	マスコミが彼の婚約の噂を嗅ぎつけ早速駆けつけた。	マスコミがかれのこんやくのうわさをかぎつけさっそくかけつけた 
\\	休日は、一日中何をするともなく、ブラブラしていることが多い。	特に~しようとおもうのではなく。 「はっきりした意図を持たないで、なんとなく、無意識に」という意味がある。
\\	日本の多くの若者が生活の目的もなくぶらぶらしている。	にほんのおおくのわかものがせいかつのもくてきもなくぶらぶらしている 
\\	彼らはどこからともなく突然現れた。	かれらはどこからともなくとつぜんあらわれた 
\\	音楽の分野では誰もこの若い女性にかなわない。	おんがくのぶんやではだれもこのわかいじょせいにかなわない 
\\	英語の単語を暗記することになると、誰も彼にはかなわない。	えいごのたんごをあんきすることになると、だれもかれにはかなわない 
\\	この辺は静かなところだが、夜ともなると、バイクで若者が集まってくるので、うるさくてかなわない。	~(の)場合は・状態になると
\\	夜ともなれば彼は好奇心を抱いて星空を見上げたこともあろうと思う。	よるともなればかれはこうきしんをだいてほしぞらをみあげたこともあろうとおもう 
\\	同じ世界ながら見る心が違えば地獄ともなれば天国ともなる。	おなじせかいながらみるこころがちがえばじごくともなればてんごくともなる 
\\	私は昨晩警察といざこざを起こした。	わたしはさくばんけいさつといざこざをおこした 
\\	幸せにも彼女はそのいざこざに巻き込まれなかった。	しあわせにもかのじょはそのいざこざにまきこまれなかった 
\\	不況とはいえ、急成長している企業もあるのだ。	~とはいっても、しかし
\\	は・・・、若かったとはいえ、しょうもない凡ミスだ。	は・・・、わかかったとはいえ、しょうもないぼんミスだ 
\\	しかしながら、利用する風の量は場所や季節によって異なる。	しかしながら、りようするかぜのりょうはばしょやきせつによってことなる 
\\	しかしながら、大部分の黒人にとって、本当の変化はやってくるのが極めて遅かった。	しかしながら、だいぶぶんのこくじんにとって、ほんとうのへんかはやってくるのがきわめておそかった 
\\	友人と居酒屋で飲んでいたら、「もう店を閉める。早く帰れ」とばかりに店員が掃除を始めた。	まるで~と言うように
\\	見知らぬ人に吠え掛かるのは多くの犬に共通の習慣です。	みしらぬひとにほえかかるのはおおくのいぬにきょうつうのしゅうかんです 
\\	私は犬に骨をやって、それで吠えるのをやめさせた。	わたしはいぬにほねをやって、それでほえるのをやめさせた 
\\	むしろロン毛のほうが禿げやすいって聞いたぞ。	むしろロンげのほうがはげやすいってみいたぞ 
\\	英国では女王は君臨するが、支配はしない。	えいこくではじょおうはくんりんするが、しはいはしない 
\\	その王は40年間にわたって人民の上に君臨した。	そのおうはよんじゅうねんのわたってじんみんのうえにくんりんした 
\\	試験に失敗したからといって、落ち込んでばかりもいられない。次のチャンスを目指して、また新しい気持ちで頑張ろう。	~だけしているわけにはいけない
\\	彼の罪を大目に見て許してあげたほうがいいのではないか。	かれのつみをおおめにみてゆるしてあげたほうがいいのではないか 
\\	今までは大目に見てきたが、今度ばかりは許さない。警察に言ってやる。	~だけは  ほかの場合と違って~は
\\	自分の家族が殺されたら復讐しないではおかないと彼は言った。	~ずにはおかない  きっと~する  必ず~する (相手に対する強い働きかけがある)
\\	母は困った人を見たら、決して見捨ててはおかない。必ず助けようと力を尽くす人だ。	ておく+ない
\\	彼の気持ちを変えさせるよう、彼に働きかけねばならない。	かれはきもちをかえさせるよう、かれにはたらきかけねばならない 
\\	子供が犯罪したら、親として謝罪しないではすまない。 「子供と親は別の人格」などといってはいられない。	~ないわけにはいかない  ~ないことはできない  ~なければならない
\\	どんな言語を学ぶにしても辞書なしではすまない。	どんなげんごをまなぶにしてもじしょなしではすまない 
\\	パスポートは外国に行ったときなしではすまされないものだ。	パスポートはがいこくにいったときなしではすまされないものだ 
\\	私は日曜までこの電卓なしではすまされない。	わたしはにちようびまでこのでんたくなしではすまされない 
\\	その偉大な画家の傑作が壁にさかさまにかかっているを見て、彼は驚いた。	そのいだいながかのけっさくがかべにさかさまにかかっているをみて、かれはおどろいた 
\\	その大富豪は費用には関係なくその傑作を購入するつもりだった。	そのだいふごうはひようにはかんけいなくそのけっさくをこうにゅうするつもりだった 
\\	その大富豪は貧しい少年として生涯を始めた。	そのだいふごうはまずしいしょうねんとしてしょうがいをはじめた 
\\	母の死は私の生涯に大きな空白を残した。	ははのしはわたしのしょうがいにおおきなくうはくをのこした 
\\	生涯教育は絶え間ない再訓練を意味する。	しょうがいきょういくはたえまないさいくんれんをいみする 
\\	私は自分の理想の追求に生涯を費やそうと決心した。	わたしはじぶんのりそうのついきゅうにしょうがいをついやそうとけっしんした 
\\	その老婆は熱と絶え間ない咳で弱っていた。	そのろうばはねつとたえまないせきでよわっていた 
\\	1人の老婆が不自由な足で通りを歩いていた。	ひとりのろうばがふじゆうなあしでとおりをあるいていた 
\\	恵子が手際よく食器を重ねて、シンクへ運んでゆく。	けいこがてぎわよくしょっきをかさねて、シンクへはこんでゆく 
\\	机の上には漫画本が重ねてあった。	つくえのうえにはまんがほんがかさねてあった 
\\	部屋の隅に本がきちんと積み重ねられていた。	へやのすみにほんがきちんとつみかさねられていた 
\\	彼女は天才とは言えないまでも、相当優秀な人だと評判です。	~ほどではないが  ~とは言わないけれど
\\	僅か6ヶ月後に最初の仕事を止めるとしたら、愚かとは言わないまでも賢いとは言えない。	わずかろっかげつごにさいしょのしごとをやめるとしたら、おろかとはいわないまでもかしこいとはいえない 
\\	君にはまったく往生する。	きみにはまったくおうじょうする 
\\	警察署内での拷問の残酷さは筆舌に尽くし難い。	けいさつしょないでのごうもんのざんこくさはひつぜつにつくしがたい 
\\	東京は、日本一大きい都市だが、24時間目覚めている。	とうきょうは、にほんいちおおきいとしだが、にじゅうじかんめざめている 
\\	いくらお礼を言っても言い切れない。	いくらおれいをいってもいいきれない 
\\	大変な仕事ですが、みんなで力を合わせて一生懸命やれば、できないものでもないから、やってみましょうか。	~とは言い切れない  ~なくはない  ~ないこともない
\\	彼は、涙ながらに苦しかった戦争の思い出を語った。	【なみだながらに】 泣いている状態で
\\	彼は生まれながらの悪人だと言わざるを得ない。	~そのままの変わらない状態
\\	人は誰でも何らかの生まれながらの才能があるものですが、それを生かせるかどうかが問題です。	ひとはだれでもなんらかのうまれながらのさいのうがあるものですが、それをいかせるかどうかがもんだいです 
\\	当時彼女は、何らかの仕事に従事していた。	とうじかのじょは、なんらかのしごとにじゅうじしていた 
\\	ピョンヤンとワシントンの間で何らかの妥協に至ることが不可欠だ。	ピョンヤンとワシントンのあいだでなんらかのだきょうにいたることがふかけつだ 
\\	及ばずながらお手伝いしましょう。	【およばずながら】 不十分ではあるが
\\	私は彼の成功をかげながら応援しています。	目立たずに、そっと
\\	彼は今本を読んでいるからそっとしておこう。	他人に気づかれないように物事をするさま。こっそり。ひそかに。
\\	彼女が許してくれるまでそっとしておいたほうがいいかもね。	かのじょがゆるしてくれるまでそっとしておいたほうがいいかもね 
\\	彼は密かに庭に入り込んだ。	かれはひそかににわにはいりこんだ 
\\	私の知っている最大の喜びは、密かによい行いをして偶然人に知られることである。	わたしのしっているさいだいのよろこびは、ひそかによいおこないをしてぐうぜんひとにしられることである 
\\	あなたは血も涙も無い人ね。	あなたはちもなみだもないひとね 
\\	鬼の目にも涙。	おにのめにもなみだ 
\\	この家は狭いながらも庭もあるし、駅にも近いので、満足しています。	~が  ~しかし  ~けれど
\\	ぜいぜい言いながらも、気合をいれて走り続ける。	ぜいぜいいいながらも、きあいをいれてはしりつづける 
\\	がばっと気合を入れて身を起こした。	がばっときあいをいれてみをおこした 
\\	もっと気合いを入れて仕事しろ。	もっときあいをいれてしごとしろ 
\\	期末試験に備えて本当に気合いを入れて勉強しなきゃ。	きまつしけんにそなえてほんとうにきあいをいれてべんきょうしなきゃ 
\\	明日までに、私は学期末レポートを仕上げることはできない。	あしたまでに、わたしはがっきまつレポートをしあげることはできない 
\\	焦って仕事を仕上げようとすれば、無用な間違いを犯す。	あせってしごとをしあげようとすれば、むようなまちがいをおかす 
\\	ジョンは、その不誠実なセールスマンが、彼を騙して無用な機械を買わせたと主張した。	ジョンは、そのふせいじつなセールスマンが、かれをだましてむようなきかいをかわせたとしゅちょうした 
\\	我々の軍隊に完全に包囲されてしまって敵はとうとう降伏した。	われわれのぐんたいにかんぜんにほういされてしまっててきはとうとうこうふくした 
\\	軍隊は城を何日間も包囲した。	ぐんたいはしろをなんにちかんもほういした 
\\	彼女は少ないながらも持っていた全ての硬貨をそのものもらいに与えました。	かのじょはすくないながらもっていたすべてのこうかをそのものもらいにあたえました 
\\	少し悪いと思いながらも、彼は笑うのを抑えることはできなかった。	すこしわるいとおもいながらも、かれはわらうのをおさえることはできなかった 
\\	彼の活躍なくしてチームの優勝はあり得なかった。流石、キャプテンだ。	~ない場合は  ~がなくては  ~がなければ  ~なしには
\\	たゆまぬ努力が成功の鍵であることは言うまでもない。	たやまぬどりょくがせいこうのかぎであることはいうまでもない 
\\	諸君らのたゆまぬ努力と労働によって、ついに我らがアジトが完成した!!	しょくんらのたゆまぬどりょくとろうどうによって、遂にわれらがアジトがかんせいした!! 
\\	大きな成功は、コツコツ努力した結果である。	おおきなせいこうはコツコツどりょくしたけっかである 
\\	いまだかつて偉大なもので熱烈な精神なくして成し遂げられたものは何もない。	いまだかっていだいなものでねつれつなせいしんなくしてなしとげられたものはなにもない 
\\	彼女は音楽を熱烈に愛好していた。	かのじょはおんがくをねつれつにあいこうしていた 
\\	彼は現代人類学の父として熱烈な支持を受けている。	かれはげんだいじんるいがくのちちとしてねつれつなしじをうけている 
\\	いまだかつて、神を見たものはいない。	今までに一度も。
\\	彼は未だ嘗てないぐらい偉大な物理学者である。	かれはいまだかつてないぐらいいだいなぶつりがくしゃである 
\\	忍耐なくしてはだれも成功することはできない。	にんたいなくしてはだれもせいこうすることはできない 
\\	関係者以外は、許可なしにこの建物に入ることはできません。	~がないと  ~がなければ
\\	野党が次回の選挙で勝つ可能性は無きにしもあらずだと言われている。	【なきにしもあらず】 ないこともない。  少しある。
\\	勇者にあらずんば美女を得ず。	ゆうしゃにあらずんばびじょをえず 
\\	富士山が再び噴火する恐れは無きにしも非ずだから、富士山のふもとには住まないほうがいい。	ふじさんがふたたびふんかするおそれはなきにしもあらずだから、ふじさんのふもとにはすまないほうがいい 
\\	こんな絵が描けるのは子供ならではですね。大人にはとてもこんなえはかけませんね。	~でなくてはできない  ~でなくては(~し)ない  ~以外には(ない)
\\	よっぽど印象に残る事じゃないと覚えてないんだよね。	かなりな程度であるさま。 よほど。
\\	よほど重要な用向きでやってきたことが、彼の顔色で分かった。	かなりな程度であるさま。 よっぽど。
\\	よほど疲れていたのだろう。夫は帰ってくるなり食事もせずに寝てしまった。	~とすぐ  ~や否や  ~(た)とたん
\\	ファン達は有名人と出くわすなり、彼にサインを求めた。	ファンたちはゆうめいじんとでくわすなり、かれにサインをもとめた 
\\	教室に入るなり、先生は突然怒ったような口調で話しはじめた。	きょうしつにはいるなり、せんせいはとつぜんおこったようなくちょうではなしはじめた 
\\	彼の姿を見るなり私は恐怖で震えた。	かれのすがたをみるなりわたしはきょうふでふるえた 
\\	明日我々は敵軍に出くわすだろう。	【でくわす】 偶然に出会う。ばったりと会う。
\\	彼女は行き当たりばったりに棚から一冊の本を取った。	かのじょはいきあたりばったりにたなからいっさつのほんをとった 
\\	彼は横にばったり倒れた。	かれはよこにばったりたおれた 
\\	彼は一週間前に家を出たなり帰ってこない。	~たまま  ~たきり
\\	私でお役に立つことがあったら、何なりとおっしゃってください。	【なんなり】 何でも (あらたまった言い方)
\\	何か困ったことがあったら先生なり、お母さんなりに相談しなさい。	~とか~とか
\\	春になると、冬にできた氷はとけて小川になり、川になり、湖になる。	はるになると、ふやびできたこおりはとけておがわになり、かわになり、みずうみになる 
\\	お酒を飲んでいるうちに、彼はすっかり上機嫌になり、どれほど病院がいやなものかをみんなに何度もしゃべった。	おさけをのんでいるうちに、かれはすっかりじょうきげんになり、どれほどびょういんがいやなものかをみんなになんどもしゃべった 
\\	いじめにあった場合は、すぐ教師なりに相談してください。	~など
\\	お金があれば悩みはないと思っていたが、金持ちは金持ちなりに悩みがあるものらしい。	~にあったように  ~にふさわしく
\\	初心者なりによくやった。	しょしんしゃなりによくなった 
\\	彼は弁護士だからそれなりに対応しなければならない。	その状態のままで変わらないこと。そのまま。それきり。
\\	あの国では、私は外国人だったのでそれなりに扱われた。	あのくにでは、わたしはがいこくじんだったのでそれないにあつかわれた 
\\	彼はわれわれの指導者である、だからそれなりに尊敬しなければならない。	かれはわれわれのしどうしゃである、だからそれなりにそんけいしなければならない 
\\	すべての幸福な家庭という物はお互いに似通っているが不幸な家庭という物はめいめいそれなりに違った不幸があるものだ。	すべてのこうふくなかぞくというものはおたがいににかよっているがふこうなかぞくというものはめいめいそれなりにちがったふこうがるものだ 
\\	彼はすべて妻の言いなりになっている。	【いいなり】 人の言うとおりにすること。
\\	彼は簡単に人の言いなりになるような男ではない。	かれはかんたんにひとのいいなりになるようなおとこではない 
\\	彼の言いなりになるくらいなら、独りで暮らした方がましだ。	かれのいいなりになるくらいなら、ひとりでくらしたほうがましだ 
\\	曲がりなりにも彼は原稿を書き終えた。	【まがりながら】 不完全ながら。どうにかこうにか。
\\	「まがりなりにも通じている」ということと「正しい英語を使っている」ということには雲泥の差があります。	"「まがりなりにもつうじている」ということと「ただしいえいごをつかっている」ということにはうんでいのさがあります 
\\	彼は苦学して曲がりなりにも大学を出た。	かれはくがくしてまがりにもだいがくをでた 
\\	バイトで学費を稼ぎながら大学に通ってる。まあ苦学生ってとこかな。	バイトでがくひをかせぎながらだいがくにかよっている。まあくがくせいってとこかな 
\\	あの人と巡り会えたのは、一期一会なのでしょうか。	あのひととめぐりあえたのは、いちごいちえなのでしょうか 
\\	いろいろなデザインのネクタイが沢山並んでいるけど、これといって買いたいものはない。	特に~ない
\\	未だにその事件は解決していない。	【いまだに】 今になってもまだ~ない。
\\	彼女がどうしてその犯罪に及んだかという動機については、いまだによくわかりません。	かのじょがどうしてそのはんざいにおよんだかというどうぎについては、いまだによくわかりません 
\\	もうとっくに学校へ出かけている時間じゃないの。	ずっと以前。とう。
\\	うちの息子は、大学に入ったのに、ろくに勉強もしないで遊んでばかりいるんですよ。困ったものです。	十分に~しない。  満足に~しない。
\\	一生懸命働いているのに、給料もろくにもらえない。こんな仕事、もう辞めようか。	十分に~ない。  満足に~ない。
\\	そうですね。一概には言えませんね。	そうですね。いちがいにはいえませんね 
\\	日本人は集団で行動するのを好むといわれるが、一概にそうとは言えない。	【いちがい~ない】 必ずしも~ない。  全部が~とは言えない。
\\	ダイエットをしたり、ジムに通ったりしているのに、一向に体重が減らない。	【いっこう~ない】 全然~ない。  いつまでたっても~ない。
\\	彼はサッカー選手としては大して優秀ではない。でも、ハンサムなので人気があるそうです。	そんなに~ない。
\\	ロンドンの空気はどうよくみても大して自慢できるものでなかった。	ロンドンのくうきはどうよくみてもたいしてじまんできるものではなかった 
\\	彼女が二度と口をきいてくれないとしても、それは自業自得さ。	かのじょはにどとくちをきいてくれないとしても、それはじごうじとくさ 
\\	かれは事故を起こして悩んでいるが、全く同情するにあたらない。飲酒運転したんだから、自業自得だ。	~(する)ことはない。  ~(する)必要はない。  ~(する)価値はない。
\\	政治家の汚職なんて今さら驚くにあたらない。	もっと早ければともかく、今となっては遅すぎる、という意を表す。
\\	「女のくせに」と言う発言はセクハラにあたるのだそうです。	~と同じだ。
\\	去年の夏にようやく私の長男は泳げるようになった。	長い間待ち望んでいた事態が遂に実現するさま。やっとのことで。
\\	悪戦苦闘の末、ようやく彼は事業を立ち直らせた。	あくせんくとうのすえ、ようやくかれはじぎょうをたちなおらせた 
\\	彼は、私が2時間も悪戦苦闘した問題を5分で解いてしまった。	【あくせんくとう】 困難な状況の中で、苦しみながら努力すること。
\\	トムはもう3時間もの間、眠りにつこうと悪戦苦闘しています。	トムはもうさんじかんものかん、ねむりつこうとあくせんくとうしています 
\\	幼児期にあって心を受けた傷は、生涯トラウマとして心に残り続けるそうだ。	~において  ~の状態で
\\	年金生活者が厳しい生活を強いられているのは事実だ。	ねんきんせいかつしゃがきびしいせいかつをしいられているのはじじつだ 
\\	エジプトは混乱期にあり、国民は苦労の多い生活を強いられている。	~にあって
\\	お心遣い重ねて感謝します。	おこころづかいかさねてかんしゃします 
\\	東京電力の清水社長が13日に会見し、福島第1原発の事故について重ねて謝罪した。	【かさねて】 同じことを繰り返すさま。もう一度。再び。
\\	ジョンソン氏は自営業で、家具の修理をやっている。	ジョンソンしはじえいぎょうで、かぐのしゅうりをやっている 
\\	私の乗った飛行機が無事に着陸してほっとした。	緊張がとけて、安心するさま。
\\	自営業から大企業に至るまで、すべての商業が不況に陥った。	【にいたるまで】 ~まで行く  ~まで及ぶ
\\	重要なのはゴールではなく、そこに至る道程である。	じゅうようなのはゴールではなく、そこにいたるみちのりである 
\\	昔から今に至るまで存在する、あらゆる社会の歴史は階級闘争の歴史である。	むかしからいまにいたるまでそんざいする、あらゆるしゃかいのれきしはかいきゅうとうそうのれきしである 
\\	近代医学の進歩は長い道程を歩んだ。	きんだいいがくのしんぽはながいみちのりをあゆんだ 
\\	論点の中心は、近代化という問題である。	ろんてんのちゅうしんは、きんだいかというもんだいである 
\\	火事は大事に至らず鎮火した。	かじはだいじにいたらずちんかした 
\\	彼らは鎮火するために互いに協力した。	【ちんか】 火事が消えること。火事を消すこと。
\\	高級旅館に泊まれば。細かいところにまで気を配った至れり尽くせりのサービスが受けられる。	【いたれりつくせんり】 注意がとてもよく行き届いている。
\\	その庭は手入れが行き届いている。	そのにわはていれがいきとどいている 
\\	首脳会議の最終日に至って、やっと結論が出た。	【いたって】 (その段階・状況)になって
\\	はっきりした証拠が、裁判所に提出された。彼の罪は明白だ。しかし、この場に至っても、彼はまだ真実のこと話そうとしない。	【いたっても】 (その段階・状況)になっても
\\	彼がそんなに早く来た理由は明白ではない。	【めいはく】 あきらかで疑う余地のないこと。
\\	人間至る所青山有り。	にんげんいたるところせいざんあり 
\\	道はここから緩やかな上りだ。	みちはここからゆるやかなあがりだ。 
\\	川幅が広くなるにつれて、水流はいっそう緩やかになった。	かわはばがひろくなるにつれて、すいりゅういっそうゆるやかになった。 
\\	ゆったり座って休みなさい。そうすればずっと気分が良くなるでしょう。	落ち着いてのんびりしているさま。
\\	会社の信用にかかわることは絶対にしてはいけないと、社長に言われています。	~に関係する  ~影響がある
\\	これは我が校の名誉にかかわる問題だ。	これはわがこうのめいよにかかわるもんだいだ。 
\\	よく注意して運転しないと、命にかかわるよ。	よくちゅういしてうんてんしないと、いのちにかかわるよ 
\\	酸素が不足することは大部分の動物にとって命にかかわることである。	さんそがふそくすることはだいぶぶんのどうぶつにとっていのちにかかわることである 
\\	あんな悪人とは、かかわり合いにならないほうがいいですよ。	【かかわりあいにならない】 関係を持たない。
\\	彼は街頭で買った絵を額縁に入れた。	かれはがいとうでかったえをがくぶちにいれた 
\\	混雑する街頭で数人の男性が喧嘩をしている。通行人はかかわり合いになるのを恐れて、急いで通り過ぎていく。	【かかわりあいになる】 関係を持つ。
\\	原子力発電所でまた事故があった、周辺住民の不安は想像にかたくない。	容易に~できる  簡単に~できる
\\	評判で察すると、彼女はその仕事に相応しいようだ。	ひょうばんでさっすると、かのじょはそのしごとにふさわしいようだ 
\\	彼女の表情から怒っているのを察した。	かのじょのひょうじょうからおこっているのをさっした 
\\	彼はノーベル賞に値する研究を成し遂げた。これは、偉大な科学者にして初めてできることであろう。	~だけ  ~だからこそ
\\	あの人、ベテランなんですか。それにしては、下手ですね。	~なのに  ~だけれど(普通考えることと反対)
\\	この高級車は中古にしてこの値段だ。新車なら、3,000万ぐらいするそうだ。	~でも  ~なのに  ~さえ
\\	いつも自分以外の人の気持ちを思いやらなくてはならない。	いつもじぶんいがいのひとのきもちをおもいやらなくてはならない 
\\	他人を思いやる気持ちが大切です。	【おもいやる】 他人の身の上や心情を推し量って、同情する。また、配慮する。
\\	彼は即時断行を強調した。	かれはそくじだんこうをきょうちょうした 
\\	国の作成したカリキュラムは古い、時代に即した新しいカリキュラムを作成しなければならない、と教師達は主張している。	【にそくした】 ~の通り  ~に沿った  ~に会う
\\	シンガポールでは罪人を鞭打ちで処罰する。	シンガポールではざいにんをむちうちでしょばつする 
\\	ジムは私を侮辱したから鞭で叩いてやる必要がある。	ジムはわたしをぶじょくしたからむちでたたいてやるひつようがある 
\\	この規則を破った人は厳しい処罰を免ぜられない。	このきそくをやぶったひとはきびしいしょばつをめんぜられない 
\\	犯罪は法律に則して処罰される。	【にそくして】  基準に従って
\\	私の判断の基準は楽しく働く能力だ。	わたしのはんだんのきじゅんはたのしくはたらくのうりょくだ 
\\	私たちの家は西欧の基準から見ると小さすぎるとのことです。	わたしたちのいえはせいおうのきじゅんからみるとちいさすぎるとのことです 
\\	相手の基準を受け入れるのは、その相手の力に服従することだ。	あいてのきじゅんをうけいれるのは、そのあいてのちからにふくじゅうすることだ 
\\	観光客の服装は地元の礼儀正しさの基準にかなっていない。	かんこうきゃくのふくそうはじもとのれいぎただしさのきじゅんにかなっていない 
\\	この車はそんな目的にかなうように作られている。	条件・基準などによく当てはまる。ぴったり合う。適合する。
\\	私は自分の計画をその新しい事態に適合させた。	わたしはじぶんのけいかくをそのあたらしいじたいにてきごうさせた 
\\	これは素人が書いた小説だが、十分読むにたえる作品だ。	~(の)価値がある  ~(が)できる
\\	彼は、芸術に対する鑑賞力がある。	かれは、げいじゅつにたいするかんしょうりょくがある 
\\	バッハの作品は万人の鑑賞にたえる名曲です。	【かんしょう】 芸術作品などを見たり聞いたり読んだりして、それが表現しようとするところをつかみとり、そのよさを味わうこと。
\\	その画廊では多くの古い名画を展示することになっている。	そのがろうではおおくのふるいめいがをてんじすることになっている 
\\	安物買いの銭失い。	やすものがいのぜにうしない 
\\	これは安物のアクセサリーです。人前でつけるにたえないものですね。	~できない  がまんできない
\\	いやしくも勉強をするなら、一生懸命やりなさい。	いやしくもべんきょうをするなら、いっしょうけんめいやりなさい 
\\	いやしくもなすに足る事なら立派にやるだけの価値がある。	いやしくもなすにたることならりっぱにやるだけのかちがある 
\\	仮にも為す価値があるものならば何でも良く為す価値がある。	かりにもなすかちがあるものならばなんでもよくなすかちがある 
\\	為すことができる者が為し、為すことができない者が教える。	なすことができるものがなし、なすことができないものがおしえる 
\\	仮にも彼に成功したい気があるのなら、もっとせっせと働かなければならない。	かりにもかれはせいこうしたいきがあるのなら、もっとせっせととはたらかなければならない 
\\	脇目も振らずに。	わきめもふらずに 
\\	彼はせっせと試験準備をしている。	わきめもふらず、熱心に物事をするさま。
\\	台風で木の実がすっかり落ちた。	たいふうできのみがすっかりおちた 
\\	彼女はせっせと編み物に精を出していた。	かのじょはせっせとあみものにせいをだしていた 
\\	この農園は私たちの必要を満たすに足るだけの野菜を産出する。	こののうえんはわたしたちのひつようをみたすにたるだけのやさいをさんしゅつする 
\\	それはより多くの二酸化炭素の産出につながり、それが世界的な温暖化の原因になる主な気体なのだ。	それはよりおおくのにさんかたんそのさんしゅつにつながり、それがせかいてきなおんだんかのげんいんになるしゅなきたいなのだ 
\\	新しい調査では、65歳以上の病院患者の診察記録には誤りが多く、重大な診察ミスにつながりかねない、ということです。	あたらしいちょうさでは、ろくじゅうさいいじょうのびょういんかんじゃのしんさつきろくにはあやまりがおおく、じゅうだいなしんさつミスにつながりかねない、ということです 
\\	熱がこの化学薬品を無害な気体に分解するだろう。	ねつがこのかがくやくひんをむがいなきたいにぶんかいするだろう 
\\	多くの画期的な発明をした彼は、尊敬に足る人物である。	十分ある  ~(する)価値がある
\\	八方美人頼むに足らず。	はっぽうびじんたのむにたらず 
\\	反抗はいつの世でも金になる。	はんこうはいつのよでもかねになる 
\\	「敵の軍隊など恐れるに足らず、我々には神がついている」と指導者はいつの世も国民に言うものらしい。	【にたらず】 ~必要はない
\\	公立の学校にひきかえ、私立の学校は学費がたかすぎる。	~に比べて  ~にちがって
\\	わが社の業績は去年にもまして悪化してきた。	~以上
\\	彼は何物にもまして名誉を重んじる。	かれはなにものにもましてめいよをおもんじる 
\\	大衆によって重んじられている人が必ずしもそれを受けるに足る人とは言えない。	たいしゅうによっておもんじられているひとがかならずしもそれをうけるにたるひととはいえない 
\\	今日は台風が近づいているので、かなり波が荒いですよ。泳げないことはないんですが、危険ですから、十分気をつけてください。	「~ない」とは言えないが、でも
\\	過去問がもしあるならば一応目を通すほうがいいと思います。	かこもんがもしあるならばいちおうめをとおすほうがいいとおもいます 
\\	兄は出社前に新聞にざっと目を通すことが習慣になっている。	あにはしゅっしゃまえにしんぶんにざっとめをとおすことがしゅうかんになっている 
\\	読むには読んだんですが、半分ぐらいしか理解できなくて・・・・・・。	一応~する(した)、でも
\\	社長のお宅のパーティーに招待していただき、光栄の至りでございます。	とても~  最高の~  ~の結果
\\	試験中カンニングしているところを見つけられた大学生の中には、赤面さえしないものもいる。	しけんちゅうカンニングしているところをみつけられただいがくせいのなかには、せきめんさえしないものもいる 
\\	彼女は友人の結婚披露宴ですばらしい挨拶を述べた。	かのじょはゆうじんのけっこんひろうえんですばらしいあいさつをのべた 
\\	結婚式の後で盛大な披露宴が催された。	けっこんしきのあとでせいだいなひろうえんがもよおされた 
\\	聴衆は、その歌手に盛大な拍手を送った。	ちょうしゅうは、そのかしゅにせいだいなはくしゅをおくった 
\\	家族一同、おかげさまでいたって元気にしております。	程度のはなはだしいさま。きわめて。非常に。
\\	大工が息子に、鶏小屋を造ってやるのは、いたって簡単なことである。	【だいく】 主として木造の家屋などを建てたり、修理したりする職人。また、その仕事。
\\	その政治家は選挙における支持に対して一同に大いに感謝している、と述べた。	【いちどう】 そこにいる人々全部。または、仲間の者全体。
\\	彼らは路上で1ペニー硬貨を奪い合った。	かれらはろじょうでいちペニーこうかをうばいあった 
\\	路上で暮らすホームレスを見るたびに「寂しさの極み」という言葉が頭に浮かぶ。	一番~  最高に~
\\	彼女の夫はひどい麻薬依存症だ。	かのじょのおっとはひどいまやくいぞんしょうだ 
\\	男性はおろか女性もアルコール依存症になる。	~はもちろん  ~どころか  はいうまでもなく
\\	私の連れは疲れすぎていて走るのはおろか、歩くこともできない、と言った。	わたしのつれはつかれすぎていてはしるのはおろか、あるくともできない、といった 
\\	その農民は稲が実る日を待ち望んでいる。	そののうみんはいながみのるひをまちのぞんでいる 
\\	ドイツに滞在した1年間は非常に実りの多いものだった。	ドイツにたいざいしたいちねんかんはひじょうにみのりのおおいものだった 
\\	次の20年が私たち皆にとってさらに実り多いものになりますように。	つぎのにじゅうねんがわたしたちみんなにとってさらにみのりおおいものになりますように 
\\	「雨が嫌い」なんて言ってはいけません。雨が降ればこそ作物も実るのですから。	~からこそ
\\	愛すればこそ彼女と結婚しなかった。	あいすればこそかのじょとけっこんしなかった 
\\	彼女らの妬みは彼女の美しさに向けられた。	かのじょらのねたみはかのじょのうつくしさにむけられた 
\\	誰かが「アフリカでは空はいつも晴れているということだよ」と妬ましそうに言った。	"だれかが「アフリカではそらはいつもはれているということだよ」とねたましそうにいった 
\\	僕がどんな仕事をしても、同僚は妬みこそすれ、ほめることはない。	【ねたみこそすれ、ほめることはない】 妬むだけで、ほめない
\\	洋服店を開こうというトムの努力は水の泡になった。	【みずのあわ】 努力・苦心がすべてむだになること。
\\	どんなによく勉強しても、試験開始時間に遅れてしまえばそれまでだ。それまでの努力は水の泡になる。	~ば、それで終わりだ   ~場合はだめだ
\\	「騙された」と言ってしまえばそれまでですが、「うまい話」に乗った当人であることは間違いありません。	"「だまされた」といってしまえばそれまでですが、「うまいはなし」にのったとうにんであることはまちがえません 
\\	雨が降ったらそれまでだ。試合は中止せざるを得なくなる。	~ばそれまでだ
\\	ワインを保管する際は、ひとり温度のみならず、湿度にも気を配らねばならない。	単に~だけでなく
\\	「ゴミを捨てるべからず」「たばこのポイ捨てをするべからず」という看板が駅前に立っている。	~してはいけない
\\	流言飛語に惑わされるべからず、とはいうものの、言うは易く行うは難し、と思わない?	りゅうげんひごにまどわされるべからず、というものの、いうはやすくおこなうはむずかし、とおもわない? 
\\	公園の掲示に「芝生に入るべからず」と書いてあった。	"こうえんのけいじに「しばふにはいるべからず」とかいてあった 
\\	後期逸すべからず。	こうきいっすべからず 
\\	彼女の行動は常軌を逸したものだ。	かのじょのこうどうはじょうきをいっしたものだ 
\\	彼の後期の絵はすべて傑作とみなされた。	かれのこうきのえはすべてけっさくとみなされた 
\\	動議が何であれ、殺人は許すべからざる行為である。	~ことができない~   ~てはいけない~
\\	30階建の超高層ビルが突然爆発炎上した。	さんじゅうかいだてのちょうこうそうビルがとつぜんばくはつえんじょうした 
\\	彼は幸運にも炎上しているビルから救出された。	かれはこううんにもえんじょうしているビルからきゅうしゅつされた 
\\	彼ほどの確立した科学者がそのような初歩的な誤りを犯すとは実際上考えられないことである。	かれほどのかくりつしたかがくしゃがそのようなしょほてきなあやまりをおかすとはじっさいじょうかんがえられないことである 
\\	水は生きるうえで欠くべからざるものだ。	みずはいきるうえでかくべからざるものだ 
\\	実際にこれらの目標を実現するのに欠くべからざる役割を果たしてきた。	じっさいにこれらのもくひょうをじつげんするのにかくべからざるやくがいやくわりをはたしてきた 
\\	毎日怠けず仕事をするべし。	~しなさい
\\	来年イギリスに留学をすべく、現在準備をしている。	~ために  ~(する)つもり  ~(し)ようと思って
\\	このお支払いの問題が解決できるよう、ご指摘の件を改善すべく努力いたします。	このおしはらいのもんだいがかいけつでくるよう、ごしてきのけんをかいぜんすべくどりょくいたします 
\\	彼らは男を逮捕すべく追いかけた。	【おいかける】 先に行くものに追いつこうとして、あとから追う。おっかける。
\\	無断で会議に欠席するなど、会社員としてあるまじき行為だ。	~してはならない  ~べきではない
\\	彼には彼に仕える召し使いが3人いた。	かれにはかれにつかえるめしつかいがさんにんいた 
\\	彼は年輩の婦人に仕えている。	かれはねんぱいのふじんにつかえている 
\\	すまじきものは宮仕え。	すまじきものはみやづかえ 
\\	人質にとられながら、彼は努めて勇敢に振る舞った。	ひとじちにとられながら、かれはつとめてゆうかんにふるまった 
\\	人質は無事全員解放された。	ひとじちはぶじぜんいんかいほうされた 
\\	私達のようきゅうに応じなければ、人質を殺すまでだ」とテロリストは言った。	~だけだ
\\	「強盗を捕まえてくれて本当にありがとう」 「警察官として当然のことをしたまでのことだから、お礼には及びません」	~だけのことだ
\\	あれは、わざわざ映画館まで見に行くまでもないつまらない映画です。	~ことはない  ~(する)必要はない
\\	論じるまでもなく基本的人権は尊重されなければならない。	ろんじるまでもなくきほんてきじんけんはそんちょうされなければならない 
\\	「冬のソナタ」を例に挙げるまでもなく、いまや時ならぬ韓流ブームである。	"「ふゆのソナタ」をれいにあげるまでもなく、いまやときならぬかんりゅうブームである 
\\	彼の時ならぬ発言は秘密を漏らしたばかりではなく、平和運動の計画をも、覆してしまった。	かれのときならぬはつげんはひみつももらしたばかりではなく、へいわうんどうのけいかくをも、くつがえしてしまった 
\\	最高裁は原判決を覆しました。	さいこうさいはげんはんけつをくつがえしました 
\\	この本物の宝石とその偽物とを比較してみなさい。	このほんもののほうせきとそのにせものとをひかくしてみなさい 
\\	雨の中での試合となり、子供達のユニフォームは泥まみれだった。	~だらけ  ~がいっぱい付いている
\\	そのナイフは血に塗れていた。	そのナイフはちにまみれていた 
\\	彼の袖が油まみれの鍋に触れた。	かれのそでがあぶらまみれのなべにふれた 
\\	難民たちは汗と泥にまみれて、難民キャンプにたどりついた。	~まみれになって(~する)
\\	そこにたどりつく方法はない。	尋ね求めながら、やっと目的地に行き着く。また、苦労のすえに、やっと行き着く。
\\	彼は重傷を負っていたけれども、何とか電話までたどりつくことができた。	かれはじゅうしょうをおっていたけれども、なんとかでんわまでたどりつくことができた 
\\	毎年3月には花が咲き始めて、山は春めく。4月になると、すっかり春になる。	~的になる  ~の様子が現れる  ~の感じが出てくる
\\	静かだった会場が、その発言でざわめいた。	ざわざわした。  騒がしくなった。
\\	重大事件発生の連絡を受けて、新聞記者は色めきたった。	【いろめきたつ】 緊張・興奮した様子になった。
\\	この推理小説はまだ半分ぐらいしか読んでいない。	このすいりしょうせつはまだはんぶんぐらいしかよんでいない 
\\	彼は殺人推理小説を異常な程好む。	かれはさつじんすいりしょうせつをいじょうなほどこのむ 
\\	駅前にできたホテルは便利さもさることながら、眺めも素晴らしいので人気があるそうだ。	~も、もちろんが
\\	失業問題もさることながら、環境問題も大切だ。	しつぎょうもんだいもさることながら、かんきょうもんだいもたいせつだ。 
\\	彼ら私たちを未熟者として軽蔑する。	かれらはわたしたちをみじゅくしゃとしたけいべつする 
\\	こんな未熟な親が出産、子育てなどするのが間違いだったんだ。	こんなみじゅくなおやはしゅっさん、こそだてなどするのがまちがいだったんだ 
\\	船長は無線通信士に遭難信号を打つように命令した。	せんちょうはむせんつうしんしにそうなんしんごうをうつようにめいれいした 
\\	悪天候の影響もさることながら、登山者の未熟さが遭難を引き起こした。	【そうなん】 災難に出あうこと。特に、登山や航海などで命を失うような危険にあうこと。
\\	電話で予約すれば簡単にすむものを、彼ははわざわざ店まで行ったんだって。	~のに
\\	一族が苦労して積み上げてきたものを、お前は一瞬で無駄にするつもりか。	いちぞくがくろうしてつみあげてきたものを、おまえはいっしゅんでむだにするつもりか 
\\	私たちの世界をもっとよいものにするように努めよう。	完全にする
\\	最近、我が社の機器を最新のものにした。	さいきん、わがしゃのききをさいしんのものにした 
\\	日本は北方領土を日本固有のものにしたがっている。	にほんはほっぽうりょうどをにほんこゆうのものにしたがっている 
\\	宇宙旅行はそのうち珍しくないものになるだろう。	完全になる
\\	その国では、何をするにも親の地位がものをいうのです。	結果を発揮する
\\	この業界では金がものをいうんだ。	このぎょうかいではかねがものをいうんだ。 
\\	当人の終始変わらない努力が最後にものを言うのである。	とうにんのしゅうしかわらないどりょくがさいごにものをいうのである 
\\	私服警官が終始大統領の身辺を固めていた。	しふくけいかんがしゅうしだいとうりょうのしんぺんをかためていた 
\\	植物の種子は始終呼吸している。	しょくぶつのしゅしはしじゅうこきゅうしている 
\\	彼が始終文句を言うので私の怒りが爆発した。	かれがしじゅうもんくをいうのでわたしのいかりがばくはつした 
\\	ジョニーは46年間りんごの種子を蒔き付けた。	ジョニーはよんじゅうろくねんかんりんごのしゅしをまきつけた 
\\	農民が穀物の種を蒔いた。	のうみんがこくもつのしゅをまいた 
\\	乗客が一斉に乗ってきた。	【いっせい】 同時にそろって何かをすること。同時。いちどき。
\\	教会の鐘が一斉に鳴り出した。	きょうかいのかねがいっせいになりだした 
\\	試験開始のベルが鳴るや否や、学生は一斉に答えを書き始めた。	【やいなや】 ~とすぐ  ~と同時に  ~(する)か~ないかのうちに
\\	彼と近づきになるやいなや、私は彼をよく知るようになろうと決心した。	かれとちかづきになるやいなや、わたしはかれをよくしるようになろうとけっしんした 
\\	彼は帰宅するやいなや、誇らしげに衝撃の発表をした。	かれはきたくするやいなや、ほこらしげにしょうげきのはっぴょうをした 
\\	彼女の顔は誇らしさで赤く上気していた。	かのじょのかおはほこらしさであかくじょうきしていた 
\\	彼は子供の時に盗みをしたことがあった。それは貧しさゆえの過ちだった。	~だから  ~という理由での
\\	太陽がまともに照り付けている。	じかに。 まっすぐに。 面と向かって。
\\	僕の部屋は西日をまともにうける。	ぼくのひゃはにしびをまともにうける 
\\	彼に嘘などつかなければよかった。二度と彼の顔をまともに見られない。	かれにうそなどつかなければよかった。にどとかれのかおをまともにみられない 
\\	その起源ゆえに、カナダ英語にはアメリカ英語とイギリス英語の両方の特徴がある。	そのきげんゆえに、カナダえいごにはアメリカえいごとイギリスえいごのりょうほうのとくちょうがある 
\\	それゆえに一年の残りの期間は閉鎖される事になるだろう。	それゆえにいちねんののこりのきかんはへいさされることになるだろう 
\\	みなさんはある種の特権を持つゆえに相応の責任もあります。	みなさんはあるしゅのとっけんをもつゆえにそうおうのせきにんもあります 
\\	君はこのまま収入不相応な暮らしを続ければ金に困って身動きがとれなくなるだろう。	きみはこのまましゅうにゅうふそうおうなくらしをつづければかねにこまってみうごきがとれなくなるだろう 
\\	その職に就く為にはそれ相応の資格が必要です。	そのしょくにつくためにはそれそうおうのしかくがひつようです 
\\	保守的な人はすぐ分相応に振る舞えという。	ほしゅてきなひとはすぐぶんそうおうにふるまえという 
\\	我思う故に我あり。	われおもうゆえにわれあり 
\\	山高きが故に貴からず。	やまたかきがゆえにたっとからず 
\\	甘い物は別腹。	あまいものはべつばら 
\\	悪人でも手柄は認めてやれ。	あくにんでもてがらはみとめてやれ 
\\	リンドバーグの大西洋横断無着陸単独飛行はめざましい手柄であった。	リンドバーグのたいせいようおうだんむちゃくりくたんどくひこうはめざましいてがらであった 
\\	日本は戦後目覚ましい産業の進歩をとげた。	【めざましい】 目が覚めるほどすばらしい。驚くほどすばらしい。
\\	不幸は決して単独では来ない。	ふこうはけっしてたんどくではこない 
\\	単独で太平洋を航海するのは勇気を要した。	たんどくでたいへいようをこうかいするのはゆうきをようした 
\\	肩関節は、単独で動くほか、腕の動きにも連動して動きます。	かたかんせつは、たんどくでうごくほか、うでのうごきにもれんどうしてうごきます 
\\	さまざまな辛苦を乗り越え、彼は南極大陸単独横断を成し遂げた。	さまざまなしんくをのりこえ、かれはなんきょくたいりくたんどくおうだんをなしとげた 
\\	彼のお母さんはそのことで絶えず愚痴をこぼしています。	かれのおかあさんはそのことでたえずぐちをこぼしています 
\\	鬼の首を取ったよう。	【おにのくびをとったよう】 大変な手柄を立てたかのように得意になるさま。
\\	壁に耳あり障子に目あり。	【かべにみみありしょうじにめあり】 どこでだれが聞いているか分からないということ。秘密が漏れやすいことのたとえ。
\\	蛙の面に小便。	【かえるのつらにしょうべん】 蛙は顔に水をかけられても平気でいることから、どんな仕打ちをされても、何も感じず平気でいること。
\\	船頭多くして船山に上る。	せんどうおおくしてふねやまにのぼる 
\\	石の上にも三年。	【いしのうえにもさんねん】 冷たい石の上でも三年も座り続ければ暖まる。どんなに辛くても辛抱すれば、やがて成功する。
\\	鬼の居ぬ間に洗濯。	【おにのいぬまにせんたく】 気兼ねする人やこわい人のいない間に、したいことをしたり、息ぬきしたりすること。鬼の留守に洗濯。
\\	彼は袋を背負っている。	かれはふくろをせおっている 
\\	人生は重い荷を背負って遠い道を行くようなものだ。	じんせいはおもいにをせおってとおいみちをいくようなものだ 
\\	目糞鼻屎を笑う。	めくそはなくそをわらう 
\\	柳に雪折れなし。	やなぎにゆきおれなし 
\\	神も仏もない。	かみもほとけもない 
\\	番茶も出花。	ばんちゃもでばな 
\\	豆腐の角に頭をぶつけて死ね。	とうふのかどにあたまをぶつけてしね 
\\	「金利が安い今をおいて、家を買うことはできませんよ」とセールスマンが客を説得している。	~を除いて  ~以外に
\\	彼は遅刻したので、われわれは彼をおいて出発した。	かれはちこくしたので、われわれはかれをおいてしゅっぱつした 
\\	人は、その身分によって、判断されるべきではない。	ひとは、そのみぶんによって、はんだんされるべきではない 
\\	これは身分不相応な贅沢なものだ。	これはみぶんふそうおうなぜいたくなものだ 
\\	江戸時代においては、身分の違うものは結婚することができなかった。	~に  ~で (時代・場所)
\\	この数年は異常高温が続いているが、今年の夏はことに暑く、熱射病による死者もたくさん出た。	~とくに
\\	その映画の主人公はこともなげに人を殺す。子供には見せたくない映画だ。	平気で。 とても簡単だと言う様子で。
\\	あんな物を投げ下ろせば、運悪く当たった人は、怪我をする。	あんなものをなげおろせば、うんわるくあたったひとは、けがをする 
\\	妹を受験勉強を励んでいたが、こともあろうに入学試験の日に高熱を出して、試験を受けられなくなってしまった。	運悪く
\\	こともあろうに、なぜ彼は誕生日に帽子なんかくれたんだろう。私は帽子をかぶらないのに。	こともあろうに、なぜかれはたんじょうびにぼうしなんかくれたんだろう。わたしはぼうしをかぶらないのに 
\\	ひょっとしたら彼が産業スパイではないかと思い浮かんだ。	もしかしたら。
\\	しまった!ウッカリして、携帯電話をお店に忘れてきちゃった。	ぼんやりして注意が行き届かないさま。
\\	彼らはひょっとすると来るかもしれないし、来ないかもしれない。	かれらはひょっとするとくるかもしれないし、こないかもしれない 
\\	彼が結局足をまた使えるようになる可能性は、もしかするとあるかも知れない。	かれはけっきょくあしをまたつかえるようになるかのうせいは、もしかするとあるかもしれない 
\\	検査の結果はまだ出ていませんが、ことによると手術するかもしれません。	もしかすると。 ひょっとすると。
\\	ことによると父は次の列車で帰るかもしれません。	ことによるとちちはつぎのれっしゃでくるかもしれません 
\\	私は父の本棚にある本はどれもみんな読みました。	わたしのちちのほんだなにあるほんはどれもみんなよみました 
\\	最近うちの子は親の言うことにことごとく反抗する。反抗期なのだろうか。	ぜんぶ。  どれもみんな。
\\	部長は今年七月を限りに退職なさるそうです。	~を最後に  ~の限界まで
\\	考えたんだけど、やっぱりこういうのはよくないから、今日を限りに別れよう。	かんがえたんだけど、やっぱりこいうのはよくないから、きょうをかぎりにわかれよう 
\\	あらん限りの力を尽くしたが、戦争は防げなかった。	【あらんかぎり】 あるだけ全部の
\\	医者はあらん限りの手段を尽くして患者の命を救おうとした。	いしゃはあらんかぎりのしゅだんをつくしてかんじゃのいのちをすくおうとした 
\\	暴風雨があらん限りの猛威を振るった。	ぼうふううがあらんかぎりのもういをふるった 
\\	暴風雨のため道路工事が中止された。	【ぼうふうう】 激しい風を伴った雨。台風や発達した低気圧によって起こる。あらし。
\\	これらの船は数日間低気圧の中にいることができるくらい速く走れる。	【ていきあつ】 周囲よりも気圧が低いこと。また、その領域。
\\	君は危険な領域に踏み込んでいるよ。	きみはきけんなりょういきにふみこんでいるよ 
\\	世論は政治の領域で重要な役割を演ずる。	【りょういき】 ある力・作用・規定などが及ぶ範囲。また、その物事・人がかかわりをもつ範囲。
\\	民間機が軍事的領域を侵犯したとのことです。	みんかんきがぐんじてきりょういきをしんぱんしたとのことです 
\\	インフルエンザが都市部で猛威を振るった。	インフルエンザがとしぶでもういをふるった 
\\	この空模様では、おそらく明日の今ごろは台風が猛威をふるっているだろう。	このそらもようでは、おそらくあしたのいまごろはたいふうがもういをふるっているだろう 
\\	それを皮切りとして欧州の詩や文学を多数紹介するようになりました。	それをかわきりとしておうしゅうのしやぶんがくをたすうしょうかいするようになりました 
\\	かわりに「ある程度」「多くの」「しばしば」という言葉を使い、「私の経験では」「間違っていたら申し訳ないのですが」「多くの例では」のような言葉を皮切りに話を始めなさい。	"かわりに「あるていど」「おおくの」「しばしば」ということばをつかい、「わたしのけいけんでは」「まちがっていたらもうしわけないのですが」「おおくのれつでは」のようなことばをかわきりにはなしをはじめなさい 
\\	災い転じて福となせ。	わざわいてんじてふくとなせ 
\\	私はあなたが国債に投資なさる事をお勧めします。	わたしはあなたがこくさいにとうしなさることをおすすめします 
\\	彼のコンサートは東京を皮切りにして、全国10都市で行われる予定です。	【をかわきりに】 ~から始まって
\\	僕だったらそんな図々しいことは言えない。	【ずうずうしい】 恥を知らない。厚かましい。
\\	彼はスキャンダルにも、批判にも平気な顔をしている。あのぐらい面の皮が厚くなければ、政治家にはならないのだろう。	【つらのかわがあつい】 恥を恥とも思わない。ずうずうしい。厚かましい。
\\	持てば持つほど、欲張りになる。	【よくばり】 欲が深いこと。また、その人。
\\	シャイロックは欲張りで、更に悪いことには大変なけちだ。	シャイロックはよくばりで、さらにわるいことにはたいへんなけちだ 
\\	「私の物は私の物。あなたの物も私の物」だって?なんて欲の皮が突っ張っているんだ。	【よくのかわがつっぱっている】 よくばりだ。
\\	突然の解雇に、社員はいかりを禁じえないでいる。	【をきんじえない】 ~をおさえることができない
\\	なにも彼の過去のミスを償うことはできないでしょう。	なにもかれのかこのミスをつぐなうことはできないでしょう 
\\	私たちがおかけしたご苦労に対してどうしたら償いができるでしょう。	わたしたちがおかけしたごくろうにたいしてどうしたらつぐないができるでしょう 
\\	失われた幸福は金では償えない。	うしなわれたこうふくはかねではつぐなえない 
\\	被害者たちにはけがの補償を受ける権利がある。	ひがいしゃたちにはけがのほしょうをうけるけんりがある 
\\	政府は作物が受けた被害に対して農民に補償した。	せいふはさくぶつがうけたひがいにたいしてのうみんにほしょうした 
\\	「あなたの罪は、死をもって償うべきです」と裁判官は犯人に死刑を宣告した。	~で  ~によって
\\	私はどうして彼らがレストランを閉店せざるをえなかったのかよくわからない。	わたしはどうしてかれらがレストランをへいてんせざるをえなかったのかよくわからない 
\\	本日の営業は9時をもって閉店とさせていただきます。	~で  ~に
\\	体の障害をものともせずに最後まで走り抜く選手達を見て、私はとても感動した。	~に負けずに立ち向かって
\\	息子は試合前、「あんな弱いチームなんか、ものの数ではない。あっという間に負かしてやる」と言っていたのに、結局負けてしまった。	【もののかずではない】 たいしたことはない。
\\	その校舎は夕暮れの中にきらきらと輝いていた。	そのこうしゃはゆうぐれのなかにきらきらとかがやいていた 
\\	秋の夕暮れはなんとなくもの寂しい気分がします。	【ものさみしい】 (雰囲気が)寂しい。
\\	彼女はもの静かな人だ。彼女が大声を出したりあわてたりするのを私はみたことがない。	【ものしずか】 (雰囲気が)静か
\\	うまい写真をとるにはちょっとした工夫とコツがあるんです。	うまいしゃしんをとるにはちょっとしたくふうとコツがあるんです 
\\	私は直ちに出発した。さもなければ彼に会いそこなっただろう。	そうでなければ。さもなくば。
\\	彼は最善を尽くした。さもなければ一等賞は取れなかっただろう。	かれはさいぜんをつくした。さもなければいっとうしょうはとれなかっただろう 
\\	黙っていなさい。さもなくば何か聞く価値のあることを言いなさい。	そうでなければ。さもなければ。
\\	私たちは兄弟として共に生きることを知らねばならない。さもなくば、愚か者として共に滅びるであろう。	わたしたちはきょうだいとしてともにいきることをしらねばならない。さもなくば、おろかものとしてともにほろびるであろう 
\\	滅びに至る門は大きく、その道は広い。	ほろびにいたるもんはおおきく、そのみちはひろい 
\\	「ちがうちがう僕らは求めてる笑顔はあんなじゃない」 それは『嗤い』もしくは『嘲笑』。	"「ちがうちがうぼくらはもとめてるえがおはあんなじゃない」 それは『わらい』もしくは『ちょうしょう』。 
\\	ペダルを踏むたびごとに、車輪が一回転する。	そのときにはいつも。
\\	枠が出来上がると、クモはちょうど自転車の車輪の輻のように、それに絹の糸をかける。	わくができあがると、クモはちょうどじてんしゃのしゃりんのやのように、それにきぬのいとをかける 
\\	彼はドア枠の下に立った。	かれはドアわくのしたにたった 
\\	そんなにお金が無くても、なんとかやっていけるでしょう。	あれこれ工夫や努力をするさま。どうにか。
\\	電力不足で、工場は一時閉鎖を余儀なくされている。	【をよぎなく】 やむを得ないので~する  仕方ないので~する
\\	余儀ない事情で、参加を取りやめることになってしまいました。	【よぎない】 やむを得ない~  仕方がない~
\\	残念なことに、私の父は長いわずらいから回復できなかった。	病気。やまい。
\\	彼女が死人のように顔色が悪いのは長患いのせいだ。	かのじょがしにんのようにかおいろがわるいのはながわずらいのせいだ 
\\	敗北した軍はその国から撤退した。	はいぼくしたぐんはそのくにからてったいした 
\\	ソ連軍はアフガニスタンからの撤退を開始した。	ソれんぐんはアフガニスタンからのてったいをかいしした 
\\	将軍は結局のところ敗北を認めたが、予想だにしなかったことである。	しょうぐんはけっきょくのところはいぼくをみとめたが、よそうだにしなかったことである 
\\	次には、これらの輻の上に絹の糸をさらに張り、巣の中央に滑らかで、粘りのない部分を残す。	つぎには、これらのやのうえにきぬのいとをさらにはり、すのちゅうおうになめらかで、ねばりのないぶぶんをのこす 
\\	どうぞ、皆様も最後の一瞬まで粘り抜いてください。	どうぞ、みんなさまもさいごのいっしゅんまでねばりぬいてください 
\\	敗北や失敗は人間を非常に卑屈にする。	はいぼくやしっぱいはにんげんをひじょうにひくつにする 
\\	彼らはなぜあんなに卑屈に彼に仕えたのか。	かれらはなぜあんなにひくつにかれにつかえたのか 
\\	彼は親の心配をよそに遊んでばかりいたから、今年も進級できないらしい。	~を考えず  ~に無関心で  ~に関係なく
\\	彼らは連勝記録が途切れてから10連敗している。	かれらはれんしょうきろくがとぎれてからじゅうれんぱいしている 
\\	みじめな連敗で我々は意気消沈した。	【いきしょうちん】 意気込みがすっかり衰えること。元気がなくなること。意気阻喪(いきそそう)。
\\	ファンの期待をよそに、あのチームはすでに10連敗だ。	その時点ではもうその状態になっていることを表す。もはや。とっくに。
\\	人間が月に住むのももはや夢ではない。	ある事態が変えられないところまで進んでいるさま。今となっては。もう。
\\	その音楽はもはや我々の心に訴えない。	そのおんがくはもはやわれわれのこころにうったえない 
\\	この藁のマット(日本語で畳)は、もはや手で作られない。	"このわらのマット(にほんごでたたみ)は、もはやてでつくられない 
\\	政府はもはやホームレスの問題を避けてとおるわけにはいかない。	せいふはもはやホームレスのもんだいをさけてとおるわけにはいかない 
\\	南アフリカ政府は、もはや黒人に平等の公民権を求める運動を抑制できない。	みなみアフリカせいふは、もはやこくじんにびょうどうのこうみんけんをもとめるうんどうをよくせいできない 
\\	もしも平和が名誉をもって維持されるのでなければ、それはもはや平和ではない。	もしもへいわがめいよをもっていじされるのでなければ、それはもはやへいわではない 
\\	多くの川は汚染がひどいのでもはや飲料水をとるために利用することはできない。	おおくのかわはおせんがひどいのでもはやいんりょうすいをとるためにりようすることはできない 
\\	アメリカには憲法上の抑制と均衡という入念な制度がある。	アメリカにはけんぽうじょうのよくせいときんこうというにゅうねんなせいどがある 
\\	この上なく入念に準備した計画でもしばしば失敗する。	このうえなくにゅうねんにじゅんびしたけいかくでもしばしばしっぱいする 
\\	彼女はこの上なく幸せだった。	【このうえ】 今の程度以上。これ以上。
\\	運動中はよそ見をしないでください。わき見運動は事故のもとです。	【よそみ】 (ちゃんと前をみないで)横を見る  わき見をする
\\	よそ行きの服を着て、今日はどこへ行くの?	【よそいきのふく】 特別な時に着る服
\\	都会の人は態度がよそよそしい、と田舎から来た友人は嘆いている。	親しみを見せない  冷たい
\\	彼女のよそよそしさを打破し、うちとけさせるのにずいぶん時間がかかった。	かのじょのよそよそしさをだはし、うちとけさせるのにずいぶんじかんがかかった 
\\	ファッション・デザイナーたちは伝統を打破しています。	ファッション・デザイナーたちはでんとうをだはしています 
\\	私は社長の信頼を得んがため、朝早くから深夜まで働き続けた。	~たいために  ~(よ)うと思って
\\	彼らの話し振りから、私は彼らが結婚していると推測した。	かれらのはなしぶりから、わたしはかれらがけっこんしているとすいそくした 
\\	いずれにしてもあなたの推測は間違っている。	いずれにしてもあなたのすいそくはまちがっている 
\\	いずれにしても、彼を助ける手段はない。	どちらを選ぶにしても。事情がどうであろうとも。どっちみち。いずれにせよ。
\\	あなたはたくさんの仕事を狂わんばかりに急いで片づけた。	あなたはたくさんのしごとをくるわんばかりにいそいでかたづけた 
\\	私はさっさと彼の手紙に返事を書いてしまうつもりだ。	動作のすばやいさま。
\\	怒った彼は、さっさと帰れと叫ばんばかりにドアを開けた。	今にも~(し)そう  ~とばかりに
\\	彼らは「かわいそうな奴」と言わんばかりに私達をじっと見た。	"かれらは「かわいそうなやつ」といわんばかりにわたしたちをじっとみた 
\\	「あなたを愛しているのよ」と言わんばかりに彼女は僕にウィンクした。	「あなたをあいしているのよ」といわんばかりにかのじょはぼくにウィンクした 
\\	彼女はその知らせを聞いて、気も狂わんばかりに喜んだ。	かのじょはそのしらせをきいて、きもくるわんばかりによろこんだ 
\\	やがて厳しい北風が吹き始めるだろう。	あまり時間や日数がたたないうちに、ある事が起こるさま、また、ある事態になるさま。そのうちに。まもなく。じきに。
\\	しかし独身でいることのメリットにも関わらず、やがていつかは結婚したいと彼らは考えている。	しかしどくしんでいることのメリットにもかかわらず、やがていつかはけっこんしたいとかれらはかんがえている 
\\	彼女はあっさりと答えた。	時間や手間をかけずに物事が行われるさま。簡単に。
\\	私はプールでひと泳ぎした後とてもさっぱりした。	気にかかるものなどを処理して、さわやかな気分であるさま。
\\	寒さの後の暖かさで我々はうきうきした。	楽しさで心がはずむさま。うれしさのあまり落ち着いていられないさま。
\\	僕はそのグループの演技にうっとりしたよ。	美しいものなどに心を奪われて、ぼうっとしているさま。また、気抜けしたさま。
\\	テレビの画像はぼやけて見えた。	テレビのがぞうはぼやけてみえた 
\\	画像処理のソフトウエアを開発した会社をご存知でしたら教えてください。	がぞうしょりのソフトウエアをかいはつしたかいしゃをごぞんじでしたらおしえてください 
\\	彼は頭を殴られてぼうっとなった。	意識が正常でなく、ぼんやりしているさま。
\\	本当に眠ってなんかいない、うとうとしているだけだ。	眠けを催して浅い眠りに落ちるさま。
\\	その問題をうやむやにしておくことはできない。	物事がどうなのかはっきりしないこと。曖昧なさま。
\\	わが国のサッカー代表チームが歴史的大敗を喫しガックリした。	一時に疲れが出たり、気落ちしたりして、元気がめっきりなくなるさま。
\\	めっきり春らしくなった。	状態の変化がはっきり感じられるさま。
\\	犬は空腹の時、えさをがつがつ食べる。	飢えてむやみに食物を欲しがるさま。また、むさぼり食うさま。
\\	お腹を空かせたその男は、食物をむさぼり食った。	がつがつと食べる。
\\	そんなにがぶがぶ飲み続けると、アル中になるのがオチですよ。	水などをむさぼるように飲むさま。また、その音。
\\	子供の学費を考えると、オチオチビールも飲んでられないな。	落ち着いて。安心して。
\\	記者は大統領の登場前にがやがやと話していた。	大ぜいが勝手にうるさく話し合うさま。
\\	エレベーターにぎっしりいっぱいに乗った。	すきまなく詰まっているさま。ぎっちり。
\\	この鍋は色々な使い道がある。	【つかいみち】 使う方法。使い方。
\\	あのコートは高かったかもしれませんが、それだけの値うちはある。	あのコートはたかかったかもしれませんが、それだけのねうちはある 
\\	彼女はもう二度と私の顔を見たくないときっぱり言った。	態度をはっきりと決めるさま。
\\	彼の娘は動作がきびきびしている。	人の動作や話し方などが生き生きとして気持ちのよいさま。
\\	どの電車も通勤者でぎゅうぎゅう詰めだった。	全く余裕がないほど、たくさん詰め込むこと。
\\	彼女は驚いてきょとんとした顔をした。	びっくりしたり、事情がのみこめなかったりして、目を見開いてぼんやりしているさま。
\\	田舎者は街中できょろきょろとしていた。	落ち着きなく、絶えずあたりを見まわすさま。
\\	重荷で机がきしるほどだった。	おもにでつくえがきしるほどだった 
\\	グズグズしてないでさっさと行動しろ!	のろのろといたずらに時間を費やすさま。
\\	ぐずぐずしていたために、高いホテルに一晩泊まらざるをえなくなった。	ぐずぐずしていたために、たかいホテルにひとばんとまらざるをえなくなった 
\\	休日で車は街道をのろのろと動いた。	動きがにぶく、ゆっくりしているさま。
\\	タクシーは蝸牛と同じくらいのろのろ進んでいるように思えた。	タクシーはかたつむりとおなじくらいのろのろすすんでいるようにおもえた 
\\	くちゃくちゃとガムをかまないで下さい。	口の中で物をかむときの音を表す語。
\\	4マイルも歩かないうちに彼はくたくたになった。	疲れたり弱ったりして、力の抜けたさま。また、古くなって張りのなくなったさま。
\\	その木は空にくっきりと浮かび上がっていた。	物の姿や形が非常にはっきりとしているさま。
\\	そのような痕跡がくっきりと残っているかどうかによって、これらの人々がいつも重労働に従事していたかどうかがわかる。	そのようなこんせきがくっきりとのこっているかどうかによって、これらのひとびとがいつもじゅうろうどうにじゅうじしていたかどうかがわかる 
\\	メディアが憎悪の痕跡を安売りする。	メディアがぞうおのこんせきをやすうりする 
\\	黒人に対するひどい人種的憎悪はまだ存在している。	こくじんにたいするひどいじんしゅてきぞうおはまだそんざいしている 
\\	暑いしむしむしするので熟睡できなかった。	あつしむしむしするのでじゅくすいできなかった 
\\	彼女が熟睡しているのを確かめて、彼はこっそり部屋を抜け出して行った。	かのじょがじゅくすいしているのをたしかめて、かれはこっそりへやをぬけだしていった 
\\	一晩ぐっすり眠ったので彼はさわやかに見えた。	深く眠っているさま。熟睡するさま。
\\	私たちは5時間の旅でぐったりしてしまった。	疲れたり弱ったりして、力が抜けたさま。ぐたっと。
\\	トムは自分の感情をぐっと抑えた。	すっかり。ことごとく。
\\	たった1回失敗したぐらいでクヨクヨするなよ!	いつまでも気にかけて、あれこれと思い悩むさま。
\\	この歯がグラグラします。	物が揺れ動いて安定しないさま。また、事柄や気持ちなどが動揺するさま。
\\	行けども行けども、やはり同じ場所をグルグル回っている気がしてならない。	物が続いて回るさま。
\\	おばさんたちが電車でげらげら笑っていた。	閉まりなく、大声で笑うさま。
\\	水がなければこの錠剤は飲み込めません。	みずがなければこのじょうざいはのみこめません 
\\	これらの錠剤を飲めば腹痛は治るでしょう。	これらのじょうざいをもめばふくつうはなおるでしょう 
\\	ごくっと唾を飲み込んだ。	液体や、錠剤などの小さな固形物を一気に飲み込むさま。ごくりと。
\\	影でこそこそせず面と向かって彼にそれをいいたまえ。	人目につかないように物事をするさま。こっそり。
\\	熱帯雨林は地球に多くの恩恵を与える。	ねったいうりんはちきゅうにおおくのおんけいをあたえる 
\\	オーストラリア人は全体として政治システムが安定していることによる恩恵に満足している。	オーストラリアじんはぜんたいとしてせいじシステムがあんていしていることによるおんけいにまんぞくしている 
\\	神は許し給うとも、そのゆえに人間は忘れるべきではない。	かみはゆるしたまうとも、そのゆえににんげんはわすれるべきではない 
\\	弟の部屋はいつも雑然としている。	【ざつぜん】 いろいろなものが入り乱れて、まとまりのないさま。
\\	ごたごたが君をごたごたさせるまでごたごたをごたごたさせるな。	ごたごたがきみをごたごたさせるまでごたごたをごたごたさせるな。 
\\	ごたごたに巻き込まれるようなことはゆうな。	混乱や争いが起こっているさま。
\\	地道な者はいつか勝つ。	じみちなものはいつかかつ 
\\	悪銭身につかず、というじゃない。結局は地道に稼ぐしかないと思うよ。	あくせんみにつかず、とうじゃない。けっきょくはじみちにかせぐしかないとおもうよ 
\\	彼はこつこつ事実を調べていた。	地道に働くさま。たゆまず努め励むさま。
\\	彼は同僚にこまごまとしたことを言わない。	細かいところまで行き届くさま。
\\	断崖の天辺に古い城が立っている。	だんがいのてっぺんにふるいしろがたっている 
\\	断崖から身を乗り出しこわごわ下を覗きこむ。	恐ろしく思いながら物事をするさま。おそるおそる。
\\	マージャンの遊び方がさっぱりわからん。	全然。まったく。
\\	冬の関東平野はさむざむとしていた。	いかにも寒そうなさま。
\\	私たちは山の上から平野を見下ろした。	わたしたちはやまのうえからへいやをみおろした 
\\	てんで見当つきません。	てんでけんとうつきません 
\\	てんで役にたたない。	まるっきり。まったく。てんから。
\\	その運転手は、まるっきり違う街の間違った球場にチームを運んで行ってしまうという大ドジを踏んでしまった。	そのうんてんしゅは、まるっきりちがうまちのまちがったきゅうじょうにチームをはこんでいってしまうというおおドジをふんでしまった 
\\	私の妻は料理がまるきりへたです。	全く。まるで。まるっきり。
\\	あう~、あたしったらまたドジっちゃいました。	間の抜けた失敗をすること。
\\	その事故はまだありありと彼の記憶に残っている。	はっきりと外に現れるさま。明らかに。
\\	むやみに自殺を否定しようとは思いません。	結果や是非を考えないで、いちずに物事をすること。
\\	彼女はやたらと靴を買う。	根拠・節度がないさま。筋が通らないさま。めちゃくちゃ。むやみ。
\\	どんなに節度のある家庭でも事故は起こるもの。	どんなにせつどのあるかていでもじこはおこるもの 
\\	ひっきりなしの騒音にいらいらする。	絶え間なく続くさま。切れ目のないさま。
\\	金の切れ目が縁の切れ目。	かねのきれめがえんのきれめ 
\\	サメの皮はマグロの皮よりはるかにざらざらしている。	触った感じが粗く滑らかでないさま。
\\	一人の小さな女の子がしくしく泣いているのをみつけた。	声をひそめて弱々しく泣くさま。
\\	その弱々しい患者は胃癌に苦しんでいる。	【よわよわしい】 いかにも弱そうである。力や元気がなさそうである。
\\	これはとてもしっかりした家だ。どこへでも持っていけますよ。	かたく強いさま。
\\	医師は難しい手術についてじっくり考えた。	落ち着いて、また、念入りに物事をするさま。
\\	ジョンは、念入りにその事故を調査した。	細かい点にまでよく気をつけて物事をすること。
\\	その問題について手を打つ前に彼は3日間じっくり考えた。	そのもんだいについててをうつまえにかれはみっかかんじっくりかんがえた 
\\	政府はしぶしぶ経済政策を変更した。	気が進まぬまま、しかたなく物事をするさま。嫌々ながら。
\\	私はそれを嫌々引き受けた。	【いやいや】 しかたなく物事を行うさま。嫌だとは思いながら。しぶしぶ。
\\	あの人に頼むのはちょっと気が進まない。	すすんでしようとは思わない。気乗りがしない。
\\	このところあまり仕事に気乗りがしない。	【きのり】 興味がわいて、それをしようという気持ちになること。気が進むこと。
\\	彼は主たる論点を明らかにすることに気乗り薄である。	【きのりうす】 あまり気が進まないこと。
\\	老齢は気づかぬうちに我々に忍び寄る。	ろうれいはきづかぬうちにわれわれにしのびよる 
\\	老齢人口は、健康管理にますます多くの出費が必要となるだろう。	ろうれいじんこうは、けんこうかんりにますますおおくのしゅっぴがひつようとなるだろう 
\\	私たちの主たる関心は社会の老齢化にあるべきだ。	【しゅたる】 おもな。主要な。
\\	じめじめした寒い日は健康に悪い。	湿気が多く不快なさま。
\\	夢に欠けている主な事は首尾一貫性である。	ゆめにかけているおもなことはしゅびいっかんせいである 
\\	それぞれの文化には首尾一貫した世界観がある。	それぞれのぶんかにはしゅびいっかんしたせかいかんがある 
\\	彼女の説明はしりめつれつである。	物事に一貫性がなく、ばらばらで、まとまりのないこと。
\\	彼らは彼女の水着を驚きの目でじろじろと見た。	無遠慮に目を向けるさま。
\\	外国人がずかずかくつのまま畳に上がり込んだ。	遠慮なく乱暴に入ったり近寄ったりするさま。
\\	部下は上司にずけずけ物が言えない。	遠慮や加減をしないで、はっきりとものを言うさま。つけつけ。
\\	あの犬は雄か雌か。	あのいぬはおすかめすか 
\\	雄の孔雀は尾の羽毛が色彩豊かである。	おすのくじゃくはおのうもうがしきさいゆたかである 
\\	その羽毛枕が高そうです。	そのうもうまくらがたかそうです 
\\	孔雀は本当に目の覚めるような美しい尾をしている。	くじゃくはほんとうにめのさめるようなうつくしいおをしている 
\\	赤ん坊は母親の腕の中ですやすやと眠っている。	静かによく眠っているさま。また、そのときの寝息の音を表す語。
\\	すらすら言えるようになるまでこの文を暗記しなさい。	物事が滞りなくなめらかに進行するさま。
\\	滞りのない、優雅な仕草でグラスに水を注ぎ込んだ。	とどこおりのない、ゆうがなしぐさでグラスにみずをそそぎこんだ 
\\	ポニーテールが翻って思わず見とれる仕草だ。	ポニーテールがひるがえっておもわずみとれるしぐさだ 
\\	支払いを滞る原因となるような問題があるのですが。	しはらいをとどこおるげんいんとなるようなもんだいがあるのですが 
\\	ずらりと並んだアルミのポットやなべ。	人や物がたくさん並び連なっているさま。ずらっと。
\\	自動車は角をすれすれに通った。	触れそうになるくらい近づいていること。
\\	僕は終電にすれすれのところで間に合った。	限界をもう少しで越えそうなこと。
\\	スーザンは賢いから試験にすんなり通ると思うよ。	物事が滞ることなく、なめらかに進むさま。
\\	その少女は大きくなってすらりとした女性になった。	ほっそりと形よく伸びているさま。すらっと。
\\	外に出ると寒さでぞくぞくした。	寒けがするさま。
\\	まもなく彼女に会えるとうれしくてぞくぞくした。	感情の高ぶりや緊張、また、恐怖などのために身震いするさま。
\\	彼らは森の中で死体を見つけた時、身震いせずにはいられなかった。	かれらはもりのなかでしたいをみつけたとき、みぶるいせずにはいられなかった 
\\	われにもなく、彼はちょっと身震いした。	無意識のうちに。我を忘れて。
\\	われわれは皆その大きなショックで身震いがした、ぞっとした。	寒さや恐怖などのために、また、強い感動を受けて、からだが震え上がるさま。
\\	私は朝の4時半ごろにそわそわし出したが、6時まで起きなかった。	気持ちや態度が落ち着かないさま。
\\	夫は妻の詰問にたじたじとなった。	困難に直面したり、相手の力に圧倒されたりしてひるむさま。
\\	だぶだぶの上着が最新の流行なのだ。	衣服などが大きすぎてゆるいさま。また、太りすぎて、肉がたるんでいるさま。ぶかぶか。
\\	話は三時までだらだらと続いた。	変化の乏しい状態が長く続くさま。
\\	汗がダラダラです。	液体がたくさん流れつづけるさま。
\\	工事はちゃくちゃくと完成に近づいている。	仕事などが次々と順序よくはかどるさま。
\\	小屋は次々に風で吹き倒された。	【つぎつぎ】 物事が次から次へと続くさま。
\\	たまにはビール以外のお酒をちびちび舐めるのも良いでしょう。	一度にしてしまわないで、ほんの少しずつするさま。ちびりちびり。
\\	この靴下はちぐはぐだ。	二つ以上の物事が、食い違っていたり、調和していなかったりするさま。
\\	私の猫は一生懸命私のご機嫌をとろうとする。	わたしのねこはいっしょけんめいわたしのごきげんをとろうとした 
\\	この歌手はあまりにもちやほやされすぎる。	相手の機嫌をとるようなさま。
\\	遠くで小さな明かりがちらちらしていた。	小さい光が強まったり弱まったり、また、細かく揺れ動いたりするさま。
\\	時がたつのは何と早いことかと彼はつくづく考えた。	物事を、静かに深く考えたり、注意深く観察したりするさま。よくよく。じっくり。
\\	私の父はビリヤードの玉のように頭がつるつるだ。	物の表面がなめらかで、つやのあるさま。
\\	この金属は磨くと艶が出る。	【つや】 物の表面から出るしっとりとした光。光沢。
\\	この指輪は光沢を失った。	このゆびわはこうたくをうしなった 
\\	ぶらぶら油を売っていないで、とっとと仕事にもどったらどうなのさ。	さっさと。はやく。
\\	ちょっとその辺をブラブラしました。	あてもなくのんびり歩きまわるさま。
\\	妻子は運命に与えられた人質である。	さいしはうんめいにあたえられたひとじちである 
\\	彼は、妻子を田舎にのこしておいて、職を求めてぶらぶらしていた。	なすこともなく毎日を暮らすさま。
\\	老人がとぼとぼ歩いていた。	元気なく歩くさま。
\\	梅雨らしく空はどんよりしている。	空が曇って重苦しく感じられるさま。
\\	彼女が物事をてきぱきできないのを驚いた。	処理や対応がはっきりしていて、歯切れのよいさま。
\\	さっきからあの人一人でにやにやして。怪しい。	声を出さないで薄笑いを浮かべるさま。
\\	うなぎはぬるぬるしていてつかみにくい。	粘液状のものがついたりしていてすべりやすいさま。また、そのような感じで不快なさま。ぬらぬら。
\\	はきはきと意見を言ってください。	話し方・態度・性格・行動などがはっきりしているさま。
\\	飢えた人たちの写真にはっと息をのむ思いをした。	思いがけない出来事にびっくりするさま。
\\	コンピューターは非常に複雑な仕事を瞬時にすることができる。	コンピューターはひじょうにふくざつなしごとをしゅんじにすることができる 
\\	火はぱっと明るく燃え上がった。	すばやく動作をするさま。また、瞬時に物事が起こるさま。
\\	学校から団体で行く人たちもいるが、ほとんどの人たちはばらばらに行く。	互いに別々になること。また、別々であるさま。
\\	木の葉がはらはらと散った。	小さいものや軽いものが、静かに続けて落ちかかるさま。
\\	ホテルの食堂でランチの最中に、ステラという若い女性がばったり倒れ、医師のスチュワートがそのからだを調べて言うことには・・・。	勢いよく倒れたり落ちたりするさま。ばたり。
\\	私が出かけようとした際、旧友が訪ねてきた。	わたしはでかけようとしたさい、きゅうゆうがたずねてきた 
\\	私は旧友に出会い、更に不思議な事に恩師に出会った。	わたしはきゅうゆうにであい、さらにふしぎなことにおんしにであった 
\\	私は恩師の世話でこの仕事に就いた。	わたしはおんしのせわでこのしごとをついた 
\\	駅前でばったり旧友に会った。	思いがけなく人に出会うさま。
\\	思いがけず私達は空港で出くわした。	【おもいがけず】 予期しなかったのに。思いがけなく。
\\	トムは車をピカピカの新車で買った。	真新しいさま。また、なりたてであるさま。
\\	その車はワックスがかけられてピカピカしている。	つやがあって照り輝いているさま。
\\	両親が昨晩ひそひそと話をしているのを聞いた。	他人に聞こえないように小声で話すさま。
\\	その問題は我々の日常生活に密着している。	そのものだいはわれわれのにちじょうせいかつにみっちゃくしている 
\\	今朝ひょっこり旧友に会った。	思いがけなくそのことが起きるさま。
\\	作者がこの本の最後の章でひょっこり登場した人物に罪を着せました。	さくしゃがこのほんのさいごのしょうでひょっこりとうじょうしたじんぶつにつみをきせました 
\\	父はついさきほど出かけた。	少し前。いましがた。先刻。
\\	先刻承知。	せんこくしょうち 
\\	巨大なタンカーがついに今し方出港した。	きょだいなタンカーがついにいましがたしゅっこうした 
\\	今し方地面が揺れたのを感じましたか。	【いましがた】 ついさっき。たった今。
\\	私がついさっき書いた詩を君に読んであげよう。	たった今。 今し方。
\\	男は壁の隙間から覗いた。	おとこはかべのすきまからのぞいた 
\\	北風が隙間からひゅーひゅー吹き込む。	強い風が連続して吹く音を表す語。
\\	風が吹くたびに、桜の花びらが、ひらひらと舞い降りてました。	かぜがふくたびに、さくらのはなびらが、ひらひらとまいおりてました 
\\	ヒラヒラと桜の花が舞っている。もうお花見の季節も終わりだな。	薄くて軽いものが揺れ動くさま。
\\	泥棒はひらりと屋根から屋根へ飛び移った。	すばやく身をかわしたり飛び移ったりするさま。
\\	どんな俳優も劇が始まる前にはびくびくするものだ。	絶えず恐れや不安を感じて落ち着かないでいるさま。
\\	彼の重さでロープがぴんと張った。	ゆるみなく、まっすぐに張るさま。
\\	毎日大食していると、ブクブク太り過ぎること間違えなし。	しまりなく太っているさま。
\\	風邪には大食、熱には小食。	かぜにはたいしょく、ねつにはしょうしょく 
\\	ワインの飲みすぎでふらふらになった。	揺れ動いて安定しないさま。
\\	しゃべり出した時彼の両手はぶるぶる震えた。	寒さや恐怖などのために震えるさま。
\\	日干し後の布団はふわふわしている。	柔らかくふくらんでいるさま。
\\	部屋は香水の匂いでぷんぷんしていた。	へやはこうすいのにおいでぷんぷんしていた 
\\	隣に座った男の人がウイスキーのにおいをプンプンとさせて、がまんできなかった。	強いにおいがしきりに鼻をつくさま。比喩的にも用いる。
\\	その比喩的意味はもはや使われていない。	そのひゆてきいみはもはやつかわれていない 
\\	この表現は日本語にはない英語の比喩表現として、私は大変気に入っています。	このひょうげんはにほんごにはないえいごのひゆひょうげんとして、わたしはたいへんきにいっています 
\\	1年が過ぎ、彼の死体は塵と化した。	いちねんがすぎ、かれのしたいはちりとかした 
\\	彼は借金を頼みに頭をペコペコ下げてやって来た。	頭をしきりに下げるさま。また、へつらうさま。へこへこ。
\\	あなたは上役に諂わなくてもよい。	あなたはうわやくにへつらわなくてもよい 
\\	彼は走り続けてついに完全にへとへとになった。	非常に疲れて力がすっかり抜けてしまうさま。
\\	猫はぺろぺろと前足をなめている。	舌で物をなめまわすさま。
\\	春の陽気はぽかぽか呼ばれる。	暖かく感じるさま。暖かくて気持ちのよいさま。
\\	パイプから水がぽたぽた落ちているのが聞こえるでしょ。	液体などが続けて落ちるさま。
\\	青空に雲がぽっかりと浮かんでいた。	軽く浮かんでいるさま。
\\	彼女はぽっちゃりとしてきた。	ふっくらとして愛らしいさま。
\\	彼は電話で初めて彼女の愛らしい声を聞いてすっかり彼女に惚れ込んでしまった。	かれはでんわではじめてかのじょのあいらしいこえをきいてすっかりかのじょにほれこんでしまった 
\\	「どんなことでも直ぐに分かります、・・・ググれば」「ググれば?」	"「どんなことでもすぐにわかります、・・・ググれば」「ググれば?」 
\\	その建物は内外ともぼろぼろになっている。	もろく崩れたり、砕けたりするさま。
\\	女性は涙もろい。	【なみだもろい】 ちょっとしたことにも涙が出がちである。情にほだされやすい
\\	情にほだされて思わず涙を流した。	じょうにほだされておもわずなみだをながした 
\\	ぽつぽつと雨が降り始めた。	物事が少しずつ行われるさま。また、物事を少しずつゆっくり行うさま。
\\	ぼつぼつ始めようか。	物事が勢いよくわき起こるさま。
\\	彼に話し掛けられたときまごついてしまった。	かれははなしかけられたときもごついてしまった 
\\	彼女は不注意な間違いにまごついて、わっと泣き出した。	急に声をあげて泣くさま。
\\	まごまごしていないで、彼女に早く告白しなさい。	まごつくさま。うろたえるさま。
\\	気候の突然の変化にうろたえるな。	不意を打たれ、驚いたり慌てたりして取り乱す。
\\	彼はその悲報を取り乱さずに聞いた。	かれはそのひほうをとりみださずにきいた 
\\	その男は、終身刑であると裁判官が判決を下した時、取り乱した。	そのおとこは、しゅうしんけいであるとさいばんかんがはんけつをおろしたとき、とりみだした 
\\	ボブは酷く取り乱していて、現実と虚構の区別がほとんど出来なかった。	ボブはひどくとりみだしていて、げんじつときょこうのくべつがほとんどできなかった 
\\	殺人の有罪宣告を受け、彼は終身刑を科せられた。	さつじんのゆうざいせんこくをうけ、かれはしゅうしんけいをかせられた 
\\	あの赤ん坊はまるまるしていて健康的だ。	よく太っているさま。
\\	彼女はご主人の態度にむかっときた。	急に怒り出すさま。
\\	朝起きた時むかむかします。	吐き気がして気持ちの悪いさま。
\\	私はむらむらとその本が買いたくなった。	抑えがたい感情や思いがわき起こるさま。
\\	彼は英語の力がめきめきついてきた。	目に見えて、進歩・発展するさま。
\\	その目茶苦茶な状態をいったいどうしたらいいのだ。	【めちゃくちゃ】 どうにもならないほどに壊れたり、混乱したりすること。めちゃめちゃ。
\\	彼女の胸の内に野心の火がめらめらと燃えていた。	炎を立てて勢いよく燃え広がるさま。
\\	「どーした、もじもじして」「あーいや、何かパンツのゴム切れちゃったみたいで」	遠慮や恥ずかしさなどのために、はっきりした態度がとれないさま、また、落ち着かないさま。
\\	彼女は生まれた赤ん坊を見せたくてやきもきしていた。	あれこれと気をもんでいらいらするさま。
\\	私たちの家は快適ですが、やっぱり前の家がなつかしい。	予測したとおりになるさま。案の定。
\\	さいは麻酔でよたよたしていた。	足の動きがしっかりしていないさま。
\\	彼は彼の几帳面さを自慢した。	かれはかれのきちょうめんさをじまんした 
\\	誰でも大なり小なり自惚れはある。	だれでもだいなりしょうなりうぬぼれはある 
\\	彼女は英語を話すのが一番うまいと自惚れている。	かのじょはえいごをはなすのがいちばんうまいとうぬぼれている 
\\	大欲は無欲に似たり。	たいよくはむよくににたり 
\\	しばしば彼は自分で行かざるをえなかった。	同じ事が何度も重なって行われるさま。たびたび。
\\	私は彼がそのメロディーをトランペットで吹いているのをたびたび耳にした。	何度も繰り返し行われるさま。いくども。しばしば。
\\	彼は父親にそっくりだと誰もが言っている。	非常によく似ているさま。
\\	彼女は入ってからきっちりドアを閉めた。	隙間やずれがないさま。ぴったり。
\\	私はきっちり3時間待った。	時間や数量に端数がないさま。きっかり。
\\	8時きっかりに、自分の部屋で朝食を食べたいんだ。	時間・数量などが正確で過不足のないさま。かっきり。ちょうど。
\\	距離で100メートルといったらかっきり100メートルである。	数量・時間などに、端数のないさま。ぴったり。きっかり。
\\	暑かったので、彼女は汗びっしょりになりました。	ひどくぬれるさま。
\\	がっちり組んで助け合わねばならない。	すきまなく組み合うさま。
\\	これは掛けるのに大層頑丈な椅子だ。	これはかけるのにたいそうがんじょうないすだ 
\\	この鞄は君が持っているのと同じほど頑丈そうに見える。	このかばんはきみがもっているのとおなじほどがんじょうそうにみえる 
\\	彼は背が高く骨組みのがっちりした人だ。	引き締まっていて丈夫そうなさま。頑丈なさま。がっしり。
\\	金融引き締めで金利が上昇するだろう。	きんゆうひきしめできんりがじょうしょうしゅるだろう 
\\	がっしりした私の父の隣にいて、彼は小さく見えた。	物の構造や体格がしっかりしていて、力強く、また、簡単には壊れそうにないさま。がっちり。
\\	社会構造はそれほど変わらない。	しゃかいこうぞうはそれほどかわらない 
\\	この増加に加えて、世界の経済構造の変化があった。	このぞうかにくわえて、せかいのけいざいこうぞうのへんかがあった 
\\	骨格たくましい体格なので彼は柔道家でとおっている。	こっかくたくましいたいかくなのでかれはじゅうどうかでとおっている 
\\	当時はまだ彼はたくましい精神力だった。	からだが頑丈で、いかにも強そうに見える。
\\	異様な光景が彼女の目に留った。	いようなこうけいがかのじょのめにとまった 
\\	彼女はこってりした食べ物が好きだ。	味や色などが濃くて、しつこいさま。
\\	こってり化粧した顔は異様である。	こってりけしょうしたかおはいようである 
\\	彼女は髪を無造作に束ねている。	かのじょはかみをむぞうさにたばねている 
\\	彼は、薪を束ねた。	かれは、まきをたばねた 
\\	それをざっくりと各工程に割り振ったものです。	それをざっくりとかくこうていにわりふったものです 
\\	登場人物がうまく割り振られていた。	とうじょうじんぶつがうまくわりふられている 
\\	ジョンは抜け目がないので成功は間違いない。	ジョンはぬけめがないのでせいこうはまちがいない 
\\	彼女にはちゃっかりしたところがある。	自分の利益のために抜け目なく振る舞うさま。
\\	てっきりあなたが我々といっしょに来られるものと思っていました。	確かだと思っていた予想・推測が反対の結果となって現れた場合に用いる語。きっと。
\\	彼はウイスキーをちょっぴり飲んだ。	分量や程度がわずかであるさま。ほんのすこし。
\\	町が一面すっぽり雪をかぶった。	全体を余すところなく覆うさま。
\\	それはでっぷり太った猫ですね。	太っていて恰幅(かっぷく)のよいさま。
\\	京都には有名な古い建造物がたくさんある。	きょうとにはゆうめいなふるいけんぞうぶつがたくさんある 
\\	これは私が見た中で一番どっしりとした建造物です。	いかにも重みのあるさま。ずっしり。
\\	一晩水につかっていた着物はずっしり重たかった。	きわめて重いさま。また、重そうなさま。ずしり。
\\	彼はいつもどっしりしている。	落ち着きがあって重々しいさま。
\\	彼は群衆に向かって重々しく話しかけた。	かれはぐんしゅうにむかっておもおもしくはなしかけた 
\\	彼の手はねっとりして気持ち悪かった。	とろりとしてねばりけのあるさま。また、ねばねばするさま。
\\	コンピューターは、たとえ緩慢にせよ大きな変化を引き起こした。	コンピューターは、たとえかんまんにせよおおきいなへんかをひきおこした 
\\	ジョンはのっそり起き上がった。	動作がのろいさま。また、行動が緩慢で、とらえどころがないさま。
\\	商店街は平日ひっそりしている。	物音や人声がせず静かなさま。
\\	私たちはその討論を争点となっている問題に限定すべきだ。	わたしたちはそのとうろんをそうてんとなっているもんだいにげんていすべきだ 
\\	ここでは私は議論を、なぜ相撲の好きな外国人が多いか、に限定したい。	ここではわたしはぎろんを、なぜすもうのすきながいこくじんがおおいか、にげんていしたい 
\\	私は五百円ぽっきりしか持っていない。	数量を表す語に付いて、ちょうどそれだけと限定する意を表す。
\\	彼はぽっくり死にました。	人が突然に死ぬさま。
\\	祖父は祖母がぽっくり逝ってから急に老け込みました。	そふはそぼがぽっくりいってからきゅうにふけこみました 
\\	先生は学生にえいごみっしり教えた。	一つのことを十分に行うさま。みっちり。
\\	彼女はむっつりとした表情を浮かべていた。	口数が少なく、愛想のないさま。
\\	ご家庭でふっくらパンを作ってみませんか。	やわらかにふくらんでいるさま。ふっくり。
\\	さしあたり大きな台風はこないだろう。	先のことはともかく、今のところ。今しばらくの間。当面。
\\	ともかく我々はベストを尽くしたのだ。	とにかく。ともかくも。
\\	何はともあれありがとう。	【なにはともあれ】 ほかの事はどうであろうとも。ともかく。
\\	何はともあれ、大事にいたらなかったのは、不幸中の幸いであった。	なにはともあれ、だいじにいたらなかったのは、ふこうちゅうのさいわいであった 
\\	だれにも死亡がなかったのは不幸中の幸いでした。	【ふこうちゅうのさいわい】 不幸な出来事の中でせめてもの救いとなること。
\\	試験期日を照会してみる必要がある。	しけんきじつをしょうかいしてみるひつようがある 
\\	お支払いの期日を二ヶ月間延長していただけませんでしょうか。	おしはらいのきじつをにかげつかんえんちょうしていただけませんでしょうか 
\\	彼は取引先に照会状を書いた。	かれはとりひきさきにしょうかいじょうをかいた 
\\	君を生かすも殺すも僕次第だ。	きみをいかすもころすもぼくしだいだ 
\\	我々を生かそうとして彼は死んだ。	われわれをいかそうとしてかれはしんだ 
\\	幾多、苦戦したのち、我々は勝利を収め、新政府を樹立することができた。	いくた、くせんしたのち、われわれはしょうりをおさめ、しんせいふをじゅりつすることができた 
\\	彼は革製品の売買をしている。	かれはかわせいひんのばいばいをしている 
\\	ディール市内と近辺の家屋の売買を仲介しています。	ディールしないときんぺんのかおくのばいばいをちゅうかいしています 
\\	彼は経済学に相当な貢献をした。	【こうけん】 ある物事や社会のために役立つように尽力すること。
\\	日本はそうすることによって、文化及び教育の面で貢献することが出来る。	にほんはそうすることによって、ぶんかおよびきょういくのめんでこうけんすることができる 
\\	日常生活について言えば彼はとてもだらしない。	きちんとしていない。整っていない。
\\	長くてだらしがないより短くて簡潔のほうがよい。	ながくてだらしがないよりみじかくてかんけつのほうがよい 
\\	道はなだらかな上がりになっていた。	傾斜の度合いがゆるやかなさま。
\\	屋根は鋭い角度で傾斜している。	やねはするどいかくどでけいしゃしている 
\\	その土地は川に向かって緩やかに傾斜している。	そのとちはかわにむかってゆるやかにけいしゃしている 
\\	騒々しい音楽を聞くとフレッドは頭にくるんだ。	そうぞうしいおんがくをきくとフレッドはあたまにくるんだ 
\\	私は騒々しいクラスで声が通らなかった。	【そうぞうしい】 物音や人声が多くてうるさい。さわがしい。
\\	ためらうものは失敗する。	あれこれ考えて迷う。決心がつかずにぐずぐずする。
\\	彼は帰宅しようか居残って仕事を続けようかためらった。	かれはきたくしようかいのこってしごとをつづけようかためらった 
\\	ためらうことなく、その陰謀にたいする徹底した対抗処置をとった。	ためらうことなく、そのいんぼうにたいするていこうしたたいこうしょちをとった 
\\	記者達は個人の生活を侵害することにためらいを感じない。	きしゃたちはこじんのせいかつをしんがいすることにためらいをかんじない 
\\	その薬は彼に不思議なほどよく効いた。	ちょうどよいくらいに。
\\	列車のコンパートメントはすぐ窮屈になる。	れっしゃのコンパートメントはすぐきゅうくつになる 
\\	非の打ち所がない計画は、窮屈だなあ。	ひのうちどころがないけいかくは、きゅうくつだなあ 
\\	父は若いころ車を買うゆとりもなかった。	物事に余裕があり窮屈でないこと。余裕。
\\	隊長は部下に直ちに集合するように命令した。	たいちょうはぶかにただちにしゅうごうするようにめいれいした 
\\	この研究は各国の移民政策を比較するものである。	このけんきゅうはかっこくのいみんせいさくをひかくするものである 
\\	委員会は、大気汚染を抑制するために互いに協力し合うよう各国に要請した。	いいんかいは、たいきおせんをよくせいするためにたがいにきょうりょくしあうようかっこくにようせいした 
\\	私は彼の終わりのない説教にうんざりした。	わたしはかれのおわりのないせっきょうにうんざりした 
\\	彼女の悲しみは無言でこぼす涙になって表れた。	かのじょのかなしみはむごんでこぼすなみだになってあらわれる 
\\	彼女は叫びながら、ナイフを振り回しました。	かのじょはさけびながら、ナイフをふりまわした 
\\	そのように刃物を振り回すことは危険だ。	手や手に持った物などを、勢いよく振ったり回したりする。
\\	彼女は非常に美人だったので、彼女が通り過ぎると誰でも振り向いたものです。	かのじょはひじょうにびじんだったので、かのじょがとおりすぎるとだれでもふりむいたものです 
\\	立ち去るおまえを俺は振り向くことができなかった。	【ふりむく】 顔やからだを後方へ向ける。振り返って見る。
\\	キング牧師は首を撃たれ、後方に倒れた。	キングぼくしはくびをうたれ、こうほうにたおれた 
\\	彼は僕の方を振り向くとニヤリと笑った。	声を出さないでちょっと笑いを浮かべるさま。
\\	薬屋はこの道の突き当たりにあります。	くすりやはこのみちのつきあたりにあります 
\\	彼女は大勢の少年たちに凝視されているのを意識していたと思う。	かのじょはおおぜいのしょうねんたちにぎょうしされているのをいしきしていたとおもう 
\\	人をじっと見つめるのは無礼である。	【みつめる】 対象から目をそらさずにじっと見つづける。凝視する。
\\	両者の利害の調整を図りつつ、国際的視野に立った人口政策を考えていかなければならない。	りょうしゃのりがいのちょうせいをはかりつつ、こくさいてきしやにたったじんこうせいさくをかんがえていかなければならない 
\\	その著名な詩人は自分の書斎で自殺を図ろうとした。	そのちょめいなしじんはじぶんのしょさいでじさつをはかろうとした 
\\	聴衆は一斉に立ち上がって拍手喝采した。	ちょうしゅうはいっせいにたちあがってはくしゅかっさいした 
\\	欠点があるからそれだけよけいに彼を愛している。	普通より分量の多いこと。程度が上なこと。たくさん。
\\	私はくだらない議論に巻き込まれた。	まじめに取り合うだけの価値がない。程度が低くてばからしい。くだらぬ。くだらん。
\\	生徒たちはくだらない質問で先生を困らせた。	せいとたちはくだらないしつもんでせんせいをこまらせた 
\\	自分の友達の悪口を言うなんてみっともないぞ。	見た目にわるい。体裁がわるい。
\\	彼女は彼の紳士らしい体裁にだまされてしまった。	かのじょはかれのしんしらしいていさいにだまされてしまった 
\\	一部の病院は幼児用ミルクの無料見本を配布する。	いちぶのびょういんはようじようミルクのむりょうみほんをはいふする 
\\	にわかに雨が降ってきた。	物事が急に起こるさま。突然。
\\	私たちの惑星は飛ぶ鳥の気軽さで宇宙を動いている。	わたしたちのわくせいはとぶとりのきがるさでうちゅうをうごいている 
\\	万が一に備えて、武器になるものを探した。	まんがいちにそなえて、ぶきになるものをさがした 
\\	昨秋は晴天続きだった。	さくしゅうはせいてんつづきだった 
\\	真っ赤な夕焼けは明日の晴天を告げた。	まっかなゆうやけはあしたのせいてんをつげた 
\\	しびれを切らして借金の催促をした。	しびれをきらしてしゃっきんのさいそくをした 
\\	催促がましくて恐縮ですが、先日お貸ししたお金を返していただけませんか。	さいそくがましくてきょうしゅくですが、せんじつおかししたおかねをかえしていただけませんか 
\\	その電線に触るとしびれるよ。	電気などを感じてびりびりふるえる。
\\	正座して足がしびれる。	体の一部または全体の感覚が失われ、自由がきかなくなる。
\\	通りですれ違った時私をわざと無視した。	とおりですれちがったときわたしをわざとむしした 
\\	彼の言うことすべてが、私の神経を逆なでするのです。	かれのいうことすべてが、わたしのしんけいをさかなでするのです 
\\	鯨は魚ではなくて哺乳類である。	くじらはさかなではなくてほにゅうるいである 
\\	爬虫類が大嫌い。	はちゅうるいがだいきらい 
\\	与党の首脳たちは政治改革法案で知恵をしぼっています。	よとうのしゅのうたちはせいじかいかくほうあんでちえをしぼっています 
\\	もう少し脳味噌をしぼって考えろ。	もうすこしのうみそをしぼってかんがえる 
\\	支配人が不在の時は彼女が業務を管理する。	しはいにんがふざいのときはかのじょがぎょうむをしょりする 
\\	彼をはるばる訪ねて行ったがあいにく不在だった。	遠く離れているさま。遠くから、または遠くへ物事の及ぶさま。はるかに。
\\	群集がひとりでに集まり始めた。	他からの力は加わっていないはずなのに、自然に。おのずから。
\\	蝋燭は独りでに消えた。	ろうそくはひとりでにきえた 
\\	油断なく注意さえしておれば好機はおのずから生まれる。	そのもの自体の力、成り行きに基づくさま。自然に。ひとりでに。おのずと。
\\	その問題はおのずと解決するだろう。	ひとりでに。おのずから。
\\	過去を志向する社会では、人々は過去と伝統にいつまでもこだわり続ける。	かこをしこうするしゃかいでは、ひとびとはかことでんとうにいつもこだわりつづける 
\\	こだわりはお客に伝わってこそ意味をなす。	こだわりはおきゃくにつたわってこそいみをなす 
\\	彼女の問題についての説明は、結局意味をなさなかった。	かのじょのもんだいについてのせつめいは、けっきょくいみをなさなかった 
\\	子供は積み木で遊ぶ。	こどもはつみきであそぶ 
\\	人柄がにじみ出ている。	ひとがらがにじみでている 
\\	臆病な鼓動とともに血はにじみ続ける。	おくびょうなこどうとともにちはにじみつづける 
\\	汗がじわじわとにじみ出る	液体が少しずつしみ込むさま。また、にじみ出るさま。
\\	借金がじわじわと増える	物事がゆっくりと確実に進行するさま。
\\	わたしたちは、さんざんふざけていたから、そろそろ仕事にとりかかる時だ。	物事の程度が著しいさま。
\\	あのいたずらっ子はさんざんたたいてやる必要がある。	人の迷惑になることをすること。悪ふざけ。
\\	敵は戦場からちりぢりに逃げ去った。	まとまっていたものがはなればなれになるさま。ばらばら。
\\	どっちみち私は彼は好きではない。	どういうふうにしても、結局はある状態になることを表す。どちらにしても。いずれにしても。どのみち。
\\	この辞書は簡約版だ。	このじしょはかんやくばんだ 
\\	彼も大人になって、物事を総合的な視野で見られるようになった。	かれもおとなになって、ものごとをそうごうてきなしやでみられるようになった 
\\	工業区と商業区が一体となった総合的な開発区を目指しております。	こうぎょうくとしょうぎょうくがいったいになったそうごうてきなかいはつくをめざしております 
\\	えっへんと大いばりでアリスは胸を張った。	"えっへんとおおおばりであるすはむねをはった 
\\	今引き返すには遅すぎる。	もとへ戻ること。
\\	町は見渡す限りの焼け野原であった。	まちはみわたすかぎりのやけのはらであった 
\\	何年か前には、結核にかかっていると知らされることは死の宣告を聞くのに等しかった。	なんねんかまえには、けっかくにかかっているとしらされることはしのせんこくときくのにひとしかった 
\\	彼女は、美術学校を出ただけあって、絵が上手です。	~から、なるほど  ~から確かに  ~だけに
\\	その店は料理もサービスも素晴らしかった。さすが評判のレストランだけのことはある。	たしかに~からだ
\\	一連の爆発で、その研究施設は瓦礫の山と化した。	いちれんのばくはつで、そのけんきゅうしせつはがれきのやまとかした 
\\	案ずるより生むが易し。	あんずるよりうむがやすし 
\\	彼女の目つきは陰気になった。	かのじょのめつきはいんきになった 
\\	君の写真を申込書に添付することを忘れないでね。	きみのしゃしんをもうしこみしょにてんぷすることをわすれないでね 
\\	上司に相談してからでないと、お返事できません。	先に~しないと  まず~しなければ
\\	本当に困ったときにこそ助け合うのが友情というものだ。	まさに~だ。
\\	どんなこともお金で解決できる、というものではない。	必ずしも~ない  ~わけではない
\\	この国の経済はよくなるどころか、ますます悪くなっている。	~ではなく、まったく反対に
\\	忙しくて、旅行するどころか、テレビを見る暇さえない。	~はもちろん
\\	忙しくてテレビを見る暇もないのだから、旅行どころではない。	全然~できない  ~は、とんでもない
\\	小説を書いても、おもしろくないことには売れませんよ。	~なければ  ~ないと
\\	ストレスがたまったときは、酒を飲まないではいられない。	~しないことは難しい  ~したい気持ちが抑えられない
\\	卒業にあたって、クラスのアルバムを作った。	~(の)とき  ~(の)際  ~にあたり
\\	優勝チームは、体の大きさにかけても作戦のうまさにかけても素晴らしいチームだ。	~についていうと  ~の点で
\\	英語のできない社長にかわって、部長が電話に出た。	(人の)代理に
\\	木の机にかわり、プラスチックの机が多く使用されるようになった。	(物の)代用に
\\	卒業式に際して、市長からを祝いの言葉が卒業生に送られた。	~の時に  ~の場合に
\\	「医者になって欲しい。」という親の期待にこたえて、彼は医者になった。	~に会うように
\\	南に行くにしたがって、気温が高くなる。	~とともなう  ~につれて  ~にともなって  ~と一緒に
\\	弟にしたら、私のようなうるさい姉の存在は、いやでたまらないだろう。	~の立場では  ~にとって  ~からすれば  ~から見れば
\\	彼は有能であるから、その選挙で当選するに相違ない。	【にそういない】 ~に違いない  きっと~だ
\\	冷房中につき、ドアをお閉めください。	~なので
\\	祖母が編んだセーターを見るにつけ、やさしかった祖母を思い出す。	~ときは、いつも  ~ときも、~ときも
\\	彼は何事につけても全力で取り組むので、評判がいい。	~に関して
\\	水不足は農業にとっての大問題である。	~から見た
\\	両親の期待に反して女の子が生まれた。	~に違って
\\	この活動はボランティアによって支えられている。	~で
\\	たいてい遅く帰宅しますが、日によっては6時ごろ帰れる場合もあります。	ある~の場合は
\\	台風による農作物の被害が全国に広がっている。	~で  ~が原因である
\\	明け方近くは冷え込む。	あけがたちかくはひえこむ 
\\	友達によれば、今日の明け方、地震があったそうだ。	~の話では
\\	首相をぬきにしては会議が始められないので、首相が見えるまで待ちましょう。	~なしで
\\	これ、あなたが作ったケーキですか。お世辞ぬきにおいしいです。	~なしに
\\	アルコールぬきのパーティーなんて、つまらないなあ。	~なしの
\\	困難にあっても最後までやりぬく気持ちを捨てないで、がんばりなさい。	完全に~する  最後まで~する
\\	激しい戦いの末に、57対58で私のチームが優勝した。	【のすえに】 ~の結果  ~の後で、とうとう・やっと・ついに
\\	「開放」という美しい言葉のもとに戦争が繰り返される。	~の下で
\\	最近は人間ばかりか猫さえも美容院に行くそうだ。	~だけでなく  ~のみならず  ~ばかりでなく
\\	社内のスキャンダルを話したばかりに会社を首になった。	~だけの理由で・原因で
\\	あの食堂の料理は、味はともかくとして、量が多いので人気があるんだ。	~は今問題にしないが  ~はともかく
\\	国が違うと、言葉と文化はもちろん、人々の考え方まで違う。	~だけでなく(~も)  ~に限らず
\\	彼のサッカー選手としての人気は国内はもとより、広く海外でも高まっている。	~だけでなく(~も)  ~はもちろん(~も)
\\	生活は豊かになったが、その反面、公害で病気になる人が増えた。	~と別の面では
\\	母に言うと心配するから、このことは母に言うまい。	つもりはない  ~(し)ないつもりだ  ~(し)ない
\\	この病気で死ぬまいと思うが、念のため医者に診てもらおう。	いっそう注意するため。確認のため。
\\	彼は帰国したのではあるまいか、最近ずっと顔を見ていない。	~ないだろう
\\	猫が死んだことを子供に言おうか言うまいか迷った。	~するか、しないか
\\	行けるものなら、ヨーロッパへ行ってみたい。でも、僕お金ないし、無理かな。	もし~できたら
\\	パソコンを買ったものの、使い方がぜんぜんわからない。	~のに  ~けれども
\\	若いうちにぜひ外国で1年ぐらいくらしたいものだ。	とても~たいと思っている
\\	ひどい虫歯だね。治しようがない。もう抜くしかない。	~できない  ~する方法がない
\\	子供が4人もいるから、私は毎日がんばって働いているわけだ。	~(の)は当たり前だ  ~(の)はとうぜんの結果だ
\\	そんな難しいことが子供にわかるわけがない。	はずがない  当然~ない
\\	人間は働くために生まれてきたわけではない。	~のではない
\\	試験の前だから、遊びたいけれど遊んでいるわけにはいかない。	(~したいけれど)できない
\\	その店は田舎のレストランのわりに、なかなかしゃれている。	~だけど  ~なのに  ~にしては
\\	子供が生まれたのをきっかけに、煙草をやめることにした。	~(の)機会に  ~(の)ときから  ~を契機に
\\	その戦争を契機として世界経済は悪化した。	【をきかいとして】 ~(の)機会に  ~(の)ときから  ~をきっかけに
\\	もう少し心をこめて歌うと、もっと人に感動を与えられます。	~をなかに入れて  ~を含めて  ~を一緒にして
\\	収入の多少を問わず、だれでもこと施設が利用できる。	~に関係なく  ~にかかわらず
\\	有名な歌手の家に泥棒が入り、現金をはじめ、宝石などが盗まれた。	第一に~、そして
\\	委員会のメンバーは教育問題をめぐって活発に話し合った。	~に関して  ~について
\\	これは歴史上の現実をもとにして書かれた小説です。	~に基づいて  ~を基礎にして
\\	彼らは女王がおいでになるので旗を掲げた。	かれらはじょおうがおいでになるのではたをかかげた 
\\	私はあなたがおいでになるのを楽しみにしています。	「行く」「来る」「居る」の尊敬語。いらっしゃる。
\\	全くあなたのおっしゃる通りです。	「言う」の尊敬語。
\\	お父さんが引退なさるときにはあの人が後を継ぐのです。	「する」「なす」の尊敬語
\\	お目にかかるのは4年ぶりですね。	「会う」の謙譲語。お会いする。
\\	文には普通、主語と動詞がある。	ぶんにはふつう、しゅごとどうしがある 
\\	英語では動詞が目的語の前に来る。	えいごではどうしがもくてきごのまえにくる 
\\	作文は、今日中に書かなければならないのに、まだ書きかけだ。	はじめたが、まだ終わっていない  まだ~している途中だ
\\	彼はまるで父親に合ったことがないかのようだ。	まるで~ようだ・みたいだ  ~にとても近い・似ている
\\	手伝いたいけど、多分足手まといになるよね。	【あしでまとい】 手足にまつわりついて自由な活動の妨げとなること。また、そのようなさま。あしてがらみ。
\\	ぬれたスカートが足にまつわりつく。	からみついて離れない。まとわりつく。
\\	子供が母親にまつわりつく。	そばにいて離れない。いつもつきまとっている。まとわりつく。
\\	足にひもがからみつく。	物のまわりに巻きつく。まといつく。まつわりつく。
\\	野党の足並みがそろう。	【あしなみ】 考え方や行動のそろいぐあい。
\\	足元の明るいうちに手を引きなさい。	【あしもとのあかるいうち】 《日が暮れて足元が見えなくならないうち、の意から》自分が不利な状態にならないうち。
\\	その実業家にはその取引から手を引く勇気がなかった。	関係をなくす。
\\	やくざな稼業から足を洗う。	悪い仲間から離れる。好ましくない生活をやめる。職業・仕事をやめる場合にも用いる。
\\	環境を救うために、みんなが拠出も含めた協力をしなければならないでしょう。	かんきょうをすくうために、みんながきょしゅつもふくめたどりょくをしなければならないでしょう 
\\	赤十字は被災者に食料と医療を分配した。	せきじゅうじはひさいしゃにしょくりょうといりょうをぶんぱいした 
\\	そのような分配された支配はほとんど存在しない。	そのようなぶんぱいされたしはいはほとんどそんざいしない 
\\	ちゃんとペース配分は考えてありますもの。	ちゃんとペースはいぶんはかんがえてありますもの 
\\	食料の供給が不足したので、我々は残された少量を配分しなければならなかった。	しょくりょうはきょうきゅうがふそくしたので、われわれはのこされたしょうりょうをはいぶんしなければならなかった 
\\	ガスストーブは料理するのに最も均等な熱を供給する。	ガスストーブはりょうりするのにもっともきんとうなねつをきょうきゅうする 
\\	男女間に不均等が存在することは許されるべきではない。	だんじょかんにふきんとうがそんざいすることはゆるされるべきではない 
\\	研究開発にお金を割り当てる。	全体をいくつかに分けて、それぞれにあてがう。配分する。わりふる。
\\	費用を頭割りで払った。	【あたまわり】 金品の拠出・分配や仕事の配分をする場合に、人数に応じて均等に割り当てること。
\\	彼は課長に頭を抑えつけられている。	かれはかちょうにあたまをおさえつけられている 
\\	旅行に出かける前に頭を刈っておきたい。	りょこうにでかけるまえにあたまをかっておきたい 
\\	君のために危ない橋を渡る気はないね。一度だって君に助けてもらったことなどないだろ。	【あぶないはしをわたる】 危険を知った上で、その方法で物事を行うことの例え。 危ない橋を渡ることになるが、成功すれば大きな利益が得られる。
\\	忍者ごっこをしよう。	いっしょにある動作のまねをすること、特に子供の遊びについていう。
\\	鼬はその悪臭で知られている。	いたちはそのあくしゅうでしられている 
\\	私は返答に窮した。	わたしはへんとうにきゅうした 
\\	結婚してくれと言われたとき、私は言葉に窮した。	けっこんしてくれといわれたとき、わたしはことばにきゅうした 
\\	この花は強い香りを放つ。	このはなはつよいかおりをはなつ 
\\	それは虎を野に放つようなものだ。	それはとらをのにはなつようなものだ 
\\	鹿をめがけて彼は見事な1発を放った。	しかをめがけてかれはみごとないっぱつをはなった 
\\	一度放たれた石と言葉は呼び戻せない。	いちどはなたれたいしとことばはよびもどせない 
\\	その物体は光を放ちながら南の方角へ飛んで行った。	そのぶったいはひかりをはなちながらみなみのほうがくへとんでいった 
\\	彼の演説は気迫に欠けていた。	かれのえんぜつはきはくにかけていた 
\\	競技者は気迫と自信に満ちている。	きょうぎしゃはきはくとじしんにみちている 
\\	敵の気迫におじけだつ。	恐怖心が生じる。おじけづく。
\\	人前だと気後れして話ができない。	【きおくれ】 相手の勢いやその場の雰囲気などに押されて、心がひるむこと。気おじ。
\\	彼は非常な剣幕で私をにらみ付けた。	【けんまく】 怒って興奮しているようす。いきり立った、荒々しい態度や顔つき。
\\	相手の剣幕に怯む。	【ひるむ】 おじけづいてしりごみする。気後れする。
\\	鼬の最後っ屁だった。	【いたちのさいごっぺ】 《イタチが窮したときに悪臭を放って敵をひるませるところから》困ったときに非常手段に訴えること。
\\	自分の仕事に一心不乱に打ち込みなさい。	【いっしんふらん】 心を一つの事に集中して、他の事に気をとられないこと。
\\	あなたは否でも応でも行かなければならない。	【いやでもおうでも】 承知でも不承知でも。どうしても。なにがなんでも。否が応でも。
\\	牛の小便のようでした。	【うしのしょうべん】 だらだらと長く続くたとえ。
\\	隣り近所に気兼ねする。	【きがね】 他人の思わくなどに気をつかうこと。遠慮。
\\	彼は他人を欺くようなことはしない。	【あざむく】 言葉巧みにうそを言って、相手に本当だと思わせる。言いくるめる。だます。
\\	外見で他人をごまかす。	人目を欺いて不正をする。
\\	やつの巧妙な話に僕は簡単に騙されてしまった。	やつのこうみょうなはなしにぼくはかんたんにだまされてしまった 
\\	その機械は精巧に出来ているので、すぐ壊れる。	そのきかいはせいこうにできているので、すぐこわれる 
\\	9時以降に電話した方が安いですか。	【いこう】 ある時からのち。
\\	彼はそうする事に後ろめたさを感じなかった。	かれはそうすることにうしろめたさをかんじなかった 
\\	彼女は大量の血を目にして恐怖で顔をそむけた。	かのじょはたいりょうのちをめにしてきょうふでかおをそむけた 
\\	彼に涙を見せまいと彼女は顔を背けた。	かれになみだをみせまいとかのじょはかおをそむけた 
\\	病気の子どもの痛ましい泣き声を聞くのは我々には耐えられなかった。	【いたましい】 目をそむけたくなるほど悲惨である。痛々しい。
\\	戦争には悲惨と悲しみが伴う。	【ひさん】 見聞きに耐えられないほどいたましいこと。
\\	自分の目耳で見聞きしていないものは、本当かもしれないが、本当ではないかもしれない。	【みきき】 見たり、人から聞いたりすること。けんぶん。
\\	検分を広める。	【けんぶん】 実際に見たり聞いたりすること。また、それによって得た経験・知識。けんもん。
\\	人間ならだれでもそんな罪悪に対して不快感を持つ。	【ざいあく】 道徳や宗教の教えに背くこと。つみ。とが。
\\	親の言いつけに背く。	【そむく】 取り決めたことや目上の人の考え・命令などに従わずに反抗したり反対したりする。さからう。
\\	目上の人には礼儀正しくしなければならない。	【めうえ】 階級・地位や年齢が自分より上であること。また、その人。
\\	彼は目上の人に対して丁重である。	【ていちょう】 礼儀正しく、手厚いこと。
\\	クック船長はその原住民達の手厚いもてなしに感謝した。	【てあつい】 扱い方やもてなし方が、親切で丁寧である。
\\	私はその家族の親切なもてなしを受けた。	客を取り扱うこと。待遇。
\\	その家は目下建築中である。	【もっか】 ただいま。さしあたり。現在。副詞的にも用いる。
\\	少年にしろ、成人にしろ、悪いことをしたら、厳しい罰を受けるべきです。	~にせよ  ~ても・でも  ~場合でも  どちらの場合も
\\	旅行に行くにせよ、行かないにせよ、早く決めたほうがいい。	~にしろ  ~ても・でも  ~場合でも  どちらの場合も
\\	群馬銀行は経営が危なくなっているとか。預けた金は大丈夫だろうか。	~そうだ  ~ということだ
\\	両親は田舎で二人だけで暮らしている。年をとっているだけに、娘としては心配だ。	~から、なるほど  ~から確かに  ~だけあって
\\	私の預金残高はどうなっていますか。	わたしのよきんざんだかはどうなっていますか 
\\	この薬には悪い副作用はありません。	【ふくさよう】 薬物の、病気を治す作用とは別の作用。有害なことが多い。
\\	いったん彼女が到着すれば私たちは出発できます。	一度。
\\	遅かれ早かれ彼女はショックから立ち直るだろう。	【たちなおる】 悪い状態から、もとの状態に戻る。
\\	隣の家の人々は我々が昨夜大騒ぎをしたので閉口した。	【へいこう】 手に負えなくて困ること。
\\	彼は黒人を見下すのが癖になっている。	【みくだす】 相手をばかにして見る。あなどりみる。見さげる。
\\	虫歯を侮ると命に係わる場合もあるのです。	【あなどる】 軽蔑する。
\\	慣れすぎは侮りのもと。	なれすぎはあなどりのもと 
\\	侮り難いぜ、地元の遊園地。カップルやら家族やら・・・見渡す限り人、人、人。	あなどりがたいぜ、じもとのゆうえんち。かっぷるやらかぞくやら・・・みわたすかぎりひと、ひと、ひと。 
\\	そのいたずら小僧から目を離すな。	【こぞう】 年少の男子を見下していう語。小僧っ子。こわっぱ。
\\	先生は何かと言うと人のことに口出しする。	【くちだし】 他人の話に割り込んで自分の意見を言うこと。差し出口。
\\	彼は会社員として働く一方、作曲家としても活躍している。	~が  ~けれども  しかし~  一つの面では
\\	小さな違反をしただけなのに、警察に呼ばれたうえ、罰金まで払わされた。	~だけではなく、さらに
\\	大変重要なことですから、よく考えた上で返事をください。	~してから
\\	動物を飼う上は、責任を持って世話をするべきです。	~ときは  ~以上は  ~からには
\\	毎年、夏から秋にかけて、台風が何回か日本列島に接近する。	~から~までのあいだ
\\	言葉の数はその辞書のほうが多いけど、例文の多さからいえば、この辞書のほうがいいんじゃないかな。	~の点を見ていうと  ~を考えていうと  ~についていうと
\\	母は、紙一枚からして無駄にはしないで節約する。	~さえ
\\	部長のあの表情からすると、先月の営業成績はあまりよくなかった。	~から判断すると  ~を見て判断すると
\\	日本人一般の考え方からすると、彼の態度は非常識だと言われるかもしれない。	【ひじょうしき】 常識のないこと。常識を欠いていること。
\\	外国語を勉強するからには、その言葉で冗談が言えるようになるまでがんばりたい。	~のだから、それなら  以上・以上は  からは
\\	社会人であるからは、自分の義務と責任を忘れてはいけない。	~のだから、それなら  以上・以上は  からには
\\	彼らはその問題に最終的な決着をつけた。	【けっちゃく】 物事のきまりがついて終わりになること。
\\	洪水が治まり始めた。	【おさまり】 物事のきまりがつくこと。始末。決着。
\\	観測されたデータから見て、火山の噴火はおさまりつつあるようだ。	~についていうと  ~の立場からいうと  ~からいうと
\\	健康のため、エレベーターに乗るかわりに、階段を上ることにしている。	~をしないで  ~(すること)をやめて
\\	自分達の持っている力を試合で出しきれば、きっと勝てますよ。	~きる=完全に~する  全部~(し)終わる
\\	町並みが昔のままに保存され、古い寺も多いことから、その町は「小京都」と呼ばれている。	~ので
\\	一人で悩んでいないで、家族や友達に相談してみることですよ。何かいいアドバイスがもらえるでしょう。	~なさい  ~ほうがいい
\\	彼のことだから、きっと最後までかんばりぬくと思います。	~は、いつもそうだから
\\	息子は、新しいパソコンの前で飽きることなくキーボードをたたいている。	~しないで  ~ずに
\\	嬉しいことに、4月から給料が1割ほどあがる。	非常に~  とても~
\\	ちょっと待たされただけで、なにもそんなに怒ることはないでしょう。	~(する)必要はない  ~(し)なくてもいい
\\	見かけが美しくさえあれば、内容はどうでもいいと言うんですか。	~だけ~ば、それでいい
\\	金額が大きくて私には支払えないので、父に援助を頼まざるをえない。	~(し)なければならない  ~(し)ないわけにはいかない
\\	手足を漱ぐ。	【すすぐ】 汚れを水で洗い落とす。水で清める。
\\	台風の進み方次第では、飛行機が飛ばなくなる恐れもあります。	~によって決まる  ~によって変わる  ~にかかっている
\\	私達は、一緒に暮らしているけれども、法律上は夫婦ではないんです。	~的に  ~の点で  ~の面で  ~に関係することで
\\	合格の知らせを受けた時は、思わず、「やった、やった。」と叫ばずにはいられなかった。	~(し)ないではいられない
\\	春、くしゃみなどのアレルギー症状が起きるのは、主に杉の花粉のせいだと言われる。	【しょうじょう】 病気や怪我の状態。病気などによる肉体的、精神的な異状。
\\	線写真で異状が認められます。	【いじょう】 普通とは違う状態。
\\	寝不足のせいで、今日は頭がぼんやりしている。	~から・ので
\\	頭がぼんやりしているのは、寝不足のせいだ。	~が理由だ
\\	よく寝たせいか、今朝は頭がすっきりしている。	たぶん~(だ)からだろう
\\	日曜日は、寝たいだけ寝ることにしているので、起きるのは昼ごろになる。	全部  みんな
\\	たとえ手術をうけても、完全には治らないだろう。	~(の)場合でも
\\	担当者に電話で問い合わせてみたところ、社員旅行の申し込みはもう終わっていた。	~たら  ~た結果
\\	走り出したとたん、右のひざにピリッと痛みを感じた。	瞬間的に強い刺激を受けてしびれを感じるさま。
\\	店に入ったとたん、いい匂いがした。急にお腹が空いてきた。	~した後、ほとのど同時に
\\	西田先生は授業のたびに、面白い話をしてくださる。	~ときは、いつも同じように
\\	このテストの答案は、間違いだらけだ。	~がいっぱいある  がいっぱいついている
\\	研修旅行のついでに、大阪にいる先輩に会いに行った。	~とき、一緒に  ~の機会を利用して
\\	新しい流行語ができる一方で、古くからあるきれいな言葉が消えつつあるのは残念だ。	今、~(し)ている
\\	目の前の美しい山を眺めつつ、自分の絵の才能がないことを残念に思った。	~ながら
\\	試験の準備を早く始めようと思いつつも、まだ始めていない。	~しているのに
\\	彼女は、飽きっぽい性格で、今までに職場を何回か変えている。	~傾向がある  少し~(の)性質がある  少し~(の)感じがある
\\	彼女を知って以来、僕の人生はバラ色だ。	~てから今までずっと
\\	最近の日本人は礼儀を忘れてしまったのではないかと、残念に思われてならない。	非常に~で、おさえることができない
\\	あの二人は秋に結婚するということだ。	~そうだ(聞いたり、読んだりして知ったこと・情報)
\\	出席、欠席について何も連絡がないということは、欠席するということだろう。	つまり~だ  ~ということがわかる
\\	言い換えると、世の中にはいろいろな人間が必要だということだ。	同じ事柄を他の言葉で言い表す。言い直す。換言する。
\\	以上のことを換言すれば。	【かんげん】 別の言葉で言い表すこと。言いかえること。
\\	政治に対する私の関心は専ら学問上のものです。	【もっぱら】 他はさしおいて、ある一つの事に集中するさま。また、ある一つの事を主とするさま。ひたすら。ただただ。
\\	彼女はその知らせを聞いてただただ泣くばかりであった。	「ただ」を強めていう語。ひたすら。もっぱら。
\\	この川は源を北アルプスに発する。	【みなもと】 川の水などの流れ出るもと。水源。
\\	日本を代表するものといったら、やはり、桜、富士山なんかでしょう。	~を聞いてすぐ頭に浮かぶことは
\\	彼の仕事ぶりといったら、「仕事の鬼」と言われるくらいだ。	~は  ~についていうと
\\	「奥様はとても陽気な方ですね」 「いやあ、陽気というより、うるさいというべきでしょうね。」	~もっと適切にいうと
\\	熱帯の国といっても、朝晩は涼しくて過ごしやすくなる。	~けれども
\\	大丈夫。マニュアルに書いてあるとおりにすれば、うまくいきます。	~ように  ~と同じように
\\	ぐっすり寝ているところに地震が起きて、はっと目が覚めた。	ちょうど~とき  ~(の)状況に
\\	このまま不況からぬけ出せないとすれば、日本の将来に希望はない。	もし~たら・ば・と・なら
\\	ノーベル賞の受賞は、学者として最高の名誉だ。	~の場合で  ~の資格で
\\	その人が親切そうに見えたとしても、本当に親切かどうかはわからない。	~(の)場合でも  たとえ~でも
\\	時代の流れとともに、人の考え方も変化する。	~と一緒に  ~につれて  ~にしたがって
\\	納豆は、食べないこともないんですが、あまり好きじゃないんです。	「全然~ない」というわけではないけれど、でも。100パーセントの否定はしないが、なにか問題があることを表す。
\\	彼らは、貧しいながらも明るく生きている。	~のに  ~(だ)が  ~つつ(も)
\\	あなたなんか嫌い。もう来ないで。	~は  ~とは  ~なんて ~など
\\	台風の日に風雨のなかでテニスをするなど、考えられない。	~は  ~とは  ~なんて  ~なんか
\\	彼女、大人っぽいね。まだ15歳だなんて信じられないよ。	~は  ~とは  ~なんか  ~など
\\	結婚するにあたり、両家の家族が集まって食事をした。	~(の)とき  ~(の)際  ~にあたって
\\	親が忙しいときに限って子供が熱を出す。	特に~(の)場合だけは  特に~ときだけ
\\	ビザを持っている人に限り入国することができます。	~だけ
\\	原子力発電所の周辺住民に限らず遠くの住民も原子力発電の反対運動にさんかした。	【にかぎらず】 ~だけでなく(~も)
\\	わが社に関して詳しく知りたい場合は、ホームページをご覧ください。	【にかんして】 ~に関係して  ~について
\\	彼は怠け者にきまっている。彼が働いているところを見たことがない。	絶対に~だ
\\	森林火災に加え、高速道路の建設で、森はだんだん減ってしまった。	~と  ~だけでなく
\\	大統領は「戦争をしない。」といっている。国民にしても、考えは同じだ。	~も  ~にとっても  ~の場合でも  ~から見ても
\\	子供の希望に沿って公園が作られた。その公園は大人の考えたものとはかなり違っていた。	~から離れないで
\\	あの人は奥さんに対しては「彼女と別れる。」と言い、恋人に対しては「すぐ離婚する。」と言っているそうだ。	~に  ~に向かって
\\	大統領は新しい外交方針について国民に説明した。	~に関して  ~につけても
\\	工場の家事はようやく消されたが、原因についてはまだ調査が進んでいない。	~に関しては
\\	私は東京だけでなく、地方の生活や習慣についても知りたいと思います。	~に関しても
\\	旅行日程についてのご質問がある方は手をあげてください。	~に関しての
\\	冬の凍った道路は、自転車にはもちろん、歩く人にとっても大変危険なものだ。	~から見ても
\\	秋が深まるに伴って、山の紅葉が進む。	~とともに  ~につれて  ~にしたがって  ~と一緒に  ~と一緒の
\\	仕事は予定表に基づいて、きちんと進められなければならない。	~を基本にして
\\	国家である以上、法律に基づく制度が定められているはずだ。	~を基本にする
\\	法の定めるところに従う。	【さだめる】 よりどころや従うべきものとしてきめる。制定する。
\\	彼の趣味は音楽、歴史、政治など多くの分野にわたる。	~に広がる  ~の間続く
\\	首相の記者会見は2時間にわたって行われた。	~の間(の)  ~に広がって  ~の間続いて
\\	人間のみならず動物もストレスを感じるという。	~だけではなく(~も)
\\	この花は色がきれいなばかりでなく、香りも素晴らしいですよ。	~だけでなく  ~のみならず  ~ばかりか
\\	酒はともかく、たばこは肺ガンになるからやめるべきだ。	~は今問題にしないが  ~はともかくとして
\\	わが社の車は、外国向けのものは左ハンドルで、国内向けのものは右ハンドルです。	【むけの】 ~を対象に  ~のためを考えて
\\	いろいろな機能のあるパソコンより、初心者向きの、使い方の簡単なものがほしい。	~に合う  ~に適する
\\	彼女は頭もよければ、顔もきれいでみんなが憧れるわけです。	~も~し、~も~
\\	娘が結婚する日の父親の心理には複雑なものがある。	
\\	深い感じること
\\	親の言うことは素直に聞くものだよ	一般的なこと・当然のこと・普通のことに使う。
\\	世界が互いに文化を理解しあうということは難しいものだ。	本当に~  実に~
\\	子供のころ、悪いことばかりして親を怒らせたものだ。	よく~(し)た  いつも~だった
\\	すみません、お待たせして。電車が遅れたもので・・・。	~から  ~ので
\\	諦めるものか。最後まで頑張りぬくぞ。	絶対~したくない
\\	子供でも月に100万円は楽に稼げるって? そんな仕事があるものか。	絶対~ないと思う
\\	世界から戦争がなくなるように運動している平和団体が沢山ある。	~ために
\\	新幹線をはじめとする交通機関が雪のためストップしています。	
\\	をはじめとするB」  
\\	は代表的なもの。
\\	御辞儀の習慣はその島の人達に特有なものである。	【おじぎ】 頭を下げて礼をすること。頭を下げてするあいさつ。
\\	ロケットは静かに着地したので器具を壊さずにすんだ。	ロケットはしずかにちゃくちしたのできぐをこわさずにすんだ 
\\	いったん着地した怪物は、瞬間再び跳躍して私の頭上にいた。	いったんちゃくちしたかいぶつは、しゅんかんふたたびちょうやくしてわたしのずじょうにいた 
\\	そんな無茶はよせ。	そんなむちゃはよせ 
\\	わずかな額のことで言い争うのはよせ。	わずかながくのことでいいあらそうのはよせ 
\\	痛い!由紀子!痛いよ。グーで殴るのはよせよ!	いたい!ゆきこ!いたいよ。グーでなぐるのはよせよ! 
\\	女々しい振る舞いはよせ。	【めめしい】 態度や気性が柔弱である。意気地がない。主に男性についていう。
\\	柔弱な気質の人だ。	【にゅうじゃく】 気力や体質が弱々しいこと。
\\	彼女はその気性を祖父から受け継いでいる。	【きしょう】 生まれつきの性質。気質。きだて。気象。
\\	彼は雄々しく振る舞った。	【おおしい】 男らしいさま。勇ましい。
\\	母親になってもからきし意気地がない。	【いくじ】 自分自身や他人に対する面目から、自分の意志をあくまで通そうとする気構え。意地。いきじ。
\\	サービス産業には、通信、運輸、流通、金融をはじめものの生産には関係ないたくさんの分野が含まれる。	サービスざんぎょうには、つうしん、うんゆ、りゅうつう、きんゆうをはじめもののせいさんにはかんけいないたくさんのぶんやがふくまれる 
\\	貨幣は銀行制度を通じて流通する。	かへいはぎんこうせいどをつうじてりゅうつうする 
\\	英国は、1971年にその貨幣制度を十進法に移行させた。	【じゅうしんほう】 基数を10とし、0から9まで10個の数字を使い、10倍ごとに上の位に上げていく表し方。日常生活で最も使われている。
\\	農場生活から都市の生活への移行は困難なことが多い。	【いこう】 ある状態から他の状態へ移っていくこと。
\\	良い経営陣なら正当な要求に耳を貸すものだ。	よいけいえいじんならせいとうなようきゅうにみみをかすものだ 
\\	経営陣は果たして本気で我々のボーナスについて考えようとしているのか、それとも単に口先だけだったのだろうか。	けいえいじんははたしてほんきでわれわれのボーナスについてかんがえようとしているのか、それともたんにくちさきだけだったのだろうか 
\\	でも、それで士気が下がっては本末転倒ではないかと。	【しき】 兵士の、戦いに対する意気込み。また、人々が団結して物事を行うときの意気込み。
\\	団結すれば、どんなことでもできる。	【だんけつ】 多くの人が共通の目的のために一つにまとまること。
\\	我々が手に手を取って一致団結すれば、きっとこの不況の時代も乗り切れるだろう。	われわれはてにてをとっていっちだんけつすれば、きっとこのふきょうのじだいものりきれるだろう 
\\	兵士たちはその歌に士気を鼓舞された。	へいしたちはそのうたにしきをかぶされた 
\\	将校は部下を鼓舞して勇気を出させた。	しょうこうはぶかをかぶしてゆうきをださせた 
\\	一週間熟考した末に新しい計画を考えついた。	いっしゅうかんじゅっこうしたすえにあたらしいけいかくをかんがえついた 
\\	長々と協議した末に、売り手と買い手は結局折り合った。	ながながときょうぎしたすえに、うりてとかいてはけっきょくおりあった 
\\	確かに、脳と意識の関係はもともと因果関係ではないのだと考える学者もいる。	【いんが】 原因と結果。また、その関係。
\\	ここで迷わず迂回路を取ることにする。	ここでまよわずうかいろをとることにする 
\\	車を迂回路に導く。	【みちびく】 道案内をする。案内して目的の所に連れていく。
\\	彼はとても親切なので彼女の道案内をした。	【みちあんない】 道を知らない人を導いて連れていくこと。また、その人。
\\	コーチはチームを勝利に導いた。	コーチはチームをしょうりにみちびいた 
\\	その赤ん坊はちょいちょい母親を困らせる。	間を置いて同じことが何度も繰り返されるさま。度々。おりおり。ちょくちょく。
\\	そのクーデターの間接的な情報しか得ていない。	【かんせつ】 遠回しに示すこと。
\\	遠回しの言い方はやめて、要点をお話ください。	【とおまわし】 直接的な表現を避けて、それとなく言うこと。
\\	駅はその二つの町の中間にある。	【ちゅうかん】 物と物との間の空間や位置。
\\	君の家と私に家の中間で落ち合うことにしよう。	【おちあう】 一つ所で出合う。また、打ち合わせておいて、ある場所で一緒になる。
\\	彼は世界平和のために超人的な努力をした。	【ちょうじん】 並み外れた能力をもった人。スーパーマン。
\\	その事故を犠牲にしてもはじまらない。	無意味
\\	彼は風の音に耳を澄ましながら、長い間座っていた。	【みみをすます】 聞こうとして注意を集中する。耳をそばだてる。
\\	日本では自分でビールを注がずに誰かが注ぐ。	【つぐ・そそぐ】 流れ入る。流れ込む。
\\	これはとてもまろやかなコーヒーだ。	口あたりが柔らかいさま。味が穏やかなさま。
\\	彼女はそのビールをグイと飲みほした。	そのまま、すぐに、一息に、勢いよく。
\\	新しい花瓶の花が彼女の気分を爽快にした。	【そうかい】 さわやかで気持ちがよいこと。
\\	私は走る前に、体を解すのにちょっと体操をする。	【ほぐす】 こりかたまっているものをやわらかくする。
\\	じゃ、せめてテーブルの準備でもしましょう。	不満足ながら、これだけは実現させたいという最低限の願望を表す。少なくとも。十分ではないが、これだけでも。
\\	新型は来年はじめ市販される見込みです。	【しはん】 市場・商店で普通に売ること。
\\	詩人としての彼は20代が花盛りだった。	【はなざかり】 物事が非常に盛んであること。また、その時期。
\\	私たちは囲炉裏の周りに座りました。	【いろり】 室内の床の一部を四角に切り抜いて火をたくようにした場所。暖房・煮炊きに用いる。炉。
\\	私は解読プログラムを持っていません。	【かいどく】 古文書・暗号・古代言語などを読み解くこと。
\\	こんなに安い値段ならこのペンは本当にお買い得だ。	【おかいどく】 割安で、買ったほうが得であること。⇔買い損
\\	シーズンオフはホテルは割安だ。	【わりやす】 品質や分量の割合からみると安価であること。⇔割高。
\\	ワインがお気に召すといいのですが。	【おきにめす】 「気に入る」「好む」の尊敬語。
\\	日本のあるインドカレー屋のカレーの味って、やっぱり日本人の口に合わせて作られているよね。	【くちにあう】 飲食物の味が好みに合う。
\\	彼の謙遜さは賞賛に値する。	【けんそん】 へりくだること。控え目な態度をとること。
\\	彼女は三味線による新しいジャズの演奏法を始めた。	【しゃみせん】
\\	賃貸アパートをさがしています。	【ちんたい】 賃料を取り、物を相手方に貸すこと。賃貸し。
\\	見渡す限り広々とした草原で、ところどころに小さな森があった。	【ひろびろ】 非常に広く感じられるさま。
\\	できることなら俗語は使わないようにしなさい。	【ぞくご】 あらたまった場面では使われないような卑俗な言葉。「てめえ」「やばい」の類。スラング。
\\	2ヶ月分の敷金を入れていただきます。	【しききん】 不動産、特に家屋の賃貸借にさいして賃料などの債務の担保にする目的で、賃借人が賃貸人に預けておく保証金。しきがね。
\\	その家は南向きなので、とても日当たりがよい。	【ひあたり】 日光が当たること。また、その当たる場所や、当たりぐあい。
\\	すぐに一人住まいに慣れますよ。	すぐにひとりずまいになれますよ 
\\	荷造りを除いては、することはなにも残っていない。	【にづくり】 品物を袋や箱に詰めたり、紙や布などで覆い囲んだり、縄をかけたりしてひとまとめにし、荷物にすること。
\\	電子レンジの調子がおかしい。修理するより買い換えるほうがいいかな。	【かいかえる】 新しく買い入れて、今までの物ととりかえる。
\\	省エネのためにコンビニの24時間営業を廃止しろ!	【しょうエネ】 石油・電力・ガスなどのエネルギーを効率的に使用し、その消費量を節約すること。省エネルギー。
\\	これらの古雑誌を処分して下さい。	【しょぶん】 不要なものや余分なものなどを、捨てる、売り払う、消滅させる、など適当な方法で始末すること。
\\	今日は不燃物のゴミの日です。	【ふねんぶつ】 燃えないこと。また、燃えにくいこと。
\\	彼女は窓からステッカーを引き剥がした。	【はがす】 付着しているものを剥ぎ取る。めくり取る。
\\	ミルクは全部こぼれた。	【こぼれる】 液体、粉末、粒状の物などが容器などから外へ出る。すきまなどから漏れ落ちる。
\\	若い親はしばしば子どもを甘やかす。	【あまやかす】 子供などを厳しくしつけないで、わがままにさせておく。
\\	アリスは厳格な母を怖がっていた。	【げんかく】 規律や道徳にきびしく、不正や怠慢を許さないこと。
\\	彼は子供をしつけるのに厳格だった。	教えて身につけさせる。
\\	彼女は赤ん坊をおんぶしていた。	子供などを背負うこと、また、背負われることをいう幼児語。
\\	子供を負ぶって買い物に行く。	【おぶう】 背負う。おんぶする。
\\	彼は少年時代はとてもやんちゃだった。	子供がわがままにふるまうこと。子供がだだをこねたりいたずらしたりすること。また、そのさまやそのような子供。やんちゃん。
\\	彼女たちが庭で燥いでいる。	【はしゃぐ】 調子にのってふざけ騒ぐ。
\\	子供たちがどたばたとはしゃぎ回る。	騒がしく走り回ったりあばれたりする音や、そのさまを表す語。
\\	少年たちは2階でどたばた暴れていた。	【あばれる】 乱暴な行動をする。被害が出るほど乱暴に動く。
\\	私の母は行儀作法にやかましい。	きびしい。
\\	ジョーはやかましい隣人と口をきく間柄ではなかった。	声や物音などが騒がしい。うるさく、不快である。
\\	ハワイでは1年中海水浴が楽しめる。	【かいすいよく】 海に行って泳いだり、日光浴をしたりすること。
\\	ロッククライミングも、海で深く潜ることもしたし、インドネシアの熱帯雨林で眠ったこともある。	【もぐる】 水の中にくぐって入る。
\\	潜水して泳げるかい。	【せんすい】 水中にもぐること。
\\	風船が木の枝に引っ掛かる。	【ひっかかる】 物にかかってそこに止まる。
\\	彼らは池でボートを漕いでいる。	【こぐ】 櫓(ろ)や櫂(かい)を使って水をかき、舟を進める。
\\	かわるがわるボート漕ごう。	順番に代わり合って。交代に。
\\	じゃあ2人乗りを許してやるからお前漕げ、俺が荷台に乗るから。	【こぐ】 足や腰を曲げたり伸ばしたりして、乗り物を進めたり振り動かしたりする。
\\	彼女は胸の中で心臓が引っ繰り返る思いだった。	【ひっくりかえる】 上下・表裏などが反対になる。さかさまになる。裏返る。
\\	税金のために彼は銀行口座を個別にしておきたかった。	ぜいきんのためにかれはぎんこうこうざをこべつにしておきたかった 
\\	ハンバーガーばかり食べていると、栄養が偏るよ。	【かたよる】 ある部分・方面に集中して、全体のつりあいを欠いた状態になる。不均衡になる。偏する。
\\	支出は合計20万円になった。	【ししゅつ】 ある目的のために自分の所有する金銭や物品を支払うこと。また、その金品。
\\	彼の発明は製品を製造する際の時間を削減する。	【さくげん】 現にあるものを、けずってへらすこと。
\\	したがって生産費を削る必要がある。	【けずる】 全体からその部分を取り除く。削除する。
\\	私は自炊に慣れている。	【じすい】 自分で食事をつくること。
\\	機械は多くの人手を省く。	【ひとで】 働く人。働き手。
\\	親の中には、子供たちが数について基本的なことを十分教わっていないのではないか、と心配しているものも多い。	【おそわる】 教えを受ける。教えてもらう。
\\	概して、人生という競技はやりがいがある。	そのことをするだけの価値と、それにともなう気持ちの張り。
\\	その生徒は3回連続して授業をサボった。	【サボる】 怠ける。怠けて休む。
\\	あいつ、上司にゴマをすってやがる。	【ゴマをする】 他人にへつらって自分の利益を図る。
\\	ペンをとる前に考えをまとめる必要がある。	ばらばらのものを集めてひとかたまりのものにする。
\\	張り切って仕事をする。	【はりきる】 元気や気力が満ちあふれる。意気込む。
\\	もうこれ以上生徒を引き受ける時間はない。	【ひきうける】 自分が責任をもってその物事を受け持つ。
\\	彼らは会合の打ち合わせをした。	【うちあわせ】 前もって相談すること。下相談。
\\	電話を取り次ぐ。	【とりつぐ】 間に立って、一方から他方へ用件などを伝える。
\\	パソコンを起動する。	【きどう】 動きや働きを起こすこと。特に、機器類が運転を開始すること。始動。
\\	パソコンを再起動する。	【さいきどう】 コンピューターや周辺機器の使用を中止し、起動しなおすこと。リブート。リスタート。
\\	ここで改行しなさい。	【かいぎょう】 文章の中で行を新しくすること。行を変えること。また、そのように活字を組むこと。
\\	資料を郵便で取り寄せる。	【とりよせる】 注文して送らせたり、持って来させたりする。
\\	イラストを挿入する。	【そうにゅう】 中にさし入れること。中にはさみこむこと。
\\	あ、上書きしちゃった。	【うわがき】 コンピューターのファイルや記録メディアにデータを記録する際、もとのデータの上に新しいデータを書き込むこと。または、文字入力の際に、もとの文章の上に新たに書き込むこと。オーバーライト。重ね書き。
\\	彼は二国間について論評した。	【ろんぴょう】 ある物事の内容・結果などを論じ、批評すること。また、その文章。
\\	彼は作文の余白に自分の論評を書いた。	【よはく】 字や絵などが書いてある紙面で、何も記されないで白く残っている部分。
\\	印刷がずれる。	元あったところから、少しすべり動いて移る。あるべき位置から少し動いたり、基準の位置に合わない状態になる。
\\	新制度に切り替える。	【きりかえる】 今までのものをやめて別のものにする。
\\	この停戦が世界平和に役立つことを私達はみな望んでいる。	【ていせん】 交戦中の両軍が何らかの目的のため、合意の上で一時的に戦闘行為を中止すること。
\\	バスに傘を忘れるとは彼女はそそっかしい。	態度や行動に落ち着きがない、あわて者である。不注意で、とかく失敗をしがちである。軽はずみである。
\\	彼は頼もしい男だ。	【たのもしい】 信頼できる。頼みにできて心強い。
\\	君が味方になってくれれば何よりも心強い。	【こころづよい】 頼りになるものがあって安心である。心丈夫だ。気強い。
\\	回りくどい言い方はやめてはっきり言ってよ!	【まわりくどい】 遠回しでわずらわしい。
\\	そんなふうに言うなんて卑怯だ。	【ひきょう】 勇気がなく、物事に正面から取り組もうとしないこと。正々堂々としていないこと。
\\	君のお姉さんは王女のように上品な顔をしている。	【じょうひん】 品質のよいこと。また、高級品。⇔下品。
\\	彼の使った下品な言葉はとてもほかの人に伝えられない。	【げひん】 品格・品性が劣ること。卑しいこと。⇔上品。
\\	彼は卑しい振舞いをした。	【いやしい】 品位に欠けている。下品だ。
\\	飛行機の辺りで人の動きが慌わただしかった。	【あわただしい】 次から次へと用事や変化があって落ち着かない。
\\	全くありふれた人間にすぎない。	どこにでもある。ざらにある。普通であって珍しくない。
\\	彼はいわゆるありふれた親馬鹿ではない。	【おやばか】 わが子かわいさのあまり、子供の的確な評価ができないで、他人から見ると愚かに思える行動をすること。また、その親。
\\	ボクシングは必ずしも荒っぽいスポーツではない。	【あらっぽい】 行動や性格にやさしいところがないさま。 乱暴。
\\	そのニュースを聞いたとたんに彼女は真っ青になった。	【まっさお】 血の気がひいて顔色が悪いこと。青ざめること。
\\	この派手な服は私には向きません。	【はで】 姿・形・色彩などが華やかで人目をひくこと。⇔地味。
\\	華やかな都会での生活です。	【はなやか】 花が開いたように、明るく人目を引きつけるさま。
\\	就職の面接には派手なネクタイよりも地味なネクタイの方が好ましい。	【じみ】 形や模様などにはなやかさがなく、目立たないこと。
\\	パイプをくわえたその紳士は有名な評論家です。	【くわえる】 口に軽く挟んで支える。
\\	ねずみが齧って壁に穴をあけた。	【かじる】 かたい物の端を歯でかむ、また、かみとる。
\\	みずみずしい桃をかじった時の味が何とも言えません。	【もも】
\\	何とも信じられないほどの仕事を彼はしたもんだね。	【なんとも】 言葉にいえないほど程度がはなはだしいさま。
\\	彼はなにか彼女の耳にささやいた。	【ささやく】 小さな声で話す。ひそひそと話す。
\\	かごをぶら下げて買物に行く。	【ぶらさげる】 無造作に手にさげて持つ。
\\	水道の蛇口をひねる。	指先でつまんで回す。
\\	私は石につまずいて、足首をひねってしまった。	【つまずく】 歩いていて、誤って足先を物に突き当ててよろける。けつまずく。
\\	水たまりを跨ぐ。	【またぐ】 足を開いて物の上を越える。
\\	少し離れてみると、その岩は人がしゃがんだ姿に見える。	【しゃがむ】 ひざを曲げ、腰を落として姿勢を低くする。
\\	私は彼が壁にもたれて座っているのに気づいた。	【もたれる】 人や物に自分のからだの重みをあずける。寄りかかる。
\\	きのう銀座でロバートとすれ違ったよ。	【すれちがう】 触れ合うほど近くを反対方向に通りすぎる。
\\	赤ん坊は歩く前にはう。	手足を地面・床などにつけて進む。腹這いになって進む。また、腹這いになる。
\\	初日の出を拝む。	【おがむ】 神仏などに、手を合わせ、頭を下げて祈る。
\\	あのいじめっ子を、遣っ付けておいで。	【やっつける】 打ち負かす。
\\	君は、いつも私の服をけなすのだから。	ことさらに悪い点を取り上げて非難する。くさす。
\\	俺のことはほっといてくれ!	おれのことはほっといてくれ! 
\\	その男は私の足を踏んだのにわびることさえしなかった。	自分の非を認めて、相手の許しを請う。あやまる。
\\	先輩は、嫌味なくらいの完璧超人でしたからね。	【いやみ】 人に不快な思いを与える言動。あてつけや皮肉。また、それによって不快感を与えるさま。
\\	あの人はあまり威張るから好きになれない。	【いばる】 威勢を張って偉そうにする。えばる。
\\	権力者の威勢に恐れをなす。	【いせい】 人を恐れ従わせる力。
\\	一杯飲んで威勢をつける。	【いせい】 言語や動作に活気があること。意気の盛んなこと。
\\	その子供たちはいつも親ともめる。	争いが起きてごたごたする。
\\	彼らはその制度を改めた。	【あらためる】 新しくする。古いもの、旧来(きゅうらい)のものを新しいものと入れ替える。
\\	男子学生たちは引っ切り無しにふざけあっていた。	【ひっきりなし】 絶え間なく続くさま。切れ目のないさま。
\\	彼はふざけた調子でそう言った。	【ふざける】 おどけたり冗談を言ったりする。
\\	彼女は左手の薬指にダイヤの指輪をはめていた。	【くすりゆび】 親指から数えて4番目の指。
\\	彼の鞄が重くて、更に悪いことには、片足のかかとにまめができてしまっていた。	足の裏の後部、足首の下にあたる部分。くびす。きびす。
\\	かかとがすり減る。	靴など、履き物(はきもの)の後部。
\\	あんなにかかとの高い靴はいてたら、すぐに捻挫しちゃうよな。	【ねんざ】 手や足などの関節に無理な力がかかり、外れかかって靭帯(じんたい)や腱(けん)が損傷された状態。痛み・腫(は)れなどを伴う。
\\	アキレスはかかとを除いては不死身だった。	【ふじみ】 不死であること。どんな病気・苦痛・傷・打撃にも耐えうるからだであること。また、そのからだやそのさま。
\\	混雑したバスの中で私は誰かに爪先を踏まれた。	【つまさき】 足の指の先。
\\	一日中家に閉じ籠もるのは健康に良くない。	【とじこもる】 家や部屋などに入ったまま、外へ出ないでいる。
\\	肩をもんであげましょう。	【もむ】 両手の間に挟んでこする。また、両手をこする。
\\	目をこする。	物を他の物に強く押し当てたままで動かす。また、そのように繰り返し続けて動かす。
\\	もうこの失恋の痛みに堪えることができない。	【こらえる】 苦しみなどに、耐えてがまんする。しんぼうする。
\\	肌に艶がある。	【つや】 なめらかで張りがあり美しいこと。
\\	湿疹も伝染性だって。	【しっしん】 皮膚の表面にできる炎症。赤くなったり、ぶつぶつができたりして、かゆみを伴う。
\\	部屋は全部塞がっている。	【ふさがる】 他のものに占められていて、使うことができない。いっぱいであきがない。
\\	家が建って空き地が塞がる。	【ふさがる】 あいていた箇所が詰まる。すき間や穴がなくなる。
\\	傷口が塞がる。	【ふさがる】 開いていたものが閉じた状態になる。
\\	つい話し込んでしまった。	そのつもりがないのにしてしまうさま。うっかり。思わず知らず。
\\	何度やってもどうもうまくいかない。	あれこれ考えたり試したりしてもなかなか満足できない気持ちを表す。
\\	どうも調子がおかしい。	物事の原因や理由がはっきりわからない気持ちを表す。
\\	なんとか暮らしていける。	完全・十分とはいえないが、条件・要求などに一応かなうさま。かろうじて。どうにか。
\\	私はなんとなく彼の家を見つけた。	言動などに、はっきりとした理由・目的がないさま。なんとはなしに。なぜか。なんだか。
\\	どうにか助かった。	かろうじて。やっとのことで。
\\	わたしはどうにかそこに着くのに間に合った。	まがりなりにも。なんとか。
\\	いずれ改めてお伺いいたします。	あまり遠くない将来をいう。そのうちに。近々。
\\	君の努力はいずれ報われるだろう。	【むくわれる】 受けた事に対して、それに見合う行為を相手に行う。むくう。
\\	改めて、日本の文化や歴史に触れてみるのもいいものだと思った。	【あらてめて】 再び新しく行うさま。もう一度。別の機会に。
\\	人間はいずれ死ぬのだ。	いつか。結局。
\\	ただちに大阪に行ってもらいたい。	間に何も置かないで接しているさま。直接。じかに。
\\	英国大使は大統領とじかに会見することを要求した。	間にほかのものを入れないで直接にするさま。ただちに。
\\	彼はつねに黒眼鏡をかけている。	どんな時でも。いつも。絶えず。
\\	兄はしきりに名声を望んでいる。	程度・度合いが著しいさま。むやみ。やたら。
\\	しきりにほめる。	何度も。
\\	彼はしょっちゅう私を訪れる。	いつも。しじゅう。
\\	私は始終あなたのことを考えています。	【しじゅう】 絶え間ないさま。頻繁に行われるさま。いつも。しょっちゅう。
\\	この部屋はほぼ十分な広さだ。	全部あるいは完全にではないが、それに近い状態であるさま。だいたい。おおよそ。
\\	彼女のおおよその年齢しか知らない。	細部にこだわらず概略を判断するさま。だいたい。大ざっぱに。およそ。
\\	その穴は直径およそ5フィートだ。	大まかに言って。だいたい。約。
\\	何だ、まだほんの子供じゃないか。	次にくる言葉が取るに足りないものであることを表す語。まったくわずかの。
\\	私はたった3千円しか持っていない。	数量が少ないことを強調するさま。わずか。ほんの。
\\	ヘレンの体重はせいぜい40キロだ。	できるだけ多く見積もってもその程度であるさま。たかだか。
\\	彼は私のしたことにいちいち文句を言う。	一つ一つの物事。それぞれ。めいめい。
\\	彼の言うことはいちいち頭に来る。	【あたまにくる】 怒りで興奮する。かっとなる。
\\	その子供達はそれぞれ賞を獲得した。	複数の物・人の、ひとつひとつ・ひとりひとり。おのおの。めいめい。
\\	生徒はめいめい自分の机を持っている。	ひとりひとり。それぞれ。おのおの。
\\	おのおのの建物のは名前がある。	多くのもののそれぞれ。めいめい。
\\	その知らせを聞いて、皆シーンとしてしまった。	物音一つ聞こえないようすを表す語。もの静かなさま。
\\	群衆が競技場からどっと流れ出した。	たくさんの人や物が一時に押し寄せるさま。
\\	多くの観光客がその島に押し寄せた。	【おしよせる】 激しい勢いで迫る。
\\	波また波が岸辺に押し寄せた。	【きしべ】 岸に沿った所。岸のあたり。
\\	人影がすっと消える。	すばやく、とどこおらずに動作をするさま。または、変化が起こるさま。
\\	今週は大いに書きました。	【おおいに】 非常に。はなはだ。たくさん。
\\	中古車にしてはこの値段はやや高い。	数量、程度などが少しであるさま。
\\	その子供にいくらか同情した。	数量・程度があまり多くないさま。多少。
\\	部屋は割合にきれいだ。	【わりあい】 他の物事や場合に比べてそれらの程度を超えているさま。比較的。割に。
\\	僕は君が欠点を持っているので、なおいっそう愛してる。	ぼくはきみがけってんをもっているので、なおいっそうあいしてる 
\\	むしろ名誉ある死を選びたい。	二つを比べて、あれよりもこれを選ぶ、また、これのほうがよりよいという気持ちを表す。どちらかといえば。
\\	私が黙っていたので彼女は余計に腹をたててしまった。	【よけい】 程度・分量がさらに増すさま。もっと。なおさら。
\\	船が徐々に見えて来た。	【じょじょ】 少しずつ進行したり変化したりするさま。
\\	彼はさらにお金を要求した。	同じことが重なったり加わったりするさま。重ねて。加えて。その上に。
\\	彼女はさらにもっと美しい。	今までよりも程度が増すさま。前にも増して。いっそう。ますます。
\\	酒をぐっと飲み干す。	瞬間的に力を入れるさま。また、物事をひと息に行うさま。
\\	値段は去年に比べてぐっと高い。	状態の程度が今までと大きく隔っているさま。一段と。ぐんと。
\\	お母様にくれぐれもよろしく。	何度も心をこめて依頼・懇願したり、忠告したりするさま。
\\	これで一応でき上がりだ。	十分ではないが、ひととおり。大略。
\\	どのくらいでこの洗濯物は出来上がりますか。	【できあがる】 すっかりできる。完成する。仕上がる。
\\	彼は、いわば迷える子羊だ。	言ってみれば。たとえて言えば。
\\	彼はいわゆるりっぱな紳士だ。	世間一般に言われる。俗に言う。よく言う。
\\	まさか彼に会えるとは思わなかった。	ある事態の生じる可能性を強く否定したり、実現することが意外だと思う気持ちを表わす。
\\	これはまさに私が読みたかった本だ。	ある事が確かな事実であるさま。まちがいなく。本当に。
\\	彼女はまさに出発するところである。	実現・継続の時点を強調するさま。ちょうど。あたかも。
\\	彼らはあたかも蟻のように働いた。	あるものが他によく似ていることを表す。まるで。まさしく。ちょうど。
\\	北海道には一遍も行ったことがない。	【いっぺん】 一回。一度。
\\	今こそ一気に取引をまとめるときだ。	【いっきに】 いっぺんに物事を行うさま。
\\	思い切り遊びたい。	【おもいきり】 満足できるまでするさま。思う存分。
\\	彼はスキーを思う存分楽しんだ。	【おもうぞんぶん】 満足がいくまで。思いきり。
\\	私は思い切ってそこに行った。	【おもいきって】 ためらう気持ちを振り切って物事をするさま。決心して。
\\	私は彼を気の毒に思わずにはいられない。	【おもわず】 思いのほか。意外。案外。
\\	彼女はうまくゆかなかったが、なにしろ初めてのことだったからね。	どんな事実や事情があっても、という気持ちを表す。なんにせよ。とにかく。
\\	来週はなにかと忙しい。	一つの物事に限定しない気持ちを表す。何やかやと。あれやこれやと。いろいろと。
\\	状況は相変わらずそのままだ。	【あいかわらず】 今までと変わったようすが見られないさま。以前と同じように。
\\	とりあえず母に合格を知らせる。	ほかのことはさしおいて、まず第一に。なにはさておき。
\\	寒さが一層厳しくなる。	【いっそう】 程度がいちだんと進むさま。ひときわ。ますます。
\\	そんな絵ならいっそ掛けないほうがましだ。	中途半端な状態を排して思いきったことを選ぶときに用いる。とやかく言わないで。むしろ。いっそのこと。
\\	今に後悔するぞ。	やがて。間もなく。
\\	今にも嵐になりそうだ。	目前に何かが起こりそうなさま。すぐにも。今まさに。
\\	ただ今すべて満席です。	今ちょうど。
\\	どうやら誤解があるようだ。	確実ではないが、なんとなく。
\\	だめだろうと思っていたが果たして失敗だった。	【はたして】 結末が予期したとおりであるさま。思ったとおり。案の定。
\\	どうせやるからには上手にやるようにしなさい。	経過がどうであろうと、結果は明らかだと認める気持ちを表す語。いずれにせよ。結局は。
\\	彼には欠点があるからかえって好きだ。	予想とは反対になるさま。反対に。逆に。
\\	私は彼を訪ねたが生憎留守だった。	【あいにく】 ぐあい悪く。
\\	我々はあくまでも闘う。	物事を最後までやりとおすさま。徹底的に。
\\	空はあくまでも青い。	どこまでも。全く。
\\	なんだかめまいと吐き気がします。	物事がはっきりしないさま。原因・理由などがよくわからないさま。なんとなく。なぜか。
\\	なぜか、急ぐ気は起こらない。時間は、十分ある。	理由・原因がはっきりしないさま。どういうわけか。なぜだか。なんとなく。なんだか。
\\	町並みからひときわ高く昔のお城が立っている。	他と比べて特に目立っているさま。一段と。
\\	彼は物置の中で偶然いくつかの古文書を見つけた。	【ものおき】 当面必要としない器具などを入れておく場所。また、そのための小屋。
\\	月日がどんどん過ぎていったが、彼の消息は何も聞こえてこなかった。	【つきひ】 過ぎていく時間。時日。としつき。
\\	後日伺います。	【ごじつ】 その日よりあとの日。ある出来事よりもあとの日。他日。
\\	道路上の大小の石に気をつけなさい。	【だいしょう】 大きいことと小さいこと。大きいものと小さいもの。
\\	彼はその大木を斧で切り倒した。	【たいぼく】 大きな木。大樹。巨木。
\\	この映画はまさしく不朽の名作である。	【ふきゅう】 朽ちないこと。いつまでも価値を失わずに残ること。
\\	朽ちない物は何もない。	【くちる】 腐って形がくずれたりぼろぼろになったりする。
\\	彼は見本請求の手紙を書いた。	【みほん】 商品などの質や形状を買い手に知らせるために示す品。また、そのために作った物。サンプル。
\\	この地方の名物料理がありますか。	【めいぶつ】 その土地で名高い産物。名産。
\\	私をそのあだ名で呼ばないでくれ。	【あだな】 本名とは別に、その人の容姿や性質などの特徴から、他人がつける名。ニックネーム。あざな。
\\	郵便の宛名ははっきり正確に。	【あてな】 手紙や書類などに書く、先方の氏名。また、住所と氏名。名宛。
\\	傷口を手当てする。	【てあて】 病気やけがの処置を施すこと。また、その処置。
\\	彼は強気だ。	【つよき】 気が強いこと。積極的な態度に出ること。
\\	私の姉はよく花壇の手入れをしたものだった。	【かだん】 庭などで、一部分を区切り土を盛り上げるなどして草花を植えた所。
\\	手前の交差点を右折する。	【てまえ】 自分に近い方。また、目標とするものの前。こちら。
\\	二社が合同して計画した事業です。	【ごうどう】 独立している二つ以上のものが一つになること。また、一つに合わせること。
\\	勉強の合間に私はテレビをみた。	【あいま】 物事のとぎれる間の時間。あいだ。ひま。絶え間。
\\	この詩は無名の詩人によって書かれた。	【むめい】 名がないこと。名がわからないこと。また、名を記さないこと。無記名。
\\	費用の目安を立てる。	【めやす】 目当て。目標。おおよその基準。また、おおよその見当。
\\	その男性は60歳を超えているに違いない。髪が白髪だから。	【しらが】 色素がなくなったために白くなった髪。はくはつ。
\\	彼女は冒険心が旺盛だ。	【おうせい】 活動力が非常に盛んであること。
\\	彼らはその有名な科学者に敬意を表して宴会を催した。	【えんかい】 酒食を共にし、歌や踊りを楽しむ集まり。えん。うたげ。
\\	この制度には改良の余地がない。	【かいりょう】 不備な点や悪い点を改めて、よくすること。改善。
\\	毳々しくない上着を必ず選びましょう。	【けばけばしい】 品がなくはでなさま。特に、色彩などがどぎつくて、派手なさま。
\\	ボールが道の向こう側に転がった。	【ころがる】 ころころと回転しながら進む。ころげる。
\\	丸太を転がす。	【ころがす】 力を加えて転がるようにする。ころころと回転させて動かす。
\\	彼は肉を食い千切った。	【ちぎる】 手などでこまかに切り離す。こまかにばらばらにする。
\\	知らない人にほえるように犬をしつける。	獣などが大声で鳴く。
\\	人間は、考え、話すことができるという点で、獣と違う。	【けだもの】 全身に毛が生え、4足で歩く哺乳動物。けもの。
\\	昔は、裕福な人たちだけが乳母を雇える余裕があったのです。	【うば】 母親に代わって乳児に乳を飲ませたりして、養育する女。おんば。めのと。
\\	私は祖母に養育された。	【よういく】 養い育てること。
\\	青山さんは養う家族が多い。	【やしなう】 自分の収入で家族などが生活できるようにする。扶養する。
\\	君は家族扶養の責任を忘れてはならない。	【ふよう】 助け養うこと。生活できるように世話すること。
\\	犬がうなる。	獣が低く力の入った声を出す。
\\	私はこの国に骨を埋めるつもりです。	【うめる】 土の中などに物を入れ込んで外から見えないようにする。うずめる。
\\	耳を塞ぐ。	【ふさぐ】 耳・目・口などを手で押さえて覆う。また、目・口を閉じる。
\\	前髪を垂らす。	【たらす】 たれるようにする。ぶらさげる。
\\	カーテンを吊るす。	【つるす】 物をひもや縄などで結んで下へ垂らす。つり下げる。ぶらさげる。
\\	草花の根は赤ん坊の指のように弱い。	【くさばな】 花の咲く草。また、草に咲く花。
\\	花が凋む(萎む)。	【しぼむ】 草花などが生気をなくしてしおれたり縮んだりする。
\\	私は夏は凍った棒アイスクリームをしゃぶるのが好きです。	舌で物の表面に触れてぬらす。
\\	その足を退けなさい。	【どける】 場所をあけるために、そこにあったものを他の場所へ移す。のける。どかす。
\\	予定を一週間ずらそう。	【ずらす】 位置や日時などを重ならないように動かす。
\\	犬が跳ねるのをごらん。	【はねる】 勢いよくとび上がる。躍り上がる。
\\	人類は高等哺乳動物である。	【こうとう】 同種のものの中で、進化の度合いが高いこと。
\\	その国は高等教育社会に変わりつつある。	【こうとう】 程度・等級・品位などが高いこと。高級。
\\	あの人は案外いい人かもしれない。	【あんがい】 予想が外れること。思いがけないこと。また、そのさま。思いのほか。
\\	パーティーの費用は一人当たり4000円です。	パーティーのひようはひとりあたりよんせんえんです。 
\\	国民体育大会が開幕した。	【かいまく】 舞台の幕が開いて、芝居などが始まること。また、始めること。⇔閉幕。
\\	閉幕は九時半の予定です。	【へいまく】 幕が閉じて演劇などが終わること。また、終えること。
\\	幕が上がってにぎやかなカクテルパーティーの場面となる。	【まく】 演劇や映画で使う垂れ布。また、演劇の場面の区切り。
\\	賞味期限が切れる。	【しょうみきげん】 定められた方法により保存した場合において、期待されるすべての品質の保持が十分に可能であると認められる期限を示す年月日。
\\	水気の多い果物。	【みずけ】 物に含まれている水の量。水分。すいき。
\\	これだけのことをされて、ブチキレないほうがおかしい。	【ぶちきれる】 突然ひどく怒り出す。
\\	窓ガラスに水滴がつく。	【すいてき】 しずく。水のしたたり。
\\	気密性でない窓ならば、水滴ができるだろう。	【きみつ】 密閉して気体の流通を妨げ、気圧の変化の影響を受けないようにすること。
\\	この貯金には三分の利子が付く。	【りし】 金銭の貸借が行われた場合、その使用の対価として借り手が貸し手に支払う金銭。利息。
\\	彼は利息を付けて借金払いをした。	【りそく】 金銭などの使用の対価として、金額と期間とに比例して一定の割合(利率)で支払われる金銭その他の代替物。利子。
\\	彼は、結婚するためにイランに帰るかもしれないと言いながらも、日本から帰った後の彼の計画は依然としてめどが立っていない。	目指すところ。目当て。また、物事の見通し。
\\	ズボンに折り目をつける。	【おりめ】 紙・衣服などを折りたたむときにできる境目の筋。
\\	夜道を1人で歩くのは危険だと思います。	【よみち】 夜の道。また、夜の道を行くこと。
\\	彼らはけんかをしているのではなく、劇の稽古をしているところだ。	【けいこ】 芸能・武術・技術などを習うこと。また、練習。
\\	無差別に攻撃する。	【むさべつ】 差別のないこと。同一のものとして扱うこと。また、そのさま。むしゃべつ。
\\	個人情報を知らせたくないので、本名や住所は非公開にしています。	【ひこうかい】 一般の人には公開しないこと。
\\	その本には索引がついていますか。	【さくいん】 ある書物の中の語句や事項などを、容易に探し出せるように抽出して一定の順序に配列し、その所在を示した表。インデックス。
\\	塩は食べ物を腐らせず長持ちさせるのに役立つ。	【ながもち】 長くよい状態などを保つこと。
\\	ボートの貸し賃はいくらですか。	【かしちん】 物を貸して取る料金。⇔借り賃。
\\	貸し自転車の借り賃。	【かりちん】 物を借りた代わりに支払う料金。⇔貸し賃。
\\	我々はその問題を立体的に調査した。	【りったい】 いくつかの平面や曲面で囲まれて、三次元の空間の一部を占める物体。また、幾何学の対象としての空間図形。
\\	日本は高齢化社会に対処しようとしています。	【こうれいかしゃかい】 総人口に占める老年人口の比率が高まりつつある社会。日本では65歳以上の人口比率が7パーセントに達した昭和45年
\\	から始まったとされる。老人福祉などの対策が課題となる。老齢化社会。
\\	ヘンリーは高齢を理由に解雇された。	【こうれい】 年老いていること。老年であること。老齢。
\\	世紀の変わり目に子供たちはまだ工場で働いていた。	【かわりめ】 物事の状態や季節が移り変わる時。また、その境目のところ。
\\	彼女は1課を丸ごと暗記することで満点を取った。	分割したり変形したりしない、その形のまま。そっくり全部。まるのまま。
\\	失敗するごとに上達する。	たびに。
\\	3軒置きくらいに飲み屋がある。	さんげんおきくらいにのみやがある 
\\	5分置きに電車が着く。	ごふんおきにでんしゃがつく 
\\	一行おきに書け。	いっぎょうおきにかけ 
\\	何でもこなす者は名人にはなれない。	技術などを習って、それを思うままに使う。また、身につけた技術でうまく扱う。自在に扱う。
\\	あなたはまるで幽霊でも見たような顔付きをしている。	【かおつき】 気持ちを表す顔のようす。表情。
\\	炊きたての御飯。	【たきたて】 炊きあがったばかりであること。
\\	この会の名称はE
\\	とする。	【めいしょう】 呼び名。名前。呼称。
\\	公式に記念公園と呼称する。	【こしょう】 名をつけて呼ぶこと。また、その名。称呼。
\\	私があなたに1ドル払えば清算がつく。	【せいさん】 相互の貸し借りを計算して、きまりをつけること。
\\	法律文書では難しい言葉や語句がよく使用される。	【ごく】 語や句。また、言葉。
\\	ひと欠片のパンでは彼の飢えを満たすには足りなかった。	【かけら】 物の欠けた一片。断片。
\\	その女の子は彼の残酷な仕打ちになすがままになった。	そのおんなのこはかれのざんこくなしうちになすがままになった 
\\	一年生を受け持つ。	【うけもつ】 自分の仕事として引き受けて行う。担当する。担任する。
\\	英語の単語は他の言語にずいぶん取り入れられている。	【とりいれる】 他のよい点を採用する。
\\	洗濯物を中へ取り入れて下さい。	【とりいれる】 外にあるものを取って中に入れる。とりこむ。
\\	難民に食糧を施す。	【ほどこす】 恵まれない人に物質的な援助を与える。あわれみの気持ちで、人が困っている状態を助けるような行為をする。恵み与える。
\\	防音設備が施されプライベートを重視したゲストルーム。	【ぼうおん】 外部の音が室内に入るのを防ぎ、また室内の音が外に漏れるのを防ぐこと。
\\	防音壁を取り付ける。	【とりつける】 ある物を他の物に装置する。
\\	電車賃を立て替える。	【たてかえる】 他人に代わって一時、代金を支払う。
\\	医学では日本は欧米に追いつきました。	【おいつく】 能力・技術などが目標とするものに達する。
\\	先に歩いて下さい、後で追いつくから。	【おいつく】 追いかけて先に出たものに行き着く。
\\	この規則は貴方には当てはめることはできない。	【あてはめる】 うまく合うようにする。適用する。
\\	同じ事が国家についても当てはまる。	【あてはまる】 物事にぴったり合う。適合する。適応する。
\\	今日車はとても普及しているので、私達は誰でも車を持っていると思い込んでいる。	【おもいこむ】 深く心に思う。固く心に決める。
\\	重要だと思える事は何でも書き留めるべきです。	【かきとめる】 のちのちのために書いて残しておく。
\\	後々まで語り継がれる。	【のちのち】 それよりずっとあと。また、これから先。将来。
\\	私は押入れに閉じ込められるのがこわかった。	【おしいれ】 家屋内の、ふすまなどで仕切り、寝具・道具などを入れておく所。押し込み。
\\	ふすま越しに話し声がきこえてくる。	木で骨組みを作り、その両面に紙または布を張った建具。襖障子。
\\	ときに君の喜びと君の真剣な職業とを交流せしめよ。	【こうりゅう】 互いに行き来すること。特に、異なる地域・組織・系統の人々が行き来すること。また、その間でさまざまな物事のやりとりが行われること。
\\	友人の結婚式を司会する。	【しかい】 会の進行をつかさどること。また、その役。
\\	丈の長い黒いコートを着た、あの長身の男だった。	【たけ】 頭頂からかかとまでの、体の長さ。
\\	トムは長身の男を疑いの目で見た。	【ちょうしん】 背が高いこと。また、そのからだ。
\\	権利が消失する。	【しょうしつ】 物が消えてなくなること。また、今まで有効だったものが、その効力などを失うこと。
\\	医学は劇的な進歩をしてきた。	【げきてき】 劇を見ているように緊張や感動をおぼえるさま。ドラマチック。
\\	杉の大木を伐採する。	【ばっさい】 山林などの樹木を切り出すこと。
\\	耐震性を高める。	【たいしんせい】 建物などの、地震に耐えられる度合い。
\\	耐震性の悪い建物に亀裂が生じることもある。	【きれつ】 亀の甲の模様のように、ひびが入ること。また、その割れ目。ひび割れ。
\\	ひどくなると、亀裂が生じたり、断裂することもあります。	【だんれつ】 切れて裂けること。
\\	私はフライパンで野菜を炒めた。	【いためる】 野菜や肉などを、少量の油でいりつけて料理する。
\\	彼女はろくに微笑みをみせず答えた。	ある事柄がほとんど行われないさま。
\\	布団の出し入れは面倒だ。	【だしいれ】 出すことと入れること。出したり入れたりすること。
\\	食べ物の好き嫌いが激しい。	【すききらい】 好きなことと、嫌いなこと。また、えりごのみ。
\\	乞食は選り好みできない。	【えりごのみ】 自分の好きなものだけを選び取ること。よりごのみ。えりぎらい。
\\	困るのはお互い様です。	【おたがいさま】 両方とも同じ立場や状態に置かれていること。
\\	いい人だけどイマイチね。	少しだけ不足しているさま。もう一息。
\\	まともな人間なら、そんなことはしない。	まじめなこと。正当であること。
\\	二つの方法を併用する。	【へいよう】 あるものを他のものとともに用いること。
\\	その試合の最終得点は3対1だった。	【とくてん】 競技・試験などで、点をとること。また、その点数。
\\	彼は外国で2年間研究する特典を得た。	【とくてん】 特別に与えられる恩典。特別の待遇。
\\	授業料免除の恩典がある。	【おんてん】 ありがたい処置。情けある取り計らい。
\\	申込書は全ページにもれなく記入する必要がある。	のこらず。ことごとく。
\\	クラブ会員は1人残らずみな出席していた。	【のこらず】 余すところなく。すべて。
\\	それぞれの品物に値札をつけなさい。	【ねふだ】 商品につける、値段を書いたふだ。
\\	生活費を補助する。	【ほじょ】 不足しているところを補い助けること。また、その助けとなるもの。
\\	これらの単語の使い分け方を教えて下さい。	【つかいわける】 場合・目的・用途などに応じて、選んで使う。
\\	その発見はいろいろな用途に応用できる。	【ようと】 物や金銭などの使いみち。
\\	彼は模型飛行機作りに夢中だ。	【もけい】 実物の形に似せて作ったもの。
\\	正直は最も大事な美徳だ。	【びとく】 美しい徳。道にかなった行い。
\\	新しい世紀に変わる前後の状況はこんなものであった。	【ぜんご】 時間的にみた、まえとあと。
\\	電気のために、ろうそくは我々の生活にほとんど無用になった。	【むよう】 役に立たないこと。使い道のないこと。無益。
\\	室内を清掃する。	【せいそう】 きれいに掃除すること。
\\	は、家電製品みたいな感覚で買われてるんだろうね。	【かでん】 家庭用電気製品のこと。
\\	間取りも狭いし、周りもうるさいけど、住めば都なんだよ、この部屋。	【まどり】 部屋の配置。各室の位置。
\\	標本を種別する。	【しゅべつ】 種類・種目によって区別すること。また、その区別。
\\	駅前が満車だったから公園の無料駐輪場まで戻った。	【ちゅうりんじょう】 自転車専用の置き場。鉄道駅や商店の近くなどに設けられる。
\\	雨の中に放置しておくと自転車はさびるでしょう。	【ほうち】 そのままにしてほうっておくこと。所かまわず置きっぱなしにしておくこと。
\\	彼は盗難車を見たと知らせてきた。	【とうなん】 金品を盗まれること。また、その災難。
\\	私は月極で部屋を借りている。	【つきぎめ】 1か月を単位として契約などをきめること。
\\	私は妹と性格および習慣が大きく異なる。	複数の事物・事柄を並列して挙げたり、別の事物・事柄を付け加えて言ったりするのに用いる語。と。ならびに。また。そして。
\\	必ず弱火で煮立たないように煮ること。	【にたつ】 煮えて沸騰する。煮え立つ。
\\	煮え立たせる寸前で弱火にして、もう一度灰汁を取り除きます。	【あく】 肉などを煮たときに、煮汁の表面に浮き出る白く濁ったもの。
\\	厚手な鍋。	【あつで】 紙・布・陶器などの地の厚いこと。
\\	ふんだんな軍資金。	絶え間なく続くさま。転じて、あまるほど多くあるさま。豊富。
\\	とろ火で煮込む。	【にこむ】 時間をかけて煮る。
\\	妹は来年早々に結婚します。	【そうそう】 その状態になってすぐの時。多く、他の語の下に付いて用いられる。
\\	早々に用事を済ませる。	【そうそう】 急いで物事をするさま。はやばや。
\\	残っているのは冒頭の部分だけであった。	【ぼうとう】 文章・談話のはじめの部分。
\\	夕食後の談話で彼らは政治について話し合った。	【だんわ】 話をすること。会話。はなし。
\\	文章の末尾。	【まつび】 ものの最後。終わり。
\\	強い雨のため私はずぶ濡れになった。	【ずぶぬれ】 雨などが衣服にしみとおって、からだ全体がぬれること。びしょぬれ。ぐしょぬれ。
\\	僕なら、何を置いても、同窓会は必ず出席するだろう。	【どうそうかい】 同窓の人たちの親睦のための団体。
\\	我々は親睦を深めた。	【しんぼく】 互いに親しみ合い、仲よくすること。
\\	その部屋は紙くずだらけだった。	【かみくず】 不用になった紙切れ。くず紙。
\\	株券を盗んだと言ってジルを責める理由はあなたにはない。	【かぶけん】 株主としての地位を表す有価証券。
\\	遊牧の生活を送る。	【ゆうぼく】 1か所に定住しないで、牛や羊などの家畜とともに水や牧草を求めて、移動しながら牧畜を行うこと。
\\	その馬は一着になった。	【いっちゃく】 最初に到着すること。競走などで、一番になること。
\\	進路のことで先生に助言を求めた。	【しんろ】 将来進むべき道。将来の方向。
\\	船は北方に進路をとった。	【しんろ】 進んで行く道。行く手。
\\	日本の農村風景は大きく変わったといわれています。	【のうそん】 住民の大部分が農業を生業としている村落。
\\	彼らは一酸化炭素中毒で死んだ。	【いっさんかたんそ】 無色・無臭の猛毒気体。都市ガスや木炭などの不完全燃焼によって生じ、自動車の排ガスにも含まれている。点火すると青白い炎をあげて燃え、二酸化炭素になる。メチルアルコールの合成原料、還元剤などに利用。化学式
\\	交差点周辺の一酸化炭素の濃度はかなり高かった。	【のうど】 溶液や混合気体・固溶体などに含まれる組成成分の量の割合。表し方には、質量百分率(重量パーセント)・体積百分率(容量パーセント)・モル濃度・規定度などがある。
\\	猛暑をどうしのいでいますか。	【もうしょ】 激しい暑さ。酷暑。
\\	彼は都会の酷暑を避けて箱根に行った。	【こくしょ】 ひどく暑いこと。真夏の厳しい暑さ。
\\	猛暑で病人が続出した。	【ぞくしゅつ】 同じようなことが次々と続いて出たり起こったりすること。
\\	それ以来彼は全力をあげて自分の仕事に没頭した。	【ぼっとう】 一つの事に熱中して他を顧みないこと。
\\	二人の愛を育む。	【はぐくむ】 育成する。
\\	ワインにはいろいろなタイプがあり、それによって育成の仕方もさまざまです。	【いくせい】 育て上げること。育ててりっぱにすること。
\\	彼はまずユーモアのある逸話を話して講義を始めた。	【いつわ】 その人についての、あまり知られていない興味深い話。エピソード。
\\	平和な状態がしばらく持続した。	【じぞく】 ある状態がそのまま続くこと。また、保ち続けること。継続。
\\	隣の人が若い女の人と不倫しているらしいよ。	【ふりん】 道徳にはずれること。特に、男女関係で、人の道に背くこと。
\\	一郎は親の意に背いて俳優になった	【そむく】 規則、約束、命令、秩序などに従わない。
\\	お手洗いを拝借できますか。	【はいしゃく】 借りることをへりくだっていう語。
\\	英語のコミュニティでお名前とコメントを拝見し、プロフィールを拝読しました。	【はいどく】 読むことを、その筆者を敬っていう謙譲語
\\	お手紙拝受いたしました。	【はいじゅ】 受けること、受け取ることをへりくだっていう語。
\\	公衆の面前で彼を嘲笑するのはよくない。	【こうしゅう】 社会一般の人々。
\\	ワールドカップにおける予選は、ワールドカップのエントリー国の中から本大会に出場できる国を決めるための大会である。	【よせん】 数あるものの中から前もって選び出すこと。
\\	私の愛猫がもう一週間も行方が解らない。	【あいびょう】 かわいがって大切にしている猫。
\\	愛犬の病気は重かった。	【あいけん】 かわいがって飼っている犬。
\\	目前に迫った試験のことが彼女の心に大きく広がった。	【もくぜん】 見ている目の前。転じて、きわめて近いこと。
\\	彼女は深呼吸してから、身の上を語り始めた。	【みのうえ】 その人にかかわること。また、その人の境遇。
\\	彼は十中八九会議に遅れる。	【じっちゅうはっく】 十のうちの八か九まで。副詞的にも用いる。ほとんど。おおかた。
\\	新工場の建設用地はまだ未定である。	【みてい】 まだ決まっていないこと。
\\	彼の陳述はだんだんしどろもどろになった。	言葉の使い方や話の内容などが、とりとめなく、ひどく乱れたさま。
\\	原始人はその猛獣を見ておびえた。	【もうじゅう】 肉食で、荒々しい性質の動物。
\\	咄嗟ににブレーキをふむ。	【とっさに】 その瞬間に。たちどころに。
\\	咄嗟の思い付きであんな返事をしてまずかった。	【とっさ】 ごくわずかな時間。
\\	前述のとおり筆者はその意見に同意できない。	【ぜんじゅつ】 前に述べたこと。既述。先述。
\\	寝室のカーテンは端が色あせてきた。	【しんしつ】 寝るために使う部屋。ねや。
\\	この豪邸には寝室が十二もあります。	【ごうてい】 大きくてりっぱな邸宅。大邸宅。
\\	この薬を飲むと胃の痙攣が治ります。	【けいれん】 全身的または部分的に筋肉が収縮し、不随意運動を起こすこと。持続的にみられる強直性のもの、間欠的にみられる間代性のものなどがあり、脳疾患・髄膜炎・中毒・ホルモンの異常などが原因。
\\	私達は交代で車を運転しました。	【こうたい】 役割や位置などを互いに入れかえること。また、互いに入れかわること。
\\	医者の器具は常に完全に清潔でなければならない。	【せいけつ】 汚れがないこと。衛生的であること。
\\	難民キャンプの衛生状態はひどいものだった。	【えいせい】 健康の維持と向上を図るとともに、疾病の予防と治療につとめること。
\\	清潔は日本人の習性だ。	【しゅうせい】 後天的に習慣が性質となったもの。習癖。
\\	言語能力は後天的ではなく、生まれつきです。	【こうてんてき】 生まれてからのちに身にそなわるさま。
\\	その毒性のある植物から離れているべきです。	【どくせい】 有毒の性質。
\\	この流動体は接着剤の代用になります。	【せっちゃくざい】 固体と固体とをはり合わせるのに用いる物質。でんぷんのり・カゼイン・にかわ・ゴムなどやフェノール樹脂・ビニル樹脂・エポキシ樹脂などの合成樹脂がある。
\\	その魚は真水に住む。	【まみず】 塩分などのまじらない水。淡水。さみず。
\\	太陽は信じられないほど、とてつもない熱と光を多量に放出している。	途方もない。また、並み外れている。
\\	植物は栄養物を作っているとき、酸素を放出する。	【ほうしゅつ】 吹き出すこと。また、あふれ出ること。
\\	警官がやってくるのを見ると彼は猛烈に走り出した。	【もうれつ】 勢いが強くはげしいさま。程度がはなはだしいさま。
\\	その鍵を弄るな!	【いじる】 指先や手で触ったりなでたりする。
\\	編成を弄る。	【いじる】 物事を少し変えたり、動かしたりする。
\\	彼の身元は腕時計で確認できた。	【みもと】 その人の出生・出自・経歴などの事柄。
\\	警察はその男の経歴を調べた。	【けいれき】 今まで経験してきた仕事・身分・地位・学業などの事柄。履歴。
\\	路上の血痕は俺のものに違いない。	【けっこん】 血のついた跡。
\\	語学に主力を注ぐ。	【しゅりょく】 出せる力のうちのおもな部分。おもな力。
\\	当社の主力商品。	【しゅりょく】 中心となって力を発揮するもの。主要な戦力・勢力。
\\	彼らは果樹に農薬を散布している。	かれらはかじゅにのうやくをさんぷしている 
\\	今日はジョンが当番です。	【とうばん】 順送りに仕事の番に当たること。また、その番に当たる人。
\\	次の兎の飼育当番は彼らです。	【しいく】 家畜などを飼い育てること。飼いならすこと。
\\	私たちは庭で野菜を栽培している。	【さいばい】 植物を植えて育てること。魚介類の養殖にもいう。
\\	メロンは温室で栽培する。	【おんしつ】 内部の温度を一定に保てるように設備した、ガラスやビニール張りの建物。
\\	人間は複雑な有機体だ。	【ゆうきたい】 生活機能をもち、有機物からなる組織体。生物のこと。
\\	5分ばかりこの道を行けば、右手にその百貨店があります。	【ひゃっかてん】 デパート。
\\	核の拡散を防止する。	【かくさん】 広がり、散らばること。
\\	科学者たちはエイズ・ウイルスの拡散を食い止めようと戦っています。	【くいとめる】 [動マ下一][文]くひと・む[マ下二]物事がそれ以上進まないように防ぎ止める。
\\	誰かがその家に放火した。	【ほうか】 火事を起こす目的で、火をつけること。付け火。火付け。
\\	昨晩の火事は放火と断定された。	【だんてい】 物事にはっきりした判断をくだすこと。また、その判断。
\\	その発表は死傷者の数を誇張していた。	【こちょう】 実際よりも大げさに表現すること。
\\	全国を宣教して歩く。	【せんきょう】 宗教上の教えを広めること。特に、キリスト教の伝道にいう。
\\	風俗習慣は国によって大きな違いがある。	【ふうぞく】 ある時代やある社会における、生活上の習わしやしきたり。風習。
\\	この風習の起源は12世紀にさかのぼる。	【ふうしゅう】 その土地や国に伝わる生活や行事などの習わし。風俗習慣。しきたり。
\\	五つの等級に分ける。	【とうきゅう】 上下・優劣の順位を表す段階。くらい。階級。
\\	低い順位にある。	【じゅんい】 一定の基準によって上下あるいはあとさきの関係で順に並べられるときの、それぞれの位置。
\\	それらの作品には優劣をつけがたい。	【ゆうれつ】 すぐれていることと、おとっていること。まさりおとり。
\\	口コミで売れる。	【くちコミ】 うわさ・評判などを口伝えに広めること。
\\	山上に陣取る。	【じんどる】 ある場所に陣地を構える。
\\	ヒトラーは後生の悪い人だろう。	【ごしょう】 あとから生まれてくる人。後世の人。
\\	就きましてはカタログを郵送してください。	【つきましては】 「就いては」の丁寧な言い方。それですから。したがいましては。
\\	彼は粘土で像を形作った。	【かたちづくる】 形成する。構成する。
\\	水は水素と酸素で構成されている。	【こうせい】 いくつかの要素を一つのまとまりのあるものに組み立てること。また、組み立てたもの。
\\	彼は私の不平を軽んじた。	かれはわたしのふへいをかろんじた 
\\	夏には節水して下さい。	【せっすい】 水をむだに使わないようにすること。
\\	テレビの音声を低くしてもいい?	【おんせい】 人間が音声器官を通じて発する音の総称。おんじょう。
\\	あなたを一人前の男にしてあげよう。	【いちにんまえ】 成人であること。また、成人の資格・能力があること。ひとりまえ。
\\	俳句は季節と関連が深い。	【かんれん】 ある事柄と他の事柄との間につながりがあること。連関。
\\	彼は2年生のとき学校を中退した。	【ちゅうたい】 修業年限の中途で退学すること。
\\	只今、僕は旅立ちの日に向けて修業中です。	【しゅうぎょう】 学術・技芸などを学んで身につけること。また、その分野で規定される課程または年限を済ますこと。しゅぎょう。
\\	旅立ちの時が近づいている。	【たびだち】 旅に出ること。出発すること。かどで。
\\	その船はアメリカ国旗を掲げていた。	【かかげる】 人目につく高い所へ上げる。また、手に持って高く差し上げる。
\\	昨日今日に始まったことではない。	【きのうきょう】 つい最近。このごろ。
\\	この文章は作者の気持ちを的確に表現している。	【てきかく】 的をはずさないで、まちがいがないこと
\\	せっかくの招待とあれば否めない。	【いなめない】 断ることができない。
\\	責任の重さを犇犇と感じる。	【ひしひし】 強く身に迫るさま。切実に感じるさま。
\\	今時良い仕事はなかなかないが得難いのを忘れないでね。	【いまどき】 今の時世。現代。現今。当世。
\\	あなたはユーモアのセンスが抜群です。	【ばつぐん】 多くの中で、特にすぐれていること。ぬきんでていること。
\\	彼は抜群の騎手だ。	【きしゅ】 
\\	馬に乗る人。 
\\	競馬で、出場馬の乗り手。ジョッキー。
\\	彼は減税を唱えた。	【げんぜい】 税金の額を減らすこと。
\\	彼らは減税は自分たちの功績だと主張した。	【こうせき】 あることを成し遂げた手柄。すぐれた働きや成果。
\\	彼は年の割には老けて見える。	【ふける】 年をとる。また、年寄りじみる。
\\	文明を更新し再建して世界を自殺から救う。	【さいけん】 焼けたり、壊れたりした建造物を建て直すこと。
\\	その城は、1485年に全焼して、再建されなかった。	【ぜんしょう】 火事で、建物などが全部焼けてしまうこと。まるやけ
\\	貿易は諸国の発展を促進する。	【そくしん】 物事がはやくはかどるようにうながすこと。
\\	その少年は両親は付き添われてきた。	【つきそう】 世話などをするためにそばについている。
\\	彼女は結婚式で花嫁の付き添い役をつとめた。	【つきそい】 人のそばについて、世話をすること。
\\	彼女の夫は3年間服役している。	【ふくえき】 懲役に服すること。
\\	裁判官はその囚人の一年の刑期を赦免した。	【しゃめん】 罪や過ちを許すこと。
\\	その囚人は刑期に服した後赦免された。	【ふくする】 言われたとおりにする。従う。服従する。また、従わせる。
\\	裁判官が有罪の判決を下した以上、潔く服役しなければならない。	【いさぎよい】 思い切りがよい。未練がましくない。また、さっぱりとしていて小気味がよい。
\\	彼女にはまだ未練がある。	【みれん】 執心が残って思い切れないこと。あきらめきれないこと。
\\	りりしい若者。	物事に対してひるむことなく積極的に向かって行くさま。
\\	彼はまめに働く人だ。	労苦をいとわず物事にはげむこと。勤勉。
\\	その婦人の態度はしとやかだ。	性質や動作がもの静かで上品であるさま。また、つつしみ深いさま。
\\	彼女は太っているとはいえないまでも大柄な人だ。	【おおがら】 体格が普通より大きいこと。
\\	息子は年の割には小柄だ。	【こがら】 体格が普通より小さいこと。
\\	ある意味では、丁寧語は気さくな雰囲気を壊す。	【きさく】 人柄がさっぱりしていて、こだわらないさま。気取りがなく親しみやすいさま。
\\	気持ちをおおらかにする。	心がゆったりとして、こせこせしないさま。おおよう。
\\	こせこせする必要はない。	細かなことにこだわって、ゆとりや落ち着きがないさま。
\\	私たちがポルトガルへ行こうと決めたのはまったくの気まぐれだった。	【きまぐれ】 気が変わりやすいこと。その時々の思いつきや気分で行動すること。
\\	あの社長は近寄りにくい。	【ちかよる】 近くに寄る。近づく。
\\	その家は見かけがたいへん陰気だった。	【いんき】 気分・雰囲気・天候などが、晴れ晴れしないこと。
\\	彼は物好きにもその珍しい果物をかじってみた。	【ものずき】 変わったことを好むこと。好奇心が強く、普通と違ったことを好むこと。
\\	いつもとは趣向の異なるパーティー。	【しゅこう】 おもむき。意向。趣意。
\\	年上の人にあまり馴れ馴れしくしてはいけない。	【なれなれしい】 非常に親しいようすである。
\\	彼はしぶとく自分の意見にこだわる。	【しぶとい】 強情で臆するところがない。また、困難にあってもへこたれずねばり強い。
\\	彼は強情すぎるが、他方では頼りになった。	【ごうじょう】 意地を張って、なかなか自分の考えを変えないこと。
\\	だれのことにもお節介をやく人だ。	【おせっかい】 出しゃばって、いらぬ世話をやくこと。
\\	君のような出しゃばりは嫌いだ。	【でしゃばり】 おせっかいな人。
\\	彼のきざな態度は頭にくる。	服装や言動などが気どっていて嫌な感じをもたせること。
\\	雨の日は出掛けるのが煩わしい。	【わずらわしい】 心を悩ましてうるさい。面倒で、できれば避けたい気持ちである。
\\	このパソコンはわずらわしい配線の必要がありません。	【はいせん】 電気機器・通信装置などを導線で接続して回路を構成すること。また、その導線。
\\	日本では心臓病を患う人が多いようです。	【わずらう】 病気になる。
\\	愛の輝きのない人生は何だろうか。	【かがやき】 かがやくこと。また、かがやく光。
\\	今仮に突然目が見えなくなったら、どうしますか。	【かりに】 仮定を表す。もし。
\\	今日は清清しいお天気ですね。	【すがすがしい】 さわやかで気持ちがいい。
\\	学生たちは夏休みが来るのが待ち遠しい。	【まちどおしい】 待っていてもなかなか来ず、早く来るようにと願っているさま。
\\	一人だけで行くのは心細い。	【こころぼそい】 頼るものがなく不安である。
\\	彼と離れるのはとても切ない。	【せつない】 悲しさや恋しさで、胸がしめつけられるようである。やりきれない。やるせない。
\\	いつもうっとうしい感じがするのです。	心がふさいで晴れ晴れしない。気分が重苦しい。
\\	顔を合わせるのが決まり悪い。	【きまりわるい】 体裁が悪く恥ずかしい。気恥ずかしい。きまりがわるい。
\\	おっかない目にあう。	怖い。恐ろしい。
\\	浅ましい世の中になったものだ。	【あさましい】 見苦しく情けない。嘆かわしい。
\\	嘆かわしい事件が起こる。	【なげかわしい】 悲しく情けなく感じられる。残念に思う。
\\	彼の苦労を思うと私も心苦しい。	【こころぐるしい】 心に痛みを感じるさま。つらく切ない。
\\	彼は走って10ポンド減量した。	【げんりょう】 目方や分量が減ること。また、減らすこと。特に、体重を減らすこと。⇔増量。
\\	薬を増量する。	【ぞうりょう】 分量がふえること。分量をふやすこと。⇔減量。
\\	ここだけの話だけれど、あの太った見苦しい魔女は減量中なのだ。	【みぐるしい】 見た感じが不愉快である。みっともない。みにくい。
\\	すぐすねる子供。	すなおに人に従わないで、不平がましい態度をとる。
\\	彼は文章の切れ目ごとに息をついた。	【きれめ】 継続して行われている物事の、いったんとぎれるところ。ひと区切りついたところ。
\\	この辺で一応仕事に区切りをつけよう。	【くぎり】 物事の切れ目。段落。きり。
\\	肉に切れ目を入れて。	【きれめ】 切れてできたあと。
\\	目先の利益だけにとらわれてはいけない。	【めさき】 目の前にある物事。その時その場。当座。
\\	当座のところ、僕は叔父の家に泊めてもらっているが、将来小さなアパートに移るつもりだ。	【とうざ】 物事に直面した、すぐその場。即座。
\\	あいつは田中の家来だ。	【けらい】 主君や主家に仕える者。家臣。従者。
\\	従者にとっては誰も英雄ではない。	【じゅうしゃ】 主人の供をする者。供の者。供人。ずさ。
\\	家主は彼がドアを赤く塗ることを許してくれない。	【やぬし】 貸し家の所有者。いえぬし。おおや。
\\	その会社は中国への進出を目指している。	【しんしゅつ】 進み出ること。新しい方面や分野に進み出て、活動領域を広げること。
\\	あの子は勉強の進度が速い。	【しんど】 物事の進み方の程度。はかどりぐあい。
\\	先行の3人は間もなく引き返した。	【せんこう】 他の人に先だって行くこと。前行。
\\	彼の思想は時代に先行していた。	【せんこう】 他の事柄よりも先に進むこと。
\\	先着200名まで特別席に入れます。	【せんちゃく】 先に到着すること。
\\	先方のお電話番号は何番ですか。	【せんぽう】 相手の人。
\\	政府は税制改革に着手した。	【ちゃくしゅ】 ある仕事に手をつけること。とりかかること。
\\	着色した食品。	【ちゃくしょく】 物に色をつけること。また、その色。
\\	着目に値する提案。	【ちゃくもく】 特に注意して見ること。目をつけること。また、目のつけどころ。着眼。
\\	なかなかよいところに着眼している。	【ちゃくがん】 目をつけること。また、目のつけ方。着目。
\\	そのビルの建設は来年着工されます。	【ちゃっこう】 土木・建築などの工事を始めること。
\\	彼は自分の家族が楽に暮らせるように日夜働いていた。	【にちや】 昼と夜。昼夜。
\\	彼は金持ちになるために昼夜働いた。	【ちゅうや】 ひるとよる。
\\	私は終日小説を読んで過ごした。	【しゅうじつ】 1日中。朝から晩まで。まる1日。ひねもす。
\\	多くの切符は前売りされている。	【まえうり】 切符や入場券などを、使用当日よりも前に売ること。また、その券。
\\	自主性に欠ける。	【じしゅ】 他からの干渉や保護を受けず、独立して事を行うこと。
\\	私は彼を説得して警察に自首させた。	【じしゅ】 犯罪事実または犯人の発覚する前に、犯人が自ら捜査機関に犯罪事実を申告し、その訴追を求めること。刑が減軽または免除されうる事由となる。
\\	恥ずかしげに俯く。	【うつむく】 [動カ五(四)]顔が下の方へ傾く。下を向く。
\\	彼女は瞬きして涙を止めようとした。	【まばたき】 まぶたを閉じて、またすぐ開くこと。またたき。
\\	彼は学校に対する不満を呟いた。	【つぶやく】 小さい声でひとりごとを言う。
\\	母は台所でせっせと料理しながら歌を口ずさんでいた。	【くちずさむ】 詩や歌などを思いつくままに口にしたり歌ったりする。
\\	庭の草をむしる。	つかんだりつまんだりして引き抜く。
\\	カレンダーをめくる。	上に重なっているものをはがすように上げる。
\\	鼻を摘んで薬を飲み込んだ。	【つまむ】 指先ではさむ。指先や箸(はし)などではさみもつ。
\\	花を摘みに行く。	【つむ】 指先や爪の先ではさみとる。つまみとる。
\\	母親は構い過ぎて子供の独立心の芽を摘んでしまった。	【つむ】 大きくならないうちに取り除く。
\\	ばらの新芽が出てきた。	【め】 植物の、少しふくらんでいて、やがて生長して葉や茎(くき)になる部分。
\\	いすの背をはたく。	打ち払う。ほこりなどをたたいて払う。
\\	疲れた足を摩る。	【さする】 [動ラ五(四)]手のひらなどでからだや物の表面を、くりかえし軽くこする。
\\	赤ん坊をバスタオルで包む。	【くるむ】 巻くようにして物をつつむ。
\\	彼はその紙を握りつぶして丸めた。	【まるめる】 丸い形にする。
\\	彼はマッチを擦った。	【する】 物に、他の物を強く触れ合わせて動かす。
\\	床に俯せる。	【うつぶせる】 顔を下に向けてからだを伏せる。腹を下にして横たわる。
\\	馬にまたがる。	またを広げて両足で挟むようにして乗る。
\\	股を広げて立つ。	【また】 胴から足が分かれている所。また、ズボン・パンツなどのその部分にもいう。
\\	この計画は来年度にもまたがるものだ。	時間的、空間的に一方から他方におよぶ。わたる。ひろがる。
\\	彼は膝まで泥に浸かっていた。	【つかる】 液体の中にひたる。転じて、ある状態などにはいりきる。
\\	ボートは橋の下を潜り抜けた。	【くぐる】 物の下や狭い間・中を、姿勢を低くして通って向こう側へ出る。また、門やトンネルなどを通り抜ける。
\\	家のない人たちは冷たいにわか雨をよける場所を探した。	触れたり出あったりしないようにわきに寄る。また、身をかわしてさける。
\\	もがけばもがくほど縄が食い込んだ。	もだえ苦しんで手足をやたらに動かす。あがく。
\\	彼は水面に浮かび上がろうと必死で足掻いた。	【あがく】 手足を振り動かしてもがく。じたばたする。
\\	彼は長年あちこち彷徨ったあげく,ここに落ち着いた。	【さまよう】 あてもなく歩きまわる。また、迷って歩きまわる。
\\	彼女は自分の容姿については、思い違いはしていなかった。	【ようし】 顔だちとからだつき。すがたかたち。
\\	新聞に広告を掲載する。	【けいさい】 新聞・雑誌などに、文章・写真などを載せること。
\\	「先生の容態は?」「絶対安静だ」	【あんせい】 病気治療のため、静かに寝ていること。
\\	君は無茶を言っている。	【むちゃ】 筋道が立たず、道理に合わないこと。
\\	引退したら余生を田舎で過ごしたい。	【よせい】 盛りの時期を過ぎた残りの生涯。残された人生。
\\	一日の仕事が終わると皆家路を急ぐ。	【いえじ】 わが家へ帰る道。
\\	人前で彼を冷やかすなんて君は意地悪だ。	【ひやかす】 相手が困ったり恥ずかしがったりするような言葉をかけてからかう。
\\	大口を叩く。	【おおくちをたたく】 おおげさなことをいう。偉そうにいう。
\\	彼はすぐに雑用を終えた。	【ざつよう】 こまごました、いろいろの用事。
\\	彼女は雑用をするのを厭わない。	【いとう】 嫌って避ける。嫌がる。
\\	その事故は単なる過失から起こった。	【かしつ】 不注意などによって生じたしくじり。過ち。
\\	清水ですすぐ。	汚れを水で洗い落とす。水で清める。
\\	髪を洗ったら、濯ぎなさい。	【ゆすぐ】 水の中で揺り動かして、また、中の水を揺り動かして洗う。すすぐ。
\\	不純物をこして取り除く。	【こす】 かすや不純物を取り除くために、布・網・紙や砂などをくぐらせる。
\\	豆をばらまく。	ばらばらに散らしてまく。方々にまき散らす。
\\	名刺をばらまく。	金銭や物品を多くの人に配る。
\\	眠っている子を揺すって起こした。	【ゆする】 ゆり動かす。
\\	手紙を寄越す。	【よこす】 こちらへ送ってくる。こちらへ渡す。
\\	話しながら目を逸らすのは失礼である。	【そらす】 向かうべき方向・目標からわきの方へ向ける。他へ転じる。
\\	にんじんを細かく刻む。	【きざむ】 刃物で物を細かく切る。
\\	刻んだキャベツを水にさらしてぱりっとさせる。	【さらす】 野菜などのあく・臭みなどを抜くために水に浸す。
\\	工夫を凝らす。	【こらす】 一心に考えをめぐらす。
\\	山頂への小道を辿った。	【たどる】 道筋に沿ってめざす方向へ進む。
\\	ぼんやりした思い出を辿ればここに井戸があった。	【たどる】 筋道を追ったり、手がかりを頼ったりして探し求めていく。次から次へと尋ね求める。
\\	防犯カメラを据え付ける。	【すえつける】 物をある場所に据えて固定する。
\\	彼は何かよからぬことを目論んでいるらしい。	【もくろむ】 物事をしようとして考えをめぐらす。計画する。企てる。たくらむ。
\\	敵軍によって退路を遮られた。	【さえぎる】 進行・行動を邪魔してやめさせる。妨げる。
\\	木々に太陽光線が遮られる。	【さえぎる】 間に隔てになるものを置いて、向こうを見えなくする。
\\	彼らの侵入を阻む。	【はばむ】 進もうとするのをさまたげる。防ぎとめる。また、こばむ。
\\	駐車中の車に爆発物が仕掛けてあった。	【しかける】 作用するように、装置・工夫などを設ける。
\\	弁護士は真実を話すように証人を促した。	【うながす】 物事を早くするようにせきたてる。また、ある行為をするように仕向ける。催促する。
\\	あなたは自分の意見に固執すべきではない。	【こしつ】 あくまでも自分の意見を主張して譲らないこと。
\\	インフレが家計を脅かす。	【おびやかす】 危険な状態にする。危うくする。
\\	その店に何度か行くうちに主人と馴染みになった。	【なじみ】 なれ親しんで知っていること。また、その人。
\\	彼はすぐ外国の慣習に馴染んだ。	【なじむ】 人になれて親しくなる。また、物事や場所になれて親しみをもつ。
\\	私は京都の古い寺に深い愛着を感じる。	【あいちゃく】 なれ親しんだものに深く心が引かれること。あいじゃく。
\\	疑問を抱く。	【いだく】 ある考えや感情をもつ。
\\	その別荘は清潔で整然としていた。	【せいぜん】 秩序正しく整っているさま。
\\	消極的なその男はめったに自己表現しない。	【しょうきょくてき】 自分から進んで物事をしないさま。引っ込みがちなさま。また、否定的であるさま。
\\	庭は草むしりが必要だ。	【くさむしり】 雑草をむしり取ること。草取り。除草。
\\	彼は大きな花束を抱えてやって来た。	【はなたば】 花を何本か束ねたもの。
\\	我々は時代に一歩先んじた見識を持たねばならない。	【けんしき】 物事を深く見通し、本質をとらえる、すぐれた判断力。ある物事に対する確かな考えや意見。識見。
\\	泣く子を宥める。	【なだめる】 怒りや不満などをやわらげ静める。事が荒だたないようにとりなす。
\\	老人を労わりなさい。	【いたわる】 弱い立場にある人などに同情の気持ちをもって親切に接する。気を配って大切に世話をする。
\\	彼を煽てれば、彼は何でもしてくれる。	【おだてる】 うれしがることを言って、相手を得意にさせる。何かをさせようと、ことさらに褒める。もちあげる。
\\	友達の昇進を妬んでいる。	【ねたむ】 他人が自分よりすぐれている状態をうらやましく思って憎む。ねたましく思う。そねむ。
\\	自転車の男がハンドバッグを攫った。	【さらう】 油断につけこんで奪い去る。気づかれないように連れ去る。
\\	彼女は大変話好きでいつも我々の会話を攫ってしまう。	【さらう】 その場にあるものを残らず持ち去る。関心を一人占めにする。
\\	事件を公平に裁く。	【さばく】 理非を明らかにする。裁判する。
\\	彼は酔っ払いから妻を庇おうと前へ出た。	【かばう】 他から害を受けないように、助け守る。いたわり守る。
\\	ごちそうを出して客を持て成した。	【もてなす】 人を取り扱う。待遇する。あしらう。
\\	妻の不注意を詰った。	【なじる】 相手を問いつめて責める。詰問する。
\\	男は通行人にたばこを強請った。	【ねだる】 甘えたり、無理に頼んだりしてほしいものを請い求める。せがむ。せびる。
\\	奢った生活をする。	【おごる】 程度を超えたぜいたくをする。
\\	成功しても驕るな。	【おごる】 地位・権力・財産・才能などを誇って、思い上がった振る舞いをする。
\\	彼は自分の考えを息子たちに強いた。	【しいる】 相手の意向を無視して、むりにやらせる。強制する。
\\	男の子たちは新しい先生にすぐ懐いた。	【なつく】 慣れ親しむ。慣れて付き従う。
\\	彼は嫌がらせをしているだけだよ。	【いやがらせ】 相手の嫌がることをしたり言ったりして、わざと困らせること。また、その言行。
\\	その件はすっかり拗れてしまった。	【こじれる】 物事がもつれて、うまく進まなくなる。
\\	風邪を拗らせて肺炎になってしまった。	【こじらせる】 病気を治しそこねて長引かせる。
\\	意識がもうろうとしている。	意識が確かでないさま。
\\	山に霞がかかっていた。	【かすみ】 細かい水滴やちりが空気中に漂っている現象。
\\	雨に霞む街。	【かすむ】 霞がかかったような状態になる。ぼんやりして、物の姿や形がはっきり見えなくなる。
\\	涙で目が霞んだ。	【かすむ】 目が疲れたり故障があったりして物が見えにくくなる。
\\	今は政界から遠ざかっています。	【とおざかる】 遠くに離れてゆく。遠のく。
\\	貧血症にかかっている。	【ひんけつしょう】 
\\	足が浮腫む。	【むくむ】 体組織に余分な組織液がたまり、からだの全体、または一部分がはれたようになる。
\\	彼は吃りながら礼を言った。	【どもる】 ものを言うとき、言い出しの音が容易に発音できなかったり、ある音が何度も繰り返されたりする。
\\	どもりを矯正する。	【きょうせい】 欠点・悪習などを正常な状態に直すこと。
\\	蕁麻疹が全身に出た。	【じんましん】 
\\	判断力いかんによらず、二十歳以上であれば選挙権がある。ただし認知症の場合などは問題があるだろう。	【にんちしょう】 
\\	心臓の発作が起きた。	【ほっさ】 病気の症状が急激に起こること。ふつう短時間でおさまる。
\\	喘息の発作が起きました。	【ぜんそく】 
\\	うん、誤診だったみたい。	【ごしん】 医師が診断を誤ること。
\\	その痛ましいエピソードは私には痛切に感じられた。	【つうせつ】 身にしみて強く感じること。
\\	彼女の親切が身に染みた。	【しみる】 心にしみじみと感じる。しむ。
\\	私は彼の親切を染み染み感じた。	【しみじみ】 心の底から深く感じるさま。
\\	国会は多分この不評の法律を改正するだろう。	【ふひょう】 評判の悪いこと。評価の低いこと。
\\	誰もが永久平和を望んでいる。	【えいきゅう】 いつまでも限りなく続くこと。
\\	ガソリンは燃料として使われる。	【ねんりょう】 燃焼させて熱・光や動力などを得る材料。石炭・薪(まき)・ガソリン・アルコール・ガス・ウランなど。
\\	憂慮の面もちで患者を見ていた。	【ゆうりょ】 心配すること。思いわずらうこと。
\\	日照りは9月まで続いた。	【ひでり】 日が照りつけること。特に、真夏に晴天が続き雨が降らないこと。
\\	期待通り晴天となった。	【せいてん】 晴れた空。よい天気。青天。
\\	日照り続きで井戸が枯渇した。	【こかつ】 水がかれること。かわいて水分がなくなること。
\\	戦争で国の財源は枯渇した。	【こかつ】 物が尽きてなくなること。
\\	壁に掲示してある順路に従って進む。	【じゅんろ】 順序よく進めるように定めた道筋。道順。
\\	大砲が偶然発射してしまった。	【たいほう】 火薬の爆発力を利用して大きな弾丸を発射する兵器。
\\	日本は兵器にあまりお金を使うべきではない。	【へいき】 戦闘に用いる器材の総称。武器。
\\	条約は科学兵器の使用を禁止している。	【じょうやく】 国家間または国家と国際機関との間の文書による合意。
\\	人に鉄砲を向ける。	【てっぽう】 飛び道具(とびどうぐ)  ピストル  短銃(たんじゅう)  拳銃(けんじゅう)  はじき  機関銃(きかんじゅう)  機関砲(きかんほう)  小銃(しょうじゅう)  ライフル  猟銃(りょうじゅう)
\\	あいつは無鉄砲な男だ。	【むてっぽう】 是非や結果を考えずにむやみに行動すること。また、そのさまや、そのような人。むこうみず。
\\	地上のすべての生物は互いに依存し合っている。	【ちじょう】 地面の上。
\\	地上のすべての人は同胞だ。	【どうほう】 同じ父母から生まれた兄弟姉妹。
\\	海外で働く同胞を支援する。	【どうほう】 同じ国土に生まれた人々。同じ国民。また、同じ民族。
\\	この地上で唯一無二の平等は死である。	【ゆいつむに】 ただ一つあって、二つとないこと。
\\	病人を徹夜で介抱する。	【かいほう】 病人・けが人・酔っぱらいなどの世話をすること。看護。
\\	彼のミスがその計画を水泡に帰した。	【すいほうにきす】 努力したのに、何の甲斐もなく無駄に終わること
\\	飽和状態になるまで塩をこの水に入れなさい	【ほうわ】 含みもつことのできる最大限度に達して、それ以上余地のないこと。
\\	いえ、これは秘宝ですから。わたくしたちもこの度、初めて見たのです。	【ひほう】 大切にして他人には見せない宝。秘蔵の宝物。
\\	満員電車でぎゅうぎゅうドアに押し付けられた。	【おしつける】 押して、離れないようにする。力を入れて押す。おっつける。
\\	彼はいつも私に彼の意見を押し付けます。	【おしつける】 無理にやらせる。また、無理に受け入れさせようとする。おっつける。
\\	陸路ではそこへ行けない。	【りくろ】 陸上を通る道。
\\	両親は空港まで私を見送ってくれた。	【みおくる】 遠ざかる物や人をその後方で眺める。
\\	その車は制限速度を超過している。	【ちょうか】 数量や時間が、一定の限度をこえること。
\\	喜びも悲しみも分かち合う。	【わかちあう】 互いに分ける。分け合う。
\\	葬儀は厳かに行われた。	【おごそか】 重々しくいかめしいさま。礼儀正しく近寄りにくいさま。
\\	彼の家は喫茶店と花屋を兼業している。	【けんぎょう】 本業のほかに他の事業・仕事を兼ね行うこと。また、その事業・仕事。
\\	これはソファーとベッドに兼用できる。	【けんよう】 一つのものを二つ以上の用途に使うこと。
\\	自転車を兄と兼用する。	【けんよう】 一つのものを二人以上で一緒に使うこと。共用。
\\	彼は蛇を本能的に嫌悪していた。	【けんお】 憎みきらうこと。強い不快感を持つこと。
\\	午後8時に干潮になる。	【かんちょう】 潮が引いて、海水面が最も低くなる現象。ふつう、1日に2回起こる。低潮。引き潮。⇔満潮(まんちょう)。
\\	病人を発汗させる。	【はっかん】 汗が出ること。汗を出すこと。
\\	会報は年2回刊行される。	【かんこう】 書籍などを印刷して世に出すこと。出版。
\\	この薬は肝臓に効く。	【かんぞう】 
\\	彼はその問題について幹部の何人かと協議した。	【かんぶ】 団体や組織の中で中心となる人。
\\	光は水に入る時屈折する。	【くっせつ】 光や音などの波動が、ある媒質から他の媒質に進むとき、その境界面で進行方向を変えること。
\\	埋蔵金が発掘される。	【はっくつ】 地中に埋もれているものを掘り出すこと。
\\	採掘が爆発した時、中には誰もいなかった。	【さいくつ】 岩石・土砂や地中の鉱物などを掘り出すこと。
\\	険悪な顔でにらむ。	【けんあく】 表情や性質がとげとげしくなること。
\\	そんなトゲトゲしい言い方しなくたっていいだろう。	態度や言葉づかいにとげがある。つっけんどんである。
\\	突っけんどんに答える。	【つっけんどん】 無遠慮でとげとげしいさま。冷淡なさま。
\\	私の演説は冷淡な沈黙で迎えられた。	【れいたん】 物事に熱心でないこと。関心や興味を示さないこと。淡白。
\\	彼は淡泊な気性だから付き合いやすい。	【たんぱく】 性格や態度がさっぱりしていること。こだわりやしつこさがないこと。
\\	彼女は自分の英文を英国人に点検してもらった。	【てんけん】 悪い箇所や異常はないか、一つ一つ検査すること。
\\	聴講切符を前もって入手しておかなければならない。	【ちょうこう】 講義を聞くこと。
\\	排水のために溝を掘った。	【はいすい】 不用な水を排出すること。
\\	彼の船は無人島に漂流した。	【ひょうりゅう】 風や潮のままに海上をただよい流れること。
\\	それは海底で発見された。	【かいてい】 海の底。
\\	彼らは大邸宅に住んでいる。	【ていたく】 家。すまい。特に、構えが大きくて、りっぱな造りの家。やしき。邸第。
\\	首相官邸。	【かんてい】 大臣や長官など高級官吏の在任中に、住居として政府が提供する邸宅。
\\	この企画は採算が取れないかもしれない。	【さいさん】 利益があるかどうか、収支を計算してみること。商売や事業の、収支のつりあい。
\\	メニューにはいろいろ多彩な料理があった。	【たさい】 変化や種類が多くにぎやかなこと。
\\	多彩なネオンサイン。	【たさい】 色の種類の多いこと。いろどりが多く美しいこと。
\\	彼らはその英雄の銅像を建てた。	【どうぞう】 青銅で鋳造した像。特に、野外に置かれる記念碑的な像をさすことが多い。
\\	陸上鉄道運輸。	【うんゆ】 旅客・貨物を運び送ること。ふつう、鉄道・自動車・船舶・航空機によるものを総合していう。輸送。
\\	指の切り傷は3日で治癒した。	【ちゆ】 病気やけがなどがなおること。
\\	大統領は一部の資本家と癒着していると非難された。	【ゆちゃく】 好ましくない状態で強く結びついていること。
\\	腸が癒着する。	【ゆちゃく】 本来は分離しているはずの臓器・組織面が、外傷や炎症のために、くっつくこと。
\\	胃腸のぐあいが悪い。	【いちょう】 消化器官の胃と腸。
\\	死体を解剖する。	【かいぼう】 生物体を切り開いて、内部の構造、あるいは病変・死因なども観察すること。腑分(ふわ)け。解体。
\\	私たちは内臓器官を調べるためにカエルを解剖した。	【きかん】 多細胞生物において、いくつかの組織が集まって一定の形・大きさおよび生理機能をもつ部分。
\\	盲腸炎の手術をする。	【もうちょう】 小腸に続く、大腸の初部。小腸が横から連なるため、下端が盲管となり、その先に虫垂(ちゅうすい)がある。草食動物では比較的長く、消化に関与する。
\\	あなたは知力で彼に匹敵している。	【ひってき】 比べてみて能力や価値などが同程度であること。肩を並べること。
\\	それには絶対的で普遍的な意味はない。	【ふへんてき】 広く行き渡るさま。極めて多くの物事にあてはまるさま。
\\	彼はピアノを楽譜なしで弾いた。	【がくふ】 歌曲または楽曲を、一定の約束に従って、記号を用いて書き表したもの。
\\	君には脱帽するよ。	【だつぼう】 敬意を表して、かぶっている帽子をぬぐこと。
\\	男性は一人で運搬トラックに荷物を詰め込んでいる。	【うんぱん】 物品を運び移すこと。
\\	椅子とかテーブルとかの搬入はどうするの?	【はんにゅう】 品物を運び入れること。
\\	源をすっかり消耗してしまった。	【しょうもう】 使って減らすこと。また、使って減ること。
\\	一日中歩き回って消耗した。	【しょうもう】 体力や気力を使い果たすこと。
\\	大統領は憲法を一時停止し、戒厳令をしきました。	【かいげんれい】 「戒厳」を宣告する命令。明治憲法下では天皇がこれを宣告した。
\\	候補者たちは減税法案をめぐって応酬しました。	【おうしゅう】 互いにやり取りすること。また、先方からしてきたことに対して、こちらからもやり返すこと。
\\	彼女に花束を進呈した。	【しんてい】 人に物を差し上げること。進上。
\\	私達は彼に時計を贈呈した。	【ぞうてい】 人に物を贈ること。物を差し上げること。進呈。
\\	捜査員らは暗殺計画を摘発しました。	【てきはつ】 悪事などをあばいて世間に発表すること。
\\	事件の後失踪する。	【しっそう】 行方をくらますこと。また、行方が知れないこと。失跡。
\\	我々は行動を強制されたり禁止されたりする。	【きょうせい】 権力や威力によって、その人の意思にかかわりなく、ある事を無理にさせること。
\\	戦争で兵士達は戦線に行かざるをえなかった。	【せんせん】 戦闘を交えている地域。戦闘の第一線。戦場。戦闘線。
\\	重税のため民心は政府から離反している。	【りはん】 従っていたものなどが、そむきはなれること。
\\	彼はその政党の主導権を握った。	【しゅどうけん】 主となって物事を動かし進めることができる力。イニシアチブ。
\\	政府はこれらの法規を廃棄すべきである。	【はいき】 条約を当事国の一方の意思によって効力を失わせること。
\\	彼が外人客の接待にあたっている。	【せったい】 客をもてなすこと。もてなし。
\\	抜本的な対策を打ち出した。	【ばっぽんてき】 根本に立ち戻って是正するさま。
\\	具体策を打ち出す。	【うちだす】 主張などをはっきり示す。提案する。
\\	彼女は才色兼備だ。	【さいしょくけんび】 女性がすぐれた才知と美しい顔かたちをもっていること。
\\	彼らは自由党員たちと同盟した。	【どうめい】 個人・団体または国家などが、互いに共通の目的を達成するために同一の行動をとることを約束すること。また、それによって成立した関係。
\\	彼らの親密さは年月とともに深まった。	【しんみつ】 互いの交際の深いこと。きわめて仲のよいこと。
\\	今日では対外関係と国内問題には親密な関係がある。	【たいがい】 外部、あるいは外国に対すること。
\\	対外貿易は巨額の収入をもたらす。	【きょがく】 金額が非常に多いこと。
\\	党を牛耳っているのは彼だ。	【ぎゅうじる】 団体や組織を支配し、思いのままに動かす。牛耳を執る。
\\	1杯の茶で蘇った気持ちになった。	【よみがえる】 一度衰退したものが、再び盛んになる。
\\	簡潔に言うと、ホブボームの論点は20世紀の歴史は文化衰退の歴史であるということだ。	【すいたい】 勢いや活力が衰え弱まること。
\\	あの店はカメラを桁外れの安値で売る。	【けたはずれ】 価値や規模などが、他からかけ離れていること。けたちがい。
\\	彼は地雷の上を運転して、彼のジープは爆裂した。	【じらい】 地中に埋め、その上を通る人や戦車などを爆破する装置の爆薬。
\\	爆薬がどかんと鳴った。	【ばくやく】 火薬類のうち、物を破壊するための爆破薬、および爆弾・砲弾を炸裂(さくれつ)させるための炸薬のこと。ダイナマイト・トリニトロトルエンなど。
\\	花火をどかんと打ち上げた。	爆発したり破裂したときの大きな音を表す。
\\	航空機の発達のおかげで、世界はずっと狭くなった。	【こうくうき】 人が乗って空中を航行する機器の総称。飛行船・気球・グライダー・飛行機・ヘリコプターなど。現在では主に飛行機をさす。
\\	その惨事はあのテロリストのせいだ。	【さんじ】 悲惨な出来事。いたましい事件。
\\	囚人は警官を殺害したことを否定した。	【さつがい】 人を殺すこと。
\\	枕を高くして熟眠する。	【じゅくみん】 ぐっすり眠ること。熟睡。
\\	彼らはボストンを永住の地と定めた。	【えいじゅう】 長く、ある土地に住み着くこと。死ぬまでその土地に住むこと。
\\	便座におしっこしちゃったら、ちゃんと拭くのよ!	【べんざ】 洋式便器の腰かける部分。
\\	音楽家が死んで遺族は貧困にさらされた。	【いぞく】 死んだ人のあとに残された家族・親族。
\\	老若男女を問わず、	【ろうにゃくなんにょ】 老人も若者も、男も女も含む、あらゆる人々。
\\	川下りでびしょびしょになった。	【かわくだり】 船で周囲の景観を楽しみながら川を下ること。
\\	船が転覆し大勢の乗客が海に投げ出された。	【てんぷく】 列車・船などがひっくり返ること。また、ひっくり返すこと。
\\	虫眼鏡で花粉を見る。	【虫眼鏡】 小さい物体を拡大して見るために用いる、焦点距離の短い凸レンズ。拡大鏡。ルーペ。
\\	ピアノとバイオリンでは音色が違う。	【ねいろ】 発音体の違い、あるいは同じ発音体でも音の出し方によって生じる、音の感覚的な特性。高さや強さが同じ音でも、それに含まれる部分音の種類や強さなどによって違いが生じる。
\\	戦争中多くの残虐行為が行われた。	【ざんぎゃく】 人や生き物に対してする行為のむごたらしいこと。
\\	動物に対して惨いことをしてはいけない。	【むごい】 見るにたえないほど痛ましい。残酷である。
\\	先生は答案の採点に忙しい。	【さいてん】 評価して点数をつけること。また、つけた点数。
\\	先生は我々の試験を採点する時とても公正だった。	【こうせい】 偏りのないこと。
\\	栄養に偏りのないようにする。	【かたより】 かたよること。
\\	脱走犯人はまだ捕まらない。	【だっそう】 束縛されている場所から抜け出して逃げること。
\\	まつわりつく子供を振り切って家を出て来ました	【ふりきる】 からだにまとわりつくものなどを、強く振るようにして離れさせる。振り離す。
\\	泥棒は警察署に連行された。	【れんこう】 本人の意思にかかわらず、連れて行くこと。特に、警察官が犯人・容疑者などを警察署へ連れて行くこと。
\\	囚人は連行していた看守の手を振り切って脱走した。	【かんしゅ】 刑務所などで、囚人の監督・警備などに従事する法務事務官。
\\	そのレストランはニューヨークの楽しい穴場ですよ。	【あなば】 一般の人にあまり知られていない、いいところ。あな。
\\	発電所の事故で6時まで送電出来ない。	【そうでん】 発電所から変電所または配電所に電力を送ること。
\\	この国は食糧を自給自足している。	【じきゅうじそく】 必要な物資を自分自身の力で生産して満たすこと。
\\	1639年以降日本は外国料理に対して門戸を閉ざした。	【もんこ】 他と交流し、また外部のものを受け入れるための入り口。
\\	彼の手法は全く驚くべきものだった。	【しゅほう】 物事のやり方。特に、芸術作品などをつくるうえでの表現方法。技法。
\\	外交対話のおかげでその紛争に終止符を打つ事が出来た。	【しゅうしふをうつ】 終わりにする。結末をつける。ピリオドを打つ。
\\	暗雲が垂れ込める。	【あんうん】 真っ黒な雲。今にも雨や雪が降りだしそうな気配のある暗い雲。
\\	その辺までしか許容できない。	【きょよう】 そこまではよいとして認めること。大目にみること。
\\	彼らはその格子を取り壊した。	【こうし】 細い角材や竹などを、碁盤の目のように組み合わせて作った建具。戸・窓などに用いる。
\\	囚人は今日は独房の中で静かにしている。	【どくぼう】 刑務所や拘置所で、収容者を一人だけ入れておく監房。
\\	殺人の有罪宣告を受け、彼は終身刑を科せられた。	【しゅうしんけい】 無期限で、人が生涯服する自由刑。無期懲役と無期禁固とがある。
\\	社会に反逆する。	【はんぎゃく】 権威・権力などにさからうこと。
\\	反逆者達は首都を制圧した。	【せいあつ】 威力で相手を押さえつけること。
\\	その反逆者は市民権を剥奪された。	【はくだつ】 はぎ取ってうばうこと。力ずくで取り上げること
\\	反逆者はついに捕まり牢獄に監禁された。	【ろうごく】 罪人を入れておく所。牢屋。
\\	トムは小川さんから多くの人が成金を軽蔑すると聞いた	【なりきん】 急に金持ちになること。また、その人。
\\	「なに、アンタ疑ってんの」「だ、だってそんな、いきなり魔界とか、信じろという方がおかしいよ」	【まかい】 悪魔のいる世界。
\\	叔父は大の愛煙家です。彼にとってタバコは欠くことができない。	【あいえんか】 タバコを好んで吸う人。タバコ好き。
\\	の洞窟が私たちの隠れ家になった。	【かくれが】 人目を避けて隠れている場所。また、隠れ住む家。
\\	寒くて歯がガチガチ鳴った。	堅い物が連続してぶつかり合う音を表す語。
\\	全ての子供には、尊敬し模倣する人が必要だ。	【もほう】 他のものをまねること。似せること。
\\	各党派の色分けを説明した	【いろわけ】 人や事物をある基準によって種類分けすること。ふるいわけ。
\\	二つの党派に割れる。	【とうは】 考え方・主義や利害関係などを同じくする人々の集まり。党。
\\	彼女と話をしているとくたびれる。	長時間からだや頭を使ったため、疲れて元気がなくなる。
\\	原油は精製されて多くの製品を産出する。	【せいせい】 まじりものを除いて、純良なものをつくりあげること。
\\	言うまでもないが、ノルウェーは世界第二の原油供給国となっている。	【げんゆ】 油井(ゆせい)から採掘されたままの精製していない石油。通常は黒色の悪臭ある液体。
\\	バスの車掌は彼女に、料金を払えないので降りるようにと言った。	【しゃしょう】 電車・汽車・バスなどに乗って、旅客・荷物などの車内の事務を取り扱う者。
\\	その夜はその小屋で雑魚寝した。	【ざこね】 大勢の人が雑然と入り交じって寝ること。
\\	彼は浴槽に水を出しっぱなしにしている。	【よくそう】 湯ぶね。ふろおけ。
\\	野党は政府の怠慢を糾弾した。	【きゅうだん】 罪や責任を問いただし、非難すること。
\\	彼は他の大勢の人達に共に亡命した。	【ぼうめい】 政治的弾圧や思想の相違、宗教・人種的な理由による迫害を避けるために自国から外国へ逃れること。
\\	マヨネーズの油は分離することがある。	【ぶんり】 分かれて離れること。また、分けて離すこと。
\\	我々のクラブは国際的な組織に加入した。	【かにゅう】 団体や組織などの仲間に加わること。
\\	うわさはあっという間に広く伝播する。	【でんぱ】 伝わり広まること。広く伝わること。
\\	クリスト教が伝播する前の多神教の時代が浮き上がって感じられる。	【たしんきょう】 多数の神々を信じ礼拝する宗教。それぞれの神が固有の活動領域をもつ。古代ギリシャ・ローマの宗教など。
\\	彼女の4人の姉妹のうち1人は他界したが、ほかは健在だ。	【けんざい】 元気で無事に暮らしていること。
\\	予想は、寸分違わぬくらい当った。	【すんぶん】 ごくわずかの分量または程度。少し。わずか。
\\	幸福と言うものを世俗的な成功と言う点から考えるのは間違っている。	【せぞく】 俗世間。また、俗世間の人。
\\	概ねは了承した。	【おおむね】 だいたいの趣旨。あらまし。
\\	まず第1に、雑貨店を経営するには資金が不十分だ。	【ざっか】 日常生活に必要なこまごました品物。
\\	人類は太古の昔から指を用いて食べ物を食してきたのである。	【たいこ】 非常に遠い昔。大昔。有史以前。
\\	眼下の景色を見る。	【がんか】 見下ろした辺り一帯。
\\	古都の名残をとどめる。	【なごり】 ある事柄が過ぎ去ったあとに、なおその気配や影響が残っていること。また、その気配や影響。
\\	新しい実験は定説をくつがえした。	【ていせつ】 一般に認められ、確定的であるとされている説。
\\	脇の下を擽る。	【くすぐる】 皮膚の敏感な部分を軽く刺激し、むずむずして笑いたくなるような感じを起こさせる。
\\	彼の手紙で彼女は虚栄心を傷つけられた。	【きょえいしん】 自分を実質以上に見せようと、みえを張りたがる心。
\\	その言葉は彼女の虚栄心を擽った。	【くすぐる】 人の心を軽く刺激し、そわそわさせたり、いい気持ちにさせたりする。
\\	官能をくすぐられる。	【かんのう】 肉体的快感、特に性的感覚を享受する働き。
\\	快楽しか求めない人もいる。	【かいらく】 心地よく楽しいこと。官能的な欲望の満足によって生じる、快い感情。けらく。
\\	彼は家に帰りたくてむずむずしていた。	やる気にあふれ、落ち着かないさま。また、やりたいことができなくてもどかしく思うさま。うずうず。
\\	彼は新しい単車が買いたくてうずうずしている。	ある行動をしたくて、じっとしていられないさま。むずむず。
\\	彼は妻を毒殺したと言われている。	【どくさつ】 毒物・毒薬を用いて殺すこと。毒害。
\\	その州では未だに古い慣習が根強い、とその文化人類学者は言っている。	【ねづよい】 深く根を張っていて揺るがない。基本がしっかりしていて強い。
\\	しかし、イエスはご自分の身体の神殿の事を言われたのである。	【しんでん】 神社の中心で、神体・神像など崇拝の対象を安置する殿社。
\\	黄金を捜し求めて多くの男たちが西部へと出かけていった。	【おうごん】 こがね。きん。
\\	教育にもっとお金をかければ経済成長に拍車が掛かるだろう。	【はくしゃがかかる】 進行が一段と速くなる。
\\	チームが弱体化する。	【じゃくたいか】 組織などの力が衰えること。
\\	部外者の立ち入りを禁ず。	【ぶがい】 その機関・組織に関係のないこと。また、役所・会社などの部に属していないこと。
\\	像の土台に彼らの願いが刻まれている。	【どだい】 建築物の最下部にあって、上の重みを支えるもの。基礎。
\\	彼は考古学者の助手である。	【こうこがく】 遺跡や遺物によって、古い時代の人類の生活・文化を研究する学問。
\\	ローマは至る所に遺跡がある。	【いせき】 貝塚・古墳・集落跡など、過去の人類の生活・活動のあと。遺物・遺構のある場所。
\\	スミスさんは奥さんに先立たれた。	【さきだつ】 先に死ぬ。
\\	そんな非情なことはできない	【ひじょう】 人間らしい感情をもたないこと。感情に左右されないこと。
\\	今週は公園の桜が満開だ。	【まんかい】 花が十分に開くこと。また、すべての花が開くこと。はなざかり。
\\	花はまだ蕾だ。	【つぼみ】 花の、まだ咲き開かないもの。
\\	バラの花は満開よりも蕾の方が甘美である。	【かんび】 心地よくうっとりとした気持ちにさせること。
\\	その泥棒は少年にナイフを突き付けようとした。	【つきつける】 荒々しく凶器などを相手の目の前に差し出す。
\\	淡い味付け。	【あわい】 色や味などが際立たず、薄い。
\\	真っ青な空では雲か万華鏡のように姿を変え、淡く輝く夜空は光を吸い込む。	【まんげきょう】 
\\	技術革新のおかげで、その工場の最大生産量は2倍になった。	【かくしん】 旧来の制度・組織・方法・習慣などを改めて新しくすること。特に、政治では、現状を改革しようとする立場。
\\	彼らは不屈の抵抗をした。	【ふくつ】 どんな困難にぶつかっても、意志を貫くこと。
\\	下水が完全に詰まっている。	【げすい】 
\\	住宅の台所・風呂場や、工場などから流れ出る汚れた水。⇔上水。 
\\	下水を流す溝。下水道。下水管。
\\	数年後、ヨーロッパ人たちは沿岸の植民地に住み着いた。	【えんがん】 陸地の、海・川・湖などに沿った部分。
\\	一見してその空家には修繕が大いに必要であるのがわかった。	【しゅうぜん】 壊れたり悪くなったりしたところを繕い直すこと。修理。
\\	その寝室には、美しい装飾品がいっぱいあった。	【そうしょく】 飾ること。美しく装うこと。また、その装い・飾り。
\\	この雄大な海の眺めをごらんなさい。	【ゆうだい】 規模が大きく堂々としていること。また、そのさま。
\\	谷には農家が点在していた。	【てんざい】 あちこちに散らばって存在すること。散在。
\\	その出版社は児童文学を専門にしている。	【じどう】 心身ともにまだ十分発達していない者。子供。特に、学校教育法で、満6~12歳の学齢児童。
\\	われわれの仲間ではあなたは異色の存在だった。	【いしょく】 他と異なって特色のあること。
\\	ジョンはトラを捕まえ、2匹のライオンを射殺した。	【しゃさつ】 銃や弓などでうち殺すこと。
\\	射殺された時、その警官は非番だった。	【ひばん】 当番でないこと。また、その人。
\\	私はその重たい袋を背負って運んだ。	【おもたい】 目方が多い。
\\	うちはパパが別居中なの。(ママがどうしようもない人だから)	【べっきょ】 夫婦・家族などが別れて住むこと。
\\	彼の死因はいまだに謎である。	【しいん】 死亡の原因。
\\	過度の脂肪の摂取は心臓病の原因になるとされている。	【せっしゅ】 取り入れて自分のものにすること。また、栄養物などを体内に取り入れること。
\\	家の柵から抜け出せない。	【しがらみ】 引き留め、まとわりつくもの。じゃまをするもの。
\\	彼が早い汽車に乗ったのは定かだ	【さだか】 事実として、はっきりしているさま。確実。あきらか。
\\	ロシア北西部のムルマンスクから出航し、1万5000㌧砕氷船で北極海を約6000㌔航海する16日間の旅だ。	【さいひょうせん】 氷海の氷を砕きながら航行できるように設計された船。
\\	泥が服に飛び散った。	【とびちる】 飛んであちこちへ散る。飛散する。
\\	火花が飛び散った。	【ひばな】 放電の際に発する光。スパーク。
\\	主導権をめぐって火花を散らす。	【ひばなをちらす】 互いに激しく刀を打ち合わせて戦う。転じて、激しく争う。火を散らす。
\\	ギターを奏でる。	【かなでる】 楽器、特に管弦楽器を演奏する。
\\	かなりあくどい事をしてきたそうだな。	程度を超えてどぎつい。やり方が行きすぎてたちが悪い。
\\	わき目も振らず猛進する。	【もうしん】 勢い激しく突き進むこと。
\\	右傾化している現在。	【うけい】 思想が国粋主義・ファシズムの右翼的な立場に傾いていること。⇔左傾。
\\	彼は右翼的である。	【うよく】 保守的または国粋的な思想、立場の一派。
\\	この団体はいくらか左翼に傾いている。	【さよく】 社会主義・共産主義・無政府主義などの革新的な思想。また、そのような立場の人・団体。左派。
\\	私の部屋は誰も人をいれない安息の場所です。	【あんそく】 何のわずらいもなく、くつろいで休むこと。
\\	多元的な考え方。	【たげんてき】 物事の要素・根源がいくつもあるさま。⇔一元的。
\\	民衆の怒りと不満が融合して暴動が起こった。	【ゆうごう】 とけあうこと。とけあって一つのものになること。
\\	銀行はその会社に融資をした。	【ゆうし】 資金を融通すること。
\\	資金を融通する。	【ゆうずう】 必要な物や金を都合すること。やりくり。
\\	遣り繰りするために一生懸命働いた。	【やりくり】 不十分なものをあれこれ工夫して都合をつけること。
\\	私たちの学校は健全な環境に囲まれている。	【けんぜん】 身心が正常に働き、健康であること。
\\	高度に硫黄を含む。	【いおう】 
\\	彼は前言を取り消した。	【ぜんげん】 前に言った言葉。
\\	郵便局が民営化された。	【みんえいか】 国や地方公共団体が経営する企業・特殊法人などを民間会社や特殊会社にすること。
\\	この病院は民営である。	【みんえい】 民間で経営すること。
\\	共産主義の下では、生産手段は国有化される。	【こくゆう】 国家が所有すること。
\\	次の選挙では民主党が共和党に勝つものと予想されている。	【きょうわとう】 民主党と並ぶ米国の二大政党の一つ。
\\	今夏の降水量は普通でなかった。	【こうすい】 大気中の水蒸気が雨や雪などになって地上に落下する現象。また、その雨や雪。
\\	明日、月見の会があるだろう。	【つきみ】 月をながめて楽しむこと。特に陰暦八月十五夜
\\	ようやく研究完成の目処がついた。	【めど】 目指すところ。目当て。また、物事の見通し。
\\	象牙に彫った仏像。	【ぞうげ】 象の上あごにある長く伸びた一対の門歯。細かい木目状の縞模様があり、適度の硬さなので細工物に用いられた。
\\	彼女は血を見て卒倒した。	【そっとう】 脳貧血(のうひんけつ)などにより突然意識を失って倒れること。
\\	鼻血が止まりません。	【はなぢ】 鼻からの出血。
\\	本当に悪いんだけど、君の襟巻き、どっかに置いてきてしまったらしいんだよ。	【えりまき】 防寒または装飾用に首に巻くもの。毛糸・毛皮・絹布などで作る。首巻き。マフラー。
\\	コンピューター端末はずっと一列に並んでいた。	【たんまつ】 コンピューターの入出力装置のある部分。
\\	彼は捕虜に対して極悪非道な扱いをした。	【ひどう】 人としてのあり方や生き方にはずれていること。
\\	極悪な犯罪。	【ごくあく】 この上なく悪いこと。
\\	彼女は浮気な女で本当に誰でも相手にする。	【うわき】 異性に心をひかれやすいこと。
\\	彼女は異性の前では非常に恥ずかしがる。	【いせい】 男女・雌雄の性が異なること。特に、男性から女性を、女性から男性をさしていう。
\\	選手権を奪回する。	【だっかい】 奪われたものを取り戻すこと。奪い返すこと。
\\	彼女の成功は彼女を嫉妬の標的にした。	【ひょうてき】 攻撃目標。ターゲット。
\\	あの種の商売なら堅実だ。	【けんじつ】 手堅く確実なこと。確かであぶなげのないこと。
\\	飛行機は滑走路に着陸した。	【かっそうろ】 飛行機の離着陸時の滑走に用いる、飛行場内に設けられた直線状の舗装路。
\\	雇い主は、時に従業員を搾取する。	【さくしゅ】 本来他人のものになるはずのものを自分のものにすること。
\\	節操を守って行動する	【せっそう】 節義を堅く守って変えないこと。自分の信じる主義・主張などを守りとおすこと。みさお。
\\	彼は私の手伝いをするのを露骨にいやがった。	【ろこつ】 感情などを隠さずに、ありのまま外に表すこと。また、そのさま。むきだし。あらわ。
\\	私達は通例、1日に3回食事をする。	【つうれい】 一般に。通常。
\\	この交響曲は真の傑作だ。	【こうきょうきょく】 シンフォニー。
\\	それでは好ましくない先例を作ることになる	【せんれい】 以前にあった同類の例。また、これまでのしきたり。前例。
\\	彼は洗礼を受けてカトリック教徒となった。	【せんれい】 キリスト教徒となるために教会が執行する儀式。全身を水にひたすか、または頭部に水を注ぐことによって罪を洗い清め、神の子として新しい生命を与えられるあかしとする。バプテスマ。
\\	ニューヨークには高層ビルが林立している。	【りんりつ】 林のように、たくさんのものが並び立つこと。
\\	入試の前に校舎の下見をする。	【したみ】 前もって見て調べておくこと。
\\	すぐ警察に通報してください。	【つうほう】 情報・ニュースなどを告げ知らせること。また、その知らせ。
\\	標高がぐんと高くなると呼吸困難を感じる。	【ひょうこう】 ある地点の、平均海水面からの高さ。日本では東京湾の平均海水面を零メートルとする。海抜。
\\	紅葉の渓谷をさかのぼる。	【けいこく】 山にはさまれた、川のある所。たに。
\\	川を遡る。	【さかのぼる】 流れに逆らって上流に進む。
\\	露は太陽が昇ると蒸発した。	【じょうはつ】 液体がその表面から気化する現象。
\\	降伏条件は過酷だった。	【かこく】 厳しすぎるさま。ひどすぎるさま。
\\	その兄弟の間には強い絆がある。	【きずな】 人と人との断つことのできないつながり。離れがたい結びつき。
\\	彼はロンドンの郊外に定住することにした。	【ていじゅう】 一定の場所に住居を構え、そこに住みつくこと。
\\	その病気は蔓延しつつあるようだ。	【まんえん】 病気や悪習などがいっぱいに広がること。
\\	彼らは上司の考え方に懐疑的だった。	【かいぎ】 物事の意味・価値、また自他の存在や見解などについて疑いを持つこと。
\\	彼はその事件とまったく無縁ではない。	【むえん】 縁のないこと。関係のないこと。
\\	キリシタンの迫害の犠牲になって殉教した。	【じゅんきょう】 自らの信仰のために生命をささげること。
\\	ここの人たちは気風が荒い。	【きふう】 気性。気だて。特に、ある集団・地域内の人々に共通する気質。
\\	近頃は地震が頻繁で不気味だ。	【ひんぱん】 しきりに行われること。しばしばであること。
\\	この小島ではあらゆる文明から隔絶された感じがした	【かくぜつ】 かけ離れていること。遠くへだたっていること。
\\	さまざまな原因が複合して起きた事故。	【ふくごう】 複数のものが合わさって一つのものになること。
\\	来週は東京に寒波の襲来がある。	【しゅうらい】 激しい勢いでおそいかかってくること。来襲。
\\	彼らは村を洪水から守るために堤防を補強した。	【ほきょう】 弱い部分や足りないところを補って強くすること。
\\	まもなく彼女は舞台に復帰するだろう。	【ふっき】 もとの位置・状態などに戻ること。
\\	人の価値はその人の外観とは無関係だ。	【がいかん】 外側から見た感じ。見かけ。うわべ。外見。
\\	あの映画は主題歌の人気と相まって大ヒットした。	【しゅだいか】 テーマソング。
\\	戦車を偽装する。	【ぎそう】 周囲のものと似た色や形にして姿を見分けにくくすること。特に、戦場などで行うもの。カムフラージュ。
\\	心中に偽装した殺人。	【ぎそう】 ある事実をおおい隠すために、他の物事・状況をよそおうこと。
\\	ここでは人間から微生物に至るまで、あらゆる生物の研究が行われている。	【びせいぶつ】 顕微鏡で拡大しなければよく見えない微細な生物。細菌・酵母・原生動物、菌類の一部など。ウイルスを含め、また藻類まで含めることもある。
\\	大使館は最高裁判所に隣接している。	【りんせつ】 となり合っていること。
\\	要所を兵隊が固めていた。	【ようしょ】 重要な地点・場所。
\\	列車の後部3両はひどい損傷を受けた。	【そんしょう】 人や物などが損なわれ傷つくこと。また、損ない傷つけること。
\\	外国人労働者の流入が、この地域で深刻な住宅難を引き起こした。	【りゅうにゅう】 多くの人や金などが外部から入り込んでくること。
\\	梅は春の先駆けである。	【さきがけ】 他のものより先になること。
\\	春に先駆けて梅が咲いた。	【さきがける】 他に先んじて事をする。
\\	溶け続ける氷河を監視するために、人工衛星が軌道に打ち上げられた。	【ひょうが】 
\\	その孤児は金持ちに育てられた。	【みなしご】 両親のいない子。
\\	国家の存亡にかかわることである。	【そんぼう】 存在と滅亡。存続するか消滅するかということ。
\\	寺を建立する。	【こんりゅう】 寺院や堂・塔などを建てること。
\\	その大聖堂の建立は中世にまでさかのぼる。	【だいせいどう】 カテドラル。
\\	私の車は馬力が足らない。	【ばりき】 
\\	コンクリートは中に鋼鉄の棒を入れる事で補強される。	【こうてつ】 
\\	妻は家を買いたい欲求に取りつかれている。	【よっきゅう】 強くほしがって求めること。
\\	太陽はどんなに麗しく輝いていようとも沈まなくてはならない。	【うるわしい】 精神的に豊かで気高く、人に感銘を与えるさま。心あたたまり、うつくしい。
\\	彼女は花嫁のような衣装を着ている。	【いしょう】 着物。衣服。
\\	彼は自分自身に似た連中が大嫌いだった。	【れんちゅう】 仲間である者たち。また、同じようなことをする者たちをひとまとめにしていう語。
\\	クラスで最下位である。	【さいかい】 地位や順位などが最も下であること。
\\	もし万一再び失敗すると、私はその計画を断念するだろう。	【だんねん】 自分の希望などを、きっぱりとあきらめること。
\\	彼は経済にかけては誰にも負けないと自負している。	【じふ】 名](スル)自分の才能・知識・業績などに自信と誇りを持つこと。
\\	今年は桜の花が例年よりも少し遅れている。	【れいねん】 いつもの年。毎年。
\\	昨日の株価下落に伴い、今日の取引は低迷だった。	【ていめい】 よくない状態から抜け出せないでいること。
\\	濃厚な味付けの料理。	【のうこう】 味・色・におい・成分などが濃いさま。こってりとしたさま。
\\	彼らの濃厚なキスにあてられた。	【のうこう】 男女の仲がきわめて熱情的なさま。
\\	彼女は自分の運転の腕前を誇りにしている。	【うでまえ】 巧みに物事をなしうる能力や技術。手並み。技量。うで。
\\	音楽に拍子を合わせて踊った。	【ひょうし】 音楽用語。 ①音楽のリズムを形成する基本単位。
\\	交渉はとんとん拍子に進んだ。	【とんとんひょうし】 物事が順調にはかどること。
\\	祖母は少しも生活様式を変えなかった。	【せいかつようしき】 ある社会・集団に属する人に共通してみられる生活の型。
\end{CJK}
\end{document}