\documentclass[8pt]{extreport} 
\usepackage{hyperref}
\usepackage{CJKutf8}
\begin{document}
\begin{CJK}{UTF8}{min}
\\	一片の石
\\	會津八一
\\	【テキスト中に現れる記号について】
\\	ルビ
\\	(例)極《はて》
\\	入力者注 主に外字の説明や、傍点の位置の指定
\\	(数字は、
\\	0213の面区点番号、または底本のページと行数)
\\	(例)羊※[#「示+古」、第3水準1-89-26]
\\	人間が石にたよるやうになつて、もうよほど久しいことであるのに、まだ根気よくそれをやつてゐる。石にたより、石に縋り、石を崇め、石を拝む。この心から城壁も、祭壇も、神像も、殿堂も、石で作られた。いつまでもこの世に留めたいと思ふ物を作るために、東洋でも、西洋でも、あるひは何処の極《はて》でも、昔から人間が努めてゐる姿は目ざましい。人は死ぬ。そのまま地びたに棄てておいても、膿血や腐肉が流れつくした後に、骨だけは石に似て永く遺るべき素質であるのに、遺族友人と称へるものが集つて、火を点けて焼く。せつかくの骨までが粉々に砕けてしまふ。それを拾ひ集めて、底深く地中に埋めて、その上にいかつい四角な石を立てる。御参りをするといへば、まるでそれが故人であるやうに、その石を拝む。そして、その石が大きいほど貞女孝子と褒められる。貧乏ものは、こんな点でも孝行がむづかしい。
\\	なるほど、像なり、建物なり、または墓なり何なり、凡そ人間の手わざで、遠い時代から遺つてゐるものはある。しかし遺つてゐるといつても、時代にもよるが、少し古いところは、作られた数に較べると、千に一つにも当らない。つまり、石といへども、千年の風霜に曝露されて、平気でゐるものではない。それに野火や山火事が崩壊を早めることもある。いかに立派な墓や石碑でも、その人の名を、まだ世間が忘れきらぬうちから、もう押し倒されて、倉の土台や石垣の下積みになることもある。追慕だ研究だといつて跡を絶たない人たちの、搨拓の手のために、磨滅を促すこともある。そこで漢の時代には、いづれの村里にも、あり余るほどあつた石碑が、今では支那全土で百基ほどしか遺つてゐない。国破れて山河ありといふが、国も山河もまだそのままであるのに、さしもに人間の思ひを籠めた記念物が、もう無くなつてゐることは、いくらもある。まことに寂しいことである。
\\	むかし晋の世に、羊※[#「示+古」、第3水準1-89-26]といふ人があつた。学識もあり、手腕もあり、情味の深い、立派な大官で、晋の政府のために、呉国の懐柔につくして功があつた。この人は平素山水の眺めが好きで、襄陽に在任の頃はいつもすぐ近い※[#「山+見」、第3水準1-47-77]山といふのに登つて、酒を飲みながら、友人と詩などを作つて楽しんだものであるが、ある時、ふと同行の友人に向つて、一体この山は、宇宙開闢の初めからあるのだから、昔からずゐぶん偉い人たちも遊びにやつて来てゐるわけだ。それがみんな湮滅して何の云ひ伝へも無い。こんなことを考へると、ほんとに悲しくなる。もし百年の後にここへ来て、今の我々を思ひ出してくれる人があるなら、私の魂魄は必ずここへ登つて来る、と嘆いたものだ。そこでその友人が、いやあなたのやうに功績の大きな、感化の深い方は、その令聞は永くこの山とともに、いつまでも世間に伝はるにちがひありませんと、やうやくこのさびしい気持を慰めたといふことである。それから間もなくこの人が亡くなると、果して土地の人民どもは金を出し合つてこの山の上に碑を立てた。すると通りかかりにこの碑を見るものは、遺徳を想ひ出しては涙に暮れたものであつた。そのうちに堕涙の碑といふ名もついてしまつた。
\\	同じ頃、晋の貴族に杜預といふ人があつた。年は羊※[#「示+古」、第3水準1-89-26]よりも一つ下であつたが、これも多識な通人で、人の気受けもよろしかつた。襄陽へ出かけて来て、やはり呉の国を平げることに手柄があつた。堕涙の碑といふ名なども、実はこの人がつけたものらしい。羊※[#「示+古」、第3水準1-89-26]とは少し考へ方が違つてゐたが、この人も、やはりひどく身後の名声を気にしてゐた。そこで自分の一生の業績を石碑に刻んで、二基同じものを作らせて、一つを同じ※[#「山+見」、第3水準1-47-77]山の上に立て、今一つをば漢江の深い淵に沈めさせた。万世の後に、如何なる天変地異が起つて、よしんば山上の一碑が蒼海の底に隠れるやうになつても、その時には、たぶん谷底の方が現はれて来る。こんな期待をかけてゐたものと見える。
\\	ところが後に唐の時代になつて、同じ襄陽から孟浩然といふ優れた詩人が出た。この人もある時弟子たちを連れて※[#「山+見」、第3水準1-47-77]山の頂に登つた。そして先づ羊※[#「示+古」、第3水準1-89-26]のことなどを思ひ出して、こんな詩を作つた。
\\	[#ここから2字下げ]
\\	人事代謝あり、
\\	往来して古今を成す。
\\	江山は勝迹を留め、
\\	我輩また登臨す。
\\	水落ちて魚梁浅く、
\\	天寒うして夢沢深し。
\\	羊公碑尚ほあり。
\\	読み罷めて涙襟を沾す。
\\	[#ここで字下げ終わり]
\\	この一篇は、この人の集中でも傑作とされてゐるが、その気持は全く羊※[#「示+古」、第3水準1-89-26]と同じものに打たれてゐるらしかつた。
\\	この人よりも十二年遅れて生れた李白は、かつて若い頃この襄陽の地に来て作つた歌曲には、
\\	[#ここから2字下げ]
\\	※[#「山+見」、第3水準1-47-77]山は漢江に臨み、
\\	水は緑に、沙は雪のごとし。
\\	上に堕涙の碑のあり、
\\	青苔して久しく磨滅せり。
\\	[#ここで字下げ終わり]
\\	とか、また
\\	[#ここから2字下げ]
\\	君見ずや、晋朝の羊公一片の石、
\\	亀頭剥落して莓苔を生ず。
\\	涙またこれがために堕つ能はず、
\\	心またこれがために哀しむ能はず。
\\	[#ここで字下げ終わり]
\\	とか、あるひはまた後に追懐の詩の中に
\\	[#ここから2字下げ]
\\	空しく思ふ羊叔子、
\\	涙を堕す※[#「山+見」、第3水準1-47-77]山のいただき。
\\	[#ここで字下げ終わり]
\\	と感慨を詠じたりしてゐる。
\\	なるほど、さすがの羊公も、今は一片の石で、しかも剥落して青苔を蒙つてゐる。だから人生はやはり酒でも飲めと李白はいふのであらうが、ここに一つ大切なことがある。孟浩然や李白が涙を流して眺め入つた石碑は、羊公歿後に立てられたままでは無かつたらしい。といふのは、歿後わづか二百七十二年にして、破損が甚しかつたために、梁の大同十年といふ年に、原碑の残石を用ゐて文字を彫り直すことになつた。そして別にその裏面に、劉之※[#「二点しんにょう+隣のつくり」、105-8]の属文を劉霊正が書いて彫らせた。二人が見たのは、まさしくそれであつたにちがひない。こんなわけで碑を背負つてゐる台石の亀も、一度修繕を経てゐる筈であるのに、それを李白などがまだ見ないうちに、もうまた剥落して一面にあをあをと苔蒸してゐたといふのである。そこのところが私にはほんとに面白い。
\\	この堕涙の碑は、つひに有名になつたために、李商隠とか白居易とか、詩人たちの作で、これに触れてゐるものはもとより多い。しかし大中九年に李景遜といふものが、別にまた一基の堕涙の碑を営んで、羊※[#「示+古」、第3水準1-89-26]のために※[#「山+見」、第3水準1-47-77]山に立てたといはれてゐる。が、明の于奕正の編んだ碑目には、もはやその名が見えないところを見ると、もつと早く失はれたのであらう。そしてその碑目には、やはり梁の重修のものだけを挙げてゐるから、こちらはその頃にはまだあつたものと見えるが、今はそれも無くなつた。
\\	羊※[#「示+古」、第3水準1-89-26]は身後の名を気にしてゐたものの、自分のために人が立ててくれた石碑が、三代目さへ亡び果てた今日に至つても、「文選」や「晋書」や「隋書経籍志」のあらむかぎり、いつの世までも、何処かに彼の名を知る人は絶えぬことであらう。彼の魂魄は、もうこれに気づいてゐることであらう。またその友人、杜預が企画した石碑は、二基ともに亡びて、いまにして行くところを知るよしもないが、彼の著述として、やや得意のものであつたらしい「左氏経伝集解」は、今も尚ほ世に行はれて、往々日本の若い学生の手にもそれを見ることがある。だから、大昔から、人間の深い期待にもかかはらず、石は案外脆いもので寿命はかへつて紙墨にも及ばないから、人間はもつと確かなものに憑らなければならぬ、と云ふことが出来やう。杜預の魂魄も、かなり大きな見込み違ひをして、たぶん初めはどぎまぎしたものの、そこを通り越して、今ではもう安心を得てゐるのであらう。
\\	底本:「日本の名随筆88 石」作品社
\\	1990(平成2)年2月25日第1刷発行
\\	1996(平成8)年8月25日第5刷発行
\\	底本の親本:「續 渾霽随筆」中公文庫、中央公論社
\\	1980(昭和55)年1月発行
\\	入力:門田裕志
\\	校正:仙酔ゑびす
\\	2006年11月18日作成
\\	青空文庫作成ファイル:
\\	このファイルは、インターネットの図書館、青空文庫
\\	で作られました。入力、校正、制作にあたったのは、ボランティアの皆さんです。
\end{CJK}
\end{document}