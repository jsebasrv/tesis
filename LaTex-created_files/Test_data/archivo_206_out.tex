\documentclass[8pt]{extreport} 
\usepackage{hyperref}
\usepackage{CJKutf8}
\begin{document}
\begin{CJK}{UTF8}{min}
\\	キュリー夫人
\\	石原純
\\	【テキスト中に現れる記号について】
\\	ルビ
\\	(例)云《い》えば
\\	入力者注 主に外字の説明や、傍点の位置の指定
\\	(例)[#7字下げ]
\\	[#7字下げ]女性と科学[#「女性と科学」は中見出し]
\\	科学は今では女性の方々にとっても必要な大切なものであるということは、もちろん皆さんも知っていられるでしょう。なぜと云《い》えば、ふだんの生活を科学的に合理的に行って、すべての物資を節約することは、つまりは国家に尽す重要な道であることは確かであるからです。
\\	西洋の大学では、どこへ行っても、男子の学生と肩を並べて教授の講義を熱心に聴き入っている女子の学生を見ない処《ところ》は殆《ほとん》どありません。そして大学の研究室のなかにさえも多くの女性が働いています。我が国はいくらかそれとは事情を異《こと》にしていますが、それでも今では学位をもっている女性の方方がかなりに見られるようになりました。
\\	科学の上ですぐれた仕事を成し遂げることは容易ではありません。それでも熱情をこめて励みさえすれば、ある程度には到達することができるのです。しかも科学の歴史を繙《ひもと》いて見ると、女性でありながらすばらしい仕事をした人たちがそこにいくらも現れて来るのです。私はそのなかからただ一、二の例を採り出して見ましょう。その一人は、ロシヤの数学者として名だかいソーニヤ・コヴァレフスカヤです。彼女は一八五〇年に生まれ、ドイツで数学を勉強して、すばらしい研究をなし遂げ、後にスウェーデンのストックホルム大学の教授に任ぜられて、一八九一年に四十一歳で逝去したのでしたが、女流数学者として他に比類を見ないと称せられているばかりでなく、同時に文学者としても著名であって、その自伝は広く愛読されています。もう一人は、ここでお話ししようとするキュリー夫人で、その名は誰も知らないものもないほどですが、更にキュリー夫人の長女であるジョリオ夫人もまた母に劣らぬ科学上の大きな仕事を成し遂げたので、一層有名ともなっているのです。もちろんこのような成功は特別な場合でもあり、また偶々《たまたま》僥倖《ぎょうこう》のある問題にゆき当ったという点もないわけではないでしょうが、しかし熱心に科学の仕事に携わらなければそこには到達できないのでありますし、何《いず》れにしても女性の名を科学の上で高からしめたことは確かであります。
\\	[#7字下げ]故国ポーランド[#「故国ポーランド」は中見出し]
\\	キュリー夫人の故国はポーランドであって、一八六七年の十一月七日にその首都ワルソーで生まれたのでした。その名をマリー・スクロドフスカと称しましたが、父はギムナジウム(中等学校)の教師で、物理学と数学とを教えて居り、母も以前に女学校を立てたことのある人であったというのですから、学問に縁故の深い家柄であったわけです。そしてこれがすでに後にマリーを学問の研究に携わらせる何かの動機となっていたのかも知れません。しかし母はマリーが五歳に達したときに不幸にして亡くなってしまったので、その後は専《もっぱ》ら父の手で育てられました。ところで父はギムナジウムの教師ではあったのですが、その家計は決して豊かではなかったのに、おまけに非常にまじめな人であったので、学校に物理学の実験器械などが十分に備えてなかったのを遺憾に感じ、予算も少なかったので、自費でそれらを買いととのえたものですから、授業の方はいくらかうまく出来たものの、家計はひどく困ったということです。でもそういう父の感化のおかげで、子供たちも自然に学問の尊さをしみじみと知ったのでした。
\\	マリーは女学校に入学して熱心に勉強しました。ところが、その頃のポーランドは悲しいことに、ロシヤ、ドイツ、オーストリーの三箇国に分割されてしまっていたので、ワルソーの町はロシヤ領に属して、殊に極度な圧制を受けていたのでした。学校の書物などもすべてロシヤ語のもので、ポーランドの言葉などはまるで許されなかったのです。マリーはこのようななかで、しかし学問を勉強する準備としてフランス、ドイツ、イギリスの言語を大いに勉強して、それらにも能《よ》く通じたので、十七歳で女学校を卒業してからは、語学の家庭教師を勤めて、不足がちな家計を助けたということです。
\\	さて、どこの国でも圧制が度を過ごすと、それに対する反抗のおこって来るのは当然のことであります。ましてポーランドには祖国愛に強い人たちがたくさんにありました。それでいろいろな形でその運動がおこされましたけれども、ワルソーでは十分にその目的が達せられなかったので、オーストリー領にあるクラカウの町に赴いてこれに携わる人々もたくさんにありました。そしてマリーもやがてまたその仲間に加わりました。ところがマリーは暫くクラカウに居るうちに、自分の心の奥に一つの悩みを覚え始めました。それは祖国のために尽そうとする心に変りはないのですが、自分が幼少の頃から熱心に希望していたのは学問の上で仕事をしたいということであったので、今まるでそれと違ったことをしているのがいかにも心残りに感ぜられたのです。そこでいろいろと考えなやんだ末に、学問の上で成功することも祖国に尽す心の上に変りはないのだと云《い》うことをひたすらに感じ、その後は科学研究に身を任せようと固く決心したのでした。
\\	それにしても女学校を出ただけの学問では進んで何を研究するというわけにもゆきませんから、もっと勉強しなくてはいけないと感じましたが、クラカウでは万事に不便なので、何かよい方法をと考慮していました。ところが、ちょうどその頃姉がフランスのパリで医学を学んでいたので、そこへ赴いて、どんなに苦学してもよいから一生懸命に勉強してみたいと決心を定めました。この決心こそ実にマリーが学問の生活に立ち入る大切な出発点なのでありました。人間には誰にもこういう決心が必要なので、ここに一生の運命がかかっているとも見られるのでしょう。
\\	[#7字下げ]科学研究のよろこび[#「科学研究のよろこび」は中見出し]
\\	この固い決心のもとにマリーは遠くパリに赴き、姉に遇《あ》った上で、東寄りのある町に屋根裏の一室を借り、そこで極度に切りつめた生活を始めました。粗末な食事にも堪え、冬の寒いなかに焚物《たきもの》の乏しいのをも厭《いと》わず[#「厭わず」は底本では「壓はず」]、熱心にソルボンヌの大学へ通って、物理学の講義を聞きました。物質的にはどんなに貧しくても、ここでは故国とはまるで事情を異《こと》にしてすべてが自由で明るく、それに日毎に学問を覚えこんでゆくことのできたのが、何よりも彼女の心を楽しませました。この時の物理学の教授は名だかいリップマンという人で、そのほかに当時世界に並ぶものもないと云《い》われたアンリ・ポアンカレの講義などもあったのでした。
\\	一八九三年に彼女は二十六歳で物理学の学科を卒《お》え、翌年には数学をも卒《お》えましたが、引続いて物理学の教室に通い、リップマン教授の指導を受けながら研究に従事しました。遠く故国を離れて貧しい生活のもとに身なりもかまわず、ひたすらに学問の道に励むこの若い女性の姿こそ、実にこの上もなく尊いものであったに違いありません。しかし「これが祖国を愛する自分の歩むべき唯々一つの道なのだ」と彼女はむしろ限りない喜びをもって、その研究のためにすべてを忘れたのでした。
\\	暫くしてマリーは同じ教室に出入りしていた若い物理学者ピエール・キュリーと知合い、だんだんに親しくなりました。ピエールはすでに物理学の上で相当な仕事をして将来を嘱望され、理化学専門学校の教授となっていたのですが、自分の研究のために大学の教室へも来ていたのです。マリーはこのピエールと親しくなってから、その人物を敬愛する念がだんだんに高まりましたが、それがこの二人の間に限りない愛情を育くみ、遂に一八九五年の七月二十五日に両人の結婚となって、それが実を結んだのでした。それだけにこの家庭は幸福で美しいものでありましたが、それでも生活のためにピエールは講義に多くの時間を費さなくてはなりませんでしたし、また共に研究を続けようとした研究室の設備がいかにも粗末であったというような嘆きは免がれなかったのでした。
\\	ところでこの一八九五年は物理学にとっては大いに記念すべき年でありました。と云《い》うのは、ドイツでヴュルツブルグ大学の教授レンチェンが謂《い》わゆるエックス線を発見して学界を驚かしたからです。それに刺戟《しげき》されていろいろな新しい研究が行われ出しましたが、その翌年にはパリの工科大学の教授ベックレルがウランと云《い》う元素から一種の放射線の出ることを発見し、これが謂《い》わゆる放射能の最初の事実として、これもまた世を驚かしたのでした。
\\	キュリー夫妻はこの発見を知って、ここにいろいろの研究問題のあることを悟り、それを自分達で解決しようと決心しました。最も重要な問題は、ウランの外にも放射能を示す物質があるかどうかと云《い》うことなのでありましたが、それには先《ま》ず原子量がウランに次ぐような物質についてしらべて見なくてはなりませんでした。ウランは実は我々の知っている元素の中で最大の原子量をもつものなのですが、その次はトリウムという元素です。ところがトリウムに就《つ》いて実験してみると、これもウランと同じく放射能を示すことがわかりました。そこで更に進んでウランやトリウムを含むいろいろな鉱石についてなおこの実験をつづけようとしましたが、そこにはなかなか言葉に云《い》いつくせないような苦心が重ねられたのでした。それはなかなか簡単には語りつくせないのですが、その最大のものはウランやトリウムを沢山《たくさん》に含んでいるピッチブレンドと[#「ピッチブレンドと」は底本では「ピッチブレントと」]いう鉱石を多量に手に入れなくてはならないと云《い》うことでした。何かよい方法をと考えなやんでいるうちに、この鉱石を沢山に産出するボヘミヤのヨアヒムスタールという鉱山に、以前にソルボンヌ大学で知合っていたアルファンと云《い》う人が技師として働いていることがわかり、この人に依頼して見ようと云《い》うので、マリー自身がそこへ出かけることに決心しました。女性の身でありながら途中の困難な旅を漸《ようや》くきり抜けて疲れきった体でアルファンを尋ねましたので、彼はその熱心さに感激し、オーストリーの政府に、義侠《ぎきょう》的にその希望を申し入れたので、そのお蔭《かげ》で数トンに上るピッチブレンドがキュリー夫妻のパリの研究室に送られることになったのでした。こうしてともかく材料は都合よくととのえられたものの、次にはそれの分析に多くの困難があったのでした。それについてのこまかい話は省きますが、幾度か失望の淵《ふち》に陥りながら、それでも夫妻で互《たがい》に励まし合い、遂に一八九八年の夏になって最初の成功をかち得たのでした。このときのキュリー夫妻のよろこびはどんなであったか、言葉につくせない程であったにちがいありません。
\\	[#7字下げ]ポロニウム及びラジウムの発見[#「ポロニウム及びラジウムの発見」は中見出し]
\\	キュリー夫妻は、このようにして強い放射能を有する二つの新しい元素を見つけ出しました。最初に発見した元素には、キュリー夫人が祖国ポーランドの名に因んで、ポロニウムという名称を附しました。彼女の絶えずもち続けた祖国愛のための学問上の成果がここに実を結んだものとして、彼女がどれほどこれに満足を感じたかが想察されます。それから数箇月後に発見された第二の元素には、ラジウムという名を与えました。このラジウムはウランに比べて数百万倍も強い放射能をもっている点で、実に驚くべきものであったのです。
\\	この発見によってキュリー夫妻の名声は、学界ばかりでなく一般の社会にまでも俄《にわ》かに広まりました。余りに世間から騒がれることは学者にとっては却って迷惑をも生ずるのですが、学界での厚遇にはキュリー夫妻はいつも心からの感謝を示しました。かくて一九〇〇年にピエールはソルボンヌ大学の教授となり、マリー夫人はセーヴルの女子高等師範学校の教授となり、更に一九〇三年にはベックレルと共にキュリー夫妻に対して学界最高の栄誉とせられているノーベル物理学賞が授与されました。
\\	この後キュリー夫妻はなお熱心に研究を続けて種々の事実を明らかにしましたが、図らずも一九〇六年に悲しむべき不幸の事件が突発しました。丁度《ちょうど》その年の四月十九日のことです。ピエールがある宴会からの帰途に過《あや》まって自動車に[#「自動車に」はママ]轢《ひ》かれて負傷し、その儘《まま》不慮の死を遂げてしまったのでした。その折のマリー夫人の大きな悲痛と落胆とは想像に余りある程でありました。一生のよい伴侶として、またその指導と協力とのもとに学問の研究をさえ共にして来た良人《おっと》を俄《にわ》かに失ったことがどれほど堪え難いことであったかは云《い》う迄《まで》もないことでしょう。これからは自分一人で研究を励まなくてはならないのですし、その上二人の幼い娘をも育ててゆかなくてはならないのでした。これを思いわずらって暫くは途方に暮れましたが、しかし彼女の鞏固《きょうこ》な意力は遂にこの試煉《しれん》にも耐えて、その大きな悲しみをも見事に克服してしまったのでした。
\\	良人《おっと》の死後に彼女はその後を継いでソルボンヌ大学の教授に任ぜられました。またキュリー夫妻のために数年前に特に新設されたラジウム研究所の所長ともなり、そこで若い人たちをよく指導して研究を進めました。その後のすぐれた業績により、一九一一年には再びノーベル物化学賞を授与されましたが、同じ人で二度もノーベル賞を授けられたのは、今日までキュリー夫人以外にはありません。この事からも科学界における女性の栄誉を夫人が一身に荷《にな》っていることがよくわかるでしょう。
\\	一九一四年に起った世界大戦の際には、ラジウム医療によって国家のために活動尽力しました。大戦後に郷里ワルソーに彼女のために研究所が建設されたことなども、彼女のこの上もない喜びの一つでありました。かくてその晩年にはあらゆる栄誉が彼女に与えられましたが、一九三四年の六月に病を得て、南フランスの山地に療養に赴いたまま七月四日に遂に六十七歳でその輝かしい一生を終りました。しかし彼女の遺した科学上の偉大な仕事は永遠に忘れられることなく、その歴史の上に著しい跡をとどめているのですから、またすばらしいではありませんか。
\\	それにしても彼女の晩年において唯々一つの心残りであったのは、嘗《かつ》て困苦を共にして来た最愛の良人《おっと》の不慮の死であったに違いありません。この事はいつも悲しく想い出されたに違いありませんが、彼女はそこに自分の義務として長女イレーヌを同じく立派な科学者に育て上げようと決心したのでした。イレーヌもこれに従って熱心に物理学を修め、ソルボンヌ大学を卒業してから母の研究所で研究を続けているうちに、ジャン・フレデリック・ジョリオという青年と知合い、母の許しを得て結婚しましたが、この二人のその後の研究によって人工放射能という新しい現象が見つけ出されました。これは一九三四年のことでしたが、この現象が物理学の上で同じくすばらしい意味をもっているので、翌年このジョリオ夫妻にまたノーベル賞が授与されました。これで母親とその娘とが共に学界最高の栄誉をになうことになったわけで、これはすでにキュリー夫人の歿後《ぼつご》でありましたが、しかしそのすぐれた教育の効果がここに実を結んだのだと云《い》わなければならないのでしょう。キュリー夫人の偉大さはここにもまた見られるのです。
\\	底本:「偉い科學者」實業之日本社
\\	1942(昭和17)年10月10日発行
\\	※「旧字、旧仮名で書かれた作品を、現代表記にあらためる際の作業指針」に基づいて、底本の表記をあらためました。
\\	「或る」は「ある」に、「之」は「これ」に、「併し」は「しかし」に、「既に」は「すでに」に、「於ける」「於て」は「おける」「おいて」に、「噸」は「トン」に、「エッキス線」は「エックス線」に、「ラヂウム」は「ラジウム」に、置き換えました。
\\	※「方々」と「方方」の混在は、底本通りです。
\\	※読みにくい言葉、読み誤りやすい言葉に振り仮名を付しました。底本には振り仮名が付されていません。
\\	※国立国会図書館デジタルコレクション
\\	で公開されている当該書籍画像に基づいて、作業しました。
\\	入力:高瀬竜一
\\	校正:
\\	2018年10月24日作成
\\	青空文庫作成ファイル:
\\	このファイルは、インターネットの図書館、青空文庫
\\	で作られました。入力、校正、制作にあたったのは、ボランティアの皆さんです。
\end{CJK}
\end{document}