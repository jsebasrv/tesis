\documentclass[8pt]{extreport} 
\usepackage{hyperref}
\usepackage{CJKutf8}
\begin{document}
\begin{CJK}{UTF8}{min}
\\	相性問題	あいせいもんだい	
\\	相手固定接続	あいてこていせつぞく	
\\	相手先固定接続	あいてさきこていせつぞく	
\\	相手先選択接続	あいてさきせんたくせつぞく	
\\	相手選択接続	あいてせんたくせつぞく	
\\	相手選択接続機能	あいてせんたくせつぞくきのう	
\\	曖昧	あいまい	
\\	あいまい検索・曖昧検索	あいまいけんさく	
\\	あいまい制御・曖昧制御	あいまいせいぎょ	
\\	あいまい度・曖昧度	あいまいど	
\\	あいまい量・曖昧量	あいまいりょう	
\\	あいまい理論・曖昧理論	あいまいりろん	
\\	あいまい論理・曖昧論理	あいまいろんり	
\\	青歯	あおば	
\\	空き媒体	あきばいたい	
\\	空き容量	あきようりょう	
\\	空きファイル	あきファイル	
\\	足跡・足あと	あしあと	
\\	脚書き領域	あしがきりょういき	
\\	値	あたい	
\\	値参照名	あたいさんしょうめい	
\\	頭そろえ	あたまそろえ	
\\	当たり判定	あたりはんてい	
\\	圧縮技術	あっしゅくぎじゅつ	
\\	圧縮効果	あっしゅくこうか	
\\	圧縮アルゴリズム	あっしゅくアルゴリズム	
\\	圧縮アーカイバ	あっしゅくアーカイバ	
\\	圧縮モード	あっしゅくモード	
\\	圧伸器	あつしんき	
\\	圧電	あつでん	
\\	あて先変更・宛て先変更・宛先変更	あてさきへんこう	
\\	宛先アドレス	あてさきアドレス	
\\	後入れ先出し記憶装置	あといれさきだしきおくそうち	
\\	後入れ先出しリスト	あといれさきだしリスト	
\\	後書きラベル	あとがきラベル	
\\	溢れ検査・あふれ検査	あふれけんさ	
\\	溢れ表示・あふれ表示	あふれひょうじ	
\\	網掛け・網かけ	あみかけ	
\\	網点	あみてん	
\\	網目状網	あみめじょうもう	
\\	網目状ネットワーク	あみめじょうネットワーク	
\\	誤り回復	あやまりかいふく	
\\	誤り検出	あやまりけんしゅつ	
\\	誤り検出符号	あやまりけんしゅつふごう	
\\	誤り状態	あやまりじょうたい	
\\	誤り制御	あやまりせいぎょ	
\\	誤り制御ソフトウェア	あやまりせいぎょソフトウェア	
\\	誤り通知	あやまりつうち	
\\	誤り訂正	あやまりていせい	
\\	誤り訂正符号	あやまりていせいふごう	
\\	誤り表示	あやまりひょうじ	
\\	誤り率	あやまりりつ	
\\	誤りバースト	あやまりバースト	
\\	暗号化技術	あんごうかぎじゅつ	
\\	暗号化ファイルシステム	あんごうかファイルシステム	
\\	暗号技術	あんごうぎじゅつ	
\\	暗号検査値	あんごうけんさち	
\\	暗号同期	あんごうどうき	
\\	暗号キー	あんごうキー	
\\	暗号システム	あんごうシステム	
\\	暗号メール	あんごうメール	
\\	暗証フレーズ	あんしょうフレーズ	
\\	暗示宣言	あんじせんげん	
\\	暗示番地指定	あんじばんちしてい	
\\	暗示表現	あんじひょうげん	
\\	安全設計要件	あんぜんせっけいようけん	
\\	安全保護	あんぜんほご	
\\	安全保護監査	あんぜんほごかんさ	
\\	安全保護監査証跡	あんぜんほごかんさしょうせき	
\\	安全保護付きアクセス管理	あんぜんほごつきアクセスかんり	
\\	安全保護能力	あんぜんほごのうりょく	
\\	安全保護方針	あんぜんほごほうしん	
\\	安全保護サービス	あんぜんほごサービス	
\\	安全保護ラベル	あんぜんほごラベル	
\\	安全インターロック	あんぜんインターロック	
\\	暗黙的	あんもくてき	
\\	暗黙的引用仕様	あんもくてきいんようしよう	
\\	暗黙範囲符	あんもくはんいふ	
\\	暗黙変換	あんもくへんかん	
\\	暗黙連結	あんもくれんけつ	
\\	暗黙連結処理定義	あんもくれんけつしょりていぎ	
\\	暗黙アドレシング	あんもくアドレシング	
\\	暗黙アドレス指定	あんもくアドレスしてい	
\\	行先指示コード	いきさきしじコード	
\\	異機種環境	いきしゅかんきょう	
\\	移行期	いこうき	
\\	意思決定支援システム・意志決定支援システム	いしけっていしえんシステム	
\\	異種計算機ネットワーク	いしゅけいさんきネットワーク	
\\	移植可能性	いしょくかのうせい	
\\	維持可能スループット	いじかのうスループット	
\\	維持可能セル速度	いじかのうセルそくど	
\\	異常状態	いじょうじょうたい	
\\	異常切断	いじょうせつだん	
\\	位相誤差	いそうごさ	
\\	位相歪み	いそうひずみ	
\\	位相符号化	いそうふごうか	
\\	位相偏移変調	いそうへんいへんちょう	
\\	位相変調記録	いそうへんちょうきろく	
\\	位相変調方式	いそうへんちょうほうしき	
\\	位相コード化	いそうコードか	
\\	位相シフトキーイング	いそうシフトキーイング	
\\	依存適合性	いぞんてきごうせい	
\\	一意な識別子	いちいなしきべつし	
\\	一意名	いちいめい	
\\	一意郵便名	いちいゆうびんめい	
\\	位置決め領域	いちきめりょういき	
\\	位置決め	いちぎめ	
\\	位置決め孔	いちぎめこう	
\\	位置決め時間	いちぎめじかん	
\\	位置決め点	いちぎめてん	
\\	一時誤り	いちじあやまり	
\\	一次局	いちじきょく	
\\	一次群速度インターフェース	いちじぐんそくどインターフェース	
\\	一次資料	いちじしりょう	
\\	一時的ウィンドウ	いちじてきウィンドウ	
\\	一時的エラー	いちじてきエラー	
\\	一時的ユーザ	いちじてきユーザ	
\\	一時的リンクパック領域	いちじてきリンクパックりょういき	
\\	一次文献	いちじぶんけん	
\\	一次キャッシュ	いちじキャッシュ	
\\	一時ファイル	いちじファイル	
\\	1対1の通信	いちたいいちのつうしん	
\\	一太郎	いちたろう	
\\	一段階層分類体系	いちだんかいそうぶんるいたいけい	
\\	位置調整	いちちょうせい	
\\	位置入力装置	いちにゅうりょくそうち	
\\	一方向性関数	いちほうこうせいかんすう	
\\	一様参照	いちようさんしょう	
\\	一様乱数	いちようらんすう	
\\	一覧済み	いちらんずみ	
\\	一覧抽象操作	いちらんちゅうしょうそうさ	
\\	位置カーソル	いちカーソル	
\\	位置パラメタ	いちパラメタ	
\\	一括処理	いっかつしょり	
\\	一括置換	いっかつちかん	
\\	一括適用	いっかつてきよう	
\\	一括発注	いっかつはっちゅう	
\\	一括変換	いっかつへんかん	
\\	一括方式	いっかつほうしき	
\\	1次キャッシュ	いっしキャッシュ	
\\	一斉送信	いっせいそうしん	
\\	一体型モニタ	いったいがたモニタ	
\\	一致演算	いっちえんざん	
\\	一致素子	いっちそし	
\\	一致判定	いっちはんてい	
\\	一致ゲート	いっちゲート	
\\	一定ビットレート	いっていビットレート	
\\	一般化構造体要素	いっぱんかこうぞうたいようそ	
\\	一般化作画基本要素	いっぱんかさくがきほんようそ	
\\	一般区切り子	いっぱんくぎりし	
\\	一般区切り子機能	いっぱんくぎりしきのう	
\\	一般原則	いっぱんげんそく	
\\	一般公開	いっぱんこうかい	
\\	一般色	いっぱんしょく	
\\	一般実体	いっぱんじったい	
\\	一般実体参照	いっぱんじったいさんしょう	
\\	一般自動動作	いっぱんじどうどうさ	
\\	一般属性	いっぱんぞくせい	
\\	一般適合性	いっぱんてきごうせい	
\\	一般配布	いっぱんはいふ	
\\	一般分類体系	いっぱんぶんるいたいけい	
\\	一般保護違反	いっぱんほごいはん	
\\	一般命令演算機構	いっぱんめいれいえんざんきこう	
\\	一般ユーザ	いっぱんユーザ	
\\	一般リリース	いっぱんリリース	
\\	遺伝的アルゴリズム	いでんてきアルゴリズム	
\\	移動局	いどうきょく	
\\	移動受信	いどうじゅしん	
\\	移動端末	いどうたんまつ	
\\	移動不可データセット	いどうふかデータセット	
\\	違法コピー	いほうコピー	
\\	意味関係	いみかんけい	
\\	意味分析	いみぶんせき	
\\	意味ネットワーク	いみネットワーク	
\\	いもづる接続・芋蔓接続・芋づる接続	いもづるせつぞく	
\\	色温度	いろおんど	
\\	色空間	いろくうかん	
\\	色座標系	いろざひょうけい	
\\	色指標	いろしひょう	
\\	色写像	いろしゃぞう	
\\	色選択モード	いろせんたくモード	
\\	色付き折れ線集合	いろつきおれせんしゅうごう	
\\	色定義表	いろていぎひょう	
\\	色定義スプライン	いろていぎスプライン	
\\	色分解	いろぶんかい	
\\	色補正	いろほせい	
\\	色テーブル	いろテーブル	
\\	色モデル	いろモデル	
\\	印刷装置	いんさつそうち	
\\	印刷スケジューラ	いんさつスケジューラ	
\\	印刷プレビュー	いんさつプレビュー	
\\	因子集合	いんししゅうごう	
\\	印字位置	いんじいち	
\\	印字解像度	いんじかいぞうど	
\\	印字項目	いんじこうもく	
\\	印字後行送り	いんじごうこうおくり	
\\	印字式計算器	いんじしきけいさんき	
\\	印字集団	いんじしゅうだん	
\\	印字出力	いんじしゅつりょく	
\\	印字前行送り	いんじぜんぎょうおくり	
\\	印字装置	いんじそうち	
\\	印字抑止機能	いんじよくしきのう	
\\	印字欄	いんじらん	
\\	印字ドラム	いんじドラム	
\\	印字バー	いんじバー	
\\	印字ホイール	いんじホイール	
\\	隠線隠面消去	いんせんいんめんしょうきょ	
\\	隠線消去	いんせんしょうきょ	
\\	陰線処理	いんせんしょり	
\\	隠線処理	いんせんしょり	
\\	陰面消去	いんめんしょうきょ	
\\	隠面処理	いんめんしょり	
\\	引用索引	いんようさくいん	
\\	引用仕様	いんようしよう	
\\	引用仕様宣言	いんようしようせんげん	
\\	引用仕様本体	いんようしようほんたい	
\\	引用文字列	いんようもじれつ	
\\	上付き添え字・上付き添字	うえつきそえじ	
\\	迂回ルーティング	うかいルーティング	
\\	受け入れ機能	うけいれきのう	
\\	受け入れ検査	うけいれけんさ	
\\	受入れ試験	うけいれしけん	
\\	受け側ドライブ	うけがわドライブ	
\\	受渡し検査	うけわたしけんさ	
\\	受渡し試験	うけわたししけん	
\\	動きベクトル	うごきベクトル	
\\	後向き推論	うしろむきすいろん	
\\	薄膜トランジスタ	うすまくトランジスタ	
\\	薄膜トランジスター	うすまくトランジスター	
\\	打ち切り誤差・打切り誤差	うちきりごさ	
\\	打ち込み	うちこみ	
\\	有無	うむ	
\\	埋め草文字	うめくさもじ	
\\	埋め込みプロセッサ	うめこみプロセッサ	
\\	埋め字	うめじ	
\\	裏技	うらわざ	
\\	裏プロセス	うらプロセス	
\\	裏プロセスグループ	うらプロセスグループ	
\\	上書き	うわがき	
\\	上書きモード	うわがきモード	
\\	上付き文字	うわつきもじ	
\\	運転時間	うんてんじかん	
\\	運搬クラス	うんぱんクラス	
\\	運用管理	うんようかんり	
\\	運用規則に基づく安全保護方針	うんようきそくにもとづくあんぜんほごほうしん	
\\	運用形態	うんようけいたい	
\\	英大文字	えいおおもじ	
\\	永久記憶装置	えいきゅうきおくそうち	
\\	影響円すい・影響円錐	えいきょうえんすい	
\\	英語モード	えいごモード	
\\	英字集合	えいじしゅうごう	
\\	英字コード	えいじコード	
\\	英字コード化集合	えいじコードかしゅうごう	
\\	英数	えいすう	
\\	英数字集合	えいすうじしゅうごう	
\\	英数字コード	えいすうじコード	
\\	英数字コード化集合	えいすうじコードかしゅうごう	
\\	英数字データ	えいすうじデータ	
\\	英数文字	えいすうもじ	
\\	衛星系	えいせいけい	
\\	衛星インターネット接続	えいせいインターネットせつぞく	
\\	衛星リンク	えいせいリンク	
\\	映像圧縮	えいぞうあっしゅく	
\\	映像信号	えいぞうしんごう	
\\	映像品質	えいぞうひんしつ	
\\	映像データ	えいぞうデータ	
\\	英文字	えいもじ	
\\	描き直し	えがきなおし	
\\	液晶ディスプレイ	えきしょうディスプレイ	
\\	液晶パネル	えきしょうパネル	
\\	液晶プロジェクタ	えきしょうプロジェクタ	
\\	液タブ	えきタブ	
\\	枝刈り・枝狩り	えだかり	
\\	閲覧	えつらん	
\\	遠隔一括処理	えんかくいっかつしょり	
\\	遠隔一括入力	えんかくいっかつにゅうりょく	
\\	遠隔医療	えんかくいりょう	
\\	遠隔監視	えんかくかんし	
\\	遠隔試験法	えんかくしけんほう	
\\	遠隔資源	えんかくしげん	
\\	遠隔処理	えんかくしょり	
\\	遠隔操作サービス要素	えんかくそうさサービスようそ	
\\	遠隔操作プロトコル機械	えんかくそうさプロトコルきかい	
\\	遠隔測定	えんかくそくてい	
\\	遠隔端末装置	えんかくたんまつそうち	
\\	遠隔通信	えんかくつうしん	
\\	遠隔保守	えんかくほしゅ	
\\	遠隔保守システム	えんかくほしゅシステム	
\\	遠隔アクセス	えんかくアクセス	
\\	遠隔アクセスデータ処理	えんかくアクセスデータしょり	
\\	遠隔サイト	えんかくサイト	
\\	遠隔ジョブ入力	えんかくジョブにゅうりょく	
\\	遠隔バッチ処理	えんかくバッチしょり	
\\	遠隔バッチ入力	えんかくバッチにゅうりょく	
\\	遠隔マシン	えんかくマシン	
\\	永久変数	えんきゅうへんすう	
\\	演算機	えんざんき	
\\	演算器	えんざんき	
\\	演算記号省略	えんざんきごうしょうりゃく	
\\	演算処理ユニット	えんざんしょりユニット	
\\	演算時間	えんざんじかん	
\\	演算数	えんざんすう	
\\	演算制御装置	えんざんせいぎょそうち	
\\	演算装置	えんざんそうち	
\\	演算増幅器	えんざんぞうふくき	
\\	演算対象	えんざんたいしょう	
\\	演算表	えんざんひょう	
\\	演算符号	えんざんふごう	
\\	演算部	えんざんぶ	
\\	演算方式	えんざんほうしき	
\\	演算論理装置	えんざんろんりそうち	
\\	演算コード	えんざんコード	
\\	演算モード	えんざんモード	
\\	円筒プリンター	えんとうプリンター	
\\	欧字集合	おうじしゅうごう	
\\	欧字列	おうじれつ	
\\	欧字コード	おうじコード	
\\	欧字コード化集合	おうじコードかしゅうごう	
\\	欧数字英数字集合	おうすうじえいすうじしゅうごう	
\\	欧数字英数字データ	おうすうじえいすうじデータ	
\\	欧数字集合	おうすうじしゅうごう	
\\	欧数字コード	おうすうじコード	
\\	欧数字コード化集合	おうすうじコードかしゅうごう	
\\	欧数字データ	おうすうじデータ	
\\	応答確認	おうとうかくにん	
\\	応答側	おうとうがわ	
\\	応答側SS利用者	おうとうがわエスエスりようしゃ	
\\	応答側Nアドレス	おうとうがわエンアドレス	
\\	応答時間	おうとうじかん	
\\	応答時間ウインドウ	おうとうじかんウインドウ	
\\	応答メッセージ	おうとうメッセージ	
\\	往復伝搬時間	おうふくでんぱんじかん	
\\	欧文ピッチ処理	おうぶんピッチしょり	
\\	欧文フォント	おうぶんフォント	
\\	応用管理	おうようかんり	
\\	応用管理応用エンティティ	おうようかんりおうようエンティティ	
\\	応用規則	おうようきそく	
\\	応用技術	おうようぎじゅつ	
\\	応用研究	おうようけんきゅう	
\\	応用特有情報	おうようとくゆうじょうほう	
\\	応用アソシエーション	おうようアソシエーション	
\\	応用コンテキスト	おうようコンテキスト	
\\	応用ソフトウェア	おうようソフトウェア	
\\	応用ソフトウェアプログラム	おうようソフトウェアプログラム	
\\	応用データ	おうようデータ	
\\	応用パッケージ	おうようパッケージ	
\\	応用BOS	おうようビーオーエス	
\\	応用プログラム	おうようプログラム	
\\	応用プロセス	おうようプロセス	
\\	大きさ引継ぎ配列	おおきさひきつぎはいれつ	
\\	大文字小文字を区別	おおもじこもじをくべつ	
\\	大文字名前開始文字	おおもじなまえかいしもじ	
\\	大文字名前文字	おおもじなまえもじ	
\\	屋内配線	おくないはいせん	
\\	屋内ケーブル	おくないケーブル	
\\	奥行き表現	おくゆきひょうげん	
\\	奥行き表現定義表	おくゆきひょうげんていぎひょう	
\\	奥行き表現モード	おくゆきひょうげんモード	
\\	送り側ドライブ	おくりがわドライブ	
\\	送り孔	おくりこう	
\\	送り込み	おくりこみ	
\\	送り点	おくりてん	
\\	送りレジスタ	おくりレジスタ	
\\	雄コネクタ	おすコネクタ	
\\	帯状連結三角形	おびじょうれんけつさんかくけい	
\\	重い故障	おもいこしょう	
\\	表	おもて	
\\	表プロセス	おもてプロセス	
\\	表プロセスグループ	おもてプロセスグループ	
\\	表プロセスグループID	おもてプロセスグループアイディー	
\\	親局	おやきょく	
\\	親構造体	おやこうぞうたい	
\\	親子結合	おやこけつごう	
\\	親順序番号	おやじゅんじょばんごう	
\\	親操作	おやそうさ	
\\	親のないプロセスグループ	おやのないプロセスグループ	
\\	親指シフト	おやゆびシフト	
\\	親指シフトキーボード	おやゆびシフトキーボード	
\\	親要素	おやようそ	
\\	親エントリ	おやエントリ	
\\	親ディレクトリ	おやディレクトリ	
\\	親プログラム	おやプログラム	
\\	親プロセス	おやプロセス	
\\	親プロセスID	おやプロセスアイディー	
\\	折り返し試験	おりかえししけん	
\\	折返し試験	おりかえししけん	
\\	折返しテスト	おりかえしテスト	
\\	折り返しポイント	おりかえしポイント	
\\	折りたたみ連続紙	おりたたみれんぞくし	
\\	折れ線	おれせん	
\\	折れ線束表	おれせんたばひょう	
\\	音楽ソフト	おんがくソフト	
\\	音源モジュール	おんげんモジュール	
\\	音声会議システム	おんじょうかいぎシステム	
\\	音声トラック	おんじょうトラック	
\\	音声圧縮	おんせいあっしゅく	
\\	音声応答装置	おんせいおうとうそうち	
\\	音声応答ユニット	おんせいおうとうユニット	
\\	音声合成	おんせいごうせい	
\\	音声合成装置	おんせいごうせいそうち	
\\	音声処理	おんせいしょり	
\\	音声処理技術	おんせいしょりぎじゅつ	
\\	音声転送	おんせいてんそう	
\\	音声認識技術	おんせいにんしきぎじゅつ	
\\	音声符号化	おんせいふごうか	
\\	音声チャネル	おんせいチャネル	
\\	音声ネットワーク	おんせいネットワーク	
\\	音声ファイル	おんせいファイル	
\\	音声メイルボックス	おんせいメイルボックス	
\\	音声メール	おんせいメール	
\\	音量調節	おんりょうちょうせつ	
\\	回帰テスト	かいきテスト	
\\	会議資料	かいぎしりょう	
\\	改行	かいぎょう	
\\	改行文字	かいぎょうもじ	
\\	改行コード	かいぎょうコード	
\\	下位けたあふれ	かいけたあふれ	
\\	下位けたあふれ表示	かいけたあふれひょうじ	
\\	下位語	かいご	
\\	下位参照	かいさんしょう	
\\	改ざん検出	かいざんけんしゅつ	
\\	下位試験手段	かいしけんしゅだん	
\\	開始変数	かいしへんすう	
\\	解釈型言語	かいしゃくかたげんご	
\\	解釈実行	かいしゃくじっこう	
\\	解釈実行プログラム	かいしゃくじっこうプログラム	
\\	開始タグ	かいしタグ	
\\	開始ビット	かいしビット	
\\	階乗機能	かいじょうきのう	
\\	回線切り換え	かいせんきりかえ	
\\	回線業者	かいせんぎょうしゃ	
\\	回線群	かいせんぐん	
\\	回線終端	かいせんしゅうたん	
\\	回線終端装置	かいせんしゅうたんそうち	
\\	回線種別	かいせんすべつ	
\\	回線切断	かいせんせつだん	
\\	回線争奪	かいせんそうだつ	
\\	回線速度	かいせんそくど	
\\	回線帯域	かいせんたいいき	
\\	回線追跡	かいせんついせき	
\\	回線費用	かいせんひよう	
\\	回線網	かいせんもう	
\\	回線容量	かいせんようりょう	
\\	回線利用効率	かいせんりようこうりつ	
\\	回線インタフェース	かいせんインタフェース	
\\	回線インタフェースモジュール	かいせんインタフェースモジュール	
\\	回線リセール	かいせんリセール	
\\	回線レート	かいせんレート	
\\	下位層	かいそう	
\\	階層化	かいそうか	
\\	階層型ディレクトリ	かいそうかたディレクトリ	
\\	階層型メニュー	かいそうかたメニュー	
\\	階層関係	かいそうかんけい	
\\	階層化ディレクトリ	かいそうかディレクトリ	
\\	階層化メニュー	かいそうかメニュー	
\\	階層形計算機ネットワーク	かいそうがたけいさんきネットワーク	
\\	階層型ファイルシステム	かいそうがたファイルシステム	
\\	階層型ルーティング	かいそうがたルーティング	
\\	階層記憶制御	かいそうきおくせいぎょ	
\\	階層木構造	かいそうきこうぞう	
\\	階層構造	かいそうこうぞう	
\\	回送先住所要求	かいそうさきじゅうしょようきゅう	
\\	階層制約集合	かいそうせいやくしゅうごう	
\\	階層表記法	かいそうひょうきほう	
\\	階層表示	かいそうひょうじ	
\\	階層分類体系	かいそうぶんるいたいけい	
\\	回送要求	かいそうようきゅう	
\\	回送IPメッセージ表示	かいそうアイピーメッセージひょうじ	
\\	階層アーキテクチャ	かいそうアーキテクチャ	
\\	階層ディレクトリ	かいそうディレクトリ	
\\	階層ファイルモデル	かいそうファイルモデル	
\\	階層プロトコル	かいそうプロトコル	
\\	階層メニュー	かいそうメニュー	
\\	改造	かいぞう	
\\	海賊版ソフトウェア	かいぞくはんソフトウェア	
\\	改訂履歴	かいていりれき	
\\	回転位置感知	かいてんいちかんち	
\\	回転位置検出	かいてんいちけんしゅつ	
\\	回転位置検出機構	かいてんいちけんしゅつきこう	
\\	回転待ち時間	かいてんまちじかん	
\\	解凍	かいとう	
\\	解凍ソフト	かいとうソフト	
\\	解読キー	かいどくキー	
\\	開発環境	かいはつかんきょう	
\\	開発基本線	かいはつきほんせん	
\\	開発言語	かいはつげんご	
\\	開発後評価	かいはつごひょうか	
\\	開発中	かいはつちゅう	
\\	開発費用	かいはつひよう	
\\	開発元	かいはつもと	
\\	開発キット	かいはつキット	
\\	開発グループ	かいはつグループ	
\\	開発コスト	かいはつコスト	
\\	開発ツール	かいはつツール	
\\	開発プロセス	かいはつプロセス	
\\	改版管理	かいはんかんり	
\\	改版理由	かいはんりゆう	
\\	回避番号	かいひばんごう	
\\	回避文字	かいひもじ	
\\	回復管理機能	かいふくかんりきのう	
\\	回復機能	かいふくきのう	
\\	回復時間	かいふくじかん	
\\	回復不能誤り	かいふくふのうあやまり	
\\	回復用情報	かいふくようじょうほう	
\\	回復ログ	かいふくログ	
\\	下位部分木	かいぶぶんもく	
\\	解放	かいほう	
\\	開放型文書体系	かいほうがたぶんしょたいけい	
\\	開放型システム	かいほうがたシステム	
\\	開放型システム間相互接続	かいほうがたシステムかんそうごせつぞく	
\\	開放型システム間相互接続環境	かいほうがたシステムかんそうごせつぞくかんきょう	
\\	解放要求	かいほうようきゅう	
\\	回路機構	かいろきこう	
\\	回路構成要素	かいろこうせいようそ	
\\	回路図記号	かいろずきごう	
\\	回路容量	かいろようりょう	
\\	会話形	かいわがた	
\\	会話型言語	かいわけいげんご	
\\	会話型処理	かいわけいしょり	
\\	会話モード	かいわモード	
\\	下位グループ	かいグループ	
\\	下位ダイアログ	かいダイアログ	
\\	下位バイト	かいバイト	
\\	改ページ・改頁	かいページ	
\\	開ループ	かいループ	
\\	科学技術計算	かがくぎじゅつけいさん	
\\	科学的記数法	かがくてききすうほう	
\\	書き換え可能	かきかえかのう	
\\	書き込み可能	かきこみかのう	
\\	書き込み許可	かきこみきょか	
\\	書込み許可リング	かきこみきょかリング	
\\	書き込み禁止	かきこみきんし	
\\	書き込み保護スイッチ	かきこみほごスイッチ	
\\	書込み保護ラベル	かきこみほごラベル	
\\	書込みサイクル時間	かきこみサイクルじかん	
\\	書込みヘッド	かきこみヘッド	
\\	課金機能	かきんきのう	
\\	課金情報	かきんじょうほう	
\\	課金制	かきんせい	
\\	課金制御データ	かきんせいぎょデータ	
\\	課金動作有効期間	かきんどうさゆうこうきかん	
\\	課金データ	かきんデータ	
\\	課金レコード	かきんレコード	
\\	鍵管理	かぎかんり	
\\	鍵の配布	かぎのはいふ	
\\	可逆圧縮	かぎゃくあっしゅく	
\\	可逆計数器	かぎゃくけいすうき	
\\	可逆カウンタ	かぎゃくカウンタ	
\\	核具象構文	かくぐしょうこうぶん	
\\	各国語キーボード	かくこくごキーボード	
\\	拡散反射	かくさんはんしゃ	
\\	拡散反射係数	かくさんはんしゃけいすう	
\\	拡散率	かくさんりつ	
\\	拡散スペクトル	かくさんスペクトル	
\\	隠し機能	かくしきのう	
\\	隠し属性	かくしぞくせい	
\\	隠しコマンド	かくしコマンド	
\\	隠しファイル	かくしファイル	
\\	確実性係数	かくじつせいけいすう	
\\	拡大行列	かくだいぎょうれつ	
\\	拡大縮小	かくだいしゅくしょう	
\\	拡大保証	かくだいほしょう	
\\	拡張記憶装置	かくちょうきおくそうち	
\\	拡張記号	かくちょうきごう	
\\	拡張機能	かくちょうきのう	
\\	拡張業界標準アーキテクチャ	かくちょうぎょうかいひょうじゅんアーキテクチャ	
\\	拡張小型デバイスインタフェース	かくちょうこがたデバイスインタフェース	
\\	拡張子	かくちょうし	
\\	拡張指定表	かくちょうしていひょう	
\\	拡張性	かくちょうせい	
\\	拡張精度浮動小数点数	かくちょうせいどふどうしょうすうてんすう	
\\	拡張日本語キーボード	かくちょうにほんごキーボード	
\\	拡張版SpeedStepテクノロジ	かくちょうはんスピードステップテクノロジ	
\\	拡張物理的配達アドレス構成要素	かくちょうぶつりてきはいたつアドレスこうせいようそ	
\\	拡張部分	かくちょうぶぶん	
\\	拡張命令コード	かくちょうめいれいコード	
\\	拡張郵便ORアドレス構成要素	かくちょうゆうびんオーアーアドレスこうせいようそ	
\\	拡張ISA	かくちょうアイエスエー	
\\	拡張スロット	かくちょうスロット	
\\	拡張テキスト	かくちょうテキスト	
\\	拡張バス	かくちょうバス	
\\	拡張ボード	かくちょうボード	
\\	拡張メモリー	かくちょうメモリー	
\\	拡張モード	かくちょうモード	
\\	拡張UEF	かくちょうユーイーエフ	
\\	確定可能	かくていかのう	
\\	確定的過程	かくていてきかてい	
\\	確認試験	かくにんしけん	
\\	格納	かくのう	
\\	格納密度	かくのうみつど	
\\	格納領域	かくのうりょういき	
\\	格納メッセージ	かくのうメッセージ	
\\	格納メッセージ一覧	かくのうメッセージいちらん	
\\	格納メッセージ該当数	かくのうメッセージがいとうすう	
\\	格納メッセージ警報	かくのうメッセージけいほう	
\\	格納メッセージ削除	かくのうメッセージさくじょ	
\\	格納メッセージ自動回送	かくのうメッセージじどうかいそう	
\\	格納メッセージ取出し	かくのうメッセージとりだし	
\\	確率密度	かくりつみつど	
\\	確率理論	かくりつりろん	
\\	隠れ線	かくれせん	
\\	隠れキャラ	かくれキャラ	
\\	隠れマルコフモデル	かくれマルコフモデル	
\\	陰付け処理	かげつけしょり	
\\	加減演算子	かげんえんざんし	
\\	加減作用素	かげんさようそ	
\\	加減算器	かげんさんき	
\\	加減算時間	かげんさんじかん	
\\	加工データ	かこうデータ	
\\	重ね印刷	かさねいんさつ	
\\	重ね打ち	かさねうち	
\\	加算機式計算器	かさんきしきけいさんき	
\\	加算積分器	かさんせきぶんき	
\\	可視化処理	かしかしょり	
\\	可視化順序	かしかじゅんじょ	
\\	頭文字語	かしらもじご	
\\	過剰規制	かじょうきせい	
\\	仮説の検定	かせつのけんてい	
\\	仮想押しボタン	かそうおしボタン	
\\	仮想化	かそうか	
\\	仮想回線	かそうかいせん	
\\	仮想回路	かそうかいろ	
\\	仮想記憶	かそうきおく	
\\	仮想記憶装置	かそうきおくそうち	
\\	仮想記憶システム	かそうきおくシステム	
\\	仮想機械	かそうきかい	
\\	仮想企業	かそうきぎょう	
\\	仮想局	かそうきょく	
\\	仮想空間	かそうくうかん	
\\	仮想計算機	かそうけいさんき	
\\	仮想現実	かそうげんじつ	
\\	仮想現実感	かそうげんじつかん	
\\	仮想工場	かそうこうじょう	
\\	仮想私設通信網	かそうしせつつうしんもう	
\\	仮想私設網	かそうしせつもう	
\\	仮想接続	かそうせつぞく	
\\	仮想専用網	かそうせんようもう	
\\	仮想装置	かそうそうち	
\\	仮想装置座標	かそうそうちざひょう	
\\	仮想装置座標領域	かそうそうちざひょうりょういき	
\\	仮想的	かそうてき	
\\	仮想的交換局	かそうてきこうかんきょく	
\\	仮想86モード	かそうはちろくモード	
\\	仮想8086モード	かそうはちゼロはちろくモード	
\\	仮想閉域網	かそうへいいきもう	
\\	仮想文字集合	かそうもじしゅうごう	
\\	仮想アドレス	かそうアドレス	
\\	仮想アドレス空間	かそうアドレスくうかん	
\\	仮想アドレス領域	かそうアドレスりょういき	
\\	仮想コネクション	かそうコネクション	
\\	仮想コネクション識別子	かそうコネクションしきべつし	
\\	仮想サーバー・仮想サーバ	かそうサーバー・かそうサーバ	
\\	仮想ステム	かそうステム	
\\	仮想チャネル	かそうチャネル	
\\	仮想チャネル識別子	かそうチャネルしきべつし	
\\	仮想デバイスドライバ	かそうデバイスドライバ	
\\	仮想ドライブ	かそうドライブ	
\\	仮想パス	かそうパス	
\\	仮想パス識別子	かそうパスしきべつし	
\\	仮想ファイル	かそうファイル	
\\	仮想ファイルストア	かそうファイルストア	
\\	仮想マシン	かそうマシン	
\\	仮想メモリ・仮想メモリー	かそうメモリ・かそうメモリー	
\\	加速時間	かそくじかん	
\\	過疎スレ	かそスレ	
\\	片側制御モード	かたがわせいぎょモード	
\\	型参照名	かたさんしょうめい	
\\	型宣言文	かたせんげんぶん	
\\	型定義	かたていぎ	
\\	型変換演算子	かたへんかんえんざんし	
\\	片方向相互動作	かたほうこうそうごどうさ	
\\	片方向通信	かたほうこうつうしん	
\\	片面ディスク	かためんディスク	
\\	肩文字	かたもじ	
\\	偏り誤差	かたよりごさ	
\\	型パラメタ	かたパラメタ	
\\	型パラメタ値	かたパラメタち	
\\	可聴周波数	かちょうしゅうはすう	
\\	活性文書型	かっせいぶんしょがた	
\\	活性連結型	かっせいれんけつがた	
\\	活字ドラム	かつじドラム	
\\	活字バー	かつじバー	
\\	活字ホイール	かつじホイール	
\\	活用率	かつようりつ	
\\	稼働中	かどうちゅう	
\\	仮名漢字変換・かな漢字変換	かなかんじへんかん	
\\	仮名漢字変換形日本文入力装置	かなかんじへんかんがたにほんぶんにゅうりょくそうち	
\\	かな入力・仮名入力	かなにゅうりょく	
\\	加入者回線	かにゅうしゃかいせん	
\\	加入者線交換機	かにゅうしゃせんこうかんき	
\\	加入者宅内装置	かにゅうしゃたくないそうち	
\\	加入者番号	かにゅうしゃばんごう	
\\	加入者アドレス	かにゅうしゃアドレス	
\\	加入者データ	かにゅうしゃデータ	
\\	可搬型	かはんがた	
\\	過負荷	かふか	
\\	株式会社日本レジストリサービス	かぶしきがいしゃにほんレジストリサービス	
\\	可変関数発生器	かへんかんすうはっせいき	
\\	可変小数点表示法	かへんしょうすうてんひょうじほう	
\\	可変字送り	かへんじおくり	
\\	可変長レコード	かへんちょうレコード	
\\	可変反復データ項目	かへんはんぷくデータこうもく	
\\	可変部	かへんぶ	
\\	可変ビットレート	かへんビットレート	
\\	壁紙	かべがみ	
\\	加法性白色ガウス雑音	かほうせいはくしょくガウスざつおん	
\\	紙送り	かみおくり	
\\	紙テープ	かみテープ	
\\	可用時間	かようじかん	
\\	可用率	かようりつ	
\\	空の文字列	からのもじれつ	
\\	空のディスケット	からのディスケット	
\\	空のディレクトリ	からのディレクトリ	
\\	仮原文	かりげんぶん	
\\	仮原文区切り記号	かりげんぶんくぎりきごう	
\\	借り数	かりすう	
\\	仮手続	かりてつづき	
\\	仮配列	かりはいれつ	
\\	仮引数	かりひきすう	
\\	仮パラメタ	かりパラメタ	
\\	仮ポインタ	かりポインタ	
\\	軽い故障	かるいこしょう	
\\	簡易言語	かんいげんご	
\\	簡易端末	かんいたんまつ	
\\	簡易認証	かんいにんしょう	
\\	間隔計時機構	かんかくけいじきこう	
\\	間隔比率	かんかくひりつ	
\\	間隔文字	かんかくもじ	
\\	管轄ポート	かんかつポート	
\\	環境記述	かんきょうきじゅつ	
\\	環境句	かんきょうく	
\\	環境光	かんきょうこう	
\\	環境光源	かんきょうこうげん	
\\	環境光反射係数	かんきょうこうはんしゃけいすう	
\\	環境条件	かんきょうじょうけん	
\\	環境設定	かんきょうせってい	
\\	環境変数	かんきょうへんすう	
\\	環境モニタリング	かんきょうモニタリング	
\\	関係演算子	かんけいえんざんし	
\\	関係管理	かんけいかんり	
\\	関係指示記号	かんけいしじきごう	
\\	関係データベース	かんけいデータベース	
\\	感光装置	かんこうそうち	
\\	感光素子	かんこうそし	
\\	監査検討ファイル	かんさけんとうファイル	
\\	監視局	かんしきょく	
\\	緩衝記憶	かんしょうきおく	
\\	干渉信号	かんしょうしんごう	
\\	干渉電力	かんしょうでんりょく	
\\	関心領域	かんしんりょういき	
\\	監視プログラム	かんしプログラム	
\\	漢字対応	かんじたいおう	
\\	漢字直接入力	かんじちょくせつにゅうりょく	
\\	漢字統合	かんじとうごう	
\\	漢字符号化方式	かんじふごうかほうしき	
\\	環状計数器	かんじょうけいすうき	
\\	環状網	かんじょうもう	
\\	環状ネットワーク	かんじょうネットワーク	
\\	漢字コード	かんじコード	
\\	漢字コード化集合	かんじコードかしゅうごう	
\\	漢字Talk	かんじトーク	
\\	漢字URL	かんじユーアールエル	
\\	漢字ROM	かんじロム	
\\	関数・函数	かんすう	
\\	関数型言語	かんすうがたげんご	
\\	関数形言語	かんすうがたげんご	
\\	関数型プログラミング	かんすうがたプログラミング	
\\	関数形プログラミング	かんすうがたプログラミング	
\\	関数結果	かんすうけっか	
\\	関数手続き	かんすうてつづき	
\\	関数発生器	かんすうはっせいき	
\\	関数副プログラム	かんすうふくプログラム	
\\	関数名	かんすうめい	
\\	関数呼び出し	かんすうよびだし	
\\	関数ライブラリ	かんすうライブラリ	
\\	間接色指定	かんせついろしてい	
\\	間接参照	かんせつさんしょう	
\\	間接発信	かんせつはっしん	
\\	間接発進ポート	かんせつはっしんポート	
\\	間接利用者	かんせつりようしゃ	
\\	間接アドレス	かんせつアドレス	
\\	幹線結合器	かんせんけつごうき	
\\	幹線ケーブル	かんせんケーブル	
\\	完全修飾ドメイン名	かんぜんしゅうしょくドメインめい	
\\	完全実装	かんぜんじっそう	
\\	完全冗長	かんぜんじょうちょう	
\\	完全冗長システム	かんぜんじょうちょうシステム	
\\	完全性	かんぜんせい	
\\	完全接続網	かんぜんせつぞくもう	
\\	完全接続ネットワーク	かんぜんせつぞくネットワーク	
\\	完全対応	かんぜんたいおう	
\\	完全なパス名	かんぜんなパスめい	
\\	完全版	かんぜんはん	
\\	完全表記	かんぜんひょうき	
\\	観測値	かんそくち	
\\	観測データ	かんそくデータ	
\\	漢直	かんちょく	
\\	観点表示	かんてんひょうじ	
\\	監督機能	かんとくきのう	
\\	感熱印字装置	かんねついんじそうち	
\\	感熱式プリンタ	かんねつしきプリンタ	
\\	感熱プリンタ	かんねつプリンタ	
\\	完備した表	かんびしたひょう	
\\	簡約化命令集合計算機	かんやくかめいれいしゅうごうけいさんき	
\\	慣用暗号方式	かんようあんごうほうしき	
\\	慣用暗号システム	かんようあんごうシステム	
\\	管理開放型システム	かんりかいほうがたシステム	
\\	管理環境	かんりかんきょう	
\\	管理機関	かんりきかん	
\\	管理業務	かんりぎょうむ	
\\	管理作業	かんりさぎょう	
\\	管理者	かんりしゃ	
\\	管理情報ベース	かんりじょうほうベース	
\\	管理対象	かんりたいしょう	
\\	管理体制	かんりたいせい	
\\	管理提供オブジェクト	かんりていきょうオブジェクト	
\\	簡略記述ORアドレス	かんりゃくきじゅつオーアーアドレス	
\\	簡略版	かんりゃくばん	
\\	簡略分類体系	かんりゃくぶんるいたいけい	
\\	完了コード	かんりょうコード	
\\	管理領域	かんりりょういき	
\\	管理領域名	かんりりょういきめい	
\\	管理オブジェクトクラス	かんりオブジェクトクラス	
\\	管理システム	かんりシステム	
\\	管理ツール	かんりツール	
\\	関連機器	かんれんきき	
\\	関連技術	かんれんぎじゅつ	
\\	関連情報	かんれんじょうほう	
\\	関連情報速度	かんれんじょうほうそくど	
\\	関連図	かんれんず	
\\	関連製品	かんれんせいひん	
\\	関連単語	かんれんたんご	
\\	関連付け	かんれんづけ	
\\	外因損失時間	がいいんそんしつじかん	
\\	外観と感触	がいかんとかんしょく	
\\	外形寸法	がいけいすんぽう	
\\	外字番号	がいじばんごう	
\\	外字符号	がいじふごう	
\\	外字ファイル	がいじファイル	
\\	概念記号	がいねんきごう	
\\	概念文書	がいねんぶんしょ	
\\	概念スキーマ	がいねんスキーマ	
\\	外部管理	がいぶかんり	
\\	外部型	がいぶがた	
\\	外部記憶	がいぶきおく	
\\	外部記憶装置	がいぶきおくそうち	
\\	外部結合	がいぶけつごう	
\\	外部識別子	がいぶしきべつし	
\\	外部試験法	がいぶしけんほう	
\\	外部実体	がいぶじったい	
\\	外部宣言	がいぶせんげん	
\\	外部装置	がいぶそうち	
\\	外部手続	がいぶてつづき	
\\	外部副プログラム	がいぶふくプログラム	
\\	外部文書クラス	がいぶぶんしょクラス	
\\	外部変数	がいぶへんすう	
\\	外部要素	がいぶようそ	
\\	外部割り込み	がいぶわりこみ	
\\	外部アクセス	がいぶアクセス	
\\	外部アクセス可能閉域利用者グループ	がいぶアクセスかのうへいいきりようしゃグループ	
\\	外部キャッシュ	がいぶキャッシュ	
\\	外部キー	がいぶキー	
\\	外部クロック	がいぶクロック	
\\	外部クロック同期	がいぶクロックどうき	
\\	外部ゲートウエイプロトコル	がいぶゲートウエイプロトコル	
\\	外部コマンド	がいぶコマンド	
\\	外部コンポーネント	がいぶコンポーネント	
\\	外部スイッチ	がいぶスイッチ	
\\	外部データ	がいぶデータ	
\\	外部データ項目	がいぶデータこうもく	
\\	外部データレコード	がいぶデータレコード	
\\	外部ファイル	がいぶファイル	
\\	外部ファイル結合子	がいぶファイルけつごうし	
\\	外部ファイルサービス	がいぶファイルサービス	
\\	外部プログラム入力式計算器	がいぶプログラムにゅうりょくしきけいさんき	
\\	外部メモリ	がいぶメモリ	
\\	外部ラベル	がいぶラベル	
\\	学習機械	がくしゅうきかい	
\\	学習機能	がくしゅうきのう	
\\	学習プログラム	がくしゅうプログラム	
\\	画素間隔	がそかんかく	
\\	画素進行方向	がそしんこうほうこう	
\\	画素配列	がそはいれつ	
\\	画像圧縮	がぞうあっしゅく	
\\	画像圧縮技術	がぞうあっしゅくぎじゅつ	
\\	画像資料	がぞうしりょう	
\\	画像情報	がぞうじょうほう	
\\	画像要素	がぞうようそ	
\\	画像劣化	がぞうれっか	
\\	画像データ	がぞうデータ	
\\	画像パターン認識	がぞうパターンにんしき	
\\	画像ファイル	がぞうファイル	
\\	画面共有	がめんきょうゆう	
\\	画面上	がめんじょう	
\\	画面静止	がめんせいし	
\\	画面取り込み	がめんとりこみ	
\\	画面分割	がめんぶんかつ	
\\	画面サイズ	がめんサイズ	
\\	柄様式	がらようしき	
\\	含意素子	がんいそし	
\\	記憶管理	きおくかんり	
\\	記憶機構	きおくきこう	
\\	記憶装置	きおくそうち	
\\	記憶装置の保護	きおくそうちのほご	
\\	記憶装置の保護キー	きおくそうちのほごキー	
\\	記憶装置表示	きおくそうちひょうじ	
\\	記憶素子	きおくそし	
\\	記憶単位	きおくたんい	
\\	記憶媒体	きおくばいたい	
\\	記憶場所	きおくばしょ	
\\	記憶番地	きおくばんち	
\\	記憶保護	きおくほご	
\\	記憶容量	きおくようりょう	
\\	記憶列	きおくれつ	
\\	記憶列結合	きおくれつけつごう	
\\	記憶イメージ	きおくイメージ	
\\	記憶セル	きおくセル	
\\	機械化	きかいか	
\\	機械学習	きかいがくしゅう	
\\	機械機能	きかいきのう	
\\	機会検討	きかいけんとう	
\\	機械語	きかいご	
\\	機械語命令	きかいごめいれい	
\\	機械最小値	きかいさいしょうち	
\\	機械設計	きかいせっけい	
\\	機械の語	きかいのご	
\\	機械翻訳	きかいほんやく	
\\	機械命令	きかいめいれい	
\\	機械コード	きかいコード	
\\	幾何学図形要素	きかがくずけいようそ	
\\	規格参照区切り子集合	きかくさんしょうくぎりししゅうごう	
\\	規格参照具象構文	きかくさんしょうぐしょうこうぶん	
\\	規格参照容量集合	きかくさんしょうようりょうしゅうごう	
\\	規格参照予約名	きかくさんしょうよやくめい	
\\	規格参照量集合	きかくさんしょうりょうしゅうごう	
\\	規格制定	きかくせいてい	
\\	基幹業務データ	きかんぎょうむデータ	
\\	機関名	きかんめい	
\\	機関名の階層性	きかんめいのかいそうせい	
\\	基幹システム	きかんシステム	
\\	危機的	ききてき	
\\	企業情報ポータル	きぎょうじょうほうポータル	
\\	企業内情報ポータル	きぎょうないじょうほうポータル	
\\	企業ネットワーク	きぎょうネットワーク	
\\	奇偶検査	きぐうけんさ	
\\	奇偶検査ビット	きぐうけんさビット	
\\	奇遇性	きぐうせい	
\\	危険域	きけんいき	
\\	危険地域	きけんちいき	
\\	危険領域	きけんりょういき	
\\	期限切れ検査	きげんぎれけんさ	
\\	木構造	きこうぞう	
\\	記号型言語	きごうかたげんご	
\\	記号文字	きごうもじ	
\\	記号列	きごうれつ	
\\	記号論理学	きごうろんりがく	
\\	記号アドレス	きごうアドレス	
\\	刻み幅	きざみはば	
\\	記者発表	きしゃはっぴょう	
\\	機種依存文字	きしゅいそんもじ・きしゅいぞんもじ	
\\	機種依存	きしゅいぞん	
\\	記述項	きじゅつこう	
\\	記述項の右辺	きじゅつこうのうへん	
\\	記述項の左辺	きじゅつこうのさへん	
\\	記述子要素	きじゅつしようそ	
\\	記述的マーク	きじゅつてきマーク	
\\	記述部	きじゅつぶ	
\\	記述名	きじゅつめい	
\\	基準縁	きじゅんえん	
\\	基準形式	きじゅんけいしき	
\\	基準試験	きじゅんしけん	
\\	基準時点	きじゅんじてん	
\\	基準値・規準値	きじゅんち	
\\	基準復帰記録	きじゅんふっききろく	
\\	基準領域	きじゅんりょういき	
\\	基準クロック	きじゅんクロック	
\\	机上検査	きじょうけんさ	
\\	木状網	きじょうもう	
\\	机上チェック	きじょうチェック	
\\	机上デバッグ	きじょうデバッグ	
\\	木状ネットワーク	きじょうネットワーク	
\\	基数記数法	きすうきすうほう	
\\	基数表記法	きすうひょうきほう	
\\	奇数パリティ	きすうパリティ	
\\	規制機能	きせいきのう	
\\	既存装置	きそんそうち	
\\	寄託図書館	きたくとしょかん	
\\	木探索	きたんさく	
\\	帰着	きちゃく	
\\	基調演説	きちょうえんぜつ	
\\	基底解	きていかい	
\\	基底アドレス	きていアドレス	
\\	基底アドレスレジスタ	きていアドレスレジスタ	
\\	基底クラス	きていクラス	
\\	基底レジスタ	きていレジスタ	
\\	起点シソーラス	きてんシソーラス	
\\	起点デスクリプタ	きてんデスクリプタ	
\\	起動画面	きどうがめん	
\\	起動側	きどうがわ	
\\	起動側応用エンティティ	きどうがわおうようエンティティ	
\\	起動側CMISEサービス利用者	きどうがわシーエムアイエスイーサービスりようしゃ	
\\	起動項目	きどうこうもく	
\\	起動者	きどうしゃ	
\\	起動ディスク	きどうディスク	
\\	輝度信号	きどしんごう	
\\	機能概要	きのうがいよう	
\\	機能強化	きのうきょうか	
\\	機能試験	きのうしけん	
\\	機能試験マクロ	きのうしけんマクロ	
\\	機能指示記号	きのうしじきごう	
\\	機能縮退	きのうしゅくたい	
\\	機能仕様	きのうしよう	
\\	機能診断テスト	きのうしんだんテスト	
\\	機能設計	きのうせっけい	
\\	機能単位	きのうたんい	
\\	機能停止	きのうていし	
\\	帰納的関数	きのうてきかんすう	
\\	機能分析	きのうぶんせき	
\\	機能文字	きのうもじ	
\\	機能文字識別引数	きのうもじしきべつひきすう	
\\	揮発性記憶装置	きはつせいきおくそうち	
\\	揮発性メモリー	きはつせいメモリー	
\\	基板	きばん	
\\	基盤設備	きばんせつび	
\\	記法識別子	きほうしきべつし	
\\	記法宣言	きほうせんげん	
\\	記法属性	きほうぞくせい	
\\	記法名	きほうめい	
\\	基本外	きほんがい	
\\	基本型符号化	きほんがたふごうか	
\\	基本形リンク制御	きほんがたリンクせいぎょ	
\\	基本記号	きほんきごう	
\\	基本機能	きほんきのう	
\\	基本技術	きほんぎじゅつ	
\\	基本形式	きほんけいしき	
\\	基本構成部品	きほんこうせいぶひん	
\\	基本項目	きほんこうもく	
\\	基本状態	きほんじょうたい	
\\	基本制御プログラム	きほんせいぎょプログラム	
\\	基本相互接続試験	きほんそうごせつぞくしけん	
\\	基本測定単位	きほんそくていたんい	
\\	基本多言語面	きほんたげんごめん	
\\	基本値	きほんち	
\\	基本電話サービス	きほんでんわサービス	
\\	基本入出力システム	きほんにゅうしゅつりょくシステム	
\\	基本物理表現	きほんぶつりひょうげん	
\\	基本文	きほんぶん	
\\	基本文書型	きほんぶんしょがた	
\\	基本文書要素	きほんぶんしょようそ	
\\	基本変数	きほんへんすう	
\\	基本名	きほんめい	
\\	基本目標	きほんもくひょう	
\\	基本要素属性	きほんようそぞくせい	
\\	基本要素の表示様相	きほんようそのひょうじようそう	
\\	基本論理対象体	きほんろんりたいしょうたい	
\\	基本割付け対象体	きほんわりつけたいしょうたい	
\\	基本アクセス法	きほんアクセスほう	
\\	基本アクセスインターフェース	きほんアクセスインターフェース	
\\	基本インターフェース	きほんインターフェース	
\\	基本ウィンドウ	きほんウィンドウ	
\\	基本SGML文書	きほんエスジーエムエルぶんしょ	
\\	基本サービス	きほんサービス	
\\	基本ソフト	きほんソフト	
\\	基本ソフトウェア	きほんソフトウェア	
\\	機密保護	きみつほご	
\\	客先開発プログラム	きゃくさきかいはつプログラム	
\\	休止状態	きゅうしじょうたい	
\\	給送	きゅうそう	
\\	急速充電	きゅうそくじゅうでん	
\\	旧JIS漢字コード	きゅうジェーアイエスかんじコード・きゅうジスかんじコード	
\\	旧バージョン	きゅうバージョン	
\\	鏡映	きょうえい	
\\	共役直径対	きょうえきちょっけいたい	
\\	境界合わせ	きょうかいあわせ	
\\	境界内オブジェクト集合	きょうかいないオブジェクトしゅうごう	
\\	境界表現	きょうかいひょうげん	
\\	強化学習	きょうかがくしゅう	
\\	供給停止	きょうきゅうていし	
\\	競合整理処理	きょうごうせいりしょり	
\\	競合アクセス業者	きょうごうアクセスぎょうしゃ	
\\	教材センタ	きょうざいセンタ	
\\	教材プログラム	きょうざいプログラム	
\\	強制改行	きょうせいかいぎょう	
\\	強制終了	きょうせいしゅうりょう	
\\	強制衝突	きょうせいしょうとつ	
\\	強制ページ区切り	きょうせいページくぎり	
\\	強制モード	きょうせいモード	
\\	共存するサービス要素	きょうぞんするサービスようそ	
\\	共存セグメント	きょうぞんセグメント	
\\	筐体	きょうたい	
\\	狭帯域	きょうたいいき	
\\	協調試験法	きょうちょうしけんほう	
\\	協調的マルチタスク	きょうちょうてきマルチタスク	
\\	強調表示	きょうちょうひょうじ	
\\	強調表示フィルタ	きょうちょうひょうじフィルタ	
\\	共通因数	きょうつういんすう	
\\	共通化	きょうつうか	
\\	共通管理情報サービス	きょうつうかんりじょうほうサービス	
\\	共通管理情報サービス要素	きょうつうかんりじょうほうサービスようそ	
\\	共通業務	きょうつうぎょうむ	
\\	共通細目	きょうつうさいもく	
\\	共通識別子	きょうつうしきべつし	
\\	共通線	きょうつうせん	
\\	共通線信号	きょうつうせんしんごう	
\\	共通線信号網	きょうつうせんしんごうもう	
\\	共通内容部	きょうつうないようぶ	
\\	共通内容部記述部	きょうつうないようぶきじゅつぶ	
\\	共通文書	きょうつうぶんしょ	
\\	共通領域	きょうつうりょういき	
\\	共通論理構造	きょうつうろんりこうぞう	
\\	共通割付け構造	きょうつうわりつけこうぞう	
\\	共通アプリケーション環境	きょうつうアプリケーションかんきょう	
\\	共通オブジェクトリクエストブローカアーキテクチャ	きょうつうオブジェクトリクエストブローカアーキテクチャ	
\\	共通システム領域	きょうつうシステムりょういき	
\\	共通セグメント	きょうつうセグメント	
\\	共通デスクトップ環境	きょうつうデスクトップかんきょう	
\\	共通バッファ	きょうつうバッファ	
\\	共通フォーマット	きょうつうフォーマット	
\\	共通ブロック	きょうつうブロック	
\\	共通プログラム	きょうつうプログラム	
\\	共通モデル	きょうつモデル	
\\	協同収集	きょうどうしゅうしゅう	
\\	共同実験	きょうどうじっけん	
\\	共同利用型音声照会通知システム	きょうどうりようけいおんせいしょうかいつうちシステム	
\\	鏡面反射	きょうめんはんしゃ	
\\	鏡面反射係数	きょうめんはんしゃけいすう	
\\	鏡面反射指数	きょうめんはんしゃしすう	
\\	鏡面反射色	きょうめんはんしゃしょく	
\\	共役転置	きょうやくてんち	
\\	共有鍵	きょうゆうかぎ	
\\	共有鍵暗号方式	きょうゆうかぎあんごうほうしき	
\\	共有資源	きょうゆうしげん	
\\	共有資産	きょうゆうしさん	
\\	強誘電体メモリ	きょうゆうでんたいメモリ	
\\	共有サーバー	きょうゆうサーバー	
\\	共有ディスク	きょうゆうディスク	
\\	共有ディレクトリ	きょうゆうディレクトリ	
\\	共有フォルダ	きょうゆうフォルダ	
\\	共有プリンタ	きょうゆうプリンタ	
\\	共有メディア	きょうゆうメディア	
\\	共有メモリー・共有メモリ	きょうゆうメモリー・きょうゆうメモリ	
\\	共有ライブラリ	きょうゆうライブラリ	
\\	共用仮想領域	きょうようかそうりょういき	
\\	共用変数	きょうようへんすう	
\\	共用プログラム	きょうようプログラム	
\\	共用ルーチン	きょうようルーチン	
\\	許可信号	きょかしんごう	
\\	局間回線	きょくかんかいせん	
\\	局管理	きょくかんり	
\\	局建	きょくげん	
\\	局所線形計画法	きょくしょせんけいけいかくほう	
\\	局所的	きょくしょてき	
\\	局所的試験法	きょくしょてきしけんほう	
\\	局所分岐	きょくしょぶんき	
\\	局所変数	きょくしょへんすう	
\\	局所名	きょくしょめい	
\\	局所要素	きょくしょようそ	
\\	局所アドレス管理	きょくしょアドレスかんり	
\\	局所モデリング変換	きょくしょモデリングへんかん	
\\	局所レベル	きょくしょレベル	
\\	極性ゼロ復帰記録	きょくせいゼロふっききろく	
\\	曲線近似圧縮	きょくせんきんじあっしゅく	
\\	曲線発生器	きょくせんはっせいき	
\\	局留めアドレス	きょくどめアドレス	
\\	局内クロストーク	きょくないクロストーク	
\\	局コード	きょくコード	
\\	局データ	きょくデータ	
\\	許諾契約	きょだくけいやく	
\\	許容情報速度	きょようじょうほうそくど	
\\	切上げ機能	きりあげきのう	
\\	切り替え時・切り換え時	きりかえとき・きりかえどき	
\\	切替えシステム	きりかえシステム	
\\	切捨て機能	きりすてきのう	
\\	切取り画素配列	きりとりがそはいれつ	
\\	切取り指示子	きりとりしじし	
\\	切取り枠	きりとりわく	
\\	切り張り	きりばり	
\\	切りばり法	きりばりほう	
\\	切戻し	きりもどし	
\\	切戻しなし	きりもどしなし	
\\	記録開始	きろくかいし	
\\	記録境界文字	きろくきょうかいもじ	
\\	記録再生	きろくさいせい	
\\	記録終了	きろくしゅうりょう	
\\	記録情報	きろくじょうほう	
\\	記録媒体	きろくばいたい	
\\	記録変調方式	きろくへんちょうほうしき	
\\	記録密度	きろくみつど	
\\	緊急度	きんきゅうど	
\\	近似雑音	きんじざつおん	
\\	近似雑音系列	きんじざつおんけいれつ	
\\	近似ランダム	きんじランダム	
\\	近似ランダム系列	きんじランダムけいれつ	
\\	禁則処理	きんそくしょり	
\\	金属酸化物半導体	きんぞくさんかぶつはんどうたい	
\\	金属酸化膜半導体	きんぞくさんかまくはんどうたい	
\\	金属被膜加工	きんぞくひまくかこう	
\\	均等割り付け・均等割付	きんとうわりつけ	
\\	疑似命令	ぎじめいれい	
\\	技術委員会	ぎじゅついいんかい	
\\	技術革命	ぎじゅつかくめい	
\\	技術検討	ぎじゅつけんとう	
\\	技術原稿	ぎじゅつげんこう	
\\	技術交換	ぎじゅつこうかん	
\\	技術講習	ぎじゅつこうしゅう	
\\	技術支援	ぎじゅつしえん	
\\	技術仕様書	ぎじゅつしようしょ	
\\	技術調査	ぎじゅつちょうさ	
\\	技術的限界	ぎじゅつてきげんかい	
\\	技術的名称	ぎじゅつてきめいしょう	
\\	技術的問題	ぎじゅつてきもんだい	
\\	技術標準	ぎじゅつひょうじゅん	
\\	技術面で	ぎじゅつめんで	
\\	技術要件	ぎじゅつようけん	
\\	技術コスト	ぎじゅつコスト	
\\	擬似乱数列	ぎじらんすうれつ	
\\	疑似SRAM	ぎじエスラム	
\\	疑似カラー	ぎじカラー	
\\	疑似コード	ぎじコード	
\\	疑似マルチタスク	ぎじマルチタスク	
\\	疑似リエントラントプログラム	ぎじリエントラントプログラム	
\\	擬寸法集合体	ぎすんぽうしゅうごうたい	
\\	逆切り落し	ぎゃくきりおとし	
\\	逆多重化	ぎゃくたじゅうか	
\\	逆符号化	ぎゃくふごうか	
\\	逆編集	ぎゃくへんしゅう	
\\	逆方向通信路	ぎゃくほうこうつうしんろ	
\\	逆方向読取り	ぎゃくほうこうよみとり	
\\	逆方向LANチャネル	ぎゃくほうこうランチャネル	
\\	逆読み	ぎゃくよみ	
\\	逆アセンブラ	ぎゃくアセンブラ	
\\	逆アセンブル	ぎゃくアセンブル	
\\	逆クリッピング	ぎゃくクリッピング	
\\	逆ポーランド記法	ぎゃくポーランドきほう	
\\	逆ポーランド式	ぎゃくポーランドしき	
\\	逆ポーランド表記式計算器	ぎゃくポーランドひょうきしきけいさんき	
\\	逆ポーランド表記法	ぎゃくポーランドひょうきほう	
\\	行位置	ぎょういち	
\\	行印字装置	ぎょういんじそうち	
\\	行送り	ぎょうおくり	
\\	行送り幅	ぎょうおくりはば	
\\	業界規範	ぎょうかいきはん	
\\	業界精通者	ぎょうかいせいつうしゃ	
\\	業界標準化	ぎょうかいひょうじゅんか	
\\	業界標準アーキテクチャ	ぎょうかいひょうじゅんアーキテクチャ	
\\	行間隔	ぎょうかんかく	
\\	行起点	ぎょうきてん	
\\	行始端	ぎょうしたん	
\\	行進行方向	ぎょうしんこうほうこう	
\\	行数カウンタ	ぎょうすうカウンタ	
\\	行揃え	ぎょうそろえ	
\\	行端揃え	ぎょうたんそろえ	
\\	行頭	ぎょうとう	
\\	行の進む方向	ぎょうのすすむほうこう	
\\	行の高さ	ぎょうのたかさ	
\\	行の長さ	ぎょうのながさ	
\\	行番号	ぎょうばんごう	
\\	行方向	ぎょうほうこう	
\\	行方向奇偶検査	ぎょうほうこうきぐうけんさ	
\\	業務概要	ぎょうむがいよう	
\\	業務のカテゴリィ	ぎょうむのカテゴリィ	
\\	業務アプリケーション	ぎょうむアプリケーション	
\\	業務スケジュール	ぎょうむスケジュール	
\\	行列演算	ぎょうれつえんざん	
\\	行列記法	ぎょうれつきほう	
\\	行列代数	ぎょうれつだいすう	
\\	行列表現	ぎょうれつひょうげん	
\\	行列要素	ぎょうれつようそ	
\\	行枠	ぎょうわく	
\\	行ピッチ	ぎょうピッチ	
\\	銀行ANSER	ぎんこうアンサー	
\\	空状態	くうじょうたい	
\\	空対応表	くうたいおうひょう	
\\	空内部様式	くうないぶようしき	
\\	空白化	くうはくか	
\\	空白列	くうはくれつ	
\\	空白デリミター	くうはくデリミター	
\\	空冷ファン	くうれいファン	
\\	空列	くうれつ	
\\	空連結集合	くうれんけつしゅうごう	
\\	空アドレス	くうアドレス	
\\	区きり子	くきりし	
\\	区切り記号	くぎりきごう	
\\	区切り子機能	くぎりしきのう	
\\	区切り子集合	くぎりししゅうごう	
\\	区切り子集合引数	くぎりししゅうごうひきすう	
\\	区切り子文字	くぎりしもじ	
\\	区切り子文字列	くぎりしもじれつ	
\\	区切り点	くぎりてん	
\\	区切り文字	くぎりもじ	
\\	草の根ネット	くさのねネット	
\\	草の根BBS	くさのねビービーエス	
\\	屑	くず	
\\	下りリンク	くだりリンク	
\\	区点	くてん	
\\	区点コード	くてんコード	
\\	句読文字	くとうもじ	
\\	国番号	くにばんごう	
\\	国コード	くにコード	
\\	区分番号	くぶんばんごう	
\\	区分編成法	くぶんへんせいほう	
\\	区分編成データセット	くぶんへんせいデータセット	
\\	区分編成ファイル	くぶんへんせいファイル	
\\	区分データセット	くぶんデータセット	
\\	区分ファイル	くぶんファイル	
\\	組上り濃度	くみあがりのうど	
\\	組み合わせ回路・組合せ回路	くみあわせかいろ	
\\	組合せ条件	くみあわせじょうけん	
\\	組み込みコマンド	くみこみコマンド	
\\	組み込みプロセッサ	くみこみプロセッサ	
\\	組始端	くみしたん	
\\	組終端	くみしゅうたん	
\\	位取り表記法	くらいどりひょうきほう	
\\	位取り表現	くらいどりひょうげん	
\\	位取り表現法	くらいどりひょうげんほう	
\\	繰返し演算	くりかえしえんざん	
\\	繰出し孔	くりだしこう	
\\	繰出し孔トラック	くりだしこうトラック	
\\	繰出し孔ピッチ	くりだしこうピッチ	
\\	加わった呼量	くわわったこりょう	
\\	区オクテット	くオクテット	
\\	偶数パリティ	ぐうすうパリティ	
\\	偶発時用手続き	ぐうはつじようてつづき	
\\	具象構文	ぐしょうこうぶん	
\\	具象構文引数	ぐしょうこうぶんひきすう	
\\	具体的対象	ぐたいてきたいしょう	
\\	群オクテット	ぐんオクテット	
\\	経営情報システム	けいえいじょうほうシステム	
\\	計画停止	けいかくていし	
\\	計画評価	けいかくひょうか	
\\	経過計時機構	けいかけいじきこう	
\\	経過時間	けいかじかん	
\\	計算化学	けいさんかがく	
\\	計算可能性	けいさんかのうせい	
\\	計算機依存言語	けいさんきいぞんげんご	
\\	計算機科学	けいさんきかがく	
\\	計算機構成	けいさんきこうせい	
\\	計算機出力マイクロフィルミング	けいさんきしゅつりょくマイクロフィルミング	
\\	計算機出力マイクロフィルム印字装置	けいさんきしゅつりょくマイクロフィルムいんじそうち	
\\	計算機出力マイクロフィルム装置	けいさんきしゅつりょくマイクロフィルムそうち	
\\	計算機図形処理	けいさんきずけいしょり	
\\	計算機設計言語	けいさんきせっけいげんご	
\\	計算機の語	けいさんきのご	
\\	計算機間	けいさんきま	
\\	計算機名	けいさんきめい	
\\	計算機システム	けいさんきシステム	
\\	計算機システム監査	けいさんきシステムかんさ	
\\	計算機システムの安全保護	けいさんきシステムのあんぜんほご	
\\	計算機センター	けいさんきセンター	
\\	計算機ネットワーク	けいさんきネットワーク	
\\	計算木	けいさんぎ	
\\	計算構造	けいさんこうぞう	
\\	計算順序	けいさんじゅんじょ	
\\	計算の複雑性	けいさんのふくざつせい	
\\	形式関係	けいしきかんけい	
\\	形式仕様書	けいしきしようしょ	
\\	形式変換	けいしきへんかん	
\\	警視庁ハイテク犯罪対策センター	けいしちょうハイテクはんざいたいさくセンター	
\\	計時機構	けいじきこう	
\\	経時優先	けいじゆうせん	
\\	形状定義スプライン	けいじょうていぎスプライン	
\\	形状適合	けいじょうてきごう	
\\	形状保存変換	けいじょうほぞんへんかん	
\\	形状明示配列	けいじょうめいじはいれつ	
\\	形状モデル	けいじょうモデル	
\\	係数器	けいすうき	
\\	計数器	けいすうき	
\\	係数行列	けいすうぎょうれつ	
\\	係数設定モード	けいすうせっていモード	
\\	計数抽象操作	けいすうちゅうしょうそうさ	
\\	計測プロセス	けいそくプロセス	
\\	継続行	けいぞくぎょう	
\\	継続問い合わせ応答処理	けいぞくといあわせおうとうしょり	
\\	継続発注	けいぞくはっちゅう	
\\	継続文	けいぞくぶん	
\\	継続シグナル	けいぞくシグナル	
\\	携帯可能	けいたいかのう	
\\	携帯端末	けいたいたんまつ	
\\	携帯用	けいたいよう	
\\	警報抽象操作	けいほうちゅうしょうそうさ	
\\	警報メッセージ	けいほうメッセージ	
\\	契約者回線	けいやくしゃかいせん	
\\	計量言語学	けいりょうげんごがく	
\\	軽量端末	けいりょうたんまつ	
\\	経路制御	けいろせいぎょ	
\\	経路選択	けいろせんたく	
\\	桁上り・けた上り	けたあがり	
\\	桁上げ・けた上げ	けたあげ	
\\	けた上げ数・桁上げ数	けたあげすう	
\\	桁上げなし加算・けた上げなし加算	けたあげなしかさん	
\\	桁あふれ	けたあふれ	
\\	けた位置・桁位置	けたいち	
\\	けた移動・桁移動	けたいどう	
\\	けた送り・桁送り	けたおくり	
\\	結果出力拡張機能	けっかしゅつりょくかくちょうきのう	
\\	結果の一意名	けっかのいちいめい	
\\	結果文書型	けっかぶんしょがた	
\\	結果要素型	けっかようそがた	
\\	結線	けっせん	
\\	決定性言語	けっていせいげんご	
\\	決定的文脈自由文法	けっていてきぶんみゃくじゆうぶんぽう	
\\	決定表	けっていひょう	
\\	決定理論	けっていりろん	
\\	決定論的	けっていろんてき	
\\	結合記法名	けつごうきほうめい	
\\	結合試験	けつごうしけん	
\\	結合情報量	けつごうじょうほうりょう	
\\	結合性	けつごうせい	
\\	結合要素型	けつごうようそがた	
\\	結合システム	けつごうシステム	
\\	欠落検査	けつらくけんさ	
\\	研究展望報告書	けんきゅうてんぼうほうこくしょ	
\\	研究図書館	けんきゅうとしょかん	
\\	研究報告書	けんきゅうほうこくしょ	
\\	研究グループ	けんきゅうグループ	
\\	検索機能	けんさくきのう	
\\	検索結果	けんさくけっか	
\\	検索項目	けんさくこうもく	
\\	検索語	けんさくご	
\\	検索者	けんさくしゃ	
\\	検索時概念組合せ索引作業	けんさくじがいねんくみあわせさくいんさぎょう	
\\	検索条件	けんさくじょうけん	
\\	検索速度	けんさくそくど	
\\	検索インタフェース	けんさくインタフェース	
\\	検索エンジン	けんさくエンジン	
\\	検索エンジン対策	けんさくエンジンたいさく	
\\	検索エンジンポジショニング	けんさくエンジンポジショニング	
\\	検索エンジンマーケティング	けんさくエンジンマーケティング	
\\	検索ツール	けんさくツール	
\\	検索パス	けんさくパス	
\\	検索ポート	けんさくポート	
\\	検索ロボット	けんさくロボット	
\\	検索ワード	けんさくワード	
\\	検査合計	けんさごうけい	
\\	検査数字	けんさすうじ	
\\	検査文字	けんさもじ	
\\	検査問題	けんさもんだい	
\\	検査キー	けんさキー	
\\	検査プログラム	けんさプログラム	
\\	検出可能要素	けんしゅつかのうようそ	
\\	検出可能セグメント	けんしゅつかのうセグメント	
\\	検出部	けんしゅつぶ	
\\	検証試験	けんしょうしけん	
\\	顕示性	けんじせい	
\\	検討結果	けんとうけっか	
\\	劇場並み	げきじょうなみ	
\\	現位置	げんいち	
\\	限界検査	げんかいけんさ	
\\	限界試験	げんかいしけん	
\\	減基数の補数	げんきすうのほすう	
\\	現金自動取引装置	げんきんじどうとりひきそうち	
\\	原稿台	げんこうだい	
\\	現行の作業ディレクトリ	げんこうのさぎょうディレクトリ	
\\	源項目	げんこうもく	
\\	言語結合	げんごけつごう	
\\	言語構成要素	げんごこうせいようそ	
\\	言語処理系	げんごしょりけい	
\\	言語対象	げんごたいしょう	
\\	言語表示	げんごひょうじ	
\\	言語名	げんごめい	
\\	言語要素	げんごようそ	
\\	言語トランスレータ	げんごトランスレータ	
\\	言語プロセッサ	げんごプロセッサ	
\\	現在行番号	げんざいぎょうばんごう	
\\	現在状態	げんざいじょうたい	
\\	現在世界	げんざいせかい	
\\	現在値	げんざいち	
\\	現在点	げんざいてん	
\\	現在レコード	げんざいレコード	
\\	減算器	げんざんき	
\\	原始言語	げんしげんご	
\\	原始文書型	げんしぶんしょがた	
\\	原始要素型	げんしようそがた	
\\	原資料	げんしりょう	
\\	原始プログラム	げんしプログラム	
\\	原始モジュール	げんしモジュール	
\\	減衰係数	げんすいけいすう	
\\	減衰時間	げんすいじかん	
\\	減衰ひずみ	げんすいひずみ	
\\	減衰歪み	げんすいひずみ	
\\	減衰量	げんすいりょう	
\\	現属性	げんぞくせい	
\\	現対応表	げんたいおうひょう	
\\	現地化	げんちか	
\\	限定子	げんていし	
\\	限定保証	げんていほしょう	
\\	限度検査	げんどけんさ	
\\	原符号化情報種別表示	げんふごうかじょうほうしゅべつひょうじ	
\\	現付番	げんふばん	
\\	原文語	げんぶんご	
\\	原文名	げんぶんめい	
\\	減法混色	げんぽうこんしょく	
\\	厳密認証	げんみつにんしょう	
\\	現要素	げんようそ	
\\	現連結集合	げんれんけつしゅうごう	
\\	現割付け位置	げんわりつけいち	
\\	原EIT	げんイーアイティー	
\\	高域通過フィルタ	こういきつうかフィルタ	
\\	広域網	こういきもう	
\\	広域イーサネット	こういきイーサネット	
\\	広域ネットワーク	こういきネットワーク	
\\	広域LAN	こういきラン	
\\	高位レベル	こういレベル	
\\	光化	こうか	
\\	公開かぎ	こうかいかぎ	
\\	公開鍵	こうかいかぎ	
\\	公開鍵暗号	こうかいかぎあんごう	
\\	公開鍵暗号方式	こうかいかぎあんごうほうしき	
\\	公開鍵基盤	こうかいかぎきばん	
\\	公開識別子	こうかいしきべつし	
\\	高解像度	こうかいぞうど	
\\	公開文	こうかいぶん	
\\	公開文記述	こうかいぶんきじゅつ	
\\	公開文言語	こうかいぶんげんご	
\\	公開文指示シーケンス	こうかいぶんしじシーケンス	
\\	公開文種別	こうかいぶんしゅべつ	
\\	公開文表示版	こうかいぶんひょうじばん	
\\	公開アーカイブ	こうかいアーカイブ	
\\	公開キー	こうかいキー	
\\	交換回線	こうかんかいせん	
\\	交換型	こうかんがた	
\\	交換機	こうかんき	
\\	交換機製品	こうかんきせいひん	
\\	交換機通信事業者	こうかんきつうしんじぎょうしゃ	
\\	交換処理能力	こうかんしょりのうりょく	
\\	交換接続	こうかんせつぞく	
\\	交換網	こうかんもう	
\\	交換様式	こうかんようしき	
\\	交換様式クラス	こうかんようしきクラス	
\\	交換システム	こうかんシステム	
\\	交換データ要素	こうかんデータようそ	
\\	交換レイヤ	こうかんレイヤ	
\\	光学式走査器	こうがくしきそうさき	
\\	光学式文字認識	こうがくしきもじにんしき	
\\	光学式文字読み取り装置	こうがくしきもじよみとりそうち	
\\	光学式文字読取り装置	こうがくしきもじよみとりそうち	
\\	光学式スキャナ	こうがくしきスキャナ	
\\	光学式マウス	こうがくしきマウス	
\\	光学式マーク読み取り装置	こうがくしきマークよみとりそうち	
\\	光学的文字認識	こうがくてきもじにんしき	
\\	光学文字認識	こうがくもじにんしき	
\\	光学ドライブ	こうがくドライブ	
\\	光学マーク読取り	こうがくマークよみとり	
\\	高画質	こうがしつ	
\\	高級言語	こうきゅうげんご	
\\	光源状態	こうげんじょうたい	
\\	光源モデル	こうげんモデル	
\\	交互配置	こうごはいち	
\\	交差合計検査	こうさごうけいけんさ	
\\	公衆回線	こうしゅうかいせん	
\\	公衆回線網	こうしゅうかいせんもう	
\\	公衆電線網	こうしゅうでんせんもう	
\\	公衆電話網	こうしゅうでんわもう	
\\	公衆サービス	こうしゅうサービス	
\\	公衆データネットワーク	こうしゅうデータネットワーク	
\\	公衆メッセージ通信処理サービス	こうしゅうメッセージつうしんしょりサービス	
\\	交渉ページ	こうしょうページ	
\\	更新履歴	こうしんりれき	
\\	更新暦	こうしんれき	
\\	降順キー	こうじゅんキー	
\\	高水準言語	こうすいじゅんげんご	
\\	構成管理	こうせいかんり	
\\	構成機器	こうせいきき	
\\	高精細度テレビ	こうせいさいどテレビ	
\\	構成制御	こうせいせいぎょ	
\\	構成制御委員会	こうせいせいぎょいいんかい	
\\	構成節	こうせいせつ	
\\	構成体	こうせいたい	
\\	構成部品	こうせいぶひん	
\\	構成変更	こうせいへんこう	
\\	構成要素	こうせいようそ	
\\	構成要素型	こうせいようそがた	
\\	構成データ	こうせいデータ	
\\	高速演算機構	こうそくえんざんきこう	
\\	高速化	こうそくか	
\\	高速回線	こうそくかいせん	
\\	高速けた上げ	こうそくけたあげ	
\\	高速処理	こうそくしょり	
\\	高速接続	こうそくせつぞく	
\\	高速多重集配信装置	こうそくたじゅうしゅうはいしんそうち	
\\	高速直列インターフェイス	こうそくちょくれつインターフェイス	
\\	高速通信	こうそくつうしん	
\\	高速通信網	こうそくつうしんもう	
\\	高速伝送	こうそくてんそう	
\\	高速アクセス	こうそくアクセス	
\\	高速イーサネット	こうそくイーサネット	
\\	高速シリアルインターフェース	こうそくシリアルインターフェース	
\\	高速スイッチング	こうそくスイッチング	
\\	高速セレクト	こうそくセレクト	
\\	高速データ通信	こうそくデータつうしん	
\\	高速ネットワーク	こうそくネットワーク	
\\	高速パケット	こうそくパケット	
\\	高速ページモード	こうそくページモード	
\\	高速ページモードDRAM	こうそくページモードディーラム	
\\	高速モデム	こうそくモデム	
\\	構造化言語	こうぞうかげんご	
\\	構造化照会言語	こうぞうかしょうかいげんご	
\\	構造化問い合わせ言語	こうぞうかといあわせげんご	
\\	構造化分析	こうぞうかぶんせき	
\\	構造化プログラミング	こうぞうかプログラミング	
\\	構造化プログラム	こうぞうかプログラム	
\\	構造型	こうぞうがた	
\\	構造型符号化	こうぞうがたふごうか	
\\	構造構文	こうぞうこうぶん	
\\	構造体参照	こうぞうたいさんしょう	
\\	構造体識別子	こうぞうたいしきべつし	
\\	構造体状態リスト	こうぞうたいじょうたいリスト	
\\	構造体成分	こうぞうたいせいぶん	
\\	構造体中央記憶	こうぞうたいちゅうおうきおく	
\\	構造体取出し	こうぞうたいとりだし	
\\	構造体編集	こうぞうたいへんしゅう	
\\	構造体変数	こうぞうたいへんすう	
\\	構造体保管	こうぞうたいほかん	
\\	構造体要素	こうぞうたいようそ	
\\	構造体ネットワーク	こうぞうたいネットワーク	
\\	構造表記法	こうぞうひょうきほう	
\\	構造要素	こうぞうようそ	
\\	広帯域	こうたいいき	
\\	広帯域総合ディジタル通信網	こうたいいきそうごうディジタルつうしんもう	
\\	広帯域伝送	こうたいいきでんそう	
\\	広帯域ネットワーク	こうたいいきネットワーク	
\\	高帯域ネットワーク	こうたいいきネットワーク	
\\	後退回復	こうたいかいふく	
\\	交替装置	こうたいそうち	
\\	後退文字	こうたいもじ	
\\	交代トラック	こうたいトラック	
\\	交代パス	こうたいパス	
\\	後退ファイル回復	こうたいファイルかいふく	
\\	後置	こうち	
\\	構築子	こうちくし	
\\	後置表記法	こうちひょうきほう	
\\	交直両用式計算器	こうちょくりょうようしきけいさんき	
\\	公的公開識別子	こうてきこうかいしきべつし	
\\	公的公開識別子誤り	こうてきこうかいしきべつしあやまり	
\\	高度技術	こうどぎじゅつ	
\\	高度交通システム	こうどこうつうシステム	
\\	高度通信システム相互接続	こうどつうしんシステムそうごせつぞく	
\\	高度サービス	こうどサービス	
\\	構内交換機	こうないこうかんき	
\\	構内通信網	こうないつうしんもう	
\\	構内電話交換機	こうないでんわこうかんき	
\\	構文解析	こうぶんかいせき	
\\	構文解析対象文字データ	こうぶんかいせきたいしょうもじデータ	
\\	構文規則	こうぶんきそく	
\\	構文参照文字集合	こうぶんさんしょうもじしゅうごう	
\\	構文素	こうぶんそ	
\\	構文エラー	こうぶんエラー	
\\	構文チェッカ	こうぶんチェッカ	
\\	後方一致	こうほういっち	
\\	後方境界面	こうほうきょうかいめん	
\\	後方推論	こうほうすいろん	
\\	後方端	こうほうたん	
\\	候補デスクリプタ	こうほデスクリプタ	
\\	高密度ディスケット	こうみつどディスケット	
\\	項目識別子	こうもくしきべつし	
\\	項目選択	こうもくせんたく	
\\	項目名	こうもくめい	
\\	攻略	こうりゃく	
\\	孔パターン	こうパターン	
\\	小型化技術	こがたかぎじゅつ	
\\	顧客サービス	こきゃくサービス	
\\	子局	こきょく	
\\	国際照明委員会	こくさいしょうめいいいんかい	
\\	国際逐次刊行物データシステム	こくさいちくじかんこうぶつデータシステム	
\\	国際通話	こくさいつうわ	
\\	国際電気標準会議	こくさいでんきひょうじゅんかいぎ	
\\	国際標準化	こくさいひょうじゅんか	
\\	国際標準化機構	こくさいひょうじゅんかきこう	
\\	国際標準逐次刊行物番号	こくさいひょうじゅんちくじかんこうぶつばんごう	
\\	国際符号化文字集合	こくさいふごうかもじしゅうごう	
\\	国際無線通信諮問委員会	こくさいむせんつうしんしもんいいんかい	
\\	刻時機構	こくじきこう	
\\	刻時信号	こくじしんごう	
\\	刻時トラック	こくじトラック	
\\	刻時パルス	こくじパルス	
\\	国別コード	こくべつコード	
\\	国名コード	こくめいコード	
\\	子構造体	ここうぞうたい	
\\	故障に強い	こしょうにこわい	
\\	呼処理	こしょり	
\\	子順序番号	こじゅんじょばんごう	
\\	個人間通信	こじんかんつうしん	
\\	個人間メッセージ通信サービス	こじんかんメッセージつうしんサービス	
\\	個人通信サービス	こじんつうしんサービス	
\\	個人向け	こじんむけ	
\\	個人用オーガナイザ	こじんようオーガナイザ	
\\	個人コミュニケーションサービス	こじんコミュニケーションサービス	
\\	呼制御	こせいぎょ	
\\	呼制御手順	こせいぎょてじゅん	
\\	子操作	こそうさ	
\\	呼損率	こそんりつ	
\\	固体撮像デバイス	こたいさつぞうデバイス	
\\	固定関数発生器	こていかんすうはっせいき	
\\	固定記憶装置	こていきおくそうち	
\\	固定基数記数法	こていきすうきすうほう	
\\	固定基数表記法	こていきすうひょうきほう	
\\	固定小数点演算	こていしょうすうてんえんざん	
\\	固定小数点数	こていしょうすうてんすう	
\\	固定小数点表示法	こていしょうすうてんひょうじほう	
\\	固定小数点方式	こていしょうすうてんほうしき	
\\	固定小数点レジスタ	こていしょうすうてんレジスタ	
\\	固定字送り	こていじおくり	
\\	固定字送り幅	こていじおくりはば	
\\	固定接続	こていせつぞく	
\\	固定属性	こていぞくせい	
\\	固定長セル	こていちょうセル	
\\	固定長パケット	こていちょうパケット	
\\	固定長レコード	こていちょうレコード	
\\	固定幅フォント	こていはばフォント	
\\	固定媒体	こていばいたい	
\\	固定ディスク	こていディスク	
\\	固定ディスク装置	こていディスクそうち	
\\	固定ディスクドライブ	こていディスクドライブ	
\\	固定ビットレート	こていビットレート	
\\	固定ヘッド	こていヘッド	
\\	固定ヘッド式ディスク	こていヘッドしきディスク	
\\	個別記号	こべつきごう	
\\	個別部品	こべつぶひん	
\\	小文字名前開始文字	こもじなまえかいしもじ	
\\	小文字名前文字	こもじなまえもじ	
\\	固有関数	こゆうかんすう	
\\	固有色	こゆうしょく	
\\	固有色データ	こゆうしょくデータ	
\\	固有選択字句	こゆうせんたくじく	
\\	固有の大小順序	こゆうのだいしょうじゅんじょ	
\\	固有文字集合	こゆうもじしゅうごう	
\\	孤立概念	こりつがいねん	
\\	混合基数記数法	こんごうきすうきすうほう	
\\	混合基数表記法	こんごうきすうひょうきほう	
\\	混合基底記数法	こんごうきていきすうほう	
\\	混合基底表記法	こんごうきていひょうきほう	
\\	混合指定表	こんごうしていひょう	
\\	子エントリ	こエントリ	
\\	子プロセス	こプロセス	
\\	合計機能	ごうけいきのう	
\\	合計検査	ごうけいけんさ	
\\	合計穿孔機	ごうけいせんこうき	
\\	合計容量	ごうけいようりょう	
\\	合計カウンタ	ごうけいカウンタ	
\\	合成形分類体系	ごうせいがたぶんるいたいけい	
\\	合成モデリング変換	ごうせいモデリングへんかん	
\\	合理性検査	ごうりせいけんさ	
\\	互換機	ごかんき	
\\	互換性試験	ごかんせいしけん	
\\	互換製品	ごかんせいひん	
\\	互換テスト	ごかんテスト	
\\	極低温記憶装置	ごくていおんきおくそうち	
\\	語構成記憶装置	ごこうせいきおくそうち	
\\	誤差拡散法	ごさかくさんほう	
\\	誤差幅	ごさはば	
\\	誤差範囲	ごさはんい	
\\	最右端	さいうたん	
\\	最下位	さいかい	
\\	最下位ビット	さいかいビット	
\\	再開フェス	さいかいフェス	
\\	再帰関数	さいきかんすう	
\\	再帰的	さいきてき	
\\	再帰的定義	さいきてきていぎ	
\\	再帰的サブルーチン	さいきてきサブルーチン	
\\	再起動	さいきどう	
\\	再帰呼び出し	さいきよびだし	
\\	再帰呼出し	さいきよびだし	
\\	最近稼働	さいきんかどう	
\\	再組立	さいくみたて	
\\	再計算	さいけいさん	
\\	最左端	さいさたん	
\\	再始動	さいしどう	
\\	最終値	さいしゅうち	
\\	最終認可	さいしゅうにんか	
\\	最終版	さいしゅうはん	
\\	最終バージョン	さいしゅうバージョン	
\\	最小化ボタン	さいしょうかボタン	
\\	最小構成	さいしょうこうせい	
\\	最小動作	さいしょうどうさ	
\\	最小有効数字	さいしょうゆうこうすうじ	
\\	最小有効ビット	さいしょうゆうこうビット	
\\	最小SGML文書	さいしょうエスジーエムエルぶんしょ	
\\	再初期設定	さいしょきせってい	
\\	最初の値	さいしょのあたい	
\\	最初の権利者	さいしょのけんりしゃ	
\\	最初のアクティビティ識別子	さいしょのアクティビティしきべつし	
\\	再実行	さいじっこう	
\\	再実行時間	さいじっこうじかん	
\\	最上位	さいじょうい	
\\	最上位ビット	さいじょういビット	
\\	再生保証領域	さいせいほしょうりょういき	
\\	再設定	さいせってい	
\\	再接続	さいせつぞく	
\\	再宣言	さいせんげん	
\\	再送信	さいそうしん	
\\	最大化ボタン	さいだいかボタン	
\\	最大許容正規過電圧	さいだいきょようせいきかでんあつ	
\\	最大許容同相過電圧	さいだいきょようどうそうかでんあつ	
\\	最大作動正規電圧	さいだいさどうせいきでんあつ	
\\	最大作動同相電圧	さいだいさどうどうそうでんあつ	
\\	最大数	さいだいすう	
\\	最大正規電圧	さいだいせいきでんあつ	
\\	最大同相電圧	さいだいどうそうでんあつ	
\\	最大表示	さいだいひょうじ	
\\	最大有効数字	さいだいゆうこうすうじ	
\\	最大有効ビット	さいだいゆうこうビット	
\\	最大バーストサイズ	さいだいバーストサイズ	
\\	最大ビット長	さいだいビットちょう	
\\	最大フレームサイズ	さいだいフレームサイズ	
\\	最長経路	さいちょうけいろ	
\\	最適ルート	さいてきルート	
\\	再度組み立て	さいどくみたて	
\\	再入可能	さいにゅうかのう	
\\	再入可能手続き	さいにゅうかのうてつづき	
\\	再入可能サブルーチン	さいにゅうかのうサブルーチン	
\\	再入可能プログラム	さいにゅうかのうプログラム	
\\	再入可能ルーチン	さいにゅうかのうルーチン	
\\	再配置可能	さいはいちかのう	
\\	再配置可能アドレス	さいはいちかのうアドレス	
\\	再配置可能コード	さいはいちかのうコード	
\\	再配置可能プログラム	さいはいちかのうプログラム	
\\	最繁正時	さいぱんせいじ	
\\	最繁トラヒック時	さいぱんトラヒックとき	
\\	再表示制御	さいひょうじせいぎょ	
\\	再描画	さいびょうが	
\\	細分化	さいぶんか	
\\	細胞状	さいぼうじょう	
\\	再読み込み・再読込み	さいよみこみ	
\\	再利用率	さいりようりつ	
\\	再割当	さいわりあて	
\\	再コンパイル	さいコンパイル	
\\	再セットアップ	さいセットアップ	
\\	再ルーチング	さいルーチング	
\\	再ロード	さいロード	
\\	先入れ先出し記憶装置	さきいれさきだしきおくそうち	
\\	先入れ先出しリスト	さきいれさきだしリスト	
\\	先取り	さきどり	
\\	作業域	さぎょういき	
\\	作業場所節	さぎょうばしょせつ	
\\	作業負荷	さぎょうふか	
\\	作業ディレクトリ	さぎょうディレクトリ	
\\	索引言語	さくいんげんご	
\\	索引語	さくいんご	
\\	索引作業	さくいんさぎょう	
\\	索引時概念組合せ索引作業	さくいんじがいねんくみあわせさくいんさぎょう	
\\	索引順次アクセス方式	さくいんじゅんじアクセスほうしき	
\\	索引順編成データセット	さくいんじゅんへんせいデータセット	
\\	索引順編成ファイル	さくいんじゅんへんせいファイル	
\\	索引順アクセス方式	さくいんじゅんアクセスほうしき	
\\	索引精度	さくいんせいど	
\\	索引編成	さくいんへんせい	
\\	索引用語の付与	さくいんようごのふよ	
\\	索引トラック	さくいんトラック	
\\	索引ファイル	さくいんファイル	
\\	削除抽象操作	さくじょちゅうしょうそうさ	
\\	作図装置	さくずそうち	
\\	作図装置増分量	さくずそうちぞうぶんりょう	
\\	作成者語	さくせいしゃご	
\\	作成日時	さくせいにちじ	
\\	作成システム	さくせいシステム	
\\	作譜	さくふ	
\\	差し替え表示	さしかえひょうじ	
\\	差し込み印刷	さしこみいんさつ	
\\	差動位相偏移変調	さどういそうへんいへんちょう	
\\	差動増幅器	さどうぞうふくき	
\\	作動電圧表示器	さどうでんあつひょうじき	
\\	差分	さぶん	
\\	差分法	さぶんほう	
\\	差分ファイル	さぶんファイル	
\\	左辺値	さへんち	
\\	差別化フィーチャ	さべつかフィーチャ	
\\	左方端	さほうたん	
\\	左右突出部	さゆうとっしゅつぶ	
\\	作用対象	さようたいしょう	
\\	三角形集合	さんかくけいしゅうごう	
\\	参考マニュアル	さんこうマニュアル	
\\	算式通り方式計算器	さんしきどおりほうしきけいさんき	
\\	三者通話機能	さんしゃつうわきのう	
\\	参照許可	さんしょうきょか	
\\	参照形式	さんしょうけいしき	
\\	参照経路	さんしょうけいろ	
\\	参照結合	さんしょうけつごう	
\\	参照権	さんしょうけん	
\\	参照検索	さんしょうけんさく	
\\	参照先	さんしょうさき	
\\	参照事項検索	さんしょうじこうけんさく	
\\	参照線	さんしょうせん	
\\	参照選択型	さんしょうせんたくがた	
\\	参照呼び出し	さんしょうよびだし	
\\	参照渡し	さんしょうわたし	
\\	参照アクセス	さんしょうアクセス	
\\	参照キー	さんしょうキー	
\\	参照モデル	さんしょうモデル	
\\	三次の多項式	さんじのたこうしき	
\\	算術演算	さんじゅつえんざん	
\\	算術演算機構	さんじゅつえんざんきこう	
\\	算術演算子	さんじゅつえんざんし	
\\	算術演算装置	さんじゅつえんざんそうち	
\\	算術演算命令	さんじゅつえんざんめいれい	
\\	算術加算	さんじゅつかさん	
\\	算術けた送り	さんじゅつけたおくり	
\\	算術作用素	さんじゅつさようそ	
\\	算術式	さんじゅつしき	
\\	算術乗算	さんじゅつじょうざん	
\\	算術文	さんじゅつぶん	
\\	算術命令	さんじゅつめいれい	
\\	算術論理演算機構	さんじゅつろんりえんざんきこう	
\\	算術論理演算装置	さんじゅつろんりえんざんそうち	
\\	算術和	さんじゅつわ	
\\	算術レジスタ	さんじゅつレジスタ	
\\	算体	さんたい	
\\	算程	さんてい	
\\	算譜	さんぷ	
\\	散布度	さんぷど	
\\	散布量	さんぷりょう	
\\	算法言語	さんぽうげんご	
\\	雑音除去	ざつおんじょきょ	
\\	雑音バースト信号	ざつおんバーストしんごう	
\\	雑時間	ざつじかん	
\\	座標格子	ざひょうこうし	
\\	座標図形処理	ざひょうずけいしょり	
\\	座標点	ざひょうてん	
\\	支援グループ	しえんグループ	
\\	視覚的エディタ	しかくてきエディタ	
\\	紫外線消去型EPROM	しがいせんしょうきょかたイープロム	
\\	敷居値演算	しきいちえんざん	
\\	敷居値関数	しきいちかんすう	
\\	敷居値素子	しきいちそし	
\\	敷居値ゲート	しきいちゲート	
\\	色値	しきち	
\\	色度信号	しきどしんごう	
\\	識別子参照値	しきべつしさんしょうち	
\\	識別子参照並び	しきべつしさんしょうならび	
\\	識別子値	しきべつしち	
\\	識別子オクテット	しきべつしオクテット	
\\	識別情報にもとづく安全保護方針	しきべつじょうほうにもとづくあんぜんほごほうしん	
\\	識別番号	しきべつばんごう	
\\	識別名	しきべつめい	
\\	試験管理プロトコル	しけんかんりプロトコル	
\\	試験機関	しけんきかん	
\\	試験協調手続	しけんきょうちょうてつづき	
\\	試験群	しけんぐん	
\\	試験群目標	しけんぐんもくひょう	
\\	試験計画書	しけんけいかくしょ	
\\	試験結果	しけんけっか	
\\	試験工程	しけんこうてい	
\\	試験項目	しけんこうもく	
\\	試験項目誤り	しけんこうもくあやまり	
\\	試験項目名	しけんこうもくめい	
\\	試験事象	しけんじしょう	
\\	試験実現者	しけんじつげんしゃ	
\\	試験条件	しけんじょうけん	
\\	試験体	しけんたい	
\\	試験対象	しけんたいしょう	
\\	試験対象システム	しけんたいしょうシステム	
\\	試験端末	しけんたんまつ	
\\	試験目的	しけんもくてき	
\\	試験用プロトコル実装補助情報	しけんようプロトコルじっそうほじょじょうほう	
\\	試験データ	しけんデータ	
\\	試験パターン	しけんパターン	
\\	資源共有	しげんきょうゆう	
\\	資源文書	しげんぶんしょ	
\\	資源分配	しげんぶんぱい	
\\	資源割振り	しげんわりふり	
\\	試作環境	しさくかんきょう	
\\	資財所要量計画	しざいしょようりょうけいかく	
\\	四捨五入機能	ししゃごにゅうきのう	
\\	私書箱番号	ししょばこばんごう	
\\	指示抄録	しじしょうろく	
\\	指示状態	しじじょうたい	
\\	指数演算子	しすうえんざんし	
\\	指数的	しすうてき	
\\	指数的成長	しすうてきせいちょう	
\\	私設回線	しせつかいせん	
\\	私設管理領域	しせつかんりりょういき	
\\	私設領域名	しせつりょういきめい	
\\	私設ディレクトリ管理領域	しせつディレクトリかんりりょういき	
\\	自然言語処理	しぜんげんごしょり	
\\	自然文検索	しぜんぶんけんさく	
\\	子孫構造体	しそんこうぞうたい	
\\	下つき	したつき	
\\	下付き文字	したつきもじ	
\\	下向き矢印	したむきやじるし	
\\	下矢印キー	したやじるしキー	
\\	始端	したん	
\\	始端部	したんぶ	
\\	視聴覚センタ	しちょうかくセンタ	
\\	失効日時表示	しっこうにちじひょうじ	
\\	失敗判定	しっぱいはんてい	
\\	室内実験	しつないじっけん	
\\	指定可能点	していかのうてん	
\\	指定子	していし	
\\	指定文字そろえ	していもじそろえ	
\\	指動輪	しどうりん	
\\	市内局番	しないきょくばん	
\\	市販のソフト	しはんのソフト	
\\	市販ソフト	しはんソフト	
\\	指標付き色指定	しひょうつきいろしてい	
\\	指標付きアドレス	しひょうつきアドレス	
\\	指標名	しひょうめい	
\\	指標データ項目	しひょうデータこうもく	
\\	指標レジスタ	しひょうレジスタ	
\\	四辺形格子	しへんけいこうし	
\\	車々間通信	しゃしゃかんつうしん	
\\	写真図書館	しゃしんとしょかん	
\\	視野移動	しやいどう	
\\	周期冗長検査	しゅうきじょうちょうけんさ	
\\	集計結果	しゅうけいけっか	
\\	集合型	しゅうごうがた	
\\	集合値属性	しゅうごうちぞくせい	
\\	修飾されたデータ名	しゅうしょくされたデータめい	
\\	修飾キー	しゅうしょくキー	
\\	集信装置	しゅうしんそうち	
\\	修正時間	しゅうせいじかん	
\\	集成体型	しゅうせいたいけい	
\\	修正版	しゅうせいばん	
\\	修正プログラム	しゅうせいプログラム	
\\	修正モード	しゅうせいモード	
\\	集積回路記憶装置	しゅうせきかいろきおくそうち	
\\	集積回路メモリ	しゅうせきかいろメモリ	
\\	集積回路メモリー	しゅうせきかいろメモリー	
\\	集線装置	しゅうせんそうち	
\\	終端間	しゅうたんかん	
\\	終端間暗号化	しゅうたんかんあんごうか	
\\	終端せず	しゅうたんせず	
\\	終端抵抗	しゅうたんていこう	
\\	終端部	しゅうたんぶ	
\\	終端文字	しゅうたんもじ	
\\	終端バイト	しゅうたんバイト	
\\	集団項目	しゅうだんこうもく	
\\	終着局	しゅうちゃくきょく	
\\	集中化	しゅうちゅうか	
\\	集中管理	しゅうちゅうかんり	
\\	集中型N多端点コネクション	しゅうちゅうがたエンたたんてんコネクション	
\\	集中局	しゅうちゅうきょく	
\\	集中局区域	しゅうちゅうきょくくいき	
\\	集中処理	しゅうちゅうしょり	
\\	集中制御	しゅうちゅうせいぎょ	
\\	集中データ処理	しゅうちゅうデータしょり	
\\	集中データ処理システム	しゅうちゅうデータしょりシステム	
\\	集配信装置	しゅうはいしんそうち	
\\	周波数管理	しゅうはすうかんり	
\\	周波数選択性フェージング	しゅうはすうせんたくせいフェージング	
\\	周波数多重	しゅうはすうたじゅう	
\\	周波数分割多重	しゅうはすうぶんかつたじゅう	
\\	周波数分割多元接続	しゅうはすうぶんわりたげんせつぞく	
\\	周波数分割マルチプレクサ	しゅうはすうぶんわりマルチプレクサ	
\\	周波数変位方式	しゅうはすうへんいほうしき	
\\	周波数変調記録	しゅうはすうへんちょうきろく	
\\	周波数変調方式	しゅうはすうへんちょうほうしき	
\\	周波数利用効率	しゅうはすうりようこうりつ	
\\	周波数スペクトル	しゅうはすうスペクトル	
\\	周波数ホッピング	しゅうはすうホッピング	
\\	周辺	しゅうへん	
\\	周辺節点	しゅうへんせってん	
\\	周辺装置	しゅうへんそうち	
\\	周辺ノード	しゅうへんノード	
\\	終了記号	しゅうりょうきごう	
\\	終了条件	しゅうりょうじょうけん	
\\	終了コード	しゅうりょうコード	
\\	終了ステータス	しゅうりょうステータス	
\\	終了タグ	しゅうりょうタグ	
\\	主管機関管理領域	しゅかんきかんかんりりょういき	
\\	主管機関領域名	しゅかんきかんりょういきめい	
\\	主管機関ディレクトリ管理領域	しゅかんきかんディレクトリかんりりょういき	
\\	主官庁管理領域	しゅかんちょうかんりりょういき	
\\	主記憶装置	しゅきおくそうち	
\\	主局	しゅきょく	
\\	縮小命令セットコンピュータ	しゅくしょうめいれいセットコンピュータ	
\\	縮退	しゅくたい	
\\	縮退故障	しゅくたいこしょう	
\\	主刻時機構	しゅこくじきこう	
\\	主制御機構	しゅせいぎょきこう	
\\	主制御装置	しゅせいぎょそうち	
\\	主題表示	しゅだいひょうじ	
\\	主題別シソーラス	しゅだいべつシソーラス	
\\	出回線	しゅつかいせん	
\\	出側	しゅつがわ	
\\	出現回数	しゅつげんかいすう	
\\	出現頻度	しゅつげんひんど	
\\	出力機構	しゅつりょくきこう	
\\	出力基本要素	しゅつりょくきほんようそ	
\\	出力先変更	しゅつりょくさきへんこう	
\\	出力処理	しゅつりょくしょり	
\\	出力手続き	しゅつりょくてつづき	
\\	出力待ち行列	しゅつりょくまちぎょうれつ	
\\	出力サブシステム	しゅつりょくサブシステム	
\\	出力チャンネル	しゅつりょくチャンネル	
\\	出力デバイス	しゅつりょくデバイス	
\\	出力データ	しゅつりょくデータ	
\\	出力ファイル	しゅつりょくファイル	
\\	出力フォーマット	しゅつりょくフォーマット	
\\	出力ポート	しゅつりょくポート	
\\	出力モード	しゅつりょくモード	
\\	出力レコード	しゅつりょくレコード	
\\	出トラヒック	しゅつトラヒック	
\\	手動応答	しゅどうおうとう	
\\	手動機能	しゅどうきのう	
\\	手動呼出し	しゅどうよびだし	
\\	種の用語	しゅのようご	
\\	主配線盤	しゅはいせんばん	
\\	主力商品	しゅりょくしょうひん	
\\	主エントリ	しゅエントリ	
\\	主キー	しゅキー	
\\	主クロック	しゅクロック	
\\	主プログラム	しゅプログラム	
\\	主メモリ	しゅメモリ	
\\	主レコードキー	しゅレコードキー	
\\	照会言語	しょうかいげんご	
\\	照会用端末	しょうかいようたんまつ	
\\	昇華型熱転写方式	しょうかがたねつてんしゃほうしき	
\\	昇華型プリンタ	しょうかがたプリンタ	
\\	障害回復	しょうがいかいふく	
\\	障害しきい値	しょうがいしきいち	
\\	障害地点	しょうがいちてん	
\\	障害追跡	しょうがいついせき	
\\	障害率しきい値	しょうがいりつしきいち	
\\	障害セクタ	しょうがいセクタ	
\\	消去及びプログラム可能読取り専用記憶装置	しょうきょおよびプログラムかのうよみとりせんようきおくそうち	
\\	消去可能記憶装置	しょうきょかのうきおくそうち	
\\	消去ヘッド	しょうきょヘッド	
\\	小計機能	しょうけいきのう	
\\	衝撃式印字装置	しょうげきしきいんじそうち	
\\	消光比	しょうこうひ	
\\	照合機	しょうごうき	
\\	照合順序	しょうごうじゅんじょ	
\\	照合順番	しょうごうじゅんばん	
\\	照合操作	しょうごうそうさ	
\\	詳細課金データ	しょうさいかきんデータ	
\\	詳細設計	しょうさいせっけい	
\\	詳細設定	しょうさいせってい	
\\	詳細説明	しょうさいせつめい	
\\	照射方向	しょうしゃほうこう	
\\	照準記号	しょうじゅんきごう	
\\	昇順整列	しょうじゅんせいれつ	
\\	昇順キー	しょうじゅんキー	
\\	小数点以下の桁数	しょうすうてんいかのけたすう	
\\	小数点記号	しょうすうてんきごう	
\\	省電力機能	しょうでんりょくきのう	
\\	衝突検出	しょうとつけんしゅつ	
\\	小なり	しょうなり・ショウナリ	
\\	承認者	しょうにんしゃ	
\\	承認者表示	しょうにんしゃひょうじ	
\\	承認履歴	しょうにんりれき	
\\	消費側非同期手続き	しょうひがわひどうきてつづき	
\\	証明機関	しょうめいきかん	
\\	証明証	しょうめいしょう	
\\	証明証経路	しょうめいしょうけいろ	
\\	証明証通し番号	しょうめいしょうとおしばんごう	
\\	小面除去	しょうめんじょきょ	
\\	小面法線	しょうめんほうせん	
\\	小面データ	しょうめんデータ	
\\	商用電源式計算器	しょうようでんげんしきけいさんき	
\\	商用版	しょうようはん	
\\	省略時実体	しょうりゃくじじったい	
\\	省略時値	しょうりゃくじち	
\\	省略時コンテキスト	しょうりゃくじコンテキスト	
\\	省略不可能	しょうりゃくふかのう	
\\	省略タグ最小化引数	しょうりゃくタグさいしょうかひきすう	
\\	抄録ページ	しょうろくページ	
\\	書簡箱	しょかんばこ	
\\	初期	しょき	
\\	初期化子	しょきかし	
\\	初期化部	しょきかぶ	
\\	初期刊本	しょきかんぼん	
\\	初期化コード	しょきかコード	
\\	初期化プログラム	しょきかプログラム	
\\	初期画面	しょきがめん	
\\	初期故障	しょきこしょう	
\\	初期条件モード	しょきじょうけんモード	
\\	初期状態	しょきじょうたい	
\\	初期値設定プログラム単位	しょきちせっていプログラムたんい	
\\	初期点	しょきてん	
\\	初期不良	しょきふりょう	
\\	初期連結集合	しょきれんけつしゅうごう	
\\	初期デフォルト	しょきデフォルト	
\\	書式送り	しょしきおくり	
\\	書式換え	しょしきかえ	
\\	書式機能	しょしききのう	
\\	書式検査	しょしきけんさ	
\\	書式制御文字	しょしきせいぎょもじ	
\\	書式付き形式	しょしきつきけいしき	
\\	書式付き処理可能形式	しょしきつきしょりかのうけいしき	
\\	書式付け	しょしきつけ	
\\	書式オーバレイ	しょしきオーバレイ	
\\	書式ファイル	しょしきファイル	
\\	書式フラッシュ	しょしきフラッシュ	
\\	所定期間保存	しょていきかんほぞん	
\\	所定期間保存対象ファイル	しょていきかんほぞんたいしょうファイル	
\\	所定期間保存ファイル	しょていきかんほぞんファイル	
\\	書評見本	しょひょうみほん	
\\	所有者識別子	しょゆうしゃしきべつし	
\\	処理可能形式	しょりかのうけいしき	
\\	処理機構	しょりきこう	
\\	処理系	しょりけい	
\\	処理系依存	しょりけいいぞん	
\\	処理系限界	しょりけいげんかい	
\\	処理系定義	しょりけいていぎ	
\\	処理系定義の動作	しょりけいていぎのどうさ	
\\	処理指令	しょりしれい	
\\	処理指令実体	しょりしれいじったい	
\\	処理済み	しょりずみ	
\\	処理装置	しょりそうち	
\\	処理速度	しょりそくど	
\\	処理対象	しょりたいしょう	
\\	処理内容	しょりないよう	
\\	処理能力	しょりのうりょく	
\\	処理利得	しょりりとく	
\\	処理システム	しょりシステム	
\\	私用かぎ	しようかぎ	
\\	使用可能時間	しようかのうじかん	
\\	使用可能性	しようかのうせい	
\\	使用規制	しようきせい	
\\	使用許諾契約	しようきょだくけいやく	
\\	使用距離	しようきょり	
\\	使用契約	しようけいやく	
\\	仕様作成	しようさくせい	
\\	使用条件	しようじょうけん	
\\	使用性試験	しようせいしけん	
\\	試用版	しようばん	
\\	使用不能時間	しようふのうじかん	
\\	使用率	しようりつ	
\\	使用率パラメータ制御	しようりつパラメータせいぎょ	
\\	使用パラメータ制御	しようパラメータせいぎょ	
\\	使用ベース	しようベース	
\\	使用メモリ	しようメモリ	
\\	指令言語	しれいげんご	
\\	白黒反転モード	しろくろはんでんモード	
\\	白黒ディスプレイ	しろくろディスプレイ	
\\	新規呼	しんきこ	
\\	新機能	しんきのう	
\\	新規ユーザ	しんきユーザ	
\\	真空槽	しんくうそう	
\\	神経回路網	しんけいかいろもう	
\\	信号局	しんごうきょく	
\\	信号局コード	しんごうきょくコード	
\\	信号検出	しんごうけんしゅつ	
\\	信号再生	しんごうさいせい	
\\	信号雑音比	しんごうざつおんひ	
\\	信号処理	しんごうしょり	
\\	信号成形	しんごうせいけい	
\\	信号対雑音比	しんごうたいざつおんひ	
\\	信号対ノイズ比	しんごうたいノイズひ	
\\	信号手順	しんごうてじゅん	
\\	信号変換	しんごうへんかん	
\\	信号網	しんごうもう	
\\	信号劣化	しんごうれっか	
\\	信号チャネル	しんごうチャネル	
\\	信号プロトコル	しんごうプロトコル	
\\	信号ラベル	しんごうラベル	
\\	信号リンク	しんごうリンク	
\\	信号リンクタイマー	しんごうリンクタイマー	
\\	新世代コンピュータ開発機構	しんせだいコンピュータかいはつきこう	
\\	新世代コンピュータ技術開発機構	しんせだいコンピュータぎじゅつかいはつきこう	
\\	深層学習	しんそうがくしゅう	
\\	深層分類体系	しんそうぶんるいたいけい	
\\	診断機能	しんだんきのう	
\\	診断出力	しんだんしゅつりょく	
\\	診断プログラム	しんだんプログラム	
\\	診断メッセージ	しんだんメッセージ	
\\	診断ユーティリティ	しんだんユーティリティ	
\\	親展書留郵便	しんてんかきとめゆうびん	
\\	侵入検知システム	しんにゅうけんちシステム	
\\	侵入テスト	しんにゅうテスト	
\\	真の部分要素	しんのぶぶんようそ	
\\	信頼係数	しんらいけいすう	
\\	信頼限界	しんらいげんかい	
\\	信頼できる機能	しんらいできるきのう	
\\	信頼できるホスト	しんらいできるホスト	
\\	真理値表	しんりちひょう	
\\	新連邦評価基準	しんれんぽうひょうかきじゅん	
\\	新JIS漢字コード	しんジェーアイエスかんじコード・しんジスかんじコード	
\\	字送り点	じおくりてん	
\\	自家製ソフトウェア	じかせいソフトウェア	
\\	直叩き	じかたたき	
\\	時間刻み	じかんきざみ	
\\	時間多重	じかんたじゅう	
\\	時間分割多元接続方式	じかんぶんかつたげんせつぞくほうしき	
\\	時間変換係数	じかんへんかんけいすう	
\\	直リンク	じかリンク・ちょくリンク	
\\	磁気印字装置	じきいんじそうち	
\\	磁気記憶装置	じききおくそうち	
\\	磁気記録	じききろく	
\\	磁気記録媒体	じききろくばいたい	
\\	磁気媒体	じきばいたい	
\\	磁気文字読み取り装置	じきもじよみとりそうち	
\\	磁極間隔	じきょくかんかく	
\\	磁気インク文字	じきインクもじ	
\\	磁気インク文字認識	じきインクもじにんしき	
\\	磁気インク文字読取装置・磁気インク文字読み取り装置・磁気インク文字読取り装置	じきインクもじよみとりそうち	
\\	磁気カード記憶装置	じきカードきおくそうち	
\\	磁気テープ	じきテープ	
\\	磁気テープ記憶装置	じきテープきおくそうち	
\\	磁気ドラム	じきドラム	
\\	磁気ドラム記憶装置	じきドラムきおくそうち	
\\	磁気バブル記憶装置	じきバブルきおくそうち	
\\	磁気バブルメモリー	じきバブルメモリー	
\\	磁気プリンタ	じきプリンタ	
\\	磁気ヘッド	じきヘッド	
\\	磁気メディア	じきメディア	
\\	字句解析	じくかいせき	
\\	字句単位	じくたんい	
\\	字句の構造	じくのこうぞう	
\\	時系列分析	じけいれつぶんせき	
\\	次元数	じげんすう	
\\	自己解凍	じこかいとう	
\\	自己解凍ファイル	じこかいとうファイル	
\\	自己規定項目	じこきていこうもく	
\\	自己結合	じこけつごう	
\\	自己検査符号	じこけんさふごう	
\\	自己資源	じこしげん	
\\	自己終端判別	じこしゅうたんはんべつ	
\\	自己診断	じこしんだん	
\\	自己相似図形	じこそうじずけい	
\\	自己展開ファイル	じこてんかいファイル	
\\	事後ジャーナル化	じごうジャーナルか	
\\	事後条件	じごじょうけん	
\\	事後分析ダンプ	じごぶんせきダンプ	
\\	事後保守	じごほしゅ	
\\	事後保全	じごほぜん	
\\	字下げ	じさげ	
\\	事象投影	じしょうとうえい	
\\	事象報告	じしょうほうこく	
\\	事象待ち行列	じしょうまちぎょうれつ	
\\	事象モード	じしょうモード	
\\	辞書攻撃	じしょこうげき	
\\	事実上の標準	じじつじょうのひょうじゅん	
\\	事実上標準	じじつじょうひょうじゅん	
\\	事前記録媒体	じぜんきろくばいたい	
\\	事前記録データ媒体	じぜんきろくデータばいたい	
\\	事前条件	じぜんじょうけん	
\\	持続可能セルレート	じぞくかのうセルレート	
\\	持続的な有効性	じぞくてきなゆうこうせい	
\\	住宅用	じたくよう	
\\	実験期間	じっけんきかん	
\\	実験網	じっけんもう	
\\	実験システム	じっけんシステム	
\\	実行可能試験項目	じっこうかのうしけんこうもく	
\\	実行可能試験項目群	じっこうかのうしけんこうもくぐん	
\\	実行可能プログラム	じっこうかのうプログラム	
\\	実行可能モジュール	じっこうかのうモジュール	
\\	実行環境	じっこうかんきょう	
\\	実行側応用エンティティ	じっこうがわおうようエンティティ	
\\	実行側CMISEサービス利用者	じっこうがわシーエムアイエスイーサービスりようしゃ	
\\	実行許可	じっこうきょか	
\\	実行形式	じっこうけいしき	
\\	実行形式ライブラリ	じっこうけいしきライブラリ	
\\	実行構文	じっこうこうぶん	
\\	実行時	じっこうじ	
\\	実行時間	じっこうじかん	
\\	実行順序	じっこうじゅんじょ	
\\	実行時エラー	じっこうじエラー	
\\	実行制御ファイル	じっこうせいぎょファイル	
\\	実行速度	じっこうそくど	
\\	実行単位	じっこうたんい	
\\	実効値	じっこうち	
\\	実効長	じっこうちょう	
\\	実効転送速度	じっこうてんそうそくど	
\\	実効輻射電力	じっこうふくしゃでんりょく	
\\	実行不成功	じっこうふせいこう	
\\	実行不能文	じっこうふのうぶん	
\\	実行文	じっこうぶん	
\\	実効命令	じっこうめいれい	
\\	実行優先順位	じっこうゆうせんじゅんい	
\\	実行用計算機	じっこうようけいさんき	
\\	実行用計算機記述項	じっこうようけいさんききじゅつこう	
\\	実行用プログラム	じっこうようプログラム	
\\	実効利用者ID	じっこうりようしゃアイディー	
\\	実効アドレッシングモード	じっこうアドレッシングモード	
\\	実効グループID	じっこうグループアイディー	
\\	実行サイクル	じっこうサイクル	
\\	実効スループット	じっこうスループット	
\\	実行ファイル	じっこうファイル	
\\	実行プロファイル	じっこうプロファイル	
\\	実行レベル	じっこうレベル	
\\	10個1組・十個一組	じっこひとくみ・じっこいちくみ	
\\	実際の小数点	じっさいのしょうすうてん	
\\	実施項目	じっしこうもく	
\\	10進演算・十進演算	じっしんえんざん	
\\	10進演算子・十進演算子	じっしんえんざんし	
\\	10進基数・十進基数	じっしんきすう	
\\	10進小数・十進小数	じっしんしょうすう	
\\	10進小数点・十進小数点	じっしんしょうすうてん	
\\	10進数・十進数	じっしんすう	
\\	10進2進変換・十進二進変換	じっしんにしんへんかん	
\\	十進表記法	じっしんひょうきほう	
\\	十進分類体系	じっしんぶんるいたいけい	
\\	実数値入力装置	じっすうちにゅうりょくそうち	
\\	実績データ	じっせきデータ	
\\	実装者定義	じっそうしゃていぎ	
\\	実装標準	じっそうひょうじゅん	
\\	実装密度	じっそうみつど	
\\	実体化	じったいか	
\\	実体管理系	じったいかんりけい	
\\	実体記述子	じったいきじゅつし	
\\	実体構造	じったいこうぞう	
\\	実体参照	じったいさんしょう	
\\	実体集合	じったいしゅうごう	
\\	実体終了信号	じったいしゅうりょうしんごう	
\\	実体宣言	じったいせんげん	
\\	実体文	じったいぶん	
\\	実体木	じったいぼく	
\\	実開放型システム	じつかいほうがたシステム	
\\	実記憶	じつきおく	
\\	実記憶装置	じつきおくそうち	
\\	実現可能性検討	じつげんかのうせいけんとう	
\\	実現可能性調査	じつげんかのうせいちょうさ	
\\	実現された辺	じつげんされたへん	
\\	実現方法	じつげんほうほう	
\\	実試験手段	じつしけんしゅだん	
\\	実小数点	じつしょうすうてん	
\\	実時間演算	じつじかんえんざん	
\\	実10進小数点・実十進小数点	じつじっしんしょうすうてん	
\\	実受信者	じつじゅしんしゃ	
\\	実引き数・実引数	じつひきすう	
\\	実利用者ID	じつりようしゃアイディー	
\\	実アドレス	じつアドレス	
\\	実アドレス領域	じつアドレスりょういき	
\\	実グループID	じつグループアイディー	
\\	実システム	じつシステム	
\\	実システム環境	じつシステムかんきょう	
\\	実データ	じつデータ	
\\	実パラメタ	じつパラメタ	
\\	実パラメータ	じつパラメータ	
\\	実ファイル	じつファイル	
\\	実ファイルストア	じつファイルストア	
\\	自動応答	じどうおうとう	
\\	自動回送	じどうかいそう	
\\	自動回送表示	じどうかいそうひょうじ	
\\	自動解凍	じどうかいとう	
\\	自動記憶割振り	じどうきおくわりふり	
\\	自動機能	じどうきのう	
\\	自動切り替え	じどうきりかえ	
\\	自動記録器	じどうきろくき	
\\	自動警報	じどうけいほう	
\\	自動航空管制	じどうこうくうかんせい	
\\	自動更新	じどうこうしん	
\\	自動再計算	じどうさいけいさん	
\\	自動再送信要求	じどうさいそうしんようきゅう	
\\	自動再送要求	じどうさいそうようきゅう	
\\	自動再表示	じどうさいひょうじ	
\\	自動索引作業	じどうさくいんさぎょう	
\\	自動削除	じどうさくじょ	
\\	自動式構内交換機	じどうしきこうないこうかんき	
\\	自動周波数制御	じどうしゅうはすうせいぎょ	
\\	自動抄録作業	じどうしょうろくさぎょう	
\\	自動資料作成支援システム	じどうしりょうさくせいしえんシステム	
\\	自動実行	じどうじっこう	
\\	自動制御装置	じどうせいぎょそうち	
\\	自動生成	じどうせいせい	
\\	自動設定繰返し演算	じどうせっていくりかえしえんざん	
\\	自動選択	じどうせんたく	
\\	自動着信	じどうちゃくしん	
\\	自動抽出	じどうちゅうしゅつ	
\\	自動定数機能	じどうていすうきのう	
\\	自動電圧切替	じどうでんあつきりかえ	
\\	自動動作	じどうどうさ	
\\	自動動作種別	じどうどうさしゅべつ	
\\	自動発効	じどうはっこう	
\\	自動発着信装置	じどうはっちゃくしんそうち	
\\	自動振り分け機能	じどうふりわけきのう	
\\	自動変更	じどうへんこう	
\\	自動保存	じどうほぞん	
\\	自動翻訳	じどうほんやく	
\\	自動優先順位グループ	じどうゆうせんじゅんいグループ	
\\	自動預金支払機	じどうよきんしはらいき	
\\	自動呼び出し	じどうよびだし	
\\	自動呼出し	じどうよびだし	
\\	自動呼び出し装置	じどうよびだしそうち	
\\	自動呼出し装置	じどうよびだしそうち	
\\	自動利得制御	じどうりとくせいぎょ	
\\	自動割付けデータ実体	じどうわりつけデータじったい	
\\	自動コンフィギュレーション	じどうコンフィギュレーション	
\\	自動コンフィグレーション	じどうコンフィグレーション	
\\	自動システム保護	じどうシステムほご	
\\	自動データ収集	じどうデータしゅうしゅう	
\\	自動プログラムツール	じどうプログラムツール	
\\	自動ボリューム認識	じどうボリュームにんしき	
\\	自動ボリューム認識機能	じどうボリュームにんしききのう	
\\	自動ポスト	じどうポスト	
\\	自発的判断による不整合判明状態	じはつてきはんだんによるふせいごうはんめいじょうたい	
\\	自発的判断による不整合見込み状態	じはつてきはんだんによるふせいごうみこみじょうたい	
\\	自発的判断ログ記録	じはつてきはんだんログきろく	
\\	時分割	じぶんかつ	
\\	時分割型マルチプレクサ	じぶんかつがたマルチプレクサ	
\\	時分割多元接続	じぶんかつたげんせつぞく	
\\	時分割多重	じぶんかつたじゅう	
\\	時分割多重化	じぶんかつたじゅうか	
\\	時分割多重装置	じぶんかつたじゅうそうち	
\\	時分割システム	じぶんかつシステム	
\\	字方向	じほうこう	
\\	事務管理データ処理	じむかんりデータしょり	
\\	事務機械化	じむきかいか	
\\	事務文書用言語	じむぶんしょようげんご	
\\	事務用	じむよう	
\\	従局	じゅうきょく	
\\	住居表示	じゅうきょひょうじ	
\\	住居表示番号	じゅうきょひょうじばんごう	
\\	充足字句	じゅうそくじく	
\\	従属局	じゅうぞくきょく	
\\	従属コンパイル	じゅうぞくコンパイル	
\\	従属リスト	じゅうぞくリスト	
\\	充填文字	じゅうてんもじ	
\\	重複部分配列	じゅうふくぶぶんはいれつ	
\\	重複名解決フラグ	じゅうふくめいかいけつフラグ	
\\	重要度表示	じゅうようどひょうじ	
\\	重要なSGML文字	じゅうようなエスジーエムエルもじ	
\\	従来型銅線	じゅうらいがたどうせん	
\\	十六進法・16進法	じゅうろくしんほう	
\\	受信相手	じゅしんあいて	
\\	受信側SS利用者	じゅしんがわエスエスりようしゃ	
\\	受信側SPM	じゅしんがわエスピーエム	
\\	受信側エンティティ	じゅしんがわエンティティ	
\\	受信側TS利用者	じゅしんがわティーエスりようしゃ	
\\	受信側トランスポートエンティティ	じゅしんがわトランスポートエンティティ	
\\	受信者指定あて先変更	じゅしんしゃしていあてさきへんこう	
\\	受信専用	じゅしんせんよう	
\\	受信通知	じゅしんつうち	
\\	受信通知要求表示	じゅしんつうちようきゅうひょうじ	
\\	受信不能通知要求表示	じゅしんふのうつうちようきゅうひょうじ	
\\	受信アンテナ	じゅしんアンテナ	
\\	受信バッファ	じゅしんバッファ	
\\	授受特性	じゅじゅとくせい	
\\	受諾側	じゅだくがわ	
\\	受諾側遠隔操作プロトコル機械	じゅだくがわえんかくそうさプロトコルきかい	
\\	受諾側アソシエーション制御プロトコル機械	じゅだくがわアソシエーションせいぎょプロトコルきかい	
\\	受諾側セションサービス利用者	じゅだくがわセションサービスりようしゃ	
\\	受動局	じゅどうきょく	
\\	需要増加	じゅようぞうか	
\\	受領システム	じゅりょうシステム	
\\	巡回冗長検査	じゅんかいじょうちょうけんさ	
\\	巡回セールスマンの問題	じゅんかいセールスマンのもんだい	
\\	巡回ポーリング	じゅんかいポーリング	
\\	循環借り	じゅんかんかり	
\\	循環けた上げ	じゅんかんけたあげ	
\\	循環けた送り	じゅんかんけたおくり	
\\	準拠セル	じゅんきょセル	
\\	準市内通話	じゅんしないつうわ	
\\	順次走査	じゅんじそうさ	
\\	順序回路	じゅんじょかいろ	
\\	順序木	じゅんじょぎ	
\\	順序性	じゅんじょせい	
\\	順序制御	じゅんじょせいぎょ	
\\	順序探索	じゅんじょたんさく	
\\	順序番号	じゅんじょばんごう	
\\	順序列型	じゅんじょれつがた	
\\	順次アクセス	じゅんじアクセス	
\\	純正品	じゅんせいひん	
\\	準内字	じゅんないじ	
\\	順番検査	じゅんばんけんさ	
\\	順番列	じゅんばんれつ	
\\	準備完了ログ記録	じゅんびかんりょうログきろく	
\\	順編成	じゅんへんせい	
\\	順編成法	じゅんへんせいほう	
\\	順編成ファイル	じゅんへんせいファイル	
\\	順方向通信路	じゅんほうこうつうしんろ	
\\	順方向読取り	じゅんほうこうよみとり	
\\	順方向LANチャネル	じゅんほうこうランチャネル	
\\	順呼出し	じゅんよびだし	
\\	順読み	じゅんよみ	
\\	順ファイル	じゅんファイル	
\\	自由曲線	じゆうきょくせん	
\\	自由形式	じゆうけいしき	
\\	上位	じょうい	
\\	上位機種	じょういきしゅ	
\\	上位語	じょういご	
\\	上位互換	じょういごかん	
\\	上位参照	じょういさんしょう	
\\	上位試験手段	じょういしけんしゅだん	
\\	上位省略表記	じょういしょうりゃくひょうき	
\\	上位層	じょういそう	
\\	上位グループ	じょういグループ	
\\	上位ダイアログ	じょういダイアログ	
\\	条件記述部	じょうけんきじゅつぶ	
\\	条件検索	じょうけんけんさく	
\\	条件構成体	じょうけんこうせいたい	
\\	条件式	じょうけんしき	
\\	条件指定	じょうけんしてい	
\\	条件付き構成要素	じょうけんつきこうせいようそ	
\\	条件付き最適化	じょうけんつきさいてきか	
\\	条件付き情報量	じょうけんつきじょうほうりょう	
\\	条件付き飛び越し	じょうけんつきとびこし	
\\	条件付き飛越し命令	じょうけんつきとびこしめいれい	
\\	条件付き分岐	じょうけんつきぶんき	
\\	条件付き平均情報量	じょうけんつきへいきんじょうほうりょう	
\\	条件付きエントロピー・条件付エントロピー	じょうけんつきエントロピー	
\\	条件付きコンパイル・条件付コンパイル	じょうけんつきコンパイル	
\\	条件付きジャンプ・条件付ジャンプ	じょうけんつきジャンプ	
\\	条件付きブランチ	じょうけんつきブランチ	
\\	条件文	じょうけんぶん	
\\	条件分岐	じょうけんぶんき	
\\	条件分岐命令	じょうけんぶんきめいれい	
\\	条件名	じょうけんめい	
\\	条件名条件	じょうけんめいじょうけん	
\\	条件命令	じょうけんめいれい	
\\	条件コード	じょうけんコード	
\\	上下限	じょうげげん	
\\	上下列	じょうげれつ	
\\	上限実行優先順位	じょうげんじっこうゆうせんじゅんい	
\\	上限値	じょうげんち	
\\	状態空間	じょうたいくうかん	
\\	状態情報	じょうたいじょうほう	
\\	状態遷移	じょうたいせんい	
\\	状態遷移図	じょうたいせんいず	
\\	状態変化	じょうたいへんか	
\\	状態変数	じょうたいへんすう	
\\	状態見出し語	じょうたいみだしご	
\\	状態マシン	じょうたいマシン	
\\	常駐型プログラム	じょうちゅうがたプログラム	
\\	常駐終了型プログラム	じょうちゅうしゅうりょうけいプログラム	
\\	常駐終了プログラム	じょうちゅうしゅうりょうプログラム	
\\	常駐制御プログラム	じょうちゅうせいぎょプログラム	
\\	冗長化	じょうちょうか	
\\	冗長検査	じょうちょうけんさ	
\\	冗長構成	じょうちょうこうせい	
\\	冗長出力	じょうちょうしゅつりょく	
\\	冗長電源	じょうちょうでんげん	
\\	冗長符号	じょうちょうふごう	
\\	冗長量	じょうちょうりょう	
\\	情報隠蔽	じょうほういんぺい	
\\	情報化	じょうほうか	
\\	情報科学	じょうほうかがく	
\\	情報革命	じょうほうかくめい	
\\	情報管理	じょうほうかんり	
\\	情報基盤	じょうほうきばん	
\\	情報技術	じょうほうぎじゅつ	
\\	情報群	じょうほうぐん	
\\	情報検索	じょうほうけんさく	
\\	情報検索代行業者	じょうほうけんさくだいこうぎょうしゃ	
\\	情報源	じょうほうげん	
\\	情報交換	じょうほうこうかん	
\\	情報交換用アメリカ標準コード	じょうほうこうかんようアメリカひょうじゅんコード	
\\	情報工学	じょうほうこうがく	
\\	情報資源	じょうほうしげん	
\\	情報資源管理	じょうほうしげんかんり	
\\	情報資源辞書システム	じょうほうしげんじしょシステム	
\\	情報処理	じょうほうしょり	
\\	情報処理技術者試験	じょうほうしょりぎじゅつしゃしけん	
\\	情報提供者	じょうほうていきょうしゃ	
\\	情報転送	じょうほうてんそう	
\\	情報伝達	じょうほうでんたつ	
\\	情報の測度	じょうほうのそくど	
\\	情報発信	じょうほうはっしん	
\\	情報分析	じょうほうぶんせき	
\\	情報分析センタ	じょうほうぶんせきセンタ	
\\	情報要素	じょうほうようそ	
\\	情報インフラ	じょうほうインフラ	
\\	情報オブジェクト	じょうほうオブジェクト	
\\	情報システム	じょうほうシステム	
\\	情報スーパーハイウェイ・情報スーパーハイウエー	じょうほうスーパーハイウェイ・じょうほうスーパーハイウエー	
\\	情報センタ	じょうほうセンタ	
\\	情報ソース	じょうほうソース	
\\	情報チャネル	じょうほうチャネル	
\\	情報トラック	じょうほうトラック	
\\	情報ハイウェイ	じょうほうハイウェイ	
\\	情報ハイウェー	じょうほうハイウェー	
\\	情報フロー	じょうほうフロー	
\\	情報ブロック	じょうほうブロック	
\\	情報ベース	じょうほうベース	
\\	情報ベース種別	じょうほうベースしゅべつ	
\\	情報リテラシ	じょうほうリテラシ	
\\	剰余検査	じょうよけんさ	
\\	除算演算子	じょざんえんざんし	
\\	自立型	じりつがた	
\\	自律動作	じりつどうさ	
\\	自律コンピューティング	じりつコンピューティング	
\\	自律メッセージ	じりつメッセージ	
\\	字類条件	じるいじょうけん	
\\	字類名	じるいめい	
\\	人工環境	じんこうかんきょう	
\\	人工言語	じんこうげんご	
\\	人工現実感	じんこうげんじつかん	
\\	人工頭脳	じんこうずのう	
\\	人工知能	じんこうちのう	
\\	人工無脳	じんこうむのう	
\\	自ホスト	じホスト	
\\	推奨環境	すいしょうかんきょう	
\\	推奨動作環境	すいしょうどうさかんきょう	
\\	垂直磁気記録	すいちょくじききろく	
\\	垂直冗長検査	すいちょくじょうちょうけんさ	
\\	垂直走査周波数	すいちょくそうさしゅうはすう	
\\	垂直同期信号	すいちょくどうきしんごう	
\\	垂直方向書式送り	すいちょくほうこうしょしきおくり	
\\	垂直パリティ	すいちょくパリティ	
\\	垂直パリティチェック	すいちょくパリティチェック	
\\	推定値	すいていち	
\\	推定データ	すいていデータ	
\\	水平磁気記録	すいへいじききろく	
\\	水平情報検査	すいへいじょうほうけんさ	
\\	水平走査周波数	すいへいそうさしゅうはすう	
\\	水平方向	すいへいほうこう	
\\	水平方向奇偶検査	すいへいほうこうきぐうけんさ	
\\	水平ピッチ	すいへいピッチ	
\\	水平ローディング	すいへいローディング	
\\	推論エンジン	すいろんエンジン	
\\	数学関数	すうがくかんすう	
\\	数学的モデル化	すうがくてきモデルか	
\\	数学モデル	すうがくモデル	
\\	数式通り方式計算器	すうしきどおりほうしきけいさんき	
\\	数字位置	すうじいち	
\\	数字句	すうじく	
\\	数字項目	すうじこうもく	
\\	数字集合	すうじしゅうごう	
\\	数字定数	すうじていすう	
\\	数字表記法	すうじひょうきほう	
\\	数字利用者識別子	すうじりようしゃしきべつし	
\\	数字ORアドレス	すうじオーアドレス	
\\	数字コード	すうじコード	
\\	数字コード化集合	すうじコードかしゅうごう	
\\	数字シフト	すうじシフト	
\\	数体系	すうたいけい	
\\	数値演算コプロセッサ	すうちえんざんコプロセッサ	
\\	数値演算プロセッサ	すうちえんざんプロセッサ	
\\	数値型	すうちがた	
\\	数値形式	すうちけいしき	
\\	数値語	すうちご	
\\	数値指定文字参照	すうちしていもじさんしょう	
\\	数値的	すうちてき	
\\	数値の語	すうちのご	
\\	数値表現	すうちひょうげん	
\\	数値変数	すうちへんすう	
\\	数値キーパッド	すうちキーパッド	
\\	数値データ	すうちデータ	
\\	数表現	すうひょうげん	
\\	数理計画システム	すうりけいかくシステム	
\\	数量化理論	すうりょうかりろん	
\\	寸法記入	すんぽうきにゅう	
\\	図形記号	ずけいきごう	
\\	図形記述子	ずけいきじゅつし	
\\	図形処理	ずけいしょり	
\\	図形表示装置	ずけいひょうじそうち	
\\	図形文字	ずけいもじ	
\\	図形文字列	ずけいもじれつ	
\\	図形要素	ずけいようそ	
\\	図形エディタ・図形エディター	ずけいエディタ・ずけいエディター	
\\	図表一覧	ずひょういちらん	
\\	生映像	せいえいぞう	
\\	正規化装置座標	せいきかそうちざひょう	
\\	正規化投影座標系	せいきかとうえいざひょうけい	
\\	正規化変換	せいきかへんかん	
\\	正規形	せいきけい	
\\	正規電圧	せいきでんあつ	
\\	正規表現	せいきひょうげん	
\\	正規ユーザー	せいきユーザー	
\\	制御脚書き	せいぎょあしがき	
\\	制御頭書き	せいぎょあたまがき	
\\	制御域	せいぎょいき	
\\	制御階層	せいぎょかいそう	
\\	制御観測点	せいぎょかんそくてん	
\\	制御機能	せいぎょきのう	
\\	制御局	せいぎょきょく	
\\	制御切れ	せいぎょぎれ	
\\	制御切れレベル	せいぎょぎれレベル	
\\	制御系	せいぎょけい	
\\	制御権	せいぎょけん	
\\	制御言語	せいぎょげんご	
\\	制御集団	せいぎょしゅうだん	
\\	制御属性パッケージ	せいぎょぞくせいパッケージ	
\\	制御卓	せいぎょたく	
\\	制御端末	せいぎょたんまつ	
\\	制御点	せいぎょてん	
\\	制御流れ	せいぎょながれ	
\\	制御部拡張	せいぎょぶかくちょう	
\\	制御文字	せいぎょもじ	
\\	制御要素	せいぎょようそ	
\\	制御領域	せいぎょりょういき	
\\	制御コード	せいぎょコード	
\\	制御シーケンス	せいぎょシーケンス	
\\	制御ストリング	せいぎょストリング	
\\	制御データ項目	せいぎょデータこうもく	
\\	制御データ名	せいぎょデータめい	
\\	制御フレーム	せいぎょフレーム	
\\	制御ブロック	せいぎょブロック	
\\	制御プログラム	せいぎょプログラム	
\\	制御プロセス	せいぎょプロセス	
\\	制御プロセッサ	せいぎょプロセッサ	
\\	制御モデル	せいぎょモデル	
\\	制限指定表	せいげんしていひょう	
\\	制限なし	せいげんなし	
\\	制限なしトークン	せいげんなしトークン	
\\	制限配信	せいげんはいしん	
\\	整構造プログラミング	せいこうぞうプログラミング	
\\	整合寸法集合体	せいごうすんぽうしゅうごうたい	
\\	製作会社	せいさくがいしゃ	
\\	生産側非同期手続き	せいさんがわひどうきてつづき	
\\	静止画像	せいしがぞう	
\\	正準形式	せいじゅんけいしき	
\\	正常動作精度	せいじょうどうさせいど	
\\	整数以外	せいすういがい	
\\	整数型	せいすうがた	
\\	整数計画法	せいすうけいかくほう	
\\	整数値	せいすうち	
\\	生成規則	せいせいきそく	
\\	生成子完全集合	せいせいしかんぜんしゅうごう	
\\	生成子部分集合	せいせいしぶぶんしゅうごう	
\\	正則表現	せいそくひょうげん	
\\	製造施設	せいぞうしせつ	
\\	製造者規定信号	せいぞうしゃきていしんごう	
\\	製造責任	せいぞうせきにん	
\\	整定時間	せいていじかん	
\\	静的解析	せいてきかいせき	
\\	静的記憶装置	せいてききおくそうち	
\\	静的適合性	せいてきてきごうせい	
\\	静的適合性審査	せいてきてきごうせいしんさ	
\\	静的適合性要件	せいてきてきごうせいようけん	
\\	静的変数	せいてきへんすう	
\\	静電印字装置	せいでんいんじそうち	
\\	静電記憶装置	せいでんきおくそうち	
\\	静電気放電	せいでんきほうでん	
\\	静電気防止シート	せいでんきぼうしシート	
\\	静電気プリンタ	せいでんきプリンタ	
\\	静電作図装置	せいでんさくずそうち	
\\	静電プリンタ	せいでんプリンタ	
\\	静電プロッタ	せいでんプロッタ	
\\	性能監視	せいのうかんし	
\\	性能管理	せいのうかんり	
\\	性能評価	せいのうひょうか	
\\	性能ペナルティ	せいのうペナルティ	
\\	製表機	せいひょうき	
\\	製品化	せいひんか	
\\	製品版	せいひんはん	
\\	製品ファミリー	せいひんファミリー	
\\	正負条件	せいふじょうけん	
\\	精密分類体系	せいみつぶんるいたいけい	
\\	制約集号	せいやくしゅうごう	
\\	制約集合	せいやくしゅうごう	
\\	西暦2000年問題	せいれきにせんねんもんだい	
\\	整列併合用ファイル記述項	せいれつへいごうようファイルきじゅつこう	
\\	整列用ファイル	せいれつようファイル	
\\	正論理	せいろんり	
\\	正・写し受信者表示・正写し受信者表示	せい・うつしじゅしんしゃひょうじ・せいうつしじゅしんしゃひょうじ	
\\	世界規模	せかいきぼ	
\\	世界座標	せかいざひょう	
\\	世界座標系	せかいざひょうけい	
\\	世界情報インフラ	せかいじょうほうインフラ	
\\	赤外線インタフェース	せきがいせんインタフェース	
\\	責任追跡	せきにんついせき	
\\	積分器	せきぶんき	
\\	世代管理	せだいかんり	
\\	接触画面	せっしょくがめん	
\\	接触面積	せっしょくめんせき	
\\	設置条件	せっちじょうけん	
\\	設定	せってい	
\\	設定時	せっていじ	
\\	設定範囲	せっていはんい	
\\	設定ファイル	せっていファイル	
\\	接点状態表示信号	せってんじょうたいひょうじしんごう	
\\	接点跳動	せってんちょうどう	
\\	接点入力	せってんにゅうりょく	
\\	接点保護	せってんほご	
\\	接続関係にあるNエンティティ	せつぞくかんけいにあるエンエンティティ	
\\	接続機器	せつぞくきき	
\\	接続機構インタフェース	せつぞくきこうインタフェース	
\\	接続技術	せつぞくぎじゅつ	
\\	接続形態	せつぞくけいたい	
\\	接続識別子	せつぞくしきべつし	
\\	接続時間	せつぞくじかん	
\\	接続方式	せつぞくほうしき	
\\	接続要求	せつぞくようきゅう	
\\	接続要望	せつぞくようぼう	
\\	接続料金	せつぞくりょうきん	
\\	接続ケーブル	せつぞくケーブル	
\\	節点ベクトル	せつてんベクトル	
\\	節電機能	せつでんきのう	
\\	節の見出し	せつのみだし	
\\	節名	せつめい	
\\	遷移確率	せんいかくりつ	
\\	遷移関数	せんいかんすう	
\\	遷移図	せんいず	
\\	線間隔	せんかんかく	
\\	線形	せんけい	
\\	線形探索	せんけいたんさく	
\\	線形二元系列生成機	せんけいにげんけいれつせいせいき	
\\	線形表記法	せんけいひょうきほう	
\\	線形分類体系	せんけいぶんるいたいけい	
\\	線形方程式	せんけいほうていしき	
\\	線形補間	せんけいほかん	
\\	線形システム	せんけいシステム	
\\	線形フィルタ	せんけいフィルタ	
\\	線形リスト	せんけいリスト	
\\	選言	せんげん	
\\	宣言完結文	せんげんかんけつぶん	
\\	宣言型言語	せんげんがたげんご	
\\	宣言形言語	せんげんがたげんご	
\\	宣言型マクロ命令	せんげんけいマクロめいれい	
\\	宣言子	せんげんし	
\\	宣言済み具象構文	せんげんずみぐしょうこうぶん	
\\	宣言の有効範囲	せんげんのゆうこうはんい	
\\	宣言部分	せんげんぶぶん	
\\	宣言部分集合	せんげんぶぶんしゅうごう	
\\	宣言文	せんげんぶん	
\\	穿孔位置	せんこういち	
\\	穿孔機構	せんこうきこう	
\\	穿孔済カード	せんこうずみカード	
\\	先行制御	せんこうせいぎょ	
\\	穿孔装置・せん孔装置	せんこうそうち	
\\	穿孔通路	せんこうつうろ	
\\	穿孔部	せんこうぶ	
\\	穿孔翻訳機	せんこうほんやくき	
\\	先行読取りヘッド	せんこうよみとりヘッド	
\\	穿孔カード	せんこうカード	
\\	先行ゼロ	せんこうゼロ	
\\	穿孔テープ	せんこうテープ	
\\	先行パッド	せんこうパッド	
\\	先行ページング	せんこうページング	
\\	潜在受信者	せんざいじゅしんしゃ	
\\	線種	せんしゅ	
\\	線進行方向	せんしんこうほうこう	
\\	先進ユーザ	せんしんユーザ	
\\	線状網	せんじょうもう	
\\	線状ネットワーク	せんじょうネットワーク	
\\	線図形処理	せんずけいしょり	
\\	先祖構造体	せんぞこうぞうたい	
\\	選択解除	せんたくかいじょ	
\\	選択型	せんたくがた	
\\	選択機能	せんたくきのう	
\\	選択された実行可能試験項目群	せんたくされたじっこうかのうしけんこうもくぐん	
\\	選択抄録	せんたくしょうろく	
\\	選択信号	せんたくしんごう	
\\	選択情報量	せんたくじょうほうりょう	
\\	選択値入力装置	せんたくちにゅうりょくそうち	
\\	選択領域	せんたくりょういき	
\\	選択クラス	せんたくクラス	
\\	選択チャネル	せんたくチャネル	
\\	選択フィールド保護	せんたくフィールドほご	
\\	先頭位置	せんとういち	
\\	先頭の空白	せんとうのくうはく	
\\	線分文字発生器	せんぶんもじはっせいき	
\\	線幅倍率	せんぷくばいりつ	
\\	専門家システム	せんもんかシステム	
\\	専門辞書	せんもんじしょ	
\\	専門図書館	せんもんとしょかん	
\\	専門分類体系	せんもんぶんるいたいけい	
\\	専門シソーラス	せんもんシソーラス	
\\	専用キー	せんようキー	
\\	専用レジスタ	せんようレジスタ	
\\	戦略情報システム	せんりゃくじょうほうシステム	
\\	絶対座標	ぜったいざひょう	
\\	絶対座標指令	ぜったいざひょうしれい	
\\	絶対座標命令	ぜったいざひょうめいれい	
\\	絶対番地	ぜったいばんち	
\\	絶対トラックアドレス	ぜったいトラックアドレス	
\\	絶対パス名	ぜったいパスめい	
\\	絶対ベクトル	ぜったいベクトル	
\\	絶縁増幅器	ぜつえんぞうふくき	
\\	全域アドレス管理	ぜんいきアドレスかんり	
\\	全角文字	ぜんかくもじ	
\\	全加算器	ぜんかさんき	
\\	全稼働時間	ぜんかどうじかん	
\\	全画面	ぜんがめん	
\\	全画面表示	ぜんがめんひょうじ	
\\	前景画像	ぜんけいがぞう	
\\	全けた上げ	ぜんけたあげ	
\\	全桁上げ	ぜんけたあげ	
\\	全減算器	ぜんげんさんき	
\\	全社規模	ぜんしゃきぼ	
\\	全集合	ぜんしゅうごう	
\\	全消去機能	ぜんしょうきょきのう	
\\	前進回復	ぜんしんかいふく	
\\	全体語	ぜんたいご	
\\	全体配列	ぜんたいはいれつ	
\\	前置表記法	ぜんちひょうきほう	
\\	前置プロセッサ	ぜんちプロセッサ	
\\	全二重通信	ぜんにじゅうつうしん	
\\	全二重伝送	ぜんにじゅうでんそう	
\\	全文検索	ぜんぶんけんさく	
\\	全米科学財団	ぜんべいかがくざいだん	
\\	全米情報インフラストラクチャ	ぜんべいじょうほうインフラストラクチャ	
\\	前方一致	ぜんぽういっち	
\\	前方境界面	ぜんぽうきょうかいめん	
\\	前方参照	ぜんぽうさんしょう	
\\	前方推論	ぜんぽうすいろん	
\\	前方端	ぜんぽうたん	
\\	総当たり攻撃	そうあたりこうげき	
\\	双安定	そうあんてい	
\\	双安定回路	そうあんていかいろ	
\\	双安定トリガ回路	そうあんていトリガかいろ	
\\	双安定トリガー回路	そうあんていトリガーかいろ	
\\	総括票	そうかつひょう	
\\	層管理	そうかんり	
\\	走行の流れ	そうこうのながれ	
\\	総合図書館	そうごうとしょかん	
\\	総合サービスディジタル網	そうごうサービスディジタルもう	
\\	総合ディスク制御機構	そうごうディスクせいぎょきこう	
\\	相互互換性	そうごごかんせい	
\\	相互参照表示	そうごさんしょうひょうじ	
\\	相互情報量	そうごじょうほうりょう	
\\	相互接続	そうごせつぞく	
\\	相互接続性	そうごせつぞくせい	
\\	相互接続性試験	そうごせつぞくせいじけん	
\\	相互接続地点	そうごせつぞくちてん	
\\	相互相関	そうごそうかん	
\\	相互動作管理	そうごどうさかんり	
\\	相互動作性	そうごどうさせい	
\\	相互動作性試験	そうごどうさせいしけん	
\\	相互排除	そうごはいじょ	
\\	走査器	そうさき	
\\	走査周波数	そうさしゅうはすう	
\\	操作進行状況	そうさしんこうじょうきょう	
\\	走査状態リスト	そうさじょうたいリスト	
\\	操作除外集合	そうさじょがいしゅうごう	
\\	操作対象集合	そうさたいしょうしゅうごう	
\\	操作卓	そうさたく	
\\	操作棒	そうさぼう	
\\	操作インタフェース	そうさインタフェース	
\\	操作コード	そうさコード	
\\	操作コードトラップ	そうさコードトラップ	
\\	操作パネル	そうさパネル	
\\	総称識別子	そうしょうしきべつし	
\\	総称アドレス	そうしょうアドレス	
\\	総所有コスト	そうしょゆうコスト	
\\	送信側	そうしんがわ	
\\	送信側SS利用者	そうしんがわエスエスりようしゃ	
\\	送信側SPM	そうしんがわエスピーエム	
\\	送信側エンティティ	そうしんがわエンティティ	
\\	送信側TS利用者	そうしんがわティーエスりようしゃ	
\\	送信側トランスポートエンティティ	そうしんがわトランスポートエンティティ	
\\	送信器	そうしんき	
\\	送信者	そうしんしゃ	
\\	送信要求	そうしんようきゅう	
\\	送信ウィンドウ	そうしんウィンドウ	
\\	送信レート	そうしんレート	
\\	送受信装置	そうじゅしんそうち	
\\	相対形式プログラム	そうたいけいしきプログラム	
\\	相対座標	そうたいざひょう	
\\	相対座標指令	そうたいざひょうしれい	
\\	相対座標命令	そうたいざひょうめいれい	
\\	相対世代番号	そうたいせだいばんごう	
\\	相対的なパス名	そうたいてきなパスめい	
\\	相対変位	そうたいへんい	
\\	相対編成	そうたいへんせい	
\\	相対アドレス	そうたいアドレス	
\\	相対エントロピー	そうたいエントロピー	
\\	相対キー	そうたいキー	
\\	相対パス名	そうたいパスめい	
\\	相対ファイル	そうたいファイル	
\\	相対レコード番号	そうたいレコードばんごう	
\\	装置依存の公開文	そうちいぞんのこうかいぶん	
\\	装置空間	そうちくうかん	
\\	装置座標	そうちざひょう	
\\	装置座標系	そうちざひょうけい	
\\	装置障害	そうちしょうがい	
\\	装置状態バイト	そうちじょうたいバイト	
\\	装置制御	そうちせいぎょ	
\\	装置制御文字	そうちせいぎょもじ	
\\	装置独立	そうちどくりつ	
\\	装置独立性	そうちどくりつせい	
\\	装置のランプ	そうちのランプ	
\\	装置名	そうちめい	
\\	装置クラス	そうちクラス	
\\	装置ドライバ	そうちドライバ	
\\	双対演算	そうついえんざん	
\\	想定小数点	そうていしょうすうてん	
\\	挿入点	そうにゅうてん	
\\	総廃棄セル数	そうはいきセルすう	
\\	相変化型光ディスク	そうへんかがたひかりディスク	
\\	双方向印刷	そうほうこういんさつ	
\\	双方向性	そうほうこうせい	
\\	双方向通信	そうほうこうつうしん	
\\	双方向CATV	そうほうこうシーエーティービー・そうほうこうシーエーティーブイ	
\\	相補演算	そうほえんざん	
\\	相補型金属酸化膜半導体	そうほがたきんぞくさんかまくはんどうたい	
\\	総セル数	そうセルすう	
\\	添字指定	そえじしてい	
\\	添字付きデータ名	そえじつきデータめい	
\\	添え字付き変数	そえじづきへんすう	
\\	添字三つ組	そえじみつぐみ	
\\	即接続	そくせつぞく	
\\	速達郵便サービス	そくたつゆうびんサービス	
\\	即値命令	そくちめいれい	
\\	即値アドレス	そくちアドレス	
\\	即値データ	そくちデータ	
\\	測定方法	そくていほうほう	
\\	速度変換	そくどへんかん	
\\	疎結合	そけつごう	
\\	疎結合システム	そけつごうシステム	
\\	疎結合マルチプロセッサ	そけつごうマルチプロセッサ	
\\	外付・外付け	そとづけ	
\\	損傷ログ記録	そんしょうログきろく	
\\	増設メモリ	ぞうせつメモリ	
\\	増分座標	ぞうぶんざひょう	
\\	増分量	ぞうぶんりょう	
\\	増分割り当て	ぞうぶんわりあて	
\\	増分ベクトル	ぞうぶんベクトル	
\\	属種関係	ぞくしゅかんけい	
\\	属性化領域	ぞくせいかりょういき	
\\	属性型	ぞくせいがた	
\\	属性型JPドメイン名	ぞくせいがたジェーピードメインめい	
\\	属性指定	ぞくせいしてい	
\\	属性指定並び	ぞくせいしていならび	
\\	属性値選択節	ぞくせいちせんたくせつ	
\\	属性値表現	ぞくせいちひょうげん	
\\	属性値アサーション	ぞくせいちアサーション	
\\	属性定義	ぞくせいていぎ	
\\	属性定義並び	ぞくせいていぎならび	
\\	属性定義並び宣言	ぞくせいていぎならびせんげん	
\\	属性並び	ぞくせいならび	
\\	属性並び宣言	ぞくせいならびせんげん	
\\	属性表	ぞくせいひょう	
\\	属性要素	ぞくせいようそ	
\\	属性データ	ぞくせいデータ	
\\	帯域圧縮	たいいきあっしゅく	
\\	帯域確保	たいいきかくほ	
\\	帯域制御	たいいきせいぎょ	
\\	帯域制限	たいいきせいげん	
\\	帯域制限信号	たいいきせいげんしんごう	
\\	帯域通過フィルタ	たいいきつうかフィルタ	
\\	大域的	たいいきてき	
\\	帯域幅利用	たいいきはばりよう	
\\	大域変数	たいいきへんすう	
\\	大域名	たいいきめい	
\\	帯域予約	たいいきよやく	
\\	対応	たいおう	
\\	対応付け索引方式	たいおうつけさくいんほうしき	
\\	対応版	たいおうばん	
\\	対応プロトコル	たいおうプロトコル	
\\	対角行列	たいかくぎょうれつ	
\\	対抗商品	たいこうしょうひん	
\\	対向循環	たいこうじゅんかん	
\\	退行テスト	たいこうテスト	
\\	対称型マルチプロセッサ	たいしょうがたマルチプロセッサ	
\\	対称型マルチプロセッシング	たいしょうがたマルチプロセッシング	
\\	対象機種	たいしょうきしゅ	
\\	対象受信者	たいしょうじゅしんしゃ	
\\	対象体記述部	たいしょうたいきじゅつぶ	
\\	対象体種別	たいしょうたいしゅべつ	
\\	対象体容量	たいしょうたいようりょう	
\\	対象体クラス	たいしょうたいクラス	
\\	対象体クラス記述部	たいしょうたいクラスきじゅつぶ	
\\	対称リスト	たいしょうリスト	
\\	対数的	たいすうてき	
\\	対数の指数	たいすうのしすう	
\\	退避利用者ID	たいひりようしゃアイディー	
\\	退避グループID	たいひグループアイディー	
\\	対話型プログラム	たいわかたプログラム	
\\	対話型	たいわがた	
\\	対話形	たいわがた	
\\	対話処理	たいわしょり	
\\	対話的	たいわてき	
\\	対話方式	たいわほうしき	
\\	耐タンパー性	たいタンパーせい	
\\	高いレベル	たかいレベル	
\\	多機能	たきのう	
\\	多義語	たぎご	
\\	卓上型コンピュータ	たくじょうがたコンピュータ	
\\	卓上計算器	たくじょうけいさんき	
\\	卓上出版	たくじょうしゅっぱん	
\\	多言語対応	たげんごたいおう	
\\	多言語サポート	たげんごサポート	
\\	多言語シソーラス	たげんごシソーラス	
\\	多元接続	たげんせつぞく	
\\	多元接続方式	たげんせつぞくほうしき	
\\	多元分類体系	たげんぶんるいたいけい	
\\	多項回帰	たこうかいき	
\\	多次元配列	たじげんはいれつ	
\\	多重階層	たじゅうかいそう	
\\	多重仮想記憶方式	たじゅうかそうきおくほうしき	
\\	多重化装置	たじゅうかそうち	
\\	多重仮想アドレス空間	たじゅうかそうアドレスくうかん	
\\	多重索引順編成法	たじゅうさくいんじゅんへんせいほう	
\\	多重処理	たじゅうしょり	
\\	多重適合決定表	たじゅうてきごうけっていひょう	
\\	多重伝送装置	たじゅうでんそうそうち	
\\	多重タスキング	たじゅうタスキング	
\\	多重タスク処理	たじゅうタスクしょり	
\\	多重チャネル	たじゅうチャネル	
\\	多重ネットワーク	たじゅうネットワーク	
\\	多重プログラミング	たじゅうプログラミング	
\\	多重プロセッサ	たじゅうプロセッサ	
\\	多重プロセッシング	たじゅうプロセッシング	
\\	多数決演算	たすうけつえんざん	
\\	多数決素子	たすうけつそし	
\\	多数決ゲート	たすうけつゲート	
\\	多態性	たたいせい	
\\	畳み込み定理	たたみこみていり	
\\	畳み込み符号	たたみこみふごう	
\\	立上がり時間	たちあがりじかん	
\\	多値属性	たちぞくせい	
\\	達成のレベル	たっせいのレベル	
\\	縦視野移動	たてしやいどう	
\\	多倍精度	たばいせいど	
\\	束指標	たばしひょう	
\\	束表	たばひょう	
\\	束表項目	たばひょうこうもく	
\\	多品種中少量生産システム	たひんしゅちゅうしょうりょうせいさんシステム	
\\	多符号核具象構文	たふごうかくぐしょうこうぶん	
\\	多符号基本具象構文	たふごうきほんぐしょうこうぶん	
\\	多符号具象構文	たふごうぐしょうこうぶん	
\\	多方向展開	たほうこうてんかい	
\\	単安定回路	たんあんていかいろ	
\\	単安定トリガ回路	たんあんていトリガかいろ	
\\	単一型集合型	たんいつがたしゅうごうがた	
\\	単一型順序列型	たんいつがたじゅんじょれつがた	
\\	単一言語シソーラス	たんいつげんごシソーラス	
\\	単一障害点	たんいつしょうがいてん	
\\	単一接続局	たんいつせつぞくきょく	
\\	単一接続コンセントレータ	たんいつせつぞくコンセントレータ	
\\	単一値属性	たんいつちぞくせい	
\\	単一表記法	たんいつひょうきほう	
\\	単一命令操作	たんいつめいれいそうさ	
\\	単位倍率	たんいばいりつ	
\\	単位料金区域	たんいりょうきんくいき	
\\	単位列	たんいれつ	
\\	単項演算	たんこうえんざん	
\\	単項演算子	たんこうえんざんし	
\\	単行出版物	たんこうしゅっぱんぶつ	
\\	単語カウント	たんごカウント	
\\	探索かぎ	たんさくかぎ	
\\	探索サイクル	たんさくサイクル	
\\	単紙	たんし	
\\	短縮参照区切り子機能	たんしゅくさんしょうくぎりしきのう	
\\	短縮参照集合	たんしゅくさんしょうしゅうごう	
\\	短縮参照使用宣言	たんしゅくさんしょうしようせんげん	
\\	短縮参照対応表宣言	たんしゅくさんしょうたいおうひょうせんげん	
\\	短縮参照列	たんしゅくさんしょうれつ	
\\	短縮アドレス呼出し	たんしゅくアドレスよびだし	
\\	短縮ダイヤル・短縮ダイアル	たんしゅくダイヤル・たんしゅくダイアル	
\\	単純階層	たんじゅんかいそう	
\\	単純緩衝法	たんじゅんかんしょうほう	
\\	単純型	たんじゅんがた	
\\	単純実行文	たんじゅんじっこうぶん	
\\	単純条件	たんじゅんじょうけん	
\\	単純連結処理定義	たんじゅんれんけつしょりていぎ	
\\	単純コマンド	たんじゅんコマンド	
\\	単純マトリックス方式	たんじゅんマトリックスほうしき	
\\	単精度浮動小数点数	たんせいどふどうしょうすうてんすう	
\\	単線式回線	たんせんしきかいせん	
\\	単層埋込み試験	たんそううめこみしけん	
\\	単体試験	たんたいしけん	
\\	単適合決定表	たんてきごうけっていひょう	
\\	端点	たんてん	
\\	探番	たんばん	
\\	単票	たんぴょう	
\\	単方向	たんほうこう	
\\	単方向通信	たんほうこうつうしん	
\\	単方向伝送	たんほうこうでんそう	
\\	端末	たんまつ	
\\	端末画面	たんまつがめん	
\\	端末識別子	たんまつしきべつし	
\\	端末種別	たんまつしゅべつ	
\\	端末装置	たんまつそうち	
\\	端末入力	たんまつにゅうりょく	
\\	端末エミュレーション	たんまつエミュレーション	
\\	端末エミュレータ	たんまつエミュレータ	
\\	端末ORアドレス	たんまつオーアドレス	
\\	端末サーバー	たんまつサーバー	
\\	単密度	たんみつど	
\\	多オクテット	たオクテット	
\\	多バイト	たバイト	
\\	多バイト文字	たバイトもじ	
\\	大域要素	だいいきようそ	
\\	大域モデリング変換	だいいきモデリングへんかん	
\\	第一レベルドメイン	だいいちレベルドメイン	
\\	第一種電気通信事業者	だいいっしゅでんきつうしんじぎょうしゃ	
\\	大改訂	だいかいてい	
\\	大記憶	だいきおく	
\\	大記憶管理システム	だいきおくかんりシステム	
\\	大記憶ファイル	だいきおくファイル	
\\	代行受信者	だいこうじゅしんしゃ	
\\	代行受信者許可	だいこうじゅしんしゃきょか	
\\	代行受信者登録	だいこうじゅしんしゃとうろく	
\\	代行送信	だいこうそうしん	
\\	第5世代コンピュータ	だいごせだいコンピュータ	
\\	第5世代コンピューター	だいごせだいコンピューター	
\\	大小順序	だいしょうじゅんじょ	
\\	代数演算子	だいすうえんざんし	
\\	代数理論	だいすうりろん	
\\	代替クラス・代替えクラス	だいたいクラス・だいがえクラス	
\\	代替トラック	だいたいトラック	
\\	代替パス・代替えパス	だいたいパス・だいがえパス	
\\	大なり	だいなり・ダイナリ	
\\	第二種電気通信事業者	だいにしゅでんきつうしんじぎょうしゃ	
\\	代入演算子	だいにゅうえんざんし	
\\	代入文	だいにゅうぶん	
\\	大容量	だいようりょう	
\\	大容量記憶装置	だいようりょうきおくそうち	
\\	大容量記憶システム	だいようりょうきおくシステム	
\\	大容量データストレージ	だいようりょうデータストレージ	
\\	第4世代言語	だいよんせだいげんご	
\\	代理受信者	だいりじゅしんしゃ	
\\	代理サーバー	だいりサーバー	
\\	抱き合せで販売	だきあわせではんばい	
\\	打けん検証・打鍵検証	だけんけんしょう	
\\	出し入れ可能媒体	だしいれかのうばいたい	
\\	打診発生源認証	だしんはっせいげんにんしょう	
\\	妥当性確認試験	だとうせいかくにんしけん	
\\	段階的詳細化	だんかいてきしょうさいか	
\\	段組み	だんぐみ	
\\	団体著者	だんたいちょしゃ	
\\	断片化	だんぺんか	
\\	断面プロット	だんめんプロット	
\\	段落の見出し	だんらくのみだし	
\\	段落名	だんらくめい	
\\	段シフト	だんシフト	
\\	地域型JPドメイン名	ちいきかたジェーピードメインめい	
\\	地域競争	ちいききょうそう	
\\	地域電話会社	ちいきでんわがいしゃ	
\\	地域電話サービス	ちいきでんわサービス	
\\	地域郵便属性	ちいきゆうびんぞくせい	
\\	地域コード	ちいきコード	
\\	地域ベル電話会社	ちいきベルでんわがいしゃ	
\\	地域ベルオペレーティング会社	ちいきベルオペレーティングがいしゃ	
\\	遅延回路	ちえんかいろ	
\\	遅延更新	ちえんこうしん	
\\	遅延処理	ちえんしょり	
\\	遅延線	ちえんせん	
\\	遅延素子	ちえんそし	
\\	遅延配信	ちえんはいしん	
\\	遅延配信取消し	ちえんはいしんとりけし	
\\	遅延歪み	ちえんひずみ	
\\	遅延広がり	ちえんひろがり	
\\	遅延変動	ちえんへんどう	
\\	遅延変動センシティブ	ちえんへんどうセンシティブ	
\\	遅延揺らぎ	ちえんゆらぎ	
\\	遅延センシティブ	ちえんセンシティブ	
\\	遅延モード	ちえんモード	
\\	置換可能引数データ	ちかんかのうひきすうデータ	
\\	置換可能文字データ	ちかんかのうもじデータ	
\\	置換文	ちかんぶん	
\\	置換文字	ちかんもじ	
\\	置換リスト	ちかんリスト	
\\	地球局	ちきゅうきょく	
\\	逐一命令操作	ちくいちめいれいそうさ	
\\	逐次印字装置	ちくじいんじそうち	
\\	逐次刊行物	ちくじかんこうぶつ	
\\	逐次刊行物の受入れ	ちくじかんこうぶつのうけいれ	
\\	逐次実行	ちくじじっこう	
\\	知識木	ちしきぎ	
\\	知識参照	ちしきさんしょう	
\\	知識情報	ちしきじょうほう	
\\	知識情報処理	ちしきじょうほうしょり	
\\	知識情報処理システム	ちしきじょうほうしょりシステム	
\\	知識ベース	ちしきベース	
\\	地上波	ちじょうは	
\\	置数消去機能	ちすうしょうきょきのう	
\\	地図帳	ちずちょう	
\\	知的端末	ちてきたんまつ	
\\	地点遠隔地	ちてんえんかくち	
\\	致命的誤り	ちめいてきあやまり	
\\	致命的エラー	ちめいてきエラー	
\\	着信者	ちゃくしんしゃ	
\\	着信転送機能	ちゃくしんてんそうきのう	
\\	着脱可能	ちゃくだつかのう	
\\	着加入者	ちゃっかにゅうしゃ	
\\	着呼	ちゃっこ	
\\	着呼受付信号	ちゃっこうけつけしんごう	
\\	着呼受付不可信号	ちゃっこうけつけふかしんごう	
\\	着呼側SS利用者	ちゃっこがわエスエスりようしゃ	
\\	着呼側Nアドレス	ちゃっこがわエンアドレス	
\\	着呼側NS利用者	ちゃっこがわエンエスりようしゃ	
\\	着呼側TS利用者	ちゃっこがわティーエスりようしゃ	
\\	中央演算処理装置	ちゅうおうえんざんしょりそうち	
\\	中央処理装置	ちゅうおうしょりそうち	
\\	中央そろえ	ちゅうおうそろえ	
\\	中央揃え	ちゅうおうぞろえ	
\\	中核文書	ちゅうかくぶんしょ	
\\	中間言語	ちゅうかんげんご	
\\	中間言語レベル	ちゅうかんげんごレベル	
\\	中間者攻撃	ちゅうかんしゃこうげき	
\\	中間節点	ちゅうかんせってん	
\\	中間装置	ちゅうかんそうち	
\\	中間速度	ちゅうかんそくど	
\\	中間配線盤	ちゅうかんはいせんばん	
\\	中間文字	ちゅうかんもじ	
\\	中間コード	ちゅうかんコード	
\\	中間システム	ちゅうかんシステム	
\\	中間ノード	ちゅうかんノード	
\\	中間バイト	ちゅうかんバイト	
\\	注記行	ちゅうきぎょう	
\\	注記項	ちゅうきこう	
\\	中軌道	ちゅうきどう	
\\	注記様式	ちゅうきようしき	
\\	注記テキスト	ちゅうきテキスト	
\\	中空内部様式	ちゅうくうないぶようしき	
\\	中継器	ちゅうけいき	
\\	中継交換機	ちゅうけいこうかんき	
\\	中継線交換機	ちゅうけいせんこうかんき	
\\	中継装置	ちゅうけいそうち	
\\	注釈宣言	ちゅうしゃくせんげん	
\\	抽出規準	ちゅうしゅつきじゅん	
\\	抽出条件	ちゅうしゅつじょうけん	
\\	抽出モード	ちゅうしゅつモード	
\\	抽象構文	ちゅうしょうこうぶん	
\\	抽象構文名	ちゅうしょうこうぶんめい	
\\	抽象的試験項目	ちゅうしょうてきしけんこうもく	
\\	抽象的試験方法論	ちゅうしょうてきしけんほうほうろん	
\\	抽象的Nサービスプリミティブ	ちゅうしょうてきエンサービスプリミティブ	
\\	抽象アソシエーション	ちゅうしょうアソシエーション	
\\	抽象データ型	ちゅうしょうデータがた	
\\	中心波長	ちゅうしんはちょう	
\\	中断点	ちゅうだんてん	
\\	中置表記法	ちゅうちひょうきほう	
\\	超過情報速度	ちょうかじょうほうそくど	
\\	超過送信	ちょうかそうしん	
\\	超過送信制御	ちょうかそうしんせいぎょ	
\\	長距離会社	ちょうきょりがいしゃ	
\\	長距離電話事業	ちょうきょりでんわじぎょう	
\\	長距離キャリア	ちょうきょりキャリア	
\\	長弧	ちょうこ	
\\	超高速	ちょうこうそく	
\\	超高速並列インターフェース	ちょうこうそくへいれつインターフェース	
\\	超広帯域無線	ちょうこうたいいきむせん	
\\	超小型	ちょうこがた	
\\	調査期間	ちょうさきかん	
\\	調整水準	ちょうせいすいじゅん	
\\	長精度浮動小数点数	ちょうせいどふどうしょうすうてんすう	
\\	超大規模集積回路	ちょうだいきぼしゅうせきかいろ	
\\	超大容量	ちょうだいようりょう	
\\	超々大規模集積回路・超超大規模集積回路	ちょうちょうだいきぼしゅうせきかいろ	
\\	超低電圧版モバイル	ちょうていでんあつばんモバイル	
\\	頂点色	ちょうてんしょく	
\\	頂点法線	ちょうてんほうせん	
\\	頂点シェーダ	ちょうてんシェーダ	
\\	頂点データ	ちょうてんデータ	
\\	超並列計算機	ちょうへいれつけいさんき	
\\	調歩式伝送	ちょうほしきでんそう	
\\	調歩同期	ちょうほどうき	
\\	調歩同期式	ちょうほどうきしき	
\\	調歩同期アダプタ	ちょうほどうきアダプタ	
\\	直接色指定	ちょくせついろしてい	
\\	直接記憶アクセス	ちょくせつきおくアクセス	
\\	直接入力	ちょくせつにゅうりょく	
\\	直接発信	ちょくせつはっしん	
\\	直接百分率機能	ちょくせつひゃくぶんりつきのう	
\\	直接編成ファイル	ちょくせつへんせいファイル	
\\	直接呼出し機能	ちょくせつよびだしきのう	
\\	直接利用者	ちょくせつりようしゃ	
\\	直接アクセス	ちょくせつアクセス	
\\	直接アクセス記憶装置	ちょくせつアクセスきおくそうち	
\\	直接アドレス	ちょくせつアドレス	
\\	直接アドレスファイル	ちょくせつアドレスファイル	
\\	直接メモリアクセス	ちょくせつメモリアクセス	
\\	直前指定要素	ちょくぜんしていようそ	
\\	直定数	ちょくていすう	
\\	直並列変換器	ちょくへいれつへんかんき	
\\	直列加算	ちょくれつかさん	
\\	直列加算器	ちょくれつかさんき	
\\	直列伝送	ちょくれつでんそう	
\\	直列データ回線	ちょくれつデータかいせん	
\\	直リン	ちょくリン・じかリン	
\\	著作権登録図書館	ちょさくけんとうろくとしょかん	
\\	著作権図書館	ちょさくけんとしょかん	
\\	著作集	ちょさくしゅう	
\\	直帰率	ちょっきりつ	
\\	直結形トランジスタ論理回路	ちょっけつけいトランジスタろんりかいろ	
\\	直交系列	ちょっこうけいれつ	
\\	直交周波数多重	ちょっこうしゅうはすうたじゅう	
\\	直交周波数多重変調	ちょっこうしゅうはすうたじゅうへんちょう	
\\	直交振幅変調	ちょっこうしんぷくへんちょう	
\\	ちらつき防止フィルタ	ちらつきぼうしフィルタ	
\\	地理的	ちりてき	
\\	追加記号	ついかきごう	
\\	追加機能	ついかきのう	
\\	追加レコード	ついかレコード	
\\	追記型光ディスク	ついきがたひかりディスク	
\\	追従保持要素	ついじゅうほじようそ	
\\	追跡記号	ついせききごう	
\\	追跡プログラム	ついせきプログラム	
\\	通貨記号	つうかきごう	
\\	通過判定	つうかはんてい	
\\	通貨編集用文字	つうかへんしゅうようもじ	
\\	通信相手	つうしんあいて	
\\	通信回線	つうしんかいせん	
\\	通信管理システム	つうしんかんりシステム	
\\	通信規格	つうしんきかく	
\\	通信機器	つうしんきき	
\\	通信記述項	つうしんきじゅつこう	
\\	通信記述名	つうしんきじゅつめい	
\\	通信規約	つうしんきやく	
\\	通信行	つうしんぎょう	
\\	通信事業者	つうしんじぎょうしゃ	
\\	通信条件	つうしんじょうけん	
\\	通信制御機構	つうしんせいぎょきこう	
\\	通信制御処理装置	つうしんせいぎょしょりそうち	
\\	通信制御装置	つうしんせいぎょそうち	
\\	通信節	つうしんせつ	
\\	通信設備	つうしんせつび	
\\	通信端末	つうしんたんまつ	
\\	通信手順	つうしんてじゅん	
\\	通信品位法	つうしんひんいほう	
\\	通信品質	つうしんひんしつ	
\\	通信負荷	つうしんふか	
\\	通信文個数	つうしんぶんこすう	
\\	通信方式	つうしんほうしき	
\\	通信量	つうしんりょう	
\\	通信料金	つうしんりょうきん	
\\	通信理論	つうしんりろん	
\\	通信路	つうしんろ	
\\	通信路容量	つうしんろようりょう	
\\	通信アクセス方式	つうしんアクセスほうしき	
\\	通信カラオケ	つうしんカラオケ	
\\	通信キャリア	つうしんキャリア	
\\	通信システム	つうしんシステム	
\\	通信ソフト	つうしんソフト	
\\	通信ソフトウェア	つうしんソフトウェア	
\\	通信チャネル	つうしんチャネル	
\\	通信プロトコル	つうしんプロトコル	
\\	通信ポート	つうしんポート	
\\	通常表示	つうじょうひょうじ	
\\	通常郵便	つうじょうゆうびん	
\\	通常ファイル	つうじょうファイル	
\\	通常モード	つうじょうモード	
\\	通知型	つうちがた	
\\	通知付き局留め	つうちつききょくどめ	
\\	通報	つうほう	
\\	通報受端	つうほうじゅたん	
\\	通話接続	つうわせつぞく	
\\	通話料金制	つうわりょうきんせい	
\\	使い捨て装置	つかいすてそうち	
\\	使い捨てパスワード	つかいすてパスワード	
\\	付き分岐	つきぶんき	
\\	次の実行完結文	つぎのじっこうかんけつぶん	
\\	次の実行文	つぎのじっこうぶん	
\\	次のレコード	つぎのレコード	
\\	次ホップ	つぎホップ	
\\	繋ぎ	つなぎ	
\\	摘み	つまみ	
\\	提案クラス	ていあんクラス	
\\	低域通過フィルタ	ていいきつうかフィルタ	
\\	低解像度	ていかいぞうど	
\\	定期業務	ていきぎょうむ	
\\	定期保守	ていきほしゅ	
\\	定期保全	ていきほぜん	
\\	低級言語	ていきゅうげんご	
\\	提供者責任分散型トランザクション	ていきょうしゃせきにんぶんさんがたトランザクション	
\\	定義可能	ていぎかのう	
\\	定義済み変数	ていぎすみへんすう	
\\	定義済み	ていぎずみ	
\\	定義済み関数	ていぎずみかんすう	
\\	定義文字実体集合	ていぎもじじったいしゅうごう	
\\	定形郵便ORアドレス	ていけいゆうびんオーアーアドレス	
\\	停止問題	ていしもんだい	
\\	低周波発振器	ていしゅうははっしんき	
\\	低消費電力	ていしょうひでんりょく	
\\	低消費電力状態	ていしょうひでんりょくじょうたい	
\\	停止シグナル	ていしシグナル	
\\	定常情報源	ていじょうじょうほうげん	
\\	定常性	ていじょうせい	
\\	低水準言語	ていすいじゅんげんご	
\\	定数	ていすう	
\\	定数機能	ていすうきのう	
\\	定数式	ていすうしき	
\\	定数表現	ていすうひょうげん	
\\	低速通信	ていそくつうしん	
\\	低損失	ていそんしつ	
\\	低遅延	ていちえん	
\\	定電圧定周波数電源装置	ていでんあつていしゅうはすうでんげんそうち	
\\	低電圧版Xeon	ていでんあつばんジーオン	
\\	低電圧版モバイル	ていでんあつばんモバイル	
\\	低電力	ていでんりょく	
\\	低密度ディスケット	ていみつどディスケット	
\\	低容量	ていようりょう	
\\	低コスト	ていコスト	
\\	低ビットレート	ていビットレート	
\\	低プライオリティ	ていプライオリティ	
\\	低レベル	ていレベル	
\\	手書き認識	てがきにんしき	
\\	手書き文字	てがきもじ	
\\	手書き文字認識	てがきもじにんしき	
\\	適応制御系	てきおうせいぎょけい	
\\	適応的差分パルス符号変調	てきおうてきさぶんパルスふごうへんちょう	
\\	適応変換符号化	てきおうへんかんふごうか	
\\	適応デルタ変調	てきおうデルタへんちょう	
\\	適格である	てきかくである	
\\	適合検査	てきごうけんさ	
\\	適合検査試験	てきごうけんさしけん	
\\	適合する実装	てきごうするじっそう	
\\	適合性解決試験	てきごうせいかいけつしけん	
\\	適合性試験	てきごうせいしけん	
\\	適合性評価過程	てきごうせいひょうかかてい	
\\	適合性文書	てきごうせいぶんしょ	
\\	適合性ログ	てきごうせいログ	
\\	適合SGML応用	てきごうエスジーエムエルおうよう	
\\	適合SGML文書	てきごうエスジーエムエルぶんしょ	
\\	適正表示状態	てきせいひょうじじょうたい	
\\	適用業務責任分散型トランザクション	てきようぎょうむせきにんぶんさんがたトランザクション	
\\	適用業務問題	てきようぎょうむもんだい	
\\	適用業務ソフトウェア	てきようぎょうむソフトウェア	
\\	適用業務ソフトウェアプログラム	てきようぎょうむソフトウェアプログラム	
\\	適用業務プログラム	てきようぎょうむプログラム	
\\	適用範囲	てきようはんい	
\\	適用分野	てきようぶんや	
\\	手順終了	てじゅんしゅうりょう	
\\	手順書	てじゅんしょ	
\\	手順名	てじゅんめい	
\\	手続引用仕様	てつづきいんようしよう	
\\	手続き型言語	てつづきがたげんご	
\\	手続き形言語	てつづきがたげんご	
\\	手続き集合	てつづきしゅうごう	
\\	手続き宣言	てつづきせんげん	
\\	手続きの入口	てつづきのいりぐち	
\\	手続き部の終わり	てつづきぶのおわり	
\\	手続き分岐文	てつづきぶんきぶん	
\\	手続き向き言語	てつづきむきげんご	
\\	手続き名	てつづきめい	
\\	手続き呼び出し・手続き呼出	てつづきよびだし	
\\	展開結果	てんかいけっか	
\\	添加部分要素	てんかぶぶんようそ	
\\	添加要素	てんかようそ	
\\	転送構文	てんそうこうぶん	
\\	転送構文名	てんそうこうぶんめい	
\\	転送処理	てんそうしょり	
\\	転送時間	てんそうじかん	
\\	転送穿孔翻訳機	てんそうせんこうほんやくき	
\\	転送速度	てんそうそくど	
\\	転送システム	てんそうシステム	
\\	転送ルート	てんそうルート	
\\	転送レート	てんそうレート	
\\	伝達コスト	てんたつコスト	
\\	転置インデックス	てんちインデックス	
\\	添付書類	てんぷしょるい	
\\	添付ファイル	てんぷファイル	
\\	点列入力装置	てんれつにゅうりょくそうち	
\\	点オクテット	てんオクテット	
\\	出口点	でぐちてん	
\\	電界効果トランジスター・電界効果トランジスタ	でんかいこうかトランジスター・でんかいこうかトランジスタ	
\\	電荷結合素子	でんかけつごうそし	
\\	電気回路ブレーカー	でんきかいろブレーカー	
\\	電源回路	でんげんかいろ	
\\	電源管理	でんげんかんり	
\\	電源切断状態	でんげんせつだんじょうたい	
\\	電源投入	でんげんとうにゅう	
\\	電源投入時パスワード	でんげんとうにゅうじパスワード	
\\	電源投入シーケンス	でんげんとうにゅうシーケンス	
\\	電源コード	でんげんコード	
\\	電源スイッチ	でんげんスイッチ	
\\	電源ユニット	でんげんユニット	
\\	電源ランプ	でんげんランプ	
\\	電算	でんさん	
\\	電算機	でんさんき	
\\	電子会議	でんしかいぎ	
\\	電子計算機	でんしけいさんき	
\\	電子掲示板	でんしけいじばん	
\\	電子雑誌	でんしざっし	
\\	電子式	でんししき	
\\	電子資金移動	でんししきんいどう	
\\	電子資金振替	でんししきんふりかえ	
\\	電子紙幣	でんししへい	
\\	電子署名	でんししょめい	
\\	電子新聞	でんししんぶん	
\\	電子透かし	でんしすかし	
\\	電子的	でんしてき	
\\	電子的資金移動	でんしてきしきんいどう	
\\	電子手帳	でんしてちょう	
\\	電子図書館	でんしとしょかん	
\\	電子番組ガイド	でんしばんぐみガイド	
\\	電子振替決済	でんしふりかえけっさい	
\\	電信電話会社	でんしんでんわがいしゃ	
\\	電子データ交換	でんしデータこうかん	
\\	電子データ処理	でんしデータしょり	
\\	電子ビジネス	でんしビジネス	
\\	電子ファイリング	でんしファイリング	
\\	電子ファイリングシステム	でんしファイリングシステム	
\\	電子ブック	でんしブック	
\\	電子マネー	でんしマネー	
\\	電子メール	でんしメール	
\\	電子メールアドレス	でんしメールアドレス	
\\	電子メールシステム	でんしメールシステム	
\\	電子メールボックス	でんしメールボックス	
\\	電磁界	でんじかい	
\\	電磁波互換性	でんじはごかんせい	
\\	電磁妨害	でんじぼうがい	
\\	伝送経路遅延	でんそうけいろちえん	
\\	伝送故障	でんそうこしょう	
\\	伝送制御文字	でんそうせいぎょもじ	
\\	伝送制御プロトコル	でんそうせいぎょプロトコル	
\\	伝送速度	でんそうそくど	
\\	伝送損失	でんそうそんしつ	
\\	伝送遅延	でんそうちえん	
\\	伝送遅延変動	でんそうちえんへんどう	
\\	伝送品質	でんそうひんしつ	
\\	伝送路長	でんそうろちょう	
\\	伝送スピード	でんそうスピード	
\\	伝送パス	でんそうパス	
\\	伝達情報量	でんたつじょうほうりょう	
\\	電池式計算器	でんちしきけいさんき	
\\	電脳	でんのう	
\\	伝播遅延	でんぱちえん	
\\	伝播側	でんぱんそく	
\\	伝搬遅延	でんぱんちえん	
\\	伝票発行処理	でんぴょうはっこうしょり	
\\	電流一致選択	でんりゅういっちせんたく	
\\	電力切れ	でんりょくぎれ	
\\	電力制限	でんりょくせいげん	
\\	電力束密度	でんりょくそくみつど	
\\	電話応答機能	でんわおうとうきのう	
\\	電話機器	でんわきき	
\\	電話ワイヤ	でんわワイヤ	
\\	問合せ関数	といあわせかんすう	
\\	問合せ機能	といあわせきのう	
\\	問合せ言語	といあわせげんご	
\\	問合せ端末	といあわせたんまつ	
\\	統一化	とういつか	
\\	統一スタンダード	とういつスタンダード	
\\	投影基準点	とうえいきじゅんてん	
\\	投影線	とうえいせん	
\\	投影方法	とうえいほうほう	
\\	投影ビューポート	とうえいビューポート	
\\	等化	とうか	
\\	等価演算	とうかえんざん	
\\	透過画像	とうかがぞう	
\\	等化器	とうかき	
\\	等角速度	とうかくそくど	
\\	透過原稿ユニット	とうかげんこうユニット	
\\	等価語	とうかご	
\\	等価参照列	とうかさんしょうれつ	
\\	等価素子	とうかそし	
\\	統括名	とうかつめい	
\\	透過的	とうかてき	
\\	透過的データ	とうかてきデータ	
\\	透過なデータ	とうかなデータ	
\\	頭切れ	とうきれ	
\\	統計処理	とうけいしょり	
\\	統計情報	とうけいじょうほう	
\\	統計多重処理	とうけいたじゅうしょり	
\\	凍結レファレンス	とうけつレファレンス	
\\	等高線図	とうこうせんず	
\\	等高線グラフ	とうこうせんグラフ	
\\	統合開発環境	とうごうかいはつかんきょう	
\\	統合開放形ハイパメディア	とうごうかいほうがたハイパメディア	
\\	統合漢字	とうごうかんじ	
\\	統合試験	とうごうしけん	
\\	統合ソフトウェア	とうごうソフトウェア	
\\	統合デジタル通信網	とうごうデジタルつうしんもう	
\\	統合プログラム	とうごうプログラム	
\\	透視投影	とうしとうえい	
\\	透視変換	とうしへんかん	
\\	投資利益率	とうしりえきりつ	
\\	等時	とうじ	
\\	等時性	とうじせい	
\\	等時性通信	とうじせいつうしん	
\\	等時性伝送	とうじせいでんそう	
\\	到達性	とうたつせい	
\\	到着プロセス	とうちゃくプロセス	
\\	等方写像	とうほうしゃぞう	
\\	頭末そろえ	とうまつそろえ	
\\	登録識別子	とうろくしきべつし	
\\	登録集原文	とうろくしゅうげんぶん	
\\	登録集名	とうろくしゅうめい	
\\	登録所有者識別子	とうろくしょゆうしゃしきべつし	
\\	登録済みアクセス	とうろくずみアクセス	
\\	特異データ	とくいデータ	
\\	特殊配達	とくしゅはいたつ	
\\	特殊名	とくしゅめい	
\\	特殊名記述項	とくしゅめいきじゅつこう	
\\	特殊文字	とくしゅもじ	
\\	特殊文字語	とくしゅもじご	
\\	特殊アクセス	とくしゅアクセス	
\\	特殊レジスタ	とくしゅレジスタ	
\\	特注ソフトウェア	とくちゅうソフトウェア	
\\	特定記号	とくていきごう	
\\	特定業務向き言語	とくていぎょうむむきげんご	
\\	特定子	とくていし	
\\	特定数字	とくていすうじ	
\\	特定の処理	とくていのしょり	
\\	特定文字データ実体	とくていもじデータじったい	
\\	特定用途向け集積回路	とくていようとむけしゅうせきかいろ	
\\	特定用途向けIC	とくていようとむけアイシー	
\\	特定論理構造	とくていろんりこうぞう	
\\	特定割付け構造	とくていわりつけこうぞう	
\\	特定マシン向け	とくていマシンむけ	
\\	匿名FTP	とくめいエフティーピー	
\\	図書館システム	としょかんシステム	
\\	図書館ネットワーク	としょかんネットワーク	
\\	閉じたシステム	とじたシステム	
\\	特権属性証明	とっけんぞくせいしょうめい	
\\	特権命令	とっけんめいれい	
\\	特権グループ	とっけんグループ	
\\	特権ユーザ	とっけんユーザ	
\\	凸最適化	とつさいてきか	
\\	飛び越し・飛越し	とびこし	
\\	飛越し命令	とびこしめいれい	
\\	取り込み	とりこみ	
\\	取り出し制限	とりだしせいげん	
\\	取り出し抽象操作	とりだしちゅうしょうそうさ	
\\	取り出しボタン	とりだしボタン	
\\	取り外し可能	とりはずしかのう	
\\	取り外し可能記憶装置	とりはずしかのうきおくそうち	
\\	同一環境	どういちかんきょう	
\\	同一分布	どういつぶんぷ	
\\	同位列	どういれつ	
\\	同位エンティティ	どういエンティティ	
\\	同位エンティティに割り当てられるウィンドウ上限	どういエンティティにわりあてられるウィンドウじょうげん	
\\	同位エンティティ認証	どういエンティティにんしょう	
\\	同音異義性	どうおんいぎせい	
\\	同格関係	どうかくかんけい	
\\	動画共有サービス	どうがきょうゆうサービス	
\\	動画投稿サイト	どうがとうこうサイト	
\\	動画コーデック	どうがコーデック	
\\	同期化	どうきか	
\\	同期先	どうきさき	
\\	同期的	どうきてき	
\\	同期点識別子	どうきてんしきべつし	
\\	同期転送モード	どうきてんそうモード	
\\	同期伝送	どうきでんそう	
\\	同期元	どうきもと	
\\	同期デジタルハイアラキー	どうきデジタルハイアラキー	
\\	同期データリンク制御	どうきデータリンクせいぎょ	
\\	同期ビットストリーム	どうきビットストリーム	
\\	同義性	どうぎせい	
\\	同義名	どうぎめい	
\\	同形異義性	どうけいいぎせい	
\\	同形同音異義性	どうけいどうおんいぎせい	
\\	動作位置	どうさいち	
\\	動作可能時間	どうさかのうじかん	
\\	動作環境	どうさかんきょう	
\\	動作記述部	どうさきじゅつぶ	
\\	動作行	どうさぎょう	
\\	動作指定	どうさしてい	
\\	動作時間	どうさじかん	
\\	動作状況	どうさじょうきょう	
\\	動作状態	どうさじょうたい	
\\	動作不能時間	どうさふのうじかん	
\\	動作有効期間	どうさゆうこうきかん	
\\	動作率	どうさりつ	
\\	動作領域	どうさりょういき	
\\	動作フィールド	どうさフィールド	
\\	動作ページ	どうさページ	
\\	動作モード	どうさモード	
\\	同種計算機ネットワーク	どうしゅけいさんきネットワーク	
\\	同時間	どうじかん	
\\	動軸回転表示	どうじくかいてんひょうじ	
\\	同相電圧	どうそうでんあつ	
\\	同相分除去	どうそうぶんじょきょ	
\\	導通試験	どうつうしけん	
\\	動的解析	どうてきかいせき	
\\	動的緩衝法	どうてきかんしょうほう	
\\	動的記憶装置	どうてききおくそうち	
\\	動的再配置	どうてきさいはいち	
\\	動的資源割り振り	どうてきしげんわりふり	
\\	動的資源割振り	どうてきしげんわりふり	
\\	動的装置再構成	どうてきそうちさいこうせい	
\\	動的適合性	どうてきてきごうせい	
\\	動的適合性要件	どうてきてきごうせいようけん	
\\	動的な帯域割当	どうてきなたいいきわりあて	
\\	動的呼出し	どうてきよびだし	
\\	動的割振り	どうてきわりふり	
\\	動的アドレス変換	どうてきアドレスへんかん	
\\	動的アドレス変換機構	どうてきアドレスへんかんきこう	
\\	動的コンテンツ	どうてきコンテンツ	
\\	動的ルーチング	どうてきルーチング	
\\	動的ルーティング	どうてきルーティング	
\\	導入済み	どうにゅうすみ	
\\	同報	どうほう	
\\	同報通信	どうほうつうしん	
\\	同報伝送	どうほうでんそう	
\\	同報サービス	どうほうサービス	
\\	同曜日	どうようび	
\\	道路交通情報通信システム	どうろこうつうじょうほうつうしんシステム	
\\	独断的	どくだんてき	
\\	独立系ソフトウェアベンダ	どくりつけいソフトウェアベンダ	
\\	独立系ソフトウェアメーカー	どくりつけいソフトウェアメーカー	
\\	独立項目	どくりつこうもく	
\\	独立コンパイル	どくりつコンパイル	
\\	独立データ記述項	どくりつデータきじゅつこう	
\\	独立ユティリティプログラム	どくりつユティリティプログラム	
\\	内字	ないじ	
\\	内蔵型アレイプロセッサー	ないぞうがたアレイプロセッサー	
\\	内蔵装置	ないぞうそうち	
\\	内蔵オーディオ回路	ないぞうオーディオかいろ	
\\	内蔵テンキーパッド	ないぞうテンキーパッド	
\\	内蔵モデム	ないぞうモデム	
\\	内部記憶	ないぶきおく	
\\	内部記憶装置	ないぶきおくそうち	
\\	内部参照	ないぶさんしょう	
\\	内部詳細	ないぶしょうさい	
\\	内部処理	ないぶしょり	
\\	内部状態	ないぶじょうたい	
\\	内部手続	ないぶてつづき	
\\	内部副プログラム	ないぶふくプログラム	
\\	内部変数	ないぶへんすう	
\\	内部様式	ないぶようしき	
\\	内部割り込み	ないぶわりこみ	
\\	内部キャッシュ	ないぶキャッシュ	
\\	内部クロック	ないぶクロック	
\\	内部クロック同期	ないぶクロックどうき	
\\	内部コマンド	ないぶコマンド	
\\	内部データ	ないぶデータ	
\\	内部データ項目	ないぶデータこうもく	
\\	内部バッファ	ないぶバッファ	
\\	内部ファイル	ないぶファイル	
\\	内部ファイル結合子	ないぶファイルけつごうし	
\\	内部ファイルサービス	ないぶファイルサービス	
\\	内部メモリ	ないぶメモリ	
\\	内部ラベル	ないぶラベル	
\\	内部リンク	ないぶリンク	
\\	内容完全性	ないようかんぜんせい	
\\	内容規則	ないようきそく	
\\	内容機密性	ないようきみつせい	
\\	内容検索	ないようけんさく	
\\	内容種別	ないようしゅべつ	
\\	内容体系	ないようたいけい	
\\	内容長	ないようちょう	
\\	内容部	ないようぶ	
\\	内容分析	ないようぶんせき	
\\	内容要素	ないようようそ	
\\	内容呼び出しメモリ・内容呼出メモリ	ないようよびだしメモリ	
\\	内容アドレス記憶装置	ないようアドレスきおくそうち	
\\	内容モデル	ないようモデル	
\\	長いファイル名	ながいファイルめい	
\\	長さ指示子	ながさしじし	
\\	長さオクテット	ながさオクテット	
\\	流れ図	ながれず	
\\	流れ図記号	ながれずきごう	
\\	流れ制御	ながれせいぎょ	
\\	流れ線	ながれせん	
\\	流れの向き	ながれのむき	
\\	名前解決	なまえかいけつ	
\\	名前開始文字	なまえかいしもじ	
\\	名前解析	なまえかいせき	
\\	名前群	なまえぐん	
\\	名前結合	なまえけつごう	
\\	名前指定実体参照	なまえしていじったいさんしょう	
\\	名前指定文字参照	なまえしていもじさんしょう	
\\	名前集合	なまえしゅうごう	
\\	名前字句	なまえじく	
\\	名前字句群	なまえじくぐん	
\\	名前付き	なまえつき	
\\	名前付き定数	なまえつきていすう	
\\	名前付きパイプ	なまえつきパイプ	
\\	名前による代入	なまえによるだいにゅう	
\\	名前文字	なまえもじ	
\\	名前コンテキスト	なまえコンテキスト	
\\	生データ	なまデータ	
\\	並び線	ならびせん	
\\	並べ替え	ならべかえ	
\\	成り済まし・成りすまし	なりすまし	
\\	名礼	なれ	
\\	難語集	なんごしゅう	
\\	何の変更もなく	なんのへんこうもなく	
\\	二項演算子	にこうえんざんし	
\\	2項式	にこうしき	
\\	2進演算	にしんえんざん	
\\	2進加算器	にしんかさんき	
\\	2進化10進数・二進化十進数	にしんかじっしんすう	
\\	2進化10進表示法・二進化十進進表示法	にしんかじっしんひょうじほう	
\\	2進化10進法・二進化十進法	にしんかじっしんほう	
\\	2進化10進コード・二進化十進コード	にしんかじっしんコード	
\\	2進
\\	5進・2進5進	にしんごしん	
\\	2進数・二進数	にしんすう	
\\	2進数字	にしんすうじ	
\\	二進表記法	にしんひょうきほう	
\\	2進データ同期通信	にしんデータどうきつうしん	
\\	二次記憶装置	にじきおくそうち	
\\	2次局	にじきょく	
\\	二次局	にじきょく	
\\	二次局状態	にじきょくじょうたい	
\\	二次資料	にじしりょう	
\\	二次文献	にじぶんけん	
\\	二重引用符	にじゅういんようふ	
\\	二重化	にじゅうか	
\\	二重書き	にじゅうかき	
\\	二重接続局	にじゅうせつぞくきょく	
\\	二重接続コンセントレータ	にじゅうせつぞくコンセントレータ	
\\	二重通信	にじゅうつうしん	
\\	二重リング	にじゅうリング	
\\	二乗機能	にじょうきのう	
\\	二次キャッシュ	にじキャッシュ	
\\	2000年問題	にせんねんもんだい	
\\	二段化構成	にだんかこうせい	
\\	二段構成	にだんこうせい	
\\	二地点同接続	にちてんどうせつぞく	
\\	日別概要	にちべつがいよう	
\\	日経MIX	にっけいミックス	
\\	二分分類体系	にぶんぶんるいたいけい	
\\	2変量補間	にへんりょうほかん	
\\	日本語化	にほんごか	
\\	日本語環境	にほんごかんきょう	
\\	日本語入力FEP	にほんごにゅうりょくエフイーピー	
\\	日本語版	にほんごばん	
\\	日本語ドメイン名	にほんごドメインめい	
\\	日本語モード	にほんごモード	
\\	日本語ワードプロセッサ	にほんごワードプロセッサ	
\\	入回線	にゅうかいせん	
\\	入手可能	にゅうしゅかのう	
\\	入手可能公開文	にゅうしゅかのうこうかいぶん	
\\	入出力管理	にゅうしゅつりょくかんり	
\\	入出力管理記述項	にゅうしゅつりょくかんりきじゅつこう	
\\	入出力機器	にゅうしゅつりょくきき	
\\	入出力機構	にゅうしゅつりょくきこう	
\\	入出力処理装置	にゅうしゅつりょくしょりそうち	
\\	入出力状態	にゅうしゅつりょくじょうたい	
\\	入出力制御装置	にゅうしゅつりょくせいぎょそうち	
\\	入出力制御システム	にゅうしゅつりょくせいぎょシステム	
\\	入出力節	にゅうしゅつりょくせつ	
\\	入出力装置	にゅうしゅつりょくそうち	
\\	入出力文	にゅうしゅつりょくぶん	
\\	入出力両用ファイル	にゅうしゅつりょくりょうようファイル	
\\	入出力両用モード	にゅうしゅつりょくりょうようモード	
\\	入出力チャネル	にゅうしゅつりょくチャネル	
\\	入出力データ	にゅうしゅつりょくデータ	
\\	入出力ファイル	にゅうしゅつりょくファイル	
\\	入出力プロセッサ	にゅうしゅつりょくプロセッサ	
\\	入手不能公開文	にゅうしゅふのうこうかいぶん	
\\	入力誤り	にゅうりょくあやまり	
\\	入力応答	にゅうりょくおうとう	
\\	入力画面	にゅうりょくがめん	
\\	入力機構	にゅうりょくきこう	
\\	入力基本要素	にゅうりょくきほんようそ	
\\	入力行	にゅうりょくぎょう	
\\	入力処理	にゅうりょくしょり	
\\	入力受理	にゅうりょくじゅり	
\\	入力順	にゅうりょくじゅん	
\\	入力制御	にゅうりょくせいぎょ	
\\	入力促進	にゅうりょくそくしん	
\\	入力促進エコー種別	にゅうりょくそくしんエコーしゅべつ	
\\	入力値	にゅうりょくち	
\\	入力中止動作	にゅうりょくちゅうしどうさ	
\\	入力手続き	にゅうりょくてつづき	
\\	入力の流れ	にゅうりょくのながれ	
\\	入力文	にゅうりょくぶん	
\\	入力保護	にゅうりょくほご	
\\	入力用	にゅうりょくよう	
\\	入力要求	にゅうりょくようきゅう	
\\	入力類	にゅうりょくるい	
\\	入力キュー	にゅうりょくキュー	
\\	入力キー	にゅうりょくキー	
\\	入力クラス	にゅうりょくクラス	
\\	入力サブシステム	にゅうりょくサブシステム	
\\	入力デバイス	にゅうりょくデバイス	
\\	入力データ	にゅうりょくデータ	
\\	入力ファイル	にゅうりょくファイル	
\\	入力ファイル名	にゅうりょくファイルめい	
\\	入力フォーカス	にゅうりょくフォーカス	
\\	入力モード	にゅうりょくモード	
\\	入トラヒック	にゅうトラヒック	
\\	任意型	にんいがた	
\\	任意信号	にんいしんごう	
\\	任意選択構成要素	にんいせんたくこうせいようそ	
\\	任意選択属性	にんいせんたくぞくせい	
\\	任意選択利用者ファシリティ	にんいせんたくりようしゃファシリティ	
\\	任意キャラクタ	にんいキャラクタ	
\\	認証局	にんしょうきょく	
\\	認証交換	にんしょうこうかん	
\\	認証情報	にんしょうじょうほう	
\\	認証媒体	にんしょうばいたい	
\\	認証トークン	にんしょうトークン	
\\	2ルート化	にルートか	
\\	塗りつぶし・塗り潰し	ぬりつぶし	
\\	熱設計電力	ねつせっけいでんりょく	
\\	熱転写プリンタ	ねつてんしゃプリンタ	
\\	熱暴走	ねつぼうそう	
\\	年日付	ねんひづけ	
\\	根コンテキスト	ねコンテキスト	
\\	能動的脅威	のうどうてききょうい	
\\	納入業者	のうにゅうぎょうしゃ	
\\	納本図書館	のうほんとしょかん	
\\	上りリンク	のぼりリンク	
\\	廃棄セル	はいきセル	
\\	廃棄プライオリティ	はいきプライオリティ	
\\	背景画像	はいけいがぞう	
\\	背景反射率	はいけいはんしゃりつ	
\\	廃止事項	はいしじこう	
\\	廃止予定	はいしよてい	
\\	廃止予定事項	はいしよていじこう	
\\	配信期限指定	はいしんきげんしてい	
\\	配信証明	はいしんしょうめい	
\\	配信時刻表示	はいしんじこくひょうじ	
\\	配信通知	はいしんつうち	
\\	配信の否認不能	はいしんのひにんふのう	
\\	配信不能	はいしんふのう	
\\	配信不能通知	はいしんふのうつうち	
\\	配信不能通知の抑止	はいしんふのうつうちのよくし	
\\	配信報告	はいしんほうこく	
\\	配信報告エントリ	はいしんほうこくエントリ	
\\	配信保留	はいしんほりゅう	
\\	配信優先度選択	はいしんゆうせんどせんたく	
\\	配信EIT	はいしんイーアイティー	
\\	配信メッセージエントリ	はいしんメッセージエントリ	
\\	排除要素	はいじょようそ	
\\	配線盤	はいせんばん	
\\	配線ケーブル	はいせんケーブル	
\\	排他演算	はいたえんざん	
\\	排他制御	はいたせいぎょ	
\\	排他素子	はいたそし	
\\	排他的論理和	はいたてきろんりわ	
\\	排他的論理和演算	はいたてきろんりわえんざん	
\\	排他的論理和素子	はいたてきろんりわそし	
\\	排他ディスパッチ機構	はいたディスパッチきこう	
\\	配電器	はいでんき	
\\	配布先表	はいふさきひょう	
\\	配布先表展開	はいふさきひょうてんかい	
\\	配布先表展開禁止	はいふさきひょうてんかいきんし	
\\	配布先表展開履歴表示	はいふさきひょうてんかいりれきひょうじ	
\\	配布先表名	はいふさきひょうめい	
\\	配列	はいれつ	
\\	配列値	はいれつち	
\\	配列分割	はいれつぶんかつ	
\\	配列名	はいれつめい	
\\	配列要素	はいれつようそ	
\\	配列ポインタ	はいれつポインタ	
\\	破壊	はかい	
\\	破壊加算	はかいかさん	
\\	破壊的読み取り	はかいてきよみとり	
\\	破壊読み出し	はかいよみだし	
\\	破壊読取り	はかいよみとり	
\\	白色雑音	はくしょくざつおん	
\\	博物館学	はくぶつかんがく	
\\	博物館資料	はくぶつかんしりょう	
\\	博物館資料記録技術	はくぶつかんしりょうきろくぎじゅつ	
\\	派生文献	はせいぶんけん	
\\	破損クラスタ	はそんクラスタ	
\\	八進細分記号	はちしんさいぶんきごう	
\\	発加入者	はっかにゅうしゃ	
\\	発局	はっきょく	
\\	発見的	はっけんてき	
\\	発見的方法	はっけんてきほうほう	
\\	発呼	はっこ	
\\	発光素子	はっこうそし	
\\	発行と引用	はっこうといんよう	
\\	発呼側SS利用者	はっこがわエスエスりようしゃ	
\\	発呼側Nアドレス	はっこがわエンアドレス	
\\	発呼側NS利用者	はっこがわエンエスりようしゃ	
\\	発呼側TS利用者	はっこがわティーエスりようしゃ	
\\	発信拠点	はっしんきょてん	
\\	発信源	はっしんげん	
\\	発信者	はっしんしゃ	
\\	発信者識別番号	はっしんしゃしきべつばんごう	
\\	発信者表示	はっしんしゃひょうじ	
\\	発信者要求代行受信者	はっしんしゃようきゅうだいこうじゅしんしゃ	
\\	発信証明	はっしんしょうめい	
\\	発信時刻表示	はっしんじこくひょうじ	
\\	発信の否認不能	はっしんのひにんふのう	
\\	発信頻度	はっしんひんど	
\\	発信元	はっしんもと	
\\	発生源の否認不能	はっせいげんのひにんふのう	
\\	発生順	はっせいじゅん	
\\	発生頻度	はっせいひんど	
\\	幅優先探索	はばゆうせんたんさく	
\\	針金細工表現	はりがねざいくひょうげん	
\\	貼付け・貼り付け・はり付け	はりつけ	
\\	範囲演算子	はんいえんざんし	
\\	範囲検査	はんいけんさ	
\\	範囲名	はんいめい	
\\	範囲明示文	はんいめいじぶん	
\\	汎欧州	はんおうしゅう	
\\	半角文字	はんかくもじ	
\\	半加算器	はんかさんき	
\\	半加算機	はんかさんき	
\\	半減算器	はんげんさんき	
\\	半語	はんご	
\\	反射型カラー液晶ディスプレイ	はんしゃがたカラーえきしょうディスプレイ	
\\	反射計算	はんしゃけいさん	
\\	反射公式	はんしゃこうしき	
\\	反射特性	はんしゃとくせい	
\\	反射法線	はんしゃほうせん	
\\	反射モデル	はんしゃモデル	
\\	版数管理	はんすうかんり	
\\	判断記号	はんだんきごう	
\\	判断命令	はんだんめいれい	
\\	半値電力幅	はんちでんりょくはば	
\\	判定基準	はんていきじゅん	
\\	反転表示	はんてんひょうじ	
\\	半導体ウエハ	はんどうたいウエハ	
\\	半導体ディスク	はんどうたいディスク	
\\	半二重通信・半2重通信	はんにじゅうつうしん	
\\	半二重伝送・半2重伝送	はんにじゅうでんそう	
\\	販売時点	はんばいじてん	
\\	販売時点管理	はんばいじてんかんり	
\\	反復演算	はんぷくえんざん	
\\	汎用階層ファイルモデル・はん用階層ファイルモデル	はんようかいそうファイルモデル	
\\	汎用型コンピュータ	はんようがたコンピュータ	
\\	汎用機	はんようき	
\\	汎用体・はん用体	はんようたい	
\\	汎用定義・はん用定義	はんようていぎ	
\\	汎用的試験項目・はん用的試験項目	はんようてきしけんこうもく	
\\	汎用的試験項目群・はん用的試験項目群	はんようてきしけんこうもくぐん	
\\	汎用名前付け規則	はんようなまえつけきそく	
\\	汎用入出力制御装置	はんようにゅうしゅつりょくせいぎょそうち	
\\	汎用コンピュータ	はんようコンピュータ	
\\	汎用JPドメイン名	はんようジェーピードメインめい	
\\	汎用レジスタ・はん用レジスタ	はんようレジスタ	
\\	倍角	ばいかく	
\\	倍精度浮動小数点	ばいせいどふどうしょうすうてん	
\\	倍速	ばいそく	
\\	倍速CDROMドライブ	ばいそくシーディーロムドライブ	
\\	倍速CPU	ばいそくシーピーユー	
\\	媒体依存インタフェース	ばいたいいぞんインタフェース	
\\	媒体接続機構	ばいたいせつぞくきこう	
\\	媒体アクセス制御	ばいたいアクセスせいぎょ	
\\	媒体アクセス制御副層	ばいたいアクセスせいぎょふくそう	
\\	媒体アクセス制御プロトコル	ばいたいアクセスせいぎょプロトコル	
\\	媒体インタフェースコネクタ	ばいたいインタフェースコネクタ	
\\	倍長語	ばいちょうご	
\\	倍長レジスタ	ばいちょうレジスタ	
\\	倍幅文字	ばいはばもじ	
\\	倍密度	ばいみつど	
\\	倍率付き測定単位	ばいりつつきそくていたんい	
\\	倍クロック技術	ばいクロックぎじゅつ	
\\	倍パルス記録	ばいパルスきろく	
\\	爆弾マーク	ばくだんマーク	
\\	化けチェック	ばけチェック	
\\	罵倒メール	ばとうメール	
\\	番号案内	ばんごうあんない	
\\	番号計画	ばんごうけいかく	
\\	番号通話	ばんごうつうわ	
\\	非圧縮	ひあっしゅく	
\\	非安定マルチバイブレータ	ひあんていマルチバイブレータ	
\\	非一致演算	ひいっちえんざん	
\\	非印刷文字	ひいんさつもじ	
\\	非英数字	ひえいすうじ	
\\	非会話モード	ひかいわモード	
\\	非可逆圧縮	ひかぎゃくあっしゅく	
\\	比較演算子	ひかくえんざんし	
\\	比較関係	ひかくかんけい	
\\	比較器	ひかくき	
\\	比較式	ひかくしき	
\\	比較条件	ひかくじょうけん	
\\	比較文字	ひかくもじ	
\\	非加算機能	ひかさんきのう	
\\	被加数	ひかすう	
\\	光記憶装置	ひかりきおくそうち	
\\	光基準面	ひかりきじゅんめん	
\\	光集中指数	ひかりしゅうちゅうしすう	
\\	光磁気ディスク	ひかりじきディスク	
\\	光造形システム	ひかりぞうけいシステム	
\\	光立ち上がり時間	ひかりたちあがりじかん	
\\	光立ち下がり時間	ひかりたちさがりじかん	
\\	光伝達網	ひかりでんたつもう	
\\	光インタフェース	ひかりインタフェース	
\\	光コンピュータ	ひかりコンピュータ	
\\	光センサーマウス	ひかりセンサーマウス	
\\	光チャネルアダプタ	ひかりチャネルアダプタ	
\\	光ディスク	ひかりディスク	
\\	光ファイバ伝送技術	ひかりファイバでんそうぎじゅつ	
\\	光ファイバ分散型データインターフェース	ひかりファイバぶんさんがたデータインターフェース	
\\	光ファイバケーブル	ひかりファイバケーブル	
\\	光メモリ	ひかりメモリ	
\\	被管理開放型システム	ひかんりかいほうがたシステム	
\\	被管理システム	ひかんりシステム	
\\	引き込みケーブル	ひきこみケーブル	
\\	引込みケーブル	ひきこみケーブル	
\\	非基準復帰記録	ひきじゅんふっききろく	
\\	引数・引き数	ひきすう	
\\	引数結合	ひきすうけつごう	
\\	引数実体	ひきすうじったい	
\\	引数実体参照	ひきすうじったいさんしょう	
\\	引数表記	ひきすうひょうき	
\\	引数分離子	ひきすうぶんりし	
\\	引数領域	ひきすうりょういき	
\\	引数キーワード	ひきすうキーワード	
\\	引数リスト	ひきすうリスト	
\\	非局所分岐	ひきょくしょぶんき	
\\	低い優先順位	ひくいゆうせんじゅんい	
\\	非空白文字	ひくうはくもじ	
\\	非結晶質	ひけっしょうしつ	
\\	非公開受信者表示	ひこうかいじゅしんしゃひょうじ	
\\	非構造制約集合	ひこうぞうせいやくしゅうごう	
\\	非集中型N多端点コネクション	ひしゅうちゅうがたエンたたんてんコネクション	
\\	非衝撃式印字装置	ひしょうげきしきいんじそうち	
\\	非実行文	ひじっこうぶん	
\\	非常駐	ひじょうちゅう	
\\	非常用取り出しホール	ひじょうようとりだしホール	
\\	非数	ひすう	
\\	非図形文字	ひずけいもじ	
\\	非整数値	ひせいすうち	
\\	非線形計画法	ひせんけいけいかくほう	
\\	非絶縁増幅器	ひぜつえんぞうふくき	
\\	非対称型マルチプロセッサ	ひたいしょうかたマルチプロセッサ	
\\	非対称型デジタルループ	ひたいしょうがたデジタルループ	
\\	非対称データフロー	ひたいしょうデータフロー	
\\	左上隅	ひだりうえすみ	
\\	左下隅	ひだりしたすみ	
\\	左揃え	ひだりそろえ	
\\	左手座標系	ひだりてざひょうけい	
\\	左シフト	ひだりシフト	
\\	左マウスボタン	ひだりマウスボタン	
\\	非直結	ひちょっけつ	
\\	必す構成要素	ひっすこうせいようそ	
\\	必す信号	ひっすしんごう	
\\	必す属性	ひっすぞくせい	
\\	必須入力	ひっすにゅうりょく	
\\	必要語	ひつようご	
\\	必要動作環境	ひつようどうさかんきょう	
\\	必要メモリ	ひつようメモリ	
\\	否定	ひてい	
\\	否定回路	ひていかいろ	
\\	否定組合せ条件	ひていくみあわせじょうけん	
\\	否定積	ひていせき	
\\	否定素子	ひていそし	
\\	否定単純条件	ひていたんじゅんじょうけん	
\\	否定論理積	ひていろんりせき	
\\	否定論理積演算	ひていろんりせきえんざん	
\\	否定論理積素子	ひていろんりせきそし	
\\	否定論理和	ひていろんりわ	
\\	否定論理和演算	ひていろんりわえんざん	
\\	否定論理和素子	ひていろんりわそし	
\\	否定和	ひていわ	
\\	非手続き型言語	ひてつづきかたげんご	
\\	非等価演算	ひとうかえんざん	
\\	非等時性伝送	ひとうじせいでんそう	
\\	非同期	ひどうき	
\\	非同期応答モード	ひどうきおうとうモード	
\\	非同期式	ひどうきしき	
\\	非同期式伝送	ひどうきしきでんそう	
\\	非同期通信アダプタ	ひどうきつうしんアダプタ	
\\	非同期手続き	ひどうきてつづき	
\\	非同期転送モード	ひどうきてんそうモード	
\\	非同期伝送	ひどうきでんそう	
\\	非同期平衡モード	ひどうきへいこうモード	
\\	非同期バス	ひどうきバス	
\\	非破壊的	ひはかいてき	
\\	非破壊読取り	ひはかいよみとり	
\\	非番号制コマンド	ひばんごうせいコマンド	
\\	非番号制レスポンス	ひばんごうせいレスポンス	
\\	非表示	ひひょうじ	
\\	非表示文字	ひひょうじもじ	
\\	非負整数	ひふせいすう	
\\	非保護領域	ひほごりょういき	
\\	秘密鍵	ひみつかぎ	
\\	秘密鍵暗号方式	ひみつかぎあんごうほうしき	
\\	秘密度表示	ひみつどひょうじ	
\\	秘密分散	ひみつぶんさん	
\\	秘密保持契約	ひみつほじけいやく	
\\	百分率機能	ひゃくぶんりつきのう	
\\	非優先語	ひゆうせんご	
\\	表意定数	ひょういていすう	
\\	評価制度	ひょうかせいど	
\\	評価値	ひょうかち	
\\	評価報告書	ひょうかほうこくしょ	
\\	表記体系	ひょうきたいけい	
\\	表記方向	ひょうきほうこう	
\\	表記法の許容性	ひょうきほうのきょようせい	
\\	表記法の展開性	ひょうきほうのてんかいせい	
\\	表計算	ひょうけいさん	
\\	表計算アプリケーション	ひょうけいさんアプリケーション	
\\	表計算プログラム	ひょうけいさんプログラム	
\\	表色系	ひょうしょくけい	
\\	表示	ひょうじ	
\\	表示域	ひょうじいき	
\\	表示印字式計算器	ひょうじいんじしきけいさんき	
\\	表示画像	ひょうじがぞう	
\\	表示器	ひょうじき	
\\	表示空間	ひょうじくうかん	
\\	表示更新モード	ひょうじこうしんモード	
\\	表示式計算器	ひょうじしきけいさんき	
\\	表示指定	ひょうじしてい	
\\	表示指令	ひょうじしれい	
\\	表示操作卓	ひょうじそうさたく	
\\	表示装置	ひょうじそうち	
\\	表示体裁	ひょうじていさい	
\\	表示取消し	ひょうじとりけし	
\\	表示媒体	ひょうじばいたい	
\\	表示命令	ひょうじめいれい	
\\	表示面	ひょうじめん	
\\	表示文字実体集合	ひょうじもじじったいしゅうごう	
\\	標準解像度	ひょうじゅんかいぞうど	
\\	標準化団体	ひょうじゅんかだんたい	
\\	標準記法	ひょうじゅんきほう	
\\	標準出力	ひょうじゅんしゅつりょく	
\\	標準準拠	ひょうじゅんじゅんきょ	
\\	標準属性	ひょうじゅんぞくせい	
\\	標準的	ひょうじゅんてき	
\\	標準入出力	ひょうじゅんにゅうしゅつりょく	
\\	標準入力	ひょうじゅんにゅうりょく	
\\	標準の応答メッセージ	ひょうじゅんのおうとうメッセージ	
\\	標準アクセス速度	ひょうじゅんアクセスそくど	
\\	標準エラー	ひょうじゅんエラー	
\\	標準データ形式	ひょうじゅんデータけいしき	
\\	標準MIDIファイル	ひょうじゅんミディファイル	
\\	標準モジュール	ひょうじゅんモジュール	
\\	標準ライブラリ	ひょうじゅんライブラリ	
\\	表示優先度	ひょうじゆうせんど	
\\	表示様式	ひょうじようしき	
\\	表示要素	ひょうじようそ	
\\	表示様相源フラグ	ひょうじようそうげんフラグ	
\\	標題誌	ひょうだいし	
\\	評定水準	ひょうていすいじゅん	
\\	標本化定理	ひょうほんかていり	
\\	標本間隔	ひょうほんかんかく	
\\	標本時刻	ひょうほんじこく	
\\	標本分布	ひょうほんぶんぷ	
\\	表見出し	ひょうみだし	
\\	表面実装技術	ひょうめんじっそうぎじゅつ	
\\	表面実装部品	ひょうめんじっそうぶひん	
\\	表面プロット	ひょうめんプロット	
\\	表要素	ひょうようそ	
\\	開いている実体	ひらいているじったい	
\\	開いている要素	ひらいているようそ	
\\	開いているマーク区間宣言	ひらいているマークくかんせんげん	
\\	開かれたシステム	ひらかれたシステム	
\\	開き口	ひらきぐち	
\\	比例に応じて	ひれいにおうじて	
\\	非連鎖シーケンス	ひれんさシーケンス	
\\	拡がり角	ひろがりかく	
\\	品質劣化	ひんしつれっか	
\\	頻度分布	ひんどぶんぷ	
\\	非アクティブ化	ひアクティブか	
\\	非アルファベット	ひアルファベット	
\\	非SGML文字	ひエスジーエムエルもじ	
\\	非SGMLデータ実体	ひエスジーエムエルデータじったい	
\\	非クライアント	ひクライアント	
\\	非シールド	ひシールド	
\\	非シールドより対線	ひシールドよりたいせん	
\\	非0復帰1記録	ひゼロふっきいちきろく	
\\	非ゼロ復帰記録	ひゼロふっききろく	
\\	非ゼロ復帰変化記録	ひゼロふっきへんかきろく	
\\	非0復帰0記録	ひゼロふっきゼロきろく	
\\	非ゼロ復帰マーク記録	ひゼロふっきマークきろく	
\\	非デスクリプタ	ひデスクリプタ	
\\	非ノイマン型・非ノイマン形	ひノイマンがた	
\\	非フレーム形式	ひフレームけいしき	
\\	非リアルタイム	ひリアルタイム	
\\	描画色モデル	びょうがいろモデル	
\\	描画像	びょうがぞう	
\\	描画パイプライン	びょうがパイプライン	
\\	描画ヘッド	びょうがヘッド	
\\	不安定状態	ふあんていじょうたい	
\\	風景画書式	ふうけいがしょしき	
\\	付加価値再販業者	ふかかちさいはんぎょうしゃ	
\\	付加価値ネットワーク	ふかかちネットワーク	
\\	不可逆圧縮	ふかぎゃくあっしゅく	
\\	不確定判定	ふかくていはんてい	
\\	負荷軽減	ふかけいげん	
\\	深さ優先探索	ふかさゆうせんたんさく	
\\	不可視フィルタ	ふかしフィルタ	
\\	負荷増大	ふかぞうたい	
\\	不活性機能文字	ふかっせいきのうもじ	
\\	付加物理表現	ふかぶつりひょうげん	
\\	負荷分散装置	ふかぶんさんそうち	
\\	負荷分散電源供給	ふかぶんさんでんげんきょうきゅう	
\\	負荷平衡	ふかへいこう	
\\	付加文字	ふかもじ	
\\	不感帯要素	ふかんたいようそ	
\\	付加サービス	ふかサービス	
\\	負荷テスト	ふかテスト	
\\	負荷バランシング	ふかバランシング	
\\	不揮発性記憶装置	ふきはつせいきおくそうち	
\\	不揮発性メモリ	ふきはつせいメモリ	
\\	不揮発性メモリー	ふきはつせいメモリー	
\\	普及過程	ふきゅうかてい	
\\	復号化・複号化	ふくごうか	
\\	復号器	ふくごうき	
\\	複合局	ふくごうきょく	
\\	複合構成部品	ふくごうこうせいぶひん	
\\	複合式	ふくごうしき	
\\	複合条件	ふくごうじょうけん	
\\	複合表記法	ふくごうひょうきほう	
\\	複合文	ふくごうぶん	
\\	複合法	ふくごうほう	
\\	複合命令セットコンピューター	ふくごうめいれいセットコンピューター	
\\	複合論理対象体	ふくごうろんりたいしょうたい	
\\	複合割付け対象体	ふくごうわりつけたいしょうたい	
\\	複雑命令セットコンピュータ	ふくざつめいれいセットコンピュータ	
\\	複式本体	ふくしきほんたい	
\\	複数拠点	ふくすうきょてん	
\\	複数行	ふくすうぎょう	
\\	複数業務	ふくすうぎょうむ	
\\	複数層試験	ふくすうそうしけん	
\\	複数地点	ふくすうちてん	
\\	複数文字	ふくすうもじ	
\\	複数カラム文字	ふくすうカラムもじ	
\\	複数バイト文字	ふくすうバイトもじ	
\\	複数フレーム伝送	ふくすうフレームでんそう	
\\	複数ワード	ふくすうワード	
\\	副層	ふくそう	
\\	輻輳状態	ふくそうじょうたい	
\\	輻輳制御	ふくそうせいぎょ	
\\	複素区間	ふくそくかん	
\\	複素定数	ふくそていすう	
\\	復調機	ふくちょうき	
\\	復調装置	ふくちょうそうち	
\\	副電力	ふくでんりょく	
\\	副待ち行列	ふくまちぎょうれつ	
\\	副要求	ふくようきゅう	
\\	袋綴じ・袋とじ	ふくろとじ	
\\	袋文字	ふくろもじ	
\\	副プログラム	ふくプログラム	
\\	副プログラムの分身	ふくプログラムのぶんしん	
\\	副ボリューム記述子	ふくボリュームきじゅつし	
\\	副レコードキー	ふくレコードキー	
\\	符号誤り率	ふごうあやまりりつ	
\\	符号位置	ふごういち	
\\	符号化画像	ふごうかがぞう	
\\	符号拡張	ふごうかくちょう	
\\	符号化形式	ふごうかけいしき	
\\	符号化10進・符号化十進	ふごうかじっしん	
\\	符号化10進法・符号化十進法	ふごうかじっしんほう	
\\	符号化情報種別	ふごうかじょうほうしゅべつ	
\\	符号化表現	ふごうかひょうげん	
\\	符号化方式	ふごうかほうしき	
\\	符号間干渉	ふごうかんかんしょう	
\\	符号系名	ふごうけいめい	
\\	符号集合	ふごうしゅうごう	
\\	符号集合位置	ふごうしゅうごういち	
\\	符号数字	ふごうすうじ	
\\	符号体系	ふごうたいけい	
\\	符号短縮法	ふごうたんしゅくほう	
\\	符号付き・符号付	ふごうつき	
\\	符号反転機能	ふごうはんてんきのう	
\\	符号表	ふごうひょう	
\\	符号分割多元接続方式	ふごうぶんかつたげんせつぞくほうしき	
\\	符号分割多重	ふごうぶんかつたじゅう	
\\	符号文字	ふごうもじ	
\\	符号理論	ふごうりろん	
\\	符号ビット	ふごうビット	
\\	符号フラグ	ふごうフラグ	
\\	浮上高	ふじょうこう	
\\	不正文字	ふせいもじ	
\\	不正アクセス	ふせいアクセス	
\\	不正コピー	ふせいコピー	
\\	不足データ	ふそくデータ	
\\	付属せず	ふぞくせず	
\\	復帰文字	ふっきもじ	
\\	復旧手順	ふっきゅうてじゅん	
\\	復帰アドレス	ふっきアドレス	
\\	復帰コード	ふっきコード	
\\	不定期業務	ふていきぎょうむ	
\\	不定長レコード	ふていちょうレコード	
\\	不定ファイル	ふていファイル	
\\	不特定下位参照	ふとくていかいさんしょう	
\\	不特定記憶単位	ふとくていきおくたんい	
\\	不動作時間	ふどうさじかん	
\\	浮動小数点	ふどうしょうすうてん	
\\	浮動小数点演算	ふどうしょうすうてんえんざん	
\\	浮動小数点演算機構	ふどうしょうすうてんえんざんきこう	
\\	浮動小数点演算アクセラレータ	ふどうしょうすうてんえんざんアクセラレータ	
\\	浮動小数点基数	ふどうしょうすうてんきすう	
\\	浮動小数点基底	ふどうしょうすうてんきてい	
\\	浮動小数点数・浮動少数点数	ふどうしょうすうてんすう	
\\	浮動小数点相対精度	ふどうしょうすうてんそうたいせいど	
\\	浮動小数点代数	ふどうしょうすうてんだいすう	
\\	浮動小数点表示	ふどうしょうすうてんひょうじ	
\\	浮動小数点表示法	ふどうしょうすうてんひょうじほう	
\\	浮動小数点方式	ふどうしょうすうてんほうしき	
\\	浮動小数点コプロセッサ	ふどうしょうすうてんコプロセッサ	
\\	浮動小数点データ	ふどうしょうすうてんデータ	
\\	浮動小数点レジスタ	ふどうしょうすうてんレジスタ	
\\	浮動ヘッド	ふどうヘッド	
\\	負の符号入力	ふのふごうにゅうりょく	
\\	負の符号表示	ふのふごうひょうじ	
\\	付番	ふばん	
\\	付番核	ふばんかく	
\\	付番群	ふばんぐん	
\\	付番要素	ふばんようそ	
\\	普遍分類体系	ふへんぶんるいたいけい	
\\	不法コピー	ふほうコピー	
\\	不要情報	ふようじょうほう	
\\	不良セクター	ふりょうセクター	
\\	不良トラック	ふりょうトラック	
\\	振舞い試験	ふるまいしけん	
\\	紛失パルス	ふんしつパルス	
\\	物理記録密度	ぶつりきろくみつど	
\\	物理構成	ぶつりこうせい	
\\	物理信号制御副層	ぶつりしんごうせいぎょふくそう	
\\	物理層	ぶつりそう	
\\	物理層媒体依存部	ぶつりそうばいたいいぞんぶ	
\\	物理的安全保護	ぶつりてきあんぜんほご	
\\	物理的回線	ぶつりてきかいせん	
\\	物理的回送許可	ぶつりてきかいそうきょか	
\\	物理的回送禁止	ぶつりてきかいそうきんし	
\\	物理的配達	ぶつりてきはいたつ	
\\	物理的配達国名	ぶつりてきはいたつこくめい	
\\	物理的配達個人名	ぶつりてきはいたつこじんめい	
\\	物理的配達組織名	ぶつりてきはいたつそしきめい	
\\	物理的配達郵便局番号	ぶつりてきはいたつゆうびんきょくばんごう	
\\	物理的配達郵便局名	ぶつりてきはいたつゆうびんきょくめい	
\\	物理的配達郵便局アドレス構成要素	ぶつりてきはいたつゆうびんきょくアドレスこうせいようそ	
\\	物理的配達領域	ぶつりてきはいたつりょういき	
\\	物理的配達アクセス単位	ぶつりてきはいたつアクセスたんい	
\\	物理的配達アドレス構成要素	ぶつりてきはいたつアドレスこうせいようそ	
\\	物理的配達サービス	ぶつりてきはいたつサービス	
\\	物理的配達サービス名	ぶつりてきはいたつサービスめい	
\\	物理的配達システム	ぶつりてきはいたつシステム	
\\	物理的媒体	ぶつりてきばいたい	
\\	物理伝送路	ぶつりでんそうろ	
\\	物理入力装置	ぶつりにゅうりょくそうち	
\\	物理番地	ぶつりばんち	
\\	物理表現	ぶつりひょうげん	
\\	物理アドレス	ぶつりアドレス	
\\	物理アドレス拡張	ぶつりアドレスかくちょう	
\\	物理インタフェース	ぶつりインタフェース	
\\	物理エラー	ぶつりエラー	
\\	物理コネクション	ぶつりコネクション	
\\	物理ノード	ぶつりノード	
\\	物理フォーマット	ぶつりフォーマット	
\\	物理ページ	ぶつりページ	
\\	物理メッセージ	ぶつりメッセージ	
\\	物理メディア	ぶつりメディア	
\\	物理メモリ	ぶつりメモリ	
\\	物理リンク	ぶつりリンク	
\\	物理レイヤ	ぶつりレイヤ	
\\	物理レコード	ぶつりレコード	
\\	物理レコードアドレス	ぶつりレコードアドレス	
\\	部の見出し	ぶのみだし	
\\	部品表	ぶひんひょう	
\\	部分一致	ぶぶんいっち	
\\	部分回復	ぶぶんかいふく	
\\	部分けた上げ	ぶぶんけたあげ	
\\	部分語	ぶぶんご	
\\	部分参照子	ぶぶんさんしょうし	
\\	部分実装	ぶぶんじっそう	
\\	部分実体	ぶぶんじったい	
\\	部分全体関係	ぶぶんぜんたいかんけい	
\\	部分属性要求	ぶぶんぞくせいようきゅう	
\\	部分配列	ぶぶんはいれつ	
\\	部分配列添字	ぶぶんはいれつそえじ	
\\	部分木	ぶぶんもく	
\\	部分文字列	ぶぶんもじれつ	
\\	部分要素	ぶぶんようそ	
\\	部分領域	ぶぶんりょういき	
\\	部分列	ぶぶんれつ	
\\	部門名	ぶもんめい	
\\	ぶら下がりインデント	ぶらさがりインデント	
\\	分解	ぶんかい	
\\	文化圏固有動作	ぶんかけんこゆうどうさ	
\\	分割統治法	ぶんかつとうちほう	
\\	分割バー	ぶんかつバー	
\\	分割プロット	ぶんかつプロット	
\\	文関数	ぶんかんすう	
\\	分岐	ぶんき	
\\	分岐接続	ぶんきせつぞく	
\\	分岐予測	ぶんきよそく	
\\	分岐構成体	ぶんぎこうせいたい	
\\	文献検索	ぶんけんけんさく	
\\	文献集	ぶんけんしゅう	
\\	文献の選択	ぶんけんのせんたく	
\\	分散化	ぶんさんか	
\\	分散型データベース	ぶんさんがたデータベース	
\\	分散型データベース管理システム	ぶんさんがたデータベースかんりシステム	
\\	分散型トランザクション	ぶんさんがたトランザクション	
\\	分散協調処理	ぶんさんきょうちょうしょり	
\\	分散試験法	ぶんさんしけんほう	
\\	分散処理	ぶんさんしょり	
\\	分散処理環境	ぶんさんしょりかんきょう	
\\	分散処理システム	ぶんさんしょりシステム	
\\	分散制御	ぶんさんせいぎょ	
\\	分散名前解析	ぶんさんなまえかいせき	
\\	分散配置	ぶんさんはいち	
\\	分散分析法	ぶんさんぶんせきほう	
\\	分散コンピューティング	ぶんさんコンピューティング	
\\	分散コンピューティング環境	ぶんさんコンピューティングかんきょう	
\\	分散COM	ぶんさんシーオーエム	
\\	分散DoS攻撃	ぶんさんディーオーエスこうげき	
\\	分散データ処理	ぶんさんデータしょり	
\\	分散データベース	ぶんさんデータベース	
\\	文識別子	ぶんしきべつし	
\\	分周回路	ぶんしゅうかいろ	
\\	分周器	ぶんしゅうき	
\\	文章処理	ぶんしょうしょり	
\\	文章単位	ぶんしょうたんい	
\\	文書応用仕様	ぶんしょおうようしよう	
\\	文書化	ぶんしょか	
\\	文書概要	ぶんしょがいよう	
\\	文書概要水準	ぶんしょがいようすいじゅん	
\\	文書型	ぶんしょがた	
\\	文書型指定	ぶんしょがたしてい	
\\	文書型宣言	ぶんしょがたせんげん	
\\	文書型宣言部分集合	ぶんしょがたせんげんぶぶんしゅうごう	
\\	文書型定義	ぶんしょがたていぎ	
\\	文書処理応用	ぶんしょしょりおうよう	
\\	文書実現値	ぶんしょじつげんち	
\\	文書実現値集合	ぶんしょじつげんちしゅうごう	
\\	文書体系	ぶんしょたいけい	
\\	文書体系水準	ぶんしょたいけいすいじゅん	
\\	文書体系クラス	ぶんしょたいけいクラス	
\\	文書文字集合	ぶんしょもじしゅうごう	
\\	文書要素	ぶんしょようそ	
\\	文書読取り装置	ぶんしょよみとりそうち	
\\	文書論理根	ぶんしょろんりこん	
\\	文書割付け根	ぶんしょわりつけこん	
\\	文書割付け処理	ぶんしょわりつけしょり	
\\	文書アイコン	ぶんしょアイコン	
\\	文書クラス	ぶんしょクラス	
\\	文書クラス記述部	ぶんしょクラスきじゅつぶ	
\\	文書ファイル	ぶんしょファイル	
\\	分析形分類体系	ぶんせきがたぶんるいたいけい	
\\	分析合成形分類体系	ぶんせきごうせいがたぶんるいたいけい	
\\	分析手法	ぶんせきしゅほう	
\\	分断後部	ぶんだんこうぶ	
\\	分断前部	ぶんだんぜんぶ	
\\	文内要素	ぶんないようそ	
\\	文番号	ぶんばんごう	
\\	文法エラー	ぶんぽうエラー	
\\	文法チェッカ	ぶんぽうチェッカ	
\\	文脈依存	ぶんみゃくいぞん	
\\	文脈依存文法	ぶんみゃくいぞんぶんぽう	
\\	文脈自由	ぶんみゃくじゆう	
\\	文脈自由文法	ぶんみゃくじゆうぶんぽう	
\\	文脈上の区切り子	ぶんみゃくじょうのくぎりし	
\\	文脈上の選択字句	ぶんみゃくじょうのせんたくじく	
\\	文脈上の選択要素	ぶんみゃくじょうのせんたくようそ	
\\	文脈上の必す字句	ぶんみゃくじょうのひっすじく	
\\	文脈上の必す要素	ぶんみゃくじょうのひっすようそ	
\\	文脈宣言	ぶんみゃくせんげん	
\\	文脈に敏感	ぶんみゃくにびんかん	
\\	文脈列	ぶんみゃくれつ	
\\	分離記号	ぶんりきごう	
\\	分離子	ぶんりし	
\\	分離子文字	ぶんりしもじ	
\\	分離符	ぶんりふ	
\\	分類記号	ぶんるいきごう	
\\	分類作業	ぶんるいさぎょう	
\\	分類体系	ぶんるいたいけい	
\\	分類番号	ぶんるいばんごう	
\\	文キーワード	ぶんキーワード	
\\	文ラベル	ぶんラベル	
\\	閉域	へいいき	
\\	閉域接続	へいいきせつぞく	
\\	閉域利用者グループ	へいいきりようしゃグループ	
\\	平均故障間隔	へいきんこしょうかんかく	
\\	平均故障時間	へいきんこしょうじかん	
\\	平均修復時間	へいきんしゅうふくじかん	
\\	平均修理時間	へいきんしゅうりじかん	
\\	平均情報量	へいきんじょうほうりょう	
\\	平均伝達情報量	へいきんでんたつじょうほうりょう	
\\	平均光パワー	へいきんひかりパワー	
\\	平均アクセス時間	へいきんアクセスじかん	
\\	平均シーク時間	へいきんシークじかん	
\\	平均トラヒック	へいきんトラヒック	
\\	平均ビットレート	へいきんビットレート	
\\	平均レート	へいきんレート	
\\	平行光源	へいこうこうげん	
\\	平衡誤差	へいこうごさ	
\\	並行処理	へいこうしょり	
\\	並行注記	へいこうちゅうき	
\\	並行的	へいこうてき	
\\	平衡的誤差	へいこうてきごさ	
\\	平行変換	へいこうへんかん	
\\	併合用ファイル	へいごうようファイル	
\\	閉塞状態	へいそくじょうたい	
\\	閉塞ウィンドウ	へいそくウィンドウ	
\\	並直列変換器	へいちょくれつへんかんき	
\\	平方根機能	へいほうこんきのう	
\\	平面作図装置	へいめんさくずそうち	
\\	平面プロッタ	へいめんプロッタ	
\\	並列運転	へいれつうんてん	
\\	並列加算	へいれつかさん	
\\	並列加算器	へいれつかさんき	
\\	並列処理	へいれつしょり	
\\	並列転送	へいれつてんそう	
\\	並列伝送	へいれつでんそう	
\\	閉ループ	へいループ	
\\	辺可視フラグ	へんかしフラグ	
\\	変換機能	へんかんきのう	
\\	変換禁止	へんかんきんし	
\\	変換済み表示	へんかんずみひょうじ	
\\	変換精度	へんかんせいど	
\\	変換パイプライン	へんかんパイプライン	
\\	変換プログラム	へんかんプログラム	
\\	変形関数	へんけいかんすう	
\\	変形具象構文	へんけいぐしょうこうぶん	
\\	変形周波数変調記録	へんけいしゅうはすうへんちょうきろく	
\\	変更域ダンプ	へんこういきダンプ	
\\	変更可能	へんこうかのう	
\\	偏向関係	へんこうかんけい	
\\	偏向抄録	へんこうしょうろく	
\\	編集機能	へんしゅうきのう	
\\	編集処理	へんしゅうしょり	
\\	編集用文字	へんしゅうようもじ	
\\	編集モード	へんしゅうモード	
\\	返信要求表示	へんしんようきゅうひょうじ	
\\	返信IPメッセージ表示	へんしんアイピーメッセージひょうじ	
\\	変数	へんすう	
\\	変数設定	へんすうせってい	
\\	変数宣言	へんすうせんげん	
\\	変数名	へんすうめい	
\\	変数置換	へんずうちかん	
\\	辺線種	へんせんしゅ	
\\	辺線幅倍率	へんせんぷくばいりつ	
\\	返送照合	へんそうしょうごう	
\\	返送内容エントリ	へんそうないようエントリ	
\\	変調装置	へんちょうそうち	
\\	変調速度	へんちょうそくど	
\\	変調方式	へんちょうほうしき	
\\	変復調装置	へんふくちょうそうち	
\\	辺フラグ	へんフラグ	
\\	冪乗演算子	べきじょうえんざんし	
\\	別アカウント	べつアカウント	
\\	別ウィンドウ・別ウインドウ・別ウィンドー・別ウインドー	べつウィンドウ・べつウインドウ・べつウィンドー・べつウインドー	
\\	包括的試験サービス	ほうかつてきしけんサービス	
\\	放棄シーケンス	ほうきシーケンス	
\\	方向性・方向姓	ほうこうせい	
\\	方向キー	ほうこうキー	
\\	報告可能マーク誤り	ほうこくかのうマークあやまり	
\\	報告集団	ほうこくしゅうだん	
\\	報告集団記述項	ほうこくしゅうだんきじゅつこう	
\\	報告書脚書き	ほうこくしょあしがき	
\\	報告書頭書き	ほうこくしょあたまがき	
\\	報告書記述項	ほうこくしょきじゅつこう	
\\	報告書行	ほうこくしょぎょう	
\\	報告書句	ほうこくしょく	
\\	報告書作成管理システム	ほうこくしょさくせいかんりシステム	
\\	報告書節	ほうこくしょせつ	
\\	報告書名	ほうこくしょめい	
\\	報告書ファイル	ほうこくしょファイル	
\\	報告書レコード	ほうこくしょレコード	
\\	報告対象打診	ほうこくたいしょうだしん	
\\	報告対象メッセージ	ほうこくたいしょうメッセージ	
\\	報告発生源認証	ほうこくはっせいげんにんしょう	
\\	放送業界	ほうそうぎょうかい	
\\	放送形式	ほうそうけいしき	
\\	放送用人工衛星	ほうそうようじんこうえいせい	
\\	放送テレビ	ほうそうテレビ	
\\	報知抄録	ほうちしょうろく	
\\	法定納本	ほうていのうほん	
\\	放熱板	ほうねつばん	
\\	包絡線	ほうらくせん	
\\	補完規則	ほかんきそく	
\\	保管ファイル	ほかんファイル	
\\	保護機能	ほごきのう	
\\	保護装置	ほごそうち	
\\	保護用スリーブ	ほごようスリーブ	
\\	保護領域	ほごりょういき	
\\	保護キー	ほごキー	
\\	星状網	ほしじょうもう	
\\	星状ネットワーク	ほしじょうネットワーク	
\\	保守契約	ほしゅけいやく	
\\	保守盤	ほしゅばん	
\\	保守パネル	ほしゅパネル	
\\	保証格納域	ほしょうかくのういき	
\\	保証の限定	ほしょうのげんてい	
\\	補助漢字	ほじょかんじ	
\\	補助記憶	ほじょきおく	
\\	補助記憶装置	ほじょきおくそうち	
\\	補助語	ほじょご	
\\	補助装置	ほじょそうち	
\\	補助入力装置	ほじょにゅうりょくそうち	
\\	補助グループID	ほじょグループアイディー	
\\	補助コンソール	ほじょコンソール	
\\	補助デスクリプタ	ほじょデスクリプタ	
\\	補助バッテリ	ほじょバッテリ	
\\	保持TPDU	ほじティーピーディーユー	
\\	保持モード	ほじモード	
\\	補数演算	ほすうえんざん	
\\	補数演算子	ほすうえんざんし	
\\	補数回路	ほすうかいろ	
\\	補数器	ほすうき	
\\	補数の底	ほすうのそこ	
\\	保全性	ほぜんせい	
\\	補足編	ほそくへん	
\\	保存期間	ほぞんきかん	
\\	保存期間検査	ほぞんきかんけんさ	
\\	保存先	ほぞんさき	
\\	保留エリア	ほりゅうエリア	
\\	翻音	ほんおん	
\\	本学システム	ほんがくシステム	
\\	香港テレコム	ほんこんテレコム	
\\	本体集団	ほんたいしゅうだん	
\\	本体種別表示	ほんたいしゅべつひょうじ	
\\	本体部	ほんたいぶ	
\\	本体部暗号化表示	ほんたいぶあんごうかひょうじ	
\\	本番運用	ほんばんうんよう	
\\	翻訳解釈実行プログラム指示文	ほんやくかいしゃくじっこうプログラムしじぶん	
\\	翻訳指示文	ほんやくしじぶん	
\\	翻訳時	ほんやくじ	
\\	翻訳時間	ほんやくじかん	
\\	翻訳単位のプログラム	ほんやくたんいのプログラム	
\\	翻訳用計算機	ほんやくようけいさんき	
\\	翻訳用計算機記述項	ほんやくようけいさんききじゅつこう	
\\	翻訳アセンブラ指示文	ほんやくアセンブラしじぶん	
\\	翻訳コンパイラ指示文	ほんやくコンパイラしじぶん	
\\	翻訳ソフトウェア	ほんやくソフトウェア	
\\	翻訳テーブル	ほんやくテーブル	
\\	翻訳プログラム	ほんやくプログラム	
\\	翻訳プログラム指示文	ほんやくプログラムしじぶん	
\\	翻訳ルーチン	ほんやくルーチン	
\\	本来受信者	ほんらいじゅしんしゃ	
\\	本システム	ほんシステム	
\\	本ドキュメント	ほんドキュメント	
\\	本メール	ほんメール	
\\	防護領域	ぼうごりょういき	
\\	防爆形	ぼうばくがた	
\\	母線	ぼせん	
\\	間合せ機能	まあわせきのう	
\\	毎正時データ	まいしょうじデータ	
\\	埋設ケーブル	まいせつケーブル	
\\	前置パス名	まえおきパスめい	
\\	前向き推論	まえむきすいろん	
\\	膜型キーボード	まくがたキーボード	
\\	孫局	まごきょく	
\\	摩擦送り	まさつおくり	
\\	待ち行列	まちぎょうれつ	
\\	待ち行列方式	まちぎょうれつほうしき	
\\	待ち行列名	まちぎょうれつめい	
\\	待ち行列理論	まちぎょうれつりろん	
\\	待ち状態	まちじょうたい	
\\	末端成分	まったんせいぶん	
\\	末そろえ	まつそろえ	
\\	末尾方式コード	まつびほうしきコード	
\\	窓表示域変換	まどひょうじいきへんかん	
\\	魔方陣行列	まほうじんぎょうれつ	
\\	丸漢フォント	まるかんフォント	
\\	丸め	まるめ	
\\	丸め誤差	まるめごさ	
\\	回り込み	まわりこみ	
\\	満了日付	まんりょうひつけ	
\\	見掛け変数	みかけへんすう	
\\	未規定	みきてい	
\\	右上隅	みぎうえすみ	
\\	右下隅	みぎしたすみ	
\\	右揃え	みぎそろえ	
\\	右大括弧	みぎだいかっこ	
\\	右手座標系	みぎてざひょうけい	
\\	右シフト	みぎシフト	
\\	見出しラベル	みだしラベル	
\\	見出しレコード	みだしレコード	
\\	未知状態	みちじょうたい	
\\	密閉型	みっぺいがた	
\\	密結合	みつけつごう	
\\	密結合システム	みつけつごうシステム	
\\	未定義値	みていぎち	
\\	未定義の動作	みていぎのどうさ	
\\	未登録所有者識別子	みとうろくしょゆうしゃしきべつし	
\\	未登録アクセス	みとうろくアクセス	
\\	見計らい送本	みはからいそうほん	
\\	魅力的機能	みりょくてききのう	
\\	無演算命令	むえんざんめいれい	
\\	無音検出	むおんけんしゅつ	
\\	無音抑制	むおんよくせい	
\\	無限遠点	むげんえんてん	
\\	無効SPDU	むこうエスピーディーユー	
\\	無効キー条件	むこうキーじょうけん	
\\	無効TPDU	むこうティーピーディーユー	
\\	無効ビット	むこうビット	
\\	無効PPDU	むこうピーピーディーユー	
\\	無効フレーム	むこうフレーム	
\\	無差別モード	むさべつモード	
\\	無指定ビットレート	むしていビットレート	
\\	無償収集	むしょうしゅうしゅう	
\\	矛盾データ	むじゅんデータ	
\\	無条件飛越し命令	むじょうけんとびこしめいれい	
\\	無条件文	むじょうけんぶん	
\\	無条件分岐	むじょうけんぶんき	
\\	無人端末	むじんたんまつ	
\\	無人モード	むじんモード	
\\	無線方式	むせんほうしき	
\\	無線LAN	むせんラン	
\\	無線リンク	むせんリンク	
\\	無駄時間	むだじかん	
\\	無通話時	むつうわじ	
\\	無停電電源	むていでんでんげん	
\\	無停電電源装置	むていでんでんげんそうち	
\\	無手順	むてじゅん	
\\	無変換	むへんかん	
\\	無変換入力	むへんかんにゅうりょく	
\\	明確化	めいかくか	
\\	名称集	めいしょうしゅう	
\\	名称変更	めいしょうへんこう	
\\	名称領域	めいしょうりょういき	
\\	名称領域名	めいしょうりょういきめい	
\\	明示的引用仕様	めいじてきいんようしよう	
\\	明示的順方向輻輳通知	めいじてきじゅんほうこうふくそうつうち	
\\	明示的並列命令コンピューティング技術	めいじてきへいれつめいれいコンピューティングぎじゅつ	
\\	明示内容参照	めいじないようさんしょう	
\\	明示範囲符	めいじはんいふ	
\\	明示番地指定	めいじばんちしてい	
\\	明示表現	めいじひょうげん	
\\	明示輻輳通知	めいじふくそうつうち	
\\	明示変換	めいじへんかん	
\\	明示連結	めいじれんけつ	
\\	明示アドレシング	めいじアドレシング	
\\	明示アドレス指定	めいじアドレスしてい	
\\	命名機関	めいめいきかん	
\\	命名規則	めいめいきそく	
\\	命名副領域	めいめいふくりょういき	
\\	命名方法引数	めいめいほうほうひきすう	
\\	命名領域	めいめいりょういき	
\\	名目転送速度	めいもくてんそうそくど	
\\	命令	めいれい	
\\	命令形式	めいれいけいしき	
\\	命令再試行	めいれいさいしこう	
\\	命令実行段階	めいれいじっこうだんかい	
\\	命令制御ユニット	めいれいせいぎょユニット	
\\	命令取出し段階	めいれいとりだしだんかい	
\\	命令取り出しサイクル	めいれいとりだしサイクル	
\\	命令アドレスレジスタ	めいれいアドレスレジスタ	
\\	命令コード	めいれいコード	
\\	命令サイクル	めいれいサイクル	
\\	命令セット	めいれいセット	
\\	命令セットアーキテクチャ	めいれいセットアーキテクチャ	
\\	命令ポインタレジスタ	めいれいポインタレジスタ	
\\	命令レジスタ	めいれいレジスタ	
\\	雌コネクタ	めすコネクタ	
\\	面オクテット	めんオクテット	
\\	網安定性	もうあんていせい	
\\	網管理	もうかんり	
\\	網効率	もうこうりつ	
\\	網識別子	もうしきべつし	
\\	網制御装置	もうせいぎょそうち	
\\	網エレメント	もうエレメント	
\\	網リソース	もうリソース	
\\	模擬	もぎ	
\\	黙示属性	もくしぞくせい	
\\	目的言語	もくてきげんご	
\\	目的要素	もくてきようそ	
\\	目的プログラム	もくてきプログラム	
\\	目的モジュール	もくてきモジュール	
\\	目標シソーラス	もくひょうシソーラス	
\\	目標デスクリプタ	もくひょうデスクリプタ	
\\	目標プログラム	もくひょうプログラム	
\\	文字位置	もじいち	
\\	文字落ち	もじおち	
\\	文字回転	もじかいてん	
\\	文字回転角	もじかいてんかく	
\\	文字間	もじかん	
\\	文字間隔	もじかんかく	
\\	文字型特殊ファイル	もじがたとくしゅファイル	
\\	文字記憶単位	もじきおくたんい	
\\	文字基準線	もじきじゅんせん	
\\	文字基準方向	もじきじゅんほうこう	
\\	文字境界	もじきょうかい	
\\	文字食込み	もじくいこみ	
\\	文字項目	もじこうもく	
\\	文字参照	もじさんしょう	
\\	文字集合	もじしゅうごう	
\\	文字修飾	もじしゅうしょく	
\\	文字種別	もじしゅべつ	
\\	文字進行方向	もじしんこうほうこう	
\\	文字実体集合	もじじったいしゅうごう	
\\	文字順列	もじじゅんれつ	
\\	文字数/インチ	もじすうインチ	
\\	文字操作	もじそうさ	
\\	文字像	もじぞう	
\\	文字高さ	もじたかさ	
\\	文字立ち上がり方向	もじたちあがりほうこう	
\\	文字定数	もじていすう	
\\	文字認識	もじにんしき	
\\	文字の大きさ	もじのおおきさ	
\\	文字の高さ	もじのたかさ	
\\	文字の大小順序	もじのだいしょうじゅんじょ	
\\	文字の幅	もじのはば	
\\	文字発生器	もじはっせいき	
\\	文字幅	もじはば	
\\	文字幅拡大率	もじはばかくだいりつ	
\\	文字化け	もじばけ	
\\	文字番号	もじばんごう	
\\	文字表示装置	もじひょうじそうち	
\\	文字符号データ要素	もじふごうデータようそ	
\\	文字列	もじれつ	
\\	文字列型	もじれつがた	
\\	文字列束表	もじれつたばひょう	
\\	文字列入力装置	もじれつにゅうりょくそうち	
\\	文字列の長さ	もじれつのながさ	
\\	文字列のフォントと表示精度	もじれつのフォントとひょうじせいど	
\\	文字枠	もじわく	
\\	文字クラス	もじクラス	
\\	文字セット	もじセット	
\\	文字テキスト	もじテキスト	
\\	文字データ	もじデータ	
\\	文字データ実体	もじデータじったい	
\\	文字ピッチ	もじピッチ	
\\	文字レパートリ	もじレパートリ	
\\	戻り	もどり	
\\	戻り値	もどりち	
\\	漏れ電流	もれでんりゅう	
\\	文書館学	もんじょかんがく	
\\	問題記述	もんだいきじゅつ	
\\	問題定義	もんだいていぎ	
\\	問題向き言語	もんだいむきげんご	
\\	焼き付き	やきつき	
\\	焼きなまし法・焼き鈍し法	やきなましほう	
\\	矢印キー	やじるしキー	
\\	山登り法	やまのぼりほう	
\\	有意瞬間	ゆういしゅんかん	
\\	有意状態	ゆういじょうたい	
\\	遊休時間	ゆうきゅうじかん	
\\	有機EL	ゆうきイーエル	
\\	有限状態機械	ゆうげんじょうたいきかい	
\\	有限オートマトン	ゆうげんオートマトン	
\\	有効域	ゆうこういき	
\\	有効化	ゆうこうか	
\\	有効けた	ゆうこうけた	
\\	有効桁	ゆうこうけた	
\\	有効けた演算	ゆうこうけたえんざん	
\\	有効試験事象	ゆうこうしけんじしょう	
\\	有効状態	ゆうこうじょうたい	
\\	有効範囲	ゆうこうはんい	
\\	有効SPDU	ゆうこうエスピーディーユー	
\\	有効データ転送速度	ゆうこうデータてんそうそくど	
\\	有向ビーム表示装置	ゆうこうビームひょうじそうち	
\\	有効PPDU	ゆうこうピーピーディーユー	
\\	有償収集	ゆうしょうしゅうしゅう	
\\	優先語	ゆうせんご	
\\	優先使用語	ゆうせんしようご	
\\	優先度	ゆうせんど	
\\	優先Nサービスデータ単位	ゆうせんエンサービスデータたんい	
\\	優先リンクセット	ゆうせんリンクセット	
\\	誘電性プラスチック	ゆうでんせいプラスチック	
\\	郵便ORアドレス	ゆうびんオーアーアドレス	
\\	郵便ORアドレス構成要素	ゆうびんオーアーアドレスこうせいようそ	
\\	有利化	ゆうりか	
\\	輸出権	ゆしゅつけん	
\\	輸送問題	ゆそうもんだい	
\\	要求側	ようきゅうがわ	
\\	要求側遠隔操作プロトコル機械	ようきゅうがわえんかくそうさプロトコルきかい	
\\	要求側アソシエーション制御プロトコル機械	ようきゅうがわアソシエーションせいぎょプロトコルきかい	
\\	要求側SS利用者	ようきゅうがわエスエスりようしゃ	
\\	要求時ページング	ようきゅうじページング	
\\	要求配信方法	ようきゅうはいしんほうほう	
\\	要求分解	ようきゅうぶんかい	
\\	要求分析	ようきゅうぶんせき	
\\	要求モード	ようきゅうモード	
\\	ようこそ画面	ようこそがめん	
\\	用語解説	ようごかいせつ	
\\	用語索引	ようごさくいん	
\\	用語の抽出	ようごのちゅうしゅつ	
\\	用語表	ようごひょう	
\\	用紙送り	ようしおくり	
\\	用紙切れ	ようしきれ	
\\	用紙速送り	ようしはやおくり	
\\	用紙スタッカー	ようしスタッカー	
\\	用紙トレイ	ようしトレイ	
\\	要素	ようそ	
\\	要素位置	ようそいち	
\\	要素型	ようそがた	
\\	要素型定義	ようそがたていぎ	
\\	要素型引数	ようそがたひきすう	
\\	要素構造	ようそこうぞう	
\\	要素参照リスト	ようそさんしょうリスト	
\\	要素集合	ようそしゅうごう	
\\	要素種別	ようそしゅべつ	
\\	要素数	ようそすう	
\\	要素宣言	ようそせんげん	
\\	要素別処理	ようそべつしょり	
\\	要素ポインタ	ようそポインタ	
\\	要望クラス	ようぼうクラス	
\\	溶融型熱転写プリンタ	ようゆうがたねつてんしゃプリンタ	
\\	揺らん期本・揺籃期本	ようらんきぼん	
\\	容量集合	ようりょうしゅうごう	
\\	抑止信号	よくししんごう	
\\	予見外試験結果	よけんがいしけんけっか	
\\	予見試験結果	よけんしけんけっか	
\\	横倍角	よこばいかく	
\\	余剰パルス	よじょうパルス	
\\	呼ばれるプログラム	よばれるプログラム	
\\	呼出側	よびだしがわ	
\\	呼び出し時間・呼出時間	よびだしじかん	
\\	呼び出し法	よびだしほう	
\\	呼び出し命令	よびだしめいれい	
\\	呼出し列	よびだしれつ	
\\	呼ぶプログラム	よぶプログラム	
\\	予防保守	よぼうほしゅ	
\\	予防保全	よぼうほぜん	
\\	読み込み	よみこみ	
\\	読み込み許可	よみこみきょか	
\\	読み出し・読出し	よみだし	
\\	読み出し専用メモリ	よみだしせんようメモリ	
\\	読み出し端末	よみだしたんまつ	
\\	読み出し保護	よみだしほご	
\\	読取り書込み開口部	よみとりかきこみかいこうぶ	
\\	読取り書込み記憶装置	よみとりかきこみきおくそうち	
\\	読取り書込みサイクル時間	よみとりかきこみサイクルじかん	
\\	読取り書込みスロット	よみとりかきこみスロット	
\\	読取り書込みヘッド	よみとりかきこみヘッド	
\\	読み取り機・読取機・読み取り器・読取り機・読取器・読取り器	よみとりき	
\\	読み取り専用	よみとりせんよう	
\\	読取り専用記憶装置	よみとりせんようきおくそうち	
\\	読取り専用ファイルシステム	よみとりせんようファイルシステム	
\\	読み取り専用メモリ	よみとりせんようメモリ	
\\	読み取り専用モード	よみとりせんようモード	
\\	読取り通路	よみとりつうろ	
\\	読取り部	よみとりぶ	
\\	読取りサイクル時間	よみとりサイクルじかん	
\\	読取りヘッド	よみとりヘッド	
\\	予約名	よやくめい	
\\	予約名使用引数	よやくめいしようひきすう	
\\	予約ファイル名	よやくファイルめい	
\\	より対線	よりついせん	
\\	乱数生成	らんすうせいせい	
\\	乱数列	らんすうれつ	
\\	乱呼出し	らんよびだし	
\\	離散的表現	りさんてきひょうげん	
\\	離散的データ	りさんてきデータ	
\\	離散コサイン変換	りさんコサインへんかん	
\\	立体集合演算	りったいしゅうごうえんざん	
\\	略字方式コード	りゃくじほうしきコード	
\\	略成語	りゃくせいご	
\\	略記組合せ比較条件	りゃっきくみあわせひかくじょうけん	
\\	領域確保	りょういきかくほ	
\\	領域型出力基本要素	りょういきがたしゅつりょくきほんようそ	
\\	領域集合	りょういきしゅうごう	
\\	領域束表	りょういきたばひょう	
\\	領域定義属性	りょういきていぎぞくせい	
\\	領域塗りつぶし	りょういきぬりつぶし	
\\	料金体系	りょうきんたいけい	
\\	量子化雑音	りょうしかざつおん	
\\	量集合	りょうしゅうごう	
\\	量子コンピュータ	りょうしコンピュータ	
\\	両方向印刷	りょうほうこういんさつ	
\\	両方向交互相互動作	りょうほうこうこうごそうごどうさ	
\\	両方向交互通信	りょうほうこうこうごつうしん	
\\	両方向交互データ通信	りょうほうこうこうごデータつうしん	
\\	両方向同時相互動作	りょうほうこうどうじそうごどうさ	
\\	両方向同時通信	りょうほうこうどうじつうしん	
\\	両方向同時データ通信	りょうほうこうどうじデータつうしん	
\\	両面ディスク	りょうめんディスク	
\\	利用可能	りようかのう	
\\	利用可能時間	りようかのうじかん	
\\	利用可能領域	りようかのうりょういき	
\\	利用可能ビットレート	りようかのうビットレート	
\\	利用者誤り指示未完了	りようしゃあやまりしじみかんりょう	
\\	利用者誤り要求未完了	りようしゃあやまりようきゅうみかんりょう	
\\	利用者機能	りようしゃきのう	
\\	利用者機能体	りようしゃきのうたい	
\\	利用者語	りようしゃご	
\\	利用者座標	りようしゃざひょう	
\\	利用者証明証	りようしゃしょうめいしょう	
\\	利用者定義演算	りようしゃていぎえんざん	
\\	利用者定義代入文	りようしゃていぎだいにゅうぶん	
\\	利用者の登録	りようしゃのとうろく	
\\	利用者の認証	りようしゃのにんしょう	
\\	利用者名	りようしゃめい	
\\	利用者ID	りようしゃアイディー	
\\	利用者ASE	りようしゃエーエスイー	
\\	利用者サービスクラス	りようしゃサービスクラス	
\\	利用者マニュアル	りようしゃマニュアル	
\\	利用頻度	りようひんど	
\\	利用便覧	りようびんらん	
\\	履歴情報	りれきじょうほう	
\\	履歴データ	りれきデータ	
\\	輪郭線要素	りんかくせんようそ	
\\	臨時従業員	りんじじゅうぎょういん	
\\	隣接局	りんせつきょく	
\\	隣接局通知	りんせつきょくつうち	
\\	隣接節点	りんせつせってん	
\\	隣接単位料金区域	りんせつたんいりょうきんくいき	
\\	隣接定義域	りんせつていぎいき	
\\	隣接ドメイン	りんせつドメイン	
\\	隣接ノード	りんせつノード	
\\	類義性	るいぎせい	
\\	累計メモリ	るいけいメモリ	
\\	累算器	るいさんき	
\\	類似製品	るいじせいひん	
\\	累積索引付け	るいせきさくいんつけ	
\\	累積差率	るいせきさりつ	
\\	累積分布関数	るいせきぶんぷかんすう	
\\	累積率	るいせきりつ	
\\	累積和	るいせきわ	
\\	例外条件	れいがいじょうけん	
\\	例外状態	れいがいじょうたい	
\\	暦週	れきしゅう	
\\	暦日付	れきひづけ	
\\	劣化故障	れっかこしょう	
\\	列挙分類体系	れっきょぶんるいたいけい	
\\	列番号	れつばんごう	
\\	列変数	れつへんすう	
\\	列方行奇偶検査	れつほうこうきぐうけんさ	
\\	列ピッチ	れつピッチ	
\\	連係編集	れんけいへんしゅう	
\\	連係編集プログラム	れんけいへんしゅうプログラム	
\\	連結演算子	れんけつえんざんし	
\\	連結型宣言	れんけつがたせんげん	
\\	連結型宣言部分集合	れんけつがたせんげんぶぶんしゅうごう	
\\	連結式	れんけつしき	
\\	連結集合	れんけつしゅうごう	
\\	連結集合宣言	れんけつしゅうごうせんげん	
\\	連結処理	れんけつしょり	
\\	連結処理定義	れんけつしょりていぎ	
\\	連結性	れんけつせい	
\\	連結操作	れんけつそうさ	
\\	連結属性	れんけつぞくせい	
\\	連結網	れんけつもう	
\\	連結網表示	れんけつもうひょうじ	
\\	連結リスト	れんけつリスト	
\\	連結リスト探索	れんけつリストたんさく	
\\	連言	れんごん	
\\	連鎖シーケンス	れんさシーケンス	
\\	連鎖リスト	れんさリスト	
\\	連鎖リスト探索	れんさリストたんさく	
\\	連想関係	れんそうかんけい	
\\	連想記憶装置	れんそうきおくそうち	
\\	連想検索	れんそうけんさく	
\\	連想配列	れんそうはいれつ	
\\	連続関係	れんぞくかんけい	
\\	連続項目	れんぞくこうもく	
\\	連続紙	れんぞくし	
\\	連続帳票	れんぞくちょうひょう	
\\	連続分布	れんぞくぶんぷ	
\\	連続用紙	れんぞくようし	
\\	連投	れんとう	
\\	連文節変換	れんぶんせつへんかん	
\\	連絡節	れんらくせつ	
\\	露光装置	ろこうそうち	
\\	論理演算	ろんりえんざん	
\\	論理演算機構	ろんりえんざんきこう	
\\	論理演算子	ろんりえんざんし	
\\	論理演算装置	ろんりえんざんそうち	
\\	論理回路	ろんりかいろ	
\\	論理型	ろんりがた	
\\	論理型言語	ろんりがたげんご	
\\	論理機構	ろんりきこう	
\\	論理記号	ろんりきごう	
\\	論理けた送り	ろんりけたおくり	
\\	論理構成	ろんりこうせい	
\\	論理構造	ろんりこうぞう	
\\	論理構造編集処理	ろんりこうぞうへんしゅうしょり	
\\	論理順番	ろんりじゅんばん	
\\	論理図	ろんりず	
\\	論理積	ろんりせき	
\\	論理積演算	ろんりせきえんざん	
\\	論理積結合	ろんりせきけつごう	
\\	論理積素子	ろんりせきそし	
\\	論理設計	ろんりせっけい	
\\	論理装置	ろんりそうち	
\\	論理素子	ろんりそし	
\\	論理対象体	ろんりたいしょうたい	
\\	論理対象体クラス	ろんりたいしょうたいクラス	
\\	論理的否定	ろんりてきひてい	
\\	論理入力装置	ろんりにゅうりょくそうち	
\\	論理爆弾	ろんりばくだん	
\\	論理番号	ろんりばんごう	
\\	論理比較	ろんりひかく	
\\	論理否定演算	ろんりひていえんざん	
\\	論理否定回路	ろんりひていかいろ	
\\	論理矛盾	ろんりむじゅん	
\\	論理和素子	ろんりわそし	
\\	論理アドレス	ろんりアドレス	
\\	論理エラー	ろんりエラー	
\\	論理ゲート	ろんりゲート	
\\	論理コントローラ	ろんりコントローラ	
\\	論理シフト	ろんりシフト	
\\	論理ドライブ	ろんりドライブ	
\\	論理ノード	ろんりノード	
\\	論理パス	ろんりパス	
\\	論理フォーマット	ろんりフォーマット	
\\	論理ブロック	ろんりブロック	
\\	論理ブロックアドレス	ろんりブロックアドレス	
\\	論理プログラミング	ろんりプログラミング	
\\	論理ページ	ろんりページ	
\\	論理ユニット	ろんりユニット	
\\	論理ユニット番号	ろんりユニットばんごう	
\\	論理リンク	ろんりリンク	
\\	論理リンク制御	ろんりリンクせいぎょ	
\\	論理リンク制御副層	ろんりリンクせいぎょふくそう	
\\	論理リンク制御プロトコル	ろんりリンクせいぎょプロトコル	
\\	論理リング	ろんりリング	
\\	論理レコード	ろんりレコード	
\\	論理レベル	ろんりレベル	
\\	割り当て・割当て	わりあて	
\\	割り込み	わりこみ	
\\	割込み記号	わりこみきごう	
\\	割り込み信号	わりこみしんごう	
\\	割込み分類	わりこみぶんるい	
\\	割り込み方式	わりこみほうしき	
\\	割り込み要求	わりこみようきゅう	
\\	割り込みコントローラ	わりこみコントローラ	
\\	割り込みサービスルーチン	わりこみサービスルーチン	
\\	割り込みハンドラ	わりこみハンドラ	
\\	割り込みベクタ	わりこみベクタ	
\\	割り込みベクトル	わりこみベクトル	
\\	割込みレジスタ	わりこみレジスタ	
\\	割付け構造	わりつけこうぞう	
\\	割付け処理	わりつけしょり	
\\	割付け順番	わりつけじゅんばん	
\\	割付け対象体	わりつけたいしょうたい	
\\	割付け対象体クラス	わりつけたいしょうたいクラス	
\\	割付け体裁	わりつけていさい	
\\	割付けの流れ	わりつけのながれ	
\\	割付け配列	わりつけはいれつ	
\\	割付け類別	わりつけるいべつ	
\\	割引サービス	わりびきサービス	
\\	割れ	われ	
\\	アイコン化	アイコンか	
\\	勧告	アイティーユーかんこく	
\\	アイドル時間	アイドルじかん	
\\	アイドル状態	アイドルじょうたい	
\\	電話	アイピーでんわ	
\\	アウトオブオーダー実行	アウトオブオーダーじっこう	
\\	アク禁	アクきん	
\\	アクセス回線	アクセスかいせん	
\\	アクセス可能	アクセスかのう	
\\	アクセス管理	アクセスかんり	
\\	アクセス機構	アクセスきこう	
\\	アクセス許可	アクセスきょか	
\\	アクセス系	アクセスけい	
\\	アクセス権限	アクセスけんげん	
\\	アクセス失敗率	アクセスしっぱいりつ	
\\	アクセス時間	アクセスじかん	
\\	アクセス情報	アクセスじょうほう	
\\	アクセス制御	アクセスせいぎょ	
\\	アクセス制御一覧	アクセスせいぎょいちらん	
\\	アクセス制御語	アクセスせいぎょご	
\\	アクセス制御フィールド	アクセスせいぎょフィールド	
\\	アクセス単位	アクセスたんい	
\\	アクセス点	アクセスてん	
\\	アクセス番号	アクセスばんごう	
\\	アクセス頻度が低いファイル	アクセスひんどがひくいファイル	
\\	アクセス不能	アクセスふのう	
\\	アクセス法	アクセスほう	
\\	アクセス方式	アクセスほうしき	
\\	アクセス保護手続き	アクセスほごてつづき	
\\	アクセス網	アクセスもう	
\\	アクセス料金	アクセスりょうきん	
\\	アクセス例外	アクセスれいがい	
\\	アクセント付き文字	アクセントつきもじ	
\\	アクティビティ属性	アクティビティぞくせい	
\\	アクティブ化	アクティブか	
\\	アクティブマトリクス方式	アクティブマトリクスほうしき	
\\	アクティブロジック機能	アクティブロジックきのう	
\\	アジマス角	アジマスかく	
\\	アス比	アスひ	
\\	アスペクト比	アスペクトひ	
\\	アセンブラ制御命令	アセンブラせいぎょめいれい	
\\	アセンブラマクロ変換支援プログラム	アセンブラマクロへんかんしえんプログラム	
\\	アセンブリ言語・アセンブラ言語・アセンブリー言語・アセンブラー言語	アセンブリげんご・アセンブラげんご・アセンブリーげんご・アセンブラーげんご	
\\	アセンブル時	アセンブルじ	
\\	アセンブル時間	アセンブルじかん	
\\	アセンブル編集	アセンブルへんしゅう	
\\	アソシエーション応答側	アソシエーションおうとうがわ	
\\	アソシエーション起動側	アソシエーションきどうがわ	
\\	アソシエーション制御サービス要素	アソシエーションせいぎょサービスようそ	
\\	アダプテーション機能	アダプテーションきのう	
\\	アダプテーション層	アダプテーションそう	
\\	アップグレード手順	アップグレードてじゅん	
\\	アップグレード版	アップグレードはん	
\\	アドレシング可能	アドレシングかのう	
\\	アドレス解決	アドレスかいけつ	
\\	アドレス解決プロトコル	アドレスかいけつプロトコル	
\\	アドレス可能点	アドレスかのうてん	
\\	アドレス可能カーソル	アドレスかのうカーソル	
\\	アドレス管理	アドレスかんり	
\\	アドレス空間	アドレスくうかん	
\\	アドレス計算	アドレスけいさん	
\\	アドレス形式	アドレスけいしき	
\\	アドレス構成要素	アドレスこうせいようそ	
\\	アドレス指定能力	アドレスしていのうりょく	
\\	アドレス指定例外	アドレスしていれいがい	
\\	アドレス修飾	アドレスしゅうしょく	
\\	アドレス情報	アドレスじょうほう	
\\	アドレス生成	アドレスせいせい	
\\	アドレス定数	アドレスていすう	
\\	アドレス部	アドレスぶ	
\\	アドレス部拡張	アドレスぶかくちょう	
\\	アドレス変換	アドレスへんかん	
\\	アドレス変換機構	アドレスへんかんきこう	
\\	アドレス変換例外	アドレスへんかんれいがい	
\\	アドレス変更	アドレスへんこう	
\\	アドレス変更子	アドレスへんこうし	
\\	アナログ回線	アナログかいせん	
\\	アナログ回路	アナログかいろ	
\\	アナログ掛算器	アナログかけざんき	
\\	アナログ加算器	アナログかさんき	
\\	アナログ計算機	アナログけいさんき	
\\	アナログ信号	アナログしんごう	
\\	アナログ乗算器	アナログじょうざんき	
\\	アナログ除算器	アナログじょざんき	
\\	アナログ装置	アナログそうち	
\\	アナログ通信路	アナログつうしんろ	
\\	アナログ伝送	アナログでんそう	
\\	アナログ入力チャネル	アナログにゅうりょくチャネル	
\\	アナログ表現	アナログひょうげん	
\\	アナログ表示	アナログひょうじ	
\\	アナログ変数	アナログへんすう	
\\	アナログディジタル変換器	アナログディジタルへんかんき	
\\	アナログデジタル変換器	アナログデジタルへんかんき	
\\	アフィン変換	アフィンへんかん	
\\	アプリオリ法	アプリオリほう	
\\	アプリケーション開発	アプリケーションかいはつ	
\\	アプリケーション開発環境	アプリケーションかいはつかんきょう	
\\	アプリケーション開発言語	アプリケーションかいはつげんご	
\\	アプリケーション開発者	アプリケーションかいはつしゃ	
\\	アプリケーション開発システム	アプリケーションかいはつシステム	
\\	アプリケーション開発ツール	アプリケーションかいはつツール	
\\	アプリケーション設計プロセス	アプリケーションせっけいプロセス	
\\	アプリケーション層	アプリケーションそう	
\\	アプリケーション部	アプリケーションぶ	
\\	アラーム検出	アラームけんしゅつ	
\\	アラーム重要度	アラームじゅうようど	
\\	アルファ版	アルファばん	
\\	アルファベットの語	アルファベットのご	
\\	アルファベット表記法	アルファベットひょうきほう	
\\	アレイ処理	アレイしょり	
\\	アレイ処理機構	アレイしょりきこう	
\\	アレイ処理装置	アレイしょりそうち	
\\	アングル括弧	アングルかっこ	
\\	アンダフロー表示	アンダフローひょうじ	
\\	アンチコピー技術	アンチコピーぎじゅつ	
\\	アンテナ利得率	アンテナりとくりつ	
\\	アンド回路・AND回路	アンドかいろ	
\\	アンナンバード接続	アンナンバードせつぞく	
\\	アンパック10進表記法・アンパック十進表記法	アンパックじっしんひょうきほう	
\\	アンビエント光	アンビエントこう	
\\	アーカイブ先	アーカイブさき	
\\	アーカイブ属性	アーカイブぞくせい	
\\	アーカイブ対象ファイル	アーカイブたいしょうファイル	
\\	アース端子付きプラグ	アースたんしづきプラグ	
\\	アース用ストラップ	アースようストラップ	
\\	言語	アールげんご	
\\	イコール機能	イコールきのう	
\\	イネーブル信号	イネーブルしんごう	
\\	イベント駆動	イベントくどう	
\\	イベント駆動型	イベントくどうがた	
\\	イベント指定子	イベントしていし	
\\	イベント処理	イベントしょり	
\\	イベント報告	イベントほうこく	
\\	イメージ検索	イメージけんさく	
\\	イメージ処理	イメージしょり	
\\	イリジウム計画	イリジウムけいかく	
\\	インクジェット印字装置	インクジェットいんじそうち	
\\	インクジェット方式	インクジェットほうしき	
\\	インサート編集	インサートへんしゅう	
\\	インスタンス化	インスタンスか	
\\	インスタンス変数	インスタンスへんすう	
\\	インストール可能デバイスドライバ	インストールかのうデバイスドライバ	
\\	インタフェース種別	インタフェースしゅべつ	
\\	インタフェース定義言語	インタフェースていぎげんご	
\\	インタプリタ型言語	インタプリタかたげんご	
\\	インタプリータ型言語	インタプリータかたげんご	
\\	インターネット閲覧ソフト	インターネットえつらんソフト	
\\	インターネット技術推進グループ	インターネットぎじゅつすいしんグループ	
\\	インターネット技術特別調査委員会	インターネットぎじゅつとくべつちょうさいいんかい	
\\	インターネット研究推進グループ	インターネットけんきゅうすいしんグループ	
\\	インターネット研究特別調査委員会	インターネットけんきゅうとくべつちょうさいいんかい	
\\	インターネット検索	インターネットけんさく	
\\	インターネット接続	インターネットせつぞく	
\\	インターネット接続共有	インターネットせつぞくきょうゆう	
\\	インターネット電話	インターネットでんわ	
\\	インタープリタ型言語	インタープリタかたげんご	
\\	インテリジェント装置	インテリジェントそうち	
\\	インテリジェント端末	インテリジェントたんまつ	
\\	インデックス付きアドレス	インデックスつきアドレス	
\\	インデックス付きサーチ	インデックスつきサーチ	
\\	インピーダンス管理グラウンド層	インピーダンスかんりグラウンドそう	
\\	インフィックス表記法	インフィックスひょうきほう	
\\	インライン画像	インラインがぞう	
\\	インライン展開	インラインてんかい	
\\	インライン変換	インラインへんかん	
\\	ウィルス隔離	ウィルスかくり	
\\	ウィンドウ下限	ウィンドウかげん	
\\	ウィンドウ環境	ウィンドウかんきょう	
\\	ウィンドウ上限	ウィンドウじょうげん	
\\	ウィンドウ情報	ウィンドウじょうほう	
\\	ウイルス監視機能	ウイルスかんしきのう	
\\	ウイルス警報	ウイルスけいほう	
\\	ウイルス検出ソフトウェア	ウイルスけんしゅつソフトウェア	
\\	ウイルス防衛用ソフト	ウイルスぼうえいようソフト	
\\	ウイルス防止	ウイルスぼうし	
\\	ウェイト状態	ウェイトじょうたい	
\\	ウェブベース企業管理	ウェブベースきぎょうかんり	
\\	ウォーホール型ワーム	ウォーホールかたワーム	
\\	エクステンドメモリ仕様	エクステンドメモリしよう	
\\	エコー減衰器	エコーげんすいき	
\\	エコー立体	エコーりったい	
\\	エコー領域	エコーりょういき	
\\	エスケープ文字	エスケープもじ	
\\	エスケープ要素	エスケープようそ	
\\	端末	エックスたんまつ	
\\	エッジトリガ方式	エッジトリガほうしき	
\\	エミッタ結合素子	エミッタけつごうそし	
\\	エラー回復	エラーかいふく	
\\	エラー回復ルーチン	エラーかいふくルーチン	
\\	エラー検出符号	エラーけんしゅつふごう	
\\	エラー終了	エラーしゅうりょう	
\\	エラー処理	エラーしょり	
\\	エラー処理手順	エラーしょりてじゅん	
\\	エラー処理方法	エラーしょりほうほう	
\\	エラー処理ルーチン	エラーしょりルーチン	
\\	エラー状態	エラーじょうたい	
\\	エラー状態リスト	エラーじょうたいリスト	
\\	エラー制御	エラーせいぎょ	
\\	エラー制御ロジック	エラーせいぎょロジック	
\\	エラー訂正	エラーていせい	
\\	エラー訂正符号	エラーていせいふごう	
\\	エラー訂正エンコーディング	エラーていせいエンコーディング	
\\	エラー訂正プロトコル	エラーていせいプロトコル	
\\	エラー訂正メモリ	エラーていせいメモリ	
\\	エラーの検出と訂正	エラーのけんしゅつとていせい	
\\	エラー表示	エラーひょうじ	
\\	エラー補正	エラーほせい	
\\	エラー履歴	エラーりれき	
\\	エリア境界	エリアきょうかい	
\\	エルゴード的	エルゴードてき	
\\	エンティティ種類	エンティティしゅるい	
\\	エントリ種別	エントリしゅべつ	
\\	エントリ順データセット	エントリじゅんデータセット	
\\	エントリ状況	エントリじょうきょう	
\\	エントリ情報	エントリじょうほう	
\\	エントリ情報選択	エントリじょうほうせんたく	
\\	エンドエンド通信パス	エンドエンドつうしんパス	
\\	エージェント指向	エージェントしこう	
\\	互換機	エーティーごかんき	
\\	回路・オア回路	オアかいろ	
\\	オクテット列型	オクテットれつがた	
\\	オストワルト表色系	オストワルトひょうしょくけい	
\\	オスメス変換	オスメスへんかん	
\\	オスメス変換器	オスメスへんかんき	
\\	オフショア開発	オフショアかいはつ	
\\	オフセット印刷	オフセットいんさつ	
\\	オフライン記憶装置	オフラインきおくそうち	
\\	オブジェクト記述子型	オブジェクトきじゅつしがた	
\\	オブジェクト技術	オブジェクトぎじゅつ	
\\	オブジェクト識別子	オブジェクトしきべつし	
\\	オブジェクト識別子型	オブジェクトしきべつしがた	
\\	オブジェクト指向	オブジェクトしこう	
\\	オブジェクト指向型データベース	オブジェクトしこうがたデータベース	
\\	オブジェクト指向言語	オブジェクトしこうげんご	
\\	オブジェクト指向インタフェース	オブジェクトしこうインタフェース	
\\	オブジェクト指向グラフィックス	オブジェクトしこうグラフィックス	
\\	オブジェクト指向データベース	オブジェクトしこうデータベース	
\\	オブジェクト指向プログラミング	オブジェクトしこうプログラミング	
\\	オブジェクト修飾	オブジェクトしゅうしょく	
\\	オブジェクトのリンクと埋め込み	オブジェクトのリンクとうめこみ	
\\	オブジェクトデータベース管理システム	オブジェクトデータベースかんりシステム	
\\	オペレーティング環境	オペレーティングかんきょう	
\\	オンザフライ印字装置	オンザフライいんじそうち	
\\	オンザフライ書き込み	オンザフライかきこみ	
\\	オンスクリーン機能	オンスクリーンきのう	
\\	オンデマンドダイヤルアップ接続	オンデマンドダイヤルアップせつぞく	
\\	オンライン科学コンピュータ	オンラインかがくコンピュータ	
\\	オンライン記憶	オンラインきおく	
\\	オンライン機械	オンラインきかい	
\\	オンライン機器	オンラインきき	
\\	オンライン計算処理	オンラインけいさんしょり	
\\	オンライン計測器パッケージ	オンラインけいそくきパッケージ	
\\	オンライン検査機能	オンラインけんさきのう	
\\	オンライン処理	オンラインしょり	
\\	オンライン処理環境	オンラインしょりかんきょう	
\\	オンライン診断	オンラインしんだん	
\\	オンライン状態	オンラインじょうたい	
\\	オンライン情報処理	オンラインじょうほうしょり	
\\	オンライン制御	オンラインせいぎょ	
\\	オンライン端末テスト	オンラインたんまつテスト	
\\	オンライン通信	オンラインつうしん	
\\	オンライン動作状態	オンラインどうさじょうたい	
\\	オンライン配布	オンラインはいふ	
\\	オンライン版	オンラインはん	
\\	オンライン分析	オンラインぶんせき	
\\	オンライン分析処理	オンラインぶんせきしょり	
\\	オンライン申し込み	オンラインもうしこみ	
\\	オンライン問題解決	オンラインもんだいかいけつ	
\\	オンライン利用者	オンラインりようしゃ	
\\	オンラインテスト制御プログラム	オンラインテストせいぎょプログラム	
\\	オンラインディレイド処理システム	オンラインディレイドしょりシステム	
\\	オンラインデータ収集システム	オンラインデータしゅうしゅうシステム	
\\	オンラインデータ処理	オンラインデータしょり	
\\	オンラインデータ入力	オンラインデータにゅうりょく	
\\	オンライントランザクション処理	オンライントランザクションしょり	
\\	オンラインバッチ処理システム	オンラインバッチしょりシステム	
\\	オンラインパターン解析認識システム	オンラインパターンかいせきにんしきシステム	
\\	オンラインプロジェクト管理	オンラインプロジェクトかんり	
\\	オンラインボリューム管理	オンラインボリュームかんり	
\\	オンラインリアルタイム監視	オンラインリアルタイムかんし	
\\	オンラインリアルタイム処理	オンラインリアルタイムしょり	
\\	オンラインリアルタイム処理システム	オンラインリアルタイムしょりシステム	
\\	オートマトン理論	オートマトンりろん	
\\	オーバレイ区分	オーバレイくぶん	
\\	オーバレイ構造	オーバレイこうぞう	
\\	オーバーフロー表示	オーバーフローひょうじ	
\\	オーバーラップウィンドウ方式	オーバーラップウィンドウほうしき	
\\	オープン最短パスファースト	オープンさいたんパスファースト	
\\	オープンファイル記述	オープンファイルきじゅつ	
\\	カスケード式けた上げ	カスケードしきけたあげ	
\\	カスケード表示	カスケードひょうじ	
\\	カット紙	カットし	
\\	カプセル化	カプセルか	
\\	カラー液晶	カラーえきしょう	
\\	カリグラフィック表示装置	カリグラフィックひょうじそうち	
\\	カンマ表示	カンマひょうじ	
\\	カーソル移動キー	カーソルいどうキー	
\\	カード送り機構	カードおくりきこう	
\\	カード型データベース	カードがたデータベース	
\\	カード穿孔機構	カードせんこうきこう	
\\	カード穿孔装置	カードせんこうそうち	
\\	カード通路	カードつうろ	
\\	カード分類機	カードぶんるいき	
\\	カード読取り機構	カードよみとりきこう	
\\	カード読み取り装置・カード読取り装置	カードよみとりそうち	
\\	ガウス消去	ガウスしょうきょ	
\\	ガス放電ディスプレイ	ガスほうでんディスプレイ	
\\	ガリウム砒素リン	ガリウムひそリン	
\\	ガンマ補正	ガンマほせい	
\\	キャッシュ記憶装置	キャッシュきおくそうち	
\\	キャッシュメモリ機構	キャッシュメモリきこう	
\\	キャビネット環境	キャビネットかんきょう	
\\	キャビネット筺体	キャビネットきょうたい	
\\	キャラクタ集合	キャラクタしゅうごう	
\\	キャラクタ定義テーブル	キャラクタていぎテーブル	
\\	キャリア検出	キャリアけんしゅつ	
\\	キャリア検知	キャリアけんち	
\\	キャリア検知多重アクセス衝突回避ネットワーク	キャリアけんちたじゅうアクセスしょうとつかいひネットワーク	
\\	キャリア検知多重アクセス衝突検出ネットワーク	キャリアけんちたじゅうアクセスしょうとつけんしゅつネットワーク	
\\	キャンディデート版	キャンディデートばん	
\\	キュー末尾	キューまつび	
\\	キューイング理論	キューイングりろん	
\\	キュービック補間	キュービックほかん	
\\	キー候補	キーこうほ	
\\	キー照合	キーしょうごう	
\\	キー順データセット	キーじゅんデータセット	
\\	キー入力	キーにゅうりょく	
\\	キー入力検証	キーにゅうりょくけんしょう	
\\	キー入力待ち	キーにゅうりょくまち	
\\	キー割り当て・キー割当	キーわりあて	
\\	キーボード及び外部プログラム入力式計算器	キーボードおよびがいぶプログラムにゅうりょくしきけいさんき	
\\	キーボード制御装置	キーボードせいぎょそうち	
\\	キーボード入力	キーボードにゅうりょく	
\\	キーボードの配置	キーボードのはいち	
\\	キーボードプログラム入力式計算器	キーボードプログラムにゅうりょくしきけいさんき	
\\	キーワード検索	キーワードけんさく	
\\	ギャランティ型	ギャランティかた	
\\	クライアント側	クライアントがわ	
\\	クライアント側アプリケーション	クライアントがわアプリケーション	
\\	クラス階層	クラスかいそう	
\\	クラスタ化されたアプリケーション	クラスタかされたアプリケーション	
\\	クラスタ装置	クラスタそうち	
\\	クラスタ番号	クラスタばんごう	
\\	クラッシュ回復	クラッシュかいふく	
\\	クラッド径	クラッドけい	
\\	クリアエントリー機能	クリアエントリーきのう	
\\	クリック保証	クリックほしょう	
\\	クリティカルパス法	クリティカルパスほう	
\\	クレジットカード与信照会用端末	クレジットカードよしんしょうかいようたんまつ	
\\	クロス開発	クロスかいはつ	
\\	クロス集計	クロスしゅうけい	
\\	クロスプラットフォーム技術	クロスプラットフォームぎじゅつ	
\\	クロスランゲージ情報検索	クロスランゲージじょうほうけんさく	
\\	クロスリファレンス形式	クロスリファレンスけいしき	
\\	クロック供給源	クロックきょうきゅうげん	
\\	クロック周波数	クロックしゅうはすう	
\\	クロック信号	クロックしんごう	
\\	クロック速度	クロックそくど	
\\	クロック発振ロジック	クロックはっしんロジック	
\\	クロック分配	クロックぶんぱい	
\\	クロマクリア管	クロマクリアかん	
\\	グラフ領域	グラフりょういき	
\\	グラフィクス装置	グラフィクスそうち	
\\	グラフィクス中核系	グラフィクスちゅうかくけい	
\\	グラフィック基本要素	グラフィックきほんようそ	
\\	グラフィック文字	グラフィックもじ	
\\	グリフ形状	グリフけいじょう	
\\	グリフ座標系	グリフざひょうけい	
\\	グリフ集合	グリフしゅうごう	
\\	グリフ像	グリフぞう	
\\	グリフ手続き	グリフてつづき	
\\	グリフ手続きインタプリタ	グリフてつづきインタプリタ	
\\	グリフ配置量	グリフはいちりょう	
\\	グリフ表現	グリフひょうげん	
\\	グループ化	グループか	
\\	グループ絶縁	グループぜつえん	
\\	グループ同報	グループどうほう	
\\	グループ分離キャラクタ	グループぶんりキャラクタ	
\\	グループ料金制	グループりょうきんせい	
\\	グローバル形式	グローバルけいしき	
\\	グローバル検索	グローバルけんさく	
\\	グローバル宣言	グローバルせんげん	
\\	グローバル変数	グローバルへんすう	
\\	グローバル名称	グローバルめいしょう	
\\	グローバルネットワークアドレス領域	グローバルネットワークアドレスりょういき	
\\	ケーブルとアース線の配置	ケーブルとアースせんのはいち	
\\	ケーブルの切断	ケーブルのせつだん	
\\	ケーブル配線図	ケーブルはいせんず	
\\	ゲートウェー交換機	ゲートウェーこうかんき	
\\	ゲーム理論	ゲームりろん	
\\	コア技術	コアぎじゅつ	
\\	コア径	コアけい	
\\	コネクション管理	コネクションかんり	
\\	コネクション型	コネクションがた	
\\	コネクション許可制御	コネクションきょかせいぎょ	
\\	コネクション識別子	コネクションしきべつし	
\\	コネクション指向	コネクションしこう	
\\	コネクション制御	コネクションせいぎょ	
\\	コネクション設定	コネクションせってい	
\\	コネクション単位	コネクションたんい	
\\	コネクション統計	コネクションとうけい	
\\	コネクション要求	コネクションようきゅう	
\\	コネクション容量	コネクションようりょう	
\\	コネクショントラヒック記述子	コネクショントラヒックきじゅつし	
\\	コネクションレス型	コネクションレスがた	
\\	コネクションレス型ネットワーク伝送	コネクションレスがたネットワークでんそう	
\\	コピー機器	コピーきき	
\\	コマンド記述ブロック	コマンドきじゅつブロック	
\\	コマンド行	コマンドぎょう	
\\	コマンド行引数	コマンドぎょうひきすう	
\\	コマンド駆動型システム	コマンドくどうがたシステム	
\\	コマンド検索パス	コマンドけんさくパス	
\\	コマンド入力	コマンドにゅうりょく	
\\	コマンド入力型インタフェース	コマンドにゅうりょくがたインタフェース	
\\	コマンド待ち状態	コマンドまちじょうたい	
\\	コマンド名	コマンドめい	
\\	コマンド要求	コマンドようきゅう	
\\	コマンド履歴	コマンドりれき	
\\	コミックストリップ状画像	コミックストリップじょうがぞう	
\\	コミット準備完了状態	コミットじゅんびかんりょうじょうたい	
\\	コミット調整担当	コミットちょうせいたんとう	
\\	コミットメント単位	コミットメントたんい	
\\	コミットメント調整者	コミットメントちょうせいしゃ	
\\	コミットログ記録	コミットログきろく	
\\	コミュニケーション理論	コミュニケーションりろん	
\\	コメント化	コメントか	
\\	コメント欄	コメントらん	
\\	コンテキスト接頭部	コンテキストせっとうぶ	
\\	コンテキスト定義集合	コンテキストていぎしゅうごう	
\\	コンテクスト切り換え	コンテクストきりかえ	
\\	コンテンツ管理システム	コンテンツかんりシステム	
\\	コントローラ番号	コントローラばんごう	
\\	コントローラ保守診断バス	コントローラほしゅしんだんバス	
\\	コントロール機器	コントロールきき	
\\	コントロール文字	コントロールもじ	
\\	コンパイラ型言語	コンパイラかたげんご	
\\	コンパイラ言語	コンパイラげんご	
\\	コンパイラ指示命令	コンパイラしじめいれい	
\\	コンパイラによる最適化	コンパイラによるさいてきか	
\\	コンパイラの最適化能力	コンパイラのさいてきかのうりょく	
\\	コンパイル時	コンパイルじ	
\\	コンパイル時間	コンパイルじかん	
\\	コンパイル済みモジュール	コンパイルすみモジュール	
\\	コンパイル単位	コンパイルたんい	
\\	コンピュータ依存言語	コンピュータいぞんげんご	
\\	コンピュータ化	コンピュータか	
\\	コンピュータ切り替え器	コンピュータきりかえき	
\\	コンピュータ緊急対応センター	コンピュータきんきゅうたいおうセンター	
\\	コンピュータ支援協調作業	コンピュータしえんきょうちょうさぎょう	
\\	コンピュータ資源	コンピュータしげん	
\\	コンピュータ端末	コンピュータたんまつ	
\\	コンピュータ断層撮影	コンピュータだんそうさつえい	
\\	コンピュータ統合生産	コンピュータとうごうせいさん	
\\	コンピュータ犯罪	コンピュータはんざい	
\\	コンピュータ媒介言語	コンピュータばいかいげんご	
\\	コンピュータ用語	コンピュータようご	
\\	コンピューター援用生産	コンピューターえんようせいさん	
\\	コンピューター援用設計	コンピューターえんようせっけい	
\\	コンピューター化	コンピューターか	
\\	コンピューター断層撮影法	コンピューターだんそうさつえいほう	
\\	コンピューター犯罪	コンピューターはんざい	
\\	コンピューティング環境	コンピューティングかんきょう	
\\	コンポジット信号	コンポジットしんごう	
\\	コンポーネント信号	コンポーネントしんごう	
\\	コーディング体系	コーディングたいけい	
\\	コーディング用紙	コーディングようし	
\\	コード化	コードか	
\\	コード化画像	コードかがぞう	
\\	コード拡張文字	コードかくちょうもじ	
\\	コード化集合	コードかしゅうごう	
\\	コード化表現	コードかひょうげん	
\\	コード化文字集合	コードかもじしゅうごう	
\\	コード管理	コードかんり	
\\	コード構成	コードこうせい	
\\	コード設計	コードせっけい	
\\	コード体系	コードたいけい	
\\	コード値	コードち	
\\	コード透過形データ通信	コードとうかがたデータつうしん	
\\	コード独立形データ通信	コードどくりつがたデータつうしん	
\\	コード入力	コードにゅうりょく	
\\	コード変換	コードへんかん	
\\	コード変換器	コードへんかんき	
\\	コード要素	コードようそ	
\\	コードページ切り替え	コードページきりかえ	
\\	コールバック機能	コールバックきのう	
\\	サイクル時間	サイクルじかん	
\\	サインチェンジ機能	サインチェンジきのう	
\\	サウンド機能	サウンドきのう	
\\	サブ垢	サブあか・サブアカ	
\\	サブネット化	サブネットか	
\\	サブネットワーク接続点	サブネットワークせつぞくてん	
\\	サブネットワーク接続点アドレス	サブネットワークせつぞくてんアドレス	
\\	サブノート型パソコン	サブノートがたパソコン	
\\	サブルーチン副プログラム	サブルーチンふくプログラム	
\\	サブルーチン名	サブルーチンめい	
\\	サプレス可	サプレスか	
\\	サンプリング定理	サンプリングていり	
\\	サンプル集団	サンプルしゅうだん	
\\	サンプルホールド装置	サンプルホールドそうち	
\\	サージ耐力	サージたいりょく	
\\	サージ抵抗	サージていこう	
\\	サーチ許可	サーチきょか	
\\	サーチ時間	サーチじかん	
\\	サーチ不可能	サーチふかのう	
\\	サーチ文字列	サーチもじれつ	
\\	サーチエンジン最適化	サーチエンジンさいてきか	
\\	サーチエンジン対策	サーチエンジンたいさく	
\\	サーバ側・サーバー側	サーバがわ・サーバーがわ	
\\	サーバクライアント型	サーバクライアントがた	
\\	サービス拒否攻撃	サービスきょひこうげき	
\\	サービス種別	サービスしゅべつ	
\\	サービス信頼性	サービスしんらいせい	
\\	サービス実現	サービスじつげん	
\\	サービス制御群	サービスせいぎょぐん	
\\	サービス提供区域	サービスていきょうくいき	
\\	サービス提供者	サービスていきょうしゃ	
\\	サービス提供地域	サービスていきょうちいき	
\\	サービス統合ディジタル網	サービスとうごうディジタルもう	
\\	サービスに依存	サービスにいそん	
\\	サービスの復旧	サービスのふっきゅう	
\\	サービスの妨害	サービスのぼうがい	
\\	サービス品質	サービスひんしつ	
\\	サービス品質契約	サービスひんしつけいやく	
\\	サービス便覧	サービスびんらん	
\\	サービス不能攻撃	サービスふのうこうげき	
\\	サービス妨害攻撃	サービスぼうがいこうげき	
\\	サービス要素	サービスようそ	
\\	サービスアクセス点	サービスアクセスてん	
\\	サービスデータ単位	サービスデータたんい	
\\	シェル変数	シェルへんすう	
\\	シグナリング機能	シグナリングきのう	
\\	シグナリング層	シグナリングそう	
\\	シグナル名	シグナルめい	
\\	システム開発	システムかいはつ	
\\	システム監査	システムかんさ	
\\	システム管理	システムかんり	
\\	システム管理応用エンティティ	システムかんりおうようエンティティ	
\\	システム管理応用サービス要素	システムかんりおうようサービスようそ	
\\	システム管理応用プロセス	システムかんりおうようプロセス	
\\	システム管理機能	システムかんりきのう	
\\	システム管理機能単位	システムかんりきのうたんい	
\\	システム管理機能単位パッケージ	システムかんりきのうたんいパッケージ	
\\	システム管理機能領域	システムかんりきのうりょういき	
\\	システム管理者	システムかんりしゃ	
\\	システム管理操作	システムかんりそうさ	
\\	システム管理オブジェクト	システムかんりオブジェクト	
\\	システム管理サービス	システムかんりサービス	
\\	システム概念設計	システムがいねんせっけい	
\\	システム記述	システムきじゅつ	
\\	システム共通領域	システムきょうつうりょういき	
\\	システム支援	システムしえん	
\\	システム識別子	システムしきべつし	
\\	システム試験時間	システムしけんじかん	
\\	システム試験評価計画	システムしけんひょうかけいかく	
\\	システム資源	システムしげん	
\\	システム修復ディスク	システムしゅうふくディスク	
\\	システム障害	システムしょうがい	
\\	システム実動時間	システムじつどうじかん	
\\	システム制御プログラム	システムせいぎょプログラム	
\\	システム生成	システムせいせい	
\\	システム設計	システムせっけい	
\\	システム宣言	システムせんげん	
\\	システム立ち上げ	システムたちあげ	
\\	システム適合性試験報告書	システムてきごうせいしけんほうこくしょ	
\\	システム適合性宣言	システムてきごうせいせんげん	
\\	システム統合	システムとうごう	
\\	システム負荷レベル	システムふかレベル	
\\	システム文書	システムぶんしょ	
\\	システム分析	システムぶんせき	
\\	システム保守	システムほしゅ	
\\	システム名	システムめい	
\\	システム名称	システムめいしょう	
\\	システム文字集合	システムもじしゅうごう	
\\	システム要求	システムようきゅう	
\\	システム要件	システムようけん	
\\	システム量	システムりょう	
\\	システム領域	システムりょういき	
\\	システム論理装置	システムろんりそうち	
\\	システムファイル記述子	システムファイルきじゅつし	
\\	シフト演算機構	シフトえんざんきこう	
\\	シフト機能	シフトきのう	
\\	シャノンの理論	シャノンのりろん	
\\	シャーシ接地	シャーシせっち	
\\	シリアル接続	シリアルせつぞく	
\\	シリアル転送	シリアルてんそう	
\\	シリアル伝送	シリアルでんそう	
\\	シリアル番号	シリアルばんごう	
\\	シルク形式	シルクけいしき	
\\	シングルリンク手順	シングルリンクてじゅん	
\\	シンプレックス法	シンプレックスほう	
\\	言語	シーげんご	
\\	シーク時間	シークじかん	
\\	シーケンス番号	シーケンスばんごう	
\\	言語	シープラスプラスげんご	
\\	ジェネリックなPC互換機	ジェネリックなピーシーごかんき	
\\	ジェネリックセル速度アルゴリズム	ジェネリックセルそくどアルゴリズム	
\\	ジオメトリック平均値	ジオメトリックへいきんち	
\\	ジャックナイフ法	ジャックナイフほう	
\\	ジャバ制御	ジャバせいぎょ	
\\	ジャム信号	ジャムしんごう	
\\	ジャンパ選択可能	ジャンパせんたくかのう	
\\	ジョセフソン接合	ジョセフソンせつごう	
\\	ジョブ固有領域	ジョブこゆうりょういき	
\\	ジョブ制御	ジョブせいぎょ	
\\	ジョブ制御言語	ジョブせいぎょげんご	
\\	ジョブの流れ	ジョブのながれ	
\\	ジョブ番号	ジョブばんごう	
\\	スイッチ間通信	スイッチあいだつうし	
\\	スイッチ状態条件	スイッチじょうたいじょうけん	
\\	スイッチ選択可能	スイッチせんたくかのう	
\\	スイッチ容量	スイッチようりょう	
\\	スイッチング関数	スイッチングかんすう	
\\	スイッチング遅延	スイッチングちえん	
\\	スイッチング電源	スイッチングでんげん	
\\	スイッチング能力	スイッチングのうりょく	
\\	スイッチング変数	スイッチングへんすう	
\\	スカラ値	スカラち	
\\	スカラ変数	スカラへんずう	
\\	スキマティク図	スキマティクず	
\\	スキャンライン法	スキャンラインほう	
\\	スクリプト言語	スクリプトげんご	
\\	スタンダード版	スタンダードはん	
\\	スター型・スター形	スターがた	
\\	スター型ネットワーク	スターがたネットワーク	
\\	スター形配線リングネットワーク	スターけいはいせんリングネットワーク	
\\	スター状	スターじょう	
\\	スタート信号	スタートしんごう	
\\	スティック型ポインティングデバイス	スティックかたポインティングデバイス	
\\	ステム幅整合	ステムはばせいごう	
\\	ステータス行	ステータスぎょう	
\\	ステートメント番号	ステートメントばんごう	
\\	ストップ信号	ストップしんごう	
\\	ストリング変数	ストリングへんすう	
\\	ストリーミングSIMD拡張命令	ストリーミングエスアイエムディーかくちょうめいれい	
\\	ストリーミングテープ駆動機構	ストリーミングテープくどうきこう	
\\	ストレージ構造	ストレージこうぞう	
\\	ストローク入力装置	ストロークにゅうりょくそうち	
\\	ストローク文字発生器	ストロークもじはっせいき	
\\	スプライン曲線	スプラインきょくせん	
\\	スプライン曲面	スプラインきょくめん	
\\	スプライン補間	スプラインほかん	
\\	スペクトラム拡散通信	スペクトラムかくさんつうしん	
\\	スペクトル幅	スペクトルはば	
\\	スペクトル半値幅	スペクトルはんねはば	
\\	スポット光源	スポットこうげん	
\\	スポット穿孔機	スポットせんこうき	
\\	スリム型	スリムかた	
\\	スリム型コンピュータ	スリムかたコンピュータ	
\\	スリム型PC	スリムかたピーシー	
\\	スルー接続	スルーせつぞく	
\\	スロット形リングネットワーク	スロットがたリングネットワーク	
\\	スロット時間	スロットじかん	
\\	スワップ領域	スワップりょういき	
\\	スーパーネット化	スーパーネットか	
\\	セキュリティ管理	セキュリティかんり	
\\	セクション区画	セクションくかく	
\\	セクタのデータ領域	セクタのデータりょういき	
\\	セグメンテーション違反	セグメンテーションいはん	
\\	セグメント化	セグメントか	
\\	セグメント属性	セグメントぞくせい	
\\	セグメント変換	セグメントへんかん	
\\	セグメント優先度	セグメントゆうせんど	
\\	セション生存期間	セションせいぞんきかん	
\\	セションコネクション同期	セションコネクションどうき	
\\	セションサービス利用者	セションサービスりようしゃ	
\\	セションプロトコル機械	セションプロトコルきかい	
\\	セッション層	セッションそう	
\\	セットアップ時間	セットアップじかん	
\\	セットアップ手続き	セットアップてつづき	
\\	セル化	セルか	
\\	セル化遅延	セルかちえん	
\\	セル間の到着時間	セルかんのとうちゃくじかん	
\\	セル組立分解	セルくみたてぶんかい	
\\	セル交換	セルこうかん	
\\	セル式電話	セルしきでんわ	
\\	セル処理	セルしょり	
\\	セル損失	セルそんしつ	
\\	セル損失比	セルそんしつひ	
\\	セル損失優先権	セルそんしつゆせんけん	
\\	セル損失率	セルそんしつりつ	
\\	セル損失プライオリティ	セルそんしつプライオリティ	
\\	セル遅延変動	セルちえんへんどう	
\\	セル廃棄	セルはいき	
\\	セル廃棄率	セルはいきりつ	
\\	セル配列	セルはいれつ	
\\	セル分割	セルぶんかつ	
\\	セルスイッチング容量	セルスイッチングようりょう	
\\	セルラー方式	セルラーほうしき	
\\	セルループバック試験	セルループバックしけん	
\\	ゼロ充てん	ゼロあてん	
\\	ゼロ以外	ゼロいがい	
\\	ゼロ行列	ゼロぎょうれつ	
\\	ゼロ番目	ゼロばんめ	
\\	ゼロ復帰記録	ゼロふっききろく	
\\	ゼロ抑制	ゼロよくせい	
\\	ゼロ抑制機能	ゼロよくせいきのう	
\\	ゼロアドレス命令	ゼロアドレスめいれい	
\\	ゼロサプレス機能	ゼロサプレスきのう	
\\	ソフト行末	ソフトぎょうまつ	
\\	ソフトウェア開発	ソフトウェアかいはつ	
\\	ソフトウェア許諾契約	ソフトウェアきょだくけいやく	
\\	ソフトウェア工学	ソフトウェアこうがく	
\\	ソフトウェア構成	ソフトウェアこうせい	
\\	ソフトウェア支援	ソフトウェアしえん	
\\	ソフトウェア使用許諾契約	ソフトウェアしようきょだくけいやく	
\\	ソフトウェア使用契約	ソフトウェアしようけいやく	
\\	ソフトウェア製品	ソフトウェアせいひん	
\\	ソフトウェア品質	ソフトウェアひんしつ	
\\	ソフトウェア品質測定法	ソフトウェアひんしつそくていほう	
\\	ソフトウェア品質特性	ソフトウェアひんしつとくせい	
\\	ソフトウェア割り込み	ソフトウェアわりこみ	
\\	ソフトウェアフロー制御	ソフトウェアフローせいぎょ	
\\	ソフトセクタ式ディスク	ソフトセクタしきディスク	
\\	ソースコード管理システム	ソースコードかんりシステム	
\\	ソースレベル互換	ソースレベルごかん	
\\	ゾーン形式	ゾーンけいしき	
\\	タイミング信号	タイミングしんごう	
\\	タイミング図	タイミングず	
\\	タイリングウィンドウ方式	タイリングウィンドウほうしき	
\\	タイル方式	タイルほうしき	
\\	タグ付き	タグつき	
\\	タグ付き型	タグつきがた	
\\	タグ付け	タグつけ	
\\	タブ移動	タブいどう	
\\	タブ区切り	タブくぎり	
\\	タブ停止位置	タブていしいち	
\\	タブ文字	タブもじ	
\\	タワー型	タワーがた	
\\	ターンオン安定時間	ターンオンあんていじかん	
\\	ダイアルアップ接続	ダイアルアップせつぞく	
\\	ダイアルアップIP接続	ダイアルアップアイピーせつぞく	
\\	ダイアログ回復	ダイアログかいふく	
\\	ダイアログ確立指示未完了	ダイアログかくりつしじみかんりょう	
\\	ダイアログ確立要求未完了	ダイアログかくりつようきゅうみかんりょう	
\\	ダイアログ終了指示未完了	ダイアログしゅうりょうしじみかんりょう	
\\	ダイアログ終了要求未完了	ダイアログしゅうりょうようきゅうみかんりょう	
\\	ダイオード関数発生器	ダイオードかんすうはっせいき	
\\	ダイオードトランジスタ論理回路	ダイオードトランジスタろんりかいろ	
\\	ダイヤル音	ダイヤルおん	
\\	ダイヤルアップ接続	ダイヤルアップせつぞく	
\\	ダイヤルアップIP接続	ダイヤルアップアイピーせつぞく	
\\	ダイヤルイン番号サービス	ダイヤルインばんごうサービス	
\\	ダイレクト接続	ダイレクトせつぞく	
\\	ダイレクトコール機能	ダイレクトコールきのう	
\\	ダウン時間	ダウンじかん	
\\	ダム端末	ダムたんまつ	
\\	ダンピング抵抗	ダンピングていこう	
\\	チェーン印字装置	チェーンいんじそうち	
\\	チャネル当たり	チャネルあたり	
\\	チャネル間結合装置	チャネルかんけつごうそうち	
\\	チャネル間分離	チャネルかんぶんり	
\\	チャネル間アダプタ	チャネルかんアダプタ	
\\	チャネル状態語	チャネルじょうたいご	
\\	チャネルパス再接続	チャネルパスさいせつぞく	
\\	チャネルプロトコル機械	チャネルプロトコルきかい	
\\	ツリー構造	ツリーこうぞう	
\\	ヅアル運転	ヅアルうんてん	
\\	テキスト位置	テキストいち	
\\	テキスト開始	テキストかいし	
\\	テキスト画面	テキストがめん	
\\	テキスト局所座標系	テキストきょくしょざひょうけい	
\\	テキスト区域	テキストくいき	
\\	テキスト形式	テキストけいしき	
\\	テキスト終結	テキストしゅうけつ	
\\	テキスト処理	テキストしょり	
\\	テキスト配置	テキストはいち	
\\	テキスト表示精度	テキストひょうじせいど	
\\	テキスト平面	テキストへいめん	
\\	テキスト平面定義ベクトル	テキストへいめんていぎベクトル	
\\	テキスト方向	テキストほうこう	
\\	テスト結果	テストけっか	
\\	テスト仕様	テストしよう	
\\	テスト版	テストばん	
\\	テレビ信号	テレビしんごう	
\\	テレビ方式	テレビほうしき	
\\	テンポラリ変数	テンポラリへんすう	
\\	テープ終わりマーク	テープおわりマーク	
\\	テープ駆動機構	テープくどうきこう	
\\	テープ駆動部	テープくどうぶ	
\\	テープ終端マーカ	テープしゅうたんマーカ	
\\	テープ穿孔機構	テープせんこうきこう	
\\	テープ穿孔装置	テープせんこうそうち	
\\	テープ装置	テープそうち	
\\	テープ複製機構	テープふくせいきこう	
\\	テープ複製装置	テープふくせいそうち	
\\	ディザ法	ディザほう	
\\	ディジタル化	ディジタルか	
\\	ディジタル回線	ディジタルかいせん	
\\	ディジタル計数形	ディジタルけいすうがた	
\\	ディジタル検出遅延時間	ディジタルけんしゅつちえんじかん	
\\	ディジタル署名	ディジタルしょめい	
\\	ディジタル信号	ディジタルしんごう	
\\	ディジタル光ディスク	ディジタルひかりディスク	
\\	ディジタル表現	ディジタルひょうげん	
\\	ディジタルアナログ変換器	ディジタルアナログへんかんき	
\\	ディスク圧縮	ディスクあっしゅく	
\\	ディスク駆動機構	ディスクくどうきこう	
\\	ディスク装置	ディスクそうち	
\\	ディスク二重化	ディスクにじゅうか	
\\	ディスクの空き領域	ディスクのあきりょういき	
\\	ディスク容量	ディスクようりょう	
\\	ディスプレイ装置	ディスプレイそうち	
\\	ディファレンシャルマンチェスタ符号化方式	ディファレンシャルマンチェスタふごうかほうしき	
\\	ディレクトリ管理領域	ディレクトリかんりりょういき	
\\	ディレクトリ構造	ディレクトリこうぞう	
\\	ディレクトリ情報ベース	ディレクトリじょうほうベース	
\\	ディレクトリ名	ディレクトリめい	
\\	ディレクトリ名による受信者指定	ディレクトリめいによるじゅしんしゃしてい	
\\	ディレクトリ利用者機能体	ディレクトリりようしゃきのうたい	
\\	ディレクトリ利用者エージェント	ディレクトリりようしゃエージェント	
\\	ディレクトリシステム機能体	ディレクトリシステムきのうたい	
\\	攻撃	ディーオーエスこうげき	
\\	デザイン寸法	デザインすんぽう	
\\	デジタル衛星放送	デジタルえいせいほうそう	
\\	デジタル音楽	デジタルおんがく	
\\	デジタル音楽ファイル	デジタルおんがくファイル	
\\	デジタル化	デジタルか	
\\	デジタル回線	デジタルかいせん	
\\	デジタル計算機	デジタルけいさんき	
\\	デジタル交換機	デジタルこうかんき	
\\	デジタル交換網	デジタルこうかんもう	
\\	デジタル証明	デジタルしょうめい	
\\	デジタル署名	デジタルしょめい	
\\	デジタル署名アルゴリズム	デジタルしょめいアルゴリズム	
\\	デジタル信号	デジタルしんごう	
\\	デジタル信号プロセッサ	デジタルしんごうプロセッサ	
\\	デジタルアナログ変換器	デジタルアナログへんかんき	
\\	デスクトップ会議	デスクトップかいぎ	
\\	デスクトップ型	デスクトップかた	
\\	デスクトップ型コンピュータ	デスクトップかたコンピュータ	
\\	デスクトップ型PC	デスクトップかたピーシー	
\\	デスクトップ環境	デスクトップかんきょう	
\\	デスクトップ出版	デスクトップしゅっぱん	
\\	デッドロック回避	デッドロックかいひ	
\\	デバイス名	デバイスめい	
\\	デバッグ行	デバッグぎょう	
\\	デバッグ節	デバッグせつ	
\\	デフォルト設定	デフォルトせってい	
\\	デモ版	デモばん・デモはん	
\\	デージーチェーン接続	デージーチェーンせつぞく	
\\	データ圧縮	データあっしゅく	
\\	データ暗号化規格	データあんごうかきかく	
\\	データ依存形障害	データいぞんがたしょうがい	
\\	データ依存ジッタ	データいぞんジッタ	
\\	データ意味論	データいみろん	
\\	データ解析	データかいせき	
\\	データ回線	データかいせん	
\\	データ回線終端装置	データかいせんしゅうたんそうち	
\\	データ回線透過性	データかいせんとうかせい	
\\	データ確認	データかくにん	
\\	データ完全性	データかんぜんせい	
\\	データ管理者	データかんりしゃ	
\\	データ型	データがた	
\\	データ記述言語	データきじゅつげんご	
\\	データ記述項	データきじゅつこう	
\\	データ記述ファイル	データきじゅつファイル	
\\	データ記述レコード	データきじゅつレコード	
\\	データ機密保護	データきみつほご	
\\	データ共用	データきょうよう	
\\	データ局	データきょく	
\\	データ記録装置	データきろくそうち	
\\	データ句	データく	
\\	データ群	データぐん	
\\	データ傾向	データけいこう	
\\	データ形式	データけいしき	
\\	データ欠損	データけっそん	
\\	データ交換	データこうかん	
\\	データ交換機	データこうかんき	
\\	データ交換制御装置	データこうかんせいぎょそうち	
\\	データ交換装置	データこうかんそうち	
\\	データ構造	データこうぞう	
\\	データ項目	データこうもく	
\\	データ写像	データしゃぞう	
\\	データ収集	データしゅうしゅう	
\\	データ収集管理システム	データしゅうしゅうかんりシステム	
\\	データ収集端末	データしゅうしゅうたんまつ	
\\	データ集線装置	データしゅうせんそうち	
\\	データ取得	データしゅとく	
\\	データ処理	データしょり	
\\	データ処理機械	データしょりきかい	
\\	データ処理システム	データしょりシステム	
\\	データ処理システムの安全保護	データしょりシステムのあんぜんほご	
\\	データ処理ステーション	データしょりステーション	
\\	データ処理センター	データしょりセンター	
\\	データ処理ノード	データしょりノード	
\\	データ信号速度	データしんごうそくど	
\\	データ辞書	データじしょ	
\\	データ実体	データじったい	
\\	データ実体化	データじったいか	
\\	データ受信装置	データじゅしんそうち	
\\	データ整合性	データせいごうせい	
\\	データ精度	データせいど	
\\	データ専用文字	データせんようもじ	
\\	データ操作言語	データそうさげんご	
\\	データ送信装置	データそうしんそうち	
\\	データ損失	データそんしつ	
\\	データ属性	データぞくせい	
\\	データ対象	データたいしょう	
\\	データ多重変換装置	データたじゅうへんかんそうち	
\\	データ単位	データたんい	
\\	データ端末装置	データたんまつそうち	
\\	データ段階化	データだんかいか	
\\	データ値	データち	
\\	データ抽象化	データちゅうしょうか	
\\	データ通信	データつうしん	
\\	データ通信機器	データつうしんきき	
\\	データ通信装置	データつうしんそうち	
\\	データ付き出力基本要素	データつきしゅつりょくきほんようそ	
\\	データ付き領域集合	データつきりょういきしゅうごう	
\\	データ付き領域集合群	データつきりょういきしゅうごうぐん	
\\	データ定義言語	データていぎげんご	
\\	データ点	データてん	
\\	データ転送速度	データてんそうそくど	
\\	データ転送段階	データてんそうだんかい	
\\	データ転送フェーズ	データてんそうフェーズ	
\\	データ伝送	データでんそう	
\\	データ伝送回線	データでんそうかいせん	
\\	データ伝送路	データでんそうろ	
\\	データ投入	データとうにゅう	
\\	データ登録	データとうろく	
\\	データ独立性	データどくりつせい	
\\	データ内容	データないよう	
\\	データ内容記法	データないようきほう	
\\	データ流れ	データながれ	
\\	データ入力	データにゅうりょく	
\\	データ入力端末	データにゅうりょくたんまつ	
\\	データの汚染	データのおせん	
\\	データの階層・データ階層	データのかいそう・データかいそう	
\\	データの完全性	データのかんぜんせい	
\\	データの先頭	データのせんとう	
\\	データの品質	データのひんしつ	
\\	データの保全性	データのほぜんせい	
\\	データの丸め	データのまるめ	
\\	データの要素	データのようそ	
\\	データ発信元認証	データはっしんもとにんしょう	
\\	データ媒体	データばいたい	
\\	データ媒体変換	データばいたいへんかん	
\\	データ媒体保護機構	データばいたいほごきこう	
\\	データ表示	データひょうじ	
\\	データ分析	データぶんせき	
\\	データ変換	データへんかん	
\\	データ変換器	データへんかんき	
\\	データ編集者	データへんしゅうしゃ	
\\	データ放送	データほうそう	
\\	データ補間	データほかん	
\\	データ保護	データほご	
\\	データ密度	データみつど	
\\	データ名	データめい	
\\	データ網	データもう	
\\	データ文字	データもじ	
\\	データ要素	データようそ	
\\	データ連鎖	データれんさ	
\\	データタグ群	データタグぐん	
\\	データタグ形式	データタグけいしき	
\\	データチャネル装置	データチャネルそうち	
\\	データビット長	データビットちょう	
\\	データベース管理	データベースかんり	
\\	データベース管理者	データベースかんりしゃ	
\\	データベース管理システム	データベースかんりシステム	
\\	データベース技術	データベースぎじゅつ	
\\	データベース再編成	データベースさいへんせい	
\\	データリンク層	データリンクそう	
\\	データリンクコネクション識別子	データリンクコネクションしきべつし	
\\	トップダウン設計	トップダウンせっけい	
\\	トラッキング記号	トラッキングきごう	
\\	トラック密度	トラックみつど	
\\	トラックアクセス時間	トラックアクセスじかん	
\\	トラバーサル順序	トラバーサルじゅんじょ	
\\	トラヒック記述子	トラヒックきじゅつし	
\\	トラヒック契約	トラヒックけいやく	
\\	トラヒック全体	トラヒックぜんたい	
\\	トラヒック特性	トラヒックとくせい	
\\	トラフィック解析	トラフィックかいせき	
\\	トラフィック制御	トラフィックせいぎょ	
\\	トラフィック輻輳	トラフィックふくそう	
\\	トラフィック量	トラフィックりょう	
\\	トラフィックフロー機密性	トラフィックフローきみつせい	
\\	トランク予備	トランクよび	
\\	トランザクション回復	トランザクションかいふく	
\\	トランザクション記録動作	トランザクションきろくどうさ	
\\	トランザクション識別子	トランザクションしきべつし	
\\	トランザクション終了フェーズ	トランザクションしゅうりょうフェーズ	
\\	トランザクション処理	トランザクションしょり	
\\	トランザクション処理応用サービス要素	トランザクションしょりおうようサービスようそ	
\\	トランザクション処理サービス提供者	トランザクションしょりサービスていきょうしゃ	
\\	トランザクション処理サービス利用者	トランザクションしょりサービスりようしゃ	
\\	トランザクション処理チャネル	トランザクションしょりチャネル	
\\	トランザクション処理プロトコル機械	トランザクションしょりプロトコルきかい	
\\	トランザクション分枝	トランザクションぶんし	
\\	トランザクション分枝識別子	トランザクションぶんししきべつし	
\\	トランザクション木の制約条件	トランザクションもくのせいやくじょうけん	
\\	トランスポート層	トランスポートそう	
\\	トランスポートサービス提供者	トランスポートサービスていきょうしゃ	
\\	トランスポートサービス利用者	トランスポートサービスりようしゃ	
\\	トリガ回路	トリガかいろ	
\\	トークン化	トークンか	
\\	トークン発生レート	トークンはっせいレート	
\\	トークンパッシング手順	トークンパッシングてじゅん	
\\	トーン検出	トーンけんしゅつ	
\\	ドイツ工業規格	ドイツこうぎょうきかく	
\\	ドキュメンテーション言語	ドキュメンテーションげんご	
\\	ドキュメント型	ドキュメントがた	
\\	ドキュメント共有	ドキュメントきょうゆう	
\\	ドキュメント整理	ドキュメントせいり	
\\	ドキュメント読取り装置	ドキュメントよみとりそうち	
\\	ドット印字装置	ドットいんじそうち	
\\	ドット絵	ドットえ	
\\	ドット字形	ドットじけい	
\\	ドットマトリックス文字発生器	ドットマトリックスもじはっせいき	
\\	ドプラー周波数	ドプラーしゅうはすう	
\\	ドライブ位置	ドライブいち	
\\	ドライブ名	ドライブめい	
\\	ドライブ文字	ドライブもじ	
\\	ドラム印字装置	ドラムいんじそうち	
\\	ドラム記憶装置	ドラムきおくそうち	
\\	ドラム作図装置	ドラムさくずそうち	
\\	ドラム式プロッタ	ドラムしきプロッタ	
\\	ナイキスト周波数	ナイキストしゅうはすう	
\\	ナル文字列	ナルもじれつ	
\\	ニッカド電池	ニッカドでんち	
\\	ニッケル水素	ニッケルすいそ	
\\	ニッケル水素電池	ニッケルすいそでんち	
\\	ニッケル水素バッテリ	ニッケルすいそバッテリ	
\\	ニュース記事	ニュースきじ	
\\	ヌル型・ナル型	ヌルがた・ナルがた	
\\	ヌル値	ヌルち	
\\	ヌル文字	ヌルもじ	
\\	ヌル文字列	ヌルもじれつ	
\\	ネイティブ対応	ネイティブたいおう	
\\	ネガティブ表示	ネガティブひょうじ	
\\	ネット混雑	ネットこんざつ	
\\	ネット署名	ネットしょめい	
\\	ネット住民	ネットじゅうみん	
\\	ネットワーク改造	ネットワークかいぞう	
\\	ネットワーク監視	ネットワークかんし	
\\	ネットワーク管理	ネットワークかんり	
\\	ネットワーク管理者	ネットワークかんりしゃ	
\\	ネットワーク管理システム	ネットワークかんりシステム	
\\	ネットワーク機器	ネットワークきき	
\\	ネットワーク計画	ネットワークけいかく	
\\	ネットワーク構成	ネットワークこうせい	
\\	ネットワーク構造	ネットワークこうぞう	
\\	ネットワーク効率	ネットワークこうりつ	
\\	ネットワーク資源	ネットワークしげん	
\\	ネットワーク資産	ネットワークしさん	
\\	ネットワーク状況	ネットワークじょうきょう	
\\	ネットワーク図	ネットワークず	
\\	ネットワーク接続	ネットワークせつぞく	
\\	ネットワーク接続形態	ネットワークせつぞくけいたい	
\\	ネットワーク全体	ネットワークぜんたい	
\\	ネットワーク層	ネットワークそう	
\\	ネットワーク統合	ネットワークとうごう	
\\	ネットワーク内	ネットワークない	
\\	ネットワーク名	ネットワークめい	
\\	ネットワークアクセス点	ネットワークアクセスてん	
\\	ネットワークアドレス付与機関	ネットワークアドレスふよきかん	
\\	ネットワークアドレス変換	ネットワークアドレスへんかん	
\\	ネットワークアドレス領域	ネットワークアドレスりょういき	
\\	ネットワークサービス提供者	ネットワークサービスていきょうしゃ	
\\	ネットワークプロトコルアドレス指定情報	ネットワークプロトコルアドレスしていじょうほう	
\\	ノイマン型・ノイマン形	ノイマンがた	
\\	ノンアド機能	ノンアドきのう	
\\	ノングレア処理	ノングレアしょり	
\\	ノンプリント機能	ノンプリントきのう	
\\	ノンリニア編集	ノンリニアへんしゅう	
\\	ノート型	ノートがた	
\\	ノート型コンピュータ	ノートがたコンピュータ	
\\	ノート型パソコン	ノートがたパソコン	
\\	ノート型PC	ノートがたピーシー	
\\	ノートブック型パーソナルコンピュータ	ノートブックがたパーソナルコンピュータ	
\\	ハイエンド機種	ハイエンドきしゅ	
\\	ハイパ文書	ハイパぶんしょ	
\\	ハイパメディア応用	ハイパメディアおうよう	
\\	ハイパメディア及び時間依存情報の構造化言語	ハイパメディアおよびじかんいぞんじょうほうのこうぞうかげんご	
\\	ハイパメディア文書	ハイパメディアぶんしょ	
\\	ハイパーテキスト転送プロトコル	ハイパーテキストてんそうプロトコル	
\\	ハイパーテキストマークアップ言語	ハイパーテキストマークアップげんご	
\\	ハイブリッド計算機	ハイブリッドけいさんき	
\\	ハイレベルデータリンク制御手順	ハイレベルデータリンクせいぎょてじゅん	
\\	ハッシュ関数	ハッシュかんすう	
\\	ハッシュ合計	ハッシュごうけい	
\\	ハッシュ処理	ハッシュしょり	
\\	ハッシュ表探索	ハッシュひょうたんさく	
\\	ハッシュ法	ハッシュほう	
\\	ハッチ定義表	ハッチていぎひょう	
\\	ハッチ内部様式	ハッチないぶようしき	
\\	ハッチ様式	ハッチようしき	
\\	ハフマン符号	ハフマンふごう	
\\	ハンドシェーク指示未完了	ハンドシェークしじみかんりょう	
\\	ハンドシェーク要求未完了	ハンドシェークようきゅうみかんりょう	
\\	ハンドヘルド計算器	ハンドヘルドけいさんき	
\\	ハートビート信号	ハートビートしんごう	
\\	ハード行末	ハードぎょうまつ	
\\	ハードウェア構成	ハードウェアこうせい	
\\	ハードウェア割り込み	ハードウェアわりこみ	
\\	ハードウェアフロー制御	ハードウェアフローせいぎょ	
\\	ハードセクタ式ディスク	ハードセクタしきディスク	
\\	ハードセクタ方式	ハードセクタほうしき	
\\	バイチ図	バイチず	
\\	バイト値	バイトち	
\\	バイナリ互換	バイナリごかん	
\\	バイナリレベル互換	バイナリレベルごかん	
\\	バウンダリ調整	バウンダリちょうせい	
\\	バケツリレー攻撃	バケツリレーこうげき	
\\	バケツリレー素子	バケツリレーそし	
\\	バス型	バスがた	
\\	バス調停	バスちょうてい	
\\	バス幅	バスはば	
\\	バッカス記法	バッカスきほう	
\\	バックエンド処理	バックエンドしょり	
\\	バックグラウンド印刷	バックグラウンドいんさつ	
\\	バックドア型コンピュータウイルス	バックドアかたコンピュータウイルス	
\\	バックワード回復	バックワードかいふく	
\\	バッチ処理	バッチしょり	
\\	バッテリ駆動	バッテリくどう	
\\	バッファ記憶	バッファきおく	
\\	バッファ付き	バッファづき	
\\	バッファ付き装置	バッファづきそうち	
\\	バッファオーバーフロー攻撃	バッファオーバーフローこうげき	
\\	バナー広告	バナーこうこく	
\\	バブル記憶装置	バブルきおくそうち	
\\	バリイング変数	バリイングへんすう	
\\	バルク品	バルクひん	
\\	バンド印字装置	バンドいんじそうち	
\\	バンド幅	バンドはば	
\\	バー印字装置	バーいんじそうち	
\\	バージョン管理システム	バージョンかんりシステム	
\\	バージョン番号	バージョンばんごう	
\\	バースト性	バーストせい	
\\	バースト長	バーストちょう	
\\	バースト的	バーストてき	
\\	バースト転送	バーストてんそう	
\\	バースト伝送	バーストでんそう	
\\	バースト頻度	バーストひんど	
\\	バーストモード転送	バーストモードてんそう	
\\	バーチャルサーキット識別子	バーチャルサーキットしきべつし	
\\	バーチャルパス識別子	バーチャルパスしきべつし	
\\	パイプライン処理	パイプラインしょり	
\\	パイプライン処理機構	パイプラインしょりきこう	
\\	パイプライン処理装置	パイプラインしょりそうち	
\\	パイロット試験	パイロットしけん	
\\	パケット組立て分解機能	パケットくみたてぶんかいきのう	
\\	パケット形態端末	パケットけいたいたんまつ	
\\	パケット交換	パケットこうかん	
\\	パケット交換網	パケットこうかんもう	
\\	パケット順序制御	パケットじゅんじょせいぎょ	
\\	パケット通信	パケットつうしん	
\\	パス指向	パスしこう	
\\	パス番号	パスばんごう	
\\	パス名	パスめい	
\\	パス名の構成要素	パスめいのこうせいようそ	
\\	パスオーバーヘッド部	パスオーバーヘッドぶ	
\\	パスワード失効	パスワードしっこう	
\\	パスワード制限	パスワードせいげん	
\\	パスワード保護	パスワードほご	
\\	パソコン減税	パソコンげんぜい	
\\	パソコン通信	パソコンつうしん	
\\	パソコン通信サービス	パソコンつうしんサービス	
\\	パターン一致	パターンいっち	
\\	パターン定義表	パターンていぎひょう	
\\	パターン内部様式	パターンないぶようしき	
\\	パック10進表記法・パック1十進表記法	パックじっしんひょうきほう	
\\	パッケージ名	パッケージめい	
\\	パニック画面	パニックがめん	
\\	パラメタ空間	パラメタくうかん	
\\	パラメタ結合	パラメタけつごう	
\\	パラメタ識別子	パラメタしきべつし	
\\	パラメタ値	パラメタち	
\\	パラメタ範囲	パラメタはんい	
\\	パラレル転送	パラレルてんそう	
\\	パラレル伝送	パラレルでんそう	
\\	パリティ検査	パリティけんさ	
\\	パルス符号変調	パルスふごうへんちょう	
\\	パルス列	パルスれつ	
\\	パワーマネージメント機能	パワーマネージメントきのう	
\\	パーセント記号	パーセントきごう	
\\	パートナー関係	パートナーかんけい	
\\	ヒストリの置換	ヒストリのちかん	
\\	ヒット数	ヒットすう	
\\	ヒット率	ヒットりつ	
\\	ヒープ領域	ヒープりょういき	
\\	ビタビ復号	ビタビふくごう	
\\	ビッグエンディアン式	ビッグエンディアンしき	
\\	ビット誤り率	ビットあやまりりつ	
\\	ビット位置	ビットいち	
\\	ビット演算	ビットえんざん	
\\	ビット組み	ビットくみ	
\\	ビット組合せ	ビットくみあわせ	
\\	ビット構成	ビットこうせい	
\\	ビット単位	ビットたんい	
\\	ビット長	ビットちょう	
\\	ビット番号	ビットばんごう	
\\	ビット密度	ビットみつど	
\\	ビット列	ビットれつ	
\\	ビット列型	ビットれつがた	
\\	ビデオ会議	ビデオかいぎ	
\\	ビデオ会議システム	ビデオかいぎシステム	
\\	ビデオ同期装置	ビデオどうきそうち	
\\	ビュー基準座標系	ビューきじゅんざひょうけい	
\\	ビュー基準点	ビューきじゅんてん	
\\	ビュー指標	ビューしひょう	
\\	ビュー写像変換行列	ビューしゃぞうへんかんぎょうれつ	
\\	ビュー上方向定義ベクトル	ビューじょうほうこうていぎベクトル	
\\	ビュー定義表	ビューていぎひょう	
\\	ビュー表現	ビューひょうげん	
\\	ビュー平面	ビューへいめん	
\\	ビュー平面法線	ビューへいめんほうせん	
\\	ビュー変換入力優先度	ビューへんかんにゅうりょくゆうせんど	
\\	ビュー方向変換行列	ビューほうこうへんかんぎょうれつ	
\\	ビュー立体	ビューりったい	
\\	ビューイング変換	ビューイングへんかん	
\\	ビューロファックスサービスによる配達	ビューロファックスサービスによるはいたつ	
\\	ビルド番号	ビルドばんごう	
\\	スプライン関数	ビースプラインかんすう	
\\	ビープ音	ビープおん	
\\	ピコ秒	ピコびょう	
\\	ピザボックス型	ピザボックスがた	
\\	ピック経路	ピックけいろ	
\\	ピック識別子	ピックしきべつし	
\\	ピック入力装置	ピックにゅうりょくそうち	
\\	ピン互換	ピンごかん	
\\	ピン説明	ピンせつめい	
\\	ピント送り	ピントおくり	
\\	ピーク速度	ピークそくど	
\\	ピーク帯域幅	ピークたいいきはば	
\\	ピークセル速度	ピークセルそくど	
\\	ファイバー切断	ファイバーせつだん	
\\	ファイル圧縮	ファイルあっしゅく	
\\	ファイル位置指示子	ファイルいちしじし	
\\	ファイル終わりマーク	ファイルおわりマーク	
\\	ファイル終わりラベル	ファイルおわりラベル	
\\	ファイル拡張子	ファイルかくちょうし	
\\	ファイル感染型ウイルス	ファイルかんせんかたウイルス	
\\	ファイル管理	ファイルかんり	
\\	ファイル管理記述項	ファイルかんりきじゅつこう	
\\	ファイル記憶装置	ファイルきおくそうち	
\\	ファイル記述	ファイルきじゅつ	
\\	ファイル記述項	ファイルきじゅつこう	
\\	ファイル記述子	ファイルきじゅつし	
\\	ファイル共有	ファイルきょうゆう	
\\	ファイル共用	ファイルきょうよう	
\\	ファイル許可ビット	ファイルきょかビット	
\\	ファイル句	ファイルく	
\\	ファイル形式	ファイルけいしき	
\\	ファイル結合子	ファイルけつごうし	
\\	ファイル交換ソフトウェア	ファイルこうかんソフトウェア	
\\	ファイル固有属性	ファイルこゆうぞくせい	
\\	ファイル作成時	ファイルさくせいじ	
\\	ファイル指定	ファイルしてい	
\\	ファイル終了条件	ファイルしゅうりょうじょうけん	
\\	ファイル所有者クラス	ファイルしょゆうしゃクラス	
\\	ファイル処理ルーチン	ファイルしょりルーチン	
\\	ファイル節	ファイルせつ	
\\	ファイル送信	ファイルそうしん	
\\	ファイル掃除	ファイルそうじ	
\\	ファイル属性	ファイルぞくせい	
\\	ファイル属性不整合条件	ファイルぞくせいふせいごうじょうけん	
\\	ファイル他者クラス	ファイルたしゃクラス	
\\	ファイル転送	ファイルてんそう	
\\	ファイル転送プロトコル	ファイルてんそうプロトコル	
\\	ファイル通し番号	ファイルとおしばんごう	
\\	ファイル入出力	ファイルにゅうしゅつりょく	
\\	ファイルの圧縮	ファイルのあっしゅく	
\\	ファイルの更新	ファイルのこうしん	
\\	ファイルの削除	ファイルのさくじょ	
\\	ファイルの修復	ファイルのしゅうふく	
\\	ファイルの断片化	ファイルのだんぺんか	
\\	ファイルの変換	ファイルのへんかん	
\\	ファイルの保護	ファイルのほご	
\\	ファイルの保守	ファイルのほしゅ	
\\	ファイル始めラベル	ファイルはじめラベル	
\\	ファイル分割	ファイルぶんかつ	
\\	ファイル分離キャラクタ	ファイルぶんりキャラクタ	
\\	ファイル編成	ファイルへんせい	
\\	ファイル保護リング	ファイルほごリング	
\\	ファイル名	ファイルめい	
\\	ファイル名拡張子	ファイルめいかくちょうし	
\\	ファイル名置換	ファイルめいちかん	
\\	ファイル名展開	ファイルめいてんかい	
\\	ファイル名の完結	ファイルめいのかんけつ	
\\	ファイルアクセス構造	ファイルアクセスこうぞう	
\\	ファイルアクセスデータ単位	ファイルアクセスデータたんい	
\\	ファイルサービス利用者	ファイルサービスりようしゃ	
\\	ファイルストア動作	ファイルストアどうさ	
\\	ファクス送信	ファクスそうしん	
\\	ファジィ論理	ファジィろんり	
\\	ファジー制御	ファジーせいぎょ	
\\	ファジー論理	ファジーろんり	
\\	ファセット順位	ファセットじゅんい	
\\	ファセット表示記号	ファセットひょうじきごう	
\\	ファセット分類体系	ファセットぶんるいたいけい	
\\	ファセット連結記号	ファセットれんけつきごう	
\\	ファックス通信機能	ファックスつうしんきのう	
\\	フィボナッチ探索	フィボナッチたんさく	
\\	フィボネッチ数列	フィボネッチすうれつ	
\\	フィラー文字	フィラーもじ	
\\	フィルタ項目	フィルタこうもく	
\\	フィルム感度	フィルムかんど	
\\	フィールド分離文字	フィールドぶんりもじ	
\\	フェライト磁心	フェライトじしん	
\\	フェールセーフ動作	フェールセーフどうさ	
\\	フォント参照	フォントさんしょう	
\\	フォント資源	フォントしげん	
\\	フォント寸法	フォントすんぽう	
\\	フォント配置量	フォントはいちりょう	
\\	フォント名	フォントめい	
\\	フォールバック機能	フォールバックきのう	
\\	フラクタル圧縮	フラクタルあっしゅく	
\\	フラット制約集号	フラットせいやくしゅうごう	
\\	フラットアドレス空間	フラットアドレスくうかん	
\\	フラットベッド式プロッタ	フラットベッドしきプロッタ	
\\	フリップフロップ回路	フリップフロップかいろ	
\\	フル冗長	フルじょうちょう	
\\	フルタワー型	フルタワーがた	
\\	フルタワー型コンピュータ	フルタワーがたコンピュータ	
\\	フルタワー型PC	フルタワーがたピーシー	
\\	フルテキスト検索	フルテキストけんさく	
\\	フレーム間タイムフィル	フレームかんタイムフィル	
\\	フレーム形式	フレームけいしき	
\\	フレーム検査シーケンス	フレームけんさシーケンス	
\\	フレーム除去	フレームじょきょ	
\\	フレーム同期方式	フレームどうきほうしき	
\\	フレーム廃棄	フレームはいき	
\\	フロントエンド処理	フロントエンドしょり	
\\	フロントエンド通信プロセッサ	フロントエンドつうしんプロセッサ	
\\	ブラインド入力	ブラインドにゅうりょく	
\\	ブラウザ戦争	ブラウザせんそう	
\\	ブリッジ入力回路	ブリッジにゅうりょくかいろ	
\\	ブルックの法則	ブルックのほうそく	
\\	ブルートフォース攻撃	ブルートフォースこうげき	
\\	ブロッキング系数	ブロッキングけいすう	
\\	ブロック化	ブロックか	
\\	ブロック化因数	ブロックかいんすう	
\\	ブロック間隔	ブロックかんかく	
\\	ブロック間ギャップ	ブロックかんギャップ	
\\	ブロック型特殊ファイル	ブロックがたとくしゅファイル	
\\	ブロック検査	ブロックけんさ	
\\	ブロック長	ブロックちょう	
\\	ブロック転送	ブロックてんそう	
\\	ブロック符号	ブロックふごう	
\\	ブロック連鎖	ブロックれんさ	
\\	ブロックチェック文字	ブロックチェックもじ	
\\	ブート可能ディスケット	ブートかのうディスケット	
\\	ブート時に	ブートときに	
\\	ブート不可能ディスケット	ブートふかのうディスケット	
\\	ブール演算	ブールえんざん	
\\	ブール演算子	ブールえんざんし	
\\	ブール演算表	ブールえんざんひょう	
\\	ブール関数	ブールかんすう	
\\	ブール型	ブールがた	
\\	ブール式	ブールしき	
\\	ブール値	ブールち	
\\	プッシュ技術	プッシュぎじゅつ	
\\	プライバシ保護	プライバシほご	
\\	プラグ及びジャック	プラグおよびジャック	
\\	プラグ互換	プラグごかん	
\\	プラスチック光ファイバ	プラスチックひかりファイバ	
\\	プラットフォーム特有	プラットフォームとくゆう	
\\	プリエンプティブ式のマルチタスキング	プリエンプティブしきのマルチタスキング	
\\	プリセット値	プリセットち	
\\	プリンタ指定	プリンタしてい	
\\	プリンタ制御コード	プリンタせいぎょコード	
\\	プリント回路	プリントかいろ	
\\	プリント配線盤	プリントはいせんばん	
\\	プリント板	プリントばん	
\\	プレゼンテーション層	プレゼンテーションそう	
\\	プレゼンテーションコンテキスト識別	プレゼンテーションコンテキストしきべつ	
\\	プレゼンテーションコンテキスト識別子	プレゼンテーションコンテキストしきべつし	
\\	プレゼンテーションデータ値	プレゼンテーションデータち	
\\	プレフィックス形シフトキー	プレフィックスがたシフトキー	
\\	プレフィックス表記法	プレフィックスひょうきほう	
\\	プレーンテキスト版	プレーンテキストはん	
\\	プロ版	プロはん	
\\	プログラミング可能	プログラミングかのう	
\\	プログラミング環境	プログラミングかんきょう	
\\	プログラミング言語	プログラミングげんご	
\\	プログラム依存形障害	プログラムいぞんがたしょうがい	
\\	プログラム終わり見出し	プログラムおわりみだし	
\\	プログラム化	プログラムか	
\\	プログラム可能	プログラムかのう	
\\	プログラム可能読取り専用記憶装置	プログラムかのうよみとりせんようきおくそうち	
\\	プログラム言語	プログラムげんご	
\\	プログラム固定式計算器	プログラムこていしきけいさんき	
\\	プログラム式計算器	プログラムしきけいさんき	
\\	プログラム試験時間	プログラムしけんじかん	
\\	プログラム仕様	プログラムしよう	
\\	プログラム実動時間	プログラムじつどうじかん	
\\	プログラム状態語	プログラムじょうたいご	
\\	プログラム単位	プログラムたんい	
\\	プログラム保守説明書	プログラムほしゅせつめいしょ	
\\	プログラム保守マニュアル	プログラムほしゅマニュアル	
\\	プログラム名	プログラムめい	
\\	プログラム名記述項	プログラムめいきじゅつこう	
\\	プロジェクト管理	プロジェクトかんり	
\\	プロジェクト管理ソフト	プロジェクトかんりソフト	
\\	プロジェクト計画	プロジェクトけいかく	
\\	プロジェクト仕様書	プロジェクトしようしょ	
\\	プロジェクト制御	プロジェクトせいぎょ	
\\	プロジェクト統制	プロジェクトとうせい	
\\	プロジェクト立案	プロジェクトりつあん	
\\	プロセス間通信	プロセスかんつうしん	
\\	プロセス制御	プロセスせいぎょ	
\\	プロセス制御装置	プロセスせいぎょそうち	
\\	プロセスの生存期間	プロセスのせいぞんきかん	
\\	プロセス番号	プロセスばんごう	
\\	プロセス割込み信号	プロセスわりこみしんごう	
\\	プロセッサ処理主体	プロセッサしょりしゅたい	
\\	プロッタ増分量	プロッタぞうぶんりょう	
\\	プロトコル誤り	プロトコルあやまり	
\\	プロトコル階層	プロトコルかいそう	
\\	プロトコル機械	プロトコルきかい	
\\	プロトコル群	プロトコルぐん	
\\	プロトコル実装適合性宣言	プロトコルじっそうてきごうせいせんげん	
\\	プロトコル適合性試験報告書	プロトコルてきごうせいしけんほうこくしょ	
\\	プロトコル変換	プロトコルへんかん	
\\	プロトコルデータ単位	プロトコルデータたんい	
\\	プロンプトに対して	プロンプトにたいして	
\\	ヘイズ互換	ヘイズごかん	
\\	ヘッディング開始	ヘッディングかいし	
\\	ヘッド切換え	ヘッドきりかえ	
\\	ヘッド高	ヘッドこう	
\\	ヘッド衝突	ヘッドしょうとつ	
\\	ヘッドローディング域	ヘッドローディングいき	
\\	ヘルプ画面	ヘルプがめん	
\\	ヘルプ機能	ヘルプきのう	
\\	ベクトル処理機構	ベクトルしょりきこう	
\\	ベクトル処理装置	ベクトルしょりそうち	
\\	ベクトル添字	ベクトルそえじ	
\\	ベクトル発生器	ベクトルはっせいき	
\\	ベジェ曲線	ベジェきょくせん	
\\	ベストエフォート型	ベストエフォートかた	
\\	ベル型	ベルがた	
\\	ベルト印字装置	ベルトいんじそうち	
\\	ベンダ側	ベンダがわ	
\\	ベンチマーク試験	ベンチマークしけん	
\\	ベース基地	ベースきち	
\\	ベースバンド伝送	ベースバンドでんそう	
\\	ベータ版・β版	ベータばん	
\\	ペイント系プログラム	ペイントけいプログラム	
\\	ペン入力	ペンにゅうりょく	
\\	ページ脚書き	ページあしがき	
\\	ページ印字装置	ページいんじそうち	
\\	頁送り	ページおくり	
\\	ページ書き込み保護	ページかきこみほご	
\\	ページ記述言語	ページきじゅつげんご	
\\	ページ区切り	ページくぎり	
\\	ページ座標系	ページざひょうけい	
\\	ページ集合	ページしゅうごう	
\\	ページ置換アルゴリズム	ページちかんアルゴリズム	
\\	ページの向き	ページのむき	
\\	ページ番号	ページばんごう	
\\	ページ変換例外	ページへんかんれいがい	
\\	ページ本体	ページほんたい	
\\	ページ読取り装置	ページよみとりそうち	
\\	ページ枠	ページわく	
\\	ページ分け	ページわけ	
\\	ページング可能	ページングかのう	
\\	ページング技法	ページングぎほう	
\\	ホスト計算機	ホストけいさんき	
\\	ホスト名	ホストめい	
\\	ホーム位置	ホームいち	
\\	ホールド時間	ホールドじかん	
\\	ボリューム終わりラベル	ボリュームおわりラベル	
\\	ボリューム指示子	ボリュームしじし	
\\	ボリューム集合	ボリュームしゅうごう	
\\	ボリューム始めラベル	ボリュームはじめラベル	
\\	ボリューム見出しラベル	ボリュームみだしラベル	
\\	ボールド体	ボールドたい	
\\	ポインタ結合	ポインタけつごう	
\\	ポインタ代入	ポインタだいにゅう	
\\	ポインタ代入文	ポインタだいにゅうぶん	
\\	ポイントツーポイント接続	ポイントツーポイントせつぞく	
\\	ポケット計算器	ポケットけいさんき	
\\	ポジティブ表示	ポジティブひょうじ	
\\	ポストフィックス表記法	ポストフィックスひょうきほう	
\\	ポップアップ表示	ポップアップひょうじ	
\\	ポリトープ法	ポリトープほう	
\\	ポロジ状態要素	ポロジじょうたいようそ	
\\	ポート番号	ポートばんごう	
\\	ポート輻輳	ポートふくそう	
\\	ポーランド記法	ポーランドきほう	
\\	ポーランド表記法	ポーランドひょうきほう	
\\	マイクロ秒	マイクロびょう	
\\	マイクロ命令	マイクロめいれい	
\\	マイクロソフトライセンス認証	マイクロソフトライセンスにんしょう	
\\	マイナスのねじ回し	マイナスのねじまわし	
\\	マイナーバージョン番号	マイナーバージョンばんごう	
\\	マウス移動感度	マウスいどうかんど	
\\	マウスの左ボタン	マウスのひだりボタン	
\\	マウント処理	マウントしょり	
\\	マクロ感染型ウイルス	マクロかんせんかたウイルス	
\\	マクロ機能	マクロきのう	
\\	マクロ言語	マクロげんご	
\\	マクロ式	マクロしき	
\\	マクロ処理系	マクロしょりけい	
\\	マクロ処理プログラム	マクロしょりプログラム	
\\	マクロ生成系	マクロせいせいけい	
\\	マクロ生成プログラム	マクロせいせいプログラム	
\\	マクロ定義	マクロていぎ	
\\	マクロの展開	マクロのてんかい	
\\	マクロ本体	マクロほんたい	
\\	マクロ命令	マクロめいれい	
\\	マクロ呼出し	マクロよびだし	
\\	マシン依存型	マシンいぞんがた	
\\	マシンが読み取り可能	マシンがよみとりかのう	
\\	マシン語	マシンご	
\\	マシン独立型	マシンどくりつがた	
\\	マスク不可能割り込み	マスクふかのうわりこみ	
\\	マック版	マックはん	
\\	マッチ演算子	マッチえんざんし	
\\	マルコフ解析	マルコフかいせき	
\\	マルコフ連鎖	マルコフれんさ	
\\	マルコフ連鎖モンテカルロ法	マルコフれんさモンテカルロほう	
\\	マルチタスク処理	マルチタスクしょり	
\\	マルチバイト文字	マルチバイトもじ	
\\	マルチブロックメッセージ伝送	マルチブロックメッセージでんそう	
\\	マルチベンダー環境	マルチベンダーかんきょう	
\\	マルチポイント接続	マルチポイントせつぞく	
\\	マルチメディア化	マルチメディアか	
\\	マルチメディア通信	マルチメディアつうしん	
\\	マルチレンジ増幅器	マルチレンジぞうふくき	
\\	マンセル表色系	マンセルひょうしょくけい	
\\	マーカ種	マーカしゅ	
\\	マーカ倍率	マーカばいりつ	
\\	マーカ列	マーカれつ	
\\	マーカ列束表	マーカれつたばひょう	
\\	マーク規則	マークきそく	
\\	マーク区間	マークくかん	
\\	マーク区間開始	マークくかんかいし	
\\	マーク区間終了	マークくかんしゅうりょう	
\\	マーク区間宣言	マークくかんせんげん	
\\	マーク区間の有効状態	マークくかんのゆうこうじょうたい	
\\	マーク最小化機構	マークさいしょうかきこう	
\\	マーク宣言	マークせんげん	
\\	マーク付け	マークづけ	
\\	マーク認知開始文字	マークにんちかいしもじ	
\\	マーク認知終了文字	マークにんちしゅうりょうもじ	
\\	マーク認知抑止文字	マークにんちよくしもじ	
\\	マーク文字	マークもじ	
\\	マーク読取り	マークよみとり	
\\	マーク読取り装置	マークよみとりそうち	
\\	マークアップ言語	マークアップげんご	
\\	マーケティング情報システム	マーケティングじょうほうシステム	
\\	ミシン目	ミシンめ	
\\	ミドルタワー型	ミドルタワーがた	
\\	ミドルタワー型コンピュータ	ミドルタワーがたコンピュータ	
\\	ミドルタワー型PC	ミドルタワーがたピーシー	
\\	ミニタワー型	ミニタワーがた	
\\	ミニタワー型コンピュータ	ミニタワーがたコンピュータ	
\\	ミニタワー型PC	ミニタワーがたピーシー	
\\	ミリ秒	ミリびょう	
\\	メイト系	メイトけい	
\\	メジャーバージョン番号	メジャーバージョンばんごう	
\\	メタ言語	メタげんご	
\\	メタ文字	メタもじ	
\\	メタファイル解釈系	メタファイルかいしゃくけい	
\\	メタファイル記述子	メタファイルきじゅつし	
\\	メタファイル生成系	メタファイルせいせいけい	
\\	メタファイル要素	メタファイルようそ	
\\	メタリック回線	メタリックかいせん	
\\	メッセジ転送	メッセジてんそう	
\\	メッセージ安全保護ラベル	メッセージあんぜんほごラベル	
\\	メッセージ格納	メッセージかくのう	
\\	メッセージ格納域	メッセージかくのういき	
\\	メッセージ交換	メッセージこうかん	
\\	メッセージ識別	メッセージしきべつ	
\\	メッセージ順序完全性	メッセージじゅんじょかんぜんせい	
\\	メッセージ通信処理	メッセージつうしんしょり	
\\	メッセージ通信処理環境	メッセージつうしんしょりかんきょう	
\\	メッセージ通信処理サービス	メッセージつうしんしょりサービス	
\\	メッセージ通信処理システム	メッセージつうしんしょりシステム	
\\	メッセージ通信システム	メッセージつうしんシステム	
\\	メッセージ転送	メッセージてんそう	
\\	メッセージ転送機能体	メッセージてんそうきのうたい	
\\	メッセージ転送サービス	メッセージてんそうサービス	
\\	メッセージ転送システム	メッセージてんそうシステム	
\\	メッセージ流れ機密性	メッセージながれきみつせい	
\\	メッセージ発信抽象操作	メッセージはっしんちゅうしょうそうさ	
\\	メッセージ発生源認証	メッセージはっせいげんにんしょう	
\\	メディア消磁装置	メディアしょうじそうち	
\\	メディア変換	メディアへんかん	
\\	メディアアクセス制御	メディアアクセスせいぎょ	
\\	メトロポリス・ヘイスティングス法	メトロポリス・ヘイスティングスほう	
\\	メニュー画面	メニューがめん	
\\	メニュー項目	メニューこうもく	
\\	メニュー方式	メニューほうしき	
\\	メニュー方式のプログラム	メニューほうしきのプログラム	
\\	メモリ拡張カード	メモリかくちょうカード	
\\	メモリ拡張ボード	メモリかくちょうボード	
\\	メモリ管理装置	メモリかんりそうち	
\\	メモリ管理ユニット	メモリかんりユニット	
\\	メモリ効果	メモリこうか	
\\	メモリ構成	メモリこうせい	
\\	メモリ消去機能	メモリしょうきょきのう	
\\	メモリ常駐	メモリじょうちゅう	
\\	メモリ表示	メモリひょうじ	
\\	メモリ不足	メモリぶそく	
\\	メモリ分割	メモリぶんかつ	
\\	メモリ保護	メモリほご	
\\	メモリ容量	メモリようりょう	
\\	メモリ領域	メモリりょういき	
\\	メモリ割り当て	メモリわりあて	
\\	メモリー管理	メモリーかんり	
\\	メモリー効果	メモリーこうか	
\\	メンバー関数	メンバーかんすう	
\\	メール爆弾	メールばくだん	
\\	モジュラー式コンセント	モジュラーしきコンセント	
\\	モジュロN計数器	モジュロエンけいすうき	
\\	モジュロN検査	モジュロエンけんさ	
\\	モジュール化	モジュールか	
\\	モジュール性	モジュールせい	
\\	モジュール手続	モジュールてつづき	
\\	モジュール副プログラム	モジュールふくプログラム	
\\	モデム接続	モデムせつぞく	
\\	モデリング座標系	モデリングざひょうけい	
\\	モデリング変換	モデリングへんかん	
\\	モデル化	モデルか	
\\	モデル群	モデルぐん	
\\	モバイル電話	モバイルでんわ	
\\	モーション広告	モーションこうこく	
\\	モード方式	モードほうしき	
\\	ユニコード技術委員会	ユニコードぎじゅついいんかい	
\\	ユニット分離キャラクタ	ユニットぶんりキャラクタ	
\\	ユビキタス環境	ユビキタスかんきょう	
\\	能力の登録	ユーエーのうりょくのとうろく	
\\	ユーザ関数	ユーザかんすう	
\\	ユーザ側	ユーザがわ	
\\	ユーザ登録	ユーザとうろく	
\\	ユーザ部	ユーザぶ	
\\	ユーザ要望	ユーザようぼう	
\\	ユーザー網インタフェース	ユーザーあみインタフェース	
\\	ユーザー側	ユーザーがわ	
\\	ユーザー辞書	ユーザーじしょ	
\\	ユーザー情報	ユーザーじょうほう	
\\	ユーザー定義	ユーザーていぎ	
\\	ユーザー登録	ユーザーとうろく	
\\	ユーザー認証	ユーザーにんしょう	
\\	ユーザー名	ユーザーめい	
\\	ライセンス契約	ライセンスけいやく	
\\	ライセンス文書	ライセンスぶんしょ	
\\	ライトペン検出	ライトペンけんしゅつ	
\\	ライブ映像	ライブえいぞう	
\\	ライブラリー関数	ライブラリーかんすう	
\\	ラウンドトリップ時間	ラウンドトリップじかん	
\\	ラジオシティ法	ラジオシティほう	
\\	ラスタ化	ラスタか	
\\	ラスタ作図装置	ラスタさくずそうち	
\\	ラスタ図形処理	ラスタずけいしょり	
\\	ラスタ図形要素	ラスタずけいようそ	
\\	ラスタ単位	ラスタたんい	
\\	ラスタ表示装置	ラスタひょうじそうち	
\\	ラスタ方式グラフィクス	ラスタほうしきグラフィクス	
\\	ラベル識別子	ラベルしきべつし	
\\	ラベル番号	ラベルばんごう	
\\	ランダマイズ技法	ランダマイズぎほう	
\\	ランプ点灯	ランプてんとう	
\\	ランレングス法	ランレングスほう	
\\	リアルタイム処理	リアルタイムしょり	
\\	リスト演算子	リストえんざんし	
\\	リスト処理	リストしょり	
\\	リスト装置	リストそうち	
\\	リソース割当	リソースわりあて	
\\	リターン値	リターンち	
\\	リチウムイオン電池	リチウムイオンでんち	
\\	リチウムポリマー電池	リチウムポリマーでんち	
\\	リップルけた上げ	リップルけたあげ	
\\	リトルエンディアン式	リトルエンディアンしき	
\\	リバース表示	リバースひょうじ	
\\	リフレッシュ速度	リフレッシュそくど	
\\	リモート印刷	リモートいんさつ	
\\	リモート管理	リモートかんり	
\\	リモート接続	リモートせつぞく	
\\	リモート操作	リモートそうさ	
\\	リモート端末	リモートたんまつ	
\\	リリース情報	リリースじょうほう	
\\	リリース番号	リリースばんごう	
\\	リンク暗号化	リンクあんごうか	
\\	リンク型	リンクがた	
\\	リンク故障	リンクこしょう	
\\	リンク処理定義	リンクしょりていぎ	
\\	リンク状態	リンクじょうたい	
\\	リンク数	リンクすう	
\\	リンク制御プロトコル	リンクせいぎょプロトコル	
\\	リンク端	リンクたん	
\\	リンク付リスト	リンクつきリスト	
\\	リンク付け	リンクづけ	
\\	リンク番号	リンクばんごう	
\\	リング型接続	リングがたせつぞく	
\\	リング遅延	リングちえん	
\\	リード線	リードせん	
\\	ルカーシェビッチ表記法	ルカーシェビッチひょうきほう	
\\	ルックアップ機能	ルックアップきのう	
\\	ルックアヘッド方式	ルックアヘッドほうしき	
\\	ルーチング情報	ルーチングじょうほう	
\\	ルーティング機能	ルーティングきのう	
\\	ルーティング決定	ルーティングけってい	
\\	ルーティング情報	ルーティングじょうほう	
\\	ループ構成体	ループこうせいたい	
\\	ループ状	ループじょう	
\\	ループ抵抗	ループていこう	
\\	ループ不変式	ループふへんしき	
\\	ループ変数	ループへんすう	
\\	ループバック試験	ループバックしけん	
\\	レイアウト表示	レイアウトひょうじ	
\\	レコーディング細目	レコーディングさいもく	
\\	レコーディング年	レコーディングねん	
\\	レコード終わりマーク	レコードおわりマーク	
\\	レコード間隔	レコードかんかく	
\\	レコード間ギャップ	レコードかんギャップ	
\\	レコード記述	レコードきじゅつ	
\\	レコード記述項	レコードきじゅつこう	
\\	レコード形式	レコードけいしき	
\\	レコード長	レコードちょう	
\\	レコード内データ構造	レコードないデータこうぞう	
\\	レコード番号	レコードばんごう	
\\	レコード変数	レコードへんすう	
\\	レコード名	レコードめい	
\\	レコード領域	レコードりょういき	
\\	レジスタ退避領域	レジスタたいひりょういき	
\\	レジスタ長	レジスタちょう	
\\	レジスタ変数	レジスタへんすう	
\\	レジューム機能	レジュームきのう	
\\	レベル指示語	レベルしじご	
\\	レベル番号	レベルばんごう	
\\	レベルセンス方式	レベルセンスほうしき	
\\	レーク受信機	レークじゅしんき	
\\	レーザ印字装置	レーザいんじそうち	
\\	レーザー印字装置	レーザーいんじそうち	
\\	レートをベースとしたフロー制御	レートをベースとしたフローせいぎょ	
\\	レートベース輻輳制御	レートベースふくそうせいぎょ	
\\	ログイン中	ログインちゅう	
\\	ログイン名	ログインめい	
\\	ロジック回路	ロジックかいろ	
\\	ロジック爆弾	ロジックばくだん	
\\	ロスレス圧縮	ロスレスあっしゅく	
\\	ロックアウト機能	ロックアウトきのう	
\\	ロッシー圧縮	ロッシーあっしゅく	
\\	ロボット型検索エンジン	ロボットがたけんさくエンジン	
\\	ロボット検索エンジン	ロボットけんさくエンジン	
\\	ローカル競争	ローカルきょうそう	
\\	ローカル事項	ローカルじこう	
\\	ローカル変数	ローカルへんすう	
\\	ローカル名称	ローカルめいしょう	
\\	ローマ字入力	ローマじにゅうりょく	
\\	ロール用紙	ロールようし	
\\	ロールバック起動指示	ロールバックきどうしじ	
\\	ロールバック起動要求	ロールバックきどうようきゅう	
\\	ロールバック起動サービスプリミティブ	ロールバックきどうサービスプリミティブ	
\\	ワイヤフレーム表現	ワイヤフレームひょうげん	
\\	ワイルドカード文字	ワイルドカードもじ	
\\	ワーク環境	ワークかんきょう	
\\	ワークステーション依存セグメント記憶域	ワークステーションいぞんセグメントきおくいき	
\\	ワークステーション記述表	ワークステーションきじゅつひょう	
\\	ワークステーション型式	ワークステーションけいしき	
\\	ワークステーション光源	ワークステーションこうげん	
\\	ワークステーション識別子	ワークステーションしきべつし	
\\	ワークステーション種別	ワークステーションしゅべつ	
\\	ワークステーション状態リスト	ワークステーションじょうたいリスト	
\\	ワークステーション独立セグメント記憶域	ワークステーションどくりつセグメントきおくいき	
\\	ワークステーション変換	ワークステーションへんかん	
\\	ワード境界	ワードきょうかい	
\\	ワード数	ワードすう	
\\	ワーニング無しに	ワーニングなしに	
\\	ワープロ通信	ワープロつうしん	
\\	ワープロ馬鹿	ワープロばか	
\\	ワールド座標	ワールドざひょう	
\\	ワールド座標系	ワールドざひょうけい	
\\	挙げられる	あげられる	
\\	与える	あたえる	
\\	荒らす・荒す	あらす	
\\	落ちる	おちる	
\\	落とす	おとす	
\\	折り返す	おりかえす	
\\	過疎る	かそる	
\\	桁上げる・けた上げる	けたあげる	
\\	叩く	たたく	
\\	立ち上がる	たちあがる	
\\	取りこぼす	とりこぼす	
\\	取り込む	とりこむ	
\\	流し込む	ながしこむ	
\\	抜け出す	ぬけだす	
\\	抜ける	ぬける	
\\	載せ込む・乗せ込む	のせこむ	
\\	開く	ひらく	
\\	呼び出す	よびだす	
\\	読み込む・読みこむ・読込む	よみこむ	
\\	読み出す	よみだす	
\\	折り返し	おりかえし	
\end{CJK}
\end{document}