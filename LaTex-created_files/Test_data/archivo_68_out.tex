\documentclass[8pt]{extreport} 
\usepackage{hyperref}
\usepackage{CJKutf8}
\begin{document}
\begin{CJK}{UTF8}{min}
\\	ても/でもよろしい(ん)でしょうか	
\\	お
\\	する 
\\	ても 
\\	じゃなくてもよろしいでしょうか 
\\	させていただいてもよろしいでしょうか。 今度行くレストランですが、安くていい所なら少し遠くてもよろしいでしょうか? 
\\	ても) 待ち合わせ場所は駅の近くならどこでもよろしいでしょうか? 
\\	私が代表のスピーチを?!私は構いませんけど、話すのが下手(へた)でもよろしいんでしょうか? 
\\	今度の旅行ですが、食事はあまり豪華じゃなくてもよろしいでしょうか? 
\\	てもかまわないでしょうか 
\\	ちょっと, 少し, んです, んですが, んですけど 
\\	なくても, い
\\	くても 
\\	ないでも...) 
\\	てもかまわない? 
\\	構う・かまう, 
\\	よろしければ
\\	でもいかがですか? よければ
\\	でもどうですか? 
\\	(でも). 
\\	~たらどうですか? 
\\	~てもいいですか? 
\\	に よければ土曜日にでもどうですか?
\\	~んですけど/~んですが、実は~んです	
\\	ん
\\	(けど/が), 
\\	(実) 
\\	場合	
\\	ばあい 
\\	な
\\	場合, 
\\	の場合 
\\	は 
\\	場合 
\\	場合(は)、~。 
\\	時 
\\	場合 
\\	その場合 
\\	場合 
\\	明日の野球の試合が雨で延期になった場合は、今度はいつできるでしょうか。なかなかスケジュールが難しいですね。 この自転車を修理する/した場合は、どのくらい費用がかかるでしょうか。
\\	~時~分前	
\\	今、4時 5分前です。試験は4時 10分までですから、あと15分です。 
\\	今からちょっと寝るけど、9時 10分前になったら起こしてくれない? 
\\	で、~	
\\	では、~ 
\\	では、少々お待ちください。 
\\	まま	
\\	まだ
\\	まま. 
\\	あけたまま 
\\	まだ
\\	まま 
\\	のまま, 
\\	なまま 
\\	い
\\	まま 
\\	ままに、ままの
\\	、まま
\\	、まま
\\	、ままです、
\\	ままにする 
\\	今のまま, 
\\	そのまま 
\\	この資料、直そうと思ったけど、やっぱり今の内容のままにしておこう。 
\\	やらなければならないことをついそのままにしてしまって、あとでやっておけばよかったと思うことがある。 
\\	ことがある 
\\	このまま私達が何もせず地球の温度が上がり続けたら、この辺りは沈みかねない。 
\\	~と、~た	
\\	~たら/ところ、~た、
\\	たら、
\\	た。 
\\	たところ、
\\	た。 
\\	と、
\\	た。 
\\	ところ 
\\	~と、~たので、びっくりした。 
\\	校長に相談したところ、すぐに体育館の使用を許してもらえた。 
\\	古い時計を新しいのに変えたところ、時間を意識するようになって、時間に遅れなくなった。 
\\	~だって	も 
\\	僕だって=僕も =もまた 
\\	だって言った。 
\\	以後	いご =後 三時以後なら、~ 
\\	三時間後に~ 
\\	以後 
\\	京都へ行ったことある? まだ行ったことないんだ。 まだ 
\\	ことない 
\\	上で	
\\	うえ 
\\	上で、
\\	上での
\\	させる上. 
\\	大勢の人の前で発表する上で大切なことは、自信を持って大きい声ではっきりと話すことです。 
\\	上で、
\\	私達が成長していく上で、失敗することは必要なことだと思います。 
\\	生徒達に文化祭の準備をさせる上でいろいろな問題が出てくると思いますが、先生方、ご協力をお願いします。 
\\	今日は負けてしまったが、もう一度しっかり練習した上で、試合を申し込もう。 
\\	この企画については資料をもう一度しっかり読んだ上で、実際にできるかどうかお答えいたします。 
\\	うえでの
\\	これはみんなでよく話し合ったうえでの結論ですから、できれば変えたくありません。 
\\	インターネットで商品を買う上で気をつけなければならないことは、そのサイトが信用できるものかしっかり確認することです。 リーダーを選ぶ上で大切なポイントは、リーダーシップがあって、信頼できる人かどうかだと思う。 他の人の家へ行く時は、先に連絡した上で、行ったほうがいいでしょうね。お留守かもしれませんから。 校長と相談した上で改めて結果をお知らせします。 自信がありそうですね。確実な根拠(こんきょ)があった上で自信だったらいいですが。 
\\	上は  
\\	じょうは 
\\	上では  
\\	うえでは 
\\	国籍上はアメリカ人でも、英語はほとんどできなかったわけで、それが返って日本語を習う上では良かったようだ。 
\\	返って かえって 
\\	ねばらない	"~なければいけない 時代が変わってきたのに、男が変わってくれないので、外で働く女は、三重(さんじゅう)の苦しみを受けねばならなくなりました。(母、妻、仕事) ~ないと。
\\	いけない) 
\\	なきゃ。 
\\	あ、佐藤さんにこの本借りっぱなしだ!早く返さなきゃ! 
\\	もうすぐ) 
\\	いけません>なりません. 
\\	なければなりません・なくてはなりません 
\\	なければいけません・なくてはいけません なければー
\\	なきゃ なくてはー
\\	なくちゃ ならないー
\\	なんない なければいけない 
\\	(~と)同様	"どうよう 
\\	(=と同じように, 
\\	同じ) 原題
\\	が表す通り、彼は「ある国のカッコよさ、つまり大衆に考える文化的影響力は、GNPと同様、その国のパワーとして認知されるべきではないか」と説き、日本は既にその分野の超王国となっていると指摘する。 
\\	とともに. 
\\	は
\\	と同様、
\\	死んだも同様  しんだもどうよう 
\\	たびに
\\	とともに~ 
\\	と 
\\	と
\\	もに~) 
\\	は
\\	とともに
\\	とともに~ 
\\	とともに~ 
\\	ながら, 
\\	につれて 
\\	何だか雪降るにつれて室内の温度も下がってきた気がします。 
\\	経済の発展につれて生活が豊かになった。 
\\	少し待てば雨は止むだろうと思って待っていたが、時間が経つにつれてだんだん強くなってきた。 
\\	病気がよくなるにつれて食欲が出てきて、よかったです。 
\\	ー新聞によると人口増加につれて食料不足(ぶそく)や水不足(みずぶそく)、ごみ問題などが生じるそうですよ。 ーそうなんですか。今から対策を考えておかなくてはいけませんね。 ーその本、おもしろい? ーうん。科学が発達するにつれて・科学の発達につれて私達の生活がどんなふうに変わってきたかよくわかるよ。
\\	に(は)敵わない	かなわない 
\\	素人(しろうと)はいくらがんばってもプロにはかなわない。 
\\	どんなに正当な理由もこの言葉には敵わない。 
\\	敵う かなう 
\\	(匹敵する), 
\\	づらい	
\\	にくい. 
\\	自分の失敗は他人には話しづらい。 
\\	「きまりですから」もタイ語には訳しづらい。 喉が痛いと、ものが食べづらいだろう。 
\\	辛い つらい 
\\	~めにあう 
\\	~立場にある  たちば 
\\	辛く当たる 
\\	に拘る・拘らない	
\\	にこだわる・こだわらない  う
\\	可能にこだわらない   (かのう, 
\\	アクセントに拘っていると、日本語を話すのは難しいかもしれない。 
\\	場所に拘らなければ、安く家を買える。 
\\	拘る こだわる  う
\\	拘泥する  こうでいする 
\\	において	=で 
\\	で. 
\\	明治時代において女性が留学するのは大変珍しいことだった。 
\\	第一回のオリンピックはギリシャにおいて開催(かいさい)された。 
\\	三十分ぐらいした(ころ) 
\\	学生たちが来てから三十分ぐらいしたころ、前田先生が着いた。
\\	てばかり(いる)	
\\	いる・います 
\\	ばかりいないで, ばかりいたけど, ばかりいました, ばかりいても. 
\\	だけ スキーを始めた頃は転んでばかりいたけど、今では上手に滑れるようになりました。 
\\	うちの犬は昔いたずらっ子で私が遊んでいても邪魔してばかり痛んだよ。でも今は大(おとな)の犬になって、とても利口になったんだ。 家では、のんびりしてばかりいるし、私も料理に挑戦してみようかな。 
\\	ばかり 
\\	(まだ)
\\	たばかり 
\\	~たところ 
\\	まだ 
\\	ばかり 
\\	ばかりだ, ばかりで, ばかりなのに. 
\\	~たばかりの頃. 
\\	私は1年前にこの町に引っ越して来たんですが、引っ越して来たばかりの頃は昔の友達を思い出してばかりいたんですが、今は新しい友達ができてよかったですね。 私が日本へ来てもう一ヶ月経ったが(たった)、まだ来たばかりだと友達は思っている。 
\\	新入生はまだ入学したばかりでわからないことが多いので、いろいろと教えてあげてください。 
\\	たて 
\\	たての
\\	焼きたてパン, 塗りたてのペンキ (ぬりたて) 
\\	たて	
\\	たての
\\	焼きたてパン, 塗りたてのペンキ (ぬりたて) 
\\	てばかり(います) 
\\	いる・います 
\\	ばかりいないで, ばかりいたけど, ばかりいました, ばかりいても. 
\\	だけ スキーを始めた頃は転んでばかりいたけど、今では上手に滑れるようになりました。 
\\	うちの犬は昔いたずらっ子で私が遊んでいても邪魔してばかり痛んだよ。でも今は大(おとな)の犬になって、とても利口になったんだ。 家では、のんびりしてばかりいるし、私も料理に挑戦してみようかな。 (まだ)
\\	たばかり 
\\	~たところ 
\\	まだ 
\\	ばかり 
\\	ばかりだ, ばかりで, ばかりなのに. 
\\	~たばかりの頃. 
\\	私は1年前にこの町に引っ越して来たんですが、引っ越して来たばかりの頃は昔の友達を思い出してばかりいたんですが、今は新しい友達ができてよかったですね。 私が日本へ来てもう一ヶ月経ったが(たった)、まだ来たばかりだと友達は思っている。 
\\	新入生はまだ入学したばかりでわからないことが多いので、いろいろと教えてあげてください。 
\\	ばかり 
\\	ことになっている	
\\	ことに決まっている 
\\	ことになっております. おります 
\\	なっております, 
\\	てはいけない
\\	ことになっている, 
\\	ない
\\	ことになっている. 
\\	いただく 
\\	こと. 
\\	(てはいけない 
\\	おります 
\\	この橋は古くて危ないので、5年前から誰も渡ってはいけないことになっております。 
\\	こちらでは写真は取れないことになっておりまして、カメラはこちらでお預かりしております。(あずかりする) 
\\	いただく 
\\	ここからはスリッパを履いていただくことになっておりますので、靴を脱いで、こちらに置いてくださいますか? 
\\	このチケットは普段はご利用できないことになっておりましたが、今回と特別にご利用できることになりました。 
\\	ことになった。 
\\	ことにする 
\\	することにする 
\\	ーこのお店お人は感じがいいね。 ーそうだね。店員さんはみんな親切だから、わからないことは誰に聞いても丁寧に(ていねい)教えてくれるよ。 ーへえ、そうなんだ。これからはこのお店で買うことにしよう。 午後から何だか体調が悪い。無理して仕事を続けて明日休むぐらいなら今日早く帰ったほうがましだと思って、定時になったらすぐに帰ることにした。 
\\	ことにしています 
\\	覚えることにしている 
\\	ように言う	
\\	てください 
\\	お願いする
\\	頼む たのむ 
\\	注意する 
\\	言う. 
\\	言われた 
\\	犯人
\\	は山田さんという女性を誘拐し、その家族の5,000万円払うように要求した。 (ようきゅう) 
\\	ようになる 
\\	い
\\	=くなる, な
\\	=になる, 
\\	ようになる, 
\\	になる 
\\	ように感じます 
\\	なくなる, な
\\	じゃなくなる 
\\	静かじゃなくなる, 
\\	静かにならなくなる) 
\\	ってくる 
\\	だんだん 
\\	最近、インターネットでニュースを読むようになって、だんだん新聞を読まなくなってきています。 
\\	生徒の顔色を見ていると、だんだん心配になってきて、学校へ連絡したあと、すぐ病院へ連れて行くことにしました。 
\\	この町も人がたくさん住むようになって、だんだん静かじゃなくなってきました。 
\\	な
\\	ー昼は時間もないし、手軽に食べられるから、おにぎりはいいですよね。 ーええ。でも、私は、昼だけじゃないんです。朝も、夜もなんです。 ーええ?種類が豊富でも、それはだんだん嫌(いや)になってきませんか? ー嫌になりませんが、だんだん飽きてきましたね。(あきる) 手術が近づくと、生徒はだんだん落ち着かなくなってきましたが、ジムさんの励ましで、少し安心したようでした。 ようにする 
\\	気をつけるようにします。 
\\	10時に間に合うようにするには、今からだと、タクシーに乗らない限り無理です。
\\	出入り口の近くに荷物をたくさん置いておくと、家事や地震で逃げようと思ってもドアが開かない恐れがあるので、荷物は置かないようにしてください。 
\\	午後の授業に遅れないようにしてくださいね。 
\\	ようにしている 
\\	ように
\\	(分かる, いる, ある). 
\\	ために 
\\	食べ物をもらえるようにお願い事をしましょう。 
\\	忙しい時はあいさつをしなかったり、返事をしなかったりすることがあるが、失礼な人だと思われかねない。いくら忙しくても、どんな時でも、気がついたら、あいさつや返事をするように気をつけたい。 
\\	ように 私は山登りが好きで、今までいろいろな山に登ってきました。山登りができるのは体が丈夫なおかげですから、これからも病気や怪我をしないように気をつけたいと思います。 
\\	ように
\\	ように言う。 思うように 
\\	なかなか思うように
\\	(なかなか 
\\	思ったより~ 
\\	ーあれ、逃げられた!なかなか紙の上に乗ってくれません。 ーあ、いけない!紙に穴が空いちゃった。思ったより難しいですね。 
\\	このゲームは思ったより難しいから、一回でクリアというわけにはいかないんじゃないかなと思うよ。 
\\	期待したほど  きたい 
\\	昨日見に行った映画は、期待したほどおもしろくなかった。お金を払って見るにはあたらない映画だった。 
\\	お金を払って見る
\\	見に払う 
\\	ように思う。 
\\	ーちょっと味が薄いように思うんですが。 ーあ、本当ですね。水を多く入れすぎたかな。 多く入れすぎる=
\\	毎日のように 
\\	週, 回, 年, 時間
\\	そのような~ 
\\	そんな
\\	相談センターに聞いたところ、そのような複雑な内容は、短い時間ではお話しかねるとの事でした。 
\\	どんな人からみてもいいと思えるような、もっとかっこいいポスターを作ってみせますよ! 
\\	とても~ない	
\\	なかなか~ない 
\\	包丁(ほうちょう)でもなかなか切れませんよ。 
\\	~らしい	
\\	日本人らしい、 静からしい 海外から来ている人も多いから、日本らしいのにしたら? 
\\	~ようだ 
\\	日本人のようだ、静かなようだ 
\\	たようだった。 
\\	よう) 他の先生は前の会議が延びたようだった。 
\\	は
\\	が延びた。 
\\	安井先生は昨日遅くまで仕事をされたようですよ。 
\\	その地域の人からすると普通でも、他の地域の人からするとびっくりするような習慣もある。 
\\	~みたいだ 
\\	ようだ 日本人みたいだ、にぎやかみたいだ 
\\	だ 
\\	な ~に/く見える 
\\	1のような
\\	1のように
\\	そうだ 
\\	そうだ 
\\	そうだ 
\\	高くなそう/高そうじゃない 元気じゃなそう/げんきそうじゃない ふりそうに(も)ない 
\\	ふる) ご家族が最近ホームに来られないようで、五十嵐さんがちょっとさびしそうにしていたのも気になって・・・。 
\\	そうだ 
\\	町田校長は毎朝6時に起きられるそうです。 
\\	らしい 
\\	そうだ 
\\	ようだ, みたいだ, 
\\	そうだ 
\\	ようだ, みたいだ, 
\\	そうだ 
\\	~ばいいのに	
\\	のに。 
\\	電話かければよかったのに。 
\\	いやならいやだと言えばいいのに。 
\\	といい 
\\	~た方がいい 
\\	ないほうがいい, 
\\	のに~。 な
\\	なのに
\\	ただ	
\\	ただ
\\	しか/だけ
\\	。 ただ
\\	だけ/しか
\\	は
\\	ただ見ているだけです。 
\\	しか~ない 
\\	も, 
\\	が,は, 
\\	を, 
\\	で, に.
\\	わけ(だ)	
\\	つまり~ 
\\	上手なわけだ, 上手だわけだ わけではない・わけじゃない・わけじゃありません 
\\	たわけじゃない... 
\\	な, な
\\	なわけ... 
\\	ん 
\\	仕事が忙しいからと言って、いつも彼女を手伝わないわけじゃないんです。 
\\	このクラスはいつもうるさいわけじゃないんですが、今日は何だか落ち着かないですね。 
\\	ここへ来るのが嫌なわけではないんですが、その日はどうしても来られないんです。 
\\	必ずしも
\\	というわけではない 
\\	わけにはいかない・いきません 
\\	~というわけにはいかない 
\\	という 
\\	という 
\\	という. 
\\	か) 
\\	という 
\\	な
\\	という 
\\	な) 
\\	という= 
\\	""=しないといけない 
\\	金曜日の飲み会には上司も出席するので、私だけ出ないというわけにはいかないんです。 
\\	な
\\	完璧(かんぺき)というわけには行かないが、できるだけミスのないように企画したつもりだ。 
\\	絶対に安全というわけには行きませんが、事故が起きないように毎日厳しく(きびしく)検査しています。 
\\	まだ仕事が終らないんですが、残りは明日にするというわけにはいきませんか? 
\\	職場で日本語を話してはいけないということでしたが、詳しい説明をする時はたぶん日本語をつかわないというわけにはいかないでしょう。 
\\	(でしょう 
\\	日本語を使うというわけにはいかない: 
\\	このゲームは思ったより難しいから、一回でクリアというわけにはいかないんじゃないかなと思うよ。 
\\	ー申し訳ございませんが、此方の商品(しょうひん)は、もう発送段階(はっそうだんかい)に入っていまして、今からキャンセルというわけにはいかないのですが・・・。 ーそこを何とかキャンセルしていただくわけにはいきませんか? 
\\	この1枚1,600円のTシャツを2枚で3,000円にしてもらうというわけにはいかないだろうか。 
\\	~とは限らない 
\\	必ずしもと~ 
\\	/な
\\	だとは 
\\	からと言って 
\\	からって 
\\	からと言って 
\\	(だ)とは, だ 
\\	な
\\	だとは, だ 
\\	家の電話にかけても出ないからと言って、(必ずしも)家に誰もいないとは限りませんよ。 
\\	顔色が悪いからと言って、(必ずしも)病気(だ)とは限らないよ。何かショックなことがあったのかもしれないよ。 
\\	必ず 
\\	きっと. わけがない 
\\	わけじゃない? 
\\	のわけ, な
\\	なわけ, 
\\	わけ 
\\	わけがないじゃない. 
\\	わけがない. 
\\	がないじゃない 
\\	今は夏ですよ。雪が降るわけがないじゃないですか。 
\\	の 
\\	佐々木君が私の彼のわけがありません!ただの友達です。 
\\	小学生の私がそんなに高いホテルに泊まれるわけがありませんよ。 
\\	谷さんにおもしろい
\\	があったら貸してほしいと言われたから、サッカーの試合の
\\	を貸したが、あまりうれしそうではなかった。よく考えたら、谷さんは全然サッカーに興味がないのだ。サッカーの
\\	を貸して喜ぶわけがなかった。 
\\	わけがない 
\\	おもしろくないわけがないですよ。山本監督(かんとく)は映画の賞を何度ももらっている人ですよ。
\\	~限り	"~かぎり 
\\	限り 
\\	である限り, な
\\	な限り 
\\	な
\\	である限り. 
\\	だ限り 
\\	~たら、~できる. 
\\	限りは~ 
\\	限り
\\	限りの
\\	この世界にいる限り、出会いと別れは誰にでもあるものなんじゃないかと思います。 
\\	今、両親とは別々に暮らしているが、両親とも健康(けんこう)である限り(な限り 
\\	は離れていても(はならていて)安心だ。 皆さんが私の力が必要だと言ってくれる限り、精一杯努力するつもりです。 精一杯  せいいっぱい 
\\	車が止まった場所は、見る限り人の家が全くない山道だった。(やまみち) 限りの
\\	思いつく限りの所を探しましたが、なくした財布は見つかりませんでした。 
\\	なくす
\\	なくなる 
\\	台風などのひどい天気で/じゃ ない限り、毎日自転車で通勤しています。 
\\	私は周りが静かで/じゃ ない限り集中できないから、音楽を聞きながら本を読むなんてできません。 
\\	10時に間に合うようにするには、今からだと、タクシーに乗らない限り無理です。
\\	限りでは、~ 
\\	このデータを見る限りでは、この町の人口は10年前ほとんど変わっていない。 
\\	私が知っている限りでは、町田校長が一番日本の歴史について詳しいです。 (一番詳しい) 
\\	ちょっと写して見た限りではかなりきれいに撮れているから、これはいいデジカメなんじゃないかな。 
\\	できれば限り  
\\	に限らない 
\\	に限られている 
\\	は)
\\	に限る 
\\	の 
\\	携帯電話は使い方が(簡単なの・シンプルなの・易しいの(やさしい)・楽なの(らく))に限りますよ。 パソコンを買うなら、軽くて小さいのに限りますね。持ち運びに便利ですから。 
\\	嫌なことは忘れるに限りますよ。 
\\	魚も野菜も新鮮なのに限る。 
\\	~に越したことはない  こした 
\\	~とは限らない 
\\	必ずしもと~ 
\\	/な
\\	だとは 
\\	からと言って 
\\	からって 
\\	からと言って 
\\	(だ)とは, だ 
\\	な
\\	だとは, だ 
\\	家の電話にかけても出ないからと言って、(必ずしも)家に誰もいないとは限りませんよ。 
\\	顔色が悪いからと言って、(必ずしも)病気(だ)とは限らないよ。何かショックなことがあったのかもしれないよ。 
\\	わけではない・わけじゃない・わけじゃありません 
\\	必ずしも
\\	というわけではない 
\\	必ず 
\\	きっと. ~ほど~は
\\	ない 
\\	に限る 
\\	ことほど
\\	ことはない 
\\	~はない 
\\	は 
\\	ない. 
\\	両親ほど私のことを心配してくれる人は他にいません。 (ほかに) 
\\	自分の国の料理ほど食べてほっとする料理はありませんから。 
\\	新宿ほど一晩中にぎやかで明るい町はないと聞きましたが、本当ですか? 
\\	疲れると、友達に電話をかけたくなります。仲のいいともだちとのおしゃべりほどリラックスできることはありませんから。 
\\	以前は
\\	町(ちょう)ほどにぎやかな所はなかったが、今は人口が少なくなってさびしくなってしまった。 
\\	自転車ほど環境(かんきょう)にいい乗物はないよ。空気が汚れないから。 それほど~(ない) 
\\	作るのはそれほど難しくないし、家でも家族と作って見ます。 
\\	~に越したことはない	"~にこしたことはない 
\\	は)
\\	に限る 
\\	の 
\\	携帯電話は使い方が(簡単なの・シンプルなの・易しいの(やさしい)・楽なの(らく))に限りますよ。 パソコンを買うなら、軽くて小さいのに限りますね。持ち運びに便利ですから。 
\\	嫌なことは忘れるに限りますよ。 
\\	魚も野菜も新鮮なのに限る。 
\\	~ほど~は
\\	ない 
\\	に限る 
\\	ことほど
\\	ことはない 
\\	~はない 
\\	は 
\\	ない. 
\\	両親ほど私のことを心配してくれる人は他にいません。 (ほかに) 
\\	自分の国の料理ほど食べてほっとする料理はありませんから。 
\\	新宿ほど一晩中にぎやかで明るい町はないと聞きましたが、本当ですか? 
\\	疲れると、友達に電話をかけたくなります。仲のいいともだちとのおしゃべりほどリラックスできることはありませんから。 
\\	以前は
\\	町(ちょう)ほどにぎやかな所はなかったが、今は人口が少なくなってさびしくなってしまった。 
\\	自転車ほど環境(かんきょう)にいい乗物はないよ。空気が汚れないから。 それほど~(ない) 
\\	作るのはそれほど難しくないし、家でも家族と作って見ます。 
\\	~てある	
\\	てある, が 
\\	を 
\\	を 
\\	「持ち込み」というのは外から何かが持って入るということで、「持ち込み
\\	と書いてあるカラオケなどはお菓子やジュースを持って入っていいんですよ。 
\\	書きっぱなしの日記が机に置いてあります。 
\\	~ておく 
\\	~中に	
\\	今日中, 今年中 (ことし)= じゅう 
\\	ちゅう 毎年=まいとし ~までに (に 
\\	いつまでに 
\\	まで 
\\	に) 
\\	から~まで 
\\	までに 
\\	まで 
\\	五時まで、ここにいます。 五時までに、ここに来てください。
\\	させる	
\\	来させる=こさせる  る
\\	=させる  う
\\	/せる 
\\	ない 
\\	させられる 
\\	来させられる=こさせられる  る
\\	=させられる  う
\\	/せられる 
\\	/される, 
\\	ささ 
\\	から
\\	(さ)せていただきます 
\\	から 
\\	私, こちら 
\\	その件については今日他の先生方とも相談して、明日(あす)校長からご連絡させていただきます。(ご 
\\	こちらからいくつかアイディアを出させていただきますので、その中から決めていただいてもいいでしょうか? 
\\	すみません。私からも意見を言わせていただきたいのですが、よろしいでしょうか? 
\\	(からも, 
\\	もから) 今日はいろいろと話を聞いていただいてありがとうございました。また何かありましたら、次回は私達からご相談に伺わせていただきたいと思っておりますので、よろしくおお願いいたします。 
\\	ご 
\\	伺う 
\\	ー駅からそちらまでどうやって行ったらいいですか? ー駅に着かれた時にお電話をいただければ、こちらからお迎えに行かせていただきますよ。 
\\	ー先週の見合い、どうでしたか? ーうーん・・・。いい人だったんですが、ちょっと・・・。 それで、こちらから断らせていただくことにしたんです。 ーまずは最近の子供達が抱えている問題点について、こちらから30分ほどお話をさせていただきます。その後(ご)、ご意見やご質問をお聞きします。 ーはい。わかりました。
\\	意外に	いがいに 
\\	は意外に
\\	以外の
\\	ほかに(も) 
\\	以外に
\\	以上・以下 いじょう・いか 
\\	首相以下の政治家  しゅしょういかのせいじか 
\\	以上前に 300年以上前に宝物を埋めたという資料が発見され、それもとにして町中の人が驚きました。(おどろきました) 以内  いない 五週か以内に  ごしゅうかんいないに 
\\	以外	いがい 
\\	以外の
\\	ほかに(も) 
\\	以外に
\\	意外に  いがいに 
\\	は意外に
\\	以上・以下 いじょう・いか 
\\	首相以下の政治家  しゅしょういかのせいじか 
\\	以上前に 300年以上前に宝物を埋めたという資料が発見され、それもとにして町中の人が驚きました。(おどろきました) 以内  いない 五週か以内に  ごしゅうかんいないに 
\\	始める	"る
\\	勉強し始めたの 
\\	読み始める 
\\	終る (おわる) 
\\	終える 
\\	~を始め  はじめ 
\\	~を始め、
\\	3など、~。 始めの~ 
\\	初めて
\\	を
\\	て初めての
\\	が始まって初めての
\\	1て初めて
\\	こんなに笑ったのは、生まれて初めてです。 
\\	のは、
\\	てはじめてだ。 
\\	て初めて、
\\	のだ。) ー安井先生はとても優しくてすてきですよ。 ーえ?そんなうれしいこと、生まれて初めて言われました。 日本に来て初めての日本語でのスーぴちでしたから、少し緊張しました。 
\\	家族と別々に住んでみて初めて、家族のありがたみがわかりました。 
\\	知り合いに連れて行ってもらってはじめて、歌舞伎を見ました。 
\\	はじめて 
\\	連れて行く, 
\\	て.) ー山田さん、大ざっぱに見えるけど、すごく几帳面だよ。部屋もいつもきれいにしてるし。 ーうん、この間(あいだ)山田さんのうちに遊びに行って始めてわかった。 一度死んだつもりで初めからがんばります。 
\\	初めて	"初めて
\\	を
\\	て初め 
\\	て初めての
\\	が始まって初めての
\\	1て初めて
\\	こんなに笑ったのは、生まれて初めてです。 
\\	のは、
\\	てはじめてだ。 
\\	て初めて、
\\	のだ。) ー安井先生はとても優しくてすてきですよ。 ーえ?そんなうれしいこと、生まれて初めて言われました。 日本に来て初めての日本語でのスーぴちでしたから、少し緊張しました。 
\\	家族と別々に住んでみて初めて、家族のありがたみがわかりました。 
\\	知り合いに連れて行ってもらってはじめて、歌舞伎を見ました。 
\\	はじめて 
\\	連れて行く, 
\\	て.) ー山田さん、大ざっぱに見えるけど、すごく几帳面だよ。部屋もいつもきれいにしてるし。 ーうん、この間(あいだ)山田さんのうちに遊びに行って始めてわかった。 一度死んだつもりで初めからがんばります。 
\\	始める る
\\	勉強し始めたの 
\\	読み始める 
\\	終る 
\\	終える 
\\	~を始め  はじめ 
\\	~を始め、
\\	3など、~。
\\	なるべく	
\\	できれば 教室の外でも、なるべく日本語を使ったほうがいいでしょう。 
\\	宿題は、なるべく次の日に出してください。 
\\	できるだけ 
\\	この時のポイントは、フライパンに対してお箸(はし)をできるだけまっすぐ立てて、こんなふうにフライパンとお箸を両方動かしながら、早く混ぜる(まぜる)ことです。 完璧(かんぺき)というわけには行かないが、できるだけミスのないように企画したつもりだ。 
\\	一人だと外で食事をしがちなので、栄養を考えてできるだけ野菜を多く食べるようにしています。 
\\	外で食事をする  そとでしょくじ 
\\	外で食べる/
\\	ても	
\\	い
\\	=くても 
\\	/な
\\	=でも 
\\	であっても 
\\	だって?? 
\\	卒業式の日にでもみんなの前で歌おと思って準備していたんですよ。 ても 
\\	明日はどうせ暇だし、映画にでも行こうかな。 
\\	どうせ今日は大雨で野球は中止になると思うし、その時は家で本でも読もう。 
\\	ても・でも~たら~ 
\\	ても 
\\	たら 
\\	ても. もし上司に飲みに誘われても、都合が悪かったら無理に付き合わなくても大丈夫ですよ。 
\\	今は平気でも、夜になったらまた熱が出るかもしれないからもう少し休んでください。 
\\	と言っても
\\	だ 
\\	~からと言って(も、) 
\\	わり 
\\	そろそろ寝たほうがいいんじゃないですか?冬休みだからと言って、いつまでも寝ないでいると朝起きられなくなりますよ。 
\\	仕事が忙しいからと言って、いつも彼女を手伝わないわけじゃないんです。 
\\	とは限らない 
\\	からって 
\\	からと言って 
\\	(だ)とは, だ 
\\	な
\\	だとは, だ 
\\	家の電話にかけても出ないからと言って、(必ずしも)家に誰もいないとは限りませんよ。 
\\	顔色が悪いからと言って、(必ずしも)病気(だ)とは限らないよ。何かショックなことがあったのかもしれないよ。 
\\	そうかと言って 
\\	も! 
\\	でも, けれども. 
\\	わけにはいかない, わけではない, 
\\	それにしても 
\\	1、それにしても
\\	1。それにしても
\\	2。 返って  かえって 
\\	1、返って
\\	~は、返って~。 
\\	たらかえって 
\\	たら 
\\	かえって. 
\\	たら/かえって 
\\	体を急に動かしたらかえって体によくないかもしれませんが、横になってばかりいるのもよくないですよ。 
\\	怒られると思いながらはっきり自分の意見を言ったら、かえって先生に褒められた。 
\\	お金を使う  
\\	バーゲンで安いからと思ってあまり考えずに買っていたら、かえってお金を使いすぎてしまった。 
\\	新鮮な野菜を食べたくて、自分で野菜を育てて見たら、できた野菜が自分の子供のようにかわいくなって、かえって食べられない。 
\\	野菜を育てる いろいろと考えずに、思い切ってやったらかえってうまくいきました。 引越しの手伝いに行ったらかえって邪魔になってしまいました。 どうしても 
\\	ここへ来るのが嫌なわけではないんですが、その日はどうしても来られないんです。 
\\	何と言っても 
\\	何と言っても
\\	ても 
\\	どこへ行っても, 何を食べても いくら: 
\\	どんなに: 
\\	いくら 
\\	この店では、1時間以内ならいくら食べても1,000円です。 
\\	それに、いくら家に必要な物を準備していても、外出中だったら持って行けないのではないかと思いますが・・・。 彼がいくら背が高くても、あの壁の時計には届かないだろう。 
\\	どんな 
\\	どんなに素敵でも(すてき)このかばんは買えません。高すぎます。 
\\	何を頼んでもいいんですか? 
\\	ナオミさんは何をしても上手にできる。 
\\	ーこのお店お人は感じがいいね。 ーそうだね。店員さんはみんな親切だから、わからないことは誰に聞いても丁寧に(ていねい)教えてくれるよ。 ーへえ、そうなんだ。これからはこのお店で買うことにしよう。 彼は家にいるのが好きなので、1週間どこへも行かなくても平気だ。 
\\	ーアランさんは何を食べてもおいしいと言いますね。好き嫌いはないんですか? ーないですよ。日本料理は特に何でも好きです。 どんなに悲しいことや落ち込むことが当てえも、この曲を聞けば元気になれる。 日本は地震の多い国だから、地震がいつ来ても慌てることのないように、日ごろから必要な物を準備しておく必要があります。 何歳になっても知らないことはたくさんある。世界は知らないことだらけだ。 
\\	て 
\\	どんな時も自分より相手を大切に思う気持ちこそすばらしいというものだ。 
\\	忙しい時はあいさつをしなかったり、返事をしなかったりすることがあるが、失礼な人だと思われかねない。いくら忙しくても、どんな時でも、気がついたら、あいさつや返事をするように気をつけたい。 
\\	どんな仕事でも楽だなんてことはありません。(らくだ) 
\\	さえ 
\\	こと. 
\\	も. 
\\	さえも ~て、~さえだ/
\\	さえ
\\	ば 
\\	さえすれば 
\\	てさえいれば 
\\	が, を, は, も 
\\	に, で, 
\\	にさえ 
\\	さえ 
\\	さえ, 
\\	とは言え 
\\	先生、お待たせしてしまって申し訳ありません。急用とは言え、あやまるよりほかありません。 
\\	~にしろ~にしろ~ 
\\	~にせよ~にせよ 
\\	にしても 
\\	にしろ
\\	にしろ 
\\	にしろ
\\	にしろ. 
\\	にしろ 
\\	どちにしろ 
\\	にしても, 
\\	レストランに行くにしても行かないしても、連絡してください。 
\\	どちにしろ 
\\	男の子か女の子が、どちらにしろ早く生まれてきてほしい。 
\\	森さんは怒るにしろ泣くにしろ、よく騒ぐ人だ。 
\\	おしゃべりが大好きな弟のことは、うるさいにしろおとなしいにしろ、気になってしまう。 
\\	上手にしろ下手にしろ、精一杯やることが何でも大切です。 
\\	手紙にしろメールにしろ、とにかく返事をください。 
\\	べき	
\\	する= すべき 
\\	するべき. すべき 
\\	=べきではない, 
\\	べき 
\\	べきだった 
\\	自分が今何をすべきで、何をすべきでないのかしっかり考えて行動したい。 
\\	て行動する いつ...すべきか 
\\	はず 
\\	のはず, な
\\	なはず, 
\\	い
\\	はず, 
\\	はず はず 
\\	でしょう: あのレストランのご主人のことだから、予算を言えばその中でメニューを考えて作ってくれるはずだ。 
\\	~はずだった 
\\	はずじゃなかった・はずではなかった 
\\	のはず, な
\\	なはず 卵を4個も使ったので、夕食のオムレツは大きいはずだったが、零してしまったので小さくなった。 確か~たはずです(が) 
\\	はず 
\\	たはずです. 
\\	が 
\\	けれども, 
\\	よ. 
\\	はずだった 
\\	確か. ここの本なら、確か2週間借りられたはずですが。1週間しか借りられないと言われたんですが? 
\\	北海道は、確か5月でも寒かったはずですが。でも旅行は北海道にしますか? 
\\	このお店が閉まるのは確かそんなに早くなかったはずですよ。どうして今日は閉まっているんだろう? 
\\	山田さん、授業に出てないんですか? 確か、朝は元気だったはずですが・・・ 具合が悪くなったのかなあ? 
\\	確かセールの開始は明後日(あさって)の朝10時からだったはずですけど、早めに行って並びますか? 
\\	確か、去年の(期末試験の)問題は平均点が85点くらいで簡単だったはずですが・・・。 私~なきゃならない 
\\	私. 
\\	べき. ないといけない、ないと 
\\	珍しくありません	=珍しくないです 良くありません, 
\\	良くないです. そんなに安くはないけど、~ 
\\	は 
\\	そんなにすっごい反応でしたか? 
\\	そんなに 
\\	これはみんなでよく話し合ったうえでの結論ですから、できれば変えたくありません。 
\\	ろくに~ない	
\\	あまり. 
\\	別に~ない 
\\	別に。 
\\	だから~	
\\	それで 
\\	私はこの曲が好きだからだ。
\\	かな	
\\	~かなと思っていたんです。 
\\	かしら 
\\	確か	"たしか 
\\	かも. ー何をお探しですか? ー時計を探しているんです。確かここに置いたと思うんですが、ないんです・・・。 ーあ、
\\	さんがお探しの時計って、もしかしてこれじゃありませんか? 確かに~ 
\\	確か~たはずです(が) 
\\	はず 
\\	たはずです. 
\\	が 
\\	けれども, 
\\	よ. 
\\	はずだった 
\\	ここの本なら、確か2週間借りられたはずですが。1週間しか借りられないと言われたんですが? 
\\	北海道は、確か5月でも寒かったはずですが。でも旅行は北海道にしますか? 
\\	このお店が閉まるのは確かそんなに早くなかったはずですよ。どうして今日は閉まっているんだろう? 
\\	山田さん、授業に出てないんですか? 確か、朝は元気だったはずですが・・・ 具合が悪くなったのかなあ? 
\\	さえ	
\\	こと. 
\\	も. 
\\	さえも ~て、~さえだ/
\\	さえ
\\	ば 
\\	さえすれば 
\\	てさえいれば 
\\	が, を, は, も 
\\	に, で, 
\\	にさえ 
\\	さえ 
\\	さえ, 
\\	ても 
\\	い
\\	=くても 
\\	/な
\\	=でも 
\\	であっても 
\\	だって?? 
\\	卒業式の日にでもみんなの前で歌おと思って準備していたんですよ。 ても 
\\	明日はどうせ暇だし、映画にでも行こうかな。 
\\	どうせ今日は大雨で野球は中止になると思うし、その時は家で本でも読もう。 
\\	ても・でも~たら~ 
\\	ても 
\\	たら 
\\	ても. もし上司に飲みに誘われても、都合が悪かったら無理に付き合わなくても大丈夫ですよ。 
\\	今は平気でも、夜になったらまた熱が出るかもしれないからもう少し休んでください。 
\\	と言っても
\\	だ 
\\	~からと言って(も、) 
\\	わり 
\\	だ/な
\\	だから 
\\	からと言っても、 
\\	から
\\	と言っても
\\	と言って 
\\	のに 
\\	からと言って 若いからと言って、あまり無理をしてはいけない。 
\\	歌手は人気があるからと言って、歌が上手かどうかはわからない。 
\\	きれいだからと言って、自然お花を取るのはかわいそうだと私は思います。 
\\	仕事が忙しいからと言って、いつも彼女を手伝わないわけじゃないんです。 
\\	とは限らない 
\\	からって 
\\	からと言って 
\\	(だ)とは, だ 
\\	な
\\	だとは, だ 
\\	家の電話にかけても出ないからと言って、(必ずしも)家に誰もいないとは限りませんよ。 
\\	顔色が悪いからと言って、(必ずしも)病気(だ)とは限らないよ。何かショックなことがあったのかもしれないよ。 
\\	だからと言って 
\\	でも、だからと言って日本語だけにするおのなあ・・・。 
\\	そろそろ寝たほうがいいんじゃないですか?冬休みだからと言って、いつまでも寝ないでいると朝起きられなくなりますよ。 
\\	そうかと言って 
\\	も! 
\\	でも, けれども. 
\\	わけにはいかない, わけではない, 
\\	それにしても 
\\	1、それにしても
\\	1。それにしても
\\	2。 返って かえって 
\\	1、返って
\\	~は、返って~。 
\\	たらかえって 
\\	たら 
\\	かえって. 
\\	たら/かえって 
\\	体を急に動かしたらかえって体によくないかもしれませんが、横になってばかりいるのもよくないですよ。 
\\	怒られると思いながらはっきり自分の意見を言ったら、かえって先生に褒められた。 
\\	お金を使う  
\\	バーゲンで安いからと思ってあまり考えずに買っていたら、かえってお金を使いすぎてしまった。 
\\	新鮮な野菜を食べたくて、自分で野菜を育てて見たら、できた野菜が自分の子供のようにかわいくなって、かえって食べられない。 
\\	野菜を育てる いろいろと考えずに、思い切ってやったらかえってうまくいきました。 引越しの手伝いに行ったらかえって邪魔になってしまいました。 どうしても 
\\	ここへ来るのが嫌なわけではないんですが、その日はどうしても来られないんです。 
\\	何と言っても 
\\	何って, 
\\	何て, 
\\	何と言っても
\\	が一番いい。
\\	旅行は何と言っても京都が一番! 清水寺(きよみずでら)、金閣寺(きんかくじ)など多くの文化遺産がありますから。 
\\	お肉なら何といってもコムラストアだって聞いたんですけど、コムラストアの場所、知っている? 
\\	ても 
\\	どこへ行っても, 何を食べても いくら: 
\\	どんなに: 
\\	いくら 
\\	この店では、1時間以内ならいくら食べても1,000円です。 
\\	それに、いくら家に必要な物を準備していても、外出中だったら持って行けないのではないかと思いますが・・・。 彼がいくら背が高くても、あの壁の時計には届かないだろう。 
\\	どんな 
\\	どんなに素敵でも(すてき)このかばんは買えません。高すぎます。 
\\	何を頼んでもいいんですか? 
\\	ナオミさんは何をしても上手にできる。 
\\	ーこのお店お人は感じがいいね。 ーそうだね。店員さんはみんな親切だから、わからないことは誰に聞いても丁寧に(ていねい)教えてくれるよ。 ーへえ、そうなんだ。これからはこのお店で買うことにしよう。 彼は家にいるのが好きなので、1週間どこへも行かなくても平気だ。 
\\	ーアランさんは何を食べてもおいしいと言いますね。好き嫌いはないんですか? ーないですよ。日本料理は特に何でも好きです。 どんなに悲しいことや落ち込むことが当てえも、この曲を聞けば元気になれる。 日本は地震の多い国だから、地震がいつ来ても慌てることのないように、日ごろから必要な物を準備しておく必要があります。 何歳になっても知らないことはたくさんある。世界は知らないことだらけだ。 
\\	て 
\\	どんな時も自分より相手を大切に思う気持ちこそすばらしいというものだ。 
\\	忙しい時はあいさつをしなかったり、返事をしなかったりすることがあるが、失礼な人だと思われかねない。いくら忙しくても、どんな時でも、気がついたら、あいさつや返事をするように気をつけたい。 
\\	どんな仕事でも楽だなんてことはありません。(らくだ) 
\\	とは言え 
\\	先生、お待たせしてしまって申し訳ありません。急用とは言え、あやまるよりほかありません。 
\\	~にしろ~にしろ~ 
\\	~にせよ~にせよ 
\\	にしても 
\\	にしろ
\\	にしろ 
\\	にしろ
\\	にしろ. 
\\	にしろ 
\\	どちにしろ 
\\	にしても, 
\\	レストランに行くにしても行かないしても、連絡してください。 
\\	どちにしろ 
\\	男の子か女の子が、どちらにしろ早く生まれてきてほしい。 
\\	森さんは怒るにしろ泣くにしろ、よく騒ぐ人だ。 
\\	おしゃべりが大好きな弟のことは、うるさいにしろおとなしいにしろ、気になってしまう。 
\\	上手にしろ下手にしろ、精一杯やることが何でも大切です。 
\\	手紙にしろメールにしろ、とにかく返事をください。 
\\	いつの間にか	"いつのまにか 
\\	その間に、暇つぶしに時刻表を眺めていたら(ながめ)、いつの間にか電車の路線にも詳しくなっていたんです。
\\	~に慣れる	"~になれる る
\\	の. 
\\	に 
\\	~ずつ	
\\	と
\\	を
\\	ずつ
\\	と
\\	の
\\	を
\\	ずつ
\\	を
\\	ずつ
\\	。 一人一枚ずつ
\\	と
\\	と
\\	一つずつ。  
\\	少しずつ  
\\	みんなの分  ぶん 
\\	でも、部活が終った後は、後輩達に優しく、時々みんなの分のジュースを買ってあげたりしていました。 
\\	気	
\\	が気になる 
\\	の おしゃべりが大好きな弟のことは、うるさいにしろおとなしいにしろ、気になってしまう。 
\\	気になる 
\\	ー全く見られないんですか? 
\\	ーいえ、全く見られないわけじゃないんですが、やっぱり気になって。 
\\	~が気にかかる 
\\	気をする 
\\	気をしますか。 
\\	気づかう  
\\	気遣う. 気にしないでください 
\\	(心配しないで)
\\	に)気をつける 
\\	気をつけるようにします。 
\\	いろいろお世話になりました。どうぞお体にお気をつけください。 
\\	気をつかう (使う?) 
\\	(~に)気がつく 
\\	の, 
\\	こと 
\\	気づく 
\\	気がついたら、お客さんが立ち上がって大きな拍手(はくしゅ)をしてくれていた。 (~が)きに入る・気に入ってる   いる・いっている 
\\	この店の雰囲気、どうです?最近気に入って(い)まして、よく来るんですよ。 それに、公民館のわりに白っぽいちょっとおしゃれな建物で、そこも気に入ってます。(いってます) 
\\	お気に入り  いり 
\\	~そうな気がする 
\\	だから、日本人が温かく迎えてくれそうな気がした。 
\\	くれる? 
\\	~そう? 温かく あたたかく 
\\	~ような気がする 
\\	しませんか, しました 
\\	ような, 
\\	たような, 
\\	ような (まじめ), 
\\	たような...した 
\\	のような, な
\\	なような この分だとすぐに積もるような気がします。 (つもる) 
\\	今日は昨日より少し涼しいような気がしませんか? 
\\	夜はいつも静かですが、今晩は特に静かなような気がします。 
\\	ような: 今度のプレゼンテーションは失敗しないような気がします。 
\\	たような: 作りすぎたような気がします。 
\\	ような: でも、校長先生ほどまじめではない・まじめじゃないような気がする。 
\\	たような...した: 隣の部屋から人の声がしたような気がしました。 日本語が少し上達したような気がします。 
\\	日本人は自分の入ってるグループを重んじるような気がします。 
\\	佐藤先生(さとう)が、去年行った旅館は1万円ぐらいで泊まれたと言っていたような気がする、はっきり覚えていない。 気がする 
\\	何だか雪降るにつれて室内の温度も下がってきた気がします。 
\\	気を許す ゆるす 
\\	やる気が出てくる 褒められたとたん(二)うれしくなって、やる気が出てきた。 
\\	ええ、ぜひたくさんの人に来てほしいとやる気が出てきましたよ。 
\\	せっかく	
\\	わざわざ 
\\	私の提案は大した提案じゃないから、みんなでわざわざ集まって検討するには当たらないよ。 
\\	なかなか でも、なかなかいい服がなかったので、結局手袋だけしか買わなかった。 ずっと行きたくて、なかなかチャンスがなかったが、今年こそ北海道へ行きたい。 
\\	日曜日は暇かと言うと、そうではない。平日になかなかできない家の仕事がたくさなるのだ。 
\\	引越しは荷物をまとめるやら市役所で手続きをするやら、やらなくてはいけないことが多くて、なかなか休めない。 
\\	ず	
\\	なくて 
\\	ないで 
\\	に: このまま私達が何もせず地球の温度が上がり続けたら、この辺りは沈みかねない。 
\\	ずに 
\\	ないで 
\\	ない 
\\	読む=読まない=読まず  貸す=貸さない=貸さず  来る=来ない(こない)=来ず(こず) 
\\	する=せず
\\	ことは
\\	(が)	
\\	な
\\	なことは~
\\	でけっこう(です)	
\\	でいいです 
\\	てもいいです 
\\	で構わないです・かまわない・かまいません  
\\	構う  かまう 
\\	いつでも構いません。 
\\	何
\\	も	
\\	たくさn 
\\	も
\\	ぜんぜん 
\\	ある, ない, 
\\	どんな
\\	でも	
\\	が, を 
\\	と, から, に(でも) 
\\	なんでも 
\\	か
\\	かと
\\	心配する, 考える, 迷う (まよう 
\\	んじゃないか(と思う)	
\\	のではないだろうか/のではないか のではないでしょうか 
\\	かも. 
\\	/な
\\	なのではない... と思う 
\\	心配する 
\\	店を出て行くお客さんの様子を見ていると、あの店の料理は評判ほどおいしくないのではないかと思います。 
\\	途中でコースを間違えなかったら、木村さんがマラソン大会で1位(いちい)だったのではないかと思う。 
\\	な
\\	これだけ資料があれば十分なのではないかと思います。 
\\	そして、山田さんを置いて、その後(ご)犯人が逃げた場所は、大阪なのではないかと思う。 なと思う 
\\	例えば少し寒気がするとか頭が痛いとか、おかしいなと思ったらすぐ知らせてください。 
\\	むしろ	
\\	は
\\	よりむしろ
\\	は、
\\	の方が
\\	よりむしろ
\\	より、
\\	のほうが、むしろ
\\	。 夏風邪は冬の風邪よりむしろ治りにくいから、気をつけてください。
\\	って	"そうです, 
\\	(と言った=って言った=って) 
\\	だって。, 
\\	って。 
\\	聞きました, 
\\	としては	
\\	電気街として知られている。 
\\	によって 
\\	によって
\\	にとって 
\\	に. 
\\	に. 日本人にとって~  
\\	~によると 
\\	ー新聞によると人口増加につれて食料不足(ぶそく)や水不足(みずぶそく)、ごみ問題などが生じるそうですよ。 ーそうなんですか。今から対策を考えておかなくてはいけませんね。 ~をもとに(して) 
\\	によると 
\\	これらの写真をもとに、プレゼンテーションソフトを使って授業を進めて行こうと思います。 
\\	子供の頃の思い出をもとに(して)30年前のこの町の様子を絵に描いてみました。 
\\	この映画は本当にあったことをもとに(して)作られた。 
\\	インターネットで集めた情報をもとに(して)まとめた情報をブログで公開したら、みんなからわかりやすい、ありがたいというコメントをたくさんもらった。 
\\	目撃者(もくげきしゃ)の証言(しょうげん)もとにして犯人の絵を描きました。 300年以上前に宝物を埋めたという資料が発見され、それもとにして町中の人が驚きました。(おどろきました) 日本で経験したこともとにして異文化についての本を書こうと思っています。 今年は生徒へのアンケートもとにして修学旅行の行き先を考えます。
\\	~たっけ、だっけ	
\\	かな, でしょうか 
\\	だったけ, でしたっけ 
\\	ですっけ. 
\\	だっけ, 
\\	だったっけ, 
\\	でしたっけ い
\\	かったっけ 
\\	寒かったっけ) 
\\	たっけ, 
\\	ましたっけ  
\\	行ったっけ) ん 
\\	~んだっけ, 
\\	~んだったっけ, 
\\	~んでしたっけ ~んじゃありませんでしたか ~んじゃなかったでしたっけ ~んじゃなかったの? 
\\	/な
\\	なんじゃな... 
\\	人が多い場所は苦手なんじゃありませんでしたか。 
\\	て、
\\	てしまう	
\\	ちゃう 
\\	てしまう 
\\	じゃう 
\\	でしまう. 僕だって、明後日の朝出さなきゃならないレポートがあるから、今晩徹夜で書いちゃおうと思ってるんだ。 電車の中に珍しい服装をした人がいたので、ついじっくり見てしまって、反対にその人に変な目で見られた。 
\\	さっきはちょっといらいらしていて、つい怒ってしまった。申し訳なかった。 
\\	彼女の誕生日をつい忘れてしまったために、彼女に怒られた。 
\\	やらなければならないことをついそのままにしてしまって、あとでやっておけばよかったと思うことがある。 
\\	私がつい注文し忘れてしまって・・・。
\\	~的	"~てき 
\\	な 
\\	に
\\	何の	なんの? 
\\	ー一緒に行かない? ー何の映画。 何に  なんに 
\\	なにに 出し物は何にしましょうか?
\\	の中で
\\	と
\\	とどち(ら)の方が
\\	の仲で~  
\\	も
\\	も
\\	ずっと 
\\	やっぱり	
\\	やっぱり 
\\	やはり 
\\	ー全く見られないんですか? 
\\	ーいえ、全く見られないわけじゃないんですが、やっぱり気になって。 
\\	より仕方がない	
\\	しかたない, しょうがない. 仕方なく 
\\	仕方なく買った
\\	より.
\\	お~になる, いらしゃる. 
\\	分かる, できる 
\\	こそ	
\\	は
\\	てこそ)
\\	が, を, も, は. 
\\	へこそ, とこそ, 
\\	こそ, 
\\	てこそ, 
\\	であればこそ
\\	とにかく~. 
\\	特に・とくに. 
\\	おとなしい子供こそしっかり気をつけて見てあげたい。 
\\	ずっと行きたくて、なかなかチャンスがなかったが、今年こそ北海道へ行きたい。 
\\	であれば 
\\	元気でいればこそ、悲しいこともあるけれど、それ以上にきっと楽しいこともありますよ。 
\\	こそ 富士山は雪があってこそきれいに見える。 
\\	運動は続けてこそ効果が出るんじゃないだろうか。 
\\	私は、その土地の人と触れ合ってこそ、その土地のことをよく理解できると思うんです。 いつも、日曜日は6時には起きています。休みの日こそ早起き(を)して、日ごろなかなかできないことをしようと思っているんですよ。 どんな時も自分より相手を大切に思う気持ちこそすばらしいというものだ。 
\\	それでこそ男だ。 
\\	特に 
\\	ーアランさんは何を食べてもおいしいと言いますね。好き嫌いはないんですか? ーないですよ。日本料理は特に何でも好きです。
\\	かまたは	
\\	それとも 
\\	。それとも
\\	、それとも
\\	。 それとも、~。 もまた  
\\	若しくは  もしくは、~ 
\\	もしくは、週末であれば、18日は今のところ大丈夫です。 
\\	1ほど
\\	程 
\\	ほどではない. でも、心配するほどではなく、数日で調子を取り戻すことができた。 動けないほどお腹がいっぱいだ。 
\\	東京はびっくりするほど人が多い。 
\\	店を出て行くお客さんの様子を見ていると、あの店の料理は評判ほどおいしくないのではないかと思います。 
\\	2て
\\	1ほどです。 
\\	~ば~ほど 
\\	/な
\\	であればあるほど, 
\\	ある. 程 
\\	小さければ小さいほど毒が強い。  勉強しなければ勉強しないほど~ 
\\	いいレストランであればあるほど良い。 
\\	説明は簡単であればあるほどいい。 
\\	1くらい
\\	ぐらい 
\\	な
\\	なくらい 
\\	くらいの
\\	ぐらい. 
\\	彼女となら結婚してもいいと思えるぐらいすてきな女性に出会った。 
\\	ちょっと失敗したぐらい大丈夫ですよ。そんなに気にしないでください。 
\\	まだ3月だというのに、今日の天気は暑いくらいだ。 
\\	会議に来て行くスーツは、ちょっと地味なくらいがちょっといいよ。 
\\	少しいたずらをしても、元気すぎるぐらいの子供が好きだ。 
\\	リンリンさんは指を痛めるぐらい何度もピアノの練習をした。 リンリンさんは本番の時、お客さんの方を見られないぐらい緊張していた。 
\\	2て、
\\	1くらいだ。 
\\	くらい 
\\	ぐらい. せめた
\\	は~ 
\\	せめては
\\	くらい/ぐらい 
\\	せめて 
\\	くらい, 
\\	せめて一回日本へ行ってみたいなあ。 
\\	は
\\	くらい(だ。) 
\\	くらい。 
\\	だ 
\\	かもしれない, 
\\	ぐらい
\\	てください 
\\	ぐらい
\\	ください 
\\	させる 
\\	くらい
\\	くらい
\\	くらい
\\	くらい? 
\\	ぐらい. 
\\	くらい
\\	お腹がすいたと言ってばかりいないで、早く食べたいならサラダの用意ぐらい手伝ってよ。 
\\	サラダくらい...) させる/
\\	疲れているので、今日ぐらいゆっくり寝させてください。 
\\	少しぐらい休ませてくださいよ。 
\\	開けっぱなしにしないで、ドアぐらい閉めてください。 
\\	帰る前に電話ぐらいしてよ。 故障で今シャワーが使えませんが、2,3時間ぐらい我慢してください。 
\\	部屋を使ったあとは、電気ぐらい消してください。 
\\	ーねえ。このケーキ、作ったんだ。食べてみて。 ーええ?なんか形が崩れてて、あまり・・・。 ー見た目ぐらい我慢してよ。味は本当においしいんだから。 ーうん、じゃ、一口。 ~ほど~は
\\	ない 
\\	に限る 
\\	ことほど
\\	ことはない 
\\	~はない 
\\	は 
\\	ない. 
\\	両親ほど私のことを心配してくれる人は他にいません。 (ほかに) 
\\	自分の国の料理ほど食べてほっとする料理はありませんから。 
\\	新宿ほど一晩中にぎやかで明るい町はないと聞きましたが、本当ですか? 
\\	疲れると、友達に電話をかけたくなります。仲のいいともだちとのおしゃべりほどリラックスできることはありませんから。 
\\	以前は
\\	町(ちょう)ほどにぎやかな所はなかったが、今は人口が少なくなってさびしくなってしまった。 
\\	自転車ほど環境(かんきょう)にいい乗物はないよ。空気が汚れないから。 それほど~(ない) 
\\	作るのはそれほど難しくないし、家でも家族と作って見ます。 
\\	くらい	
\\	ぐらい. せめた
\\	は~ 
\\	せめては
\\	くらい/ぐらい 
\\	せめて 
\\	くらい, 
\\	せめて一回日本へ行ってみたいなあ。 少なくとも 
\\	じゃ、新卒(者)で入ったとしても、少なくとも今年で30歳なるということですよね。 一応  いちおう 
\\	ー行きたいけど、この仕事、どうしても今日中にやらないといけないから・・・。 ーそうなんだ。でも、もし早く終ったら一応電話くれる? そうですね。一応私から教頭(きょうとう)には伝えておくので、出勤してから教頭に確認してみてくれますか? 
\\	は
\\	くらい(だ。) 
\\	くらい。 
\\	だ 
\\	かもしれない, 
\\	1くらい
\\	ぐらい 
\\	な
\\	なくらい 
\\	くらいの
\\	ぐらい. 
\\	彼女となら結婚してもいいと思えるぐらいすてきな女性に出会った。 
\\	ちょっと失敗したぐらい大丈夫ですよ。そんなに気にしないでください。 
\\	まだ3月だというのに、今日の天気は暑いくらいだ。 
\\	会議に来て行くスーツは、ちょっと地味なくらいがちょっといいよ。 
\\	少しいたずらをしても、元気すぎるぐらいの子供が好きだ。 
\\	リンリンさんは指を痛めるぐらい何度もピアノの練習をした。 リンリンさんは本番の時、お客さんの方を見られないぐらい緊張していた。 
\\	2て、
\\	1くらいだ。 
\\	1ほど
\\	程 
\\	ほどではない. でも、心配するほどではなく、数日で調子を取り戻すことができた。 動けないほどお腹がいっぱいだ。 
\\	東京はびっくりするほど人が多い。 
\\	2て
\\	1ほどです。 
\\	~ば~ほど 
\\	/な
\\	であればあるほど 程 
\\	小さければ小さいほど毒が強い。  勉強しなければ勉強しないほど~ 
\\	ぐらい
\\	てください 
\\	ぐらい
\\	ください 
\\	させる 
\\	くらい
\\	くらい
\\	くらい
\\	くらい? 
\\	ぐらい. 
\\	くらい
\\	お腹がすいたと言ってばかりいないで、早く食べたいならサラダの用意ぐらい手伝ってよ。 
\\	サラダくらい...) させる/
\\	疲れているので、今日ぐらいゆっくり寝させてください。 
\\	少しぐらい休ませてくださいよ。 
\\	開けっぱなしにしないで、ドアぐらい閉めてください。 
\\	帰る前に電話ぐらいしてよ。 故障で今シャワーが使えませんが、2,3時間ぐらい我慢してください。 
\\	部屋を使ったあとは、電気ぐらい消してください。 
\\	ーねえ。このケーキ、作ったんだ。食べてみて。 ーええ?なんか形が崩れてて、あまり・・・。 ー見た目ぐらい我慢してよ。味は本当においしいんだから。 ーうん、じゃ、一口。 ~ほど~は
\\	ない 
\\	に限る 
\\	ことほど
\\	ことはない 
\\	~はない 
\\	は 
\\	ない. 
\\	両親ほど私のことを心配してくれる人は他にいません。 (ほかに) 
\\	自分の国の料理ほど食べてほっとする料理はありませんから。 
\\	新宿ほど一晩中にぎやかで明るい町はないと聞きましたが、本当ですか? 
\\	疲れると、友達に電話をかけたくなります。仲のいいともだちとのおしゃべりほどリラックスできることはありませんから。 
\\	以前は
\\	町(ちょう)ほどにぎやかな所はなかったが、今は人口が少なくなってさびしくなってしまった。 
\\	自転車ほど環境(かんきょう)にいい乗物はないよ。空気が汚れないから。 それほど~(ない) 
\\	作るのはそれほど難しくないし、家でも家族と作って見ます。 
\\	何よりも	"なんよりも 
\\	何よりも
\\	=欲しい, 一番 それは何よりもです。 
\\	どうしても
\\	たい。 
\\	何より なにより 
\\	それは何よりです。 
\\	ものですから	
\\	もんだから 来月のサッカーの試合で、県の運動場を使用できないものでしょうか?近くにいい運動場がないものですから。 
\\	~と思っていたものですから 
\\	(から) 
\\	もの
\\	です) 
\\	感じていた 
\\	考えていた. 
\\	今晩お時間があったらお食事でもいかがですか? 私、山本さんと一度ゆっくり話してみたいと思っていたものですから。 
\\	から 
\\	今やっている仕事が終ってから準備しようと考えていたものですから、まだ全く準備していないのですが・・・。 
\\	昨日から何だか体調が変だなあと感じていたものですから、もしかしてと思ったものですが。 
\\	~うちに	
\\	のうちに, な
\\	なうちに, 
\\	(分かる, ある, 
\\	お餅が柔らかいうちに食べましょう! 
\\	しっかりとデータを集めて、自分でまとめた資料を作らないと、あとで上司に書き直しさせられかねないから、今のうちからがんばろう。 
\\	うちに 
\\	さあ、ピザが冷めないうちにいただきましょう。 
\\	~のうち 
\\	内 
\\	日本は、同僚やお客さんと飲みに行くのも仕事のうちなのよ。 うち, ない 
\\	(内部 ないぶ) 
\\	(うちに) 
\\	(宅 たく) 
\\	(時) 
\\	のうちに) 
\\	(中[で]) 
\\	(二つの), 
\\	(三つ以上の), 
\\	~にいる(いない) 
\\	時 サラダをのうちの食べながら持っているうちに、魚のメインコースが出てきた。 すごくよかった!観光地(かんこうち)でも旅館でもいろいろおもしろいことがあって・・・。ああ、休み時間内(ない)には全部話しきれないなあ。 そのうちの 
\\	その中の いない 
\\	五週か以内に  ごしゅうかんいないに 
\\	この店では、1時間以内ならいくら食べても1,000円です。 
\\	つい	"(ほんの) 
\\	(うっかり) 
\\	~今したが 
\\	ついに, 
\\	つい 
\\	てしまう 電車の中に珍しい服装をした人がいたので、ついじっくり見てしまって、反対にその人に変な目で見られた。 
\\	さっきはちょっといらいらしていて、つい怒ってしまった。申し訳なかった。 
\\	彼女の誕生日をつい忘れてしまったために、彼女に怒られた。 
\\	やらなければならないことをついそのままにしてしまって、あとでやっておけばよかったと思うことがある。 
\\	私がつい注文し忘れてしまって・・・。 部屋の掃除をずっとしようと思っていたのに、つい後回しにしてしまって、部屋がとても汚くなってしまった。 最後に主人公が家族と無事に再会したシーンでは、つい泣いてしまって、ハンカチが必要だった。 本当に申し訳ございません。ついうっかりしてしまいまして・・・。 偶然  ぐうぜん 
\\	~の(に) 
\\	~のできごと(一致 いっち) 
\\	ばったり 
\\	ばったり出会う
\\	たものだ。	
\\	昔は~, 
\\	の頃は~. 
\\	ものです。 もちろん相手が納得するまで説明したが、しまいには、こちらから先手を打って、理由を言ったものだった。 学生の頃、友達に勝ちたくて一生懸命練習したものですよ。 
\\	としたら	
\\	とする=
\\	たら=
\\	丁度~しようとしている  (ちょうど) 
\\	食べようとしている。 
\\	とする 
\\	男性が自転車を駐輪しようとした時に、他の自転車に当たってしまった。 (あたって)   
\\	とおもう 
\\	直そうと思った。 
\\	せい	
\\	だから, 
\\	だから, 理由. 
\\	せい
\\	のせい, な
\\	なせい 学生ができないのは先生のせいだろうか。 
\\	自分の間違いを他人(たにん)のせいにするのはよくない。 
\\	それは気のせいだよ。 
\\	1せいか、
\\	おかげで. 
\\	期末試験が近いせいか、休みが多い。 
\\	おかげで 
\\	だから, 
\\	お蔭様で・おかげさまで 
\\	は
\\	おかげだ(から)。 
\\	な
\\	なおかげ, 
\\	のおかげ 
\\	な
\\	私は山登りが好きで、今までいろいろな山に登ってきました。山登りができるのは体が丈夫なおかげですから、これからも病気や怪我をしないように気をつけたいと思います。 
\\	は
\\	おかげだから 
\\	私が毎日楽しく仕事ができるのは周りの先生方が明るいおかげだからだと思います。 
\\	は
\\	おかげだ 
\\	ー阿部選手(あべ)はおにぎりが大好きで、外国の試合に行っても必ずおにぎりを食べるそうです。オリンピックの時も試合の前に食べそうですよ。 ー阿部選手がオリンピックで優勝できたのはおにぎりのおかげだったんですね。 
\\	て御覧なさい	
\\	てごらんなさい 
\\	て見なさい・てみなさい 
\\	なさい 
\\	てごらん。 
\\	思うように	
\\	なかなか思うように
\\	(なかなか 
\\	思ったより~ 
\\	ーあれ、逃げられた!なかなか紙の上に乗ってくれません。 ーあ、いけない!紙に穴が空いちゃった。思ったより難しいですね。 
\\	このゲームは思ったより難しいから、一回でクリアというわけにはいかないんじゃないかなと思うよ。 
\\	期待したほど  きたい 
\\	昨日見に行った映画は、期待したほどおもしろくなかった。お金を払って見るにはあたらない映画だった。 
\\	お金を払って見る
\\	見に払う 
\\	ように思う。 
\\	ーちょっと味が薄いように思うんですが。 ーあ、本当ですね。水を多く入れすぎたかな。 多く入れすぎる=
\\	毎日のように 
\\	週, 回, 年, 時間
\\	ように言う 
\\	てください(と言う) 
\\	お願いする
\\	頼む たのむ 
\\	注意する 
\\	言う. 
\\	言われた 
\\	ようになる 
\\	い
\\	=くなる, な
\\	=になる, 
\\	ようになる, 
\\	になる 
\\	ように感じます 
\\	なくなる, な
\\	じゃなくなる 
\\	静かじゃなくなる, 
\\	静かにならなくなる) 
\\	ってくる 
\\	だんだん 
\\	最近、インターネットでニュースを読むようになって、だんだん新聞を読まなくなってきています。 
\\	生徒の顔色を見ていると、だんだん心配になってきて、学校へ連絡したあと、すぐ病院へ連れて行くことにしました。 
\\	この町も人がたくさん住むようになって、だんだん静かじゃなくなってきました。 
\\	な
\\	ー昼は時間もないし、手軽に食べられるから、おにぎりはいいですよね。 ーええ。でも、私は、昼だけじゃないんです。朝も、夜もなんです。 ーええ?種類が豊富でも、それはだんだん嫌(いや)になってきませんか? ー嫌になりませんが、だんだん飽きてきましたね。(あきる) 手術が近づくと、生徒はだんだん落ち着かなくなってきましたが、ジムさんの励ましで、少し安心したようでした。 ようにする 
\\	気をつけるようにします。 
\\	10時に間に合うようにするには、今からだと、タクシーに乗らない限り無理です。
\\	出入り口の近くに荷物をたくさん置いておくと、家事や地震で逃げようと思ってもドアが開かない恐れがあるので、荷物は置かないようにしてください。 
\\	午後の授業に遅れないようにしてくださいね。 
\\	ようにしている 
\\	ように
\\	(分かる, いる, ある). 
\\	ために 
\\	食べ物をもらえるようにお願い事をしましょう。 
\\	忙しい時はあいさつをしなかったり、返事をしなかったりすることがあるが、失礼な人だと思われかねない。いくら忙しくても、どんな時でも、気がついたら、あいさつや返事をするように気をつけたい。 
\\	ように 私は山登りが好きで、今までいろいろな山に登ってきました。山登りができるのは体が丈夫なおかげですから、これからも病気や怪我をしないように気をつけたいと思います。 
\\	ように
\\	ように言う。 そのような~ 
\\	そんな
\\	相談センターに聞いたところ、そのような複雑な内容は、短い時間ではお話しかねるとの事でした。 
\\	どんな人からみてもいいと思えるような、もっとかっこいいポスターを作ってみせますよ! 
\\	たら/ったら	
\\	は 
\\	は~. 
\\	うちの母ったら、文句ばかり言うのよ。 
\\	あの人ったら、こんなこと言うのよ。 
\\	~に違いない	
\\	は
\\	に違いない=: 
\\	違いません・ちがいません. 
\\	あそこでお辞儀(おじぎ)をしているのは日本人に違いない。 
\\	私が「渋谷マンション」に住んでいるもんだから、その
\\	""という主所を見て、
\\	に住んでいるなら、すごい暮らしをしているに違いないと考えたらしいんです。 
\\	さすが(に)	
\\	に 
\\	だけあって (だから) 
\\	1だけあって、さすが
\\	さすが
\\	さすが
\\	な
\\	日本に十年も住んでいただけあって、さすが日本語が上手だ。 
\\	お習字(しゅうじ)の先生だけあって、さすがにすばらしい字だ。 
\\	さすが
\\	さすが言葉を習うのが得意な川口さんだ。彼は外国旅行に行くと、すぐにその国の言葉を覚えてしまう。 
\\	さすがコンピューターエンジニアですね。パソコンの故障はすぐに直せるんですね。 
\\	は 
\\	を 
\\	さすが質がいい靴ですね。歩きやすくて足が疲れません。 
\\	短い時間で、こんなにたくさんの料理を作れるんですねか。しかもすごくおいしい。彼女の料理の実力はさすがですね。 
\\	なんか 
\\	なんて	"=とか 
\\	とか なんか 
\\	なんて 
\\	などは, などと, などというのは 私は周りが静かで/じゃ ない限り集中できないから、音楽を聞きながら本を読むなんてできません。 
\\	なんか 
\\	会場はかなり広くて屋台もたくさん出ているんですが、中(なか)には毎年欠かさず行っているお店なんかもあるんだよ。 
\\	なんて 
\\	サンタクロースなんていないと言われた。 
\\	サンタクロース 
\\	などの
\\	台風などのひどい天気で/じゃ ない限り、毎日自転車で通勤しています。 
\\	でも 
\\	明日はどうせ暇だし、映画にでも行こうかな。 
\\	どうせ今日は大雨で野球は中止になると思うし、その時は家で本でも読もう。 
\\	や
\\	など   妻は猫が大好きなので、家の中は猫の写真や猫の人形など、猫に関係したものだらけなんです。 
\\	一方(、)(は)	"いっぽう 
\\	一方では~、他方では~。  (たほう) 
\\	~が, 
\\	も. 
\\	一方では他の人とシャワーやトイレを共有するのが嫌だと言う人もいますが、他方ではこういうホテルで旅の情報を交換したいと言う人もいます。 
\\	も~も 
\\	が 
\\	この映画について一方ではすばらしいと言う人もいるが、他方ではつまらないと言う人もいる。 
\\	一方では生徒のいいところを褒め(ほめ)、他方では間違ったことをした時に寂しく(さびしく)注意するのがいい教師だ。 
\\	も~も 
\\	世界には一方では物が豊富な国もあり、他方では食べる物もない国もある。 
\\	経済発展は一方では国を豊かにしますが、他方では森林伐採や大気汚染など地球環境を悪くする(という)心配もあります。 彼については一方では頼りになると言う人も多い(です)けれど、わがままだと感じている人もいますから、リーダーにするかどうか、もう少し様子を見たほうがいいと思いますよ。
\\	せいぜい	
\\	せめて~ 
\\	せいぜい
\\	一日, 
\\	私は金がなく、昼の中華料理屋で、三十セントと三十五セントの定食のうち、せいぜい月に一度か二度、高いのにありつければ良いほうだった。 (~に)ありつき  
\\	日本のサラリーマンは、休みを取ったとしても、せいぜい四、五日でしょう。 
\\	大きなクラスでは、一時間にせいぜい一、二度当たればいい方だ。 
\\	(当たれば あたれば) 
\\	せめた
\\	は~ 
\\	せめては
\\	くらい/ぐらい 
\\	せめて 
\\	くらい, 
\\	せめて一回日本へ行ってみたいなあ。 
\\	くらい 
\\	ぐらい. 
\\	ぐらい
\\	てください 
\\	ぐらい
\\	ください 
\\	させる 
\\	くらい
\\	くらい
\\	くらい
\\	くらい? 
\\	ぐらい. 
\\	くらい
\\	お腹がすいたと言ってばかりいないで、早く食べたいならサラダの用意ぐらい手伝ってよ。 
\\	サラダくらい...) させる/
\\	疲れているので、今日ぐらいゆっくり寝させてください。 
\\	少しぐらい休ませてくださいよ。 
\\	開けっぱなしにしないで、ドアぐらい閉めてください。 
\\	帰る前に電話ぐらいしてよ。 故障で今シャワーが使えませんが、2,3時間ぐらい我慢してください。 
\\	部屋を使ったあとは、電気ぐらい消してください。 
\\	ーねえ。このケーキ、作ったんだ。食べてみて。 ーええ?なんか形が崩れてて、あまり・・・。 ー見た目ぐらい我慢してよ。味は本当においしいんだから。 ーうん、じゃ、一口。
\\	よっぽど	
\\	とても 
\\	よほど.
\\	~ものの	
\\	である(ない)ものの, な
\\	なものの ~ながら 
\\	1ながら
\\	ある, 思う, 住む, 分かる, 飲む
\\	であり 
\\	残念ながら 
\\	くせに 
\\	のくせに, な
\\	なくせに 
\\	隣の部屋の人は毎晩大きい音で音楽を聞くくせに、私には少しのことで「うるさい!」と言っている怒る。 
\\	最近若いくせにすぐに「疲れた」と言う人が多い。 
\\	友達は練習にはあまり熱心じゃないくせ試合には出たがって、困る。 
\\	たがっている 
\\	て 
\\	わり(に) 
\\	のわり, な
\\	なわり 
\\	た1
\\	わりに(は), は 
\\	からと言って 
\\	のに 
\\	わり 
\\	が、そのわりに、~。 
\\	テストは心配したわりに(は)よくできた。 
\\	テスト, 
\\	私 
\\	私は昨日、丈夫なわりに(は)値段が安い傘を買った。 
\\	この店のラーメンはとてもおいしいが、そのわりに(は)あまり人に知られていない。 
\\	冬のわりに(は)暖かい日が続いている。 
\\	続く) 
\\	と言っても
\\	だ 
\\	返って	"かえって 
\\	1、返って
\\	~は、返って~。 
\\	たらかえって 
\\	たら 
\\	かえって. 
\\	たら/かえって 
\\	体を急に動かしたらかえって体によくないかもしれませんが、横になってばかりいるのもよくないですよ。 
\\	怒られると思いながらはっきり自分の意見を言ったら、かえって先生に褒められた。 
\\	お金を使う  
\\	バーゲンで安いからと思ってあまり考えずに買っていたら、かえってお金を使いすぎてしまった。 
\\	新鮮な野菜を食べたくて、自分で野菜を育てて見たら、できた野菜が自分の子供のようにかわいくなって、かえって食べられない。 
\\	野菜を育てる いろいろと考えずに、思い切ってやったらかえってうまくいきました。 引越しの手伝いに行ったらかえって邪魔になってしまいました。 ても 
\\	い
\\	=くても 
\\	/な
\\	=でも 
\\	であっても 
\\	だって?? 
\\	卒業式の日にでもみんなの前で歌おと思って準備していたんですよ。 ても 
\\	明日はどうせ暇だし、映画にでも行こうかな。 
\\	どうせ今日は大雨で野球は中止になると思うし、その時は家で本でも読もう。 
\\	ても・でも~たら~ 
\\	ても 
\\	たら 
\\	ても. もし上司に飲みに誘われても、都合が悪かったら無理に付き合わなくても大丈夫ですよ。 
\\	今は平気でも、夜になったらまた熱が出るかもしれないからもう少し休んでください。 
\\	と言っても
\\	だ 
\\	~からと言って(も、) 
\\	わり 
\\	そろそろ寝たほうがいいんじゃないですか?冬休みだからと言って、いつまでも寝ないでいると朝起きられなくなりますよ。 
\\	仕事が忙しいからと言って、いつも彼女を手伝わないわけじゃないんです。 
\\	とは限らない 
\\	からって 
\\	からと言って 
\\	(だ)とは, だ 
\\	な
\\	だとは, だ 
\\	家の電話にかけても出ないからと言って、(必ずしも)家に誰もいないとは限りませんよ。 
\\	顔色が悪いからと言って、(必ずしも)病気(だ)とは限らないよ。何かショックなことがあったのかもしれないよ。 
\\	そうかと言って 
\\	も! 
\\	でも, けれども. 
\\	わけにはいかない, わけではない, 
\\	それにしても 
\\	1、それにしても
\\	1。それにしても
\\	2。 どうしても 
\\	ここへ来るのが嫌なわけではないんですが、その日はどうしても来られないんです。 
\\	何と言っても 
\\	何と言っても
\\	ても 
\\	どこへ行っても, 何を食べても いくら: 
\\	どんなに: 
\\	いくら 
\\	この店では、1時間以内ならいくら食べても1,000円です。 
\\	それに、いくら家に必要な物を準備していても、外出中だったら持って行けないのではないかと思いますが・・・。 彼がいくら背が高くても、あの壁の時計には届かないだろう。 
\\	どんな 
\\	どんなに素敵でも(すてき)このかばんは買えません。高すぎます。 
\\	何を頼んでもいいんですか? 
\\	ナオミさんは何をしても上手にできる。 
\\	ーこのお店お人は感じがいいね。 ーそうだね。店員さんはみんな親切だから、わからないことは誰に聞いても丁寧に(ていねい)教えてくれるよ。 ーへえ、そうなんだ。これからはこのお店で買うことにしよう。 彼は家にいるのが好きなので、1週間どこへも行かなくても平気だ。 
\\	ーアランさんは何を食べてもおいしいと言いますね。好き嫌いはないんですか? ーないですよ。日本料理は特に何でも好きです。 どんなに悲しいことや落ち込むことが当てえも、この曲を聞けば元気になれる。 日本は地震の多い国だから、地震がいつ来ても慌てることのないように、日ごろから必要な物を準備しておく必要があります。 何歳になっても知らないことはたくさんある。世界は知らないことだらけだ。 
\\	て 
\\	どんな時も自分より相手を大切に思う気持ちこそすばらしいというものだ。 
\\	忙しい時はあいさつをしなかったり、返事をしなかったりすることがあるが、失礼な人だと思われかねない。いくら忙しくても、どんな時でも、気がついたら、あいさつや返事をするように気をつけたい。 
\\	どんな仕事でも楽だなんてことはありません。(らくだ) 
\\	さえ 
\\	こと. 
\\	も. 
\\	さえも ~て、~さえだ/
\\	さえ
\\	ば 
\\	さえすれば 
\\	てさえいれば 
\\	が, を, は, も 
\\	に, で, 
\\	にさえ 
\\	さえ 
\\	さえ, 
\\	とは言え 
\\	先生、お待たせしてしまって申し訳ありません。急用とは言え、あやまるよりほかありません。 
\\	~にしろ~にしろ~ 
\\	~にせよ~にせよ 
\\	にしても 
\\	にしろ
\\	にしろ 
\\	にしろ
\\	にしろ. 
\\	にしろ 
\\	どちにしろ 
\\	にしても, 
\\	レストランに行くにしても行かないしても、連絡してください。 
\\	どちにしろ 
\\	男の子か女の子が、どちらにしろ早く生まれてきてほしい。 
\\	森さんは怒るにしろ泣くにしろ、よく騒ぐ人だ。 
\\	おしゃべりが大好きな弟のことは、うるさいにしろおとなしいにしろ、気になってしまう。 
\\	上手にしろ下手にしろ、精一杯やることが何でも大切です。 
\\	手紙にしろメールにしろ、とにかく返事をください。 
\\	には
\\	の~がある	
\\	親には親の生活がある。 
\\	子供には子供の考えがあるんだから、まず聞いてやることが必要だ。 
\\	1は
\\	1なりの
\\	には, 
\\	アメリカにはアメリカなりのよさがあり、日本には日本なりのよさがあるから、どちらがいいとも言えない。 
\\	1は
\\	1なりに
\\	私, 妻 ーもうちょっと勉強したら? ーこれでも、私は私なりに頑張っているつもりですけど。 
\\	それに日本じゃ、妻は妻なりに生活を楽しんでるのよ。
\\	~という理由で	
\\	から 
\\	という 
\\	っていう理由で
\\	とともに~	
\\	も 
\\	は
\\	とともに
\\	とともに~ 
\\	とともに~ 
\\	たびに
\\	と同様 どうよう 
\\	(=と同じように) 原題
\\	が表す通り、彼は「ある国のカッコよさ、つまり大衆に考える文化的影響力は、GNPと同様、その国のパワーとして認知されるべきではないか」と説き、日本は既にその分野の超王国となっていると指摘する。 
\\	とともに. 
\\	は
\\	と同様、
\\	死んだも同様  しんだもどうよう 
\\	につれて 
\\	何だか雪降るにつれて室内の温度も下がってきた気がします。 
\\	経済の発展につれて生活が豊かになった。 
\\	少し待てば雨は止むだろうと思って待っていたが、時間が経つにつれてだんだん強くなってきた。 
\\	病気がよくなるにつれて食欲が出てきて、よかったです。 
\\	ー新聞によると人口増加につれて食料不足(ぶそく)や水不足(みずぶそく)、ごみ問題などが生じるそうですよ。 ーそうなんですか。今から対策を考えておかなくてはいけませんね。 ーその本、おもしろい? ーうん。科学が発達するにつれて・科学の発達につれて私達の生活がどんなふうに変わってきたかよくわかるよ。
\\	たびに
\\	とともに~ 
\\	も 
\\	は
\\	とともに
\\	とともに~ 
\\	とともに~ 
\\	と同様 どうよう 
\\	(=と同じように) 原題
\\	が表す通り、彼は「ある国のカッコよさ、つまり大衆に考える文化的影響力は、GNPと同様、その国のパワーとして認知されるべきではないか」と説き、日本は既にその分野の超王国となっていると指摘する。 
\\	とともに. 
\\	は
\\	と同様、
\\	死んだも同様  しんだもどうよう 
\\	につれて 
\\	何だか雪降るにつれて室内の温度も下がってきた気がします。 
\\	経済の発展につれて生活が豊かになった。 
\\	少し待てば雨は止むだろうと思って待っていたが、時間が経つにつれてだんだん強くなってきた。 
\\	病気がよくなるにつれて食欲が出てきて、よかったです。 
\\	ー新聞によると人口増加につれて食料不足(ぶそく)や水不足(みずぶそく)、ごみ問題などが生じるそうですよ。 ーそうなんですか。今から対策を考えておかなくてはいけませんね。 ーその本、おもしろい? ーうん。科学が発達するにつれて・科学の発達につれて私達の生活がどんなふうに変わってきたかよくわかるよ。
\\	は(を)
\\	に任せる	
\\	まかせる る
\\	に 
\\	は 
\\	てくれたらと思います	
\\	てくれたらいい(のに)と思う。 
\\	(させる) 
\\	っていうか
\\	ていうか というか 
\\	昨日パイを作った、ていうか作るのを手伝っただけなんだけど。 
\\	兄弟ていうか、双子ですけど。 (ふたご) 
\\	どちらかと言うとV	
\\	どちかと言うと 
\\	ら) 
\\	どちかって言うと 
\\	と 
\\	のが
\\	のほうが
\\	ほうが
\\	の) 
\\	よりも/
\\	よりも 
\\	も) 
\\	犬か猫であれば、どちらかと言うと、猫のほうが好きだ。 
\\	どちら, 
\\	どちら, 
\\	ー私にもできることがあったら何でも言って下さいね。 ーありがとう。でもどちらかと言うと今よりも当日手が足りなくて助けてもらうことが多そうです。 
\\	企画を主催する側(がわ)がいいか客として参加する側(がわ)がいいか、どちらがいいかと聞かれれば、私はどちらかと言うと主催する側がいい。 
\\	聞かれる=
\\	ー安井(やすい)先生はあまり怒らなさそうですね。 ーいいえ。私はどちらかと言うと怒りっぽいほうですよ。 ーえ?そんなふうに見えませんよ。 (怒りがち 
\\	ほう, 
\\	怒りがち 
\\	とか(言う)	
\\	と 
\\	と言う(聞く, おしゃる, 話す) 
\\	んですけど, 
\\	けど 
\\	が
\\	は
\\	とか(聞・言・話)んですが、
\\	小川先生が、ナガイ先生は日本の歴史にとても詳しいとかおっしゃっていたんですけど、どこで勉強なさったんですか? 
\\	誰かから、金澤さん(かなざわ)は若い頃からダンスを習っているとか聞いたんですが、どんなダンスがお好きなんですか? 
\\	単に~ではない	
\\	日本は単に外国文化を模倣(もほう)しているのではない。 
\\	だらけ	
\\	どろ 
\\	間違い 
\\	穴 あな 
\\	きまり 
\\	きまりだらけ日本) 
\\	だらけある. 
\\	だらけになる, だらけで~, だらけのところの
\\	ある 
\\	妻は猫が大好きなので、家の中は猫の写真や猫の人形など、猫に関係したものだらけなんです。 
\\	ここは以前は石だらけのところで、とても野菜を作ることはできませんでした。 
\\	何歳になっても知らないことはたくさんある。世界は知らないことだらけだ。 
\\	昨日野球の試合は予想していなかったことだらけで、とてもおもしろかった。 
\\	今はビルだらけなんですよ。私が子供の頃は緑や古い町並みがあって(まち、なみ)、とてもいい所だったんですけど、すっかり変わってしまいました。 ー日本の生活はどうですか? ーもうすっかり慣れました。来た時はわからないことだらけで戸惑いましたが、今はだいぶ詳しくなりましたよ。
\\	の余地がない	
\\	のよちがない 
\\	(妥協 だきょう), 
\\	(議論 ぎろん), 
\\	妥協の余地がない。 
\\	相手が話しを聞いてくれない人だと、議論の余地がない。 
\\	余地 よち 
\\	(余白 よはく), 
\\	~がある 
\\	もつ	"もつ・もって・もった
\\	・もてる
\\	つ)・もたない・もたず・もとう かつ・かって・かった・かてる・かとう 
\\	かってる・かっている 
\\	なさい 
\\	ないで 
\\	る
\\	=/お/ う
\\	=/え/ 
\\	な 見る・見なさい・見ないで・見ろ・見るな  食べる・食べなさい・食べないで・食べろ・食べるな 買う・買いなさい・買わないで・買え・買うな たつ・たちなさい・たたないで・たて・たつな かえる・かえりなさい・かえらないで・かえれ・かえるな 書く・書きなさい・書かないで・書け・書くな 読む・読みなさい・読まないで・読め・読むな 呼ぶ・呼びなさい・呼ばないで・呼べ・呼ぶな 死ぬ・死になさい・死にないで・死ね・死ぬな
\\	と言えば、~	
\\	1つと言えば~ 
\\	大ざっぱに言えば~ おおざっぱ, 
\\	大雑把 
\\	計算に言えば~  けいさん 
\\	正確に言えば  せいかくにいえば 
\\	ついで(に) 
\\	つい 
\\	ところで 
\\	ちなみに 
\\	たって	
\\	ても 
\\	行かなくたっていい=行かなくてもいい
\\	体験する	たいけんする 
\\	経験する けいけん 
\\	一般のに	いっぱんのに 
\\	大体 だいたい 
\\	普段 ふだん 
\\	いつもより
\\	明日は朝8時からミーテイングがあるからいつもより早く来てほしい。 いつもと違う 
\\	ここへ旅行に来た人は海のスポーツを楽しむかたわら、いつもとは違う時間をゆっくり過ごしている。 
\\	~ている最中	
\\	さいちゅう 
\\	ことにしよう そんなに気にしないことにしましょう。 
\\	ないでおこう・ないでおきましょう 
\\	(置く) 食べないでおこう。 
\\	-あいで
\\	んとこう たべんとこう (食べる), いかんとこう (行く), かかんとこう (書く)
\\	こんにちは・こんばんわ	
\\	くないです・じゃないです	
\\	じゃありません 
\\	ないです 
\\	いいじゃないですか。 
\\	だって、お雑煮(ぞうに)もお汁粉(しるこ)も両方食べたいじゃないですか! 
\\	失礼します	しつれいします 
\\	さようなら 
\\	どうそよろしく	
\\	お願いします. 
\\	よろしくお願いします 
\\	ホストファミリーの鈴木さん	
\\	鈴木さんのホストファミリー 
\\	どちらからいらっしゃいましたか。	
\\	から来ました。 どこの~ 
\\	ああ、車がほしいなあ。 買うなら、どこの車がいいかなあ。 
\\	どこ, どちら 
\\	はどこですか。
\\	その	
\\	あの 
\\	横	よこ 
\\	隣・となり 
\\	(どうぞ)上がってください。	
\\	お邪魔します。  おじゃまします 
\\	が
\\	は~。	
\\	は 
\\	が 
\\	です。	
\\	にあります 
\\	あのう失礼ですが、~	
\\	きゅうじゅう 
\\	ななじゅう、しちじゅう
\\	とてもよく
\\	はとてもよく勉強する。 
\\	お願いします	
\\	あのうちょっとすみませんが/失礼ですが~ 
\\	六ヶ月	ろっかげつ=半年 はんとし
\\	好きな
\\	~んです	
\\	んです= 
\\	(から 
\\	百万 ひゃくまん 
\\	万 
\\	十万 
\\	三十五万 十億  じゅうおく 
\\	一億  いちおく 
\\	とう	
\\	とうつ 
\\	(つ
\\	地下~	ちか~ 
\\	昨日/先日はどうもご馳走様でした。	きのう・せんじつはどうもごちそうさまでした。 
\\	ええ、ぜひ。	
\\	二日	ふつか 
\\	みっか, 
\\	十日  とおか ~年 
\\	~日 
\\	三日 
\\	三日間)
\\	たり
\\	たりする	
\\	や
\\	た 
\\	昔の橋は木でつくられていて、流されたり焼けたりして橋が何度も落ちたからだと言われていますよ。 
\\	忙しい時はあいさつをしなかったり、返事をしなかったりすることがあるが、失礼な人だと思われかねない。いくら忙しくても、どんな時でも、気がついたら、あいさつや返事をするように気をつけたい。 
\\	クリスマスには、家族が全員集まって、ご飯を食べたりします。 私は旅行が好きで、時刻表を持って普通電車で一人でのんびり出かけたりもするんですよ。 でも、彼女はとても上手に読んだり書いたりできますよ。 
\\	読んで書く 
\\	上手に 
\\	よく 
\\	(よくわかる) ー最近、私、忘れっぽいんですよ。 ー私もです。昨日買った本を読みかけてから、まだ最後まで読んでいない本があったことを思い出したりして・・・。 ーそう、そう。私もありますよ。 
\\	休みの日は例えば近くの公園を散歩するとか映画を見るとかして楽しいんでいます。 
\\	とかして楽しい>
\\	とかするのが好き
\\	の・が	の 
\\	が 
\\	る	られる 
\\	れる 
\\	ら 
\\	食べれる
\\	て
\\	~たがっている・います	
\\	~たい. たがっている 
\\	て 
\\	友達は練習にはあまり熱心じゃないくせ試合には出たがって、困る。 
\\	どうしたらいいですか/でしょうか。	
\\	どうかしたんですか。 
\\	どうしたんですか。 
\\	はい、いいです。 
\\	はい、どうぞ/かまいません/もちろん.
\\	行っている・来ている	
\\	結婚している, 
\\	そうだ 
\\	高くなそう/高そうじゃない 元気じゃなそう/げんきそうじゃない ふりそうに(も)ない 
\\	ふる)
\\	~でしょうか 
\\	~だろうか 
\\	ます
\\	でしょう 
\\	もしもし、
\\	と申しますが、
\\	のご担当者様はいらっしゃいますでしょうか? 
\\	でしょうか 
\\	みんなどのような相談をしたでしょうか?
\\	来よう	こよう 
\\	来る. 
\\	いただく 
\\	最後に皆さんからの質問に答えていただこうと思います。 
\\	ので
\\	から. ので 
\\	そのつもり。	
\\	そのつもりだ。 
\\	つもり 
\\	~つもりはありません/ない。 
\\	予定はありません/ない。 (よてい)
\\	何て	
\\	なんて 
\\	この漢字、何て読むの?
\\	前に/は	
\\	後	
\\	ご(に) 
\\	のあと(で) 
\\	たあと(で) 
\\	2が~後で
\\	1は~。 
\\	が 
\\	て上げましょうか。	あげましょうか・あげようか 
\\	~ましょうか. 
\\	てはいけない/だめ	
\\	/な
\\	=じゃ, い/
\\	=ちゃ 
\\	てはいけない? 
\\	てはだめ 
\\	(な/
\\	だ)し、
\\	し、) ~。	
\\	て
\\	し、
\\	から
\\	も
\\	2し、
\\	も 
\\	~なら	
\\	のなら/
\\	んなら 
\\	なら= 
\\	たら= 
\\	(たら) 
\\	と: となら ー私の友達、愛がなくてもお金がある人となら結婚すると言うんですよ。 ーええ!私はお金のためだけに結婚できませんよ!
\\	ため~。	
\\	から) 
\\	ための 
\\	て ハードディスクを買って送料が1万円だなんてことはありえないよ。 
\\	で
\\	円	2枚で3,000円にしてもらいます 
\\	は五つで百円。 
\\	で 
\\	この1枚1,600円のTシャツを2枚で3,000円にしてもらうというわけにはいかないだろうか。 
\\	ことができる	
\\	てから
\\	が 
\\	いつ来るかわからない地震には、そうするよりほかありませんし、準備した物は、地震が落ち着いてからあとで取りに帰ることもできますから。
\\	だけで	=でだけ=にだけ=だけに 
\\	に, で). 
\\	とだけ, 
\\	だけと 
\\	ー私の友達、愛がなくてもお金がある人となら結婚すると言うんですよ。 ーええ!私はお金のためだけに結婚できませんよ! が, を, 
\\	だけで(は)ない 質問はイギリスのことだけでなく、何でもいいですよ。 
\\	欲しい	ほしい 
\\	欲しいそうです, 欲しいんです 
\\	欲しがっている. 
\\	~て欲しくない ~ないで欲しい ~てもらいたくない ー子供達が(携帯電話)欲しがるんでしょうか? 
\\	ーそれもそうだと思いますが、保護者からすれば、子供とすぐ連絡が取れる手段として必要なんだと思いますよ。
\\	良すぎる	よすぎる=いい+すぎる 書きたいことがたくさんありすぎて・・・。  
\\	てください	"させる+ください 
\\	の・こと	
\\	の 
\\	の 
\\	こと, 
\\	"だと思う 
\\	だろうと思う 
\\	んじゃないかと思う 
\\	んじゃないですか/でしょうか, んじゃない?, んじゃないかな/かしら 
\\	~のことですが	
\\	の~ 明日の打ち合わせのことですが、何を用意しておけばいいですか。 
\\	旅行ですが、~ 
\\	旅行のことですが、~ 
\\	旅行には、~ 
\\	~が必要です。 旅行って、~ 
\\	旅行については、 その件については、~ 
\\	と言えば、~ 
\\	来週の会議は何について話し合うんですか?   なんについて, 
\\	なにについて 
\\	で 
\\	校長の奥さんの立場で考えて、答えましょう。
\\	せっかく~のに~	
\\	/な
\\	なのに 
\\	/な
\\	~だったのに 
\\	せっかく
\\	のに、
\\	の 
\\	せっかくの
\\	せっかく(の)お休みだったのに、昨日はどこにも出かけないで一日中家で寝て過ごした。 
\\	のに 
\\	せっかっく 
\\	せっかく梶田さんが手伝うって言ってくれているのに何も頼まないのは申し訳ないよ。 (もうしわけない) 
\\	せっかく家にパソコンがあるのに、家では殆ど使いません。 
\\	この部屋は、せっかくこんなに広いのに、お客さんが来た時しか使わないんですか。 もったいないですね。 
\\	もったいない 
\\	時間が~
\\	二週目	にしゅうめ 
\\	11月の2週目の日曜日に、~
\\	~だったら、~と思うんですけど	
\\	だったら 
\\	と思うんだけど・ですけど 
\\	い
\\	~たい, 
\\	いんだったら~ 
\\	い
\\	~たい 
\\	かったら 
\\	けど 
\\	どう?, 
\\	駅前のスーパーだったら、まだ開いていると思うんですけど、商店街はもう閉まっています。 
\\	それなら~ 
\\	それじゃ 
\\	それなら? それじゃその餃子、全部食べられないんじゃないですか? 
\\	~っと	
\\	じゃ、私も秋冬の新作をチェックしようっと。 
\\	新作 しんさく 
\\	~たことにする	
\\	する 
\\	してください, 
\\	して 
\\	1て
\\	だったことにする 
\\	ことだったことにする 
\\	たことにした 
\\	ふりをする 今日のお弁当は母が作ったものだが、自分で作ったことにして友達の家へ持っていった。 
\\	自分の悩みを上司に相談する時、友達の悩みだったことにして相談した。 
\\	ちなみに、少年は飛行機の模型(もけい)が好きだったことにして、少年の将来(しょうらい)の夢も、パイロットになることだったことにするつもりです。 デパートで服を選んでいた時、スカートを履いてみた(はいてみた)が入らなかった。悔しかったので、デザインがよくなかったことにして、他のを選んだ。 
\\	今の話、聞こえてしまったかもしれませんが、関係者の間だけの話なので聞かなかったことにしてください。 
\\	振りをする  ふり 
\\	振りをする 聞こえない~をする 
\\	けれど、電車の中の様子を見ていると、寝た振りをしたり音楽を聞いていて気がつかない振りをして席を譲らない人もけっこういるように思います。
\\	~ということでした	
\\	ってことでした. 
\\	いう 
\\	今連絡があって、今日の山登りは中止だということでした/ってことでした。 
\\	~とのことでした 
\\	と言う, そうだ, 
\\	~というものだ 
\\	だろう 
\\	というもの. 
\\	だ 
\\	というもの. だろう 
\\	初めて会った人のことを全部知ろうとするのは無理というものだろう。 
\\	日本の人に私の国について少しでも知ってもらえたら、私が日本へ来た意味があったというものです。 
\\	これこそイギリス観光の醍醐味というものです。 
\\	どんな時も自分より相手を大切に思う気持ちこそすばらしいというものだ。 
\\	勉強だけではなく、社会のルールも教えるのが教育というものです。 
\\	ーうちの高校のサッカー部、去年優勝した学校に勝って、県大会で優勝したんですよ! ーすごいですね!これがスポーツのおもしろさおいうものですね。 
\\	そんなに喜んでもらえたら、並んだ甲斐があるというものだよ。 
\\	たら~た(のに) 
\\	ば~た(のに)	
\\	""たら~た
\\	のに 
\\	なら
\\	ば, ば 
\\	~たらよかった。= 
\\	もっと道が簡単だったらこんなに遅れなかったのに。 もっと道が簡単なら(ば)こんなに遅れなかったのに。 
\\	私も一緒に選びたかったのに・・・。あ~あ、出かけなきゃよかった。 私も一緒に選びたかったのに・・・。あ~あ、出かけなかったらよかった。 
\\	始めのほう	
\\	始めのほうは見ることができたんですが、~ 
\\	最初のほう  さいしょ 
\\	私も、最初のほうだけですが、~ 
\\	たつもり	
\\	たつもりだ・たつもりだった・たつもりに~
\\	なって...) 私は、私ができることを一生懸命やったつもりだった。 
\\	すみません、メールを送ったつもりだったんですが、送られていなかったみたいです。 
\\	説明を聞いてわかったつもりになっていたが、実際にやってみたらよくわかっていないことに気がついた。 
\\	たつもり 
\\	つもりで, 
\\	つもりでいる 
\\	つもりになって. つもりになって 
\\	つもりで 
\\	だったつもり これなら生徒達もイギリスを旅行したつもりで楽しんでくれると思います。 
\\	一度死んだつもりで初めからがんばります。 
\\	妹のダンスの発表会を見に行った。他の人に比べたらそんなに上手ではなかったが 最後に満足そうな顔をしていたので、うまく踊れたつもりでいるようだ。 
\\	毎日アイスクリームを1つ食べたつもりになって、100円ずつ貯金しています。 
\\	~代わりに	
\\	代わりの
\\	の代わり, 
\\	かわり, 代わりに
\\	その代わり, 
\\	代わりに~ 
\\	の代わり= 
\\	が代わりに= 
\\	その代わりに~ 
\\	元々は出なくていいことになっていたのだが、出勤(しゅっきん)の予定だった浜田先生のお子さんに赤ちゃんが生まれそうだということで、浜田先生が急にお休みすることになったので、浜田先生が来られない代わりに私が出勤することになったのだ。
\\	てもらってもいい?	
\\	ていただいてもよろしいでしょうか。 
\\	てもらう/いただく 
\\	てもいい/てもよろしいですか。 
\\	~ください. 
\\	明日、着いたら山田さんから電話してもらっていい? 
\\	しばらくの間ここで待っていていただいてもよろしいですか? 
\\	~と伝えてくれる?	
\\	ように伝える. 
\\	と伝える 
\\	~、と。 
\\	と 
\\	って. 
\\	伝えてくださいますか, 
\\	市川さん(いしかわ)に授業が終ったら職員室に来てって伝えてくれる? 北村さん(きたむら)がお戻り(もどり)になったらお電話いただけるように伝えてくださいますか。 
\\	次第	"しだい 
\\	次第、すぐに~。 
\\	する 
\\	次第 
\\	し次第 
\\	たらすぐ~。 
\\	次第. 海外出張に出ている木村が帰国(し)次第、木村からご報告します。 
\\	その件については、確認ができ次第、すぐにこちらからもう一度ご連絡します。 
\\	キャンペーンは、プレゼントがなくなり次第終らせていただきます。 
\\	次第  しだい 
\\	か
\\	ないかのうちに 
\\	2か
\\	2ないか, 
\\	雪が降り出すか(ふりだす)(降り)出さないのうちに、ヒーターを出しておいた。 
\\	昨日の会議(かいぎ)で、新しい企画について説明し始めるか(説明し)始めないかのうちに、上司に席を立たれてしまった。 
\\	昨日は仕事が終って映画を見に行った。時間ぎりぎりに入ったので、席に着くか着かないかのうちに、映画が始まった。 
\\	着く 
\\	昨日、とてもサービスがいいレストランに友達と行った。全部食べ終わるか終らないのうちに、コーヒーが運ばれて来た。 
\\	彼のスピーチは人々を感動させ、スピーチが終るか終らないかのうちに会場中(じゅう)に拍手が起こった。 
\\	彼女はとても急いでいたのだろう。 タクシーが来るが来ないかのうちに家から出てきた。 
\\	卵の外側(そとがわ)が少しだけ焦げるか焦げないかのうちに、ひっくり返してください。 
\\	たとたん(に) 
\\	元気になった, やる気ができた). 
\\	二 
\\	もうちょっとで~ところだった, 
\\	としたら. 
\\	褒められたとたん(二)うれしくなって、やる気が出てきた。 
\\	佐藤さんのことを話したとたん(に)、佐藤さんから電話がかかって来た。 
\\	人にいいことをしたとたん(に)、私にもいいことがたくさん起きた。 
\\	ジムさんが来たとたん(に)、パテイーはとても盛り上がった。 
\\	実は昨日、晩ご飯を食べ終わったとたんに眠くなって、早く寝たので、朝早く目を覚めたんです。
\\	~と思っていたものですから	
\\	(から) 
\\	もの
\\	です) 
\\	感じていた 
\\	考えていた. 
\\	今晩お時間があったらお食事でもいかがですか? 私、山本さんと一度ゆっくり話してみたいと思っていたものですから。 
\\	から 
\\	今やっている仕事が終ってから準備しようと考えていたものですから、まだ全く準備していないのですが・・・。 
\\	昨日から何だか体調が変だなあと感じていたものですから、もしかしてと思ったものですが。 
\\	ものですから 
\\	もんだから 来月のサッカーの試合で、県の運動場を使用できないものでしょうか?近くにいい運動場がないものですから。 
\\	何とも	なんとも 
\\	何とも言えない、何とも言わない 何も言う 
\\	何とも言う 
\\	何もと言う 
\\	なー/あ/+さ	
\\	な 
\\	い
\\	ー/あ/+さ 
\\	安さ・やすさ 
\\	汚さ・きたなさ. 親切さ 簡単さ 静かさ 丈夫さ じょうぶさ 大切さ 大変さ にぎやかさ 熱心さ 必要さ 複雑さ 平和さ 便利さ 真面目さ  まじめさ 有名さ 新鮮さ 真剣さ  しんけんさ 確実さ 残念さ 大事さ 苦手さ 不便さ 危険さ  きけんさ 洋服は派手さ、地味さよりも本人に似合っているかどうかじゃないでしょうか。 
\\	~まして・でして	"て 
\\	て
\\	まして 
\\	でして 
\\	から. 
\\	て, 
\\	て 
\\	この店の雰囲気、どうです?最近気に入って(い)まして、よく来るんですよ。 すみません、その件は後でお返事させていただいていいですか? 今ちょっと急いで(い)まして。 犬と遊ぶことだけが最近の楽しみでしてねえ。 
\\	て 
\\	た, 
\\	ええ、私の買いたい本が近所の本屋ではもう売切れていて。 
\\	あ、お客様はクレジットカードでお支払いの予定ですが、カードの番号に間違いがあったようで、もう一度ご確認くださいというご連絡をメールでさせていただいていますが。 
\\	させていただいた・させていただきました 
\\	て 
\\	います 
\\	が 
\\	~て、~て	
\\	て 
\\	うれしくて、うれしくて さびしくて、さびしくて 来週、友人と行く予定なんですよ。もう楽しみで、楽しみで。 
\\	もう: 
\\	~に対して	
\\	の 
\\	に対して、
\\	に対する
\\	に対しての
\\	に対しても. 
\\	のことですが, 
\\	校長先生のご提案に対して、何かご意見のある方(かた)はいらっしゃいますか? 
\\	皆さん、子供に対しても、お年寄りに対しても、誰に対しても優しい人になってください。 
\\	相手チームが50点入れたのに対して、私達のチームは48点入れて、バスケの試合が面白くなってきた。 
\\	今までは自分のことだけに一生懸命だったが、これからはもうちょっと周りの人に対しての優しさを持ちたいと思う。 
\\	上司に対する言葉の使い方にもうちょっと気をつけたほうがいいかもしれません。 
\\	ー授業の準備っていつもどのぐらいしているんですか? ーそうですね。1つの授業に対して、3時間ぐらいでしょうか。 ーええ?そんなに時間をかけているんですか?
\\	~まで
\\	に, 
\\	~にまで
\\	お花なんかはよくプレゼントであげますが、車までプレゼントであげる人は普通いないでしょう。 
\\	さんは、今私達が使っている日本語だけでなく、昔の日本語まで勉強しているんですか。 
\\	学校は日曜日は普通休みですが、部活があるため日曜日まで働かなければならない先生もいるでしょうね。 
\\	こんなふうに、そんなふうに、あんなふうに、どんなふうに こういうふうに、そういうふうに、ああいうふうに、どういうふうに	
\\	(どんな) 
\\	(どう
\\	いう) 
\\	どう
\\	風, 
\\	ふう. 
\\	ジャンボジェットのパイロットって、カッコいいね。僕もあんなふうになりたいな。 
\\	どんなふうに打ったんですか。 
\\	そんなふうに見えません。 
\\	ー安井(やすい)先生はあまり怒らなさそうですね。 ーいいえ。私はどちらかと言うと怒りっぽいほうですよ。 ーえ?そんなふうに見えませんよ。 (怒りがち 
\\	ほう, 
\\	怒りがち 
\\	こういう
\\	こんな 
\\	こんなに) 最初はこういうレストランのメニューも全然わからなくて大変でしたよねえ。 こうなる ーこの自転車、ずいぶん古いの? ーううん、去年買ったばかりなんだけど、もうこんなに汚れちゃったんだよ。 ーへえ、外に置きっぱなしにしてると、こうなるんだ。
\\	~かどうかはともなく	
\\	駅の裏のレストランがおいしいかどうかはともなく、レストランなどの大きい会場を予約するとかなりお金がかかってしまいそうですね。 
\\	かどうか 間違いがないかどうかご確認ください。
\\	~にしろ~にしろ~ 
\\	~にせよ~にせよ 
\\	にしても 
\\	しても 
\\	にしろ
\\	にしろ 
\\	にしろ
\\	にしろ. 
\\	にしろ 
\\	どちにしろ 
\\	にしても, 
\\	レストランに行くにしても行かないしても、連絡してください。 
\\	どちにしろ 
\\	男の子か女の子が、どちらにしろ早く生まれてきてほしい。 
\\	森さんは怒るにしろ泣くにしろ、よく騒ぐ人だ。 
\\	おしゃべりが大好きな弟のことは、うるさいにしろおとなしいにしろ、気になってしまう。 
\\	上手にしろ下手にしろ、精一杯やることが何でも大切です。 
\\	手紙にしろメールにしろ、とにかく返事をください。 
\\	ー今日の旅行は楽しかったね。 ーうん。朝の集合にしろ、ランチの予約にしろ、最初は不安だったけど、何とかなったし。
\\	か~はありませんか/いませんか	
\\	か 
\\	は 
\\	が 
\\	ありますか/いる, 
\\	~ございませんか. 
\\	の
\\	何かわからない問題はありませんか? 
\\	この中で、どれか使っていないパソコンはありませんか? 
\\	どれか 
\\	何かパソコン...
\\	でなければ	
\\	じゃなければ 服装は派手じゃなければ、何でも構いません。 服装は派手でなければ、何でも構いません。 
\\	どうか	
\\	佐藤先生(さとう)、さっきから何か悩んでいる(なやんでいる)ようですが、どうかなさったねすか?
\\	た(っ)きりだ 
\\	た(っ)きり~ない	
\\	(ただ)
\\	きり 
\\	たきり 
\\	きり~ない。 
\\	きりだ、きりで、きりに、きり、 
\\	たきりです/だ。 
\\	きり~/
\\	きりです/だ。 
\\	きり~
\\	ない。 
\\	たきり~
\\	ない。 
\\	だけ, しか, 
\\	きりしか
\\	。きり 
\\	っきり 
\\	きり 
\\	まま. 
\\	先月はお金をたくさん使ったので、今残っているお金はこれ(っ)きりだ。 
\\	外国語は学生時代にフランス語を勉強したっきりだ。 
\\	(ただ, 
\\	ーあ、この本、友達に借りたまま、返してなかった! ーじゃ、早く返さないと! 
\\	友達に借りたっきり 
\\	これは親からもらった時計です。世界に一つきりの物だから、大切に仕舞っています。 
\\	世界に一つきりの物 チャンスは後一回きりしか残っていませんから、頑張ってください。 
\\	あと
\\	残っている 赤ちゃんは一人きりにしないで、いつも周りの人が注意を払うようにした方がいい。 
\\	一人にしないで 注意を払う 
\\	今月は5日に雨が降った(っ)きり、それからは一度も降っていない。 
\\	それから 
\\	田中先生とは三年前に会った(っ)きり、一度も会っていません。 
\\	それから 
\\	しかもね、その花は町中の店で一軒(いっけん)きりしか売っていなかったそうで、わざわざ遠くまで買いに行ってくれたそうなんです。 ーバンドは続けているの? ーいや、最後にやったのは大学祭でやった一回きりのらいぶだよ。 ー少し心配になってきた頃、やっと帰って来たので「どこへ行ってたの?」と聞いたんですが、私の顔を黙ってみるだけで何も言わないんです。 ーそうですか。それで? ーそれで、「どうしたの?」と聞いたら、それでもただ「あのね・・・」と言ったきり、ニコニコ笑っているだけなんです。
\\	ちょっと
\\	たいことがあるんですが	
\\	相談, 話したい, 
\\	伝える, 知らせる, 打ち合わせる 
\\	教える, 確認する, 連絡する, 知る. 
\\	ちょっと
\\	ですけど 
\\	ちょっと教えてもらいたいことがあるんですが、夜電話してもいいですか? 
\\	ちょっと伝えておきたいことがあるんですが、今お時間はありますか? 
\\	には	
\\	=ために(は). ため 
\\	のには 
\\	には~=
\\	を
\\	には~=
\\	ここの交差点はよく事故がある。事故をなくすにはこの交差点に信号をつける必要があるだろう。 
\\	町の人と親しくなるにはどうしたらいいでしょうか? 
\\	明日、駅で待ち合わせ(を)するには、どこがわかりやすいですか?
\\	を~て~と~	
\\	て/と 
\\	て~と
\\	て~たら・なら・て
\\	て
\\	て
\\	と
\\	て
\\	学校の階段を上がって、屋上に出てから空を見ると、とても気持ちがいい。 
\\	階段を下りて、左に行くとトイレがあります。 
\\	小さい川を渡って、まっすぐ進むと大きな気が見えた。 
\\	喫茶店を出てから、右へまっすぐ行って、1つ目の信号を越えると突き当たりがあったので、そこを右に曲がって来ました。 ~の
\\	を~て 
\\	を(右に)曲がって~ 一つ目の突き当たりを左、二つ目の突き当たりを右に曲がって少し行くと私の家がある。 
\\	曲がる 
\\	猫は花壇のそばを通って、それから隣の家の壁を登って行った。 
\\	ホームの階段を下りて、駅の改札を出てから、傘を電車の中に忘れて来たことに気がついた。 
\\	ず(に)	
\\	ずに=
\\	ないで 
\\	ないで
\\	する=せず(に) 来る=来ず(に) (こず) る
\\	ず(に) う
\\	ず(に) 
\\	-ない) 
\\	ないで, なくて 
\\	あれ、休みの時間なのに、休憩(きゅうけい)せずに、仕事をしているんですか? 予習をして来るように言っておいたのに、わからない言葉も調べずに来たんですか? 
\\	ず=なくて、 
\\	約束の時間を30分過ぎても友達は来ず、連絡も取れない。どうしたのだろう? 
\\	に: このまま私達が何もせず地球の温度が上がり続けたら、この辺りは沈みかねない。 
\\	か
\\	ないかのうちに	
\\	2か
\\	2ないか, 
\\	雪が降り出すか(ふりだす)(降り)出さないかのうちに、ヒーターを出しておいた。 
\\	昨日の会議(かいぎ)で、新しい企画について説明し始めるか(説明し)始めないかのうちに、上司に席を立たれてしまった。 
\\	昨日は仕事が終って映画を見に行った。時間ぎりぎりに入ったので、席に着くか着かないかのうちに、映画が始まった。 
\\	着く 
\\	昨日、とてもサービスがいいレストランに友達と行った。全部食べ終わるか終らないのうちに、コーヒーが運ばれて来た。 
\\	彼のスピーチは人々を感動させ、スピーチが終るか終らないかのうちに会場中(じゅう)に拍手が起こった。 
\\	彼女はとても急いでいたのだろう。 タクシーが来るが来ないかのうちに家から出てきた。 
\\	卵の外側(そとがわ)が少しだけ焦げるか焦げないかのうちに、ひっくり返してください。 
\\	たとたん(に) 
\\	元気になった, やる気ができた). 
\\	二 
\\	もうちょっとで~ところだった, 
\\	としたら. 
\\	褒められたとたん(二)うれしくなって、やる気が出てきた。 
\\	佐藤さんのことを話したとたん(に)、佐藤さんから電話がかかって来た。 
\\	人にいいことをしたとたん(に)、私にもいいことがたくさん起きた。 
\\	ジムさんが来たとたん(に)、パテイーはとても盛り上がった。 
\\	実は昨日、晩ご飯を食べ終わったとたんに眠くなって、早く寝たので、朝早く目を覚めたんです。 
\\	としたら 
\\	とする=
\\	たら=
\\	丁度~しようとしている  (ちょうど) 
\\	食べようとしている。 
\\	とする 
\\	男性が自転車を駐輪しようとした時に、他の自転車に当たってしまた。 (あたって) 
\\	とおもう 
\\	直そうと思った。 
\\	いろいろと
\\	いろんなことを
\\	新入生はまだ入学したばかりでわからないことが多いので、いろいろと教えてあげてください。 
\\	私でよかったら	
\\	ええ、私でよかったら何でも言ってください。 
\\	はもちろん、
\\	(も)だ。	
\\	1はもちろん、
\\	2も~。 
\\	いい企画にするのはもちろん、これから宣伝活動にも力を入れていかなきゃ。 
\\	っぽい	
\\	のような, 
\\	らしい. 
\\	春らしい色のスカート=春っぽい 
\\	っぽい= 
\\	たっぽい= 
\\	っぽかった. 
\\	っぽくなる. 
\\	一昨日(おととい)公園の木の下で小さくて黒っぽい猫を見かけた。 
\\	私は都会的な店よりも田舎っぽい雰囲気の店が好きだ。 
\\	っぽい 例: 飽きっぽい(あきっぽい, 
\\	怒りっぽい 
\\	明日は大事なミテイングだったよね? 山田さんは忘れっぽいから、今晩確認の電話をしておくね。 
\\	ー安井(やすい)先生はあまり怒らなさそうですね。 ーいいえ。私はどちらかと言うと怒りっぽいほうですよ。 ーえ?そんなふうに見えませんよ。 (怒りがち 
\\	ほう, 
\\	怒りがち 
\\	部屋の電気は消えているし、とても静かだし、みんな帰ったっぽいなあ。 
\\	がちだ 
\\	だ 
\\	で, なので, がちなこと, がちにする, 
\\	てしまいがちだ。 
\\	っぽい) 
\\	やすい/なりやすい. 
\\	病気がちな人=病気になりやすい人 
\\	させる 
\\	私は車の運転があまり得意じゃないので、家族で出かける時は、父に運転させてしまいがちです。 
\\	このレストラン、建物が立派なので高そう家で入りにくいと思われがちですが、そうじゃないんです。 
\\	になる 
\\	朝は、あまり食べません。コーヒーだけになりがちで・・・。 
\\	ある 
\\	日本語が少し話せても読み書きは全然できないというのは、外国人の場合ありがちなことだ。 
\\	歩き始めたばかりの赤ちゃんは転びがちだから、注意して見ていないといけません。
\\	ーああ、最近、仕事が進まなくて、いらいらしてしまう・・・。 ー私も、月末(げつまつ)になると忙しいからいらいらしがちだけど、そういう時は、ちょっと外の空気を吸い(すい)に行くといいよ。 ああ、部長は仕事に厳しい(きびしい)からよく怖い人って勘違いされがちだけど、面倒の良い、部下思いの良い上司なんだよ。 
\\	ー細かくて(こまかくて)難しい計算(けいさん)も、パソコンなら速いし、正確だから、安心していいですよね。 ー確かに、パソコンでの計算は正確だからと安心しがちだけど、入力ミスとかもあるかもしれないから、最後はちゃんと人の目で確認しなきゃね。 ー昨日、インスタント食品を買いに行ったんだけど・・・。 ーえ、私、母によく「インスタント食品は体によくないからあんまり食べるな!」って言われるよ。 ー確かに、そう言われがちだけど、最近は安全な材料で、健康的なものが出ているんだよ。 一人だと外で食事をしがちなので、栄養を考えてできるだけ野菜を多く食べるようにしています。 
\\	外で食事をする  そとでしょくじ 
\\	外で食べる/
\\	てみせる	
\\	つもりだ. どんな人からみてもいいと思えるような、もっとかっこいいポスターを作ってみせますよ! 
\\	今度の試合は絶対に勝ってみせる。 
\\	方	"ほう 
\\	楽しみは最後に取っておくほうだ。 
\\	一般的にロンドンのホテルは高い方だと思います。 (たかいほう) 
\\	ー安井(やすい)先生はあまり怒らなさそうですね。 ーいいえ。私はどちらかと言うと怒りっぽいほうですよ。 ーえ?そんなふうに見えませんよ。 (怒りがち 
\\	ほう, 
\\	怒りがち 
\\	み・さ 
\\	"さ 
\\	い 
\\	な
\\	~っぽい 
\\	~らしい. 
\\	客観的  きゃっかんてき 
\\	客観 きゃっかん 
\\	程度 (を表す あらわす)(一定のレベルにある) 程度 ていど 
\\	一定  いってい 
\\	尺度  (を表す)(どのくらいか) 尺度  しゃくど 
\\	み 
\\	い 
\\	な
\\	主観的 しゅかんてき 
\\	主観 しゅかん 
\\	感覚 (を表す) 感覚 かんかく  
\\	場所・点 (を表す) 場所: 
\\	点: 
\\	さ, 程度 
\\	このケーキの甘さはちょうどいい。 子供達の元気さにいつも負けてしまう。 ミュージカルの衣装のカラフルさに感動した。 (カラフルな) 足を折った時の痛さは口では表現できない。 彼女の話し方の大人っぽさにみんな驚いた。(おどろいた)(~っぽい 
\\	新しい職場で私らしさをアピールしたい。 (~らしい 
\\	いい 
\\	よさ. 静かな 
\\	静かさ・静けさ. 
\\	な
\\	さ 
\\	な. 
\\	嫌な(いやな), 暇な(ひまな), 変な 
\\	的 
\\	国際的, 自動的, 具体的 (ぐたい 
\\	さ, 尺度 
\\	新しいテレビの大きさは何インチ? (小ささ 
\\	この建物の高さはどのくらい? (低さ ひくさ 
\\	この荷物の重さは5kgだった。 (軽さ かるさ 
\\	髪の長さは肩まである。 (短さ みじかさ 
\\	さ 
\\	反対語でペアになる場合は、一般的に顕著さ(けんちょさ 
\\	が高いほうの語(ご)に「さ」が付く。 
\\	厚い・薄い(あつい・うすい)、暑い・寒い(あつい・さむい)、うれしい・悲しい、重い・軽い、大きい・小さい、濃い・薄い(こい・うすい)、高い・低い、強い・弱い、広い・狭い(ひろい・せまい)、深い・浅い(ふかい・あさい)、難しい・易しい(やさしい) み, 感覚 
\\	このケーキの甘みは砂糖(さとう)ではない。 (味覚 みかく) 試合のあと、右足(みぎあし)に痛み(いたみ)を感じた。 (痛覚  つうかく) 秋になって、木の葉(このは)が赤み(あかみ)を帯びてきた(おびてきた・帯びる 
\\	。 (視覚 しかく 
\\	けっこう厚み(あつみ)がある本だね。 (視覚・触覚 しかく・しょっかく 
\\	毎年、お正月(しょうがつ)が楽しみだ。 (情的感覚 じょうてきかんかく 
\\	この学校に親しみを覚えた。 (情的感覚) 彼はああ見えて面白みがある。 (情的価値 価値・かち 
\\	親切にしてもらって人の温かみを感じた。 (情的価値) (親切にしてもらって 
\\	親切にもらった, 
\\	校長先生は人として重みがある。(精神的価値 精神・せいしん 
\\	これは深み(ふかみ)のあるコーヒーですね。 (味覚的価値 みかくてきかち) 
\\	さ 
\\	感覚 
\\	青い、赤い、黒い、白い、甘い、辛い(つらい)、苦い(にがい)、うまい、すごい、暖かい・温かい(あたたかい)、厚い(あつい)、痛い(いたい)、重い、おもしろい、悲しい、苦しい(くるしい)、親しい、楽しい、懐かしい(なつかしい) み, 場所・点 
\\	川の深み(ふかみ)にはまった。 (場所) (はまる 
\\	高みの見物に行った。 (場所) (高みの見物: 
\\	事件(じけん)の真相が明るみに出た。 (点?) (真相 しんそう 
\\	強そうに見える彼にも弱みを見つけた。 (点) 
\\	明るい、高い、深い(ふかい)、強い、弱い 
\\	校長先生のお話は、生徒それぞれの高みに向かってがんばろうというお話だった。 
\\	楽しみは最後に取っておくほうだ。 
\\	ほう=方, 
\\	皆の心の痛みがわかる人になってください。 
\\	でもそれで文字ばかりのポスターになっちゃったらおもしろみがなくなるしなあ・・・。 
\\	ーあれ?これ、ちょっと色が変だよ。赤っぽいよね? ー本当だ。赤みが強すぎるんだ。 ーもう一度調整してみてよ。
\\	ー歌のテマは何にしましょうか? ー例えば命の大切さとかはどうですか?
\\	からみると、	
\\	昨日の試験の結果からみると、彼はきっと希望している高校に合格できると思います。 
\\	あの様子からみると、彼女はずいぶん疲れているらしい。 
\\	からすると・すれば 
\\	親からすると、子供は何歳になっても心配です。 
\\	思う): 佐々木君からすれば、私の意見は間違っていると思われるでしょう。 
\\	その地域の人からすると普通でも、他の地域の人からするとびっくりするような習慣もある。 
\\	最近服を着ているペットが多い。飼っている人からすればかわいいのだろうが、ぺっとからすれば、服を着せられてストレスを感じているかもしれない。 
\\	ー子供達が(携帯電話)欲しがるんでしょうか? 
\\	ーそれもそうだと思いますが、保護者からすれば、子供とすぐ連絡が取れる手段として必要なんだと思いますよ。
\\	でしたら	
\\	だったら 本当でしたらそうしていただきたいのですが、皆さん忙しいと思いますので、代表の方お一人に答えていただくというのはいかがでしょう? ー今日の会議の司会は、英語科(えいごか)の先生だったら誰でもいい? ー今日の会議の司会は、英語科の先生でしたら誰でもよろしいでしょうか? 司会 しかい 
\\	それなら~ 
\\	出す	
\\	はじめる 
\\	走り出す  
\\	朝の通勤途中、ジムさんは生徒たちと話しながら歩いていたら、だんだん時間がなくなってきたのに気づいて、慌てて走り出しました。 作り出す  
\\	作り始める 
\\	作り出したら、うちで作るのも、だんだん楽しくなってきますよ。 
\\	走り出る 
\\	でる 
\\	だす 走り回る 走りまわる 
\\	走り込む 走りこむ 
\\	走り抜ける 走りぬける 
\\	もうちょとで
\\	ところだった もう少しで
\\	ところだった	
\\	ちょっと, 少し 
\\	で 
\\	そうになった. 
\\	ぶつかりそうになった, 
\\	としたら 
\\	だった 
\\	んだよ 
\\	んだそうだ 
\\	電車を間違えて、もうちょっとで大事な会議に遅刻するところでした。 
\\	さっき野球のボールが飛んで来て、もうちょっとで窓が割れるところだったんだよ。 
\\	今朝、雪で道が滑ったでしょう?私、もう少しで転ぶところでしたよ。 
\\	できるだけ	
\\	この時のポイントは、フライパンに対してお箸(はし)をできるだけまっすぐ立てて、こんなふうにフライパンとお箸を両方動かしながら、早く混ぜる(まぜる)ことです。 完璧(かんぺき)というわけには行かないが、できるだけミスのないように企画したつもりだ。 
\\	一人だと外で食事をしがちなので、栄養を考えてできるだけ野菜を多く食べるようにしています。 
\\	外で食事をする  そとでしょくじ 
\\	外で食べる/
\\	なるべく 
\\	できれば 教室の外でも、なるべく日本語を使ったほうがいいでしょう。 
\\	宿題は、なるべく次の日に出してください。 
\\	そこを何とか~	
\\	ー申し訳ございませんが、此方の商品(しょうひん)は、もう発送段階(はっそうだんかい)に入っていまして、今からキャンセルというわけにはいかないのですが・・・。 ーそこを何とかキャンセルしていただくわけにはいきませんか? 
\\	くださる	
\\	ください チケットを買うのは、ここに並べばいいかな。あ、でも電話がかかって来た。佐藤先生(さとう)は待ってくださるかな?聞いてみよう。 
\\	かねる	
\\	(できない), 
\\	る 
\\	する 
\\	しかねる 
\\	ことができない, 
\\	かねる 
\\	いたす 
\\	お
\\	する. 
\\	できる 
\\	する. しかねる=
\\	できかねる=
\\	できる 
\\	する 
\\	ー期末テストの成績表(せいせきひょう)を明日中(あしたじゅう)に出していただくことはできますか? ーえ?テストは今日終ったばかりですよ。それはいくらなんでもできかねます。 いたす 
\\	する 
\\	パソコンの故障ですか。テクニカルなご質問は、私ではお答えしかねます/お答えいたしかねます。修理担当のスタッフに代わりので、少々お待ちになっていただけますか? 
\\	申し訳ございません。私では、わかりかねますが・・・。 
\\	申し訳ございませんが、一度お召しになったものはお取り替えしかねるのですが。 
\\	その提案はとてもいい提案だと思いますが、今日の会議は話し合うことが多いので、今日中(きょうじゅう)にはちょっと検討しかねます。次の会議で検討することにしましょう。 
\\	ーあのう、申し訳ないんですが、このセーター、洗濯したら縮んでしまったので、取り替えていただけないでしょうか? ー申し訳ございませんが、一度お召しになったものはお取り替えしかねるのですが。  私一人では判断しかねますので、他の先生のご意見も聞いて検討してみたいと思います。 
\\	テキスト) 相談センターに聞いたところ、そのような複雑な内容は、短い時間ではお話しかねるとの事でした。 
\\	そのような~ 
\\	そんな
\\	かねない 
\\	丸めた雪はかたいので、当たるとは怪我をしかねませんからね。 
\\	このまま私達が何もせず地球の温度が上がり続けたら、この辺りは沈みかねない。 
\\	忙しい時はあいさつをしなかったり、返事をしなかったりすることがあるが、失礼な人だと思われかねない。いくら忙しくても、どんな時でも、気がついたら、あいさつや返事をするように気をつけたい。 
\\	学校の廊下を走ったら誰かとぶつかって、自分はもちろん、相手にも怪我をさせかねないので、廊下を走ってはいけません。 
\\	しっかりとデータを集めて、自分でまとめた資料を作らないと、あとで上司に書き直しさせられかねないから、今のうちからがんばろう。 
\\	ー最近、忙しくて、食事もきちんとできないんです。 ー食事だけはちゃんとしないと、体を壊しかねません・壊しかねないですよ。 ーアランさんは誕生日のパーティーのことを内緒にして、びっくりさせましょう。 ーそうですね。でも
\\	さんはパーティーの前にアランさんに言いかねないですよ。 ーそうですね。じゃ、
\\	さんにも内緒ですね。 ーニュースで
\\	国と
\\	国の首脳会談が決裂したら、戦争になりかねないと言っていましたよ。 ー心配ですね。何とか話し合いで解決してもらいたいですね。
\\	と
\\	と、どちら~	
\\	今年は、このテキストとそのテキスト、どちらにしましょうか? 
\\	は
\\	と
\\	のどちらですか? 会話に合うデパート駅の絵は
\\	と
\\	のどちらですか? どちらの
\\	は~? 
\\	どの) じゃ、どちらの出口がお店に近い
\\	さんに聞いておきます。
\\	~をもらえませんか?	
\\	一日だけでいいんで、休みをもらえませんか? 
\\	だけでいいので	
\\	んで 
\\	から 
\\	ので. 
\\	から). 
\\	だけがいい. 
\\	だけでいい
\\	ので. 
\\	でいい 
\\	だけ 
\\	これでいい. 
\\	ので 
\\	んで 
\\	一日だけでいいんで、休みをもらえませんか? 
\\	会場の準備、ちょっと手伝ってくれないかな?いすを並べてくれるだけでいいから。 
\\	何も意見を言わなくても、参加するだけでいいので、会議に出てくださいませんか? 
\\	こっちの小さいのだけでいいから、荷物持ってくれない? 
\\	(て)からでいいので 
\\	から 
\\	ので. 
\\	から 
\\	て 
\\	てから. 
\\	~たあとでいい. 
\\	から). 
\\	お願い!付き合って!5時からでいいから。 
\\	デパートに買い物に行きませんか?そのビデオを見終わってからでいいので。 
\\	これからもボランテイア活動を続けるかどうか、よく検討して(けんとう)もらってからでいいので、報告してください。 
\\	てくださると助かるんですが	
\\	くれる~だけど 
\\	ですが/だけど 
\\	のを手伝って... 
\\	と 
\\	助かる 
\\	と
\\	今度のセミナー、一緒に来てくれると助かるんだけど、無理かな・・・? 
\\	この仕事を手伝ってくださると助かるんですが・・・。 
\\	仕事を 
\\	あのう、書く物を忘れちゃったので、鉛筆(えんぴつ)かボールペンを貸してくれると助かるんだけど・・・。 
\\	がちだ	
\\	だ 
\\	で, なので, がちなこと, がちにする, 
\\	てしまいがちだ。 
\\	っぽい) 
\\	やすい/なりやすい. 
\\	病気がちな人=病気になりやすい人 
\\	させる 
\\	私は車の運転があまり得意じゃないので、家族で出かける時は、父に運転させてしまいがちです。 
\\	このレストラン、建物が立派なので高そう家で入りにくいと思われがちですが、そうじゃないんです。 
\\	になる 
\\	朝は、あまり食べません。コーヒーだけになりがちで・・・。 
\\	ある 
\\	日本語が少し話せても読み書きは全然できないというのは、外国人の場合ありがちなことだ。 
\\	歩き始めたばかりの赤ちゃんは転びがちだから、注意して見ていないといけません。
\\	ーああ、最近、仕事が進まなくて、いらいらしてしまう・・・。 ー私も、月末(げつまつ)になると忙しいからいらいらしがちだけど、そういう時は、ちょっと外の空気を吸い(すい)に行くといいよ。 ああ、部長は仕事に厳しい(きびしい)からよく怖い人って勘違いされがちだけど、面倒の良い、部下思いの良い上司なんだよ。 
\\	ー細かくて(こまかくて)難しい計算(けいさん)も、パソコンなら速いし、正確だから、安心していいですよね。 ー確かに、パソコンでの計算は正確だからと安心しがちだけど、入力ミスとかもあるかもしれないから、最後はちゃんと人の目で確認しなきゃね。 ー昨日、インスタント食品を買いに行ったんだけど・・・。 ーえ、私、母によく「インスタント食品は体によくないからあんまり食べるな!」って言われるよ。 ー確かに、そう言われがちだけど、最近は安全な材料で、健康的なものが出ているんだよ。 一人だと外で食事をしがちなので、栄養を考えてできるだけ野菜を多く食べるようにしています。 
\\	外で食事をする  そとでしょくじ 
\\	外で食べる/
\\	期限を何でも忘れがちなんだから、ちゃんとカレンダーにメモしておくとか、工夫しておきなさいよ。 
\\	っぽい 
\\	のような, 
\\	らしい. 
\\	春らしい色のスカート=春っぽい 
\\	っぽい= 
\\	たっぽい= 
\\	っぽかった. 
\\	っぽくなる. 
\\	一昨日(おととい)公園の木の下で小さくて黒っぽい猫を見かけた。 
\\	私は都会的な店よりも田舎っぽい雰囲気の店が好きだ。 
\\	っぽい 例: 飽きっぽい(あきっぽい, 
\\	怒りっぽい 
\\	明日は大事なミテイングだったよね? 山田さんは忘れっぽいから、今晩確認の電話をしておくね。 
\\	ー安井(やすい)先生はあまり怒らなさそうですね。 ーいいえ。私はどちらかと言うと怒りっぽいほうですよ。 ーえ?そんなふうに見えませんよ。 (怒りがち 
\\	ほう, 
\\	怒りがち 
\\	部屋の電気は消えているし、とても静かだし、みんな帰ったっぽいなあ。 
\\	よさそう	
\\	いい 
\\	よさ 
\\	よい?)
\\	そう 
\\	ない 
\\	なさ
\\	そう ー安井(やすい)先生はあまり怒らなさそうですね。 ーいいえ。私はどちらかと言うと怒りっぽいほうですよ。 ーえ?そんなふうに見えませんよ。 (怒りがち 
\\	ほう, 
\\	怒りがち 
\\	か~。	
\\	か 
\\	の. どこへ行ったのかわからない。
\\	そうしないと、~	
\\	自分の国の料理が食べたい時、自分で作らなければなりませんでした。そうしないと、日本では国の料理は決して食べられなかったからです。だから、二本に来て、ずいぶん料理が上手になりました。
\\	そういえば、	
\\	ーそうだ、今年も新年会をしませんか? ーいいですね。そういえばミーラ先生、去年の新年会には遅刻して来ましたよね。 
\\	ことのないように	
\\	ない) 
\\	が 
\\	の. 
\\	ー今年は遅れることのないようにしてくださいね。 ーええ、もちろん!去年はたまたま時計が壊れていて・・・。(こわれ) 
\\	パスワードは絶対に他の人に知られることのないようにしてください。 
\\	このレポート、時間もあまりないから、書き直しをさせられることのないようにがんばりましょう! 
\\	日本は地震の多い国だから、地震がいつ来ても慌てることのないように、日ごろから必要な物を準備しておく必要があります。
\\	きる	
\\	終わる/終える 
\\	きる 
\\	う
\\	(きらない, きれない
\\	きったまま, きれますか 
\\	きった
\\	う
\\	いきる, 
\\	ってきる 山田さん、そんな疲れきった顔をしてどうしたんですか? 
\\	急だから空いているとは言いきれないけど・・・。 
\\	プリンターのインク、最後まで使いきらないうちに交換するのはもったいないですよ。 
\\	1組ではあるゲームにみんなが興味を持って盛り上がったけど、2組でも必ず盛り上がるとは言いきれませんよね。 
\\	い)
\\	これから	
\\	これからが楽しくなるのに・・・。 
\\	~というと	
\\	というと
\\	というと 
\\	というと? 
\\	というと? 
\\	というと、~
\\	というと~ 
\\	といいますと. 
\\	っていうと ー
\\	さん、もう帰ってしまうんですか?これからが楽しくなるのに・・・。 ー(これからが楽しくなる)というと?/といいますと? 
\\	ー「サファリ」っていう映画、もう見た? ー「サファリ」っていうと、最近公開された、ラオインが主人公のアニメの? 
\\	選手、今日は調子が悪かったと(って)いうと、本当の実力はもっとすごいということですか? ー昨日は本当に運が悪い日だったよ。 ーっていうと? ~ということですよね 
\\	ということ 
\\	~ね 
\\	~ね. 
\\	ということでした 
\\	とのことでした) 
\\	だよね 
\\	っていうこと... 
\\	ということですよね 
\\	ということ 
\\	再提出(さいていしゅつ)しろと言われたということは、この書類のどこかに間違いがあるということですよね? 
\\	このメニューの数字、1が一番辛くないということだよね? 
\\	じゃ、新卒(者)で入ったとしても、少なくとも今年で30歳なるということですよね。 ーこの「現地解散」ってホテルで解散(する)ということだよね? ーそうそう。 ーそれから一日目の食事って、予定表に「7時からホテルの食堂で」と書いてあるから、みんなで食べるということだよね? 
\\	食事(を)するということだよね) ーうん。そういうことだよね。 ~って~よね 
\\	よ:
\\	ね:
\\	って 
\\	~ということですよね 
\\	って
\\	よね。 
\\	ーメルベイユ神戸って・・・。 (こうべ) ー最近オープンしたばかりのケーキ屋さんだよね? ーそうそう。よく知っているね。 
\\	村山さんが結婚するって本当だよね? 
\\	安くておいしいってこのレストランのことですよね? 
\\	内容が複雑ってたぶん山本さんはこの本のことを言ってたんですよね? 
\\	確か、カリスマ教師の田沼先生(たぬま)が講演なさるのって
\\	日って 
\\	のって)英語教師のセミナーの最終日(さいしゅうび)でしたよね? お茶の産地って京都が一番有名だよね? ~なくない? 
\\	くない 
\\	今日は1月のわりに、寒くなくないですか? 
\\	もうこの古いかばんは使わなくない?捨てようか。 
\\	~というのは~ことで、~ 
\\	という 
\\	ことで 
\\	というのは~. 
\\	ことで、 
\\	というのは
\\	ことで、
\\	ことで 
\\	この「ペット禁止」というのは、このアパートでペットを飼ってはだめだということで、ペットがいる人はこのアパートに住むことはできません。 
\\	「鼻が高い」というのはとてもすばらしくてうれしいと思うということで、例えば「私は友達の木村さんがテニスの高校チャンピオンになって、鼻が高いです」というように使います。 
\\	「持ち込み」というのは外から何かが持って入るということで、「持ち込み
\\	と書いてあるカラオケなどはお菓子やジュースを持って入っていいんですよ。 
\\	さっき田中さんが言っていた明日締め切りの仕事というのは部長から急に頼まれた会議の資料作り(づくり)のことで、元々は田中さんの仕事じゃないんです。 
\\	さっき隣のクラスの山田先生が言っていた「このクラスは元気すぎますねえ」というのは、このクラスはうるさいということで、ちょっと静かにしてほしいということを遠回しに言ったんだと思います。 女性専用車両というのは原則として女の人だけが利用する車両のことで、主(おも)にラッシュの時間帯(たい)に導入されているんですよ。 ー
\\	さんのいない旅行は、ピエロのいないサーカスのようなものだから。 ーえ?どういう意味? ー人気者(もの)がいないから、盛り上がらないってことだよ。 線というのは、どのようなものでしょうか? 
\\	~かと言うと 
\\	な
\\	かというと 
\\	な) 
\\	な 
\\	日曜日は暇かと言うと、そうではない。平日になかなかできない家の仕事がたくさなるのだ。 
\\	冬、北海道の家の中は寒いかと言うと、その反対で、とても暖かくて
\\	シャツで過ごせます。 
\\	小学生だからと言って子供のような考えしか持っていないかと言うと、とてもしっかりした考え方を持っていることもあって、びっくりさせられることがある。 
\\	友達が家を買った。長い間(あいだ)自分の家を持つことが夢だったそうなので、これで満足したかと言うと、まだ100%満足していないそうで、次は庭を立派にしたいそうだ。 
\\	野球部はとても厳しい部活として知られている。そのため、やめる生徒も多いが、みんなやめてしまうかと言うと、がんばって続ける生徒もいて、全国大会でもいい成績(せいせき)を取っている。 
\\	例えば~とか~とか	
\\	例えば 
\\	とかの
\\	とか 
\\	とかの
\\	なんか~ 
\\	とかの
\\	とか 
\\	とか/なんか 
\\	プレゼントは例えば
\\	とかの今人気のブランドのアクセサリーなんかが喜ばれるんじゃない? 
\\	例えば少し寒気がするとか頭が痛いとか、おかしいなと思ったらすぐ知らせてください。 
\\	休みの日は例えば近くの公園を散歩するとか映画を見るとかして楽しいんでいます。 
\\	とかして楽しい>
\\	とかするのが好き 
\\	例えば内容が複雑だったとか参加費が高かったとか、何でもいいので、今日は講演の感想をお願いします。 
\\	ー歌のテマは何にしましょうか? ー例えば命の大切さとかはどうですか?
\\	行ったことはない	
\\	は 
\\	が 
\\	私は~.
\\	かけ(だ) 
\\	かける	
\\	(かけだ) 
\\	かけだから, かけで, かけの
\\	(かける) 
\\	かけた, かけたら, かけている/た. 
\\	(~てはじめる), 
\\	(もう少しで~ところだった), 
\\	としたら, 
\\	か
\\	ないかのうちに). 
\\	をする 
\\	をしかける 病気が治りかけだから、あと少し家で休みます。 
\\	パソコンの調子が悪くて、作りかけのデータが消えてしまった。 
\\	もう少しで~ところだった 
\\	朝、急いでいたので、間違って違う電車に乗りかけた。 
\\	その時計は電池が切れかけているから、5分くらい遅れているよ! 気をつけて! 
\\	ちょっと待って。今、昨日会った人の名前を思い出しかけているところだから。 
\\	ううん、干しっぱなしの服は、ちょうど今しまいかけていたんだよ。 ー最近、私、忘れっぽいんですよ。 ー私もです。昨日買った本を読みかけてから、まだ最後まで読んでいない本があったことを思い出したりして・・・。 ーそう、そう。私もありますよ。
\\	っぱなしだ	
\\	っぱなしだ, っぱなしだった, っぱなしで, っぱなしの
\\	っぱなしにする 
\\	ている 
\\	まま 
\\	まま 
\\	今日は初めての授業だったので、一日緊張しっぱなしだった。 
\\	あ、佐藤さんにこの本借りっぱなしだ!早く返さなきゃ! 
\\	もうすぐ) 資料を机の上に出しっぱなしで帰ってしまった。 
\\	書きっぱなしの日記が机に置いてあります。 
\\	雨が降っているのに、洗濯物を外に干しっぱなし(ほし)じゃないの!それに、窓も開けっぱなしだ!こんなひどい雨なのに窓を開けっぱなしにしていると、雨が降り込んじゃうじゃないの。それから、机の上に置きっぱなしのジュースの缶(かん)があるじゃない。 ーこの自転車、ずいぶん古いの? ーううん、去年買ったばかりなんだけど、もうこんなに汚れちゃったんだよ。 ーへえ、外に置きっぱなしにしてると、こうなるんだ。 ー歌の内容はどんなものですか? ーあ、それはまだ。短い言葉を書きっぱなしのままで。 
\\	プレーヤーで映画を見ようと思ったら、モニターに線が入りっぱなしで、見にくいんです。 でも、最初はそのこつがわからなかったので、失敗しっぱなしでした。 
\\	日本の夏は蒸し暑いので、夜寝られなくて一晩中クーラーをつけっぱなしにしたくてどうしようもない時がありますね。 
\\	~というのはいかがでしょう?	
\\	たらどう? 
\\	たら 
\\	というのは. 
\\	たらどう, 
\\	という?) 
\\	どう?, 
\\	たらどう?, 
\\	というどう? 
\\	たら 
\\	というのはどう. 
\\	でしょう 
\\	どうですか 
\\	どう 
\\	っていうのはどう?, どうでしょう 
\\	いかが) 今度の保護者会儀では、代表で佐藤先生にあいさつしていただくというのはいかがですか? 
\\	で 
\\	明日のパーテイーの場所は、久しぶりに私のうちでというのはどう? 
\\	午後はお客さんがたくさん来るので、午前中にこの仕事をしてしまうっていうのはどうでしょう? 
\\	じゃ、ナガイ先生の見たいお寺が他にもあったら教えてくださいね。それから、休憩も兼ねて茶道を体験するというのはどうでしょう? 本当でしたらそうしていただきたいのですが、皆さん忙しいと思いますので、代表の方お一人に答えていただくというのはいかがでしょう?
\\	~ようなもの(だ)	
\\	みたいなものだ 
\\	もの 
\\	ような/みたいな. 
\\	の/ことは
\\	のようなものだ。 
\\	ようなものらしい。 
\\	のようなもの. 
\\	写真を撮ることは(とる)、私にとっては日記のようなものです。 
\\	でも、給料日にほしかった物を買うことは、私にとってはがんばった自分へのご褒美のようなものだから。 夫と妻というのは空気のようなものらしい。普段は気づかないが、いないとお互いに困るんだとか。 
\\	最近の携帯電話は機能(きのう)が増えて、小さいパソコンを持っているようなものだ。 
\\	ペットを飼うことは、子供を育てるようなものです。大変ですけど、かわいいですよ。 
\\	実際のところは、単に習慣としてやっているようなものですけど。 
\\	さんのいない旅行は、ピエロのいないサーカスのようなものだから。 ーえ?どういう意味? ー人気者(もの)がいないから、盛り上がらないってことだよ。 私のような者でもお役に立てればうれしいです。 
\\	の方が~より	
\\	できれば平日よりはお休みの方がいいかな・・・。 
\\	~だけしか~ない	
\\	だけ
\\	しか~
\\	ない 
\\	~しか~ない. 昨日の夜、国の友達と電話で話しので、3時間だけしか寝ていない。 
\\	ケーキバイキングに行ったが、お腹が痛くなって、1皿だけしか食べられなかった。 
\\	そうなると、18日だけしか今のところ候補になる日がないことになりますね。 
\\	授業の準備が終わり、会議の時間になったので、会議室に行った。 すると、会議室には、安井先生だけしかいなかったので、他の先生方(がた)が来るまで安井先生とお話した。 でも、なかなかいい服がなかったので、結局手袋だけしか買わなかった。 ただの
\\	佐々木君が私の彼のわけがありません!ただの友達です。 
\\	ないものでしょうか?	
\\	だろうか. 
\\	でしょうか 
\\	ないもの 
\\	でしょうか, 
\\	か 
\\	かどうか 
\\	私, 
\\	もらえる/ いただく 
\\	させる. 
\\	何とか. 
\\	/私 
\\	大塚さんが一人で忙しそうだが、何か手伝って上げられないものなのかなあ。 
\\	来月のサッカーの試合で、県の運動場を使用できないものでしょうか?近くにいい運動場がないものですから。 
\\	何とか, 
\\	もらえる 
\\	何とか金曜日までに予定をはっきり決めてもらえないものか、もう一度山田さんに相談してみるつもりだ。 
\\	もらえる, させる 
\\	来週国から家族が来るから、一日だけでも休ませてもらえないものだろうか。 
\\	かどうか 
\\	それから、もしみんなの都合を合わせるのが難しければ、他にいい方法がないものかどうかまたご相談させてください。 
\\	いただく 
\\	今度の集まりには、もっと多くの保護者の方に来ていただけないものでしょうか? 
\\	大塚さんが一人で忙しそうだが、何か手伝って上げられないものなのかなあ。 
\\	な 
\\	の, 
\\	か 
\\	なあ 
\\	かな(あ)) ー
\\	先生、このデータのことなんですけど、計算方法が複雑なのか、ファイルにたびたびエラーが出るので困っているんですが・・・。
\\	先生、パソコンが詳しいですよね?ちょっと見ていただけますか? ーわかりました。・・・うーん、確かに、複雑な方法を使っているみたいですね。もうちょっとシンプルな方法がないものか、私も考えてみますよ。 パソコンが詳しい: 
\\	シンプルな 
\\	複雑
\\	が	
\\	だ 
\\	日本語で書くのは時間がかかるが、楽しい。
\\	の一つ	
\\	お金が一番大事だという人がいるが、お金はただ生活に便利な物の一つ(である)にすぎないと思う。 
\\	そんな言葉の一つに
\\	がある。使いようは幅広い。 日本社会の「きまり」にかかわる例の一つに、会社への就職活動がある。 日本人はスポーツが好きで、いろいろなスポーツが盛んだが、一番人気のあるスポーツの一つは、野球だろう。
\\	/な
\\	だと言う	
\\	だ.  お金が一番大事だという人がいるが、お金はただ生活に便利な物の一つ(である)にすぎないと思う。 
\\	ブルーチーズを食べたら、とてもおいしかった!今まで苦手だと思っていたが、においが嫌いだったにすぎないのがわかった。 
\\	を手伝う	
\\	を 
\\	に 私はいつも少し掃除を手伝っているにすぎないのに、母はとても喜んでくれる。 
\\	片付けを手伝う 
\\	~にすぎない	
\\	ただ 
\\	にすぎない 
\\	と思う, のに~, のではないだろうか, のがわかった. 
\\	にすぎない 
\\	であるにすぎない. 
\\	な
\\	だったにすぎない. 
\\	それは迷信にすぎませんね。 
\\	お金が一番大事だという人がいるが、お金はただ生活に便利な物の一つ(である)にすぎないと思う。 
\\	情報公開の時代になったと言っても、実際のところ利用者にはほんの少しの情報が公開されるようになったにすぎないのではないだろうか。 
\\	な
\\	ブルーチーズを食べたら、とてもおいしかった!今まで苦手だと思っていたが、においが嫌いだったにすぎないのがわかった。 
\\	私はいつも少し掃除を手伝っているにすぎないのに、母はとても喜んでくれる。 
\\	ー泣いているんですか? ーええ。つい感動してしまって。 ーまだテーマが決まったにすぎないんです。 
\\	には当たらない	
\\	わざわざ 
\\	よ。. 
\\	私/
\\	いえいえ、そんな大したことじゃありません。こんなこと、褒めていただくには当たりませんよ。 
\\	昨日見に行った映画は、期待したほどおもしろくなかった。お金を払って見るにはあたらない映画だった。 
\\	お金を払って見て
\\	見に払う 
\\	さんは、いつも人に優しくて、強い心を持っていて、私は
\\	さんを尊敬しています! ーいえいえ、尊敬されるには当たりませんよ。私はまだまだですから。 
\\	わざわざ 
\\	私の提案は大した提案じゃないから、みんなでわざわざ集まって検討するには当たらないよ。 
\\	ー新しいショッピングセンターはどうでしたか? ー近くのショッピングセンターと入っているお店はほとんど同じでしたよ。場所も遠いですし、わざわざ行くには当たりませんよ。 
\\	行く必要には当たりません。 
\\	ー明日の野球の試合、大丈夫なあ・・・。 ー心配するには当たりませんよ!たくさん練習しましたし、大丈夫ですよ。自信を持ってください! 
\\	心配には当たりません。
\\	に当たる  う
\\	に 
\\	当たる う
\\	当てる あてる  る
\\	当て嵌まる  あてはまる  う
\\	例のように、会話にあうように
\\	の
\\	に当て嵌まる表現を入れましょう。 当て嵌める る
\\	当て物  あてもの 
\\	(当てがうもの) 
\\	こうやって	
\\	こうやって、ほんのちょっとレンジで温めると・・・。 
\\	どうやって  
\\	どうやって知り合いになったんですか? 
\\	た
\\	です(か)
\\	部のみんなは~ 
\\	はみんなが~ 
\\	みんな 
\\	たとたん(に)	
\\	元気になった, やる気ができた). 
\\	に 
\\	もうちょっとで~ところだった, 
\\	としたら. 
\\	た)+とたん(に) 
\\	褒められたとたん(に)うれしくなって、やる気が出てきた。 
\\	佐藤さんのことを話したとたん(に)、佐藤さんから電話がかかって来た。 
\\	人にいいことをしたとたん(に)、私にもいいことがたくさん起きた。 
\\	ジムさんが来たとたん(に)、パテイーはとても盛り上がった。 
\\	実は昨日、晩ご飯を食べ終わったとたんに眠くなって、早く寝たので、朝早く目を覚めたんです。 
\\	次第  しだい 
\\	次第、すぐに~。 
\\	する 
\\	次第 
\\	し次第 
\\	たらすぐ~。 
\\	次第. 海外出張に出ている木村が帰国(し)次第、木村からご報告します。 
\\	その件については、確認ができ次第、すぐにこちらからもう一度ご連絡します。 
\\	キャンペーンは、プレゼントがなくなり次第終らせていただきます。 
\\	次第  しだい 
\\	か
\\	ないかのうちに 
\\	2か
\\	2ないか, 
\\	雪が降り出すか(ふりだす)(降り)出さないのうちに、ヒーターを出しておいた。 
\\	昨日の会議(かいぎ)で、新しい企画について説明し始めるか(説明し)始めないかのうちに、上司に席を立たれてしまった。 
\\	昨日は仕事が終って映画を見に行った。時間ぎりぎりに入ったので、席に着くか着かないかのうちに、映画が始まった。 
\\	着く 
\\	昨日、とてもサービスがいいレストランに友達と行った。全部食べ終わるか終らないのうちに、コーヒーが運ばれて来た。 
\\	彼のスピーチは人々を感動させ、スピーチが終るか終らないかのうちに会場中(じゅう)に拍手が起こった。 
\\	彼女はとても急いでいたのだろう。 タクシーが来るが来ないかのうちに家から出てきた。 
\\	卵の外側(そとがわ)が少しだけ焦げるか焦げないかのうちに、ひっくり返してください。
\\	ていただけるとありがたいんですが 
\\	てもらえるとありがたいんだけど	
\\	もらえる~んだけど 
\\	、お願いできませんか。. 
\\	もらえる/いただける 
\\	ですけど. お願いできない 
\\	飛行機のチケットを一緒に予約してもらえるとありがたいんですが、お願いできませんか? 
\\	傘を持っていないんだけど、今すごい雨が降ってるんだ。だから、駅まで迎えに来てもらえるとありがたいんだけど、お願いできないかな? 
\\	家にいないことが多いので、いらっしゃる前に電話をしていただけるとありがたいんですが。 
\\	先生、すみません。この机をあそこへ運ぶので、手伝っていただけるとありがたいんですが。 
\\	音楽が聞こえてくるので、もう少し音を小さくしていただけるとありがたいんですが。 
\\	ていただけないかと思いまして 
\\	てもらえないかと思って 
\\	校長先生、記念に校長先生にも私と一緒に写真に入っていただけないかと思いまして・・・。よろしいでしょうか? 
\\	英語教育で有名な佐藤先生に会わせていただけないかと思いまして、北海道から参りました。 (まいりました) 
\\	工藤さん、今ちょっと時間ない?この資料を見てもらえないかと思って。 
\\	こと(は)ない	
\\	~が必要ない 
\\	なくてもいい 
\\	は 
\\	わざわざ. 
\\	(気にすることはない) 
\\	(落ち込むことはない). は 
\\	そんなにひどい風邪を引いているんだったら、無理して学校へ来ることなかったのに。 わざわざ 
\\	でもわざわざ持って来てもらうことはないですから、今度会う時まで預かっておいてください。  (あずかって) いや、もう遅いし、今からわざわざスーパーへ行って、作ることはないですよ。 
\\	結婚式の受付のお手伝い、急に頼んでしまってすみません。受付をしている時、ずっと立っていることはないですから、人が来ない時は座って行ってくださいね。 
\\	一人で悩むことはないですよ。困った時は、みんなで相談して決めましょう。 
\\	その仕事は全部高田さんにやってもらうことはないですよ。グループで分担すればいいですから。 
\\	明日の講演会のために、今日中にいすを並べておくことはありませんよ。 
\\	ないでいる	
\\	ないで 
\\	いる. 
\\	と, のは~, たら. 
\\	いる 
\\	たら, 
\\	こんな寒い日にコートを着ないでいると風邪を引きますよ。 
\\	学校をずっと休まないでいるのはすごいことです。 
\\	ずっと彼女に連絡しないでいたんですよ。そうしたら、電話番号がわからなくなってしまいました。 
\\	そろそろ寝たほうがいいんじゃないですか?冬休みだからと言って、いつまでも寝ないでいると朝起きられなくなりますよ。 
\\	しばらくこの店に来ないでいたら、新しい店員さんが増えていました。 
\\	2ヶ月間、髪(かみ)を切らないでいましたからね。そろそろカットに行かないと。 しばらくの間(あいだ)無駄遣いしないでいたんですが、先週本や洋服を買いすぎてしまって、実は今あまりお金がないんです。
\\	よりほか~ない。	
\\	しかたがない 
\\	どうしよう. 
\\	が(い)ない 
\\	ない 
\\	電話番号を書いたメモをなくしてしまったので、パソコンにメールを送るよりほか彼と連絡を取る方法がない。 
\\	どうしよう 
\\	この文の意味がわかりませんが、今は辞書で調べるよりほかどうしようもありません。 
\\	5分くらいなら、大丈夫だよ。結局、待つよりほかどうしようもないしね。 
\\	しかたがない 
\\	村上さんが休むなら、私がかわりに会議に出席する(しゅっせき)よりほかしかたがないですね。 
\\	先生、お待たせしてしまって申し訳ありません。急用とは言え、あやまるよりほかありません。 
\\	週末買い物するのを忘れていたので、昨日はインスタントラーメンよりほかしかたがないですね。 (の 
\\	こと 
\\	あの時の私には、ミラーさんよりほかお願いできる人がいなかった。 
\\	あの時の私には
\\	あの時に私にとって 
\\	の
\\	に
\\	いつ来るかわからない地震には、そうするよりほかありませんし、準備した物は、地震が落ち着いてからあとで取りに帰ることもできますから。  ~てどうしようもない 
\\	しかたがない 
\\	て 
\\	しかた 
\\	先日お会いしたすてきな女性にもう一度会いたくてどうしようもないんです。 時々やる気が出なくてどうしようもない/しかたがない時があるが、そんな時は町に出かけるとリフレッシュできる。 
\\	のどが乾いてどうしようもない/しかたがないんですが、何か飲むものはありませんか? 
\\	ぐらい
\\	てください	
\\	ぐらい
\\	ください 
\\	させる 
\\	くらい
\\	くらい
\\	くらい
\\	くらい? 
\\	ぐらい. 
\\	くらい
\\	お腹がすいたと言ってばかりいないで、早く食べたいならサラダの用意ぐらい手伝ってよ。 
\\	サラダくらい...) させる/
\\	疲れているので、今日ぐらいゆっくり寝させてください。 
\\	少しぐらい休ませてくださいよ。 
\\	開けっぱなしにしないで、ドアぐらい閉めてください。 
\\	帰る前に電話ぐらいしてよ。 故障で今シャワーが使えませんが、2,3時間ぐらい我慢してください。 
\\	部屋を使ったあとは、電気ぐらい消してください。 
\\	ーねえ。このケーキ、作ったんだ。食べてみて。 ーええ?なんか形が崩れてて、あまり・・・。 ー見た目ぐらい我慢してよ。味は本当においしいんだから。 ーうん、じゃ、一口。 
\\	くらい 
\\	ぐらい. せめた
\\	は~ 
\\	せめては
\\	くらい/ぐらい 
\\	せめて 
\\	くらい, 
\\	せめて一回日本へ行ってみたいなあ。 少なくとも 
\\	じゃ、新卒(者)で入ったとしても、少なくとも今年で30歳なるということですよね。 一応  いちおう 
\\	ー行きたいけど、この仕事、どうしても今日中にやらないといけないから・・・。 ーそうなんだ。でも、もし早く終ったら一応電話くれる? そうですね。一応私から教頭(きょうとう)には伝えておくので、出勤してから教頭に確認してみてくれますか? 
\\	は
\\	くらい(だ。) 
\\	くらい。 
\\	だ 
\\	かもしれない, 
\\	1くらい
\\	ぐらい 
\\	な
\\	なくらい 
\\	くらいの
\\	ぐらい. 
\\	彼女となら結婚してもいいと思えるぐらいすてきな女性に出会った。 
\\	ちょっと失敗したぐらい大丈夫ですよ。そんなに気にしないでください。 
\\	まだ3月だというのに、今日の天気は暑いくらいだ。 
\\	会議に来て行くスーツは、ちょっと地味なくらいがちょっといいよ。 
\\	少しいたずらをしても、元気すぎるぐらいの子供が好きだ。 
\\	リンリンさんは指を痛めるぐらい何度もピアノの練習をした。 リンリンさんは本番の時、お客さんの方を見られないぐらい緊張していた。 
\\	2て、
\\	1くらいだ。 
\\	1ほど
\\	程 
\\	ほどではない. でも、心配するほどではなく、数日で調子を取り戻すことができた。 動けないほどお腹がいっぱいだ。 
\\	東京はびっくりするほど人が多い。 
\\	2て
\\	1ほどです。 
\\	~ば~ほど 
\\	/な
\\	であればあるほど 程 
\\	小さければ小さいほど毒が強い。  勉強しなければ勉強しないほど~ 
\\	どうせ~し...	
\\	(どうせ~~し...) 
\\	とにかく~), 
\\	""/し, 
\\	どうせ
\\	/な
\\	だ
\\	し. 
\\	だし 
\\	明日は家族がみんな出かけて、どうせ一人だし、晩ご飯は軽く済ませようと思います。 
\\	な
\\	だし 
\\	明日はどうせ暇だし、映画にでも行こうかな。 
\\	どうせ今日は大雨で野球は中止になると思うし、その時は家で本でも読もう。 
\\	あのデパートに行くのはやめたよ。どうせ高いし。 
\\	このツアー旅行ではどうせたくさんのお土産屋へ連れて行かれるし、今この店で急いでお土産を買うのはやめておこう。 
\\	明日はどうせ暇だし、映画にでも行こうかな。 市川さん(いちかわ)を誘っても(さそっても)どうせ断るし/断られるし、二人で行こう。 どうせお金もないし、疲れているから、明日はうちにいよう。 
\\	こともある	
\\	時々 
\\	それに餃子の中にお金やピーナッツ、飴などを入れておくこともあるそうです。 
\\	元気でいればこそ、悲しいこともあるけれど、それ以上にきっと楽しいこともありますよ。 
\\	~にしろ~にしろ~	
\\	~にせよ~にせよ 
\\	にしても 
\\	しても 
\\	にしろ
\\	にしろ 
\\	にしろ
\\	にしろ. 
\\	にしろ 
\\	どちにしろ 
\\	にしても, 
\\	レストランに行くにしても行かないしても、連絡してください。 
\\	どちにしろ 
\\	男の子か女の子が、どちらにしろ早く生まれてきてほしい。 
\\	森さんは怒るにしろ泣くにしろ、よく騒ぐ人だ。 
\\	おしゃべりが大好きな弟のことは、うるさいにしろおとなしいにしろ、気になってしまう。 
\\	上手にしろ下手にしろ、精一杯やることが何でも大切です。 
\\	手紙にしろメールにしろ、とにかく返事をください。 
\\	ー今日の旅行は楽しかったね。 ーうん。朝の集合にしろ、ランチの予約にしろ、最初は不安だったけど、何とかなったし。 ~かどうかはともなく 
\\	駅の裏のレストランがおいしいかどうかはともなく、レストランなどの大きい会場を予約するとかなりお金がかかってしまいそうですね。 
\\	かどうか 間違いがないかどうかご確認ください。
\\	1といい
\\	2といい 
\\	も~も~ 
\\	にしろ~にしろ~ 
\\	あのサッカー選手はスピードといいテクニックといいすばらしい。 この服はデザインといい色といいとてもユニークです。 
\\	おいしい料理といい楽しいダンスといい、このパーテイーは最高だ。 
\\	東京といいニューヨークといいソウルといい、大都市はたいてい地下鉄が通っている。 
\\	すごくよかった!観光地(かんこうち)でも旅館でもいろいろおもしろいことがあって・・・。ああ、休み時間内(ない)には全部話しきれないなあ。 魚も野菜も新鮮なのに限る。 
\\	この映画について一方ではすばらしいと言う人もいるが、他方ではつまらないと言う人もいる。 
\\	世界には一方では物が豊富な国もあり、他方では食べる物もない国もある。 
\\	何度も何度も練習した。
\\	1といい
\\	2といい	
\\	も~も~ 
\\	にしろ~にしろ~ 
\\	あのサッカー選手はスピードといいテクニックといいすばらしい。 この服はデザインといい色といいとてもユニークです。 
\\	おいしい料理といい楽しいダンスといい、このパーテイーは最高だ。 
\\	東京といいニューヨークといいソウルといい、大都市はたいてい地下鉄が通っている。 
\\	~にしろ~にしろ~ 
\\	~にせよ~にせよ 
\\	にしても 
\\	しても 
\\	にしろ
\\	にしろ 
\\	にしろ
\\	にしろ. 
\\	にしろ 
\\	どちにしろ 
\\	にしても, 
\\	レストランに行くにしても行かないしても、連絡してください。 
\\	どちにしろ 
\\	男の子か女の子が、どちらにしろ早く生まれてきてほしい。 
\\	森さんは怒るにしろ泣くにしろ、よく騒ぐ人だ。 
\\	おしゃべりが大好きな弟のことは、うるさいにしろおとなしいにしろ、気になってしまう。 
\\	上手にしろ下手にしろ、精一杯やることが何でも大切です。 
\\	手紙にしろメールにしろ、とにかく返事をください。 
\\	ー今日の旅行は楽しかったね。 ーうん。朝の集合にしろ、ランチの予約にしろ、最初は不安だったけど、何とかなったし。 ~かどうかはともなく 
\\	駅の裏のレストランがおいしいかどうかはともなく、レストランなどの大きい会場を予約するとかなりお金がかかってしまいそうですね。 
\\	かどうか 間違いがないかどうかご確認ください。
\\	すごくよかった!観光地(かんこうち)でも旅館でもいろいろおもしろいことがあって・・・。ああ、休み時間内(ない)には全部話しきれないなあ。 魚も野菜も新鮮なのに限る。 
\\	この映画について一方ではすばらしいと言う人もいるが、他方ではつまらないと言う人もいる。 
\\	~ているところありません。	
\\	ところ. 
\\	は	
\\	そうしたくはないが、他に方法がないためしかたがない。 そんなに安くはないけど、~
\\	彼らは鳥と同じくらい高く飛んだ。
\\	「日本での活動」
\\	小川先生とナオミ、リンリンがナオミのことを話しています。
\\	お+
\\	な
\\	ご 
\\	おきれい、お好き(すき)、お得意(例外 れいがい)、お早い、お優しい(やさしい)、お若い、お忙しいところありがとうございます、お元気ですか?、ご丁寧(ていねい) ー授業中は英語しか使わないので、日本語にはまだまだ自信がありません。失礼があったらすみません。 ーいえいえ、とてもお上手じゃないですか! 
\\	ております	
\\	ております. 
\\	ております 
\\	しておって 
\\	しており 
\\	私は健康のために毎日走っております。 
\\	私は去年まで北側町に住んでおりました。 
\\	辻は今授業をしておりまして、電話に出られないのですが。 
\\	ており 
\\	父は医者をしており、毎日遅くまで病院で働いております。 
\\	お/ご
\\	ください	
\\	お~ください 
\\	する, 来る, 
\\	見る, 着る, いる) 
\\	する, 
\\	ご~ください. 
\\	返事 
\\	お 
\\	ご. お返事 
\\	ご返事 
\\	お 
\\	ご 
\\	ます(か). 
\\	越す 
\\	ご覧(らん) 
\\	来る 
\\	見る. 
\\	国際交流フエアにお越しくださった皆様、本日はどうもありがとうございました。 
\\	ます 
\\	会議の前にこの資料をお読みください。わからない点は、会議でご質問くださいますか? 
\\	お任せください! 
\\	どうぞこちらのお席におかけください。 
\\	お
\\	です	
\\	お
\\	じゃありません 
\\	です, です 
\\	でしたら 
\\	なの 
\\	ありです 
\\	お
\\	の
\\	/お
\\	のようだ. 
\\	ているところ 
\\	小川先生はまだお見えじゃありませんね。もう少し待ちしましょうか。 
\\	なの 
\\	ミラー先生は
\\	教室をお使いじゃありません。今お使いなのは安井先生です。 
\\	あり, たら 
\\	校長先生は午後はお出かけです。校長に急用がおありでしたら、携帯に連絡をお取りして見ましょうか? 
\\	の
\\	ー何をお探しですか? ー時計を探しているんです。確かここに置いたと思うんですが、ないんです・・・。 ーあ、
\\	さんがお探しの時計って、もしかしてこれじゃありませんか? よう 
\\	ー大丈夫ですか?だいぶお疲れようですねえ。 ーいえ、疲れているんじゃなくて、深呼吸をしただけなんですが。 (しんこきゅう) 
\\	ているところ 
\\	ー最近安井先生、調子が悪そうですね。お気づきですか? ーいいえ、全然気づきませんでした。 
\\	ー今日の予定ではどちらにお泊りですか? ー友達のうちに泊まろうと思っています。 ー何かお困りですか? ーええ。道に迷ってしまったんです。 
\\	ます
\\	こんにちは。先日そちらでお世話になりました北側小学校のナオミナガイと申します。 
\\	1あっての
\\	県大会で優勝できたのはラッキーだったからではない。毎日の練習あっての優勝だ。 
\\	私はサッカーが大好きで、サッカーがない高校生活は考えられません。サッカーあっての高校生活です! 
\\	生徒あっての学校です。もっと生徒の意見を聞くべきです。 
\\	愛あっての結婚だ!お金のための結婚はできない! 健康あっての仕事だ!やりたい仕事をしたいなら、体を大切にするべきだ。 
\\	健康な体あって・・・) 
\\	あっての
\\	だ。 
\\	~をもとに(して)	
\\	によると 
\\	これらの写真をもとに、プレゼンテーションソフトを使って授業を進めて行こうと思います。 
\\	子供の頃の思い出をもとに(して)30年前のこの町の様子を絵に描いてみました。 
\\	この映画は本当にあったことをもとに(して)作られた。 
\\	インターネットで集めた情報をもとに(して)まとめた情報をブログで公開したら、みんなからわかりやすい、ありがたいというコメントをたくさんもらった。 
\\	目撃者(もくげきしゃ)の証言(しょうげん)もとにして犯人の絵を描きました。 300年以上前に宝物を埋めたという資料が発見され、それもとにして町中の人が驚きました。(おどろきました) 日本で経験したこともとにして異文化についての本を書こうと思っています。 今年は生徒へのアンケートもとにして修学旅行の行き先を考えます。 
\\	としては 
\\	電気街として知られている。 
\\	によって 
\\	によって
\\	にとって 
\\	に. 
\\	に. 日本人にとって~  
\\	~によると 
\\	ー新聞によると人口増加につれて食料不足(ぶそく)や水不足(みずぶそく)、ごみ問題などが生じるそうですよ。 ーそうなんですか。今から対策を考えておかなくてはいけませんね。
\\	本当にあったこと この映画は本当にあったことをもとに(して)作られた。 
\\	顔色が悪いからと言って、(必ずしも)病気(だ)とは限らないよ。何かショックなことがあったのかもしれないよ。 
\\	さんが何だかうれしそうですね。何かいいことがあったんでしょうか?
\\	への	
\\	今年は生徒へのアンケートもとにして修学旅行の行き先を考えます。
\\	のついでに	
\\	の. 
\\	の, 
\\	(ついで) 
\\	買い物のついでにこの手紙を出して来てください。 
\\	台所を掃除するついでに冷蔵庫の中もきれいにした。 
\\	京都へ出張したついでに大学の時の友達に会った。 
\\	京都へ来たついでに琵琶湖(びわこ)を見に行きませんか。 
\\	~というものだ	
\\	だろう 
\\	というもの. 
\\	だ 
\\	というもの. だろう 
\\	初めて会った人のことを全部知ろうとするのは無理というものだろう。 
\\	日本の人に私の国について少しでも知ってもらえたら、私が日本へ来た意味があったというものです。 
\\	これこそイギリス観光の醍醐味というものです。 
\\	どんな時も自分より相手を大切に思う気持ちこそすばらしいというものだ。 
\\	勉強だけではなく、社会のルールも教えるのが教育というものです。 
\\	ーうちの高校のサッカー部、去年優勝した学校に勝って、県大会で優勝したんですよ! ーすごいですね!これがスポーツのおもしろさというものですね。 
\\	そんなに喜んでもらえたら、並んだ甲斐があるというものだよ。 
\\	~ということでした 
\\	ってことでした. 
\\	いう 
\\	今連絡があって、今日の山登りは中止だということでした/ってことでした。 
\\	~とのことでした 
\\	と言う, そうだ, 
\\	の私	小学生の私がそんなに高いホテルに泊まれるわけがありませんよ。 
\\	て 
\\	来た時, 
\\	日本に来て大変だったことはありますか? 
\\	めったに~ない	
\\	珍しく~ 「七人の侍」なんてめったにやらないんだから、来た時に見なきゃ。 やる 
\\	小村市にはめったに雪が降らないからなおさら喜びますよ。 
\\	の/
\\	恐れがある	
\\	恐れ 
\\	恐れはない. 
\\	てしまう. 雪で足元が滑る恐れがありますから、十分注意してください。 
\\	今晩は大雪の恐れがあります。多いところでは60cm以上積もるでしょう。 
\\	これは食べると危険です。子供が食べる恐れがあるため、子供の手が届かない所に置いてください。 
\\	出入り口の近くに荷物をたくさん置いておくと、家事や地震で逃げようと思ってもドアが開かない恐れがあるので、荷物は置かないようにしてください。 
\\	てしまう 
\\	この動物は減ってきていて、10年後には地球からいなくなってしまう恐れがあるそうだ。 
\\	恐れはない 
\\	この薬を飲めば、手術したあとで痛む恐れはありませんから、安心してください。 
\\	心配しないで この形の鍵(かぎ)を使っている人は気をつけてください。簡単に泥棒(どろぼう)にうちに入られる恐れがあります。
\\	このきのこは毒があります。食べたら死ぬ恐れがあります・死んでしまう恐れがありますから、山で見つけても絶対に採って、食べないでください。 授業中地震が起こった場合、パニックになって生徒が怪我(を)する・してしまう恐れがありますから、慌てないように注意してください。 恐れ・怖れ  おそれ 
\\	(戦
\\	念), 
\\	怖), 
\\	かたわら	
\\	の, 
\\	しながら, 
\\	3年生は受験(じゅけん)勉強のかたわら、卒業アルバムをみんなで一生懸命作っている。 
\\	あの人は仕事のかたわら、ボランティア活動にも熱心に参加している。 
\\	ここへ旅行に来た人は海のスポーツを楽しむかたわら、いつもとは違う時間をゆっくり過ごしている。 
\\	~なんてことはありえない	
\\	~なんてありえない, 
\\	なんてことはありえません. 
\\	だ, 
\\	な
\\	だ. 
\\	なんてことはありえなかった 
\\	なんてことはありえない. 
\\	でしょう, が、. 
\\	な
\\	どんな仕事でも楽だなんてことはありません。(らくだ) 
\\	ハードディスクを買って送料が1万円だなんてことはありえないよ。 
\\	でしょう 
\\	佐藤先生(さとう)が連絡しないで学校を休むなんてことはありえないでしょう。 
\\	ちゃんと鍵をかけたのだから、泥棒に入られるなんてことはありえない。 
\\	~ありえない, 
\\	あれいない, ~が、 
\\	私が子供の時は、冬にこの町に雪が降らないなんてことはありえなかったが、最近はあまり降らない年(とし)もある。 
\\	ありえない 
\\	アランさんが人と話さなかったなんてことはありえない!友達の中で彼が一番おしゃべりなんだから。 
\\	アランさんが人と話さないなんてことはありえない!友達の中で彼が一番おしゃべりなんだから。 リンリンさんがご飯を全部食べないで残すなんてことはありえないよ!いつももったいないと言って、全部食べるんだから。調子が悪かったのかなあ・・・。
\\	~ならまだしも	
\\	な
\\	なら 
\\	だ, 
\\	な) 
\\	ても 
\\	まだしも 
\\	それだけならまだしも... 
\\	だ, な): 普通の会話ならまだしもスピーチはまだ無理です。 
\\	な
\\	にぎやかならまだしも、静かな所を夜一人で歩くのは怖いです。 
\\	兄は少しでも貯金があるならまだしも、全く貯金をしていないのに車を買うつもりなんです。 
\\	大切な書類をうちに忘れてしまったならまだしも、間違えて捨ててしまったそうです。 
\\	部屋がとても広いならまだしも、そんなに広くないのに10人も友達が遊びに来て困りました。 
\\	ー生徒達にたくさん雪をぶつけられてしまいました。 ーそのようですね。靴だけならまだしも、洋服までびっしょりですからね。 
\\	大丈夫ですよ。外国ならまだしも日本だからね。言葉が通じますから、わからなければ聞けば大丈夫です。
\\	~やら~やら	
\\	や 
\\	やらで
\\	やら
\\	やら言う 
\\	やら 
\\	やら
\\	やら テストやら宿題やらやることがたくさなって、生徒達も毎日忙しそうだ。 
\\	やるで 
\\	旅行は暑いやら荷物が重いやらで疲れてしまった。 
\\	引越しは荷物をまとめるやら市役所で手続きをするやら、、やらなくてはいけないことが多くて、なかなか休めない。 
\\	山口さんは山に登る前は疲れるから行きたくないやら面白くないやら言っていたが、実際に行ったら一番楽しんでいた。 
\\	昨日弟の部屋に入ったら、漫画(まんが)やら
\\	やらいろいろな物が散らかっていて、とても汚くてびっくりしました。 それが大変だったんですよ。みんな歌うやら踊るやら(おどる)大騒ぎで、大家さんに怒られてしまったんです。 今朝、バス停で転んでしまったんです。しかも、人がたくさn並んでいて・・・。腰が痛いやら、見られて(は)恥ずかしいやらでさんざんでした。
\\	ぐらいなら~
\\	(た/ない)ほうがましだ	
\\	くらい 
\\	ぐらい 
\\	ほうがまし 
\\	ほうがいい ましである 
\\	(人が主語) 
\\	午後から何だか体調が悪い。無理して仕事を続けて明日休むぐらいなら今日早く帰ったほうがましだと思って、定時になったらすぐに帰ることにした。 
\\	こんなことを言って林さん困らせてしまうぐらいなら、何も言わないほうがましです。 
\\	ほうがまし 
\\	この絵は自分で描いたものですが、誰にも見てもらえないぐらいなら飾らないほうがましなんじゃないかと思って、しまってあったんです。 
\\	のことだから	
\\	あの, そんな. 明るい中村君のことだから、高校へ行ってもたくさん友達ができると思いますよ。 
\\	父はゴルフ好きで暇があったらゴルフをしている。そんな父のことだから今週末もゴルフに行くだろう。 
\\	ナオミさんのことだから、小学生に楽しく英語を教えているでしょう。子供が好きですから。 
\\	うちの近くの喫茶店は春になると桜を使った面白いメニューを出すので人気がある。あの店のことだから、今年も新しいメニューを出すだろう。今年はどんなものか、今から楽しみだ。 
\\	あのレストランのご主人のことだから、予算を言えばその中でメニューを考えて作ってくれるはずだ。 
\\	か~ない	
\\	も 
\\	久しぶりの休みだから、どこか出かけない?
\end{CJK}
\end{document}