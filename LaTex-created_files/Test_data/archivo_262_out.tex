\documentclass[8pt]{extreport} 
\usepackage{hyperref}
\usepackage{CJKutf8}
\begin{document}
\begin{CJK}{UTF8}{min}
\\	少年少女におくる言葉
\\	會津八一
\\	私は新潟の生れで小學校は西堀小學校(今はないが、廣小路の消防の詰署のある附近)へ通つたものだ。そこを出て大畑の高等小學校へ進んだが、成績はけつして優等どころでなく、やうやく眞中へとどくかとどかないかといふ程度だつた。
\\	卒業する時、學校へ自分の目的を紙に書いて出すこととなつた。その時私の同級生は總理大臣になりたいとか、陸軍大臣けん海軍大臣になるとか、さういふことをはなばなしく書いて出した人が多かつた。私は今でもわすれないが、小學校を出たなら百姓になる、ただの百姓で一生くらしたいといふことを書いて出した記憶がある。
\\	當時そんなことを書いたのは私だけだつたと思ふ。當時の私は年齡的にも希望に輝いてをらず成績もあまりよくなかつたために、そんなことを書いたのだらうと思ふ。
\\	けつして今いふところの平民思想とかを當時もつてゐたのではない。ただ私が、ふるはない、平凡な、そして學問もあまりはなばなしくないただの子供だつたことを示すものだ。
\\	しかしそれから中學へやつてもらひ、進んで大學も出ることができ、今日まで學問をつづけることができた。最初體がよわかつたので、希望も消極的だつたと思ふが、今日七十二歳の高齡に達しても、わりあひ丈夫でゐる。人間の一生といふものはけつして二年や三年で勝負のつく、いはば短距離競走ではなく、六十年、七十年、時として百年にもわたる長距離競走だから、なんといつても體が一番大切だ。
\\	しかしその體も、もちやうによつてはもつものだ。私の知人で八十何歳になる人で、子供の時體が弱かつたといふ人が二人も三人もゐる。
\\	自分の體のくせと、弱みを守つてゆく、その手かげんさへわかれば、あんぐわい長く、最初體の強いのをほこつてゐた人よりも、かへつて長生きをすることができるといふことがわかる。
\\	底本:「會津八一全集 第七卷」中央公論社
\\	1982(昭和57)年4月25日初版発行
\\	底本の親本:「會津八一全集 第六卷」中央公論社
\\	1968(昭和43)年刊行
\\	初出:「少年少女新潟日報」
\\	1952(昭和27)年1月13日
\\	入力:フクポー
\\	校正:鴨川佳一郎
\\	2017年6月13日作成
\\	青空文庫作成ファイル:
\\	このファイルは、インターネットの図書館、青空文庫
\\	で作られました。入力、校正、制作にあたったのは、ボランティアの皆さんです。
\end{CJK}
\end{document}