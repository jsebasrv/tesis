\documentclass[8pt]{extreport} 
\usepackage{hyperref}
\usepackage{CJKutf8}
\begin{document}
\begin{CJK}{UTF8}{min}
\\	それ	それ	それ	
\\	それはとってもいい話だ。	それはとってもいい 話[はなし]だ。	それ は とっても いい はなし だ	
\\	はとってもいい 話[はなし]だ。			
\\	一つ	一[ひと]つ	ひとつ	
\\	それを一つください。	それを 一[ひと]つください。	それ を ひとつ ください	
\\	それを
\\	ください。			
\\	一	一[いち]	いち	
\\	一から始めましょう。	一[いち]から 始[はじ]めましょう。	いち から はじめましょう	
\\	から 始[はじ]めましょう。			
\\	二	二[に]	に	
\\	その人には二回会った。	その 人[ひと]には 二[に] 回会[かい あ]った。	その ひと に は に かい あった	
\\	その 人[ひと]には
\\	回会[かい あ]った。			
\\	二つ	二[ふた]つ	ふたつ	
\\	ソフトクリームを二つください。	ソフトクリームを 二[ふた]つください。	そふとくりーむ を ふたつ ください	
\\	ソフトクリームを
\\	ください。			
\\	三	三[さん]	さん	
\\	彼女は三人の子供の母親だ。	彼女[かのじょ]は 三[さん] 人[にん]の 子供[こども]の 母親[ははおや]だ。	かのじょ は さんにん の こども の ははおや だ	
\\	彼女[かのじょ]は
\\	人[にん]の 子供[こども]の 母親[ははおや]だ。			
\\	三つ	三[みっ]つ	みっつ	
\\	コップを三つ買いました。	コップを 三[みっ]つ 買[か]いました。	こっぷ を みっつ かいました	
\\	コップを
\\	買[か]いました。			
\\	これ	これ	これ	
\\	これをください。	これをください。	これをください。	
\\	をください。			
\\	五つ	五[いつ]つ	いつつ	
\\	桃を五つください。	桃[もも]を 五[いつ]つください。	もも を いつつ ください	
\\	桃[もも]を
\\	ください。			
\\	五	五[ご]	ご	
\\	五人で旅行に行きました。	五[ご] 人[にん]で 旅行[りょこう]に 行[い]きました。	ごにん で りょこう に いきました	
\\	人[にん]で 旅行[りょこう]に 行[い]きました。			
\\	六つ	六[むっ]つ	むっつ	
\\	息子は六つになりました。	息子[むすこ]は 六[むっ]つになりました。	むすこ は むっつ に なりました	
\\	息子[むすこ]は
\\	になりました。			
\\	六	六[ろく]	ろく	
\\	彼には子供が六人います。	彼[かれ]には 子供[こども]が 六[ろく] 人[にん]います。	かれ に は こども が ろくにん います	
\\	彼[かれ]には 子供[こども]が
\\	人[にん]います。			
\\	七	七[なな]	なな	
\\	バナナが七本あります。	バナナが 七[なな] 本[ほん]あります。	ばなな が ななほん あります	
\\	バナナが
\\	本[ほん]あります。			
\\	七つ	七[なな]つ	ななつ	
\\	この子は今年七つになります。	この 子[こ]は 今年[ことし] 七[なな]つになります。	この こ は ことし ななつ に なります	
\\	この 子[こ]は 今年[ことし]
\\	になります。			
\\	八	八[はち]	はち	
\\	りんごを八個ください。	りんごを 八[はち] 個[こ]ください。	りんご を はちこ ください	
\\	りんごを
\\	個[こ]ください。			
\\	八つ	八[やっ]つ	やっつ	
\\	あの家には時計が八つあります。	あの 家[いえ]には 時計[とけい]が 八[やっ]つあります。	あの いえ に は とけい が やっつ あります	
\\	あの 家[いえ]には 時計[とけい]が
\\	あります。			
\\	なる	なる	なる	
\\	彼は医者になりました。	彼[かれ]は 医者[いしゃ]になりました。	かれ は いしゃ に なりました	
\\	彼[かれ]は 医者[いしゃ]に
\\	九	九[きゅう]	きゅう	
\\	野球は九人で1チームです。	野球[やきゅう]は 九[きゅう] 人[にん]で 1[ひと]チームです。	やきゅう は きゅうにん で ひとちーむ です	
\\	野球[やきゅう]は
\\	人[にん]で 1[ひと]チームです。			
\\	九つ	九[ここの]つ	ここのつ	
\\	娘は明日九つになります。	娘[むすめ]は 明日[あす] 九[ここの]つになります。	むすめ は あす ここのつ に なります	
\\	娘[むすめ]は 明日[あす]
\\	になります。			
\\	十	十[じゅう]	じゅう	
\\	その子は指で十数えました。	その 子[こ]は 指[ゆび]で 十[じゅう] 数[かぞ]えました。	その こ は ゆび で じゅう かぞえました	
\\	その 子[こ]は 指[ゆび]で
\\	数[かぞ]えました。			
\\	千	千[せん]	せん	
\\	千円貸してください。	千[せん] 円貸[えん か]してください。	せんえん かして ください	
\\	円貸[えん か]してください。			
\\	万	万[まん]	まん	
\\	この靴は1万円です。	この 靴[くつ]は 1[いち] 万[まん] 円[えん]です。	この くつ は いちまんえん です	
\\	この 靴[くつ]は 1[いち]
\\	円[えん]です。			
\\	円	円[えん]	えん	
\\	そこに大きな円を描いて。	そこに 大[おお]きな 円[えん]を 描[か]いて。	そこ に おおき な えん を かいて	
\\	そこに 大[おお]きな
\\	を 描[か]いて。			
\\	円	円[えん]	えん	
\\	カレーライスは700円です。	カレーライスは 700[ななひゃく] 円[えん]です。	かれーらいす は ななひゃくえん です	
\\	カレーライスは 700[ななひゃく]
\\	です。			
\\	する	する	する	
\\	友達と一緒に宿題をした。	友達[ともだち]と 一緒[いっしょ]に 宿題[しゅくだい]をした。	ともだち と いっしょ に しゅくだい を した	
\\	友達[ともだち]と 一緒[いっしょ]に 宿題[しゅくだい]を
\\	六日	六日[むいか]	むいか	
\\	六日前に日本に帰ってきました。	六日[むいか] 前[まえ]に 日本[にほん]に 帰[かえ]ってきました。	むいかまえ に にほん に かえって きました	
\\	前[まえ]に 日本[にほん]に 帰[かえ]ってきました。			
\\	三日	三日[みっか]	みっか	
\\	手紙が届くのに三日かかりました。	手紙[てがみ]が 届[とど]くのに 三日[みっか]かかりました。	てがみ が とどく の に みっか かかりました	
\\	手紙[てがみ]が 届[とど]くのに
\\	かかりました。			
\\	五日	五日[いつか]	いつか	
\\	五月五日は祝日です	五月[ごがつ] 五日[いつか]は 祝日[しゅくじつ]です	ごがつ いつか は しゅくじつ です	
\\	五月[ごがつ]
\\	は 祝日[しゅくじつ]です			
\\	八日	八日[ようか]	ようか	
\\	八日からイギリスに行きます。	八日[ようか]からイギリスに 行[い]きます。	ようか から いぎりす に いきます	
\\	からイギリスに 行[い]きます。			
\\	二十日	二十日[はつか]	はつか	
\\	来月の二十日は弟の誕生日です。	来月[らいげつ]の 二十日[はつか]は 弟[おとうと]の 誕生日[たんじょうび]です。	らいげつ の はつか は おとうと の たんじょうび です	
\\	来月[らいげつ]の
\\	は 弟[おとうと]の 誕生日[たんじょうび]です。			
\\	二日	二日[ふつか]	ふつか	
\\	私は二日待った。	私[わたし]は 二日[ふつか] 待[ま]った。	わたし は ふつか まった	
\\	私[わたし]は
\\	待[ま]った。			
\\	九日	九日[ここのか]	ここのか	
\\	九日に荷物が届きます。	九日[ここのか]に 荷物[にもつ]が 届[とど]きます。	ここのか に にもつ が とどきます	
\\	に 荷物[にもつ]が 届[とど]きます。			
\\	ところ	ところ	ところ	
\\	私は友達のところに泊まった。	私[わたし]は 友達[ともだち]のところに 泊[と]まった。	わたし は ともだち の ところ に とまった	
\\	私[わたし]は 友達[ともだち]の
\\	に 泊[と]まった。			
\\	一日	一日[ついたち]	ついたち	
\\	来月の一日は空いていますか。	来月[らいげつ]の 一日[ついたち]は 空[あ]いていますか。	らいげつ の ついたち は あいています か	
\\	来月[らいげつ]の
\\	は 空[あ]いていますか。			
\\	十日	十日[とおか]	とおか	
\\	十日後に帰ります。	十日[とおか] 後[ご]に 帰[かえ]ります。	とおかご に かえります	
\\	後[ご]に 帰[かえ]ります。			
\\	七日	七日[なのか]	なのか	
\\	先月の七日に孫が生まれました。	先月[せんげつ]の 七日[なのか]に 孫[まご]が 生[う]まれました。	せんげつ の なのか に まご が うまれました	
\\	先月[せんげつ]の
\\	に 孫[まご]が 生[う]まれました。			
\\	やる	やる	やる	
\\	一緒に宿題をやろう。	一緒[いっしょ]に 宿題[しゅくだい]をやろう。	いっしょ に しゅくだい を やろう	
\\	一緒[いっしょ]に 宿題[しゅくだい]を
\\	分かる	分[わ]かる	わかる	
\\	質問の意味は分かりましたか。	質問[しつもん]の 意味[いみ]は 分[わ]かりましたか。	しつもん の いみ は わかりました か	
\\	質問[しつもん]の 意味[いみ]は
\\	か。			
\\	そう	そう	そう	
\\	私もそう思います。	私[わたし]もそう 思[おも]います。	わたし も そう おもいます	
\\	私[わたし]も
\\	思[おも]います。			
\\	何	何[なに]	なに	
\\	夕食には何を食べたいですか。	夕食[ゆうしょく]には 何[なに]を 食[た]べたいですか。	ゆうしょく に は なに を たべたい です か	
\\	夕食[ゆうしょく]には
\\	を 食[た]べたいですか。			
\\	先	先[さき]	さき	
\\	お先にどうぞ。	お 先[さき]にどうぞ。	おさきに どうぞ	
\\	お
\\	にどうぞ。			
\\	今年	今年[ことし]	ことし	
\\	今年はイタリアに旅行したい。	今年[ことし]はイタリアに 旅行[りょこう]したい。	ことし は いたりあ に りょこう したい	
\\	はイタリアに 旅行[りょこう]したい。			
\\	今	今[いま]	いま	
\\	彼は今、勉強しています。	彼[かれ]は 今[いま]、 勉強[べんきょう]しています。	かれ は いま べんきょう して います	
\\	彼[かれ]は
\\	、 勉強[べんきょう]しています。			
\\	今日	今日[きょう]	きょう	
\\	今日は仕事がありません。	今日[きょう]は 仕事[しごと]がありません。	きょう は しごと が ありません	
\\	は 仕事[しごと]がありません。			
\\	今月	今月[こんげつ]	こんげつ	
\\	今月はとても忙しい。	今月[こんげつ]はとても 忙[いそが]しい。	こんげつ は とても いそがしい	
\\	はとても 忙[いそが]しい。			
\\	今週	今週[こんしゅう]	こんしゅう	
\\	今週は日本語のテストがあります。	今週[こんしゅう]は 日本語[にほんご]のテストがあります。	こんしゅう は にほんご の てすと が あります	
\\	は 日本語[にほんご]のテストがあります。			
\\	もう	もう	もう	
\\	彼はもう帰りました。	彼[かれ]はもう 帰[かえ]りました。	かれ は もう かえりました 。	
\\	彼[かれ]は
\\	帰[かえ]りました。			
\\	上げる	上[あ]げる	あげる	
\\	彼は荷物をあみだなに上げた。	彼[かれ]は 荷物[にもつ]をあみだなに 上[あ]げた。	かれ は にもつ を あみだな に あげた	
\\	彼[かれ]は 荷物[にもつ]をあみだなに
\\	よく	よく	よく	
\\	彼女はよく旅行に行きます。	彼女[かのじょ]はよく 旅行[りょこう]に 行[い]きます。	かのじょ は よく りょこう に いきます	
\\	彼女[かのじょ]は
\\	旅行[りょこう]に 行[い]きます。			
\\	上る	上[のぼ]る	のぼる	
\\	猫が屋根に上っている。	猫[ねこ]が 屋根[やね]に 上[のぼ]っている。	ねこ が やね に のぼって いる	
\\	猫[ねこ]が 屋根[やね]に
\\	上	上[うえ]	うえ	
\\	上を向いて。	上[うえ]を 向[む]いて。	うえ を むいて	
\\	を 向[む]いて。			
\\	下	下[した]	した	
\\	財布は机の下にあった。	財布[さいふ]は 机[つくえ]の 下[した]にあった。	さいふ は つくえ の した に あった	
\\	財布[さいふ]は 机[つくえ]の
\\	にあった。			
\\	人	人[ひと]	ひと	
\\	彼は優しい人です。	彼[かれ]は 優[やさ]しい 人[ひと]です。	かれ は やさしい ひと です	
\\	彼[かれ]は 優[やさ]しい
\\	です。			
\\	一人	一人[ひとり]	ひとり	
\\	そこには私一人しかいなかった。	そこには 私[わたし] 一人[ひとり]しかいなかった。	そこ に は わたし ひとり しか いなかった	
\\	そこには 私[わたし]
\\	しかいなかった。			
\\	どう	どう	どう	
\\	あなたはどう思いますか。	あなたはどう 思[おも]いますか。	あなた は どう おもいます か	
\\	あなたは
\\	思[おも]いますか。			
\\	入れる	入[い]れる	いれる	
\\	彼はかばんに手帳を入れた。	彼[かれ]はかばんに 手帳[てちょう]を 入[い]れた。	かれ は かばん に てちょう を いれた	
\\	彼[かれ]はかばんに 手帳[てちょう]を
\\	入る	入[はい]る	はいる	
\\	寒いので中に入ってください。	寒[さむ]いので 中[なか]に 入[はい]ってください。	さむい の で なか に はいって ください	
\\	寒[さむ]いので 中[なか]に
\\	ください。			
\\	出る	出[で]る	でる	
\\	今朝は早く家を出ました。	今朝[けさ]は 早[はや]く 家[いえ]を 出[で]ました。	けさ は はやく いえ を でました	
\\	今朝[けさ]は 早[はや]く 家[いえ]を
\\	出来る	出来[でき]る	できる	
\\	彼女はイタリア語が出来ます。	彼女[かのじょ]はイタリア 語[ご]が 出来[でき]ます。	かのじょ は いたりあご が できます	
\\	彼女[かのじょ]はイタリア 語[ご]が
\\	出す	出[だ]す	だす	
\\	彼はかばんから教科書を出した。	彼[かれ]はかばんから 教科書[きょうかしょ]を 出[だ]した。	かれ は かばん から きょうかしょ を だした	
\\	彼[かれ]はかばんから 教科書[きょうかしょ]を
\\	休む	休[やす]む	やすむ	
\\	明日、会社を休みます。	明日[あした]、 会社[かいしゃ]を 休[やす]みます。	あした かいしゃ を やすみます	
\\	明日[あした]、 会社[かいしゃ]を
\\	体	体[からだ]	からだ	
\\	私は体が丈夫だ。	私[わたし]は 体[からだ]が 丈夫[じょうぶ]だ。	わたし は からだ が じょうぶ だ	
\\	私[わたし]は
\\	が 丈夫[じょうぶ]だ。			
\\	どこ	どこ	どこ	
\\	あの本をどこに置きましたか。	あの 本[ほん]をどこに 置[お]きましたか。	あの ほん を どこ に おきました か	
\\	あの 本[ほん]を
\\	に 置[お]きましたか。			
\\	上手	上手[じょうず]	じょうず	
\\	妹は歌が上手です。	妹[いもうと]は 歌[うた]が 上手[じょうず]です。	いもうと は うた が じょうず です	
\\	妹[いもうと]は 歌[うた]が
\\	です。			
\\	あげる	あげる	あげる	
\\	この本、あなたにあげます。	この 本[ほん]、あなたにあげます。	この ほん あなた に あげます	
\\	この 本[ほん]、あなたに
\\	友達	友達[ともだち]	ともだち	
\\	私には友達がたくさんいます。	私[わたし]には 友達[ともだち]がたくさんいます。	わたし に は ともだち が たくさん います	
\\	私[わたし]には
\\	がたくさんいます。			
\\	元気	元気[げんき]	げんき	
\\	おかげさまで元気です。	おかげさまで 元気[げんき]です。	おかげさま で げんき です	
\\	おかげさまで
\\	です。			
\\	こう	こう	こう	
\\	こう小さい字は読めない。	こう 小[ちい]さい 字[じ]は 読[よ]めない。	こう ちいさい じ は よめない	
\\	小[ちい]さい 字[じ]は 読[よ]めない。			
\\	くれる	くれる	くれる	
\\	友達が誕生日プレゼントをくれた。	友達[ともだち]が 誕生日[たんじょうび]プレゼントをくれた。	ともだち が たんじょうび ぷれぜんと を くれた	
\\	友達[ともだち]が 誕生日[たんじょうび]プレゼントを
\\	低い	低[ひく]い	ひくい	
\\	彼は背が低い。	彼[かれ]は 背[せ]が 低[ひく]い。	かれ は せ が ひくい	
\\	彼[かれ]は 背[せ]が
\\	前	前[まえ]	まえ	
\\	その店の前で会いましょう。	その 店[みせ]の 前[まえ]で 会[あ]いましょう。	その みせ の まえ で あいましょう	
\\	その 店[みせ]の
\\	で 会[あ]いましょう。			
\\	午後	午後[ごご]	ごご	
\\	明日の午後、お客様が来る。	明日[あす]の 午後[ごご]、お 客様[きゃくさま]が 来[く]る。	あす の ごご おきゃくさま が くる	
\\	明日[あす]の
\\	、お 客様[きゃくさま]が 来[く]る。			
\\	午前	午前[ごぜん]	ごぜん	
\\	午前9時のニュースです。	午前[ごぜん] 9時[くじ]のニュースです。	ごぜん くじ の にゅーす です	
\\	9時[くじ]のニュースです。			
\\	かなり	かなり	かなり	
\\	彼はかなり英語が上手です。	彼[かれ]はかなり 英語[えいご]が 上手[じょうず]です。	かれ は かなり えいご が じょうず です	
\\	彼[かれ]は
\\	英語[えいご]が 上手[じょうず]です。			
\\	今晩	今晩[こんばん]	こんばん	
\\	今晩のパーティーは何時からですか。	今晩[こんばん]のパーティーは 何時[なんじ]からですか。	こんばん の ぱーてぃー は なんじ から です か	
\\	のパーティーは 何時[なんじ]からですか。			
\\	ご飯	ご 飯[はん]	ごはん	
\\	私はパンよりご飯が好きだ。	私[わたし]はパンよりご 飯[はん]が 好[す]きだ。	わたし は ぱん より ごはん が すき だ	
\\	私[わたし]はパンより
\\	が 好[す]きだ。			
\\	もっと	もっと	もっと	
\\	もっと近くに来てください。	もっと 近[ちか]くに 来[き]てください。	もっと ちかく に きて ください	
\\	近[ちか]くに 来[き]てください。			
\\	そこ	そこ	そこ	
\\	そこに座ってください。	そこに 座[すわ]ってください。	そこ に すわって ください	
\\	に 座[すわ]ってください。			
\\	会う	会[あ]う	あう	
\\	また会いましょう。	また 会[あ]いましょう。	また あいましょう	
\\	また
\\	仕事	仕事[しごと]	しごと	
\\	3月は仕事が忙しい。	3月[さんがつ]は 仕事[しごと]が 忙[いそが]しい。	さんがつ は しごと が いそがしい	
\\	3月[さんがつ]は
\\	が 忙[いそが]しい。			
\\	他	他[ほか]	ほか	
\\	他に方法がありません。	他[ほか]に 方法[ほうほう]がありません。	ほか に ほうほう が ありません	
\\	に 方法[ほうほう]がありません。			
\\	ここ	ここ	ここ	
\\	ここに本があります。	ここに 本[ほん]があります。	ここ に ほん が あります	
\\	に 本[ほん]があります。			
\\	もらう	もらう	もらう	
\\	彼女からプレゼントをもらいました。	彼女[かのじょ]からプレゼントをもらいました。	かのじょ から ぷれぜんと を もらいました	
\\	彼女[かのじょ]からプレゼントを
\\	全部	全部[ぜんぶ]	ぜんぶ	
\\	それ、全部ください。	それ、 全部[ぜんぶ]ください。	それ ぜんぶ ください	
\\	それ、
\\	ください。			
\\	とても	とても	とても	
\\	この本はとてもおもしろい。	この 本[ほん]はとてもおもしろい。	この ほん は とても おもしろい	
\\	この 本[ほん]は
\\	おもしろい。			
\\	作る	作[つく]る	つくる	
\\	今、朝ご飯を作っています。	今[いま]、 朝[あさ]ご 飯[はん]を 作[つく]っています。	いま あさごはん を つくって います	
\\	今[いま]、 朝[あさ]ご 飯[はん]を
\\	使う	使[つか]う	つかう	
\\	このパソコンを使ってください。	このパソコンを 使[つか]ってください。	この ぱそこん を つかって ください	
\\	このパソコンを
\\	ください。			
\\	冬	冬[ふゆ]	ふゆ	
\\	カナダの冬はとても寒いです。	カナダの 冬[ふゆ]はとても 寒[さむ]いです。	かなだ の ふゆ は とても さむい です	
\\	カナダの
\\	はとても 寒[さむ]いです。			
\\	みんな	みんな	みんな	
\\	みんなにお菓子をあげましょう。	みんなにお 菓子[かし]をあげましょう。	みんな に おかし を あげましょう	
\\	にお 菓子[かし]をあげましょう。			
\\	いつも	いつも	いつも	
\\	彼女はいつも元気だ。	彼女[かのじょ]はいつも 元気[げんき]だ。	かのじょ は いつも げんき だ	
\\	彼女[かのじょ]は
\\	元気[げんき]だ。			
\\	一番	一番[いちばん]	いちばん	
\\	彼はクラスで一番背が高い。	彼[かれ]はクラスで 一番[いちばん] 背[せ]が 高[たか]い。	かれ は くらす で いちばん せ が たかい	
\\	彼[かれ]はクラスで
\\	背[せ]が 高[たか]い。			
\\	住む	住[す]む	すむ	
\\	彼は会社の近くに住んでいる。	彼[かれ]は 会社[かいしゃ]の 近[ちか]くに 住[す]んでいる。	かれ は かいしゃ の ちかく に すんで いる	
\\	彼[かれ]は 会社[かいしゃ]の 近[ちか]くに
\\	いつ	いつ	いつ	
\\	彼女はいつ来ますか。	彼女[かのじょ]はいつ 来[き]ますか。	かのじょ は いつ きます か	
\\	彼女[かのじょ]は
\\	来[き]ますか。			
\\	取る	取[と]る	とる	
\\	テストでいい点を取った。	テストでいい 点[てん]を 取[と]った。	てすと で いい てん を とった	
\\	テストでいい 点[てん]を
\\	どちら	どちら	どちら	
\\	肉と魚とどちらが好きですか。	肉[にく]と 魚[さかな]とどちらが 好[す]きですか。	にく と さかな と どちら が すき です か	
\\	肉[にく]と 魚[さかな]と
\\	が 好[す]きですか。			
\\	先生	先生[せんせい]	せんせい	
\\	私は日本語の先生になりたいです。	私[わたし]は 日本語[にほんご]の 先生[せんせい]になりたいです。	わたし は にほんご の せんせい に なりたい です	
\\	私[わたし]は 日本語[にほんご]の
\\	になりたいです。			
\\	勉強	勉強[べんきょう]	べんきょう	
\\	私は日本語を勉強しています。	私[わたし]は 日本語[にほんご]を 勉強[べんきょう]しています。	わたし は にほんご を べんきょう して います	
\\	私[わたし]は 日本語[にほんご]を
\\	しています。			
\\	どれ	どれ	どれ	
\\	この中でどれが好きですか。	この 中[なか]でどれが 好[す]きですか。	この なか で どれ が すき です か	
\\	この 中[なか]で
\\	が 好[す]きですか。			
\\	勝つ	勝[か]つ	かつ	
\\	今日はヤンキースが勝った。	今日[きょう]はヤンキースが 勝[か]った。	きょう は やんきーす が かった	
\\	今日[きょう]はヤンキースが
\\	ドア	ドア	ドア	
\\	ドアを開けてください。	ドアを 開[あ]けてください。	どあ を あけて ください	
\\	を 開[あ]けてください。			
\\	切る	切[き]る	きる	
\\	この紙を半分に切ってください。	この 紙[かみ]を 半分[はんぶん]に 切[き]ってください。	この かみ を はんぶん に きって ください	
\\	この 紙[かみ]を 半分[はんぶん]に
\\	ください。			
\\	乗る	乗[の]る	のる	
\\	駅からはタクシーに乗ってください。	駅[えき]からはタクシーに 乗[の]ってください。	えき からは たくしー に のって ください	
\\	駅[えき]からはタクシーに
\\	ください。			
\\	しゃべる	しゃべる	しゃべる	
\\	彼女はよくしゃべるね。	彼女[かのじょ]はよくしゃべるね。	かのじょ は よく しゃべる ね	
\\	彼女[かのじょ]はよく
\\	ね。			
\\	動く	動[うご]く	うごく	
\\	動かないで。	動[うご]かないで 。	うごかない で	
\\	働く	働[はたら]く	はたらく	
\\	姉は銀行で働いています。	姉[あね]は 銀行[ぎんこう]で 働[はたら]いています。	あね は ぎんこう で はたらいています。	
\\	姉[あね]は 銀行[ぎんこう]で
\\	テーブル	テーブル	テーブル	
\\	新しいテーブルを買いました。	新[あたら]しいテーブルを 買[か]いました。	あたらしい てーぶる を かいました	
\\	新[あたら]しい
\\	を 買[か]いました。			
\\	去年	去年[きょねん]	きょねん	
\\	私は去年フランスへ行った。	私[わたし]は 去年[きょねん]フランスへ 行[い]った。	わたし は きょねん ふらんす へ いった	
\\	私[わたし]は
\\	フランスへ 行[い]った。			
\\	ゲーム	ゲーム	ゲーム	
\\	私たちはビデオゲームをした。	私[わたし]たちはビデオゲームをした。	わたしたち は びでおげーむ を した	
\\	私[わたし]たちはビデオ
\\	をした。			
\\	どうして	どうして	どうして	
\\	どうして泣いているの。	どうして 泣[な]いているの。	どうして ないて いる の	
\\	泣[な]いているの。			
\\	お父さん	お 父[とう]さん	おとうさん	
\\	お父さんは会社員です。	お 父[とう]さんは 会社員[かいしゃいん]です。	おとうさん は かいしゃいん です	
\\	は 会社員[かいしゃいん]です。			
\\	お母さん	お 母[かあ]さん	おかあさん	
\\	お母さんによろしくお伝えください。	お 母[かあ]さんによろしくお 伝[つた]えください。	おかあさん に よろしく おつたえ ください	
\\	によろしくお 伝[つた]えください。			
\\	お姉さん	お 姉[ねえ]さん	おねえさん	
\\	昨日、あなたのお姉さんに会ったよ。	昨日[きのう]、あなたのお 姉[ねえ]さんに 会[あ]ったよ。	きのう あなた の おねえさん に あった よ	
\\	昨日[きのう]、あなたの
\\	に 会[あ]ったよ。			
\\	おなか	おなか	おなか	
\\	おなかが空きました。	おなかが 空[す]きました。	おなか が すきました	
\\	が 空[す]きました。			
\\	兄	兄[あに]	あに	
\\	兄は水泳が得意です。	兄[あに]は 水泳[すいえい]が 得意[とくい]です。	あに は すいえい が とくい です	
\\	は 水泳[すいえい]が 得意[とくい]です。			
\\	お兄さん	お 兄[にい]さん	おにいさん	
\\	あなたのお兄さんは何歳?	あなたのお 兄[にい]さんは 何歳?[なんさい]	あなた の おにいさん は なんさい	
\\	あなたの
\\	は 何歳?[なんさい]			
\\	ミーティング	ミーティング	ミーティング	
\\	朝9時からミーティングが始まった。	朝9時[あさ 
\\	じ]からミーティングが 始[はじ]まった。	あさ 
\\	じ から みーてぃんぐ が はじまった	
\\	朝9時[あさ 
\\	じ]から
\\	が 始[はじ]まった。			
\\	ご主人	ご 主人[しゅじん]	ごしゅじん	
\\	ご主人はお元気ですか。	ご 主人[しゅじん]はお 元気[げんき]ですか。	ごしゅじん は おげんき です か	
\\	はお 元気[げんき]ですか。			
\\	別	別[べつ]	べつ	
\\	別の本も見せてください。	別[べつ]の 本[ほん]も 見[み]せてください。	べつ の ほん も みせて ください	
\\	の 本[ほん]も 見[み]せてください。			
\\	つける	つける	つける	
\\	電気をつけてください。	電気[でんき]をつけてください。	でんき を つけて ください	
\\	電気[でんき]を
\\	ください。			
\\	一杯	一杯[いっぱい]	いっぱい	
\\	プールは人で一杯です。	プールは 人[ひと]で 一杯[いっぱい]です。	ぷーる は ひと で いっぱい です	
\\	プールは 人[ひと]で
\\	です。			
\\	お願い	お 願[ねが]い	おねがい	
\\	お願いがあります。	お 願[ねが]いがあります。	おねがい が あります	
\\	があります。			
\\	あれ	あれ	あれ	
\\	あれは何ですか。	あれは 何[なん]ですか。	あれ は なん です か	
\\	は 何[なん]ですか。			
\\	お手洗い	お 手洗[てあら]い	おてあらい	
\\	お手洗いはどこですか。	お 手洗[てあら]いはどこですか。	おてあらい は どこ です か	
\\	はどこですか。			
\\	冷たい	冷[つめ]たい	つめたい	
\\	冷たい飲み物をください。	冷[つめ]たい 飲[の]み 物[もの]をください。	つめたい のみもの を ください	
\\	飲[の]み 物[もの]をください。			
\\	まずい	まずい	まずい	
\\	ここの料理はまずい。	ここの 料理[りょうり]はまずい。	ここ の りょうり は まずい	
\\	ここの 料理[りょうり]は
\\	一緒に	一緒[いっしょ]に	いっしょに	
\\	一緒に帰りましょう。	一緒[いっしょ]に 帰[かえ]りましょう。	いっしょに かえりましょう	
\\	帰[かえ]りましょう。			
\\	そば	そば	そば	
\\	そばにいて下さい。	そばにいて 下[くだ]さい。	そば に いて ください	
\\	にいて 下[くだ]さい。			
\\	こっち	こっち	こっち	
\\	こっちに来て下さい。	こっちに 来[き]て 下[くだ]さい。	こっち に きて ください	
\\	に 来[き]て 下[くだ]さい。			
\\	優しい	優[やさ]しい	やさしい	
\\	彼はとても優しい人です。	彼[かれ]はとても 優[やさ]しい 人[ひと]です。	かれ は とても やさしい ひと です	
\\	彼[かれ]はとても
\\	人[ひと]です。			
\\	よう	よう	よう	
\\	彼女は眠いようです。	彼女[かのじょ]は 眠[ねむ]いようです。	かのじょ は ねむい よう です	
\\	彼女[かのじょ]は 眠[ねむ]い
\\	です。			
\\	あそこ	あそこ	あそこ	
\\	あそこにバス停があります。	あそこにバス 停[てい]があります。	あそこ に ばすてい が あります	
\\	にバス 停[てい]があります。			
\\	おいしい	おいしい	おいしい	
\\	このケーキはおいしいね。	このケーキはおいしいね。	この けーき は おいしい ね	
\\	このケーキは
\\	ね。			
\\	僕	僕[ぼく]	ぼく	
\\	僕は学生です。	僕[ぼく]は 学生[がくせい]です。	ぼく は がくせい です	
\\	は 学生[がくせい]です。			
\\	なる	なる	なる	
\\	柿の木に実がたくさんなっています。	柿[かき]の 木[き]に 実[み]がたくさんなっています。	かきのき に み が たくさん なって います	
\\	柿[かき]の 木[き]に 実[み]がたくさん
\\	ため	ため	ため	
\\	これは医者のためのサイトです。	これは 医者[いしゃ]のためのサイトです。	これ は いしゃ の ため の さいと です	
\\	これは 医者[いしゃ]の
\\	のサイトです。			
\\	より	より	より	
\\	これから、より一層努力します。	これから、より 一層努力[いっそう どりょく]します。	これから より いっそう どりょく します	
\\	これから、
\\	一層努力[いっそう どりょく]します。			
\\	七	七[しち]	しち	
\\	そのグループのメンバーは全部で七人だ。	そのグループのメンバーは 全部[ぜんぶ]で 七[しち] 人[にん]だ。	その ぐるーぷ の めんばー は ぜんぶ で しちにん だ	
\\	そのグループのメンバーは 全部[ぜんぶ]で
\\	人[にん]だ。			
\\	九	九[く]	く	
\\	私は九月に行く予定です。	私[わたし]は 九[く] 月[がつ]に 行[い]く 予定[よてい]です。	わたし は くがつ に いく よてい です	
\\	私[わたし]は
\\	月[がつ]に 行[い]く 予定[よてい]です。			
\\	億	億[おく]	おく	
\\	世界人口は65億人だ。	世界人口[せかい じんこう]は 65[ろくじゅうご] 億[おく] 人[にん]だ。	せかい じんこう は ろくじゅうごおくにん だ	
\\	世界人口[せかい じんこう]は 65[ろくじゅうご]
\\	人[にん]だ。			
\\	ドル	ドル	ドル	
\\	この服は300ドルしました。	この 服[ふく]は 300[さんびゃく]ドルしました。	この ふく は さんびゃくどる しました	
\\	この 服[ふく]は 300[さんびゃく]
\\	しました。			
\\	システム	システム	システム	
\\	全てのシステムが停止した。	全[すべ]てのシステムが 停止[ていし]した。	すべて の しすてむ が ていし した	
\\	全[すべ]ての
\\	が 停止[ていし]した。			
\\	十分	十分[じゅうぶん]	じゅうぶん	
\\	お金はまだ十分あります。	お 金[かね]はまだ 十分[じゅうぶん]あります。	おかね は まだ じゅうぶん あります	
\\	お 金[かね]はまだ
\\	あります。			
\\	分	分[ぶん]	ぶん	
\\	このケーキはあなたの分です。	このケーキはあなたの 分[ぶん]です。	この けーき は あなた の ぶん です	
\\	このケーキはあなたの
\\	です。			
\\	分ける	分[わ]ける	わける	
\\	お菓子をみんなで分けました。	お 菓子[かし]をみんなで 分[わ]けました。	おかし を みんな で わけました	
\\	お 菓子[かし]をみんなで
\\	分かれる	分[わ]かれる	わかれる	
\\	グループの中で意見が分かれました。	グループの 中[なか]で 意見[いけん]が 分[わ]かれました。	ぐるーぷ の なか で いけん が わかれました	
\\	グループの 中[なか]で 意見[いけん]が
\\	コンピューター	コンピューター	コンピューター	
\\	新しいコンピューターを買った。	新[あたら]しいコンピューターを 買[か]った。	あたらしい こんぴゅーたー を かった	
\\	新[あたら]しい
\\	を 買[か]った。			
\\	何か	何[なに]か	なにか	
\\	道に何か落ちています。	道[みち]に 何[なに]か 落[お]ちています。	みち に なにか おちて います	
\\	道[みち]に
\\	落[お]ちています。			
\\	何	何[なん]	なん	
\\	質問は何ですか。	質問[しつもん]は 何[なん]ですか。	しつもん は なん です か	
\\	質問[しつもん]は
\\	ですか。			
\\	先ず	先[ま]ず	まず	
\\	帰ったら先ず手を洗いましょう。	帰[かえ]ったら 先[ま]ず 手[て]を 洗[あら]いましょう。	かえったら まず て を あらいましょう 。	
\\	帰[かえ]ったら
\\	手[て]を 洗[あら]いましょう。			
\\	先月	先月[せんげつ]	せんげつ	
\\	先月、友達の結婚式があった。	先月[せんげつ]、 友達[ともだち]の 結婚式[けっこんしき]があった。	せんげつ ともだち の けっこんしき が あった	
\\	、 友達[ともだち]の 結婚式[けっこんしき]があった。			
\\	やる	やる	やる	
\\	犬にえさをやった。	犬[いぬ]にえさをやった。	いぬ に えさ を やった	
\\	犬[いぬ]にえさを
\\	先週	先週[せんしゅう]	せんしゅう	
\\	先週は海に行った。	先週[せんしゅう]は 海[うみ]に 行[い]った。	せんしゅう は うみ に いった	
\\	は 海[うみ]に 行[い]った。			
\\	今まで	今[いま]まで	いままで	
\\	今までどこにいたのですか。	今[いま]までどこにいたのですか。	いままで どこ に いた の です か	
\\	どこにいたのですか。			
\\	ロボット	ロボット	ロボット	
\\	彼はロボットの研究をしています。	彼[かれ]はロボットの 研究[けんきゅう]をしています。	かれ は ろぼっと の けんきゅう を しています 。	
\\	彼[かれ]は
\\	の 研究[けんきゅう]をしています。			
\\	ほとんど	ほとんど	ほとんど	
\\	お金がほとんどありません。	お 金[かね]がほとんどありません。	おかね が ほとんど ありません	
\\	お 金[かね]が
\\	ありません。			
\\	中	中[なか]	なか	
\\	財布は引き出しの中にあります。	財布[さいふ]は 引[ひ]き 出[だ]しの 中[なか]にあります。	さいふ は ひきだし の なか に あります	
\\	財布[さいふ]は 引[ひ]き 出[だ]しの
\\	にあります。			
\\	パソコン	パソコン	パソコン	
\\	彼はパソコンを2台持っています。	彼[かれ]はパソコンを 2台持[にだい も]っています。	かれ は ぱそこん を にだい もって います	
\\	彼[かれ]は
\\	を 2台持[にだい も]っています。			
\\	上がる	上[あ]がる	あがる	
\\	私たちは2階に上がった。	私[わたし]たちは 2階[にかい]に 上[あ]がった。	わたしたち は にかい に あがった	
\\	私[わたし]たちは 2階[にかい]に
\\	もう	もう	もう	
\\	コーヒーをもう一杯ください。	コーヒーをもう 一杯[いっぱい]ください。	こーひー を もう いっぱい ください	
\\	コーヒーを
\\	一杯[いっぱい]ください。			
\\	上がる	上[あ]がる	あがる	
\\	彼は人前だと上がってしまう。	彼[かれ]は 人前[ひとまえ]だと 上[あ]がってしまう。	かれ は ひとまえ だ と あがって しまう	
\\	彼[かれ]は 人前[ひとまえ]だと
\\	上り	上[のぼ]り	のぼり	
\\	これは上り電車です。	これは 上[のぼ]り 電車[でんしゃ]です。	これ は のぼり でんしゃ です	
\\	これは
\\	電車[でんしゃ]です。			
\\	下げる	下[さ]げる	さげる	
\\	少し音量を下げてください。	少[すこ]し 音量[おんりょう]を 下[さ]げてください。	すこし おんりょう を さげて ください	
\\	少[すこ]し 音量[おんりょう]を
\\	ください。			
\\	プログラム	プログラム	プログラム	
\\	受付でプログラムを受け取った。	受付[うけつけ]でプログラムを 受[う]け 取[と]った。	うけつけ で ぷろぐらむ を うけとった	
\\	受付[うけつけ]で
\\	を 受[う]け 取[と]った。			
\\	下がる	下[さ]がる	さがる	
\\	やっと熱が下がった。	やっと 熱[ねつ]が 下[さ]がった。	やっと ねつ が さがった	
\\	やっと 熱[ねつ]が
\\	下りる	下[お]りる	おりる	
\\	そこの階段を下りてください。	そこの 階段[かいだん]を 下[お]りてください。	そこ の かいだん を おりて ください	
\\	そこの 階段[かいだん]を
\\	ください。			
\\	下ろす	下[お]ろす	おろす	
\\	棚からその箱を下ろしてください。	棚[たな]からその 箱[はこ]を 下[お]ろしてください。	たな から その はこ を おろして ください	
\\	棚[たな]からその 箱[はこ]を
\\	ください。			
\\	下り	下[くだ]り	くだり	
\\	もうすぐ下りの電車が発車します。	もうすぐ 下[くだ]りの 電車[でんしゃ]が 発車[はっしゃ]します。	もうすぐ くだり の でんしゃ が はっしゃ します	
\\	もうすぐ
\\	の 電車[でんしゃ]が 発車[はっしゃ]します。			
\\	よく	よく	よく	
\\	よく答えが分かりましたね。	よく 答[こた]えが 分[わ]かりましたね。	よく こたえ が わかりました ね	
\\	答[こた]えが 分[わ]かりましたね。			
\\	下る	下[くだ]る	くだる	
\\	小さな船が川を下っています。	小[ちい]さな 船[ふね]が 川[かわ]を 下[くだ]っています。	ちいさ な ふね が かわ を くだって います	
\\	小[ちい]さな 船[ふね]が 川[かわ]を
\\	一方	一方[いっぽう]	いっぽう	
\\	ここは一方通行です。	ここは 一方[いっぽう] 通行[つうこう]です。	ここ は いっぽう つうこう です	
\\	ここは
\\	通行[つうこう]です。			
\\	まま	まま	まま	
\\	電気がついたままですよ。	電気[でんき]がついたままですよ。	でんき が ついた まま です よ	
\\	電気[でんき]がついた
\\	ですよ。			
\\	二人	二人[ふたり]	ふたり	
\\	今日は妻と二人で食事をします。	今日[きょう]は 妻[つま]と 二人[ふたり]で 食事[しょくじ]をします。	きょう は つま と ふたり で しょくじ を します	
\\	今日[きょう]は 妻[つま]と
\\	で 食事[しょくじ]をします。			
\\	人々	人々[ひとびと]	ひとびと	
\\	あの村の人々はとても親切です。	あの 村[むら]の 人々[ひとびと]はとても 親切[しんせつ]です。	あの むら の ひとびと は とても しんせつ です	
\\	あの 村[むら]の
\\	はとても 親切[しんせつ]です。			
\\	一人で	一人[ひとり]で	ひとりで	
\\	今日は一人で映画を見ます。	今日[きょう]は 一人[ひとり]で 映画[えいが]を 見[み]ます。	きょう は ひとりで えいが を みます	
\\	今日[きょう]は
\\	映画[えいが]を 見[み]ます。			
\\	テレビ	テレビ	テレビ	
\\	私はテレビをあまり見ません。	私[わたし]はテレビをあまり 見[み]ません。	わたし は てれび を あまり みません	
\\	私[わたし]は
\\	をあまり 見[み]ません。			
\\	休日	休日[きゅうじつ]	きゅうじつ	
\\	休日は家でよくテレビを見ます。	休日[きゅうじつ]は 家[いえ]でよくテレビを 見[み]ます。	きゅうじつ は いえ で よく てれび を みます	
\\	は 家[いえ]でよくテレビを 見[み]ます。			
\\	休み	休[やす]み	やすみ	
\\	木曜日は仕事が休みです。	木曜日[もくようび]は 仕事[しごと]が 休[やす]みです。	もくようび は しごと が やすみ です	
\\	木曜日[もくようび]は 仕事[しごと]が
\\	です。			
\\	ソフト	ソフト	ソフト	
\\	このソフトで日本語を勉強することができます。	このソフトで 日本語[にほんご]を 勉強[べんきょう]することができます。	この そふと で にほんご を べんきょう する こと が できます	
\\	この
\\	で 日本語[にほんご]を 勉強[べんきょう]することができます。			
\\	力	力[ちから]	ちから	
\\	お相撲さんは、みんな力持ちだ。	お 相撲[すもう]さんは、みんな 力[ちから] 持[も]ちだ。	おすもうさん は みんな ちからもち だ	
\\	お 相撲[すもう]さんは、みんな
\\	持[も]ちだ。			
\\	協力	協力[きょうりょく]	きょうりょく	
\\	このプロジェクトにはみんなの協力が必要です。	このプロジェクトにはみんなの 協力[きょうりょく]が 必要[ひつよう]です。	この ぷろじぇくと に は みんな の きょうりょく が ひつよう です	
\\	このプロジェクトにはみんなの
\\	が 必要[ひつよう]です。			
\\	人口	人口[じんこう]	じんこう	
\\	その国の人口はどのくらいですか。	その 国[くに]の 人口[じんこう]はどのくらいですか。	その くに の じんこう は どの くらい です か	
\\	その 国[くに]の
\\	はどのくらいですか。			
\\	出口	出口[でぐち]	でぐち	
\\	出口はあそこです。	出口[でぐち]はあそこです。	でぐち は あそこ です	
\\	はあそこです。			
\\	ただ	ただ	ただ	
\\	この温泉はただです。	この 温泉[おんせん]はただです。	この おんせん は ただ です	
\\	この 温泉[おんせん]は
\\	です。			
\\	入り口	入[い]り 口[ぐち]	いりぐち	
\\	入り口は向こうです。	入[い]り 口[ぐち]は 向[む]こうです。	いりぐち は むこう です	
\\	は 向[む]こうです。			
\\	下手	下手[へた]	へた	
\\	私は歌が下手だ。	私[わたし]は 歌[うた]が 下手[へた]だ。	わたし は うた が へた だ	
\\	私[わたし]は 歌[うた]が
\\	だ。			
\\	これら	これら	これら	
\\	今日はこれらの問題について話し合います。	今日[きょう]はこれらの 問題[もんだい]について 話[はな]し 合[あ]います。	きょう は これら の もんだい に ついて はなしあいます	
\\	今日[きょう]は
\\	の 問題[もんだい]について 話[はな]し 合[あ]います。			
\\	いずれ	いずれ	いずれ	
\\	いずれまたお会いしましょう。	いずれまたお 会[あ]いしましょう。	いずれ また おあい しましょう	
\\	またお 会[あ]いしましょう。			
\\	あまり	あまり	あまり	
\\	このビールはあまり美味しくありません。	このビールはあまり 美味[おい]しくありません。	この びーる は あまり おいしく ありません	
\\	このビールは
\\	美味[おい]しくありません。			
\\	なお	なお	なお	
\\	なお、雨の場合は中止です。	なお、 雨[あめ]の 場合[ばあい]は 中止[ちゅうし]です。	なお あめ の ばあい は ちゅうし です	
\\	、 雨[あめ]の 場合[ばあい]は 中止[ちゅうし]です。			
\\	ほぼ	ほぼ	ほぼ	
\\	仕事がほぼ終わりました。	仕事[しごと]がほぼ 終[お]わりました。	しごと が ほぼ おわりました	
\\	仕事[しごと]が
\\	終[お]わりました。			
\\	サービス	サービス	サービス	
\\	この店はサービスがいい。	この 店[みせ]はサービスがいい。	この みせ は さーびす が いい 。	
\\	この 店[みせ]は
\\	がいい。			
\\	グループ	グループ	グループ	
\\	店に学生のグループが来た。	店[みせ]に 学生[がくせい]のグループが 来[き]た。	みせ に がくせい の ぐるーぷ が きた	
\\	店[みせ]に 学生[がくせい]の
\\	が 来[き]た。			
\\	ホテル	ホテル	ホテル	
\\	今ホテルに着きました。	今[いま]ホテルに 着[つ]きました。	いま ほてる に つきました	
\\	今[いま]
\\	に 着[つ]きました。			
\\	人気	人気[にんき]	にんき	
\\	このバンドはとても人気があるよ。	このバンドはとても 人気[にんき]があるよ。	この ばんど は とても にんき が ある よ	
\\	このバンドはとても
\\	があるよ。			
\\	まとめる	まとめる	まとめる	
\\	彼女は荷物をまとめて出て行った。	彼女[かのじょ]は 荷物[にもつ]をまとめて 出[で]て 行[い]った。	かのじょ は にもつ を まとめて でて いった	
\\	彼女[かのじょ]は 荷物[にもつ]を
\\	出[で]て 行[い]った。			
\\	やはり	やはり	やはり	
\\	彼はやはり遅刻しました。	彼[かれ]はやはり 遅刻[ちこく]しました。	かれ は やはり ちこく しました	
\\	彼[かれ]は
\\	遅刻[ちこく]しました。			
\\	はっきり	はっきり	はっきり	
\\	今日は山がはっきり見える。	今日[きょう]は 山[やま]がはっきり 見[み]える。	きょう は やま が はっきり みえる	
\\	今日[きょう]は 山[やま]が
\\	見[み]える。			
\\	一昨年	一昨年[おととし]	おととし	
\\	一昨年初めて京都へ旅行しました。	一昨年[おととし] 初[はじ]めて 京都[きょうと]へ 旅行[りょこう]しました。	おととし はじめて きょうと へ りょこう しました	
\\	初[はじ]めて 京都[きょうと]へ 旅行[りょこう]しました。			
\\	一昨日	一昨日[おととい]	おととい	
\\	一昨日彼から電話がありました。	一昨日[おととい] 彼[かれ]から 電話[でんわ]がありました。	おととい かれ から でんわ が ありました	
\\	彼[かれ]から 電話[でんわ]がありました。			
\\	つまり	つまり	つまり	
\\	つまり、あなたは何も知らないのですね。	つまり、あなたは 何[なに]も 知[し]らないのですね。	つまり あなた は なに も しらない の です ね	
\\	、あなたは 何[なに]も 知[し]らないのですね。			
\\	南	南[みなみ]	みなみ	
\\	私の家は町の南にあります。	私[わたし]の 家[いえ]は 町[まち]の 南[みなみ]にあります。	わたし の いえ は まち の みなみ に あります	
\\	私[わたし]の 家[いえ]は 町[まち]の
\\	にあります。			
\\	北	北[きた]	きた	
\\	ロシアは日本の北にあります。	ロシアは 日本[にっぽん]の 北[きた]にあります。	ろしあ は にっぽん の きた に あります	
\\	ロシアは 日本[にっぽん]の
\\	にあります。			
\\	ビル	ビル	ビル	
\\	私の会社はあのビルの8階です。	私[わたし]の 会社[かいしゃ]はあのビルの 8階[はちかい]です。	わたし の かいしゃ は あの びる の はちかい です	
\\	私[わたし]の 会社[かいしゃ]はあの
\\	の 8階[はちかい]です。			
\\	もちろん	もちろん	もちろん	
\\	もちろん一緒に行きます。	もちろん 一緒[いっしょ]に 行[い]きます。	もちろん いっしょ に いきます	
\\	一緒[いっしょ]に 行[い]きます。			
\\	この間	この 間[あいだ]	このあいだ	
\\	この間彼女に会った。	この 間[あいだ] 彼女[かのじょ]に 会[あ]った。	このあいだ かのじょ に あった	
\\	彼女[かのじょ]に 会[あ]った。			
\\	人間	人間[にんげん]	にんげん	
\\	人間の心は複雑です。	人間[にんげん]の 心[こころ]は 複雑[ふくざつ]です。	にんげん の こころ は ふくざつ です	
\\	の 心[こころ]は 複雑[ふくざつ]です。			
\\	かつて	かつて	かつて	
\\	かつて私が学生だった頃のことです。	かつて 私[わたし]が 学生[がくせい]だった 頃[ころ]のことです。	かつて わたし が がくせい だった ころ の こと です	
\\	私[わたし]が 学生[がくせい]だった 頃[ころ]のことです。			
\\	初めて	初[はじ]めて	はじめて	
\\	東京に来るのは初めてです。	東京[とうきょう]に 来[く]るのは 初[はじ]めてです。	とうきょう に くる の は はじめて です 。	
\\	東京[とうきょう]に 来[く]るのは
\\	です。			
\\	スポーツ	スポーツ	スポーツ	
\\	あなたは何かスポーツをしていますか。	あなたは 何[なに]かスポーツをしていますか。	あなた は なにか すぽーつ を して います か	
\\	あなたは 何[なに]か
\\	をしていますか。			
\\	初め	初[はじ]め	はじめ	
\\	初めは上手くできませんでした。	初[はじ]めは 上手[うま]くできませんでした。	はじめ は うまく できません でした 。	
\\	は 上手[うま]くできませんでした。			
\\	今後	今後[こんご]	こんご	
\\	今後ともよろしくお願いします。	今後[こんご]ともよろしくお 願[ねが]いします。	こんご とも よろしく おねがい します	
\\	ともよろしくお 願[ねが]いします。			
\\	なぜ	なぜ	なぜ	
\\	なぜ来なかったの。	なぜ 来[こ]なかったの。	なぜ こなかった の	
\\	来[こ]なかったの。			
\\	半分	半分[はんぶん]	はんぶん	
\\	お菓子を友達に半分あげた。	お 菓子[かし]を 友達[ともだち]に 半分[はんぶん]あげた。	おかし を ともだち に はんぶん あげた	
\\	お 菓子[かし]を 友達[ともだち]に
\\	あげた。			
\\	半年	半年[はんとし]	はんとし	
\\	日本に来て半年になります。	日本[にっぽん]に 来[き]て 半年[はんとし]になります。	にっぽん に きて はんとし に なります	
\\	日本[にっぽん]に 来[き]て
\\	になります。			
\\	そのまま	そのまま	そのまま	
\\	そのままお待ちください。	そのままお 待[ま]ちください。	そのまま おまち ください	
\\	お 待[ま]ちください。			
\\	半月	半月[はんつき]	はんつき	
\\	半月前に日本に来ました。	半月[はんつき] 前[まえ]に 日本[にほん]に 来[き]ました。	はんつきまえ に にほん に きました	
\\	前[まえ]に 日本[にほん]に 来[き]ました。			
\\	半日	半日[はんにち]	はんにち	
\\	今日は半日だけ仕事だ。	今日[きょう]は 半日[はんにち]だけ 仕事[しごと]だ。	きょう は はんにち だけ しごと だ	
\\	今日[きょう]は
\\	だけ 仕事[しごと]だ。			
\\	今朝	今朝[けさ]	けさ	
\\	今朝から頭が痛い。	今朝[けさ]から 頭[あたま]が 痛[いた]い。	けさ から あたま が いたい	
\\	から 頭[あたま]が 痛[いた]い。			
\\	もし	もし	もし	
\\	もし雨が降ったら、行きません。	もし 雨[あめ]が 降[ふ]ったら、 行[い]きません。	もし あめ が ふったら いきません	
\\	雨[あめ]が 降[ふ]ったら、 行[い]きません。			
\\	つもり	つもり	つもり	
\\	明日からタバコを止めるつもりです。	明日[あした]からタバコを 止[や]めるつもりです。	あした から たばこ を やめる つもり です	
\\	明日[あした]からタバコを 止[や]める
\\	です。			
\\	今夜	今夜[こんや]	こんや	
\\	今夜は月がとてもきれいです。	今夜[こんや]は 月[つき]がとてもきれいです。	こんや は つき が とても きれい です	
\\	は 月[つき]がとてもきれいです。			
\\	やっと	やっと	やっと	
\\	やっと仕事が終わりました。	やっと 仕事[しごと]が 終[お]わりました。	やっと しごと が おわりました	
\\	仕事[しごと]が 終[お]わりました。			
\\	ニュース	ニュース	ニュース	
\\	夜のニュースを見ましたか。	夜[よる]のニュースを 見[み]ましたか。	よる の にゅーす を みました か	
\\	夜[よる]の
\\	を 見[み]ましたか。			
\\	ずっと	ずっと	ずっと	
\\	父は休みの日はずっとテレビを見ている。	父[ちち]は 休[やす]みの 日[ひ]はずっとテレビを 見[み]ている。	ちち は やすみ の ひ は ずっと てれび を みて いる	
\\	父[ちち]は 休[やす]みの 日[ひ]は
\\	テレビを 見[み]ている。			
\\	ビデオ	ビデオ	ビデオ	
\\	私はその番組をビデオに撮った。	私[わたし]はその 番組[ばんぐみ]をビデオに 撮[と]った。	わたし は その ばんぐみ を びでお に とった	
\\	私[わたし]はその 番組[ばんぐみ]を
\\	に 撮[と]った。			
\\	ローマ字	ローマ 字[じ]	ろーまじ	
\\	ローマ字で名前を書いてください。	ローマ 字[じ]で 名前[なまえ]を 書[か]いてください。	ろーまじ で なまえ を かいて ください	
\\	で 名前[なまえ]を 書[か]いてください。			
\\	マンション	マンション	マンション	
\\	彼はマンションに住んでいます。	彼[かれ]はマンションに 住[す]んでいます。	かれ は まんしょん に すんで います	
\\	彼[かれ]は
\\	に 住[す]んでいます。			
\\	しばらく	しばらく	しばらく	
\\	そこでしばらく休んでいます。	そこでしばらく 休[やす]んでいます。	そこで しばらく やすんで います	
\\	そこで
\\	休[やす]んでいます。			
\\	会話	会話[かいわ]	かいわ	
\\	親子の会話は大切です。	親子[おやこ]の 会話[かいわ]は 大切[たいせつ]です。	おやこ の かいわ は たいせつ です	
\\	親子[おやこ]の
\\	は 大切[たいせつ]です。			
\\	ガス	ガス	ガス	
\\	地震でガスが止まった。	地震[じしん]でガスが 止[と]まった。	じしん で がす が とまった	
\\	地震[じしん]で
\\	が 止[と]まった。			
\\	会社	会社[かいしゃ]	かいしゃ	
\\	彼は小さな会社に勤めています。	彼[かれ]は 小[ちい]さな 会社[かいしゃ]に 勤[つと]めています。	かれ は ちいさ な かいしゃ に つとめて います	
\\	彼[かれ]は 小[ちい]さな
\\	に 勤[つと]めています。			
\\	仕方	仕方[しかた]	しかた	
\\	ファイルのダウンロードの仕方が分かりません。	ファイルのダウンロードの 仕方[しかた]が 分[わ]かりません。	ふぁいる の だうんろーど の しかた が わかりません	
\\	ファイルのダウンロードの
\\	が 分[わ]かりません。			
\\	うまい	うまい	うまい	
\\	彼は野球がうまい。	彼[かれ]は 野球[やきゅう]がうまい。	かれ は やきゅう が うまい	
\\	彼[かれ]は 野球[やきゅう]が
\\	事故	事故[じこ]	じこ	
\\	彼は事故で怪我をしました。	彼[かれ]は 事故[じこ]で 怪我[けが]をしました。	かれ は じこ で けが を しました	
\\	彼[かれ]は
\\	で 怪我[けが]をしました。			
\\	サラリーマン	サラリーマン	サラリーマン	
\\	父はサラリーマンです。	父[ちち]はサラリーマンです。	ちち は さらりーまん です	
\\	父[ちち]は
\\	です。			
\\	バス	バス	バス	
\\	バスで行こう。	バスで 行[い]こう。	ばす で いこう	
\\	で 行[い]こう。			
\\	クラス	クラス	クラス	
\\	この学校は1クラス30人です。	この 学校[がっこう]は1クラス30 人[にん]です。	この がっこう は 
\\	くらす 
\\	にん です 。	
\\	この 学校[がっこう]は1
\\	人[にん]です。			
\\	交通事故	交通事故[こうつうじこ]	こうつうじこ	
\\	彼は交通事故を起こした。	彼[かれ]は 交通事故[こうつうじこ]を 起[お]こした。	かれ は こうつうじこ を おこした	
\\	彼[かれ]は
\\	を 起[お]こした。			
\\	トラック	トラック	トラック	
\\	トラックを運転できますか。	トラックを 運転[うんてん]できますか。	とらっく を うんてん できます か	
\\	を 運転[うんてん]できますか。			
\\	他	他[た]	た	
\\	留学生はアジア人が多く、その他は3割です。	留学生[りゅうがくせい]はアジア 人[じん]が 多[おお]く、その 他[た]は 3割[さんわり]です。	りゅうがくせい は あじあじん が おおく そのた は さんわり です	
\\	留学生[りゅうがくせい]はアジア 人[じん]が 多[おお]く、その
\\	は 3割[さんわり]です。			
\\	パーティー	パーティー	パーティー	
\\	明日、うちでパーティーを開きます。	明日[あした]、うちでパーティーを 開[ひら]きます。	あした うち で ぱーてぃー を ひらきます	
\\	明日[あした]、うちで
\\	を 開[ひら]きます。			
\\	うまい	うまい	うまい	
\\	うまい寿司屋を見つけたよ。	うまい 寿司屋[すしや]を 見[み]つけたよ。	うまい すしや を みつけた よ	
\\	寿司屋[すしや]を 見[み]つけたよ。			
\\	今度	今度[こんど]	こんど	
\\	今度はどこに行きたいですか。	今度[こんど]はどこに 行[い]きたいですか。	こんど は どこ に いきたい です か	
\\	はどこに 行[い]きたいですか。			
\\	ガラス	ガラス	ガラス	
\\	クリスタルガラスの花瓶を買いました。	クリスタルガラスの 花瓶[かびん]を 買[か]いました。	くりすたるがらす の かびん を かいました	
\\	クリスタル
\\	の 花瓶[かびん]を 買[か]いました。			
\\	何度	何度[なんど]	なんど	
\\	あの店には何度も行きました。	あの 店[みせ]には 何度[なんど]も 行[い]きました。	あの みせ に は なんど も いきました	
\\	あの 店[みせ]には
\\	も 行[い]きました。			
\\	コース	コース	コース	
\\	私は日本語コースを取っています。	私[わたし]は 日本語[にほんご]コースを 取[と]っています。	わたし は にほんご こーす を とって います	
\\	私[わたし]は 日本語[にほんご]
\\	を 取[と]っています。			
\\	会長	会長[かいちょう]	かいちょう	
\\	初めに会長が挨拶した。	初[はじ]めに 会長[かいちょう]が 挨拶[あいさつ]した。	はじめ に かいちょう が あいさつ した	
\\	初[はじ]めに
\\	が 挨拶[あいさつ]した。			
\\	アパート	アパート	アパート	
\\	彼はアパートに住んでいます。	彼[かれ]はアパートに 住[す]んでいます。	かれ は あぱーと に すんで います	
\\	彼[かれ]は
\\	に 住[す]んでいます。			
\\	全体	全体[ぜんたい]	ぜんたい	
\\	全体の80パーセントが完成しました。	全体[ぜんたい]の 80[はちじゅっ]パーセントが 完成[かんせい]しました。	ぜんたい の はちじゅっぱーせんと が かんせい しました	
\\	の 80[はちじゅっ]パーセントが 完成[かんせい]しました。			
\\	全く	全[まった]く	まったく	
\\	私には全く分かりません。	私[わたし]には 全[まった]く 分[わ]かりません。	わたし に は まったく わかりません	
\\	私[わたし]には
\\	分[わ]かりません。			
\\	レコード	レコード	レコード	
\\	ジャズのレコードをかけました。	ジャズのレコードをかけました。	じゃず の れこーど を かけました	
\\	ジャズの
\\	をかけました。			
\\	一部	一部[いちぶ]	いちぶ	
\\	計画を一部変更しましょう。	計画[けいかく]を 一部[いちぶ] 変更[へんこう]しましょう。	けいかく を いちぶ へんこう しましょう	
\\	計画[けいかく]を
\\	変更[へんこう]しましょう。			
\\	どんどん	どんどん	どんどん	
\\	ドアをどんどんとたたいた。	ドアをどんどんとたたいた。	どあ を どんどん と たたいた	
\\	ドアを
\\	とたたいた。			
\\	全国	全国[ぜんこく]	ぜんこく	
\\	次は全国のお天気です。	次[つぎ]は 全国[ぜんこく]のお 天気[てんき]です。	つぎ は ぜんこく の おてんき です	
\\	次[つぎ]は
\\	のお 天気[てんき]です。			
\\	カメラ	カメラ	カメラ	
\\	カメラが壊れた。	カメラが 壊[こわ]れた。	かめら が こわれた	
\\	が 壊[こわ]れた。			
\\	世界	世界[せかい]	せかい	
\\	私は世界旅行をしたい。	私[わたし]は 世界[せかい] 旅行[りょこう]をしたい。	わたし は せかい りょこう を したい	
\\	私[わたし]は
\\	旅行[りょこう]をしたい。			
\\	テープ	テープ	テープ	
\\	彼女はその会話をテープに録音した。	彼女[かのじょ]はその 会話[かいわ]をテープに 録音[ろくおん]した。	かのじょ は その かいわ を てーぷ に ろくおん した	
\\	彼女[かのじょ]はその 会話[かいわ]を
\\	に 録音[ろくおん]した。			
\\	ビール	ビール	ビール	
\\	夏はビールがとても美味しい。	夏[なつ]はビールがとても 美味[おい]しい。	なつ は びーる が とても おいしい	
\\	夏[なつ]は
\\	がとても 美味[おい]しい。			
\\	ページ	ページ	ページ	
\\	32ページを開いてください。	32[さんじゅうに]ページを 開[ひら]いてください。	さんじゅうにぺーじ を ひらいて ください	
\\	32[さんじゅうに]
\\	を 開[ひら]いてください。			
\\	分野	分野[ぶんや]	ぶんや	
\\	音楽は彼の得意な分野です。	音楽[おんがく]は 彼[かれ]の 得意[とくい]な 分野[ぶんや]です。	おんがく は かれ の とくい な ぶんや です	
\\	音楽[おんがく]は 彼[かれ]の 得意[とくい]な
\\	です。			
\\	グラフ	グラフ	グラフ	
\\	彼は売上をグラフにした。	彼[かれ]は 売上[うりあげ]をグラフにした。	かれ は うりあげ を ぐらふ に した	
\\	彼[かれ]は 売上[うりあげ]を
\\	にした。			
\\	八百屋	八百屋[やおや]	やおや	
\\	八百屋でみかんを買いました。	八百屋[やおや]でみかんを 買[か]いました。	やおや で みかん を かいました	
\\	でみかんを 買[か]いました。			
\\	そば屋	そば 屋[や]	そばや	
\\	昼はそば屋に行きました。	昼[ひる]はそば 屋[や]に 行[い]きました。	ひる は そばや に いきました 。	
\\	昼[ひる]は
\\	に 行[い]きました。			
\\	たばこ屋	たばこ 屋[や]	たばこや	
\\	私はたばこ屋でライターを買った。	私[わたし]はたばこ 屋[や]でライターを 買[か]った。	わたし は たばこや で らいたー を かった	
\\	私[わたし]は
\\	でライターを 買[か]った。			
\\	ポスト	ポスト	ポスト	
\\	手紙をポストに入れました。	手紙[てがみ]をポストに 入[い]れました。	てがみ を ぽすと に いれました	
\\	手紙[てがみ]を
\\	に 入[い]れました。			
\\	お茶	お 茶[ちゃ]	おちゃ	
\\	お茶を入れましょうか。	お 茶[ちゃ]を 入[い]れましょうか。	おちゃ を いれましょう か	
\\	を 入[い]れましょうか。			
\\	テスト	テスト	テスト	
\\	テストを始めてください。	テストを 始[はじ]めてください。	てすと を はじめて ください	
\\	を 始[はじ]めてください。			
\\	あちこち	あちこち	あちこち	
\\	私たちは朝からあちこち散歩しました。	私[わたし]たちは 朝[あさ]からあちこち 散歩[さんぽ]しました。	わたしたち は あさ から あちこち さんぽ しました	
\\	私[わたし]たちは 朝[あさ]から
\\	散歩[さんぽ]しました。			
\\	作文	作文[さくぶん]	さくぶん	
\\	日本語で作文を書きました。	日本語[にほんご]で 作文[さくぶん]を 書[か]きました。	にほんご で さくぶん を かきました	
\\	日本語[にほんご]で
\\	を 書[か]きました。			
\\	ゴルフ	ゴルフ	ゴルフ	
\\	兄はゴルフを始めました。	兄[あに]はゴルフを 始[はじ]めました。	あに は ごるふ を はじめました	
\\	兄[あに]は
\\	を 始[はじ]めました。			
\\	交通費	交通費[こうつうひ]	こうつうひ	
\\	会社までの交通費は一ヶ月8,000円です。	会社[かいしゃ]までの 交通費[こうつうひ]は 一ヶ月8,000円[いっかげつ はっせんえん]です。	かいしゃ まで の こうつうひ は いっかげつ はっせんえん です	
\\	会社[かいしゃ]までの
\\	は 一ヶ月8,000円[いっかげつ はっせんえん]です。			
\\	ラジオ	ラジオ	ラジオ	
\\	彼女はラジオを聞いています。	彼女[かのじょ]はラジオを 聞[き]いています。	かのじょ は らじお を きいて います	
\\	彼女[かのじょ]は
\\	を 聞[き]いています。			
\\	タクシー	タクシー	タクシー	
\\	タクシーを呼んでください。	タクシーを 呼[よ]んでください。	たくしー を よんで ください	
\\	を 呼[よ]んでください。			
\\	作品	作品[さくひん]	さくひん	
\\	この絵はゴッホの作品だ。	この 絵[え]はゴッホの 作品[さくひん]だ。	この え は ごっほ の さくひん だ	
\\	この 絵[え]はゴッホの
\\	だ。			
\\	二階	二階[にかい]	にかい	
\\	兄は二階にいます。	兄[あに]は 二階[にかい]にいます。	あに は にかい に います	
\\	兄[あに]は
\\	にいます。			
\\	ゆっくり	ゆっくり	ゆっくり	
\\	もっとゆっくり話してください。	もっとゆっくり 話[はな]してください。	もっと ゆっくり はなして ください	
\\	もっと
\\	話[はな]してください。			
\\	値段	値段[ねだん]	ねだん	
\\	このベルトの値段は2500円でした。	このベルトの 値段[ねだん]は 2500円[にせんごひゃくえん]でした。	この べると の ねだん は にせんごひゃくえん でした	
\\	このベルトの
\\	は 2500円[にせんごひゃくえん]でした。			
\\	レストラン	レストラン	レストラン	
\\	レストランでインド料理を食べました。	レストランでインド 料理[りょうり]を 食[た]べました。	れすとらん で いんど りょうり を たべました	
\\	でインド 料理[りょうり]を 食[た]べました。			
\\	価格	価格[かかく]	かかく	
\\	ガソリンの価格がどんどん上がっている。	ガソリンの 価格[かかく]がどんどん 上[あ]がっている。	がそりん の かかく が どんどん あがって いる	
\\	ガソリンの
\\	がどんどん 上[あ]がっている。			
\\	冬休み	冬休[ふゆやす]み	ふゆやすみ	
\\	冬休みにお祖父ちゃんの家に行きます。	冬休[ふゆやす]みにお 祖父[じい]ちゃんの 家[うち]に 行[い]きます。	ふゆやすみ に おじいちゃん の うち に いきます	
\\	にお 祖父[じい]ちゃんの 家[うち]に 行[い]きます。			
\\	カード	カード	カード	
\\	支払いはカードでお願いします。	支払[しはら]いはカードでお 願[ねが]いします。	しはらい は かーど で おねがい します	
\\	支払[しはら]いは
\\	でお 願[ねが]いします。			
\\	アルバイト	アルバイト	アルバイト	
\\	兄はアルバイトをしています。	兄[あに]はアルバイトをしています。	あに は あるばいと を して います	
\\	兄[あに]は
\\	をしています。			
\\	コピー	コピー	コピー	
\\	会議で書類のコピーを配った。	会議[かいぎ]で 書類[しょるい]のコピーを 配[くば]った。	かいぎ で しょるい の こぴー を くばった	
\\	会議[かいぎ]で 書類[しょるい]の
\\	を 配[くば]った。			
\\	事情	事情[じじょう]	じじょう	
\\	あなたの事情はよく分かりました。	あなたの 事情[じじょう]はよく 分[わ]かりました。	あなた の じじょう は よく わかりました	
\\	あなたの
\\	はよく 分[わ]かりました。			
\\	ぶつかる	ぶつかる	ぶつかる	
\\	車が電柱にぶつかった。	車[くるま]が 電柱[でんちゅう]にぶつかった。	くるま が でんちゅう に ぶつかった	
\\	車[くるま]が 電柱[でんちゅう]に
\\	フィルム	フィルム	フィルム	
\\	旅行のためにたくさんフィルムを買った。	旅行[りょこう]のためにたくさんフィルムを 買[か]った。	りょこう の ため に たくさん ふぃるむ を かった	
\\	旅行[りょこう]のためにたくさん
\\	を 買[か]った。			
\\	中心	中心[ちゅうしん]	ちゅうしん	
\\	ここが建物の中心です。	ここが 建物[たてもの]の 中心[ちゅうしん]です。	ここ が たてもの の ちゅうしん です	
\\	ここが 建物[たてもの]の
\\	です。			
\\	デパート	デパート	デパート	
\\	私はデパートで靴を買った。	私[わたし]はデパートで 靴[くつ]を 買[か]った。	わたし は でぱーと で くつ を かった	
\\	私[わたし]は
\\	で 靴[くつ]を 買[か]った。			
\\	ベッド	ベッド	ベッド	
\\	彼はベッドで寝ています。	彼[かれ]はベッドで 寝[ね]ています。	かれ は べっど で ねて います	
\\	彼[かれ]は
\\	で 寝[ね]ています。			
\\	コート	コート	コート	
\\	寒かったのでコートを着た。	寒[さむ]かったのでコートを 着[き]た。	さむかった の で こーと を きた	
\\	寒[さむ]かったので
\\	を 着[き]た。			
\\	不便	不便[ふべん]	ふべん	
\\	私の家は駅から遠くて不便です。	私[わたし]の 家[いえ]は 駅[えき]から 遠[とお]くて 不便[ふべん]です。	わたし の いえ は えき から とおく て ふべん です	
\\	私[わたし]の 家[いえ]は 駅[えき]から 遠[とお]くて
\\	です。			
\\	交番	交番[こうばん]	こうばん	
\\	あそこの交番で道を聞きましょう。	あそこの 交番[こうばん]で 道[みち]を 聞[き]きましょう。	あそこ の こうばん で みち を ききましょう	
\\	あそこの
\\	で 道[みち]を 聞[き]きましょう。			
\\	ノート	ノート	ノート	
\\	ノートを開いてください。	ノートを 開[ひら]いてください。	のーと を ひらいて ください	
\\	を 開[ひら]いてください。			
\\	ワイン	ワイン	ワイン	
\\	ワインを少し飲みました。	ワインを 少[すこ]し 飲[の]みました。	わいん を すこし のみました	
\\	を 少[すこ]し 飲[の]みました。			
\\	住所	住所[じゅうしょ]	じゅうしょ	
\\	この住所に行ってください。	この 住所[じゅうしょ]に 行[い]ってください。	この じゅうしょ に いって ください	
\\	この
\\	に 行[い]ってください。			
\\	便所	便所[べんじょ]	べんじょ	
\\	便所はそこです。	便所[べんじょ]はそこです。	べんじょ は そこ です	
\\	はそこです。			
\\	おかしい	おかしい	おかしい	
\\	彼の様子がおかしい。	彼[かれ]の 様子[ようす]がおかしい。	かれ の ようす が おかしい	
\\	彼[かれ]の 様子[ようす]が
\\	トイレ	トイレ	トイレ	
\\	トイレを掃除しましたか。	トイレを 掃除[そうじ]しましたか。	といれ を そうじ しました か	
\\	を 掃除[そうじ]しましたか。			
\\	キャンプ	キャンプ	キャンプ	
\\	友達とキャンプに行った。	友達[ともだち]とキャンプに 行[い]った。	ともだち と きゃんぷ に いった	
\\	友達[ともだち]と
\\	に 行[い]った。			
\\	プラスチック	プラスチック	プラスチック	
\\	このカップはプラスチックです。	このカップはプラスチックです。	この かっぷ は ぷらすちっく です	
\\	このカップは
\\	です。			
\\	カラー	カラー	カラー	
\\	カラーコピーは一枚幾らですか。	カラーコピーは 一枚幾[いちまい いく]らですか。	からーこぴー は いちまい いくら です か	
\\	コピーは 一枚幾[いちまい いく]らですか。			
\\	ピアノ	ピアノ	ピアノ	
\\	昔、ピアノを習っていました。	昔[むかし]、ピアノを 習[なら]っていました。	むかし ぴあの を ならって いました	
\\	昔[むかし]、
\\	を 習[なら]っていました。			
\\	取れる	取[と]れる	とれる	
\\	シャツのボタンが取れた。	シャツのボタンが 取[と]れた。	しゃつ の ぼたん が とれた	
\\	シャツのボタンが
\\	スキー	スキー	スキー	
\\	冬はよくスキーに行きます。	冬[ふゆ]はよくスキーに 行[い]きます。	ふゆ は よく すきー に いきます	
\\	冬[ふゆ]はよく
\\	に 行[い]きます。			
\\	なかなか	なかなか	なかなか	
\\	荷物がなかなか届きません。	荷物[にもつ]がなかなか 届[とど]きません。	にもつ が なかなか とどきません	
\\	荷物[にもつ]が
\\	届[とど]きません。			
\\	プール	プール	プール	
\\	私は夏休みにプールに行った。	私[わたし]は 夏休[なつやす]みにプールに 行[い]った。	わたし は なつやすみ に ぷーる に いった	
\\	私[わたし]は 夏休[なつやす]みに
\\	に 行[い]った。			
\\	お土産	お 土産[みやげ]	おみやげ	
\\	父はお土産にお菓子を買ってきた。	父[ちち]はお 土産[みやげ]にお 菓子[かし]を 買[か]ってきた。	ちち は おみやげ に おかし を かって きた	
\\	父[ちち]は
\\	にお 菓子[かし]を 買[か]ってきた。			
\\	ホーム	ホーム	ホーム	
\\	もうすぐこのホームに電車が来ます。	もうすぐこのホームに 電車[でんしゃ]が 来[き]ます。	もうすぐ この ほーむ に でんしゃ が きます	
\\	もうすぐこの
\\	に 電車[でんしゃ]が 来[き]ます。			
\\	中学	中学[ちゅうがく]	ちゅうがく	
\\	息子は中学に通っています。	息子[むすこ]は 中学[ちゅうがく]に 通[かよ]っています。	むすこ は ちゅうがく に かよって います	
\\	息子[むすこ]は
\\	に 通[かよ]っています。			
\\	入学	入学[にゅうがく]	にゅうがく	
\\	妹は九月にアメリカの大学に入学します。	妹[いもうと]は 九月[くがつ]にアメリカの 大学[だいがく]に 入学[にゅうがく]します。	いもうと は くがつ に あめりか の だいがく に にゅうがく します	
\\	妹[いもうと]は 九月[くがつ]にアメリカの 大学[だいがく]に
\\	します。			
\\	エレベーター	エレベーター	エレベーター	
\\	エレベーターで下に降りましょう。	エレベーターで 下[した]に 降[お]りましょう。	えれべーたー で した に おりましょう	
\\	で 下[した]に 降[お]りましょう。			
\\	中学生	中学生[ちゅうがくせい]	ちゅうがくせい	
\\	息子は中学生です。	息子[むすこ]は 中学生[ちゅうがくせい]です。	むすこ は ちゅうがくせい です	
\\	息子[むすこ]は
\\	です。			
\\	メモ	メモ	メモ	
\\	メモを取ってください。	メモを 取[と]ってください。	めも を とって ください	
\\	を 取[と]ってください。			
\\	中学校	中学校[ちゅうがっこう]	ちゅうがっこう	
\\	息子の中学校は家から5分です。	息子[むすこ]の 中学校[ちゅうがっこう]は 家[いえ]から 5分[ごふん]です。	むすこ の ちゅうがっこう は いえ から ごふん です	
\\	息子[むすこ]の
\\	は 家[いえ]から 5分[ごふん]です。			
\\	パン	パン	パン	
\\	朝ご飯にはいつもパンを食べる。	朝[あさ]ご 飯[はん]にはいつもパンを 食[た]べる。	あさごはん に は いつも ぱん を たべる	
\\	朝[あさ]ご 飯[はん]にはいつも
\\	を 食[た]べる。			
\\	休校	休校[きゅうこう]	きゅうこう	
\\	学校は今週は休校です。	学校[がっこう]は 今週[こんしゅう]は 休校[きゅうこう]です。	がっこう は こんしゅう は きゅうこう です	
\\	学校[がっこう]は 今週[こんしゅう]は
\\	です。			
\\	びっくりする	びっくりする	びっくりする	
\\	大きな音にびっくりしました。	大[おお]きな 音[おと]にびっくりしました。	おおきな おと に びっくり しました	
\\	大[おお]きな 音[おと]に
\\	制度	制度[せいど]	せいど	
\\	来年から新しい制度が始まります。	来年[らいねん]から 新[あたら]しい 制度[せいど]が 始[はじ]まります。	らいねん から あたらしい せいど が はじまります	
\\	来年[らいねん]から 新[あたら]しい
\\	が 始[はじ]まります。			
\\	ズボン	ズボン	ズボン	
\\	ズボンが汚れた。	ズボンが 汚[よご]れた。	ずぼん が よごれた	
\\	が 汚[よご]れた。			
\\	おもちゃ	おもちゃ	おもちゃ	
\\	赤ちゃんが自動車のおもちゃで遊んでいる。	赤[あか]ちゃんが 自動車[じどうしゃ]のおもちゃで 遊[あそ]んでいる。	あかちゃん が じどうしゃ の おもちゃ で あそんで いる	
\\	赤[あか]ちゃんが 自動車[じどうしゃ]の
\\	で 遊[あそ]んでいる。			
\\	グラム	グラム	グラム	
\\	ひき肉を200グラムください。	ひき 肉[にく]を 200[にひゃく]グラムください。	ひきにく を にひゃくぐらむ ください	
\\	ひき 肉[にく]を 200[にひゃく]
\\	ください。			
\\	コーヒー	コーヒー	コーヒー	
\\	私は毎朝コーヒーを飲みます。	私[わたし]は 毎朝[まいあさ]コーヒーを 飲[の]みます。	わたし は まいあさ こーひー を のみます	
\\	私[わたし]は 毎朝[まいあさ]
\\	を 飲[の]みます。			
\\	今回	今回[こんかい]	こんかい	
\\	まあ今回は許してあげよう。	まあ 今回[こんかい]は 許[ゆる]してあげよう。	まあ こんかい は ゆるして あげよう	
\\	まあ
\\	は 許[ゆる]してあげよう。			
\\	テント	テント	テント	
\\	みんなでテントを張りました。	みんなでテントを 張[は]りました。	みんな で てんと を はりました	
\\	みんなで
\\	を 張[は]りました。			
\\	個人	個人[こじん]	こじん	
\\	これは私個人の意見です。	これは 私[わたし] 個人[こじん]の 意見[いけん]です。	これ は わたし こじん の いけん です	
\\	これは 私[わたし]
\\	の 意見[いけん]です。			
\\	ボート	ボート	ボート	
\\	池でボートに乗りました。	池[いけ]でボートに 乗[の]りました。	いけ で ぼーと に のりました	
\\	池[いけ]で
\\	に 乗[の]りました。			
\\	ボール	ボール	ボール	
\\	彼はボールを投げた。	彼[かれ]はボールを 投[な]げた。	かれ は ぼーる を なげた	
\\	彼[かれ]は
\\	を 投[な]げた。			
\\	全然	全然[ぜんぜん]	ぜんぜん	
\\	この本は全然面白くなかった。	この 本[ほん]は 全然[ぜんぜん] 面白[おもしろ]くなかった。	この ほん は ぜんぜん おもしろく なかった	
\\	この 本[ほん]は
\\	面白[おもしろ]くなかった。			
\\	オートバイ	オートバイ	オートバイ	
\\	彼はオートバイに乗っている。	彼[かれ]はオートバイに 乗[の]っている。	かれ は おーとばい に のって いる	
\\	彼[かれ]は
\\	に 乗[の]っている。			
\\	ひどい	ひどい	ひどい	
\\	妹とひどい喧嘩をした。	妹[いもうと]とひどい 喧嘩[けんか]をした。	いもうと と ひどい けんか を した	
\\	妹[いもうと]と
\\	喧嘩[けんか]をした。			
\\	企業	企業[きぎょう]	きぎょう	
\\	彼女はアメリカの企業で働いています。	彼女[かのじょ]はアメリカの 企業[きぎょう]で 働[はたら]いています。	かのじょ は あめりか の きぎょう で はたらいて います	
\\	彼女[かのじょ]はアメリカの
\\	で 働[はたら]いています。			
\\	作業	作業[さぎょう]	さぎょう	
\\	作業するにはもっと広いスペースが必要だ。	作業[さぎょう]するにはもっと 広[ひろ]いスペースが 必要[ひつよう]だ。	さぎょう する に は もっと ひろい すぺーす が ひつよう だ	
\\	するにはもっと 広[ひろ]いスペースが 必要[ひつよう]だ。			
\\	あなた	あなた	あなた	
\\	この本、あなたにあげます。	この 本[ほん]、あなたにあげます。	この ほん あなた に あげます	
\\	この 本[ほん]、
\\	にあげます。			
\\	利用	利用[りよう]	りよう	
\\	私はよく図書館を利用します。	私[わたし]はよく 図書館[としょかん]を 利用[りよう]します。	わたし は よく としょかん を りよう します	
\\	私[わたし]はよく 図書館[としょかん]を
\\	します。			
\\	便利	便利[べんり]	べんり	
\\	インターネットはとても便利です。	インターネットはとても 便利[べんり]です。	いんたーねっと は とても べんり です	
\\	インターネットはとても
\\	です。			
\\	スイッチ	スイッチ	スイッチ	
\\	彼はカーラジオのスイッチを入れた。	彼[かれ]はカーラジオのスイッチを 入[い]れた。	かれ は かーらじお の すいっち を いれた	
\\	彼[かれ]はカーラジオの
\\	を 入[い]れた。			
\\	プレゼント	プレゼント	プレゼント	
\\	誕生日にプレゼントをもらいました。	誕生日[たんじょうび]にプレゼントをもらいました。	たんじょうび に ぷれぜんと を もらいました	
\\	誕生日[たんじょうび]に
\\	をもらいました。			
\\	いつでも	いつでも	いつでも	
\\	いつでもうちに来てください。	いつでもうちに 来[き]てください。	いつでも うち に きて ください	
\\	うちに 来[き]てください。			
\\	交差点	交差点[こうさてん]	こうさてん	
\\	次の交差点を左に曲がってください。	次[つぎ]の 交差点[こうさてん]を 左[ひだり]に 曲[ま]がってください。	つぎ の こうさてん を ひだり に まがって ください	
\\	次[つぎ]の
\\	を 左[ひだり]に 曲[ま]がってください。			
\\	テニス	テニス	テニス	
\\	彼らはよくテニスをしています。	彼[かれ]らはよくテニスをしています。	かれら は よく てにす を して います	
\\	彼[かれ]らはよく
\\	をしています。			
\\	割る	割[わ]る	わる	
\\	皿を落として割った。	皿[さら]を 落[お]として 割[わ]った。	さら を おとして わった	
\\	皿[さら]を 落[お]として
\\	割れる	割[わ]れる	われる	
\\	コップが落ちて割れた。	コップが 落[お]ちて 割[わ]れた。	こっぷ が おちて われた	
\\	コップが 落[お]ちて
\\	割り算	割[わ]り 算[ざん]	わりざん	
\\	割り算は小学校で習います。	割[わ]り 算[ざん]は 小学校[しょうがっこう]で 習[なら]います。	わりざん は しょうがっこう で ならいます	
\\	は 小学校[しょうがっこう]で 習[なら]います。			
\\	こちら	こちら	こちら	
\\	受付はこちらです。	受付[うけつけ]はこちらです。	うけつけ は こちら です	
\\	受付[うけつけ]は
\\	です。			
\\	借りる	借[か]りる	かりる	
\\	彼にビデオを借りました。	彼[かれ]にビデオを 借[か]りました。	かれ に びでお を かりました	
\\	彼[かれ]にビデオを
\\	ボタン	ボタン	ボタン	
\\	ボタンを押してください。	ボタンを 押[お]してください。	ぼたん を おして ください	
\\	を 押[お]してください。			
\\	おかしい	おかしい	おかしい	
\\	彼の話はおかしかった。	彼[かれ]の 話[はなし]はおかしかった。	かれ の はなし は おかしかった	
\\	彼[かれ]の 話[はなし]は
\\	コート	コート	コート	
\\	新しいコートでテニスをしました。	新[あたら]しいコートでテニスをしました。	あたらしい こーと で てにす を しました	
\\	新[あたら]しい
\\	でテニスをしました。			
\\	切れる	切[き]れる	きれる	
\\	このはさみはよく切れますね。	このはさみはよく 切[き]れますね。	この はさみ は よく きれます ね	
\\	このはさみはよく
\\	ね。			
\\	ドラマ	ドラマ	ドラマ	
\\	私はドラマを見るのが好きです。	私[わたし]はドラマを 見[み]るのが 好[す]きです。	わたし は どらま を みる の が すき です	
\\	私[わたし]は
\\	を 見[み]るのが 好[す]きです。			
\\	切手	切手[きって]	きって	
\\	郵便局で切手を買いました。	郵便局[ゆうびんきょく]で 切手[きって]を 買[か]いました。	ゆうびんきょく で きって を かいました	
\\	郵便局[ゆうびんきょく]で
\\	を 買[か]いました。			
\\	ビザ	ビザ	ビザ	
\\	学生ビザを持っています。	学生[がくせい]ビザを 持[も]っています。	がくせい びざ を もって います	
\\	学生[がくせい]
\\	を 持[も]っています。			
\\	切符	切符[きっぷ]	きっぷ	
\\	東京までの切符を買った。	東京[とうきょう]までの 切符[きっぷ]を 買[か]った。	とうきょう まで の きっぷ を かった	
\\	東京[とうきょう]までの
\\	を 買[か]った。			
\\	入場券	入場券[にゅうじょうけん]	にゅうじょうけん	
\\	入場券は3000円です。	入場券[にゅうじょうけん]は 3000円[さんぜんえん]です。	にゅうじょうけん は さんぜんえん です	
\\	は 3000円[さんぜんえん]です。			
\\	ポケット	ポケット	ポケット	
\\	財布をポケットにしまった。	財布[さいふ]をポケットにしまった。	さいふ を ぽけっと に しまった	
\\	財布[さいふ]を
\\	にしまった。			
\\	代わる	代[か]わる	かわる	
\\	上司に代わって会議に出た。	上司[じょうし]に 代[か]わって 会議[かいぎ]に 出[で]た。	じょうし に かわって かいぎ に でた	
\\	上司[じょうし]に
\\	会議[かいぎ]に 出[で]た。			
\\	代える	代[か]える	かえる	
\\	社長に代えて部長を出席させます。	社長[しゃちょう]に 代[か]えて 部長[ぶちょう]を 出席[しゅっせき]させます。	しゃちょう に かえて ぶちょう を しゅっせき させます	
\\	社長[しゃちょう]に
\\	部長[ぶちょう]を 出席[しゅっせき]させます。			
\\	そろそろ	そろそろ	そろそろ	
\\	そろそろ始めましょうか。	そろそろ 始[はじ]めましょうか。	そろそろ はじめましょう か	
\\	始[はじ]めましょうか。			
\\	一定	一定[いってい]	いってい	
\\	申し込むには一定の資格が要る。	申[もう]し 込[こ]むには 一定[いってい]の 資格[しかく]が 要[い]る。	もうしこむ に は いってい の しかく が いる	
\\	申[もう]し 込[こ]むには
\\	の 資格[しかく]が 要[い]る。			
\\	予定	予定[よてい]	よてい	
\\	今日の予定を教えてください。	今日[きょう]の 予定[よてい]を 教[おし]えてください。	きょう の よてい を おしえて ください	
\\	今日[きょう]の
\\	を 教[おし]えてください。			
\\	ぶどう	ぶどう	ぶどう	
\\	私はぶどうが好きです。	私[わたし]はぶどうが 好[す]きです。	わたし は ぶどう が すき です	
\\	私[わたし]は
\\	が 好[す]きです。			
\\	予習	予習[よしゅう]	よしゅう	
\\	明日の予習をしましょう。	明日[あした]の 予習[よしゅう]をしましょう。	あした の よしゅう を しましょう	
\\	明日[あした]の
\\	をしましょう。			
\\	予約	予約[よやく]	よやく	
\\	レストランを予約しました。	レストランを 予約[よやく]しました。	れすとらん を よやく しました	
\\	レストランを
\\	しました。			
\\	めったに	めったに	めったに	
\\	彼女はめったに怒りません。	彼女[かのじょ]はめったに 怒[おこ]りません。	かのじょ は めったに おこりません 。	
\\	彼女[かのじょ]は
\\	怒[おこ]りません。			
\\	クリスマス	クリスマス	クリスマス	
\\	クリスマスにはケーキを食べます。	クリスマスにはケーキを 食[た]べます。	くりすます に は けーき を たべます	
\\	にはケーキを 食[た]べます。			
\\	ネクタイ	ネクタイ	ネクタイ	
\\	父の日にネクタイをプレゼントした。	父[ちち]の 日[ひ]にネクタイをプレゼントした。	ちちのひ に ねくたい を ぷれぜんと した	
\\	父[ちち]の 日[ひ]に
\\	をプレゼントした。			
\\	乗り物	乗[の]り 物[もの]	のりもの	
\\	自転車は便利な乗り物です。	自転車[じてんしゃ]は 便利[べんり]な 乗[の]り 物[もの]です。	じてんしゃ は べんり な のりもの です	
\\	自転車[じてんしゃ]は 便利[べんり]な
\\	です。			
\\	バイオリン	バイオリン	バイオリン	
\\	彼女はバイオリンを習っています。	彼女[かのじょ]はバイオリンを 習[なら]っています。	かのじょ は ばいおりん を ならって います	
\\	彼女[かのじょ]は
\\	を 習[なら]っています。			
\\	下着	下着[したぎ]	したぎ	
\\	私は下着を手で洗う。	私[わたし]は 下着[したぎ]を 手[て]で 洗[あら]う。	わたし は したぎ を て で あらう	
\\	私[わたし]は
\\	を 手[て]で 洗[あら]う。			
\\	パスポート	パスポート	パスポート	
\\	パスポートが見つからない。	パスポートが 見[み]つからない。	ぱすぽーと が みつからない	
\\	が 見[み]つからない。			
\\	上着	上着[うわぎ]	うわぎ	
\\	暑いので上着を脱ぎました。	暑[あつ]いので 上着[うわぎ]を 脱[ぬ]ぎました。	あつい の で うわぎ を ぬぎました	
\\	暑[あつ]いので
\\	を 脱[ぬ]ぎました。			
\\	バッグ	バッグ	バッグ	
\\	私は黒いバッグを持っています。	私[わたし]は 黒[くろ]いバッグを 持[も]っています。	わたし は くろい ばっぐ を もって います	
\\	私[わたし]は 黒[くろ]い
\\	を 持[も]っています。			
\\	出席	出席[しゅっせき]	しゅっせき	
\\	午後は会議に出席します。	午後[ごご]は 会議[かいぎ]に 出席[しゅっせき]します。	ごご は かいぎ に しゅっせき します	
\\	午後[ごご]は 会議[かいぎ]に
\\	します。			
\\	ゴム	ゴム	ゴム	
\\	ゴムが伸びてしまった。	ゴムが 伸[の]びてしまった。	ごむ が のびて しまった	
\\	が 伸[の]びてしまった。			
\\	ギター	ギター	ギター	
\\	彼女はギターが得意です。	彼女[かのじょ]はギターが 得意[とくい]です。	かのじょ は ぎたー が とくい です	
\\	彼女[かのじょ]は
\\	が 得意[とくい]です。			
\\	セーター	セーター	セーター	
\\	このセーターはウールだ。	このセーターはウールだ。	この せーたー は うーる だ	
\\	この
\\	はウールだ。			
\\	動き	動[うご]き	うごき	
\\	まず相手の動きを見ましょう。	まず 相手[あいて]の 動[うご]きを 見[み]ましょう。	まず あいて の うごき を みましょう	
\\	まず 相手[あいて]の
\\	を 見[み]ましょう。			
\\	やっぱり	やっぱり	やっぱり	
\\	やっぱり旅が大好きだ。	やっぱり 旅[たび]が 大好[だいす]きだ。	やっぱり たび が だいすき だ	
\\	旅[たび]が 大好[だいす]きだ。			
\\	動物	動物[どうぶつ]	どうぶつ	
\\	私は動物が大好きです。	私[わたし]は 動物[どうぶつ]が 大好[だいす]きです。	わたし は どうぶつ が だいすき です	
\\	私[わたし]は
\\	が 大好[だいす]きです。			
\\	動かす	動[うご]かす	うごかす	
\\	部屋の家具を動かしました。	部屋[へや]の 家具[かぐ]を 動[うご]かしました。	へや の かぐ を うごかしました	
\\	部屋[へや]の 家具[かぐ]を
\\	不動産屋	不動産屋[ふどうさんや]	ふどうさんや	
\\	私の父は不動産屋です。	私[わたし]の 父[ちち]は 不動産屋[ふどうさんや]です。	わたし の ちち は ふどうさんや です	
\\	私[わたし]の 父[ちち]は
\\	です。			
\\	カーテン	カーテン	カーテン	
\\	朝、カーテンを開けた。	朝[あさ]、カーテンを 開[あ]けた。	あさ かーてん を あけた	
\\	朝[あさ]、
\\	を 開[あ]けた。			
\\	シャツ	シャツ	シャツ	
\\	このシャツはアイロンが必要だ。	このシャツはアイロンが 必要[ひつよう]だ。	この しゃつ は あいろん が ひつよう だ	
\\	この
\\	はアイロンが 必要[ひつよう]だ。			
\\	ナイフ	ナイフ	ナイフ	
\\	ナイフで手を切った。	ナイフで 手[て]を 切[き]った。	ないふ で て を きった	
\\	で 手[て]を 切[き]った。			
\\	事実	事実[じじつ]	じじつ	
\\	それは全て事実ですか。	それは 全[すべ]て 事実[じじつ]ですか。	それ は すべて じじつ です か	
\\	それは 全[すべ]て
\\	ですか。			
\\	バイク	バイク	バイク	
\\	兄はバイクが大好きです。	兄[あに]はバイクが 大好[だいす]きです。	あに は ばいく が だいすき です	
\\	兄[あに]は
\\	が 大好[だいす]きです。			
\\	いつか	いつか	いつか	
\\	僕はいつかアフリカに行きたい。	僕[ぼく]はいつかアフリカに 行[い]きたい。	ぼく は いつか あふりか に いきたい	
\\	僕[ぼく]は
\\	アフリカに 行[い]きたい。			
\\	オレンジ	オレンジ	オレンジ	
\\	私はオレンジが好きです。	私[わたし]はオレンジが 好[す]きです。	わたし は おれんじ が すき です	
\\	私[わたし]は
\\	が 好[す]きです。			
\\	出発	出発[しゅっぱつ]	しゅっぱつ	
\\	あと15分で出発です。	あと 15分[じゅうごふん]で 出発[しゅっぱつ]です。	あと じゅうごふん で しゅっぱつ です	
\\	あと 15分[じゅうごふん]で
\\	です。			
\\	きつい	きつい	きつい	
\\	このシャツは少しきついです。	このシャツは 少[すこ]しきついです。	この しゃつ は すこし きつい です	
\\	このシャツは 少[すこ]し
\\	です。			
\\	代表	代表[だいひょう]	だいひょう	
\\	彼がクラスの代表だ。	彼[かれ]がクラスの 代表[だいひょう]だ。	かれ が くらす の だいひょう だ	
\\	彼[かれ]がクラスの
\\	だ。			
\\	さっき	さっき	さっき	
\\	さっきの話を続けましょう。	さっきの 話[はなし]を 続[つづ]けましょう。	さっき の はなし を つづけましょう	
\\	の 話[はなし]を 続[つづ]けましょう。			
\\	テキスト	テキスト	テキスト	
\\	テキストを読んでください。	テキストを 読[よ]んでください。	てきすと を よんで ください	
\\	を 読[よ]んでください。			
\\	ビニール	ビニール	ビニール	
\\	ゴミはそのビニール袋に入れてください。	ゴミはそのビニール 袋[ぶくろ]に 入[い]れてください。	ごみ は その びにーるぶくろ に いれて ください	
\\	ゴミはその
\\	袋[ぶくろ]に 入[い]れてください。			
\\	りんご	りんご	りんご	
\\	りんごを一つください。	りんごを 一[ひと]つください。	りんご を ひとつ ください	
\\	を 一[ひと]つください。			
\\	写る	写[うつ]る	うつる	
\\	このカメラはよく写りますよ。	このカメラはよく 写[うつ]りますよ。	この かめら は よく うつります よ	
\\	このカメラはよく
\\	よ。			
\\	写す	写[うつ]す	うつす	
\\	彼は友達の答えを写した。	彼[かれ]は 友達[ともだち]の 答[こた]えを 写[うつ]した。	かれ は ともだち の こたえ を うつした	
\\	彼[かれ]は 友達[ともだち]の 答[こた]えを
\\	写真	写真[しゃしん]	しゃしん	
\\	写真は良い思い出になります。	写真[しゃしん]は 良[い]い 思[おも]い 出[で]になります。	しゃしん は いい おもいで に なります	
\\	は 良[い]い 思[おも]い 出[で]になります。			
\\	アルバム	アルバム	アルバム	
\\	私は彼のアルバムを見た。	私[わたし]は 彼[かれ]のアルバムを 見[み]た。	わたし は かれ の あるばむ を みた	
\\	私[わたし]は 彼[かれ]の
\\	を 見[み]た。			
\\	スカート	スカート	スカート	
\\	彼女はあまりスカートははかない。	彼女[かのじょ]はあまりスカートははかない。	かのじょ は あまり すかーと は はかない	
\\	彼女[かのじょ]はあまり
\\	ははかない。			
\\	ペン	ペン	ペン	
\\	ペンを貸してください。	ペンを 貸[か]してください。	ぺん を かして ください	
\\	を 貸[か]してください。			
\\	人形	人形[にんぎょう]	にんぎょう	
\\	彼女は人形をたくさん持っています。	彼女[かのじょ]は 人形[にんぎょう]をたくさん 持[も]っています。	かのじょ は にんぎょう を たくさん もって います	
\\	彼女[かのじょ]は
\\	をたくさん 持[も]っています。			
\\	おしゃべり	おしゃべり	おしゃべり	
\\	妹はとてもおしゃべりです。	妹[いもうと]はとてもおしゃべりです。	いもうと は とても おしゃべり です	
\\	妹[いもうと]はとても
\\	です。			
\\	きっと	きっと	きっと	
\\	明日はきっと雨が降ります。	明日[あした]はきっと 雨[あめ]が 降[ふ]ります。	あした は きっと あめ が ふります	
\\	明日[あした]は
\\	雨[あめ]が 降[ふ]ります。			
\\	三角	三角[さんかく]	さんかく	
\\	紙を三角に切りました。	紙[かみ]を 三角[さんかく]に 切[き]りました。	かみ を さんかく に きりました	
\\	紙[かみ]を
\\	に 切[き]りました。			
\\	メニュー	メニュー	メニュー	
\\	メニューをください。	メニューをください。	めにゅー を ください	
\\	をください。			
\\	アイスクリーム	アイスクリーム	アイスクリーム	
\\	弟はアイスクリームが大好きです。	弟[おとうと]はアイスクリームが 大好[だいす]きです。	おとうと は あいすくりーむ が だいすき です	
\\	弟[おとうと]は
\\	が 大好[だいす]きです。			
\\	似ている	似[に]ている	にている	
\\	私は母に似ています。	私[わたし]は 母[はは]に 似[に]ています。	わたし は はは に にて います	
\\	私[わたし]は 母[はは]に
\\	おしゃれ	おしゃれ	おしゃれ	
\\	あの子はおしゃれだね。	あの 子[こ]はおしゃれだね。	あの こ は おしゃれ だ ね	
\\	あの 子[こ]は
\\	だね。			
\\	以上	以上[いじょう]	いじょう	
\\	飛行機が1時間以上遅れた。	飛行機[ひこうき]が 1時間[いちじかん] 以上[いじょう] 遅[おく]れた。	ひこうき が いちじかん いじょう おくれた	
\\	飛行機[ひこうき]が 1時間[いちじかん]
\\	遅[おく]れた。			
\\	ジュース	ジュース	ジュース	
\\	このジュースは甘すぎる。	このジュースは 甘[あま]すぎる。	この じゅーす は あま すぎる	
\\	この
\\	は 甘[あま]すぎる。			
\\	ゼロ	ゼロ	ゼロ	
\\	今日の交通事故はゼロです。	今日[きょう]の 交通事故[こうつう じこ]はゼロです。	きょう の こうつう じこ は ぜろ です	
\\	今日[きょう]の 交通事故[こうつう じこ]は
\\	です。			
\\	ピンク	ピンク	ピンク	
\\	娘がピンクのドレスを着ている。	娘[むすめ]がピンクのドレスを 着[き]ている。	むすめ が ぴんく の どれす を きて いる	
\\	娘[むすめ]が
\\	のドレスを 着[き]ている。			
\\	不親切	不親切[ふしんせつ]	ふしんせつ	
\\	その店員は不親切だった。	その 店員[てんいん]は 不親切[ふしんせつ]だった。	その てんいん は ふしんせつ だった	
\\	その 店員[てんいん]は
\\	だった。			
\\	グラス	グラス	グラス	
\\	これはきれいなグラスですね。	これはきれいなグラスですね。	これ は きれい な ぐらす です ね	
\\	これはきれいな
\\	ですね。			
\\	万歳	万歳[ばんざい]	ばんざい	
\\	勝った、万歳!	勝[か]った、 万歳[ばんざい]!	かった ばんざい	
\\	勝[か]った、
\\	二十歳	二十歳[はたち]	はたち	
\\	妹は来年、二十歳になります。	妹[いもうと]は 来年[らいねん]、 二十歳[はたち]になります。	いもうと は らいねん はたち に なります	
\\	妹[いもうと]は 来年[らいねん]、
\\	になります。			
\\	兄さん	兄[にい]さん	にいさん	
\\	兄さん、おめでとう。	兄[にい]さん、おめでとう。	にいさん おめでとう	
\\	、おめでとう。			
\\	ダンス	ダンス	ダンス	
\\	彼女はダンスが大好きです。	彼女[かのじょ]はダンスが 大好[だいす]きです。	かのじょ は だんす が だいすき です	
\\	彼女[かのじょ]は
\\	が 大好[だいす]きです。			
\\	兄弟	兄弟[きょうだい]	きょうだい	
\\	彼は3人兄弟です。	彼[かれ]は 3人[さんにん] 兄弟[きょうだい]です。	かれ は さんにん きょうだい です	
\\	彼[かれ]は 3人[さんにん]
\\	です。			
\\	おじ	おじ	おじ	
\\	おじは銀行に勤めています。	おじは 銀行[ぎんこう]に 勤[つと]めています。	おじ は ぎんこう に つとめて います	
\\	は 銀行[ぎんこう]に 勤[つと]めています。			
\\	サッカー	サッカー	サッカー	
\\	彼はサッカーの選手です。	彼[かれ]はサッカーの 選手[せんしゅ]です。	かれ は さっかー の せんしゅ です	
\\	彼[かれ]は
\\	の 選手[せんしゅ]です。			
\\	効果	効果[こうか]	こうか	
\\	この薬には胃を守る効果がある。	この 薬[くすり]には 胃[い]を 守[まも]る 効果[こうか]がある。	この くすり に は い を まもる こうか が ある	
\\	この 薬[くすり]には 胃[い]を 守[まも]る
\\	がある。			
\\	スープ	スープ	スープ	
\\	母がコーンスープを作っている。	母[はは]がコーンスープを 作[つく]っている。	はは が こーんすーぷ を つくって いる	
\\	母[はは]がコーン
\\	を 作[つく]っている。			
\\	バナナ	バナナ	バナナ	
\\	私は毎朝バナナを食べます。	私[わたし]は 毎朝[まいあさ]バナナを 食[た]べます。	わたし は まいあさ ばなな を たべます	
\\	私[わたし]は 毎朝[まいあさ]
\\	を 食[た]べます。			
\\	信じる	信[しん]じる	しんじる	
\\	彼はキリストを信じている。	彼[かれ]はキリストを 信[しん]じている。	かれ は きりすと を しんじて いる	
\\	彼[かれ]はキリストを
\\	信号	信号[しんごう]	しんごう	
\\	信号が青になった。	信号[しんごう]が 青[あお]になった。	しんごう が あお に なった	
\\	が 青[あお]になった。			
\\	うるさい	うるさい	うるさい	
\\	この通りは車の音がうるさい。	この 通[とお]りは 車[くるま]の 音[おと]がうるさい。	この とおり は くるま の おと が うるさい	
\\	この 通[とお]りは 車[くるま]の 音[おと]が
\\	住民	住民[じゅうみん]	じゅうみん	
\\	地域の住民が集まって話合いをした。	地域[ちいき]の 住民[じゅうみん]が 集[あつ]まって 話合[はなしあ]いをした。	ちいき の じゅうみん が あつまって はなしあい を した	
\\	地域[ちいき]の
\\	が 集[あつ]まって 話合[はなしあ]いをした。			
\\	主人	主人[しゅじん]	しゅじん	
\\	主人は今、留守です。	主人[しゅじん]は 今[いま]、 留守[るす]です。	しゅじん は いま るす です	
\\	は 今[いま]、 留守[るす]です。			
\\	会議	会議[かいぎ]	かいぎ	
\\	今日の午後、大事な会議があります。	今日[きょう]の 午後[ごご]、 大事[だいじ]な 会議[かいぎ]があります。	きょう の ごご だいじ な かいぎ が あります	
\\	今日[きょう]の 午後[ごご]、 大事[だいじ]な
\\	があります。			
\\	パトカー	パトカー	パトカー	
\\	あそこにパトカーがいる。	あそこにパトカーがいる。	あそこ に ぱとかー が いる	
\\	あそこに
\\	がいる。			
\\	反対	反対[はんたい]	はんたい	
\\	私は反対です。	私[わたし]は 反対[はんたい]です。	わたし は はんたい です	
\\	私[わたし]は
\\	です。			
\\	ハンバーガー	ハンバーガー	ハンバーガー	
\\	今日の昼ご飯はハンバーガーでした。	今日[きょう]の 昼[ひる]ご 飯[はん]はハンバーガーでした。	きょう の ひるごはん は はんばーがー でした	
\\	今日[きょう]の 昼[ひる]ご 飯[はん]は
\\	でした。			
\\	別々	別々[べつべつ]	べつべつ	
\\	別々に払いましょう。	別々[べつべつ]に 払[はら]いましょう。	べつべつ に はらいましょう	
\\	に 払[はら]いましょう。			
\\	エスカレーター	エスカレーター	エスカレーター	
\\	3階までエスカレーターで行きましょう。	3階[さんかい]までエスカレーターで 行[い]きましょう。	さんかい まで えすかれーたー で いきましょう	
\\	3階[さんかい]まで
\\	で 行[い]きましょう。			
\\	別れる	別[わか]れる	わかれる	
\\	駅で友だちと別れました。	駅[えき]で 友[とも]だちと 別[わか]れました。	えき で ともだち と わかれました	
\\	駅[えき]で 友[とも]だちと
\\	一般	一般[いっぱん]	いっぱん	
\\	一般の方はこちらの席へどうぞ。	一般[いっぱん]の 方[かた]はこちらの 席[せき]へどうぞ。	いっぱん の かた は こちら の せき へ どうぞ	
\\	の 方[かた]はこちらの 席[せき]へどうぞ。			
\\	タオル	タオル	タオル	
\\	私はタオルで顔をふいた。	私[わたし]はタオルで 顔[かお]をふいた。	わたし は たおる で かお を ふいた	
\\	私[わたし]は
\\	で 顔[かお]をふいた。			
\\	並ぶ	並[なら]ぶ	ならぶ	
\\	ここに並んでください。	ここに 並[なら]んでください。	ここ に ならんで ください	
\\	ここに
\\	ください。			
\\	並べる	並[なら]べる	ならべる	
\\	私は料理をテーブルに並べた。	私[わたし]は 料理[りょうり]をテーブルに 並[なら]べた。	わたし は りょうり を てーぶる に ならべた	
\\	私[わたし]は 料理[りょうり]をテーブルに
\\	パチンコ	パチンコ	パチンコ	
\\	彼は毎日パチンコをしています。	彼[かれ]は 毎日[まいにち]パチンコをしています。	かれ は まいにち ぱちんこ を して います	
\\	彼[かれ]は 毎日[まいにち]
\\	をしています。			
\\	入院	入院[にゅういん]	にゅういん	
\\	昨日、母が入院しました。	昨日[きのう]、 母[はは]が 入院[にゅういん]しました。	きのう はは が にゅういん しました	
\\	昨日[きのう]、 母[はは]が
\\	しました。			
\\	医者	医者[いしゃ]	いしゃ	
\\	私は医者に相談した。	私[わたし]は 医者[いしゃ]に 相談[そうだん]した。	わたし は いしゃ に そうだん した	
\\	私[わたし]は
\\	に 相談[そうだん]した。			
\\	みかん	みかん	みかん	
\\	みかんを1つください。	みかんを 1[ひと]つください。	みかん を ひとつ ください	
\\	を 1[ひと]つください。			
\\	お医者さん	お 医者[いしゃ]さん	おいしゃさん	
\\	熱があるのでお医者さんに行った。	熱[ねつ]があるのでお 医者[いしゃ]さんに 行[い]った。	ねつ が ある の で おいしゃさん に いった	
\\	熱[ねつ]があるので
\\	に 行[い]った。			
\\	ケーキ	ケーキ	ケーキ	
\\	誕生日にケーキを食べました。	誕生日[たんじょうび]にケーキを 食[た]べました。	たんじょうび に けーき を たべました	
\\	誕生日[たんじょうび]に
\\	を 食[た]べました。			
\\	亡くなる	亡[な]くなる	なくなる	
\\	おととい、昔の友人が亡くなった。	おととい、 昔[むかし]の 友人[ゆうじん]が 亡[な]くなった。	おととい むかし の ゆうじん が なくなった	
\\	おととい、 昔[むかし]の 友人[ゆうじん]が
\\	コップ	コップ	コップ	
\\	私は毎朝コップ一杯の水を飲む。	私[わたし]は 毎朝[まいあさ]コップ 一杯[いっぱい]の 水[みず]を 飲[の]む。	わたし は まいあさ こっぷ いっぱい の みず を のむ	
\\	私[わたし]は 毎朝[まいあさ]
\\	一杯[いっぱい]の 水[みず]を 飲[の]む。			
\\	ナイロン	ナイロン	ナイロン	
\\	このジャケットはナイロンでできています。	このジャケットはナイロンでできています。	この じゃけっと は ないろん で できて います	
\\	このジャケットは
\\	でできています。			
\\	危ない	危[あぶ]ない	あぶない	
\\	その道は車が多くて危ない。	その 道[みち]は 車[くるま]が 多[おお]くて 危[あぶ]ない。	その みち は くるま が おおくて あぶない	
\\	その 道[みち]は 車[くるま]が 多[おお]くて
\\	危険	危険[きけん]	きけん	
\\	その地域は今、危険だ。	その 地域[ちいき]は 今[いま]、 危険[きけん]だ。	その ちいき は いま きけん だ	
\\	その 地域[ちいき]は 今[いま]、
\\	だ。			
\\	スーツ	スーツ	スーツ	
\\	あのスーツはそんなに高くない。	あのスーツはそんなに 高[たか]くない。	あの すーつ は そんなに たかく ない	
\\	あの
\\	はそんなに 高[たか]くない。			
\\	チケット	チケット	チケット	
\\	この遊園地のチケットは3000円です。	この 遊園地[ゆうえんち]のチケットは 3000円[さんぜんえん]です。	この ゆうえんち の ちけっと は さんぜんえん です	
\\	この 遊園地[ゆうえんち]の
\\	は 3000円[さんぜんえん]です。			
\\	チョコレート	チョコレート	チョコレート	
\\	妹はチョコレートが大好きです。	妹[いもうと]はチョコレートが 大好[だいす]きです。	いもうと は ちょこれーと が だいすき です	
\\	妹[いもうと]は
\\	が 大好[だいす]きです。			
\\	ジェット機	ジェット 機[き]	ジェットき	
\\	ジェット機が飛んでいる。	ジェット 機[き]が 飛[と]んでいる。	じぇっとき が とんで いる	
\\	が 飛[と]んでいる。			
\\	チャンネル	チャンネル	チャンネル	
\\	テレビのチャンネルを変えてください。	テレビのチャンネルを 変[か]えてください。	てれび の ちゃんねる を かえて ください	
\\	テレビの
\\	を 変[か]えてください。			
\\	具体的	具体的[ぐたいてき]	ぐたいてき	
\\	具体的な例をいくつか見せてください。	具体的[ぐたいてき]な 例[れい]をいくつか 見[み]せてください。	ぐたいてき な れい を いくつか みせて ください	
\\	な 例[れい]をいくつか 見[み]せてください。			
\\	つまらない	つまらない	つまらない	
\\	彼の話はつまらないですね。	彼[かれ]の 話[はなし]はつまらないですね。	かれ の はなし は つまらない です ね	
\\	彼[かれ]の 話[はなし]は
\\	ですね。			
\\	備える	備[そな]える	そなえる	
\\	災害に備えて大量の水を買い込んだ。	災害[さいがい]に 備[そな]えて 大量[たいりょう]の 水[みず]を 買[か]い 込[こ]んだ。	さいがい に そなえて たいりょう の みず を かいこんだ	
\\	災害[さいがい]に
\\	大量[たいりょう]の 水[みず]を 買[か]い 込[こ]んだ。			
\\	マッチ	マッチ	マッチ	
\\	マッチを持っていますか。	マッチを 持[も]っていますか。	まっち を もって います か	
\\	を 持[も]っていますか。			
\\	公開	公開[こうかい]	こうかい	
\\	その映画は今日、公開されます。	その 映画[えいが]は 今日[きょう]、 公開[こうかい]されます。	その えいが は きょう こうかい されます	
\\	その 映画[えいが]は 今日[きょう]、
\\	されます。			
\\	いとこ	いとこ	いとこ	
\\	従兄弟と私は同じ年です。	従兄弟[いとこ]と 私[わたし]は 同[おな]じ 年[とし]です。	いとこ と わたし は おなじ とし です	
\\	と 私[わたし]は 同[おな]じ 年[とし]です。			
\\	公園	公園[こうえん]	こうえん	
\\	子供たちが公園で遊んでいる。	子供[こども]たちが 公園[こうえん]で 遊[あそ]んでいる。	こどもたち が こうえん で あそんで いる	
\\	子供[こども]たちが
\\	で 遊[あそ]んでいる。			
\\	動物園	動物園[どうぶつえん]	どうぶつえん	
\\	昨日子供たちと動物園に行きました。	昨日子供[きのう こども]たちと 動物園[どうぶつえん]に 行[い]きました。	きのう こどもたち と どうぶつえん に いきました	
\\	昨日子供[きのう こども]たちと
\\	に 行[い]きました。			
\\	ストーブ	ストーブ	ストーブ	
\\	うちではまだストーブを使っています。	うちではまだストーブを 使[つか]っています。	うち で は まだ すとーぶ を つかって います	
\\	うちではまだ
\\	を 使[つか]っています。			
\\	ガソリンスタンド	ガソリンスタンド	ガソリンスタンド	
\\	この近くにガソリンスタンドはありますか。	この 近[ちか]くにガソリンスタンドはありますか。	この ちかく に がそりんすたんど は あります か	
\\	この 近[ちか]くに
\\	はありますか。			
\\	デート	デート	デート	
\\	デートで遊園地に行きました。	デートで 遊園地[ゆうえんち]に 行[い]きました。	でーと で ゆうえんち に いきました	
\\	で 遊園地[ゆうえんち]に 行[い]きました。			
\\	原因	原因[げんいん]	げんいん	
\\	この事故の原因は何ですか。	この 事故[じこ]の 原因[げんいん]は 何[なん]ですか。	このじこ の げんいん は なん です か	
\\	この 事故[じこ]の
\\	は 何[なん]ですか。			
\\	ふすま	ふすま	ふすま	
\\	ふすまを閉めてください。	ふすまを 閉[し]めてください。	ふすま を しめて ください	
\\	を 閉[し]めてください。			
\\	レモン	レモン	レモン	
\\	紅茶にレモンを入れて飲んだ。	紅茶[こうちゃ]にレモンを 入[い]れて 飲[の]んだ。	こうちゃ に れもん を いれて のんだ	
\\	紅茶[こうちゃ]に
\\	を 入[い]れて 飲[の]んだ。			
\\	チーズ	チーズ	チーズ	
\\	チーズを一切れ食べました。	チーズを 一切[ひとき]れ 食[た]べました。	ちーず を ひときれ たべました	
\\	を 一切[ひとき]れ 食[た]べました。			
\\	ドライブ	ドライブ	ドライブ	
\\	今日は群馬までドライブしました。	今日[きょう]は 群馬[ぐんま]までドライブしました。	きょう は ぐんま まで どらいぶ しました	
\\	今日[きょう]は 群馬[ぐんま]まで
\\	しました。			
\\	ラケット	ラケット	ラケット	
\\	テニスのラケットを買いました。	テニスのラケットを 買[か]いました。	てにす の らけっと を かいました	
\\	テニスの
\\	を 買[か]いました。			
\\	スチュワーデス	スチュワーデス	スチュワーデス	
\\	彼女はスチュワーデスになった。	彼女[かのじょ]はスチュワーデスになった。	かのじょ は すちゅわーです に なった	
\\	彼女[かのじょ]は
\\	になった。			
\\	会談	会談[かいだん]	かいだん	
\\	会談の内容が発表されました。	会談[かいだん]の 内容[ないよう]が 発表[はっぴょう]されました。	かいだん の ないよう が はっぴょう されました	
\\	の 内容[ないよう]が 発表[はっぴょう]されました。			
\\	テープレコーダー	テープレコーダー	テープレコーダー	
\\	テープレコーダーで自分の声を録音しました。	テープレコーダーで 自分[じぶん]の 声[こえ]を 録音[ろくおん]しました。	てーぷれこーだー で じぶん の こえ を ろくおん しました	
\\	で 自分[じぶん]の 声[こえ]を 録音[ろくおん]しました。			
\\	ひげ	ひげ	ひげ	
\\	父はひげをはやしています。	父[ちち]はひげをはやしています。	ちち は ひげ を はやして います	
\\	父[ちち]は
\\	をはやしています。			
\\	ぐっすり	ぐっすり	ぐっすり	
\\	赤ちゃんがぐっすり寝ている。	赤[あか]ちゃんがぐっすり 寝[ね]ている。	あかちゃん が ぐっすり ねて いる	
\\	赤[あか]ちゃんが
\\	寝[ね]ている。			
\\	ソース	ソース	ソース	
\\	ソースはどれですか。	ソースはどれですか。	そーす は どれ です か	
\\	はどれですか。			
\\	党	党[とう]	とう	
\\	党の代表が質問に答えました。	党[とう]の 代表[だいひょう]が 質問[しつもん]に 答[こた]えました。	とう の だいひょう が しつもん に こたえました	
\\	の 代表[だいひょう]が 質問[しつもん]に 答[こた]えました。			
\\	タイプライター	タイプライター	タイプライター	
\\	母は古いタイプライターを持っています。	母[はは]は 古[ふる]いタイプライターを 持[も]っています。	はは は ふるい たいぷらいたー を もって います	
\\	母[はは]は 古[ふる]い
\\	を 持[も]っています。			
\\	候補	候補[こうほ]	こうほ	
\\	会長の候補は3人います。	会長[かいちょう]の 候補[こうほ]は 3人[さんにん]います。	かいちょう の こうほ は さんにん います	
\\	会長[かいちょう]の
\\	は 3人[さんにん]います。			
\\	トマト	トマト	トマト	
\\	私はトマトが大好きです。	私[わたし]はトマトが 大好[だいす]きです。	わたし は とまと が だいすき です	
\\	私[わたし]は
\\	が 大好[だいす]きです。			
\\	バター	バター	バター	
\\	パンにバターをぬって食べました。	パンにバターをぬって 食[た]べました。	ぱん に ばたー を ぬって たべました	
\\	パンに
\\	をぬって 食[た]べました。			
\\	バレーボール	バレーボール	バレーボール	
\\	妹はバレーボールが得意です。	妹[いもうと]はバレーボールが 得意[とくい]です。	いもうと は ばれーぼーる が とくい です	
\\	妹[いもうと]は
\\	が 得意[とくい]です。			
\\	おかず	おかず	おかず	
\\	晩ご飯のおかずは何?	晩[ばん]ご 飯[はん]のおかずは 何?[なに]	ばんごはん の おかず は なに	
\\	晩[ばん]ご 飯[はん]の
\\	は 何?[なに]			
\\	カレンダー	カレンダー	カレンダー	
\\	カレンダーに予定を書いた。	カレンダーに 予定[よてい]を 書[か]いた。	かれんだー に よてい を かいた	
\\	に 予定[よてい]を 書[か]いた。			
\\	勤める	勤[つと]める	つとめる	
\\	私は銀行に勤めています。	私[わたし]は 銀行[ぎんこう]に 勤[つと]めています。	わたし は ぎんこう に つとめて います	
\\	私[わたし]は 銀行[ぎんこう]に
\\	事務所	事務所[じむしょ]	じむしょ	
\\	後で事務所に来てください。	後[あと]で 事務所[じむしょ]に 来[き]てください。	あとで じむしょ に きて ください	
\\	後[あと]で
\\	に 来[き]てください。			
\\	事務室	事務室[じむしつ]	じむしつ	
\\	事務室でコピーを取って来ます。	事務室[じむしつ]でコピーを 取[と]って 来[き]ます。	じむしつ で こぴー を とって きます	
\\	でコピーを 取[と]って 来[き]ます。			
\\	ハンドバッグ	ハンドバッグ	ハンドバッグ	
\\	ハンドバッグを忘れました。	ハンドバッグを 忘[わす]れました。	はんどばっぐ を わすれました	
\\	を 忘[わす]れました。			
\\	ベル	ベル	ベル	
\\	玄関のベルが鳴った。	玄関[げんかん]のベルが 鳴[な]った。	げんかん の べる が なった	
\\	玄関[げんかん]の
\\	が 鳴[な]った。			
\\	卒業	卒業[そつぎょう]	そつぎょう	
\\	私は去年、大学を卒業した。	私[わたし]は 去年[きょねん]、 大学[だいがく]を 卒業[そつぎょう]した。	わたし は きょねん だいがく を そつぎょう した	
\\	私[わたし]は 去年[きょねん]、 大学[だいがく]を
\\	した。			
\\	インク	インク	インク	
\\	プリンターのインクを買った。	プリンターのインクを 買[か]った。	ぷりんたー の いんく を かった	
\\	プリンターの
\\	を 買[か]った。			
\\	交渉	交渉[こうしょう]	こうしょう	
\\	今、値段を交渉しています。	今[いま]、 値段[ねだん]を 交渉[こうしょう]しています。	いま ねだん を こうしょう して います	
\\	今[いま]、 値段[ねだん]を
\\	しています。			
\\	事件	事件[じけん]	じけん	
\\	その事件の犯人はまだ捕まっていない。	その 事件[じけん]の 犯人[はんにん]はまだ 捕[つか]まっていない。	その じけん の はんにん は まだ つかまって いない	
\\	その
\\	の 犯人[はんにん]はまだ 捕[つか]まっていない。			
\\	クーラー	クーラー	クーラー	
\\	暑いのでクーラーをつけました。	暑[あつ]いのでクーラーをつけました。	あつい の で くーらー を つけました	
\\	暑[あつ]いので
\\	をつけました。			
\\	参加	参加[さんか]	さんか	
\\	明日は市民マラソンに参加します。	明日[あす]は 市民[しみん]マラソンに 参加[さんか]します。	あす は しみん まらそん に さんか します	
\\	明日[あす]は 市民[しみん]マラソンに
\\	します。			
\\	加える	加[くわ]える	くわえる	
\\	塩、コショウを加えてください。	塩[しお]、コショウを 加[くわ]えてください。	しお こしょう を くわえて ください	
\\	塩[しお]、コショウを
\\	ください。			
\\	加工	加工[かこう]	かこう	
\\	この工場では魚を加工している。	この 工場[こうじょう]では 魚[さかな]を 加工[かこう]している。	この こうじょう で は さかな を かこう して いる	
\\	この 工場[こうじょう]では 魚[さかな]を
\\	している。			
\\	ミルク	ミルク	ミルク	
\\	赤ちゃんにミルクをあげた。	赤[あか]ちゃんにミルクをあげた。	あかちゃん に みるく を あげた	
\\	赤[あか]ちゃんに
\\	をあげた。			
\\	サラダ	サラダ	サラダ	
\\	サラダをたくさん食べました。	サラダをたくさん 食[た]べました。	さらだ を たくさん たべました	
\\	をたくさん 食[た]べました。			
\\	修理	修理[しゅうり]	しゅうり	
\\	車を修理に出した。	車[くるま]を 修理[しゅうり]に 出[だ]した。	くるま を しゅうり に だした	
\\	車[くるま]を
\\	に 出[だ]した。			
\\	乗り換える	乗[の]り 換[か]える	のりかえる	
\\	次の駅で地下鉄に乗り換えます。	次[つぎ]の 駅[えき]で 地下鉄[ちかてつ]に 乗[の]り 換[か]えます。	つぎ の えき で ちかてつ に のりかえます	
\\	次[つぎ]の 駅[えき]で 地下鉄[ちかてつ]に
\\	乗り換え	乗[の]り 換[か]え	のりかえ	
\\	次の駅で乗り換えです。	次[つぎ]の 駅[えき]で 乗[の]り 換[か]えです。	つぎ の えき で のりかえ です	
\\	次[つぎ]の 駅[えき]で
\\	です。			
\\	おじさん	おじさん	おじさん	
\\	昨日おじさんに会いました。	昨日[きのう]おじさんに 会[あ]いました。	きのう おじさん に あいました	
\\	昨日[きのう]
\\	に 会[あ]いました。			
\\	ベルト	ベルト	ベルト	
\\	ベルトがきつくなりました。	ベルトがきつくなりました。	べると が きつく なりました	
\\	がきつくなりました。			
\\	ラーメン	ラーメン	ラーメン	
\\	夕食にラーメンを食べました。	夕食[ゆうしょく]にラーメンを 食[た]べました。	ゆうしょく に らーめん を たべました	
\\	夕食[ゆうしょく]に
\\	を 食[た]べました。			
\\	助ける	助[たす]ける	たすける	
\\	彼女は病気の犬を助けた。	彼女[かのじょ]は 病気[びょうき]の 犬[いぬ]を 助[たす]けた。	かのじょ は びょうき の いぬ を たすけた	
\\	彼女[かのじょ]は 病気[びょうき]の 犬[いぬ]を
\\	ライター	ライター	ライター	
\\	店にライターを忘れました。	店[みせ]にライターを 忘[わす]れました。	みせ に らいたー を わすれました	
\\	店[みせ]に
\\	を 忘[わす]れました。			
\\	ワイシャツ	ワイシャツ	ワイシャツ	
\\	彼は白いワイシャツを着ている。	彼[かれ]は 白[しろ]いワイシャツを 着[き]ている。	かれ は しろい わいしゃつ を きて いる	
\\	彼[かれ]は 白[しろ]い
\\	を 着[き]ている。			
\\	シャワー	シャワー	シャワー	
\\	朝、急いでシャワーを浴びました。	朝[あさ]、 急[いそ]いでシャワーを 浴[あ]びました。	あさ いそいで しゃわー を あびました	
\\	朝[あさ]、 急[いそ]いで
\\	を 浴[あ]びました。			
\\	うどん	うどん	うどん	
\\	私はお昼にうどんを食べました。	私[わたし]はお 昼[ひる]にうどんを 食[た]べました。	わたし は おひる に うどん を たべました	
\\	私[わたし]はお 昼[ひる]に
\\	を 食[た]べました。			
\\	ごみ	ごみ	ごみ	
\\	今日はごみの日だ。	今日[きょう]はごみの 日[ひ]だ。	きょう は ごみ の ひ だ	
\\	今日[きょう]は
\\	の 日[ひ]だ。			
\\	スーパーマーケット	スーパーマーケット	スーパーマーケット	
\\	スーパーマーケットで人参を買った。	スーパーマーケットで 人参[にんじん]を 買[か]った。	すーぱーまーけっと で にんじん を かった	
\\	で 人参[にんじん]を 買[か]った。			
\\	ハンカチ	ハンカチ	ハンカチ	
\\	ハンカチで手をふきました。	ハンカチで 手[て]をふきました。	はんかち で て を ふきました	
\\	で 手[て]をふきました。			
\\	久しぶり	久[ひさ]しぶり	ひさしぶり	
\\	明日、久しぶりに友達に会います。	明日[あした]、 久[ひさ]しぶりに 友達[ともだち]に 会[あ]います。	あした ひさしぶり に ともだち に あいます	
\\	明日[あした]、
\\	に 友達[ともだち]に 会[あ]います。			
\\	冷える	冷[ひ]える	ひえる	
\\	クーラーで体が冷えた。	クーラーで 体[からだ]が 冷[ひ]えた。	くーらー で からだ が ひえた	
\\	クーラーで 体[からだ]が
\\	冷やす	冷[ひ]やす	ひやす	
\\	頭を冷やしなさい。	頭[あたま]を 冷[ひ]やしなさい。	あたま を ひやしなさい	
\\	頭[あたま]を
\\	ビデオテープ	ビデオテープ	ビデオテープ	
\\	私はビデオテープを30本持っています。	私[わたし]はビデオテープを 30本持[さんじゅっぽん も]っています。	わたし は びでおてーぷ を さんじゅっぽん もって います	
\\	私[わたし]は
\\	を 30本持[さんじゅっぽん も]っています。			
\\	冷める	冷[さ]める	さめる	
\\	彼への気持ちが冷めた。	彼[かれ]への 気持[きも]ちが 冷[さ]めた。	かれ へ の きもち が さめた	
\\	彼[かれ]への 気持[きも]ちが
\\	凍る	凍[こお]る	こおる	
\\	寒い朝は道路が凍ります。	寒[さむ]い 朝[あさ]は 道路[どうろ]が 凍[こお]ります。	さむい あさ は どうろ が こおります	
\\	寒[さむ]い 朝[あさ]は 道路[どうろ]が
\\	冷蔵庫	冷蔵庫[れいぞうこ]	れいぞうこ	
\\	飲み物は冷蔵庫にあります。	飲[の]み 物[もの]は 冷蔵庫[れいぞうこ]にあります。	のみもの は れいぞうこ に あります	
\\	飲[の]み 物[もの]は
\\	にあります。			
\\	ぶらぶら	ぶらぶら	ぶらぶら	
\\	朝、近所をぶらぶらした。	朝[あさ]、 近所[きんじょ]をぶらぶらした。	あさ きんじょ を ぶらぶら した	
\\	朝[あさ]、 近所[きんじょ]を
\\	した。			
\\	アナウンサー	アナウンサー	アナウンサー	
\\	私はアナウンサーになりたい。	私[わたし]はアナウンサーになりたい。	わたし は あなうんさー に なりたい	
\\	私[わたし]は
\\	になりたい。			
\\	光る	光[ひか]る	ひかる	
\\	波がきらきら光っていました。	波[なみ]がきらきら 光[ひか]っていました。	なみ が きらきら ひかって いました	
\\	波[なみ]がきらきら
\\	スプーン	スプーン	スプーン	
\\	カレーライスはスプーンで食べます。	カレーライスはスプーンで 食[た]べます。	かれーらいす は すぷーん で たべます	
\\	カレーライスは
\\	で 食[た]べます。			
\\	ブラウス	ブラウス	ブラウス	
\\	彼女は白いブラウスを着ている。	彼女[かのじょ]は 白[しろ]いブラウスを 着[き]ている。	かのじょ は しろい ぶらうす を きて いる	
\\	彼女[かのじょ]は 白[しろ]い
\\	を 着[き]ている。			
\\	リボン	リボン	リボン	
\\	プレゼントにリボンを付けた。	プレゼントにリボンを 付[つ]けた。	ぷれぜんと に りぼん を つけた	
\\	プレゼントに
\\	を 付[つ]けた。			
\\	厳しい	厳[きび]しい	きびしい	
\\	私の上司はとても厳しい。	私[わたし]の 上司[じょうし]はとても 厳[きび]しい。	わたし の じょうし は とても きびしい	
\\	私[わたし]の 上司[じょうし]はとても
\\	いかが	いかが	いかが	
\\	お加減はいかがですか。	お 加減[かげん]はいかがですか。	おかげん は いかが です か	
\\	お 加減[かげん]は
\\	ですか。			
\\	世紀	世紀[せいき]	せいき	
\\	新しい世紀の始まりです。	新[あたら]しい 世紀[せいき]の 始[はじ]まりです。	あたらしい せいき の はじまり です	
\\	新[あたら]しい
\\	の 始[はじ]まりです。			
\\	ステレオ	ステレオ	ステレオ	
\\	彼はステレオで音楽を聞いた。	彼[かれ]はステレオで 音楽[おんがく]を 聞[き]いた。	かれ は すてれお で おんがく を きいた	
\\	彼[かれ]は
\\	で 音楽[おんがく]を 聞[き]いた。			
\\	フォーク	フォーク	フォーク	
\\	フォークをもらえますか。	フォークをもらえますか。	ふぉーく を もらえます か	
\\	をもらえますか。			
\\	位置	位置[いち]	いち	
\\	私の町は東京の北に位置します。	私[わたし]の 町[まち]は 東京[とうきょう]の 北[きた]に 位置[いち]します。	わたし の まち は とうきょう の きた に いち します	
\\	私[わたし]の 町[まち]は 東京[とうきょう]の 北[きた]に
\\	します。			
\\	おば	おば	おば	
\\	おばは大阪に住んでいます。	おばは 大阪[おおさか]に 住[す]んでいます。	おば は おおさか に すんで います	
\\	は 大阪[おおさか]に 住[す]んでいます。			
\\	停車	停車[ていしゃ]	ていしゃ	
\\	この電車は東京まで停車致しません。	この 電車[でんしゃ]は 東京[とうきょう]まで 停車[ていしゃ] 致[いた]しません。	この でんしゃ は とうきょう まで ていしゃ いたしません	
\\	この 電車[でんしゃ]は 東京[とうきょう]まで
\\	致[いた]しません。			
\\	バス停	バス 停[てい]	ばすてい	
\\	次のバス停で降ります。	次[つぎ]のバス 停[てい]で 降[お]ります。	つぎ の ばすてい で おります	
\\	次[つぎ]の
\\	で 降[お]ります。			
\\	きらきら	きらきら	きらきら	
\\	星がきらきら光っている。	星[ほし]がきらきら 光[ひか]っている。	ほし が きらきら ひかって いる	
\\	星[ほし]が
\\	光[ひか]っている。			
\\	判断	判断[はんだん]	はんだん	
\\	彼の判断は正しい。	彼[かれ]の 判断[はんだん]は 正[ただ]しい。	かれ の はんだん は ただしい	
\\	彼[かれ]の
\\	は 正[ただ]しい。			
\\	どうぞ	どうぞ	どうぞ	
\\	こちらへどうぞ。	こちらへどうぞ。	こちらへどうぞ。	
\\	こちらへ
\\	ネックレス	ネックレス	ネックレス	
\\	妻にネックレスをプレゼントしました。	妻[つま]にネックレスをプレゼントしました。	つま に ねっくれす を ぷれぜんと しました	
\\	妻[つま]に
\\	をプレゼントしました。			
\\	冷房	冷房[れいぼう]	れいぼう	
\\	冷房を入れてください。	冷房[れいぼう]を 入[い]れてください。	れいぼう を いれて ください	
\\	を 入[い]れてください。			
\\	パジャマ	パジャマ	パジャマ	
\\	このパジャマを着て。	このパジャマを 着[き]て。	この ぱじゃま を きて	
\\	この
\\	を 着[き]て。			
\\	ボールペン	ボールペン	ボールペン	
\\	ボールペンで名前を書いてください。	ボールペンで 名前[なまえ]を 書[か]いてください。	ぼーるぺん で なまえ を かいて ください	
\\	で 名前[なまえ]を 書[か]いてください。			
\\	すいか	すいか	すいか	
\\	夏はすいかが美味しい。	夏[なつ]はすいかが 美味[おい]しい。	なつ は すいか が おいしい	
\\	夏[なつ]は
\\	が 美味[おい]しい。			
\\	停留所	停留所[ていりゅうじょ]	ていりゅうじょ	
\\	バスの停留所で10分待ちました。	バスの 停留所[ていりゅうじょ]で 10分待[じゅっぷん ま]ちました。	ばす の ていりゅうじょ で じゅっぷん まちました	
\\	バスの
\\	で 10分待[じゅっぷん ま]ちました。			
\\	そちら	そちら	そちら	
\\	夕方そちらに着きます。	夕方[ゆうがた]そちらに 着[つ]きます。	ゆうがた そちら に つきます	
\\	夕方[ゆうがた]
\\	に 着[つ]きます。			
\\	住宅	住宅[じゅうたく]	じゅうたく	
\\	ここは静かな住宅地だ。	ここは 静[しず]かな 住宅[じゅうたく] 地[ち]だ。	ここ は しずかな じゅうたくち だ	
\\	ここは 静[しず]かな
\\	地[ち]だ。			
\\	そば	そば	そば	
\\	お昼にそばを食べた。	お 昼[ひる]にそばを 食[た]べた。	おひる に そば を たべた	
\\	お 昼[ひる]に
\\	を 食[た]べた。			
\\	お宅	お 宅[たく]	おたく	
\\	先生のお宅はどちらですか。	先生[せんせい]のお 宅[たく]はどちらですか。	せんせい の おたく は どちら です か	
\\	先生[せんせい]の
\\	はどちらですか。			
\\	どっち	どっち	どっち	
\\	どっちの色が好きですか。	どっちの 色[いろ]が 好[す]きですか。	どっち の いろ が すき です か 。	
\\	の 色[いろ]が 好[す]きですか。			
\\	両方	両方[りょうほう]	りょうほう	
\\	チョコレートとケーキを両方ください。	チョコレートとケーキを 両方[りょうほう]ください。	ちょこれーと と けーき を りょうほう ください	
\\	チョコレートとケーキを
\\	ください。			
\\	両替	両替[りょうがえ]	りょうがえ	
\\	1万円を両替してください。	1万円[いちまんえん]を 両替[りょうがえ]してください。	いちまんえん を りょうがえ して ください	
\\	1万円[いちまんえん]を
\\	してください。			
\\	バスケットボール	バスケットボール	バスケットボール	
\\	姉はバスケットボールの選手です。	姉[あね]はバスケットボールの 選手[せんしゅ]です。	あね は ばすけっとぼーる の せんしゅ です	
\\	姉[あね]は
\\	の 選手[せんしゅ]です。			
\\	両親	両親[りょうしん]	りょうしん	
\\	私の両親は大阪に住んでいます。	私[わたし]の 両親[りょうしん]は 大阪[おおさか]に 住[す]んでいます。	わたし の りょうしん は おおさか に すんで います	
\\	私[わたし]の
\\	は 大阪[おおさか]に 住[す]んでいます。			
\\	内側	内側[うちがわ]	うちがわ	
\\	白線の内側に下がってください。	白線[はくせん]の 内側[うちがわ]に 下[さ]がってください。	はくせん の うちがわ に さがって ください	
\\	白線[はくせん]の
\\	に 下[さ]がってください。			
\\	サンドイッチ	サンドイッチ	サンドイッチ	
\\	今日のお昼はサンドイッチです。	今日[きょう]のお 昼[ひる]はサンドイッチです。	きょう の おひる は さんどいっち です	
\\	今日[きょう]のお 昼[ひる]は
\\	です。			
\\	にこにこ	にこにこ	にこにこ	
\\	彼女はいつもにこにこしています。	彼女[かのじょ]はいつもにこにこしています。	かのじょ は いつも にこにこ して います	
\\	彼女[かのじょ]はいつも
\\	しています。			
\\	ウール	ウール	ウール	
\\	このウールのセーターは暖かい。	このウールのセーターは 暖[あたた]かい。	この うーる の せーたー は あたたかい	
\\	この
\\	のセーターは 暖[あたた]かい。			
\\	コーラ	コーラ	コーラ	
\\	私の兄はコーラが好きだ。	私[わたし]の 兄[あに]はコーラが 好[す]きだ。	わたし の あに は こーら が すき だ	
\\	私[わたし]の 兄[あに]は
\\	が 好[す]きだ。			
\\	トイレットペーパー	トイレットペーパー	トイレットペーパー	
\\	トイレットペーパーを交換しました。	トイレットペーパーを 交換[こうかん]しました。	といれっとぺーぱー を こうかん しました	
\\	を 交換[こうかん]しました。			
\\	付く	付[つ]く	つく	
\\	靴に泥が付いています。	靴[くつ]に 泥[どろ]が 付[つ]いています。	くつ に どろ が ついて います	
\\	靴[くつ]に 泥[どろ]が
\\	付ける	付[つ]ける	つける	
\\	おまけを付けました。	おまけを 付[つ]けました。	おまけ を つけました	
\\	おまけを
\\	どなた	どなた	どなた	
\\	失礼ですが、どなたですか。	失礼[しつれい]ですが、どなたですか。	しつれい です が 、 どなた です か 。	
\\	失礼[しつれい]ですが、
\\	ですか。			
\\	ハイキング	ハイキング	ハイキング	
\\	昨日は友人とハイキングに行きました。	昨日[きのう]は 友人[ゆうじん]とハイキングに 行[い]きました。	きのう は ゆうじん と はいきんぐ に いきました	
\\	昨日[きのう]は 友人[ゆうじん]と
\\	に 行[い]きました。			
\\	例えば	例[たと]えば	たとえば	
\\	例えば、このソフトで日本語を勉強することができます。	例[たと]えば、このソフトで 日本語[にほんご]を 勉強[べんきょう]することができます。	たとえば この そふと で にほんご を べんきょう する こと が できます	
\\	、このソフトで 日本語[にほんご]を 勉強[べんきょう]することができます。			
\\	ハム	ハム	ハム	
\\	ハムサンドをください。	ハムサンドをください。	はむ さんど を ください	
\\	サンドをください。			
\\	例文	例文[れいぶん]	れいぶん	
\\	例文を3つ作ってください。	例文[れいぶん]を 3[みっ]つ 作[つく]ってください。	れいぶん を みっつ つくって ください	
\\	を 3[みっ]つ 作[つく]ってください。			
\\	例	例[れい]	れい	
\\	一つ例をあげてください。	一[ひと]つ 例[れい]をあげてください。	ひとつ れい を あげて ください	
\\	一[ひと]つ
\\	をあげてください。			
\\	余る	余[あま]る	あまる	
\\	夕食の料理がたくさん余った。	夕食[ゆうしょく]の 料理[りょうり]がたくさん 余[あま]った。	ゆうしょく の りょうり が たくさん あまった	
\\	夕食[ゆうしょく]の 料理[りょうり]がたくさん
\\	ボーイフレンド	ボーイフレンド	ボーイフレンド	
\\	私のボーイフレンドはハンサムな方でした。	私[わたし]のボーイフレンドはハンサムな 方[かた]でした。	わたし の ぼーいふれんど は はんさむな かた でした	
\\	私[わたし]の
\\	はハンサムな 方[かた]でした。			
\\	削る	削[けず]る	けずる	
\\	彼は家族のために仕事の時間を削った。	彼[かれ]は 家族[かぞく]のために 仕事[しごと]の 時間[じかん]を 削[けず]った。	かれ は かぞく の ため に しごと の じかん を けずった	
\\	彼[かれ]は 家族[かぞく]のために 仕事[しごと]の 時間[じかん]を
\\	アイロン	アイロン	アイロン	
\\	彼女はシャツにアイロンをかけた。	彼女[かのじょ]はシャツにアイロンをかけた。	かのじょ は しゃつ に あいろん を かけた	
\\	彼女[かのじょ]はシャツに
\\	をかけた。			
\\	カレーライス	カレーライス	カレーライス	
\\	子供はカレーライスが好きです。	子供[こども]はカレーライスが 好[す]きです。	こども は かれーらいす が すき です	
\\	子供[こども]は
\\	が 好[す]きです。			
\\	不幸	不幸[ふこう]	ふこう	
\\	彼女の家族に不幸があった。	彼女[かのじょ]の 家族[かぞく]に 不幸[ふこう]があった。	かのじょ の かぞく に ふこう が あった	
\\	彼女[かのじょ]の 家族[かぞく]に
\\	があった。			
\\	キャッシュカード	キャッシュカード	キャッシュカード	
\\	キャッシュカードでお金を下ろした。	キャッシュカードでお 金[かね]を 下[お]ろした。	きゃっしゅかーど で おかね を おろした	
\\	でお 金[かね]を 下[お]ろした。			
\\	イヤリング	イヤリング	イヤリング	
\\	彼女はすてきなイヤリングをしている。	彼女[かのじょ]はすてきなイヤリングをしている。	かのじょ は すてき な いやりんぐ を して いる	
\\	彼女[かのじょ]はすてきな
\\	をしている。			
\\	厚い	厚[あつ]い	あつい	
\\	その辞書はとても厚い。	その 辞書[じしょ]はとても 厚[あつ]い。	その じしょ は とても あつい	
\\	その 辞書[じしょ]はとても
\\	厚さ	厚[あつ]さ	あつさ	
\\	私は板の厚さを測った。	私[わたし]は 板[いた]の 厚[あつ]さを 測[はか]った。	わたし は いた の あつさ を はかった	
\\	私[わたし]は 板[いた]の
\\	を 測[はか]った。			
\\	ガールフレンド	ガールフレンド	ガールフレンド	
\\	昔は彼女のことをガールフレンドと言っていました。	昔[むかし]は 彼女[かのじょ]のことをガールフレンドと 言[い]っていました。	むかし は かのじょ の こと を がーるふれんど と いって いました	
\\	昔[むかし]は 彼女[かのじょ]のことを
\\	と 言[い]っていました。			
\\	伸びる	伸[の]びる	のびる	
\\	髪がだいぶ伸びたね。	髪[かみ]がだいぶ 伸[の]びたね。	かみ が だいぶ のびた ね	
\\	髪[かみ]がだいぶ
\\	ね。			
\\	カセットテープ	カセットテープ	カセットテープ	
\\	昔はカセットテープを使っていました。	昔[むかし]はカセットテープを 使[つか]っていました。	むかし は かせっとてーぷ を つかっていました 。	
\\	昔[むかし]は
\\	を 使[つか]っていました。			
\\	上昇	上昇[じょうしょう]	じょうしょう	
\\	地球の気温は上昇している。	地球[ちきゅう]の 気温[きおん]は 上昇[じょうしょう]している。	ちきゅう の きおん は じょうしょう して いる	
\\	地球[ちきゅう]の 気温[きおん]は
\\	している。			
\\	かゆい	かゆい	かゆい	
\\	背中がかゆいです。	背中[せなか]がかゆいです。	せなか が かゆい です	
\\	背中[せなか]が
\\	です。			
\\	お礼	お 礼[れい]	おれい	
\\	彼女にお礼の手紙を書きました。	彼女[かのじょ]にお 礼[れい]の 手紙[てがみ]を 書[か]きました。	かのじょ に おれい の てがみ を かきました	
\\	彼女[かのじょ]に
\\	の 手紙[てがみ]を 書[か]きました。			
\\	ぐらぐら	ぐらぐら	ぐらぐら	
\\	地震で家がぐらぐらと揺れた。	地震[じしん]で 家[いえ]がぐらぐらと 揺[ゆ]れた。	じしん で いえ が ぐらぐら と ゆれた	
\\	地震[じしん]で 家[いえ]が
\\	と 揺[ゆ]れた。			
\\	努力	努力[どりょく]	どりょく	
\\	もっと努力しよう。	もっと 努力[どりょく]しよう。	もっと どりょく しよう	
\\	もっと
\\	しよう。			
\\	ジャム	ジャム	ジャム	
\\	このイチゴでジャムを作りましょう。	このイチゴでジャムを 作[つく]りましょう。	この いちご で じゃむ を つくりましょう	
\\	このイチゴで
\\	を 作[つく]りましょう。			
\\	占める	占[し]める	しめる	
\\	私に届くメールのうち、迷惑メールが7割を占めている。	私[わたし]に 届[とど]くメールのうち、 迷惑[めいわく]メールが 7割[ななわり]を 占[し]めている。	わたし に とどく めーる の うち めいわく めーる が ななわり を しめて いる	
\\	私[わたし]に 届[とど]くメールのうち、 迷惑[めいわく]メールが 7割[ななわり]を
\\	処理	処理[しょり]	しょり	
\\	事務的な処理に1週間かかります。	事務的[じむてき]な 処理[しょり]に 1週間[いっしゅうかん]かかります。	じむてき な しょり に いっしゅうかん かかります	
\\	事務的[じむてき]な
\\	に 1週間[いっしゅうかん]かかります。			
\\	スリッパ	スリッパ	スリッパ	
\\	スリッパをはいてください。	スリッパをはいてください。	すりっぱ を はいて ください	
\\	をはいてください。			
\\	トランプ	トランプ	トランプ	
\\	友達とトランプをして遊びました。	友達[ともだち]とトランプをして 遊[あそ]びました。	ともだち と とらんぷ を して あそびました	
\\	友達[ともだち]と
\\	をして 遊[あそ]びました。			
\\	よろしい	よろしい	よろしい	
\\	レポートはこれでよろしいですか。	レポートはこれでよろしいですか。	れぽーと は これ で よろしい です か	
\\	レポートはこれで
\\	ですか。			
\\	ラッシュアワー	ラッシュアワー	ラッシュアワー	
\\	私はラッシュアワーの電車が嫌いです。	私[わたし]はラッシュアワーの 電車[でんしゃ]が 嫌[きら]いです。	わたし は らっしゅあわー の でんしゃ が きらい です	
\\	私[わたし]は
\\	の 電車[でんしゃ]が 嫌[きら]いです。			
\\	お祝い	お 祝[いわ]い	おいわい	
\\	彼の就職のお祝いをしよう。	彼[かれ]の 就職[しゅうしょく]のお 祝[いわ]いをしよう。	かれ の しゅうしょく の おいわい を しよう	
\\	彼[かれ]の 就職[しゅうしょく]の
\\	をしよう。			
\\	カップ	カップ	カップ	
\\	水を1カップ入れてください。	水[みず]を 1[いち]カップ 入[い]れてください。	みず を いちかっぷ いれて ください	
\\	水[みず]を 1[いち]
\\	入[い]れてください。			
\\	スケート	スケート	スケート	
\\	湖でスケートをした。	湖[みずうみ]でスケートをした。	みずうみ で すけーと を した	
\\	湖[みずうみ]で
\\	をした。			
\\	たくさん	たくさん	たくさん	
\\	昨日はたくさん泳ぎました。	昨日[きのう]はたくさん 泳[およ]ぎました。	きのう は たくさん およぎました	
\\	昨日[きのう]は
\\	泳[およ]ぎました。			
\\	ピクニック	ピクニック	ピクニック	
\\	今日はピクニックに行きましょう。	今日[きょう]はピクニックに 行[い]きましょう。	きょう は ぴくにっく に いきましょう	
\\	今日[きょう]は
\\	に 行[い]きましょう。			
\\	与える	与[あた]える	あたえる	
\\	そのニュースは彼に大きなショックを与えた。	そのニュースは 彼[かれ]に 大[おお]きなショックを 与[あた]えた。	その にゅーす は かれ に おおき な しょっく を あたえた	
\\	そのニュースは 彼[かれ]に 大[おお]きなショックを
\\	あちら	あちら	あちら	
\\	あちらに行ってみよう。	あちらに 行[い]ってみよう。	あちら に いって みよう	
\\	に 行[い]ってみよう。			
\\	どきどき	どきどき	どきどき	
\\	彼女に会うとどきどきします。	彼女[かのじょ]に 会[あ]うとどきどきします。	かのじょ に あう と どきどき します	
\\	彼女[かのじょ]に 会[あ]うと
\\	します。			
\\	すき焼き	すき 焼[や]き	すきやき	
\\	昨日の夜はすき焼きを食べた。	昨日[きのう]の 夜[よる]はすき 焼[や]きを 食[た]べた。	きのう の よる は すきやき を たべた	
\\	昨日[きのう]の 夜[よる]は
\\	を 食[た]べた。			
\\	まあまあ	まあまあ	まあまあ	
\\	彼の成績はまあまあです。	彼[かれ]の 成績[せいせき]はまあまあです。	かれ の せいせき は まあまあ です	
\\	彼[かれ]の 成績[せいせき]は
\\	です。			
\\	乾く	乾[かわ]く	かわく	
\\	夏は洗濯物がすぐ乾く。	夏[なつ]は 洗濯物[せんたくもの]がすぐ 乾[かわ]く。	なつ は せんたくもの が すぐ かわく	
\\	夏[なつ]は 洗濯物[せんたくもの]がすぐ
\\	乾杯	乾杯[かんぱい]	かんぱい	
\\	お二人の未来に乾杯しましょう。	お 二人[ふたり]の 未来[みらい]に 乾杯[かんぱい]しましょう。	おふたり の みらい に かんぱい しましょう	
\\	お 二人[ふたり]の 未来[みらい]に
\\	しましょう。			
\\	乾かす	乾[かわ]かす	かわかす	
\\	ぬれた服を乾かした。	ぬれた 服[ふく]を 乾[かわ]かした。	ぬれた ふく を かわかした	
\\	ぬれた 服[ふく]を
\\	マフラー	マフラー	マフラー	
\\	彼女は首にマフラーをまいていた。	彼女[かのじょ]は 首[くび]にマフラーをまいていた。	かのじょ は くび に まふらー を まいて いた	
\\	彼女[かのじょ]は 首[くび]に
\\	をまいていた。			
\\	アクセサリー	アクセサリー	アクセサリー	
\\	このアクセサリーは素敵ね。	このアクセサリーは 素敵[すてき]ね。	この あくせさりー は すてき ね	
\\	この
\\	は 素敵[すてき]ね。			
\\	あくび	あくび	あくび	
\\	父があくびをした。	父[ちち]があくびをした。	ちち が あくび を した	
\\	父[ちち]が
\\	をした。			
\\	からから	からから	からから	
\\	おしゃべりして喉がからからになりました。	おしゃべりして 喉[のど]がからからになりました。	おしゃべり して のど が からから に なりました	
\\	おしゃべりして 喉[のど]が
\\	になりました。			
\\	ふらふら	ふらふら	ふらふら	
\\	私は疲れてふらふらです。	私[わたし]は 疲[つか]れてふらふらです。	わたし は つかれて ふらふら です	
\\	私[わたし]は 疲[つか]れて
\\	です。			
\\	こぼす	こぼす	こぼす	
\\	ソファにワインをこぼした。	ソファにワインをこぼした。	そふぁ に わいん を こぼした	
\\	ソファにワインを
\\	内容	内容[ないよう]	ないよう	
\\	この本の内容を説明してください。	この 本[ほん]の 内容[ないよう]を 説明[せつめい]してください。	この ほん の ないよう を せつめい して ください	
\\	この 本[ほん]の
\\	を 説明[せつめい]してください。			
\\	ランチ	ランチ	ランチ	
\\	一緒にランチに行きませんか。	一緒[いっしょ]にランチに 行[い]きませんか。	いっしょ に らんち に いきません か	
\\	一緒[いっしょ]に
\\	に 行[い]きませんか。			
\\	健康	健康[けんこう]	けんこう	
\\	私は健康な生活を送っています。	私[わたし]は 健康[けんこう]な 生活[せいかつ]を 送[おく]っています。	わたし は けんこう な せいかつ を おくって います	
\\	私[わたし]は
\\	な 生活[せいかつ]を 送[おく]っています。			
\\	レシート	レシート	レシート	
\\	レシートを財布に入れました。	レシートを 財布[さいふ]に 入[い]れました。	れしーと を さいふ に いれました	
\\	を 財布[さいふ]に 入[い]れました。			
\\	再び	再[ふたた]び	ふたたび	
\\	彼は再びここに戻ってきた。	彼[かれ]は 再[ふたた]びここに 戻[もど]ってきた。	かれ は ふたたび ここ に もどって きた	
\\	彼[かれ]は
\\	ここに 戻[もど]ってきた。			
\\	再来年	再来年[さらいねん]	さらいねん	
\\	次のオリンピックは再来年です。	次[つぎ]のオリンピックは 再来年[さらいねん]です。	つぎ の おりんぴっく は さらいねん です	
\\	次[つぎ]のオリンピックは
\\	です。			
\\	チョーク	チョーク	チョーク	
\\	チョークで黒板に字を書きました。	チョークで 黒板[こくばん]に 字[じ]を 書[か]きました。	ちょーく で こくばん に じ を かきました	
\\	で 黒板[こくばん]に 字[じ]を 書[か]きました。			
\\	再来月	再来月[さらいげつ]	さらいげつ	
\\	再来月まで予約で一杯です。	再来月[さらいげつ]まで 予約[よやく]で 一杯[いっぱい]です。	さらいげつ まで よやく で いっぱい です	
\\	まで 予約[よやく]で 一杯[いっぱい]です。			
\\	再来週	再来週[さらいしゅう]	さらいしゅう	
\\	再来週は忙しいです。	再来週[さらいしゅう]は 忙[いそが]しいです。	さらいしゅう は いそがしい です	
\\	は 忙[いそが]しいです。			
\\	ティッシュ	ティッシュ	ティッシュ	
\\	ティッシュを取ってください。	ティッシュを 取[と]ってください。	てぃっしゅ を とって ください	
\\	を 取[と]ってください。			
\\	仮名	仮名[かな]	かな	
\\	お名前に仮名を振ってください。	お 名前[なまえ]に 仮名[かな]を 振[ふ]ってください。	おなまえ に かな を ふって ください	
\\	お 名前[なまえ]に
\\	を 振[ふ]ってください。			
\\	ノック	ノック	ノック	
\\	入る時はドアをノックしてください。	入[はい]る 時[とき]はドアをノックしてください。	はいる とき は どあ を のっく して ください	
\\	入[はい]る 時[とき]はドアを
\\	してください。			
\\	よろしく	よろしく	よろしく	
\\	よろしくお願いします。	よろしくお 願[ねが]いします。	よろしく おねがい します	
\\	お 願[ねが]いします。			
\\	博物館	博物館[はくぶつかん]	はくぶつかん	
\\	昨日、車の博物館に行った。	昨日[きのう]、 車[くるま]の 博物館[はくぶつかん]に 行[い]った。	きのう くるま の はくぶつかん に いった	
\\	昨日[きのう]、 車[くるま]の
\\	に 行[い]った。			
\\	がらがら	がらがら	がらがら	
\\	映画館はがらがらでした。	映画館[えいがかん]はがらがらでした。	えいがかん は がらがら でした	
\\	映画館[えいがかん]は
\\	でした。			
\\	免許証	免許証[めんきょしょう]	めんきょしょう	
\\	免許証を見せてください。	免許証[めんきょしょう]を 見[み]せてください。	めんきょしょう を みせて ください	
\\	を 見[み]せてください。			
\\	伝える	伝[つた]える	つたえる	
\\	みんなにこのことを伝えてください。	みんなにこのことを 伝[つた]えてください。	みんな に この こと を つたえて ください	
\\	みんなにこのことを
\\	ください。			
\\	ぎらぎら	ぎらぎら	ぎらぎら	
\\	太陽がぎらぎらしている。	太陽[たいよう]がぎらぎらしている。	たいよう が ぎらぎら して いる	
\\	太陽[たいよう]が
\\	している。			
\\	そっち	そっち	そっち	
\\	そっちが私の部屋です。	そっちが 私[わたし]の 部屋[へや]です。	そっち が わたし の へや です	
\\	が 私[わたし]の 部屋[へや]です。			
\\	卵	卵[たまご]	たまご	
\\	ニワトリは卵を産みます。	ニワトリは 卵[たまご]を 産[う]みます。	にわとり は たまご を うみます	
\\	ニワトリは
\\	を 産[う]みます。			
\\	ピンポン	ピンポン	ピンポン	
\\	昨日の夕方、友達とピンポンをした。	昨日[きのう]の 夕方[ゆうがた]、 友達[ともだち]とピンポンをした。	きのう の ゆうがた ともだち と ぴんぽん を した	
\\	昨日[きのう]の 夕方[ゆうがた]、 友達[ともだち]と
\\	をした。			
\\	刺す	刺[さ]す	さす	
\\	虫に腕を刺されました。	虫[むし]に 腕[うで]を 刺[さ]されました。	むし に うで を さされました	
\\	虫[むし]に 腕[うで]を
\\	刺身	刺身[さしみ]	さしみ	
\\	私は刺身は食べません。	私[わたし]は 刺身[さしみ]は 食[た]べません。	わたし は さしみ は たべません	
\\	私[わたし]は
\\	は 食[た]べません。			
\\	ぺこぺこ	ぺこぺこ	ぺこぺこ	
\\	私はおなかがぺこぺこです。	私[わたし]はおなかがぺこぺこです。	わたし は おなか が ぺこぺこ です	
\\	私[わたし]はおなかが
\\	です。			
\\	倒れる	倒[たお]れる	たおれる	
\\	強風で木が倒れた。	強風[きょうふう]で 木[き]が 倒[たお]れた。	きょうふう で き が たおれた	
\\	強風[きょうふう]で 木[き]が
\\	倒す	倒[たお]す	たおす	
\\	そのスキーヤーはポールを倒した。	そのスキーヤーはポールを 倒[たお]した。	その すきーやー は ぽーる を たおした	
\\	そのスキーヤーはポールを
\\	傾向	傾向[けいこう]	けいこう	
\\	最近の若者は難しい本を読まない傾向がある。	最近[さいきん]の 若者[わかもの]は 難[むずか]しい 本[ほん]を 読[よ]まない 傾向[けいこう]がある。	さいきん の わかもの は むずかしい ほん を よまない けいこう が ある	
\\	最近[さいきん]の 若者[わかもの]は 難[むずか]しい 本[ほん]を 読[よ]まない
\\	がある。			
\\	ぺらぺら	ぺらぺら	ぺらぺら	
\\	彼女はドイツ語がぺらぺらです。	彼女[かのじょ]はドイツ 語[ご]がぺらぺらです。	かのじょ は どいつご が ぺらぺら です	
\\	彼女[かのじょ]はドイツ 語[ご]が
\\	です。			
\\	主張	主張[しゅちょう]	しゅちょう	
\\	上司は私の主張を受け入れた。	上司[じょうし]は 私[わたし]の 主張[しゅちょう]を 受[う]け 入[い]れた。	じょうし は わたし の しゅちょう を うけいれた	
\\	上司[じょうし]は 私[わたし]の
\\	を 受[う]け 入[い]れた。			
\\	レインコート	レインコート	レインコート	
\\	雨なのでレインコートを着ました。	雨[あめ]なのでレインコートを 着[き]ました。	あめ な の で れいんこーと を きました	
\\	雨[あめ]なので
\\	を 着[き]ました。			
\\	アイス	アイス	アイス	
\\	暑いのでアイスを食べました。	暑[あつ]いのでアイスを 食[た]べました。	あつい の で あいす を たべました	
\\	暑[あつ]いので
\\	を 食[た]べました。			
\\	丸い	丸[まる]い	まるい	
\\	地球は丸い。	地球[ちきゅう]は 丸[まる]い。	ちきゅう は まるい。	
\\	地球[ちきゅう]は
\\	丸	丸[まる]	まる	
\\	紙に大きな丸を書きました。	紙[かみ]に 大[おお]きな 丸[まる]を 書[か]きました。	かみ に おおき な まる を かきました	
\\	紙[かみ]に 大[おお]きな
\\	を 書[か]きました。			
\\	いつごろ	いつごろ	いつごろ	
\\	いつごろ夏休みを取りますか。	いつごろ 夏休[なつやす]みを 取[と]りますか。	いつごろ なつやすみ を とります か	
\\	夏休[なつやす]みを 取[と]りますか。			
\\	ウィスキー	ウィスキー	ウィスキー	
\\	このウィスキーは強いよ。	このウィスキーは 強[つよ]いよ。	この うぃすきー は つよい よ	
\\	この
\\	は 強[つよ]いよ。			
\\	エアメール	エアメール	エアメール	
\\	彼にエアメールを送りました。	彼[かれ]にエアメールを 送[おく]りました。	かれ に えあめーる を おくりました	
\\	彼[かれ]に
\\	を 送[おく]りました。			
\\	シャープペンシル	シャープペンシル	シャープペンシル	
\\	私のシャープペンシルがありません。	私[わたし]のシャープペンシルがありません。	わたし の しゃーぷぺんしる が ありません	
\\	私[わたし]の
\\	がありません。			
\\	偉い	偉[えら]い	えらい	
\\	彼は偉い学者です。	彼[かれ]は 偉[えら]い 学者[がくしゃ]です。	かれ は えらい がくしゃ です	
\\	彼[かれ]は
\\	学者[がくしゃ]です。			
\\	セロテープ	セロテープ	セロテープ	
\\	セロテープはありますか。	セロテープはありますか。	せろてーぷ は あります か	
\\	はありますか。			
\\	やけど	やけど	やけど	
\\	彼は手にやけどをしました。	彼[かれ]は 手[て]にやけどをしました。	かれ は て に やけど を しました	
\\	彼[かれ]は 手[て]に
\\	をしました。			
\\	お釣り	お 釣[つ]り	おつり	
\\	母はお釣りを確かめた。	母[はは]はお 釣[つ]りを 確[たし]かめた。	はは は おつり を たしかめた	
\\	母[はは]は
\\	を 確[たし]かめた。			
\\	アイスコーヒー	アイスコーヒー	アイスコーヒー	
\\	アイスコーヒーをください。	アイスコーヒーをください。	あいすこーひー を ください	
\\	をください。			
\\	あっち	あっち	あっち	
\\	郵便局はあっちです。	郵便局[ゆうびんきょく]はあっちです。	ゆうびんきょく は あっち です	
\\	郵便局[ゆうびんきょく]は
\\	です。			
\\	あんなに	あんなに	あんなに	
\\	あんなにいい人はいません。	あんなにいい 人[ひと]はいません。	あんなに いい ひと は いません	
\\	いい 人[ひと]はいません。			
\\	一生懸命	一生懸命[いっしょうけんめい]	いっしょうけんめい	
\\	彼は毎日一生懸命働いている。	彼[かれ]は 毎日[まいにち] 一生懸命[いっしょうけんめい] 働[はたら]いている。	かれ は まいにち いっしょうけんめい はたらいて いる	
\\	彼[かれ]は 毎日[まいにち]
\\	働[はたら]いている。			
\\	休憩	休憩[きゅうけい]	きゅうけい	
\\	少し休憩しましょう。	少[すこ]し 休憩[きゅうけい]しましょう。	すこし きゅうけい しましょう	
\\	少[すこ]し
\\	しましょう。			
\\	いたずら	いたずら	いたずら	
\\	その子はいたずらが大好きだ。	その 子[こ]はいたずらが 大好[だいす]きだ。	その こ は いたずら が だいすき だ	
\\	その 子[こ]は
\\	が 大好[だいす]きだ。			
\\	化粧	化粧[けしょう]	けしょう	
\\	彼女は化粧が上手い。	彼女[かのじょ]は 化粧[けしょう]が 上手[うま]い。	かのじょ は けしょう が うまい	
\\	彼女[かのじょ]は
\\	が 上手[うま]い。			
\\	ウェートレス	ウェートレス	ウェートレス	
\\	ウェートレスを呼んだ。	ウェートレスを 呼[よ]んだ。	うぇーとれす を よんだ	
\\	を 呼[よ]んだ。			
\\	おばさん	おばさん	おばさん	
\\	おばさん一家が遊びに来ました。	おばさん 一家[いっか]が 遊[あそ]びに 来[き]ました。	おばさん いっか が あそび に きました	
\\	一家[いっか]が 遊[あそ]びに 来[き]ました。			
\\	くし	くし	くし	
\\	彼はくしで髪をとかした。	彼[かれ]はくしで 髪[かみ]をとかした。	かれ は くし で かみ を とかした	
\\	彼[かれ]は
\\	で 髪[かみ]をとかした。			
\\	匂い	匂[にお]い	におい	
\\	いい匂いがする。	いい 匂[にお]いがする。	いい におい が する	
\\	いい
\\	がする。			
\\	クリーニング	クリーニング	クリーニング	
\\	スーツをクリーニングに出しました。	スーツをクリーニングに 出[だ]しました。	すーつ を くりーにんぐ に だしました	
\\	スーツを
\\	に 出[だ]しました。			
\\	ごちそう	ごちそう	ごちそう	
\\	テーブルの上にごちそうが並んでいる。	テーブルの 上[うえ]にごちそうが 並[なら]んでいる。	てーぶる の うえ に ごちそう が ならんで いる	
\\	テーブルの 上[うえ]に
\\	が 並[なら]んでいる。			
\\	お菓子	お 菓子[かし]	おかし	
\\	みんなにお菓子をあげましょう。	みんなにお 菓子[かし]をあげましょう。	みんな に おかし を あげましょう	
\\	みんなに
\\	をあげましょう。			
\\	お辞儀	お 辞儀[じぎ]	おじぎ	
\\	皆、社長にお辞儀をした。	皆[みんな]、 社長[しゃちょう]にお 辞儀[じぎ]をした。	みんな しゃちょう に おじぎ を した	
\\	皆[みんな]、 社長[しゃちょう]に
\\	をした。			
\\	ごちそうする	ごちそうする	ごちそうする	
\\	今日の夕食は私がごちそうします。	今日[きょう]の 夕食[ゆうしょく]は 私[わたし]がごちそうします。	きょう の ゆうしょく は わたし が ごちそう します	
\\	今日[きょう]の 夕食[ゆうしょく]は 私[わたし]が
\\	万年筆	万年筆[まんねんひつ]	まんねんひつ	
\\	父に万年筆をプレゼントしました。	父[ちち]に 万年筆[まんねんひつ]をプレゼントしました。	ちち に まんねんひつ を ぷれぜんと しました	
\\	父[ちち]に
\\	をプレゼントしました。			
\\	こぼれる	こぼれる	こぼれる	
\\	彼女の目から涙がこぼれた。	彼女[かのじょ]の 目[め]から 涙[なみだ]がこぼれた。	かのじょ の め から なみだ が こぼれた	
\\	彼女[かのじょ]の 目[め]から 涙[なみだ]が
\\	こんなに	こんなに	こんなに	
\\	こんなに面白い本は初めて読んだ。	こんなに 面白[おもしろ]い 本[ほん]は 初[はじ]めて 読[よ]んだ。	こんなに おもしろい ほん は はじめて よんだ	
\\	面白[おもしろ]い 本[ほん]は 初[はじ]めて 読[よ]んだ。			
\\	ざあざあ	ざあざあ	ざあざあ	
\\	雨がざあざあ降っている。	雨[あめ]がざあざあ 降[ふ]っている。	あめ が ざあざあ ふって いる	
\\	雨[あめ]が
\\	降[ふ]っている。			
\\	包む	包[つつ]む	つつむ	
\\	プレゼントをきれいな紙で包みました。	プレゼントをきれいな 紙[かみ]で 包[つつ]みました。	ぷれぜんと を きれい な かみ で つつみました	
\\	プレゼントをきれいな 紙[かみ]で
\\	ジャガイモ	ジャガイモ	ジャガイモ	
\\	ポテトサラダを作るから、ジャガイモを買ってきて。	ポテトサラダを 作[つく]るから、ジャガイモを 買[か]ってきて。	ぽてと さらだ を つくる から じゃがいも を かって きて	
\\	ポテトサラダを 作[つく]るから、
\\	を 買[か]ってきて。			
\\	ソファー	ソファー	ソファー	
\\	このソファーは気持ちがいい。	このソファーは 気持[きも]ちがいい。	この そふぁー は きもち が いい	
\\	この
\\	は 気持[きも]ちがいい。			
\\	出掛ける	出掛[でか]ける	でかける	
\\	主人はもう出掛けました。	主人[しゅじん]はもう 出掛[でか]けました。	しゅじん は もう でかけました	
\\	主人[しゅじん]はもう
\\	たばこ	たばこ	たばこ	
\\	彼はたばこを吸いません。	彼[かれ]はたばこを 吸[す]いません。	かれ は たばこ を すいません	
\\	彼[かれ]は
\\	を 吸[す]いません。			
\\	ちょうど	ちょうど	ちょうど	
\\	値段はちょうど3万円です。	値段[ねだん]はちょうど 3万円[さんまんえん]です。	ねだん は ちょうど さんまんえん です	
\\	値段[ねだん]は
\\	3万円[さんまんえん]です。			
\\	ちょっと	ちょっと	ちょっと	
\\	もうちょっと塩を入れて。	もうちょっと 塩[しお]を 入[い]れて。	もう ちょっと しお を いれて	
\\	もう
\\	塩[しお]を 入[い]れて。			
\\	お巡りさん	お 巡[まわ]りさん	おまわりさん	
\\	あのお巡りさんに道を聞きましょう。	あのお 巡[まわ]りさんに 道[みち]を 聞[き]きましょう。	あの おまわりさん に みち を ききましょう	
\\	あの
\\	に 道[みち]を 聞[き]きましょう。			
\\	冗談	冗談[じょうだん]	じょうだん	
\\	冗談は止めてください。	冗談[じょうだん]は 止[や]めてください。	じょうだん は やめて ください	
\\	は 止[や]めてください。			
\\	つく	つく	つく	
\\	部屋の電気がついています。	部屋[へや]の 電気[でんき]がついています。	へや の でんき が ついて います	
\\	部屋[へや]の 電気[でんき]が
\\	丁寧	丁寧[ていねい]	ていねい	
\\	彼女はいつも丁寧に仕事をします。	彼女[かのじょ]はいつも 丁寧[ていねい]に 仕事[しごと]をします。	かのじょ は いつも ていねい に しごと を します	
\\	彼女[かのじょ]はいつも
\\	に 仕事[しごと]をします。			
\\	てんぷら	てんぷら	てんぷら	
\\	私はえびのてんぷらが好きです。	私[わたし]はえびのてんぷらが 好[す]きです。	わたし は えび の てんぷら が すき です	
\\	私[わたし]はえびの
\\	が 好[す]きです。			
\\	とんとん	とんとん	とんとん	
\\	母の肩をとんとんたたいた。	母[はは]の 肩[かた]をとんとんたたいた。	はは の かた を とんとん たたいた	
\\	母[はは]の 肩[かた]を
\\	たたいた。			
\\	にんじん	にんじん	にんじん	
\\	私はにんじんが嫌いだ。	私[わたし]はにんじんが 嫌[きら]いだ。	わたし は にんじん が きらい だ	
\\	私[わたし]は
\\	が 嫌[きら]いだ。			
\\	この頃	この 頃[ごろ]	このごろ	
\\	この頃、よく肩が凝る。	この 頃[ごろ]、よく 肩[かた]が 凝[こ]る。	このごろ よく かた が こる	
\\	、よく 肩[かた]が 凝[こ]る。			
\\	ぬるい	ぬるい	ぬるい	
\\	風呂がぬるかった。	風呂[ふろ]がぬるかった。	ふろ が ぬるかった	
\\	風呂[ふろ]が
\\	パーセント	パーセント	パーセント	
\\	この村の80パーセントは老人です。	この 村[むら]の 80[はちじゅっ]パーセントは 老人[ろうじん]です。	この むら の はちじゅっぱーせんと は ろうじん です	
\\	この 村[むら]の 80[はちじゅっ]
\\	は 老人[ろうじん]です。			
\\	ひざ	ひざ	ひざ	
\\	スキーでひざを怪我しました。	スキーでひざを 怪我[けが]しました。	すきー で ひざ を けがしました	
\\	スキーで
\\	を 怪我[けが]しました。			
\\	ひじ	ひじ	ひじ	
\\	彼はひじに怪我をした。	彼[かれ]はひじに 怪我[けが]をした。	かれ は ひじ に けが を した	
\\	彼[かれ]は
\\	に 怪我[けが]をした。			
\\	仕舞う	仕舞[しま]う	しまう	
\\	彼女は大切な書類を机に仕舞った。	彼女[かのじょ]は 大切[たいせつ]な 書類[しょるい]を 机[つくえ]に 仕舞[しま]った。	かのじょ は たいせつ な しょるい を つくえ に しまった。	
\\	彼女[かのじょ]は 大切[たいせつ]な 書類[しょるい]を 机[つくえ]に
\\	ひも	ひも	ひも	
\\	靴のひもが切れた。	靴[くつ]のひもが 切[き]れた。	くつ の ひも が きれた	
\\	靴[くつ]の
\\	が 切[き]れた。			
\\	まく	まく	まく	
\\	日本では、二月に豆をまく行事がある。	日本[にほん]では、 二月[にがつ]に 豆[まめ]をまく 行事[ぎょうじ]がある。	にほん で は にがつ に まめ を まく ぎょうじ が ある	
\\	日本[にほん]では、 二月[にがつ]に 豆[まめ]を
\\	行事[ぎょうじ]がある。			
\\	やかん	やかん	やかん	
\\	やかんでお湯を沸かしました。	やかんでお 湯[ゆ]を 沸[わ]かしました。	やかん で おゆ を わかしました	
\\	でお 湯[ゆ]を 沸[わ]かしました。			
\\	丈夫	丈夫[じょうぶ]	じょうぶ	
\\	祖母は身体が丈夫だ。	祖母[そぼ]は 身体[からだ]が 丈夫[じょうぶ]だ。	そぼ は からだ が じょうぶ だ	
\\	祖母[そぼ]は 身体[からだ]が
\\	だ。			
\\	又	又[また]	また	
\\	明日、また来ます。	明日[あした]、また 来[き]ます。	あした また きます	
\\	明日[あした]、
\\	来[き]ます。			
\\	お祖父さん	お 祖父[じい]さん	おじいさん	
\\	私のお祖父さんは毎日散歩します。	私[わたし]のお 祖父[じい]さんは 毎日散歩[まいにち さんぽ]します。	わたし の おじいさん は まいにち さんぽ します 。	
\\	私[わたし]の
\\	は 毎日散歩[まいにち さんぽ]します。			
\\	お祖母さん	お 祖母[ばあ]さん	おばあさん	
\\	彼女はお祖母さんと住んでいる。	彼女[かのじょ]はお 祖母[ばあ]さんと 住[す]んでいる。	かのじょ は おばあさん と すん でいる 。	
\\	彼女[かのじょ]は
\\	と 住[す]んでいる。			
\\	ようこそ	ようこそ	ようこそ	
\\	日本へようこそ。	日本[にっぽん]へようこそ。	にっぽん へ ようこそ	
\\	日本[にっぽん]へ
\\	ウェーター	ウェーター	ウェーター	
\\	ウェーターが水を運んできました。	ウェーターが 水[みず]を 運[はこ]んできました。	うぇーたー が みず を はこんで きました	
\\	が 水[みず]を 運[はこ]んできました。			
\\	そんなに	そんなに	そんなに	
\\	一度にそんなにたくさんはできない。	一度[いちど]にそんなにたくさんはできない。	いちど に そんなに たくさん は できない	
\\	一度[いちど]に
\\	たくさんはできない。			
\\	傘	傘[かさ]	かさ	
\\	電車に傘を忘れた。	電車[でんしゃ]に 傘[かさ]を 忘[わす]れた。	でんしゃ に かさ を わすれた	
\\	電車[でんしゃ]に
\\	を 忘[わす]れた。			
\\	お嬢さん	お 嬢[じょう]さん	おじょうさん	
\\	お嬢さんはおいくつですか。	お 嬢[じょう]さんはおいくつですか。	おじょうさん は おいくつ です か	
\\	はおいくつですか。			
\\	伴う	伴[ともな]う	ともなう	
\\	その仕事は危険を伴う。	その 仕事[しごと]は 危険[きけん]を 伴[ともな]う。	その しごと は きけん を ともなう	
\\	その 仕事[しごと]は 危険[きけん]を
\\	元旦	元旦[がんたん]	がんたん	
\\	元旦に彼から年賀状が来た。	元旦[がんたん]に 彼[かれ]から 年賀状[ねんがじょう]が 来[き]た。	がんたん に かれ から ねんがじょう が きた 。	
\\	に 彼[かれ]から 年賀状[ねんがじょう]が 来[き]た。			
\\	半袖	半袖[はんそで]	はんそで	
\\	今日は、半袖のシャツを着よう。	今日[きょう]は、 半袖[はんそで]のシャツを 着[き]よう。	きょう は 、 はんそで の しゃつ を きよう 。	
\\	今日[きょう]は、
\\	のシャツを 着[き]よう。			
\\	凄い	凄[すご]い	すごい	
\\	凄い雨になった。	凄[すご]い 雨[あめ]になった。	すごい あめ に なった	
\\	雨[あめ]になった。			
\\	剃る	剃[そ]る	そる	
\\	父は毎日髭を剃ります。	父[ちち]は 毎日髭[まいにち ひげ]を 剃[そ]ります。	ちち は まいにち ひげ を そります	
\\	父[ちち]は 毎日髭[まいにち ひげ]を
\\	一時	一時[いちじ]	いちじ	
\\	お店は一時休業になったんだ。	お 店[みせ]は 一時[いちじ] 休業[きゅうぎょう]になったんだ。	おみせ は いちじ きゅうぎょう に なった ん だ	
\\	お 店[みせ]は
\\	休業[きゅうぎょう]になったんだ。			
\\	先に	先[さき]に	さきに	
\\	では、私たちは先に出発します。	では、 私[わたし]たちは 先[さき]に 出発[しゅっぱつ]します。	では わたしたち は さきに しゅっぱつ します	
\\	では、 私[わたし]たちは
\\	出発[しゅっぱつ]します。			
\\	先日	先日[せんじつ]	せんじつ	
\\	先日の件はどうなりましたか。	先日[せんじつ]の 件[けん]はどうなりましたか。	せんじつ の けん は どう なりました か	
\\	の 件[けん]はどうなりましたか。			
\\	今日	今日[こんにち]	こんにち	
\\	今日の日本の若者は欧米化している。	今日[こんにち]の 日本[にほん]の 若者[わかもの]は 欧米化[おうべいか]している。	こんにち の にほん の わかもの は おうべいか して いる	
\\	の 日本[にほん]の 若者[わかもの]は 欧米化[おうべいか]している。			
\\	今ごろ	今[いま]ごろ	いまごろ	
\\	今ごろそんなこと言わないで。	今[いま]ごろそんなこと 言[い]わないで。	いまごろ そんな こと いわない で	
\\	そんなこと 言[い]わないで。			
\\	ケース	ケース	ケース	
\\	このケースには
\\	が50枚入ります。	このケースには
\\	が50 枚入[まい はい]ります。	この けーす に は 
\\	が 
\\	まい はいります	
\\	この
\\	には
\\	が50 枚入[まい はい]ります。			
\\	今にも	今[いま]にも	いまにも	
\\	今にも雨が降りそうですね。	今[いま]にも 雨[あめ]が 降[ふ]りそうですね。	いまにも あめ が ふりそう です ね	
\\	雨[あめ]が 降[ふ]りそうですね。			
\\	今に	今[いま]に	いまに	
\\	あなたも今に分かるでしょう。	あなたも 今[いま]に 分[わ]かるでしょう。	あなた も いまに わかる でしょう	
\\	あなたも
\\	分[わ]かるでしょう。			
\\	イメージ	イメージ	イメージ	
\\	彼には清潔なイメージがあるね。	彼[かれ]には 清潔[せいけつ]なイメージがあるね。	かれ に は せいけつ な いめーじ が ある ね	
\\	彼[かれ]には 清潔[せいけつ]な
\\	があるね。			
\\	上がる	上[あ]がる	あがる	
\\	今日は仕事が早く上がったんだ。	今日[きょう]は 仕事[しごと]が 早[はや]く 上[あ]がったんだ。	きょう は しごと が はやく あがった ん だ	
\\	今日[きょう]は 仕事[しごと]が 早[はや]く
\\	んだ。			
\\	上がる	上[あ]がる	あがる	
\\	冷めないうちにどうぞお上がり下さい。	冷[さ]めないうちにどうぞお 上[あ]がり 下[くだ]さい。	さめない うち に どうぞ おあがり ください	
\\	冷[さ]めないうちにどうぞ
\\	下[くだ]さい。			
\\	せい	せい	せい	
\\	自分の失敗を他人のせいにするな。	自分[じぶん]の 失敗[しっぱい]を 他人[たにん]のせいにするな。	じぶん の しっぱい を たにん の せい に する な	
\\	自分[じぶん]の 失敗[しっぱい]を 他人[たにん]の
\\	にするな。			
\\	上	上[じょう]	じょう	
\\	握り寿司の上を注文したよ。	握[にぎ]り 寿司[ずし]の 上[じょう]を 注文[ちゅうもん]したよ。	にぎりずし の じょう を ちゅうもん した よ	
\\	握[にぎ]り 寿司[ずし]の
\\	を 注文[ちゅうもん]したよ。			
\\	上	上[かみ]	かみ	
\\	お上は何を考えてるんだろうね。	お 上[かみ]は 何[なに]を 考[かんが]えてるんだろうね。	おかみ は なに を かんがえて る ん だろう ね	
\\	お
\\	は 何[なに]を 考[かんが]えてるんだろうね。			
\\	下さる	下[くだ]さる	くださる	
\\	先生が手紙を下さいました。	先生[せんせい]が 手紙[てがみ]を 下[くだ]さいました。	せんせい が てがみ を くださいました	
\\	先生[せんせい]が 手紙[てがみ]を
\\	下水	下水[げすい]	げすい	
\\	その道は今、下水の工事をしているよ。	その 道[みち]は 今[いま]、 下水[げすい]の 工事[こうじ]をしているよ。	その みち は いま げすい の こうじ を して いる よ	
\\	その 道[みち]は 今[いま]、
\\	の 工事[こうじ]をしているよ。			
\\	上下	上下[じょうげ]	じょうげ	
\\	この服は上下セットで買いました。	この 服[ふく]は 上下[じょうげ]セットで 買[か]いました。	この ふく は じょうげ せっと で かいました	
\\	この 服[ふく]は
\\	セットで 買[か]いました。			
\\	上下	上下[うえした]	うえした	
\\	彼は上下おそろいの服を着ていますね。	彼[かれ]は 上下[うえした]おそろいの 服[ふく]を 着[き]ていますね。	かれ は うえした おそろい の ふく を きています ね 。	
\\	彼[かれ]は
\\	おそろいの 服[ふく]を 着[き]ていますね。			
\\	下	下[げ]	げ	
\\	彼女の成績は上の下です。	彼女[かのじょ]の 成績[せいせき]は 上[じょう]の 下[げ]です。	かのじょ の せいせき は じょう の げ です	
\\	彼女[かのじょ]の 成績[せいせき]は 上[じょう]の
\\	です。			
\\	いつも	いつも	いつも	
\\	彼女の様子がいつもと違う。	彼女[かのじょ]の 様子[ようす]がいつもと 違[ちが]う。	かのじょ の ようす が いつも と ちがう	
\\	彼女[かのじょ]の 様子[ようす]が
\\	と 違[ちが]う。			
\\	内	内[うち]	うち	
\\	心の内をお話し下さい。	心[こころ]の 内[うち]をお 話[はな]し 下[くだ]さい。	こころ の うち を お はなし ください	
\\	心[こころ]の
\\	をお 話[はな]し 下[くだ]さい。			
\\	一体	一体[いったい]	いったい	
\\	一体何事ですか。	一体[いったい] 何事[なにごと]ですか。	いったい なにごと です か	
\\	何事[なにごと]ですか。			
\\	きっかけ	きっかけ	きっかけ	
\\	大学に通うのが上京のきっかけでした。	大学[だいがく]に 通[かよ]うのが 上京[じょうきょう]のきっかけでした。	だいがく に かよう の が じょうきょう の きっかけ でした	
\\	大学[だいがく]に 通[かよ]うのが 上京[じょうきょう]の
\\	でした。			
\\	出力	出力[しゅつりょく]	しゅつりょく	
\\	このプリンターで写真を出力できます。	このプリンターで 写真[しゃしん]を 出力[しゅつりょく]できます。	この ぷりんたー で しゃしん を しゅつりょく できます	
\\	このプリンターで 写真[しゃしん]を
\\	できます。			
\\	すっかり	すっかり	すっかり	
\\	買い物をすっかり忘れていた。	買[か]い 物[もの]をすっかり 忘[わす]れていた。	かいもの を すっかり わすれて いた	
\\	買[か]い 物[もの]を
\\	忘[わす]れていた。			
\\	シリーズ	シリーズ	シリーズ	
\\	このシリーズは50話もあるそうよ。	このシリーズは 50話[ごじゅうわ]もあるそうよ。	この しりーず は ごじゅうわ も ある そう よ	
\\	この
\\	は 50話[ごじゅうわ]もあるそうよ。			
\\	エイズ	エイズ	エイズ	
\\	その国ではエイズが大きな社会問題ね。	その 国[くに]ではエイズが 大[おお]きな 社会問題[しゃかい もんだい]ね。	その くに で は えいず が おおき な しゃかい もんだい ね	
\\	その 国[くに]では
\\	が 大[おお]きな 社会問題[しゃかい もんだい]ね。			
\\	上達	上達[じょうたつ]	じょうたつ	
\\	短い時間にずいぶん上達しましたね。	短[みじか]い 時間[じかん]にずいぶん 上達[じょうたつ]しましたね。	みじかい じかん に ずいぶん じょうたつ しました ね	
\\	短[みじか]い 時間[じかん]にずいぶん
\\	しましたね。			
\\	一家	一家[いっか]	いっか	
\\	あの一家は仲がいい。	あの 一家[いっか]は 仲[なか]がいい。	あの いっか は なか が いい	
\\	あの
\\	は 仲[なか]がいい。			
\\	元来	元来[がんらい]	がんらい	
\\	彼は元来、真面目な人です。	彼[かれ]は 元来[がんらい]、 真面目[まじめ]な 人[ひと]です。	かれ は がんらい まじめ な ひと です	
\\	彼[かれ]は
\\	、 真面目[まじめ]な 人[ひと]です。			
\\	元日	元日[がんじつ]	がんじつ	
\\	元日には初詣でに行きます。	元日[がんじつ]には 初詣[はつもう]でに 行[い]きます。	がんじつ に は はつもうで に いきます	
\\	には 初詣[はつもう]でに 行[い]きます。			
\\	エンジン	エンジン	エンジン	
\\	車のエンジンが調子悪い。	車[くるま]のエンジンが 調子悪[ちょうし わる]い。	くるま の えんじん が ちょうし わるい	
\\	車[くるま]の
\\	が 調子悪[ちょうし わる]い。			
\\	お中元	お 中元[ちゅうげん]	おちゅうげん	
\\	お中元にビールが届きました。	お 中元[ちゅうげん]にビールが 届[とど]きました。	おちゅうげん に びーる が とどきました	
\\	にビールが 届[とど]きました。			
\\	元気	元気[げんき]	げんき	
\\	彼女は最近元気がないですね。	彼女[かのじょ]は 最近[さいきん] 元気[げんき]がないですね。	かのじょ は さいきん げんき が ない です ね	
\\	彼女[かのじょ]は 最近[さいきん]
\\	がないですね。			
\\	クラス	クラス	クラス	
\\	彼はビジネスクラスに乗ったの。	彼[かれ]はビジネスクラスに 乗[の]ったの。	かれ は びじねすくらす に のった の	
\\	彼[かれ]はビジネス
\\	に 乗[の]ったの。			
\\	スピード	スピード	スピード	
\\	カーブを曲がるときはスピードを落としましょう。	カーブを 曲[ま]がるときはスピードを 落[お]としましょう。	かーぶ を まがる とき は すぴーど を おとしましょう	
\\	カーブを 曲[ま]がるときは
\\	を 落[お]としましょう。			
\\	一昨年	一昨年[いっさくねん]	いっさくねん	
\\	大学卒業は一昨年です。	大学卒業[だいがく そつぎょう]は 一昨年[いっさくねん]です。	だいがく そつぎょう は いっさくねん です	
\\	大学卒業[だいがく そつぎょう]は
\\	です。			
\\	一昨日	一昨日[いっさくじつ]	いっさくじつ	
\\	一昨日、お電話を差し上げました。	一昨日[いっさくじつ]、お 電話[でんわ]を 差[さ]し 上[あ]げました。	いっさくじつ おでんわ を さしあげました	
\\	、お 電話[でんわ]を 差[さ]し 上[あ]げました。			
\\	一向に	一向[いっこう]に	いっこうに	
\\	彼は一向に興味を示さないの。	彼[かれ]は 一向[いっこう]に 興味[きょうみ]を 示[しめ]さないの。	かれ は いっこうに きょうみ を しめさない の	
\\	彼[かれ]は
\\	興味[きょうみ]を 示[しめ]さないの。			
\\	いかに	いかに	いかに	
\\	彼に会えば、彼がいかに良い人か分かります。	彼[かれ]に 会[あ]えば、 彼[かれ]がいかに 良[い]い 人[ひと]か 分[わ]かります。	かれ に あえば かれ が いかに いい ひと か わかります	
\\	彼[かれ]に 会[あ]えば、 彼[かれ]が
\\	良[い]い 人[ひと]か 分[わ]かります。			
\\	円高	円高[えんだか]	えんだか	
\\	円高の影響で海外製品が安く買えますよ。	円高[えんだか]の 影響[えいきょう]で 海外製品[かいがい せいひん]が 安[やす]く 買[か]えますよ。	えんだか の えいきょう で かいがい せいひん が やすく かえます よ	
\\	の 影響[えいきょう]で 海外製品[かいがい せいひん]が 安[やす]く 買[か]えますよ。			
\\	スタート	スタート	スタート	
\\	マラソンは雨の中でスタートしたの。	マラソンは 雨[あめ]の 中[なか]でスタートしたの。	まらそん は あめ の なか で すたーと した の	
\\	マラソンは 雨[あめ]の 中[なか]で
\\	したの。			
\\	初日	初日[しょにち]	しょにち	
\\	会議の初日に市長がスピーチをしたよ。	会議[かいぎ]の 初日[しょにち]に 市長[しちょう]がスピーチをしたよ。	かいぎ の しょにち に しちょう が すぴーち を した よ	
\\	会議[かいぎ]の
\\	に 市長[しちょう]がスピーチをしたよ。			
\\	お前	お 前[まえ]	おまえ	
\\	お前の言うことは信じられない。	お 前[まえ]の 言[い]うことは 信[しん]じられない。	おまえ の いう こと は しんじられ ない	
\\	の 言[い]うことは 信[しん]じられない。			
\\	前後	前後[ぜんご]	ぜんご	
\\	そちらに着くのは6時前後です。	そちらに 着[つ]くのは 6時[ろくじ] 前後[ぜんご]です。	そちら に つく の は ろくじ ぜんご です	
\\	そちらに 着[つ]くのは 6時[ろくじ]
\\	です。			
\\	セット	セット	セット	
\\	このメニューはサラダと飲み物がセットになっています。	このメニューはサラダと 飲[の]み 物[もの]がセットになっています。	この めにゅー は さらだ と のみもの が せっと に なって います	
\\	このメニューはサラダと 飲[の]み 物[もの]が
\\	になっています。			
\\	アップ	アップ	アップ	
\\	写真をアップで撮ろう。	写真[しゃしん]をアップで 撮[と]ろう。	しゃしん を あっぷ で とろう	
\\	写真[しゃしん]を
\\	で 撮[と]ろう。			
\\	一見	一見[いっけん]	いっけん	
\\	彼は一見サラリーマン風ですね。	彼[かれ]は 一見[いっけん]サラリーマン 風[ふう]ですね。	かれ は いっけん さらりーまんふう です ね	
\\	彼[かれ]は
\\	サラリーマン 風[ふう]ですね。			
\\	エネルギー	エネルギー	エネルギー	
\\	若者たちはエネルギーにあふれていますね。	若者[わかもの]たちはエネルギーにあふれていますね。	わかものたち は えねるぎー に あふれて います ね	
\\	若者[わかもの]たちは
\\	にあふれていますね。			
\\	下書き	下書[したが]き	したがき	
\\	今、論文の下書きをしているところです。	今[いま]、 論文[ろんぶん]の 下書[したが]きをしているところです。	いま ろんぶん の したがき を して いる ところ です	
\\	今[いま]、 論文[ろんぶん]の
\\	をしているところです。			
\\	会	会[かい]	かい	
\\	会は午後9時に終わりました。	会[かい]は 午後9時[ごご くじ]に 終[お]わりました。	かい は ごご くじ に おわりました	
\\	は 午後9時[ごご くじ]に 終[お]わりました。			
\\	かえって	かえって	かえって	
\\	そんなことをしたら、かえってよくないよ。	そんなことをしたら、かえってよくないよ。	そんなことをしたら、かえってよくないよ。	
\\	そんなことをしたら、
\\	よくないよ。			
\\	会員	会員[かいいん]	かいいん	
\\	彼女はクラブの会員です。	彼女[かのじょ]はクラブの 会員[かいいん]です。	かのじょ は くらぶ の かいいん です	
\\	彼女[かのじょ]はクラブの
\\	です。			
\\	一員	一員[いちいん]	いちいん	
\\	彼は野球部の一員です。	彼[かれ]は 野球部[やきゅうぶ]の 一員[いちいん]です。	かれ は やきゅうぶ の いちいん です	
\\	彼[かれ]は 野球部[やきゅうぶ]の
\\	です。			
\\	仕上げ	仕上[しあ]げ	しあげ	
\\	彼は論文の仕上げに入ったの。	彼[かれ]は 論文[ろんぶん]の 仕上[しあ]げに 入[はい]ったの。	かれ は ろんぶん の しあげ に はいった の	
\\	彼[かれ]は 論文[ろんぶん]の
\\	に 入[はい]ったの。			
\\	仕上げる	仕上[しあ]げる	しあげる	
\\	彼はやっと報告書を仕上げたよ。	彼[かれ]はやっと 報告書[ほうこくしょ]を 仕上[しあ]げたよ。	かれ は やっと ほうこくしょ を しあげた よ	
\\	彼[かれ]はやっと 報告書[ほうこくしょ]を
\\	よ。			
\\	人事	人事[じんじ]	じんじ	
\\	彼は人事を担当しているんだ。	彼[かれ]は 人事[じんじ]を 担当[たんとう]しているんだ。	かれ は じんじ を たんとう して いる ん だ	
\\	彼[かれ]は
\\	を 担当[たんとう]しているんだ。			
\\	事	事[こと]	こと	
\\	そんな事をしてはいけません。	そんな 事[こと]をしてはいけません。	そんな こと を して は いけません	
\\	そんな
\\	をしてはいけません。			
\\	コンクリート	コンクリート	コンクリート	
\\	コンクリートにひびが入っていますね。	コンクリートにひびが 入[はい]っていますね。	こんくりーと に ひび が はいって います ね	
\\	にひびが 入[はい]っていますね。			
\\	人工	人工[じんこう]	じんこう	
\\	これは人工の湖です。	これは 人工[じんこう]の 湖[みずうみ]です。	これ は じんこう の みずうみ です	
\\	これは
\\	の 湖[みずうみ]です。			
\\	会場	会場[かいじょう]	かいじょう	
\\	会場は人で一杯になりました。	会場[かいじょう]は 人[ひと]で 一杯[いっぱい]になりました。	かいじょう は ひと で いっぱい に なりました	
\\	は 人[ひと]で 一杯[いっぱい]になりました。			
\\	出場	出場[しゅつじょう]	しゅつじょう	
\\	今日は8チームが出場しました。	今日[きょう]は 8[はち]チームが 出場[しゅつじょう]しました。	きょう は はち ちーむ が しゅつじょう しました	
\\	今日[きょう]は 8[はち]チームが
\\	しました。			
\\	下車	下車[げしゃ]	げしゃ	
\\	景色が良かったので途中下車したよ。	景色[けしき]が 良[よ]かったので 途中[とちゅう] 下車[げしゃ]したよ。	けしき が よかった の で とちゅう げしゃ した よ	
\\	景色[けしき]が 良[よ]かったので 途中[とちゅう]
\\	したよ。			
\\	いよいよ	いよいよ	いよいよ	
\\	明日はいよいよ出発の日です。	明日[あす]はいよいよ 出発[しゅっぱつ]の 日[ひ]です。	あす は いよいよ しゅっぱつ の ひ です	
\\	明日[あす]は
\\	出発[しゅっぱつ]の 日[ひ]です。			
\\	交わす	交[か]わす	かわす	
\\	彼は私と再会の約束を交わした。	彼[かれ]は 私[わたし]と 再会[さいかい]の 約束[やくそく]を 交[か]わした。	かれ は わたし と さいかい の やくそく を かわした	
\\	彼[かれ]は 私[わたし]と 再会[さいかい]の 約束[やくそく]を
\\	交通	交通[こうつう]	こうつう	
\\	ここは交通の便がよいですね。	ここは 交通[こうつう]の 便[べん]がよいですね。	ここ は こうつう の べん が よい です ね	
\\	ここは
\\	の 便[べん]がよいですね。			
\\	十字路	十字路[じゅうじろ]	じゅうじろ	
\\	十字路で車とバイクが衝突したぞ。	十字路[じゅうじろ]で 車[くるま]とバイクが 衝突[しょうとつ]したぞ。	じゅうじろ で くるま と ばいく が しょうとつ した ぞ	
\\	で 車[くるま]とバイクが 衝突[しょうとつ]したぞ。			
\\	オリンピック	オリンピック	オリンピック	
\\	オリンピックは4年に1度開催されます。	オリンピックは 4年[よねん]に 1度開催[いちど かいさい]されます。	おりんぴっく は よねん に いちど かいさい されます	
\\	は 4年[よねん]に 1度開催[いちど かいさい]されます。			
\\	初歩	初歩[しょほ]	しょほ	
\\	英語を初歩から勉強しています。	英語[えいご]を 初歩[しょほ]から 勉強[べんきょう]しています。	えいご を しょほ から べんきょう して います	
\\	英語[えいご]を
\\	から 勉強[べんきょう]しています。			
\\	一度に	一度[いちど]に	いちどに	
\\	皆の顔と名前を一度には覚えられません。	皆[みんな]の 顔[かお]と 名前[なまえ]を 一度[いちど]には 覚[おぼ]えられません。	みんな の かお と なまえ を いちどに は おぼえられません	
\\	皆[みんな]の 顔[かお]と 名前[なまえ]を
\\	は 覚[おぼ]えられません。			
\\	ストップ	ストップ	ストップ	
\\	ここでいったんストップしてください。	ここでいったんストップしてください。	ここ で いったん すとっぷ して ください	
\\	ここでいったん
\\	してください。			
\\	全員	全員[ぜんいん]	ぜんいん	
\\	チーム全員に名札が配られました。	チーム 全員[ぜんいん]に 名札[なふだ]が 配[くば]られました。	ちーむ ぜんいん に なふだ が くばられました	
\\	チーム
\\	に 名札[なふだ]が 配[くば]られました。			
\\	全て	全[すべ]て	すべて	
\\	彼に全てのことを伝えた。	彼[かれ]に 全[すべ]てのことを 伝[つた]えた。	かれ に すべて の こと を つたえた	
\\	彼[かれ]に
\\	のことを 伝[つた]えた。			
\\	一部分	一部分[いちぶぶん]	いちぶぶん	
\\	僕が知っているのは一部分に過ぎない。	僕[ぼく]が 知[し]っているのは 一部分[いちぶぶん]に 過[す]ぎない。	ぼく が しって いる の は いちぶぶん に すぎない	
\\	僕[ぼく]が 知[し]っているのは
\\	に 過[す]ぎない。			
\\	きちんと	きちんと	きちんと	
\\	問題についてきちんと話し合ったよ。	問題[もんだい]についてきちんと 話[はな]し 合[あ]ったよ。	もんだい に ついて きちんと はなしあった よ	
\\	問題[もんだい]について
\\	話[はな]し 合[あ]ったよ。			
\\	出国	出国[しゅっこく]	しゅっこく	
\\	1週間後に出国します。	1週間後[いっしゅうかん ご]に 出国[しゅっこく]します。	いっしゅうかん ご に しゅっこく します	
\\	1週間後[いっしゅうかん ご]に
\\	します。			
\\	世間	世間[せけん]	せけん	
\\	世間の反応は冷たかったよ。	世間[せけん]の 反応[はんのう]は 冷[つめ]たかったよ。	せけん の はんのう は つめたかった よ	
\\	の 反応[はんのう]は 冷[つめ]たかったよ。			
\\	世話	世話[せわ]	せわ	
\\	旅行の間、犬の世話をしてください。	旅行[りょこう]の 間[あいだ]、 犬[いぬ]の 世話[せわ]をしてください。	りょこう の あいだ いぬ の せわ を して ください	
\\	旅行[りょこう]の 間[あいだ]、 犬[いぬ]の
\\	をしてください。			
\\	クラブ	クラブ	クラブ	
\\	夏休みにはクラブの合宿に参加します。	夏休[なつやす]みにはクラブの 合宿[がっしゅく]に 参加[さんか]します。	なつやすみ に は くらぶ の がっしゅく に さんか します	
\\	夏休[なつやす]みには
\\	の 合宿[がっしゅく]に 参加[さんか]します。			
\\	出世	出世[しゅっせ]	しゅっせ	
\\	彼は出世するタイプだな。	彼[かれ]は 出世[しゅっせ]するタイプだな。	かれ は しゅっせ する たいぷ だ な	
\\	彼[かれ]は
\\	するタイプだな。			
\\	じっと	じっと	じっと	
\\	あの生徒はじっと先生の話を聞いていたね。	あの 生徒[せいと]はじっと 先生[せんせい]の 話[はなし]を 聞[き]いていたね。	あの せいと は じっと せんせい の はなし を きいて いた ね	
\\	あの 生徒[せいと]は
\\	先生[せんせい]の 話[はなし]を 聞[き]いていたね。			
\\	コンサート	コンサート	コンサート	
\\	友達とジャズのコンサートに行きました。	友達[ともだち]とジャズのコンサートに 行[い]きました。	ともだち と じゃず の こんさーと に いきました 。	
\\	友達[ともだち]とジャズの
\\	に 行[い]きました。			
\\	作家	作家[さっか]	さっか	
\\	彼女は有名な作家です。	彼女[かのじょ]は 有名[ゆうめい]な 作家[さっか]です。	かのじょ は ゆうめい な さっか です	
\\	彼女[かのじょ]は 有名[ゆうめい]な
\\	です。			
\\	使用	使用[しよう]	しよう	
\\	この製品を使用する前に、説明書をお読みください。	この 製品[せいひん]を 使用[しよう]する 前[まえ]に、 説明書[せつめいしょ]をお 読[よ]みください。	この せいひん を しよう する まえ に せつめいしょ を およみ ください	
\\	この 製品[せいひん]を
\\	する 前[まえ]に、 説明書[せつめいしょ]をお 読[よ]みください。			
\\	作用	作用[さよう]	さよう	
\\	この薬は神経に作用します。	この 薬[くすり]は 神経[しんけい]に 作用[さよう]します。	この くすり は しんけい に さよう します	
\\	この 薬[くすり]は 神経[しんけい]に
\\	します。			
\\	使用人	使用人[しようにん]	しようにん	
\\	彼は使用人を首にしたよ。	彼[かれ]は 使用人[しようにん]を 首[くび]にしたよ。	かれ は しようにん を くび に した よ	
\\	彼[かれ]は
\\	を 首[くび]にしたよ。			
\\	いけない	いけない	いけない	
\\	勉強を怠けてはいけないよ。	勉強[べんきょう]を 怠[なま]けてはいけないよ。	べんきょう を なまけて は いけない よ	
\\	勉強[べんきょう]を 怠[なま]けては
\\	よ。			
\\	会費	会費[かいひ]	かいひ	
\\	会費が少し高いね。	会費[かいひ]が 少[すこ]し 高[たか]いね。	かいひ が すこし たかい ね	
\\	が 少[すこ]し 高[たか]いね。			
\\	あらかじめ	あらかじめ	あらかじめ	
\\	あらかじめ必要な物を書き出して下さい。	あらかじめ 必要[ひつよう]な 物[もの]を 書[か]き 出[だ]して 下[くだ]さい。	あらかじめ ひつよう な もの を かきだして ください	
\\	必要[ひつよう]な 物[もの]を 書[か]き 出[だ]して 下[くだ]さい。			
\\	しばしば	しばしば	しばしば	
\\	課長はしばしば出張します。	課長[かちょう]はしばしば 出張[しゅっちょう]します。	かちょう は しばしば しゅっちょう します	
\\	課長[かちょう]は
\\	出張[しゅっちょう]します。			
\\	上品	上品[じょうひん]	じょうひん	
\\	このドレスはとても上品なデザインですね。	このドレスはとても 上品[じょうひん]なデザインですね。	この どれす は とても じょうひん な でざいん です ね	
\\	このドレスはとても
\\	なデザインですね。			
\\	下品	下品[げひん]	げひん	
\\	彼女の下品なふるまいには我慢できない。	彼女[かのじょ]の 下品[げひん]なふるまいには 我慢[がまん]できない。	かのじょ の げひん な ふるまい に は がまん できない	
\\	彼女[かのじょ]の
\\	なふるまいには 我慢[がまん]できない。			
\\	一段と	一段[いちだん]と	いちだんと	
\\	物価は一段と上昇しそうだね。	物価[ぶっか]は 一段[いちだん]と 上昇[じょうしょう]しそうだね。	ぶっか は いちだんと じょうしょう しそう だ ね	
\\	物価[ぶっか]は
\\	上昇[じょうしょう]しそうだね。			
\\	ショック	ショック	ショック	
\\	彼女はショックで口もきけなかったわ。	彼女[かのじょ]はショックで 口[くち]もきけなかったわ。	かのじょ は しょっく で くち も きけなかった わ	
\\	彼女[かのじょ]は
\\	で 口[くち]もきけなかったわ。			
\\	値	値[あたい]	あたい	
\\	の値を求めなさい。	
\\	[えっくす]の 値[あたい]を 求[もと]めなさい。	えっくす の あたい を もとめなさい	
\\	[えっくす]の
\\	を 求[もと]めなさい。			
\\	値する	値[あたい]する	あたいする	
\\	その絵は一見に値しますよ。	その 絵[え]は 一見[いっけん]に 値[あたい]しますよ。	その え は いっけん に あたい します よ	
\\	その 絵[え]は 一見[いっけん]に
\\	よ。			
\\	価値	価値[かち]	かち	
\\	とても価値のある話を聞いたよ。	とても 価値[かち]のある 話[はなし]を 聞[き]いたよ。	とても かち の ある はなし を きいた よ	
\\	とても
\\	のある 話[はなし]を 聞[き]いたよ。			
\\	人格	人格[じんかく]	じんかく	
\\	私は彼の人格を尊敬しています。	私[わたし]は 彼[かれ]の 人格[じんかく]を 尊敬[そんけい]しています。	わたし は かれ の じんかく を そんけい して います	
\\	私[わたし]は 彼[かれ]の
\\	を 尊敬[そんけい]しています。			
\\	アンテナ	アンテナ	アンテナ	
\\	屋上にアンテナが立っていますね。	屋上[おくじょう]にアンテナが 立[た]っていますね。	おくじょう に あんてな が たって います ね	
\\	屋上[おくじょう]に
\\	が 立[た]っていますね。			
\\	初夏	初夏[しょか]	しょか	
\\	初夏の高原は気持ちがいいですね。	初夏[しょか]の 高原[こうげん]は 気持[きも]ちがいいですね。	しょか の こうげん は きもち が いい です ね	
\\	の 高原[こうげん]は 気持[きも]ちがいいですね。			
\\	あえて	あえて	あえて	
\\	彼はあえて危険を冒したの。	彼[かれ]はあえて 危険[きけん]を 冒[おか]したの。	かれ は あえて きけん を おかした	
\\	彼[かれ]は
\\	危険[きけん]を 冒[おか]したの。			
\\	北風	北風[きたかぜ]	きたかぜ	
\\	北風が冷たいです。	北風[きたかぜ]が 冷[つめ]たいです。	きたかぜ が つめたい です	
\\	が 冷[つめ]たいです。			
\\	カバー	カバー	カバー	
\\	本にカバーを掛けました。	本[ほん]にカバーを 掛[か]けました。	ほん に かばー を かけました	
\\	本[ほん]に
\\	を 掛[か]けました。			
\\	ジャーナリスト	ジャーナリスト	ジャーナリスト	
\\	彼女は有能なジャーナリストだ。	彼女[かのじょ]は 有能[ゆうのう]なジャーナリストだ。	かのじょ は ゆうのう な じゃーなりすと だ	
\\	彼女[かのじょ]は 有能[ゆうのう]な
\\	だ。			
\\	スタイル	スタイル	スタイル	
\\	彼女はモデルのようにスタイルがいいね。	彼女[かのじょ]はモデルのようにスタイルがいいね。	かのじょ は もでる の よう に すたいる が いい ね	
\\	彼女[かのじょ]はモデルのように
\\	がいいね。			
\\	インタビュー	インタビュー	インタビュー	
\\	彼はインタビューに、はきはきと答えてたよ。	彼[かれ]はインタビューに、はきはきと 答[こた]えてたよ。	かれ は いんたびゅー に はきはき と こたえて た よ	
\\	彼[かれ]は
\\	に、はきはきと 答[こた]えてたよ。			
\\	住まい	住[す]まい	すまい	
\\	私の住まいは東京にあります。	私[わたし]の 住[す]まいは 東京[とうきょう]にあります。	わたし の すまい は とうきょう に あります	
\\	私[わたし]の
\\	は 東京[とうきょう]にあります。			
\\	人名	人名[じんめい]	じんめい	
\\	これは日本の人名ですか。	これは 日本[にほん]の 人名[じんめい]ですか。	これ は にほん の じんめい です か	
\\	これは 日本[にほん]の
\\	ですか。			
\\	シーズン	シーズン	シーズン	
\\	その選手は今シーズンも好調だね。	その 選手[せんしゅ]は 今[こん]シーズンも 好調[こうちょう]だね。	その せんしゅ は こん しーずん も こうちょう だ ね 。	
\\	その 選手[せんしゅ]は 今[こん]
\\	も 好調[こうちょう]だね。			
\\	下町	下町[したまち]	したまち	
\\	あの子は下町育ちだ。	あの 子[こ]は 下町[したまち] 育[そだ]ちだ。	あの こ は したまち そだち だ	
\\	あの 子[こ]は
\\	育[そだ]ちだ。			
\\	区分	区分[くぶん]	くぶん	
\\	この表は年齢区分ごとの人口を表しています。	この 表[ひょう]は 年齢[ねんれい] 区分[くぶん]ごとの 人口[じんこう]を 表[あらわ]しています。	この ひょう は ねんれい くぶん ごと の じんこう を あらわして います	
\\	この 表[ひょう]は 年齢[ねんれい]
\\	ごとの 人口[じんこう]を 表[あらわ]しています。			
\\	区	区[く]	く	
\\	東京には23の区がある。	東京[とうきょう]には23の 区[く]がある。	とうきょう に は 
\\	の く が ある 。	
\\	東京[とうきょう]には23の
\\	がある。			
\\	アイデア	アイデア	アイデア	
\\	彼がいいアイデアを出したね。	彼[かれ]がいいアイデアを 出[だ]したね。	かれ が いい あいであ を だした ね	
\\	彼[かれ]がいい
\\	を 出[だ]したね。			
\\	出荷	出荷[しゅっか]	しゅっか	
\\	ご注文の品は明日出荷致します。	ご 注文[ちゅうもん]の 品[しな]は 明日[あす] 出荷[しゅっか] 致[いた]します。	ごちゅうもん の しな は あす しゅっか いたします	
\\	ご 注文[ちゅうもん]の 品[しな]は 明日[あす]
\\	致[いた]します。			
\\	人物	人物[じんぶつ]	じんぶつ	
\\	彼は会社の重要な人物です。	彼[かれ]は 会社[かいしゃ]の 重要[じゅうよう]な 人物[じんぶつ]です。	かれ は かいしゃ の じゅうよう な じんぶつ です	
\\	彼[かれ]は 会社[かいしゃ]の 重要[じゅうよう]な
\\	です。			
\\	入れ物	入[い]れ 物[もの]	いれもの	
\\	荷物が多いので大きな入れ物が必要です。	荷物[にもつ]が 多[おお]いので 大[おお]きな 入[い]れ 物[もの]が 必要[ひつよう]です。	にもつ が おおい の で おおき な いれもの が ひつよう です	
\\	荷物[にもつ]が 多[おお]いので 大[おお]きな
\\	が 必要[ひつよう]です。			
\\	作物	作物[さくもつ]	さくもつ	
\\	米はアジアでは大切な作物だ。	米[こめ]はアジアでは 大切[たいせつ]な 作物[さくもつ]だ。	こめ は あじあ で は たいせつ な さくもつ だ	
\\	米[こめ]はアジアでは 大切[たいせつ]な
\\	だ。			
\\	いかにも	いかにも	いかにも	
\\	彼はいかにもスポーツマンらしいですね。	彼[かれ]はいかにもスポーツマンらしいですね。	かれ は いかにも すぽーつまん らしい です ね	
\\	彼[かれ]は
\\	スポーツマンらしいですね。			
\\	いきなり	いきなり	いきなり	
\\	後ろからいきなり肩をたたかれた。	後[うし]ろからいきなり 肩[かた]をたたかれた。	うしろ から いきなり かた を たたかれた	
\\	後[うし]ろから
\\	肩[かた]をたたかれた。			
\\	いつまでも	いつまでも	いつまでも	
\\	いつまでもあなたを忘れません。	いつまでもあなたを 忘[わす]れません。	いつまでも あなた を わすれません	
\\	あなたを 忘[わす]れません。			
\\	しっかり	しっかり	しっかり	
\\	彼は若いのにしっかりしてるね。	彼[かれ]は 若[わか]いのにしっかりしてるね。	かれ は わかい の に しっかり してる ね	
\\	彼[かれ]は 若[わか]いのに
\\	してるね。			
\\	区役所	区役所[くやくしょ]	くやくしょ	
\\	区役所に書類を届けてください。	区役所[くやくしょ]に 書類[しょるい]を 届[とど]けてください。	くやくしょ に しょるい を とどけて ください	
\\	に 書類[しょるい]を 届[とど]けてください。			
\\	人生	人生[じんせい]	じんせい	
\\	彼は自分の人生を振り返ったわ。	彼[かれ]は 自分[じぶん]の 人生[じんせい]を 振[ふ]り 返[かえ]ったわ。	かれ は じぶん の じんせい を ふりかえった わ	
\\	彼[かれ]は 自分[じぶん]の
\\	を 振[ふ]り 返[かえ]ったわ。			
\\	一生	一生[いっしょう]	いっしょう	
\\	一生のお願いがあります。	一生[いっしょう]のお 願[ねが]いがあります。	いっしょう の おねがい が あります	
\\	のお 願[ねが]いがあります。			
\\	アンケート	アンケート	アンケート	
\\	彼女はアンケートに答えたよ。	彼女[かのじょ]はアンケートに 答[こた]えたよ。	かのじょ は あんけーと に こたえた よ	
\\	彼女[かのじょ]は
\\	に 答[こた]えたよ。			
\\	スケジュール	スケジュール	スケジュール	
\\	スケジュールの調整は君に任せるわ。	スケジュールの 調整[ちょうせい]は 君[きみ]に 任[まか]せるわ。	すけじゅーる の ちょうせい は きみ に まかせる わ	
\\	の 調整[ちょうせい]は 君[きみ]に 任[まか]せるわ。			
\\	ガソリン	ガソリン	ガソリン	
\\	車にガソリンを入れました。	車[くるま]にガソリンを 入[い]れました。	くるま に がそりん を いれました	
\\	車[くるま]に
\\	を 入[い]れました。			
\\	出産	出産[しゅっさん]	しゅっさん	
\\	彼女は女の子を出産したんだ。	彼女[かのじょ]は 女[おんな]の 子[こ]を 出産[しゅっさん]したんだ。	かのじょ は おんな の こ を しゅっさん した ん だ	
\\	彼女[かのじょ]は 女[おんな]の 子[こ]を
\\	したんだ。			
\\	カット	カット	かっと	
\\	市長がテープをカットした。	市長[しちょう]がテープをカットした。	しちょう が てーぷ を かっと した	
\\	市長[しちょう]がテープを
\\	した。			
\\	休学	休学[きゅうがく]	きゅうがく	
\\	1年休学することにしました。	1年[いちねん] 休学[きゅうがく]することにしました。	いちねん きゅうがく する こと に しました	
\\	1年[いちねん]
\\	することにしました。			
\\	スター	スター	スター	
\\	彼はその時スターだったよ。	彼[かれ]はその 時[とき]スターだったよ。	かれ は その とき すたー だった よ 。	
\\	彼[かれ]はその 時[とき]
\\	だったよ。			
\\	キリスト教	キリスト 教[きょう]	きりすときょう	
\\	この先にキリスト教の教会があります。	この 先[さき]にキリスト 教[きょう]の 教会[きょうかい]があります。	この さき に きりすときょう の きょうかい が あります	
\\	この 先[さき]に
\\	の 教会[きょうかい]があります。			
\\	イスラム教	イスラム 教[きょう]	イスラムきょう	
\\	これはイスラム教の寺院です。	これはイスラム 教[きょう]の 寺院[じいん]です。	これ は いすらむきょう の じいん です	
\\	これは
\\	の 寺院[じいん]です。			
\\	制御	制御[せいぎょ]	せいぎょ	
\\	ここのパネルで機械全体を制御できます。	ここのパネルで 機械全体[きかい ぜんたい]を 制御[せいぎょ]できます。	ここ の ぱねる で きかい ぜんたい を せいぎょ できます	
\\	ここのパネルで 機械全体[きかい ぜんたい]を
\\	できます。			
\\	カメラマン	カメラマン	カメラマン	
\\	将来はプロのカメラマンになりたいです。	将来[しょうらい]はプロのカメラマンになりたいです。	しょうらい は ぷろ の かめらまん に なりたい です	
\\	将来[しょうらい]はプロの
\\	になりたいです。			
\\	アルコール	アルコール	アルコール	
\\	彼はアルコールに強い体質です。	彼[かれ]はアルコールに 強[つよ]い 体質[たいしつ]です。	かれ は あるこーる に つよい たいしつ です	
\\	彼[かれ]は
\\	に 強[つよ]い 体質[たいしつ]です。			
\\	せめて	せめて	せめて	
\\	せめてこれだけは約束してください。	せめてこれだけは 約束[やくそく]してください。	せめて これ だけ は やくそく して ください 。	
\\	これだけは 約束[やくそく]してください。			
\\	前回	前回[ぜんかい]	ぜんかい	
\\	前回の続きから始めます。	前回[ぜんかい]の 続[つづ]きから 始[はじ]めます。	ぜんかい の つづき から はじめます	
\\	の 続[つづ]きから 始[はじ]めます。			
\\	ストレス	ストレス	ストレス	
\\	ストレスがいろいろな病気の元になっているの。	ストレスがいろいろな 病気[びょうき]の 元[もと]になっているの。	すとれす が いろいろ な びょうき の もと に なっている の 。	
\\	がいろいろな 病気[びょうき]の 元[もと]になっているの。			
\\	個性	個性[こせい]	こせい	
\\	彼女たちはそれぞれ個性が強いですね。	彼女[かのじょ]たちはそれぞれ 個性[こせい]が 強[つよ]いですね。	かのじょたち は それぞれ こせい が つよい です ね	
\\	彼女[かのじょ]たちはそれぞれ
\\	が 強[つよ]いですね。			
\\	個々	個々[ここ]	ここ	
\\	個々の問題を解決しましょう。	個々[ここ]の 問題[もんだい]を 解決[かいけつ]しましょう。	ここ の もんだい を かいけつ しましょう	
\\	の 問題[もんだい]を 解決[かいけつ]しましょう。			
\\	勝手	勝手[かって]	かって	
\\	勝手なことばかり言わないでくれ。	勝手[かって]なことばかり 言[い]わないでくれ。	かって な こと ばかり いわないで くれ	
\\	なことばかり 言[い]わないでくれ。			
\\	勝ち	勝[か]ち	かち	
\\	歌合戦は赤組の勝ちでしたね。	歌合戦[うたがっせん]は 赤組[あか ぐみ]の 勝[か]ちでしたね。	うたがっせん は あか ぐみ の かち でした ね	
\\	歌合戦[うたがっせん]は 赤組[あか ぐみ]の
\\	でしたね。			
\\	勝負	勝負[しょうぶ]	しょうぶ	
\\	勝負はまだ始まったばかりよ。	勝負[しょうぶ]はまだ 始[はじ]まったばかりよ。	しょうぶ は まだ はじまった ばかり よ	
\\	はまだ 始[はじ]まったばかりよ。			
\\	コード	コード	コード	
\\	コードが短くてコンセントに届きません。	コードが 短[みじか]くてコンセントに 届[とど]きません。	こーど が みじかく て こんせんと に とどきません	
\\	が 短[みじか]くてコンセントに 届[とど]きません。			
\\	コーチ	コーチ	コーチ	
\\	新しいコーチの指導は厳しかった。	新[あたら]しいコーチの 指導[しどう]は 厳[きび]しかった。	あたらしい こーち の しどう は きびしかった	
\\	新[あたら]しい
\\	の 指導[しどう]は 厳[きび]しかった。			
\\	休業	休業[きゅうぎょう]	きゅうぎょう	
\\	明日は臨時に休業します。	明日[あした]は 臨時[りんじ]に 休業[きゅうぎょう]します。	あした は りんじ に きゅうぎょう します	
\\	明日[あした]は 臨時[りんじ]に
\\	します。			
\\	ステージ	ステージ	ステージ	
\\	彼はステージに立ったよ。	彼[かれ]はステージに 立[た]ったよ。	かれ は すてーじ に たった よ	
\\	彼[かれ]は
\\	に 立[た]ったよ。			
\\	事業	事業[じぎょう]	じぎょう	
\\	彼は事業に失敗したの。	彼[かれ]は 事業[じぎょう]に 失敗[しっぱい]したの。	かれ は じぎょう に しっぱい した の	
\\	彼[かれ]は
\\	に 失敗[しっぱい]したの。			
\\	利く	利[き]く	きく	
\\	彼女は気が利いている。	彼女[かのじょ]は 気[き]が 利[き]いている。	かのじょ は き が きいて いる	
\\	彼女[かのじょ]は 気[き]が
\\	収入	収入[しゅうにゅう]	しゅうにゅう	
\\	日本人の平均収入はどの位ですか。	日本人[にほんじん]の 平均[へいきん] 収入[しゅうにゅう]はどの 位[くらい]ですか。	にほんじん の へいきん しゅうにゅう は どの くらい です か	
\\	日本人[にほんじん]の 平均[へいきん]
\\	はどの 位[くらい]ですか。			
\\	作製	作製[さくせい]	さくせい	
\\	合鍵の作製には2日ほどかかります。	合鍵[あいかぎ]の 作製[さくせい]には 2日[ふつか]ほどかかります。	あいかぎ の さくせい に は ふつか ほど かかります	
\\	合鍵[あいかぎ]の
\\	には 2日[ふつか]ほどかかります。			
\\	アドバイス	アドバイス	アドバイス	
\\	彼のアドバイスはいつも有り難いわね。	彼[かれ]のアドバイスはいつも 有[あ]り 難[がた]いわね。	かれ の あどばいす は いつも ありがたい わ ね	
\\	彼[かれ]の
\\	はいつも 有[あ]り 難[がた]いわね。			
\\	アルミニウム	アルミニウム	アルミニウム	
\\	このお鍋はアルミニウム製です。	このお 鍋[なべ]はアルミニウム 製[せい]です。	この おなべ は あるみにうむ せい です	
\\	このお 鍋[なべ]は
\\	製[せい]です。			
\\	会計	会計[かいけい]	かいけい	
\\	会計を済ませて店を出たんだ。	会計[かいけい]を 済[す]ませて 店[みせ]を 出[で]たんだ。	かいけい を すませて みせ を でた ん だ	
\\	を 済[す]ませて 店[みせ]を 出[で]たんだ。			
\\	オーケストラ	オーケストラ	オーケストラ	
\\	彼はオーケストラを指揮しているの。	彼[かれ]はオーケストラを 指揮[しき]しているの。	かれ は おーけすとら を しき して いる の	
\\	彼[かれ]は
\\	を 指揮[しき]しているの。			
\\	おじさん	おじさん	おじさん	
\\	今そこで隣のおじさんに会ったよ。	今[いま]そこで 隣[となり]のおじさんに 会[あ]ったよ。	いま そこで となり の おじさん に あった よ	
\\	今[いま]そこで 隣[となり]の
\\	に 会[あ]ったよ。			
\\	収支	収支[しゅうし]	しゅうし	
\\	家計の収支が合わないの。	家計[かけい]の 収支[しゅうし]が 合[あ]わないの。	かけい の しゅうし が あわない の	
\\	家計[かけい]の
\\	が 合[あ]わないの。			
\\	コンクール	コンクール	コンクール	
\\	来年のコンクールに出場するつもりです。	来年[らいねん]のコンクールに 出場[しゅつじょう]するつもりです。	らいねん の こんくーる に しゅつじょう する つもり です	
\\	来年[らいねん]の
\\	に 出場[しゅつじょう]するつもりです。			
\\	借り	借[か]り	かり	
\\	この借りは必ず返します。	この 借[か]りは 必[かなら]ず 返[かえ]します。	この かり は かならず かえします	
\\	この
\\	は 必[かなら]ず 返[かえ]します。			
\\	借金	借金[しゃっきん]	しゃっきん	
\\	彼は友達に借金をしたんだ。	彼[かれ]は 友達[ともだち]に 借金[しゃっきん]をしたんだ。	かれ は ともだち に しゃっきん を した ん だ	
\\	彼[かれ]は 友達[ともだち]に
\\	をしたんだ。			
\\	アマチュア	アマチュア	アマチュア	
\\	彼はアマチュア音楽家です。	彼[かれ]はアマチュア 音楽家[おんがくか]です。	かれ は あまちゅあ おんがくか です	
\\	彼[かれ]は
\\	音楽家[おんがくか]です。			
\\	初期	初期[しょき]	しょき	
\\	鼻水は風邪の初期症状のひとつです。	鼻水[はなみず]は 風邪[かぜ]の 初期[しょき] 症状[しょうじょう]のひとつです。	はなみず は かぜ の しょき しょうじょう の ひとつ です	
\\	鼻水[はなみず]は 風邪[かぜ]の
\\	症状[しょうじょう]のひとつです。			
\\	前期	前期[ぜんき]	ぜんき	
\\	前期の売上はとても良かったわ。	前期[ぜんき]の 売上[うりあげ]はとても 良[よ]かったわ。	ぜんき の うりあげ は とても よかった わ	
\\	の 売上[うりあげ]はとても 良[よ]かったわ。			
\\	きつい	きつい	きつい	
\\	彼女は性格がきついよね。	彼女[かのじょ]は 性格[せいかく]がきついよね。	かのじょ は せいかく が きつい よ ね	
\\	彼女[かのじょ]は 性格[せいかく]が
\\	よね。			
\\	制限	制限[せいげん]	せいげん	
\\	彼女は食事を制限しています。	彼女[かのじょ]は 食事[しょくじ]を 制限[せいげん]しています。	かのじょ は しょくじ を せいげん して います	
\\	彼女[かのじょ]は 食事[しょくじ]を
\\	しています。			
\\	コーナー	コーナー	コーナー	
\\	バーゲンコーナーで
\\	を買いました。	バーゲンコーナーで 
\\	[でぃーぶいでぃー]を 買[か]いました。	ばーげんこーなー で でぃーぶいでぃー を かいました	
\\	バーゲン
\\	で 
\\	[でぃーぶいでぃー]を 買[か]いました。			
\\	一切	一切[いっさい]	いっさい	
\\	私は一切その問題とは関係がありません。	私[わたし]は 一切[いっさい]その 問題[もんだい]とは 関係[かんけい]がありません。	わたし は いっさい その もんだい と は かんけい が ありません	
\\	私[わたし]は
\\	その 問題[もんだい]とは 関係[かんけい]がありません。			
\\	区切る	区切[くぎ]る	くぎる	
\\	フロアはパーティションで区切られています。	フロアはパーティションで 区切[くぎ]られています。	ふろあ は ぱーてぃしょん で くぎられて います	
\\	フロアはパーティションで
\\	切れ	切[き]れ	きれ	
\\	この包丁は切れが良いな。	この 包丁[ほうちょう]は 切[き]れが 良[い]いな。	この ほうちょう は きれ が いい な	
\\	この 包丁[ほうちょう]は
\\	が 良[い]いな。			
\\	ぎりぎり	ぎりぎり	ぎりぎり	
\\	会社の始業時間にぎりぎりで間に合ったよ。	会社[かいしゃ]の 始業時間[しぎょうじかん]にぎりぎりで 間[ま]に 合[あ]ったよ。	かいしゃ の しぎょうじかん に ぎりぎり で まにあった よ	
\\	会社[かいしゃ]の 始業時間[しぎょうじかん]に
\\	で 間[ま]に 合[あ]ったよ。			
\\	券	券[けん]	けん	
\\	入場券をお持ちですか。	入場[にゅうじょう] 券[けん]をお 持[も]ちですか。	にゅうじょうけん を お もち です か	
\\	入場[にゅうじょう]
\\	をお 持[も]ちですか。			
\\	世代	世代[せだい]	せだい	
\\	私は彼と同じ世代です。	私[わたし]は 彼[かれ]と 同[おな]じ 世代[せだい]です。	わたし は かれ と おなじ せだい です	
\\	私[わたし]は 彼[かれ]と 同[おな]じ
\\	です。			
\\	代わり	代[か]わり	かわり	
\\	ごま油の代わりにオリーブ油を使いましょう。	ごま 油[あぶら]の 代[か]わりにオリーブ 油[ゆ]を 使[つか]いましょう。	ごまあぶら の かわり に おりーぶゆ を つかいましょう	
\\	ごま 油[あぶら]の
\\	にオリーブ 油[ゆ]を 使[つか]いましょう。			
\\	代わる代わる	代[か]わる 代[が]わる	かわるがわる	
\\	皆が代わる代わる彼らを祝福したよ。	皆[みな]が 代[か]わる 代[が]わる 彼[かれ]らを 祝福[しゅくふく]したよ。	みな が かわるがわる かれら を しゅくふく した よ	
\\	皆[みな]が
\\	彼[かれ]らを 祝福[しゅくふく]したよ。			
\\	ウサギ	ウサギ	ウサギ	
\\	ウサギとカメの話を知っていますか。	ウサギとカメの 話[はなし]を 知[し]っていますか。	うさぎ と かめ の はなし を しって います か	
\\	とカメの 話[はなし]を 知[し]っていますか。			
\\	化学	化学[かがく]	かがく	
\\	彼は化学の教授です。	彼[かれ]は 化学[かがく]の 教授[きょうじゅ]です。	かれ は かがく の きょうじゅ です	
\\	彼[かれ]は
\\	の 教授[きょうじゅ]です。			
\\	グラウンド	グラウンド	グラウンド	
\\	野球部はグラウンドで練習しています。	野球部[やきゅうぶ]はグラウンドで 練習[れんしゅう]しています。	やきゅうぶ は ぐらうんど で れんしゅう して います	
\\	野球部[やきゅうぶ]は
\\	で 練習[れんしゅう]しています。			
\\	今更	今更[いまさら]	いまさら	
\\	今更後悔しても、もう遅いよ。	今更[いまさら] 後悔[こうかい]しても、もう 遅[おそ]いよ。	いまさら こうかい して も もう おそい よ	
\\	後悔[こうかい]しても、もう 遅[おそ]いよ。			
\\	乗車券	乗車券[じょうしゃけん]	じょうしゃけん	
\\	乗車券は無くさないように。	乗車券[じょうしゃけん]は 無[な]くさないように。	じょうしゃけん は なくさない よう に	
\\	は 無[な]くさないように。			
\\	乗車	乗車[じょうしゃ]	じょうしゃ	
\\	このバスの運賃は乗車するときに払います。	このバスの 運賃[うんちん]は 乗車[じょうしゃ]するときに 払[はら]います。	この ばす の うんちん は じょうしゃ する とき に はらいます	
\\	このバスの 運賃[うんちん]は
\\	するときに 払[はら]います。			
\\	さっぱり	さっぱり	さっぱり	
\\	彼女の言っていることがさっぱり分からないの。	彼女[かのじょ]の 言[い]っていることがさっぱり 分[わ]からないの。	かのじょ の いって いる こと が さっぱり わからない の	
\\	彼女[かのじょ]の 言[い]っていることが
\\	分[わ]からないの。			
\\	乗客	乗客[じょうきゃく]	じょうきゃく	
\\	乗客の一人の具合が悪くなったの。	乗客[じょうきゃく]の 一人[いち にん]の 具合[ぐあい]が 悪[わる]くなったの。	じょうきゃく の いち にん の ぐあい が わるく なった の 。	
\\	の 一人[いち にん]の 具合[ぐあい]が 悪[わる]くなったの。			
\\	シャッター	シャッター	シャッター	
\\	丸いボタンを押すとシャッターが閉まります。	丸[まる]いボタンを 押[お]すとシャッターが 閉[し]まります。	まるい ぼたん を おすと しゃったー が しまります	
\\	丸[まる]いボタンを 押[お]すと
\\	が 閉[し]まります。			
\\	スピーカー	スピーカー	スピーカー	
\\	このスピーカーは音がいいですね。	このスピーカーは 音[おと]がいいですね。	この すぴーかー は おと が いい です ね	
\\	この
\\	は 音[おと]がいいですね。			
\\	セールスマン	セールスマン	セールスマン	
\\	以前、セールスマンをしていたことがあります。	以前[いぜん]、セールスマンをしていたことがあります。	いぜん せーるすまん を して いた こと が あります	
\\	以前[いぜん]、
\\	をしていたことがあります。			
\\	おしゃべり	おしゃべり	おしゃべり	
\\	彼女たちはおしゃべりに夢中です。	彼女[かのじょ]たちはおしゃべりに 夢中[むちゅう]です。	かのじょたち は おしゃべり に むちゅう です	
\\	彼女[かのじょ]たちは
\\	に 夢中[むちゅう]です。			
\\	全速力	全速力[ぜんそくりょく]	ぜんそくりょく	
\\	久しぶりに全速力で走りました。	久[ひさ]しぶりに 全速力[ぜんそくりょく]で 走[はし]りました。	ひさしぶり に ぜんそくりょく で はしりました	
\\	久[ひさ]しぶりに
\\	で 走[はし]りました。			
\\	カラオケ	カラオケ	カラオケ	
\\	カラオケで思い切り歌ったの。	カラオケで 思[おも]い 切[き]り 歌[うた]ったの。	からおけ で おもいきり うたった の	
\\	で 思[おも]い 切[き]り 歌[うた]ったの。			
\\	こっそり	こっそり	こっそり	
\\	彼女はこっそりダイエットを始めたの。	彼女[かのじょ]はこっそりダイエットを 始[はじ]めたの。	かのじょ は こっそり だいえっと を はじめた の	
\\	彼女[かのじょ]は
\\	ダイエットを 始[はじ]めたの。			
\\	ジョギング	ジョギング	ジョギング	
\\	私は毎日ジョギングをしています。	私[わたし]は 毎日[まいにち]ジョギングをしています。	わたし は まいにち じょぎんぐ を して います	
\\	私[わたし]は 毎日[まいにち]
\\	をしています。			
\\	あれこれ	あれこれ	あれこれ	
\\	服を買うのにあれこれ迷いました。	服[ふく]を 買[か]うのにあれこれ 迷[まよ]いました。	ふく を かう の に あれこれ まよいました	
\\	服[ふく]を 買[か]うのに
\\	迷[まよ]いました。			
\\	去る	去[さ]る	さる	
\\	去る者は追わず。	去[さ]る 者[もの]は 追[お]わず。	さる もの は おわず	
\\	者[もの]は 追[お]わず。			
\\	およそ	およそ	およそ	
\\	家から駅までおよそ1キロあります。	家[いえ]から 駅[えき]までおよそ 1[いち]キロあります。	いえ から えき まで およそ いちきろ あります	
\\	家[いえ]から 駅[えき]まで
\\	1[いち]キロあります。			
\\	五十音	五十音[ごじゅうおん]	ごじゅうおん	
\\	名前は五十音順に並んでいます。	名前[なまえ]は 五十音[ごじゅうおん] 順[じゅん]に 並[なら]んでいます。	なまえ は ごじゅうおんじゅん に ならんで います	
\\	名前[なまえ]は
\\	順[じゅん]に 並[なら]んでいます。			
\\	サークル	サークル	サークル	
\\	どのサークルに入るか迷っちゃった。	どのサークルに 入[はい]るか 迷[まよ]っちゃった。	どの さーくる に はいる か まよっちゃった	
\\	どの
\\	に 入[はい]るか 迷[まよ]っちゃった。			
\\	企画	企画[きかく]	きかく	
\\	独身者パーティーを企画した。	独身者[どくしんしゃ]パーティーを 企画[きかく]した。	どくしんしゃ ぱーてぃー を きかく した	
\\	独身者[どくしんしゃ]パーティーを
\\	した。			
\\	区画	区画[くかく]	くかく	
\\	あそこの3区画は売り出し中です。	あそこの 3[さん] 区画[くかく]は 売[う]り 出[だ]し 中[ちゅう]です。	あそこ の さんくかく は うりだしちゅう です	
\\	あそこの 3[さん]
\\	は 売[う]り 出[だ]し 中[ちゅう]です。			
\\	キリスト	キリスト	キリスト	
\\	キリストの母親の名前はマリアだ。	キリストの 母親[ははおや]の 名前[なまえ]はマリアだ。	きりすと の ははおや の なまえ は まりあ だ	
\\	の 母親[ははおや]の 名前[なまえ]はマリアだ。			
\\	一面	一面[いちめん]	いちめん	
\\	外は一面の雪景色だったね。	外[そと]は 一面[いちめん]の 雪景色[ゆきげしき]だったね。	そと は いちめん の ゆきげしき だった ね	
\\	外[そと]は
\\	の 雪景色[ゆきげしき]だったね。			
\\	あっさり	あっさり	あっさり	
\\	そうあっさり言わないで。	そうあっさり 言[い]わないで。	そう あっさり いわない で	
\\	そう
\\	言[い]わないで。			
\\	一種	一種[いっしゅ]	いっしゅ	
\\	それは果物の一種です。	それは 果物[くだもの]の 一種[いっしゅ]です。	それ は くだもの の いっしゅ です	
\\	それは 果物[くだもの]の
\\	です。			
\\	一種	一種[いっしゅ]	いっしゅ	
\\	これは一種独特の味がするね。	これは 一種[いっしゅ] 独特[どくとく]の 味[あじ]がするね。	これ は いっしゅ どくとく の あじ が する ね	
\\	これは
\\	独特[どくとく]の 味[あじ]がするね。			
\\	人種	人種[じんしゅ]	じんしゅ	
\\	この国には様々な人種がいます。	この 国[くに]には 様々[さまざま]な 人種[じんしゅ]がいます。	この くに に は さまざま な じんしゅ が います	
\\	この 国[くに]には 様々[さまざま]な
\\	がいます。			
\\	人類	人類[じんるい]	じんるい	
\\	人類は2本の足で歩きます。	人類[じんるい]は 2本[にほん]の 足[あし]で 歩[ある]きます。	じんるい は にほん の あし で あるきます	
\\	は 2本[にほん]の 足[あし]で 歩[ある]きます。			
\\	せっせと	せっせと	せっせと	
\\	彼は朝から晩までせっせと働いたの。	彼[かれ]は 朝[あさ]から 晩[ばん]までせっせと 働[はたら]いたの。	かれ は あさ から ばん まで せっせと はたらいた の	
\\	彼[かれ]は 朝[あさ]から 晩[ばん]まで
\\	働[はたら]いたの。			
\\	三角形	三角形[さんかくけい]	さんかくけい	
\\	この三角形の面積を出しなさい。	この 三角形[さんかくけい]の 面積[めんせき]を 出[だ]しなさい。	この さんかくけい の めんせき を だしなさい	
\\	この
\\	の 面積[めんせき]を 出[だ]しなさい。			
\\	カタログ	カタログ	カタログ	
\\	カタログを見て注文したんだ。	カタログを 見[み]て 注文[ちゅうもん]したんだ。	かたろぐ を みて ちゅうもん した ん だ	
\\	を 見[み]て 注文[ちゅうもん]したんだ。			
\\	作曲	作曲[さっきょく]	さっきょく	
\\	この曲は誰が作曲したのですか。	この 曲[きょく]は 誰[だれ]が 作曲[さっきょく]したのですか。	この きょく は だれ が さっきょく した の です か	
\\	この 曲[きょく]は 誰[だれ]が
\\	したのですか。			
\\	共通	共通[きょうつう]	きょうつう	
\\	私たちは共通の趣味を持っています。	私[わたし]たちは 共通[きょうつう]の 趣味[しゅみ]を 持[も]っています。	わたしたち は きょうつう の しゅみ を もって います	
\\	私[わたし]たちは
\\	の 趣味[しゅみ]を 持[も]っています。			
\\	共通語	共通語[きょうつうご]	きょうつうご	
\\	インドでは英語は共通語だよ。	インドでは 英語[えいご]は 共通語[きょうつうご]だよ。	いんど で は えいご は きょうつうご だ よ	
\\	インドでは 英語[えいご]は
\\	だよ。			
\\	共同	共同[きょうどう]	きょうどう	
\\	2社が共同で新製品を開発したんだ。	2社[にしゃ]が 共同[きょうどう]で 新製品[しんせいひん]を 開発[かいはつ]したんだ。	にしゃ が きょうどう で しんせいひん を かいはつ した ん だ	
\\	2社[にしゃ]が
\\	で 新製品[しんせいひん]を 開発[かいはつ]したんだ。			
\\	一同	一同[いちどう]	いちどう	
\\	一同顔を見合わせました。	一同[いちどう] 顔[かお]を 見合[みあ]わせました。	いちどう かお を みあわせました	
\\	顔[かお]を 見合[みあ]わせました。			
\\	以前	以前[いぜん]	いぜん	
\\	以前彼はこの町に住んでいたんだ。	以前[いぜん] 彼[かれ]はこの 町[まち]に 住[す]んでいたんだ。	いぜん かれ は この まち に すんで いた ん だ	
\\	彼[かれ]はこの 町[まち]に 住[す]んでいたんだ。			
\\	ジャズ	ジャズ	ジャズ	
\\	昨日の夜、ジャズを聴きに行きました。	昨日[きのう]の 夜[よる]、ジャズを 聴[き]きに 行[い]きました。	きのう の よる じゃず を きき に いきました	
\\	昨日[きのう]の 夜[よる]、
\\	を 聴[き]きに 行[い]きました。			
\\	以下	以下[いか]	いか	
\\	数学が平均点以下だった。	数学[すうがく]が 平均点[へいきんてん] 以下[いか]だった。	すうがく が へいきんてん いか だった	
\\	数学[すうがく]が 平均点[へいきんてん]
\\	だった。			
\\	以降	以降[いこう]	いこう	
\\	夕方以降にお電話を下さい。	夕方[ゆうがた] 以降[いこう]にお 電話[でんわ]を 下[くだ]さい。	ゆうがた いこう に お でんわ を ください	
\\	夕方[ゆうがた]
\\	にお 電話[でんわ]を 下[くだ]さい。			
\\	以来	以来[いらい]	いらい	
\\	それ以来彼女に会ってない。	それ 以来[いらい] 彼女[かのじょ]に 会[あ]ってない。	それ いらい かのじょ に あってない	
\\	それ
\\	彼女[かのじょ]に 会[あ]ってない。			
\\	以後	以後[いご]	いご	
\\	すみません、以後気を付けます。	すみません、 以後[いご] 気[き]を 付[つ]けます。	すみません いご き を つけます	
\\	すみません、
\\	気[き]を 付[つ]けます。			
\\	下宿	下宿[げしゅく]	げしゅく	
\\	親類の家に4年間下宿しました。	親類[しんるい]の 家[いえ]に 4年間[よねんかん] 下宿[げしゅく]しました。	しんるい の いえ に よねんかん げしゅく しました	
\\	親類[しんるい]の 家[いえ]に 4年間[よねんかん]
\\	しました。			
\\	サイン	サイン	サイン	
\\	ここにサインしてください。	ここにサインしてください。	ここ に さいん して ください	
\\	ここに
\\	してください。			
\\	制服	制服[せいふく]	せいふく	
\\	私の学校には制服がありません。	私[わたし]の 学校[がっこう]には 制服[せいふく]がありません。	わたし の がっこう に は せいふく が ありません	
\\	私[わたし]の 学校[がっこう]には
\\	がありません。			
\\	ごまかす	ごまかす	ごまかす	
\\	彼は質問の答えをごまかしたね。	彼[かれ]は 質問[しつもん]の 答[こた]えをごまかしたね。	かれ は しつもん の こたえ を ごまかした ね	
\\	彼[かれ]は 質問[しつもん]の 答[こた]えを
\\	ね。			
\\	インテリ	インテリ	インテリ	
\\	彼女はインテリで、しかも美人ですね。	彼女[かのじょ]はインテリで、しかも 美人[びじん]ですね。	かのじょ は いんてり で しかも びじん です ね	
\\	彼女[かのじょ]は
\\	で、しかも 美人[びじん]ですね。			
\\	休息	休息[きゅうそく]	きゅうそく	
\\	休息をとることは大切です。	休息[きゅうそく]をとることは 大切[たいせつ]です。	きゅうそく を とる こと は たいせつ です	
\\	をとることは 大切[たいせつ]です。			
\\	作者	作者[さくしゃ]	さくしゃ	
\\	この本の作者は誰ですか。	この 本[ほん]の 作者[さくしゃ]は 誰[だれ]ですか。	この ほん の さくしゃ は だれ です か	
\\	この 本[ほん]の
\\	は 誰[だれ]ですか。			
\\	いちいち	いちいち	いちいち	
\\	彼はいちいち私に指図する。	彼[かれ]はいちいち 私[わたし]に 指図[さしず]する。	かれ は いちいち わたし に さしず する	
\\	彼[かれ]は
\\	私[わたし]に 指図[さしず]する。			
\\	前者	前者[ぜんしゃ]	ぜんしゃ	
\\	私は前者の方が優れていると思います。	私[わたし]は 前者[ぜんしゃ]の 方[ほう]が 優[すぐ]れていると 思[おも]います。	わたし は ぜんしゃ の ほう が すぐれて いる と おもいます	
\\	私[わたし]は
\\	の 方[ほう]が 優[すぐ]れていると 思[おも]います。			
\\	効く	効[き]く	きく	
\\	この薬はあまりよく効かないよ。	この 薬[くすり]はあまりよく 効[き]かないよ。	この くすり は あまり よく きかない よ	
\\	この 薬[くすり]はあまりよく
\\	よ。			
\\	効力	効力[こうりょく]	こうりょく	
\\	その薬は効力が強いよ。	その 薬[くすり]は 効力[こうりょく]が 強[つよ]いよ。	その くすり は こうりょく が つよい よ	
\\	その 薬[くすり]は
\\	が 強[つよ]いよ。			
\\	ストライキ	ストライキ	ストライキ	
\\	従業員は明日からストライキです。	従業員[じゅうぎょういん]は 明日[あす]からストライキです。	じゅうぎょういん は あす から すとらいき です	
\\	従業員[じゅうぎょういん]は 明日[あす]から
\\	です。			
\\	効き目	効[き]き 目[め]	ききめ	
\\	この薬の効き目は素晴らしいの。	この 薬[くすり]の 効[き]き 目[め]は 素晴[すば]らしいの。	この くすり の ききめ は すばらしい の	
\\	この 薬[くすり]の
\\	は 素晴[すば]らしいの。			
\\	あんまり	あんまり	あんまり	
\\	これはあんまり好きじゃないな。	これはあんまり 好[す]きじゃないな。	これ は あんまり すき じゃ ない な	
\\	これは
\\	好[す]きじゃないな。			
\\	信用	信用[しんよう]	しんよう	
\\	彼の言うことは信用できない。	彼[かれ]の 言[い]うことは 信用[しんよう]できない。	かれ の いう こと は しんよう できない	
\\	彼[かれ]の 言[い]うことは
\\	できない。			
\\	信頼	信頼[しんらい]	しんらい	
\\	私は部下を信頼しています。	私[わたし]は 部下[ぶか]を 信頼[しんらい]しています。	わたし は ぶか を しんらい して います	
\\	私[わたし]は 部下[ぶか]を
\\	しています。			
\\	依頼	依頼[いらい]	いらい	
\\	彼に協力を依頼しました。	彼[かれ]に 協力[きょうりょく]を 依頼[いらい]しました。	かれ に きょうりょく を いらい しました	
\\	彼[かれ]に 協力[きょうりょく]を
\\	しました。			
\\	人民	人民[じんみん]	じんみん	
\\	彼は人民のための政治を行った。	彼[かれ]は 人民[じんみん]のための 政治[せいじ]を 行[おこな]った。	かれ は じんみん の ため の せいじ を おこなった	
\\	彼[かれ]は
\\	のための 政治[せいじ]を 行[おこな]った。			
\\	主要	主要[しゅよう]	しゅよう	
\\	産業はインドの主要産業よね。	
\\	産業[あいてぃー さんぎょう]はインドの 主要[しゅよう] 産業[さんぎょう]よね。	あいてぃー さんぎょう は いんど の しゅよう さんぎょう よ ね	
\\	産業[あいてぃー さんぎょう]はインドの
\\	産業[さんぎょう]よね。			
\\	主に	主[おも]に	おもに	
\\	この商品は主に女性に人気がありまして。	この 商品[しょうひん]は 主[おも]に 女性[じょせい]に 人気[にんき]がありまして。	この しょうひん は おもに じょせい に にんき が ありまし て	
\\	この 商品[しょうひん]は
\\	女性[じょせい]に 人気[にんき]がありまして。			
\\	主役	主役[しゅやく]	しゅやく	
\\	彼は学校の劇で主役に選ばれたよ。	彼[かれ]は 学校[がっこう]の 劇[げき]で 主役[しゅやく]に 選[えら]ばれたよ。	かれ は がっこう の げき で しゅやく に えらばれた よ	
\\	彼[かれ]は 学校[がっこう]の 劇[げき]で
\\	に 選[えら]ばれたよ。			
\\	うるさい	うるさい	うるさい	
\\	あまりうるさく言わないで下さい。	あまりうるさく 言[い]わないで 下[くだ]さい。	あまり うるさく いわない で ください	
\\	あまり
\\	言[い]わないで 下[くだ]さい。			
\\	主体	主体[しゅたい]	しゅたい	
\\	政治は国民主体であるべきです。	政治[せいじ]は 国民[こくみん] 主体[しゅたい]であるべきです。	せいじ は こくみん しゅたい で ある べき です	
\\	政治[せいじ]は 国民[こくみん]
\\	であるべきです。			
\\	主題	主題[しゅだい]	しゅだい	
\\	この小説の主題は反戦です。	この 小説[しょうせつ]の 主題[しゅだい]は 反戦[はんせん]です。	この しょうせつ の しゅだい は はんせん です	
\\	この 小説[しょうせつ]の
\\	は 反戦[はんせん]です。			
\\	主食	主食[しゅしょく]	しゅしょく	
\\	アジアの主食は米です。	アジアの 主食[しゅしょく]は 米[こめ]です。	あじあ の しゅしょく は こめ です	
\\	アジアの
\\	は 米[こめ]です。			
\\	主	主[おも]	おも	
\\	彼女の主な仕事は接客だよ。	彼女[かのじょ]の 主[おも]な 仕事[しごと]は 接客[せっきゃく]だよ。	かのじょ の おも な しごと は せっきゃく だ よ	
\\	彼女[かのじょ]の
\\	な 仕事[しごと]は 接客[せっきゃく]だよ。			
\\	主義	主義[しゅぎ]	しゅぎ	
\\	私は車は持たない主義です。	私[わたし]は 車[くるま]は 持[も]たない 主義[しゅぎ]です。	わたし は くるま は もたない しゅぎ です	
\\	私[わたし]は 車[くるま]は 持[も]たない
\\	です。			
\\	共産主義	共産主義[きょうさんしゅぎ]	きょうさんしゅぎ	
\\	その国は共産主義の国だよ。	その 国[くに]は 共産主義[きょうさんしゅぎ]の 国[くに]だよ。	その くに は きょうさんしゅぎ の くに だ よ	
\\	その 国[くに]は
\\	の 国[くに]だよ。			
\\	カーブ	カーブ	カーブ	
\\	このカーブは気をつけて。	このカーブは 気[き]をつけて。	この かーぶ は き を つけて	
\\	この
\\	は 気[き]をつけて。			
\\	協議	協議[きょうぎ]	きょうぎ	
\\	委員会はその問題について協議しましたわ。	委員会[いいんかい]はその 問題[もんだい]について 協議[きょうぎ]しましたわ。	いいんかい は その もんだい に ついて きょうぎ しました わ	
\\	委員会[いいんかい]はその 問題[もんだい]について
\\	しましたわ。			
\\	ガイド	ガイド	ガイド	
\\	ガイドつきの旅行を申し込んだの。	ガイドつきの 旅行[りょこう]を 申[もう]し 込[こ]んだの。	がいど つき の りょこう を もうしこんだ の	
\\	つきの 旅行[りょこう]を 申[もう]し 込[こ]んだの。			
\\	世論	世論[せろん]	せろん	
\\	世論はあなたの味方です。	世論[せろん]はあなたの 味方[みかた]です。	せろん は あなた の みかた です	
\\	はあなたの 味方[みかた]です。			
\\	一応	一応[いちおう]	いちおう	
\\	彼にも一応知らせておいたほうがいいだろう。	彼[かれ]にも 一応[いちおう] 知[し]らせておいたほうがいいだろう。	かれ に も いちおう しらせて おいた ほう が いいだろう	
\\	彼[かれ]にも
\\	知[し]らせておいたほうがいいだろう。			
\\	かえる	かえる	かえる	
\\	田んぼの中でかえるが合唱している。	田[た]んぼの 中[なか]でかえるが 合唱[がっしょう]している。	たんぼ の なか で かえる が がっしょう している 。	
\\	田[た]んぼの 中[なか]で
\\	が 合唱[がっしょう]している。			
\\	区別	区別[くべつ]	くべつ	
\\	あの二人はそっくりで区別できない。	あの 二人[ふたり]はそっくりで 区別[くべつ]できない。	あの ふたり は そっくり で くべつ できない	
\\	あの 二人[ふたり]はそっくりで
\\	できない。			
\\	一般に	一般[いっぱん]に	いっぱんに	
\\	一般に老人は早起きです。	一般[いっぱん]に 老人[ろうじん]は 早起[はやお]きです。	いっぱんに ろうじん は はやおき です	
\\	老人[ろうじん]は 早起[はやお]きです。			
\\	ずれる	ずれる	ずれる	
\\	ポスターの位置が少しずれているよ。	ポスターの 位置[いち]が 少[すこ]しずれているよ。	ぽすたー の いち が すこし ずれて いる よ	
\\	ポスターの 位置[いち]が 少[すこ]し
\\	よ。			
\\	上等	上等[じょうとう]	じょうとう	
\\	上等なワインを飲んだの。	上等[じょうとう]なワインを 飲[の]んだの。	じょうとう な わいん を のんだ の	
\\	なワインを 飲[の]んだの。			
\\	オーバー	オーバー	オーバー	
\\	彼の話はいつもオーバーです。	彼[かれ]の 話[はなし]はいつもオーバーです。	かれ の はなし は いつも おーばー です	
\\	彼[かれ]の 話[はなし]はいつも
\\	です。			
\\	医学	医学[いがく]	いがく	
\\	彼は医学を学んでいます。	彼[かれ]は 医学[いがく]を 学[まな]んでいます。	かれ は いがく を まなんで います	
\\	彼[かれ]は
\\	を 学[まな]んでいます。			
\\	エアコン	エアコン	エアコン	
\\	部屋にエアコンを取り付けたんだ。	部屋[へや]にエアコンを 取[と]り 付[つ]けたんだ。	へや に えあこん を とりつけた ん だ	
\\	部屋[へや]に
\\	を 取[と]り 付[つ]けたんだ。			
\\	人文科学	人文科学[じんぶんかがく]	じんぶんかがく	
\\	大学で人文科学を専攻しました。	大学[だいがく]で 人文科学[じんぶんかがく]を 専攻[せんこう]しました。	だいがく で じんぶんかがく を せんこう しました	
\\	大学[だいがく]で
\\	を 専攻[せんこう]しました。			
\\	しょっちゅう	しょっちゅう	しょっちゅう	
\\	彼はしょっちゅう遅刻しているな。	彼[かれ]はしょっちゅう 遅刻[ちこく]しているな。	かれ は しょっちゅう ちこく して いる な	
\\	彼[かれ]は
\\	遅刻[ちこく]しているな。			
\\	スタンド	スタンド	スタンド	
\\	姉はスタンドの下で本を読んでいたんだ。	姉[あね]はスタンドの 下[した]で 本[ほん]を 読[よ]んでいたんだ。	あね は すたんど の した で ほん を よんで いた ん だ	
\\	姉[あね]は
\\	の 下[した]で 本[ほん]を 読[よ]んでいたんだ。			
\\	ご存じ	ご 存[ぞん]じ	ごぞんじ	
\\	彼の名前をご存じですか。	彼[かれ]の 名前[なまえ]をご 存[ぞん]じですか。	かれ の なまえ を ごぞんじ です か	
\\	彼[かれ]の 名前[なまえ]を
\\	ですか。			
\\	がっかり	がっかり	がっかり	
\\	成績が落ちてがっかりした。	成績[せいせき]が 落[お]ちてがっかりした。	せいせき が おちて がっかり した	
\\	成績[せいせき]が 落[お]ちて
\\	した。			
\\	スライド	スライド	スライド	
\\	このふたはスライドします。	このふたはスライドします。	この ふた は すらいど します	
\\	このふたは
\\	します。			
\\	効率	効率[こうりつ]	こうりつ	
\\	作業の効率を上げるにはどうしたらいいですか。	作業[さぎょう]の 効率[こうりつ]を 上[あ]げるにはどうしたらいいですか。	さぎょう の こうりつ を あげる に は どう したら いい です か	
\\	作業[さぎょう]の
\\	を 上[あ]げるにはどうしたらいいですか。			
\\	おっしゃる	おっしゃる	おっしゃる	
\\	あなたのおっしゃる通りです。	あなたのおっしゃる 通[とお]りです。	あなた の おっしゃる とおり です	
\\	あなたの
\\	通[とお]りです。			
\\	危機	危機[きき]	きき	
\\	地球環境の危機が叫ばれているのよ。	地球環境[ちきゅう かんきょう]の 危機[きき]が 叫[さけ]ばれているのよ。	ちきゅう かんきょう の きき が さけばれて いる の よ	
\\	地球環境[ちきゅう かんきょう]の
\\	が 叫[さけ]ばれているのよ。			
\\	クラシック	クラシック	クラシック	
\\	彼女はクラシック音楽が好きです。	彼女[かのじょ]はクラシック 音楽[おんがく]が 好[す]きです。	かのじょ は くらしっく おんがく が すき です	
\\	彼女[かのじょ]は
\\	音楽[おんがく]が 好[す]きです。			
\\	具合	具合[ぐあい]	ぐあい	
\\	今日は体の具合が悪いです。	今日[きょう]は 体[からだ]の 具合[ぐあい]が 悪[わる]いです。	きょう は からだ の ぐあい が わるい です	
\\	今日[きょう]は 体[からだ]の
\\	が 悪[わる]いです。			
\\	さっと	さっと	さっと	
\\	こぼれた牛乳をさっと拭き取ったの。	こぼれた 牛乳[ぎゅうにゅう]をさっと 拭[ふ]き 取[と]ったの。	こぼれた ぎゅうにゅう を さっと ふきとった の	
\\	こぼれた 牛乳[ぎゅうにゅう]を
\\	拭[ふ]き 取[と]ったの。			
\\	キャンセル	キャンセル	キャンセル	
\\	予約をキャンセルしたよ。	予約[よやく]をキャンセルしたよ。	よやく を きゃんせる した よ	
\\	予約[よやく]を
\\	したよ。			
\\	公式	公式[こうしき]	こうしき	
\\	今後の方針が公式に発表されました。	今後[こんご]の 方針[ほうしん]が 公式[こうしき]に 発表[はっぴょう]されました。	こんご の ほうしん が こうしき に はっぴょう されました	
\\	今後[こんご]の 方針[ほうしん]が
\\	に 発表[はっぴょう]されました。			
\\	公共	公共[こうきょう]	こうきょう	
\\	公共の乗り物は誰でも利用できます。	公共[こうきょう]の 乗[の]り 物[もの]は 誰[だれ]でも 利用[りよう]できます。	こうきょう の のりもの は だれ で も りよう できます	
\\	の 乗[の]り 物[もの]は 誰[だれ]でも 利用[りよう]できます。			
\\	シーツ	シーツ	シーツ	
\\	メイドがシーツを交換してくれたね。	メイドがシーツを 交換[こうかん]してくれたね。	めいど が しーつ を こうかん して くれた ね	
\\	メイドが
\\	を 交換[こうかん]してくれたね。			
\\	公平	公平[こうへい]	こうへい	
\\	賞金をもらったら公平に分けましょう。	賞金[しょうきん]をもらったら 公平[こうへい]に 分[わ]けましょう。	しょうきん を もらったら こうへい に わけましょう	
\\	賞金[しょうきん]をもらったら
\\	に 分[わ]けましょう。			
\\	公立	公立[こうりつ]	こうりつ	
\\	駅前に公立の図書館があるよ。	駅前[えきまえ]に 公立[こうりつ]の 図書館[としょかん]があるよ。	えきまえ に こうりつ の としょかん が ある よ	
\\	駅前[えきまえ]に
\\	の 図書館[としょかん]があるよ。			
\\	交際	交際[こうさい]	こうさい	
\\	彼はモデルと交際していたんだ。	彼[かれ]はモデルと 交際[こうさい]していたんだ。	かれ は もでる と こうさい して いた ん だ	
\\	彼[かれ]はモデルと
\\	していたんだ。			
\\	かかる	かかる	かかる	
\\	彼女はインフルエンザにかかっていますね。	彼女[かのじょ]はインフルエンザにかかっていますね。	かのじょ は いんふるえんざ に かかって います ね	
\\	彼女[かのじょ]はインフルエンザに
\\	ね。			
\\	コック	コック	コック	
\\	彼は腕のいいコックです。	彼[かれ]は 腕[うで]のいいコックです。	かれ は うで の いい こっく です	
\\	彼[かれ]は 腕[うで]のいい
\\	です。			
\\	えび	えび	えび	
\\	私はえびのてんぷらが好きです。	私[わたし]はえびのてんぷらが 好[す]きです。	わたし は えび の てんぷら が すき です	
\\	私[わたし]は
\\	のてんぷらが 好[す]きです。			
\\	勝敗	勝敗[しょうはい]	しょうはい	
\\	このセットで勝敗が決まります。	このセットで 勝敗[しょうはい]が 決[き]まります。	この せっと で しょうはい が きまります	
\\	このセットで
\\	が 決[き]まります。			
\\	原則	原則[げんそく]	げんそく	
\\	原則としてキャンセルできません。	原則[げんそく]としてキャンセルできません。	げんそく と して きゃんせる できません	
\\	としてキャンセルできません。			
\\	原料	原料[げんりょう]	げんりょう	
\\	石油はいろいろな製品の原料になります。	石油[せきゆ]はいろいろな 製品[せいひん]の 原料[げんりょう]になります。	せきゆ は いろいろ な せいひん の げんりょう に なります	
\\	石油[せきゆ]はいろいろな 製品[せいひん]の
\\	になります。			
\\	原理	原理[げんり]	げんり	
\\	見学者に機械の動く原理を説明しました。	見学者[けんがくしゃ]に 機械[きかい]の 動[うご]く 原理[げんり]を 説明[せつめい]しました。	けんがくしゃ に きかい の うごく げんり を せつめい しました	
\\	見学者[けんがくしゃ]に 機械[きかい]の 動[うご]く
\\	を 説明[せつめい]しました。			
\\	原子力	原子力[げんしりょく]	げんしりょく	
\\	原子力の安全な利用について考えましょう。	原子力[げんしりょく]の 安全[あんぜん]な 利用[りよう]について 考[かんが]えましょう。	げんしりょく の あんぜん な りよう に ついて かんがえましょう	
\\	の 安全[あんぜん]な 利用[りよう]について 考[かんが]えましょう。			
\\	グリーン	グリーン	グリーン	
\\	そのグリーンのスカーフ、素敵ですね。	そのグリーンのスカーフ、 素敵[すてき]ですね。	その ぐりーん の すかーふ すてき です ね	
\\	その
\\	のスカーフ、 素敵[すてき]ですね。			
\\	ジーパン	ジーパン	ジーパン	
\\	彼はいつもジーパンをはいているね。	彼[かれ]はいつもジーパンをはいているね。	かれ は いつも じーぱん を はいて いる ね	
\\	彼[かれ]はいつも
\\	をはいているね。			
\\	ジャーナリズム	ジャーナリズム	ジャーナリズム	
\\	彼はジャーナリズムを専攻している。	彼[かれ]はジャーナリズムを 専攻[せんこう]している。	かれ は じゃーなりずむ を せんこう して いる	
\\	彼[かれ]は
\\	を 専攻[せんこう]している。			
\\	スピーチ	スピーチ	スピーチ	
\\	彼のスピーチは素晴らしかった。	彼[かれ]のスピーチは 素晴[すば]らしかった。	かれ の すぴーち は すばらしかった 。	
\\	彼[かれ]の
\\	は 素晴[すば]らしかった。			
\\	下調べ	下調[したしら]べ	したしらべ	
\\	まず第一に、しっかり下調べをしなさい。	まず 第一[だいいち]に、しっかり 下調[したしら]べをしなさい。	まず だいいち に しっかり したしらべ を しなさい	
\\	まず 第一[だいいち]に、しっかり
\\	をしなさい。			
\\	ああ	ああ	ああ	
\\	ああうるさい人は苦手です。	ああうるさい 人[ひと]は 苦手[にがて]です。	ああ うるさい ひと は にがて です	
\\	うるさい 人[ひと]は 苦手[にがて]です。			
\\	アクセント	アクセント	アクセント	
\\	アメリカ英語とイギリス英語ではアクセントが違うことがあるね。	アメリカ 英語[えいご]とイギリス 英語[えいご]ではアクセントが 違[ちが]うことがあるね。	あめりか えいご と いぎりす えいご で は あくせんと が ちがう こと が ある ね	
\\	アメリカ 英語[えいご]とイギリス 英語[えいご]では
\\	が 違[ちが]うことがあるね。			
\\	コマーシャル	コマーシャル	コマーシャル	
\\	彼女はテレビのコマーシャルに出ているわ。	彼女[かのじょ]はテレビのコマーシャルに 出[で]ているわ。	かのじょ は てれび の こまーしゃる に でて いる わ	
\\	彼女[かのじょ]はテレビの
\\	に 出[で]ているわ。			
\\	係	係[かかり]	かかり	
\\	彼女は会場整理の係だったの。	彼女[かのじょ]は 会場整理[かいじょう せいり]の 係[かかり]だったの。	かのじょ は かいじょう せいり の かかり だった の	
\\	彼女[かのじょ]は 会場整理[かいじょう せいり]の
\\	だったの。			
\\	事態	事態[じたい]	じたい	
\\	事態は深刻です。	事態[じたい]は 深刻[しんこく]です。	じたい は しんこく です	
\\	は 深刻[しんこく]です。			
\\	しつこい	しつこい	しつこい	
\\	しつこい迷惑メールに困っている。	しつこい 迷惑[めいわく]メールに 困[こま]っている。	しつこい めいわく めーる に こまって いる	
\\	迷惑[めいわく]メールに 困[こま]っている。			
\\	しゃがむ	しゃがむ	しゃがむ	
\\	お年寄りが道端にしゃがんでいますね。	お 年寄[としよ]りが 道端[みちばた]にしゃがんでいますね。	おとしより が みちばた に しゃがんで います ね	
\\	お 年寄[としよ]りが 道端[みちばた]に
\\	ね。			
\\	コンテスト	コンテスト	コンテスト	
\\	このコンテストに優勝すると車がもらえるんだ。	このコンテストに 優勝[ゆうしょう]すると 車[くるま]がもらえるんだ。	この こんてすと に ゆうしょう する と くるま が もらえる ん だ	
\\	この
\\	に 優勝[ゆうしょう]すると 車[くるま]がもらえるんだ。			
\\	キャベツ	キャベツ	キャベツ	
\\	キャベツの千切りを添えたよ。	キャベツの 千切[せんぎ]りを 添[そ]えたよ。	きゃべつ の せんぎり を そえた よ	
\\	の 千切[せんぎ]りを 添[そ]えたよ。			
\\	仕組み	仕組[しく]み	しくみ	
\\	この機械の仕組みは複雑だ。	この 機械[きかい]の 仕組[しく]みは 複雑[ふくざつ]だ。	この きかい の しくみ は ふくざつ だ	
\\	この 機械[きかい]の
\\	は 複雑[ふくざつ]だ。			
\\	アイドル	アイドル	アイドル	
\\	彼女は若者のアイドルです。	彼女[かのじょ]は 若者[わかもの]のアイドルです。	かのじょ は わかもの の あいどる です	
\\	彼女[かのじょ]は 若者[わかもの]の
\\	です。			
\\	前進	前進[ぜんしん]	ぜんしん	
\\	一列に並んで前進して下さい。	一列[いちれつ]に 並[なら]んで 前進[ぜんしん]して 下[くだ]さい。	いちれつ に ならんで ぜんしん して ください	
\\	一列[いちれつ]に 並[なら]んで
\\	して 下[くだ]さい。			
\\	からかう	からかう	からかう	
\\	彼は時々妹をからかいます。	彼[かれ]は 時々妹[ときどき いもうと]をからかいます。	かれ は ときどき いもうと を からかいます	
\\	彼[かれ]は 時々妹[ときどき いもうと]を
\\	主任	主任[しゅにん]	しゅにん	
\\	彼女は会計主任です。	彼女[かのじょ]は 会計[かいけい] 主任[しゅにん]です。	かのじょ は かいけい しゅにん です	
\\	彼女[かのじょ]は 会計[かいけい]
\\	です。			
\\	お世辞	お 世辞[せじ]	おせじ	
\\	彼はお世辞を言うのが上手い。	彼[かれ]はお 世辞[せじ]を 言[い]うのが 上手[うま]い。	かれ は おせじ を いう の が うまい	
\\	彼[かれ]は
\\	を 言[い]うのが 上手[うま]い。			
\\	すっきり	すっきり	すっきり	
\\	よく眠ったら気分がすっきりした。	よく 眠[ねむ]ったら 気分[きぶん]がすっきりした。	よく ねむったら きぶん が すっきり した	
\\	よく 眠[ねむ]ったら 気分[きぶん]が
\\	した。			
\\	出勤	出勤[しゅっきん]	しゅっきん	
\\	毎朝7時5分に出勤します。	毎朝7時5分[まいあさ しち じ ご ふん]に 出勤[しゅっきん]します。	まいあさ しち じ ご ふん に しゅっきん します	
\\	毎朝7時5分[まいあさ しち じ ご ふん]に
\\	します。			
\\	勤勉	勤勉[きんべん]	きんべん	
\\	彼は勤勉な人です。	彼[かれ]は 勤勉[きんべん]な 人[ひと]です。	かれ は きんべん な ひと です	
\\	彼[かれ]は
\\	な 人[ひと]です。			
\\	勤務	勤務[きんむ]	きんむ	
\\	私の一日の勤務時間は8時間です。	私[わたし]の 一日[いちにち]の 勤務[きんむ] 時間[じかん]は 8時間[はちじかん]です。	わたし の いちにち の きんむじかん は はちじかん です	
\\	私[わたし]の 一日[いちにち]の
\\	時間[じかん]は 8時間[はちじかん]です。			
\\	事務	事務[じむ]	じむ	
\\	事務の経験が3年あります。	事務[じむ]の 経験[けいけん]が 3年[さんねん]あります。	じむ の けいけん が さんねん あります	
\\	の 経験[けいけん]が 3年[さんねん]あります。			
\\	公務員	公務員[こうむいん]	こうむいん	
\\	私の父は公務員です。	私[わたし]の 父[ちち]は 公務員[こうむいん]です。	わたし の ちち は こうむいん です	
\\	私[わたし]の 父[ちち]は
\\	です。			
\\	あいにく	あいにく	あいにく	
\\	途中であいにく雨が降り出したの。	途中[とちゅう]であいにく 雨[あめ]が 降[ふ]り 出[だ]したの。	とちゅう で あいにく あめ が ふりだした の	
\\	途中[とちゅう]で
\\	雨[あめ]が 降[ふ]り 出[だ]したの。			
\\	事務員	事務員[じむいん]	じむいん	
\\	新しい事務員が入りました。	新[あたら]しい 事務員[じむいん]が 入[はい]りました。	あたらしい じむいん が はいりました	
\\	新[あたら]しい
\\	が 入[はい]りました。			
\\	おばさん	おばさん	おばさん	
\\	おばさん、こんにちは。	おばさん、こんにちは。	おばさん、こんにちは。	
\\	、こんにちは。			
\\	収集	収集[しゅうしゅう]	しゅうしゅう	
\\	彼の趣味は切手収集です。	彼[かれ]の 趣味[しゅみ]は 切手[きって] 収集[しゅうしゅう]です。	かれ の しゅみ は きって しゅうしゅう です	
\\	彼[かれ]の 趣味[しゅみ]は 切手[きって]
\\	です。			
\\	供給	供給[きょうきゅう]	きょうきゅう	
\\	彼の会社は電力を供給しています。	彼[かれ]の 会社[かいしゃ]は 電力[でんりょく]を 供給[きょうきゅう]しています。	かれ の かいしゃ は でんりょく を きょうきゅう して います	
\\	彼[かれ]の 会社[かいしゃ]は 電力[でんりょく]を
\\	しています。			
\\	ジャンプ	ジャンプ	ジャンプ	
\\	猿が高い木にジャンプした。	猿[さる]が 高[たか]い 木[き]にジャンプした。	さる が たかい き に じゃんぷ した	
\\	猿[さる]が 高[たか]い 木[き]に
\\	した。			
\\	あちらこちら	あちらこちら	あちらこちら	
\\	あちらこちらで紅葉がきれいですね。	あちらこちらで 紅葉[こうよう]がきれいですね。	あちらこちら で こうよう が きれい です ね	
\\	で 紅葉[こうよう]がきれいですね。			
\\	参考	参考[さんこう]	さんこう	
\\	この本を参考にして下さい。	この 本[ほん]を 参考[さんこう]にして 下[くだ]さい。	この ほん を さんこう に して ください	
\\	この 本[ほん]を
\\	にして 下[くだ]さい。			
\\	参議院	参議院[さんぎいん]	さんぎいん	
\\	彼は参議院議員です。	彼[かれ]は 参議院[さんぎいん] 議員[ぎいん]です。	かれ は さんぎいん ぎいん です	
\\	彼[かれ]は
\\	議員[ぎいん]です。			
\\	お参り	お 参[まい]り	おまいり	
\\	家族でお寺にお参りに行きました。	家族[かぞく]でお 寺[てら]にお 参[まい]りに 行[い]きました。	かぞく で おてら に おまいり に いきました	
\\	家族[かぞく]でお 寺[てら]に
\\	に 行[い]きました。			
\\	参考書	参考書[さんこうしょ]	さんこうしょ	
\\	この参考書はとても役に立つよ。	この 参考書[さんこうしょ]はとても 役[やく]に 立[た]つよ。	この さんこうしょ は とても やく に たつ よ	
\\	この
\\	はとても 役[やく]に 立[た]つよ。			
\\	加わる	加[くわ]わる	くわわる	
\\	私たちのチームに彼が加わった。	私[わたし]たちのチームに 彼[かれ]が 加[くわ]わった。	わたしたち の ちーむ に かれ が くわわった	
\\	私[わたし]たちのチームに 彼[かれ]が
\\	いい加減	いい 加減[かげん]	いいかげん	
\\	いい加減なことを言ってはいけません。	いい 加減[かげん]なことを 言[い]ってはいけません。	いいかげん な こと を いって は いけません	
\\	なことを 言[い]ってはいけません。			
\\	いつのまにか	いつのまにか	いつのまにか	
\\	いつのまにか夜が明けていたね。	いつのまにか 夜[よ]が 明[あ]けていたね。	いつのまにか よ が あけて いた ね	
\\	夜[よ]が 明[あ]けていたね。			
\\	おごる	おごる	おごる	
\\	彼に食事をおごってもらった。	彼[かれ]に 食事[しょくじ]をおごってもらった。	かれ に しょくじ を おごって もらった	
\\	彼[かれ]に 食事[しょくじ]を
\\	修正	修正[しゅうせい]	しゅうせい	
\\	検討の結果、案を修正したよ。	検討[けんとう]の 結果[けっか]、 案[あん]を 修正[しゅうせい]したよ。	けんとう の けっか あん を しゅうせい した よ	
\\	検討[けんとう]の 結果[けっか]、 案[あん]を
\\	したよ。			
\\	おやつ	おやつ	おやつ	
\\	今日のおやつはプリンだった。	今日[きょう]のおやつはプリンだった。	きょう の おやつ は ぷりん だった	
\\	今日[きょう]の
\\	はプリンだった。			
\\	傷	傷[きず]	きず	
\\	足の傷が痛みます。	足[あし]の 傷[きず]が 痛[いた]みます。	あし の きず が いたみます	
\\	足[あし]の
\\	が 痛[いた]みます。			
\\	傷める	傷[いた]める	いためる	
\\	彼は柔道で腰を傷めたんだ。	彼[かれ]は 柔道[じゅうどう]で 腰[こし]を 傷[いた]めたんだ。	かれ は じゅうどう で こし を いためた ん だ	
\\	彼[かれ]は 柔道[じゅうどう]で 腰[こし]を
\\	んだ。			
\\	交換	交換[こうかん]	こうかん	
\\	試合の相手とユニフォームを交換したよ。	試合[しあい]の 相手[あいて]とユニフォームを 交換[こうかん]したよ。	しあい の あいて と ゆにふぉーむ を こうかん した よ	
\\	試合[しあい]の 相手[あいて]とユニフォームを
\\	したよ。			
\\	交替	交替[こうたい]	こうたい	
\\	家まで交替で荷物を持ちました。	家[いえ]まで 交替[こうたい]で 荷物[にもつ]を 持[も]ちました。	いえ まで こうたい で にもつ を もちました	
\\	家[いえ]まで
\\	で 荷物[にもつ]を 持[も]ちました。			
\\	ごろごろ	ごろごろ	ごろごろ	
\\	雷がごろごろ鳴っている。	雷[かみなり]がごろごろ 鳴[な]っている。	かみなり が ごろごろ なって いる	
\\	雷[かみなり]が
\\	鳴[な]っている。			
\\	公害	公害[こうがい]	こうがい	
\\	私たちは公害を減らすよう努力しています。	私[わたし]たちは 公害[こうがい]を 減[へ]らすよう 努力[どりょく]しています。	わたしたち は こうがい を へらす よう どりょく して います	
\\	私[わたし]たちは
\\	を 減[へ]らすよう 努力[どりょく]しています。			
\\	シューズ	シューズ	シューズ	
\\	ジョギング用のシューズは安くないな。	ジョギング 用[よう]のシューズは 安[やす]くないな。	じょぎんぐ よう の しゅーず は やすくない な	
\\	ジョギング 用[よう]の
\\	は 安[やす]くないな。			
\\	助手	助手[じょしゅ]	じょしゅ	
\\	資料は助手に預けておいてください。	資料[しりょう]は 助手[じょしゅ]に 預[あず]けておいてください。	しりょう は じょしゅ に あずけて おいて ください	
\\	資料[しりょう]は
\\	に 預[あず]けておいてください。			
\\	キス	キス	キス	
\\	初めてのキスは海岸でだったの。	初[はじ]めてのキスは 海岸[かいがん]でだったの。	はじめて の きす は かいがん で だった の	
\\	初[はじ]めての
\\	は 海岸[かいがん]でだったの。			
\\	くるくる	くるくる	くるくる	
\\	猫は私の足元をくるくると回ったの。	猫[ねこ]は 私[わたし]の 足元[あしもと]をくるくると 回[まわ]ったの。	ねこ は わたし の あしもと を くるくる と まわった の	
\\	猫[ねこ]は 私[わたし]の 足元[あしもと]を
\\	と 回[まわ]ったの。			
\\	作戦	作戦[さくせん]	さくせん	
\\	彼らは次の試合のために作戦を立てたのよ。	彼[かれ]らは 次[つぎ]の 試合[しあい]のために 作戦[さくせん]を 立[た]てたのよ。	かれら は つぎ の しあい の ため に さくせん を たてた の よ	
\\	彼[かれ]らは 次[つぎ]の 試合[しあい]のために
\\	を 立[た]てたのよ。			
\\	争い	争[あらそ]い	あらそい	
\\	その地域では争いが絶えません。	その 地域[ちいき]では 争[あらそ]いが 絶[た]えません。	その ちいき で は あらそい が たえません	
\\	その 地域[ちいき]では
\\	が 絶[た]えません。			
\\	争う	争[あらそ]う	あらそう	
\\	その2国は資源をめぐって争っています。	その 2国[にこく]は 資源[しげん]をめぐって 争[あらそ]っています。	その にこく は しげん を めぐって あらそって います	
\\	その 2国[にこく]は 資源[しげん]をめぐって
\\	サボる	サボる	サボる	
\\	また仕事をサボっていますね。	また 仕事[しごと]をサボっていますね。	また しごと を さぼって います ね	
\\	また 仕事[しごと]を
\\	ね。			
\\	ショッピング	ショッピング	ショッピング	
\\	彼女のいちばんの楽しみはショッピングです。	彼女[かのじょ]のいちばんの 楽[たの]しみはショッピングです。	かのじょ の いちばん の たのしみ は しょっぴんぐ です	
\\	彼女[かのじょ]のいちばんの 楽[たの]しみは
\\	です。			
\\	交流	交流[こうりゅう]	こうりゅう	
\\	もっと他の町との交流を深めましょう。	もっと 他[ほか]の 町[まち]との 交流[こうりゅう]を 深[ふか]めましょう。	もっと ほか の まち と の こうりゅう を ふかめましょう	
\\	もっと 他[ほか]の 町[まち]との
\\	を 深[ふか]めましょう。			
\\	一流	一流[いちりゅう]	いちりゅう	
\\	彼は一流の選手です。	彼[かれ]は 一流[いちりゅう]の 選手[せんしゅ]です。	かれ は いちりゅう の せんしゅ です	
\\	彼[かれ]は
\\	の 選手[せんしゅ]です。			
\\	上流	上流[じょうりゅう]	じょうりゅう	
\\	上流には滝があります。	上流[じょうりゅう]には 滝[たき]があります。	じょうりゅう に は たき が あります	
\\	には 滝[たき]があります。			
\\	下流	下流[かりゅう]	かりゅう	
\\	下流に小さな滝があるよ。	下流[かりゅう]に 小[ちい]さな 滝[たき]があるよ。	かりゅう に ちいさ な たき が ある よ	
\\	に 小[ちい]さな 滝[たき]があるよ。			
\\	三流	三流[さんりゅう]	さんりゅう	
\\	彼は三流大学を出たが、今は社長だよ。	彼[かれ]は 三流[さんりゅう] 大学[だいがく]を 出[で]たが、 今[いま]は 社長[しゃちょう]だよ。	かれ は さんりゅう だいがく を でた が いま は しゃちょう だ よ	
\\	彼[かれ]は
\\	大学[だいがく]を 出[で]たが、 今[いま]は 社長[しゃちょう]だよ。			
\\	スマート	スマート	スマート	
\\	彼はいつもスマートに行動するね。	彼[かれ]はいつもスマートに 行動[こうどう]するね。	かれ は いつも すまーと に こうどう する ね	
\\	彼[かれ]はいつも
\\	に 行動[こうどう]するね。			
\\	ずらり	ずらり	ずらり	
\\	玄関に靴がずらりと並んでいたな。	玄関[げんかん]に 靴[くつ]がずらりと 並[なら]んでいたな。	げんかん に くつ が ずらり と ならんで いた な	
\\	玄関[げんかん]に 靴[くつ]が
\\	と 並[なら]んでいたな。			
\\	いらっしゃる	いらっしゃる	いらっしゃる	
\\	お客様がいらっしゃいました。	お 客様[きゃくさま]がいらっしゃいました。	おきゃくさま が いらっしゃいました	
\\	お 客様[きゃくさま]が
\\	冷ます	冷[さ]ます	さます	
\\	お茶がとても熱かったので冷ましてから飲んだの。	お 茶[ちゃ]がとても 熱[あつ]かったので 冷[さ]ましてから 飲[の]んだの。	おちゃ が とても あつかった の で さまして から のんだ の	
\\	お 茶[ちゃ]がとても 熱[あつ]かったので
\\	から 飲[の]んだの。			
\\	ウナギ	ウナギ	ウナギ	
\\	ウナギの蒲焼きは美味しいね。	ウナギの 蒲焼[かばや]きは 美味[おい]しいね。	うなぎ の かばやき は おいしい ね	
\\	の 蒲焼[かばや]きは 美味[おい]しいね。			
\\	ガード	ガード	ガード	
\\	歩道に新しくガードが付きましたね。	歩道[ほどう]に 新[あたら]しくガードが 付[つ]きましたね。	ほどう に あたらしく がーど が つきました ね	
\\	歩道[ほどう]に 新[あたら]しく
\\	が 付[つ]きましたね。			
\\	かぼちゃ	かぼちゃ	かぼちゃ	
\\	かぼちゃのスープは美味しいです。	かぼちゃのスープは 美味[おい]しいです。	かぼちゃの すーぷ は おいしい です	
\\	のスープは 美味[おい]しいです。			
\\	主観	主観[しゅかん]	しゅかん	
\\	主観だけで物事を見てはいけない。	主観[しゅかん]だけで 物事[ものごと]を 見[み]てはいけない。	しゅかん だけ で ものごと を みて は いけない	
\\	だけで 物事[ものごと]を 見[み]てはいけない。			
\\	主観的	主観的[しゅかんてき]	しゅかんてき	
\\	それは主観的な意見だ。	それは 主観的[しゅかんてき]な 意見[いけん]だ。	それ は しゅかんてき な いけん だ	
\\	それは
\\	な 意見[いけん]だ。			
\\	クイズ	クイズ	クイズ	
\\	クイズ番組を見るのが好きです。	クイズ 番組[ばんぐみ]を 見[み]るのが 好[す]きです。	くいず ばんぐみ を みる の が すき です	
\\	番組[ばんぐみ]を 見[み]るのが 好[す]きです。			
\\	クリーム	クリーム	クリーム	
\\	私は洗い物をしたあと、手にクリームをぬります。	私[わたし]は 洗[あら]い 物[もの]をしたあと、 手[て]にクリームをぬります。	わたし は あらいもの を した あと て に くりーむ を ぬります	
\\	私[わたし]は 洗[あら]い 物[もの]をしたあと、 手[て]に
\\	をぬります。			
\\	厳重	厳重[げんじゅう]	げんじゅう	
\\	上司から厳重に注意されました。	上司[じょうし]から 厳重[げんじゅう]に 注意[ちゅうい]されました。	じょうし から げんじゅう に ちゅうい されました	
\\	上司[じょうし]から
\\	に 注意[ちゅうい]されました。			
\\	ぐるぐる	ぐるぐる	ぐるぐる	
\\	犬が自分のしっぽを追いかけてぐるぐる回っている。	犬[いぬ]が 自分[じぶん]のしっぽを 追[お]いかけてぐるぐる 回[まわ]っている。	いぬ が じぶん の しっぽ を おいかけて ぐるぐる まわって いる	
\\	犬[いぬ]が 自分[じぶん]のしっぽを 追[お]いかけて
\\	回[まわ]っている。			
\\	さぞ	さぞ	さぞ	
\\	それはさぞがっかりしたことでしょう。	それはさぞがっかりしたことでしょう。	それはさぞがっかりしたことでしょう。	
\\	それは
\\	がっかりしたことでしょう。			
\\	サングラス	サングラス	サングラス	
\\	彼はサングラスが似合うね。	彼[かれ]はサングラスが 似合[にあ]うね。	かれ は さんぐらす が にあう ね	
\\	彼[かれ]は
\\	が 似合[にあ]うね。			
\\	位	位[くらい]	くらい	
\\	位が上がれば責任も増えるものだ。	位[くらい]が 上[あ]がれば 責任[せきにん]も 増[ふ]えるものだ。	くらい が あがれ ば せきにん も ふえる もの だ	
\\	が 上[あ]がれば 責任[せきにん]も 増[ふ]えるものだ。			
\\	一周	一周[いっしゅう]	いっしゅう	
\\	私たちは庭園を一周しました。	私[わたし]たちは 庭園[ていえん]を 一周[いっしゅう]しました。	わたしたち は ていえん を いっしゅう しました	
\\	私[わたし]たちは 庭園[ていえん]を
\\	しました。			
\\	きっちり	きっちり	きっちり	
\\	彼女はきっちり3時に来たわ。	彼女[かのじょ]はきっちり 3時[さんじ]に 来[き]たわ。	かのじょ は きっちり さんじ に きた わ	
\\	彼女[かのじょ]は
\\	3時[さんじ]に 来[き]たわ。			
\\	きゅうり	きゅうり	きゅうり	
\\	きゅうりに味噌をつけて食べたの。	きゅうりに 味噌[みそ]をつけて 食[た]べたの。	きゅうり に みそ を つけて たべた の	
\\	に 味噌[みそ]をつけて 食[た]べたの。			
\\	一緒	一緒[いっしょ]	いっしょ	
\\	一緒に食事しようか。	一緒[いっしょ]に 食事[しょくじ]しようか。	いっしょ に しょくじ しよう か	
\\	に 食事[しょくじ]しようか。			
\\	すっと	すっと	すっと	
\\	言いたいことを言ったら胸がすっとした。	言[い]いたいことを 言[い]ったら 胸[むね]がすっとした。	いいたい こと を いったら むね が すっと した	
\\	言[い]いたいことを 言[い]ったら 胸[むね]が
\\	した。			
\\	一帯	一帯[いったい]	いったい	
\\	この辺一帯はリンゴ畑です。	この 辺[へん] 一帯[いったい]はリンゴ 畑[ばたけ]です。	この へん いったい は りんごばたけ です	
\\	この 辺[へん]
\\	はリンゴ 畑[ばたけ]です。			
\\	すらすら	すらすら	すらすら	
\\	その小学生は難しい本をすらすら読んだの。	その 小学生[しょうがくせい]は 難[むずか]しい 本[ほん]をすらすら 読[よ]んだの。	その しょうがくせい は むずかしい ほん を すらすら よんだ の	
\\	その 小学生[しょうがくせい]は 難[むずか]しい 本[ほん]を
\\	読[よ]んだの。			
\\	住宅地	住宅地[じゅうたくち]	じゅうたくち	
\\	その住宅地は便利な場所にあるね。	その 住宅地[じゅうたくち]は 便利[べんり]な 場所[ばしょ]にあるね。	その じゅうたくち は べんり な ばしょ に ある ね	
\\	その
\\	は 便利[べんり]な 場所[ばしょ]にあるね。			
\\	うろうろ	うろうろ	うろうろ	
\\	その男はロビーをしばらくうろうろしていたよ。	その 男[おとこ]はロビーをしばらくうろうろしていたよ。	その おとこ は ろびー を しばらく うろうろ して いた よ	
\\	その 男[おとこ]はロビーをしばらく
\\	していたよ。			
\\	休暇	休暇[きゅうか]	きゅうか	
\\	今度の休暇にフィリピンに行きます。	今度[こんど]の 休暇[きゅうか]にフィリピンに 行[い]きます。	こんど の きゅうか に ふぃりぴん に いきます	
\\	今度[こんど]の
\\	にフィリピンに 行[い]きます。			
\\	おまけ	おまけ	おまけ	
\\	八百屋さんでたくさんおまけしてもらったよ。	八百屋[やおや]さんでたくさんおまけしてもらったよ。	やおや さん で たくさん おまけ して もらった よ	
\\	八百屋[やおや]さんでたくさん
\\	してもらったよ。			
\\	印象	印象[いんしょう]	いんしょう	
\\	彼から良い印象を受けたわ。	彼[かれ]から 良[い]い 印象[いんしょう]を 受[う]けたわ。	かれ から いい いんしょう を うけた わ	
\\	彼[かれ]から 良[い]い
\\	を 受[う]けたわ。			
\\	印	印[しるし]	しるし	
\\	間違いに印を付けておきました。	間違[まちが]いに 印[しるし]を 付[つ]けておきました。	まちがい に しるし を つけて おきました	
\\	間違[まちが]いに
\\	を 付[つ]けておきました。			
\\	クッキー	クッキー	クッキー	
\\	クッキーと紅茶をいただきました。	クッキーと 紅茶[こうちゃ]をいただきました。	くっきー と こうちゃ を いただきました	
\\	と 紅茶[こうちゃ]をいただきました。			
\\	印刷	印刷[いんさつ]	いんさつ	
\\	年賀状を印刷したの。	年賀状[ねんがじょう]を 印刷[いんさつ]したの。	ねんがじょう を いんさつ した の	
\\	年賀状[ねんがじょう]を
\\	したの。			
\\	出版	出版[しゅっぱん]	しゅっぱん	
\\	この本は15年前に出版された。	この 本[ほん]は 15年前[じゅうごねんまえ]に 出版[しゅっぱん]された。	この ほん は じゅうごねんまえ に しゅっぱん された	
\\	この 本[ほん]は 15年前[じゅうごねんまえ]に
\\	された。			
\\	出版社	出版社[しゅっぱんしゃ]	しゅっぱんしゃ	
\\	彼女は出版社で働いています。	彼女[かのじょ]は 出版社[しゅっぱんしゃ]で 働[はたら]いています。	かのじょ は しゅっぱんしゃ で はたらいて います	
\\	彼女[かのじょ]は
\\	で 働[はたら]いています。			
\\	コンセント	コンセント	コンセント	
\\	プラグをコンセントに挿し込んだよ。	プラグをコンセントに 挿[さ]し 込[こ]んだよ。	ぷらぐ を こんせんと に さしこんだ よ	
\\	プラグを
\\	に 挿[さ]し 込[こ]んだよ。			
\\	くっ付く	くっ 付[つ]く	くっつく	
\\	靴の底にガムがくっ付いてしまった。	靴[くつ]の 底[そこ]にガムがくっ 付[つ]いてしまった。	くつ の そこ に がむ が くっついて しまった	
\\	靴[くつ]の 底[そこ]にガムが
\\	ずるい	ずるい	ずるい	
\\	あなたのやり方はずるい。	あなたのやり 方[かた]はずるい。	あなた の やりかた は ずるい	
\\	あなたのやり 方[かた]は
\\	傷付く	傷付[きずつ]く	きずつく	
\\	私は彼女の言葉に傷付きました。	私[わたし]は 彼女[かのじょ]の 言葉[ことば]に 傷付[きずつ]きました。	わたし は かのじょ の ことば に きずつきました	
\\	私[わたし]は 彼女[かのじょ]の 言葉[ことば]に
\\	傷付ける	傷付[きずつ]ける	きずつける	
\\	あなたを傷付けるつもりはありませんでした。	あなたを 傷付[きずつ]けるつもりはありませんでした。	あなた を きずつける つもり は ありませんでした	
\\	あなたを
\\	つもりはありませんでした。			
\\	くっ付ける	くっ 付[つ]ける	くっつける	
\\	彼は壁に耳をくっ付けて隣の話を聞いていたの。	彼[かれ]は 壁[かべ]に 耳[みみ]をくっ 付[つ]けて 隣[となり]の 話[はなし]を 聞[き]いていたの。	かれ は かべ に みみ を くっつけて となり の はなし を きいて いた の	
\\	彼[かれ]は 壁[かべ]に 耳[みみ]を
\\	隣[となり]の 話[はなし]を 聞[き]いていたの。			
\\	せっかち	せっかち	せっかち	
\\	彼はせっかちで困ります。	彼[かれ]はせっかちで 困[こま]ります。	かれ は せっかち で こまります	
\\	彼[かれ]は
\\	で 困[こま]ります。			
\\	うちわ	うちわ	うちわ	
\\	父はうちわで扇いでいるの。	父[ちち]はうちわで 扇[あお]いでいるの。	ちち は うちわ で あおいで いる の	
\\	父[ちち]は
\\	で 扇[あお]いでいるの。			
\\	信念	信念[しんねん]	しんねん	
\\	自分の信念に従いたいと思います。	自分[じぶん]の 信念[しんねん]に 従[したが]いたいと 思[おも]います。	じぶん の しんねん に したがいたい と おもいます	
\\	自分[じぶん]の
\\	に 従[したが]いたいと 思[おも]います。			
\\	うっかり	うっかり	うっかり	
\\	大事なメールをうっかり消してしまった。	大事[だいじ]なメールをうっかり 消[け]してしまった。	だいじ な めーる を うっかり けして しまった	
\\	大事[だいじ]なメールを
\\	消[け]してしまった。			
\\	くだらない	くだらない	くだらない	
\\	くだらないおしゃべりはやめなさい。	くだらないおしゃべりはやめなさい。	くだらないおしゃべりはやめなさい。	
\\	おしゃべりはやめなさい。			
\\	余り	余[あま]り	あまり	
\\	余りは次回使いましょう。	余[あま]りは 次回使[じかい つか]いましょう。	あまり は じかい つかいましょう	
\\	は 次回使[じかい つか]いましょう。			
\\	削減	削減[さくげん]	さくげん	
\\	福祉予算は削減されないことに決まったのよ。	福祉予算[ふくしよさん]は 削減[さくげん]されないことに 決[き]まったのよ。	ふくしよさん は さくげん されない こと に きまった の よ	
\\	福祉予算[ふくしよさん]は
\\	されないことに 決[き]まったのよ。			
\\	削除	削除[さくじょ]	さくじょ	
\\	要らないファイルは削除して下さい。	要[い]らないファイルは 削除[さくじょ]して 下[くだ]さい。	いらない ふぁいる は さくじょ して ください	
\\	要[い]らないファイルは
\\	して 下[くだ]さい。			
\\	刻む	刻[きざ]む	きざむ	
\\	玉ねぎを細かく刻んでください。	玉[たま]ねぎを 細[こま]かく 刻[きざ]んでください。	たまねぎ を こまかく きざんで ください	
\\	玉[たま]ねぎを 細[こま]かく
\\	ください。			
\\	ずうずうしい	ずうずうしい	ずうずうしい	
\\	何てずうずうしい人なんだろう。	何[なん]てずうずうしい 人[ひと]なんだろう。	なんて ずうずうしい ひと なん だろう	
\\	何[なん]て
\\	人[ひと]なんだろう。			
\\	すやすや	すやすや	すやすや	
\\	子供がすやすや眠っているね。	子供[こども]がすやすや 眠[ねむ]っているね。	こども が すやすや ねむって いる ね	
\\	子供[こども]が
\\	眠[ねむ]っているね。			
\\	休養	休養[きゅうよう]	きゅうよう	
\\	彼女は今、休養中です。	彼女[かのじょ]は 今[いま]、 休養[きゅうよう] 中[ちゅう]です。	かのじょ は いま きゅうようちゅう です	
\\	彼女[かのじょ]は 今[いま]、
\\	中[ちゅう]です。			
\\	セルフサービス	セルフサービス	セルフサービス	
\\	この食堂はセルフサービスです。	この 食堂[しょくどう]はセルフサービスです。	この しょくどう は せるふさーびす です	
\\	この 食堂[しょくどう]は
\\	です。			
\\	勤労	勤労[きんろう]	きんろう	
\\	父は長い勤労生活を送ったんです。	父[ちち]は 長[なが]い 勤労[きんろう] 生活[せいかつ]を 送[おく]ったんです。	ちち は ながい きんろう せいかつ を おくった ん です	
\\	父[ちち]は 長[なが]い
\\	生活[せいかつ]を 送[おく]ったんです。			
\\	アルファベット	アルファベット	アルファベット	
\\	この子はアルファベットを全部言えます。	この 子[こ]はアルファベットを 全部言[ぜんぶ い]えます。	この こ は あるふぁべっと を ぜんぶ いえます	
\\	この 子[こ]は
\\	を 全部言[ぜんぶ い]えます。			
\\	いびき	いびき	いびき	
\\	父のいびきは大きいんだ。	父[ちち]のいびきは 大[おお]きいんだ。	ちち の いびき は おおきい ん だ	
\\	父[ちち]の
\\	は 大[おお]きいんだ。			
\\	厚着	厚着[あつぎ]	あつぎ	
\\	普段から厚着しないようにしています。	普段[ふだん]から 厚着[あつぎ]しないようにしています。	ふだん から あつぎ しない よう に して います	
\\	普段[ふだん]から
\\	しないようにしています。			
\\	いやらしい	いやらしい	いやらしい	
\\	彼はいやらしい男ね。	彼[かれ]はいやらしい 男[おとこ]ね。	かれ は いやらしい おとこ ね	
\\	彼[かれ]は
\\	男[おとこ]ね。			
\\	イントネーション	イントネーション	イントネーション	
\\	イントネーションの違いに気を付けて。	イントネーションの 違[ちが]いに 気[き]を 付[つ]けて。	いんとねーしょん の ちがい に き を つけて	
\\	の 違[ちが]いに 気[き]を 付[つ]けて。			
\\	先程	先程[さきほど]	さきほど	
\\	その件でしたら、先程上司から許可を得ております。	その 件[けん]でしたら、 先程[さきほど] 上司[じょうし]から 許可[きょか]を 得[え]ております。	その けん でしたら さきほど じょうし から きょか を えて おります	
\\	その 件[けん]でしたら、
\\	上司[じょうし]から 許可[きょか]を 得[え]ております。			
\\	優れる	優[すぐ]れる	すぐれる	
\\	彼は非常に優れた選手です。	彼[かれ]は 非常[ひじょう]に 優[すぐ]れた 選手[せんしゅ]です。	かれ は ひじょうに すぐれた せんしゅ です	
\\	彼[かれ]は 非常[ひじょう]に
\\	選手[せんしゅ]です。			
\\	がたがた	がたがた	がたがた	
\\	この椅子はがたがたしているね。	この 椅子[いす]はがたがたしているね。	この いす は がたがた して いる ね	
\\	この 椅子[いす]は
\\	しているね。			
\\	出身	出身[しゅっしん]	しゅっしん	
\\	家内は九州出身です。	家内[かない]は 九州[きゅうしゅう] 出身[しゅっしん]です。	かない は きゅうしゅう しゅっしん です	
\\	家内[かない]は 九州[きゅうしゅう]
\\	です。			
\\	全身	全身[ぜんしん]	ぜんしん	
\\	運動した翌日は全身の筋肉が痛い。	運動[うんどう]した 翌日[よくじつ]は 全身[ぜんしん]の 筋肉[きんにく]が 痛[いた]い。	うんどう した よくじつ は ぜんしん の きんにく が いたい	
\\	運動[うんどう]した 翌日[よくじつ]は
\\	の 筋肉[きんにく]が 痛[いた]い。			
\\	カンニング	カンニング	カンニング	
\\	カンニングをした人は0点です。	カンニングをした 人[ひと]は 0点[れいてん]です。	かんにんぐ を した ひと は れいてん です	
\\	をした 人[ひと]は 0点[れいてん]です。			
\\	主婦	主婦[しゅふ]	しゅふ	
\\	主婦の主な仕事は家事よ。	主婦[しゅふ]の 主[おも]な 仕事[しごと]は 家事[かじ]よ。	しゅふ の おも な しごと は かじ よ	
\\	の 主[おも]な 仕事[しごと]は 家事[かじ]よ。			
\\	あぐら	あぐら	あぐら	
\\	彼はあぐらをかいてしゃべっていますね。	彼[かれ]はあぐらをかいてしゃべっていますね。	かれ は あぐら を かいて しゃべって います ね	
\\	彼[かれ]は
\\	をかいてしゃべっていますね。			
\\	お互い	お 互[たが]い	おたがい	
\\	お互いを信頼することが大事です。	お 互[たが]いを 信頼[しんらい]することが 大事[だいじ]です。	おたがい を しんらい する こと が だいじ です	
\\	を 信頼[しんらい]することが 大事[だいじ]です。			
\\	交互	交互[こうご]	こうご	
\\	男女交互に並んでください。	男女[だんじょ] 交互[こうご]に 並[なら]んでください。	だんじょ こうご に ならんで ください	
\\	男女[だんじょ]
\\	に 並[なら]んでください。			
\\	いやいや	いやいや	いやいや	
\\	その子はいやいや学校へ行ったの。	その 子[こ]はいやいや 学校[がっこう]へ 行[い]ったの。	その こ は いやいや がっこう へ いった の	
\\	その 子[こ]は
\\	学校[がっこう]へ 行[い]ったの。			
\\	ウエスト	ウエスト	ウエスト	
\\	最近ウエストが太くなったな。	最近[さいきん]ウエストが 太[ふと]くなったな。	さいきん うえすと が ふとく なった な	
\\	最近[さいきん]
\\	が 太[ふと]くなったな。			
\\	オルガン	オルガン	オルガン	
\\	僕たちはオルガンに合わせて賛美歌を歌ったんだ。	僕[ぼく]たちはオルガンに 合[あ]わせて 賛美歌[さんびか]を 歌[うた]ったんだ。	ぼくたち は おるがん に あわせて さんびか を うたった んだ	
\\	僕[ぼく]たちは
\\	に 合[あ]わせて 賛美歌[さんびか]を 歌[うた]ったんだ。			
\\	区域	区域[くいき]	くいき	
\\	ここは危険区域よ。	ここは 危険[きけん] 区域[くいき]よ。	ここ は きけん くいき よ	
\\	ここは 危険[きけん]
\\	よ。			
\\	からから	からから	からから	
\\	彼はからからと笑いました。	彼[かれ]はからからと 笑[わら]いました。	かれ は からから と わらいました	
\\	彼[かれ]は
\\	と 笑[わら]いました。			
\\	人権	人権[じんけん]	じんけん	
\\	全ての人に人権がある。	全[すべ]ての 人[ひと]に 人権[じんけん]がある。	すべて の ひと に じんけん が ある	
\\	全[すべ]ての 人[ひと]に
\\	がある。			
\\	きっちん	きっちん	きっちん	
\\	彼女はキッチンで夕食を作っています。	彼女[かのじょ]はキッチンで 夕食[ゆうしょく]を 作[つく]っています。	かのじょ は きっちん で ゆうしょく を つくって います	
\\	彼女[かのじょ]は
\\	で 夕食[ゆうしょく]を 作[つく]っています。			
\\	侵す	侵[おか]す	おかす	
\\	他人の権利を侵してはなりません。	他人[たにん]の 権利[けんり]を 侵[おか]してはなりません。	たにん の けんり を おかしては なりません	
\\	他人[たにん]の 権利[けんり]を
\\	はなりません。			
\\	くしゃくしゃ	くしゃくしゃ	くしゃくしゃ	
\\	紙をくしゃくしゃに丸めました。	紙[かみ]をくしゃくしゃに 丸[まる]めました。	かみ を くしゃくしゃ に まるめました	
\\	紙[かみ]を
\\	に 丸[まる]めました。			
\\	修了	修了[しゅうりょう]	しゅうりょう	
\\	先月、そのコースを修了しました。	先月[せんげつ]、そのコースを 修了[しゅうりょう]しました。	せんげつ その こーす を しゅうりょう しました	
\\	先月[せんげつ]、そのコースを
\\	しました。			
\\	けち	けち	けち	
\\	彼は金持ちだけど、けちね。	彼[かれ]は 金持[かねも]ちだけど、けちね。	かれ は かねもち だ けど けち ね	
\\	彼[かれ]は 金持[かねも]ちだけど、
\\	ね。			
\\	乾燥	乾燥[かんそう]	かんそう	
\\	冬は空気が乾燥しますね。	冬[ふゆ]は 空気[くうき]が 乾燥[かんそう]しますね。	ふゆ は くうき が かんそう します ね	
\\	冬[ふゆ]は 空気[くうき]が
\\	しますね。			
\\	ざらざら	ざらざら	ざらざら	
\\	砂ぼこりで机がざらざらしているね。	砂[すな]ぼこりで 机[つくえ]がざらざらしているね。	すなぼこり で つくえ が ざらざら して いる ね	
\\	砂[すな]ぼこりで 机[つくえ]が
\\	しているね。			
\\	スモッグ	スモッグ	スモッグ	
\\	都会の空はスモッグで灰色ね。	都会[とかい]の 空[そら]はスモッグで 灰色[はいいろ]ね。	とかい の そら は すもっぐ で はいいろ ね	
\\	都会[とかい]の 空[そら]は
\\	で 灰色[はいいろ]ね。			
\\	すり	すり	すり	
\\	すりに財布をとられた。	すりに 財布[さいふ]をとられた。	すり に さいふ を とられた	
\\	に 財布[さいふ]をとられた。			
\\	おにぎり	おにぎり	おにぎり	
\\	昼食におにぎりを食べました。	昼食[ちゅうしょく]におにぎりを 食[た]べました。	ちゅうしょく に おにぎり を たべました	
\\	昼食[ちゅうしょく]に
\\	を 食[た]べました。			
\\	及ぶ	及[およ]ぶ	およぶ	
\\	あなたにまで迷惑が及んでごめんなさい。	あなたにまで 迷惑[めいわく]が 及[およ]んでごめんなさい。	あなた に まで めいわく が およんで ごめんなさい	
\\	あなたにまで 迷惑[めいわく]が
\\	ごめんなさい。			
\\	上級	上級[じょうきゅう]	じょうきゅう	
\\	彼は上級のコースに上がったよ。	彼[かれ]は 上級[じょうきゅう]のコースに 上[あ]がったよ。	かれ は じょうきゅう の こーす に あがった よ	
\\	彼[かれ]は
\\	のコースに 上[あ]がったよ。			
\\	かみそり	かみそり	かみそり	
\\	かみそりの刃で指を切りました。	かみそりの 刃[は]で 指[ゆび]を 切[き]りました。	かみそり の は で ゆび を きりました	
\\	の 刃[は]で 指[ゆび]を 切[き]りました。			
\\	初級	初級[しょきゅう]	しょきゅう	
\\	これは初級の教科書です。	これは 初級[しょきゅう]の 教科書[きょうかしょ]です。	これ は しょきゅう の きょうかしょ です 。	
\\	これは
\\	の 教科書[きょうかしょ]です。			
\\	出血	出血[しゅっけつ]	しゅっけつ	
\\	出血がひどいので、医者に行ったほうがいい。	出血[しゅっけつ]がひどいので、 医者[いしゃ]に 行[い]ったほうがいい。	しゅっけつ が ひどい の で いしゃ に いった ほう が いい	
\\	がひどいので、 医者[いしゃ]に 行[い]ったほうがいい。			
\\	がやがや	がやがや	がやがや	
\\	生徒たちががやがやと騒いでいますね。	生徒[せいと]たちががやがやと 騒[さわ]いでいますね。	せいとたち が がやがや と さわいで います ね	
\\	生徒[せいと]たちが
\\	と 騒[さわ]いでいますね。			
\\	健全	健全[けんぜん]	けんぜん	
\\	子供の健全な心を育てましょう。	子供[こども]の 健全[けんぜん]な 心[こころ]を 育[そだ]てましょう。	こども の けんぜん な こころ を そだてましょう	
\\	子供[こども]の
\\	な 心[こころ]を 育[そだ]てましょう。			
\\	健康	健康[けんこう]	けんこう	
\\	健康が一番大切だ。	健康[けんこう]が 一番大切[いちばん たいせつ]だ。	けんこう が いちばん たいせつ だ 。	
\\	が 一番大切[いちばん たいせつ]だ。			
\\	医療	医療[いりょう]	いりょう	
\\	医療関係の仕事をしています。	医療[いりょう] 関係[かんけい]の 仕事[しごと]をしています。	いりょう かんけい の しごと を して います	
\\	関係[かんけい]の 仕事[しごと]をしています。			
\\	きょろきょろ	きょろきょろ	きょろきょろ	
\\	会場で子供がきょろきょろしているわよ。	会場[かいじょう]で 子供[こども]がきょろきょろしているわよ。	かいじょう で こども が きょろきょろ して いる わ よ	
\\	会場[かいじょう]で 子供[こども]が
\\	しているわよ。			
\\	くしゃみ	くしゃみ	くしゃみ	
\\	風邪でくしゃみが止まりません。	風邪[かぜ]でくしゃみが 止[と]まりません。	かぜ で くしゃみ が とまりません	
\\	風邪[かぜ]で
\\	が 止[と]まりません。			
\\	再開	再開[さいかい]	さいかい	
\\	試合はすぐに再開されたの。	試合[しあい]はすぐに 再開[さいかい]されたの。	しあい は すぐ に さいかい された の	
\\	試合[しあい]はすぐに
\\	されたの。			
\\	再生	再生[さいせい]	さいせい	
\\	留守番電話のメッセージを再生したの。	留守番電話[るすばんでんわ]のメッセージを 再生[さいせい]したの。	るすばんでんわ の めっせーじ を さいせい した の	
\\	留守番電話[るすばんでんわ]のメッセージを
\\	したの。			
\\	再会	再会[さいかい]	さいかい	
\\	彼らは互いに再会を喜んだの。	彼[かれ]らは 互[たが]いに 再会[さいかい]を 喜[よろこ]んだの。	かれら は たがいに さいかい を よろこんだ の	
\\	彼[かれ]らは 互[たが]いに
\\	を 喜[よろこ]んだの。			
\\	ぐずぐず	ぐずぐず	ぐずぐず	
\\	ぐずぐずしていると電車に乗り遅れますよ。	ぐずぐずしていると 電車[でんしゃ]に 乗[の]り 遅[おく]れますよ。	ぐずぐず して いる と でんしゃ に のりおくれます よ	
\\	していると 電車[でんしゃ]に 乗[の]り 遅[おく]れますよ。			
\\	くすぐったい	くすぐったい	くすぐったい	
\\	犬に顔をなめられてくすぐったいよ。	犬[いぬ]に 顔[かお]をなめられてくすぐったいよ。	いぬ に かお を なめられて くすぐったい よ 。	
\\	犬[いぬ]に 顔[かお]をなめられて
\\	よ。			
\\	仮に	仮[かり]に	かりに	
\\	仮にそれが事実だとしてももう遅いよ。	仮[かり]にそれが 事実[じじつ]だとしてももう 遅[おそ]いよ。	かりに それ が じじつ だ と して も もう おそい よ	
\\	それが 事実[じじつ]だとしてももう 遅[おそ]いよ。			
\\	仮定	仮定[かてい]	かてい	
\\	仮定の話だけでは結論は出ません。	仮定[かてい]の 話[はなし]だけでは 結論[けつろん]は 出[で]ません。	かてい の はなし だけ で は けつろん は でません	
\\	の 話[はなし]だけでは 結論[けつろん]は 出[で]ません。			
\\	仮	仮[かり]	かり	
\\	仮の申し込みをしました。	仮[かり]の 申[もう]し 込[こ]みをしました。	かり の もうしこみ を しました	
\\	の 申[もう]し 込[こ]みをしました。			
\\	仮名遣い	仮名遣[かなづか]い	かなづかい	
\\	祖父は古い仮名遣いで書くことがあるの。	祖父[そふ]は 古[ふる]い 仮名遣[かなづか]いで 書[か]くことがあるの。	そふ は ふるい かなづかい で かく こと が ある の	
\\	祖父[そふ]は 古[ふる]い
\\	で 書[か]くことがあるの。			
\\	公演	公演[こうえん]	こうえん	
\\	彼らの海外公演が発表されたよ。	彼[かれ]らの 海外[かいがい] 公演[こうえん]が 発表[はっぴょう]されたよ。	かれら の かいがいこうえん が はっぴょう された よ	
\\	彼[かれ]らの 海外[かいがい]
\\	が 発表[はっぴょう]されたよ。			
\\	出演	出演[しゅつえん]	しゅつえん	
\\	彼女はその映画に出演しているよ。	彼女[かのじょ]はその 映画[えいが]に 出演[しゅつえん]しているよ。	かのじょ は その えいが に しゅつえん して いる よ	
\\	彼女[かのじょ]はその 映画[えいが]に
\\	しているよ。			
\\	くたびれる	くたびれる	くたびれる	
\\	たくさん歩いてくたびれました。	たくさん 歩[ある]いてくたびれました。	たくさん あるいて くたびれました	
\\	たくさん 歩[ある]いて
\\	劇	劇[げき]	げき	
\\	小学校で子供たちの劇を見ました。	小学校[しょうがっこう]で 子供[こども]たちの 劇[げき]を 見[み]ました。	しょうがっこう で こどもたち の げき を みました	
\\	小学校[しょうがっこう]で 子供[こども]たちの
\\	を 見[み]ました。			
\\	劇場	劇場[げきじょう]	げきじょう	
\\	午後8時に劇場の前で会いましょう。	午後8時[ごご はちじ]に 劇場[げきじょう]の 前[まえ]で 会[あ]いましょう。	ごご はちじ に げきじょう の まえ で あいましょう	
\\	午後8時[ごご はちじ]に
\\	の 前[まえ]で 会[あ]いましょう。			
\\	いらいら	いらいら	いらいら	
\\	彼女は朝からいらいらしている。	彼女[かのじょ]は 朝[あさ]からいらいらしている。	かのじょ は あさ から いらいら して いる	
\\	彼女[かのじょ]は 朝[あさ]から
\\	している。			
\\	主催	主催[しゅさい]	しゅさい	
\\	その芸術祭は市が主催しています。	その 芸術祭[げいじゅつさい]は 市[し]が 主催[しゅさい]しています。	その げいじゅつさい は し が しゅさい して います	
\\	その 芸術祭[げいじゅつさい]は 市[し]が
\\	しています。			
\\	促す	促[うなが]す	うながす	
\\	彼に集中するよう注意を促しといたよ。	彼[かれ]に 集中[しゅうちゅう]するよう 注意[ちゅうい]を 促[うなが]しといたよ。	かれ に しゅうちゅう する よう ちゅうい を うながし とい た よ	
\\	彼[かれ]に 集中[しゅうちゅう]するよう 注意[ちゅうい]を
\\	よ。			
\\	催促	催促[さいそく]	さいそく	
\\	彼女に本を返すように催促したの。	彼女[かのじょ]に 本[ほん]を 返[かえ]すように 催促[さいそく]したの。	かのじょ に ほん を かえす よう に さいそく した の	
\\	彼女[かのじょ]に 本[ほん]を 返[かえ]すように
\\	したの。			
\\	エチケット	エチケット	エチケット	
\\	エチケットを守ることは大切です。	エチケットを 守[まも]ることは 大切[たいせつ]です。	えちけっと を まもる こと は たいせつ です	
\\	を 守[まも]ることは 大切[たいせつ]です。			
\\	ご免	ご 免[めん]	ごめん	
\\	遅れてご免。	遅[おく]れてご 免[めん]。	おくれて ごめん	
\\	遅[おく]れて
\\	かっと	かっと	かっと	
\\	かっとなって余計なことを言ってしまったな。	かっとなって 余計[よけい]なことを 言[い]ってしまったな。	かっと なって よけい な こと を いって しまった な	
\\	なって 余計[よけい]なことを 言[い]ってしまったな。			
\\	助詞	助詞[じょし]	じょし	
\\	助詞を変えると文の意味が変わります。	助詞[じょし]を 変[か]えると 文[ぶん]の 意味[いみ]が 変[か]わります。	じょし を かえる と ぶん の いみ が かわります	
\\	を 変[か]えると 文[ぶん]の 意味[いみ]が 変[か]わります。			
\\	上司	上司[じょうし]	じょうし	
\\	上司に相談してみます。	上司[じょうし]に 相談[そうだん]してみます。	じょうし に そうだん して みます	
\\	に 相談[そうだん]してみます。			
\\	がらがら	がらがら	がらがら	
\\	誰かがガラガラとうがいをしているね。	誰[だれ]かがガラガラとうがいをしているね。	だれか が がらがら と うがい を して いる ね	
\\	誰[だれ]かが
\\	とうがいをしているね。			
\\	休講	休講[きゅうこう]	きゅうこう	
\\	今日のフランス語の講座は休講です。	今日[きょう]のフランス 語[ご]の 講座[こうざ]は 休講[きゅうこう]です。	きょう の ふらんすご の こうざ は きゅうこう です	
\\	今日[きょう]のフランス 語[ご]の 講座[こうざ]は
\\	です。			
\\	医師	医師[いし]	いし	
\\	医師に入院を勧められたの。	医師[いし]に 入院[にゅういん]を 勧[すす]められたの。	いし に にゅういん を すすめられた の	
\\	に 入院[にゅういん]を 勧[すす]められたの。			
\\	かんかん	かんかん	かんかん	
\\	彼女はかんかんに怒っています。	彼女[かのじょ]はかんかんに 怒[おこ]っています。	かのじょ は かんかん に おこって います	
\\	彼女[かのじょ]は
\\	に 怒[おこ]っています。			
\\	お手伝い	お 手伝[てつだ]い	おてつだい	
\\	何かお手伝いしましょうか。	何[なに]かお 手伝[てつだ]いしましょうか。	なに か おてつだい しましょう か	
\\	何[なに]か
\\	しましょうか。			
\\	これまで	これまで	これまで	
\\	これまでの私とは違うんです。	これまでの 私[わたし]とは 違[ちが]うんです。	これまで の わたし と は ちがうん です	
\\	の 私[わたし]とは 違[ちが]うんです。			
\\	刺さる	刺[さ]さる	ささる	
\\	靴の底に釘が刺さってしまったんだ。	靴[くつ]の 底[そこ]に 釘[くぎ]が 刺[さ]さってしまったんだ。	くつ の そこ に くぎ が ささって しまった ん だ	
\\	靴[くつ]の 底[そこ]に 釘[くぎ]が
\\	んだ。			
\\	刺激	刺激[しげき]	しげき	
\\	そのクイズ番組は脳を刺激するね。	そのクイズ 番組[ばんぐみ]は 脳[のう]を 刺激[しげき]するね。	その くいずばんぐみ は のう を しげき する ね	
\\	そのクイズ 番組[ばんぐみ]は 脳[のう]を
\\	するね。			
\\	じめじめ	じめじめ	じめじめ	
\\	梅雨時はじめじめする。	梅雨時[つゆどき]はじめじめする。	つゆどき は じめじめ する	
\\	梅雨時[つゆどき]は
\\	する。			
\\	一致	一致[いっち]	いっち	
\\	皆の意見が一致しました。	皆[みんな]の 意見[いけん]が 一致[いっち]しました。	みんな の いけん が いっち しました	
\\	皆[みんな]の 意見[いけん]が
\\	しました。			
\\	傾く	傾[かたむ]く	かたむく	
\\	お日様が西に傾きましたね。	お 日様[ひさま]が 西[にし]に 傾[かたむ]きましたね。	おひさま が にし に かたむきました ね	
\\	お 日様[ひさま]が 西[にし]に
\\	ね。			
\\	傾ける	傾[かたむ]ける	かたむける	
\\	彼女は首を少し傾けて笑うの。	彼女[かのじょ]は 首[くび]を 少[すこ]し 傾[かたむ]けて 笑[わら]うの。	かのじょ は くび を すこし かたむけて わらう の	
\\	彼女[かのじょ]は 首[くび]を 少[すこ]し
\\	笑[わら]うの。			
\\	じろじろ	じろじろ	じろじろ	
\\	人をじろじろ見るものではありません。	人[ひと]をじろじろ 見[み]るものではありません。	ひと を じろじろ みる もの で は ありません	
\\	人[ひと]を
\\	見[み]るものではありません。			
\\	あり	あり	あり	
\\	ありの群れが砂糖にたかっている。	ありの 群[む]れが 砂糖[さとう]にたかっている。	あり の むれ が さとう に たかっている 。	
\\	の 群[む]れが 砂糖[さとう]にたかっている。			
\\	出張	出張[しゅっちょう]	しゅっちょう	
\\	部長は京都に出張中です。	部長[ぶちょう]は 京都[きょうと]に 出張[しゅっちょう] 中[ちゅう]です。	ぶちょう は きょうと に しゅっちょうちゅう です	
\\	部長[ぶちょう]は 京都[きょうと]に
\\	中[ちゅう]です。			
\\	あいづち	あいづち	あいづち	
\\	彼の話に私はあいづちを打ったの。	彼[かれ]の 話[はなし]に 私[わたし]はあいづちを 打[う]ったの。	かれ の はなし に わたし は あいづち を うった の	
\\	彼[かれ]の 話[はなし]に 私[わたし]は
\\	を 打[う]ったの。			
\\	かちかち	かちかち	かちかち	
\\	池がかちかちに凍っています。	池[いけ]がかちかちに 凍[こお]っています。	いけ が かちかち に こおって います	
\\	池[いけ]が
\\	に 凍[こお]っています。			
\\	くたくた	くたくた	くたくた	
\\	今日はたくさん歩いてくたくたです。	今日[きょう]はたくさん 歩[ある]いてくたくたです。	きょう は たくさん あるいて くたくた です	
\\	今日[きょう]はたくさん 歩[ある]いて
\\	です。			
\\	こしょう	こしょう	こしょう	
\\	こしょうを入れ過ぎてスープが辛い。	こしょうを 入[い]れ 過[す]ぎてスープが 辛[から]い。	こしょう を いれすぎて すーぷ が からい	
\\	を 入[い]れ 過[す]ぎてスープが 辛[から]い。			
\\	勢力	勢力[せいりょく]	せいりょく	
\\	この頃は、ローマが勢力を伸ばしていました。	この 頃[ころ]は、ローマが 勢力[せいりょく]を 伸[の]ばしていました。	この ころ は ろーま が せいりょく を のばして いました	
\\	この 頃[ころ]は、ローマが
\\	を 伸[の]ばしていました。			
\\	勢い	勢[いきお]い	いきおい	
\\	そのチームには勢いがありますね。	そのチームには 勢[いきお]いがありますね。	その ちーむ に は いきおい が あります ね	
\\	そのチームには
\\	がありますね。			
\\	すべすべ	すべすべ	すべすべ	
\\	彼女の肌はすべすべしているね。	彼女[かのじょ]の 肌[はだ]はすべすべしているね。	かのじょ の はだ は すべすべ して いる ね	
\\	彼女[かのじょ]の 肌[はだ]は
\\	しているね。			
\\	せっかく	せっかく	せっかく	
\\	せっかく来たんだからゆっくりして行きなさい。	せっかく 来[き]たんだからゆっくりして 行[い]きなさい。	せっかく きたん だから ゆっくり して いきなさい	
\\	来[き]たんだからゆっくりして 行[い]きなさい。			
\\	偉大	偉大[いだい]	いだい	
\\	彼は偉大な事業をなしとげました。	彼[かれ]は 偉大[いだい]な 事業[じぎょう]をなしとげました。	かれ は いだい な じぎょう を なしとげ ました	
\\	彼[かれ]は
\\	な 事業[じぎょう]をなしとげました。			
\\	刀	刀[かたな]	かたな	
\\	日本映画で刀を見た。	日本映画[にほん えいが]で 刀[かたな]を 見[み]た。	にほん えいが で かたな を みた	
\\	日本映画[にほん えいが]で
\\	を 見[み]た。			
\\	事柄	事柄[ことがら]	ことがら	
\\	これまで話し合った事柄をまとめてください。	これまで 話[はな]し 合[あ]った 事柄[ことがら]をまとめてください。	これ まで はなしあった ことがら を まとめて ください	
\\	これまで 話[はな]し 合[あ]った
\\	をまとめてください。			
\\	あきれる	あきれる	あきれる	
\\	彼の頑固さにはあきれました。	彼[かれ]の 頑固[がんこ]さにはあきれました。	かれ の がんこさ に は あきれました	
\\	彼[かれ]の 頑固[がんこ]さには
\\	兼ねる	兼[か]ねる	かねる	
\\	彼女は秘書と事務の担当を兼ねています。	彼女[かのじょ]は 秘書[ひしょ]と 事務[じむ]の 担当[たんとう]を 兼[か]ねています。	かのじょ は ひしょ と じむ の たんとう を かねて います	
\\	彼女[かのじょ]は 秘書[ひしょ]と 事務[じむ]の 担当[たんとう]を
\\	あだな	あだな	あだな	
\\	彼のあだなは「りき」です。	彼[かれ]のあだなは「りき」です。	かれ の あだな は 
\\	りき 
\\	です 。	
\\	彼[かれ]の
\\	は「りき」です。			
\\	印鑑	印鑑[いんかん]	いんかん	
\\	ここに受け取りの印鑑をお願いします。	ここに 受[う]け 取[と]りの 印鑑[いんかん]をお 願[ねが]いします。	ここ に うけとり の いんかん を おねがい します	
\\	ここに 受[う]け 取[と]りの
\\	をお 願[ねが]いします。			
\\	あふれる	あふれる	あふれる	
\\	バスタブからお湯があふれました。	バスタブからお 湯[ゆ]があふれました。	ばすたぶ から おゆ が あふれました	
\\	バスタブからお 湯[ゆ]が
\\	いか	いか	いか	
\\	私はイカの刺し身が大好きです。	私[わたし]はイカの 刺[さ]し 身[み]が 大好[だいす]きです。	わたし は いか の さしみ が だいすき です	
\\	私[わたし]は
\\	の 刺[さ]し 身[み]が 大好[だいす]きです。			
\\	いつのまに	いつのまに	いつのまに	
\\	彼はいつのまにいなくなったの。	彼[かれ]はいつのまにいなくなったの。	かれ は いつのまに いなく なった の	
\\	彼[かれ]は
\\	いなくなったの。			
\\	インフレーション	インフレーション	インフレーション	
\\	市場にはインフレーションの影響が出ているね。	市場[しじょう]にはインフレーションの 影響[えいきょう]が 出[で]ているね。	しじょう に は いんふれーしょん の えいきょう が でて いる ね	
\\	市場[しじょう]には
\\	の 影響[えいきょう]が 出[で]ているね。			
\\	一層	一層[いっそう]	いっそう	
\\	雨が一層激しくなったね。	雨[あめ]が 一層[いっそう] 激[はげ]しくなったね。	あめ が いっそう はげしく なった ね	
\\	雨[あめ]が
\\	激[はげ]しくなったね。			
\\	ウィークエンド	ウィークエンド	ウィークエンド	
\\	今度のウィークエンドは映画を見ます。	今度[こんど]のウィークエンドは 映画[えいが]を 見[み]ます。	こんど の うぃーくえんど は えいが を みます	
\\	今度[こんど]の
\\	は 映画[えいが]を 見[み]ます。			
\\	ガラス戸	ガラス 戸[ど]	がらすど	
\\	お店のガラス戸が割られたんだ。	お 店[みせ]のガラス 戸[ど]が 割[わ]られたんだ。	おみせ の がらすど が わられた ん だ	
\\	お 店[みせ]の
\\	が 割[わ]られたんだ。			
\\	うがい	うがい	うがい	
\\	冬はまめにうがいをします。	冬[ふゆ]はまめにうがいをします。	ふゆ は まめ に うがい を します	
\\	冬[ふゆ]はまめに
\\	をします。			
\\	うきうき	うきうき	うきうき	
\\	もうすぐ夏休みなのでうきうきしています。	もうすぐ 夏休[なつやす]みなのでうきうきしています。	もうすぐ なつやすみ な の で うきうき して います	
\\	もうすぐ 夏休[なつやす]みなので
\\	しています。			
\\	お歳暮	お 歳暮[せいぼ]	おせいぼ	
\\	デパートでお歳暮を送りました。	デパートでお 歳暮[せいぼ]を 送[おく]りました。	でぱーと で おせいぼ を おくりました	
\\	デパートで
\\	を 送[おく]りました。			
\\	うつむく	うつむく	うつむく	
\\	彼はうつむいて何かを考えているね。	彼[かれ]はうつむいて 何[なに]かを 考[かんが]えているね。	かれ は うつむいて なにか を かんがえて いる ね	
\\	彼[かれ]は
\\	何[なに]かを 考[かんが]えているね。			
\\	儀式	儀式[ぎしき]	ぎしき	
\\	儀式には作法があります。	儀式[ぎしき]には 作法[さほう]があります。	ぎしき に は さほう が あります	
\\	には 作法[さほう]があります。			
\\	うなずく	うなずく	うなずく	
\\	彼女はうなずいて同意を示したの。	彼女[かのじょ]はうなずいて 同意[どうい]を 示[しめ]したの。	かのじょ は うなずいて どうい を しめした の	
\\	彼女[かのじょ]は
\\	同意[どうい]を 示[しめ]したの。			
\\	お坊さん	お 坊[ぼう]さん	おぼうさん	
\\	お寺からお坊さんが出て来ましたよ。	お 寺[てら]からお 坊[ぼう]さんが 出[で]て 来[き]ましたよ。	おてら から おぼうさん が でて きました よ	
\\	お 寺[てら]から
\\	が 出[で]て 来[き]ましたよ。			
\\	おだてる	おだてる	おだてる	
\\	彼は人をおだてるのが上手です。	彼[かれ]は 人[ひと]をおだてるのが 上手[じょうず]です。	かれ は ひと を おだてる の が じょうず です	
\\	彼[かれ]は 人[ひと]を
\\	のが 上手[じょうず]です。			
\\	かかと	かかと	かかと	
\\	靴のかかとで彼の足を踏んでしまったの。	靴[くつ]のかかとで 彼[かれ]の 足[あし]を 踏[ふ]んでしまったの。	くつ の かかと で かれ の あし を ふん でしまった の 。	
\\	靴[くつ]の
\\	で 彼[かれ]の 足[あし]を 踏[ふ]んでしまったの。			
\\	下旬	下旬[げじゅん]	げじゅん	
\\	この仕事は来月の下旬には終わります。	この 仕事[しごと]は 来月[らいげつ]の 下旬[げじゅん]には 終[お]わります。	この しごと は らいげつ の げじゅん に は おわります	
\\	この 仕事[しごと]は 来月[らいげつ]の
\\	には 終[お]わります。			
\\	上旬	上旬[じょうじゅん]	じょうじゅん	
\\	7月上旬に夏祭りがありますよ。	7月[しちがつ] 上旬[じょうじゅん]に 夏祭[なつまつ]りがありますよ。	しちがつ じょうじゅん に なつまつり が あります よ	
\\	7月[しちがつ]
\\	に 夏祭[なつまつ]りがありますよ。			
\\	がさがさ	がさがさ	がさがさ	
\\	玄関でがさがさと音がしましたよ。	玄関[げんかん]でがさがさと 音[おと]がしましたよ。	げんかん で がさがさ と おと が しました よ	
\\	玄関[げんかん]で
\\	と 音[おと]がしましたよ。			
\\	一瞬	一瞬[いっしゅん]	いっしゅん	
\\	一瞬自分の耳を疑ったよ。	一瞬[いっしゅん] 自分[じぶん]の 耳[みみ]を 疑[うたが]ったよ。	いっしゅん じぶん の みみ を うたがった よ	
\\	自分[じぶん]の 耳[みみ]を 疑[うたが]ったよ。			
\\	かじる	かじる	かじる	
\\	弟は美味しそうにりんごをかじっていたよ。	弟[おとうと]は 美味[おい]しそうにりんごをかじっていたよ。	おとうと は おいし そう に りんご を かじって いた よ	
\\	弟[おとうと]は 美味[おい]しそうにりんごを
\\	よ。			
\\	がぶがぶ	がぶがぶ	がぶがぶ	
\\	彼は水をがぶがぶ飲んだんだ。	彼[かれ]は 水[みず]をがぶがぶ 飲[の]んだんだ。	かれ は みず を がぶがぶ のんだ ん だ	
\\	彼[かれ]は 水[みず]を
\\	飲[の]んだんだ。			
\\	お目に掛かる	お 目[め]に 掛[か]かる	おめにかかる	
\\	お目に掛かれて嬉しいです。	お 目[め]に 掛[か]かれて 嬉[うれ]しいです。	おめにかかれて うれしい です	
\\	嬉[うれ]しいです。			
\\	信仰	信仰[しんこう]	しんこう	
\\	信仰は心の支えです。	信仰[しんこう]は 心[こころ]の 支[ささ]えです。	しんこう は こころ の ささえ です	
\\	は 心[こころ]の 支[ささ]えです。			
\\	からす	からす	からす	
\\	木の上でからすが鳴いています。	木[き]の 上[うえ]でからすが 鳴[な]いています。	き の うえ で からす が ないて います	
\\	木[き]の 上[うえ]で
\\	が 鳴[な]いています。			
\\	偶然	偶然[ぐうぜん]	ぐうぜん	
\\	街で偶然友人に会いました。	街[まち]で 偶然[ぐうぜん] 友人[ゆうじん]に 会[あ]いました。	まち で ぐうぜん ゆうじん に あいました	
\\	街[まち]で
\\	友人[ゆうじん]に 会[あ]いました。			
\\	偶数	偶数[ぐうすう]	ぐうすう	
\\	偶数は2で割り切れます。	偶数[ぐうすう]は 2[に]で 割[わ]り 切[き]れます。	ぐうすう は に で わりきれます	
\\	は 2[に]で 割[わ]り 切[き]れます。			
\\	伺う	伺[うかが]う	うかがう	
\\	お話しを伺いたいのですが。	お 話[はな]しを 伺[うかが]いたいのですが。	おはなし を うかがいたい の です が	
\\	お 話[はな]しを
\\	のですが。			
\\	がん	がん	がん	
\\	祖父はがんで亡くなりました。	祖父[そふ]はがんで 亡[な]くなりました。	そふ は がん で なくなりました	
\\	祖父[そふ]は
\\	で 亡[な]くなりました。			
\\	俺	俺[おれ]	おれ	
\\	俺の頼みを聞いてくれ。	俺[おれ]の 頼[たの]みを 聞[き]いてくれ。	おれ の たのみ を きいて くれ	
\\	の 頼[たの]みを 聞[き]いてくれ。			
\\	偏る	偏[かたよ]る	かたよる	
\\	栄養が偏らないように食事に気を付けています。	栄養[えいよう]が 偏[かたよ]らないように 食事[しょくじ]に 気[き]を 付[つ]けています。	えいよう が かたよらない よう に しょくじ に き を つけて います	
\\	栄養[えいよう]が
\\	ように 食事[しょくじ]に 気[き]を 付[つ]けています。			
\\	一遍に	一遍[いっぺん]に	いっぺんに	
\\	春は一遍に花が咲く。	春[はる]は 一遍[いっぺん]に 花[はな]が 咲[さ]く。	はる は いっぺんに はな が さく	
\\	春[はる]は
\\	花[はな]が 咲[さ]く。			
\\	ぎっしり	ぎっしり	ぎっしり	
\\	この箱にはぎっしり物が詰まっています。	この 箱[はこ]にはぎっしり 物[もの]が 詰[つ]まっています。	この はこ に は ぎっしり もの が つまって います	
\\	この 箱[はこ]には
\\	物[もの]が 詰[つ]まっています。			
\\	ぎゅうぎゅう	ぎゅうぎゅう	ぎゅうぎゅう	
\\	電車がぎゅうぎゅうに込んでいますね。	電車[でんしゃ]がぎゅうぎゅうに 込[こ]んでいますね。	でんしゃ が ぎゅうぎゅう に こんで います ね	
\\	電車[でんしゃ]が
\\	に 込[こ]んでいますね。			
\\	勧める	勧[すす]める	すすめる	
\\	勧められて欲しくもないものを買っちゃったよ。	勧[すす]められて 欲[ほ]しくもないものを 買[か]っちゃったよ。	すすめられて ほしく も ない もの を かっちゃった よ	
\\	欲[ほ]しくもないものを 買[か]っちゃったよ。			
\\	ぎょうざ	ぎょうざ	ぎょうざ	
\\	ここのぎょうざは美味しいよ。	ここのぎょうざは 美味[おい]しいよ。	ここ の ぎょうざ は おいしい よ	
\\	ここの
\\	は 美味[おい]しいよ。			
\\	収穫	収穫[しゅうかく]	しゅうかく	
\\	お米の収穫が始まったね。	お 米[こめ]の 収穫[しゅうかく]が 始[はじ]まったね。	おこめ の しゅうかく が はじまった ね	
\\	お 米[こめ]の
\\	が 始[はじ]まったね。			
\\	きれ	きれ	きれ	
\\	余ったきれで人形を作りましょう。	余[あま]ったきれで 人形[にんぎょう]を 作[つく]りましょう。	あまった きれ で にんぎょう を つくりましょう	
\\	余[あま]った
\\	で 人形[にんぎょう]を 作[つく]りましょう。			
\\	原稿用紙	原稿用紙[げんこうようし]	げんこうようし	
\\	作文を書くのに原稿用紙を使ったの。	作文[さくぶん]を 書[か]くのに 原稿用紙[げんこうようし]を 使[つか]ったの。	さくぶん を かく の に げんこうようし を つかった の	
\\	作文[さくぶん]を 書[か]くのに
\\	を 使[つか]ったの。			
\\	お陰	お 陰[かげ]	おかげ	
\\	先生のお陰で大学に合格できました。	先生[せんせい]のお 陰[かげ]で 大学[だいがく]に 合格[ごうかく]できました。	せんせい の おかげ で だいがく に ごうかく できました	
\\	先生[せんせい]の
\\	で 大学[だいがく]に 合格[ごうかく]できました。			
\\	くすぐる	くすぐる	くすぐる	
\\	弟に足の裏をくすぐられた。	弟[おとうと]に 足[あし]の 裏[うら]をくすぐられた。	おとうと に あし の うら を くすぐられた	
\\	弟[おとうと]に 足[あし]の 裏[うら]を
\\	卑しい	卑[いや]しい	いやしい	
\\	彼は金に卑しいね。	彼[かれ]は 金[かね]に 卑[いや]しいね。	かれ は かね に いやしい ね	
\\	彼[かれ]は 金[かね]に
\\	ね。			
\\	げらげら	げらげら	げらげら	
\\	彼女はいつまでもげらげら笑っていたな。	彼女[かのじょ]はいつまでもげらげら 笑[わら]っていたな。	かのじょ は いつ まで も げらげら わらって いた な	
\\	彼女[かのじょ]はいつまでも
\\	笑[わら]っていたな。			
\\	一斉に	一斉[いっせい]に	いっせいに	
\\	彼らは一斉に拍手したね。	彼[かれ]らは 一斉[いっせい]に 拍手[はくしゅ]したね。	かれら は いっせいに はくしゅ した ね	
\\	彼[かれ]らは
\\	拍手[はくしゅ]したね。			
\\	刑務所	刑務所[けいむしょ]	けいむしょ	
\\	彼は二度と刑務所から出ることができないの。	彼[かれ]は 二度[にど]と 刑務所[けいむしょ]から 出[で]ることができないの。	かれ は にどと けいむしょ から でる こと が できない の	
\\	彼[かれ]は 二度[にど]と
\\	から 出[で]ることができないの。			
\\	刑事	刑事[けいじ]	けいじ	
\\	刑事が現場を調べているわ。	刑事[けいじ]が 現場[げんば]を 調[しら]べているわ。	けいじ が げんば を しらべて いる わ	
\\	が 現場[げんば]を 調[しら]べているわ。			
\\	ゴールデンウィーク	ゴールデンウィーク	ゴールデンウィーク	
\\	ゴールデンウィークに海外旅行をします。	ゴールデンウィークに 海外旅行[かいがい りょこう]をします。	ごーるでんうぃーく に かいがい りょこう を します	
\\	に 海外旅行[かいがい りょこう]をします。			
\\	丘	丘[おか]	おか	
\\	丘の上にホテルが建ちましたね。	丘[おか]の 上[うえ]にホテルが 建[た]ちましたね。	おか の うえ に ほてる が たちました ね	
\\	の 上[うえ]にホテルが 建[た]ちましたね。			
\\	こたつ	こたつ	こたつ	
\\	寒いのでこたつを出しました。	寒[さむ]いのでこたつを 出[だ]しました。	さむい の で こたつ を だしました	
\\	寒[さむ]いので
\\	を 出[だ]しました。			
\\	ことわざ	ことわざ	ことわざ	
\\	日本のことわざをいくつくらい知っていますか。	日本[にほん]のことわざをいくつくらい 知[し]っていますか。	にほん の ことわざ を いくつ くらい しって います か	
\\	日本[にほん]の
\\	をいくつくらい 知[し]っていますか。			
\\	劣る	劣[おと]る	おとる	
\\	私は体力では誰にも劣りません。	私[わたし]は 体力[たいりょく]では 誰[だれ]にも 劣[おと]りません。	わたし は たいりょく で は だれ に も おとりません	
\\	私[わたし]は 体力[たいりょく]では 誰[だれ]にも
\\	勘定	勘定[かんじょう]	かんじょう	
\\	お勘定をして下さい。	お 勘定[かんじょう]をして 下[くだ]さい。	おかんじょう を して ください	
\\	お
\\	をして 下[くだ]さい。			
\\	勘	勘[かん]	かん	
\\	女の勘を甘く見てはいけません。	女[おんな]の 勘[かん]を 甘[あま]く 見[み]てはいけません。	おんな の かん を あまく みて は いけません	
\\	女[おんな]の
\\	を 甘[あま]く 見[み]てはいけません。			
\\	勘違い	勘違[かんちが]い	かんちがい	
\\	待ち合わせは2時だと勘違いしていました。	待[ま]ち 合[あ]わせは 2時[にじ]だと 勘違[かんちが]いしていました。	まちあわせ は にじ だ と かんちがい して いました	
\\	待[ま]ち 合[あ]わせは 2時[にじ]だと
\\	していました。			
\\	これから	これから	これから	
\\	これからはもっと気を付けます。	これからはもっと 気[き]を 付[つ]けます。	これから は もっと き を つけます	
\\	はもっと 気[き]を 付[つ]けます。			
\\	公衆	公衆[こうしゅう]	こうしゅう	
\\	彼は公衆電話を探したの。	彼[かれ]は 公衆[こうしゅう] 電話[でんわ]を 探[さが]したの。	かれ は こうしゅう でんわ を さがした の	
\\	彼[かれ]は
\\	電話[でんわ]を 探[さが]したの。			
\\	お仕舞い	お 仕舞[しま]い	おしまい	
\\	話はこれでお仕舞いです。	話[はなし]はこれでお 仕舞[しま]いです。	はなし は これ で おしまい です	
\\	話[はなし]はこれで
\\	です。			
\\	さす	さす	さす	
\\	自転車に油をさしたの。	自転車[じてんしゃ]に 油[あぶら]をさしたの。	じてんしゃ に あぶら を さした の 。	
\\	自転車[じてんしゃ]に 油[あぶら]を
\\	の。			
\\	児童	児童[じどう]	じどう	
\\	ここは児童の通学路です。	ここは 児童[じどう]の 通学路[つうがくろ]です。	ここ は じどう の つうがくろ です	
\\	ここは
\\	の 通学路[つうがくろ]です。			
\\	しびれる	しびれる	しびれる	
\\	足がしびれた。	足[あし]がしびれた。	あし が しびれた	
\\	足[あし]が
\\	先祖	先祖[せんぞ]	せんぞ	
\\	神道では先祖を大切にします。	神道[しんとう]では 先祖[せんぞ]を 大切[たいせつ]にします。	しんとう で は せんぞ を たいせつ に します	
\\	神道[しんとう]では
\\	を 大切[たいせつ]にします。			
\\	じゃぶじゃぶ	じゃぶじゃぶ	じゃぶじゃぶ	
\\	彼は顔をじゃぶじゃぶ洗ったの。	彼[かれ]は 顔[かお]をじゃぶじゃぶ 洗[あら]ったの。	かれ は かお を じゃぶじゃぶ あらった の	
\\	彼[かれ]は 顔[かお]を
\\	洗[あら]ったの。			
\\	お年玉	お 年玉[としだま]	おとしだま	
\\	甥と姪にお年玉をあげたの。	甥[おい]と 姪[めい]にお 年玉[としだま]をあげたの。	おい と めい に おとしだま を あげた の	
\\	甥[おい]と 姪[めい]に
\\	をあげたの。			
\\	じゃんじゃん	じゃんじゃん	じゃんじゃん	
\\	電話がじゃんじゃん掛かってきたな。	電話[でんわ]がじゃんじゃん 掛[か]かってきたな。	でんわ が じゃんじゃん かかって きた な	
\\	電話[でんわ]が
\\	掛[か]かってきたな。			
\\	しわ	しわ	しわ	
\\	スカートにしわがよっているよ。	スカートにしわがよっているよ。	すかーと に しわ が よって いる よ	
\\	スカートに
\\	がよっているよ。			
\\	下痢	下痢[げり]	げり	
\\	古いお寿司を食べて下痢をしてしまったんだ。	古[ふる]いお 寿司[すし]を 食[た]べて 下痢[げり]をしてしまったんだ。	ふるい おすし を たべて げり を して しまった ん だ	
\\	古[ふる]いお 寿司[すし]を 食[た]べて
\\	をしてしまったんだ。			
\\	ずるがしこい	ずるがしこい	ずるがしこい	
\\	あいつはずるがしこい顔をしているね。	あいつはずるがしこい 顔[かお]をしているね。	あいつ は ずるがしこい かお を して いる ね	
\\	あいつは
\\	顔[かお]をしているね。			
\\	お嬢さん	お 嬢[じょう]さん	おじょうさん	
\\	彼はお嬢さんと一緒でした。	彼[かれ]はお 嬢[じょう]さんと 一緒[いっしょ]でした。	かれ は おじょうさん と いっしょ でした	
\\	彼[かれ]は
\\	と 一緒[いっしょ]でした。			
\\	せいぜい	せいぜい	せいぜい	
\\	どんなに頑張っても、せいぜい3位くらいにしかなれないだろうな。	どんなに 頑張[がんば]っても、せいぜい 3位[さんい]くらいにしかなれないだろうな。	どんな に がんばって も せいぜい さんい くらい に しか なれない だろう な	
\\	どんなに 頑張[がんば]っても、
\\	3位[さんい]くらいにしかなれないだろうな。			
\\	ぜいたく	ぜいたく	ぜいたく	
\\	彼女は一生ぜいたくに暮らしたんだ。	彼女[かのじょ]は 一生[いっしょう]ぜいたくに 暮[く]らしたんだ。	かのじょ は いっしょう ぜいたく に くらした ん だ	
\\	彼女[かのじょ]は 一生[いっしょう]
\\	に 暮[く]らしたんだ。			
\\	一旦	一旦[いったん]	いったん	
\\	疲れたでしょう、一旦休みましょう。	疲[つか]れたでしょう、 一旦[いったん] 休[やす]みましょう。	つかれた でしょう いったん やすみましょう	
\\	疲[つか]れたでしょう、
\\	休[やす]みましょう。			
\\	乞食	乞食[こじき]	こじき	
\\	乞食が公園のベンチで寝ている。	乞食[こじき]が 公園[こうえん]のベンチで 寝[ね]ている。	こじき が こうえん の べんち で ねて いる	
\\	が 公園[こうえん]のベンチで 寝[ね]ている。			
\\	亀	亀[かめ]	かめ	
\\	この池には亀がいますね。	この 池[いけ]には 亀[かめ]がいますね。	この いけ に は かめ が います ね	
\\	この 池[いけ]には
\\	がいますね。			
\\	がくんと	がくんと	がくんと	
\\	今学期は成績ががくんと落ちてしまった。	今学期[こんがっき]は 成績[せいせき]ががくんと 落[お]ちてしまった。	こんがっき は せいせき が がくんと おちて しまった	
\\	今学期[こんがっき]は 成績[せいせき]が
\\	落[お]ちてしまった。			
\\	きらりと	きらりと	きらりと	
\\	ダイヤモンドがきらりと光った。	ダイヤモンドがきらりと 光[ひか]った。	だいやもんど が きらりと ひかった	
\\	ダイヤモンドが
\\	光[ひか]った。			
\\	ごくんと	ごくんと	ごくんと	
\\	薬をごくんと飲み込んだんだ。	薬[くすり]をごくんと 飲[の]み 込[こ]んだんだ。	くすり を ごくんと のみこんだ ん だ	
\\	薬[くすり]を
\\	飲[の]み 込[こ]んだんだ。			
\\	お詫び	お 詫[わ]び	おわび	
\\	誤解があったことをお詫びします。	誤解[ごかい]があったことをお 詫[わ]びします。	ごかい が あった こと を おわび します	
\\	誤解[ごかい]があったことを
\\	します。			
\\	一まず	一[ひと]まず	ひとまず	
\\	一まず休憩しましょう。	一[ひと]まず 休憩[きゅうけい]しましょう。	ひとまず きゅうけい しましょう	
\\	休憩[きゅうけい]しましょう。			
\\	万一	万一[まんいち]	まんいち	
\\	万一のために保険に入ったの。	万一[まんいち]のために 保険[ほけん]に 入[はい]ったの。	まんいち の ため に ほけん に はいった の	
\\	のために 保険[ほけん]に 入[はい]ったの。			
\\	データ	データ	データ	
\\	去年のデータを見せてください。	去年[きょねん]のデータを 見[み]せてください。	きょねん の でーた を みせて ください 。	
\\	去年[きょねん]の
\\	を 見[み]せてください。			
\\	三日月	三日月[みかづき]	みかづき	
\\	空に三日月が見えました。	空[そら]に 三日月[みかづき]が 見[み]えました。	そら に みかづき が みえました	
\\	空[そら]に
\\	が 見[み]えました。			
\\	ただ	ただ	ただ	
\\	私はただ彼女と話したかっただけです。	私[わたし]はただ 彼女[かのじょ]と 話[はな]したかっただけです。	わたし は ただ かのじょ と はなしたかった だけ です	
\\	私[わたし]は
\\	彼女[かのじょ]と 話[はな]したかっただけです。			
\\	もっとも	もっとも	もっとも	
\\	彼の意見はもっともです。	彼[かれ]の 意見[いけん]はもっともです。	かれ の いけん は もっとも です	
\\	彼[かれ]の 意見[いけん]は
\\	です。			
\\	何か	何[なに]か	なにか	
\\	何か質問はありますか。	何[なに]か 質問[しつもん]はありますか。	なにか しつもん は あります か	
\\	質問[しつもん]はありますか。			
\\	何とか	何[なん]とか	なんとか	
\\	何とかお願いします。	何[なん]とかお 願[ねが]いします。	なんとか おねがい します	
\\	お 願[ねが]いします。			
\\	何より	何[なに]より	なにより	
\\	ご無事で何よりです。	ご 無事[ぶじ]で 何[なに]よりです。	ごぶじ で なにより です	
\\	ご 無事[ぶじ]で
\\	です。			
\\	何と	何[なん]と	なんと	
\\	何と彼らは結婚したそうです。	何[なん]と 彼[かれ]らは 結婚[けっこん]したそうです。	なんと かれら は けっこん した そう です	
\\	彼[かれ]らは 結婚[けっこん]したそうです。			
\\	何となく	何[なん]となく	なんとなく	
\\	パーティーに行くのは何となく気が進まないな。	パーティーに 行[い]くのは 何[なん]となく 気[き]が 進[すす]まないな。	ぱーてぃー に いく の は なんとなく き が すすまない な	
\\	パーティーに 行[い]くのは
\\	気[き]が 進[すす]まないな。			
\\	何だか	何[なん]だか	なんだか	
\\	茶柱が立って何だか少し幸せな気分です。	茶柱[ちゃばしら]が 立[た]って 何[なん]だか 少[すこ]し 幸[しあわ]せな 気分[きぶん]です。	ちゃばしら が たって なんだか すこし しあわせ な きぶん です	
\\	茶柱[ちゃばしら]が 立[た]って
\\	少[すこ]し 幸[しあわ]せな 気分[きぶん]です。			
\\	メーカー	メーカー	メーカー	
\\	彼は靴のメーカーで働いています。	彼[かれ]は 靴[くつ]のメーカーで 働[はたら]いています。	かれ は くつ の めーかー で はたらいて います	
\\	彼[かれ]は 靴[くつ]の
\\	で 働[はたら]いています。			
\\	何で	何[なん]で	なんで	
\\	彼女は何で来ないんだろう。	彼女[かのじょ]は 何[なん]で 来[こ]ないんだろう。	かのじょ は なんで こないん だろう	
\\	彼女[かのじょ]は
\\	来[こ]ないんだろう。			
\\	何十	何十[なんじゅう]	なんじゅう	
\\	その店に何十人も行列していたよ。	その 店[みせ]に 何十[なんじゅう] 人[にん]も 行列[ぎょうれつ]していたよ。	その みせ に なんじゅうにん も ぎょうれつ して いた よ	
\\	その 店[みせ]に
\\	人[にん]も 行列[ぎょうれつ]していたよ。			
\\	何て	何[なん]て	なんて	
\\	何てきれいな人なんだ。	何[なん]てきれいな 人[ひと]なんだ。	なんて きれい な ひと なんだ	
\\	きれいな 人[ひと]なんだ。			
\\	何しろ	何[なに]しろ	なにしろ	
\\	彼は何しろよくしゃべります。	彼[かれ]は 何[なに]しろよくしゃべります。	かれ は なにしろ よく しゃべります	
\\	彼[かれ]は
\\	よくしゃべります。			
\\	何千	何千[なんぜん]	なんぜん	
\\	新聞社に何千ものメールが寄せられたよ。	新聞社[しんぶんしゃ]に 何千[なんぜん]ものメールが 寄[よ]せられたよ。	しんぶんしゃ に なんぜん も の めーる が よせられた よ	
\\	新聞社[しんぶんしゃ]に
\\	ものメールが 寄[よ]せられたよ。			
\\	レベル	レベル	レベル	
\\	自分のレベルに合った授業を選んでください。	自分[じぶん]のレベルに 合[あ]った 授業[じゅぎょう]を 選[えら]んでください。	じぶん の れべる に あった じゅぎょう を えらんで ください	
\\	自分[じぶん]の
\\	に 合[あ]った 授業[じゅぎょう]を 選[えら]んでください。			
\\	何百	何百[なんびゃく]	なんびゃく	
\\	海で何百もの美しい魚を見ました。	海[うみ]で 何百[なんびゃく]もの 美[うつく]しい 魚[さかな]を 見[み]ました。	うみ で なんびゃく も の うつくしい さかな を みました	
\\	海[うみ]で
\\	もの 美[うつく]しい 魚[さかな]を 見[み]ました。			
\\	やって来る	やって 来[く]る	やってくる	
\\	フランスからサーカスがやって来ます。	フランスからサーカスがやって 来[き]ます。	ふらんす から さーかす が やってきます	
\\	フランスからサーカスが
\\	むしろ	むしろ	むしろ	
\\	彼は建築家というよりむしろ芸術家ね。	彼[かれ]は 建築家[けんちくか]というよりむしろ 芸術家[げいじゅつか]ね。	かれ は けんちくか と いう より むしろ げいじゅつか ね	
\\	彼[かれ]は 建築家[けんちくか]というより
\\	芸術家[げいじゅつか]ね。			
\\	中年	中年[ちゅうねん]	ちゅうねん	
\\	これは、中年の男性によく見られる症状です。	これは、 中年[ちゅうねん]の 男性[だんせい]によく 見[み]られる 症状[しょうじょう]です。	これ は ちゅうねん の だんせい に よく みられる しょうじょう です	
\\	これは、
\\	の 男性[だんせい]によく 見[み]られる 症状[しょうじょう]です。			
\\	中	中[ちゅう]	ちゅう	
\\	うちの子の身長はクラスで中くらいです。	うちの 子[こ]の 身長[しんちょう]はクラスで 中[ちゅう]くらいです。	うち の こ の しんちょう は くらす で ちゅう くらい です	
\\	うちの 子[こ]の 身長[しんちょう]はクラスで
\\	くらいです。			
\\	ますます	ますます	ますます	
\\	状況はますます悪くなったね。	状況[じょうきょう]はますます 悪[わる]くなったね。	じょうきょう は ますます わるく なった ね	
\\	状況[じょうきょう]は
\\	悪[わる]くなったね。			
\\	ぶら下がる	ぶら 下[さ]がる	ぶらさがる	
\\	猿が木にぶら下がってるな。	猿[さる]が 木[き]にぶら 下[さ]がってるな。	さる が き に ぶらさがって る な	
\\	猿[さる]が 木[き]に
\\	な。			
\\	ぶら下げる	ぶら 下[さ]げる	ぶらさげる	
\\	彼は首にカメラをぶら下げているね。	彼[かれ]は 首[くび]にカメラをぶら 下[さ]げているね。	かれ は くび に かめら を ぶらさげて いる ね	
\\	彼[かれ]は 首[くび]にカメラを
\\	ね。			
\\	モデル	モデル	モデル	
\\	彼女はこの小説の主人公のモデルです。	彼女[かのじょ]はこの 小説[しょうせつ]の 主人公[しゅじんこう]のモデルです。	かのじょ は この しょうせつ の しゅじんこう の もでる です	
\\	彼女[かのじょ]はこの 小説[しょうせつ]の 主人公[しゅじんこう]の
\\	です。			
\\	下	下[もと]	もと	
\\	私はその教授の下で論文を書いたの。	私[わたし]はその 教授[きょうじゅ]の 下[もと]で 論文[ろんぶん]を 書[か]いたの。	わたし は その きょうじゅ の もと で ろんぶん を かいた の	
\\	私[わたし]はその 教授[きょうじゅ]の
\\	で 論文[ろんぶん]を 書[か]いたの。			
\\	やり方	やり 方[かた]	やりかた	
\\	仕事のやり方を教えてください。	仕事[しごと]のやり 方[かた]を 教[おし]えてください。	しごと の やりかた を おしえて ください	
\\	仕事[しごと]の
\\	を 教[おし]えてください。			
\\	一人一人	一人一人[ひとりひとり]	ひとりひとり	
\\	この学校では一人一人の生徒を大切にしているの。	この 学校[がっこう]では 一人一人[ひとりひとり]の 生徒[せいと]を 大切[たいせつ]にしているの。	この がっこう で は ひとりひとり の せいと を たいせつ に して いる の	
\\	この 学校[がっこう]では
\\	の 生徒[せいと]を 大切[たいせつ]にしているの。			
\\	もたらす	もたらす	もたらす	
\\	彼はこの国に平和をもたらしたのよ。	彼[かれ]はこの 国[くに]に 平和[へいわ]をもたらしたのよ。	かれ は この くに に へいわ を もたらした の よ	
\\	彼[かれ]はこの 国[くに]に 平和[へいわ]を
\\	のよ。			
\\	出入り	出入[でい]り	でいり	
\\	彼の家は人の出入りが多いね。	彼[かれ]の 家[いえ]は 人[ひと]の 出入[でい]りが 多[おお]いね。	かれ の いえ は ひと の でいり が おおい ね	
\\	彼[かれ]の 家[いえ]は 人[ひと]の
\\	が 多[おお]いね。			
\\	出来上がる	出来上[できあ]がる	できあがる	
\\	この家は来年出来上がります。	この 家[いえ]は 来年[らいねん] 出来上[できあ]がります。	この いえ は らいねん できあがります	
\\	この 家[いえ]は 来年[らいねん]
\\	人出	人出[ひとで]	ひとで	
\\	今日も遊園地は大変な人出だったよ。	今日[きょう]も 遊園地[ゆうえんち]は 大変[たいへん]な 人出[ひとで]だったよ。	きょう も ゆうえんち は たいへん な ひとで だった よ	
\\	今日[きょう]も 遊園地[ゆうえんち]は 大変[たいへん]な
\\	だったよ。			
\\	それぞれ	それぞれ	それぞれ	
\\	人はそれぞれ考え方が違います。	人[ひと]はそれぞれ 考[かんが]え 方[かた]が 違[ちが]います。	ひと は それぞれ かんがえかた が ちがいます	
\\	人[ひと]は
\\	考[かんが]え 方[かた]が 違[ちが]います。			
\\	出来上がり	出来上[できあ]がり	できあがり	
\\	私は作品の出来上がりに満足しています。	私[わたし]は 作品[さくひん]の 出来上[できあ]がりに 満足[まんぞく]しています。	わたし は さくひん の できあがり に まんぞく して います	
\\	私[わたし]は 作品[さくひん]の
\\	に 満足[まんぞく]しています。			
\\	出来るだけ	出来[でき]るだけ	できるだけ	
\\	出来るだけ早く来てください。	出来[でき]るだけ 早[はや]く 来[き]てください。	できるだけ はやく きて ください	
\\	早[はや]く 来[き]てください。			
\\	その内	その 内[うち]	そのうち	
\\	彼女もその内、親の気持ちが分かるだろう。	彼女[かのじょ]もその 内[うち]、 親[おや]の 気持[きも]ちが 分[わ]かるだろう。	かのじょ も そのうち おや の きもち が わかる だろう	
\\	彼女[かのじょ]も
\\	、 親[おや]の 気持[きも]ちが 分[わ]かるだろう。			
\\	トップ	トップ	トップ	
\\	彼はトップの成績で合格しました。	彼[かれ]はトップの 成績[せいせき]で 合格[ごうかく]しました。	かれ は とっぷ の せいせき で ごうかく しました	
\\	彼[かれ]は
\\	の 成績[せいせき]で 合格[ごうかく]しました。			
\\	休める	休[やす]める	やすめる	
\\	疲れた体を休めてください。	疲[つか]れた 体[からだ]を 休[やす]めてください。	つかれた からだ を やすめて ください	
\\	疲[つか]れた 体[からだ]を
\\	ください。			
\\	ブーム	ブーム	ブーム	
\\	最近、日本は健康ブームです。	最近[さいきん]、 日本[にほん]は 健康[けんこう]ブームです。	さいきん にほん は けんこう ぶーむ です	
\\	最近[さいきん]、 日本[にほん]は 健康[けんこう]
\\	です。			
\\	一休み	一休[ひとやす]み	ひとやすみ	
\\	ここで一休みしましょう。	ここで 一休[ひとやす]みしましょう。	ここ で ひとやすみ しましょう	
\\	ここで
\\	しましょう。			
\\	入力	入力[にゅうりょく]	にゅうりょく	
\\	彼女の仕事はデータの入力です。	彼女[かのじょ]の 仕事[しごと]はデータの 入力[にゅうりょく]です。	かのじょ の しごと は でーた の にゅうりょく です	
\\	彼女[かのじょ]の 仕事[しごと]はデータの
\\	です。			
\\	体力	体力[たいりょく]	たいりょく	
\\	若者は体力がありますね。	若者[わかもの]は 体力[たいりょく]がありますね。	わかもの は たいりょく が あります ね	
\\	若者[わかもの]は
\\	がありますね。			
\\	ようやく	ようやく	ようやく	
\\	ようやくゴールが見えてきました。	ようやくゴールが 見[み]えてきました。	ようやく ごーる が みえて きました	
\\	ゴールが 見[み]えてきました。			
\\	出入り口	出入[でい]り 口[ぐち]	でいりぐち	
\\	出入り口に車を停めないでください。	出入[でい]り 口[ぐち]に 車[くるま]を 停[と]めないでください。	でいりぐち に くるま を とめない で ください	
\\	に 車[くるま]を 停[と]めないでください。			
\\	一口	一口[ひとくち]	ひとくち	
\\	彼はおまんじゅうを一口で食べたの。	彼[かれ]はおまんじゅうを 一口[ひとくち]で 食[た]べたの。	かれ は おまんじゅう を ひとくち で たべた の	
\\	彼[かれ]はおまんじゅうを
\\	で 食[た]べたの。			
\\	人手	人手[ひとで]	ひとで	
\\	人手が足りなくて忙しい。	人手[ひとで]が 足[た]りなくて 忙[いそが]しい。	ひとで が たりなくて いそがしい	
\\	が 足[た]りなくて 忙[いそが]しい。			
\\	チーム	チーム	チーム	
\\	彼はチームの一員です。	彼[かれ]はチームの 一員[いちいん]です。	かれ は ちーむ の いちいん です	
\\	彼[かれ]は
\\	の 一員[いちいん]です。			
\\	タイプ	タイプ	タイプ	
\\	同じタイプで色の違うものはありますか。	同[おな]じタイプで 色[いろ]の 違[ちが]うものはありますか。	おなじ たいぷ で いろ の ちがう もの は あります か	
\\	同[おな]じ
\\	で 色[いろ]の 違[ちが]うものはありますか。			
\\	友人	友人[ゆうじん]	ゆうじん	
\\	彼は高校時代からの友人です。	彼[かれ]は 高校時代[こうこう じだい]からの 友人[ゆうじん]です。	かれ は こうこう じだい から の ゆうじん です	
\\	彼[かれ]は 高校時代[こうこう じだい]からの
\\	です。			
\\	友	友[とも]	とも	
\\	持つべきものは友よね。	持[も]つべきものは 友[とも]よね。	もつべき もの は とも よ ね	
\\	持[も]つべきものは
\\	よね。			
\\	ビジネス	ビジネス	ビジネス	
\\	彼はビジネスクラスに乗ったの。	彼[かれ]はビジネスクラスに 乗[の]ったの。	かれ は びじねすくらす に のった の	
\\	彼[かれ]は
\\	クラスに 乗[の]ったの。			
\\	元	元[もと]	もと	
\\	あの人は卓球の元選手です。	あの 人[ひと]は 卓球[たっきゅう]の 元[もと] 選手[せんしゅ]です。	あの ひと は たっきゅう の もとせんしゅ です	
\\	あの 人[ひと]は 卓球[たっきゅう]の
\\	選手[せんしゅ]です。			
\\	元々	元々[もともと]	もともと	
\\	彼女は元々フランスに行くつもりだったの。	彼女[かのじょ]は 元々[もともと]フランスに 行[い]くつもりだったの。	かのじょ は もともと ふらんす に いく つもり だった の	
\\	彼女[かのじょ]は
\\	フランスに 行[い]くつもりだったの。			
\\	ネットワーク	ネットワーク	ネットワーク	
\\	最近、社内ネットワークの調子が悪い。	最近[さいきん]、 社内[しゃない]ネットワークの 調子[ちょうし]が 悪[わる]い。	さいきん しゃない ねっとわーく の ちょうし が わるい	
\\	最近[さいきん]、 社内[しゃない]
\\	の 調子[ちょうし]が 悪[わる]い。			
\\	南西	南西[なんせい]	なんせい	
\\	ここから南西に進むと村があります。	ここから 南西[なんせい]に 進[すす]むと 村[むら]があります。	ここ から なんせい に すすむ と むら が あります	
\\	ここから
\\	に 進[すす]むと 村[むら]があります。			
\\	南東	南東[なんとう]	なんとう	
\\	南東の方角に月が見えるよ。	南東[なんとう]の 方角[ほうがく]に 月[つき]が 見[み]えるよ。	なんとう の ほうがく に つき が みえる よ	
\\	の 方角[ほうがく]に 月[つき]が 見[み]えるよ。			
\\	南北	南北[なんぼく]	なんぼく	
\\	南北に山が広がっています。	南北[なんぼく]に 山[やま]が 広[ひろ]がっています。	なんぼく に やま が ひろがって います	
\\	に 山[やま]が 広[ひろ]がっています。			
\\	北西	北西[ほくせい]	ほくせい	
\\	台風は北西に進んでいます。	台風[たいふう]は 北西[ほくせい]に 進[すす]んでいます。	たいふう は ほくせい に すすんで います	
\\	台風[たいふう]は
\\	に 進[すす]んでいます。			
\\	やや	やや	やや	
\\	この服、私にはやや小さいみたい。	この 服[ふく]、 私[わたし]にはやや 小[ちい]さいみたい。	この ふく わたし に は やや ちいさい みたい	
\\	この 服[ふく]、 私[わたし]には
\\	小[ちい]さいみたい。			
\\	北東	北東[ほくとう]	ほくとう	
\\	町の北東に山があります。	町[まち]の 北東[ほくとう]に 山[やま]があります。	まち の ほくとう に やま が あります	
\\	町[まち]の
\\	に 山[やま]があります。			
\\	入門	入門[にゅうもん]	にゅうもん	
\\	私は相撲部屋に入門しました。	私[わたし]は 相撲部屋[すもう べや]に 入門[にゅうもん]しました。	わたし は すもう べや に にゅうもん しました	
\\	私[わたし]は 相撲部屋[すもう べや]に
\\	しました。			
\\	デモ	デモ	デモ	
\\	彼女はデモに参加したよ。	彼女[かのじょ]はデモに 参加[さんか]したよ。	かのじょ は でも に さんか した よ	
\\	彼女[かのじょ]は
\\	に 参加[さんか]したよ。			
\\	中間	中間[ちゅうかん]	ちゅうかん	
\\	そのふたつの町の中間には川が流れているの。	そのふたつの 町[まち]の 中間[ちゅうかん]には 川[かわ]が 流[なが]れているの。	その ふたつ の まち の ちゅうかん に は かわ が ながれて いる の	
\\	そのふたつの 町[まち]の
\\	には 川[かわ]が 流[なが]れているの。			
\\	ワープロ	ワープロ	ワープロ	
\\	ワープロで報告書を打ったよ。	ワープロで 報告書[ほうこくしょ]を 打[う]ったよ。	わーぷろ で ほうこくしょ を うった よ	
\\	で 報告書[ほうこくしょ]を 打[う]ったよ。			
\\	低下	低下[ていか]	ていか	
\\	最近、教育レベルが低下しています。	最近[さいきん]、 教育[きょういく]レベルが 低下[ていか]しています。	さいきん きょういく れべる が ていか して います	
\\	最近[さいきん]、 教育[きょういく]レベルが
\\	しています。			
\\	低	低[てい]	てい	
\\	彼は低賃金で働いているんだ。	彼[かれ]は 低[てい] 賃金[ちんぎん]で 働[はたら]いているんだ。	かれ は ていちんぎん で はたらいて いる ん だ	
\\	彼[かれ]は
\\	賃金[ちんぎん]で 働[はたら]いているんだ。			
\\	メンバー	メンバー	メンバー	
\\	グループのメンバーは10人です。	グループのメンバーは10 人[にん]です。	ぐるーぷ の めんばー は 
\\	にん です 。	
\\	グループの
\\	は10 人[にん]です。			
\\	前年	前年[ぜんねん]	ぜんねん	
\\	前年に比べて利益が上がりました。	前年[ぜんねん]に 比[くら]べて 利益[りえき]が 上[あ]がりました。	ぜんねん に くらべて りえき が あがりました	
\\	に 比[くら]べて 利益[りえき]が 上[あ]がりました。			
\\	前方	前方[ぜんぽう]	ぜんぽう	
\\	前方に山が見えますね。	前方[ぜんぽう]に 山[やま]が 見[み]えますね。	ぜんぽう に やま が みえます ね	
\\	に 山[やま]が 見[み]えますね。			
\\	前向き	前向[まえむ]き	まえむき	
\\	前向きに検討します。	前向[まえむ]きに 検討[けんとう]します。	まえむき に けんとう します	
\\	に 検討[けんとう]します。			
\\	ハードウェア	ハードウェア	ハードウェア	
\\	この問題はハードウェアの故障が原因です。	この 問題[もんだい]はハードウェアの 故障[こしょう]が 原因[げんいん]です。	この もんだい は はーどうぇあ の こしょう が げんいん です	
\\	この 問題[もんだい]は
\\	の 故障[こしょう]が 原因[げんいん]です。			
\\	人前	人前[ひとまえ]	ひとまえ	
\\	彼女は人前に出ると緊張します。	彼女[かのじょ]は 人前[ひとまえ]に 出[で]ると 緊張[きんちょう]します。	かのじょ は ひとまえ に でる と きんちょう します	
\\	彼女[かのじょ]は
\\	に 出[で]ると 緊張[きんちょう]します。			
\\	前もって	前[まえ]もって	まえもって	
\\	休む時は前もって連絡ください。	休[やす]む 時[とき]は 前[まえ]もって 連絡[れんらく]ください。	やすむ とき は まえもって れんらく ください	
\\	休[やす]む 時[とき]は
\\	連絡[れんらく]ください。			
\\	出前	出前[でまえ]	でまえ	
\\	寿司の出前を頼んだよ。	寿司[すし]の 出前[でまえ]を 頼[たの]んだよ。	すし の でまえ を たのんだ よ	
\\	寿司[すし]の
\\	を 頼[たの]んだよ。			
\\	前半	前半[ぜんはん]	ぜんはん	
\\	相手チームのリードで前半が終わったよ。	相手[あいて]チームのリードで 前半[ぜんはん]が 終[お]わったよ。	あいて ちーむ の りーど で ぜんはん が おわった よ	
\\	相手[あいて]チームのリードで
\\	が 終[お]わったよ。			
\\	半ば	半[なか]ば	なかば	
\\	彼は30代の半ばです。	彼[かれ]は30 代[だい]の 半[なか]ばです。	かれ は 
\\	だい の なかば です	
\\	彼[かれ]は30 代[だい]の
\\	です。			
\\	マスコミ	マスコミ	マスコミ	
\\	彼はよくマスコミにも登場するね。	彼[かれ]はよくマスコミにも 登場[とうじょう]するね。	かれ は よく ますこみ に も とうじょう する ね	
\\	彼[かれ]はよく
\\	にも 登場[とうじょう]するね。			
\\	半年	半年[はんとし]	はんとし	
\\	日本に来て半年になります。	日本[にっぽん]に 来[き]て 半年[はんとし]になります。	にっぽん に きて はんとし に なります	
\\	日本[にっぽん]に 来[き]て
\\	になります。			
\\	やがて	やがて	やがて	
\\	やがて雪も止むでしょう。	やがて 雪[ゆき]も 止[や]むでしょう。	やがて ゆき も やむ でしょう	
\\	雪[ゆき]も 止[や]むでしょう。			
\\	七夕	七夕[たなばた]	たなばた	
\\	日本では7月7日は七夕の日です。	日本[にほん]では7 月7日[がつ 
\\	か]は 七夕[たなばた]の 日[ひ]です。	にほん で は 
\\	がつ 
\\	か は たなばた の ひ です	
\\	日本[にほん]では7 月7日[がつ 
\\	か]は
\\	の 日[ひ]です。			
\\	わずか	わずか	わずか	
\\	彼はわずかな貯金で暮らしている。	彼[かれ]はわずかな 貯金[ちょきん]で 暮[く]らしている。	かれ は わずか な ちょきん で くらして いる	
\\	彼[かれ]は
\\	な 貯金[ちょきん]で 暮[く]らしている。			
\\	パターン	パターン	パターン	
\\	彼女の行動パターンは興味深いな。	彼女[かのじょ]の 行動[こうどう]パターンは 興味深[きょうみぶか]いな。	かのじょ の こうどう ぱたーん は きょうみぶかい な	
\\	彼女[かのじょ]の 行動[こうどう]
\\	は 興味深[きょうみぶか]いな。			
\\	よそ見	よそ 見[み]	よそみ	
\\	運転中はよそ見をしてはいけません。	運転中[うんてんちゅう]はよそ 見[み]をしてはいけません。	うんてんちゅう は よそみ を して は いけません	
\\	運転中[うんてんちゅう]は
\\	をしてはいけません。			
\\	ついに	ついに	ついに	
\\	彼はついに弁護士の資格を取ったよ。	彼[かれ]はついに 弁護士[べんごし]の 資格[しかく]を 取[と]ったよ。	かれ は ついに べんごし の しかく を とった よ	
\\	彼[かれ]は
\\	弁護士[べんごし]の 資格[しかく]を 取[と]ったよ。			
\\	一言	一言[ひとこと]	ひとこと	
\\	社長に一言お願いしたの。	社長[しゃちょう]に 一言[ひとこと]お 願[ねが]いしたの。	しゃちょう に ひとこと おねがい した の	
\\	社長[しゃちょう]に
\\	お 願[ねが]いしたの。			
\\	まるで	まるで	まるで	
\\	彼はまるで子供のようにはしゃいだよ。	彼[かれ]はまるで 子供[こども]のようにはしゃいだよ。	かれ は まるで こども の よう に はしゃいだ よ	
\\	彼[かれ]は
\\	子供[こども]のようにはしゃいだよ。			
\\	どんどん	どんどん	どんどん	
\\	彼は山道をどんどん進んで行ったの。	彼[かれ]は 山道[やまみち]をどんどん 進[すす]んで 行[い]ったの。	かれ は やまみち を どんどん すすんで いった の	
\\	彼[かれ]は 山道[やまみち]を
\\	進[すす]んで 行[い]ったの。			
\\	出会う	出会[であ]う	であう	
\\	アメリカで彼女と出会いました。	アメリカで 彼女[かのじょ]と 出会[であ]いました。	あめりか で かのじょ と であいました。	
\\	アメリカで 彼女[かのじょ]と
\\	それほど	それほど	それほど	
\\	彼がそれほど悩んでいたとは知らなかった。	彼[かれ]がそれほど 悩[なや]んでいたとは 知[し]らなかった。	かれ が それほど なやんで いた と は しらなかった	
\\	彼[かれ]が
\\	悩[なや]んでいたとは 知[し]らなかった。			
\\	入社	入社[にゅうしゃ]	にゅうしゃ	
\\	彼は昨年入社したの。	彼[かれ]は 昨年[さくねん] 入社[にゅうしゃ]したの。	かれ は さくねん にゅうしゃ した の	
\\	彼[かれ]は 昨年[さくねん]
\\	したの。			
\\	もはや	もはや	もはや	
\\	もはや彼の助けは必要じゃないの。	もはや 彼[かれ]の 助[たす]けは 必要[ひつよう]じゃないの。	もはや かれ の たすけ は ひつよう じゃ ない の	
\\	彼[かれ]の 助[たす]けは 必要[ひつよう]じゃないの。			
\\	出来事	出来事[できごと]	できごと	
\\	面白い出来事がありました。	面白[おもしろ]い 出来事[できごと]がありました。	おもしろい できごと が ありました	
\\	面白[おもしろ]い
\\	がありました。			
\\	入場	入場[にゅうじょう]	にゅうじょう	
\\	選手の入場です。	選手[せんしゅ]の 入場[にゅうじょう]です。	せんしゅ の にゅうじょう です	
\\	選手[せんしゅ]の
\\	です。			
\\	デザイン	デザイン	デザイン	
\\	この服のデザインは素敵ですね。	この 服[ふく]のデザインは 素敵[すてき]ですね。	この ふく の でざいん は すてき です ね 。	
\\	この 服[ふく]の
\\	は 素敵[すてき]ですね。			
\\	交じる	交[ま]じる	まじる	
\\	彼女は外国人に交じってダンスをしたの。	彼女[かのじょ]は 外国人[がいこくじん]に 交[ま]じってダンスをしたの。	かのじょ は がいこくじん に まじって だんす を した の	
\\	彼女[かのじょ]は 外国人[がいこくじん]に
\\	ダンスをしたの。			
\\	交わる	交[まじ]わる	まじわる	
\\	人と交わって、多くのことを学びました。	人[ひと]と 交[まじ]わって、 多[おお]くのことを 学[まな]びました。	ひと と まじわって おおく の こと を まなびました	
\\	人[ひと]と
\\	、 多[おお]くのことを 学[まな]びました。			
\\	交ぜる	交[ま]ぜる	まぜる	
\\	私も交ぜてください。	私[わたし]も 交[ま]ぜてください。	わたし も まぜて ください	
\\	私[わたし]も
\\	ください。			
\\	ファン	ファン	ファン	
\\	彼は大のサッカーファンです。	彼[かれ]は 大[だい]のサッカーファンです。	かれ は だい の さっかーふぁん です	
\\	彼[かれ]は 大[だい]のサッカー
\\	です。			
\\	交ざる	交[ま]ざる	まざる	
\\	大人も子供も交ざって遊んだの。	大人[おとな]も 子供[こども]も 交[ま]ざって 遊[あそ]んだの。	おとな も こども も まざって あそんだ の	
\\	大人[おとな]も 子供[こども]も
\\	遊[あそ]んだの。			
\\	交わる	交[まじ]わる	まじわる	
\\	二つの国道はここで交わります。	二[ふた]つの 国道[こくどう]はここで 交[まじ]わります。	ふたつ の こくどう は ここ で まじわります	
\\	二[ふた]つの 国道[こくどう]はここで
\\	なさる	なさる	なさる	
\\	お申し込みなさるのでしたら、こちらにお並びください。	お 申[もう]し 込[こ]みなさるのでしたら、こちらにお 並[なら]びください。	おもうしこみ なさる の でしたら こちら に お ならび ください	
\\	お 申[もう]し 込[こ]み
\\	のでしたら、こちらにお 並[なら]びください。			
\\	人通り	人通[ひとどお]り	ひとどおり	
\\	ここは人通りが激しいね。	ここは 人通[ひとどお]りが 激[はげ]しいね。	ここ は ひとどおり が はげしい ね	
\\	ここは
\\	が 激[はげ]しいね。			
\\	一通り	一通[ひととお]り	ひととおり	
\\	説明書を一通り読んだの。	説明書[せつめいしょ]を 一通[ひととお]り 読[よ]んだの。	せつめいしょ を ひととおり よんだ の	
\\	説明書[せつめいしょ]を
\\	読[よ]んだの。			
\\	まとまる	まとまる	まとまる	
\\	みんなの意見がまとまりました。	みんなの 意見[いけん]がまとまりました。	みんな の いけん が まとまりました	
\\	みんなの 意見[いけん]が
\\	ルール	ルール	ルール	
\\	このゲームのルールは簡単です。	このゲームのルールは 簡単[かんたん]です。	この げーむ の るーる は かんたん です	
\\	このゲームの
\\	は 簡単[かんたん]です。			
\\	他方	他方[たほう]	たほう	
\\	他方の視点からも見てみましょう。	他方[たほう]の 視点[してん]からも 見[み]てみましょう。	たほう の してん から も みて みましょう	
\\	の 視点[してん]からも 見[み]てみましょう。			
\\	他人	他人[たにん]	たにん	
\\	私は他人に住所を教えたくない。	私[わたし]は 他人[たにん]に 住所[じゅうしょ]を 教[おし]えたくない。	わたし は たにん に じゅうしょ を おしえたく ない	
\\	私[わたし]は
\\	に 住所[じゅうしょ]を 教[おし]えたくない。			
\\	中止	中止[ちゅうし]	ちゅうし	
\\	雨で運動会が中止になったの。	雨[あめ]で 運動会[うんどうかい]が 中止[ちゅうし]になったの。	あめ で うんどうかい が ちゅうし に なった の	
\\	雨[あめ]で 運動会[うんどうかい]が
\\	になったの。			
\\	二度と	二度[にど]と	にどと	
\\	もうここには二度と来ません。	もうここには 二度[にど]と 来[き]ません。	もう ここ に は にどと きません	
\\	もうここには
\\	来[き]ません。			
\\	ヘリコプター	ヘリコプター	ヘリコプター	
\\	事故現場の上空をヘリコプターが飛んでいます。	事故現場[じこ げんば]の 上空[じょうくう]をヘリコプターが 飛[と]んでいます。	じこ げんば の じょうくう を へりこぷたー が とんで います	
\\	事故現場[じこ げんば]の 上空[じょうくう]を
\\	が 飛[と]んでいます。			
\\	この度	この 度[たび]	このたび	
\\	この度はご結婚おめでとうございます。	この 度[たび]はご 結婚[けっこん]おめでとうございます。	このたび は ごけっこん おめでとう ございます	
\\	はご 結婚[けっこん]おめでとうございます。			
\\	ともかく	ともかく	ともかく	
\\	ともかく一度考え直しましょう。	ともかく 一度考[いちど かんが]え 直[なお]しましょう。	ともかく いちど かんがえなおしましょう	
\\	一度考[いちど かんが]え 直[なお]しましょう。			
\\	つい	つい	つい	
\\	会議中、ついあくびをしてしまいました。	会議中[かいぎちゅう]、ついあくびをしてしまいました。	かいぎちゅう つい あくび を して しまいました	
\\	会議中[かいぎちゅう]、
\\	あくびをしてしまいました。			
\\	トンネル	トンネル	トンネル	
\\	トンネルを抜けると海が見えたよ。	トンネルを 抜[ぬ]けると 海[うみ]が 見[み]えたよ。	とんねる を ぬける と うみ が みえた よ	
\\	を 抜[ぬ]けると 海[うみ]が 見[み]えたよ。			
\\	全力	全力[ぜんりょく]	ぜんりょく	
\\	全力で走れ。	全力[ぜんりょく]で 走[はし]れ。	ぜんりょく で はしれ	
\\	で 走[はし]れ。			
\\	内部	内部[ないぶ]	ないぶ	
\\	これは機械の内部の問題です。	これは 機械[きかい]の 内部[ないぶ]の 問題[もんだい]です。	これ は きかい の ないぶ の もんだい です	
\\	これは 機械[きかい]の
\\	の 問題[もんだい]です。			
\\	バランス	バランス	バランス	
\\	栄養バランスの良い食事をしよう。	栄養[えいよう]バランスの 良[い]い 食事[しょくじ]をしよう。	えいよう ばらんす の いい しょくじ を しよう	
\\	栄養[えいよう]
\\	の 良[い]い 食事[しょくじ]をしよう。			
\\	入国	入国[にゅうこく]	にゅうこく	
\\	私は留学生として日本に入国しました。	私[わたし]は 留学生[りゅうがくせい]として 日本[にっぽん]に 入国[にゅうこく]しました。	わたし は りゅうがくせい として にっぽん に にゅうこく しました	
\\	私[わたし]は 留学生[りゅうがくせい]として 日本[にっぽん]に
\\	しました。			
\\	どうしても	どうしても	どうしても	
\\	どうしてもその訳を知りたい。	どうしてもその 訳[わけ]を 知[し]りたい。	どうしても その わけ を しりたい	
\\	その 訳[わけ]を 知[し]りたい。			
\\	世の中	世[よ]の 中[なか]	よのなか	
\\	世の中にはいろいろな人がいます。	世[よ]の 中[なか]にはいろいろな 人[ひと]がいます。	よのなか に は いろいろ な ひと が います	
\\	にはいろいろな 人[ひと]がいます。			
\\	中世	中世[ちゅうせい]	ちゅうせい	
\\	彼女は中世の音楽が好きです。	彼女[かのじょ]は 中世[ちゅうせい]の 音楽[おんがく]が 好[す]きです。	かのじょ は ちゅうせい の おんがく が すき です	
\\	彼女[かのじょ]は
\\	の 音楽[おんがく]が 好[す]きです。			
\\	世	世[よ]	よ	
\\	やっと私の作品が世に出たの。	やっと 私[わたし]の 作品[さくひん]が 世[よ]に 出[で]たの。	やっと わたし の さくひん が よ に でた の	
\\	やっと 私[わたし]の 作品[さくひん]が
\\	に 出[で]たの。			
\\	どうか	どうか	どうか	
\\	どうかお許しください。	どうかお 許[ゆる]しください。	どうか おゆるし ください	
\\	お 許[ゆる]しください。			
\\	プロ	プロ	プロ	
\\	彼はスケートのプロです。	彼[かれ]はスケートのプロです。	かれ は すけーと の ぷろ です	
\\	彼[かれ]はスケートの
\\	です。			
\\	チャンス	チャンス	チャンス	
\\	これは素晴らしいチャンスだ。	これは 素晴[すば]らしいチャンスだ。	これ は すばらしい ちゃんす だ	
\\	これは 素晴[すば]らしい
\\	だ。			
\\	分解	分解[ぶんかい]	ぶんかい	
\\	機械を分解してみたの。	機械[きかい]を 分解[ぶんかい]してみたの。	きかい を ぶんかい して みた の	
\\	機械[きかい]を
\\	してみたの。			
\\	どうも	どうも	どうも	
\\	今日はどうも体の調子が悪い。	今日[きょう]はどうも 体[からだ]の 調子[ちょうし]が 悪[わる]い。	きょう は どうも からだ の ちょうし が わるい	
\\	今日[きょう]は
\\	体[からだ]の 調子[ちょうし]が 悪[わる]い。			
\\	ハード	ハード	ハード	
\\	最近、仕事がかなりハードです。	最近[さいきん]、 仕事[しごと]がかなりハードです。	さいきん しごと が かなり はーど です	
\\	最近[さいきん]、 仕事[しごと]がかなり
\\	です。			
\\	作り上げる	作[つく]り 上[あ]げる	つくりあげる	
\\	それは彼女が作り上げた話です。	それは 彼女[かのじょ]が 作[つく]り 上[あ]げた 話[はなし]です。	それ は かのじょ が つくりあげた はなし です	
\\	それは 彼女[かのじょ]が
\\	話[はなし]です。			
\\	作り話	作[つく]り 話[ばなし]	つくりばなし	
\\	その子が言っていることは作り話です。	その 子[こ]が 言[い]っていることは 作[つく]り 話[ばなし]です。	その こ が いって いる こと は つくりばなし です	
\\	その 子[こ]が 言[い]っていることは
\\	です。			
\\	マイナス	マイナス	マイナス	
\\	外の温度はマイナス3度です。	外[そと]の 温度[おんど]はマイナス3 度[ど]です。	そと の おんど は まいなす 
\\	ど です	
\\	外[そと]の 温度[おんど]は
\\	度[ど]です。			
\\	使い道	使[つか]い 道[みち]	つかいみち	
\\	ボーナスの使い道を考えているところです。	ボーナスの 使[つか]い 道[みち]を 考[かんが]えているところです。	ぼーなす の つかいみち を かんがえて いる ところ です	
\\	ボーナスの
\\	を 考[かんが]えているところです。			
\\	使い	使[つか]い	つかい	
\\	主人の使いで市役所に行くところです。	主人[しゅじん]の 使[つか]いで 市役所[しやくしょ]に 行[い]くところです。	しゅじん の つかい で しやくしょ に いく ところ です	
\\	主人[しゅじん]の
\\	で 市役所[しやくしょ]に 行[い]くところです。			
\\	とにかく	とにかく	とにかく	
\\	とにかく現場へ行ってみましょう。	とにかく 現場[げんば]へ 行[い]ってみましょう。	とにかく げんば へ いって みましょう	
\\	現場[げんば]へ 行[い]ってみましょう。			
\\	パイプ	パイプ	パイプ	
\\	このパイプは詰まっていますよ。	このパイプは 詰[つ]まっていますよ。	この ぱいぷ は つまって います よ	
\\	この
\\	は 詰[つ]まっていますよ。			
\\	値上がり	値上[ねあ]がり	ねあがり	
\\	野菜が値上がりしていますね。	野菜[やさい]が 値上[ねあ]がりしていますね。	やさい が ねあがり して います ね	
\\	野菜[やさい]が
\\	していますね。			
\\	値上げ	値上[ねあ]げ	ねあげ	
\\	バス代が値上げされました。	バス 代[だい]が 値上[ねあ]げされました。	ばすだい が ねあげ されました	
\\	バス 代[だい]が
\\	されました。			
\\	値下がり	値下[ねさ]がり	ねさがり	
\\	ガソリンが値下がりしました。	ガソリンが 値下[ねさ]がりしました。	がそりん が ねさがり しました	
\\	ガソリンが
\\	しました。			
\\	たちまち	たちまち	たちまち	
\\	空がたちまち曇ってきたね。	空[そら]がたちまち 曇[くも]ってきたね。	そら が たちまち くもって きた ね	
\\	空[そら]が
\\	曇[くも]ってきたね。			
\\	値下げ	値下[ねさ]げ	ねさげ	
\\	電話料金が値下げされた。	電話料金[でんわ りょうきん]が 値下[ねさ]げされた。	でんわ りょうきん が ねさげ された	
\\	電話料金[でんわ りょうきん]が
\\	された。			
\\	体格	体格[たいかく]	たいかく	
\\	彼はとても体格がいいわ。	彼[かれ]はとても 体格[たいかく]がいいわ。	かれ は とても たいかく が いい わ	
\\	彼[かれ]はとても
\\	がいいわ。			
\\	冬季	冬季[とうき]	とうき	
\\	次の冬季オリンピックは2年後です。	次[つぎ]の 冬季[とうき]オリンピックは2 年後[ねんご]です。	つぎ の とうき おりんぴっく は 
\\	ねんご です	
\\	次[つぎ]の
\\	オリンピックは2 年後[ねんご]です。			
\\	プラス	プラス	プラス	
\\	この経験はあなたにとってプラスになるでしょう。	この 経験[けいけん]はあなたにとってプラスになるでしょう。	この けいけん は あなた に とって ぷらす に なる でしょう	
\\	この 経験[けいけん]はあなたにとって
\\	になるでしょう。			
\\	低温	低温[ていおん]	ていおん	
\\	この製品は低温で保存してください。	この 製品[せいひん]は 低温[ていおん]で 保存[ほぞん]してください。	この せいひん は ていおん で ほぞん して ください	
\\	この 製品[せいひん]は
\\	で 保存[ほぞん]してください。			
\\	体温	体温[たいおん]	たいおん	
\\	今朝の体温は36度でした。	今朝[けさ]の 体温[たいおん]は36 度[ど]でした。	けさ の たいおん は 
\\	ど でした	
\\	今朝[けさ]の
\\	は36 度[ど]でした。			
\\	友情	友情[ゆうじょう]	ゆうじょう	
\\	彼らは強い友情で結ばれているな。	彼[かれ]らは 強[つよ]い 友情[ゆうじょう]で 結[むす]ばれているな。	かれら は つよい ゆうじょう で むすばれて いる な	
\\	彼[かれ]らは 強[つよ]い
\\	で 結[むす]ばれているな。			
\\	人情	人情[にんじょう]	にんじょう	
\\	この町の人たちには人情があるね。	この 町[まち]の 人[ひと]たちには 人情[にんじょう]があるね。	この まち の ひとたち に は にんじょう が ある ね	
\\	この 町[まち]の 人[ひと]たちには
\\	があるね。			
\\	リード	リード	リード	
\\	彼は彼女をリードしながら踊ったね。	彼[かれ]は 彼女[かのじょ]をリードしながら 踊[おど]ったね。	かれ は かのじょ を りーど しながら おどった ね	
\\	彼[かれ]は 彼女[かのじょ]を
\\	しながら 踊[おど]ったね。			
\\	中古	中古[ちゅうこ]	ちゅうこ	
\\	その車は中古で買ったんだ。	その 車[くるま]は 中古[ちゅうこ]で 買[か]ったんだ。	その くるま は ちゅうこ で かった ん だ	
\\	その 車[くるま]は
\\	で 買[か]ったんだ。			
\\	たとえ	たとえ	たとえ	
\\	たとえ、嵐になっても絶対に行く。	たとえ、 嵐[あらし]になっても 絶対[ぜったい]に 行[い]く。	たとえ あらし に なって も ぜったい に いく	
\\	、 嵐[あらし]になっても 絶対[ぜったい]に 行[い]く。			
\\	内心	内心[ないしん]	ないしん	
\\	彼女は内心どきどきしていましたよ。	彼女[かのじょ]は 内心[ないしん]どきどきしていましたよ。	かのじょ は ないしん どきどき して いました よ	
\\	彼女[かのじょ]は
\\	どきどきしていましたよ。			
\\	なるべく	なるべく	なるべく	
\\	なるべく早く仕事を終わらせてください。	なるべく 早[はや]く 仕事[しごと]を 終[お]わらせてください。	なるべく はやく しごと を おわらせて ください	
\\	早[はや]く 仕事[しごと]を 終[お]わらせてください。			
\\	テーマ	テーマ	テーマ	
\\	講演のテーマは何ですか。	講演[こうえん]のテーマは 何[なん]ですか。	こうえん の てーま は なん です か	
\\	講演[こうえん]の
\\	は 何[なん]ですか。			
\\	ムード	ムード	ムード	
\\	部屋を暗くしてムードを出してみたよ。	部屋[へや]を 暗[くら]くしてムードを 出[だ]してみたよ。	へや を くらく して むーど を だして みた よ	
\\	部屋[へや]を 暗[くら]くして
\\	を 出[だ]してみたよ。			
\\	不安	不安[ふあん]	ふあん	
\\	明日、病院に検査に行くので少し不安です。	明日[あす]、 病院[びょういん]に 検査[けんさ]に 行[い]くので 少[すこ]し 不安[ふあん]です。	あす びょういん に けんさ に いく の で すこし ふあん です	
\\	明日[あす]、 病院[びょういん]に 検査[けんさ]に 行[い]くので 少[すこ]し
\\	です。			
\\	不満	不満[ふまん]	ふまん	
\\	私は彼のやり方には不満です。	私[わたし]は 彼[かれ]のやり 方[かた]には 不満[ふまん]です。	わたし は かれ の やりかた に は ふまん です	
\\	私[わたし]は 彼[かれ]のやり 方[かた]には
\\	です。			
\\	不足	不足[ふそく]	ふそく	
\\	今年はひどい水不足よ。	今年[ことし]はひどい 水[みず] 不足[ぶそく]よ。	ことし は ひどい みずぶそく よ	
\\	今年[ことし]はひどい 水[みず]
\\	よ。			
\\	どく	どく	どく	
\\	そこをどいてください。	そこをどいてください。	そこをどいてください。	
\\	そこを
\\	ください。			
\\	不可能	不可能[ふかのう]	ふかのう	
\\	レポートを1日で仕上げるのは不可能です。	レポートを1 日[にち]で 仕上[しあ]げるのは 不可能[ふかのう]です。	れぽーと を 
\\	にち で しあげる の は ふかのう です	
\\	レポートを1 日[にち]で 仕上[しあ]げるのは
\\	です。			
\\	不十分	不十分[ふじゅうぶん]	ふじゅうぶん	
\\	その程度の努力では不十分です。	その 程度[ていど]の 努力[どりょく]では 不十分[ふじゅうぶん]です。	その ていど の どりょく で は ふじゅうぶん です	
\\	その 程度[ていど]の 努力[どりょく]では
\\	です。			
\\	不明	不明[ふめい]	ふめい	
\\	その病気は原因不明と言われているんだよ。	その 病気[びょうき]は 原因[げんいん] 不明[ふめい]と 言[い]われているんだよ。	その びょうき は げんいん ふめい と いわれている ん だ よ	
\\	その 病気[びょうき]は 原因[げんいん]
\\	と 言[い]われているんだよ。			
\\	不良	不良[ふりょう]	ふりょう	
\\	不良品を返品したよ。	不良[ふりょう] 品[ひん]を 返品[へんぴん]したよ。	ふりょうひん を へんぴん した よ	
\\	品[ひん]を 返品[へんぴん]したよ。			
\\	不合格	不合格[ふごうかく]	ふごうかく	
\\	残念ながら試験は不合格でした。	残念[ざんねん]ながら 試験[しけん]は 不合格[ふごうかく]でした。	ざんねん ながら しけん は ふごうかく でした	
\\	残念[ざんねん]ながら 試験[しけん]は
\\	でした。			
\\	マーケット	マーケット	マーケット	
\\	彼はイタリアのマーケットを開拓しました。	彼[かれ]はイタリアのマーケットを 開拓[かいたく]しました。	かれ は いたりあ の まーけっと を かいたく しました	
\\	彼[かれ]はイタリアの
\\	を 開拓[かいたく]しました。			
\\	不通	不通[ふつう]	ふつう	
\\	今朝、停電で電車が不通になったよ。	今朝[けさ]、 停電[ていでん]で 電車[でんしゃ]が 不通[ふつう]になったよ。	けさ ていでん で でんしゃ が ふつう に なった よ	
\\	今朝[けさ]、 停電[ていでん]で 電車[でんしゃ]が
\\	になったよ。			
\\	便り	便[たよ]り	たより	
\\	月に一度母から便りが来ます。	月[つき]に 一度母[いちど はは]から 便[たよ]りが 来[き]ます。	つき に いちど はは から たより が きます	
\\	月[つき]に 一度母[いちど はは]から
\\	が 来[き]ます。			
\\	ファッション	ファッション	ファッション	
\\	彼女はファッションの専門家だ。	彼女[かのじょ]はファッションの 専門家[せんもんか]だ。	かのじょ は ふぁっしょん の せんもんか だ	
\\	彼女[かのじょ]は
\\	の 専門家[せんもんか]だ。			
\\	ブレーキ	ブレーキ	ブレーキ	
\\	赤信号でブレーキを踏んだの。	赤信号[あかしんごう]でブレーキを 踏[ふ]んだの。	あかしんごう で ぶれーき を ふんだ の	
\\	赤信号[あかしんごう]で
\\	を 踏[ふ]んだの。			
\\	ミス	ミス	ミス	
\\	ミスは誰にでもあります。	ミスは 誰[だれ]にでもあります。	みす は だれ に で も あります	
\\	は 誰[だれ]にでもあります。			
\\	ヨット	ヨット	ヨット	
\\	ヨットでクルージングを楽しみましたわ。	ヨットでクルージングを 楽[たの]しみましたわ。	よっと で くるーじんぐ を たのしみました わ	
\\	でクルージングを 楽[たの]しみましたわ。			
\\	なかなか	なかなか	なかなか	
\\	彼女は絵がなかなか上手です。	彼女[かのじょ]は 絵[え]がなかなか 上手[じょうず]です。	かのじょ は え が なかなか じょうず です	
\\	彼女[かのじょ]は 絵[え]が
\\	上手[じょうず]です。			
\\	体重	体重[たいじゅう]	たいじゅう	
\\	私は毎朝体重を測っています。	私[わたし]は 毎朝[まいあさ] 体重[たいじゅう]を 測[はか]っています。	わたし は まいあさ たいじゅう を はかって います	
\\	私[わたし]は 毎朝[まいあさ]
\\	を 測[はか]っています。			
\\	ロック	ロック	ロック	
\\	俺はロックを聞くのが好きだ。	俺[おれ]はロックを 聞[き]くのが 好[す]きだ。	おれ は ろっく を きく の が すき だ	
\\	俺[おれ]は
\\	を 聞[き]くのが 好[す]きだ。			
\\	わざわざ	わざわざ	わざわざ	
\\	わざわざ来てくれてありがとう。	わざわざ 来[き]てくれてありがとう。	わざわざ きて くれて ありがとう	
\\	来[き]てくれてありがとう。			
\\	取り上げる	取[と]り 上[あ]げる	とりあげる	
\\	危ないので子供からライターを取り上げました。	危[あぶ]ないので 子供[こども]からライターを 取[と]り 上[あ]げました。	あぶない の で こども から らいたー を とりあげました	
\\	危[あぶ]ないので 子供[こども]からライターを
\\	取り入れる	取[と]り 入[い]れる	とりいれる	
\\	彼は新しいアイデアをすぐ取り入れます。	彼[かれ]は 新[あたら]しいアイデアをすぐ 取[と]り 入[い]れます。	かれ は あたらしい あいであ を すぐ とりいれます	
\\	彼[かれ]は 新[あたら]しいアイデアをすぐ
\\	取り決め	取[と]り 決[き]め	とりきめ	
\\	これは会社間の取り決めです。	これは 会社間[かいしゃかん]の 取[と]り 決[き]めです。	これ は かいしゃかん の とりきめ です	
\\	これは 会社間[かいしゃかん]の
\\	です。			
\\	取り消す	取[と]り 消[け]す	とりけす	
\\	ホテルの予約を取り消したよ。	ホテルの 予約[よやく]を 取[と]り 消[け]したよ。	ほてる の よやく を とりけした よ	
\\	ホテルの 予約[よやく]を
\\	よ。			
\\	パイロット	パイロット	パイロット	
\\	彼はパイロットです。	彼[かれ]はパイロットです。	かれ は ぱいろっと です	
\\	彼[かれ]は
\\	です。			
\\	取り出す	取[と]り 出[だ]す	とりだす	
\\	彼はポケットから財布を取り出したの。	彼[かれ]はポケットから 財布[さいふ]を 取[と]り 出[だ]したの。	かれ は ぽけっと から さいふ を とりだした の	
\\	彼[かれ]はポケットから 財布[さいふ]を
\\	の。			
\\	どうやら	どうやら	どうやら	
\\	明日はどうやら雨らしいよ。	明日[あした]はどうやら 雨[あめ]らしいよ。	あした は どうやら あめ らしい よ	
\\	明日[あした]は
\\	雨[あめ]らしいよ。			
\\	値打ち	値打[ねう]ち	ねうち	
\\	この絵はとても値打ちがあります。	この 絵[え]はとても 値打[ねう]ちがあります。	この え は とても ねうち が あります	
\\	この 絵[え]はとても
\\	があります。			
\\	ふと	ふと	ふと	
\\	ふと昔の事を思い出したんだ。	ふと 昔[むかし]の 事[こと]を 思[おも]い 出[だ]したんだ。	ふと むかし の こと を おもいだした ん だ	
\\	昔[むかし]の 事[こと]を 思[おも]い 出[だ]したんだ。			
\\	ベテラン	ベテラン	ベテラン	
\\	彼はベテランの運転手です。	彼[かれ]はベテランの 運転手[うんてんしゅ]です。	かれ は べてらん の うんてんしゅ です	
\\	彼[かれ]は
\\	の 運転手[うんてんしゅ]です。			
\\	ボーナス	ボーナス	ボーナス	
\\	今年はボーナスがたくさん出ました。	今年[ことし]はボーナスがたくさん 出[で]ました。	ことし は ぼーなす が たくさん でました 。	
\\	今年[ことし]は
\\	がたくさん 出[で]ました。			
\\	体育	体育[たいいく]	たいいく	
\\	今日は体育の授業があります。	今日[きょう]は 体育[たいいく]の 授業[じゅぎょう]があります。	きょう は たいいく の じゅぎょう が あります	
\\	今日[きょう]は
\\	の 授業[じゅぎょう]があります。			
\\	そっくり	そっくり	そっくり	
\\	あの親子はそっくりだね。	あの 親子[おやこ]はそっくりだね。	あの おやこ は そっくり だ ね	
\\	あの 親子[おやこ]は
\\	だね。			
\\	体制	体制[たいせい]	たいせい	
\\	政治の体制を変える必要があります。	政治[せいじ]の 体制[たいせい]を 変[か]える 必要[ひつよう]があります。	せいじ の たいせい を かえる ひつよう が あります	
\\	政治[せいじ]の
\\	を 変[か]える 必要[ひつよう]があります。			
\\	力強い	力強[ちからづよ]い	ちからづよい	
\\	彼は力強い演技をするね。	彼[かれ]は 力強[ちからづよ]い 演技[えんぎ]をするね。	かれ は ちからづよい えんぎ を する ね	
\\	彼[かれ]は
\\	演技[えんぎ]をするね。			
\\	レンズ	レンズ	レンズ	
\\	眼鏡のレンズを替えました。	眼鏡[めがね]のレンズを 替[か]えました。	めがね の れんず を かえました	
\\	眼鏡[めがね]の
\\	を 替[か]えました。			
\\	ペース	ペース	ペース	
\\	彼はとても早いペースで走っているね。	彼[かれ]はとても 早[はや]いペースで 走[はし]っているね。	かれ は とても はやい ぺーす で はしって いる ね	
\\	彼[かれ]はとても 早[はや]い
\\	で 走[はし]っているね。			
\\	値引き	値引[ねび]き	ねびき	
\\	あと1000円値引きしましょう。	あと1000 円[えん] 値引[ねび]きしましょう。	あと 
\\	えん ねびき しましょう	
\\	あと1000 円[えん]
\\	しましょう。			
\\	マラソン	マラソン	マラソン	
\\	彼はマラソンに出場したよ。	彼[かれ]はマラソンに 出場[しゅつじょう]したよ。	かれ は まらそん に しゅつじょう した よ	
\\	彼[かれ]は
\\	に 出場[しゅつじょう]したよ。			
\\	よほど	よほど	よほど	
\\	今日の遠足はよほど楽しかったらしい。	今日[きょう]の 遠足[えんそく]はよほど 楽[たの]しかったらしい。	きょう の えんそく は よほど たのしかった らしい	
\\	今日[きょう]の 遠足[えんそく]は
\\	楽[たの]しかったらしい。			
\\	入試	入試[にゅうし]	にゅうし	
\\	明日は高校の入試です。	明日[あした]は 高校[こうこう]の 入試[にゅうし]です。	あした は こうこう の にゅうし です	
\\	明日[あした]は 高校[こうこう]の
\\	です。			
\\	体験	体験[たいけん]	たいけん	
\\	今日、珍しい体験をしました。	今日[きょう]、 珍[めずら]しい 体験[たいけん]をしました。	きょう めずらしい たいけん を しました	
\\	今日[きょう]、 珍[めずら]しい
\\	をしました。			
\\	ラグビー	ラグビー	ラグビー	
\\	父は若いころラグビーの選手でした。	父[ちち]は 若[わか]いころラグビーの 選手[せんしゅ]でした。	ちち は わかい ころ らぐびー の せんしゅ でした	
\\	父[ちち]は 若[わか]いころ
\\	の 選手[せんしゅ]でした。			
\\	ラッシュ	ラッシュ	ラッシュ	
\\	私は毎朝ラッシュの電車に乗っています。	私[わたし]は 毎朝[まいあさ]ラッシュの 電車[でんしゃ]に 乗[の]っています。	わたし は まいあさ らっしゅ の でんしゃ に のって います	
\\	私[わたし]は 毎朝[まいあさ]
\\	の 電車[でんしゃ]に 乗[の]っています。			
\\	単語	単語[たんご]	たんご	
\\	日本語の単語をいくつ知っていますか。	日本語[にほんご]の 単語[たんご]をいくつ 知[し]っていますか。	にほんご の たんご を いくつ しって います か	
\\	日本語[にほんご]の
\\	をいくつ 知[し]っていますか。			
\\	ロープ	ロープ	ロープ	
\\	ロープを使って崖を下りたんだ。	ロープを 使[つか]って 崖[がけ]を 下[お]りたんだ。	ろーぷ を つかって がけ を おりた ん だ	
\\	を 使[つか]って 崖[がけ]を 下[お]りたんだ。			
\\	単に	単[たん]に	たんに	
\\	心配しないで、単に眠いだけです。	心配[しんぱい]しないで、 単[たん]に 眠[ねむ]いだけです。	しんぱい しない で たんに ねむい だけ です	
\\	心配[しんぱい]しないで、
\\	眠[ねむ]いだけです。			
\\	ダム	ダム	ダム	
\\	今、ダムの水が少ないね。	今[いま]、ダムの 水[みず]が 少[すく]ないね。	いま だむ の みず が すくない ね	
\\	今[いま]、
\\	の 水[みず]が 少[すく]ないね。			
\\	単数	単数[たんすう]	たんすう	
\\	は単数で
\\	は複数です。	
\\	[ぺん]は 単数[たんすう]で 
\\	[ぺんず]は 複数[ふくすう]です。	ぺんは たんすう で ぺんず は ふくすう です	
\\	[ぺん]は
\\	で 
\\	[ぺんず]は 複数[ふくすう]です。			
\\	ダウン	ダウン	ダウン	
\\	サーバーがダウンしています。	サーバーがダウンしています。	さーばー が だうん して います	
\\	サーバーが
\\	しています。			
\\	人数	人数[にんずう]	にんずう	
\\	参加者の人数を教えてください。	参加者[さんかしゃ]の 人数[にんずう]を 教[おし]えてください。	さんかしゃ の にんずう を おしえて ください	
\\	参加者[さんかしゃ]の
\\	を 教[おし]えてください。			
\\	トレーニング	トレーニング	トレーニング	
\\	彼は毎日トレーニングをするの。	彼[かれ]は 毎日[まいにち]トレーニングをするの。	かれ は まいにち とれーにんぐ を する の	
\\	彼[かれ]は 毎日[まいにち]
\\	をするの。			
\\	分担	分担[ぶんたん]	ぶんたん	
\\	私たち夫婦は家事を分担しています。	私[わたし]たち 夫婦[ふうふ]は 家事[かじ]を 分担[ぶんたん]しています。	わたしたち ふうふ は かじ を ぶんたん して います	
\\	私[わたし]たち 夫婦[ふうふ]は 家事[かじ]を
\\	しています。			
\\	そっと	そっと	そっと	
\\	母親は娘の髪をそっとなでたの。	母親[ははおや]は 娘[むすめ]の 髪[かみ]をそっとなでたの。	ははおや は むすめ の かみ を そっと なでた の	
\\	母親[ははおや]は 娘[むすめ]の 髪[かみ]を
\\	なでたの。			
\\	マーク	マーク	マーク	
\\	このブランドのマークは可愛いですね。	このブランドのマークは 可愛[かわい]いですね。	この ぶらんど の まーく は かわいい です ね	
\\	このブランドの
\\	は 可愛[かわい]いですね。			
\\	不規則	不規則[ふきそく]	ふきそく	
\\	最近、不規則な生活をしている。	最近[さいきん]、 不規則[ふきそく]な 生活[せいかつ]をしている。	さいきん ふきそく な せいかつ を して いる	
\\	最近[さいきん]、
\\	な 生活[せいかつ]をしている。			
\\	たった	たった	たった	
\\	財布の中にたった1000円しかないよ。	財布[さいふ]の 中[なか]にたった 1000円[せんえん]しかないよ。	さいふ の なか に たった せんえん しか ない よ	
\\	財布[さいふ]の 中[なか]に
\\	1000円[せんえん]しかないよ。			
\\	不利	不利[ふり]	ふり	
\\	彼は今、不利な立場にいます。	彼[かれ]は 今[いま]、 不利[ふり]な 立場[たちば]にいます。	かれ は いま ふり な たちば に います	
\\	彼[かれ]は 今[いま]、
\\	な 立場[たちば]にいます。			
\\	利口	利口[りこう]	りこう	
\\	あの犬はとても利口ですね。	あの 犬[いぬ]はとても 利口[りこう]ですね。	あの いぬ は とても りこう です ね	
\\	あの 犬[いぬ]はとても
\\	ですね。			
\\	利子	利子[りし]	りし	
\\	借金に利子をつけて返したの。	借金[しゃっきん]に 利子[りし]をつけて 返[かえ]したの。	しゃっきん に りし を つけて かえした の	
\\	借金[しゃっきん]に
\\	をつけて 返[かえ]したの。			
\\	たまたま	たまたま	たまたま	
\\	道でたまたま友達に会った。	道[みち]でたまたま 友達[ともだち]に 会[あ]った。	みち で たまたま ともだち に あった	
\\	道[みち]で
\\	友達[ともだち]に 会[あ]った。			
\\	利益	利益[りえき]	りえき	
\\	先月の利益は200万円でした。	先月[せんげつ]の 利益[りえき]は200 万円[まんえん]でした。	せんげつ の りえき は 
\\	まんえん でした	
\\	先月[せんげつ]の
\\	は200 万円[まんえん]でした。			
\\	テンポ	テンポ	テンポ	
\\	この曲はテンポが速いですね。	この 曲[きょく]はテンポが 速[はや]いですね。	この きょく は てんぽ が はやい です ね	
\\	この 曲[きょく]は
\\	が 速[はや]いですね。			
\\	不要	不要[ふよう]	ふよう	
\\	不要になったパソコンを処分したんだ。	不要[ふよう]になったパソコンを 処分[しょぶん]したんだ。	ふよう に なった ぱそこん を しょぶん した ん だ	
\\	になったパソコンを 処分[しょぶん]したんだ。			
\\	不必要	不必要[ふひつよう]	ふひつよう	
\\	不必要なファイルは削除してください。	不必要[ふひつよう]なファイルは 削除[さくじょ]してください。	ふひつよう な ふぁいる は さくじょ して ください	
\\	なファイルは 削除[さくじょ]してください。			
\\	プラン	プラン	プラン	
\\	彼女と旅行のプランを考えました。	彼女[かのじょ]と 旅行[りょこう]のプランを 考[かんが]えました。	かのじょ と りょこう の ぷらん を かんがえました	
\\	彼女[かのじょ]と 旅行[りょこう]の
\\	を 考[かんが]えました。			
\\	ほっと	ほっと	ほっと	
\\	家に着いてほっとしたよ。	家[いえ]に 着[つ]いてほっとしたよ。	いえ に ついて ほっと した よ	
\\	家[いえ]に 着[つ]いて
\\	したよ。			
\\	体温計	体温計[たいおんけい]	たいおんけい	
\\	体温計が壊れてしまった。	体温計[たいおんけい]が 壊[こわ]れてしまった。	たいおんけい が こわれて しまった	
\\	が 壊[こわ]れてしまった。			
\\	割合	割合[わりあい]	わりあい	
\\	二つの薬品を1対3の割合で混ぜたんだ。	二[ふた]つの 薬品[やくひん]を1 対3[たい 
\\	の 割合[わりあい]で 混[ま]ぜたんだ。	ふたつ の やくひん を 
\\	たい 
\\	の わりあい で まぜた ん だ	
\\	二[ふた]つの 薬品[やくひん]を1 対3[たい 
\\	の
\\	で 混[ま]ぜたんだ。			
\\	なるほど	なるほど	なるほど	
\\	なるほど、よく分かりました。	なるほど、よく 分[わ]かりました。	なるほど よく わかりました	
\\	、よく 分[わ]かりました。			
\\	割り当て	割[わ]り 当[あ]て	わりあて	
\\	チケットの割り当ては1人10枚です。	チケットの 割[わ]り 当[あ]ては 1人10枚[ひとり じゅうまい]です。	ちけっと の わりあて は ひとり じゅうまい です	
\\	チケットの
\\	は 1人10枚[ひとり じゅうまい]です。			
\\	割り当てる	割[わ]り 当[あ]てる	わりあてる	
\\	全員に作業が割り当てられました。	全員[ぜんいん]に 作業[さぎょう]が 割[わ]り 当[あ]てられました。	ぜんいん に さぎょう が わりあてられました	
\\	全員[ぜんいん]に 作業[さぎょう]が
\\	割り引く	割[わ]り 引[び]く	わりびく	
\\	定価から2000円割り引きますよ。	定価[ていか]から2 000円[せんえん] 割[わ]り 引[び]きますよ。	ていか から 
\\	せんえん わりびきます よ	
\\	定価[ていか]から2 000円[せんえん]
\\	よ。			
\\	割に	割[わり]に	わりに	
\\	この映画は割に面白いわ。	この 映画[えいが]は 割[わり]に 面白[おもしろ]いわ。	この えいが は わりに おもしろい わ	
\\	この 映画[えいが]は
\\	面白[おもしろ]いわ。			
\\	割り	割[わ]り	わり	
\\	3日に1度の割りで彼からメールが来るの。	3日[みっか]に1 度[ど]の 割[わ]りで 彼[かれ]からメールが 来[く]るの。	みっか に 
\\	ど の わり で かれ から めーる が くる の	
\\	3日[みっか]に1 度[ど]の
\\	で 彼[かれ]からメールが 来[く]るの。			
\\	割り引き	割[わ]り 引[び]き	わりびき	
\\	今日は全品10
\\	割り引きです。	今日[きょう]は 全品10[ぜんぴん 
\\	割[わ]り 引[び]きです。	きょう は ぜんぴん 
\\	わりびき です	
\\	今日[きょう]は 全品10[ぜんぴん 
\\	です。			
\\	バイト	バイト	バイト	
\\	今日は6時からバイトです。	今日[きょう]は6 時[じ]からバイトです。	きょう は 
\\	じ から ばいと です	
\\	今日[きょう]は6 時[じ]から
\\	です。			
\\	割合に	割合[わりあい]に	わりあいに	
\\	今回のテストは割合に簡単でした。	今回[こんかい]のテストは 割合[わりあい]に 簡単[かんたん]でした。	こんかい の てすと は わりあいに かんたん でした	
\\	今回[こんかい]のテストは
\\	簡単[かんたん]でした。			
\\	ぶつける	ぶつける	ぶつける	
\\	車を壁にぶつけてしまいました。	車[くるま]を 壁[かべ]にぶつけてしまいました。	くるま を かべ に ぶつけて しまいました	
\\	車[くるま]を 壁[かべ]に
\\	取り返す	取[と]り 返[かえ]す	とりかえす	
\\	彼はチャンピオンのタイトルを取り返したね。	彼[かれ]はチャンピオンのタイトルを 取[と]り 返[かえ]したね。	かれ は ちゃんぴおん の たいとる を とりかえした ね	
\\	彼[かれ]はチャンピオンのタイトルを
\\	ね。			
\\	タイヤ	タイヤ	タイヤ	
\\	車のタイヤを換えた。	車[くるま]のタイヤを 換[か]えた。	くるま の たいや を かえた	
\\	車[くるま]の
\\	を 換[か]えた。			
\\	割り込む	割[わ]り 込[こ]む	わりこむ	
\\	車が前に割り込んできた。	車[くるま]が 前[まえ]に 割[わ]り 込[こ]んできた。	くるま が まえ に わりこんで きた	
\\	車[くるま]が 前[まえ]に
\\	マイク	マイク	マイク	
\\	彼女はマイクを持って話し始めたの。	彼女[かのじょ]はマイクを 持[も]って 話[はな]し 始[はじ]めたの。	かのじょ は まいく を もって はなしはじめた の	
\\	彼女[かのじょ]は
\\	を 持[も]って 話[はな]し 始[はじ]めたの。			
\\	人込み	人込[ひとご]み	ひとごみ	
\\	私は人込みが好きではありません。	私[わたし]は 人込[ひとご]みが 好[す]きではありません。	わたし は ひとごみ が すき で は ありません	
\\	私[わたし]は
\\	が 好[す]きではありません。			
\\	ゆとり	ゆとり	ゆとり	
\\	最近、生活にゆとりがでてきました。	最近[さいきん]、 生活[せいかつ]にゆとりがでてきました。	さいきん せいかつ に ゆとり が でて きました	
\\	最近[さいきん]、 生活[せいかつ]に
\\	がでてきました。			
\\	値切る	値切[ねぎ]る	ねぎる	
\\	彼は値切るのが上手です。	彼[かれ]は 値切[ねぎ]るのが 上手[じょうず]です。	かれ は ねぎる の が じょうず です	
\\	彼[かれ]は
\\	のが 上手[じょうず]です。			
\\	前売り券	前売[まえう]り 券[けん]	まえうりけん	
\\	コンサートの前売り券を手に入れたんだ。	コンサートの 前売[まえう]り 券[けん]を 手[て]に 入[い]れたんだ。	こんさーと の まえうりけん を て に いれた ん だ	
\\	コンサートの
\\	を 手[て]に 入[い]れたんだ。			
\\	ぼんやり	ぼんやり	ぼんやり	
\\	彼は遠くをぼんやり見ていたの。	彼[かれ]は 遠[とお]くをぼんやり 見[み]ていたの。	かれ は とおく を ぼんやり みて いた の	
\\	彼[かれ]は 遠[とお]くを
\\	見[み]ていたの。			
\\	代金	代金[だいきん]	だいきん	
\\	ここで代金をお支払いください。	ここで 代金[だいきん]をお 支払[しはら]いください。	ここ で だいきん を お しはらい ください	
\\	ここで
\\	をお 支払[しはら]いください。			
\\	人差し指	人差[ひとさ]し 指[ゆび]	ひとさしゆび	
\\	彼女は人差し指を怪我したの。	彼女[かのじょ]は 人差[ひとさ]し 指[ゆび]を 怪我[けが]したの。	かのじょ は ひとさしゆび を けが した の 。	
\\	彼女[かのじょ]は
\\	を 怪我[けが]したの。			
\\	中指	中指[なかゆび]	なかゆび	
\\	中指をドアに挟んでしまった。	中指[なかゆび]をドアに 挟[はさ]んでしまった。	なかゆび を どあ に はさんで しまった	
\\	をドアに 挟[はさ]んでしまった。			
\\	たまらない	たまらない	たまらない	
\\	頭が痛くてたまらない。	頭[あたま]が 痛[いた]くてたまらない。	あたま が いたくて たまらない	
\\	頭[あたま]が 痛[いた]くて
\\	不安定	不安定[ふあんてい]	ふあんてい	
\\	最近、体調が少し不安定です。	最近[さいきん]、 体調[たいちょう]が 少[すこ]し 不安定[ふあんてい]です。	さいきん たいちょう が すこし ふあんてい です	
\\	最近[さいきん]、 体調[たいちょう]が 少[すこ]し
\\	です。			
\\	モーター	モーター	モーター	
\\	車のモーターを修理したぜ!	車[くるま]のモーターを 修理[しゅうり]したぜ!	くるま の もーたー を しゅうり した ぜ	
\\	車[くるま]の
\\	を 修理[しゅうり]したぜ!			
\\	予算	予算[よさん]	よさん	
\\	車の費用、予算オーバーだ。	車[くるま]の 費用[ひよう]、 予算[よさん]オーバーだ。	くるま の ひよう、 よさん おーばー だ	
\\	車[くるま]の 費用[ひよう]、
\\	オーバーだ。			
\\	予報	予報[よほう]	よほう	
\\	予報では明日は雨ですね。	予報[よほう]では 明日[あした]は 雨[あめ]ですね。	よほう で は あした は あめ です ね	
\\	では 明日[あした]は 雨[あめ]ですね。			
\\	予言	予言[よげん]	よげん	
\\	彼の予言は当たったことがないね。	彼[かれ]の 予言[よげん]は 当[あ]たったことがないね。	かれ の よげん は あたった こと が ない ね	
\\	彼[かれ]の
\\	は 当[あ]たったことがないね。			
\\	まして	まして	まして	
\\	他人でも悲しいのだから、まして本人はどれほどでしょう。	他人[たにん]でも 悲[かな]しいのだから、まして 本人[ほんにん]はどれほどでしょう。	たにん で も かなしい の だ から まして ほんにん は どれほど でしょう	
\\	他人[たにん]でも 悲[かな]しいのだから、
\\	本人[ほんにん]はどれほどでしょう。			
\\	ピストル	ピストル	ピストル	
\\	犯人はピストルを持っているわ。	犯人[はんにん]はピストルを 持[も]っているわ。	はんにん は ぴすとる を もって いる わ	
\\	犯人[はんにん]は
\\	を 持[も]っているわ。			
\\	乗り出す	乗[の]り 出[だ]す	のりだす	
\\	船が長い航海に乗り出したの。	船[ふね]が 長[なが]い 航海[こうかい]に 乗[の]り 出[だ]したの。	ふね が ながい こうかい に のりだした の	
\\	船[ふね]が 長[なが]い 航海[こうかい]に
\\	の。			
\\	乗り込む	乗[の]り 込[こ]む	のりこむ	
\\	あの駅で学生がたくさん乗り込んだね。	あの 駅[えき]で 学生[がくせい]がたくさん 乗[の]り 込[こ]んだね。	あの えき で がくせい が たくさん のりこんだ ね	
\\	あの 駅[えき]で 学生[がくせい]がたくさん
\\	ね。			
\\	乗り降り	乗[の]り 降[お]り	のりおり	
\\	この駅でたくさんの人が乗り降りしますね。	この 駅[えき]でたくさんの 人[ひと]が 乗[の]り 降[お]りしますね。	この えき で たくさん の ひと が のりおり します ね	
\\	この 駅[えき]でたくさんの 人[ひと]が
\\	しますね。			
\\	たっぷり	たっぷり	たっぷり	
\\	たっぷりとマッサージしてもらいました。	たっぷりとマッサージしてもらいました。	たっぷり と まっさーじ して もらいました	
\\	とマッサージしてもらいました。			
\\	中立	中立[ちゅうりつ]	ちゅうりつ	
\\	私は中立の立場を取っています。	私[わたし]は 中立[ちゅうりつ]の 立場[たちば]を 取[と]っています。	わたし は ちゅうりつ の たちば を とって います	
\\	私[わたし]は
\\	の 立場[たちば]を 取[と]っています。			
\\	つくづく	つくづく	つくづく	
\\	来てよかったとつくづく思います。	来[き]てよかったとつくづく 思[おも]います。	きて よかったと つくづく おもいます	
\\	来[き]てよかったと
\\	思[おも]います。			
\\	どうせ	どうせ	どうせ	
\\	どうせ間に合わないならゆっくり行こう。	どうせ 間[ま]に 合[あ]わないならゆっくり 行[い]こう。	どうせ まにあわない なら ゆっくり いこう	
\\	間[ま]に 合[あ]わないならゆっくり 行[い]こう。			
\\	取り次ぐ	取[と]り 次[つ]ぐ	とりつぐ	
\\	電話があったら取り次いでください。	電話[でんわ]があったら 取[と]り 次[つ]いでください。	でんわ が あったら とりついで ください	
\\	電話[でんわ]があったら
\\	ください。			
\\	不運	不運[ふうん]	ふうん	
\\	彼に不運な出来事が起こったの。	彼[かれ]に 不運[ふうん]な 出来事[できごと]が 起[お]こったの。	かれ に ふうん な できごと が おこった の	
\\	彼[かれ]に
\\	な 出来事[できごと]が 起[お]こったの。			
\\	動向	動向[どうこう]	どうこう	
\\	今、経済の動向は読みにくいわ。	今[いま]、 経済[けいざい]の 動向[どうこう]は 読[よ]みにくいわ。	いま けいざい の どうこう は よみ にくい わ	
\\	今[いま]、 経済[けいざい]の
\\	は 読[よ]みにくいわ。			
\\	動作	動作[どうさ]	どうさ	
\\	彼は動作が機敏です。	彼[かれ]は 動作[どうさ]が 機敏[きびん]です。	かれ は どうさ が きびん です	
\\	彼[かれ]は
\\	が 機敏[きびん]です。			
\\	リズム	リズム	リズム	
\\	最近彼は生活のリズムが乱れています。	最近彼[さいきん かれ]は 生活[せいかつ]のリズムが 乱[みだ]れています。	さいきん かれ は せいかつ の りずむ が みだれて います	
\\	最近彼[さいきん かれ]は 生活[せいかつ]の
\\	が 乱[みだ]れています。			
\\	動力	動力[どうりょく]	どうりょく	
\\	この車の動力は電気です。	この 車[くるま]の 動力[どうりょく]は 電気[でんき]です。	この くるま の どうりょく は でんき です	
\\	この 車[くるま]の
\\	は 電気[でんき]です。			
\\	働き	働[はたら]き	はたらき	
\\	部下が素晴らしい働きをしたな。	部下[ぶか]が 素晴[すば]らしい 働[はたら]きをしたな。	ぶか が すばらしい はたらき を した な	
\\	部下[ぶか]が 素晴[すば]らしい
\\	をしたな。			
\\	ハンドル	ハンドル	ハンドル	
\\	この車は左ハンドルです。	この 車[くるま]は 左[ひだり]ハンドルです。	この くるま は ひだり はんどる です	
\\	この 車[くるま]は 左[ひだり]
\\	です。			
\\	乗り遅れる	乗[の]り 遅[おく]れる	のりおくれる	
\\	寝坊して新幹線に乗り遅れたよ。	寝坊[ねぼう]して 新幹線[しんかんせん]に 乗[の]り 遅[おく]れたよ。	ねぼう して しんかんせん に のりおくれた よ	
\\	寝坊[ねぼう]して 新幹線[しんかんせん]に
\\	よ。			
\\	プリント	プリント	プリント	
\\	デジカメで撮った写真をプリントしたんだ。	デジカメで 撮[と]った 写真[しゃしん]をプリントしたんだ。	でじかめ で とった しゃしん を ぷりんと した ん だ	
\\	デジカメで 撮[と]った 写真[しゃしん]を
\\	したんだ。			
\\	不在	不在[ふざい]	ふざい	
\\	妻の不在中、夫は毎日外食したんだ。	妻[つま]の 不在[ふざい] 中[ちゅう]、 夫[おっと]は 毎日外食[まいにち がいしょく]したんだ。	つま の ふざいちゅう おっと は まいにち がいしょく した ん だ	
\\	妻[つま]の
\\	中[ちゅう]、 夫[おっと]は 毎日外食[まいにち がいしょく]したんだ。			
\\	まぶしい	まぶしい	まぶしい	
\\	夏の日差しがまぶしかったよ。	夏[なつ]の 日差[ひざ]しがまぶしかったよ。	なつ の ひざし が まぶしかった よ	
\\	夏[なつ]の 日差[ひざ]しが
\\	よ。			
\\	よそ	よそ	よそ	
\\	夕食はよそでごちそうになりました。	夕食[ゆうしょく]はよそでごちそうになりました。	ゆうしょく は よそ で ごちそう に なりました	
\\	夕食[ゆうしょく]は
\\	でごちそうになりました。			
\\	つぐ	つぐ	つぐ	
\\	お酒をおつぎしましょう。	お 酒[さけ]をおつぎしましょう。	おさけ を お つぎ しましょう	
\\	お 酒[さけ]をお
\\	ディスコ	ディスコ	ディスコ	
\\	昔はよくディスコに行ったな。	昔[むかし]はよくディスコに 行[い]ったな。	むかし は よく でぃすこ に いった な	
\\	昔[むかし]はよく
\\	に 行[い]ったな。			
\\	ぴったり	ぴったり	ぴったり	
\\	この服はあなたにぴったりですね。	この 服[ふく]はあなたにぴったりですね。	この ふく は あなた に ぴったり です ね	
\\	この 服[ふく]はあなたに
\\	ですね。			
\\	半面	半面[はんめん]	はんめん	
\\	テニスコートの半面を使って試合をしたの。	テニスコートの 半面[はんめん]を 使[つか]って 試合[しあい]をしたの。	てにすこーと の はんめん を つかって しあい を した の	
\\	テニスコートの
\\	を 使[つか]って 試合[しあい]をしたの。			
\\	まさか	まさか	まさか	
\\	まさか車が当たるとは思わなかった。	まさか 車[くるま]が 当[あ]たるとは 思[おも]わなかった。	まさか くるま が あたる と は おもわなかった	
\\	車[くるま]が 当[あ]たるとは 思[おも]わなかった。			
\\	どっと	どっと	どっと	
\\	観衆がどっと笑ったんだ。	観衆[かんしゅう]がどっと 笑[わら]ったんだ。	かんしゅう が どっと わらった ん だ	
\\	観衆[かんしゅう]が
\\	笑[わら]ったんだ。			
\\	ばら色	ばら 色[いろ]	ばらいろ	
\\	彼女の人生はばら色だったの。	彼女[かのじょ]の 人生[じんせい]はばら 色[いろ]だったの。	かのじょ の じんせい は ばらいろ だった の 。	
\\	彼女[かのじょ]の 人生[じんせい]は
\\	だったの。			
\\	もうける	もうける	もうける	
\\	先月は株で100万円もうけたの。	先月[せんげつ]は 株[かぶ]で100 万円[まん えん]もうけたの。	せんげつ は かぶ で 
\\	まん えん もうけた の	
\\	先月[せんげつ]は 株[かぶ]で100 万円[まん えん]
\\	の。			
\\	分類	分類[ぶんるい]	ぶんるい	
\\	これらの本を分類してください。	これらの 本[ほん]を 分類[ぶんるい]してください。	これら の ほん を ぶんるい して ください	
\\	これらの 本[ほん]を
\\	してください。			
\\	ゆったり	ゆったり	ゆったり	
\\	彼女はゆったりした服を着ていますね。	彼女[かのじょ]はゆったりした 服[ふく]を 着[き]ていますね。	かのじょ は ゆったり した ふく を きて います ね	
\\	彼女[かのじょ]は
\\	した 服[ふく]を 着[き]ていますね。			
\\	やり直し	やり 直[なお]し	やりなおし	
\\	仕事がやり直しになった。	仕事[しごと]がやり 直[なお]しになった。	しごと が やりなおし に なった	
\\	仕事[しごと]が
\\	になった。			
\\	やり直す	やり 直[なお]す	やりなおす	
\\	急いでやり直します。	急[いそ]いでやり 直[なお]します。	いそい で やりなおします。	
\\	急[いそ]いで
\\	ちっとも	ちっとも	ちっとも	
\\	彼女が結婚したなんて、ちっとも知らなかった。	彼女[かのじょ]が 結婚[けっこん]したなんて、ちっとも 知[し]らなかった。	かのじょ が けっこん した なんて ちっとも しらなかった	
\\	彼女[かのじょ]が 結婚[けっこん]したなんて、
\\	知[し]らなかった。			
\\	ドレス	ドレス	ドレス	
\\	昨日素敵なドレスを買いました。	昨日素敵[きのう すてき]なドレスを 買[か]いました。	きのう すてき な どれす を かいました	
\\	昨日素敵[きのう すてき]な
\\	を 買[か]いました。			
\\	パンク	パンク	パンク	
\\	自転車のタイヤがパンクしました。	自転車[じてんしゃ]のタイヤがパンクしました。	じてんしゃ の たいや が ぱんく しました	
\\	自転車[じてんしゃ]のタイヤが
\\	しました。			
\\	共に	共[とも]に	ともに	
\\	最後まで共にがんばりましょう。	最後[さいご]まで 共[とも]にがんばりましょう。	さいご まで ともに がんばりましょう	
\\	最後[さいご]まで
\\	がんばりましょう。			
\\	のんびり	のんびり	のんびり	
\\	休みの日は家でのんびり過ごします。	休[やす]みの 日[ひ]は 家[うち]でのんびり 過[す]ごします。	やすみ の ひ は うち で のんびり すごします	
\\	休[やす]みの 日[ひ]は 家[うち]で
\\	過[す]ごします。			
\\	人違い	人違[ひとちが]い	ひとちがい	
\\	すみません、人違いでした。	すみません、 人違[ひとちが]いでした。	すみません ひとちがい でした	
\\	すみません、
\\	でした。			
\\	似る	似[に]る	にる	
\\	女の子は話し方まで母親に似るわね。	女[おんな]の 子[こ]は 話[はな]し 方[かた]まで 母親[ははおや]に 似[に]るわね。	おんな の こ は はなしかた まで ははおや に にる わ ね	
\\	女[おんな]の 子[こ]は 話[はな]し 方[かた]まで 母親[ははおや]に
\\	わね。			
\\	はめる	はめる	はめる	
\\	彼女は指輪をたくさんはめていますね。	彼女[かのじょ]は 指輪[ゆびわ]をたくさんはめていますね。	かのじょ は ゆびわ を たくさん はめて います ね	
\\	彼女[かのじょ]は 指輪[ゆびわ]をたくさん
\\	いますね。			
\\	似合う	似合[にあ]う	にあう	
\\	彼女は着物がよく似合います。	彼女[かのじょ]は 着物[きもの]がよく 似合[にあ]います。	かのじょ は きもの が よく にあいます	
\\	彼女[かのじょ]は 着物[きもの]がよく
\\	似せる	似[に]せる	にせる	
\\	彼の字に似せて書きました。	彼[かれ]の 字[じ]に 似[に]せて 書[か]きました。	かれ の じ に にせて かきました	
\\	彼[かれ]の 字[じ]に
\\	書[か]きました。			
\\	ふた	ふた	ふた	
\\	箱にふたをしたよ。	箱[はこ]にふたをしたよ。	はこ に ふた を した よ	
\\	箱[はこ]に
\\	をしたよ。			
\\	レポート	レポート	レポート	
\\	教授にレポートを提出しました。	教授[きょうじゅ]にレポートを 提出[ていしゅつ]しました。	きょうじゅ に れぽーと を ていしゅつ しました	
\\	教授[きょうじゅ]に
\\	を 提出[ていしゅつ]しました。			
\\	マネー	マネー	マネー	
\\	インターネットでマネー情報をチェックしたよ。	インターネットでマネー 情報[じょうほう]をチェックしたよ。	いんたーねっと で まねー じょうほう を ちぇっく した よ	
\\	インターネットで
\\	情報[じょうほう]をチェックしたよ。			
\\	もろい	もろい	もろい	
\\	この岩は意外ともろい。	この 岩[いわ]は 意外[いがい]ともろい。	この いわ は いがい と もろい	
\\	この 岩[いわ]は 意外[いがい]と
\\	ユーモア	ユーモア	ユーモア	
\\	彼女はユーモアのセンスがあります。	彼女[かのじょ]はユーモアのセンスがあります。	かのじょ は ゆーもあ の せんす が あります 。	
\\	彼女[かのじょ]は
\\	のセンスがあります。			
\\	利息	利息[りそく]	りそく	
\\	預金には利息がつきます。	預金[よきん]には 利息[りそく]がつきます。	よきん に は りそく が つきます	
\\	預金[よきん]には
\\	がつきます。			
\\	バック	バック	バック	
\\	車のギアをバックに入れたよ。	車[くるま]のギアをバックに 入[い]れたよ。	くるま の ぎあ を ばっく に いれた よ	
\\	車[くるま]のギアを
\\	に 入[い]れたよ。			
\\	そびえる	そびえる	そびえる	
\\	いくつもの高層ビルがそびえていたんだ。	いくつもの 高層[こうそう]ビルがそびえていたんだ。	いくつ も の こうそう びる が そびえていた ん だ 。	
\\	いくつもの 高層[こうそう]ビルが
\\	んだ。			
\\	タレント	タレント	タレント	
\\	彼はタレントなのでよくテレビに出ています。	彼[かれ]はタレントなのでよくテレビに 出[で]ています。	かれ は たれんと な の で よく てれび に でて います	
\\	彼[かれ]は
\\	なのでよくテレビに 出[で]ています。			
\\	不自然	不自然[ふしぜん]	ふしぜん	
\\	彼の態度はどこか不自然だったな。	彼[かれ]の 態度[たいど]はどこか 不自然[ふしぜん]だったな。	かれ の たいど は どこか ふしぜん だった な	
\\	彼[かれ]の 態度[たいど]はどこか
\\	だったな。			
\\	不自由	不自由[ふじゆう]	ふじゆう	
\\	彼は不自由な暮らしに慣れたようだね。	彼[かれ]は 不自由[ふじゆう]な 暮[く]らしに 慣[な]れたようだね。	かれ は ふじゆう な くらし に なれた よう だ ね	
\\	彼[かれ]は
\\	な 暮[く]らしに 慣[な]れたようだね。			
\\	たまに	たまに	たまに	
\\	彼はたまに料理をします。	彼[かれ]はたまに 料理[りょうり]をします。	かれ は たまに りょうり を します	
\\	彼[かれ]は
\\	料理[りょうり]をします。			
\\	ペンキ	ペンキ	ペンキ	
\\	床にペンキをこぼしてしまいました。	床[ゆか]にペンキをこぼしてしまいました。	ゆか に ぺんき を こぼして しまいました	
\\	床[ゆか]に
\\	をこぼしてしまいました。			
\\	不思議	不思議[ふしぎ]	ふしぎ	
\\	それはとても不思議な話だね。	それはとても 不思議[ふしぎ]な 話[はなし]だね。	それ は とても ふしぎ な はなし だ ね	
\\	それはとても
\\	な 話[はなし]だね。			
\\	ベンチ	ベンチ	ベンチ	
\\	公園のベンチで本を読んでいます。	公園[こうえん]のベンチで 本[ほん]を 読[よ]んでいます。	こうえん の べんち で ほん を よんで います	
\\	公園[こうえん]の
\\	で 本[ほん]を 読[よ]んでいます。			
\\	世論	世論[よろん]	よろん	
\\	世論は新しい法律に反対です。	世論[よろん]は 新[あたら]しい 法律[ほうりつ]に 反対[はんたい]です。	よろん は あたらしい ほうりつ に はんたい です	
\\	は 新[あたら]しい 法律[ほうりつ]に 反対[はんたい]です。			
\\	めいめい	めいめい	めいめい	
\\	交通費はめいめい支払ってください。	交通費[こうつうひ]はめいめい 支払[しはら]ってください。	こうつうひ は めいめい しはらって ください	
\\	交通費[こうつうひ]は
\\	支払[しはら]ってください。			
\\	反発	反発[はんぱつ]	はんぱつ	
\\	彼は先生に反発していたよ。	彼[かれ]は 先生[せんせい]に 反発[はんぱつ]していたよ。	かれ は せんせい に はんぱつ して いた よ	
\\	彼[かれ]は 先生[せんせい]に
\\	していたよ。			
\\	反映	反映[はんえい]	はんえい	
\\	新型車に研究の成果が反映されています。	新型車[しんがたしゃ]に 研究[けんきゅう]の 成果[せいか]が 反映[はんえい]されています。	しんがたしゃ に けんきゅう の せいか が はんえい されて います	
\\	新型車[しんがたしゃ]に 研究[けんきゅう]の 成果[せいか]が
\\	されています。			
\\	反論	反論[はんろん]	はんろん	
\\	反論がある人はどうぞ。	反論[はんろん]がある 人[ひと]はどうぞ。	はんろん が ある ひと は どうぞ	
\\	がある 人[ひと]はどうぞ。			
\\	反する	反[はん]する	はんする	
\\	彼の行いはモラルに反しています。	彼[かれ]の 行[おこな]いはモラルに 反[はん]しています。	かれ の おこない は もらる に はんして います	
\\	彼[かれ]の 行[おこな]いはモラルに
\\	います。			
\\	どうにか	どうにか	どうにか	
\\	どうにか締め切りに間に合った。	どうにか 締[し]め 切[き]りに 間[ま]に 合[あ]った。	どうにか しめきり に まにあった	
\\	締[し]め 切[き]りに 間[ま]に 合[あ]った。			
\\	反則	反則[はんそく]	はんそく	
\\	その選手は反則で負けました。	その 選手[せんしゅ]は 反則[はんそく]で 負[ま]けました。	その せんしゅ は はんそく で まけました	
\\	その 選手[せんしゅ]は
\\	で 負[ま]けました。			
\\	反応	反応[はんのう]	はんのう	
\\	彼女は良い反応を示したわ。	彼女[かのじょ]は 良[よ]い 反応[はんのう]を 示[しめ]したわ。	かのじょ は よい はんのう を しめした わ	
\\	彼女[かのじょ]は 良[よ]い
\\	を 示[しめ]したわ。			
\\	ちゃんと	ちゃんと	ちゃんと	
\\	朝食はちゃんと食べましたか。	朝食[ちょうしょく]はちゃんと 食[た]べましたか。	ちょうしょく は ちゃんと たべました か	
\\	朝食[ちょうしょく]は
\\	食[た]べましたか。			
\\	別に	別[べつ]に	べつに	
\\	私は別に気になりません。	私[わたし]は 別[べつ]に 気[き]になりません。	わたし は べつに き に なりません	
\\	私[わたし]は
\\	気[き]になりません。			
\\	別れ	別[わか]れ	わかれ	
\\	別れはいつでも悲しいものです。	別[わか]れはいつでも 悲[かな]しいものです。	わかれ は いつでも かなしい もの です	
\\	はいつでも 悲[かな]しいものです。			
\\	のどか	のどか	のどか	
\\	私の田舎はのどかな所です。	私[わたし]の 田舎[いなか]はのどかな 所[ところ]です。	わたし の いなか は のどか な ところ です	
\\	私[わたし]の 田舎[いなか]は
\\	な 所[ところ]です。			
\\	メロディー	メロディー	メロディー	
\\	このメロディーは聞いたことがある。	このメロディーは 聞[き]いたことがある。	この めろでぃー は きいた こと が ある	
\\	この
\\	は 聞[き]いたことがある。			
\\	全般	全般[ぜんぱん]	ぜんぱん	
\\	彼は植物全般に詳しいね。	彼[かれ]は 植物[しょくぶつ] 全般[ぜんぱん]に 詳[くわ]しいね。	かれ は しょくぶつ ぜんぱん に くわしい ね	
\\	彼[かれ]は 植物[しょくぶつ]
\\	に 詳[くわ]しいね。			
\\	全面的	全面的[ぜんめんてき]	ぜんめんてき	
\\	彼が全面的に協力してくれるそうです。	彼[かれ]が 全面的[ぜんめんてき]に 協力[きょうりょく]してくれるそうです。	かれ が ぜんめんてき に きょうりょく して くれる そう です	
\\	彼[かれ]が
\\	に 協力[きょうりょく]してくれるそうです。			
\\	リットル	リットル	リットル	
\\	今日は水を2リットル以上飲んだわ。	今日[きょう]は 水[みず]を2リットル 以上飲[いじょう の]んだわ。	きょう は みず を 
\\	りっとる いじょう のんだ わ	
\\	今日[きょう]は 水[みず]を2
\\	以上飲[いじょう の]んだわ。			
\\	並み	並[な]み	なみ	
\\	彼は並みの人間ではありません。	彼[かれ]は 並[な]みの 人間[にんげん]ではありません。	かれ は なみ の にんげん で は ありません	
\\	彼[かれ]は
\\	の 人間[にんげん]ではありません。			
\\	並木	並木[なみき]	なみき	
\\	駅前の並木が台風で倒れたらしい。	駅前[えきまえ]の 並木[なみき]が 台風[たいふう]で 倒[たお]れたらしい。	えきまえ の なみき が たいふう で たおれた らしい	
\\	駅前[えきまえ]の
\\	が 台風[たいふう]で 倒[たお]れたらしい。			
\\	わざと	わざと	わざと	
\\	彼はわざと負けたように見えたな。	彼[かれ]はわざと 負[ま]けたように 見[み]えたな。	かれ は わざと まけた よう に みえた な	
\\	彼[かれ]は
\\	負[ま]けたように 見[み]えたな。			
\\	不平	不平[ふへい]	ふへい	
\\	私はいつも同僚の不平を聞いているんだ。	私[わたし]はいつも 同僚[どうりょう]の 不平[ふへい]を 聞[き]いているんだ。	わたし は いつも どうりょう の ふへい を きいて いる ん だ	
\\	私[わたし]はいつも 同僚[どうりょう]の
\\	を 聞[き]いているんだ。			
\\	とっくに	とっくに	とっくに	
\\	彼ならとっくに帰りましたよ。	彼[かれ]ならとっくに 帰[かえ]りましたよ。	かれ なら とっくに かえりました よ	
\\	彼[かれ]なら
\\	帰[かえ]りましたよ。			
\\	ベスト	ベスト	ベスト	
\\	私はベストを尽くしました。	私[わたし]はベストを 尽[つ]くしました。	わたし は べすと を つくしました	
\\	私[わたし]は
\\	を 尽[つ]くしました。			
\\	不平等	不平等[ふびょうどう]	ふびょうどう	
\\	職場での男女不平等はよく見られるね。	職場[しょくば]での 男女[だんじょ] 不平等[ふびょうどう]はよく 見[み]られるね。	しょくば で の だんじょ ふびょうどう は よく みられる ね	
\\	職場[しょくば]での 男女[だんじょ]
\\	はよく 見[み]られるね。			
\\	ワンピース	ワンピース	ワンピース	
\\	彼女は白いワンピースを着ていたよ。	彼女[かのじょ]は 白[しろ]いワンピースを 着[き]ていたよ。	かのじょ は しろい わんぴーす を きて いた よ	
\\	彼女[かのじょ]は 白[しろ]い
\\	を 着[き]ていたよ。			
\\	内科	内科[ないか]	ないか	
\\	内科で胃の調子を診てもらったよ。	内科[ないか]で 胃[い]の 調子[ちょうし]を 診[み]てもらったよ。	ないか で い の ちょうし を みて もらった よ	
\\	で 胃[い]の 調子[ちょうし]を 診[み]てもらったよ。			
\\	ダイヤモンド	ダイヤモンド	ダイヤモンド	
\\	ダイヤモンドはとても高価ですね。	ダイヤモンドはとても 高価[こうか]ですね。	だいやもんど は とても こうか です ね	
\\	はとても 高価[こうか]ですね。			
\\	亡くす	亡[な]くす	なくす	
\\	昨年、友人を亡くしました。	昨年[さくねん]、 友人[ゆうじん]を 亡[な]くしました。	さくねん ゆうじん を なくしました	
\\	昨年[さくねん]、 友人[ゆうじん]を
\\	ナンバー	ナンバー	ナンバー	
\\	その車のナンバーを覚えていますか。	その 車[くるま]のナンバーを 覚[おぼ]えていますか。	その くるま の なんばー を おぼえて います か	
\\	その 車[くるま]の
\\	を 覚[おぼ]えていますか。			
\\	保つ	保[たも]つ	たもつ	
\\	彼女は若さを保とうと必死だ。	彼女[かのじょ]は 若[わか]さを 保[たも]とうと 必死[ひっし]だ。	かのじょ は わかさ を たもとう と ひっし だ	
\\	彼女[かのじょ]は 若[わか]さを
\\	と 必死[ひっし]だ。			
\\	保険	保険[ほけん]	ほけん	
\\	あなたは保険に加入していますか。	あなたは 保険[ほけん]に 加入[かにゅう]していますか。	あなた は ほけん に かにゅう して います か	
\\	あなたは
\\	に 加入[かにゅう]していますか。			
\\	保証	保証[ほしょう]	ほしょう	
\\	この製品の保証期間は5年間となっております。	この 製品[せいひん]の 保証[ほしょう] 期間[きかん]は5 年間[ねん かん]となっております。	この せいひん の ほしょう きかん は 
\\	ねん かん と なって おります	
\\	この 製品[せいひん]の
\\	期間[きかん]は5 年間[ねん かん]となっております。			
\\	保証人	保証人[ほしょうにん]	ほしょうにん	
\\	彼が私の保証人になってくれました。	彼[かれ]が 私[わたし]の 保証人[ほしょうにん]になってくれました。	かれ が わたし の ほしょうにん に なって くれました	
\\	彼[かれ]が 私[わたし]の
\\	になってくれました。			
\\	とうとう	とうとう	とうとう	
\\	とうとう引っ越しの日がきました。	とうとう 引[ひ]っ 越[こ]しの 日[ひ]がきました。	とうとう ひっこし の ひ が きました	
\\	引[ひ]っ 越[こ]しの 日[ひ]がきました。			
\\	保存	保存[ほぞん]	ほぞん	
\\	データを30分おきに保存してください。	データを30 分[ぷん]おきに 保存[ほぞん]してください。	でーた を 
\\	ぷん おき に ほぞん して ください	
\\	データを30 分[ぷん]おきに
\\	してください。			
\\	とんでもない	とんでもない	とんでもない	
\\	仕事中に帰るなんてとんでもない。	仕事中[しごとちゅう]に 帰[かえ]るなんてとんでもない。	しごとちゅう に かえる なんて とんでもない	
\\	仕事中[しごとちゅう]に 帰[かえ]るなんて
\\	不注意	不注意[ふちゅうい]	ふちゅうい	
\\	その事故は運転手の不注意が原因で起きたの。	その 事故[じこ]は 運転手[うんてんしゅ]の 不注意[ふちゅうい]が 原因[げんいん]で 起[お]きたの。	その じこ は うんてんしゅ の ふちゅうい が げんいん で おきた の	
\\	その 事故[じこ]は 運転手[うんてんしゅ]の
\\	が 原因[げんいん]で 起[お]きたの。			
\\	にこやか	にこやか	にこやか	
\\	彼らはにこやかに挨拶を交わしたけどね。	彼[かれ]らはにこやかに 挨拶[あいさつ]を 交[か]わしたけどね。	かれら は にこやか に あいさつ を かわした けど ね	
\\	彼[かれ]らは
\\	に 挨拶[あいさつ]を 交[か]わしたけどね。			
\\	不確か	不確[ふたし]か	ふたしか	
\\	人の記憶は不確かよ。	人[ひと]の 記憶[きおく]は 不確[ふたし]かよ。	ひと の きおく は ふたしか よ	
\\	人[ひと]の 記憶[きおく]は
\\	よ。			
\\	ロッカー	ロッカー	ロッカー	
\\	荷物を駅のロッカーに入れた。	荷物[にもつ]を 駅[えき]のロッカーに 入[い]れた。	にもつ を えき の ろっかー に いれた	
\\	荷物[にもつ]を 駅[えき]の
\\	に 入[い]れた。			
\\	不器用	不器用[ぶきよう]	ぶきよう	
\\	妹は不器用で、料理も苦手なの。	妹[いもうと]は 不器用[ぶきよう]で、 料理[りょうり]も 苦手[にがて]なの。	いもうと は ぶきよう で りょうり も にがて な の	
\\	妹[いもうと]は
\\	で、 料理[りょうり]も 苦手[にがて]なの。			
\\	予備	予備[よび]	よび	
\\	旅行には予備の靴を持って行きます。	旅行[りょこう]には 予備[よび]の 靴[くつ]を 持[も]って 行[い]きます。	りょこう に は よび の くつ を もって いきます	
\\	旅行[りょこう]には
\\	の 靴[くつ]を 持[も]って 行[い]きます。			
\\	プライド	プライド	プライド	
\\	彼はプライドが高い人です。	彼[かれ]はプライドが 高[たか]い 人[ひと]です。	かれ は ぷらいど が たかい ひと です	
\\	彼[かれ]は
\\	が 高[たか]い 人[ひと]です。			
\\	不公平	不公平[ふこうへい]	ふこうへい	
\\	彼のやり方は不公平です。	彼[かれ]のやり 方[かた]は 不公平[ふこうへい]です。	かれ の やりかた は ふこうへい です	
\\	彼[かれ]のやり 方[かた]は
\\	です。			
\\	保育園	保育園[ほいくえん]	ほいくえん	
\\	娘を保育園に迎えに行きます。	娘[むすめ]を 保育園[ほいくえん]に 迎[むか]えに 行[い]きます。	むすめ を ほいくえん に むかえ に いきます	
\\	娘[むすめ]を
\\	に 迎[むか]えに 行[い]きます。			
\\	ママ	ママ	ママ	
\\	ママに聞いてみよう。	ママに 聞[き]いてみよう。	まま に きいて みよう	
\\	に 聞[き]いてみよう。			
\\	よける	よける	よける	
\\	椅子の荷物をよけて座ったけど。	椅子[いす]の 荷物[にもつ]をよけて 座[すわ]ったけど。	いす の にもつ を よけて すわった けど	
\\	椅子[いす]の 荷物[にもつ]を
\\	座[すわ]ったけど。			
\\	半島	半島[はんとう]	はんとう	
\\	台風がその半島を通過したの。	台風[たいふう]がその 半島[はんとう]を 通過[つうか]したの。	たいふう が その はんとう を つうか した の	
\\	台風[たいふう]がその
\\	を 通過[つうか]したの。			
\\	不完全	不完全[ふかんぜん]	ふかんぜん	
\\	このデータはまだ不完全ね。	このデータはまだ 不完全[ふかんぜん]ね。	この でーた は まだ ふかんぜん ね	
\\	このデータはまだ
\\	ね。			
\\	バッジ	バッジ	バッジ	
\\	彼は弁護士バッジを付けていたよ。	彼[かれ]は 弁護士[べんごし]バッジを 付[つ]けていたよ。	かれ は べんごし ばっじ を つけて いた よ	
\\	彼[かれ]は 弁護士[べんごし]
\\	を 付[つ]けていたよ。			
\\	原	原[はら]	はら	
\\	クローバーの原でピクニックをしたよ。	クローバーの 原[はら]でピクニックをしたよ。	くろーばー の はら で ぴくにっく を した よ	
\\	クローバーの
\\	でピクニックをしたよ。			
\\	ピン	ピン	ピン	
\\	彼女はいつも髪をピンで留めているね。	彼女[かのじょ]はいつも 髪[かみ]をピンで 留[と]めているね。	かのじょ は いつも かみ を ぴん で とめて いる ね	
\\	彼女[かのじょ]はいつも 髪[かみ]を
\\	で 留[と]めているね。			
\\	不正	不正[ふせい]	ふせい	
\\	試験で不正が見つかったよ。	試験[しけん]で 不正[ふせい]が 見[み]つかったよ。	しけん で ふせい が みつかった よ	
\\	試験[しけん]で
\\	が 見[み]つかったよ。			
\\	まあ	まあ	まあ	
\\	まあこれでいいだろう。	まあこれでいいだろう。	まあこれでいいだろう。	
\\	これでいいだろう。			
\\	むなしい	むなしい	むなしい	
\\	愛がなければ人生はむなしいよ。	愛[あい]がなければ 人生[じんせい]はむなしいよ。	あい が なければ じんせい は むなしい よ	
\\	愛[あい]がなければ 人生[じんせい]は
\\	よ。			
\\	ついで	ついで	ついで	
\\	ついでだから彼も呼ぼうよ。	ついでだから 彼[かれ]も 呼[よ]ぼうよ。	ついで だから かれ も よぼうよ	
\\	だから 彼[かれ]も 呼[よ]ぼうよ。			
\\	体調	体調[たいちょう]	たいちょう	
\\	今日は体調が悪いです。	今日[きょう]は 体調[たいちょう]が 悪[わる]いです。	きょう は たいちょう が わるい です	
\\	今日[きょう]は
\\	が 悪[わる]いです。			
\\	単調	単調[たんちょう]	たんちょう	
\\	この曲は単調でつまらないな。	この 曲[きょく]は 単調[たんちょう]でつまらないな。	この きょく は たんちょう で つまらない な	
\\	この 曲[きょく]は
\\	でつまらないな。			
\\	バケツ	バケツ	バケツ	
\\	バケツに水を汲んできてください。	バケツに 水[みず]を 汲[く]んできてください。	ばけつ に みず を くんで きて ください	
\\	に 水[みず]を 汲[く]んできてください。			
\\	前提	前提[ぜんてい]	ぜんてい	
\\	交渉が成立するという前提で話を進めます。	交渉[こうしょう]が 成立[せいりつ]するという 前提[ぜんてい]で 話[はなし]を 進[すす]めます。	こうしょう が せいりつ する と いう ぜんてい で はなし を すすめます	
\\	交渉[こうしょう]が 成立[せいりつ]するという
\\	で 話[はなし]を 進[すす]めます。			
\\	はっと	はっと	はっと	
\\	彼女の美しさにはっとしたよ。	彼女[かのじょ]の 美[うつく]しさにはっとしたよ。	かのじょ の うつくしさ に はっと した よ	
\\	彼女[かのじょ]の 美[うつく]しさに
\\	したよ。			
\\	まずい	まずい	まずい	
\\	彼女の運転がまずくて、はらはらした。	彼女[かのじょ]の 運転[うんてん]がまずくて、はらはらした。	かのじょ の うんてん が まずくて はらはら した	
\\	彼女[かのじょ]の 運転[うんてん]が
\\	、はらはらした。			
\\	もしも	もしも	もしも	
\\	もしも彼女と結婚できたらどんなに嬉しいだろう。	もしも 彼女[かのじょ]と 結婚[けっこん]できたらどんなに 嬉[うれ]しいだろう。	もしも かのじょ と けっこん できたら どんなに うれしい だろう	
\\	彼女[かのじょ]と 結婚[けっこん]できたらどんなに 嬉[うれ]しいだろう。			
\\	やかましい	やかましい	やかましい	
\\	スピーカーの音がやかましいな。	スピーカーの 音[おと]がやかましいな。	すぴーかー の おと が やかましい な	
\\	スピーカーの 音[おと]が
\\	な。			
\\	バット	バット	バット	
\\	このバットは金属で出来ています。	このバットは 金属[きんぞく]で 出来[でき]ています。	この ばっと は きんぞく で できて います	
\\	この
\\	は 金属[きんぞく]で 出来[でき]ています。			
\\	マナー	マナー	マナー	
\\	食事のマナーを守りましょう。	食事[しょくじ]のマナーを 守[まも]りましょう。	しょくじ の まなー を まもりましょう	
\\	食事[しょくじ]の
\\	を 守[まも]りましょう。			
\\	不況	不況[ふきょう]	ふきょう	
\\	不況の影響で仕事が少ないね。	不況[ふきょう]の 影響[えいきょう]で 仕事[しごと]が 少[すく]ないね。	ふきょう の えいきょう で しごと が すくない ね	
\\	の 影響[えいきょう]で 仕事[しごと]が 少[すく]ないね。			
\\	ばらばら	ばらばら	ばらばら	
\\	生徒たちはばらばらに帰宅したね。	生徒[せいと]たちはばらばらに 帰宅[きたく]したね。	せいとたち は ばらばら に きたく した ね	
\\	生徒[せいと]たちは
\\	に 帰宅[きたく]したね。			
\\	入選	入選[にゅうせん]	にゅうせん	
\\	彼の絵がコンクールに入選した。	彼[かれ]の 絵[え]がコンクールに 入選[にゅうせん]した。	かれ の え が こんくーる に にゅうせん した	
\\	彼[かれ]の 絵[え]がコンクールに
\\	した。			
\\	プラットホーム	プラットホーム	プラットホーム	
\\	プラットホームで電車が来るのを待ったの。	プラットホームで 電車[でんしゃ]が 来[く]るのを 待[ま]ったの。	ぷらっとほーむ で でんしゃ が くる の を まった の	
\\	で 電車[でんしゃ]が 来[く]るのを 待[ま]ったの。			
\\	先頭	先頭[せんとう]	せんとう	
\\	先頭の人はプラカードを持ってください。	先頭[せんとう]の 人[ひと]はプラカードを 持[も]ってください。	せんとう の ひと は ぷらかーど を もって ください	
\\	の 人[ひと]はプラカードを 持[も]ってください。			
\\	まぐろ	まぐろ	まぐろ	
\\	まぐろの刺身を食べました。	まぐろの 刺身[さしみ]を 食[た]べました。	まぐろ の さしみ を たべました	
\\	の 刺身[さしみ]を 食[た]べました。			
\\	取り組む	取[と]り 組[く]む	とりくむ	
\\	彼女はスペイン語に取り組んでいます。	彼女[かのじょ]はスペイン 語[ご]に 取[と]り 組[く]んでいます。	かのじょ は すぺいんご に とりくんで います	
\\	彼女[かのじょ]はスペイン 語[ご]に
\\	任せる	任[まか]せる	まかせる	
\\	この仕事は君に任せる。	この 仕事[しごと]は 君[きみ]に 任[まか]せる。	この しごと は きみ に まかせる	
\\	この 仕事[しごと]は 君[きみ]に
\\	ろくに	ろくに	ろくに	
\\	ゆうべはろくに寝ていない。	ゆうべはろくに 寝[ね]ていない。	ゆうべ は ろくに ねて いない	
\\	ゆうべは
\\	寝[ね]ていない。			
\\	勤め先	勤[つと]め 先[さき]	つとめさき	
\\	私の勤め先にご連絡ください。	私[わたし]の 勤[つと]め 先[さき]にご 連絡[れんらく]ください。	わたし の つとめさき に ごれんらく ください	
\\	私[わたし]の
\\	にご 連絡[れんらく]ください。			
\\	勤め	勤[つと]め	つとめ	
\\	来月で勤めをやめます。	来月[らいげつ]で 勤[つと]めをやめます。	らいげつ で つとめ を やめます	
\\	来月[らいげつ]で
\\	をやめます。			
\\	務める	務[つと]める	つとめる	
\\	私が司会を務めます。	私[わたし]が 司会[しかい]を 務[つと]めます。	わたし が しかい を つとめます	
\\	私[わたし]が 司会[しかい]を
\\	つまむ	つまむ	つまむ	
\\	お菓子でもつまんでください。	お 菓子[かし]でもつまんでください。	おかし で も つまんで ください	
\\	お 菓子[かし]でも
\\	ください。			
\\	任務	任務[にんむ]	にんむ	
\\	彼は最後まで自分の任務を果たした。	彼[かれ]は 最後[さいご]まで 自分[じぶん]の 任務[にんむ]を 果[は]たした。	かれ は さいご まで じぶん の にんむ を はたした	
\\	彼[かれ]は 最後[さいご]まで 自分[じぶん]の
\\	を 果[は]たした。			
\\	務め	務[つと]め	つとめ	
\\	人々を守るのが私の務めです。	人々[ひとびと]を 守[まも]るのが 私[わたし]の 務[つと]めです。	ひとびと を まもる の が わたし の つとめ です	
\\	人々[ひとびと]を 守[まも]るのが 私[わたし]の
\\	です。			
\\	募集	募集[ぼしゅう]	ぼしゅう	
\\	私の会社で社員を募集しています。	私[わたし]の 会社[かいしゃ]で 社員[しゃいん]を 募集[ぼしゅう]しています。	わたし の かいしゃ で しゃいん を ぼしゅう して います	
\\	私[わたし]の 会社[かいしゃ]で 社員[しゃいん]を
\\	しています。			
\\	ボーイ	ボーイ	ボーイ	
\\	ボーイに荷物を運んでもらった。	ボーイに 荷物[にもつ]を 運[はこ]んでもらった。	ぼーい に にもつ を はこんで もらった	
\\	に 荷物[にもつ]を 運[はこ]んでもらった。			
\\	ホット	ホット	ホット	
\\	コーヒーをホットでください。	コーヒーをホットでください。	こーひー を ほっと で ください	
\\	コーヒーを
\\	でください。			
\\	参る	参[まい]る	まいる	
\\	さあ、参りましょうか。	さあ、 参[まい]りましょうか。	さあ まいりましょう か	
\\	さあ、
\\	か。			
\\	まし	まし	まし	
\\	これでもないよりましだ。	これでもないよりましだ。	これでもないよりましだ。	
\\	これでもないより
\\	だ。			
\\	判決	判決[はんけつ]	はんけつ	
\\	判決が下された。	判決[はんけつ]が 下[くだ]された。	はんけつ が くだされた	
\\	が 下[くだ]された。			
\\	ぱっと	ぱっと	ぱっと	
\\	彼女はぱっと目を開いた。	彼女[かのじょ]はぱっと 目[め]を 開[ひら]いた。	かのじょ は ぱっと め を ひらいた	
\\	彼女[かのじょ]は
\\	目[め]を 開[ひら]いた。			
\\	判子	判子[はんこ]	はんこ	
\\	ここに判子を押してください。	ここに 判子[はんこ]を 押[お]してください。	ここ に はんこ を おして ください	
\\	ここに
\\	を 押[お]してください。			
\\	反感	反感[はんかん]	はんかん	
\\	俺たちは彼の言動に反感を持ったね。	俺[おれ]たちは 彼[かれ]の 言動[げんどう]に 反感[はんかん]を 持[も]ったね。	おれたち は かれ の げんどう に はんかん を もった ね	
\\	俺[おれ]たちは 彼[かれ]の 言動[げんどう]に
\\	を 持[も]ったね。			
\\	予想	予想[よそう]	よそう	
\\	私の予想が当たった。	私[わたし]の 予想[よそう]が 当[あ]たった。	わたし の よそう が あたった	
\\	私[わたし]の
\\	が 当[あ]たった。			
\\	つなぐ	つなぐ	つなぐ	
\\	インターネットは世界の人々をつなぎますね。	インターネットは 世界[せかい]の 人々[ひとびと]をつなぎますね。	いんたーねっと は せかい の ひとびと を つなぎます ね	
\\	インターネットは 世界[せかい]の 人々[ひとびと]を
\\	ね。			
\\	保障	保障[ほしょう]	ほしょう	
\\	私があなたの安全を保障します。	私[わたし]があなたの 安全[あんぜん]を 保障[ほしょう]します。	わたし が あなた の あんぜん を ほしょう します	
\\	私[わたし]があなたの 安全[あんぜん]を
\\	します。			
\\	ひょっとしたら	ひょっとしたら	ひょっとしたら	
\\	ひょっとしたら彼はそのことを知らないのかもしれない。	ひょっとしたら 彼[かれ]はそのことを 知[し]らないのかもしれない。	ひょっとしたら かれ は その こと を しらない の かも しれない	
\\	彼[かれ]はそのことを 知[し]らないのかもしれない。			
\\	取り替える	取[と]り 替[か]える	とりかえる	
\\	シーツを取り替えました。	シーツを 取[と]り 替[か]えました。	しーつ を とりかえました	
\\	シーツを
\\	ひとりでに	ひとりでに	ひとりでに	
\\	ドアがひとりでに閉まったな。	ドアがひとりでに 閉[し]まったな。	どあ が ひとりでに しまった な	
\\	ドアが
\\	閉[し]まったな。			
\\	助かる	助[たす]かる	たすかる	
\\	彼はシートベルトをしていたので助かったんだ。	彼[かれ]はシートベルトをしていたので 助[たす]かったんだ。	かれ は しーとべると を して いた の で たすかった ん だ	
\\	彼[かれ]はシートベルトをしていたので
\\	んだ。			
\\	ひねる	ひねる	ひねる	
\\	彼は水道の蛇口をひねったの。	彼[かれ]は 水道[すいどう]の 蛇口[じゃぐち]をひねったの。	かれ は すいどう の じゃぐち を ひねった の	
\\	彼[かれ]は 水道[すいどう]の 蛇口[じゃぐち]を
\\	の。			
\\	ひび	ひび	ひび	
\\	窓ガラスにひびが入ったよ。	窓[まど]ガラスにひびが 入[はい]ったよ。	まどがらす に ひび が はいった よ	
\\	窓[まど]ガラスに
\\	が 入[はい]ったよ。			
\\	やっつける	やっつける	やっつける	
\\	主人公が悪者をやっつけた。	主人公[しゅじんこう]が 悪者[わるもの]をやっつけた。	しゅじんこう が わるもの を やっつけた	
\\	主人公[しゅじんこう]が 悪者[わるもの]を
\\	レントゲン	レントゲン	レントゲン	
\\	病院でレントゲンを撮ったよ。	病院[びょういん]でレントゲンを 撮[と]ったよ。	びょういん で れんとげん を とった よ	
\\	病院[びょういん]で
\\	を 撮[と]ったよ。			
\\	乱れる	乱[みだ]れる	みだれる	
\\	風で髪が乱れたね。	風[かぜ]で 髪[かみ]が 乱[みだ]れたね。	かぜ で かみ が みだれた ね	
\\	風[かぜ]で 髪[かみ]が
\\	ね。			
\\	パンツ	パンツ	パンツ	
\\	彼女はスカートよりパンツが似合うね。	彼女[かのじょ]はスカートよりパンツが 似合[にあ]うね。	かのじょ は すかーと より ぱんつ が にあう ね	
\\	彼女[かのじょ]はスカートより
\\	が 似合[にあ]うね。			
\\	乱す	乱[みだ]す	みだす	
\\	彼は風紀を乱している。	彼[かれ]は 風紀[ふうき]を 乱[みだ]している。	かれ は ふうき を みだして いる	
\\	彼[かれ]は 風紀[ふうき]を
\\	むける	むける	むける	
\\	日焼けで皮がむけた。	日焼[ひや]けで 皮[かわ]がむけた。	ひやけ で かわ が むけた	
\\	日焼[ひや]けで 皮[かわ]が
\\	ぶつぶつ	ぶつぶつ	ぶつぶつ	
\\	彼はぶつぶつと独り言を言ったんだ。	彼[かれ]はぶつぶつと 独[ひと]り 言[ごと]を 言[い]ったんだ。	かれ は ぶつぶつ と ひとりごと を いった ん だ	
\\	彼[かれ]は
\\	と 独[ひと]り 言[ごと]を 言[い]ったんだ。			
\\	ぶるぶる	ぶるぶる	ぶるぶる	
\\	彼はぶるぶる震えていたよ。	彼[かれ]はぶるぶる 震[ふる]えていたよ。	かれ は ぶるぶる ふるえて いた よ	
\\	彼[かれ]は
\\	震[ふる]えていたよ。			
\\	中流	中流[ちゅうりゅう]	ちゅうりゅう	
\\	川の中流あたりにその村はあるよ。	川[かわ]の 中流[ちゅうりゅう]あたりにその 村[むら]はあるよ。	かわ の ちゅうりゅう あたり に その むら は ある よ	
\\	川[かわ]の
\\	あたりにその 村[むら]はあるよ。			
\\	ボリューム	ボリューム	ボリューム	
\\	ラジオのボリュームを上げてください。	ラジオのボリュームを 上[あ]げてください。	らじお の ぼりゅーむ を あげて ください	
\\	ラジオの
\\	を 上[あ]げてください。			
\\	二流	二流[にりゅう]	にりゅう	
\\	彼はまだまだ二流の芸人だね。	彼[かれ]はまだまだ 二流[にりゅう]の 芸人[げいにん]だね。	かれ は まだまだ にりゅう の げいにん だ ね	
\\	彼[かれ]はまだまだ
\\	の 芸人[げいにん]だね。			
\\	ちり	ちり	ちり	
\\	ちりも積もれば山となる。	ちりも 積[つ]もれば 山[やま]となる。	ちり も つもれば やま と なる 。	
\\	も 積[つ]もれば 山[やま]となる。			
\\	冷え込む	冷[ひ]え 込[こ]む	ひえこむ	
\\	明日から急に冷え込むそうです。	明日[あす]から 急[きゅう]に 冷[ひ]え 込[こ]むそうです。	あす から きゅう に ひえこむ そう です	
\\	明日[あす]から 急[きゅう]に
\\	そうです。			
\\	冷やかす	冷[ひ]やかす	ひやかす	
\\	友達はそのカップルを冷やかしたんだ。	友達[ともだち]はそのカップルを 冷[ひ]やかしたんだ。	ともだち は その かっぷる を ひやかした ん だ	
\\	友達[ともだち]はそのカップルを
\\	んだ。			
\\	冷凍	冷凍[れいとう]	れいとう	
\\	残ったカレーを冷凍しました。	残[のこ]ったカレーを 冷凍[れいとう]しました。	のこった かれー を れいとう しました	
\\	残[のこ]ったカレーを
\\	しました。			
\\	内臓	内臓[ないぞう]	ないぞう	
\\	来週、内臓を検査します。	来週[らいしゅう]、 内臓[ないぞう]を 検査[けんさ]します。	らいしゅう ないぞう を けんさ します	
\\	来週[らいしゅう]、
\\	を 検査[けんさ]します。			
\\	はげる	はげる	はげる	
\\	壁のペンキがはげてきたな。	壁[かべ]のペンキがはげてきたな。	かべ の ぺんき が はげて きた な	
\\	壁[かべ]のペンキが
\\	きたな。			
\\	パパ	パパ	パパ	
\\	私のパパは36歳です。	私[わたし]のパパは36 歳[さい]です。	わたし の ぱぱ は 
\\	さい です	
\\	私[わたし]の
\\	は36 歳[さい]です。			
\\	不景気	不景気[ふけいき]	ふけいき	
\\	今、あの国は不景気らしいよ。	今[いま]、あの 国[くに]は 不景気[ふけいき]らしいよ。	いま あの くに は ふけいき らしい よ	
\\	今[いま]、あの 国[くに]は
\\	らしいよ。			
\\	つかむ	つかむ	つかむ	
\\	彼女が僕の手をつかみました。	彼女[かのじょ]が 僕[ぼく]の 手[て]をつかみました。	かのじょ が ぼく の て を つかみました	
\\	彼女[かのじょ]が 僕[ぼく]の 手[て]を
\\	光	光[ひかり]	ひかり	
\\	一筋の光が窓から差し込んだの。	一筋[ひとすじ]の 光[ひかり]が 窓[まど]から 差[さ]し 込[こ]んだの。	ひとすじ の ひかり が まど から さしこんだ の	
\\	一筋[ひとすじ]の
\\	が 窓[まど]から 差[さ]し 込[こ]んだの。			
\\	もったいない	もったいない	もったいない	
\\	食べ物を残してはもったいないわよ。	食[た]べ 物[もの]を 残[のこ]してはもったいないわよ。	たべもの を のこして は もったいない わ よ	
\\	食[た]べ 物[もの]を 残[のこ]しては
\\	わよ。			
\\	プロ野球	プロ 野球[やきゅう]	プロやきゅう	
\\	最近のプロ野球は面白くなってきたね。	最近[さいきん]のプロ 野球[やきゅう]は 面白[おもしろ]くなってきたね。	さいきん の ぷろやきゅう は おもしろく なって きた ね	
\\	最近[さいきん]の
\\	は 面白[おもしろ]くなってきたね。			
\\	ふざける	ふざける	ふざける	
\\	ふざけるのは止めて。	ふざけるのは 止[や]めて。	ふざける の は やめて	
\\	のは 止[や]めて。			
\\	だるい	だるい	だるい	
\\	昨日は風邪で少しだるかったんだ。	昨日[きのう]は 風邪[かぜ]で 少[すこ]しだるかったんだ。	きのう は かぜ で すこし だるかった ん だ	
\\	昨日[きのう]は 風邪[かぜ]で 少[すこ]し
\\	んだ。			
\\	ちぎる	ちぎる	ちぎる	
\\	紙を細かくちぎってください。	紙[かみ]を 細[こま]かくちぎってください。	かみ を こまかく ちぎって ください	
\\	紙[かみ]を 細[こま]かく
\\	ください。			
\\	単位	単位[たんい]	たんい	
\\	メートルは長さの単位です。	メートルは 長[なが]さの 単位[たんい]です。	めーとる は ながさ の たんい です	
\\	メートルは 長[なが]さの
\\	です。			
\\	分離	分離[ぶんり]	ぶんり	
\\	自民党から新しい党が分離したね。	自民党[じみんとう]から 新[あたら]しい 党[とう]が 分離[ぶんり]したね。	じみんとう から あたらしい とう が ぶんり した ね	
\\	自民党[じみんとう]から 新[あたら]しい 党[とう]が
\\	したね。			
\\	ねぎ	ねぎ	ねぎ	
\\	みそ汁にねぎを入れました。	みそ 汁[しる]にねぎを 入[い]れました。	みそしる に ねぎ を いれました	
\\	みそ 汁[しる]に
\\	を 入[い]れました。			
\\	停電	停電[ていでん]	ていでん	
\\	台風で停電したよ。	台風[たいふう]で 停電[ていでん]したよ。	たいふう で ていでん した よ	
\\	台風[たいふう]で
\\	したよ。			
\\	停止	停止[ていし]	ていし	
\\	突然、機械が停止してしまったの。	突然[とつぜん]、 機械[きかい]が 停止[ていし]してしまったの。	とつぜん きかい が ていし して しまった の	
\\	突然[とつぜん]、 機械[きかい]が
\\	してしまったの。			
\\	びしょびしょ	びしょびしょ	びしょびしょ	
\\	服が雨でびしょびしょになったよ。	服[ふく]が 雨[あめ]でびしょびしょになったよ。	ふく が あめ で びしょびしょ に なった よ	
\\	服[ふく]が 雨[あめ]で
\\	になったよ。			
\\	中断	中断[ちゅうだん]	ちゅうだん	
\\	停電のため仕事を中断しました。	停電[ていでん]のため 仕事[しごと]を 中断[ちゅうだん]しました。	ていでん の ため しごと を ちゅうだん しました	
\\	停電[ていでん]のため 仕事[しごと]を
\\	しました。			
\\	バーゲン	バーゲン	バーゲン	
\\	昨日バーゲンでスーツを買いました。	昨日[きのう]バーゲンでスーツを 買[か]いました。	きのう ばーげん で すーつ を かいました	
\\	昨日[きのう]
\\	でスーツを 買[か]いました。			
\\	中継	中継[ちゅうけい]	ちゅうけい	
\\	京都から中継で放送しています。	京都[きょうと]から 中継[ちゅうけい]で 放送[ほうそう]しています。	きょうと から ちゅうけい で ほうそう して います	
\\	京都[きょうと]から
\\	で 放送[ほうそう]しています。			
\\	メーター	メーター	メーター	
\\	ガス会社がメーターを調べにきたぞ。	ガス 会社[がいしゃ]がメーターを 調[しら]べにきたぞ。	がす がいしゃ が めーたー を しらべ に きた ぞ	
\\	ガス 会社[がいしゃ]が
\\	を 調[しら]べにきたぞ。			
\\	内陸	内陸[ないりく]	ないりく	
\\	彼は中国の内陸を旅行しました。	彼[かれ]は 中国[ちゅうごく]の 内陸[ないりく]を 旅行[りょこう]しました。	かれ は ちゅうごく の ないりく を りょこう しました	
\\	彼[かれ]は 中国[ちゅうごく]の
\\	を 旅行[りょこう]しました。			
\\	もしかすると	もしかすると	もしかすると	
\\	もしかするとあなたに一度お会いしてますか。	もしかするとあなたに 一度[いちど]お 会[あ]いしてますか。	もしかすると あなた に いちど おあい してます か	
\\	あなたに 一度[いちど]お 会[あ]いしてますか。			
\\	南極	南極[なんきょく]	なんきょく	
\\	南極でペンギンを見ました。	南極[なんきょく]でペンギンを 見[み]ました。	なんきょく で ぺんぎん を みました	
\\	でペンギンを 見[み]ました。			
\\	北極	北極[ほっきょく]	ほっきょく	
\\	北極にペンギンはいません。	北極[ほっきょく]にペンギンはいません。	ほっきょく に ぺんぎん は いません	
\\	にペンギンはいません。			
\\	先端	先端[せんたん]	せんたん	
\\	棒の先端を持って下さい。	棒[ぼう]の 先端[せんたん]を 持[も]って 下[くだ]さい。	ぼう の せんたん を もって ください	
\\	棒[ぼう]の
\\	を 持[も]って 下[くだ]さい。			
\\	ちぎれる	ちぎれる	ちぎれる	
\\	荷物が重くて手がちぎれそうだ。	荷物[にもつ]が 重[おも]くて 手[て]がちぎれそうだ。	にもつ が おもくて て が ちぎれ そう だ	
\\	荷物[にもつ]が 重[おも]くて 手[て]が
\\	だ。			
\\	内緒	内緒[ないしょ]	ないしょ	
\\	この話は課長には内緒ですよ。	この 話[はなし]は 課長[かちょう]には 内緒[ないしょ]ですよ。	この はなし は かちょう に は ないしょ です よ	
\\	この 話[はなし]は 課長[かちょう]には
\\	ですよ。			
\\	にっこり	にっこり	にっこり	
\\	彼女はにっこりほほえんだ。	彼女[かのじょ]はにっこりほほえんだ。	かのじょ は にっこり ほほえんだ	
\\	彼女[かのじょ]は
\\	ほほえんだ。			
\\	体操	体操[たいそう]	たいそう	
\\	兄は体操の選手です。	兄[あに]は 体操[たいそう]の 選手[せんしゅ]です。	あに は たいそう の せんしゅ です	
\\	兄[あに]は
\\	の 選手[せんしゅ]です。			
\\	ぱいなっぷる	ぱいなっぷる	ぱいなっぷる	
\\	このパイナップルを切ってください。	このパイナップルを 切[き]ってください。	この ぱいなっぷる を きって ください	
\\	この
\\	を 切[き]ってください。			
\\	保守	保守[ほしゅ]	ほしゅ	
\\	あの村は伝統を保守している。	あの 村[むら]は 伝統[でんとう]を 保守[ほしゅ]している。	あの むら は でんとう を ほしゅ して いる	
\\	あの 村[むら]は 伝統[でんとう]を
\\	している。			
\\	ぱんだ	ぱんだ	ぱんだ	
\\	パンダは笹を食べます。	パンダは 笹[ささ]を 食[た]べます。	ぱんだ は ささ を たべます	
\\	は 笹[ささ]を 食[た]べます。			
\\	取り戻す	取[と]り 戻[もど]す	とりもどす	
\\	緑を取り戻す必要があります。	緑[みどり]を 取[と]り 戻[もど]す 必要[ひつよう]があります。	みどり を とりもどす ひつよう が あります	
\\	緑[みどり]を
\\	必要[ひつよう]があります。			
\\	びっしょり	びっしょり	びっしょり	
\\	運動して汗びっしょりです。	運動[うんどう]して 汗[あせ]びっしょりです。	うんどう して あせ びっしょり です	
\\	運動[うんどう]して 汗[あせ]
\\	です。			
\\	冷静	冷静[れいせい]	れいせい	
\\	冷静に話し合いましょう。	冷静[れいせい]に 話[はな]し 合[あ]いましょう。	れいせい に はなしあいましょう	
\\	に 話[はな]し 合[あ]いましょう。			
\\	両者	両者[りょうしゃ]	りょうしゃ	
\\	両者の意見を聞きましょう。	両者[りょうしゃ]の 意見[いけん]を 聞[き]きましょう。	りょうしゃ の いけん を ききましょう	
\\	の 意見[いけん]を 聞[き]きましょう。			
\\	両立	両立[りょうりつ]	りょうりつ	
\\	勉強と仕事の両立は難しいよ。	勉強[べんきょう]と 仕事[しごと]の 両立[りょうりつ]は 難[むずか]しいよ。	べんきょう と しごと の りょうりつ は むずかしい よ	
\\	勉強[べんきょう]と 仕事[しごと]の
\\	は 難[むずか]しいよ。			
\\	ペンチ	ペンチ	ペンチ	
\\	ペンチで針金を切ります。	ペンチで 針金[はりがね]を 切[き]ります。	ぺんち で はりがね を きります	
\\	で 針金[はりがね]を 切[き]ります。			
\\	両手	両手[りょうて]	りょうて	
\\	彼は両手を握り締めたの。	彼[かれ]は 両手[りょうて]を 握[にぎ]り 締[し]めたの。	かれ は りょうて を にぎりしめた の	
\\	彼[かれ]は
\\	を 握[にぎ]り 締[し]めたの。			
\\	両面	両面[りょうめん]	りょうめん	
\\	紙の両面に字が書かれていたよ。	紙[かみ]の 両面[りょうめん]に 字[じ]が 書[か]かれていたよ。	かみ の りょうめん に じ が かかれて いた よ	
\\	紙[かみ]の
\\	に 字[じ]が 書[か]かれていたよ。			
\\	側面	側面[そくめん]	そくめん	
\\	これが建物の側面の写真です。	これが 建物[たてもの]の 側面[そくめん]の 写真[しゃしん]です。	これ が たてもの の そくめん の しゃしん です	
\\	これが 建物[たてもの]の
\\	の 写真[しゃしん]です。			
\\	両側	両側[りょうがわ]	りょうがわ	
\\	道の両側にたくさんお店がありますよ。	道[みち]の 両側[りょうがわ]にたくさんお 店[みせ]がありますよ。	みち の りょうがわ に たくさん おみせ が あります よ	
\\	道[みち]の
\\	にたくさんお 店[みせ]がありますよ。			
\\	モダン	モダン	モダン	
\\	あの建物はモダンだね。	あの 建物[たてもの]はモダンだね。	あの たてもの は もだん だ ね	
\\	あの 建物[たてもの]は
\\	だね。			
\\	リクエスト	リクエスト	リクエスト	
\\	ラジオ番組にリクエストを送ったの。	ラジオ 番組[ばんぐみ]にリクエストを 送[おく]ったの。	らじお ばんぐみ に りくえすと を おくった の	
\\	ラジオ 番組[ばんぐみ]に
\\	を 送[おく]ったの。			
\\	ビーチ	ビーチ	ビーチ	
\\	ビーチで友達とサーフィンしたよ。	ビーチで 友達[ともだち]とサーフィンしたよ。	びーち で ともだち と さーふぃん した よ	
\\	で 友達[ともだち]とサーフィンしたよ。			
\\	付近	付近[ふきん]	ふきん	
\\	事件現場付近で怪しい人物を見た。	事件現場[じけん げんば] 付近[ふきん]で 怪[あや]しい 人物[じんぶつ]を 見[み]た。	じけん げんば ふきん で あやしい じんぶつ を みた	
\\	事件現場[じけん げんば]
\\	で 怪[あや]しい 人物[じんぶつ]を 見[み]た。			
\\	付き合い	付[つ]き 合[あ]い	つきあい	
\\	今日は付き合いで帰りが遅くなります。	今日[きょう]は 付[つ]き 合[あ]いで 帰[かえ]りが 遅[おそ]くなります。	きょう は つきあい で かえり が おそく なります	
\\	今日[きょう]は
\\	で 帰[かえ]りが 遅[おそ]くなります。			
\\	ビタミン	ビタミン	ビタミン	
\\	ビタミンを毎日とりましょう。	ビタミンを 毎日[まいにち]とりましょう。	びたみん を まいにち とりましょう	
\\	を 毎日[まいにち]とりましょう。			
\\	付け加える	付[つ]け 加[くわ]える	つけくわえる	
\\	自分の考えを付け加えました。	自分[じぶん]の 考[かんが]えを 付[つ]け 加[くわ]えました。	じぶん の かんがえ を つけくわえました	
\\	自分[じぶん]の 考[かんが]えを
\\	付き合う	付[つ]き 合[あ]う	つきあう	
\\	彼女と4年ほど付き合っています。	彼女[かのじょ]と4 年[ねん]ほど 付[つ]き 合[あ]っています。	かのじょ と 
\\	ねん ほど つきあって います	
\\	彼女[かのじょ]と4 年[ねん]ほど
\\	います。			
\\	付録	付録[ふろく]	ふろく	
\\	今月の付録はアニメの
\\	です。	今月[こんげつ]の 付録[ふろく]はアニメの 
\\	[でぃーぶぃーでぃー]です。	こんげつ の ふろく は あにめ の でぃーぶぃーでぃー です	
\\	今月[こんげつ]の
\\	はアニメの 
\\	[でぃーぶぃーでぃー]です。			
\\	付け足す	付[つ]け 足[た]す	つけたす	
\\	何か付け足すことはありますか。	何[なに]か 付[つ]け 足[た]すことはありますか。	なにか つけたす こと は あります か	
\\	何[なに]か
\\	ことはありますか。			
\\	フライパン	フライパン	フライパン	
\\	フライパンで目玉焼きを作ったよ。	フライパンで 目玉焼[めだまや]きを 作[つく]ったよ。	ふらいぱん で めだまやき を つくった よ	
\\	で 目玉焼[めだまや]きを 作[つく]ったよ。			
\\	付属	付属[ふぞく]	ふぞく	
\\	このソフトには色々なツールが付属しています。	このソフトには 色々[いろいろ]なツールが 付属[ふぞく]しています。	この そふと に は いろいろ な つーる が ふぞく して います	
\\	このソフトには 色々[いろいろ]なツールが
\\	しています。			
\\	反省	反省[はんせい]	はんせい	
\\	彼は十分反省しています。	彼[かれ]は 十分[じゅうぶん] 反省[はんせい]しています。	かれ は じゅうぶん はんせい して います	
\\	彼[かれ]は 十分[じゅうぶん]
\\	しています。			
\\	まあまあ	まあまあ	まあまあ	
\\	まあまあ、落ち着いてください。	まあまあ、 落[お]ち 着[つ]いてください。	まあまあ おちついて ください	
\\	、 落[お]ち 着[つ]いてください。			
\\	たこ	たこ	たこ	
\\	タコには足が8本ある。	タコには 足[あし]が8 本[ぽん]ある。	たこ に は あし が 
\\	ぽん ある	
\\	には 足[あし]が8 本[ぽん]ある。			
\\	列車	列車[れっしゃ]	れっしゃ	
\\	列車がホームに入ってきた。	列車[れっしゃ]がホームに 入[はい]ってきた。	れっしゃ が ほーむ に はいって きた	
\\	がホームに 入[はい]ってきた。			
\\	列島	列島[れっとう]	れっとう	
\\	日本は列島です。	日本[にっぽん]は 列島[れっとう]です。	にっぽん は れっとう です	
\\	日本[にっぽん]は
\\	です。			
\\	列	列[れつ]	れつ	
\\	店の前に長い列ができていたね。	店[みせ]の 前[まえ]に 長[なが]い 列[れつ]ができていたね。	みせ の まえ に ながい れつ が できて いた ね	
\\	店[みせ]の 前[まえ]に 長[なが]い
\\	ができていたね。			
\\	例外	例外[れいがい]	れいがい	
\\	例外は認めません。	例外[れいがい]は 認[みと]めません。	れいがい は みとめません	
\\	は 認[みと]めません。			
\\	デザート	デザート	デザート	
\\	デザートにケーキを食べました。	デザートにケーキを 食[た]べました。	でざーと に けーき を たべました	
\\	にケーキを 食[た]べました。			
\\	例える	例[たと]える	たとえる	
\\	彼は彼女をバラに例えたんだ。	彼[かれ]は 彼女[かのじょ]をバラに 例[たと]えたんだ。	かれ は かのじょ を ばら に たとえた ん だ	
\\	彼[かれ]は 彼女[かのじょ]をバラに
\\	んだ。			
\\	例題	例題[れいだい]	れいだい	
\\	みんなで例題を解いてみましょう。	みんなで 例題[れいだい]を 解[と]いてみましょう。	みんな で れいだい を といて みましょう	
\\	みんなで
\\	を 解[と]いてみましょう。			
\\	例年	例年[れいねん]	れいねん	
\\	例年8月は雨が少ないね。	例年[れいねん] 
\\	月[がつ]は 雨[あめ]が 少[すく]ないね。	れいねん 
\\	がつ は あめ が すくない ね	
\\	月[がつ]は 雨[あめ]が 少[すく]ないね。			
\\	例え	例[たと]え	たとえ	
\\	例えを上げると話が分かり易くなる。	例[たと]えを 上[あ]げると 話[はなし]が 分[わ]かり 易[やす]くなる。	たとえ を あげる と はなし が わかり やすく なる	
\\	を 上[あ]げると 話[はなし]が 分[わ]かり 易[やす]くなる。			
\\	とっさに	とっさに	とっさに	
\\	とっさに彼の名前が出てこなかった。	とっさに 彼[かれ]の 名前[なまえ]が 出[で]てこなかった。	とっさ に かれ の なまえ が でてこなかった 。	
\\	に 彼[かれ]の 名前[なまえ]が 出[で]てこなかった。			
\\	余地	余地[よち]	よち	
\\	彼に言い訳の余地はありません。	彼[かれ]に 言[い]い 訳[わけ]の 余地[よち]はありません。	かれ に いいわけ の よち は ありません	
\\	彼[かれ]に 言[い]い 訳[わけ]の
\\	はありません。			
\\	余計	余計[よけい]	よけい	
\\	余計なことを言ってはだめよ。	余計[よけい]なことを 言[い]ってはだめよ。	よけい な こと を いって は だめ よ	
\\	なことを 言[い]ってはだめよ。			
\\	余分	余分[よぶん]	よぶん	
\\	食事は余分に用意してあります。	食事[しょくじ]は 余分[よぶん]に 用意[ようい]してあります。	しょくじ は よぶん に ようい して あります	
\\	食事[しょくじ]は
\\	に 用意[ようい]してあります。			
\\	ぬるぬる	ぬるぬる	ぬるぬる	
\\	うなぎはぬるぬるしていますね。	うなぎはぬるぬるしていますね。	うなぎはぬるぬるしていますね。	
\\	うなぎは
\\	していますね。			
\\	中途	中途[ちゅうと]	ちゅうと	
\\	私は中途採用で入社しました。	私[わたし]は 中途[ちゅうと] 採用[さいよう]で 入社[にゅうしゃ]しました。	わたし は ちゅうと さいよう で にゅうしゃ しました	
\\	私[わたし]は
\\	採用[さいよう]で 入社[にゅうしゃ]しました。			
\\	取り締まり	取[と]り 締[し]まり	とりしまり	
\\	違法駐車の取り締まりが厳しくなったの。	違法駐車[いほう ちゅうしゃ]の 取[と]り 締[し]まりが 厳[きび]しくなったの。	いほう ちゅうしゃ の とりしまり が きびしく なった の	
\\	違法駐車[いほう ちゅうしゃ]の
\\	が 厳[きび]しくなったの。			
\\	取り締まる	取[と]り 締[し]まる	とりしまる	
\\	警察が飲酒運転を取り締まっていますよ。	警察[けいさつ]が 飲酒運転[いんしゅ うんてん]を 取[と]り 締[し]まっていますよ。	けいさつ が いんしゅ うんてん を とりしまって います よ	
\\	警察[けいさつ]が 飲酒運転[いんしゅ うんてん]を
\\	よ。			
\\	のろい	のろい	のろい	
\\	この電車は本当にのろいですね。	この 電車[でんしゃ]は 本当[ほんとう]にのろいですね。	この でんしゃ は ほんとうに のろい です ね	
\\	この 電車[でんしゃ]は 本当[ほんとう]に
\\	ですね。			
\\	ぴょんぴょん	ぴょんぴょん	ぴょんぴょん	
\\	カエルがぴょんぴょんはねているね。	カエルがぴょんぴょんはねているね。	かえる が ぴょんぴょん はねて いる ね	
\\	カエルが
\\	はねているね。			
\\	労働者	労働者[ろうどうしゃ]	ろうどうしゃ	
\\	労働者がストをしたそうだよ。	労働者[ろうどうしゃ]がストをしたそうだよ。	ろうどうしゃ が すと を した そう だ よ	
\\	がストをしたそうだよ。			
\\	労働	労働[ろうどう]	ろうどう	
\\	彼らは1日7時間労働している。	彼[かれ]らは1 日7時間[にち 
\\	じかん] 労働[ろうどう]している。	かれら は 
\\	にち 
\\	じかん ろうどう して いる	
\\	彼[かれ]らは1 日7時間[にち 
\\	じかん]
\\	している。			
\\	労力	労力[ろうりょく]	ろうりょく	
\\	仕事には時間と労力が必要です。	仕事[しごと]には 時間[じかん]と 労力[ろうりょく]が 必要[ひつよう]です。	しごと に は じかん と ろうりょく が ひつよう です	
\\	仕事[しごと]には 時間[じかん]と
\\	が 必要[ひつよう]です。			
\\	ブラシ	ブラシ	ブラシ	
\\	犬の毛をブラシでとかしてやったの。	犬[いぬ]の 毛[け]をブラシでとかしてやったの。	いぬ の け を ぶらし で とかして やった の	
\\	犬[いぬ]の 毛[け]を
\\	でとかしてやったの。			
\\	乏しい	乏[とぼ]しい	とぼしい	
\\	彼女はまだ仕事の経験が乏しい。	彼女[かのじょ]はまだ 仕事[しごと]の 経験[けいけん]が 乏[とぼ]しい。	かのじょ は まだ しごと の けいけん が とぼしい。	
\\	彼女[かのじょ]はまだ 仕事[しごと]の 経験[けいけん]が
\\	不幸せ	不幸[ふしあわ]せ	ふしあわせ	
\\	彼ほど不幸せな人はいないな。	彼[かれ]ほど 不幸[ふしあわ]せな 人[ひと]はいないな。	かれ ほど ふしあわせ な ひと は いない な	
\\	彼[かれ]ほど
\\	な 人[ひと]はいないな。			
\\	マヨネーズ	マヨネーズ	マヨネーズ	
\\	サラダにマヨネーズをかけたの。	サラダにマヨネーズをかけたの。	さらだ に まよねーず を かけた の	
\\	サラダに
\\	をかけたの。			
\\	低気圧	低気圧[ていきあつ]	ていきあつ	
\\	低気圧が近づいています。	低気圧[ていきあつ]が 近[ちか]づいています。	ていきあつ が ちかづいて います	
\\	が 近[ちか]づいています。			
\\	だらしない	だらしない	だらしない	
\\	だらしない格好をしないでください。	だらしない 格好[かっこう]をしないでください。	だらしない かっこう を しない で ください	
\\	格好[かっこう]をしないでください。			
\\	とうもろこし	とうもろこし	とうもろこし	
\\	このとうもろこしは甘くておいしい。	このとうもろこしは 甘[あま]くておいしい。	この とうもろこし は あまく て おいしい	
\\	この
\\	は 甘[あま]くておいしい。			
\\	伸ばす	伸[の]ばす	のばす	
\\	ストレッチで筋肉を伸ばしましょう。	ストレッチで 筋肉[きんにく]を 伸[の]ばしましょう。	すとれっち で きんにく を のばしましょう	
\\	ストレッチで 筋肉[きんにく]を
\\	倍	倍[ばい]	ばい	
\\	以前の収入は今の倍はあった。	以前[いぜん]の 収入[しゅうにゅう]は 今[いま]の 倍[ばい]はあった。	いぜん の しゅうにゅう は いま の ばい は あった	
\\	以前[いぜん]の 収入[しゅうにゅう]は 今[いま]の
\\	はあった。			
\\	ひらひら	ひらひら	ひらひら	
\\	木の葉がひらひらと落ちたの。	木[こ]の 葉[は]がひらひらと 落[お]ちたの。	このは が ひらひら と おちた の	
\\	木[こ]の 葉[は]が
\\	と 落[お]ちたの。			
\\	乗り越える	乗[の]り 越[こ]える	のりこえる	
\\	彼は悲しみを乗り越えて強く生きた。	彼[かれ]は 悲[かな]しみを 乗[の]り 越[こ]えて 強[つよ]く 生[い]きた。	かれ は かなしみ を のりこえて つよく いきた	
\\	彼[かれ]は 悲[かな]しみを
\\	強[つよ]く 生[い]きた。			
\\	乗り越し	乗[の]り 越[こ]し	のりこし	
\\	乗り越しを機械で精算したの。	乗[の]り 越[こ]しを 機械[きかい]で 精算[せいさん]したの。	のりこし を きかい で せいさん した の	
\\	を 機械[きかい]で 精算[せいさん]したの。			
\\	乗り越す	乗[の]り 越[こ]す	のりこす	
\\	居眠りして降りる駅を乗り越した。	居眠[いねむ]りして 降[お]りる 駅[えき]を 乗[の]り 越[こ]した。	いねむり して おりる えき を のりこした	
\\	居眠[いねむ]りして 降[お]りる 駅[えき]を
\\	ぽかぽか	ぽかぽか	ぽかぽか	
\\	今日はぽかぽか暖かい日です。	今日[きょう]はぽかぽか 暖[あたた]かい 日[ひ]です。	きょう は ぽかぽか あたたかい ひ です	
\\	今日[きょう]は
\\	暖[あたた]かい 日[ひ]です。			
\\	よる	よる	よる	
\\	人は見かけによらないな。	人[ひと]は 見[み]かけによらないな。	ひと は みかけ に よらない な	
\\	人[ひと]は 見[み]かけに
\\	な。			
\\	反射	反射[はんしゃ]	はんしゃ	
\\	車のライトが反射していますね。	車[くるま]のライトが 反射[はんしゃ]していますね。	くるま の らいと が はんしゃ して います ね	
\\	車[くるま]のライトが
\\	していますね。			
\\	優勝	優勝[ゆうしょう]	ゆうしょう	
\\	私たちのチームが優勝しました。	私[わたし]たちのチームが 優勝[ゆうしょう]しました。	わたしたち の ちーむ が ゆうしょう しました	
\\	私[わたし]たちのチームが
\\	しました。			
\\	優秀	優秀[ゆうしゅう]	ゆうしゅう	
\\	彼はとても優秀な生徒です。	彼[かれ]はとても 優秀[ゆうしゅう]な 生徒[せいと]です。	かれ は とても ゆうしゅう な せいと です	
\\	彼[かれ]はとても
\\	な 生徒[せいと]です。			
\\	だぶだぶ	だぶだぶ	だぶだぶ	
\\	この服は大き過ぎてだぶだぶです。	この 服[ふく]は 大[おお]き 過[す]ぎてだぶだぶです。	この ふく は おおき すぎて だぶだぶ です	
\\	この 服[ふく]は 大[おお]き 過[す]ぎて
\\	です。			
\\	半導体	半導体[はんどうたい]	はんどうたい	
\\	半導体は様々な製品に使われているよ。	半導体[はんどうたい]は 様々[さまざま]な 製品[せいひん]に 使[つか]われているよ。	はんどうたい は さまざま な せいひん に つかわれて いる よ	
\\	は 様々[さまざま]な 製品[せいひん]に 使[つか]われているよ。			
\\	つるつる	つるつる	つるつる	
\\	床がつるつるすべります。	床[ゆか]がつるつるすべります。	ゆか が つるつる すべります	
\\	床[ゆか]が
\\	すべります。			
\\	努める	努[つと]める	つとめる	
\\	良い成績が取れるように努めます。	良[い]い 成績[せいせき]が 取[と]れるように 努[つと]めます。	いい せいせき が とれる よう に つとめます	
\\	良[い]い 成績[せいせき]が 取[と]れるように
\\	トースト	トースト	トースト	
\\	私は毎朝トーストを2枚食べます。	私[わたし]は 毎朝[まいあさ]トーストを2 枚食[まい た]べます。	わたし は まいあさ とーすと を 
\\	まい たべます	
\\	私[わたし]は 毎朝[まいあさ]
\\	を2 枚食[まい た]べます。			
\\	中身	中身[なかみ]	なかみ	
\\	かばんの中身を見せてください。	かばんの 中身[なかみ]を 見[み]せてください。	かばん の なかみ を みせて ください	
\\	かばんの
\\	を 見[み]せてください。			
\\	どきっと	どきっと	どきっと	
\\	突然の質問にどきっとしたよ。	突然[とつぜん]の 質問[しつもん]にどきっとしたよ。	とつぜん の しつもん に どきっと した よ	
\\	突然[とつぜん]の 質問[しつもん]に
\\	したよ。			
\\	のろのろ	のろのろ	のろのろ	
\\	車は雪道をのろのろと走ったの。	車[くるま]は 雪道[ゆきみち]をのろのろと 走[はし]ったの。	くるま は ゆきみち を のろのろ と はしった の	
\\	車[くるま]は 雪道[ゆきみち]を
\\	と 走[はし]ったの。			
\\	仲良く	仲良[なかよ]く	なかよく	
\\	あの夫婦は仲良く暮らしているよ。	あの 夫婦[ふうふ]は 仲良[なかよ]く 暮[く]らしているよ。	あの ふうふ は なかよく くらして いる よ	
\\	あの 夫婦[ふうふ]は
\\	暮[く]らしているよ。			
\\	仲良し	仲良[なかよ]し	なかよし	
\\	あの3人組は仲良しですね。	あの3 人組[にんぐみ]は 仲良[なかよ]しですね。	あの 
\\	にんぐみ は なかよし です ね	
\\	あの3 人組[にんぐみ]は
\\	ですね。			
\\	仲直り	仲直[なかなお]り	なかなおり	
\\	けんかした友達と仲直りしました。	けんかした 友達[ともだち]と 仲直[なかなお]りしました。	けんか した ともだち と なかなおり しました	
\\	けんかした 友達[ともだち]と
\\	しました。			
\\	仲	仲[なか]	なか	
\\	あなたたちは仲がいいですね。	あなたたちは 仲[なか]がいいですね。	あなたたち は なか が いい です ね	
\\	あなたたちは
\\	がいいですね。			
\\	仲間	仲間[なかま]	なかま	
\\	彼には仲間がたくさんいる。	彼[かれ]には 仲間[なかま]がたくさんいる。	かれ に は なかま が たくさん いる	
\\	彼[かれ]には
\\	がたくさんいる。			
\\	仲人	仲人[なこうど]	なこうど	
\\	あの夫婦は私たちの仲人です。	あの 夫婦[ふうふ]は 私[わたし]たちの 仲人[なこうど]です。	あの ふうふ は わたしたち の なこうど です	
\\	あの 夫婦[ふうふ]は 私[わたし]たちの
\\	です。			
\\	ぴかぴか	ぴかぴか	ぴかぴか	
\\	新車はぴかぴかですね。	新車[しんしゃ]はぴかぴかですね。	しんしゃ は ぴかぴか です ね	
\\	新車[しんしゃ]は
\\	ですね。			
\\	ぺこぺこ	ぺこぺこ	ぺこぺこ	
\\	彼は上司にぺこぺこしている。	彼[かれ]は 上司[じょうし]にぺこぺこしている。	かれ は じょうし に ぺこぺこ して いる	
\\	彼[かれ]は 上司[じょうし]に
\\	している。			
\\	互い	互[たが]い	たがい	
\\	互いの話をよく聞きなさい。	互[たが]いの 話[はなし]をよく 聞[き]きなさい。	たがい の はなし を よく ききなさい	
\\	の 話[はなし]をよく 聞[き]きなさい。			
\\	ぺらぺら	ぺらぺら	ぺらぺら	
\\	この本は薄くてぺらぺらですね。	この 本[ほん]は 薄[うす]くてぺらぺらですね。	この ほん は うすくて ぺらぺら です ね	
\\	この 本[ほん]は 薄[うす]くて
\\	ですね。			
\\	初恋	初恋[はつこい]	はつこい	
\\	私の初恋は小学生の時です。	私[わたし]の 初恋[はつこい]は 小学生[しょうがくせい]の 時[とき]です。	わたし の はつこい は しょうがくせい の とき です	
\\	私[わたし]の
\\	は 小学生[しょうがくせい]の 時[とき]です。			
\\	ほこり	ほこり	ほこり	
\\	この部屋はほこりだらけですね。	この 部屋[へや]はほこりだらけですね。	この へや は ほこり だらけ です ね 。	
\\	この 部屋[へや]は
\\	だらけですね。			
\\	単純	単純[たんじゅん]	たんじゅん	
\\	彼は単純な人です。	彼[かれ]は 単純[たんじゅん]な 人[ひと]です。	かれ は たんじゅん な ひと です	
\\	彼[かれ]は
\\	な 人[ひと]です。			
\\	ポット	ポット	ポット	
\\	お湯はポットに入っています。	お 湯[ゆ]はポットに 入[はい]っています。	おゆ は ぽっと に はいって います	
\\	お 湯[ゆ]は
\\	に 入[はい]っています。			
\\	ミュージック	ミュージック	ミュージック	
\\	彼はソウルミュージックをよく聞くの。	彼[かれ]はソウルミュージックをよく 聞[き]くの。	かれ は そうる みゅーじっく を よく きく の	
\\	彼[かれ]はソウル
\\	をよく 聞[き]くの。			
\\	保護	保護[ほご]	ほご	
\\	みんなで環境を保護しましょう。	みんなで 環境[かんきょう]を 保護[ほご]しましょう。	みんな で かんきょう を ほご しましょう	
\\	みんなで 環境[かんきょう]を
\\	しましょう。			
\\	保護者	保護者[ほごしゃ]	ほごしゃ	
\\	私はこの子の保護者です。	私[わたし]はこの 子[こ]の 保護者[ほごしゃ]です。	わたし は この こ の ほごしゃ です	
\\	私[わたし]はこの 子[こ]の
\\	です。			
\\	ロマンチック	ロマンチック	ロマンチック	
\\	彼女はロマンチックな人です。	彼女[かのじょ]はロマンチックな 人[ひと]です。	かのじょ は ろまんちっく な ひと です	
\\	彼女[かのじょ]は
\\	な 人[ひと]です。			
\\	わくわく	わくわく	わくわく	
\\	遠足が楽しみでわくわくしています。	遠足[えんそく]が 楽[たの]しみでわくわくしています。	えんそく が たのしみ で わくわく して います	
\\	遠足[えんそく]が 楽[たの]しみで
\\	しています。			
\\	偽物	偽物[にせもの]	にせもの	
\\	彼らが売っていたのは偽物だ。	彼[かれ]らが 売[う]っていたのは 偽物[にせもの]だ。	かれら が うって いた の は にせもの だ	
\\	彼[かれ]らが 売[う]っていたのは
\\	だ。			
\\	ノーベル賞	ノーベル 賞[しょう]	ノーベルしょう	
\\	日本人がノーベル賞を取ったよ。	日本人[にほんじん]がノーベル 賞[しょう]を 取[と]ったよ。	にほんじん が のーべるしょう を とった よ	
\\	日本人[にほんじん]が
\\	を 取[と]ったよ。			
\\	わさび	わさび	わさび	
\\	刺身にわさびは欠かせませんね。	刺身[さしみ]にわさびは 欠[か]かせませんね。	さしみ に わさび は かかせません ね	
\\	刺身[さしみ]に
\\	は 欠[か]かせませんね。			
\\	不得意	不得意[ふとくい]	ふとくい	
\\	私は数学が不得意です。	私[わたし]は 数学[すうがく]が 不得意[ふとくい]です。	わたし は すうがく が ふとくい です	
\\	私[わたし]は 数学[すうがく]が
\\	です。			
\\	チキン	チキン	チキン	
\\	チキンソテーは私の大好物です。	チキンソテーは 私[わたし]の 大好物[だいこうぶつ]です。	ちきんそてー は わたし の だいこうぶつ です	
\\	ソテーは 私[わたし]の 大好物[だいこうぶつ]です。			
\\	とじる	とじる	とじる	
\\	書類はこのファイルにとじてください。	書類[しょるい]はこのファイルにとじてください。	しょるい は この ふぁいる に とじて ください	
\\	書類[しょるい]はこのファイルに
\\	ください。			
\\	分散	分散[ぶんさん]	ぶんさん	
\\	その会社はいろいろな国に投資を分散しているね。	その 会社[かいしゃ]はいろいろな 国[くに]に 投資[とうし]を 分散[ぶんさん]しているね。	その かいしゃ は いろいろ な くに に とうし を ぶんさん して いる ね	
\\	その 会社[かいしゃ]はいろいろな 国[くに]に 投資[とうし]を
\\	しているね。			
\\	にやにや	にやにや	にやにや	
\\	なぜか彼はにやにやしています。	なぜか 彼[かれ]はにやにやしています。	なぜ か かれ は にやにや して います	
\\	なぜか 彼[かれ]は
\\	しています。			
\\	ばら	ばら	ばら	
\\	お祝いにばらの花束を贈りました。	お 祝[いわ]いにばらの 花束[はなたば]を 贈[おく]りました。	おいわい に ばら の はなたば を おくりました	
\\	お 祝[いわ]いに
\\	の 花束[はなたば]を 贈[おく]りました。			
\\	まな板	まな 板[いた]	まないた	
\\	魚を切った後、まな板を洗ったの。	魚[さかな]を 切[き]った 後[あと]、まな 板[いた]を 洗[あら]ったの。	さかな を きった あと、 まないた を あらった の	
\\	魚[さかな]を 切[き]った 後[あと]、
\\	を 洗[あら]ったの。			
\\	ほうれん草	ほうれん 草[そう]	ほうれんそう	
\\	ほうれん草はビタミンが豊富です。	ほうれん 草[そう]はビタミンが 豊富[ほうふ]です。	ほうれんそう は びたみん が ほうふ です 。	
\\	はビタミンが 豊富[ほうふ]です。			
\\	パンティー	パンティー	パンティー	
\\	白いパンティーを買いました。	白[しろ]いパンティーを 買[か]いました。	しろい ぱんてぃー を かいました	
\\	白[しろ]い
\\	を 買[か]いました。			
\\	取り扱う	取[と]り 扱[あつか]う	とりあつかう	
\\	当店ではお酒を取り扱っておりません。	当店[とうてん]ではお 酒[さけ]を 取[と]り 扱[あつか]っておりません。	とうてん で は お さけ を とりあつかって おりません	
\\	当店[とうてん]ではお 酒[さけ]を
\\	おりません。			
\\	取り扱い	取[と]り 扱[あつか]い	とりあつかい	
\\	この機械は取り扱いに注意してください。	この 機械[きかい]は 取[と]り 扱[あつか]いに 注意[ちゅうい]してください。	この きかい は とりあつかい に ちゅうい して ください	
\\	この 機械[きかい]は
\\	に 注意[ちゅうい]してください。			
\\	中級	中級[ちゅうきゅう]	ちゅうきゅう	
\\	彼は中級レベルの日本語を習っています。	彼[かれ]は 中級[ちゅうきゅう]レベルの 日本語[にほんご]を 習[なら]っています。	かれ は ちゅうきゅう れべる の にほんご を ならって います	
\\	彼[かれ]は
\\	レベルの 日本語[にほんご]を 習[なら]っています。			
\\	ふわふわ	ふわふわ	ふわふわ	
\\	ふわふわの布団に寝たよ。	ふわふわの 布団[ふとん]に 寝[ね]たよ。	ふわふわ の ふとん に ねた よ	
\\	の 布団[ふとん]に 寝[ね]たよ。			
\\	まずい	まずい	まずい	
\\	今日中に返答しなければまずい。	今日中[きょうじゅう]に 返答[へんとう]しなければまずい。	きょうじゅう に へんとう しなければ まずい	
\\	今日中[きょうじゅう]に 返答[へんとう]しなければ
\\	分析	分析[ぶんせき]	ぶんせき	
\\	今の経済の動きを分析しています。	今[いま]の 経済[けいざい]の 動[うご]きを 分析[ぶんせき]しています。	いま の けいざい の うごき を ぶんせき して います	
\\	今[いま]の 経済[けいざい]の 動[うご]きを
\\	しています。			
\\	保健	保健[ほけん]	ほけん	
\\	保健室で少し休んだわ。	保健[ほけん] 室[しつ]で 少[すこ]し 休[やす]んだわ。	ほけんしつ で すこし やすんだ わ	
\\	室[しつ]で 少[すこ]し 休[やす]んだわ。			
\\	もむ	もむ	もむ	
\\	肩をもんでください。	肩[かた]をもんでください。	かた を もんで ください	
\\	肩[かた]を
\\	ください。			
\\	中毒	中毒[ちゅうどく]	ちゅうどく	
\\	そのホテルで食中毒が発生したんだ。	そのホテルで 食[しょく] 中毒[ちゅうどく]が 発生[はっせい]したんだ。	その ほてる で しょくちゅうどく が はっせい した ん だ	
\\	そのホテルで 食[しょく]
\\	が 発生[はっせい]したんだ。			
\\	ばい菌	ばい 菌[きん]	ばいきん	
\\	傷口にばい菌が入った。	傷口[きずぐち]にばい 菌[きん]が 入[はい]った。	きずぐち に ばいきん が はいった	
\\	傷口[きずぐち]に
\\	が 入[はい]った。			
\\	センター	センター	センター	
\\	このセンターで工場全体を管理しています。	このセンターで 工場全体[こうじょう ぜんたい]を 管理[かんり]しています。	この せんたー で こうじょう ぜんたい を かんり して います	
\\	この
\\	で 工場全体[こうじょう ぜんたい]を 管理[かんり]しています。			
\\	不潔	不潔[ふけつ]	ふけつ	
\\	体を不潔にしているとかゆくなりますよ。	体[からだ]を 不潔[ふけつ]にしているとかゆくなりますよ。	からだ を ふけつ に して いる と かゆく なります よ	
\\	体[からだ]を
\\	にしているとかゆくなりますよ。			
\\	ぞんざい	ぞんざい	ぞんざい	
\\	そんなぞんざいな口のききかたはよくないよ。	そんなぞんざいな 口[くち]のききかたはよくないよ。	そんな ぞんざい な くち の ききかた は よくない よ	
\\	そんな
\\	な 口[くち]のききかたはよくないよ。			
\\	たんす	たんす	たんす	
\\	昨日、たんすを買いました。	昨日[きのう]、たんすを 買[か]いました。	きのう たんす を かいました	
\\	昨日[きのう]、
\\	を 買[か]いました。			
\\	博士	博士[はくし]	はくし	
\\	彼は数学の博士だそうです。	彼[かれ]は 数学[すうがく]の 博士[はくし]だそうです。	かれ は すうがく の はくし だ そう です	
\\	彼[かれ]は 数学[すうがく]の
\\	だそうです。			
\\	博士	博士[はかせ]	はかせ	
\\	彼は物理学の博士です。	彼[かれ]は 物理学[ぶつりがく]の 博士[はかせ]です。	かれ は ぶつりがく の はかせ です	
\\	彼[かれ]は 物理学[ぶつりがく]の
\\	です。			
\\	チェンジ	チェンジ	チェンジ	
\\	坂道でギアーをチェンジしたよ。	坂道[さかみち]でギアーをチェンジしたよ。	さかみち で ぎあー を ちぇんじ した よ	
\\	坂道[さかみち]でギアーを
\\	したよ。			
\\	催す	催[もよお]す	もよおす	
\\	静かな音楽で眠気を催したよ。	静[しず]かな 音楽[おんがく]で 眠気[ねむけ]を 催[もよお]したよ。	しずか な おんがく で ねむけ を もよおした よ	
\\	静[しず]かな 音楽[おんがく]で 眠気[ねむけ]を
\\	よ。			
\\	催し	催[もよお]し	もよおし	
\\	学校の催しに参加しました。	学校[がっこう]の 催[もよお]しに 参加[さんか]しました。	がっこう の もよおし に さんか しました	
\\	学校[がっこう]の
\\	に 参加[さんか]しました。			
\\	典型的	典型的[てんけいてき]	てんけいてき	
\\	彼女は典型的なイタリア人です。	彼女[かのじょ]は 典型的[てんけいてき]なイタリア 人[じん]です。	かのじょ は てんけいてき な いたりあじん です	
\\	彼女[かのじょ]は
\\	なイタリア 人[じん]です。			
\\	つかまる	つかまる	つかまる	
\\	しっかりとつかまっていてください。	しっかりとつかまっていてください。	しっかりとつかまっていてください。	
\\	しっかりと
\\	いてください。			
\\	典型	典型[てんけい]	てんけい	
\\	彼は職人の典型です。	彼[かれ]は 職人[しょくにん]の 典型[てんけい]です。	かれ は しょくにん の てんけい です	
\\	彼[かれ]は 職人[しょくにん]の
\\	です。			
\\	免税	免税[めんぜい]	めんぜい	
\\	この商品は免税です。	この 商品[しょうひん]は 免税[めんぜい]です。	この しょうひん は めんぜい です	
\\	この 商品[しょうひん]は
\\	です。			
\\	つながり	つながり	つながり	
\\	この文のつながりはおかしいです。	この 文[ぶん]のつながりはおかしいです。	この ぶん の つながり は おかしい です	
\\	この 文[ぶん]の
\\	はおかしいです。			
\\	免許	免許[めんきょ]	めんきょ	
\\	やっと運転免許を手に入れたよ。	やっと 運転[うんてん] 免許[めんきょ]を 手[て]に 入[い]れたよ。	やっと うんてん めんきょ を て に いれた よ	
\\	やっと 運転[うんてん]
\\	を 手[て]に 入[い]れたよ。			
\\	動詞	動詞[どうし]	どうし	
\\	「食べる」は動詞です。	
\\	食[た]べる」は 動詞[どうし]です。	たべる は どうし です	
\\	食[た]べる」は
\\	です。			
\\	副詞	副詞[ふくし]	ふくし	
\\	「ゆっくり歩く」の「ゆっくり」は副詞です。	「ゆっくり 歩[ある]く」の「ゆっくり」は 副詞[ふくし]です。	ゆっくり あるく の ゆっくり は ふくし です	
\\	「ゆっくり 歩[ある]く」の「ゆっくり」は
\\	です。			
\\	代名詞	代名詞[だいめいし]	だいめいし	
\\	「彼」は代名詞の一つです。	
\\	彼」[かれ]は 代名詞[だいめいし]の 一[ひと]つです。	かれ は だいめいし の ひとつ です	
\\	彼」[かれ]は
\\	の 一[ひと]つです。			
\\	つながる	つながる	つながる	
\\	電話がつながりません。	電話[でんわ]がつながりません。	でんわ が つながりません	
\\	電話[でんわ]が
\\	伝統	伝統[でんとう]	でんとう	
\\	私は日本の伝統を大切にします。	私[わたし]は 日本[にっぽん]の 伝統[でんとう]を 大切[たいせつ]にします。	わたし は にっぽん の でんとう を たいせつ に します	
\\	私[わたし]は 日本[にっぽん]の
\\	を 大切[たいせつ]にします。			
\\	伝わる	伝[つた]わる	つたわる	
\\	漢字は中国から伝わった。	漢字[かんじ]は 中国[ちゅうごく]から 伝[つた]わった。	かんじ は ちゅうごく から つたわった	
\\	漢字[かんじ]は 中国[ちゅうごく]から
\\	伝説	伝説[でんせつ]	でんせつ	
\\	彼は数々の伝説を残しました。	彼[かれ]は 数々[かずかず]の 伝説[でんせつ]を 残[のこ]しました。	かれ は かずかず の でんせつ を のこしました	
\\	彼[かれ]は 数々[かずかず]の
\\	を 残[のこ]しました。			
\\	伝言	伝言[でんごん]	でんごん	
\\	課長から伝言があります。	課長[かちょう]から 伝言[でんごん]があります。	かちょう から でんごん が あります	
\\	課長[かちょう]から
\\	があります。			
\\	伝染	伝染[でんせん]	でんせん	
\\	その国ではコレラの伝染が蔓延しています。	その 国[くに]ではコレラの 伝染[でんせん]が 蔓延[まんえん]しています。	その くに で は これら の でんせん が まんえん して います	
\\	その 国[くに]ではコレラの
\\	が 蔓延[まんえん]しています。			
\\	つばめ	つばめ	つばめ	
\\	つばめが飛んでいますね。	つばめが 飛[と]んでいますね。	つばめ が とんで います ね	
\\	が 飛[と]んでいますね。			
\\	つぶやく	つぶやく	つぶやく	
\\	彼は一人で何かつぶやいていたの。	彼[かれ]は 一人[ひとり]で 何[なに]かつぶやいていたの。	かれ は ひとり で なにか つぶやいて いた の	
\\	彼[かれ]は 一人[ひとり]で 何[なに]かつぶやいていたの。			
\\	つぶる	つぶる	つぶる	
\\	目をつぶってください。	目[め]をつぶってください。	め を つぶって ください	
\\	目[め]を
\\	ください。			
\\	到着	到着[とうちゃく]	とうちゃく	
\\	夜9時に東京に到着しました。	夜9時[よる 
\\	じ]に 東京[とうきょう]に 到着[とうちゃく]しました。	よる 
\\	じ に とうきょう に とうちゃく しました	
\\	夜9時[よる 
\\	じ]に 東京[とうきょう]に
\\	しました。			
\\	到達	到達[とうたつ]	とうたつ	
\\	ついに山頂に到達しました。	ついに 山頂[さんちょう]に 到達[とうたつ]しました。	ついに さんちょう に とうたつ しました	
\\	ついに 山頂[さんちょう]に
\\	しました。			
\\	倒産	倒産[とうさん]	とうさん	
\\	会社が倒産しました。	会社[かいしゃ]が 倒産[とうさん]しました。	かいしゃ が とうさん しました	
\\	会社[かいしゃ]が
\\	しました。			
\\	つぼみ	つぼみ	つぼみ	
\\	花のつぼみがたくさんついていますね。	花[はな]のつぼみがたくさんついていますね。	はな の つぼみ が たくさん ついています ね 。	
\\	花[はな]の
\\	がたくさんついていますね。			
\\	つまずく	つまずく	つまずく	
\\	石につまずきました。	石[いし]につまずきました。	いし に つまずきました 。	
\\	石[いし]に
\\	ツル	ツル	ツル	
\\	湖にツルがいますよ。	湖[みずうみ]にツルがいますよ。	みずうみ に つる が います よ	
\\	湖[みずうみ]に
\\	がいますよ。			
\\	兵士	兵士[へいし]	へいし	
\\	その爆撃で兵士が3人負傷しました。	その 爆撃[ばくげき]で 兵士[へいし]が 3人負傷[さんにん ふしょう]しました。	その ばくげき で へいし が さんにん ふしょう しました	
\\	その 爆撃[ばくげき]で
\\	が 3人負傷[さんにん ふしょう]しました。			
\\	兵器	兵器[へいき]	へいき	
\\	あの国は強力な兵器を持っている。	あの 国[くに]は 強力[きょうりょく]な 兵器[へいき]を 持[も]っている。	あの くに は きょうりょく な へいき を もって いる	
\\	あの 国[くに]は 強力[きょうりょく]な
\\	を 持[も]っている。			
\\	兵隊	兵隊[へいたい]	へいたい	
\\	兵隊が銃をかまえていたよ。	兵隊[へいたい]が 銃[じゅう]をかまえていたよ。	へいたい が じゅう を かまえて いた よ	
\\	が 銃[じゅう]をかまえていたよ。			
\\	でこぼこ	でこぼこ	でこぼこ	
\\	この道はでこぼこしていますね。	この 道[みち]はでこぼこしていますね。	この みち は でこぼこ して います ね	
\\	この 道[みち]は
\\	していますね。			
\\	てのひら	てのひら	てのひら	
\\	てのひらに汗をかきました。	てのひらに 汗[あせ]をかきました。	てのひら に あせ を かきました	
\\	に 汗[あせ]をかきました。			
\\	乱暴	乱暴[らんぼう]	らんぼう	
\\	人に乱暴してはいけません。	人[ひと]に 乱暴[らんぼう]してはいけません。	ひと に らんぼう して は いけません	
\\	人[ひと]に
\\	してはいけません。			
\\	とがる	とがる	とがる	
\\	この鉛筆はとがっていますね。	この 鉛筆[えんぴつ]はとがっていますね。	この えんぴつ は とがって います ね	
\\	この 鉛筆[えんぴつ]は
\\	ね。			
\\	予防	予防[よぼう]	よぼう	
\\	虫歯は予防できます。	虫歯[むしば]は 予防[よぼう]できます。	むしば は よぼう できます	
\\	虫歯[むしば]は
\\	できます。			
\\	とんかつ	とんかつ	とんかつ	
\\	昼食にとんかつを食べました。	昼食[ちゅうしょく]にとんかつを 食[た]べました。	ちゅうしょく に とん かつ を たべました 。	
\\	昼食[ちゅうしょく]に
\\	を 食[た]べました。			
\\	不機嫌	不機嫌[ふきげん]	ふきげん	
\\	この頃、あの人は不機嫌ですね。	この 頃[ごろ]、あの 人[ひと]は 不機嫌[ふきげん]ですね。	このごろ あの ひと は ふきげん です ね	
\\	この 頃[ごろ]、あの 人[ひと]は
\\	ですね。			
\\	どんなに	どんなに	どんなに	
\\	彼女はどんなに嬉しかったろう。	彼女[かのじょ]はどんなに 嬉[うれ]しかったろう。	かのじょ は どんなに うれしかったろう	
\\	彼女[かのじょ]は
\\	嬉[うれ]しかったろう。			
\\	到底	到底[とうてい]	とうてい	
\\	今夜中に東京に着くのは到底無理でしょう。	今夜中[こんやじゅう]に 東京[とうきょう]に 着[つ]くのは 到底[とうてい] 無理[むり]でしょう。	こんやじゅう に とうきょう に つく の は とうてい むり でしょう	
\\	今夜中[こんやじゅう]に 東京[とうきょう]に 着[つ]くのは
\\	無理[むり]でしょう。			
\\	体系	体系[たいけい]	たいけい	
\\	勤務の体系を見直すことにしました。	勤務[きんむ]の 体系[たいけい]を 見直[みなお]すことにしました。	きんむ の たいけい を みなおす こと に しました	
\\	勤務[きんむ]の
\\	を 見直[みなお]すことにしました。			
\\	なめる	なめる	なめる	
\\	犬に顔をなめられました。	犬[いぬ]に 顔[かお]をなめられました。	いぬ に かお を なめられました	
\\	犬[いぬ]に 顔[かお]を
\\	にらむ	にらむ	にらむ	
\\	彼女は私をにらんだの。	彼女[かのじょ]は 私[わたし]をにらんだの。	かのじょ は わたし を にらんだ の	
\\	彼女[かのじょ]は 私[わたし]を
\\	の。			
\\	人柄	人柄[ひとがら]	ひとがら	
\\	みんな彼の温かい人柄が大好きだよ。	みんな 彼[かれ]の 温[あたた]かい 人柄[ひとがら]が 大好[だいす]きだよ。	みんな かれ の あたたかい ひとがら が だいすき だ よ	
\\	みんな 彼[かれ]の 温[あたた]かい
\\	が 大好[だいす]きだよ。			
\\	ねずみ	ねずみ	ねずみ	
\\	台所にねずみが出た。	台所[だいどころ]にねずみが 出[で]た。	だいどころ に ねずみ が でた	
\\	台所[だいどころ]に
\\	が 出[で]た。			
\\	のぞく	のぞく	のぞく	
\\	ドアの隙間から中をのぞいたんだ。	ドアの 隙間[すきま]から 中[なか]をのぞいたんだ。	どあ の すきま から なか を のぞいた ん だ	
\\	ドアの 隙間[すきま]から 中[なか]を
\\	んだ。			
\\	のんき	のんき	のんき	
\\	彼はずいぶんのんきなやつだ。	彼[かれ]はずいぶんのんきなやつだ。	かれ は ずいぶん のんき な やつ だ	
\\	彼[かれ]はずいぶん
\\	なやつだ。			
\\	余裕	余裕[よゆう]	よゆう	
\\	出発まで時間の余裕があります。	出発[しゅっぱつ]まで 時間[じかん]の 余裕[よゆう]があります。	しゅっぱつ まで じかん の よゆう が あります	
\\	出発[しゅっぱつ]まで 時間[じかん]の
\\	があります。			
\\	入浴	入浴[にゅうよく]	にゅうよく	
\\	私は毎日寝る前に入浴します。	私[わたし]は 毎日寝[まいにち ね]る 前[まえ]に 入浴[にゅうよく]します。	わたし は まいにち ねる まえ に にゅうよく します	
\\	私[わたし]は 毎日寝[まいにち ね]る 前[まえ]に
\\	します。			
\\	はえ	はえ	はえ	
\\	はえがうるさく飛んでいるね。	はえがうるさく 飛[と]んでいるね。	はえ が うるさく とん でいる ね 。	
\\	がうるさく 飛[と]んでいるね。			
\\	ばかばかしい	ばかばかしい	ばかばかしい	
\\	ばかばかしい映画だが面白かった。	ばかばかしい 映画[えいが]だが 面白[おもしろ]かった。	ばかばかしい えいが だ が おもしろかった	
\\	映画[えいが]だが 面白[おもしろ]かった。			
\\	冷淡	冷淡[れいたん]	れいたん	
\\	彼は彼女に冷淡な態度をとったね。	彼[かれ]は 彼女[かのじょ]に 冷淡[れいたん]な 態度[たいど]をとったね。	かれ は かのじょ に れいたん な たいど を とった ね	
\\	彼[かれ]は 彼女[かのじょ]に
\\	な 態度[たいど]をとったね。			
\\	はかり	はかり	はかり	
\\	ケーキを作る時ははかりを使います。	ケーキを 作[つく]る 時[とき]ははかりを 使[つか]います。	けーき を つくる とき は はかり を つかいます 。	
\\	ケーキを 作[つく]る 時[とき]は
\\	を 使[つか]います。			
\\	はしご	はしご	はしご	
\\	はしごを使って屋根に上りました。	はしごを 使[つか]って 屋根[やね]に 上[のぼ]りました。	はしご を つかって やね に のぼりました	
\\	を 使[つか]って 屋根[やね]に 上[のぼ]りました。			
\\	はだし	はだし	はだし	
\\	砂浜をはだしでかけ回ったんだ。	砂浜[すなはま]をはだしでかけ 回[まわ]ったんだ。	すなはま を はだし で かけまわった ん だ	
\\	砂浜[すなはま]を
\\	でかけ 回[まわ]ったんだ。			
\\	半径	半径[はんけい]	はんけい	
\\	この円の半径は5センチです。	この 円[えん]の 半径[はんけい]は5センチです。	この えん の はんけい は 
\\	せんち です	
\\	この 円[えん]の
\\	は5センチです。			
\\	丼	丼[どんぶり]	どんぶり	
\\	彼はご飯を丼で食べたよ。	彼[かれ]はご 飯[はん]を 丼[どんぶり]で 食[た]べたよ。	かれ は ごはん を どんぶり で たべた よ	
\\	彼[かれ]はご 飯[はん]を
\\	で 食[た]べたよ。			
\\	バツ	バツ	バツ	
\\	バツが三つあるので85点です。	バツが 三[みっ]つあるので85 点[てん]です。	ばつ が みっつ ある の で 
\\	てん です	
\\	が 三[みっ]つあるので85 点[てん]です。			
\\	はねる	はねる	はねる	
\\	彼は車にはねられたけど無事だったの。	彼[かれ]は 車[くるま]にはねられたけど 無事[ぶじ]だったの。	かれ は くるま に はねられた けど ぶじ だった の	
\\	彼[かれ]は 車[くるま]に
\\	けど 無事[ぶじ]だったの。			
\\	はれる	はれる	はれる	
\\	今朝は目がはれています。	今朝[けさ]は 目[め]がはれています。	けさ は め が はれて います	
\\	今朝[けさ]は 目[め]が
\\	匂う	匂[にお]う	におう	
\\	バラの花が甘く匂っているね。	バラの 花[はな]が 甘[あま]く 匂[にお]っているね。	ばら の はな が あまく におって いる ね	
\\	バラの 花[はな]が 甘[あま]く
\\	ね。			
\\	ひく	ひく	ひく	
\\	車にひかれないよう気をつけなさい。	車[くるま]にひかれないよう 気[き]をつけなさい。	くるま に ひかれない よう き を つけなさい 。	
\\	車[くるま]に
\\	気[き]をつけなさい。			
\\	ヒント	ヒント	ヒント	
\\	何かヒントをください。	何[なに]かヒントをください。	なにか ひんと を ください	
\\	何[なに]か
\\	をください。			
\\	ファックス	ファックス	ファックス	
\\	詳細はファックスで送ります。	詳細[しょうさい]はファックスで 送[おく]ります。	しょうさい は ふぁっくす で おくります	
\\	詳細[しょうさい]は
\\	で 送[おく]ります。			
\\	ふもと	ふもと	ふもと	
\\	彼は山のふもとに住んでいるよ。	彼[かれ]は 山[やま]のふもとに 住[す]んでいるよ。	かれ は やま の ふもと に すんで いる よ	
\\	彼[かれ]は 山[やま]の
\\	に 住[す]んでいるよ。			
\\	フルーツ	フルーツ	フルーツ	
\\	おいしそうなフルーツゼリーだね。	おいしそうなフルーツゼリーだね。	おいし そう な ふるーつぜりー だ ね	
\\	おいしそうな
\\	ゼリーだね。			
\\	へそ	へそ	へそ	
\\	カエルにはへそがないんだ。	カエルにはへそがないんだ。	かえる に は へそ が ない ん だ	
\\	カエルには
\\	がないんだ。			
\\	仏	仏[ほとけ]	ほとけ	
\\	仏の教えを勉強しました。	仏[ほとけ]の 教[おし]えを 勉強[べんきょう]しました。	ほとけ の おしえ を べんきょう しました	
\\	の 教[おし]えを 勉強[べんきょう]しました。			
\\	べたべた	べたべた	べたべた	
\\	カップルがべたべたしているね。	カップルがべたべたしているね。	かっぷる が べたべた して いる ね	
\\	カップルが
\\	しているね。			
\\	仏教	仏教[ぶっきょう]	ぶっきょう	
\\	お葬式は仏教で行うことが多いです。	お 葬式[そうしき]は 仏教[ぶっきょう]で 行[おこな]うことが 多[おお]いです。	おそうしき は ぶっきょう で おこなう こと が おおい です	
\\	お 葬式[そうしき]は
\\	で 行[おこな]うことが 多[おお]いです。			
\\	ほうき	ほうき	ほうき	
\\	ほうきで庭を掃除しました。	ほうきで 庭[にわ]を 掃除[そうじ]しました。	ほうき で にわ を そうじ しました 。	
\\	で 庭[にわ]を 掃除[そうじ]しました。			
\\	分裂	分裂[ぶんれつ]	ぶんれつ	
\\	会社が二つの派閥に分裂しているんだ。	会社[かいしゃ]が 二[ふた]つの 派閥[はばつ]に 分裂[ぶんれつ]しているんだ。	かいしゃ が ふたつ の はばつ に ぶんれつ して いる ん だ	
\\	会社[かいしゃ]が 二[ふた]つの 派閥[はばつ]に
\\	しているんだ。			
\\	ポスター	ポスター	ポスター	
\\	電柱にポスターが貼ってあった。	電柱[でんちゅう]にポスターが 貼[は]ってあった。	でんちゅう に ぽすたー が はって あった	
\\	電柱[でんちゅう]に
\\	が 貼[は]ってあった。			
\\	包丁	包丁[ほうちょう]	ほうちょう	
\\	包丁で指を切った。	包丁[ほうちょう]で 指[ゆび]を 切[き]った。	ほうちょう で ゆび を きった	
\\	で 指[ゆび]を 切[き]った。			
\\	包帯	包帯[ほうたい]	ほうたい	
\\	彼女は手に包帯を巻いていたんだ。	彼女[かのじょ]は 手[て]に 包帯[ほうたい]を 巻[ま]いていたんだ。	かのじょ は て に ほうたい を まいて いた ん だ	
\\	彼女[かのじょ]は 手[て]に
\\	を 巻[ま]いていたんだ。			
\\	包み	包[つつ]み	つつみ	
\\	この包みは誰のですか。	この 包[つつ]みは 誰[だれ]のですか。	この つつみ は だれ の です か	
\\	この
\\	は 誰[だれ]のですか。			
\\	包み紙	包[つつ]み 紙[がみ]	つつみがみ	
\\	包み紙はたたんでください。	包[つつ]み 紙[がみ]はたたんでください。	つつみがみ は たたんで ください	
\\	はたたんでください。			
\\	ほどく	ほどく	ほどく	
\\	靴のひもをほどいたの。	靴[くつ]のひもをほどいたの。	くつ の ひも を ほどいた の	
\\	靴[くつ]のひもを
\\	の。			
\\	中旬	中旬[ちゅうじゅん]	ちゅうじゅん	
\\	来月中旬にフランスへ行きます。	来月[らいげつ] 中旬[ちゅうじゅん]にフランスへ 行[い]きます。	らいげつ ちゅうじゅん に ふらんす へ いきます 。	
\\	来月[らいげつ]
\\	にフランスへ 行[い]きます。			
\\	ほほ	ほほ	ほほ	
\\	彼女は真っ赤なほほをしているね。	彼女[かのじょ]は 真[ま]っ 赤[か]なほほをしているね。	かのじょ は まっか な ほほ を している ね 。	
\\	彼女[かのじょ]は 真[ま]っ 赤[か]な
\\	をしているね。			
\\	ほほえむ	ほほえむ	ほほえむ	
\\	彼女は私にほほえんだよ。	彼女[かのじょ]は 私[わたし]にほほえんだよ。	かのじょ は わたし に ほほえんだ よ	
\\	彼女[かのじょ]は 私[わたし]に
\\	よ。			
\\	まく	まく	まく	
\\	庭に花の種をまきました。	庭[にわ]に 花[はな]の 種[たね]をまきました。	にわ に はな の たね を まきました	
\\	庭[にわ]に 花[はな]の 種[たね]を
\\	まとめ	まとめ	まとめ	
\\	論文のまとめを書いています。	論文[ろんぶん]のまとめを 書[か]いています。	ろんぶん の まとめ を かいています 。	
\\	論文[ろんぶん]の
\\	を 書[か]いています。			
\\	まね	まね	まね	
\\	娘はよく私のまねをします。	娘[むすめ]はよく 私[わたし]のまねをします。	むすめ は よく わたし の まね を します	
\\	娘[むすめ]はよく 私[わたし]の
\\	をします。			
\\	出迎え	出迎[でむか]え	でむかえ	
\\	空港へ彼女を出迎えに行きました。	空港[くうこう]へ 彼女[かのじょ]を 出迎[でむか]えに 行[い]きました。	くうこう へ かのじょ を でむかえ に いきました	
\\	空港[くうこう]へ 彼女[かのじょ]を
\\	に 行[い]きました。			
\\	まねる	まねる	まねる	
\\	私の動きをまねてください。	私[わたし]の 動[うご]きをまねてください。	わたし の うごき を まねて ください	
\\	私[わたし]の 動[うご]きを
\\	ください。			
\\	俳句	俳句[はいく]	はいく	
\\	「さくら」という言葉を使って俳句を書いたよ。	「さくら」という 言葉[ことば]を 使[つか]って 俳句[はいく]を 書[か]いたよ。	「さくら」 という ことば を つかって はいく を かいた よ	
\\	「さくら」という 言葉[ことば]を 使[つか]って
\\	を 書[か]いたよ。			
\\	俳優	俳優[はいゆう]	はいゆう	
\\	彼は俳優です。	彼[かれ]は 俳優[はいゆう]です。	かれ は はいゆう です	
\\	彼[かれ]は
\\	です。			
\\	まぶた	まぶた	まぶた	
\\	眠くてまぶたが重くなってきた。	眠[ねむ]くてまぶたが 重[おも]くなってきた。	ねむくて まぶた が おもく なってきた 。	
\\	眠[ねむ]くて
\\	が 重[おも]くなってきた。			
\\	倣う	倣[なら]う	ならう	
\\	この街はロンドンに倣って作られました。	この 街[まち]はロンドンに 倣[なら]って 作[つく]られました。	この まち は ろんどん に ならって つくられました	
\\	この 街[まち]はロンドンに
\\	作[つく]られました。			
\\	偏見	偏見[へんけん]	へんけん	
\\	彼は世の中の偏見と戦ったんだ。	彼[かれ]は 世[よ]の 中[なか]の 偏見[へんけん]と 戦[たたか]ったんだ。	かれ は よのなか の へんけん と たたかった ん だ	
\\	彼[かれ]は 世[よ]の 中[なか]の
\\	と 戦[たたか]ったんだ。			
\\	何遍	何遍[なんべん]	なんべん	
\\	この映画は何遍も見ました。	この 映画[えいが]は 何遍[なんべん]も 見[み]ました。	この えいが は なんべん も みました	
\\	この 映画[えいが]は
\\	も 見[み]ました。			
\\	まれ	まれ	まれ	
\\	彼が仕事を休むのはまれです。	彼[かれ]が 仕事[しごと]を 休[やす]むのはまれです。	かれ が しごと を やすむ の は まれ です	
\\	彼[かれ]が 仕事[しごと]を 休[やす]むのは
\\	です。			
\\	ミスプリント	ミスプリント	ミスプリント	
\\	資料にミスプリントがありました。	資料[しりょう]にミスプリントがありました。	しりょう に みすぷりんと が ありました	
\\	資料[しりょう]に
\\	がありました。			
\\	内閣	内閣[ないかく]	ないかく	
\\	新しい内閣が誕生したね。	新[あたら]しい 内閣[ないかく]が 誕生[たんじょう]したね。	あたらしい ないかく が たんじょう した ね	
\\	新[あたら]しい
\\	が 誕生[たんじょう]したね。			
\\	むく	むく	むく	
\\	人参の皮をむきました。	人参[にんじん]の 皮[かわ]をむきました。	にんじん の かわ を むきました	
\\	人参[にんじん]の 皮[かわ]を
\\	不愉快	不愉快[ふゆかい]	ふゆかい	
\\	彼の話を聞いて不愉快になった。	彼[かれ]の 話[はなし]を 聞[き]いて 不愉快[ふゆかい]になった。	かれ の はなし を きいて ふゆかい に なった	
\\	彼[かれ]の 話[はなし]を 聞[き]いて
\\	になった。			
\\	めくる	めくる	めくる	
\\	彼はページをめくったの。	彼[かれ]はページをめくったの。	かれ は ぺーじ を めくった の	
\\	彼[かれ]はページを
\\	の。			
\\	めでたい	めでたい	めでたい	
\\	それはめでたいことだ。	それはめでたいことだ。	それはめでたいことだ。	
\\	それは
\\	ことだ。			
\\	もうかる	もうかる	もうかる	
\\	彼の商売はもうかっているらしいね。	彼[かれ]の 商売[しょうばい]はもうかっているらしいね。	かれ の しょうばい は もうかって いる らしい ね	
\\	彼[かれ]の 商売[しょうばい]は
\\	らしいね。			
\\	倉庫	倉庫[そうこ]	そうこ	
\\	この荷物を倉庫に運びましょう。	この 荷物[にもつ]を 倉庫[そうこ]に 運[はこ]びましょう。	この にもつ を そうこ に はこびましょう	
\\	この 荷物[にもつ]を
\\	に 運[はこ]びましょう。			
\\	創立	創立[そうりつ]	そうりつ	
\\	この学校は1962年に創立されました。	この 学校[がっこう]は1962 年[ねん]に 創立[そうりつ]されました。	この がっこう は 
\\	ねん に そうりつ されました	
\\	この 学校[がっこう]は1962 年[ねん]に
\\	されました。			
\\	もうけ	もうけ	もうけ	
\\	今日のもうけは2万円でした。	今日[きょう]のもうけは2 万円[まんえん]でした。	きょう の もうけ は 
\\	まんえん でした	
\\	今日[きょう]の
\\	は2 万円[まんえん]でした。			
\\	もしかしたら	もしかしたら	もしかしたら	
\\	もしかしたら来年転勤になるかも知れない。	もしかしたら 来年転勤[らいねん てんきん]になるかも 知[し]れない。	もしかしたら らいねん てんきん に なる かも しれない	
\\	来年転勤[らいねん てんきん]になるかも 知[し]れない。			
\\	もてる	もてる	もてる	
\\	彼は非常にもてるね。	彼[かれ]は 非常[ひじょう]にもてるね。	かれ は ひじょう に もてる ね	
\\	彼[かれ]は 非常[ひじょう]に
\\	ね。			
\\	先輩	先輩[せんぱい]	せんぱい	
\\	日本人は先輩、後輩の関係をとても大切に考えます。	日本人[にっぽんじん]は 先輩[せんぱい]、 後輩[こうはい]の 関係[かんけい]をとても 大切[たいせつ]に 考[かんが]えます。	にっぽんじん は せんぱい こうはい の かんけい を とても たいせつ に かんがえます	
\\	日本人[にっぽんじん]は
\\	、 後輩[こうはい]の 関係[かんけい]をとても 大切[たいせつ]に 考[かんが]えます。			
\\	冒険	冒険[ぼうけん]	ぼうけん	
\\	昔の子供たちは冒険小説をよく読みました。	昔[むかし]の 子供[こども]たちは 冒険[ぼうけん] 小説[しょうせつ]をよく 読[よ]みました。	むかし の こどもたち は ぼうけん しょうせつ を よく よみました	
\\	昔[むかし]の 子供[こども]たちは
\\	小説[しょうせつ]をよく 読[よ]みました。			
\\	ゆでる	ゆでる	ゆでる	
\\	今、野菜をゆでています。	今[いま]、 野菜[やさい]をゆでています。	いま やさい を ゆでて います	
\\	今[いま]、 野菜[やさい]を
\\	取り敢えず	取[と]り 敢[あ]えず	とりあえず	
\\	取り敢えずお知らせしておきます。	取[と]り 敢[あ]えずお 知[し]らせしておきます。	とりあえず おしらせ して おきます	
\\	お 知[し]らせしておきます。			
\\	勇気	勇気[ゆうき]	ゆうき	
\\	彼は勇気があるね。	彼[かれ]は 勇気[ゆうき]があるね。	かれ は ゆうき が ある ね	
\\	彼[かれ]は
\\	があるね。			
\\	勇敢	勇敢[ゆうかん]	ゆうかん	
\\	彼女はとても勇敢でした。	彼女[かのじょ]はとても 勇敢[ゆうかん]でした。	かのじょ は とても ゆうかん でした	
\\	彼女[かのじょ]はとても
\\	でした。			
\\	励ます	励[はげ]ます	はげます	
\\	父が息子を励ましたの。	父[ちち]が 息子[むすこ]を 励[はげ]ましたの。	ちち が むすこ を はげました の	
\\	父[ちち]が 息子[むすこ]を
\\	の。			
\\	よける	よける	よける	
\\	彼は飛んできたボールをよけたの。	彼[かれ]は 飛[と]んできたボールをよけたの。	かれ は とんで きた ぼーる を よけた の	
\\	彼[かれ]は 飛[と]んできたボールを
\\	の。			
\\	よこす	よこす	よこす	
\\	父が長い手紙をよこしたよ。	父[ちち]が 長[なが]い 手紙[てがみ]をよこしたよ。	ちち が ながい てがみ を よこした よ	
\\	父[ちち]が 長[なが]い 手紙[てがみ]を
\\	よ。			
\\	よす	よす	よす	
\\	人をからかうのはよしなさい。	人[ひと]をからかうのはよしなさい。	ひと を からかう の は よしなさい	
\\	人[ひと]をからかうのは
\\	中央	中央[ちゅうおう]	ちゅうおう	
\\	その公園は町のほぼ中央に位置するんだ。	その 公園[こうえん]は 町[まち]のほぼ 中央[ちゅうおう]に 位置[いち]するんだ。	その こうえん は まち の ほぼ ちゅうおう に いち する ん だ	
\\	その 公園[こうえん]は 町[まち]のほぼ
\\	に 位置[いち]するんだ。			
\\	レクリエーション	レクリエーション	レクリエーション	
\\	町内会のレクリエーションに参加したの。	町内会[ちょうないかい]のレクリエーションに 参加[さんか]したの。	ちょうないかい の れくりえーしょん に さんか した の	
\\	町内会[ちょうないかい]の
\\	に 参加[さんか]したの。			
\\	双子	双子[ふたご]	ふたご	
\\	友達に双子が生まれたよ。	友達[ともだち]に 双子[ふたご]が 生[う]まれたよ。	ともだち に ふたご が うまれた よ	
\\	友達[ともだち]に
\\	が 生[う]まれたよ。			
\\	ろうそく	ろうそく	ろうそく	
\\	バースデーケーキにろうそくを立てました。	バースデーケーキにろうそくを 立[た]てました。	ばーすでーけーき に ろうそく を たてました	
\\	バースデーケーキに
\\	を 立[た]てました。			
\\	わがまま	わがまま	わがまま	
\\	子供のわがままを叱ったの。	子供[こども]のわがままを 叱[しか]ったの。	こども の わがまま を しかった の	
\\	子供[こども]の
\\	を 叱[しか]ったの。			
\\	わく	わく	わく	
\\	友達に励まされて勇気がわいたよ。	友達[ともだち]に 励[はげ]まされて 勇気[ゆうき]がわいたよ。	ともだち に はげまされて ゆうき が わいた よ	
\\	友達[ともだち]に 励[はげ]まされて 勇気[ゆうき]が
\\	よ。			
\\	ばからしい	ばからしい	ばからしい	
\\	小さなことにくよくよするのはばからしいよ。	小[ちい]さなことにくよくよするのはばからしいよ。	ちいさ な こと に くよくよ する の は ばからしい よ	
\\	小[ちい]さなことにくよくよするのは
\\	よ。			
\\	なぜか	なぜか	なぜか	
\\	今日はなぜか体がだるい。	今日[きょう]はなぜか 体[からだ]がだるい。	きょう は なぜか からだ が だるい	
\\	今日[きょう]は
\\	体[からだ]がだるい。			
\\	にわかに	にわかに	にわかに	
\\	空がにわかに暗くなったな。	空[そら]がにわかに 暗[くら]くなったな。	そら が にわかに くらく なった な	
\\	空[そら]が
\\	暗[くら]くなったな。			
\\	ぴたりと	ぴたりと	ぴたりと	
\\	風がぴたりと止んだね。	風[かぜ]がぴたりと 止[や]んだね。	かぜ が ぴたりと やんだ ね	
\\	風[かぜ]が
\\	止[や]んだね。			
\\	ぴょんと	ぴょんと	ぴょんと	
\\	子供が水たまりをぴょんと飛びこえたね。	子供[こども]が 水[みず]たまりをぴょんと 飛[と]びこえたね。	こども が みずたまり を ぴょんと とびこえた ね	
\\	子供[こども]が 水[みず]たまりを
\\	飛[と]びこえたね。			
\\	卑怯	卑怯[ひきょう]	ひきょう	
\\	彼らは卑怯な方法で勝利したの。	彼[かれ]らは 卑怯[ひきょう]な 方法[ほうほう]で 勝利[しょうり]したの。	かれら は ひきょう な ほうほう で しょうり した の	
\\	彼[かれ]らは
\\	な 方法[ほうほう]で 勝利[しょうり]したの。			
\\	ひらりと	ひらりと	ひらりと	
\\	桜の花びらがひらりと舞い落ちたな。	桜[さくら]の 花[はな]びらがひらりと 舞[ま]い 落[お]ちたな。	さくら の はなびら が ひらりと まいおちた な	
\\	桜[さくら]の 花[はな]びらが
\\	舞[ま]い 落[お]ちたな。			
\\	もうじき	もうじき	もうじき	
\\	彼はもうじき父親になります。	彼[かれ]はもうじき 父親[ちちおや]になります。	かれ は もうじき ちちおや に なります	
\\	彼[かれ]は
\\	父親[ちちおや]になります。			
\\	やたらに	やたらに	やたらに	
\\	彼はやたらに話しかけてきた。	彼[かれ]はやたらに 話[はな]しかけてきた。	かれ は やたらに はなしかけて きた	
\\	彼[かれ]は
\\	話[はな]しかけてきた。			
\\	割り箸	割[わ]り 箸[ばし]	わりばし	
\\	彼女は割り箸を使わず、自分の箸を使うの。	彼女[かのじょ]は 割[わ]り 箸[ばし]を 使[つか]わず、 自分[じぶん]の 箸[はし]を 使[つか]うの。	かのじょ は わりばし を つかわず じぶん の はし を つかう の	
\\	彼女[かのじょ]は
\\	を 使[つか]わず、 自分[じぶん]の 箸[はし]を 使[つか]うの。			
\\	初詣で	初詣[はつもう]で	はつもうで	
\\	近くの神社に初詣でに行きました。	近[ちか]くの 神社[じんじゃ]に 初詣[はつもう]でに 行[い]きました。	ちかく の じんじゃ に はつもうで に いきました	
\\	近[ちか]くの 神社[じんじゃ]に
\\	に 行[い]きました。			
\\	便箋	便箋[びんせん]	びんせん	
\\	友達に手紙を書こうと便箋を買ったんだ。	友達[ともだち]に 手紙[てがみ]を 書[か]こうと 便箋[びんせん]を 買[か]ったんだ。	ともだち に てがみ を かこう と びんせん を かった ん だ	
\\	友達[ともだち]に 手紙[てがみ]を 書[か]こうと
\\	を 買[か]ったんだ。			
\end{CJK}
\end{document}