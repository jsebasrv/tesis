\documentclass[8pt]{extreport} 
\usepackage{hyperref}
\usepackage{CJKutf8}
\begin{document}
\begin{CJK}{UTF8}{min}
\\	個人の意志は尊重しなければいけない	
\\	〜を尊重する 
\\	意志 
\\	個人
\\	気軽にいけよ。大丈夫、すべてうまくいくさ。	
\\	気軽にやれ 
\\	に〜を確約する(保証する) 
\\	(最終的に)〜という結果になる
\\	悲観的な人生観を捨てて、前向きな態度を常に持ち続けよう	
\\	~を離す、捨てる 
\\	悲観的な、消極的な 
\\	見方(態度)、~観 
\\	~を維持する 
\\	積極的な、前向きな 
\\	態度
\\	生まれた国、性別、信条に関係なく、誰に対しても公平でなくてはならない。	
\\	公平な、公正な 
\\	~にかかわらず 
\\	国の~、国内の~ 
\\	起源、生まれ 
\\	性別
\\	(宗教のための)信条
\\	平等は憲法で守られている	
\\	平等 
\\	~を保証する、~を約束する 
\\	憲法
\\	彼は柱に寄りかかって、自自由の女神像をじっと見つめた。	
\\	~に寄りかかる 
\\	柱、支柱 
\\	~をじっと見つめる 
\\	彫像 
\\	自由
\\	香水のほのかな香りを漂わせながら、一人の女性が僕の前を通り過ぎた。前の彼女のことを思い出してしまった。	
\\	(~の)そばを通り過ぎる 
\\	~を発する 
\\	かすかな 
\\	香り、におい 
\\	香水 
\\	~(人)に思い出させる 
\\	に
\\	を思い出させる 
\\	前の~
\\	納豆は臭いはひどいけれど味は最高。	
\\	においがする 
\\	ひどい 
\\	~な味がする 
\\	すばらしい
\\	「汗でびしょぬれだよ。」 「来ないで!臭いわ。シャワーを浴びて。」	
\\	びしょ濡れの 
\\	汗 
\\	後ろに下がる 
\\	悪臭がする、(ひどく)臭い 
\\	シャワーを浴びる
\\	ボブはひどく取り乱していて、現実と虚構の区別がほとんどできなかった。	
\\	ひどく興奮する、取り乱す 
\\	ほとんど(~し)ない 
\\	と
\\	を区別する 
\\	現実、事実 
\\	作り話、虚構
\\	まもなく出版になる彼の小説は、散文に彼の詩の才能を融合させたものだ。	
\\	小説 
\\	組み合わさる 
\\	と
\\	を組み合わせつ 
\\	散文 
\\	才能 
\\	詩 
\\	~を出版する
\\	来月、その百科事典の最新版が出る。	
\\	最新の 
\\	版 
\\	百科事典 
\\	発売される、出版される
\\	山田さんがその心惹かれるおとぎ話をやさしい日本語に翻訳した。	
\\	から
\\	に)翻訳する 
\\	魅了する 
\\	妖精 
\\	話、物語 
\\	平易な
\\	以下の一節は有名な寓話からの引用です。	
\\	一節 
\\	引用する 
\\	有名な 
\\	寓話(ぐうわ)=擬人化された動物などを通じて教訓を伝える物語
\\	イソップ物語、
\\	「現代文学に詳しいですか?」 「ほとんど知りません」	
\\	よく知っている 
\\	~をよく知っている 
\\	現代の 
\\	文学 
\\	ほとんど~
\\	"時々、
\\	""と
\\	""を混同してしまう。
\\	時々、たまに 
\\	と
\\	を混同する 
\\	曲線 
\\	~を彫る
\\	恥ずかしがらないで。あなたの発音はだいたいあっています。	
\\	恥ずかしがりの 
\\	発音 
\\	だいたい 
\\	正しい
\\	この記事には、語彙を本気で増やしたいと思う人々に役立つ情報が含まれています。	
\\	~を含んでいる 
\\	役立つ情報 
\\	~という人たち
\\	彼の最新の作品が市庁舎で展示されているの。言葉にできないほど素敵よ。	
\\	作品 
\\	展示されている 
\\	市庁舎 
\\	すてきな 
\\	言葉では言い表せない(ほど)
\\	「少し詰めていただけませんか?」 「あ、ごめんなさい。こんなに場所をとっていたなんて気づきませんでした。」	
\\	(人が座れるように)席を詰める 
\\	(必要以上に)~(空間、時間などを)取る
\\	「この不格好な物体は何?」 「これは抽象芸術の作品だよ!」	
\\	不格好な、醜い 
\\	1つの~
\\	100ドル貸してくれとリッチーに頼んだら、彼は「俺もお金がないんだよ!」と言って首を横に振った。 ワロタww	
\\	すがってお願いする 
\\	~に
\\	(物、金)を貸す 
\\	ドル 
\\	と言う、首を横に振る 
\\	お金がない、すっからかんだ
\\	「安売りをしているといつも衝動買いしてしまうたちです。」 「わたしも。」	
\\	~しがちである 
\\	衝動的に 
\\	バーゲン(特売)で 
\\	(選考分に答えて)私もそうです。
\\	実際のところ、普通に人々にはそんなぜいたく品を購入する余裕はない。	
\\	実際のところ 
\\	~する余裕がある 
\\	贅沢品
\\	予測していた通り、失業率は3四半期連続で上昇でした。	
\\	~を予測する 
\\	失業 
\\	四半期 
\\	連続で
\\	出費を切り詰めないと。だから、これからは使ったお金は一日単位で記録していくつもりだ。	
\\	~を減らす 
\\	出費 
\\	これからは 
\\	~の動き(流れ)を常に記録する 
\\	~という基準で
\\	いずれにせよ、組合側はある程度妥協しなければならない。	
\\	労働組合 
\\	妥協する 
\\	(特定はできないが)ある 
\\	~の程度
\\	有能な整備士は引っ張りだこだから、なかなかの資金を稼げる。	
\\	能力を持つ 
\\	整備士 
\\	需要がある 
\\	~を稼ぐ 
\\	かなりよい 
\\	賃金
\\	大統領は福祉改革を実行するための具体的な計画を発表した。	
\\	大統領 
\\	~うを公表する 
\\	~を実行する 
\\	福祉 
\\	改革
\\	彼の政策はきっと暗い結果を招くだろう。徹底的な見直しが必要だ。	
\\	きっと 
\\	~に至る、(を導く) 
\\	気分を暗くさせる 
\\	結果、影響 
\\	徹底的な 
\\	見直し
\\	社会の高齢化が避けられないことは言うまでもない。	
\\	~は言うまでもない 
\\	高齢化 
\\	避けられない
\\	この表をご覧になってください。少年非行が深刻な割合で増加していることを示します。	
\\	~をちょっと見る 
\\	図、表 
\\	~を示す 
\\	青少年の、少年の 
\\	非行 
\\	増加している 
\\	深刻な 
\\	~なペースで
\\	多くの商業地区では、数年間売りに出されている空き地がたくさんある。	
\\	地区、地方 
\\	たくさんの(数) 
\\	誰もいない 
\\	一区画 
\\	売り物である
\\	大都市の人口密度が徐々に低下している。	
\\	密度 
\\	大都市
\\	ほぼ間違いなくこの先数年間、出生率は減少の一途をたどるだろう。	
\\	ほぼ間違いなく 
\\	出生率 
\\	~をし続ける 
\\	着実に 
\\	これから来る~
\\	その発電所は遠く離れた群に電気を供給している。	
\\	発電所 
\\	に
\\	を提供する 
\\	遠く離れた 
\\	群 
\\	電気
\\	許可なくこの装置を操作することはできません。	
\\	に~することを許す 
\\	~を動かす、操作する 
\\	装置 
\\	(正式な)許可
\\	実は、住民たちは放射能にさらされていた。	
\\	実際は 
\\	住民 
\\	を
\\	にさらす 
\\	放射能、放射線
\\	核エネルギーの潜在的な危険性は、いくら強調してもし過ぎるということはない。	
\\	いくら~してもし過ぎることはない 
\\	~を強調する 
\\	潜在的な 
\\	核の、原子力の
\\	過去数年間その問題を見過ごしていた報道機関が、今ではどちらかといえばそのことを重視し過ぎている。	
\\	報道陣 
\\	~を見落とす、見過ごす 
\\	むしろ、どちらかと言えば 
\\	に対して~な評価をする
\\	穀物を栽培するだけでなく、その農場経営者は食料雑貨店も経営している。	
\\	~だけではなく 
\\	~を栽培する 
\\	穀物 
\\	農場経営者 
\\	~を経営する 
\\	食料雑貨店
\\	その研究所は1960年代後半に設立された。	
\\	理系の大学、組織 
\\	~後半の 
\\	1960年代
\\	スターさんに直接電話したらどうですか?この辺りでは文句なしでいちばん有名な弁護士ですよ。	
\\	~に相談する 
\\	本人と直接会って 
\\	はるかに、間違いなく 
\\	傑出した、著名な 
\\	弁護士
\\	念のためにメモしておきます。記憶力が悪いから。	
\\	~を書き留める 
\\	念のため 
\\	記憶力が悪い
\\	弁護士は依頼人に、保険会社に対して法的手段をとることを勧めた。	
\\	に~するのがいいと勧める 
\\	法的な 
\\	~な行動をとる 
\\	保険
\end{CJK}
\end{document}