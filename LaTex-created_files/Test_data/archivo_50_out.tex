\documentclass[8pt]{extreport} 
\usepackage{hyperref}
\usepackage{CJKutf8}
\begin{document}
\begin{CJK}{UTF8}{min}
\\	あの…	
\\	あの…	
\\	今	
\\	今[いま]	イꜜマ
\\	英語	
\\	英[えい]語[ご]	エイコ°
\\	学生	
\\	学[がく]生[せい]	ガクセイ
\\	〜語	
\\	〜 語[ご]	ゴꜜ
\\	午後	
\\	午[ご]後[ご]	ゴꜜコ°
\\	午前	
\\	午[ご]前[ぜん]	ゴꜜゼン
\\	〜歳	
\\	〜 歳[さい]	
\\	〜さん	
\\	〜さん	
\\	〜時	
\\	〜 時[じ]	
\\	〜人	
\\	〜 人[にん]	
\\	先生	
\\	先[せん]生[せい]	センセꜜイ
\\	専門	
\\	専[せん]門[もん]	センモン
\\	そうです	
\\	そうです	
\\	大学	
\\	大[だい]学[がく]	ダイカ°ク
\\	電話	
\\	電[でん]話[わ]	デンワ
\\	友達	
\\	友[とも]達[だち]	トモダチ
\\	名前	
\\	名[な]前[まえ]	ナマエ
\\	何	
\\	何[なに]	ナꜜニ 
\\	ナニ
\\	日本	
\\	日本[にっぽん]	ニッポꜜン 
\\	ニホꜜン
\\	〜年生	
\\	〜 年[ねん]生[せい]	
\\	はい	
\\	はい	ハꜜイ 
\\	ハイ
\\	半	
\\	半[はん]	ハꜜン
\\	番号	
\\	番[ばん]号[ごう]	バンコ°ꜜー
\\	留学生	
\\	留[りゅう]学[がく]生[せい]	リューカ°ꜜクセイ 
\\	リューカ°クꜜセイ
\\	私	
\\	私[わたし]	ワタシ
\\	アメリカ	
\\	アメリカ	
\\	イギリス	
\\	イギリス	
\\	オーストラリア	
\\	オーストラリア	
\\	韓国	
\\	韓[かん]国[こく]	
\\	スウェーデン	
\\	スウェーデン	
\\	中国	
\\	中[ちゅう]国[ごく]	
\\	科学	
\\	科[か]学[がく]	カꜜカ°ク
\\	アジア研究	
\\	アジア 研[けん]究[きゅう]	
\\	国際関係	
\\	国[こく]際[さい] 関[かん]係[けい]	
\\	コンピューター 
\\	パソコン	
\\	コンピューター 
\\	パソコン	
\\	人類学	
\\	人[じん]類[るい] 学[がく]	ジンルꜜイカ°ク
\\	政治	
\\	政[せい]治[じ]	セイジ
\\	ビジネス 
\\	業務	
\\	ビジネス 
\\	業[ぎょう]務[む]	
\\	文学	
\\	文[ぶん]学[がく]	ブꜜンカ°ク
\\	歴史	
\\	歴[れき]史[し]	レキシ
\\	仕事	
\\	仕[し]事[ごと]	シコ°ト
\\	医者	
\\	医[い]者[しゃ]	イシャ
\\	会社員	
\\	会[かい]社[しゃ] 員[いん]	カイシャꜜイン
\\	高校生	
\\	高[こう]校[こう]生[せい]	コーコꜜーセイ
\\	主婦	
\\	主[しゅ]婦[ふ]	シュꜜフ
\\	大学院生	
\\	大[だい]学[がく]院[いん]生[せい]	
\\	大学生	
\\	大[だい]学[がく]生[せい]	ダイカ°ꜜクセイ 
\\	ダイカ°クꜜセイ
\\	弁護士	
\\	弁[べん]護[ご]士[し]	ベンコ°ꜜシ
\\	お母さん	
\\	お 母[かあ]さん	オカꜜーサン
\\	お父さん	
\\	お 父[とう]さん	オトꜜーサン
\\	お姉さん	
\\	お 姉[ねえ]さん	オネꜜーサン
\\	お兄さん	
\\	お 兄[にい]さん	オニꜜーサン
\\	妹	
\\	妹[いもうと]	イモートꜜ
\\	弟	
\\	弟[おとうと]	オトートꜜ
\\	一歳	
\\	一歳[いっさい]	
\\	一分	
\\	一[いっ]分[ぷん]	イチꜜブ 
\\	イチꜜブン 
\\	イチブン
\\	九歳	
\\	九[きゅう]歳[さい]	
\\	九分	
\\	九[きゅう]分[ふん]	
\\	九時	
\\	九[く]時[じ]	クꜜジ
\\	五歳	
\\	五[ご]歳[さい]	
\\	五時	
\\	五[ご]時[じ]	ゴꜜジ
\\	五分	
\\	五[ご]分[ふん]	ゴꜜブ
\\	三歳	
\\	三[さん]歳[さい]	サꜜンサイ
\\	三時	
\\	三[さん]時[じ]	サꜜンジ
\\	三十分	
\\	三[さん]十[じゅっ]分[ぷん]	
\\	三分	
\\	三[さん]分[ぷん]	サンブン
\\	七時	
\\	七[しち]時[じ]	
\\	十一時	
\\	十[じゅう]一[いち]時[じ]	
\\	十一歳	
\\	十[じゅう]一[いっ]歳[さい]	
\\	十一分	
\\	十[じゅう]一[いっ]分[ぷん]	
\\	十九分	
\\	十[じゅう]九[きゅう]分[ふん]	
\\	十五分	
\\	十[じゅう]五[ご]分[ふん]	
\\	十三分	
\\	十[じゅう]三[さん]分[ぷん]	
\\	十時	
\\	十[じゅう]時[じ]	
\\	十七分	
\\	十[じゅう]七[なな]分[ふん]	
\\	十二時	
\\	十[じゅう]二[に]時[じ]	
\\	十二分	
\\	十[じゅう]二[に]分[ふん]	ジューニブン
\\	十八分	
\\	十[じゅう]八[はっ]分[ぷん]	
\\	十四分	
\\	十[じゅう]四[よん]分[ぷん]	
\\	十六分	
\\	十[じゅう]六[ろっ]分[ぷん]	
\\	十歳	
\\	十[じゅっ]歳[さい]	
\\	十分	
\\	十[じゅっ]分[ぷん]	ジューブꜜン
\\	七歳	
\\	七[なな]歳[さい]	
\\	七分	
\\	七[なな]分[ふん]	シチꜜブ
\\	二歳	
\\	二[に]歳[さい]	
\\	二時	
\\	二[に]時[じ]	ニꜜジ
\\	二十分	
\\	二[に]十[じゅっ]分[ぷん]	
\\	二分	
\\	二[に]分[ふん]	ニブン
\\	二十歳	
\\	二十歳[はたち]	ハꜜタチ
\\	八時	
\\	八[はち]時[じ]	ハチꜜジ
\\	八歳	
\\	八[はっ]歳[さい]	
\\	八分	
\\	八[はっ]分[ぷん]	
\\	四時	
\\	四[よ]時[じ]	シꜜジ 
\\	ヨꜜジ
\\	四歳	
\\	四[よん]歳[さい]	
\\	四分	
\\	四[よん]分[ぷん]	
\\	六歳	
\\	六[ろく]歳[さい]	
\\	六時	
\\	六[ろく]時[じ]	ロクꜜジ
\\	六分	
\\	六[ろっ]分[ぷん]	
\\	此れ	
\\	此[こ]れ	
\\	其れ	
\\	其[そ]れ	ソレ
\\	彼れ	
\\	彼[あ]れ	
\\	何れ	
\\	何[ど]れ	ドレ
\\	此の	
\\	此[こ]の	
\\	其の	
\\	其[そ]の	ソノ
\\	彼の	
\\	彼[あ]の	アꜜノ
\\	何の	
\\	何[ど]の	ドꜜノ 
\\	ナꜜンノ
\\	彼処	
\\	彼[あそ]処[こ]	アスコ 
\\	アソコ
\\	何処	
\\	何[ど]処[こ]	イꜜズク 
\\	イꜜズコ 
\\	イズコ 
\\	ドꜜコ
\\	誰	
\\	誰[だれ]	ダꜜレ 
\\	タꜜレ
\\	美味しい	
\\	美[お]味[い]しい	オイシイ 
\\	オイシꜜイ
\\	魚 
\\	魚	
\\	魚[さかな] 
\\	魚[うお]	
\\	豚かつ	
\\	豚[とん]かつ	
\\	肉	
\\	肉[にく]	ニクꜜ
\\	メニュー	
\\	メニュー	メꜜニュー
\\	野菜	
\\	野[や]菜[さい]	ヤサイ
\\	鉛筆	
\\	鉛[えん]筆[ぴつ]	エンピツ
\\	傘	
\\	傘[かさ]	カꜜサ
\\	鞄	
\\	鞄[かばん]	カバン
\\	靴	
\\	靴[くつ]	クツꜜ
\\	財布	
\\	財[さい]布[ふ]	サイフ
\\	ジーンズ	
\\	ジーンズ	ジꜜーンズ
\\	辞書 
\\	辞典	
\\	辞[じ]書[しょ] 
\\	辞[じ]典[てん]	
\\	自転車	
\\	自[じ]転[てん]車[しゃ]	ジテꜜンシャ 
\\	ジテンシャ
\\	新聞	
\\	新[しん]聞[ぶん]	シンブン
\\	テープ	
\\	テープ	テꜜープ
\\	時計	
\\	時[と]計[けい]	トケイ
\\	トレーナー	
\\	トレーナー	トレꜜーナー 
\\	トレーナー
\\	ノート	
\\	ノート	ノꜜート
\\	ペン	
\\	ペン	ペꜜン
\\	帽子	
\\	帽[ぼう]子[し]	ボーシ
\\	本	
\\	本[ほん]	ホꜜン 
\\	モꜜト 
\\	モトꜜ 
\\	モト
\\	お手洗い	
\\	お 手[て]洗[あら]い	オテアꜜライ
\\	喫茶店	
\\	喫[きっ]茶[さ]店[てん]	キッサテン 
\\	キッサꜜテン
\\	銀行	
\\	銀[ぎん]行[こう]	ギンコー
\\	図書館	
\\	図[と]書[しょ]館[かん]	トショꜜカン
\\	郵便局	
\\	郵[ゆう]便[びん]局[きょく]	ユービꜜンキョク
\\	経済	
\\	経[けい]済[ざい]	ケꜜイザイ
\\	幾ら	
\\	幾[いく]ら	
\\	〜円	
\\	〜 円[えん]	
\\	高い	
\\	高[たか]い	タカꜜイ
\\	いらっしゃいませ	
\\	いらっしゃいませ	
\\	(〜を)お願いします 
\\	(〜を)下さい	
\\	(〜を) お 願[ねが]いします 
\\	(〜を) 下[くだ]さい	
\\	じゃあ	
\\	じゃあ	
\\	(〜を)どうぞ	
\\	(〜を)どうぞ	
\\	どうも	
\\	どうも	ドꜜーモ
\\	椅子	
\\	椅[い]子[す]	イス
\\	カーテン 
\\	幕	
\\	カーテン 
\\	幕[まく]	
\\	消しゴム	
\\	消[け]し ゴ[ご]ム[む]	ケシゴム
\\	黒板	
\\	黒[こく]板[ばん]	コクバン
\\	戸 
\\	ドア	
\\	戸[と] 
\\	ドア	
\\	映画	
\\	映[えい]画[が]	エꜜイカ° 
\\	エイカ°
\\	音楽	
\\	音[おん]楽[がく]	オꜜンカ°ク 
\\	オンカ°ク
\\	雑誌	
\\	雑[ざっ]誌[し]	ザッシ
\\	スポーツ	
\\	スポーツ	スポꜜーツ
\\	デート	
\\	デート	デꜜート
\\	テニス	
\\	テニス	テꜜニス
\\	テレビ	
\\	テレビ	テꜜレビ
\\	ビデオ	
\\	ビデオ	ビꜜデオ
\\	朝御飯	
\\	朝[あさ]御[ご]飯[はん]	
\\	お酒	
\\	お 酒[さけ]	
\\	お茶	
\\	お 茶[ちゃ]	オチャ
\\	コーヒー	
\\	コーヒー	コーヒꜜー
\\	晩御飯	
\\	晩[ばん]御[ご]飯[はん]	
\\	ハンバーガー	
\\	ハンバーガー	ハンバꜜーガー
\\	昼御飯	
\\	昼[ひる] 御[ご]飯[はん]	
\\	水	
\\	水[みず]	ミズ
\\	家 
\\	家	
\\	家[いえ] 
\\	家[うち]	
\\	(僕の)家	
\\	僕[ぼく]の) 家[うち]	
\\	[エル]
\\	[エル]	
\\	学校	
\\	学[がっ]校[こう]	ガッコー
\\	朝	
\\	朝[あさ]	アꜜサ 
\\	アシタꜜ
\\	明日	
\\	明日[あした]	アシタꜜ 
\\	アシタ 
\\	アスꜜ 
\\	アス 
\\	ミョꜜーニチ
\\	何時	
\\	何[い]時[つ]	イꜜツ 
\\	ナꜜンジ 
\\	ナンドキ
\\	今日	
\\	今[きょ]日[う]	キョꜜー 
\\	コꜜンニチ
\\	〜頃	
\\	〜 頃[ごろ]	
\\	今晩	
\\	今[こん]晩[ばん]	コꜜンバン
\\	週末	
\\	週[しゅう]末[まつ]	シューマツ
\\	土曜日	
\\	土[ど]曜[よう]日[び]	ドヨꜜービ
\\	日曜日	
\\	日[にち]曜[よう]日[び]	ニチヨꜜービ
\\	毎日	
\\	毎[まい]日[にち]	マꜜイニチ
\\	毎晩	
\\	毎[まい]晩[ばん]	マꜜイバン 
\\	マイバン
\\	行く 
\\	行って 
\\	行きます	
\\	行[い]く 
\\	行[い]って 
\\	行[い]きます	
\\	帰る 
\\	帰って 
\\	帰ります	
\\	帰[かえ]る 
\\	帰[かえ]って 
\\	帰[かえ]ります 
\\	換える 
\\	変える)	
\\	聞く 
\\	聞いて 
\\	聞きます	
\\	聞[き]く 
\\	聞[き]いて 
\\	聞[き]きます	
\\	飲む 
\\	飲んで 
\\	飲みます	
\\	飲[の]む 
\\	飲[の]んで 
\\	飲[の]みます	
\\	話す 
\\	話して 
\\	話します	
\\	話[はな]す 
\\	話[はな]して 
\\	話[はな]します	
\\	読む 
\\	読んで 
\\	読みます	
\\	読[よ]む 
\\	読[よ]んで 
\\	読[よ]みます	
\\	起きる 
\\	起きて 
\\	起きます	
\\	起[お]きる 
\\	起[お]きて 
\\	起[お]きます	
\\	食べる 
\\	食べて 
\\	食べます	
\\	食[た]べる 
\\	食[た]べて 
\\	食[た]べます	
\\	寝る 
\\	寝て 
\\	寝ます	
\\	寝[ね]る 
\\	寝[ね]て 
\\	寝[ね]ます	
\\	見る 
\\	見て 
\\	見ます	
\\	見[み]る 
\\	見[み]て 
\\	見[み]ます	
\\	来る 
\\	来て 
\\	来ます	
\\	来[く]る 
\\	来[き]て 
\\	来[き]ます	
\\	する 
\\	して 
\\	します	
\\	する 
\\	して 
\\	します	
\\	勉強する 
\\	勉強して 
\\	勉強します	
\\	勉[べん]強[きょう]する 
\\	勉[べん]強[きょう]して 
\\	勉[べん]強[きょう]します	
\\	良い	
\\	良[い]い	イꜜイ 
\\	ヨꜜイ
\\	早い	
\\	早[はや]い 
\\	速い)	ハヤꜜイ
\\	余り	
\\	余[あま]り	アꜜマリ 
\\	アマリꜜ
\\	全然 
\\	些とも	
\\	全[ぜん]然[ぜん] 
\\	些[ちっ]とも	
\\	大抵	
\\	大[たい]抵[てい]	タイテイ
\\	一寸 
\\	少し	
\\	一[ちょっ]寸[と] 
\\	少[すこ]し	
\\	時々	
\\	時[とき]々[どき]	トキꜜドキ 
\\	トキドキ
\\	良く	
\\	良[よ]く	ヨꜜク
\\	そうですね	
\\	そうですね	
\\	でも 
\\	~けど 
\\	が	
\\	でも 
\\	~けど 
\\	が	
\\	どうですか	
\\	どうですか	
\\	アルバイト	
\\	アルバイト	アルバꜜイト
\\	買い物	
\\	買[か]い 物[もの]	カイモノ
\\	貴方 
\\	貴女 
\\	貴男	
\\	貴方[あなた] 
\\	貴女[あなた] 
\\	貴男[あなた]	
\\	犬	
\\	犬[いぬ]	イヌꜜ
\\	お土産	
\\	お 土産[みやげ]	オミꜜヤ 
\\	オミヤケ°
\\	子供	
\\	子[こ]供[ども]	
\\	御飯	
\\	御[ご]飯[はん]	
\\	写真	
\\	写[しゃ]真[しん]	シャシン
\\	机	
\\	机[つくえ]	ツクエ
\\	手紙	
\\	手[て]紙[がみ]	テカ°ミ
\\	猫	
\\	猫[ねこ]	ネꜜコ
\\	パン	
\\	パン	パꜜン
\\	人	
\\	人[ひと]	ニꜜン 
\\	ヒトꜜ 
\\	ヒト
\\	お寺 
\\	寺	
\\	お 寺[てら] 
\\	寺[てら]	
\\	公園	
\\	公[こう]園[えん]	コーエン
\\	スーパー	
\\	スーパー	スꜜーパー
\\	デパート	
\\	デパート	デパꜜート
\\	バス停	
\\	バ[ば]ス[す]停[てい]	バステイ
\\	病院	
\\	病[びょう]院[いん]	ビョーイン
\\	ホテル	
\\	ホテル	ホꜜテル
\\	本屋 
\\	書店	
\\	本[ほん]屋[や] 
\\	書[しょ]店[てん]	
\\	町	
\\	町[まち]	チョꜜー 
\\	マチꜜ
\\	レストラン	
\\	レストラン	レꜜストラン
\\	昨日	
\\	昨[きの]日[う]	キノꜜー 
\\	キノー 
\\	サクꜜジツ
\\	さっき 
\\	先ほど	
\\	さっき 
\\	先[さき]ほど	
\\	〜時間	
\\	〜 時[じ]間[かん]	
\\	一時間	
\\	一[いち]時[じ]間[かん]	イチジꜜカン
\\	先週	
\\	先[せん]週[しゅう]	センシュー
\\	時	
\\	時[とき]	トキꜜ
\\	火曜日	
\\	火[か]曜[よう]日[び]	カヨꜜービ
\\	水曜日	
\\	水[すい]曜[よう]日[び]	スイヨꜜービ
\\	木曜日	
\\	木[もく]曜[よう]日[び]	モクヨꜜービ
\\	金曜日	
\\	金[きん]曜[よう]日[び]	キンヨꜜービ
\\	会う 
\\	会って 
\\	会います	
\\	会[あ]う 
\\	会[あ]って 
\\	会[あ]います	
\\	有る 
\\	有って 
\\	有ります	
\\	有[あ]る 
\\	有[あ]って 
\\	有[あ]ります	
\\	買う 
\\	買って 
\\	買います	
\\	買[か]う 
\\	買[か]って 
\\	買[か]います 
\\	飼う)	
\\	書く 
\\	書いて 
\\	書きます	
\\	書[か]く 
\\	書[か]いて 
\\	書[か]きます 
\\	描く)	
\\	撮る 
\\	撮って 
\\	撮ります	
\\	撮[と]る 
\\	撮[と]って 
\\	撮[と]ります 
\\	取る)	
\\	待つ 
\\	待って 
\\	待ちます	
\\	待[ま]つ 
\\	待[ま]って 
\\	待[ま]ちます	
\\	分かる 
\\	分かって 
\\	分かります	
\\	分[わ]かる 
\\	分[わ]かって 
\\	分[わ]かります	
\\	居る 
\\	居て 
\\	居ます	
\\	居[い]る 
\\	居[い]て 
\\	居[い]ます 
\\	要る)	
\\	〜位	
\\	〜 位[ぐらい]	
\\	御免なさい	
\\	御[ご]免[めん]なさい	
\\	だから	
\\	だから	ダꜜカラ
\\	沢山	
\\	沢[たく]山[さん]	タクサꜜン 
\\	タクサン
\\	〜と	
\\	〜と	
\\	一人で	
\\	一[ひと]人[り]で	
\\	もしもし 
\\	申し申し	
\\	もしもし 
\\	申[もう]し 申[もう]し	
\\	右	
\\	右[みぎ]	ミꜜキ° 
\\	ミキ°
\\	左	
\\	左[ひだり]	サꜜ 
\\	ヒダリ
\\	前	
\\	前[まえ]	ゼꜜン 
\\	マꜜエ
\\	後ろ	
\\	後[うし]ろ	ウシロ
\\	中	
\\	中[なか]	チュꜜー 
\\	ナꜜカ
\\	上	
\\	上[うえ]	ウエ 
\\	カꜜミ 
\\	ジョꜜー
\\	下	
\\	下[した] 
\\	舌)	シタꜜ 
\\	シタ
\\	蕎麦	
\\	蕎[そ]麦[ば] 
\\	側)	ソꜜバ
\\	隣	
\\	隣[となり]	トナリ
\\	間	
\\	間[あいだ]	アイダ
\\	其処	
\\	其[そ]処[こ]	ソコ
\\	此処	
\\	此[こ]処[こ]	ココ
\\	月曜日	
\\	月[げつ]曜[よう]日[び]	ゲツヨꜜービ
\\	一月	
\\	一[いち]月[がつ]	イチカ°ツꜜ 
\\	イチカ°ツ 
\\	ヒトꜜツキ
\\	五日	
\\	五[いつ]日[か] 
\\	何時か)	イツカ
\\	一昨日	
\\	一昨日[おととい]	オトトꜜイ 
\\	オトトイ
\\	一昨年	
\\	一昨年[おととし]	オトꜜトシ
\\	去年	
\\	去[きょ]年[ねん]	キョꜜネン 
\\	コꜜゾ
\\	九月	
\\	九[く]月[がつ]	クꜜカ°ツ
\\	五月	
\\	五[ご]月[がつ]	ゴꜜカ°ツ 
\\	サツキ
\\	九日	
\\	九[ここの]日[か]	ココノカꜜ 
\\	ココノカ
\\	再来月	
\\	再[さ]来[らい]月[げつ]	サライケ°ツ 
\\	サラꜜイケ°ツ
\\	再来週	
\\	再[さ]来[らい]週[しゅう]	サライシュー
\\	再来年	
\\	再[さ]来[らい]年[ねん]	サライネン
\\	三月	
\\	三[さん]月[がつ]	サꜜンカ°ツ 
\\	ミꜜツキ
\\	四月	
\\	四[し]月[がつ]	シカ°ツꜜ 
\\	シカ°ツ
\\	七月	
\\	七[しち]月[がつ]	シチカ°ツꜜ 
\\	シチカ°ツ
\\	十一月	
\\	十[じゅう]一[いち]月[がつ]	ジューイチカ°ツꜜ 
\\	ジューイチカ°ツ
\\	十一日	
\\	十[じゅう]一[いち]日[にち]	
\\	十月	
\\	十[じゅう]月[がつ]	ジューカ°ツꜜ 
\\	ジューカ°ツ
\\	十二月	
\\	十[じゅう]二[に]月[がつ]	ジューニカ°ツꜜ 
\\	ジューニカ°ツ
\\	十四日	
\\	十[じゅう]四[よっ]日[か]	
\\	先々週	
\\	先[せん]々[せん] 週[しゅう]	センセンシュー 
\\	センセꜜンシュー
\\	一日	
\\	一[つい]日[たち]	イチジツ 
\\	イチニチꜜ 
\\	イチニチ 
\\	ツイタチꜜ 
\\	ヒトꜜヒ
\\	十日	
\\	十[とお]日[か]	トーカ
\\	七日	
\\	七[なの]日[か]	ナヌカ 
\\	ナノカ
\\	二か月前	
\\	二[に]か 月[げつ] 前[まえ]	
\\	二月	
\\	二[に]月[がつ]	ニカ°ツꜜ 
\\	ニカ°ツ 
\\	フタꜜツキ 
\\	フタツキꜜ 
\\	フタツキ
\\	二十四日	
\\	二[に]十[じゅう]四[よっ]日[か]	
\\	八月	
\\	八[はち]月[がつ]	ハチカ°ツꜜ 
\\	ハチカ°ツ 
\\	ヤꜜツキ
\\	二十日	
\\	二十[はつ]日[か]	ハツカ
\\	二日	
\\	二[ふつ]日[か]	フツカ
\\	三日	
\\	三[みっ]日[か]	ミッカ
\\	六日	
\\	六[むい]日[か]	ムイカ
\\	八日	
\\	八[よう]日[か]	ヨーカ
\\	四日	
\\	四[よっ]日[か]	ヨッカ
\\	六月	
\\	六[ろく]月[がつ]	ロクカ°ツꜜ 
\\	ロクカ°ツ
\\	海	
\\	海[うみ]	ウꜜミ
\\	切手	
\\	切[きっ]手[て]	キッテ 
\\	キッテꜜ
\\	切符	
\\	切[きっ]符[ぷ]	キップ
\\	サーフィン	
\\	サーフィン	サꜜーフィン
\\	宿題	
\\	宿[しゅく]題[だい]	シュクダイ
\\	食べ物	
\\	食[た]べ 物[もの]	タベモꜜノ 
\\	タベꜜモノ
\\	誕生日	
\\	誕[たん]生[じょう]日[び]	タンジョꜜービ
\\	テスト	
\\	テスト	テꜜスト
\\	天気	
\\	天[てん]気[き]	テꜜンキ
\\	飲み物	
\\	飲[の]み 物[もの]	ノミꜜモノ
\\	葉書	
\\	葉[は]書[がき]	
\\	バス	
\\	バス	バꜜス
\\	飛行機	
\\	飛[ひ]行[こう]機[き]	ヒコꜜーキ
\\	部屋	
\\	室)	部[へ]屋[や]	ヘヤꜜ
\\	僕	
\\	僕[ぼく]	シモベ 
\\	シモベꜜ 
\\	ボꜜク 
\\	ボク
\\	休み	
\\	休[やす]み	ヤスミꜜ
\\	旅行	
\\	旅[りょ]行[こう]	リョコー
\\	新しい	
\\	新[あたら]しい	アタラシꜜイ
\\	熱い 
\\	暑い	
\\	熱[あつ]い 
\\	暑[あつ]い	
\\	忙しい	
\\	忙[いそが]しい	イソカ°シꜜイ 
\\	コゼワシꜜイ 
\\	セワシꜜイ
\\	大きい	
\\	大[おお]きい	オーキꜜイ
\\	面白い	
\\	面[おも]白[しろ]い	オモシロꜜイ
\\	怖い	
\\	怖[こわ]い	コワꜜイ
\\	寒い 
\\	冷たい	
\\	寒[さむ]い 
\\	冷[つめ]たい	
\\	楽しい	
\\	楽[たの]しい	タノシꜜイ
\\	小さい	
\\	小[ちい]さい	チーサꜜイ
\\	詰らない	
\\	詰[つま]らない	
\\	古い	
\\	古[ふる]い	フルꜜイ
\\	難しい	
\\	難[むずか]しい	ムズカシイ 
\\	ムズカシꜜイ
\\	易しい	
\\	易[やさ]しい 
\\	優しい)	ヤサシイ
\\	安い	
\\	安[やす]い	ヤスꜜイ
\\	嫌い(な)	
\\	嫌[きら]い(な)	
\\	綺麗(な)	
\\	綺[き]麗[れい](な)	
\\	元気(な)	
\\	元[げん]気[き](な)	
\\	静か(な)	
\\	静[しず]か(な)	
\\	好き(な)	
\\	好[す]き(な)	
\\	大嫌い(な)	
\\	大[だい]嫌[きら]い(な)	
\\	大好き(な)	
\\	大[だい]好[す]き(な)	
\\	賑やか(な)	
\\	賑[にぎ]やか(な)	
\\	ハンサム(な)	
\\	ハンサム(な)	
\\	暇(な)	
\\	暇[ひま](な)	
\\	泳ぐ 
\\	泳いで 
\\	泳ぎます	
\\	泳[およ]ぐ 
\\	泳[およ]いで 
\\	泳[およ]ぎます	
\\	乗る 
\\	乗って 
\\	乗ります	
\\	乗[の]る 
\\	乗[の]って 
\\	乗[の]ります	
\\	やる 
\\	やって 
\\	やります	
\\	やる 
\\	やって 
\\	やります	
\\	出掛ける 
\\	出掛けて 
\\	出掛けます	
\\	出[で]掛[か]ける 
\\	出[で]掛[か]けて 
\\	出[で]掛[か]けます	
\\	一緒に	
\\	一[いっ]緒[しょ]に	
\\	それから	
\\	それから	ソレカラ
\\	大丈夫	
\\	大[だい]丈[じょう]夫[ぶ]	ダイジョꜜーフ 
\\	ダイジョꜜーブ
\\	迚も	
\\	迚[とて]も	
\\	どんな	
\\	どんな	ドꜜンナ
\\	〜迄	
\\	〜 迄[まで]	
\\	〜枚	
\\	〜 枚[まい]	
\\	エアログラム	
\\	エアログラム	
\\	書留	
\\	書[かき]留[とめ]	カキトメ
\\	現像	
\\	開発)	現[げん]像[ぞう]	ゲンゾー
\\	航空便	
\\	航[こう]空[くう] 便[びん]	コークービン 
\\	コークꜜービン
\\	光沢あり	
\\	光[こう]沢[たく]あり	
\\	光沢なし	
\\	光[こう]沢[たく]なし	
\\	小包	
\\	小[こ]包[づつみ]	コズꜜツミ
\\	仕上がり	
\\	仕[し]上[あ]がり	シアカ°リ
\\	スライド	
\\	スライド	スライド
\\	速達	
\\	速[そく]達[たつ]	ソクタツ
\\	使い捨てカメラ	
\\	使[つか]い 捨[す]てカメラ	
\\	電池	
\\	電[でん]池[ち]	デꜜンチ
\\	ネガ	
\\	ネガ	ネꜜガ
\\	パノラマ	
\\	パノラマ	パノラマ
\\	引換券 
\\	引き換え券	
\\	引[ひき]換[かえ]券[けん] 
\\	引[ひ]き 換[か]え 券[けん]	
\\	フィルム	
\\	フィルム	フィꜜルム
\\	封書	
\\	封[ふう]書[しょ]	フーショ
\\	船便	
\\	船[ふな]便[びん]	フナビン 
\\	フナꜜビン
\\	プリント	
\\	プリント	プリント
\\	保険	
\\	保[ほ]険[けん]	ホケン
\\	窓口	
\\	窓[まど]口[ぐち]	マドꜜク°チ
\\	焼き増し	
\\	焼[や]き 増[ま]し	ヤキマシ
\\	お金	
\\	お 金[かね]	オカネ
\\	お祖母さん	
\\	お 祖[ば]母[あ]さん	オバꜜーサン
\\	お風呂	
\\	お 風[ふ]呂[ろ]	オフꜜロ
\\	漢字	
\\	漢[かん]字[じ]	カンジ
\\	教科書	
\\	教[きょう]科[か]書[しょ]	キョーカꜜショ
\\	今週	
\\	今[こん]週[しゅう]	コンシュー
\\	市民病院	
\\	市[し]民[みん]病[びょう]院[いん]	
\\	次	
\\	次[つぎ]	ツキ°ꜜ
\\	テレビゲーム	
\\	テレビゲーム	テレビゲꜜーム
\\	電気	
\\	電[でん]気[き]	デꜜンキ
\\	電車	
\\	電[でん]車[しゃ]	デンシャ 
\\	デꜜンシャ
\\	荷物	
\\	荷[に]物[もつ]	ニꜜモツ
\\	~頁	
\\	頁[ぺえじ]	
\\	窓	
\\	窓[まど]	マꜜド
\\	夜	
\\	夜[よる] 
\\	寄る)	ヨꜜ 
\\	ヨꜜル
\\	来週	
\\	来[らい]週[しゅう]	ライシュー
\\	来年	
\\	来[らい]年[ねん]	ライネン
\\	大変(な)	
\\	大[たい]変[へん](な)	
\\	遊ぶ 
\\	遊んで 
\\	遊びます	
\\	遊[あそ]ぶ 
\\	遊[あそ]んで 
\\	遊[あそ]びます	
\\	急ぐ 
\\	急いで 
\\	急ぎます	
\\	急[いそ]ぐ 
\\	急[いそ]いで 
\\	急[いそ]ぎます	
\\	お風呂に入る 
\\	お風呂に入って 
\\	お風呂に入ります	
\\	お 風[ふ]呂[ろ]に 入[はい]る 
\\	お 風[ふ]呂[ろ]に 入[はい]って 
\\	お 風[ふ]呂[ろ]に 入[はい]ります	
\\	返す 
\\	返して 
\\	返します	
\\	返[かえ]す 
\\	返[かえ]して 
\\	返[かえ]します	
\\	消す 
\\	消して 
\\	消します	
\\	消[け]す 
\\	消[け]して 
\\	消[け]します	
\\	死ぬ 
\\	死んで 
\\	死にます	
\\	死[し]ぬ 
\\	死[し]んで 
\\	死[し]にます	
\\	座る 
\\	座って 
\\	座ります	
\\	座[すわ]る 
\\	座[すわ]って 
\\	座[すわ]ります	
\\	立つ 
\\	立って 
\\	立ちます	
\\	立[た]つ 
\\	立[た]って 
\\	立[た]ちます	
\\	煙草を吸う 
\\	煙草を吸って 
\\	煙草を吸います	
\\	煙草[たばこ]を 吸[す]う 
\\	煙草[たばこ]を 吸[す]って 
\\	煙草[たばこ]を 吸[す]います	
\\	使う 
\\	使って 
\\	使います	
\\	使[つか]う 
\\	使[つか]って 
\\	使[つか]います	
\\	手伝う 
\\	手伝って 
\\	手伝います	
\\	手[て]伝[つだ]う 
\\	手[て]伝[つだ]って 
\\	手[て]伝[つだ]います	
\\	入る 
\\	入って 
\\	入ります	
\\	入[はい]る 
\\	入[はい]って 
\\	入[はい]ります	
\\	持つ 
\\	持って 
\\	持ちます	
\\	持[も]つ 
\\	持[も]って 
\\	持[も]ちます	
\\	休む 
\\	休んで 
\\	休みます	
\\	休[やす]む 
\\	休[やす]んで 
\\	休[やす]みます	
\\	開ける 
\\	開けて 
\\	開けます	
\\	開[あ]ける 
\\	開[あ]けて 
\\	開[あ]けます	
\\	教える 
\\	教えて 
\\	教えます	
\\	教[おし]える 
\\	教[おし]えて 
\\	教[おし]えます	
\\	降りる 
\\	降りて 
\\	降ります	
\\	降[お]りる 
\\	降[お]りて 
\\	降[お]ります	
\\	借りる 
\\	借りて 
\\	借ります	
\\	借[か]りる 
\\	借[か]りて 
\\	借[か]ります	
\\	閉める 
\\	閉めて 
\\	閉めます	
\\	閉[し]める 
\\	閉[し]めて 
\\	閉[し]めます	
\\	点ける 
\\	点けて 
\\	点けます	
\\	点[つ]ける 
\\	点[つ]けて 
\\	点[つ]けます 
\\	漬ける)	
\\	電話を掛ける 
\\	電話を掛けて 
\\	電話を掛けます	
\\	電[でん]話[わ]を 掛[か]ける 
\\	電[でん]話[わ]を 掛[か]けて 
\\	電[でん]話[わ]を 掛[か]けます	
\\	忘れる 
\\	忘れて 
\\	忘れます	
\\	忘[わす]れる 
\\	忘[わす]れて 
\\	忘[わす]れます	
\\	連れて来る 
\\	連れて来て 
\\	連れて来ます	
\\	連[つ]れて 来[く]る 
\\	連[つ]れて 来[き]て 
\\	連[つ]れて 来[き]ます	
\\	持って来る 
\\	持って来て 
\\	持って来ます	
\\	持[も]って 来[く]る 
\\	持[も]って 来[き]て 
\\	持[も]って 来[き]ます	
\\	後で	
\\	後[あと]で	
\\	遅く	
\\	遅[おそ]く	
\\	結構です。	
\\	結[けっ]構[こう]です。	
\\	直ぐ	
\\	直[す]ぐ	スꜜク°
\\	本当ですか。	
\\	本[ほん]当[とう]ですか。	
\\	ゆっくり	
\\	ゆっくり	ユックꜜリ
\\	角	
\\	角[かど]	カꜜク 
\\	カク 
\\	カクꜜ 
\\	カꜜド 
\\	ツノꜜ
\\	北	
\\	北[きた]	キタ 
\\	キタꜜ
\\	信号	
\\	信[しん]号[ごう]	シンコ°ー
\\	西	
\\	西[にし]	ニシ
\\	東	
\\	東[ひがし]	アꜜズマ 
\\	ヒカ°シ 
\\	ヒカ°シꜜ
\\	左側	
\\	左[ひだり]側[がわ]	ヒダリカ°ワ
\\	一つ目	
\\	一[ひと]つ 目[め]	ヒトツメꜜ
\\	二つ目	
\\	二[ふた]つ 目[め]	
\\	曲がる 
\\	曲がって 
\\	曲がります	
\\	曲[ま]がる 
\\	曲[ま]がって 
\\	曲[ま]がります	
\\	右側	
\\	右[みぎ]側[がわ]	ミキ°カ°ワ
\\	南	
\\	南[みなみ]	ミナミ
\\	渡る 
\\	渡って 
\\	渡ります	
\\	渡[わた]る 
\\	渡[わた]って 
\\	渡[わた]ります	
\\	姉	
\\	姉[あね]	アネ
\\	アパート	
\\	アパート	アパꜜート
\\	歌	
\\	歌[うた]	ウタꜜ
\\	男の人	
\\	男[おとこ]の 人[ひと]	
\\	会社	
\\	会[かい]社[しゃ]	カイシャ
\\	家族	
\\	家[か]族[ぞく]	カꜜゾク
\\	髪	
\\	髪[かみ] 
\\	紙)	カミꜜ
\\	兄弟	
\\	兄[きょう]弟[だい]	キョꜜーダイ 
\\	ケイテイ
\\	国	
\\	国[くに]	クニ
\\	車	
\\	車[くるま]	クルマ
\\	コンビニ	
\\	コンビニ	コンビニ
\\	食堂	
\\	食[しょく]堂[どう]	ショクドー
\\	父	
\\	父[ちち]	チチꜜ 
\\	チꜜチ
\\	シャツ	
\\	[てぃー]シャツ[しゃつ]	
\\	目	
\\	目[め]	メꜜ
\\	眼鏡	
\\	眼[め]鏡[がね]	メꜜカ°ネ
\\	頭が良い	
\\	頭[あたま]が 良[い]い	
\\	格好いい	
\\	格[かっ]好[こ]いい	
\\	可愛い	
\\	可愛[かわい]い	カワイꜜイ
\\	背が高い	
\\	背[せ]が 高[たか]い	
\\	背が低い	
\\	背[せ]が 低[ひく]い	
\\	長い	
\\	長[なが]い	ナカ°ꜜイ
\\	短い	
\\	短[みじか]い	ミジカꜜイ
\\	親切(な)	
\\	親[しん]切[せつ](な)	
\\	歌う 
\\	歌って 
\\	歌います	
\\	歌[うた]う 
\\	歌[うた]って 
\\	歌[うた]います	
\\	被る 
\\	被って 
\\	被ります	
\\	被[かぶ]る 
\\	被[かぶ]って 
\\	被[かぶ]ります	
\\	知る 
\\	知って 
\\	知ります	
\\	知[し]る 
\\	知[し]って 
\\	知[し]ります	
\\	知っています	
\\	知[し]っています	
\\	知りません	
\\	知[し]りません	
\\	住む 
\\	住んで 
\\	住みます	
\\	住[す]む 
\\	住[す]んで 
\\	住[す]みます 
\\	済む)	
\\	履く 
\\	履いて 
\\	履きます	
\\	履[は]く 
\\	履[は]いて 
\\	履[は]きます 
\\	吐く)	
\\	太る 
\\	太って 
\\	太ります	
\\	太[ふと]る 
\\	太[ふと]って 
\\	太[ふと]ります	
\\	太っています	
\\	太[ふと]っています	
\\	眼鏡を掛ける 
\\	眼鏡を掛けて 
\\	眼鏡を掛けます	
\\	眼[め]鏡[がね]を 掛[か]ける 
\\	眼[め]鏡[がね]を 掛[か]けて 
\\	眼[め]鏡[がね]を 掛[か]けます	
\\	着る 
\\	着て 
\\	着ます	
\\	着[き]る 
\\	着[き]て 
\\	着[き]ます 
\\	切る)	
\\	痩せる 
\\	痩せて 
\\	痩せます	
\\	痩[や]せる 
\\	痩[や]せて 
\\	痩[や]せます	
\\	痩せている	
\\	痩[や]せている	
\\	結婚する 
\\	結婚して 
\\	結婚します	
\\	結[けっ]婚[こん]する 
\\	結[けっ]婚[こん]して 
\\	結[けっ]婚[こん]します	
\\	一人	
\\	一[ひと]人[り]	イチꜜニン 
\\	ヒトꜜリ
\\	二人	
\\	二[ふた]人[り]	フタリꜜ 
\\	フタリ
\\	勿論	
\\	勿[もち]論[ろん]	モチꜜロン
\\	良かったら	
\\	良[よ]かったら	
\\	頭	
\\	頭[あたま]	アタマꜜ 
\\	アタꜜマ 
\\	カシラꜜ 
\\	カブリꜜ 
\\	カꜜブリ 
\\	カブリ 
\\	コーベꜜ 
\\	コーベ 
\\	コꜜーベ 
\\	ズ 
\\	ツムリꜜ
\\	兄	
\\	兄[あに]	アꜜニ 
\\	ケꜜイ
\\	うちの子	
\\	うちの 子[こ]	
\\	内の人	
\\	内[うち]の 人[ひと]	ウチノヒトꜜ
\\	奥さん	
\\	奥[おく]さん	オꜜクサン
\\	お子さん	
\\	お 子[こ]さん	オコサン
\\	お祖父さん	
\\	お 祖[じ]父[い]さん	オジꜜーサン
\\	お尻	
\\	お 尻[しり]	オシリ
\\	夫	
\\	夫[おっと]	オット 
\\	ツꜜマ
\\	肩	
\\	肩[かた]	カꜜタ
\\	家内	
\\	家[か]内[ない]	カꜜナイ
\\	上さん	
\\	上[かみ]さん	
\\	口	
\\	口[くち]	クチ
\\	首	
\\	首[くび]	クビ 
\\	コーベꜜ 
\\	コーベ 
\\	コꜜーベ
\\	ご主人	
\\	ご 主[しゅ]人[じん]	
\\	主人	
\\	主人[しゅじん]	シュꜜジン
\\	尻	
\\	尻[しり]	シリꜜ
\\	背中	
\\	背[せ]中[なか]	セナカ
\\	祖父	
\\	祖[そ]父[ふ]	ジꜜー 
\\	ジジꜜー 
\\	ソꜜフ
\\	祖母	
\\	祖[そ]母[ぼ]	ソꜜボ
\\	旦那	
\\	旦[だん]那[な]	ダンナ
\\	妻	
\\	妻[つま]	サꜜイ 
\\	ツꜜマ
\\	手	
\\	手[て]	テꜜ
\\	女房	
\\	女[にょう]房[ぼう]	ニョꜜーボー
\\	鼻	
\\	鼻[はな] 
\\	花)	ハナ
\\	母	
\\	母[はは]	ハꜜハ
\\	耳	
\\	耳[みみ]	ミミꜜ
\\	胸	
\\	胸[むね]	ムネꜜ
\\	指	
\\	指[ゆび]	ユビꜜ
\\	ワイフ	
\\	ワイフ	ワꜜイフ
\\	残念ですね。	
\\	残[ざん]念[ねん]ですね。	
\\	明後日	
\\	明後日[あさって]	アサꜜッテ 
\\	ミョーコ°ꜜニチ
\\	雨	
\\	雨[あめ] 
\\	飴)	アꜜメ 
\\	ウꜜ
\\	カメラ	
\\	カメラ	カꜜメラ
\\	カラオケ	
\\	カラオケ	
\\	空気	
\\	空[くう]気[き]	クꜜーキ
\\	今朝	
\\	今[け]朝[さ]	ケꜜサ
\\	今月	
\\	今[こん]月[げつ]	コンケ°ツ
\\	ディスコ	
\\	ディスコ	ディꜜスコ
\\	天気予報	
\\	天[てん]気[き] 予[よ]報[ほう]	テンキヨꜜホー
\\	トマト	
\\	トマト	トꜜマト
\\	夏	
\\	夏[なつ]	ナツꜜ
\\	何か	
\\	何[なに]か	ナꜜニカ
\\	パーティー	
\\	パーティー	パꜜーティー
\\	バーベキュー	
\\	バーベキュー	バーベꜜキュー
\\	箸	
\\	箸[はし] 
\\	橋)	ハꜜシ
\\	冬	
\\	冬[ふゆ]	フユꜜ
\\	ホームステイ	
\\	ホームステイ	ホームステꜜイ
\\	毎週	
\\	毎[まい]週[しゅう]	マイシュー
\\	来月	
\\	来[らい]月[げつ]	ラꜜイケ°ツ
\\	上手	
\\	上[じょう]手[ず]	ウワテ 
\\	カミテ 
\\	カミテꜜ 
\\	ジョーズꜜ
\\	下手	
\\	下[へ]手[た]	シタテ 
\\	シタデ 
\\	シモテ 
\\	シモテꜜ 
\\	ヘタꜜ
\\	有名(な)	
\\	有[ゆう]名[めい](な)	
\\	雨が降る 
\\	雨が降って 
\\	雨が降ります	
\\	雨[あめ]が 降[ふ]る 
\\	雨[あめ]が 降[ふ]って 
\\	雨[あめ]が 降[ふ]ります	
\\	洗う 
\\	洗って 
\\	洗います	
\\	洗[あら]う 
\\	洗[あら]って 
\\	洗[あら]います	
\\	言う 
\\	言って 
\\	言います 
\\	ゆ)	
\\	言[い]う 
\\	言[い]って 
\\	言[い]います	
\\	遅く成る 
\\	遅く成って 
\\	遅く成ります	
\\	遅[おそ]く 成[な]る 
\\	遅[おそ]く 成[な]って 
\\	遅[おそ]く 成[な]ります	
\\	思う 
\\	思って 
\\	思います	
\\	思[おも]う 
\\	思[おも]って 
\\	思[おも]います	
\\	切る 
\\	切って 
\\	切ります	
\\	切[き]る 
\\	切[き]って 
\\	切[き]ります 
\\	着る)	
\\	作る 
\\	作って 
\\	作ります	
\\	作[つく]る 
\\	作[つく]って 
\\	作[つく]ります	
\\	持って行く 
\\	持って行って 
\\	持って行きます	
\\	持[も]って 行[い]く 
\\	持[も]って 行[い]って 
\\	持[も]って 行[い]きます	
\\	じろじろ見る 
\\	じろじろ見て 
\\	じろじろ見ます	
\\	じろじろ 見[み]る 
\\	じろじろ 見[み]て 
\\	じろじろ 見[み]ます	
\\	始める 
\\	始めて 
\\	始めます	
\\	始[はじ]める 
\\	始[はじ]めて 
\\	始[はじ]めます	
\\	運転する 
\\	運転して 
\\	運転します	
\\	運[うん]転[てん]する 
\\	運[うん]転[てん]して 
\\	運[うん]転[てん]します	
\\	洗濯する 
\\	洗濯して 
\\	洗濯します	
\\	洗[せん]濯[たく]する 
\\	洗[せん]濯[たく]して 
\\	洗[せん]濯[たく]します	
\\	掃除する 
\\	掃除して 
\\	掃除します	
\\	掃[そう]除[じ]する 
\\	掃[そう]除[じ]して 
\\	掃[そう]除[じ]します	
\\	料理する 
\\	料理して 
\\	料理します	
\\	料[りょう]理[り]する 
\\	料[りょう]理[り]して 
\\	料[りょう]理[り]します	
\\	ううん	
\\	ううん	
\\	うん	
\\	うん	
\\	乾杯	
\\	乾[かん]杯[ぱい]	カンパイ
\\	皆で	
\\	皆[みんな]で	
\\	いい子	
\\	いい 子[こ]	
\\	色	
\\	色[いろ]	イロꜜ
\\	お弁当	
\\	お 弁[べん]当[とう]	オベントー
\\	温泉	
\\	温[おん]泉[せん]	オンセン
\\	歌舞伎	
\\	歌[か]舞[ぶ]伎[き]	カブキ
\\	ギター	
\\	ギター	ギꜜター
\\	薬	
\\	薬[くすり]	クスリ 
\\	ヤꜜク
\\	薬を飲む 
\\	薬を飲んで 
\\	薬を飲みます	
\\	薬[くすり]を 飲[の]む 
\\	薬[くすり]を 飲[の]んで 
\\	薬[くすり]を 飲[の]みます	
\\	コンサート 
\\	音楽会 
\\	演奏会	
\\	コンサート 
\\	音[おん]楽[がく]会[かい] 
\\	演[えん]奏[そう]会[かい]	
\\	今度	
\\	今[こん]度[ど]	コꜜンド
\\	作文	
\\	作[さく]文[ぶん]	サクブン
\\	試験	
\\	試[し]験[けん]	シケꜜン
\\	新幹線	
\\	新[しん]幹[かん]線[せん]	シンカꜜンセン
\\	スキー	
\\	スキー	スキꜜー
\\	先月	
\\	先[せん]月[げつ]	セꜜンケ°ツ
\\	単語	
\\	単[たん]語[ご]	タンコ°
\\	ピアノ	
\\	ピアノ	ピアノ
\\	病気	
\\	病[びょう]気[き]	ビョーキ
\\	青い	
\\	青[あお]い	アオꜜイ
\\	赤い	
\\	赤[あか]い	アカイ
\\	黒い	
\\	黒[くろ]い	クロꜜイ
\\	寂しい	
\\	寂[さび]しい	サビシꜜイ 
\\	サミシꜜイ
\\	白い	
\\	白[しろ]い	シロꜜイ
\\	若い	
\\	若[わか]い	ワカꜜイ
\\	意地悪(な)	
\\	意[い]地[じ]悪[わる](な)	
\\	踊る 
\\	踊って 
\\	踊ります	
\\	踊[おど]る 
\\	踊[おど]って 
\\	踊[おど]ります	
\\	終わる 
\\	終わって 
\\	終わります	
\\	終[お]わる 
\\	終[お]わって 
\\	終[お]わります	
\\	人気が有る 
\\	人気が有って 
\\	人気が有ります	
\\	人[にん]気[き]が 有[あ]る 
\\	人[にん]気[き]が 有[あ]って 
\\	人[にん]気[き]が 有[あ]ります	
\\	始まる	
\\	始[はじ]まる	ハジマル
\\	弾く 
\\	弾いて 
\\	弾きます	
\\	弾[ひ]く 
\\	弾[ひ]いて 
\\	弾[ひ]きます	
\\	貰う 
\\	貰って 
\\	貰います	
\\	貰[もら]う 
\\	貰[もら]って 
\\	貰[もら]います	
\\	覚える 
\\	覚えて 
\\	覚えます	
\\	覚[おぼ]える 
\\	覚[おぼ]えて 
\\	覚[おぼ]えます	
\\	出る 
\\	出て 
\\	出ます	
\\	出[で]る 
\\	出[で]て 
\\	出[で]ます	
\\	運動する 
\\	運動して 
\\	運転します	
\\	運[うん]動[どう]する 
\\	運[うん]動[どう]して 
\\	運[うん]動[どう]します	
\\	是非	
\\	是[ぜ]非[ひ]	ゼꜜヒ
\\	皆	
\\	皆[みんな]	ミナꜜ 
\\	ミナ 
\\	ミンナꜜ 
\\	ミンナ
\\	もう	
\\	もう	モꜜー 
\\	モー
\\	一つ	
\\	一[ひと]つ	ヒトꜜツ
\\	二つ	
\\	二[ふた]つ	フタツꜜ 
\\	フタツ
\\	三つ	
\\	三[みっ]つ	ミッツꜜ 
\\	ミッツ
\\	四つ	
\\	四[よっ]つ	ヨꜜツ 
\\	ヨッツꜜ 
\\	ヨッツ
\\	五つ	
\\	五[いつ]つ	イツꜜツ
\\	六つ	
\\	六[むっ]つ	ムッツꜜ 
\\	ムッツ
\\	七つ	
\\	七[なな]つ	ナナꜜツ
\\	八つ	
\\	八[やっ]つ	ヤツꜜ 
\\	ヤッツꜜ 
\\	ヤッツ
\\	九つ	
\\	九[ここの]つ	ココꜜノツ
\\	十	
\\	十[とお]	ジュꜜー 
\\	トꜜー 
\\	トー
\\	顔が青い	
\\	顔[かお]が 青[あお]い	
\\	黄色い	
\\	黄[き]色[いろ]い	キイロイ
\\	金色 
\\	ゴールド	
\\	金[きん]色[いろ] 
\\	ゴールド	
\\	銀色 
\\	シルバー	
\\	銀[ぎん]色[いろ] 
\\	シルバー	
\\	金髪	
\\	金[きん]髪[ぱつ]	キンパツ
\\	グリーン 
\\	緑	
\\	グリーン 
\\	緑[みどり]	
\\	グレー 
\\	灰色 
\\	鼠色	
\\	グレー 
\\	灰[はい]色[いろ] 
\\	鼠[ねずみ]色[いろ]	
\\	白黒	
\\	白[しろ]黒[くろ]	シꜜロクロ 
\\	シロクロ
\\	茶色い	
\\	茶[ちゃ]色[いろ]い	
\\	ピンク 
\\	桃色	
\\	ピンク 
\\	桃[もも]色[いろ]	
\\	水色	
\\	水[みず]色[いろ]	ミズイロ
\\	紫	
\\	紫[むらさき]	ムラꜜサキ
\\	秋	
\\	秋[あき]	アꜜキ
\\	駅	
\\	駅[えき]	エꜜキ
\\	お金持ち	
\\	お 金[かね]持[も]ち	
\\	顔	
\\	顔[かお]	カオ 
\\	カンバセ
\\	季節	
\\	季[き]節[せつ]	キꜜセツ 
\\	キセꜜツ
\\	クレジットカード	
\\	クレジットカード	クレジットカꜜード
\\	今年	
\\	今年[ことし]	コトシ 
\\	コꜜンネン
\\	サッカー	
\\	サッカー	サꜜッカー
\\	シャツ	
\\	シャツ	シャꜜツ
\\	生活	
\\	生[せい]活[かつ]	セイカツ
\\	世界	
\\	世)	世[せ]界[かい]	セꜜカイ 
\\	セカꜜイ
\\	地下鉄	
\\	地[ち]下[か]鉄[てつ]	チカテツ
\\	手袋	
\\	手[て]袋[ぶくろ]	テブꜜクロ
\\	床屋	
\\	床[とこ]屋[や]	トコヤ
\\	春	
\\	春[はる]	シュꜜン 
\\	ハꜜル
\\	パンツ	
\\	パンツ	パꜜンツ 
\\	パンツ
\\	美容院	
\\	美[び]容[よう]院[いん]	ビヨꜜーイン
\\	便	
\\	便[びん]	ビꜜン 
\\	ベꜜン
\\	船 
\\	舟	
\\	船[ふね] 
\\	舟[ふね]	
\\	野球	
\\	野[や]球[きゅう]	ヤキュー
\\	有名人	
\\	有[ゆう]名[めい]人[じん]	ユーメꜜイジン
\\	予約	
\\	予[よ]約[やく]	ヨヤク
\\	来学期	
\\	来[らい] 学[がっ]期[き]	ライカ°ꜜッキ
\\	林檎	
\\	林[りん]檎[ご]	リンコ°
\\	暖かい 
\\	温かい	
\\	暖[あたた]かい 
\\	温[あたた]かい	
\\	遅い	
\\	遅[おそ]い	オソイ
\\	涼しい	
\\	涼[すず]しい	スズシꜜイ
\\	眠い	
\\	眠[ねむ]い	ネムイ
\\	簡単(な)	
\\	簡[かん]単[たん](な)	
\\	掛かる 
\\	掛かって 
\\	掛かります	
\\	掛[か]かる 
\\	掛[か]かって 
\\	掛[か]かります	
\\	泊まる 
\\	泊まって 
\\	泊まります	
\\	泊[と]まる 
\\	泊[と]まって 
\\	泊[と]まります	
\\	成る 
\\	成って 
\\	成ります	
\\	成[な]る 
\\	成[な]って 
\\	成[な]ります 
\\	鳴る)	
\\	払う 
\\	払って 
\\	払います	
\\	払[はら]う 
\\	払[はら]って 
\\	払[はら]います	
\\	決める 
\\	決めて 
\\	決めます	
\\	決[き]める 
\\	決[き]めて 
\\	決[き]めます	
\\	練習する 
\\	練習して 
\\	練習します	
\\	練[れん]習[しゅう]する 
\\	練[れん]習[しゅう]して 
\\	練[れん]習[しゅう]します	
\\	歩いて	
\\	歩[ある]いて	
\\	一番	
\\	一[いち]番[ばん]	イチꜜバン 
\\	イチバン 
\\	ヒトꜜツカ°イ
\\	此の頃	
\\	此[こ]の 頃[ごろ]	
\\	どうやって	
\\	どうやって	
\\	何方	
\\	何方[どちら]	ドꜜチラ 
\\	ドꜜッチ 
\\	ドꜜナタ
\\	何の位	
\\	何[ど]の 位[ぐらい]	
\\	早く	
\\	早[はや]く	ハꜜヤク
\\	~行き	
\\	行[い]き	
\\	一番後ろ	
\\	一[いち]番[ばん] 後[うし]ろ	
\\	一番前	
\\	一[いち]番[ばん] 前[まえ]	
\\	入口	
\\	入[いり]口[ぐち]	
\\	往復	
\\	往[おう]復[ふく]	オーフク
\\	改札	
\\	改[かい]札[さつ]	カイサツ
\\	回数券	
\\	回[かい]数[すう] 券[けん]	カイスꜜーケン
\\	階段	
\\	階[かい]段[だん]	カイダン
\\	学割	
\\	学[がく]割[わり]	ガクワリ
\\	片道	
\\	片[かた]道[みち]	カタミチ
\\	切符売り場	
\\	切[きっ]符[ぷ] 売[う]り 場[ば]	
\\	急行	
\\	急[きゅう]行[こう]	キューコー
\\	禁煙車	
\\	禁[きん]煙[えん] 車[しゃ]	
\\	指定席	
\\	指[し]定[てい] 席[せき]	シテꜜイセキ
\\	次発	
\\	次[つぎ]発[はつ]	
\\	自由席	
\\	自[じ]由[ゆう] 席[せき]	ジユꜜーセキ
\\	終電	
\\	終[しゅう]電[でん]	シューデン
\\	乗車券	
\\	乗[じょう]車[しゃ] 券[けん]	ジョーシャꜜケン
\\	整理券	
\\	整[せい]理[り] 券[けん]	
\\	先発	
\\	先[せん]発[ぱつ]	センパツ
\\	次は~	
\\	次[つぎ]は~	
\\	定期券	
\\	定[てい]期[き] 券[けん]	テイキꜜケン
\\	出口	
\\	出[で]口[ぐち]	デꜜク°チ
\\	特急	
\\	特[とっ]急[きゅう]	トッキュー
\\	乗り換え	
\\	乗[の]り 換[か]え	ノリカエ
\\	売店	
\\	売[ばい]店[てん]	バイテン
\\	~番線	
\\	番[ばん]線[せん]	
\\	普通	
\\	普[ふ]通[つう]	フツー
\\	~方面	
\\	方[ほう]面[めん]	
\\	ホーム 
\\	プラットホーム	
\\	ホーム 
\\	プラットホーム	
\\	お菓子 
\\	甘い物 
\\	飴	
\\	お 菓[か]子[し] 
\\	甘[あま]い 物[もの] 
\\	飴[あめ]	
\\	お正月 
\\	御正月	
\\	お 正[しょう]月[がつ] 
\\	御[お]正[しょう]月[がつ]	
\\	男の子	
\\	男[おとこ]の 子[こ]	オトコꜜノコ
\\	玩具	
\\	玩[おも]具[ちゃ]	オモꜜチャ
\\	女の子	
\\	女[おんな]の 子[こ]	オンナꜜノコ
\\	外国	
\\	外[がい]国[こく]	ガイコク 
\\	トツクニ 
\\	トꜜツクニ 
\\	トツꜜクニ
\\	歌手	
\\	歌[か]手[しゅ]	カꜜシュ
\\	キャンプ	
\\	キャンプ	キャꜜンプ
\\	今学期	
\\	今[こん]学[がっ]期[き]	
\\	社長	
\\	社[しゃ]長[ちょう]	シャチョー
\\	授業 
\\	クラス	
\\	授[じゅ]業[ぎょう] 
\\	クラス	
\\	将来	
\\	将[しょう]来[らい]	ショꜜーライ
\\	ドライブ	
\\	ドライブ	ドラꜜイブ
\\	ビール	
\\	ビール	ビꜜール
\\	美術館	
\\	美[び]術[じゅつ]館[かん]	ビジュツꜜカン 
\\	ビジュꜜツカン
\\	ホストファミリー	
\\	ホストファミリー	
\\	湖	
\\	湖[みずうみ]	ミズウꜜミ
\\	山	
\\	山[やま]	ヤマꜜ
\\	夢	
\\	夢[ゆめ]	ユメꜜ
\\	ルームメート	
\\	ルームメート	ルームメꜜート
\\	嘘をつく 
\\	嘘をついて 
\\	嘘をつきます	
\\	嘘[うそ]をつく 
\\	嘘[うそ]をついて 
\\	嘘[うそ]をつきます	
\\	お腹が空く 
\\	お腹が空いて 
\\	お腹が空きます	
\\	お 腹[なか]が 空[す]く 
\\	お 腹[なか]が 空[す]いて 
\\	お 腹[なか]が 空[す]きます	
\\	飼う 
\\	飼って 
\\	飼います	
\\	飼[か]う 
\\	飼[か]って 
\\	飼[か]います 
\\	買う)	
\\	サボる 
\\	サボって 
\\	サボります	
\\	サ[さ]ボ[ぼ]る 
\\	サ[さ]ボ[ぼ]って 
\\	サ[さ]ボ[ぼ]ります	
\\	取る 
\\	取って 
\\	取ります	
\\	取[と]る 
\\	取[と]って 
\\	取[と]ります 
\\	撮る)	
\\	習う 
\\	習って 
\\	習います	
\\	学ぶ)	習[なら]う 
\\	習[なら]って 
\\	習[なら]います	
\\	登る 
\\	登って 
\\	登ります	
\\	登[のぼ]る 
\\	登[のぼ]って 
\\	登[のぼ]ります	
\\	働く 
\\	働いて 
\\	働きます	
\\	働[はたら]く 
\\	働[はたら]いて 
\\	働[はたら]きます	
\\	疲れる 
\\	疲れて 
\\	疲れます	
\\	疲[つか]れる 
\\	疲[つか]れて 
\\	疲[つか]れます	
\\	辞める 
\\	辞めて 
\\	辞めます	
\\	辞[や]める 
\\	辞[や]めて 
\\	辞[や]めます	
\\	喧嘩する	
\\	喧[けん]嘩[か]する	
\\	紹介する	
\\	紹[しょう]介[かい]する	
\\	ダイエットする	
\\	ダイエットする	
\\	遅刻する	
\\	遅[ち]刻[こく]する	
\\	留学する	
\\	留[りゅう]学[がく]する	
\\	後	
\\	後[あと]	アꜜト 
\\	ゴ 
\\	シリꜜ 
\\	ノチꜜ 
\\	ノチ
\\	出身	
\\	出[しゅっ]身[しん]	シュッシン
\\	凄く	
\\	とても) 
\\	凄[すご]く	スꜜコ°ク 
\\	スコ°ꜜク
\\	そして	
\\	そして	ソシテ
\\	久しぶり	
\\	久[ひさ]しぶり	ヒサシブリ 
\\	ヒサシブリꜜ
\\	まあまあ	
\\	まあまあ	
\\	もっと	
\\	もっと	モꜜット
\\	硬い言い方	
\\	硬[かた]い 言[い]い 方[かた]	
\\	括弧	
\\	括[かっ]弧[こ]	カꜜッコ
\\	~行目	
\\	行[ぎょう]目[め]	
\\	砕けた言い方	
\\	砕[くだ]けた 言[い]い 方[かた]	
\\	答	
\\	答[こたえ]	
\\	質問	
\\	質[しつ]問[もん]	シツモン
\\	締切 
\\	締め切り	
\\	締[しめ]切[きり] 
\\	締[し]め 切[き]り	
\\	丁寧な言い方	
\\	丁[てい]寧[ねい]な 言[い]い 方[かた]	
\\	[ばつ]	
\\	発音	
\\	発[はつ]音[おん]	ハツオン
\\	~番	
\\	番[ばん]	
\\	標準語	
\\	標[ひょう]準[じゅん] 語[ご]	ヒョージュンコ°
\\	二人宛	
\\	二[ふた]人[り]宛[ずつ]	
\\	文法	
\\	文[ぶん]法[ぽう]	ブンポー
\\	方言	
\\	方[ほう]言[げん]	ホーケ°ꜜン
\\	外に	
\\	外[ほか]に	
\\	まる	〇, 
\\	まる	
\\	例	
\\	例[れい] 
\\	礼)	レꜜイ
\\	練習	
\\	練[れん]習[しゅう]	レンシュー
\\	足	
\\	足[あし]	アシꜜ
\\	意味	
\\	意[い]味[み]	イꜜミ
\\	風	
\\	風[かぜ]	カゼ 
\\	フꜜー
\\	彼女	
\\	彼[かの]女[じょ]	カꜜノジョ
\\	彼	
\\	彼[かれ]	アレ 
\\	カꜜレ
\\	気温	
\\	気[き]温[おん]	キオン
\\	曇り	
\\	曇[くも]り	クモリꜜ
\\	試合	
\\	試[し]合[あい]	シアイ
\\	ジュース	
\\	ジュース	ジュꜜース
\\	成績	
\\	成[せい]績[せき]	セイセキ
\\	咳	
\\	咳[せき] 
\\	席)	セキꜜ
\\	喉	
\\	喉[のど]	
\\	歯	
\\	歯[は] 
\\	葉)	ハꜜ
\\	花	
\\	花[はな] 
\\	鼻)	ハナꜜ
\\	晴れ	
\\	晴[は]れ	ハレꜜ
\\	服	
\\	服[ふく] 
\\	吹く)	フクꜜ
\\	二日酔い	
\\	二[ふつ]日[か]酔[よ]い	フツカヨイ
\\	ホームシック	
\\	ホームシック	ホームシꜜック
\\	物	
\\	物[もの]	モノꜜ 
\\	モノ
\\	雪	
\\	雪[ゆき]	ユキꜜ
\\	用事	
\\	用[よう]事[じ]	ヨージ
\\	甘い	
\\	甘[あま]い	アマイ 
\\	ウマꜜイ
\\	痛い	
\\	痛[いた]い	イタꜜイ
\\	多い	
\\	多[おお]い	オꜜーイ 
\\	オオꜜイ
\\	狭い	
\\	狭[せま]い	セマꜜイ
\\	都合が悪い	
\\	都[つ]合[ごう]が 悪[わる]い	
\\	悪い	
\\	悪[わる]い	ワルꜜイ
\\	素敵(な)	
\\	素[す]敵[てき](な)	
\\	風邪を引く 
\\	風邪を引いて 
\\	風邪を引きます	
\\	風邪[かぜ]を 引[ひ]く 
\\	風邪[かぜ]を 引[ひ]いて 
\\	風邪[かぜ]を 引[ひ]きます	
\\	興味が有る 
\\	興味が有って 
\\	興味が有ります	
\\	興[きょう]味[み]が 有[あ]る 
\\	興[きょう]味[み]が 有[あ]って 
\\	興[きょう]味[み]が 有[あ]ります	
\\	無くす 
\\	無くして 
\\	無くします	
\\	無[な]くす 
\\	無[な]くして 
\\	無[な]くします	
\\	喉が渇く 
\\	喉が渇いて 
\\	喉が渇きます	
\\	喉[のど]が 渇[かわ]く 
\\	喉[のど]が 渇[かわ]いて 
\\	喉[のど]が 渇[かわ]きます	
\\	熱が有る 
\\	熱が有って 
\\	熱が有ります	
\\	熱[ねつ]が 有[あ]る 
\\	熱[ねつ]が 有[あ]って 
\\	熱[ねつ]が 有[あ]ります	
\\	咳が出る 
\\	咳が出て 
\\	咳が出ます	
\\	咳[せき]が 出[で]る 
\\	咳[せき]が 出[で]て 
\\	咳[せき]が 出[で]ます	
\\	別れる 
\\	別れて 
\\	別れます	
\\	別[わか]れる 
\\	別[わか]れて 
\\	別[わか]れます	
\\	緊張する	
\\	緊[きん]張[ちょう]する	
\\	心配する	
\\	心[しん]配[ぱい]する	
\\	何時も	
\\	何[い]時[つ]も	
\\	お大事に。	
\\	お 大[だい]事[じ]に。	
\\	元気が無い	
\\	元[げん]気[き]が 無[な]い	
\\	多分	
\\	多[た]分[ぶん]	タꜜブン 
\\	タブン
\\	出来るだけ 
\\	成るべく	
\\	出[で]来[き]るだけ 
\\	成[な]るべく	
\\	二三日	
\\	二[に]三[さん]日[にち]	ニꜜサンニチ 
\\	ニサꜜンニチ 
\\	ニꜜサンチ 
\\	ニサꜜンチ
\\	初めて	
\\	初[はじ]めて	ハジꜜメテ
\\	もう直ぐ	
\\	もう 直[す]ぐ	
\\	眼科医	
\\	眼[がん]科[か]医[い]	ガンカꜜイ
\\	怪我	
\\	怪[け]我[が]	ケカ°ꜜ
\\	外科医	
\\	外[げ]科[か]医[い]	ゲカꜜイ
\\	抗生物質	
\\	抗[こう]生[せい]物[ぶっ]質[しつ]	コーセイブꜜッシツ
\\	産婦人科医	
\\	産[さん]婦[ふ]人[じん]科[か]医[い]	
\\	歯科医 
\\	歯医者	
\\	歯[し]科[か]医[い] 
\\	歯[は]医[い]者[しゃ]	
\\	耳鼻科医	
\\	耳[じ]鼻[び] 科[か]医[い]	
\\	手術	
\\	手[しゅ]術[じゅつ]	シュꜜジュツ
\\	整形外科	
\\	整[せい]形[けい] 外[げ]科[か]	セイケイゲꜜカ
\\	体温計	
\\	体[たい]温[おん]計[けい]	タイオンケイ 
\\	タイオꜜンケイ
\\	注射	
\\	注[ちゅう]射[しゃ]	チューシャ
\\	内科医	
\\	内[ない]科[か]医[い]	
\\	皮膚科医	
\\	皮[ひ]膚[ふ] 科[か]医[い]	
\\	レントゲン	
\\	レントゲン	レントゲン
\\	ウェイター	
\\	ウェイター	
\\	大人	
\\	大人[おとな]	ウꜜシ 
\\	オトナ 
\\	タイジン 
\\	ダイニン
\\	外国語	
\\	外[がい]国[こく]語[ご]	ガイコクコ°
\\	楽器	
\\	楽[がっ]器[き] 
\\	学期)	ガッキ
\\	空手	
\\	空[から]手[て]	カラテ
\\	カレー	
\\	カレー	カレー
\\	着物	
\\	着[き]物[もの]	キモノ
\\	ケーキ	
\\	ケーキ	ケꜜーキ
\\	広告	
\\	広[こう]告[こく]	コーコク
\\	紅茶	
\\	紅[こう]茶[ちゃ]	コーチャ
\\	言葉	
\\	言[こと]葉[ば]	コトバꜜ
\\	ゴルフ	
\\	ゴルフ	ゴꜜルフ
\\	セーター	
\\	セーター	セꜜーター
\\	象	
\\	象[ぞう]	ゾꜜー
\\	バイオリン	
\\	バイオリン	バイオリン
\\	オートバイ 
\\	バイク	
\\	オートバイ 
\\	バイク	
\\	物価	
\\	物[ぶっ]価[か]	ブッカ
\\	募集	
\\	募[ぼ]集[しゅう]	ボシュー
\\	店	
\\	店[みせ]	タナ 
\\	ミセꜜ
\\	やくざ	
\\	やくざ	ヤꜜクザ
\\	約束	
\\	約[やく]束[そく]	ヤクソク
\\	レポート	
\\	レポート	
\\	私 
\\	私[わたくし] 
\\	嬉しい	
\\	嬉[うれ]しい	ウレシꜜイ
\\	悲しい	
\\	悲[かな]しい	カナシイ
\\	辛い	
\\	辛[から]い	カラꜜイ 
\\	ツライ
\\	厳しい	
\\	厳[きび]しい	キビシꜜイ
\\	近い	
\\	側)	近[ちか]い	チカꜜイ
\\	色々(な) 
\\	色んな	
\\	色[いろ]々[いろ](な) 
\\	色[いろ]んな	
\\	幸せ (な)	
\\	幸[しあわ]せ (な)	
\\	駄目	
\\	駄[だ]目[め]	ダメꜜ
\\	編む 
\\	編んで 
\\	編みます	
\\	編[あ]む 
\\	編[あ]んで 
\\	編[あ]みます	
\\	貸す 
\\	貸して 
\\	貸します	
\\	貸[か]す 
\\	貸[か]して 
\\	貸[か]します	
\\	頑張る 
\\	頑張って 
\\	頑張ります	
\\	頑[がん]張[ば]る 
\\	頑[がん]張[ば]って 
\\	頑[がん]張[ば]ります	
\\	泣く 
\\	泣いて 
\\	泣きます	
\\	泣[な]く 
\\	泣[な]いて 
\\	泣[な]きます	
\\	磨く 
\\	磨いて 
\\	磨きます	
\\	磨[みが]く 
\\	磨[みが]いて 
\\	磨[みが]きます	
\\	約束を守る 
\\	約束を守って 
\\	約束を守ります	
\\	約[やく]束[そく]を 守[まも]る 
\\	約[やく]束[そく]を 守[まも]って 
\\	約[やく]束[そく]を 守[まも]ります	
\\	感動する	
\\	感[かん]動[どう]する	
\\	いらっしゃいます	
\\	いらっしゃいます	
\\	実は	
\\	実[じつ]は	ジツꜜワ
\\	全部	
\\	全[ぜん]部[ぶ]	ゼꜜンブ
\\	一日 
\\	一[いち] 日[にち] 
\\	二日 
\\	二[ふつ]日[か] 
\\	三日 
\\	三[みっ]日[か] 
\\	四日 
\\	四[よっ]日[か] 
\\	五日 
\\	五[いつ]日[か] 
\\	六日 
\\	六[むい]日[か] 
\\	七日 
\\	七[なの]日[か] 
\\	八日 
\\	八[よう]日[か] 
\\	九日 
\\	九[ここの]日[か] 
\\	十日 
\\	十[とお]日[か] 
\\	宜しくお願いします	
\\	宜[よろ]しくお 願[ねが]いします	
\\	縫い包み	
\\	縫[ぬ]い 包[ぐる]み	
\\	大家さん	
\\	大[おお]家[や]さん	
\\	お返し	
\\	お 返[かえ]し	オカエシ
\\	伯父さん 
\\	叔父さん	
\\	伯父[おじ]さん 
\\	叔父[おじ]さん	
\\	伯母さん 
\\	叔母さん	
\\	伯母[おば]さん 
\\	叔母[おば]さん	
\\	倶楽部	
\\	倶[ク]楽[ラ]部[ブ]	
\\	クリスマス	
\\	クリスマス	クリスꜜマス
\\	コップ	
\\	コップ	コップ
\\	皿	
\\	皿[さら]	サラ
\\	時間	
\\	時[じ]間[かん]	ジカン
\\	チョコレート	
\\	チョコレート	チョコレꜜート
\\	電子メール	
\\	電[でん]子[し]メール	デンシメꜜール
\\	ネクタイ	
\\	ネクタイ	ネꜜクタイ
\\	バレンタインデー	
\\	バレンタインデー	バレンタイꜜンデー
\\	ビデオカメラ	
\\	ビデオカメラ	ビデオカꜜメラ
\\	夫婦	
\\	夫[ふう]婦[ふ]	フꜜーフ
\\	ホワイトデー	
\\	ホワイトデー	
\\	マフラー	
\\	マフラー	マꜜフラー
\\	漫画	
\\	漫[まん]画[が]	マンカ°
\\	マンション	
\\	マンション	マꜜンション
\\	蜜柑	
\\	蜜[み]柑[かん]	ミꜜカン
\\	指輪	
\\	指[ゆび]輪[わ]	ユビワ
\\	両親	
\\	両[りょう]親[しん]	リョꜜーシン
\\	履歴書	
\\	履[り]歴[れき] 書[しょ]	リレキショ 
\\	リレキショꜜ
\\	欲しい	
\\	欲[ほ]しい 
\\	欲する 
\\	ホシꜜイ
\\	けち(な)	
\\	けち(な)	
\\	送る 
\\	送って 
\\	送ります	
\\	送[おく]る 
\\	送[おく]って 
\\	送[おく]ります	
\\	似合う 
\\	似合って 
\\	似合います	
\\	似[に]合[あ]う 
\\	似[に]合[あ]って 
\\	似[に]合[あ]います	
\\	諦める 
\\	諦めて 
\\	諦めます	
\\	諦[あきら]める 
\\	諦[あきら]めて 
\\	諦[あきら]めます	
\\	上げる 
\\	上げて 
\\	上げます	
\\	上[あ]げる 
\\	上[あ]げて 
\\	上[あ]げます	
\\	呉れる 
\\	呉れて 
\\	呉れます	
\\	呉[く]れる 
\\	呉[く]れて 
\\	呉[く]れます 
\\	暮れる)	
\\	相談する	
\\	相[そう]談[だん]する	
\\	プロポーズする	
\\	プロポーズする	
\\	同じ	
\\	同[おな]じ	オナジ 
\\	オンナジ
\\	私達	
\\	私[わたし]達[たち]	ワタシꜜタチ
\\	丁度	
\\	丁[ちょう]度[ど]	チョード
\\	~個	
\\	個[こ]	
\\	~冊	
\\	冊[さつ]	
\\	~台	
\\	台[だい]	
\\	~匹	
\\	匹[ひき]	
\\	絵	
\\	絵[え]	エꜜ
\\	映画館	
\\	映[えい]画[が]館[かん]	エイカ°ꜜカン
\\	外国人	
\\	外[がい]国[こく]人[じん]	ガイコクꜜジン
\\	家具	
\\	家[か]具[ぐ]	カꜜク°
\\	観光 
\\	見物	
\\	観[かん]光[こう] 
\\	見[けん]物[ぶつ]	
\\	結婚式	
\\	結[けっ]婚[こん]式[しき]	ケッコꜜンシキ
\\	地震	
\\	地[じ]震[しん]	ジシン
\\	ジャケット	
\\	ジャケット	ジャꜜケット 
\\	ジャケꜜット
\\	習慣	
\\	習[しゅう]慣[かん] 
\\	週間)	シューカン
\\	税金	
\\	税[ぜい]金[きん]	ゼイキン
\\	卒業式	
\\	卒[そつ]業[ぎょう]式[しき]	ソツキ°ョꜜーシキ
\\	地図	
\\	地[ち]図[ず]	チꜜズ
\\	庭	
\\	庭[にわ]	ニワ
\\	発表	
\\	発[はっ]表[ぴょう]	ハッピョー
\\	番組	
\\	番[ばん]組[ぐみ]	バンク°ミ 
\\	バンク°ミꜜ
\\	プール	
\\	プール	プꜜール
\\	ペット	
\\	ペット	ペꜜット
\\	予定	
\\	予[よ]定[てい]	ヨテイ
\\	旅館	
\\	旅[りょ]館[かん]	リョカン
\\	割引券	
\\	割[わり]引[びき]券[けん]	ワリビキꜜケン
\\	広い	
\\	広[ひろ]い	ヒロꜜイ
\\	売る 
\\	売って 
\\	売ります	
\\	売[う]る 
\\	売[う]って 
\\	売[う]ります	
\\	下ろす 
\\	下ろして 
\\	下ろします	
\\	下[お]ろす 
\\	下[お]ろして 
\\	下[お]ろします	
\\	描く 
\\	描いて 
\\	描きます	
\\	描[か]く 
\\	描[か]いて 
\\	描[か]きます 
\\	書く)	
\\	探す 
\\	探して 
\\	探します	
\\	探[さが]す 
\\	探[さが]して 
\\	探[さが]します	
\\	招待する	
\\	招[しょう]待[たい]する	
\\	付き合う 
\\	付き合って 
\\	付き合います	
\\	付[つ]き 合[あ]う 
\\	付[つ]き 合[あ]って 
\\	付[つ]き 合[あ]います	
\\	着く 
\\	着いて 
\\	着きます	
\\	着[つ]く 
\\	着[つ]いて 
\\	着[つ]きます 
\\	点く 
\\	付く)	
\\	保険に入る 
\\	保険に入って 
\\	保険に入ります	
\\	保[ほ]険[けん]に 入[はい]る 
\\	保[ほ]険[けん]に 入[はい]って 
\\	保[ほ]険[けん]に 入[はい]ります	
\\	調べる 
\\	調べて 
\\	調べます	
\\	調[しら]べる 
\\	調[しら]べて 
\\	調[しら]べます	
\\	捨てる 
\\	捨てて 
\\	捨てます	
\\	捨[す]てる 
\\	捨[す]てて 
\\	捨[す]てます	
\\	見える 
\\	見えて 
\\	見えます	
\\	見[み]える 
\\	見[み]えて 
\\	見[み]えます	
\\	卒業する	
\\	卒[そつ]業[ぎょう]する	
\\	予約する	
\\	予[よ]約[やく]する	
\\	一日中	
\\	一[いち]日[にち] 中[じゅう]	
\\	最近	
\\	最[さい]近[きん]	サイキン
\\	駅員	
\\	駅[えき]員[いん]	エキꜜイン 
\\	エキイン
\\	親	
\\	"親[おや] 
\\	オヤꜜ 
\\	シꜜン
\\	塵	
\\	塵[ごみ]	チリ
\\	砂糖	
\\	砂[さ]糖[とう]	サトꜜー
\\	親戚	
\\	親[しん]戚[せき]	シンセキ
\\	推薦状	
\\	推[すい]薦[せん] 状[じょう]	スイセンジョー 
\\	スイセꜜンジョー
\\	大学院	
\\	大[だい]学[がく]院[いん]	ダイカ°クꜜイン
\\	台風	
\\	台[たい]風[ふう]	タイフꜜー
\\	日	
\\	日[ひ] 
\\	火)	ヒꜜ 
\\	ヒ
\\	ファイル	
\\	ファイル	ファꜜイル
\\	道	
\\	道[みち]	ドꜜー 
\\	ミチ
\\	目覚まし時計	
\\	目[め]覚[ざ]まし 時[と]計[けい]	メザマシドꜜケイ
\\	汚い	
\\	汚[きたな]い	キタナꜜイ
\\	起こす 
\\	起こして 
\\	起こします	
\\	起[お]こす 
\\	起[お]こして 
\\	起[お]こします	
\\	奢る 
\\	奢って 
\\	奢ります	
\\	奢[おご]る 
\\	奢[おご]って 
\\	奢[おご]ります	
\\	落ち込む 
\\	落ち込んで 
\\	落ち込みます	
\\	落[お]ち 込[こ]む 
\\	落[お]ち 込[こ]んで 
\\	落[お]ち 込[こ]みます	
\\	困る 
\\	困って 
\\	困ります	
\\	困[こま]る 
\\	困[こま]って 
\\	困[こま]ります	
\\	出す 
\\	出して 
\\	出します	
\\	出[だ]す 
\\	出[だ]して 
\\	出[だ]します	
\\	連れて行く 
\\	連れて行って 
\\	連れて行きます	
\\	連[つ]れて 行[い]く 
\\	連[つ]れて 行[い]って 
\\	連[つ]れて 行[い]きます	
\\	直す 
\\	直して 
\\	直します	
\\	直[なお]す 
\\	直[なお]して 
\\	直[なお]します	
\\	道に迷う 
\\	道に迷って 
\\	道に迷います	
\\	道[みち]に 迷[まよ]う 
\\	道[みち]に 迷[まよ]って 
\\	道[みち]に 迷[まよ]います	
\\	笑う 
\\	笑って 
\\	笑います	
\\	笑[わら]う 
\\	笑[わら]って 
\\	笑[わら]います	
\\	アイロンを掛ける 
\\	アイロンを掛けて 
\\	アイロンを掛けます	
\\	アイロンを 掛[か]ける 
\\	アイロンを 掛[か]けて 
\\	アイロンを 掛[か]けます	
\\	集める 
\\	集めて 
\\	集めます	
\\	集[あつ]める 
\\	集[あつ]めて 
\\	集[あつ]めます	
\\	入れる 
\\	入れて 
\\	入れます	
\\	入[い]れる 
\\	入[い]れて 
\\	入[い]れます 
\\	淹れる)	
\\	乗り遅れる 
\\	乗り遅れて 
\\	乗り遅れます	
\\	乗[の]り 遅[おく]れる 
\\	乗[の]り 遅[おく]れて 
\\	乗[の]り 遅[おく]れます	
\\	見せる 
\\	見せて 
\\	見せます	
\\	見[み]せる 
\\	見[み]せて 
\\	見[み]せます	
\\	朝寝坊する	
\\	朝[あさ]寝[ね]坊[ぼう]する	
\\	案内する	
\\	案[あん]内[ない]する	
\\	説明する	
\\	説[せつ]明[めい]する	
\\	迎えに来る 
\\	迎えに来て 
\\	迎えに来ます	
\\	迎[むか]えに 来[く]る 
\\	迎[むか]えに 来[き]て 
\\	迎[むか]えに 来[き]ます	
\\	ええと	
\\	ええと	
\\	今日中に	
\\	今日[きょう] 中[ちゅう]に	
\\	此の間	
\\	此[こ]の 間[かん]	
\\	此の位	
\\	此[こ]の 位[ぐらい]	
\\	御免	
\\	御[ご]免[めん]	ゴメン
\\	失礼します	
\\	失[しつ]礼[れい]します	
\\	自分で	
\\	自[じ]分[ぶん]で	
\\	授業中に	
\\	授[じゅ]業[ぎょう] 中[ちゅう]に	
\\	他の	
\\	他[ほか]の	
\\	男	
\\	男[おとこ]	オトコꜜ
\\	お湯 
\\	湯	
\\	お 湯[ゆ] 
\\	湯[ゆ]	
\\	女 
\\	女の人 
\\	女性	
\\	女[おんな] 
\\	女[おんな]の 人[ひと] 
\\	女[じょ]性[せい]	
\\	鍵	
\\	鍵[かぎ]	カキ°ꜜ
\\	紙	
\\	紙[かみ] 
\\	髪)	カミꜜ
\\	給料	
\\	給[きゅう]料[りょう]	キュꜜーリョー
\\	コンタクト	
\\	コンタクト	コꜜンタクト 
\\	コンタꜜクト
\\	最低	
\\	最[さい]低[てい]	サイテイ
\\	サラリーマン	
\\	サラリーマン	サラリꜜーマン
\\	残業	
\\	残[ざん]業[ぎょう]	ザンキ°ョー
\\	ストーブ	
\\	ストーブ	ストꜜーブ
\\	宝籤	
\\	宝[たから]籤[くじ]	
\\	違い	
\\	違[ちが]い	チカ°イ
\\	ニュース	
\\	ニュース	ニュꜜース
\\	髭	
\\	髭[ひげ]	ヒケ°
\\	秘密	
\\	秘[ひ]密[みつ]	ヒミツ
\\	寮	
\\	寮[りょう]	リョꜜー
\\	旅行会社	
\\	旅[りょ]行[こう]会[がい]社[しゃ]	
\\	羨ましい	
\\	羨[うらや]ましい	ウラヤマシꜜイ
\\	少ない	
\\	少[すく]ない	スクナꜜイ
\\	強い	
\\	強[つよ]い	キツイ 
\\	コワꜜイ 
\\	ツヨꜜイ
\\	選ぶ 
\\	選んで 
\\	選びます	
\\	選[えら]ぶ 
\\	選[えら]んで 
\\	選[えら]びます	
\\	髪を梳かす 
\\	髪を梳かして 
\\	髪を梳かします	
\\	髪[かみ]を 梳[と]かす 
\\	髪[かみ]を 梳[と]かして 
\\	髪[かみ]を 梳[と]かします	
\\	お湯を沸かす 
\\	お湯を沸かして 
\\	お湯を沸かします	
\\	お 湯[ゆ]を 沸[わ]かす 
\\	お 湯[ゆ]を 沸[わ]かして 
\\	お 湯[ゆ]を 沸[わ]かします	
\\	込む 
\\	込んで 
\\	込みます	
\\	込[こ]む 
\\	込[こ]んで 
\\	込[こ]みます	
\\	宝籤に当たる 
\\	宝籤に当たって 
\\	宝籤に当たります	
\\	宝[たから]籤[くじ]に 当[あ]たる 
\\	宝[たから]籤[くじ]に 当[あ]たって 
\\	宝[たから]籤[くじ]に 当[あ]たります	
\\	脱ぐ 
\\	脱いで 
\\	脱ぎます	
\\	脱[ぬ]ぐ 
\\	脱[ぬ]いで 
\\	脱[ぬ]ぎます	
\\	髭を剃る 
\\	髭を剃って 
\\	髭を剃ります	
\\	髭[ひげ]を 剃[そ]る 
\\	髭[ひげ]を 剃[そ]って 
\\	髭[ひげ]を 剃[そ]ります	
\\	淹れる 
\\	淹れて 
\\	淹れます	
\\	淹[い]れる 
\\	淹[い]れて 
\\	淹[い]れます 
\\	入れる)	
\\	鍵を掛ける 
\\	鍵を掛けて 
\\	鍵を掛けます	
\\	鍵[かぎ]を 掛[か]ける 
\\	鍵[かぎ]を 掛[か]けて 
\\	鍵[かぎ]を 掛[か]けます	
\\	足りる 
\\	足りて 
\\	足ります	
\\	足[た]りる 
\\	足[た]りて 
\\	足[た]ります	
\\	慣れる 
\\	慣れて 
\\	慣れます	
\\	慣[な]れる 
\\	慣[な]れて 
\\	慣[な]れます	
\\	化粧する	
\\	化[け]粧[しょう]する	
\\	就職する	
\\	就[しゅう]職[しょく]する	
\\	離婚する	
\\	離[り]婚[こん]する	
\\	御目出度う御座います	
\\	御[お]目[め]出[で]度[と]う 御[ご]座[ざ]います	
\\	自分	
\\	自[じ]分[ぶん]	ジブン
\\	そうか	
\\	そうか	
\\	それに	
\\	それに	ソレニ
\\	例えば 
\\	〜とか	
\\	例[たと]えば 
\\	〜とか	
\\	〜に比べて	
\\	〜に 比[くら]べて	
\\	〜に依ると	
\\	〜に 依[よ]ると	
\\	矢っ張り	
\\	矢[や]っ 張[ぱ]り	
\\	エアコン	
\\	エアコン	エアコン
\\	牛乳	
\\	牛[ぎゅう]乳[にゅう]	ギューニュー
\\	シャンプー	
\\	シャンプー	シャꜜンプー
\\	スイッチ	
\\	スイッチ	スイꜜッチ 
\\	スꜜイッチ
\\	スカート	
\\	スカート	スカꜜート
\\	醤油	
\\	醤[しょう]油[ゆ]	ショーユ
\\	外	
\\	外[そと]	ソꜜト 
\\	ホカ
\\	タオル	
\\	タオル	タꜜオル
\\	日記	
\\	日[にっ]記[き]	ニッキ
\\	ポップコーン	
\\	ポップコーン	ポップコꜜーン
\\	家賃	
\\	家[や]賃[ちん]	ヤꜜチン
\\	夕方	
\\	夕[ゆう]方[がた]	ユーカ°タ
\\	ラジカセ	
\\	ラジカセ	ラジカセ
\\	留守番電話	
\\	留[る]守[す]番[ばん]電[でん]話[わ]	ルスバンデꜜンワ
\\	冷蔵庫	
\\	冷[れい]蔵[ぞう]庫[こ]	レイゾꜜーコ
\\	蝋燭	
\\	蝋[ろう]燭[そく]	ローソꜜク 
\\	ローソクꜜ
\\	明るい	
\\	明[あか]るい	アカルイ
\\	気分が悪い	
\\	気[き]分[ぶん]が 悪[わる]い	
\\	恥ずかしい	
\\	恥[は]ずかしい	ハズカシꜜイ
\\	開く 
\\	開いて 
\\	開きます 開く 
\\	開いて 
\\	開きます	
\\	開[ひら]く 
\\	開[ひら]いて 
\\	開[ひら]きます 開[あ]く 
\\	開[あ]いて 
\\	開[あ]きます	
\\	謝る 
\\	謝って 
\\	謝ります	
\\	謝[あやま]る 
\\	謝[あやま]って 
\\	謝[あやま]ります	
\\	押す 
\\	押して 
\\	押します	
\\	押[お]す 
\\	押[お]して 
\\	押[お]します	
\\	落とす 
\\	落として 
\\	落とします	
\\	落[お]とす 
\\	落[お]として 
\\	落[お]としま	
\\	お湯が沸く 
\\	お湯が沸いて 
\\	お湯が沸きます	
\\	お 湯[ゆ]が 沸[わ]く 
\\	お 湯[ゆ]が 沸[わ]いて 
\\	お 湯[ゆ]が 沸[わ]きます	
\\	転ぶ 
\\	転んで 
\\	転びます	
\\	転[ころ]ぶ 
\\	転[ころ]んで 
\\	転[ころ]びます	
\\	壊す 
\\	壊して 
\\	壊します	
\\	壊[こわ]す 
\\	壊[こわ]して 
\\	壊[こわ]します	
\\	咲く 
\\	咲いて 
\\	咲きます	
\\	咲[さ]く 
\\	咲[さ]いて 
\\	咲[さ]きます	
\\	閉まる 
\\	閉まって 
\\	閉まります	
\\	閉[し]まる 
\\	閉[し]まって 
\\	閉[し]まります	
\\	助かる 
\\	助かって 
\\	助かります	
\\	助[たす]かる 
\\	助[たす]かって 
\\	助[たす]かります	
\\	頼む 
\\	頼んで 
\\	頼みます	
\\	頼[たの]む 
\\	頼[たの]んで 
\\	頼[たの]みます	
\\	点く 
\\	点いて 
\\	点きます	
\\	点[つ]く 
\\	点[つ]いて 
\\	点[つ]きます 
\\	着く 
\\	付く)	
\\	汚す 
\\	汚して 
\\	汚します 汚す 
\\	汚して 
\\	汚します	
\\	汚[けが]す 
\\	汚[けが]して 
\\	汚[けが]します 汚[よご]す 
\\	汚[よご]して 
\\	汚[よご]します	
\\	落ちる 
\\	落ちて 
\\	落ちます	
\\	落[お]ちる 
\\	落[お]ちて 
\\	落[お]ちます	
\\	考える 
\\	考えて 
\\	考えます	
\\	考[かんが]える 
\\	考[かんが]えて 
\\	考[かんが]えます	
\\	消える 
\\	消えて 
\\	消えます	
\\	消[き]える 
\\	消[き]えて 
\\	消[き]えます	
\\	壊れる 
\\	壊れて 
\\	壊れます	
\\	壊[こわ]れる 
\\	壊[こわ]れて 
\\	壊[こわ]れます	
\\	汚れる 
\\	汚れて 
\\	汚れます	
\\	汚[よご]れる 
\\	汚[よご]れて 
\\	汚[よご]れます	
\\	注文する	
\\	注[ちゅう]文[もん]する	
\\	今直ぐ	
\\	今[いま]直[す]ぐ	
\\	お陰で	
\\	お 陰[かげ]で	
\\	お先に失礼します	
\\	お 先[さき]に 失[しつ]礼[れい]します	
\\	お疲れ様でした	
\\	お 疲[つか]れ 様[さま]でした	
\\	本当に	
\\	本[ほん]当[とう]に	
\\	先ず	
\\	先[ま]ず	マꜜズ
\\	〜 迄に	
\\	〜 迄[まで]に	
\\	どうしよう	
\\	どうしよう	
\\	奥様	
\\	奥[おく]様[さま]	オꜜクサマ
\\	お礼	
\\	お 礼[れい]	オレイ
\\	敬語	
\\	敬[けい]語[ご]	ケイコ°
\\	出張	
\\	出[しゅっ]張[ちょう]	シュッチョー
\\	此方	
\\	此[こ]方[ちら]	コチラ 
\\	コッチꜜ 
\\	コꜜナタ 
\\	コナꜜタ 
\\	コノꜜカタ 
\\	コノカタꜜ 
\\	コノカꜜタ
\\	種類	
\\	種[しゅ]類[るい]	シュꜜルイ
\\	性格	
\\	性[せい]格[かく] 
\\	正確)	セイカク
\\	中学生	
\\	中[ちゅう]学[がく]生[せい]	チューカ°ꜜクセイ 
\\	チューカ°クꜜセイ
\\	怠け者	
\\	怠[なま]け 者[もの]	ナマケモノ 
\\	ナマケモノꜜ
\\	悩み	
\\	悩[なや]み	ナヤミꜜ
\\	話	
\\	話[はなし]	ハナシꜜ
\\	恥ずかしがり屋	
\\	恥[は]ずかしがり 屋[や]	ハズカシカ°リヤ
\\	部長	
\\	部[ぶ]長[ちょう]	ブチョー
\\	文化	
\\	文[ぶん]化[か]	ブꜜンカ
\\	間違い	
\\	間[ま]違[ちが]い	
\\	仲が良い	
\\	仲[なか]が 良[い]い	
\\	真面目	
\\	真[ま]面[じ]目[め]	シンメꜜンモク 
\\	シンメꜜンボク 
\\	マジメ
\\	いらっしゃる 
\\	いらっしゃって 
\\	いらっしゃいます	
\\	行く, 来る, 
\\	居る	いらっしゃる 
\\	いらっしゃって 
\\	いらっしゃいます	
\\	怒る 
\\	怒って 
\\	怒ります	
\\	怒[おこ]る 
\\	怒[おこ]って 
\\	怒[おこ]ります	
\\	仰っしゃる 
\\	仰っしゃって 
\\	仰っしゃいます	
\\	言う	仰[お]っしゃる 
\\	仰[お]っしゃって 
\\	仰[お]っしゃいます	
\\	お休みに成る 
\\	お休みに成って 
\\	お休みに成ります	
\\	寝る	お 休[やす]みに 成[な]る 
\\	お 休[やす]みに 成[な]って 
\\	お 休[やす]みに 成[な]ります	
\\	決まる 
\\	決まって 
\\	決まります	
\\	決[き]まる 
\\	決[き]まって 
\\	決[き]まります	
\\	下さる 
\\	下さって 
\\	下さいます	
\\	呉れる	下[くだ]さる 
\\	下[くだ]さって 
\\	下[くだ]さいます	
\\	ご覧に成る 
\\	ご覧に成って 
\\	ご覧に成ります	
\\	見る	ご 覧[らん]に 成[な]る 
\\	ご 覧[らん]に 成[な]って 
\\	ご 覧[らん]に 成[な]ります	
\\	知り合う 
\\	知り合って 
\\	知り合います	
\\	知[し]り 合[あ]う 
\\	知[し]り 合[あ]って 
\\	知[し]り 合[あ]います	
\\	〜ていらっしゃる 
\\	〜ていらっしゃって 
\\	〜ていらっしゃいます	
\\	〜ている	〜ていらっしゃる 
\\	〜ていらっしゃって 
\\	〜ていらっしゃいます	
\\	なさる 
\\	なさって 
\\	なさいます	
\\	する	なさる 
\\	なさって 
\\	なさいます	
\\	引っ越す 
\\	引っ越して 
\\	引っ越します	
\\	引[ひ]っ 越[こ]す 
\\	引[ひ]っ 越[こ]して 
\\	引[ひ]っ 越[こ]します	
\\	召し上がる 
\\	召し上がって 
\\	召し上がります	
\\	食べる 
\\	飲む	召[め]し 上[あ]がる 
\\	召[め]し 上[あ]がって 
\\	召[め]し 上[あ]がります	
\\	呼ぶ 
\\	呼んで 
\\	呼びます	
\\	呼[よ]ぶ 
\\	呼[よ]んで 
\\	呼[よ]びます	
\\	寄る 
\\	寄って 
\\	寄ります	
\\	寄[よ]る 
\\	寄[よ]って 
\\	寄[よ]ります	
\\	遅れる 
\\	遅れて 
\\	遅れます	
\\	遅[おく]れる 
\\	遅[おく]れて 
\\	遅[おく]れます	
\\	掛ける 
\\	掛けて 
\\	掛けます	
\\	掛[か]ける 
\\	掛[か]けて 
\\	掛[か]けます	
\\	晴れる 
\\	晴れて 
\\	晴れます	
\\	晴[は]れる 
\\	晴[は]れて 
\\	晴[は]れます	
\\	持てる 
\\	持てて 
\\	持てます	
\\	持[も]てる 
\\	持[も]てて 
\\	持[も]てます	
\\	遠慮する	
\\	遠[えん]慮[りょ]する	
\\	ご馳走する 
\\	ご馳走して 
\\	ご馳走します	
\\	ご 馳[ち]走[そう]する 
\\	ご 馳[ち]走[そう]して 
\\	ご 馳[ち]走[そう]します	
\\	注意する 
\\	気を付ける	
\\	注[ちゅう]意[い]する 
\\	気[き]を 付[つ]ける	
\\	話をする	
\\	話[はなし]をする	
\\	何故 
\\	如何して	
\\	何[な]故[ぜ] 
\\	如[ど]何[う]して	
\\	本当は	
\\	本[ほん]当[とう]は	
\\	毎朝	
\\	毎[まい]朝[あさ]	マꜜイアサ 
\\	マイアサ
\\	未だ	
\\	未[ま]だ	イマダ 
\\	イꜜマダ 
\\	マꜜダ
\\	〜名様	
\\	〜 名[めい] 様[さま]	
\\	ようこそ	
\\	ようこそ	ヨꜜーコソ
\\	宜しくお伝え下さい	
\\	宜[よろ]しくお 伝[つた]え 下[くだ]さい	
\\	彼方	
\\	彼方[あちら]	アチラ 
\\	アッチꜜ 
\\	アꜜナタ 
\\	カꜜナタ
\\	宇宙人	
\\	宇[う]宙[ちゅう] 人[じん]	ウチュꜜージン
\\	音	
\\	音[おと]	オトꜜ
\\	係の者	
\\	係[かかり]の 者[もの]	
\\	神様	
\\	神[かみ]様[さま]	カꜜミサマ
\\	字	
\\	字[じ]	アꜜザ 
\\	アザナ 
\\	ジꜜ
\\	支店	
\\	支[し]店[てん]	シテン
\\	趣味	
\\	趣[しゅ]味[み]	シュꜜミ
\\	小説	
\\	小[しょう]説[せつ]	ショーセツ
\\	スニーカー	
\\	スニーカー	スニꜜーカー 
\\	スニーカー
\\	扇子	
\\	扇[せん]子[す]	センス
\\	月	
\\	月[つき]	ツキꜜ
\\	ハイヒール	
\\	ハイヒール	ハイヒꜜール
\\	夕べ	
\\	夕[ゆう]べ	ユーベꜜ
\\	重い	
\\	重[おも]い	オモイ
\\	歩く 
\\	歩いて 
\\	歩きます	
\\	歩[ある]く 
\\	歩[ある]いて 
\\	歩[ある]きます	
\\	致す 
\\	致して 
\\	致します	
\\	する	致[いた]す 
\\	致[いた]して 
\\	致[いた]します	
\\	頂く 
\\	頂いて 
\\	頂きます	
\\	食べる 
\\	飲む	頂[いただ]く 
\\	頂[いただ]いて 
\\	頂[いただ]きます	
\\	頂く(貰う) 
\\	頂いて 
\\	頂きます	
\\	貰う	頂[いただ]く
\\	貰[もら]う) 
\\	頂[いただ]いて 
\\	頂[いただ]きます	
\\	伺う 
\\	伺って 
\\	伺います	
\\	伺[うかが]う/ 伺[うかが]って 
\\	伺[うかが]います	
\\	居る 
\\	居って 
\\	居ります 
\\	居る	居[お]る 
\\	居[お]って 
\\	居[お]ります	
\\	御座る 
\\	御座って 
\\	御座います	
\\	有る	御[ご]座[ざ]る 
\\	御[ご]座[ざ]って 
\\	御[ご]座[ざ]います	
\\	〜ておる 
\\	〜ておって 
\\	〜ております	
\\	〜ている	〜ておる 
\\	〜ておって 
\\	〜ております	
\\	〜で御座る 
\\	〜で御座います	
\\	です	〜で 御[ご]座[ざ]る 
\\	〜で 御[ご]座[ざ]います	
\\	参る 
\\	参って 
\\	参ります	
\\	行く 
\\	来る	参[まい]る 
\\	参[まい]って 
\\	参[まい]ります	
\\	申す 
\\	申して 
\\	申します	
\\	言う	申[もう]す 
\\	申[もう]して 
\\	申[もう]します	
\\	戻る 
\\	戻って 
\\	戻ります	
\\	戻[もど]る 
\\	戻[もど]って 
\\	戻[もど]ります	
\\	聞こえる 
\\	聞こえて 
\\	聞こえます	
\\	聞[き]こえる 
\\	聞[き]こえて 
\\	聞[き]こえます	
\\	差し上げる 
\\	差し上げて 
\\	差し上げます	
\\	上げる	差[さ]し 上[あ]げる 
\\	差[さ]し 上[あ]げて 
\\	差[さ]し 上[あ]げます	
\\	伝える 
\\	伝えて 
\\	伝えます	
\\	伝[つた]える 
\\	伝[つた]えて 
\\	伝[つた]えます	
\\	待たせる 
\\	待たせて 
\\	待たせます	
\\	待[ま]たせる 
\\	待[ま]たせて 
\\	待[ま]たせます	
\\	交換する 
\\	取り替える 
\\	取り替えて 
\\	取り替えます	
\\	交[こう]換[かん]する 
\\	取[と]り 替[か]える 
\\	取[と]り 替[か]えて 
\\	取[と]り 替[か]えます	
\\	生活する	
\\	生[せい]活[かつ]する	
\\	返品する	
\\	返[へん]品[ぴん]する	
\\	おや	
\\	おや 
\\	親)	
\\	〜階	
\\	〜 階[かい] 
\\	回)	
\\	畏まりました。	
\\	畏[かしこ]まりました。	
\\	さあ	
\\	さあ	
\\	失礼しました	
\\	失[しつ]礼[れい]しました	
\\	少々	
\\	少[しょう]々[しょう]	ショꜜーショー
\\	其れでは	
\\	其[そ]れでは	
\\	出来れば	
\\	出[で]来[き]れば	
\\	誠に	
\\	誠[まこと]に	
\\	又	
\\	又[また]	マタꜜ 
\\	マタ
\\	申し訳有りません	
\\	申[もう]し 訳[わけ]有[あ]りません	
\\	宜しかったら	
\\	宜[よろ]しかったら	
\\	赤ちゃん 
\\	赤ん坊	
\\	赤[あか]ちゃん 
\\	赤[あか]ん 坊[ぼう]	
\\	蚊	
\\	蚊[か]	カ
\\	会議	
\\	会[かい]議[ぎ]	カꜜイキ° 
\\	カイキ°ꜜ
\\	ガソリン	
\\	ガソリン	ガソリン
\\	環境	
\\	環[かん]境[きょう]	カンキョー
\\	警察	
\\	警[けい]察[さつ]	ケイサツ
\\	工場	
\\	工[こう]場[じょう]	コージョꜜー 
\\	コーバꜜ
\\	事	
\\	事[こと]	コトꜜ
\\	スピーチ	
\\	スピーチ	スピꜜーチ
\\	政府	
\\	政[せい]府[ふ]	セꜜイフ
\\	痴漢	
\\	痴[ち]漢[かん]	チカン
\\	泥棒	
\\	泥[どろ]棒[ぼう]	ドロボー
\\	バイト	
\\	アルバイト	バイト	バイト 
\\	バꜜイト
\\	犯人	
\\	犯[はん]人[にん]	ハꜜンニン
\\	ポスター	
\\	ポスター	ポꜜスター
\\	昔	
\\	昔[むかし]	ムカシ
\\	文句	
\\	文[もん]句[く]	モꜜンク
\\	文句を言う 
\\	文句を言って 
\\	文句を言います	
\\	文[もん]句[く]を 言[い]う 
\\	文[もん]句[く]を 言[い]って 
\\	文[もん]句[く]を 言[い]います	
\\	留守	
\\	留[る]守[す]	ルꜜス
\\	遠い	
\\	遠[とお]い	トーイ
\\	酷い	
\\	酷[ひど]い	ヒドꜜイ
\\	安全	
\\	安[あん]全[ぜん]	アンゼン
\\	大切(な)	
\\	大[たい]切[せつ](な)	
\\	目茶苦茶	
\\	目[め]茶[ちゃ]苦[く]茶[ちゃ]	
\\	置く 
\\	置いて 
\\	置きます	
\\	置[お]く 
\\	置[お]いて 
\\	置[お]きます	
\\	気が付く 
\\	気が付いて 
\\	気が付きます	
\\	気[き]が 付[つ]く 
\\	気[き]が 付[つ]いて 
\\	気[き]が 付[つ]きます	
\\	蹴る 
\\	蹴って 
\\	蹴ります	
\\	蹴[け]る 
\\	蹴[け]って 
\\	蹴[け]ります	
\\	刺す 
\\	刺して 
\\	刺します	
\\	刺[さ]す 
\\	刺[さ]して 
\\	刺[さ]します	
\\	触る 
\\	触って 
\\	触ります	
\\	触[さわ]る 
\\	触[さわ]って 
\\	触[さわ]ります	
\\	捕まる 
\\	捕まって 
\\	捕まります	
\\	捕[つか]まる 
\\	捕[つか]まって 
\\	捕[つか]まります	
\\	包む 
\\	包んで 
\\	包みます	
\\	包[つつ]む 
\\	包[つつ]んで 
\\	包[つつ]みます	
\\	殴る 
\\	殴って 
\\	殴ります 
\\	打つ 
\\	打って 
\\	打ちます	
\\	殴[なぐ]る 
\\	殴[なぐ]って 
\\	殴[なぐ]ります 
\\	打[う]つ 
\\	打[う]って 
\\	打[う]ちます	
\\	盗む 
\\	盗んで 
\\	盗みます	
\\	盗[ぬす]む 
\\	盗[ぬす]んで 
\\	盗[ぬす]みます	
\\	貼る 
\\	貼って 
\\	貼ります	
\\	貼[は]る 
\\	貼[は]って 
\\	貼[は]ります 
\\	春)	
\\	踏む 
\\	踏んで 
\\	踏みます	
\\	踏[ふ]む 
\\	踏[ふ]んで 
\\	踏[ふ]みます	
\\	振る 
\\	振って 
\\	振ります	
\\	振[ふ]る 
\\	振[ふ]って 
\\	振[ふ]ります 
\\	降る)	
\\	焼く 
\\	焼いて 
\\	焼きます	
\\	焼[や]く 
\\	焼[や]いて 
\\	焼[や]きます	
\\	苛める 
\\	苛めて 
\\	苛めます	
\\	苛[いじ]める 
\\	苛[いじ]めて 
\\	苛[いじ]めます	
\\	着替える 
\\	着替えて 
\\	着替えます	
\\	着[き]替[が]える 
\\	着[き]替[が]えて 
\\	着[き]替[が]えます	
\\	溜める 
\\	溜めて 
\\	溜めます	
\\	溜[た]める 
\\	溜[た]めて 
\\	溜[た]めます	
\\	続ける 
\\	続けて 
\\	続けます	
\\	続[つづ]ける 
\\	続[つづ]けて 
\\	続[つづ]けます	
\\	褒める 
\\	褒めて 
\\	褒めます	
\\	褒[ほ]める 
\\	褒[ほ]めて 
\\	褒[ほ]めます	
\\	間違える 
\\	間違えて 
\\	間違えます	
\\	間[ま]違[ちが]える 
\\	間[ま]違[ちが]えて 
\\	間[ま]違[ちが]えます	
\\	見付ける 
\\	見付けて 
\\	見付けます	
\\	見[み]付[つ]ける 
\\	見[み]付[つ]けて 
\\	見[み]付[つ]けます	
\\	準備する	
\\	準[じゅん]備[び]する	
\\	馬鹿にする	
\\	馬[ば]鹿[か]にする	
\\	吃驚する 
\\	驚く 
\\	驚いて 
\\	驚きます	
\\	吃[びっ]驚[くり]する 
\\	驚[おどろ]く 
\\	驚[おどろ]いて 
\\	驚[おどろ]きます	
\\	昼寝をする	
\\	昼[ひる]寝[ね]をする	
\\	連絡する	
\\	連[れん]絡[らく]する	
\\	〜間に	
\\	〜 間[かん]に	
\\	頃	
\\	頃[ころ]	コꜜロ
\\	兎に角	
\\	兎[と]に 角[かく]	トꜜニカク
\\	同僚	
\\	同[どう]僚[りょう]	ドーリョー
\\	相手	
\\	相[あい]手[て]	アイテꜜ
\\	受付	
\\	受[うけ]付[つけ]	ウケツケ
\\	英会話	
\\	英[えい]会[かい]話[わ]	エイカꜜイワ
\\	お嬢さん	
\\	お 嬢[じょう]さん	オジョꜜーサン
\\	家事	
\\	家[か]事[じ] 
\\	火事)	カꜜジ
\\	木	
\\	木[き]	キꜜ 
\\	ボク
\\	期末試験	
\\	期[き]末[まつ] 試[し]験[けん]	キマツシꜜケン 
\\	キマツシケꜜン
\\	後輩	
\\	後[こう]輩[はい]	コーハイ
\\	猿	
\\	猿[さる]	サꜜル 
\\	マシラ 
\\	マꜜシラ
\\	自由	
\\	自[じ]由[ゆう]	ジユꜜー
\\	塾	
\\	塾[じゅく]	ジュꜜク
\\	書類	
\\	書[しょ]類[るい]	ショルイ
\\	先輩	
\\	先[せん]輩[ぱい]	センパイ
\\	一人暮らし	
\\	一[ひと]人[り]暮[ぐ]らし	ヒトリク°ꜜラシ
\\	部下	
\\	部[ぶ]下[か]	ブꜜカ
\\	復習	
\\	復[ふく]習[しゅう]	フクシュー
\\	プロジェクト 
\\	事業	
\\	プロジェクト 
\\	事[じ]業[ぎょう]	
\\	ボール	
\\	ボール	ボール 
\\	ボꜜール
\\	無駄遣い	
\\	無[む]駄[だ]遣[づか]い	
\\	免許	
\\	免[めん]許[きょ]	メꜜンキョ
\\	予習	
\\	予[よ]習[しゅう]	ヨシュー
\\	煩い	
\\	煩[うるさ]い	ウルサꜜイ 
\\	ワズライ
\\	心配	
\\	心[しん]配[ぱい]	シンパイ
\\	貧乏(な)	
\\	貧[びん]乏[ぼう](な)	
\\	ペラペラ	
\\	ペラペラ	
\\	変	
\\	変[へん] 
\\	辺)	ヘꜜン
\\	楽	
\\	楽[らく]	ガꜜク 
\\	ガク 
\\	ラꜜク 
\\	ラクꜜ
\\	勝つ 
\\	勝って 
\\	勝ちます	
\\	勝[か]つ 
\\	勝[か]って 
\\	勝[か]ちます	
\\	コピーを取る 
\\	コピーを取って 
\\	コピーを取ります	
\\	コピーを 取[と]る 
\\	コピーを 取[と]って 
\\	コピーを 取[と]ります	
\\	運ぶ 
\\	運んで 
\\	運びます	
\\	運[はこ]ぶ 
\\	運[はこ]んで 
\\	運[はこ]びます	
\\	走る 
\\	走って 
\\	走ります	
\\	走[はし]る 
\\	走[はし]って 
\\	走[はし]ります	
\\	拾う 
\\	拾って 
\\	拾います	
\\	拾[ひろ]う 
\\	拾[ひろ]って 
\\	拾[ひろ]います	
\\	吹く 
\\	吹いて 
\\	吹きます	
\\	吹[ふ]く 
\\	吹[ふ]いて 
\\	吹[ふ]きます	
\\	放って置く 
\\	放って置いて 
\\	放って置きます	
\\	放[はな]って 置[お]く 
\\	放[はな]って 置[お]いて 
\\	放[はな]って 置[お]きます	
\\	間に合う 
\\	間に合って 
\\	間に合います	
\\	間[ま]に 合[あ]う 
\\	間[ま]に 合[あ]って 
\\	間[ま]に 合[あ]います	
\\	迎えに行く 
\\	迎えに行いて 
\\	迎えに行きます	
\\	迎[むか]えに 行[い]く 
\\	迎[むか]えに 行[い]いて 
\\	迎[むか]えに 行[い]きます	
\\	計画を立てる 
\\	計画を立てて 
\\	計画を立てます	
\\	計[けい]画[かく]を 立[た]てる 
\\	計[けい]画[かく]を 立[た]てて 
\\	計[けい]画[かく]を 立[た]てます	
\\	育てる 
\\	育てて 
\\	育てます	
\\	育[そだ]てる 
\\	育[そだ]てて 
\\	育[そだ]てます	
\\	助ける 
\\	助けて 
\\	助けます	
\\	助[たす]ける 
\\	助[たす]けて 
\\	助[たす]けます	
\\	負ける 
\\	負けて 
\\	負けます	
\\	負[ま]ける 
\\	負[ま]けて 
\\	負[ま]けます	
\\	お願いする	
\\	お 願[ねが]いする	
\\	賛成する	
\\	賛[さん]成[せい]する	
\\	失敗する	
\\	失[しっ]敗[ぱい]する	
\\	徹夜する	
\\	徹[てつ]夜[や]する	
\\	反対する	
\\	反[はん]対[たい]する	
\\	翻訳する 
\\	訳す 
\\	訳して 
\\	訳します	
\\	翻[ほん]訳[やく]する 
\\	訳[やく]す 
\\	訳[やく]して 
\\	訳[やく]します	
\\	こんな風に	
\\	こんな 風[ふう]に	
\\	ずっと	
\\	ずっと	ズット
\\	絶対に	
\\	絶[ぜっ]対[たい]に	
\\	そんな事はない	
\\	そんな 事[こと]はない	
\\	何度も	
\\	何[なん] 度[ど]も	
\\	のんびり	
\\	のんびり	ノンビꜜリ
\\	もう少し	
\\	もう 少[すこ]し	モースコꜜシ
\\	思い出	
\\	思[おも]い 出[で]	オモイデ
\\	海外旅行	
\\	海[かい]外[がい] 旅[りょ]行[こう]	
\\	体	
\\	体[からだ]	カラダ 
\\	タꜜイ 
\\	テꜜイ
\\	靴下	
\\	靴[くつ]下[した]	クツꜜシタ 
\\	クツシタꜜ
\\	国際電話	
\\	国[こく]際[さい]電[でん]話[わ]	
\\	授業料	
\\	授[じゅ]業[ぎょう] 料[りょう]	ジュキ°ョꜜーリョー
\\	小学校	
\\	小[しょう]学[がっ]校[こう]	ショーカ°ꜜッコー
\\	選挙	
\\	選挙[せんきょ]	セꜜンキョ
\\	タイヤ	
\\	タイヤ	タイヤ
\\	只	
\\	只[ただ]	タꜜダ
\\	場所 
\\	所	
\\	場[ば]所[しょ] 
\\	所[ところ]	
\\	別荘	
\\	別[べっ]荘[そう]	ベッソꜜー 
\\	ベッソー
\\	ボーナス	
\\	ボーナス	ボꜜーナス
\\	盆踊り	
\\	盆[ぼん]踊[おど]り	ボンオꜜドリ
\\	面接	
\\	面[めん]接[せつ]	メンセツ
\\	夕食	
\\	夕[ゆう]食[しょく]	ユーショク
\\	理想	
\\	理[り]想[そう]	リソー
\\	留守番	
\\	留[る]守[す]番[ばん]	ルスバン
\\	不味い	
\\	不[ま]味[ず]い	マズꜜイ
\\	雨が止む 
\\	雨が止んで 
\\	雨が止みます	
\\	雨[あめ]が 止[や]む 
\\	雨[あめ]が 止[や]んで 
\\	雨[あめ]が 止[や]みます	
\\	居なく成る	
\\	居[い]なく 成[な]る	
\\	お世話に成る 
\\	お世話に成って 
\\	お世話に成ります	
\\	お 世[せ]話[わ]に 成[な]る 
\\	お世[せ]話[わ]に 成[な]って 
\\	お 世[せ]話[わ]に 成[な]ります	
\\	お腹を壊す 
\\	お腹を壊して 
\\	お腹を壊します	
\\	お 腹[なか]を 壊[こわ]す 
\\	お 腹[なか]を 壊[こわ]して 
\\	お 腹[なか]を 壊[こわ]します	
\\	違う 
\\	違って 
\\	違います	
\\	違[ちが]う 
\\	違[ちが]って 
\\	違[ちが]います	
\\	無く成る	
\\	無[な]く 成[な]る	
\\	悪口を言う 
\\	悪口を言って 
\\	悪口を言います	
\\	悪[わる]口[ぐち]を 言[い]う 
\\	悪[わる]口[ぐち]を 言[い]って 
\\	悪[わる]口[ぐち]を 言[い]います	
\\	受ける 
\\	受けて 
\\	受けます	
\\	受[う]ける 
\\	受[う]けて 
\\	受[う]けます	
\\	換える 
\\	換えて 
\\	換えます	
\\	換[か]える 
\\	換[か]えて 
\\	換[か]えます 
\\	帰る)	
\\	離れる 
\\	離れて 
\\	離れます	
\\	離[はな]れる 
\\	離[はな]れて 
\\	離[はな]れます	
\\	〜顔をする	
\\	〜 顔[がお]をする	
\\	がっかりする	
\\	がっかりする	
\\	我慢する	
\\	我[が]慢[まん]する	
\\	世話をする	
\\	世[せ]話[わ]をする	
\\	同情する	
\\	同[どう]情[じょう]する	
\\	パンクする	
\\	パンクする	
\\	戻って来る	
\\	戻[もど]って 来[く]る	
\\	優勝する	
\\	優[ゆう]勝[しょう]する	
\\	いや	
\\	いや	
\\	元気でね。	
\\	元[げん]気[き]でね。
\\	そうそう	
\\	そうそう	ソーソー
\\	そろそろ	
\\	そろそろ	ソꜜロソロ
\\	そんな〜	
\\	そんな〜	
\\	物凄く 
\\	非常に 
\\	随分	
\\	物[もの]凄[すご]く 
\\	非[ひ]常[じょう]に 
\\	随[ずい]分[ぶん]	
\\	横	
\\	横[よこ]	ヨコ
\\	~点	
\\	点[てん]	
\\	速い	
\\	速[はや]い 
\\	早い)	ハヤꜜイ
\\	ええ	
\\	ええ	
\\	勤める 
\\	勤めて 
\\	勤めます	
\\	勤[つと]める 
\\	勤[つと]めて 
\\	勤[つと]めます	
\\	何も
\\	何[なに]も
\\	~人	
\\	人[にん]	
\\	別に 
\\	別[べつ]に
\\	未だ 
\\	未[ま]だ
\\	~から 
\\	迄)	
\\	~から 
\\	迄[まで])	
\\	所で	
\\	所[ところ]で	
\\	~か~	
\\	~か~	
\\	~か月	
\\	~か 月[げつ]	
\\	~後	
\\	後[ご]	
\\	~週間	
\\	週[しゅう]間[かん] 
\\	習慣)	
\\	~で	
\\	~で	
\\	どっち	
\\	どっち	ドꜜッチ
\\	~丈	
\\	丈[だけ]	
\\	~でしょう	
\\	~でしょう	
\\	~度	
\\	度[ど]	
\\	~ので 
\\	~から	
\\	~ので 
\\	~から	
\\	お宅	
\\	お 宅[たく]	オタク
\\	~回	
\\	回[かい] 
\\	階)	
\\	~キロ	
\\	~キロ	
\\	~と申します	
\\	~と 申[もう]します	
\\	ラジオ	
\\	ラジオ	ラꜜジオ
\\	~君	
\\	君[くん]	
\\	こんな~	
\\	こんな~	
\\	~達	
\\	達[たち]	
\\	一個、二個、三個、四個、五個、六個、七個、八個、九個、十個、何個	
\\	ーつ): 
\\	一[いっ]個[こ]、 二[に]個[こ]、 三[さん]個[こ]、 四[よん]個[こ]、 五[ご]個[こ]、 六[ろっ]個[こ]、 七[なな]個[こ]、 八[はっ]個[こ]、 九[きゅう]個[こ]、 十[じゅっ]個[こ]、 何[なん]個[こ]	
\\	一冊、二冊、三冊、四冊、五冊、六冊、七冊、八冊、九冊、十冊、何冊	
\\	一[いっ]冊[さつ]、 二[に]冊[さつ]、 三[さん]冊[さつ]、 四[よん]冊[さつ]、 五[ご]冊[さつ]、 六[ろく]冊[さつ]、 七[なな]冊[さつ]、 八[はっ]冊[さつ]、 九[きゅう]冊[さつ]、 十[じゅっ]冊[さつ]、 何[なん]冊[さつ]	
\\	一匹、二匹、三匹、四匹、五匹、六匹、七匹、八匹、九匹、十匹、何匹	
\\	一[いっ]匹[ぴき]、 二[に]匹[ひき]、 三[さん]匹[びき]、 四[よん]匹[ひき]、 五[ご]匹[ひき]、 六[ろっ]匹[ぴき]、 七[なな]匹[ひき]、 八[はっ]匹[ぴき]、 九[きゅう]匹[ひき]、 十[じゅっ]匹[ぴき]、 何[なん]匹[びき]	
\\	一本、二本、三本、四本、五本、六本、七本、八本、九本、十本、何本	
\\	一[いっ]本[ぽん]、 二[に]本[ほん]、 三[さん]本[ぼん]、 四[よん]本[ほん]、 五[ご]本[ほん]、 六[ろっ]本[ぽん]、 七[なな]本[ほん]、 八[はっ]本[ぽん]、 九[きゅう]本[ほん]、 十[じゅっ]本[ぽん]、 何[なん]本[ぼん]	
\\	一台、二台、三台、四台、五台、六台、七台、八台、九台、十台、何台	
\\	一[いち]台[だい]、 二[に]台[だい]、 三[さん]台[だい]、 四[よん]台[だい]、 五[ご]台[だい]、 六[ろく]台[だい]、 七[なな]台[だい]、 八[はち]台[だい]、 九[きゅう]台[だい]、 十[じゅう]台[だい]、 何[なん]台[だい]	
\\	一枚、二枚、三枚、四枚、五枚、六枚、七枚、八枚、九枚、十枚、何枚	
\\	一[いち]枚[まい]、 二[に]枚[まい]、 三
\\	さん]枚[まい]、 四[よん]枚[まい]、 五[ご]枚[まい]、 六[ろく]枚[まい]、 七[なな]枚[まい]、 八[はち]枚[まい]、 九[きゅう]枚[まい]、 十[じゅう]枚[まい]、 何[なん]枚[まい]	
\\	~に就いて	
\\	~に 就[つ]いて	
\\	~目	
\\	目[め]	
\\	もう一度	
\\	もう 一[いち]度[ど]	
\\	要る 
\\	要って 
\\	要ります	
\\	要[い]る 
\\	要[い]って 
\\	要[い]ります 
\\	居る)	
\\	携帯電話	
\\	携[けい]帯[たい]電[でん]話[わ]	ケイタイデꜜンワ
\\	一輪車	
\\	一[いち]輪[りん]車[しゃ]
\\	側	
\\	近い)	側[そば] 
\\	蕎麦)	カワꜜ 
\\	ガワ 
\\	ソꜜバ
\\	殺気	
\\	殺[さっ]気[き] 
\\	先ほど)	サッキ 
\\	サꜜッキ
\\	新参者	
\\	新[しん]参[ざん]者[もの]	シンザンモノ
\\	殺す 
\\	殺して 
\\	殺します	
\\	殺[ころ]す 
\\	殺[ころ]して 
\\	殺[ころ]します	
\\	光沢	
\\	光[こう]沢[たく]	コータク
\\	市民	
\\	市[し]民[みん]	シꜜミン
\\	坐禅	
\\	坐[ざ]禅[ぜん]	
\\	無所得	
\\	無[む]所[しょ]得[とく]	
\\	御家族	
\\	御[ご]家[か]族[ぞく]	
\\	残念	
\\	残[ざん]念[ねん]	ザンネꜜン
\\	運転免許証	
\\	運[うん]転[てん]免[めん]許[きょ]証[しょう]	
\\	散歩する	
\\	散[さん]歩[ぽ]する	
\\	売り場	
\\	売[う]り 場[ば]	ウリバ
\\	お腹	
\\	お 腹[なか]	オナカ
\\	議会	
\\	議[ぎ]会[かい]	ギꜜカイ
\\	都合が良い 
\\	便利な	
\\	都[つ]合[ごう]が 良[い]い 
\\	便[べん]利[り]な
\\	言い方	
\\	言[い]い 方[かた]	イーカタ
\\	丁寧(な)	
\\	丁[てい]寧[ねい](な)
\\	巻末	
\\	巻[かん]末[まつ]	カンマツ
\\	活用表	
\\	活[かつ]用[よう]表[ひょう]	
\\	活用	
\\	活[かつ]用[よう]	カツヨー
\\	音読み	
\\	音[おん]読[よ]み	オンヨミ
\\	訓読み	
\\	訓[くん]読[よ]み	クンヨミ
\\	お早う	
\\	お 早[はよ]う	
\\	お早う御座います	
\\	お 早[はよ]う 御[ご]座[ざ]います	
\\	今日は	
\\	今[こん]日[にち]は	
\\	今晩は	
\\	今[こん]晩[ばん]は	
\\	左様なら	
\\	左[さ]様[よう]なら	
\\	お休みなさい	
\\	お 休[やす]みなさい	
\\	有難う	
\\	有[あり]難[がと]う	
\\	有難う御座います	
\\	有[あり]難[がと]う 御[ご]座[ざ]いいます	
\\	済みません	
\\	済[す]みません
\\	行って来ます	
\\	行[い]って 来[き]ます
\\	行ってらっしゃい	
\\	行[い]ってらっしゃい
\\	只今	
\\	只[ただ]今[いま]	タダꜜイマ 
\\	タダイマꜜ
\\	お帰りなさい	
\\	お 帰[かえ]りなさい	
\\	頂きます	
\\	頂[いただ]きます
\\	御馳走様	
\\	御[ご]馳[ち]走[そう]様[さま]	ゴチソーサマ
\\	どうぞ宜しく	
\\	どうぞ 宜[よろ]しく
\\	十二時半です。	
\\	十[じゅう]二[に]時[じ]半[はん]です。
\\	学生です。	
\\	学[がく]生[せい]です。
\\	専門は日本語です。	
\\	専[せん]門[もん]は 日[に]本[ほん]語[ご]です。
\\	留学生です。	
\\	留[りゅう]学[がく]生[せい]です。
\\	留学生ですか。	
\\	留[りゅう]学[がく]生[せい]ですか。
\\	専門は何ですか。	
\\	専[せん]門[もん]は 何[なに]ですか。
\\	専門は英語です。	
\\	専[せん]門[もん]は 英[えい]語[ご]です。
\\	今は何時ですか。	
\\	今[いま]は 何[なん]時[じ]ですか。
\\	黒田さんは何歳ですか。	
\\	黒[くろ]田[だ]さんは 何[なん]歳[さい]ですか。
\\	十九歳です。	
\\	十[じゅう]九[きゅう]歳[さい]です。
\\	何年生ですか。	
\\	何[なん]年[ねん]生[せい]ですか。
\\	二年生です。	
\\	二[に]年[ねん]生[せい]です。
\\	電話番号は何ですか。	
\\	電[でん]話[わ]番[ばん]号[ごう]は 何[なん]ですか。
\\	大学の先生	
\\	大[だい]学[がく]の 先[せん]生[せい]
\\	日本語の学生	
\\	日[に]本[ほん]語[ご]の 学[がく]生[せい]
\\	日本の大学	
\\	日[に]本[ほん]の 大[だい]学[がく]	
\\	たけしの電話番号	
\\	たけしの 電[でん]話[わ]番[ばん]号[ごう]
\\	たけしさんのお母さんは高校の先生です。	
\\	たけしさんのお 母[かあ]さんは 高[こう]校[こう]の 先[せん]生[せい]です。
\\	私の電話番号は37-8667です。	
\\	私[わたし]の 電[でん]話[わ]番[ばん]号[ごう]は 3[さん]7[なな]の 8[はち]6[ろく]6[ろく]7[なな]です。
\\	教師	
\\	教[きょう]師[し]	キョꜜーシ
\\	零	
\\	零[れい/ぜろ]	レꜜイ
\\	百	
\\	百[ひゃく]	ヒャクꜜ
\\	二十	
\\	二[に]十[じゅう]
\\	三十	
\\	三[さん]十[じゅう]
\\	四十	
\\	四[よん]十[じゅう]	シジュꜜー 
\\	ヨꜜンジュー
\\	五十	
\\	五[ご]十[じゅう]
\\	六十	
\\	六[ろく]十[じゅう]	ロクジュꜜー
\\	七十	
\\	七[なな]十[じゅう]
\\	八十	
\\	八[はち]十[じゅう]	ハチジュꜜー
\\	九十	
\\	九[きゅう]十[じゅう]
\\	四十五	
\\	四[よん]十[じゅう]五[ご]	
\\	八十三	
\\	八[はち]十[じゅう]三[さん]	
\\	七十六	
\\	七[なな]十[じゅう]六[ろく]
\\	五十二	
\\	五[ご]十[じゅう]二[に]
\\	三十八	
\\	三[さん]十[じゅう]八[はち]
\\	六十一	
\\	六[ろく]十[じゅう]一[いち]	
\\	二十四	
\\	二[に]十[じゅう]四[よん] 
\\	二[に]十[じゅう]四[し]
\\	九十七	
\\	九[きゅう]十[じゅう]七[なな] 
\\	九[きゅう]十[じゅう]七[しち]
\\	東京は今何時ですか。	
\\	東[とう]京[きょう]は 今[いま]何[なん]時[じ]ですか。
\\	午前三時です。	
\\	午[ご]前[ぜん]三[さん]時[じ]です。
\\	いいえ。	
\\	いいえ。
\\	午前四時半です。	
\\	午[ご]前[ぜん]四[よ]時[じ]半[はん]です。
\\	午後四時半です。	
\\	午[ご]後[ご]四[よ]時[じ]半[はん]です。
\\	午後九時半です。	
\\	午[ご]後[ご]九[く]時[じ]半[はん]です。
\\	午前九時半です。	
\\	午[ご]前[ぜん]九[く]時[じ]半[はん]です。
\\	午後十一時半です。	
\\	午[ご]後[ご]十[じゅう]一[いち]時[じ]半[はん]です。
\\	携帯電話番号は何ですか。	
\\	携[けい]帯[たい]電[でん]話[わ]番[ばん]号[ごう]は 何[なん]ですか。
\\	私の携帯電話番号は37-8667です。	
\\	私[わたし]の 携[けい]帯[たい]電[でん]話[わ]番[ばん]号[ごう]は 3[さん]7[なな]の 8[はち]6[ろく]6[ろく]7[なな]です。
\\	お名前は?	
\\	お 名[な]前[まえ]は?
\\	何処から来ましたか。	
\\	何[ど]処[こ]から 来[き]ましたか。
\\	仕事は何ですか。	
\\	仕[し]事[ごと]は 何[なん]ですか。
\\	何歳ですか。	
\\	何[なん]歳[さい]ですか。
\\	有難う御座います。	
\\	有[あり]難[がと]う 御[ご]座[ざ]います。
\\	どう致しまして。	
\\	どう 致[いた]しまして。
\\	会話	
\\	会[かい]話[わ]	カイワ
\\	此れは幾らですか。	
\\	此[こ]れは 幾[いく]らですか。
\\	それは三千円です。	
\\	それは 三[さん]千[ぜん]円[えん]です。
\\	高いですね。	
\\	高[たか]いですね。
\\	じゃあ、あの時計は幾らですか。	
\\	じゃあ、あの 時[と]計[けい]は 幾[いく]らですか。
\\	あれは三千五百円です。	
\\	あれは 三[さん]千[ぜん]五[ご]百[ひゃく]円[えん]です。
\\	そうですか。あれも高いですね。	
\\	そうですか。あれも 高[たか]いですね。
\\	千八百円	
\\	千[せん]八[はっ]百[ぴゃく]円[えん]	
\\	じゃあ、其の時計を下さい。	
\\	じゃあ、 其[そ]の 時[と]計[けい]を 下[くだ]さい。
\\	此れは誰の財布ですか。	
\\	此[こ]れは 誰[だれ]の 財[さい]布[ふ]ですか。
\\	私の財布です。有難う御座います。	
\\	私[わたし]の 財[さい]布[ふ]です。 有[あり]難[がと]う 御[ご]座[ざ]います。
\\	いらっしゃいませ。メニューをどうぞ。	
\\	いらっしゃいませ。メニューをどうぞ。
\\	どうも。此れは何ですか。	
\\	どうも。 此[こ]れは 何[なん]ですか。
\\	何れですか。ああ、豚かつです。	
\\	何[ど]れですか。ああ、 豚[とん]かつです。
\\	豚かつ?魚ですか。	
\\	豚[とん]かつ? 魚[さかな]ですか。
\\	いいえ、魚じゃありません。肉です。美味しいですよ。	
\\	いいえ、 魚[さかな]じゃありません。 肉[にく]です。 美[お]味[い]しいですよ。
\\	じゃあ、此れをお願いします。	
\\	じゃあ、 此[こ]れをお 願[ねが]いします。
\\	済みません。お手洗いは何処ですか。	
\\	済[す]みません。お 手[て]洗[あら]いは 何[ど]処[こ]ですか。
\\	彼処です。	
\\	彼[あそ]処[こ]です。
\\	此の時計は幾らですか。	
\\	此[こ]の 時[と]計[けい]は 幾[いく]らですか。
\\	あの時計は九千円です。	
\\	あの 時[と]計[けい]は 九[きゅう]千[せん]円[えん]です。
\\	何の時計が九千円ですか。	
\\	何[ど]の 時[と]計[けい]が 九[きゅう]千[せん]円[えん]ですか。
\\	此れは誰の鞄ですか。	
\\	此[こ]れは 誰[だれ]の 鞄[かばん]ですか。
\\	済みません、郵便局は何処ですか。	
\\	済[す]みません 、 郵[ゆう]便[びん]局[きょく]は 何[ど]処[こ]ですか。
\\	郵便局は彼処です。	
\\	郵[ゆう]便[びん]局[きょく]は 彼[あそ]処[こ]です。
\\	たけしさんは日本人です。	
\\	たけしさんは 日[に]本[ほん]人[じん]です。
\\	美智子さんも日本人です。	
\\	美[み]智[ち]子[こ]さんも 日[に]本[ほん]人[じん]です。
\\	山田さんは学生じゃありません。	
\\	山[やま]田[だ]さんは 学[がく]生[せい]じゃありません。
\\	リーさんの専門は文学ですね。	
\\	リーさんの 専[せん]門[もん]は 文[ぶん]学[がく]ですね。
\\	此れは肉じゃありませんね。	
\\	此[こ]れは 肉[にく]じゃありませんね。
\\	豚かつは魚じゃありませんよ。	
\\	豚[とん]かつは 魚[さかな]じゃありませんよ。
\\	表現	
\\	表[ひょう]現[げん]	ヒョーケ°ꜜン 
\\	ヒョーケ°ン
\\	写経	
\\	写[しゃ]経[きょう]	シャキョー
\\	二百	
\\	二[に]百[ひゃく]
\\	三百	
\\	三[さん]百[びゃく]
\\	四百	
\\	四[よん]百[ひゃく]	
\\	五百	
\\	五[ご]百[ひゃく]
\\	六百	
\\	六[ろっ]百[ぴゃく]	
\\	七百	
\\	七[なな]百[ひゃく]
\\	八百	
\\	八[はっ]百[ぴゃく]	
\\	九百	
\\	九[きゅう]百[ひゃく]
\\	千	
\\	千[せん] 
\\	線)	セꜜン
\\	二千	
\\	二[に]千[せん]
\\	三千	
\\	三[さん]千[ぜん]
\\	四千	
\\	四[よん]千[せん]	
\\	五千	
\\	五[ご]千[せん]
\\	六千	
\\	六[ろく]千[せん]	
\\	七千	
\\	七[なな]千[せん]
\\	八千	
\\	八[はっ]千[せん]	
\\	九千	
\\	九[きゅう]千[せん]
\\	一万	
\\	一[いち]万[まん]
\\	二万	
\\	二[に]万[まん]
\\	三万	
\\	三[さん]万[まん]
\\	四万	
\\	四[よん]万[まん]	
\\	五万	
\\	五[ご]万[まん]
\\	六万	
\\	六[ろく]万[まん]	
\\	七万	
\\	七[なな]万[まん]
\\	八万	
\\	八[はち]万[まん]	
\\	九万	
\\	九[きゅう]万[まん]
\\	三十四	
\\	三[さん]十[じゅう]四[よん]
\\	六十七	
\\	六[ろく]十[じゅう]七[しち] 
\\	六[ろく]十[じゅう]七[なな]	
\\	九十九	
\\	九[きゅう]十[じゅう]九[きゅう]
\\	百二十五	
\\	百[ひゃく]二[に]十[じゅう]五[ご]	
\\	五百十五	
\\	五[ご]百[ひゃく]十[じゅう]五[ご]
\\	六百三	
\\	六[ろっ]百[ぴゃく]三[さん]	
\\	八百五十	
\\	八[はっ]百[ぴゃく]五[ご]十[じゅう]	
\\	千三百	
\\	千[せん]三[さん]百[びゃく]	
\\	三千四百	
\\	三[さん]千[ぜん]四[よん]百[ひゃく]
\\	八千九百	
\\	八[はっ]千[せん]九[きゅう]百[ひゃく]	
\\	三万五千	
\\	三[さん]万[まん]五[ご]千[せん]
\\	六万四千五百	
\\	六[ろく]万[まん]四[よん]千[せん]五[ご]百[ひゃく]	
\\	此れは誰の傘ですか。	
\\	此[こ]れは 誰[だれ]の 傘[かさ]ですか。
\\	分かりました。	
\\	分[わ]かりました。
\\	分かりません。	
\\	分[わ]かりません。
\\	ゆっくり言って下さい。	
\\	ゆっくり 言[い]って 下[くだ]さい。
\\	もう一度言って下さい。	
\\	もう 一[いち]度[ど]言[い]って 下[くだ]さい。
\\	一寸待って下さい。	
\\	一[ちょっ]寸[と]待[ま]って 下[くだ]さい。
\\	京都	
\\	京[きょう]都[と]	キョꜜート
\\	週末は大抵何をしますか。	
\\	週[しゅう]末[まつ]は 大[たい]抵[てい]何[なに]をしますか。
\\	大抵家で勉強します。	
\\	大[たい]抵[てい]家[うち]で 勉[べん]強[きょう]します。
\\	でも、時々映画を見ます。	
\\	でも、 時[とき]々[どき]映[えい]画[が]を 見[み]ます。
\\	土曜日に映画を見ませんか。	
\\	土[ど]曜[よう]日[び]に 映[えい]画[が]を 見[み]ませんか。
\\	土曜日は一寸...	
\\	土[ど]曜[よう]日[び]は 一[ちょっ]寸[と]...
\\	じゃあ、日曜日はどうですか。	
\\	じゃあ、 日[にち]曜[よう]日[び]はどうですか。
\\	良いですね。	
\\	良[い]いですね。
\\	早いですね。	
\\	早[はや]いですね。
\\	今日は京都に行きます。	
\\	今日[きょう]は 京[きょう]都[と]に 行[い]きます。
\\	京都で映画を見ます。	
\\	京[きょう]都[と]で 映[えい]画[が]を 見[み]ます。
\\	何時頃帰りますか。	
\\	何[なん]時[じ]頃[ごろ]帰[かえ]りますか。
\\	九時頃です。	
\\	九[く]時[じ]頃[ごろ]です。
\\	晩御飯は?	
\\	晩[ばん]御[ご]飯[はん]は?
\\	食べません。	
\\	食[た]べません。
\\	私は良くテレビを見ます。	
\\	私[わたし]は 良[よ]くテレビを 見[み]ます。
\\	メアリーさんは時々朝御飯を食べません。	
\\	メアリーさんは 時[とき]々[どき]朝[あさ]御[ご]飯[はん]を 食[た]べません。
\\	私は明日京都に行きます。	
\\	私[わたし]は 明日[あした]京[きょう]都[と]に 行[い]きます。
\\	スーさんは今日家に帰りません。	
\\	スーさんは 今日[きょう]家[うち]に 帰[かえ]りません。
\\	図書館で本を読みます。	
\\	図[と]書[しょ]館[かん]で 本[ほん]を 読[よ]みます。
\\	家でテレビを見ます。	
\\	家[うち]でテレビを 見[み]ます。
\\	私は今日学校に行きません。 
\\	私は今日学校へ行きません。	
\\	私[わたし]は 今[きょ]日[う]学[がっ]校[こう]に 行[い]きません。 
\\	私[わたし]は 今[きょ]日[う]学[がっ]校[こう]へ 行[い]きません。
\\	私は家に帰ります。 
\\	私は家へ帰ります。	
\\	私[わたし]は 家[うち]に 帰[かえ]ります。 
\\	私[わたし]は 家[うち]へ 帰[かえ]ります。
\\	日曜日に京都に行きます。	
\\	日[にち]曜[よう]日[び]に 京[きょう]都[と]に 行[い]きます。
\\	十二時に寝ます。	
\\	十[じゅう]二[に]時[じ]に 寝[ね]ます。
\\	コーヒーを飲みます。	
\\	コーヒーを 飲[の]みます。
\\	テープを聞きます。	
\\	テープを 聞[き]きます。
\\	テレビを見ます。	
\\	テレビを 見[み]ます。
\\	日曜日に行きます。	
\\	日[にち]曜[よう]日[び]に 行[い]きます。
\\	十時四十二分に起きます。	
\\	十[じゅう]時[じ]四[よん]十[じゅう]二[に]分[ふん]に 起[お]きます。
\\	九月に帰ります。	
\\	九[く]月[がつ]に 帰[かえ]ります。
\\	明日来ます。	
\\	明日[あした]来[き]ます。
\\	毎晩テレビを見ます。	
\\	毎[まい]晩[ばん]テレビを 見[み]ます。
\\	何時行きますか。	
\\	何[い]時[つ]行[い]きますか。
\\	朝(に)新聞を読みます。	
\\	朝[あさ](に) 新[しん]聞[ぶん]を 読[よ]みます。
\\	週末(に)何をしますか。	
\\	週[しゅう]末[まつ](に) 何[なに]をしますか。
\\	昼御飯を食べませんか。	
\\	昼[ひる] 御[ご]飯[はん]を 食[た]べませんか。
\\	私は今日図書館で日本語を勉強します。	
\\	私[わたし]は 今[きょ]日[う]図[と]書[しょ]館[かん]で 日[に]本[ほん]語[ご]を 勉[べん]強[きょう]します。
\\	私は良く七時頃家へ帰ります。	
\\	私[わたし]は 良[よ]く 七[しち]時[じ]頃[ごろ]家[うち]へ 帰[かえ]ります。
\\	私は時々喫茶店に行きます。	
\\	私[わたし]は 時[とき]々[どき]喫[きっ]茶[さ]店[てん]に 行[い]きます。
\\	私は全然テレビを見ません。	
\\	私[わたし]は 全[ぜん]然[ぜん]テレビを 見[み]ません。
\\	たけしさんは余り勉強しません。	
\\	たけしさんは余[あま]り 勉[べん]強[きょう]しません。
\\	でテープを聞きます。	
\\	[エル]
\\	[エル]でテープを 聞[き]きます。
\\	図書館で雑誌を読みます。	
\\	図[と]書[しょ]館[かん]で 雑[ざっ]誌[し]を 読[よ]みます。
\\	学校でテニスをします。	
\\	学[がっ]校[こう]でテニスをします。
\\	マクドナルドでハンバーガーを食べます。	
\\	マクドナルドでハンバーガーを 食[た]べます。
\\	喫茶店でコーヒーを飲みます。	
\\	喫[きっ]茶[さ]店[てん]でコーヒーを 飲[の]みます。
\\	家でテレビを見ます。	
\\	家[うち]でテレビを 見[み]ます。
\\	大学で日本語を話します。	
\\	大[だい]学[がく]で 日[に]本[ほん]語[ご]を 話[はな]します。
\\	郵便局に行きます。	
\\	郵[ゆう]便[びん] 局[きょく]に 行[い]きます。
\\	ポーランド	
\\	ポーランド
\\	ポーランドに帰ります。	
\\	ポーランドに 帰[かえ]ります。
\\	何時に起きますか。	
\\	何[なん]時[じ]に 起[お]きますか。
\\	何時に学校に行きますか。 
\\	何時に学校へ行きますか。	
\\	何[なん]時[じ]に 学[がっ]校[こう]に 行[い]きますか。 
\\	何[なん]時[じ]に 学[がっ]校[こう]へ 行[い]きますか。
\\	何時に昼御飯を食べますか。	
\\	何[なん]時[じ]に 昼[ひる] 御[ご]飯[はん]を 食[た]べますか。
\\	何時にコーヒーを飲みますか。	
\\	何[なん]時[じ]にコーヒーを 飲[の]みますか。
\\	何時に家に帰りますか。 
\\	何時に家へ帰りますか。	
\\	何[なん]時[じ]に 家[うち]に 帰[かえ]りますか。 
\\	何[なん]時[じ]に 家[うち]へ 帰[かえ]りますか。
\\	何時に勉強しますか。	
\\	何[なん]時[じ]に 勉[べん]強[きょう]しますか。
\\	何時に寝ますか。	
\\	何[なん]時[じ]に 寝[ね]ますか。
\\	二時に図書館で雑誌を読みます。	
\\	二[に]時[じ]に 図[と]書[しょ]館[かん]で 雑[ざっ]誌[し]を 読[よ]みます。
\\	四時半に
\\	でテープを聞きます。	
\\	四[よ]時[じ]半[はん]に 
\\	[エル]
\\	[エル]でテープを 聞[き]きます。
\\	土曜日に学校でテニスをします。	
\\	土[ど]曜[よう]日[び]に 学[がっ]校[こう]でテニスをします。
\\	五時にマクドナルドでハンバーガーを食べます。	
\\	五[ご]時[じ]にマクドナルドでハンバーガーを 食[た]べます。
\\	三時に喫茶店でコーヒーを飲みます。	
\\	三[さん]時[じ]に 喫[きっ]茶[さ]店[てん]でコーヒーを 飲[の]みます。
\\	今晩(に)家でテレビを見ます。	
\\	今[こん]晩[ばん](に) 家[うち]でテレビを 見[み]ます。
\\	毎日大学で日本語を話します。	
\\	毎[まい]日[にち]大[だい]学[がく]で 日[に]本[ほん]語[ご]を 話[はな]します。
\\	一時に郵便局に行きます。	
\\	一[いち]時[じ]に 郵[ゆう]便[びん] 局[きょく]に 行[い]きます。
\\	明日ポーランドに帰ります。	
\\	明日[あした]ポーランドに 帰[かえ]ります。
\\	映画を見ませんか。	
\\	映[えい]画[が]を 見[み]ませんか。
\\	僕の家に行きませんか。	
\\	僕[ぼく]の 家[うち]に 行[い]きませんか。
\\	晩御飯を食べませんか。	
\\	晩[ばん]御[ご]飯[はん]を 食[た]べませんか。
\\	図書館で勉強しませんか。	
\\	図[と]書[しょ]館[かん]で 勉[べん]強[きょう]しませんか。
\\	喫茶店で話しませんか。	
\\	喫[きっ]茶[さ]店[てん]で 話[はな]しませんか。
\\	家でお茶を飲みませんか。	
\\	家[うち]でお 茶[ちゃ]を 飲[の]みませんか。
\\	音楽を聞きませんか。	
\\	音[おん]楽[がく]を 聞[き]きませんか。
\\	済みません。マクドナルドは何処ですか。	
\\	済[す]みません。マクドナルドは 何[ど]処[こ]ですか。
\\	彼のデパートの前ですよ。	
\\	彼[あ]のデパートの 前[まえ]ですよ。
\\	お帰りなさい。映画はどうでしたか。	
\\	お 帰[かえ]りなさい。 映[えい]画[が]はどうでしたか。
\\	見ませんでした。たけしさんは来ませんでした。	
\\	見[み]ませんでした。たけしさんは 来[き]ませんでした。
\\	如何してですか。	
\\	如[ど]何[う]してですか。
\\	分かりません。だから、一人で本屋とお寺に行きました。	
\\	分[わ]かりません。だから、 一[ひと]人[り]で 本[ほん]屋[や]とお 寺[てら]に 行[い]きました。
\\	人が沢山居ましたか。	
\\	人[ひと]が 沢[たく]山[さん]居[い]ましたか。
\\	お寺で写真を沢山撮りました。	
\\	お 寺[てら]で 写[しゃ]真[しん]を 沢[たく]山[さん]撮[と]りました。
\\	デパートにも行きました。はい、お土産です。	
\\	デパートにも 行[い]きました。はい、お 土産[みやげ]です。
\\	さっき電話が有りましたよ。	
\\	さっき 電[でん]話[わ]が 有[あ]りましたよ。
\\	木村	
\\	木[き]村[むら]	
\\	たけしさん、今日来ませんでしたね。	
\\	たけしさん、 今日[きょう]来[き]ませんでしたね。
\\	行きましたよ。マクドナルドの前で一時間待ちました。	
\\	行[い]きましたよ。マクドナルドの 前[まえ]で 一[いち]時[じ]間[かん]待[ま]ちました。
\\	彼処にマクドナルドが有ります。	
\\	彼[あそ]処[こ]にマクドナルドが 有[あ]ります。
\\	テレビが有りません。	
\\	テレビが 有[あ]りません。
\\	有る 
\\	有ります 
\\	に 
\\	で, 
\\	が 
\\	は,	
\\	有る 
\\	有ります 
\\	時間が有りますか。	
\\	時[じ]間[かん]が 有[あ]りますか。
\\	火曜日にテストが有りますか。	
\\	火[か]曜[よう]日[び]にテストが 有[あ]りますか。
\\	明日は日本語のクラスが有りません。	
\\	明日[あした]は 日[に]本[ほん]語[ご]のクラスが 有[あ]りません。
\\	彼処に留学生が居ます。	
\\	彼[あそ]処[こ]に 留[りゅう]学[がく]生[せい]が 居[い]ます。
\\	銀行は図書館の隣です。	
\\	銀[ぎん]行[こう]は 図[と]書[しょ]館[かん]の 隣[となり]です。
\\	傘は机の下です。	
\\	傘[かさ]は 机[つくえ]の 下[した]です。
\\	レストランはデパートと病院の間です。	
\\	レストランはデパートと 病[びょう]院[いん]の 間[あいだ]です。
\\	山下	
\\	山[やま]下[した]	
\\	東西大学	
\\	東[とう]西[ざい]大[だい]学[がく]	
\\	山下先生は東西大学の学生でした。	
\\	山[やま]下[した]先[せん]生[せい]は 東[とう]西[ざい]大[だい]学[がく]の 学[がく]生[せい]でした。
\\	彼れは日本の映画じゃありませんでした。	
\\	彼[あ]れは 日[に]本[ほん]の 映[えい]画[が]じゃありませんでした。
\\	私は京都で写真を沢山撮りました。 
\\	私は京都で沢山写真を撮りました。	
\\	を)	私[わたし]は 京[きょう]都[と]で 写[しゃ]真[しん]を 沢[たく]山[さん]撮[と]りました。 
\\	私[わたし]は 京[きょう]都[と]で 沢[たく]山[さん]写[しゃ]真[しん]を 撮[と]りました。
\\	野菜を沢山食べました。 
\\	沢山野菜を食べました。	
\\	を)	野[や]菜[さい]を 沢[たく]山[さん]食[た]べました。 
\\	沢[たく]山[さん]野[や]菜[さい]を 食[た]べました。
\\	私は昨日日本語を三時間位勉強しました。	
\\	私[わたし]は 昨[きの]日[う]日[に]本[ほん]語[ご]を 三[さん]時[じ]間[かん]位[ぐらい]勉[べん]強[きょう]しました。
\\	大阪	
\\	大[おお]阪[さか]	オーサカ
\\	京都と大阪に行きました。	
\\	京[きょう]都[と]と 大[おお]阪[さか]に 行[い]きました。
\\	日本語と英語を話します。	
\\	日[に]本[ほん]語[ご]と 英[えい]語[ご]を 話[はな]します。
\\	メアリーさんはスーさんと韓国に行きます。	
\\	メアリーさんはスーさんと 韓[かん]国[こく]に 行[い]きます。
\\	山下先生も昨日京都に行きました。	
\\	山[やま]下[した]先[せん]生[せい]も 昨[きの]日[う]京[きょう]都[と]に 行[い]きました。
\\	私は昨日京都に行きました。	
\\	私[わたし]は 昨[きの]日[う]京[きょう]都[と]に 行[い]きました。
\\	メアリーさんは靴を買いました。	
\\	メアリーさんは 靴[くつ]を 買[か]いました。
\\	メアリーさんは鞄も買いました。	
\\	メアリーさんは 鞄[かばん]も 買[か]いました。
\\	私は先週京都に行きました。 大阪にも行きました。	
\\	私[わたし]は 先[せん]週[しゅう]京[きょう]都[と]に 行[い]きました。 大[おお]阪[さか]にも 行[い]きました。
\\	二時間半	
\\	二[に]時[じ]間[かん]半[はん]
\\	貴方の町に日本のレストランが有りますか。	
\\	貴方[あなた]の 町[まち]に 日[に]本[ほん]のレストランが 有[あ]りますか。
\\	貴方の家に猫が居ますか。	
\\	貴方[あなた]の 家[うち]に 猫[ねこ]が 居[い]ますか。
\\	貴方の学校に日本人の学生が居ますか。	
\\	貴方[あなた]の 学[がっ]校[こう]に 日[に]本[ほん]人[じん]の 学[がく]生[せい]が 居[い]ますか。
\\	デパートに何が有りますか。	
\\	デパートに 何[なん]が 有[あ]りますか。
\\	教室	
\\	教[きょう]室[しつ]	キョーシツ
\\	此の教室に誰が居ますか。	
\\	此[こ]の 教[きょう]室[しつ]に 誰[だれ]が 居[い]ますか。
\\	動物園	
\\	動[どう]物[ぶつ]園[えん]	ドーブツꜜエン
\\	動物園に何が居ますか。	
\\	動[どう]物[ぶつ]園[えん]に 何[なん]が 居[い]ますか。
\\	動物	
\\	動[どう]物[ぶつ]	ドーブツ
\\	動物園に動物が居ます。	
\\	動[どう]物[ぶつ]園[えん]に 動[どう]物[ぶつ]が 居[い]ます。
\\	月曜日に英語のクラスが有りますか。	
\\	月[げつ]曜[よう]日[び]に 英[えい]語[ご]のクラスが 有[あ]りますか。
\\	仏語 
\\	フランス語	
\\	仏[ふつ]語[ご] 
\\	フランス 語[ご]
\\	図書館はスーパーの隣です。	
\\	図[と]書[しょ]館[かん]はスーパーの 隣[となり]です。
\\	図書館は大学の後ろです。	
\\	図[と]書[しょ]館[かん]は 大[だい]学[がく]の 後[うし]ろです。
\\	郵便局は病院の前です。	
\\	郵[ゆう]便[びん]局[きょく]は 病[びょう]院[いん]の 前[まえ]です。
\\	喫茶店はホテルの中です。	
\\	喫[きっ]茶[さ]店[てん]はホテルの 中[なか]です。
\\	バス停は大学の前です。	
\\	バス 停[てい]は 大[だい]学[がく]の 前[まえ]です。
\\	病院は大学とホテルの間です。	
\\	病[びょう]院[いん]は 大[だい]学[がく]とホテルの 間[あいだ]です。
\\	本は机の上です。	
\\	本[ほん]は 机[つくえ]の 上[うえ]です。
\\	辞書は鞄の中です。	
\\	辞[じ]書[しょ]は 鞄[かばん]の 中[なか]です。
\\	電話はテレビの右です。	
\\	電[でん]話[わ]はテレビの 右[みぎ]です。
\\	帽子は傘の左です。	
\\	帽[ぼう]子[し]は 傘[かさ]の 左[ひだり]です。
\\	山下先生は二十歳でした。	
\\	山[やま]下[した]先[せん]生[せい]は 二十歳[はたち]でした。
\\	山下先生は大学生でしたか。 はい、山下先生は大学生でした。	
\\	山[やま]下[した]先[せん]生[せい]は 大[だい]学[がく]生[せい]でしたか。 はい、 山[やま]下[した]先[せん]生[せい]は 大[だい]学[がく]生[せい]でした。
\\	山下先生は一年生でしたか。 いいえ、山下先生は一年生じゃありませんでした。	
\\	山[やま]下[した]先[せん]生[せい]は 一[いち]年[ねん]生[せい]でしたか。 いいえ、 山[やま]下[した]先[せん]生[せい]は 一[いち]年[ねん]生[せい]じゃありませんでした。
\\	山下先生は良い学生でしたか。 いいえ、山下先生は良い学生じゃありませんでした。	
\\	山[やま]下[した]先[せん]生[せい]は 良[い]い 学[がく]生[せい]でしたか。 いいえ、 山[やま]下[した]先[せん]生[せい]は 良[い]い 学[がく]生[せい]じゃありませんでした。
\\	山下先生の専門は英語でしたか。 いいえ、山下先生の専門は英語じゃありませんでした。	
\\	山[やま]下[した]先[せん]生[せい]の 専[せん]門[もん]は 英[えい]語[ご]でしたか。 いいえ、 山[やま]下[した]先[せん]生[せい]の 専[せん]門[もん]は 英[えい]語[ご]じゃありませんでした。
\\	山下先生の専門は日本の歴史でしたか。 はい、山下先生の専門は日本の歴史でした。	
\\	山[やま]下[した]先[せん]生[せい]の 専[せん]門[もん]は 日[に]本[ほん]の 歴[れき]史[し]でしたか。 はい、 山[やま]下[した]先[せん]生[せい]の 専[せん]門[もん]は 日[に]本[ほん]の 歴[れき]史[し]でした。
\\	其の時計は二万円でしたか。 いいえ、其の時計は二万円じゃありませんでした。 一万円でした。	
\\	其[そ]の 時[と]計[けい]は 二[に]万[まん]円[えん]でしたか。 いいえ、 其[そ]の 時[と]計[けい]は 二[に]万[まん]円[えん]じゃありませんでした。 一[いち]万[まん]円[えん]でした。
\\	メアリーさんは火曜日に家で手紙を書きました。	
\\	メアリーさんは 火[か]曜[よう]日[び]に 家[うち]で 手[て]紙[がみ]を 書[か]きました。
\\	メアリーさんは月曜日に図書館で勉強しました。	
\\	メアリーさんは 月[げつ]曜[よう]日[び]に 図[と]書[しょ]館[かん]で 勉[べん]強[きょう]しました。
\\	メアリーさんは木曜日に喫茶店で日本人の友達に会いました。	
\\	メアリーさんは 木[もく]曜[よう]日[び]に 喫[きっ]茶[さ]店[てん]で 日[に]本[ほん]人[じん]の 友[とも]達[だち]に 会[あ]いました。
\\	メアリーさんは金曜日に友達の家で晩御飯を食べました。	
\\	メアリーさんは 金[きん]曜[よう]日[び]に 友[とも]達[だち]の 家[うち]で 晩[ばん]御[ご]飯[はん]を 食[た]べました。
\\	メアリーさんは日曜日にデパートで買い物をしました。	
\\	メアリーさんは 日[にち]曜[よう]日[び]にデパートで 買[か]い 物[もの]をしました。
\\	メアリーさんは何時映画を見ましたか。	
\\	メアリーさんは 何[い]時[つ]映[えい]画[が]を 見[み]ましたか。
\\	メアリーさんは何時買い物をしましたか。	
\\	メアリーさんは 何[い]時[つ]買[か]い 物[もの]をしましたか。
\\	メアリーさんは金曜日に何処で晩御飯を食べましたか。	
\\	メアリーさんは 金[きん]曜[よう]日[び]に 何[ど]処[こ]で 晩[ばん]御[ご]飯[はん]を 食[た]べましたか。
\\	メアリーさんは木曜日に何処で友達に会いましたか。	
\\	メアリーさんは 木[もく]曜[よう]日[び]に 何[ど]処[こ]で 友[とも]達[だち]に 会[あ]いましたか。
\\	子供の時良く本を読みましたか。 いいえ、全然読みませんでした。	
\\	子[こ]供[ども]の 時[とき]良[よ]く 本[ほん]を 読[よ]みましたか。 いいえ、 全[ぜん]然[ぜん]読[よ]みませんでした。
\\	高校の時良く本を読みましたか。 
\\	いいえ、余り読みませんでした。	
\\	高[こう]校[こう]の 時[とき]良[よ]く 本[ほん]を 読[よ]みましたか。 
\\	いいえ、 余[あま]り 読[よ]みませんでした。
\\	子供の時良く公園に行きましたか。 はい、時々行きました。	
\\	子[こ]供[ども]の 時[とき]良[よ]く 公[こう]園[えん]に 行[い]きましたか。 はい、 時[とき]々[どき]行[い]きました。
\\	山本	
\\	山[やま]本[もと]	
\\	西語 
\\	スペイン語	
\\	西[せい]語[ご] 
\\	スペイン 語[ご]	
\\	田中	
\\	田[た]中[なか]	
\\	何時間?	
\\	何[なん]時[じ]間[かん]?
\\	大抵何時間位寝ますか。 大抵八時間位寝ます。	
\\	大[たい]抵[てい]何[なん]時[じ]間[かん]位[ぐらい]寝[ね]ますか。 大[たい]抵[てい]八[はち]時[じ]間[かん]位[ぐらい]寝[ね]ます。
\\	毎日大抵何時間勉強しますか。 毎日五時間勉強します。	
\\	毎[まい]日[にち]大[たい]抵[てい]何[なん]時[じ]間[かん]勉[べん]強[きょう]しますか。 毎[まい]日[にち]五[ご]時[じ]間[かん]勉[べん]強[きょう]します。
\\	誰と昼御飯を食べますか。	
\\	誰[だれ]と 昼[ひる] 御[ご]飯[はん]を 食[た]べますか。
\\	先週写真を沢山撮りましたか。	
\\	先[せん]週[しゅう]写[しゃ]真[しん]を 沢[たく]山[さん]撮[と]りましたか。
\\	横浜	
\\	横[よこ]浜[はま]	ヨコハマ
\\	日・週・月・年	
\\	日[ひ]・ 週[しゅう]・ 月[つき]・ 年[とし]	
\\	十二日	
\\	十[じゅう]二[に]日[にち]	
\\	十三日	
\\	十[じゅう]三[さん]日[にち]	
\\	十五日	
\\	十[じゅう]五[ご]日[にち]	ジュꜜーゴニチ
\\	十六日	
\\	十[じゅう]六[ろく]日[にち]	
\\	十七日	
\\	十[じゅう]七[しち]日[にち]	
\\	十八日	
\\	十[じゅう]八[はち]日[にち]	
\\	十九日	
\\	十[じゅう]九[く]日[にち]	
\\	一昨日 ・ 昨日 ・ 今日 ・ 明日 ・ 明後日	
\\	一昨日[おととい]・ 昨[きの]日[う]・ 今日[きょう]・ 明日[あした]・ 明後日[あさって]
\\	先々週 ・ 先週 ・ 今週 ・ 来週 ・ 再来週	
\\	先[せん]々[せん] 週[しゅう]・ 先[せん]週[しゅう]・ 今[こん]週[しゅう]・ 来[らい]週[しゅう]・ 再[さ]来[らい]週[しゅう]	
\\	二か月前 ・ 先月 ・ 今月 ・ 来月 ・ 再来月	
\\	二[に]か 月[げつ] 前[まえ]・ 先[せん]月[げつ]・ 今[こん]月[げつ]・ 来[らい]月[げつ]・ 再[さ]来[らい]月[げつ]
\\	一昨年 ・ 去年 ・ 今年 ・ 来年 ・ 再来年	
\\	一昨年[おととし]・ 去[きょ]年[ねん]・ 今年[ことし]・ 来[らい]年[ねん]・ 再[さ]来[らい]年[ねん]
\\	良い天気ですね。	
\\	良[い]い 天[てん]気[き]ですね。
\\	そうですね。でも、一寸暑いですね。	
\\	そうですね。でも、 一[ちょっ]寸[と]暑[あつ]いですね。
\\	わあ	
\\	わあ	
\\	わあ、綺麗な海!	
\\	わあ、 綺[き]麗[れい]な 海[うみ]!
\\	泳ぎましょう。	
\\	泳[およ]ぎましょう。
\\	ロバートさんはどんなスポーツが好きですか。	
\\	ロバートさんはどんなスポーツが 好[す]きですか。
\\	サーフィンが好きです。	
\\	サーフィンが 好[す]きです。
\\	明日一緒にやりましょうか。	
\\	明日[あした]一[いっ]緒[しょ]にやりましょうか。
\\	でも、難しくありませんか。	
\\	でも、 難[むずか]しくありませんか。
\\	大丈夫ですよ。	
\\	大[だい]丈[じょう]夫[ぶ]ですよ。
\\	済みません。葉書は、イギリス迄幾らですか。	
\\	済[す]みません。 葉[は]書[がき]は、イギリス 迄[まで]幾[いく]らですか。
\\	じゃあ、七十円切手を二枚お願いします。	
\\	じゃあ、 七[なな]十[じゅう]円[えん]切[きっ]手[て]を 二[に]枚[まい]お 願[ねが]いします。
\\	それから、五十円切手を一枚下さい。	
\\	それから、 五[ご]十[じゅう]円[えん]切[きっ]手[て]を 一[いち]枚[まい]下[くだ]さい。
\\	葉書、有難う。旅行は楽しかったですか。	
\\	葉[は]書[がき]、 有[あり]難[がと]う。 旅[りょ]行[こう]は 楽[たの]しかったですか。
\\	沖縄	
\\	沖[おき]縄[なわ]	オキナワ
\\	沖縄の海は迚も綺麗でした。	
\\	沖[おき]縄[なわ]の 海[うみ]は 迚[とて]も 綺[き]麗[れい]でした。
\\	僕も海が大好きです。	
\\	僕[ぼく]も 海[うみ]が 大[だい]好[す]きです。
\\	飛行機の切符は高かったですか。	
\\	飛[ひ]行[こう]機[き]の 切[きっ]符[ぷ]は 高[たか]かったですか。
\\	いいえ、余り高くありませんでした。	
\\	いいえ、 余[あま]り 高[たか]くありませんでした。
\\	たけしさんのデートはどうでしたか。	
\\	たけしさんのデートはどうでしたか。
\\	郵便局員	
\\	郵[ゆう]便[びん] 局[きょく]員[いん]	
\\	面白い映画	
\\	面[おも]白[しろ]い 映[えい]画[が]
\\	昨日、面白い映画を見ました。	
\\	昨[きの]日[う]、 面[おも]白[しろ]い 映[えい]画[が]を 見[み]ました。
\\	怖い先生	
\\	怖[こわ]い 先[せん]生[せい]
\\	山下先生は怖い先生です。	
\\	山[やま]下[した]先[せん]生[せい]は 怖[こわ]い 先[せん]生[せい]です。
\\	綺麗な写真	
\\	綺[き]麗[れい]な 写[しゃ]真[しん]
\\	京都で綺麗な写真を撮りました。	
\\	京[きょう]都[と]で 綺[き]麗[れい]な 写[しゃ]真[しん]を 撮[と]りました。
\\	元気な先生	
\\	元[げん]気[き]な 先[せん]生[せい]
\\	山下先生は元気な先生です。	
\\	山[やま]下[した]先[せん]生[せい]は 元[げん]気[き]な 先[せん]生[せい]です。
\\	面白いです。 面白くありません。	
\\	面[おも]白[しろ]いです。 面[おも]白[しろ]くありません。
\\	面白かったです。 面白くありませんでした。	
\\	面[おも]白[しろ]かったです。 面[おも]白[しろ]くありませんでした。
\\	良いです。 良くありません。	
\\	良[い]いです。 良[よ]くありません。
\\	良かったです。 良くありませんでした。	
\\	良[よ]かったです。 良[よ]くありませんでした。
\\	元気です。 元気じゃありません。	
\\	元[げん]気[き]です。 元[げん]気[き]じゃありません。
\\	元気でした。 元気じゃありませんでした。	
\\	元[げん]気[き]でした。 元[げん]気[き]じゃありませんでした。
\\	山口	
\\	山[やま]口[ぐち]	ヤマꜜク°チ
\\	山口県	
\\	山[やま]口[ぐち]県[けん]	
\\	ロバートさんは日本語のクラスが好きです。	
\\	ロバートさんは 日[に]本[ほん]語[ご]のクラスが 好[す]きです。
\\	山下先生は魚が嫌いです。	
\\	山[やま]下[した]先[せん]生[せい]は 魚[さかな]が 嫌[きら]いです。
\\	好きでも嫌いでもありません。	
\\	好[す]きでも 嫌[きら]いでもありません。
\\	此れは私の好きな本です。	
\\	此[こ]れは私[わたし]の 好[す]きな 本[ほん]です。
\\	此の部屋は一寸暑いです。	
\\	此[こ]の 部[へ]屋[や]は 一[ちょっ]寸[と]暑[あつ]いです。
\\	コーヒーが大好きです。	
\\	コーヒーが 大[だい]好[す]きです。
\\	納豆	
\\	納[なっ]豆[とう]	ナットꜜー
\\	納豆が大嫌いです。	
\\	納[なっ]豆[とう]が 大[だい]嫌[きら]いです。
\\	一緒に図書館で勉強しましょう。	
\\	一[いっ]緒[しょ]に 図[と]書[しょ]館[かん]で 勉[べん]強[きょう]しましょう。
\\	喫茶店でコーヒーを飲みましょうか。	
\\	喫[きっ]茶[さ]店[てん]でコーヒーを 飲[の]みましょうか。
\\	リーさんは切手を三枚買いました。	
\\	リーさんは 切[きっ]手[て]を 三[さん]枚[まい]買[か]いました。
\\	たけしさんは忙しいです。	
\\	たけしさんは 忙[いそが]しいです。
\\	東京は賑やかです。	
\\	東[とう]京[きょう]は 賑[にぎ]やかです。
\\	安いです	
\\	安[やす]いです
\\	暑いです	
\\	暑[あつ]いです
\\	寒いです	
\\	寒[さむ]いです
\\	面白いです	
\\	面[おも]白[しろ]いです
\\	詰らないです	
\\	詰[つま]らないです
\\	忙しいです	
\\	忙[いそが]しいです
\\	良いです	
\\	良[い]いです
\\	元気です	
\\	元[げん]気[き]です
\\	静かです	
\\	静[しず]かです
\\	賑やかです	
\\	賑[にぎ]やかです
\\	綺麗です	
\\	綺[き]麗[れい]です
\\	暇です	
\\	暇[ひま]です
\\	寒くありません	
\\	寒[さむ]くありません
\\	古くありません	
\\	古[ふる]くありません
\\	怖くありません	
\\	怖[こわ]くありません
\\	新しくありません	
\\	新[あたら]しくありません
\\	難しくありません	
\\	難[むずか]しくありません
\\	小さくありません	
\\	小[ちい]さくありません
\\	良くありません	
\\	良[よ]くありません
\\	元気じゃありません	
\\	元[げん]気[き]じゃありません
\\	静かじゃありません	
\\	静[しず]かじゃありません
\\	綺麗じゃありません	
\\	綺[き]麗[れい]じゃありません
\\	ハンサムじゃありません	
\\	ハンサムじゃありません
\\	今日は暇ですか。 はい、暇です。 
\\	いいえ、暇じゃありません。	
\\	今日[きょう]は 暇[ひま]ですか。 はい、 暇[ひま]です。 
\\	いいえ、 暇[ひま]じゃありません。
\\	学校は大きいですか。 はい、 大きいです。 
\\	いいえ、 大きくありません。	
\\	学[がっ]校[こう]は 大[おお]きいですか。 はい、 大[おお]きいです。 
\\	いいえ、 大[おお]きくありません。
\\	日本の食べ物は美味しいですか。 はい、 美味しいです。 
\\	いいえ、 美味しくありません。	
\\	日[に]本[ほん]の 食[た]べ 物[もの]は 美[お]味[い]しいですか。 はい、 美[お]味[い]しいです。 
\\	いいえ、 美[お]味[い]しくありません。
\\	先生は易しいですか。 はい、 易しいです。 
\\	いいえ、 易しくありません。	
\\	先[せん]生[せい]は 易[やさ]しいですか。 はい、 易[やさ]しいです。 
\\	いいえ、 易[やさ]しくありません。
\\	高かったです	
\\	高[たか]かったです
\\	暑かったです	
\\	暑[あつ]かったです
\\	早かったです	
\\	早[はや]かったです
\\	寒かったです	
\\	寒[さむ]かったです
\\	面白かったです	
\\	面[おも]白[しろ]かったです
\\	詰らなかったです	
\\	詰[つま]らなかったです
\\	忙しかったです	
\\	忙[いそが]しかったです
\\	良かったです	
\\	良[よ]かったです
\\	静かでした	
\\	静[しず]かでした
\\	賑やかでした	
\\	賑[にぎ]やかでした
\\	綺麗でした	
\\	綺[き]麗[れい]でした
\\	暇でした	
\\	暇[ひま]でした
\\	安くありませんでした	
\\	安[やす]くありませんでした
\\	元気じゃありませんでした	
\\	元[げん]気[き]じゃありませんでした
\\	高くありませんでした	
\\	高[たか]くありませんでした
\\	楽しくありませんでした	
\\	楽[たの]しくありませんでした
\\	易しくありませんでした	
\\	易[やさ]しくありませんでした
\\	詰らなくありませんでした	
\\	詰[つま]らなくありませんでした
\\	大きくありませんでした	
\\	大[おお]きくありませんでした
\\	良くありませんでした	
\\	良[よ]くありませんでした
\\	忙しくありませんでした	
\\	忙[いそが]しくありませんでした
\\	賑やかじゃありませんでした	
\\	賑[にぎ]やかじゃありませんでした
\\	静かじゃありませんでした	
\\	静[しず]かじゃありませんでした
\\	綺麗じゃありませんでした	
\\	綺[き]麗[れい]じゃありませんでした
\\	暇じゃありませんでした	
\\	暇[ひま]じゃありませんでした
\\	沖縄は暑かったです。	
\\	沖[おき]縄[なわ]は 暑[あつ]かったです。
\\	食べ物は高くありませんでした。	
\\	食[た]べ 物[もの]は 高[たか]くありませんでした。
\\	飲み物は美味しかったです。	
\\	飲[の]み 物[もの]は 美[お]味[い]しかったです。
\\	ホテルは大きくありませんでした。	
\\	ホテルは 大[おお]きくありませんでした。
\\	ホテルは新しかったです。	
\\	ホテルは 新[あたら]しかったです。
\\	新しかったです	
\\	新[あたら]しかったです
\\	新しくありませんでした	
\\	新[あたら]しくありませんでした
\\	レストランは静かじゃありませんでした。	
\\	レストランは 静[しず]かじゃありませんでした。
\\	海は綺麗でした。	
\\	海[うみ]は 綺[き]麗[れい]でした。
\\	サーフィンは面白かったです。	
\\	サーフィンは 面[おも]白[しろ]かったです。
\\	休みに沖縄に行きました。	
\\	休[やす]みに 沖[おき]縄[なわ]に 行[い]きました。
\\	映画は迚も怖かったです。	
\\	映[えい]画[が]は 迚[とて]も 怖[こわ]かったです。
\\	パーティーは楽しくありませんでした。	
\\	パーティーは 楽[たの]しくありませんでした。
\\	たけしさんはどんな人ですか。	
\\	たけしさんはどんな 人[ひと]ですか。
\\	魚が好きですか。	
\\	魚[さかな]が 好[す]きですか。
\\	アイスクリーム	
\\	アイスクリーム	アイスクリꜜーム
\\	ハードロックが好きですか。	
\\	ハードロックが 好[す]きですか。
\\	空手が大好きです。	
\\	空[から]手[て]が 大[だい]好[す]きです。
\\	どんな食べ物が好きですか。	
\\	どんな 食[た]べ 物[もの]が 好[す]きですか。
\\	どんな音楽が好きですか。	
\\	どんな 音[おん]楽[がく]が 好[す]きですか。
\\	日本語を話しましょう。	
\\	日[に]本[ほん]語[ご]を 話[はな]しましょう。
\\	家に帰りましょう。	
\\	家[うち]に 帰[かえ]りましょう。
\\	先生に聞きましょう。	
\\	先[せん]生[せい]に 聞[き]きましょう。
\\	映画を見ましょう。	
\\	映[えい]画[が]を 見[み]ましょう。
\\	葉書を書きましょう。	
\\	葉[は]書[がき]を 書[か]きましょう。
\\	出掛けましょう。	
\\	出[で]掛[か]けましょう。
\\	待ちましょう。	
\\	待[ま]ちましょう。
\\	写真を撮りましょう。	
\\	写[しゃ]真[しん]を 撮[と]りましょう。
\\	バスに乗りましょう。	
\\	バスに 乗[の]りましょう。
\\	そうしましょう。	
\\	そうしましょう。
\\	お茶を飲みましょうか。	
\\	お 茶[ちゃ]を 飲[の]みましょうか。
\\	先生の誕生日	
\\	先[せん]生[せい]の 誕[たん]生[じょう]日[び]
\\	大きかったです	
\\	大[おお]きかったです
\\	此れ、お願いします。	
\\	此[こ]れ、お 願[ねが]いします。
\\	航空便でお願いします。	
\\	航[こう]空[くう] 便[びん]でお 願[ねが]いします。
\\	船便でお願いします。	
\\	船[ふな]便[びん]でお 願[ねが]いします。
\\	何日位掛かりますか。	
\\	何[なん]日[にち]位[ぐらい]掛[か]かりますか。
\\	何日	
\\	何[なん]日[にち]	ナꜜンニチ
\\	百円に成ります。	
\\	百[ひゃく]円[えん]に 成[な]ります。
\\	後百円です。	
\\	後[あと]百[ひゃく]円[えん]です。
\\	済みません。焼き増しをお願いします。	
\\	済[す]みません。 焼[や]き 増[ま]しをお 願[ねが]いします。
\\	光沢ありで宜しいですか。	
\\	光[こう]沢[たく]ありで 宜[よろ]しいですか。
\\	此処に名前と電話番号をお願いします。	
\\	此[こ]処[こ]に 名[な]前[まえ]と 電[でん]話[わ]番[ばん]号[ごう]をお 願[ねが]いします。
\\	仕上がりは十五日の三時です。	
\\	仕[し]上[あ]がりは 十[じゅう]五[ご]日[にち]の 三[さん]時[じ]です。
\\	此の引き換え券を持って来て下さい。	
\\	此[こ]の 引[ひ]き 換[か]え 券[けん]を 持[も]って 来[き]て 下[くだ]さい。
\\	ロバートさん、次の頁を読んで下さい。	
\\	ロバートさん、 次[つぎ]の 頁[ぺえじ]を 読[よ]んで 下[くだ]さい。
\\	起きて下さい。	
\\	起[お]きて 下[くだ]さい。
\\	クラスで寝ては行けません。	
\\	クラスで 寝[ね]ては 行[い]けません。
\\	先生、教科書を忘れました。	
\\	先[せん]生[せい]、 教[きょう]科[か]書[しょ]を 忘[わす]れました。
\\	教科書を持って来て下さいね。毎日使いますから。	
\\	教[きょう]科[か]書[しょ]を 持[も]って 来[き]て 下[くだ]さいね。 毎[まい]日[にち]使[つか]いますから。
\\	ロバートさん、今日は大変でしたね。	
\\	ロバートさん、 今[きょ]日[う]は 大[たい]変[へん]でしたね。
\\	後で貴方のノートを借りてもいいですか。	
\\	後[あと]で 貴方[あなた]のノートを 借[か]りてもいいですか。
\\	直ぐ返します。	
\\	直[す]ぐ 返[かえ]します。
\\	金曜日に休みましたからね。	
\\	金[きん]曜[よう]日[び]に 休[やす]みましたからね。
\\	今日は家に帰って勉強します。	
\\	今[きょ]日[う]は 家[いえ]に 帰[かえ]って 勉[べん]強[きょう]します。
\\	此のバスは市民病院へ行きますか。	
\\	此[こ]のバスは 市[し]民[みん] 病[びょう]院[いん]へ 行[い]きますか。
\\	あの、お祖母さん。どうぞ座って下さい。	
\\	あの、お 祖母[ばあ]さん。 どうぞ 座[すわ]って 下[くだ]さい。
\\	直ぐ降りますから。	
\\	直[す]ぐ 降[お]りますから。
\\	荷物を持ちましょうか。	
\\	荷[に]物[もつ]を 持[も]ちましょうか。
\\	教科書のテープを聞いて下さい。	
\\	教[きょう]科[か]書[しょ]のテープを 聞[き]いて 下[くだ]さい。
\\	一寸教えて下さい。	
\\	一[ちょっ]寸[と]教[おし]えて 下[くだ]さい。
\\	教科書を見てもいいですか。	
\\	教[きょう]科[か]書[しょ]を 見[み]てもいいですか。
\\	てもいいですか。	
\\	てもいいですか。
\\	てもいいです。	
\\	てもいいです。
\\	ては行けません。	
\\	ては 行[い]けません。
\\	いいえ、教科書を見ては行けません。	
\\	いいえ、 教[きょう]科[か]書[しょ]を 見[み]ては 行[い]けません。
\\	窓を開けて。	
\\	窓[まど]を 開[あ]けて。
\\	コピー	
\\	コピー	コꜜピー
\\	今日は、六時に起きて、勉強しました。	
\\	今[きょ]日[う]は、 六[ろく]時[じ]に 起[お]きて、 勉[べん]強[きょう]しました。
\\	食堂に行って、昼御飯を食べましょう。	
\\	食[しょく]堂[どう]に 行[い]って、 昼[ひる] 御[ご]飯[はん]を 食[た]べましょう。
\\	バスに乗って、会社に行きます。	
\\	バスに 乗[の]って、 会[かい]社[しゃ]に 行[い]きます。
\\	私は今晩勉強します。明日テストが有りますから。	
\\	私[わたし]は 今[こん]晩[ばん]勉[べん]強[きょう]します。 明日[あした]テストが 有[あ]りますから。 
\\	から。
\\	タクシー	
\\	タクシー	タꜜクシー
\\	バスに乗りましょう。タクシーは高いですから。	
\\	バスに 乗[の]りましょう。タクシーは 高[たか]いですから。 
\\	から。
\\	(私が)やりましょうか。	
\\	私[わたし]が)やりましょうか。
\\	荷物を持ちましょうか。	
\\	荷[に]物[もつ]を 持[も]ちましょうか。
\\	昨日一時に寝ました。 遅いですね。	
\\	昨[きの]日[う]一[いち]時[じ]に 寝[ね]ました。 遅[おそ]いですね。
\\	十時頃起きて、遅い朝御飯を食べました。	
\\	十[じゅう]時[じ]頃[ごろ]起[お]きて、 遅[おそ]い 朝[あさ]御[ご]飯[はん]を 食[た]べました。
\\	昨日、遅く寝ました。	
\\	昨[きの]日[う]、 遅[おそ]く 寝[ね]ました。
\\	明日電話を掛けて下さい。	
\\	明日[あした]電[でん]話[わ]を 掛[か]けて 下[くだ]さい。
\\	窓を開けて下さい。	
\\	窓[まど]を 開[あ]けて 下[くだ]さい。
\\	手紙を書いて下さい。	
\\	手[て]紙[がみ]を 書[か]いて 下[くだ]さい。
\\	お茶を飲んで下さい。	
\\	お 茶[ちゃ]を 飲[の]んで 下[くだ]さい。
\\	漢字を教えて下さい。	
\\	漢[かん]字[じ]を 教[おし]えて 下[くだ]さい。
\\	飲み物を持って来て下さい。	
\\	飲[の]み 物[もの]を 持[も]って 来[き]て 下[くだ]さい。
\\	待って下さい。	
\\	待[ま]って 下[くだ]さい。
\\	私と来て下さい。	
\\	私[わたし]と 来[き]て 下[くだ]さい。
\\	病院に行って下さい。	
\\	病[びょう]院[いん]に 行[い]って 下[くだ]さい。
\\	本を返して下さい。	
\\	本[ほん]を 返[かえ]して 下[くだ]さい。
\\	友達を連れて来て下さい。	
\\	友[とも]達[だち]を 連[つ]れて 来[き]て 下[くだ]さい。
\\	立って下さい。	
\\	立[た]って 下[くだ]さい。
\\	電気を点けて下さい。	
\\	電[でん]気[き]を 点[つ]けて 下[くだ]さい。
\\	窓を閉めて下さい。	
\\	窓[まど]を 閉[し]めて 下[くだ]さい。
\\	急いで下さい。	
\\	急[いそ]いで 下[くだ]さい。
\\	煙草を吸ってもいいですか。	
\\	煙草[たばこ]を 吸[す]ってもいいですか。
\\	遅く帰ってもいいですか。	
\\	遅[おそ]く 帰[かえ]ってもいいですか。
\\	夜、出掛けてもいいですか。	
\\	夜[よる]、 出[で]掛[か]けてもいいですか。
\\	電話を掛けてもいいですか。	
\\	電[でん]話[わ]を 掛[か]けてもいいですか。
\\	友達を連れて来てもいいですか。	
\\	友[とも]達[だち]を 連[つ]れて 来[き]てもいいですか。
\\	朝、お風呂に入ってもいいですか。	
\\	朝[あさ]、お 風[ふ]呂[ろ]に 入[はい]ってもいいですか。
\\	音楽を聞いてもいいですか。	
\\	音[おん]楽[がく]を 聞[き]いてもいいですか。
\\	お手洗いに行ってもいいですか。	
\\	お 手[て]洗[あら]いに 行[い]ってもいいですか。
\\	宿題を忘れました。明日、持って来てもいいですか。	
\\	宿[しゅく]題[だい]を 忘[わす]れました。 明日[あした]、 持[も]って 来[き]てもいいですか。
\\	英語を話してもいいですか。	
\\	英[えい]語[ご]を 話[はな]してもいいですか。
\\	写真を撮ってもいいですか。	
\\	写[しゃ]真[しん]を 撮[と]ってもいいですか。
\\	窓を開けてもいいですか。	
\\	窓[まど]を 開[あ]けてもいいですか。
\\	電気を点けてもいいですか。	
\\	電[でん]気[き]を 点[つ]けてもいいですか。
\\	煙草を吸っては行けません。	
\\	煙草[たばこ]を 吸[す]っては 行[い]けません。
\\	電話を掛けては行けません。	
\\	電[でん]話[わ]を 掛[か]けては 行[い]けません。
\\	朝、お風呂に入っては行けません。	
\\	朝[あさ]、お 風[ふ]呂[ろ]に 入[はい]っては 行[い]けません。
\\	遅く帰っては行けません。	
\\	遅[おそ]く 帰[かえ]っては 行[い]けません。
\\	友達を連れて来ては行けません。	
\\	友[とも]達[だち]を 連[つ]れて 来[き]ては 行[い]けません。
\\	音楽を聞いては行けません。	
\\	音[おん]楽[がく]を 聞[き]いては 行[い]けません。
\\	夜、出掛けては行けません。	
\\	夜[よる]、 出[で]掛[か]けては 行[い]けません。
\\	朝起きて、コーヒーを飲みます。	
\\	朝[あさ]起[お]きて、コーヒーを 飲[の]みます。
\\	テレビを消して、出掛けます。	
\\	テレビを 消[け]して、 出[で]掛[か]けます。
\\	朝御飯を食べて、お手洗いに行きます。	
\\	朝[あさ] 御[ご]飯[はん]を 食[た]べて、 お 手[て]洗[あら]いに 行[い]きます。
\\	お風呂に入って、デートをします。	
\\	お 風[ふ]呂[ろ]に 入[はい]って、 デートをします。
\\	テープを聞いて、寝ます。	
\\	テープを 聞[き]いて、 寝[ね]ます。
\\	新聞を読んで、コーヒーを飲みます。	
\\	新[しん]聞[ぶん]を 読[よ]んで、コーヒーを 飲[の]みます。
\\	大学に行って、勉強します。	
\\	大[だい]学[がく]に 行[い]って、 勉[べん]強[きょう]します。
\\	電車を降りて、公園に帰りました。	
\\	電[でん]車[しゃ]を 降[お]りて、 公[こう]園[えん]に 帰[かえ]りました。
\\	バスに乗ります。時間が有りませんから。	
\\	バスに 乗[の]ります。 時[じ]間[かん]が 有[あ]りませんから。
\\	先週は大変でした。毎日仕事に行きましたから。	
\\	先[せん]週[しゅう]は 大[たい]変[へん]でした。 毎[まい]日[にち]仕[し]事[ごと]に 行[い]きましたから。
\\	彼の映画を見ません。詰らないですから。	
\\	彼[あ]の 映[えい]画[が]を 見[み]ません。 詰[つま]らないですから。
\\	僕の友達が大好きです。面白い人ですから。	
\\	僕[ぼく]の 友[とも]達[だち]が 大[だい]好[す]きです。 面[おも]白[しろ]い 人[ひと]ですから。
\\	彼のレストランに行きました。美味しい食べ物が有りますから。	
\\	彼[あ]のレストランに 行[い]きました。 美[お]味[い]しい 食[た]べ 物[もの]が 有[あ]りますから。
\\	朝御飯を食べませんでした。余りお金が有りませんから。	
\\	朝[あさ]御[ご]飯[はん]を 食[た]べませんでした。 余[あま]りお 金[かね]が 有[あ]りませんから。
\\	私も朝御飯を食べませんでした。朝、忙しかったですから。	
\\	私[わたし]も 朝[あさ]御[ご]飯[はん]を 食[た]べませんでした。 朝[あさ]、 忙[いそが]しかったですから。
\\	お金が全然有りません。本を沢山買いましたから。	
\\	お 金[かね]が 全[ぜん]然[ぜん]有[あ]りません。 本[ほん]を 沢[たく]山[さん]買[か]いましたから。
\\	窓を開けましょうか。	
\\	窓[まど]を 開[あ]けましょうか。
\\	テレビを消しましょうか。	
\\	テレビを 消[け]しましょうか。
\\	手伝いましょうか。	
\\	手[て]伝[つだ]いましょうか。
\\	電気を消しましょうか。	
\\	電[でん]気[き]を 消[け]しましょうか。
\\	窓を閉めましょうか。	
\\	窓[まど]を 閉[し]めましょうか。
\\	お金を借りてもいいですか。明日友達の誕生日ですから。	
\\	お 金[かね]を 借[か]りてもいいですか。 明日[あした]友[とも]達[だち]の 誕[たん]生[じょう]日[び]ですから。
\\	貴方の車を借りてもいいですか。	
\\	貴方[あなた]の 車[くるま]を 借[か]りてもいいですか。
\\	其のケーキを食べてもいいですか。ケーキが大好きですから。	
\\	其[そ]のケーキを 食[た]べてもいいですか。ケーキが 大[だい]好[す]きですから。
\\	今週の週末何をしますか。	
\\	今[こん]週[しゅう]の 週[しゅう]末[まつ]何[なに]をしますか。
\\	今週の週末勉強して、食べて、寝ます。	
\\	今[こん]週[しゅう]の 週[しゅう]末[まつ]勉[べん]強[きょう]して、 食[た]べて、 寝[ね]ます。
\\	子供の時良く何をしましたか。	
\\	子[こ]供[ども]の 時[とき]良[よ]く 何[なに]をしましたか。
\\	子供の時良く本を読んで、テレビゲームをしました。	
\\	子[こ]供[ども]の 時[とき]良[よ]く 本[ほん]を 読[よ]んで、テレビゲームをしました。
\\	図書館で何をしては行けませんか。	
\\	図[と]書[しょ]館[かん]で 何[なに]をしては 行[い]けませんか。
\\	図書館で話しては行けません。	
\\	図[と]書[しょ]館[かん]で 話[はな]しては 行[い]けません。
\\	電車の中で煙草を吸ってもいいですか。 いいえ、電車の中で煙草を吸っては行けません。	
\\	電[でん]車[しゃ]の 中[なか]で 煙草[たばこ]を 吸[す]ってもいいですか。 いいえ、 電[でん]車[しゃ]の 中[なか]で 煙草[たばこ]を 吸[す]っては 行[い]けません。
\\	大学に何を持って来ますか。	
\\	大[だい]学[がく]に 何[なに]を 持[も]って 来[き]ますか。
\\	良く電車に乗りますか。	
\\	良[よ]く 電[でん]車[しゃ]に 乗[の]りますか。
\\	宿題を忘れましたか。	
\\	宿[しゅく]題[だい]を 忘[わす]れましたか。
\\	子供の時、何処で遊びましたか。	
\\	子[こ]供[ども]の 時[とき]、 何[ど]処[こ]で 遊[あそ]びましたか。
\\	子供の時良くお母さんを手伝いましたか。	
\\	子[こ]供[ども]の 時[とき]良[よ]くお 母[かあ]さんを 手[て]伝[つだ]いましたか。
\\	図書館で良く本を借りますか。	
\\	図[と]書[しょ]館[かん]で 良[よ]く 本[ほん]を 借[か]りますか。
\\	道を聞く 
\\	道を聞いて 
\\	道を聞きます	
\\	道[みち]を 聞[き]く 
\\	道[みち]を 聞[き]いて 
\\	道[みち]を 聞[き]きます
\\	真っ直ぐ行く 
\\	真っ直ぐ行って 
\\	真っ直ぐ行きます	
\\	真[ま]っ 直[す]ぐ 行[い]く 
\\	真[ま]っ 直[す]ぐ 行[い]って 
\\	真[ま]っ 直[す]ぐ 行[い]きます
\\	真っ直ぐ	
\\	真[ま]っ 直[す]ぐ	
\\	右に曲がる 
\\	右に曲がって 
\\	右に曲がります	
\\	右[みぎ]に 曲[ま]がる 
\\	右[みぎ]に 曲[ま]がって 
\\	右[みぎ]に 曲[ま]がります
\\	左に曲がる 
\\	左に曲がって 
\\	左に曲がります	
\\	左[ひだり]に 曲[ま]がる 
\\	左[ひだり]に 曲[ま]がって 
\\	左[ひだり]に 曲[ま]がります
\\	一つ目の信号を左に曲がる 
\\	一つ目の信号を左に曲がって 
\\	一つ目の信号を左に曲がります	
\\	一[ひと]つ 目[め]の 信[しん]号[ごう]を 左[ひだり]に 曲[ま]がる 
\\	一[ひと]つ 目[め]の 信[しん]号[ごう]を 左[ひだり]に 曲[ま]がって 
\\	一[ひと]つ 目[め]の 信[しん]号[ごう]を 左[ひだり]に 曲[ま]がります
\\	一つ目の信号を右に曲がる 
\\	一つ目の信号を右に曲がって 
\\	一つ目の信号を右に曲がります	
\\	一[ひと]つ 目[め]の 信[しん]号[ごう]を 右[みぎ]に 曲[ま]がる 
\\	一[ひと]つ 目[め]の 信[しん]号[ごう]を 右[みぎ]に 曲[ま]がって 
\\	一[ひと]つ 目[め]の 信[しん]号[ごう]を 右[みぎ]に 曲[ま]がります
\\	二つ目の角を左に曲がる 
\\	二つ目の角を左に曲がって 
\\	二つ目の角を左に曲がります	
\\	二[ふた]つ 目[め]の 角[かど]を 左[ひだり]に 曲[ま]がる 
\\	二[ふた]つ 目[め]の 角[かど]を 左[ひだり]に 曲[ま]がって 
\\	二[ふた]つ 目[め]の 角[かど]を 左[ひだり]に 曲[ま]がります
\\	一つ目の角を右に曲がる 
\\	一つ目の角を右に曲がって 
\\	一つ目の角を右に曲がります	
\\	一[ひと]つ 目[め]の 角[かど]を 右[みぎ]に 曲[ま]がる 
\\	一[ひと]つ 目[め]の 角[かど]を 右[みぎ]に 曲[ま]がって 
\\	一[ひと]つ 目[め]の 角[かど]を 右[みぎ]に 曲[ま]がります
\\	道を渡る 
\\	道を渡って 
\\	道を渡ります	
\\	道[みち]を 渡[わた]る 
\\	道[みち]を 渡[わた]って 
\\	道[みち]を 渡[わた]ります
\\	道の右側	
\\	道[みち]の 右[みぎ]側[がわ]	
\\	道の左側	
\\	道[みち]の 左[ひだり]側[がわ]	
\\	真っ直ぐ行って、三つ目の角を右に曲がって下さい。郵便局は道の右側に有りますよ。	
\\	真[ま]っ 直[す]ぐ 行[い]って、 三[みっ]つ 目[め]の 角[かど]を 右[みぎ]に 曲[ま]がって 下[くだ]さい。 郵[ゆう]便[びん]局[きょく]は 道[みち]の 右[みぎ]側[がわ]に 有[あ]りますよ。
\\	此れはスーさんの家族の写真ですか。	
\\	此[こ]れはスーさんの 家[か]族[ぞく]の 写[しゃ]真[しん]ですか。
\\	スーさんは何れですか。	
\\	スーさんは 何[ど]れですか。
\\	此れです。高校の時は眼鏡を掛けていました。	
\\	此[こ]れです。 高[こう]校[こう]の 時[とき]は 眼[め]鏡[がね]を 掛[か]けていました。
\\	可愛いですね。	
\\	可愛[かわい]いですね。
\\	此れは父です。	
\\	此[こ]れは 父[ちち]です。
\\	アメリカの会社に勤めています。	
\\	アメリカの 会[かい]社[しゃ]に 勤[つと]めています。
\\	背が高くてハンサムですね。	
\\	背[せ]が 高[たか]くてハンサムですね。
\\	此れはお姉さんですか。	
\\	此[こ]れはお 姉[ねえ]さんですか。
\\	姉は結婚しています。	
\\	姉[あね]は 結[けっ]婚[こん]しています。
\\	今ソウルに住んでいます。	
\\	今[いま]ソウルに 住[す]んでいます。
\\	子供が一人居ます。三歳です。	
\\	子[こ]供[ども]が 一[ひと]人[り]居[い]ます。 三[さん]歳[さい]です。
\\	あっ、猫が居ますね。	
\\	あっ、 猫[ねこ]が 居[い]ますね。
\\	でも、一寸太っていますね。良く食べますから。	
\\	でも、 一[ちょっ]寸[と]太[ふと]っていますね。 良[よ]く 食[た]べますから。
\\	美智子	
\\	美[み]智[ち]子[こ]	
\\	美智子さん、今何をしていますか。	
\\	美[み]智[ち]子[こ]さん、 今[いま]何[なに]をしていますか。
\\	別に何もしていません。今、スーさんの写真を見ています。	
\\	別[べつ]に 何[なに]もしていません。 今[いま]、スーさんの 写[しゃ]真[しん]を 見[み]ています。
\\	面白いビデオが有りますから、良かったら見に来ませんか。	
\\	面[おも]白[しろ]いビデオが 有[あ]りますから、 良[よ]かったら 見[み]に 来[き]ませんか。
\\	いいですね。スーさんも一緒に行ってもいいですか。	
\\	いいですね。スーさんも 一[いっ]緒[しょ]に 行[い]ってもいいですか。
\\	じゃあ、直ぐ行きます。	
\\	じゃあ、 直[す]ぐ 行[い]きます。
\\	スーさんは今勉強しています。	
\\	スーさんは 今[いま]勉[べん]強[きょう]しています。
\\	たけしさんは英語の本を読んでいます。	
\\	たけしさんは 英[えい]語[ご]の 本[ほん]を 読[よ]んでいます。
\\	私は英語を教えています。	
\\	私[わたし]は 英[えい]語[ご]を 教[おし]えています。
\\	メアリーさんは日本語を勉強しています。	
\\	メアリーさんは 日[に]本[ほん]語[ご]を 勉[べん]強[きょう]しています。
\\	山下先生は結婚しています。	
\\	山[やま]下[した]先[せん]生[せい]は 結[けっ]婚[こん]しています。
\\	美智子さんは窓の側に座っています。	
\\	美[み]智[ち]子[こ]さんは 窓[まど]の 側[そば]に 座[すわ]っています。
\\	持っている	
\\	持[も]っている
\\	知っている	
\\	知[し]っている
\\	太っている	
\\	太[ふと]っている
\\	痩せている	
\\	痩[や]せている	
\\	スーさんはお金を沢山持っています。	
\\	スーさんはお 金[かね]を 沢[たく]山[さん]持[も]っています。
\\	山下先生は英語を知っています。	
\\	山[やま]下[した]先[せん]生[せい]は 英[えい]語[ご]を 知[し]っています。
\\	トムさんは一寸太っています。	
\\	トムさんは 一[ちょっ]寸[と]太[ふと]っています。
\\	私の弟は迚も痩せています。	
\\	私[わたし]の 弟[おとうと]は 迚[とて]も 痩[や]せています。
\\	着ている	
\\	着[き]ている
\\	起きている	
\\	起[お]きている
\\	住んでいる	
\\	住[す]んでいる
\\	勤めている	
\\	勤[つと]めている
\\	メアリーさんは
\\	シャツを着ています。	
\\	メアリーさんは 
\\	シャツを 着[き]ています。
\\	お父さんは起きています。	
\\	お 父[とう]さんは 起[お]きています。
\\	父と母は東京に住んでいます。	
\\	父[ちち]と 母[はは]は 東[とう]京[きょう]に 住[す]んでいます。
\\	姉は日本の会社に勤めています。	
\\	姉[あね]は 日[に]本[ほん]の 会[かい]社[しゃ]に 勤[つと]めています。
\\	中国に行っています。	
\\	中[ちゅう]国[ごく]に 行[い]っています。
\\	食べています。	
\\	食[た]べています。
\\	食べていません。	
\\	食[た]べていません。
\\	食べていました。	
\\	食[た]べていました。
\\	食べていませんでした。	
\\	食[た]べていませんでした。
\\	トムさんは髪が長いです。	
\\	トムさんは 髪[かみ]が 長[なが]いです。
\\	彼の店の食べ物は安くて、美味しいです。	
\\	彼[あ]の 店[みせ]の 食[た]べ 物[もの]は 安[やす]くて、 美[お]味[い]しいです。
\\	彼の人は何時も元気で、面白いです。	
\\	彼[あ]の 人[ひと]は 何[い]時[つ]も 元[げん]気[き]で、 面[おも]白[しろ]いです。
\\	山下先生は日本人で、四十歳です。	
\\	山[やま]下[した]先[せん]生[せい]は日[に]本[ほん]人[じん]で、 四[よん]十[じゅう]歳[さい]です。
\\	デパートに鞄を買いに行きました。	
\\	デパートに 鞄[かばん]を 買[か]いに 行[い]きました。
\\	メアリーさんは日本に日本語を勉強しに来ました。	
\\	メアリーさんは 日[に]本[ほん]に 日[に]本[ほん]語[ご]を 勉[べん]強[きょう]しに 来[き]ました。
\\	三人	
\\	三[さん]人[にん]
\\	四人	
\\	四[よ]人[にん]	ヨッタリꜜ 
\\	ヨッタリ
\\	五人	
\\	五[ご]人[にん]
\\	六人	
\\	六[ろく]人[にん]	
\\	七人 
\\	七人	
\\	七[なな]人[にん] 
\\	七[しち]人[にん]
\\	八人	
\\	八[はち]人[にん]	
\\	九人	
\\	九[きゅう]人[にん]
\\	十人	
\\	十[じゅう]人[にん]	
\\	私のクラスに(は)スウェーデン人の学生が一人居ます。	
\\	私[わたし]のクラスに(は)スウェーデン 人[じん]の 学[がく]生[せい]が 一[ひと]人[り]居[い]ます。
\\	子供の時良く友達と遊びました。	
\\	子[こ]供[ども]の 時[とき]良[よ]く 友[とも]達[だち]と 遊[あそ]びました。
\\	遅く迄遊んでは行けません。	
\\	遅[おそ]く 迄[まで]遊[あそ]んでは 行[い]けません。
\\	先週の週末は東京に遊びに行きました。	
\\	先[せん]週[しゅう]の 週[しゅう]末[まつ]は 東[とう]京[きょう]に 遊[あそ]びに 行[い]きました。
\\	僕の家に遊びに来て下さい。	
\\	僕[ぼく]の 家[うち]に 遊[あそ]びに 来[き]て 下[くだ]さい。
\\	バスケット(ボール)	
\\	バスケット(ボール)	
\\	トランプ	
\\	トランプ	トラꜜンプ
\\	ギターを弾く 
\\	ギターを弾いて 
\\	ギターを弾きます	
\\	ギターを 弾[ひ]く 
\\	ギターを 弾[ひ]いて 
\\	ギターを 弾[ひ]きます
\\	メアリーさんは何をしていますか。 
\\	メアリーさんはテレビを見ています。	
\\	メアリーさんは 何[なに]をしていますか。 
\\	メアリーさんはテレビを 見[み]ています。
\\	メアリーさんは本を読んでいます。	
\\	メアリーさんは 本[ほん]を 読[よ]んでいます。
\\	メアリーさんは泳いでいます。	
\\	メアリーさんは 泳[およ]いでいます。
\\	メアリーさんは日本語のテープを聞いています。	
\\	メアリーさんは 日[に]本[ほん]語[ご]のテープを 聞[き]いています。
\\	メアリーさんは写真を撮っています。	
\\	メアリーさんは 写[しゃ]真[しん]を 撮[と]っています。
\\	メアリーさんは歌を歌っています。	
\\	メアリーさんは 歌[うた]を 歌[うた]っています。
\\	メアリーさんは日本語を話しています。	
\\	メアリーさんは 日[に]本[ほん]語[ご]を 話[はな]しています。
\\	メアリーさんは友達を待っています。	
\\	メアリーさんは 友[とも]達[だち]を 待[ま]っています。
\\	メアリーさんは煙草を吸っています。	
\\	メアリーさんは 煙草[たばこ]を 吸[す]っています。
\\	メアリーさんはテニスをしています。	
\\	メアリーさんはテニスをしています。
\\	メアリーさんはコーヒーを飲んでいます。	
\\	メアリーさんはコーヒーを 飲[の]んでいます。
\\	メアリーさんは電話を掛けています。	
\\	メアリーさんは 電[でん]話[わ]を 掛[か]けています。
\\	午後二時頃何をしていますか。	
\\	午[ご]後[ご]二[に]時[じ]頃[ごろ]何[なに]をしていますか。
\\	友達と部屋で勉強しています。	
\\	友[とも]達[だち]と 部[へ]屋[や]で 勉[べん]強[きょう]しています。
\\	田中さんは海で泳いでいます。	
\\	田[た]中[なか]さんは 海[うみ]で 泳[およ]いでいます。
\\	スーさんのお父さんは何処に住んでいますか。	
\\	スーさんのお 父[とう]さんは 何[ど]処[こ]に 住[す]んでいますか。
\\	ニューヨーク	
\\	ニューヨーク	
\\	お父さんはニューヨークに住んでいます。	
\\	お 父[とう]さんはニューヨークに 住[す]んでいます。
\\	お姉さんは何処に住んでいますか。 
\\	お姉さんはソウルに住んでいます。	
\\	お 姉[ねえ]さんは 何[ど]処[こ]に 住[す]んでいますか。 
\\	お 姉[ねえ]さんはソウルに 住[す]んでいます。
\\	弟さんはアメリカに住んでいますか。 
\\	いいえ、アメリカに住んでいません。 ロンドンに住んでいます。	
\\	弟[おとうと]さんはアメリカに 住[す]んでいますか。 
\\	いいえ、 アメリカに 住[す]んでいません。 ロンドンに 住[す]んでいます。
\\	ロンドン	
\\	ロンドン
\\	お母さんは何をしていますか。 
\\	お母さんは高校の先生です。	
\\	お 母[かあ]さんは 何[なに]をしていますか。 
\\	お 母[かあ]さんは 高[こう]校[こう]の 先[せん]生[せい]です。
\\	お姉さんは何をしていますか。 
\\	お姉さんは銀行に勤めています。	
\\	お 姉[ねえ]さんは 何[なに]をしていますか。 
\\	お 姉[ねえ]さんは 銀[ぎん]行[こう]に 勤[つと]めています。
\\	お姉さんは結婚していますか。 
\\	はい、お姉さんは結婚しています。	
\\	お 姉[ねえ]さんは 結[けっ]婚[こん]していますか。 
\\	はい、お 姉[ねえ]さんは 結[けっ]婚[こん]しています。
\\	弟さんは結婚していますか。 
\\	いいえ、弟さんは結婚していません。	
\\	弟[おとうと]さんは 結[けっ]婚[こん]していますか。 
\\	いいえ、 弟[おとうと]さんは 結[けっ]婚[こん]していません。
\\	お父さんは何歳ですか。 
\\	お父さんは四十八歳です。	
\\	お 父[とう]さんは 何[なん]歳[さい]ですか。 
\\	お 父[とう]さんは 四[よん]十[じゅう]八[はっ]歳[さい]です。
\\	お父さんは日本の会社に勤めていますか。 
\\	いいえ、お父さんはアメリカの会社に勤めています。	
\\	お 父[とう]さんは 日[に]本[ほん]の 会[かい]社[しゃ]に 勤[つと]めていますか。 
\\	いいえ、お 父[とう]さんはアメリカの 会[かい]社[しゃ]に 勤[つと]めています。
\\	山田	
\\	山[やま]田[だ]	ヤマダ
\\	吉川	
\\	吉[よし]川[かわ]	
\\	川村	
\\	川[かわ]村[むら]	
\\	中山	
\\	中[なか]山[やま]
\\	小川	
\\	小[お]川[がわ]	オカ°ワ
\\	水野	
\\	水[みず]野[の]	
\\	山中	
\\	山[やま]中[なか]	サꜜンチュー 
\\	ヤマナカ
\\	山田さんは太っていますか。 
\\	いいえ、山田さんは太っていません。	
\\	山[やま]田[だ]さんは 太[ふと]っていますか。 
\\	いいえ、 山[やま]田[だ]さんは 太[ふと]っていません。
\\	山田さんは
\\	シャツを着ていますか。 
\\	いいえ、山田さんはセーターを着ています。	
\\	山[やま]田[だ]さんは 
\\	シャツを 着[き]ていますか。 
\\	いいえ、 山[やま]田[だ]さんはセーターを 着[き]ています。
\\	吉川さんは何を着ていますか。 
\\	吉川さんはトレーナーを着ています。	
\\	吉[よし]川[かわ]さんは 何[なに]を 着[き]ていますか。 
\\	吉[よし]川[かわ]さんはトレーナーを 着[き]ています。
\\	山田さんはジーンズを履いていますか。 
\\	はい、山田さんはジーンズを履いています。	
\\	山[やま]田[だ]さんはジーンズを 履[は]いていますか。 
\\	はい、 山[やま]田[だ]さんはジーンズを 履[は]いています。
\\	吉川さんは眼鏡を掛けていますか。 
\\	いいえ、吉川さんは 眼鏡を掛けていません。	
\\	吉[よし]川[かわ]さんは 眼[め]鏡[がね]を 掛[か]けていますか。 
\\	いいえ、 吉[よし]川[かわ]さんは 眼[め]鏡[がね]を 掛[か]けていません。
\\	吉川さんは傘を持っていますか。 
\\	いいえ、吉川さんは傘を持っていません。	
\\	吉[よし]川[かわ]さんは 傘[かさ]を 持[も]っていますか。 
\\	いいえ、 吉[よし]川[かわ]さんは 傘[かさ]を 持[も]っていません。
\\	山田さんは背が高いですか。 
\\	はい、山田さんは背が高いです。	
\\	山[やま]田[だ]さんは 背[せ]が 高[たか]いですか。 
\\	はい、 山[やま]田[だ]さんは 背[せ]が 高[たか]いです。
\\	吉川さんは背が低いですか。 
\\	はい、吉川さんは背が低いです。	
\\	吉[よし]川[かわ]さんは 背[せ]が 低[ひく]いですか。 
\\	はい、 吉[よし]川[かわ]さんは 背[せ]が 低[ひく]いです。
\\	山田さんは髪が長いですか。 
\\	いいえ、山田さんは髪が短いです。	
\\	山[やま]田[だ]さんは 髪[かみ]が 長[なが]いですか。 
\\	いいえ、 山[やま]田[だ]さんは 髪[かみ]が 短[みじか]いです。
\\	吉川さんは目が小さいですか。 
\\	いいえ、吉川さんは目が大きいです。	
\\	吉[よし]川[かわ]さんは 目[め]が 小[ちい]さいですか。 
\\	いいえ、 吉[よし]川[かわ]さんは 目[め]が 大[おお]きいです。
\\	水野さんは帽子を被っています。	
\\	水[みず]野[の]さんは 帽[ぼう]子[し]を 被[かぶ]っています。
\\	小川さんは髪が長いです。	
\\	小[お]川[がわ]さんは 髪[かみ]が 長[なが]いです。
\\	中山さんは髪が短いです。	
\\	中[なか]山[やま]さんは 髪[かみ]が 短[みじか]いです。
\\	山中さんは本を持っています。	
\\	山[やま]中[なか]さんは 本[ほん]を 持[も]っています。
\\	大学は新しくて綺麗です。	
\\	大[だい]学[がく]は 新[あたら]しくて 綺[き]麗[れい]です。
\\	美智子さんは綺麗で易しいです。	
\\	美[み]智[ち]子[こ]さんは 綺[き]麗[れい]で 易[やさ]しいです。
\\	東京は大きくて賑やかです。	
\\	東[とう]京[きょう]は 大[おお]きくて 賑[にぎ]やかです。
\\	たけしさんは背が高くて格好いいです。	
\\	たけしさんは 背[せ]が 高[たか]くて 格[かっ]好[こ]いいです。
\\	ロバートさんは静かで大きいです。	
\\	ロバートさんは 静[しず]かで 大[おお]きいです。
\\	新幹線は速くて便利です。	
\\	新[しん]幹[かん]線[せん]は 速[はや]くて 便[べん]利[り]です。
\\	スーさんは頭が良くて親切です。	
\\	スーさんは 頭[あたま]が 良[よ]くて 親[しん]切[せつ]です。
\\	私の国の人は元気で賑やかです。	
\\	私[わたし]の 国[くに]の 人[ひと]は 元[げん]気[き]で 賑[にぎ]やかです。
\\	隣の人は背が高くて親切でした。	
\\	隣[となり]の 人[ひと]は 背[せ]が 高[たか]くて 親[しん]切[せつ]でした。
\\	隣の人	
\\	隣[となり]の 人[ひと]	
\\	私の町は静かで良かったです。	
\\	私[わたし]の 町[まち]は 静[しず]かで 良[よ]かったです。
\\	私の先生は大きくて怖かったです。	
\\	私[わたし]の 先[せん]生[せい]は 大[おお]きくて 怖[こわ]かったです。
\\	私の家は古くて綺麗じゃありませんでした。	
\\	私[わたし]の 家[いえ]は 古[ふる]くて 綺[き]麗[れい]じゃありませんでした。
\\	クラスは長くて面白くありませんでした。	
\\	クラスは 長[なが]くて 面[おも]白[しろ]くありませんでした。
\\	私の友達は親切で面白かったです。	
\\	私[わたし]の 友[とも]達[だち]は 親[しん]切[せつ]で 面[おも]白[しろ]かったです。
\\	高校は賑やかで楽しかったです。	
\\	高[こう]校[こう]は 賑[にぎ]やかで 楽[たの]しかったです。
\\	宿題は難しくて大変でした。	
\\	宿[しゅく]題[だい]は 難[むずか]しくて 大[たい]変[へん]でした。
\\	私は小さくて可愛かったです。	
\\	私[わたし]は 小[ちい]さくて 可愛[かわい]かったです。
\\	私のルームメートは親切で面白いです。	
\\	私[わたし]のルームメートは 親[しん]切[せつ]で 面[おも]白[しろ]いです。
\\	私の町は大きくて賑やかです。	
\\	私[わたし]の 町[まち]は 大[おお]きくて 賑[にぎ]やかです。
\\	私のクラスは長くて詰らないです。	
\\	私[わたし]のクラスは 長[なが]くて 詰[つま]らないです。
\\	日本人は背が低くて元気です。	
\\	日[に]本[ほん]人[じん]は 背[せ]が 低[ひく]くて 元[げん]気[き]です。
\\	スーさんは京都に歌舞伎を見に行きます。	
\\	スーさんは 京[きょう]都[と]に 歌[か]舞[ぶ]伎[き]を 見[み]に 行[い]きます。
\\	スーさんは図書館に本を借りに行きます。	
\\	スーさんは 図[と]書[しょ]館[かん]に 本[ほん]を 借[か]りに 行[い]きます。
\\	スーさんは
\\	にテープを聞きに行きます。	
\\	スーさんは 
\\	[エル]
\\	[エル]にテープを 聞[き]きに 行[い]きます。
\\	スーさんは食堂に昼御飯を食べに行きます。	
\\	スーさんは 食[しょく]堂[どう]に 昼[ひる] 御[ご]飯[はん]を 食[た]べに 行[い]きます。
\\	スーさんは郵便局に切手を買いに行きます。	
\\	スーさんは 郵[ゆう]便[びん]局[きょく]に 切[きっ]手[て]を 買[か]いに 行[い]きます。
\\	スーさんは公園に写真を撮りに行きます。	
\\	スーさんは 公[こう]園[えん]に 写[しゃ]真[しん]を 撮[と]りに 行[い]きます。
\\	スーさんは友達の家に勉強しに行きます。	
\\	スーさんは 友[とも]達[だち]の 家[うち]に 勉[べん]強[きょう]しに 行[い]きます。
\\	スーさんは町に遊びに行きます。	
\\	スーさんは 町[まち]に 遊[あそ]びに 行[い]きます。
\\	スーさんはデパートに靴を買いに行きます。	
\\	スーさんはデパートに 靴[くつ]を 買[か]いに 行[い]きます。
\\	スーさんは高校に英語を教えに行きます。	
\\	スーさんは 高[こう]校[こう]に 英[えい]語[ご]を 教[おし]えに 行[い]きます。
\\	スーさんは喫茶店にコーヒーを飲みに行きます。	
\\	スーさんは 喫[きっ]茶[さ]店[てん]にコーヒーを 飲[の]みに 行[い]きます。
\\	コンビニにパンを買いに行きます。	
\\	コンビニにパンを 買[か]いに 行[い]きます。
\\	東京に友達に会いに行きます。	
\\	東[とう]京[きょう]に 友[とも]達[だち]に 会[あ]いに 行[い]きます。
\\	図書館に本を返しに行きます。	
\\	図[と]書[しょ]館[かん]に 本[ほん]を 返[かえ]しに 行[い]きます。
\\	家に寝に帰ります。	
\\	家[いえ]に 寝[ね]に 帰[かえ]ります。
\\	大学に仏語を勉強しに来ました。	
\\	大[だい]学[がく]に 仏[ふつ]語[ご]を 勉[べん]強[きょう]しに 来[き]ました。
\\	此の部屋に女の人が何人居ますか。 
\\	二人居ます。	
\\	此[こ]の 部[へ]屋[や]に 女[おんな]の 人[ひと]が 何[なん]人[にん]居[い]ますか。 
\\	二[ふた]人[り]居[い]ます。
\\	何人	
\\	何[なん]人[にん]	ナニビト 
\\	ナꜜンニン 
\\	ナンピト
\\	此の部屋に男の人が何人居ますか。 
\\	一人居ます。	
\\	此[こ]の 部[へ]屋[や]に 男[おとこ]の 人[ひと]が 何[なん]人[にん]居[い]ますか。 
\\	一[ひと]人[り]居[い]ます。
\\	此の部屋にポーランド人が何人居ますか。 
\\	十人居ます。	
\\	此[こ]の 部[へ]屋[や]にポーランド 人[じん]が 何[なん]人[にん]居[い]ますか。 
\\	十[じゅう]人[にん]居[い]ます。
\\	此の部屋に髪が長い人が何人居ますか。 
\\	三人居ます。	
\\	此[こ]の 部[へ]屋[や]に 髪[かみ]が 長[なが]い 人[ひと]が 何[なん]人[にん]居[い]ますか。 
\\	三[さん]人[にん]居[い]ます。
\\	此の部屋に元気な人が何人居ますか。 
\\	十五人居ます。	
\\	此[こ]の 部[へ]屋[や]に 元[げん]気[き]な 人[ひと]が 何[なん]人[にん]居[い]ますか。 
\\	十[じゅう]五[ご]人[にん]居[い]ます。
\\	何処に住んでいますか。	
\\	何[ど]処[こ]に 住[す]んでいますか。
\\	結婚していますか。	
\\	結[けっ]婚[こん]していますか。
\\	自転車を持っていますか。	
\\	自[じ]転[てん]車[しゃ]を 持[も]っていますか。
\\	一輪車を持っていますか。	
\\	一[いち]輪[りん]車[しゃ]を 持[も]っていますか。
\\	車を持っていますか。	
\\	車[くるま]を 持[も]っていますか。
\\	日本の歌を知っていますか。	
\\	日[に]本[ほん]の 歌[うた]を 知[し]っていますか。
\\	日本の歌を沢山知っています。	
\\	日[に]本[ほん]の 歌[うた]を 沢[たく]山[さん]知[し]っています。
\\	日本語の先生は今日何を着ていますか。	
\\	日[に]本[ほん]語[ご]の 先[せん]生[せい]は 今[きょ]日[う]何[なに]を 着[き]ていますか。
\\	兄弟が居ますか。何人居ますか。	
\\	兄[きょう]弟[だい]が 居[い]ますか。 何[なん]人[にん]居[い]ますか。
\\	お母さんは何処に勤めていますか。	
\\	お 母[かあ]さんは 何[ど]処[こ]に 勤[つと]めていますか。
\\	子供の時、自転車を持っていましたか。	
\\	子[こ]供[ども]の 時[とき]、 自[じ]転[てん]車[しゃ]を 持[も]っていましたか。
\\	高校の時、日本語を知っていましたか。	
\\	高[こう]校[こう]の 時[とき]、 日[に]本[ほん]語[ご]を 知[し]っていましたか。
\\	妹さん	
\\	妹[いもうと]さん
\\	弟さん	
\\	弟[おとうと]さん
\\	田中さんのお父さんは何歳ですか。	
\\	田[た]中[なか]さんのお 父[とう]さんは 何[なん]歳[さい]ですか。
\\	父は五十歳です。	
\\	父[ちち]は 五[ご]十[じゅう]歳[さい]です。
\\	お父さんは五十歳です。	
\\	お 父[とう]さんは 五[ご]十[じゅう]歳[さい]です。
\\	田中さんのお兄さんは何処に住んでいますか。	
\\	田[た]中[なか]さんのお 兄[にい]さんは 何[ど]処[こ]に 住[す]んでいますか。
\\	お兄さんは東京に住んでいます。	
\\	お 兄[にい]さんは 東[とう]京[きょう]に 住[す]んでいます。
\\	兄は東京に住んでいます。	
\\	兄[あに]は 東[とう]京[きょう]に 住[す]んでいます。
\\	たけしさん、明日皆でバーベキューをしませんか。	
\\	たけしさん、 明日[あした]皆[みんな]でバーベキューをしませんか。
\\	いいですね。誰が来ますか。	
\\	いいですね。 誰[だれ]が 来[き]ますか。
\\	スーさんとロバートさんが来ます。メアリーさんも来ると思います。	
\\	スーさんとロバートさんが 来[き]ます。メアリーさんも 来[く]ると 思[おも]います。
\\	けんさんはアルバイトが有ると言っていました。	
\\	けんさんはアルバイトが 有[あ]ると 言[い]っていました。
\\	残念ですね。何か持って行きましょうか。	
\\	残[ざん]念[ねん]ですね。 何[なに]か 持[も]って 行[い]きましょうか。
\\	何も要らないと思います。	
\\	何[なに]も 要[い]らないと 思[おも]います。
\\	上手ですね。ロバートさんは料理するのが好きですか。	
\\	上[じょう]手[ず]ですね。ロバートさんは 料[りょう]理[り]するのが 好[す]きですか。
\\	ええ、良く家で作ります。	
\\	ええ、 良[よ]く 家[うち]で 作[つく]ります。
\\	何か手伝いましょうか。	
\\	何[なに]か 手[て]伝[つだ]いましょうか。
\\	じゃあ、トマトを切って下さい。	
\\	じゃあ、トマトを 切[き]って 下[くだ]さい。
\\	始めましょうか。	
\\	始[はじ]めましょうか。
\\	あっ、未だ飲まないで下さい。	
\\	あっ、 未[ま]だ 飲[の]まないで 下[くだ]さい。
\\	メアリーさんも来ると言っていましたから。	
\\	メアリーさんも 来[く]ると 言[い]っていましたから。
\\	遅く成って済みません。	
\\	遅[おそ]く 成[な]って 済[す]みません。
\\	じゃあ、乾杯。	
\\	じゃあ、 乾[かん]杯[ぱい]。
\\	飲む 
\\	飲まない	
\\	飲[の]む 
\\	飲[の]まない
\\	食べる 
\\	食べない	
\\	食[た]べる 
\\	食[た]べない
\\	電話を掛ける 
\\	電話を掛けない	
\\	電[でん]話[わ]を 掛[か]ける 
\\	電[でん]話[わ]を 掛[か]けない
\\	雨が降る 
\\	雨が降らない	
\\	雨[あめ]が 降[ふ]る 
\\	雨[あめ]が 降[ふ]らない
\\	降りる 
\\	降りない	
\\	降[お]りる 
\\	降[お]りない
\\	開ける 
\\	開けない	
\\	開[あ]ける 
\\	開[あ]けない
\\	閉める 
\\	閉めない	
\\	閉[し]める 
\\	閉[し]めない
\\	運転する 
\\	運転しない	
\\	運[うん]転[てん]する 
\\	運[うん]転[てん]しない
\\	運動する 
\\	運動しない	
\\	運[うん]動[どう]する 
\\	運[うん]動[どう]しない
\\	遊ぶ 
\\	遊ばない	
\\	遊[あそ]ぶ 
\\	遊[あそ]ばない
\\	遅くなる 
\\	遅くならない	
\\	遅[おそ]くなる 
\\	遅[おそ]くならない
\\	連れて来る 
\\	連れて来ない	
\\	連[つ]れて 来[く]る 
\\	連[つ]れて 来[こ]ない
\\	返す 
\\	返さない	
\\	返[かえ]す 
\\	返[かえ]さない
\\	踊る 
\\	踊らない	
\\	踊[おど]る 
\\	踊[おど]らない
\\	起きる 
\\	起きない	
\\	起[お]きる 
\\	起[お]きない
\\	買う 
\\	買わない	
\\	買[か]う 
\\	買[か]わない
\\	貰う 
\\	貰わない	
\\	貰[もら]う 
\\	貰[もら]わない
\\	読む 
\\	読まない	
\\	読[よ]む 
\\	読[よ]まない
\\	話す 
\\	話さない	
\\	話[はな]す 
\\	話[はな]さない
\\	言う 
\\	言わない	
\\	言[い]う 
\\	言[い]わない
\\	覚える 
\\	覚えない	
\\	覚[おぼ]える 
\\	覚[おぼ]えない
\\	見る 
\\	見ない	
\\	見[み]る 
\\	見[み]ない
\\	要る 
\\	要らない	
\\	要[い]る 
\\	要[い]らない 
\\	居る)
\\	被る 
\\	被らない	
\\	被[かぶ]る 
\\	被[かぶ]らない
\\	行く 
\\	行かない	
\\	行[い]く 
\\	行[い]かない
\\	薬を飲む 
\\	薬を飲まない	
\\	薬[くすり]を 飲[の]む 
\\	薬[くすり]を 飲[の]まない
\\	聞く 
\\	聞かない	
\\	聞[き]く 
\\	聞[き]かない
\\	結婚する 
\\	結婚しない	
\\	結[けっ]婚[こん]する 
\\	結[けっ]婚[こん]しない
\\	終わる 
\\	終わらない	
\\	終[お]わる 
\\	終[お]わらない
\\	立つ 
\\	立たない	
\\	立[た]つ 
\\	立[た]たない
\\	知る 
\\	知らない	
\\	知[し]る 
\\	知[し]らない
\\	着る 
\\	着ない	
\\	着[き]る 
\\	着[き]ない 
\\	切る)
\\	眼鏡を掛ける 
\\	眼鏡を掛けない	
\\	眼[め]鏡[がね]を 掛[か]ける 
\\	眼[め]鏡[がね]を 掛[か]けない
\\	痩せる 
\\	痩せない	
\\	痩[や]せる 
\\	痩[や]せない
\\	点ける 
\\	点けない	
\\	点[つ]ける 
\\	点[つ]けない
\\	渡る 
\\	渡らない	
\\	渡[わた]る 
\\	渡[わた]らない
\\	消す 
\\	消さない	
\\	消[け]す 
\\	消[け]さない
\\	洗濯する 
\\	洗濯しない	
\\	洗[せん]濯[たく]する 
\\	洗[せん]濯[たく]しない
\\	洗う 
\\	洗わない	
\\	洗[あら]う 
\\	洗[あら]わない
\\	泳ぐ 
\\	泳がない	
\\	泳[およ]ぐ 
\\	泳[およ]がない
\\	死ぬ 
\\	死なない	
\\	死[し]ぬ 
\\	死[し]なない
\\	歌う 
\\	歌わない	
\\	歌[うた]う 
\\	歌[うた]わない
\\	来る 
\\	来ない	
\\	来[く]る 
\\	来[こ]ない
\\	有る 
\\	ない	
\\	有[あ]る 
\\	ない
\\	書く 
\\	書かない	
\\	書[か]く 
\\	書[か]かない 
\\	描く)
\\	曲がる 
\\	曲がらない	
\\	曲[ま]がる 
\\	曲[ま]がらない
\\	料理する 
\\	料理しない	
\\	料[りょう]理[り]する 
\\	料[りょう]理[り]しない
\\	教える 
\\	教えない	
\\	教[おし]える 
\\	教[おし]えない
\\	撮る 
\\	撮らない	
\\	撮[と]る 
\\	撮[と]らない 
\\	取る)
\\	掃除する 
\\	掃除しない	
\\	掃[そう]除[じ]する 
\\	掃[そう]除[じ]しない
\\	持つ 
\\	持たない	
\\	持[も]つ 
\\	持[も]たない
\\	持って行く 
\\	持って行かない	
\\	持[も]って 行[い]く 
\\	持[も]って 行[い]かない
\\	持って来る 
\\	持って来ない	
\\	持[も]って 来[く]る 
\\	持[も]って 来[こ]ない
\\	手伝う 
\\	手伝わない	
\\	手[て]伝[つだ]う 
\\	手[て]伝[つだ]わない
\\	急ぐ 
\\	急がない	
\\	急[いそ]ぐ 
\\	急[いそ]がない
\\	思う 
\\	思わない	
\\	思[おも]う 
\\	思[おも]わない
\\	忘れる 
\\	忘れない	
\\	忘[わす]れる 
\\	忘[わす]れない
\\	待つ 
\\	待たない	
\\	待[ま]つ 
\\	待[ま]たない
\\	弾く 
\\	弾かない	
\\	弾[ひ]く 
\\	弾[ひ]かない
\\	座る 
\\	座らない	
\\	座[すわ]る 
\\	座[すわ]らない
\\	帰る 
\\	帰らない	
\\	帰[かえ]る 
\\	帰[かえ]らない
\\	履く 
\\	履かない	
\\	履[は]く 
\\	履[は]かない
\\	居る 
\\	居ない	
\\	居[い]る 
\\	居[い]ない
\\	寝る 
\\	寝ない	
\\	寝[ね]る 
\\	寝[ね]ない
\\	始める 
\\	始めない	
\\	始[はじ]める 
\\	始[はじ]めない
\\	太る 
\\	太らない	
\\	太[ふと]る 
\\	太[ふと]らない
\\	勤める 
\\	勤めない	
\\	勤[つと]める 
\\	勤[つと]めない
\\	勉強する 
\\	勉強しない	
\\	勉[べん]強[きょう]する 
\\	勉[べん]強[きょう]しない
\\	切る 
\\	切らない	
\\	切[き]る 
\\	切[き]らない 
\\	着る)
\\	分かる 
\\	分からない	
\\	分[わ]かる 
\\	分[わ]からない
\\	出掛ける 
\\	出掛けない	
\\	出[で]掛[か]ける 
\\	出[で]掛[か]けない
\\	出る 
\\	出ない	
\\	出[で]る 
\\	出[で]ない
\\	入る 
\\	入らない	
\\	入[はい]る 
\\	入[はい]らない
\\	借りる 
\\	借りない	
\\	借[か]りる 
\\	借[か]りない
\\	使う 
\\	使わない	
\\	使[つか]う 
\\	使[つか]わない
\\	作る 
\\	作らない	
\\	作[つく]る 
\\	作[つく]らない
\\	住む 
\\	住まない	
\\	住[す]む 
\\	住[す]まない
\\	会う 
\\	会わない	
\\	会[あ]う 
\\	会[あ]わない
\\	休む 
\\	休まない	
\\	休[やす]む 
\\	休[やす]まない
\\	人気が有る 
\\	人気が無い	
\\	人[にん]気[き]が 有[あ]る 
\\	人[にん]気[き]が 無[な]い
\\	乗る 
\\	乗らない	
\\	乗[の]る 
\\	乗[の]らない
\\	やる 
\\	やらない	
\\	やる 
\\	やらない
\\	煙草を吸う 
\\	煙草を吸わない	
\\	煙草[たばこ]を 吸[す]う 
\\	煙草[たばこ]を 吸[す]わない
\\	する 
\\	しない	
\\	する 
\\	しない
\\	じろじろ見る 
\\	じろじろ見ない	
\\	じろじろ 見[み]る 
\\	じろじろ 見[み]ない
\\	明日試験が有ります。 スーさんは明日試験が有ると言っていました。	
\\	明日[あした]試[し]験[けん]が 有[あ]ります。 スーさんは 明日[あした]試[し]験[けん]が 有[あ]ると 言[い]っていました。
\\	たけしさんはメアリーさんが好きだと思います。	
\\	たけしさんはメアリーさんが 好[す]きだと 思[おも]います。
\\	メアリーさんはたけしさんが好きじゃないと思います。	
\\	メアリーさんはたけしさんが 好[す]きじゃないと 思[おも]います。
\\	此処で写真を撮らないで下さい。	
\\	此[こ]処[こ]で 写[しゃ]真[しん]を 撮[と]らないで 下[くだ]さい。
\\	私は日本語を勉強するのが好きです。	
\\	私[わたし]は 日[に]本[ほん]語[ご]を 勉[べん]強[きょう]するのが 好[す]きです。
\\	私は部屋を掃除するのが嫌いです。	
\\	私[わたし]は 部[へ]屋[や]を 掃[そう]除[じ]するのが 嫌[きら]いです。
\\	ロバートさんは料理を作るのが上手です。	
\\	ロバートさんは 料[りょう]理[り]を 作[つく]るのが 上[じょう]手[ず]です。
\\	たけしさんは英語を話すのが下手です。	
\\	たけしさんは 英[えい]語[ご]を 話[はな]すのが 下[へ]手[た]です。
\\	誰が沖縄に行きましたか。 ロバートさんが沖縄に行きました。	
\\	誰[だれ]が 沖[おき]縄[なわ]に 行[い]きましたか。 ロバートさんが 沖[おき]縄[なわ]に 行[い]きました。
\\	何のクラスが面白いですか。 日本語のクラスが面白いです。	
\\	何[ど]のクラスが 面[おも]白[しろ]いですか。 日[に]本[ほん]語[ご]のクラスが 面[おも]白[しろ]いです。
\\	(此のクラスで)誰が眼鏡を掛けていますか。 
\\	山下先生が眼鏡を掛けています。	
\\	此[こ]のクラスで) 誰[だれ]が 眼[め]鏡[がね]を 掛[か]けていますか。 
\\	山[やま]下[した]先[せん]生[せい]が 眼[め]鏡[がね]を 掛[か]けています。
\\	猫が何か持って来ました。	
\\	猫[ねこ]が 何[なに]か 持[も]って 来[き]ました。
\\	猫が何か食べましたか。	
\\	猫[ねこ]が 何[なに]か 食[た]べましたか。
\\	いいえ、猫が何も食べませんでした。	
\\	いいえ、 猫[ねこ]が 何[なに]も 食[た]べませんでした。
\\	勉強	
\\	勉[べん]強[きょう]	ベンキョー
\\	日本語の勉強は楽しいです。	
\\	日[に]本[ほん]語[ご]の 勉[べん]強[きょう]は 楽[たの]しいです。
\\	料理	
\\	料[りょう]理[り]	リョꜜーリ
\\	ロバートさんの料理は美味しいです。	
\\	ロバートさんの 料[りょう]理[り]は 美[お]味[い]しいです。
\\	高い 
\\	高くない	
\\	高[たか]い 
\\	高[たか]くない
\\	元気だ 
\\	元気じゃない	
\\	元[げん]気[き]だ 
\\	元[げん]気[き]じゃない
\\	学生だ 
\\	学生じゃない	
\\	学[がく]生[せい]だ 
\\	学[がく]生[せい]じゃない
\\	有名だ 
\\	有名じゃない	
\\	有[ゆう]名[めい]だ 
\\	有[ゆう]名[めい]じゃない
\\	雨だ 
\\	雨じゃない	
\\	雨[あめ]だ 
\\	雨[あめ]じゃない
\\	忙しい 
\\	忙しくない	
\\	忙[いそが]しい 
\\	忙[いそが]しくない
\\	可愛い 
\\	可愛くない	
\\	可愛[かわい]い 
\\	可愛[かわい]くない
\\	短い 
\\	短くない	
\\	短[みじか]い 
\\	短[みじか]くない
\\	親切だ 
\\	親切じゃない	
\\	親[しん]切[せつ]だ 
\\	親[しん]切[せつ]じゃない
\\	早い 
\\	早くない	
\\	早[はや]い 
\\	早[はや]くない 
\\	速い)
\\	綺麗だ 
\\	綺麗じゃない	
\\	綺[き]麗[れい]だ 
\\	綺[き]麗[れい]じゃない
\\	大変だ 
\\	大変じゃない	
\\	大[たい]変[へん]だ 
\\	大[たい]変[へん]じゃない
\\	良い 
\\	良くない	
\\	良[い]い 
\\	良[よ]くない
\\	格好いい 
\\	格好よくない	
\\	格[かっ]好[こ]いい 
\\	格[かっ]好[こ]よくない
\\	好きだ 
\\	好きじゃない	
\\	好[す]きだ 
\\	好[す]きじゃない
\\	嫌いだ 
\\	嫌いじゃない	
\\	嫌[きら]いだ 
\\	嫌[きら]いじゃない
\\	良く魚を食べる? 
\\	うん、食べる。 
\\	ううん、食べない。	
\\	良[よ]く 魚[さかな]を 食[た]べる? 
\\	うん、 食[た]べる。 
\\	ううん、 食[た]べない。
\\	今日、勉強する? 
\\	うん、勉強する。 
\\	ううん、勉強しない。	
\\	今日[きょう]、 勉[べん]強[きょう]する? 
\\	うん、 勉[べん]強[きょう]する。 
\\	ううん、 勉[べん]強[きょう]しない。
\\	良くお茶を飲む? 
\\	うん、飲む。 
\\	ううん、飲まない。	
\\	良[よ]くお 茶[ちゃ]を 飲[の]む? 
\\	うん、 飲[の]む。 
\\	ううん、 飲[の]まない。
\\	毎日、日本語を話す? 
\\	うん、話す。 
\\	ううん、話さない。	
\\	毎[まい]日[にち]、 日[に]本[ほん]語[ご]を 話[はな]す? 
\\	うん、 話[はな]す。 
\\	ううん、 話[はな]さない。
\\	今日友達に会う? 
\\	うん、会う。 
\\	ううん、会わない。	
\\	今[きょ]日[う]友[とも]達[だち]に 会[あ]う? 
\\	うん、 会[あ]う。 
\\	ううん、 会[あ]わない。
\\	良く電車に乗る? 
\\	うん、乗る。 
\\	ううん、乗らない。	
\\	良[よ]く 電[でん]車[しゃ]に 乗[の]る? 
\\	うん、 乗[の]る。 
\\	ううん、 乗[の]らない。
\\	毎日、テレビを見る? 
\\	うん、見る。 
\\	ううん、見ない。	
\\	毎[まい]日[にち]、テレビを 見[み]る? 
\\	うん、 見[み]る。 
\\	ううん、 見[み]ない。
\\	明日、大学に来る? 
\\	うん、来る。 
\\	ううん、来ない。	
\\	明日[あした]、 大[だい]学[がく]に 来[く]る? 
\\	うん、 来[く]る。 
\\	ううん、 来[こ]ない。
\\	今日、宿題が有る? 
\\	うん、有る。 
\\	ううん、無い。	
\\	今[きょ]日[う]、 宿[しゅく]題[だい]が 有[あ]る? 
\\	うん、 有[あ]る。 
\\	ううん、 無[な]い。
\\	自転車を持っている? 
\\	うん、持っている。 
\\	ううん、持っていない。	
\\	自[じ]転[てん]車[しゃ]を 持[も]っている? 
\\	うん、 持[も]っている。 
\\	ううん、 持[も]っていない。
\\	来週ディスコに行く? 
\\	うん、行く。 
\\	ううん、行かない。	
\\	来[らい]週[しゅう]ディスコに 行[い]く? 
\\	うん、 行[い]く。 
\\	ううん、 行[い]かない。
\\	毎週、部屋を掃除する? 
\\	うん、掃除する。 
\\	ううん、掃除しない。	
\\	毎[まい]週[しゅう]、 部[へ]屋[や]を 掃[そう]除[じ]する? 
\\	うん、 掃[そう]除[じ]する。 
\\	ううん、 掃[そう]除[じ]しない。
\\	毎日、洗濯する? 
\\	うん、洗濯する。 
\\	ううん、洗濯しない。	
\\	毎[まい]日[にち]、 洗[せん]濯[たく]する? 
\\	うん、 洗[せん]濯[たく]する。 
\\	ううん、 洗[せん]濯[たく]しない。
\\	暇? 
\\	うん、暇。 
\\	ううん、暇じゃない。	
\\	暇[ひま]? 
\\	うん、 暇[ひま]。 
\\	ううん、 暇[ひま]じゃない。
\\	元気? 
\\	うん、元気。 
\\	ううん、元気じゃない。	
\\	元[げん]気[き]? 
\\	うん、 元[げん]気[き]。 
\\	ううん、 元[げん]気[き]じゃない。
\\	忙しい? 
\\	うん、忙しい。 
\\	ううん、忙しくない。	
\\	忙[いそが]しい? 
\\	うん、 忙[いそが]しい。 
\\	ううん、 忙[いそが]しくない。
\\	日本語のクラスは面白い? 
\\	うん、面白い。 
\\	ううん、面白くない。	
\\	日[に]本[ほん]語[ご]のクラスは 面[おも]白[しろ]い? 
\\	うん、 面[おも]白[しろ]い。 
\\	ううん、 面[おも]白[しろ]くない。
\\	日本語のクラスは難しい? 
\\	うん、難しい。 
\\	ううん、難しくない。	
\\	日[に]本[ほん]語[ご]のクラスは 難[むずか]しい? 
\\	うん、 難[むずか]しい。 
\\	ううん、 難[むずか]しくない。
\\	此の教科書は良い? 
\\	うん、良い。 
\\	ううん、良くない。	
\\	此[こ]の 教[きょう]科[か]書[しょ]は 良[い]い? 
\\	うん、 良[い]い。 
\\	ううん、 良[よ]くない。
\\	先生は怖い? 
\\	うん、怖い。/ ううん、怖くない。	
\\	先[せん]生[せい]は 怖[こわ]い? 
\\	うん、 怖[こわ]い。 
\\	ううん、 怖[こわ]くない。
\\	料理が上手? 
\\	うん、上手。/ ううん、上手じゃない。	
\\	料[りょう]理[り]が 上[じょう]手[ず]? 
\\	うん、 上[じょう]手[ず]。 
\\	ううん、 上[じょう]手[ず]じゃない。
\\	お風呂が好き? 
\\	うん、好き。/ ううん、好きじゃない。	
\\	お 風[ふ]呂[ろ]が 好[す]き? 
\\	うん、 好[す]き。 
\\	ううん、 好[す]きじゃない。
\\	スポーツが嫌い? 
\\	うん、嫌い。/ ううん、嫌いじゃない。	
\\	スポーツが 嫌[きら]い? 
\\	うん、 嫌[きら]い。 
\\	ううん、 嫌[きら]いじゃない。
\\	今日は月曜日? 
\\	うん、月曜日。 
\\	ううん、月曜日じゃない。	
\\	今[きょ]日[う]は 月[げつ]曜[よう]日[び]? 
\\	うん、 月[げつ]曜[よう]日[び]。 
\\	ううん、 月[げつ]曜[よう]日[び]じゃない。
\\	日本人だと思います。	
\\	日[に]本[ほん]人[じん]だと 思[おも]います。
\\	メアリーさんは日本語が上手だと思います。	
\\	メアリーさんは 日[に]本[ほん]語[ご]が 上[じょう]手[ず]だと 思[おも]います。
\\	メアリーさんは良く料理すると思います。	
\\	メアリーさんは 良[よ]く 料[りょう]理[り]すると 思[おも]います。
\\	メアリーさんは車を運転すると思います。	
\\	メアリーさんは 車[くるま]を 運[うん]転[てん]すると 思[おも]います。
\\	メアリーさんは煙草を吸わないと思います。	
\\	メアリーさんは 煙草[たばこ]を 吸[す]わないと 思[おも]います。
\\	メアリーさんは毎日日本語のテープを聞くと思います。	
\\	メアリーさんは 毎[まい]日[にち]日[に]本[ほん]語[ご]のテープを 聞[き]くと 思[おも]います。
\\	メアリーさんは夜遅く家に帰らないと思います。	
\\	メアリーさんは 夜[よる]遅[おそ]く 家[いえ]に 帰[かえ]らないと 思[おも]います。
\\	メアリーさんは余りお酒を飲まないと思います。	
\\	メアリーさんは 余[あま]りお 酒[さけ]を 飲[の]まないと 思[おも]います。
\\	メアリーさんは良く映画を見に行くと思います。	
\\	メアリーさんは 良[よ]く 映[えい]画[が]を 見[み]に 行[い]くと 思[おも]います。
\\	メアリーさんは結婚していないと思います。	
\\	メアリーさんは 結[けっ]婚[こん]していないと 思[おも]います。
\\	メアリーさんはたけしさんが好きだと思います。	
\\	メアリーさんはたけしさんが 好[す]きだと 思[おも]います。
\\	メアリーさんは忙しいと思います。	
\\	メアリーさんは 忙[いそが]しいと 思[おも]います。
\\	メアリーさんは良い学生だと思います。	
\\	メアリーさんは 良[い]い 学[がく]生[せい]だと 思[おも]います。
\\	メアリーさんは背が高くないと思います。	
\\	メアリーさんは 背[せ]が 高[たか]くないと 思[おも]います。
\\	メアリーさんは怖くないと思います。	
\\	メアリーさんは 怖[こわ]くないと 思[おも]います。
\\	メアリーさんは静かじゃないと思います。	
\\	メアリーさんは 静[しず]かじゃないと 思[おも]います。
\\	メアリーさんは一年生じゃないと思います。	
\\	メアリーさんは 一[いち]年[ねん]生[せい]じゃないと 思[おも]います。
\\	此の人は日本人ですか。 
\\	ええ、日本人だと思います。 
\\	いいえ、日本人じゃないと思います。	
\\	此[こ]の 人[ひと]は 日[に]本[ほん]人[じん]ですか。 
\\	ええ、 日[に]本[ほん]人[じん]だと 思[おも]います。 
\\	いいえ、 日[に]本[ほん]人[じん]じゃないと 思[おも]います。
\\	此の人は会社員ですか? 
\\	ええ、会社員だと思います。 
\\	いいえ、会社員じゃないと思います。	
\\	此[こ]の 人[ひと]は 会[かい]社[しゃ] 員[いん]ですか? 
\\	ええ、 会[かい]社[しゃ] 員[いん]だと 思[おも]います。 
\\	いいえ、 会[かい]社[しゃ] 員[いん]じゃないと 思[おも]います。
\\	此の人は有名ですか? 
\\	ええ、有名だと思います。 
\\	いいえ、有名じゃないと思います。	
\\	此[こ]の 人[ひと]は 有[ゆう]名[めい]ですか? 
\\	ええ、 有[ゆう]名[めい]だと 思[おも]います。 
\\	いいえ、 有[ゆう]名[めい]じゃないと 思[おも]います。
\\	此の人は暇ですか? 
\\	ええ、暇だと思います。 
\\	いいえ、暇じゃないと思います。	
\\	此[こ]の 人[ひと]は 暇[ひま]ですか? 
\\	ええ、 暇[ひま]だと 思[おも]います。 
\\	いいえ、 暇[ひま]じゃないと 思[おも]います。
\\	此の人は頭が良いですか? 
\\	ええ、頭が良いと思います。 
\\	いいえ、頭が良くないと思います。	
\\	此[こ]の 人[ひと]は 頭[あたま]が 良[い]いですか? 
\\	ええ、 頭[あたま]が 良[い]いと 思[おも]います。 
\\	いいえ、 頭[あたま]が 良[よ]くないと 思[おも]います。
\\	此の人は背が高いですか? 
\\	ええ、背が高いと思います。 
\\	いいえ、背が高くないと思います。	
\\	此[こ]の 人[ひと]は 背[せ]が 高[たか]いですか? 
\\	ええ、 背[せ]が 高[たか]いと 思[おも]います。 
\\	いいえ、 背[せ]が 高[たか]くないと 思[おも]います。
\\	此の人は結婚していますか? 
\\	ええ、結婚していると思います。 
\\	いいえ、結婚していないと思います。	
\\	此[こ]の 人[ひと]は 結[けっ]婚[こん]していますか? 
\\	ええ、 結[けっ]婚[こん]していると 思[おも]います。 
\\	いいえ、 結[けっ]婚[こん]していないと 思[おも]います。
\\	此の人はお金を沢山持っていますか? 
\\	ええ、お金を沢山持っていると思います。 
\\	いいえ、お金を沢山持っていないと思います。	
\\	此[こ]の 人[ひと]はお 金[かね]を 沢[たく]山[さん]持[も]っていますか? 
\\	ええ、 お 金[かね]を 沢[たく]山[さん]持[も]っていると 思[おも]います。 
\\	いいえ、お 金[かね]を 沢[たく]山[さん]持[も]っていないと 思[おも]います。
\\	此の人は良く食べますか? 
\\	ええ、食べると思います。 
\\	いいえ、食べないと思います。	
\\	此[こ]の 人[ひと]は 良[よ]く 食[た]べますか? 
\\	ええ、 食[た]べると 思[おも]います。 
\\	いいえ、 食[た]べないと 思[おも]います。
\\	此処は日本ですか。 
\\	ええ、日本だと思います。 
\\	いいえ、日本じゃないと思います。	
\\	此[こ]処[こ]は 日[に]本[ほん]ですか。 
\\	ええ、 日[に]本[ほん]だと 思[おも]います。 
\\	いいえ、 日[に]本[ほん]じゃないと 思[おも]います。
\\	有名な所ですか。 
\\	ええ、有名だと思います。 
\\	いいえ、有名じゃないと思います。	
\\	有[ゆう]名[めい]な 所[ところ]ですか。 
\\	ええ、 有[ゆう]名[めい]だと 思[おも]います。 
\\	いいえ、 有[ゆう]名[めい]じゃないと 思[おも]います。
\\	空気は綺麗ですか。 
\\	ええ、綺麗だと思います。 
\\	いいえ、綺麗じゃないと思います。	
\\	空[くう]気[き]は 綺[き]麗[れい]ですか。 
\\	ええ、 綺[き]麗[れい]だと 思[おも]います。 
\\	いいえ、 綺[き]麗[れい]じゃないと 思[おも]います。
\\	暑いですか。 
\\	ええ、暑いと思います。 
\\	いいえ、暑くないと思います。	
\\	暑[あつ]いですか。 
\\	ええ、 暑[あつ]いと 思[おも]います。 
\\	いいえ、 暑[あつ]くないと 思[おも]います。
\\	冬は寒いですか。 
\\	ええ、寒いと思います。 
\\	いいえ、寒くないと思います。	
\\	冬[ふゆ]は 寒[さむ]いですか。 
\\	ええ、 寒[さむ]いと 思[おも]います。 
\\	いいえ、 寒[さむ]くないと 思[おも]います。
\\	人が沢山住んでいますか。 
\\	ええ、住んでいると思います。 
\\	いいえ、住んでいないと思います。	
\\	人[ひと]が 沢[たく]山[さん]住[す]んでいますか。 
\\	ええ、 住[す]んでいると 思[おも]います。 
\\	いいえ、 住[す]んでいないと 思[おも]います。
\\	良く仕事をしますか。 
\\	ええ、すると思います。 
\\	いいえ、しないと思います。	
\\	良[よ]く 仕[し]事[ごと]をしますか。 
\\	ええ、すると 思[おも]います。 
\\	いいえ、しないと 思[おも]います。
\\	夏に良く雨が降りますか。 
\\	ええ、雨が降ると思います。 
\\	いいえ、雨が降らないと思います。	
\\	夏[なつ]に 良[よ]く 雨[あめ]が 降[ふ]りますか。 
\\	ええ、 雨[あめ]が 降[ふ]ると 思[おも]います。 
\\	いいえ、 雨[あめ]が 降[ふ]らないと 思[おも]います。
\\	メアリーさんは忙しいと言っていました。	
\\	メアリーさんは 忙[いそが]しいと 言[い]っていました。
\\	メアリーさんは何と言っていましたか。	
\\	メアリーさんは 何[なん]と 言[い]っていましたか。
\\	来月は暇じゃないと言っていました。	
\\	来[らい]月[げつ]は 暇[ひま]じゃないと 言[い]っていました。
\\	明日は買い物をすると言っていました。	
\\	明日[あした]は 買[か]い 物[もの]をすると 言[い]っていました。
\\	京都に住んでいると言っていました。	
\\	京[きょう]都[と]に 住[す]んでいると 言[い]っていました。
\\	ロバートさんは何と言っていましたか。	
\\	ロバートさんは 何[なん]と 言[い]っていましたか。
\\	ホームステイをしていると言っていました。	
\\	ホームステイをしていると 言[い]っていました。
\\	お父さんは親切だと言っていました。	
\\	お 父[とう]さんは 親[しん]切[せつ]だと 言[い]っていました。
\\	お母さんは料理が上手だと言っていました。	
\\	お 母[かあ]さんは 料[りょう]理[り]が 上[じょう]手[ず]だと 言[い]っていました。
\\	お兄さんは大学生だと言っていました。	
\\	お 兄[にい]さんは 大[だい]学[がく]生[せい]だと 言[い]っていました。
\\	家族は英語を話さないと言っていました。	
\\	家[か]族[ぞく]は 英[えい]語[ご]を 話[はな]さないと 言[い]っていました。
\\	天気予報は何と言っていましたか。	
\\	天[てん]気[き] 予[よ]報[ほう]は 何[なん]と 言[い]っていましたか。
\\	明日は良い天気だと言っていました。	
\\	明日[あした]は 良[い]い 天[てん]気[き]だと 言[い]っていました。
\\	明日は寒くないと言っていました。	
\\	明日[あした]は 寒[さむ]くないと 言[い]っていました。
\\	明後日は雨が降ると言っていました。	
\\	明後日[あさって]は 雨[あめ]が 降[ふ]ると 言[い]っていました。
\\	明後日は寒いと言っていました。	
\\	明後日[あさって]は 寒[さむ]いと 言[い]っていました。
\\	食べないで下さい。	
\\	食[た]べないで 下[くだ]さい。
\\	写真を撮らないで下さい。	
\\	写[しゃ]真[しん]を 撮[と]らないで 下[くだ]さい。
\\	写真を見ないで下さい。	
\\	写[しゃ]真[しん]を 見[み]ないで 下[くだ]さい。
\\	英語を話さないで下さい。	
\\	英[えい]語[ご]を 話[はな]さないで 下[くだ]さい。
\\	電話を掛けないで下さい。	
\\	電[でん]話[わ]を 掛[か]けないで 下[くだ]さい。
\\	僕の家に来ないで下さい。	
\\	僕[ぼく]の 家[うち]に 来[こ]ないで 下[くだ]さい。
\\	行かないで下さい。	
\\	行[い]かないで 下[くだ]さい。
\\	煙草を吸わないで下さい。	
\\	煙草[たばこ]を 吸[す]わないで 下[くだ]さい。
\\	クラスで寝ないで下さい。	
\\	クラスで 寝[ね]ないで 下[くだ]さい。
\\	忘れないで下さい。	
\\	忘[わす]れないで 下[くだ]さい。
\\	じろじろ見ないで下さい。	
\\	じろじろ 見[み]ないで 下[くだ]さい。
\\	未だクラスを始めないで下さい。	
\\	未[ま]だクラスを 始[はじ]めないで 下[くだ]さい。
\\	遅く成らないで下さい。	
\\	遅[おそ]く 成[な]らないで 下[くだ]さい。
\\	未だ黒板を消さないで下さい。	
\\	未[ま]だ 黒[こく]板[ばん]を 消[け]さないで 下[くだ]さい。
\\	勉強するのが好きですか。	
\\	勉[べん]強[きょう]するのが 好[す]きですか。
\\	メアリーさんはフランス語が上手です。	
\\	メアリーさんはフランス 語[ご]が 上[じょう]手[ず]です。
\\	メアリーさんはテレビゲームが下手です。	
\\	メアリーさんはテレビゲームが 下[へ]手[た]です。
\\	メアリーさんは料理が上手です。	
\\	メアリーさんは 料[りょう]理[り]が 上[じょう]手[ず]です。
\\	寿司	
\\	寿[す]司[し]	
\\	メアリーさんは寿司を作るのが下手です。	
\\	メアリーさんは 寿[す]司[し]を 作[つく]るのが 下[へ]手[た]です。
\\	メアリーさんは日本語を話すのが上手です。	
\\	メアリーさんは 日[に]本[ほん]語[ご]を 話[はな]すのが 上[じょう]手[ず]です。
\\	メアリーさんは写真を撮るのが上手です。	
\\	メアリーさんは 写[しゃ]真[しん]を 撮[と]るのが 上[じょう]手[ず]です。
\\	メアリーさんは車を運転するのが上手です。	
\\	メアリーさんは 車[くるま]を 運[うん]転[てん]するのが 上[じょう]手[ず]です。
\\	メアリーさんは箸で食べるのが上手です。	
\\	メアリーさんは 箸[はし]で 食[た]べるのが 上[じょう]手[ず]です。
\\	ラブレター	
\\	ラブレター
\\	メアリーさんはラブレターを書くのが下手です。	
\\	メアリーさんはラブレターを 書[か]くのが 下[へ]手[た]です。
\\	食べるのが好きですか。	
\\	食[た]べるのが 好[す]きですか。
\\	寝るのが好きですか。	
\\	寝[ね]るのが 好[す]きですか。
\\	歌うのが好きですか。	
\\	歌[うた]うのが 好[す]きですか。
\\	買い物をするのが好きですか。	
\\	買[か]い 物[もの]をするのが 好[す]きですか。
\\	スポーツをするのが好きですか。	
\\	スポーツをするのが 好[す]きですか。
\\	日本語を勉強するのが好きですか。	
\\	日[に]本[ほん]語[ご]を 勉[べん]強[きょう]するのが 好[す]きですか。
\\	洗濯をするのが好きですか。	
\\	洗[せん]濯[たく]をするのが 好[す]きですか。
\\	掃除をするのが好きですか。	
\\	掃[そう]除[じ]をするのが 好[す]きですか。
\\	料理をするのが好きですか。	
\\	料[りょう]理[り]をするのが 好[す]きですか。
\\	お風呂に入るのが好きですか。	
\\	お 風[ふ]呂[ろ]に 入[はい]るのが 好[す]きですか。
\\	車を運転するのが好きですか。	
\\	車[くるま]を 運[うん]転[てん]するのが 好[す]きですか。
\\	車を掃除するのが好きですか。	
\\	車[くるま]を 掃[そう]除[じ]するのが 好[す]きですか。
\\	誰がイギリス人ですか。	
\\	誰[だれ]がイギリス 人[じん]ですか。
\\	ロバートさんがイギリス人です。	
\\	ロバートさんがイギリス 人[じん]です。
\\	誰が韓国人ですか。	
\\	誰[だれ]が 韓[かん]国[こく]人[じん]ですか。
\\	スーさんが韓国人です。	
\\	スーさんが 韓[かん]国[こく]人[じん]です。
\\	誰が料理をするのが上手ですか。	
\\	誰[だれ]が 料[りょう]理[り]をするのが 上[じょう]手[ず]ですか。
\\	誰が何時も食堂で食べますか。	
\\	誰[だれ]が 何[い]時[つ]も 食[しょく]堂[どう]で 食[た]べますか。
\\	誰がデートをしましたか。	
\\	誰[だれ]がデートをしましたか。
\\	誰が犬が好きですか。	
\\	誰[だれ]が 犬[いぬ]が 好[す]きですか。
\\	週末何もしませんでした。	
\\	週[しゅう]末[まつ]何[なに]もしませんでした。
\\	パーティーに行きましたが、何も食べませんでした。	
\\	パーティーに 行[い]きましたが、 何[なに]も 食[た]べませんでした。
\\	カラオケが有りましたが、何も歌いませんでした。	
\\	カラオケが 有[あ]りましたが、 何[なに]も 歌[うた]いませんでした。
\\	テレビが有りましたが、何も見ませんでした。	
\\	テレビが 有[あ]りましたが、 何[なに]も 見[み]ませんでした。
\\	カメラを持っていましたが、何も撮りませんでした。	
\\	カメラを 持[も]っていましたが、 何[なに]も 撮[と]りませんでした。
\\	ゆみさんに会いましたが、何も話しませんでした。	
\\	ゆみさんに 会[あ]いましたが、 何[なに]も 話[はな]しませんでした。
\\	昨日の晩御飯は何か作りましたか。 
\\	はい、スパゲティを作りました。 
\\	いいえ、何も作りませんでした。	
\\	昨[きの]日[う]の 晩[ばん]御[ご]飯[はん]は 何[なに]か 作[つく]りましたか。 
\\	はい、スパゲティを 作[つく]りました。 
\\	いいえ、 何[なに]も 作[つく]りませんでした。
\\	今朝、何か食べましたか。 
\\	はい、寿司を食べました。/ いいえ、何も食べませんでした。	
\\	今[け]朝[さ]、 何[なに]か 食[た]べましたか。 
\\	はい、 寿[す]司[し]を 食[た]べました。 
\\	いいえ、 何[なに]も 食[た]べませんでした。
\\	昨日、何か買いましたか。 
\\	はい、本を買いました。 
\\	いいえ、何も買いませんでした。	
\\	昨[きの]日[う]、 何[なに]か 買[か]いましたか。 
\\	はい、 本[ほん]を 買[か]いました。 
\\	いいえ、 何[なに]も 買[か]いませんでした。
\\	週末、何かしますか。 
\\	はい、ドイツ語を勉強します。 
\\	いいえ、何もしません。	
\\	週[しゅう]末[まつ]、 何[なに]かしますか。 
\\	はい、 ドイツ 語[ご]を 勉[べん]強[きょう]します。 
\\	いいえ、 何[なに]もしません。
\\	ドイツ語	
\\	ドイツ 語[ご]	ドイツコ°
\\	メアリーさんは歌舞伎が好きですか。	
\\	メアリーさんは 歌[か]舞[ぶ]伎[き]が 好[す]きですか。
\\	歌舞伎ですか。余り知りません。	
\\	歌[か]舞[ぶ]伎[き]ですか。 余[あま]り 知[し]りません。
\\	でもロバートさんは面白かったと言っていました。	
\\	でもロバートさんは 面[おも]白[しろ]かったと 言[い]っていました。
\\	歌舞伎の切符を二枚貰ったから、見に行きませんか。	
\\	歌[か]舞[ぶ]伎[き]の 切[きっ]符[ぷ]を 二[に]枚[まい]貰[もら]ったから、 見[み]に 行[い]きませんか。
\\	ええ、是非。何時ですか。	
\\	ええ、 是[ぜ]非[ひ]。 何[い]時[つ]ですか。
\\	木曜日です。十ニ時から四時迄です。	
\\	木[もく]曜[よう]日[び]です。 十[じゅう]二[に]時[じ]から 四[よ]時[じ]迄[まで]です。
\\	綺麗でしたね。	
\\	綺[き]麗[れい]でしたね。
\\	出ている人は皆男の人ですよ。	
\\	出[で]ている 人[ひと]は 皆[みんな]男[おとこ]の 人[ひと]ですよ。
\\	本当ですか。	
\\	本[ほん]当[とう]ですか。	
\\	所でもう昼御飯を食べましたか。	
\\	所[ところ]でもう 昼[ひる] 御[ご]飯[はん]を 食[た]べましたか。
\\	いいえ、未だ食べていません。	
\\	いいえ、 未[ま]だ 食[た]べていません。
\\	じゃあ、買いに行きましょう。	
\\	じゃあ、 買[か]いに 行[い]きましょう。
\\	済みません、お弁当を二つ下さい。	
\\	済[す]みません、 お 弁[べん]当[とう]を 二[ふた]つ 下[くだ]さい。
\\	それから、お茶を一つとコーヒーを一つ。	
\\	それから、 お 茶[ちゃ]を 一[ひと]つとコーヒーを 一[ひと]つ。
\\	二千八百円です。どうも有難う御座いました。	
\\	二[に]千[せん]八[はっ]百[ぴゃく]円[えん]です。どうも 有[あり]難[がと]う 御[ご]座[ざ]いました。
\\	読んだ 
\\	読まなかった	
\\	読[よ]んだ 
\\	読[よ]まなかった
\\	じろじろ見た 
\\	じろじろ見なかった	
\\	じろじろ 見[み]た 
\\	じろじろ 見[み]なかった
\\	した 
\\	しなかった	
\\	した 
\\	しなかった
\\	煙草を吸った 
\\	煙草を吸わなかった	
\\	煙草[たばこ]を 吸[す]った 
\\	煙草[たばこ]を 吸[す]わなかった
\\	やった 
\\	やらなかった	
\\	やった 
\\	やらなかった
\\	乗った 
\\	乗らなかった	
\\	乗[の]った 
\\	乗[の]らなかった
\\	人気が有った 
\\	人気がなかった	
\\	人[にん]気[き]が 有[あ]った 
\\	人[にん]気[き]がなかった
\\	休んだ 
\\	休まなかった	
\\	休[やす]んだ 
\\	休[やす]まなかった
\\	会った 
\\	会わなかった	
\\	会[あ]った 
\\	会[あ]わなかった
\\	住んだ 
\\	住まなかった	
\\	住[す]んだ 
\\	住[す]まなかった
\\	作った 
\\	作らなかった	
\\	作[つく]った 
\\	作[つく]らなかった
\\	使った 
\\	使わなかった	
\\	使[つか]った 
\\	使[つか]わなかった
\\	借りた 
\\	借りなかった	
\\	借[か]りた 
\\	借[か]りなかった
\\	入った 
\\	入らなかった	
\\	入[はい]った 
\\	入[はい]らなかった
\\	出た 
\\	出なかった	
\\	出[で]た 
\\	出[で]なかった
\\	出掛けた 
\\	出掛けなかった	
\\	出[で]掛[か]けた 
\\	出[で]掛[か]けなかった
\\	分かった 
\\	分からなかった	
\\	分[わ]かった 
\\	分[わ]からなかった
\\	切った 
\\	切らなかった	
\\	切[き]った 
\\	切[き]らなかった
\\	勉強した 
\\	勉強しなかった	
\\	勉[べん]強[きょう]した 
\\	勉[べん]強[きょう]しなかった
\\	勤めた 
\\	勤めなかった	
\\	勤[つと]めた 
\\	勤[つと]めなかった
\\	太った 
\\	太らなかった	
\\	太[ふと]った 
\\	太[ふと]らなかった
\\	始めた 
\\	始めなかった	
\\	始[はじ]めた 
\\	始[はじ]めなかった
\\	寝た 
\\	寝なかった	
\\	寝[ね]た 
\\	寝[ね]なかった
\\	居た 
\\	居なかった	
\\	居[い]た 
\\	居[い]なかった
\\	履いた 
\\	履かなかった	
\\	履[は]いた 
\\	履[は]かなかった
\\	帰った 
\\	帰らなかった	
\\	帰[かえ]った 
\\	帰[かえ]らなかった
\\	座った 
\\	座らなかった	
\\	座[すわ]った 
\\	座[すわ]らなかった
\\	忘れた 
\\	忘れなかった	
\\	忘[わす]れた 
\\	忘[わす]れなかった
\\	思った 
\\	思わなかった	
\\	思[おも]った 
\\	思[おも]わなかった
\\	急いだ 
\\	急がなかった	
\\	急[いそ]いだ 
\\	急[いそ]がなかった
\\	成った 
\\	成らなかった	
\\	成[な]った 
\\	成[な]らなかった
\\	成る 
\\	成らない	
\\	成[な]る 
\\	成[な]らない
\\	手伝った 
\\	手伝わなかった	
\\	手[て]伝[つだ]った 
\\	手[て]伝[つだ]わなかった
\\	払った 
\\	払わなかった	
\\	払[はら]った 
\\	払[はら]わなかった
\\	払う 
\\	払わない	
\\	払[はら]う 
\\	払[はら]わない
\\	持って来た 
\\	持って来なかった	
\\	持[も]って 来[き]た 
\\	持[も]って 来[こ]なかった
\\	持った 
\\	持たなかった	
\\	持[も]った 
\\	持[も]たなかった
\\	掃除した 
\\	掃除しなかった	
\\	掃[そう]除[じ]した 
\\	掃[そう]除[じ]しなかった
\\	掛かった 
\\	掛からなかった	
\\	掛[か]かった 
\\	掛[か]からなかった
\\	撮った 
\\	撮らなかった	
\\	撮[と]った 
\\	撮[と]らなかった
\\	教えた 
\\	教えなかった	
\\	教[おし]えた 
\\	教[おし]えなかった
\\	料理した 
\\	料理しなかった	
\\	料[りょう]理[り]した 
\\	料[りょう]理[り]しなかった
\\	曲がった 
\\	曲がらなかった	
\\	曲[ま]がった 
\\	曲[ま]がらなかった
\\	書いた 
\\	書かなかった	
\\	書[か]いた 
\\	書[か]かなかった
\\	有った 
\\	なかった	
\\	有[あ]った 
\\	なかった
\\	来た 
\\	来なかった	
\\	来[き]た 
\\	来[こ]なかった
\\	歌った 
\\	歌わなかった	
\\	歌[うた]った 
\\	歌[うた]わなかった
\\	死んだ 
\\	死ななかった	
\\	死[し]んだ 
\\	死[し]ななかった
\\	決めた 
\\	決めなかった	
\\	決[き]めた 
\\	決[き]めなかった
\\	泊まった 
\\	泊まらなかった	
\\	泊[と]まった 
\\	泊[と]まらなかった
\\	泳いだ 
\\	泳がなかった	
\\	泳[およ]いだ 
\\	泳[およ]がなかった
\\	洗った 
\\	洗わなかった	
\\	洗[あら]った 
\\	洗[あら]わなかった
\\	洗濯した 
\\	洗濯しなかった	
\\	洗[せん]濯[たく]した 
\\	洗[せん]濯[たく]しなかった
\\	消した 
\\	消さなかった	
\\	消[け]した 
\\	消[け]さなかった
\\	渡った 
\\	渡らなかった	
\\	渡[わた]った 
\\	渡[わた]らなかった
\\	点けた 
\\	点けなかった	
\\	点[つ]けた 
\\	点[つ]けなかった
\\	痩せた 
\\	痩せなかった	
\\	痩[や]せた 
\\	痩[や]せなかった
\\	眼鏡を掛けた 
\\	眼鏡を掛けなかった	
\\	眼[め]鏡[がね]を 掛[か]けた 
\\	眼[め]鏡[がね]を 掛[か]けなかった
\\	着た 
\\	着なかった	
\\	着[き]た 
\\	着[き]なかった
\\	知った 
\\	知らなかった	
\\	知[し]った 
\\	知[し]らなかった
\\	立った 
\\	立たなかった	
\\	立[た]った 
\\	立[た]たなかった
\\	終わった 
\\	終わらなかった	
\\	終[お]わった 
\\	終[お]わらなかった
\\	結婚した 
\\	結婚しなかった	
\\	結[けっ]婚[こん]した 
\\	結[けっ]婚[こん]しなかった
\\	練習した 
\\	練習しなかった	
\\	練[れん]習[しゅう]した 
\\	練[れん]習[しゅう]しなかった
\\	聞いた 
\\	聞かなかった	
\\	聞[き]いた 
\\	聞[き]かなかった
\\	薬を飲んだ 
\\	薬を飲まなかった	
\\	薬[くすり]を 飲[の]んだ 
\\	薬[くすり]を 飲[の]まなかった
\\	行った 
\\	行かなかった	
\\	行[い]った 
\\	行[い]かなかった
\\	被った 
\\	被らなかった	
\\	被[かぶ]った 
\\	被[かぶ]らなかった
\\	要った 
\\	要らなかった	
\\	要[い]った 
\\	要[い]らなかった
\\	見た 
\\	見なかった	
\\	見[み]た 
\\	見[み]なかった
\\	覚えた 
\\	覚えなかった	
\\	覚[おぼ]えた 
\\	覚[おぼ]えなかった
\\	言った 
\\	言わなかった	
\\	言[い]った 
\\	言[い]わなかった
\\	話した 
\\	話さなかった	
\\	話[はな]した 
\\	話[はな]さなかった
\\	貰った 
\\	貰わなかった	
\\	貰[もら]った 
\\	貰[もら]わなかった
\\	買った 
\\	買わなかった	
\\	買[か]った 
\\	買[か]わなかった
\\	起きた 
\\	起きなかった	
\\	起[お]きた 
\\	起[お]きなかった
\\	踊った 
\\	踊らなかった	
\\	踊[おど]った 
\\	踊[おど]らなかった
\\	返した 
\\	返さなかった	
\\	返[かえ]した 
\\	返[かえ]さなかった
\\	連れて来た 
\\	連れて来なかった	
\\	連[つ]れて 来[き]た 
\\	連[つ]れて 来[こ]なかった
\\	遅く成った 
\\	遅く成らなかった	
\\	遅[おそ]く 成[な]った 
\\	遅[おそ]く 成[な]らなかった
\\	遊んだ 
\\	遊ばなかった	
\\	遊[あそ]んだ 
\\	遊[あそ]ばなかった
\\	運動した 
\\	運動しなかった	
\\	運[うん]動[どう]した 
\\	運[うん]動[どう]しなかった
\\	運転した 
\\	運転しなかった	
\\	運[うん]転[てん]した 
\\	運[うん]転[てん]しなかった
\\	閉めた 
\\	閉めなかった	
\\	閉[し]めた 
\\	閉[し]めなかった
\\	開けた 
\\	開けなかった	
\\	開[あ]けた 
\\	開[あ]けなかった
\\	降りた 
\\	降りなかった	
\\	降[お]りた 
\\	降[お]りなかった
\\	雨が降った 
\\	雨が降らなかった	
\\	雨[あめ]が 降[ふ]った 
\\	雨[あめ]が 降[ふ]らなかった
\\	電話を掛けた 
\\	電話を掛けなかった	
\\	電[でん]話[わ]を 掛[か]けた 
\\	電[でん]話[わ]を 掛[か]けなかった
\\	食べた 
\\	食べなかった	
\\	食[た]べた 
\\	食[た]べなかった
\\	飲んだ 
\\	飲まなかった	
\\	飲[の]んだ 
\\	飲[の]まなかった
\\	メアリーさんは、高校の時眼鏡を掛けたと言っていました。	
\\	メアリーさんは、 高[こう]校[こう]の 時[とき]眼[め]鏡[がね]を 掛[か]けたと 言[い]っていました。
\\	(私は)トムさんがやったと思います。	
\\	私[わたし]は)トムさんがやったと 思[おも]います。
\\	晩御飯、食べた? 
\\	うん、食べた。	
\\	晩[ばん]御[ご]飯[はん]、 食[た]べた? 
\\	うん、 食[た]べた。
\\	スーさんは日本語を勉強していると言っていました。	
\\	スーさんは 日[に]本[ほん]語[ご]を 勉[べん]強[きょう]していると 言[い]っていました。
\\	彼処で本を読んでいる学生は美智子さんです。	
\\	彼[あそ]処[こ]で 本[ほん]を 読[よ]んでいる学[がく]生[せい]は 美[み]智[ち]子[こ]さんです。
\\	面白い人	
\\	面[おも]白[しろ]い 人[ひと]
\\	髪が長い人	
\\	髪[かみ]が 長[なが]い 人[ひと]
\\	眼鏡を掛けている人	
\\	眼[め]鏡[がね]を 掛[か]けている 人[ひと]
\\	猫が好きな人	
\\	猫[ねこ]が 好[す]きな 人[ひと]
\\	彼処で写真を撮っている人 (は誰ですか。)	
\\	彼[あそ]処[こ]で 写[しゃ]真[しん]を 撮[と]っている 人[ひと] (は 誰[だれ]ですか。)
\\	毎日運動する人(は元気です)。	
\\	毎[まい]日[にち]運[うん]動[どう]する 人[ひと](は 元[げん]気[き]です)。
\\	煙草を吸わない人(が好きです)。	
\\	煙草[たばこ]を 吸[す]わない 人[ひと](が 好[す]きです)。
\\	去年結婚した友達(から手紙が来ました。)	
\\	去[きょ]年[ねん]結[けっ]婚[こん]した 友[とも]達[だち](から 手[て]紙[がみ]が 来[き]ました。)
\\	私は昨日宿題をしました。	
\\	私[わたし]は 昨[きの]日[う]宿[しゅく]題[だい]をしました。
\\	私はもう宿題をしました。	
\\	私[わたし]はもう 宿[しゅく]題[だい]をしました。
\\	私は昨日宿題をしませんでした。	
\\	私[わたし]は 昨[きの]日[う]宿[しゅく]題[だい]をしませんでした。
\\	私は未だ宿題をしていません。	
\\	私[わたし]は 未[ま]だ 宿[しゅく]題[だい]をしていません。
\\	未だ 
\\	ていません	
\\	未[ま]だ... ていません
\\	スーさんは未だ起きていません。 (ている 
\\	スーさんは 未[ま]だ 起[お]きていません。 (ている 
\\	私は未だ昼御飯を食べていません。 (ている 
\\	私[わたし]は 未[ま]だ 昼[ひる] 御[ご]飯[はん]を 食[た]べていません。 (ている 
\\	私は朝御飯を食べませんでした。忙しかったですから。	
\\	私[わたし]は 朝[あさ]御[ご]飯[はん]を 食[た]べませんでした。 忙[いそが]しかったですから。
\\	明日試験が有るから、私は今晩勉強します。	
\\	明日[あした]試[し]験[けん]が 有[あ]るから、 私[わたし]は 今[こん]晩[ばん]勉[べん]強[きょう]します。
\\	寒かったから、出掛けませんでした。	
\\	寒[さむ]かったから、 出[で]掛[か]けませんでした。
\\	歌舞伎の切符が有りますから、一緒に見に行きましょう。	
\\	歌[か]舞[ぶ]伎[き]の 切[きっ]符[ぷ]が 有[あ]りますから、 一[いっ]緒[しょ]に 見[み]に 行[い]きましょう。
\\	十課	
\\	十[じゅっ]課[か]	
\\	赤が一番好きです。	
\\	赤[あか]が 一[いち]番[ばん]好[す]きです。
\\	緑のセーター	
\\	緑[みどり]のセーター
\\	顔が青いですね。	
\\	顔[かお]が 青[あお]いですね。
\\	白黒の写真	
\\	白[しろ]黒[くろ]の 写[しゃ]真[しん]	
\\	メアリーさんは金髪です。	
\\	メアリーさんは 金[きん]髪[ぱつ]です。
\\	冬休みの予定	
\\	冬[ふゆ]休[やす]みの 予[よ]定[てい]	
\\	寒く成りましたね。	
\\	寒[さむ]く 成[な]りましたね。
\\	冬休みはどうしますか。	
\\	冬[ふゆ]休[やす]みはどうしますか。
\\	冬休み	
\\	冬[ふゆ]休[やす]み	フユヤꜜスミ
\\	韓国か台湾に行くつもりですが、未だ決めていません。	
\\	韓[かん]国[こく]か 台[たい]湾[わん]に 行[い]くつもりですが、 未[ま]だ 決[き]めていません。
\\	台湾	
\\	台[たい]湾[わん]	
\\	韓国と台湾とどっちの方が良いと思いますか。	
\\	韓[かん]国[こく]と 台[たい]湾[わん]とどっちの 方[ほう]が 良[い]いと 思[おも]いますか。
\\	ううん、台湾の方が暖かいと思います。	
\\	ううん、 台[たい]湾[わん]の 方[ほう]が 暖[あたた]かいと 思[おも]います。
\\	でも、スーさんは韓国の食べ物は美味しいと言っていましたよ。	
\\	でも、スーさんは 韓[かん]国[こく]の 食[た]べ 物[もの]は 美[お]味[い]しいと 言[い]っていましたよ。
\\	所で、たけしさんは何処かに行きますか。	
\\	所[ところ]で、たけしさんは 何[ど]処[こ]かに 行[い]きますか。
\\	何処にも行きません。お金が無いから此処に居ます。	
\\	何[ど]処[こ]にも 行[い]きません。お 金[かね]が 無[な]いから 此[こ]処[こ]に 居[い]ます。
\\	じゃあ、たけしさんにお土産を買って来ますよ。	
\\	じゃあ、たけしさんにお 土産[みやげ]を 買[か]って 来[き]ますよ。
\\	大阪からソウル迄飛行機の予約をお願いします。	
\\	大[おお]阪[さか]からソウル 迄[まで]飛[ひ]行[こう]機[き]の 予[よ]約[やく]をお 願[ねが]いします。
\\	十二月十九日です。	
\\	十[じゅう]二[に]月[がつ]十[じゅう]九[く]日[にち]です。
\\	午前と午後の便が有りますが・・・	
\\	午[ご]前[ぜん]と 午[ご]後[ご]の 便[びん]が 有[あ]りますが・・・
\\	クレジットカードで払ってもいいですか。	
\\	クレジットカードで 払[はら]ってもいいですか。
\\	ソウル迄何の位掛かりますか。	
\\	ソウル 迄[まで]何[ど]の 位[ぐらい]掛[か]かりますか。
\\	一時間位です。	
\\	一[いち]時[じ]間[かん]位[ぐらい]です。
\\	山田さんの方が山口さんより格好いいです。	
\\	山[やま]田[だ]さんの 方[ほう]が 山[やま]口[ぐち]さんより 格[かっ]好[こ]いいです。
\\	スーさんの方がメアリーさんより歌が上手です。	
\\	スーさんの 方[ほう]がメアリーさんより 歌[うた]が 上[じょう]手[ず]です。
\\	バスと電車とどっちの方が安いですか。	
\\	バスと 電[でん]車[しゃ]とどっちの 方[ほう]が 安[やす]いですか。
\\	スーさんとメアリーさんとロバートさんの中で、誰が一番歌が上手だと思いますか。	
\\	スーさんとメアリーさんとロバートさんの 中[なか]で、 誰[だれ]が 一[いち]番[ばん]歌[うた]が 上[じょう]手[ず]だと 思[おも]いますか。
\\	勿論、ロバートさんが一番歌が上手です。	
\\	勿[もち]論[ろん]、ロバートさんが 一[いち]番[ばん]歌[うた]が 上[じょう]手[ず]です。
\\	私は黒いセーターを持っています。赤いのも持っています。	
\\	私[わたし]は 黒[くろ]いセーターを 持[も]っています。 赤[あか]いのも 持[も]っています。
\\	安い辞書を買いに行きました。でも良いのが有りませんでした。	
\\	安[やす]い 辞[じ]書[しょ]を 買[か]いに 行[い]きました。でも 良[い]いのが 有[あ]りませんでした。
\\	此れはスーさんの鞄ですか。 
\\	いいえ、其れはメアリーさんの___です。	
\\	此[こ]れはスーさんの 鞄[かばん]ですか。 
\\	いいえ、 其[そ]れはメアリーさんの___です。
\\	アメリカのアイスクリームの方が日本の___より美味しいです。	
\\	アメリカのアイスクリームの 方[ほう]が 日[に]本[ほん]の___より 美[お]味[い]しいです。
\\	(私は)週末たけしさんとテニスをするつもりです。	
\\	私[わたし]は) 週[しゅう]末[まつ]たけしさんとテニスをするつもりです。
\\	山下先生は明日大学に来ないつもりです。	
\\	山[やま]下[した]先[せん]生[せい]は 明日[あした]大[だい]学[がく]に 来[こ]ないつもりです。
\\	お寺を見に行くつもりでしたけど、天気が良くなかったから、行きませんでした。	
\\	お 寺[てら]を 見[み]に 行[い]くつもりでしたけど、 天[てん]気[き]が 良[よ]くなかったから、 行[い]きませんでした。
\\	暖かく成る	
\\	暖[あたた]かく 成[な]る
\\	静かに成る	
\\	静[しず]かに 成[な]る
\\	会社員に成る	
\\	会[かい]社[しゃ] 員[いん]に 成[な]る
\\	日本語の勉強が楽しく成りました。	
\\	日[に]本[ほん]語[ご]の 勉[べん]強[きょう]が 楽[たの]しく 成[な]りました。
\\	日本語の勉強が好きに成りました。	
\\	日[に]本[ほん]語[ご]の 勉[べん]強[きょう]が 好[す]きに 成[な]りました。
\\	メアリーさんは前より日本語が上手に成りました。	
\\	メアリーさんは 前[まえ]より 日[に]本[ほん]語[ご]が 上[じょう]手[ず]に 成[な]りました。
\\	何か___ 何___も	
\\	何[なに]か___ 何[なに]___も
\\	誰か___ 誰___も	
\\	誰[だれ]か___ 誰[だれ]___も
\\	何処か___ 何処___も	
\\	何[ど]処[こ]か___ 何[ど]処[こ]___も
\\	何処かに行きましたか。 
\\	いいえ、何処にも行きませんでした。	
\\	何[ど]処[こ]かに 行[い]きましたか。 
\\	いいえ、 何[ど]処[こ]にも 行[い]きませんでした。
\\	誰かに会いましたか。 
\\	いいえ、誰にも会いませんでした。	
\\	誰[だれ]かに 会[あ]いましたか。 
\\	いいえ、 誰[だれ]にも 会[あ]いませんでした。
\\	何かしましたか。 
\\	いいえ、何もしませんでした。	
\\	何[なに]かしましたか。 
\\	いいえ、 何[なに]もしませんでした。
\\	箸で御飯を食べます。	
\\	箸[はし]で 御[ご]飯[はん]を 食[た]べます。
\\	日本語で話しましょう。	
\\	日[に]本[ほん]語[ご]で 話[はな]しましょう。
\\	バスで駅迄行きました。	
\\	バスで 駅[えき]迄[まで]行[い]きました。
\\	テレビで映画を見ました。	
\\	テレビで 映[えい]画[が]を 見[み]ました。
\\	メアリーさん、久しぶりですね。休みはどうでしたか。	
\\	メアリーさん、 久[ひさ]しぶりですね。 休[やす]みはどうでしたか。
\\	凄く楽しかったです。	
\\	凄[すご]く 楽[たの]しかったです。
\\	韓国で買い物をしたり、韓国料理を食べたりしました。	
\\	韓[かん]国[こく]で 買[か]い 物[もの]をしたり、 韓[かん]国[こく]料[りょう]理[り]を 食[た]べたりしました。
\\	私も旅行したいです。	
\\	私[わたし]も 旅[りょ]行[こう]したいです。
\\	美智子さんの休みは楽しかったですか。	
\\	美[み]智[ち]子[こ]さんの 休[やす]みは 楽[たの]しかったですか。
\\	まあまあでした。	
\\	まあまあでした。
\\	一日丈ドライブに行きましたが、毎日アルバイトをしていました。	
\\	一[いち]日[にち]丈[だけ]ドライブに 行[い]きましたが、 毎[まい]日[にち]アルバイトをしていました。
\\	美智子さん、友達を紹介します。	
\\	美[み]智[ち]子[こ]さん、 友[とも]達[だち]を 紹[しょう]介[かい]します。
\\	此方はジョンさんです。ジョンさんは先月日本に来ました。	
\\	此[こ]方[ちら]はジョンさんです。ジョンさんは 先[せん]月[げつ]日[に]本[ほん]に 来[き]ました。
\\	初めまして。	
\\	初[はじ]めまして。
\\	初めまして。どうぞ宜しく。	
\\	初[はじ]めまして。どうぞ 宜[よろ]しく。
\\	ジョンさんは出身は何処ですか。	
\\	ジョンさんは 出[しゅっ]身[しん]は 何[ど]処[こ]ですか。
\\	オーストラリアです。	
\\	オーストラリアです。
\\	美智子さんはオーストラリアに行ったことが有りますか。	
\\	美[み]智[ち]子[こ]さんはオーストラリアに 行[い]ったことが 有[あ]りますか。
\\	山や海が有って、綺麗な所ですよ。	
\\	山[やま]や 海[うみ]が 有[あ]って、 綺[き]麗[れい]な 所[ところ]ですよ。
\\	グレートバリアリーフで有名です。美智子さんは何処の出身ですか。	
\\	グレートバリアリーフで 有[ゆう]名[めい]です。 美[み]智[ち]子[こ]さんは 何[ど]処[こ]の 出[しゅっ]身[しん]ですか。
\\	長野です。今度遊びに来て下さい。食べ物も美味しいですよ。	
\\	長[なが]野[の]です。 今[こん]度[ど]遊[あそ]びに 来[き]て 下[くだ]さい。 食[た]べ 物[もの]も 美[お]味[い]しいですよ。
\\	是非、行きたいです。	
\\	是[ぜ]非[ひ]、 行[い]きたいです。
\\	今度の週末は、映画を見たいです。 
\\	映画が見たいです。	
\\	今[こん]度[ど]の 週[しゅう]末[まつ]は、 映[えい]画[が]を 見[み]たいです。 
\\	映[えい]画[が]が 見[み]たいです。
\\	何時か	
\\	何[い]時[つ]か 
\\	五日)	
\\	何時か中国に行きたいです。	
\\	何[い]時[つ]か 中[ちゅう]国[ごく]に 行[い]きたいです。
\\	たいです	
\\	たいです
\\	彼の人に会いたくありません。	
\\	彼[あ]の 人[ひと]に 会[あ]いたくありません。
\\	セーターが買いたかったから、デパートに行きました。	
\\	セーターが 買[か]いたかったから、 デパートに 行[い]きました。
\\	メアリーさんはトイレに行きたいと言っていました。	
\\	メアリーさんはトイレに 行[い]きたいと 言[い]っていました。
\\	メアリーさんはコーヒーを飲みたがっています。	
\\	メアリーさんはコーヒーを 飲[の]みたがっています。
\\	韓国料理	
\\	韓[かん]国[こく]料[りょう]理[り]	
\\	中国料理	
\\	中[ちゅう]国[ごく]料[りょう]理[り]
\\	大阪で買い物をしたり、韓国料理を食べたりします。	
\\	大[おお]阪[さか]で 買[か]い 物[もの]をしたり、 韓[かん]国[こく]料[りょう]理[り]を 食[た]べたりします。
\\	大阪で買い物をして、韓国料理を食べます。	
\\	大[おお]阪[さか]で 買[か]い 物[もの]をして、 韓[かん]国[こく]料[りょう]理[り]を 食[た]べます。
\\	ヨーロッパに行ったことが有りますか。 
\\	はい、行ったことが有ります。	
\\	ヨーロッパに 行[い]ったことが 有[あ]りますか。 
\\	はい、 行[い]ったことが 有[あ]ります。
\\	たけしさんは授業を休んだことが有りません。	
\\	たけしさんは 授[じゅ]業[ぎょう]を 休[やす]んだことが 有[あ]りません。
\\	ことがある	
\\	ことがある
\\	京都や奈良に行きました。	
\\	京[きょう]都[と]や 奈[な]良[ら]に 行[い]きました。
\\	奈良	
\\	奈[な]良[ら]	ナꜜラ
\\	や
\\	や
\\	ヨーロッパ	
\\	ヨーロッパ
\\	メアリーさん、元気が有りませんね。	
\\	メアリーさん、 元[げん]気[き]が 有[あ]りませんね。
\\	うーん、一寸お腹が痛いんです。	
\\	うーん、 一[ちょっ]寸[と]お 腹[なか]が 痛[いた]いんです。
\\	如何したんですか。	
\\	如[ど]何[う]したんですか。
\\	昨日、友達と晩御飯を食べに行ったんです。	
\\	昨[きの]日[う]、 友[とも]達[だち]と 晩[ばん]御[ご]飯[はん]を 食[た]べに 行[い]ったんです。
\\	多分食べ過ぎたんだと思います。	
\\	多[た]分[ぶん]食[た]べ 過[す]ぎたんだと 思[おも]います。
\\	大丈夫ですか。	
\\	大[だい]丈[じょう]夫[ぶ]ですか。
\\	ええ心配しないで下さい。・・・ああ、痛い。	
\\	ええ 心[しん]配[ぱい]しないで 下[くだ]さい。・・・ああ、 痛[いた]い。
\\	病院に行った方が良いですよ。	
\\	病[びょう]院[いん]に 行[い]った 方[ほう]が 良[い]いですよ。
\\	先生、喉が痛いんです。	
\\	先[せん]生[せい]、 喉[のど]が 痛[いた]いんです。
\\	昨日はお腹が痛かったんです。	
\\	昨[きの]日[う]はお 腹[なか]が 痛[いた]かったんです。
\\	熱も有りますね。風邪ですね。	
\\	熱[ねつ]も 有[あ]りますね。 風邪[かぜ]ですね。
\\	あの、もう 直ぐテニスの試合があるので練習しなくちゃ行けないんですが・・・	
\\	あの、 もう 直[す]ぐテニスの 試[し]合[あい]があるので 練[れん]習[しゅう]しなくちゃ 行[い]けないんですが・・・
\\	二三日、運動しない方がいいでしょう。	
\\	二[に]三[さん]日[にち]、 運[うん]動[どう]しない 方[ほう]がいいでしょう。
\\	今日は薬を飲んで、早く寝て下さい。	
\\	今[きょ]日[う]は 薬[くすり]を 飲[の]んで、 早[はや]く 寝[ね]て 下[くだ]さい。
\\	風邪	
\\	風邪[かぜ] 
\\	カゼ
\\	バスが来なかったんです。	
\\	バスが 来[こ]なかったんです。
\\	明日試験が有ります。 明日試験が有るんです。	
\\	明日[あした]試[し]験[けん]が 有[あ]ります。 明日[あした]試[し]験[けん]が 有[あ]るんです。
\\	トイレ	
\\	トイレ	トꜜイレ
\\	トイレに行きたいです。 トイレに行きたいんです。	
\\	トイレに 行[い]きたいです。 トイレに 行[い]きたいんです。
\\	成績が良くないんです。	
\\	成[せい]績[せき]が 良[よ]くないんです。
\\	試験が終わったんです。	
\\	試[し]験[けん]が 終[お]わったんです。
\\	学生なんです	
\\	学[がく]生[せい]なんです
\\	静かなんです	
\\	静[しず]かなんです
\\	如何した	
\\	如[ど]何[う]した
\\	如何して彼を別れたんですか。 
\\	彼、全然お風呂に入らないんです。	
\\	如[ど]何[う]して 彼[かれ]を 別[わか]れたんですか。 
\\	彼[かれ]、 全[ぜん]然[ぜん]お 風[ふ]呂[ろ]に 入[はい]らないんです。
\\	如何したんですか。 
\\	猫が死んだんです。	
\\	如[ど]何[う]したんですか。 
\\	猫[ねこ]が 死[し]んだんです。
\\	高かった 
\\	高くなかった	
\\	高[たか]かった 
\\	高[たか]くなかった
\\	元気だった 
\\	元気じゃなかった	
\\	元[げん]気[き]だった 
\\	元[げん]気[き]じゃなかった
\\	学生だった 
\\	学生じゃなかった	
\\	学[がく]生[せい]だった 
\\	学[がく]生[せい]じゃなかった
\\	有名だった 
\\	有名じゃなかった	
\\	有[ゆう]名[めい]だった 
\\	有[ゆう]名[めい]じゃなかった
\\	雨だった 
\\	雨じゃなかった	
\\	雨[あめ]だった 
\\	雨[あめ]じゃなかった
\\	赤かった 
\\	赤くなかった	
\\	赤[あか]かった 
\\	赤[あか]くなかった
\\	可愛かった 
\\	可愛くなかった	
\\	可愛[かわい]かった 
\\	可愛[かわい]くなかった
\\	短かった 
\\	短くなかった	
\\	短[みじか]かった 
\\	短[みじか]くなかった
\\	親切だった 
\\	親切じゃなかった	
\\	親[しん]切[せつ]だった 
\\	親[しん]切[せつ]じゃなかった
\\	安かった 
\\	安くなかった	
\\	安[やす]かった 
\\	安[やす]くなかった
\\	綺麗だった 
\\	綺麗じゃなかった	
\\	綺[き]麗[れい]だった 
\\	綺[き]麗[れい]じゃなかった
\\	良い天気だった 
\\	良い天気じゃなかった	
\\	良[い]い 天[てん]気[き]だった 
\\	良[い]い 天[てん]気[き]じゃなかった
\\	格好よかった 
\\	格好よくなかった	
\\	格[かっ]好[こ]よかった 
\\	格[かっ]好[こ]よくなかった
\\	寂しかった 
\\	寂しくなかった	
\\	寂[さび]しかった 
\\	寂[さび]しくなかった
\\	眠かった 
\\	眠くなかった	
\\	眠[ねむ]かった 
\\	眠[ねむ]くなかった
\\	意地悪だった 
\\	意地悪じゃなかった	
\\	意[い]地[じ]悪[わる]だった 
\\	意[い]地[じ]悪[わる]じゃなかった
\\	病気だった 
\\	病気じゃなかった	
\\	病[びょう]気[き]だった 
\\	病[びょう]気[き]じゃなかった
\\	若かった 
\\	若くなかった	
\\	若[わか]かった 
\\	若[わか]くなかった
\\	長かった 
\\	長くなかった	
\\	長[なが]かった 
\\	長[なが]くなかった
\\	便利だった 
\\	便利じゃなかった	
\\	便[べん]利[り]だった 
\\	便[べん]利[り]じゃなかった
\\	青かった 
\\	青くなかった	
\\	青[あお]かった 
\\	青[あお]くなかった
\\	静かだった 
\\	静かじゃなかった	
\\	静[しず]かだった 
\\	静[しず]かじゃなかった
\\	面白かった 
\\	面白くなかった	
\\	面[おも]白[しろ]かった 
\\	面[おも]白[しろ]くなかった
\\	昨日テレビを見た? 
\\	うん、見た。 
\\	ううん、見なかった。	
\\	昨[きの]日[う]テレビを 見[み]た? 
\\	うん、 見[み]た。 
\\	ううん、 見[み]なかった。
\\	昨日ピザを食べた? 
\\	うん、食べた。 
\\	ううん、食べなかった。	
\\	昨[きの]日[う]ピザを 食[た]べた? 
\\	うん、 食[た]べた。 
\\	ううん、 食[た]べなかった。
\\	昨日散歩した? 
\\	うん、散歩した。 
\\	ううん、散歩しなかった。	
\\	昨[きの]日[う]散[さん]歩[ぽ]した? 
\\	うん、 散[さん]歩[ぽ]した。 
\\	ううん、 散[さん]歩[ぽ]しなかった。
\\	昨日図書館で本を借りた? 
\\	うん、借りた。 
\\	ううん、借りなかった。	
\\	昨[きの]日[う]図[と]書[しょ]館[かん]で 本[ほん]を 借[か]りた? 
\\	うん、 借[か]りた。 
\\	ううん、 借[か]りなかった。
\\	昨日家を掃除した? 
\\	うん、掃除した。 
\\	ううん、掃除しなかった。	
\\	昨[きの]日[う]家[うち]を 掃[そう]除[じ]した? 
\\	うん、 掃[そう]除[じ]した。 
\\	ううん、 掃[そう]除[じ]しなかった。
\\	昨日家で料理した? 
\\	うん、料理した。 
\\	ううん、料理しなかった。	
\\	昨[きの]日[う]家[うち]で 料[りょう]理[り]した? 
\\	うん、 料[りょう]理[り]した。 
\\	ううん、 料[りょう]理[り]しなかった。
\\	昨日友達に会った? 
\\	うん、会った。 
\\	ううん、会わなかった。	
\\	昨[きの]日[う]友[とも]達[だち]に 会[あ]った? 
\\	うん、 会[あ]った。 
\\	ううん、 会[あ]わなかった。
\\	昨日単語を覚えた? 
\\	うん、覚えた。 
\\	ううん、覚えなかった。	
\\	昨[きの]日[う]単[たん]語[ご]を 覚[おぼ]えた? 
\\	うん、 覚[おぼ]えた。 
\\	ううん、 覚[おぼ]えなかった。
\\	昨日学校に来た? 
\\	うん、来た。 
\\	ううん、来なかった。	
\\	昨[きの]日[う]学[がっ]校[こう]に 来[き]た? 
\\	うん、 来[き]た。 
\\	ううん、 来[こ]なかった。
\\	昨日家族に電話を掛けた? 
\\	うん、掛けた。 
\\	ううん、掛けなかった。	
\\	昨[きの]日[う]家[か]族[ぞく]に 電[でん]話[わ]を 掛[か]けた? 
\\	うん、 掛[か]けた。 
\\	ううん、 掛[か]けなかった。
\\	ピザ	
\\	ピザ	ピꜜザ
\\	昨日コンピューターを使った? 
\\	うん、使った。 
\\	ううん、使わなかった。	
\\	昨[きの]日[う]コンピューターを 使[つか]った? 
\\	うん、 使[つか]った。 
\\	ううん、 使[つか]わなかった。
\\	昨日手紙を貰った? 
\\	うん、貰った。 
\\	ううん、貰わなかった。	
\\	昨[きの]日[う]手[て]紙[がみ]を 貰[もら]った? 
\\	うん、 貰[もら]った。 
\\	ううん、 貰[もら]わなかった。
\\	昨日遊びに行った? 
\\	うん、行った。 
\\	ううん、行かなかった。	
\\	昨[きの]日[う]遊[あそ]びに 行[い]った? 
\\	うん、 行[い]った。 
\\	ううん、 行[い]かなかった。
\\	子供の時元気だった? 
\\	うん、元気だった。 
\\	ううん、元気じゃなかった。	
\\	子[こ]供[ども]の 時[とき]元[げん]気[き]だった? 
\\	うん、 元[げん]気[き]だった。 
\\	ううん、 元[げん]気[き]じゃなかった。
\\	子供の時背が高かった? 
\\	うん、背が高かった。 
\\	ううん、背が高くなかった。	
\\	子[こ]供[ども]の 時[とき]背[せ]が 高[たか]かった? 
\\	うん、 背[せ]が 高[たか]かった。 
\\	ううん、 背[せ]が 高[たか]くなかった。
\\	子供の時スキーが上手だった? 
\\	うん、上手だった。 
\\	ううん、上手じゃなかった。	
\\	子[こ]供[ども]の 時[とき]スキーが 上[じょう]手[ず]だった? 
\\	うん、 上[じょう]手[ず]だった。 
\\	ううん、 上[じょう]手[ず]じゃなかった。
\\	子供の時髪が長かった? 
\\	うん、髪が長かった。 
\\	ううん、髪が長くなかった。	
\\	子[こ]供[ども]の 時[とき]髪[かみ]が 長[なが]かった? 
\\	うん、 髪[かみ]が 長[なが]かった。 
\\	ううん、 髪[かみ]が 長[なが]くなかった。
\\	子供の時勉強が好きだった? 
\\	うん、好きだった。 
\\	ううん、好きじゃなかった。	
\\	子[こ]供[ども]の 時[とき]勉[べん]強[きょう]が 好[す]きだった? 
\\	うん、 好[す]きだった。 
\\	ううん、 好[す]きじゃなかった。
\\	此の人は子供の時元気でしたか? 
\\	はい、元気だったと思います。/ いいえ、元気じゃなかったと思います。	
\\	此[こ]の 人[ひと]は 子[こ]供[ども]の 時[とき]元[げん]気[き]でしたか? 
\\	はい、 元[げん]気[き]だったと 思[おも]います。 
\\	いいえ、 元[げん]気[き]じゃなかったと 思[おも]います。
\\	此の人は子供の時綺麗でしたか? 
\\	はい、綺麗だったと思います。 
\\	いいえ、綺麗じゃなかったと思います。	
\\	此[こ]の 人[ひと]は 子[こ]供[ども]の 時[とき]綺[き]麗[れい]でしたか? 
\\	はい、 綺[き]麗[れい]だったと 思[おも]います。 
\\	いいえ、 綺[き]麗[れい]じゃなかったと 思[おも]います。
\\	此の人は子供の時意地悪でしたか? 
\\	はい、意地悪だったと思います。 
\\	いいえ、意地悪じゃなかったと思いま 。	
\\	此[こ]の 人[ひと]は 子[こ]供[ども]の 時[とき]意[い]地[じ]悪[わる]でしたか? 
\\	はい、 意[い]地[じ]悪[わる]だったと 思[おも]います。 
\\	いいえ、 意[い]地[じ]悪[わる]じゃなかったと 思[おも]います。
\\	此の人は子供の時歌が上手でしたか? 
\\	はい、上手だったと思います。/ いいえ、上手じゃなかったと思います。	
\\	此[こ]の 人[ひと]は 子[こ]供[ども]の 時[とき]歌[うた]が 上[じょう]手[ず]でしたか? 
\\	はい、 上[じょう]手[ず]だったと 思[おも]います。 
\\	いいえ、 上[じょう]手[ず]じゃなかったと 思[おも]います。
\\	此の人は子供の時頭が良かったですか。 
\\	はい、頭が良かったと思います。 
\\	いいえ、頭が良くなかったと思います。	
\\	此[こ]の 人[ひと]は 子[こ]供[ども]の 時[とき]頭[あたま]が 良[よ]かったですか。 
\\	はい、 頭[あたま]が 良[よ]かったと 思[おも]います。 
\\	いいえ、 頭[あたま]が 良[よ]くなかったと 思[おも]います。
\\	此の人は子供の時髪が黒かったですか。 
\\	はい、髪が黒かったと思います。 
\\	いいえ、髪が黒くなかったと思います。	
\\	此[こ]の 人[ひと]は 子[こ]供[ども]の 時[とき]髪[かみ]が 黒[くろ]かったですか。 
\\	はい、 髪[かみ]が 黒[くろ]かったと 思[おも]います。 
\\	いいえ、 髪[かみ]が 黒[くろ]くなかったと 思[おも]います。
\\	此の人は子供の時痩せていましたか。 
\\	はい、痩せていたと思います。 
\\	いいえ、痩せていなかったと思います。	
\\	此[こ]の 人[ひと]は 子[こ]供[ども]の 時[とき]痩[や]せていましたか。 
\\	はい、 痩[や]せていたと 思[おも]います。 
\\	いいえ、 痩[や]せていなかったと 思[おも]います。
\\	日本語の先生に就いて	
\\	日[に]本[ほん]語[ご]の 先[せん]生[せい]に 就[つ]いて
\\	日本語の先生は子供の時太っていましたか。 
\\	はい、太っていたと思います。 
\\	いいえ、太っていなかったと思います。	
\\	日[に]本[ほん]語[ご]の 先[せん]生[せい]は 子[こ]供[ども]の 時[とき]太[ふと]っていましたか。 
\\	はい、 太[ふと]っていたと 思[おも]います。/ いいえ、 太[ふと]っていなかったと 思[おも]います。
\\	日本語の先生は子供の時意地悪でしたか。 
\\	はい、意地悪だったと思います。 
\\	いいえ、意地悪じゃなかったと思います。	
\\	日[に]本[ほん]語[ご]の 先[せん]生[せい]は 子[こ]供[ども]の 時[とき]意[い]地[じ]悪[わる]でしたか。 
\\	はい、 意[い]地[じ]悪[わる]だったと 思[おも]います。/ いいえ、 意[い]地[じ]悪[わる]じゃなかったと 思[おも]います。
\\	水野さんは子供の時いい子だったと思います。	
\\	水[みず]野[の]さんは 子[こ]供[ども]の 時[とき]いい 子[こ]だったと 思[おも]います。
\\	水野さんは子供の時いい子じゃなかったと思います。	
\\	水[みず]野[の]さんは 子[こ]供[ども]の 時[とき]いい 子[こ]じゃなかったと 思[おも]います。
\\	子供の時、意地悪でしたか。 
\\	はい、意地悪でした。 
\\	水野さんは子供の時意地悪だったと言っていました。	
\\	子[こ]供[ども]の 時[とき]、 意[い]地[じ]悪[わる]でしたか。 
\\	はい、 意[い]地[じ]悪[わる]でした。 
\\	水[みず]野[の]さんは 子[こ]供[ども]の 時[とき]意[い]地[じ]悪[わる]だったと 言[い]っていました。
\\	子供の時、眼鏡を掛けていましたか。 
\\	はい、眼鏡を掛けていました。 
\\	水野さんは子供の時眼鏡を掛けていたと言っていました。	
\\	子[こ]供[ども]の 時[とき]、 眼[め]鏡[がね]を 掛[か]けていましたか。 
\\	はい、 眼[め]鏡[がね]を 掛[か]けていました。 
\\	水[みず]野[の]さんは 子[こ]供[ども]の 時[とき]眼[め]鏡[がね]を 掛[か]けていたと 言[い]っていました。
\\	子供の時、眼鏡を掛けていましたか。 
\\	いいえ、眼鏡を掛けていませんでした。 
\\	水野さんは子供の時眼鏡を掛けていなかったと言っていました。	
\\	子[こ]供[ども]の 時[とき]、 眼[め]鏡[がね]を 掛[か]けていましたか。 
\\	いいえ、 眼[め]鏡[がね]を 掛[か]けていませんでした。 
\\	水[みず]野[の]さんは 子[こ]供[ども]の 時[とき]眼[め]鏡[がね]を 掛[か]けていなかったと 言[い]っていました。
\\	笑っている人は田中さんです。	
\\	笑[わら]っている 人[ひと]は 田[た]中[なか]さんです。
\\	踊っている人は大川さんです。	
\\	踊[おど]っている 人[ひと]は 大[おお]川[がわ]さんです。
\\	帽子を被っている人は山口さんです。	
\\	帽[ぼう]子[し]を 被[かぶ]っている 人[ひと]は 山[やま]口[ぐち]さんです。
\\	お握り	
\\	お 握[にぎ]り	オニꜜキ°リ
\\	お握りを食べている人は中村さんです。	
\\	お 握[にぎ]りを 食[た]べている 人[ひと]は 中[なか]村[むら]さんです。
\\	着物を着ている人は野村さんです。	
\\	着[き]物[もの]を 着[き]ている 人[ひと]は 野[の]村[むら]さんです。
\\	ギターを弾いている人は鈴木さんです。	
\\	ギターを 弾[ひ]いている 人[ひと]は 鈴[すず]木[き]さんです。
\\	もう朝御飯を食べましたか。 
\\	いいえ、未だ食べていません。	
\\	もう 朝[あさ]御[ご]飯[はん]を 食[た]べましたか。 
\\	いいえ、 未[ま]だ 食[た]べていません。
\\	もう切符を買いましたか。 
\\	いいえ、未だ買っていません。	
\\	もう 切[きっ]符[ぷ]を 買[か]いましたか。 
\\	いいえ、 未[ま]だ 買[か]っていません。
\\	もう作文を書きましたか。 
\\	いいえ、未だ書いていません。	
\\	もう 作[さく]文[ぶん]を 書[か]きましたか。 
\\	いいえ、 未[ま]だ 書[か]いていません。
\\	もう薬を飲みましたか。 
\\	いいえ、未だ飲んでいません。	
\\	もう 薬[くすり]を 飲[の]みましたか。 
\\	いいえ、 未[ま]だ 飲[の]んでいません。
\\	もう新しい単語を覚えましたか。 
\\	いいえ、未だ覚えていません。	
\\	もう 新[あたら]しい 単[たん]語[ご]を 覚[おぼ]えましたか。 
\\	いいえ、 未[ま]だ 覚[おぼ]えていません。
\\	天ぷら	
\\	天[てん]ぷら	テンプラ
\\	魚が嫌いだから、寿司を食べません。	
\\	魚[さかな]が 嫌[きら]いだから、 寿[す]司[し]を 食[た]べません。
\\	試験が終わったから、今は暇です。	
\\	試[し]験[けん]が 終[お]わったから、 今[いま]は 暇[ひま]です。
\\	旅行に行ったから、学校を休みました。	
\\	旅[りょ]行[こう]に 行[い]ったから、 学[がっ]校[こう]を 休[やす]みました。
\\	演奏会の切符を二枚貰ったから、行きませんか。	
\\	演[えん]奏[そう]会[かい]の 切[きっ]符[ぷ]を 二[に]枚[まい]貰[もら]ったから、 行[い]きませんか。
\\	天気が良くなかったから、遊びに行きませんでした。	
\\	天[てん]気[き]が 良[よ]くなかったから、 遊[あそ]びに 行[い]きませんでした。
\\	授業が始まるから、急ぎましょう。	
\\	授[じゅ]業[ぎょう]が 始[はじ]まるから、 急[いそ]ぎましょう。
\\	病気の時	
\\	病[びょう]気[き]の 時[とき]	
\\	病気の時、良く薬を飲みますか。	
\\	病[びょう]気[き]の 時[とき]、 良[よ]く 薬[くすり]を 飲[の]みますか。
\\	良く散歩しますか。	
\\	良[よ]く 散[さん]歩[ぽ]しますか。
\\	去年の誕生日に何か貰いましたか。	
\\	去[きょ]年[ねん]の 誕[たん]生[じょう]日[び]に 何[なに]か 貰[もら]いましたか。
\\	今日授業は何時に始まりましたか。 何時に終わりましたか。	
\\	今[きょ]日[う]授[じゅ]業[ぎょう]は 何[なん]時[じ]に 始[はじ]まりましたか。 何[なん]時[じ]に 終[お]わりましたか。
\\	どんな色のトレーナーを持っていますか。	
\\	どんな 色[いろ]のトレーナーを 持[も]っていますか。
\\	今度の試験は難しいと思いますか。	
\\	今[こん]度[ど]の 試[し]験[けん]は 難[むずか]しいと 思[おも]いますか。
\\	新幹線と電車とどっちの方が遅いですか。 
\\	電車の方が新幹線より遅いです。	
\\	新[しん]幹[かん]線[せん]と 電[でん]車[しゃ]とどっちの 方[ほう]が 遅[おそ]いですか。 
\\	電[でん]車[しゃ]の 方[ほう]が 新[しん]幹[かん]線[せん]より 遅[おそ]いです。
\\	東京と大阪とどっちの方が大きいですか。 
\\	東京の方が大阪より大きいです。	
\\	東[とう]京[きょう]と 大[おお]阪[さか]とどっちの 方[ほう]が 大[おお]きいですか。 
\\	東[とう]京[きょう]の 方[ほう]が 大[おお]阪[さか]より 大[おお]きいです。
\\	大阪と京都とどっちの方が小さいですか。 
\\	京都の方が大阪より小さいです。	
\\	大[おお]阪[さか]と 京[きょう]都[と]とどっちの 方[ほう]が 小[ちい]さいですか。 
\\	京[きょう]都[と]の 方[ほう]が 大[おお]阪[さか]より 小[ちい]さいです。
\\	夏と冬とどっちの方が好きですか。 
\\	夏も冬も好きです。	
\\	夏[なつ]と 冬[ふゆ]とどっちの 方[ほう]が 好[す]きですか。 
\\	夏[なつ]も 冬[ふゆ]も 好[す]きです。
\\	寿司と天ぷらとどっちの方が美味しいですか。 
\\	寿司の方が天ぷらより美味しいです。	
\\	寿[す]司[し]と 天[てん]ぷらとどっちの 方[ほう]が 美[お]味[い]しいですか。 
\\	寿[す]司[し]の 方[ほう]が 天[てん]ぷらより 美[お]味[い]しいです。
\\	格好いい人と頭が良い人とどっちの方が好きですか。 
\\	頭が良い人の方が格好いい人より好きです。	
\\	格[かっ]好[こ]いい 人[ひと]と 頭[あたま]が 良[い]い 人[ひと]とどっちの 方[ほう]が 好[す]きですか。 
\\	頭[あたま]が 良[い]い 人[ひと]の 方[ほう]が 格[かっ]好[こ]いい 人[ひと]より 好[す]きです。
\\	中国料理と日本料理とどっちの方が好きですか。 
\\	中国料理も日本料理も好きです。	
\\	中[ちゅう]国[ごく]料[りょう]理[り]と 日[に]本[ほん]料[りょう]理[り]とどっちの 方[ほう]が 好[す]きですか。 
\\	中[ちゅう]国[ごく]料[りょう]理[り]も 日[に]本[ほん]料[りょう]理[り]も 好[す]きです。
\\	船と飛行機とどっちの方が人気が有りますか。 
\\	飛行機の方が船より人気が有ります。	
\\	船[ふね]と 飛[ひ]行[こう]機[き]とどっちの 方[ほう]が 人[にん]気[き]が 有[あ]りますか。 
\\	飛[ひ]行[こう]機[き]の 方[ほう]が 船[ふね]より 人[にん]気[き]が 有[あ]ります。
\\	ドイツの車と日本の車とどっちの方が良いですか。 
\\	日本の車の方がドイツの車より良いです。	
\\	ドイツの 車[くるま]と 日[に]本[ほん]の 車[くるま]とどっちの 方[ほう]が 良[い]いですか。 
\\	日[に]本[ほん]の 車[くるま]の 方[ほう]がドイツの 車[くるま]より 良[い]いです。
\\	漢字と片仮名とどっちの方が簡単ですか。 
\\	片仮名の方が漢字より簡単です。	
\\	漢[かん]字[じ]と 片[かた]仮[か]名[な]とどっちの 方[ほう]が 簡[かん]単[たん]ですか。 
\\	片[かた]仮[か]名[な]の 方[ほう]が 漢[かん]字[じ]より 簡[かん]単[たん]です。
\\	春と秋とどっちの方が好きですか。 
\\	春も秋も好きです。	
\\	春[はる]と 秋[あき]とどっちの 方[ほう]が 好[す]きですか。 
\\	春[はる]も 秋[あき]も 好[す]きです。
\\	貴方の国の冬と日本の冬とどっちの方が暖かいですか。 
\\	日本の冬の方が私の国の冬より暖かいです。	
\\	貴方[あなた]の 国[くに]の 冬[ふゆ]と 日[に]本[ほん]の 冬[ふゆ]とどっちの 方[ほう]が 暖[あたた]かいですか。 
\\	日[に]本[ほん]の 冬[ふゆ]の 方[ほう]が 私[わたし]の 国[くに]の 冬[ふゆ]より 暖[あたた]かいです。
\\	貴方の国の生活と日本の生活とどっちの方が楽しいですか。 
\\	日本の生活の方が私の国の生活より楽しいです。	
\\	貴方[あなた]の 国[くに]の 生[せい]活[かつ]と 日[に]本[ほん]の 生[せい]活[かつ]とどっちの 方[ほう]が 楽[たの]しいですか。 
\\	日[に]本[ほん]の 生[せい]活[かつ]の 方[ほう]が 私[わたし]の 国[くに]の 生[せい]活[かつ]より 楽[たの]しいです。
\\	新幹線と電車とバスの中で、何れが一番速いですか。 
\\	新幹線が一番速いです。	
\\	新[しん]幹[かん]線[せん]と 電[でん]車[しゃ]とバスの 中[なか]で、 何[ど]れが 一[いち]番[ばん]速[はや]いですか。 
\\	新[しん]幹[かん]線[せん]が 一[いち]番[ばん]速[はや]いです。
\\	此の中で、何れが一番大きいですか。 
\\	横浜が一番大きいと思います。	
\\	此[こ]の 中[なか]で、 何[ど]れが 一[いち]番[ばん]大[おお]きいですか。 
\\	横[よこ]浜[はま]が 一[いち]番[ばん]大[おお]きいと 思[おも]います。
\\	此の中で、何れが一番小さいですか。 
\\	沖縄が一番小さいです。	
\\	此[こ]の 中[なか]で、 何[ど]れが 一[いち]番[ばん]小[ちい]さいですか。 
\\	沖[おき]縄[なわ]が 一[いち]番[ばん]小[ちい]さいです。
\\	田中さんと山田さんと鈴木さんの中で、誰が一番背が高いですか。 
\\	鈴木さんが一番背が高いです。	
\\	田[た]中[なか]さんと 山[やま]田[だ]さんと 鈴[すず]木[き]さんの 中[なか]で、 誰[だれ]が 一[いち]番[ばん]背[せ]が 高[たか]いですか。 
\\	鈴[すず]木[き]さんが 一[いち]番[ばん]背[せ]が 高[たか]いです。
\\	此の中で、誰が一番若いですか。 
\\	山田さんが一番若いです。	
\\	此[こ]の 中[なか]で、 誰[だれ]が 一[いち]番[ばん]若[わか]いですか。 
\\	山[やま]田[だ]さんが 一[いち]番[ばん]若[わか]いです。
\\	食べ物の中で、何が一番好きですか。 
\\	寿司が一番好きです。	
\\	食[た]べ 物[もの]の 中[なか]で、 何[なに]が 一[いち]番[ばん]好[す]きですか。 
\\	寿[す]司[し]が 一[いち]番[ばん]好[す]きです。
\\	飲み物の中で、何が一番好きですか。 
\\	牛乳が一番好きです。	
\\	飲[の]み 物[もの]の 中[なか]で、 何[なに]が 一[いち]番[ばん]好[す]きですか。 
\\	牛[ぎゅう]乳[にゅう]が 一[いち]番[ばん]好[す]きです。
\\	世界の町の中で、何れが一番好きですか。 
\\	世界の町の中で、東京が一番好きです。	
\\	世[せ]界[かい]の 町[まち]の 中[なか]で、 何[ど]れが 一[いち]番[ばん]好[す]きですか。 
\\	世[せ]界[かい]の 町[まち]の 中[なか]で、 東[とう]京[きょう]が 一[いち]番[ばん]好[す]きです。
\\	有名人の中で、誰が一番好きですか。 
\\	天皇が一番好きです。	
\\	有[ゆう]名[めい]人[じん]の 中[なか]で、 誰[だれ]が 一[いち]番[ばん]好[す]きですか。 
\\	天[てん]皇[のう]が 一[いち]番[ばん]好[す]きです。
\\	天皇	
\\	天[てん]皇[のう]	テンノꜜー
\\	日本のロック	
\\	日[に]本[ほん]のロック	
\\	音楽の中で、何れが一番好きですか。 
\\	日本のロックが一番好きです。	
\\	音[おん]楽[がく]の 中[なか]で、 何[ど]れが 一[いち]番[ばん]好[す]きですか。 
\\	日[に]本[ほん]のロックが 一[いち]番[ばん]好[す]きです。
\\	季節の中で、何れが一番好きですか。 
\\	秋が一番好きです。	
\\	季[き]節[せつ]の 中[なか]で、 何[ど]れが 一[いち]番[ばん]好[す]きですか。 
\\	秋[あき]が 一[いち]番[ばん]好[す]きです。
\\	クラスの中で、誰が一番良い学生ですか。 
\\	僕が一番良い学生です。	
\\	クラスの 中[なか]で、 誰[だれ]が 一[いち]番[ばん]良[い]い 学[がく]生[せい]ですか。 
\\	僕[ぼく]が 一[いち]番[ばん]良[い]い 学[がく]生[せい]です。
\\	クラスの中で、誰が一番髪が長いですか。 
\\	水野さんが一番髪が長いです。	
\\	クラスの 中[なか]で、 誰[だれ]が 一[いち]番[ばん]髪[かみ]が 長[なが]いですか。 
\\	水[みず]野[の]さんが 一[いち]番[ばん]髪[かみ]が 長[なが]いです。
\\	富士山	
\\	富[ふ]士[じ]山[さん]	
\\	エベレスト	
\\	エベレスト	
\\	富士山とエベレストと何方の方が高いですか。	
\\	富[ふ]士[じ]山[さん]とエベレストと 何方[どちら]の 方[ほう]が 高[たか]いですか。
\\	田中さんと山田さんと何方の方が髪が長いですか。	
\\	田[た]中[なか]さんと 山[やま]田[だ]さんと 何方[どちら]の 方[ほう]が 髪[かみ]が 長[なが]いですか。
\\	クラスの中で今日誰が一番お金を持っていますか。	
\\	クラスの 中[なか]で 今日[きょう]誰[だれ]が 一[いち]番[ばん]お 金[かね]を 持[も]っていますか。
\\	世界の国の中で、何処が一番小さいですか。	
\\	世[せ]界[かい]の 国[くに]の 中[なか]で、 何[ど]処[こ]が 一[いち]番[ばん]小[ちい]さいですか。
\\	此の林檎は水野さんのです。	
\\	此[こ]の 林[りん]檎[ご]は 水[みず]野[の]さんのです。
\\	此の牛乳は鈴木さんのです。	
\\	此[こ]の 牛[ぎゅう]乳[にゅう]は 鈴[すず]木[き]さんのです。
\\	何方のコーヒーが良いですか。 
\\	冷たいのを下さい。	
\\	何方[どちら]のコーヒーが 良[い]いですか。 
\\	冷[つめ]たいのを 下[くだ]さい。
\\	何方の辞書が良いですか。 
\\	電子のを下さい。	
\\	何方[どちら]の 辞[じ]書[しょ]が 良[い]いですか。 
\\	電[でん]子[し]のを 下[くだ]さい。
\\	電子辞書	
\\	電[でん]子[し]辞[じ]書[しょ]	
\\	何方の手袋が良いですか。 
\\	緑のを下さい。	
\\	何方[どちら]の 手[て]袋[ぶくろ]が 良[い]いですか。 
\\	緑[みどり]のを 下[くだ]さい。
\\	月曜日に本を読むつもりです。	
\\	月[げつ]曜[よう]日[び]に 本[ほん]を 読[よ]むつもりです。
\\	月曜日にピアノを練習するつもりです。	
\\	月[げつ]曜[よう]日[び]にピアノを 練[れん]習[しゅう]するつもりです。
\\	火曜日に運動するつもりです。	
\\	火[か]曜[よう]日[び]に 運[うん]動[どう]するつもりです。
\\	水曜日に洗濯するつもりです。	
\\	水[すい]曜[よう]日[び]に 洗[せん]濯[たく]するつもりです。
\\	木曜日に友達に手紙を書くつもりです。	
\\	木[もく]曜[よう]日[び]に 友[とも]達[だち]に 手[て]紙[がみ]を 書[か]くつもりです。
\\	木曜日に出掛けないつもりです。	
\\	木[もく]曜[よう]日[び]に 出[で]掛[か]けないつもりです。
\\	金曜日に友達と晩御飯を食べるつもりです。	
\\	金[きん]曜[よう]日[び]に 友[とも]達[だち]と 晩[ばん]御[ご]飯[はん]を 食[た]べるつもりです。
\\	金曜日に日本語を勉強しないつもりです。	
\\	金[きん]曜[よう]日[び]に 日[に]本[ほん]語[ご]を 勉[べん]強[きょう]しないつもりです。
\\	土曜日に友達の家に泊まるつもりです。	
\\	土[ど]曜[よう]日[び]に 友[とも]達[だち]の 家[うち]に 泊[と]まるつもりです。
\\	土曜日に家に帰らないつもりです。	
\\	土[ど]曜[よう]日[び]に 家[いえ]に 帰[かえ]らないつもりです。
\\	日曜日に部屋を掃除するつもりです。	
\\	日[にち]曜[よう]日[び]に 部[へ]屋[や]を 掃[そう]除[じ]するつもりです。
\\	日曜日に早く起きないつもりです。	
\\	日[にち]曜[よう]日[び]に 早[はや]く 起[お]きないつもりです。
\\	週末、映画を見に行きますか。 
\\	ええ、見に行くつもりです。 
\\	いいえ、見に行かないつもりです。	
\\	週[しゅう]末[まつ]、 映[えい]画[が]を 見[み]に 行[い]きますか。 
\\	ええ、 見[み]に 行[い]くつもりです。 
\\	いいえ、 見[み]に 行[い]かないつもりです。
\\	今日の午後勉強しますか。 
\\	ええ、勉強するつもりです。 
\\	いいえ、勉強しないつもりです。	
\\	今日[きょう]の 午[ご]後[ご]勉[べん]強[きょう]しますか。 
\\	ええ、 勉[べん]強[きょう]するつもりです。 
\\	いいえ、 勉[べん]強[きょう]しないつもりです。
\\	冬休みに旅行しますか。 
\\	ええ、するつもりです。 
\\	いいえ、しないつもりです。	
\\	冬[ふゆ]休[やす]みに 旅[りょ]行[こう]しますか。 
\\	ええ、するつもりです。 
\\	いいえ、しないつもりです。
\\	三年後、日本に居ますか。 
\\	ええ、居るつもりです。 
\\	いいえ、居ないつもりです。	
\\	三[さん]年[ねん]後[ご]、 日[に]本[ほん]に 居[い]ますか。 
\\	ええ、 居[い]るつもりです。 
\\	いいえ、 居[い]ないつもりです。
\\	来学期も日本語を勉強しますか。 
\\	ええ、勉強するつもりです。 
\\	いいえ、勉強しないつもりです。	
\\	来[らい] 学[がっ]期[き]も 日[に]本[ほん]語[ご]を 勉[べん]強[きょう]しますか。 
\\	ええ、 勉[べん]強[きょう]するつもりです。 
\\	いいえ、 勉[べん]強[きょう]しないつもりです。
\\	眠く成りました。	
\\	眠[ねむ]く 成[な]りました。
\\	元気に成りました。	
\\	元[げん]気[き]に 成[な]りました。
\\	髪が短く成りました。	
\\	髪[かみ]が 短[みじか]く 成[な]りました。
\\	大きく成りました。	
\\	大[おお]きく 成[な]りました。
\\	暑く成りました。	
\\	暑[あつ]く 成[な]りました。
\\	涼しく成りました。	
\\	涼[すず]しく 成[な]りました。
\\	医者に成りました。	
\\	医[い]者[しゃ]に 成[な]りました。
\\	春に成りました。	
\\	春[はる]に 成[な]りました。
\\	円が安く成りました。	
\\	円[えん]が 安[やす]く 成[な]りました。
\\	掃除したから、部屋が綺麗に成りました。	
\\	掃[そう]除[じ]したから、 部[へ]屋[や]が 綺[き]麗[れい]に 成[な]りました。
\\	美容院に行ったから、髪が短く成りました。	
\\	美[び]容[よう]院[いん]に 行[い]ったから、 髪[かみ]が 短[みじか]く 成[な]りました。
\\	毎日ピアノを練習しているから、上手に成りました。	
\\	毎[まい]日[にち]ピアノを 練[れん]習[しゅう]しているから、 上[じょう]手[ず]に 成[な]りました。
\\	お酒を沢山飲んで、顔が青く成りました。	
\\	お 酒[さけ]を 沢[たく]山[さん]飲[の]んで、 顔[かお]が 青[あお]く 成[な]りました。
\\	十一月は暖かかったですが、此の頃寒く成りました。	
\\	十[じゅう]一[いち]月[がつ]は 暖[あたた]かかったですが、 此[こ]の 頃[ごろ]寒[さむ]く 成[な]りました。
\\	教えるのが好きだから、教師に成るつもりです。	
\\	教[おし]えるのが 好[す]きだから、 教[きょう]師[し]に 成[な]るつもりです。
\\	家から駅迄自転車で行きます。	
\\	家[うち]から 駅[えき]迄[まで]自[じ]転[てん]車[しゃ]で 行[い]きます。
\\	床屋から学校迄新幹線で行きます。	
\\	床[とこ]屋[や]から 学[がっ]校[こう]迄[まで]新[しん]幹[かん]線[せん]で 行[い]きます。
\\	家からバス停迄歩いて行きます。	
\\	家[うち]から バ[ば]ス[す]停[てい]迄[まで]歩[ある]いて 行[い]きます。
\\	横浜から東京迄電車で行きます。	
\\	横[よこ]浜[はま]から 東[とう]京[きょう]迄[まで]電[でん]車[しゃ]で 行[い]きます。
\\	会社から市民病院迄地下鉄で行きます。	
\\	会[かい]社[しゃ]から 市[し]民[みん]病[びょう]院[いん]迄[まで]地[ち]下[か]鉄[てつ]で 行[い]きます。
\\	名古屋	
\\	名[な]古[ご]屋[や]	ナꜜコ°ヤ
\\	名古屋から横浜迄新幹線で行きます。	
\\	名[な]古[ご]屋[や]から 横[よこ]浜[はま]迄[まで]新[しん]幹[かん]線[せん]で 行[い]きます。
\\	日本からドイツ迄飛行機で行きます。	
\\	日[に]本[ほん]からドイツ 迄[まで]飛[ひ]行[こう]機[き]で 行[い]きます。
\\	台湾から中国迄船で行きます。	
\\	台[たい]湾[わん]から 中[ちゅう]国[ごく]迄[まで]船[ふね]で 行[い]きます。
\\	家から駅迄十分掛かります。	
\\	家[うち]から 駅[えき]迄[まで]十[じゅっ]分[ぷん]掛[か]かります。
\\	床屋から学校迄四十分掛かります。	
\\	床[とこ]屋[や]から 学[がっ]校[こう]迄[まで]四[よん]十[じゅっ]分[ぷん]掛[か]かります。
\\	会社から市民病院迄一時間掛かります。	
\\	会[かい]社[しゃ]から 市[し]民[みん]病[びょう]院[いん]迄[まで]一[いち]時[じ]間[かん]掛[か]かります。
\\	名古屋から横浜迄八時間掛かります。	
\\	名[な]古[ご]屋[や]から 横[よこ]浜[はま]迄[まで]八[はち]時[じ]間[かん]掛[か]かります。
\\	家からバス停迄八週間掛かりました。	
\\	家[うち]から バ[ば]ス[す]停[てい]迄[まで]八[はっ]週[しゅう]間[かん]掛[か]かりました。
\\	台湾から中国迄一週間掛かります。	
\\	台[たい]湾[わん]から 中[ちゅう]国[ごく]迄[まで]一[いっ]週[しゅう]間[かん]掛[か]かります。
\\	家から南駅迄自転車で行きます。五分掛かります。	
\\	家[うち]から 南[みなみ]駅[えき]迄[まで]自[じ]転[てん]車[しゃ]で 行[い]きます。 五[ご]分[ふん]掛[か]かります。
\\	南駅から西駅迄地下鉄に乗ります。二十分掛かります。	
\\	南[みなみ]駅[えき]から 西[にし]駅[えき]迄[まで]地[ち]下[か]鉄[てつ]に 乗[の]ります。 二[に]十[じゅっ]分[ぷん]掛[か]かります。
\\	西駅から学校迄歩いて行きます。十分位掛かります。	
\\	西[にし]駅[えき]から 学[がっ]校[こう]迄[まで]歩[ある]いて 行[い]きます。 十[じゅっ]分[ぷん]位[ぐらい]掛[か]かります。
\\	メアリーさんは今年の冬休みに何処かに行くつもりですか。	
\\	メアリーさんは 今年[ことし]の 冬[ふゆ]休[やす]みに 何[ど]処[こ]かに 行[い]くつもりですか。
\\	どうやって韓国へ行きますか。誰と行きますか。	
\\	どうやって 韓[かん]国[こく]へ 行[い]きますか。 誰[だれ]と 行[い]きますか。
\\	何処に泊まりますか。	
\\	何[ど]処[こ]に 泊[と]まりますか。
\\	韓国で何をするつもりですか。	
\\	韓[かん]国[こく]で 何[なに]をするつもりですか。
\\	間も無く発車します。	
\\	間[ま]も 無[な]く 発[はっ]車[しゃ]します。
\\	間も無く	
\\	間[ま]も 無[な]く	マモꜜナク
\\	発車	
\\	発[はっ]車[しゃ]	ハッシャ
\\	電車が参ります。	
\\	電[でん]車[しゃ]が 参[まい]ります。
\\	次は南駅止まります。	
\\	次[つぎ]は 南[みなみ]駅[えき]止[と]まります。
\\	止める 
\\	止めて 
\\	止めます	
\\	止[と]める 
\\	止[と]めて 
\\	止[と]めます	
\\	ドアが閉まります。ご注意下さい。	
\\	ドアが 閉[し]まります。ご 注[ちゅう]意[い]下[くだ]さい。
\\	秋葉原	
\\	秋[あき]葉[は]原[ばら]	
\\	此の電車は秋葉原に止まりますか。	
\\	此[こ]の 電[でん]車[しゃ]は 秋[あき]葉[は]原[ばら]に 止[と]まりますか。
\\	終電は何時ですか。	
\\	終[しゅう]電[でん]は 何[なん]時[じ]ですか。
\\	東京迄の指定席を一枚お願いします。	
\\	東[とう]京[きょう]迄[まで]の 指[し]定[てい] 席[せき]を 一[いち]枚[まい]お 願[ねが]いします。
\\	学割が使えますか。	
\\	学[がく]割[わり]が 使[つか]えますか。
\\	使える 
\\	使えて 
\\	使えます	
\\	使[つか]える 
\\	使[つか]えて 
\\	使[つか]えます	
\\	鎌倉	
\\	鎌[かま]倉[くら]	
\\	鎌倉行きの電車は何れですか。	
\\	鎌[かま]倉[くら]行[い]きの 電[でん]車[しゃ]は 何[ど]れですか。
\\	二番線です。	
\\	二[に]番[ばん]線[せん]です。
\\	週末は、勉強したり、友達と話したりしました。	
\\	週[しゅう]末[まつ]は、 勉[べん]強[きょう]したり、 友[とも]達[だち]と 話[はな]したりしました。
\\	踊ったり、音楽を聞いたりのが好きです。	
\\	踊[おど]ったり、 音[おん]楽[がく]を 聞[き]いたりのが 好[す]きです。
\\	山下先生はテレビを見ますか。 
\\	いいえ、テレビは見ません。	
\\	山[やま]下[した]先[せん]生[せい]はテレビを 見[み]ますか。 
\\	いいえ、テレビは 見[み]ません。
\\	コーヒーが飲みたいですか。 
\\	いいえ、コーヒーが飲みたくありません。	
\\	コーヒーが 飲[の]みたいですか。 
\\	いいえ、コーヒーが 飲[の]みたくありません。
\\	英語では話したくありません。	
\\	英[えい]語[ご]では 話[はな]したくありません。
\\	広島	
\\	広[ひろ]島[しま]	ヒロシマ
\\	広島県	
\\	広[ひろ]島[しま]県[けん]	
\\	広島には行ったことが有りません。	
\\	広[ひろ]島[しま]には 行[い]ったことが 有[あ]りません。
\\	其の人に一回丈会ったことが有ります。	
\\	其[そ]の 人[ひと]に 一[いっ]回[かい]丈[だけ]会[あ]ったことが 有[あ]ります。
\\	一つ丈質問が有ります。	
\\	一[ひと]つ丈[だけ]質[しつ]問[もん]が 有[あ]ります。
\\	三十分丈寝ました。	
\\	三[さん]十[じゅっ]分[ぷん]丈[だけ]寝[ね]ました。
\\	サラダ	
\\	サラダ	サꜜラダ
\\	晩御飯にサラダを食べました。	
\\	晩[ばん]御[ご]飯[はん]にサラダを 食[た]べました。
\\	お土産に絵葉書を買いました。	
\\	お 土産[みやげ]に 絵[え]葉[は]書[がき]を 買[か]いました。
\\	湖迄ドライブに行きました。	
\\	湖[みずうみ]迄[まで]ドライブに 行[い]きました。
\\	日本で車を運転したことが有りますか。	
\\	日[に]本[ほん]で 車[くるま]を 運[うん]転[てん]したことが 有[あ]りますか。
\\	夕べ怖い夢を見ました。	
\\	夕[ゆう]べ 怖[こわ]い 夢[ゆめ]を 見[み]ました。
\\	夢を持っています。 夢が有ります。	
\\	夢[ゆめ]を 持[も]っています。 夢[ゆめ]が 有[あ]ります。
\\	将来の夢	
\\	将[しょう]来[らい]の 夢[ゆめ]	
\\	貴方の将来の夢は何ですか。	
\\	貴方[あなた]の 将[しょう]来[らい]の 夢[ゆめ]は 何[なに]ですか。
\\	東京には大きいデパートが沢山有ります。	
\\	東[とう]京[きょう]には 大[おお]きいデパートが 沢[たく]山[さん]有[あ]ります。
\\	私の大学には良い日本語の先生が居ます。	
\\	私[わたし]の 大[だい]学[がく]には 良[い]い 日[に]本[ほん]語[ご]の 先[せん]生[せい]が 居[い]ます。
\\	ハンバーガーが食べたいです。 ハンバーガーが食べたくありません。	
\\	ハンバーガーが 食[た]べたいです。 ハンバーガーが 食[た]べたくありません。
\\	湖に行きたいです。 湖に行きたくありません。	
\\	湖[みずうみ]に 行[い]きたいです。 湖[みずうみ]に 行[い]きたくありません。
\\	日本語のテープが聞きたいです。 日本語のテープが聞きたくありません。	
\\	日[に]本[ほん]語[ご]のテープが 聞[き]きたいです。 日[に]本[ほん]語[ご]のテープが 聞[き]きたくありません。
\\	旅行がしたいです。 旅行がしたくありません。	
\\	旅[りょ]行[こう]がしたいです。 旅[りょ]行[こう]がしたくありません。
\\	社長に成りたいです。 社長に成りたくありません。	
\\	社[しゃ]長[ちょう]に 成[な]りたいです。 社[しゃ]長[ちょう]に 成[な]りたくありません。
\\	日本で働きたいです。 日本で働きたくありません。	
\\	日[に]本[ほん]で 働[はたら]きたいです。 日[に]本[ほん]で 働[はたら]きたくありません。
\\	車が買いたいです。 車が買いたくありません。	
\\	車[くるま]が 買[か]いたいです。 車[くるま]が 買[か]いたくありません。
\\	日本に住みたいです。 日本に住みたくありません。	
\\	日[に]本[ほん]に 住[す]みたいです。 日[に]本[ほん]に 住[す]みたくありません。
\\	留学したいです。 留学したくありません。	
\\	留[りゅう]学[がく]したいです。 留[りゅう]学[がく]したくありません。
\\	山に登りたいです。 山に登りたくありません。	
\\	山[やま]に 登[のぼ]りたいです。 山[やま]に 登[のぼ]りたくありません。
\\	お腹が空いていますから、ハンバーガーが食べたいです。	
\\	お 腹[なか]が 空[す]いていますから、ハンバーガーが 食[た]べたいです。
\\	先ほど食べましたから、ハンバーガーが食べたくありません。	
\\	先[さき]ほど 食[た]べましたから、ハンバーガーが 食[た]べたくありません。
\\	子供の時、玩具の電車で遊びたかったです。 子供の時、玩具の電車で遊びたくありませんでした。	
\\	子[こ]供[ども]の 時[とき]、 玩[おも]具[ちゃ]の 電[でん]車[しゃ]で 遊[あそ]びたかったです。 子[こ]供[ども]の 時[とき]、 玩[おも]具[ちゃ]の 電[でん]車[しゃ]で 遊[あそ]びたくありませんでした。
\\	テレビが見たかったです。 テレビが見たくありませんでした。	
\\	テレビが 見[み]たかったです。 テレビが 見[み]たくありませんでした。
\\	飛行機に乗りたかったです。 飛行機に乗りたくありませんでした。	
\\	飛[ひ]行[こう]機[き]に 乗[の]りたかったです。 飛[ひ]行[こう]機[き]に 乗[の]りたくありませんでした。
\\	お風呂に入りたかったです。 お風呂に入りたくありませんでした。	
\\	お 風[ふ]呂[ろ]に 入[はい]りたかったです。 お 風[ふ]呂[ろ]に 入[はい]りたくありませんでした。
\\	犬が飼いたかったです。 犬が飼いたくありませんでした。	
\\	犬[いぬ]が 飼[か]いたかったです。 犬[いぬ]が 飼[か]いたくありませんでした。
\\	学校が辞めたかったです。 学校が辞めたくありませんでした。	
\\	学[がっ]校[こう]が 辞[や]めたかったです。 学[がっ]校[こう]が 辞[や]めたくありませんでした。
\\	魚が食べたかったです。 魚が食べたくありませんでした。	
\\	魚[さかな]が 食[た]べたかったです。 魚[さかな]が 食[た]べたくありませんでした。
\\	子供の時、男の子と話したかったです。 子供の時、女の子と話したくありませんでした。	
\\	子[こ]供[ども]の 時[とき]、 男[おとこ]の 子[こ]と 話[はな]したかったです。 子[こ]供[ども]の 時[とき]、 女[おんな]の 子[こ]と 話[はな]したくありませんでした。
\\	ピアノが習いたかったです。 ピアノが習いたくありませんでした。	
\\	ピアノが 習[なら]いたかったです。 ピアノが 習[なら]いたくありませんでした。
\\	車が運転したかったです。 車が運転したくありませんでした。	
\\	車[くるま]が 運[うん]転[てん]したかったです。 車[くるま]が 運[うん]転[てん]したくありませんでした。
\\	有名に成りたいです。 有名に成りたくありません。	
\\	有[ゆう]名[めい]に 成[な]りたいです。 有[ゆう]名[めい]に 成[な]りたくありません。
\\	けんさんはピザが食べたいと言っていました。 けんさんはピザが食べたがっています。	
\\	けんさんはピザが 食[た]べたいと 言[い]っていました。 けんさんはピザが 食[た]べたがっています。
\\	昼御飯に何が食べたいですか。	
\\	昼[ひる] 御[ご]飯[はん]に 何[なに]が 食[た]べたいですか。
\\	けんさんは昼御飯に寿司が食べたいと言っていました。 けんさんは昼御飯に寿司が食べたがっています。	
\\	けんさんは 昼[ひる] 御[ご]飯[はん]に 寿[す]司[し]が 食[た]べたいと 言[い]っていました。 けんさんは 昼[ひる] 御[ご]飯[はん]に 寿[す]司[し]が 食[た]べたがっています。
\\	何が一番買いたいですか。	
\\	何[なに]が 一[いち]番[ばん]買[か]いたいですか。
\\	けんさんが一番一輪車が買いたいと言っていました。 けんさんが一番一輪車が買いたがっています。	
\\	けんさんが 一[いち]番[ばん]一[いち]輪[りん]車[しゃ]が 買[か]いたいと 言[い]っていました。 けんさんが 一[いち]番[ばん]一[いち]輪[りん]車[しゃ]が 買[か]いたがっています。
\\	何処に一番行きたいですか。	
\\	何[ど]処[こ]に 一[いち]番[ばん]行[い]きたいですか。
\\	けんさんが一番名古屋に行きたいと言っていました。 けんさんが一番名古屋に行きたがっています。	
\\	けんさんが 一[いち]番[ばん]名[な]古[ご]屋[や]に 行[い]きたいと 言[い]っていました。 けんさんが 一[いち]番[ばん]名[な]古[ご]屋[や]に 行[い]きたがっています。
\\	誰に一番会いたいですか。	
\\	誰[だれ]に 一[いち]番[ばん]会[あ]いたいですか。
\\	けんさんが一番天皇に会いたいと言っていました。 けんさんが一番天皇に会いたがっています。	
\\	けんさんが 一[いち]番[ばん]天[てん]皇[のう]に 会[あ]いたいと 言[い]っていました。 けんさんが 一[いち]番[ばん]天[てん]皇[のう]に 会[あ]いたがっています。
\\	何が習いたいですか。	
\\	何[なに]が 習[なら]いたいですか。
\\	けんさんは日本語が習いたいと言っていました。 けんさんは日本語が習いたがっています。	
\\	けんさんは 日[に]本[ほん]語[ご]が 習[なら]いたいと 言[い]っていました。 けんさんは 日[に]本[ほん]語[ご]が 習[なら]いたがっています。
\\	何がしたいですか。	
\\	何[なに]がしたいですか。
\\	何がしたくありませんか。	
\\	何[なに]がしたくありませんか。
\\	子供の時、何に成りたかったですか。	
\\	子[こ]供[ども]の 時[とき]、 何[なに]に 成[な]りたかったですか。
\\	将来、何に成りたいですか。	
\\	将[しょう]来[らい]、 何[なに]に 成[な]りたいですか。
\\	けんさんは子供の時、弁護士に成りたいと言っていました。 けんさんは子供の時、弁護士に成りたがっています。	
\\	けんさんは 子[こ]供[ども]の 時[とき]、 弁[べん]護[ご]士[し]に 成[な]りたいと 言[い]っていました。 けんさんは 子[こ]供[ども]の 時[とき]、 弁[べん]護[ご]士[し]に 成[な]りたがっています。
\\	今学期の後、何がしたいですか。	
\\	今[こん]学[がっ]期[き]の 後[あと]、 何[なに]がしたいですか。
\\	作家	
\\	作[さっ]家[か]	サッカ 
\\	サꜜッカ
\\	警察官 
\\	警官	
\\	警[けい]察[さつ]官[かん] 
\\	警[けい]官[かん]	
\\	俳優	
\\	俳[はい]優[ゆう]	ハイユー
\\	看護婦	
\\	看[かん]護[ご]婦[ふ]	カンコ°ꜜフ
\\	野球選手	
\\	野[や]球[きゅう]選[せん]手[しゅ]	
\\	大統領	
\\	大[だい]統[とう]領[りょう]	ダイトꜜーリョー
\\	ジャーナリスト	
\\	ジャーナリスト	ジャーナリꜜスト
\\	女優	
\\	女[じょ]優[ゆう]	ジョユー
\\	消防士	
\\	消[しょう]防[ぼう]士[し]	ショーボꜜーシ
\\	今日は良い天気だから、散歩したいです。	
\\	今[きょ]日[う]は 良[い]い 天[てん]気[き]だから、 散[さん]歩[ぽ]したいです。
\\	明日は休みだから、山に登りたいです。	
\\	明日[あした]は 休[やす]みだから、 山[やま]に 登[のぼ]りたいです。
\\	疲れたから、出掛けたくありません。	
\\	疲[つか]れたから、 出[で]掛[か]けたくありません。
\\	掃除したり、洗濯したりします。	
\\	掃[そう]除[じ]したり、 洗[せん]濯[たく]したりします。
\\	ジョンさんはお寺を見たり、美術館に行ったりしました。	
\\	ジョンさんはお 寺[てら]を 見[み]たり、 美[び]術[じゅつ]館[かん]に 行[い]ったりしました。
\\	たけしさんはキャンプに行ったり、ドライブに行ったりしました。	
\\	たけしさんはキャンプに 行[い]ったり、 ドライブに 行[い]ったりしました。
\\	きょうこさんはお菓子を作ったり、家で本を読んだりしました。	
\\	きょうこさんはお 菓[か]子[し]を 作[つく]ったり、 家[うち]で 本[ほん]を 読[よ]んだりしました。
\\	スーさんは広島に遊びに行ったり、食べに行ったりしました。	
\\	スーさんは 広[ひろ]島[しま]に 遊[あそ]びに 行[い]ったり、 食[た]べに 行[い]ったりしました。
\\	けんさんは部屋を掃除したり、洗濯したりしました。	
\\	けんさんは 部[へ]屋[や]を 掃[そう]除[じ]したり、 洗[せん]濯[たく]したりしました。
\\	ロバートさんは友達と会ったり、ビデオを見たりしました。	
\\	ロバートさんは 友[とも]達[だち]と 会[あ]ったり、ビデオを 見[み]たりしました。
\\	山下先生は温泉に行ったり、休んだりしました。	
\\	山[やま]下[した]先[せん]生[せい]は 温[おん]泉[せん]に 行[い]ったり、 休[やす]んだりしました。
\\	日本で何がしたいですか。 
\\	日本のお菓子を食べたり、富士山に登ったりしたいです。	
\\	日[に]本[ほん]で 何[なに]がしたいですか。 
\\	日[に]本[ほん]のお 菓[か]子[し]を 食[た]べたり、 富[ふ]士[じ]山[さん]に 登[のぼ]ったりしたいです。
\\	週末良く何をしますか。 
\\	週末良く本を読んだり、音楽を聞いたりします。	
\\	週[しゅう]末[まつ]良[よ]く 何[なに]をしますか。 
\\	週[しゅう]末[まつ]良[よ]く 本[ほん]を 読[よ]んだり、 音[おん]楽[がく]を 聞[き]いたりします。
\\	デートの時、何をしますか。 
\\	デートの時、沢山踊ったり、お酒を沢山飲んだりします。	
\\	デートの 時[とき]、 何[なに]をしますか。 
\\	デートの 時[とき]、 沢[たく]山[さん]踊[おど]ったり、 お 酒[さけ]を 沢[たく]山[さん]飲[の]んだりします。
\\	貴方の国ではお正月何をしますか。 
\\	私の国でお正月家族に会ったり、一緒に歌ったりします。	
\\	貴方[あなた]の 国[くに]ではお 正[しょう]月[がつ]何[なに]をしますか。 
\\	私[わたし]の 国[くに]でお 正[しょう]月[がつ]家[か]族[ぞく]に 会[あ]ったり、 一[いっ]緒[しょ]に 歌[うた]ったりします。
\\	子供の時、良く何をしましたか。 
\\	兄弟と遊んだり、お母さんを手伝ったりしました。	
\\	子[こ]供[ども]の 時[とき]、 良[よ]く 何[なに]をしましたか。 
\\	兄[きょう]弟[だい]と 遊[あそ]んだり、 お 母[かあ]さんを 手[て]伝[つだ]ったりしました。
\\	日本で何をしましたか。 
\\	学校で教えたり、日本語を沢山勉強したりしました。	
\\	日[に]本[ほん]で 何[なに]をしましたか。 
\\	学[がっ]校[こう]で 教[おし]えたり、 日[に]本[ほん]語[ご]を 沢[たく]山[さん]勉[べん]強[きょう]したりしました。
\\	夏休みに何をしましたか。 
\\	夏休みに泳いだり、水を沢山飲んだりしました。	
\\	夏[なつ]休[やす]みに 何[なに]をしましたか。 
\\	夏[なつ]休[やす]みに 泳[およ]いだり、 水[みず]を 沢[たく]山[さん]飲[の]んだりしました。
\\	有名人に会ったことが有りますか。	
\\	有[ゆう]名[めい]人[じん]に 会[あ]ったことが 有[あ]りますか。
\\	天ぷらを食べたことが有ります。	
\\	天[てん]ぷらを 食[た]べたことが 有[あ]ります。
\\	東京に行ったことが有りません。	
\\	東[とう]京[きょう]に 行[い]ったことが 有[あ]りません。
\\	寿司を食べたことが有ります。	
\\	寿[す]司[し]を 食[た]べたことが 有[あ]ります。
\\	仏語を勉強したことが有ります。	
\\	仏[ふつ]語[ご]を 勉[べん]強[きょう]したことが 有[あ]ります。
\\	レストランで働いたことが有ります。	
\\	レストランで 働[はたら]いたことが 有[あ]ります。
\\	広島に行ったことが有りません。	
\\	広[ひろ]島[しま]に 行[い]ったことが 有[あ]りません。
\\	ラブレターを書いたことが有りません。	
\\	ラブレターを 書[か]いたことが 有[あ]りません。
\\	授業で寝たことが有ります。	
\\	授[じゅ]業[ぎょう]で 寝[ね]たことが 有[あ]ります。
\\	富士山に登ったことが有ります。	
\\	富[ふ]士[じ]山[さん]に 登[のぼ]ったことが 有[あ]ります。
\\	日本で車を運転したことが有りません。	
\\	日[に]本[ほん]で 車[くるま]を 運[うん]転[てん]したことが 有[あ]りません。
\\	日本の映画を見たことが有りません。	
\\	日[に]本[ほん]の 映[えい]画[が]を 見[み]たことが 有[あ]りません。
\\	日本のお酒を飲んだことが有りますか。 
\\	はい、有ります。美味しかったです。	
\\	日[に]本[ほん]のお 酒[さけ]を 飲[の]んだことが 有[あ]りますか。 
\\	はい、 有[あ]ります。 美[お]味[い]しかったです。
\\	ダイエットをしたことが有りますか。 
\\	いいえ、有りません。何時も痩せていましたから。	
\\	ダイエットをしたことが 有[あ]りますか。 
\\	いいえ、 有[あ]りません。 何[い]時[つ]も 痩[や]せていましたから。
\\	試験で零点取ったことが有りますか。 
\\	いいえ、有りません。何時も沢山勉強していましたから。	
\\	試[し]験[けん]で 零[れい]点[てん]取[と]ったことが 有[あ]りますか。 
\\	いいえ、 有[あ]りません。 何[い]時[つ]も 沢[たく]山[さん]勉[べん]強[きょう]していましたから。
\\	英語を教えたことが有りますか。 
\\	はい、有りま。教えるのが好きですから。	
\\	英[えい]語[ご]を 教[おし]えたことが 有[あ]りますか。 
\\	はい、 有[あ]りま。 教[おし]えるのが 好[す]きですから。
\\	有名人に会ったことが有りますか。 
\\	はい、有ります。天皇に会ったことが有ります。	
\\	有[ゆう]名[めい]人[じん]に 会[あ]ったことが 有[あ]りますか。 
\\	はい、 有[あ]ります。 天[てん]皇[のう]に 会[あ]ったことが 有[あ]ります。
\\	中国語	
\\	中[ちゅう]国[ごく]語[ご]
\\	河豚	
\\	河[ふ]豚[ぐ]	フꜜク°
\\	河豚を食べたことが有りますか。 
\\	はい、有りま。美味しかったです。	
\\	河[ふ]豚[ぐ]を 食[た]べたことが 有[あ]りますか。 
\\	はい、 有[あ]りま。 美[お]味[い]しかったです。
\\	中国語を勉強したことが有りますか。 
\\	はい、有りま。難しくて大変でした。	
\\	中[ちゅう]国[ごく]語[ご]を 勉[べん]強[きょう]したことが 有[あ]りますか。 
\\	はい、 有[あ]りま。 難[むずか]しくて 大[たい]変[へん]でした。
\\	新幹線に乗ったことが有りますか。 
\\	いいえ、有りません。日本に居たことが有りませんから。	
\\	新[しん]幹[かん]線[せん]に 乗[の]ったことが 有[あ]りますか。 
\\	いいえ、 有[あ]りません。 日[に]本[ほん]に 居[い]たことが 有[あ]りませんから。
\\	嘘をついたことが有りますか。 
\\	いいえ、有りません。何時も嘘をつきますから。	
\\	嘘[うそ]をついたことが 有[あ]りますか。 
\\	いいえ、 有[あ]りません。 何[い]時[つ]も 嘘[うそ]をつきますから。
\\	日本料理を作ったことが有りますか。 
\\	はい、有りま。それに今晩河豚を作るつもりです。	
\\	日[に]本[ほん]料[りょう]理[り]を 作[つく]ったことが 有[あ]りますか。 
\\	はい、 有[あ]りま。それに 今[こん]晩[ばん]河[ふ]豚[ぐ]を 作[つく]るつもりです。
\\	遅刻したことが有りますか。	
\\	遅[ち]刻[こく]したことが 有[あ]りますか。
\\	授業をサボったことが有りますか。 
\\	はい、有りま。高校の時良くサボりました。それに今日もサボるつもりです。	
\\	授[じゅ]業[ぎょう]を サ[さ]ボ[ぼ]ったことが 有[あ]りますか。 
\\	はい、 有[あ]りま。 高[こう]校[こう]の 時[とき]良[よ]く サ[さ]ボ[ぼ]りました。それに 今日[きょう]も サ[さ]ボ[ぼ]るつもりです。
\\	友達と喧嘩したことが有りますか。 
\\	いいえ、有りません。喧嘩をするのが大嫌いですから。	
\\	友[とも]達[だち]と 喧[けん]嘩[か]したことが 有[あ]りますか。 
\\	いいえ、 有[あ]りません。 喧[けん]嘩[か]をするのが 大[だい]嫌[きら]いですから。
\\	寿司や天ぷらを良く食べます。	
\\	寿[す]司[し]や 天[てん]ぷらを 良[よ]く 食[た]べます。
\\	野球やサッカーを良く見ます。	
\\	野[や]球[きゅう]やサッカーを 良[よ]く 見[み]ます。
\\	韓国料理やポーランド料理を良く作ります。	
\\	韓[かん]国[こく]料[りょう]理[り]やポーランド 料[りょう]理[り]を 良[よ]く 作[つく]ります。
\\	貴方の大学の食堂には、どんな食べ物が有りますか。	
\\	貴方[あなた]の 大[だい]学[がく]の 食[しょく]堂[どう]には、どんな 食[た]べ 物[もの]が 有[あ]りますか。
\\	貴方の大学には、何処の国の人が居ますか。	
\\	貴方[あなた]の 大[だい]学[がく]には、 何[ど]処[こ]の 国[くに]の 人[ひと]が 居[い]ますか。
\\	私の大学には、ポーランド人やドイツ人が居ます。	
\\	私[わたし]の 大[だい]学[がく]には、ポーランド 人[じん]やドイツ 人[じん]が 居[い]ます。
\\	外国に行ったことが有りますか。何処ですか。	
\\	外[がい]国[こく]に 行[い]ったことが 有[あ]りますか。 何[ど]処[こ]ですか。
\\	イギリスやスペイン迄行ったことが有ります。	
\\	イギリスやスペイン 迄[まで]行[い]ったことが 有[あ]ります。
\\	今、どんな授業を取っていますか。	
\\	今[いま]、どんな 授[じゅ]業[ぎょう]を 取[と]っていますか。
\\	俳優の中で、誰が一番好きですか。	
\\	俳[はい]優[ゆう]の 中[なか]で、 誰[だれ]が 一[いち]番[ばん]好[す]きですか。
\\	歌手の中で、誰が一番好きですか。	
\\	歌[か]手[しゅ]の 中[なか]で、 誰[だれ]が 一[いち]番[ばん]好[す]きですか。
\\	貴方の夢は何ですか。 
\\	私は将来、お金持ちに成りたいです。そして色々な国に行きたいです。	
\\	貴方[あなた]の 夢[ゆめ]は 何[なん]ですか。 
\\	私[わたし]は 将[しょう]来[らい]、お 金[かね]持[も]ちに 成[な]りたいです。そして 色[いろ]々[いろ]な 国[くに]に 行[い]きたいです。
\\	色々な国	
\\	色[いろ]々[いろ]な 国[くに]
\\	子供の時の夢は何でしたか。 
\\	子供の時、歌手に成りたかったです。	
\\	子[こ]供[ども]の 時[とき]の 夢[ゆめ]は 何[なん]でしたか。 
\\	子[こ]供[ども]の 時[とき]、 歌[か]手[しゅ]に 成[な]りたかったです。
\\	私はニューヨークの出身です。ニューヨークは迚も大きくて賑やかです。	
\\	私[わたし]はニューヨークの 出[しゅっ]身[しん]です。ニューヨークは 迚[とて]も 大[おお]きくて 賑[にぎ]やかです。
\\	綺麗な公園や有名な美術館や沢山劇場が有ります。	
\\	綺[き]麗[れい]な 公[こう]園[えん]や 有[ゆう]名[めい]な 美[び]術[じゅつ]館[かん]や 沢[たく]山[さん]劇[げき]場[じょう]が 有[あ]ります。
\\	劇場	
\\	劇[げき]場[じょう]	ゲキジョー
\\	ミュージカル	
\\	ミュージカル
\\	良くミュージカルを見たり、公園で散歩したりしました。	
\\	良[よ]くミュージカルを 見[み]たり、 公[こう]園[えん]で 散[さん]歩[ぽ]したりしました。
\\	夏休みに帰って、友達に会いたいです。	
\\	夏[なつ]休[やす]みに 帰[かえ]って、 友[とも]達[だち]に 会[あ]いたいです。
\\	何方でも良いです。	
\\	何方[どちら]でも 良[い]いです。
\\	同じです。	
\\	同[おな]じです。
\\	手を上げて下さい。	
\\	手[て]を 上[あ]げて 下[くだ]さい。
\\	読んで来て下さい。	
\\	読[よ]んで 来[き]て 下[くだ]さい。
\\	宿題を出して下さい。	
\\	宿[しゅく]題[だい]を 出[だ]して 下[くだ]さい。
\\	教科書を閉じて下さい。	
\\	教[きょう]科[か]書[しょ]を 閉[と]じて 下[くだ]さい。
\\	閉じる 
\\	閉じて 
\\	閉じます	
\\	閉[と]じる 
\\	閉[と]じて 
\\	閉[と]じます
\\	隣の人に聞いて下さい。	
\\	隣[となり]の 人[ひと]に 聞[き]いて 下[くだ]さい。
\\	辞めて下さい。	
\\	辞[や]めて 下[くだ]さい。
\\	今日は此れで終わります。	
\\	今[きょ]日[う]は 此[こ]れで 終[お]わります。
\\	迚も良い教科書ですね。 
\\	ええ。私の大学の先生が書いたんです。	
\\	迚[とて]も 良[い]い 教[きょう]科[か]書[しょ]ですね。 
\\	ええ。 私[わたし]の 大[だい]学[がく]の 先[せん]生[せい]が 書[か]いたんです。 (んです 
\\	食べ過ぎては行けません。	
\\	食[た]べ 過[す]ぎては 行[い]けません。
\\	早く起き過ぎました。	
\\	早[はや]く 起[お]き 過[す]ぎました。
\\	過ぎる 
\\	過ぎて 
\\	過ぎます	
\\	過[す]ぎる 
\\	過[す]ぎて 
\\	過[す]ぎます
\\	此の本は高過ぎます。	
\\	此[こ]の 本[ほん]は 高[たか]過[す]ぎます。
\\	彼の人は親切過ぎます。	
\\	彼[あ]の 人[ひと]は 親[しん]切[せつ]過[す]ぎます。
\\	もっと野菜を食べた方が良いですよ。	
\\	もっと 野[や]菜[さい]を 食[た]べた 方[ほう]が 良[い]いですよ。
\\	授業を休まない方が良いですよ。	
\\	授[じゅ]業[ぎょう]を 休[やす]まない方[ほう]が 良[い]いですよ。
\\	何時も日本語を話すので、日本語が上手に成りました。	
\\	何[い]時[つ]も 日[に]本[ほん]語[ご]を 話[はな]すので、 日[に]本[ほん]語[ご]が 上[じょう]手[ず]に 成[な]りました。
\\	宿題が沢山有ったので、昨日の夜、寝ませんでした。	
\\	宿[しゅく]題[だい]が 沢[たく]山[さん]有[あ]ったので、 昨[きの]日[う]の 夜[よる]、 寝[ね]ませんでした。
\\	其の人は意地悪なので、嫌いです。	
\\	其[そ]の 人[ひと]は 意[い]地[じ]悪[わる]なので、 嫌[きら]いです。
\\	今日は日曜日なので、銀行は休みです。	
\\	今[きょ]日[う]は 日[にち]曜[よう]日[び]なので、 銀[ぎん]行[こう]は 休[やす]みです。
\\	来週試験が有るから、沢山勉強しなくちゃ行けません。	
\\	来[らい]週[しゅう]試[し]験[けん]が 有[あ]るから、 沢[たく]山[さん]勉[べん]強[きょう]しなくちゃ 行[い]けません。
\\	~なくちゃ行けません	
\\	~なくちゃ 行[い]けません
\\	食べなくちゃ行けません	
\\	食[た]べなくちゃ 行[い]けません
\\	言わなくちゃ行けません	
\\	言[い]わなくちゃ 行[い]けません
\\	しなくちゃ行けません	
\\	しなくちゃ 行[い]けません
\\	来なくちゃ行けません	
\\	来[こ]なくちゃ 行[い]けません
\\	今朝六時に起きなくちゃ行けませんでした。	
\\	今[け]朝[さ]六[ろく]時[じ]に 起[お]きなくちゃ 行[い]けませんでした。
\\	食べなくちゃ行けませんでした	
\\	食[た]べなくちゃ 行[い]けませんでした
\\	言わなくちゃ行けませんでした	
\\	言[い]わなくちゃ 行[い]けませんでした
\\	しなくちゃ行けませんでした	
\\	しなくちゃ 行[い]けませんでした
\\	来なくちゃ行けませんでした	
\\	来[こ]なくちゃ 行[い]けませんでした
\\	毎日、練習しなくちゃ行けないんです。	
\\	毎[まい]日[にち]、 練[れん]習[しゅう]しなくちゃ 行[い]けないんです。
\\	明日は雨が降るでしょう。 明日は雨が降らないでしょう。	
\\	明日[あした]は 雨[あめ]が 降[ふ]るでしょう。 明日[あした]は 雨[あめ]が 降[ふ]らないでしょう。
\\	北海道	
\\	北[ほっ]海[かい]道[どう]	ホッカꜜイドー
\\	北海道は寒いでしょう。 北海道は寒くないでしょう。	
\\	北[ほっ]海[かい]道[どう]は 寒[さむ]いでしょう。 北[ほっ]海[かい]道[どう]は 寒[さむ]くないでしょう。
\\	山下先生は魚が好きでしょう。 山下先生は魚が好きじゃないでしょう。	
\\	山[やま]下[した]先[せん]生[せい]は 魚[さかな]が 好[す]きでしょう。 山[やま]下[した]先[せん]生[せい]は 魚[さかな]が 好[す]きじゃないでしょう。
\\	彼の人はドイツ人でしょう。 彼の人はドイツ人じゃないでしょう。	
\\	彼[あ]の 人[ひと]はドイツ 人[じん]でしょう。 彼[あ]の 人[ひと]はドイツ 人[じん]じゃないでしょう。
\\	日本語と韓国語と、どっちの方が難しいでしょうか。	
\\	日[に]本[ほん]語[ご]と 韓[かん]国[こく]語[ご]と、どっちの 方[ほう]が 難[むずか]しいでしょうか。
\\	たけしさんは興味が有るだろうと思います。	
\\	たけしさんは 興[きょう]味[み]が 有[あ]るだろうと 思[おも]います。
\\	ジョン、中国語が分かるでしょ?此れ、読んで。	
\\	ジョン、 中[ちゅう]国[ごく]語[ご]が 分[わ]かるでしょ? 此[こ]れ、 読[よ]んで。
\\	如何したんですか。 
\\	頭が痛いんです。	
\\	如[ど]何[う]したんですか。 
\\	頭[あたま]が 痛[いた]いんです。
\\	如何したんですか。 
\\	彼から電話が有ったんです。	
\\	如[ど]何[う]したんですか。 
\\	彼[かれ]から 電[でん]話[わ]が 有[あ]ったんです。
\\	如何したんですか。 
\\	プレゼントを貰ったんです。	
\\	如[ど]何[う]したんですか。 
\\	プレゼントを 貰[もら]ったんです。
\\	如何したんですか。 
\\	明日は休みなんです。	
\\	如[ど]何[う]したんですか。 
\\	明日[あした]は 休[やす]みなんです。
\\	如何したんですか。 
\\	昨日は誕生日だったんです。	
\\	如[ど]何[う]したんですか。 
\\	昨[きの]日[う]は 誕[たん]生[じょう]日[び]だったんです。
\\	如何したんですか。 
\\	テストが難しくなかったんです。	
\\	如[ど]何[う]したんですか。 
\\	テストが 難[むずか]しくなかったんです。
\\	如何したんですか。 
\\	喉が痛いんです。	
\\	如[ど]何[う]したんですか。 
\\	喉[のど]が 痛[いた]いんです。
\\	如何したんですか。 
\\	風邪を引いたんです。	
\\	如[ど]何[う]したんですか。 
\\	風邪[かぜ]を 引[ひ]いたんです。
\\	如何したんですか。 
\\	切符を無くしたんです。	
\\	如[ど]何[う]したんですか。 
\\	切[きっ]符[ぷ]を 無[な]くしたんです。
\\	如何したんですか。 
\\	明日テストが有るんです。	
\\	如[ど]何[う]したんですか。 
\\	明日[あした]テストが 有[あ]るんです。
\\	如何したんですか。 
\\	咳が出るんです。	
\\	如[ど]何[う]したんですか。 
\\	咳[せき]が 出[で]るんです。
\\	如何したんですか。 
\\	彼女と別れたんです。	
\\	如[ど]何[う]したんですか。 
\\	彼[かの]女[じょ]と 別[わか]れたんです。
\\	素敵な車ですね。 
\\	父のなんです。	
\\	素[す]敵[てき]な 車[くるま]ですね。 
\\	父[ちち]のなんです。
\\	綺麗な花ですね。 
\\	友達に貰ったんです。	
\\	綺[き]麗[れい]な 花[はな]ですね。 
\\	友[とも]達[だち]に 貰[もら]ったんです。
\\	新しい靴ですね。 
\\	イタリアのなんです。	
\\	新[あたら]しい 靴[くつ]ですね。 
\\	イタリアのなんです。
\\	イタリア	
\\	イタリア	
\\	可愛い服ですね。 
\\	自分で作ったんです。	
\\	可愛[かわい]い 服[ふく]ですね。 
\\	自[じ]分[ぶん]で 作[つく]ったんです。
\\	素敵なセーターですね。 
\\	母が作ったんです。	
\\	素[す]敵[てき]なセーターですね。 
\\	母[はは]が 作[つく]ったんです。
\\	良い鞄ですね。 
\\	安かったんです。	
\\	良[い]い 鞄[かばん]ですね。 
\\	安[やす]かったんです。
\\	格好いい彼ですね。 
\\	親切なんです。	
\\	格[かっ]好[こ]いい 彼[かれ]ですね。 
\\	親[しん]切[せつ]なんです。
\\	素敵な時計ですね。 
\\	友達に貰ったんです。	
\\	素[す]敵[てき]な 時[と]計[けい]ですね。 
\\	友[とも]達[だち]に 貰[もら]ったんです。
\\	先週東京に行きました。 
\\	如何して行ったんですか。 
\\	母がアメリカから来たんです。	
\\	先[せん]週[しゅう]東[とう]京[きょう]に 行[い]きました。 
\\	如[ど]何[う]して 行[い]ったんですか。 
\\	母[はは]がアメリカから 来[き]たんです。
\\	長野	
\\	長[なが]野[の]	ナꜜカ°ノ
\\	迚も疲れました。 
\\	如何して疲れたんですか。 
\\	沢山働いたんです。	
\\	迚[とて]も 疲[つか]れました。 
\\	如[ど]何[う]して 疲[つか]れたんですか。 
\\	沢[たく]山[さん]働[はたら]いたんです。
\\	お金が全然有りません。 
\\	如何してお金が全然ないんですか。 
\\	本を沢山買ったんです。	
\\	お 金[かね]が 全[ぜん]然[ぜん]有[あ]りません。 
\\	如[ど]何[う]してお 金[かね]が 全[ぜん]然[ぜん]ないんですか。 
\\	本[ほん]を 沢[たく]山[さん]買[か]ったんです。
\\	今日の午後	
\\	今日[きょう]の 午[ご]後[ご]
\\	三年後	
\\	三[さん]年[ねん]後[ご]
\\	鎌倉行きの電車	
\\	鎌[かま]倉[くら]行[い]きの 電[でん]車[しゃ]
\\	鑿	
\\	鑿[のみ]	ノꜜミ
\\	北海	
\\	北[ほっ]海[かい]	
\\	茶道 
\\	茶の湯	
\\	茶[さ]道[どう] 
\\	茶[ちゃ]の 湯[ゆ]	
\\	薄茶	
\\	薄[うす]茶[ちゃ]	ウスチャ
\\	濃茶	
\\	濃[こい]茶[ちゃ]	
\\	抹茶	
\\	抹[まっ]茶[ちゃ]	マッチャ
\\	茶杓	
\\	茶[ちゃ]杓[しゃく]	チャシャク
\\	杓	
\\	杓[しゃく]	
\\	茶碗	
\\	茶[ちゃ]碗[わん]	チャワン
\\	鎌	
\\	鎌[かま]	カꜜマ
\\	厚い	
\\	厚[あつ]い 
\\	暑, 
\\	熱)	アツイ
\\	危ない	
\\	危[あぶ]ない	アブナイ 
\\	アブナꜜイ
\\	家庭	
\\	家[か]庭[てい]	カテイ
\\	花瓶	
\\	花[か]瓶[びん]	カビン
\\	軽い	
\\	軽[かる]い	カルイ
\\	川 
\\	河	
\\	川[かわ] 
\\	河[かわ]	
\\	牛肉	
\\	牛[ぎゅう]肉[にく]	ギューニク
\\	キロ 
\\	キログラム	
\\	キロ 
\\	キログラム	
\\	キロ 
\\	キロメートル	
\\	キロ 
\\	キロメートル	
\\	果物	
\\	果[くだ]物[もの]	クダꜜモノ
\\	曇る 
\\	曇って 
\\	曇ります	
\\	曇[くも]る 
\\	曇[くも]って 
\\	曇[くも]ります	
\\	暗い	
\\	暗[くら]い	クライ
\\	玄関	
\\	玄[げん]関[かん]	ゲꜜンカン
\\	交差点	
\\	交[こう]差[さ]点[てん]	コーサテン 
\\	コーサꜜテン
\\	交番	
\\	交[こう]番[ばん]	コーバン
\\	声	
\\	声[こえ]	コꜜエ
\\	コート	
\\	コート	コꜜート
\\	答える 
\\	答えて 
\\	答えます	
\\	答[こた]える 
\\	答[こた]えて 
\\	答[こた]えます	
\\	差す 
\\	差して 
\\	差します	
\\	差[さ]す 
\\	差[さ]して 
\\	差[さ]します 
\\	刺す)	
\\	塩	
\\	塩[しお]	シオꜜ
\\	然し	
\\	然[しか]し	
\\	自動車	
\\	自[じ]動[どう]車[しゃ]	ジドꜜーシャ 
\\	ジドーシャ
\\	締める 
\\	締めて 
\\	締めます	
\\	締[し]める 
\\	締[し]めて 
\\	締[し]めます 
\\	閉める)	
\\	シャワー	
\\	シャワー	シャꜜワー
\\	丈夫(な)	
\\	丈[じょう]夫[ぶ](な)	
\\	生徒	
\\	生[せい]徒[と]	セꜜイト
\\	背広	
\\	背[せ]広[びろ]	セビロ
\\	空	
\\	空[そら]	ウロ 
\\	カラꜜ 
\\	クꜜー 
\\	クー 
\\	ソꜜラ
\\	大使館	
\\	大[たい]使[し]館[かん]	タイシꜜカン
\\	台所	
\\	台[だい]所[どころ]	ダイドコロ
\\	建物	
\\	建[たて]物[もの]	タテꜜモノ 
\\	タテモꜜノ
\\	卵	
\\	卵[たまご]	タマꜜコ° 
\\	タマコ° 
\\	ラꜜン
\\	段々	
\\	段[だん]々[だん]	ダꜜンダン 
\\	ダンダン
\\	飛ぶ 
\\	飛んで 
\\	飛びます	
\\	飛[と]ぶ 
\\	飛[と]んで 
\\	飛[と]びます	
\\	鳥	
\\	鳥[とり]	トリ
\\	鳥肉	
\\	鳥[とり]肉[にく]	トリニク
\\	鳴く 
\\	鳴いて 
\\	鳴きます	
\\	鳴[な]く 
\\	鳴[な]いて 
\\	鳴[な]きます 
\\	泣く)	
\\	等	
\\	等[など]	ナꜜド
\\	並ぶ 
\\	並んで 
\\	並びます	
\\	並[なら]ぶ 
\\	並[なら]んで 
\\	並[なら]びます	
\\	並べる 
\\	並べて 
\\	並べます	
\\	並[なら]べる 
\\	並[なら]べて 
\\	並[なら]べます	
\\	温い	
\\	温[ぬる]い	ヌクꜜイ 
\\	ヌルꜜイ
\\	灰皿	
\\	灰[はい]皿[ざら]	ハイザラ 
\\	ハイザꜜラ
\\	箱	
\\	箱[はこ]	ハコ
\\	橋	
\\	橋[はし] 
\\	箸)	ハシꜜ
\\	初め 
\\	始め	
\\	初[はじ]め 
\\	始[はじ]め	
\\	バター	
\\	バター	バꜜター
\\	ハンカチ	
\\	ハンカチ	ハンカチ 
\\	ハンカꜜチ
\\	低い	
\\	低[ひく]い	ヒクꜜイ
\\	昼	
\\	昼[ひる]	ヒルꜜ
\\	封筒	
\\	封[ふう]筒[とう]	フートー
\\	フォーク	
\\	フォーク	フォꜜーク
\\	豚肉	
\\	豚[ぶた]肉[にく]	トンニク 
\\	ブタニク
\\	降る 
\\	降って 
\\	降ります	
\\	降[ふ]る 
\\	降[ふ]って 
\\	降[ふ]ります 
\\	振る)	
\\	文章	
\\	文[ぶん]章[しょう]	ブꜜンショー
\\	辺	
\\	辺[へん] 
\\	変)	ヘꜜン 
\\	ヘン
\\	ポケット	
\\	ポケット	ポケꜜット 
\\	ポꜜケット
\\	細い	
\\	細[ほそ]い	ホソꜜイ
\\	ボタン	
\\	ボタン	ボタン 
\\	ボꜜタン
\\	本棚	
\\	本[ほん]棚[だな]	ホꜜンダナ
\\	毎年 
\\	毎年	
\\	毎[まい]年[ねん] 
\\	毎[まい]年[とし]	
\\	丸い 
\\	円い	
\\	丸[まる]い 
\\	円[まる]い	
\\	万年筆	
\\	万[まん]年[ねん]筆[ひつ]	マンネꜜンヒツ
\\	向こう	
\\	向[む]こう	ムコꜜー 
\\	ムコー
\\	村	
\\	村[むら]	ソꜜン 
\\	ムラꜜ
\\	メートル	
\\	メートル	メートル
\\	門	
\\	門[もん]	カꜜド 
\\	モꜜン
\\	問題	
\\	問[もん]題[だい]	モンダイ
\\	八百屋	
\\	八[や]百[お]屋[や]	ヤオヤ
\\	洋服	
\\	洋[よう]服[ふく]	ヨーフク
\\	弱い	
\\	弱[よわ]い	ヨワꜜイ
\\	立派	
\\	立[りっ]派[ぱ]	リッパ
\\	廊下	
\\	廊[ろう]下[か]	ローカ
\\	ワイシャツ	
\\	ワイシャツ	ワイシャツ
\\	渡す 
\\	渡して 
\\	渡します	
\\	渡[わた]す 
\\	渡[わた]して 
\\	渡[わた]します	
\\	浅い	
\\	浅[あさ]い	アサイ
\\	味	
\\	味[あじ]	アジ
\\	集る 
\\	集って 
\\	集ります	
\\	集[あつま]る 
\\	集[あつま]って 
\\	集[あつま]ります	
\\	安心	
\\	安[あん]心[しん]	アンシン
\\	以下	
\\	以[い]下[か]	イꜜカ
\\	以外	
\\	以[い]外[がい] 
\\	意外)	イꜜカ°イ
\\	医学	
\\	医[い]学[がく]	イꜜカ°ク
\\	生きる 
\\	生きて 
\\	生きます	
\\	生[い]きる 
\\	生[い]きて 
\\	生[い]きます	
\\	意見	
\\	意[い]見[けん]	イꜜケン
\\	石	
\\	石[いし]	イシꜜ
\\	以上	
\\	以[い]上[じょう]	イꜜジョー
\\	一度	
\\	一[いち]度[ど]	イチꜜド 
\\	イチドꜜ 
\\	イチド 
\\	ヒトꜜタビ
\\	一生懸命	
\\	一[いっ]生[しょう]懸[けん]命[めい]	イッショーケꜜンメイ
\\	一杯	
\\	一[いっ]杯[ぱい]	イꜜッパイ
\\	糸	
\\	糸[いと]	イꜜト
\\	以内	
\\	以[い]内[ない] 
\\	田舎)	イꜜナイ
\\	田舎	
\\	田舎[いなか] 
\\	以内)	イナカ
\\	祈る 
\\	祈って 
\\	祈ります	
\\	祈[いの]る 
\\	祈[いの]って 
\\	祈[いの]ります	
\\	植える 
\\	植えて 
\\	植えます	
\\	植[う]える 
\\	植[う]えて 
\\	植[う]えます	
\\	動く 
\\	動いて /動きます	
\\	動[うご]く 
\\	動[うご]いて 
\\	動[うご]きます	
\\	内	
\\	内[うち]	ウチ
\\	美しい	
\\	美[うつく]しい	ウツクシꜜイ
\\	写す 
\\	写して 
\\	写します	
\\	写[うつ]す 
\\	写[うつ]して 
\\	写[うつ]します	
\\	移る 
\\	移って 
\\	移ります	
\\	移[うつ]る 
\\	移[うつ]って 
\\	移[うつ]ります	
\\	腕	
\\	腕[うで]	ウデꜜ 
\\	カꜜイナ 
\\	カイナ
\\	裏	
\\	裏[うら]	ウラꜜ
\\	運転手	
\\	運[うん]転[てん]手[しゅ]	ウンテꜜンシュ
\\	枝	
\\	枝[えだ]	エ 
\\	エダ
\\	お祝い	
\\	お 祝[いわ]い	オイワイ
\\	可笑しい	
\\	可[お]笑[か]しい	オカシꜜイ
\\	億	
\\	億[おく]	オꜜク
\\	屋上	
\\	屋[おく]上[じょう]	
\\	贈り物 
\\	プレゼント	
\\	贈[おく]り 物[もの] 
\\	プレゼント	
\\	押し入れ	
\\	押[お]し 入[い]れ	オシイレ
\\	お釣り	
\\	お 釣[つ]り	オツリ
\\	踊り	
\\	踊[おど]り	オドリ
\\	お祭り	
\\	お 祭[まつ]り	
\\	お見舞い	
\\	お 見[み]舞[ま]い	
\\	思い出す 
\\	思い出して 
\\	思い出します	
\\	思[おも]い 出[だ]す 
\\	思[おも]い 出[だ]して 
\\	思[おも]い 出[だ]します	
\\	表	
\\	表[おもて]	オモテꜜ 
\\	ヒョー
\\	泳ぎ方	
\\	泳[およ]ぎ 方[かた]	
\\	折る 
\\	折って 
\\	折ります	
\\	折[お]る 
\\	折[お]って 
\\	折[お]ります	
\\	折れる 
\\	折れて 
\\	折れます	
\\	折[お]れる 
\\	折[お]れて 
\\	折[お]れます	
\\	終わり	
\\	終[お]わり	オワリ
\\	海岸	
\\	海[かい]岸[がん]	カイカ°ン
\\	会議室	
\\	会[かい]議[ぎ]室[しつ]	
\\	会場	
\\	会[かい]場[じょう]	カイジョー
\\	帰り	
\\	帰[かえ]り	カエリꜜ
\\	変える 
\\	変えて 
\\	変えます	
\\	変[か]える 
\\	変[か]えて 
\\	変[か]えます 
\\	帰る 
\\	換える)	
\\	鏡	
\\	鏡[かがみ]	カカ°ミꜜ
\\	飾る 
\\	飾って/ 飾ります	
\\	飾[かざ]る 
\\	飾[かざ]って/ 飾[かざ]ります	
\\	火事	
\\	火[か]事[じ] 
\\	家事)	カꜜジ
\\	ガス	
\\	ガス	ガꜜス
\\	堅い 
\\	硬い 
\\	固い	
\\	堅[かた]い 
\\	硬[かた]い 
\\	固[かた]い	
\\	形	
\\	形[かたち]	カタꜜ 
\\	カタチ 
\\	ナリꜜ
\\	片付ける 
\\	片付けて 
\\	片付けます	
\\	片[かた]付[づ]ける 
\\	片[かた]付[づ]けて 
\\	片[かた]付[づ]けます	
\\	課長	
\\	課[か]長[ちょう]	カチョー
\\	必ず	
\\	必[かなら]ず	カナラズ
\\	壁	
\\	壁[かべ]	カベ
\\	噛む 
\\	噛んで 
\\	噛みます	
\\	噛[か]む 
\\	噛[か]んで 
\\	噛[か]みます	
\\	通う 
\\	通って 
\\	通います	
\\	通[かよ]う 
\\	通[かよ]って 
\\	通[かよ]います	
\\	乾く 
\\	乾いて 
\\	乾きます	
\\	乾[かわ]く 
\\	乾[かわ]いて 
\\	乾[かわ]きます 
\\	渇く)	
\\	代わり	
\\	代[か]わり	カワリ
\\	関係	
\\	関[かん]係[けい]	カンケイ
\\	気	
\\	気[き] 
\\	木)	キ 
\\	ケ 
\\	ケꜜ
\\	機会	
\\	機[き]会[かい] 
\\	機械)	キカꜜイ 
\\	キカイ
\\	危険	
\\	危[き]険[けん]	キケン
\\	汽車	
\\	汽[き]車[しゃ]	キシャꜜ 
\\	キꜜシャ
\\	技術	
\\	技[ぎ]術[じゅつ]	ギꜜジュツ
\\	規則	
\\	規[き]則[そく]	キꜜソク 
\\	キソꜜク
\\	屹度	
\\	屹[きっ]度[と]	
\\	絹	
\\	絹[きぬ]	キꜜヌ
\\	気分	
\\	気[き]分[ぶん]	キꜜブン
\\	君	
\\	君[きみ]	キミ
\\	気持ち	
\\	気[き]持[も]ち	
\\	客 
\\	お客さん	
\\	客[きゃく] 
\\	お 客[きゃく]さん	
\\	急	
\\	急[きゅう] 
\\	九)	キュー
\\	教育	
\\	教[きょう]育[いく]	キョーイク
\\	教会	
\\	教[きょう]会[かい] 
\\	協会)	キョーカイ
\\	競争	
\\	競[きょう]争[そう]	キョーソー
\\	興味	
\\	興[きょう]味[み]	キョꜜーミ 
\\	キョーミꜜ
\\	近所	
\\	近[きん]所[じょ]	キꜜンジョ
\\	具合	
\\	具[ぐ]合[あい]	グアイ
\\	草	
\\	草[くさ]	クサꜜ 
\\	ソꜜー
\\	比べる 
\\	比べて 
\\	比べます	
\\	比[くら]べる 
\\	比[くら]べて 
\\	比[くら]べます	
\\	暮れる 
\\	暮れて 
\\	暮れます	
\\	暮[く]れる 
\\	暮[く]れて 
\\	暮[く]れます 
\\	呉れる)	
\\	毛	
\\	毛[け]	ケ 
\\	モꜜー
\\	景色	
\\	景[け]色[しき]	ケꜜシキ
\\	下宿	
\\	下[げ]宿[しゅく]	ゲシュク
\\	決して	
\\	決[けっ]して	ケッシテ
\\	原因	
\\	原[げん]因[いん]	ゲンイン
\\	研究室	
\\	研[けん]究[きゅう]室[しつ]	ケンキュꜜーシツ
\\	郊外	
\\	郊[こう]外[がい]	コꜜーカ°イ
\\	講義	
\\	講[こう]義[ぎ]	コꜜーキ° 
\\	コーキ°ꜜ
\\	工業	
\\	工[こう]業[ぎょう]	コꜜーキ°ョー
\\	校長	
\\	校[こう]長[ちょう]	コーチョー
\\	交通	
\\	交[こう]通[つう]	コーツー
\\	講堂	
\\	講[こう]堂[どう]	コードー
\\	高等学校 
\\	高校	
\\	高[こう]等[とう]学[がっ]校[こう] 
\\	高[こう]校[こう]	
\\	国際	
\\	国[こく]際[さい]	コクサイ
\\	心	
\\	心[こころ]	ココꜜロ 
\\	ココロꜜ 
\\	シꜜン
\\	故障する	
\\	故[こ]障[しょう]する	
\\	故障	
\\	故[こ]障[しょう]	コショー
\\	ご存じ	
\\	ご 存[ぞん]じ	ゴゾꜜンジ
\\	ご馳走	
\\	ご 馳[ち]走[そう]	
\\	小鳥	
\\	小[こ]鳥[とり]	コトリ
\\	細かい	
\\	細[こま]かい	コマカꜜイ
\\	米	
\\	米[こめ]	コメꜜ 
\\	ヨꜜネ
\\	此れから	
\\	此[こ]れから	
\\	今夜	
\\	今[こん]夜[や]	コꜜンヤ
\\	最後	
\\	最[さい]後[ご]	サꜜイコ°
\\	最初	
\\	最[さい]初[しょ]	サイショ
\\	坂	
\\	坂[さか]	サカꜜ
\\	下る 
\\	下って 
\\	下ります	
\\	下[さが]る 
\\	下[さが]って 
\\	下[さが]ります	
\\	盛ん	
\\	盛[さか]ん	サカン
\\	下げる 
\\	下げて 
\\	下げます	
\\	下[さ]げる 
\\	下[さ]げて 
\\	下[さ]げます	
\\	騒ぐ 
\\	騒いで 
\\	騒ぎます	
\\	騒[さわ]ぐ 
\\	騒[さわ]いで 
\\	騒[さわ]ぎます	
\\	産業	
\\	産[さん]業[ぎょう]	サンキ°ョー
\\	市	
\\	市[し]	イꜜチ 
\\	シꜜ
\\	仕方	
\\	仕[し]方[かた]	シカタ
\\	事故	
\\	事[じ]故[こ]	ジꜜコ
\\	時代	
\\	時[じ]代[だい]	ジダイ
\\	下着	
\\	下[した]着[ぎ]	シタキ°
\\	確り	
\\	確[しっか]り	シッカꜜリ
\\	失敗	
\\	失[しっ]敗[ぱい]	シッパイ
\\	品物	
\\	品[しな]物[もの]	シナモノ
\\	暫く	
\\	暫[しばら]く	シバꜜラク
\\	島	
\\	島[しま]	シマꜜ
\\	事務所	
\\	事[じ]務[む]所[しょ]	ジムꜜショ
\\	社会	
\\	社[しゃ]会[かい]	シャꜜカイ
\\	住所	
\\	住[じゅう]所[しょ]	ジュꜜーショ
\\	柔道	
\\	柔[じゅう]道[どう]	ジュꜜードー
\\	"十分 
\\	"十[じゅう]分[ぶん] 
\\	紹介	
\\	紹[しょう]介[かい]	ショーカイ
\\	食料品	
\\	食[しょく]料[りょう]品[ひん]	ショクリョーヒン 
\\	ショクリョꜜーヒン
\\	知らせる 
\\	知らせて 
\\	知らせます	
\\	知[し]らせる 
\\	知[し]らせて 
\\	知[し]らせます	
\\	人口	
\\	人[じん]口[こう]	ジンコー
\\	神社	
\\	神[じん]社[じゃ]	ジꜜンジャ
\\	親切	
\\	親[しん]切[せつ]	シꜜンセツ
\\	新聞社	
\\	新[しん]聞[ぶん]社[しゃ]	シンブꜜンシャ
\\	水泳	
\\	水[すい]泳[えい]	スイエイ
\\	水道	
\\	水[すい]道[どう]	スイドー
\\	数学	
\\	数[すう]学[がく]	スーカ°ク
\\	凄い	
\\	凄[すご]い	スコ°ꜜイ
\\	進む 
\\	進んで 
\\	進みます	
\\	進[すす]む 
\\	進[すす]んで 
\\	進[すす]みます	
\\	すっかり	
\\	すっかり	スッカꜜリ
\\	すっと	
\\	すっと	
\\	砂	
\\	砂[すな]	イサコ° 
\\	スナ
\\	滑る 
\\	滑って 
\\	滑ります	
\\	滑[すべ]る 
\\	滑[すべ]って 
\\	滑[すべ]ります	
\\	隅	
\\	隅[すみ]	スꜜミ
\\	済む 
\\	済んで 
\\	済みます	
\\	済[す]む 
\\	済[す]んで 
\\	済[す]みます 
\\	住む)	
\\	煙草	
\\	煙草[たばこ]	タバコ
\\	すり	
\\	すり	スꜜリ
\\	すると	
\\	すると	スルト
\\	生産する	
\\	生[せい]産[さん]する	
\\	生産	
\\	生[せい]産[さん]	セイサン
\\	西洋	
\\	西[せい]洋[よう]	セꜜイヨー
\\	席	
\\	席[せき] 
\\	咳)	セꜜキ
\\	説明	
\\	説[せつ]明[めい]	セツメイ
\\	線	
\\	線[せん] 
\\	千)	セꜜン
\\	戦争	
\\	戦[せん]争[そう]	センソー
\\	退院	
\\	退[たい]院[いん]	タイイン
\\	退院する	
\\	退[たい]院[いん]する	
\\	大事	
\\	大[だい]事[じ]	ダꜜイジ 
\\	ダイジꜜ 
\\	ダイジ
\\	大体	
\\	大[だい]体[たい]	ダイタイ
\\	大分	
\\	大[だい]分[ぶ]	ダイブ
\\	倒れる 
\\	倒れて 
\\	倒れます	
\\	倒[たお]れる 
\\	倒[たお]れて 
\\	倒[たお]れます	
\\	確か	
\\	確[たし]か	タꜜシカ
\\	足す 
\\	足して 
\\	足します	
\\	足[た]す 
\\	足[た]して 
\\	足[た]します	
\\	訪ねる 
\\	訪ねて 
\\	訪ねます	
\\	訪[たず]ねる 
\\	訪[たず]ねて 
\\	訪[たず]ねます 
\\	尋ねる)	
\\	尋ねる 
\\	尋ねて 
\\	尋ねます	
\\	尋[たず]ねる 
\\	尋[たず]ねて 
\\	尋[たず]ねます 
\\	訪ねる)	
\\	正しい	
\\	正[ただ]しい	タダシꜜイ
\\	畳	
\\	畳[たたみ]	タタミ
\\	"立てる 
\\	立てて 
\\	立てます 
\\	"立[た]てる 
\\	立[た]てて 
\\	立[た]てます 
\\	建てる 
\\	建てる 
\\	建てて 
\\	建てます	
\\	建[た]てる 
\\	建[た]てて 
\\	建[た]てます 
\\	立てる)	
\\	棚	
\\	棚[たな]	タナ
\\	楽しみ	
\\	楽[たの]しみ	タノシꜜミ 
\\	タノシミꜜ 
\\	タノシミ
\\	楽しむ 
\\	楽しんで 
\\	楽しみます	
\\	楽[たの]しむ 
\\	楽[たの]しんで 
\\	楽[たの]しみます	
\\	偶に	
\\	偶[たま]に	
\\	為	
\\	為[ため]	タメꜜ
\\	男性	
\\	男[だん]性[せい]	ダンセイ
\\	暖房	
\\	暖[だん]房[ぼう]	ダンボー
\\	血	
\\	血[ち]	チ
\\	チェックする	
\\	チェックする	
\\	力	
\\	力[ちから]	チカラꜜ
\\	ちゃん	
\\	ちゃん	
\\	注意	
\\	注[ちゅう]意[い]	チュꜜーイ
\\	駐車場	
\\	駐[ちゅう]車[しゃ]場[じょう]	チューシャジョー
\\	地理	
\\	地[ち]理[り]	チꜜリ
\\	捕まえる 
\\	捕まえて 
\\	捕まえます	
\\	捕[つか]まえる 
\\	捕[つか]まえて 
\\	捕[つか]まえます	
\\	付く 
\\	付いて 
\\	付きます	
\\	付[つ]く 
\\	付[つ]いて 
\\	付[つ]きます 
\\	着く 
\\	点く)	
\\	漬ける 
\\	漬けて 
\\	漬けます	
\\	漬[つ]ける 
\\	漬[つ]けて 
\\	漬[つ]けます 
\\	点ける)	
\\	都合	
\\	都[つ]合[ごう]	ツꜜコ°ー 
\\	ツコ°ー
\\	続く 
\\	続いて 
\\	続きます	
\\	続[つづ]く 
\\	続[つづ]いて 
\\	続[つづ]きます	
\\	積もり	
\\	積[つ]もり	
\\	釣る 
\\	釣って 
\\	釣ります	
\\	釣[つ]る 
\\	釣[つ]って 
\\	釣[つ]ります	
\\	適当	
\\	適[てき]当[とう]	テキトー
\\	点	
\\	点[てん]	テン 
\\	ポꜜチ
\\	店員	
\\	店[てん]員[いん]	テンイン
\\	電灯	
\\	電[でん]灯[とう]	デントー
\\	電報	
\\	電[でん]報[ぽう]	デンポー
\\	展覧会	
\\	展[てん]覧[らん]会[かい]	テンラꜜンカイ
\\	都	
\\	都[と]	トꜜ 
\\	ミヤコ
\\	道具	
\\	道[どう]具[ぐ]	ドーク°ꜜ
\\	到頭	
\\	到[とう]頭[とう]	トꜜートー
\\	通る 
\\	通って 
\\	通ります	
\\	通[とお]る 
\\	通[とお]って 
\\	通[とお]ります	
\\	特に	
\\	特[とく]に	トꜜクニ
\\	特別	
\\	特[とく]別[べつ]	トクベツ
\\	途中	
\\	途[と]中[ちゅう]	トチュー
\\	届ける 
\\	届けて 
\\	届けます	
\\	届[とど]ける 
\\	届[とど]けて 
\\	届[とど]けます	
\\	どんどん	
\\	どんどん	ドꜜンドン
\\	直る 
\\	直って 
\\	直ります	
\\	直[なお]る 
\\	直[なお]って 
\\	直[なお]ります 
\\	治る)	
\\	治る 
\\	治って 
\\	治ります	
\\	治[なお]る 
\\	治[なお]って 
\\	治[なお]ります 
\\	直る)	
\\	亡く成る 
\\	亡く成って 
\\	亡く成ります	
\\	死ぬ)	亡[な]く 成[な]る 
\\	亡[な]く 成[な]って 
\\	亡[な]く 成[な]ります 
\\	無く成る)	
\\	投げる 
\\	投げて 
\\	投げます	
\\	投[な]げる 
\\	投[な]げて 
\\	投[な]げます	
\\	鳴る 
\\	鳴って/ 鳴ります	
\\	鳴[な]る 
\\	鳴[な]って/ 鳴[な]ります 
\\	成る)	
\\	なる程	
\\	なる 程[ほど]	
\\	苦い	
\\	苦[にが]い	ニカ°ꜜイ
\\	二階建て	
\\	二[に]階[かい]建[だ]て	
\\	何奴も此奴も	
\\	何[ど]奴[いつ]も 此[こ]奴[いつ]も	
\\	逃げる 
\\	逃げて 
\\	逃げます	
\\	逃[に]げる 
\\	逃[に]げて 
\\	逃[に]げます	
\\	入院	
\\	入[にゅう]院[いん]	ニューイン
\\	入院する	
\\	入[にゅう]院[いん]する	
\\	入学	
\\	入[にゅう]学[がく]	ニューカ°ク
\\	入学する	
\\	入[にゅう]学[がく]する	
\\	似る 
\\	似て 
\\	似ます	
\\	似[に]る 
\\	似[に]て 
\\	似[に]ます	
\\	人形	
\\	人[にん]形[ぎょう]	ニンキ°ョー 
\\	ヒトカ°タ
\\	塗る 
\\	塗って 
\\	塗ります	
\\	塗[ぬ]る 
\\	塗[ぬ]って 
\\	塗[ぬ]ります	
\\	濡れる 
\\	濡れて 
\\	濡れます	
\\	濡[ぬ]れる 
\\	濡[ぬ]れて 
\\	濡[ぬ]れます	
\\	値段	
\\	値[ね]段[だん]	ネダン
\\	熱	
\\	熱[ねつ]	ネツꜜ
\\	寝坊	
\\	寝[ね]坊[ぼう]	ネボー
\\	眠る 
\\	眠って 
\\	眠ります	
\\	眠[ねむ]る 
\\	眠[ねむ]って 
\\	眠[ねむ]ります	
\\	残る 
\\	残って 
\\	残ります	
\\	残[のこ]る 
\\	残[のこ]って 
\\	残[のこ]ります	
\\	乗り換える 
\\	乗り換えて 
\\	乗り換えます	
\\	乗[の]り 換[か]える 
\\	乗[の]り 換[か]えて 
\\	乗[の]り 換[か]えます	
\\	乗り物	
\\	乗[の]り 物[もの]	ノリモノ
\\	葉	
\\	葉[は] 
\\	歯)	ハ 
\\	ヨꜜー
\\	場合	
\\	場[ば]合[あい]	バアイ
\\	倍	
\\	倍[ばい]	バイ 
\\	バꜜイ
\\	拝見する	
\\	拝[はい]見[けん]する	
\\	筈	
\\	筈[はず]	ハズ
\\	はっきり	
\\	はっきり	ハッキꜜリ
\\	花見	
\\	花[はな]見[み]	ハナミꜜ
\\	林	
\\	林[はやし]	ハヤシ 
\\	ハヤシꜜ
\\	反対	
\\	反[はん]対[たい]	ハンタイ
\\	火	
\\	火[ひ] 
\\	日)	ヒꜜ
\\	冷える 
\\	冷えて 
\\	冷えます	
\\	冷[ひ]える 
\\	冷[ひ]えて 
\\	冷[ひ]えます	
\\	光	
\\	光[ひかり]	ヒカリꜜ
\\	光る 
\\	光って 
\\	光ります	
\\	光[ひか]る 
\\	光[ひか]って 
\\	光[ひか]ります	
\\	引き出し	
\\	引[ひ]き 出[だ]し	ヒキダシ
\\	引き出す 
\\	引き出して 
\\	引き出します	
\\	引[ひ]き 出[だ]す 
\\	引[ひ]き 出[だ]して 
\\	引[ひ]き 出[だ]します	
\\	飛行場 
\\	空港	
\\	飛[ひ]行[こう]場[じょう] 
\\	空[くう]港[こう]	
\\	必要	
\\	必[ひつ]要[よう]	ヒツヨー
\\	昼間	
\\	昼[ひる]間[ま]	チューカン 
\\	ヒルマꜜ
\\	昼休み	
\\	昼[ひる]休[やす]み	ヒルヤꜜスミ
\\	増える 
\\	増えて 
\\	増えます	
\\	増[ふ]える 
\\	増[ふ]えて 
\\	増[ふ]えます	
\\	深い	
\\	深[ふか]い	フカꜜイ
\\	複雑	
\\	複[ふく]雑[ざつ]	フクザツ
\\	葡萄	
\\	葡[ぶ]萄[どう]	ブドー
\\	布団	
\\	布[ふ]団[とん]	フトン
\\	不便	
\\	不[ふ]便[べん]	フꜜベン
\\	降り出す 
\\	降り出して 
\\	降り出します	
\\	降[ふ]り 出[だ]す 
\\	降[ふ]り 出[だ]して 
\\	降[ふ]り 出[だ]します	
\\	別	
\\	別[べつ]	ベツ
\\	返事	
\\	返[へん]事[じ]	ヘンジꜜ
\\	貿易	
\\	貿[ぼう]易[えき]	ボーエキ
\\	法律	
\\	法[ほう]律[りつ]	ホーリツ
\\	星	
\\	星[ほし]	ホシ
\\	殆ど	
\\	殆[ほとん]ど	ホトꜜンド
\\	翻訳	
\\	翻[ほん]訳[やく]	ホンヤク
\\	又は	
\\	又[また]は	
\\	周り	
\\	周[まわ]り	マワリ
\\	回る 
\\	回って 
\\	回ります	
\\	回[まわ]る 
\\	回[まわ]って 
\\	回[まわ]ります	
\\	真中	
\\	真[まん]中[なか]	
\\	味噌	
\\	味[み]噌[そ]	ミꜜソ
\\	見付かる 
\\	見付かって 
\\	見付かります	
\\	見[み]付[つ]かる 
\\	見[み]付[つ]かって 
\\	見[み]付[つ]かります	
\\	港	
\\	港[みなと]	ミナト
\\	向かう 
\\	向かって 
\\	向かいます	
\\	向[む]かう 
\\	向[む]かって 
\\	向[む]かいます	
\\	迎える 
\\	迎えて 
\\	迎えます	
\\	迎[むか]える 
\\	迎[むか]えて 
\\	迎[むか]えます	
\\	虫	
\\	虫[むし]	ムシ
\\	息子	
\\	息[むす]子[こ]	ムスコ
\\	娘	
\\	娘[むすめ]	ムスメꜜ
\\	無理	
\\	無[む]理[り]	ムꜜリ
\\	珍しい	
\\	珍[めずら]しい	メズラシꜜイ
\\	もし	
\\	もし	モꜜシ
\\	木綿	
\\	木綿[もめん]	モメン
\\	森	
\\	森[もり]	モリ
\\	役に立つ 
\\	役に立って 
\\	役に立ちます	
\\	役[やく]に 立[た]つ 
\\	役[やく]に 立[た]って 
\\	役[やく]に 立[た]ちます	
\\	焼ける 
\\	焼けて 
\\	焼けます	
\\	焼[や]ける 
\\	焼[や]けて 
\\	焼[や]けます	
\\	優しい	
\\	優[やさ]しい 
\\	易しい)	ヤサシイ
\\	やっと	
\\	やっと	ヤット 
\\	ヤットꜜ
\\	柔らかい	
\\	柔[やわ]らかい	ヤワラカꜜイ
\\	揺れる 
\\	揺れて 
\\	揺れます	
\\	揺[ゆ]れる 
\\	揺[ゆ]れて 
\\	揺[ゆ]れます	
\\	用	
\\	用[よう]	ヨꜜー
\\	用意	
\\	支度)	用[よう]意[い] 
\\	容易)	ヨꜜーイ
\\	喜ぶ 
\\	喜んで 
\\	喜びます	
\\	喜[よろこ]ぶ 
\\	喜[よろこ]んで 
\\	喜[よろこ]びます	
\\	理由	
\\	理[り]由[ゆう]	リユー
\\	利用	
\\	利[り]用[よう]	リヨー
\\	両方	
\\	両[りょう]方[ほう]	リョーホꜜー 
\\	リョーホー
\\	冷房	
\\	冷[れい]房[ぼう]	レイボー
\\	連絡	
\\	連[れん]絡[らく]	レンラク
\\	沸かす 
\\	沸かして 
\\	沸かします	
\\	沸[わ]かす 
\\	沸[わ]かして 
\\	沸[わ]かします	
\\	沸く 
\\	沸いて 
\\	沸きます	
\\	沸[わ]く 
\\	沸[わ]いて 
\\	沸[わ]きます	
\\	訳	
\\	訳[わけ]	ヤꜜク 
\\	ワꜜケ
\\	忘れ物	
\\	忘[わす]れ 物[もの]	ワスレモノ
\\	割合	
\\	割[わり]合[あい]	ワリアイ
\\	割れる 
\\	割れて 
\\	割れます	
\\	割[わ]れる 
\\	割[わ]れて 
\\	割[わ]れます	
\\	アジア	
\\	アジア	
\\	アクセサリー	
\\	アクセサリー	アꜜクセサリー 
\\	アクセꜜサリー
\\	アナウンサー	
\\	アナウンサー	アナウꜜンサー
\\	アフリカ	
\\	アフリカ	
\\	アルコール	
\\	アルコール	アルコール
\\	エスカレーター	
\\	エスカレーター	エスカレꜜーター
\\	ガソリンスタンド	
\\	ガソリンスタンド	ガソリンスタꜜンド
\\	ガラス	
\\	ガラス	ガラス
\\	サンダル	
\\	サンダル	サンダル 
\\	サꜜンダル
\\	サンドイッチ	
\\	サンドイッチ	サンドイꜜッチ
\\	ジャム	
\\	ジャム	ジャꜜム
\\	スーツ	
\\	スーツ	スꜜーツ
\\	スーツケース	
\\	スーツケース	スーツケꜜース
\\	スクリーン	
\\	スクリーン	スクリꜜーン
\\	ステーキ	
\\	ステーキ	ステꜜーキ
\\	ステレオ	
\\	ステレオ	ステレオ
\\	タイプ	
\\	タイプ	タꜜイプ
\\	テキスト	
\\	テキスト	テꜜキスト 
\\	テキꜜスト
\\	パート	
\\	パート	パꜜート 
\\	パート
\\	ハンドバッグ	
\\	ハンドバッグ	ハンドバꜜッグ
\\	ビル	
\\	ビル	ビꜜル
\\	ファックス	
\\	ファックス	ファꜜックス
\\	ベル	
\\	ベル	ベꜜル
\\	レジ	
\\	レジ	レꜜジ
\\	ワープロ	
\\	ワープロ	ワープロ
\\	賛成	
\\	賛[さん]成[せい]	サンセイ
\\	化粧	
\\	化[け]粧[しょう]	ケショꜜー
\\	はい、リトルアジアです。	
\\	はい、リトルアジアです。
\\	私、ジョン・ワンと申します。	
\\	私[わたくし]、ジョン・ワンと 申[もう]します。
\\	アルバイト募集の広告を見たんですが。	
\\	アルバイト 募[ぼ]集[しゅう]の 広[こう]告[こく]を 見[み]たんですが。
\\	そうですか。じゃあ、会って話しましょうか。	
\\	そうですか。じゃあ、 会[あ]って 話[はな]しましょうか。
\\	今日店に来られますか。	
\\	今日[きょう]店[みせ]に 来[こ]られますか。
\\	今日は一寸行けないんですが、明日なら行けると思います。	
\\	今日[きょう]は 一[ちょっ]寸[と]行[い]けないんですが、 明日[あした]なら 行[い]けると 思[おも]います。
\\	そうですか。今日は駄目ですか。	
\\	そうですか。 今日[きょう]は 駄[だ]目[め]ですか。
\\	じゃあ、明日の一時頃はどうですか。	
\\	じゃあ、 明日[あした]の 一[いち]時[じ]頃[ごろ]はどうですか。
\\	一時ですね。分かりました。	
\\	一[いち]時[じ]ですね。 分[わ]かりました。
\\	ワンさんは如何して此のアルバイトに興味が有るんですか。	
\\	ワンさんは 如[ど]何[う]して 此[こ]のアルバイトに 興[きょう]味[み]が 有[あ]るんですか。
\\	面白そうですから。色々な人に会えるし、日本語も使えるし。	
\\	面[おも]白[しろ]そうですから。 色[いろ]々[いろ]な 人[ひと]に 会[あ]えるし、 日[に]本[ほん]語[ご]も 使[つか]えるし。
\\	レストランで働いたことが有りますか。	
\\	レストランで 働[はたら]いたことが 有[あ]りますか。
\\	はい、ウェーターなら、したことが有ります。	
\\	はい、ウェーターなら、したことが 有[あ]ります。
\\	明日から始められますか。	
\\	明日[あした]から 始[はじ]められますか。
\\	はい、宜しくお願いします。	
\\	はい、 宜[よろ]しくお 願[ねが]いします。
\\	頑張って下さい。	
\\	頑[がん]張[ば]って 下[くだ]さい。
\\	いらっしゃいませ。あ、山下先生。	
\\	いらっしゃいませ。あ、 山[やま]下[した]先[せん]生[せい]。
\\	ジョンさん、此処でアルバイトをしているんですか。	
\\	ジョンさん、 此[こ]処[こ]でアルバイトをしているんですか。
\\	ええ、一週間に三日働いています。	
\\	ええ、 一[いっ]週[しゅう]間[かん]に 三[みっ]日[か]働[はたら]いています。
\\	そうですか。何れが美味しいですか。	
\\	そうですか。 何[ど]れが 美[お]味[い]しいですか。
\\	此のカレーが一番人気が有ります。	
\\	此[こ]のカレーが 一[いち]番[ばん]人[にん]気[き]が 有[あ]ります。
\\	美味しそうですね。じゃあ、食べてみます。	
\\	美[お]味[い]しそうですね。じゃあ、 食[た]べてみます。
\\	馬鹿	
\\	馬[ば]鹿[か]	バꜜカ
\\	遊べる 
\\	遊べて 
\\	遊べます	
\\	遊[あそ]べる 
\\	遊[あそ]べて 
\\	遊[あそ]べます
\\	買える 
\\	買えて 
\\	買えます	
\\	買[か]える 
\\	買[か]えて 
\\	買[か]えます
\\	読める 
\\	読めて 
\\	読めます	
\\	読[よ]める 
\\	読[よ]めて 
\\	読[よ]めます
\\	話せる 
\\	話せて 
\\	話せます	
\\	話[はな]せる 
\\	話[はな]せて 
\\	話[はな]せます
\\	行ける 
\\	行けて 
\\	行けます	
\\	行[い]ける 
\\	行[い]けて 
\\	行[い]けます
\\	泳げる 
\\	泳げて 
\\	泳げます	
\\	泳[およ]げる 
\\	泳[およ]げて 
\\	泳[およ]げます
\\	待てる 
\\	待てて 
\\	待てます	
\\	待[ま]てる 
\\	待[ま]てて 
\\	待[ま]てます
\\	死ねる 
\\	死ねて 
\\	死ねます	
\\	死[し]ねる 
\\	死[し]ねて 
\\	死[し]ねます
\\	取れる 
\\	取れて 
\\	取れます	
\\	取[と]れる 
\\	取[と]れて 
\\	取[と]れます
\\	来られる 
\\	来られて 
\\	来られます	
\\	来[こ]られる 
\\	来[こ]られて 
\\	来[こ]られます
\\	日本語が話せます。	
\\	日[に]本[ほん]語[ご]が 話[はな]せます。
\\	私は泳げないんです。	
\\	私[わたし]は 泳[およ]げないんです。
\\	雨が降ったので、海に行けませんでした。	
\\	雨[あめ]が 降[ふ]ったので、 海[うみ]に 行[い]けませんでした。
\\	出来る 
\\	出来て 
\\	出来ます	
\\	出[で]来[き]る 
\\	出[で]来[き]て 
\\	出[で]来[き]ます
\\	漢字が/を読めます。 
\\	が 
\\	を)	
\\	漢[かん]字[じ]が/を 読[よ]めます。
\\	仕事が出来ます。 
\\	する 
\\	が 
\\	仕[し]事[ごと]が 出[で]来[き]ます。
\\	山に登れます。	
\\	山[やま]に 登[のぼ]れます。
\\	如何してパーティーに来ないんですか。	
\\	如[ど]何[う]してパーティーに 来[こ]ないんですか。
\\	明日試験が有るから、今日は勉強しなくちゃ行けないんです。	
\\	明日[あした]試[し]験[けん]が 有[あ]るから、 今[きょ]日[う]は 勉[べん]強[きょう]しなくちゃ 行[い]けないんです。
\\	日本語は面白いし、先生は良いし、私は日本語の授業が大好きです。	
\\	日[に]本[ほん]語[ご]は 面[おも]白[しろ]いし、 先[せん]生[せい]は 良[い]いし、 私[わたし]は 日[に]本[ほん]語[ご]の 授[じゅ]業[ぎょう]が 大[だい]好[す]きです。
\\	家族から手紙が来たし、彼と電話で話したし、今日は迚も良い日でした。	
\\	家[か]族[ぞく]から 手[て]紙[がみ]が 来[き]たし、 彼[かれ]と 電[でん]話[わ]で 話[はな]したし、 今[きょ]日[う]は 迚[とて]も 良[い]い 日[ひ]でした。
\\	国に帰りたいですか。 
\\	いいえ、日本の生活は楽しいし、良い友達が居るし、帰りたくありません。	
\\	国[くに]に 帰[かえ]りたいですか。 
\\	いいえ、 日[に]本[ほん]の 生[せい]活[かつ]は 楽[たの]しいし、 良[い]い 友[とも]達[だち]が 居[い]るし、 帰[かえ]りたくありません。
\\	物価が安いし、此の町の生活は楽です。	
\\	物[ぶっ]価[か]が 安[やす]いし、 此[こ]の 町[まち]の 生[せい]活[かつ]は 楽[らく]です。
\\	山下先生は良い先生です。教えるのが上手だし、親切だし。	
\\	山[やま]下[した]先[せん]生[せい]は 良[い]い 先[せん]生[せい]です。 教[おし]えるのが 上[じょう]手[ず]だし、 親[しん]切[せつ]だし。
\\	此の林檎は美味しそうです。	
\\	此[こ]の 林[りん]檎[ご]は 美[お]味[い]しそうです。
\\	明日は天気が良さそうです。	
\\	明日[あした]は 天[てん]気[き]が 良[よ]さそうです。
\\	良さそうです	
\\	良[よ]さそうです
\\	メアリーさんは元気そうでした。	
\\	メアリーさんは 元[げん]気[き]そうでした。
\\	元気そうです	
\\	元[げん]気[き]そうです
\\	美味しそうです	
\\	美[お]味[い]しそうです
\\	此の本は難しくなさそうです。	
\\	此[こ]の 本[ほん]は 難[むずか]しくなさそうです。
\\	知子	
\\	知[とも]子[こ]	
\\	知子さんはテニスが上手じゃなさそうです。	
\\	知[とも]子[こ]さんはテニスが 上[じょう]手[ず]じゃなさそうです。
\\	暖かそうなセーターを着ています。	
\\	暖[あたた]かそうなセーターを 着[き]ています。
\\	暖かそうなセーター	
\\	暖[あたた]かそうなセーター
\\	漢字が分からなかったので、日本人の友達に聞いて見ました。	
\\	漢[かん]字[じ]が 分[わ]からなかったので、 日[に]本[ほん]人[じん]の 友[とも]達[だち]に 聞[き]いて 見[み]ました。
\\	友達が彼の店のケーキは美味しいと言っていましたから、今度食べてみます。	
\\	友[とも]達[だち]が 彼[あ]の 店[みせ]のケーキは 美[お]味[い]しいと 言[い]っていましたから、 今[こん]度[ど]食[た]べてみます。
\\	ブラジル	
\\	ブラジル	
\\	メキシコ	
\\	メキシコ
\\	ブラジルに行ったことが有りますか。 
\\	メキシコなら行ったことが有りますが、ブラジルは行ったことが有りません。	
\\	ブラジルに 行[い]ったことが 有[あ]りますか。 
\\	メキシコなら 行[い]ったことが 有[あ]りますが、ブラジルは 行[い]ったことが 有[あ]りません。
\\	平仮名	
\\	平[ひら]仮[が]名[な]	
\\	片仮名	
\\	片[かた]仮[か]名[な]	カタカꜜナ 
\\	カタꜜカナ
\\	日本語が分かりますか。 
\\	平仮名なら分かります。	
\\	日[に]本[ほん]語[ご]が 分[わ]かりますか。 
\\	平[ひら]仮[が]名[な]なら 分[わ]かります。
\\	一週間に三回	
\\	一[いっ]週[しゅう]間[かん]に 三[さん]回[かい]
\\	に 
\\	に 
\\	に 
\\	私は一週間に三回髪を洗います。	
\\	私[わたし]は 一[いっ]週[しゅう]間[かん]に 三[さん]回[かい]髪[かみ]を 洗[あら]います。
\\	私は一か月に一回家族に電話を掛けます。	
\\	私[わたし]は 一[いっ]か 月[げつ]に 一[いっ]回[かい]家[か]族[ぞく]に 電[でん]話[わ]を 掛[か]けます。
\\	父は一年に二回旅行をします。	
\\	父[ちち]は 一[いち]年[ねん]に 二[に]回[かい]旅[りょ]行[こう]をします。
\\	サックス	
\\	サックス	サꜜックス
\\	サックスを吹く 
\\	サックスを吹いて 
\\	サックスを吹きます	
\\	サックスを 吹[ふ]く 
\\	サックスを 吹[ふ]いて 
\\	サックスを 吹[ふ]きます
\\	太鼓 
\\	ドラム	
\\	太[たい]鼓[こ] 
\\	ドラム	
\\	太鼓を叩く 
\\	太鼓を叩いて 
\\	太鼓を叩きます	
\\	太[たい]鼓[こ]を 叩[たた]く 
\\	太[たい]鼓[こ]を 叩[たた]いて 
\\	太[たい]鼓[こ]を 叩[たた]きます
\\	何か楽器が出来ますか。	
\\	何[なに]か 楽[がっ]器[き]が 出[で]来[き]ますか。
\\	何か楽器をやりますか。	
\\	何[なに]か 楽[がっ]器[き]をやりますか。
\\	料理が上手です。 上手に料理が出来ます。	
\\	料[りょう]理[り]が 上[じょう]手[ず]です。 上[じょう]手[ず]に 料[りょう]理[り]が 出[で]来[き]ます。
\\	毎日日本語を楽しく勉強しています。	
\\	毎[まい]日[にち]日[に]本[ほん]語[ご]を 楽[たの]しく 勉[べん]強[きょう]しています。
\\	[シー]
\\	[ディー]	
\\	インド	
\\	インド	
\\	タイ	
\\	タイ	タꜜイ
\\	チベット	
\\	チベット	
\\	ニュージーランド	
\\	ニュージーランド	
\\	口座	
\\	口[こう]座[ざ]	コーザ
\\	手数料	
\\	手[て]数[すう]料[りょう]	テスꜜーリョー
\\	キャッシュカード	
\\	キャッシュカード	キャッシュカꜜード
\\	通帳	
\\	通[つう]帳[ちょう]	ツーチョー
\\	預金	
\\	預[よ]金[きん]	ヨキン
\\	百円玉	
\\	百[ひゃく]円[えん]玉[だま]	
\\	玉	
\\	玉[たま]	ギョꜜク 
\\	ギョク 
\\	タマꜜ
\\	金額	
\\	金[きん]額[がく]	キンカ°ク
\\	暗証	
\\	暗[あん]証[しょう]	アンショー
\\	暗証番号	
\\	暗[あん]証[しょう]番[ばん]号[ごう]	
\\	口座を開きたいんですが。	
\\	口[こう]座[ざ]を 開[ひら]きたいんですが。
\\	口座を閉じたいんですが。	
\\	口[こう]座[ざ]を 閉[と]じたいんですが。
\\	ドル	
\\	ドル	ドꜜル
\\	ドルを円に換えて下さい。	
\\	ドルを 円[えん]に 換[か]えて 下[くだ]さい。
\\	小切手	
\\	小[こ]切[ぎっ]手[て]	コキ°ꜜッテ
\\	現金	
\\	現[げん]金[きん]	ゲンキꜜン 
\\	ゲンナマ
\\	小切手を現金にしたいんですけど。	
\\	小[こ]切[ぎっ]手[て]を 現[げん]金[きん]にしたいんですけど。
\\	口座にお金を振り込みたいんです。	
\\	口[こう]座[ざ]にお 金[かね]を 振[ふ]り 込[こ]みたいんです。
\\	振り込む 
\\	振り込んで 
\\	振り込みます	
\\	振[ふ]り 込[こ]む 
\\	振[ふ]り 込[こ]んで 
\\	振[ふ]り 込[こ]みます	
\\	送金	
\\	送[そう]金[きん]	ソーキン
\\	アメリカに送金したいんですが。	
\\	アメリカに 送[そう]金[きん]したいんですが。
\\	札	
\\	札[さつ] 
\\	冊)	サツ 
\\	フダ
\\	一万円札を千円札十枚に両替出来ますか。	
\\	一[いち]万[まん]円[えん]札[さつ]を 千[せん]円[えん]札[さつ]十[じゅう]枚[まい]に 両[りょう]替[がえ]出[で]来[き]ますか。
\\	両替	
\\	両[りょう]替[がえ]
\\	お金を下ろします。	
\\	お 金[かね]を 下[お]ろします。
\\	バレンタインデーのプレゼントは何が良いと思いますか。	
\\	バレンタインデーのプレゼントは 何[なに]が 良[い]いと 思[おも]いますか。
\\	たけしさんは何時も同じセーターを着ているから、セーターを上げたらどうですか。	
\\	たけしさんは 何[い]時[つ]も 同[おな]じセーターを 着[き]ているから、セーターを 上[あ]げたらどうですか。
\\	其れは良いかも知れませんね。	
\\	其[そ]れは 良[い]いかも 知[し]れませんね。
\\	たけしさん。はい、此れ。	
\\	たけしさん。はい、 此[こ]れ。
\\	えっ、僕に。どうも有難う。開けてもいい?	
\\	えっ、 僕[ぼく]に。どうも 有[あり]難[がと]う。 開[あ]けてもいい?
\\	わあ、素敵なセーター。此んなのが欲しかったんだ。メアリーさんが編んだの。	
\\	わあ、 素[す]敵[てき]なセーター。 此[こ]んなのが 欲[ほ]しかったんだ。メアリーさんが 編[あ]んだの。
\\	うん、小さいかも知れないから着てみて。	
\\	うん、 小[ちい]さいかも 知[し]れないから 着[き]てみて。
\\	丁度良いよ。有難う。	
\\	丁[ちょう]度[ど]良[い]いよ。 有[あり]難[がと]う。
\\	暖かそうなセーターですね。	
\\	暖[あたた]かそうなセーターですね。
\\	此れ、メアリーさんが呉れたんです。	
\\	此[こ]れ、メアリーさんが 呉[く]れたんです。
\\	良く似合っていますよ。僕も彼女が欲しいなあ。	
\\	良[よ]く 似[に]合[あ]っていますよ。 僕[ぼく]も 彼[かの]女[じょ]が 欲[ほ]しいなあ。
\\	ロバートさんは、チョコレートを十個も貰ったんですよ。	
\\	ロバートさんは、チョコレートを 十[じゅっ]個[こ]も 貰[もら]ったんですよ。
\\	僕は一個しか貰いませんでした。大家さんから。寂しいなあ。	
\\	僕[ぼく]は 一[いっ]個[こ]しか 貰[もら]いませんでした。 大[おお]家[や]さんから。 寂[さび]しいなあ。
\\	でも、ロバートさんはホワイトデーが大変ですよ。	
\\	でも、ロバートさんはホワイトデーが 大[たい]変[へん]ですよ。
\\	折り紙	
\\	折[お]り 紙[がみ]	オリꜜカ°ミ 
\\	オリカ°ミ
\\	ええ、男の人は三月十四日にお返しをしなくちゃ行けないんですよ。	
\\	ええ、 男[おとこ]の 人[ひと]は 三[さん]月[がつ]十[じゅう]四[よっ]日[か]にお 返[かえ]しをしなくちゃ 行[い]けないんですよ。
\\	良い漢字の辞書が欲しいです。	
\\	良[い]い 漢[かん]字[じ]の 辞[じ]書[しょ]が 欲[ほ]しいです。
\\	ゴジラ	
\\	ゴジラ	
\\	子供の時、ゴジラの玩具が欲しかったです。	
\\	子[こ]供[ども]の 時[とき]、ゴジラの 玩[おも]具[ちゃ]が 欲[ほ]しかったです。
\\	お金は余り欲しくありません。	
\\	お 金[かね]は 余[あま]り 欲[ほ]しくありません。
\\	私は 
\\	が 欲しい	
\\	私[わたし]は 
\\	が 欲[ほ]しい
\\	ロバートさんはコンピューターが欲しいと言っています。	
\\	ロバートさんはコンピューターが 欲[ほ]しいと 言[い]っています。
\\	クラシック	
\\	クラシック	クラシꜜック 
\\	クラꜜシック
\\	きょうこさんはクラシックのシーディーが欲しくないでしょう。	
\\	きょうこさんはクラシックの 
\\	[シー]
\\	[ディー]が 欲[ほ]しくないでしょう。
\\	知子さんは英語を習いたがっています。	
\\	知[とも]子[こ]さんは 英[えい]語[ご]を 習[なら]いたがっています。
\\	トムさんは友達を欲しがっています。	
\\	トムさんは 友[とも]達[だち]を 欲[ほ]しがっています。
\\	明日は雨が降るかも知れません。	
\\	明日[あした]は 雨[あめ]が 降[ふ]るかも 知[し]れません。
\\	田中さんより、鈴木さんの方が背が高いかも知れません。	
\\	田[た]中[なか]さんより、 鈴[すず]木[き]さんの 方[ほう]が 背[せ]が 高[たか]いかも 知[し]れません。
\\	明日は天気が良くないかも知れません。	
\\	明日[あした]は 天[てん]気[き]が 良[よ]くないかも 知[し]れません。
\\	トムさんは、子供の時、意地悪だったかも知れません。	
\\	トムさんは、 子[こ]供[ども]の 時[とき]、 意[い]地[じ]悪[わる]だったかも 知[し]れません。
\\	カナダ	
\\	カナダ	
\\	トムさんはカナダ人だ。 トムさんはカナダ人かも知れません。	
\\	トムさんはカナダ 人[じん]だ。 トムさんはカナダ 人[じん]かも 知[し]れません。
\\	山下先生は犬が嫌いだ。 山下先生は犬が嫌いかも知れません。	
\\	山[やま]下[した]先[せん]生[せい]は 犬[いぬ]が 嫌[きら]いだ。 山[やま]下[した]先[せん]生[せい]は 犬[いぬ]が 嫌[きら]いかも 知[し]れません。
\\	行くかも知れません	
\\	行[い]くかも 知[し]れません
\\	寒いかも知れません	
\\	寒[さむ]いかも 知[し]れません
\\	元気かも知れません	
\\	元[げん]気[き]かも 知[し]れません
\\	学生かも知れません	
\\	学[がく]生[せい]かも 知[し]れません
\\	私は其の女の人に花を上げます。	
\\	私[わたし]は 其[そ]の 女[おんな]の 人[ひと]に 花[はな]を 上[あ]げます。
\\	其の女の人は男の人に時計を上げました。	
\\	其[そ]の 女[おんな]の 人[ひと]は 男[おとこ]の 人[ひと]に 時[と]計[けい]を 上[あ]げました。
\\	両親が私に新しい車を呉れるかも知れません。	
\\	両[りょう]親[しん]が 私[わたし]に 新[あたら]しい 車[くるま]を 呉[く]れるかも 知[し]れません。
\\	は/が 
\\	に 呉れる/上げる	
\\	は/が 
\\	に 呉[く]れる/ 上[あ]げる
\\	私は姉に/から古い辞書を貰いました。 姉が私に古い辞書を呉れました。	
\\	私[わたし]は 姉[あね]に/から 古[ふる]い 辞[じ]書[しょ]を 貰[もら]いました。 姉[あね]が 私[わたし]に 古[ふる]い 辞[じ]書[しょ]を 呉[く]れました。
\\	は/が 
\\	に/から 貰う	
\\	は/が 
\\	に/から 貰[もら]う
\\	もっと勉強したらどうですか。 (たらどうですか 
\\	もっと 勉[べん]強[きょう]したらどうですか。 (たらどうですか 
\\	薬を飲んだらどうですか。 (たらどうですか 
\\	薬[くすり]を 飲[の]んだらどうですか。 (たらどうですか 
\\	家に来ませんか。 (たらどうですか 
\\	家[うち]に 来[き]ませんか。 (たらどうですか 
\\	私の家には猫が三匹居ます。	
\\	私[わたし]の 家[うち]には 猫[ねこ]が 三[さん]匹[びき]居[い]ます。
\\	傘を三本買いました。	
\\	傘[かさ]を 三[さん]本[ぼん]買[か]いました。
\\	"私の母は猫を三匹も飼っています。 
\\	も 
\\	"私[わたし]の 母[はは]は 猫[ねこ]を 三[さん]匹[びき]も 飼[か]っています。 
\\	も 
\\	"昨日のパーティーには学生が二十人も来ました。 
\\	も 
\\	"昨[きの]日[う]のパーティーには 学[がく]生[せい]が 二[に]十[じゅう]人[にん]も 来[き]ました。 
\\	も 
\\	"私は日本語の辞書を一冊しか持っていません。 
\\	しか 
\\	"私[わたし]は 日[に]本[ほん]語[ご]の 辞[じ]書[しょ]を 一[いっ]冊[さつ]しか 持[も]っていません。 
\\	しか 
\\	"此の会社にはコンピューターが二台しか有りません。 
\\	しか 
\\	"此[こ]の 会[かい]社[しゃ]にはコンピューターが 二[に]台[だい]しか 有[あ]りません。 
\\	しか 
\\	日本語の先生は易しいなあ。 (なあ 
\\	日[に]本[ほん]語[ご]の 先[せん]生[せい]は 易[やさ]しいなあ。 (なあ 
\\	良い教科書だなあ。 (なあ 
\\	良[い]い 教[きょう]科[か]書[しょ]だなあ。 (なあ 
\\	お腹が空いたなあ。	
\\	お 腹[なか]が 空[す]いたなあ。
\\	彼の人はけちだなあ。	
\\	彼[あ]の 人[ひと]はけちだなあ。
\\	"日本語が出来ます。 (出来る 
\\	"日[に]本[ほん]語[ご]が 出[で]来[き]ます。 
\\	出[で]来[き]る 
\\	"彼はスキーが出来ません。 (出来る 
\\	"彼[かれ]はスキーが 出[で]来[き]ません。 
\\	出[で]来[き]る 
\\	"試験は余り出来ませんでした。 (出来る 
\\	"試[し]験[けん]は 余[あま]り 出[で]来[き]ませんでした。 
\\	出[で]来[き]る 
\\	"晩御飯が出来ました。 (出来る 
\\	"晩[ばん]御[ご]飯[はん]が 出[で]来[き]ました。 
\\	出[で]来[き]る 
\\	"宿題は出来ましたか。 (出来る 
\\	"宿[しゅく]題[だい]は 出[で]来[き]ましたか。 
\\	出[で]来[き]る 
\\	"新しい店が出来ました。 (出来る 
\\	"新[あたら]しい 店[みせ]が 出[で]来[き]ました。 
\\	出[で]来[き]る 
\\	"友達が沢山出来ました。 (出来る 
\\	"友[とも]達[だち]が 沢[たく]山[さん]出[で]来[き]ました。 
\\	出[で]来[き]る 
\\	たけし君、今度の休み、予定有る?	
\\	たけし 君[くん]、 今[こん]度[ど]の 休[やす]み、 予[よ]定[てい]有[あ]る?
\\	ううん、別に。如何して?	
\\	ううん、 別[べつ]に。 如[ど]何[う]して?
\\	美智子さんの長野の家に行こうと思っているんだけど、一緒に行かない?	
\\	美[み]智[ち]子[こ]さんの 長[なが]野[の]の 家[うち]に 行[い]こうと 思[おも]っているんだけど、 一[いっ]緒[しょ]に 行[い]かない?
\\	良いの?	
\\	良[い]いの?
\\	うん。美智子さんが「たけし君も誘って」と言ってたから。	
\\	うん。 美[み]智[ち]子[こ]さんが「たけし 君[くん]も 誘[さそ]って」と 言[い]ってたから。
\\	誘う 
\\	誘って 
\\	誘います	
\\	誘[さそ]う 
\\	誘[さそ]って 
\\	誘[さそ]います
\\	じゃあ、行く。電車の時間、調べておくよ。	
\\	じゃあ、 行[い]く。 電[でん]車[しゃ]の 時[じ]間[かん]、 調[しら]べておくよ。
\\	有難う。じゃあ、私、美智子さんに電話しておく。	
\\	有[あり]難[がと]う。じゃあ、 私[わたし]、 美[み]智[ち]子[こ]さんに 電[でん]話[わ]しておく。
\\	早く着いたから、一寸観光しない。	
\\	早[はや]く 着[つ]いたから、 一[ちょっ]寸[と]観[かん]光[こう]しない。
\\	うん。何処に行く?	
\\	うん。 何[ど]処[こ]に 行[い]く?
\\	善光寺	
\\	善[ぜん]光[こう]寺[じ]	
\\	善光寺はどう?有名なお寺だよ。	
\\	善[ぜん]光[こう]寺[じ]はどう? 有[ゆう]名[めい]なお 寺[てら]だよ。
\\	そうだね。昼御飯は何にする?	
\\	そうだね。 昼[ひる] 御[ご]飯[はん]は 何[なに]にする?
\\	長野は蕎麦が美味しいから蕎麦を食べようよ。	
\\	長[なが]野[の]は 蕎[そ]麦[ば]が 美[お]味[い]しいから 蕎[そ]麦[ば]を 食[た]べようよ。
\\	済みません、善光寺に行くバスは何れですか。	
\\	済[す]みません、 善[ぜん]光[こう]寺[じ]に 行[い]くバスは 何[ど]れですか。
\\	善光寺なら、五番のバスですよ。	
\\	善[ぜん]光[こう]寺[じ]なら、 五[ご]番[ばん]のバスですよ。
\\	有難う御座います。此の地図、貰ってもいいですか。	
\\	有[あり]難[がと]う 御[ご]座[ざ]います。 此[こ]の 地[ち]図[ず]、 貰[もら]ってもいいですか。
\\	ええ、どうぞ。それから、此れ美術館の割引券ですが、良かったらどうぞ。	
\\	ええ、どうぞ。それから、 此[こ]れ 美[び]術[じゅつ]館[かん]の 割[わり]引[びき]券[けん]ですが、 良[よ]かったらどうぞ。
\\	東山かいい	
\\	東[ひがし]山[やま]かいい	
\\	此れ、東山かいいの絵が有る美術館ですよね。	
\\	此[こ]れ、 東[ひがし]山[やま]かいいの 絵[え]が 有[あ]る 美[び]術[じゅつ]館[かん]ですよね。
\\	明日行く予定なんです。どうも有難う。	
\\	明日[あした] 行[い]く 予[よ]定[てい]なんです。どうも 有[あり]難[がと]う。
\\	気を付けて。	
\\	気[き]を 付[つ]けて。
\\	食べよう	
\\	食[た]べよう
\\	行こう	
\\	行[い]こう
\\	話そう	
\\	話[はな]そう
\\	買おう	
\\	買[か]おう
\\	泳ごう	
\\	泳[およ]ごう
\\	遊ぼう	
\\	遊[あそ]ぼう
\\	待とう	
\\	待[ま]とう
\\	読もう	
\\	読[よ]もう
\\	死のう	
\\	死[し]のう
\\	取ろう	
\\	取[と]ろう
\\	来よう	
\\	来[こ]よう
\\	しよう	
\\	しよう
\\	明日は授業が無いから、今夜、何処かに食べに行こう。	
\\	明日[あした]は 授[じゅ]業[ぎょう]が 無[な]いから、 今[こん]夜[や]、 何[ど]処[こ]かに 食[た]べに 行[い]こう。
\\	結婚しようよ。	
\\	結[けっ]婚[こん]しようよ。
\\	手伝おうか。	
\\	か 
\\	手[て]伝[つだ]おうか。
\\	友達が面白いと言っていましたから、此の映画を見ようか。	
\\	か 
\\	友[とも]達[だち]が 面[おも]白[しろ]いと 言[い]っていましたから、 此[こ]の 映[えい]画[が]を 見[み]ようか。
\\	今度、何時会おうか。	
\\	か 
\\	今[こん]度[ど]、 何[い]時[つ]会[あ]おうか。
\\	毎日三時間日本語を勉強しようと思っています。	
\\	と思っています 
\\	毎[まい]日[にち]三[さん]時[じ]間[かん]日[に]本[ほん]語[ご]を 勉[べん]強[きょう]しようと 思[おも]っています。
\\	一万円上げましょう。何に使いますか。 
\\	漢字の辞書を買おうと思います。	
\\	と思います 
\\	一[いち]万[まん]円[えん]上[あ]げましょう。 何[なに]に 使[つか]いますか。 
\\	漢[かん]字[じ]の 辞[じ]書[しょ]を 買[か]おうと 思[おも]います。
\\	両親から一万円貰ったんですか。 
\\	漢字の辞書を買おうと思っています。	
\\	と思っています 
\\	両[りょう]親[しん]から 一[いち]万[まん]円[えん]貰[もら]ったんですか。 
\\	漢[かん]字[じ]の 辞[じ]書[しょ]を 買[か]おうと 思[おも]っています。
\\	明日試験が有るので、今晩勉強しておきます。	
\\	明日[あした]試[し]験[けん]が 有[あ]るので、 今[こん]晩[ばん]勉[べん]強[きょう]しておきます。
\\	寒く成るから、ストーブを点けておきました。	
\\	寒[さむ]く 成[な]るから、ストーブを 点[つ]けておきました。
\\	ホテルを予約しておかなくちゃ行けません。	
\\	ホテルを 予[よ]約[やく]しておかなくちゃ 行[い]けません。
\\	昨日買った本	
\\	昨[きの]日[う]買[か]った 本[ほん]
\\	彼が呉れた本	
\\	彼[かれ]が 呉[く]れた 本[ほん]
\\	机の上に有る本	
\\	机[つくえ]の 上[うえ]に 有[あ]る 本[ほん]
\\	日本で買えない本	
\\	日[に]本[ほん]で 買[か]えない 本[ほん]
\\	此れは去年の誕生日に彼女が呉れた本です。 
\\	此れは本です。)	
\\	此[こ]れは 去[きょ]年[ねん]の 誕[たん]生[じょう]日[び]に 彼[かの]女[じょ]が 呉[く]れた 本[ほん]です。 
\\	此[こ]れは 本[ほん]です。)
\\	父が村上春樹が書いた本を呉れました。 
\\	父が本呉れました。)	
\\	父[ちち]が 村[むら]上[かみ]春[はる]樹[き]が 書[か]いた 本[ほん]を 呉[く]れました。 
\\	父[ちち]が 本[ほん]を 呉[く]れました。)
\\	私が一番感動した映画は「生きる」です。 
\\	映画は「生きる」です)	
\\	私[わたし]が 一[いち]番[ばん]感[かん]動[どう]した 映[えい]画[が]は
\\	生[い]きる」です。 
\\	映[えい]画[が]は
\\	生[い]きる」です。)
\\	村上春樹	
\\	村[むら]上[かみ]春[はる]樹[き]	
\\	専攻	
\\	専[せん]攻[こう]	センコー
\\	失礼します、先生。今日授業に来られなくて済みませんでした。	
\\	失[しつ]礼[れい]します、 先[せん]生[せい]。 今日[きょう]授[じゅ]業[ぎょう]に 来[こ]られなくて 済[す]みませんでした。
\\	実は、朝寝坊して電車に乗り遅れたんです。済みません。	
\\	実[じつ]は、 朝[あさ]寝[ね]坊[ぼう]して 電[でん]車[しゃ]に 乗[の]り 遅[おく]れたんです。 済[す]みません。
\\	もう三回目ですよ。目覚まし時計を買ったらどうですか。	
\\	もう 三[さん]回[かい]目[め]ですよ。 目[め]覚[ざ]まし 時[と]計[けい]を 買[か]ったらどうですか。
\\	はい。あの、先生、宿題は明日迄待って頂けませんか。	
\\	はい。あの、 先[せん]生[せい]、 宿[しゅく]題[だい]は 明日[あした]迄[まで]待[ま]って 頂[いただ]けませんか。
\\	宿題を入れたファイルが無いんです。	
\\	宿[しゅく]題[だい]を 入[い]れたファイルが 無[な]いんです。
\\	困りましたね。有るといいですね。	
\\	困[こま]りましたね。 有[あ]るといいですね。
\\	済みません、ファイルを無くしたんですが。	
\\	済[す]みません、ファイルを 無[な]くしたんですが。
\\	どんなファイルですか。	
\\	どんなファイルですか。
\\	此の位の青いファイルです。	
\\	此[こ]の 位[ぐらい]の 青[あお]いファイルです。
\\	電車を降りる時、忘れたと思うんですが。	
\\	電[でん]車[しゃ]を 降[お]りる 時[とき]、 忘[わす]れたと 思[おも]うんですが。
\\	ええと・・・一寸待って下さい。電話して聞いてみます。	
\\	ええと・・・ 一[ちょっ]寸[と]待[ま]って 下[くだ]さい。 電[でん]話[わ]して 聞[き]いてみます。
\\	ジョンさん、ファイルは有りましたか。	
\\	ジョンさん、ファイルは 有[あ]りましたか。
\\	はい、駅員さんが探して呉れたんです。	
\\	はい、 駅[えき]員[いん]さんが 探[さが]して 呉[く]れたんです。
\\	良かったですね。	
\\	良[よ]かったですね。
\\	此れ、宿題です。遅く成って済みませんでした。	
\\	此[こ]れ、 宿[しゅく]題[だい]です。 遅[おそ]く 成[な]って 済[す]みませんでした。
\\	いいえ、良く出来ていますね。	
\\	いいえ、 良[よ]く 出[で]来[き]ていますね。
\\	ええ。駅員さんに手伝って貰いましたから。	
\\	ええ。 駅[えき]員[いん]さんに 手[て]伝[つだ]って 貰[もら]いましたから。
\\	私は妹にお金を貸して上げました。 
\\	私は妹にお金を貸しました。	
\\	私[わたし]は 妹[いもうと]にお 金[かね]を 貸[か]して 上[あ]げました。 
\\	私[わたし]は 妹[いもうと]にお 金[かね]を 貸[か]しました。
\\	きょうこさんはトムさんを駅に連れて行って上げました。 
\\	きょうこさんはトムさんを駅に連れて行きました。	
\\	きょうこさんはトムさんを 駅[えき]に 連[つ]れて 行[い]って 上[あ]げました。 
\\	きょうこさんはトムさんを 駅[えき]に 連[つ]れて 行[い]きました。
\\	友達が宿題を手伝って呉れました。	
\\	友[とも]達[だち]が 宿[しゅく]題[だい]を 手[て]伝[つだ]って 呉[く]れました。
\\	親戚がクッキーを送って呉れました。	
\\	親[しん]戚[せき]がクッキーを 送[おく]って 呉[く]れました。
\\	私は友達に宿題を手伝って貰いました。	
\\	私[わたし]は 友[とも]達[だち]に 宿[しゅく]題[だい]を 手[て]伝[つだ]って 貰[もら]いました。 
\\	貰[もら]う 
\\	一寸手伝って頂けませんか。	
\\	一[ちょっ]寸[と]手[て]伝[つだ]って 頂[いただ]けませんか。 
\\	頂[いただ]けませんか 
\\	一寸待って呉れませんか。	
\\	下さい)	一[ちょっ]寸[と]待[ま]って 呉[く]れませんか。 
\\	呉[く]れませんか 
\\	下[くだ]さい)
\\	其れ取って呉れない。 
\\	其れ取って。	
\\	其[そ]れ 取[と]って 呉[く]れない。 
\\	其[そ]れ 取[と]って。 
\\	呉[く]れない 
\\	良いアルバイトが有るといいですね。	
\\	良[い]いアルバイトが 有[あ]るといいですね。 
\\	といいですね 
\\	雨が降らないといいですね。	
\\	雨[あめ]が 降[ふ]らないといいですね。 
\\	といいですね 
\\	試験が易しいといいんですが。	
\\	試[し]験[けん]が 易[やさ]しいといいんですが。 
\\	といいんですが 
\\	八時の電車に乗れるといいんですが。	
\\	八[はち]時[じ]の 電[でん]車[しゃ]に 乗[の]れるといいんですが。 
\\	といいんですが 
\\	といいですね	
\\	といいですね
\\	といいんですが	
\\	といいんですが
\\	大学に行けるといいんですが。	
\\	大[だい]学[がく]に 行[い]けるといいんですが。
\\	時, 
\\	時[とき] 
\\	チベットに行く時、ビザを取ります。	
\\	チベットに 行[い]く 時[とき]、ビザを 取[と]ります。
\\	チベットに行く時、ビザを取りました。	
\\	チベットに 行[い]く 時[とき]、ビザを 取[と]りました。
\\	寝る時、コンタクトを取ります。	
\\	寝[ね]る 時[とき]、コンタクトを 取[と]ります。
\\	出掛ける時、ドアを鍵を掛けました。	
\\	出[で]掛[か]ける 時[とき]、ドアを 鍵[かぎ]を 掛[か]けました。
\\	寂しい時、妻に電話します。	
\\	寂[さび]しい 時[とき]、 妻[つま]に 電[でん]話[わ]します。
\\	寒い時、頭が痛く成ります。	
\\	(な-
\\	な 
\\	時)	寒[さむ]い 時[とき]、 頭[あたま]が 痛[いた]く 成[な]ります。
\\	元気な時、公園を走ります。	
\\	(な-
\\	な 
\\	時)	元[げん]気[き]な 時[とき]、 公[こう]園[えん]を 走[はし]ります。
\\	犬が病気の時、病院に連れて行きました。	
\\	(な-
\\	な 
\\	時)	犬[いぬ]が 病[びょう]気[き]の 時[とき]、 病[びょう]院[いん]に 連[つ]れて 行[い]きました。
\\	中国に行った時、ウーロン茶を買います。	
\\	中[ちゅう]国[ごく]に 行[い]った 時[とき]、ウーロン 茶[ちゃ]を 買[か]います。
\\	ウーロン茶	
\\	ウーロン 茶[ちゃ]	ウーロꜜンチャ
\\	ビザ	
\\	ビザ	ビꜜザ
\\	中国に行った時、ウーロン茶を買いました。	
\\	中[ちゅう]国[ごく]に 行[い]った 時[とき]、ウーロン 茶[ちゃ]を 買[か]いました。
\\	疲れた時、ゆっくりお風呂に入ります。	
\\	疲[つか]れた 時[とき]、ゆっくりお 風[ふ]呂[ろ]に 入[はい]ります。
\\	宿題を忘れた時、泣いて仕舞いました。	
\\	宿[しゅく]題[だい]を 忘[わす]れた 時[とき]、 泣[な]いて 仕[し]舞[ま]いました。
\\	汚い言葉	
\\	汚[きたな]い 言[こと]葉[ば]	
\\	汚い言葉を使って、済みませんでした。	
\\	(て-
\\	汚[きたな]い 言[こと]葉[ば]を 使[つか]って、 済[す]みませんでした。
\\	デートの約束を忘れて、御免。	
\\	(て-
\\	デートの 約[やく]束[そく]を 忘[わす]れて、 御[ご]免[めん]。
\\	宿題を持って来なくて、済みませんでした。	
\\	(なくて 
\\	て-
\\	宿[しゅく]題[だい]を 持[も]って 来[こ]なくて、 済[す]みませんでした。
\\	もっと早く	
\\	もっと 早[はや]く	
\\	もっと早く言わなくて、御免。	
\\	(なくて 
\\	て-
\\	もっと 早[はや]く 言[い]わなくて、 御[ご]免[めん]。
\\	たけしさん、久しぶりですね。	
\\	たけしさん、 久[ひさ]しぶりですね。
\\	旅行会社に就職したそうですね。御目出度う御座います。	
\\	旅[りょ]行[こう]会[がい]社[しゃ]に 就[しゅう]職[しょく]したそうですね。 御[お]目[め]出[で]度[と]う 御[ご]座[ざ]います。
\\	有難う御座います。	
\\	有[あり]難[がと]う 御[ご]座[ざ]います。
\\	もう仕事に慣れましたか。	
\\	もう 仕[し]事[ごと]に 慣[な]れましたか。
\\	ええ。でも、学生の時に比べて凄く忙しく成りました。	
\\	ええ。でも、 学[がく]生[せい]の 時[とき]に 比[くら]べて 凄[すご]く 忙[いそが]しく 成[な]りました。
\\	自分の時間が全然無いんです。	
\\	自[じ]分[ぶん]の 時[じ]間[かん]が 全[ぜん]然[ぜん]無[な]いんです。
\\	大変ですね。	
\\	大[たい]変[へん]ですね。
\\	私の友達の会社は、休みが多くて残業をしなくてもいいそうですよ。	
\\	私[わたし]の 友[とも]達[だち]の 会[かい]社[しゃ]は、 休[やす]みが 多[おお]くて 残[ざん]業[ぎょう]をしなくてもいいそうですよ。
\\	羨ましいですよ。	
\\	羨[うらや]ましいですよ。
\\	僕の会社は休みも少ないし、給料も安いし、最低です。	
\\	僕[ぼく]の 会[かい]社[しゃ]は 休[やす]みも 少[すく]ないし、 給[きゅう]料[りょう]も 安[やす]いし、 最[さい]低[てい]です。
\\	会社に入る前に如何してもっと調べなかったんですか。	
\\	会[かい]社[しゃ]に 入[はい]る 前[まえ]に 如[ど]何[う]してもっと 調[しら]べなかったんですか。
\\	旅行会社に入ったら、旅行が出来ると思ったんです。	
\\	旅[りょ]行[こう]会[がい]社[しゃ]に 入[はい]ったら、 旅[りょ]行[こう]が 出[で]来[き]ると 思[おも]ったんです。
\\	たけしさんが卒業してから全然会ってないけど、元気だった?	
\\	たけしさんが 卒[そつ]業[ぎょう]してから 全[ぜん]然[ぜん]会[あ]ってないけど、 元[げん]気[き]だった?
\\	随分疲れているみたい。	
\\	随[ずい]分[ぶん]疲[つか]れているみたい。
\\	毎晩四、五時間しか寝ていないそうよ。	
\\	毎[まい]晩[ばん]四[し]、 五[ご]時[じ]間[かん]しか 寝[ね]ていないそうよ。
\\	矢っ張りサラリーマンは大変だなあ。	
\\	矢[や]っ 張[ぱ]りサラリーマンは 大[たい]変[へん]だなあ。
\\	それに、忙し過ぎてメアリーとデートする時間もないって。	
\\	それに、 忙[いそが]し 過[す]ぎてメアリーとデートする 時[じ]間[かん]もないって。
\\	そうか。僕だったら、仕事より彼女を選ぶけど。	
\\	そうか。 僕[ぼく]だったら、 仕[し]事[ごと]より 彼[かの]女[じょ]を 選[えら]ぶけど。
\\	彼の二人、大丈夫かなあ。	
\\	彼[あ]の 二[ふた]人[り]、 大[だい]丈[じょう]夫[ぶ]かなあ。
\\	「日本語の授業は楽しいです。」 
\\	日本語の授業は楽しいそうです。	
\\	日[に]本[ほん]語[ご]の 授[じゅ]業[ぎょう]は 楽[たの]しいです。」 
\\	日[に]本[ほん]語[ご]の 授[じゅ]業[ぎょう]は 楽[たの]しいそうです。
\\	「先生は迚も親切です。」 
\\	先生は迚も親切だそうです。	
\\	先[せん]生[せい]は 迚[とて]も 親[しん]切[せつ]です。」 
\\	先[せん]生[せい]は 迚[とて]も 親[しん]切[せつ]だそうです。
\\	今日は授業が有りませんでした。 
\\	其の日は授業がなかったそうです。	
\\	今[きょ]日[う]は 授[じゅ]業[ぎょう]が 有[あ]りませんでした。 
\\	其[そ]の 日[ひ]は 授[じゅ]業[ぎょう]がなかったそうです。
\\	話すそうです	
\\	話[はな]すそうです
\\	寂しいそうです	
\\	寂[さび]しいそうです
\\	好きだそうです	
\\	好[す]きだそうです
\\	学生だそうです	
\\	学[がく]生[せい]だそうです
\\	新聞に依ると	
\\	新[しん]聞[ぶん]に 依[よ]ると
\\	天気予報に依ると	
\\	天[てん]気[き] 予[よ]報[ほう]に 依[よ]ると
\\	「今日は忙しいです。明日、試験が有るんです。」 
\\	メアリーさん、今日は忙しいって。明日、試験が有るんだって。	
\\	今日[きょう]は 忙[いそが]しいです。 明日[あした]、 試[し]験[けん]が 有[あ]るんです。」 
\\	メアリーさん、 今日[きょう]は 忙[いそが]しいって。 明日[あした]、 試[し]験[けん]が 有[あ]るんだって。
\\	あきらさんは何て言ってた。 
\\	チョコレートを食べ過ぎたって言ってた。 (って 
\\	て 
\\	ん; 
\\	って 
\\	と
\\	言う)	
\\	あきらさんは 何[なん]て 言[い]ってた。 
\\	チョコレートを 食[た]べ 過[す]ぎたって 言[い]ってた。 (って 
\\	て 
\\	ん; 
\\	って 
\\	と
\\	言[い]う)
\\	たら
\\	たら
\\	日本に行ったら、着物を買います。	
\\	日[に]本[ほん]に 行[い]ったら、 着[き]物[もの]を 買[か]います。
\\	読んだら	
\\	読[よ]んだら
\\	易しかったら	
\\	易[やさ]しかったら
\\	静かだったら	
\\	静[しず]かだったら
\\	休みだったら	
\\	休[やす]みだったら
\\	元気じゃなかったら	
\\	元[げん]気[き]じゃなかったら
\\	天気が良かったら、散歩に行きます。	
\\	天[てん]気[き]が 良[よ]かったら、 散[さん]歩[ぽ]に 行[い]きます。
\\	山下先生に会ったら、其の事を聞こうと思います。	
\\	山[やま]下[した]先[せん]生[せい]に 会[あ]ったら、 其[そ]の 事[こと]を 聞[き]こうと 思[おも]います。
\\	其の事	
\\	其[そ]の 事[こと]	
\\	日本人だったら、此の言葉を知っているでしょう。	
\\	日[に]本[ほん]人[じん]だったら、 此[こ]の 言[こと]葉[ば]を 知[し]っているでしょう。
\\	アムネスティ・インターナショナル	
\\	アムネスティ・インターナショナル	
\\	宝籤に当たったら、アムネスティ・インターナショナルにお金を送ります。	
\\	宝[たから]籤[くじ]に 当[あ]たったら、 アムネスティ・インターナショナルにお 金[かね]を 送[おく]ります。
\\	今晩、家に帰ったら、電話します。	
\\	今[こん]晩[ばん]、 家[うち]に 帰[かえ]ったら、 電[でん]話[わ]します。
\\	宿題が終わったら、遊びに行きましょう。	
\\	宿[しゅく]題[だい]が 終[お]わったら、 遊[あそ]びに 行[い]きましょう。
\\	私が猫だったら、一日中寝ているでしょう。	
\\	私[わたし]が 猫[ねこ]だったら、 一[いち]日[にち] 中[じゅう]寝[ね]ているでしょう。
\\	お金が有ったら、車を買うんですけど。	
\\	お 金[かね]が 有[あ]ったら、 車[くるま]を 買[か]うんですけど。
\\	靴を脱がなくてもいいです。	
\\	靴[くつ]を 脱[ぬ]がなくてもいいです。
\\	贈り物は高くなくてもいいです。	
\\	贈[おく]り 物[もの]は 高[たか]くなくてもいいです。
\\	私の父は村上春樹みたいです。	
\\	私[わたし]の 父[ちち]は 村[むら]上[かみ]春[はる]樹[き]みたいです。
\\	彼の人は猿みたいです。	
\\	彼[あ]の 人[ひと]は 猿[さる]みたいです。
\\	雨が降ったみたいです。	
\\	雨[あめ]が 降[ふ]ったみたいです。
\\	彼の人はお腹が空いているみたいです。	
\\	彼[あ]の 人[ひと]はお 腹[なか]が 空[す]いているみたいです。
\\	彼の人は夕べ寝なかったみたいです。	
\\	彼[あ]の 人[ひと]は 夕[ゆう]べ 寝[ね]なかったみたいです。
\\	先生は明日学校に来ないみたいです。	
\\	先[せん]生[せい]は 明日[あした]学[がっ]校[こう]に 来[こ]ないみたいです。
\\	国に帰る前に、もう一度東京に行きます。	
\\	国[くに]に 帰[かえ]る 前[まえ]に、もう 一[いち]度[ど]東[とう]京[きょう]に 行[い]きます。 
\\	前に 
\\	日本に来る前に、一学期日本語を勉強しました。	
\\	日[に]本[ほん]に 来[く]る 前[まえ]に、 一[いち]学[がっ]期[き]日[に]本[ほん]語[ご]を 勉[べん]強[きょう]しました。 
\\	前に 
\\	勉強してから、友達に手紙を書きました。	
\\	勉[べん]強[きょう]してから、 友[とも]達[だち]に 手[て]紙[がみ]を 書[か]きました。 
\\	てから、 
\\	けんさんが来てから、食べましょう。	
\\	けんさんが 来[き]てから、 食[た]べましょう。 
\\	てから、 
\\	森田	
\\	森[もり]田[た]	
\\	ジョン、今日は森田君、風邪で来られないそうだ。	
\\	ジョン、 今日[きょう]は 森[もり]田[た]君[くん]、 風邪[かぜ]で 来[こ]られないそうだ。
\\	夕方に成ると忙しく成るから、頼むよ。	
\\	夕[ゆう]方[がた]に 成[な]ると 忙[いそが]しく 成[な]るから、 頼[たの]むよ。
\\	はい、頑張ります。	
\\	はい、 頑[がん]張[ば]ります。
\\	先ず、冷蔵庫に野菜が入っているから、出しておいて。	
\\	先[ま]ず、 冷[れい]蔵[ぞう]庫[こ]に 野[や]菜[さい]が 入[はい]っているから、 出[だ]しておいて。
\\	それから、外の電気は点いている?	
\\	それから、 外[そと]の 電[でん]気[き]は 点[つ]いている?
\\	いいえ、点いていません。点けましょうか。	
\\	いいえ、 点[つ]いていません。 点[つ]けましょうか。
\\	うん。其処のスイッチを押すと点くよ。	
\\	うん。 其[そ]処[こ]のスイッチを 押[お]すと 点[つ]くよ。
\\	済みません、醤油を落としちゃったんです。御免なさい。	
\\	済[す]みません、 醤[しょう]油[ゆ]を 落[お]としちゃったんです。 御[ご]免[めん]なさい。
\\	いえ、大丈夫です。あっ、スカートが汚れて仕舞いましたね。	
\\	いえ、 大[だい]丈[じょう]夫[ぶ]です。あっ、スカートが 汚[よご]れて 仕[し]舞[ま]いましたね。
\\	本当だ。どうしよう。	
\\	本[ほん]当[とう]だ。どうしよう。
\\	今直ぐタオルを持って来ます。	
\\	今[いま]直[す]ぐタオルを 持[も]って 来[き]ます。
\\	今日はジョンのお陰で助かったよ。	
\\	今[きょ]日[う]はジョンのお 陰[かげ]で 助[たす]かったよ。
\\	いいえ。でも、本当に忙しかったですね。	
\\	いいえ。でも、 本[ほん]当[とう]に 忙[いそが]しかったですね。
\\	明日は高校が有るんだろう。	
\\	明日[あした]は 高[こう]校[こう]が 有[あ]るんだろう。
\\	アルバイトをし乍ら高校に行くのは大変だね。	
\\	アルバイトをし 乍[なが]ら 高[こう]校[こう]に 行[い]くのは 大[たい]変[へん]だね。
\\	ええ。時々遅刻しちゃうんですよ。	
\\	ええ。 時[とき]々[どき]遅[ち]刻[こく]しちゃうんですよ。
\\	僕も学生の時は良く授業をサボったよ。	
\\	僕[ぼく]も 学[がく]生[せい]の 時[とき]は 良[よ]く 授[じゅ]業[ぎょう]をサボったよ。
\\	もっと勉強すれば良かったなあ。後は僕がやっておくから。	
\\	もっと 勉[べん]強[きょう]すれば 良[よ]かったなあ。 後[あと]は 僕[ぼく]がやっておくから。
\\	じゃあ、お先に失礼します。お疲れ様でした。	
\\	じゃあ、 お 先[さき]に 失[しつ]礼[れい]します。お 疲[つか]れ 様[さま]でした。
\\	お疲れ様。	
\\	お 疲[つか]れ 様[さま]。
\\	たけしさんが電気を点けました。 電気が点きました。	
\\	たけしさんが 電[でん]気[き]を 点[つ]けました。 電[でん]気[き]が 点[つ]きました。
\\	たけしさんがお湯を沸かしました。 お湯が沸きました。	
\\	たけしさんがお 湯[ゆ]を 沸[わ]かしました。 お 湯[ゆ]が 沸[わ]きました。
\\	スーさんは今、携帯電話でお母さんと話しています。	
\\	ている 
\\	スーさんは 今[いま]、 携[けい]帯[たい]電[でん]話[わ]でお 母[かあ]さんと 話[はな]しています。
\\	森田先生は結婚しています。	
\\	ている 
\\	森[もり]田[た]先[せん]生[せい]は 結[けっ]婚[こん]しています。
\\	ロバートさんは窓を開けています。 ドアが開いています。	
\\	ている 
\\	ている 
\\	ロバートさんは 窓[まど]を 開[あ]けています。 ドアが 開[あ]いています。
\\	知子さんは電気を消しています。 テレビは消えています。	
\\	ている 
\\	ている 
\\	知[とも]子[こ]さんは 電[でん]気[き]を 消[け]しています。 テレビは 消[き]えています。
\\	ゴジラが町を壊しています。 此のコンピューターは壊れています。	
\\	ている 
\\	ている 
\\	ゴジラが 町[まち]を 壊[こわ]しています。 此[こ]のコンピューターは 壊[こわ]れています。
\\	本を読んで仕舞いました。	
\\	本[ほん]を 読[よ]んで 仕[し]舞[ま]いました。
\\	仕舞う 
\\	仕舞って 
\\	仕舞います	
\\	仕[し]舞[ま]う 
\\	仕[し]舞[ま]って 
\\	仕[し]舞[ま]います	
\\	電車の中に鞄を忘れて仕舞いました。	
\\	電[でん]車[しゃ]の 中[なか]に 鞄[かばん]を 忘[わす]れて 仕[し]舞[ま]いました。
\\	宿題を忘れたので、 先生は怒って仕舞いました。	
\\	宿[しゅく]題[だい]を 忘[わす]れたので、 先[せん]生[せい]は 怒[おこ]って 仕[し]舞[ま]いました。
\\	宿題を無くしちゃった。	
\\	宿[しゅく]題[だい]を 無[な]くしちゃった。
\\	食事	
\\	食[しょく]事[じ]	ショクジ
\\	私は其の人と話すと元気に成る。	
\\	私[わたし]は 其[そ]の 人[ひと]と 話[はな]すと 元[げん]気[き]に 成[な]る。 
\\	と 
\\	道が込んでいると時間が掛かる。	
\\	道[みち]が 込[こ]んでいると 時[じ]間[かん]が 掛[か]かる。 
\\	と 
\\	メアリーさんが国に帰ると寂しく成ります。	
\\	メアリーさんが 国[くに]に 帰[かえ]ると 寂[さび]しく 成[な]ります。
\\	子供の時、冬に成ると風邪を引きました。	
\\	子[こ]供[ども]の 時[とき]、 冬[ふゆ]に 成[な]ると 風邪[かぜ]を 引[ひ]きました。
\\	秋に成ると木が赤く成ります。	
\\	秋[あき]に 成[な]ると 木[き]が 赤[あか]く 成[な]ります。
\\	夜に成ると町が静かに成ります。	
\\	夜[よる]に 成[な]ると 町[まち]が 静[しず]かに 成[な]ります。
\\	私は何時も音楽を聞き乍ら日本語を勉強します。	
\\	私[わたし]は 何[い]時[つ]も 音[おん]楽[がく]を 聞[き]き 乍[なが]ら 日[に]本[ほん]語[ご]を 勉[べん]強[きょう]します。 
\\	乍[なが]ら 
\\	たけしさんは歌を歌い乍ら洗濯をしています。	
\\	たけしさんは 歌[うた]を 歌[うた]い 乍[なが]ら 洗[せん]濯[たく]をしています。 
\\	乍[なが]ら 
\\	アルバイトをし乍ら学校に行くのは大変です。	
\\	アルバイトをし 乍[なが]ら 学[がっ]校[こう]に 行[い]くのは 大[たい]変[へん]です。 
\\	乍[なが]ら 
\\	彼の時、「愛している」と言えば良かったです。	
\\	彼[あ]の 時[とき]、
\\	愛[あい]している」と 言[い]えば 良[よ]かったです。 
\\	えば 良[よ]かったです 
\\	彼女と別れなければ良かったです。	
\\	彼[かの]女[じょ]と 別[わか]れなければ 良[よ]かったです。 
\\	えば 良[よ]かったです 
\\	食べれば良かったです。	
\\	食[た]べれば 良[よ]かったです。
\\	行けば良かったです。	
\\	行[い]けば 良[よ]かったです。
\\	待てば良かったです。	
\\	待[ま]てば 良[よ]かったです。
\\	買えば良かったです。	
\\	買[か]えば 良[よ]かったです。
\\	来れば良かったです。	
\\	来[く]れば 良[よ]かったです。
\\	すれば良かったです。	
\\	すれば 良[よ]かったです。
\\	食べなければ良かったです。	
\\	食[た]べなければ 良[よ]かったです。
\\	行かなければ良かったです。	
\\	行[い]かなければ 良[よ]かったです。
\\	待たなければ良かったです。	
\\	待[ま]たなければ 良[よ]かったです。
\\	買わなければ良かったです。	
\\	買[か]わなければ 良[よ]かったです。
\\	来なければ良かったです。	
\\	来[こ]なければ 良[よ]かったです。
\\	しなければ良かったです。	
\\	しなければ 良[よ]かったです。
\\	部長、出張お疲れ様でした。	
\\	部[ぶ]長[ちょう]、 出[しゅっ]張[ちょう]お 疲[つか]れ 様[さま]でした。
\\	木村君、迎えに来て呉れて有難う。	
\\	木[き]村[むら]君[くん]、 迎[むか]えに 来[き]て 呉[く]れて 有[あり]難[がと]う。
\\	本当は神戸を一時に出る筈だったんだけど、遅れちゃってね。	
\\	本[ほん]当[とう]は 神[こう]戸[べ]を 一[いち]時[じ]に 出[で]る 筈[はず]だったんだけど、 遅[おく]れちゃってね。
\\	神戸	
\\	神[こう]戸[べ]	コꜜーベ
\\	川崎	
\\	川[かわ]崎[さき]	
\\	じゃあ、お疲れに成ったでしょう。	
\\	じゃあ、お 疲[つか]れに 成[な]ったでしょう。
\\	大丈夫だけど、一寸お腹が空いているんだ。	
\\	大[だい]丈[じょう]夫[ぶ]だけど、 一[ちょっ]寸[と]お 腹[なか]が 空[す]いているんだ。
\\	じゃあ、何か召し上がってからお帰りに成りますか。	
\\	じゃあ、 何[なに]か 召[め]し 上[あ]がってからお 帰[かえ]りに 成[な]りますか。
\\	いらっしゃいませ。何名様ですか。	
\\	いらっしゃいませ。 何[なん]名[めい] 様[さま]ですか。
\\	二人です。	
\\	二[ふた]人[り]です。
\\	お煙草をお吸いに成りますか。	
\\	お 煙草[たばこ]をお 吸[す]いに 成[な]りますか。
\\	此方へどうぞ。お決まりに成りましたらお呼び下さい。	
\\	此[こ]方[ちら]へどうぞ。お 決[き]まりに 成[な]りましたらお 呼[よ]び 下[くだ]さい。
\\	家迄送って呉れて有難う。	
\\	家[うち]迄[まで]送[おく]って 呉[く]れて 有[あり]難[がと]う。
\\	いいえ。今日はご馳走して下さって、有難う御座いました。	
\\	いいえ。 今[きょ]日[う]はご 馳[ち]走[そう]して 下[くだ]さって、 有[あり]難[がと]う 御[ご]座[ざ]いました。
\\	ゆっくり話が出来て良かったよ。一寸家に寄らない?	
\\	ゆっくり 話[はなし]が 出[で]来[き]て 良[よ]かったよ。 一[ちょっ]寸[と]家[うち]に 寄[よ]らない?
\\	いえ、もう遅いし奥様もお休みに成っているでしょう。	
\\	いえ、もう 遅[おそ]いし 奥[おく]様[さま]もお 休[やす]みに 成[な]っているでしょう。
\\	此の時間なら未だ起きている筈だよ。	
\\	此[こ]の 時[じ]間[かん]なら 未[ま]だ 起[お]きている 筈[はず]だよ。
\\	でも、今日は遠慮しておきます。奥様に宜しくお伝え下さい。	
\\	でも、 今日[きょう]は 遠[えん]慮[りょ]しておきます。 奥[おく]様[さま]に 宜[よろ]しくお 伝[つた]え 下[くだ]さい。
\\	出迎え	
\\	出[で]迎[むか]え	デムカエ
\\	先生は今日学校にいらっしゃいません。 
\\	行きません 
\\	来ません 
\\	居ません)	
\\	先[せん]生[せい]は 今[きょ]日[う]学[がっ]校[こう]にいらっしゃいません。 
\\	行[い]きません 
\\	来[き]ません 
\\	居[い]ません)
\\	何を召し上がりますか。	
\\	何[なに]を 召[め]し 上[あ]がりますか。
\\	田中さんのお母さんが此の本を下さいました。 
\\	呉れました)	
\\	田[た]中[なか]さんのお 母[かあ]さんが 此[こ]の 本[ほん]を 下[くだ]さいました。 
\\	呉[く]れました)
\\	先生は自分で料理なさるそうです。 
\\	料理するそうです)	
\\	先[せん]生[せい]は 自[じ]分[ぶん]で 料[りょう]理[り]なさるそうです。 
\\	料[りょう]理[り]するそうです)
\\	心配なさらないで下さい。 
\\	心配しないで下さい)	
\\	心[しん]配[ぱい]なさらないで 下[くだ]さい。 
\\	心[しん]配[ぱい]しないで 下[くだ]さい)
\\	先生はテレビをご覧に成っています。 
\\	見ています)	
\\	先[せん]生[せい]はテレビをご 覧[らん]に 成[な]っています。 
\\	見[み]ています)
\\	先生はテープを聞いていらっしゃいます。 
\\	聞いています)	
\\	先[せん]生[せい]はテープを 聞[き]いていらっしゃいます。 
\\	聞[き]いています) (〜ていらっしゃる 
\\	ている)
\\	先生は疲れていらっしゃるみたいです。 
\\	疲れているみたいです)	
\\	先[せん]生[せい]は 疲[つか]れていらっしゃるみたいです。 
\\	疲[つか]れているみたいです) (〜ていらっしゃる 
\\	先生はもうお帰りに成りました。 
\\	帰りました) (お 
\\	に成る 
\\	先[せん]生[せい]はもうお 帰[かえ]りに 成[な]りました。 
\\	帰[かえ]りました) (お 
\\	に 成[な]る 
\\	此の雑誌をお読みに成ったことが有りますか。 
\\	読んだことが有りますか) (お 
\\	に成る 
\\	此[こ]の 雑[ざっ]誌[し]をお 読[よ]みに 成[な]ったことが 有[あ]りますか。 
\\	読[よ]んだことが有りますか) (お 
\\	に 成[な]る 
\\	どうぞお使いに成って下さい。 
\\	使って下さい)	
\\	どうぞお 使[つか]いに 成[な]って 下[くだ]さい。 
\\	使[つか]って 下[くだ]さい) (お 
\\	に 成[な]る 
\\	整理券をお取り下さい。 
\\	取る) (お 
\\	下さい 
\\	整[せい]理[り] 券[けん]をお 取[と]り 下[くだ]さい。 
\\	取[と]る) (お 
\\	下[くだ]さい 
\\	説明をお読み下さい。 
\\	読む)	
\\	説[せつ]明[めい]をお 読[よ]み 下[くだ]さい。 
\\	読[よ]む) (お 
\\	下[くだ]さい 
\\	ご注意下さい。 
\\	注意する)	
\\	ご 注[ちゅう]意[い]下[くだ]さい。 
\\	注[ちゅう]意[い]する)
\\	ご覧下さい。 
\\	ご覧に成る 
\\	見る)	
\\	ご 覧[らん]下[くだ]さい。 
\\	ご 覧[らん]に成る 
\\	見[み]る)
\\	お召し上がり下さい。 
\\	召し上がる 
\\	食べる/飲む)	
\\	お 召[め]し 上[あ]がり 下[くだ]さい。 
\\	召[め]し 上[あ]がる 
\\	食[た]べる/ 飲[の]む)
\\	お休み下さい。 
\\	お休みに成る 
\\	寝る)	
\\	お 休[やす]み 下[くだ]さい。 
\\	お 休[やす]みに 成[な]る 
\\	寝[ね]る)
\\	手伝って呉れて有難う。	
\\	手[て]伝[つだ]って 呉[く]れて 有[あり]難[がと]う。 
\\	呉[く]れて 有[あり]難[がと]う 
\\	推薦状を書いて下さって有難う御座いました。	
\\	推[すい]薦[せん] 状[じょう]を 書[か]いて 下[くだ]さって 有[あり]難[がと]う 御[ご]座[ざ]いました。 
\\	下[くだ]さって 有[あり]難[がと]う 御[ご]座[ざ]いました 
\\	日本語を勉強して良かったです。 
\\	良かった 
\\	日[に]本[ほん]語[ご]を 勉[べん]強[きょう]して 良[よ]かったです。 
\\	良[よ]かった 
\\	知子さんが元気に成って良かったです。 
\\	良かった 
\\	知[とも]子[こ]さんが 元[げん]気[き]に 成[な]って 良[よ]かったです。 
\\	良[よ]かった 
\\	昨日のパーティーに行かなくて良かったです。 
\\	良かった 
\\	昨[きの]日[う]のパーティーに 行[い]かなくて 良[よ]かったです。 
\\	良[よ]かった 
\\	良い友達でいて呉れて有難う。	
\\	良[い]い 友[とも]達[だち]でいて 呉[く]れて 有[あり]難[がと]う。 
\\	呉[く]れて 有[あり]難[がと]う 
\\	でいる 
\\	今日は日曜日だから、銀行は閉まっている筈です。 
\\	筈です 
\\	今[きょ]日[う]は 日[にち]曜[よう]日[び]だから、 銀[ぎん]行[こう]は 閉[し]まっている 筈[はず]です。 
\\	筈[はず]です 
\\	昨日花子さんは何処にも行かなかった筈です。 
\\	筈です 
\\	昨[きの]日[う]花[はな]子[こ]さんは 何[ど]処[こ]にも 行[い]かなかった 筈[はず]です。 
\\	筈[はず]です 
\\	花子	
\\	花[はな]子[こ]	
\\	先週電話を貰う筈でしたが、電話が有りませんでした。 
\\	筈です 
\\	先[せん]週[しゅう]電[でん]話[わ]を 貰[もら]う 筈[はず]でしたが、 電[でん]話[わ]が 有[あ]りませんでした。 
\\	筈[はず]です 
\\	面白い筈です	
\\	面[おも]白[しろ]い 筈[はず]です
\\	元気な筈です	
\\	元[げん]気[き]な 筈[はず]です
\\	日本人の筈です	
\\	日[に]本[ほん]人[じん]の 筈[はず]です
\\	太郎	
\\	太[た]郎[ろう]	
\\	済みません。此の間此のウオークマンを買ったんですが、音が聞こえないんです。	
\\	済[す]みません。 此[こ]の 間[かん]此[こ]のウオークマンを 買[か]ったんですが、 音[おと]が 聞[き]こえないんです。
\\	少々、お待ち下さい。今、係の者を呼んで参ります。	
\\	少[しょう]々[しょう]、お 待[ま]ち 下[くだ]さい。 今[いま]、 係[かかり]の 者[もの]を 呼[よ]んで 参[まい]ります。
\\	お待たせしました。田中と申します。ウオークマンを見せて頂けますか。	
\\	お 待[ま]たせしました。 田[た]中[なか]と 申[もう]します。ウオークマンを 見[み]せて 頂[いただ]けますか。
\\	壊れているみたいですね。失礼致しました。	
\\	壊[こわ]れているみたいですね。 失[しつ]礼[れい]致[いた]しました。
\\	宜しかったら交換致しますが。	
\\	宜[よろ]しかったら 交[こう]換[かん]致[いた]しますが。
\\	申し訳御座いません。今、同じ物が御座いませんので、二、三週間待って頂けませんか。	
\\	申[もう]し 訳[わけ]御[ご]座[ざ]いません。 今[いま]、 同[おな]じ 物[もの]が 御[ご]座[ざ]いませんので、 二[に]、 三[さん]週[しゅう]間[かん]待[ま]って 頂[いただ]けませんか。
\\	其れは一寸・・・。もう直ぐ国に帰るので、出来れば返品したいんですが。	
\\	其[そ]れは 一[ちょっ]寸[と]・・・。もう 直[す]ぐ 国[くに]に 帰[かえ]るので、 出[で]来[き]れば 返[へん]品[ぴん]したいんですが。
\\	そうですか。畏まりました。誠に申し訳御座いませんでした。	
\\	そうですか。 畏[かしこ]まりました。 誠[まこと]に 申[もう]し 訳[わけ]御[ご]座[ざ]いませんでした。
\\	西木屋と言う店は何処に有るか教えて頂けませんか。	
\\	西[にし]木[き]屋[や]と 言[い]う 店[みせ]は 何[ど]処[こ]に 有[あ]るか 教[おし]えて 頂[いただ]けませんか。
\\	地図が有るんですけど、分かり難いんです。	
\\	地[ち]図[ず]が 有[あ]るんですけど、 分[わ]かり 難[にく]いんです。
\\	扇子の店ですね。次の角を左に曲がったら見えますよ。	
\\	扇[せん]子[す]の 店[みせ]ですね。 次[つぎ]の 角[かど]を 左[ひだり]に 曲[ま]がったら 見[み]えますよ。
\\	扇子を買いに行くんですか。	
\\	扇[せん]子[す]を 買[か]いに 行[い]くんですか。
\\	お土産に扇子を買おうと思っているんです。	
\\	お 土産[みやげ]に 扇[せん]子[す]を 買[か]おうと 思[おも]っているんです。
\\	良いお土産に成りますよ。おや、雨ですね。傘を持っていますか。	
\\	良[い]いお 土産[みやげ]に 成[な]りますよ。おや、 雨[あめ]ですね。 傘[かさ]を 持[も]っていますか。
\\	急いでいたから、傘を持たないで来ちゃったんです。	
\\	急[いそ]いでいたから、 傘[かさ]を 持[も]たないで 来[き]ちゃったんです。
\\	じゃあ、一緒に店迄行きましょう。	
\\	じゃあ、 一[いっ]緒[しょ]に 店[みせ]迄[まで]行[い]きましょう。
\\	どうも済みません。荷物が重そうですね。お持ちします。	
\\	どうも 済[す]みません。 荷[に]物[もつ]が 重[おも]そうですね。お 持[も]ちします。
\\	道場	
\\	道[どう]場[じょう] 
\\	同情)	ドꜜージョー
\\	本部	
\\	本[ほん]部[ぶ]	ホꜜンブ
\\	合気道	
\\	合[あい]気[き]道[どう]	アイキꜜドー
\\	合気道家	
\\	合[あい]気[き]道[どう]家[か]	
\\	私は来年も日本に居ります。 
\\	居ます)	
\\	私[わたくし]は 来[らい]年[ねん]も 日[に]本[ほん]に 居[お]ります。 
\\	居[い]ます)
\\	私は今年の六月に大学を卒業致しました。 
\\	卒業しました)	
\\	私[わたくし]は 今年[ことし]の 六[ろく]月[がつ]に 大[だい]学[がく]を 卒[そつ]業[ぎょう]致[いた]しました。 
\\	卒[そつ]業[ぎょう]しました)
\\	私は一年間日本語を勉強しております。 
\\	勉強しています)	
\\	私[わたくし]は 一[いち]年[ねん]間[かん]日[に]本[ほん]語[ご]を 勉[べん]強[きょう]しております。 
\\	勉[べん]強[きょう]しています) (〜ておる 
\\	〜ている)
\\	私は日本の文化に興味が御座います。 
\\	興味が有ります)	
\\	私[わたくし]は 日[に]本[ほん]の 文[ぶん]化[か]に 興[きょう]味[み]が 御[ご]座[ざ]います。 
\\	興[きょう]味[み]が 有[あ]ります)
\\	私の母は医者で御座います。 
\\	です)	
\\	私[わたくし]の 母[はは]は 医[い]者[しゃ]で 御[ご]座[ざ]います。 
\\	です)
\\	電車が参ります。 
\\	来ます)	
\\	電[でん]車[しゃ]が 参[まい]ります。 
\\	来[き]ます)
\\	お手洗いは二階で御座います。 
\\	です)	
\\	お 手[て]洗[あら]いは 二[に]階[かい]で 御[ご]座[ざ]います。 
\\	です)
\\	私は昨日先生にお会いしました。	
\\	私[わたし]は 昨[きの]日[う]先[せん]生[せい]にお 会[あ]いしました。 (お 
\\	する 
\\	私は先生に本をお貸しするつもりです。 (お 
\\	する 
\\	私[わたし]は 先[せん]生[せい]に 本[ほん]をお 貸[か]しするつもりです。 (お 
\\	する 
\\	私は先生に辞典をお借りしました。	
\\	私[わたし]は 先[せん]生[せい]に 辞[じ]典[てん]をお 借[か]りしました。 (お 
\\	する 
\\	ご紹介する	
\\	ご 紹[しょう]介[かい]する 
\\	する 
\\	ご 
\\	お 
\\	ご案内する	
\\	ご 案[あん]内[ない]する 
\\	する 
\\	ご 
\\	お 
\\	ご説明する	
\\	ご 説[せつ]明[めい]する 
\\	する 
\\	ご 
\\	お 
\\	お電話する	
\\	する)	お 電[でん]話[わ]する 
\\	する 
\\	ご 
\\	お 
\\	私は先生に此の本を頂きました。	
\\	私[わたし]は 先[せん]生[せい]に 此[こ]の 本[ほん]を 頂[いただ]きました。 
\\	頂[いただ]く 
\\	貰[もら]う)
\\	私は先生に漢字を教えて頂きました。 (頂く 
\\	貰う)	
\\	私[わたし]は 先[せん]生[せい]に 漢[かん]字[じ]を 教[おし]えて 頂[いただ]きました。 
\\	頂[いただ]く 
\\	貰[もら]う)
\\	私は先生に贈り物を差し上げます。 (差し上げる 
\\	上げる)	
\\	私[わたし]は 先[せん]生[せい]に 贈[おく]り 物[もの]を 差[さ]し 上[あ]げます。 
\\	差[さ]し 上[あ]げる 
\\	上[あ]げる)
\\	私は先生に地図をお見せしました。	
\\	私[わたし]は 先[せん]生[せい]に 地[ち]図[ず]をお 見[み]せしました。 (お 
\\	する 
\\	私の父はお客さんにお茶をお淹れしました。 (お 
\\	する 
\\	私[わたし]の 父[ちち]はお 客[きゃく]さんにお 茶[ちゃ]をお 淹[い]れしました。 (お 
\\	する 
\\	私の母は大統領に手紙を頂きました。	
\\	私[わたし]の 母[はは]は 大[だい]統[とう]領[りょう]に 手[て]紙[がみ]を 頂[いただ]きました。 
\\	頂[いただ]く 
\\	貰[もら]う)
\\	(私は貴方を)駅迄お送りします。	
\\	私[わたし]は 貴方[あなた]を) 駅[えき]迄[まで]お 送[おく]りします。
\\	テレビをお点けしましょう。	
\\	テレビをお 点[つ]けしましょう。
\\	先生が
\\	いらっしゃいました/お帰りに成りました。	
\\	先[せん]生[せい]が
\\	いらっしゃいました/お 帰[かえ]りに 成[な]りました。 
\\	私は
\\	山田太郎と申します。	
\\	私[わたし]は
\\	山[やま]田[だ]太[た]郎[ろう]と 申[もう]します。 
\\	私は
\\	先生に
\\	本をお返ししました。	
\\	私[わたし]は
\\	先[せん]生[せい]に
\\	本[ほん]をお 返[かえ]ししました。 
\\	夕べは、寝ないで、勉強しました。	
\\	夕[ゆう]べは、 寝[ね]ないで、 勉[べん]強[きょう]しました。 
\\	ないで 
\\	辞典を使わないで、雑誌を読みます。	
\\	辞[じ]典[てん]を 使[つか]わないで、 雑[ざっ]誌[し]を 読[よ]みます。 
\\	ないで 
\\	森田先生は昨日何を食べたか覚えていません。	
\\	森[もり]田[た]先[せん]生[せい]は 昨[きの]日[う]何[なに]を 食[た]べたか 覚[おぼ]えていません。 
\\	何 
\\	誰 
\\	か 
\\	かどうか)
\\	花子さんが何処に住んでいるか知っていますか。	
\\	花[はな]子[こ]さんが 何[ど]処[こ]に 住[す]んでいるか 知[し]っていますか。 
\\	何 
\\	誰 
\\	か 
\\	かどうか)
\\	私は明日学校に行くかどうか分かりません。	
\\	私[わたし]は 明日[あした]学[がっ]校[こう]に 行[い]くかどうか 分[わ]かりません。 
\\	何 
\\	誰 
\\	か 
\\	かどうか)
\\	誰が一番上手だか分かりません。	
\\	誰[だれ]が 一[いち]番[ばん]上[じょう]手[ず]だか 分[わ]かりません。 (だ 
\\	彼の人が消防士だかどうか分かりません。	
\\	彼[あ]の 人[ひと]が 消[しょう]防[ぼう]士[し]だかどうか 分[わ]かりません。 (だ 
\\	太郎さんは何を食べましたか。 
\\	私は太郎さんが何を食べたか知りません。	
\\	太[た]郎[ろう]さんは 何[なに]を 食[た]べましたか。 
\\	私[わたし]は 太[た]郎[ろう]さんが 何[なに]を 食[た]べたか 知[し]りません。 (は 
\\	が 
\\	と言う 
\\	と 言[い]う 
\\	ポチと言う犬(を飼っていました。)	
\\	ポチと 言[い]う 犬[いぬ](を 飼[か]っていました。)
\\	「花」と言う歌(を歌いました。)	
\\	花[はな]」と 言[い]う 歌[うた](を 歌[うた]いました。)
\\	使う 
\\	使い易い	
\\	使[つか]う 
\\	使[つか]い 易[やす]い
\\	読む 
\\	読み易い	
\\	読[よ]む 
\\	読[よ]み 易[やす]い
\\	此のコンピューターは使い易いです。	
\\	此[こ]のコンピューターは 使[つか]い 易[やす]いです。
\\	此の小説は読み易いです。	
\\	此[こ]の 小[しょう]説[せつ]は 読[よ]み 易[やす]いです。
\\	~易い	
\\	易[やす]い
\\	難い	
\\	難[にく]い
\\	食べる 
\\	食べ難い	
\\	食[た]べる 
\\	食[た]べ 難[にく]い
\\	骨	
\\	骨[ほね]	コツꜜ 
\\	コツ 
\\	ホネꜜ
\\	(骨が多くて)魚は食べ難いです。	
\\	骨[ほね]が 多[おお]くて) 魚[さかな]は 食[た]べ 難[にく]いです。
\\	此の町は迚も住み易いです。	
\\	此[こ]の 町[まち]は 迚[とて]も 住[す]み 易[やす]いです。
\\	此のコップは飲み難いです。	
\\	此[こ]のコップは 飲[の]み 難[にく]いです。
\\	此の椅子は座り難い。	
\\	此[こ]の 椅[い]子[す]は 座[すわ]り 難[にく]い。 
\\	易[やす]い 
\\	難[にく]い 
\\	此の椅子に座るのは難しい。	
\\	此[こ]の 椅[い]子[す]に 座[すわ]るのは 難[むずか]しい。 
\\	易[やす]い 
\\	難[にく]い 
\\	此の雑誌は買い難い。	
\\	此[こ]の 雑[ざっ]誌[し]は 買[か]い 難[にく]い。 
\\	易[やす]い 
\\	難[にく]い 
\\	此の雑誌を買うのは難しい。	
\\	此[こ]の 雑[ざっ]誌[し]を 買[か]うのは 難[むずか]しい。 
\\	易[やす]い 
\\	難[にく]い 
\\	大家さん、大変です。泥棒に入られました。	
\\	大[おお]家[や]さん、 大[たい]変[へん]です。 泥[どろ]棒[ぼう]に 入[はい]られました。
\\	えっ。何か取られましたか。	
\\	えっ。 何[なに]か 取[と]られましたか。
\\	コンピューターと… バイトで溜めたお金も有りません。	
\\	コンピューターと… バイトで 溜[た]めたお 金[かね]も 有[あ]りません。
\\	兎に角警察に連絡した方がいいですよ。	
\\	兎[と]に 角[かく]警[けい]察[さつ]に 連[れん]絡[らく]した 方[ほう]がいいですよ。
\\	鍵を掛けて有ったんですか。	
\\	鍵[かぎ]を 掛[か]けて 有[あ]ったんですか。
\\	さあ…夕べは飲んで帰ったから、鍵を掛けたかどうか良く覚えていないんです。	
\\	さあ… 夕[ゆう]べは 飲[の]んで 帰[かえ]ったから、 鍵[かぎ]を 掛[か]けたかどうか 良[よ]く 覚[おぼ]えていないんです。
\\	何時頃帰ったか覚えていますか。	
\\	何[なん]時[じ]頃[ごろ]帰[かえ]ったか 覚[おぼ]えていますか。
\\	終電だったから… 多分一時半頃です。	
\\	終[しゅう]電[でん]だったから… 多[た]分[ぶん]一[いち]時[じ]半[はん]頃[ごろ]です。
\\	泥棒は其の後入ったんですね。	
\\	泥[どろ]棒[ぼう]は 其[そ]の 後[あと]入[はい]ったんですね。
\\	ええ、朝、部屋が目茶苦茶だったんで、吃驚したんです。	
\\	ええ、 朝[あさ]、 部[へ]屋[や]が 目[め]茶[ちゃ]苦[く]茶[ちゃ]だったんで、 吃[びっ]驚[くり]したんです。
\\	寝ている間に泥棒に入られて、気が付かなかったんですか。	
\\	寝[ね]ている 間[あいだ]に 泥[どろ]棒[ぼう]に 入[はい]られて、 気[き]が 付[つ]かなかったんですか。
\\	ジョンさん、留守の間に、警察から電話が有りましたよ。	
\\	ジョンさん、 留[る]守[す]の 間[あいだ]に、 警[けい]察[さつ]から 電[でん]話[わ]が 有[あ]りましたよ。
\\	犯人が捕まったので、警察に来て欲しいそうです。	
\\	犯[はん]人[にん]が 捕[つか]まったので、 警[けい]察[さつ]に 来[き]て 欲[ほ]しいそうです。
\\	私の財布です。良かった。	
\\	私[わたし]の 財[さい]布[ふ]です。 良[よ]かった。
\\	それから、鍵を新しくしましたから、どうぞ。本当に大変でしたね。	
\\	それから、 鍵[かぎ]を 新[あたら]しくしましたから、どうぞ。 本[ほん]当[とう]に 大[たい]変[へん]でしたね。
\\	ええ、でも其のお陰で、良い事も有りました。	
\\	ええ、でも 其[そ]のお 陰[かげ]で、 良[い]い 事[こと]も 有[あ]りました。
\\	皆色々な物を呉れたり、奢って呉れたりしたんです。	
\\	皆[みんな]色[いろ]々[いろ]な 物[もの]を 呉[く]れたり、 奢[おご]って 呉[く]れたりしたんです。
\\	ジョンさんは良い友達が沢山居て、幸せですね。	
\\	ジョンさんは 良[い]い 友[とも]達[だち]が 沢[たく]山[さん]居[い]て、 幸[しあわ]せですね。
\\	友達が車を使いました。 
\\	私は友達に車を使われました。	
\\	友[とも]達[だち]が 車[くるま]を 使[つか]いました。 
\\	私[わたし]は 友[とも]達[だち]に 車[くるま]を 使[つか]われました。
\\	食べられる 
\\	食べられて 
\\	食べられます	
\\	食[た]べられる 
\\	食[た]べられて 
\\	食[た]べられます
\\	行かれる 
\\	行かれて 
\\	行かれます	
\\	行[い]かれる 
\\	行[い]かれて 
\\	行[い]かれます
\\	待たれる 
\\	待たれて 
\\	待たれます	
\\	待[ま]たれる 
\\	待[ま]たれて 
\\	待[ま]たれます
\\	読まれる 
\\	読まれて 
\\	読まれます	
\\	読[よ]まれる 
\\	読[よ]まれて 
\\	読[よ]まれます
\\	泳がれる 
\\	泳がれて 
\\	泳がれます	
\\	泳[およ]がれる 
\\	泳[およ]がれて 
\\	泳[およ]がれます
\\	買われる 
\\	買われて 
\\	買われます	
\\	買[か]われる 
\\	買[か]われて 
\\	買[か]われます
\\	話される 
\\	話されて 
\\	話されます	
\\	話[はな]される 
\\	話[はな]されて 
\\	話[はな]されます
\\	死なれる 
\\	死なれて 
\\	死なれます	
\\	死[し]なれる 
\\	死[し]なれて 
\\	死[し]なれます
\\	取られる 
\\	取られて 
\\	取られます	
\\	取[と]られる 
\\	取[と]られて 
\\	取[と]られます
\\	遊ばれる 
\\	遊ばれて 
\\	遊ばれます	
\\	遊[あそ]ばれる 
\\	遊[あそ]ばれて 
\\	遊[あそ]ばれます
\\	来られる 
\\	来られて 
\\	来られます	
\\	来[こ]られる 
\\	来[こ]られて 
\\	来[こ]られます
\\	される 
\\	されて 
\\	されます	
\\	される 
\\	されて 
\\	されます
\\	読まれる 
\\	読まれない	
\\	読[よ]まれる 
\\	読[よ]まれない
\\	読まれた 
\\	読まれなかった	
\\	読[よ]まれた 
\\	読[よ]まれなかった
\\	私は隣の人に煙草を吸われました。	
\\	私[わたし]は 隣[となり]の 人[ひと]に 煙草[たばこ]を 吸[す]われました。
\\	太郎さんは花子さんに良く笑われました。	
\\	太[た]郎[ろう]さんは 花[はな]子[こ]さんに 良[よ]く 笑[わら]われました。
\\	森田先生は誰かにパスワードを盗まれたそうです。	
\\	森[もり]田[た]先[せん]生[せい]は 誰[だれ]かにパスワードを 盗[ぬす]まれたそうです。
\\	私は友達に日記を読まれました。 私は友達に手紙を読んで貰いました。	
\\	私[わたし]は 友[とも]達[だち]に 日[にっ]記[き]を 読[よ]まれました。 私[わたし]は 友[とも]達[だち]に 手[て]紙[がみ]を 読[よ]んで 貰[もら]いました。
\\	私は其の人にデートに誘われました。	
\\	私[わたし]は 其[そ]の 人[ひと]にデートに 誘[さそ]われました。 
\\	私は兄に友達に紹介されました。	
\\	私[わたし]は 兄[あに]に 友[とも]達[だち]に 紹[しょう]介[かい]されました。 
\\	私は友達にパーティーに呼ばれました。	
\\	私[わたし]は 友[とも]達[だち]にパーティーに 呼[よ]ばれました。 
\\	尊敬	
\\	尊[そん]敬[けい]	ソンケイ
\\	尊敬する	
\\	尊[そん]敬[けい]する
\\	其の人は皆に尊敬されています。	
\\	其[そ]の 人[ひと]は 皆[みんな]に 尊[そん]敬[けい]されています。 
\\	炬燵	
\\	炬[こ]燵[たつ]	コタツ
\\	寒いので、ストーブが点けて有ります。	
\\	寒[さむ]いので、ストーブが 点[つ]けて 有[あ]ります。 
\\	有る 
\\	炬燵の上に本が置いて有ります。	
\\	炬[こ]燵[たつ]の 上[うえ]に 本[ほん]が 置[お]いて 有[あ]ります。 
\\	有る 
\\	レストランの予約がして有ります。 
\\	レストランの予約がしておきました。	
\\	レストランの 予[よ]約[やく]がして 有[あ]ります。 
\\	レストランの 予[よ]約[やく]がしておきました。 
\\	~て有ります 
\\	パンが買って有ります。 
\\	パンを買っておきました。	
\\	パンが 買[か]って 有[あ]ります。 
\\	パンを 買[か]っておきました。 
\\	~て有ります 
\\	窓が閉めて有ります。 (閉める 
\\	窓が閉まっています。 (閉まる 
\\	窓[まど]が 閉[し]めて 有[あ]ります。 
\\	閉[し]める 
\\	窓[まど]が 閉[し]まっています。 
\\	閉[し]まる 
\\	(て 有[あ]る 
\\	ている 
\\	(ている) 間に 
\\	(ている) 間[あいだ]に 
\\	お風呂に入っている間に電話が有りました。	
\\	お 風[ふ]呂[ろ]に 入[はい]っている 間[あいだ]に 電[でん]話[わ]が 有[あ]りました。
\\	夕べ、寝ている間に地震が有りました。	
\\	夕[ゆう]べ、 寝[ね]ている 間[あいだ]に 地[じ]震[しん]が 有[あ]りました。
\\	両親が日本に居る間に神戸に連れて行きたいです。	
\\	両[りょう]親[しん]が 日[に]本[ほん]に 居[い]る 間[あいだ]に 神[こう]戸[べ]に 連[つ]れて 行[い]きたいです。
\\	留守の間に警察官が来ました。	
\\	留[る]守[す]の 間[あいだ]に 警[けい]察[さつ]官[かん]が 来[き]ました。
\\	ルームメートがコンピューターを使っている間、私は本を読んで待ちました。	
\\	ルームメートがコンピューターを 使[つか]っている 間[あいだ]、 私[わたし]は 本[ほん]を 読[よ]んで 待[ま]ちました。 
\\	間[あいだ] 
\\	間[あいだ]に)
\\	冷たい 
\\	冷たくする	
\\	冷[つめ]たい 
\\	冷[つめ]たくする
\\	簡単な 
\\	簡単にする	
\\	簡[かん]単[たん]な 
\\	簡[かん]単[たん]にする
\\	此の間の試験は難し過ぎたので、次の試験は易しくして下さい。	
\\	此[こ]の 間[かん]の 試[し]験[けん]は 難[むずか]し 過[す]ぎたので、 次[つぎ]の 試[し]験[けん]は 易[やさ]しくして 下[くだ]さい。
\\	皆で世界を良くしましょう。	
\\	皆[みんな]で 世[せ]界[かい]を 良[よ]くしましょう。
\\	部屋を綺麗にしました。	
\\	部[へ]屋[や]を 綺[き]麗[れい]にしました。
\\	首相	
\\	首[しゅ]相[しょう]	シュショー
\\	小泉	
\\	小[こ]泉[いずみ]	
\\	髪を小泉首相みたいにしたいです。	
\\	髪[かみ]を 小[こ]泉[いずみ]首[しゅ]相[しょう]みたいにしたいです。
\\	静かにする 
\\	静かにして下さい。	
\\	静[しず]かにする 
\\	静[しず]かにして 下[くだ]さい。
\\	私は病気の生徒に元気に成って欲しいです。	
\\	私[わたし]は 病[びょう]気[き]の 生[せい]徒[と]に 元[げん]気[き]に 成[な]って 欲[ほ]しいです。 
\\	欲[ほ]しい 
\\	に)
\\	私は先輩に宿題を手伝って欲しかったです。	
\\	私[わたし]は 先[せん]輩[ぱい]に 宿[しゅく]題[だい]を 手[て]伝[つだ]って 欲[ほ]しかったです。 
\\	欲[ほ]しい 
\\	に)
\\	(私は) 
\\	に 
\\	欲しい	
\\	私[わたし]は) 
\\	に 
\\	欲[ほ]しい
\\	私は叔父さんに昔の話をして欲しくありません。	
\\	私[わたし]は 叔父[おじ]さんに 昔[むかし]の 話[はなし]をして 欲[ほ]しくありません。 
\\	欲[ほ]しい 
\\	~て 欲[ほ]しくありません 
\\	~ないで 欲[ほ]しいです)
\\	私はホストファミリーに英語で話さないで欲しいです。	
\\	私[わたし]はホストファミリーに 英[えい]語[ご]で 話[はな]さないで 欲[ほ]しいです。 
\\	欲[ほ]しい 
\\	~て 欲[ほ]しくありません 
\\	~ないで 欲[ほ]しいです)
\\	ベトナム	
\\	ベトナム	
\\	私はベトナムに行きたいです。 私は漢字の辞書が欲しいです。 私は首相に電話をして欲しいです。	
\\	私[わたし]はベトナムに 行[い]きたいです。 
\\	たい 
\\	私[わたし]は 漢[かん]字[じ]の 辞[じ]書[しょ]が 欲[ほ]しいです。 
\\	が 欲[ほ]しい 
\\	私[わたし]は 首[しゅ]相[しょう]に 電[でん]話[わ]をして 欲[ほ]しいです。 
\\	欲[ほ]しい 
\\	先生は生徒に沢山勉強して欲しいと言っています。 先生は生徒に沢山勉強して欲しがっています。	
\\	先[せん]生[せい]は 生[せい]徒[と]に 沢[たく]山[さん]勉[べん]強[きょう]して 欲[ほ]しいと 言[い]っています。 先[せん]生[せい]は 生[せい]徒[と]に 沢[たく]山[さん]勉[べん]強[きょう]して 欲[ほ]しがっています。
\\	ゆみ、勉強しなさい。	
\\	ゆみ、 勉[べん]強[きょう]しなさい。
\\	来週は期末試験が有るのに、全然勉強していないでしょ。	
\\	来[らい]週[しゅう]は 期[き]末[まつ] 試[し]験[けん]が 有[あ]るのに、 全[ぜん]然[ぜん]勉[べん]強[きょう]していないでしょ。
\\	お母さん、私もう十七なんだから少しほっておいてよ。	
\\	お 母[かあ]さん、 私[わたし]もう 十[じゅう]七[なな]なんだから 少[すこ]しほっておいてよ。
\\	今、頑張っておけば、良い大学に入れて、後で楽に成るんだから。	
\\	今[いま]、 頑[がん]張[ば]っておけば、 良[い]い 大[だい]学[がく]に 入[はい]れて、 後[あと]で 楽[らく]に 成[な]るんだから。
\\	私、別に良い大学に行けなくてもいい。	
\\	私[わたし]、 別[べつ]に 良[い]い 大[だい]学[がく]に 行[い]けなくてもいい。
\\	お父さんとお母さんは、貴方を良い大学に行かせて上げたいの。	
\\	お 父[とう]さんとお 母[かあ]さんは、 貴方[あなた]を 良[い]い 大[だい]学[がく]に 行[い]かせて 上[あ]げたいの。
\\	お母さんの家は貧乏だったから、良い大学に行かせて呉れなかったのよ。	
\\	お 母[かあ]さんの 家[うち]は 貧[びん]乏[ぼう]だったから、 良[い]い 大[だい]学[がく]に 行[い]かせて 呉[く]れなかったのよ。
\\	分かった、わかった。其の話はもう何度も聞いた。	
\\	分[わ]かった、 分[わ]かった。 其[そ]の 話[はなし]はもう 何[なん] 度[ど]も 聞[き]いた。
\\	メアリーも高校の時こんな風だった。	
\\	メアリーも 高[こう]校[こう]の 時[とき]こんな 風[ふう]だった。
\\	そうね、矢っ張り親は煩かったけど、もう少し自由が有ったかな。	
\\	そうね、 矢[や]っ 張[ぱ]り 親[おや]は 煩[うるさ]かったけど、もう 少[すこ]し 自[じ]由[ゆう]が 有[あ]ったかな。
\\	うちの親、一寸変だと思うでしょ。	
\\	うちの 親[おや]、 一[ちょっ]寸[と]変[へん]だと 思[おも]うでしょ。
\\	そんな事ないよ。一寸厳しいかも知れないけど、ゆみちゃんの事を心配しているんだよ。	
\\	そんな 事[こと]ないよ。 一[ちょっ]寸[と]厳[きび]しいかも 知[し]れないけど、ゆみちゃんの 事[こと]を 心[しん]配[ぱい]しているんだよ。
\\	うちのゆみちゃん、高校生なのに忙しくて全然遊ぶ時間が無いみたい。	
\\	うちのゆみちゃん、 高[こう]校[こう]生[せい]なのに 忙[いそが]しくて 全[ぜん]然[ぜん]遊[あそ]ぶ 時[じ]間[かん]が 無[な]いみたい。
\\	子供も大変だけど、親も大変だと思うよ。	
\\	子[こ]供[ども]も 大[たい]変[へん]だけど、 親[おや]も大[たい]変[へん]だと 思[おも]うよ。
\\	塾に行かせたり、英会話を習わせたり、お金が掛かるだろうな。	
\\	塾[じゅく]に 行[い]かせたり、 英[えい]会[かい]話[わ]を 習[なら]わせたり、お 金[かね]が 掛[か]かるだろうな。
\\	たけし君も子供の時、ゆみちゃんのように塾に行ってた?	
\\	たけし 君[くん]も 子[こ]供[ども]の 時[とき]、ゆみちゃんのように 塾[じゅく]に 行[い]ってた?
\\	僕はずっと遊んでた。	
\\	僕[ぼく]はずっと 遊[あそ]んでた。
\\	自分の子供にものんびり遊ばせてあげたいなあ。	
\\	自[じ]分[ぶん]の 子[こ]供[ども]にものんびり 遊[あそ]ばせてあげたいなあ。
\\	でも、日本で子供を育てるのは大変そうだね。	
\\	でも、 日[に]本[ほん]で 子[こ]供[ども]を 育[そだ]てるのは 大[たい]変[へん]そうだね。
\\	食べさせる 
\\	食べさせて 
\\	食べさせます	
\\	食[た]べさせる 
\\	食[た]べさせて 
\\	食[た]べさせます
\\	行かせる 
\\	行かせて 
\\	行かせます	
\\	行[い]かせる 
\\	行[い]かせて 
\\	行[い]かせます
\\	待たせる 
\\	待たせて 
\\	待たせます	
\\	待[ま]たせる 
\\	待[ま]たせて 
\\	待[ま]たせます	
\\	読ませる 
\\	読ませて 
\\	読ませます	
\\	読[よ]ませる 
\\	読[よ]ませて 
\\	読[よ]ませます
\\	泳がせる 
\\	泳がせて 
\\	泳がせます	
\\	泳[およ]がせる 
\\	泳[およ]がせて 
\\	泳[およ]がせます
\\	買わせる 
\\	買わせて 
\\	買わせます	
\\	買[か]わせる 
\\	買[か]わせて 
\\	買[か]わせます
\\	話させる 
\\	話させて 
\\	話させます	
\\	話[はな]させる 
\\	話[はな]させて 
\\	話[はな]させます
\\	死なせる 
\\	死なせて 
\\	死なせます	
\\	死[し]なせる 
\\	死[し]なせて 
\\	死[し]なせます
\\	取らせる 
\\	取らせて 
\\	取らせます	
\\	取[と]らせる 
\\	取[と]らせて 
\\	取[と]らせます
\\	遊ばせる 
\\	遊ばせて 
\\	遊ばせます	
\\	遊[あそ]ばせる 
\\	遊[あそ]ばせて 
\\	遊[あそ]ばせます
\\	来させる 
\\	来させて 
\\	来させます	
\\	来[こ]させる 
\\	来[こ]させて 
\\	来[こ]させます
\\	させる 
\\	させて 
\\	させます	
\\	させる 
\\	させて 
\\	させます
\\	は/が 
\\	に 
\\	は 
\\	が, 
\\	に, 
\\	は/が 
\\	に 
\\	先生は学生に会話を覚えさせました。	
\\	私[わたし]は 学[がく]生[せい]に 会[かい]話[わ]を 覚[おぼ]えさせました。
\\	お父さんは子供に野菜を食べさせました。	
\\	お 父[とう]さんは 子[こ]供[ども]に 野[や]菜[さい]を 食[た]べさせました。
\\	お母さんは子供に本を読ませました。	
\\	お 母[かあ]さんは 子[こ]供[ども]に 本[ほん]を 読[よ]ませました。
\\	先生は私に英語を話せて呉れませんでした。	
\\	先[せん]生[せい]は 私[わたし]に 英[えい]語[ご]を 話[はな]せて 呉[く]れませんでした。
\\	私は自分の子供に好きな所に行かせて上げるつもりです。	
\\	私[わたし]は 自[じ]分[ぶん]の 子[こ]供[ども]に 好[す]きな 所[ところ]に 行[い]かせて 上[あ]げるつもりです。
\\	私に此の仕事をやらせて下さい。	
\\	私[わたし]に 此[こ]の 仕[し]事[ごと]をやらせて 下[くだ]さい。
\\	私は其の子供を泣かせて仕舞いました。	
\\	"私[わたし]は 其[そ]の 子[こ]供[ども]を 泣[な]かせて 仕[し]舞[ま]いました。 
\\	を 
\\	に 
\\	スタジオジブリは面白い映画を作って皆を笑わせました。	
\\	"スタジオジブリは 面[おも]白[しろ]い 映[えい]画[が]を 作[つく]って 皆[みんな]を 笑[わら]わせました。 
\\	を 
\\	に 
\\	は/が 
\\	に 
\\	は 
\\	に 
\\	は/が 
\\	に 
\\	て上げる 
\\	て呉れる	
\\	は/が 
\\	に 
\\	て 上[あ]げる 
\\	て 呉[く]れる
\\	て下さい	
\\	て 下[くだ]さい
\\	家に毎日電話しなさい。	
\\	家[うち]に 毎[まい]日[にち]電[でん]話[わ]しなさい。 
\\	なさい 
\\	~は 行[い]けません)
\\	先生の話を良く聞きなさい。	
\\	先[せん]生[せい]の 話[はなし]を 良[よ]く 聞[き]きなさい。 
\\	なさい 
\\	~は 行[い]けません)
\\	括弧の中に単語を入れなさい。	
\\	括[かっ]弧[こ]の中[なか]に 単[たん]語[ご]を 入[い]れなさい。 
\\	なさい 
\\	~は 行[い]けません)
\\	食べれば 
\\	食べなければ	
\\	食[た]べれば 
\\	食[た]べなければ
\\	-う 
\\	-えば; 
\\	-い 
\\	-ければ)
\\	行けば 
\\	行かなければ	
\\	行[い]けば 
\\	行[い]かなければ 
\\	-う 
\\	-えば; 
\\	-い 
\\	-ければ)
\\	すれば 
\\	しなければ	
\\	すれば 
\\	しなければ 
\\	-う 
\\	-えば; 
\\	-い 
\\	-ければ)
\\	来れば 
\\	来なければ	
\\	来[く]れば 
\\	来[こ]なければ 
\\	-う 
\\	-えば; 
\\	-い 
\\	-ければ)
\\	待てば 
\\	待たなければ	
\\	待[ま]てば 
\\	待[ま]たなければ 
\\	-う 
\\	-えば; 
\\	-い 
\\	-ければ)
\\	買えば 
\\	買わなければ	
\\	買[か]えば 
\\	買[か]わなければ 
\\	-う 
\\	-えば; 
\\	-い 
\\	-ければ)
\\	車が有れば、色々な所に行けます。	
\\	車[くるま]が 有[あ]れば、 色[いろ]々[いろ]な 所[ところ]に 行[い]けます。 
\\	ば 
\\	鍵を掛けておけば、泥棒に入られません。	
\\	鍵[かぎ]を 掛[か]けておけば、 泥[どろ]棒[ぼう]に 入[はい]られません。 
\\	ば 
\\	大家さんに言わなければ、分かりませんよ。	
\\	大[おお]家[や]さんに 言[い]わなければ、 分[わ]かりませんよ。 
\\	ば 
\\	走れば、電車に間に合います。	
\\	走[はし]れば、 電[でん]車[しゃ]に 間[ま]に 合[あ]います。 
\\	ば 
\\	此の薬を飲めば大丈夫です。	
\\	此[こ]の 薬[くすり]を 飲[の]めば 大[だい]丈[じょう]夫[ぶ]です。 
\\	ば 
\\	先生に聞けば、良いんですよ。	
\\	先[せん]生[せい]に 聞[き]けば、 良[い]いんですよ。 
\\	ば 
\\	此の会社はお金が有るのに、給料は低いです。	
\\	"此[こ]の 会[かい]社[しゃ]はお 金[かね]が 有[あ]るのに、 給[きゅう]料[りょう]は 低[ひく]いです。 
\\	のに 
\\	家に居るのに、電話に出ない。	
\\	"家[いえ]に 居[い]るのに、 電[でん]話[わ]に 出[で]ない。 
\\	のに 
\\	田中さんは親切なのに、山田さんは田中さんが嫌いです。	
\\	田[た]中[なか]さんは 親[しん]切[せつ]なのに、 山[やま]田[だ]さんは 田[た]中[なか]さんが 嫌[きら]いです。 
\\	のに 
\\	な-
\\	です, 
\\	な, 
\\	んです 
\\	大きい問題なのに、彼の人は誰にも相談しません。	
\\	大[おお]きい 問[もん]題[だい]なのに、 彼[あ]の 人[ひと]は 誰[だれ]にも 相[そう]談[だん]しません。 
\\	のに 
\\	な-
\\	です, 
\\	な, 
\\	んです 
\\	日本語が少し難しいけど、此れを読んで下さい。	
\\	日[に]本[ほん]語[ご]が 少[すこ]し 難[むずか]しいけど、 此[こ]れを 読[よ]んで 下[くだ]さい。
\\	のに 
\\	余り美味しそうじゃないけど、此処で食べましょう。	
\\	余[あま]り 美[お]味[い]しそうじゃないけど、 此[こ]処[こ]で 食[た]べましょう。 
\\	のに 
\\	札幌	
\\	札[さっ]幌[ぽろ]	サッポロ
\\	私は鎌倉のような町が好きです。	
\\	"私[わたし]は 鎌[かま]倉[くら]のような 町[まち]が 好[す]きです。 
\\	のような 
\\	のような 
\\	夏目漱石	
\\	夏[なつ]目[め]漱[そう]石[せき]	
\\	私は夏目漱石のような作家に成りたいです。	
\\	"私[わたし]は 夏[なつ]目[め]漱[そう]石[せき]のような 作[さっ]家[か]に 成[な]りたいです。 
\\	のような 
\\	のような 
\\	花子さんは魚のように上手に泳ぎます。	
\\	"花[はな]子[こ]さんは 魚[さかな]のように 上[じょう]手[ず]に 泳[およ]ぎます。 
\\	のように
\\	三島由紀夫	
\\	三[み]島[しま]由[ゆ]紀[き]夫[お]
\\	太郎さんは三島由紀夫のように強いです。	
\\	"太[た]郎[ろう]さんは 三[み]島[しま]由[ゆ]紀[き]夫[お]のように 強[つよ]いです。 
\\	のように
\\	彼の人は壊れたレコードのように同じ事を言います。	
\\	"彼[あ]の 人[ひと]は 壊[こわ]れたレコードのように 同[おな]じ 事[こと]を 言[い]います。 
\\	のように
\\	レコード	
\\	レコード
\\	墓場 
\\	墓地	
\\	墓[はか]場[ば] 
\\	墓[ぼ]地[ち]	
\\	此の町は夜の墓場のように静かです。	
\\	"此[こ]の 町[まち]は 夜[よる]の 墓[はか]場[ば]のように 静[しず]かです。 
\\	のように
\\	メアリーが居なく成ると寂しく成るね。	
\\	メアリーが 居[い]なく 成[な]ると 寂[さび]しく 成[な]るね。
\\	でも、来年大学を卒業したら、又日本に戻って来ますから。	
\\	でも、 来[らい]年[ねん]大[だい]学[がく]を 卒[そつ]業[ぎょう]したら、 又[また]日[に]本[ほん]に 戻[もど]って 来[き]ますから。
\\	お父さん、お母さん、本当にお世話に成りました。	
\\	お 父[とう]さん、お 母[かあ]さん、 本[ほん]当[とう]にお 世[せ]話[わ]に 成[な]りました。
\\	いや、私達もメアリーがいて、迚も楽しかったよ。	
\\	いや、 私[わたし]達[たち]もメアリーがいて、迚[とて]も 楽[たの]しかったよ。
\\	国に帰っても、私達のことを忘れないでね。	
\\	国[くに]に 帰[かえ]っても、 私[わたし]達[たち]のことを 忘[わす]れないでね。
\\	勿論。アメリカにも遊びに来て下さい。	
\\	勿[もち]論[ろん]。 アメリカにも 遊[あそ]びに 来[き]て 下[くだ]さい。
\\	じゃあ、此の夏アメリカに行く事にしようか。	
\\	じゃあ、 此[こ]の 夏[なつ]アメリカに 行[い]く 事[こと]にしようか。
\\	此の一年、色んな事が有ったね。	
\\	此[こ]の 一[いち]年[ねん]、 色[いろ]んな 事[こと]が 有[あ]ったね。
\\	そうそう。デートの時良く待たされた。	
\\	そうそう。デートの 時[とき]良[よ]く 待[ま]たされた。
\\	僕が約束の場所を間違えて、後で、物凄く怒られたり。	
\\	僕[ぼく]が 約[やく]束[そく]の 場[ば]所[しょ]を 間[ま]違[ちが]えて、 後[あと]で、 物[もの]凄[すご]く 怒[おこ]られたり。
\\	たけし君が作った料理を食べさせられて、お腹を壊したり。	
\\	たけし 君[くん]が 作[つく]った 料[りょう]理[り]を 食[た]べさせられて、お 腹[なか]を 壊[こわ]したり。
\\	初めて一緒に踊った時、「盆踊りみたいだ」って笑われた。	
\\	初[はじ]めて 一[いっ]緒[しょ]に 踊[おど]った 時[とき]、
\\	盆[ぼん]踊[おど]りみたいだ」って 笑[わら]われた。
\\	彼の時は足を踏まれて痛かった。	
\\	彼[あ]の 時[とき]は 足[あし]を 踏[ふ]まれて 痛[いた]かった。
\\	考え方が違うから、良く喧嘩もしたね。皆良い思い出だね。	
\\	考[かんが]え 方[かた]が 違[ちが]うから、 良[よ]く 喧[けん]嘩[か]もしたね。 皆[みんな]良[い]い 思[おも]い 出[で]だね。
\\	じゃあ、元気でね。	
\\	じゃあ、 元[げん]気[き]でね。
\\	うん。たけし君も。たけし君に会えて本当に良かった。	
\\	うん。たけし 君[くん]も。たけし 君[くん]に 会[あ]えて 本[ほん]当[とう]に 良[よ]かった。
\\	そんな悲しそうな顔しないで。	
\\	そんな 悲[かな]しそうな 顔[かお]しないで。
\\	分かってる。じゃあ、そろそろ行かなくちゃ。	
\\	分[わ]かってる。じゃあ、そろそろ 行[い]かなくちゃ。
\\	メアリーが卒業して日本に戻って来る迄待っているから。	
\\	メアリーが 卒[そつ]業[ぎょう]して 日[に]本[ほん]に 戻[もど]って 来[く]る 迄[まで]待[ま]っているから。
\\	(下手だから歌いたくなかったのに)歌を歌わされました。	
\\	下[へ]手[た]だから 歌[うた]いたくなかったのに) 歌[うた]を 歌[うた]わされました。
\\	(嫌いだから食べたくないんですが、何時も)肉を食べさせられます。	
\\	嫌[きら]いだから 食[た]べたくないんですが、 何[い]時[つ]も) 肉[にく]を 食[た]べさせられます。 
\\	食べさせられる 
\\	食べさせられて 
\\	食べさせられます	
\\	食[た]べさせられる 
\\	食[た]べさせられて 
\\	食[た]べさせられます
\\	話させられる 
\\	話させられて 
\\	話させられます	
\\	話[はな]させられる 
\\	話[はな]させられて 
\\	話[はな]させられます
\\	書かされる 
\\	書かされて 
\\	書かされます	
\\	書[か]かされる 
\\	書[か]かされて 
\\	書[か]かされます
\\	読まされる 
\\	読まされて 
\\	読まされます	
\\	読[よ]まされる 
\\	読[よ]まされて 
\\	読[よ]まされます
\\	泳がされる 
\\	泳がされて 
\\	泳がされます	
\\	泳[およ]がされる 
\\	泳[およ]がされて 
\\	泳[およ]がされます
\\	買わされる 
\\	買わされて 
\\	買わされます	
\\	買[か]わされる 
\\	買[か]わされて 
\\	買[か]わされます
\\	立たされる 
\\	立たされて 
\\	立たされます	
\\	立[た]たされる 
\\	立[た]たされて 
\\	立[た]たされます
\\	撮らされる 
\\	撮らされて 
\\	撮らされます	
\\	撮[と]らされる 
\\	撮[と]らされて 
\\	撮[と]らされます
\\	呼ばされる 
\\	呼ばされて 
\\	呼ばされます	
\\	呼[よ]ばされる 
\\	呼[よ]ばされて 
\\	呼[よ]ばされます
\\	来させられる 
\\	来させられて 
\\	来させられます	
\\	来[こ]させられる 
\\	来[こ]させられて 
\\	来[こ]させられます
\\	させられる 
\\	させられて 
\\	させられます	
\\	させられる 
\\	させられて 
\\	させられます
\\	は/が 
\\	に 
\\	は 
\\	が, 
\\	に, 
\\	は/が 
\\	に 
\\	私は友達に宿題を手伝わされました。 
\\	友達は私に宿題を手伝わせました。	
\\	私[わたし]は 友[とも]達[だち]に 宿[しゅく]題[だい]を 手[て]伝[つだ]わされました。 
\\	友[とも]達[だち]は 私[わたし]に 宿[しゅく]題[だい]を 手[て]伝[つだ]わせました。 
\\	ゆみはお母さんに勉強させられました。 
\\	ゆみは勉強しました。	
\\	ゆみはお 母[かあ]さんに 勉[べん]強[きょう]させられました。 
\\	ゆみは 勉[べん]強[きょう]しました。 
\\	雨が降っても、ピクニックに行きます。 雨が降ったら、ピクニックに行きません。	
\\	"雨[あめ]が 降[ふ]っても、ピクニックに 行[い]きます。 雨[あめ]が 降[ふ]ったら、ピクニックに 行[い]きません。 
\\	ても 
\\	暑くても、冷房を点けません。 暑かったら、冷房を点けます。	
\\	"暑[あつ]くても、 冷[れい]房[ぼう]を 点[つ]けません。 暑[あつ]かったら、 冷[れい]房[ぼう]を 点[つ]けます。 
\\	ても 
\\	子供でも、分かります。 子供だったら、分かりません。	
\\	"子[こ]供[ども]でも、 分[わ]かります。 子[こ]供[ども]だったら、 分[わ]かりません。 
\\	ても 
\\	買う 
\\	買っても 
\\	買ったら	
\\	買[か]う 
\\	買[か]っても 
\\	買[か]ったら
\\	悲しい 
\\	悲しくても 
\\	悲しかったら	
\\	悲[かな]しい 
\\	悲[かな]しくても 
\\	悲[かな]しかったら
\\	元気な 
\\	元気でも 
\\	元気だったら	
\\	元[げん]気[き]な 
\\	元[げん]気[き]でも 
\\	元[げん]気[き]だったら
\\	学生 
\\	学生でも 
\\	学生だったら	
\\	学[がく]生[せい] 
\\	学[がく]生[せい]でも 
\\	学[がく]生[せい]だったら
\\	買わない 
\\	買わなくても 
\\	買わなかったら	
\\	買[か]わない 
\\	買[か]わなくても 
\\	買[か]わなかったら
\\	悲しくない 
\\	悲しくなくても 
\\	悲しくなかったら	
\\	悲[かな]しくない 
\\	悲[かな]しくなくても 
\\	悲[かな]しくなかったら
\\	元気じゃない 
\\	元気じゃなくても 
\\	元気じゃなかったら	
\\	元[げん]気[き]じゃない 
\\	元[げん]気[き]じゃなくても 
\\	元[げん]気[き]じゃなかったら
\\	学生じゃない 
\\	学生じゃなくても 
\\	学生じゃなかったら	
\\	学[がく]生[せい]じゃない 
\\	学[がく]生[せい]じゃなくても 
\\	学[がく]生[せい]じゃなかったら
\\	私は、雨が降っていても、毎日、授業に行きました。	
\\	私[わたし]は、 雨[あめ]が 降[ふ]っていても、 毎[まい]日[にち]、 授[じゅ]業[ぎょう]に 行[い]きました。 (ても 
\\	日本語の授業が難しくても、取ったでしょう。	
\\	日[に]本[ほん]語[ご]の 授[じゅ]業[ぎょう]が 難[むずか]しくても、 取[と]ったでしょう。 (ても 
\\	車を買う事にしました。	
\\	"車[くるま]を 買[か]う 事[こと]にしました。 
\\	事[こと]にする 
\\	~事にする	
\\	事[こと]にする
\\	余り文句を言わない事にします。	
\\	"余[あま]り 文[もん]句[く]を 言[い]わない 事[こと]にします。 
\\	事[こと]にする 
\\	今年の夏は札幌に行く事にしよう。	
\\	今年[ことし]の 夏[なつ]は 札[さっ]幌[ぽろ]に 行[い]く 事[こと]にしよう。 
\\	事[こと]にしよう 
\\	毎晩十一時に寝る事にしています。	
\\	"毎[まい]晩[ばん]十[じゅう]一[いち]時[じ]に 寝[ね]る 事[こと]にしています。 (事[こと]にしている 
\\	絶対にお酒を飲まない事にしています。	
\\	"絶[ぜっ]対[たい]にお 酒[さけ]を 飲[の]まない 事[こと]にしています。 
\\	事[こと]にしている 
\\	~事にしている	
\\	事[こと]にしている
\\	スタジオジブリ	
\\	スタジオジブリ	
\\	天国	
\\	天[てん]国[ごく]	テꜜンコ°ク
\\	地獄	
\\	地[じ]獄[ごく]	ジコ°クꜜ
\\	晴れる迄、喫茶店でコーヒーを飲みながら、待ちます。	
\\	"晴[は]れる 迄[まで]、 喫[きっ]茶[さ]店[てん]でコーヒーを 飲[の]みながら、 待[ま]ちます。 
\\	迄[まで]
\\	日本語が上手に成る迄、国に帰りません。	
\\	"日[に]本[ほん]語[ご]が 上[じょう]手[ず]に 成[な]る 迄[まで]、 国[くに]に 帰[かえ]りません。 
\\	迄[まで]
\\	赤ちゃんが寝る迄、(私は)歌を歌って上げます。	
\\	"赤[あか]ちゃんが 寝[ね]る 迄[まで]、
\\	私[わたし]は) 歌[うた]を 歌[うた]って 上[あ]げます。 
\\	迄[まで]
\\	が 
\\	は)
\\	日本の生活に慣れる迄、大変でした。	
\\	"日[に]本[ほん]の 生[せい]活[かつ]に 慣[な]れる 迄[まで]、 大[たい]変[へん]でした。
\\	迄[まで]
\\	考え方	
\\	"考[かんが]え 方[かた] 
\\	方[かた] 
\\	漢字を読む 
\\	漢字の読み方	
\\	漢[かん]字[じ]を 読[よ]む 
\\	漢[かん]字[じ]の 読[よ]み 方[かた]
\\	箸を使う 
\\	箸の使い方	
\\	箸[はし]を 使[つか]う 
\\	箸[はし]の 使[つか]い方[かた]
\\	空港に行く 
\\	空港の行き方	
\\	空[くう]港[こう]に 行[い]く 
\\	空[くう]港[こう]の 行[い]き 方[かた]
\\	お風呂に入る 
\\	お風呂の入り方	
\\	お 風[ふ]呂[ろ]に 入[はい]る 
\\	お 風[ふ]呂[ろ]の 入[はい]り 方[かた]
\\	日本語を勉強する 
\\	日本語の勉強のし方	
\\	日[に]本[ほん]語[ご]を 勉[べん]強[きょう]する 
\\	日[に]本[ほん]語[ご]の 勉[べん]強[きょう]のし 方[かた]
\\	旅館を予約する 
\\	旅館の予約のし方	
\\	旅[りょ]館[かん]を 予[よ]約[やく]する 
\\	旅[りょ]館[かん]の 予[よ]約[やく]のし 方[かた]
\\	たけしさんのスパゲティの食べ方は面白いです。	
\\	たけしさんのスパゲティの 食[た]べ 方[かた]は 面[おも]白[しろ]いです。 
\\	方[かた] 
\\	は 
\\	を)
\\	済みません。此の漢字の書き方を教えて頂けませんか。	
\\	済[す]みません。 此[こ]の 漢[かん]字[じ]の 書[か]き 方[かた]を 教[おし]えて 頂[いただ]けませんか。 
\\	方[かた] 
\\	は 
\\	を)
\\	作り過ぎました。	
\\	作[つく]り 過[す]ぎました。
\\	此の部屋は狭過ぎます。	
\\	此[こ]の 部[へ]屋[や]は 狭[せま]過[す]ぎます。
\\	食べ過ぎました。	
\\	食[た]べ 過[す]ぎました。
\\	酒を飲み過ぎました。	
\\	酒[さけ]を 飲[の]み 過[す]ぎました。
\\	テレビを見過ぎました。	
\\	テレビを 見[み]過[す]ぎました。
\\	食料品を買い過ぎました。	
\\	食[しょく]料[りょう]品[ひん]を 買[か]い 過[す]ぎました。
\\	此の服は大き過ぎます。	
\\	此[こ]の 服[ふく]は 大[おお]き 過[す]ぎます。
\\	此の試験は難し過ぎます。	
\\	此[こ]の 試[し]験[けん]は 難[むずか]し 過[す]ぎます。
\\	此のセーターは高過ぎます。	
\\	此[こ]のセーターは 高[たか]過[す]ぎます。
\\	此のお風呂は熱過ぎます。	
\\	此[こ]のお 風[ふ]呂[ろ]は 熱[あつ]過[す]ぎます。
\\	宿題は多過ぎます。	
\\	宿[しゅく]題[だい]は 多[おお]過[す]ぎます。
\\	此の犬は元気過ぎます。	
\\	此[こ]の 犬[いぬ]は 元[げん]気[き]過[す]ぎます。
\\	食べ過ぎたから、お腹が痛いんです。	
\\	食[た]べ 過[す]ぎたから、お 腹[なか]が 痛[いた]いんです。
\\	飲み過ぎたから、寝なくちゃ行けませんでした。	
\\	飲[の]み 過[す]ぎたから、 寝[ね]なくちゃ 行[い]けませんでした。
\\	勉強し過ぎましたから、頭が痛く成ったです。	
\\	勉[べん]強[きょう]し 過[す]ぎましたから、 頭[あたま]が 痛[いた]く 成[な]ったです。
\\	コンピューターは使い過ぎましたから、良く寝なかったんです。	
\\	コンピューターは 使[つか]い 過[す]ぎましたから、 良[よ]く 寝[ね]なかったんです。
\\	本を読み過ぎたから、目が乾いたんです。	
\\	本[ほん]を 読[よ]み 過[す]ぎたから、 目[め]が 乾[かわ]いたんです。
\\	歌を歌い過ぎたから、喉が痛いんです。	
\\	歌[うた]を 歌[うた]い 過[す]ぎたから、 喉[のど]が 痛[いた]いんです。
\\	薬を飲んだ方が良いですよ。	
\\	薬[くすり]を 飲[の]んだ 方[ほう]が 良[い]いですよ。
\\	早く寝た方が良いですよ。	
\\	早[はや]く 寝[ね]た 方[ほう]が 良[い]いですよ。
\\	遊びに行かない方が良いですよ。	
\\	遊[あそ]びに 行[い]かない 方[ほう]が 良[い]いですよ。
\\	仕事を休んだ方が良いですよ。	
\\	仕[し]事[ごと]を 休[やす]んだ 方[ほう]が 良[い]いですよ。
\\	家に帰った方が良いですよ。	
\\	家[いえ]に 帰[かえ]った 方[ほう]が 良[い]いですよ。
\\	日本語が上手に成りたいんです。 
\\	日本人の友達を作った方が良いですよ。 
\\	出来るだけ英語を話さない方が良いですよ。	
\\	日[に]本[ほん]語[ご]が 上[じょう]手[ず]に 成[な]りたいんです。 
\\	日[に]本[ほん]人[じん]の 友[とも]達[だち]を 作[つく]った 方[ほう]が 良[い]いですよ。 
\\	出[で]来[き]るだけ 英[えい]語[ご]を 話[はな]さない 方[ほう]が 良[い]いですよ。
\\	ホームシックなんです。 
\\	貴方のお母さんを忘れた方が良いですよ。 毎週、家族に電話を掛けない方が良いですよ。	
\\	ホームシックなんです。 
\\	貴方[あなた]のお 母[かあ]さんを 忘[わす]れた 方[ほう]が 良[い]いですよ。 毎[まい]週[しゅう]、 家[か]族[ぞく]に 電[でん]話[わ]を 掛[か]けない 方[ほう]が 良[い]いですよ。
\\	痩せたいんです。 
\\	もっと運動した方が良いですよ。 お菓子を食べない方が良いですよ。	
\\	痩[や]せたいんです。 
\\	もっと 運[うん]動[どう]した 方[ほう]が 良[い]いですよ。 お 菓[か]子[し]を 食[た]べない 方[ほう]が 良[い]いですよ。
\\	友達と喧嘩したくないんです。 
\\	もっと友達と話してみた方が良いですよ。 友達を殺さない方が良いですよ。	
\\	友[とも]達[だち]と 喧[けん]嘩[か]したくないんです。 
\\	もっと 友[とも]達[だち]と 話[はな]してみた 方[ほう]が 良[い]いですよ。 友[とも]達[だち]を 殺[ころ]さない 方[ほう]が 良[い]いですよ。
\\	お金が無いんです。 
\\	もっと働いた方が良いですよ。 高い服を買わない方が良いですよ。	
\\	お 金[かね]が 無[な]いんです。 
\\	もっと 働[はたら]いた 方[ほう]が 良[い]いですよ。 高[たか]い 服[ふく]を 買[か]わない 方[ほう]が 良[い]いですよ。
\\	成績が悪いんです。 
\\	学校が辞めた方が良いですよ。 試験の前に酒を飲まない方が良いですよ。	
\\	成[せい]績[せき]が 悪[わる]いんです。 
\\	学[がっ]校[こう]が 辞[や]めた 方[ほう]が 良[い]いですよ。 試[し]験[けん]の 前[まえ]に 酒[さけ]を 飲[の]まない 方[ほう]が 良[い]いですよ。
\\	二日酔いなんです。 
\\	もっと水を飲んだ方が良いですよ。 安い酒を飲まない方が良いですよ。	
\\	二[ふつ]日[か]酔[よ]いなんです。 
\\	もっと 水[みず]を 飲[の]んだ 方[ほう]が 良[い]いですよ。 安[やす]い 酒[さけ]を 飲[の]まない 方[ほう]が 良[い]いですよ。
\\	歯が痛いんです。 
\\	歯医者に行った方が良いですよ。 砂糖を食べない方が良いですよ。	
\\	歯[は]が 痛[いた]いんです。 
\\	歯[は]医[い]者[しゃ]に 行[い]った 方[ほう]が 良[い]いですよ。 砂[さ]糖[とう]を 食[た]べない 方[ほう]が 良[い]いですよ。
\\	赤ちゃんを無くしたんです。 
\\	今直ぐ警察を電話した方が良いですよ。 赤ちゃんを見付ける前に日本に引っ越しない方が良いですよ。	
\\	赤[あか]ちゃんを 無[な]くしたんです。 
\\	今[いま]直[す]ぐ 警[けい]察[さつ]を 電[でん]話[わ]した 方[ほう]が 良[い]いですよ。 赤[あか]ちゃんを 見[み]付[つ]ける 前[まえ]に 日[に]本[ほん]に 引[ひ]っ 越[こ]しない 方[ほう]が 良[い]いですよ。
\\	何時も授業を遅刻するんです。 
\\	もっと早く起きた方が良いですよ。	
\\	何[い]時[つ]も 授[じゅ]業[ぎょう]を 遅[ち]刻[こく]するんです。 
\\	もっと 早[はや]く 起[お]きた 方[ほう]が 良[い]いですよ。
\\	良く運動した方が良いです。	
\\	良[よ]く 運[うん]動[どう]した 方[ほう]が 良[い]いです。
\\	良く甘い物を食べない方が良いです。	
\\	良[よ]く 甘[あま]い 物[もの]を 食[た]べない 方[ほう]が 良[い]いです。
\\	良く野菜を食べた方が良いです。	
\\	良[よ]く 野[や]菜[さい]を 食[た]べた 方[ほう]が 良[い]いです。
\\	一寸太っていますから、朝御飯を食べない方が良いですよ。	
\\	一[ちょっ]寸[と]太[ふと]っていますから、 朝[あさ]御[ご]飯[はん]を 食[た]べない 方[ほう]が 良[い]いですよ。
\\	煙草を吸わない方が良いです。	
\\	煙草[たばこ]を 吸[す]わない 方[ほう]が 良[い]いです。
\\	お酒を飲まない方が良いです。	
\\	お酒[さけ]を 飲[の]まない 方[ほう]が 良[い]いです。
\\	もっと歩いた方が良いです。	
\\	もっと 歩[ある]いた 方[ほう]が 良[い]いです。
\\	安いので、買います。	
\\	から)	安[やす]いので、 買[か]います。
\\	彼の映画は面白くないので、見たくありません。	
\\	から)	彼[あ]の 映[えい]画[が]は 面[おも]白[しろ]くないので、 見[み]たくありません。
\\	今週は忙しかったので、疲れています。	
\\	から)	今[こん]週[しゅう]は 忙[いそが]しかったので、 疲[つか]れています。
\\	病気だったので、授業を休みました。	
\\	から)	病[びょう]気[き]だったので、 授[じゅ]業[ぎょう]を 休[やす]みました。
\\	彼女は何時も親切なので、人気が有ります。	
\\	から)	彼[かの]女[じょ]は 何[い]時[つ]も 親[しん]切[せつ]なので、 人[にん]気[き]が 有[あ]ります。
\\	政治に興味が無いので、新聞を読みません。	
\\	から)	政[せい]治[じ]に 興[きょう]味[み]が 無[な]いので、 新[しん]聞[ぶん]を 読[よ]みません。
\\	明日試験が有るので、勉強します。	
\\	から)	明日[あした]試[し]験[けん]が 有[あ]るので、 勉[べん]強[きょう]します。
\\	喉が渇いたので、ジュースが飲みたいです。	
\\	から)	喉[のど]が 渇[かわ]いたので、ジュースが 飲[の]みたいです。
\\	歩き過ぎたので、足が痛いです。	
\\	から)	歩[ある]き 過[す]ぎたので、 足[あし]が 痛[いた]いです。
\\	風邪を引いたので、授業を休みました。	
\\	から)	風邪[かぜ]を 引[ひ]いたので、 授[じゅ]業[ぎょう]を 休[やす]みました。
\\	お金が無いので、会社に入りたいです。	
\\	から)	お 金[かね]が 無[な]いので、 会[かい]社[しゃ]に 入[はい]りたいです。
\\	お腹が空いているので、何か食べたいです。でも、お金が無いんです。助けて下さい。	
\\	から)	お 腹[なか]が 空[す]いているので、 何[なに]か 食[た]べたいです。でも、お 金[かね]が 無[な]いんです。 助[たす]けて 下[くだ]さい。
\\	ホームシックなので、家族を忘れてみます。	
\\	から)	ホームシックなので、 家[か]族[ぞく]を 忘[わす]れてみます。
\\	用事が全然無いので、今、寝ます。	
\\	から)	用[よう]事[じ]が 全[ぜん]然[ぜん]無[な]いので、 今[いま]、 寝[ね]ます。
\\	此の単語の意味が分からないので、今、自殺します。お先に失礼します。	
\\	から)	此[こ]の 単[たん]語[ご]の 意[い]味[み]が 分[わ]からないので、 今[いま]、 自[じ]殺[さつ]します。お 先[さき]に 失[しつ]礼[れい]します。
\\	自殺	
\\	自[じ]殺[さつ]	ジサツ
\\	七時に起きなくちゃ行けません。	
\\	七[しち]時[じ]に 起[お]きなくちゃ 行[い]けません。
\\	八時に家を出なくちゃ行けません。	
\\	八[はち]時[じ]に 家[うち]を 出[で]なくちゃ 行[い]けません。
\\	太郎さんは九時に授業に出なくちゃ行けません。	
\\	太[た]郎[ろう]さんは 九[く]時[じ]に 授[じゅ]業[ぎょう]に 出[で]なくちゃ 行[い]けません。
\\	太郎さんは一時に森田先生に会わなくちゃ行けません。	
\\	太[た]郎[ろう]さんは 一[いち]時[じ]に 森[もり]田[た]先[せん]生[せい]に 会[あ]わなくちゃ 行[い]けません。
\\	花子さんは二時に西語を教えなくちゃ行けません。	
\\	花[はな]子[こ]さんは 二[に]時[じ]に 西[せい]語[ご]を 教[おし]えなくちゃ 行[い]けません。
\\	花子さんは三時に
\\	に行ってテープを聞かなくちゃ行けません。	
\\	花[はな]子[こ]さんは 三[さん]時[じ]に 
\\	[エル]
\\	[エル]に 行[い]ってテープを 聞[き]かなくちゃ 行[い]けません。
\\	五時に家に帰らなくちゃ行けません。	
\\	五[ご]時[じ]に 家[いえ]に 帰[かえ]らなくちゃ 行[い]けません。
\\	六時にホストファミリーと夕食を食べなくちゃ行けません。	
\\	六[ろく]時[じ]にホストファミリーと 夕[ゆう]食[しょく]を 食[た]べなくちゃ 行[い]けません。
\\	知子さんは八時に宿題をしなくちゃ行けません。	
\\	知[とも]子[こ]さんは 八[はち]時[じ]に 宿[しゅく]題[だい]をしなくちゃ 行[い]けません。
\\	知子さんは九時にお風呂に入らなくちゃ行けません。裸な。	
\\	知[とも]子[こ]さんは 九[く]時[じ]にお 風[ふ]呂[ろ]に 入[はい]らなくちゃ 行[い]けません。 裸[はだか]な。
\\	裸(な)	
\\	裸[はだか](な)	
\\	美智子さんは十時に薬を飲まなくちゃ行けません。	
\\	美[み]智[ち]子[こ]さんは 十[じゅう]時[じ]に 薬[くすり]を 飲[の]まなくちゃ 行[い]けません。
\\	明日の朝、一緒にテニスをしませんか。 
\\	一寸都合が悪いんです。明日は授業に出なくちゃ行けないんです。	
\\	明日[あした]の 朝[あさ]、 一[いっ]緒[しょ]にテニスをしませんか。 
\\	一[ちょっ]寸[と]都[つ]合[ごう]が 悪[わる]いんです。 明日[あした]は 授[じゅ]業[ぎょう]に 出[で]なくちゃ 行[い]けないんです。
\\	日本語の授業で何をしなくちゃ行けませんか。	
\\	日[に]本[ほん]語[ご]の 授[じゅ]業[ぎょう]で 何[なに]をしなくちゃ 行[い]けませんか。
\\	教科書を持って来なくちゃ行けません。	
\\	教[きょう]科[か]書[しょ]を 持[も]って 来[こ]なくちゃ 行[い]けません。
\\	日本で外国人は何をしなくちゃ行けませんか。	
\\	日[に]本[ほん]で 外[がい]国[こく]人[じん]は 何[なに]をしなくちゃ 行[い]けませんか。
\\	日本で外国人は早く標準語を習わなくちゃ行けません。	
\\	日[に]本[ほん]で 外[がい]国[こく]人[じん]は 早[はや]く 標[ひょう]準[じゅん] 語[ご]を 習[なら]わなくちゃ 行[い]けません。
\\	格好よく成りたいんです。何をしなくちゃ行けませんか。	
\\	格[かっ]好[こ]よく 成[な]りたいんです。 何[なに]をしなくちゃ 行[い]けませんか。
\\	明日は晴れでしょう。	
\\	明日[あした]は 晴[は]れでしょう。
\\	東京は明日雪でしょう。	
\\	東[とう]京[きょう]は 明日[あした]雪[ゆき]でしょう。
\\	東京の気温は、二度位でしょう。	
\\	東[とう]京[きょう]の 気[き]温[おん]は、 二[に]度[ど]位[ぐらい]でしょう。
\\	神戸は明日晴れでしょう。	
\\	神[こう]戸[べ]は 明日[あした]晴[は]れでしょう。
\\	神戸は明日暑いでしょう。	
\\	神[こう]戸[べ]は 明日[あした]暑[あつ]いでしょう。
\\	神戸の気温は、三十度位でしょう。	
\\	神[こう]戸[べ]の 気[き]温[おん]は、 三[さん]十[じゅう]度[ど]位[ぐらい]でしょう。
\\	札幌は明日雨でしょう。	
\\	札[さっ]幌[ぽろ]は 明日[あした]雨[あめ]でしょう。
\\	札幌は明日涼しいでしょう。	
\\	札[さっ]幌[ぽろ]は 明日[あした]涼[すず]しいでしょう。
\\	札幌の気温は、十八度位でしょう。	
\\	札[さっ]幌[ぽろ]の 気[き]温[おん]は、 十[じゅう]八[はち]度[ど]位[ぐらい]でしょう。
\\	川崎は明日曇りでしょう。	
\\	川[かわ]崎[さき]は 明日[あした]曇[くも]りでしょう。
\\	川崎は明日暖かいでしょう。	
\\	川[かわ]崎[さき]は 明日[あした]暖[あたた]かいでしょう。
\\	川崎の気温は、二十度位でしょう。	
\\	川[かわ]崎[さき]の 気[き]温[おん]は、 二[に]十[じゅう]度[ど]位[ぐらい]でしょう。
\\	健康	
\\	健[けん]康[こう]	ケンコー
\\	健康(な)	
\\	健[けん]康[こう](な)	
\\	済みません、初めてなんですが。	
\\	済[す]みません、 初[はじ]めてなんですが。
\\	保険証	
\\	保[ほ]険[けん]証[しょう]	
\\	保険証を見せて下さい。	
\\	保[ほ]険[けん]証[しょう]を 見[み]せて 下[くだ]さい。
\\	此の紙に名前と住所を書いて下さい。	
\\	此[こ]の 紙[かみ]に 名[な]前[まえ]と 住[じゅう]所[しょ]を 書[か]いて 下[くだ]さい。
\\	此れは何の薬ですか。	
\\	此[こ]れは 何[なん]の 薬[くすり]ですか。
\\	痛み止め 
\\	鎮痛剤	
\\	痛[いた]み 止[ど]め 
\\	鎮[ちん]痛[つう]剤[ざい]
\\	食後	
\\	食[しょく]後[ご]	ショクコ°
\\	鎮痛剤です。食後に飲んで下さい。	
\\	鎮[ちん]痛[つう]剤[ざい]です。 食[しょく]後[ご]に 飲[の]んで 下[くだ]さい。
\\	下痢	
\\	下[げ]痢[り]
\\	下痢です。	
\\	下[げ]痢[り]です。
\\	便秘	
\\	便[べん]秘[ぴ]	ベンピ 
\\	ベꜜンピ
\\	便秘です。	
\\	便[べん]秘[ぴ]です。
\\	生理	
\\	生[せい]理[り] 
\\	整理)	セꜜイリ
\\	生理です。	
\\	生[せい]理[り]です。
\\	花粉症	
\\	花[か]粉[ふん]症[しょう]	カフンショー 
\\	カフꜜンショー
\\	花粉	
\\	花[か]粉[ふん]	カフン
\\	花粉症です。	
\\	花[か]粉[ふん]症[しょう]です。
\\	アレルギー	
\\	アレルギー	アレꜜルギー 
\\	アレルꜜギー
\\	~にアレルギーが有ります。	
\\	~にアレルギーが 有[あ]ります。
\\	虫歯	
\\	虫[むし]歯[ば]	ムシバ
\\	虫歯が有ります。	
\\	虫[むし]歯[ば]が 有[あ]ります。
\\	くしゃみ	
\\	くしゃみ	クシャꜜミ
\\	くしゃみが出ます。	
\\	くしゃみが 出[で]ます。
\\	鼻水	
\\	鼻[はな]水[みず]	ハナミズ 
\\	ハナミꜜズ
\\	鼻水が出ます。	
\\	鼻[はな]水[みず]が 出[で]ます。
\\	痒い	
\\	痒[かゆ]い	カユꜜイ 
\\	カイꜜイ
\\	痒み	
\\	痒[かゆ]み	カユミꜜ 
\\	カユミ
\\	背中が痒いです。	
\\	背[せ]中[なか]が 痒[かゆ]いです。
\\	発疹	
\\	発[はっ]疹[しん]	ハッシン
\\	発疹が有ります。	
\\	発[はっ]疹[しん]が 有[あ]ります。
\\	目眩	
\\	目[め]眩[まい]	
\\	目眩がします。	
\\	目[め]眩[まい]がします。
\\	吐く 
\\	吐いて 
\\	吐きます	
\\	吐[は]く 
\\	吐[は]いて 
\\	吐[は]きます 
\\	履く)	
\\	吐きました。	
\\	吐[は]きました。	
\\	気分が悪いです。	
\\	気[き]分[ぶん]が 悪[わる]いです。
\\	火傷	
\\	火傷[やけど]	カショー 
\\	ヤケド
\\	火傷をしました。	
\\	火傷[やけど]をしました。
\\	足の骨を折りました。	
\\	足[あし]の 骨[ほね]を 折[お]りました。	
\\	怪我をしました。	
\\	怪[け]我[が]をしました。	
\\	富美子	
\\	富[ふ]美[み]子[こ]	
\\	富美子さんは一キロ泳げます。	
\\	富[ふ]美[み]子[こ]さんは 一[いち]キロ 泳[およ]げます。
\\	富美子さんは日本の歌が歌えます。	
\\	富[ふ]美[み]子[こ]さんは 日[に]本[ほん]の 歌[うた]が 歌[うた]えます。
\\	富美子さんはバイオリンが弾けます。	
\\	富[ふ]美[み]子[こ]さんはバイオリンが 弾[ひ]けます。
\\	富美子さんは空手が出来ます。	
\\	富[ふ]美[み]子[こ]さんは 空[から]手[て]が 出[で]来[き]ます。
\\	富美子さんは寿司が食べられます。	
\\	富[ふ]美[み]子[こ]さんは 寿[す]司[し]が 食[た]べられます。
\\	富美子さんは料理が出来ます。	
\\	富[ふ]美[み]子[こ]さんは 料[りょう]理[り]が 出[で]来[き]ます。
\\	富美子さんは日本語で電話が掛けられます。	
\\	富[ふ]美[み]子[こ]さんは 日[に]本[ほん]語[ご]で 電[でん]話[わ]が 掛[か]けられます。
\\	富美子さんは車が運転出来ます。	
\\	富[ふ]美[み]子[こ]さんは 車[くるま]が 運[うん]転[てん]出[で]来[き]ます。
\\	富美子さんはセーターが編めます。	
\\	富[ふ]美[み]子[こ]さんはセーターが 編[あ]めます。
\\	富美子さんは日本語で手紙が書けます。	
\\	富[ふ]美[み]子[こ]さんは 日[に]本[ほん]語[ご]で 手[て]紙[がみ]が 書[か]けます。
\\	富美子さんは朝早く起きられます。	
\\	富[ふ]美[み]子[こ]さんは 朝[あさ] 早[はや]く 起[お]きられます。
\\	富美子さんは熱いお風呂に入れます。裸な。	
\\	富[ふ]美[み]子[こ]さんは 熱[あつ]いお 風[ふ]呂[ろ]に 入[はい]れます。 裸[はだか]な。
\\	一キロ泳げますか。 
\\	はい、泳げます。 
\\	いいえ、泳げません。	
\\	一[いち]キロ 泳[およ]げますか。 
\\	はい、 泳[およ]げます。 
\\	いいえ、 泳[およ]げません。
\\	外国語が話せますか。	
\\	外[がい]国[こく]語[ご]が 話[はな]せますか。
\\	誰かが殺せますか。	
\\	誰[だれ]かが 殺[ころ]せますか。
\\	自殺が出来ますか。	
\\	自[じ]殺[さつ]が 出[で]来[き]ますか。
\\	着物を買いましたか。 
\\	いいえ。高過ぎて買えませんでした。	
\\	着[き]物[もの]を 買[か]いましたか。 
\\	いいえ。 高[たか]過[す]ぎて 買[か]えませんでした。
\\	神戸の牛肉を食べましたか。 
\\	いいえ。高過ぎて食べられませんでした。	
\\	神[こう]戸[べ]の 牛[ぎゅう]肉[にく]を 食[た]べましたか。 
\\	いいえ。 高[たか]過[す]ぎて 食[た]べられませんでした。
\\	インドのカレーを食べましたか。 
\\	いいえ。辛過ぎて食べられませんでした。	
\\	インドのカレーを 食[た]べましたか。 
\\	いいえ。 辛[から]過[す]ぎて 食[た]べられませんでした。
\\	宿題をしましたか。 
\\	いいえ。難し過ぎて出来ませんでした。	
\\	宿[しゅく]題[だい]をしましたか。 
\\	いいえ。 難[むずか]し 過[す]ぎて 出[で]来[き]ませんでした。
\\	温泉に入りましたか。 
\\	いいえ。熱過ぎて入れませんでした。	
\\	温[おん]泉[せん]に 入[はい]りましたか。 
\\	いいえ。 熱[あつ]過[す]ぎて 入[はい]れませんでした。
\\	昨日出掛けましたか。 
\\	いいえ。忙し過ぎて出掛けられませんでした。	
\\	昨[きの]日[う]出[で]掛[か]けましたか。 
\\	いいえ。 忙[いそが]し 過[す]ぎて 出[で]掛[か]けられませんでした。
\\	漢字を全部覚えましたか。 
\\	いいえ。多過ぎて覚えられませんでした。	
\\	漢[かん]字[じ]を 全[ぜん]部[ぶ]覚[おぼ]えましたか。 
\\	いいえ。 多[おお]過[す]ぎて 覚[おぼ]えられませんでした。
\\	海で泳ぎましたか。 
\\	いいえ。寒過ぎて泳げませんでした。	
\\	海[うみ]で 泳[およ]ぎましたか。 
\\	いいえ。 寒[さむ]過[す]ぎて 泳[およ]げませんでした。
\\	日本に住みたいですか。 
\\	物価が高いし、人が沢山居るし、住みたくありません。	
\\	日[に]本[ほん]に 住[す]みたいですか。 
\\	物[ぶっ]価[か]が 高[たか]いし、 人[ひと]が 沢[たく]山[さん]居[い]るし、 住[す]みたくありません。
\\	今週は忙しいですか。 
\\	試験が有るし、宿題が沢山有るし、忙しいです。	
\\	今[こん]週[しゅう]は 忙[いそが]しいですか。 
\\	試[し]験[けん]が 有[あ]るし、 宿[しゅく]題[だい]が 沢[たく]山[さん]有[あ]るし、 忙[いそが]しいです。
\\	太郎さんは良い人ですか。 
\\	易しいし、親切だし、良い人です。	
\\	太[た]郎[ろう]さんは 良[い]い 人[ひと]ですか。 
\\	易[やさ]しいし、 親[しん]切[せつ]だし、 良[い]い 人[ひと]です。
\\	経済の授業を取りますか。 
\\	先生は厳しいし、長いレポートを書かなくちゃ行けないし、取りません。	
\\	経[けい]済[ざい]の 授[じゅ]業[ぎょう]を 取[と]りますか。 
\\	先[せん]生[せい]は 厳[きび]しいし、 長[なが]いレポートを 書[か]かなくちゃ 行[い]けないし、 取[と]りません。
\\	旅行は楽しかったですか。 
\\	食べ物は美味しくなかったし、言葉が分からなかったし、楽しくありませんでした。	
\\	旅[りょ]行[こう]は 楽[たの]しかったですか。 
\\	食[た]べ 物[もの]は 美[お]味[い]しくなかったし、 言[こと]葉[ば]が 分[わ]からなかったし、 楽[たの]しくありませんでした。
\\	今晩はパーティーに行きますか。 
\\	風邪を引いているし、昨日もパーティーに行ったし、行きません。	
\\	今[こん]晩[ばん]はパーティーに 行[い]きますか。 
\\	風邪[かぜ]を 引[ひ]いているし、 昨[きの]日[う]もパーティーに 行[い]ったし、 行[い]きません。
\\	日本語の新聞が読めますか。 
\\	漢字が読めないし、文法が分からないし、読めません。	
\\	日[に]本[ほん]語[ご]の 新[しん]聞[ぶん]が 読[よ]めますか。 
\\	漢[かん]字[じ]が 読[よ]めないし、 文[ぶん]法[ぽう]が 分[わ]からないし、 読[よ]めません。
\\	一人で旅行が出来ますか。 
\\	日本語が話せるし、もう大人だし、出来ます。	
\\	一[ひと]人[り]で 旅[りょ]行[こう]が 出[で]来[き]ますか。 
\\	日[に]本[ほん]語[ご]が 話[はな]せるし、 もう 大人[おとな]だし、 出[で]来[き]ます。
\\	鈴木さんが好きですか。 
\\	嘘をつくし、約束を守らないし、好きじゃありません。	
\\	鈴[すず]木[き]さんが 好[す]きですか。 
\\	嘘[うそ]をつくし、 約[やく]束[そく]を 守[まも]らないし、 好[す]きじゃありません。
\\	新しいアパートは良いですか。 
\\	会社に近いし、静かだし、良いです。	
\\	新[あたら]しいアパートは 良[い]いですか。 
\\	会[かい]社[しゃ]に 近[ちか]いし、 静[しず]かだし、 良[い]いです。
\\	日本の生活は楽しいですか。 
\\	はい、楽しいです。友達が沢山居るし、皆は親切だし。	
\\	日[に]本[ほん]の 生[せい]活[かつ]は 楽[たの]しいですか。 
\\	はい、 楽[たの]しいです。 友[とも]達[だち]が 沢[たく]山[さん]居[い]るし、 皆[みんな]は 親[しん]切[せつ]だし。
\\	此の頃忙しいですか。 
\\	はい、単語が沢山暗記しなくちゃ行けないし、最近就職したし、忙しいです。	
\\	此[こ]の 頃[ごろ]忙[いそが]しいですか。 
\\	はい、 単[たん]語[ご]が 沢[たく]山[さん]暗[あん]記[き]しなくちゃ 行[い]けないし、 最[さい]近[きん]就[しゅう]職[しょく]したし、 忙[いそが]しいです。
\\	今、幸せですか。 
\\	はい、綺麗な妻が居るし、元気な息子が居るし、幸せです。	
\\	今[いま]、 幸[しあわ]せですか。 
\\	はい、 綺[き]麗[れい]な 妻[つま]が 居[い]るし、 元[げん]気[き]な 息[むす]子[こ]が 居[い]るし、 幸[しあわ]せです。
\\	来学期も日本語の授業を取りますか。 
\\	はい、日本語は物凄く面白いし、日本人の姫と結婚したいし、取ります。	
\\	来[らい] 学[がっ]期[き]も 日[に]本[ほん]語[ご]の 授[じゅ]業[ぎょう]を 取[と]りますか。 
\\	はい、 日[に]本[ほん]語[ご]は 物[もの]凄[すご]く 面[おも]白[しろ]いし、 日[に]本[ほん]人[じん]の 姫[ひめ]と 結[けっ]婚[こん]したいし、 取[と]ります。
\\	姫	
\\	王女)	姫[ひめ]	ヒꜜメ
\\	暗記	
\\	暗[あん]記[き]	アンキ
\\	日本語の授業は大変ですか。 
\\	いいえ、日本語の文法は易しいし、漢字は簡単だし、大変じゃありません。	
\\	日[に]本[ほん]語[ご]の 授[じゅ]業[ぎょう]は 大[たい]変[へん]ですか。 
\\	いいえ、 日[に]本[ほん]語[ご]の 文[ぶん]法[ぽう]は 易[やさ]しいし、 漢[かん]字[じ]は 簡[かん]単[たん]だし、 大[たい]変[へん]じゃありません。
\\	此の寿司は美味しそうですね。	
\\	此[こ]の 寿[す]司[し]は 美[お]味[い]しそうですね。
\\	此のケーキは甘そうですね。	
\\	此[こ]のケーキは 甘[あま]そうですね。
\\	此のカレーは辛そうですね。	
\\	此[こ]のカレーは 辛[から]そうですね。
\\	此の服は古そうですね。	
\\	此[こ]の 服[ふく]は 古[ふる]そうですね。
\\	此の先生は厳しそうですね。	
\\	此[こ]の 先[せん]生[せい]は 厳[きび]しそうですね。
\\	此の時計は新しそうですね。	
\\	此[こ]の 時[と]計[けい]は 新[あたら]しそうですね。
\\	此のやくざは怖そうですね。	
\\	此[こ]のやくざは 怖[こわ]そうですね。
\\	此の男の人は寂しそうですね。	
\\	此[こ]の 男[おとこ]の 人[ひと]は 寂[さび]しそうですね。
\\	此の女の人は嬉しそうですね。	
\\	此[こ]の 女[おんな]の 人[ひと]は 嬉[うれ]しそうですね。
\\	此のお祖父さんは元気そうですね。	
\\	此[こ]のお 祖[じ]父[い]さんは 元[げん]気[き]そうですね。
\\	此のお祖母さんは意地悪そうですね。	
\\	此[こ]のお 祖[ば]母[あ]さんは 意[い]地[じ]悪[わる]そうですね。
\\	貴方の奥さんは親切そうですね。	
\\	貴方[あなた]の 奥[おく]さんは 親[しん]切[せつ]そうですね。
\\	此の弁護士は頭が良さそうですね。	
\\	此[こ]の 弁[べん]護[ご]士[し]は 頭[あたま]が 良[よ]さそうですね。
\\	此の生徒は眠そうですね。	
\\	此[こ]の 生[せい]徒[と]は 眠[ねむ]そうですね。
\\	此の子供は悲しそうですね。	
\\	此[こ]の 子[こ]供[ども]は 悲[かな]しそうですね。
\\	美味しそうな寿司ですね。	
\\	美[お]味[い]しそうな 寿[す]司[し]ですね。
\\	甘そうなケーキですね。	
\\	甘[あま]そうなケーキですね。
\\	辛そうなカレーですね。	
\\	辛[から]そうなカレーですね。
\\	古そうな服ですね。	
\\	古[ふる]そうな 服[ふく]ですね。
\\	厳しそうな先生ですね。	
\\	厳[きび]しそうな 先[せん]生[せい]ですね。
\\	新しそうな時計ですね。	
\\	新[あたら]しそうな 時[と]計[けい]ですね。
\\	怖そうなやくざですね。	
\\	怖[こわ]そうなやくざですね。
\\	寂しそうな男の人ですね。	
\\	寂[さび]しそうな 男[おとこ]の 人[ひと]ですね。
\\	嬉しそうな女の人ですね。	
\\	嬉[うれ]しそうな 女[おんな]の 人[ひと]ですね。
\\	元気そうなお祖父さんですね。	
\\	元[げん]気[き]そうなお 祖[じ]父[い]さんですね。
\\	意地悪そうなお祖母さんですね。	
\\	意[い]地[じ]悪[わる]そうなお 祖[ば]母[あ]さんですね。
\\	親切そうな奥さんですね。	
\\	親[しん]切[せつ]そうな 奥[おく]さんですね。
\\	頭が良さそうな弁護士ですね。	
\\	頭[あたま]が 良[よ]さそうな 弁[べん]護[ご]士[し]ですね。
\\	眠そうな生徒ですね。	
\\	眠[ねむ]そうな 生[せい]徒[と]ですね。
\\	悲しそうな子供ですね。	
\\	悲[かな]しそうな 子[こ]供[ども]ですね。
\\	寒そうな女の人ですね。 此の女の人は寒そうですね。	
\\	寒[さむ]そうな 女[おんな]の 人[ひと]ですね。 此[こ]の 女[おんな]の 人[ひと]は 寒[さむ]そうですね。
\\	面白そうな本ですね。	
\\	面[おも]白[しろ]そうな 本[ほん]ですね。
\\	教科書なので、読まなくちゃ行けないんです。	
\\	教[きょう]科[か]書[しょ]なので、 読[よ]まなくちゃ 行[い]けないんです。
\\	此の服は素敵ですよ。 
\\	じゃあ、着てきます。	
\\	此[こ]の 服[ふく]は 素[す]敵[てき]ですよ。 
\\	じゃあ、 着[き]てみます。
\\	経済の授業は面白かったですよ。 
\\	じゃあ、取ってみます。	
\\	経[けい]済[ざい]の 授[じゅ]業[ぎょう]は 面[おも]白[しろ]かったですよ。 
\\	じゃあ、 取[と]ってみます。
\\	彼の映画を見て泣きました。 
\\	じゃあ、見てみます。	
\\	彼[あ]の 映[えい]画[が]を 見[み]て 泣[な]きました。 
\\	じゃあ、 見[み]てみます。
\\	此の本は感動しました。 
\\	じゃあ、読んでみます。	
\\	此[こ]の 本[ほん]は 感[かん]動[どう]しました。 
\\	じゃあ、 読[よ]んでみます。
\\	此のケーキは美味しかったですよ。 
\\	じゃあ、食べてみます。	
\\	此[こ]のケーキは 美[お]味[い]しかったですよ。 
\\	じゃあ、 食[た]べてみます。
\\	東京は大きかったですよ。 
\\	じゃあ、行ってみます。	
\\	東[とう]京[きょう]は 大[おお]きかったですよ。 
\\	じゃあ、 行[い]ってみます。
\\	此の
\\	は良かったですよ。 
\\	じゃあ、聞いてみます。	
\\	此[こ]の 
\\	[シー]
\\	[ディー]は 良[よ]かったですよ。 
\\	じゃあ、 聞[き]いてみます。
\\	此の電子辞書は便利でしたよ。 
\\	じゃあ、使ってみます。	
\\	此[こ]の 電[でん]子[し]辞[じ]書[しょ]は 便[べん]利[り]でしたよ。 
\\	じゃあ、 使[つか]ってみます。
\\	神戸で何をしてみたいですか。 
\\	神戸で牛肉を食べてみたいです。	
\\	神[こう]戸[べ]で 何[なに]をしてみたいですか。 
\\	神[こう]戸[べ]で 牛[ぎゅう]肉[にく]を 食[た]べてみたいです。
\\	札幌で何をしてみたいですか。 
\\	札幌で札幌雪祭りを見てみたいです。	
\\	札[さっ]幌[ぽろ]で 何[なに]をしてみたいですか。 
\\	札[さっ]幌[ぽろ]で 札[さっ]幌[ぽろ]雪[ゆき]祭[まつ]りを 見[み]てみたいです。
\\	川崎で何をしてみたいですか。 
\\	川崎で野球選手と話してみたいです。	
\\	川[かわ]崎[さき]で 何[なに]をしてみたいですか。 
\\	川[かわ]崎[さき]で 野[や]球[きゅう]選[せん]手[しゅ]と 話[はな]してみたいです。
\\	鎌倉で何をしてみたいですか。 
\\	鎌倉で神社に行ってみたいです。	
\\	鎌[かま]倉[くら]で 何[なに]をしてみたいですか。 
\\	鎌[かま]倉[くら]で 神[じん]社[じゃ]に 行[い]ってみたいです。
\\	花子さんは今朝、コーヒーを飲みましたか。 
\\	紅茶なら飲みましたが、コーヒーは飲みませんでした。	
\\	花[はな]子[こ]さんは 今[け]朝[さ]、コーヒーを 飲[の]みましたか。 
\\	紅[こう]茶[ちゃ]なら 飲[の]みましたが、コーヒーは 飲[の]みませんでした。
\\	花子さんはバイクに乗れますか。 
\\	自転車なら乗れますが、バイクは乗れません。	
\\	花[はな]子[こ]さんはバイクに 乗[の]れますか。 
\\	自[じ]転[てん]車[しゃ]なら 乗[の]れますが、バイクは 乗[の]れません。
\\	花子さんは神戸に行ったことが有りますか。 
\\	札幌なら行ったことが有りますが、神戸は行ったことが有りません。	
\\	花[はな]子[こ]さんは 神[こう]戸[べ]に 行[い]ったことが 有[あ]りますか。 
\\	札[さっ]幌[ぽろ]なら 行[い]ったことが 有[あ]りますが、 神[こう]戸[べ]は 行[い]ったことが 有[あ]りません。
\\	富美子さんは野球をしますか。 
\\	合気道ならしますが、野球はしません。	
\\	富[ふ]美[み]子[こ]さんは 野[や]球[きゅう]をしますか。 
\\	合[あい]気[き]道[どう]ならしますが、 野[や]球[きゅう]はしません。
\\	富美子さんは経済に興味が有りますか。 
\\	日本の歴史なら興味が有りますが、経済は興味が有りません。	
\\	富[ふ]美[み]子[こ]さんは 経[けい]済[ざい]に 興[きょう]味[み]が 有[あ]りますか。 
\\	日[に]本[ほん]の 歴[れき]史[し]なら 興[きょう]味[み]が 有[あ]りますが、 経[けい]済[ざい]は 興[きょう]味[み]が 有[あ]りません。
\\	太郎さんは彼女が居ますか。 
\\	彼なら居ますが、彼女は居ません。同性愛者ですから。	
\\	太[た]郎[ろう]さんは 彼[かの]女[じょ]が 居[い]ますか。 
\\	彼[かれ]なら 居[い]ますが、 彼[かの]女[じょ]は 居[い]ません。 同[どう]性[せい]愛[あい]者[しゃ]ですから。
\\	同性愛者	
\\	同[どう]性[せい]愛[あい]者[しゃ]	
\\	同性愛	
\\	同[どう]性[せい]愛[あい]	ドーセꜜイアイ
\\	スポーツを良く見ますか。 
\\	ええ、野球なら見ます。 
\\	いいえ、見ません。	
\\	スポーツを 良[よ]く 見[み]ますか。 
\\	ええ、 野[や]球[きゅう]なら 見[み]ます。/ いいえ、 見[み]ません。
\\	外国語が出来ますか。 
\\	ええ、西語なら出来ます。 
\\	いいえ、出来ません。	
\\	外[がい]国[こく]語[ご]が 出[で]来[き]ますか。 
\\	ええ、 西[せい]語[ご]なら 出[で]来[き]ます。/ いいえ、 出[で]来[き]ません。
\\	日本料理が作れますか。 
\\	ええ、河豚なら作れます。 
\\	いいえ、作れません。	
\\	日[に]本[ほん]料[りょう]理[り]が 作[つく]れますか。 
\\	ええ、 河[ふ]豚[ぐ]なら 作[つく]れます。/ いいえ、 作[つく]れません。
\\	有名人に会ったことが有りますか。 
\\	ええ、日本の首相なら会ったことが有ります。 
\\	いいえ、有りません。	
\\	有[ゆう]名[めい]人[じん]に 会[あ]ったことが 有[あ]りますか。 
\\	ええ、 日[に]本[ほん]の 首[しゅ]相[しょう]なら 会[あ]ったことが 有[あ]ります。/ いいえ、 有[あ]りません。
\\	楽器が出来ますか。 
\\	ええ、太鼓なら叩けます。 
\\	いいえ、出来ません。	
\\	楽[がっ]器[き]が 出[で]来[き]ますか。 
\\	ええ、 太[たい]鼓[こ]なら 叩[たた]けます。/ いいえ、 出[で]来[き]ません。
\\	現金が貸せますか。 
\\	ええ、百円玉なら貸せます。 
\\	いいえ、貸せません。	
\\	現[げん]金[きん]が 貸[か]せますか 。 
\\	ええ、 百[ひゃく]円[えん]玉[だま]なら 貸[か]せます。/ いいえ、 貸[か]せません。
\\	一か月に一回床屋に行きます。	
\\	一[いっ]か 月[げつ]に 一[いっ]回[かい]床[とこ]屋[や]に 行[い]きます。
\\	一日に二回食べます。	
\\	一[いち]日[にち]に 二[に]回[かい]食[た]べます。
\\	一日に三回歯を磨きます。	
\\	一[いち]日[にち]に 三[さん]回[かい]歯[は]を 磨[みが]きます。
\\	一日に七時間寝ます。	
\\	一[いち]日[にち]に 七[しち]時[じ]間[かん]寝[ね]ます。
\\	一日に三時間勉強します。	
\\	一[いち]日[にち]に 三[さん]時[じ]間[かん]勉[べん]強[きょう]します。
\\	一週間に一回部屋を掃除します。	
\\	一[いっ]週[しゅう]間[かん]に 一[いっ]回[かい]部[へ]屋[や]を 掃[そう]除[じ]します。
\\	一週間に二回洗濯します。	
\\	一[いっ]週[しゅう]間[かん]に 二[に]回[かい]洗[せん]濯[たく]します。
\\	一週間に三日アルバイトをします。	
\\	一[いっ]週[しゅう]間[かん]に 三[みっ]日[か]アルバイトをします。
\\	一週間に五日学校に行きます。	
\\	一[いっ]週[しゅう]間[かん]に 五[いつ]日[か]学[がっ]校[こう]に 行[い]きます。
\\	一か月に一回映画館に映画を見に行きます。	
\\	一[いっ]か 月[げつ]に 一[いっ]回[かい]映[えい]画[が]館[かん]に 映[えい]画[が]を 見[み]に 行[い]きます。
\\	一週間に一回	
\\	一[いっ]週[しゅう]間[かん]に 一[いっ]回[かい]
\\	一週間に二回	
\\	一[いっ]週[しゅう]間[かん]に 二[に]回[かい]
\\	一週間に四回	
\\	一[いっ]週[しゅう]間[かん]に 四[よん]回[かい]
\\	一週間に五回	
\\	一[いっ]週[しゅう]間[かん]に 五[ご]回[かい]
\\	一週間に六回	
\\	一[いっ]週[しゅう]間[かん]に 六[ろっ]回[かい]
\\	一週間に七回	
\\	一[いっ]週[しゅう]間[かん]に 七[なな]回[かい]
\\	一週間に八回	
\\	一[いっ]週[しゅう]間[かん]に 八[はっ]回[かい]
\\	一週間に九回	
\\	一[いっ]週[しゅう]間[かん]に 九[きゅう]回[かい]
\\	一週間に十回	
\\	一[いっ]週[しゅう]間[かん]に 十[じゅっ]回[かい]
\\	一日に一回	
\\	一[いち]日[にち]に 一[いっ]回[かい]
\\	一日に二回	
\\	一[いち]日[にち]に 二[に]回[かい]
\\	一日に三回	
\\	一[いち]日[にち]に 三[さん]回[かい]
\\	一日に四回	
\\	一[いち]日[にち]に 四[よん]回[かい]
\\	一日に五回	
\\	一[いち]日[にち]に 五[ご]回[かい]
\\	一日に六回	
\\	一[いち]日[にち]に 六[ろっ]回[かい]
\\	一日に七回	
\\	一[いち]日[にち]に 七[なな]回[かい]
\\	一日に八回	
\\	一[いち]日[にち]に 八[はっ]回[かい]
\\	一日に九回	
\\	一[いち]日[にち]に 九[きゅう]回[かい]
\\	一日に十回	
\\	一[いち]日[にち]に 十[じゅっ]回[かい]
\\	1年間に1日	
\\	1[いち]年[ねん]間[かん]に 1[いち] 日[にち]
\\	1年間に2日	
\\	1[いち]年[ねん]間[かん]に 2[ふつ]日[か]
\\	1年間に3日	
\\	1[いち]年[ねん]間[かん]に 3[みっ]日[か]
\\	1年間に4日	
\\	1[いち]年[ねん]間[かん]に 4[よっ]日[か]
\\	1年間に5日	
\\	1[いち]年[ねん]間[かん]に 5[いつ]日[か]
\\	1年間に6日	
\\	1[いち]年[ねん]間[かん]に 6[むい]日[か]
\\	1年間に7日	
\\	1[いち]年[ねん]間[かん]に 7[なの]日[か]
\\	1年間に8日	
\\	1[いち]年[ねん]間[かん]に 8[よう]日[か]
\\	1年間に9日	
\\	1[いち]年[ねん]間[かん]に 9[ここの]日[か]
\\	1年間に10日	
\\	1[いち]年[ねん]間[かん]に 10[とお]日[か]
\\	佐藤	
\\	佐[さ]藤[とう]	
\\	佐藤さんは一日に何回食べますか。 
\\	そうですね。大抵一日に二回食べます。	
\\	佐[さ]藤[とう]さんは 一[いち]日[にち]に 何[なん]回[かい]食[た]べますか。 
\\	そうですね。 大[たい]抵[てい]一[いち]日[にち]に 二[に]回[かい]食[た]べます。
\\	何回	
\\	何[なん]回[かい]	ナꜜンカイ
\\	一日に何回食べますか。	
\\	一[いち]日[にち]に 何[なん]回[かい]食[た]べますか。
\\	一日に何時間勉強しますか。	
\\	一[いち]日[にち]に 何[なん]時[じ]間[かん]勉[べん]強[きょう]しますか。
\\	一週間に何回酒を飲みますか。	
\\	一[いっ]週[しゅう]間[かん]に 何[なん]回[かい]酒[さけ]を 飲[の]みますか。
\\	一週間に何時間運動しますか。	
\\	一[いっ]週[しゅう]間[かん]に 何[なん]時[じ]間[かん]運[うん]動[どう]しますか。
\\	一週間に何日徹夜しますか。	
\\	一[いっ]週[しゅう]間[かん]に 何[なん]日[にち] 徹[てつ]夜[や]しますか。
\\	一か月に何回図書館で本を借りますか。	
\\	一[いっ]か 月[げつ]に 何[なん]回[かい]図[と]書[しょ]館[かん]で 本[ほん]を 借[か]りますか。
\\	一か月に何時間電車に乗りますか。	
\\	一[いっ]か 月[げつ]に 何[なん]時[じ]間[かん]電[でん]車[しゃ]に 乗[の]りますか。
\\	一か月に何日休みが有りますか。	
\\	一[いっ]か 月[げつ]に 何[なん]日[にち]休[やす]みが 有[あ]りますか。
\\	一か月に四回位美容院に行きます。	
\\	一[いっ]か 月[げつ]に 四[よん]回[かい]位[ぐらい]美[び]容[よう]院[いん]に 行[い]きます。
\\	10年間に1回位演奏会に行きます。	
\\	10[じゅう]年[ねん]間[かん]に 1[いっ]回[かい]位[ぐらい]演[えん]奏[そう]会[かい]に 行[い]きます。
\\	子供の時に何が出来ましたか。何が出来ませんでしたか。	
\\	子[こ]供[ども]の 時[とき]に 何[なに]が 出[で]来[き]ましたか。 何[なに]が 出[で]来[き]ませんでしたか。
\\	百円で何が買えますか。	
\\	百[ひゃく]円[えん]で 何[なに]が 買[か]えますか。
\\	何処に行ってみたいですか。如何してですか。	
\\	何[ど]処[こ]に 行[い]ってみたいですか。 如[ど]何[う]してですか。
\\	子供の時、何がしてみたかったですか。	
\\	子[こ]供[ども]の 時[とき]、 何[なに]がしてみたかったですか。
\\	今、何がしてみたいですか。	
\\	今[いま]、 何[なに]がしてみたいですか。
\\	一日に何時間位起きていますか。	
\\	一[いち]日[にち]に 何[なん]時[じ]間[かん]位[ぐらい]起[お]きていますか。
\\	一週間に何回喫茶店へ行きますか。	
\\	一[いっ]週[しゅう]間[かん]に 何[なん]回[かい]喫[きっ]茶[さ]店[てん]へ 行[い]きますか。
\\	一か月に幾ら使いますか。	
\\	一[いっ]か 月[げつ]に 幾[いく]ら 使[つか]いますか。
\\	一時間に幾ら貰いますか。	
\\	一[いち]時[じ]間[かん]に 幾[いく]ら 貰[もら]いますか。
\\	一週間に何日働きますか。	
\\	一[いっ]週[しゅう]間[かん]に 何[なん]日[にち]働[はたら]きますか。
\\	日本語学校	
\\	日[に]本[ほん]語[ご]学[がっ]校[こう]
\\	外国語学校	
\\	外[がい]国[こく]語[ご]学[がっ]校[こう]	
\\	もしもし、山田さんのお宅ですか。	
\\	もしもし、 山[やま]田[だ]さんのお 宅[たく]ですか。
\\	又電話します。	
\\	又[また]電[でん]話[わ]します。
\\	佐藤から電話が有ったと伝えて下さい。	
\\	佐[さ]藤[とう]から 電[でん]話[わ]が 有[あ]ったと 伝[つた]えて 下[くだ]さい。
\\	ミーティング	
\\	ミーティング
\\	ミーティングに行けないと伝えて下さい。	
\\	ミーティングに 行[い]けないと 伝[つた]えて 下[くだ]さい。
\\	小林	
\\	小[こ]林[ばやし]	
\\	正確な	
\\	正[せい]確[かく]な	
\\	正直な	
\\	正[しょう]直[じき]な	
\\	道元禅師	
\\	道[どう]元[げん]禅[ぜん]師[じ]	
\\	金閣寺	
\\	金[きん]閣[かく]寺[じ]	
\\	硬い言い方	
\\	硬[かた]い 言[い]い 方[かた]	
\\	木漏れ日	
\\	木[こ]漏[も]れ 日[び]	コモレꜜビ 
\\	コモレビ
\\	新選組	
\\	新[しん]選[せん]組[ぐみ]	
\\	海苔	
\\	海苔[のり]	ノリꜜ
\end{CJK}
\end{document}