\documentclass[8pt]{extreport} 
\usepackage{hyperref}
\usepackage{CJKutf8}
\begin{document}
\begin{CJK}{UTF8}{min}
\\	私	
\\	私[わたし]		
\\	あなた	
\\	あなた		
\\	あの人(あの方)	
\\	あの 人[ひと](あの 方[かた])		
\\	~さん	
\\	~さん		
\\	~ちゃん	
\\	~ちゃん		
\\	~人	
\\	人[じん]		
\\	先生	
\\	先[せん] 生[せい]		
\\	教師	
\\	教[きょう] 師[し]		
\\	学生	
\\	学[がく] 生[せい]		
\\	会社員	
\\	会[かい] 社[しゃ] 員[いん]		
\\	社員	
\\	社[しゃ] 員[いん]		
\\	銀行員	
\\	銀[ぎん] 行[こう] 員[いん]		
\\	医者	
\\	医[い] 者[しゃ]		
\\	研究者	
\\	研[けん] 究[きゅう] 者[しゃ]		
\\	大学	
\\	大[だい] 学[がく]		
\\	病院	
\\	病[びょう] 院[いん]		
\\	誰	
\\	(どなた 
\\	誰[だれ]		
\\	ー歳	ー 
\\	ー 歳[さい]		
\\	何歳	
\\	(おいくつ 
\\	何[なん] 歳[さい]		
\\	はい	
\\	はい		
\\	いいえ	
\\	いいえ		
\\	始めまして	
\\	始[はじ]めまして		
\\	~からきました。	
\\	~からきました。		
\\	どうぞよろしくお願いします。	
\\	どうぞよろしくお 願[ねが]いします。		
\\	失礼ですが	
\\	失[しつ] 礼[れい]ですが		
\\	お名前は	
\\	お 名[な] 前[まえ]は		
\\	こちらは~さんです。	
\\	こちらは~さんです。		
\\	アメリカ	
\\	アメリカ		
\\	イギリス	
\\	イギリス		
\\	インド	
\\	インド		
\\	インドネシア	
\\	インドネシア		
\\	韓国	
\\	韓[かん] 国[こく]		
\\	タイ	
\\	タイ		
\\	中国	
\\	中[ちゅう] 国[ごく]		
\\	ドイツ	
\\	ドイツ		
\\	日本	
\\	日本[にほん]		
\\	ブラジル	
\\	ブラジル		
\\	イタリア	
\\	イタリア		
\\	イラン	
\\	イラン		
\\	エジプト	
\\	エジプト		
\\	オーストラリア	
\\	オーストラリア		
\\	カナダ	
\\	カナダ		
\\	サウジアラビア	
\\	サウジアラビア		
\\	シンガポール	
\\	シンガポール		
\\	スペイン	
\\	スペイン		
\\	フランス	
\\	フランス		
\\	フィリピン	
\\	フィリピン		
\\	ベトナム	
\\	ベトナム		
\\	マレーシア	
\\	マレーシア		
\\	メキシコ	
\\	メキシコ		
\\	ロシア	
\\	ロシア		
\\	ペルシア語	
\\	ペルシア 語[ご]		
\\	ポルトガル語	
\\	ポルトガル 語[ご]		
\\	これ	
\\	これ		
\\	それ	
\\	それ		
\\	あれ	
\\	あれ		
\\	この~	
\\	この~		
\\	その~	
\\	その~		
\\	あの~	
\\	あの~		
\\	本	
\\	本[ほん]		
\\	辞書	
\\	辞[じ] 書[しょ]		
\\	雑誌	
\\	雑[ざっ] 誌[し]		
\\	新聞	
\\	新[しん] 聞[ぶん]		
\\	手帳	
\\	手[て] 帳[ちょう]		
\\	名刺	
\\	名[めい] 刺[し]		
\\	カード	
\\	カード		
\\	鉛筆	
\\	鉛[えん] 筆[ぴつ]		
\\	ボールペン	
\\	ボールペン		
\\	シャープペンシル	
\\	シャープペンシル		
\\	鍵	
\\	鍵[かぎ]		
\\	時計	
\\	時[と] 計[けい]		
\\	傘	
\\	傘[かさ]		
\\	鞄	
\\	鞄[かばん]		
\\	テレビ	
\\	テレビ		
\\	ラジオ	
\\	ラジオ		
\\	カメラ	
\\	カメラ		
\\	コンピューター	
\\	コンピューター		
\\	車	
\\	車[くるま]		
\\	机	
\\	机[つくえ]		
\\	椅子	
\\	椅子[いす]		
\\	チョコレート	
\\	チョコレート		
\\	コーヒー	
\\	コーヒー		
\\	お土産	
\\	お 土産[みやげ]		
\\	英語	
\\	英[えい] 語[ご]		
\\	日本語	
\\	日本[にほん] 語[ご]		
\\	何	
\\	何[なん]		
\\	そう	
\\	そう		
\\	あのう	
\\	あのう		
\\	えっ	
\\	えっ		
\\	どうぞ。	
\\	どうぞ。		
\\	どうもありがとうございます。	
\\	どうもありがとうございます。		
\\	そうですか。	
\\	そうですか。		
\\	違います。	
\\	違[ちが]います。		
\\	あっ	
\\	あっ		
\\	これからお世話になります。	
\\	これからお 世[せ] 話[わ]になります。		
\\	こちらこそどうぞよろしくお願いします。	
\\	こちらこそどうぞよろしくお 願[ねが]いします。		
\\	佐藤	
\\	佐[さ] 藤[とう]		
\\	鈴木	
\\	鈴[すず] 木[き]		
\\	高橋	
\\	高[たか] 橋[はし]		
\\	田中	
\\	田[た] 中[なか]		
\\	渡辺	
\\	渡[わた] 辺[なべ]		
\\	伊藤	
\\	伊[い] 藤[とう]		
\\	山本	
\\	山[やま] 本[もと]		
\\	中村	
\\	中[なか] 村[むら]		
\\	小林	
\\	小[こ] 林[ばやし]		
\\	加藤	
\\	加[か] 藤[とう]		
\\	吉田	
\\	吉[よし] 田[だ]		
\\	山田	
\\	山[やま] 田[だ]		
\\	佐々木	
\\	佐[さ] 々[さ] 木[き]		
\\	斎藤	
\\	斎[さい] 藤[とう]		
\\	山口	
\\	山[やま] 口[ぐち]		
\\	松本	
\\	松[まつ] 本[もと]		
\\	井上	
\\	井[いの] 上[うえ]		
\\	木村	
\\	木[き] 村[むら]		
\\	清水	
\\	清[し] 水[みず]		
\\	ここ	
\\	ここ		
\\	そこ	
\\	そこ		
\\	あそこ	
\\	あそこ		
\\	どこ	
\\	どこ		
\\	こちら	
\\	こちら		
\\	そちら	
\\	そちら		
\\	あちら	
\\	あちら		
\\	どちら	
\\	どちら		
\\	教室	
\\	教[きょう] 室[しつ]		
\\	食堂	
\\	食[しょく] 堂[どう]		
\\	事務所	
\\	事[じ] 務[む] 所[しょ]		
\\	会議室	
\\	会[かい] 議[ぎ] 室[しつ]		
\\	受付	
\\	受[うけ] 付[つけ]		
\\	ロビー	
\\	ロビー		
\\	部屋	
\\	部[へ] 屋[や]		
\\	トイレ(お手洗い)	
\\	トイレ(お 手[て] 洗[あら]い)		
\\	階段	
\\	階[かい] 段[だん]		
\\	エレベーター	
\\	エレベーター		
\\	エスカレーター	
\\	エスカレーター		
\\	自動販売機	
\\	自[じ] 動[どう] 販[はん] 売[ばい] 機[き]		
\\	電話	
\\	電[でん] 話[わ]		
\\	お国	
\\	お 国[くに]		
\\	会社	
\\	会[かい] 社[しゃ]		
\\	家、家	
\\	家[いえ]、家 [うち]		
\\	靴	
\\	靴[くつ]		
\\	ネクタイ	
\\	ネクタイ		
\\	ワイン	
\\	ワイン		
\\	売り場	
\\	売[う]り 場[ば]		
\\	地下	
\\	地[ち] 下[か]		
\\	ー階、階	
\\	ー 階[かい]、 階[がい]		
\\	何階	
\\	何[なん] 階[がい]		
\\	ー円	
\\	ー 円[えん]		
\\	幾ら	
\\	幾[いく]ら		
\\	百	
\\	百[ひゃく]		
\\	千	
\\	千[せん]		
\\	万	
\\	万[まん]		
\\	すみません。	
\\	すみません。		
\\	どうも。	
\\	どうも。		
\\	いらっしゃいませ。	
\\	いらっしゃいませ。		
\\	[~を]見せてください。	
\\	[~を] 見[み]せてください。		
\\	じゃ	
\\	じゃ		
\\	[~を]ください。	
\\	[~を]ください。		
\\	スイス	
\\	スイス		
\\	ジャカルタ	
\\	ジャカルタ		
\\	バンコク	
\\	バンコク		
\\	ベルリン	
\\	ベルリン		
\\	新大阪	
\\	新[しん] 大[おお] 阪[さか]		
\\	駐車場	
\\	駐[ちゅう] 車[しゃ] 場[じょう]		
\\	食品	
\\	食[しょく] 品[ひん]		
\\	アクセサリー	
\\	アクセサリー		
\\	化粧品	
\\	化[け] 粧[しょう] 品[ひん]		
\\	婦人服	
\\	婦[ふ] 人[じん] 服[ふく]		
\\	紳士服	
\\	紳[しん] 士[し] 服[ふく]		
\\	家具	
\\	家[か] 具[ぐ]		
\\	食器	
\\	食[しょっ] 器[き]		
\\	電化製品	
\\	電[でん] 化[か] 製[せい] 品[ひん]		
\\	子供服	
\\	子[こ] 供[ども] 服[ふく]		
\\	玩具	
\\	玩具[おもちゃ]		
\\	文房具	
\\	文[ぶん] 房[ぼう] 具[ぐ]		
\\	スポーツ用品	
\\	スポーツ 用[よう] 品[ひん]		
\\	旅行用品	
\\	旅[りょ] 行[こう] 用[よう] 品[ひん]		
\\	眼鏡	
\\	眼[め] 鏡[がね]		
\\	レストラン	
\\	レストラン		
\\	催し物会場	
\\	催[もよお]し 物[もの] 会[かい] 場[じょう]		
\\	屋上	
\\	屋[おく] 上[じょう]		
\\	遊園地	
\\	遊[ゆう] 園[えん] 地[ち]		
\\	起きます	
\\	起[お]きます		
\\	寝ます	
\\	寝[ね]ます		
\\	働きます	
\\	働[はたら]きます		
\\	休みます	
\\	休[やす]みます		
\\	勉強します	
\\	勉[べん] 強[きょう]します		
\\	終わります	
\\	終[お]わります		
\\	デパート	
\\	デパート		
\\	銀行	
\\	銀[ぎん] 行[こう]		
\\	郵便局	
\\	郵[ゆう] 便[びん] 局[きょく]		
\\	図書館	
\\	図[と] 書[しょ] 館[かん]		
\\	美術館	
\\	美[び] 術[じゅつ] 館[かん]		
\\	今	
\\	今[いま]		
\\	ー時	ー 
\\	ー 時[じ]		
\\	ー分	ー 
\\	ー 分[ぶん](ぷん)		
\\	半	
\\	半[はん]		
\\	何時	
\\	何[なん] 時[じ]		
\\	何分	
\\	何[なん] 分[ぷん]		
\\	午前	
\\	午[ご] 前[ぜん]		
\\	午後	
\\	午[ご] 後[ご]		
\\	朝	
\\	朝[あさ]		
\\	昼	
\\	昼[ひる]		
\\	晩(夜)	
\\	晩[ばん]
\\	夜[よる])		
\\	一昨日	
\\	一昨日[おととい]		
\\	昨日	
\\	昨日[きのう]		
\\	今日	
\\	今日[きょう]		
\\	明日	
\\	明日[あした]		
\\	明後日	
\\	明後日[あさって]		
\\	今朝	
\\	今朝[けさ]		
\\	今晩	
\\	今[こん] 晩[ばん]		
\\	休み	
\\	休[やす]み		
\\	昼休み	
\\	昼[ひる] 休[やす]み		
\\	試験	
\\	試[し] 験[けん]		
\\	会議をします	
\\	会[かい] 議[ぎ]をします		
\\	映画	
\\	映[えい] 画[が]		
\\	毎朝	
\\	毎[まい] 朝[あさ]		
\\	毎晩	
\\	毎[まい] 晩[ばん]		
\\	毎日	
\\	毎[まい] 日[にち]		
\\	月曜日	
\\	月[げつ] 曜[よ] 日[うび]		
\\	火曜日	
\\	火[か] 曜[よう] 日[び]		
\\	水曜日	
\\	水[すい] 曜[よう] 日[び]		
\\	木曜日	
\\	木[もく] 曜[よう] 日[び]		
\\	土曜日	
\\	土[ど] 曜[よう] 日[び]		
\\	日曜日	
\\	日[にち] 曜[よう] 日[び]		
\\	何曜日	
\\	何[なん] 曜[よう] 日[び]		
\\	~から	
\\	~から		
\\	~まで	
\\	~まで		
\\	~と~	
\\	~と~		
\\	大変ですね。	
\\	大[たい] 変[へん]ですね。		
\\	番号	
\\	番[ばん] 号[ごう]		
\\	何番	
\\	何[なん] 番[ばん]		
\\	ニューヨーク	
\\	ニューヨーク		
\\	ペキン	
\\	(北京)	ペキン		
\\	ロサンゼルス	
\\	ロサンゼルス		
\\	ロンドン	
\\	ロンドン		
\\	警察署	
\\	警[けい] 察[さつ] 署[しょ]		
\\	消防署	
\\	消[しょう] 防[ぼう] 署[しょ]		
\\	時報	
\\	時[じ] 報[ほう]		
\\	天気予報	
\\	天[てん] 気[き] 予[よ] 報[ほう]		
\\	電話番号案内	
\\	電[でん] 話[わ] 番[ばん] 号[ごう] 案[あん] 内[ない]		
\\	県	
\\	県[けん]		
\\	市、市	
\\	市[し]、 市[いち]		
\\	中央	
\\	中[ちゅう] 央[おう]		
\\	行きます	
\\	行[い]きます		
\\	来ます	
\\	来[き]ます		
\\	帰ります	
\\	帰[かえ]ります		
\\	学校	
\\	学[がっ] 校[こう]		
\\	スーパー	
\\	スーパー		
\\	駅	
\\	駅[えき]		
\\	飛行機	
\\	飛[ひ] 行[こう] 機[き]		
\\	船	
\\	船[ふね]		
\\	電車	
\\	電[でん] 車[しゃ]		
\\	地下鉄	
\\	地[ち] 下[か] 鉄[てつ]		
\\	新幹線	
\\	新[しん] 幹[かん] 線[せん]		
\\	バス	
\\	バス		
\\	人	
\\	人[ひと]		
\\	友達	
\\	友[とも] 達[だち]		
\\	彼	
\\	彼[かれ]		
\\	彼女	
\\	彼[かの] 女[じょ]		
\\	家族	
\\	家[か] 族[ぞく]		
\\	一人で	
\\	一人[ひとり]で		
\\	先週	
\\	先[せん] 週[しゅう]		
\\	今週	
\\	今[こん] 週[しゅう]		
\\	来週	
\\	来[らい] 週[しゅう]		
\\	先月	
\\	先[せん] 月[げつ]		
\\	今月	
\\	今[こん] 月[げつ]		
\\	来月	
\\	来[らい] 月[げつ]		
\\	去年	
\\	去[きょ] 年[ねん]		
\\	今年	
\\	今年[ことし]		
\\	来年	
\\	来[らい] 年[ねん]		
\\	ー年	
\\	ー 年[ねん]		
\\	何年	
\\	何[なん] 年[ねん]		
\\	ー月	
\\	ー 月[がつ]		
\\	何月	
\\	何[なん] 月[がつ]		
\\	一日	
\\	一日[ついたち]		
\\	二日	
\\	二[ふつ] 日[か]		
\\	三日	
\\	三[みっ] 日[か]		
\\	四日	
\\	四[よっ] 日[か]		
\\	五日	
\\	五[いつ] 日[か]		
\\	六日	
\\	六[むい] 日[か]		
\\	七日	
\\	七[なの] 日[か]		
\\	八日	
\\	八[よう] 日[か]		
\\	九日	
\\	九[ここの] 日[か]		
\\	十日	
\\	十[とお] 日[か]		
\\	十四日	
\\	十[じゅう] 四[よっ] 日[か]		
\\	二十日	
\\	二十日[はつか]		
\\	二十四日	
\\	二[に] 十[じゅう] 四[よっ] 日[か]		
\\	ー日	
\\	ー 日[にち]		
\\	いつ	
\\	いつ		
\\	誕生日	
\\	誕[たん] 生[じょう] 日[び]		
\\	そうですね。	
\\	そうですね。		
\\	どうもありがとうございました。	
\\	どうもありがとうございました。		
\\	どういたしまして。	
\\	どういたしまして。		
\\	ー番線	
\\	ー 番[ばん] 線[せん]		
\\	次の	
\\	次[つぎ]の		
\\	普通	
\\	普[ふ] 通[つう]		
\\	急行	
\\	急[きゅう] 行[こう]		
\\	特急	
\\	特[とっ] 急[きゅう]		
\\	甲子園	
\\	甲[こう] 子[し] 園[えん]		
\\	大阪城	
\\	大[おお] 阪[さか] 城[じょう]		
\\	元日	
\\	(1月1日)	元[がん] 日[じつ]		
\\	成人の日	
\\	(1月第2月曜日)	成[せい] 人[じん]の 日[ひ]		
\\	春分の日	
\\	(3月20日/21日)	春[しゅん] 分[ぶん]の 日[ひ]		
\\	昭和の日	
\\	(4月29日)	昭[しょう] 和[わ]の 日[ひ]		
\\	憲法記念日	
\\	(5月3日)	憲[けん] 法[ぽう] 記[き] 念[ねん] 日[ひ]		
\\	みどりの日	
\\	(5月4日)	みどりの 日[ひ]		
\\	子供の日	
\\	(5月5日)	子[こ] 供[ども]の 日[ひ]		
\\	海の日	
\\	(7月第3月曜日)	海[うみ]の 日[ひ]		
\\	山の日	
\\	(8月11日)	山[やま]の 日[ひ]		
\\	敬老の日	
\\	(9月第3月曜日)	敬[けい] 老[ろう]の 日[ひ]		
\\	秋分の日	
\\	(9月22日/23日)	秋[しゅう] 分[ぶん]の 日[ひ]		
\\	体育の日	
\\	(10月第2月曜日)	体[たい] 育[いく]の 日[ひ]		
\\	文化の日	
\\	(11月3日)	文[ぶん] 化[か]の 日[ひ]		
\\	勤労感謝の日	
\\	(11月23日)	勤[きん] 労[ろう] 感[かん] 謝[しゃ]の 日[ひ]		
\\	天皇誕生日	
\\	(12月23日)	天[てん] 皇[のう] 誕[たん] 生[じょう] 日[び]		
\\	ゴールデンウェーク	
\\	(4月29日~5月5日)	ゴールデンウェーク		
\\	食べます	
\\	食[た]べます		
\\	飲みます	
\\	飲[の]みます		
\\	吸います	
\\	吸[す]います		
\\	見ます	
\\	見[み]ます		
\\	聞きます	
\\	聞[き]きます		
\\	読みます	
\\	読[よ]みます		
\\	書きます	
\\	書[か]きます		
\\	買います	
\\	買[か]います		
\\	写真を撮ります	
\\	写[しゃ] 真[しん]を 撮[と]ります		
\\	します	
\\	します		
\\	友達に会います	
\\	友[とも] 達[だち]に 会[あ]います		
\\	ごはん	
\\	ごはん		
\\	朝ごはん	
\\	朝[あさ]ごはん		
\\	昼ごはん	
\\	昼[ひる]ごはん		
\\	晩ごはん	
\\	晩[ばん]ごはん		
\\	パン	
\\	パン		
\\	卵	
\\	卵[たまご]		
\\	肉	
\\	肉[にく]		
\\	魚	
\\	魚[さかな]		
\\	野菜	
\\	野[や] 菜[さい]		
\\	果物	
\\	果[くだ] 物[もの]		
\\	水	
\\	水[みず]		
\\	お茶	
\\	お 茶[ちゃ]		
\\	紅茶	
\\	紅[こう] 茶[ちゃ]		
\\	牛乳	
\\	牛[ぎゅう] 乳[にゅう](ミリク)		
\\	ジュース	
\\	ジュース		
\\	ビール	
\\	ビール		
\\	お酒	
\\	お 酒[さけ]		
\\	たばこ	
\\	たばこ		
\\	手紙	
\\	手[て] 紙[がみ]		
\\	レポート	
\\	レポート		
\\	写真	
\\	写[しゃ] 真[しん]		
\\	ビデオ	
\\	ビデオ		
\\	店	
\\	店[みせ]		
\\	庭	
\\	庭[にわ]		
\\	宿題をします	
\\	宿[しゅく] 題[だい]をします		
\\	テニスをします	
\\	テニスをします		
\\	サッカーをします	
\\	サッカーをします		
\\	[お]花見をします	
\\	[お] 花[はな] 見[み]をします		
\\	一緒に	
\\	一[いっ] 緒[しょ]に		
\\	一寸	
\\	一寸[ちょっと]		
\\	いつも	
\\	いつも		
\\	時々	
\\	時[とき] 々[どき]		
\\	それから	
\\	それから		
\\	いいですね。	
\\	いいですね。		
\\	分かりました。	
\\	分[わ]かりました。		
\\	何ですか。	
\\	何[なん]ですか。		
\\	じゃ、またあした。	
\\	じゃ、またあした。		
\\	食べ物	
\\	食[た]べ 物[もの]		
\\	きゅうり	
\\	きゅうり		
\\	トマト	
\\	トマト		
\\	なす	
\\	なす		
\\	豆	
\\	豆[まめ]		
\\	キャベツ	
\\	キャベツ		
\\	ねぎ	
\\	ねぎ		
\\	白菜	
\\	白[はく] 菜[さい]		
\\	ほうれん草	
\\	ほうれん 草[そう]		
\\	レタス	
\\	レタス		
\\	じゃがいも	
\\	じゃがいも		
\\	大根	
\\	大[だい] 根[こん]		
\\	たまねぎ	
\\	たまねぎ		
\\	にんじん	
\\	にんじん		
\\	いちご	
\\	いちご		
\\	桃	
\\	桃[もも]		
\\	すいか	
\\	すいか		
\\	ぶどう	
\\	ぶどう		
\\	なし	
\\	なし		
\\	柿	
\\	柿[かき]		
\\	みかん	
\\	みかん		
\\	りんご	
\\	りんご		
\\	バナナ	
\\	バナナ		
\\	ソーセージ	
\\	ソーセージ		
\\	米	
\\	米[こめ]		
\\	牛肉	
\\	牛[ぎゅう] 肉[にく]		
\\	鳥肉	
\\	鳥[とり] 肉[にく]		
\\	豚肉	
\\	豚[ぶた] 肉[にく]		
\\	ハム	
\\	ハム		
\\	あじ	
\\	あじ		
\\	いわし	
\\	いわし		
\\	さば	
\\	さば		
\\	さんま	
\\	さんま		
\\	鮭	
\\	鮭[さけ]		
\\	まぐろ	
\\	まぐろ		
\\	たい	
\\	たい		
\\	鱈	
\\	鱈[たら]		
\\	蝦	
\\	蝦[えび]		
\\	かに	
\\	かに		
\\	いか	
\\	いか		
\\	たこ	
\\	たこ		
\\	貝	
\\	貝[かい]		
\\	切ります	
\\	切[き]ります		
\\	送ります	
\\	送[おく]ります		
\\	上げます	
\\	上[あ]げます		
\\	もらいます	
\\	もらいます		
\\	貸します	
\\	貸[か]します		
\\	借ります	
\\	借[か]ります		
\\	教えます	
\\	教[おし]えます		
\\	習います	
\\	習[なら]います		
\\	電話をかけます	
\\	電[でん] 話[わ]をかけます		
\\	手	
\\	手[て]		
\\	箸	
\\	箸[はし]		
\\	スポーン	
\\	スポーン		
\\	ナイフ	
\\	ナイフ		
\\	フォーク	
\\	フォーク		
\\	はさみ	
\\	はさみ		
\\	パソコン	
\\	パソコン		
\\	ケータイ	
\\	ケータイ		
\\	メール	
\\	メール		
\\	年賀状	
\\	年[ねん] 賀[が] 状[じょう]		
\\	パンチ	
\\	パンチ		
\\	ホッチキス	
\\	ホッチキス		
\\	セロテープ	
\\	セロテープ		
\\	消しゴム	
\\	消[け]しゴム		
\\	紙	
\\	紙[かみ]		
\\	花	
\\	花[はな]		
\\	シャツ	
\\	シャツ		
\\	プレゼント	
\\	プレゼント		
\\	荷物	
\\	荷[に] 物[もつ]		
\\	お金	
\\	お 金[かね]		
\\	切符	
\\	切[きっ] 符[ぷ]		
\\	クリスマス	
\\	クリスマス		
\\	父	
\\	父[ちち]		
\\	母	
\\	母[はは]		
\\	お父さん	
\\	お 父[とう]さん		
\\	お母さん	
\\	お 母[かあ]さん		
\\	もう	
\\	もう		
\\	まだ	
\\	まだ		
\\	これから	
\\	これから		
\\	[~、]すてきですね。	
\\	[~、]すてきですね。		
\\	いらっしゃい。	
\\	いらっしゃい。		
\\	どうぞお上がりください。	
\\	どうぞお 上[あ]がりください。		
\\	失礼します。	
\\	失[しつ] 礼[れい]します。		
\\	いただきます。	
\\	たべます、のみます、もらいます)	いただきます。		
\\	ごちそうさまでした。	
\\	ごちそうさまでした。		
\\	祖母	
\\	祖[そ] 母[ぼ]		
\\	祖父	
\\	祖[そ] 父[ふ]		
\\	祖父母	
\\	祖[そ] 父[ふ] 母[ぼ]		
\\	両親	
\\	両[りょう] 親[しん]		
\\	妹	
\\	妹[いもうと]		
\\	弟	
\\	弟[おとうと]		
\\	姉	
\\	姉[あね]		
\\	兄	
\\	兄[あに]		
\\	兄弟	
\\	兄[きょう] 弟[だい]		
\\	妻	
\\	妻[つま]		
\\	夫	
\\	夫[おっと]		
\\	夫婦	
\\	夫[ふう] 婦[ふ]		
\\	娘	
\\	娘[むすめ]		
\\	息子	
\\	息[むす] 子[こ]		
\\	子供	
\\	子[こ] 供[ども]		
\\	ハンサ厶(な)	
\\	ハンサ厶(な)		
\\	きれい(な)	
\\	きれい(な)		
\\	静か(な)	
\\	静[しず]か(な)		
\\	賑やか(な)	
\\	賑[にぎ]やか(な)		
\\	有名(な)	
\\	有[ゆう] 名[めい](な)		
\\	親切(な)	
\\	親[しん] 切[せつ](な)		
\\	元気(な)	
\\	元[げん] 気[き](な)		
\\	暇(な)	
\\	暇[ひま](な)		
\\	便利(な)	
\\	便[べん] 利[り](な)		
\\	すてき(な)	
\\	すてき(な)		
\\	大きい	
\\	大[おお]きい		
\\	小さい	
\\	小[ちい]さい		
\\	新しい	
\\	新[あたら]しい		
\\	古い	
\\	古[ふる]い		
\\	悪い	
\\	悪[わる]い		
\\	暑い	
\\	暑[あつ]い		
\\	熱い	
\\	熱[あつ]い		
\\	寒い	
\\	寒[さむ]い		
\\	冷たい	
\\	冷[つめ]たい		
\\	難しい	
\\	難[むずか]しい		
\\	易しい	
\\	易[やさ]しい		
\\	高い	
\\	高[たか]い		
\\	安い	
\\	安[やす]い		
\\	低い	
\\	低[ひく]い		
\\	面白い	
\\	面[おも] 白[しろ]い		
\\	おいしい	
\\	おいしい		
\\	忙しい	
\\	忙[いそが]しい		
\\	楽しい	
\\	楽[たの]しい		
\\	白い	
\\	白[しろ]い		
\\	黒い	
\\	黒[くろ]い		
\\	赤い	
\\	赤[あか]い		
\\	青い	
\\	青[あお]い		
\\	桜	
\\	桜[さくら]		
\\	山	
\\	山[やま]		
\\	町	
\\	町[まち]		
\\	所	
\\	所[ところ]		
\\	寮	
\\	寮[りょう]		
\\	生活	
\\	生[せい] 活[かつ]		
\\	お仕事をします	
\\	お 仕[し] 事[ごと]をします		
\\	どう	
\\	どう		
\\	どんな〜	
\\	〜	どんな〜		
\\	とても	
\\	とても		
\\	あまり	
\\	あまり		
\\	そして	
\\	そして		
\\	〜が、〜	〜
\\	〜	〜が、〜		
\\	お元気ですか。	
\\	お 元[げん] 気[き]ですか。		
\\	[〜、]もう一杯いかがですか。	
\\	[〜]?	[〜、]もう 一[いっ] 杯[ぱい]いかがですか。		
\\	いいえ、けっこうです。	
\\	いいえ、けっこうです。		
\\	もう~です [ね]。	
\\	もう~です [ね]。		
\\	そろそろ失礼します。	
\\	そろそろ 失[しつ] 礼[れい]します。		
\\	いいえ。	
\\	いいえ。		
\\	またいらっしゃってください。	
\\	またいらっしゃってください。		
\\	上海	
\\	上[シャン] 海[ハイ]		
\\	金閣寺	
\\	金[きん] 閣[かく] 寺[じ]		
\\	奈良公園	
\\	奈[な] 良[ら] 公[こう] 園[えん]		
\\	富士山	
\\	富[ふ] 士[じ] 山[さん]		
\\	七人の侍	
\\	七[しち] 人[にん]の 侍[さむらい]		
\\	黄色い	
\\	黄[き] 色[いろ]い		
\\	茶色い	
\\	茶[ちゃ] 色[いろ]い		
\\	ピンク	
\\	ピンク		
\\	オレンジ	
\\	オレンジ		
\\	緑	
\\	緑[みどり]		
\\	紫	
\\	紫[むらさき]		
\\	グレー	
\\	グレー		
\\	ベージュ	
\\	ベージュ		
\\	甘い	
\\	甘[あま]い		
\\	辛い	
\\	辛[つら]い		
\\	苦い	
\\	苦[にが]い		
\\	塩辛い	
\\	塩[しお] 辛[から]い		
\\	酸っぱい	
\\	酸[す]っぱい		
\\	濃い	
\\	濃[こ]い		
\\	薄い	
\\	薄[うす]い		
\\	味	
\\	味[あじ]		
\\	色	
\\	色[いろ]		
\\	分かります	
\\	分[わ]かります		
\\	あります	
\\	あります		
\\	好き(な)	
\\	好[す]き(な)		
\\	嫌い(な)	
\\	嫌[きら]い(な)		
\\	上手(な)	
\\	上[じょう] 手[ず](な)		
\\	下手(な)	
\\	下手[へた](な)		
\\	飲み物	
\\	飲[の]み 物[もの]		
\\	料理をします	
\\	料[りょう] 理[り]をします		
\\	スポーツをします	
\\	スポーツをします		
\\	野球をします	
\\	野[や] 球[きゅう]をします		
\\	ダンス	
\\	ダンス		
\\	旅行をします	
\\	旅[りょ] 行[こう]をします		
\\	音楽	
\\	音[おん] 楽[がく]		
\\	歌	
\\	歌[うた]		
\\	クラシック	
\\	クラシック		
\\	ジャズ	
\\	ジャズ		
\\	コンサート	
\\	コンサート		
\\	カラオケ	
\\	カラオケ		
\\	歌舞伎	
\\	歌[か] 舞[ぶ] 伎[き]		
\\	絵	
\\	絵[え]		
\\	字	
\\	字[じ]		
\\	漢字	
\\	漢[かん] 字[じ]		
\\	ひらがな	
\\	ひらがな		
\\	かたかな	
\\	かたかな		
\\	ローマ字	
\\	ローマ 字[じ]		
\\	細かいお金	
\\	細[こま]かいお 金[かね]		
\\	チケット	
\\	チケット		
\\	用事	
\\	用[よう] 事[じ]		
\\	約束をします	
\\	約[やく] 束[そく]をします		
\\	アルバイトをします	
\\	アルバイトをします		
\\	ご主人	
\\	ご 主[しゅ] 人[じん]		
\\	奥さん	
\\	奥[おく]さん		
\\	よく	
\\	よく		
\\	だいたい	
\\	だいたい		
\\	たくさん	
\\	たくさん		
\\	少し	
\\	少[すこ]し		
\\	全然	
\\	全[ぜん] 然[ぜん]		
\\	早く	
\\	早[はや]く		
\\	速く	
\\	速[はや]く		
\\	どうして	
\\	どうして		
\\	貸してください。	
\\	貸[か]してください。		
\\	いいですよ。	
\\	いいですよ。		
\\	残念ですが	
\\	残[ざん] 念[ねん]ですが		
\\	ああ	
\\	ああ		
\\	一緒にいかがですか。	
\\	一[いっ] 緒[しょ]にいかがですか。		
\\	[~は]ちょっと……。	
\\	[~は]ちょっと……。		
\\	だめですか。	
\\	だめですか。		
\\	また今度お願いします。	
\\	また 今[こん] 度[ど]お 願[ねが]いします。		
\\	ポップス	
\\	ポップス		
\\	ロック	
\\	ロック		
\\	ラテン	
\\	ラテン		
\\	民謡	
\\	民[みん] 謡[よう]		
\\	演歌	
\\	演[えん] 歌[か]		
\\	ミュージカル	
\\	ミュージカル		
\\	オペラ	
\\	オペラ		
\\	ホラー	
\\	ホラー		
\\	アニメ	
\\	アニメ		
\\	ドキュメンタリー	
\\	ドキュメンタリー		
\\	恋愛	
\\	恋[れん] 愛[あい]		
\\	ミステリー	
\\	ミステリー		
\\	文芸	
\\	文[ぶん] 芸[げい]		
\\	戦争	
\\	戦[せん] 争[そう]		
\\	アクション	
\\	アクション		
\\	喜劇	
\\	喜[き] 劇[げき]		
\\	ソフトボール	
\\	ソフトボール		
\\	ラグビー	
\\	ラグビー		
\\	バレーボール	
\\	バレーボール		
\\	バスケットボール	
\\	バスケットボール		
\\	ボウリング	
\\	ボウリング		
\\	スキー	
\\	スキー		
\\	スケート	
\\	スケート		
\\	野球	
\\	野[や] 球[きゅう]		
\\	卓球(ピンポン)	
\\	卓[たっ] 球[きゅう](ピンポン)		
\\	相撲	
\\	相[す] 撲[もう]		
\\	柔道	
\\	柔[じゅう] 道[どう]		
\\	剣道	
\\	剣[けん] 道[どう]		
\\	水泳	
\\	水[すい] 泳[えい]		
\\	います	
\\	います		
\\	いろいろ(な)	
\\	いろいろ(な)		
\\	男の人	
\\	男[おとこ]の 人[ひと]		
\\	女の人	
\\	女[おんな]の 人[ひと]		
\\	男の子	
\\	男[おとこ]の 子[こ]		
\\	女の子	
\\	女[おんな]の 子[こ]		
\\	犬	
\\	犬[いぬ]		
\\	猫	
\\	猫[ねこ]		
\\	パンダ	
\\	パンダ		
\\	象	
\\	象[ぞう]		
\\	木	
\\	木[き]		
\\	物	
\\	物[もの]		
\\	電池	
\\	電[でん] 池[ち]		
\\	箱	
\\	箱[はこ]		
\\	スイッチ	
\\	スイッチ		
\\	冷蔵庫	
\\	冷[れい] 蔵[ぞう] 庫[こ]		
\\	テーブル	
\\	テーブル		
\\	ベッド	
\\	ベッド		
\\	棚	
\\	棚[たな]		
\\	ドア	
\\	ドア		
\\	窓	
\\	窓[まど]		
\\	ポスト	
\\	ポスト		
\\	ビル	
\\	ビル		
\\	コンビニ	
\\	コンビニ		
\\	公園	
\\	公[こう] 園[えん]		
\\	喫茶店	
\\	喫[きっ] 茶[さ] 店[てん]		
\\	~屋	
\\	屋[や]		
\\	乗り場	
\\	乗[の]り 場[ば]		
\\	上	
\\	上[うえ]		
\\	下	
\\	下[した]		
\\	前	
\\	前[まえ]		
\\	後ろ	
\\	後[うし]ろ		
\\	右	
\\	右[みぎ]		
\\	左	
\\	左[ひだり]		
\\	中	
\\	中[なか]		
\\	外	
\\	外[そと]		
\\	隣	
\\	隣[となり]		
\\	近く	
\\	近[ちか]く		
\\	間	
\\	間[あいだ]		
\\	~や~ [など]	
\\	~や~ [など]		
\\	どうもすみません。	
\\	どうもすみません。		
\\	ナンプラー	
\\	ナンプラー		
\\	コーナー	
\\	コーナー		
\\	一番下	
\\	一[いち] 番[ばん] 下[した]		
\\	東京ディズニーランド	
\\	東[とう] 京[きょう]ディズニーランド		
\\	玄関	
\\	玄[げん] 関[かん]		
\\	トイレ	
\\	トイレ		
\\	風呂場	
\\	風[ふ] 呂[ろ] 場[ば]		
\\	洗面所	
\\	洗[せん] 面[めん] 所[じょ]		
\\	台所	
\\	台[だい] 所[どころ]		
\\	寝室	
\\	寝[しん] 室[しつ]		
\\	廊下	
\\	廊[ろう] 下[か]		
\\	ベランダ	
\\	ベランダ		
\\	子供がいます	
\\	子[こ] 供[ども]がいます		
\\	日本にいます	
\\	日本[にほん]にいます		
\\	かかります	
\\	かかります		
\\	会社を休みます	
\\	会[かい] 社[しゃ]を 休[やす]みます		
\\	一人	
\\	一人[ひとり]		
\\	二人	
\\	二人[ふたり]		
\\	ー人	ー 
\\	ー 人[にん]		
\\	ー台	
\\	ー 台[だい]		
\\	ー枚	
\\	ー 枚[まい]		
\\	ー回	ー
\\	ー 回[かい]		
\\	サンドイッチ	
\\	サンドイッチ		
\\	カレーライス	
\\	カレーライス		
\\	アイスクリーム	
\\	アイスクリーム		
\\	切手	
\\	切[きっ] 手[て]		
\\	はがき	
\\	はがき		
\\	封筒	
\\	封[ふう] 筒[とう]		
\\	外国	
\\	外[がい] 国[こく]		
\\	留学生	
\\	留[りゅう] 学[がく] 生[せい]		
\\	クラス	
\\	クラス		
\\	ー時間	ー 
\\	ー 時[じ] 間[かん]		
\\	ー週間	ー 
\\	ー 週[しゅう] 間[かん]		
\\	ーか月	ー 
\\	ーか 月[げつ]		
\\	~ぐらい	
\\	~ぐらい		
\\	どのくらい	
\\	どのくらい		
\\	全部で	
\\	全[ぜん] 部[ぶ]で		
\\	みんな	
\\	みんな		
\\	~だけ	
\\	~だけ		
\\	かしこまりました。	
\\	かしこまりました。		
\\	いいおてんきですね。	
\\	いいおてんきですね。		
\\	お出かけですか。	
\\	お 出[で]かけですか。		
\\	ちょっと~まで。	
\\	ちょっと~まで。		
\\	行ってらっしゃい。	
\\	行ってらっしゃい。		
\\	行ってきます。	
\\	行[い]ってきます。		
\\	船便	
\\	船[ふな] 便[びん]		
\\	航空便	
\\	航[こう] 空[くう] 便[びん](エアメール)		
\\	お願いします。	
\\	お 願[ねが]いします。		
\\	メニュー	
\\	メニュー		
\\	定食	
\\	定[てい] 食[しょく]		
\\	ランチ	
\\	ランチ		
\\	天丼	
\\	天[てん] 丼[どん]		
\\	親子丼	
\\	親[おや] 子[こ] 丼[どん]		
\\	牛丼	
\\	牛[ぎゅう] 丼[どん]		
\\	焼き肉	
\\	焼[や]き 肉[にく]		
\\	野菜いため	
\\	野[や] 菜[さい]いため		
\\	漬物	
\\	漬[つけ] 物[もの]		
\\	みそ汁	
\\	みそ 汁[しる]		
\\	おにぎり	
\\	おにぎり		
\\	てんぷら	
\\	てんぷら		
\\	すし	
\\	すし		
\\	うどん	
\\	うどん		
\\	そば	
\\	そば		
\\	ラーメン	
\\	ラーメン		
\\	焼きそば	
\\	焼[や]きそば		
\\	お好み焼き	
\\	お 好[この]み 焼[や]き		
\\	ハンバーグ	
\\	ハンバーグ		
\\	コロッケ	
\\	コロッケ		
\\	えびフライ	
\\	えびフライ		
\\	フライドチケン	
\\	フライドチケン		
\\	サラダ	
\\	サラダ		
\\	スープ	
\\	スープ		
\\	スパゲッティ	
\\	スパゲッティ		
\\	ピザ	
\\	ピザ		
\\	ハンバーガー	
\\	ハンバーガー		
\\	トースト	
\\	トースト		
\\	ココア	
\\	ココア		
\\	コーラ	
\\	コーラ		
\\	簡単(な)	
\\	簡[かん]単[たん](な)		
\\	近い	
\\	近[ちか]い		
\\	遠い	
\\	遠[とお]い		
\\	速い	
\\	速[はや]い		
\\	早い	
\\	早[はや]い		
\\	遅い	
\\	遅[おそ]い		
\\	人が多い	
\\	人[ひと]が 多[おお]い		
\\	人が少ない	
\\	人[ひと]が 少[すく]ない		
\\	暖かい、温かい	
\\	暖[あたた]かい、 温[あたた]かい		
\\	涼しい	
\\	涼[すず]しい		
\\	重い	
\\	重[おも]い		
\\	軽い	
\\	軽[かる]い		
\\	コーヒーがいい	
\\	コーヒーがいい		
\\	季節	
\\	季[き] 節[せつ]		
\\	春	
\\	春[はる]		
\\	夏	
\\	夏[なつ]		
\\	秋	
\\	秋[あき]		
\\	冬	
\\	冬[ふゆ]		
\\	天気	
\\	天[てん] 気[き]		
\\	雨	
\\	雨[あめ]		
\\	雪	
\\	雪[ゆき]		
\\	曇り	
\\	曇[くも]り		
\\	ホテル	
\\	ホテル		
\\	空港	
\\	空[くう] 港[こう]		
\\	海	
\\	海[うみ]		
\\	世界	
\\	世[せ] 界[かい]		
\\	パーティーをします	
\\	パーティーをします		
\\	お祭り	
\\	お 祭[まつ]り		
\\	すき焼き	
\\	すき 焼[や]き		
\\	刺身	
\\	刺[さし] 身[み]		
\\	レモン	
\\	レモン		
\\	生け花をします	
\\	生[い]け 花[ばな]をします		
\\	紅葉、紅葉	もみじ:
\\	こうよう:
\\	紅葉[もみじ]、 紅[こう] 葉[よう]		
\\	一番	
\\	一[いち] 番[ばん]		
\\	ずっと	
\\	ずっと		
\\	初めて	
\\	初[はじ]めて		
\\	ただいま。	
\\	ただいま。		
\\	お帰りなさい。	
\\	お 帰[かえ]りなさい。		
\\	わあ、すごい人ですね。	
\\	わあ、すごい 人[ひと]ですね。		
\\	疲れました。	
\\	疲[つか]れました。		
\\	祇園祭	
\\	祇[ぎ] 園[おん] 祭[まつり]		
\\	ホンコン	
\\	ホンコン		
\\	名所	
\\	名[めい] 所[しょ]		
\\	鹿苑寺(金閣寺)	
\\	鹿[ろく] 苑[おん] 寺[じ]
\\	金[きん]閣[かく]寺[じ])		
\\	姫路城	
\\	姫[ひめ] 路[じ] 城[じょう]		
\\	原爆ドーム	
\\	原[げん] 爆[ばく]ドーム		
\\	天神祭	
\\	天[てん] 神[じん] 祭[まつり]		
\\	東大寺	
\\	東[とう] 大[だい] 寺[じ]		
\\	神田祭	
\\	神[かん] 田[だ] 祭[まつり]		
\\	皇居	
\\	皇[こう] 居[きょ]		
\\	東照宮	
\\	東[とう] 照[しょう] 宮[ぐう]		
\\	遊びます	
\\	遊[あそ]びます		
\\	泳ぎます	
\\	泳[およ]ぎます		
\\	迎えます	
\\	迎[むか]えます		
\\	疲れます	
\\	疲[つか]れます		
\\	結婚します	
\\	結[けっ] 婚[こん]します		
\\	買い物します	
\\	買[か]い 物[もの]します		
\\	食事します	
\\	食[しょく]事[じ]します		
\\	公園を散歩します	
\\	公[こう] 園[えん]を 散[さん] 歩[ぽ]します		
\\	大変(な)	
\\	大[たい] 変[へん](な)		
\\	欲しい	
\\	欲[ほ]しい		
\\	広い	
\\	広[ひろ]い		
\\	狭い	
\\	狭[せま]い		
\\	プール	
\\	プール		
\\	川	
\\	川[かわ]		
\\	美術	
\\	美[び] 術[じゅつ]		
\\	釣りをします	
\\	釣[つ]りをします		
\\	週末	
\\	週[しゅう] 末[まつ]		
\\	~ごろ	
\\	~ごろ		
\\	何か	
\\	何[なに]か		
\\	どこか	
\\	どこか		
\\	喉が乾きます	
\\	喉[のど]が 乾[かわ]きます		
\\	お腹が空きます	
\\	お 腹[なか]が 空[す]きます		
\\	そうしましょう。	
\\	そうしましょう。		
\\	ご注文は?	
\\	ご 注[ちゅう] 文[もん]は?		
\\	少々お待ちください。	
\\	少[しょう] 々[しょう]お 待[ま]ちください。		
\\	別々に	
\\	別[べつ] 々[べつ]に		
\\	博物館	
\\	博[はく] 物[ぶつ] 館[かん]		
\\	映画館	
\\	映[えい] 画[が] 館[かん]		
\\	動物園	
\\	動[どう] 物[ぶつ] 園[えん]		
\\	植物園	
\\	植[しょく] 物[ぶつ] 園[えん]		
\\	お寺	
\\	お 寺[てら]		
\\	神社	
\\	神[じん] 社[じゃ]		
\\	教会	
\\	教[きょう] 会[かい]		
\\	モスク	
\\	モスク		
\\	体育館	
\\	体[たい] 育[いく] 館[かん]		
\\	大使館	
\\	大[たい] 使[し] 館[かん]		
\\	市役所	
\\	市[し] 役[やく] 所[しょ]		
\\	交番	
\\	交[こう] 番[ばん]		
\\	高校	
\\	高[こう] 校[こう]		
\\	中学校	
\\	中[ちゅう] 学[がっ] 校[こう]		
\\	小学校	
\\	小[しょう] 学[がっ] 校[こう]		
\\	幼稚園	
\\	幼[よう] 稚[ち] 園[えん]		
\\	肉屋	
\\	肉[にく] 屋[や]		
\\	パン屋	
\\	パン 屋[や]		
\\	魚屋	
\\	魚[さかな] 屋[や]		
\\	酒屋	
\\	酒[さか] 屋[や]		
\\	八百屋	
\\	八[や] 百[お] 屋[や]		
\\	つけます	
\\	つけます		
\\	消します	
\\	消[け]します		
\\	開けます	
\\	開[あ]けます		
\\	閉めます	
\\	閉[し]めます		
\\	急ぎます	
\\	急[いそ]ぎます		
\\	待ちます	
\\	待[ま]ちます		
\\	持ちます	
\\	持[も]ちます		
\\	取ります	
\\	取[と]ります		
\\	手伝います	
\\	手[て] 伝[つだ]います		
\\	呼びます	
\\	呼[よ]びます		
\\	話します	
\\	話[はな]します		
\\	使います	
\\	使[つか]います		
\\	止めます	
\\	止[と]めます		
\\	見せます	
\\	見[み]せます		
\\	住所を教えます	
\\	住[じゅう] 所[しょ]を 教[おし]えます		
\\	座ります	
\\	座[すわ]ります		
\\	立ちます	
\\	立[た]ちます		
\\	喫茶店に入ります	
\\	喫[きっ] 茶[さ] 店[てん]に 入[はい]ります		
\\	喫茶店を出ます	
\\	喫[きっ] 茶[さ] 店[てん]を 出[で]ます		
\\	雨が降ります	
\\	雨[あめ]が 降[ふ]ります		
\\	コピーします	
\\	コピーします		
\\	電気	
\\	電[でん] 気[き]		
\\	エアコン	
\\	エアコン		
\\	パスポート	
\\	パスポート		
\\	名前	
\\	名[な] 前[まえ]		
\\	住所	
\\	住[じゅう] 所[しょ]		
\\	地図	
\\	地[ち] 図[ず]		
\\	塩	
\\	塩[しお]		
\\	砂糖	
\\	砂[さ] 糖[とう]		
\\	問題	
\\	問[もん] 題[だい]		
\\	答え	
\\	答[こた]え		
\\	読み方	
\\	読[よ]み 方[かた]		
\\	~方	
\\	方[かた]		
\\	まっすぐ	
\\	まっすぐ		
\\	ゆっくり	
\\	ゆっくり		
\\	すぐ	
\\	すぐ		
\\	また	
\\	また		
\\	あとで	
\\	あとで		
\\	もう少し	
\\	もう 少[すこ]し		
\\	さあ	
\\	さあ		
\\	あれ?	
\\	あれ?		
\\	信号を右へ曲がってください。	
\\	信[しん] 号[ごう]を 右[みぎ]へ 曲[ま]がってください。		
\\	お釣り	
\\	お 釣[つ]り		
\\	切符売り場	
\\	切[きっ] 符[ぷ] 売[う]り 場[ば]		
\\	自動券売機	
\\	自[じ] 動[どう] 券[けん] 売[ばい] 機[き]		
\\	精算機	
\\	精[せい] 算[さん] 機[き]		
\\	改札口	
\\	改[かい] 札[さつ] 口[ぐち]		
\\	出口	
\\	出[で] 口[ぐち]		
\\	入口	
\\	入[いり] 口[ぐち]		
\\	東口	
\\	東[ひがし] 口[ぐち]		
\\	西口	
\\	西[にし] 口[ぐち]		
\\	南口	
\\	南[みなみ] 口[ぐち]		
\\	北口	
\\	北[きた] 口[ぐち]		
\\	中央口	
\\	中[ちゅう] 央[おう] 口[ぐち]		
\\	プラットホーム	
\\	プラットホーム		
\\	売店	
\\	売[ばい] 店[てん]		
\\	コインロッカー	
\\	コインロッカー		
\\	タクシー乗り場	
\\	タクシー 乗[の]り 場[ば]		
\\	バスターミナル	
\\	バスターミナル		
\\	バス停	
\\	バス 停[てい]		
\\	快速	
\\	快[かい] 速[そく]		
\\	準急	
\\	準[じゅん] 急[きゅう]		
\\	~発	
\\	発[はつ]		
\\	~着	
\\	着[ちゃく]		
\\	東京	
\\	東[とう] 京[きょう]		
\\	定期券	
\\	定[てい] 期[き] 券[けん]		
\\	回数券	
\\	回[かい] 数[すう] 券[けん]		
\\	片道	
\\	片[かた] 道[みち]		
\\	往復	
\\	往[おう] 復[ふく]		
\\	置きます	
\\	置[お]きます		
\\	作ります	
\\	作[つく]ります		
\\	造ります	
\\	造[つく]ります		
\\	売ります	
\\	売[う]ります		
\\	知ります	
\\	知[し]ります		
\\	住みます	
\\	住[す]みます		
\\	研究します	
\\	研[けん] 究[きゅう]します		
\\	資料	
\\	資[し] 料[りょう]		
\\	カタログ	
\\	カタログ		
\\	時刻表	
\\	時[じ] 刻[こく] 表[ひょう]		
\\	服	
\\	服[ふく]		
\\	製品	
\\	製[せい] 品[ひん]		
\\	ソフト	
\\	ソフト		
\\	電子辞書	
\\	電[でん] 子[し] 辞[じ] 書[しょ]		
\\	経済	
\\	経[けい] 済[ざい]		
\\	歯医者	
\\	歯[は] 医[い] 者[しゃ]		
\\	独身	
\\	独[どく] 身[しん]		
\\	すみません	
\\	すみません		
\\	皆さん	
\\	皆[みな]さん		
\\	思い出します	
\\	思[おも]い 出[だ]します		
\\	いらっしゃいます、おいでになります	
\\	いきます、きます、います)	いらっしゃいます、おいでになります		
\\	日本橋	
\\	日本[にっぽん] 橋[ばし]		
\\	職業	
\\	職[しょく] 業[ぎょう]		
\\	公務員	
\\	公[こう] 務[む] 員[いん]		
\\	駅員	
\\	駅[えき] 員[いん]		
\\	郵便局員	
\\	郵[ゆう] 便[びん] 局[きょく] 員[いん]		
\\	店員	
\\	店[てん] 員[いん]		
\\	調理師	
\\	調[ちょう] 理[り] 師[し]		
\\	理容師	
\\	理[り] 容[よう] 師[し]		
\\	美容師	
\\	美[び] 容[よう] 師[し]		
\\	弁護士	
\\	弁[べん] 護[ご] 士[し]		
\\	看護師	
\\	看[かん] 護[ご] 師[し]		
\\	運転手	
\\	運[うん] 転[てん] 手[しゅ]		
\\	警察官	
\\	警[けい] 察[さつ] 官[かん]		
\\	外交官	
\\	外[がい] 交[こう] 官[かん]		
\\	政治家	
\\	政[せい] 治[じ] 家[か]		
\\	画家	
\\	画[が] 家[か]		
\\	作家	
\\	作[さっ] 家[か]		
\\	音楽家	
\\	音[おん] 楽[がく] 家[か]		
\\	建築家	
\\	建[けん] 築[ちく] 家[か]		
\\	デザイナー	
\\	デザイナー		
\\	ジャーナリスト	
\\	ジャーナリスト		
\\	歌手	
\\	歌[か] 手[しゅ]		
\\	俳優	
\\	俳[はい] 優[ゆう]		
\\	スポーツ選手	
\\	スポーツ 選[せん] 手[しゅ]		
\\	十九日	
\\	十[じゅう] 九[く] 日[にち]		
\\	電車に乗ります	
\\	電[でん] 車[しゃ]に 乗[の]ります		
\\	電車を降ります	
\\	電[でん] 車[しゃ]を 降[お]ります		
\\	乗り換えます	
\\	乗[の]り 換[か]えます		
\\	シャワーを浴びる	
\\	シャワーを 浴[あ]びる		
\\	入れます	
\\	入[い]れます		
\\	出します	
\\	出[だ]します		
\\	お金を下ろします	
\\	お 金[かね]を 下[お]ろします		
\\	大学に入ります	
\\	大[だい] 学[がく]に 入[はい]ります		
\\	大学を出ます	
\\	大[だい] 学[がく]を 出[で]ます		
\\	押します	
\\	押[お]します		
\\	始めます	
\\	始[はじ]めます		
\\	見学します	
\\	見[けん] 学[がく]します		
\\	電話します	
\\	電[でん] 話[わ]します		
\\	若い	
\\	若[わか]い		
\\	長い	
\\	長[なが]い		
\\	短い	
\\	短[みじか]い		
\\	明るい	
\\	明[あか]るい		
\\	暗い	
\\	暗[くら]い		
\\	体	
\\	体[からだ]		
\\	頭	
\\	頭[あたま]		
\\	髪	
\\	髪[かみ]		
\\	顔	
\\	顔[かお]		
\\	目	
\\	目[め]		
\\	耳	
\\	耳[みみ]		
\\	鼻	
\\	鼻[はな]		
\\	口	
\\	口[くち]		
\\	歯	
\\	歯[は]		
\\	お腹	
\\	お 腹[なか]		
\\	足	
\\	足[あし]		
\\	背	
\\	背[せ]		
\\	サービス	
\\	サービス		
\\	ジョギングをします	
\\	ジョギングをします		
\\	ー番	
\\	ー	ー 番[ばん]		
\\	どうやって	
\\	どうやって		
\\	どの~	
\\	どの~		
\\	どれ	
\\	どれ		
\\	すごいですね。	
\\	すごいですね。		
\\	いいえ、まだまだです。	
\\	いいえ、まだまだです。		
\\	お引き出しですか。	
\\	お 引[ひ]き 出[だ]しですか。		
\\	まず	
\\	まず		
\\	次に	
\\	次[つぎ]に		
\\	キャッシュカード	
\\	キャッシュカード		
\\	暗証番号	
\\	暗[あん] 証[しょう] 番[ばん] 号[ごう]		
\\	金額	
\\	金[きん] 額[がく]		
\\	確認	
\\	確[かく] 認[にん]		
\\	ボタン	
\\	ボタン		
\\	雪祭り	
\\	雪[ゆき] 祭[まつ]り		
\\	バンドン	
\\	バンドン		
\\	フランケン	
\\	フランケン		
\\	ベラクルス	
\\	ベラクルス		
\\	梅田	
\\	梅[うめ] 田[だ]		
\\	お振り込み	
\\	お 振[ふ]り 込[こ]み		
\\	お振り替え	
\\	お 振[ふ]り 替[か]え		
\\	お引き出し	
\\	お 引[ひ]き 出[だ]し		
\\	通帳記入	
\\	通[つう] 帳[ちょう] 記[き] 入[にゅう]		
\\	残高照会	
\\	残[ざん] 高[だか] 照[しょう] 会[かい]		
\\	覚えます	
\\	覚[おぼ]えます		
\\	忘れます	
\\	忘[わす]れます		
\\	なくします	
\\	なくします		
\\	払います	
\\	払[はら]います		
\\	返します	
\\	返[かえ]します		
\\	出かけます	
\\	出[で]かけます		
\\	脱ぎます	
\\	脱[ぬ]ぎます		
\\	持って行きます	
\\	持[も]って 行[い]きます		
\\	持って来ます	
\\	持[も]って 来[き]ます		
\\	心配します	
\\	心[しん] 配[ぱい]します		
\\	残業します	
\\	残[ざん] 業[ぎょう]します		
\\	出張します	
\\	出[しゅっ] 張[ちょう]します		
\\	薬を飲みます	
\\	薬[くすり]を 飲[の]みます		
\\	入ります	
\\	入[はい]ります		
\\	大切(な)	
\\	大[たい] 切[せつ](な)		
\\	大丈夫(な)	
\\	大[だい] 丈[じょう] 夫[ぶ](な)		
\\	危ない	
\\	危[あぶ]ない		
\\	禁煙	
\\	禁[きん] 煙[えん]		
\\	健康	
\\	健[けん] 康[こう]		
\\	保険証	
\\	保[ほ] 険[けん] 証[しょう]		
\\	熱	
\\	熱[ねつ]		
\\	病気	
\\	病[びょう] 気[き]		
\\	薬	
\\	薬[くすり]		
\\	お風呂	
\\	お 風[ふ] 呂[ろ]		
\\	上着	
\\	上[うわ] 着[ぎ]		
\\	下着	
\\	下[した] 着[ぎ]		
\\	3日	
\\	日[にち]		
\\	~までに	
\\	〜までに		
\\	ですから	
\\	ですから		
\\	どうしましたか。	
\\	どうしましたか。		
\\	のど	
\\	のど		
\\	痛い	
\\	痛[いた]い		
\\	風邪	
\\	風[か] 邪[ぜ]		
\\	お大事に。	
\\	お 大[だい] 事[じ]に。		
\\	あご	
\\	あご		
\\	首	
\\	首[くび]		
\\	胸	
\\	胸[むね]		
\\	肩	
\\	肩[かた]		
\\	腕	
\\	腕[うで]		
\\	肘	
\\	肘[ひじ]		
\\	指	
\\	指[ゆび]		
\\	爪	
\\	爪[つめ]		
\\	背中	
\\	背[せ] 中[なか]		
\\	腰	
\\	腰[こし]		
\\	尻	
\\	尻[しり]		
\\	膝	
\\	膝[ひざ]		
\\	骨	
\\	骨[ほね]		
\\	インフルエンザ	
\\	インフルエンザ		
\\	盲腸	
\\	盲[もう] 腸[ちょう]		
\\	ぎっくり腰	
\\	ぎっくり 腰[ごし]		
\\	捻挫	
\\	捻[ねん] 挫[ざ]		
\\	骨折	
\\	骨[こっ] 折[せつ]		
\\	二日酔い	
\\	二[ふつ] 日[か] 酔[よ]い		
\\	頭が痛い	
\\	頭[あたま]が 痛[いた]い		
\\	お腹が痛い	
\\	お 腹[なか]が 痛[いた]い		
\\	歯が痛い	
\\	歯[は]が 痛[いた]い		
\\	熱があります	
\\	熱[ねつ]があります		
\\	咳が出ます	
\\	咳[せき]が 出[で]ます		
\\	鼻水が出ます	
\\	鼻[はな] 水[みず]が 出[で]ます		
\\	血が出ます	
\\	血[ち]が 出[で]ます		
\\	吐き気がします	
\\	吐[は]き 気[け]がします		
\\	寒気がします	
\\	寒[さむ] 気[け]がします		
\\	めまいがします	
\\	めまいがします		
\\	下痢をします	
\\	下[げ] 痢[り]をします		
\\	便秘をします	
\\	便[べん] 秘[ぴ]をします		
\\	けがをします	
\\	けがをします		
\\	やけどをします	
\\	やけどをします		
\\	食欲がありません	
\\	食[しょく] 欲[よく]がありません		
\\	肩が凝ります	
\\	肩[かた]が 凝[こ]ります		
\\	体がだるい	
\\	体[からだ]がだるい		
\\	かゆい	
\\	かゆい		
\\	林	
\\	林[はやし]		
\\	金曜日	
\\	金[きん] 曜[よう] 日[び]		
\\	タクシー	
\\	タクシー		
\\	自転車	
\\	自[じ] 転[てん] 車[しゃ]		
\\	歩いて	
\\	歩[ある]いて		
\\	何日	
\\	何[なん] 日[にち]		
\\	建国記念の日	
\\	(2月11日)	建[けん] 国[こく] 記[き] 念[ねん]の 日[ひ]		
\\	ええ	
\\	ええ		
\\	居間	
\\	居[い] 間[ま]		
\\	エンジニア	
\\	エンジニア		
\\	お預け入れ	
\\	お 預[あず]け 入[い]れ		
\\	できます	
\\	できます		
\\	洗います	
\\	洗[あら]います		
\\	弾きます	
\\	弾[ひ]きます		
\\	歌います	
\\	歌[うた]います		
\\	集めます	
\\	集[あつ]めます		
\\	捨てます	
\\	捨[す]てます		
\\	換えます	
\\	換[か]えます		
\\	運転します	
\\	運[うん] 転[てん]します		
\\	予約します	
\\	予[よ] 約[やく]します		
\\	ピアノ	
\\	ピアノ		
\\	ーメートル	ー 
\\	ーメートル		
\\	現金	
\\	現[げん] 金[きん]		
\\	趣味	
\\	趣[しゅ] 味[み]		
\\	日記	
\\	日[にっ] 記[き]		
\\	お祈りをします	
\\	お 祈[いの]り		
\\	課長	
\\	課[か] 長[ちょう]		
\\	部長	
\\	部[ぶ] 長[ちょう]		
\\	社長	
\\	社[しゃ] 長[ちょう]		
\\	動物	
\\	動[どう] 物[ぶつ]		
\\	馬	
\\	馬[うま]		
\\	インターネット	
\\	インターネット		
\\	特に	
\\	特[とく]に		
\\	へえ	
\\	へえ		
\\	それはおもしろいですね。	
\\	それはおもしろいですね。		
\\	なかなか	
\\	なかなか		
\\	ほんとうですか。	
\\	ほんとうですか。		
\\	ぜひ	
\\	ぜひ		
\\	動き	
\\	動[うご]き		
\\	飛ぶ	
\\	飛[と]ぶ		
\\	跳ぶ	
\\	跳[と]ぶ		
\\	登る	
\\	登[のぼ]る		
\\	走る	
\\	走[はし]る		
\\	泳ぐ	
\\	泳[およ]ぐ		
\\	潜る	
\\	潜[もぐ]る		
\\	飛び込む	
\\	飛[と]び 込[こ]む		
\\	逆立ちます	
\\	逆[さか] 立[だ]ちます		
\\	はう	
\\	はう		
\\	ける	
\\	ける		
\\	振る	
\\	振[ふ]る		
\\	持ち上げる	
\\	持[も]ち 上[あ]げる		
\\	投げる	
\\	投[な]げる		
\\	たたく	
\\	たたく		
\\	引く	
\\	引[ひ]く		
\\	押す	
\\	押[お]す		
\\	曲げる	
\\	曲[ま]げる		
\\	伸ばす	
\\	伸[の]ばす		
\\	転ぶ	
\\	転[ころ]ぶ		
\\	振り向く	
\\	振[ふ]り 向[む]く		
\\	登ります、上ります	
\\	登[のぼ]ります、 上[のぼ]ります		
\\	ホテルに泊まります	
\\	ホテルに 泊[と]まります		
\\	掃除します	
\\	掃[そう] 除[じ]します		
\\	洗濯します	
\\	洗[せん] 濯[たく]します		
\\	なります	
\\	なります		
\\	眠い	
\\	眠[ねむ]い		
\\	強い	
\\	強[つよ]い		
\\	弱い	
\\	弱[よわ]い		
\\	練習します	
\\	練[れん] 習[しゅう]します		
\\	ゴルフをします	
\\	ゴルフをします		
\\	日	
\\	日[ひ]		
\\	調子	
\\	調[ちょう] 子[し]		
\\	一度	
\\	一[いち] 度[ど]		
\\	一度も	
\\	一[いち] 度[ど]も		
\\	だんだん	
\\	だんだん		
\\	もうすぐ	
\\	もうすぐ		
\\	おかげさまで	
\\	おかげさまで		
\\	でも	
\\	でも		
\\	乾杯	
\\	乾[かん] 杯[ぱい]		
\\	ダイエットをします	
\\	ダイエットをします		
\\	無理(な)	
\\	無[む] 理[り](な)		
\\	体にいい	
\\	体[からだ]にいい		
\\	伝統文化	
\\	伝[でん] 統[とう] 文[ぶん] 化[か]		
\\	娯楽	
\\	娯[ご] 楽[らく]		
\\	茶道(お茶)	
\\	茶[さ] 道[どう](お 茶[ちゃ])		
\\	華道	
\\	華[か] 道[どう]		
\\	書道	
\\	書[しょ] 道[どう]		
\\	能	
\\	能[のう]		
\\	文楽	
\\	文[ぶん] 楽[らく]		
\\	空手	
\\	空[から] 手[て]		
\\	漫才	
\\	漫[まん] 才[ざい]		
\\	落語	
\\	落[らく] 語[ご]		
\\	囲碁	
\\	囲[い] 碁[ご]		
\\	将棋	
\\	将[しょう] 棋[ぎ]		
\\	盆踊り	
\\	盆[ぼん] 踊[おど]り		
\\	ビザが要ります	
\\	ビザが 要[い]ります		
\\	調べます	
\\	調[しら]べます		
\\	修理します	
\\	修[しゅう] 理[り]します		
\\	僕	
\\	わたし 
\\	僕[ぼく]		
\\	君	
\\	あなた)	君[きみ]		
\\	~君	
\\	~さん)	
\\	君[くん]		
\\	うん	
\\	はい)	うん		
\\	ううん	
\\	いいえ)	ううん		
\\	言葉	
\\	言[こと] 葉[ば]		
\\	着物	
\\	着[き] 物[もの]		
\\	ビザ	
\\	ビザ		
\\	初め	
\\	初[はじ]め		
\\	終わり	
\\	終[お]わり		
\\	こっち	
\\	こちら)	こっち		
\\	そっち	
\\	そちら)	そっち		
\\	あっち	
\\	あちら)	あっち		
\\	どっち	
\\	どちら)	どっち		
\\	みんなで	
\\	みんなで		
\\	~けど	
\\	が)	~けど		
\\	お腹がいっぱいです	
\\	お 腹[なか]がいっぱいです		
\\	よかったら	
\\	よかったら		
\\	思います	
\\	思[おも]います		
\\	言います	
\\	言[い]います		
\\	勝ちます	
\\	勝[か]ちます		
\\	負けます	
\\	負[ま]けます		
\\	お祭りがあります	
\\	お 祭[まつ]りがあります		
\\	役に立ちます	
\\	役[やく]に 立[だ]ちます		
\\	動きます	
\\	動[うご]きます		
\\	会社をやめます	
\\	会[かい] 社[しゃ]をやめます		
\\	気をつけます	
\\	気[き]をつけます		
\\	留学します	
\\	留[りゅう] 学[がく]します		
\\	むだ(な)	
\\	むだ(な)		
\\	不便(な)	
\\	不[ふ] 便[べん](な)		
\\	すごい	
\\	すごい		
\\	本当	
\\	本[ほん] 当[とう]		
\\	うそ	
\\	うそ		
\\	自動車	
\\	自[じ] 動[どう] 車[しゃ]		
\\	交通	
\\	交[こう] 通[つう]		
\\	物価	
\\	物[ぶっ] 価[か]		
\\	放送	
\\	放[ほう] 送[そう]		
\\	パチンコ	
\\	パチンコ		
\\	ニュース	
\\	ニュース		
\\	マンガ	
\\	マンガ		
\\	デザイン	
\\	デザイン		
\\	夢	
\\	夢[ゆめ]		
\\	天才	
\\	天[てん] 才[さい]		
\\	試合をします	
\\	試[し] 合[あい]をします		
\\	意見	
\\	意[い] 見[けん]		
\\	話をします	
\\	話[はなし]をします		
\\	地球	
\\	地[ち] 球[きゅう]		
\\	月	
\\	月[つき]		
\\	最近	
\\	最[さい] 近[きん]		
\\	たぶん	
\\	たぶん		
\\	きっと	
\\	きっと		
\\	本当に	
\\	本[ほん] 当[とう]に		
\\	そんなに	
\\	そんなに		
\\	~について	
\\	~について		
\\	久しぶりですね。	
\\	久[ひさ]しぶりですね。		
\\	~でも飲みませんか。	
\\	~でも 飲[の]みませんか。		
\\	もちろん	
\\	もちろん		
\\	もう帰らないと……。	
\\	もう 帰[かえ]らないと……。		
\\	役職名	
\\	役[やく] 職[しょく] 名[めい]		
\\	国	
\\	国[くに]		
\\	都道府県	
\\	都[と] 道[どう] 府[ふ] 県[けん]		
\\	村	
\\	村[むら]		
\\	首相	
\\	首[しゅ] 相[しょう]		
\\	内閣総理大臣	
\\	内[ない] 閣[かく] 総[そう] 理[り] 大[だい] 臣[じん]		
\\	知事	
\\	知[ち] 事[じ]		
\\	市長	
\\	市[し] 長[ちょう]		
\\	町長	
\\	町[ちょう] 長[ちょう]		
\\	村長	
\\	村[そん] 長[ちょう]		
\\	高等学校	
\\	高[こう] 等[とう] 学[がっ] 校[こう]		
\\	学長	
\\	学[がく] 長[ちょう]		
\\	校長	
\\	校[こう] 長[ちょう]		
\\	園長	
\\	園[えん] 長[ちょう]		
\\	会長	
\\	会[かい] 長[ちょう]		
\\	重役	
\\	重[じゅう] 役[やく]		
\\	院長	
\\	院[いん] 長[ちょう]		
\\	看護師長	
\\	看[かん] 護[ご] 師[し] 長[ちょう]		
\\	頭取	
\\	頭[とう] 取[どり]		
\\	支店長	
\\	支[し] 店[てん] 長[ちょう]		
\\	駅長	
\\	駅[えき] 長[ちょう]		
\\	警察	
\\	警[けい] 察[さつ]		
\\	署長	
\\	署[しょ] 長[ちょう]		
\\	着ます	
\\	着[き]ます		
\\	はきます	
\\	はきます		
\\	かぶります	
\\	かぶります		
\\	眼鏡をかけます	
\\	眼[め] 鏡[がね]をかけます		
\\	ネクタイをします	
\\	ネクタイをします		
\\	生まれます	
\\	生[う]まれます		
\\	私たち	
\\	私[わたし]たち		
\\	コート	
\\	コート		
\\	セーター	
\\	セーター		
\\	スーツ	
\\	スーツ		
\\	帽子	
\\	帽[ぼう] 子[し]		
\\	ケーキ	
\\	ケーキ		
\\	お弁当	
\\	お 弁[べん] 当[とう]		
\\	ロボット	
\\	ロボット		
\\	ユーモア	
\\	ユーモア		
\\	都合	
\\	都[つ] 合[ごう]		
\\	えーと	
\\	えーと		
\\	おめでとうございます	
\\	おめでとうございます		
\\	お探しですか。	
\\	お 探[さが]しですか。		
\\	では	
\\	では		
\\	家賃	
\\	家[や] 賃[ちん]		
\\	ダイニングキッチン	
\\	ダイニングキッチン		
\\	和室	
\\	和[わ] 室[しつ]		
\\	押し入れ	
\\	押[お]し 入[い]れ		
\\	布団	
\\	布[ふ] 団[とん]		
\\	パリ	
\\	パリ		
\\	万里の長城	
\\	万[ばん] 里[り]の 長[ちょう] 城[じょう]		
\\	衣服	
\\	衣[い] 服[ふく]		
\\	ワンピース	
\\	ワンピース		
\\	ズボン	
\\	ズボン		
\\	パンツ	
\\	パンツ		
\\	ジーンズ	
\\	ジーンズ		
\\	スカート	
\\	スカート		
\\	ブラウス	
\\	ブラウス		
\\	ワイシャツ	
\\	ワイシャツ		
\\	マフラー	
\\	マフラー		
\\	手袋	
\\	手[て] 袋[ぶくろ]		
\\	靴下	
\\	靴[くつ] 下[した]		
\\	(パンティー)ストッキング	
\\	(パンティー)ストッキング		
\\	オーバコート	
\\	オーバコート		
\\	レインコート	
\\	レインコート		
\\	ベルト	
\\	ベルト		
\\	ハイヒール	
\\	ハイヒール		
\\	ブーツ	
\\	ブーツ		
\\	運動靴	
\\	運[うん] 動[どう] 靴[くつ]		
\\	帯	
\\	帯[おび]		
\\	ぞうり	
\\	ぞうり		
\\	たび	
\\	たび		
\\	先生に聞きます	
\\	先[せん] 生[せい]に 聞[き]きます		
\\	回します	
\\	回[まわ]します		
\\	引きます	
\\	引[ひ]きます		
\\	変えます	
\\	変[か]えます		
\\	ドアに触ります	
\\	ドアに 触[さわ]ります		
\\	お釣りが出ます	
\\	お 釣[つ]りが 出[で]ます		
\\	歩きます	
\\	歩[ある]きます		
\\	橋を渡ります	
\\	橋[はし]を 渡[わた]ります		
\\	右へ曲がります	
\\	右[みぎ]へ 曲[ま]がります		
\\	寂しい	
\\	寂[さび]しい		
\\	お湯	
\\	お 湯[ゆ]		
\\	音	
\\	音[おと]		
\\	サイズ	
\\	サイズ		
\\	故障します	
\\	故[こ] 障[しょう]します		
\\	道	
\\	道[みち]		
\\	交差点	
\\	交[こう] 差[さ] 点[てん]		
\\	信号	
\\	信[しん] 号[ごう]		
\\	角	
\\	角[かど]		
\\	橋	
\\	橋[はし]		
\\	建物	
\\	建[たて] 物[もの]		
\\	何回も	
\\	何[なん] 回[かい]も		
\\	ー目	
\\	ー 目[め]		
\\	道路	
\\	道[どう] 路[ろ]		
\\	歩道	
\\	歩[ほ] 道[どう]		
\\	車道	
\\	車[しゃ] 道[どう]		
\\	高速道路	
\\	高[こう] 速[そく] 道[どう] 路[ろ]		
\\	通り	
\\	通[とお]り		
\\	横断歩道	
\\	横[おう] 断[だん] 歩[ほ] 道[どう]		
\\	歩道橋	
\\	歩[ほ] 道[どう] 橋[きょう]		
\\	坂	
\\	坂[さか]		
\\	踏切	
\\	踏[ふみ] 切[きり]		
\\	ガソリンスタンド	
\\	ガソリンスタンド		
\\	止まれ	
\\	止[と]まれ		
\\	進入禁止	
\\	進[しん] 入[にゅう] 禁[きん] 止[し]		
\\	一方通行	
\\	一[いっ] 方[ぽう] 通[つう] 行[こう]		
\\	駐車禁止	
\\	駐[ちゅう] 車[しゃ] 禁[きん] 止[し]		
\\	右折禁止	
\\	右[う] 折[せつ] 禁[きん] 止[し]		
\\	くれます	
\\	くれます		
\\	直します	
\\	直[なお]します		
\\	連れていきます	
\\	連[つ]れていきます		
\\	連れてきます	
\\	連[つ]れてきます		
\\	人を送ります	
\\	人[ひと]を 送[おく]ります		
\\	紹介します	
\\	紹[しょう] 介[かい]します		
\\	案内します	
\\	案[あん] 内[ない]します		
\\	説明します	
\\	説[せつ] 明[めい]します		
\\	おじいさん、おじいちゃん	
\\	おじいさん、おじいちゃん		
\\	おばあさん、おばあちゃん	
\\	おばあさん、おばあちゃん		
\\	準備をします	
\\	準[じゅん] 備[び]をします		
\\	引っ越しします	
\\	引[ひ]っ 越[こ]しします		
\\	お菓子	
\\	お 菓子[かし]		
\\	ホームステイ	
\\	ホームステイ		
\\	全部	
\\	全[ぜん] 部[ぶ]		
\\	自分で	
\\	自[じ] 分[ぶん]で		
\\	他に	
\\	他[ほか]に		
\\	母の日	
\\	母[はは]の 日[ひ]		
\\	贈答	
\\	贈[ぞう] 答[とう]		
\\	習慣	
\\	習[しゅう] 慣[かん]		
\\	お年玉	
\\	お 年[とし] 玉[だま]		
\\	入学祝い	
\\	入[にゅう] 学[がく] 祝[いわ]い		
\\	卒業祝い	
\\	卒[そつ] 業[ぎょう] 祝[いわ]い		
\\	結婚祝い	
\\	結[けっ] 婚[こん] 祝[いわ]い		
\\	出産祝い	
\\	出[しゅっ] 産[さん] 祝[いわ]い		
\\	お中元	
\\	お 中[ちゅう] 元[げん]		
\\	お歳暮	
\\	お 歳[せい] 暮[ぼ]		
\\	お香典	
\\	お 香[こう] 典[でん]		
\\	お見舞い	
\\	お 見[み] 舞[ま]い		
\\	熨斗袋	
\\	熨[の] 斗[し] 袋[ぶくろ]		
\\	考えます	
\\	考[かんが]えます		
\\	着きます	
\\	着[つ]きます		
\\	年を取ります	
\\	年[とし]を 取[と]ります		
\\	田舎	
\\	田舎[いなか]		
\\	チャンス	
\\	チャンス		
\\	億	
\\	億[おく]		
\\	もし(~たら)	
\\	もし(~たら)		
\\	意味	
\\	意[い] 味[み]		
\\	もしもし	
\\	もしもし		
\\	転勤します	
\\	転[てん] 勤[きん]します		
\\	暇	
\\	暇[ひま]		
\\	いろいろお世話になりました。	
\\	いろいろお 世[せ] 話[わ]になりました。		
\\	頑張ります	
\\	頑[がん] 張[ば]ります		
\\	どうぞお元気です。	
\\	どうぞお 元[げん] 気[き]です。		
\\	一生	
\\	一[いっ] 生[しょう]		
\\	赤ちゃん	
\\	赤[あか]ちゃん		
\\	保育園	
\\	保[ほ] 育[いく] 園[えん]		
\\	青年	
\\	青[せい] 年[ねん]		
\\	短大	
\\	短[たん] 大[だい]		
\\	専門学校	
\\	専[せん] 門[もん] 学[がっ] 校[こう]		
\\	大学院	
\\	大[だい] 学[がく] 院[いん]		
\\	就職をします	
\\	就[しゅう] 職[しょく]をします		
\\	中年	
\\	中[ちゅう] 年[ねん]		
\\	離婚します	
\\	離[り] 婚[こん]します		
\\	再婚します	
\\	再[さい] 婚[こん]します		
\\	老人	
\\	老[ろう] 人[じん]		
\\	仕事をやめます	
\\	仕[し] 事[ごと]をやめます		
\\	死にます	
\\	死[し]にます		
\\	見ます、診ます	
\\	見[み]ます、 診[み]ます		
\\	探します、捜します	
\\	探[さが]します、 捜[さが]します		
\\	時間に遅れます	
\\	時[じ] 間[かん]に 遅[おく]れます		
\\	時間に間に合います	
\\	時[じ] 間[かん]に 間[ま]に 合[あ]います		
\\	やります	
\\	やります		
\\	拾います	
\\	拾[ひろ]います		
\\	連絡します	
\\	連[れん] 絡[らく]します		
\\	気分がいい	
\\	気[き] 分[ぶん]がいい		
\\	気分が悪い	
\\	気[き] 分[ぶん]が 悪[わる]い		
\\	運動会	
\\	運[うん] 動[どう] 会[かい]		
\\	フリーマーケット	
\\	フリーマーケット		
\\	場所	
\\	場[ば] 所[しょ]		
\\	ボランティア	
\\	ボランティア		
\\	財布	
\\	財[さい] 布[ふ]		
\\	ごみ	
\\	ごみ		
\\	国会議事堂	
\\	国[こっ] 会[かい] 議[ぎ] 事[じ] 堂[どう]		
\\	平日	
\\	平[へい] 日[じつ]		
\\	~弁	
\\	弁[べん]		
\\	今度	
\\	今[こん] 度[ど]		
\\	ずいぶん	
\\	ずいぶん		
\\	直接	
\\	直[ちょく] 接[せつ]		
\\	いつでも	
\\	いつでも		
\\	どこでも	
\\	どこでも		
\\	だれでも	
\\	だれでも		
\\	何でも	
\\	何[なん]でも		
\\	こんな~	
\\	こんな~		
\\	そんな~	
\\	そんな~		
\\	あんな~	
\\	あんな~		
\\	荷物が片づきます	
\\	荷[に] 物[もつ]が 片[かた]づきます		
\\	ごみを出します	
\\	ごみを 出[だ]します		
\\	燃えるごみ	
\\	燃[も]えるごみ		
\\	置き場	
\\	置[お]き 場[ば]		
\\	横	
\\	横[よこ]		
\\	瓶	
\\	瓶[びん]		
\\	缶	
\\	缶[かん]		
\\	ガス	
\\	ガス		
\\	宇宙	
\\	宇[う] 宙[ちゅう]		
\\	~様	
\\	さん)	
\\	様[さま]		
\\	宇宙船	
\\	宇[う] 宙[ちゅう] 船[せん]		
\\	怖い	
\\	怖[こわ]い		
\\	宇宙ステーション	
\\	宇[う] 宙[ちゅう]ステーション		
\\	違います	
\\	違[ちが]います		
\\	宇宙飛行士	
\\	宇[う] 宙[ちゅう] 飛[ひ] 行[こう] 士[し]		
\\	足ります	
\\	足[た]ります		
\\	収集日	
\\	収[しゅう] 集[しゅう] 日[び]		
\\	お知らせ	
\\	お 知[し]らせ		
\\	可燃ごみ	
\\	可[か] 燃[ねん]ごみ		
\\	燃える	
\\	燃[も]える		
\\	生ごみ	
\\	生[なま]ごみ		
\\	紙くず	
\\	紙[かみ]くず		
\\	不燃	
\\	不[ふ] 燃[ねん]		
\\	グラス	
\\	グラス		
\\	瀬戸物	
\\	瀬[せ] 戸[と] 物[もの]		
\\	金属製	
\\	金[きん] 属[ぞく] 製[せい]		
\\	資源ごみ	
\\	資[し] 源[げん]ごみ		
\\	ペットボトル	
\\	ペットボトル		
\\	粗大ごみ	
\\	粗[そ] 大[だい]ごみ		
\\	事前	
\\	事[じ] 前[ぜん]		
\\	申し込み	
\\	申[もう]し 込[こ]み		
\\	飼います	
\\	飼[か]います		
\\	道を走ります	
\\	道[みち]を 走[はし]ります		
\\	山が見えます	
\\	山[やま]が 見[み]えます		
\\	音が聞こえます	
\\	音[おと]が 聞[き]こえます		
\\	道ができます	
\\	道[みち]ができます		
\\	教室を開きます	
\\	教[きょう] 室[しつ]を 開[ひら]きます		
\\	心配(な)	
\\	心[しん] 配[ぱい](な)		
\\	ペット	
\\	ペット		
\\	鳥	
\\	鳥[とり]		
\\	声	
\\	声[こえ]		
\\	波	
\\	波[なみ]		
\\	花火	
\\	花[はな] 火[び]		
\\	道具	
\\	道[どう] 具[ぐ]		
\\	クリーニング	
\\	クリーニング		
\\	マンション	
\\	マンション		
\\	キッチン	
\\	キッチン		
\\	~教室	
\\	教[きょう] 室[しつ]		
\\	パーティールーム	
\\	パーティールーム		
\\	方	
\\	ひと)	方[かた]		
\\	~後	
\\	後[ご]		
\\	~しか	
\\	~しか		
\\	ほかの	
\\	ほかの		
\\	はっきり	
\\	はっきり		
\\	本棚	
\\	本[ほん] 棚[だな]		
\\	いつか	
\\	いつか		
\\	建てます	
\\	建[た]てます		
\\	すばらしい	
\\	すばらしい		
\\	子供たち	
\\	子[こ] 供[ども]たち		
\\	大好き(な)	
\\	大[だい] 好[す]き(な)		
\\	主人公	
\\	主[しゅ] 人[じん] 公[こう]		
\\	形	
\\	形[かたち]		
\\	不思議(な)	
\\	不[ふ] 思[し] 議[ぎ](な)		
\\	ポケット	
\\	ポケット		
\\	例えば	
\\	例[たと]えば		
\\	付けます	
\\	付[つ]けます		
\\	自由に	
\\	自[じ] 由[ゆう]に		
\\	空	
\\	空[そら]		
\\	飛びます	
\\	飛[と]びます		
\\	昔	
\\	昔[むかし]		
\\	自分	
\\	自[じ] 分[ぶん]		
\\	将来	
\\	将[しょう] 来[らい]		
\\	合鍵	
\\	合[あい] 鍵[かぎ]		
\\	かかと	
\\	かかと		
\\	つま先	
\\	つま 先[さき]		
\\	中敷き	
\\	中[なか] 敷[じ]き		
\\	交換	
\\	交[こう] 換[かん]		
\\	ファスナー	
\\	ファスナー		
\\	ハンドル	
\\	ハンドル		
\\	持ち手	
\\	持[も]ち 手[て]		
\\	ほつれ	
\\	ほつれ		
\\	縫い目	
\\	縫[ぬ]い 目[め]		
\\	水洗い	
\\	水[みず] 洗[あら]い		
\\	染み抜き	
\\	染[し]み 抜[ぬ]き		
\\	はっ水	
\\	はっ 水[すい]		
\\	加工	
\\	加[か] 工[こう]		
\\	サイズ直し	
\\	サイズ 直[なお]し		
\\	縮む	
\\	縮[ちぢ]む		
\\	伸びる	
\\	伸[の]びる		
\\	宅配便	
\\	宅[たく] 配[はい] 便[びん]		
\\	公共料金	
\\	公[こう] 共[きょう] 料[りょう] 金[きん]		
\\	支払い	
\\	支[し] 払[はら]い		
\\	コピー	
\\	コピー		
\\	ファクス	
\\	ファクス		
\\	販売	
\\	販[はん] 売[ばい]		
\\	パンが売れます	
\\	パンが 売[う]れます		
\\	踊ります	
\\	踊[おど]ります		
\\	噛みます	
\\	噛[か]みます		
\\	選びます	
\\	選[えら]びます		
\\	大学に通います	
\\	大[だい] 学[がく]に 通[かよ]います		
\\	メモします	
\\	メモします		
\\	まじめ(な)	
\\	まじめ(な)		
\\	熱心(な)	
\\	熱[ねっ] 心[しん](な)		
\\	偉い	
\\	偉[えら]い		
\\	ちょうどいい	
\\	ちょうどいい		
\\	景色	
\\	景[け] 色[しき]		
\\	美容院	
\\	美[び] 容[よう] 院[いん]		
\\	経験があります	
\\	経[けい] 験[けん]があります		
\\	経験をします	
\\	経[けい] 験[けん]をします		
\\	力	
\\	力[ちから]		
\\	学生に人気があります	
\\	学[がく] 生[せい]に 人[にん] 気[き]があります		
\\	ガム	
\\	ガム		
\\	品物	
\\	品[しな] 物[もの]		
\\	値段	
\\	値[ね] 段[だん]		
\\	給料	
\\	給[きゅう] 料[りょう]		
\\	ボーナス	
\\	ボーナス		
\\	ゲーム	
\\	ゲーム		
\\	番組	
\\	番[ばん] 組[ぐみ]		
\\	ドラマ	
\\	ドラマ		
\\	小説	
\\	小[しょう] 説[せつ]		
\\	小説家	
\\	小[しょう] 説[せつ] 家[か]		
\\	~家	
\\	家[か]		
\\	~機	
\\	機[き]		
\\	息子さん	
\\	息[むす] 子[こ]さん		
\\	娘さん	
\\	娘[むすめ]さん		
\\	しばらく	
\\	しばらく		
\\	たいてい	
\\	たいてい		
\\	それに	
\\	それに		
\\	それで	
\\	それで		
\\	ちょっとお願いがあるんですが。	
\\	ちょっとお 願[ねが]いがあるんですが。		
\\	実は	
\\	実[じつ]は		
\\	会話	
\\	会[かい] 話[わ]		
\\	うーん	
\\	うーん		
\\	参加します	
\\	参[さん] 加[か]します		
\\	日にち	
\\	日[ひ]にち		
\\	ど	
\\	ど		
\\	無料	
\\	無[む] 料[りょう]		
\\	誘います	
\\	誘[さそ]います		
\\	イベント	
\\	イベント		
\\	徒歩	
\\	徒[と] 歩[ほ]		
\\	築	
\\	築[ちく]		
\\	敷金	
\\	敷[しき] 金[きん]		
\\	礼金	
\\	礼[れい] 金[きん]		
\\	管理費	
\\	管[かん] 理[り] 費[ひ]		
\\	向き	
\\	向[む]き		
\\	ー階建て	ー 
\\	ー 階[かい] 建[だ]て		
\\	ー畳	
\\	ー 畳[じょう]		
\\	不動産	
\\	不[ふ] 動[どう] 産[さん]		
\\	ドアが開きます	
\\	ドアが 開[あき]ます		
\\	ドアが閉まります	
\\	ドアが 閉[し]まります		
\\	電気がつきます	
\\	電[でん] 気[き]がつきます		
\\	電気が消えます	
\\	電[でん] 気[き]が 消[き]えます		
\\	いすが壊れます	
\\	いすが 壊[こわ]れます		
\\	コップが割れます	
\\	コップが 割[わ]れます		
\\	木が折れます	
\\	木[き]が 折[お]れます		
\\	紙が破れます	
\\	紙[かみ]が 破[やぶ]れます		
\\	服が汚れます	
\\	服[ふく]が 汚[よご]れます		
\\	ポケットが付きます	
\\	ポケットが 付[つ]きます		
\\	ボタンが外れます	
\\	ボタンが 外[はず]れます		
\\	車が止まります	
\\	車[くるま]が 止[と]まります		
\\	まちがえます	
\\	まちがえます		
\\	落とします	
\\	落[お]とします		
\\	鍵が掛かります	
\\	鍵[かぎ]が 掛[か]かります		
\\	拭きます	
\\	拭[ふ]きます		
\\	取り替えます	
\\	取[と]り 替[か]えます		
\\	片づけます	
\\	片[かた]づけます		
\\	お皿	
\\	お 皿[さら]		
\\	おちゃわん	
\\	おちゃわん		
\\	コップ	
\\	コップ		
\\	袋	
\\	袋[ふくろ]		
\\	書類	
\\	書[しょ] 類[るい]		
\\	枝	
\\	枝[えだ]		
\\	スピーチをします	
\\	スピーチをします		
\\	返事をします	
\\	返[へん] 事[じ]をします		
\\	お先にどうぞ。	
\\	お 先[さき]にどうぞ。		
\\	今の電車	
\\	今[いま]の 電[でん] 車[しゃ]		
\\	忘れ物	
\\	忘[わす]れ 物[もの]		
\\	このくらい	
\\	このくらい		
\\	~側	
\\	側[がわ]		
\\	~辺	
\\	辺[へん]		
\\	覚えていません	
\\	覚[おぼ]えていません。		
\\	網棚	
\\	網[あみ] 棚[だな]		
\\	確か	
\\	確[たし]か		
\\	ああ、よかった。	
\\	ああ、よかった。		
\\	地震	
\\	地[じ] 震[しん]		
\\	壁	
\\	壁[かべ]		
\\	針	
\\	針[はり]		
\\	指します	
\\	指[さ]します		
\\	駅前	
\\	駅[えき] 前[まえ]		
\\	倒れます	
\\	倒[たお]れます		
\\	西	
\\	西[にし]		
\\	~の方	
\\	~の 方[ほう]		
\\	燃えます	
\\	燃[も]えます		
\\	レポーター	
\\	レポーター		
\\	状態	
\\	状[じょう] 態[たい]		
\\	様子	
\\	様[よう] 子[す]		
\\	太っている	
\\	太[ふと]っている		
\\	やせている	
\\	やせている		
\\	膨らんでいる	
\\	膨[ふく]らんでいる		
\\	穴が開いている	
\\	穴[あな]が 開[あ]いている		
\\	曲がっている	
\\	曲[ま]がっている		
\\	ゆがんでいる	
\\	ゆがんでいる		
\\	へこんでいる	
\\	へこんでいる		
\\	ねじれている	
\\	ねじれている		
\\	欠けている	
\\	欠[か]けている		
\\	ひびが入っている	
\\	ひびが 入[はい]っている		
\\	腐っている	
\\	腐[くさ]っている		
\\	乾いている	
\\	乾[かわ]いている		
\\	濡れている	
\\	濡[ぬ]れている		
\\	凍っている	
\\	凍[こお]っている		
\\	はります	
\\	はります		
\\	掛けます	
\\	掛[か]けます		
\\	飾ります	
\\	飾[かざ]ります		
\\	並べます	
\\	並[なら]べます		
\\	植えます	
\\	植[う]えます		
\\	戻します	
\\	戻[もど]します		
\\	まとめます	
\\	まとめます		
\\	しまいます	
\\	しまいます		
\\	決めます	
\\	決[き]めます		
\\	予習します	
\\	予[よ] 習[しゅう]します		
\\	復習します	
\\	復[ふく] 習[しゅう]します		
\\	そのままにします	
\\	そのままにします		
\\	授業	
\\	授[じゅ] 業[ぎょう]		
\\	講義	
\\	講[こう] 義[ぎ]		
\\	ミーティング	
\\	ミーティング		
\\	予定	
\\	予[よ] 定[てい]		
\\	ごみ箱	
\\	ごみ 箱[ばこ]		
\\	人形	
\\	人[にん] 形[ぎょう]		
\\	花瓶	
\\	花[か] 瓶[びん]		
\\	鏡	
\\	鏡[かがみ]		
\\	引き出し	
\\	引[ひ]き 出[だ]し		
\\	池	
\\	池[いけ]		
\\	元の所	
\\	元[もと]の 所[ところ]		
\\	周り	
\\	周[まわ]り		
\\	真ん中	
\\	真[ま]ん 中[なか]		
\\	隅	
\\	隅[すみ]		
\\	リュック	
\\	リュック		
\\	非常袋	
\\	非[ひ] 常[じょう] 袋[ぶくろ]		
\\	非常時	
\\	非[ひ] 常[じょう] 時[じ]		
\\	生活します	
\\	生[せい] 活[かつ]します		
\\	懐中電灯	
\\	懐[かい] 中[ちゅう] 電[でん] 灯[とう]		
\\	丸い	
\\	丸[まる]い		
\\	ある~	
\\	ある~		
\\	夢を見ます	
\\	夢[ゆめ]を 見[み]ます		
\\	嬉しい	
\\	嬉[うれ]しい		
\\	嫌(な)	
\\	嫌[いや](な)		
\\	すると	
\\	すると		
\\	目が覚めます	
\\	目[め]が 覚[さ]めます		
\\	非常	
\\	非[ひ] 常[じょう]		
\\	場合	
\\	場[ば] 合[あい]		
\\	備え	
\\	備[そな]え		
\\	倒れる	
\\	倒[たお]れる		
\\	消火器	
\\	消[しょう] 火[か] 器[き]		
\\	備える	
\\	備[そな]える		
\\	貯える	
\\	貯[たくわ]える		
\\	用意	
\\	用[よう] 意[い]		
\\	地域	
\\	地[ち] 域[いき]		
\\	避難場所	
\\	避[ひ] 難[なん] 場[ば] 所[しょ]		
\\	知人	
\\	知[ち] 人[じん]		
\\	友人	
\\	友達)	友[ゆう] 人[じん]		
\\	連絡先	
\\	連[れん] 絡[らく] 先[さき]		
\\	万一	
\\	万[まん] 一[いち]		
\\	丈夫	
\\	丈[じょう] 夫[ぶ]		
\\	落ち着く	
\\	落[お]ち 着[つ]く		
\\	始末	
\\	始[し] 末[まつ]		
\\	戸	
\\	戸[と]		
\\	確保	
\\	確[かく] 保[ほ]		
\\	慌てて	
\\	慌[あわ]てて		
\\	飛び出す	
\\	飛[と]び 出[だ]す		
\\	収まる	
\\	収[おさ]まる		
\\	情報	
\\	情[じょう] 報[ほう]		
\\	山崩れ	
\\	山[やま] 崩[くず]れ		
\\	崖崩れ	
\\	崖[がけ] 崩[くず]れ		
\\	津波	
\\	津[つ] 波[なみ]		
\\	注意	
\\	注[ちゅう] 意[い]		
\\	避難	
\\	避[ひ] 難[なん]		
\\	必ず	
\\	必[かなら]ず		
\\	台風	
\\	台[たい] 風[ふう]		
\\	気象情報	
\\	気[き] 象[しょう] 情[じょう] 報[ほう]		
\\	点検	
\\	点[てん] 検[けん]		
\\	緊急	
\\	緊[きん] 急[きゅう]		
\\	続けます	
\\	続[つづ]けます		
\\	見つけます	
\\	見[み]つけます		
\\	休みを取ります	
\\	休[やす]みを 取[と]ります		
\\	試験を受けます	
\\	試[し] 験[けん]を 受[う]けます		
\\	申し込みます	
\\	申[もう]し 込[こ]みます		
\\	休憩します	
\\	休[きゅう] 憩[けい]します		
\\	連休	
\\	連[れん] 休[きゅう]		
\\	作文	
\\	作[さく] 文[ぶん]		
\\	発表します	
\\	発[はっ] 表[ぴょう]します		
\\	展覧会	
\\	展[てん] 覧[らん] 会[かい]		
\\	結婚式	
\\	結[けっ] 婚[こん] 式[しき]		
\\	お葬式	
\\	お 葬[そう] 式[しき]		
\\	式	
\\	式[しき]		
\\	本社	
\\	本[ほん] 社[しゃ]		
\\	支店	
\\	支[し] 店[てん]		
\\	温泉	
\\	温[おん] 泉[せん]		
\\	帰り	
\\	帰[かえ]り		
\\	ー号	
\\	ー 号[ごう]		
\\	残ります	
\\	残[のこ]ります		
\\	入学試験	
\\	入[にゅう] 学[がく] 試[し] 験[けん]		
\\	月に	
\\	月[つき]に		
\\	卒業します	
\\	卒[そつ] 業[ぎょう]します		
\\	閉じます	
\\	閉[と]じます		
\\	都会	
\\	都[と] 会[かい]		
\\	専門	
\\	専[せん] 門[もん]		
\\	医学	
\\	医[い] 学[がく]		
\\	薬学	
\\	薬[やく] 学[がく]		
\\	化学	
\\	化[か] 学[がく]		
\\	生化学	
\\	生[せい] 化[か] 学[がく]		
\\	生物学	
\\	生[せい] 物[ぶつ] 学[がく]		
\\	農学	
\\	農[のう] 学[がく]		
\\	地学	
\\	地[ち] 学[がく]		
\\	地理学	
\\	地[ち] 理[り] 学[がく]		
\\	数学	
\\	数[すう] 学[がく]		
\\	物理学	
\\	物[ぶつ] 理[り] 学[がく]		
\\	工学	
\\	工[こう] 学[がく]		
\\	土木工学	
\\	土[ど] 木[ぼく] 工[こう] 学[がく]		
\\	電子工学	
\\	電[でん] 子[し] 工[こう] 学[がく]		
\\	電気工学	
\\	電[でん] 気[き] 工[こう] 学[がく]		
\\	機械工学	
\\	機[き] 械[かい] 工[こう] 学[がく]		
\\	コンピューター工学	
\\	コンピューター 工[こう] 学[がく]		
\\	遺伝子工学	
\\	遺[い] 伝[でん] 子[し] 工[こう] 学[がく]		
\\	建築学	
\\	建[けん] 築[ちく] 学[がく]		
\\	天文学	
\\	天[てん] 文[もん] 学[がく]		
\\	環境科学	
\\	環[かん] 境[きょう] 科[か] 学[がく]		
\\	政治学	
\\	政[せい] 治[じ] 学[がく]		
\\	国際関係学	
\\	国[こく] 際[さい] 関[かん] 係[けい] 学[がく]		
\\	法律学	
\\	法[ほう] 律[りつ] 学[がく]		
\\	経済学	
\\	経[けい] 済[ざい] 学[がく]		
\\	経営学	
\\	経[けい] 営[えい] 学[がく]		
\\	社会学	
\\	社[しゃ] 会[かい] 学[がく]		
\\	教育学	
\\	教[きょう] 育[いく] 学[がく]		
\\	文学	
\\	文[ぶん] 学[がく]		
\\	言語学	
\\	言[げん] 語[ご] 学[がく]		
\\	心理学	
\\	心[しん] 理[り] 学[がく]		
\\	哲学	
\\	哲[てつ] 学[がく]		
\\	宗教学	
\\	宗[しゅう] 教[きょう] 学[がく]		
\\	芸術	
\\	芸[げい] 術[じゅつ]		
\\	体育学	
\\	体[たい] 育[いく] 学[がく]		
\\	運動します	
\\	運[うん] 動[どう]します		
\\	成功します	
\\	成[せい] 功[こう]します		
\\	試験に失敗します	
\\	試[し] 験[けん]に 失[しっ] 敗[ぱい]します		
\\	試験に合格します	
\\	試[し] 験[けん]に 合[ごう] 格[かく]します		
\\	雨がやみます	
\\	雨[あめ]がやみます		
\\	晴れます	
\\	晴[は]れます		
\\	曇ります	
\\	曇[くも]ります		
\\	熱が続きます	
\\	熱[ねつ]が 続[つづ]きます		
\\	風邪をひきます	
\\	風[か] 邪[ぜ]をひきます		
\\	冷やします	
\\	冷[ひ]やします		
\\	道が込みます	
\\	道[みち]が 込[こ]みます		
\\	道がすきます	
\\	道[みち]がすきます		
\\	試合に出ます	
\\	試[し] 合[あい]に 出[で]ます		
\\	パーティーに出ます	
\\	パーティーに 出[で]ます		
\\	無理をします	
\\	無[む] 理[り]をします		
\\	十分(な)	
\\	十[じゅう]分[ぶん](な)		
\\	おかしい	
\\	おかしい		
\\	うるさい	
\\	うるさい		
\\	太陽	
\\	太[たい] 陽[よう]		
\\	星	
\\	星[ほし]		
\\	風	
\\	風[かぜ]		
\\	東	
\\	東[ひがし]		
\\	南	
\\	南[みなみ]		
\\	北	
\\	北[きた]		
\\	国際~	
\\	国[こく] 際[さい]~		
\\	水道	
\\	水[すい] 道[どう]		
\\	エンジン	
\\	エンジン		
\\	チーム	
\\	チーム		
\\	今夜	
\\	今[こん] 夜[や]		
\\	夕方	
\\	夕[ゆう] 方[がた]		
\\	まえ	
\\	まえ		
\\	遅く	
\\	遅[おそ]く		
\\	こんなに	
\\	こんなに		
\\	あんなに	
\\	あんなに		
\\	ヨーロッパ	
\\	ヨーロッパ		
\\	元気	
\\	元[げん] 気[き]		
\\	胃	
\\	胃[い]		
\\	ストレス	
\\	ストレス		
\\	それはいけませんね。	
\\	それはいけませんね。		
\\	星占い	
\\	星[ほし] 占[うらな]い		
\\	牡牛座	
\\	牡[お] 牛[うし] 座[ざ]		
\\	働きすぎ	
\\	働[はたら]きすぎ		
\\	困ります	
\\	困[こま]ります		
\\	宝くじが当たります	
\\	宝[たから]くじが 当[あ]たります		
\\	恋人	
\\	恋[こい] 人[びと]		
\\	ラッキーアイテム	
\\	ラッキーアイテム		
\\	石	
\\	石[いし]		
\\	晴れ	
\\	晴[は]れ		
\\	晴れのち曇り	
\\	晴[は]れのち 曇[くも]り		
\\	曇り時々雨	
\\	曇[くも]り 時[とき]々[どき] 雨[あめ]		
\\	曇り所によって雨	
\\	曇[くも]り 所[ところ]によって 雨[あめ]		
\\	にわか雨、夕立	
\\	にわか 雨[あめ]、 夕[ゆう]立[だち]		
\\	虹	
\\	虹[にじ]		
\\	雲	
\\	雲[くも]		
\\	湿度	
\\	湿[しつ] 度[ど]		
\\	蒸し暑い	
\\	蒸[む]し 暑[あつ]い		
\\	さわやか(な)	
\\	さわやか(な)		
\\	降水確率	
\\	降[こう] 水[すい] 確[かく] 率[りつ]		
\\	最高気温	
\\	最[さい] 高[こう] 気[き] 温[おん]		
\\	最低気温	
\\	最[さい] 低[てい] 気[き] 温[おん]		
\\	北海道地方	
\\	北[ほっ] 海[かい] 道[どう] 地[ち] 方[ほう]		
\\	札幌	
\\	札[さっ] 幌[ぽろ]		
\\	東北地方	
\\	東[とう] 北[ほく] 地[ち] 方[ほう]		
\\	仙台	
\\	仙[せん] 台[だい]		
\\	長野	
\\	長[なが] 野[の]		
\\	関東地方	
\\	関[かん] 東[とう] 地[ち] 方[ほう]		
\\	中部地方	
\\	中[ちゅう] 部[ぶ] 地[ち] 方[ほう]		
\\	近畿地方	
\\	近[きん] 畿[き] 地[ち] 方[ほう]		
\\	名古屋	
\\	名[な] 古[ご] 屋[や]		
\\	大阪	
\\	大[おお] 阪[さか]		
\\	四国地方	
\\	四[し] 国[こく] 地[ち] 方[ほう]		
\\	中国地方	
\\	中[ちゅう] 国[ごく] 地[ち] 方[ほう]		
\\	松江	
\\	松[まつ] 江[え]		
\\	高知	
\\	高[こう] 知[ち]		
\\	九州地方	
\\	九[きゅう] 州[しゅう] 地[ち] 方[ほう]		
\\	鹿児島	
\\	鹿[か] 児[ご] 島[しま]		
\\	那覇	
\\	那[な] 覇[は]		
\\	逃げます	
\\	逃[に]げます		
\\	騒ぎます	
\\	騒[さわ]ぎます		
\\	あきらめます	
\\	あきらめます		
\\	投げます	
\\	投[な]げます		
\\	守ります	
\\	守[まも]ります		
\\	式が始まります	
\\	式[しき]が 始[はじ]まります		
\\	会議に出席します	
\\	会[かい] 議[ぎ]に 出[しゅっ] 席[せき]します		
\\	伝えます	
\\	伝[つた]えます		
\\	車に注意します	
\\	車[くるま]に 注[ちゅう] 意[い]します		
\\	席を外します	
\\	席[せき]を 外[はず]します		
\\	戻ります	
\\	戻[もど]ります		
\\	電話があります	
\\	電[でん] 話[わ]があります		
\\	リサイクルします	
\\	リサイクルします		
\\	だめ(な)	
\\	だめ(な)		
\\	同じ	
\\	同[おな]じ		
\\	席	
\\	席[せき]		
\\	マーク	
\\	マーク		
\\	ボール	
\\	ボール		
\\	締め切り	
\\	締[し]め 切[き]り		
\\	規則	
\\	規[き] 則[そく]		
\\	危険	
\\	危[き] 険[けん]		
\\	使用禁止	
\\	使[し] 用[よう] 禁[きん] 止[し]		
\\	立入禁止	
\\	立[たち] 入[いり] 禁[きん] 止[し]		
\\	徐行	
\\	徐[じょ] 行[こう]		
\\	非常口	
\\	非[ひ] 常[じょう] 口[ぐち]		
\\	割引	
\\	割[わり] 引[びき]		
\\	飲み放題	
\\	飲[の]み 放[ほう] 題[だい]		
\\	使用中	
\\	使[し] 用[よう] 中[ちゅう]		
\\	募集中	
\\	募[ぼ] 集[しゅう] 中[ちゅう]		
\\	~中	
\\	中[ちゅう]		
\\	どういう~	
\\	どういう~		
\\	いくら(~ても)	
\\	いくら(~ても)		
\\	あと~	
\\	あと~		
\\	~ほど	
\\	~ほど		
\\	駐車違反	
\\	駐[ちゅう] 車[しゃ] 違[い] 反[はん]		
\\	罰金	
\\	罰[ばっ] 金[きん]		
\\	助け合います	
\\	助[たす]け 合[あ]います		
\\	もともと	
\\	もともと		
\\	悲しい	
\\	悲[かな]しい		
\\	もっと	
\\	もっと		
\\	あいさつをします	
\\	あいさつをします		
\\	相手	
\\	相[あい] 手[て]		
\\	気持ち	
\\	気[き] 持[も]ち		
\\	標識	
\\	標[ひょう] 識[しき]		
\\	営業中	
\\	営[えい] 業[ぎょう] 中[ちゅう]		
\\	準備中	
\\	準[じゅん] 備[び] 中[ちゅう]		
\\	閉店	
\\	閉[へい] 店[てん]		
\\	定休日	
\\	定[てい] 休[きゅう] 日[び]		
\\	化粧室	
\\	化[け] 粧[しょう] 室[しつ]		
\\	禁煙席	
\\	禁[きん] 煙[えん] 席[せき]		
\\	予約席	
\\	予[よ] 約[やく] 席[せき]		
\\	火気厳禁	
\\	火[か] 気[き] 厳[げん] 禁[きん]		
\\	割れ物注意	
\\	割[わ]れ 物[もの] 注[ちゅう] 意[い]		
\\	運転初心者注意	
\\	運[うん] 転[てん] 初[しょ] 心[しん] 者[しゃ] 注[ちゅう] 意[い]		
\\	工事中	
\\	工[こう] 事[じ] 中[ちゅう]		
\\	塩素系漂白剤不可	
\\	塩[えん] 素[そ] 系[けい] 漂[ひょう] 白[はく] 剤[ざい] 不[ふ] 可[か]		
\\	手洗い	
\\	手[て] 洗[あら]い		
\\	アイロン(低温)	
\\	アイロン
\\	低[てい]温[おん])		
\\	ドライクリーニング	
\\	ドライクリーニング		
\\	歯を磨きます	
\\	歯[は]を 磨[みが]きます		
\\	組み立てます	
\\	組[く]み 立[た]てます		
\\	折ります	
\\	折[お]ります		
\\	忘れ物に気がつきます	
\\	忘[わす]れ 物[もの]に 気[き]がつきます		
\\	しょうゆをつけます	
\\	しょうゆをつけます		
\\	鍵が見つかります	
\\	鍵[かぎ]が 見[み]つかります		
\\	質問します	
\\	質[しつ] 問[もん]します		
\\	傘をさします	
\\	傘[かさ]をさします		
\\	スポーツクラブ	
\\	スポーツクラブ		
\\	お城	
\\	お 城[しろ]		
\\	説明書	
\\	説[せつ] 明[めい] 書[しょ]		
\\	図	
\\	図[ず]		
\\	線	
\\	線[せん]		
\\	矢印	
\\	矢[や] 印[じるし]		
\\	黒	
\\	黒[くろ]		
\\	白	
\\	白[しろ]		
\\	赤	
\\	赤[あか]		
\\	青	
\\	青[あお]		
\\	紺	
\\	紺[こん]		
\\	黄色	
\\	黄[き] 色[いろ]		
\\	茶色	
\\	茶[ちゃ] 色[いろ]		
\\	しょうゆ	
\\	しょうゆ		
\\	ソース	
\\	ソース		
\\	お客さん	
\\	お 客[きゃく]さん		
\\	~か~	
\\	~か~		
\\	夕べ	
\\	夕[ゆう]べ		
\\	さっき	
\\	さっき		
\\	お茶をたてます	
\\	お 茶[ちゃ]をたてます		
\\	先に	
\\	先[さき]に		
\\	載せます	
\\	載[の]せます		
\\	これでいいですか。	
\\	これでいいですか。		
\\	いかがですか。	
\\	いかがですか。		
\\	材料	
\\	材[ざい] 料[りょう]		
\\	~分	
\\	分[ぶん]		
\\	ーグラム	ー 
\\	ーグラム		
\\	ー個	
\\	ー 個[こ]		
\\	4分の1
\\	分[ぶん]の1
\\	調味料	
\\	調[ちょう] 味[み] 料[りょう]		
\\	適当な大きさに	
\\	適[てき]当[とう]な 大[おお]きさに		
\\	なべ	
\\	なべ		
\\	火	
\\	火[ひ]		
\\	火にかけます	
\\	火[ひ]にかけます		
\\	煮ます	
\\	煮[に]ます		
\\	煮えます	
\\	煮[に]えます		
\\	どんぶり	
\\	どんぶり		
\\	たちます	
\\	たちます		
\\	料理	
\\	料[りょう] 理[り]		
\\	煮る	
\\	煮[に]る		
\\	焼く	
\\	焼[や]く		
\\	揚げる	
\\	揚[あ]げる		
\\	いためる	
\\	いためる		
\\	ゆでる	
\\	ゆでる		
\\	蒸す	
\\	蒸[む]す		
\\	炊く	
\\	炊[た]く		
\\	むく	
\\	むく		
\\	刻む	
\\	刻[きざ]む		
\\	かき混ぜる	
\\	かき 混[ま]ぜる		
\\	酢	
\\	酢[す]		
\\	みそ	
\\	みそ		
\\	油	
\\	油[あぶら]		
\\	マヨネーズ	
\\	マヨネーズ		
\\	ケチャップ	
\\	ケチャップ		
\\	からし(マスタード)	
\\	からし(マスタード)		
\\	こしょう	
\\	こしょう		
\\	とうがらし	
\\	とうがらし		
\\	しょうが	
\\	しょうが		
\\	わさび	
\\	わさび		
\\	カレー粉	
\\	カレー 粉[こ]		
\\	大所用品	
\\	大[だい] 所[どころ] 用[よう] 品[ひん]		
\\	やかん	
\\	やかん		
\\	ふた	
\\	ふた		
\\	おたま	
\\	おたま		
\\	まな板	
\\	まな 板[いた]		
\\	包丁	
\\	包[ほう] 丁[ちょう]		
\\	ふきん	
\\	ふきん		
\\	フライパン	
\\	フライパン		
\\	電子レンジ	
\\	電[でん] 子[し]レンジ		
\\	炊飯器	
\\	炊[すい] 飯[はん] 器[き]		
\\	しゃもじ	
\\	しゃもじ		
\\	缶切り	
\\	缶[かん] 切[き]り		
\\	栓抜き	
\\	栓[せん] 抜[ぬ]き		
\\	ざる	
\\	ざる		
\\	ポット	
\\	ポット		
\\	ガス台	
\\	ガス 台[だい]		
\\	流し台	
\\	流[なが]し 台[だい]		
\\	換気扇	
\\	換[かん] 気[き] 扇[せん]		
\\	花が咲きます	
\\	花[はな]が 咲[さ]きます		
\\	色が変わります	
\\	色[いろ]が 変[か]わります		
\\	丸を付けます	
\\	丸[まる]を 付[つ]けます		
\\	病気が治ります	
\\	病[びょう] 気[き]が 治[なお]ります		
\\	故障が直ります	
\\	故[こ] 障[しょう]が 直[なお]ります		
\\	クリックします	
\\	クリックします		
\\	入力します	
\\	入[にゅう] 力[りょく]します		
\\	正しい	
\\	正[ただ]しい		
\\	向こう	
\\	向[む]こう		
\\	島	
\\	島[しま]		
\\	港	
\\	港[みなと]		
\\	近所	
\\	近[きん] 所[じょ]		
\\	海外	
\\	海[かい] 外[がい]		
\\	山登り	
\\	山[やま] 登[のぼ]り		
\\	歴史	
\\	歴[れき] 史[し]		
\\	機会	
\\	機[き] 会[かい]		
\\	許可	
\\	許[きょ] 可[か]		
\\	円、丸	
\\	円[えん]、 丸[まる]		
\\	ふりがな	
\\	ふりがな		
\\	設備	
\\	設[せつ] 備[び]		
\\	レバー	
\\	レバー		
\\	キー	
\\	キー		
\\	カーテン	
\\	カーテン		
\\	ひも	
\\	ひも		
\\	葉	
\\	葉[は]		
\\	これで終わりましょう。	
\\	これで 終[お]わりましょう。		
\\	それなら	
\\	それなら		
\\	夜行バス	
\\	夜[や] 行[こう]バス		
\\	旅行社	
\\	旅[りょ] 行[こう] 社[しゃ]		
\\	詳しい	
\\	詳[くわ]しい		
\\	スキー場	
\\	スキー 場[じょう]		
\\	朱	
\\	朱[しゅ]		
\\	交わります	
\\	交[まじ]わります		
\\	ことわざ	
\\	ことわざ		
\\	関係	
\\	関[かん] 係[けい]		
\\	仲よくします	
\\	仲[なか]よくします		
\\	必要(な)	
\\	必[ひつ]要[よう](な)		
\\	都	
\\	都[みやこ]		
\\	知恵	
\\	知[ち] 恵[え]		
\\	姿	
\\	姿[すがた]		
\\	積もる	
\\	積[つ]もる		
\\	影	
\\	影[かげ]		
\\	事故にあいます	
\\	事[じ] 故[こ]にあいます		
\\	貯金します	
\\	貯[ちょ] 金[きん]します		
\\	7時を過ぎます	
\\	時[じ]を 過[す]ぎます		
\\	仕事に慣れます	
\\	仕[し] 事[ごと]に 慣[な]れます		
\\	食べ物が腐ります	
\\	食[た]べ 物[もの]が 腐[くさ]ります		
\\	ラッシュ	
\\	ラッシュ		
\\	曲	
\\	曲[きょく]		
\\	毎週	
\\	毎[まい] 週[しゅう]		
\\	毎月	
\\	毎[まい] 月[つき]		
\\	毎年、毎年	
\\	毎[まい] 年[とし]、 毎[まい] 年[ねん]		
\\	このごろ	
\\	このごろ		
\\	やっと	
\\	やっと		
\\	かなり	
\\	かなり		
\\	絶対に	
\\	絶[ぜっ] 対[たい]に		
\\	上手に	
\\	上[じょう] 手[ず]に		
\\	できるだけ	
\\	できるだけ		
\\	ほとんど	
\\	ほとんど		
\\	特別(な)	
\\	特[とく]別[べつ](な)		
\\	していらっしゃいます	
\\	しています)	していらっしゃいます		
\\	使っていらっしゃるんですね。	
\\	つかっているんですね)	使[つか]っていらっしゃるんですね。		
\\	チャレンジします	
\\	チャレンジします		
\\	乗り物	
\\	乗[の]り 物[もの]		
\\	ー世紀	
\\	ー 世[せい] 紀[き]		
\\	遠く	
\\	遠[とお]く		
\\	珍しい	
\\	珍[めずら]しい		
\\	汽車	
\\	汽[き] 車[しゃ]		
\\	汽船	
\\	汽[き] 船[せん]		
\\	大勢の~	
\\	大[おお] 勢[ぜい]の~		
\\	運びます	
\\	運[はこ]びます		
\\	利用します	
\\	利[り] 用[よう]します		
\\	栄養	
\\	栄[えい] 養[よう]		
\\	健康診断	
\\	健[けん] 康[こう] 診[しん] 断[だん]		
\\	夜更かし	
\\	夜[よ] 更[ふ]かし		
\\	栄養素	
\\	栄[えい] 養[よう] 素[そ]		
\\	含む	
\\	含[ふく]む		
\\	炭水化物	
\\	炭[たん] 水[すい] 化[か] 物[ぶつ]		
\\	芋	
\\	芋[いも]		
\\	のり	
\\	のり		
\\	カルシウム	
\\	カルシウム		
\\	海草	
\\	海[かい] 草[そう]		
\\	とうふ	
\\	とうふ		
\\	たんぽく質	
\\	たんぽく 質[しつ]		
\\	脂肪	
\\	脂[し] 肪[ぼう]		
\\	ビタミン	
\\	ビタミン		
\\	褒めます	
\\	褒[ほ]めます		
\\	しかります	
\\	しかります		
\\	招待します	
\\	招[しょう] 待[たい]します		
\\	頼みます	
\\	頼[たの]みます		
\\	注意します	
\\	注[ちゅう] 意[い]します		
\\	盗ります	
\\	盗[と]ります		
\\	踏みます	
\\	踏[ふ]みます		
\\	壊します	
\\	壊[こわ]します		
\\	汚します	
\\	汚[よご]します		
\\	行います	
\\	行[おこな]います		
\\	輸出します	
\\	輸[ゆ] 出[しゅつ]します		
\\	輸入します	
\\	輸[ゆ] 入[にゅう]します		
\\	翻訳します	
\\	翻[ほん] 訳[やく]します		
\\	発明します	
\\	発[はつ] 明[めい]します		
\\	発見します	
\\	発[はっ] 見[けん]します		
\\	麦	
\\	麦[むぎ]		
\\	石油	
\\	石[せき] 油[ゆ]		
\\	原料	
\\	原[げん] 料[りょう]		
\\	インスタントラーメン	
\\	インスタントラーメン		
\\	デート	
\\	デート		
\\	泥棒	
\\	泥[どろ] 棒[ぼう]		
\\	警官	
\\	警[けい] 官[かん]		
\\	世界中	
\\	世[せ] 界[かい] 中[じゅう]		
\\	何語	
\\	何[なに] 語[ご]		
\\	誰か	
\\	誰[だれ]か		
\\	よかったですね。	
\\	よかったですね。		
\\	オリンピック	
\\	オリンピック		
\\	ワールドカップ	
\\	ワールドカップ		
\\	大仏	
\\	大[だい] 仏[ぶつ]		
\\	江戸時代	
\\	江[え] 戸[ど] 時[じ] 代[だい]		
\\	ポルトガル	
\\	ポルトガル		
\\	皆様	
\\	みなさん)	皆[みな] 様[さま]		
\\	うちが焼けます	
\\	うちが 焼[や]けます		
\\	その後	
\\	その 後[ご]		
\\	世界遺産	
\\	世[せ] 界[かい] 遺[い] 産[さん]		
\\	~の一つ	
\\	~の 一[ひと]つ		
\\	金色	
\\	金[きん] 色[いろ]		
\\	本物	
\\	本[ほん] 物[もの]		
\\	キロ	
\\	キロ		
\\	美しい	
\\	美[うつく]しい		
\\	豪華(な)	
\\	豪[ごう] 華[か](な)		
\\	彫刻	
\\	彫[ちょう] 刻[こく]		
\\	言い伝え	
\\	言[い]い 伝[つた]え		
\\	眠ります	
\\	眠[ねむ]ります		
\\	彫ります	
\\	彫[ほ]ります		
\\	仲間	
\\	仲[なか] 間[ま]		
\\	しかし	
\\	しかし		
\\	そのあと	
\\	そのあと		
\\	一所懸命	
\\	一[いっ] 所[しょう] 懸[けん] 命[めい]		
\\	ねずみ	
\\	ねずみ		
\\	一匹もいません。	
\\	一[いっ] 匹[ぴき]もいません。		
\\	事故	
\\	事[じ] 故[こ]		
\\	事件	
\\	事[じ] 件[けん]		
\\	殺す	
\\	殺[ころ]す		
\\	撃つ	
\\	撃[う]つ		
\\	刺す	
\\	刺[さ]す		
\\	噛む	
\\	噛[か]む		
\\	轢く	
\\	轢[ひ]く		
\\	跳ねる	
\\	跳[は]ねる		
\\	衝突する	
\\	衝[しょう] 突[とつ]する		
\\	追突する	
\\	追[つい] 突[とつ]する		
\\	盗む	
\\	盗[ぬす]む		
\\	誘拐する	
\\	誘[ゆう] 拐[かい]する		
\\	ハイジャックする	
\\	ハイジャックする		
\\	墜落する	
\\	墜[つい] 落[らく]する		
\\	運ぶ	
\\	運[はこ]ぶ		
\\	爆発する	
\\	爆[ばく] 発[はつ]する		
\\	助ける	
\\	助[たす]ける		
\\	沈没する	
\\	沈[ちん] 没[ぼつ]する		
\\	旅行に参加します	
\\	旅[りょ] 行[こう]に 参[さん] 加[か]します		
\\	育てます	
\\	育[そだ]てます		
\\	入院します	
\\	入[にゅう] 院[いん]します		
\\	退院します	
\\	退[たい] 院[いん]します		
\\	電源を入れます	
\\	電[でん] 源[げん]を 入[い]れます		
\\	電源を切ります	
\\	電[でん] 源[げん]を 切[き]ります		
\\	鍵を掛けます	
\\	鍵[かぎ]を 掛[か]けます		
\\	嘘をつきます	
\\	嘘[うそ]をつきます		
\\	気持ちがいい	
\\	気[き] 持[も]ちがいい		
\\	気持ちが悪い	
\\	気[き] 持[も]ちが 悪[わる]い		
\\	大きな~	
\\	大[おお]きな~		
\\	小さな~	
\\	小[ちい]さな~		
\\	海岸	
\\	海[かい] 岸[がん]		
\\	工場	
\\	工[こう] 場[じょう]		
\\	かな	
\\	かな		
\\	指輪	
\\	指[ゆび] 輪[わ]		
\\	電源	
\\	電[でん] 源[げん]		
\\	~製	
\\	製[せい]		
\\	おととし	
\\	おととし		
\\	あ、いけない。	
\\	あ、いけない。		
\\	お先に失礼します。	
\\	お 先[さき]に 失[しつ] 礼[れい]します。		
\\	出雲大社	
\\	出[い] 雲[ずも]大[たい]社[しゃ]		
\\	チェンマイ	
\\	チェンマイ		
\\	回覧	
\\	回[かい] 覧[らん]		
\\	研究室	
\\	研[けん] 究[きゅう] 室[しつ]		
\\	きちんと	
\\	きちんと		
\\	整理します	
\\	整[せい] 理[り]します		
\\	方法	
\\	方[ほう] 法[ほう]		
\\	~という	
\\	~という		
\\	ー冊	
\\	ー 冊[さつ]		
\\	はんこ	
\\	はんこ		
\\	はんこを押します	
\\	はんこを 押[お]します		
\\	双子	
\\	双[ふた] 子[ご]		
\\	姉妹	
\\	姉[し] 妹[まい]		
\\	5年生	
\\	年[ねん] 生[せい]		
\\	似ています	
\\	似[に]ています		
\\	性格	
\\	性[せい] 格[かく]		
\\	おとなしい	
\\	おとなしい		
\\	優しい	
\\	優[やさ]しい		
\\	世話をします	
\\	世[せ] 話[わ]をします		
\\	時間がたちます	
\\	時[じ] 間[かん]がたちます		
\\	ー点	ー 
\\	ー 点[てん]		
\\	気が強い	
\\	気[き]が 強[つよ]い		
\\	喧嘩します	
\\	喧[けん] 嘩[か]します		
\\	年齢	
\\	年[ねん] 齢[れい]		
\\	しかた	
\\	しかた		
\\	位置	
\\	位[い] 置[ち]		
\\	上から2段目	
\\	上[うえ]から2 段[だん] 目[め]		
\\	奥	
\\	奥[おく]		
\\	手前	
\\	手[て] 前[まえ]		
\\	前から2列目	
\\	前[まえ]から2 列[れつ] 目[め]		
\\	机の周り	
\\	机[つくえ]の 周[まわ]り		
\\	2行目	
\\	2行[ぎょう] 目[め]		
\\	4ページ	
\\	4ページ		
\\	3行目	
\\	行[ぎょう] 目[め]		
\\	本のそば	
\\	本[ほん]のそば		
\\	教室の真ん中	
\\	教[きょう] 室[しつ]の 真[ま]ん 中[なか]		
\\	斜め後ろ	
\\	斜[なな]め 後[うし]ろ		
\\	斜め前	
\\	斜[なな]め 前[まえ]		
\\	テレビの横	
\\	テレビの 横[よこ]		
\\	質問に答えます	
\\	質[しつ] 問[もん]に 答[こた]えます		
\\	ビルが倒れます	
\\	ビルが 倒[たお]れます		
\\	道を通ります	
\\	道[みち]を 通[とお]ります		
\\	びっくりします	
\\	びっくりします		
\\	がっかりします	
\\	がっかりします		
\\	安心します	
\\	安[あん] 心[しん]します		
\\	太ります	
\\	太[ふと]ります		
\\	やせます	
\\	やせます		
\\	複雑(な)	
\\	複[ふく] 雑[ざつ](な)		
\\	邪魔(な)	
\\	邪[じゃ] 魔[ま](な)		
\\	硬い	
\\	硬[かた]い		
\\	軟らかい	
\\	軟[やわ]らかい		
\\	汚い	
\\	汚[きたな]い		
\\	恥ずかしい	
\\	恥[は]ずかしい		
\\	雷	
\\	雷[かみなり]		
\\	火事	
\\	火[か] 事[じ]		
\\	ハイキング	
\\	ハイキング		
\\	お見合い	
\\	お 見[み]合[あ]い		
\\	操作します	
\\	操[そう] 作[さ]します		
\\	会場	
\\	会[かい] 場[じょう]		
\\	~代	
\\	代[だい]		
\\	フロント	
\\	フロント		
\\	ー号室	
\\	ー	ー 号[ごう] 室[しつ]		
\\	タオル	
\\	タオル		
\\	石鹸	
\\	石[せっ] 鹸[けん]		
\\	お疲れさまでした。	
\\	お 疲[つか]れさまでした。		
\\	伺います	
\\	ききます、たずねます)	伺[うかが]います		
\\	途中で	
\\	途[と] 中[ちゅう]で		
\\	トラック	
\\	トラック		
\\	打つかります	
\\	打[ぶ]つかります		
\\	大人	
\\	大人[おとな]		
\\	洋服	
\\	洋[よう] 服[ふく]		
\\	西洋化します	
\\	西[せい] 洋[よう] 化[か]します		
\\	合います	
\\	合[あ]います		
\\	今では	
\\	今[いま]では		
\\	成人式	
\\	成[せい] 人[じん] 式[しき]		
\\	伝統的(な)	
\\	伝[でん] 統[とう] 的[てき](な)		
\\	羨ましい	
\\	羨[うらや]ましい		
\\	懐かしい	
\\	懐[なつ]かしい		
\\	うっとりする	
\\	うっとりする		
\\	いらいらする	
\\	いらいらする		
\\	どきどきする	
\\	どきどきする		
\\	はらはらする	
\\	はらはらする		
\\	わくわくする	
\\	わくわくする		
\\	数える	
\\	数[かぞ]える		
\\	測ります、量ります	
\\	測[はか]ります、 量[はか]ります		
\\	確かめます	
\\	確[たし]かめます		
\\	サイズが合います	
\\	サイズが 合[あ]います		
\\	出発します	
\\	出[しゅっ] 発[ぱつ]します		
\\	到着します	
\\	到[とう] 着[ちゃく]します		
\\	酔います	
\\	酔[よ]います		
\\	上手くいきます	
\\	上手[うま]くいきます		
\\	問題が出ます	
\\	問[もん] 題[だい]が 出[で]ます		
\\	相談します	
\\	相[そう] 談[だん]します		
\\	忘年会	
\\	忘[ぼう] 年[ねん] 会[かい]		
\\	新年会	
\\	新[しん] 年[ねん] 会[かい]		
\\	二次会	
\\	二[に] 次[じ] 会[かい]		
\\	発表会	
\\	発[はっ] 表[ぴょう] 会[かい]		
\\	大会	
\\	大[たい] 会[かい]		
\\	マラソン	
\\	マラソン		
\\	コンテスト	
\\	コンテスト		
\\	表	
\\	表[おもて]		
\\	裏	
\\	裏[うら]		
\\	間違い	
\\	間[ま] 違[ちが]い		
\\	傷	
\\	傷[きず]		
\\	お年寄り	
\\	お 年[とし] 寄[よ]り		
\\	長さ	
\\	長[なが]さ		
\\	重さ	
\\	重[おも]さ		
\\	高さ	
\\	高[たか]さ		
\\	大きさ	
\\	大[おお]きさ		
\\	[ー]便	
\\	[ー] 便[びん]		
\\	ー本	
\\	ー 本[ほん](ーぽん、ーぼん)		
\\	ー杯	ー 
\\	ー 杯[はい](ーぱい、-ばい)		
\\	ーセンチ	ー 
\\	ーセンチ		
\\	ーミリ	ー 
\\	ーミリ		
\\	~以上	
\\	以[い] 上[じょう]		
\\	~以下	
\\	以[い] 下[か]		
\\	長崎	
\\	長[なが] 崎[さき]		
\\	七夕祭り	
\\	(7月7日)	七夕[たなばた] 祭[まつ]り		
\\	どうでしょうか。	
\\	どうですか)	どうでしょうか。		
\\	テスト	
\\	テスト		
\\	成績	
\\	成[せい] 績[せき]		
\\	ところで	
\\	ところで		
\\	オートバイ	
\\	オートバイ		
\\	爆弾	
\\	爆[ばく] 弾[だん]		
\\	積みます	
\\	積[つ]みます		
\\	離れた	
\\	離[はな]れた		
\\	急に	
\\	急[きゅう]に		
\\	動かします	
\\	動[うご]かします		
\\	犯人	
\\	犯[はん] 人[にん]		
\\	男	
\\	男[おとこ]		
\\	手に入れます	
\\	手[て]に 入[い]れます		
\\	今でも	
\\	今[いま]でも		
\\	単位	
\\	単[たん] 位[い]		
\\	模様	
\\	模[も] 様[よう]		
\\	面積	
\\	面[めん] 積[せき]		
\\	平方センチメートル	
\\	平[へい] 方[ほう]センチメートル		
\\	平方メートル	
\\	平[へい] 方[ほう]メートル		
\\	平方キロメートル	
\\	平[へい] 方[ほう]キロメートル		
\\	ミリ	
\\	ミリ		
\\	センチ	
\\	センチ		
\\	メートル	
\\	メートル		
\\	キロ
\\	キロ
\\	体積	
\\	体[たい] 積[せき]		
\\	容積	
\\	容[よう] 積[せき]		
\\	立方センチメートル	
\\	立[りっ] 方[ぽう]センチメートル		
\\	立方メートル	
\\	立[りっ] 方[ぽう]メートル		
\\	ミリリットル	
\\	ミリリットル		
\\	シーシー	
\\	シーシー		
\\	リットル	
\\	リットル		
\\	ミリグラム	
\\	ミリグラム		
\\	グラム	
\\	グラム		
\\	トン	
\\	トン		
\\	計算	
\\	計[けい] 算[さん]		
\\	たす	
\\	たす		
\\	ひく	
\\	ひく		
\\	かける	
\\	かける		
\\	わる	
\\	わる		
\\	イコール	
\\	イコール		
\\	直線	
\\	直[ちょく] 線[せん]		
\\	曲線	
\\	曲[きょく] 線[せん]		
\\	点線	
\\	点[てん] 線[せん]		
\\	三角形	
\\	三[さん] 角[かく] 形[けい]		
\\	四角形	
\\	四[し] 角[かく] 形[けい]		
\\	縦じま	
\\	縦[たて]じま		
\\	横縞	
\\	横[よこ] 縞[じま]		
\\	チェック	
\\	チェック		
\\	水玉	
\\	水[みず] 玉[たま]		
\\	花柄	
\\	花[はな] 柄[がら]		
\\	無地	
\\	無[む] 地[じ]		
\\	くださいます	
\\	くれます)	くださいます		
\\	下げます	
\\	下[さ]げます		
\\	親切にします	
\\	親[しん] 切[せつ]にします		
\\	可愛い	
\\	可愛[かわい]い		
\\	お祝いをします	
\\	お 祝[いわ]いをします		
\\	興味	
\\	興[きょう] 味[み]		
\\	コンピューターに興味があります	
\\	コンピューターに 興[きょう] 味[み]があります		
\\	文法	
\\	文[ぶん] 法[ぽう]		
\\	発音	
\\	発[はつ] 音[おん]		
\\	猿	
\\	猿[さる]		
\\	えさ	
\\	えさ		
\\	絵本	
\\	絵[え] 本[ほん]		
\\	絵はがき	
\\	絵[え]はがき		
\\	ドライバー	
\\	ドライバー		
\\	ハンカチ	
\\	ハンカチ		
\\	暖房	
\\	暖[だん] 房[ぼう]		
\\	冷房	
\\	冷[れい] 房[ぼう]		
\\	温度	
\\	温[おん] 度[ど]		
\\	孫	
\\	孫[まご]		
\\	お孫さん	
\\	お 孫[まご]さん		
\\	おじ	
\\	おじ		
\\	おじさん	
\\	おじさん		
\\	おば	
\\	おば		
\\	おばさん	
\\	おばさん		
\\	管理人	
\\	さん 
\\	管[かん] 理[り] 人[にん]		
\\	この間	
\\	この 間[あいだ]		
\\	ひとこと	
\\	ひとこと		
\\	~ずつ	
\\	~ずつ		
\\	お宅	
\\	(=うち/いえ)	お 宅[たく]		
\\	どうぞお幸せに。	
\\	どうぞお 幸[しあわ]せに。		
\\	昔話	
\\	昔[むかし] 話[ばなし]		
\\	苛めます	
\\	苛[いじ]めます		
\\	かめ	
\\	かめ		
\\	助けます	
\\	助[たす]けます		
\\	お姫様	
\\	お 姫[ひめ] 様[さま]		
\\	暮らします	
\\	暮[く]らします		
\\	陸	
\\	陸[りく]		
\\	煙	
\\	煙[けむり]		
\\	真っ白(な)	
\\	真[ま]っ 白[しろ](な)		
\\	中身	
\\	中[なか] 身[み]		
\\	衣装	
\\	衣[い] 装[しょう]		
\\	晴れ着	
\\	晴[は]れ 着[ぎ]		
\\	新作	
\\	新[しん] 作[さく]		
\\	七五三	
\\	七[しち] 五[ご] 三[さん]		
\\	卒業式	
\\	卒[そつ] 業[ぎょう] 式[しき]		
\\	民宿	
\\	民[みん] 宿[しゅく]		
\\	家庭的	
\\	家[か] 庭[てい] 的[てき]		
\\	公民館	
\\	公[こう] 民[みん] 館[かん]		
\\	講習会	
\\	講[こう] 習[しゅう] 会[かい]		
\\	レンタル	
\\	レンタル		
\\	ビデオカメラ	
\\	ビデオカメラ		
\\	携帯電話	
\\	携[けい] 帯[たい] 電[でん] 話[わ]		
\\	ベビー用品	
\\	ベビー 用[よう] 品[ひん]		
\\	レジャー用品	
\\	レジャー 用[よう] 品[ひん]		
\\	話し相手	
\\	話[はな]し 相[あい] 手[て]		
\\	体験	
\\	体[たい] 験[けん]		
\\	禅	
\\	禅[ぜん]		
\\	精進料理	
\\	精[しょう] 進[じん] 料[りょう] 理[り]		
\\	包みます	
\\	包[つつ]みます		
\\	沸かします	
\\	沸[わ]かします		
\\	計算します	
\\	計[けい] 算[さん]します		
\\	並びます	
\\	並[なら]びます		
\\	丈夫(な)	
\\	丈[じょう] 夫[ぶ](な)		
\\	自然	
\\	自[し] 然[ぜん]		
\\	教育	
\\	教[きょう] 育[いく]		
\\	文化	
\\	文[ぶん] 化[か]		
\\	社会	
\\	社[しゃ] 会[かい]		
\\	政治	
\\	政[せい] 治[じ]		
\\	法律	
\\	法[ほう] 律[りつ]		
\\	平和	
\\	平[へい] 和[わ]		
\\	目的	
\\	目[もく] 的[てき]		
\\	論文	
\\	論[ろん] 文[ぶん]		
\\	楽しみ	
\\	楽[たの]しみ		
\\	ミキサー	
\\	ミキサー		
\\	風呂敷	
\\	風[ふ] 呂[ろ] 敷[しき]		
\\	そろばん	
\\	そろばん		
\\	体温計	
\\	体[たい] 温[おん] 計[けい]		
\\	なぜ	
\\	なぜ		
\\	国連	
\\	国[こく] 連[れん]		
\\	ボーナスが出ます	
\\	ボーナスが 出[で]ます		
\\	半分	
\\	半[はん] 分[ぶん]		
\\	ローン	
\\	ローン		
\\	カップ麺	
\\	カップ 麺[めん]		
\\	世界初	
\\	世[せ] 界[かい] 初[はつ]		
\\	~によって	
\\	~によって		
\\	麺	
\\	麺[めん]		
\\	広めます	
\\	広[ひろ]めます		
\\	市場調査	
\\	市[し] 場[じょう] 調[ちょう] 査[さ]		
\\	割ります	
\\	割[わ]ります		
\\	注ぎます	
\\	注[そそ]ぎます		
\\	事務用品	
\\	事[じ] 務[む] 用[よう] 品[ひん]		
\\	綴じる	
\\	綴[と]じる		
\\	クリップ	
\\	クリップ		
\\	挟む	
\\	挟[はさ]む		
\\	画鋲	
\\	画[が] 鋲[びょう]		
\\	留める	
\\	留[と]める		
\\	カッター	
\\	カッター		
\\	切る	
\\	切[き]る		
\\	ガムテープ	
\\	ガムテープ		
\\	糊	
\\	糊[のり]		
\\	鉛筆削り	
\\	鉛[えん] 筆[ぴつ] 削[けず]り		
\\	削る	
\\	削[けず]る		
\\	ファイル	
\\	ファイル		
\\	ファイルする	
\\	ファイルする		
\\	修正液	
\\	修[しゅう] 正[せい] 液[えき]		
\\	消す	
\\	消[け]す		
\\	穴を開ける	
\\	穴[あな]を 開[あ]ける		
\\	電卓	
\\	電[でん] 卓[たく]		
\\	計算する	
\\	計[けい] 算[さん]する		
\\	定規(物差し)	
\\	定[じょう] 規[ぎ]
\\	物[もの]差[さ]し)		
\\	線を引く	
\\	線[せん]を 引[ひ]く		
\\	線を測る	
\\	線[せん]を 測[はか]る		
\\	のこぎり	
\\	のこぎり		
\\	金槌	
\\	金[かな] 槌[づち]		
\\	くぎを打つ	
\\	くぎを 打[う]つ		
\\	ペンチ	
\\	ペンチ		
\\	ねじを締める	
\\	ねじを 締[し]める		
\\	ねじを緩める	
\\	ねじを 緩[ゆる]める		
\\	浴室	
\\	浴[よく] 室[しつ]		
\\	洋室	
\\	洋[よう] 室[しつ]		
\\	物入れ	
\\	物[もの] 入[い]れ		
\\	輸出が増えます	
\\	輸[ゆ] 出[しゅつ]が 増[ふ]えます		
\\	輸出が減ります	
\\	輸[ゆ] 出[しゅつ]が 減[へ]ります		
\\	値段が上がります	
\\	値[ね] 段[だん]が 上[あ]がります		
\\	値段が下がります	
\\	値[ね] 段[だん]が 下[さ]がります		
\\	ひもが切れます	
\\	ひもが 切[き]れます		
\\	ボタンが取れます	
\\	ボタンが 取[と]れます		
\\	荷物が落ちます	
\\	荷[に] 物[もつ]が 落[お]ちます		
\\	ガソリンがなくなります	
\\	ガソリンがなくなります		
\\	変(な)	
\\	変[へん](な)		
\\	幸せ(な)	
\\	幸[しあわ]せ(な)		
\\	楽(な)	
\\	楽[らく](な)		
\\	うまい	
\\	うまい		
\\	まずい	
\\	まずい		
\\	つまらない	
\\	つまらない		
\\	パンフレット	
\\	パンフレット		
\\	今にも	
\\	今[いま]にも		
\\	わあ	
\\	わあ		
\\	ばら	
\\	ばら		
\\	ドライブ	
\\	ドライブ		
\\	理由	
\\	理[り] 由[ゆう]		
\\	謝ります	
\\	謝[あやま]ります		
\\	知り合います	
\\	知[し]り 合[あ]います		
\\	性質	
\\	性[せい] 質[しつ]		
\\	厳しい	
\\	厳[きび]しい		
\\	気が長い	
\\	気[き]が 長[なが]い		
\\	気が短い	
\\	気[き]が 短[みじか]い		
\\	気が弱い	
\\	気[き]が 弱[よわ]い		
\\	活発(な)	
\\	活[かっ] 発[ぱつ](な)		
\\	誠実(な)	
\\	誠[せい] 実[じつ](な)		
\\	わがまま(な)	
\\	わがまま(な)		
\\	ふまじめ(な)	
\\	ふまじめ(な)		
\\	頑固(な)	
\\	頑[がん] 固[こ](な)		
\\	素直(な)	
\\	素[す] 直[なお](な)		
\\	意地悪(な)	
\\	意[い] 地[じ] 悪[わる](な)		
\\	勝ち気(な)	
\\	勝[か]ち 気[き](な)		
\\	神経質(な)	
\\	神[しん] 経[けい] 質[しつ](な)		
\\	泣きます	
\\	泣[な]きます		
\\	笑います	
\\	笑[わら]います		
\\	シャツが乾きます	
\\	シャツが 乾[かわ]きます		
\\	シャツが濡れます	
\\	シャツが 濡[ぬ]れます		
\\	滑ります	
\\	滑[すべ]ります		
\\	事故が起きます	
\\	事[じ] 故[こ]が 起[お]きます		
\\	調節します	
\\	調[ちょう] 節[せつ]します		
\\	危険(な)	
\\	危[き] 険[けん](な)		
\\	厚い	
\\	厚[あつ]い		
\\	太い	
\\	太[ふと]い		
\\	細い	
\\	細[ほそ]い		
\\	空気	
\\	空[くう] 気[き]		
\\	涙	
\\	涙[なみだ]		
\\	和食	
\\	和[わ] 食[しょく]		
\\	洋食	
\\	洋[よう] 食[しょく]		
\\	おかず	
\\	おかず		
\\	量	
\\	量[りょう]		
\\	ー倍	ー 
\\	ー 倍[ばい]		
\\	シングル	
\\	シングル		
\\	ツイン	
\\	ツイン		
\\	洗濯物	
\\	洗[せん] 濯[たく] 物[もの]		
\\	どうなさいますか。	
\\	どうなさいますか。		
\\	シャンプーをします	
\\	シャンプーをします		
\\	どういうふうになさいますか。	
\\	どういうふうになさいますか。		
\\	ショート	
\\	ショート		
\\	~みたいにしてください。	
\\	~みたいにしてください。		
\\	これでよろしいでしょうか。	
\\	これでよろしいでしょうか。		
\\	どうもお疲れさまでした。	
\\	どうもお 疲[つか]れさまでした。		
\\	嫌がります	
\\	嫌[いや]がります		
\\	上手く	
\\	上手[うま]く		
\\	順序	
\\	順[じゅん] 序[じょ]		
\\	安心(な)	
\\	安[あん] 心[しん](な)		
\\	表現	
\\	表[ひょう] 現[げん]		
\\	別れます	
\\	別[わか]れます		
\\	これら	
\\	これら		
\\	縁起が悪い	
\\	縁[えん] 起[ぎ]が 悪[わる]い		
\\	理髪店	
\\	理[り] 髪[はつ] 店[てん]		
\\	パーム	
\\	パーム		
\\	シャンプー	
\\	シャンプー		
\\	トリートメント	
\\	トリートメント		
\\	ブロー	
\\	ブロー		
\\	カラー	
\\	カラー		
\\	エクステ	
\\	エクステ		
\\	ネイル	
\\	ネイル		
\\	フェイシャルマッサージ	
\\	フェイシャルマッサージ		
\\	メイク	
\\	メイク		
\\	着付け	
\\	着[き] 付[つ]け		
\\	切ってください。	
\\	切[き]ってください。		
\\	耳が見えるくらいに~	
\\	耳[みみ]が 見[み]えるくらいに~		
\\	肩にかかるくらいに~	
\\	肩[かた]にかかるくらいに~		
\\	眉が隠れるくらいに~	
\\	眉[まゆ]が 隠[かく]れるくらいに~		
\\	この写真みたいに~	
\\	この 写[しゃ] 真[しん]みたいに~		
\\	髪をとかす	
\\	髪[かみ]をとかす		
\\	髪を分ける	
\\	髪[かみ]を 分[わ]ける		
\\	髪をまとめる	
\\	髪[かみ]をまとめる		
\\	髪をアップにする	
\\	髪[かみ]をアップにする		
\\	髪を染める	
\\	髪[かみ]を 染[そ]める		
\\	ひげ/顔をそる	
\\	ひげ/ 顔[かお]をそる		
\\	化粧	
\\	化[け] 粧[しょう](メイクする)		
\\	三つ編みにする	
\\	三[み]つ 編[あ]みにする		
\\	刈り上げる	
\\	刈[か]り 上[あ]げる		
\\	パーマをかける	
\\	パーマをかける		
\\	信じます	
\\	信[しん]じます		
\\	キャンセルします	
\\	キャンセルします		
\\	知らせます	
\\	知[し]らせます		
\\	保証書	
\\	保[ほ] 証[しょう] 書[しょ]		
\\	領収書	
\\	領[りょう] 収[しゅう] 書[しょ]		
\\	中止	
\\	中[ちゅう] 止[し]		
\\	点	
\\	点[てん]		
\\	梅	
\\	梅[うめ]		
\\	110番	
\\	番[ばん]		
\\	119番	
\\	番[ばん]		
\\	無理に	
\\	無[む]理[り]に		
\\	楽しみにしています	
\\	楽[たの]しみにしています		
\\	以上です。	
\\	以[い] 上[じょう]です。		
\\	係員	
\\	係[かかり] 員[いん]		
\\	コース	
\\	コース		
\\	スタート	
\\	スタート		
\\	―位	
\\	位[い]		
\\	優勝します	
\\	優[ゆう] 勝[しょう]します		
\\	悩み	
\\	悩[なや]み		
\\	目覚まし時計	
\\	目[め] 覚[ざ]まし 時[ど] 計[けい]		
\\	大学生	
\\	大[だい] 学[がく] 生[せい]		
\\	回答します	
\\	回[かい] 答[とう]します		
\\	鳴ります	
\\	鳴[な]ります		
\\	セットします	
\\	セットします		
\\	それでも	
\\	それでも		
\\	整形外科	
\\	整[せい] 形[けい] 外[げ] 科[か]		
\\	皮膚科	
\\	皮[ひ] 膚[ふ] 科[か]		
\\	産婦人科	
\\	産[さん] 婦[ふ] 人[じん] 科[か]		
\\	内科	
\\	内[ない] 科[か]		
\\	待合室	
\\	待[まち] 合[あい] 室[しつ]		
\\	外科	
\\	外[げ] 科[か]		
\\	眼科	
\\	眼[がん] 科[か]		
\\	小児科	
\\	小[しょう] 児[に] 科[か]		
\\	歯科	
\\	歯[し] 科[か]		
\\	泌尿器科	
\\	泌[ひ] 尿[にょう] 器[き] 科[か]		
\\	会計	
\\	会[かい] 計[けい]		
\\	耳鼻咽喉科	
\\	耳[じ] 鼻[び] 咽[いん] 喉[こう] 科[か]		
\\	薬局	
\\	薬[やっ] 局[きょく]		
\\	診察する	
\\	診[しん] 察[さつ]する		
\\	検査する	
\\	検[けん] 査[さ]する		
\\	注射する	
\\	注[ちゅう] 射[しゃ]する		
\\	レントゲンを撮る	
\\	レントゲンを 撮[と]る		
\\	入院する	
\\	入[にゅう] 院[いん]する		
\\	手術する	
\\	手[しゅ] 術[じゅつ]する		
\\	麻酔する	
\\	麻[ま] 酔[すい]する		
\\	処方箋	
\\	処[しょ] 方[ほう] 箋[せん]		
\\	カルテ	
\\	カルテ		
\\	診察券	
\\	診[しん] 察[さつ] 券[けん]		
\\	薬の種類	
\\	薬[くすり]の 種[しゅ] 類[るい]		
\\	痛み止め	
\\	痛[いた]み 止[ど]め		
\\	湿布薬	
\\	湿[しっ] 布[ぷ] 薬[やく]		
\\	解熱剤	
\\	解[げ] 熱[ねつ] 剤[ざい]		
\\	錠剤	
\\	錠[じょう] 剤[ざい]		
\\	粉薬	
\\	粉[こな] 薬[ぐすり]		
\\	カプセル	
\\	カプセル		
\\	渡します	
\\	渡[わた]します		
\\	帰ってきます	
\\	帰[かえ]ってきます		
\\	バスが出ます	
\\	バスが 出[で]ます		
\\	荷物が届きます	
\\	荷[に] 物[もつ]が 届[とど]きます		
\\	大学に入学します	
\\	大[だい] 学[がく]に 入[にゅう] 学[がく]します		
\\	大学を卒業します	
\\	大[だい] 学[がく]を 卒[そつ] 業[ぎょう]します		
\\	焼きます	
\\	焼[や]きます		
\\	パンが焼けます	
\\	パンが 焼[や]けます		
\\	肉が焼けます	
\\	肉[にく]が 焼[や]けます		
\\	留守	
\\	留[る] 守[す]		
\\	原因	
\\	原[げん] 因[いん]		
\\	~の所	
\\	~の 所[ところ]		
\\	半年	
\\	半[はん] 年[とし]		
\\	ちょうど	
\\	ちょうど		
\\	たった今	
\\	たった 今[いま]		
\\	今いいですか。	
\\	今[いま]いいですか。		
\\	ガスサービスセンター	
\\	ガスサービスセンター		
\\	ガスレンジ	
\\	ガスレンジ		
\\	具合	
\\	具[ぐ] 合[あい]		
\\	申し訳ありません。	
\\	申[もう]し 訳[わけ]ありません。		
\\	向かいます	
\\	向[む]かいます		
\\	ついています	
\\	ついています		
\\	床	
\\	床[ゆか]		
\\	転びます	
\\	転[ころ]びます		
\\	ベル	
\\	ベル		
\\	順番に	
\\	順[じゅん] 番[ばん]に		
\\	出来事	
\\	出[で] 来[き] 事[ごと]		
\\	服飾	
\\	服[ふく] 飾[しょく]		
\\	医療	
\\	医[い] 療[りょう]		
\\	その他	
\\	その 他[た]		
\\	ジャム	
\\	ジャム		
\\	クッキー	
\\	クッキー		
\\	チーズ	
\\	チーズ		
\\	オムレツ	
\\	オムレツ		
\\	フランクフルト	
\\	フランクフルト		
\\	カステラ	
\\	カステラ		
\\	マカロニ	
\\	マカロニ		
\\	パスタ	
\\	パスタ		
\\	エプロン	
\\	エプロン		
\\	ランジェリー	
\\	ランジェリー		
\\	ホック	
\\	ホック		
\\	ズック	
\\	ズック		
\\	ビロード	
\\	ビロード		
\\	レントゲン	
\\	レントゲン		
\\	アレルギー	
\\	アレルギー		
\\	メス	
\\	メス		
\\	ピンセット	
\\	ピンセット		
\\	コーラス	
\\	コーラス		
\\	メロディー	
\\	メロディー		
\\	バレエ	
\\	バレエ		
\\	アトリエ	
\\	アトリエ		
\\	メルヘン	
\\	メルヘン		
\\	オルゴール	
\\	オルゴール		
\\	スケジュール	
\\	スケジュール		
\\	ティッシュペーパー	
\\	ティッシュペーパー		
\\	アンケート	
\\	アンケート		
\\	コンクール	
\\	コンクール		
\\	エネルギー	
\\	エネルギー		
\\	テーム	
\\	テーム		
\\	ゴム	
\\	ゴム		
\\	ペンキ	
\\	ペンキ		
\\	カルタ	
\\	カルタ		
\\	風が吹きます	
\\	風[かぜ]が 吹[ふ]きます		
\\	ごみが燃えます	
\\	ごみが 燃[も]えます		
\\	亡くなります(死にます)	
\\	亡[な]くなります
\\	死[し]にます)		
\\	人が集まります	
\\	人[ひと]が 集[あつ]まります		
\\	人が別れます	
\\	人[ひと]が 別[わか]れます		
\\	音がします	
\\	音[おと]がします		
\\	声がします	
\\	声[こえ]がします		
\\	味がします	
\\	味[あじ]がします		
\\	匂いがします	
\\	匂[にお]いがします		
\\	ひどい	
\\	ひどい		
\\	実験	
\\	実[じっ] 験[けん]		
\\	データ	
\\	データ		
\\	人口	
\\	人[じん] 口[こう]		
\\	匂い	
\\	匂[にお]い		
\\	科学	
\\	科[か] 学[がく]		
\\	パトカー	
\\	パトカー		
\\	救急車	
\\	救[きゅう] 急[きゅう] 車[しゃ]		
\\	賛成	
\\	賛[さん] 成[せい]		
\\	反対	
\\	反[はん] 対[たい]		
\\	大統領	
\\	大[だい] 統[とう] 領[りょう]		
\\	~によると	
\\	~によると		
\\	婚約します	
\\	婚[こん] 約[やく]します		
\\	どうも	
\\	どうも		
\\	化粧をします	
\\	化[け] 粧[しょう]をします		
\\	女性	
\\	女[じょ] 性[せい]		
\\	男性	
\\	男[だん] 性[せい]		
\\	長生きします	
\\	長[なが] 生[い]きします		
\\	擬音語	
\\	擬[ぎ] 音[おん] 語[ご]		
\\	擬態語	
\\	擬[ぎ] 態[たい] 語[ご]		
\\	ザーザー(降る)	
\\	ザーザー
\\	降[ふ]る)		
\\	ピューピュー(吹く)	
\\	ピューピュー
\\	吹[ふ]く)		
\\	ゴロゴロ(鳴る)	
\\	ゴロゴロ
\\	鳴[な]る)		
\\	ワンワン(ほえる)	
\\	ワンワン(ほえる)		
\\	ニャーニャー(鳴く)	
\\	ニャーニャー
\\	鳴[な]く)		
\\	かーかー(鳴く)	
\\	かーかー
\\	鳴[な]く)		
\\	げらげら(笑う)	
\\	げらげら
\\	笑[わら]う)		
\\	しくしく(泣く)	
\\	しくしく
\\	泣[な]く)		
\\	きょろきょろ(見る)	
\\	きょろきょろ
\\	見[み]る)		
\\	ぱくぱく(食べる)	
\\	ぱくぱく
\\	食[た]べる)		
\\	ぐうぐう(寝る)	
\\	ぐうぐう
\\	寝[ね]る)		
\\	すらすら(読む)	
\\	すらすら
\\	読[よ]む)		
\\	ざらざら(している)	
\\	ざらざら(している)		
\\	べたべた(している)	
\\	べたべた(している)		
\\	つるつる(している)	
\\	つるつる(している)		
\\	降ろします、下ろします	
\\	降[お]ろします、 下[お]ろします		
\\	届けます	
\\	届[とど]けます		
\\	録音します	
\\	録[ろく] 音[おん]します		
\\	塾	
\\	塾[じゅく]		
\\	生徒	
\\	生[せい] 徒[と]		
\\	~間	
\\	間[かん]		
\\	いいことですね。	
\\	いいことですね。		
\\	お忙しいですか。	
\\	お 忙[いそが]しいですか。		
\\	営業	
\\	営[えい] 業[ぎょう]		
\\	それまでに	
\\	それまでに		
\\	かまいません	
\\	かまいません		
\\	楽しみます	
\\	楽[たの]しみます		
\\	親	
\\	親[おや]		
\\	小学生	
\\	小[しょう] 学[がく] 生[せい]		
\\	ーパーセント	ー 
\\	ーパーセント		
\\	その次	
\\	その 次[つぎ]		
\\	習字	
\\	習[しゅう] 字[じ]		
\\	普通の	
\\	普[ふ] 通[つう]の		
\\	しつける	
\\	しつける		
\\	鍛える	
\\	鍛[きた]える		
\\	経験	
\\	経[けい] 験[けん]		
\\	年寄り	
\\	年[とし] 寄[よ]り		
\\	決める	
\\	決[き]める		
\\	自信	
\\	自[じ] 信[しん]		
\\	責任	
\\	責[せき] 任[にん]		
\\	我慢	
\\	我[が] 慢[まん]		
\\	会社に勤めます	
\\	会[かい] 社[しゃ]に 勤[つと]めます		
\\	椅子に掛けます	
\\	椅子[いす]に 掛[か]けます		
\\	過ごします	
\\	過[す]ごします		
\\	召し上がります	
\\	たべます、のみます)	召[め]し 上[あ]がります		
\\	おっしゃいます	
\\	いいます)	おっしゃいます		
\\	なさいます	
\\	します)	なさいます		
\\	ご覧になります	
\\	みます)	ご 覧[らん]になります		
\\	ご存知です	
\\	しっています)	ご 存[ぞん] 知[じ]です		
\\	旅館	
\\	旅[りょ] 館[かん]		
\\	奥様	
\\	おくさん)	奥[おく] 様[さま]		
\\	たまに	
\\	たまに		
\\	どなたでも	
\\	だれでも)	どなたでも		
\\	~といいます	
\\	~といいます		
\\	ー年ー組	
\\	ー 年[ねん]ー 組[くみ]		
\\	熱を出します	
\\	熱[ねつ]を 出[だ]します		
\\	よろしくお伝えください。	
\\	よろしくお 伝[つた]えください。		
\\	失礼いたします。	
\\	しつれいします)	失[しつ] 礼[れい]いたします。		
\\	経歴	
\\	経[けい] 歴[れき]		
\\	医学部	
\\	医[い] 学[がく] 部[ぶ]		
\\	目指します	
\\	目[め] 指[ざ]します		
\\	進みます	
\\	進[すす]みます		
\\	細胞	
\\	細[さい] 胞[ぼう]		
\\	開発します	
\\	開[かい] 発[はつ]します		
\\	マウス	
\\	マウス		
\\	受賞します	
\\	受[じゅ] 賞[しょう]します		
\\	講演会	
\\	講[こう] 演[えん] 会[かい]		
\\	行事	
\\	行[ぎょう] 事[じ]		
\\	お正月	
\\	(1月1日~3日)	お 正[しょう] 月[がつ]		
\\	豆まき	
\\	(2月3日ごろ)	豆[まめ]まき		
\\	ひな祭り	
\\	(3月3日)	ひな 祭[まつ]り		
\\	お月見	
\\	(9月15日ごろ)	お 月[つき] 見[み]		
\\	大晦日	
\\	(12月31日)	大[おお] 晦[みそ] 日[か]		
\\	参ります	
\\	いきます、きます)	参[まい]ります		
\\	おります	
\\	います)	おります		
\\	申します、申し上ます	
\\	いいます)	申[もう]します、 申[もう]し 上[あげ]ます		
\\	いたします	
\\	します)	いたします		
\\	拝見します	
\\	みます)	拝[はい] 見[けん]します		
\\	存じます	
\\	しっています)	存[ぞん]じます		
\\	お目にかかります	
\\	あいます)	お 目[め]にかかります		
\\	コーヒーを入れます	
\\	コーヒーを 入[い]れます		
\\	用意します	
\\	用[よう] 意[い]します		
\\	私
\\	わたし)	私[わたくし]		
\\	ガイド	
\\	ガイド		
\\	メールアドレス	
\\	メールアドレス		
\\	再来週	
\\	再[さ] 来[らい] 週[しゅう]		
\\	再来月	
\\	再[さ] 来[らい] 月[げつ]		
\\	再来年	
\\	再[さ] 来[らい] 年[ねん]		
\\	初めに	
\\	初[はじ]めに		
\\	緊張します	
\\	緊[きん] 張[ちょう]します		
\\	賞金	
\\	賞[しょう] 金[きん]		
\\	きりん	
\\	きりん		
\\	ころ	
\\	ころ		
\\	夢がかないます	
\\	夢[ゆめ]がかないます		
\\	応援します	
\\	応[おう] 援[えん]します		
\\	心から	
\\	心[こころ]から		
\\	感謝します	
\\	感[かん] 謝[しゃ]します		
\\	お元気でいらっしゃいますか。	
\\	おげんきですか)	お 元[げん] 気[き]でいらっしゃいますか。		
\\	迷惑をかけます	
\\	迷[めい] 惑[わく]をかけます		
\\	生かします	
\\	生[い]かします		
\\	宛名	
\\	宛[あて] 名[な]		
\\	おとといの朝	
\\	おとといの 朝[あさ]		
\\	昨日の朝	
\\	昨日[きのう]の 朝[あさ]		
\\	明日の朝	
\\	明日[あした]の 朝[あさ]		
\\	明後日の朝	
\\	明後日[あさって]の 朝[あさ]		
\\	おとといの晩(夜)	
\\	おとといの 晩[ばん]
\\	夜[よる])		
\\	昨日の晩(夜)	
\\	昨日[きのう]の 晩[ばん]
\\	夜[よる])		
\\	明日の晩(夜)	
\\	明日[あした]の 晩[ばん]
\\	夜[よる])		
\\	明後日の晩(夜)	
\\	明後日[あさって]の 晩[ばん]
\\	夜[よる])		
\\	先々週(二週間前)	
\\	先[せん]々[せん] 週[しゅう](二[に]週[しゅう]間[かん] 前[まえ])		
\\	先々月(二週間前)	
\\	先[せん]々[せん] 月[げつ]
\\	二[に]週[しゅう]間[かん] 前[まえ])		
\\	カレンダー	
\\	カレンダー		
\\	ポスター	
\\	ポスター		
\\	予定表	
\\	予[よ] 定[てい] 表[ひょう]		
\\	ガイドブック	
\\	ガイドブック		
\\	~とか、~とか	
\\	~とか、~とか		
\\	宝くじ	
\\	宝[たから]くじ		
\\	アフリカ	
\\	アフリカ		
\\	バザー	
\\	バザー		
\\	カット	
\\	カット		
\\	1センチくらい~	
\\	1センチくらい~		
\\	どちら様でしょうか。	
\\	どちら 様[さま]でしょうか。		
\\	お待たせしました。	
\\	お 待[ま]たせしました。		
\\	お礼	
\\	お 礼[れい]		
\\	安全(な)	
\\	安[あん] 全[ぜん](な)		
\\	お客様	
\\	お 客[きゃく]様[さま](お 客[きゃく]さん)		
\\	お子さん	
\\	お 子[こ]さん		
\\	ガソリン	
\\	ガソリン		
\\	金	
\\	金[きん]		
\\	キャンプ	
\\	キャンプ		
\\	~会社	
\\	会[がい] 社[しゃ]		
\\	缶詰	
\\	缶[かん] 詰[づめ]		
\\	みえます	
\\	きます)	みえます		
\\	差し上げます	
\\	あげます)	差[さ]し 上[あ]げます		
\\	早寝	
\\	早[はや] 寝[ね]		
\\	早起き	
\\	早[はや] 起[お]き		
\\	外食	
\\	外[がい] 食[しょく]		
\\	宿	
\\	宿[やど]		
\\	便利屋	
\\	便[べん] 利[り] 屋[や]		
\\	金銀	
\\	金[きん] 銀[ぎん]		
\\	貴重品	
\\	貴[き] 重[ちょう] 品[ひん]		
\\	救急薬品	
\\	救[きゅう] 急[きゅう] 薬[やく] 品[ひん]		
\\	退院する	
\\	退[たい] 院[いん]する		
\end{CJK}
\end{document}