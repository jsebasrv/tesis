\documentclass[8pt]{extreport} 
\usepackage{hyperref}
\usepackage{CJKutf8}
\begin{document}
\begin{CJK}{UTF8}{min}
\\	毎日同じ物を食
\\	こと]は、面白くない。	毎日同じ物を食べるのは、面白くない。	まいにちおなじものをたべるのは、おもしろくない。	
\\	「の」 
\\	毎日[まいにち] 同[おな]じ 物[もの]を 食[た]べるのは、 面白[おもしろ]くない。		
\\	電気が点
\\	。	電気が点いた。	でんきがついた。	
\\	電気[でんき]が 点[つ]いた。		
\\	ボブは、魚が好きなんだ
\\	。/そうだね。	ボブは、魚が好きなんだよね。/そうだね。	ボブは、さかながすきなんだよね。/そうだね。	
\\	「よね」	ボブは、 魚[さかな]が 好[す]きなんだよね。/そうだね。		
\\	魚
\\	。	魚だ。	さかなだ。	
\\	「だ」	魚[さかな]だ。		
\\	猫は
\\	。	猫はいない。	ねこはいない。	
\\	猫[ねこ]はいない。		
\\	晩ご飯を食
\\	人は、映画で見
\\	銀行に行った。	晩ご飯を食べなかった人は、映画で見た銀行に行った。	ばんごはんをたべなかったひとは、えいがでみたぎんこうにいった。	
\\	晩[ばん]ご 飯[はん]を 食[た]べなかった 人[ひと]は、 映画[えいが]で 見[み]た 銀行[ぎんこう]に 行[い]った。		
\\	学生
\\	。	学生じゃない。	がくせいじゃない。	
\\	学生[がくせい]じゃない。		
\\	いい天気だ
\\	。/そう
\\	。	いい天気だね。/そうね。	いいてんきだね。/そうね。	
\\	「ね」 
\\	いい 天気[てんき]だね。/そうね。		
\\	高速道路
\\	走る。	高速道路を走る。	こうそくどうろをはしる。	
\\	「を」 
\\	高速[こうそく] 道路[どうろ]を 走[はし]る。		
\\	朝ご飯を食
\\	。/どうして?	朝ご飯を食べるんじゃなかった。/どうして?	あさごはんをたべるんじゃなかった。/どうして?	
\\	「の」 
\\	朝[あさ]ご 飯[はん]を 食[た]べるんじゃなかった。/どうして?		
\\	魚が好
\\	タイプ。	魚が好きなタイプ。	さかながすきなタイプ。	
\\	魚[さかな]が 好[す]きなタイプ。		
\\	静
\\	が、アリス
\\	部屋だ。	静かなのが、アリスの部屋だ。	しずかなのが、アリスのへやだ。	
\\	「の」 
\\	静[しず]かなのが、アリスの 部屋[へや]だ。		
\\	私は買
\\	。	私は買わない。	わたしはかわない。	
\\	私[わたし]は 買[か]わない。		
\\	今、授業がない
\\	/ううん、ある。	今、授業がないんじゃない?/ううん、ある。	いま、じゅぎょうがないんじゃない?/ううん、ある。	
\\	「の」 
\\	今[いま]、 授業[じゅぎょう]がないんじゃない?/ううん、ある。		
\\	勝ち
\\	向かう。	勝ちへ向かう。	かちへむかう。	
\\	「へ」 
\\	勝[か]ちへ 向[む]かう。		
\\	部屋を出
\\	。	部屋を出た。	へやをでた。	
\\	部屋[へや]を 出[で]た。		
\\	お金が
\\	。	お金がない。	おかねがない。	
\\	お 金[かね]がない。		
\\	本
\\	雑誌
\\	葉書を買った。	本と雑誌と葉書を買った。	ほんとざっしとはがきをかった。	
\\	「と」 
\\	本[ほん]と 雑誌[ざっし]と 葉書[はがき]を 買[か]った。		
\\	誰が窓を開
\\	誰が窓を開けた?	だれがまどをあけた?	
\\	誰[だれ]が 窓[まど]を 開[あ]けた?		
\\	私も遊
\\	。	私も遊んだ。	わたしもあそんだ。	
\\	私[わたし]も 遊[あそ]んだ。		
\\	今日はいい天気だ
\\	。/うん。でも、明日雨が降る
\\	。	今日はいい天気だね。/うん。でも、明日雨が降るよ。	きょうはいいてんきだね。/うん。でも、あしたあめがふるよ。	
\\	「よ」 
\\	今日[きょう]はいい 天気[てんき]だね。/うん。でも、 明日[あした] 雨[あめ]が 降[ふ]るよ。		
\\	今は忙しい
\\	。	今は忙しいの。	いまはいそがしいの。	
\\	「の」 
\\	今[いま]は 忙[いそが]しいの。		
\\	今日
\\	試験だ。/ジム
\\	/ジム
\\	明日。	今日は試験だ。/ジムは?/ジムは明日。	きょうはしけんだ。/ジムは?/ジムはあした。	
\\	「は」 
\\	今日[きょう]は 試験[しけん]だ。/ジムは?/ジムは 明日[あした]。		
\\	魚が好
\\	タイプ。	魚が好きだったタイプ。	さかながすきだったタイプ。	
\\	魚[さかな]が 好[す]きだったタイプ。		
\\	友達は、来年、日本
\\	行く。	友達は、来年、日本に行く。	ともだちは、らいねん、にほんにいく。	
\\	「に」 
\\	友達[ともだち]は、 来年[らいねん]、 日本[にっぽん]に 行[い]く。		
\\	街
\\	ぶらぶら歩く。	街をぶらぶら歩く。	まちをぶらぶらあるく。	
\\	「を」 
\\	街[まち]をぶらぶら 歩[ある]く。		
\\	毎日、日本語
\\	勉強する。	毎日、日本語を勉強する。	まいにち、にほんごをべんきょうする。	
\\	「を」 
\\	毎日[まいにち]、 日本語[にほんご]を 勉強[べんきょう]する。		
\\	ジュース
\\	飲んだ。	ジュースを飲んだ。	ジュースをのんだ。	
\\	「を」 
\\	ジュースを 飲[の]んだ。		
\\	アリスは学生?/うん、トム
\\	学生。	アリスは学生?/うん、トムも学生。	アリスはがくせい?/うん、トムもがくせい。	
\\	「も」 
\\	アリスは 学生[がくせい]?/うん、トムも 学生[がくせい]。		
\\	お金が
\\	。	お金がなかった。	おかねがなかった。	
\\	お 金[かね]がなかった。		
\\	その人が買
\\	/ううん、先生が買
\\	。	その人が買うんじゃなかったの?/ううん、先生が買うんだ。	そのひとがかうんじゃなかったの?/ううん、せんせいがかうんだ。	
\\	「の」 
\\	その 人[ひと]が 買[か]うんじゃなかったの?/ううん、 先生[せんせい]が 買[か]うんだ。		
\\	ボブの声は、
\\	大きい。	ボブの声は、結構大きい。	ボブのこえは、けっこうおおきい。	
\\	ボブの 声[こえ]は、 結構[けっこう] 大[おお]きい。		
\\	ボブは魚が好
\\	。	ボブは魚が好きじゃない。	ボブはさかながすきじゃない。	
\\	ボブは 魚[さかな]が 好[す]きじゃない。		
\\	高
\\	ビル。	高くなかったビル。	たかくなかったビル。	
\\	高[たか]くなかったビル。		
\\	何できた?/バスできた。
\\	何で?	何できた?/バスできた。
\\	何で?	なにできた?/バスできた。	
\\	「で」 
\\	「何」	何[なん]できた?/バスできた。 
\\	何[なに]で?		
\\	ボブは魚が好
\\	。	ボブは魚が好きだった。	ボブはさかながすきだった。	
\\	ボブは 魚[さかな]が 好[す]きだった。		
\\	お金が
\\	。	お金がある。	おかねがある。	
\\	お 金[かね]がある。		
\\	ジムが遊
\\	。	ジムが遊ばない。	ジムはあそばない。	
\\	ジムが 遊[あそ]ばない。		
\\	時間がない
\\	。/大丈夫だ
\\	。	時間がないよ。/大丈夫だよ。	じかんがないよ。/だいじょうぶだよ。	
\\	「よ」 
\\	時間[じかん]がないよ。/ 大丈夫[だいじょうぶ]だよ。		
\\	猫は
\\	。	猫はいなかった。	ねこはいなかった。	
\\	猫[ねこ]はいなかった。		
\\	ボブも
\\	。	ボブもしない。	ボブもしない。	
\\	ボブもしない。		
\\	友達は、来年
\\	日本
\\	行く。	友達は、来年に日本に行く。	ともだちは、らいねんににほんにいく。	
\\	「に」 
\\	友達[ともだち]は、 来年[らいねん]に 日本[にっぽん]に 行[い]く。		
\\	面白い映画だった
\\	全然面白くなかった。	面白い映画だったね。/え?全然面白くなかった。	おもしろいえいがだったね。/え?ぜんぜんおもしろくなかった。	
\\	「ね」 
\\	面白[おもしろ]い 映画[えいが]だったね。/え? 全然[ぜんぜん] 面白[おもしろ]くなかった。		
\\	友達
\\	話した。	友達と話した。	ともだちとはなした。	
\\	「と」 
\\	友達[ともだち]と 話[はな]した。		
\\	値段が高
\\	レストランはあまり好きじゃない。	値段が高いレストランはあまり好きじゃない。	ねだんがたかいレストランはあまりすきじゃない。	
\\	値段[ねだん]が 高[たか]いレストランはあまり 好[す]きじゃない。		
\\	毎日勉強
\\	こと
\\	は大変。	毎日勉強するのは大変。	まいにちべんきょうのはたいへん。	
\\	「の」 
\\	毎日[まいにち] 勉強[べんきょう]するのは 大変[たいへん]。		
\\	アリスは学生?/うん、
\\	トム
\\	学生じゃない。	アリスは学生?/うん、でもトムは学生じゃない。	アリスはがくせい?/うん、でもトムはがくせいじゃない。	
\\	「も」 
\\	アリスは 学生[がくせい]?/うん、でもトムは 学生[がくせい]じゃない。		
\\	今日は、走
\\	。	今日は、走った。	きょうは、はしった。	
\\	今日[きょう]は、 走[はし]った。		
\\	静かな高
\\	ビル。	静かな高いビル。	すずかなたかいビル。	
\\	静[しず]かな 高[たか]いビル。		
\\	。	食べる。	たべる。	
\\	食[た]べる。		
\\	誰
\\	学生?	誰が学生?	だれががくせい?	
\\	「が」 
\\	誰[だれ]が 学生[がくせい]?		
\\	魚
\\	。	魚じゃない。	さかなじゃない。	
\\	魚[さかな]じゃない。		
\\	バス
\\	帰る。	バスで帰る。	バスでかえる。	
\\	「で」 
\\	バスで 帰[かえ]る。		
\\	ジム
\\	。	ジムなのだ。	ジムなのだ。	
\\	「の」 
\\	ジムなのだ。		
\\	魚が好
\\	タイプは、肉が好
\\	。	魚が好きじゃないタイプは、肉が好きだ。	さかながすきじゃないタイプは、にくがすきだ。	
\\	魚[さかな]が 好[す]きじゃないタイプは、 肉[にく]が 好[す]きだ。		
\\	先生
\\	会った。	先生と会った。	せんせいとあった。	
\\	「と」 
\\	先生[せんせい]と 会[あ]った。		
\\	電気を消
\\	。	電気を消す。	でんきをけす。	
\\	電気[でんき]を 消[け]す。		
\\	靴
\\	シャツを買う。	靴やシャツを買う。	くつやシャツをかう。	
\\	「や」 
\\	「とか」 
\\	靴[くつ]やシャツを 買[か]う。		
\\	魚が好
\\	タイプ。	魚が好きじゃないタイプ。	さかながすきじゃないタイプ。	
\\	魚[さかな]が 好[す]きじゃないタイプ。		
\\	日本語
\\	、習う。	日本語は、習う。	にほんごは、ならう。	
\\	日本語[にほんご]は、 習[なら]う。		
\\	静か
\\	。	静かじゃない。	しずかじゃない。	
\\	静[しず]かじゃない。		
\\	学校に行った?/行かなかった。/図書館
\\	は? /図書館
\\	行かなかった。	学校に行った?/行かなかった。/図書館には? /図書館にも行かなかった。	「がっこうにいった?/いかなかった。/としょかんには?/としょかんにもいかなかった。	
\\	学校[がっこう]に 行[い]った?/ 行[い]かなかった。/ 図書館[としょかん]には?/ 図書館[としょかん]にも 行[い]かなかった。		
\\	家
\\	帰らない。	家に帰らない。	{いえ
\\	うち}にかえらない。	
\\	「に」 
\\	家[いえ]に 帰[かえ]らない。		
\\	日本語
\\	習う。	日本語を習う。	にほんごをならう。	
\\	日本語[にほんご]を 習[なら]う。		
\\	魚
\\	食べる。	魚を食べる。	さかなをたべる。	
\\	「を」 
\\	魚[さかな]を 食[た]べる。		
\\	ジムが
\\	。	ジムがしなかった。	ジムがしなかった。	
\\	ジムがしなかった。		
\\	勉強は、
\\	。	勉強は、した。	べんきょうは、した。	
\\	勉強[べんきょう]は、した。		
\\	友達は親切
\\	人。	友達は親切な人。	ともだちはしんせつなひと。	
\\	友達[ともだち]は 親切[しんせつ]な 人[ひと]。		
\\	アリスは食
\\	。	アリスは食べなかった。	アリスはたべなかった。	
\\	アリスは 食[た]べなかった。		
\\	高
\\	ビル。	高かったビル。	たかかったビル。	
\\	高[たか]かったビル。		
\\	飲み物
\\	カップ
\\	ナプキンは、いらない?	飲み物やカップやナプキンは、いらない?	のみものやカップやナプキンは、いらない?	
\\	「や」 
\\	「とか」 
\\	飲み物[のみもの]やカップやナプキンは、いらない?		
\\	魚
\\	。	魚だった。	さかなだった。	
\\	魚[さかな]だった。		
\\	この町は、最近大
\\	変わった。	この町は、最近大きく変わった。	このまちは、さいきんおおきくかわった。	
\\	この 町[まち]は、 最近[さいきん] 大[おお]きく 変[か]わった。		
\\	授業に行く
\\	こと]を忘れた。	授業に行くのを忘れた。	じゅぎょうにいくのをわすれた。	
\\	「の」 
\\	授業[じゅぎょう]に 行[い]くのを 忘[わす]れた。		
\\	先週に図書館
\\	行った。	先週に図書館に行った。	せんしゅうにとしょかんにいった。	
\\	「に」 
\\	先週[せんしゅう]に 図書館[としょかん]に 行[い]った。		
\\	どこに行
\\	/授業に行
\\	。	どこに行くの?/授業に行くんだ。	どこにいくの?/じゅぎょうにいくんだ。	
\\	「の」 
\\	どこに 行[い]くの?/ 授業[じゅぎょう]に 行[い]くんだ。		
\\	高
\\	ビル。	高いビル。	たかいビル。	
\\	高[たか]いビル。		
\\	部屋
\\	来る。	部屋へ来る。	へやへくる。	
\\	「へ」 
\\	部屋[へや]へ 来[く]る。		
\\	映画は、全部見
\\	。	映画は、全部見た。	えいがは、ぜんぶみた。	
\\	映画[えいが]は、 全部[ぜんぶ] 見[み]た。		
\\	部屋
\\	来る。	部屋に来る。	へやにくる。	
\\	「に」 
\\	部屋[へや]に 来[く]る。		
\\	ジムは医者
\\	なる。	ジムは医者になる。	ジムはいしゃになる。	
\\	「に」 
\\	ジムは 医者[いしゃ]になる。		
\\	ジム
\\	明日?/明日じゃない。	ジムは明日?/明日じゃない。	ジムはあした?/あしたじゃない。	
\\	「は」 
\\	ジムは 明日[あした]?/ 明日[あした]じゃない。		
\\	何できた?/暇だから。
\\	何で?	何できた?/暇だから。
\\	何で?	なんできた?/ひまだから。	
\\	「で」 
\\	「何」	何[なん]できた?/ 暇[ひま]だから。 
\\	何[なに]で?		
\\	値段があんまりよ
\\	。	値段があんまりよくない。	ねだんがあんまりよくない。	
\\	値段[ねだん]があんまりよくない。		
\\	家
\\	帰らない。	家へ帰らない。	{いえ
\\	うち}へかえらない。	
\\	「へ」 
\\	家[いえ]へ 帰[かえ]らない。		
\\	子供
\\	アリスが立派な大人になった。	子供だったアリスが立派な大人になった。	こどもだったアリスがりっぱなおとなになった。	
\\	子供[こども]だったアリスが 立派[りっぱ]な 大人[おとな]になった。		
\\	今は忙しい
\\	今は忙しいの?	いまはいそがしいの?	
\\	「の」 
\\	今[いま]は 忙[いそが]しいの?		
\\	魚が好
\\	タイプ。	魚が好きじゃなかったタイプ。	さかながすきじゃなかったタイプ。	
\\	魚[さかな]が 好[す]きじゃなかったタイプ。		
\\	私が電気を点
\\	。	私が電気を点けた。	わたしがでんきをつけた。	
\\	私[わたし]が 電気[でんき]を 点[つ]けた。		
\\	レストラン
\\	昼ご飯を食べた。	レストランで昼ご飯を食べた。	レストランでひるごはんをたべた。	
\\	「で」 
\\	レストランで 昼[ひる]ご 飯[はん]を 食[た]べた。		
\\	私は買
\\	。	私は買わなかった。	わたしはかわなかった。	
\\	私[わたし]は 買[か]わなかった。		
\\	学生
\\	人は、学校に行かない。	学生じゃない人は、学校に行かない。	がくせいじゃないひとは、がっこうにいかない。	
\\	学生[がくせい]じゃない 人[ひと]は、 学校[がっこう]に 行[い]かない。		
\\	ボブ
\\	本。	ボブの本。	ボブのほん。	
\\	「の」 
\\	ボブの 本[ほん]。		
\\	静
\\	人。	静かな人。	しずかなひと。	
\\	静[しず]かな 人[ひと]。		
\\	ボブも行
\\	。	ボブも行かなかった。	ボブもいかなかった。	
\\	ボブも 行[い]かなかった。		
\\	白
\\	こと]は、かわいい。	白いのは、かわいい。	しろいのは、かわいい。	
\\	「の」 
\\	白[しろ]いのは、かわいい。		
\\	ジムが
\\	。	ジムが遊ぶ。	ジムはあそぶ。	
\\	ジムが 遊[あそ]ぶ。		
\\	猫は部屋
\\	いる。	猫は部屋にいる。	ねこはへやにいる。	
\\	「に」 
\\	猫[ねこ]は 部屋[へや]にいる。		
\\	アリスは食
\\	。	アリスは食べない。	アリスはたべない。	
\\	アリスは 食[た]べない。		
\\	アリスは、アメリカ
\\	来た。	アリスは、アメリカから来た。	アリスは、アメリカからきた。	
\\	「に」 
\\	アリスは、アメリカから 来[き]た。		
\\	映画を
\\	見た。	映画をたくさん見た。	えいがをたくさんみた。	
\\	映画[えいが]をたくさん 見[み]た。		
\\	そのシャツは誰
\\	/ボブ
\\	だ。	そのシャツは誰の?/ボブのだ。	そのシャツはだれの?/ボブのだ。	
\\	「の」 
\\	そのシャツは 誰[だれ]の?/ボブのだ。		
\\	。	魚。	さかな。	
\\	「だ」	魚[さかな]。		
\\	今は忙しい
\\	。	今は忙しいのだ。	いまはいそがしいのだ。	
\\	「の」 
\\	今[いま]は 忙[いそが]しいのだ。		
\\	最近、
\\	食べない。	最近、全然食べない。	さいきん、ぜんぜんたべない。	
\\	最近[さいきん]、 全然[ぜんぜん] 食[た]べない。		
\\	ボブは、アメリカ
\\	大学
\\	学生だ。	ボブは、アメリカの大学の学生だ。	ボブは、アメリカのだいがくのがくせいだ。	
\\	「の」 
\\	ボブは、アメリカの 大学[だいがく]の 学生[がくせい]だ。		
\\	映画館
\\	見た。	映画館で見た。	えいがかんでみた。	
\\	「で」 
\\	映画[えいが] 館[かん]で 見[み]た。		
\\	赤いズボンを買
\\	友達はボブだ。	赤いズボンを買う友達はボブだ。	あかいズボンをかうともだちはボブだ。	
\\	赤[あか]いズボンを 買[か]う 友達[ともだち]はボブだ。		
\\	ボブは日本
\\	行った。	ボブは日本へ行った。	ボブはにほんへいった。	
\\	「へ」 
\\	ボブは 日本[にっぽん]へ 行[い]った。		
\\	ボブも
\\	。	ボブもする。	ボブもする。	
\\	ボブもする。		
\\	静か
\\	。	静かじゃなかった。	しぞかじゃなかった。	
\\	静[しず]かじゃなかった。		
\\	椅子が台所
\\	あった。	椅子が台所にあった。	いすがだいどころにあった。	
\\	「に」 
\\	椅子[いす]が 台所[だいどころ]にあった。		
\\	白
\\	は、かわいい。	白い物は、かわいい。	しろいものは、かわいい。	
\\	「の」 
\\	もの 
\\	白[しろ]い 物[もの]は、かわいい。		
\\	靴
\\	や)]シャツを買う。	靴とかシャツを買う。	くつとかシャツをかう。	
\\	「や」 
\\	「とか」 
\\	靴[くつ]とかシャツを 買[か]う。		
\\	ボブは、いつも勉強
\\	人だ。	ボブは、いつも勉強する人だ。	ボブは、いつもべんきょうするひとだ。	
\\	ボブは、いつも 勉強[べんきょう]する 人[ひと]だ。		
\\	窓がどうして開
\\	窓がどうして開いた?	まどがどうしてあいた?	
\\	窓[まど]がどうして 開[ひら]いた?		
\\	私は
\\	。	私は買う。	わたしはかう。	
\\	私[わたし]は 買[か]う。		
\\	いい友達
\\	会った。	いい友達に会った。	いいともだちにあった。	
\\	「に」 
\\	いい 友達[ともだち]に 会[あ]った。		
\\	ボブは朝ご飯を早
\\	食べた。	ボブは朝ご飯を早く食べた。	ボブはあさごはんをはやくたべた。	
\\	ボブは 朝[あさ]ご 飯[はん]を 早[はや]く 食[た]べた。		
\\	飲み物
\\	や)]カップ
\\	や)]ナプキンは、いらない?	飲み物とかカップとかナプキンは、いらない?	のみものとかカップとかナプキンは、いらない?	
\\	「や」 
\\	「とか」 
\\	飲み物[のみもの]とかカップとかナプキンは、いらない?		
\\	アリス
\\	学生?/うん、学生。	アリスは学生?/うん、学生。	アリスはがくせい。/うん、がくせい	
\\	「は」 
\\	アリスは 学生[がくせい]?/うん、 学生[がくせい]。		
\\	先週に映画を見
\\	人は誰?	先週に映画を見た人は誰?	せんしゅうにえいがをみたひとはだれ?	
\\	先週[せんしゅう]に 映画[えいが]を 見[み]た 人[ひと]は 誰[だれ]?		
\\	どこで食べる? /イタリアレストラン
\\	はどう?	どこで食べる? /イタリアレストランではどう?	どこでだべる?/イタリアレストランではどう?	
\\	どこで 食[た]べる?/イタリアレストランではどう?		
\\	学生
\\	誰?	学生は誰?	がくせいはだれ?	
\\	「が」 
\\	学生[がくせい]は 誰[だれ]?		
\\	スプーン
\\	フォークで魚を食べた。	スプーンとフォークで魚を食べた。	スプーンとフォークでさかなをたべた。	
\\	「と」 
\\	スプーンとフォークで 魚[さかな]を 食[た]べた。		
\\	図書館の中では、静
\\	する。	図書館の中では、静かにする。	としょかんのなかでは、しずかにする。	
\\	図書館[としょかん]の 中[なか]では、 静[しず]かにする。		
\\	アリスは学生?/ううん、トム
\\	学生じゃない。	アリスは学生?/ううん、トムも学生じゃない。	アリスはがくせい?/ううん、トムもがくせいじゃない。	
\\	「も」 
\\	アリスは 学生[がくせい]?/ううん、トムも 学生[がくせい]じゃない。		
\\	彼はかっこよ
\\	彼はかっこよかった!	かれはかっこよかった。	
\\	彼[かれ]はかっこよかった!		
\\	今、授業がある
\\	/今は、ないんだ。	今、授業があるんじゃない?/今は、ないんだ。	いま、じゅぎょうがあるんじゃない?/いまは、ないんだ。	
\\	「の」 
\\	今[いま]、 授業[じゅぎょう]があるんじゃない?/ 今[いま]は、ないんだ。		
\\	宿題を今日
\\	明日までする。	宿題を今日から明日までする。	しゅくだいをきょうからあしたまでする。	
\\	「に」 
\\	宿題[しゅくだい]を 今日[きょう]から 明日[あした]までする。		
\\	誰
\\	学生?/ジム
\\	学生。	誰が学生?/ジムが学生。	だれががくせい?/ジムはがくせい。	
\\	「が」 
\\	誰[だれ]が 学生[がくせい]?/ジムが 学生[がくせい]。		
\\	アリスは
\\	。	アリスは食べる。	アリスはたべる。	
\\	アリスは 食[た]べる。		
\\	ボブは日本
\\	行った。	ボブは日本に行った。	ボブはにほんにいった。	
\\	「に」 
\\	ボブは 日本[にっぽん]に 行[い]った。		
\\	ボブは魚が
\\	。	ボブは魚が好きだ。	ボブはさかながすきだ。	
\\	ボブは 魚[さかな]が 好[す]きだ。		
\\	学生
\\	。	学生じゃなかった。	がくせいじゃなかった。	
\\	学生[がくせい]じゃなかった。		
\\	友達が来
\\	。	友達が来た。	ともだちがきた。	
\\	友達[ともだち]が 来[き]た。		
\\	ジム
\\	。	ジムのだ。	ジムのだ。	
\\	「の」 
\\	ジムのだ。		
\\	高
\\	ビル。	高くないビル。	たかくないビル。	
\\	高[たか]くないビル。		
\\	友達
\\	アリスは、いい友達になった。	友達じゃなかったアリスは、いい友達になった。	ともだちじゃなかったアリスは、いいともだちになった。	
\\	友達[ともだち]じゃなかったアリスは、いい 友達[ともだち]になった。		
\\	友達は
\\	。	友達は親切。	ともだちはしんせつ。	
\\	友達[ともだち]は 親切[しんせつ]。		
\\	アリスは自分の部屋をきれい
\\	した。	アリスは自分の部屋をきれいにした。	アリスはじぶんのへやをきれいにした。	
\\	アリスは 自分[じぶん]の 部屋[へや]をきれいにした。		
\\	授業に行く
\\	の]を忘れた。	授業に行くことを忘れた。	じゅぎょうにいくことをわすれた。	
\\	「の」 
\\	こと 
\\	授業[じゅぎょう]に 行[い]くことを 忘[わす]れた。		
\\	猫は
\\	。	猫はいる。	ねこはいる。	
\\	猫[ねこ]はいる。		
\\	ボブは魚が好
\\	。	ボブは魚が好きじゃなかった。	ボブはさかながすきじゃなかった。	
\\	ボブは 魚[さかな]が 好[す]きじゃなかった。		
\\	メールアドレス
\\	登録した。	メールアドレスを登録した。	メールアドレスをとうろくした。	
\\	「を」 
\\	メールアドレスを 登録[とうろく]した。		
\\	高
\\	静かなビル。	高くない静かなビル。	たかくないしずかなビル。	
\\	高[たか]くない 静[しず]かなビル。		
\\	ご飯は、食
\\	。	ご飯は、食べた。	ごはんは、たべた。	
\\	ご 飯[はん]は、 食[た]べた。		
\\	先週に医者
\\	ボブは、仕事を辞めた。	先週に医者だったボブは、仕事を辞めた。	せんしゅうにいしゃだったボブは、しごとをやめた。	
\\	先週[せんしゅう]に 医者[いしゃ]だったボブは、 仕事[しごと]を 辞[や]めた。		
\\	電気が消
\\	。	電気が消える。	でんきがきえる。	
\\	電気[でんき]が 消[き]える。		
\\	毎日運動
\\	のに、全然痩せなかった。	毎日運動したのに、全然痩せなかった。	まいにちうんどうしたのに、ぜんぜんやせなかった。	
\\	「のに」 
\\	毎日[まいにち] 運動[うんどう]したのに、 全然[ぜんぜん] 痩[や]せなかった。		
\\	それは、
\\	話だよ。	それは、ありえない話だよ。	それは、ありえないはなしだよ。	
\\	「ある得る」, 
\\	それは、ありえない 話[はなし]だよ。		
\\	去年から背が高
\\	なったね。	去年から背が高くなったね。	きょねんからせがたかくなったね。	
\\	する 
\\	なる 
\\	に 
\\	「なる」 
\\	去年[きょねん]から 背[せ]が 高[たか]くなったね。		
\\	残念だが、今週末は行
\\	。	残念だが、今週末は行けない。	ざんねんだが、こんしゅうまつはいけない。	
\\	が	残念[ざんねん]だが、 今週[こんしゅう] 末[まつ]は 行[い]けない。		
\\	友達に会
\\	ら、買い物に行きます。	友達に会えたら、買い物に行きます。	ともだちにあえたら、かいものにいきます。	
\\	「たら(ば)」	友達[ともだち]に 会[あ]えたら、 買い物[かいもの]に 行[い]きます。		
\\	これは、日本語で何
\\	言
\\	か。	これは、日本語で何と言いますか。	これは、にほんごでなんといいますか。	
\\	これは、 日本語[にほんご]で 何[なに]と 言[い]いますか。		
\\	晩ご飯を作
\\	おく。	晩ご飯を作っておく。	ばんごはんをつくっておく。	
\\	「~ておく」 
\\	晩[ばん]ご 飯[はん]を 作[つく]っておく。		
\\	その部屋はあまり静
\\	よ。	その部屋はあまり静かじゃないですよ。	そのへやはあまりしずかじゃないですよ。	
\\	「です」 
\\	その 部屋[へや]はあまり 静[しず]かじゃないですよ。		
\\	勉強
\\	する, 
\\	ちゃ 。	勉強しなくちゃ 。	べんきょうしなくちゃ。	
\\	「なくて」
\\	勉強[べんきょう]しなくちゃ。		
\\	そんなのは、
\\	よ!	そんなのは、あるかよ!	そんなのは、あるかよ!	
\\	そんなのは、あるかよ!		
\\	先生が学校に行
\\	か教えない?	先生が学校に行ったか教えない?	せんせいががっこうにいったかおしえない?	
\\	「か」 
\\	先生[せんせい]が 学校[がっこう]に 行[い]ったか 教[おし]えない?		
\\	マトリックスを見
\\	けど、面白かった。	マトリックスを見たけど、面白かった。	マトリックスをみたけど、おもしろかった。	
\\	「が」 
\\	「けど」	マトリックスを 見[み]たけど、 面白[おもしろ]かった。		
\\	日本に来て、寿司が食
\\	れるようになった。	日本に来て、寿司が食べられるようになった。	にほんにきて、すしがたべられるようになった。	
\\	する 
\\	なる 
\\	に 
\\	「なる」 
\\	「する」 
\\	日本[にっぽん]に 来[き]て、 寿司[すし]が 食[た]べられるようになった。		
\\	ゆっくり話
\\	ください。	ゆっくり話してください。	ゆっくりはなしてください。	
\\	「~ください」- 
\\	「くださる」	ゆっくり 話[はな]してください。		
\\	パーティーに行かなかったの?/うん、時間が
\\	から。	パーティーに行かなかったの?/うん、時間がなかったから。	パーティーにいかなかったの?/うん、じかんがなかったから。	
\\	「から」 
\\	「ので」	パーティーに 行[い]かなかったの?/うん、 時間[じかん]がなかったから。		
\\	代わりに行
\\	。	代わりに行ってあげる。	かわりにいってあげる。	
\\	「あげる」	代[か]わりに 行[い]ってあげる。		
\\	これ、ちょっと見
\\	いい?	これ、ちょっと見ていい?	これ、ちょっとみていい?	
\\	これ、ちょっと 見[み]ていい?		
\\	話を聞
\\	いますか。/ううん、聞
\\	ない。	話を聞いていますか。/ううん、聞いてない。	はなしをきいていますか。/ううん、きいてない。	
\\	「~てる」 
\\	話[はなし]を 聞[き]いていますか。/ううん、 聞[き]いてない。		
\\	何を読
\\	いる?教科書を読
\\	います。	何を読んでいる?教科書を読んでいます。	なにをよんでいる? /きょうかしょをよんでいます。	
\\	「~ている」 
\\	何[なに]を 読[よ]んでいる? 教科書[きょうかしょ]を 読[よ]んでいます。		
\\	みきちゃんが、明日
\\	来る, 
\\	って。	みきちゃんが、明日こないって。	みきちゃんが、あしたこないって。	
\\	「って」 
\\	「て」 
\\	「という」 
\\	みきちゃんが、 明日[あした]こないって。		
\\	無理矢理に部屋に入
\\	うとしている。	無理矢理に部屋に入ろうとしている。	むりやりにへやにはいろうとしている。	
\\	無理矢理[むりやり]に 部屋[へや]に 入[にゅう]ろうとしている。		
\\	みんなが行
\\	なら私も行く。	みんなが行くなら私も行く。	みんながいくならわたしもいく。	
\\	「なら(ば)」	みんなが 行[い]くなら 私[わたし]も 行[い]く。		
\\	学生
\\	のに、彼女は勉強しない。	学生なのに、彼女は勉強しない。	がくせいなのに、かのじょはべんきょうしない。	
\\	「のに」 
\\	学生[がくせい]なのに、 彼女[かのじょ]は 勉強[べんきょう]しない。		
\\	代わりに行
\\	代わりに行ってくれる?	かわりにいってくれる?	
\\	「くれる」	代[か]わりに 行[い]ってくれる?		
\\	友達が父にいいことを教
\\	。	友達が父にいいことを教えてあげた。	ともだちがちちにいいことをおしえてあげた。	
\\	「あげる」	友達[ともだち]が 父[ちち]にいいことを 教[おし]えてあげた。		
\\	先週、ボブに会
\\	よ。	先週、ボブに会いましたよ。	せんしゅう、ボブにあいましたよ。	
\\	「~ます」 
\\	先週[せんしゅう]、ボブに 会[あ]いましたよ。		
\\	てゆうか、もう帰らないとだめですけど。	てゆうか、もう帰らないとだめですけど。	てゆうか、もうかえらないとだめですけど。	
\\	「ゆう」 
\\	「いう」	てゆうか、もう 帰[かえ]らないとだめですけど。		
\\	来年留学するって智子のこと?	来年留学するって智子のこと?	らいねんりゅうがくするってともこのこと?	
\\	「って」 
\\	「て」 
\\	「という」	来年[らいねん] 留学[りゅうがく]するって 智子[さとこ]のこと?		
\\	食
\\	ば病気になるよ。	食べなければ病気になるよ。	たべなければびょうきになるよ。	
\\	「ば」	食[た]べなければ 病気[びょうき]になるよ。		
\\	遠い所に行
\\	。	遠い所に行かないで。	とおいところにいかないで。	
\\	「~ください」- 
\\	「くださる」 
\\	遠[とお]い 所[ところ]に 行[い]かないで。		
\\	友達のおかげで、映画はただで見
\\	。	友達のおかげで、映画はただで見られた。	ともだちのあかげで、えいがはただでみられた。	
\\	「見える」 
\\	「聞こえる」 
\\	友達[ともだち]のおかげで、 映画[えいが]はただで 見[み]られた。		
\\	私が全部食
\\	。	私が全部食べてあげました。	わたしがぜんぶたべてあげました。	
\\	「くれる」 
\\	「あげる」 
\\	私[わたし]が 全部[ぜんぶ] 食[た]べてあげました。		
\\	ちょっと忙
\\	ので、そろそろ失礼します。	ちょっと忙しいので、そろそろ失礼します。	ちょっといそがしいので、そろそろしつれいします。	
\\	「から」 
\\	「ので」	ちょっと 忙[いそが]しいので、そろそろ 失礼[しつれい]します。		
\\	先生
\\	と、きっと年上なんじゃないですか?	先生だと、きっと年上なんじゃないですか?	せんせいだと、きっととしうえなんじゃないですか?	
\\	「と」	先生[せんせい]だと、きっと 年上[としうえ]なんじゃないですか?		
\\	私が友達にプレゼントを
\\	。	私が友達にプレゼントをあげた。	わたしがともだちにプレゼントをあげた。	
\\	「あげる」	私[わたし]が 友達[ともだち]にプレゼントをあげた。		
\\	時間がなかった。/
\\	からパーティーに行かなかったの?	時間がなかった。/だからパーティーに行かなかったの?	じかんがなかった。/だからパーティーにいかなかったの?	
\\	「から」 
\\	「ので」	時間[じかん]がなかった。/だからパーティーに 行[い]かなかったの?		
\\	今、時間が
\\	って、本当?	今、時間がないって、本当?	いま、じかんがないって、ほんとう?	
\\	「って」 
\\	「と」	今[いま]、 時間[じかん]がないって、 本当[ほんとう]?		
\\	友達に会
\\	ば、買い物に行きます。	友達に会えれば、買い物に行きます。	ともだちにあえれば、かいものにいきます。	
\\	「ば」	友達[ともだち]に 会[あ]えれば、 買い物[かいもの]に 行[い]きます。		
\\	友達は何を
\\	する]いるの?昼ご飯を食
\\	いる。	友達は何をしているの?昼ご飯を食べている。	ともだちはなにをしているの?/ひるごはんをたべている。	
\\	「~ている」 
\\	友達[ともだち]は 何[なに]をしているの? 昼[ひる]ご 飯[はん]を 食[た]べている。		
\\	んで、友達に会う時間がない。	なんで、友達に会う時間がない。	なんで、ともだちにあうじかんがない。	
\\	「から」 
\\	「ので」 
\\	なんで、 友達[ともだち]に 会[あ]う 時間[じかん]がない。		
\\	重い荷物が持
\\	。	重い荷物が持てます。	おもいにもつがもてます。	
\\	重[おも]い 荷物[にもつ]が 持[も]てます。		
\\	学校に行
\\	と。	学校に行かないと。	がっこうにいかないと。	
\\	「と」
\\	学校[がっこう]に 行[い]かないと。		
\\	ボールを落
\\	と落ちる。	ボールを落すと落ちる。	ボールをおとすとおちる。	
\\	「と」	ボールを 落[おと]すと 落[お]ちる。		
\\	独身だ
\\	いうのは、嘘だったの?	独身だというのは、嘘だったの?	どくしんだというのは、うそだったの?	
\\	「いう」 
\\	独身[どくしん]だというのは、 嘘[うそ]だったの?		
\\	お父さんが
\\	が壊れた。	お父さんがくれた時計が壊れた。	おとうさんがくれたとけいがこわれた。	
\\	「~ください」- 
\\	「くださる」	お 父[とう]さんがくれた 時計[とけい]が 壊[こわ]れた。		
\\	全部食
\\	くれますか。	全部食べないでくれますか。	ぜんぶたべないでくれますか。	
\\	「くれる」 
\\	「もらえる」 
\\	全部[ぜんぶ] 食[た]べないでくれますか。		
\\	全部食
\\	ほしいんだけど・・・。	全部食べてほしいんだけど・・・。	ぜんぶたべてほしいんだけど・・・。	
\\	「欲しい」	全部[ぜんぶ] 食[た]べてほしいんだけど・・・。		
\\	もうお金が
\\	って。	もうお金がないって。	もうおかねがないって。	
\\	「って」 
\\	「と」	もうお 金[かね]がないって。		
\\	友達
\\	プレゼントが来た。	友達からプレゼントが来た。	ともだちからプレゼントがきた。	
\\	「から」 
\\	「ので」	友達[ともだち]からプレゼントが 来[き]た。		
\\	千円を貸
\\	くれる?	千円を貸してくれる?	せんえんをかしてくれる。	
\\	「くれる」 
\\	「もらえる」	千[せん] 円[えん]を 貸[か]してくれる?		
\\	運動しているから、強
\\	なる。	運動しているから、強くなる。	うんどうしているから、つよくなる。	
\\	する 
\\	なる 
\\	に 
\\	「なる」 
\\	運動[うんどう]しているから、 強[つよ]くなる。		
\\	今日は何を
\\	する, 
\\	か?/テーマパークに行
\\	今日は何をしようか?/テーマパークに行こう!	あしたはなにをしようか?/テーマパークにいこう!	
\\	今日[きょう]は 何[なに]をしようか?/テーマパークに 行[い]こう!		
\\	デパートに行
\\	が、何も欲しくなかったです。	デパートに行きましたが、何も欲しくなかったです。	デパートにいきましたが、なにもほしくなかったです。	
\\	「が」 
\\	「けど」	デパートに 行[い]きましたが、 何[なに]も 欲[ほ]しくなかったです。		
\\	そんなことは
\\	~うる]	そんなことはありえる	そんなことはありえる	
\\	「ある得る」, 
\\	そんなことはありえる		
\\	そうゆうことじゃないって!	そうゆうことじゃないって!	そうゆうことじゃないって!	
\\	「ゆう」 
\\	「いう」	そうゆうことじゃないって!		
\\	学生がこれを先生に
\\	。	学生がこれを先生にあげる。	がくせいがこれをせんせいにあげる。	
\\	「あげる」	学生[がくせい]がこれを 先生[せんせい]にあげる。		
\\	アリス
\\	って、すごくいい人でしょ?	アリスって、すごくいい人でしょ?	アリスって、すごくいいひとでしょ?	
\\	「って」 
\\	「と」 
\\	「は」	アリ ス[す]って、すごくいい 人[ひと]でしょ?		
\\	暇
\\	ら、遊びに行くよ。	暇だったら、遊びに行くよ。	ひまだったら、あそびにいくよ。	
\\	「たら(ば)」	暇[ひま]だったら、 遊[あそ]びに 行[い]くよ。		
\\	あの人は、本当に
\\	食べる。	あの人は、本当に何でも食べる。	あのひとは、ほんとうになんでもたべる。	
\\	あの 人[ひと]は、 本当[ほんとう]に 何[なに]でも 食[た]べる。		
\\	それでいいと思うなら、そう
\\	する]なよ。	それでいいと思うなら、そうしなよ。	それでいいとおもうなら、そうしなよ。	
\\	「~なさい」 
\\	それでいいと 思[おも]うなら、そうしなよ。		
\\	早く寝
\\	はなりませんでした。	早く寝てはなりませんでした。	はやくねてはなりませんでした。	
\\	「だめ」, 「いけない」, 
\\	「ならない」 
\\	早[はや]く 寝[ね]てはなりませんでした。		
\\	私の部屋は、きれい
\\	、静
\\	、とても好き。	私の部屋は、きれいで、静かで、とても好き。	わたしのへやは、きれいで、しずかで、とてもすき。	
\\	私[わたし]の 部屋[へや]は、きれいで、 静[しず]かで、とても 好[す]き。		
\\	時間がないなら、明日でもいいよ。	もし時間がないなら、明日でもいいよ。	もしじかんがないなら、あしたでもいいよ。	
\\	「もし」 
\\	なら	もし 時間[じかん]がないなら、 明日[あした]でもいいよ。		
\\	智子は来年、海外に行
\\	んだって。	智子は来年、海外に行くんだって。	ともこはらいねん、かいがいにいくんだって。	
\\	「って」 
\\	「と」	智子[さとこ]は 来年[らいねん]、 海外[かいがい]に 行[い]くんだって。		
\\	デパートに行
\\	が、いい物がたくさんありました。	デパートに行きましたが、いい物がたくさんありました。	デパートにいきましたが、いいものがたくさんありました。	
\\	「が」 
\\	「けど」
\\	デパートに 行[い]きましたが、いい 物[もの]がたくさんありました。		
\\	明日、映画を見
\\	行く。	明日、映画を見に行く。	あした、えいがをみにいく。	
\\	明日[あした]、 映画[えいが]を 見[み]に 行[い]く。		
\\	誰が盗んだのか、
\\	知りませんか。	誰が盗んだのか、誰か知りませんか。	だれがぬすんだのか、だれかしりませんか。	
\\	誰[だれ]が 盗[ぬす]んだのか、 誰[だれ]か 知[し]りませんか。		
\\	それを食
\\	はだめ!	それを食べてはだめ!	それをたべてはだめ!	
\\	「だめ」, 「いけない」, 
\\	「ならない」 
\\	ては
\\	それを 食[た]べてはだめ!		
\\	けど、彼がまだ好きなの。	だけど、彼がまだ好きなの。	だけど、かれがまだすきなの。	
\\	「が」 
\\	「けど」	だけど、 彼[かれ]がまだ 好[す]きなの。		
\\	食堂に行
\\	、昼ご飯を食
\\	、昼寝をした。	食堂に行って、昼ご飯を食べて、昼寝をした。	しょくどうにいって、ひるごはんをたべて、ひるねをした。	
\\	食堂[しょくどう]に 行[い]って、 昼[ひる]ご 飯[はん]を 食[た]べて、 昼寝[ひるね]をした。		
\\	学校に行
\\	と友達と会えないよ。	学校に行かないと友達と会えないよ。	がっこうにいかないとともだちとあえないよ。	
\\	「と」	学校[がっこう]に 行[い]かないと 友達[ともだち]と 会[あ]えないよ。		
\\	宿題を
\\	する, 
\\	はいけなかった。	宿題をしなくてはいけなかった。	しゅくだいをしなくてはいけなかった。	
\\	ては
\\	いけない
\\	宿題[しゅくだい]をしなくてはいけなかった。		
\\	ずっと一緒に
\\	たい。	ずっと一緒にいたい。	ずっといっしょにいたい。	
\\	「たい」	ずっと 一緒[いっしょ]にいたい。		
\\	名前はアリス。	あたしの名前はアリス。	あたしのなまえはアリス。	
\\	あたしの 名前[なまえ]はアリス。		
\\	どうして彼が好きなの?優
\\	し、かっこ
\\	し、面白いから。	どうして彼が好きなの?優しいし、かっこいいし、面白いから。	どうしてかれがすきなの?やさしいし、かっこいいし、おもしろいから。	
\\	「し」	どうして 彼[かれ]が 好[す]きなの? 優[やさ]しいし、かっこいいし、 面白[おもしろ]いから。		
\\	遠い所に行
\\	くれ。	遠い所に行かないでくれ。	とおいところにいかないでくれ。	
\\	「~ください」- 
\\	「くださる」 
\\	「くれ」	遠[とお]い 所[ところ]に 行[い]かないでくれ。		
\\	友達に聞
\\	けど、知らなかった。	友達に聞いたけど、知らなかった。	ともだちにきいたけど、しらなかった。	
\\	「が」 
\\	「けど」	友達[ともだち]に 聞[き]いたけど、 知[し]らなかった。		
\\	消しゴムを貸
\\	くれ。	消しゴムを貸してくれ。	けしゴムをかしてくれ。	
\\	「~ください」- 
\\	「くださる」-
\\	「くれ」	消しゴム[けしごむ]を 貸[か]してくれ。		
\\	先生から今日は授業がない
\\	聞
\\	んだけど。	先生から今日は授業がないと聞いたんだけど。	せんせいからきょうはじゅぎょうがないときいたんだけど。	
\\	先生[せんせい]から 今日[きょう]は 授業[じゅぎょう]がないと 聞[き]いたんだけど。		
\\	時間が
\\	からパーティーに行きませんでした。	時間がなかったからパーティーに行きませんでした。	じかんがなかったからパーティーにいきませんでした。	
\\	「から」 
\\	「ので」	時間[じかん]がなかったからパーティーに 行[い]きませんでした。		
\\	全部飲
\\	も構わないよ。	全部飲んでも構わないよ。	ぜんぶのんでもかまわないよ。	
\\	全部[ぜんぶ] 飲[の]んでも 構[かま]わないよ。		
\\	そういう人と一緒に仕事をするのは、嫌だよね。	そういう人と一緒に仕事をするのは、嫌だよね。	そういうひとといっしょにしごとをするのは、いやだよね。	
\\	「いう」 
\\	「こう」、「そう」、「ああ」、
\\	「どう」	そういう 人[ひと]と 一緒[いっしょ]に 仕事[しごと]をするのは、 嫌[いや]だよね。		
\\	ああいう人と結婚できたら、幸せになれると思います。	ああいう人と結婚できたら、幸せになれると思います。	ああいうひととけっこんできたら、しあわせになれるとおもいます。	
\\	「いう」 
\\	「こう」、「そう」、「ああ」、
\\	「どう」	ああいう 人[ひと]と 結婚[けっこん]できたら、 幸[しあわ]せになれると 思[おも]います。		
\\	ご飯を食
\\	きゃ。	ご飯を食べなきゃ。	ごはんをたべなきゃ。	
\\	「なければ」
\\	ご 飯[はん]を 食[た]べなきゃ。		
\\	ルミネ
\\	いうデパートはどこにあるか、知っていますか?	ルミネというデパートはどこにあるか、知っていますか?	ルミネというデパートはどこにあるか、しっていますか?	
\\	「いう」 
\\	ルミネというデパートはどこにあるか、 知[し]っていますか?		
\\	彼は何を言
\\	かわからない。	彼は何を言ったかわからない。	かれはなにをいったかわからない。	
\\	「か」 
\\	彼[かれ]は 何[なに]を 言[い]ったかわからない。		
\\	犬に餌を
\\	あげる]?	犬に餌をやった?	いぬにえさをやった?	
\\	「やる」 
\\	「あげる」	犬[いぬ]に 餌[えさ]をやった?		
\\	時間が
\\	んでパーティーに行かなかった。	時間がなかったんでパーティーに行かなかった。	じかんがなかったんでパーティーにいかなかった。	
\\	「から」 
\\	「ので」 
\\	時間[じかん]がなかったんでパーティーに 行[い]かなかった。		
\\	図書館はどこですか。/図書館
\\	なら、あそこです。	図書館はどこですか。/図書館なら、あそこです。	としょかんはどこですか。/としょかんなら、あそこです。	
\\	「なら(ば)」	図書館[としょかん]はどこですか。/ 図書館[としょかん]なら、あそこです。		
\\	先生が教
\\	んですか。	先生が教えてくれるんですか。	せんせいがおしえてくれるんですか。	
\\	「くれる」 
\\	「あげる」 
\\	先生[せんせい]が 教[おし]えてくれるんですか。		
\\	彼は、これは何
\\	言いましたか。	彼は、これは何だと言いましたか。	かれは、これはなんだといいましたか。	
\\	彼[かれ]は、これは 何[なに]だと 言[い]いましたか。		
\\	この中から
\\	を選ぶの。	この中からどれかを選ぶの。	このなかからどれかをえらぶの。	
\\	この 中[なか]からどれかを 選[えら]ぶの。		
\\	毎日、肉を食
\\	ようになった。	毎日、肉を食べるようになった。	まいにち、にくをたべるようになった。	
\\	する 
\\	なる 
\\	に 
\\	「なる」 
\\	「する」 
\\	毎日[まいにち]、 肉[にく]を 食[た]べるようになった。		
\\	お金が
\\	ある]らいいね。	お金があったらいいね。	おかねがったらいいね。	
\\	「たら(ば)」	お 金[かね]があったらいいね。		
\\	友達からプレゼントを
\\	。	友達からプレゼントをもらった。	ともだちからプレゼントをもらった。	
\\	「もらう」	友達[ともだち]からプレゼントをもらった。		
\\	カレーを食べよう
\\	思ったけど、食べる時間がなかった。	カレーを食べようと思ったけど、食べる時間がなかった。	カレーをたべようとおもったけど、たべるじかんがなかった。	
\\	カレーを 食[た]べようと 思[おも]ったけど、 食[た]べる 時間[じかん]がなかった。		
\\	私は、アリス
\\	言
\\	。	私は、アリスと言います。	わたしは、アリスといいます。	
\\	私[わたし]は、アリスと 言[い]います。		
\\	久しぶりに彼の声が聞
\\	。	久しぶりに彼の声が聞けた。	ひさしぶりにかれのこえがきけた。	
\\	「見える」 
\\	「聞こえる」 
\\	久[ひさ]しぶりに 彼[かれ]の 声[こえ]が 聞[き]けた。		
\\	彼は高校生
\\	聞いたけど、信じられない。	彼は高校生だと聞いたけど、信じられない。	かれはこうこうせいだときいたけど、しんじられない。	
\\	彼[かれ]は 高校生[こうこうせい]だと 聞[き]いたけど、 信[しん]じられない。		
\\	ここに入
\\	はいけません。	ここに入ってはいけません。	ここにはいってはいけません。	
\\	「だめ」, 「いけない」, 
\\	「ならない」 
\\	ここに 入[はい]ってはいけません。		
\\	鈴木さんはどこですか。/もう、家に帰
\\	いる。	鈴木さんはどこですか。/もう、家に帰っている。	すずきさんはどこですか。/もう{うち
\\	いえ}にかえっている。	
\\	鈴木[すずき]さんはどこですか。/もう、 家[いえ]に 帰[かえ]っている。		
\\	ここに座
\\	なさい。	ここに座りなさい。	ここにすわりなさい。	
\\	「~なさい」 
\\	ここに 座[すわ]りなさい。		
\\	お酒を飲
\\	みましたが、すごく眠くなりました。	お酒を飲んでみましたが、すごく眠くなりました。	おさけをのんでみましたが、すごくねむくなりました。	
\\	お 酒[さけ]を 飲[の]んでみましたが、すごく 眠[ねむ]くなりました。		
\\	昼ご飯は、
\\	いいです。	昼ご飯は、どこでもいいです。	ひるごはんは、どこでもいいです。	
\\	昼[ひる]ご 飯[はん]は、どこでもいいです。		
\\	それを
\\	それをくれる?	それをくれる?	
\\	「~ください」- 
\\	「くださる」	それをくれる?		
\\	何を読
\\	る?/教科書を読
\\	います。	何を読んでる?/教科書を読んでいます。	なにをよんでる?/きょうかしょをよんでいます。	
\\	「~てる」 
\\	何[なに]を 読[よ]んでる?/ 教科書[きょうかしょ]を 読[よ]んでいます。		
\\	面白くない映画は見
\\	。	面白くない映画は見ません。	おもしろくないえいがほみません。	
\\	「~ます」 
\\	面白[おもしろ]くない 映画[えいが]は 見[み]ません。		
\\	鉛筆を家に持
\\	くる?	鉛筆を家に持ってくる?	えんぴつを{うち
\\	いえ}にもってくる。	
\\	(行く、来る) 
\\	鉛筆[えんぴつ]を 家[いえ]に 持[も]ってくる?		
\\	私が友達にプレゼントを
\\	。	私が友達にプレゼントをもらった。	わたしがともだちにプレゼントをもらった。	
\\	「もらう」	私[わたし]が 友達[ともだち]にプレゼントをもらった。		
\\	日本語をずっと前から勉強
\\	きて、結局はやめた。	日本語をずっと前から勉強してきて、結局はやめた。	にほんをずっとまえからべんきょうしてきて、けっきょくはやめた。	
\\	(行く、来る) 
\\	日本語[にほんご]をずっと 前[まえ]から 勉強[べんきょう]してきて、 結局[けっきょく]はやめた。		
\\	時間が
\\	、映画を見
\\	。	時間がありまして、映画を見ました。	じかんがありまして、えいがをみました。	
\\	時間[じかん]がありまして、 映画[えいが]を 見[み]ました。		
\\	何を
\\	する]たいですか。	何をしたいですか。	なにをしたいですか。	
\\	「たい」	何[なに]をしたいですか。		
\\	明日は何を食
\\	か?/カレーを食
\\	明日は何を食べようか?/カレーを食べよう!	あしたはなにをたべようか?/カレーをたべよう!	
\\	明日[あした]は 何[なに]を 食[た]べようか?/カレーを 食[た]べよう!		
\\	大きい縫い包みが欲
\\	大きい縫い包みが欲しい!	おおきいぬいぐるみがほしい!	
\\	「欲しい」	大[おお]きい 縫[ぬ]い 包[つつ]みが 欲[ほ]しい!		
\\	これは、先生が
\\	。	これは、先生がくれた。	これは、せんせいがくれた。	
\\	「くれる」	これは、 先生[せんせい]がくれた。		
\\	名前はボブ
\\	。	俺の名前はボブだ。	おれのなまえはボブだ。	
\\	俺[おれ]の 名前[なまえ]はボブだ。		
\\	私は、ハンバーガーとサラダ
\\	します。	私は、ハンバーガーとサラダにします。	わたしは、ハンバーガーとサラダにします。	
\\	する 
\\	なる 
\\	に 
\\	「なる」 
\\	「する」 
\\	私[わたし]は、ハンバーガーとサラダにします。		
\\	ケーキ、食
\\	たくないの?	ケーキ、食べたくないの?	ケーキ、たべたくないの?	
\\	「たい」	ケーキ、 食[た]べたくないの?		
\\	明日は何を食
\\	/カレーを食
\\	明日は何を食べましょうか?/カレーを食べましょう, 
\\	あしたはなにをたべましょうか?/カレーをたべましょう!	
\\	明日[あした]は 何[なに]を 食[た]べましょうか?/カレーを 食[た]べましょう,!		
\\	高い物を買
\\	くれる?	高い物を買わないでくれる?	たかいものをかわないでくれる?	
\\	「くれる」 
\\	「もらえる」 
\\	高[たか]い 物[もの]を 買[か]わないでくれる?		
\\	車を買
\\	の?	車を買ってくれるの?	くるまをかってくれるの?	
\\	「くれる」	車[くるま]を 買[か]ってくれるの?		
\\	ちょっと静かに
\\	する]くれない?	ちょっと静かにしてくれない?	ちょっとしずかにしてくれない?	
\\	「くれる」 
\\	「もらえる」	ちょっと 静[しず]かにしてくれない?		
\\	今週末は、
\\	行かなかった。	今週末は、どこにも行かなかった。	こんしゅうまつは、どこにもいかなかった。	
\\	今週[こんしゅう] 末[まつ]は、どこにも 行[い]かなかった。		
\\	今、時間が
\\	って聞いたんだけど、本当?	今、時間がないって聞いたんだけど、本当?	いま、じかんがないってきいたんだけど、ほんとう?	
\\	「って」 
\\	「と」	今[いま]、 時間[じかん]がないって 聞[き]いたんだけど、 本当[ほんとう]?		
\\	というか、もう帰らないとだめですけど。	というか、もう帰らないとだめですけど。	というか、もうかえらないとだめですけど。	
\\	「という」 
\\	「か」 
\\	というか、もう 帰[かえ]らないとだめですけど。		
\\	お父さんが言った。
\\	「それをください」とお父さんが言った。	「それをください」とおとうさんがいった。	
\\	「~ください」- 
\\	「くださる」	「それをください」とお 父[とう]さんが 言[い]った。		
\\	好きに
\\	する, 
\\	。	好きにしろ。	すきにしろ。	
\\	好[す]きにしろ。		
\\	「今日は授業がない」
\\	先生から聞
\\	んだけど。	「今日は授業がない」と先生から聞いたんだけど。	「きょうはじゅぎょうがない」とせんせいからきいたんだけど。	
\\	今日[きょう]は 授業[じゅぎょう]がない」と 先生[せんせい]から 聞[き]いたんだけど。		
\\	彼の日本語が上手
\\	なった。	彼の日本語が上手になった。	かれのにほんががじょうずになった。	
\\	する 
\\	なる 
\\	に 
\\	「なる」 
\\	「する」 
\\	彼[かれ]の 日本語[にほんご]が 上手[じょうず]になった。		
\\	スプーンを
\\	。	スプーンをちょうだい。	スプーンをちょうだい。	
\\	「~ちょうだい」 
\\	スプーンをちょうだい。		
\\	ここにあるレストランは
\\	美味しくない。	ここにあるレストランはどれも美味しくない。	ここにあるレストランはどれもおいしくない。	
\\	ここにあるレストランはどれも 美味[おい]しくない。		
\\	全部食
\\	もいいよ。	全部食べてもいいよ。	ぜんぶたべてもいいよ。	
\\	全部[ぜんぶ] 食[た]べてもいいよ。		
\\	ということは]、みきちゃんは、今彼氏がいない
\\	ということ]?	てことは、みきちゃんは、今彼氏がいないてこと?	てことは、みきちゃんは、いまかれしがいないてこと?	
\\	「って」 
\\	「て」 
\\	「という」	てことは、みきちゃんは、 今[こん] 彼氏[かれし]がいないてこと?		
\\	楽
\\	ば、私も行かない。	楽しくなければ、私も行かない。	たのしくなければ、わたしもいかない。	
\\	「ば」	楽[たの]しくなければ、 私[わたし]も 行[い]かない。		
\\	「友達」は、英語で
\\	いう意味です。	「友達」は、英語で
\\	という意味です。	「ともだち」は、えいがで
\\	といういみです。	
\\	「いう」 
\\	友達[ともだち]」は、 英語[えいご]で
\\	という 意味[いみ]です。		
\\	全部食
\\	もいいよ。	全部食べなくてもいいよ。	ぜんぶたべなくてもいいよ。	
\\	全部[ぜんぶ] 食[た]べなくてもいいよ。		
\\	お酒を飲
\\	うとしたが、奥さんが止めた。	お酒を飲もうとしたが、奥さんが止めた。	おさけをのもうとしたが、おくさんがとめた。	
\\	お 酒[さけ]を 飲[の]もうとしたが、 奥[おく]さんが 止[と]めた。		
\\	友達のおかげで、映画をただで見
\\	ことができた。	友達のおかげで、映画をただで見ることができた。	ともだちのおかげで、えいがをただでみることができた。	
\\	「見える」 
\\	「聞こえる」 
\\	友達[ともだち]のおかげで、 映画[えいが]をただで 見[み]ることができた。		
\\	食
\\	たくなかったけど食
\\	たくなった。	食べたくなかったけど食べたくなった。	たべたくなかったけどたべtくなった。	
\\	「たい」	食[た]べたくなかったけど 食[た]べたくなった。		
\\	たくさん食
\\	と太るよ。	たくさん食べると太るよ。	たくさんたべるとふとるよ。	
\\	「と」	たくさん 食[た]べると 太[ふと]るよ。		
\\	お好み焼きを始めて食
\\	みたけど、とてもおいしかった!	お好み焼きを始めて食べてみたけど、とてもおいしかった!	おこのみやきをはじめてたべてみたけど、とてもおいしかった!	
\\	お 好み焼[このみや]きを 始[はじ]めて 食[た]べてみたけど、とてもおいしかった!		
\\	映画を見たり、本を読
\\	。	映画を見たり、本を読んだりしない。	えいがをみたり、ほんをよんだりしない。	
\\	「~たりする」	映画[えいが]を 見[み]たり、 本[ほん]を 読[よ]んだりしない。		
\\	先に行
\\	いるよ。/美恵ちゃんは、もう来
\\	いるよ。	先に行っているよ。/美恵ちゃんは、もう来ているよ。	さきにいっているよ。/みえちゃんは、もうきているよ。	
\\	先[さき]に 行[い]っているよ。/ 美恵[みえ]ちゃんは、もう 来[き]ているよ。		
\\	毎日学校に行
\\	とだめです。	毎日学校に行かないとだめです。	まいにちがっこうにいかないとだめです。	
\\	と
\\	だめ
\\	毎日[まいにち] 学校[がっこう]に 行[い]かないとだめです。		
\\	そんなことは
\\	~える]。	そんなことはありうる。	そんなことはありうる。	
\\	「ある得る」, 
\\	そんなことはありうる。		
\\	毎日、肉を食
\\	ようにする。	毎日、肉を食べるようにする。	まいにち、にくをたべるようにする。	
\\	する 
\\	なる 
\\	に 
\\	「なる」 
\\	「する」 
\\	毎日[まいにち]、 肉[にく]を 食[た]べるようにする。		
\\	死
\\	じゃだめだよ!	死んじゃだめだよ!	しんじゃだめだよ。	
\\	「では」
\\	「だめ」
\\	死[し]んじゃだめだよ!		
\\	ここは静
\\	んで、とても穏やかです。	ここは静かなんで、とても穏やかです。	ここはしずかなんで、とてもおだやかです。	
\\	「から」 
\\	「ので」 
\\	ここは 静[しず]かなんで、とても 穏[おだ]やかです。		
\\	映画を見たり、本を読
\\	。	映画を見たり、本を読んだりしなかった。	えいがをみたり、ほんをよんだりしなかった。	
\\	「~たりする」	映画[えいが]を 見[み]たり、 本[ほん]を 読[よ]んだりしなかった。		
\\	鉛筆を持
\\	いる?	鉛筆を持っている?	えんぴつをもっている。	
\\	(行く、来る) 
\\	鉛筆[えんぴつ]を 持[も]っている?		
\\	晩ご飯を食
\\	ね。	晩ご飯を食べませんでしたね。	ばんごはんをたべませんでしたね。	
\\	「~ます」 
\\	晩[ばん]ご 飯[はん]を 食[た]べませんでしたね。		
\\	日本語で話
\\	くれ。	日本語で話してくれ。	にほんごではなしてくれ。	
\\	「~ください」- 
\\	「くださる」 
\\	「くれ」	日本語[にほんご]で 話[はな]してくれ。		
\\	車を買
\\	よ。	車を買ってあげるよ。	くるまをかってあげるよ。	
\\	「あげる」	車[くるま]を 買[か]ってあげるよ。		
\\	私は有名な人
\\	なる。	私は有名な人になる。	わたしはゆうめいなひとになる。	
\\	する 
\\	なる 
\\	に 
\\	「なる」 
\\	「する」 
\\	私[わたし]は 有名[ゆうめい]な 人[ひと]になる。		
\\	お父さんは、早く帰
\\	きました。	お父さんは、早く帰ってきました。	おとうさんは、はやくかえってきました。	
\\	(行く、来る) 
\\	お 父[とう]さんは、 早[はや]く 帰[かえ]ってきました。		
\\	新しいデパートに行
\\	みる。	新しいデパートに行ってみる。	あたらしいデパートにいってみる。	
\\	新[あたら]しいデパートに 行[い]ってみる。		
\\	地下に入って、富士山が見
\\	なった。	地下に入って、富士山が見えなくなった。	ちかにはいって、ふじさんがみえなくなった。	
\\	する 
\\	なる 
\\	に 
\\	「なる」 
\\	「する」 
\\	地下[ちか]に 入[はい]って、 富士山[ふじさん]が 見[み]えなくなった。		
\\	友達は何を
\\	する, 
\\	るの?/昼ご飯を食
\\	る。	友達は何をしてるの?/昼ご飯を食べてる。	ともだちはなにをしてるの?/ひるごはんをたべている。	
\\	「~てる」 
\\	友達[ともだち]は 何[なに]をしてるの?/ 昼[ひる]ご 飯[はん]を 食[た]べてる。		
\\	映画を見
\\	り、本を読
\\	り、昼寝
\\	りする。	映画を見たり、本を読んだり、昼寝したりする。	えいがをみたり、ほんをやんだり、ひるねしたりする。	
\\	「~たりする」	映画[えいが]を 見[み]たり、 本[ほん]を 読[よ]んだり、 昼寝[ひるね]したりする。		
\\	この大学の授業は簡単
\\	り、難
\\	りする。	この大学の授業は簡単だったり、難しかったりする。	このだいがくのじゅぎょうはかんたんだったり、むずかしかったりする。	
\\	「~たりする」	この 大学[だいがく]の 授業[じゅぎょう]は 簡単[かんたん]だったり、 難[むずか]しかったりする。		
\\	銀行に行
\\	らどうですか。	銀行に行ったらどうですか。	ぎんこうにいったらどうですか。	
\\	「ば」 
\\	「たら」 
\\	銀行[ぎんこう]に 行[い]ったらどうですか。		
\\	この質問の答えは、
\\	分かる。	この質問の答えは、誰でも分かる。	このしつもんのこたえは、だれでもわかる。	
\\	この 質問[しつもん]の 答[こた]えは、 誰[だれ]でも 分[わ]かる。		
\\	毎日学校に行
\\	はなりません。	毎日学校に行かなくてはなりません。	まいにちがっこうにいかなくてはなりません。	
\\	ては
\\	ならない
\\	毎日[まいにち] 学校[がっこう]に 行[い]かなくてはなりません。		
\\	夜、遅くまで電話
\\	はならない。	夜、遅くまで電話してはならない。	よる、おそくまででんわしてはならない。	
\\	「だめ」, 「いけない」, 
\\	「ならない」 
\\	夜[よる]、 遅[おそ]くまで 電話[でんわ]してはならない。		
\\	楽
\\	ば、私も行く。	楽しければ、私も行く。	たのしければ、わたしもいく。	
\\	「ば」	楽[たの]しければ、 私[わたし]も 行[い]く。		
\\	宿題をチェック
\\	する]もらいたかったけど、時間がなくて無理だった。	宿題をチェックしてもらいたかったけど、時間がなくて無理だった。	しゅくだいをチェックしてもらいたかったけど、じかんがなくてむりだった。	
\\	「もらう」	宿題[しゅくだい]をチェックしてもらいたかったけど、 時間[じかん]がなくて 無理[むり]だった。		
\\	こんなのを本当に食
\\	こんなのを本当に食べるか?	こんなのをほんとうにたべるか?	
\\	こんなのを 本当[ほんとう]に 食[た]べるか?		
\\	晩ご飯を作
\\	とく。	晩ご飯を作っとく。	ばんごはんをつくっとく。	
\\	「~とく」 
\\	晩[ばん]ご 飯[はん]を 作[つく]っとく。		
\\	ここに来
\\	ください。	ここに来ないでください。	ここにこないでください。	
\\	「~ください」- 
\\	「くださる」	ここに 来[こ]ないでください。		
\\	家に帰
\\	ら、誰もいなかった。	家に帰ったら、誰もいなかった。	いえにかえったら、だれもいなかった。	
\\	「たら(ば)」	家[いえ]に 帰[かえ]ったら、 誰[だれ]もいなかった。		
\\	え?何
\\	って?	え?何だって?	え?なんだって?	
\\	「って」 
\\	「と」	え? 何[なに]だって?		
\\	こんなのを本当に食
\\	こんなのを本当に食べる?	こんなのをほんとうにたべる?	
\\	こんなのを 本当[ほんとう]に 食[た]べる?		
\\	昨日何を食
\\	か忘れた。	昨日何を食べたか忘れた。	きのうなにをたべたかわすれた。	
\\	「か」 
\\	昨日[きのう] 何[なに]を 食[た]べたか 忘[わす]れた。		
\\	これは何だ
\\	言いましたか。	これは何だと言いましたか。	これはなんだといいましたか。	
\\	これは 何[なに]だと 言[い]いましたか。		
\\	アリスさんが言
\\	なら問題ないよ。	アリスさんが言うなら問題ないよ。	アリスさんがいうならもんだいないよ。	
\\	「なら(ば)」	アリスさんが 言[い]うなら 問題[もんだい]ないよ。		
\\	部屋を綺麗に
\\	する]ほしいのよ。	部屋を綺麗にしてほしいのよ。	へやをきれいにしてほしいのよ。	
\\	「欲しい」	部屋[へや]を 綺麗[きれい]にしてほしいのよ。		
\\	イタリア料理を食べに行
\\	。/すみません。ちょっと、お腹がいっぱいです。	イタリア料理を食べに行きませんか。/すみません。ちょっと、お腹がいっぱいです。	イタリアりょうりをたべにいきませんか。/すみません。ちょっと、おなかがいっぱいです。	
\\	イタリア 料理[りょうり]を 食[た]べに 行[い]きませんか。/すみません。ちょっと、お 腹[なか]がいっぱいです。		
\\	主人公が犯人だった
\\	いうのが一番面白かった。	主人公が犯人だったというのが一番面白かった。	しゅじんこうがはんにんだったというのがいちばんおもしろかった。	
\\	「いう」 
\\	主人公[しゅじんこう]が 犯人[はんにん]だったというのが 一番[いちばん] 面白[おもしろ]かった。		
\\	先生が学校に行
\\	か行
\\	か知らない。	先生が学校に行ったか行かなかったか知らない。	せんせいががっこうにいったかいかなかったかしらない。	
\\	「か」 
\\	先生[せんせい]が 学校[がっこう]に 行[い]ったか 行[い]かなかったか 知[し]らない。		
\\	そんなことは
\\	。	そんなことはありえない。	そんなことはありえない。	
\\	「ある得る」, 
\\	そんなことはありえない。		
\\	名前はボブ
\\	。	僕の名前はボブだ。	ぼくのなまえはボブだ。	
\\	僕[ぼく]の 名前[なまえ]はボブだ。		
\\	ここは静
\\	ので、とても穏やかです。	ここは静かなので、とても穏やかです。	ここはすずかなので、とてもおだやかです。	
\\	「から」 
\\	「ので」	ここは 静[しず]かなので、とても 穏[おだ]やかです。		
\\	名前はキム
\\	。	僕の名前はキムです。	ぼくのなまえはキムです。	
\\	僕[ぼく]の 名前[なまえ]はキムです。		
\\	どうしてパーティーに行きませんでしたか。/時間が
\\	からです。	どうしてパーティーに行きませんでしたか。/時間がなかったからです。	どうしてパーティーにいきませんでしたか。/じかんがなかったからです。	
\\	「から」 
\\	「ので」	どうしてパーティーに 行[い]きませんでしたか。/ 時間[じかん]がなかったからです。		
\\	まだ一杯あるから、たくさん食
\\	な。	まだ一杯あるから、たくさん食べな。	まだいっぱいあるから、たくさんたべな。	
\\	「~なさい」 
\\	まだ 一杯[いっぱい]あるから、たくさん 食[た]べな。		
\\	先生が学校に行
\\	かどうか知らない。	先生が学校に行ったかどうか知らない。	せんせいががっこうにいったかどうかしらない。	
\\	「どうか」 
\\	先生[せんせい]が 学校[がっこう]に 行[い]ったかどうか 知[し]らない。		
\\	消しゴムを貸
\\	。	消しゴムを貸して。	けしゴムをかして。	
\\	「~ください」- 
\\	「くださる」 
\\	消しゴム[けしごむ]を 貸[か]して。		
\\	私は医者
\\	なった。	私は医者になった。	わたしはいしゃになった。	
\\	する 
\\	なる 
\\	に 
\\	「なる」 
\\	「する」 
\\	私[わたし]は 医者[いしゃ]になった。		
\\	一年間練習したから、ピアノが弾
\\	ようになった。	一年間練習したから、ピアノが弾けるようになった。	いちねんかんれんしゅうしたから、ピアノがひけるようになった。	
\\	する 
\\	なる 
\\	に 
\\	「なる」 
\\	「する」 
\\	一 年間[ねんかん] 練習[れんしゅう]したから、ピアノが 弾[はじ]けるようになった。		
\\	友達がプレゼントを私に
\\	。	友達がプレゼントを私にくれた。	ともだちがプレゼントをわたしにくれた。	
\\	「くれる」 
\\	「あげる」 
\\	友達[ともだち]がプレゼントを 私[わたし]にくれた。		
\\	友達
\\	プレゼントが来た。	友達だからプレゼントが来た。	ともだちだからプレゼントがきた。	
\\	「から」 
\\	「ので」	友達[ともだち]だからプレゼントが 来[き]た。		
\\	はい、はい、
\\	。	はい、はい、分かった、分かった。	はい、はい、わかった、わかった。	
\\	はい、はい、 分[わ]かった、 分[わ]かった。		
\\	お酒は好きというか、ないと生きていけない。	お酒は好きというか、ないと生きていけない。	おさけはすきというか、ないといきていけない。	
\\	「という」 
\\	「か」 
\\	お 酒[さけ]は 好[す]きというか、ないと 生[い]きていけない。		
\\	毎日、勉強を避
\\	うとする。	毎日、勉強を避けようとする。	まいにち、べんきょうをさけようとする。	
\\	毎日[まいにち]、 勉強[べんきょう]を 避[さ]けようとする。		
\\	子犬はとても好
\\	。	子犬はとても好きです。	こいぬはとてもすきです。	
\\	「です」 
\\	子犬[こいぬ]はとても 好[す]きです。		
\\	早く寝
\\	うとしたけど、結局は徹夜した。	早く寝ようとしたけど、結局は徹夜した。	はやくねようとしたけど、けっきょくはてつやした。	
\\	早[はや]く 寝[ね]ようとしたけど、 結局[けっきょく]は 徹夜[てつや]した。		
\\	アメリカに行
\\	ら、たくさん太りました。	アメリカに行ったら、たくさん太りました。	アメリカにいったら、たくさんふとりました。	
\\	「たら(ば)」	アメリカに 行[い]ったら、たくさん 太[ふと]りました。		
\\	友達に会
\\	ば、買い物に行きます。	友達に会えれば、買い物に行きます。	ともだちにあえれば、かいものにいきます。	
\\	「ば」 
\\	友達[ともだち]に 会[あ]えれば、 買い物[かいもの]に 行[い]きます。		
\\	落書きを書
\\	ください。	落書きを書かないでください。	らくがきをかかないでください。	
\\	「~ください」- 
\\	「くださる」	落書[らくが]きを 書[か]かないでください。		
\\	千円を貸
\\	もらえる?	千円を貸してもらえる?	せんえんをかしてもらえる?	
\\	「くれる」 
\\	「もらえる」	千[せん] 円[えん]を 貸[か]してもらえる?		
\\	彼女は、学生
\\	、先生だ。	彼女は、学生じゃなくて、先生だ。	彼女は、がくせいじゃなくて、せんせいだ。	
\\	彼女[かのじょ]は、 学生[がくせい]じゃなくて、 先生[せんせい]だ。		
\\	リブート
\\	いうのは、パソコンを再起動する
\\	いうことです。	リブートというのは、パソコンを再起動するということです。	リブートというのは、パソコンをさいきどうするということです。	
\\	「いう」 
\\	リブートというのは、パソコンを 再[さい] 起動[きどう]するということです。		
\\	明日、大学に行
\\	。	明日、大学に行きます。	あした、だいがくにいきます。	
\\	「~ます」 
\\	明日[あした]、 大学[だいがく]に 行[い]きます。		
\\	もう信
\\	。	もう信じられない。	もうしんじられない。	
\\	もう 信[しん]じられない。		
\\	多分行かないと思う。というか、お金がないから、行けない。	多分行かないと思う。というか、お金がないから、行けない。	たぶんいかないとおもう。というか、おかねがないから、いけない。	
\\	「という」 
\\	「か」 
\\	多分[たぶん] 行[い]かないと 思[おも]う。というか、お 金[かね]がないから、 行[い]けない。		
\\	旅行の計画は終った?/うん、切符を買ったし、ホテルの予約も
\\	する]ある。	旅行の計画は終った?/うん、切符を買ったし、ホテルの予約もしてある。	りょこうのけいかくはおわった?/うん、きっぷをかったし、ホテルのよやくもしてある。	
\\	「~てある」 
\\	旅行[りょこう]の 計画[けいかく]は 終[おわ]った?/うん、 切符[きっぷ]を 買[か]ったし、ホテルの 予約[よやく]もしてある。		
\\	たまにご両親と話
\\	ばどう?	たまにご両親と話せばどう?	たまにごりょうしんとはなせばどう?	
\\	「ば」 
\\	「たら」 
\\	たまにご 両親[りょうしん]と 話[はな]せばどう?		
\\	私は学生
\\	ので、お金がないんです。	私は学生なので、お金がないんです。	わたしはがくせいなので、おかながないんです。	
\\	「から」 
\\	「ので」	私[わたし]は 学生[がくせい]なので、お 金[かね]がないんです。		
\\	みきちゃんは、あんたの彼女でしょう?/う~ん、彼女というか、友達というか、なんというか・・・	みきちゃんは、あんたの彼女でしょう?/う~ん、彼女というか、友達というか、なんというか・・・	みきちゃんは、あんたのかのじょでしょう?/う~ん、かのじょというか、ともだちというか、なんというか・・・	
\\	「という」 
\\	「か」 
\\	みきちゃんは、あんたの 彼女[かのじょ]でしょう?/う~ん、 彼女[かのじょ]というか、 友達[ともだち]というか、なんというか・・・		
\\	この魚は、鯛
\\	いいます。	この魚は、鯛といいます。	このさかなは、たいといいます。	
\\	「いう」 
\\	この 魚[さかな]は、 鯛[たい]といいます。		
\\	友達は
\\	遅れる。	友達はいつも遅れる。	ともだちはいつもおくれる。	
\\	友達[ともだち]はいつも 遅[おく]れる。		
\\	しないとだめだよ。/
\\	とはいっても]、時間がないからできないよ。	しないとだめだよ。/だって、時間がないからできないよ。	しないとだめだよ。/だって、じかんがないからできないよ。	
\\	「だって」 
\\	「とはいっても」	しないとだめだよ。/だって、 時間[じかん]がないからできないよ。		
\\	あんたは、いつもこういう時に来るんだから、困るんだよ。	あんたは、いつもこういう時に来るんだから、困るんだよ。	あんたは、いつもこういうときにくるんだから、こまるんだよ。	
\\	「いう」 
\\	「こう」、「そう」、「ああ」、
\\	「どう」	あんたは、いつもこういう 時[とき]に 来[く]るんだから、 困[こま]るんだよ。		
\\	昨日、友達が遊
\\	きた。	昨日、友達が遊びにきた。	きのう、ともだちがあそびにきた。	
\\	昨日[きのう]、 友達[ともだち]が 遊[あそ]びにきた。		
\\	学生
\\	ら、学生割引で買えます。	学生だったら、学生割引で買えます。	がくせいだったら、がくせいわりびきでかえます。	
\\	「たら(ば)」	学生[がくせい]だったら、 学生[がくせい] 割引[わりびき]で 買[か]えます。		
\\	そんなのは、
\\	か]	そんなのは、あるの?	そんなのは、あるの?	
\\	(の)
\\	そんなのは、あるの?		
\\	それを
\\	。	それをください。	それをください。	
\\	「~ください」- 
\\	「くださる」	それをください。		
\\	どうして友達じゃないんですか?先生
\\	し、年上
\\	し・・・。	どうして友達じゃないんですか?先生だし、年上だし・・・。	どうしてともだちじゃないんですか?せんせいだし、ねんうえだし・・・。	
\\	「し」	どうして 友達[ともだち]じゃないんですか? 先生[せんせい]だし、 年上[としうえ]だし・・・。		
\\	その部屋はあまり静
\\	。	その部屋はあまり静かじゃないです。	そのへやはあまりしずかじゃないです。	
\\	「です」 
\\	その 部屋[へや]はあまり 静[しず]かじゃないです。		
\\	一生懸命、頑張
\\	いく!	一生懸命、頑張っていく!	いっしょうけんめい、がんばっていく!	
\\	(行く、来る) 
\\	一生懸命[いっしょうけんめい]、 頑張[がんば]っていく!		
\\	ここに名前を
\\	。	ここに名前を書いてちょうだい。	ここになまえをかいてちょうだい。	
\\	「~ちょうだい」 
\\	ここに 名前[なまえ]を 書[か]いてちょうだい。		
\\	もう帰
\\	いい?	もう帰っていい?	もうかえっていい?	
\\	もう 帰[かえ]っていい?		
\\	アリスが、「寒い」と言
\\	。	アリスが、「寒い」と言った。	アリスが、「さむい」といった。	
\\	アリスが、
\\	寒[さむ]い」と 言[い]った。		
\\	駅の方へ走
\\	いった。	駅の方へ走っていった。	えきのほうへはしっていった。	
\\	(行く、来る) 
\\	駅[えき]の 方[ほう]へ 走[はし]っていった。		
\\	今、どこに行こう
\\	考えている。	今、どこに行こうかと考えている。	いま、どこにいこうかとかんがえている。	
\\	今[いま]、どこに 行[い]こうかと 考[かんが]えている。		
\\	周りがうるさくて、彼が言っていることがあんまり聞
\\	。	周りがうるさくて、彼が言っていることがあんまり聞こえなかった。	まわりがうるさくて、かれがいっていることがあんまりきこえなかった。	
\\	「見える」 
\\	「聞こえる」 
\\	周[まわ]りがうるさくて、 彼[かれ]が 言[い]っていることがあんまり 聞[き]こえなかった。		
\\	電池を買
\\	おきます。	電池を買っておきます。	でんちをかっておきます。	
\\	「~ておく」 
\\	電池[でんち]を 買[か]っておきます。		
\\	何
\\	言いましたか。	何と言いましたか。	なんといいましたか。	
\\	何[なに]と 言[い]いましたか。		
\\	話を聞
\\	いますか。/ううん、聞
\\	いない。	話を聞いていますか。/ううん、聞いていない。	はなしをきいていますか。/ううん、きいていない。	
\\	「~ている」 
\\	話[はなし]を 聞[き]いていますか。/ううん、 聞[き]いていない。		
\\	みきちゃんが洋介と別れたんだって。/ということは、みきちゃんは、今彼氏がいないということ?/ そう。そういうこと。	みきちゃんが洋介と別れたんだって。/ということは、みきちゃんは、今彼氏がいないということ?/ そう。そういうこと。	みきちゃんがようすけとわかれたんだって。/ということは、みきちゃんは、いまかれしがいないということ?/ そう。そういうこと。	
\\	「という」 
\\	「こと」 
\\	みきちゃんが 洋介[ようすけ]と 別[わか]れたんだって。/ということは、みきちゃんは、 今[こん] 彼氏[かれし]がいないということ?/そう。そういうこと。		
\\	電気を消
\\	と暗くなる。	電気を消すと暗くなる。	でんきをけすとくらくなる。	
\\	「と」	電気[でんき]を 消[け]すと 暗[くら]くなる。		
\\	宿題を
\\	する, 
\\	といけない。	宿題をしないといけない。	しゅくだいをしないといけない。	
\\	と
\\	いけない
\\	宿題[しゅくだい]をしないといけない。		
\\	「寒い」
\\	アリスが田中に言った。	「寒い」とアリスが田中に言った。	「さむい」とアリスがたなかにいった。	
\\	寒[さむ]い」とアリスが 田中[たなか]に 言[い]った。		
\\	お母さんは
\\	。母は買い物に行きました。	お母さんはどこですか。母は買い物に行きました。	おかあさんはどこですか。はははかいものにいきますた。	
\\	お 母[かあ]さんはどこですか。 母[はは]は 買い物[かいもの]に 行[い]きました。		
\\	先週に見た映画は、とても面白
\\	。	先週に見た映画は、とても面白かったです。	せんしゅうにみたえいがは、とてもおもしろかったです。	
\\	「です」 
\\	先週[せんしゅう]に 見[み]た 映画[えいが]は、とても 面白[おもしろ]かったです。		
\\	日本人はお酒に弱い
\\	いうのは本当?	日本人はお酒に弱いというのは本当?	にほんじんはおさけによわいというのはほんとう?	
\\	「いう」 
\\	日本人[にっぽんじん]はお 酒[さけ]に 弱[よわ]いというのは 本当[ほんとう]?		
\\	温泉に行
\\	たい。	温泉に行きたい。	おんせんにいきたい。	
\\	「たい」	温泉[おんせん]に 行[い]きたい。		
\\	漢字で書
\\	ください。	漢字で書いてください。	かんじでかいてください。	
\\	「~ください」- 
\\	「くださる」	漢字[かんじ]で 書[か]いてください。		
\\	明日
\\	って、雨が降
\\	んだって。	明日って、雨が降るんだって。	あしたって、あめがふるんだって。	
\\	「って」 
\\	「と」 
\\	「は」	明日[あした]って、 雨[あめ]が 降[ふ]るんだって。		
\\	友達が父にいいことを教
\\	。	友達が父にいいことを教えてくれた。	ともだちがちちにいいことをおしえてくれた。	
\\	「くれる」	友達[ともだち]が 父[ちち]にいいことを 教[おし]えてくれた。		
\\	食堂に行
\\	、昼ご飯を食
\\	、昼寝をする。	食堂に行って、昼ご飯を食べて、昼寝をする。	しょくどうにいって、ひるごはんをたべて、ひるねをする。	
\\	食堂[しょくどう]に 行[い]って、 昼[ひる]ご 飯[はん]を 食[た]べて、 昼寝[ひるね]をする。		
\\	冬に入って、コートを着ている人が増
\\	いきます。	冬に入って、コートを着ている人が増えていきます。	ふゆにはいって、コートをきているひとがふえていきます。	
\\	(行く、来る) 
\\	冬[ふゆ]に 入[はい]って、コートを 着[き]ている 人[ひと]が 増[ふ]えていきます。		
\\	色々な人と付き合
\\	きたけど、いい人はまだ見つからない。	色々な人と付き合ってきたけど、いい人はまだ見つからない。	いろいろなひととつきあってきたけど、いいひとはまだみつからない。	
\\	(行く、来る) 
\\	色々[いろいろ]な 人[ひと]と 付き合[つきあ]ってきたけど、いい 人[ひと]はまだ 見[み]つからない。		
\\	富士山が登
\\	。	富士山が登れた。	ふじさんがのぼれた。	
\\	富士山[ふじさん]が 登[のぼ]れた。		
\\	映画を見たり、本を読
\\	。	映画を見たり、本を読んだりした。	えいがをみたり、ほんをよんだりした。	
\\	「~たりする」	映画[えいが]を 見[み]たり、 本[ほん]を 読[よ]んだりした。		
\\	よく聞
\\	なさい!	よく聞きなさい!	よくききなさい!	
\\	「~なさい」 
\\	よく 聞[き]きなさい!		
\\	道は分
\\	か。	道は分かりますか。	みちはわかりますか。	
\\	道[みち]は 分[わ]かりますか。		
\\	今日は暇
\\	けど、明日は忙しい。	今日は暇だけど、明日は忙しい。	きょうはひまだけど、あしたはいそがしい。	
\\	「が」 
\\	「けど」	今日[きょう]は 暇[ひま]だけど、 明日[あした]は 忙[いそが]しい。		
\\	彼が寝坊したことも
\\	ね。	彼が寝坊したこともありうるね。	かれがねぼうしたこともありうるね。	
\\	「ある得る」, 
\\	彼[かれ]が 寝坊[ねぼう]したこともありうるね。		
\\	今日、知
\\	。	今日、知りました。	きょう、しりました。	
\\	今日[きょう]、 知[し]りました。		
\\	日本語で話
\\	。	日本語で話して。	にほんごではなして。	
\\	「~ください」- 
\\	「くださる」 
\\	日本語[にほんご]で 話[はな]して。		
\\	ので、友達に会う時間がない。	なので、友達に会う時間がない。	なので、ともだちにあうじかんがない。	
\\	「から」 
\\	「ので」	なので、 友達[ともだち]に 会[あ]う 時間[じかん]がない。		
\\	昨日、時間が
\\	。	昨日、時間がなかったんです。	きのう、じかんがなかったんです。	
\\	「です」 
\\	昨日[きのう]、 時間[じかん]がなかったんです。		
\\	犯人を
\\	で見ましたか。	犯人をどこかで見ましたか。	はんにんをどこかでみましたか。	
\\	犯人[はんにん]をどこかで 見[み]ましたか。		
\\	友達が私にプレゼントを
\\	。	友達が私にプレゼントをくれた。	ともだちがわたしにプレゼントをくれた。	
\\	「くれる」	友達[ともだち]が 私[わたし]にプレゼントをくれた。		
\\	漢字を書
\\	もらえませんか。	漢字を書いてもらえませんか。	かんじをかいてもらえませんか。	
\\	「くれる」 
\\	「もらえる」	漢字[かんじ]を 書[か]いてもらえませんか。		
\\	大学に行かないって、どういう意味なの?	大学に行かないって、どういう意味なの?	だいがくにいかないって、どういういみなの?	
\\	「いう」 
\\	「こう」、「そう」、「ああ」、
\\	「どう」	大学[だいがく]に 行[い]かないって、どういう 意味[いみ]なの?		
\\	名前はキム
\\	。	私の名前はキムです。	わたしのなまえはキムです。	
\\	私[わたし]の 名前[なまえ]はキムです。		
\\	毎日学校に行
\\	ばいけません。	毎日学校に行かなければいけません。	まいにちがっこうにいかなければいけません。	
\\	ば
\\	いけない
\\	毎日[まいにち] 学校[がっこう]に 行[い]かなければいけません。		
\\	他にいいものがたくさんあるけど、やっぱりこれ
\\	する。	他にいいものがたくさんあるけど、やっぱりこれにする。	ほかにいいものがたくさんあるけど、やっぱりこれにする。	
\\	する 
\\	なる 
\\	に 
\\	「なる」 
\\	「する」 
\\	他[た]にいいものがたくさんあるけど、やっぱりこれにする。		
\\	漢字は書
\\	か?	漢字は書けますか?	かんじはかけますか。	
\\	漢字[かんじ]は 書[か]けますか?		
\\	その部屋はあまり静
\\	よ。	その部屋はあまり静かじゃありませんよ。	そのへやはあまりしずかじゃありませんよ。	
\\	ある)	
\\	「です」 
\\	その 部屋[へや]はあまり 静[しず]かじゃありませんよ。		
\\	この質問の答えは、
\\	知らない。	この質問の答えは、誰も知らない。	このしつもんのこたえは、だれもしらない。	
\\	この 質問[しつもん]の 答[こた]えは、 誰[だれ]も 知[し]らない。		
\\	先生がこれを学生に
\\	。	先生がこれを学生にくれる。	せんせいがこれをがくせいにくれる。	
\\	「くれる」	先生[せんせい]がこれを 学生[がくせい]にくれる。		
\\	その時計は私から
\\	のよ。	その時計は私からもらったのよ。	そのとけいはわたしからもらったのよ。	
\\	「もらう」	その 時計[とけい]は 私[わたし]からもらったのよ。		
\\	今日は晴れて、富士山が見
\\	。	今日は晴れて、富士山が見える。	きょうははれて、ふじさんがみえる。	
\\	「見える」 
\\	「聞こえる」 
\\	今日[きょう]は 晴[は]れて、 富士山[ふじさん]が 見[み]える。		
\\	この歌を知
\\	いますか?	この歌を知っていますか?	このうたをしっていますか。	
\\	この 歌[うた]を 知[し]っていますか?		
\\	先生が教
\\	んですか。	先生が教えてあげるんですか。	せんせいがおしえてあげるんですか。	
\\	「くれる」 
\\	「あげる」 
\\	先生[せんせい]が 教[おし]えてあげるんですか。		
\\	電池を買
\\	ときます。	電池を買っときます。	だんちをかっときます。	
\\	「~とく」 
\\	電池[でんち]を 買[か]っときます。		
\\	来年留学するというのは、智子のこと?	来年留学するというのは、智子のこと?	らいねんりゅうがくするというのは、ともこのこと?	
\\	「って」 
\\	「て」 
\\	「という」, 
\\	来年[らいねん] 留学[りゅうがく]するというのは、 智子[さとこ]のこと?		
\\	ここに入
\\	ちゃだめだよ。	ここに入っちゃだめだよ。	ここにはいっちゃだめだよ。	
\\	「ては」
\\	だめ
\\	ここに 入[はい]っちゃだめだよ。		
\\	勉強をなるべく避
\\	うと思った。	勉強をなるべく避けようと思った。	べんきょうをなるべくさけようとおもった。	
\\	勉強[べんきょう]をなるべく 避[さ]けようと 思[おも]った。		
\\	勉強をたくさんしたから、頭がよ
\\	なった。	勉強をたくさんしたから、頭がよくなった。	べんきょうをたくさんしたから、あたまがよくなった。	
\\	する 
\\	なる 
\\	に 
\\	「なる」 
\\	勉強[べんきょう]をたくさんしたから、 頭[あたま]がよくなった。		
\\	鉛筆を学校へ持
\\	いく?	鉛筆を学校へ持っていく?	えんぴつをがっこうへもっていく?	
\\	(行く、来る) 
\\	鉛筆[えんぴつ]を 学校[がっこう]へ 持[も]っていく?		
\\	がおいしいクッキーを全部食べた。	誰かがおいしいクッキーを全部食べた。	だれかがおいしいクッキーをぜんぶたべた。	
\\	誰[だれ]かがおいしいクッキーを 全部[ぜんぶ] 食[た]べた。		
\\	行かなくてもいいよ。/
\\	とはいっても]、みんな行くって。私も行かないと。	行かなくてもいいよ。/だって、みんな行くって。私も行かないと。	いかなくてもいいよ。/だって、みんないくって。わたしもいかないと。	
\\	「だって」 
\\	「とはいっても」	行[い]かなくてもいいよ。/だって、みんな 行[い]くって。 私[わたし]も 行[い]かないと。		
\\	犬と遊
\\	たいですか。	犬と遊びたいですか。	いぬとあそびたいですか。	
\\	「たい」	犬[いぬ]と 遊[あそ]びたいですか。		
\\	全部飲
\\	も大丈夫だよ。	全部飲んでも大丈夫だよ。	ぜんぶのんでもだいじょうぶだよ。	
\\	全部[ぜんぶ] 飲[の]んでも 大丈夫[だいじょうぶ]だよ。		
\\	海外に行
\\	ことになった。	海外に行くことになった。	かいがいにいくことになった。	
\\	する 
\\	なる 
\\	に 
\\	「なる」 
\\	「する」 
\\	海外[かいがい]に 行[い]くことになった。		
\\	広島のお好み焼きを食
\\	みたい!	広島のお好み焼きを食べてみたい!	ひろしまのおこのみやきをたべてみたい!	
\\	広島[ひろしま]のお 好み焼[このみや]きを 食[た]べてみたい!		
\\	海外に行
\\	ことにした。	海外に行くことにした。	かいがいにいくことにした。	
\\	する 
\\	なる 
\\	に 
\\	「なる」 
\\	「する」 
\\	海外[かいがい]に 行[い]くことにした。		
\\	宿題を
\\	する, 
\\	ばだめだった。	宿題をしなければだめだった。	しゅくだいをしなければだめだった。	
\\	ば
\\	だめ
\\	宿題[しゅくだい]をしなければだめだった。		
\\	これは先生に
\\	。	これは先生にあげる。	これはせんせいにあげる。	
\\	「あげる」	これは 先生[せんせい]にあげる。		
\\	毎日ジムに行
\\	うと決めた。	毎日ジムに行こうと決めた。	まいにちジムにいこうときめた。	
\\	毎日[まいにち]ジムに 行[い]こうと 決[き]めた。		
\\	お金が
\\	ある]ばいいね。	お金があればいいね。	おかねがあればいいね。	
\\	「(ば)」	お 金[かね]があればいいね。		
\\	準備はどうですか。準備は、もう
\\	する]あるよ。	準備はどうですか。準備は、もうしてあるよ。	じゅんびはどうですか。/じゅんびは、もうしてあるよ。	
\\	「~てある」 
\\	準備[じゅんび]はどうですか。 準備[じゅんび]は、もうしてあるよ。		
\\	今日は何を
\\	する, 
\\	/テーマパークに行
\\	今日は何をしましょうか?/テーマパークに行きましょう!	あしたはなにをしましょうか?/テーマパークにいきましょう!	
\\	今日[きょう]は 何[なに]をしましょうか?/テーマパークに 行[い]きましょう!		
\\	というか]、もう帰らないとだめですけど。	ていうか、もう帰らないとだめですけど。	ていうか、もうかえらないとだめですけど。	
\\	「って」 
\\	「て」 
\\	「という」	ていうか、もう 帰[かえ]らないとだめですけど。		
\\	お金が
\\	ある]ばいいね。	お金があればいいね。	おかねがあればいいね。	
\\	「ば」	お 金[かね]があればいいね。		
\\	これは、なん
\\	いう魚ですか。	これは、なんという魚ですか。	これは、なんというさかなですか。	
\\	「いう」 
\\	これは、なんという 魚[さかな]ですか。		
\\	田中さんは、お金持
\\	、かっこよ
\\	、魅力的ですね。	田中さんは、お金持ちで、かっこよくて、魅力的ですね。	たなかさんは、おかねもちで、かっこよくて、みりょくてきですね。	
\\	田中[たなか]さんは、お 金持[かねも]ちで、かっこよくて、 魅力[みりょく] 的[てき]ですね。		
\\	よかったら、映画を観に行きますか?	もしよかったら、映画を観に行きますか?	もしよかったら、えいがをみにいきますか?	
\\	「もし」 
\\	たら	もしよかったら、 映画[えいが]を 観[かん]に 行[い]きますか?		
\\	これは友達に買
\\	もらった。	これは友達に買ってもらった。	これはともだちにかってもらった。	
\\	「もらう」	これは 友達[ともだち]に 買[か]ってもらった。		
\\	もう時間がないわ。	もう時間がないわ。	もうじかんがないわ。	
\\	「わ」、「ぞ」、「ぜ」、「かしら」	もう 時間[じかん]がないわ。		
\\	あっち行
\\	あっち行け!	あっちいけ!	
\\	あっち 行[い]け!		
\\	いい大学に行
\\	かな?	いい大学に行けるかな?	いいだいがくにいけるかな?	
\\	「な」 
\\	「さ」	いい 大学[だいがく]に 行[い]けるかな?		
\\	今日は雨が降
\\	かな?	今日は雨が降るかな?	きょうはあめがふるかな?	
\\	「な」 
\\	「さ」	今日[きょう]は 雨[あめ]が 降[ふ]るかな?		
\\	それを食
\\	な!	それを食べるな!	それをたべるな!	
\\	それを 食[た]べるな!		
\\	早く酒を持ってきて
\\	くれる, 
\\	。	早く酒を持ってきてくれ。	はやくさけをもってきてくれ。	
\\	早[はや]く 酒[さけ]を 持[も]ってきてくれ。		
\\	今、図書館に行くんだよ
\\	な。/うん、なんで?	今、図書館に行くんだよな。/うん、なんで?	いま、としょかんにいくんだよな。/うん、 なんで?	
\\	「な」 
\\	「さ」	今[いま]、 図書館[としょかん]に 行[い]くんだよな。/うん、なんで?		
\\	いい大学に入
\\	かしら?	いい大学に入れるかしら?	いいだいがくにはいれるかしら?	
\\	「わ」、「ぞ」、「ぜ」、「かしら」	いい 大学[だいがく]に 入[い]れるかしら?		
\\	変なことを言
\\	な!	変なことを言うな!	へんなことをいうな!	
\\	変[へん]なことを 言[い]うな!		
\\	日本語は、たくさん勉強したけどな。まだ全然分からない。/大丈夫よ。きっと分かるようになるからさ。/ならいいけどな。	日本語は、たくさん勉強したけどな。まだ全然分からない。/大丈夫よ。きっと分かるようになるからさ。/ならいいけどな。	にほんごは、たくさんべんきょうしたけどな。まだぜんぜんわからない。/だいじょうぶよ。きっとわかるようになるからさ。/ならいいけどな。	
\\	「な」 
\\	「さ」	日本語[にほんご]は、たくさん 勉強[べんきょう]したけどな。まだ 全然[ぜんぜん] 分[わ]からない。/ 大丈夫[だいじょうぶ]よ。きっと 分[わ]かるようになるからさ。/ならいいけどな。		
\\	これで、もう終わりだぜ。	これで、もう終わりだぜ。	これで、もうおわりだぜ。	
\\	「わ」、「ぞ」、「ぜ」、「かしら」	これで、もう 終[お]わりだぜ。		
\\	おい、行くぞ!	おい、行くぞ!	おい、いくぞ!	
\\	「わ」、「ぞ」、「ぜ」、「かしら」	おい、 行[い]くぞ!		
\\	先生が学生に宿題をたくさん
\\	する]せた。	先生が学生に宿題をたくさんさせた。	先生ががくせいにしゅくだいをたくさんさせた。	
\\	先生[せんせい]が 学生[がくせい]に 宿題[しゅくだい]をたくさんさせた。		
\\	先生が質問をたくさん聞
\\	せてくれた。	先生が質問をたくさん聞かせてくれた。	せんせいがしつもんをたくさんきかせてくれた。	
\\	先生[せんせい]が 質問[しつもん]をたくさん 聞[き]かせてくれた。		
\\	全部食
\\	せてくれた。	全部食べさせてくれた。	せんぶたべさせてくれた。	
\\	全部[ぜんぶ] 食[た]べさせてくれた。		
\\	みんなに変だと言
\\	れます。	みんなに変だと言われます。	みんなにへんだといわれます。	
\\	みんなに 変[へん]だと 言[い]われます。		
\\	全部食
\\	せた。	全部食べさせた。	ぜんぶたべさせた。	
\\	全部[ぜんぶ] 食[た]べさせた。		
\\	トイレに行
\\	もいいですか。	トイレに行ってもいいですか。	といれにいってもいいですか。	
\\	トイレに 行[い]ってもいいですか。		
\\	このパッケージには、あらゆるものが含
\\	れている。	このパッケージには、あらゆるものが含まれている。	このパッケージには、あらゆるものがふくまれている。	
\\	このパッケージには、あらゆるものが 含[ふく]まれている。		
\\	外国人に質問を聞
\\	れたが、答
\\	れなかった。	外国人に質問を聞かれたが、答えられなかった。	がいこくじんにしつもんをきかれたが、こたえられなかった。	
\\	外国[がいこく] 人[じん]に 質問[しつもん]を 聞[き]かれたが、 答[こた]えられなかった。		
\\	同じことを何回も言
\\	すな!	同じことを何回も言わすな!	おなじことをなんかいもいわすな!	
\\	同[おな]じことを 何[なん] 回[かい]も 言[い]わすな!		
\\	光の速さを超えるのは、不可能だと思
\\	れる。	光の速さを超えるのは、不可能だと思われる。	ひかりのはやさをこえるのは、ふかのうだとおもわれる。	
\\	光[ひかり]の 速[はや]さを 超[こ]えるのは、 不可能[ふかのう]だと 思[おも]われる。		
\\	その部長は、よく長時間働
\\	せる。	その部長は、よく長時間働かせる。	そのぶちょうは、よくちょうじかんはたらかせる。	
\\	その 部長[ぶちょう]は、よく 長時間[ちょうじかん] 働[はたら]かせる。		
\\	トイレに行
\\	せてくれますか。	トイレに行かせてくれますか。	トイレにいかせてくれますか。	
\\	トイレに 行[い]かせてくれますか。		
\\	ポリッジが誰かに食
\\	れた!	ポリッジが誰かに食べられた!	ポリッジがだれかにたべられた。	
\\	ポリッジが 誰[だれ]かに 食[た]べられた!		
\\	今日は仕事を休
\\	せてください。	今日は仕事を休ませてください。	きょうはしごとをやすませてください。	
\\	今日[きょう]は 仕事[しごと]を 休[やす]ませてください。		
\\	広子の家に行ったことある?/
\\	ある]わけないでしょう。	広子の家に行ったことある?/あるわけないでしょう。	広子のいえにいったことある?/あるわけないでしょう。	
\\	「わけ」 
\\	ない 
\\	広子[ひろこ]の 家[いえ]に 行[い]ったことある?/あるわけないでしょう。		
\\	この字は読
\\	にくい 。	この字は読みにくい 。	このじはよみにくい。	
\\	「~やすい、~にくい」 
\\	この 字[じ]は 読[よ]みにくい。		
\\	もう逃
\\	れないかもしれんぞ。	もう逃げられないかもしれんぞ。	もうにげられないかもしれんぞ。	
\\	「かもしれない」 
\\	もう 逃[に]げられないかもしれんぞ。		
\\	仕事で何をなさっているんですか。	仕事で何をなさっているんですか。	しごとでなにをなさっているんですか。	
\\	仕事[しごと]で 何[なに]をなさっているんですか。		
\\	朝ご飯は食べたくなかったのに、食
\\	せられた。	朝ご飯は食べたくなかったのに、食べさせられた。	あさごはんはたべたくなかったのに、たべさせられた。	
\\	朝[あさ]ご 飯[はん]は 食[た]べたくなかったのに、 食[た]べさせられた。		
\\	すまん。	すまん。	すまん。	
\\	「ん」	すまん。		
\\	学生が廊下に立
\\	された。	学生が廊下に立たされた。	がくせいがろうかにたたされた。	
\\	学生[がくせい]が 廊下[ろうか]に 立[た]たされた。		
\\	日本では、お酒を飲
\\	せられることが多い。	日本では、お酒を飲ませられることが多い。	にほんでは、おさけをのませられることがおおい。	
\\	日本[にっぽん]では、お 酒[さけ]を 飲[の]ませられることが 多[おお]い。		
\\	この漬物は美味
\\	そう!	この漬物は美味しそう!	このつけものはおいしそう!	
\\	「~そう」- 
\\	「い」 
\\	この 漬物[つけもの]は 美味[おい]しそう!		
\\	航空券は安
\\	ば安
\\	いいとは限らない。	航空券は安ければ安いほどいいとは限らない。	こうくうけんはやすければやすいほどいいとはかぎらない。	
\\	「ほど」 
\\	航空[こうくう] 券[けん]は 安[やす]ければ 安[やす]いほどいいとは 限[かぎ]らない。		
\\	このピザはお好み焼き
\\	ように見える。	このピザはお好み焼きのように見える。	このピザはおこのみやきのようにみえる。	
\\	「みたい」 
\\	よう 
\\	このピザはお 好み焼[このみや]きのように 見[み]える。		
\\	商品の品質を
\\	大切にしています。	商品の品質を何より大切にしています。	しょうひんのひんしつをなによりたいせつにしています。	
\\	方 
\\	よる 
\\	「より」 
\\	「より」 
\\	商品[しょうひん]の 品質[ひんしつ]を 何[なに]より 大切[たいせつ]にしています。		
\\	一人で行
\\	ことはありません。	一人で行くことはありません	ひとりでいくことはありません。	
\\	「こと」 
\\	一 人[にん]で 行[い]くことはありません		
\\	観客
\\	して参加させてもらった。	観客として参加させてもらった。	かんきゃくとしてさんかさせてもらった。	
\\	「とする」	観客[かんきゃく]として 参加[さんか]させてもらった。		
\\	アリス
\\	から]何ももらってない。	アリスからしか何ももらってない。	アリスからしかなにももらってない。	
\\	「しか」 
\\	アリスからしか 何[なに]ももらってない。		
\\	彼は麻雀
\\	ばかりです。	彼は麻雀ばかりです。	かれはマージャンばかりです。	
\\	「だけ」 
\\	「ばかり」	彼[かれ]は 麻雀[まーじゃん]ばかりです。		
\\	それ
\\	は]、食べないでください。	それだけは、食べないでください。	それだけは、たべないでください。	
\\	「だけ」 
\\	それだけは、 食[た]べないでください。		
\\	とても信
\\	がたい話だが、本当に起こったらしい。	とても信じがたい話だが、本当に起こったらしい。	とてもしんじがたいはなしだが、ほんとうにおこったらしい。	
\\	「~にくい」 
\\	「~がたい」 
\\	「~づらい」 
\\	「~がたい」 
\\	とても 信[しん]じがたい 話[はなし]だが、 本当[ほんとう]に 起[お]こったらしい。		
\\	雨で試合は中止に
\\	なる]かもしれないね。	雨で試合は中止になるかもしれないね。	あめでしあいはちゅうしになるかもしれないね。	
\\	「かもしれない」 
\\	雨[あめ]で 試合[しあい]は 中止[ちゅうし]になるかもしれないね。		
\\	お酒を飲
\\	すぎないように気をつけてね。	お酒を飲みすぎないように気をつけてね。	おさけをのみすぎないようにきをつけてね。	
\\	「すぎる」	お 酒[さけ]を 飲[の]みすぎないように 気[き]をつけてね。		
\\	どうしてこなかったの?/授業があったの。	どうしてこなかったの?/授業があったの。	どうしてこなっかったの?/じゅぎょうがあったの。	
\\	「の」 
\\	どうしてこなかったの?/ 授業[じゅぎょう]があったの。		
\\	どうしてこなかったの?/授業があったもの。	どうしてこなかったの?/授業があったもの 。	どうしてこなっかったの?/じゅぎょうがあったもの。	
\\	「もの」 
\\	どうしてこなかったの?/ 授業[じゅぎょう]があったもの。		
\\	どうぞ、ごゆっく
\\	なさいませ。	どうぞ、ごゆっくりなさいませ。	どうぞ、ごゆっくりなさいませ。	
\\	どうぞ、ごゆっくりなさいませ。		
\\	時間が足
\\	すぎて、何もできなかった。	時間が足りなさすぎて、何もできなかった。	じかんがたりなさすぎて、なにもできなかった。	
\\	「すぎる」	時間[じかん]が 足[た]りなさすぎて、 何[なに]もできなかった。		
\\	日本の映画を観
\\	ことないの?	日本の映画を観たことないの?	にほんのえいがをみたことないの?	
\\	「こと」 
\\	日本[にっぽん]の 映画[えいが]を 観[み]たことないの?		
\\	花
\\	より団子。	花より団子。	はなよりだんご。	
\\	方 
\\	よる 
\\	「より」 
\\	花[はな]より 団子[だんご]。		
\\	彼は何も言
\\	ず、帰ってしまった。	彼は何も言わず、帰ってしまった。	かれはなにもいわず、かえってしまった。	
\\	ない 
\\	ず	彼[かれ]は 何[なに]も 言[い]わず、 帰[かえ]ってしまった。		
\\	先生はお見
\\	になりますか。	先生はお見えになりますか。	せんせいほかえになりますか。	
\\	お 
\\	に 
\\	なる	先生[せんせい]はお 見[み]えになりますか。		
\\	この犬は可哀相。	この犬は可哀相。	このいぬはかわいそう。	
\\	「~そう」 
\\	かわい 
\\	この 犬[いぬ]は 可哀相[かわいそう]。		
\\	明日、雨が降る
\\	そうだ。	明日、雨が降るそうだ。	あした、あめがふるそうだ。	
\\	「~そうだ」	明日[あした]、 雨[あめ]が 降[ふ]るそうだ。		
\\	すみません、お待
\\	せしました。	すみません、お待たせしました。	すみません、おまたせしました。	
\\	お 
\\	する	すみません、お 待[ま]たせしました。		
\\	これからどこへ行くんでしょうか?	これからどこへ行くんでしょうか?	これからどこへいくんでしょうか?	
\\	「でしょう」 
\\	これからどこへ 行[い]くんでしょうか?		
\\	漢字の書
\\	方を教えてくれますか?	漢字の書き方を教えてくれますか?	かんじのかきかたをおしえてくれますか?	
\\	方 
\\	よる 
\\	「方」 
\\	漢字[かんじ]の 書き方[かきかた]を 教[おし]えてくれますか?		
\\	カクテルはビールより飲
\\	やすい。	カクテルはビールより飲みやすい。	カクテルはビールよりのみやすい。	
\\	「~やすい、~にくい」 
\\	カクテルはビールより 飲[の]みやすい。		
\\	今は授業が終
\\	ところです。	今は授業が終ったところです。	いまはじゅぎょうがおわったところです。	
\\	「ところ」 
\\	今[いま]は 授業[じゅぎょう]が 終[おわ]ったところです。		
\\	今日、田中さんはこないの?/だそうです。	今日、田中さんはこないの?/だそうです。	きょう、たなかさんはこないの?/だそうです。	
\\	「~そうだ」	今日[きょう]、 田中[たなか]さんはこないの?/だそうです。		
\\	その人
\\	が]好きだったんだ。	その人だけが好きだったんだ。	そのひとだけがすきだったんだ。	
\\	「だけ」 
\\	その 人[ひと]だけが 好[す]きだったんだ。		
\\	あの人はちょっと韓国人
\\	ぽいよね。	あの人はちょっと韓国人っぽいよね。	あのひとはちょっとかんこくじんっぽいよね。	
\\	「っぽい」: 
\\	あの 人[ひと]はちょっと 韓国[かんこく] 人[じん]っぽいよね。		
\\	休ませていただけますでしょうか。	休ませていただけますでしょうか。	やすませていただけますでしょうか。	
\\	「でしょう」 
\\	休[やす]ませていただけますでしょうか。		
\\	貧乏
\\	ながらも、高級なバッグを買っちゃったよ。	貧乏ながらも、高級なバッグを買っちゃったよ。	びんぼうながらも、こうきゅうなバッグをかっちゃったよ。	
\\	「ながら」 
\\	ながらも 
\\	貧乏[びんぼう]ながらも、 高級[こうきゅう]なバッグを 買[か]っちゃったよ。		
\\	この犬は可愛い。	この犬は可愛い。	このいぬ犬はかわいい。	
\\	「~そう」 
\\	この 犬[いぬ]は 可愛[かわい]い。		
\\	これはただの試合じゃ
\\	そうだ。	これはただの試合じゃなさそうだ。	これはただのしあいじゃなさそうだ。	
\\	「~そう」 
\\	「い」 
\\	「さ」.	これはただの 試合[しあい]じゃなさそうだ。		
\\	天気予報によ
\\	と、今日は雨だそうだ。	天気予報によると、今日は雨だそうだ。	てんきよほうによると、きょうはあめだそうだ。	
\\	方 
\\	よる 
\\	「によると」	天気[てんき] 予報[よほう]によると、 今日[きょう]は 雨[あめ]だそうだ。		
\\	すみません、今食
\\	ばかりなので、お腹がいっぱいです。	すみません、今食べたばかりなので、お腹がいっぱいです。	すみません、いまたべたばかりなので、おなかがいっぱいです。	
\\	「~ばかり」	すみません、 今[いま] 食[た]べたばかりなので、お 腹[なか]がいっぱいです。		
\\	日本では、お酒を飲
\\	されることが多い。	日本では、お酒を飲まされることが多い。	にほんでは、おさけをのまされることがおおい。	
\\	日本[にっぽん]では、お 酒[さけ]を 飲[の]まされることが 多[おお]い。		
\\	人によ
\\	話が違う。	人によって話が違う。	ひとによってはなしがちがう。	
\\	方 
\\	よる 
\\	「によって」 
\\	人[ひと]によって 話[はなし]が 違[ちが]う。		
\\	これから頑張
\\	しかない!	これから頑張るしかない!	これからがんばるしかない!	
\\	「しか」 
\\	これから 頑張[がんば]るしかない!		
\\	準備が終わったから、これからは食
\\	だけだ。	準備が終わったから、これからは食べるだけだ。	じゅんびがおわったから、これからはたべるだけだ。	
\\	「だけ」 
\\	準備[じゅんび]が 終[お]わったから、これからは 食[た]べるだけだ。		
\\	親に毎日宿題を
\\	する]せられる。	親に毎日宿題をさせられる。	おやにまいにちしょくだいをさせられる。	
\\	親[おや]に 毎日[まいにち] 宿題[しゅくだい]をさせられる。		
\\	待ち合わせは、分
\\	づらい場所にしないでね。	待ち合わせは、分かりづらい場所にしないでね。	まちあわせは、わかりづらいばしょにしないでね。	
\\	「~にくい」 
\\	「~がたい」 
\\	「~づらい」 
\\	「~づらい」 
\\	待ち合[まちあ]わせは、 分[わ]かりづらい 場所[ばしょ]にしないでね。		
\\	あの人を見
\\	よう
\\	気がした。	あの人を見たような気がした。	あのひとをみたようなきがした。	
\\	よう (様) 
\\	な 
\\	あの 人[ひと]を 見[み]たような 気[き]がした。		
\\	あなたは、学生さんでしょうか。	あなたは、学生さんでしょうか。	あなたは、がくせいさんでしょうか。	
\\	「でしょう」 
\\	あなたは、 学生[がくせい]さんでしょうか。		
\\	友達の話によ
\\	と、朋子はやっとボーイフレンドを見つけたらしい。	友達の話によると、朋子はやっとボーイフレンドを見つけたらしい。	ともだちのはなしによると、ともこはやっとボーイフレンドをみつけたらしい。	
\\	方 
\\	よる 
\\	「によると」	友達[ともだち]の 話[はなし]によると、 朋子[ともこ]はやっとボーイフレンドを 見[み]つけたらしい。		
\\	佐藤さんは料理が上手で、また食
\\	過ぎました。	佐藤さんは料理が上手で、また食べ過ぎました。	さとうさんはりょうりがじょうずで、またたべすぎました。	
\\	「すぎる」	佐藤[さとう]さんは 料理[りょうり]が 上手[じょうず]で、また 食[た]べ 過[す]ぎました。		
\\	お手洗いはこのビルの二階に
\\	ある]ます。	お手洗いはこのビルの二階にあります。	おてあらいはこのビルのにかいにあります。	
\\	お 手洗[てあら]いはこのビルの二 階[かい]にあります。		
\\	明日に行
\\	とする。	明日に行くとする。	あしたにいくとする。	
\\	「とする」	明日[あした]に 行[い]くとする。		
\\	また遅刻
\\	する]ちまったよ。	また遅刻しちまったよ。	またちこくしちまったよ。	
\\	「~てしまう」	また 遅刻[ちこく]しちまったよ。		
\\	金魚がもう死
\\	しまった。	金魚がもう死んでしまった。	きんぎょがもうしんでしまった。	
\\	「しまう」 
\\	金魚[きんぎょ]がもう 死[し]んでしまった。		
\\	勉強
\\	する]ずに東大に入れると思わないな。	勉強せずに東大に入れると思わないな。	べんきょうせずにとうだいにはいれるとおもわないな。	
\\	勉強[べんきょう]せずに 東大[とうだい]に 入[い]れると 思[おも]わないな。		
\\	もう帰
\\	ゃっていい?	もう帰っちゃっていい?	もうかえちゃっていい?	
\\	「~てしまう」	もう 帰[かえ]っちゃっていい?		
\\	このビルの高さは何ですか?	このビルの高さは何ですか?	このビルのたかさはなんですか?	
\\	「~さ」 
\\	このビルの 高[たか]さは 何[なに]ですか?		
\\	学生
\\	ようだ。	学生のようだ。	がくせいのようだ。	
\\	よう (様) 
\\	の 
\\	な 
\\	学生[がくせい]のようだ。		
\\	ここには、誰もいない
\\	ようだ。	ここには、誰もいないようだ。	ここには、だれもいないようだ。	
\\	よう (様) 
\\	ここには、 誰[だれ]もいないようだ。		
\\	これから食べに行くんでしょ。/だったら?	これから食べに行くんでしょ。/だったら?	これからたべにいくんでしょ。/だったら?	
\\	「でしょう」 
\\	「だろう」 
\\	これから 食[た]べに 行[い]くんでしょ。/だったら?		
\\	この映画は一回見
\\	ことあるかも!	この映画は一回見たことあるかも!	このえいがはいっかいみたことあるかも!	
\\	「かもしれない」 
\\	この 映画[えいが]は一 回[かい] 見[み]たことあるかも!		
\\	ゆっくり食
\\	方が健康にいいよ。	ゆっくり食べた方が健康にいいよ。	ゆっくりたべたほうがけんこうにいいよ。	
\\	方 
\\	よる 
\\	「方」 
\\	ゆっくり 食[た]べた 方[ほう]が 健康[けんこう]にいいよ。		
\\	あそこが代々木公園
\\	かもしれない。	あそこが代々木公園かもしれない。	あそこがよよぎこうえんかもしれない。	
\\	「かもしれない」 
\\	あそこが 代々木公園[よよぎこうえん]かもしれない。		
\\	そのケーキを全部食
\\	しまった。	そのケーキを全部食べてしまった。	そのケーキをぜんぶたべてしまった。	
\\	「しまう」 
\\	そのケーキを 全部[ぜんぶ] 食[た]べてしまった。		
\\	もう10時になったから、来
\\	そうだね。	もう10時になったから、来なさそうだね。	もう10じになったから、こなさそうだね。	
\\	「~そう」 
\\	「い」 
\\	「さ」.	もう 
\\	時[じ]になったから、 来[き]なさそうだね。		
\\	歩
\\	ら歩
\\	、迷ってしまった。	歩いたら歩くほど、迷ってしまった。	あるいたらあるくほど、まよってしまった。	
\\	「ほど」 
\\	歩[ある]いたら 歩[ある]くほど、 迷[まよ]ってしまった。		
\\	スミスさんは食堂に行
\\	かもしれません。	スミスさんは食堂に行ったかもしれません。	スミスさんはしょくどうにいったかもしれません。	
\\	「かもしれない」 
\\	スミスさんは 食堂[しょくどう]に 行[い]ったかもしれません。		
\\	いらっしゃいませ。	いらっしゃいませ。	いらっしゃいませ。	
\\	いらっしゃいませ。		
\\	今日の天気はそれ
\\	寒くない。	今日の天気はそれほど寒くない。	きょうのてんきはそれほどさむくない。	
\\	「ほど」 
\\	今日[きょう]の 天気[てんき]はそれほど 寒[さむ]くない。		
\\	これも結構
\\	そうだけど、やっぱり高いよね。	これも結構よさそうだけど、やっぱり高いよね。	これもけっこうよさそうだけど、やっぱりたかいよね。	
\\	「~そう」- 
\\	「い」 
\\	いい 
\\	これも 結構[けっこう]よさそうだけど、やっぱり 高[たか]いよね。		
\\	彼は、優
\\	ところもあるよ。	彼は、優しいところもあるよ。	かれは、やさしいところもあるよ。	
\\	「ところ」 
\\	彼[かれ]は、 優[やさ]しいところもあるよ。		
\\	部屋が暗かったので、見
\\	にくかった。	部屋が暗かったので、見にくかった。	へやがくらかったので、みにくかった。	
\\	「~やすい、~にくい」 
\\	部屋[へや]が 暗[くら]かったので、 見[み]にくかった。		
\\	ちゃんと食べないと、痩
\\	しまいますよ。	ちゃんと食べないと、痩せてしまいますよ。	ちゃんとたべないと、やせてしまいますよ。	
\\	「しまう」 
\\	ちゃんと 食[た]べないと、 痩[や]せてしまいますよ。		
\\	あっ!遅刻しちゃう!/だから、時間がないって言ったでしょう!	あっ!遅刻しちゃう!/だから、時間がないって言ったでしょう!	あっ!ちこくしちゃう!/だから、じかんがないっていったでしょう!	
\\	「でしょう」 
\\	「だろう」 
\\	あっ! 遅刻[ちこく]しちゃう!/だから、 時間[じかん]がないって 言[い]ったでしょう!		
\\	もう売り切れ
\\	ようだ。	もう売り切れのようだ。	もううりきれのようだ。	
\\	「みたい」 
\\	よう 
\\	もう 売り切[うりき]れのようだ。		
\\	仕事が一杯入って、残念
\\	ながら、今日は行けなくなりました。	仕事が一杯入って、残念ながら、今日は行けなくなりました。	しごとがいっぱいはいって、ざんねんながら、きょうはいけなくなりました。	
\\	「ながら」 
\\	仕事[しごと]が一 杯[はい] 入[はい]って、 残念[ざんねん]ながら、 今日[きょう]は 行[い]けなくなりました。		
\\	今度は負
\\	わけにはいかない。	今度は負けるわけにはいかない。	こんどはまけるわけにはいかない。	
\\	「わけ」 
\\	今度[こんど]は 負[ま]けるわけにはいかない。		
\\	毎日仕事に行くのが嫌だ。/仕事が
\\	よりましだよ。	毎日仕事に行くのが嫌だ。/仕事がないよりましだよ。	まいにちしごとにいくのがいやだ。/しごとがないよりましだよ。	
\\	方 
\\	よる 
\\	「より」 
\\	毎日[まいにち] 仕事[しごと]に 行[い]くのが 嫌[いや]だ。/ 仕事[しごと]がないよりましだよ。		
\\	掃除、手伝ってくれるでしょう。/え?そうなの?	掃除、手伝ってくれるでしょう。/え?そうなの?	そうじ、てつだってくれるでしょう。/え?そうなの?	
\\	「でしょう」 
\\	「だろう」 
\\	掃除[そうじ]、 手伝[てつだ]ってくれるでしょう。/え?そうなの?		
\\	文章は、短
\\	ば短
\\	、簡単
\\	ら簡単
\\	よいです。	文章は、短ければ短いほど、簡単なら簡単なほどよいです。	ぶんしょうは、みじかければみじかいほど、かんたんならかんたんなほどよいです。	
\\	「ほど」 
\\	文章[ぶんしょう]は、 短[みじか]ければ 短[みじか]いほど、 簡単[かんたん]なら 簡単[かんたん]なほどよいです。		
\\	有り難う御座いました。またお越
\\	くださいませ。	有り難う御座いました。またお越しくださいませ。	ありがとうございました。またおこしくださいませ。	
\\	有り難[ありがた]う 御座[ござ]いました。またお 越[こ]しくださいませ。		
\\	どうしてこなかったの?/授業があったもん。	どうしてこなかったの?/授業があったもん。	どうしてこなっかったの?/じゅぎょうがあったもん。	
\\	「もの」 
\\	どうしてこなかったの?/ 授業[じゅぎょう]があったもん。		
\\	もうお帰
\\	ですか。	もうお帰りですか。	もうおかえりですか。	
\\	お 
\\	です	もうお 帰[かえ]りですか。		
\\	ここは静
\\	ようだ。	ここは静かなようだ。	ここはしずかなようだ。	
\\	よう (様)- 
\\	の 
\\	な 
\\	ここは 静[しず]かなようだ。		
\\	ポップコーンを食
\\	ながら、映画を観る。	ポップコーンを食べながら、映画を観る。	ポップコーンをたべながら、えいがをみる。	
\\	「ながら」 
\\	ポップコーンを 食[た]べながら、 映画[えいが]を 観[み]る。		
\\	韓国人と結婚
\\	する]はならぬ!	韓国人と結婚してはならぬ!	かんこくじんとけっこんしてはならぬ!	
\\	「ぬ」	韓国[かんこく] 人[じん]と 結婚[けっこん]してはならぬ!		
\\	全部買うの?/ううん、これ
\\	買
\\	。	全部買うの?/ううん、これしか買わない。	ぜんぶかうの?/ううん、これしかかわない。	
\\	「しか」	全部[ぜんぶ] 買[か]うの?/ううん、これしか 買[か]わない。		
\\	韓国人と結婚
\\	する, 
\\	はならん!	韓国人と結婚しなくてはならん!	かんこくじんとけっこんしなくてはならん!	
\\	「ん」	韓国[かんこく] 人[じん]と 結婚[けっこん]しなくてはならん!		
\\	宿題をやった?/しまった!	宿題をやった?/しまった!	しゅくだいをやった?/しまった!	
\\	宿題[しゅくだい]をやった?/しまった!		
\\	みんな、どっか行
\\	ゃったよ。	みんな、どっか行っちゃったよ。	みんな、どっかいっちゃったよ。	
\\	「~てしまう」	みんな、どっか 行[い]っちゃったよ。		
\\	一度行
\\	こともないんです。	一度行ったこともないんです。	いちどいったこともないんです。	
\\	「こと」 
\\	一度[いちど] 行[い]ったこともないんです。		
\\	制服を着ている姿を見ると、学生
\\	みたいです。	制服を着ている姿を見ると、学生みたいです。	せいふくをきているすがたをみると、がくせいみたいです。	
\\	「みたい」 
\\	制服[せいふく]を 着[き]ている 姿[すがた]を 見[み]ると、 学生[がくせい]みたいです。		
\\	口笛を
\\	する]ながら、手紙を書いていた。	口笛をしながら、手紙を書いていた。	くちぶえをしながら、てがみをかいていた。	
\\	「ながら」 
\\	口笛[くちぶえ]をしながら、 手紙[てがみ]を 書[か]いていた。		
\\	もう売り切れ
\\	みたい。	もう売り切れみたい。	もううりきれみたい。	
\\	「みたい」 
\\	もう 売り切[うりき]れみたい。		
\\	アリスはどこだ?/もう寝ているだろう。	アリスはどこだ?/もう寝ているだろう。	アリスはどこだ?もうねているだろう。	
\\	「でしょう」 
\\	「だろう」 
\\	アリスはどこだ?/もう 寝[ね]ているだろう。		
\\	よろしくお願
\\	します。	よろしくお願いします。	よろしくおねがいします。	
\\	お 
\\	する	よろしくお 願[ねが]いします。		
\\	全部買うの?/ううん、これ
\\	。	全部買うの?/ううん、これだけ。	ぜんぶかうの?/ううん、これだけ。	
\\	「しか」	全部[ぜんぶ] 買[か]うの?/ううん、これだけ。		
\\	何も食
\\	ずにそんなにお酒を飲むと当然酔っ払いますよ。	何も食べずにそんなにお酒を飲むと当然酔っ払いますよ。	なにもたべずにそんなにおさけをのむととうぜんよっぱらいますよ。	
\\	ず	何[なに]も 食[た]べずにそんなにお 酒[さけ]を 飲[の]むと 当然[とうぜん] 酔っ払[よっぱら]いますよ。		
\\	今日だけは悲
\\	ままでいさせてほしい。	今日だけは悲しいままでいさせてほしい。	きょうだけはかなしいままでいさせてほしい。	
\\	「まま」 
\\	今日[きょう]だけは 悲[かな]しいままでいさせてほしい。		
\\	徹夜して、宿題
\\	ことはある。	徹夜して、宿題することはある。	てつやして、しゅくだいすることはある。	
\\	「こと」 
\\	徹夜[てつや]して、 宿題[しゅくだい]することはある。		
\\	先生と相談
\\	する, 
\\	で、この授業を取ることは出来ない。	先生と相談しないで、この授業を取ることは出来ない。	せんせいとそうだんしないで、このじゅぎょうをとることはできない。	
\\	ないで	先生[せんせい]と 相談[そうだん]しないで、この 授業[じゅぎょう]を 取[と]ることは 出来[でき]ない。		
\\	昨晩のこと、全然覚えてないな。/それは飲
\\	すぎだよ。	昨晩のこと、全然覚えてないな。/それは飲みすぎだよ。	さくばんのこと、ぜんぜんおぼえてないな。/せれはのみすぎだよ。	
\\	「すぎる」 
\\	昨晩[さくばん]のこと、 全然[ぜんぜん] 覚[おぼ]えてないな。/それは 飲[の]みすぎだよ。		
\\	崇君は漫画ばっかり読んでてさ。かっこ悪い。	崇君は漫画ばっかり読んでてさ。かっこ悪い。	たかしくんはまんがばっかりよんでてさ。かっこわるい。	
\\	「だけ」 
\\	「ばかり」	崇[たかし] 君[くん]は 漫画[まんが]ばっかり 読[よ]んでてさ。かっこ 悪[わる]い。		
\\	学生じゃ
\\	方がいいよ。	学生じゃない方がいいよ。	がくせいじゃないほうがいいよ。	
\\	方 
\\	よる 
\\	「方」 
\\	学生[がくせい]じゃない 方[ほう]がいいよ。		
\\	まさか、今起
\\	ばっかなの?	まさか、今起きたばっかなの?	まさか、いまおきたばっかなの?	
\\	「~ばかり」 
\\	まさか、 今[いま] 起[お]きたばっかなの?		
\\	新宿の行
\\	方は分かりますか。	新宿の行き方は分かりますか。	しんじゅくのいきかたはわかりますか。	
\\	方 
\\	よる 
\\	「方」 
\\	新宿[しんじゅく]の 行[い]き 方[かた]は 分[わ]かりますか。		
\\	この販売機
\\	では]、500円玉が使えない。	この販売機だけでは、500円玉が使えない。	このはんばいきだけは、500えんだまがつかえない。	
\\	「だけ」 
\\	この 販売[はんばい] 機[き]だけでは、 
\\	円[えん] 玉[だま]が 使[つか]えない。		
\\	これとそれ
\\	。	これとそれだけ。	これとそれだけ。	
\\	「だけ」	これとそれだけ。		
\\	ゆっくり食べた方が早く食
\\	よりいい。	ゆっくり食べた方が早く食べるよりいい。	ゆっくりたべたほうがはやくたべるよりいい。	
\\	方 
\\	よる 
\\	「より」 
\\	ゆっくり 食[た]べた 方[ほう]が 早[はや]く 食[た]べるよりいい。		
\\	大人
\\	らしくするつもりだったのに、大騒ぎしてしまった。	大人らしくするつもりだったのに、大騒ぎしてしまった。	おとならしくするつもりだったのに、おおさわぎしてしまった。	
\\	「~らしい」	大人[おとな]らしくするつもりだったのに、 大騒[おおさわ]ぎしてしまった。		
\\	ご飯の方が、パン
\\	より美味しい。	ご飯の方が、パンより美味しい。	ごはんのほうが、パンよりおいしい。	
\\	方 
\\	よる 
\\	「より」 
\\	ご 飯[はん]の 方[ほう]が、パンより 美味[おい]しい。		
\\	テレビを付
\\	ぱなしにしなければ眠れない人は、結構いる。	テレビを付けっぱなしにしなければ眠れない人は、結構いる。	テレビをつけっぱなしにしなければねむれないひとは、けっこういる。	
\\	「っぱなし」 
\\	テレビを 付[つ]けっぱなしにしなければ 眠[ねむ]れない 人[ひと]は、 結構[けっこう]いる。		
\\	ご飯
\\	方がおいしい。	ご飯の方がおいしい。	ごはんのほうがおいしい。	
\\	方 
\\	よる 
\\	「方」 
\\	ご 飯[はん]の 方[ほう]がおいしい。		
\\	鈴木さん
\\	方が若い。	鈴木さんの方が若い。	すずきさんのほうがわかい。	
\\	方 
\\	よる 
\\	「方」 
\\	鈴木[すずき]さんの 方[ほう]が 若[わか]い。		
\\	何も食
\\	寝ました。	何も食べないで寝ました。	なにもたべないでねました。	
\\	ないで	何[なに]も 食[た]べないで 寝[ね]ました。		
\\	こちらは、私の部屋でございます。	こちらは、私の部屋でございます。	こちらは、わたしのへやでございます。	
\\	こちらは、 私[わたし]の 部屋[へや]でございます。		
\\	10キロを走
\\	ばかりで、凄く疲れた。	10キロを走ったばかりで、凄く疲れた。	じゅっキロをはしったばかりで、すごくつかれた。	
\\	「~ばかり」	10キロを 走[はし]ったばかりで、 凄[すご]く 疲[つか]れた。		
\\	小林さん
\\	から, には]、返事が来なかった。	小林さんからだけには、返事が来なかった。	こばやしさんからだけには、へんじがこなかった。	
\\	「だけ」 
\\	小林[こばやし]さんからだけには、 返事[へんじ]が 来[こ]なかった。		
\\	パソコンの使
\\	方は、みんな知っているでしょう。	パソコンの使い方は、みんな知っているでしょう。	パソコンのつかいかたは、みんなしっているでしょう。	
\\	方 
\\	よる 
\\	「方」 
\\	パソコンの 使い方[つかいかた]は、みんな 知[し]っているでしょう。		
\\	私が書いたレポートを見
\\	いただ
\\	ますか。	私が書いたレポートを見ていただけますか。	わたしがかいたレポートをみていただけますか。	
\\	私[わたし]が 書[か]いたレポートを 見[み]ていただけますか。		
\\	相手に何も言
\\	ながら、自分の気持ちを分かってほしいのは単なる我が儘だと思わない?	相手に何も言わないながら、自分の気持ちを分かってほしいのは単なる我が儘だと思わない?	あいてになにもいわないながら、じぶんのきもちをわかってほしいのはたんなるわがままだとおもわない?	
\\	「ながら」 
\\	相手[あいて]に 何[なに]も 言[い]わないながら、 自分[じぶん]の 気持[きも]ちを 分[わ]かってほしいのは 単[たん]なる 我[わ]が儘だと 思[おも]わない?		
\\	もう家に帰るんだろう。/そうよ。	もう家に帰るんだろう。/そうよ。	もう{うち
\\	いえ}にかえるんだろう。/そうよ。	
\\	「でしょう」 
\\	「だろう」 
\\	もう 家[いえ]に 帰[かえ]るんだろう。/そうよ。		
\\	音楽を聴
\\	ながら、学校へ歩くのが好き。	音楽を聴きながら、学校へ歩くのが好き。	おんがくをききながら、がっこうへあるくのがすき。	
\\	「ながら」 
\\	音楽[おんがく]を 聴[き]きながら、 学校[がっこう]へ 歩[ある]くのが 好[す]き。		
\\	今日は忙しくて、朝ご飯
\\	食
\\	れなかった。	今日は忙しくて、朝ご飯しか食べられなかった。	きょうはいそがしくて、あさごはんしかたべられなかった。	
\\	「しか」	今日[きょう]は 忙[いそが]しくて、 朝[あさ]ご 飯[はん]しか 食[た]べられなかった。		
\\	彼は学生
\\	よう
\\	雰囲気ですね。	彼は学生のような雰囲気ですね。	かれはがくせいのようなふんいきですね。	
\\	よう (様) 
\\	な 
\\	彼[かれ]は 学生[がくせい]のような 雰囲気[ふんいき]ですね。		
\\	閉まるドアにご注意
\\	ください。	閉まるドアにご注意下さい。	しまるドアにごちゅういください。	
\\	「ください」 
\\	閉[し]まるドアにご 注意[ちゅうい] 下[くだ]さい。		
\\	中国語が読
\\	わけがない。	中国語が読めるわけがない。	ちゅうごくごがよめるわけがない。	
\\	「わけ」 
\\	ない 
\\	中国[ちゅうごく] 語[ご]が 読[よ]めるわけがない。		
\\	この乗車券は発売当日
\\	有効です。	この乗車券は発売当日のみ有効です。	このじょうしゃけんははつばいとうじつのみゆうこうです。	
\\	「のみ」 
\\	「だけ」	この 乗車[じょうしゃ] 券[けん]は 発売[はつばい] 当日[とうじつ]のみ 有効[ゆうこう]です。		
\\	何だよ!おばさん
\\	ばっかりじゃないか?	何だよ!おばさんばっかりじゃないか?	なんだよ!おばさんばっかりじゃないか?	
\\	「だけ」 
\\	「ばかり」	何[なに]だよ!おばさんばっかりじゃないか?		
\\	少々お待
\\	ください。	少々お待ちください。	しょうしょうおたちください。	
\\	「ください」 
\\	少々[しょうしょう]お 待[ま]ちください。		
\\	お寿司を食
\\	ことがある。	お寿司を食べたことがある。	おすしをたべたことがある。	
\\	「こと」 
\\	お 寿司[すし]を 食[た]べたことがある。		
\\	こうなったら、逃
\\	しかない。	こうなったら、逃げるしかない。	こうなったら、にげるしかない。	
\\	「しか」 
\\	こうなったら、 逃[に]げるしかない。		
\\	これ
\\	。	これしかない。	これしかない。	
\\	「しか」	これしかない。		
\\	ここに名前を書
\\	だけでいいですか?	ここに名前を書くだけでいいですか?	ここになまえをかくだけでいいですか?	
\\	「だけ」 
\\	ここに 名前[なまえ]を 書[か]くだけでいいですか?		
\\	ごめん、ついお前を呼
\\	じまった。	ごめん、ついお前を呼んじまった。	ごめん、ついおまえをよんじまった。	
\\	「~てしまう」	ごめん、ついお 前[まえ]を 呼[よ]んじまった。		
\\	あの肉を食
\\	のは難しい。	あの肉を食べるのは難しい。	あのにくをたべるのはむずかしい。	
\\	「~やすい、~にくい」 
\\	あの 肉[にく]を 食[た]べるのは 難[むずか]しい。		
\\	試験のために三時間も勉強した。	試験のために三時間も勉強した。	しけんのためにさんじかんもべんきょうした。	
\\	「も」 
\\	試験[しけん]のために三 時間[じかん]も 勉強[べんきょう]した。		
\\	アンケート対象は大学生
\\	です。	アンケート対象は大学生のみです。	アンケートたいしょうはだいがくのみです。	
\\	「のみ」 
\\	「だけ」	アンケート 対象[たいしょう]は 大学生[だいがくせい]のみです。		
\\	こちらから行
\\	方が早かった。	こちらから行った方が早かった。	こちらからいったほうがはやかった。	
\\	方 
\\	よる 
\\	「方」 
\\	こちらから 行[おこな]った 方[ほう]が 早[はや]かった。		
\\	マトリックス・レボリューションを観
\\	方がいいよ	マトリックス・レボリューションを観ない方がいいよ。	マトリックス・レボリューションをみないほうがいいよ。	
\\	方 
\\	よる 
\\	「方」 
\\	マトリックス・レボリューションを 観[み]ない 方[ほう]がいいよ。		
\\	いやだ。おばさん
\\	ばっかり。	いやだ。おばさんばっかり。	いやだ。おばさんばっかり。	
\\	「だけ」 
\\	「ばかり」	いやだ。おばさんばっかり。		
\\	ごめん、待
\\	せてしまって!	ごめん、待たせてしまって!	ごめん、またせてしまって!	
\\	「しまう」 
\\	ごめん、 待[ま]たせてしまって!		
\\	明日の会議に行
\\	れるんですか?	明日の会議に行かれるんですか?	あしたのかいぎにいかれるんですか?	
\\	明日[あした]の 会議[かいぎ]に 行[い]かれるんですか?		
\\	彼は、初心者
\\	ながらも、実力はプロと同じだ。	彼は、初心者ながらも、実力はプロと同じだ。	かれは、しょしんしゃながらも、じつりょくはプロとおなじだ。	
\\	「ながら」 
\\	ながらも 
\\	彼[かれ]は、 初心者[しょしんしゃ]ながらも、 実力[じつりょく]はプロと 同[おな]じだ。		
\\	りんご
\\	。	りんごだけ。	りんごだけ。	
\\	「だけ」	りんごだけ。		
\\	彼には、彼女がもったい
\\	すぎるよ。	彼には、彼女がもったいなさすぎるよ。	かれには、かのじょがもったいなさすぎるよ。	
\\	「すぎる」	彼[かれ]には、 彼女[かのじょ]がもったいなさすぎるよ。		
\\	恭子は全然女
\\	ぽくないね。	恭子は全然女っぽくないね。	きょうこはぜんぜんおんなっぽくないね。	
\\	「っぽい」: 
\\	恭子[きょうこ]は 全然[ぜんぜん] 女[おんな]っぽくないね。		
\\	あの肉は食
\\	にくい。	あの肉は食べにくい。	あのにくはたべにくい。	
\\	「~やすい、~にくい」 
\\	あの 肉[にく]は 食[た]べにくい。		
\\	今、家に帰
\\	ばかりです。	今、家に帰ったばかりです。	いま、{いえ;うち}にかえったばかりです。	
\\	「~ばかり」	今[いま]、 家[いえ]に 帰[かえ]ったばかりです。		
\\	朝ご飯を食
\\	としても、もう昼だからお腹が空いたでしょう。	朝ご飯を食べたとしても、もう昼だからお腹が空いたでしょう。	あさごはんをたべたとしても、もうひるだからおなかがすいたでしょう。	
\\	「とする」	朝[あさ]ご 飯[はん]を 食[た]べたとしても、もう 昼[ひる]だからお 腹[なか]が 空[あ]いたでしょう。		
\\	昨日、電話三回もしたよ!	昨日、電話三回もしたよ!	きのう、でんわさんかいもしたよ!	
\\	「も」 
\\	昨日[きのう]、 電話[でんわ]三 回[かい]もしたよ!		
\\	パリに行
\\	ことはありますか。	パリに行ったことはありますか。	バリにいったことはありますか。	
\\	「こと」 
\\	パリに 行[い]ったことはありますか。		
\\	そろそろ遅くな
\\	ゃうよ。	そろそろ遅くなっちゃうよ。	そろそろおそくなっちゃうよ。	
\\	「~てしまう」	そろそろ 遅[おそ]くなっちゃうよ。		
\\	そういう食
\\	方は体によくないよ。	そういう食べ方は体によくないよ。	そういうたべかたはからだによくないよ。	
\\	方 
\\	よる 
\\	「方」 
\\	そういう 食[た]べ 方[かた]は 体[からだ]によくないよ。		
\\	今日、田中さんはこないの?/こない
\\	らしい。	今日、田中さんはこないの?/こないらしい。	きょう、たなかさんはこないの?/こないらしい。	
\\	「~らしい」	今日[きょう]、 田中[たなか]さんはこないの?/こないらしい。		
\\	お手洗いはこのビルの二階にございます。	お手洗いはこのビルの二階にございます。	おてあらいはこのビルのにかいにございます。	
\\	お 手洗[てあら]いはこのビルの二 階[かい]にございます。		
\\	このピザはお好み焼き
\\	みたいじゃない?	このピザはお好み焼きみたいじゃない?	このピザはおこのみやきみたいじゃない?	
\\	「みたい」 
\\	このピザはお 好み焼[このみや]きみたいじゃない?		
\\	そんなに飲
\\	方がよかった。	そんなに飲まなかった方がよかった。	そんなにのまなかったほうがよかった。	
\\	方 
\\	よる 
\\	「方」 
\\	そんなに 飲[の]まなかった 方[ほう]がよかった。		
\\	日本語は読
\\	づらいな。	日本語は読みづらいな。	にほんごはよみづらいな。	
\\	「~にくい」 
\\	「~がたい」 
\\	「~づらい」 
\\	「~づらい」 
\\	日本語[にほんご]は 読[よ]みづらいな。		
\\	半分しか食べて
\\	ままで捨てちゃ駄目!	半分しか食べてないままで捨てちゃ駄目!	はんぶんしかたべてないままですてちゃダメ!	
\\	「まま」 
\\	半分[はんぶん]しか 食[た]べてないままで 捨[す]てちゃ 駄目[だめ]!		
\\	寝る時間がない
\\	忙しい。	寝る時間がないほど忙しい。	ねるじかんがないほどいそがしい。	
\\	「ほど」 
\\	寝[ね]る 時間[じかん]がないほど 忙[いそが]しい。		
\\	季節によ
\\	果物は美味しくなったり、不味くなったりする。	季節によって果物は美味しくなったり、不味くなったりする。	きせつによってくだものはおいしくなったり、まずくなったりする。	
\\	方 
\\	よる 
\\	「によって」 
\\	季節[きせつ]によって 果物[くだもの]は 美味[おい]しくなったり、 不味[まず]くなったりする。		
\\	大
\\	すぎるからトランクに入らないぞ。	大きすぎるからトランクに入らないぞ。	おおきすぎるからトランクにはいらないぞ。	
\\	「すぎる」	大[おお]きすぎるからトランクに 入[はい]らないぞ。		
\\	あいつに二時間も待
\\	せられた。	あいつに二時間も待たせられた。	あいつににじかんもまたせられた。	
\\	あいつに二 時間[じかん]も 待[ま]たせられた。		
\\	こちらは、私の部屋です。	こちらは、私の部屋です。	こちらは、わたしのへやです。	
\\	こちらは、 私[わたし]の 部屋[へや]です。		
\\	ゲームに嵌まっちゃって、最近パソコンを使
\\	まくっているよ。	ゲームに嵌まっちゃって、最近パソコンを使いまくっているよ。	ゲームにはまっちゃって、さいきんパソコンをつかいまくっているよ。	
\\	「まくる」	ゲームに 嵌[はま]まっちゃって、 最近[さいきん]パソコンを 使[つか]いまくっているよ。		
\\	その人は学生だろう。	その人は学生だろう。	そのひとはがくせいだろう。	
\\	「~そう」 
\\	その 人[ひと]は 学生[がくせい]だろう。		
\\	この歌
\\	を]を歌わなかった。	この歌だけを歌わなかった。	このうただけをうたわなかった。	
\\	「だけ」 
\\	この 歌[うた]だけを 歌[うた]わなかった。		
\\	これ
\\	見ない。	これだけ見ない。	これだけみない。	
\\	「しか」	これだけ 見[み]ない。		
\\	宿題を
\\	する, 
\\	、授業に行くのは、やめた方がいいよ。	宿題をしないで、授業に行くのは、やめた方がいいよ。	しゅくだいをしないで、じゅぎょうにいくのは、やめたほうがいいよ。	
\\	ないで	宿題[しゅくだい]をしないで、 授業[じゅぎょう]に 行[い]くのは、やめた 方[ほう]がいいよ。		
\\	毎日会いに行った
\\	そうです。	毎日会いに行ったそうです。	まいにちあいにいったそうです。	
\\	「~そうだ」	毎日[まいにち] 会[あ]いに 行[い]ったそうです。		
\\	最近は仕事
\\	ばっかだよ。	最近は仕事ばっかだよ。	さいきんはしごとばっかだよ。	
\\	「だけ」 
\\	「ばかり」	最近[さいきん]は 仕事[しごと]ばっかだよ。		
\\	金魚がもう死
\\	ゃった。	金魚がもう死んじゃった。	きんぎょがもうしんじゃった。	
\\	「~てしまう」	金魚[きんぎょ]がもう 死[し]んじゃった。		
\\	こうなったら、もう
\\	やる]っきゃない!	こうなったら、もうやるっきゃない!	こうなったら、もうやるっきゃない!	
\\	「しか」- 「っきゃ」, 
\\	「しか」	こうなったら、もうやるっきゃない!		
\\	もう腐っているから、捨
\\	しかないよ。	もう腐っているから、捨てるしかないよ。	もうくさっているから、すてるしかないよ。	
\\	「しか」 
\\	もう 腐[くさ]っているから、 捨[す]てるしかないよ。		
\\	彼との忘
\\	がたい思い出を大切にしている。	彼との忘れがたい思い出を大切にしている。	かれとのわすれがたいおもいでをたいせつにしている。	
\\	「~にくい」 
\\	「~がたい」 
\\	「~づらい」 
\\	「~がたい」 
\\	彼[かれ]との 忘[わす]れがたい 思い出[おもいで]を 大切[たいせつ]にしている。		
\\	この
\\	ままで宜しいですか?	このままで宜しいですか?	このままでよろしいですか?	
\\	「まま」 
\\	このままで 宜[よろ]しいですか?		
\\	その格好
\\	ままでクラブに入れないよ。	その格好のままでクラブに入れないよ。	そのかっこうのままでクラブにはいれないよ。	
\\	「まま」 
\\	その 格好[かっこう]のままでクラブに 入[い]れないよ。		
\\	先生、お聞
\\	したいことがありますが。	先生、お聞きしたいことがありますが。	せんせい、おききしたいことがありますが。	
\\	お 
\\	する	先生[せんせい]、お 聞[き]きしたいことがありますが。		
\\	キムさん
\\	より鈴木さんの方が若い。	キムさんより鈴木さんの方が若い。	キムさんよりすずきさんのほうがわかい。	
\\	方 
\\	よる 
\\	「より」 
\\	キムさんより 鈴木[すずき]さんの 方[ほう]が 若[わか]い。		
\\	どちらからいらっしゃいましたか。	どちらからいらっしゃいましたか。	どちらからいらっしゃいましたか。	
\\	どちらからいらっしゃいましたか。		
\\	毎日ケーキを食べて、2キロ太
\\	しまいました。	毎日ケーキを食べて、2キロ太ってしまいました。	まいにちケーキをたべて、2キロふとってしまいました。	
\\	「しまう」 
\\	毎日[まいにち]ケーキを 食[た]べて、2キロ 太[ふと]ってしまいました。		
\\	明日も雨でしょう。	明日も雨でしょう。	あしたもあめでしょう。	
\\	「でしょう」 
\\	明日[あした]も 雨[あめ]でしょう。		
\\	勉強を
\\	する]ば
\\	する, 
\\	、頭がよくなるよ。	勉強をすればするほど、頭がよくなるよ。	べんきょうをすればするほど、あたまがよくなるよ。	
\\	「ほど」 
\\	勉強[べんきょう]をすればするほど、 頭[あたま]がよくなるよ。		
\\	皆で、もう全部食べてしまった
\\	ぽいよ。	皆で、もう全部食べてしまったっぽいよ。	みんなで、もうぜんぶたべてしまったっぽいよ。	
\\	「っぽい」: 
\\	皆[みな]で、もう 全部[ぜんぶ] 食[た]べてしまったっぽいよ。		
\\	結局、嫌なことを
\\	する]せてしまった。	結局、嫌なことをさせてしまった。	けっきょく、いやなことをさせてしまった。	
\\	「しまう」 
\\	結局[けっきょく]、 嫌[いや]なことをさせてしまった。		
\\	韓国料理は食
\\	ば食
\\	、おいしくなる。	韓国料理は食べれば食べるほど、おいしくなる。	かんこくりょうりはたべればたべるほど、おいしくなる。	
\\	「ほど」 
\\	韓国[かんこく] 料理[りょうり]は 食[た]べれば 食[た]べるほど、おいしくなる。		
\\	こちらにご覧
\\	ください。	こちらにご覧下さい。	こちらにごらんください。	
\\	「ください」 
\\	「ご覧になる」	こちらにご 覧[らん] 下[くだ]さい。		
\\	そういうのを見
\\	ことがなかった。	そういうのを見たことがなかった。	そういうのをみたことがなかった。	
\\	「こと」 
\\	そういうのを 見[み]たことがなかった。		
\\	これ
\\	見ない。	これしか見ない。	これしかみない。	
\\	「しか」	これしか 見[み]ない。		
\\	直美ちゃんと遊
\\	ばっかりでしょう!	直美ちゃんと遊ぶばっかりでしょう!	なおみちゃんとあそぶばっかりでしょう!	
\\	「だけ」 
\\	「ばかり」	直美[なおみ]ちゃんと 遊[あそ]ぶばっかりでしょう!		
\\	いくら英語を勉強しても、うまくならないの。/つまり、語学には、能力がないという訳か。/失礼ね。	いくら英語を勉強しても、うまくならないの。/つまり、語学には、能力がないという訳か。/失礼ね。	いくらえいごをべんきょうしても、うまくならないの。/つまり、ごがくには、のうりょくがないというわけか。しつれいね。	
\\	「わけ」訳	いくら 英語[えいご]を 勉強[べんきょう]しても、うまくならないの。/つまり、 語学[ごがく]には、 能力[のうりょく]がないという 訳[わけ]か。/ 失礼[しつれい]ね。		
\\	いらっしゃい!	いらっしゃい!	いらっしゃい!	
\\	いらっしゃい!		
\\	ちょっと怒
\\	よう
\\	聞こえた。	ちょっと怒ったように聞こえた。	ちょっとおこったようにきこえた。	
\\	よう (様) 
\\	に 
\\	ちょっと 怒[おこ]ったように 聞[き]こえた。		
\\	あの子は子供
\\	らしくない。	あの子は子供らしくない。	あのこはこどもらしくない。	
\\	「~らしい」	あの 子[こ]は 子供[こども]らしくない。		
\\	テレビを観
\\	ながら、宿題をする。	テレビを観ながら、宿題をする。	テレビをみながら、しゅくだいをする。	
\\	「ながら」 
\\	テレビを 観[み]ながら、 宿題[しゅくだい]をする。		
\\	今から行
\\	としたら、9時に着くと思います。	今から行くとしたら、9時に着くと思います。	いまからいくとしたら、くじにつくとおもいます。	
\\	「とする」	今[いま]から 行[い]くとしたら、9 時[じ]に 着[つ]くと 思[おも]います。		
\\	この仕事は
\\	も早く出来ます。	この仕事は誰よりも早く出来ます。	このしごとはだれよりもはやくできます。	
\\	方 
\\	よる 
\\	「より」 
\\	「より」 
\\	この 仕事[しごと]は 誰[だれ]よりも 早[はや]く 出来[でき]ます。		
\\	あの人は何なの?/美由紀さんの友達
\\	らしいですよ。	あの人は何なの?/美由紀さんの友達らしいですよ。	あのひとはなんなの?/みゆきさんのともだちらしいですよ。	
\\	「~らしい」	あの 人[ひと]は 何[なに]なの?/ 美由紀[みゆき]さんの 友達[ともだち]らしいですよ。		
\\	これから行
\\	ところでした。	これから行くところでした。	これからいくところでした。	
\\	「ところ」 
\\	これから 行[い]くところでした。		
\\	歯を磨
\\	、学校に行っちゃいました。	歯を磨かないで、学校に行っちゃいました。	はをみがかないで、がっこうにいっちゃいました。	
\\	ないで	歯[は]を 磨[みが]かないで、 学校[がっこう]に 行[い]っちゃいました。		
\\	アリスさん、もう召し上
\\	ましたか。	アリスさん、もう召し上がりましたか。	アリスさん、もうめしあがりましたか。	
\\	アリスさん、もう 召し上[めしあ]がりましたか。		
\\	今日は飲みに行こうか?/それは、裕子によ
\\	ね。	今日は飲みに行こうか?/それは、裕子によるね。	きょうはのみにいこうか?/それは、ゆうこによるね。	
\\	方 
\\	よる 
\\	「によって」 
\\	よる	今日[きょう]は 飲[の]みに 行[い]こうか?/それは、 裕子[ゆうこ]によるね。		
\\	今年、十キロも太っちゃった!	今年、十キロも太っちゃった!	ことし、じゅっキロもふとちゃった!	
\\	「も」 
\\	今年[ことし]、十キロも 太[ふと]っちゃった!		
\\	は、ハードディスクの容量が大
\\	ば大
\\	もっとたくさんの曲が保存できます。	
\\	は、ハードディスクの容量が大きければ大きいほどもっとたくさんの曲が保存できます。	
\\	は、ハードディスクのようりょうがおおきければおおきいほどもっとたくさんのきょくがほぞんできます。	
\\	「ほど」 
\\	は、ハードディスクの 容量[ようりょう]が 大[おお]きければ 大[おお]きいほどもっとたくさんの 曲[きょく]が 保存[ほぞん]できます。		
\\	その人は学生でしょう。	その人は学生でしょう。	そのひとはがくせいでしょう。	
\\	「~そう」 
\\	その 人[ひと]は 学生[がくせい]でしょう。		
\\	早くきて。映画は、今ちょうど良
\\	ところだよ。	早くきて。映画は、今ちょうど良いところだよ。	はやくきて。えいがは、いまちょうどいいところだよ。	
\\	「ところ」 
\\	早[はや]くきて。 映画[えいが]は、 今[いま]ちょうど 良[よ]いところだよ。		
\\	昼ご飯を食
\\	ばっかなのに、もうお腹が空いた。	昼ご飯を食べたばっかなのに、もうお腹が空いた。	ひるごはんをたべたばっかなのに、もうおなかがすいた。	
\\	「~ばかり」 
\\	昼[ひる]ご 飯[はん]を 食[た]べたばっかなのに、もうお 腹[なか]が 空[あ]いた。		
\\	これ
\\	見る。	これだけ見る。	これだけみる。	
\\	「しか」	これだけ 見[み]る。		
\\	今日は、どちらへいらっしゃいますか。	今日は、どちらへいらっしゃいますか。	きょうは、どちらへいらっしゃいますか。	
\\	今日[きょう]は、どちらへいらっしゃいますか。		
\\	赤ちゃんは、静
\\	方が好き。	赤ちゃんは、静かな方が好き。	あかちゃんは、しずかなほうがすき。	
\\	方 
\\	よる 
\\	「方」 
\\	赤[あか]ちゃんは、 静[しず]かな 方[ほう]が 好[す]き。		
\\	私はキムと申
\\	ます。	私はキムと申します。	わたしはキムともうします。	
\\	私[わたし]はキムと 申[もう]します。		
\\	ポップコーンを食
\\	ながら、映画を観た。	ポップコーンを食べながら、映画を観た。	ポップコーンをたべながら、えいがをみた。	
\\	「ながら」 
\\	ポップコーンを 食[た]べながら、 映画[えいが]を 観[み]た。		
\\	窓が開
\\	放しだったので、蚊が一杯入った。	窓が開けっ放しだったので、蚊が一杯入った。	まどがあけっぱなしだったので、かがいっぱいはいった。	
\\	「っぱなし」 
\\	窓[まど]が 開けっ放[あけっぱな]しだったので、 蚊[か]が一 杯[はい] 入[はい]った。		
\\	昨日の喧嘩で何も言
\\	ばかりに、平気な顔をしている。	昨日の喧嘩で何も言わなかったばかりに、平気な顔をしている。	きのうのケンカでなにもいわなかったばかりに、へいきなかおをしている。	
\\	「ばかり」 
\\	昨日[きのう]の 喧嘩[けんか]で 何[なに]も 言[い]わなかったばかりに、 平気[へいき]な 顔[かお]をしている。		
\\	早く帰
\\	べき。	早く帰るべき。	はやくかえるべき。	
\\	「べく」 
\\	早[はや]く 帰[かえ]るべき。		
\\	最近のウィルスは強力で、プログラムを実行
\\	する 
\\	が
\\	する]まいが、ページを見るだけで感染するらしい。	最近のウィルスは強力で、プログラムを実行しようがしまいが、ページを見るだけで感染するらしい。	さいきんのウィルスはきょうりょくで、プログラムをじっこうしようがしまいが、ページをみるだけでかんせんするらしい。	
\\	最近[さいきん]のウィルスは 強力[きょうりょく]で、プログラムを 実行[じっこう]しようがしまいが、ページを 見[み]るだけで 感染[かんせん]するらしい。		
\\	相手は剣の達人だ。そう簡単には勝
\\	まい。	相手は剣の達人だ。そう簡単には勝てまい。	あいてはつるぎのたつじんだ。そうかんたんにはかてまい。	
\\	相手[あいて]は 剣[けん]の 達人[たつじん]だ。そう 簡単[かんたん]には 勝[か]てまい。		
\\	企業内の顧客データを利用
\\	、彼の行方を調べることが出来た。。	企業内の顧客データを利用し、彼の行方を調べることが出来た。	きぎょうないのこきゃくデータをりようし、かれのゆくえをしらべることができた。	
\\	企業[きぎょう] 内[ない]の 顧客[こきゃく]データを 利用[りよう]し、 彼[かれ]の 行方[ゆくえ]を 調[しら]べることが 出来[でき]た。		
\\	今後50年、人間が直面するであろう問題に正面から向き合って、自ら解決をはかりつつ、そのノウハウが次の産業となるシナリオを考えたい。	今後50年、人間が直面するであろう問題に正面から向き合って、自ら解決をはかりつつ、そのノウハウが次の産業となるシナリオを考えたい。	こんご50ねん、にんげんがちょくめんするであろうもんだいにしょうめんからむきあって、みずからかいけつをはかりつつ、そのノウハウがつぎのさんぎょうとなるシナリオをかんがえたい。	
\\	「であろう」 
\\	今後[こんご] 
\\	年[ねん]、 人間[にんげん]が 直面[ちょくめん]するであろう 問題[もんだい]に 正面[しょうめん]から 向き合[むきあ]って、 自[みずか]ら 解決[かいけつ]をはかりつつ、そのノウハウが 次[つぎ]の 産業[さんぎょう]となるシナリオを 考[かんが]えたい。		
\\	コーディングが好きな開発者は、ちゃんとしたドキュメント作成と十分なテストを怠
\\	きらいがある。	コーディングが好きな開発者は、ちゃんとしたドキュメント作成と十分なテストを怠るきらいがある。	コーディングがすきなかいはつしゃは、ちゃんとしたドキュメントさくせいとじゅうぶんなテストをおこたるきらいがある。	
\\	「きらいがある」	コーディングが 好[す]きな 開発[かいはつ] 者[しゃ]は、ちゃんとしたドキュメント 作成[さくせい]と 十分[じゅうぶん]なテストを 怠[おこた]るきらいがある。		
\\	そんな謎
\\	めいた顔をされても、うまく説明できないよ。。	そんな謎めいた顔をされても、うまく説明できないよ。	そんななぞめいたかおをされても、うまくせつめいできないよ。	
\\	「めく」 
\\	そんな 謎[なぞ]めいた 顔[かお]をされても、うまく 説明[せつめい]できないよ。		
\\	妻はルイヴィトンのバッグを欲
\\	がっているんだけど、そんなもん、買えるわけないでしょう!	妻はルイヴィトンのバッグを欲しがっているんだけど、そんなもん、買えるわけないでしょう!	つまはルイヴィトンのバッグをほしがっているんだけど、そんなもん、かえるわけないでしょう!	
\\	「~がる」	妻[つま]はルイヴィトンのバッグを 欲[ほ]しがっているんだけど、そんなもん、 買[か]えるわけないでしょう!		
\\	そんな無茶な手段は認
\\	まい!	そんな無茶な手段は認めますまい!	そんなむちゃなしゅだんはみとめますまい!	
\\	そんな 無茶[むちゃ]な 手段[しゅだん]は 認[みと]めますまい!		
\\	預金者が大手銀行を相手取って訴訟を起こすケースも出ており、金融庁は被害者の救済を優先させて、金融機関に犯罪防止対策の強化を促すべきだと判断。	預金者が大手銀行を相手取って訴訟を起こすケースも出ており、金融庁は被害者の救済を優先させて、金融機関に犯罪防止対策の強化を促すべきだと判断。	よきんものがおおてぎんこうをあいてとってそしょうをおこすケースもでており、きんゆうちょうはひがいしゃのきゅうさいをゆうせんさせて、きんゆうきかんにはんざいぼうしたいさくのきょうかをうながすべきだとはんだん。	
\\	「べき」 
\\	預金[よきん] 者[しゃ]が 大手[おおて] 銀行[ぎんこう]を 相手取[あいてど]って 訴訟[そしょう]を 起[お]こすケースも 出[で]ており、 金融[きんゆう] 庁[ちょう]は 被害[ひがい] 者[しゃ]の 救済[きゅうさい]を 優先[ゆうせん]させて、 金融[きんゆう] 機関[きかん]に 犯罪[はんざい] 防止[ぼうし] 対策[たいさく]の 強化[きょうか]を 促[うなが]すべきだと 判断[はんだん]。		
\\	お弁当を買うお金
\\	さえなかった。	お弁当を買うお金さえなかった。	おべんとうをかうおきんさえなかった。	
\\	「(で)さえ」 
\\	お 弁当[べんとう]を 買[か]うお 金[かね]さえなかった。		
\\	この天才の私
\\	すら分からなかった。	この天才の私ですら分からなかった。	このてんさいのわたしですらわからなかった。	
\\	「(で)すら」 
\\	「(で)さえ」 
\\	この 天才[てんさい]の 私[わたし]ですら 分[わ]からなかった。		
\\	国土交通省は2年後に利用率を70%まで引き上げる考えで、買い替え時に利用する気になるかどうかがカギになりそうだ。	国土交通省は2年後に利用率を70%まで引き上げる考えで、買い替え時に利用する気になるかどうかがカギになりそうだ。	こくどこうつうしょうは2ねんごにりようりつを70%までひきあげるかんがえで、かいかえときにりようするきになるかどうかがカギになりそうだ。	
\\	「である」 
\\	国土[こくど] 交通省[こうつうしょう]は 2[に] 年[ねん] 後[ご]に 利用[りよう] 率[りつ]を 7[なな] 0[ぜろ] 
\\	[ぱーせんと]まで 引き上[ひきあ]げる 考[かんが]えで、 買い替[かいか]え 時[じ]に 利用[りよう]する 気[き]になるかどうかがカギになりそうだ。		
\\	教科書を読
\\	そばから忘れてしまうので勉強ができない。	教科書を読むそばから忘れてしまうので勉強ができない。	きょうかしょをよむそばからわすれてしまうのでべんきょうができない。	
\\	「そばから」 
\\	教科書[きょうかしょ]を 読[よ]むそばから 忘[わす]れてしまうので 勉強[べんきょう]ができない。		
\\	あの人は、二度と嘘を付
\\	まいと誓ったのです。	あの人は、二度と嘘を付くまいと誓ったのです。	あのにんは、にどとうそをつくまいとちかったのです。	
\\	あの 人[ひと]は、 二度[にど]と 嘘[うそ]を 付[つ]くまいと 誓[ちか]ったのです。		
\\	彼女は朝早く起こされるのを嫌
\\	がるタイプです。	彼女は朝早く起こされるのを嫌がるタイプです。	かのじょはあさはやくおこされるのをいやがるタイプです。	
\\	「~がる」	彼女[かのじょ]は 朝[あさ] 早[はや]く 起[お]こされるのを 嫌[いや]がるタイプです。		
\\	先生と相談
\\	あげく、退学することにした。	先生と相談のあげく、退学することにした。	せんせいとそうだんのあげく、たいがくすることにした。	
\\	「~あげく(挙句)」 
\\	先生[せんせい]と 相談[そうだん]のあげく、 退学[たいがく]することにした。		
\\	私は緊張しすぎて、ちらっと見ること
\\	すら出来ませんでした。	私は緊張しすぎて、ちらっと見ることすら出来ませんでした。	わたしはきんちょうしすぎて、ちらっとみることすらできませんでした。	
\\	「(で)すら」 
\\	「(で)さえ」 
\\	私[わたし]は 緊張[きんちょう]しすぎて、ちらっと 見[み]ることすら 出来[でき]ませんでした。		
\\	多くの大学生は、締切日ぎりぎりまで、宿題を
\\	やる, 
\\	きらいがある。	多くの大学生は、締切日ぎりぎりまで、宿題をやらないきらいがある。	おおくのだいがくせいは、しめきりびぎりぎりまで、しゅくだいをやらないきらいがある。	
\\	「きらいがある」	多[おお]くの 大学生[だいがくせい]は、 締切[しめきり] 日[び]ぎりぎりまで、 宿題[しゅくだい]をやらないきらいがある。		
\\	いつも皮肉
\\	めいた言い方をしたら、みんなを嫌がらせるよ。。	いつも皮肉めいた言い方をしたら、みんなを嫌がらせるよ。	いつもひにくめいたいいかたをしたら、みんなをいやがらせるよ。 いやがる【嫌がる】 〔拒否する〕 
\\	「めく」 
\\	いつも 皮肉[ひにく]めいた 言い方[いいかた]をしたら、みんなを 嫌[いや]がらせるよ。		
\\	安全措置を忘
\\	べからず。	安全措置を忘れるべからず。	あんぜんそちをわすれるべからず。	
\\	「べからず」 
\\	安全[あんぜん] 措置[そち]を 忘[わす]れるべからず。		
\\	ゴミ捨
\\	べからず。	ゴミ捨てるべからず。	ゴミすてるべからず。	
\\	「べからず」 
\\	ゴミ 捨[す]てるべからず。		
\\	花火は、火薬と金属の粉末を混ぜたものに火を付け、燃焼時の火花を楽しむためのもの。	花火は、火薬と金属の粉末を混ぜたものに火を付け、燃焼時の火花を楽しむためのもの。	はなびは、かやくときんぞくのふんまつをまぜたものにひをづけ、ねんしょうときのひばなをたのしむためのもの。	
\\	花火[はなび]は、 火薬[かやく]と 金属[きんぞく]の 粉末[ふんまつ]を 混[ま]ぜたものに 火[ひ]を 付[つ]け、 燃焼[ねんしょう] 時[じ]の 火花[ひばな]を 楽[たの]しむためのもの。		
\\	このままでは、個人情報が漏洩
\\	する]かねないので、速やかに対応をお願い致します。	このままでは、個人情報が漏洩しかねないので、速やかに対応をお願い致します。	このままでは、こじんじょうほうがろうえいしかねないので、すみやかにたいおうをおねがいいたします。	
\\	「~かねる」	このままでは、 個人[こじん] 情報[じょうほう]が 漏洩[ろうせつ]しかねないので、 速[すみ]やかに 対応[たいおう]をお 願[ねが]い 致[いた]します。		
\\	何かを買う前に本当に必要かどうかをよく考
\\	べきだ。	何かを買う前に本当に必要かどうかをよく考えるべきだ。	なにかをかうまえにほんとうにひつようかどうかをよくかんがえるべきだ。	
\\	「べき」 
\\	何[なに]かを 買[か]う 前[まえ]に 本当[ほんとう]に 必要[ひつよう]かどうかをよく 考[かんが]えるべきだ。		
\\	父親は病気
\\	がちで、みんなが心配している。	父親は病気がちで、みんなが心配している。	ちちおやはびょうきがちで、みんながしんぱいしている。	
\\	「~がち」	父親[ちちおや]は 病気[びょうき]がちで、みんなが 心配[しんぱい]している。		
\\	「あんたと関係ない」と言
\\	ばかりに彼女は彼を無視していた。	「あんたと関係ない」と言わんばかりに彼女は彼を無視していた。	「あんたとかんけいない」といわんばかりにかのじょはかれをむししていた。	
\\	「ばかり」 
\\	「あんたと 関係[かんけい]ない」と 言[い]わんばかりに 彼女[かのじょ]は 彼[かれ]を 無視[むし]していた。		
\\	大学
\\	おろか、高校すら卒業しなかった。	大学はおろか、高校すら卒業しなかった。	だいがくはおろか、こうこうすらそつぎょうしなかった。	
\\	「おろか」 
\\	大学[だいがく]はおろか、 高校[こうこう]すら 卒業[そつぎょう]しなかった。		
\\	やむ
\\	を得ない事由により手続が遅れた場合、必ずご連絡下さい。	やむを得ない事由により手続が遅れた場合、必ずご連絡下さい。	やむをえないじゆうによりてつづきがおくれたばあい、かならずごれんらくください。	
\\	「やむを得ない」	やむを 得[え]ない 事由[じゆう]により 手続[てつづき]が 遅[おく]れた 場合[ばあい]、 必[かなら]ずご 連絡[れんらく] 下[くだ]さい。		
\\	もちろん、生徒数減少の現在、学科の新設は困難であろうが、職業科の統廃合や科内コースの改編などで時代に合わせた変革が求められているはずである。	もちろん、生徒数減少の現在、学科の新設は困難であろうが、職業科の統廃合や科内コースの改編などで時代に合わせた変革が求められているはずである。	もちろん、せいとすうげんしょうのげんざい、がっかのしんせつはこんなんであろうが、しょくぎょうかのとうはいごうやかないコースのかいへんなどでじだいにあわせたへんかくがもとめられているはずである。	
\\	「であろう」 
\\	もちろん、 生徒[せいと] 数[すう] 減少[げんしょう]の 現在[げんざい]、 学科[がっか]の 新設[しんせつ]は 困難[こんなん]であろうが、 職業[しょくぎょう] 科[か]の 統廃合[とうはいごう]や 科[か] 内[ない]コースの 改編[かいへん]などで 時代[じだい]に 合[あ]わせた 変革[へんかく]が 求[もと]められているはずである。		
\\	打合せは毎週2時から始
\\	はずじゃないですか?	打合せは毎週2時から始まるはずじゃないですか?	うちあわせはまいしゅう2ときからはじまるはずじゃないですか?	
\\	「はず」 
\\	打合[うちあわ]せは 毎週[まいしゅう]2 時[じ]から 始[はじ]まるはずじゃないですか?		
\\	彼は油
\\	まみれになりながら、車の修理に頑張りました。	彼は油まみれになりながら、車の修理に頑張りました。	かれはあぶらまみれになりながら、くるまのしゅうりにがんばりました。	
\\	「まみれ」 
\\	彼[かれ]は 油[あぶら]まみれになりながら、 車[くるま]の 修理[しゅうり]に 頑張[がんば]りました。		
\\	留守
\\	がちなご家庭には、犬よりも、猫の方がお勧めです。	留守がちなご家庭には、犬よりも、猫の方がお勧めです。	るすがちなごかていには、いぬよりも、ねこのほうがおすすめです。	
\\	「~がち」	留守[るす]がちなご 家庭[かてい]には、 犬[いぬ]よりも、 猫[ねこ]の 方[ほう]がお 勧[すす]めです。		
\\	漢字
\\	おろか、平仮名さえ読めないよ!	漢字はおろか、平仮名さえ読めないよ!	かんじはおろか、ひらがなさえよめないよ!	
\\	「おろか」 
\\	漢字[かんじ]はおろか、 平仮名[ひらがな]さえ 読[よ]めないよ!		
\\	あいつが大学に入
\\	うが入
\\	まいが、俺とは関係ないよ。	あいつが大学に入ろうが入るまいが、俺とは関係ないよ。	あいつがだいがくにいろうがいるまいが、おれとはかんけいないよ。	
\\	あいつが 大学[だいがく]に 入[にゅう]ろうが 入[はい]るまいが、 俺[おれ]とは 関係[かんけい]ないよ。		
\\	「人」の漢字
\\	すら知らない生徒は、いないでしょ!	「人」の漢字すら知らない生徒は、いないでしょ!	「にん」のかんじすらしらないせいとは、いないでしょ!	
\\	「人」 
\\	「(で)すら」 
\\	「(で)さえ」 
\\	人[ひと]」の 漢字[かんじ]すら 知[し]らない 生徒[せいと]は、いないでしょ!		
\\	運動を始めるのが早かろうが遅かろうが、健康にいいというのは変わりません。	運動を始めるのが早かろうが遅かろうが、健康にいいというのは変わりません。	うんどうをはじめるのがはやかろうがおそかろうが、けんこうにいいというのはかわりません。	
\\	「かろう」 
\\	「い」 
\\	運動[うんどう]を 始[はじ]めるのが 早[はや]かろうが 遅[おそ]かろうが、 健康[けんこう]にいいというのは 変[か]わりません。		
\\	この料理は美味
\\	はずだったが、焦げちゃって、まずくなった。	この料理は美味いはずだったが、焦げちゃって、まずくなった。	このりょうりはおいしいはずだったが、こげちゃって、まずくなった。	
\\	「はず」 
\\	この 料理[りょうり]は 美味[うま]いはずだったが、 焦[こ]げちゃって、まずくなった。		
\\	色々予定してあるから、今年は楽しいクリスマス
\\	はず。	色々予定してあるから、今年は楽しいクリスマスのはず。	いろいろよていしてあるから、こんねんはたのしいクリスマスのはず。	
\\	「はず」 
\\	色々[いろいろ] 予定[よてい]してあるから、 今年[ことし]は 楽[たの]しいクリスマスのはず。		
\\	みんなイタリアに行
\\	がってるんだけど、私の予算で行けるかどうかはとても怪しい。	みんなイタリアに行きたがってるんだけど、私の予算で行けるかどうかはとても怪しい。	みんなイタリアにいきたがってるんだけど、わたしのよさんでいけるかどうかはとてもあやしい。	
\\	「~がる」	みんなイタリアに 行[い]きたがってるんだけど、 私[わたし]の 予算[よさん]で 行[い]けるかどうかはとても 怪[あや]しい。		
\\	紅葉が始まり、すっかり秋
\\	めいた空気になってきた。	紅葉が始まり、すっかり秋めいた空気になってきた。	こうようがはじまり、すっかりあきめいたくうきになってきた。	
\\	「めく」 
\\	紅葉[こうよう]が 始[はじ]まり、すっかり 秋[あき]めいた 空気[くうき]になってきた。		
\\	早く帰
\\	べく、準備をし始めた。	早く帰るべく、準備をし始めた。	はやくかえるべく、じゅんびをしはじめた。	
\\	「べく」 
\\	早[はや]く 帰[かえ]るべく、 準備[じゅんび]をし 始[はじ]めた。		
\\	私
\\	さえ出来れば、あんたには楽ちんでしょう。	私でさえ出来れば、あんたには楽ちんでしょう。	わたしでさえできれば、あんたにはらくちんでしょう。	
\\	「(で)さえ」 
\\	私[わたし]でさえ 出来[でき]れば、あんたには 楽[らく]ちんでしょう。		
\\	言語は簡単にマスターできること
\\	言語は簡単にマスターできることではない。	げんごはかんたんにマスターできることではない。	
\\	「である」	言語[げんご]は 簡単[かんたん]にマスターできることではない。		
\\	休日であろうが、なかろうが、この仕事では関係ないみたい。	休日であろうが、なかろうが、この仕事では関係ないみたい。	きゅうじつであろうが、なかろうが、このしごとではかんけいないみたい。	
\\	「かろう」 
\\	「い」 
\\	休日[きゅうじつ]であろうが、なかろうが、この 仕事[しごと]では 関係[かんけい]ないみたい。		
\\	これは不公平
\\	でしょうか。	これは不公平ではないでしょうか。	これはふこうへいではないでしょうか。	
\\	「である」	これは 不公平[ふこうへい]ではないでしょうか。		
\\	上司の話を聞くと、どうしても海外に出張を
\\	する]を得ないようです。	上司の話を聞くと、どうしても海外に出張をせざるを得ないようです。	じょうしのはなしをきくと、どうしてもかいがいにしゅっちょうをせざるをえないようです。	
\\	「~ざるを得ない」	上司[じょうし]の 話[はなし]を 聞[き]くと、どうしても 海外[かいがい]に 出張[しゅっちょう]をせざるを 得[え]ないようです。		
\\	このレストランは安いと思
\\	きや、会計は5千円以上だった!	このレストランは安いと思いきや、会計は5千円以上だった!	このレストランはやすいとおもいきや、かいけいは5せんえんいじょうだった!	
\\	「思いきや」 
\\	このレストランは 安[やす]いと 思[おも]いきや、 会計[かいけい]は 
\\	千[せん] 円[えん] 以上[いじょう]だった!		
\\	「食べてみよう」と言
\\	が早いか、口の中に放り込んだ。	「食べてみよう」と言ったが早いか、口の中に放り込んだ。	「たべてみよう」といったがはやいか、くちのなかにほうりこんだ。	
\\	「が早いか」 
\\	食[た]べてみよう」と 言[い]ったが 早[はや]いか、 口[くち]の 中[なか]に 放り込[ほうりこ]んだ。		
\\	封筒には写真が数枚入
\\	、手紙が添えられていた。	封筒には写真が数枚入っており、手紙が添えられていた。	ふうとうにはしゃしんがすうまいはいっており、てがみがそえられていた。	
\\	封筒[ふうとう]には 写真[しゃしん]が 数[すう] 枚[まい] 入[はい]っており、 手紙[てがみ]が 添[そ]えられていた。		
\\	このテレビがこれ以上壊れたら、新しいのを買
\\	を得ないな。	このテレビがこれ以上壊れたら、新しいのを買わざるを得ないな。	このテレビがこれいじょうこわれたら、あたらしいのをかわざるをえないな。	
\\	「~ざるを得ない」	このテレビがこれ 以上[いじょう] 壊[こわ]れたら、 新[あたら]しいのを 買[か]わざるを 得[え]ないな。		
\\	ずっと我慢してきたが、この状態だと歯医者さんに行
\\	を得ない。	ずっと我慢してきたが、この状態だと歯医者さんに行かざるを得ない。	ずっとがまんしてきたが、このじょうたいだとはいしゃさんにいかざるをえない。	
\\	「~ざるを得ない」	ずっと 我慢[がまん]してきたが、この 状態[じょうたい]だと 歯医者[はいしゃ]さんに 行[い]かざるを 得[え]ない。		
\\	この埃
\\	だらけのテレビをちゃんと拭いてくれない?	この埃だらけのテレビをちゃんと拭いてくれない?	このほこりだらけのテレビをちゃんとぬぐいてくれない?	
\\	「だらけ」 
\\	この 埃[ほこり]だらけのテレビをちゃんと 拭[ふ]いてくれない?		
\\	体によくないと思
\\	つつ、最近は全然運動してない。	体によくないと思いつつ、最近は全然運動してない。	からだによくないとおもいつつ、さいきんはぜんぜんうんどうしてない。	
\\	「~つつ」	体[からだ]によくないと 思[おも]いつつ、 最近[さいきん]は 全然[ぜんぜん] 運動[うんどう]してない。		
\\	今後もお客様との対話の窓口として、より充実して行
\\	べく 努力してまいります。	今後もお客様との対話の窓口として、より充実していくべく 努力してまいります。	こんごもおきゃくさまとのたいわのまどぐちとして、よりじゅうじつしていくべくどりょくしてまいります。	
\\	「べく」 
\\	今後[こんご]もお 客様[きゃくさま]との 対話[たいわ]の 窓口[まどぐち]として、より 充実[じゅうじつ]していくべく 努力[どりょく]してまいります。		
\\	二日酔いで痛む頭を押
\\	つつ、トイレに入った。	二日酔いで痛む頭を押さえつつ、トイレに入った。	ふつかよいでいたむあたまをおさえつつ、トイレにいっった。	
\\	「~つつ」	二日酔[ふつかよ]いで 痛[いた]む 頭[あたま]を 押[お]さえつつ、トイレに 入[はい]った。		
\\	この場ではちょっと決
\\	かねますので、また別途会議を設けましょう。	この場ではちょっと決めかねますので、また別途会議を設けましょう。	このばではちょっときめかねますので、またべっとかいぎをもうけましょう。	
\\	「~かねる」	この 場[ば]ではちょっと 決[き]めかねますので、また 別途[べっと] 会議[かいぎ]を 設[もう]けましょう。		
\\	そう簡単に直せ
\\	はずがないよ。	そう簡単に直せるはずがないよ。	そうかんたんになおせるはずがないよ。	
\\	「はず」 
\\	そう 簡単[かんたん]に 直[なお]せるはずがないよ。		
\\	宿題が多すぎて、トイレに行く時間
\\	さえなかった。	宿題が多すぎて、トイレに行く時間さえなかった。	しゅくだいがおおすぎて、トイレにいくじかんさえなかった。	
\\	「(で)さえ」 
\\	宿題[しゅくだい]が 多[おお]すぎて、トイレに 行[い]く 時間[じかん]さえなかった。		
\\	子供が掃除
\\	そばから散らかすから、もう諦めたくなった。	子供が掃除するそばから散らかすから、もう諦めたくなった。	こどもがそうじするそばからちらかすから、もうあきらめたくなった。	
\\	「そばから」 
\\	子供[こども]が 掃除[そうじ]するそばから 散[ち]らかすから、もう 諦[あきら]めたくなった。		
\\	テレビを見
\\	ながら、寝ちゃダメよ!	テレビを見ながら、寝ちゃダメよ!	テレビをみながら、ねちゃダメよ!	
\\	「~つつ」	テレビを 見[み]ながら、 寝[ね]ちゃダメよ!		
\\	肉を食
\\	ようにしている。	肉を食べないようにしている。	にくをたべないようにしている。	
\\	肉[にく]を 食[た]べないようにしている。		
\\	女の子と共通の話題ができて、自分の体も健康になる。いいことずくめですよ。	女の子と共通の話題ができて、自分の体も健康になる。いいことずくめですよ。	おんなのこときょうつうのわだいができて、じぶんのからだもけんこうになる。いいことずくめですよ。	
\\	「ずくめ」 
\\	女の子[おんなのこ]と 共通[きょうつう]の 話題[わだい]ができて、 自分[じぶん]の 体[からだ]も 健康[けんこう]になる。いいことずくめですよ。		
\\	今の日本では、雇用慣行が崩
\\	つつある。	今の日本では、雇用慣行が崩れつつある。	いまのにっぽんでは、こようかんこうがくずれつつある。	
\\	「~つつ」	今[いま]の 日本[にっぽん]では、 雇用[こよう] 慣行[かんこう]が 崩[くず]れつつある。		
\\	時間が
\\	ある 
\\	うが
\\	ある]まいが、間に合わせるしかない。	時間があろうがあるまいが、間に合わせるしかない。	じかんがあろうがあるまいが、まにあわせるしかない。	
\\	時間[じかん]があろうがあるまいが、 間に合[まにあ]わせるしかない。		
\\	その時までは決して彼に会
\\	まいと心に決めていた。	その時までは決して彼に会うまいと心に決めていた。	そのときまではけっしてかれにあうまいとこころにきめていた。	
\\	その 時[とき]までは 決[けっ]して 彼[かれ]に 会[あ]うまいと 心[こころ]に 決[き]めていた。		
\\	純物質が混じりあっている物質
\\	。	純物質が混じりあっている物質である。	こんごうぶつとは、2しゅるいいじょうのじゅんぶっしつがまじりあっているぶっしつである。	
\\	「である」 
\\	純[じゅん] 物質[ぶっしつ]が 混[ま]じりあっている 物質[ぶっしつ]である。		
\\	ボールは爆発
\\	する]ばかりに、膨らんでいた。	ボールは爆発せんばかりに、膨らんでいた。	ボールはばくはつせんばかりに、ふくらんでいた。	
\\	「ばかり」 
\\	ボールは 爆発[ばくはつ]せんばかりに、 膨[ふく]らんでいた。		
\\	搭乗のアナウンスが聞
\\	や否や、みんながゲートの方へ走り出した。	搭乗のアナウンスが聞こえるや否や、みんながゲートの方へ走り出した。	とうじょうのアナウンスがきこえるやいなや、みんながゲートのほうへはしりだした。	
\\	「や/や否や」 
\\	搭乗[とうじょう]のアナウンスが 聞[き]こえるや 否[いな]や、みんながゲートの 方[ほう]へ 走り出[はしりだ]した。		
\\	白
\\	ずくめ団体は去年ニューズになっていた。	白ずくめ団体は去年ニューズになっていた。	しろずくめだんたいはきょねんニューズになっていた。	
\\	「ずくめ」 
\\	白[しろ]ずくめ 団体[だんたい]は 去年[きょねん]ニューズになっていた。		
\\	携帯を2年間使ってたら、傷
\\	だらけになった。	携帯を2年間使ってたら、傷だらけになった。	けいたいを2ねんかんつかってたら、きずだらけになった。	
\\	「だらけ」 
\\	携帯[けいたい]を2 年間[ねんかん] 使[つか]ってたら、 傷[きず]だらけになった。		
\\	「食べてみよう」と言
\\	が早いか、口の中に放り込んだ。	「食べてみよう」と言うが早いか、口の中に放り込んだ。	「たべてみよう」というがはやいか、くちのなかにほうりこんだ。	
\\	「が早いか」 
\\	食[た]べてみよう」と 言[い]うが 早[はや]いか、 口[くち]の 中[なか]に 放り込[ほうりこ]んだ。		
\\	彼女は、教授の姿を見
\\	が早いか、教室から逃げ出した。	彼女は、教授の姿を見るが早いか、教室から逃げ出した。	かのじょは、きょうじゅのすがたをみるがはやいか、きょうしつからにげだした。	
\\	「が早いか」 
\\	彼女[かのじょ]は、 教授[きょうじゅ]の 姿[すがた]を 見[み]るが 早[はや]いか、 教室[きょうしつ]から 逃げ出[にげだ]した。		
\\	私の顔を見
\\	や否や、何か言おうとした。	私の顔を見るや否や、何か言おうとした。	わたしのかおをみるやいなや、なにかいおうとした。	
\\	「や/や否や」 
\\	私[わたし]の 顔[かお]を 見[み]るや 否[いな]や、 何[なに]か 言[い]おうとした。		
\\	吾輩は猫
\\	。	吾輩は猫である。	わがはいはねこである。	
\\	「である」 
\\	吾輩[わがはい]は 猫[ねこ]である。		
\\	どんな商品でもネットで販売するだけで売上が伸びるというものではなかろう。	どんな商品でもネットで販売するだけで売上が伸びるというものではなかろう。	どんなしょうひんでもネットではんばいするだけでうりあがのびるというものではなかろう。	
\\	「かろう」 
\\	「い」 
\\	どんな 商品[しょうひん]でもネットで 販売[はんばい]するだけで 売上[うりあげ]が 伸[の]びるというものではなかろう。		
\\	たった1キロを走っただけで、汗
\\	まみれになるのは情けない。	たった1キロを走っただけで、汗まみれになるのは情けない。	たった1キロをはしっただけで、あせまみれになるのはなさけない。	
\\	「まみれ」 
\\	たった1キロを 走[はし]っただけで、 汗[あせ]まみれになるのは 情[なさ]けない。		
\\	うちの子供はプールに入るのを理由もなく怖
\\	がる。	うちの子供はプールに入るのを理由もなく怖がる。	うちのこどもはプールにいるのをりゆうもなくこわがる。	
\\	「~がる」	うちの 子供[こども]はプールに 入[はい]るのを 理由[りゆう]もなく 怖[こわ]がる。		
\\	早くきてよ!何を恥
\\	がっているの?	早くきてよ!何を恥ずかしがっているの?	はやくきてよ!なにをはずかしがっているの?	
\\	「~がる」	早[はや]くきてよ! 何[なに]を 恥[は]ずかしがっているの?		
\\	試験に合格
\\	べく、皆一生懸命に勉強している。	試験に合格すべく、皆一生懸命に勉強している。	しけんにごうかくすべく、みないっしょうけんめいにべんきょうしている。	
\\	「べく」 
\\	試験[しけん]に 合格[ごうかく]すべく、 皆[みな] 一生懸命[いっしょうけんめい]に 勉強[べんきょう]している。		
\\	お金
\\	さえあれば、何でも出来るよ。	お金さえあれば、何でも出来るよ。	おきんさえあれば、なんでもできるよ。	
\\	「(で)さえ」 
\\	お 金[かね]さえあれば、 何[なに]でも 出来[でき]るよ。		
\\	自分の過ちを認
\\	さえしなければ、問題は解決しないよ。	自分の過ちを認めさえしなければ、問題は解決しないよ。	じぶんのあやまちをみとめさえしなければ、もんだいはかいけつしないよ。	
\\	「(で)さえ」 
\\	自分[じぶん]の 過[あやま]ちを 認[みと]めさえしなければ、 問題[もんだい]は 解決[かいけつ]しないよ。		
\\	結婚
\\	おろか、2ヶ月付き合って、結局別れてしまった。	結婚はおろか、2ヶ月付き合って、結局別れてしまった。	けっこんはおろか、2がつつきあって、けっきょくわかれてしまった。	
\\	「おろか」 
\\	結婚[けっこん]はおろか、 
\\	ヶ月[かげつ] 付き合[つきあ]って、 結局[けっきょく] 別[わか]れてしまった。		
\\	家に帰ったら、すぐパソコンを使
\\	がる。	家に帰ったら、すぐパソコンを使いたがる。	いえにかえったら、すぐパソコンをつかいたがる。	
\\	「~がる」	家[いえ]に 帰[かえ]ったら、すぐパソコンを 使[つか]いたがる。		
\\	明日に行くのを止
\\	う。	明日に行くのを止めよう。	あしたにいくのをやめよう。	
\\	明日[あした]に 行[い]くのを 止[と]めよう。		
\\	散歩
\\	がてら、タバコを買いに行きました。	散歩がてら、タバコを買いに行きました。	さんぽがてら、タバコをかいにいきました。	
\\	「~がてら」 
\\	散歩[さんぽ]がてら、タバコを 買[か]いに 行[い]きました。		
\\	彼は漫画マニアだから、これらをもう全部読
\\	はずだよ。	彼は漫画マニアだから、これらをもう全部読んだはずだよ。	かれはまんがマニアだから、これらをもうぜんぶよんだはずだよ。	
\\	「はず」 
\\	彼[かれ]は 漫画[まんが]マニアだから、これらをもう 全部[ぜんぶ] 読[よ]んだはずだよ。		
\\	例え国のためであっても、国民を騙
\\	べきではないと思う。	例え国のためであっても、国民を騙すべきではないと思う。	たとえくにのためであっても、こくみんをだますべきではないとおもう。	
\\	「べき」 
\\	例[たと]え 国[こく]のためであっても、 国民[こくみん]を 騙[だま]すべきではないと 思[おも]う。		
\\	この旅館は、様々な新しい設備が備
\\	、とても快適だった。	この旅館は、様々な新しい設備が備えており、とても快適だった。	このりょかんは、さまざまなあたらしいせつびがそなえており、とてもかいてきだった。	
\\	この 旅館[りょかん]は、 様々[さまざま]な 新[あたら]しい 設備[せつび]が 備[そな]えており、とても 快適[かいてき]だった。		
\\	ビタミンを食
\\	さえすれば、健康が保証されますよ。	ビタミンを食べさえすれば、健康が保証されますよ。	ビタミンをたべさえすれば、けんこうがほしょうされますよ。	
\\	「(で)さえ」 
\\	ビタミンを 食[た]べさえすれば、 健康[けんこう]が 保証[ほしょう]されますよ。		
\\	事情を2時間かけて説明を
\\	する]あげく、納得してもらえなかった	事情を2時間かけて説明をしたあげく、納得してもらえなかった。	じじょうを2じかんかけてせつめいをしたあげく、なっとくしてもらえなかった。	
\\	「~あげく(挙句)」 
\\	事情[じじょう]を 
\\	時間[じかん]かけて 説明[せつめい]をしたあげく、 納得[なっとく]してもらえなかった。		
\\	昼間だから絶対込んでいると思
\\	きや、一人もいなかった。	昼間だから絶対込んでいると思いきや、一人もいなかった。	ひるまだからぜったいこんでいるとおもいきや、ひとりもいなかった。	
\\	「思いきや」 
\\	昼間[ひるま]だから 絶対[ぜったい] 込[こ]んでいると 思[おも]いきや、一 人[にん]もいなかった。		
\\	このドキュメントは間違え
\\	だらけで、全然役に立たない。	このドキュメントは間違えだらけで、全然役に立たない。	このドキュメントはまちがえだらけで、ぜんぜんやくにたたない。	
\\	「だらけ」 
\\	このドキュメントは 間違[まちが]えだらけで、 全然[ぜんぜん] 役に立[やくにた]たない。		
\\	博物館を見
\\	がてらに、お土産を買うつもりです。	博物館を見がてらに、お土産を買うつもりです。	はくぶつかんをけんがてらに、おみやげをかうつもりです。	
\\	「~がてら」 
\\	博物館[はくぶつかん]を 見[み]がてらに、お 土産[みやげ]を 買[か]うつもりです。		
\\	新しい仕事が見つからないのでやむ
\\	を得ない状態です。	新しい仕事が見つからないのでやむを得ない状態です。	あたらしいしごとがみつからないのでやむをえないじょうたいです。	
\\	「やむを得ない」	新[あたら]しい 仕事[しごと]が 見[み]つからないのでやむを 得[え]ない 状態[じょうたい]です。		
\\	電気製品の発展につれて、ハードディスクの容量はますます大きく
\\	なる]つつある。	電気製品の発展につれて、ハードディスクの容量はますます大きくなりつつある。	でんきせいひんのはってんにつれて、ハードディスクのようりょうはますますおおきくなりつつある。	
\\	「~つつ」	電気[でんき] 製品[せいひん]の 発展[はってん]につれて、ハードディスクの 容量[ようりょう]はますます 大[おお]きくなりつつある。		
\\	私は寒
\\	がり屋だから、ミネソタで暮らすのは辛かった。	私は寒がり屋だから、ミネソタで暮らすのは辛かった。	わたしはさむがりやだから、ミネソタでくらすのはつらかった。	
\\	「~がる」	私[わたし]は 寒[さむ]がり 屋[や]だから、ミネソタで 暮[く]らすのは 辛[つら]かった。		
\\	確定申告は忘
\\	がちな手続のひとつだ。	確定申告は忘れがちな手続のひとつだ。	かくていしんこくはわすれがちなてつづきのひとつだ。	
\\	「~がち」	確定[かくてい] 申告[しんこく]は 忘[わす]れがちな 手続[てつづき]のひとつだ。		
\end{CJK}
\end{document}