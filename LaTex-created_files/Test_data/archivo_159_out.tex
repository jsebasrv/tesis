\documentclass[8pt]{extreport} 
\usepackage{hyperref}
\usepackage{CJKutf8}
\begin{document}
\begin{CJK}{UTF8}{min}
\\	大唾液腺からの分泌物の性状は?	①耳下腺:純漿液腺
\\	②顎下腺:混合腺(漿
\\	粘)
\\	③舌下腺:混合腺(粘
\\	漿)	内臓系
\\	口唇は外面が皮膚、内面が口腔粘膜で覆われている。その移行部は何と呼ばれるか?	赤唇縁(唇紅)	内臓系
\\	口腔は上下歯列弓を境に前方を
\\	、後方を
\\	と呼ぶ。	①口腔前庭
\\	②固有口腔	内臓系
\\	口腔前庭は
\\	腺からの分泌を受け、固有口腔は
\\	腺と
\\	腺の分泌を受ける。	①耳下
\\	②顎下
\\	③舌下	内臓系
\\	口蓋舌弓と口蓋咽頭弓の舌根付着部の間はくぼんでおり
\\	という。ここに
\\	が存在する。この部位は口腔と咽頭との境にあり
\\	という。	①扁桃窩
\\	②口蓋扁桃
\\	③口峡
\\	内臓系
\\	大唾液腺を大きい順に並べると?	耳下腺
\\	顎下腺
\\	舌下腺	内臓系
\\	乳歯は総数
\\	本で、上下顎の左右両側にそれぞれ、乳切歯
\\	本、乳犬歯
\\	本、乳臼歯
\\	本ある。	
\\	内臓系
\\	永久歯は総数
\\	本で、上下顎の左右両側にそれぞれ、切歯
\\	本、犬歯
\\	本、小臼歯
\\	本、大臼歯
\\	本ある。	
\\	内臓系
\\	歯髄腔の外は歯の大部分を占める
\\	が取り囲む。さらに
\\	の外すなわち歯の表面は、歯冠部では
\\	で、歯根部では
\\	でおおわれている。	①象牙質
\\	②エナメル質
\\	③セメント質
\\	内臓系
\\	舌表面特有のザラザラした感じをつくり出している小突起を
\\	といい、
\\	の4種類に区分する。	①舌乳頭
\\	②糸状乳頭(味蕾×)
\\	③茸状乳頭(味蕾まれに)
\\	④葉状乳頭(味蕾
\\	⑤有郭乳頭(味蕾◯)
\\	内臓系
\\	味覚は、舌前2/3は
\\	神経の枝の鼓索神経が、舌後1/3は
\\	神経が支配している。	①顔面
\\	②舌咽
\\	内臓系
\\	知覚は、舌前2/3が
\\	神経
\\	神経の第3枝である下顎神経の枝)で、舌後1/3は
\\	神経によって支配されている。	①舌
\\	②三叉
\\	③舌咽
\\	内臓系
\\	口腔と咽頭が連動して行う重要な機能に嚥下があるが、これは
\\	の3相に分けられる。	①口腔相
\\	②咽頭相
\\	③食道相
\\	内臓系
\\	咽頭の内腔を
\\	といい、上方から
\\	の三部に区分される。	①咽頭腔
\\	②咽頭鼻部
\\	③咽頭口部
\\	④咽頭喉頭部
\\	内臓系
\\	下鼻道の後方の両側壁には、中耳に通じる何が開いているか?	耳管咽頭口
\\	内臓系
\\	咽頭扁桃、耳管扁桃、口蓋扁桃および舌扁桃は、全体として咽頭を取り囲むように配列しているため、これらを合わせて何と呼ぶか?	リンパ咽頭輪(ワルダイエルの咽頭輪
\\	内臓系
\\	食道の長さは約
\\	の筋性の管で、前方では
\\	の高さ、後方では
\\	の高さで咽頭に続く。脊柱の前を下行し、ほぼ第10胸椎の高さで横隔膜の
\\	を通過して腹腔に至り、第10~12胸椎の左前方で胃の噴門につらなる。	
\\	②輪状軟骨下縁
\\	③第6頚椎
\\	④食道裂孔
\\	='新規ドキュメント 
\\	内臓系
\\	食道にある3箇所の生理的狭窄部位は?	①食道の起始部
\\	②大動脈弓と分岐直後の左主気管支と交叉する気管分岐部
\\	③横隔膜の貫通部
\\	内臓系
\\	食道壁は粘膜、筋層、線維膜(外膜)からなる。
\\	粘膜の表面は
\\	上皮に覆われている。
\\	食道の筋は部位によってその構成が異なり、上部1/3は
\\	、中央1/3は
\\	、そして下部1/3は
\\	からなる。	①重層扁平
\\	②横紋筋
\\	③横紋筋と平滑筋
\\	④平滑筋	内臓系
\\	胃の小弯の幽門側にあるくびれを
\\	という。
\\	周囲の小弯側は潰瘍や癌の好発する部位である。	①角切痕
\\	内臓系
\\	胃底の上端は胃の最高位にあたり、
\\	の高さに達する。
\\	噴門は正中線よりやや左側で、だいたい
\\	の高さにある。
\\	幽門は
\\	の右前方に位置する。	①左第5肋骨
\\	②第11胸椎
\\	③第1腰椎
\\	内臓系
\\	胃の粘膜上皮層は
\\	からなり、これが胃小窩の底で粘膜内に落ち込んで多数の
\\	を形成する。	①単層円柱上皮
\\	②胃腺
\\	内臓系
\\	固有胃腺は胃にもっとも特有の腺で、その上皮にある主細胞は
\\	、壁細胞は
\\	、副細胞は
\\	を分泌している。	①ペプシノゲン
\\	②塩酸
\\	③粘液
\\	内臓系
\\	胃の筋層で中層の輪走筋は3層中もっともよく発達しており、とくに幽門で著しく発達し
\\	をつくる。	①幽門括約筋	内臓系
\\	小腸は、胃の幽門に続き盲腸に至る長さ約
\\	ほどの細長い管である。
\\	腸間膜の有無によって、それをもたない
\\	と腸間膜をもつ
\\	と
\\	に分けられる。	
\\	②十二指腸
\\	③空腸
\\	④回腸
\\	内臓系
\\	十二指腸は胃の幽門に続く腸管で
\\	字状を呈し、ここで
\\	部を取り囲んでいる。
\\	十二指腸の後腹壁は癒着しているため、
\\	を欠いている。
\\	十二指腸は、その走行によって上部、下行部、水平部、上行部の四つに区分されている。
\\	上部のはじめの部分は完全に腹膜に覆われ、可動性に富み、
\\	と呼ばれ
\\	の好発部位である。内腔は十二指腸の中で一番広く、他の部位と異なり粘膜に
\\	を欠いている。
\\	下行部の後内側壁には縦走する
\\	があり、その下端には
\\	という隆起が認められている。ここに総胆管と膵管が合一して、あるいは別々に開口している。この開口部の周囲を
\\	が取り囲んでいる。
\\	の上方2~3
\\	のところに、しばしば小十二指腸乳頭という小突起を認めることがあり、ここに
\\	が開口する。
\\	水平部は第4腰椎位で始まり、やや上方に傾き、第3腰椎の高さで
\\	の前を左方に走る。水平部の前面を
\\	が横切って下行する。
\\	上行部は第2腰椎の高さまで上行し、その左側で急に腹側に屈曲し空腸に移行する。その屈曲部は
\\	とよばれ、ここに横隔膜の右脚周囲の結合組織から始まる平滑筋を含む線維束の
\\	または
\\	が付く。これは
\\	を固定支持している。	①膵頭
\\	②腸間膜
\\	③十二指腸球部
\\	④十二指腸潰瘍
\\	⑤輪状ヒダ
\\	⑥十二指腸縦ヒダ
\\	⑦大十二指腸乳頭
\\	⑧オッディ括約筋
\\	⑨副膵管
\\	⑩下大静脈
\\	⑪腹大動脈
\\	⑫上腸間膜動脈
\\	⑬上腸間膜静脈
\\	⑭十二指腸空腸曲
\\	⑮十二指腸提筋
\\	⑯トライツ靭帯
\\	内臓系
\\	小腸壁は粘膜、筋層、漿膜の3層からなる。空腸、回腸の粘膜は、吸収率を高めるために表面積を大きくする構造をとっている。腸管内腔に向かって隆起する
\\	と粘膜表面にある無数の
\\	がその役目を果たしている。
\\	は小腸上部、すなわち
\\	より
\\	において発達しており、その密度も
\\	で大きい。	①輪状ヒダ
\\	②腸じゅう毛
\\	③回腸
\\	④空腸
\\	内臓系
\\	回腸ではリンパ小節が多数集まってできた
\\	がみられ、20~30個存在する。	①パイエル板
\\	内臓系
\\	上皮である小腸粘膜上皮が、粘膜全域に
\\	をつくっている。小腸粘膜上皮の間には粘液物質であるムチンを分泌する
\\	があり、
\\	の底部には抗菌物質を分泌する
\\	がある。	①単層円柱
\\	②腸腺(リーベルキューン腺)
\\	③杯細胞
\\	④パネート細胞
\\	内臓系
\\	盲腸は回腸と大腸の連結部より下方の大腸はゆきづまりの盲管となっており、長さは約
\\	である。右腸骨窩にあるこの連結部位を
\\	といい、回腸の大腸への開口部である回盲口には回腸末端が大腸に突出し弁状となった
\\	がある。
\\	は上・下唇を区別する。機能的には、
\\	は大腸内容物が小腸へ逆流するのを防いでいる。	
\\	②回盲部
\\	③回盲弁(バウヒン弁)
\\	内臓系
\\	盲腸下端の後内側壁からは
\\	とよばれる通常6~8
\\	の細長い管が伸びている。
\\	虫垂炎の圧痛点として、
\\	点や
\\	点などが知られている。
\\	点は、
\\	と
\\	を結ぶ線上で
\\	の点をさす。この点は体表において
\\	の基部を反映している。また
\\	点は左右の
\\	を結んだ線を
\\	の点である。これは、虫垂の先端を投影する。	①虫垂
\\	②マック・バーニー
\\	③ランツ
\\	④臍
\\	⑤右上前腸骨棘
\\	⑥外側2/3
\\	⑦上前腸骨棘
\\	⑧3等分した右側
\\	内臓系
\\	結腸は上行結腸、横行結腸、下行結腸、
\\	状結腸の4部を区別する。上行結腸は
\\	に続き、右側腹部を上行し、肝臓右葉に達する。ここで左方に屈曲し、横行結腸となり腹腔上部を左上方に向かって横走する。この屈曲部を
\\	とよぶ。脾臓の下端の内側に達した横行結腸は、下方に強く屈曲し
\\	をつくる。
\\	から肛側は下行結腸といい、腹腔の左後壁を下行し、左腸骨窩に至り
\\	状結腸に移行する。	①盲腸
\\	②右結腸曲
\\	③左結腸曲
\\	内臓系
\\	結腸には小腸には認められない形態的特徴がある。結腸平滑筋層の外層をなす縦走筋が3箇所で集まり、3本の
\\	をつくる。それぞれ
\\	とよばれる。この腸管長軸に沿う
\\	によって短縮するため、外方に向かっては嚢状のふくらみが、内腔にはヒダが形成される。これらをそれぞれ
\\	という。さらに、結腸では漿膜で包まれた脂肪組織塊が結腸壁に観察される。これを
\\	といい、
\\	と
\\	に沿って付着し、
\\	結腸でよく認められる。	①結腸ヒモ
\\	②大網ヒモ
\\	③間膜ヒモ
\\	④自由ヒモ
\\	⑤結腸膨起
\\	⑥結腸半月ヒダ
\\	⑦腹膜垂
\\	⑧横行
\\	内臓系
\\	直腸は、
\\	状結腸に続き、骨盤隔膜をつらぬいて肛門管となり、さらに肛門として外部に開いている。長さは約
\\	である。仙骨の前面をその弯曲に沿って下行するため、この部の直腸は前方に向かってくぼんでおり、これを仙骨曲という。さらに下方で、尾骨下端を回ってから後方に屈曲しており、これは会陰曲とよばれる。肛門管のすぐ上の部分は拡張しており、
\\	とよばれる。
\\	の上方の直腸内腔面には3条のヒダがあり、これを
\\	という。肛門では輪走の平滑筋よりなる
\\	と、その外周には横紋筋性の
\\	がある。	
\\	②直腸膨大部
\\	③直腸横ヒダ
\\	④内肛門括約筋
\\	⑤外肛門括約筋
\\	内臓系
\\	静脈系については動脈と同名の静脈が大腸に分布している。肛門管の粘膜下には直腸静脈叢が発達しており、これがうっ血拡張すると静脈瘤をつくるときがある。静脈瘤による粘膜下の結節状隆起を
\\	という。	①痔核
\\	内臓系
\\	肝臓は人体中の最大の実質性器官、そして腺であり、主として腹腔の右上部に位置する。すなわち、右下肋部から上胃部、そして小部分が左下肋部にある。重さは約
\\	で、暗赤褐色を呈している。形は底面を右側に、頂を左側に向けた楔形である。上面と下面を区別し、上面は横隔膜の円蓋に応じ凸面をなし
\\	といい、下面は不平坦で諸臓器と接し
\\	といわれる。	
\\	②横隔面
\\	③臓側面
\\	内臓系
\\	肝臓の横隔面は後部を除いて大部分が腹膜でおおわれる。左右からの腹膜が、正中線のやや右側で合わさり、前後に走る腹膜ヒダをつくっている。これを
\\	といい、肝臓を
\\	と
\\	に分ける境となっている。右葉は大きく肝臓の約4/5を占め、左葉は約1/5をなす。横隔面の後方部は腹膜におおわれず、横隔膜にじかに接着しているため、
\\	とよばれる。臓側面には隣接する内臓による陥凹や圧痕がみられる。
\\	の臓側面には
\\	圧痕、
\\	圧痕が、
\\	には
\\	圧痕、
\\	圧痕、
\\	圧痕などが認められる。また、臓側面には
\\	状を呈する1本の横溝と2本の縦溝がある。横溝は肝門にあたり、肝臓に出入りする血管、肝管、神経が通る。	①肝鎌状間膜
\\	②右葉
\\	③左葉
\\	④無漿膜野
\\	⑤胃
\\	⑥食道
\\	⑦結腸
\\	⑧腎
\\	⑨十二指腸
\\	内臓系
\\	肝臓の左側の縦溝の前部は、
\\	裂といわれ
\\	を入れ、後部は
\\	裂で
\\	が存在する。右側の縦溝の前部には胆嚢を入れる
\\	を認め、後部には下大静脈を入れる
\\	がある。
\\	状の両縦溝にはさまれる部分は右葉に属し、肝門によって前後二部に分けられる。前部は長方形で
\\	といい、後部は不規則な形で
\\	という。	①肝円索
\\	②静脈管索
\\	③胆嚢窩
\\	④大静脈窩
\\	⑤方形葉
\\	⑥尾状葉
\\	内臓系
\\	肝実質を構成するものとして、
\\	の3要素があり、これらが肝内で分枝し独特な構造をつくっている。肝臓はその構造的単位である
\\	(直径1~2
\\	の集合体である。六角柱をなす
\\	の中央には
\\	が縦走し、その周囲には放射状に配列する
\\	がある。肝細胞で分泌された胆汁の通路は、肝細胞ではさまれてできた
\\	に始まる。次に
\\	を去って
\\	を通り、左右の
\\	を経て肝臓を去り、肝門で合わさった1本の
\\	へと流れる。	①門脈
\\	②肝動脈
\\	③胆管
\\	④肝小葉
\\	⑤中心静脈
\\	⑥肝細胞板(肝細胞索)
\\	⑦毛細胆管
\\	⑧小葉間胆管
\\	⑨肝管
\\	⑩総肝管
\\	内臓系
\\	肝臓内に分布する血管には、栄養血管である
\\	と機能血管の
\\	がある。両血管はそれぞれ小葉間動・静脈となり、小葉間結合組織中を走り、直接に
\\	と連絡する。
\\	は近くの中心静脈にそそぎ込む。中心静脈は次第に集まって太くなり、結局、
\\	となり肝臓後面から出て
\\	にそそぐ。	①肝動脈
\\	②門脈
\\	③洞様毛細血管(類洞)
\\	④肝静脈
\\	⑤下大静脈
\\	内臓系
\\	肝内血流量のうち4/5は
\\	、1/5は
\\	によるとされる。小葉内毛細血管壁には、大型で食作用を有し、生体防御にあずかる
\\	がある。類洞と肝細胞板との間には
\\	とよばれる腔所があり、ここに類洞周囲脂肪細胞(井東細胞)がみられる。小葉間結合組織、すなわち
\\	を小葉間動・静脈、小葉間胆管が走行しており、ここに血流分布の中心としてとらえることができる。このため、この部位を中心静脈を中心とした構造的単位に対して機能的単位とみなすことができる。	①門脈
\\	②肝動脈
\\	③クッパーの星細胞
\\	④類洞周囲隙(ディッセ腔)
\\	⑤血管周囲線維鞘(グリソン鞘)
\\	内臓系
\\	胆嚢は胆汁を蓄え濃縮するところで、肝臓下面の
\\	におさまっているナス状の嚢状器官である。長さは8~10
\\	、幅2~4
\\	、容積は30~70
\\	である。胆嚢上面は肝臓下面と接着し、下面は肝臓とともに腹膜でおおわれている。胆嚢は底、体、頚の三部に区別される。底は先端部で盲管となっており、肝臓下縁より少し前下方に突出し、前腹壁に接する。体は中央の大部分を占める。頚部とそれに続く胆嚢管の内面には著明な粘膜ヒダを認め、これを
\\	という。	①胆嚢窩
\\	②ラセンヒダ
\\	内臓系
\\	肝臓下面、総肝管、胆嚢管の三者で囲まれる三角を
\\	といい、
\\	はこの三角内を走ることが多い。	①胆嚢三角(カロー三角)
\\	②胆嚢動脈
\\	内臓系
\\	肝臓で産生された胆汁は以下に述べる道筋、すなわち胆路を経て
\\	へとそそぐ。肝門から出た左右の
\\	は、1本に合して
\\	となる。
\\	は胆嚢からの
\\	と合し、
\\	となる。
\\	は十二指腸下行部の後内側壁に達し、膵臓からの
\\	と合して胆膵管膨大部をつくったのち
\\	にひらく。
\\	への開口部には、
\\	がある。	①十二指腸
\\	②肝管
\\	③総肝管
\\	④胆嚢管
\\	⑤総胆管
\\	⑥膵管
\\	⑦大十二指腸乳頭(ファーター乳頭)
\\	⑧オッディ括約筋
\\	内臓系
\\	膵臓は第1~2腰椎の高さでその前面を横切り、後腹壁に密着している
\\	である。扁平で舌状の細長い臓器で、長さ13~16
\\	、重さは約70
\\	である。膵臓は
\\	の三部に区分される。
\\	は右端の膨大した部分で、
\\	状に弯曲した十二指腸に囲まれている。
\\	は
\\	から左方に向かって脊柱の前を横走する部分で、
\\	は左端の細い部分で、その先端は
\\	に接する。	①後腹膜器官
\\	②膵頭
\\	③膵体
\\	④膵尾
\\	⑤脾臓
\\	内臓系
\\	膵臓は、その大部分は膵液を分泌する外分泌部から構成され、内分泌部はその中に散在する細胞集団として存在する。この内分泌性細胞群を
\\	といい、直径約200
\\	の球形を呈し、総数は約100万である。
\\	は
\\	に比較的多い。
\\	は特殊染色法によって
\\	細胞に区別できる。
\\	細胞は
\\	を産生し、
\\	の中で多数を占める
\\	細胞は
\\	を産生し、
\\	細胞は
\\	を産生している。	①膵島(ランゲルハンス島)
\\	②膵尾
\\	③グルカゴン
\\	④インスリン
\\	⑤ソマトスタチン
\\	内臓系
\\	外鼻は
\\	と
\\	からでき、上方から
\\	に分けられる。
\\	の両側の隆起部は
\\	とよばれ、
\\	は、
\\	、および鼻中隔の前方部にある。	①鼻骨
\\	②鼻軟骨
\\	③鼻根
\\	④鼻背
\\	⑤鼻尖
\\	⑥鼻翼
\\	内臓系
\\	鼻腔は鼻翼部に相当する
\\	と、その奥に続く固有の鼻腔とに分けられる。後方は
\\	により咽頭と交通する。
\\	は外鼻孔から約1
\\	の長さをもち、内壁は皮膚の続きで鼻毛、汗腺、脂腺がみられる。固有の鼻腔は
\\	という正中の仕切りで左右に分かれる。鼻腔の上壁は篩骨の篩板で、下壁は上顎骨と口蓋骨で構成される。外側壁は上・中・下鼻甲介が張り出し、各鼻甲介の下に
\\	をつくる。また、鼻甲介と鼻中隔の間の上下方向にのびた共通の空所を
\\	とよぶ。	①鼻前庭
\\	②後鼻孔
\\	③鼻中隔
\\	④上・中・下鼻道
\\	⑤総鼻道
\\	内臓系
\\	上鼻道の上壁には、匂いをかぐ(嗅覚)特殊な粘膜上皮に分化した
\\	という嗅覚受容細胞がある。
\\	の神経突起は約20本集まって
\\	となり、
\\	の小孔をつらぬいて頭蓋内に入り脳(嗅球)に接続する。	①嗅細胞
\\	②嗅神経(嗅糸)
\\	③篩骨篩板
\\	内臓系
\\	鼻腔を取り囲む骨
\\	は含気骨に属し、内部に空気を入れる空洞をもつ。この含気洞は
\\	とよばれ、さらに鼻腔と交通して鼻粘膜の続きが
\\	をおおっている。そのため鼻粘膜に炎症が生じると、容易に
\\	の粘膜に炎症が波及し
\\	炎を起こす。また
\\	のうち出口が狭かったり出口より低位にあるもの
\\	は、
\\	炎のとき膿がたまりしばしば蓄膿症をきたす。	①上顎骨
\\	②篩骨
\\	③前頭骨
\\	④蝶形骨
\\	⑤副鼻腔
\\	⑥上顎洞
\\	⑦蝶形骨洞
\\	内臓系
\\	喉頭の前面は
\\	におおわれ、男性の喉仏(アダムのリンゴ)として突出する長さ約5
\\	の管腔器官で、気道の一部をなすとともに発生器としての機能も営む。喉頭は軟骨性の支柱を有し(喉頭軟骨)、各軟骨は靭帯と多くの小筋(喉頭筋)で結合される。喉頭軟骨は不対性の
\\	および対性の
\\	が主要なものである。	①甲状軟骨
\\	②輪状軟骨
\\	③喉頭蓋軟骨
\\	④披裂軟骨
\\	内臓系
\\	喉頭の甲状軟骨は舌骨の下方にあり、喉頭軟骨のうち最大のもので、左・右板が喉頭の前・側壁を形成する。両板が正中で合する縁は、思春期以後の男性で特に前方に突隆した
\\	をつくる。正中部の内面には1対の
\\	の前端が付着する。	①喉頭隆起
\\	②声帯
\\	内臓系
\\	喉頭の輪状軟骨は
\\	の下方にある輪状の軟骨で、前部、外側部は細く、後部は広い四角板状を呈する。外側面は
\\	と、上縁は
\\	と関節をつくる。	①甲状軟骨
\\	②披裂軟骨
\\	内臓系
\\	喉頭の披裂軟骨は
\\	の小さな三角錘状の軟骨で、その前端からは甲状軟骨に向かって
\\	が張っている。	①左右1対
\\	②声帯ヒダ(声帯靭帯と声帯筋)
\\	内臓系
\\	喉頭の喉頭蓋軟骨はスプーン状の形状を呈し、舌根の後上方に突出し、喉頭口の前壁を形成する
\\	の基礎をなす。この軟骨の弾力性のある弾性軟骨で形成される。
\\	は物を飲み込むとき(嚥下)、反射的に喉頭が挙上し舌根部が
\\	を押し倒し喉頭口を閉じることで、食塊の気道内流入を防ぐ。	①喉頭蓋
\\	内臓系
\\	喉頭の喉頭腔は側壁を走る上下のヒダ、すなわち
\\	によって
\\	に区分される。左右の
\\	の間を
\\	といい、
\\	と合わせて
\\	という。	①前庭ヒダ
\\	②声帯ヒダ
\\	③喉頭前庭
\\	④喉頭室
\\	⑤声門下腔
\\	⑥声門裂
\\	⑦声門
\\	内臓系
\\	気管は喉頭(声門下腔)の続きで、
\\	位または
\\	の高さから、胸腔内の
\\	位で左右の気管支に分岐するまでの長さ約10~12
\\	、径約2
\\	の管状器官である。
\\	気管壁は、硝子軟骨で形成される約15~20個の馬蹄形の
\\	が靭帯で連結されてできる。後壁は
\\	を欠き、膜性壁とよばれ
\\	を含む。気管内面は、
\\	上皮で
\\	を多く含む。	①第6頚椎
\\	②輪状軟骨下縁
\\	③第4~5胸椎
\\	④気管軟骨
\\	⑤軟骨
\\	⑥平滑筋(気管筋)
\\	⑦多列線毛円柱
\\	⑧気管腺
\\	内臓系
\\	気管支は気管分岐部から肺門までの部分で、成人で右主気管支は長さ約2.5
\\	、径1.2~1.5
\\	、左主気管支は長さ約5
\\	、径1.0~1.3
\\	である。
\\	主気管支は
\\	よりも太く、しかも垂直に近い走行をとるので、吸入された異物や微生物は
\\	主気管支に入りやすい。
\\	肺門から肺に達した気管支は樹枝状に分岐を繰り返し、徐々に細くなり管壁の
\\	や
\\	を失い、最終的に末端は
\\	という顕微鏡的に小さい袋をなす。気管支の粘膜は
\\	上皮でおおわれ、
\\	に混じって
\\	や発達した
\\	が存在する。	①右
\\	②左
\\	③軟骨
\\	④腺
\\	⑤肺胞
\\	⑥線毛
\\	⑦杯細胞
\\	⑧気管支腺
\\	内臓系
\\	肺は胸腔内にある1対の半円錘状の器官で、心臓が
\\	側に偏在するため
\\	肺は
\\	肺よりも小さく、幅が狭い
\\	肺500
\\	肺600
\\	。肺の表面は
\\	におおわれ、平滑である。肺の上端はやや尖っていて
\\	とよばれ、鎖骨の上方およそ
\\	ほどに及ぶ。肺の下面
\\	は
\\	に接しており、そのドーム状の形に対応して全体にくぼんでいる。肺の外側面は肋骨と接する肋骨面で、内側面(縦隔面)は左・右肺の間で
\\	と接しており、とくにその部位は
\\	とよばれるくぼみを生じ、
\\	肺に著明である。内側面の中央には
\\	にかぶらない
\\	があり、気管支、肺動静脈、気管支動静脈、リンパ管、神経などが肺実質に出入りする。	①左
\\	②右
\\	③肺胸膜
\\	④肺尖
\\	⑥肺底
\\	⑦横隔膜
\\	⑧心臓
\\	⑨心圧痕
\\	⑩左
\\	⑪胸膜
\\	⑫肺門
\\	='ダウンロード.
\\	内臓系
\\	肺は、表面を裂く深い切れ込みによって各肺葉
\\	裂、
\\	裂)、
\\	肺が上葉、中葉、下葉に、
\\	肺は上葉、下葉に分けられる。これらの肺葉の分割は、肺門に入る気管支においても同様で
\\	肺で3本、
\\	肺で2本の
\\	に分岐する。さらにこれら
\\	は、
\\	肺で10本、
\\	肺で9本の
\\	に枝分かれする。肺実質はそれぞれの
\\	に相当する肺区域に区分され、それは機能上の単位をなす。なお肺動脈の分枝はそれらの
\\	と伴行することから、肺区域が肺切除の単位となり臨床的に重要視される。
\\	は分岐を続け、それらの分枝は、次々に細気管支、終末細気管支、呼吸細気管支、肺胞管と名付けられる。肺胞管からは多数の
\\	が外側にでている。	①斜
\\	②水平
\\	③右
\\	④左
\\	⑤葉気管支
\\	⑥区域気管支
\\	⑦肺胞
\\	内臓系
\\	肺胞は直径
\\	位の小さな袋であり、成人の両肺には
\\	個の肺胞があり、全ての肺胞の表面積を合わせると
\\	に達する。隣接する肺胞が共有する壁は
\\	とよばれる。
\\	は、肺胞上皮細胞でおおわれ、その中に豊富な
\\	と
\\	がある。肺胞上皮細胞には、多数の薄い
\\	と少数の
\\	がある。後者は、
\\	を分泌して、肺胞が虚脱するのを防いでいる。呼吸に伴うガス交換は、肺胞上皮細胞、基底膜、毛細血管の内皮細胞を通して行われる。これを
\\	とよび、その厚さは約0.5
\\	である。	
\\	②3~5億
\\	④肺胞中隔
\\	⑤毛細血管
\\	⑥弾性線維
\\	⑦扁平肺胞上皮細胞(Ⅰ型肺胞上皮細胞)
\\	⑧大肺胞上皮細胞(Ⅱ型肺胞上皮細胞)
\\	⑨界面活性物質
\\	⑩血液空気関門
\\	内臓系
\\	肺は2種類の血管系が分布する。機能血管としては、心臓から直接出入りする
\\	が肺胞壁の毛細血管網を形成する。栄養血管としては
\\	が胸大動脈から起こり、肺組織な分布する。	①肺動・静脈
\\	②気管支動脈
\\	内臓系
\\	肺の表面と胸壁の内面をおおう漿膜を
\\	という。肺実質をおおう
\\	は肺門で折り返って
\\	に移行する。
\\	は肺表面の区分と同様に、
\\	の三つの部位からなる。この2枚の胸膜の間には狭い胸膜腔があって外気圧より
\\	となっており、胸郭や横隔膜の運動時に胸膜内腔圧は変動する。胸膜腔は少量の漿液(胸膜液)を含み、呼吸運動の際、摩擦を防ぐ。とくに肺の前縁と下縁に沿って
\\	が反転する部位は
\\	とよばれ、深呼吸時にも間隙を残す。	①肺胸膜
\\	②壁側胸膜
\\	③肋骨胸膜
\\	④横隔胸膜
\\	⑤縦隔胸膜
\\	⑥陰圧
\\	⑦胸膜洞(肋骨縦隔洞、肋骨横隔洞)
\\	内臓系
\\	縦隔とは、左右の胸膜腔の間にはさまれた胸腔の中央部をいう。これは両側は
\\	、前方は
\\	、後方は
\\	、下方は
\\	で囲まれ、上方の境はなく胸郭上口で開放している。縦隔は心膜を基準に上部、前部、中部、後部の四つに分けられる。縦隔の中部には
\\	が、上部の前方には心臓に出入りする大血管、
\\	、上部の後方から後部にかけて
\\	などがある。	①縦隔胸膜
\\	②胸骨
\\	③脊柱
\\	④横隔膜
\\	⑤心臓
\\	⑥胸腺
\\	⑦気管
\\	⑧気管支
\\	⑨横隔神経
\\	⑩食道
\\	⑪迷走神経
\\	⑫胸大動脈
\\	⑬奇静脈系
\\	⑭胸管
\\	⑮交感神経幹
\\	内臓系
\\	腎臓は腹膜後隙にある
\\	で、脊柱の両側に位置する。色は暗褐色で対をなし、形は内側がくぼんだそら豆状で、重量は
\\	、大きさは長さ約10
\\	、幅約5
\\	、厚さ約3
\\	である。脊柱に対する位置関係は、だいたい
\\	から
\\	の高さにあるが、多くの場合
\\	腎の方が半分ないし1椎体だけ低位にある。これは肝臓の
\\	葉が
\\	葉より著しく発達するため、
\\	葉が
\\	腎を上方から圧排することに起因する。	①腹膜後器官
\\	③第11胸椎
\\	④第3腰椎
\\	⑤右
\\	⑥左
\\	内臓系
\\	腎臓の表面は線維性結合組織である
\\	でおおわれる。
\\	の外側は、腎臓の上端に接する副腎とともに
\\	とよばれる脂肪組織で囲まれる。さらに腎臓と副腎は脂肪組織内にある線維性の膜、すなわち
\\	に包まれる。このように腎臓は後腹壁にゆるく固定されているにすぎない。腎臓は
\\	そして血管によって位置が保持されている。極端にやせた場合など脂肪組織が減少し、
\\	がゆるんで腎臓の支持固定が悪くなり下垂することがある。これを
\\	という。	①線維皮膜
\\	②脂肪皮膜
\\	③腎筋膜(ゲロータ筋膜)
\\	④遊走腎
\\	内臓系
\\	腎臓は上端、下端、前面、後面、内側縁、外側縁を区別できる。腎臓の長軸は身体の正中面と平行せず、長軸の延長線は
\\	で交叉する。すなわち腎臓の下端は少し外側に開いている。内側縁の中央部は陥凹があるが、ここを
\\	といい、腎動脈、腎静脈、尿管、神経、リンパ管などが出入りする。これらの
\\	における位置関係は
\\	が最前方に、
\\	がその後ろに、
\\	が最後方にある。腎実質はこの
\\	から深くえぐり取られたような腔所をなし、これを
\\	という。
\\	の中には
\\	、枝分かれした腎動・静脈があり、それらのすきまを
\\	が満たしている。	①上方
\\	②腎門
\\	③腎静脈
\\	④腎動脈
\\	⑤尿管
\\	⑥腎洞
\\	⑦腎盤
\\	⑧腎杯
\\	⑨脂肪組織
\\	内臓系
\\	腎臓は内部の
\\	と表層の
\\	からなる。
\\	は深側の約2/3部で、腎洞を囲むように10数個の
\\	が放射状に並んでいる。
\\	の基底面は皮質側を向き錐体底とよばれ、先端は
\\	といい腎洞に向かって突出している。
\\	は杯状の管である
\\	に包まれる。
\\	は集まって広い内腔をもつ
\\	となる。
\\	は腎門において漏斗状から、ついには細い管、すなわち
\\	に移行する。皮質は表面約1/3部で、線維被膜直下にあり、
\\	基底面を弓状におおっている。さらに皮質は隣接する錐体間にも伸びており、この部を
\\	という。一つの
\\	と、これを取り囲む皮質の部を
\\	という。	①髄質
\\	②皮質
\\	③腎錐体
\\	④腎乳頭
\\	⑤腎杯
\\	⑥腎盤(腎盂)
\\	⑦尿管
\\	⑧腎柱
\\	⑨腎葉
\\	内臓系
\\	腎臓の微細構造は、多数の尿細管と血管が互いに結合しあってできている。腎皮質には
\\	とよばれる直径約200
\\	の球状の構造がみられる。
\\	は毛細血管の塊である
\\	と、これを包む袋状の
\\	からなる。
\\	は
\\	をおおっている内葉、すなわち被蓋細胞(タコ足細胞)と、
\\	の腔所を囲んで外壁となっている外葉を区別できる。
\\	で
\\	を形成する細動脈が出入りする側を
\\	から尿細管への連結側を
\\	という。
\\	の数は尿細管に一致して、1個の腎臓につき100~200万個といわれる。
\\	とそれに続く1本の尿細管を、尿の産生と排泄に関する機能上そして構造上の基本単位とみなし、
\\	とよぶ。	①腎小体(マルピギー小体)
\\	②糸球体
\\	③ボーマン嚢(糸球体嚢)
\\	④血管極
\\	⑤尿管極
\\	⑥ネフロン(腎単位)
\\	内臓系
\\	腎小体に続く尿細管はたんなる原尿を運ぶ管ではなく、尿細管における原尿の輸送過程で
\\	などを再吸収する。尿細管の起始部は皮質内にあって、腎小体の周辺をはなはだしく迂曲するので
\\	とよばれる。つづいて直尿細管(直部)となりまっすぐ髄質に下行し、
\\	の下行脚へと移行し、折れ返って上行脚となり、ふたたび皮質に向かう。そして腎小体周辺で迂曲して
\\	となる。最後に髄質に下行して
\\	となり、順次合して太くなりつつ腎乳頭に至り
\\	を形成する。腎乳頭における
\\	の腎杯への開口を
\\	という。	①ブドウ糖
\\	②アミノ酸
\\	③水分
\\	④近位尿細管(近位曲部)
\\	⑤ヘンレのワナ(係蹄)
\\	⑥遠位尿細管(遠位曲部)
\\	⑦集合管
\\	⑧乳頭管
\\	⑨乳頭孔
\\	内臓系
\\	腎動脈は腎門近くで前枝と後枝に分かれる。前枝はさらに約
\\	本の枝に分かれる。結局、腎動脈は一般に約
\\	本の枝に分岐する。これらの各動脈枝を
\\	といい、その分布域を
\\	という。腎臓は上区、上前区、下前区、下区、後区の5区域に分けられ、各区域に同名の区域動脈が分布する。区域動脈からの枝は腎乳頭の間を上行し皮質に向かうが、これを
\\	という。ついで皮質と髄質の間を弓状に走る
\\	となり、さらに
\\	となる。
\\	から
\\	が腎小体に入り、毛細血管網である糸球体を形成する。
\\	として腎小体を去り、再度毛細血管となり尿細管各部を取り巻く。その後、静脈系へと移行し、
\\	、ついで
\\	となる。
\\	は合して
\\	となり腎門を出て
\\	に入る。
\\	が血管極に至る直前で、その血管壁に類上皮細胞の集団がみられる。この細胞を
\\	といい、これは血圧上昇ホルモンである
\\	を分泌している。	
\\	③区域動脈
\\	④腎区域
\\	⑤葉間動脈
\\	⑥弓状動脈
\\	⑦小葉間動脈
\\	⑧輸入細動脈(輸入管)
\\	⑨輸出細動脈(輸出管)
\\	⑩小葉間静脈
\\	⑪弓状静脈
\\	⑫葉間静脈
\\	⑬腎静脈
\\	⑭下大静脈
\\	⑮糸球体傍細胞
\\	⑯レニン
\\	内臓系
\\	尿管は腎盤に続き尿を膀胱に導く長さ約30
\\	の管である。
\\	の下内側から起こり、
\\	によって前方からおおわれ、
\\	筋の前を下内側に下行する。さらに
\\	動脈または
\\	動脈の後ろを通り、これと交叉して下行する。小骨盤口で
\\	動・静脈の前を通って小骨盤腔に至り、骨盤の後壁に沿って走ったのち、膀胱底の後ろから膀胱壁を斜めにつらぬき膀胱に開く。	①腎門
\\	②腹膜
\\	③大腰
\\	④精巣
\\	⑤卵巣
\\	⑥総腸骨
\\	内臓系
\\	尿管は以下に述べる3個所に、やや細い狭窄部がある。
\\	①腎盤から尿管への移行部
\\	②腹部から骨盤部への移行部、すなわち小骨盤入口部(この狭窄部は総腸骨動静脈の前面を乗り越える部でもある)
\\	③膀胱壁の貫通部
\\	内臓系
\\	尿管壁は粘膜、筋層、外膜の3層からなる。粘膜上皮は
\\	上皮で、尿管が空虚の場合、粘膜は縦走するヒダをなす。筋層は
\\	よりなり、これの周期的な煽動(1分間に4~5回)により尿が膀胱へと送られる。外膜は疎性結合組織である。	①移行
\\	②平滑筋
\\	内臓系
\\	膀胱は腎臓で産生され尿管によって送られた尿を貯留し、一定量に達したときにこれを排出する嚢状器官である。小骨盤中で
\\	の後方にあり、尿の充満時には膀胱の上面は
\\	の上縁を越えて上方に膨隆する。膀胱の容量には個体差があるが、成人では一般に
\\	である。膀胱の上部は
\\	でおおわれ、後方には男性では
\\	が、女性では
\\	および
\\	が位置する。膀胱は
\\	の三部を区別する。
\\	は後方を向き、ここに尿管が斜めに前下内側に向かって膀胱壁をつらぬいている。この膀胱への尿管の開口部を
\\	という。
\\	の前下部にある尿道への移行口を
\\	という。
\\	は
\\	と
\\	の間に位置する。	①恥骨結合
\\	③腹膜
\\	④直腸
\\	⑤子宮
\\	⑥膣
\\	⑦膀胱尖
\\	⑧膀胱体
\\	⑨膀胱底
\\	⑩尿管口
\\	⑪尿道口
\\	内臓系
\\	膀胱が空虚なときには、膀胱内面に不規則な多くのヒダがみられる。しかし左右の尿管口と内尿道口でつくられる三角形、すなわち
\\	の粘膜表面は膀胱の収縮、伸展にかかわらずつねに平滑である。	①膀胱三角
\\	内臓系
\\	膀胱壁は粘膜、筋層、漿膜の3層からなる。粘膜の上皮は
\\	上皮である。膀胱三角以外の粘膜には、膀胱が空虚で収縮しているときは多数のヒダがみられるが、充満拡張時には粘膜が伸展しヒダが消失し平滑になる。筋層は
\\	で、内縦層、中輪層、外縦層の3層からなる。輪走筋が発達しているが、とくに内尿道口の周囲で発達しており、これを
\\	とよぶ。
\\	は壁側腹膜の一部で、膀胱の上面と後面の上部をおおう。膀胱の前壁と底では
\\	を欠き、外膜があるだけである。	①移行
\\	②平滑筋
\\	③膀胱括約筋
\\	④漿膜
\\	内臓系
\\	男性尿道の走行は側方からみると全体として
\\	状を呈し、 長さは
\\	である。壁内部、前立腺部、隔膜部、海綿体部の四部に区分できる。
\\	壁内部は、
\\	に始まり膀胱壁内にある短い部である。
\\	前立腺部は、前立腺の中をつらぬく部で、長さ約3
\\	である。前立腺部の後壁の正中線には上下方向に粘膜の高まりがあり、これを尿道稜という。尿道稜に続く後壁の中央部は紡錐状に隆起している。この隆起を
\\	といい、この中央には
\\	とよぶ陥凹があり、その両側には
\\	が開口する。
\\	の両側には多数の
\\	が開口している。
\\	隔膜部は、尿生殖隔膜をつらぬく部で、壁内部についで短く、長さ約1
\\	である。この部を骨格筋線維が取り囲み、
\\	をつくる。隔膜部は恥骨結合の後方約2.5
\\	にあり、尿道のうちもっとも狭い部である。
\\	海綿体部は、陰茎の
\\	中を走るもっとも長い部で、長さは約10
\\	前後である。亀頭内の前端部は広く、
\\	という。亀頭先端の体外への開口を
\\	という。	
\\	②内尿道口
\\	③精丘
\\	④前立腺小室
\\	⑤射精管
\\	⑥前立腺管
\\	⑦尿道括約筋
\\	⑧尿道海綿体
\\	⑨尿道舟状窩
\\	⑩外尿道口
\\	内臓系
\\	女性尿道は
\\	に始まり、
\\	をもって
\\	に開いている。長さは
\\	である。男性と同様に、
\\	の周囲には平滑筋である
\\	が、隔膜部には横紋筋である
\\	がある。女性はこのように尿道が短いため、
\\	からの上行性感染を起こしやすい。	①内尿道口
\\	②外尿道口
\\	③膣前庭
\\	⑤膀胱括約筋(内尿道括約筋)
\\	⑥尿道括約筋(外尿道括約筋)
\\	内臓系
\end{CJK}
\end{document}