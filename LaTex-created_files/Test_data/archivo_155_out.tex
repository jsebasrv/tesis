\documentclass[8pt]{extreport} 
\usepackage{hyperref}
\usepackage{CJKutf8}
\begin{document}
\begin{CJK}{UTF8}{min}
\\	図書室	
\\	としょしつ			
\\	図書室で勉強した。	
\\	図書室 
\\	としょしつ で べんきょう した			
\\	入場券	
\\	にゅうじょうけん			
\\	入場券は3000円です。	
\\	入場券 
\\	にゅうじょうけん は さんぜんえん です			
\\	ネックレス	
\\	妻にネックレスをプレゼントしました。	
\\	ネックレス 
\\	つま に ねっくれす を ぷれぜんと しました			
\\	パジャマ	
\\	このパジャマを着て。	
\\	パジャマ 
\\	この ぱじゃま を きて			
\\	ボールペン	
\\	ボールペンで名前を書いてください。	
\\	ボールペン 
\\	ぼーるぺん で なまえ を かいて ください			
\\	万年筆	
\\	まんねんひつ			
\\	父に万年筆をプレゼントしました。	
\\	万年筆 
\\	ちち に まんねんひつ を ぷれぜんと しました			
\\	柔らか	
\\	やわらか			
\\	彼の声は柔らかだ。	
\\	柔らか 
\\	かれ の こえ は やわらか だ			
\\	覚ます	
\\	さます			
\\	子供が目を覚ました。	
\\	覚ます 
\\	こども が め を さました			
\\	すいか	
\\	夏はすいかが美味しい。	
\\	すいか 
\\	なつ は すいか が おいしい			
\\	そちら	
\\	夕方そちらに着きます。	
\\	そちら 
\\	ゆうがた そちら に つきます			
\\	そば	
\\	お昼にそばを食べた。	
\\	そば 
\\	おひる に そば を たべた			
\\	大抵	
\\	たいてい			
\\	朝食は大抵7時頃に食べます。	
\\	大抵 
\\	ちょうしょく は たいてい しちじごろ に たべます			
\\	特急	
\\	とっきゅう			
\\	東京まで特急で3時間かかります。	
\\	特急 
\\	とうきょう まで とっきゅう で さんじかん かかります			
\\	どっち	
\\	どっちの色が好きですか。	
\\	どっち 
\\	どっち の いろ が すき です か 。			
\\	糊	
\\	のり			
\\	糊で2枚の紙を貼り合わせた。	
\\	糊 
\\	のり で にまい の かみ を はり あわせた			
\\	バスケットボール	
\\	姉はバスケットボールの選手です。	
\\	バスケットボール 
\\	あね は ばすけっとぼーる の せんしゅ です			
\\	晴れ	
\\	はれ			
\\	明日の天気は晴れです。	
\\	晴れ 
\\	あす の てんき は はれ です			
\\	欲しがる	
\\	ほしがる			
\\	子供がジュースを欲しがっています。	
\\	欲しがる 
\\	こども が じゅーす を ほしがって います			
\\	味噌汁	
\\	みそしる			
\\	私は毎日味噌汁を飲みます。	
\\	味噌汁 
\\	わたし は まいにち みそしる を のみます 。			
\\	読み方	
\\	よみかた			
\\	この漢字の読み方を教えてください。	
\\	読み方 
\\	この かんじ の よみかた を おしえて ください			
\\	嫌	
\\	いや			
\\	私は待つのが嫌だ。	
\\	嫌 
\\	わたし は まつ の が いや だ			
\\	入り口	
\\	いりぐち			
\\	入り口は向こうです。	
\\	入り口 
\\	いりぐち は むこう です			
\\	色々	
\\	いろいろ			
\\	彼は色々なことを知っている。	
\\	色々 
\\	かれ は いろいろ な こと を しって いる			
\\	降ろす	
\\	おろす			
\\	彼は車から荷物を降ろした。	
\\	降ろす 
\\	かれ は くるま から にもつ を おろした			
\\	掛かる	
\\	かかる			
\\	壁に大きな時計が掛かっています。	
\\	掛かる 
\\	かべ に おおき な とけい が かかって います			
\\	サンドイッチ	
\\	今日のお昼はサンドイッチです。	
\\	サンドイッチ 
\\	きょう の おひる は さんどいっち です			
\\	四角	
\\	しかく			
\\	紙を四角に切ってください。	
\\	四角 
\\	かみ を しかく に きって ください			
\\	四角い	
\\	しかくい			
\\	こっちの四角いテーブルを買おうよ。	
\\	四角い 
\\	こっち の しかくい てーぶる を かおうよ			
\\	突き当たる	
\\	つきあたる			
\\	突き当たったら右に曲がってください。	
\\	突き当たる 
\\	つきあたったら みぎ に まがって ください			
\\	定期券	
\\	ていきけん			
\\	定期券は1万2千円でした。	
\\	定期券 
\\	ていきけん は いちまんにせんえん でした			
\\	にこにこ	
\\	彼女はいつもにこにこしています。	
\\	にこにこ 
\\	かのじょ は いつも にこにこ して います			
\\	馬鹿	
\\	ばか			
\\	私は馬鹿だった。	
\\	馬鹿 
\\	わたし は ばか だった			
\\	名字	
\\	みょうじ			
\\	あなたの名字は何ですか。	
\\	名字 
\\	あなた の みょうじ は なん です か			
\\	下る	
\\	くだる			
\\	小さな船が川を下っています。	
\\	下る 
\\	ちいさ な ふね が かわ を くだって います			
\\	ウール	
\\	このウールのセーターは暖かい。	
\\	ウール 
\\	この うーる の せーたー は あたたかい			
\\	コーラ	
\\	私の兄はコーラが好きだ。	
\\	コーラ 
\\	わたし の あに は こーら が すき だ			
\\	トイレットペーパー	
\\	トイレットペーパーを交換しました。	
\\	トイレットペーパー 
\\	といれっとぺーぱー を こうかん しました			
\\	どなた	
\\	失礼ですが、どなたですか。	
\\	どなた 
\\	しつれい です が 、 どなた です か 。			
\\	乗り換え	
\\	のりかえ			
\\	次の駅で乗り換えです。	
\\	乗り換え 
\\	つぎ の えき で のりかえ です			
\\	ハイキング	
\\	昨日は友人とハイキングに行きました。	
\\	ハイキング 
\\	きのう は ゆうじん と はいきんぐ に いきました			
\\	灰皿	
\\	はいざら			
\\	灰皿をください。	
\\	灰皿 
\\	はいざら を ください			
\\	バス停	
\\	ばすてい			
\\	次のバス停で降ります。	
\\	バス停 
\\	つぎ の ばすてい で おります			
\\	ハム	
\\	ハムサンドをください。	
\\	ハム 
\\	はむ さんど を ください			
\\	ボーイフレンド	
\\	私のボーイフレンドはハンサムな方でした。	
\\	ボーイフレンド 
\\	わたし の ぼーいふれんど は はんさむな かた でした			
\\	床	
\\	ゆか			
\\	床がぬれている。	
\\	床 
\\	ゆか が ぬれて いる			
\\	例文	
\\	れいぶん			
\\	例文を3つ作ってください。	
\\	例文 
\\	れいぶん を みっつ つくって ください			
\\	ローマ字	
\\	ろーまじ			
\\	ローマ字で名前を書いてください。	
\\	ローマ字 
\\	ろーまじ で なまえ を かいて ください			
\\	アイロン	
\\	彼女はシャツにアイロンをかけた。	
\\	アイロン 
\\	かのじょ は しゃつ に あいろん を かけた			
\\	苺	
\\	いちご			
\\	この苺はとても甘い。	
\\	苺 
\\	この いちご は とても あまい			
\\	カレーライス	
\\	子供はカレーライスが好きです。	
\\	カレーライス 
\\	こども は かれーらいす が すき です			
\\	キャッシュカード	
\\	キャッシュカードでお金を下ろした。	
\\	キャッシュカード 
\\	きゃっしゅかーど で おかね を おろした			
\\	曇り	
\\	くもり			
\\	今日は一日曇りでした。	
\\	曇り 
\\	きょう は いちにち くもり でした			
\\	魚	
\\	さかな			
\\	塩	
\\	しお			
\\	もうちょっと塩を入れて。	
\\	塩 
\\	もう ちょっと しお を いれて			
\\	停留所	
\\	ていりゅうじょ			
\\	バスの停留所で10分待ちました。	
\\	停留所 
\\	ばす の ていりゅうじょ で じゅっぷん まちました			
\\	床屋	
\\	とこや			
\\	昨日床屋で髪を切った。	
\\	床屋 
\\	きのう とこや で かみ を きった			
\\	歯医者	
\\	はいしゃ			
\\	私は歯医者が嫌いです。	
\\	歯医者 
\\	わたし は はいしゃ が きらい です			
\\	歯磨き	
\\	はみがき			
\\	歯磨きはしましたか。	
\\	歯磨き 
\\	はみがき は しました か			
\\	早起き	
\\	はやおき			
\\	祖父は早起きです。	
\\	早起き 
\\	そふ は はやおき です			
\\	冬休み	
\\	ふゆやすみ			
\\	冬休みにお祖父ちゃんの家に行きます。	
\\	冬休み 
\\	ふゆやすみ に おじいちゃん の うち に いきます			
\\	持って行く	
\\	もっていく			
\\	水を持って行きましょう。	
\\	持って行く 
\\	みず を もっていきましょう			
\\	夕飯	
\\	ゆうはん			
\\	夕飯は寿司でした。	
\\	夕飯 
\\	ゆうはん は すし でした			
\\	酔っ払い	
\\	よっぱらい			
\\	例	
\\	れい			
\\	一つ例をあげてください。	
\\	例 
\\	ひとつ れい を あげて ください			
\\	イヤリング	
\\	彼女はすてきなイヤリングをしている。	
\\	イヤリング 
\\	かのじょ は すてき な いやりんぐ を して いる			
\\	横断歩道	
\\	おうだんほどう			
\\	あそこに横断歩道があります。	
\\	横断歩道 
\\	あそこ に おうだんほどう が あります			
\\	お辞儀	
\\	おじぎ			
\\	皆、社長にお辞儀をした。	
\\	お辞儀 
\\	みんな しゃちょう に おじぎ を した			
\\	お宅	
\\	おたく			
\\	先生のお宅はどちらですか。	
\\	お宅 
\\	せんせい の おたく は どちら です か			
\\	ガールフレンド	
\\	昔は彼女のことをガールフレンドと言っていました。	
\\	ガールフレンド 
\\	むかし は かのじょ の こと を がーるふれんど と いって いました			
\\	カセットテープ	
\\	昔はカセットテープを使っていました。	
\\	カセットテープ 
\\	むかし は カセットテープ を つかっていました 。			
\\	かゆい	
\\	背中がかゆいです。	
\\	かゆい 
\\	せなか が かゆい です			
\\	乾かす	
\\	かわかす			
\\	ぬれた服を乾かした。	
\\	乾かす 
\\	ぬれた ふく を かわかした			
\\	休校	
\\	きゅうこう			
\\	学校は今週は休校です。	
\\	休校 
\\	がっこう は こんしゅう は きゅうこう です			
\\	ぐらぐら	
\\	地震で家がぐらぐらと揺れた。	
\\	ぐらぐら 
\\	じしん で いえ が ぐらぐら と ゆれた			
\\	航空便	
\\	こうくうびん			
\\	航空便で書類が届きました。	
\\	航空便 
\\	こうくうびん で しょるい が とどきました			
\\	ジャム	
\\	このイチゴでジャムを作りましょう。	
\\	ジャム 
\\	この いちご で じゃむ を つくりましょう			
\\	スリッパ	
\\	スリッパをはいてください。	
\\	スリッパ 
\\	すりっぱ を はいて ください			
\\	大嫌い	
\\	だいきらい			
\\	私はテストが大嫌い。	
\\	大嫌い 
\\	わたし は てすと が だいきらい			
\\	畳む	
\\	たたむ			
\\	布団を畳んでください。	
\\	畳む 
\\	ふとん を たたんで ください			
\\	トランプ	
\\	友達とトランプをして遊びました。	
\\	トランプ 
\\	ともだち と とらんぷ を して あそびました			
\\	万歳	
\\	ばんざい			
\\	勝った、万歳!	
\\	万歳 
\\	かった ばんざい			
\\	番地	
\\	ばんち			
\\	その建物の番地は分かりますか。	
\\	番地 
\\	その たてもの の ばんち は わかります か			
\\	日陰	
\\	ひかげ			
\\	暑いので日陰で休みましょう。	
\\	日陰 
\\	あつい の で ひかげ で やすみましょう			
\\	美容院	
\\	びよういん			
\\	父は美容院で髪を切ります。	
\\	美容院 
\\	ちち は びよういん で かみ を きります			
\\	文房具	
\\	ぶんぼうぐ			
\\	新しい文房具を買いました。	
\\	文房具 
\\	あたらしい ぶんぼうぐ を かいました			
\\	免許証	
\\	めんきょしょう			
\\	免許証を見せてください。	
\\	免許証 
\\	めんきょしょう を みせて ください			
\\	訳す	
\\	やくす			
\\	この文を訳してください。	
\\	訳す 
\\	この ぶん を やくして ください			
\\	よろしい	
\\	レポートはこれでよろしいですか。	
\\	よろしい 
\\	れぽーと は これ で よろしい です か			
\\	ラッシュアワー	
\\	私はラッシュアワーの電車が嫌いです。	
\\	ラッシュアワー 
\\	わたし は らっしゅあわー の でんしゃ が きらい です			
\\	両替	
\\	りょうがえ			
\\	1万円を両替してください。	
\\	両替 
\\	いちまんえん を りょうがえ して ください			
\\	明日	
\\	あした			
\\	では、また明日。	
\\	明日 
\\	では 、 また あした。			
\\	送り仮名	
\\	おくりがな			
\\	その送り仮名は間違っています。	
\\	送り仮名 
\\	その おくりがな は まちがって います			
\\	片付く	
\\	かたづく			
\\	仕事が大分片付いた。	
\\	片付く 
\\	しごと が だいぶ かたづいた			
\\	カップ	
\\	水を1カップ入れてください。	
\\	カップ 
\\	みず を いちかっぷ いれて ください			
\\	消しゴム	
\\	けしごむ			
\\	消しゴムを貸して下さい。	
\\	消しゴム 
\\	けしごむ を かして ください			
\\	主人	
\\	しゅじん			
\\	主人は今、留守です。	
\\	主人 
\\	しゅじん は いま るす です			
\\	スケート	
\\	湖でスケートをした。	
\\	スケート 
\\	みずうみ で すけーと を した			
\\	たくさん	
\\	昨日はたくさん泳ぎました。	
\\	たくさん 
\\	きのう は たくさん およぎました			
\\	中	
\\	なか			
\\	鍋	
\\	なべ			
\\	鍋にスープが入っているよ。	
\\	鍋 
\\	なべ に すーぷ が はいって いる よ			
\\	怠ける	
\\	なまける			
\\	怠けていないで、手伝って。	
\\	怠ける 
\\	なまけて いない で てつだって			
\\	二階	
\\	にかい			
\\	兄は二階にいます。	
\\	二階 
\\	あに は にかい に います			
\\	寝坊	
\\	ねぼう			
\\	今朝は寝坊しました。	
\\	寝坊 
\\	けさ は ねぼう しました			
\\	鋏	
\\	はさみ			
\\	この鋏はよく切れる。	
\\	鋏 
\\	この はさみ は よく きれる			
\\	畑	
\\	はたけ			
\\	母は畑仕事が好きだ。	
\\	畑 
\\	はは は はたけ しごと が すき だ			
\\	ピクニック	
\\	今日はピクニックに行きましょう。	
\\	ピクニック 
\\	きょう は ぴくにっく に いきましょう			
\\	真っ青	
\\	まっさお			
\\	空が真っ青です。	
\\	真っ青 
\\	そら が まっさお です			
\\	汚す	
\\	よごす			
\\	彼は服を汚した。	
\\	汚す 
\\	かれ は ふく を よごした			
\\	酔っ払う	
\\	よっぱらう			
\\	彼はただの酔っ払いです。	
\\	酔っ払い 
\\	酔っ払う 
\\	かれ は ただ の よっぱらい です			
\\	昨夜は酔っ払いました。	
\\	酔っ払い 
\\	酔っ払う 
\\	ゆうべ は よっぱらいました			
\\	忘れ物	
\\	わすれもの			
\\	学校に忘れ物をしました。	
\\	忘れ物 
\\	がっこう に わすれもの を しました			
\\	あちら	
\\	あちらに行ってみよう。	
\\	あちら 
\\	あちら に いって みよう			
\\	掛け算	
\\	かけざん			
\\	弟は掛け算を習っている。	
\\	掛け算 
\\	おとうと は かけざん を ならって いる			
\\	臭い	
\\	くさい			
\\	納豆は臭い。	
\\	臭い 
\\	なっとう は くさい			
\\	自習	
\\	じしゅう			
\\	明日は自習の時間がある。	
\\	自習 
\\	あす は じしゅう の じかん が ある			
\\	突き当たり	
\\	つきあたり			
\\	この先は突き当たりです。	
\\	突き当たり 
\\	この さき は つきあたり です			
\\	どきどき	
\\	彼女に会うとどきどきします。	
\\	どきどき 
\\	かのじょ に あう と どきどき します			
\\	無くす	
\\	なくす			
\\	今日、鍵を無くしました。	
\\	無くす 
\\	きょう かぎ を なくしました			
\\	納豆	
\\	なっとう			
\\	私は納豆をよく食べます。	
\\	納豆 
\\	わたし は なっとう を よく たべます			
\\	二十歳	
\\	はたち			
\\	妹は来年、二十歳になります。	
\\	二十歳 
\\	いもうと は らいねん はたち に なります			
\\	東	
\\	ひがし			
\\	日は東から上ります。	
\\	東 
\\	ひ は ひがし から のぼります			
\\	まあまあ	
\\	彼の成績はまあまあです。	
\\	まあまあ 
\\	かれ の せいせき は まあまあ です			
\\	マフラー	
\\	彼女は首にマフラーをまいていた。	
\\	マフラー 
\\	かのじょ は くび に まふらー を まいて いた			
\\	桃	
\\	もも			
\\	私の一番好きな果物は桃です。	
\\	桃 
\\	わたし の いちばん すき な くだもの は もも です			
\\	アクセサリー	
\\	このアクセサリーは素敵ね。	
\\	アクセサリー 
\\	この あくせさりー は すてき ね			
\\	あくび	
\\	父があくびをした。	
\\	あくび 
\\	ちち が あくび を した			
\\	飴	
\\	あめ			
\\	缶に飴が入っています。	
\\	飴 
\\	かん に あめ が はいって います			
\\	売り切れ	
\\	うりきれ			
\\	チケットはもう売り切れだって。	
\\	売り切れ 
\\	ちけっと は もう うりきれ だって			
\\	お巡りさん	
\\	おまわりさん			
\\	あのお巡りさんに道を聞きましょう。	
\\	お巡りさん 
\\	あの おまわりさん に みち を ききましょう			
\\	からから	
\\	おしゃべりして喉がからからになりました。	
\\	からから 
\\	おしゃべり して のど が からから に なりました			
\\	黄色	
\\	きいろ			
\\	信号は黄色でした。	
\\	黄色 
\\	しんごう は きいろ でした			
\\	誘う	
\\	さそう			
\\	彼女をデートに誘った。	
\\	誘う 
\\	かのじょ を でーと に さそった			
\\	速達	
\\	そくたつ			
\\	これを速達で送りたいのですが。	
\\	速達 
\\	これ を そくたつ で おくりたい の です が			
\\	歯	
\\	は			
\\	私の歯は丈夫です。	
\\	歯 
\\	わたし の は は じょうぶ です			
\\	引き算	
\\	ひきざん			
\\	娘は学校で引き算を習っている。	
\\	引き算 
\\	むすめ は がっこう で ひきざん を ならって いる			
\\	昼前	
\\	ひるまえ			
\\	昼前に会議があった。	
\\	昼前 
\\	ひるまえ に かいぎ が あった			
\\	ふらふら	
\\	私は疲れてふらふらです。	
\\	ふらふら 
\\	わたし は つかれて ふらふら です			
\\	目覚まし時計	
\\	めざましどけい			
\\	7時に目覚まし時計が鳴りました。	
\\	目覚まし時計 
\\	しちじ に めざましどけい が なりました			
\\	予習	
\\	よしゅう			
\\	明日の予習をしましょう。	
\\	予習 
\\	あした の よしゅう を しましょう			
\\	こぼす	
\\	ソファにワインをこぼした。	
\\	こぼす 
\\	そふぁ に わいん を こぼした			
\\	ランチ	
\\	一緒にランチに行きませんか。	
\\	ランチ 
\\	いっしょ に らんち に いきません か			
\\	レシート	
\\	レシートを財布に入れました。	
\\	レシート 
\\	れしーと を さいふ に いれました			
\\	持って来る	
\\	もってくる			
\\	そのいすを持って来てください。	
\\	持って来る 
\\	その いす を もって きて ください			
\\	気を付ける	
\\	きをつける			
\\	体に気を付けてください。	
\\	気を付ける 
\\	からだ に き を つけて ください			
\\	刺身	
\\	さしみ			
\\	私は刺身は食べません。	
\\	刺身 
\\	わたし は さしみ は たべません			
\\	洗面所	
\\	せんめんじょ			
\\	洗面所で顔を洗った。	
\\	洗面所 
\\	せんめんじょ で かお を あらった			
\\	足し算	
\\	たしざん			
\\	娘は学校で足し算を習った。	
\\	足し算 
\\	むすめ は がっこう で たしざん を ならった			
\\	段々	
\\	だんだん			
\\	段々仕事が楽しくなってきました。	
\\	段々 
\\	だんだん しごと が たのしく なって きました			
\\	チョーク	
\\	包む	
\\	つつむ			
\\	プレゼントをきれいな紙で包みました。	
\\	包む 
\\	プレゼント を きれい な かみ で つつみました			
\\	ティッシュ	
\\	ティッシュを取ってください。	
\\	ティッシュ 
\\	てぃっしゅ を とって ください			
\\	止まる	
\\	とまる			
\\	今朝、事故で電車が止まりました。	
\\	止まる 
\\	けさ じこ で でんしゃ が とまりました			
\\	無くなる	
\\	なくなる			
\\	もうお金が無くなりました。	
\\	無くなる 
\\	もう おかね が なくなりました			
\\	西	
\\	にし			
\\	太陽は西に沈みます。	
\\	西 
\\	たいよう は にし に しずみます			
\\	ノック	
\\	入る時はドアをノックしてください。	
\\	ノック 
\\	はいる とき は どあ を のっく して ください			
\\	歯ブラシ	
\\	はぶらし			
\\	新しい歯ブラシが必要だ。	
\\	歯ブラシ 
\\	あたらしい はぶらし が ひつよう だ			
\\	暇	
\\	ひま			
\\	明日は暇ですか。	
\\	暇 
\\	あした は ひま です か			
\\	豚	
\\	ぶた			
\\	豚はすごく鼻が良い。	
\\	豚 
\\	ぶた は すごく はな が いい			
\\	二人	
\\	ふたり			
\\	今日は妻と二人で食事をします。	
\\	二人 
\\	きょう は つま と ふたり で しょくじ を します			
\\	褒める	
\\	ほめる			
\\	頑張ったので褒められました。	
\\	褒める 
\\	がんばった の で ほめられました			
\\	本箱	
\\	ほんばこ			
\\	雑誌を本箱に入れました。	
\\	本箱 
\\	ざっし を ほんばこ に いれました			
\\	止む	
\\	やむ			
\\	雨が止みました。	
\\	止む 
\\	あめ が やみました			
\\	よろしく	
\\	よろしくお願いします。	
\\	よろしく 
\\	よろしく おねがい します			
\\	沸かす	
\\	わかす			
\\	お湯を沸かしてください。	
\\	沸かす 
\\	おゆ を わかして ください			
\\	明後日	
\\	あさって			
\\	明後日は休日です。	
\\	明後日 
\\	あさって は きゅうじつ です 。			
\\	幾ら	
\\	いくら			
\\	この靴は幾らですか。	
\\	幾ら 
\\	この くつ は いくら です か			
\\	絵葉書	
\\	えはがき			
\\	友達から絵葉書が届きました。	
\\	絵葉書 
\\	ともだち から えはがき が とどきました			
\\	暖める	
\\	あたためる			
\\	今、車を暖めています。	
\\	暖める 
\\	いま くるま を あたためています			
\\	一昨日	
\\	おととい			
\\	一昨日彼から電話がありました。	
\\	一昨日 
\\	おととい かれ から でんわ が ありました			
\\	換える	
\\	かえる			
\\	車のタイヤを換えた。	
\\	換える 
\\	くるま の たいや を かえた			
\\	書留	
\\	かきとめ			
\\	これを書留で送りたいのですが。	
\\	書留 
\\	これ を かきとめ で おくりたい の です が			
\\	がらがら	
\\	映画館はがらがらでした。	
\\	がらがら 
\\	えいがかん は がらがら でした			
\\	缶詰	
\\	かんづめ			
\\	缶詰のフルーツはとても甘い。	
\\	缶詰 
\\	かんづめ の ふるーつ は とても あまい			
\\	ぎらぎら	
\\	太陽がぎらぎらしている。	
\\	ぎらぎら 
\\	たいよう が ぎらぎら して いる			
\\	九	
\\	く			
\\	私は九月に行く予定です。	
\\	九 
\\	わたし は くがつ に いく よてい です			
\\	怪我	
\\	けが			
\\	彼女は腕を怪我した。	
\\	怪我 
\\	かのじょ は うで を けがした			
\\	寒さ	
\\	さむさ			
\\	今日は厳しい寒さになるでしょう。	
\\	寒さ 
\\	きょう は きびしい さむさ に なる でしょう			
\\	祝日	
\\	しゅくじつ			
\\	5月3日は祝日だ。	
\\	祝日 
\\	ごがつ みっか は しゅくじつ だ			
\\	すき焼き	
\\	すきやき			
\\	昨日の夜はすき焼きを食べた。	
\\	すき焼き 
\\	きのう の よる は すきやき を たべた			
\\	そっち	
\\	そっちが私の部屋です。	
\\	そっち 
\\	そっち が わたし の へや です			
\\	葉書	
\\	はがき			
\\	家族に葉書を書いています。	
\\	葉書 
\\	かぞく に はがき を かいています 。			
\\	掃く	
\\	はく			
\\	床をほうきで掃きました。	
\\	床 
\\	掃く 
\\	ゆか を ほうき で はきました			
\\	昼間	
\\	ひるま			
\\	昼間は仕事で忙しいです。	
\\	昼間 
\\	ひるま は しごと で いそがしい です			
\\	ピンポン	
\\	昨日の夕方、友達とピンポンをした。	
\\	ピンポン 
\\	きのう の ゆうがた ともだち と ぴんぽん を した			
\\	深さ	
\\	ふかさ			
\\	このプールの深さは2メートルです。	
\\	深さ 
\\	この ぷーる の ふかさ は にめーとる です			
\\	不親切	
\\	ふしんせつ			
\\	その店員は不親切だった。	
\\	不親切 
\\	その てんいん は ふしんせつ だった			
\\	ぺこぺこ	
\\	私はおなかがぺこぺこです。	
\\	ぺこぺこ 
\\	わたし は おなか が ぺこぺこ です			
\\	ぺらぺら	
\\	彼女はドイツ語がぺらぺらです。	
\\	ぺらぺら 
\\	かのじょ は どいつご が ぺらぺら です			
\\	横書き	
\\	よこがき			
\\	この本は横書きです。	
\\	横書き 
\\	この ほん は よこがき です			
\\	レインコート	
\\	雨なのでレインコートを着ました。	
\\	レインコート 
\\	あめ な の で れいんこーと を きました			
\\	アイス	
\\	暑いのでアイスを食べました。	
\\	アイス 
\\	あつい の で あいす を たべました			
\\	いつごろ	
\\	いつごろ夏休みを取りますか。	
\\	いつごろ 
\\	いつごろ なつやすみ を とります か			
\\	ウィスキー	
\\	このウィスキーは強いよ。	
\\	ウィスキー 
\\	この うぃすきー は つよい よ			
\\	エアメール	
\\	彼にエアメールを送りました。	
\\	エアメール 
\\	かれ に えあめーる を おくりました			
\\	字	
\\	じ			
\\	チョークで黒板に字を書きました。	
\\	チョーク 
\\	字 
\\	ちょーく で こくばん に じ を かきました			
\\	もっと大きく字を書いてください。	
\\	字 
\\	もっと おおきく じ を かいて ください			
\\	塩辛い	
\\	しおからい			
\\	海の水は塩辛い。	
\\	塩辛い 
\\	うみ の みず は しおからい			
\\	七	
\\	しち			
\\	そのグループのメンバーは全部で七人だ。	
\\	七 
\\	その ぐるーぷ の めんばー は ぜんぶ で しちにん だ			
\\	シャープペンシル	
\\	私のシャープペンシルがありません。	
\\	シャープペンシル 
\\	わたし の しゃーぷぺんしる が ありません			
\\	セロテープ	
\\	セロテープはありますか。	
\\	セロテープ 
\\	せろてーぷ は あります か			
\\	外側	
\\	そとがわ			
\\	白線の外側を歩かないでください。	
\\	外側 
\\	はくせん の そとがわ を あるかない で ください			
\\	そば屋	
\\	そばや			
\\	昼はそば屋に行きました。	
\\	そば 
\\	そば屋 
\\	ひる は そばや に いきました 。			
\end{CJK}
\end{document}