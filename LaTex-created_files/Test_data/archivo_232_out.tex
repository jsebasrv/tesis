\documentclass[8pt]{extreport} 
\usepackage{hyperref}
\usepackage{CJKutf8}
\begin{document}
\begin{CJK}{UTF8}{min}
\\	ほとんど	
\\	5時までに仕事を終えるのはほとんど不可能だ
\\	誰にも劣らない	
\\	言うまでもなく	
\\	彼女は、ポルトガル語は言うまでもなく、英語を話せない
\\	それどころか	
\\	自分で、本人が	
\\	途方に暮れて、困って	
\\	彼はとても大きなショックを受けたので言葉に詰まった
\\	(~の)じゃまになって	
\\	本当にじゃま
\\	途中で	
\\	ところで	
\\	問題にならない、不可能な	
\\	昇給は不可能だ
\\	一般に、概して:一般の	
\\	特に	
\\	特に~
\\	概して、普通は	
\\	概して、~
\\	概して、大体において	
\\	概して、~
\\	逃走中で、捕まらないで	
\\	彼は先週以来逃走中である
\\	永久に、ずっと	
\\	私はこれから先ずっと酒をやめることを決心した
\\	むだに	
\\	私の努力は全て無駄になってしまった
\\	むだに、むなしく	
\\	色々な試みがなされてきたが、うまくいかなかった
\\	仕事中で、働いて	
\\	彼は夜遅くまで仕事をしていた
\\	進行中で	
\\	新空港の建設が進行中である
\\	偶然に	
\\	彼女は偶然友達に出会った
\\	故意に、わざと	
\\	私は故意に間違えた
\\	忙しく働いて、活動して	
\\	彼女は朝から晩まで働きづめである
\\	今度は	
\\	今度は彼女が私にいくつかの質問をした
\\	前もって、あらかじめ	
\\	あなたの出発時刻を前もって教えてください
\\	ときどき(3つ)	
\\	故障して	
\\	そのパソコンは故障していた
\\	時代遅れの	
\\	この型の靴は時代遅れだ
\\	見えないところに	
\\	その巨人は(姿が)見えなくなった
\\	手に負えない	
\\	当分の間、さしあたり	
\\	当分の間それを使わなければいけない
\\	引き続いて、れんぞくして	
\\	まもなく、すぐに	
\\	万一に備えて	
\\	結局は、最後には	
\\	品質の良いものを買うことが結局は特になる
\\	当然のこととして、もちろん	
\\	実際のところ	
\\	言葉では表現できない(ほど)	
\\	時代遅れで、時勢に遅れて	
\\	とにかく、いずれにしても	
\\	その場で、即座に	
\\	警察官はその男を現行犯逮捕した
\\	ひとつには	
\\	気軽に、くつろいで	
\\	看護師は患者を安心させるためにできることは何でもします
\\	急いで	
\\	とりわけ	
\\	たしかに、きっと	
\\	まじめに、本気で	
\\	詳しく、詳細に	
\\	すぐに、あっという間に	
\\	市全体が瞬時にして壊滅した
\\	言い換えれば、つまり	
\\	勤務中で、当番で	
\\	初めのうちは	
\\	初めて	
\\	~年(月・日)ぶりに	
\\	の陰で、
\\	のいないところで	
\\	陰で他人の悪口を言ってはいけない
\\	独力で、ひとりで	
\\	要領を得た、的外れの	
\\	あなたの説明はとても要領を得ていました。
\\	要領を得ない、的外れの	
\\	間に合って、遅れずに	
\\	時間通りに	
\\	(1分もたがわずに)きっかり	
\\	生まれつき	
\\	すぐ近くに、間近に	
\\	徐々に、次第に	
\\	要するに、手短にいえば	
\\	必ず、きっと、間違いなく	
\\	来週の金曜日に必ずお金を返してください
\\	ある程度は、ある程度まで	
\\	ある程度はあなたの見解に賛成するが、その決断については支持できない
\\	~の見地からすると	
\\	政治的見地からすると、その計画はあまり推薦できない
\\	ある意味で	
\\	手の届くところに	
\\	私は辞書をすぐ手の届くところに置いておきたい
\\	人前で、公然と	
\\	直接に、じかに	
\\	の自由になる(ように)	
\\	私は自由に使える金をたくさん持っている
\\	突然(2つ)	
\\	私の考えでは	
\\	うつぶせになって	
\\	うつぶせに眠っている男性は私の夫です
\\	ダイエット中で	
\\	~について言えば・~に関する限りでは	
\\	私はと言えば、その提案に反対です
\\	~について	
\\	彼は次になにをすべきかわからなかった
\\	~に関して(は)(3つ)	
\\	その理論に関しては、彼は正しい
\\	~する目的で、~するために	
\\	~する目的で、~するつもりで	
\\	~しないように、~するといけないから	
\\	~のために、~を目的として	
\\	~のために、~が原因で(4つ)	
\\	~のおかげで、~のせいで	
\\	~という理由で	
\\	トムはいつも病気を理由に、パーティーの招待を断っている
\\	~にもかかわらず(3つ)	
\\	私達の努力にもかかわらず、その計画は失敗に終わった 
\\	私達は彼の意向にかまわず自分たちの計画を実行する決心です
\\	~(情報源)によれば	
\\	~の代わりに、~でなく、~しないで	
\\	マイクはパーティーに行かないで家で勉強した
\\	~の代わりに
\\	じゃないほう)	
\\	期末試験に代えて、レポートを書いてもらおうと思います
\\	~の点から、~に関して	
\\	教育は単に年数の点からのみはかるべきものではありません
\\	~によって、~を用いて	
\\	私達は言葉を用いて考えを表現する
\\	~経由で	
\\	彼は日本経由でヨーロッパからオーストラリアに戻った
\\	~を犠牲にして(2つ)
\\	つかわないやつ)	
\\	~のなすがままに、~に左右されて	
\\	かわいそうにその幼い女の子は残酷な運命に翻弄された
\\	~を探して、~を求めて	
\\	徳川の秘宝を求めて
\\	~に加えて、~のほかに	
\\	首相のほかにも多くの重要人物がその場に居合わせた
\\	~に照らして、~から考えて
\\	つかわないやつ)	
\\	過去の経験に照らして彼らはその計画を再検討した
\\	~の責任で、~次第で	
\\	君が何を考えているのかみんなに話すことが君の責任だ
\\	~を代表して、~に代わって
\\	つかわないやつ)	
\\	アメリカ人を代表して
\\	~に敬意を表して、~のために	
\\	歓迎会がスミス氏のために開かれた
\\	~に賛成して	
\\	~を担当(世話・管理)して	
\\	ともこが昼食のお金を管理している
\\	~に反して
\\	と区別するように)	
\\	私の予想に反して、~
\\	~はべつとして、~はさておき	
\\	~に直面して、~をものともせず	
\\	彼は数々の障害を物ともせず前進した
\\	~を見て	
\\	どろぼうは警官を見て逃げ去った
\\	~のお返しに、~の返礼として	
\\	親切にしてくれたお返しに
\end{CJK}
\end{document}