\documentclass[8pt]{extreport} 
\usepackage{hyperref}
\usepackage{CJKutf8}
\begin{document}
\begin{CJK}{UTF8}{min}
\\	おおざっぱ	おおざっぱ	
\\	〔大づかみな様子〕 大ざっぱな (に) 
\\	大ざっぱな計算で約100ドルかかる 
\\	大ざっぱに見積もって損害は2,000万円になる 
\\	大ざっぱに言って 
\\	〔雑(ざつ)な様子〕
\\	彼の仕事は大ざっぱだ 
\\	大通り	おおどおり	
\\	大凡	おおよそ	
\\	お帰り	おかえり	
\\	おかけください	おかけください	
\\	おかげさまで	おかげさまで	
\\	おかず	【▲御数】	
\\	お弁当のおかずは何がいい 
\\	おかまいなく	おかまいなく	
\\	拝む	おがむ	
\\	〔礼拝する〕
\\	〔祈願する〕
\\	助けを求めて神を拝んだ 
\\	日の出を拝む 
\\	〔懇願する〕
\\	金を貸してくれと拝まれた 
\\	〔拝見する〕 秘蔵(ひぞう)の美術品を拝ませてください 
\\	お代わり	おかわり	
\\	((口)) 
\\	お代わりをどうぞ 
\\	彼は何杯もご飯をお代わりした 
\\	お茶のお代わりはいかがですか 
\\	コーヒーのお代わりは無料です 
\\	お銚子(ちょうし),お代わり4本(よんほん) 
\\	補う	おぎなう	
\\	〔欠損を〕
\\	家庭教師をして生活費を補った 
\\	不足額は臨時費で補った 
\\	彼女の長所は短所を補って余りある 
\\	お気の毒に	おきのどくに	
\\	屋外	おくがい	
\\	送り仮名	おくりがな	
\\	お元気で	おげんきで	
\\	怠る	おこたる	
\\	勉強を怠る 
\\	ドアの戸締(し)まりを怠る 
\\	返事を書くのを怠っている 
\\	警戒(けいかい)を怠るな 
\\	注意を怠ると事故を起こす 
\\	お先に	おさきに	
\\	伯父	おじ	
\\	叔父	おじ	
\\	惜しい	おしい	
\\	〔失いがたい〕 
\\	命が惜しい 
\\	惜しい人をなくしたものだ 
\\	惜しそうに 
\\	彼は惜しそうにペンを差し出した 
\\	〔もったいない〕 
\\	くだらない会議に時間を使うのは惜しい 
\\	秘書にしておくには惜しい男だ 
\\	いま捨ててしまっては惜しい 
\\	君にあげるのなら惜しくない 
\\	〔残念だ〕 
\\	この経験を生(い)かせなかったのは惜しい 
\\	生かす 
\\	もう5分時間があればよかったのだが惜しかった 
\\	惜しいことに[惜しくも]彼に会いそこなった 
\\	惜しい勝負(しょうぶ)だったなあ 
\\	伯父さん	おじさん	
\\	小父さん	おじさん	
\\	叔父さん	おじさん	
\\	お邪魔します	おじゃまします	
\\	世話	せわ	
\\	〔めんどうを見ること〕
\\	〔助力〕
\\	病人の世話をする 
\\	その施設では母のない子供を5人世話している 
\\	その少年は私の伯母の世話になっている 
\\	伯父には金銭上の世話になった 
\\	たいそうお世話 (様) になりました 
\\	息子はその婦人にたいへんお世話になった 
\\	東京に滞在中はいろいろお世話になり,ありがとうございました 
\\	〔やっかい,迷惑〕 他人にあまり世話を掛けないように 
\\	お世話を掛けてすみませんが… 
\\	〔紹介,斡旋(あっせん)〕 よい医者を世話してください 
\\	彼は親切にもよい就職口を世話してくれた 
\\	大きなお世話 大きなお世話だ 
\\	世話がない 君はいつも自分で自分をほめるんだから世話がない 
\\	世話が焼ける うちの息子は本当に世話が焼ける[焼けない] 
\\	世話を焼く 彼女はよく弟の世話を焼く 
\\	若い人たちは世話を焼かれるのを嫌がる 
\\	世話狂言 
\\	世話好き 
\\	世話女房 
\\	世話人 
\\	世話物 
\\	世話焼き 〔人の世話が好きな人〕
\\	〔おせっかい〕
\\	世話役 
\\	この会の世話役は鈴木さんです 
\\	お大事に	おだいじに	
\\	落着く	おちつく	
\\	お出掛け	おでかけ	
\\	お手伝いさん	おてつだいさん	
\\	脅かす	おどかす	
\\	落し物	おとしもの	
\\	驚かす	おどろかす	
\\	お願いします	おねがいします	
\\	各々	おのおの	
\\	伯母	おば	
\\	叔母	おば	
\\	小母さん	おばさん	
\\	お参り	おまいり	
\\	お待たせしました	おまたせしました	
\\	お待ちください	おまちください	
\\	おまちどおさま	おまちどおさま	
\\	おめでたい	おめでたい	
\\	思い掛けない	おもいがけない	
\\	思いがけない幸運が転(ころ)がり込んできた 
\\	旅行中に思いがけない災難に出会った 
\\	駅で思いがけない人に会った 
\\	思い切り	おもいきり	
\\	〔思う存分〕
\\	〔力一杯〕
\\	彼を思い切りなぐった 
\\	思い切り泣いた 
\\	在庫品を思い切り安いお値段にて提供(ていきょう)いたします 
\\	〔思い切ること〕 実に思い切りのよい人だった〔あきらめが速い〕 
\\	思い込む	おもいこむ	
\\	〔決心する〕 思い込んだら命がけ 
\\	〔ひとり決めする〕 彼女は僕が言い触らしたと思い込んでいる 
\\	てっきり自分の傘だと思い込んでいたら彼のだった 
\\	自分が正しいと思い込んでいた 
\\	重たい	おもたい	
\\	お休み	おやすみ	
\\	おやつ	おやつ	
\\	おやつの時間です 
\\	おやつを食べ過ぎる 
\\	親指	おやゆび	
\\	恩恵	おんけい	
\\	((英)) 
\\	人に恩恵を施す 
\\	人間は日光の恩恵を受けている 
\\	社会の恩恵に浴する 
\\	温室	おんしつ	
\\	温泉	おんせん	
\\	温帯	おんたい	
\\	御中	おんちゅう	水野商会御中 
\\	会計課御中 
\\	特に言葉を添える必要はない)
\\	女の人	おんなのひと	
\\	~日	~か	
\\	~科	~か	
\\	~歌	~か	
\\	~画	~が	
\\	外~	がい~	
\\	~外	~がい	
\\	開会	かいかい	
\\	会館	かいかん	
\\	改札	かいさつ	
\\	解散	かいさん	
\\	海水浴	かいすいよく	
\\	回数	かいすう	
\\	回数券	かいすうけん	
\\	改正	かいせい	
\\	快晴	かいせい	
\\	解説	かいせつ	
\\	改造	かいぞう	
\\	〔作り直すこと〕((米)) 
\\	((英)) 
\\	改造する 〔改装する〕
\\	〔用途を変えて作り直す〕
\\	屋根裏を改造して仕事部屋にする 
\\	〔組織などの改変〕 内閣を改造する 
\\	開通	かいつう	開通する 〔新線・道路などが〕
\\	両都市間に鉄道が開通した 
\\	不通区間は午後9時に開通した 
\\	離れ島と本土との間に電話が開通した 
\\	開通式 
\\	回転	かいてん	
\\	〔一点を中心に回ること〕
\\	〔軸を中心に〕
\\	〔自転〕
\\	〔転がり〕
\\	回転する 
\\	〔軸を中心に〕
\\	〔自転する〕
\\	月は地球の周りを回転する 
\\	軸を中心に回転する 
\\	回転しているこま 
\\	この扇風機(せんぷうき)は1秒に何回回転するか 
\\	彼女は一回転してプールに飛び込んだ 
\\	〔頭の働き〕 彼は頭の回転が早い 
\\	/((口)) 
\\	頭の回転が遅い 
\\	〔資金などの〕
\\	〔資本・在庫品などの〕
\\	資本を回転させる 
\\	資金の回転を早める 
\\	この商品は回転が早そうだ 
\\	あの店は客の回転がいい 
\\	/((英)) 
\\	回転椅子(いす) 
\\	回転運動 
\\	回転運動する 
\\	回転儀 
\\	回転競技 〔スキーの〕
\\	大回転競技 
\\	回転式の 
\\	回転式の展望室 
\\	回転資金 
\\	回転軸 
\\	回転数[速度] 
\\	回転ずし 
\\	回転体 
\\	回転ドア 
\\	回転灯 
\\	回転半径 この車の回転半径は5メートルである 
\\	回転窓 
\\	回転面 
\\	回転木馬 
\\	回転翼 〔ヘリコプターの〕
\\	〔送風機などの〕
\\	回転率 〔資産・負債などの〕
\\	〔資金回転率〕
\\	回転炉 〔ロータリー窯〕
\\	解答	かいとう	
\\	回答	かいとう	
\\	外部	がいぶ	
\\	解放	かいほう	
\\	開放	かいほう	
\\	〔開け放すこと〕 開放する 
\\	開放厳禁〔掲示〕 
\\	この家は開放的に出来ている 
\\	〔出入りが自由なこと〕 そのホールは一般に開放されている 
\\	門戸開放政策 
\\	〔明けっ広げ〕 開放的な人 
\\	開放経済 
\\	海洋	かいよう	
\\	概論	がいろん	〔概説〕
\\	〔入門〕
\\	インド哲学概論 
\\	却って	かえって	
\\	家屋	かおく	
\\	係わる	かかわる	
\\	書留	かきとめ	
\\	書取	かきとり	
\\	垣根	かきね	
\\	限り	かぎり	
\\	〔限界〕
\\	買いたい物をあげたら限りがない 
\\	声を限りに叫んだ 
\\	[((英)) 
\\	独裁者は暴虐の限りを尽くした 
\\	力の限り戦おう 
\\	〔最後〕 どうせ限りある身だ 
\\	この日を限りに彼を村で見掛けた者はいなかった 
\\	〔範囲〕 非常の場合はこの限りではない 
\\	かぎり
\\	限り】 
\\	〔限度いっぱい〕 出来る限り早く来なさい 
\\	あらん限りの努力をしたが駄目だった 
\\	うれしい限りです 
\\	〔最後〕 今日限り酒は飲まない 
\\	〔制限〕 申し込みは来月5日限り 
\\	当日限り有効 
\\	その場限り ⇒そのば(其の場) 
\\	〔範囲〕 私の知る限りでは… 
\\	事情の許す[私に関する]限り 
\\	見渡す限り焼野原だった 〔野焼きで〕
\\	/〔大火災で〕
\\	支障のない限り会に出席します 
\\	〔間〕 生きている限りあなたに不自由はさせません 
\\	彼が帰らない限り我々も席を立てない 
\\	〔条件〕 謝らない限り許してやらない 
\\	地域の人々が協力しない限り暴力団追放は難しい 
\\	各~	かく~	
\\	架空	かくう	
\\	〔空中にかけた〕 架空の 
\\	〔高架の〕
\\	〔空想の〕 架空の 
\\	架空の人物 
\\	架空の物語 
\\	架空ケーブル 
\\	架空索道 
\\	架空請求詐欺 
\\	架空装置 
\\	架空電線 
\\	架空取引 
\\	架空名義 
\\	架空名義で貯金する 
\\	各自	かくじ	
\\	拡充	かくじゅう	
\\	拡充する 
\\	生産の拡充 
\\	教育施設(しせつ)を拡充する 
\\	学術	がくじゅつ	
\\	〔学芸〕 学術上の 〔科学的〕
\\	/〔専門的〕
\\	日本学術振興会 
\\	学術振興のために必要な措置(そち)をとる 
\\	〔学問〕
\\	〔専門分野の知識〕
\\	〔個人の研究〕
\\	彼は学術優秀(ゆうしゅう)である 
\\	彼女は学術研究のため渡米(とべい)した 
\\	学術会議 
\\	学術講演 
\\	学術雑誌 
\\	学術書 
\\	学術調査 
\\	学術調査団 
\\	学術用語 
\\	学術論文 
\\	各地	かくち	
\\	拡張	かくちょう	〔幅・範囲などの〕
\\	〔大きさ・量などの〕
\\	〔具体的な物を大きくする〕
\\	拡張する 
\\	軍備拡張 
\\	領土の拡張 
\\	病院を拡張する 
\\	土地が狭くて店舗を拡張できない 
\\	事業を海外に拡張する 
\\	道路を拡張する 
\\	拡張カード 
\\	拡張機能 
\\	そのソフト会社は無料でダウンロードできる拡張機能を提供している 
\\	拡張子 
\\	拡張メモリー 
\\	角度	かくど	
\\	〔角の大きさ〕
\\	線と線が交(まじ)わったときに出来る角度 
\\	道は30度の角度で曲がっている 
\\	〔観点〕 違った角度から考えてみなさい 
\\	学年	がくねん	
\\	格別	かくべつ	
\\	確率	かくりつ	
\\	学力	がくりょく	
\\	掛け算	かけざん	
\\	可決	かけつ	
\\	火口	かこう	
\\	下降	かこう	
\\	火山	かざん	
\\	かしこまりました	かしこまりました	
\\	〔正座する〕
\\	〔堅苦しい態度を取る〕
\\	そんなにかしこまらないでください 〔気楽になさい〕
\\	/〔堅苦しくしないでください〕
\\	彼はかしこまって座っていた 
\\	かしこまった態度で口上を述べた 
\\	〔承る〕 はい,かしこまりました 
\\	貸し出し	かしだし	
\\	過失	かしつ	
\\	業務上過失 
\\	大きな[小さな]過失を犯(おか)す 
\\	彼は自分の過失だと認めた 
\\	その殺人は過失か故意か決め手がない 
\\	過失傷害(罪) 
\\	過失致死(罪) 
\\	業務上過失致死の疑いで逮捕された 
\\	過失犯 
\\	果実	かじつ	
\\	貸間	かしま	
\\	貸家	かしや	
\\	箇所	かしょ	〔場所〕
\\	〔点〕
\\	〔部分〕
\\	〔文章の〕
\\	この登山道には危険な箇所が3つある 
\\	全過程の中でここが一番難しい箇所だ 
\\	その箇所を読み直した 
\\	過剰	かじょう	
\\	かじる	かじる	
\\	課税	かぜい	
\\	下線	かせん	
\\	加速	かそく	
\\	加速度	かそくど	
\\	塊	かたまり	
\\	固まる	かたまる	
\\	片道	かたみち	
\\	傾く	かたむく	
\\	片寄る	かたよる	
\\	~がち	~がち	
\\	学科	がっか	
\\	学会	がっかい	
\\	学級	がっきゅう	
\\	担ぐ	かつぐ	
\\	〔背負う〕
\\	鉄砲を担いで行進する 
\\	〔祭り上げる〕 山田氏を会長に担ぐことにした 
\\	労働組合に担がれて国会議員に立候補した 
\\	〔だます〕
\\	まんまと担がれた 
\\	〔縁起を気にする〕 そんなに担ぐことはない 
\\	括弧	かっこ	
\\	活字	かつじ	
\\	勝手	かって	
\\	(台所)
\\	(様子・事情)
\\	(都合)
\\	〔台所〕
\\	〔暮らし向き〕 勝手が苦しい 
\\	〔具合,都合〕
\\	勝手の良い 
\\	勝手の悪い部屋 
\\	いつも自分の勝手をはかっている 
\\	〔様子〕 ここはどうも勝手が違う 
\\	今度の仕事は勝手が違っている 
\\	勝手知った他人の家だ 
\\	新入生は勝手が分からずうろうろしていた 
\\	この辺は勝手を知りません 
\\	〔気まま,自由〕 勝手気ままな人 
\\	彼は何かにつけ勝手なまねをしたがる 
\\	彼の勝手にさせておきなさい 
\\	それは私の勝手にはなりません 
\\	彼が勝手に決めたことだ 
\\	勝手に他人の所有地(しょゆうち)に入ってはいけない 
\\	この辞書は勝手にお使いください 
\\	どこへ行こうが自分の勝手だ 
\\	〔掲示〕勝手ながら本日は休業します 
\\	勝手にしろ 〔好きなようにしろ〕
\\	/〔ちくしょう〕
\\	乱暴,野卑な表現) 勝手口 
\\	〔裏口〕
\\	勝手仕事 
\\	勝手次第 
\\	それはおまえの勝手次第だ 
\\	勝手道具 
\\	勝手向き ⇒詳細を見る
\\	活力	かつりょく	
\\	仮名	かな	
\\	仮名遣い	かなづかい	
\\	加熱	かねつ	
\\	兼ねる	かねる	〔両方に役に立つ〕
\\	〔両方である〕
\\	仕事と遊びを兼ねる 
\\	この箱は机と本棚を兼ねている 
\\	彼が作曲と作詞を兼ねた 
\\	大は小を兼ねる ⇒だい(大) 〔…できない〕
\\	〔…するのが難しい〕
\\	〔…する立場・境遇にない〕
\\	〔ためらう〕
\\	それは私には分かり兼ねます 〔理解できない〕
\\	/〔知らない〕
\\	それはなんとも申し上げ兼ねます 
\\	申し上げ兼ねますが思い違いをしておられます 
\\	彼らは吉報を待ち兼ねていた 
\\	過半数	かはんすう	
\\	かび
\\	黴】	かび	
\\	〔総称〕((米)) 
\\	((英)) 
\\	白かび 
\\	青かび 
\\	かびの生えたパン 
\\	チーズにかびが生えた 
\\	本にかびが生えた 
\\	〔古臭さ〕 かびの生えた考え方 
\\	かび防止剤 
\\	被せる	かぶせる	
\\	〔上から覆う〕
\\	なべにふたをかぶせる 
\\	歯に金冠をかぶせる 
\\	魔術師は箱に布をかぶせた 
\\	〔上から注ぐ〕⇒かける(掛ける)
\\	〔責任・罪などを負わせる〕 罪を他人にかぶせる 
\\	自分のしたことの責任を人にかぶせるなんてひどい 
\\	釜	かま	
\\	構いません	かまいません	
\\	紙屑	かみくず	
\\	神様	かみさま	
\\	剃刀	かみそり	
\\	貨物	かもつ	
\\	からかう	からかう	
\\	〔悪気なく〕
\\	古めかしい言葉); 〔ふざけて〕
\\	((口)) 
\\	〔気さくに〕((口)) 
\\	彼がのろいからといってからかってはいけない 
\\	若い女の子をからかっては喜んでいる 
\\	彼はからかわれているのに気づかなかった 
\\	からかい半分に言っただけだ 
\\	からかわないでくれ ((口)) 
\\	かるた	かるた	
\\	枯れる	かれる	
\\	〔草木が〕
\\	枯れた葉[花] 
\\	この松(まつ)の木は枯れている 
\\	鉢植(はちう)えの植物がすべて枯れた 
\\	〔材木が乾燥している〕⇒かわいた(乾いた) 
\\	〔芸など深み,渋みが出る〕 彼の芸は枯れてきた 
\\	可愛がる	かわいがる	
\\	為替	かわせ	外国[国内]為替 
\\	長期[短期]為替 
\\	売[買]為替 
\\	不渡り為替手形 
\\	円[ドル/ポンド]為替 
\\	為替 (レート) の騰貴[下落] 
\\	郵便為替 ((米))
\\	((略
\\	((英))
\\	((略
\\	小為替 ((米))
\\	/((英))
\\	為替を現金に替えた 
\\	000円為替で送金します 
\\	為替受け取り人 
\\	為替管理 
\\	為替管理法 
\\	為替裁定 
\\	為替差益[差損] 
\\	為替相場 ⇒かわせレート(為替レート) 為替ダンピング 
\\	為替手形 ⇒詳細を見る 為替投機 
\\	為替振り出し人 
\\	為替ブローカー 
\\	為替変動 
\\	為替リスク 
\\	この投資信託には為替リスクが伴う 
\\	為替レート ⇒詳細を見る
\\	瓦	かわら	
\\	~刊	~かん	
\\	~間	~かん	
\\	~巻	~かん	
\\	~館	~かん	
\\	~感	~かん	
\\	換気	かんき	
\\	感激	かんげき	
\\	関西	かんさい	
\\	元日	がんじつ	
\\	鑑賞	かんしょう	鑑賞する 〔楽しむ〕
\\	〔真価を認めて味わう〕
\\	音楽鑑賞 
\\	詩文を鑑賞する 
\\	趣味は音楽[映画]鑑賞です 
\\	絵画の鑑賞力がある 
\\	感ずる	かんずる	
\\	間接	かんせつ	
\\	観測	かんそく	
\\	寒帯	かんたい	
\\	官庁	かんちょう	〔役所〕
\\	〔当局〕
\\	行政官庁 
\\	当該[関係]官庁 
\\	官庁街 
\\	官庁公示事項 
\\	勘違い	かんちがい	
\\	缶詰	かんづめ	
\\	乾電池	かんでんち	
\\	関東	かんとう	
\\	観念	かんねん	
\\	〔ある事に関する考え〕
\\	平等の観念 
\\	それは誤った観念だ 
\\	彼女はアメリカ人に関して固定観念を抱いていた 
\\	戦争の悲惨さは観念として知っているだけだ 
\\	この際,観念的でなく実際的な計画を立てるべきだ 
\\	〔物事についての意識〕
\\	彼には時間[距離]の観念がない 
\\	〔あきらめ〕
\\	〔覚悟〕
\\	死を観念する 
\\	観念のほぞを固める 
\\	病気は治らないものと観念した 
\\	犯人は観念してピストルを投げ捨てた 
\\	観念論 ⇒詳細を見る
\\	乾杯	かんぱい	
\\	看板	かんばん	
\\	看病	かんびょう	
\\	冠	かんむり	
\\	漢和	かんわ	
\\	~期	~き	
\\	~器	~き	
\\	~機	~き	
\\	気圧	きあつ	
\\	着替え	きがえ	
\\	着替える	きがえる	
\\	機関車	きかんしゃ	
\\	飢饉	ききん	
\\	器具	きぐ	
\\	〔道具〕
\\	台所用器具 
\\	測定用[精密]器具 
\\	〔簡単な器械〕⇒きき(機器) 消毒器具 
\\	照明器具 
\\	ガス器具 
\\	記号	きごう	
\\	記号を付ける 
\\	の記号 
\\	発音記号 
\\	化学記号 
\\	ハ音記号〔音楽〕 
\\	これは何を表す記号ですか 
\\	その割合を記号
\\	で表す 
\\	記号化 
\\	記号論 
\\	記号論理学 
\\	刻む	きざむ	
\\	〔肉などを〕
\\	のほうが細かい); 〔野菜を〕
\\	にんじんを細かく刻む 
\\	〔刻み目をつける〕
\\	⇒きざみ(刻み); 〔彫る〕
\\	〔表面に彫る〕
\\	子供の成長を記録するために柱に印を刻んだ 
\\	石仏を刻む 
\\	彼の墓石には彼の詩が1行刻まれている 
\\	彼の顔には長年の苦悩を表すしわが刻まれていた 
\\	〔時を刻む〕
\\	時計が時を刻んでいる 
\\	〔心に〕 その光景は彼の心に深く刻まれた 
\\	儀式	ぎしき	
\\	基準	きじゅん	
\\	規準	きじゅん	
\\	起床	きしょう	
\\	着せる	きせる	
\\	〔衣装をつけさせる〕 子供に服を着せる 
\\	手伝って娘に着物を着せた 
\\	彼は妻に毛皮のコートを着せてやれたらなあと思った 
\\	お年寄に上着を着せてあげた 
\\	〔かぶせる〕 うたた寝している子に毛布を着せてやった 
\\	うたた寝 
\\	そのメダルには金が着せてある 
\\	〔人に責任を負わせる〕 人に罪を着せる 
\\	彼は汚名(おめい)を着せられた 
\\	彼は何かにつけ私に恩(おん)を着せた 
\\	恩を着せる 
\\	基礎	きそ	
\\	〔建築物の土台〕
\\	建物の基礎を築く 
\\	この建物の基礎はしっかりしている 
\\	〔根底となるもの〕
\\	((複
\\	基礎的,基礎の 
\\	私は英語を基礎からやり直すつもりです 
\\	数学の基礎がしっかりできている 
\\	アンケートを基礎として新しい計画をたてた 
\\	その理論は最近の実験の結果を基礎としている 
\\	その事件は我が社を基礎から揺るがすおそれがある 
\\	彼の音楽家としての基礎はヨーロッパ留学中に築かれた 
\\	これらの作品の基礎となっているのは作者の人間愛である 
\\	基礎英語講座 
\\	彼は化学の基礎知識がある 
\\	物理学の基礎的原理 
\\	基礎学科 
\\	基礎工事 
\\	基礎控除 
\\	基礎産業 
\\	基礎体温 
\\	((略
\\	基礎代謝 
\\	基礎年金 
\\	基地	きち	
\\	きっかけ	きっかけ	〔初め〕
\\	〔機会〕
\\	〔手掛かり〕
\\	話の切っ掛けをつかむ 
\\	それが事件の切っ掛けとなった 
\\	これを切っ掛けに末長くお付き合いしましょう 
\\	警察は事件解決の切っ掛けがつかめない 
\\	ぎっしり	ぎっしり	
\\	ぎっしりと書かれた手紙 
\\	トランクにぎっしり詰め込む 
\\	我々は一室にぎっしりと詰め込まれた 
\\	ホールには若い人がぎっしり詰まっていた 
\\	彼の日程はぎっしり詰まっている 
\\	基盤	きばん	
\\	((複
\\	キリスト教が彼の精神的基盤を成している 
\\	確固とした基盤 
\\	彼の支持基盤は農村部だ 
\\	反戦運動の基盤を育てる 
\\	基盤産業 
\\	~気味	~ぎみ	
\\	客席	きゃくせき	
\\	客間	きゃくま	
\\	休業	きゅうぎょう	
\\	休講	きゅうこう	
\\	給与	きゅうよ	
\\	休養	きゅうよう	
\\	清い	きよい	
\\	〔濁りがない〕
\\	清い水 
\\	清い空気 
\\	〔純粋な〕
\\	清い心 
\\	清い選挙 
\\	清き一票を投じましょう 
\\	清い乙女 
\\	彼らの愛情は清いものだ 
\\	清く[清らかに]暮らす 
\\	〔潔い〕 彼はわだかまりを清く水に流した 
\\	~教	~きょう	
\\	~行	~ぎょう	
\\	~業	~ぎょう	
\\	強化	きょうか	
\\	境界	きょうかい	
\\	共産~	きょうさん~	
\\	行事	ぎょうじ	
\\	〔宗教的・社会的儀式〕
\\	年中行事 
\\	春の主な行事 
\\	学校行事 
\\	宮中の行事 
\\	1学期の行事予定表 
\\	恐縮	きょうしゅく	恐縮する 〔感謝する〕
\\	〔すまないと思う〕
\\	〔恥じ入る〕
\\	ご親切にしていただいて恐縮です 
\\	お褒めいただき恐縮です 
\\	恐縮ですがこのかばんを見ていてくださいませんか 
\\	恐縮ですがお名前をおっしゃってください 
\\	教養	きょうよう	
\\	教養のある 
\\	教養のない 
\\	教養を身に付ける 
\\	教養を高める 
\\	教養学部 
\\	教養課程 
\\	教養科目 
\\	教養小説 
\\	教養番組 
\\	行列	ぎょうれつ	
\\	〔行進〕
\\	〔示威的な〕
\\	行列する 
\\	仮装行列 
\\	ちょうちん[旗]行列 
\\	大名行列 
\\	彼女は行列の先頭に立った 
\\	彼らは大通りを行列した 
\\	〔並んだ列〕
\\	((英)) 
\\	3時間行列してやっと切符を手に入れた 
\\	行列に割り込む 
\\	店の前に人々が行列を作った 
\\	〔数学で〕
\\	((複-
\\	行列式 〔数学で〕
\\	漁業	ぎょぎょう	
\\	曲線	きょくせん	
\\	規律	きりつ	
\\	〔規律〕
\\	⇒きそく(規則) 規律を守る[破る] 
\\	〔秩序〕
\\	〔鍛練による規律正しさ〕
\\	規律正しく 
\\	規律正しい生活をする 
\\	この学校は規律がいいかげんだ 
\\	~切れ	~きれ	
\\	時間切れで作文が書き切れなかった 
\\	気を付ける	きをつける	
\\	金魚	きんぎょ	
\\	区域	くいき	
\\	空~	くう~	
\\	偶数	ぐうすう	
\\	空想	くうそう	
\\	空中	くうちゅう	
\\	釘	くぎ	
\\	区切る	くぎる	
\\	文を句読点(くとうてん)で区切る 
\\	従属節(じゅうぞくせつ)はコンマで区切られている 
\\	言葉を区切って話した 
\\	その村は川で二つに区切られている 
\\	櫛	くし	
\\	くしゃみ	くしゃみ	
\\	苦情	くじょう	
\\	苦心	くしん	
\\	常に複数形); 
\\	苦心する 
\\	苦心してそれを調べあげた 
\\	この本は彼の苦心が実ったものである 
\\	せっかくの苦心も水の泡になった 
\\	難問を解こうと苦心している 
\\	この筆使いに苦心の跡が見られる 
\\	この辞書は大勢の人の苦心の作である 
\\	苦心惨憺(さんたん) この金を工面するのに苦心さんたんした 
\\	苦心談 それを発明するまでの苦心談を語った 
\\	屑	くず	
\\	崩す	くずす	
\\	札を崩す, 
\\	字を崩す 
\\	顔を崩す, 
\\	〔砕き壊す〕
\\	((やや文)) 
\\	山を崩す 
\\	石垣(いしがき)を崩す 
\\	〔整ったものを乱す〕 敵陣(てきじん)を崩す 
\\	どうぞひざをお崩しください 
\\	孫を抱き上げながら相好(そうごう)を崩した 
\\	相好を崩す 
\\	姿勢(しせい)を崩さずに立っていた 
\\	〔両替する〕 この千円札を100円玉に崩してください 
\\	週刊誌を買って1万円を崩した 
\\	〔崩し書きにする〕 字を崩して書く 
\\	薬指	くすりゆび	
\\	崩れる	くずれる	
\\	砕く	くだく	
\\	砕ける	くだける	
\\	くたびれる	くたびれる	
\\	〔疲れる〕
\\	ああ,くたびれた 
\\	歩き回ってくたびれた 
\\	彼女と話をしているとくたびれる 
\\	彼のおしゃべりにすっかりくたびれた 
\\	タクシーを待ちくたびれた 
\\	私はもう生きていくのにくたびれた 
\\	人生にくたびれた顔付き 
\\	〔長く使って形が崩れる〕
\\	彼はいつもくたびれた服を着ていた 
\\	くだらない	くだらない	
\\	唇	くちびる	
\\	口紅	くちべに	
\\	くっつく	くっつく	
\\	〔密着する〕
\\	〔くっつき合う〕
\\	靴の底にチューインガムがくっついた 
\\	静電気でスカートが脚にくっつく 
\\	2枚の板は接着剤でぴたりとくっついていた 
\\	この糊はよくくっつかない 
\\	〔そばにいる〕 母にくっついて買物に行った 
\\	そうくっついて歩くな 
\\	〔男女が付き合う〕
\\	〔同棲する〕
\\	〔結婚する〕
\\	彼女はいつのまにか彼とくっついていた 
\\	くっつける	くっつける	
\\	くどい	くどい	
\\	〔言葉数の多い〕
\\	〔長ったらしい〕
\\	〔冗長な〕
\\	彼の話はいつもくどい 
\\	彼の文はくどい 
\\	/((文)) 
\\	彼の説明はくどくてうんざりだ 
\\	くどいようだがそれは重要なことなんだ 
\\	君はくどいよ 
\\	〔味・色などがしつこい〕 この料理はくどくて私の口に合わない 
\\	そのカーテンはちょっと色がくどい 
\\	句読点	くとうてん	
\\	配る	くばる	
\\	工夫	くふう	〔仕掛け〕
\\	〔発明〕
\\	〔手段〕
\\	通例複数形) 工夫する 
\\	使いやすいようにその辞書にいろいろと工夫を凝らした 
\\	これは私が工夫したのです 
\\	工夫の才に富んだ人です 
\\	何とか工夫してみましょう 
\\	区分	くぶん	
\\	組合せ	くみあわせ	
\\	組み立てる	くみたてる	
\\	彼は模型(もけい)飛行機を組み立てた 
\\	悔やむ	くやむ	
\\	くるむ	くるむ	
\\	赤ん坊をおくるみにくるむ 
\\	くれぐれも	くれぐれも	くれぐれもご主人によろしく 
\\	くれぐれもご自愛のほどを 
\\	~家	~け	
\\	~形	~けい	
\\	~系	~けい	
\\	稽古	けいこ	
\\	敬語	けいご	
\\	蛍光灯	けいこうとう	
\\	形式	けいしき	
\\	〔型,様式〕
\\	形式の 
\\	この書類は実に複雑な形式である 
\\	彼の報告書は形式にとらわれていて内容がなかった 
\\	注意書きは質疑応答の形式で書いてある 
\\	その祭は昔ながらの形式で進められた 
\\	〔うわべだけのこと〕 形式的 (に) 
\\	彼からは形式的な転勤あいさつ状が1通来たきりだ 
\\	形式的ですが一応ここにサインをしてください 
\\	形式ばった挨拶 
\\	形式主義 
\\	継続	けいぞく	
\\	毛糸	けいと	
\\	経度	けいど	
\\	系統	けいとう	
\\	〔秩序だったつながり〕
\\	系統的 (に) 
\\	系統づける 
\\	神経[筋肉]系統 
\\	バスの運行系統 
\\	命令系統 
\\	系統を立てて話を進める 
\\	近代科学を系統的に研究する 
\\	〔系譜〕
\\	〔流派〕
\\	〔党派〕
\\	アリストテレスの系統を受け継ぐ学派 
\\	このことばはインド・アーリア語系統のことばだ 
\\	茶系統にもいろいろな色がある 
\\	系統学 
\\	系統樹 〔言語の〕
\\	芸能	げいのう	
\\	競馬	けいば	
\\	警備	けいび	
\\	((英)) 
\\	警備する 
\\	暴徒に対する警備の厳重な大使館 
\\	警備は間もなく解かれた 
\\	警備員 
\\	〔夜間の〕
\\	〔巡回する〕
\\	警備会社 
\\	警備隊 
\\	沿岸(えんがん)警備隊 
\\	警備艇 
\\	形容詞	けいようし	
\\	形容動詞	けいようどうし	
\\	外科	げか	
\\	毛皮	けがわ	
\\	激増	げきぞう	
\\	下車	げしゃ	
\\	下旬	げじゅん	
\\	下水	げすい	
\\	削る	けずる	
\\	〔薄くそぎ取る〕
\\	〔削って平らにする〕
\\	板を削る 
\\	川の水が両岸を削った 
\\	木の皮を削り取る 
\\	盛り土の上を削る 
\\	鉛筆を削る 
\\	〔除く〕
\\	〔消す〕
\\	〔線で消す〕
\\	⇒けす(消す)3, 
\\	〔減らす〕
\\	福祉予算が大幅に削られた 
\\	給料を大幅に削られた 
\\	けた	けた	
\\	〔建物の〕
\\	〔橋などの〕
\\	〔そろばんの〕
\\	〔数の位〕
\\	けたの数 
\\	小数点以下2けたまで計算する 
\\	計算を1けた間違えた 
\\	その建物の建築費は10けたに達する 
\\	桁が違う 彼は弟とは人物のけたが違う 
\\	⇒けたちがい(桁違い)
\\	下駄	げた	
\\	血圧	けつあつ	
\\	月給	げっきゅう	
\\	傑作	けっさく	
\\	月末	げつまつ	
\\	気配	けはい	
\\	隣室に人のいる気配がした 
\\	その事について話し合った気配がなかった 
\\	物価が上がりそうな気配が濃い 
\\	下品	げひん	
\\	煙い	けむい	
\\	険しい	けわしい	
\\	~圏	~けん	
\\	〔ブロック〕
\\	人の勢力圏 
\\	共産圏 
\\	ポンド[ドル]圏 
\\	首都圏 
\\	南極圏 
\\	現~	げん~	
\\	見学	けんがく	
\\	謙虚	けんきょ	
\\	原稿	げんこう	
\\	原産	げんさん	
\\	原始	げんし	
\\	研修	けんしゅう	
\\	厳重	げんじゅう	
\\	謙遜	けんそん	謙遜する 〔控え目にする〕
\\	〔へりくだる〕
\\	謙遜な 
\\	謙遜して物を言う 
\\	言葉遣い[態度]が謙遜だ 
\\	彼は謙遜して何も言わなかった 
\\	謙遜家 
\\	謙遜語 ⇒けんじょう(謙譲)
\\	県庁	けんちょう	
\\	限度	げんど	
\\	現に	げんに	
\\	顕微鏡	けんびきょう	
\\	懸命な	けんめいな	
\\	救助隊(きゅうじょたい)の懸命な努力のおかげで 
\\	⇒いっしょ(う)けんめい(一生懸命・一所懸命)
\\	原理	げんり	
\\	原料	げんりょう	
\\	小~	こ~	
\\	恋しい	こいしい	私の恋しい人 
\\	あの人が恋しい〔そばにいないので〕 
\\	故郷が恋しい 
\\	寒くなると火が恋しい 
\\	~校	~こう	
\\	~港	~こう	
\\	~号	~ごう	
\\	工員	こういん	
\\	強引	ごういん	
\\	公害	こうがい	
\\	高級	こうきゅう	
\\	公共	こうきょう	
\\	工芸	こうげい	
\\	美術工芸 
\\	工芸家 
\\	工芸美術 
\\	工芸品 
\\	〔手工芸品〕
\\	孝行	こうこう	
\\	交差	こうさ	
\\	講師	こうし	
\\	工事	こうじ	
\\	公式	こうしき	
\\	〔数学で〕
\\	((複 〜
\\	公式で表す 
\\	〔儀式〕
\\	公式の (に) 〔公の〕
\\	公式訪問する 
\\	政府は首相の暗殺計画があったことを公式に認めた 
\\	その問題についての公式見解が間もなく発表されるでしょう 
\\	彼の引退はまだ公式発表されていない 
\\	公式主義 
\\	公式戦 
\\	〔野球で〕
\\	口実	こうじつ	
\\	私は母の病気を口実にして学校を欠席した 
\\	うまい[下手な]口実 
\\	こうして	こうして	私はこうして会社再建に成功した 
\\	[((文))
\\	こうして結べばほどけない 
\\	公衆	こうしゅう	
\\	香水	こうすい	
\\	功績	こうせき	
\\	光線	こうせん	
\\	高層	こうそう	
\\	構造	こうぞう	
\\	交替	こうたい	〔入れ替わり〕
\\	〔職務や任務の〕
\\	〔勤務の時間分割〕
\\	世代の交替 
\\	/〔比ゆ的に〕
\\	昼夜交替で 
\\	その工場は3部交代制である 
\\	彼らは4時間交替で働く 
\\	交替で見張る 
\\	妹と交替で部屋の掃除をする 
\\	監督は投手を交替した 
\\	私は彼と交替した 
\\	総会で役員の交替が行われた 
\\	交替員 彼は夜間の交替員だ 
\\	交替組 〔昼の〕
\\	〔夜の〕
\\	交替時間 交替時間は何時ですか 
\\	交替制 
\\	ここは (三) 交替制になっている 
\\	耕地	こうち	
\\	交通機関	こうつうきかん	
\\	校庭	こうてい	
\\	肯定	こうてい	
\\	肯定する 〔そうだと言う〕
\\	〔認める〕
\\	肯定の[肯定的] 
\\	彼らは肯定も否定もしなかった 
\\	彼はその批判を肯定した 
\\	彼は人生を肯定している 
\\	肯定文 〔文法で〕
\\	高度	こうど	
\\	高等	こうとう	
\\	合同	ごうどう	
\\	〔結合〕
\\	〔合一,統一〕
\\	合同する 〔一つにまとまる〕
\\	〔一体になる,統一する〕
\\	合同で 
\\	野党が合同して知事選に臨(のぞ)んだ 
\\	二国の軍隊が合同して演習した 
\\	〔数学で〕
\\	合同委員会 
\\	合同慰霊祭 
\\	合同演習 
\\	合同会議 
\\	合同会社説明会 
\\	合同三角形 
\\	合同事業 
\\	合同葬 
\\	合同練習 
\\	高等学校	こうとうがっこう	
\\	公表	こうひょう	
\\	鉱物	こうぶつ	
\\	公務	こうむ	
\\	項目	こうもく	〔題目〕
\\	〔表や計算書などの細目〕
\\	〔条項〕
\\	項目に分ける 
\\	内容を項目ごとに検討する 
\\	内容の項目別表 
\\	論文を5項目に分けた 
\\	爬虫(はちゅう)類はこの項目に入る 
\\	合理	ごうり	
\\	交流	こうりゅう	
\\	〔互いに混じること〕
\\	地方自治体間の人事の交流 
\\	2国間の文化の交流が活発になった 
\\	〔電流の〕
\\	((略
\\	交流発電機 
\\	合流	ごうりゅう	
\\	効力	こうりょく	
\\	焦がす	こがす	
\\	~国	~こく	
\\	国王	こくおう	
\\	国立	こくりつ	
\\	ご苦労様	ごくろうさま	
\\	焦げる	こげる	
\\	凍える	こごえる	
\\	心当たり	こころあたり	
\\	心当たりの人みんなに電話して彼の居所を聞いた 
\\	彼がどこへ行ったか心当たりがある[ない] 
\\	彼に良い就職口の心当たりはありませんか 
\\	心得る	こころえる	
\\	〔理解する〕
\\	〔みなす〕
\\	〔気付く〕
\\	茶道を少しばかり心得ております 
\\	それを自分の義務と心得ている 
\\	私をだれだと心得ているのか 
\\	あなたを師と心得ております 
\\	資金が乏しくなったことは心得ている 
\\	〔理解した上で引き受ける〕 心得ました 
\\	/〔敬語〕
\\	腰掛け	こしかけ	
\\	腰掛ける	こしかける	
\\	五十音	ごじゅうおん	
\\	こしらえる	
\\	拵える】	
\\	〔作る〕
\\	コートをこしらえさせた 
\\	〔準備する,調える〕 お昼を3人分こしらえてくれ 
\\	来週までに500万円こしらえなくてはならない 
\\	〔身を飾る〕
\\	妹は顔をこしらえているところです 
\\	〔偽り作る〕
\\	彼は遅刻の言い訳をこしらえた 
\\	〔持つ〕 男をこしらえたらしい 
\\	こする	こする	
\\	〔ごしごしと〕
\\	〔こすり取る〕
\\	こすり落とす 
\\	手をこすり合わせる 
\\	目をこする 
\\	馬をこする 
\\	いくらこすっても跡が消えない 
\\	床をこすって洗う 
\\	靴の泥をこすって取る 
\\	このドアは床をこする 
\\	ペンが紙にこすれる音 
\\	間違いを消そうとこすっているうちに紙に穴が開いてしまった 
\\	フライパンをこすって磨(みが)く 
\\	個体	こたい	
\\	ごちそうさま	ごちそうさま	
\\	こちらこそ	こちらこそ	
\\	小遣い	こづかい	
\\	こっそり	こっそり	〔忍びやかに〕
\\	〔秘密に〕
\\	こっそりと鹿(しか)に近付いた 
\\	彼らはその計画をこっそりと練っていた 
\\	彼は花瓶をこっそりと盗み出した 
\\	こっそりと部屋から抜け出そう 
\\	彼の行状をこっそり調べてみた 
\\	[((口))
\\	は不当なやり方で,陰険な感じがある) 二人をこっそり盗み見した 
\\	古典	こてん	
\\	~毎	~ごと	
\\	〜 (まる) ごと	〜 (まる) ごと	
\\	言付ける	ことづける	〔伝言する〕
\\	〔物を〕
\\	遅く帰ると言付けた 
\\	原稿をお届けするよう助手に言付けました 
\\	言葉遣い	ことばづかい	
\\	こないだ	こないだ	
\\	このあいだ 此間
\\	御無沙汰	ごぶさた	
\\	長らくごぶさたしております 〔音信〕
\\	/〔訪問〕
\\	ごぶさたしてすみません 
\\	御免	ごめん	
\\	ごめんください	ごめんください	
\\	小指	こゆび	
\\	こらえる	こらえる	
\\	〔我慢する〕
\\	⇒たえる(堪える) 痛いだろうがちょっとこらえて! 
\\	この騒音はこらえられない 
\\	彼の無礼(ぶれい)はこらえられない 
\\	〔抑えて外に出さない〕 あくびをこらえる 
\\	涙をこらえる 
\\	怒りをこらえる 
\\	小便をこらえていた 
\\	娯楽	ごらく	
\\	御覧	ごらん	
\\	〔見る〕
\\	ご覧 
\\	ご覧の通り元気になりました 
\\	〔…してみなさい〕 もう一度やってごらん 
\\	もう一度言えるものなら言ってごらんよ 
\\	〔私の言った通りでしょう〕 それご覧 
\\	転がす	ころがす	
\\	転がる	ころがる	
\\	紺	こん	
\\	混合	こんごう	
\\	混合する 
\\	混合した 
\\	砂と小石を混合する 
\\	水と油は混合しない 
\\	ウイスキーを混合する 
\\	混合機 
\\	混合競争[競泳] 
\\	混合経済 
\\	混合色 
\\	混合ダブルス 
\\	混合肥料 
\\	混合物 
\\	〔酒・コーヒーなど〕
\\	献立	こんだて	
\\	〔料理の〕
\\	献立を作る 
\\	献立を見る 
\\	夕食の献立は何ですか 〔家庭で〕
\\	/〔レストランで〕
\\	〔準備,手配〕
\\	献立どおりにことが運んだ 
\\	献立表 
\\	こんばんは	こんばんは	
\\	再~	さい~	
\\	在学	ざいがく	
\\	再三	さいさん	
\\	祭日	さいじつ	
\\	催促	さいそく	
\\	催促する 
\\	彼はしつこく家賃の催促をした 
\\	[((口))
\\	彼は借金で矢の催促を受けた 
\\	催促がましいのですが,いつかお貸しした本を返していただけますか 
\\	彼に返事の催促を書いて出した 
\\	催促状 
\\	採点	さいてん	
\\	((米)) 
\\	採点する 
\\	((米)) 
\\	彼は採点が厳しい[あまい] 
\\	入試の採点をする 
\\	100点満点で採点する 
\\	災難	さいなん	
\\	裁縫	さいほう	
\\	材木	ざいもく	
\\	逆さ	さかさ	
\\	逆様	さかさま	
\\	捜す	さがす	
\\	遡る	さかのぼる	
\\	〔川・流れを〕
\\	〔舟で〕
\\	〔過去や根源に立ち返る〕 この祭りは中世にさかのぼる 
\\	今から100年前にさかのぼってみよう 
\\	この法律は10月にさかのぼって適用される 
\\	根源にさかのぼって調べる 
\\	酒場	さかば	
\\	一昨昨日	さきおととい	
\\	先程	さきほど	
\\	索引	さくいん	
\\	((複 〜
\\	本に索引を付ける 
\\	索引を作る 
\\	索引カード 
\\	作者	さくしゃ	
\\	削除	さくじょ	
\\	作成	さくせい	
\\	作製	さくせい	
\\	探る	さぐる	
\\	〔手足などで捜す〕
\\	小銭がないかとポケットを探った 
\\	〔調べる〕
\\	事件の原因を探る 
\\	彼の経歴を探ってみよう 
\\	敵情を探る必要がある 
\\	彼の意中を探ってみたかい 
\\	〔風景などをたずねる〕 武蔵野の秋を探る 
\\	囁く	ささやく	
\\	匙	さじ	
\\	座敷	ざしき	
\\	差し支え	さしつかえ	
\\	差し引き	さしひき	
\\	刺身	さしみ	
\\	さすさす		さす【刺す】 
\\	〔突き刺す〕
\\	〔針などで〕
\\	男ののどを刺した 
\\	指にとげを刺した 
\\	風が刺すように冷たい 
\\	舌を刺すような味 
\\	鼻を刺すような臭い 
\\	胸を刺すような言葉 
\\	〔虫が〕
\\	蚊[はち]に刺された 
\\	〔縫う〕
\\	〔野球で,走者を〕
\\	さす【指す】 
\\	〔指し示す〕
\\	その子は自分の家の方を指した 
\\	時計の針は10時5分を指していた 
\\	〔名を示す〕 だれのことを指して言っているのですか 
\\	先生に指された 
\\	〔目指す〕⇒めざす(目指す) 
\\	〔将棋をする〕 一局指しませんか 
\\	駒(こま)を指す 
\\	もう指す手がない 
\\	〔密告する〕
\\	((口)) 
\\	仲間を指す 
\\	さす【差す・▲注す】 
\\	〔そそぐ〕
\\	コップに水を差す 
\\	父は毎朝鉢植えに水を差している 
\\	この薬を3滴,両眼に差しなさい 
\\	〔加える〕
\\	沸騰しているやかんに水を差す 
\\	〔つける〕
\\	口紅を差す 
\\	芝刈り機に油を注す 
\\	〔かざす〕
\\	傘を差す 
\\	〔さしはさむ〕 さむらいは腰に刀を差していた 
\\	〔舟を動かす〕 棹をさす 
\\	〔酒をすすめる〕 彼に杯を差した 
\\	さす【差す・▲射す】 
\\	〔光などが〕
\\	屋根に朝日が差している 
\\	この部屋はよく日が差す 
\\	障子にだれかの影が差した 
\\	〔表面に出てくる〕 ほおに赤みが差した 
\\	〔気持ちがおこる〕 嫌気が差す ⇒いやけ(嫌気) あの時は魔が差したのだろう 
\\	さす【挿す】 
\\	〔さしはさむ〕
\\	髪にかんざしを挿す 
\\	〔生ける〕
\\	花瓶にばらの花を挿す 
\\	〔挿し木をする〕
\\	椿は梅雨時に挿すとよく根付く 
\\	流石	さすが	
\\	〔いかにも〕 さすがに彼は冷静だった 
\\	さすがは医者の息子だ 
\\	〔とはいうものの〕 彼の切手が欲しくてたまらなかったが,さすがにくれとは言えなかった 
\\	〔あれほどの〕 さすがの頑固親父も子供には勝てなかった 
\\	さすがの大男もその石は持ち上げられなかった 
\\	撮影	さつえい	
\\	雑音	ざつおん	
\\	さっさと	さっさと	〔速く〕
\\	〔活発に〕
\\	さっさと答えなさい 
\\	さっさと歩け 
\\	!/((口)) 
\\	さっさと出て行け 
\\	学校が終わると子供たちはさっさと帰って行った 
\\	仕事をさっさと片づける 
\\	早速	さっそく	
\\	錆	さび	
\\	錆びる	さびる	
\\	座布団	ざぶとん	
\\	妨げる	さまたげる	
\\	〔邪魔をする〕
\\	((文)) 
\\	進歩を妨げる 
\\	飛行機の音で安眠を妨げられた 
\\	彼の出世を妨げてはいけない 
\\	悪天候に妨げられてそこに行けなかった 
\\	〔「妨げない」の形で,差し支えない〕 市長の再選を妨げない 
\\	さようなら	さようなら	
\\	騒がしい	さわがしい	
\\	さわやか	さわやか	
\\	〔すがすがしい様子〕 さわやかな 
\\	さわやかな秋の空気 
\\	は身が引き締まるような) さわやかな気分になる 
\\	〔はっきりしている様子〕 弁舌さわやかな (に) 
\\	弁舌さわやかな人 
\\	~山	~さん	
\\	~産	~さん	
\\	三角	さんかく	
\\	算数	さんすう	
\\	産地	さんち	
\\	〔産物の〕
\\	[((英)) 
\\	ぶどうの産地 
\\	産地直送のりんご 
\\	静岡は茶とみかんの産地として名高い 
\\	〔動物などの〕 羊(ひつじ)の産地 
\\	この動物の産地は南米だ 
\\	山林	さんりん	
\\	~紙	~し	
\\	~寺	~じ	
\\	仕上がる	しあがる	
\\	明明後日	しあさって	
\\	寺院	じいん	
\\	しいんと (する)	しいんと (する)	
\\	自衛	じえい	
\\	塩辛い	しおからい	
\\	司会	しかい	司会する 
\\	((口)) 
\\	田中氏が会の司会をした 
\\	本日の披露宴(ひろうえん)の司会は不肖(ふしょう)土井がつとめさせていただきます 
\\	司会者 〔パーティーなどの〕
\\	((略
\\	〔会議の〕
\\	テレビのクイズの司会者 
\\	四角い	しかくい	
\\	仕方がない	しかたがない	
\\	~時間目	~じかんめ	
\\	時間割	じかんわり	
\\	〜(日本) 式	~(にほん) しき	
\\	敷く	しく	
\\	〔布団などを広げる〕
\\	ござを敷く 
\\	布団を敷く 
\\	〔一面に散らす〕
\\	道に砂利を敷く 
\\	道に石を敷く 
\\	〔敷設する〕
\\	鉄道を敷く 
\\	陣を敷く 
\\	〔上にのる〕
\\	座ぶとんを敷く 
\\	尻に敷く ⇒しり(尻) 
\\	〔行きわたらせる〕 法律を敷く 
\\	村には町制が敷かれた 
\\	茂る	しげる	〔草木が〕
\\	〔役に立たない植物が〕
\\	〔場所が主語で〕
\\	木の茂った山腹(さんぷく) 
\\	よく茂った森 
\\	雑草の生(お)い茂った庭 
\\	葉の茂ったもみじ(紅葉)の木 
\\	庭の木は茂るにまかせた 
\\	紅葉 もみじ 
\\	こうよう 
\\	持参	じさん	持参する 〔持って来る〕
\\	〔持って行く〕
\\	ご注文の本を持参しました 
\\	明日みんな弁当持参だよ 
\\	持参金 
\\	娘の嫁入りに持参金を持たせてやった 
\\	持参人 
\\	持参人払小切手 
\\	磁石	じしゃく	
\\	四捨五入	ししゃごにゅう	
\\	始終	しじゅう	
\\	自習	じしゅう	
\\	静まる	しずまる	
\\	姿勢	しせい	
\\	自然科学	しぜんかがく	
\\	時速	じそく	
\\	子孫	しそん	
\\	死体	したい	
\\	下書き	したがき	
\\	〔草稿〕
\\	〔清書する前の〕
\\	下書きする 
\\	演説[手紙]の下書きをする 
\\	下書きに手を入れる 
\\	〔おおよその形〕
\\	自宅	じたく	
\\	自分の家 (じぶんのいえ))
\\	下町	したまち	
\\	自治	じち	
\\	自治の 
\\	地方自治 
\\	自治能力がある[ない] 
\\	自治会 〔学生の〕
\\	〔町の〕
\\	自治権 
\\	自治省 〔日本の〕
\\	〔英国の〕
\\	自治体[団体] 
\\	地方自治体 
\\	自治大臣 〔日本の〕
\\	自治領 
\\	室~	しつ~	
\\	~室	~しつ	
\\	~日	~じつ	
\\	実感	じっかん	
\\	〔実際の感じ〕
\\	実感を込めて歌う 
\\	温泉町の実感が出た絵だ 
\\	実感の乏しい作品だ 
\\	恐ろしいというのが実感だ 
\\	〔体得〕 大学生の実感がまだわかない 
\\	[((口)) 
\\	/((口)) 
\\	自分の立場の重みを実感した 
\\	湿気	しっき	
\\	〔空気の〕
\\	湿気を含んだ 
\\	/〔空気が〕
\\	湿気を取る 
\\	湿気を吸う 
\\	湿気をおびる 
\\	湿気で金具がさびている 
\\	この電気器具は湿気を嫌う 
\\	しつこい	しつこい	
\\	〔執拗(しつよう)な〕
\\	((文)) 
\\	〔うるさくせがむ〕((文)) 
\\	彼は兄に切手をくれとしつこくせがんだ 
\\	しつこく質問する 
\\	セールスマンがしつこくて困った 
\\	しつこい子だね 
\\	いつまでもしつこく言うなよ 
\\	〔味などが濃厚な〕 しつこい食物 
\\	実習	じっしゅう	
\\	実績	じっせき	
\\	執筆	しっぴつ	
\\	実物	じつぶつ	〔実際の物〕
\\	〔本物〕
\\	〔絵に対し〕
\\	〔写真に対し〕
\\	実物そっくりの肖像画(しょうぞうが) 
\\	彼女の写真の顔は実物と違う 〔実物の方がよい場合〕
\\	/〔悪い場合〕
\\	実物大 
\\	実物大の写真 
\\	実物取引 
\\	しっぽ	しっぽ
\\	尻▲尾】	
\\	⇒お(尾)犬がしっぽを巻いていた
\\	犬がしっぽを振った
\\	⇒はし(端)にんじんのしっぽ
\\	尻尾を出すまんまとしっぽを出した
\\	尻尾をつかむ彼のしっぽをつかんでいるので何だって命令できる
\\	尻尾を振る彼は社長に尻尾を振っている
\\	尻尾を巻く彼はしっぽを巻いて逃げた
\\	実用	じつよう	
\\	しつれいしました (かん)	しつれいしました (かん)	
\\	実例	じつれい	
\\	その説を実例を挙げて説明しなさい 
\\	教育がなくとも成功した実例はいくらもある 
\\	これは私が言おうとしていることのよい実例である 
\\	失恋	しつれん	
\\	指定	してい	指定する 〔特に明示する〕
\\	〔指示する〕
\\	彼は指定の時間より30分も遅れて来た 
\\	お会いする場所と時間を指定してください 
\\	学校指定の書店 
\\	全席指定〔掲示〕 
\\	指定校〔ある目的で指定された学校〕 この小学校が実験指定校になった 
\\	指定席 
\\	〔掲示〕
\\	指定席券 
\\	指定都市 政令指定都市 
\\	指定銘柄 
\\	私鉄	してつ	
\\	縛る	しばる	
\\	〔縄などで結わえる〕
\\	縛って動けなくさせるときには
\\	も
\\	も使えるが,犬をくいに結び付けるようなときには
\\	を用いる) しっかりと縄で縛る 
\\	馬をくいに縛った 
\\	彼らは私の両脚を縛った 
\\	傷口を縛る 
\\	/〔包帯をする〕
\\	〔拘束する〕
\\	人を規則で縛る 
\\	彼は一日中その仕事に縛られていた 
\\	私は時間に縛られるのはごめんだ 
\\	高級レストランで外国式礼儀に縛られて食べたような気がしなかった 
\\	職場では複雑な人間関係に縛られて思い通りにいかない 
\\	地盤	じばん	
\\	〔地面〕
\\	硬い[軟らかい]地盤 
\\	地盤がゆるい 
\\	〔土台〕
\\	地盤を固める 
\\	〔基礎〕
\\	〔成功に通ずる足場〕
\\	将来発展の地盤を築(きづ)いた 
\\	〔政党の後援者たち〕
\\	〔勢力範囲〕
\\	農村を地盤として立候補(りっこうほ)した 
\\	自民党の地盤に食い込んだ 
\\	ーーー 地盤沈下(ちんか) 
\\	〔地面の〕
\\	この地方では地盤沈下が数年続いている 
\\	〔政党の〕
\\	当地では近年共和党の地盤沈下が続いている 
\\	しびれる	しびれる	
\\	〔麻痺(まひ)する〕
\\	〔電気で〕
\\	寒さで手足がしびれた 
\\	畳に行儀よく座るとすぐ足がしびれる 
\\	〔陶酔(とうすい)する〕
\\	強烈(きょうれつ)なサンバの律動(りつどう)にしびれた 
\\	紙幣	しへい	
\\	しぼむ	しぼむ	
\\	〔しなびる〕
\\	花がしぼんだ 
\\	風船がしぼんだ 
\\	〔勢いがなくなる〕 それを見て気持ちがしぼんでしまった 
\\	小説家になりたいという希望は年とともにしぼんでいった 
\\	絞る	しぼる	しぼる【絞る・搾る】 
\\	〔押したりしめつけたりして水気を取る〕
\\	〔ねじって〕
\\	手拭(てぬぐい)を絞る 
\\	レモンの汁を絞る 
\\	牛乳を絞る 
\\	ぶどうの汁を絞ってコップに入れる 
\\	〔広がっているものをまとめる〕 カーテンを絞る 
\\	袋の口を絞る 
\\	〔レンズの絞りを小さくする〕 レンズを絞る 
\\	〔音を小さくする〕 ステレオのボリュームを絞る 
\\	〔範囲を狭める〕 問題の範囲を絞って考えてみよう 
\\	〔無理に出させる〕 智恵を絞る 
\\	声を絞る 
\\	〔強く責める〕 先生に絞られた 
\\	警官に飲酒運転を疑われて絞られた 
\\	〔鍛える〕 昨日は数学[クラブ]で絞られた 
\\	縞	しま	
\\	しみじみ	しみじみ	しみじみ反省する 
\\	彼女の親切をしみじみと感じた 
\\	彼のことをしみじみと思い出した 
\\	当時のことをしみじみと語った 
\\	は「なつかしく」) 父の小言がしみじみと胸にこたえた 
\\	氏名	しめい	
\\	締切	しめきり	
\\	締め切る	しめきる	
\\	〔申し込みなどを〕
\\	10日に応募を締め切る 
\\	〔戸などを〕
\\	戸をぴったりと締め切る 
\\	扉を締め切って会議中です 
\\	しめた (かん)	しめた (かん)	
\\	地面	じめん	
\\	〔土地の表面〕 地面がまだぬれている 
\\	〔土地,地所〕
\\	地面師 
\\	~車	~しゃ	
\\	~者	~しゃ	
\\	~社	~しゃ	
\\	社会科学	しゃかいかがく	
\\	しゃがむ	しゃがむ	
\\	しゃがんだ姿勢で 
\\	しゃがんでバスを待つ 
\\	しゃがんですみれをよく見た 
\\	蛇口	じゃぐち	
\\	弱点	じゃくてん	
\\	車庫	しゃこ	
\\	車掌	しゃしょう	
\\	写生	しゃせい	
\\	社説	しゃせつ	
\\	しゃっくり	しゃっくり	
\\	しゃぶる	しゃぶる	
\\	〔なめる〕
\\	赤ん坊が指をしゃぶっている 
\\	あめをしゃぶる 
\\	〔徹底的に取り尽くす〕 財産はもちろんのこと骨の髄まで彼にしゃぶられた 
\\	車輪	しゃりん	
\\	洒落	しゃれ	
\\	〔冗談〕
\\	〔気の利いた文句〕
\\	((口)) 
\\	〔地口〕
\\	しゃれを飛ばす 
\\	気の利いたしゃれ 
\\	このしゃれが分かるか 
\\	彼はしゃれがうまい 
\\	彼にはしゃれがてんで通じない 
\\	しゃれにもなってない 
\\	〔しゃれていること〕⇒おしゃれ(御洒落) 洒落者 
\\	((口)) 
\\	じゃんけん	じゃんけん	
\\	~手	~しゅ	
\\	~酒	~しゅ	
\\	~集	~しゅう	
\\	重~	じゅう~	
\\	集会	しゅうかい	
\\	集金	しゅうきん	
\\	集合	しゅうごう	
\\	習字	しゅうじ	
\\	修繕	しゅうぜん	
\\	じゅうたん	じゅうたん	
\\	終点	しゅうてん	
\\	重点	じゅうてん	
\\	就任	しゅうにん	
\\	周辺	しゅうへん	
\\	重役	じゅうやく	
\\	終了	しゅうりょう	
\\	重量	じゅうりょう	
\\	重力	じゅうりょく	
\\	熟語	じゅくご	
\\	祝日	しゅくじつ	
\\	縮小	しゅくしょう	
\\	受験	じゅけん	
\\	主語	しゅご	
\\	主人	しゅじん	
\\	出勤	しゅっきん	
\\	述語	じゅつご	
\\	出張	しゅっちょう	
\\	寿命	じゅみょう	
\\	主役	しゅやく	
\\	〔主人公の役〕
\\	〔役者〕
\\	彼はその映画の主役を務めた 
\\	〔主要な役を務める人〕 彼女は清潔な選挙運動の主役であった 
\\	クーデターの主役は空軍将校たちであった 
\\	受話器	じゅわき	
\\	循環	じゅんかん	
\\	循環する 
\\	血液の循環 
\\	血液が体内を循環する 
\\	周期的に循環する 
\\	景気の循環 
\\	物価と賃金は悪循環で上昇している 
\\	⇒あくじゅんかん(悪循環) 循環期 
\\	太陽暦循環期 
\\	循環器 
\\	循環級数 
\\	循環系 
\\	循環小数 〔数学で〕
\\	循環線 
\\	循環バス 
\\	循環論法 
\\	巡査	じゅんさ	
\\	順々	じゅんじゅん	
\\	順序	じゅんじょ	
\\	〔順番〕
\\	順序正しく 
\\	順序を狂わせる 
\\	この名簿は順序が違っている 
\\	順序立てて話す 
\\	順序立った[立たない]説明 
\\	〔手順〕
\\	順序[手続き]を踏む 
\\	まず校長に話すのが順序だろう 
\\	順序不同 〔ただし書きで〕
\\	純情	じゅんじょう	
\\	純粋	じゅんすい	
\\	初~	しょ~	
\\	諸~	しょ~	〔多くの〕
\\	〔いくつかの〕
\\	〔様々な〕
\\	諸外国 
\\	~所	~しょ	
\\	~所	~じょ	
\\	~女	~じょ	
\\	省~	しょう~	
\\	~省	~しょう	
\\	~商	~しょう	
\\	~勝	~しょう	
\\	~条	~じょう	
\\	〔箇条〕 第9条 
\\	〔すじ〕 一条の煙 
\\	一条の光 
\\	~畳	~じょう	
\\	消化	しょうか	
\\	小学生	しょうがくせい	
\\	将棋	しょうぎ	
\\	蒸気	じょうき	
\\	定規	じょうぎ	
\\	上級	じょうきゅう	
\\	商業	しょうぎょう	
\\	消極的	しょうきょくてき	
\\	賞金	しょうきん	
\\	上下	じょうげ	
\\	障子	しょうじ	
\\	商社	しょうしゃ	
\\	乗車	じょうしゃ	乗車する ⇒のる(乗る)2 京都駅で大阪行きの電車に乗車する 
\\	タクシーに乗車する 
\\	ご乗車ください 
\\	乗車拒否 
\\	そのタクシーは乗車拒否をした 
\\	乗車口 
\\	乗車券 
\\	割引乗車券 
\\	乗車券売場 
\\	乗車券自動販売機 
\\	乗車賃 
\\	乗車ホーム 
\\	上旬	じょうじゅん	
\\	小数	しょうすう	
\\	有限[無限]小数 
\\	小数位 
\\	小数点 
\\	小数点以下切り捨てる 
\\	小数点以下を切り上げる 
\\	小数点以下5位まで計算する 
\\	商店	しょうてん	
\\	焦点	しょうてん	
\\	消毒	しょうどく	
\\	勝敗	しょうはい	
\\	蒸発	じょうはつ	
\\	〔気化〕
\\	蒸発する[させる] 
\\	蒸発の 
\\	蒸発しやすい液体 
\\	〔人・物がなくなること〕 彼女の夫は3年前に蒸発したきりである 
\\	蒸発器 
\\	蒸発皿 
\\	蒸発熱 
\\	蒸発力 
\\	蒸発性 
\\	蒸発性の 〔揮発性の〕
\\	〔気化できる〕
\\	上品	じょうひん	上品な 〔洗練された〕
\\	〔優雅な〕
\\	上品な言葉遣いをする 
\\	態度も趣味(しゅみ)も上品な人 
\\	上品な服装をしている 
\\	上品ぶる 
\\	勝負	しょうぶ	
\\	小便	しょうべん	
\\	消防署	しょうぼうしょ	
\\	正味	しょうみ	正味50グラム〔表示〕 
\\	正味8時間働いた 
\\	正味価格 
\\	正味重量 
\\	正面	しょうめん	
\\	消耗	しょうもう	
\\	〔物を使って減らすこと〕
\\	〔完全に使い切ること〕
\\	消耗する 
\\	電力[燃料]を消耗する 
\\	資源をすっかり消耗してしまった 
\\	洗濯しすぎると衣類(いるい)は早く消耗する 
\\	〔体力・気力を使い果たすこと〕 その仕事で体力を消耗した 
\\	消耗した顔付きで 
\\	一日中歩き回って消耗した 
\\	消耗戦 
\\	消耗品 
\\	〔特に軍需品など〕
\\	初級	しょきゅう	
\\	助教授	じょきょうじゅ	
\\	~色	~しょく	
\\	食塩	しょくえん	
\\	職人	しょくにん	
\\	〔昔の〕
\\	修理のため職人が入っている 
\\	職人気質(かたぎ) 
\\	職場	しょくば	
\\	初旬	しょじゅん	
\\	書籍	しょせき	
\\	⇒ほん(本) 書籍小包 〔表示〕
\\	書籍小包として送りたい 
\\	書籍商[店] ((米)) 
\\	((英)) 
\\	〔人〕
\\	書籍代 〔手当て〕
\\	〔本屋の勘定〕
\\	書籍目録 
\\	[((また米)) 
\\	食器	しょっき	
\\	書店	しょてん	
\\	書道	しょどう	
\\	初歩	しょほ	
\\	白髪	しらが	
\\	私立	しりつ	
\\	資料	しりょう	
\\	汁	しる	
\\	〔果実の液〕
\\	〔草木の液〕
\\	汁の多い 
\\	汁をしぼる 
\\	レモンの汁をしぼる 
\\	〔吸い物〕
\\	〔肉汁〕
\\	みそ汁 
\\	煮出し汁 
\\	すまし汁 
\\	〔もうけ〕
\\	((俗)) 
\\	店は私との共同経営なのに彼独りうまい汁を吸っている 
\\	素人	しろうと	
\\	しわ (かおの~)	しわ (かおの~)	
\\	芯	しん	
\\	〔物の真中にある物〕 ろうそく(蝋燭)の芯 
\\	りんごの芯 
\\	鉛筆の芯 
\\	襟の芯 
\\	この柱は芯が腐っている 
\\	芯のある御飯 
\\	〔伸びる枝の先の芽〕
\\	豆の芯を止める 
\\	〔心根〕 彼は芯の芯まで紳士である 
\\	芯まで冷えた 
\\	体の芯まで疲れた 
\\	新幹線	しんかんせん	
\\	真空	しんくう	
\\	人事	じんじ	
\\	〔被雇用者に関する事柄〕
\\	〔人のなしうること〕 人事を尽くして天命を待つのみだ 
\\	人事異動 
\\	人事委員会 
\\	人事院 
\\	人事院勧告 
\\	人事課 
\\	人事消息欄 
\\	信ずる	しんずる	
\\	心身	しんしん	
\\	申請	しんせい	
\\	〔請願〕
\\	申請する 
\\	彼は領事館にビザの交付を申請した 
\\	証人喚問を申請する 
\\	申請書 
\\	申請書を出す 
\\	申請人 
\\	人造	じんぞう	
\\	寝台	しんだい	
\\	診断	しんだん	
\\	侵入	しんにゅう	
\\	人文科学	じんぶんかがく	
\\	人命	じんめい	
\\	深夜	しんや	
\\	森林	しんりん	
\\	親類	しんるい	
\\	親戚 (しんせき))
\\	針路	しんろ	
\\	神話	しんわ	
\\	水産	すいさん	
\\	水産加工品 
\\	水産業 
\\	水産試験所 
\\	水産大学 
\\	水産庁 
\\	水産物 
\\	炊事	すいじ	
\\	水蒸気	すいじょうき	
\\	水素	すいそ	
\\	垂直	すいちょく	
\\	推定	すいてい	
\\	水滴	すいてき	
\\	水筒	すいとう	
\\	随筆	ずいひつ	
\\	水分	すいぶん	
\\	水平	すいへい	水平の 
\\	は通例面について,
\\	は垂直に対する語で通例線について言う) 横木を地面と水平にする[保つ] 
\\	水平器 
\\	水平距離 
\\	水平軸 
\\	水平思考 
\\	水平線 
\\	水平線上[下]に 
\\	水平線に船が現れた 
\\	水平舵(だ) 
\\	〔飛行機の〕
\\	水平動 
\\	は衝撃,
\\	は振動) 水平飛行 
\\	水平飛行をする 
\\	水平貿易 
\\	水平面 
\\	水平線	すいへいせん	
\\	水面	すいめん	
\\	図々しい	ずうずうしい	〔無礼で生意気な〕
\\	((口)) 
\\	〔鉄面皮な〕
\\	〔恥知らずな〕
\\	図々しく 
\\	何とずうずうしいやつだろう 
\\	!/((口)) 
\\	彼はずうずうしくまたやって来た 
\\	/((口)) 
\\	彼の図々しさにはあきれた 
\\	末っ子	すえっこ	
\\	図鑑	ずかん	
\\	透き	すき	すき【透き・×隙】 
\\	⇒すきま(隙間) 足を踏み入れる透きもない 
\\	〔わずかなひま・余裕〕 透きを見てやっておきます 
\\	仕事の透きを見てあの病人を見舞ってください 
\\	〔気のゆるみ,油断〕 
\\	相手の透きに乗(じょう)ずる 
\\	透きをつかれた 
\\	あの男はまったく油断も透きもあったもんじゃない 
\\	そんな男に跡をつけられるのは君の方に透きがあったからだ 
\\	彼は透きを見せまいと緊張しきっている 
\\	一分の透きもない演説だった 
\\	透きをねらって逃げ出すつもりだ 
\\	杉	すぎ	
\\	好き嫌い	すききらい	
\\	好き好き	すきずき	
\\	好き好きですから何とも言えない 
\\	透き通る	すきとおる	
\\	〔透けて見える〕
\\	透き通った 
\\	((文)) 
\\	透き通って見える服 
\\	〔声がよく透る〕 透き通る声で話した 
\\	隙間	すきま	
\\	~過ぎる	~すぎる	
\\	図形	ずけい	
\\	鈴	すず	
\\	涼む	すずむ	
\\	すっきり	すっきり	
\\	〔気持ちのいい様子〕 お茶を1杯飲んだらすっきりした 
\\	暑い日は冷たい飲物(のみもの)がすっきりする 
\\	[((口)) 
\\	早くこの仕事を仕上げてすっきりした気分になりたい 
\\	髪を短くしてすっきりした 
\\	〔無駄がなく垢抜(あかぬ)けしている様子〕 垢抜ける 
\\	すっきりした文章 
\\	すっきりした服 
\\	彼女の服装はすっきりしている 
\\	この絵にはすっきりした美しさがある 
\\	〔はっきりする様子〕 彼の病気はまだすっきりしない 
\\	彼の返答には何かすっきりしないものがある 
\\	素直	すなお	
\\	頭脳	ずのう	
\\	図表	ずひょう	
\\	住まい	すまい	
\\	すまない	すまない	
\\	~済	~ずみ	
\\	相撲	すもう	
\\	ずらす	ずらす	
\\	〔物を〕
\\	机を左にずらす 
\\	体を後ろにずらして座り直した 
\\	〔時間差を設ける〕
\\	道路が込まないように出勤時間をずらす 
\\	〔時間を遅らす〕
\\	会議が2日ずらされた 
\\	ずらり	ずらり	
\\	ステージには歌手がずらりと並んでいた 
\\	ホテルの前には高級車がずらりと並んでいた 
\\	ずるい	狡い	
\\	〔こすい〕
\\	〔悪賢い〕
\\	〔こそこそした〕((口)) 
\\	〔不正直な〕
\\	ずるい企(たくら)み 
\\	ずるい商法 
\\	あいつはずるい事をする 
\\	/〔競技で〕
\\	ずるそうな目付きをした男 
\\	私に言わないなんてずるい 
\\	写真を隠していたなんてずるいぞ 
\\	寸法	すんぽう	
\\	〔長さ〕
\\	胸の寸法は95センチある 
\\	箱の寸法を物差しで測った 
\\	紙の寸法は縦20センチ,横30センチだった 
\\	寸法どおりに机を作った 
\\	〔やり方,計画〕 まずデマを飛ばしてあいつを失脚させるという寸法だ 
\\	税関	ぜいかん	
\\	((英))では複数扱い,((米))では単数扱い); 
\\	は空港・港で旅行者の荷物を調べる所
\\	は主に港で輸入品などを検査する建物) 横浜税関 
\\	税関を通る 
\\	税関で厳重に荷物を調べられた 
\\	私たちの一行には税関の調べがなかった 
\\	この国の税関は外国観光客に対して特にやかましくない 
\\	税関検査官 
\\	税関申告書 
\\	税関長 
\\	税関手続き 
\\	税関吏 
\\	製作	せいさく	
\\	制作	せいさく	
\\	清書	せいしょ	
\\	青少年	せいしょうねん	
\\	整数	せいすう	
\\	清掃	せいそう	
\\	清掃する 
\\	清掃作業員 〔道路の〕
\\	〔ごみ収集の〕((米)) 
\\	((英)) 
\\	((米)) 
\\	婉曲的) 清掃車 ((米)) 
\\	((英)) 
\\	生存	せいぞん	
\\	政党	せいとう	
\\	性能	せいのう	
\\	機械の性能 
\\	このシェーバーはあまり性能がよくない 
\\	この機械はたった5年で性能が落ちた 
\\	高性能のエンジン 
\\	性能試験 
\\	整備	せいび	
\\	〔用意して整えること〕
\\	実験室に最新のコンピュータを整備した 
\\	会場はいつでも使えるように整備されている 
\\	〔良い状態にしておくこと〕
\\	クレーコートの整備は高くつく 
\\	グラウンドを整備する 
\\	〔点検・修理しておくこと〕
\\	飛行機を整備する 
\\	車を十分に整備しておく 
\\	車を整備に出す 
\\	整備員 
\\	〔飛行機の〕
\\	整備工場 
\\	自動車整備工場 
\\	整備新幹線 
\\	成分	せいぶん	
\\	性別	せいべつ	
\\	正方形	せいほうけい	
\\	正門	せいもん	
\\	成立	せいりつ	
\\	西暦	せいれき	
\\	背負う	せおう	
\\	~席	~せき	
\\	赤道	せきどう	
\\	折角	せっかく	
\\	〔骨を折って,わざわざ〕 母がせっかく作ったご馳走がテーブルの上で冷たくなっている 
\\	せっかく訪ねてくれたのに私はあいにく留守だった 
\\	彼の失敗でせっかくの苦心も水の泡だ 
\\	〔相手の努力を謝して〕 せっかくのご招待ですから喜んで伺います 
\\	せっかくですが,今回はお伴できません 
\\	〔たまにしかないこと〕 せっかくの 
\\	〔たまにしかなくて貴重な〕
\\	せっかくの機会だったのに 
\\	せっかくの日曜日が彼のために駄目になってしまった 
\\	雨でせっかくの晴れ着が台無しだ 
\\	〔気を付けて〕 折角ご自愛下さい 
\\	接近	せっきん	
\\	接する	せっする	
\\	〔隣り合う〕
\\	国境に接する地帯 
\\	道路に接する地所 
\\	倉庫が軒を接している通り 
\\	〔一点で出合う〕
\\	二つの円が接する点 
\\	〔近づける〕 額を接してなにやら相談をしていた 
\\	きびすを接して ⇒きびす(踵) 
\\	〔接触する〕
\\	職業柄,著名人に接する機会が多い 
\\	初めて彼の芸術に接した 
\\	〔応対する〕 彼は客に丁寧に接した 
\\	〔受け取る〕
\\	訃報(ふほう)に接する 
\\	せっせと	せっせと	
\\	せっせと働く 
\\	せっせと手紙を書いている 
\\	図書館にせっせと通う 
\\	/((文))
\\	接続	せつぞく	
\\	瀬戸物	せともの	
\\	ぜひとも	ぜひとも	
\\	迫る	せまる	
\\	〔距離が近くなる〕
\\	私たちは山の頂上に迫った 
\\	崖(がけ)が海に迫っている 
\\	トムはわずかの差で第2位に迫っていた 
\\	眼前に迫った危険を前になすすべを知らなかった 
\\	〔ある時刻が近付く〕 時間が迫った 
\\	締め切りが迫っている 
\\	試合はあと1週間後に迫った 
\\	彼は自分の死期の迫ったことを感じた 
\\	〔詰まって苦しくなる〕 彼女は悲しさに胸が迫った 
\\	〔相手に強く求める〕
\\	彼は私に金をくれと迫った 
\\	彼女は必要に迫られて金を盗んだ 
\\	彼は我々に証拠を見せろと迫った 
\\	彼は私に交際を迫った 
\\	せめて	せめて	
\\	せめてパリだけでも見たい 
\\	せめてもう1時間でも長くいられたらなあ 
\\	せめて2キロくらいなら駅まで歩くのだが 
\\	彼女にとって絵をかくことがせめてもの楽しみだ 
\\	そっとしておくのが彼に対するせめてもの思いやりだ 
\\	台詞	せりふ	
\\	栓	せん	
\\	~船	~せん	
\\	~戦	~せん	
\\	前~	ぜん~	
\\	~前	~ぜん	
\\	前後	ぜんご	
\\	全集	ぜんしゅう	
\\	扇子	せんす	
\\	専制	せんせい	
\\	権力を濫用するという含みがある) 専制の 
\\	専制君主 
\\	〔暴君〕
\\	専制君主政体 
\\	専制政治 
\\	先々月	せんせんげつ	
\\	先々週	せんせんしゅう	
\\	先祖	せんぞ	
\\	先端	せんたん	
\\	先頭	せんとう	
\\	全般	ぜんぱん	
\\	扇風機	せんぷうき	
\\	洗面	せんめん	洗面する 
\\	洗面器 
\\	洗面所 
\\	洗面台 
\\	洗面用具 
\\	全力	ぜんりょく	
\\	線路	せんろ	
\\	((米)) 
\\	線路を敷く 
\\	線路に入ること厳禁 
\\	線路工事 〔線路敷設(ふせつ)〕
\\	〔補強,修理など〕
\\	~沿い	~そい	
\\	線路沿いに西へ500メートルほど行く 
\\	湖沿いの遊歩道(ゆうほどう) 
\\	総~	そう~	
\\	~艘	~そう	
\\	[接尾]助数詞。比較的小さい船を数えるのに用いる。「屋形船一―」
\\	相違	そうい	
\\	そういえば	そういえば	
\\	雑巾	ぞうきん	
\\	増減	ぞうげん	
\\	倉庫	そうこ	
\\	相互	そうご	
\\	創作	そうさく	
\\	葬式	そうしき	
\\	造船	ぞうせん	
\\	騒々しい	そうぞうしい	
\\	増大	ぞうだい	
\\	そうっと	そうっと	
\\	送別	そうべつ	
\\	彼が送別の辞(じ)を述べた 
\\	送別会 彼の送別会が行われた 
\\	草履	ぞうり	
\\	総理大臣	そうりだいじん	
\\	送料	そうりょう	〔郵便の〕
\\	〔貨物の〕
\\	((英文)) 
\\	外国郵便の送料 
\\	送料込みで10,000円 
\\	御返品(へんぴん)の場合は送料無料 
\\	~足	~そく	
\\	属する	ぞくする	
\\	〔種類・区分に入る〕
\\	虎はネコ科に属する 
\\	鯨は魚ではなく哺乳(ほにゅう)類に属する 
\\	〔所属する〕
\\	君はどのチームに属していたの 
\\	私はどの政党にも属する気はない 
\\	我々のグループは国際団体に属しています 
\\	この島はついこの間までフランスに属していた 
\\	続々	ぞくぞく	
\\	速達	そくたつ	
\\	測定	そくてい	
\\	測定する 
\\	〔土地を〕
\\	〔水深を〕
\\	放射線測定 
\\	水温の変化を測定する 
\\	体重を測定する 
\\	測定器 
\\	測定法 
\\	測量	そくりょう	
\\	速力	そくりょく	
\\	素質	そしつ	〔適性〕
\\	〔才能〕
\\	恵子はピアニストの素質を持っている 
\\	彼女は絵かきの素質はない 
\\	遺伝的な犯罪の素質など私は信じない 
\\	祖先	そせん	
\\	そそっかしい	そそっかしい	
\\	あの人は実にそそっかしい 
\\	/((口)) 
\\	そそっかしい人はお断りだ 
\\	彼はまたそそっかしい間違いをした 
\\	[((口)) 
\\	/((俗)) 
\\	卒直	そっちょく	
\\	そのころ	そのころ	
\\	そのため	そのため	
\\	その他	そのほか	
\\	〔残り全部〕 その外の人たちはみな賛成した 
\\	そこは暑いがその外の部屋は涼しい 
\\	〔更に〕 その外に何が欲しいのか 
\\	子供たちに本やその外たくさんの物をやった 
\\	その外に次の企画を説明してくれた 
\\	剃る	そる	
\\	それなのに	それなのに	
\\	彼のためにはずいぶん尽くした
\\	それなのに彼は私を裏切った 
\\	それなら	それなら	
\\	それなら行きましょう 
\\	それならそうと前もって言ってくださればよいのに 
\\	東京に来てくださいますか
\\	それなら航空券をお送りします 
\\	それならどうして自分でやらないの 
\\	それはいけませんね (かん)	それはいけませんね (かん)	
\\	逸れる	それる	
\\	〔狙いが外れる〕 弾丸が的からそれた 
\\	〔ほかの方へ行く〕 そこから道は東へそれる 
\\	飛行機は予定の航路からそれていた 
\\	彼は私を見るといきなり横町へそれていった 
\\	彼の注意がそれた 
\\	〔本筋から離れる〕 彼の話はよくわき道にそれる 
\\	正道からそれて悪の道に走った 
\\	算盤	そろばん	
\\	存じる	ぞんじる	
\\	〔知っている〕 存じません 
\\	ご存じの通り 
\\	存じております 
\\	あの人をご存じですか 
\\	/〔格式ばって〕
\\	〔思う〕
\\	ご一緒したいと存じます 
\\	存する	ぞんする	
\\	〔存在する〕
\\	〔残っている〕
\\	〔生存する〕
\\	地球の存する限り 
\\	〔…にある〕((文)) 
\\	((文)) 
\\	人間の価値は心の尊さに存する 
\\	損得	そんとく	
\\	田ぼ	たんぼ	
\\	第~	だい~	
\\	だいいち (とりわけ)	だいいち (とりわけ)	
\\	大学院	だいがくいん	
\\	大工	だいく	
\\	体系	たいけい	
\\	体系的 (に) 
\\	賃金体系 
\\	思想体系を立てる 
\\	体系の整(ととの)った論文 
\\	体系的な調査 
\\	太鼓	たいこ	
\\	対策	たいさく	
\\	大して	たいして	この本は大して面白くない 
\\	彼の家は大して立派でもない 
\\	彼は大して学のある男ではない 
\\	そんなことは大して気にしません 
\\	あの人を大して気に入っているわけではない 
\\	大小	だいしょう	
\\	体制	たいせい	〔組織体系〕
\\	〔構造〕
\\	経済[政治]体制 
\\	資本主義体制 
\\	現体制 
\\	反体制の 
\\	新[旧]体制 
\\	その国は戦時体制下にある 
\\	体制側につく 
\\	体制主義者 
\\	体積	たいせき	
\\	この容器の体積は2立方メートルある 
\\	大層	たいそう	
\\	体操	たいそう	
\\	大分	だいぶん	
\\	大木	たいぼく	
\\	代名詞	だいめいし	
\\	対立	たいりつ	
\\	〔敵対〕
\\	対立する 
\\	〔対立させる〕
\\	その二派は対立状態にあった 
\\	この問題に関して父とは真っ向から意見が対立している 
\\	現職知事に対立させて若い候補者を立てた 
\\	それにより労使間の深刻な対立が生まれた 
\\	田植え	たうえ	
\\	絶えず	たえず	
\\	楕円	だえん	
\\	楕円 (形) の 
\\	楕円軌道 
\\	耕す	たがやす	
\\	滝	たき	
\\	蓄える	たくわえる	
\\	〔ためておく〕
\\	〔残しておく〕
\\	金を蓄える 
\\	明日のレースに備えて精力を蓄えておく 
\\	冬期用の燃料を蓄える 
\\	彼は豊富な知識を蓄えている 
\\	〔ひげ・髪などをはやしておく〕 あごひげを蓄える[蓄えている] 
\\	竹	たけ	
\\	ただいま	ただいま	
\\	但し	ただし	〔しかし〕
\\	〔…という条件で〕
\\	外出は自由だ
\\	ただし10時までには帰らなければならない 
\\	あれでなかなかいい男なんだよ,ただし時々飲みすぎなければね 
\\	喜んで引き受けましょう,ただし健康が許せばです 
\\	立ち止まる	たちどまる	〔歩くのをやめて止まる〕
\\	〔静止する〕
\\	先生が来るのを見て彼女は立ち止まった 
\\	彼はポスターの前で立ち止まった 
\\	馬は障害の前で急に立ち止まった 
\\	立ち止まらないように 
\\	たちまち	たちまち	
\\	〔すぐに〕 火はたちまち2階に燃え広がった 
\\	この薬はたちまち効(き)くということだ 
\\	〔にわかに〕 口論(こうろん)はたちまち殴(なぐ)り合いに変わった 
\\	たちまち大粒(おおつぶ)の雨が降ってきた 
\\	発	はつ	
\\	脱線	だっせん	
\\	〔列車などの〕
\\	列車が脱線した 
\\	〔本筋から外れること〕
\\	((文)) 
\\	(主題から) 脱線する 
\\	話が脱線してしまった 
\\	/((文)) 
\\	/((米口)) 
\\	脱線しないようにしましょう 
\\	彼の話は脱線が多過ぎる 
\\	妥当	だとう	妥当な 〔目的などにかなった〕
\\	〔正当な〕
\\	あの場合妥当な処置であった 
\\	妥当な批判 
\\	彼が偉大な小説家と呼ばれるのは妥当ではない 
\\	その手段が妥当かどうかあやしい 
\\	妥当性 
\\	例える	たとえる	
\\	頼もしい	たのもしい	
\\	溜息	ためいき	
\\	ためらう	ためらう	〔ちゅうちょする〕
\\	〔考えがぐらつく〕
\\	彼はそれを言うのをためらった 
\\	彼はその危険を冒(おか)すのをためらった 
\\	彼は広間に入ろうとしてちょっとためらった 
\\	なぜ私がためらっているかお分かりですか 
\\	~だらけ	~だらけ	
\\	だらしない	だらしない	
\\	〔態度・姿などにしまりがない〕
\\	だらしのない様子をしている 
\\	彼の仕事ぶりはだらしない 
\\	[((英)) 
\\	だらしなく開いた唇 
\\	いすにだらしなく掛けていた 
\\	〔ふがいない〕 だらしないへまをやったものだ 
\\	そんなことで泣くなんてだらしないぞ 
\\	〔ふしだらな〕
\\	だらしない女 
\\	彼は女にだらしない 
\\	足る	たる	彼は信頼するに足る人物だ 
\\	彼の行為は称賛するに足る 
\\	その計画は考えてみるに足る 
\\	その本は読むに足らない 
\\	彼が怒ったって驚くに足らない 
\\	彼女が来なかったとしても怪しむに足らない 
\\	短~	たん~	
\\	~団	~だん	
\\	あいまい	曖昧	
\\	〔はっきりしないこと〕 あいまいな 〔不明瞭な〕
\\	〔明確につかめない〕
\\	〔どっちつかずの〕
\\	〔多義・両義にとれる〕
\\	〔いかようにもとれる〕
\\	意図的に人を混乱させるための) あいまいな説明 
\\	あいまいな表現 
\\	あいまいに答える 
\\	あいまいな態度 
\\	彼はいつもあいまいなことしか言わない 
\\	⇒いかがわしい(如何わしい) 曖昧検索(けんさく) 〔コンピュータで〕
\\	曖昧模糊(もこ) 曖昧模糊とした 
\\	曖昧宿 〔売春婦を置いている店〕
\\	〔あやしげな商売をしている店〕
\\	扇ぐ	あおぐ	
\\	青白い	あおじろい	
\\	呆れる	あきれる	〔あっけにとられる〕
\\	((口)) 
\\	〔驚く〕
\\	〔うんざりする〕
\\	その大きさにはただあきれた 
\\	あきれた値段 
\\	まあ,あきれた 
\\	これは主に女性のことば) あきれたやつだ 
\\	彼の馬鹿さ加減にはあきれて物が言えない 
\\	あくび	あくび	
\\	飽くまで	あくまで	
\\	〔最後まで〕
\\	〔根気強く〕
\\	彼は飽くまでも否定(ひてい)した 
\\	飽くまでも戦い抜く 
\\	彼は提案に飽くまでも反対した 
\\	/((口)) 
\\	彼女は飽くまでもしらを切り続けようとした 
\\	しらを切る 
\\	明くる~	あくる~	
\\	明くる1月4日(よっか)に 
\\	明くる日[年] 
\\	明くる 
\\	月 
\\	日に 
\\	明け方	あけがた	
\\	憧れる	あこがれる	
\\	〔そうしたい・そのようになりたいと思う〕
\\	彼は海にあこがれている 
\\	甥(おい)は操縦士(そうじゅうし)にあこがれている 
\\	まぶたの母にあこがれていた 
\\	若い人は大都会にあこがれるものである 
\\	[((口))
\\	〔異性にひかれる〕 
\\	女の子たちはみんなそのテニスコーチにあこがれている 
\\	異性に対して
\\	と言うと,恋い焦がれる意になる)
\\	朝寝坊	あさねぼう	
\\	足跡	あしあと	
\\	足元	あしもと	
\\	〔足の周辺〕 懐中電灯(かいちゅうでんとう)が足下の水溜(みずたま)りを照(て)らし出した 
\\	足下の地面が揺(ゆ)れた 
\\	足下に気を付けなさい 
\\	〔歩き振り〕⇒あしどり(足取り)
\\	〔身近な所〕 警察は首相の足下から出た収賄(しゅうわい)事件を摘発(てきはつ)した 
\\	〔足場〕 足下を固めてから計画に取り掛かろう 
\\	足下から鳥が立つ 
\\	〔急に思い立ってする〕 足下から鳥が飛び立つように彼は荷物をまとめて出て行った 
\\	〔身近に意外なことが起きる〕 まさか足下から鳥が立つとは思わなかった 
\\	足下に付け込む 彼は我々の足下につけこんで法外な要求をしてきた 
\\	足下に火がつく 商売は足下に火がついた 
\\	彼は足下に火がついていた 
\\	足下の明るいうちに 
\\	〔暗くならないうちに〕 足下の明るいうちに早く家に帰りなさい 
\\	〔不利にならないうちに〕 足下の明るいうちにその仕事から手を引きなさい 
\\	足下の明るいうちにさっさと帰ったほうがいいぞ 
\\	[((米))
\\	足下へも寄りつけない 私の絵は色の美しさの点で姉の足下へも寄りつけない 
\\	彼は秀才で私は足下へも寄りつけない 
\\	足下を見透かす 彼に足下を見透かされているようだ 
\\	足下を見る 彼に足下を見られた 
\\	味わう	あじわう	
\\	あちらこちら	あちらこちら	
\\	厚かましい	あつかましい	〔鉄面皮の〕
\\	〔無礼で生意気な〕
\\	((口)) 
\\	〔僭越な〕
\\	厚かましいやつだ 
\\	厚かましくも…する 
\\	/((口))
\\	なんて厚かましい 
\\	厚かましいとは思いますが… 
\\	圧縮	あっしゅく	
\\	宛名	あてな	〔受信人〕
\\	〔住所〕
\\	宛名が違っている 
\\	宛名不明で返送された 
\\	当てはまる	あてはまる	
\\	この理論はすべての無理数に当てはまる
\\	当てはめる	あてはめる	
\\	〔適用する〕
\\	その方法はすべての場合には当てはめることはできない 
\\	〔合わせる〕
\\	彼は自説にマルクスの理論を器用に当てはめた 
\\	人の事と思わずに,我が身に当てはめて考えてごらん 
\\	暴れる	あばれる	
\\	〔乱暴な行いをする〕
\\	〔激して〕
\\	〔逆上して〕
\\	〔群衆が〕
\\	〔馬などが〕
\\	〔もがく〕
\\	酔っ払いがバーで暴れた 
\\	暴れた群衆は警察署へなだれ込んだ 
\\	取り押さえられた犯人は逃げようとして暴れた 
\\	犬が暴れないようにしておいてくれ 
\\	〔思う存分動き回る〕 今日はグラウンドで暴れ回るぞ〔スポーツで〕 
\\	あぶる	あぶる	
\\	〔火で暖める,乾かす〕 手をあぶって暖める 
\\	服をストーブであぶって乾(かわ)かす 
\\	〔火から少し離して焼く〕 干し魚をあぶる 
\\	のりをあぶる 
\\	あふれる	あふれる	
\\	雨戸	あまど	
\\	甘やかす	あまやかす	
\\	余る	あまる	
\\	〔残る〕
\\	9を2で割ると商(しょう)が4で1余る 
\\	先週は小遣いが全然余らなかった 
\\	ご飯がたくさん余った 
\\	ブリッジをするには1人余った 
\\	金は余るほどある 
\\	余った仕事[金] 
\\	余った分を私によこしなさい 
\\	余った金は銀行に預ける 
\\	余った時間なんてない 
\\	〔まさる〕
\\	〔力が及ばない〕
\\	その渓谷(けいこく)の壮大(そうだい)な眺めは想像に余る 
\\	車は勢(いきお)い余って柵(さく)を突き破った 
\\	彼の力に余るものは何もない 
\\	編み物	あみもの	
\\	編む	あむ	
\\	危うい	あやうい	
\\	⇒あぶない(危ない)
\\	怪しい	あやしい	
\\	〔変な〕
\\	〔不思議な〕
\\	あの空き家(あきや)には怪しい話がある 
\\	彼女には怪しい魅力がある 
\\	怪しい光 
\\	〔疑わしい〕
\\	((口)) 
\\	〔信用できない〕
\\	〔当てにならない〕
\\	怪しいと思う ⇒あやしむ(怪しむ) その話が本当かどうかは怪しい 
\\	怪しい話だなあ 
\\	彼は怪しい人物だ 
\\	約束を守ってくれるかどうか怪しい 
\\	君の情報は怪しいと思うよ 
\\	二人の間は怪しいとにらんでいる 
\\	彼の理論づけは怪しいと思う 
\\	〔おぼつかない〕
\\	〔不確かな〕
\\	怪しい手付きでりんごをむいていた 
\\	彼の英語は相当怪しい 
\\	スケートの腕は怪しい 
\\	〔悪く変わりそうな〕 この黒雲では明日の天気は怪しい 
\\	荒い	あらい	
\\	〔性質などが強く激しい〕
\\	〔乱暴な〕
\\	荒い扱(あつか)い方 
\\	荒い海 
\\	/((文))
\\	彼は気性が荒い 
\\	/〔口〕
\\	彼は言葉遣いが荒い 
\\	は冷酷な,
\\	は激情的な,
\\	は粗野で下品な,
\\	は粗暴な感じを含む) 
\\	〔勢いが激しい〕 荒い呼吸(こきゅう)をする 
\\	彼は人使いが荒い 
\\	彼女は金遣いが荒かった 
\\	粗い	あらい	
\\	〔細かでない〕
\\	粗い粉 
\\	粗い布地(きれじ) 
\\	粗い目のふるい 
\\	粗いしまの布 
\\	布を粗い針目(はりめ)で縫(ぬ)い合わせた 
\\	〔ざらざらしている〕
\\	きめの粗い木 
\\	〔大ざっぱな〕
\\	〔おざなりの〕
\\	粗い細工 
\\	粗い仕上げ 
\\	粗い磨き方 
\\	粗い調査 
\\	彼は仕事が粗い 
\\	粗筋	あらすじ	
\\	芝居の粗筋 
\\	改めて	あらためて	
\\	〔別の時に〕
\\	〔のちほど〕
\\	改めてお電話いたします 
\\	そのことなら改めてお話しいたします 
\\	〔もう一度〕
\\	改めて怒りを感じた 
\\	改めて言うまでもないがしっかり勉強しなさい 
\\	改めて今言うことは何もない 
\\	改めて事故の原因を調査した 
\\	改める	あらためる	
\\	〔新しくする〕
\\	〔変える〕
\\	契約を改める 
\\	12月の末までに規則を改めることになっている 
\\	随筆(ずいひつ)の稿(こう)を初めから改めた 
\\	こう【稿】 ⇒げんこう(原稿) 稿を起こす 
\\	この点については稿を改めて書く 
\\	日付の「3月」を「4月」に改める〔正誤表などで〕 
\\	日を改めて来ます 
\\	げんこう【原稿】
\\	雑誌に原稿を書いている 
\\	原稿を書いて生活をする 
\\	原稿なしで話をする 
\\	原稿用紙 
\\	〔とじてある〕
\\	400字詰め原稿用紙 
\\	原稿料 
\\	〔変えてよくする〕
\\	〔正しくする〕
\\	彼は行いを改めた 
\\	誤字[つづりの誤り]を改める 
\\	〔言葉・態度などを変える〕 態度を改める 
\\	服装を改める 〔みなりを整える〕
\\	/〔着替える〕
\\	〔調べる〕
\\	〔数える〕
\\	切符[旅券(りょけん)]を改める 
\\	人数を改める 
\\	有難い	ありがたい	
\\	あれこれ	あれこれ	
\\	あれこれ迷(まよ)った揚句(あげく)転職(てんしょく)することにした 
\\	彼は親切にあれこれ忠告してくれた 
\\	あれこれ言うのはやめてくれ 
\\	[((口)) 
\\	荒れる	あれる	
\\	〔特に海がしける〕
\\	〔あらしになる〕
\\	今日は荒れそうだ 
\\	3日海が荒れている 
\\	〔荒廃する,建物などが崩れかかる〕
\\	〔家などが荒れ果てる〕
\\	〔土地が〕
\\	人のいなくなった家は荒れていた 
\\	畑が荒れている 
\\	〔皮膚がかさかさになる〕
\\	〔あかぎれなどで〕
\\	庭仕事で手が荒れた 
\\	胃が悪くて舌が荒れている 
\\	唇が荒れている 
\\	〔行為・秩序などが乱れる〕 彼は荒れた生活をしている 
\\	昨夜の宴会(えんかい)は荒れた 
\\	双方(そうほう)にエラーが続出(ぞくしゅつ)して試合が荒れた 
\\	慌ただしい	あわただしい	〔忙しい〕
\\	〔せかされた〕
\\	慌ただしく 〔忙しく〕
\\	〔あわてて〕
\\	慌ただしい一日 
\\	慌ただしい食事 
\\	記者たちが建物(たてもの)から慌ただしく出入りしている 
\\	慌ただしく動き回る 
\\	慌ただしく出掛けて,かぎを忘れた 
\\	慌ただしくせかされるのは嫌だ 
\\	急(せ)かす 
\\	慌ただしい情勢(じょうせい)の変化 
\\	安易	あんい	安易な 〔手軽な〕
\\	〔のんきな,気楽な〕
\\	安易な方法を選ぶ 
\\	安易な生き方をする 
\\	安易に返事をする 
\\	そんな安易な考え方ではだめだ 
\\	~位	~い	
\\	言い出す	いいだす	
\\	〔口を切る〕 言い出しにくいこと 
\\	言い出し渋る 
\\	彼女は言い出したら後へ引かない 
\\	地球は回ると言い出したのはだれだ 
\\	だれかがそれを言い出しはしないかと心配した 
\\	〔提案する〕
\\	尾瀬に行こうと彼が言い出した 
\\	言い付ける	いいつける	
\\	〔命じる〕
\\	支配人は秘書に次々と用を言い付けた 
\\	〔告げ口をする〕
\\	同僚のことを人に言い付けるような人ではない 
\\	先生にみんな言い付けます 
\\	そんなことをすると言い付けるぞ 
\\	意義	いぎ	
\\	生き生き	いきいき	
\\	いきなり	いきなり	
\\	〔突然〕
\\	いきなり部屋から飛び出した 
\\	相手はいきなり彼に突っ掛かってきた 
\\	彼はいきなり立ち上がった 
\\	いきなり彼に尋ねられてびっくりした 
\\	〔知らせなしに〕
\\	いきなり解雇を言い渡された 
\\	幾~	いく~	
\\	育児	いくじ	
\\	幾分	いくぶん	
\\	〔いくらか〕
\\	⇒すこし(少し) 今日は幾分気分がいい 
\\	幾分そういう傾向がある 
\\	彼は幾分かは私の考えが分かっている 
\\	〔一部分〕
\\	収入の幾分か 
\\	生け花	いけばな	
\\	以後	いご	
\\	〔ある時より後〕
\\	〔…以来〕
\\	夜10時以後はテレビの音を小さくしてください 
\\	来週の水曜日以後は家にいません 
\\	それ以後今日まで 
\\	それ以後彼女からなんの便りもない 
\\	彼のそれ以後の作品は読んでいません 
\\	〔今後〕
\\	〔将来は〕
\\	5年以後に 
\\	[((文))
\\	以後気を付けます 
\\	以後そんなことは口にしなかった 
\\	以降	いこう	⇒いご(以後) 3日以降 
\\	1960年以降 
\\	勇ましい	いさましい	
\\	衣食住	いしょくじゅう	
\\	~いち (にほんいち)	~いち (にほんいち)	
\\	いちいち	いちいち	
\\	一応	いちおう	
\\	〔一通り〕 その手紙に一応目を通しただけで破(やぶ)り捨てた 
\\	彼の主張は一応筋(すじ)が通っている 
\\	筋が通る 
\\	念のため一応この書類を見てください 
\\	〔ひとまず〕 
\\	一応これでよかろう 
\\	一応その申し出を承諾(しょうだく)した 
\\	一応身元証明書を持って行きなさい 
\\	一段	いちだん	
\\	〔階段・段階などの〕
\\	〔梯子(はしご)の〕
\\	一段高いところに 
\\	地位が一段上がる 
\\	〔ひときわ〕 一段と進歩する 
\\	これで一段と引き立つでしょう 
\\	引き立つ 
\\	〔文章の段落〕
\\	〔一節〕
\\	一流	いちりゅう	
\\	一昨日	いっさくじつ	
\\	一昨年	いっさくねん	
\\	一斉	いっせい	
\\	一旦	いったん	
\\	〔一度〕
\\	一旦した約束は履行しなければならない 
\\	彼は一旦こうと決めたら一歩も後へ引かない 
\\	一旦は大学進学をあきらめた 
\\	〔もしも〕 一旦事緩急あれば 
\\	〔一度〕 踏切では一旦停止しなければならない 
\\	一旦ここで休憩してまた続けよう 
\\	一定	いってい	
\\	行っていらっしゃい	いっていらっしゃい	
\\	行ってらっしゃい	いってらっしゃい	
\\	いってまいります	いってまいります	
\\	いってきます	いってきます	
\\	移転	いてん	
\\	井戸	いど	
\\	緯度	いど	
\\	威張る	いばる	
\\	〔偉そうにする〕 彼は劣等感(れっとうかん)から威張るのだ 
\\	あの男の威張った態度は鼻持(はなも)ちならない 
\\	彼らはなんと威張って歩いていることだろう 
\\	彼は部下に威張り散らす 
\\	〔自慢する〕⇒じまん(自慢) 彼は息子が成功したと威張っている 
\\	彼女はいくつもの賞を得たが少しも威張ったところがない 
\\	嫌がる	いやがる	
\\	いよいよ	いよいよ	
\\	〔ますます〕
\\	嵐(あらし)はいよいよ激しくなってくる 
\\	彼はいよいよしっかり勉強しなければならない 
\\	彼を知れば知るほどいよいよ敬愛の念を持つ 
\\	〔ついに〕 いよいよという時に 
\\	/〔せっぱつまった時〕
\\	いよいよ式典の当日となった 
\\	いよいよ本降りだ 
\\	いよいよ試合が始まる 
\\	いよいよとなればやります 
\\	いよいよという時に力が抜けてしまった 
\\	いよいよ金に困ったら借りに行くよ 
\\	いよいよ大会も終りに近づいた 
\\	煎る	いる	いる
\\	炒る・×煎る】 〔コーヒー豆・ナッツなどを〕
\\	〔豆・穀物などを〕
\\	南京豆をいる 
\\	いり豆 
\\	ごまをいる 
\\	炒る	いる	〔コーヒー豆・ナッツなどを〕
\\	〔豆・穀物などを〕
\\	南京豆(ナンキンまめ)をいる 
\\	いり豆 
\\	ごまをいる 
\\	入れ物	いれもの	
\\	引力	いんりょく	
\\	植木	うえき	
\\	飢える	うえる	
\\	浮ぶ	うかぶ	
\\	浮かべる	うかべる	
\\	〔浮くようにする〕
\\	〔進水させる〕
\\	船を浮かべる 〔進水させる〕
\\	/〔走らせる〕
\\	小舟を浮かべて湖水に一日遊んだ 
\\	〔表面に出す〕 彼女は目に涙を浮かべていた 
\\	彼女は口元に笑みを浮かべてお辞儀をした 
\\	町中が悲しみの色を浮かべていた 
\\	〔思い出す〕 両親のことを心に浮かべたらそんなことはできなかった 
\\	浮く	うく	
\\	〔水面などに〕
\\	木片が水に浮いている 
\\	一瞬体が宙に浮くのを感じた 
\\	〔表面に出る〕 男の額に汗が浮いていた 
\\	〔緩み離れる〕 歯が浮いた 
\\	土台が浮いている 
\\	〔人の気持ちから離れる〕 政府は国民から浮いてしまっている 
\\	〔余分が出る〕 倹約すれば月に1万円は浮く 
\\	ご馳走になった分食費が浮いた 
\\	費用はどれだけ浮きましたか 
\\	〔浮き浮きする〕 浮かない声でそう言った 
\\	浮かぬ顔 ⇒詳細を見る
\\	承る	うけたまわる	
\\	〔謹んで聞く〕
\\	⇒きく(聞く) ご意見を承りたい 
\\	早急にご返事を承りたい 
\\	ご趣旨は承りました 
\\	〔伝え聞く〕 承ればお子さんがご病気だそうで 
\\	〔謹んで受ける,承知する〕 ご依頼の件,承りました 
\\	ご用命を承ります 
\\	確かに承りました 
\\	披露宴のご相談はこの部屋で承ります 
\\	受取	うけとり	
\\	受け持つ	うけもつ	
\\	来年度は一年生を受け持ってもらえますか 
\\	この仕事は僕が受け持とう 
\\	彼はセールスマンとしてこの地域を受け持っている 
\\	薄暗い	うすぐらい	
\\	薄める	うすめる	
\\	打合せ	うちあわせ	
\\	複数形で) 打ち合わせをする 
\\	打ち合わせ通りに事が運んだ 
\\	打ち消す	うちけす	
\\	うどん	うどん	
\\	うなずく	うなずく	
\\	敬う	うやまう	
\\	⇒そんけい(尊敬) 彼女は町の人々から敬われていた 
\\	先生を敬う 
\\	神を敬う 
\\	小さい時は兄を敬っていたものだ 
\\	彼は神を敬い人を愛する男であった 
\\	裏返す	うらがえす	
\\	〔裏にする〕
\\	彼は上手にパンケーキを裏返した 
\\	は軽く空中にほうりあげてひっくり返す) オーバーを裏返して日に干しなさい 
\\	〔物の見方や立場を反対にする〕 裏返して考えてみれば 
\\	裏口	うらぐち	
\\	占う	うらなう	
\\	恨み	うらみ	
\\	恨む	うらむ	〔恨みを抱く〕
\\	〔非難する〕
\\	〔憤慨する〕
\\	彼に恨まれるような覚えはない 
\\	天を恨んでも始まらない 
\\	この企画の失敗に関しては我が身を恨むほかはない 
\\	どうか恨まないで欲しい 
\\	羨む	うらやむ	
\\	売上	うりあげ	
\\	当社の年間売り上げ (高) は3億円に達する 
\\	輸入品の売り上げがのびている 
\\	売上勘定 
\\	売上金(きん) 
\\	((口)) 
\\	その日の売上金20万円を奪われた 
\\	バザーの売上金 
\\	売上税(ぜい) 
\\	売上高(だか) 
\\	平均[月]売上高 
\\	総[純]売上高 
\\	売上帳 
\\	売上伝票 
\\	売り上げノルマ 
\\	売上元帳 
\\	売り切れ	うりきれ	
\\	売り切れる	うりきれる	
\\	売行き	うれゆき	
\\	〔需要〕
\\	売れ行きのよい本 
\\	/〔ベストセラー〕
\\	輸入品の売れ行きがよく[悪く]なった 
\\	この製品の売れ行きは止まった 
\\	その商品は売れ行きがよい[悪い] 
\\	浪々	うろうろ	
\\	いま浪々の身だ 
\\	家なき浪々の身だ 
\\	運河	うんが	
\\	うんと	うんと	
\\	その作品は新聞でうんとたたかれた 
\\	あとでうんとお説教してやるぞ 
\\	彼は僕よりうんと利口だ 
\\	文庫本をうんと買った 
\\	さあうんと食べなさい 
\\	/((俗)) 
\\	英文	えいぶん	
\\	英和	えいわ	
\\	ええと	ええと	
\\	液体	えきたい	
\\	絵の具	えのぐ	
\\	偉い	えらい	
\\	~園	~えん	
\\	宴会	えんかい	
\\	園芸	えんげい	
\\	演劇	えんげき	
\\	円周	えんしゅう	
\\	遠足	えんそく	〔小旅行〕
\\	〔食事を持って〕
\\	〔徒歩で行く〕
\\	学校の遠足 
\\	遠足で東京へ行った 
\\	延長	えんちょう	
\\	煙突	えんとつ	
\\	追いかける	おいかける	
\\	〔後から追う〕
\\	子供が追い掛けてきた 
\\	警官が泥棒を追い掛けた 
\\	〔せかす〕⇒おう(追う)
\\	彼は借金に追い掛けられている 
\\	〔続いて起こる〕 追い掛けるように事件が起きた 
\\	やっと一件落着したと思ったら追い掛けて別の難問題が起きた 
\\	追い越す	おいこす	
\\	〔通り越す〕
\\	他の人たちに追い越された 
\\	黒い車に追い越された 
\\	〔先んじる〕
\\	〔しのぐ〕
\\	ライバルに追い越された 
\\	始めたのは遅かったが,すぐに他の人たちを追い越した 
\\	身長は息子に追い越された 
\\	王女	おうじょ	
\\	応接	おうせつ	
\\	応接にいとまがない 問い合わせが多く,いちいち応接にいとまがない 
\\	応接にいとまがないほど志願者がある 
\\	応接係 
\\	((米)) 
\\	応接室 〔学校や会社の〕
\\	応接間 
\\	応対	おうたい	
\\	〔応接〕
\\	応対する 〔応接する〕
\\	〔相手になる〕
\\	〔店員が〕
\\	客の応対で忙しい 
\\	彼は応対が上手[下手]だ 〔人との接し方〕
\\	/〔客の扱い〕
\\	彼女は電話の応対が上手だ 
\\	往復	おうふく	
\\	欧米	おうべい	
\\	応用	おうよう	
\\	電波	でんぱ	
\\	電流	でんりゅう	
\\	電力	でんりょく	
\\	問い合わせ	といあわせ	
\\	~頭	~とう	
\\	~等	とう	とう
\\	柱が等間隔に立っている 
\\	等輩(とうはい) 
\\	とう 
\\	〔等級〕
\\	1等賞 
\\	レースで2等になった 
\\	〔など〕 暴力行為等処罰に関する法律 
\\	~島	~とう	
\\	銅	どう	
\\	同~	どう~	
\\	~道	~どう	
\\	どういたしまして (かん)	どういたしまして (かん)	
\\	統一	とういつ	
\\	同格	どうかく	
\\	峠	とうげ	
\\	統計	とうけい	統計の 
\\	統計上 
\\	出生率の統計をとる 
\\	これは信頼すべき資料による統計だ 
\\	統計学 
\\	統計学上の 
\\	統計学者 
\\	統計年鑑 
\\	統計表 
\\	動作	どうさ	
\\	東西	とうざい	
\\	〔東と西〕
\\	この通りは東西に走っている 
\\	川は東西に流れている 
\\	東西南北 
\\	英語ではこの順序で言う)/〔四方位〕
\\	〔東洋と西洋〕
\\	洋の東西を問わず 
\\	古今東西の文化 
\\	東西間の軍事協力 
\\	〔方向,進路〕 そのころは東西もわきまえない小娘でした 
\\	東西を見失う 〔方角が分からなくなる〕
\\	/〔途方に暮れる〕
\\	東西,東西 〔興行師の口上〕
\\	東西屋 〔ちんどんや〕
\\	当日	とうじつ	
\\	投書	とうしょ	
\\	登場	とうじょう	
\\	〔舞台に現れること〕
\\	舞台に登場する 
\\	ハムレット登場〔脚本用語〕 
\\	脚本(きゃくほん) 
\\	〔出現,人目を引くこと〕 彼女は彗星(すいせい)のように芸能界に登場した 
\\	折りよく名探偵(めいたんてい)が登場して事件は一挙に解決した 
\\	折りよく 
\\	このスキャンダルには2人の教授が登場する 
\\	登場人物 〔文学作品の〕
\\	〔演劇の〕
\\	どうせ	どうせ	
\\	〔いずれにせよ〕
\\	〔結局〕
\\	どうせ1日は掛かる 
\\	この額はどうせ壊れるようにできていたんだ 
\\	どうせ治らない病気だ 
\\	どうせ私は役立たずですよ 
\\	どうせそうでしょうよ 
\\	投げやりな調子) 2時間も待ったのだからどうせのことに彼が来るまでいたら 
\\	〔せいぜい〕 どうせ80点ぐらいしか取れない 
\\	どうせそんなことじゃないかと思っていた 
\\	〔いやしくも〕 どうせ行くならすぐに行きなさい 
\\	灯台	とうだい	
\\	盗難	とうなん	
\\	当番	とうばん	
\\	等分	とうぶん	等分する 
\\	等分に 
\\	彼らは土地を2等分した 
\\	遺産を4等分した 
\\	利益は君と僕とで等分しよう 
\\	透明	とうめい	
\\	灯油	とうゆ	
\\	童話	どうわ	
\\	~通り	~とおり	
\\	通り掛かる	とおりかかる	⇒とおりあわせる(通り合わせる) 事件が起きたときちょうどそこを通り掛かった 
\\	ちょうどそこへ運よく警官が通り掛かった 
\\	溶かす	とかす	
\\	とがる	とがる	
\\	〔物の先が鋭くなる〕 とがった鉛筆 
\\	先のとがった靴 
\\	〔神経が過敏(かびん)になる〕 全員の神経が針のようにとがっていた 
\\	彼はちょっとしたことにでもすぐとがる 
\\	〔声などが〕 とがった声で 
\\	どきどき	どきどき	
\\	特殊	とくしゅ	特殊な 
\\	この仕掛けは特殊な目的で作られた 
\\	回教国の中でこの国は特殊な位置を占めている 
\\	山田はその女と特殊な間柄にあった 
\\	前者は性的関係があることを表す) 特殊急襲部隊 
\\	((略
\\	特殊教育 
\\	特殊勤務 
\\	特殊鋼 
\\	特殊効果 〔映画・テレビの〕
\\	特殊合板 
\\	特殊撮影 ⇒とくさつ(特撮) 特殊事情 この条令の施行に当たっては地方の特殊事情を考慮すべきである 
\\	彼の特殊事情を考慮して税金が免除された 
\\	特殊性 〔そのものに特有の性質〕
\\	〔一般と異なる性質〕
\\	((英)) 
\\	彼の専門の特殊性 
\\	彼の才能の特殊性 
\\	特殊潜航艇 
\\	特殊取扱電報 
\\	特殊取扱郵便 
\\	書留
\\	速達
\\	など) 特殊部隊 〔米軍の〕
\\	特殊兵器 
\\	特殊法人 
\\	特色	とくしょく	⇒とくちょう(特徴) この作品には時代の特色がよく表れている 
\\	この町にはこれといった特色がない 
\\	意志の強さが彼女の特色だ 
\\	全編にみなぎる明るさが彼の作品の特色だ 
\\	首の長いのがきりんの特色だ 
\\	天衣無縫さが彼女の演奏を特色づけている 
\\	特定	とくてい	特定の 
\\	特定する 
\\	それぞれの器具には特定の用法がある 
\\	特定のボーイフレンドはいません 
\\	これこれがほしいと特定したわけではない 
\\	現状では犯人を特定するのは難しい 
\\	山火事の発火地点を特定することができない 
\\	特定財源 
\\	特定政党支持 
\\	特定政党支持率が下降している 
\\	特定人割引信用状 
\\	((略
\\	特定銘柄 
\\	特定郵便局 
\\	特売	とくばい	
\\	溶け込む	とけこむ	
\\	〔溶けて入りこむ〕
\\	この洗剤はなかなか水に溶け込まない 
\\	〔なじんで一つになる〕
\\	彼女はすぐ環境に溶け込める 
\\	クラスに溶け込む 
\\	彼の声はオルガンの調べに溶け込んだ 
\\	空は海に溶け込むように見えた 
\\	どける	退ける	
\\	⇒どかす(退かす) その足をどけなさい 
\\	どけるには及ばない 
\\	およばない 
\\	箱はみんなどけました 
\\	床の間	とこのま	
\\	所々	ところどころ	
\\	庭のところどころに花が咲(さ)いている 
\\	壁(かべ)がところどころ崩(くず)れている 
\\	都心	としん	
\\	戸棚	とだな	
\\	どっと	どっと	
\\	〔大勢が一度に立てる音〕 少年たちはどっと笑い出した 
\\	道化師は一座をどっと笑わせた 
\\	彼の冗談に聴衆はどっと笑った 
\\	〔一度に動く様子〕 人々はどっと現場に押し掛けた 
\\	各方面から問い合わせの手紙がどっと来た 
\\	こらえていた言葉がどっとせきを切って流れ出した 
\\	〔にわかに〕 どっと病の床についた 
\\	整う	ととのう	
\\	怒鳴る	どなる	
\\	殿	どの	
\\	飛び込む	とびこむ	
\\	〔身を躍らせる〕
\\	海に飛び込む 
\\	男は橋から川に飛び込んだ 
\\	〔勢いよく入る〕 弟が飛び込んで来た 
\\	〔突然入り込む〕 騒ぎの中に飛び込んでしまった 
\\	余計なやつが飛び込んで来た 
\\	[((口)) 
\\	〔参入する〕 山田警部は事件の渦中に飛び込んだ 
\\	14歳で歌謡界に飛び込んだ 
\\	留まる	とまる	
\\	〔固定される〕
\\	⇒とめる(留める) このピンでは留まらない 
\\	〔鳥が〕
\\	鳥が止まり木に留まっている 
\\	ご飯の上にハエが留まった 
\\	〔滞在する〕
\\	ハワイに1週間留まるつもりだ 
\\	〔気が付く〕 彼女の働き振(ぶ)りが目に留まった 
\\	あの時の情景がいつまでも心に留まっている 
\\	〔構(かま)える〕 お高く留まって,一体何様だと思っているんだろう 
\\	[((口)) 
\\	ともかく	ともかく	
\\	〔いずれにせよ〕 ともかく夕食にしよう 
\\	ともかく出発は延期(えんき)だ 
\\	ともかく彼は嫌いだ 
\\	〔…は別として〕 冗談はともかく,時間はきちんとしてくれないと困(こま)る 
\\	本当かどうかはともかく,そういううわさだ 
\\	ほかの事ならともかく,それだけはできない 
\\	ならともかく 
\\	顔はともかく,気立てはとてもいい 
\\	捕える	とらえる	
\\	〔捕(つか)まえる〕
\\	スパイが敵に捕らえられたようだ 
\\	警察は誘拐犯人(ゆうかいはんにん)を捕らえた 
\\	〔つかむ〕
\\	彼女は私の腕を捕らえた 
\\	垂れ下(さ)がったロープを捕らえた 
\\	〔得る,捕まえる〕 機会を捕らえて彼に話してみよう 
\\	レーダーが台風の目を捕らえた 
\\	〔把握(はあく)する〕
\\	彼女は私の言った意味[私の意味したこと]を捕らえることができなかった 
\\	この絵はモデルの特徴をよく捕らえている 
\\	真相を捕える 
\\	〔心をつかむ〕 絶望が彼を捕らえた 
\\	彼の話が私の心を捕らえた 
\\	取り入れる	とりいれる	
\\	〔取って入れる〕
\\	洗濯物を取り入れる 
\\	〔収穫する〕
\\	〔他の良い点を入れる〕 日本は西洋技術を急速に取り入れた 
\\	そのアイディアを最新作に取り入れた 
\\	少数派の意見を取り入れる 
\\	取り消す	とりけす	
\\	取り出す	とりだす	
\\	採る	とる	
\\	捕る	とる	いたちを捕る 
\\	捕らぬたぬきの皮算用はするな 
\\	丼	どんぶり	
\\	~内	~ない	
\\	内科	ないか	
\\	内線	ないせん	
\\	仲直り	なかなおり	
\\	仲直りする 
\\	もう彼らとは仲直りした 
\\	あの二人を仲直りさせるのは難しい 
\\	少女たちはすぐ仲直りした 
\\	長引く	ながびく	
\\	中指	なかゆび	
\\	仲良し	なかよし	
\\	慰める	なぐさめる	
\\	為す	なす	
\\	⇒する,おこなう(行う) なすすべを失って茫然とつっ立っていた 
\\	「なせば成る」という格言がある 
\\	息子のなすがままにさせておこう 
\\	彼はなすこともなくぶらぶらと過ごしている 
\\	捕虜は敵兵のなすがままであった 
\\	謎謎	なぞなぞ	
\\	傾らか	なだらか	
\\	懐かしい	なつかしい	
\\	なでる	なでる	〔さする〕
\\	〔愛情などの表現として軽くたたくようにさする〕
\\	あご[あごひげ]をなでる 
\\	子供の頭をなでる 
\\	髪をなでる 
\\	犬は腹をなでられると喜ぶ 
\\	風に吹き上げられた彼女のマフラーの先が私のほおをなでた 
\\	斜め	ななめ	
\\	〔傾いていること〕 斜めにする 
\\	斜めになる 
\\	斜めに傾いた板 
\\	この木は10度ばかり斜めになっている 
\\	カレンダーが斜めになっている 
\\	〔対角をなすこと〕 斜めの 
\\	斜め向かいの家 
\\	門の斜め前[後]に 
\\	斜め右の方に教会が見えてきた 
\\	斜めに線を引く 
\\	〔普通でないこと〕 ご機嫌斜めだ 
\\	ご機嫌斜めならず 
\\	なにしろ	なにしろ	
\\	何しろ話してごらん 
\\	何しろここは辺ぴな田舎ですからね 
\\	あの男は扱いにくいよ,何しろ頑固だもの 
\\	何々	なになに	
\\	〔何と何〕 何々を買ったか書いてくれ 
\\	〔なんだなんだ〕 何々,あいつが来るって 
\\	何々,なんと言ってきたかな〔手紙を読むとき〕 
\\	何々,大したことはない 
\\	何分	なにぶん	
\\	〔いくらか〕 何分のご援助をお願いしたいのですが 
\\	何分の通知があるまで待ちなさい 
\\	〔どうか〕 この事は何分よろしくお願いします 
\\	〔とにかく〕
\\	何分にも遅すぎるよ 
\\	何分にも若いからね 
\\	生意気	なまいき	生意気な 〔こしゃくな〕
\\	((口)) 
\\	((米口)) 
\\	主に若い女性に用いる); 〔厚かましい〕
\\	より口語的) 生意気なやつ 
\\	なんて生意気な 
\\	生意気千万だ 
\\	[((口))
\\	/((口))
\\	生意気な言葉 
\\	[((口))
\\	彼の生意気な態度は我慢できない 
\\	生意気なことを言う 
\\	[((米俗))
\\	生意気言うな 
\\	生意気を言うようですが,そうしないほうがいいでしょう 
\\	生意気に押し通す ((口)) 
\\	生意気にも私に指図(さしず)した 
\\	並木	なみき	
\\	倣う	ならう	
\\	南極	なんきょく	
\\	なんとなく	なんとなく	
\\	何となく虫が好かない 
\\	何となく行く気になれない 
\\	あの人には何となく変わったところがある 
\\	何となく父に反抗した 
\\	何となく彼が来そうな気がする 
\\	何となく気が沈んでいた 
\\	なんとも	なんとも	
\\	〔強調して〕 何ともお礼の申しようもありません 
\\	何ともお気の毒です 
\\	何とも楽しかった 
\\	何ともはや困ったことになった 
\\	〔否定して〕 悪口を言われても何ともない 
\\	魚は何ともないようだ 
\\	体は何ともありません 
\\	痛くも何ともない 
\\	嘘をつくことを何とも思わない奴だ 
\\	南米	なんべい	
\\	南北	なんぼく	
\\	におう	におう
\\	匂う・▲香う】	
\\	〔においがする〕
\\	〔芳香がする〕
\\	彼女のハンカチから香水がにおった 
\\	ゆりの花がよくにおう 
\\	ガス[魚]がにおう 
\\	彼の口はにおう 
\\	〔色が映える〕 
\\	朝日ににおう桜 
\\	〔感じがする〕 
\\	その取り引きには犯罪がにおう 
\\	[((口)) 
\\	逃がす	にがす	
\\	〔自由にしてやる〕
\\	鳥を逃がしてやる 
\\	どうか逃がしてください 
\\	〔捕えそこなう〕 残念ながらすりを逃がした 
\\	好機を逃がすな 
\\	逃がした魚は大きい 
\\	憎い	にくい	
\\	~難い	~がたい	
\\	憎む	にくむ	
\\	まだ彼を憎んでいるのか 
\\	罪を憎んで人を憎まず 
\\	憎むべき 
\\	略奪者(りゃくだつしゃ)たちの憎むべき暴虐(ぼうぎゃく) 
\\	憎んでも余(あま)りあるやつだ 
\\	憎んで余りある罪 
\\	いたずらだが憎めない子 
\\	憎めない 何をやっても憎めない人だ 
\\	憎らしい	にくらしい	
\\	にこにこ	にこにこ	
\\	濁る	にごる	
\\	〔透き通らなくなる〕
\\	は泥で,
\\	は沈澱物などがかき混ぜられて濁ること) 濁った水たまり 
\\	濁った酒 
\\	泥で水が濁った 
\\	濁った空気 
\\	〔鮮明でなくなる〕 濁った声 
\\	濁った色 
\\	鐘(かね)の音が濁っていた 
\\	病人は濁った目をしていた 
\\	〔けがれる〕 この濁った世の中 
\\	〔濁音(だくおん)になる〕 
\\	「ガ」は濁った音だ 
\\	虹	にじ	
\\	日時	にちじ	
\\	日用品	にちようひん	
\\	日課	にっか	
\\	日程	にってい	
\\	鈍い	にぶい	
\\	入社	にゅうしゃ	
\\	女房	にょうぼう	おれの女房 ((口)) 
\\	姉さん女房 
\\	押しかけ女房 
\\	世話女房 
\\	女房の尻にしかれている 
\\	彼女は恋女房だ 
\\	女房言葉 
\\	女房面 女房面する 
\\	女房持ち 
\\	女房役 彼は社長の女房役を10年つとめた 
\\	睨む	にらむ	
\\	⇒にらみつける(睨み付ける); 〔見詰める〕
\\	射手は的をぐっとにらんだ 
\\	〔目を付ける〕 警察ににらまれている 
\\	〔不興を買う〕 彼ににらまれるようなことをしたのか 
\\	ボスににらまれている 
\\	〔見当を付ける〕
\\	〔推断する〕
\\	〔嫌疑をかける〕
\\	費用は100万円とにらんでいる 
\\	彼は彼女が名門の出だとにらんだ 
\\	おれがにらんだことに狂いはない 
\\	彼はその男がスパイだとにらんだ 
\\	にわか	にわか	〔突然なこと〕 にわかに 
\\	にわかの変化 
\\	にわかの事で驚いた 
\\	一天にわかにかき曇った 
\\	にわかに笑い出した 
\\	にわかには返答できなかった〔すぐさま〕 
\\	試験のにわか勉強をしている 
\\	にわか仕込み にわか仕込みの資料 
\\	にわか仕込みの作法 
\\	にわか仕立て にわか仕立ての草案 
\\	にわか仕立ての委員会[チーム] 
\\	にわか作り にわか作りの納屋 〔あわてて建てた〕
\\	/〔間に合わせの〕
\\	縫う	ぬう	
\\	ねじ	ねじ	
\\	捩る	ねじる	
\\	〔よじる〕
\\	〔強力に〕
\\	紙をねじってこよりをつくる 
\\	取っ手をねじって外してしまった 
\\	体[人の手]をねじる 
\\	〔ねじを回す〕
\\	〔栓などを回す〕
\\	びんのふたをねじって開ける[締める] 
\\	栓をねじって水を出す[とめる] 
\\	熱する	ねっする	
\\	〔熱くする〕
\\	熱した鉄 
\\	熱したアイロン 
\\	液体を沸騰するまで熱する 
\\	〔熱くなる〕
\\	熱しやすい金属 
\\	〔熱中する〕⇒ねっちゅう(熱中),ねっきょう(熱狂) 熱しやすく冷めやすい 
\\	寝間着	ねまき	
\\	寝巻	ねまき	
\\	狙い	ねらい	
\\	狙う	ねらう	
\\	~年生	~ねんせい	
\\	年度	ねんど	
\\	事業年度 
\\	会計年度 ((米))
\\	/((英))
\\	年度初め[末/がわり]に 
\\	本年度の事業計画 
\\	2002年度の予算 
\\	来[先]年度〔会計年度の〕 
\\	年度が変ったので免許を更新して下さい 
\\	2002年度の卒業生 
\\	来年度の教育課程 
\\	農産物	のうさんぶつ	
\\	農村	のうそん	
\\	濃度	のうど	
\\	農薬	のうやく	
\\	能率	のうりつ	
\\	のこぎり	のこぎり	
\\	残らず	のこらず	
\\	⇒ぜんぶ(全部) 知っていることは残らずお話ししました 
\\	彼らは一人残らず不適格だと分かった 
\\	卵は一つ残らず腐っていた 
\\	大きなステーキを残らず平らげた 
\\	上り	のぼり	
\\	〔列車などの〕 次の上りは何時ですか 
\\	東名は現在上りは順調に流れています 
\\	〔上り坂〕 道は上りになっている 
\\	⇒のぼりちょうし(上り調子) 上り坂 
\\	上り線 
\\	上り列車 
\\	糊	のり	
\\	〔衣服用の〕
\\	のりで紙をはり合わせる 
\\	のりとはさみの仕事 〔つぎはぎ細工〕
\\	/〔寄せ集めで独創性のない仕事〕
\\	のり付きラベル 
\\	シャツにのりを付ける 
\\	のりのよくきいたシーツ 
\\	糊代(しろ) 
\\	乗換	のりかえ	
\\	乗り越し	のりこし	
\\	鈍い	のろい	
\\	〔遅い〕
\\	彼は足がのろい 
\\	彼は仕事が[何をやらしても]のろい 
\\	計算がのろい 
\\	〔頭が鈍い〕 頭の働きがのろい 
\\	のろのろ	のろのろ	
\\	のろのろ歩く 
\\	のろのろするな 
\\	濃霧の中を列車はのろのろと進んだ 
\\	疲れ切った観光客はのろのろとバスに戻った 
\\	呑気	のんき	のんきな 〔気楽な〕
\\	〔心配のない〕
\\	〔楽天的な〕
\\	〔不注意な〕
\\	〔安易な〕
\\	〔成り行き任せの〕
\\	のんきに暮らす 
\\	のんきそうだね 
\\	彼は生まれ付きのんきなたちだ 
\\	それに気が付かなかったとは君ものんきだね 
\\	こんな時にのんきに構えていられるか 
\\	[((俗))
\\	そんなのんきな考え方じゃ世間は渡れない 
\\	灰色	はいいろ	
\\	俳句	はいく	
\\	売店	ばいてん	
\\	売買	ばいばい	
\\	這う	はう	
\\	通例
\\	は足がないか,あっても短い生き物,腹ばいになった人間などについて用いる
\\	は四つんばいの人間や4本足の動物がゆっくり,またはひそかに進む様子) 毛虫(けむし)が枝をはっている 
\\	私たちははって進んだ 
\\	みみずがはったような字 
\\	かぼちゃの蔓(つる)が地面にはっていた 
\\	壁につたをはわせた 
\\	はがす	
\\	剥がす】	
\\	⇒はぐ(剥ぐ) 封筒(ふうとう)から切手をはがす 
\\	ばからしい	ばからしい	
\\	秤	はかり	
\\	吐き気	はきけ	
\\	はきはき	はきはき	〔きびきびと〕
\\	〔はっきりと〕
\\	少女ははきはき答えた 
\\	はきはきしない 〔無精で〕
\\	/〔活気がなくて〕
\\	/〔優柔不断で〕
\\	~泊	~はく	
\\	歯車	はぐるま	
\\	挟まる	はさまる	
\\	〔物の間に〕 食べ物のかすが歯に挟まった 
\\	着物の袖(そで)が戸に挟まった 
\\	〔対立しているものの間に〕 母と妻の間に挟まって閉口(へいこう)している 
\\	差し挟む	さしはさむ	
\\	⇒はさむ(挟む) 
\\	〔人の言動に干渉する〕 人の言葉に口をさしはさむ 
\\	私のやることに口をさしはさまないでもらいたい 
\\	〔疑いなどを抱く〕((文)) 
\\	((英)) 
\\	疑いをさしはさむ 
\\	/((文))
\\	異議をさしはさむ者はいなかった 
\\	梯子	はしご	
\\	始めに	はじめに	
\\	初めに	はじめに	
\\	はじめまして	はじめまして	
\\	はす	はす	⇒ななめ(斜め) はすに構える 
\\	はすに帽子をかぶっていた 
\\	道は鉄道をはすに横切っている 
\\	肌着	はだぎ	
\\	果して	はたして	
\\	鉢	はち	
\\	~発	~はつ	
\\	発揮	はっき	
\\	彼は実務家としての腕前(うでまえ)を発揮した 
\\	その問題は彼が実力を発揮するいい機会となった 
\\	発想	はっそう	
\\	考え方 (かんがえかた))
\\	発電	はつでん	
\\	発売	はつばい	
\\	話合い	はなしあい	
\\	話し掛ける	はなしかける	
\\	〔話を仕掛ける〕
\\	英語で話しかけられた 
\\	[((文))
\\	〔話し始める〕
\\	話し掛けてからやめるな 
\\	話中	はなしちゅう	
\\	甚だしい	はなはだしい	⇒ひじょうな(非常な),はげしい(激しい) 甚だしい間違い 
\\	[((文))
\\	それは甚だしい法律無視[特権の乱用]だ 
\\	甚だしい寒さ 
\\	甚だしい例をあげよう 
\\	公私混同も甚だしい 
\\	頭脳流出は国にとって甚だしい損失である 
\\	花火	はなび	
\\	花嫁	はなよめ	
\\	ばね	ばね	
\\	〔スプリング〕
\\	ばねの入ったマットレス 
\\	〔体のはねる力〕 ばねの利いた跳躍をした 
\\	足どりにばねがなくなった 
\\	体のばねが利かなくなった 
\\	発条仕掛け ばね仕掛けのおもちゃ 
\\	ばね仕掛けの鉄砲 
\\	発条ばかり 
\\	跳ねる	はねる	
\\	〔飛び上がる〕
\\	〔まりが〕
\\	〔馬が〕
\\	こおろぎが草の上で跳ねていた 
\\	かえるが跳ねて池の中に入った 
\\	/((英))
\\	〔とびちる〕 どろ水が靴下に跳ねた 
\\	揚げ物をしていて油が跳ねた 
\\	〔はじける〕 火の中の栗(くり)が跳ねた 
\\	炭火が跳ねた 
\\	〔興行が終わる〕 芝居は10時にはねた 
\\	破片	はへん	
\\	歯磨き	はみがき	
\\	はめる	はめる	
\\	〔ぴったりと入れる〕
\\	戸をはめる 
\\	戸にガラスをはめる 
\\	手袋をはめる 
\\	指輪をはめていた 
\\	指輪にダイヤモンドをはめる 
\\	すべての子供を同じ型にはめようとするのは誤りだ 
\\	〔わなに陥れる〕
\\	〔だます〕
\\	その計略にまんまとはめられた 
\\	早口	はやくち	
\\	払い込む	はらいこむ	
\\	払い戻す	はらいもどす	
\\	針金	はりがね	
\\	張り切る	はりきる	
\\	〔ぴんと張る〕
\\	綱はぴんと張り切っていた 
\\	〔意気込む〕 彼らは張り切っていた 
\\	彼は張り切って仕事をしている 
\\	張り切りすぎると失敗するよ 
\\	安心したとたんに張り切った気持ちがゆるんだ 
\\	反~	はん~	
\\	反映	はんえい	
\\	〔反射して映ること〕
\\	反映する 
\\	川に月の光が反映している 
\\	〔影響・本質などが外に表れること〕
\\	この小説は現代の世相を反映している 
\\	半径	はんけい	
\\	判子	はんこ	
\\	万歳	ばんざい	
\\	判事	はんじ	
\\	番地	ばんち	
\\	半島	はんとう	
\\	~番目	~ばんめ	
\\	非~	ひ~	
\\	~費	~ひ	
\\	日当たり	ひあたり	
\\	日帰り	ひがえり	
\\	日陰	ひかげ	
\\	それは日陰で干してください 
\\	日陰者 〔目立たぬ者〕
\\	〔世に入れられぬ者〕
\\	〔暗い過去を持つ人〕
\\	〔めかけ〕
\\	彼は日陰者になってしまった 
\\	彼は日陰者だと言われている 
\\	ぴかぴか	ぴかぴか	ぴかぴか光る 
\\	は金属的な物の,
\\	はぬれた物の輝きを表すことが多い) ぴかぴか光るものがすべて金ではない 
\\	無数の星がぴかぴか輝いていた 
\\	ぴかぴかの靴 
\\	靴[銀器]をぴかぴかに磨いた 
\\	引受る	ひきうける	
\\	〔責任を持って受け持つ〕
\\	仕事を引き受ける 
\\	議長の役を引き受ける 
\\	事件の処理を引き受ける 
\\	注文を引き受ける 
\\	私が引き受けよう 
\\	〔保証する〕 彼の能力は私が引き受けます 
\\	彼女の身元を引き受けた 
\\	〔手形・株式・負債などを保証する〕
\\	手形の支払いを引き受ける 
\\	長期債券の発行を引き受ける 
\\	費用の支払いを引き受ける 
\\	株を引き受ける 
\\	人の負債を引き受ける 
\\	〔世話をする〕 私たちがその子を引き受けよう 
\\	〔後を受け継ぐ〕 彼の後を引き受けて,私が事務所の主任になった 
\\	引き返す	ひきかえす	
\\	嵐のため途中から引き返した 
\\	財布を置いてきたのに気が付いて元来た道を引き返した 
\\	引算	ひきざん	
\\	引き出す	ひきだす	
\\	〔引っ張って出す〕
\\	キャビネットからファイルを引き出した 
\\	犬を小屋から引き出した 
\\	〔呼び出す〕 公聴会(こうちょうかい)に引き出された 
\\	うまくパーティーに引き出された 
\\	〔預金を出す〕 銀行から少し金を引き出した 
\\	〔金などをうまくもらう〕 友人から事業資金(しきん)を少し引き出すことが出来た 
\\	おやじをくどいてうまく旅費を引き出した 
\\	〔外に出す〕 生徒の潜在(せんざい)能力を引き出すようにしている 
\\	彼から必要な情報を引き出した 
\\	/((口)) 
\\	ここから次のような結論を引き出した 
\\	引き止める	ひきとめる	
\\	これ以上はお引き留めしません 
\\	彼に会うといつも引き留められて話し込まれる 
\\	彼女は嵐の中を出掛けようとする息子を引き留めた 
\\	卑怯	ひきょう	
\\	〔憶病〕
\\	ひきょうな 
\\	ひきょうな行為 
\\	ひきょうな振る舞いをする 
\\	今さら引っ込むなんてひきょうだ 
\\	〔卑劣〕
\\	ひきょうな 
\\	彼は私にひきょうなことをした 
\\	そりゃひきょうだよ 
\\	卑怯者 
\\	((口)) 
\\	((米俗)) 
\\	引分け	ひきわけ	
\\	陽射	ひざし	
\\	肘	ひじ	
\\	ぴたり	ぴたり	
\\	〔急に止まる様子〕 一台の車が目の前でぴたりと止まった 
\\	振り子がぴたりと止まった 
\\	風がぴたりとやんだ 
\\	その日から彼は酒をぴたりとやめた 
\\	[((口)) 
\\	それ以来彼は冗談をぴたりと言わなくなった 
\\	〔くっつく様子〕⇒ぴったり 子供は母親にぴたりとくっついていた 
\\	〔ちょうどよくあてはまる様子〕⇒ぴったり 靴が足にぴたりと合った 
\\	彼は私の考えていることをぴたりと言い当てた 
\\	引っ掛かる	ひっかかる	
\\	〔掛かって離れない〕
\\	〔鉤(はり)に〕
\\	くもの巣(す)に引っ掛かる 
\\	このペンは引っ掛かって書けない 
\\	骨がのどに引っ掛かった 
\\	つり糸が枝に引っ掛かった 
\\	〔よくないことに掛かり合う〕
\\	事件に引っ掛かる 
\\	陰謀(いんぼう)に引っ掛かった 
\\	検閲(けんえつ)に引っ掛かる 
\\	警察の検問に引っ掛かった 
\\	どこかのバーに引っ掛かったらしく主人はまだ帰らない 
\\	〔だまされる〕
\\	人の言葉に引っ掛かる 
\\	詐欺(さぎ)に引っ掛かる 
\\	もうこれ以上彼のごまかしには引っ掛からない 
\\	彼はいかがわしい女に引っ掛かった 
\\	筆記	ひっき	
\\	引っ繰り返す	ひっくりかえす	
\\	引っ繰り返る	ひっくりかえる	
\\	〔覆る〕
\\	舟が引っ繰り返った 
\\	形勢(けいせい)が引っ繰り返った 
\\	/((口)) 
\\	世の中は引っ繰り返るような大騒ぎになった 
\\	〔倒れる〕
\\	自転車で引っ繰り返る 
\\	仰向(あおむ)けに引っ繰り返る 
\\	逆様に引っ繰り返る 
\\	牛乳瓶が引っ繰り返った 
\\	寝床(ねどこ)に引っ繰り返って本を読む 
\\	〔駄目になる〕 計画が引っ繰り返った 
\\	彼が欠席したので予定が引っ繰り返った 
\\	[((口))
\\	引っ込む	ひっこむ	
\\	〔めだたない所に退く〕
\\	田舎に引っ込む 
\\	世間から引っ込む 
\\	部屋の隅へ引っ込む 
\\	家に引っ込んでいる 
\\	出そうになったくしゃみが引っ込んでしまった 
\\	彼女はいつも夫の後に引っ込んでいる 
\\	引っ込んでいろ 
\\	〔くぼむ〕
\\	引っ込んだ目 
\\	〔出ていたものが元へ戻る〕 こぶが引っ込んだ 
\\	かたつむりの角が引っ込んだ 
\\	〔奥まる〕
\\	大通りからかなり引っ込んだ家 
\\	海岸の引っ込んだ所 
\\	筆者	ひっしゃ	
\\	必需品	ひつじゅひん	
\\	人差指	ひとさしゆび	
\\	一通り	ひととおり	
\\	〔一応,だいたい〕 一通りの 
\\	彼女は一通り作法の心得(こころえ)がある 
\\	その論文に一通り目を通した 
\\	彼は英語が一通り出来る 
\\	この町は一通り見物した 
\\	〔普通〕 彼を納得(なっとく)させるのは一通りの苦労(くろう)ではない 
\\	彼は一通りの人ではない 
\\	人通り	ひとどおり	
\\	ひとまず	ひとまず	
\\	今日はひとまずこれで終わりにしましょう 
\\	ひとまず家に帰るとしよう 
\\	瞳	ひとみ	
\\	一休み	ひとやすみ	
\\	独り言	ひとりごと	
\\	ひとりでに	ひとりでに	
\\	〔主に自動的な装置で〕
\\	戸が独りでに開いた 
\\	正面のドアは前に立つと独りでに開く 
\\	〔自然に〕
\\	ほうっておいたら傷は独りでに治った 
\\	ほうっておく 
\\	皮肉	ひにく	
\\	は逆の表現を用いる皮肉
\\	には相手を傷つけようとする悪意がある
\\	は風刺) 皮肉な 
\\	皮肉に 
\\	皮肉を言う 
\\	/((口))
\\	皮肉で言ったんだ 
\\	がんの名医が自らがんで死ぬとは皮肉だ 
\\	皮肉な笑いで 
\\	彼は人間性を皮肉な目で見ている 
\\	皮肉屋 〔いやみを言う人〕
\\	〔冷笑する人〕
\\	日日	ひにち	
\\	〔日取り〕
\\	出発の日にちを決めよう 
\\	〔日数〕 風邪が治るまでだいぶ日にちが掛かった 
\\	捻る	ひねる	
\\	〔体を曲げるように回す〕
\\	彼の腕をひねった 
\\	腰をひねる 
\\	〔指先などでねじる〕 ひげをひねる 
\\	きつくひねったらドアのハンドルが取れた 
\\	鶏をひねる 
\\	ガス栓をひねって出す[止める] 
\\	スイッチをひねって電灯をつける[消す] 
\\	〔工夫をこらす,考え出す〕
\\	俳句をひねる 
\\	ひねったショーウインドーの飾り付け 〔奇妙な〕
\\	[〔技巧を弄した〕
\\	これは全くひねった問題だ 
\\	頭をひねったが打開策が見付からなかった 
\\	彼の態度には私も首をひねっている 
\\	彼はその案に少々首をひねった 
\\	〔簡単にやっつける〕 あんなやつが相手なら片手でも一ひねりでやっつけてやる 
\\	日の入り	ひのいり	
\\	日の出	ひので	
\\	響き	ひびき	
\\	響く	ひびく	
\\	〔鳴り渡る〕
\\	〔反響する〕
\\	この楽器はよく響かない 
\\	拍手が天井に響いた 
\\	彼女の声はよく響く 
\\	銃声が響いた 
\\	〔聞こえる〕 こんな言い方は変に響きますか 
\\	〔影響を与える〕 飲み過ぎが健康に響いた 
\\	悪天候が稲作に響いた 
\\	この間の失敗がよほど響いているらしい 
\\	何と言われても彼には響かない 
\\	その政策は両国の関係に大きく響くだろう 
\\	〔評判になる〕⇒ひびきわたる(響き渡る)
\\	皮膚	ひふ	
\\	百科事典	ひゃっかじてん	
\\	美容	びよう	
\\	~病	~びょう	
\\	表紙	ひょうし	
\\	標識	ひょうしき	
\\	標準	ひょうじゅん	
\\	標本	ひょうほん	
\\	評論	ひょうろん	
\\	昼寝	ひるね	
\\	広さ	ひろさ	
\\	広場	ひろば	
\\	広々	ひろびろ	
\\	便箋	びんせん	
\\	〔一つづりの〕
\\	便箋5枚に書く 
\\	瓶詰	びんづめ	
\\	~部	~ぶ	
\\	風船	ふうせん	
\\	不運	ふうん	
\\	不規則	ふきそく	
\\	普及	ふきゅう	普及する 
\\	((文)) 
\\	⇒ひろまる(広まる),ひろめる(広める) 米国では一家に2台の車が普及している 
\\	医学知識の普及に伴い伝染病が激減した 
\\	携帯電話は小学生にまで普及している 
\\	鈴木鎮一はバイオリン教育を日本中に普及させた 
\\	普及版 
\\	普及率 
\\	プレーヤーの普及率 
\\	付近	ふきん	
\\	副~	ふく~	
\\	副詞	ふくし	
\\	複写	ふくしゃ	
\\	複数	ふくすう	
\\	膨らます	ふくらます	【膨らます・▲脹らます】 
\\	〔膨張(ぼうちょう)させる〕 風が帆(ほ)を膨らませた 
\\	気球(ききゅう)を膨らます 
\\	袋に羽(はね)を入れて膨らます 
\\	ほおを膨らます 
\\	タイヤを膨らます 
\\	パンを膨らます 
\\	〔胸を一杯にする〕 希望に胸を膨らませてヒマラヤに向かった 
\\	膨らむ	ふくらむ	
\\	不潔	ふけつ	
\\	更ける	ふける	
\\	⇒よふけ(夜更け) 夜も[秋も]更けて 
\\	夜も更けてきたので結論は明日に持ち越された 
\\	夜も大分更けた 
\\	/((文)) 
\\	符号	ふごう	
\\	〔記号〕
\\	ばつ印[星印]の符号をつけてください 
\\	〔数学で〕
\\	正[負]の符号 
\\	〔コード〕
\\	符号化する 
\\	夫妻	ふさい	
\\	ふさがる	ふさがる	
\\	〔閉ざされている〕
\\	戸口がふさがっている 
\\	群衆で道がふさがってしまった 
\\	瓶の口がふさがった 
\\	〔使われている〕 その部屋はふさがっている 
\\	欠員はもうふさがった 
\\	電話がふさがっている 
\\	[((英))
\\	今日は一日中手がふさがっている 
\\	[((口))
\\	3時はふさがっていますが5時には空きます 
\\	〔とじる〕 開いた口がふさがらない 
\\	[((口)) 
\\	〔胸がいっぱいになる〕 それを聞くと胸がふさがる 
\\	ふさぐ	ふさぐ	
\\	〔閉じる〕
\\	〔つめる〕
\\	瓶の口をふさぐ 
\\	石塀の破れ目をセメントでふさいだ 
\\	さあ,みんな,口をふさいで 
\\	〔障害物で遮る〕
\\	城の入口を大きな岩でふさいだ 
\\	倒れた木が道路をふさいでいた 
\\	〔場所をとる〕
\\	だれかの荷物が席をふさいでいる 
\\	〔食欲を満たす〕 腹をふさぐ 
\\	〔果たす〕 責めをふさぐ 
\\	〔耳にふたをする〕 彼女は私が小言を言っても全く耳をふさいでいる 
\\	ふざける	ふざける	
\\	〔戯れる〕
\\	〔子供や動物がはね回って〕
\\	〔冗談を言う・行う〕
\\	ふざけて言ったんですよ 
\\	[((口))
\\	彼はふざけて僕に突き当たってきた 
\\	〔人をばかにする〕 ふざけるな 
\\	/((口)) 
\\	ふざけたやつだ 〔ばかなやつだ〕
\\	/〔生意気だ〕((口)) 
\\	〔男女がいちゃつく〕 人目も気にせずふざけ合っていた 
\\	無沙汰	ぶさた	
\\	ごぶさたをおわび申し上げます 
\\	ごぶさたしました 〔手紙で〕
\\	/〔久し振りで会って〕
\\	あの店にもずいぶんごぶさたしている 
\\	武士	ぶし	
\\	部首	ぶしゅ	
\\	襖	ふすま	
\\	附属	ふぞく	
\\	蓋	ふた	
\\	物騒	ぶっそう	
\\	〔不穏(ふおん)〕 
\\	物騒な世の中だ 
\\	〔危険〕 夜の一人歩きは物騒だ 
\\	ぶつぶつ	ぶつぶつ	
\\	〔物が煮える様子〕⇒ふつふつ1 
\\	〔次々に切れる様子〕 にんじんをぶつぶつと輪切りにする 
\\	〔刺す音・様子〕 針で紙にぶつぶつ穴を開ける 
\\	〔穴や突起 (がある様子) 
\\	鳥皮は表面がぶつぶつしている 
\\	顔にぶつぶつが出来た 
\\	オートミールの表面がぶつぶつしてきた 
\\	〔不平などを言う様子〕 何かぶつぶつ独り言を言っている 
\\	彼はいつも体の具合が悪いとぶつぶつ言っている 
\\	催し物がつまらないと皆ぶつぶつ言った 
\\	船便	ふなびん	
\\	部品	ぶひん	
\\	吹雪	ふぶき	
\\	父母	ふぼ	
\\	踏切	ふみきり	
\\	麓	ふもと	
\\	山のふもとの村 
\\	ぶらさげる	ぶらさげる	
\\	天井からランプをぶら下げる 
\\	いつもバックパックを左肩にぶら下げている 
\\	~振り	~ぶり	
\\	〔…の様子〕 彼の話しぶりが気にくわない 
\\	彼女の客に対する応対ぶりは非の打ち所がない 
\\	その会社の営業(えいぎょう)ぶりは堅実(けんじつ)だ 
\\	彼のあわてぶりを見せたかった 
\\	彼の役者としての成長ぶりには目を見張るものがあった 
\\	〔時間が経過した程度〕 
\\	彼に会うのは5年ぶりだった 
\\	3年ぶりに便りをよこした 
\\	兄が10年ぶりで故郷へ帰って来た 
\\	これは7年ぶりの豊作だ 
\\	振り仮名	ふりがな	
\\	振り向く	ふりむく	
\\	古~	ふる~	
\\	古里	ふるさと	
\\	振舞う	ふるまう	
\\	〔行動する〕 男らしく[紳士的に]振る舞う 
\\	スポーツマンらしく振る舞う 
\\	〔もてなす〕 
\\	昼食を振る舞う 
\\	彼らはよく客を呼んでごちそうを振る舞っている 
\\	風呂敷	ふろしき	
\\	ふわふわ	ふわふわ	
\\	噴火	ふんか	
\\	分解	ぶんかい	
\\	文芸	ぶんげい	〔文学と芸術〕
\\	〔文学〕
\\	〔学芸〕
\\	文芸に携わる人々 
\\	文芸映画 
\\	文芸家 
\\	文芸家協会 
\\	文芸学部 
\\	文芸作品 
\\	文芸雑誌 
\\	文芸思潮 
\\	文芸批評 
\\	文芸批評[評論]家 
\\	文芸復興 
\\	文芸復興期 
\\	文芸欄 
\\	文献	ぶんけん	〔ある主題に関する〕
\\	〔記録資料〕
\\	参考文献 
\\	/〔目録〕
\\	英語で書かれた医学の文献を調べた 
\\	この研究には豊富な文献がある 
\\	文献学 
\\	〔書誌学〕
\\	噴水	ふんすい	
\\	分数	ぶんすう	
\\	文体	ぶんたい	
\\	分布	ぶんぷ	
\\	この昆虫(こんちゅう)は世界中に分布している 
\\	この植物はアジアに分布している 
\\	分布図 
\\	分布範囲(はんい) 
\\	その方言(ほうげん)の分布範囲は秋田県から福島県に及ぶ 
\\	文房具	ぶんぼうぐ	
\\	文脈	ぶんみゃく	
\\	分脈からはそのような意味にはとれない 
\\	文脈が乱れている 
\\	分量	ぶんりょう	
\\	分類	ぶんるい	
\\	閉会	へいかい	
\\	平気	へいき	
\\	並行	へいこう	
\\	〔並んで行くこと〕 並行する 
\\	鈍行と急行列車がしばらく並行して走った 
\\	車が3台ずつ並行して進んだ 
\\	〔同時に行われること〕 二つの仕事を並行して行う 
\\	失業は不景気と並行していた 
\\	並行処理 〔コンピュータ〕
\\	並行操作 〔コンピュータ〕
\\	並行輸入 ウイスキーを並行輸入する 
\\	平日	へいじつ	
\\	兵隊	へいたい	
\\	平凡	へいぼん	
\\	平野	へいや	
\\	凹む	へこむ	
\\	へそ	へそ	
\\	隔てる	へだてる	
\\	〔時間をおく〕 それから20年隔てて彼はイギリスに渡った 
\\	〔距離をおく〕
\\	北海道は海峡で本州と隔てられている 
\\	子供たちはみな家から何百キロも隔てたところに住んでいる 
\\	〔間におく〕 隣村はあの山を隔てた向こうにある 
\\	両国の代表はテーブルを隔てて向かい合った 
\\	うちは彼の家と1軒隔てた隣です 
\\	〔疎遠にする〕 この事件は彼と友人の仲を隔てた 
\\	別荘	べっそう	
\\	別々	べつべつ	
\\	~遍	~へん	
\\	~弁	~べん	
\\	編集	へんしゅう	
\\	〔編纂(へんさん)〕
\\	本を[随筆を集めて]編集する 
\\	この本[テープ]の編集はなかなか気が利(き)いている 
\\	編集会議 
\\	編集局 
\\	編集顧問 
\\	編集者 
\\	〔編集員〕
\\	〔フィルム編集者〕
\\	編集主幹 
\\	編集長 
\\	編集部 
\\	編集部員〔集合的に〕 
\\	編集方針 
\\	便所	べんじょ	
\\	~歩	~ほ	
\\	~ぽい	~ぽい	
\\	防~	ぼう~	
\\	望遠鏡	ぼうえんきょう	
\\	方角	ほうがく	
\\	〔東西南北の方位〕
\\	⇒ほうい(方位) 南西の方角に 
\\	〔方向〕
\\	学校の方角から火災報知機の音が聞こえた 
\\	方角を知る[見失う] 
\\	私はすぐ方角を間違えてしまう 
\\	方角を間違えていますよ 
\\	この門は方角が悪い 
\\	公園はどっちの方角ですか 
\\	箒	ほうき	
\\	方言	ほうげん	
\\	坊さん	ぼうさん	
\\	防止	ぼうし	
\\	〔抑止〕
\\	防止する 
\\	火災防止 
\\	犯罪の防止 
\\	交通事故防止に努力する 
\\	青少年の非行を未然に防止する 
\\	物価上昇を防止する 
\\	防止対策 
\\	方針	ほうしん	
\\	法則	ほうそく	
\\	引力[需要供給]の法則 
\\	自然の法則 
\\	経済上の法則 
\\	人が昔偶然と見なしたものの中に今日我々は法則を見出す 
\\	包帯	ほうたい	
\\	膨大	ぼうだい	
\\	包丁	ほうちょう	
\\	方程式	ほうていしき	
\\	防犯	ぼうはん	
\\	方面	ほうめん	
\\	〔方向〕
\\	⇒ほうこう(方向); 〔地域〕
\\	どの方面に爆発の音が聞こえたのですか 
\\	その車は新宿方面に向かっている 
\\	白鳥はシベリア方面に飛び去った 
\\	銀座方面は交通がまひしている 
\\	関西方面へおいでの時はお知らせください 
\\	〔分野〕
\\	〔専門〕
\\	その方面は私の専門ではない 
\\	各方面の人が集まっていた 
\\	各方面から問い合わせがあった 
\\	その問題はいろいろな方面から検討しよう 
\\	坊や	ぼうや	
\\	〔男の子〕
\\	〔息子〕
\\	〔呼び掛け〕
\\	((口)) 
\\	坊や,年はいくつ 
\\	坊や,いい子だね 
\\	〔世間なれしていない若い男〕
\\	あの坊やにはいい薬になっただろう 
\\	[((口))
\\	放る	ほうる	
\\	〔投げる〕
\\	犬に肉片をほうってやった 
\\	松の木ををめがけて石を勢いよくほうった 
\\	硬貨をほうって行くかどうか決めた 
\\	〔あきらめてやめる〕
\\	⇒ほうりだす(放り出す)
\\	〔放置してかえりみない〕
\\	〔かまわずにおく〕
\\	⇒ほうりだす(放り出す)
\\	ほうっておく(放って置く)
\\	朗らか	ほがらか	
\\	〔快活な様子〕 朗らかな 
\\	朗らかに 
\\	朗らかな人だ 
\\	朗らかに笑った 
\\	〔晴れわたった様子〕 朗らかな青空 
\\	/((文)) 
\\	牧場	ぼくじょう	
\\	牧畜	ぼくちく	
\\	保健	ほけん	
\\	この町では住民の保健に力を入れている 
\\	世界保健機関 
\\	((略
\\	保健衛生 
\\	は主として清潔さに,
\\	は汚物の排除などに主眼をおく) 保健士 
\\	保健事業 
\\	保健室 〔学校などで〕
\\	((主に英)) 
\\	保健指導 
\\	保健所 
\\	保健体育 〔学科〕
\\	保健婦 
\\	〔学校の〕
\\	保健物理学 
\\	募集	ぼしゅう	
\\	干す	ほす	
\\	北極	ほっきょく	
\\	坊っちゃん	ぼっちゃん	
\\	掘る	ほる	
\\	彫る	ほる	
\\	ぼろ	ぼろ	
\\	盆	ぼん	
\\	盆地	ぼんち	
\\	ほんの~	ほんの~	
\\	本部	ほんぶ	
\\	本来	ほんらい	
\\	〔元来〕
\\	〔初めから〕
\\	⇒がんらい(元来) 本来の 
\\	この鉄道は本来鉄鉱石運搬のために敷設された 
\\	当校は本来女子の高等教育機関である 
\\	〔本質的に〕
\\	嫉妬(しっと)は本来善でも悪でもないというのが彼の説だ 
\\	〔生来〕
\\	犬は本来肉食動物だ 
\\	本来有害な薬でも使い方によっては効き目がある 
\\	〔当たり前〕 本来なら許されないところだ 
\\	私がこの土地の本来の所有者だ 
\\	まあまあ	まあまあ	
\\	枚数	まいすう	
\\	毎度	まいど	
\\	〔そのたびごと〕
\\	〔たびたび〕
\\	〔常に〕
\\	毎度同じ事を言う 
\\	毎度ご厄介をお掛けします 
\\	毎度ごひいき有り難うございます 
\\	「彼がまた文句を言っているよ」「毎度のことさ」 
\\	枕	まくら	
\\	曲げる	まげる	
\\	〔まっすぐの物をたわめる〕
\\	腰を曲げる 
\\	曲げやすい 
\\	針金を曲げて輪にする 
\\	〔歪める〕 帽子をわざと曲げてかぶる 
\\	粋に) 
\\	〔真の姿を歪める〕 意味を曲げて取る 
\\	真実を曲げる 
\\	主義を曲げる 
\\	結局彼は志を曲げてしまった 
\\	どうしても彼は意見を曲げようとしない 
\\	まごまご	まごまご	
\\	〔どうしてよいかわからない〕 出口がわからずまごまごした 
\\	〔ぐずぐずする〕 まごまごしていて列車に乗り遅れた 
\\	〔気を付けないと〕 まごまごしていると彼にしてやられるよ 
\\	摩擦	まさつ	
\\	またぐ	またぐ	
\\	垣根をまたぐ 
\\	溝をまたぐ 
\\	敷居をまたぐ 
\\	待合室	まちあいしつ	
\\	待ち合わせる	まちあわせる	
\\	街角	まちかど	
\\	真っ暗	まっくら	
\\	真っ黒	まっくろ	
\\	真っ青	まっさお	
\\	真っ先	まっさき	
\\	〔先頭〕 行列の真っ先にいる 
\\	彼はいつも真っ先に立ってやる 
\\	〔最初〕 真っ先に 〔最初に〕
\\	〔何よりも〕
\\	真っ先にそう言ったじゃないか 
\\	真っ先に金を作らねばならない 
\\	朝起きると真っ先に植木に水をやる 
\\	彼は真っ先に立ち上がった 
\\	真っ白	まっしろ	
\\	祭る	まつる	
\\	〔儀式を行って神霊を慰める〕
\\	祖先の霊を祭る 
\\	〔神としてあがめる〕
\\	〔安置する〕
\\	この神社は明治天皇を祭ってある 
\\	窓口	まどぐち	
\\	真似る	まねる	
\\	まぶた	まぶた【目×蓋・×瞼】	
\\	二重[一重]まぶた 
\\	上[下]まぶた 
\\	まぶたを閉じる 
\\	子供はまぶたが重そうだった 
\\	〔記憶〕  父の面影がまぶたに残っている 
\\	あの光景がまぶたに浮かんだ 
\\	瞼の母 
\\	間も無く	まもなく	
\\	稀 (希)	まれ	まれな 
\\	まれに見る美人 
\\	それは何もまれなことではない 
\\	まれな出来事 
\\	それを知っている人はまれだ 
\\	そんな事件はまれにしか起こらない 
\\	希な待遇(たいぐう) 
\\	そんな機会はきわめてまれであった 
\\	回り道	まわりみち	
\\	満員	まんいん	【例文】 満員の聴衆 
\\	満員のバス 
\\	席は満員です 
\\	ホテルは満員でした 
\\	列車は超満員であった 
\\	劇場は満員である 
\\	満員〔劇場の掲示〕 
\\	満点	まんてん	
\\	未~	み~	
\\	見上げる	みあげる	
\\	〔上を見る〕
\\	天井を見上げる 
\\	見上げるように高い 
\\	見上げるような大男 
\\	〔「見上げた」の形で,尊敬する〕 見上げた 
\\	見上げた心掛け[態度] 
\\	見送る	みおくる	
\\	〔去る人を送る〕
\\	門の前で客を見送った 
\\	家まで見送る 
\\	玄関まで見送る 
\\	角を曲がるまで見送った 
\\	空港で盛大に見送る 
\\	彼を柱の陰から見送った 
\\	〔死者を送る〕 私は18歳の時に母を見送った 
\\	〔やり過ごす〕
\\	込んでいたのでバスを1台見送った 
\\	チャンスを見送る 
\\	ボールを見送る 〔野球でバッターが〕
\\	/〔外野手などがホームランを〕
\\	〔そのままにする〕 法案を見送る 
\\	しばらく新規採用を見送ることにした 
\\	見下ろす	みおろす	
\\	〔下を見る〕
\\	屋根の上から猫が私を見下ろしていた 
\\	ホテルから町が見下ろせる 
\\	〔見下す〕⇒みくだす(見下す)
\\	見掛け	みかけ	
\\	三日月	みかづき	
\\	岬	みさき	
\\	惨め	みじめ	惨めな 
\\	は他人の目に哀れに見える) 惨めさ 
\\	惨めな生活をする 
\\	惨めな境遇(きょうぐう)にある 
\\	惨めな最期(さいご)でした 
\\	最期 
\\	自ら	みずから	〔自分自身で〕
\\	〔本人が直接に〕
\\	自らの 
\\	自ら名乗る 
\\	自ら彼に話した 
\\	自ら命を絶(た)つ 
\\	自ら進んでそれをした 
\\	水着	みずぎ	
\\	店屋	みせや	
\\	~みたい	~みたい	
\\	見出し	みだし	
\\	〔索引〕
\\	((複 〜
\\	〔目次〕
\\	〔標題〕
\\	〔新聞の〕
\\	〔辞書の〕
\\	小見出し 
\\	大見出し〔新聞の〕 
\\	道順	みちじゅん	
\\	みっともない	みっともない	
\\	〔恥ずべき〕
\\	〔不体裁な〕
\\	みっともない格好をする 〔ふさわしくない〕
\\	/〔みすぼらしい〕
\\	みっともない,いい年をして 
\\	いい年 
\\	彼は全くみっともなかった 
\\	みっともないからそんなことやめなさい 
\\	そんなことをしてみっともないと思いませんか 
\\	ワイシャツ姿で夕食会に出るのはみっともない 
\\	〔醜い〕
\\	見詰める	みつめる	
\\	見直す	みなおす	
\\	〔再び見る・調べる〕 それをもう一度見直してみた 
\\	もっとよく見直してみなさい 
\\	〔再検討する〕 その計画は見直す必要がある 
\\	〔今までより高く評価する〕 私は手紙を読んで彼を見直した 
\\	彼を見直さなければなるまい 
\\	〔好転する〕
\\	株式市場は見直すだろう 
\\	見慣れる	みなれる	
\\	醜い	みにくい	
\\	実る	みのる	
\\	〔実がなる〕
\\	この地方ではオレンジは実らない 
\\	今年はぶどうがよく実った 
\\	〔よい結果を生む〕
\\	私たちの努力は結局実らなかった 
\\	長い間の研究が実った 
\\	身分	みぶん	
\\	〔地位〕
\\	身分のある人 
\\	身分の違い 
\\	彼は10年間身分が変わらなかった 
\\	〔資力〕
\\	身分相応[不相応]に暮らす 
\\	〔境遇〕
\\	裕福な身分の人 
\\	結構なご身分ですね 
\\	学生の身分でベンツはぜいたくすぎる 
\\	〔素性〕
\\	身分の高い[低い]人 
\\	身分を明かす 
\\	身分を証明するものを何かお持ちですか 
\\	身分証明書 
\\	見本	みほん	
\\	見舞う	みまう	
\\	〔人を尋ねて慰める〕
\\	入院中の先生を皆で見舞った 
\\	〔悪い物事が襲う〕 不幸に見舞われる 
\\	この漁村はたびたび津波に見舞われる 
\\	あいつに一発見舞ってやりたい 
\\	[((口)) 
\\	未満	みまん	
\\	名字	みょうじ	
\\	民間	みんかん	
\\	民主~	みんしゅ~	民主的な 
\\	この会社は民主的な運営が行われている 
\\	民主化 
\\	民主化する 
\\	彼らは会社の民主化を要求している 
\\	民主国家 
\\	民主主義 
\\	民主政治 
\\	民主政体 
\\	民主党 〔米国の〕
\\	〔日本の〕
\\	((略
\\	民謡	みんよう	
\\	無限	むげん	
\\	無地	むじ	無地の布 
\\	無地のスカート 
\\	茶色の無地のカーテン 
\\	矛盾	むじゅん	
\\	無数	むすう	
\\	紫	むらさき	
\\	群れ	むれ	
\\	姪	めい	
\\	名~	めい~	
\\	~名	~めい	
\\	名作	めいさく	
\\	名所	めいしょ	
\\	命ずる	めいずる	
\\	迷信	めいしん	
\\	名物	めいぶつ	
\\	銘々	めいめい	
\\	恵まれる	めぐまれる	
\\	〔自然に〕 その島は海の資源に恵まれている 
\\	〔才能などに〕 彼は立派な体格に恵まれている 
\\	恵まれた才能 
\\	〔境遇に〕 恵まれない家庭の子供たち 
\\	恵まれた環境に育つ 
\\	恵まれた生活を送っている 
\\	それは少数の恵まれた人たちだけに有利な法律だ 
\\	〔好都合に〕 その1週間よい天気に恵まれた 
\\	チャンスに恵まれた 
\\	巡る	めぐる	
\\	〔ぐるりと回る〕⇒まわる(回る) 月は地球の周(まわ)りを巡る 
\\	〔周りを囲む〕 池の周囲(しゅうい)を細(ほそ)い道が巡っている 
\\	〔回って元へ戻る〕 春がまた巡ってきた 
\\	血液は体内を巡る 
\\	歴史は巡る 
\\	〔あちらこちらと回り歩く〕 ヨーロッパ諸国を巡る1か月の旅行 
\\	〔ある問題を中心にする〕 遺産(いさん)を巡って争(あらそ)う 
\\	金の出所(しゅっしょ)を巡るうわさ 
\\	目指す	めざす	
\\	目覚し	めざまし	
\\	目下	もっか、めした	もっか 
\\	目下調査中です 
\\	目下のところ何も問題はない 
\\	目下のところこれで間に合うでしょう 
\\	めした 〔地位が下の人〕
\\	〔年下の人〕
\\	目印	めじるし	
\\	目立つ	めだつ	
\\	めちゃくちゃ	めちゃくちゃ	
\\	〔道理に合わない様子〕 めちゃくちゃな 
\\	〔筋道が立たない〕
\\	〔向こう見ずな〕
\\	〔度外れの〕
\\	めちゃくちゃな値段をつける 
\\	めちゃくちゃな文章 
\\	めちゃくちゃな論法 
\\	彼の言うことなすことめちゃくちゃだ 
\\	賊はめちゃくちゃに発砲した 
\\	その映画はめちゃくちゃに面白い 
\\	〔ひどく壊れた様子〕 ガラスがめちゃくちゃに割れた 
\\	[((口))
\\	あらしで村がめちゃくちゃになった ((文)) 
\\	〔乱雑な様子〕 書類の順序がめちゃくちゃだ 
\\	彼の経理はめちゃくちゃだった 
\\	〔駄目になる様子〕 夕立でハイキングはめちゃくちゃになった 
\\	もう私の人生はめちゃくちゃです 
\\	彼の不注意のため計画はめちゃくちゃになった 
\\	めっきり	めっきり	
\\	めっきり涼しくなった 
\\	彼はめっきり白髪(しらが)がふえた 
\\	めでたい	めでたい	
\\	〔祝うべき,喜ばしい〕
\\	こんなめでたい事はない 
\\	めでたい結末でなによりだ 
\\	このおめでたい席に出席できて幸せです ((文)) 
\\	難問がめでたく解決した 
\\	めでたし,めでたし 〔おとぎ話などの結びの文句〕
\\	/〔万事うまくいった〕
\\	おめでとう ⇒詳細を見る 
\\	〔間抜けな〕⇒おめでたい(御目出度い)2 
\\	〔信頼が厚い〕 彼は社長の覚えがめでたい[めでたくない] 
\\	めまい	めまい	めまいがする 
\\	病気になる前よくめまいがした 
\\	二日酔いのせいかめまいがする 
\\	目安	めやす	〔めど〕
\\	〔基準〕
\\	目安を立てる 
\\	⇒めど(目処) 
\\	免税	めんぜい	
\\	面積	めんせき	
\\	面倒臭い	めんどうくさい	
\\	⇒めんどう(面倒) 面倒臭いからそんなことはしない 
\\	面倒臭がらず毎日ジョギングをしなさい 
\\	彼女は面倒臭がって自分のお茶も入れない 
\\	面倒臭い話はごめんだ 
\\	ああ面倒臭い 
\\	!/((口)) 
\\	儲かる	もうかる	
\\	〔利益がある〕((人が主語で)) 
\\	((物が主語で)) 
\\	500万円もうかった 
\\	この商売はもうかる[もうからない] 
\\	〔得をする〕 田中さんが車に乗せてくれてもうかった 
\\	儲ける	もうける	
\\	〔利益を得る〕
\\	彼はその取り引きで大金をもうけた 
\\	戦争でもうける人もある 
\\	1年に2千万円もうけた 
\\	駅まで歩いて240円もうけた 
\\	〔子供を得る〕 一子をもうける 
\\	/〔母親が〕
\\	〔思いがけず得をする〕 わーい,もうけた,2時間目は休講だ 
\\	申し訳ない	もうしわけない	
\\	木材	もくざい	
\\	目次	もくじ	
\\	潜る	もぐる	
\\	〔水中に〕
\\	くじらは長時間潜っていられる 
\\	〔物の下に〕⇒もぐりこむ(潜り込む) 
\\	〔隠れる〕
\\	危険を感じて彼は地下に潜った 
\\	もしかしたら	もしかしたら	
\\	もたれる	もたれる	
\\	〔寄り掛かる〕
\\	壁にもたれる 
\\	机にもたれる 
\\	いすの背にもたれていた 
\\	〔胃にたまる〕 これは胃にもたれる 
\\	餅	もち	
\\	~もち	~もち	
\\	元々	もともと	
\\	物置	ものおき	
\\	物語る	ものがたる	
\\	〔話をする〕
\\	〔報告・説明をする〕
\\	彼は事故の一部始終(しじゅう)を物語った 
\\	〔示す〕
\\	これは彼の性格[勤勉(きんべん)さ]を物語っている 
\\	その光景(こうけい)は爆発のすさまじさを物語っていた 
\\	[((文))
\\	すさまじい 
\\	物差し	ものさし	
\\	〔定規〕
\\	〔測定器〕
\\	棒の長さを物差しで測る 
\\	1メートルの物差し 
\\	1ヤードの物差し 
\\	〔判断の基準〕 出版物の量は文化の一つの物差しだ 
\\	彼女の行動は普通の物差しでは測れない 
\\	物凄い	ものすごい	
\\	〔恐ろしい〕
\\	〔ぞっとする〕
\\	ものすごい形相(ぎょうそう)で 
\\	ものすごい光景(こうけい)だった 
\\	〔程度がはなはだしい〕 ものすごい借金 
\\	ものすごい混乱 
\\	ものすごい音 
\\	ものすごいあらし 
\\	ものすごく美しい人 
\\	紅葉	こうよう 
\\	もみじ	こうよう 
\\	[((米)) 
\\	紅葉する 
\\	全山燃えるような紅葉だ 
\\	かえでが紅葉した 
\\	もみじ 
\\	〔かえで〕
\\	〔紅葉〕
\\	紅葉狩り 
\\	山へもみじ狩りに行った 
\\	もむ	もむ	
\\	〔もみくちゃに〕
\\	〔こするようにして〕
\\	〔あんまをする〕
\\	新聞紙をもんで柔らかくする 
\\	肩をもんでもらう 
\\	塩できゅうりをもむ 〔1本のきゅうりを〕
\\	/〔薄切りのきゅうりを〕
\\	板にきりをもむ 
\\	〔盛んに議論する〕 彼らは改正案をもみにもんだ 
\\	〔相手になって鍛える〕 よし一丁もんでやろう 
\\	催し	もよおし	
\\	盛る	もる	
\\	〔積み上げる〕
\\	砂を盛る 
\\	盆にオレンジを盛る 
\\	りんごを山と盛った盆 
\\	〔器に入れる〕
\\	(飯を) 盛る 
\\	〔薬などを調合して飲ませる〕 毒を盛る 
\\	~問	~もん	
\\	問答	もんどう	
\\	~夜	~や	
\\	やかましい	やかましい	
\\	〔音が大きくてうるさい〕
\\	やかましいラジオの音 
\\	隣室(りんしつ)の人がやかましくてよく眠れなかった 
\\	やかましいぞ! 
\\	〔文句をつける〕 公害に対する世論がやかましい 
\\	/((英))
\\	それはやかましい社会問題である 
\\	新しい鉄道建設(けんせつ)に地元がやかましく反対した 
\\	〔厳しい〕
\\	やかましい先生 
\\	近ごろ警察がやかましくて路上駐車が出来ない 
\\	〔気むずかしい〕 彼はワインの味にやかましい 
\\	/((口))
\\	彼女は服装[作法]にやかましい 
\\	母は口やかましいしゅうとめによく仕(つか)えた 
\\	〔小言(こごと)をいう〕 
\\	彼はやかましく言われないと何もしない 
\\	[((口)) 
\\	夜間	やかん	
\\	やかん	やかん	
\\	役者	やくしゃ	
\\	役所	やくしょ	
\\	役人	やくにん	
\\	薬品	やくひん	〔医薬品〕
\\	((米)) 
\\	は病気を治療するもの
\\	は治療するもの,有毒なもの,眠気を催すものなどその効能は様々である); 〔化学薬品〕
\\	役目	やくめ	
\\	火傷	やけど	
\\	夜行	やこう	
\\	矢印	やじるし	
\\	やたらに	やたらに	〔過度に〕
\\	〔無差別に〕
\\	〔手当り次第に〕
\\	〔めくら滅法に〕
\\	チョコレートをやたらに食べる 
\\	/((米口)) 
\\	やたらに冷たい水を飲むとおなかをこわしますよ 
\\	今日はやたらに眠い 
\\	犯人はやたらに発砲した 
\\	彼女はやたらに洋服を新調する 
\\	やたらにお世辞をいうのはやめなさい 
\\	やたらに感嘆詞を使う人だ 
\\	やたらに金を使う 
\\	やたらに金を借りて回った 
\\	薬局	やっきょく	
\\	やっつける	やっつける	
\\	〔負かす〕
\\	〔殺す〕((口)) 
\\	〔こらしめる〕((口)) 
\\	冷静な理論で彼女は相手をやっつけた 
\\	悪徳業者をやっつけろ 
\\	一発ではえをやっつけた 
\\	いたずら小僧をやっつけた 
\\	〔こきおろす〕
\\	((口)) 
\\	マスコミは政府の新政策をこっぴどくやっつけた 
\\	〔片付ける〕 この仕事を今日中にやっつけなくちゃ 
\\	家主	やぬし	
\\	やっぱり、やはり	やっぱり	
\\	〔同様に〕
\\	((否定を伴って)) 
\\	彼もやはり冒険家だ 
\\	私だってやはりそうです 
\\	/((口))
\\	父もやはりにんじんが嫌いです 
\\	〔依然として〕
\\	涼しくてもやはり夏は夏だ 
\\	彼女は今でもやはり美しい 
\\	お金があってもやはり死ぬときは死ぬ 
\\	ごみごみしていてもこの町はやはり好きだ 
\\	〔思ったとおり〕
\\	うわさはやはりデマだった 
\\	やはり雨になった 
\\	やはり本当だった 
\\	やはり言ったとおりだろう 
\\	破く	やぶく	
\\	やむをえない	やむをえない	〔避けられない〕
\\	〔必然的な〕
\\	やむを得ない事情で行かれなかった 
\\	そうなった以上離婚はやむを得なかった 
\\	ある程度の損失(そんしつ)はやむを得なかった 
\\	軟らかい	やわらかい	
\\	〔固くない〕
\\	柔らかいクッション 
\\	柔らかい肉 
\\	御飯を柔らかめに炊いた 
\\	〔柔軟な〕
\\	彼女の体は柔らかい 
\\	泳ぐ前に体操をして体を柔らかくしなさい 
\\	熱に当てたらゴムが柔らかくなった 
\\	〔穏やかな〕
\\	柔らかい声で話す 
\\	柔らかい態度で 
\\	家具は柔らかい色が好きだ 
\\	木々に柔らかい日ざしが降り注いでいた 
\\	このランプの笠だと光が柔らかくなる 
\\	〔軽い〕
\\	もっと軟らかい話をしよう 
\\	軟らかいものもたまには読みます 
\\	〔融通性がある〕 彼は頭が柔らかい 
\\	遊園地	ゆうえんち	
\\	夕刊	ゆうかん	
\\	郵送	ゆうそう	
\\	夕立	ゆうだち	
\\	夕日	ゆうひ	
\\	悠々	ゆうゆう	
\\	有料	ゆうりょう	その動物園は子供は無料だが,大人は有料だ 
\\	このサービスは有料です 
\\	有料試写会 
\\	有料駐車場 
\\	/((英)) 
\\	有料テレビ 
\\	有料道路 
\\	有料便所 
\\	浴衣	ゆかた	
\\	行方	ゆくえ	
\\	湯気	ゆげ	
\\	輸血	ゆけつ	
\\	輸送	ゆそう	
\\	油断	ゆだん	
\\	湯飲み	ゆのみ	
\\	緩い	ゆるい	
\\	〔きつくない〕
\\	緩い結び目[ボタン/包帯] 
\\	靴のひもを緩く結ぶ 
\\	〔急ではない〕 緩いカーブ 
\\	道はその先で緩くカーブしている 
\\	海岸まで緩い勾配(こうばい)になっている 
\\	この辺は川の流れが緩い 
\\	〔ゆっくりとした〕 緩い歩調で 
\\	〔固くない〕 緩い便 
\\	この一帯は地盤が緩い 
\\	〔厳重でない〕 緩い取り締まり 
\\	緩い管理 
\\	溶岩	ようがん	
\\	用語	ようご	
\\	要旨	ようし	
\\	君は要旨を全くつかんでいない 
\\	講演の要旨を話してください 
\\	話の要旨がわからなかった 
\\	幼児	ようじ	
\\	容積	ようせき	
\\	幼稚	ようち	
\\	幼稚園	ようちえん	
\\	用途	ようと	
\\	小麦粉の用途は広い 
\\	集められた金の用途は誰も知らなかった 
\\	洋品店	ようひんてん	
\\	養分	ようぶん	
\\	豆腐は養分が多い 
\\	寄生植物(きせいしょくぶつ)は宿主(しゅくしゅ)から養分を吸収(きゅうしゅう)する 
\\	松茸(まつたけ)は養分はあまりない 
\\	土は植物に養分を与える 
\\	羊毛	ようもう	
\\	漸く	ようやく	
\\	〔やっと〕
\\	ようやく結婚した 
\\	ようやく平和が戻ってきた 
\\	〔だんだんに〕
\\	ようやく秋もたけなわになった 
\\	〔かろうじて〕
\\	ようやく列車に間に合った 
\\	ようやく危険を免れた 
\\	彼はようやく試験に通った 
\\	/((口)) 
\\	要領	ようりょう	
\\	翌~	よく~	
\\	欲張り	よくばり	
\\	余計	よけい	
\\	よこす	よこす	
\\	〔こちらに送る〕
\\	〔こちらに渡す〕
\\	妹が手紙をよこして病気だと言って来た 
\\	その拳銃(けんじゅう)をよこすんだ 
\\	〔人を来させる〕 
\\	3日手伝いをよこしてくださいませんか 
\\	いつでも遠慮なくお子さんをプールに泳ぎによこしてください 
\\	余所	よそ	
\\	〔他の場所〕
\\	あの男はどこかよそで見掛けたことがある 
\\	彼への贈り物はどこかよそで買いましょう 
\\	よそに泊まる 
\\	今日はよそで食事をしてくる 
\\	〔わきの方〕 あいつは私が話す間よそを向いていた 
\\	〔他人〕 よその子 
\\	よその国 
\\	よそ様のことは気にするな 
\\	赤ん坊をよそに預けて仕事をしている 
\\	〔顧みないこと〕 両親の心配[仕事]をよそにして彼は自動車レースに夢中になった 
\\	皆の騒ぎをよそに,彼は小説に読みふけっていた 
\\	四つ角	よつかど	
\\	四つ角を右へ曲がりなさい 
\\	次の四つ角に公衆(こうしゅう)電話がある 
\\	酔っ払い	よっぱらい	
\\	予備	よび	
\\	予備の 〔とっておきの〕
\\	〔準備的段階の〕
\\	予備の食糧 
\\	まさかの時のために予備のお金をとっておく 
\\	客用の予備の部屋がない 
\\	予備の調査をする 
\\	電池が切れるかもしれない
\\	予備を持っていこう 
\\	予備役 
\\	予備役将校 
\\	予備金 
\\	予備軍 
\\	予備校 
\\	予備工作 
\\	予備交渉 
\\	予備交渉をする 
\\	予備選挙 
\\	((米)) 
\\	予備知識 
\\	予備調査 
\\	予備費 
\\	7万円を予備費にあてた 
\\	予備品 〔予備の部品〕
\\	〔非常準備品〕
\\	〔予備の貯蔵品〕
\\	呼び掛ける	よびかける	
\\	呼び出す	よびだす	
\\	蘇る	よみがえる	
\\	〔生き返る〕
\\	よみがえらせる 
\\	((文)) 
\\	/〔仮死状態の人を〕
\\	〔元気を取り戻す〕
\\	1杯の茶でよみがえった気持ちになった 
\\	〔元どおりになる〕 昔の記憶が徐々(じょじょ)によみがえってきた 
\\	あの甘い思い出がよみがえった 
\\	その光景がはっきりよみがえった 
\\	慶ぶ	よろこぶ	よろこ・ぶ【喜ぶ/
\\	悦ぶ/
\\	歓ぶ/
\\	慶ぶ】 
\\	来日	らいにち	
\\	落第	らくだい	
\\	欄	らん	
\\	乱暴	らんぼう	
\\	〔暴力を振るうこと〕
\\	ならず者たちは彼に乱暴を働いた 
\\	女性に乱暴する 
\\	〔粗野であること〕 乱暴な 
\\	乱暴な言葉 
\\	乱暴な運転 
\\	彼は動作が乱暴だ 
\\	言葉遣いが乱暴だ 
\\	そういう話し方は女性に対しては乱暴だよ 
\\	本を乱暴に扱(あつか)う 
\\	ドアを乱暴にしめた 
\\	そんな解釈の仕方は乱暴だ 
\\	乱暴者 
\\	((英)) 
\\	理科	りか	
\\	〔教科〕
\\	〔大学の部門〕
\\	理科系に進むつもりだ 
\\	理科大学 
\\	利害	りがい	
\\	略す	りゃくす	
\\	~流	~りゅう	
\\	流域	りゅういき	
\\	両~	りょう~	
\\	~料	~りょう	
\\	~領	~りょう	
\\	両側	りょうがわ	
\\	漁師	りょうし	
\\	領事	りょうじ	
\\	領収	りょうしゅう	
\\	臨時	りんじ	臨時の 〔特別の〕
\\	〔一時的〕
\\	臨時に 
\\	臨時記号 〔音楽の〕
\\	臨時休業 
\\	臨時休業いたします〔掲示〕 
\\	臨時休校 
\\	臨時国会 
\\	臨時昇給 
\\	臨時招集 
\\	臨時所得 
\\	臨時政府 
\\	臨時総会 
\\	臨時増刊 
\\	臨時手当 
\\	臨時ニュース 
\\	〔速報〕
\\	臨時費 
\\	臨時雇い 
\\	〔人〕
\\	((口)) 
\\	臨時雇いになる 
\\	臨時予算 
\\	臨時列車 スキー臨時列車 
\\	留守番	るすばん	
\\	零点	れいてん	
\\	冷凍	れいとう	
\\	列島	れっとう	
\\	煉瓦	れんが	
\\	連合	れんごう	
\\	ろうそく	ろうそく	
\\	録音	ろくおん	
\\	~論	~ろん	
\\	論ずる	ろんずる	
\\	和~	わ~	
\\	~羽	~わ	
\\	和英	わえい	
\\	我~	わが~	
\\	若々しい	わかわかしい	
\\	詫びる	わびる	
\\	彼に無礼[遅れてきたこと]をわびた 
\\	⇒あやまる(謝る) 〔わびる〕
\\	彼は店主に漫画本を万引きしたことを謝った 
\\	彼は一応「謝る」とは言いました 
\\	和服	わふく	
\\	割合に	わりあいに	
\\	大きい割合に軽い 
\\	割算	わりざん	
\\	割と	わりと	
\\	割引	わりびき	
\\	段階	だんかい	
\\	〔等級〕
\\	4つの段階に分ける 
\\	この品には
\\	から
\\	までの段階がある 
\\	五段階評価 
\\	〔過程の一時期〕
\\	〔局面〕
\\	今の段階では 
\\	戦争は最後の段階に入っていた 
\\	研究はまだ実験の段階である 
\\	交渉は新しい段階に入った 
\\	段階を踏んで説明する 
\\	インフレは段階的に進んでいる 
\\	当局はその制度を段階的に導入[廃止]する予定である 
\\	短期	たんき	
\\	炭鉱	たんこう	
\\	短所	たんしょ	
\\	たんす	たんす	
\\	淡水	たんすい	
\\	断水	だんすい	
\\	単数	たんすう	
\\	団地	だんち	
\\	断定	だんてい	
\\	短編	たんぺん	
\\	誓う	ちかう	〔神などにかけて〕
\\	〔厳粛な誓いを立てる〕
\\	〔宣誓する〕
\\	しばしば聖書に手を置いて); 〔約束する〕
\\	は固く約束する) 神かけて身の潔白(けっぱく)を誓います 
\\	彼は禁酒を誓った 
\\	私の秘密をだれにも言わないと誓ってくれますか 
\\	あの二人は将来を誓った仲です 
\\	地下水	ちかすい	
\\	近々	ちかぢか	
\\	彼は近々結婚する 
\\	近々帰国しなければならない 
\\	近々東京に引っ越します 
\\	近付ける	ちかづける	
\\	〔接近させる〕
\\	本を目に近付ける 
\\	敵を近付けない 
\\	机を窓に近付けた 
\\	この材質(ざいしつ)は火に近付けないこと 
\\	〔交わる〕
\\	あの人は近付けない方がよい 
\\	彼は人を近付けない男だ 
\\	彼女は男を全く近付けない 
\\	近寄る	ちかよる	
\\	力強い	ちからづよい	
\\	ちぎる	ちぎる	〔細かく裂く〕
\\	〔もぎ取る〕
\\	手紙を細かくちぎって捨てた 
\\	メモ用紙を1枚ちぎった 
\\	ぶどうを一房(ひとふさ)ちぎってくれた 
\\	地質	ちしつ	
\\	知人	ちじん	
\\	地帯	ちたい	
\\	縮む	ちぢむ	
\\	縮める	ちぢめる	
\\	縮れる	ちぢれる	
\\	地点	ちてん	
\\	地名	ちめい	
\\	茶色い	ちゃいろい	
\\	~着	~ちゃく	
\\	着々	ちゃくちゃく	
\\	彼は着々と計画を進めた 
\\	道路工事は着々と進行した 
\\	中間	ちゅうかん	
\\	中旬	ちゅうじゅん	
\\	抽象	ちゅうしょう	
\\	抽象的 
\\	彼はただ抽象的に言っただけだ 
\\	物事を抽象的に考える 
\\	抽象化 
\\	抽象化する 
\\	抽象画 
\\	抽象芸術 
\\	抽象主義 〔美術〕
\\	抽象主義の芸術家 
\\	抽象名詞 〔文法〕
\\	抽象論 
\\	中世	ちゅうせい	
\\	中性	ちゅうせい	
\\	〔化学で〕
\\	中性の 
\\	〔文法で〕
\\	中性の 
\\	〔男とも女ともつかないこと〕
\\	中性的な女 
\\	中性子 
\\	中性子星 
\\	中性子爆弾 
\\	中性紙 
\\	中性洗剤 
\\	中途	ちゅうと	
\\	中年	ちゅうねん	
\\	~庁	~ちょう	
\\	~兆	~ちょう	
\\	~長	~ちょう	
\\	~帳	~ちょう	
\\	超過	ちょうか	
\\	超過する 
\\	輸入超過 
\\	予算より10万円の超過 
\\	車は制限速度を20キロ超過した 
\\	この小包は規定の重量を30グラム超過している 
\\	先月は予算を超過した 
\\	超過額 
\\	超過勤務 
\\	超過勤務をする[断る] 
\\	超過勤務手当 
\\	超過勤務手当てを支給される 
\\	超過料金 超過料金は1時間につき800円です 
\\	彫刻	ちょうこく	
\\	長所	ちょうしょ	
\\	〔利点〕
\\	それには長所も短所もある 
\\	長所を生かすようにしなさい 
\\	正確さは彼女の長所ではない 
\\	私は人の長所を見るように努めている 
\\	彼の長所は認めてやらなければならない 
\\	長女	ちょうじょ	
\\	調整	ちょうせい	
\\	調整する 〔望ましい状態にする〕
\\	〔一定のレベル・状態に保つ〕
\\	再調整する 
\\	高度調整〔航空で〕 
\\	年末調整〔所得税の〕 
\\	故障につき調整中〔掲示〕 
\\	価格を調整する 
\\	テレビの画像(がぞう)を調整する 
\\	彼は機械の速度を調整した 
\\	議長は意見を調整すべきだ 
\\	調節	ちょうせつ	
\\	調節する 〔規制する〕
\\	〔規準・条件などに合わせて〕
\\	〔機械などを〕
\\	このスイッチで部屋の温度を調節できる 
\\	いすの高さを調節する 
\\	ラジオを調節する〔選局する〕 
\\	弁を調節する 
\\	音量を調節する 
\\	調節弁 
\\	長短	ちょうたん	
\\	頂点	ちょうてん	
\\	長男	ちょうなん	
\\	長方形	ちょうほうけい	
\\	調味料	ちょうみりょう	
\\	~丁目	~ちょうめ	
\\	直後	ちょくご	
\\	〔時間的に〕 事件発生直後警官が現場に急行した 
\\	〔場所的に〕 車の直後に 
\\	彼らの直後を追った 
\\	直線	ちょくせん	
\\	直前	ちょくぜん	
\\	直通	ちょくつう	
\\	直流	ちょくりゅう	
\\	貯蔵	ちょぞう	
\\	直角	ちょっかく	
\\	直角の 
\\	二つの線は直角を成(な)している[に交わっている] 
\\	直角三角形 
\\	直角定規 
\\	直径	ちょっけい	
\\	散らかす	ちらかす	〔物をまき散らす〕
\\	〔乱雑にする〕
\\	新聞を散らかしたままにする 
\\	部屋は散らかしっぱなしだ 
\\	子供は部屋におもちゃを散らかした 
\\	公園にごみを散らかす 
\\	散らかる	ちらかる	〔物が〕
\\	〔部屋が〕
\\	公園には紙くずや空き缶が散らかっていた 
\\	彼の部屋はいつも散らかっている 
\\	塵紙	ちりがみ	
\\	追加	ついか	
\\	ついで	ついで	ついでの節はお立ち寄りください 
\\	ついでの折にこれを調べてくれ 
\\	ジョンの話が出たついでだが,彼のお母さんはどうしておられますか 
\\	ついでながら彼は今どこにいるのか 
\\	講演のついでに自分の著書についてしゃべった 
\\	私はあの人が嫌いだが,ついでに言えば皆もそうだ 
\\	ついでに 
\\	〔機会に乗じて〕 ついでにもう一つ例を挙げよう 
\\	[((口))
\\	ついでに次回の議題を申し上げます 
\\	ついでにお耳に入れたいことがあります 
\\	〔途中で,…がてら〕 学校に行くついでに用を一つ足してちょうだい 
\\	そこへ行ったついでに彼に電話した 
\\	~通	~つう	
\\	通知	つうち	
\\	⇒つうこく(通告) 
\\	通知する 
\\	会社からの採用通知 
\\	/((英))
\\	送金通知 
\\	彼からクラス会の通知があった 
\\	彼から何か通知があったか 
\\	詳しくはまたご通知します 
\\	通知状 
\\	〔取り引き上の〕
\\	通知表 
\\	((英)) 
\\	通知預金 
\\	通帳	つうちょう	
\\	通用	つうよう	
\\	〔貨幣・言葉などの〕 通用する 
\\	((文)) 
\\	この言葉は今では通用しなくなった 
\\	僕の英語はイギリスでは通用しなかった 
\\	円はこの国で通用しますか 
\\	この紙幣はもう通用しない 
\\	〔考え・規則などの〕 通用する 
\\	彼の理論は今でも通用する 
\\	そんなわがままはよそへ行ったら通用しない 
\\	君の考え方は社会では通用しない 
\\	彼の顔はもうこの世界では通用しない 
\\	〔切符などの〕 通用する 
\\	通用当日限り〔注意書き〕 
\\	通用門 〔通用口〕
\\	〔商人・使用人などの使う〕
\\	通用期間 
\\	この切符の通用期間は何日ですか 
\\	通路	つうろ	
\\	〔劇場・乗物などの〕
\\	通路側の席 
\\	通路を開ける 
\\	通路をふさぐ 
\\	これは従業員専用通路です 
\\	~遣い	~づかい	
\\	~付	~つき	
\\	突き当たり	つきあたり	
\\	突き当たる	つきあたる	
\\	〔衝突する〕 ドアに突き当たった 
\\	車が電柱に突き当たった 
\\	〔行き詰まる〕 この道はゴルフ場に突き当たる 
\\	この道を突き当たって右に曲がると郵便局があります 
\\	〔直面する〕 それを追って行けば結局この問題に突き当たる 
\\	月日	つきひ	
\\	~続く	~つづく	
\\	突っ込む	つっこむ	
\\	〔突き入れる〕
\\	〔無造作に押し込む〕
\\	汚れ物を洗濯機に突っ込んだ 
\\	子供はバケツに手を突っ込んで水をかき回した 
\\	彼女はトランクの中に下着を突っ込んだままだった 
\\	〔勢いよく突き入る,突き進む〕 飛行機は林の中に突っ込んだ 
\\	彼らは敵陣に突っ込んでいった 
\\	〔深く立ち入る〕 余計なことに頭を突っ込む 
\\	我々は突っ込んだ話をした 
\\	[((口)) 
\\	その問題についてだれも突っ込んだ意見を述べる者はいなかった 
\\	〔弱点などを追及する〕 人に突っ込まれるようなことをしたことはありません 
\\	説明の矛盾点を突っ込まれて返答に窮した 
\\	務める	つとめる	
\\	〔役目を行う〕 彼は3期市会議員を務めた 
\\	彼は案内役を務めてくれた 
\\	〔役を演じる〕 彼は交渉の席で主役を務めた 
\\	老け役を務める 
\\	((英)) 
\\	((文)) 
\\	はもっとも普通に使われる
\\	は継続的に努力するの意で,多少形式ばったいい方
\\	は困難の克服に努めること) 二度とこんな間違いを繰り返さないように努めます 
\\	妻としての務めを果たそうと努めた 
\\	この難局を乗り切るよう努めなければならない 
\\	努める	つとめる	
\\	綱	つな	
\\	繋がり	つながり	
\\	これとそれの間にはつながりはない 
\\	食物と健康の間には密接なつながりがある 
\\	彼とは血のつながりがある 
\\	この語の意味は前後のつながりを見ないと分からない 
\\	あの人は今度の事件とつながりがあるんだろうか 
\\	粒	つぶ	
\\	潰す	つぶす	
\\	〔押し付けて形を崩す〕
\\	ぶどうをつぶしてぶどう酒を作る 
\\	じゃがいもをつぶす 
\\	〔金属製品をとかす〕⇒いつぶす(鋳潰す) 
\\	〔畜殺する〕 にわとりをつぶしてスープにした 
\\	〔埋めふさぐ〕 穴をつぶす 
\\	〔体面・名誉を傷付ける〕 彼女は父親の顔をつぶした 
\\	〔空いた時間を費やす,空費する〕 暇をつぶす 
\\	定年後の父はぶらぶらして時間をつぶしている 
\\	〔財産などを失う,破産させる〕 彼は身代をつぶした 
\\	彼は計画的に会社をつぶした 
\\	〔駄目にする,役に立たなくする〕 大声を出し過ぎて声をつぶしてしまった 
\\	経済の混乱がわずか1年間に3人の首相をつぶした 
\\	つぶれる	つぶれる	
\\	〔押されて形が崩れる〕
\\	混んだ電車でケーキがつぶれた 
\\	地震で家屋がつぶれた 
\\	〔体面が〕 彼の不始末で私の顔がつぶれた 
\\	〔やって行けなくなる,破産する〕 あの会社はつぶれ掛かっている 
\\	あの店はついにつぶれた 
\\	〔駄目になる,役に立たなくなる〕 緊急会議のために旅行の予定がつぶれた 
\\	風邪ですっかり声がつぶれてしまった 
\\	〔心が痛む〕 彼の死に接し悲しみに胸がつぶれた 
\\	つまずく	つまずく	
\\	詰まる	つまる	
\\	〔一杯入っている〕
\\	予定が詰まっている 
\\	[((口))
\\	ぎっしり詰まった本箱 
\\	道路は行楽客の車で詰まっている 
\\	〔ふさがる〕 下水が詰まっている 
\\	鼻が詰まっている 
\\	窓がみんな閉まっていて息が詰まりそうだった 
\\	女の子が病気の母親の世話をしているのを見て胸が詰まった 
\\	〔窮する〕 問い詰められて言葉に詰まった 
\\	金に詰まって男は持ち家を売り払った 
\\	〔縮む,短くなる〕 ジーンズを洗ったら丈が詰まった 
\\	先頭の2人のランナーの差が詰まってきた 
\\	〔野球で〕 詰まった当たりだった 
\\	爪	つめ	
\\	艶	つや	
\\	強気	つよき	
\\	~辛い	~づらい	
\\	釣り合う	つりあう	
\\	〔程度などが同じである〕 昨年の米の供給量と需要量がよく釣り合っていた 
\\	〔似合う〕 あの奥さんでは旦那(だんな)に釣り合わない 
\\	この夫婦は釣り合っている[いない] 
\\	〔調和する〕 このネクタイは洋服によく釣り合う 
\\	このカーテンはじゅうたんと釣り合う 
\\	吊る	つる	
\\	〔つるす〕 腰に剣をつっていた 
\\	けがをした腕をつり包帯でつっている 
\\	棚をつる 
\\	〔両端を固定してかけ渡す〕 橋をつる 
\\	ハンモックを木の間につる 
\\	〔縊死(いし)〕 彼は首をつった 
\\	この意味の
\\	は規則動詞) 
\\	〔上へ引っ張られる〕 彼女の目がつっている〔ひきつけなどで〕 
\\	〔引っ張られて縮む〕 縫い目がつっている 
\\	つるす	つるす	洗濯物をつるす 
\\	モビールを天井からつるす 
\\	手洗い	てあらい	
\\	低~	てい~	
\\	定員	ていいん	
\\	定価	ていか	
\\	低下	ていか	
\\	定期券	ていきけん	
\\	定休日	ていきゅうび	
\\	停止	ていし	
\\	停車	ていしゃ	
\\	出入り	でいり	
\\	出入口	でいりぐち	
\\	手入れ	ていれ	
\\	~滴	~てき	
\\	出来上がり	できあがり	
\\	出来上がる	できあがる	
\\	〔完成する〕
\\	その家は来月出来上がる 
\\	ワンピースが出来上がるのにあと何時間掛かりますか 
\\	〔酔う〕 もう出来上がっている 
\\	あいつ出来上がっちまったらしいな 
\\	的確	てきかく	
\\	適確	てきかく	
\\	手首	てくび	
\\	凸凹	でこぼこ	
\\	手頃	てごろ	
\\	〔扱いよいこと〕 机のそばに手ごろな棚がある 
\\	〔条件・能力などに合うこと〕 手ごろな 
\\	〔妥当な〕
\\	手ごろな大きさの板 
\\	この住まいは新婚夫婦に手ごろだ 
\\	家賃は手ごろだ 
\\	手ごろな値段で洋服を売る 
\\	弟子	でし	
\\	でたらめ	でたらめ	
\\	〔意味のないこと〕
\\	でたらめを言う 
\\	あの話はでたらめだった 
\\	〔いい加減〕 でたらめな本 
\\	でたらめな議論[演説] 
\\	でたらめな男 
\\	でたらめな天気 
\\	でたらめな値段 
\\	でたらめな返事をしておいた 
\\	でたらめに撃った弾が的に当たった 
\\	名簿の順序はでたらめです 
\\	でたらめな順番じゃだめだよ ((口)) 
\\	手帳	てちょう	
\\	鉄橋	てっきょう	
\\	手続き	てつづき	
\\	鉄砲	てっぽう	
\\	手拭い	てぬぐい	
\\	手前	てまえ	
\\	〔自分に近いほう〕 郵便局の手前を左へ曲がるのです 
\\	京都より一つ手前の駅で降りた 
\\	私の3歩手前で立ち止まった 
\\	川の手前に2,3軒家がある 
\\	〔へりくだって,私〕
\\	〔目下に対して,お前〕
\\	手前どもではそういう品を扱っておりません 
\\	〔体面,体裁〕 親の手前あまりひどい格好も出来ない 
\\	子供の手前,万事好都合であるような振りをした 
\\	約束した手前行かないわけにはいかない 
\\	〔茶道〕⇒おてまえ(御手前)
\\	出迎え	でむかえ	
\\	照らす	てらす	
\\	照る	てる	
\\	~点	~てん	
\\	展開	てんかい	
\\	〔くり広げられる〕 びょうびょうたる大海が目前に展開した 
\\	この原野では鶴の群が華やかな情景を展開する 
\\	[((文))
\\	援軍は海岸線に沿って展開した 
\\	〔事態・論旨などが次にと進む〕 この事件の今後の展開が待たれる 
\\	彼は独特の新理論を展開した 
\\	政局は新局面を展開した 
\\	〔数学で〕
\\	展開する 
\\	〔音楽で〕
\\	(主題が) 変奏に展開する 
\\	展開式 〔数学で〕
\\	展開図 〔数学で〕
\\	〔船〕
\\	展開部 〔音楽で〕
\\	伝記	でんき	
\\	電球	でんきゅう	
\\	点数	てんすう	
\\	伝染	でんせん	
\\	〔接触による病気の〕
\\	〔間接的な〕
\\	伝染する ((病気が主語で)) 
\\	((口)) 
\\	((人が主語で)) 
\\	自家伝染 
\\	この病気は空気伝染によって起こる 
\\	この病気は伝染する 
\\	は接触による,
\\	は空気,水などによる感染について用いるが実際にはこの二つの語は同じよう に使われることが多い) 200人の学童にインフルエンザが伝染した 
\\	〔物事が他に移ること〕 そういう流行は伝染しやすい 
\\	彼の笑い[あくび]は伝染した 
\\	伝染系統 〔経路〕
\\	その病気の伝染系統は分からなかった 
\\	伝染性 伝染性の 
\\	伝染病 
\\	接触伝染による); 
\\	間接伝染による) ⇒でんせん(伝染); 〔流行病〕
\\	法定伝染病 
\\	伝染病がその地方で大流行している 
\\	この皮膚炎は接触伝染病だ 
\\	伝染病患者 
\\	伝染病病棟 
\\	伝染病予防法 
\\	伝染病流行地 
\\	電池	でんち	
\\	点々	てんてん	床に血が点々と落ちた 
\\	空には点々と星がきらめいていた 
\\	湖には点々とボートが浮かんでいた 
\\	転々	てんてん	
\\	〔転がる様子〕 ボールが転々と転(ころ)がっていった 
\\	〔移り行く様子〕 転々と職を変える 
\\	男は各地を転々とした 
\\	そのダイヤは持ち主の間を転々とした 
\\	電柱	でんちゅう	
\\	天皇	てんのう	
\\	めど	【目▲処】	〔目標〕
\\	〔見通し〕
\\	仕事のめどがまだつかない 
\\	住まいを見つけるめどがまだつかない 
\\	景気回復のめどがつかない 
\\	[((口)) 
\\	ようやく研究完成のめどがついた 
\\	生産の倍増をめどにしている 
\\	来月中旬をめどにこの仕事を終わらせます 
\\	犯人のめどはまだ付かない 
\\	通告	つうこく	
\\	通告する 
\\	一方的な通告 
\\	最後通告 
\\	((複 〜
\\	何の通告もなしに 
\\	立ち退き通告をする[受ける] 
\\	警察への出頭通告を受ける 
\\	仕事をやめる時は1か月前に通告すること 
\\	通告書 
\\	いかがわしい	【▲如▲何わしい】	
\\	〔信用できない〕
\\	いかがわしい新説 
\\	〔怪しげな〕
\\	いかがわしい人物 
\\	彼はいかがわしい仕事に手を出しているらしい 
\\	いかがわしいうわさ 
\\	〔みだらな〕
\\	いかがわしい行為 
\\	彼はいかがわしい目的でその婦人のあとをつけ始めた 
\\	いかがわしい場所 
\\	いかがわしい女 
\\	いかがわしい話[絵] 
\\	格好	かっこう	
\\	〔姿,形,様子〕
\\	〔形〕
\\	格好のいい女 
\\	変わった格好の家 
\\	この帽子[服]は格好よくできている 
\\	あのドレスを着るとおかしな格好に見える 
\\	自分の格好などどうでもよかった 
\\	医者にも見放された格好で退院した 
\\	ぶざまな格好で踊る 
\\	大勢が見ていたので格好が悪かった 
\\	〔ころあい〕 格好な 〔適当な〕
\\	〔妥当な〕
\\	野営に格好な場所 
\\	格好な値段 
\\	贈り物に格好な品 
\\	格好がつく 部屋に絵があるとぐっと格好がつく 
\\	彼が出席しないとパーティーはどうも格好がつかない 
\\	格好をつける 彼はいつも格好をつけたがる 
\\	怪しむ	あやしむ	
\\	〔…でないと思う〕
\\	〔…ではないかと思う〕
\\	〔…を疑う〕
\\	((文)) 
\\	彼の誠実さを怪しむ 
\\	彼の成功を怪しむ 
\\	私は彼の意図を怪しむ 
\\	彼の潔白を怪しむ声が多い 
\\	彼はその金を盗んだのではないかと怪しまれた 
\\	〔不思議に思う〕
\\	一体何事かと怪しんだ 
\\	彼が裏切っても怪しむに足りない 
\\	一行	いっこう	
\\	〔役者などの〕
\\	一行は6名です 
\\	大使一行は今朝当地に到着した 
\\	ボブホープ一行 
\\	わきまえる	【▲弁える】	
\\	〔区別する〕
\\	((文)) 
\\	善悪をわきまえなさい 
\\	彼は公私の別をわきまえていない 
\\	礼儀をわきまえている[いない] 
\\	〔心得る〕
\\	自分の分際をわきまえろ 
\\	そのくらいのことは彼だってわきまえている 
\\	それは身のほどをわきまえない高望みだ 
\\	場所柄もわきまえずに高笑いした 
\\	不肖	ふしょう	
\\	〔父,師に似ず,愚かなこと〕 不肖の子[弟子] 
\\	〔自分を謙遜して〕 本日は不肖私が議長を務めさせていただきます 
\\	情景	じょうけい	
\\	⇒こうけい(光景) 痛ましい情景 
\\	なんて		
\\	〔なんという〕⇒なんと(何と) 何て美しい人だろう 
\\	この石は何て重いんだろう 
\\	⇒なに(何) それで君は彼に何て言ったの 
\\	何てったって夏はビールが一番だ 
\\	勉強なんて大嫌いだ 
\\	嫌だなんて言わせない 
\\	彼女がうそをつくなんて (信じられない) 
\\	今さら行くなんて言ったってね 
\\	覚えてろ,なんてね〔ふざけて〕 
\\	直訳はできない) 木村なんていう人はここでは働いていません 
\end{CJK}
\end{document}