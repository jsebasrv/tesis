\documentclass[8pt]{extreport} 
\usepackage{hyperref}
\usepackage{CJKutf8}
\begin{document}
\begin{CJK}{UTF8}{min}
\\	順調	
\\	じゅんちょう 仕事は順調ですか
\\	老人ホームのホームヘルパー	
\\	語学学校の講師	
\\	こうし
\\	会社の事務員	
\\	コンピューター関係の仕事	
\\	レストランのオーナー	
\\	ホテルの従業員	
\\	じゅうぎょういん
\\	フリーター	決まった会社などに就職をしないで、アルバイトなどを続けることで生活している人のことを言う。 臨時的に(リンジ 
\\	雇われて働いている人。 
\\	就職活動 就活	
\\	求人情報	
\\	年齢制限がある	
\\	ねんれいせいげん
\\	経験を問わない	
\\	雇う 採用する	やと-う=
\\	さいよう する 
\\	人手が不足している	
\\	ひとで
\\	正社員を募集する	
\\	せいしゃいん
\\	派遣社員に登録する	
\\	はけん
\\	会社に問い合わせる	
\\	面接で緊張する	
\\	面接にふさわしい服装	
\\	8時間勤務	
\\	時給で働く	
\\	交通費が支給される	
\\	お金を稼ぐ	
\\	かせ-ぐ
\\	お金を儲ける	
\\	もう-ける
\\	店が儲かる	
\\	もう.かる
\\	気軽に尋ねる	
\\	きがる
\\	通学	
\\	つうがく
\\	自由席券	
\\	上司 
\\	部下	
\\	じょうし ぶか
\\	部長	
\\	ぶちょう
\\	課長	
\\	かちょう
\\	先輩 
\\	後輩	
\\	せんぱい
\\	こうはい
\\	同僚 
\\	仕事仲間	
\\	どうりょう しごとなかま
\\	大企業に就職する	
\\	だいきぎょう
\\	四月に入社する 新入社員	
\\	我が社の女子社員	
\\	先輩に教わる	
\\	おそわる
\\	先輩を見習う	
\\	みならう
\\	早退する 
\\	遅刻する	
\\	そうたい
\\	休憩する	
\\	休みを取る 休暇を取る	
\\	きゅうか
\\	有給休暇	
\\	ゆうきゅうきゅうか
\\	席を外す	
\\	やりがいのある仕事	
\\	仕事が生きがいだ。	
\\	休む間がない 休むひまがない	
\\	やる気がない	
\\	具合が悪いふりをする	
\\	やる気が出ない	
\\	仕事を怠ける 仕事をサボる	
\\	なまける
\\	怠け者 
\\	働き者	
\\	仕事でミスをする	
\\	遅刻の言い訳をする	
\\	上司にゴマをする	
\\	ゴマをすった
\\	超忙しい	
\\	お世辞を言う	
\\	せじ
\\	次々 に・と	
\\	つぎつぎ
\\	上司から次々と仕事が与えられ、休む間もない。	
\\	体が持たない	
\\	会議が長引く	
\\	ながびく
\\	会議/ミーティングの準備をする	
\\	意見を述べる	
\\	意見をまとめる	
\\	意見を求める	
\\	具体的な案を出す 
\\	抽象的	
\\	ちゅうしょうてき 
\\	アイデアを出す	
\\	結論が出る	
\\	資料/プリント を配る	
\\	メモを取る	
\\	張り切る	
\\	はりきる
\\	引き受ける 
\\	仕事を引き受ける	
\\	仕事の打ち合わせをする 打ち合わせる	
\\	スケジュールを組む	
\\	仕事を順調にこなす	
\\	電話を取り次ぐ	
\\	とりつぐ
\\	電話を転送する	
\\	電話を保留する	
\\	ほりゅう
\\	残業する	
\\	出張が多い	
\\	出世する	
\\	彼は順調に出世した 
\\	しゅっせ
\\	昇進する	
\\	しょうしん
\\	本社に転勤になる	
\\	転職する	
\\	会社を首になる	
\\	リストラされる	(=首になる) 
\\	退職する	
\\	失業する	
\\	キャッシュバック	返金、払い戻し 
\\	料金	
\\	プリンター	
\\	ウィンドウ	
\\	モニター//ディスプレイ	
\\	パスコン本体	
\\	キーボード	
\\	(マウス)ポインター	
\\	パソコンを操作する	
\\	ソウサ
\\	基本をマスターする	
\\	パソコンを起動する/立ち上げる 
\\	終了させる	
\\	(しゅうりょう)
\\	パソコンを再起動する	
\\	次の画面を表示する	
\\	プロバイダーと契約する	
\\	(インター)ネットに 
\\	接続する 
\\	つなぐ 
\\	アクセスする	
\\	ホニャララ~が繋がる 
\\	電話が繋がらない	
\\	つながる
\\	インターネットで検索する	
\\	ホームページで情報を得る	
\\	*日本では、会社などのウェブサイト全体のことを「ホームページ」という。
\\	ダウンロードを開始する 
\\	完了する	
\\	ソフト(ウェア)をインストールする	
\\	スタートボタンをクリックする	
\\	フリーズする	
\\	パソコンのシステムを更新する	
\\	便利な機能	
\\	メールのやり取りをする	
\\	アドレスを入力する	
\\	アドレスを登録する	
\\	解決策	
\\	かいけつさく
\\	一流 
\\	一流の大学	
\\	イチリュウ
\\	根本的な	
\\	コンポンテキな
\\	財政	
\\	ザイセイ
\\	率直に	
\\	ソッチョクに
\\	し合う	
\\	ファイル	
\\	編集	
\\	挿入	
\\	そうにゅう
\\	書式	
\\	ショシキ
\\	ツール	
\\	新規作成	
\\	シンキサクセイ
\\	ファイルを開く 
\\	閉じる	
\\	上書き保存	
\\	うわがき
\\	名前を付けて保存	
\\	ページ設定	
\\	印刷	
\\	インサツ
\\	元に戻す	
\\	切り取り	
\\	コピー	
\\	貼り付け	
\\	検索	
\\	書式設定	
\\	しょしき
\\	フォント 書体	
\\	教科書体 明朝体 ゴシック体	
\\	ミンチョウ
\\	下線	
\\	太字	
\\	サイズ10
\\	左寄せ	
\\	右寄せ	
\\	中央揃え	
\\	チュウオウゾロエ
\\	両端揃え	
\\	リョウタンゾロエ
\\	書類を作成する	
\\	かなを漢字に変換する	
\\	漢字を確定する	
\\	文字を削除する	
\\	サクジョ
\\	改行する	
\\	カイギョウ
\\	文字のサイズを設定する	
\\	カーソルを移動する	
\\	範囲を選択する	
\\	目的の位置に合わせる	
\\	文字を中央に揃える	
\\	ソロエル
\\	右に寄せる	
\\	下線をつける アンダーラインをつける	
\\	記号を入れる	
\\	図を挿入する	
\\	ズをソウニュウする
\\	図を拡大する 
\\	縮小する	
\\	カクダイ 
\\	シュクショウする 
\\	内容を修正する	
\\	シュウセイ
\\	文書を編集する	
\\	ブンショ
\\	図形をコピーする	
\\	ズケイ
\\	切り取る カットする 
\\	切り取り 
\\	貼り付ける ペーストする 
\\	貼り付け	
\\	画像を取り込む	
\\	ガゾウ
\\	文書を保存する	
\\	ファイルを添付する	
\\	テンプ
\\	プリンターの用紙が切れる 
\\	を切らす	
\\	インク//トナーがなくなる	
\\	印刷の範囲を指定する	
\\	印刷の向きを確認する	
\\	余白を多くする	
\\	ヨハク
\\	標準サイズの用紙	
\\	ヒョウジュン
\\	印刷がずれる 
\\	をずらす	
\\	ずらす ・
\\	手間がかかる	
\\	手間を省く	
\\	ハブク
\\	表示(方法)を切り替える	
\\	データを消去する 
\\	データが消える	
\\	しょうきょ
\\	迷惑メール	
\\	受信を拒否する	
\\	キョヒ
\\	個人情報を盗まれる	
\\	(コンピューター)ウィルスに感染する	
\\	カンセン
\\	新しいウイルスに対応する	
\\	掲示板に書き込む	
\\	*書き込む= ・
\\	品物を売り買いする	
\\	オンラインゲームをする	
\\	チャットを楽しむ	
\\	普及	
\\	ふきゅう
\\	発展	
\\	ハッテン
\\	生地	
\\	キジ
\\	立ち入る	
\\	たちいる
\\	眺めがいい	
\\	ナガメ
\\	人通りが多い	
\\	ヒトドオリ
\\	一軒家 一戸建て	
\\	イッケンヤ イッコダテ
\\	直後 例)君が出かけた直後に彼が訪ねてきた。	
\\	チョクゴ
\\	物干し 例)ベランダには物干しも付いているからよかった。	
\\	風呂場	
\\	フロバ
\\	洗面所	
\\	一人暮らし 一人住まい	
\\	ヒトリグラシ ヒトリズマイ
\\	近所付き合い 例)近所付き合いが大変じゃないといいんだけど…	
\\	きんじょづきあい
\\	彼と付き合う	
\\	友達に付き合う	
\\	引越しの荷造りをする	
\\	ニヅクリ
\\	回覧 回覧板を回す	
\\	カイランバン
\\	この辺り この付近	
\\	真ん前	
\\	斜め前	
\\	ナナメマエ 
\\	真ん中	
\\	真後ろ	
\\	まうしろ
\\	チラシ びら 例)アパート・マンションのチラシの情報	
\\	賃貸アパート	チンタイ 
\\	家賃	
\\	管理費	
\\	無料 ただ 
\\	有料	
\\	徒歩5分	
\\	3階建て 例)3階建ての3階	
\\	南向き	
\\	築10年	
\\	自転車置き場 例)自転車置き場無料	
\\	マンション *少し高級な感じがする集合住宅。 分譲と家賃がある	
\\	こうきゅう じゅうたく ぶんじょう
\\	敷金 *家や部屋を借りるときに家主に預けておく保証金	
\\	シキキン やぬし ほしょうきん
\\	礼金 *家や部屋を借りるときに家主に払う一時金。ない場合もある。	
\\	イチジキン 
\\	家主=やぬし
\\	リビング・居間 
\\	ダイニング・食事のできる場所 
\\	キッチン・台所 数字: 
\\	以外の部屋の数字を表す	
\\	=いま 
\\	ワンディーケー 
\\	ニーエルデーィケー
\\	広々(と)している 例)ベランダも広々としていますし…	
\\	ひろびろ
\\	日当たりがいい	
\\	ひあたり
\\	ついていけない 例)君についていけない。   私は周りの話題についていけない。   学校は休まずに行っているものの、授業にはついていけない。	
\\	詰める 例)箱に詰めたらふたをしてガムテープで貼ってください。	
\\	ツメル
\\	溢れる 例)グラスが倒れて、酒がこぼれてしまった。	
\\	こぼれる
\\	水が漏れる 
\\	水を漏らす 
\\	秘密が漏れる->秘密を漏らす	
\\	モレル モラス
\\	古新聞を再利用する	
\\	フルシンブン
\\	アルミ缶を潰す	
\\	ツブス
\\	シールを貼る	
\\	に), 
\\	ラベル 例)ラベルをはがす	
\\	ペットボトル 例)ペットボトルを濯ぐ	
\\	すすぐ
\\	キャップ 例)キャップを閉める   キャップを外す	
\\	段ボール箱	
\\	ポリ袋	
\\	ポリぶくろ
\\	ガムテープ	
\\	生ゴミ	
\\	ナマ
\\	燃えるゴミ 可燃ごみ 燃やせるごみ	
\\	燃えないごみ 不燃ごみ 燃やせないごみ	
\\	ごみを分別する	
\\	不用品を処分する	フヨウヒンをショブンする 
\\	リサイクルに出す	
\\	ごみが溜まる	
\\	タマル
\\	電気代を節約する	
\\	省エネ 例)新しい冷蔵庫は省エネだから、…	
\\	れいぞうこ
\\	家電製品	
\\	(肉等の)トレイ	
\\	(卵・牛乳等の)パック	
\\	レジ袋	
\\	レジブクロ
\\	液	
\\	エキ
\\	水道代	
\\	買い換える 例)新しい洗濯機に買い換えたら…	
\\	指定のゴミ袋	
\\	ゴミブクロ
\\	資源ごみ	
\\	粗大ごみ	
\\	ソダイ
\\	掲示する 掲示を出す	
\\	お買い得だ。	
\\	「遠いところまでよくいらっしゃいました。どうぞお上り下さい。」 「お邪魔します。」	
\\	手頃な価格	
\\	半額セール	
\\	ハンガク
\\	快適に暮らす	
\\	くつろぐ	
\\	他) 
\\	自)
\\	ゆっくりする 
\\	のんびりする	
\\	家で過ごす	
\\	座り心地がいいソファー 寝心地のいいベッド 居心地がいい家	
\\	ゴコチ
\\	友人を家に招く 友人を家に招待する	
\\	「ごゆっくりお召し上がりください」 *レストランなどでよく聞く表現	
\\	「ゆっくりしていってください」	
\\	「お茶をお持ちします。」 「どうぞお構いなく。」	
\\	「つまらないものですが…。」 「ありがとうございます。遠慮なくいただきます。」 *人にお土産などをあげるときの謙遜した言い方。	
\\	けんそん(な)=
\\	「ご無沙汰しております(います)。お元気で(いらっしゃいま)したか。」 「ええ、お陰様で。」	
\\	しつこい	
\\	気の毒な	
\\	キノドク
\\	検査	
\\	(動物を)いじめる	
\\	猫の耳を引っ張る	
\\	足で蹴る	
\\	ケル
\\	顔を殴る	
\\	ナグル
\\	顔をぶつ	
\\	暴れる	
\\	アバレル
\\	落書きをする	
\\	やかましい *スピーカーの音がやかましい *彼はファッションにやかましい	
\\	騒々しい	
\\	ソウゾウシイ
\\	騒がしい	
\\	サワガシイ
\\	騒ぐ	
\\	サワグ
\\	いたずらをする	
\\	やんちゃな子供	
\\	紙おむつ	
\\	おむつ/おしめ を 換える/当てる	
\\	抱っこする 
\\	腕に抱く	
\\	ダッコ ダク(イダク)
\\	赤ちゃんを おんぶする/負ぶう	
\\	おぶう
\\	子供を産む 
\\	生まれる	
\\	妊娠する	
\\	妊娠させる=
\\	妊娠している=
\\	妊娠3ヶ月である=
\\	ニンシン
\\	よす 
\\	この話はよそう 
\\	よしなさい/よせ	「やめる」
\\	怒鳴る	
\\	ドナル
\\	お尻を叩く	
\\	オシリをタタク
\\	子供をしつける 
\\	しつけ	
\\	子供を甘やかす 
\\	子供が甘える	
\\	アマ~
\\	炊事	
\\	スイジ
\\	汚れた食器	
\\	食卓を片付ける	
\\	カタヅケル
\\	家事をする	
\\	育児	
\\	イクジ
\\	子供を可愛がる	
\\	子育てをする	
\\	洗い物を流しに持っていく	
\\	塞ぐ 例)耳を塞ぐ	
\\	フサグ
\\	安全性	
\\	回送する 回送列車	
\\	回送列車
\\	ロープに捕まる 例)これに捕まれ! 
\\	これに掴まれ!	
\\	!ツカマル
\\	ロープをつかむ	
\\	魚を捕まえる	
\\	(~が)捕まる 
\\	捕まえられる	
\\	(~を)引っ掛ける 例) ラックにコートを引っ掛ける   女を引っ掛ける	
\\	魚が網に引っかかる	
\\	アミ
\\	詐欺に引っかかる	
\\	サギ
\\	泥棒	
\\	ドロボウ
\\	栄養が偏る	
\\	かたよる 
\\	高くつく 例)電話代が高くつく	
\\	生活費が不足する	
\\	食費を節約する	
\\	旅行する余裕がない	
\\	ヨユウ
\\	犯す 例)法律を犯す	
\\	オカス
\\	急病人	
\\	きゅうびょうにん
\\	診療 診察 診断 検診	シンリョウ 
\\	しんさつ
\\	シンダン 
\\	健康診断 
\\	ケンシン 
\\	集団検診
\\	がん検診
\\	祝祭日	
\\	しゅくさいじつ
\\	休診日	
\\	きゅうしんび
\\	収入に釣り合う生活	
\\	しゅうにゅう つりあう=
\\	贅沢をする 贅沢な	
\\	ゼイタク
\\	無駄遣いをする	
\\	ムダヅカイ
\\	無駄を省く	
\\	済ませる 例)支払いを済ませる	
\\	スマセル
\\	紙幣 =お札	
\\	シヘイ
\\	肌が真っ黒になる 例)彼が真っ黒に日焼けした。	
\\	ヒヤケ
\\	小銭	
\\	コゼニ ー
\\	小銭入れ 
\\	銀行でお金を 1。下ろす 2。引き出す 
\\	預ける	
\\	銀行口座	
\\	自動で引き落とされる 例)電話代が口座から引き落とされた。 
\\	引き落とす	
\\	バランスの/が取れた食事	
\\	取れる=
\\	授業料を払い込む	
\\	家賃を振り込む 
\\	振込	
\\	公共料金を支払う 公共料金=電気代
\\	ガス代
\\	水道代
\\	電話代... 
\\	支払い	
\\	コウキョウ=
\\	シハライ=
\\	手数料がかかる	
\\	収入 
\\	支出	
\\	シュウニュウ 
\\	シシュツ=
\\	赤字です 
\\	黒字	
\\	クロジ=
\\	外食する 
\\	自炊する	
\\	ジスイ=
\\	サングラスを外す	
\\	魚を釣る	
\\	棚を釣る	
\\	タナ
\\	ボートをこぐ 自転車をこぐ	
\\	ボートがひっくり返る 
\\	ひっくり返す	
\\	ボートの行方 
\\	行方不明	
\\	ゆくえふめい=
\\	船を岸に近づける	
\\	キシ
\\	サメが 
\\	近づく 
\\	近寄る	
\\	1。あちこち 2。あちらこちら へ行く	
\\	うまそうな餌 美味しそうな餌 
\\	まずそうな餌	
\\	エサ
\\	振り込め詐欺 
\\	振り込む	
\\	サギ=
\\	フリコム=
\\	真っ暗な部屋	
\\	眼鏡	めがね
\\	海水浴に行く	
\\	カイスイヨク
\\	日光浴をする =太陽の光を浴びる	
\\	ニッコウヨク
\\	シャワーを浴びる	
\\	日焼けする	
\\	ヒヤケ
\\	ボタンを外す	
\\	サングラスの跡がついてる	
\\	アト
\\	汗を掻く	
\\	カク
\\	海に潜る	
\\	もぐる
\\	溺れる	
\\	オボレル
\\	黴	
\\	カビ
\\	待ち受け	
\\	内外 例)建物の内外	
\\	受け取り	
\\	終える	
\\	オ・エル
\\	強盗	ゴウトウ 
\\	酔った人	ヨッタ 
\\	通りで	
\\	液体	エキタイ 
\\	容器	ヨウキ 
\\	機嫌 
\\	社長は今日は機嫌が悪い。(=不機嫌)	キゲン 
\\	冷める	サメル 
\\	ちっとも	
\\	たっぷり 例)たっぷり15分   たっぷり食べる   金がたっぷりある   自信たっぷり	
\\	例)たっぷり10キロ 
\\	村上さん	ムラカミ
\\	じっと見つめる	
\\	目前	モクゼン 
\\	ハラハラ	
\\	圧し潰す 押しつぶす	オシツブス 
\\	体調	
\\	るせいか	
\\	飛行機でお酒飲むと、気圧(きあつ)のせいか酔いやすい. ギョーザ 餃子を家で作りましたが、皮が少なかったせいか、具が少し余ってしまいました. 
\\	肉体的	
\\	どれほど	
\\	表情 
\\	表情筋	
\\	ヒョウジョウキン 
\\	そっとする そっと	
\\	敬語:1.自己紹介します-> 
\\	宜しくお願いします->	
\\	させていただきます 
\\	お願い申し上げます
\\	避ける	さける, よける
\\	拗ねる	すねる 
\\	分譲する	ぶんじょう 
\\	考え出す	
\\	そんな事を考え出したら切りがない. 
\\	夜が明ける	
\\	温室効果ガス	
\\	地球	ちきゅう 
\\	無料体験	
\\	海面が上昇する	かいめん じょうしょう 
\\	公衆電話	コウシュウ 
\\	低下する	ていか 
\\	大幅	おおはば 
\\	首相	しゅしょう 
\\	南極	なんきょく 
\\	保護	ほご 
\\	その花は雨風に晒されないように保護されている 晒す)さらす
\\	取り返しのつかない	
\\	敬語 
\\	先生のことを私はよく知っています 
\\	その件について知っています 
\\	素晴らしいと思います 
\\	受け取ります 
\\	みます 
\\	読みます 
\\	借ります	
\\	存じ上げております 
\\	存じております 
\\	存じます 
\\	拝受(はいじゅ)いたします 
\\	拝見(はいけん)いたします 
\\	拝読(はいどく)いたします 
\\	拝借(はいしゃく)いたします
\\	小人	こども 
\\	大人 『白雪姫と七人お小人』=シラユキヒメ
\\	コビト
\\	反対語	
\\	はんたいご
\\	片道 
\\	往復	
\\	暖房 
\\	冷房	
\\	温水 
\\	冷水	
\\	老人 
\\	若者	
\\	年老いた 
\\	若い	
\\	反対 
\\	賛成	
\\	捨てる 
\\	拾う	ヒロウ 
\\	鋭い 
\\	鈍い	スルドイ 
\\	ニブイ 
\\	肯定 
\\	否定	コウテイ 
\\	ヒテイ 
\\	浮く 
\\	沈む	ウク 
\\	シズム 
\\	浮かれている 
\\	沈んでいる	嬉しいことがあって 
\\	悲しいことがあって 
\\	あいにく	
\\	布団を敷く	シク 
\\	整備	せいび 
\\	上中下	じょうちゅうげ ・
\\	不景気	ふけいき 
\\	消極的	
\\	運行	うんこう 
\\	1〈天体の〉 
\\	運行する 
\\	〈交通機関の〉 
\\	その地区一帯を運行しているバス 
\\	取り組む→取り組み	・
\\	問題と、新しい企画に) →取り組み 
\\	地域ぐるみで子供を育てようという取り組みが、各地で始まっている。
\\	取り込む	1〈取り入れる〉 
\\	洗濯物を取り込む 2。
\\	画像を 3〈混雑する〉 
\\	〈忙しい〉 
\\	〈もめ事がある〉 
\\	取り次ぐ ー
\\	取次	
\\	電話を取り次ぐ 
\\	取り除く	
\\	市民講座	
\\	こうざ
\\	稽古 に通う	けいこ 
\\	身に付ける	
\\	(健康)維持	いじ 
\\	する 
\\	護身術	ごしんじゅつ 
\\	参加資格	
\\	会費	
\\	実費	
\\	じっぴ
\\	指導員	しどういん 
\\	開講	かいこう 
\\	ゆったり	
\\	歪み	
\\	1。地震で歪みが発生する 
\\	2。無理な経済政策の歪み 
\\	体の ゆがみ
\\	納入	・
\\	先着順	せんちゃくじゅん 
\\	会計	
\\	会計課 
\\	会計係 
\\	会計を済ませる
\\	会員	
\\	(数回)に渡り	
\\	すうかい=
\\	未納	
\\	みのう
\\	つきましては	
\\	末日	まつじつ 
\\	振り込み	
\\	規約	
\\	=決まり、約束事(ヤクソクゴト=
\\	キヤク
\\	~条	じょう 
\\	に基づき	
\\	除名	じょめい 
\\	ご了承ください	
\\	りょうしょう
\\	万一	まんいち 
\\	早速ですが	すぐ用件に入る
\\	思いやりがある	
\\	マナーがいい	
\\	若々しい	ワカワカしい 
\\	センスがいい	
\\	生き生きとした	
\\	要領がいい	ヨウリョウ=
\\	=こつ 
\\	朗らかな	ホガラカ 
\\	穏やかな 平穏な 穏健な	オダヤカ
\\	へいおん
\\	おんけん 
\\	爽やかな	
\\	おとなしい	
\\	人懐っこい	ヒトナツッコイ 
\\	落ち着いている	
\\	無邪気な	ムジャキ 
\\	頼もしい	
\\	大雑把な	オオザッパ
\\	そそっかしい	
\\	いい加減な	
\\	いい加減にしろ-
\\	のんきな	
\\	だらしない	
\\	短気な 気が短い	タンキ 
\\	不潔な	フケツ 
\\	気が小さい	
\\	厚かましい 図々しい	
\\	アツカマシイ ズウズウシイ
\\	欲張りな	ヨクバリ 
\\	くどい	
\\	強引な	ゴウイン 
\\	強引に認めさせる 
\\	卑怯な	
\\	ヒキョウ
\\	乱暴な	ランボウ 
\\	ずるい 
\\	「ずるい!」	
\\	生意気な	ナマイキ 
\\	わがままな	
\\	ダサい	
\\	プライドが高い	
\\	上品な 品がある 品がいい	
\\	ジョウヒン ひんがいい
\\	下品な 品がない	ゲヒン 
\\	かっこいい	
\\	かっこ悪い	
\\	気が強い	
\\	気が弱い	
\\	器用な	
\\	キヨウ
\\	不器用な	ブキヨウ 
\\	真面目な	
\\	不真面目な	
\\	間抜けな	まぬけ 
\\	寝込む	ねこむ 
\\	会	カイ 
\\	興奮
\\	の	こうふん 
\\	興奮する 
\\	気を失う	
\\	続出する	ぞくしゅつ 
\\	豊かな国 
\\	貧しい	ユタカ 
\\	マズシイ 
\\	快適な暮らし	カイテキ 
\\	楽な方法	
\\	派手なシャツ 
\\	地味な	ハデ 
\\	じみ 
\\	粗末な食事	ソマツ 
\\	惨めな暮らし	ミジメ 
\\	厄介な問題	ヤッカイ 
\\	曖昧な返事	
\\	あいまい
\\	ありふれた話 
\\	珍しい	
\\	慌ただしい一日	アワタダシイ 
\\	騒がしい 騒々しい   通り	
\\	サワガシイ ソウゾウシイ
\\	(お)めでたい日	
\\	長さが等しい	ヒトシイ 
\\	醜い争い	ミニクイ 
\\	荒っぽい言葉	アラッポイ 
\\	空っぽの財布	カラッポ 
\\	真っ黒な 焼きすぎて魚が真っ黒になる	
\\	顔が真っ青になる	マッサオ 
\\	顔が真っ赤になる	
\\	まっか
\\	頭が真っ白になる	
\\	頭の中が真っ白になる	
\\	票	
\\	ヒョウ
\\	を睨む	
\\	ニラム
\\	部屋の中を覗く	
\\	ノゾク
\\	タバコを咥える//啣える	
\\	くわえる
\\	リンゴをかじる	
\\	ガムを噛む	カム 
\\	耳元で囁く	ミミモト で ササヤク 
\\	頷く	
\\	ウナズク
\\	肩に担ぐ	
\\	カツグ
\\	脇に抱える	ワキ
\\	カカエル 
\\	カゴをぶら下げる	
\\	肘をつく	
\\	ヒジ
\\	触る	サワル 
\\	頭をかく	
\\	ポケットに手を突っ込む	ツッコム 
\\	パンにハムを挟む	ハサム 
\\	ノリでくっつける	
\\	水道の蛇口をひねる	ジャグチ 
\\	*蛇=
\\	ひねる=
\\	ひねった
\\	膝を曲げる 
\\	伸ばす	ヒザ 
\\	溝を跨ぐ	ミゾ 
\\	マタグ 
\\	石に躓く	つまずく 
\\	雪道で滑る	ユキミチスベル 
\\	道路でしゃがむ	
\\	ドウロウ
\\	体をねじる	
\\	壁にもたれる	
\\	壁にぶつかる	
\\	壁にボールをぶつける	
\\	自転車とすれ違う	
\\	すれ違う=
\\	この道は狭くて、車と車がすれ違う
\\	地面を這う	
\\	蛇が) じめん ハウ
\\	お辞儀(を)する	
\\	ジギ
\\	通り過ぎる	
\\	その前を通り過ぎようとした女
\\	散る	チル 1。
\\	悔やむ 
\\	悔しい 後悔	クヤム・くやしい 
\\	コウカイ 
\\	情けない	なさけない 
\\	(こんな結果になったのは)やむを得ない	
\\	(<しょうがない)
\\	疑問に思う	
\\	私はそれを〜 
\\	納得がいかない/納得がいく	
\\	ナットク
\\	飽きる	アキル 
\\	に 飽きるほど=
\\	呆れる	アキレル 
\\	に あきれた人間 
\\	呆れて 
\\	聞いてあきれる 
\\	慌てる	
\\	アワテル
\\	焦る	アセル 
\\	気持ちが焦って勉強できない 
\\	焦げる
\\	ためらう	
\\	憧れる	アコガレル 
\\	恋しい	コイシイ 
\\	呟く	ツブヤク 
\\	溢れる	アフレル =溢れる(コボレル) 
\\	絡み合う	からみあう 
\\	懐かしい	ナツカシイ 
\\	曲。。。
\\	みっともない	
\\	態度
\\	惜しい	おしい! 
\\	惜しい。合格に1点不足だ。
\\	憎い 憎らしい	
\\	ニクラシイ
\\	夢中になる 
\\	退屈する	
\\	にムチュウ 
\\	タイクツだ&する&な
\\	~~することがなく退屈な日・退屈な話
\\	神に祈る	
\\	悩むー>悩み	
\\	神に誓う	チカウ 
\\	仏を拝む	ホトケをオガム 
\\	馬鹿らしい	バカラシイ 
\\	紙屑	カミクズ 
\\	埋め立てゴミ	
\\	茶碗=チャワン、皿、貝殻=カイガラ
\\	うめたて
\\	花だん	ハナダン 
\\	洗剤 薬剤 錠剤	センザイ 
\\	ジョウザイ
\\	原料	ゲンリョウ 
\\	焦点を当てる	ショウテン 
\\	焦点を合わす 
\\	問い合わせる	
\\	を尊敬する	
\\	を偉いと思う	エライ 
\\	信用する 
\\	疑う	
\\	いじめる	
\\	やっつける	
\\	からかう	
\\	を
\\	バカにする	
\\	を
\\	けなす	
\\	を
\\	責める	セメル 
\\	非を責める 
\\	脅かす	オドカス 
\\	頼る	
\\	に
\\	頼りにする	
\\	を
\\	慰める	ナグサメル 
\\	許す	ユルス 
\\	裏切る	ウラギル 
\\	騙す	ダマス 
\\	傷つける	
\\	キズ
\\	放っておく	ホウっておく 
\\	(私のことは)ほっといて!
\\	に従う 
\\	逆らう	
\\	サカラウ, 
\\	詫びる	ワビル 
\\	感謝する	
\\	冗談を言う	
\\	嫌味を言う	イヤミ 
\\	文句を言う	
\\	嘘をつく	
\\	嘘つき	
\\	黙る	ダマル 
\\	自慢する	
\\	を
\\	威張る	イバル 
\\	ふざける	
\\	ふざけるとぶつけられる 
\\	真似をする	マネ 
\\	悪口を言う	
\\	の
\\	友達に恵まれる	メグマレル 
\\	嫌な態度を取る	イヤ 
\\	態度を改める	
\\	失敗を人のせいにする	
\\	親指	オヤユビ 
\\	人差し指	
\\	ヒトサシ
\\	中指	ナカ 
\\	薬指	
\\	小指	コ 
\\	手首	
\\	テクビ
\\	肘	ヒジ 
\\	膝	ヒザ 
\\	かかと	
\\	つま先	
\\	裸足	ハダシ 
\\	=素足(スアシ)
\\	医師の診察を受ける	
\\	早めに医者にかかる	
\\	健康診断を受ける	
\\	血圧が高いい 
\\	低い	
\\	ケツハツ
\\	病気の症状	ショウジョウ 
\\	症状が治まる	
\\	オサマル
\\	顔色が悪い	
\\	顔が真っ青だ	
\\	体の具合が悪い	
\\	吐き気がする	ハキキ 
\\	吐く	ハク 
\\	頭がぼうっとする	
\\	熱中症	ネッチュウショウ 
\\	体に熱がこもる 
\\	家にこもる	
\\	体がだるい	
\\	肩がこる 
\\	肩凝り	
\\	カタコリ
\\	肩を揉む	カタをモム 
\\	ストレスがたまる	
\\	痛みをこらえる 痛みを我慢する 痛みに耐える	耐える 
\\	たえる
\\	に)、こらえる
\\	を) 
\\	肌が荒れる 
\\	肌のつやがいい	ハダ アレル 
\\	つや=
\\	体が痒い 
\\	(~を)かく	カユイ 
\\	湿疹ができる	シッシン 
\\	アレルギーがある	
\\	花粉症で鼻が詰まる	カフンショウ ツマル 
\\	虫歯になる	
\\	真っ白な歯	
\\	まっしろ
\\	リハビリをする	
\\	リハビリを受ける	
\\	傷口が塞がる	
\\	きずぐち ふさがる
\\	レントゲンを撮る	
\\	手術の傷跡が痛む	
\\	アト
\\	がんで死亡する	
\\	運動靴の紐が解ける	うんどうぐつ ヒモ ホドケル 
\\	必ずしも〜とは限らない	金持ちが必ずしも幸福だとは限らない 
\\	必ず	
\\	いつか	
\\	//ここにはいつか来たことがある 
\\	いつまでも	
\\	いつの間にか	マ 
\\	つい	
\\	ついに	
\\	どうしても	
\\	ーどうしても行けない 
\\	どうも	1。
\\	らしい、ようだ 2。どうも失礼しました/すみませんでした 3。
\\	どうも上手に話せない 4。昨日はどうも。•軽い挨拶 
\\	なんとか//どうにか	
\\	なんとか会議に間に合った。 この問題をなんとかしなければならない
\\	なんとなく	
\\	これはなんとなく面白そうな本だ。
\\	なんとも〜ない	
\\	(強調) そんなこと、なんとも思わない
\\	もし(も)	
\\	もしも100万円あったら
\\	もしかしたら、もしかすると	
\\	もしかすると、彼の話は嘘かもしれへん。
\\	なるべく	=できるだけ なるべく早く来てください
\\	なるほど	
\\	なるほど面白い本だ。 
\\	新しい情報の言う人(の話に)相槌を打つときにも使うが、目下の人が目上の人に言うと馴れ馴れしくて失礼な感じがするので注意。 •相槌=あいづちを打つ=
\\	馴れ馴れしい=なれなれしい=
\\	確か//確かに	1。確か=多分 あの人は確か林さんのお父さんだと思う 2。確かに=間違いなく あの人は確かに林さんのお父さんだ。
\\	統計	トウケイ 
\\	全力を尽くす	ゼンリョクヲツクス 
\\	大気	タイキ 
\\	濃度	
\\	ノウド
\\	上昇	ジョウショウ 
\\	天候 
\\	不順	テンコウ 
\\	天候不順 
\\	間も無く	まもなく 1。=もう直ぐ コンサートが間もなく始まる。 2。=少し後に 家に帰って間もなく客が来た。
\\	近々	ちかぢか =近日中に(キンジツチュウ) 
\\	そのうち	
\\	やがて	1。
\\	娘が結婚してやがて1年になる
\\	ようやく	=やっと 
\\	いずれ	
\\	=いつか 
\\	何れにしても(イズレ)
\\	たちまち	=直ぐに 
\\	直ちに	タダチに=すぐに 
\\	にわかに	
\\	=急に ー>にわか雨 
\\	絶えず	タエズ 
\\	常に	いつも 
\\	しきりに	1。=何度も 
\\	2。=絶えず 
\\	たい 
\\	じきに 
\\	しょっちゅう	
\\	//いつも//絶えず
\\	(もう)すでに	
\\	疾っくに	とっくに =ずいぶん前に、はるか前に 
\\	その店はとっくに閉まっていた 
\\	とっくの昔 
\\	前もって	
\\	=準備のために、前に 前もって、レストランの予約をしておいた
\\	ほぼ	
\\	=約, 
\\	お(お)よそ	
\\	=だいたい
\\	ほんの	
\\	たった	
\\	せいぜい	
\\	出席者はせいぜい30人程度
\\	だ。 
\\	できるだけ
\\	少なくとも	
\\	多くとも)
\\	せめて	1。
\\	3。多くとも, 
\\	度々/しばしば	タビタビ 
\\	いよいよ	
\\	ついに 
\\	ますます 
\\	いちいち	1。いちいちを述べる 
\\	2。課長は私の仕事にいちいち文句をつける(を言う)。 
\\	ふわふわ	1。
\\	(ふわふわと浮かぶ雲) 2。
\\	(ふわふわする)
\\	まごまごする	
\\	それぞれ/銘々/一人一人/各々	銘々 メイメイ 各々 オノオノ 
\\	ばったり	1。ばったり会う 
\\	に 2。ばったりと 
\\	こっそり	
\\	ぴったり	1。
\\	ぎっしり	
\\	の/が 
\\	にぎっしり入ってる 
\\	じっくり 
\\	ちらっと/ちらりと	
\\	うんざり	
\\	する 
\\	だ 
\\	びっしょり びしょびしょ	
\\	濡れる 
\\	びしょ濡れになった。
\\	しいんと=シーンと	
\\	する 
\\	静まる
\\	ずらりと ずらっと	
\\	並ぶ
\\	さっさと	
\\	せっせと	
\\	どっと	1。
\\	くる
\\	場所に
\\	すっと	
\\	立ち上がる 
\\	する 
\\	一人でに	
\\	=何もしていないのに ドアの前に立つと、一人でにドアが開いた
\\	一斉に	イッセイ =多くのものが同時に 鳥たちは一斉に湖から飛び立った 
\\	一斉にスタートする
\\	開花	カイカ 
\\	相当	
\\	する 
\\	十円は何リラに相当しますか 
\\	の
\\	能力相当の仕事 
\\	かなり, 
\\	大いに	おおいに 1。
\\	ある 
\\	うんと	
\\	ある 
\\	=とてもたくさん いたずらをして、先生にうんと叱られた
\\	たっぷり	
\\	あまりにも	
\\	やや	
\\	=少し 経済対策により、景気はやや上向いた
\\	大きめの
\\	多少
\\	幾分 幾らか	イクブン イクラカ 
\\	割に 割と	ワリ 
\\	の割に
\\	割に
\\	が 
\\	割合(に)	
\\	大きい割合に軽い 
\\	なお(いっそう)	
\\	より(一層)	イッソウ 
\\	むしろ	
\\	よりむしろ
\\	の方がいい
\\	余計(に)	
\\	、 もっと
\\	徐々に	ジョジョに 
\\	次第に	
\\	さらに	
\\	一段と	
\\	ぐっと	
\\	めっきり	
\\	主に	オモニ 
\\	くれぐれも	
\\	繰り返して、十分に
\\	「くれぐれもご両親によろしくお伝えください」 くれぐれも体に気を付けてください
\\	一応	
\\	一般に	
\\	明らかに	
\\	まし	
\\	発言	ハツゲン 
\\	政治家	せいじか 
\\	ひったくり	
\\	いわば	
\\	、何かに例えていうと 彼女はいわば「クラスのアイドル」でした 
\\	彼と私はいわば兄弟のようなものだ 
\\	いわゆる	
\\	、一般で言われている 彼はいわゆる「オレオレ詐欺」の犯人だった 
\\	まさか	
\\	まさに	1。
\\	通り 2。
\\	ようとしている
\\	宴会 披露宴	エンカイ 
\\	ひろうえん
\\	一遍に	イッペン =一度に 
\\	一気に	
\\	思い(っ)切り	
\\	思い切って	
\\	思わず	
\\	思いがけず 思いがけなく	
\\	何しろ	ナニ 
\\	(=とにかく)//
\\	なにかと	
\\	相変わらず	
\\	取り敢えず	トリアエズ 
\\	わざと	
\\	わざわざ	
\\	一層	イッソウ 
\\	いっそ	
\\	いっそ死にたい。
\\	今に	
\\	今にも 
\\	ようとしている/
\\	2そうだ	
\\	今更	イマサラ 
\\	未だに	イマダ 
\\	只今	タダイマ 
\\	たった今	
\\	書き上げる	
\\	皮を剥く	ムク 
\\	別にーない	
\\	そうーない	
\\	大してーない	タイシテ 
\\	一切ーない	イッサイ 
\\	とてもーない	
\\	おそらくーだろう	多分 
\\	どうやら ーそうだ どうやら ーようだ どうやら+
\\	どうやら+
\\	なんとか
\\	果たしてーだろうか	
\\	どうせーだろう	
\\	せっかくーのに	
\\	せっかくーから	
\\	一旦、一度、一度 〜したら	イッタン、イチド、ヒトタビ 
\\	一旦、ひとまず	
\\	かえって	
\\	近道する	チカミチ 
\\	さっそく	
\\	さっさと, 
\\	さすが(に)	
\\	あくまで(も)	
\\	主張する	シュチョウ 
\\	なんだか、なんとなく、なぜか	
\\	なんと なんて	
\\	電動+
\\	遺伝	いでん 
\\	凶器	キョウキ 
\\	人物	
\\	ジンブツ
\\	物理	ブツリ 
\\	食物	ショクモツ 
\\	植物
\\	作物 
\\	さくもつ 
\\	さくぶつ 
\\	主要な	
\\	生き物	いきもの =生物 せいぶつ 
\\	生物学//
\\	微生物(ビ)
\\	なまもの 
\\	生物につき要冷蔵 
\\	入れ物	
\\	物音	
\\	物置 にしまう	モノオキ 
\\	月日	ツキヒ 
\\	月日が流れる 
\\	がっぴ 
\\	後日	ゴジツ 
\\	今日	コンニチ 
\\	日中	ニッチュウ 
\\	日の出	ヒのデ 
\\	日の入り	ヒのイリ 
\\	来日する	
\\	年間	ネンカン 
\\	年月	ネンゲツ・トシツキ 
\\	年中 年がら年中	
\\	ねんじゅう ねんがらねんじゅう 
\\	=いつでも あの兄弟は、年中けんかしている
\\	年度	
\\	ねんど
\\	少年	
\\	青少年	
\\	せいしょうねん
\\	青年	
\\	年代	
\\	道具	
\\	指名する	
\\	大工	ダイク 
\\	重大な	
\\	大小	ダイショウ 
\\	大半	タイハン 
\\	大金	タイキン 
\\	大木 
\\	樹齢)	タイボク 
\\	樹齢(じゅれい)
\\	樹齢100年の大木
\\	中心	
\\	中世	
\\	チュウセイ
\\	中古の
\\	チュウコ 
\\	空中	
\\	くうちゅう
\\	夜中	
\\	世の中	
\\	手間	
\\	中間	
\\	中間試験 
\\	世間	
\\	せけん
\\	昼間	
\\	夜間	ヤカン 
\\	外出	
\\	週間	
\\	名作	
\\	作者	
\\	作品	
\\	作業する	サギョウ 
\\	通行する 
\\	一方通行	
\\	イッポウツウコウ 
\\	通知する	
\\	文通する	
\\	文通相手 
\\	一通り	ヒトトオリ 
\\	生じる	しょうじる 
\\	生える	はえる 
\\	一生	
\\	人生	ジンセイ 
\\	生産する	
\\	生け花	イケバナ
\\	生	ナマ 
\\	学力	
\\	学習する	
\\	学者	
\\	用いる	モチイル 
\\	閣僚(かくりょう 
\\	に民間人を用いる 
\\	用語	
\\	化学用語、専門用語
\\	用紙	
\\	用心する	
\\	に
\\	使用する	
\\	引用する	
\\	インヨウ
\\	通用する	
\\	急用	急な用事 
\\	日用品	ニチヨウヒン 
\\	見学する	
\\	見事な	
\\	見方	
\\	見本	
\\	見回る	
\\	励む	ハゲム 
\\	に 勉学に励む
\\	地方	
\\	関東地方 
\\	地区	
\\	住宅地区 
\\	地理	
\\	→東京の地理に詳しい 
\\	地理学 
\\	地下(街)	
\\	土地	
\\	地元	
\\	地味な 
\\	派手な	
\\	パンの生地	
\\	名所	メイショ 
\\	名人	
\\	名物	メイブツ 
\\	苗字|名字	
\\	下の名前 
\\	本名	
\\	ホンミョウ
\\	あだ名	ニックネーム
\\	発売する	
\\	発明する	
\\	発言する	
\\	会議で)
\\	発見する	
\\	発生する	
\\	台風が), 
\\	発車する	
\\	定刻に	テイコク 
\\	会合(を開く//する)	カイゴウ 
\\	=集会
\\	会場	
\\	開会する →開会式 (例:五輪の) →閉会式	
\\	大会	
\\	出会い	
\\	偶然の出会い 
\\	上京する	
\\	手当てする	テアテ 
\\	手入れする	テイレ 
\\	の手入れをする 状態を保つためにすること 週末に庭の手入れをするのが趣味です
\\	手書き	
\\	手書きの
\\	手書きで
\\	を書く
\\	手作り	
\\	手作りのケーキ
\\	手品	テジナ 
\\	を使う トランプの手品 
\\	手前	
\\	新橋は東京の二つ手前だ。
\\	話し手	
\\	聞き手	
\\	人手	ヒトデ 
\\	手話	シュワ 
\\	合計する	
\\	合計を出す 
\\	合理的な	
\\	合同	ゴウドウ 
\\	合同で 
\\	集合する	
\\	集合場所
\\	都合 →をつける	
\\	合図する	アイズ 
\\	合間	アイマ 
\\	の ー
\\	仕事の合間 
\\	体重	タイジュウ 
\\	体力	
\\	気力 
\\	死体	
\\	重体だ	
\\	ジュウタイ
\\	渋滞する	ジュウタイ
\\	強化する	キョウカ 
\\	強力な	
\\	接着剤	セッチャクザイ 
\\	弱気な	
\\	よわき
\\	力強い	チカラヅヨイ 
\\	休業	
\\	本年	→本年もよろしくお願いします。 
\\	ほんねん
\\	本社	
\\	本店	
\\	支店 
\\	本人	
\\	本気	本気にする 
\\	本気を出す 
\\	本来の
\\	本来なら 
\\	家屋	カオク 
\\	一家	
\\	芸術家	ゲイジュツカ 
\\	読書家	
\\	作家	
\\	画家	ガカ 
\\	無名	
\\	大家
\\	大家	オオヤ 
\\	家主 
\\	たいか 
\\	絵のたいか 
\\	目上の
\\	目下の
\\	目安	めやす 
\\	目安を立てる 
\\	目指す	
\\	を
\\	目立つ	
\\	注目する	
\\	に
\\	気体・液体・固体	
\\	気分がいい	
\\	気味が悪い いい気味だ	キミ 
\\	いい気味だ 
\\	気楽な	
\\	平気な	
\\	短気な	
\\	=気が短い
\\	白髪	シラガ 
\\	白髪交じりの髪 
\\	厳密に言えば	ゲンミツ 
\\	てへぺろ	
\\	検討 討議	ケントウ 
\\	提案を検討する 
\\	トウギ 
\\	訛	ナマリ 
\\	訛る
\\	大晦日	オオミソカ 
\\	アマ	アマチュア
\\	スト	ストライク
\\	ホイル	アルミンホイル
\\	インフレ 
\\	デフレ	インフレーション
\\	緑黄色野菜	
\\	りょくおうしょくやさい
\\	クレームをつける・いう	
\\	サービス精神旺盛	オウセイ 
\\	ガソリンスタンド 電気スタンド	
\\	予算がオーバーする オーバーに話す	
\\	トレーナー	
\\	コンパ	
\\	それぞれ	
\\	ワンパターンな	
\\	面白みがない
\\	車を
\\	ターンさせる 
\\	ターン現象	
\\	都会に出たヒヨが故郷に戻ることを言う。
\\	シルバーシート	
\\	サインペン	
\\	ホチキスで閉じる	
\\	合唱	がっしょう 
\\	スマート	
\\	心がける ー
\\	心がけ	
\\	安全運転を ー
\\	心がけがよくない 
\\	ノルマ	
\\	仕事のノルマをこなす 
\\	こなす	
\\	跳ねる	ハネル 
\\	転がる、転がす	
\\	ボールが道を転がる ボールを転がす
\\	ちぎる	
\\	破る →が破ける	紙を破る・やぶく 約束を 記録を
\\	犬が吠える	ホエル
\\	犬が唸る	ウナル 
\\	つるす	
\\	ぶら下げる	
\\	舐める	ナメル 
\\	しゃぶる	
\\	予定をずらす	
\\	→ずれる
\\	退ける	ドケル 
\\	退く
\\	木が枯れる 花がしぼむ	
\\	新たにする	
\\	改めて	
\\	改める	
\\	料金を 
\\	先に	
\\	先に述べたように。。。。
\\	先ほど	
\\	天井<ー
\\	床	テンジョウ<ー
\\	ユカ 
\\	芝生	しばふ 
\\	国宝	
\\	こくほう
\\	論戦	
\\	第一(に)	
\\	真っ先に	
\\	かつて	
\\	かつてない
\\	単なる
\\	ただの
\\	普通の
\\	馬鹿な真似をする	
\\	ふり	
\\	無理に 無理やり	
\\	上等な	
\\	高度な	
\\	高価な	
\\	続々と	
\\	(一人)当たり	
\\	田中夫人	フジン 
\\	紳士服←→婦人服	
\\	人工の真珠	
\\	特徴 象徴 徴収	トクチョウ
\\	ショウチョウ
\\	ハトは平和を象徴する チョウシュウ
\\	特長=特色	
\\	とくしょく
\\	あくる	
\\	もたらす	
\\	被害をもたらす=引き起こす
\\	実に	
\\	からと言って	
\\	賞味 →賞味期限が切れる	
\\	痺れが切れる =足がしびれる	
\\	しびれる
\\	野菜の水気を切る	
\\	スタートを切る	
\\	100メートル競走で10秒を切る	
\\	競走
\\	カードをよく切る	
\\	俗語	ぞくご 
\\	染みがつく	シミ 
\\	水滴がつく	
\\	利子がつく	
\\	りし
\\	力がつく 力をつける	
\\	差がつく 差をつける	
\\	見当がつく 見当をつける	
\\	めどがつく 
\\	めどをつける 
\\	めどにする	目処=
\\	目標、見通し 
\\	めどがつかない 
\\	・めどがつく 
\\	(の~) 
\\	の 
\\	を
\\	決心がつく	
\\	折り目をつける	
\\	日が当たる 日を当てる	
\\	アイロン	
\\	申し込む	
\\	太陽に雲がかかる	
\\	エンジンがかかる エンジンをかける	
\\	優勝がかかる	
\\	橋をかける	
\\	植木に水をかける	ウエキ 
\\	保険をかける	
\\	場所を取る	
\\	下準備	
\\	したじゅんび
\\	機嫌を取る	
\\	大事を取る	
\\	ズボンのゴムが伸びる	
\\	アイロンをかけてシワを伸ばす	
\\	口が軽い←→口が硬い	
\\	軽い(簡単な)気持ちで引き受ける	
\\	最近の人たちは政治に暗い←→明るい	
\\	理想が高い←→低い	
\\	鼻が高い	自慢だ 
\\	立派な息子を持って鼻が高い 
\\	格式	
\\	→格式が高いホテル
\\	ベルトがきつい←→ゆるい	
\\	きつい
\\	辛い、厳しい	
\\	きつい性格の女性	
\\	気が強い
\\	日差しがきつい・強い	
\\	目つきがきつい・鋭い	
\\	きつい・強い←→軽い酒	
\\	関心が高い←→低い	
\\	不愉快な	フユカイ
\\	認識	
\\	にんしき
\\	長持ちする	
\\	非〜 =ではない 非常識 非科学的 非公開 非公式	
\\	現大臣	←→前 
\\	←→元ダイジン 
\\	故田中大臣	こ 
\\	副大臣	
\\	ふくだいじん
\\	副作用	フクサヨウ 
\\	急病	
\\	ありのまま	
\\	売上金	ウリアゲキン 
\\	運送料	
\\	拝観料	
\\	本代	
\\	本題	
\\	借り賃	
\\	貸賃	
\\	手間賃	
\\	収まる	オサマル 
\\	比較的大きい	
\\	ひかくてき
\\	進歩的な	
\\	~風	
\\	サラリーマン風の男 
\\	西洋風の建物 
\\	関西風の味付け 
\\	立体感	リッタイ 
\\	立体 
\\	存在感	
\\	存在感がある人 
\\	合理化	
\\	少子化	ショウシカ 
\\	変わり目	
\\	季節の変わり目 
\\	誰も	
\\	〜ごとに	
\\	ごとに=たびに
\\	一雨	ヒトアメ 
\\	〜おきに	
\\	一行おきに書く 
\\	〜ぶり	
\\	仕事振り 
\\	話しぶり 身振り 
\\	(みぶり) 彼と5年ぶりに会った
\\	〜こなす	
\\	パソコンを使いこなす 
\\	洋服を着こなす 
\\	~つき 顔つき	
\\	顔つき 
\\	目つき 
\\	〜たて	
\\	炊きたてのご飯 
\\	焼きたてのパン ペンキ塗りたて 
\\	独特な	
\\	〜づらい
\\	~がたい	づらい=にくい, 
\\	がたい, 
\\	部品を組み立てる	ブヒン 
\\	帰る人を引き止める	
\\	引き返す	
\\	途中から(自分を)引き返す 
\\	来た道を引き返す 
\\	受け取る	
\\	受け持つ	
\\	→受け持ち 
\\	打ち消す	
\\	噂を打ち消す)
\\	売り上げを伸ばす	
\\	売れ行きがいい	
\\	売り出す→売り出し	
\\	取り上げる 問題を取り上げる	
\\	問題を取り上げる 
\\	取り入れる	
\\	取り扱う→取り扱い	
\\	取り付ける←→取り外す	
\\	障害物	
\\	しょうがいぶつ
\\	(後ろを)振り向く=振り返る	
\\	持ち上げる	
\\	その男の子は重い箱を片手で持ち上げた
\\	立て替える	
\\	=少しの間、その人の代わりに金を支払う 財布を忘れてきたので、昼食代を友達に立て替えてもらった
\\	追いつく	
\\	に
\\	追い出す→追い出される	
\\	乗り過ごす=乗り越す→乗り越し	
\\	(乗り越す
\\	(電車)に乗り遅れる	
\\	割り込む→割り込み乗車	
\\	列に) ー
\\	割り込み乗車= 
\\	当てはめる	
\\	例 自分に当てはめて考える 
\\	当てはまる	
\\	犬に当てはまることはネコにも当てはまる 
\\	思い込む→思い込み	
\\	思いつく	
\\	を
\\	見つめる	
\\	見かける	
\\	見直す→見直し	
\\	聞きなおす	
\\	見慣れる	
\\	見渡す	
\\	景色)
\\	書き留める	
\\	が仕上がる→仕上がり	
\\	仕上げる→仕上げ	
\\	呼び出す	
\\	貸し出す→貸し出し	
\\	3割引	
\\	清算	セイサン 
\\	目に見えて	
\\	目を向ける	
\\	に
\\	目が離せない	
\\	から・
\\	は
\\	目に付く	
\\	彼の家はすぐに目に付く
\\	目がない	
\\	に
\\	目が回る→目が回るように忙しい	
\\	目に浮かぶ	
\\	故郷の様子が目に浮かぶ 
\\	に目を通す	
\\	目を盗む	
\\	の
\\	目にあう	
\\	例:ひどい目にあう 大変目にあう 危険な目にあう
\\	耳にする	
\\	耳が痛い	
\\	耳が遠い	
\\	耳を貸す	ちょっと聞いて 
\\	彼は私の話に耳を貸してくれなかった
\\	耳が早い	
\\	耳を疑う	
\\	口が上手い	
\\	口が堅い	カタイ 
\\	口が悪い	
\\	口に合わない	
\\	口にする	
\\	口に出す	
\\	顔が利く	キク 
\\	私はその店に顔が利く
\\	顔が広い	
\\	彼はとても顔が広い
\\	顔を貸して	
\\	顔を出す	
\\	頭に来る=腹がたつ	
\\	に
\\	頭を下げる	
\\	頭が固い	
\\	に手をつける	
\\	手が空く	アク 
\\	手がかかる	
\\	手がない	
\\	手が離せない=手がふさがる	
\\	を手に入れる 
\\	が手に入る	
\\	手にかかる	
\\	が手につかない	
\\	が
\\	手を貸して	手伝って
\\	手を休める	
\\	手を入れる->手入れ	
\\	気が合わない	
\\	気が重い	
\\	気がしない	
\\	気が進まない	
\\	気にかかる	
\\	が
\\	気に食わない=気に入らない	イラナイ 
\\	気を落とす	
\\	に気を使う	
\\	気を悪くする	
\\	腕がいい	
\\	腕が上がる<ーー
\\	腕が落ちる	
\\	の
\\	腕を磨く	
\\	足を伸ばす	
\\	京都まで足を伸ばそう=
\\	足がない	
\\	足が運ぶ	
\\	足が出る	
\\	経費がかかりすぎて足が出た
\\	気が利く	
\\	気が利いた冗談
\\	評判	
\\	ひょうばん
\\	母親の買い物についていく	
\\	今日はついて(い)る<ーー
\\	ついて(い)ない	
\\	この料理はなかなかいける	
\\	・つまらないことにこだわる
\\	マイナスの意味 ・材料にこだわって料理を作る
\\	プラスの意味	
\\	=こだわりがある・ない
\\	ないよりはマシだ	
\\	まともな	
\\	まともな仕事 2。風邪をまともに受ける 
\\	ろくな仕事がない	
\\	ろくに料理を作らない	
\\	大した	
\\	あっという間に	"マ 
\\	あれこれ アレヤコレヤ なんだかんだ	
\\	例:あれこれ言っても、間に合わない
\\	当たり前のやり方	
\\	-仕事の覚えが悪い -そんなことを言った覚えはない	
\\	待ち合わせには余裕を持って出かけよう。	
\\	潰れる	ツブレル 
\\	胸がつぶれる 
\\	見聞きする	
\\	行き来する	イキキ 
\\	貸し借りする	
\\	出し入れする	
\\	布団の出し入れは面倒だ)
\\	付け外しできる	
\\	襟 シャツの襟 襟を正す	えり
\\	えりをただす 
\\	脱ぎ着する	ヌギキ 
\\	読み書きする	
\\	上げ下げする	
\\	売り買いする	
\\	行き帰り	
\\	好き嫌い	
\\	食べ物の好き嫌いが激しい
\\	少し様子を見ましょう	
\\	試しにやって見ましょう	
\\	(娘が)いつもお世話になっています	仕事などで最初の挨拶によく使う 
\\	そんなつもりはありません	
\\	困ったときはお互い様です	オタガイサマ=
\\	ノリが悪い人	
\\	ノリが悪い=
\\	この曲はノリがいい	
\\	砕け	クダケ 
\\	いまいち	あとちょっと 
\\	例:このケーキは、いまいちだ=あまり美味しくない
\\	秘密をバラす	
\\	バレる 
\\	パクる	
\\	うざい 煩わしい うっとおしい/うっとうしい	煩わしい (わずらわしい) 
\\	誠実な	セイジツ 
\\	自分の良心(りょうしん 
\\	に従って、一生懸命行動する人
\\	赤道	
\\	せきどう
\\	相槌	あいづち 
\\	相槌を打つ 
\\	どこぞ	
\\	よろしくやる	
\\	勘定をください	
\\	気圧	きあつ 
\\	措置 救済措置 緊急措置 延命措置 法的措置	そち 
\\	除名の措置=
\\	きゅうさい 
\\	きんきゅうそち 
\\	エンメイ
\\	ホウテキ 
\end{CJK}
\end{document}