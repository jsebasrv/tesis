\documentclass[8pt]{extreport} 
\usepackage{hyperref}
\usepackage{CJKutf8}
\begin{document}
\begin{CJK}{UTF8}{min}
\\	私は
\\	に花をあげた。	
\\	私[わたし]は 
\\	に 花[はな]をあげた。	
\\	私は
\\	にネクタイを買ってあげた。	
\\	私[わたし]は 
\\	にネクタイを 買[か]ってあげた。	
\\	私がご飯を食べている間、山田さんはテレビを見ていた。	
\\	私[わたし]がご 飯[はん]を 食[た]べている 間[あいだ]、 山田[やまだ]さんはテレビを 見[み]ていた。	
\\	いる間(に) 
\\	間(に) 
\\	な間(に) 
\\	の間(に)	
\\	この本はあまりよくない/よくありません。	
\\	この 本[ほん]はあまりよくない/よくありません。	
\\	この町(に)は大学が三つある。 エッフェル塔はパリにある。	
\\	この 町[まち](に)は 大学[だいがく]が 三[みっ]つある。 エッフェル塔[えっふぇるとう]はパリにある。	
\\	それはもうジョンに話してある/あります。	
\\	それはもうジョンに 話[はな]してある/あります。	
\\	ある	
\\	メイソンさんは日本へ行ったあとで病気になった。 私は授業のあとで図書館に行った。	
\\	メイソンさんは 日本[にっぽん]へ 行[い]ったあとで 病気[びょうき]になった。 私[わたし]は 授業[じゅぎょう]のあとで 図書館[としょかん]に 行[い]った。	
\\	あとで 
\\	のあとで	
\\	この薬を飲めば、よくなる。	
\\	この 薬[くすり]を 飲[の]めば、よくなる。	
\\	話せば 食べれば 来れば すれば 高ければ 静かなら(ば) 
\\	静かで あれば 
\\	なら(ば) 
\\	で あれば	
\\	デザットはたべるばかりになっている。 友子は遊んでばかりいる。	
\\	デザットはたべるばかりになっている。 友子[ゆうこ]は 遊[あそ]んでばかりいる。	
\\	ばかり たかい ばかり 静かな ばかり	
\\	先生に聞けばよかった(です)。	
\\	先生[せんせい]に 聞[き]けばよかった(です)。	
\\	話せば よかった 食べれば よかった くれば よかった すれば よかった	
\\	ビルは日本へ行けばよかったと思っている。 
\\	どこが静かだい。 だれが行くんだい。	
\\	どこが 静[しず]かだい。 だれが 行[い]くんだい。	
\\	静か だい 話す/話した ん だい 高い/高かった ん だい 静かな/静かだった ん だい 先生な/先生だった ん だい	
\\	ん 
\\	の) 
\\	スミスさんだけ(が)来た。	
\\	スミスさんだけ(が) 来[き]た。	
\\	先生だけ(が/を) 話す/話した だけ だ/です 高い/高かった だけ だ/です 静かな/静かだった だけ だ/です	
\\	あの人はよく勉強するだけで(は)なくよく遊ぶ。	
\\	あの 人[ひと]はよく 勉強[べんきょう]するだけで(は)なくよく 遊[あそ]ぶ。	
\\	だけ で(は) なく 
\\	な/だった 
\\	だけ で(は) なく 
\\	だった) だけ で(は) なく	
\\	さんは日本へ行くだろう/でしょう。	
\\	さんは 日本[にっぽん]へ 行[い]くだろう/でしょう。	
\\	話す/話した だろう 高い/高かった だろう 静か/静かだった だろう 先生/先生だった だろう	
\\	車が動きだした。	
\\	車[くるま]が 動[うご]きだした。	
\\	話し出す 食べ出す	
\\	先生でも間違う。	
\\	先生[せんせい]でも 間違[まちが]う。	
\\	先生 が/を 
\\	せんせい でも 先生 と でも 
\\	東京 から でも 
\\	だれでも 
\\	何でも 
\\	どれでも 
\\	お母さんはどう(です か)。	
\\	お 母[かあ]さんはどう(です か)。	
\\	ジョンは来たがメアリーは来なかった。	
\\	ジョンは 来[き]たがメアリーは 来[こ]なかった。	
\\	が 
\\	一男はスポーツカーを欲しがった。 上田さんはアイスクリームを食べたがりました。	
\\	一 男[なん]はスポーツカーを 欲[ほ]しがった。 上田[うえだ]さんはアイスクリームを 食[た]べたがりました。	
\\	がる	
\\	すずきさんは朝七時ごろうちを出る。	
\\	すずきさんは 朝[あさ]七 時[じ]ごろうちを 出[で]る。	
\\	ごろ(に)	
\\	私は三時間ごとに薬を飲んだ。 正月は家ごとに門松を立てる。	
\\	私[わたし]は三 時間[じかん]ごとに 薬[くすり]を 飲[の]んだ。 正月[しょうがつ]は 家[いえ]ごとに 門松[かどまつ]を 立[た]てる。	
\\	明日はからレポウートを書きはじめる。	
\\	明日[あした]はからレポウートを 書[か]きはじめる。	
\\	はじめる	はじめる 
\\	クラークさんはパーティーに行くはずです。	
\\	クラークさんはパーティーに 行[い]くはずです。	
\\	話す/話した はず だ 高い/高かった はず だ 静かな/静かだった はず だ 先生の/先生だった はず だ	
\\	私はケンほど強くない。 この仕事は子供でも出来るほどやさしいです。	
\\	私[わたし]はケンほど 強[つよ]くない。 この 仕事[しごと]は 子供[こども]でも 出来[でき]るほどやさしいです。	
\\	ほど 
\\	君[きみ] ほど 
\\	ほど 
\\	それ ほど 
\\	ほど 
\\	家が倒れる ほど 
\\	ビールを三本ほど 
\\	日本語の本を読んだほうがいい。	
\\	日本語[にほんご]の 本[ほん]を 読[よ]んだほうがいい。	
\\	ほうがいい	
\\	こーヒーを飲まないほうがいい。 
\\	石田さんのほうが私より若い。 車で行くほうがバスで行くより安いです。	
\\	石田[いしだ]さんのほうが 私[わたし]より 若[わか]い。 車[しゃ]で 行[い]くほうがバスで 行[い]くより 安[やす]いです。	
\\	私は車が欲しい。 弟は僕の自転車を欲しがっています。	
\\	私[わたし]は 車[くるま]が 欲[ほ]しい。 弟[おとうと]は 僕[ぼく]の 自転車[じてんしゃ]を 欲[ほ]しがっています。	
\\	私はあなたに英語を教えてほしい。	
\\	私[わたし]はあなたに 英語[えいご]を 教[おし]えてほしい。	
\\	ほしい	
\\	クラス(の中)で大川さんが一番頭がいい。	
\\	クラス(の 中[なか])で 大川[おおかわ]さんが一 番頭[ばんあたま]がいい。	
\\	田中さんは来週アメリカへ/に行きます。 私は東京から大阪までバスで行った。	
\\	田中[たなか]さんは 来週[らいしゅう]アメリカへ/に 行[い]きます。 私[わたし]は 東京[とうきょう]から 大阪[おおさか]までバスで 行[い]った。	
\\	これからは寒くなって行くよ。	
\\	これからは 寒[さむ]くなって 行[い]くよ。	
\\	行く	
\\	この町(に)は日本人がたくさんいる。 リーはこの寮にいます。	
\\	この 町[まち](に)は 日本人[にっぽんじん]がたくさんいる。 リーはこの 寮[りょう]にいます。	
\\	佐々木さんは酒を飲んでいる。	
\\	佐々木[ささき]さんは 酒[さけ]を 飲[の]んでいる。	
\\	いる	
\\	君たちは英和辞典がいる。	
\\	君[きみ]たちは 英和[えいわ] 辞典[じてん]がいる。	
\\	土田さんは幸子が自分を愛していることを知らなかった。	
\\	土田[つちた]さんは 幸子[こうじ]が 自分[じぶん]を 愛[あい]していることを 知[し]らなかった。	
\\	日本人は自分の国の文化をユニークだと思っている。 メアリさんは自分で何でもします。	
\\	日本人[にっぽんじん]は 自分[じぶん]の 国[くに]の 文化[ぶんか]をユニークだと 思[おも]っている。 メアリさんは 自分[じぶん]で 何[なに]でもします。	
\\	私は電車かバス(か)で行く。 トムが行くかメアリーが行くかどちらかです。	
\\	私[わたし]は 電車[でんしゃ]かバス(か)で 行[い]く。 トムが 行[い]くかメアリーが 行[い]くかどちらかです。	
\\	か 
\\	(か) 静か/静かだった か 先生/せんせいだった か	
\\	鈴木さんが大学に入ったか(どうか)(は)知らない。	
\\	鈴木[すずき]さんが 大学[だいがく]に 入[はい]ったか(どうか)(は) 知[し]らない。	
\\	日本語はおもしろいかい。	
\\	日本語[にほんご]はおもしろいかい。	
\\	午後雨が降るかもしれない。	
\\	午後[ごご] 雨[あめ]が 降[ふ]るかもしれない。	
\\	かもしれない 静か/静かだった かもしれない 先生/先生だった かもしれない	
\\	パーチィーは八時から始まる。	
\\	パーチィーは八 時[じ]から 始[はじ]まる。	
\\	雪子は晩ご飯を食べてから映画に行った。	
\\	雪子[ゆきこ]は 晩[ばん]ご 飯[はん]を 食[た]べてから 映画[えいが]に 行[い]った。	
\\	から	
\\	来年日本へ行くから日本語を勉強している。 来年日本へ行くからです。	
\\	来年[らいねん] 日本[にっぽん]へ 行[い]くから 日本語[にほんご]を 勉強[べんきょう]している。 来年[らいねん] 日本[にっぽん]へ 行[い]くからです。	
\\	から 静かだ/静かだった から 先生だ/先生だった から	
\\	松本さんは来るかしら。	
\\	松本[まつもと]さんは 来[く]るかしら。	
\\	かしら 静か/静かだった かしら 先生/先生だった かしら	
\\	ケーキの作り方を教えてください。	
\\	ケーキの 作り方[つくりかた]を 教[おし]えてください。	
\\	方	
\\	勉強/電話/料理/説明の仕[し]方。
\\	先生のかわりに私が教えた。 土曜日に仕事をするかわりに月曜日は休みます。	
\\	先生[せんせい]のかわりに 私[わたし]が 教[おし]えた。 土曜日[どようび]に 仕事[しごと]をするかわりに 月曜日[げつようび]は 休[やす]みます。	
\\	のかわりに 
\\	かわりに 
\\	な/だった かわりに	
\\	私は言わなかったけれども、トムはしっていた。	
\\	私[わたし]は 言[い]わなかったけれども、トムはしっていた。	
\\	けれども 静かだ/静かだった けれども 先生だ/先生だった けれども	
\\	私(に)はうぐいすの声がよく聞こえる。	
\\	私[わたし](に)はうぐいすの 声[こえ]がよく 聞[き]こえる。	
\\	僕はチーズがきらいだ。	
\\	僕[ぼく]はチーズがきらいだ。	
\\	いいことを教えてあげよう。 論文に書いたことを話してください。 ブラウンさんは日本の大学のことをよく知っています。	
\\	いいことを 教[おし]えてあげよう。 論文[ろんぶん]に 書[か]いたことを 話[はな]してください。 ブラウンさんは 日本[にっぽん]の 大学[だいがく]のことをよく 知[し]っています。	
\\	こと 
\\	な/だった こと 
\\	のこと	
\\	小説を書くことは難しい。	
\\	小説[しょうせつ]を 書[か]くことは 難[むずか]しい。	
\\	私はヨーロッパへ行ったことがある。 レタスがとても高かったことがあります。	
\\	私[わたし]はヨーロッパへ 行[い]ったことがある。 レタスがとても 高[たか]かったことがあります。	
\\	ことがある	
\\	私は朝ふろに入ることがある。	
\\	私[わたし]は 朝[あさ]ふろに 入[はい]ることがある。	
\\	田口さんは中国語を話すことが出来る。	
\\	田口[たぐち]さんは 中国語[ちゅうごくご]を 話[はな]すことが 出来[でき]る。	
\\	ことが出来る	
\\	私は来年大阪に転勤することになった。 日本では車は道の左側を走ることになっています。	
\\	私[わたし]は 来年[らいねん] 大阪[おおさか]に 転勤[てんきん]することになった。 日本[にっぽん]では 車[くるま]は 道[みち]の 左側[ひだりがわ]を 走[はし]ることになっています。	
\\	ことになる	
\\	私は会社をやめることにした。	
\\	私[わたし]は 会社[かいしゃ]をやめることにした。	
\\	ことにする	
\\	私はテニスをすることはするが上手じゃない。	
\\	私[わたし]はテニスをすることはするが 上手[じょうず]じゃない。	
\\	話すことははなす 静かなことは静かだ 静かだったことは静かだった いい人はいい人だ いい人だったことはいい人だった	
\\	日本語で書いてください。 英語を使わないでください。	
\\	日本語[にほんご]で 書[か]いてください。 英語[えいご]を 使[つか]わないでください。	
\\	ください 
\\	ないでください	
\\	東京からサンフランシスコまで飛行機で九時間くらいかかる。	
\\	東京[とうきょう]からサンフランシスコまで 飛行機[ひこうき]で九 時間[じかん]くらいかかる。	
\\	大川さんは(私に)本をくれた。	
\\	大川[おおかわ]さんは
\\	私[わたし]に) 本[ほん]をくれた。	
\\	父は(私に)カメラを買ってくれた。 道男は私をなぐさめてくれました。	
\\	父[ちち]は
\\	私[わたし]に)カメラを 買[か]ってくれた。 道[どう] 男[おとこ]は 私[わたし]をなぐさめてくれました。	
\\	くれる	
\\	道男は私に私をなぐさめてくれた 道男は私になぐさめてくれた
\\	田中さんがあしたうちへ/に来る。	
\\	田中[たなか]さんがあしたうちへ/に 来[く]る。	
\\	私はコンピューターが少し分かって来た。 私はいろいろ日本の歴史書を読んで来ました。	
\\	私[わたし]はコンピューターが 少[すこ]し 分[わ]かって 来[き]た。 私[わたし]はいろいろ 日本[にっぽん]の 歴史[れきし] 書[しょ]を 読[よ]んで 来[き]ました。	
\\	来る	
\\	木村君はまだ昼ご飯を食べている。 太田さんはまだそのことを知りません。	
\\	木村[きむら] 君[くん]はまだ 昼[ひる]ご 飯[はん]を 食[た]べている。 太田[おおた]さんはまだそのことを 知[し]りません。	
\\	きのうは三時から五時まで友達とテニスをした。 私が行くまでうちで待っていてください。	
\\	きのうは三 時[じ]から五 時[じ]まで 友達[ともだち]とテニスをした。 私[わたし]が 行[い]くまでうちで 待[ま]っていてください。	
\\	私は十時までに帰る。 学校が始まるまでにこの本を読んでおいてください。	
\\	私[わたし]は十 時[じ]までに 帰[かえ]る。 学校[がっこう]が 始[はじ]まるまでにこの 本[ほん]を 読[よ]んでおいてください。	
\\	までに 
\\	までに	
\\	ジャクソンさんは日本へいく前に日本語を勉強した。 旅行の前に風邪を引きました。	
\\	ジャクソンさんは 日本[にっぽん]へいく 前[まえ]に 日本語[にほんご]を 勉強[べんきょう]した。 旅行[りょこう]の 前[まえ]に 風邪[かぜ]を 引[ひ]きました。	
\\	前に 
\\	の前に	
\\	雨が降るまえに帰りましょう。 雨が降らないうちに帰りましょう。
\\	私は毎日一マイル泳ぐ。	
\\	私[わたし]は 毎日[まいにち]一マイル 泳[およ]ぐ。	
\\	弟はテレビをつけたまま寝てしまった。	
\\	弟[おとうと]はテレビをつけたまま 寝[ね]てしまった。	
\\	まま (電気を)つけたまま 
\\	まま 大きいまま 
\\	な まま 不便なまま 
\\	のまま 昔のまま 
\\	ドアを閉めないままでかけてしまった。 
\\	私が彼に話しましょう。 映画に行きましょう。	
\\	私[わたし]が 彼[かれ]に 話[はな]しましょう。 映画[えいが]に 行[い]きましょう。	
\\	私(に)は遠くのものがよく見える。	
\\	私[わたし](に)は 遠[とお]くのものがよく 見[み]える。	
\\	お母さん、お客さんが見いえたよ。 
\\	私は日本の小説を読んで見る。	
\\	私[わたし]は 日本[にっぽん]の 小説[しょうせつ]を 読[よ]んで 見[み]る。	
\\	見る	
\\	グレーさんはこんな難しい漢字も読める。 私は何も食べませんでした。	
\\	グレーさんはこんな 難[むずか]しい 漢字[かんじ]も 読[よ]める。 私[わたし]は 何[なに]も 食[た]べませんでした。	
\\	私はもう昼ご飯を食べた。 ヒルさんはもう日本語を勉強していません。	
\\	私[わたし]はもう 昼[ひる]ご 飯[はん]を 食[た]べた。 ヒルさんはもう 日本語[にほんご]を 勉強[べんきょう]していません。	
\\	田中さんも中山さんもデパートに勤めている。 京都へはバスでも電車でも行けます。 このうちは広くも狭くもない。	
\\	田中[たなか]さんも 中山[なかやま]さんもデパートに 勤[つと]めている。 京都[きょうと]へはバスでも 電車[でんしゃ]でも 行[い]けます。 このうちは 広[ひろ]くも 狭[せま]くもない。	
\\	も 
\\	も 
\\	くも  
\\	くも 
\\	でも 
\\	でも	
\\	どうして行かないの?だって、忙しいもの。 よくあんな男とデート出来るものです!	
\\	どうして 行[い]かないの?だって、 忙[いそが]しいもの。 よくあんな 男[おとこ]とデート 出来[でき]るものです!	
\\	(私は)山本さんに本をもらった。	
\\	私[わたし]は) 山本[やまもと]さんに 本[ほん]をもらった。	
\\	(私は)父にカメラを買ってもらった。	
\\	私[わたし]は) 父[ちち]にカメラを 買[か]ってもらった。	
\\	もらう	
\\	タバコを吸うな!	
\\	タバコを 吸[す]うな!	
\\	な	
\\	日本の食べ物の中ではすしやてんぷらなどが好きだ。	
\\	日本[にっぽん]の 食べ物[たべもの]の 中[なか]ではすしやてんぷらなどがすきだ。	
\\	田口さんはいつも新聞を読みながら、朝ご飯を食べる。	
\\	田口[たぐち]さんはいつも 新聞[しんぶん]を 読[よ]みながら、 朝[あさ]ご 飯[はん]を 食[た]べる。	
\\	ながら	
\\	ナンシーはきのう朝ご飯を食べないで学校へ行った。	
\\	ナンシーはきのう 朝[あさ]ご 飯[はん]を 食[た]べないで 学校[がっこう]へ 行[い]った。	
\\	ないで	
\\	今日は働かなくてもいいんですか?いいえ、働かなければなりません。	
\\	今日[きょう]は 働[はたら]かなくてもいいんですか?いいえ、 働[はたら]かなければなりません。	
\\	なければならない 
\\	く なければならない 
\\	で なければならない 
\\	で なければならない	
\\	やさしい漢字も書けなくなった。	
\\	やさしい 漢字[かんじ]も 書[か]けなくなった。	
\\	なくなる 
\\	く なくなる 
\\	では/じゃ なくなる 
\\	では/じゃ なくなる	
\\	朝七時に起きられなくて会社に遅れた。	
\\	朝[あさ]七 時[じ]に 起[お]きられなくて 会社[かいしゃ]に 遅[おく]れた。	
\\	なくて 
\\	く なくて 
\\	では/じゃ なくて 
\\	では/じゃ なくて	
\\	松田が来る(の)なら僕は行かない。	
\\	松田[まつだ]が 来[く]る(の)なら 僕[ぼく]は 行[い]かない。	
\\	もっとたくさん食べなさい。	
\\	もっとたくさん 食[た]べなさい。	
\\	なさい	
\\	私は毎朝六時半に起きる。	
\\	私[わたし]は 毎朝[まいあさ]六 時間[じかん]に 起[お]きる。	
\\	私は母に手紙をよく書く。	
\\	私[わたし]は 母[はは]に 手紙[てがみ]をよく 書[か]く。	
\\	一男は友達に手紙を読まれた。	
\\	一 男[なん]は 友達[ともだち]に 手紙[てがみ]を 読[よ]まれた。	
\\	子供が紙に絵を書いた。	
\\	子供[こども]が 紙[かみ]に 絵[え]を 書[か]いた。	
\\	私はデパートへ贈り物を買いに行った。	
\\	私[わたし]はデパートへ 贈り物[おくりもの]を 買[か]いに 行[い]った。	
\\	に	
\\	ヒルさんは今ジョンソンさんのアパートにいる。	
\\	ヒルさんは 今[いま]ジョンソンさんのアパートにいる。	
\\	私はきのうサンフランシスコに行った。	
\\	私[わたし]はきのうサンフランシスコに 行[い]った。	
\\	あの人は日本人にちがいない。	
\\	あの 人[ひと]は 日本人[にっぽんじん]にちがいない。	
\\	この本は大変読みにくい。	
\\	この 本[ほん]は 大変[たいへん] 読[よ]みにくい。	
\\	にくい	
\\	高山さんは日本人にしては大きい。 ボブは日本語をよく勉強しているにしては下手だ。	
\\	高山[たかやま]さんは 日本人[にっぽんじん]にしては 大[おお]きい。 ボブは 日本語[にほんご]をよく 勉強[べんきょう]しているにしては 下手[へた]だ。	
\\	私はこのアパートにする。	
\\	私[わたし]はこのアパートにする。	
\\	これは先生の本だ。	
\\	これは 先生[せんせい]の 本[ほん]だ。	
\\	私は大きいのを買った。	
\\	私[わたし]は 大[おお]きいのを 買[か]った。	
\\	の 
\\	なの 
\\	の	
\\	日本語を教えるのは難しい。	
\\	日本語[にほんご]を 教[おし]えるのは 難[むずか]しい。	
\\	あの人はとてもやさしいの。	
\\	あの 人[ひと]はとてもやさしいの。	
\\	何をしているん/のですか。日本語を勉強しているん/のです。	
\\	何[なに]をしているん/のですか。 日本語[にほんご]を 勉強[べんきょう]しているん/のです。	
\\	お酒をたくさん飲んだので、眠くなった。	
\\	お 酒[さけ]をたくさん 飲[の]んだので、 眠[ねむ]くなった。	
\\	毎日漢字を勉強しているのに、よく覚えられない。	
\\	毎日[まいにち] 漢字[かんじ]を 勉強[べんきょう]しているのに、よく 覚[おぼ]えられない。	
\\	のに 
\\	なのに/だったのに	
\\	私は日本語の新聞を読むのに辞書を使う。	
\\	私[わたし]は 日本語[にほんご]の 新聞[しんぶん]を 読[よ]むのに 辞書[じしょ]を 使[つか]う。	
\\	のに	
\\	私が中国に行ったのは三年前だ。 このクラスで一番頭がいいのは吉田さんです。	
\\	私[わたし]が 中国[ちゅうごく]に 行[い]ったのは三 年[ねん] 前[まえ]だ。 このクラスで一 番頭[ばんがしら]がいいのは 吉田[よしだ]さんです。	
\\	のは 
\\	だ 
\\	な/だった のは 
\\	だ	
\\	私は五番街を歩いた。	
\\	私[わたし]は五 番[ばん] 街[がい]を 歩[ある]いた。	
\\	私は朝七時半に家を出る。	
\\	私[わたし]は 朝[あさ]七 時半[じはん]に 家[いえ]を 出[で]る。	
\\	次郎は父の死を悲しんだ。	
\\	次郎[じろう]は 父[ちち]の 死[し]を 悲[かな]しんだ。	
\\	日本(に)は大学が多い。	
\\	日本[にっぽん](に)は 大学[だいがく]が 多[おお]い。	
\\	明日パーチーをするのでビールを買っておいた。	
\\	明日[あした]パーチーをするのでビールを 買[か]っておいた。	
\\	おく	
\\	見ておく 
\\	見とく 読んでおく 
\\	読んどく
\\	田中先生はもうお帰りになった。	
\\	田中[たなか] 先生[せんせい]はもうお 帰[かえ]りになった。	
\\	お 
\\	になる	
\\	私は先生のスーツケースをお持ちした。	
\\	私[わたし]は 先生[せんせい]のスーツケースをお 持[も]ちした。	
\\	お
\\	する	
\\	源氏物語をやっと読み終わった。	
\\	源氏物語[げんじものがたり]をやっと 読[よ]み 終[お]わった。	
\\	終わる	
\\	私は弟にケーキを食べられた。	
\\	私[わたし]は 弟[おとうと]にケーキを 食[た]べられた。	
\\	れる 「話される」 
\\	られる 「食べられる」 
\\	来られる、される	
\\	私は日本語が読める。	
\\	私[わたし]は 日本語[にほんご]が 読[よ]める。	
\\	る 「話せる」 
\\	られる 「食べられる」 
\\	来られる、出来る	
\\	松田さんはアメリカへ行くらしい(です)。	
\\	松田[まつだ]さんはアメリカへ 行[い]くらしい(です)。	
\\	田中さんが食べたステーキは高かった。	
\\	田中[たなか]さんが 食[た]べたステーキは 高[たか]かった。	
\\	このうちの広さはちょうどいい。	
\\	このうちの 広[ひろ]さはちょうどいい。	
\\	さ 高さ 
\\	静かさ 
\\	神様を信じますか。	
\\	神様[かみさま]を 信[しん]じますか。	
\\	鈴木さんはむすめを/に大学へ行かせた。
\\	父は妹にピアノを習わせました。
\\	鈴木[すずき]さんはむすめを/に 大学[だいがく]へ 行[い]かせた。 父[ちち]は 妹[いもうと]にピアノを 習[なら]わせました。	
\\	せる 「話させる」 
\\	させる 「食べさせる」 
\\	来させる、させる	
\\	せっかく会いに行ったのに友達はいなかった。 せっかくの日曜日なのに働きました。	
\\	せっかく 会[あ]いに 行[い]ったのに 友達[ともだち]はいなかった。 せっかくの 日曜日[にちようび]なのに 働[はたら]きました。	
\\	今日はテニスもしたし、映画も見た。	
\\	今日[きょう]はテニスもしたし、 映画[えいが]も 見[み]た。	
\\	静かだ/だった し 先生だ/だった し	
\\	悲しい、難しい、さびしい、苦しい。。。	
\\	悲[かな]しい、 難[むずか]しい、さびしい、 苦[くる]しい。。。	
\\	赤い、低い、高い。。。
\\	戸田さんしかタバコを吸わない。	
\\	戸田[とだ]さんしかタバコを 吸[す]わない。	
\\	しか 
\\	しか 
\\	「少ししか」 
\\	池田君は三日でその本を読んでしまった。 私はルームメートのミルクを飲んでしまいました。	
\\	池田[いけだ] 君[くん]は三 日[にち]でその 本[ほん]を 読[よ]んでしまった。 私[わたし]はルームメートのミルクを 飲[の]んでしまいました。	
\\	しまう	
\\	日本のことを知っていますか。 はい、知っています/いえ、知りません。	
\\	日本[にっぽん]のことを 知[し]っていますか。 はい、 知[し]っています。	
\\	山川さんはフランス語を勉強しているそうだ。	
\\	山川[やまかわ]さんは フランス語[ふらんすご]を 勉強[べんきょう]しているそうだ。	
\\	そうだ 
\\	だ/だった そうだ	
\\	雨が降りそうだ。 あの車は高そうだ。	
\\	雨[あめ]が 降[お]りそうだ。 あの 車[くるま]は 高[たか]そうだ。	
\\	そうだ 
\\	そうだ	
\\	きのうはかぜをひきました。それで学校を休んだんです。	
\\	きのうはかぜをひきました。それで 学校[がっこう]を 休[やす]んだんです。	
\\	このオレンジは甘くないです。それではこれはどうですか。	
\\	このオレンジは 甘[あま]くないです。それではこれはどうですか。	
\\	きのうは二時間ぐらい友達と飲んでそれから家に帰った。 きのうは二時間友達と飲んだ。それから本屋に寄ってうちに帰った。	
\\	きのうは二 時間[じかん]ぐらい 友達[ともだち]と 飲[の]んでそれから 家[いえ]に 帰[かえ]った。 きのうは二 時間[じかん] 友達[ともだち]と 飲[の]んだ。それから 本屋[ほんや]に 寄[よ]ってうちに 帰[かえ]った。	
\\	それから 
\\	く(て)、それから 
\\	で、それから 
\\	(と) 
\\	(と) それから 
\\	頭が痛いんです。それならすぐ寝なさい。	
\\	頭[あたま]が 痛[いた]いんです。それならすぐ 寝[ね]なさい。	
\\	刺身を食べますか。それともすきやきにしますか。	
\\	刺身[さしみ]を 食[た]べますか。それともすきやきにしますか。	
\\	きょうは東京にいった。そして友達に会った。	
\\	きょうは 東京[とうきょう]にいった。そして 友達[ともだち]に 会[あ]った。	
\\	ウイルソンさんは肉を食べすぎる。	
\\	ウイルソンさんは 肉[にく]を 食[た]べすぎる。	
\\	すぎる 高すぎる 静かすぎる	
\\	食べなさすぎる
\\	私はステーキが好きだ。	
\\	私[わたし]はステーキが 好[す]きだ。	
\\	この町はいいレストランがすくない。	
\\	この 町[まち]はいいレストランがすくない。	
\\	木がすくない町に住みたくありません。 
\\	中山さんはテニスをする。	
\\	中山[なかやま]さんはテニスをする。	
\\	洋子は長い足をしている。	
\\	洋子[ようこ]は 長[なが]い 足[あし]をしている。	
\\	子供たちの声がした。	
\\	子供[こども]たちの 声[こえ]がした。	
\\	この時計は十万円する。 あと一年したら大学を出ます。	
\\	この 時計[とけい]は十 万[まん] 円[えん]する。 あと一 年[ねん]したら 大学[だいがく]を 出[で]ます。	
\\	ジョギングを始めました。するとご飯がおいしくなりました。	
\\	ジョギングを 始[はじ]めました。するとご 飯[はん]がおいしくなりました。	
\\	私達、山田さん達、子供達、	
\\	私[わたし] 達[たち]、 山田[やまだ]さん 達[たち]、 子供[こども] 達[たち]、	
\\	私は日本へ行きたい。	
\\	私[わたし]は 日本[にっぽん]へ 行[い]きたい。	
\\	たい	
\\	今日は暑くてたまらない。 数学が嫌いでたまりません。	
\\	今日[きょう]は 暑[あつ]くてたまらない。 数学[すうがく]が 嫌[きら]いでたまりません。	
\\	くてたまらない 
\\	でたまらない	
\\	学生は試験のために勉強する。 私は日本のことを知るため(に)日本へ行きます。	
\\	学生[がくせい]は 試験[しけん]のために 勉強[べんきょう]する。 私[わたし]は 日本[にっぽん]のことを 知[し]るため(に) 日本[にっぽん]へ 行[い]きます。	
\\	ために 
\\	な/だった ために 
\\	の/だった ために この/そのために 
\\	山田さんがきたら私は帰る。	
\\	山田[やまだ]さんがきたら 私[わたし]は 帰[かえ]る。	
\\	ら 静だったら 先生だったら	
\\	もっと日本の本を読んだらどうですか。	
\\	もっと 日本[にっぽん]の 本[ほん]を 読[よ]んだらどうですか。	
\\	私達は歌ったり踊ったりした。	
\\	私[わたし] 達[たち]は 歌[うた]ったり 踊[おど]ったりした。	
\\	り (する)	
\\	あの人には話したって分からない。 僕はその切符を高くたって買います。	
\\	あの 人[ひと]には 話[はな]したって 分[わ]からない。 僕[ぼく]はその 切符[きっぷ]を 高[たか]くたって 買[か]います。	
\\	って 
\\	くたって 
\\	だって/だったって	
\\	ジムは日本へ行って勉強して。	
\\	ジムは 日本[にっぽん]へ 行[い]って 勉強[べんきょう]して。	
\\	私は雨が降っても行く。	
\\	私[わたし]は 雨[あめ]が 降[ふ]っても 行[い]く。	
\\	も 
\\	でも	
\\	ここでたばこを吸ってもいいですか。はい、いいです。 高くてもいいです。	
\\	ここでたばこを 吸[す]ってもいいですか。はい、いいです。 高[たか]くてもいいです。	
\\	マイクとディック(と)は学生だ。	
\\	マイクとディック(と)は 学生[がくせい]だ。	
\\	わたしはアンディと一緒にパーティー行った。	
\\	わたしはアンディと 一緒[いっしょ]にパーティー 行[い]った。	
\\	ヒルさんは私に日本語で「こんにちは。」と行った。	
\\	ヒルさんは 私[わたし]に 日本語[にほんご]で「こんにちは。」と 行[おこな]った。	
\\	ニューヨークに行くとおもしろい店がたくさんある。	
\\	ニューヨークに 行[い]くとおもしろい 店[みせ]がたくさんある。	
\\	静かだと	
\\	この本は田中さんに借りたんです。田中さんと言えばもう病気は治ったのかしら。	
\\	この 本[ほん]は 田中[たなか]さんに 借[か]りたんです。 田中[たなか]さんと 言[い]えばもう 病気[びょうき]は 治[なお]ったのかしら。	
\\	「雪国」という小説を読んだ。	
\\	雪国[ゆきぐに]」という 小説[しょうせつ]を 読[よ]んだ。	
\\	私はバッハとかモーツアルトとかベートーベンが好きだ。 疲れた時はお風呂に入るとか早く寝るとかしなさい。	
\\	私[わたし]はバッハとかモーツアルトとかベートーベンが 好[す]きだ。 疲[つか]れた 時[とき]はお 風呂[ふろ]に 入[はい]るとか 早[はや]く 寝[ね]るとかしなさい。	
\\	私は日本にいた時(に)お茶を習った。	
\\	私[わたし]は 日本[にっぽん]にいた 時[とき](に)お 茶[ちゃ]を 習[なら]った。	
\\	私はうちは駅から歩いて十分のところだ。	
\\	私[わたし]はうちは 駅[えき]から 歩[ある]いて 十分[じゅうぶん]のところだ。	
\\	春江は晩ご飯を食べるところだ。	
\\	春江[はるえ]は 晩[ばん]ご 飯[はん]を 食[た]べるところだ。	
\\	田中さんはセールスマンとして採用された。	
\\	田中[たなか]さんはセールスマンとして 採用[さいよう]された。	
\\	このステーキは日本のステーキとしては安い。	
\\	このステーキは 日本[にっぽん]のステーキとしては 安[やす]い。	
\\	としては	
\\	私は来年結婚するつもりだ。 父はまだ若いつもりです。	
\\	私[わたし]は 来年[らいねん] 結婚[けっこん]するつもりだ。 父[ちち]はまだ 若[わか]いつもりです。	
\\	つもりだ 
\\	な つもりだ 
\\	の つもりだ	
\\	アメリカ人ってフットボールが好きだね。 漢字を覚えるって大変ですね。	
\\	アメリカ 人[じん]ってフットボールが 好[す]きだね。 漢字[かんじ]を 覚[おぼ]えるって 大変[たいへん]ですね。	
\\	ジェーンは踊らないって。	
\\	ジェーンは 踊[おど]らないって。	
\\	前田さんはアメリカにいるうちに、英語が上手になった。	
\\	前田[まえだ]さんはアメリカにいるうちに、 英語[えいご]が 上手[じょうず]になった。	
\\	なうちに 
\\	のうちに	
\\	私は学生だ。 過ぎたさんは行きますが、私はいけません。	
\\	私[わたし]は 学生[がくせい]だ。 過[す]ぎたさんは 行[い]きますが、 私[わたし]はいけません。	
\\	私はもあしたのパーティーに行くわ。	
\\	私[わたし]はもあしたのパーティーに 行[い]くわ。	
\\	これは本です。	
\\	これは 本[ほん]です。	
\\	私は英語が分かる。	
\\	私[わたし]は 英語[えいご]が 分[わ]かる。	
\\	この部屋に入っては行けない。	
\\	この 部屋[へや]に 入[はい]っては 行[い]けない。	
\\	は行けない	
\\	私は(に)はこの言葉が分からない。	
\\	私[わたし]は(に)はこの 言葉[ことば]が 分[わ]からない。	
\\	あした試験ですか。じゃあ今晩忙しいわけですね。	
\\	あした 試験[しけん]ですか。じゃあ 今晩[こんばん] 忙[いそが]しいわけですね。	
\\	わけだ 
\\	な/だった わけだ 
\\	という/だった わけだ	
\\	花屋、本屋、薬や	
\\	花屋[はなや]、 本屋[ほんや]、 薬[くすり]や	
\\	山田さんや大川さんが来た。	
\\	山田[やまだ]さんや 大川[おおかわ]さんが 来[き]た。	
\\	ベイリーさんはやはり来なかったね。	
\\	ベイリーさんはやはり 来[こ]なかったね。	
\\	この漢字は覚え安い。	
\\	この 漢字[かんじ]は 覚[おぼ]え 安[やす]い。	
\\	安い	
\\	坂本さんはたばこを吸わないよ。	
\\	坂本[さかもと]さんはたばこを 吸[す]わないよ。	
\\	杉山さんはアメリカへ行くようだ。	
\\	杉山[すぎやま]さんはアメリカへ 行[い]くようだ。	
\\	ようだ 
\\	な/だった ようだ 
\\	の/だった ようだ 
\\	ようだ	
\\	みたいだ 
\\	/だった みたいだ
\\	読めるように字をきれいに書いてください。	
\\	読[よ]めるように 字[じ]をきれいに 書[か]いてください。	
\\	ように	
\\	私は言うように書いてください。	
\\	私[わたし]は 言[い]うように 書[か]いてください。	
\\	ように 
\\	な/だった 
\\	の/だった ように	
\\	先生は学生に宿題をするように言った。	
\\	先生[せんせい]は 学生[がくせい]に 宿題[しゅくだい]をするように 言[い]った。	
\\	ジャクソンさんは日本語が話せるようになった。	
\\	ジャクソンさんは 日本語[にほんご]が 話[はな]せるようになった。	
\\	私は毎日運動するようにする。	
\\	私[わたし]は 毎日[まいにち] 運動[うんどう]するようにする。	
\\	日本語はスペイン語より(も)おもしろい。	
\\	日本語[にほんご]はスペイン 語[ご]より(も)おもしろい。	
\\	メキシコは赤道より北にある。	
\\	メキシコは 赤道[せきどう]より 北[きた]にある。	
\\	私は日本歴史を読もうと思う。 私はもう酒を飲むまいと思います。	
\\	私[わたし]は 日本[にっぽん] 歴史[れきし]を 読[よ]もうと 思[おも]う。 私[わたし]はもう 酒[さけ]を 飲[の]むまいと 思[おも]います。	
\\	と思う 
\\	まいと思う	
\\	私は漢字を毎日五つずつ覚える。	
\\	私[わたし]は 漢字[かんじ]を 毎日[まいにち] 五[いつ]つずつ 覚[おぼ]える。	
\end{CJK}
\end{document}