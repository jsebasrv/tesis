\documentclass[8pt]{extreport} 
\usepackage{hyperref}
\usepackage{CJKutf8}
\begin{document}
\begin{CJK}{UTF8}{min}
\\	耳が痛い	みみがいたい	
\\	耳が遠い	みみがとおい	
\\	耳が早い	みみがはやい	
\\	耳に入る	みみにはいる	
\\	聞き耳を立てる	ききみみをたてる	
\\	耳を疑う	みみをうたがう	
\\	耳にタコができる	みみにタコができる	
\\	初耳	はつみみ	
\\	鼻が高い	はながたかい	
\\	鼻にかける	はなにかける	
\\	鼻につく	はなにつく	
\\	鼻が利く	はながきく	
\\	鼻が曲がる	はながまがる	
\\	鼻を明かす	はなをあかす	
\\	口がうまい	くちがうまい	
\\	口が重い	くちがおもい	
\\	口が軽い	くちがかるい	
\\	口が堅い	くちがかたい	
\\	口が悪い	くちがわるい	
\\	口が裂けても言えない	くちがさけてもいえない	
\\	口に合う	くちにあう	
\\	口をきく	くちをきく	
\\	口を出す	くちをだす	
\\	口をはさむ	くちをはさむ	
\\	口が肥える	くちがこえる	
\\	口を合わせる	くちをあわせる	
\\	働き口	はたらきぐち	
\\	頭が上がらない	あたまがあがらない	
\\	頭が固い	あたまがかたい	
\\	頭が痛い	あたまがいたい	
\\	頭をしぼる	あたまをしぼる	
\\	頭を冷やす	あたまをひやす	
\\	頭をたてに振る	あたまをたてにふる	
\\	頭をよこに振る	あたまをよこにふる	
\\	頭が重い	あたまがおもい	
\\	頭が切れる	あたまがきれる	
\\	頭が高い	あたまがたかい	
\\	頭が低い	あたまがひくい	
\\	頭が古い	あたまがふるい	
\\	頭打ち	あたまうち	
\\	頭割り	あたまわり	
\\	見る目がある	みるめがある	
\\	折り目正しい	おりめただしい	
\\	大目に見る	おおめにみる	
\\	長い目で見る	ながいめでみる	
\\	痛い目にあう	いたいめにあう	
\\	目が覚める	めがさめる	
\\	目が高い	めがたかい	
\\	目が届く	めがとどく	
\\	目が無い	めがない	
\\	目が利く	めがきく	
\\	目が離せない	めがはなせない	
\\	目にする	めにする	
\\	目を覚ます	めをさます	
\\	目を盗む	めをぬすむ	
\\	目を引く	めをひく	
\end{CJK}
\end{document}