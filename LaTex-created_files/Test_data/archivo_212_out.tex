\documentclass[8pt]{extreport} 
\usepackage{hyperref}
\usepackage{CJKutf8}
\begin{document}
\begin{CJK}{UTF8}{min}
\\	驛傳馬車
\\	アーヴィング
\\	高垣松雄訳
\\	【テキスト中に現れる記号について】
\\	ルビ
\\	(例)端下《はした》
\\	入力者注 主に外字の説明や、傍点の位置の指定
\\	(数字は、
\\	0213の面区点番号または
\\	、底本のページと行数)
\\	(例)※[#「插」でつくりの縦棒が下に突き抜けている、第4水準2-13-28]
\\	[#ここから2字下げ]
\\	いざ、これより樂しまむ、
\\	仕置を受くる憂なく、
\\	遊びたのしむ時ぞ來ぬ、
\\	時ぞ來ぬれば、いちはやく、
\\	讀本などは投げ捨てて行く。
\\	[#ここで字下げ終わり]
\\	[#地から2字上げ]――學校休暇の歌
\\	前章で述べたのは、イギリスに於けるクリスマス祝祭に就ての幾つかの一般的な觀察であつたが、今わたしは誘惑を感ずるままに、その具體的な例證として田舍で過したクリスマスの逸話を記してみたいと思ふ。讀者が之を讀まれる際に、わたしから辭を低くして切に願ふのは、いかめしい叡知はしばらく忘れて純一な休日氣分にひたり、愚かしきことをも寛き心を以て許し、ひたすら愉樂をのみ求められんことである。
\\	十二月のこと、ヨークシャを旅行の途上、長い道程をわたしは驛傳馬車の御厄介になつたが、それはクリスマスの前日であつた。馬車は内も外も乘客が混みあつてゐた。その語りあふところから見ると、行先は主に親戚友人の家でクリスマスの御馳走になりに行くのらしかつた。馬車に積込まれたものとしては、また狩獵の獲物の入つた大籃や、珍味を詰めた箱などもあつた。野兎が長い耳をぶらぶらさせて馭者臺の周圍に吊されてゐた、遠方の友人からの贈物で、差迫つた饗宴の用に立てるのであらう。わたしは三人の美しい薔薇色の頬をした少年と一緒に、車内に乘つて行つた。少年たちの顏に溢れるはちきれさうな健康と、男らしい氣魄とは、わたしが今迄にも此の國の子供達のうちに見て來たものであつた。彼等は休暇で歸省の途上にあつて、いかにも陽氣で、これから澤山樂しいことが待つてゐるのだと勇んでゐるのだつた。聞いてゐるだに面白さうに、この小さな腕白たちはやたらに大きな計畫や、またやれる見込もない素晴しい遊びごとを、この六週間に演じようとしてゐたのである。この休暇と云へば、あの厭な書物と鞭と先生との束縛から解放の時である。彼等が胸を躍らして思ひ描いてゐるのは、家族の人々、さては飼猫や飼犬と顏をあはせる時のことであり、自分達のポケット一杯に詰込んである贈物で小さな妹たちを喜ばせることであつた。併し何よりも待遠しく思つて會ひたがつてゐるのはバンタムであつたらしい。バンタムと云ふのは小馬のことと知れたが、少年たちの話合つてゐるところでは、その優れた性質は名馬ブーシファラス以來どんな馬も及ばないのであつた。あのトロットの具合、あの走る姿、それから例の跳躍ぶり――全國どこの生籬だつてバンタムに飛越せないところはないのだ。
\\	少年たちは特別に馭者から世話されてゐた。機會さへあれば彼等は馭者に向つて何やかやと質問をあびせかけ、そして彼を世界中で一番良い人だと云つた。實際その通りで、わたしの目にも彼の並々ならぬ樣子は映つたので、忙しく世話をやいたり、勿體ぶつた態度が見られた。帽子を少し横つちよに冠り、クリスマスの常盤木の大きなのを外套の釦孔に※[#「插」でつくりの縦棒が下に突き抜けている、第4水準2-13-28]してゐたのである。乘合馬車の馭者といふものは、きまつて氣の利いた、世話の行屆く人物であるのだが、とりわけその特色の見られるのは此のクリスマス季節のことである。贈答品の交換が盛んであるため、頼まれる用事も夥しいのである。で、今ここで、恐らく旅の經驗のない讀者のために一つスケッチを試み、この莫大な數に上る、そして重要な役柄の階級の人について、概略述べて置いても無下に邪魔扱ひはされないであらう。彼等には服裝も、作法も、言葉も、態度も、すべて獨特のものがあつて、それが同じ職業の者の間ではずつと行亙つて居り、苟くもイギリスの驛傳馬車の馭者は何處にゐても他の職業商賣の者とは決して間違へられることがないのである。
\\	馭者はたいがい幅つたい、福々しい顏で、妙に赤い斑點が出來てゐて、まるで血液が精分の強い食物のために皮膚の血管の一つ一つに漲つてでもゐるやうである。身體がぶよぶよ膨れてゐるのは、頻りに麥酒を飮むためだが、その圖體が一段と大きいのは外套の重ね着をしてゐるからで、その中に埋つてゐる恰好はコーリフラワのやう、一番上の外套の裾は踵まで屆いてゐる。冠つてゐる帽子は鍔がひろく山が低い。大きくくるくる頸に卷いた染色のハンカチーフは利いた風に結んで端を胸にたくしこんである。夏ならば大きな花の束を釦孔に※[#「插」でつくりの縦棒が下に突き抜けている、第4水準2-13-28]してゐる。それは大抵は贈物、戀焦れた田舍娘からの贈物に違ひない。チョッキは普通きまつて派手な色の縞模樣で、半ズボンは膝より下までもあつて、脛のまん中どころにも達する乘馬靴と觸れあふくらゐである。
\\	この服裝は實にきちんとしたものである。彼の自慢はそれが極上の布地で拵へてあることで、一見したところ粗大な風體のやうであるが、見別ける段になると、殆どイギリス人の天性と云つていい、あの身だしなみが判る。彼は大手を振つて道を行く。あちこちで村のおかみさんたちから聲をかけられ、信用のおける、頼りになる人だと思はれてゐる。また、眼のぱつちりした娘さんとの間にも了解が成立つてゐるらしい。馬換への場處へ着くと早速に彼は、聊か氣取つて手綱を抛り出し、馬は馬丁に任せてしまふ。彼の務は宿場から宿場へ馬車を驅るだけでいいのだ。馭者臺を降りると、外套のポケットへ兩手を突込む、そして宿の庭先を身體をゆさぶりながら、如何にも王侯氣取で歩きまはる。ここで、大抵は彼の崇拜者の群に取卷かれる。その連中は、馬丁や廐番の小僧たち、靴磨き、その他名のつけられない風來坊、宿屋や酒屋に出沒して使ひ走りをしたり、ありとあらゆる端下《はした》仕事をして、臺所の殘り物、酒場のしたみを頂戴して肥るという徒輩である。この連中はみな、彼を仰いで神の御託宣とあがめ、彼の口癖の文句を大事に覺えこみ、馬や博勞の話題について彼の語る意見を鵜呑みにし、わけても懸命になつて彼の風來物腰を眞似るのである。どんなやくざ者でも、苟くも外套を着てゐれば必ず兩手をポケットに突込み、馭者の歩き方に倣つて身體を横に振つて歩き、馬車屋の通語《スラング》をつかひ、恰も馭者君の卵になりすますのである。
\\	恐らくは、樂しく晴やかな氣持でゐたからであらうか、わたしは誰の顏にも快活な輝きを見つつ旅を續けたのであつた。尤も驛傳馬車といふものは、いつも浮々したところがあるので、自分が威勢よく走ると共に、世の中をもじつとさせては置かないのである。角笛が村の入口で吹鳴されると、あたりは一齊にざはめき立つ。友達の出迎ひに急ぐ者があれば、包や紙の箱を手にして逸早く座席を占めようとする者もある。そして急ぎ周章《あわて》るために、ついて來た人々に別れを告げる暇もないほどである。かかる間にも馭者は小さな頼まれ事が山のやうにあつて、それを一々果さなければならない。時には野兎や雉子を配達したり、時には小さい包や新聞を居酒屋の戸口に抛りこんだり、または、知つてゐるぞと思はせぶりな横目をつかひ、口にもそれを出しながら、半ば赤くなり半ば笑ひかけてゐる女中さんに、田舍の愛人から預つて來た奇妙な形の艶書を手渡したりもする。馬車がガラゴロ村の中を通りすぎる時には誰も彼もが窓のところへ走り寄つて來るので、どちらを向いても血色のいい人々や、くすくす笑ふ娘盛りの顏に出會ふ。あちこちの辻角には村の閑人《ひまじん》や賢人たちの會合が開かれてゐる。彼等がそこに陣取つてどんな重大な目的を達しようとするのかと云ふと、驛傳馬車の通るのを見物することに外ならない。併し、村一番の賢人が集るのは鍛冶屋の仕事場にきまつてゐる。彼等にとつては、馬車の通過が思索の種子になる一大事件なのである。鍛冶屋の主人は馬の脚を膝に載せたままで、車が勢よく走りすぎるのを見ながら、手を休める。鐡砧《かなしき》の周圍の巨人《サイクロプス》たちは打鳴す鐡鎚をしばし止めて、鐡の冷《さ》めるのも構はない。また煤けた顏で鳶色の紙帽を被つた怪物は※[#「韋+備のつくり」、第3水準1-93-84]《ふいご》に風を送つてゐるのだが、一寸その柄に凭りかかつて、喘息病みの器械に長い溜息をつかせ、自分は鍛冶場の黒い煙と硫黄のちらつく光の中から、目を光らして〔戸外を〕眺めるのである。
\\	目の前に迫つた祝祭のために、恐らく、村はいつになく活氣を帶びてゐたのだらう、わたしには誰もみんな美しく見えたのだつた。獵の獲物や家禽、また其の他食卓の珍味が村々の間で取引されてゐた。八百屋、肉屋、果物屋などの店頭はお客で賑つてゐた。家々の主婦たちはぴちぴちと元氣に立働いて、家の中を片附けてゐた。そしてつややかな柊《ひいらぎ》の枝が鮮かな赤い實をつけて、窓々に姿を見せ始めた。こんな風景からわたしが思ひ出したのは昔の著述家の書いたクリスマスの準備の敍述であつた。「今や牡鷄も牝鷄も、七面鳥や鵝鳥や家鴨と一緒に、また牛や羊と共に、みな屠られるのである。何故と云つて、十二日の間、大勢の人々の胃の腑を滿たすには少々の食物では濟まないのである。梅や香料、砂糖や蜂蜜も、パイやスープと並べられる。今こそ曲節《ふし》面白く音樂が奏でられる、若い者は踊つて歌つて身體を温《ぬく》めなければならない、よし老人たちは爐傍に坐りこんでゐようともである。女中は買物を半分だけ忘れて歸つたりする、さうしてクリスマス・イーヴに骨牌を忘れようものなら、も一度出直さざるをえない。喧しく言ひ爭つて、やれ柊人形、やれ常春藤《きづた》人形と、夫婦して自分の方に花を持たせようと言い張りあふ。骰子と骨牌の遊びは給仕頭の懷中を肥らせる。そして若し料理人が頭のはたらく男なら、旨い旨いと指先をしやぶるのである。」
\\	わたしの此の有難い空想を破つたのは乘合の少年旅客の叫聲であつた。少年たちは今まで數哩の間、馬車の窓から外を眺めて、故郷に近づくにつれ、あの木この家と一々見覺えてゐるのを喜んでゐたが、今やどつと一齊に歡呼の聲を擧げたのである。「ジョンがゐる、カルロだ、バンタムだ。」嬉しさで一杯の此の腕白少年たちは手を拍ち鳴した。
\\	歩道の盡きるところに、眞面目くさつた顏の老僕が仕着《しきせ》を着て、少年たちの迎ひに出てゐた。老僕の傍には盛《さかり》をすぎた一匹の獵犬《ポインター》と名だたるバンタム、これは小さな老ぼれの小馬で、もじやもじやの鬣《たてがみ》に長い赤錆色の尾をたらし、睡たげに、温和しく路傍に立つて、やがて自分が忙しい目をすることなどは夢にも思ひ及ばぬ樣子であつた。
\\	わたしは快い氣持になつて、少年たちが懷《なつ》つこさうに、このきちんとした老僕の周圍を跳びはねたり、犬を抱きしめたりするのを見てゐた、犬は體躯をくねらして喜んだ。併しバンタムこそは何よりも嬉しい遊び相手であつた。三人が一時《いちどき》に乘りたいと云ふのだ、で、まづどうやら老僕ジョンの骨折で、皆が代る代る乘ること、そして第一番に上の兄さんが乘ることになつた。
\\	いよいよ彼等は動きだした。一人は小馬に乘り、犬がその前を跳んだり、吠えたりしながら行くと、他の二人はジョンの兩手に喰ひついて、兩方から同時に話しかけ、老僕が口を利けないくらゐ我家のことを訊ねたり、自分たちの學校の話を聞かせたりした。少年たちを見送つてゐると一種の感情が湧いて來て、嬉しいのか悲しいのか自分でも分らなかつた。わたしは昔のことが想出された。嘗てはあの少年達のやうに、わたしも勞苦を知らず悲哀を知らなかつたのだ、そして休暇は地上の幸福の絶頂であつたのだ。わたしたちの馬車は數分間止まつて馬に水を飮ませた後、再び旅をつづけ、ある曲角《まがりかど》をまがると、小ざつぱりとした邸宅の見えるところへ出た。わづかにそれと見分けられるばかりだが、一人の婦人と二人の少女の姿が玄關に見え、わたしがお馴染になつた少年達がバンタム、カルロ、老僕ジョンと一緒に、一列になつて車道を進んで行くのが目についた。わたしは馬車の窓から乘りだすやうにして、この樂しい再會の有樣と[#「有樣と」はママ]見たいと思つたのだが、生憎ひとつの叢林が目路を遮つてしまつた。
\\	晩方になつて驛傳馬車は一つの村に着いたが、わたしは豫め此の村で一夜を過さうと決めてゐたのだつた。宿屋の大きな門口に乘入れると、片側には、威勢よく臺所で火が燃えてゐるのが窓から外へ射してゐた。中に入つてみて、わたしは讚歎した、もうこれで百度目なのだが、あの便利な點、小ざつぱりとしてゐる點、そして屈托のない享樂、さういふ光景がイギリスの宿屋の臺所で見られるのだ。室は廣々として、ぐるりにはよく磨いた銅や錫の器が掛けられ、あちこちにはクリスマスの常盤木が飾つてあつた。ハムやタン、鹽豚の片半身、などが天井からぶら下り、轉串機《くしまはし》は爐邊に絶え間なく音を立て、柱時計は一隅に時を刻んでゐた。よく磨かれた樅材の食卓が臺所の一方に片寄せて置かれ、その上には冷肉の腿や、その他、身になりさうな食物が並べられ、その眞中に泡立つビールの大盃が二つ、張番をするやうに置いてあつた。上品でない旅客はこの身のある御馳走に今にも飛びつかうと待構へてゐたが、他の人達は煙草をふかしたり、ビールを呑み雜談に興じ、爐邊に据ゑてある背の高い樫造りの椅子に腰をかけてゐた。小綺麗に身づくろひした女中達は忙しく行きつ戻りつして、きびきびした女主人の指圖に從つた。それでも時折は一瞬の暇を捉へては、爐の周圍の人々と戲言《ざれごと》を交へ、どつと笑つたりもした。かうした光景の中に申分なく具現されてゐるのが、あのプア・ロビンが考へてゐた仲冬の愉樂であるのだ。――
\\	[#ここから2字下げ]
\\	樹々は今、葉つぱの帽子を脱ぎすてて
\\	白髮の冬にお辭儀をいたす。
\\	きれいな主婦、陽氣な旦那、
\\	ビール罎、擧げる祝杯、
\\	煙草に、それから石炭の火、
\\	これらを缺いてはならぬ此の季節どき。
\\	[#ここで字下げ終わり]
\\	[#地から1字上げ](一六八四年版『プア・ロビンの暦』)
\\	わたしが宿屋に着いて程なく、一臺の郵便馬車が門口へ乘りつけて來た。一人の若い紳士が降りたのであるが、洋燈《ランプ》の光でちらと見た顏に見覺えがあると思つた。わたしが乘りだしてよく確めようとすると、先方と目があつた。間違つてはゐなかつたのだ。それはフランク・ブレイスブリッジであつた。快活で愛想のよい青年で、わたしが嘗て一緒にヨーロッパ大陸を旅行したことのある人物だ。この再會は限りなく樂しく懷しいものであつた。凡そ共に一度でも旅をしたことのある相手の顏といふものは、いつも數限りもなく愉快な光景や、珍妙な冒險や、うまい頓智を囘想させてくれる。かうして囘想を宿屋での短い會談で語り盡すといふことは不可能であつた、そして彼は、わたしが忙しく時間に追はれてゐるわけでなく、唯そちこち見物して歩いてゐるのだと知つて、是非一日でも二日でも彼の父の邸で過してくれるやうにと云ふのであつた。彼はこれから父のところへ行つて休暇を送らうとしてゐるのだし、邸は二三哩離れた場所だとのことであつた。「宿屋でクリスマスの御馳走を一人で食べるよりはまし[#「まし」に傍点]ですよ」と彼が云つた、「それにですね、古風な、心からの歡迎ができると思ふんですよ。」彼の説くことには理があつたし、白状すると、到る處で目につく賑かなお祭の支度は、自分の孤獨をいささか堪へがたく思はせてゐたのだつた。それで、わたしは直ぐさま彼の招きに應じたのである。馬車は玄關前に着けられた、そして數分の後、わたしはブレイスブリッジ邸へ向つて發足した。
\\	底本:「スケッチ・ブック」岩波文庫、岩波書店
\\	1935(昭和10)年9月15日第1刷発行
\\	2010(平成22)年2月23日第31刷発行
\\	入力:雀
\\	校正:小林繁雄
\\	2013年9月3日作成
\\	青空文庫作成ファイル:
\\	このファイルは、インターネットの図書館、青空文庫
\\	で作られました。入力、校正、制作にあたったのは、ボランティアの皆さんです。
\end{CJK}
\end{document}