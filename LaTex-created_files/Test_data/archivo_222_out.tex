\documentclass[8pt]{extreport} 
\usepackage{hyperref}
\usepackage{CJKutf8}
\begin{document}
\begin{CJK}{UTF8}{min}
\\	何を 
\\	お腹がすいた
\\	冷水麺
\\	カルビ
\\	それでは
\\	あのう
\\	あげる
\\	2つ以上の名詞を列挙するときに使う助詞で、パッチムがない場合は
\\	、パッチムがある場合は
\\	を使う。 
\\	はパッチムに関係なく使われ、会話でよく使われる。	
\\	教室に椅子と机があります。
\\	話し手の意志や未来を表す。主語が2、3人称である場合、話し手の推測を表す	
\\	来年、また韓国に来ます
\\	紙
\\	扇風機
\\	兄
\\	姉
\\	手
\\	足
\\	ご飯
\\	汁
\\	ここ
\\	先月
\\	初めて
\\	一人で
\\	食べ物、飲食
\\	しらない、分からない
\\	メニュー
\\	注文する
\\	ところが
\\	お店のおばさん
\\	笑う
\\	それで、だから
\\	カルビタン
\\	時間がある
\\	約束
\\	毎日
\\	普通
\\	ニュース
\\	午前
\\	到着する
\\	お休みになる
\\	退勤する
\\	会議
\\	よく
\\	山
\\	一日
\\	起きる
\\	仕事
\\	時間を言うとき、
\\	'は
\\	'と固有数字で、
\\	は 
\\	と漢数字で言う	
\\	10時55分
\\	「ひとつひとつもれなく、ごぞって」という意味の助詞である。時間を表す名詞に接続する場合は「その時間に一回ずつ」という意味になる。	
\\	毎朝、友たちと運動します。
\\	個
\\	甘い
\\	とても
\\	切手
\\	駐車場
\\	台
\\	ラーメン
\\	牛肉
\\	グラム
\\	-人分
\\	靴下
\\	キムパプ
\\	作る
\\	知る
\\	店
\\	開ける
\\	住む
\\	コンビニ
\\	ボールペン
\\	売る
\\	遊ぶ
\\	領収書
\\	ぶどう
\\	チョコレート
\\	鶏
\\	単位を表す名詞に接続して基準を表す助詞である	
\\	コーヒ1杯で5,000ウォンです。
\\	語幹がパッチムが
\\	で終わる動詞、形容詞は後ろに子音
\\	'が来ると、語幹末の 
\\	が脱落する。	
\\	東大門市場
\\	たくさん
\\	服
\\	トッポッキ
\\	少し
\\	辛い
\\	おいしい
\\	大きい
\\	きれいだ 
\\	涼しい
\\	すいか
\\	楽だ
\\	デザイン
\\	かっこいい
\\	運動靴
\\	背が高い
\\	顔
\\	きれいだ
\\	安い
\\	よい
\\	易い
\\	歌
\\	上手だ
\\	りんご
\\	みかん
\\	冷蔵庫
\\	パーティー
\\	肉
\\	魚
\\	キムチ
\\	サービス
\\	高い
\\	複雑だ
\\	小さい
\\	値段
\\	美味しくない
\\	不便だ
\\	下宿
\\	近い
\\	おばさん
\\	親切だ
\\	一緒に
\\	うるさい
\\	紹介する
\\	声
\\	発音
\\	静かだ
\\	多い
\\	生活
\\	楽しい
\\	語幹に付けて、2つ以上の事実を対等に羅列するときに使う	
\\	その人は素敵で、親切です。
\\	語幹に付けて、対立する2つの事実を述べるときに使う。	
\\	韓国語はむずかしいですが、面白いです
\\	の形で事物の状態を尋ねるときに使う。名詞に接続する場合は、
\\	~'の形を使う。	
\\	ソウルの夏の天気はどうですか?
\\	2つの文を対等に羅列したり、時間的に羅列するときに使う。	
\\	韓国に韓国語を勉強します。そして、アルバイトもします。
\\	今年
\\	3月
\\	勤務する
\\	そうだ
\\	休暇
\\	試験
\\	ページ
\\	薬
\\	週間
\\	-月
\\	出勤
\\	シャワー
\\	化粧
\\	就職する
\\	ヨーロッパ
\\	旅行
\\	準備運動
\\	洗う
\\	高校
\\	卒業する
\\	入学する
\\	留学
\\	時間や場所の始点と終点を表す助詞である。場所を表す場合には
\\	'の代わりに
\\	'をよく使う	
\\	3時から4時まで勉強します
\\	後続の動作が前の動作より時間的に先に起こることを表す。名詞に接続する場合は
\\	'を、動詞に接続する場合は
\\	'を使う。	
\\	二ヶ月前に家を買いました。
\\	テコンドー
\\	ミュージカル
\\	公演
\\	ジョンドン劇場
\\	この
\\	磨く
\\	入る
\\	来週
\\	広場
\\	フェスティバル
\\	足
\\	痛い
\\	とる
\\	文字メッセージ
\\	申し訳ない
\\	大丈夫である
\\	美しい
\\	ほんとに
\\	趣味
\\	料理
\\	上手く、よく
\\	吸う
\\	なぜ
\\	便利である
\\	この前
\\	家族
\\	引っ越しする
\\	初めて
\\	パスタ
\\	海鮮 シーフード
\\	スパゲティー
\\	叙述文や疑問文を否定文にするとき、語幹につけて使う	
\\	最近、忙しくありません
\\	語幹について、理由を表す。叙述文と疑問文に使う	
\\	午後、アルバイトをするので、時間がありません。
\\	好きだ
\\	忙しい
\\	お茶
\\	スーツ
\\	チャプチェ
\\	サムギョプサル
\\	嫌いだ
\\	冬
\\	サッカー
\\	音楽
\\	運転
\\	疲れる
\\	野球
\\	運動場
\\	野球場
\\	シカゴ
\\	チーム
\\	応援する
\\	後続する名詞の名前や種類、所属などを尋ねるときに使う	
\\	昨日どんな映画をみましたか? 「スーパーマン」を見ました。
\\	動詞、形容詞の前に
\\	'をつけて、叙述文や疑問文の否定を表す。「名詞
\\	動詞の場合、「名詞+
\\	の形で使う。	
\\	今日は学校に行きません。
\\	週末
\\	釜山
\\	日曜日
\\	夜
\\	ビール
\\	話す
\\	朝
\\	召し上がる
\\	去年
\\	結婚する
\\	昨日
\\	晩
\\	先週
\\	掃除
\\	新聞
\\	日記
\\	天気
\\	晴れ
\\	公園
\\	そこ
\\	テニス
\\	打つ
\\	食事
\\	ピザ~
\\	午後
\\	面白い
\\	銀行
\\	働く
\\	市役所
\\	駅
\\	近所
\\	スーパー
\\	物件
\\	降りる
\\	ショッピングする
\\	食堂
\\	空港
\\	飛行機
\\	猫
\\	横
\\	花
\\	果物
\\	化粧室
\\	牛乳
\\	ベッド
\\	上
\\	左側
\\	右側
\\	下
\\	かばん
\\	引き出し
\\	中
\\	写真
\\	時を表す名詞の後ろに付く助詞である。	
\\	去年の夏に韓国へきました。
\\	時を尋ねるときに使う	
\\	誕生日はいつですか。
\\	動詞の語幹について過去時制、または行動が完了したことを表す。語幹末の母音によって次のように活用する。敬語で言う時は
\\	である。 
\\	は前に付く名詞にパッチムがある場合は
\\	になり、パッチムがない場合は
\\	になる。
\\	は 
\\	になる。	
\\	先週の土曜日に映画を観ました。
\\	名詞の前に付いて、2つ以上の事物のうち、どちらなのかを尋ねるときに使う。	
\\	どちらがいいですか。
\\	場所を表す名詞に付いて、事物や人が存在する場所であることを表す助詞である。
\\	の次には 
\\	'などが来る。	
\\	オフィスが明洞にあります。
\\	同一の事実や行為を列挙するときに使う助詞である。
\\	が主格助詞
\\	'や目的格助詞
\\	'と共に使われる場合は、
\\	'が省略される。	
\\	冷蔵庫に牛乳があります。ジュースもあります。
\\	番号
\\	番
\\	靴
\\	ミリメートル
\\	事務室
\\	~階
\\	-号
\\	バス
\\	~ヴォン
\\	月
\\	何日
\\	誕生日
\\	~号線
\\	アパート
\\	~幢
\\	数字や数量を尋ねるときに使う。単位を表す名詞の前につく。なお値段を尋ねるときは
\\	ではなく 
\\	を使う。	
\\	今日は何月何日ですか。
\\	アルバイト
\\	旅行社
\\	会社
\\	書店
\\	薬局
\\	家
\\	文房具
\\	ノート
\\	デパート
\\	プレゼント
\\	美容室
\\	髪の毛
\\	切る
\\	飲食店
\\	昼食
\\	お金
\\	引き出す
\\	自転車
\\	タバコ
\\	水泳
\\	換金
\\	今日
\\	映画
\\	劇場
\\	手紙
\\	送る
\\	郵便局
\\	図書館
\\	電子街
\\	場所	場所を表す名詞につく。移動を表す動詞
\\	が後続し、その場所に移動することを意味する。	
\\	どこに行かれますか。
\\	後ろに移動を表す動詞
\\	が来て、移動の意図や目的を表す。動詞の語幹にパッチムがない場合とパッチム
\\	'があれば 
\\	'以外パッチムがあれば
\\	'が使われる。	
\\	プレゼントを買いに行きます。
\\	場所を表す名詞について、その場所である動作や状態が起こることを表す。	
\\	書店で本を買います。
\\	場所を尋ねるときに使う。	
\\	家はどこですか?
\\	いらっしゃる
\\	お客
\\	ちょっと
\\	座る
\\	使う、書く
\\	練習する
\\	最近、この頃
\\	どう
\\	過ごす
\\	夫人
\\	文の主体を敬うとき語幹につけて話す。語幹がパッチムで終わらない時は
\\	'、パッチムで終わるときは
\\	'をつける。	
\\	お父さんが新聞をお読みになります。
\\	命令や忠告をするときに使う表現。動詞の語幹がパッチムで終わらない時は
\\	'を、パッチムで終わるときは
\\	'を使う。否定形は
\\	少々お待ち下さい
\\	主格助詞
\\	'の尊敬形である。	
\\	先生がお話になります。
\\	行ってくる
\\	済州島
\\	おととい
\\	帰ってくる
\\	景色
\\	雨が降る
\\	鑑賞
\\	木曜日
\\	故郷
\\	南側
\\	皆さん
\\	海
\\	多くの
\\	毎年
\\	お祭り
\\	有名である
\\	博物館
\\	する
\\	韓国語
\\	宿題
\\	ある
\\	毎日
\\	微印旛
\\	パン
\\	ジュース
\\	飲む
\\	テレビ
\\	地下鉄
\\	乗る
\\	着る
\\	チマ
\\	待つ
\\	鉛筆
\\	今
\\	時間
\\	教室
\\	休み時間、休憩時間
\\	休む
\\	終わる
\\	授業
\\	始める
\\	-たち
\\	学院
\\	教える
\\	この方
\\	母、お母さん
\\	男
\\	妹、弟
\\	妻
\\	子供
\\	娘
\\	あの方
\\	社長
\\	見る
\\	食べる
\\	買う
\\	会う
\\	勉強する
\\	受ける
\\	読む
\\	電話する
\\	~さん
\\	来る
\\	寝る
\\	運動する
\\	結婚写真
\\	看護師
\\	父
\\	公務員
\\	そして
\\	中学校
\\	英語
\\	教師
\\	両親
\\	祖母
\\	(男性から見て)兄
\\	大学
\\	教授
\\	息子
\\	韓国語
\\	教科書
\\	本
\\	いいえ
\\	コンピューター
\\	黒板
\\	辞書
\\	傘
\\	財布
\\	消しゴム
\\	カレンダ-
\\	門
\\	帽子
\\	机
\\	友達
\\	部屋
\\	私達
\\	コーラ
\\	コーヒー
\\	携帯電話
\\	カメラ
\\	キムチチゲ
\\	スンドゥブチゲ
\\	フランス
\\	学生
\\	電子辞書
\\	エアコン
\\	手帳
\\	おはようございます
\\	日本
\\	人
\\	はい、ええ
\\	お会いできてうれしいです
\\	名前
\\	中国
\\	韓国
\\	インド
\\	アメリカ
\\	フィリピン
\\	会社員
\\	大学生
\\	医者
\\	先生
\\	映画俳優
\\	主婦
\\	行く
\\	台所
\\	外
\\	目
\\	昼
\\	馬
\\	コメ
\\	ソウル
\\	春
\\	絵
\\	口
\\	森
\\	川
\\	あちこち
\\	チャガルチ市場
\\	刺し身
\\	サンドイッチ
\\	中華料理店
\\	あちら
\\	出す
\\	出る
\\	火曜日
\\	水曜日
\end{CJK}
\end{document}