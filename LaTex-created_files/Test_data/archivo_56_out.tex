\documentclass[8pt]{extreport} 
\usepackage{hyperref}
\usepackage{CJKutf8}
\begin{document}
\begin{CJK}{UTF8}{min}
\\	魚	【名词】 
\\	鱼,鱼类,鱼肉(うお。魚類) 
\\	下酒菜(酒をのむ時におかずとしてたべるもの)	うお さかな	
\\	甥	【名词】 侄,侄儿,侄子;甥,外甥(その人の兄弟や姉妹がもうけた男の子)	おい	
\\	青	【名词】 
\\	青,蓝(七色のひとつ) 
\\	绿灯的省略(青信号の略) 〜になったら横断しよう/绿灯亮了才过马路 
\\	马的俗称(青馬の略) 
\\	年轻未成熟的(ある語に冠して「若い」「未熟の」の意を表す語)	あお	
\\	会う	【自动词・五段/一类】 
\\	会面,见面(人と顔を合わせる) 
\\	偶遇,碰见(偶然に出会う)	あう	
\\	赤	【名词】 
\\	红,红色(七色の一つ。血のような色。また、緋色・紅色・朱色・茶色などの総称) 
\\	(酱,狗毛等的)茶色(ソースや犬毛などの) 
\\	赤色分子(社会主義者や共産主義者をさす) 【接头/接尾】 
\\	红…(赤みをおびた) 
\\	完全(名詞の上に付けて、「まったくの」「すっかり」「あきらかな」「まるっきり」の意を表す語)	あか	
\\	秋	【名词】 秋,秋天,秋季(夏が終わり冬になるまでの間の季節)	あき	
\\	奥	【名词】 
\\	不为人知的,不表露在外的(表面に表れない部分。容易にはうかがい知れない部分) 
\\	对女主人的敬称(主として身分の高い人が自分の妻をいう語。また、身分の高い人の妻に対する敬称) 
\\	里头,内部;深处(中へ深く入った所) 
\\	里屋,里院(建物の表口から遠い所。日々、生活している所) 
\\	尽头,末尾,最后(行く末、文書・手紙などの終わりの部分)	おく	
\\	声	【名词】 
\\	声,声音;语声,嗓音(人や動物が発声器官から出す音。音声。話し声) 
\\	声音,声响(物の振動から発する音。ひびき) 
\\	语言,话(ことば) 
\\	想法;意见;呼声(意見。考え) 
\\	迹象(季節・時期などが近づくけはい)	こえ	
\\	池	【名词】 
\\	池,池塘(地面にできたくぼみに水のたまったところ。普通、湖沼より小さいものをいう。特に自然のもの) 
\\	水池,池子(地面を掘って、水をたたえたところ。主に庭園に風趣を添えるためにつくる。人工のものが多い) 
\\	砚池,砚端贮水或墨汁处(硯(すずり)の、水を入れるくぼみの部分。海)	いけ	
\\	蚊	【名词】 蚊(子)(双翅目カ科の昆虫の総称)	か	
\\	駅	【名词】 
\\	驿站(律令制で、公用の旅行や通信のために駅馬・駅船・人夫を常備しているところ) 
\\	车站(列車・電車を停止し旅客・貨物などの取り扱いのために常用される場所)	えき	
\\	食う	【他动词・五段/一类】 
\\	吃(食べる) 
\\	花费 
\\	打败,取胜(勝つ) 
\\	叮咬 
\\	蒙受(身に受ける) 
\\	年岁增长(年をとる) 
\\	生活(生活する) 
\\	受骗(騙される)	くう	
\\	傘	【名】 伞	かさ	
\\	塩	【名词】 
\\	食盐(塩化ナトリウムを主成分とする、塩辛い味のある白色の結晶) 
\\	咸度(塩加減)	しお えん	
\\	酒	【名词】 
\\	清酒(米と麹で醸造した、日本特有のアルコール含有飲料) 
\\	酒的总称(アルコール分を含み、飲むと酔う飲料の総称) 
\\	饮酒(さかもり)	さけ	
\\	足	【名词】 
\\	脚(くるぶしから先) 
\\	腿(あし全体) 
\\	脚步;步行(足どり) 
\\	交通工具(交通手段) 
\\	去,来,来往(行き来) 
\\	进度 【接头/接尾】 双(足などにつける一対のものを数える語)	あし そく	
\\	西瓜	【名词】 西瓜(ウリ科のつる性一年草。熱帯アフリカ原産。暖地の砂地畑で栽培。葉は羽状に深裂。雌雄同株。夏,淡黄色の花をつけ,大きな球形または長球形の液果がなる。果皮は平滑で時に縦の縞(しま)があり,果肉は多汁で甘く,赤ないし淡黄色。品種が多い。日本へは一六世紀渡来。ウオーターメロン)	すいか	
\\	席	【名词】 
\\	席位(座る場所) 
\\	聚会场所(会場) 
\\	坐垫(むしろ) 
\\	地位(地位) 
\\	曲艺场(寄席)	せき	
\\	世界	【名词】 
\\	世界,全球,寰球,天下。地球上的所有的国家、所有的地域(地球上のすべての国家・すべての地域。全人類社会) 
\\	宇宙。全部包含一切事物的无限空间(一切のものを全部含んだ無限の空間。宇宙) 
\\	世上,人间,世间,天地。以自我为中心的生活场所〔世の中。自分を中心とした生活の場。生活圏〕 
\\	世界,社会,环境。同一种类事物的群体或其社会〔ある特定の社会。環境。同一の種類のものの集まり、またその社会〕	せかい	
\\	汗	【名词】 
\\	汗(温度刺激により汗腺から排出される分泌液) ~をかく/出汗 
\\	反潮(物の表面に生じる湿滴) 壁が~をかいている/墙返潮了 
\\	努力(努力) 血と~の結晶/劳动成果	あせ	
\\	下	【名词】 下,下方;下去;下达;在…之下(したの方。位や級のしたの方。順序で、あとの方) 【名词】 连接时间空间的下方;位置低的地方;身份低的一方(空間的・時間的に連続したものの下の方。地位・地置の低い方) 【名词】 
\\	下部,底下(その物をささえている下の方の部分) 
\\	(某人的)身边,跟前,左右,周围(者の辺り、その人の勢力の及ぶ範囲) 
\\	在…下;情况、条件等所涉及的范围(影響が及ぶ範囲) 【名词】 下边,下面;在两个一组或三个一组的事物中处于末尾的一个;下来,弄下来,降下;下等,劣等 【名词】 下,下面;下,下级,级别或地位低下;内,里;低,劣,差;马上,随后 【接头/接尾】 事先,预先;〔地面〕直接接触地面	か しも もと げ した	
\\	杖	【名词】 
\\	手杖,拐杖;棍子。走路时手里拄着作为步行辅助工具的细长木棍或竹棍 
\\	滑雪杖(スキーのストック) 
\\	依靠,靠山。可以依赖的事物(頼りにするもの)	つえ	
\\	鉄	【名词】 
\\	铁(くろがね) 
\\	坚硬如铁的事物(硬いもの、強固なもののたとえ)	てつ	
\\	外	【名词】 
\\	相对于里的外边,外面〔外側、外面、外部〕 
\\	家以外的地方,外面,外头〔家に対して、家庭外〕 
\\	所定范围以外的外面,外部,外人〔部外、他人〕 
\\	相对于内心的表面〔表面〕 
\\	社会,外界〔世の中、世間〕	そと	
\\	机	【名词】 桌子;书桌,书案;办公桌;写字台;案(読書や書き物などをする台)	つくえ	
\\	男	【名词】 
\\	男的,男性;男人,男子(男性。男子) 
\\	雄的,公的(おす) 
\\	男子汉,丈夫气概 
\\	体面;声誉,声价(男ぶり)	おとこ	
\\	月	【名词】 
\\	月亮(地球をめぐる衛星) 
\\	月光(月の光) 
\\	月份(こよみの上での一ヵ月。時間の単位) 
\\	妊娠期(機の熟する期間。特に、妊娠10カッ月目の産み月のこと)	つき	
\\	テキスト	【名词】 
\\	【英】
\\	教科书,教材,课本,讲义(教材) 
\\	【英】
\\	文本,底本(原典) 
\\	【英】
\\	原文(原文)		
\\	犬	【名词】 狗(大昔から人間に飼育されてきた家畜) 【接头/接尾】 
\\	无…白…(卑しめ軽んじて、くだらないもの、むだなものの意を表す) 
\\	伪…似…(似て非なるもの、劣るもの。…まがい)	いぬ	
\\	猫	【名词】 猫(家に飼う小動物)	ねこ	
\\	姉	【名词】 
\\	姐姐,姊;[謙遜して]家姊〔姐〕(同じ親から生まれた年上の女。年上の女の姉妹) 
\\	嫂(子),嫂嫂;大姑子;大姨子(兄の妻。あるいは夫や妻の姉。義姉)	あね	
\\	兄	【名词】 
\\	哥哥(同じ親から生まれた年上の男) 
\\	姐夫。大舅子。义兄(義兄) 
\\	年长的(男の年長者) 
\\	师兄(自分より先に入門した人) 【名词】 
\\	兄长 
\\	在男人之间,加于朋友的姓后表示敬意	あに けい	
\\	貴方	【代词】 
\\	您,你(近世以後、目上や同輩である相手を敬って指す語) 
\\	夫妻间老婆对老公的称呼(夫婦間で妻が夫を呼ぶ語) 
\\	那里,那边(自分や相手から遠い所。あちら) 
\\	以前,从前(基準とする時からむこう。特に、依然。まえかた)	あなた	
\\	砂	【名词】 沙子(細かい岩石の粒の集合) 【名词】 沙石(小さい石)	すな いさご	
\\	西	【名词】 
\\	西,西方(四方のひとつ) 
\\	西风(西風) 
\\	西天。净土(西方浄土)	にし	
\\	布	【名词】 
\\	布。织物的总称。在日本旧时与丝绸相对指用麻、葛等纤维织成的物品。后进一步包括棉花,进而成为亦包括丝绸再被的所有织物的总称 
\\	在建筑中表示水平,横的意思	ぬの きれ	
\\	蟹	【名词】 蟹,螃蟹	かに	
\\	八	【名词】 八,第八个(数の名,七に一を加えた数)	はち	
\\	二つ	【名词】 
\\	两个(数の名。ひとつより1多い数) 
\\	两岁(2歳) 
\\	两方(両方) 
\\	第二,二则(2番目)	ふたつ	
\\	花	【名词】 
\\	花(被子植物の生殖器官で、雌しべ(子房)をもつことが特徴である。広義には種子植物の有性生殖にかかわる器官をいう) 
\\	樱花(桜) 
\\	梅花(梅) 
\\	芥草(しきみ) 
\\	华丽,华美,光彩,精华(うるわしいこと) 
\\	黄金时代,最美好的时期(最盛期) 
\\	插花术,花道(華道) 
\\	(给艺人的)赏钱,(给艺妓的)酬金(祝儀) 
\\	纸牌戏(花札)	はな	
\\	下手	【名・形容动词/ナ形容词】 
\\	笨拙,不高明(物事に巧みでないこと。まずいこと。不器用なこと。また、そういう人) 
\\	不慎重,不小心,马虎(なまじっかなこと。いいかげんなこと。または、そのさま。多く、結果が悪くなることにいう。うっかり) 【名词】 
\\	下边,下面(下の方。しもて。したで) 
\\	棋力较低的一方(他より低い地位。また、その地位にある人。特に、囲碁・将棋などで、段位の低い方の者) 【名词】 
\\	下边;下游(下の方。下座の方。川の下流) 
\\	(从观众方面看)舞台的左边(芝居の舞台の、見物席から見て左の方)	へた したて しもて	
\\	母	【名词】 
\\	母,母亲(母親) 
\\	事物产生的根源(生みだされるもと)	はは	
\\	臍	【名词】 
\\	脐,肚脐 
\\	小坑(ものの表面の中央部にある小さなくぼみやでっぱり) 
\\	中心,中央,关键点(物の中央、中心。また、重要な部分。ポイント) 【名词】 脐,决心,本意
\\	へそ。決心。また、本心)	へそ ほぞ	
\\	頬	【名词】 脸颊,脸蛋	ほお	
\\	松	【名词】 
\\	松,松树,松木(マツ科の針葉樹。特に,アカマツ・クロマツ・ゴヨウマツ・ハイマツなどマツ属の植物をさす) 
\\	新年装饰门前的松枝装饰;松枝的期间(門松。また,門松を飾っておく期間)	まつ	
\\	意味	【名词】 
\\	意思,意义(言葉が示す内容。また、言葉がある物事を示すこと) 
\\	意图,动机,用意,含意
\\	ある表現・行為によって示され、或いはそこに含み隠されている内容) 
\\	意义,价值(価値。重要性)	いみ	
\\	昔	【名词】 
\\	从前,很早以前,古时候,往昔,昔日,过去(過去) 
\\	十年(10年間)	むかし	
\\	米	【名词】 稻米,大米(稲の種子からもみ殻を取り去ったもの) 【名词】 米,大米 88岁,88岁大寿;米寿	こめ よね	
\\	道	【名词】 
\\	道路(人や車など行き来する通路。道路) 
\\	道;道义;道德((仏教や道教などの)教え) 
\\	方法,手段[技巧]手法(方法。手段) 
\\	路程(道のり。距離) 【接头/接尾】 
\\	道 
\\	道路(人の踏みおこなうべき道) 
\\	艺术,学问〔芸や学問〕	みち どう	
\\	耳	【名】 
\\	耳,耳朵 
\\	听觉,听力(聴力) 
\\	(器物的)耳状物 
\\	(东西的)边,缘	みみ	
\\	名前	【名词】 
\\	名,名字(事物の名称。名) 
\\	名字,姓名(氏名。特に苗字に対して、名)	なまえ	
\\	虫	【名词】 
\\	虫,虫子,昆虫(人・獣・鳥・魚・貝以外の小動物。多く、昆虫をいう) 
\\	蛔虫;害虫;蛆(人に害を与える小動物。人の体内にすむ寄生虫やノミ・ミラミ・シミナド) 
\\	抽风;疳积(子供に起こる病気の虫気) 
\\	心。寄宿于体内引起意识感情变化的思想(体内に宿り、さまざまな考えや感情を起こすもとになると考えられているもの) 
\\	怒气,气愤;郁闷(かんしゃく) 
\\	虫。热衷于某一件事的人(一つ事に熱中する人)	むし	
\\	桃	【名词】 〈植〉桃树;桃子(バラ科の落葉小高木)	もも	
\\	部屋	【名词】 
\\	〔室〕房间,屋子,…室,…间(家の内部を壁や建具で仕切った一画。人が起居し、物などを置くための空間。室) 
\\	顾问室,师傅室(相撲部屋の略)	へや	
\\	雪	【名词】 
\\	雪(冬に、空からふってくる、白くてつめたいもの。大気中の水蒸気が急にひえて氷のつぶとなり、それがあつまってふってくる) 
\\	雪白,洁白(白いもののたとえ)	ゆき	
\\	横	【名词】 
\\	横(水平の方向) 
\\	宽度(左右への方向) 
\\	侧面(物の側面) 
\\	旁边(かたわら。わき。左右) 
\\	歪,斜(ななめ)	よこ	
\\	屋根	【名词】 
\\	屋顶。屋脊(雨露などを防ぐために家屋の最上部に設けたおおい) 
\\	(物体上部的)篷,盖儿(物の上部にあって、おおうもの)	やね	
\\	山	【名词】 
\\	山(平地よりきわだって高く盛り上がっている地形のところ) 
\\	堆,一大堆,堆积如山(うずたかく積まれたもの) 
\\	矿山(鉱山) 
\\	高潮,关键,顶点(物事のいちばん大切なところ。物事の頂点。さかんな時) 
\\	碰运气,冒险,押宝(万一の成功を当てにして行う冒険)	やま	
\\	夢	【名词】 
\\	梦(眠っているときに、実際にあってことのようにいろいろの物事を見聞きしたり、感じたりする現象) ~を見る/做梦 
\\	梦想,幻想(はかないことやたよりにならないことのたとえ) 
\\	理想(心の中で、いつかは実現したいと思っているのぞみ)	ゆめ	
\\	冬	【名词】 冬,冬季,冬天(四季の一。秋と春の間の季節。現行の太陽暦では12月から2月まで。陰暦では10月から12月まで。また、二十四節気では立冬から立春の前日まで。天文学上では冬至から春分の前日まで。一年中でもっとも寒い季節)	ふゆ	
\\	嫁	【名词】 
\\	儿媳妇〔むすこの妻〕 
\\	妻,媳妇儿〔結婚相手の女性〕 
\\	新娘〔新婦〕	よめ	
\\	嫌	【形容动词/ナ形容词】 
\\	不愿意,不喜欢,讨厌(とても受け入れる気になれない) 
\\	厌腻,厌烦(それ以上続けたくない気持ち) 
\\	不愉快,不耐烦(不愉快なさま)	いや	厭
\\	読む	【他动词・五段/一类】 
\\	念,读;诵,朗读(文字や文を見て、それを声に出して言う) 
\\	看,阅读(文字、文章などを見て、その意味を理解する) 
\\	体察,忖度,揣摩,理解,看懂(他人の心や将来のことを推測する) 
\\	数(数える) 
\\	计算。围棋、将棋赛中,思考下一步的路数 
\\	训读(汉字)(漢字を訓で表す)	よむ	
\\	桜	【名词】 
\\	〈植〉樱花;樱花树(バラ科サクラ属の落葉高木または低木) 
\\	樱花色,淡红色(桜色) 
\\	马肉(馬肉の俗称) 
\\	托儿(露店などで、客の買い気をそそるため、客のふりをして買い物する仲間)	さくら	
\\	栗	【名词】 
\\	栗子树〔栗の木〕 
\\	栗子〔栗の実〕	くり	
\\	色	【名词】 色,颜色,彩色;色泽,光泽,色彩;肤色;脸色;气色;神色;景象,情景,样子,状态;让步,放宽条件;女色,色情;情人,情夫(妇);种类 【名词】 色,物质(五境の一つ。視覚の対象。色・形のあるもの。色境。五蘊の一つ。物質的存在。変化して壊れ、また空間を占めて他の障害になるもの) 【名词】 …色	いろ しき しょく	
\\	歴史	【名词】 
\\	历史(人類社会の過去における変遷・興亡のありさま) 
\\	来历(物事の現在に至る来歴)	れきし	
\\	裏	【名词】 
\\	背面〔うしろ側) 
\\	后(面),后边〔うしろ〕 
\\	里儿,里子,底子(衣服・袋物などの内側に付ける布) 
\\	内部,里(面);背后;内幕,幕后(表面には現れない隠された内部の事情。内情) 
\\	背面;后面;反面(表面と反対の面。下または陰になって見えない部分) 
\\	内情,隐情(表面には現れない隠された内部の事情。内情) 
\\	后半场(野球で,後攻チームの攻撃するイニング) 
\\	后面(前面・正面の反対側。うしろ)	うら り	
\\	虎	【名词】 
\\	虎,老虎(ネコ科の哺乳類。ライオンと並ぶ大形の猛獣) 
\\	醉汉,醉鬼(酔っぱらい)	とら	
\\	襟	【名词】 
\\	衣服的领子(衣服で身頃の首を取り囲むところに取り付けられている部分の名) 
\\	后颈(首の後部。また、首) 
\\	西装的硬领(布団などの、首のあたる部分にかける布)	えり	
\\	印	【名词】 
\\	记号,标记(他と紛れないように見分けるための心覚えとするもの) 
\\	证据(あることを証明すること) 
\\	信号(合図) 
\\	心意(心持ちを表す) 【名・他动词・サ变/三类】 
\\	印章(木・牙・角・水晶などに文字などを彫刻し、文書・書画に押して証明とするもの) 
\\	印(しるしつけること) 
\\	佛教上为表示领悟或心愿等而交叉手指的形状(両手の指をさまざまに組み合わせて宗教的理念を象徴的に表現すること) 
\\	印度的简称(インドの略)	しるし いん	
\\	袋	【名词】 
\\	袋,口袋;纸袋;布袋;皮袋 
\\	果囊,水果的内皮(みかん、ほおずきなどの果肉を包んでいる薄い皮) 
\\	死胡同	ふくろ	
\\	会話	【名・自动词・サ变/三类】 会话。谈话,对话(複数の人が互いに話すこと。また、その話)	かいわ	
\\	若い	【形容词/イ形容词】 
\\	年轻(年をとってない) 
\\	(年纪)小(年・数が少ない) 
\\	幼稚;未成熟;不够老练(未熟だ) 
\\	血气方刚的,朝气蓬勃的(血気盛んな様子。元気である)	わかい	
\\	我々	【代词】 
\\	我们,咱们,我等(わたくしたち。私たち。われら) 
\\	我(一人称の人代名詞。単数を表す。へりくだる気持ちを含んでいう語。わたくし)	われわれ	
\\	悪い	【形容词/イ形容词】 
\\	坏,不好(よくない) 
\\	恶性;恶劣;有害,不利(劣っている・害がある) 
\\	不对,错误(正しくない) 
\\	不吉利,不吉祥(縁起が悪い、不幸だ) 
\\	不佳,不舒畅;不适合,不方便;坏,腐败(よい感じを与えない。不快である) 
\\	不好意思。对不住。用于口语(すまないという気持ちを表す語) 【形容词/い形容词】 
\\	可憎,可恶,可恨(いやな相手として何か悪いことがあればよいと思うほど嫌っている。気に入らない。にくらしい) 
\\	漂亮;令人钦佩,值得佩服(みごとだ)	わるい にくい	
\\	縄	【名词】 
\\	绳,绳索(植物の繊維や茎をより合わせて細長く作ったもの。普通,綱よりは細く,ひもよりは太いものをいう。材料は多くは稲藁(いなわら)。シュロやアサのものは丈夫で水に強い。物を縛るときやつなぐときなどに用いる) 
\\	被捕;落网(とりなわ。捕縄)	なわ	
\\	案内	【名・他动词・サ变/三类】 
\\	引导,向导;导游,陪同游览(希望するところへ導いて行くこと) 
\\	传达(取り次ぎ) 
\\	通知,通告(事情を知らせること) 
\\	邀请(招待) 
\\	了解,熟悉(事情をよく知っていること)	あんない	
\\	本屋	【名词】 书店,开书店的人(書物を売る店)	ほんや	
\\	安心	【名・自动词・サ变/三类】 放心,无忧无虑(心配・不安がなくて、心が安らぐこと)	あんしん	
\\	日本	【名词】 日本(アジア大陸の東のはしと太平洋との間に横たわる島国。象徴としての天皇をもつ国民主権の国。首都は東京)	にほん にっぽん	
\\	南京	【名词】 
\\	南京(中国、江蘇省の省都。長江下流南岸に位置する。水陸交通の要衝で機械・化学・鉄鋼などの工業が発達。古く金陵・建業・建康などともいわれ、明代に北京に対し南京と称した。1927年以後は国民政府の首都であった) 
\\	南瓜〔かぼちゃ〕 【造语】 
\\	珍奇的,小巧的〔珍しいもの・小さいもの〕 ~玉/有孔玻璃球;有孔小珠 
\\	中国(外国)传来的东西(中国(外国)から渡来した物の称) ~米/从中国进口的大米	なんきん	
\\	記念	【名・他动词・サ变/三类】 
\\	纪念,留念。为今后唤起回忆而把某物保存下来(あとの思い出として残しておくこと。また、その物) 
\\	纪念。举行某种活动以对往事表示怀念(過去の出来事への思いを新たにし、何かをすること)	きねん	
\\	困難	【名/な形容词】 
\\	困难(苦しく難しいこと) 
\\	穷困(経済的な)	こんなん	
\\	右	【名词】 
\\	右,右边,右方(人の体で、心臓のない側) 
\\	上文,前文(縦書きの文章で、いまよりも前に書いたところ) 
\\	胜过的,比…强的(比較してすぐれた方) 
\\	右倾,偏右(保守的なこと。右翼)	みぎ	
\\	劇	【名词】 
\\	剧烈,程度很高(激しく強いこと) 
\\	戏剧,剧,戏,戏曲(たわむれること。芝居)	げき	【比较】“戏”“剧”:有时候用“戏”是指传统剧,而“剧”是指现代剧这种理解来区别这两个说法,但意思几乎是相同的。举例来说,京剧这种情况就两者都可用。“昆曲”“越剧”“秦腔”这一类就因地方的不同而有不同的说法
\\	外科	【名】 外科。对外伤或体内各种疾病通过手术治疗的一个医学分科	げか	
\\	会議	【名・自动词・サ变/三类】 会议,会
\\	関係者が集まり、討論・相談や決議をすること。また、その会合)	かいぎ	
\\	午後	【名词】 午后,下午,下半天,后半天(正午から午前零時まで;特に正午から日没までの時間)	ごご	
\\	科学	【名词】 科学(観察や実験をとおして、事実を順序だてて説明できる一般法則をみつけだし、また、それを応用する学問)	かがく	
\\	学歴	【名词】 学历(その人がどんな学校を出たかという経歴)	がくれき	
\\	疑問	【名词】 疑问(疑わしいこと) ~を抱く/怀疑	ぎもん	
\\	日本語	【名词】 日语(日本の国語)	にほんご	
\\	真面目	【名・形容动词/ナ形容词】 
\\	认真。老实。踏实。严肃(真剣な態度・顔つき。本気) 
\\	诚实,正派,正经(まごころがこもっていること。誠実なこと) 【名词】 
\\	真正的面目。本来形象。真本领。本来的性格(本体そのままのありさま。本来のすがた。転じて、真価) 
\\	认真。正直。耿直(実直)	まじめ しんめんもく	
\\	鈴木	【名词】 铃木。日本人名的一个姓氏	すずき	
\\	地図	【名词】 地图(地球表面の一部または全部を一定の割合で縮小し、記号・文字などを用いて平面上に表した図)	ちず	
\\	風	【名词】 
\\	风(空気の動き。一般に,気圧の高い方から低い方に向かう水平方向の空気の流れをいう) 
\\	风气(人に対する社会全体の態度) 【名词】 
\\	样子,态度;风度(様子。状態。ふり) 
\\	风习,习惯;风格(ある範囲の土地や社会にみられる生活様式。ならわし) 
\\	那样;这样(方式。やり方) 
\\	打扮,外表(趣のあるさま。しゃれているさま)	かぜ ふう	
\\	膝	【名词】 膝(脚の,ももとすねをつなぐ関節の前面。ひざがしら)	ひざ	
\\	支持	【名・他动词・サ变/三类】 
\\	支撑,维持(支えもつこと) 
\\	支持,赞成;拥护(賛成すること)	しじ	
\\	地味	【形容动词/ナ形容词】 
\\	服装、打扮、性格等朴素,不华美(服装や性格がはででないこと。質素なこと。くすんでいること) 
\\	性质、思考方式、生活态度等比较低调、不显眼,质朴,没有什么修饰。踏实(性質や物の考え方・生活態度などが、飾り気がなくて控え目なこと。また、そのさま。質素) 【名词】 
\\	土地的肥瘠;土质(作物栽培についての、地質の良否の状態) 
\\	(土地上长出的)粮食。尤指大米(土地に産する物。特に、米)	じみ ちみ	
\\	謎	【名词】 
\\	谜,谜语 ~をあてる/猜谜 
\\	暗示;示意(物事をそれとなく遠まわしに言うこと)	なぞ	
\\	是非	【名词】 是非(是と非);正确与错误,对与不对(正しいことと正しくないこと);善恶,好与坏(善と悪) 【副词】 务必,必须(かならず);一定(きっと);无论如何(どうあっても);非……不可(…でなければならぬ)	ぜひ	
\\	袖	【名词】 
\\	袖子,衣袖(衣服の左右の腕をおおう部分。和服では、袂を含めていう) 
\\	[桌子的]两侧的抽屉;[飞机的]两翼;[建筑物的]侧厅,边房,便门;[舞台的]侧面布置(机のわきの部分。建造物・工作物の両脇の部分。舞台の左右両端の部分)	そで	
\\	果物	【名词】 水果,鲜果(木や草につく果実で、食べられるもの)	くだもの	
\\	電話	【名词】 电话;电话机(電話機で通話すること。また、その通話)	でんわ	
\\	友達	【名词】 朋友,友人(互いに心を許し合って、対等に交わっている人。一緒に遊んだりしゃべったりする親しい人。友人。朋友。友)	ともだち	
\\	毎度	【名词】 每次,屡次(同じ事が繰り返されること) ~ありがとうございます/屡蒙关照,多谢	まいど	
\\	煙草	【名词】 烟草,烟(火をつけて吸煙する嗜好品)	たばこ	
\\	物理	【名词】 
\\	事物的道理(物事の道理) 
\\	物理(学)	ぶつり	
\\	鞄	【名词】 皮包,提包,公文包(革やズックなどで作り、書類その他の物を入れる携帯用具)	かばん	
\\	美感	【名词】 美感,感到美,对美的感觉(美しさを感じる気持ち。美しさに対する感受性)	びかん	
\\	母国	【名词】 祖国	ぼこく	
\\	燕	【名词】 
\\	燕,家燕,燕子(スズメ目ツバメ科の小鳥の総称。また、その一種) 
\\	年轻的情夫(年上の女が愛人である男)	つばめ	
\\	恋人	【名词】 恋人,情人,意中人(恋爱)对象(恋しい気持ちをいだいている相手)	こいびと	
\\	渋谷	【名词】 
\\	涩谷。东京都23个区之一,基本位于行政区划的中央,大部分在山手台地上,为住宅、商业、办公设施混合地区(東京都23区の一。区部のほぼ中央に位置する。大部分は山手台地にあり、住宅・商業・業務施設の混在地区) 
\\	涩谷。涩谷区涩谷车站附近的地名,为东京都屈指可数的繁华地段(渋谷区の、渋谷駅付近の地名。都内屈指の繁華街)	しぶや	
\\	鉛筆	【名词】 铅笔(筆記具の一種。黒鉛と粘土との粉末の混合物を高熱で焼いて芯を造り、木の軸にはめて造る。1565年、イギリスで考案。江戸初期にオランダ人から輸入)	えんぴつ	
\\	ポスト	【名词】 
\\	【英】
\\	邮筒,信筒,信箱(郵便箱) 
\\	【英】
\\	地位,工作岗位,职位(地位)(地位や役職の事) 
\\	【英】
\\	交易台		
\\	先輩	【名词】 
\\	先辈,先进,(老)前辈(年齢や学識·経験上の) 
\\	高年级同学,老校友,师兄,师姐,老学长(学校の上級生、先に卒業した人尊敬している人) 
\\	(年龄,学识,经验上的)前辈,先进,同一学校的早期毕业校友,老学长	せんぱい	
\\	偏頗	偏颇,偏于一方,不公平(考え方や立場などが一方にかたよっていること。不公平なこと)	へんぱ	
\\	ピアノ	【名词】 【伊】
\\	钢琴(鍵盤楽器の1つ。大きな共鳴箱の中に金属の弦を張り、キーを叩くと、ハンマーが弦を打って音を出す)		
\\	パイプ	【名词】 【英】
\\	管子、管道。输送空气、煤气、液体等到其他地方的导管(空気・ガス・液体などを通し、他に導くための管) 
\\	烟斗。装烟丝吸烟的西式烟具(刻みタバコを詰めて喫煙する洋式の用具) 
\\	烟嘴。吸纸烟用的管状的嘴(紙巻タバコの喫煙に用いる筒状の吸い口) 
\\	穿针引线者,牵线人(二者の間の橋渡しをする人や組織)		
\\	ぺこぺこ	【副词・自动词・サ变/三类】 
\\	薄东西凹瘪后又恢复原状的声音(アルミ・鉄板・プラスチックなど堅くて薄いものが、押されたり曲げられたりして発する音。) 
\\	点头哈腰。谄媚貌(幾度も卑屈に頭を下げるさま) 
\\	肚子饿(ひどく空腹さま) 腹が〜だ/肚子空了		
\\	ぺらぺら	【副词】 
\\	喋喋不休(際限なく軽薄な調子でしゃべり続けるさま) 
\\	连续翻纸页貌(本のページなどを繰る音) ~とノートをめくる/哗啦哗啦地翻笔记本 
\\	外语说得很流利(外国語を流暢にしゃべるさま) 彼女は中国語が~だ/她中国话讲得很流利 
\\	单薄,不值钱(板・紙・布などが薄くて貧弱なさま) ~の人絹物/单薄的人造丝织品		
\\	ピストル	【名词】 【英】
\\	手枪(短銃,拳銃)		
\\	私	【代词】 我(自称の人代名詞。自分をさすことば) 【名词】 
\\	私,个人的,私事(自分に関すること。個人的なことがら) 
\\	不公平,偏私,私利(自分だけの利益や都合を考えること)	わたし し わたくし	
\\	学生	【名词】 学生(学校で勉強する人。主に、大学で勉強する人を言う)	がくせい	
\\	木村	【名词】 木村。日本姓氏	きむら	
\\	会社員	【名词】 公司职员(会社に雇われ、働いている人)	かいしゃいん	
\\	中国	【名词】 
\\	中国,中华人民共和国(東アジアの国。「中華人民共和国」の略) 
\\	日本的中国地方(中国地方の略称) 
\\	国家的中央,天子所在地(国の中央。王城付近の地。歴代の皇居が置かれた地) 
\\	日本旧时的山阴道和山阳道的总称(古くは山陽道の称。後世は山陰道と山陽道の総称) 
\\	日本律令制下诸国分为大、上、中、下四等中的中等国(律令性で、国を面積や人口などで四等に分けたうち第三等の国)	ちゅうごく	
\\	辞書	【名词】 辞典(言葉を集め、一定の順序に並べ、その読み方・意味・語源・用例などを解説した書)	じしょ	
\\	誰	【代词】 谁(不定称の人代名詞。名を知らない人や不定の人をさしていう語)	だれ	
\\	田中	【名词】 田中。日本人名(姓)(姓氏の一)	たなか	
\\	留学生	【名词】 留学生(外国へ留学して勉強する学生)	りゅうがくせい	
\\	歳	【接头/接尾】 岁,年岁(助数詞。年齢・年数を数えるのに用いる) 【名词】 
\\	年,时间单位之一(一年間,時間を測る単位) 
\\	时代,岁月,光阴(時候) 
\\	年龄(年齢。よわい) 
\\	上年纪,老年,高龄(老齢,老年,高齢) 
\\	谷物,尤指水稻的收成(穀物の実ること) 
\\	年号(年号)	さい とし	
\\	何歳	【名词】 几岁,多大年纪(いくつ)	なんさい	
\\	方	【接头/接尾】 各位,们 【名词】 
\\	〔方角〕方,方向(方角、方向、方位) 
\\	正方形的各边,平方(正方形の各辺) 
\\	方面(並んで存在する物の一方。がわ) 
\\	方面,部门(方面、部門) 
\\	方法,手段(方法;手段) 【名词】 
\\	方,方向,方面(方向。方位) 
\\	位,人(人を敬っている語) 
\\	地方,场所(ところ。場所) 
\\	(下多接否定语。)手段,方法((多く下に打消しの語を伴って)手段。方法) 
\\	时期。时分(頃。時分)	がた ほう かた	
\\	先生	【名词】 
\\	教师,教员,老师;师傅(学問、技芸などを教える人。また、自分が教えを受けている人。師匠。また、特に、学校の教員) 
\\	医生;大夫『口』(医者を尊敬する呼び名) 
\\	先生(師匠、教師、医師、弁護士、国会議員などを敬って呼ぶ語)	せんせい	
\\	新聞	【名词】 报纸,报(社会の出来事について事実や解説を広く伝える定期刊行物)	しんぶん	
\\	院生	【名】 研究生,研究生院的学生	いんせい	
\\	大学	【名词】 
\\	大学(高等教育の中核をなす教育機関) 
\\	《大学》。儒家四书之一(儒教の経書。もと「礼記」の一編。唐の韓愈、宋の2程子に推重され,朱熹が章句を作って四書の一となる)	だいがく	
\\	大学生	【名词】 大学生(大学の学生)	だいがくせい	
\\	電子辞書	【名词】 电子词典
\\	やフラッシュメモリなどの物理的な媒体や 、ネットワーク上に保存されている辞書・百科事典の内容を、コンピュータや携帯機器など の機器によって、検索・表示などを行う装置またはソフトウェアの総称)	でんしじしょ	
\\	パソコン	【名词】 【英】
\\	个人电脑,计算机(パーソナル-コンピューター)		
\\	携帯電話	【名词】 手机。便携式电话(無線を用いた小型で持ち運びができる電話)	けいたいでんわ	
\\	アメリカ	【名词】 
\\	【英】
\\	美洲,美洲大陆(北アメリカ大陸と南アメリカ大陸の総称) 
\\	【英】
\\	美国		アメリカ合衆国の略称
\\	フランス	【名词】 【英】
\\	法国,法兰西共和国(ヨーロッパ西部、大西洋と地中海に面する共和国。首都パリ。正称、フランス共和国)		仏蘭西
\\	インド	【名词】 【英】
\\	印度		
\\	王	【名词】 
\\	君主(君主の称号) ~をたてる/立王;立君 
\\	王子,皇室男子(皇族の一種) 寛仁親~/宽仁亲王 
\\	王将(将棋の駒) ~をつめる/将老帅,将军 
\\	最出色的人和物(首位にあるもの) 発明~/发明大王	おう	
\\	スミス	【名词】 【英】
\\	史密斯,英文常见人名(人名の一つ)		
\\	初めまして	【接续】 (初次见面时说的寒暄语)幸会,初次见面,请多关照(初対面の人にいうあいさつの語。はじめてお目にかかります、の意)	はじめまして	
\\	どうぞよろしく	【惯用句】 请多关照(「~お願いします」を省略した表現で、相手に頼みごとをする際の言い回し。初対面の相手に対する挨拶としても用いられることが多い) 初めまして、~お願いします。/初次见面,请多关照	どうぞよろしく	
\\	幾つ	【名词】 
\\	几个,多少,几岁。多少(年头)的数量,亦指年龄(どれぐらいの数、また、年齢。何個。何歳) 年は〜ですか/几岁了? 〜あるか知らない/不知道有几个 
\\	以〜も的形式。很多(数が少なくない様。たくさん) 同じ例は〜もある/相同的例子还有很多 
\\	以〜か的形式。几个,一些(少しの数) りんごが〜かある/有几个苹果	いくつ	
\\	韓国	【国名】 南朝鲜;韩国(朝鮮王朝が,1897年定めた国号「大韓」の通称。1910年(明治43)の韓国併合後は再び朝鮮に改めた)	かんこく	
\\	図書館	【名词】 图书馆(図書・記録やそのほかの資料・情報を収集・整理・保存して利用に供する施設)	としょかん	
\\	郵便局	【名词】 邮局(郵便局株式会社のこと。また,その事業所の称)	ゆうびんきょく	
\\	銀行	【名】 
\\	银行(預金の受入,資金の貸付,手形の割引,為替の取引などを主たる業務とする金融機関。中央銀行・普通銀行・長期信用銀行・信託銀行・外国為替銀行などがある) 
\\	库;银行(提供されたものを蓄積・保管し,求めに応じて供給する組織)	ぎんこう	
\\	教室	【名词】 
\\	教室(学校で、授業・学習を行う部屋) 
\\	研究室(大学で、専攻科目ごとの研究室) 
\\	培训班(技芸などを教えるところ)	きょうしつ	
\\	事務室	【名词】 办公室(事務を取り扱う部屋)	じむしつ	
\\	トイレ	【名词】 【英】
\\	厕所,化妆室		「トイレット」的缩略
\\	食堂	【名词】 
\\	食堂,吃饭的房间 
\\	餐厅(いろいろな料理を食べさせる店)	しょくどう	
\\	寮	【名词】 
\\	宿舍(学生・従業員などのための共同宿舎) 
\\	茶室(茶室としてつくった小さな建物) 
\\	别墅(別荘)	りょう	
\\	学生寮	【名词】 学生宿舍(大学などの学校に学ぶ学生・生徒・児童 の住む寮のこと)	がくせいりょう	
\\	建物	【名词】 房屋;建筑物(建築物)	たてもの	
\\	体育館	【名】 体育馆	たいいくかん	
\\	隣	【名词】 
\\	邻居,邻家,邻人(家・場所などが並び続いて、相接すること。また、そういうもの) 
\\	旁边,隔壁,邻室(隣の家。隣の家庭)	となり	
\\	前	【名词】 前面,前方。前面,前头,前边,连续部分的开始处。前,在时间上指较早的时候。前面,正面,对面。阴部。前科 【接头/接尾】 份,表示相当的分量。表示相当的事物,相应的价值 【接头/接尾】 
\\	尖儿,尖端,头儿,末梢(先端) 
\\	前方,前面,那面,往前(前方) 
\\	对方(先方) 
\\	先,早;最先,首先(時間・順番が早いこと) 
\\	下文,以后的情况,后来;其余(あとの続き。残り) 
\\	将来,未来,以后,前途(将来) 
\\	前头,最前部(先頭) 
\\	以前;从前;以往(以前。かつて・むかし。その前) 
\\	事先,预先(あらかじめ) 
\\	去处,目的地(目的地) 【名词】 前,以前;上	まえ さき ぜん	
\\	運動場	【名词】 运动场,操场	うんどうじょう	
\\	円	【名词】 圆。圆(形)。轮形的;(数学中)圆,圆周;圆滑,圆润,圆满;日元。日本的货币单位 【名词】 
\\	圆形。平面的圆形 
\\	(隐语)钱。金钱 
\\	(日本关西地区方言。因甲壳为圆形,故称)甲鱼	えん まる	
\\	デパート	【名词】 【英】
\\	百货商店,百货公司(社会生活に必要な殆どの商品を売ったり、種々のサービスを提供したりする、大資本による小売店。百貨店)		
\\	昨日	【名词】 
\\	昨天,昨日(今日より一日前の日) 
\\	近来,最近,过去(近い過去)	きのう さくじつ	
\\	美しい	【形容词/イ形容词】 
\\	美丽的,悦目的,装饰得漂亮的(視覚的、聴覚的にきれいで心を打つ) 
\\	美好的,高尚的,善良的(精神的に価値があって人の心を打つ。心に深い感動を呼び起こす)	うつくしい	
\\	月曜日	【名词】 星期一。每周的第二天,星期日的次日(週の第二日。日曜日の次の日)	げつようび	
\\	火曜日	【名词】 星期二。每周的第三天,星期一的次日(週の第三日。月曜日の次の日)	かようび	
\\	水曜日	【名词】 星期三,礼拜三(日曜から始まる1週の第4日)	すいようび	
\\	木曜日	【名】 星期四	もくようび	
\\	金曜日	【名词】 星期五	きんようび	
\\	土曜日	【名词】 星期六,礼拜六(金曜日の次の日)	どようび	
\\	日曜日	【名】 星期天,星期日,周日。每周的第一天(週の第1日)	にちようび	
\\	一昨日	【名词】 前天(昨日の前の日)	おととい おとつい いっさくじつ	
\\	休み	【名词】 
\\	休息(休むこと) 
\\	缺勤(欠席) 
\\	休假(休暇) 
\\	睡觉(寝る)	やすみ	
\\	バナナ	【名词】 【英】
\\	香蕉(バショウ科バショウ属の大形多年草で果実を食用にする種類の総称)		
\\	受付	【名词】 
\\	受理,接受(申し込みや願書などを受けて、それぞれに応じた適切な取り扱いをすること) 
\\	接待员;传达室,接待处(参集者の用件などを取り次ぐ係りの人(がいるところ))	うけつけ	
\\	ボールペン	【名词】 【英】
\\	圆珠笔(軸の円錐形になっている先端に細いボールがはめ込んであるペン)		
\\	コンビニ	【名词】 【英】
\\	便利店		
\\	学校	【名词】 学校(一定の教育目的のもとで教師が児童・生徒に組織的・計画的に教育を行う所、またその施設)	がっこう	
\\	シャツ	【名词】 【英】
\\	(西式)衬衣,西装衬衫(上半身に着る洋風の下着。襟とカフスの付いた衣服。ワイシャツ。開襟シャツやポロシャツなどにも言う)		
\\	レストラン	【名词】 【法】
\\	餐厅,西餐馆((高級感を与える本格的な)西洋料理店。[広義では、中華料理店を含む])		
\\	喫茶店	【名词】 茶馆,咖啡馆(コーヒー、紅茶などの飲み物、菓子、くだものや軽食を客に供する飲食店)	きっさてん	
\\	後	【名词】 
\\	后,后面(順序で、あとの方。後部。物の正面・前面と反対の側。背後。背面) 
\\	背,背面,背后;背地里(せなか。かげ) 【名词】 后边,后面,后方;以后;之后,其次;以后的事,将来的事;结果,后果;其余,此外,以下;子孙,后人;后继者,后任;死后,身后 【名词】 
\\	后,之后,以后 
\\	今后,将来,未来(将来) 
\\	死后,身后,后世(死んだあと) 【名词】 以后(時間的にあと)	うしろ あと のち ご	
\\	閲覧室	【名词】 阅览室(図書館などで、書物、新聞などを調べ読む部屋)	えつらんしつ	
\\	英語	【名词】 英语,英文(インド‐ヨーロッパ語族のゲルマン語派西ゲルマン諸語の一)	えいご	
\\	春	【名词】 
\\	春,春天(冬と夏の間の、気侯のよい季節) 
\\	青春期,极盛时期(いちばんさかりの時期) 
\\	新的一年,新春(新しい年)	はる	
\\	暖かい	【形容词/い形容词】 
\\	气温暖和(温暖気温が寒くも暑くもなくて、心持ちがよい) 
\\	东西的温度不凉,暖和(心持ちがよい) 
\\	充满情意,亲切,热情,盛情,温暖	あたたかい あったかい	温かい
\\	夏	【名词】 夏,夏天,夏季(四季の一、暑い季節) 【名词】 夏朝(中国の王朝名)	なつ か	
\\	涼しい	【形容词/い形容词】 
\\	凉快,凉爽(肌に少し冷たさを感じて心地よい) 
\\	明亮,清澈(すっきりしていて清らかな感じだ)	すずしい	
\\	寒い	【形容词/い形容词】 
\\	寒碜,简陋,破旧,穷气(心細く貧弱だある。貧しい) 
\\	冷,寒冷(気温が低い)	さむい	
\\	今日	【名词】 
\\	今日,今天(本日) 
\\	现在,现今,如今,近来,最近,现代(近頃。この頃。現代〕	きょう こんにち	
\\	雨	【名词】 雨;下雨;雨天;雨量(大気中の水蒸気が高所で凝結し、水滴となって地上に落ちるもの。雨天)	あめ	
\\	あまり	【名词】 
\\	剩余;富余(あまること。また、あまったもの) 
\\	因过于...而...(以
\\	の~」的形式表示)由于事情的程度过分而造成的结果 【形容词/イ形容词】 太,过分。程度超过寻常(程度がはなはだしいさま) 【副词】 
\\	太,过于。程度超过一般(程度がはなはだしいさま) 
\\	(下接否定词语表示)不那么,不怎么(下に打ち消しの語を伴って、さほど。大して。あんまり) 【接头/接尾】 多,余。表示比某数目稍多一些的意思(それより幾分多い意を表す)		あんまり
\\	天気	【名词】 
\\	天气,天(気象状態) 
\\	晴天,好天气(晴天) 
\\	心情(機嫌)	てんき	
\\	赤い	【形容词/イ形容词】 
\\	红(赤色である。人の血や夕焼け空のような色である。輝くような色をいい、朱・だいだい・桃色および赤茶けた色・黄色などにも通じていう) 
\\	红色,革命,左倾(俗に、共産主義的である)	あかい	
\\	黒い	【形容词/イ形容词】 
\\	黑色,黑(墨のような色である) 
\\	褐色,黝黑色(黒に近い色) 
\\	晒成的黑色(日焼けの色) 
\\	脏,肮脏(汚れている) 
\\	邪恶,不正当(悪い。不正である。悪心がある)	くろい	
\\	青い	【形容词/い形容词】 
\\	青;蓝;绿(青の色をしている。広く緑など青系統の色にもいう) 
\\	〔脸色〕发青,苍白(赤みが足りない。青ざめている) 
\\	不成熟,幼稚(修行・知識などが不十分だ。まだ一人前でない)	あおい	
\\	林檎	【名词】 苹果(バラ科の落葉高木。また、その果実。葉は卵円形。4、5月ごろ、葉とともに白または淡紅色の5弁花を開き、のち球状の赤色などの実を結ぶ)	りんご	
\\	美味しい	【形容词/イ形容词】 味美的;好吃的。语气比「うまい」客气(「いしい」に接頭語「お」がついてできた語。美味である。「うまい」より上品な語。都合がよい、儲けになる、の意でも用いる)	おいしい	
\\	高い	【形容词/イ形容词】 
\\	高的(地面などの基準になるところから、かなり上の位置にある) 
\\	地位高(りっぱ) 
\\	程度高(物事の程度が他よりも上である) 
\\	数值大(数値が大きい) 
\\	声音大(声量が大きい) 
\\	金额大(金額大きい) 
\\	向前突出(垂直方向への伸び具合が大きい) 
\\	名声高(広く知れ渡っている)	たかい	
\\	どう	【副词】 
\\	怎么样(どうですか) 
\\	怎么,如何(どんなふうに)		
\\	広い	【形容词/い形容词】 
\\	宽阔(面積が大きい) 
\\	广泛(物事の範囲が大きい) 
\\	宽广(度胸が大きい) 
\\	放开的 
\\	宽松的(ゆとりがある)	ひろい	
\\	明るい	【形容词/イ形容词】 
\\	明亮(十分な光がある。また、そう感じられる状態である) 〜色/光亮的颜色 
\\	明朗;快活;光明(人の性格、表情、雰囲気などが楽しそうだ) 〜顔つき/明朗的面容 
\\	熟悉;精通(その物事についてよく知っている。精通している) 外国貿易に〜/精通对外贸易 日本の事情に〜/熟悉日本情况 
\\	颜色鲜艳的(黒や灰色などがまじらず鮮やかである) 〜黄色い/鲜艳的黄色 
\\	有希望的(希望をもてる状態である) 〜前途/前途无量 
\\	公正的;廉洁的(公正だ。公明だ) 〜政治/廉政	あかるい	
\\	ちょっと	【感叹词】 
\\	一点点,微不足道(わずかな) 
\\	喂。轻松地向对方打招呼的用语(軽く相手に呼びかける語、もしもし) 【副词】 
\\	相当,颇(かなり、相当) 
\\	不太容易,表示没那么简单(簡単にはできない) 
\\	试试,看看,以轻松的心情事做(軽い気持ちで行う様) 
\\	一会儿,一下;表示数量不多,程度不深,时间很短等(数量、程度などがわずかなさま、時間が短い様)		
\\	狭い	【形容词/イ形容词】 
\\	窄;狭小;狭窄(空間的に面積が少ない。また、幅が短い) 
\\	狭隘,浅陋(抽象的に範囲が限られている) 
\\	〔精神上〕心胸不宽广,肚量小(精神的にゆとりがない)	せまい	
\\	パーティー	【名词】 【英】
\\	(社交性或娱乐性)会,集会;茶会,舞会,晚会,联欢会,聚餐会(社交的な集まり・会合)		パーティ
\\	楽しい	【形容词/イ形容词】 快乐的,愉快的,高兴(愉快な気持ちである)	たのしい	
\\	難しい	【形容词/イ形容词】 
\\	难,难懂,费解,艰涩,晦涩;难办,难解决(分かりにくい、やりにくい) 
\\	(病)难以治好,不好治(治りにくい) 
\\	麻烦,复杂(面倒くさい) 
\\	心绪不好,不痛快,不高兴(機嫌が悪い) 
\\	爱挑剔,爱提意见,好抱怨;不好对付;脾气别扭的人(苦情が多い) 
\\	难解决的,难达成一致的。难备齐的(解決しにくい)	むずかしい	
\\	お菓子	【名词】 点心,糕点,糖果(通常の食事以外に楽しんで食べる嗜好品。甘いものが多い)	おかし	
\\	甘い	【形容词/い形容词】 
\\	甜(砂糖や蜜のような味である) 
\\	淡(塩気が少ない) 
\\	甜蜜(うっとりと快い) 
\\	宽;姑息;好说话(厳しくない。あまやかす) 
\\	藐视,小看,看得简单(見下す。軽蔑する) 【形容词/い形容词】 
\\	〈亦作「美味しい」〉好吃的,可口的,味道好的,美味的,香的(「美味い」とも書く、食物などの味がよい) 
\\	令人喜欢的,满意的,恰好的,方便的(自分にとって都合がよい。こちらの望ましい状態だ)	あまい うまい	
\\	食べ物	【名】 食物	たべもの	
\\	古い	【形容词/い形容词】 
\\	已往(むかし) 
\\	年久,古老,陈旧(年月を経ている) 
\\	旧(使い古された) 
\\	不新鲜(新鮮でない) 〜魚/不新鲜的鱼 
\\	落后,老式,旧,陈旧,陈腐,过时(時代遅れの,古くさい) 頭が〜/老脑筋	ふるい	
\\	面白い	【形容词/イ形容词】 
\\	愉快的(気持ちが晴れるようだ) 
\\	被吸引的,精彩的,有趣的(心をひかれるさまである) 
\\	滑稽可笑的(一風変わっている) 
\\	最好的(思うとおりで好ましい)	おもしろい	
\\	蒸し暑い	【形容词/い形容词】 闷热。热得像被蒸着似的(蒸されるように暑い)	むしあつい	
\\	大きい	【形容词/イ形容词】 
\\	大,巨大;多(容積、面積、身長などが他のものより上回っている。多くの範囲を占めている) 
\\	数量大,高(数量が多い) 
\\	傲慢,不谦虚(いばっている。謙虚でない) 
\\	有气度,大量(度量がある。包容力がある。スケールが雄大だ) 
\\	夸大(おおげさだ。実際より誇張されている) 
\\	年老(年上である)	おおきい	
\\	小さい	【形容词/イ形容词】 
\\	小(かたちや規模が少ない) 
\\	微少,轻微(わずか) 
\\	幼小(おさない) 
\\	琐碎,琐细(ささいな) 
\\	狭小,狭隘(度量がない) 
\\	小声(声がちいさい)	ちいさい	
\\	安い	【形容词/い形容词】 
\\	低廉(品物の量や質の割りにねだんが低い) 
\\	安心(安心だ) 
\\	轻率(かるがるしい) 
\\	非同一般(男女の間柄を羨望し、からかう卑語)	やすい	
\\	新しい	【形容词/イ形容词】 
\\	新的(初めてである) 
\\	新鲜的(生き生きとしている) 
\\	从未有过的(今までにないものや状態である) 
\\	时髦的(新式である)	あたらしい	
\\	暗い	【形容词/イ形容词】 
\\	暗,昏暗,黑暗(光の量が少なく、物がよく見えない状態である。明るさが足りない) 
\\	发黑,发暗,深色(色がくすんでいる。黒ずんでいる) 
\\	阴沉,不明朗,不欢快(性格や気分が陰気で晴れやかでない。明朗でない) 
\\	黑暗,暗淡(希望がもてない状態だ) 
\\	生疏,不熟悉,缺乏知识,不太了解(事情をよく知らない。精通していない) 
\\	沉重(犯罪、不幸、悲惨の存在を感じさせる)	くらい	
\\	宿題	【名词】 
\\	课外作业(学校の課題) 
\\	有待将来解决的问题,悬案(懸案)	しゅくだい	
\\	教科書	【名词】 教科书,课本(教授・学習の教材として使用される図書。小学校・中学校・高等学校およびこれらに準ずる学校で、教育課程の構成に応じて作成される、主たる教材としての児童・生徒用図書。これらの学校には教科書使用義務が課されている。教科用図書)	きょうかしょ	
\\	白い	【形容词/イ形容词】 
\\	白色(白の色である。雪のような色である) 
\\	空白(空白である) 
\\	干净,洁白(潔白である。無実である)	しろい	
\\	映画	【名】 电影	えいが	
\\	東京	【名词】 东京(日本の首都)	とうきょう	
\\	賑やか	【形容动词/ナ形容词】 
\\	热闹,熙熙攘攘,繁华,繁盛(盛んなさま。繁栄するさま) 
\\	极其开朗,热闹(よくしゃべり、よく笑って、陽気なさま)	にぎやか	
\\	静か	【形容动词/ナ形容词】 
\\	静止的,不动的(じっとしていて、動かない) 
\\	平静的,安静的,沉静的,文静的(穏やかである) 
\\	轻轻的,慢慢的(そっと。ゆっくりと) 
\\	寂静的,肃静的,静悄悄的,平静的(物音がない様)	しずか	
\\	町	【名词】 
\\	镇;城镇(人が多く集まり住んでいるところ) 
\\	町(市や区を構成する小区画) 【名词】 
\\	镇(地方行政の町) 
\\	〔市や区の小区画〕 
\\	町(長さの単位)(面積単位)	まち ちょう	街
\\	先週	【名词】 上星期,上周(今の週の前の週。前週)	せんしゅう	
\\	旅行	【名词】 旅行,旅游,游历(見物・保養・調査などのため,居所を離れてよその土地へ行くこと。旅をすること。たび)	りょこう	
\\	京都	【名词】 京都,日本的都市之一(日本の都市である)	きょうと	
\\	紅葉	【名词】 
\\	(树叶秋天)变红。红(枫)叶 
\\	槭树,枫树 
\\	鹿肉的别称 
\\	日本地名 【名词・自动・三类】 秋末时,树叶掉落之前变红的现象。红叶,霜叶	もみじ こうよう	
\\	綺麗	【形容动词/ナ形容词】 
\\	美丽,漂亮,好看(美しい) 
\\	洁净,干净(汚れがない) 
\\	漂亮地(完成某事)(上手で立派だ) 
\\	完全,彻底,干干净净(残りがない)	きれい	
\\	観光客	【名词】 游客,游人(観光のため旅行する人)	かんこうきゃく	
\\	多い	【形容词/い形容词】 多的。数目或者分量大。数量、次数等相对较大、较多(数や量がゆたかである。数量・度数が相対的に大である)	おおい	
\\	どんな	【形容动词/ナ形容词】 
\\	怎样,怎么样,如何,哪样的,什么样的(これぐらいとか、このようなとか指示できない、不明の状態や性質などを表したり、それを尋ねたりするときに用いる。どのような。どんな具合) 
\\	不拘什么样的,任何的([「でも」「ても」「も」などを伴って]物事の程度や状態、その時の事情などに影響されないようす。たとえ最もひどい…でも。いかなる)		
\\	親切	【名・形容动词/ナ形容词】 亲切,恳切,好心(相手の身になって、その人のために何 かをすること。思いやりをもって人のためにつくすこと。また、そのさま)	しんせつ	
\\	上手	【形容动词/ナ形容词】 
\\	好,高明,擅长,善于,拿手,能手(技術がすぐれている。手際がよいこと。また、その人やさま) 
\\	善于奉承,会说话(言葉巧みなこと。お世辞) 【名词】 
\\	(相撲)从对方伸出的胳膊上的外侧抓住对方腰带(的手)(相撲で、相手の差し手の上からまわしを取る組み手。また、その腕) 
\\	高处,上头,上边(上の方。川上や風上をいう。かみて。↔下手) 
\\	强手,高手(囲碁・将棋などで、棋力の優れた方の指し手) 
\\	石带左端附着的皮革带(石帯の左の端についている革帯) 【名词】 
\\	上方,上部,上手。地势高的地方(地勢的に高い方) 
\\	上游(川の上流の方) 
\\	上座,上首。由观众席看舞台的右手方(演劇において客席から舞台を見た時の右側のこと。撮影においては、カメラのファインダーから見て右側のこと)	じょうず うわて かみて	
\\	便利	【名词.形動】 便利,方便;便当(都合のよいこと。役に立って具合のよいこと。また、そのさま)	べんり	
\\	日本料理	【名词】 日本菜,日本料理(日本で発達した伝統的な料理。材料の持ち味を生かし、季節感や盛付けの美しさを重んずるのが特色)	にほんりょうり	
\\	好き	【名・形容动词/ナ形容词】 
\\	喜好,喜爱,爱好(好くこと。気に入って心がそれに向うこと。その気持) 
\\	嗜好,好奇(片寄った好み。また、物好きなさま) 
\\	好色(好色。色ごのみ) 
\\	随便,任性,随心所欲,随意(気まま。勝手)	すき	
\\	刺身	【名词】 生鱼片(新鮮な魚介類などの肉を生のまま、薄く切って、醬油などをつけて食べる料理。つくりみ。おつくり)	さしみ	
\\	大好き	【名词/形动】 顶喜欢,非常喜爱,最爱好(たいへん好きな・こと〔さま〕) 肉が~です/最爱吃肉	だいすき	
\\	そして	【惯用句】 就像上述那样地(そのような事をしたのに引き続いて(手段や方法をとって)何かが行われることを表す。その結果) 【接续词】 
\\	然后(前に述べた事柄を受けて、それに引き続いて起こる事柄を述べる。それから) 
\\	而,又,而且(前件に述べた事柄に後件をつけ加える。そのうえ。さらに)		
\\	元気	【名・形容动词/ナ形容词】 精神,精力(充沛),朝气,锐气〔気力がいい〕 【形容动词/ナ形容词】 身体健康;硬朗〔体がいい。健康である〕	げんき	
\\	有名	【形容动词/ナ形容词】 有名,著名,闻名,臭名昭著(たくさんの人たちに知られ、注目されていること)	ゆうめい	
\\	嫌い	【形容动词/ナ形容词】 不愿,厌烦,厌恶。嫌恶,讨厌(嫌うこと。嫌がること) 【名词】 有…之嫌,有…的倾向(気がかりな傾向)【接头/接尾】 嫌恶,厌恶(的人)(きらうこと。いやがること)	きらい ぎらい	
\\	テニス	【名词】 【英】
\\	网球(球技の一)		
\\	歌	【名词】 
\\	歌。歌曲(声に節をつけて歌う詞) 
\\	和歌。短歌(和歌) 
\\	诗(音律に合わせて数を整えた詞) 【名词】 
\\	歌。歌曲。歌词 
\\	短歌。和歌(和歌。短歌) 
\\	唱歌	うた か	
\\	繁華	【な形容词】繁华,热闹(人が多く集まり、にぎわっていること。また、そのさま)	はんか	
\\	ハンサム	【名词】 【英】
\\	美男子;有风度(的男人)(男子の顔立ちの良い様。美男)		
\\	簡単	【名词·形动】 简单;简易,容易;轻易;简便(単純で、理解や扱いが容易であること。また、時間や手数のかからないこと)	かんたん	
\\	病気	【名词】 病,疾病,病症,疾患『書』(肉体の働きまたは精神の働きに異常が起こり、不快や苦痛を感じ、通常の生活を営むために治療が必要となる状態。やまい。疾患。疾病)	びょうき	
\\	お茶	【名词】 
\\	茶(茶の丁寧語) 
\\	(2茶道(茶の湯) 
\\	吃茶点 
\\	休息,小憩(仕事の合間の小休憩)	おちゃ	
\\	紅茶	【名词】 红茶 ~をいれる/泡红茶	こうちゃ	
\\	音楽	【名词】 音乐(音による芸術)	おんがく	
\\	ロック	【名词】 
\\	【英】
\\	摇滚乐(電気機音楽を使いビートを強調した音楽) 
\\	岩石(岩。岩石) 【名・他动词・サ变/三类】 
\\	【英】
\\	锁(錠) 
\\	上锁(錠をおろすこと) 
\\	锁死,抱死 
\\	オンザ〜的略语 
\\	冰镇威士忌岩石		
\\	スポーツ	【名词】 【英】
\\	(体育)运动(陸上競技・野球・テニス・水泳・ボートレースなどから登山・狩猟などにいたわるまで、遊戯・競争・肉体的鍛錬の要素を含む身体運動の総称)		
\\	ダンス	【名词】 【英】
\\	跳舞,交际舞(踊り)		
\\	ギター	【名词】 【英】
\\	吉他,六弦琴(撥弦楽器の一)		
\\	ピンポン	【名词】 〈体〉乒乓球(卓球)		
\\	暇	【名词】 
\\	时间,工夫(余裕の時間) 
\\	闲空,余暇,闲工夫(余暇) 
\\	闲散(仕事のない間) 
\\	解雇,辞退(解雇) 
\\	休假,假(休暇) 【形容动词/ナ形容词】 闲,空闲(仕事や義務に拘束されず、自由にできる時間があるさま。同:閑) 【名词】 
\\	暇,闲暇,余暇,工夫,空(ひま) 
\\	休假(休み) 
\\	解雇,辞退;辞职(辞職) 
\\	离婚(離縁) 
\\	告辞,辞去,辞行,告别(別れ)	ひま いとま	
\\	何時も	【名词】 日常,平日,往常(普段、平日) 【副词】 无论何时,经常(常に。どんな時でも)	いつも	
\\	起きる	【自动词・五段/一类】 
\\	起,起来,立起来;坐起来(立ちあがる) 
\\	起床(ねどこをはなれる) 
\\	不睡(めざめている) 
\\	(事件等)发生	おきる	
\\	今	【名词】 
\\	现在,此时,此刻,目前 
\\	现代,当代,现今,如今 【副词】 
\\	刚才,方才 
\\	立即,马上 
\\	再,更,另外(ここで更に。その上に。もう) 【接头/接尾】 (接头)现在的,当今的(現代の。当世の)	いま	
\\	何時	【名词】 几点钟。什么时候(時刻が不明のときに用いる語。いくじ。いつのとき)	なんじ なんどき いつ	
\\	今朝	【名词】 今天早晨(早上),今朝(話をしている、その日の朝) 【名词】 今早,今天早晨。文语(今日のあさ)	けさ こんちょう	
\\	半	【名词】 
\\	半,一半(半ば。半分) 
\\	二倍半 
\\	奇数。用2除不尽的数(2で割りきれない数。奇数) 
\\	半。1小时的二分之一,古代计时的“刻”的二分之一(一時間の二分の一。昔の時刻では刻の二分の一) 
\\	表示中途,一半,不彻底的意思(半ば、半分、不完全ななどの意を表す)	はん	
\\	毎朝	【名词】 每天早晨〔早上〕(毎日の朝。朝ごと)	まいあさ	
\\	運動	【自动词・サ变/三类】 
\\	运动,体育运动(体育、保健や楽しみのために身体を動かすこと。スポーツ) 
\\	运动,活动(物体が時間の経過につれて、その空間的位置を変えること) 【名词】 宣传活动(キャンペーン。目的を達するために活動すること)	うんどう	
\\	朝	【名词】 
\\	朝,早晨(日の出からの数時間) 
\\	早上;午前(午前) 【名词】第二天早上(翌朝) 【名词】朝廷;朝,王朝,朝代;治世;君主国;朝,政府	あさ あした ちょう	
\\	授業	【自动词・サ变/三类】 授课,教课,讲课,上课(学校などで、学問などを教えること)	じゅぎょう	
\\	始まる	【自动词・五段/一类】 
\\	开始(開始する) 
\\	发生,引起(新たに起こる) 
\\	起源,缘起(起因する。最初となる) 
\\	犯(老毛病),拿出平生的本事(癖になっている言語や動作が出る)	はじまる	
\\	終わる	【自动词・五段/一类】 
\\	完,完毕,结束,告终,终了(続いていた物事がしまいになる) 仕事が〜/工作结束 
\\	死,死亡(死ぬ) 
\\	以…告终,以…而结束(終結する。終了する) 【接头/接尾】 做完(動作をし終わる) 読み〜/读完	おわる	
\\	明日	【名词】 明天(今日の次の日) 【名词】 
\\	明天(今日の次の日。明日。みょうにち) 
\\	将来(近い将来。未来)	あした あす みょうにち	
\\	働く	【自动词・五段/一类】 
\\	工作,劳动,做工(仕事をする。労働する。特に、職業として、或いは生計を維持するために、一定の職に就く) 
\\	起作用(機能する。また、作用して結果が現れる) 
\\	(精神、才能等)活动(精神などが活動する) 
\\	生效,发生效力(作用して結果が現れる) 
\\	(日语语法上指)活用(文法で、用言や助動詞の語尾が変化する。活用する) 
\\	干坏事(悪事をする)	はたらく	
\\	明後日	【名词】 后天(あすの次の日、明日の次の日。あさって)	あさって みょうごにち	
\\	昨夜	【名词】 昨夜,昨晚(きのうの夜)	ゆうべ	
\\	勉強	【名・自他・サ变/三类】 
\\	用功学习,用功读书,学习知识,积累经验(学問や技芸を学ぶこと。学習) 
\\	勤奋,热心(工作) 
\\	廉价,贱卖(値引き)	べんきょう	
\\	風邪	【名词】 感冒;伤风(感冒。風の病)	かぜ ふうじゃ	
\\	休む	【自动词・五段/一类】 
\\	休息,中止工作、动作等,使身心得到放松(仕事や動作を中止して、体や心を楽にする。休憩する) 
\\	停歇,暂停(継続的・定期的に行なってきたことを一時とりやめる) 
\\	睡,卧,安歇,就寝(床につく) 
\\	缺勤,缺席(本来行くべき学校や勤めに行かない)	やすむ	
\\	寝る	【自动词・一段/二类】 
\\	躺下(よこになる) 芝生の上に〜/躺在草坪上 
\\	倒伏(倒れる) 
\\	睡眠(ねむる) 
\\	因病卧床(病床に臥す) 
\\	发酵成熟(麹が成熟する) 
\\	闲置,滞销(資本や商品が回転しない) 
\\	男女行房(セックスする)	ねる	
\\	夜	【名词】 夜,夜里,晚上(日没から日の出までの時間。太陽が沈んで,あたりが暗くなっている間。夜間)	よる よ や	
\\	来る	【自动词・カ变/三类】 
\\	来,在时间、距离方面向说话者的方向靠近(話し手のいる方へ近づく) 
\\	到来,时间、季节等来临(時間が経過して,その季節・時期・順番になる) 
\\	发生(异常状态)(ある事態が出現する) 
\\	引起(「…からくる」の形で)…がその由来・原因となっている。…のために生ずる) 
\\	提起,说起,特别是…,表示强调某件事情(あるものを話題にとりあげて示す) 
\\	表示动作,作用在空间向说话者移动(話し手の方へ向かって動作が行われ,その話し手の方へ近づく意を表す) 
\\	表示动作、作用一直持续到现在(動作が継続・反復されて現在に至るまで続く意を表す) 
\\	表示动作和作用渐渐实现(戻ることを前提にして)動作が行われる意を表す。動作の実現・完了に重点が置かれる場合もある) 
\\	表示由某种状态慢慢变成另一种状态(事態が進行してある段階に至る意を表す) 
\\	表示发生某种现象(ある事態が出現し,またある現象が現れる意を表す)	くる	
\\	歩く	【自动词・五段/一类】 
\\	走,步行 ~行く/走着去; 步行 
\\	到处(動詞の後について「あちこち…してまわる」) 酒場を飲み~/串酒馆喝酒	あるく	
\\	講義	【名・他动词・サ变/三类】 
\\	讲义(書籍またはがくせつ) 
\\	大学课程(大学などで、教授者がその学問研究の一端を講じること)	こうぎ	
\\	散歩	【名词/自サ】 散步,随便走走(気晴らしや健康のために、ぶらぶら歩くこと。散策)	さんぽ	
\\	練習	【名・他动词・サ变/三类】 练习,反复学习(学問、技芸などの上達を目標に、繰り返して習うこと)	れんしゅう	
\\	キャンパス	【名词】 【英】
\\	大学校园(校庭)		
\\	今晩	【名词】 今宵。今晚。今天晚上。今夜。今儿晚上(きょうの晩。こんや。こよい)	こんばん	
\\	朗読	【名・他动词・サ变/三类】 朗读,朗诵(声を出して詩・文章などを読むこと)	ろうどく	
\\	自分	【代词】 
\\	自己,自个儿,自身,本人(その人自身。) 
\\	我(わたくし)	じぶん	
\\	毎日	【名词】 每天,每日,天天(どの日も。日ごと)	まいにち	
\\	一日	【名词】 
\\	一日,一天,一昼夜(一昼夜) 
\\	终日,一整天(終日) 
\\	一号,初一,一日 
\\	某日,某一天(ある日) 
\\	短期间,短时日(短時日) 【名词】 
\\	整天,终日(終日) 
\\	某天,某日(ある日)	いちにち ついたち いちじつ ひとひ	
\\	近く	【名词・副词】 
\\	近处,近旁,附近(そば) 
\\	不久,近期,即将(近い将来) 
\\	近乎,将近,几乎,快,快了(ほぼ)	ちかく	
\\	公園	【名词】 
\\	公园(主に市街地またはその周辺に設けられ,市民が休息したり散歩したりできる公共の庭園) 
\\	(観光や自然保護のために指定されている地域。国立公園や県立自然公園など)	こうえん	
\\	行く	【自动词・五段/一类】 
\\	行进,去(自分のいる所から遠ざかる) 
\\	到,往,前往(目的の場所に向かって進む) 
\\	进展,进行(事態が進展する) 
\\	满足,满意(満ち足りた状態になる) 
\\	上年纪(年を取ること) 
\\	逝去,流逝(時間が経過する) 
\\	移至他处(他の家へ移る) 
\\	到达(到着すること) 
\\	通往,通向(目的地に向かって進む)	いく ゆく	
\\	空気	【名词】 
\\	空气(地球を包んでいる無色透明の気体) 
\\	气氛(雰囲気)	くうき	
\\	景色	【名词】 
\\	景色;风景;风光(風景。自然の眺め) 
\\	(日本茶道指陶瓷器等茶具上的)精彩之处	けしき	
\\	ぐらい	【副助词】 
\\	大约;大概;左右((数量を表す語や基準を示す語について)大体の見当を表す) 
\\	像…那样;到…程度;表示比较的标准(比較的理解しやすい基準を示しつつ、問題になっている状況の程度を表す) 
\\	微不足道;一点;表示小看、藐视的心情(それに比べると、ほかの場合は物の数でないという意を表す)	くらい	
\\	本当	【名・形容动词/ナ形容词】 
\\	真,真的,真正(まこと) 
\\	实在,的确(実際) 
\\	本来,正常(本来)	ほんとう	
\\	一緒に	【副词】 一起,一同(ともに何をする様子)	いっしょに	
\\	写真	【名词】 照片,照片儿;相片;照相,摄影,拍照(人を写した、写真をとること)	しゃしん	
\\	撮る	【他动词・五段/一类】 摄影,照相,拍(写真や映画などを映す)	とる	
\\	電車	【名词】 电车(電力でレールの上を走り、客や貨物を運ぶ車両)	でんしゃ	
\\	新宿	【名词】 东京的一个区新宿是日本东京的一个地区名,指的是以新宿车站为中心的地区,隶属东京都新宿区管辖。与涩谷、池袋并列为东京都的3大副都心之一。在新宿集中了大量的企业总部和政府机关。东京都厅舍即位于西新宿(東京の新宿地方)	しんじゅく	
\\	店	【名词】 
\\	商店,店铺(商品を並べておいて売る所) 
\\	格子门状的房间(妓楼で、道路に面して格子構えなどにし、遊女がいて遊客を誘う座敷)	みせ	
\\	腕時計	【名词】 手表(手首につける小型の時計)	うでどけい	
\\	買う	【他动词・五段/一类】 
\\	买(品物や金と引き換えに、自分の望みの品物を得る) 
\\	因自己的行为招致不好结果(悪い結果を招く) 
\\	主动接受(進んで身に引き受ける) 
\\	器重(価値を認める)	かう	
\\	【名】 
\\	唱盘	シーディー	
\\	お腹	【名词】 肚子,女性用语(女性用語。腹の美化語)	おなか	
\\	空く	【自动词・五段/一类】 
\\	某个空间里的人、物数量减少(まばらになる。減る) 
\\	肚子空,肚子饿(空腹になる) 
\\	有空闲(ひまになる) 【自动词・五段/一类】 
\\	出现空隙或者空隙变大(穴ができる) 
\\	职位等出现空缺(欠員になる) 
\\	有空,有空闲 
\\	原本在某处的人或者物消失不见,某处变空(からになる) 
\\	用毕空置 
\\	空,闲	すく あく	
\\	食事	【自动词・サ变/三类】 饭,餐,食物;吃饭,进餐『書』(生命を維持する栄養を取るため、一日に何度か物を食べること。また、その食べ物)	しょくじ	
\\	入る	【自动词・五段/一类】 
\\	进,入,进入(ある区域の中に移動する) 
\\	在内,归入,有,含有,包括在其范围内(ある範囲に含まれる) 
\\	装入,容纳,放入,归入,容纳入某容量之中(ある容量の中におさまる) 
\\	得到,到手,收入,入主,变为自己所有(自分のものになる。手元に届けられる) 
\\	加入,进入,成为组织的一员(組織の一員となる) 
\\	硬加入,挤入(割り込む) 
\\	出现、产生裂纹等(割れ目などが生じる) 
\\	为感官所感知(知覚される) 
\\	精力充沛(気力がこもる) 
\\	进,入,到达某时期或某阶段(ある時期や階段に至る) 
\\	饮(酒)(酒を飲む) 
\\	备好茶,茶已准备好(お茶の用意ができる) 【自动词・五段/一类】 
\\	入,进入(外から中に移動する。はいる)(接在动词连用形下加强语气)表示一时很难摆脱某种状态(それから抜け出せない状態になる) 
\\	沉入,没入;隐入(太陽・月が没する) 
\\	进行,处于某种状态(進んで行き、ある階段に達する) 
\\	进入某个时间,某个季节(時間が経ち、ある区切られた時間・期間内になる。また、年月が重なる) 
\\	(果实)成熟,结果(果実の内部がいっぱいになる。みのる。熟する) 【古语】 (以「御入り候ふ」、「いらせたまふ」等形式表示)「行く」、「来(く)る」、「居る」的尊敬语	はいる いる	
\\	寿司	【名词】 
\\	盐腌的鱼类、贝类自然发酵而成的食品(古くは、魚介類に塩を加えて漬け込み自然発酵させた食品。後には発酵を早めるため、飯とともに漬けるようになった。なれずし) 
\\	寿司,以醋味饭为主要材料的散寿司、攥寿司等的总称(酢で味付けをした飯に刺身や卵焼き・海苔などをあしらった食べ物。握り鮨・巻き鮨・押し鮨・散らし鮨など)	すし	
\\	新鮮	【形容动词/ナ形容词】 
\\	新鲜(魚・肉・野菜などが、新しくて生き生きとしていること。また、物事に今までにない新しさが感じられるさま) 
\\	清新(汚れがなく、澄みきっていること。また、そのさま)	しんせん	
\\	会社	【名词】 公司(営利行為を事業の目的とする社団法人で、商法によって設立されたもの)	かいしゃ	
\\	背	【名词】 
\\	脊背,后背,脊梁(せなか) 
\\	后方,背景(うしろ) 
\\	身长,身高,身材,个子(せたけ) 
\\	山脊,岭巅(おね)	せ	
\\	サンドイッチ	【名词】 【英】
\\	三明治,夹心面包		
\\	パン	【名词】 
\\	【葡】
\\	面包(小麦粉を主材料とし、これに水とイーストなどを加えてこね、発酵させてから焼き上げた食品) 
\\	【葡】
\\	物质生活(転じて、生活の糧) 
\\	【英】
\\	平锅,带柄煎锅(浅めで底の平らな、取っ手の付いている鍋。平鍋)		
\\	食べる	【他动词・一段/二类】 
\\	吃(飲食物をいただく) 
\\	生活(生計を立てる)	たべる	
\\	バス	【名词】 
\\	【英】
\\	公共汽车(大勢の人を乗せることのできる大型自動車) 
\\	【英】
\\	男低音,男低音歌手(男性の声の中で一番低い音域。また、その歌手) 
\\	【英】
\\	低音大提琴(コントラバスの略) 
\\	【英】
\\	浴缸(洋式の風呂)		
\\	自転車	【名词】 自行车,脚踏车,单车(乗る人が自分でペダルを踏み車輪を回転させて走る二輪車)	じてんしゃ	
\\	地下鉄	【名词】 地下铁道,地铁(地下にトンネルを掘って敷設した鉄道)	ちかてつ	
\\	タクシー	【名词】 【英】
\\	出租汽车;的士;计程车(客の求めに応じて、目的地まで客を運び、距離、時間に応じて料金を取る営業用自動車)		
\\	コーヒー	【名词】 【蘭】
\\	咖啡		
\\	時計	【名词】 钟,表 腕〜/手表 懐中〜/怀表	とけい	
\\	飲む	【他动词・五段/一类】 
\\	喝,咽,吃(口に入れて噛まずに食道の方に送りこむ。喉に流しこむ。特に、酒を飲む) 
\\	吞下去(飲みこむ。姿を包み込んで見えなくする) 
\\	吸(烟),吸进(吸い込む。吸う) 
\\	饮(泣),吞(声)(こらえておもてに出さない) 
\\	藐视,不放在眼里,压倒(圧倒する。また、見くびる) 
\\	(无可奈何地)接受(受け入れる) 
\\	暗中携带(収める。隠し持つ)	のむ	呑む
\\	沢山	【副词・形容动词/ナ形容词】 
\\	很多,许多,大量,指数量多(数量の多いこと) 
\\	足够了,充分(十分なこと。もうそれ以上不要なこと)	たくさん	
\\	間	【名词】 
\\	间隔,距离(間隔、隔たり〕 行と行の~をあける/行间留出间隔 
\\	间,中间(相対する二つのものの関係) 
\\	期,时候,工夫(期間、一続きの時間〕 
\\	关系(間柄。仲) 
\\	开间,空隙,缝隙(2つの物に挟まれた、あいている部分) 【名词】 
\\	空隙,空当子『口』;物与物之间空开的空间(物と物の間の空間) 
\\	房间,间房屋内的一个间隔(家屋内の一区切り) 
\\	工夫,时间(事と事とのあいだの時間) 
\\	休止,板眼音乐,舞蹈,戏剧中等,拍子与拍子空开的时间间隔(音楽、舞踊、演劇などで、拍と拍の間の時間の間隔) 
\\	(適当な時機機会潮時) 
\\	当时的情形,氛围(その場の具合雰囲気) 【接尾】 
\\	间,期间,中间 
\\	间(あいだ);机会(チャンス);间隙(すき) 
\\	(へだたり)隔阂,裂痕	あいだ ま かん	
\\	どうも	【副词】 
\\	(多接否定)怎么也(どうしても) 
\\	总觉得,总有些,表示原因、理由不清晰的意思(なんだか。どことなく) 
\\	啊呀,表示轻微的惊讶或疑惑(「ありがとう」「すみあせん」などの上につけて、その気持ちを強めることば。下につくことばを省略することも多い) 
\\	真是,实在是,加在表示感谢或者谢罪的语句前,起强调作用(まったく。本当に)		
\\	教学棟	【名】 教学楼	きょうがくとう	
\\	黒板	【名词】 黑板(ボールド)	こくばん	
\\	テレビ	【名词】 【英】
\\	电视(机)(~ジョンの略)		
\\	壁	【名词】 
\\	墙,壁(室と室を隔てるもの) 
\\	障碍(物)(障害);隔阂(へだたり) 
\\	峭壁(岩壁) 
\\	(已婚妇女用语)豆腐	かべ	
\\	クラス	【名词】 
\\	【英】
\\	等级,阶级,阶层(等級。階級) 
\\	【英】
\\	学年,年级,班级(学級)		
\\	木	
\\	〔樹木〕木,树 
\\	木纹,木理 
\\	〔〜曜日〕星期四 
\\	五行的第一位。方位而言是东,季节而言是春。在五星中指木星,十干中是甲、乙 【名】 
\\	〔樹木〕树,树木 
\\	〔材木〕木头,木材,木料 
\\	〔拍子木〕梆子	き もく	
\\	湖	【名词】 湖,湖水(周囲を陸地で囲まれ、直接海と連絡のない静止した水塊)	みずうみ	
\\	庭	【名词】 
\\	门内空地(邸内または階前の、農事に使う空地) 
\\	场所(物事を行う場所)	にわ	
\\	子供	【名词】 
\\	自己的儿女(むすこ・むすめ) 
\\	儿童,小孩儿(児童) 
\\	仔,崽(動物の子) 
\\	幼稚(考えの未熟な人)	こども	
\\	スーパー	【名词】 
\\	【英】
\\	超,上,高级,超级(特に優れている、などの意) 
\\	超市「~マーケット」的省略语(高頻度に消費される食料品や日用品などをセルフサービスで短時間に買えるようにした小売業態である)		
\\	左	【名词】 
\\	左,左面(南を向いた時、東にあたる方) 
\\	左手(左の側の手) 
\\	左派,左倾,急进((思想上・政治上の)左翼) 
\\	喝酒(的人),爱喝酒(酒を好み飲むこと) 
\\	左,古时左右对称的官职中左的一方,在日本,左居于右之上(昔、左右に分れた官職の左の方)	ひだり	
\\	卵	【名词】 
\\	动物卵的总称(鳥・魚・虫などの雌が産む、殻や膜に包まれた胚や栄養分) 
\\	鸡蛋 
\\	未成熟者。尚未成形(修業中で、まだ一人前にならない人)	たまご	
\\	靴	【名词】 鞋(履物の一種)	くつ	
\\	ケーキ	【名词】 【英】
\\	洋点心,西洋糕点(洋風の生菓子の総称。特に、スポンジ‐ケーキを台にして、クリームや果物を加えた菓子)		
\\	貰う	【他动词・五段/一类】 
\\	领取,得到,收到(人がくれるというものを(頼んで)自分のものにする) 
\\	受到(批评等)(叱られる。罰される) 
\\	承担,接受,包(他人の物事を引き受ける。預かる) 
\\	娶,收养,聘用,作为自己的组织或家庭成员迎接过来(人に請うて自分のものとする) 
\\	传上,感染上,被迫接受非自己责任的东西(自分が望まないものを与えられる) 
\\	买(買うことを遠まわしに表す語) 
\\	赢,获胜(勝負で勝つ) 【助动词】 
\\	〈以「…して~」的形式〉请求,承蒙,从别人的动作中获益,委托别人去行动((動詞連用形に助詞「て(で)の付いたものに続けて」)他人の動作によって、自分が恩恵・利益を受ける意を表す。また、自分のための行為・動作を他人に依頼し、させる意を表す) 
\\	〈以「…して~」的形式〉请求,希望。自己的动作或行为给他人带来好处((動詞連用形に助詞「て(で)の付いたものに続けて」)自分の動作によって、他人が恩恵・利益を受ける意を表す)	もらう	
\\	誕生日	【名词】 生日,生辰(生まれた日。生誕記念日)	たんじょうび	
\\	チョコレート	【名词】 【英】
\\	巧克力(カカオ豆を炒って皮などを除き、すりつぶしたものに砂糖・カカオバター・粉乳などを加え、練って固めた菓子)		
\\	セーター	【名词】 【英】
\\	毛衣,毛线上衣(毛糸などで編んだ上着。普通,頭からかぶって着るものをいう)		
\\	妹	【名词】 
\\	妹妹 
\\	小姑,小姨,弟妹 【名词】 
\\	男子对关系亲密的女子的称呼(男が女を親しんでいう語) 
\\	女子之间亲热的称呼对方的词(女同士が親しんでいう語)	いもうと いも	
\\	呉れる	【他动词・一段/二类】 
\\	给(我),他人给自己的东西(相手がことらにわたす) 兄が本を~/哥哥给我书 
\\	俗语
\\	给(人),给他人东西,轻视受方的说法(自分が相手に与える) 
\\	他人为自己做有利的动作(相手の動作が、都合よく感じられる時に使う) 
\\	表示给对方带来不利(相手に不利益をおよぼす)	くれる	
\\	嬉しい	【形容词/い形容词】 高兴,快活,喜悦,欢喜(こころよく楽しい)	うれしい	
\\	頂く	【一类】 
\\	顶,戴(頭に載せる) 
\\	拥戴,推戴(敬い仕える) 
\\	领受,拜领,从长辈或上司处领受物品(ちょうだいする) 
\\	吃,喝(飲食する) 
\\	请允许,请让(我)(してもらう)	いただく	戴く
\\	皆	【名词・副词】 
\\	全体,大家(すべての人と物。「みな」の撥音化) 
\\	全,都,皆,一切(残らず。すべて。ことごとく)	みんな	
\\	話	【名词】 
\\	说话,讲话,谈话(話すこと。口に出して語ること) 
\\	话,谈话,说话,讲话(話された内容) 
\\	话题(話題) 
\\	传说,传闻(うわさ。評判) 
\\	商谈(話し合って決めるべき事柄) 
\\	话,讲给别人听的,有某些内容或条理的事(人に語り聞かせる、ある内容や筋を持った事柄) 
\\	事理(物の道理) 
\\	经过(経緯) 
\\	假话(つくりごと。嘘)	はなし	
\\	生活	【自动词・サ变/三类】 
\\	生活。人和生物为生存和发展而进行的日常活动(生きて活動すること。暮らし) 
\\	生活,谋生。维持度日的活动(暮らしを支えているもの。生計) 〜を営む/营生	せいかつ	
\\	寂しい	【形容词/い形容词】 
\\	寂寞,孤寂,孤单,凄凉,孤苦,无聊(心細い) 
\\	荒凉,凄寂,冷清,空寂(ひっそりした) 
\\	觉得不满足,空虚(ものたりない)	さびしい さみしい	
\\	時々	【名词】 每个季节;一时一时(その時節時節。その時その時) 【副词】 时常,时时。 有时(時として。ときおり。まま。おりおり。たまに)	ときどき	
\\	掛ける	【他动词・一段/二类】 悬挂;(为做菜而)放在火上;将身体的一部分放在某物上;固定;称(分量);(用线、绳等)绕上;使(机器)启动;做乘法;(按照合同)付款;报谎价;交配;使一个语句与另外的语句之间发生意义关联;付出(时间、劳力等);(用机关)捕捉;(提出问题)讨论研究;在(某个时间、场所)范围之内;施加(不好的影响);使负担;注意;显示;就诊;许愿;建造;上演;蒙;搅拌;(用工具)打扫,弄干净;攻击;放(火、箭等)	かける	
\\	お母さん	【名词】 妈妈,母亲。你妈,令堂(母の丁寧語。母親の立場にある人をいう語。母親が自分をいう場合にも用いられることがある)	おかあさん	
\\	弟	【名词】 
\\	弟弟(同じ親から生まれた年下のもの、特に男子) 
\\	后辈(年少の男を親しんでいう称)	おとうと	
\\	手紙	【名词】 信,书信,函,尺牍,书札([「手」は文字・筆跡の意]用事などをしるして、他人に送る文書。書簡。書状)	てがみ	
\\	お父さん	【名词】 
\\	父亲,爸爸,爸(父の丁寧語) 
\\	做父亲的。对已做父亲的人的称呼,也用于父亲指称自己	おとうさん	
\\	プレゼント	【名・他动词・サ变/三类】 【英】
\\	赠送礼物,礼品,赠品,礼物,送礼(贈り物をすること。また、贈り物)		
\\	風邪を引く	【惯用句】 着凉。感冒(かぜにかかる)	かぜをひく	
\\	父	【名词】 
\\	父亲,家父(自分のお父さんのこと) 
\\	先驱;奠基人(先駆者。創始者) 
\\	(基督教)上帝(〈キリスト教〉イエス。マリアを母とする。耶蘇。クリスト)	ちち	
\\	書く	【他动词・五段/一类】 
\\	记文字、记号或线条等。写,画(文字や記号・線などをしるす) 字を~/写字 
\\	做文章或创作作品。写(文章や作品をつくる)	かく	
\\	ペン	【名词】 
\\	【英】 
\\	钢笔、圆珠笔的总称
\\	万年筆・ボール-ペンなど、筆記具の総称) 
\\	【英】 
\\	文章,文笔(文章を書くこと。文筆活動) 
\\	【英】
\\	钢笔(インク・墨などをつけて字・絵などを書く先の尖った筆記用具)		
\\	小声	【名词】 小声,低声(小さな低い声) ~でささやく/小声喳喳话;窃窃私语 ~で話す/小声说话	こごえ	
\\	貸す	【他动词・五段/一类】 
\\	租,借,借出(自分の物を一時的に使わせる) 
\\	提供帮助(相手の非力を見かねて、必要とする協力、援助を行う)	かす	
\\	教える	【他动词・一段/二类】 
\\	教,教授(学問や技芸などを身につけるに導く) 日本語を〜/教日语 
\\	指点,指教(知っていることを告げ示す) 
\\	教训,教诲(注意を与えて導く)	おしえる	
\\	土産	【名词】 
\\	特产,土产(旅先等から持って帰るその土地の産物。つと) 
\\	(赠给人的)礼品,礼物(人を訪問する際持っていく贈り物。手土産)	みやげ どさん	
\\	漫画	【名词】 漫画;连环画;动画片(大胆に省略・誇張して描き、笑いを誘いながら風刺や批評をこめた絵。また、絵あるいは絵と台詞によって表現される物語)	まんが	
\\	ウォークマン	【名词】 【英】
\\	单放机,随身听(ヘッドホンまたはイヤホンで聴く、携帯用小型カセットテープのステレオ再生装置の商標名)		
\\	鳥	【名词】 
\\	鸟,禽(鳥類の総称) 
\\	禽肉,禽类的肉,尤指鸡肉(鳥の肉。特に、鶏の肉)	とり	
\\	飛ぶ	【自动词・五段/一类】 
\\	飞,飞翔,飞行。吹起,刮跑,飘飞,飘落,飞散。溅,迸(大地から離れ空に上がる。高く舞いあがる。空中を移動する) 
\\	赶快跑,快跑,飞跑(速く走る) 
\\	越过,跳过;[离题]很远,远离(順序を経ないで、間をこして先へ進む) 
\\	跑到很远的地方,逃往远方(行方をくらまして遠くへ逃亡する) 
\\	传播,传开(うわさや命命などが、すみやかに伝わる) 
\\	断,化为乌有,尽(切られてなくなる。切れてなくなる)	とぶ	
\\	通る	【自动词・五段/一类】 
\\	走过,通过(行き過ぎる) 
\\	来往,通行(行き来する) 
\\	通,开通(通じる) 
\\	透过,穿透,渗透(貫く) 
\\	(声音)响亮(声や音などが隅々まで届く) 
\\	通过(会议);(考试)合格(認められる。合格する) 
\\	知名,闻名(知れわたる) 
\\	了解,明白,理解(よく分かる) 
\\	进(屋),进入(部屋に入る)	とおる	
\\	富士山	【名词】 富士山。横跨静冈、山梨两县的圆锥状成层火山,日本最高峰,海拔3776
\\	。自1707年喷发形成宝永山之后停止活动,火山口直径约800
\\	。现在汽车可开到半山腰	ふじさん ふじやま	
\\	クラシック	【英】
\\	古典(古典) 
\\	古典的(古典的) 〜な文体/古典的文体 
\\	古典音乐(西洋の古典音楽)		
\\	飛行機	【名词】 飞机(空中を飛ぶ乗り物)	ひこうき	
\\	船	【名词】 
\\	船[小型の]舟『書』 
\\	〔水などを入れるもの〕槽,盆 
\\	〔魚を盛る器〕	ふね	舟
\\	速い	【形容词/イ形容词】 快的,迅速的,早的,急的,敏捷的,灵活的(速やかである。ある動作を完了するのに要する時間が短い)	はやい	
\\	新幹線	【名】 新干线(交通工具)	しんかんせん	
\\	低い	【形容词/い形容词】 
\\	低;矮(高さが低い) 
\\	低微,低贱,微贱(身分などが低い) 
\\	低,小(音量が少ない。また、音域が下である) 
\\	低的,小的(数値、数量が小さい。また、比率が小さい)	ひくい	
\\	名古屋	【日本地名】 名古屋(愛知県西部(尾張地方)の政令指定都市で、同県の県庁所在地である。全16区から構成される)	なごや	
\\	横浜	【名词】 横浜。日本地名(神奈川県東部の重工業都市。県庁所在地。政令指定都市の一つ。東京湾に面し、1859年(安政6)の開港以来生糸の輸出港として急激に発展。現在、全国一の国際貿易港。人口358万)	よこはま	
\\	列車	【名词】 列车,火车(鉄道の本線路を運転する車両。または、車両の連なり)	れっしゃ	
\\	梨	【名词】 梨树;梨(バラ科の落葉高木。山梨の改良品種で、日本では古くから果樹として栽培)	なし	
\\	メロン	【名词】 【英】
\\	〈植〉甜瓜,白兰瓜,华莱氏瓜
\\	。香瓜,蜜瓜(マスクメロンのこと。また、マスクメロンに変種のマクワウリやアメリカ産の品種を交配して作られている新品種の総称)		
\\	ワイン	【名词】 【英】
\\	葡萄酒(葡萄酒のこと)		
\\	ドイツ	【名词】 【荷】
\\	德意志,德国		独逸
\\	ビール	【葡】
\\	啤酒		
\\	タイ	【名词】 
\\	【英】
\\	泰国。位于东南亚的君主立宪制国家(東南アジアに位置する立憲君主制国家) 
\\	【英】
\\	比赛等同分,打成平手(試合などで同点になったときのタイ等にも使われる) 
\\	【英】
\\	领带,领结(一般的に結ぶことをタイと言い、首に結ぶのをネクタイに使われる)		
\\	蜜柑	【名词】 橘子;柑橘	みかん	
\\	見学	【名词・他动·三类】 参观学习(実際に見て知識を身につけること)	けんがく	
\\	自動車	【名词】 汽车(原動機を装置し、その動力によって車輪を回転し、軌条によらずに道路上を走る車)	じどうしゃ	
\\	如何	【副词】 
\\	如何(どんなに) ご意見は〜です/您的意见怎么样呢 
\\	为什么(どうして…か) 
\\	如何(「ーあらむ」の形で危惧もしくは非難の意を表す) 
\\	怎么样(どれほど…ことか)(「どうですか」と呼びかけすすめる語。)	いかが	
\\	ジュース	【英】
\\	汁,汁液(汁。液汁。特に、果物・野菜のしぼり汁。果汁。また、広義にはそれを加工した飲料)		
\\	奈良	【日本地名·姓氏】 
\\	奈良县,近畿地区中部的县,占昔日大和国全域,县厅所在地为奈良市(近畿地方中部の県。かつての大和国全域を占める。県庁の所在地、奈良市) 
\\	奈良市,奈良县北部的市,县厅所在地。京都府南部の市(奈良県北部にある市。県庁所在地。710 年平城京が置かれ、784 年まで 
\\	年間日本の首都として栄えた。以来、北都(京都)に対して南都と呼ばれ、東大寺・興福寺・春日大社の門前町として発達。天平文化に代表される多くの文化遺産を残す) 
\\	日本姓氏之一(姓氏の一)	なら	
\\	大阪	【名】 大阪(地名)	おおさか	
\\	試験	【名・他动词・サ变/三类】 
\\	试验,检验,化验(物の能力をためすこと) 
\\	考试,测验(人の能力をためすこと) 入学~/入学考试	しけん	
\\	橋	【名词】 
\\	桥,桥梁,天桥(交通路の上に架設して通路とするもの。橋梁) 
\\	中介。媒介(転じて、掛け渡すもの。仲介、媒介) 【名词】 
\\	脑桥(脳の一部) 
\\	桥(交通路の上に架設して通路とするもの)	はし きょう	
\\	渡る	【自动词・五段/一类】 
\\	渡过,过;去(远处)(間を隔てている物の一方から他方へ越えていく) 
\\	度日,生活,生存(世の中を生きていく。暮らす) 
\\	转让(钱物、权利、地位等),转手(他の人の所有物となる) 
\\	到达(各个角落),普及(配られて、ある範囲全体に届く。行き渡る) 
\\	〈接动词连用形后〉表示做得完全、全面之意(動詞の連用形に付いて、一面に、また、広く…する。また、…しつづける) 【古语】 
\\	〈含有尊敬之意〉有;在;「あり」、「居り」的敬语 
\\	〈接动词连用形后〉表示涉及时间和空间的广阔范围之意(動詞の連用形に付いて、その作用・状態が広範囲にわたって、行われる)	わたる	
\\	真っ直ぐ	【名・形容动词/な形容词】 
\\	笔直(直線的で少しも曲がっていないこと) 
\\	一直;照直;不绕弯;直接(途中で寄り道せずに行くこと) 
\\	正直;坦率;耿直(正直で偽ったり、ごまかしたりしないこと)	まっすぐ	
\\	サッカー	【名词】 【英】
\\	足球(フットボールの一種)		
\\	将来	【名・他动词・サ变/三类】 
\\	将来,未来,前途(これから先の時) 
\\	拿来,带来;招致(持ってくること)	しょうらい	
\\	卒業	【名・他动词・サ变/三类】 
\\	毕业(事業を完了する意味) 
\\	体验过,过时,过了阶段(前の段階を終了すること)	そつぎょう	
\\	そろそろ	【副词】 
\\	慢慢地;徐徐地(ゆっくり) 
\\	渐渐;逐渐(徐々に) 〜涼しくなってきた/渐渐地凉快起来了 
\\	就要,快要;不久(その時が間近に迫っているさま。まもなく) 〜12時だ/快要十二点了		
\\	貿易	【名词·自动·三类】 (进出口)贸易(外国と商品の売買をすること)	ぼうえき	
\\	関係	【名词】 
\\	关系;关联,联系;牵连;涉及 〜で就職する 
\\	亲属关系,亲戚裙带关系	かんけい	
\\	日系会社	【名】 日资公司	にっけいがいしゃ	
\\	面接	【名・自动词・サ变/三类】 接见,面试(能力や性格などを知るために、直接にその人と会うこと)	めんせつ	
\\	受ける	【自他・一段/二类】 
\\	承接(受けとめる) 
\\	承蒙,受到;接到;得到;奉(得る) 
\\	遭受(被る) 
\\	接受,答应,承认(応じる) 
\\	应,考(自分から進んで、あることをしてもらう) 
\\	继承,接续(受けつぐ) 
\\	稟承,享受(さずかる) 
\\	认为,理解(解する) 
\\	受欢迎(人気を得る)	うける	
\\	大学院	【名词】 大学研究生院。在日本,硕士研究生为2年,博士研究生为3年(大学の卒業生を対象にし、専門の学問・技術を授ける教育機関。修士課程と博士課程がある)	だいがくいん	
\\	復習	【名・他动词・サ变/三类】 复习(習ったことを、もう一度自分で勉強すること)	ふくしゅう	
\\	資料	【名词】 资料(もとになる材料)	しりょう	
\\	欲しい	【形容词/い形容词】 
\\	想要,要(自分の手に入れたい) 
\\	希望,愿意(望ましい)	ほしい	
\\	研究室	【名词】 研究室(研究をするとき使うルームです)	けんきゅうしつ	
\\	助かる	【自动词・五段/一类】 
\\	得救,脱险(死・災い・罪・苦しみなどをまぬかれる) 
\\	省力,省事(労力や負担などを省くことができる)	たすかる	
\\	電気屋	【名词】 电器商店,民用电器商店,电料行(電気商店)	でんきや	
\\	借りる	【他动词・一段/二类】 
\\	借用;暂用(他人のものを、後で返す約束で使う。借用する) 
\\	得到帮助(他人の助力・協力を受ける) 
\\	代用,取代(仮にほかのものをある目的に使う。代用する。亦作「籍りる」、「仮りる」)	かりる	
\\	駅前	【名词】 站前,车站附近(駅の正面に広がる地域)	えきまえ	
\\	降りる	【自动词・一段/二类】 
\\	下,下来(高い所から低い方へと移って、ある位置・場所に着く。上から下へ移動する) 
\\	下(走出交通工具或车站等)(乗っていた乗り物から出る) 
\\	下,降。霜、露水等出现在地上或叶片上(霧や霜などが地上・空中などに生じる) 
\\	退位,辞职(官位・役職を退く。職を辞める)	おりる	下りる
\\	出る	【自动词・一段/二类】 
\\	出去,从里向外走(中から外へ移動する) 
\\	离开,离去(基準となる場所から離れる) 
\\	前往(あるところを離れて、別の場所へ行く) 
\\	出发,交通工具起程(乗り物が出発する。または運行される) 
\\	毕业(それまで社会的に所属していたところから去る) 
\\	突出,溢出,漏出(外の方へ出っぱる。突き出る) 
\\	通往,通到(通る、または通じるの意味) 
\\	显现,出现(物が移動して中から外,または人の目に見える所に現れる) 
\\	找到(丢失的东西)(隠れていたり、どこに行ったか分からなくなっていたものが見つかる) 
\\	出席,出勤(集まりなどに出席、参加する) 
\\	参加,出场,演出(新たな社会的な活動に乗り出す) 
\\	登载,刊登,出版(広く世の中全体に向かって、発表される) 
\\	(在小说或剧本中)出场(活字になったものや話などの中に登場する) 
\\	上市,在市场上出现(多くの人々が見たり聞いたりするものの中に登場する) 
\\	卖出,销出,畅销(商品や金が他所に渡る) 
\\	被给予,得到(飲食物や金品が供される。与えられる) 
\\	发生(现象或疾病)(ある現象や事態が発生する。起きる) 
\\	超出(范围或标准)(ある基準を超す) 
\\	(液体)涌出,流出(液体などが迸る、または湧く) 
\\	(针对对方)采取…态度(人の感情などが外に表れる) 
\\	增加(数量、价值等)(数、値段の増加) 
\\	出产(物が産出される) 
\\	出自,继承系统(…の系統を引く。…に由来する) 
\\	得到(结果)(ある物事が明らかな形で示される)	でる	
\\	来年	【名词】 来年,明年(今年の次の年。明年)	らいねん	
\\	デジタル	【名词】 【英】
\\	数字(的),计数(的)用离散的数字,文字等的信号来表示物质,系统等的状态(物質・システムなどの状態を、離散的な数字・文字などの信号によって表現すること)		
\\	車	【名词】 
\\	轮,车轮(回転する輪) 
\\	车,小汽车(自動車・人力車) 【接头/接尾】 
\\	车,辆(車1台の分量) 
\\	车(車両)	くるま しゃ	
\\	掃除	【名词・他动·三类】 
\\	打扫,扫除(掃いたりふいたりして、ごみや汚れをなくしてきれいにすること) 
\\	淘厕所(便所の糞尿を汲み取ること) 
\\	清除(害悪を除き去ること)	そうじ	
\\	銀座	【名词】 
\\	银币铸造厂(江戸幕府の銀貨鋳造所) 
\\	银座(東京都中央区にある地名) 
\\	繁华街(繁華街)	ぎんざ	
\\	映画館	【名词】 电影院(映画を上映して、観客に見せる建物)	えいがかん	
\\	買い物	【名词/自动·三类】 
\\	买东西 
\\	要买的东西;买到的东西	かいもの	
\\	見物	【名・他动词・サ变/三类】 游览,参观(観光などをする);旁观;观众;旁观者;看热闹的(見物人) 【名词】 值得看(的东西)	けんぶつ みもの	
\\	工場	【名词】 工厂;[手工業の]工场,作坊;[作業現場]车间(機械などを使って、ものをつくったり加工したりするところ)	こうじょう こうば	
\\	実習	【名・他动词・サ变/三类】 实习,见习(実地または実物について学習すること)	じっしゅう	
\\	ステレオ	【名词】 
\\	【英】
\\	立体声音响器材;立体声(聴取者に音源の位置が分かり、立体感をえられるようにした音響装置) 
\\	立体声设备,实感音响设备〔モノラルに対し〕 
\\	〔立体〕立体。立体声		
\\	ラジカセ	【名词】 【英】
\\	收录两用机,收录音机(「ラジオカセット」の略。ラジオとカセットテープレコーダーを一つにまとめたもの)		
\\	喉	【名词】 
\\	咽喉,喉咙,嗓子(口の奥の食道、気管の通ずるところ) 
\\	嗓音,歌声(歌う声) 
\\	〔急所〕要害,致命处(急所)	のど	
\\	渇く	【自动词・五段/一类】 
\\	渴,干渴(のどがからからになって、水分が欲しくなる) 
\\	渴望,内心的要求(強くほしがる)	かわく	
\\	遊ぶ	【自动词・五段/一类】 
\\	玩,游玩(仕事や勉強をせず,遊戯などをして楽しく時を過ごす) 
\\	游荡,不务正业(仕事をしていない状態にある) 
\\	游历,游学(学芸を修めたり、見聞を広めたりするために、他郷へ行く)	あそぶ	
\\	水	【名词】 
\\	水(水素と酸素とからなる化合物。人間の日常生活や産業などのあらゆる面で利用される) 
\\	凉水,冷水(冷水) 
\\	洪水(洪水。大水) 
\\	(摔跤长时间不分胜负时)暂时休息(相撲では、水入りのこと)	みず すい	
\\	一杯	【名词】 多;很多(たくさん) 【名词・副词】 
\\	一碗;一杯;一盅(容器1つ分) 
\\	满满地;东西盛满状(満ちている) 
\\	尽量;最大限度地(全部) 
\\	一只(船やイカなどの)	いっぱい	
\\	金	【名词】 
\\	金属,铁(金属。金・銀・銅・鉄など) 
\\	钱,钱财;银子,票子,钞票(金銭) 
\\	金属(金属。金・銀・銅・鉄など) 
\\	钱财(金銭。おかね) 【名词】 
\\	金,黄金,金子 
\\	金。表示金的纯度的单位。24金为纯金 
\\	钱(金銭) 
\\	金色,金黄色(こがね色) 
\\	金将(将棋の駒の一) 
\\	星期五,礼拜五	かね	
\\	引き出す	【他动词・五段/一类】 
\\	拉出,抽出(中にあるものを引っ張って外へ出す) 
\\	调动。发挥(隠れているものを取り出してわかるようにする) 才能を~/发挥才能 
\\	提取(預金を下ろす) 貯金を~/提取存款 
\\	使…出资(資本などを取り出す)	ひきだす	
\\	コンピューター	【名词】 
\\	电脑,计算机(電子計算機)		
\\	服	【名词】 
\\	衣服,西服(身につけるもの。着物) 
\\	服,付,回(飲む回数)	ふく	
\\	宗教	【名词】 宗教(神または何らかの超越的絶対者、あるいは卑俗なものから分離され禁忌された神聖なものに関する信仰・行事。また、それらの連関的体系。帰依者は精神的共同社会(教団)を営む。アニミズム・自然崇拝・トーテミズムなどの原始宗教、特定の民族が信仰する民族宗教、世界的宗教すなわち仏教・キリスト教・イスラム教など、多種多様。多くは教祖・経典・教義・典礼などを何らかの形でもつ)	しゅうきょう	
\\	顔	【名词】 
\\	脸;面孔(顔だち。目・鼻・口がある、頭部の前面) 
\\	表情,面色,神色,样子(表情) 
\\	面子,脸面(面目。体面) 
\\	人(成員としての個々の人。顔ぶれ)【接头/接尾】 脸形,神色,神气,样子,态度(名詞や動詞の連用形などに付いて、そのような表情、またはそのような様子であることを表す) 【名词】 容颜(顔つき。顔のさま。かおばせ)	かお がお かんばせ	
\\	洗う	【他动词・五段/一类】 
\\	洗(水でよごれを落とす) 
\\	洗涤,净化,一笔勾销(心のわだかまりなどを流し去る) 
\\	(彻底)调查,查(清),查明(明らかにするため調べる) 
\\	冲刷。波浪来回拍击岸边(波が岸に寄せては返す)	あらう	
\\	朝食	【名词】 早饭(朝ご飯,朝の食事)	ちょうしょく	
\\	取る	【他动・一类】 
\\	拿;取,执,握,攥;把住,抓住(手の中におさめる。手に持つ) 
\\	操作,操纵(うまく動かして、事を行う。処理する) 
\\	坚持(いろいろな方法で自分のものにする) 
\\	夺取,强夺,强占,吞并〔支配・占有する〕	とる	
\\	戻る	【自动词・五段/一类】 
\\	返回,回到〔もとの所へかえる〕 
\\	倒退;折回(引き返す) 
\\	回家,回来〔帰る)。 
\\	归还;退回〔返ってくる〕	もどる	
\\	それから	【接续】 
\\	还有,再加上(そして。また) 
\\	其次,接着,以后;而且(その後) 
\\	请谈下去,往下讲,后来又怎样(話を促す)		
\\	もしもし	【感叹词】 
\\	喂。用于叫住对方(人に呼びかけるときにいう語) 
\\	喂。用于电话中,呼唤对方,或是回答(電話で、相手に呼びかけ、また、答える語)		
\\	久し振り	【名词/形动】 (隔了)好久,许久(前にそのことを経験してから、再び同じことになるまでに長い日数のあったこと。また、そのさま) やあ,〜ですね/啊呀,好久没见了	ひさしぶり	
\\	最近	【名词・副词】 最近,近来(近ごろ。近来)	さいきん	
\\	忙しい	【形容词/い形容词】 
\\	忙,忙碌(用事が多い) 
\\	急急忙忙,忙叨叨的 【形容词/い形容词】 
\\	忙,忙碌,忙合 
\\	匆匆忙忙,急忙,焦急(心がせく)	いそがしい せわしい	
\\	卒論	【名词】 毕业论文(卒業論文の略。卒業に際し、学生が特に研究した問題について提出する論文)	そつろん	
\\	事	【名词】 
\\	事,事情,事实(生じた事柄。出来事。事態。事件) 
\\	事务,工作(仕事。用件) 
\\	大事件,变故,事端(重大事。大変な事態)	こと	
\\	就職	【名词・自动·三类】 就职,就业,找到工作(職を得て勤めること)	しゅうしょく	
\\	所で	【接续】 可是;有时,转移话题时用(話題を変えて、話し始める時に用いる。それはそうと) 〜、彼女は最近元気ですか/那个,她最近身体好吗? 【接续助词】 〈接助动词「た」的终止形后〉即使,即便;纵令,以不起作用或毫无结果的事情为前提,表示假定的转折条件(逆接条件を表す。前の事柄が無益なものに、また、好ましくない状態になりそうだという予想のもとで、後に結びつける) もうこれ以上話した〜、無駄だ/即便再多说也是白废	ところで	
\\	実は	【副词】 说真的,老实说,事实是,说实在的,其实,原来(事実を言えば。本当のところは。打ち明けて言うと) ~わたしにもよくわからない/说真的,我也不太明白	じつは	
\\	今度	【名词】 
\\	这回,这次,此次,最近(何度か行われることのうち,現在行われている,あるいは最近行われたばかりのもの) 
\\	下次,下回(この次。次回)	こんど	
\\	同窓会	【名词】 同学会(同じ学校の出身者によって組織されている団体。また、その会合)	どうそうかい	
\\	相談	【名・他动词・サ变/三类】 
\\	商量;协商,协议,磋商;商谈;商定,一致的意见〔話し合い。協議する。口頭の商谈〕 
\\	征求意见,请教;咨询(意見を聞くこと) 
\\	提出意见,建议;提议(意見を出すこと)	そうだん	
\\	来週	【名词】 下周,下星期(この次の週)	らいしゅう	
\\	飲み物	【名】 饮料	のみもの	
\\	オレンジジュース	【名词】 【英】
\\	橙汁		
\\	少々	【名词・副词】 少许,一点,一些,稍微,稍稍,些微(すくないさま。わずかなこと。(副詞的に)少し。ちょっと)	しょうしょう	
\\	待つ	【他动词・五段/一类】 
\\	等,等待,等候(物事・人・時が来るのを予期し、願い望みながら、それまでの時間を過ごす。また、用意して備える) 
\\	待,伺机,静观(相手の反応や態度が分かるまで静観する) 
\\	期待,指望,期望(それを頼りにして任せる。望みを託する。期待する) 
\\	延期(期限をのばす)	まつ	
\\	ジョギング	【名词】 【英】
\\	慢跑		
\\	畏まる	【自动词・五段/一类】 
\\	毕恭毕敬『成』,拘谨;正襟危坐『成』,恭恭敬敬地坐着(恐れつつしむ;正座する) 神妙に~/正襟危坐 
\\	知道了(引き受ける)	かしこまる	
\\	急ぐ	【自动词・五段/一类】 快,急,加快,赶紧,着急(早く目的を達するように行動する) ~旅ではない/无须着急的旅程	いそぐ	
\\	死ぬ	【自动词・五段/一类】 
\\	死,死亡 
\\	死,死板 
\\	死,不起作用	しぬ	
\\	呼ぶ	【他动词・五段/一类】 
\\	喊,叫(声を出して、相手の名前などを言う) 
\\	叫来(大きな声を出して注意を引く) 
\\	请来(声をかけてこちらへ来させる。頼んで来てもらう) 
\\	邀请((行事・催し物などに)客として招く) 
\\	掀起,引起(引き寄せる。集める。さそう) 
\\	叫做,称作,称为(名づける。となえる。称する)	よぶ	喚ぶ
\\	乗る	【自动词・五段/一类】 
\\	乘坐;骑;坐,上,搭乘 
\\	登,上(物の上にあがる) 
\\	参与,参加(加わる) 
\\	上当。受骗(計略にかかる) 
\\	附着(十分によくつく) 
\\	乘势,乘机(勢いよく進む) 
\\	合拍,配合(調子が合う) 
\\	传导,传播(伝えられる)	のる	
\\	乗り換える	【他动词・一段/二类】 
\\	换车,倒车〔船〕,换乘,改乘(今までの乗り物を降りて、他の乗り物に乗る) 別の船に~/改坐别的船 
\\	倒换(株式・債券などの手持ちのものを売り、他の銘柄に買い替える) 薬品株に~/倒换成药品(行业的)股票 
\\	改变,改行(今までのかかわりや立場・考え方などを捨てて、他のものと変える)	のりかえる	
\\	立つ	【自动词・五段/一类】 
\\	立,站 
\\	冒,升,起上 
\\	离开,退 
\\	出发,动身〔出発する〕 
\\	行动起来,奋起(奮起する) 
\\	生,起(風・波などが起こる) 
\\	处于某种位置 
\\	保持,保住 
\\	刺,扎 
\\	引人注目,看得清楚 
\\	激动,激昂(感情が激する。高ぶる) 
\\	传闻,散步(世に知れ渡る) 
\\	成立,确立 
\\	(门、拉门等)关闭 
\\	有用,讲得通 
\\	擅长 
\\	开,开始 
\\	季节到来 
\\	谋生,生活(生活が出来る) 
\\	(除法运算等)得,商数 
\\	〈接动词连用形后〉表示加强语气	たつ	
\\	話す	【他动词・五段/一类】 
\\	说,讲(言葉で伝えて広める) 
\\	谈话,商量(相談する) 
\\	说明,告诉(伝える) 
\\	说(某种语言)	はなす	
\\	太極拳	【名词】 
\\	太极拳(中国の宋代に始まる拳法) 〜をする/打太极拳 
\\	『参考』中国の拳術の一種で、陳式・楊式・呉式・孫式などの流派がある	たいきょくけん	
\\	カレーライス	【名词】 
\\	【英】
\\	咖哩饭。把炒好的肉、蔬菜拌上咖喱粉和面粉煮成汁浇在米饭上的食物(本来、インド料理の一。肉・野菜をいためたものにカレー粉と小麦粉を混ぜて煮た汁を飯にかけたもの。ライス-カレー。カレー〔ライス-カレーが古い語形という〕) 
\\	(初高中生流行语)手被晒黑以至于和指甲的颜色分明(手のひらと甲の色がくっきりと分かれるほど日焼けした手)		
\\	構う	【自动词・五段/一类】 
\\	管,理会,干预(関心をもつ。気にかける) 
\\	在意,照料(相手になる。世話をする) 
\\	表示许可之意(…してよい) 【他动词・五段/一类】 逗,调戏,取笑(からかったり、傷つけたりする。相手にしてふざける)	かまう	
\\	吸う	【他动词・五段/一类】 
\\	吸,吸入(気体や液体を鼻や口を通して体内へ引き入れる) きれいな空気を〜/吸新鲜空气 
\\	吮,吮吸,嘬,啜,喝(水気を他から吸収する) スープを〜/喝汤 
\\	吸收(吸収する) スポンジはよく水を〜/海绵很能吸水	すう	
\\	手伝う	【他动词・五段/一类】 帮助,协助;另因,某种原因的基础上还有别的原因在起作用(他人の仕事を助ける。手助けをする) 【自动词・五段/一类】 另因,某种原因的基础上还有别的原因在起作用(ある原因の上に、更に付け加えて作用する)	てつだう	
\\	覚える	【他动词・一段/二类】 
\\	感觉,感到,觉得(体や心で感じる。…と感じる。…と思われる) 
\\	学会,掌握(技術を身につける。習得する) 
\\	不忘,记住,记忆(記憶に留めて忘れないでいる。頭に入れる。記憶する)	おぼえる	
\\	シャワー	【名词】 
\\	【英】
\\	淋浴器(如雨露のような噴水口から水または湯の出る装置) 
\\	级联簇射		
\\	浴びる	【他动词・一段/二类】 
\\	浇,淋,浴(水をかぶる) 
\\	照,晒,满身受到(太陽やほこりなどを受ける) 
\\	受,蒙,遭(大勢の人から非難・賞賛や質問の言葉を受ける蒙る)	あびる	
\\	教師	【名词】 
\\	教师,教员,老师(教員) 
\\	传教士(伝道士)	きょうし	
\\	厳しい	【形容词/い形容词】 
\\	严;严格;严厉;严峻;严肃(厳重である。おごそかである。激しく容赦ない) 
\\	严重,厉害,很甚(物事の状態や人の表情などが緊張している。はなはだしい) 
\\	严酷,残酷,毫不留情(むごい)	きびしい	
\\	直ぐ	【形容动词/な形容词】 
\\	笔直,径直(まっすぐである) 
\\	耿直,率直,正直(素直である)【副词】 
\\	立即,马上(直ちに) 
\\	(距离,关系)非常靠近,很近(距離がごく近い)	すぐ	
\\	結婚	【自动词・サ变/三类】 结婚(正式に夫婦関係を結ぶこと)	けっこん	
\\	掛かる	【自动词・五段/一类】 
\\	垂挂,悬挂,挂上(吊りさがる、ぶら下がっている) 
\\	复盖(一面に覆う) 
\\	陷入,落在…的手中(わなや計略に陥る) 
\\	着手,从事(とりかかる、着手する) 
\\	正在进行(ちょうどしている) 
\\	眼看就要,即将(…しそうになる) 
\\	花费时间,花费费用,花费劳动力或体力等(時間、費用、労力などが必要とされる。費やされる。要する)	かかる	
\\	緑	【名词】 
\\	绿色,翠绿(色の名。光の三原色の一) 
\\	翠绿,树的嫩芽,松树的嫩叶(緑色の木や草。新緑をいうことが多い)	みどり	
\\	空港	【名词】 机场(エアポート)	くうこう	
\\	時代	【名词】 时代;当代,现代;朝代(区切られた、ひとまとまりの長い期間。また、その当時。当代。現代)	じだい	
\\	高校	【名词】 高中(「高等学校」の略)	こうこう	
\\	迎える	【他动词・一段/二类】 
\\	迎,迎接(来るのを待ちうける);欢迎(歓待する);接待(接待する) 
\\	请,聘请,接(招く) 
\\	娶;招(めとる) 
\\	迎合(ごきげんをとる) 
\\	迎击,应战,对敌(待ちうけて撃つ) 
\\	来到,遇到(時期になる)	むかえる	
\\	北京大学	【名词】 北京大学(北京市にある国立総合大学。清朝末の1898年京師大学堂として創設され、1912年北京大学校と改称し、中国近代化運動の中心となった。1952年、中華人民共和国政府の大学再編成により、文理系の総合大学となった)	ぺきんだいがく	
\\	専攻	【名・他动词・サ变/三类】 专攻,专修(ある学問分野を専門的に修めきわめること)	せんこう	
\\	経済学	【名词】 
\\	经济学(経済現象を研究する学問。旧称、理財学) 
\\	经济学。研究人类社会的经济现象,特别是研究物质财富、服务的生产、交换、消费的规律的学问	けいざいがく	
\\	活動	【自动词・サ变/三类】 活动;工作	かつどう	
\\	向かう	【自动词・五段/一类】 
\\	反抗,抗拒(抵抗する;逆らう) 
\\	趋向,接近,临近(相当する) 
\\	出门,前往 
\\	相对,面对着,朝着,对着	むかう	
\\	海	【名词】 
\\	海,海洋(地球上の陸地以外の部分で、塩水をたたえた所。地球表面積の約7割を占め、その面積3億6000万平方キロメートル。平均深度3800メートル) 
\\	茫茫一片(あたり一面にひろがったもの、また、無数に多く集まっているさまにたとえていう) 
\\	砚池(硯のほりくぼめて水を貯える所)	うみ	
\\	近い	【形容词/い形容词】 
\\	〔距离、时间〕近;接近,靠近,靠;快,将近。 
\\	〔关系〕近;亲近,亲密,密切 
\\	近似;近乎…,近于	ちかい	
\\	オリンピック	【名词】 【英】
\\	奥林匹克运动会,奥运会(国際オリンピック委員会が主催する競技大会。4年に1回開かれる) 国際~大会/国际奥林匹克运动会;奥运会		
\\	御陰	【名词】 
\\	(神佛的)保佑,庇护(神仏の加護) 
\\	帮助,恩惠;托…的福,沾…的光,幸亏…,归功于…;由于…缘故(他人の助力。援助、力添え、恩恵) 
\\	亏得,怪,多亏(ある事や物が原因となって生じた結果)	おかげ	
\\	一段と	【副词】 更加,愈发。特别,例外,特殊。与其他相比有很大的区别(他と比べた場合にはっきりと相違があるさまを表す表現。特別に。格別に) ~美しくなった 
\\	越发美丽了 彼は~また進歩した/他越发进步了	いちだんと	
\\	為る	【自动词・五段/一类】 
\\	变成,变得(今までと違った状態・形に変わる) 
\\	到…时候(ある時分・時期などに至る) 
\\	到达某一数值(ある数値に達する) 
\\	促成,起作用(ある働きをする。作用する)	なる	成る
\\	訪問者	【名】 访问者,来访者	ほうもんしゃ	
\\	国際的	【形容动词/な形容词】 国际性的(諸国家に関係を有するさま。インターナショナル)	こくさいてき	
\\	都市	【名词】 都市,城市(みやこ)	とし	
\\	やっぱり	【副词】 
\\	依然,仍然,与从前一样。(依然として) 今も〜あの店へ行きますか。/(你)现在还去那家店吗? 
\\	同样,也 私たちも〜分からない。/我们也不明白 
\\	果然 〜思ったとおりだ。/果然跟想的一样。 
\\	到底还是,毕竟还是,无论怎么看		
\\	港町	【名词】 商埠,港口城市(港を中心として発達した町)	みなとまち	
\\	預ける	【他动词・一段/二类】 
\\	寄存;存放;将某人或钱物放在别人身边,请其代为照顾或保管 印鑑を君に~/把印鉴交给你掌管2.委托,托付(他人),交给;委以管理和经营的责任;把纠纷交给别人解决或把胜负交给别人决定 喧嘩を~/托人排解纠纷3.(把身体)倚靠在…上 壁に身を~/把身体靠在墙上	あずける	
\\	出掛ける	【自动词・一段/二类】 
\\	出去,出门,走,到…去(外出する) 
\\	要出去,刚要走(外出しようとする) 
\\	刚要…(出そうになる)	でかける	
\\	海辺	【名词】 
\\	海边,海滨(海岸近くの所) 
\\	日本地名,位于东京都江东区	うみべ かいへん	
\\	気持ち	【名词】 
\\	心情,感受,情绪(物事にたいして感じる好き・嫌い、快・不快などの心の状態。感情) 
\\	感情;感觉
\\	心の持ち方。考え方) 
\\	身体状况(体の具合)【副词】 稍微,稍稍	きもち	
\\	字	【名词】 
\\	字,文字(言語を表記するのに用いる符号。特に、漢字。もじ) 
\\	字体,笔迹(筆跡) 【名词】 
\\	字,别名,别号 
\\	绰号(他人が呼びならわした本名以外の名。あだな) 
\\	镇村区划名(町や村の中の一区画) 【名词】 字,闾(町村内の区画の名。大字と小字とがあり、普通は後者を単に字という)	じ あざな あざ	
\\	履歴書	【名词】 履历书,履历表。简历~を出す/提出履历书	りれきしょ	
\\	作る	【他动词・五段/一类】 
\\	作,造,制,将材料加工成其他东西(形状) 
\\	创造,发明,制造,生育 
\\	制定,建立,组成 
\\	书写,进行(文学艺术)创作 
\\	(辛苦)耕种,栽培 
\\	编造,虚构,假装,装作 
\\	形成,排列,改变位置,配置等,形成某种形状	つくる	
\\	調べる	【他动词・一段/二类】 
\\	调查;查阅(文献などを調査する) 
\\	检查(点検する);查找(さがして);查验(規則などにてらして) 
\\	审问,审讯(問いただす) 
\\	搜查(捜査する) 
\\	调音(調律する) 
\\	奏乐,演奏(乐器)(奏する)	しらべる	
\\	入場券	【名词】 门票;入场券	にゅうじょうけん	
\\	手に入れる	【他动词・一段/二类】 得到,获得(自分の物にする。入手する。) 大金を〜/得到了一大笔钱 【惯用句】 拿到手,取到。弄到手,得到	てにいれる	
\\	並ぶ	【自动词・五段】 
\\	排,排成(行列),列队(二つ以上のもの、二人以上の人が同じ方向に向いて列を作る) 
\\	比得上,伦比,匹敌(程度が同じになる。匹敵する)	ならぶ	
\\	頭	【名词】 头,脑袋;智力,识别力,判断力,思考;头发;上端,顶上;先,最初,开始,开头;首领,头目,首长;人,人数 【名词】 首领,地位高的人;开始;附近,周围;计算大型四蹄动物的量词 【接续词】 
\\	就在…的时候。刚一…(その時、そのとたん) 
\\	表示日、月和时间的开始(月日·時刻の初めの意) 
\\	最优秀、最出色的人(その中の一番の者の意)	あたま・とう・かぶり・ず・こうべ かしら・がしら	
\\	朗らか	【形容动词/ナ形容词】 
\\	〔性格〕明朗;开朗;爽快。〔心情〕愉快,快活;舒畅(心が晴れ晴れとしているさま) 
\\	晴朗。天空无云而晴好(空が曇りなく晴れているさま)	ほがらか	
\\	優しい	【形容词/い形容词】 
\\	安详的,典雅的,温厚的,优美的(上品で美しい。穏やかである) ~物腰/温和的态度 
\\	坦率的,纯真的,诚挚的(素直で大人しい) ~笑顔/清纯的笑脸 
\\	亲切的,富于同情心的(親切で思いやりがある)	やさしい	
\\	丈夫	【形容动词/な形容词】 
\\	健康,壮健(からだが元気) 
\\	坚固,结实(ものがしっかり) 【名词】 男子汉,大丈夫(成年男性,一人前の男)	じょうぶ じょうふ	
\\	段々	【名词】阶梯,楼梯。一样样,一桩桩 【副词】逐渐	だんだん	
\\	豊か	【形容动词/な形容词】 
\\	物质上丰富(好ましい事物が)十分に備わって不足のないさま。豊富) 
\\	富裕,充裕,宽裕(財物が十分あって恵まれているさま。富裕) 
\\	充实,精神上的丰富(精神的にこせこせせず,ゆとりのあるさま。おおらかなさま) 
\\	丰盈,丰满(肉づきがよいさま。豊満) 
\\	足够,十足(基準・限度を超えて十分にあるさま,余りのあるさまを表す)	ゆたか	
\\	肌	【名词】 
\\	皮肤,肌肤(人の体の表皮。皮膚) 
\\	〔表面〕表面(物の表面。きめ) 
\\	〔気性〕气质;[風格]风度(性質。気質。気性) 
\\	木纹(木目)	はだ はだえ	膚
\\	値段	【名词】 价格,价钱(商品につけられている金額。売買の相場・価格。ね。あたい)	ねだん	
\\	一生懸命 一所懸命	【形容动词/な形容词】 拼命地,努力地,一心,专心(物事を命がけですること。必死に) 【名词】 把命悬于一处。日本封建时代,拼命保住自己的领地并赖之以生活(賜った1ヶ所の領地を生命にかけて、生活の頼みとすること。また、その領地)	いっしょうけんめい いっしょけんめい	
\\	アルバイト	【名・自动词・サ变/三类】 
\\	【德】
\\	打工。一边进行学业或从事本职工作一边做副业劳动,亦指这样的劳动者(学業や本業の傍らに賃仕事をすること。内職。また、それをする人。バイト) 
\\	【德】
\\	学术成就,学术成果。学问上的业绩(学問上の業績) 
\\	【德】
\\	临时工,自由职业者(パートタイム労働法に定める短時間労働者に分類される雇用形態。臨時雇い)		
\\	軽い	【形容词/イ形容词】 
\\	轻,轻的,重量小,程度轻(目方が少ない) 
\\	轻松的,愉快的,简单(疲れや悩みがなくて、軽快に感じられる) 
\\	不重要的(重要でない、大切出でない) 
\\	轻率的,冒失的,随便,马虎(軽薄である、軽率だ) 
\\	好对付的,不费力(攻略しやすい) 
\\	程度不深,不重大(程度が小さい)	かるい	
\\	髪	【名词】 
\\	发,头发(頭部に生える毛) 
\\	发型(頭部の毛を結った形)	かみ	
\\	長い	【形容词/い形容词】 
\\	长的,远的(ある点からある点までの空間的な隔たりが大きい) 
\\	长久的(ある時点からある時点までの時間的な隔たりが大きい) 
\\	慢悠悠的。不慌不忙的。悠闲的(精神的に持続力がある。のんびりしている)	ながい	永い
\\	痛い	【形容词/い形容词】 
\\	疼的(痛みを感じるさま) 
\\	痛苦的(苦しく感じる) 
\\	吃不消(感に堪えない)	いたい	
\\	鼻	【名词】 鼻子(哺乳類の顔の中央に隆起し、呼吸・嗅覚を司り、発声を助ける器官) ~がつまる/鼻子不通气	はな	
\\	快適	【形容动词/な形容词】 舒适,舒服(ぐあいがよくて気持のよいこと;気持ちいい) 〜な生活/舒适的生活 〜な船旅/舒适的海上旅行 季節はいまがいちばん〜だ/现在是最舒适的季节	かいてき	
\\	窓	【名词】 
\\	窗户,窗子(採光、または通風を目的で、壁または屋根に開けた開口部) 
\\	〈比喻〉窗口,在联结内外方面发挥作用的事物(比喩的に、外と内をつなぐもの) 
\\	〈源自通风之意〉山脊的低凹部	まど	
\\	開ける	【他动词・一段/二类】 
\\	开,打开(閉じているものをひらく) 
\\	开办;着手(営業などを開始する)【自动词・一段/二类】 
\\	拓宽,扩大;发展(閉じているものが開く) 
\\	宽敞;开阔;敞亮(妨げが無く、広く見渡せる) 
\\	开发,变为易于生活的状态(開化する) 
\\	(心情)舒畅,痛快((閉ざされた心などが)晴れやかになる。爽やかになる) 
\\	开化;开明;通人情(人情に通じ、物分りがいい) 
\\	进步,进展(良い方に向う)【自他・一段/二类】 
\\	(衣襟)张开。敞开(着衣の合せ目がひらく。ひらける) 
\\	开始营业(営業を始める。営業を行う)	あける ひらける はだける	
\\	閉める	【他动词・一段/二类】 关闭,合上;掩上(ひらいていたものをとじる) 戸を〜/关上门 窓をきちんと〜/关严窗户 カーテンを〜/掩上窗帘 引き出しを〜/把抽屉关上 店を〜/关上店门; 上板儿; 下班打烊;歇业 表示把具有内部空间,并且可开可闭的结构物体的敞开部分关闭起来。它不仅仅是直线形的运动,而且可以作圆弧状的运动,将空间开口部分关闭	しめる	
\\	クーラー	【名词】 
\\	【英】
\\	冷气设备;冷却器;冷凝器;冷却剂,空调(冷却器、冷房装置) 
\\	(钓鱼等)携带式冰箱		
\\	点ける	【他动词・一段/二类】 
\\	点火,点燃(点火する。火を放つ) 
\\	打开(電気を働かせて明るい状態にする)電灯を〜/开电灯	つける	
\\	治る	【自动词・五段/一类】 病医好,痊愈(病気や怪我がよくなる)	なおる	直る
\\	冬休み	【名词】 
\\	寒假(正月をはさんで学校が授業を休むこと) 
\\	冬季休假(冬季休暇)	ふゆやすみ	
\\	太い	【形容词/イ形容词】 
\\	〔回りが〕粗;顸『方』(直径が大きい,棒状) 
\\	(胆子)大;无耻;不要脸(ずぶとい,大胆だ,ふてぶてしい) 
\\	肥胖的(肥えている) 
\\	〔声が〕粗(低くて声量が豊かだ)	ふとい	
\\	重い	【形容词/イ形容词】 
\\	沉重,重(めかたが) 
\\	沉重,不舒畅(気持ちが) 
\\	迟钝,懒得动弹(足や行動が) 
\\	重大,重要,严重(程度が)	おもい	
\\	細い	【形容词/い形容词】 
\\	细,纤细(直径が短い) 
\\	〔狭い〕狭窄,窄(幅が狭い) 
\\	微细,低小(声が高音だが、よく響かない) 
\\	微少(盛んでない、弱々しい) 
\\	贫乏,穷(貧しい) 
\\	微弱〔かすか〕 【形容词/イ形容词】 
\\	细小的((中国、四国、九州北部と東北、北海道方言)細い、細かい) 
\\	小的((中国、四国、九州方言)形や量、数などが小さい) 
\\	吝啬的,花钱精打细算((島根方言)けちけちしている。お金に細かい)	ほそい こまい	
\\	勤勉	【名・形容动词/ナ形容词】 勤勉,勤劳(仕事や勉強に一心に励むこと)	きんべん	
\\	今回	【名词】 此次,此番,这回(今度)	こんかい	
\\	問題	【名词】 
\\	问题。试题(問いかけて答えさせる題。解答を要する問い) 
\\	问题。事项。需要处理〔研究,讨论,解决〕的事项〔问题〕(研究・議論して解決すべき事柄) 
\\	问题。麻烦事(面倒な事件) 
\\	引人注目。受世人关注(人々の注目を集めている(集めてしかるべき)こと)	もんだい	
\\	営業部	【名词】 营业部(商業活動を行う部門)	えいぎょうぶ	
\\	部長	【名词】 
\\	部长。在官厅或公司等中管理一个部的事物并监督其部下的职务名称(官庁や会社などで、部の事務を管理し、部下を監督する職) 
\\	部长。负责全权处理俱乐部或球队事务的人(クラブやチームを取り仕切る人)	ぶちょう	
\\	操作	【名词·他动·三类】 
\\	操作。驾驶。操纵(机器等)(機械などを)あやつって働かせること) 
\\	〔金銭の〕设法安排。筹措。窜改(帐簿)(自分の都合のよいようにうまく運用・処理すること)	そうさ	
\\	試着	【名・他动词・サ变/三类】 试衣服,试穿(衣服)(体に合うかどうかを確かめるための服を着てみること) その服を〜してみなさい/把那件衣服穿上试试	しちゃく	
\\	市内	【名词】 市内(市の区域内。まちの中)	しない	
\\	店員	【名词】 店员,售货员(商店に勤めている人。店などで商品販売に従事する人)	てんいん	
\\	運動靴	【名词】 运动鞋(運動用の靴。ズックやビニール製で、底がゴム張りのものが多い)	うんどうぐつ	
\\	勧める	【他动词・一段/二类】 
\\	劝,劝告,劝诱(勧誘する。さそう。また、奨励する。激励する) 
\\	劝,让,进(献ずる) 
\\	同:奨める	すすめる	
\\	いらっしゃいませ	【寒暄语】 欢迎光临;您来了(表示欢迎,适合店家使用)		
\\	若者	【名词】 年轻人,青年(年若い人。わこうど)	わかもの	
\\	人気	【名词】 
\\	人望,人缘,声望,受欢迎,博得好评,吃香『口』,吃得开(世間の受け) 
\\	〈商〉商情,市况,行情(市況、相場模様) 
\\	人缘,声望【名词】 (某地方的)风气 【名词】 
\\	人的气息,人影(人のいる様子、人のけはい、ひとっけ) 
\\	有人的样子(人間らしいこと)	にんき じんき ひとけ/ひとげ	
\\	人気がある	【惯用句】 很受欢迎。有人气的(人気が集まる)	にんきがある	
\\	見せる	【他动词・一段/二类】 
\\	给...看;让...看;表示,显示(人に見させる) 
\\	装做...样给人看;假装(ふりをする) 
\\	使...见识;使...尝受(経験させる) 
\\	表意志的...(強い意志を示す) 
\\	给...看(人が見るようにする) 
\\	让看病(診察させる) 医者に〜/让医生看病	みせる	
\\	履く	【他动词・五段/一类】 穿(足を保護するものを足先につける) 靴を〜/穿鞋	はく	
\\	柔らかい	【形容词/い形容词】 
\\	柔软的;柔和的(固くなくて、ふんわりしている。また、しなやかである) 体が~/身体柔软 
\\	温柔(穏やかなさま) ~日差し/柔和的阳光 
\\	灵活(堅苦しくない。また、融通性に富んでいる) 頭が~/头脑灵活	やわらかい	
\\	新製品	【名词】 新产品。新制品(新しい製品。新たに開発した製品)	しんせいひん	
\\	レジ	【名词】 
\\	【英】
\\	现金出纳机(金銭登録器のこと) 
\\	【英】
\\	现金出纳处,现金出纳员 
\\	【英】
\\	寄存器		
\\	行列	【名·自サ】 
\\	行列,队伍(多人数がきちんと列をつくって並ぶこと。また、その列) デモの〜/示威游行的队伍 
\\	[数]阵,矩阵	ぎょうれつ	
\\	~だけ	【副助词】 
\\	只,仅,表示程度和范围 
\\	越…越… 
\\	正因为;无怪乎 体言~		
\\	星	【名词】 
\\	星星,星形(空の天体。星マーク) 
\\	星标,星徽(軍隊の目印) 
\\	小点,斑点(小さい点) 
\\	靶心,鹄(的の中心) 
\\	目标。(目当て。目標) 
\\	嫌疑犯;犯人(「目星」の略。容疑者) 
\\	〈相撲〉(表示胜负分数的)点;(比赛的)得分((相撲での)得点)	ほし	
\\	きらきら	【副词/自动·三类】 
\\	灿烂,烁烁;晃眼,耀眼,辉耀,闪耀,闪烁(星や宝石などが美しく輝いて見える様子) 太陽の光で露が~する/阳光照得露珠闪闪发光 
\\	闪烁,晃眼(光り輝いているさま) 星が~と輝く/星星闪烁着光辉		
\\	輝く	【自动・一类】 
\\	放光,辉耀,闪耀(光をはなつ) ぴかぴか〜/闪闪发光 
\\	充满,洋溢(あふれる) 
\\	光荣(名誉・名声が光栄だ)显赫(権勢が盛ん)	かがやく	
\\	聞こえる	【自动词・一段/二类】 
\\	听得见,能听见,听到,听得到,能听到(音・声などが耳で感じられる) 呼べば〜距離/一叫就能听见的距离 
\\	听起来觉得…听来似乎是…(受けとられる) 皮肉に〜/听来象讽刺 
\\	闻名,出名,著名(名高い) 音に〜た勇士/闻名的勇士	きこえる	
\\	薬	【名词】 
\\	药,药品(薬剤) 
\\	釉子(うわぐすり) 
\\	火药(火薬) 
\\	益处,好处,教训。(ためになること) 
\\	薬九層倍 卖药一本万利;黄金有价药无价。 
\\	薬にしたくも無い 一点也没有 
\\	薬人を殺さず,薬師(くすし)人を殺す 药不杀人而是庸医杀人。 
\\	薬より養生 (平时注意)养生胜于吃药 
\\	药品;釉药;火药;益处;教训 【名词】 
\\	药(同くすり);麻药,毒品(麻薬) 
\\	药;毒品(的总称)药	くすり やく	
\\	食後	【名词】 食后,饭后(食事の後)	しょくご	
\\	開く	【自动词・五段/一类】 
\\	开,开着,打开 心を~敞开胸怀 
\\	开,开放;敞开(広げる。) 
\\	开始;开张(物事が始まる) 
\\	加大、拉开(数量、距离、价格等的差距)(間隔があく。また、差がつく) 
\\	开朗、开阔 【自动词・五段/一类】 
\\	开,打开;隔墙、罩子等被除掉;关闭着的东西开了 ドアが~门开着 
\\	开始;开张,开业;开演 銀行が~银行开业 
\\	有了差距(差がつく)	ひらく あく	明く
\\	実行	【名・他动词・サ变/三类】 实行,实践,执行(実際に行うこと)	じっこう	
\\	持つ	【自动词・五段/一类】 维持;持续保持(永くその状態を保つ) 【他动词・五段/一类】 
\\	持;拿(手の中に入れて保つ。手に取る) 
\\	带,携带(身につける。携帯する) 
\\	有,拥有(自分のものにする。所有する。支配する) 
\\	抱有,怀有(心の中に抱く) 
\\	负责;承担;担当(受け持つ。担当する) 
\\	负担(费用等)(負担する。自分の任務として引き受ける) 
\\	举行,举办(会议)((会議などを)行う) 
\\	具有(身にそなえる。その中に含む) 
\\	有(与对方的关系)	もつ	
\\	住む	【自动词・五段/一类】 
\\	居住,住(場所をきめて、そこで生活する,居住する) 
\\	栖息,生存(動物が巣を作り、そこで生活する)	すむ	
\\	森	【名词】 
\\	森林(樹木が茂り立つところ) 
\\	神社的树林(特に神社のある地の木立)	もり	
\\	場所	【名词】 
\\	地方,场所,空间,位置,现场,所在地(ところ。場。位置) ~を考えてものを言え/说话要看场合 
\\	席位,座位(いどころ。場席) ~を空ける/腾出席位 
\\	(相扑比赛的)地点和赛期 夏~/夏季相扑比赛	ばしょ	
\\	知る	【他动词・五段/一类】 
\\	知道,知晓,得知(物事の存在・発生などを確かにそうだと認める。認識する) 
\\	懂得,理解,识别(物事の状態・内容・価値などを理解する。把握する) 
\\	认识,熟识(付き合いがある。知り合いである。面識がある) 
\\	记得(忘れずに覚えている。記憶する。また、物事に通じている) 
\\	经历,体验(了解)(経験する。体験して身に付ける) 
\\	关系,相关(そのことに関わって責任を持つ。関知する) 
\\	掌握,会使用(学んで、また、慣れて覚える) 
\\	感觉到,注意到(気付く。感じ取る)	しる	
\\	頑張る	【自动词・五段/一类】 
\\	坚持,拼命努力(努力する);加油,鼓劲(ファイトを出す);不甘落后(負けまいと);不甘示弱(弱味をみせない) 
\\	坚持己见,硬主张;顽固,固执己见(あくまでも主張する) 
\\	不动,不走,不离开(場所を動かない)	がんばる	
\\	留学	【名・自动词・サ变/三类】 留学(他の土地、特に外国に在留して学ぶこと)	りゅうがく	
\\	勤める	【他动词・一段/二类】 
\\	任职,工作,勤务,服务(職に就く。官庁・会社などで職員として働く。勤務する) 
\\	修行,念经(仏道に励む。勤行 する。また、仏事を営む)	つとめる	
\\	砂糖	【名词】 砂糖。白糖,以蔗糖为主要成分的代表性甜味作料。从甘蔗、甜菜等中提炼出来、易溶于水的白色结晶颗粒	さとう	
\\	ゆっくり	【副词】 有余地。宽敞舒适。充裕。舒畅。安静。舒适(時間・空間・気持ちなどに余裕があるさま〕 【副词・自动词・サ变/三类】 慢。不着急。没必要着急(急がないさま。動作・運動が時間をかけて行われるさま〕		
\\	もう一度	【副词】 再次(改めて。再び。また) ~お願いします/请再说一遍	もういちど	
\\	使う	【他动词・五段/一类】 
\\	使,用,使用(ある目的のために物や体の一部を用いる) 
\\	玩弄(手段として術・技を行う) 
\\	用,动,使内心、头脑发挥作用(心・頭を働かせる) 
\\	使,花费,消耗(物品、金钱、时间等)(物・金・時間などを費やす、消費する) 
\\	使唤(人などを働かせる) 
\\	耍弄,让木偶、动物等按照人的意图做动作(人形・動物などを自分の意図どおりに働かす、あやつる) 
\\	与特定的词语搭配,表示实施该行为(特定の語を受けて、その行為をする)	つかう	
\\	センター	【名词】 
\\	【英】
\\	中心地,中心区(中心地) 
\\	中心机构,中心站(中心施設) 
\\	〈野球〉中场手 
\\	中心,中央(まん中)		
\\	病院	【名词】 医院,病院(主に総合的な専門病院)	びょういん	
\\	タオル	【名词】 【英】
\\	毛巾(布の表面に糸を輪の状態に織り出した、やわらかく厚い綿織物)		
\\	シャンプー	【名词】 【英】
\\	洗发精(頭髪を洗うこと)		
\\	既に	【副词】 
\\	以前,以往(以前に。もう) 
\\	已经;早就(早くもそういう状態になっているということを表す) 【古语】 全部;完全(すっかり。全く)	すでに	
\\	座る	【自动词・五段/一类】 
\\	坐;跪坐(膝を折り曲げて、腰をおろす) 
\\	居某地位,占据席位(ある位置・地位などにつく)	すわる	坐る
\\	置く	【他动词・五段/一类】 
\\	放,置,搁(物をある場所にのせる。据える) 
\\	配置,设置(配置する。設置する) 
\\	设立,设置(機関・施設などを設ける) 
\\	留下,让住下(住まわせる) 
\\	放下,留下,丢下(そのものだけは手を付けない(変更を加えない)状態を保たせる) 
\\	掌握,控制(ある状況下に位置させる) 
\\	间隔(間をあける。間を隔てる) 
\\	保持(人や物が続けて一定の場所にある状態を、保たせる) 
\\	计算(計算の道具を操作して、算えたり、占ったりする) 
\\	当,作抵押(預ける) 
\\	定,安设(特定の箇所に集約し、他に力を分散しないようにする) 
\\	放,记,留在心里(時間・空間的な間隔を保たせる) 【自动词・五段/一类】 
\\	下露水或霜(大気中に生じた露や霜や雪が、消えないで、どの場所にある状態を続ける) 
\\	保持某种状态,保持下去(そのまま保持する) 
\\	表示预先做好某种事(前もってする) 
\\	暂时,暂且(しばらく)	おく	
\\	家内	家内,家庭,全家(家のなか。また,家族) 爱人,内人,妻子(自分の妻をいう語)	かない	
\\	女	【名词】 
\\	女人,妇女,女性,女子,女士,女同志(女子。女性。婦人) 
\\	成年女子(成年女子) 
\\	女人的容貌,姿色(容貌。女ぶり) 
\\	情妇,情人,妾(愛人。情婦) 
\\	女仆,女佣人(召使である女子。下女。女中) 【名词】 女儿(むすめ) 【接头/接尾】 女人,女性(おんな。女性の名や号につける語)	おんな じょ	
\\	製品	【名词】 制品;产品;成品(原料に手を加えて作った品物)	せいひん	
\\	パーセント	【名词】 【英】
\\	百分率,百分数,百分之…(100分の幾つであるかを表す語)		
\\	去年	【名词】 去年(今の年の前の年。昨年)	きょねん	
\\	国	【名词】 
\\	国,国家(一つの政府に治められている地域。国家。国土) 
\\	国土,领土(領土) 
\\	家乡,老家,故乡(自分の生まれ育ったところ) 
\\	封地,领地(封建領土) 
\\	地区,地方(地域。地方)	くに	
\\	住所	【名词】 住所。住址,地址(住む場所)	じゅうしょ	
\\	博物館	【名词】 博物馆,博物院	はくぶつかん	
\\	熱	【名词】 
\\	热,热度(熱いこと。あつさ) 
\\	〔体温〕发烧,体温高(病気などで普段より高くなった体温) 
\\	〔意気ごみ〕热情,干劲(一つのことに夢中になって、高ぶった気持ち。また、興奮した状態)	ねつ	
\\	咳	【名词】 咳嗽;清嗓子(のどや気管が刺激を受けたとき,呼気が急激に吐き出される現象)	せき しわぶき	
\\	鼻水	【名词】 鼻涕(水けの多い鼻汁) 寒くて~が垂れる/冻得淌鼻涕	はなみず	
\\	体	【名词】 
\\	身体;身子(頭、胴、手足など肉体全体をまとめて言う語) 
\\	体格;身材(体つき) 
\\	体质(体質) 
\\	健康;体力(肉体の健康。体力)【名词】 
\\	身体〔からだ〕 
\\	样子,体裁〔かたち) 
\\	姿态〔すがた〕 
\\	本体,本质,实体〔本質。実体〕 
\\	尊;具(人の遺体や神仏などを数える語)【名词】 
\\	样子;打扮(有り様。様子) 
\\	情况,状态(物事のようす) 
\\	姿态(見た目)	からだ たい てい	
\\	怠い	【形容词/い形容词】 发倦,发酸,懒倦,慵懒(疲れて体に力がなく、動くのが億劫である)	だるい	
\\	口	【名词】 口;嘴;说话,言语;传闻,话柄;出入口,门;口儿,地方(人間や動物が飲食物を取り入れ、音声を発する所) 【接头/接尾】 股,份,份额,分担量(分)	くち	
\\	風呂	【名词】 洗澡用热水;澡盆;澡堂〔塘〕,浴池(入って体を温めてたり清潔にする湯。また、その浴槽や設備。バス。銭湯。風呂屋)	ふろ	
\\	冷たい	【形容词/い形容词】 
\\	冷的;凉的;气候冷的;温度低的;身体感觉(触觉)凉的(温度が低く、冷ややかに感ずる) 〜飲み物/冷的饮料 
\\	(感情)冷淡的;冷冰冰的(人情に薄い。冷淡である) 〜目で見る/冷眼相看	つめたい	
\\	湯	【名词】 
\\	开水(湯を沸かした物) 
\\	浴池(浴場) 
\\	温泉(温泉) 
\\	融化的液体金属(金属を溶かした物) 
\\	洗澡水(くすりゆ)	ゆ	
\\	種類	【名词】 种类;类,种;样(いくつかの個体に共通の性質によって分類しまとめたもの。また、そのようにして総体を分類したときに生ずるまとまり) 〜が同じ/种类完全一样 〜が違う/种类不同	しゅるい	
\\	毎食後	【名】 每餐后	まいしょくご	
\\	カプセル	【名词】 
\\	【德】
\\	胶囊(飲みにくい薬品を封入して飲みやすくする、ゼラチン製の小さい容器) 
\\	密封舱(宇宙飛行体の気密容器)		
\\	〜回	【名词/接尾】 回,次(物事の度数を表すこと) 3~の授業/三节课 なん~も実験した/已实验了好多次	〜かい	
\\	大丈夫	【名/副】 
\\	牢固,可靠 
\\	放心,不要紧,没错儿(まちがいがなくて確かなさま)	だいじょうぶ	
\\	お大事に	【挨拶】 多保重(对病人用)	おだいじに	
\\	思う	【他动・一类】 
\\	想,思索,思量,思考(判断する) 
\\	相信,确信(信じる) 
\\	预想,预料,推想,推测,估计,想象,猜想(推量する) 
\\	感觉,觉得(感じる) 
\\	希望,期待(希望する) 
\\	挂念,放心不下(強い心にかかる)	おもう	
\\	出す	【他动词・五段/一类】 
\\	拿出,掏出,取出使从里面到外面(中から外へ移動させる) 
\\	出,体现出,显现出,暴露出(あらわす) 
\\	发生,使产生(生じさせる) 
\\	出,发,使出发(出発させる) 
\\	寄出,发出,打,发(送る) 
\\	提出(提出する) 
\\	供,端出,供给,供应,上,提供(飲食物を供する) 【接头/接尾】 …起来,开始做…(~し始める)	だす	
\\	芝生	【名词】 草坪,矮草地	しばふ	
\\	慌てる	【自动词・一段/二类】 
\\	惊慌,着慌,慌张(思いがけないことに出くわして、落ち着きを失う) 〜てない/别慌 
\\	急急忙忙,急忙(ひどく急いで事をする)	あわてる	
\\	入れる	【他动词・一段/二类】 
\\	装进,放入(入らせる) 
\\	送进,收容(収める) 
\\	包含,算上,计算进去(含める) 
\\	点灯,开电门;点火(つける) 
\\	承认,认可;听从,采纳,容纳(認める。うけいれる。認容する) 
\\	添,加,补足(加える) 
\\	请入,让进(通す) 
\\	镶,嵌(はめこむ) 
\\	加入(多余的东西),插入(さしはさむ) 
\\	投票(投票する) 
\\	送到(届ける) 
\\	交,缴纳(おさめる) 
\\	用;花费(費やす) 
\\	倒入热水冲泡饮料。也写作“淹れる”	いれる	
\\	止める	【他・一段/二类】 
\\	停,止,停止〔動きを停止させる) 
\\	止,堵,憋,屏,关,关闭(途絶えさせる。やめる) 
\\	戒掉,让其停止(やめさせる)【他・一段/二类】 
\\	停止,放弃,取消,作罢(続いていた動作・状態に決まりをつけさせる。絶やす。停止する。中止する) 
\\	忌(病気・癖などをなおす)【他・五段/一类】 
\\	停下,止。使正在动的事物停止(動いている物をとめる。抑止する) 
\\	止住,阻止住。先停下,使先保持某状态(とめておく。そのままにしておく)	とめる やめる とどめる	
\\	スイッチ	【名词】 
\\	【英】
\\	电门,开关(電流を通したり止めたり、また、切り替えたりする装置。開閉器。点滅器) 
\\	【英】
\\	转辙器,路闸,转换,交替(位置・方向・やり方などを切り替えること)		
\\	機械	【名词】 机械、机器(動力源から動力を受けて一定の運動を繰り返し,一定の仕事をする装置)	きかい	
\\	触る	【自动词・五段/一类】 
\\	触摸;接触(人が手などで物体や人体に意図的に接触する。触れる) 
\\	有关联,有关系(関わりを持つ) 
\\	触怒,伤害情绪(しゃくに障る)	さわる	
\\	パスポート	【名词】 
\\	【英】
\\	护照(旅券) 
\\	身分证(身分証明書)		
\\	失くす	【他动·一类】 弄没,弄丢。丢失,消除	なくす	
\\	脱ぐ	【他动词・五段/一类】 脱,摘掉(身につけていた物を取り去る)	ぬぐ	
\\	払う	【他动词・五段/一类】 
\\	去除,拂除,扫除(本体にとって、邪魔・不要・無益な物などを、手や道具を用いて、取り除く。除去する) 
\\	支付(金銭を渡す。支払う。また、納入する) 
\\	处理掉,卖掉(多余的东西)(不要品などを売り渡す) 
\\	驱赶,(将人和动物从现场)赶走(人をその場から去らせる) 
\\	(用手足、刀、棒等)横扫,横砍(横に勢いよく動かす。勢いよく横ざまに振る) 
\\	表示(尊敬和注意等的态度)(心を向けたり、注いだりする。心を傾注する) 
\\	达到(全无的状态)(何もない状態になる。すっかりなくなってしまう) 
\\	付出(珍贵的、无以替换的东西)(目的を達するために、ある物を費やす。消費する) 
\\	搬出,迁出(自分の物を取り去って、それまでいた所をあける。引き払う)	はらう	掃う
\\	禁煙	【自动词・サ变/三类】 
\\	禁止吸烟(タバコをすうことを禁じること) 
\\	戒烟,忌烟。戒除吸烟恶习(タバコを吸うのをやめること)	きんえん	
\\	心配	【名词】 
\\	操心,费心;关照;张罗,介绍(心を配って骨を折ること。いろいろとお世話をすること) 
\\	担心,挂心,挂念,牵挂,惦记,挂虑,惦念;害怕;不安;忧虑(何か起きはしないかと、気にかけること。不安がること。気がかり)【形容动词/な形容词】 担心;不安;操心(何かおきはしないかと気にかけるさま。気がかりなさま)	しんぱい	
\\	大切	【形容动词/な形容词】 
\\	要紧,重要;贵重(重要であるさま。肝要。大事) 
\\	心爱,珍惜;保重(丁寧に扱うさま。大事) 
\\	贵重,宝贵。价值很高(価値が高いさま)	たいせつ	
\\	危ない	【形容词/い形容词】 
\\	危险,不安全(身体・生命がそこなわれそうだ。危険だ) 
\\	靠不住;令人担心(信頼できない。信用がおけない〕 
\\	危险,不稳固。地位受到威胁(地位がおびやかされている) 
\\	危险,悬乎,不保险。处于预计会出现的坏结果的状态(悪い結果が予想される状態にある)	あぶない	
\\	返す	【他动词・五段/一类】 
\\	归还,退掉(物を、本来の場所や持ち主に戻す。返却する) 
\\	〈作自动词用〉返回(向きを逆にする) 
\\	翻过来(向きを逆にする) 
\\	报答,回敬,还礼,还击〔相手から受けた行為に対して、それと同じことをこちらからする。相手の働きかけに、同等の働きかけでこちらが応える) 
\\	〈接动词连用形后〉表示重复该动作或重复对方对自己的行为(その動作を初めからもう一度、または何度もしてみる。或いは相手からされたのと同じことを、こちらから相手に対してする)	かえす	
\\	テープ	【名词】 【英】
\\	带子,薄而窄的细长物体的总称(薄くて細長い帯状のものの総称) 
\\	(船舶启航时投掷的)纸带(出航時に船に投げ渡したり、ステージの歌手に投げたりする、紙製の長いひも) 
\\	(用布或尼龙等制造的具有绝缘、加固、修缮等多种作用的)封带。 
\\	(田径比赛中在终点拉的)终点线(競走で、決勝線に張り渡すひも) 
\\	(录音、录像用的)磁带(音声・画像などの信号を記録するのに用いる、表面に磁性物質を塗った帯状のもの) 
\\	(通信机、计算机的)穿孔带(穿孔テープ) 
\\	胶带(「セロハンテープ」、「ガムテープ」の略)		
\\	大声	【名词】 大声,高声(大きい声)	おおごえ	
\\	曲がる	【自动词・五段/一类】 
\\	弯,弯曲(まっすぐでなくなる) 
\\	转弯(進む方向を横に変わる) 
\\	歪曲,不合道理(道理に外れる) 
\\	倾斜(傾く) 
\\	乖僻(心や行いが素直でなくなる)	まがる	
\\	言葉	【名词】 
\\	话;语言;言词(物の言い方。言葉づかい) 
\\	语言。通过词语表达(言語にいる表現、話、文章) 
\\	语言,单词(語句、単語)	ことば	
\\	然様なら	【感叹词】 
\\	再见。再会〔あいさつ〕 
\\	告别,离开〔わかれる〕	さようなら	左様なら
\\	辛い	【形容词/い形容词】 
\\	辣(舌が刺激を受ける味) 
\\	咸(しょっぱい。塩辛い) 
\\	严格(処置や評価が情け容赦がない) 【形容词/い形容词】 
\\	辛苦的,艰苦的,吃不消的,身心上感到苦痛(心身に苦痛を感ずる。苦しい) 
\\	薄情的,残忍的,冷酷的,待人处事没有同情心的(人に対する仕打ちに思いやりがない。つめたく酷い) 
\\	为难的,困惑的,不知如何是好(どうしたらよいか分からず、苦しむ。困る)	からい つらい	
\\	親	【名词】 
\\	双亲;父母,父亲,母亲(子を生んだ人,または,他人の子を自分の子として養い育てる人。実父母・養父母の総称) 
\\	祖先(もののはじめ。元祖;祖先) 
\\	母,母体(子をもっている生物) 
\\	庄家(麻雀などのゲームで、特別の権利を持ったりするもの) 【名词】 
\\	亲近,亲密(したしいこと。したしみ。よしみ) 
\\	亲属;亲戚(親・兄弟などの近親者。親族。みうち)	おや しん	
\\	愛	【名词】 
\\	爱,爱情;友爱,恩爱(男女間の、相手を慕う情。恋。親兄弟のいつくしみ合う心。広く、人間や生物への思いやり) 
\\	爱好,热爱;爱戴(このむこと。めでること) 【接头/接尾】 
\\	心爱;得意(マは真、ナは格助詞、称美・親愛の意を表す) ~娘/心爱的女儿;掌上明珠 ~弟子/得意弟子;得意门徒;得意门生 
\\	同:真	あい なま	
\\	遅刻	【自动词・サ变/三类】 迟到,误(ひまどる) 5分~した/迟到了五分钟 学校に~した/上学迟到了	ちこく	
\\	ながら	【接续助词】 
\\	〈接动词连用形后〉一边…一边…;一面…一面…表示两个动作同时进行(2つのことが同時に進行する文脈に用いる) 
\\	〈接体言或动词连用形、形容词终止形和副词等后〉尽管,明知道……仍然…。表示既定的逆态接续(直接には関連しない事柄が、矛盾なく共存することを表す。或いは、満足はできないものの、容認することを表す) 【接头/接尾】 
\\	全部,都,表示同类事物处于相同状态中(その同類が、揃ってそのまま同じ状態にあることを表す) 
\\	原样;原封不动;如…一样(その(ままの)状態が断絶することなく続くことを表す) 
\\	表示自身之意(自分自身のことを強調して表す)		
\\	必ず	【副词】 
\\	一定,必定,必然(確かに。間違いなく。きっと) 
\\	总是,无一例外(1つの例外もなく、ある物事が起こるさま。ある条件のもとで、例外なく起こるさまにも用いる。常に) 
\\	〈以否定的形式表现〉(未)必;(不)一定((下に禁止や打ち消しの語を伴って)決して。必ずしも)	かならず	
\\	緊張	【自动词・サ变/三类】 
\\	紧张。精神、态度处于绷紧的状态(引き締まること) 
\\	争端即将发生的状态(相互の関係が悪化して、今にも争いの起こりそうな状況にあること)	きんちょう	
\\	美味い	【形容词/イ形容词】 
\\	巧妙的;极为适当的;可喜的;适当的(技術が優れていて、好ましい結果が得られる様子) 
\\	(味道)好吃的(味が好ましいので、もっと飲み食いしたい感じだ。美味しい)	うまい	
\\	困る	【自动词・五段/一类】 
\\	困难,为难,没有办法,苦恼 
\\	苦于贫困,困苦(貧乏で苦しむ) 
\\	难以处理,无法对付(相手の行為を迷惑に感じる)	こまる	
\\	父親	【名词】 父亲。双亲中的男性(男親。父。男である親、お父さん)	ちちおや	
\\	残す	【他动词・五段/一类】 
\\	留下(残るようにする) 
\\	剩下(余す。残らせる) 食事を〜/剩下饭菜 
\\	〈亦作「遺す」〉遗留(给后世)(後世に伝える) 莫大な財産を〜/遗下大笔财产 
\\	相扑中站稳(相撲で相手の攻めをこらえて踏みこたえる) 土俵ぎわで辛うじて〜/在场地边上勉强站住脚	のこす	
\\	全部	【名词・副词】 
\\	全部,都(全体) 
\\	整套书籍	ぜんぶ	
\\	悪戯	【名词】 对自己行为的自谦语 【形容动词/な形容词】 淘气,恶作剧,使坏(悪さ) 【自动词・サ变/三类】 调戏,猥亵(男女の私通。不義) 【名词・形動・自サ】 
\\	淘气,恶作剧(ふざけて) 
\\	玩笑,闹着玩儿;消遣 
\\	摆弄,玩弄	いたずら	
\\	やっと	【副词】 好容易;终于;勉勉强强;刚刚(実現や成立が難しい物事が、どうにかこうにか成り立つ様。辛うじて。ようやく) ~彼女を口説いた/好不容易把她说服了 家族4人が~生活していくだけの収入しかない/只有勉强够全家四口人生活的收入 ~汽車に間に合った/差点儿没赶上火车 ~の思いで頂上に辿り着いた/好容易爬上了山顶		
\\	怒鳴る	【自动·一类】 
\\	大声喊叫(大声で言う。さけぶ) 
\\	大声申斥〔大声でしかる〕	どなる	
\\	だから	【接续】 因此,所以(前に述べた事柄が、後に述べる事柄の原因・理由になることを表す語。そういうわけで。それゆえ) 彼はけちんぼうだ。~、みんなに嫌われる/他是个吝啬鬼,所以大家都讨厌他 ~言わないことじゃない/你看,我不是(早就)跟你说过嘛		
\\	注ぐ	【自动词・五段/一类】 
\\	流入,流(水が流れ込む) 川水が海に~/河水注入海里 
\\	降下,落下(雨などが降りかかる) 【他动词・五段/一类】 
\\	注入,灌入,引入,流(液体を流れ込む) 
\\	浇,洒(液体を掛ける) 【他动词・五段/一类】 斟;灌(进),注入;倒,倒入(液体を容器にそそぎ入れる。そそぎ入れる) 杯に酒を~/往杯里斟酒	そそぐ つぐ	
\\	愛情	【名词】 
\\	爱心。从心里觉得某人或某物很重要、很宝贵的心情,慈爱(怜爱、疼爱)的心情(人や物を心から大切に思う気持ち。いつくしみの心) 
\\	爱情,爱恋之情。恋慕异性的心情〔異性を恋しく思う心〕	あいじょう	
\\	少なくとも	【副词】 
\\	少说也要,保守估计也要…(最小の程度や最低の段階に言及するさま。うちわにみても) 費用は〜一億円かかる/费用少说也要花费1亿元 〜千人はいる/保守估计也有一千人 
\\	至少(ほかの物事はともかくとして。せめて) 〜規則だけは守れ/至少要遵守规则	すくなくとも	
\\	同等	同等(级)(等級・程度の同じであること。同じ等級)	どうとう	
\\	両親	【名词】 双亲,父母(父と母。ふたおや)	りょうしん	
\\	懐かしい	【形容词/い形容词】 怀念,眷恋(昔のことを思い出して、もう一度会いたい)	なつかしい	
\\	父の日	【名词】 父亲节,慰劳、祝颂父亲的日子,为每年6月的第三个星期日,始自美国(父に感謝を捧げる日、6月の第3日曜日。アメリカに起こった行事)	ちちのひ	
\\	何か	【副词】 
\\	什么,某种,某些(これときまっていないものとこと) 
\\	不知为什么,总觉得(何かしら)	なにか	
\\	連れる	"【自他・一段/二类】 
\\	带,领(伴う) 
\\	(常用""に〜""的形式)伴随着;跟随(〜立つ,うつりうごく)
\\	つれる	
\\	温泉	【名词】 温泉(地熱のために平均気温以上に熱せられて湧き出る泉)	おんせん	
\\	親孝行	【名・自动词・サ变/三类】 孝,孝敬父母,孝顺;孝顺父母的人,孝子(親に孝行であること)	おやこうこう	
\\	頃	【名词】 
\\	时候,时期(時) 
\\	时机(ころあい) 【造语】 1…时分…前后…左右 
\\	正好,正合适的时候〔程度〕(ちょうどよい)	ころ ごろ	
\\	怖い	【形容词/い形容词】 令人害怕的;可怕的	こわい	
\\	有り難味	【名词】 恩惠。有感谢之意(ありがたいと思って、それに感謝する気持ち) 親の〜/父母之恩	ありがたみ	
\\	楽しむ	【他动词・五段/一类】 
\\	乐,快乐,享受,欣赏。(心を喜ばせる) 
\\	以…为消遣,使快活(慰める) 
\\	期待,以愉快的心情盼望(期待をかける)	たのしむ	
\\	技術	【名词】 技术;工艺(物事を巧みにしとげる技。術芸。自然に人為を加えて人間の生活に役立てるようにする手段。また、そのために開発された科学を実際に応用する手段)	ぎじゅつ	
\\	消す	【他动词・五段/一类】 
\\	熄灭;扑灭 
\\	消去,消除,抹去 
\\	关掉,切断 
\\	(数学)消去 
\\	杀,杀死(殺す)	けす	
\\	無駄話	【名词・他动・三类】 闲聊,闲话,废话。无关紧要话。没用的话,喋喋不休(役に立たない話。おしゃべり)	むだばなし	
\\	緩和	【名・自他・サ变/三类】 
\\	缓和(ゆるめる) 
\\	和缓(なごやかにする)	かんわ	
\\	当面	【名·自サ】 目前,眼前,当前,面临(急いで解決しなければならないことがらを、目前にひかえていること。また、ある事態にまともにぶつかること) 【副】 现在,目前,眼下(今のところ。さしあたり)	とうめん	
\\	急務	【名词】 紧急任务,当务之急『成』(急いでしなければならない仕事や任務) 目下の〜	きゅうむ	
\\	不登校	【名】 不去学校。因各种原因不去学校,多为心理上的原因拒绝上学(主として心理的な抵抗感から学校に行かないこと)	ふとうこう	
\\	通用	【自动词・サ变/三类】 
\\	通用,通行,兼用,两用(共通して用いられる) 
\\	(在某期间)通用,有效(使用可能) 
\\	通常使用(ふだん使う)	つうよう	
\\	最終電車	【名】 末班车	さいしゅうでんしゃ	
\\	情勢	情势,形势(変化して行く物事の成り行きやようす。成り行き) 国際~/国际形势	じょうせい	
\\	開花	【自动词・サ变/三类】 
\\	〔植物〕开花(草木の花が咲くこと) 
\\	开花结果〔成功する〕 
\\	文化活动频繁(物事が盛んになること)	かいか	
\\	愚図愚図	【副词・自动词・サ变/三类】 慢吞吞地,吞吞吐吐地(てきぱき行動せず、のろのろしているさま) 【副词】 嘀嘀咕咕的,嘟嘟囔囔的,唠唠叨叨的(ぶつぶつ不平を言うさま。また、幼児がむずかるさま) ~言うな/别唠叨 【形容动词/ナ形容词】 松松散散的,松松垮垮的(ゆるんでしまりのないさま。また、しまりがなくなり崩れるさま)	ぐずぐず	
\\	油断	【自动词・サ变/三类】 漫不经心。粗心大意。疏忽大意。麻痹大意。缺乏警惕(気をゆるして、注意を怠ること)	ゆだん	
\\	鼻歌	【名词】 
\\	哼唱, 哼着唱的歌曲(鼻にかかった小声で歌う歌。気分のよいときなどに,口をとじたまま小声で歌う歌) 〜を歌う/哼着歌儿,哼唱歌曲 〜で仕事をする/边哼歌边工作 
\\	同:鼻唄	はなうた	
\\	夕飯	【名词】 晚饭,晚餐,傍晚吃的饭(夕べの食事。夕食。夕餉。夕方に食べる飯。夕食)	ゆうはん ゆうめし	
\\	語り合う	交谈,谈话(互いに話をする。語らう。話し合う)	かたりあう	
\\	並木道	【名词】 林荫道(两旁都是树的道路)	なみきみち	
\\	片付ける	【他动词・一段/二类】 
\\	整理,收拾,拾掇(散乱している物を、きちんと整頓する) 
\\	解决,处理(物事を処理・解決する。終わりにする) 
\\	嫁给(娘を嫁にやる) 
\\	除掉,消灭;杀死(じゃまな者を取りのぞく。殺す)	かたづける	
\\	流暢	【形容动词/な形容词】 流畅,流利(話しぶりがなめらかでよどみないこと)	りゅうちょう	
\\	定年	【名词】 退休年龄(法規・規則によって退官・退職するきまりになっている年齢)	ていねん	
\\	雷	【名词】 
\\	雷,打雷〔雲と雲の間、あるいは雲と大地との間の放電によって、発光と音響を発生する自然現象。かんなり。かむなり。いかずち) 
\\	雷神,雷公〔雷神、雷を起こす神〕 
\\	大发雷霆的人,咆哮如雷。比喻高声严厉地痛斥〔腹を立ててどなりつけること。やかましくがみがみと怒鳴りつけること〕	かみなり	
\\	歯	【名词】 
\\	齿,牙,牙齿(動物の口腔内にある咀嚼するための一番目の器官) 
\\	齿(轮),锯齿(器具などの似たもの)	は	
\\	磨く	【他动词・五段/一类】 
\\	刷净,擦亮,磨光(擦って、艶を出したり、きれいにしたりする) 
\\	锻炼,磨练,钻研(上達しようと訓練する。練磨する) 
\\	美化,保养(美しくする)	みがく	研く
\\	落ちる	【自动词・一段/二类】 
\\	亡命;逃遁(都を離れて地方へ移っていく。また、戦いに敗れて他の土地へ逃げて行く) 城を~/弃城逃亡 
\\	(柔道)气绝;断气,晕过去(柔道で、気絶する) 
\\	陷落(城などが攻め取られる。陥落する) 城が~/城堡沦陷到别人手中 
\\	没考中;落选,落后(落第する。落伍する) 試験に~/没考中 
\\	掉,落;脱落;剥落;褪落(その場所から離れてなくなる) つやが~/失掉光泽 
\\	落下,降落,掉下来,坠落(上から下へ自然に、また、急に移動する) 木の葉が~/树叶落下	おちる	
\\	漲る	【自动词・五段/一类】 
\\	涨满(水が満ちて、あふれるほど勢いが盛んになる) 
\\	充满,弥漫(力や感情などがあふれるばかりにいっぱいになる)	みなぎる	
\\	現代	【名词】 
\\	现代,现今,当代(現在の時代。現に生きている、今の時代) 
\\	〈史〉现代(歴史の時代区分の一)	げんだい	
\\	社会	【名词】 
\\	社会,世间(共同生活の集団) 
\\	(某)界,领域(同類の集団)	しゃかい	
\\	インターネット	【名词】 【英】
\\	因特网(コンピューターネットワーク)		
\\	非常に	【副词】 非常、很、特别(强调程度) ①それは、〜むずかしい問題だ /这个问题很难 ②〜申し上げにくいのですが… /很难启齿… ③このコンピュー夕一は、〜人気があるんです /这台计算机非常受欢迎	ひじょうに	
\\	役割	【名词】 分配的(任务),担当的职务(それぞれ受け持つ仕事を割りあてること。また、割りあてられた仕事) 〜を演じる/扮演角色;担当职务	やくわり	
\\	果す	【他动词・五段/一类】 
\\	完成,实现(实行。物事を成し遂げる) 
\\	杀死(殺す。しとめる) 
\\	(接在动词连用形后)(用)光,尽((動詞の連用形に付いて)すっかり…する。…してしまう)	はたす	果たす
\\	利用	【名・他动词・サ变/三类】 利用;搭乘(交通工具)(うまく使って何かに役立たせること)	りよう	
\\	情報	【名词】 情报;消息,信息(ある物事の内容や事情についての知らせ。インフォメーション)	じょうほう	
\\	自宅	【名词】 自己的住宅。自家。家(自分の家。私宅)	じたく	
\\	通信	【名词/自动·三类】 通信;通音信,通信息	つうしん	
\\	連絡	【名・自他・サ变/三类】 
\\	联络。联系。彼此关联。通讯联系(互いにつらなり続くこと。互いに関連すること) …と~がある/与……有联系 
\\	通知。通报 学校からの~/学校的通知 
\\	(交通工具)联运。连接。联合	れんらく	
\\	メール	【名词】 【英】
\\	电子邮件	いー・メール	
\\	メッセージ	【名词】 
\\	【英】
\\	电文,通讯,消息,音信,口信(伝言。言伝。言語や記号によって伝えられる情報内容) 
\\	【英】
\\	致词,祝词(あいさつ) 〜を送る/寄贺词 
\\	【英】
\\	咨文,声明(书)		
\\	送る	【他动词・五段/一类】 
\\	用假名标写;标上假名 
\\	传送;传递;依次挪动(移し送る) 
\\	度过(過ごす) 
\\	送(人)送行;送走;伴送 
\\	派;派遣;打发(派遣する) 
\\	送;寄;邮寄;汇寄(先方に届くようにする。送付する)	おくる	
\\	ソフトウェア	【英】
\\	软件,程序设备(コンピューターのプログラムを抽象的にとらえる呼称。コンピューターの運用に関する手順や処理する情報などを含めてもいう。ソフト)(计算机)软件		
\\	チャット	【英】
\\	闲谈;缅元聊天;聊天室(コンピューターネットワークで、同時に複数の人がメッセージをやり取りすること)		
\\	毎晩	【名词】 每晚,每天晚上(よごと。毎夜)	まいばん	
\\	チェック	【名词】 
\\	【英】
\\	支票(小切手) 
\\	【英】
\\	方格花纹;格子;花格(格子縞) 
\\	【英】
\\	号码单。寄存手提行李的号牌(チッキ) 
\\	【英】
\\	(国际象棋) 【名・他动词・サ变/三类】 【英】
\\	核对 对照;查对;检查;(检查的)记号(点検すること。検査すること。照合すること)		
\\	変化	【名词/自动·三类】 
\\	变化,变更(ある物事がそれまでとは違う状態・性質になること。変わること) 
\\	词尾的变化 【名词】 妖怪,鬼怪,妖精,幽灵,阴魂,吓唬人的东西(霊魂や動物などが姿を変えて現れること。化けて出ること。また,その現れたもの)	へんか へんげ	
\\	齎す	【他动词・五段/一类】 带来;招致;造成。带去(持って来る。持って行く。多く抽象的なものごとについていう) 幸福を~/带来幸福 被害を~/带来危害	もたらす	
\\	進歩	【自动词・サ变/三类】 进步	しんぽ	
\\	今後	【名词】 今后,以后,将来,从此以后(これからのち。以後。副詞的にも用いる)	こんご	
\\	変える	【他动词・一段/二类】 
\\	改变;变动(形・色・性質・内容・場所を階段的にそれまでとは違うものにする。変化させる) 
\\	变更;将住址、活动场所移动到别处(居場所や、ある動作をする場所を別の場所に移す)	かえる	
\\	詳しい	【形容词/い形容词】 
\\	详细的,详密的(細やかなところまで行き届いているようす) ~地図/详细的地图 
\\	熟悉的,精通的(あることについて、こまかいところまでよく知っているようす) 法律に〜/精通法律	くわしい	
\\	まあ	【副词】 
\\	【感】先,暂且(はやる気持ちを軽くおさえる語) 
\\	还算,还可以。不充分,但可以忍受的程度(十分ではないが、我慢できる程度であるさま) 
\\	(女性用语)嘿,哎哟,哎呀(驚き・感嘆などの気持ちで女性が発する語)		
\\	何とか	【接续】 某,什么;表示不定的名称(はっきり言えない事柄を表す) 【副词】 
\\	无论如何;想方设法;表示想尽各种手段或方法做某事(手段や方法を尽くして、何かをすることを表す) 
\\	(虽谈不上满足的)总算;勉强;差不多(満足とはいえないが、どうにか。どうやら。まずまず)	なんとか	
\\	中国語	【名词】 中文,中国话,汉语,华语〔文〕
\\	シナ‐チベット諸語に属する言語で、中華人民共和国の公用語。台湾および国外の華僑の間でも話される。方言は、北方・呉・湘(しょう)・(こう)・客家(はっか)・(びん)・粤(えつ)の七つに大別される。普通話(プートンホワ)とよばれる共通語は、北方方言を基礎とし、北京語の発音を標準音としている。形態的には孤立語的特徴をもつ)	ちゅうごくご	
\\	入力	【名・他动词・サ变/三类】 
\\	输入。向电路等设备中所送入的信号,能量等。输入功率(電気回路などの装置に入れられる信号、エネルギーなど) 
\\	输入。向计算机送入需要进行处理的数据,亦指其数据。输入数据(コンピューターに処理すべきデータを送り込むこと。また、そのデータ。インプット)	にゅうりょく	
\\	語学	【名词】 
\\	语言学(言語を研究する学問。言語学) 
\\	外语;外语学习;外语课程(外国語の学習。また,その学科) 
\\	会外语,外语能力(能力)	ごがく	
\\	側	【名词】 
\\	侧,旁边,附近〔ごく近く〕 
\\	随...随...;刚...就...〔…するとすぐ〕 
\\	旁观,局外	そば はた	傍
\\	遠慮	【名・他动词・サ变/三类】 
\\	远虑,深谋远虑(遠い先々まで考えること。深い考え) 
\\	客气(人に対して言語・行動を控え目にすること。ひかえる。気がね) 
\\	回避;谦辞;谢绝(それとなく断ること。辞退すること。公の秩序を考えて出勤・謁見・祝い事などをさしひかえること)	えんりょ	
\\	壊れる	【自动词・一段/二类】 
\\	坏,碎,毁,坍塌。物体失去固有的形状或七零八落(物の形が崩れたり、ばらばらになったりする) 
\\	坏。失去正常功能或发生故障(正常な働きが失われる。故障する) 
\\	落空,毁掉,破裂。计划或约定告吹(計画や約束が駄目になる)	こわれる	
\\	施設	【名・他动词・サ变/三类】 
\\	设施,设备 
\\	孤儿院;养老院;母子宿舍;接生站	しせつ	
\\	懇親会	【名】 联谊会,联欢会	こんしんかい	
\\	万博	【名词】 世博,万国博览会	ばんぱく	
\\	理解	【名・他动词・サ变/三类】 
\\	理解,了解(物事の仕組みや状況などを知ること。分かること) 
\\	体谅,谅解(相手の立場や気持ちを汲み取ること)	りかい	
\\	意欲	【名词】 意志,热情,积极性(積極的に何かをしようと思う気持ち)	いよく	
\\	休講	【自动词・サ变/三类】 (教师)停止讲课,停课(教師が講義を休むこと)	きゅうこう	
\\	暮らす	【自他・五段/一类】 
\\	生活,度日(生活する) 
\\	度过一天(月日を送る) 
\\	一直(ずっとその事をし続けるいをあらわす)	くらす	
\\	建つ	【自动词・五段/一类】 建,盖(事物が新たに設けられる) 駅前に新しいビルが〜/站前盖起新楼	たつ	
\\	マンション	【名词】 【英】
\\	公馆;高级公寓(中・高層の集合住宅。比較的規模の大きいものをいう)		
\\	故郷	【名词】 故乡,老家(自分が生まれた土地。郷里) 〜に帰る/回老家	こきょう ふるさと	古里 故里
\\	法律	【名词】 法律(広義では法と同じ。狭義では国会で制定された規範を指す)	ほうりつ	
\\	禿げ山	【名】 秃山、不长树木的山	はげやま	禿山
\\	水田	【名词】 稻田,水田(水を入れて稲などを作る耕地。田。たんぼ)	すいでん みずた	
\\	農作物	【名词】 农作物。在农田里栽培的蔬菜、谷类等(田畑で栽培する野菜・穀類など。農耕による生産物)	のうさくぶつ	
\\	被害	【名词】 被害,受害,受灾,损失(損害・危害を受けること。また,受けた損害・危害)	ひがい	
\\	顔色	【名词】 
\\	面色,颜色,脸色(顔の表面の色。血色。感情の動きの表れた顔のようす。顔つき) 
\\	脸色,面色(皮膚の色) 
\\	气色,神色,眼色(感情の動きの表れた顔のようす。顔つき)	かおいろ がんしょく	
\\	注文	【名・他动词・サ变/三类】 
\\	订货,定货,订〔定〕购,订〔定〕做(品物の制作・配達・送付などを依頼すること。また,その依頼) 
\\	要求,愿望。向对方表示自己的希望或条件(先方にこちらの希望を示すこと。また、その条件)	ちゅうもん	
\\	予約	【名・他动词・サ变/三类】 预约;预定(あらかじめ約束すること)	よやく	
\\	貯金	【自动词・サ变/三类】 存,存钱;存款;储蓄(お金を貯めること)	ちょきん	
\\	満点	【名词】 
\\	满分。考试等中设定的最高分数(試験などで、想定された最高の点数。誤りや失敗が全くない状態) 
\\	满分,完美无缺『成』,好到家。指能满足的状态(満足できる状態であること。非常に良いこと)	まんてん	
\\	育てる	【他动词・一段/二类】 
\\	培育,抚育(生き物が成長するよう世話をする) 
\\	教育,培养(一人前になるように仕込む) 
\\	发展(好ましい方向に発展させる)	そだてる	
\\	育児	【名・他动词・サ变/三类】 育儿(乳幼児を育てること) 母は~に忙しい/母亲忙于抚育幼儿 ~院/育婴堂	いくじ	
\\	老後	【名词】 老后,万年、晚年(年をとってのち)	ろうご	
\\	如何にも	【副词】 
\\	的的确确,完全(確かに) 
\\	实在,真(的)(ほんとうに) 
\\	果然,诚然,的确(なるほど)	いかにも	
\\	可笑しい	【形容词/い形容词】 可笑的,滑稽的,奇怪的,可疑的(普通とは違うところがあって笑いたくなるさま)	おかしい	
\\	この頃	【副词】 近来,这些天来,最近时期,现在(少し前の時から現在にかけての期間。近頃。最近)	このごろ	
\\	真剣	【名词】 真刀真剑 【形容动词/な形容词】 认真,一丝不苟(貌)(真面目,本気)	しんけん	
\\	条件	【名词】 
\\	(制約)条件 
\\	条文,条款(取り決め条項)	じょうけん	
\\	健康	【名/な形容】 健康;健全(身体に悪いところがなく心身がすこやかなこと。達者。丈夫。壮健。また、病気の有無に関する、体の状態)	けんこう	
\\	経済	【名词】 
\\	经济。物资的生产、流通、交换、分配及其消费、积累的整个过程以及在以上过程中所形成的社会各种关系的总体 
\\	经济。指关于金钱的收入支出(金銭の出入りに関すること。やりくり)	けいざい	
\\	挙げる	【动词・一段/二类】 
\\	竭尽全部 
\\	尽力,全拿出来 
\\	扬名,出名,有名 家名を~/扬家名 
\\	举例,列举,举出 
\\	举行仪式 
\\	举证,检举(検挙する) 
\\	逮捕,查检 
\\	开始战争	あげる	
\\	生き甲斐	【名词】 生存的意义,生活的价值(生きるに値するだけの価値。生きていることの喜びや幸福感)	いきがい	
\\	生きる	【自动词・一段/二类】 
\\	活,生存(人、動物などが命を保つ、生存する) 
\\	生活(生活する、暮らす) 
\\	有价值,献身于…,致力于…(一生を送る、やや文学的表現) 
\\	有效,有用,起作用(そのものが持っている本来の機能・能力が発揮される。有効に働く) 
\\	有影响,逝者或失去的东西仍有影响(死んだ者、失われたものの名残や影響が残る) 
\\	活,围棋的棋子和棒球的跑垒者处于不败的状态(囲碁で、一連の石が2つ以上の独立した目を保つ。野球で、アウトにならずにすむ) 
\\	生动,栩栩如生(生き生きする)	いきる	
\\	値打ち	【名词】 
\\	声价,品格(品位、品格) 
\\	价值(価値) 
\\	价格,价钱(値段) 
\\	估价,评价,定价(値段の高下を定めること)	ねうち	
\\	感じる	【自他・一段/二类】 
\\	(感官上)感,感觉,觉得(外からの刺激で、ある感覚を生ずる) 痛みを~/感到疼 
\\	(思想上)感到(ある気持ちや感情を心に持つ。ある考えを心にもつ) 
\\	感想,感动,感佩,有所感(感動する) ~ところあって、詩を書いた/因有所感而写诗	かんじる	
\\	具体的	【形容动词/な形容词】 具体,实际,具体的(はっきりとした実体を備えているさま) 〜な例をひとつあげる/举一个实际的例子 〜に解説する/具体地解释一下	ぐたいてき	
\\	ボランティア	【名词】 【英】
\\	义勇军;志愿军;社会事业义务工作人员(自発的にある活動に参加する人。特に,社会事業活動に無報酬で参加する人。篤志奉仕家) ~活動/志愿者活动		
\\	本の	【副词】 实在,不过,仅仅,一点点,些许(すこし) 〜少し/一点点	ほんの	
\\	趣味	【名词】 
\\	趣味;风趣;情趣(趣きのある。ものの持つ味わい) 
\\	趣味(美的感覚) 
\\	爱好,喜好;嗜尚(楽しみでする物事)	しゅみ	
\\	人間	【名词】 
\\	人,人类(一定の感情、理性、人格を有する) 
\\	品质;人品,品格,为人((ある個人の)品位、人柄、人物) 
\\	人间,社会,世上(世間)	にんげん	
\\	年を取る	【惯用句】 上年纪,上岁数(老齢になる)	としをとる	
\\	死別	【名词/自动・三类】 死别(しにわかれること) 夫に~した/与丈夫死别	しべつ	
\\	一人暮らし	【名词】 单身生活,一个人过日子,独居(ひとりで暮らすこと) ~の老人/单身生活的老人 ~をのんびり楽しむ/清闲自在地享受单身生活	ひとりぐらし	
\\	確り	【副词・自动词・サ变/三类】 
\\	结实,紧固。连接紧密,不易分离状(堅固でゆるぎないさま。堅実で信頼できるさま) 
\\	坚固。牢固。行为或变化的结果安定、坚固状(物事の基礎や構成が堅固で安定しているさま) 
\\	稳固。稳定(身心が健全であるさま。また、意識がはっきりしているさま) 
\\	坚挺。牢记(気持ちを引き締めて確実にするさま) 
\\	确实,扎实,可靠(体が健壮。意志が強い。識見が高明) 
\\	充分,多。有时带有讽刺意思(十分であるさま。たくさん。皮肉をこめて言うこともある) 【形容动词/ナ形容词】 股票市场形势好。反义词是「小確り」(相場が上昇傾向にあるさま。反対語は「こじっかり」)	しっかり	
\\	高齢化	【名】 老龄化,高龄化	こうれいか	
\\	進む	【自动词・五段/一类】 
\\	前进(前の方に出ていく) 
\\	(地位、级别等)上升(地位や学年などがより上級に移る) 
\\	进步;(能力、技能等)长进(能力や技術の程度が上がる。上達する。進歩する) 
\\	主动做;来劲儿(ある事をしようという気力·意欲が出てくる。積極的になる) 
\\	进展顺利(仕事や計画が捗る) 
\\	(钟表)快(他のものに比べて、先にある。特に時計の表示が正しい時刻よりも先になる) 
\\	(食欲)增进(食欲が出る) 
\\	(病)加重(程度がひどくなる。悪くなる) 
\\	(事物的程度)提高;先进((程度の高い)状態に向かう) 
\\	朝某一目标努力;有志于…(目標・進路などを定めて、それを始める)	すすむ	
\\	其れ程	【副词】 那么,那样(程度)(それくらい。そんなに。それほどまで。予期したほど。思ったほど)	それほど	
\\	恐ろしい	【形容词/イ形容词】 
\\	可怕(こわい) 
\\	惊人,非常,厉害(ひどい)	おそろしい	
\\	数	
\\	〔数量〕数,数目(一つ、二つ、三つなど、ものを個々にかぞえて得られる値) 
\\	〔多いこと〕多数,多种,种种(数が多いこと) 
\\	足以一提的事物,有…价值的事物 
\\	同伴(仲間)	かず すう	
\\	減る	【自动词・五段/一类】 
\\	减,减少(少なくなる) 
\\	磨损(すり切れる) 
\\	饿(空腹になる) 
\\	〈后接否定形〉畏惧,胆怯((打ち消しの語を伴って)怯む。臆する)	へる	
\\	増える	【自动词・一段/二类】 增加,增多(数が多くなる。物が生じて、数が増す) 去年より倍〜た/比去年增加了一倍 雨で川の水が〜た/因为下雨河水上涨	ふえる	殖える
\\	一方	【名词】 
\\	一方,一个方面,一个方向(二つあるうちの一つ。片方) 
\\	一方,单方面(一つの方面。一つの方向) 
\\	专门,一直,越来越(もっぱらその方向・方面にかたよること。…するばかり。だけ) 
\\	(作接续助词)一边…一边…(接続助詞的に用いて。…する反面。…と同時に) 【接续词】 另一方面,且说,转变话题时的用语(関連するもう一つのほうについて言うと。話かわって) 【名词】 一般,普通,寻常(普通の程度であるさま)	いっぽう ひとかた	
\\	夫婦	【名词】 夫妇,夫妻(婚姻関係にある男女の一組。夫と妻)	ふうふ みょうと めおと	
\\	二人	【名词】 二人,两个人,一对(人数が2であること、夫妻等)	ふたり	
\\	両方	【名词】 
\\	双方,两者,两方(二つ物事。両者。両方) 
\\	两边,两方。两个方面,两个方向(両側。二つの方面。二つの方角)	りょうほう	
\\	面倒を見る	【惯用句】 关照,照料,照顾(世話をする,人の世話をする。助ける)	めんどうをみる	
\\	老人ホーム	【名词】 老人院,老人之家(一般的に高齢者が入居する施設を指す)	ろうじんホーム	
\\	世代	【名词】 世代;一代,辈。同代的人,一代人(代)	せだい	
\\	頼る	【自他・五段/一类】 
\\	靠;依靠;借助;仰赖(頼みにする) 
\\	依赖;依靠(门路、人情等)(助けになるものとしてそこへ行く)	たよる	
\\	夕方	【名词】 傍晚,傍黑(日が暮れ始めて夜になるまでの間。夕刻。夕暮れ)	ゆうがた	
\\	年寄り	【名词】 
\\	老人;老头儿(年をとった人。老人。じいさん) 
\\	〈相撲〉相扑协会的董事(相撲で、興行に参与し、弟子を育成する者。また、親方の正式名称。日本相撲協会の評議員で、力士や行司が引退後に株を買って名跡を継ぐ。頭取)	としより	
\\	負ける	【自动词・一段/二类】 
\\	输,负,败(敗れる) 
\\	屈服,示弱,认输,低挡不住(ゆずる) 
\\	发炎,起炎症,起斑疹(かぶれる) 【他动词・一段/二类】 降低价钱,让价,减价(値段を安くしたり、品物を余分に渡したりする)	まける	
\\	チューリップ	【名词】 【英】
\\	郁金香。百合科多年生植物。用于观赏(ユリ科の多年草。観賞用)		
\\	ながら族	【名词】 惯于一心两用的人,惯于同时做两件事情的人(ラジオや音楽を聴きながら勉強したり、
\\	を見ながら食事をするといった ように、何か他事をしながらでないと集中出来ない人をいう)	ながらぞく	
\\	低炭素型	【名】 低碳型	ていたんそがた	
\\	間に合う	【自动词・五段/一类】 
\\	能起作用,能派用场(役に立つ) 
\\	够用;足够应付;过得去(その場の必要を満たす。十分である) コップがなくても、茶碗で〜/即使没有杯子,饭碗也能凑合着用 
\\	赶得上;来得及(定めの時刻に遅れない) 後悔しても〜(ない形)/就算后悔,也来不及了	まにあう	
\\	褒める	【他动词・一段/二类】 赞扬,称赞,赞美,褒奖,表扬。高度评价人和事(人や物事を高く評価して言う)	ほめる	
\\	尊敬	【名・他动词・サ变/三类】 尊敬,恭敬,敬仰(他人の人格・行為などを尊び敬うこと)	そんけい	
\\	スリ	【名词】 扒手,小偷(盗人) 大~/大盗;盗贼		
\\	財布	【名词】 
\\	钱包;腰包 
\\	钱包,钱袋;腰包(布や革でつくった、金を入れる小な入れもの)	さいふ	
\\	盗む	【他动词・五段/一类】 
\\	(他人のものを)偷盗,盗窃 金を〜/偷钱 
\\	背着…(こっそりと行う) 
\\	偷空,偷闲(時間をやりくりする);挤(時間の都合をつける)	ぬすむ	
\\	眠る	【自动词・五段/一类】 
\\	睡觉(心身の活動が休止し、目を閉じて無意識の状態に入る) 
\\	搁置不用(一時、活動をやめた状態になったり、活用されずにある) 
\\	死(死ぬ)	ねむる	
\\	上映	【名・他动词・サ变/三类】 放映,上映(映画をスクリーンに映して人々に見せること) その映画は目下〜中です/那部电影目前正在上映 次週〜する映画/下周放映的电影	じょうえい	
\\	印刷	【名・他动词・サ变/三类】 印刷。インクを使い、版面に描き出されている絵画、模様などを、紙そのほかの非印刷体の表面に刷り出すこと)	いんさつ	
\\	公平	公平,公道(かたよることなく、すべてを同等に扱う)	こうへい	
\\	大雨	【名词】 大雨,豪雨(ある時間はげしく,多量に降る雨) 大雨が〜/下大雨;下倾盆大雨	おおあめ	
\\	増す	【自他・五段/一类】 
\\	增加,增多,增大(増える。数量が多くなる) 水が~/涨水 
\\	增长,增进,增高(増やす。加える) 食欲が~/增进食欲 
\\	比…胜过〔…にもまして〕 去年にも~好成績を収める/获得了比去年更好的成绩	ます	益す
\\	署名	【名】 署名,签名	しょめい	
\\	権利	【名词】 权利	けんり	
\\	放棄	【名・他动词・サ变/三类】 放弃(自分の権利、資格、利益等を、あえて手放すこと)	ほうき	
\\	ドラマ	【名】 【英】
\\	戏剧,电视剧		
\\	ハイキング	【名词】 【英】
\\	郊游,徒步旅行,远足(山野・海辺などを自然を楽しみながら歩くこと) 子どもをつれて〜をする/带着孩子去郊游		
\\	私費	【名词】 私费,自己出资(個人が負担・支出する費用)	しひ	
\\	完成	【自他・サ变/三类】 完成,落成,完工,竣工(完全になしおえること。すっかり出来上がること)	かんせい	
\\	我慢	【名・他动词・サ变/三类】 
\\	自高,傲慢(高慢) 
\\	忍耐,忍受,容忍,自制(忍耐) これ以上〜できない/忍无可忍 
\\	饶恕,原谅(許す) 
\\	将就,克服,让步(譲歩) 
\\	顽固(がんこ)	がまん	
\\	産業	【名词】 产业,生产事业,实业,工商等企业,工业(農業・牧畜・林業・漁業・鉱業・工業など、産物を生産する事業)	さんぎょう	
\\	ロボット	【名词】 
\\	【英】
\\	机器人,机械人(人造人間) 
\\	自动机,自动仪器,自动装置。遥控设备,遥控机械装置(オートマチック) 
\\	傀儡(他人の指示のままに動く人)		
\\	退職	【自动词・サ变/三类】 退职(現職を退く)	たいしょく	
\\	終身	【名词】 终身。一生(いのちを終えるまでの間。一生涯) 〜刑/无期徒刑	しゅうしん	
\\	雇用	【名・他动词・サ变/三类】 雇佣,雇用;就业(やとう)	こよう	
\\	依存	【自动词・サ变/三类】 依存(他のものをたよりとして存在すること)	いぞん	
\\	譲る	【他动词・五段/一类】 
\\	让给,转让((自分の物を)他人に与える。譲り渡す) 
\\	让(路)(自分を後にして、他の人が先になるようにする) 
\\	谦让,让步(自分の考え、意見などを抑えて、他人の考えや主張を通させる。譲歩する) 
\\	出让,卖(売る) 
\\	改日,延期(ある行為を先へ延ばす)	ゆずる	
\\	男尊女卑	【名词】 男尊女卑(男性を尊び、女性を卑しいとする態度・思想) 〜に反対です/反对男尊女卑	だんそんじょひ	
\\	奉仕	【自动词・サ变/三类】 
\\	服务,不计报酬地效劳
\\	国家・社会・目上の者などに利害を考えずにつくすこと) 
\\	廉价销售商品(サービスとして特に安く売ること。サービス) 
\\	侍奉(神佛或君主)
\\	神仏・師・主君などにつつしんでつかえること。奉事)	ほうし	
\\	中止	【名・他动词・サ变/三类】 中止,停止进行。中途停止(途中でやめる事)	ちゅうし	
\\	行程	【名词】 
\\	行程,路程,旅程(道のり) 
\\	冲程〔ピストンの往復運動の距離〕	こうてい	
\\	縮める	【他动词・一段/二类】 
\\	缩,缩短;缩小(大きさを縮小する) 
\\	截短,弄小(短くする) 
\\	蜷曲;缩回(からだをまるめる) 
\\	减少,削减(へらす) 
\\	畏缩,惶恐,因为紧张、恐惧等原因而使身体或胆气变小 
\\	使起褶皱,弄皱 
\\	缩短;缩小;截短;弄小;蜷曲;减少;削减	ちぢめる	
\\	人生	【名词】 
\\	人生,人的生活(この世での生活) 
\\	人的一生(一生);生涯(事業などに従事する)	じんせい	
\\	様々	【形容动词/ナ形容词】 种种,形形色色,各种各样。各自不同,各式各样(それぞれ異なっていること。いろいろであるさま)	さまざま	
\\	苦しい	【形容词/い形容词】 
\\	痛苦,难受,疼(肉体的に苦痛である) 
\\	困难,艰难,难办(難しい) 
\\	艰苦,穷困(するのは不快であるの意を表す) 
\\	勉强,不自然(つじつまが合わない。無理である)	くるしい	
\\	出来事	【名词】 (偶发)的事件,变故(持ち上がった事件・事柄。事故)	できごと	
\\	悲しい	【形容词/い形容词】 悲痛的,伤心的,悲伤的,可悲的,悲哀的,遗憾的(心が痛んで泣けてくるような気持ちである。嘆いても嘆ききれぬ気持ちだ)	かなしい	哀しい
\\	何故	【副词】 为何,何故,为什么(理由・原因などを問う語。どういうわけで) 【副词】 
\\	如何;怎么样(どんな方法で。どうやって) この問題を~解決したらいいだろう/如何解决这个问题才好呢? 
\\	为什么(どんな理由で) ~きみは来なかったのか/为什么你没来呢? 【感叹词】 
\\	有力地否定前面的话,进而说相反的话时使用。不但不那样。相反。不(それどころか) ~、そんな程度じゃなかったよ/不(恰恰相反),没达到那种程度啊! 
\\	由于出乎意料非常惊讶时使用。呀。哦。哟 ~、あの人はたいしたものだ/哦!那个人真了不起 【副词】 何故;为何(どういうわけで)	なぜ どうして なにゆえ	
\\	望む	【他动词・五段/一类】 
\\	眺望(遠くをながめやる。はるかに見渡す) 泰山を〜/望泰山 
\\	希望,愿望,期望,指望,希求,要求(希望する) 幸福を〜/希望幸福 
\\	仰望〔あおぐ)	のぞむ	
\\	逆	【名词/形动】 逆,倒,反,颠倒(物事の順序・方向・位置関係などが反対であること。さかさま) 〜に言えば/反过来说	ぎゃく	
\\	考える	【他动词・一段/二类】 
\\	想;思索;思考;有条理地动脑筋思索究竟是怎么回事,比「思う」更具理性
\\	知識や経験などに基づいて、筋道を立てて、頭を働かせる) 
\\	想办法;研究
\\	工夫する。工夫してつくり出す)	かんがえる	亦作「校える」、「勘える」
\\	若し	【副词】 要,要是,如果,假如;假设,倘若(定していない物事,事実に反する物事を仮定して次に述べる物事の条件とする意を表す。かりに。確実ではないが,十分にあり得る事態を想定する気持ちを表す。あるいはひょっとして。もしかすると。もしかしたら)	もし	
\\	恐らく	【副词】 恐怕,或许,大概,一定(多分。きっと)	おそらく	
\\	深い	【形容词/イ形容词】 
\\	茂密,深邃(多く密生している) 
\\	(时间、季节)深的,晚的((時間的な推移の中で)今が盛りである) 
\\	(密度)大的,浓厚。(濃い。濃度が大きい) 
\\	深的(底・奥までの距離が長い) 
\\	(色)深的,浓的(色合いが濃い) 
\\	(关系)深的,密切的(物事の程度や分量、また、かかわりなどが多い)	ふかい	
\\	極楽トンボ	【名】 逍遥自在的人,逍遥王,悠闲懒散的人	ごくらくとんぼ	
\\	過ごす	【他动词・五段/一类】 
\\	(时间)流逝;度过(時間を経過させる) 
\\	生活;过日子(養う。暮らしを立てる。生計を立てる) 
\\	过度;过量(適度を越す) 
\\	(接动词连用形后)放任不管;听凭((動詞の連用形について)そのままにしておく。放置する) 
\\	(接动词连用形后)过分((動詞の連用形について)適度な程度を越してする)	すごす	
\\	馳走	【名】 
\\	奔走(かけはしること。奔走。はしりまわる) 
\\	款待;宴请;盛馔;美味(ふるまい。もてなし。饗応。立派な料理。おいしい食物。ごうかな料理) 御~様/承蒙款待	ちそう	
\\	感情	【名词】 感情,情绪(喜怒哀楽や好悪など、物に感じて起きる気持ち) 【惯用句】 〜に走る/感情用事	かんじよう	
\\	薄い	【形容词/い形容词】 
\\	薄。物体的厚度小〔厚みがない〕 
\\	淡,浅〔あわい。色・味・光・影などが濃厚でな〕 
\\	待人不好,冷淡。冷漠,淡漠。缺少情爱,关心,感动等的心情〔情があさい〕 
\\	少,稀,缺乏〔少ない。利益・効果・恩恵などが少ない〕	うすい	
\\	実感	【名・他动词・サ变/三类】 真实的感情。确实感觉到(実物に接して起きる感じ。また、実際に経験しているかのような生き生きとした感じ)	じっかん	
\\	苦しみ	【名词】 痛苦,苦恼,困难,困苦(苦しむこと。苦痛)	くるしみ	
\\	悲しみ	【名词】 悲哀,悲伤;忧愁,悲愁,悲痛(悲しむこと。悲しい気持ちや心。悲嘆。悲哀。愁嘆)	かなしみ	哀しみ
\\	思索	【名词/自他动・三类】 思索(論理的に筋道を立てて考えること。思惟) 〜にふける/陷入沉思	しさく	
\\	体験	【名・他サ】 体验,(亲身)经验(実際に自分で経験すること。また、その経験) 貴重な〜/宝贵的体验 おもしろい〜をした/经历了有趣的事情	たいけん	
\\	向上	【名词·自动·三类】 向上,提高;进步(進歩)	こうじょう	
\\	入学試験	【名词】 入学考试	にゅうがくしけん	入試
\\	失敗	【自动词・サ变/三类】 失败(やってみたが、うまくいかないこと)	しっぱい	
\\	悔しい	【形容词/い形容词】 悔恨的,遗憾的,窝心的,令人懊恼的,令人愤懑的(失敗や恥辱を経験して、諦めたり忘れたりできないさま)	くやしい	
\\	がっかり	【副词】 
\\	失望,颓丧 
\\	精疲力竭(がっくり)		
\\	平坦	【形动】 
\\	平坦(土地が平らな・こと(さま)) 
\\	平坦,平稳(物事に起伏がなく穏やかな・こと(さま))	へいたん	
\\	全く	【副词】 
\\	(后接否定)全然((否定表現に呼応して、それを強調する気持ちを表す)すっかり。全部。全然) 
\\	完全(全然に) 
\\	实在,简直,强调判断中没有谎言和夸张(実に。本当に) 
\\	(做感叹词用)确实,诚然	まったく	
\\	運	【名词】 
\\	送(はこぶこと) 
\\	动。运行(うごかすこと) 
\\	命运(天命)	うん	
\\	近付く	【自动词・五段/一类】 
\\	挨近,靠近(あるものがある場所の近くに移動する) 
\\	临近(それを行う時期が近くなる。ある期日・刻限が迫る) 
\\	接近,交往(積極的に親しくなろうとする。親密さを求める。近寄る) 
\\	越来越相似(めざすものに近い状態になる。似てくる)	ちかづく	
\\	盲腸炎	【病】 阑尾炎,盲肠炎	もうちょうえん	
\\	入院	【自动词・サ变/三类】 住(医)院(治療のために、ある期間病院にはいること)	にゅういん	
\\	残念	【形容动词/ナ形容词】 
\\	遗憾。可惜。对不起。抱歉(心残りなこと。未練のあること) 
\\	懊悔。悔恨。懊恼(口惜しいこと。無念)	ざんねん	
\\	挫ける	【自动词・一段/二类】 
\\	挫,拧,扭;挫伤,拧伤,扭伤〔ねんざする〕 
\\	气馁,颓丧,消沉,灰心失望〔勢いが弱る〕	くじける	
\\	成功	
\\	成功;成就;胜利(目的が達成されること。しようと思ったことをみごとにやりとげること) 
\\	功成名就;成功立业(世の中でりっぱな地位や財産をえる	せいこう	
\\	元	【名词】 
\\	起源,根源(物事の根本) 
\\	原状,原来的状态(以前の状態) 
\\	原因,起因(原因。理由) 
\\	成本,资本,本钱(元手。資金。また、原価。仕入れ値) 
\\	出身,经历(身元。素性) 
\\	原料(原料。材料) 
\\	酵母(酵母) 
\\	原,前(接頭語的に用いて、昔の、古い、などの意を表す)	もと げん	
\\	茶碗	【名词】 碗,茶杯,饭碗(飯を盛り、また湯茶を飲むための器)	ちゃわん	
\\	永久	永久,永远,恒久,永恒(いつまでも変わらずに続くこと。永遠) 【名词】 长久,永久(日本地名)	えいきゅう とこしえ とわ ながひさ	
\\	水準	【名词】 
\\	水准,水平面(水面の位置。陸地の高度を図る高度基準とする) 
\\	水平器〔水準器〕 
\\	水平,水准;标准高度,标准程度(物事の価値や性能を言う時の基準となるもの。レベル)	すいじゅん	
\\	急に	【副词】 忽然,突然,骤然,急忙(突然。急ぐ) 〜笑い出す/突然笑起来	きゅうに	
\\	探す	【他动词・五段/一类】 
\\	查找,寻找,找(見つけ出そうとして方々を見たり、歩きまわったり、人に聞いたりする、尋ね求める) 職を~/找工作 
\\	〈主要以「捜す」表示〉搜寻,寻找(必要なものや失ったものを見つけようとする)	さがす	捜す
\\	トンネル	【名词】 【英】
\\	隧道,隧洞;地道,坑道。贯通山腰或地下,供铁路,道路等通过而建造的洞穴(山腹や地下を貫いて、鉄道・道路などを通すために設けた穴) 【名・他动词・サ变/三类】 〈野球〉(没有接住的)球从两腿中间滚过。漏球。棒球比赛时,球从内外长手的两腿中间逃过(野球で、野手が球を股の間を通して後方にのがすこと)		
\\	抜ける	【自动词・一段/二类】 
\\	脱落;掉落(中に嵌り込んでいる物、生えている物が取れる。脱落する) 
\\	漏掉(構成要素の一部がなくなる。脱漏する。脱落する) 
\\	穿过;通过狭窄的地方到对面(通路(何かの中)を通って、向こう側へ出る) 
\\	偷偷溜走;逃走;退出(逃げ去る。逃れ出る) 
\\	迟钝;低下(魂・知恵などが抜け注意力が散漫である。知恵が不足している) 
\\	透彻;清澄(隔てがなくなり、どこまでも続いている。透き通っている) 
\\	(布料因不结实)变薄(生地の部分がだんだんなくなる)	ぬける	
\\	解決	【名・自他・サ变/三类】 解决(問題やもつれた事件などを、うまく処理すること。また、事件が片づくこと)	かいけつ	
\\	人体	【名词】 人体,人的身体(人間のからだ) 【名词】 
\\	风采,装束,举止,相貌(からだ。姿) 
\\	人品,品格(人の全体から受ける感じ。また,ひとがら。人品)	じんたい にんてい	
\\	首相	【名词】 
\\	首相,内阁总理大臣(内閣総理大臣) 〜になる/当首相 
\\	中国には「〜」という職名はなく,“国务院总理”がこれに相当する	しゅしょう	
\\	地域	【名词】 地域,地区	ちいき	
\\	コーチ	【名词】 【英】
\\	教练,技术指导(運動競技の技術などを指導すること)		
\\	体操	【名・他动词・サ变/三类】 
\\	体操(健全な身体を作り、健康の増進・精神の修養を図るためなどに行われる運動) 
\\	体操,体操比赛项目(「体育競技」の略) 
\\	(旧)体育课(学校教科の「体育」の旧称)	たいそう	
\\	怪我	【名词·自动·三类】 
\\	伤,受伤,负伤(負傷) 油断1秒、〜一生/疏忽一秒,后悔一生 
\\	过错,过失(思わぬ過ち。過失。損失)	けが	
\\	結果	【名・自他・サ变/三类】 
\\	结果,结局;后果〔ある物事を行った後に生じた現象、状況、物象をあらわす言葉である) 
\\	结实,结果〔実を結ぶ〕	けっか	
\\	訪ねる	【他动词・一段/二类】 访问(おとずれる) 先生の家を~/访问老师家 出しぬけに~/突然拜访	たずねる	
\\	瞬間	【名词】 瞬间,转眼,刹那,顷刻,俄顷(またたく間。きわめて短い時間)	しゅんかん	
\\	思わず	【副词】 禁不住,不由得,不由自主地,情不自禁地,不知不觉地(そうするつもりではなかったのに、無意識に行うさま。知らずに。考えずに)	おもわず	
\\	選挙	【名・他动词・サ变/三类】 选举,推选(多人数の中から投票などにより適任者を選び出すこと)	せんきょ	
\\	末っ子	【名词】 最小的儿子〔女儿〕;老小;(兄弟姐妹中)最年幼者;老儿子,小闺女,老疙瘩(兄弟姉妹のうち、一番あとに生まれた子。すえこ)	すえっこ	
\\	気が向く	【惯用句】 
\\	高兴,乐意(ある事をしたい気になる) 
\\	心血来潮(ある事をしたい気になる)	きがむく	
\\	希望	【名・他动词・サ变/三类】 
\\	希望,期望,愿望(あることの実現を願いのぞむこと。また、その願い。望み) 
\\	希望。寄予将来的期望(将来によせる期待。見通し)	きぼう	
\\	宇宙	【名词】 
\\	宇宙,空间;太空(すべての天体を含む空間の広がり。特に,地球の大気圏外) 
\\	天地,世界(存在する事物の全体。それを包む空間。天地万物。森羅万象。全世界)	うちゅう	
\\	開発	【名・他动词・サ变/三类】 
\\	开辟。开垦((天然資源を)生活に役立つようにすること) 
\\	开发。创办(产业)实用化。开发。开创。研制。发展(実用化すること) 
\\	启发(知識を開き導くこと)	かいはつ	
\\	海洋	【名词】 海洋
\\	広くて大きい海。大洋)	かいよう	
\\	濡れる	【自动词・一段/二类】 
\\	淋湿,湿润,渗入水分(物の表面にたっぷり水分がつく) 
\\	偷情(男女が情交する。色事をする)	ぬれる	
\\	初めて	【副词・他动词・一段/二类】 
\\	开始(开始する) 
\\	开创,创办(新たに起こす) 
\\	起来,开始(…しだす) 
\\	犯(老毛病)(いつもの癖や振る舞いを行い出す)	はじめて	始めて
\\	習慣	【名词】 
\\	个人习惯(生活の中でいつもくり返して行なっている、その人のきまったやりかた) 
\\	国家地区风俗习惯(ある国やある地方の人たちが、ふつうのこととして行なっているきまったやりかた)	しゅうかん	
\\	勿論	【副词】 当然;不用说,不消说,甭说,不待言;不言而喻(いうまでもない。自明である)	もちろん	
\\	特別	【名/な形容词】 特别,格外(普通一般と違うこと) 〜に安くする/特别减价 〜な場合は考慮してもいい/特殊情况也可以考虑	とくべつ	
\\	驚く	【自动词・五段/一类】 
\\	吓,惊恐,惊惧,害怕,吃惊吓了一跳(びっくりする) 
\\	惊讶,惊奇,惊叹,意想不到,感到意外	おどろく	
\\	終わり	【名词】 
\\	终,终了,末尾,末了,结束,结局,终点,尽头(物事が終わること) 
\\	末期(一生の最後)	おわり	
\\	初め	【名词】 
\\	开始,开头(開始。冒頭) 
\\	起因,起源(おこり。はじまり。最初。発端) 
\\	前者(前者) 
\\	最初,起初,当初,原先(ものごとを始めた時。てはじめ。最初のとき。もともと) 【接头/接尾】 开始、最初、初次(動詞の連用形に付いて。初めて…すること。物事の最初) 書き~/新年试笔	はじめ そめ	始め
\\	天気予報	【名】 天气预报	てんきよほう	
\\	ニュース	【名词】 
\\	【英】
\\	消息,新闻;[通过媒体的]报道(新聞・ラジオ・てれびなどにより報道される事件や出来事。また、それを伝えるための放送番組や記事) 
\\	新鲜事,新闻,稀奇事(珍しい出来事や新しい出来事。また、その知らせ)		
\\	咲く	【自动词・五段/一类】 (花)开(花の蕾が開く)	さく	
\\	騒ぐ	【自动词・五段/一类】 
\\	吵,吵闹(やかましくする);吵嚷(わめいて) 
\\	慌张,着忙(あわてる);激动,兴奋不安(動揺する) 
\\	骚动,闹事,显出不稳(騒ぎを起こす) 
\\	极力称赞(注目される);哄嚷;吹捧(もてはやされる)	さわぐ	
\\	大事	【名词・形动】 
\\	大事,重大问题,大事件(物事の根本にかかわるような重要なこと) 
\\	大祸〔危険・破滅〕 
\\	小心,慎重〔用心する) 
\\	重要,要紧,宝贵,保重,爱护(大切なさま。重要なさま)【名词】 大事,重大事件,严重事件(重大な出来事。大事件。大変)	だいじ おおごと	
\\	何時でも	【副词】 无论什么时候,随时,经常,总是(どんな時でも。常に)	いつでも	
\\	不思議	【名·な形容词】 怪,奇,奇怪,奇异;难以想象,不可思议『成』(想像を絶する)	ふしぎ	
\\	時期	【名词】 
\\	时期,时候(とき) 
\\	期间(期間) 
\\	季节(季節)	じき	
\\	見頃	【名词】 正好看的时候。正好看到,正值观赏时	みごろ	
\\	大体	【名】 大体,概要,大概,大略(重要な部分) 【副】 
\\	大致,大体,差不多(おおよそ) 
\\	本来,根本(もともと)	だいたい	
\\	末	【名词】 
\\	末尾,末了,终了(おわり) 
\\	末端(はし),头;尽头(はずれ) 
\\	将来,未来,前途(行く末) 
\\	无关紧要的事(取るに足りないもの),细微末节,支流末节,无足轻重;鸡毛蒜皮的小事;琐事,小事一段(こまごました事) 
\\	子孙(子孫);后裔(すでに死亡した人の) 
\\	(兄弟姐妹中)最年幼者(末っ子) 
\\	乱世,末世(末の世)【接头/接尾】 末;底(年・月の終わりの方)	すえ まつ	
\\	毎年	【名词】 每年(年ごと。まいねん)	まいとし	
\\	祭り	【名词】 
\\	祭祀,祭日(神や祖先の霊を祭ること。祭礼) 
\\	节日(記念・祝賀などのために催される行事)	まつり	
\\	集まる	【自动词・五段/一类】 聚集(多くの物が一つところに寄り合う)	あつまる	
\\	御喋り	【名・自动词・サ变/三类】 爱说话的人,健谈。聊天,闲谈	おしゃべり	
\\	踊る	【自动词・五段/一类】 
\\	跳舞,舞蹈(踊りを演じる。舞踊する) 
\\	不平稳,不平整(はげしく動揺する。乱れる) 
\\	活跃,行动(人の意のままに行動する。操られる)	おどる	
\\	合作	【名・自他・サ变/三类】 
\\	合制,合作;(共同で作る)合著,合编(共著) 
\\	协力,合作(提携・協力)	がっさく	
\\	交換	【名・他动词・サ变/三类】 
\\	交换(とりかえること) 
\\	交易(問う実写相互の間に財およびサービスが有償的に移転されること)	こうかん	
\\	見掛け	【名词】 外观,外表,外貌(外から見た様子。うわべ)	みかけ	
\\	案外	【形容动词/な形容词】 意想不到,出乎意外(予測と食い違うさま。思いのほかのことをすること)	あんがい	
\\	一体	【名词】 
\\	一体,同心,合力(一つの体。また、一つのまとまり) 
\\	一座,一尊(仏像や彫像などの数え方で、一つ) 
\\	总的说来,大体上,一般(全般。総じて。全体)【副词】 
\\	到底,究竟(疑問を強める。相手を責めて問いただすときも使う。本当に) 
\\	根本,本来,原来(そもそも。もともと)	いったい	
\\	気に入る	【惯用句】 喜欢,中意,称心,满意(好みや希望・理想に叶っていたりして、満足する)	きにいる	
\\	犯人	【名词】 犯人,罪人(犯罪を犯した人) 〜を逮捕する/逮捕犯人	はんにん	
\\	別に	【副词】 
\\	分开,另(他に) これについては、〜定める/关于这点另行规定 
\\	并(不),特别((下に打ち消しの語を伴って)取り立てていうこともないという気持ちを表す。取り立てて。特別に) 〜行きたくはない/并不怎么想去	べつに	
\\	怒る	【自动词・五段/一类】 
\\	骂,说,训(叱る) 
\\	生气,发怒,发火 【自动词・五段/一类】 
\\	发怒,生气,愤怒(腹を立てる。おこる。憤慨する) 
\\	激烈,狂暴起来(激しく動く。荒れ狂う) 
\\	挺起,耸起(角張って、ごつごつしている。角立つ)	おこる いかる	
\\	夕食	【名】 晚饭,晚餐,晚膳(夕方の食事,夕飯)	ゆうしょく	
\\	晩酌	【自动词・サ变/三类】 晚酌;晚饭时喝(的)酒((家庭で)晩の食事の時、酒を飲むこと。また、その酒)	ばんしゃく	
\\	ゴルフ	【名词】 【英】
\\	高尔夫球(ボールをクラブで打ち、9または18か所に設けられているホールに順次に入れ、打数の合計数が少ない者を勝ちとする球技)		
\\	御天気屋	【名】 喜怒无常的人,没准脾气的人,使性子的人	おてんきや	
\\	煩い	【形容词/い形容词】 
\\	嘈杂,烦人的(音が大きいのでやりきれない) 
\\	烦人的,纠缠不休难以忍受(しつこくて、やりきれない) 
\\	话多,爱唠叨人(小さいことまで、いちいち文句を言うのでいやだ) 
\\	说三道四,挑剔(物事に対して見識を持っていて、細かいところまで気にする) 
\\	厌恶,麻烦而令人讨厌(面倒くさくて、いやだ。わずらわしい)	うるさい	五月蝿い
\\	伺う	【他动词・五段/一类】 
\\	请教;打听(「聞く」「尋ねる」「問う」の謙譲語。また、目上の人の話などを聞く。拝聴する) 
\\	拜访 
\\	访问(訪問する)	うかがう	窺う
\\	夏休み	【名词】 暑假(学校などで、夏の一番暑い時期に設ける休暇。暑中休暇。夏期休暇)	なつやすみ	
\\	泳ぐ	【自动词・五段/一类】 
\\	游泳(手足やひれを動かし、水中・水面を進む) 
\\	向前栽去(前方によろめく) 
\\	挤过。穿过(押しわけて進む) 
\\	钻营。混世(世を渡る)	およぐ	
\\	登る	【自动词・五段/一类】 
\\	上,登,攀登(上へあがる) 
\\	逆流而上,上溯(流れの上流へ) 
\\	上升(温度が) 
\\	进京(上京する) 
\\	达到,高达(ある数量に達する)	のぼる	上る
\\	泣く	【自动词・五段/一类】 
\\	哭,啼哭,哭泣(人が,悲しみ・苦しみなどのために声を出し,涙を流す。また,喜びなどで涙を流す場合にもいう) 
\\	感觉为难,吃到苦头,伤脑筋;懊悔,懊丧(ひどい目にあって,嘆き悲しむ) 
\\	〔値引きする〕忍痛减价	なく	
\\	笑う	【自他・五段/一类】 
\\	笑(うれしい時の表情) 
\\	花开,开绽(つぼみが開く・ほころびる)	わらう	
\\	決める	【他动词・一段/二类】 
\\	(独自)断定;认定;(自己)作主(自分の意志や態度をはっきりさせる、決心する、一定の考えや、態度をしっかり保って変えない) 
\\	决定,规定;指定;选定;约定;商定(規則、方針などを作り上げる。選んで定める)	きめる	
\\	秋葉原	【名词】 秋叶原。东京都千代田区东北部的地名,第二次世界大战后,作为电器产品批发店、小型零售店街发展起来	あきはばら	
\\	頼む	【他动词・五段/一类】 
\\	请,雇(来てもらう) 
\\	依靠,依仗,指望(頼る。当てにする) 
\\	委托,托付;托付别人为自己做某事(託す。ある行為をしてくれるよう依頼する) 
\\	请求,恳求,嘱托;恳请别人能按自己所希望的那样去做(願いごとをする。自分の希望どおりにしてくれるように願う)	たのむ	
\\	野球	【名词】 棒球(米国で発達した球技の一)	やきゅう	
\\	眠い	【形容词/い形容词】 困,困倦,想睡觉(眠りたい気持ちである)	ねむい	
\\	ばかり	【副助词】 
\\	刚刚。刚才(ある動作がすぐにも実行される段階にあることを表す) 
\\	左右。上下。大约。表示大概的程度(分量・状態・程度などの、おおよその見積もりを表す) 
\\	只。仅。表示限定或显然很多的意思(それに限定する意を表す) 
\\	几乎要。快要(はっきりそうしないだけで、ありありと現れているさまを表す) 
\\	只因…(それが元になって、次の事態が生じた意を表す)		
\\	其の他	【副词】 其他;其余;另外,此外(それ以外) 食費~で10万円はかかる/饭费和其他共需十万日元 ~大勢/无名角色	そのほか	その他
\\	毎月	【名词】 每月(月ごと;つきづき)	まいつき まいげつ	
\\	十分	【形容动词/な形容词】 十分,充分,足够,充裕(物事が満ち足りて、不足・欠点のないさま。充分) 【名词】 十分钟(一時間の六分の一)	じゅうぶん じゅっぷん	
\\	ビル	【名】 
\\	大楼,高楼,大厦(建物)		
\\	しか	【副助词】 (接在体言、动词连体形、形容词及形容动词连用形后面,后续否定语)表示只此,很少,或不够之意((否定表現と呼応して)話し手にとって、狭い意識される範囲に限定されることを表わす) 結婚して半年に〜なっていない/结婚只有半年		
\\	そんな	【形容动词/ナ形容词】 
\\	那样的(そのような) ~ときには/在那样的时候 ~ことを言ってもしかたがない/你说那样话也是白说 
\\	哪里,不会(否定) ご迷惑じゃないかしら
\\	いいえ,~/是否打扰您了
\\	不,哪里		
\\	ノック	
\\	【英】
\\	敲打。敲门(ドアなどを叩くこと) 〜が聞こえる/听见敲门声 
\\	〈野球〉(为练习防守)打球(野球で、守備練習のために野球にボールを打ってやること)		
\\	どうぞ	【副词】 
\\	请(相手に頼む)(注:“请”は「恐れ入りますが」 「すみませんが」のように相手にていねいにものを頼むときに用いる) 
\\	请,可以(同意)		
\\	一人	【名词】 一人 〜虚を伝うれば万人実を伝う/ 一人传虚,万人传真 【名词】 一人,一个人(一個の人) 〜残らず/一个人也不剩	いちにん ひとり	
\\	週	【名词】 周,星期	しゅう	
\\	二回	【名词】 二回,二次(第2度)	にかい	
\\	小学校	【名词】 小学(校)(義務教育として始めの六年間の課程を教育する学校)	しょうがっこう	
\\	三年生	【名】 三年级	さんねんせい	
\\	活発	【形容动词/な形容词】 活泼,活跃(生き生きとして元気のよい様子)	かっぱつ	活溌
\\	経験	【名词・他动・三类】 经验,经历	けいけん	
\\	又	【名词】 又。另,别,他 【副词】 
\\	又,再,还(もう一度。再び) 
\\	也,亦『書』(同様で。やはり) 
\\	同时;也	また	
\\	小遣い	【名词】 零钱,零用钱(雑費に当てる金銭。また、自由に使える私用の金銭)	こづかい	
\\	稼ぐ	【自动词・五段/一类】 拼命劳动(生業に励む。精を出して働く) 【他动词・五段/一类】 
\\	挣钱(働いて収入を得る) 
\\	赚钱(利益を得る。儲ける) 
\\	取得(分数等)(点数・得点などを努力して得る)	かせぐ	
\\	信号	【名词】 
\\	信号,暗号(離れた二者以上の間において、定められた符号によって互いに医師を通ずる方法。合図。シグナル) 
\\	红绿灯;信号器(交通整理のための合図をする機械)	しんごう	
\\	目を閉じる	【惯用句】 闭上眼睛	めをとじる	
\\	診る	【他动词・一段/二类】 诊察,看(病)(診察する) 医者が患者を~/大夫给病人看病	みる	
\\	何度	【名词】 
\\	多少次,几次(何回) 
\\	多少度,几度(温度、角度、緯度、経度などの知れないときに言う語)	なんど	
\\	返事	【自动词・サ变/三类】 
\\	答应,回答,回话(質問や呼びかけなどに対して答えること。また、その言葉) 
\\	回信,复信(相手の手紙に対する答えの手紙)	へんじ	
\\	故障	【自动词・サ变/三类】 
\\	故障,事故;障碍;毛病(差しさわり。異常) 
\\	异议,反对意见(異議)	こしょう	
\\	仕上げる	【他动词・一段/二类】 完成(完成。仕事を終える)	しあげる	
\\	申し込み	【名词】 
\\	提议,提出要求(特定の契約を締結しようとする意思表示。相手方の承諾によって契約は成立する) 
\\	申请;应征,报名(申し込むこと。また,その手続き) 
\\	预约〔予約)	もうしこみ	
\\	締め切り	【名词】 
\\	截止。期限,最终期限(期日・時限などを定めて、事務の取り扱いなどを打ち切ること。また、その期日・時限) 募集の~の日。/征集的截止日期2. 关闭(「閉め切り」とも書く。戸・窓などを、閉じたままにすること。また、閉じられたままの場所)	しめきり	
\\	事情	【名词】 事情,缘故,情由,情况。事物转变为某种状态的理由或原因,亦指其结果使事物处于何种状态下(物事がある状態になった訳や原因。また、その結果、物事が今どのような状態にあるかということ)	じじょう	
\\	睡眠を取る	【惯用】 保证(保持)睡眠(时间)	すいみんをとる	
\\	一時	【名词】 
\\	某时,某时期,一段时间(過去のあるとき) 
\\	当时,那时,一时(その時) 
\\	一时,暂时,临时,短时间(いっとき〕 
\\	一点钟(時刻の名の一) 
\\	一次(一回。一度)【名词】 
\\	一时,短时间,暂时(少しの間。しばらく。片時) 
\\	一个时期,某一时期(ある一時期) 
\\	同时,一下子,一口气〔いちどき〕【名词】 
\\	一会儿,一时,暂时,片刻(しばらくの間。いっとき) 
\\	某时,有个时候(以前のある時。ひところ。いっとき) 
\\	一个时辰。古时的时间单位,等于现在的2小时(一時間。昔の時間区分で今の約2時間)	いちじ いっとき ひととき	
\\	ストップ	【名・自他・サ变/三类】 
\\	【英】
\\	止,停止,中止(停止。止まること) 
\\	停止信号(停止信号) 
\\	停车站(駐車場) 
\\	音栓(オルガンなどの音栓)		
\\	繋がる	【自动词・五段/一类】 
\\	被系在…上,绑;被…缠住(つなぎ止められる) 
\\	牵连,牵涉;株连,有(血缘,亲属)关系(関係がある。関わりある。結び付く) 
\\	排列,排队(並ぶ) 
\\	连接,连系(間が離れずに続く。連なる)	つながる	
\\	国境	【名词】 国境,边境,边界(国家と国家とのさかい。国家主権のおよぶ限界線)	こっきょう くにざかい	
\\	身に付ける	【惯用句】 
\\	带在身上(体に密着させるように所持する) 
\\	穿衣服(衣服を着る) 
\\	掌握(知識・学問・技術などを習得する) 
\\	穿在身上(着たり、履いたりする) 
\\	持有,携带(携帯する。所持する) 
\\	学会,掌握	みにつける	
\\	キムチ	【名词】 【朝】
\\	〈料理〉(拌有辣椒、姜、蒜等的)腌白菜,朝鲜泡菜(朝鮮の漬物の総称)		
\\	足す	【他动词・五段/一类】 
\\	加(すでにあるものの上にさらに加える) 
\\	添,续,补上(足りない分を補う) 
\\	办事,办完(必要なことをやる)	たす	
\\	引く	【自他・五段/一类】 
\\	撒(手),脱(身),摆脱(引き下がる。縁を切って別れる) 
\\	拉,曳,牵,拖(自分の方へ近づける。引いて前に進む。引っぱる。引きずる) 
\\	围上,拉上(張る) 
\\	领,带,引导,拉手(導く) 
\\	引诱,吸引,招惹(誘い入れる) 
\\	引进(管、线),安装(自来水等), 架设(电线等)(取りつける) 
\\	查(字典)(探す) 
\\	拔出,抽(签)(抜く) 
\\	引用(词句),举(例)(引用する) 
\\	减去,削减,扣除(減らす) 
\\	减价(値引きする) 
\\	拉长,拖长,拉(丝)(長く伸ばす) 
\\	涂,敷,涂上一层(塗りつける) 
\\	继承(血统),遗传,随(継ぐ) 
\\	画(线),描(眉),制(图)(描く) 
\\	抽回,收回(手、脚)(引っこめる) 
\\	撤退,后撤,拉回来(後へ戻す) 
\\	患,得(患う) 
\\	退,后退,落,减退,消失(しりぞく。後へさがる) 
\\	辞去,辞退(その地位から身をしりぞける。引退する)	ひく	
\\	鈍る	【自动词・五段/一类】 
\\	变弱;动摇(力や勢いなどが弱くなる) 
\\	变钝,不快(鋭さがなくなる、にぶくなる) 
\\	变迟钝(力や勢いなどが弱くなる。頭のはたらきや腕前などが衰える)【自动词・五段/一类】 
\\	钝,不快(刃物の切れ味が悪くなる) 
\\	变迟钝,笨拙(修錬の不足や老齢などのために技能や勢いが低下する)	にぶる なまる	
\\	時給	【名词】 计时工资(1時間あたりの給料)	じきゅう	
\\	収入	【名・他动・三类】 收入,所得(入って来るお金)	しゅうにゅう	
\\	何時の間にか	【惯用句】 不知什么时候,不知不觉(わからないうちにある状態になるさま。いつとはなしに。いつのまに。何時とはなく) 火は~消えた/火苗不知何时灭了	いつのまにか	
\\	半ば	【名词/副】 
\\	中央,中间 
\\	半途,中途 
\\	半。一半 
\\	半…半…某种状态达到了一半 
\\	一半相当的程度(かなりな程度。ほとんど)	なかば	
\\	冷える	【二类】 
\\	变冷,变凉。放凉(気温や温度が下がって冷たくなる。また、そう感じる) 
\\	〔愛情などが〕冷淡下来,变冷淡	ひえる	
\\	予定	【名・他动词・サ变/三类】 预定(これから行う事柄についてあらかじめ決めておくこと。前もって見込んでおくこと)	よてい	
\\	話し合う	【自动词・五段/一类】 
\\	谈话,对话(互いに話す。語らう) 
\\	商量,商谈;协商,谈判(相談する。協議する)	はなしあう	
\\	積もり	【名词/连用形/ます形】 
\\	打算,意图,企图(前もってもっている考え。意図。心ぐみ) 
\\	估计,预计(予想して计算すること。見積もり) 
\\	就当作,就算是(実际はそうでないのに、そうなったと仮定した気持ち) 
\\	宴会时最后一杯(瓶)酒(酒宴で、その酌で终わりにすること。おつもり)	つもり	
\\	そこで	【接续】 
\\	那么;可是。表示转移话题的用语((話題を転ずる意)ところで;さて) 
\\	于是;因此;所以(それゆえ。それで)		
\\	雪国	【名词】 
\\	多雪的地方,雪乡,雪国 
\\	川端康成的小说	ゆきぐに	
\\	メートル	【名词】 
\\	【仏】
\\	(长度单位)米(長さの単位) 
\\	仪表,计量器(計量器。メーター) 
\\	【仏】
\\	主人,管家;艺术大师;教师,先生(主。芸術家。先生)		
\\	積もる	【自他・五段/一类】 
\\	积,堆积(積み重なる) 
\\	累积,积攒,积存(たまる)	つもる	
\\	正月	【名词】 
\\	正月。新年(一年の最初の月。睦月。三が日) 
\\	过年似的热闹愉快	しょうがつ	
\\	等	【名词】 
\\	等(等しい) 
\\	等,等级;[賞]等奖,…名(順位。階級。クラス) 
\\	等,等等,诸如此类(同種のものを並べあげて、その他にもまだあることを表わす語) 【副助词】 
\\	同类列举或以不仅限于此的含意表示列举。等等。之类。什么的(同じようなものごとをならべあげて示す。また、それだけに限らない、という意味で、例示することを表わす) 
\\	表示没什么了不起的谦虚或轻视的心情。像…这样的。(あまりたいしたものではないという、謙遜や軽蔑の気持ちを表わす。なんか) 
\\	表示加重否定。决不…(打ち消す気持ちを強めて表現する) 【接头/接尾】 
\\	们,一些(〔人に関する名詞に付いて〕複数を表わす) 
\\	等,一些(人を表す名詞や代名詞に付いて、親愛・謙譲・蔑視の気持を表す)	とう など ら	
\\	その通り	【副词】 正是那样,就是(仰るとおり。当り。ご明察)	そのとおり	
\\	出来るだけ	【副词】 尽量地;尽可能地(できる範囲のことはすべて。できる限界まで)	できるだけ	
\\	協力	【名・自动词・サ变/三类】 协力,协作,合作,共同努力	きょうりょく	
\\	共	【名词】 
\\	一起(一緒) 
\\	全部。包括(全部。含めて) 【名词】 
\\	们,表示多数(複数化の接尾語) 
\\	表示谦逊的说法	とも ども	
\\	試合	【名词/自动·三类】 比赛(武芸・スポーツなどで,力の優劣を競い合うこと)	しあい	
\\	相手	【名词】 
\\	伙伴;共事者(一緒にする人。相棒。仲間) 
\\	对方;对手,敌手(争いなどで、対抗する人) 
\\	对象(働きかける対象)	あいて	
\\	凡そ	【名】 大概,概略(あらまし。おおよそ) 【副】 
\\	大概,大体上,大约(大体のところ。約) 
\\	凡,凡是,一般地(話を切り出すときに用いる。そもそも、一体) 
\\	完全,全然,根本。多用于否定(まったく)	およそ	
\\	書類	【名词】 文件;档案;资料(かきつけ・文書の総称。事務や記録の文書)	しょるい	
\\	ベル	【英】 
\\	铃,电铃;钟(呼びリン。電鈴。西洋風の鐘)		
\\	人数	【名词】 
\\	人数;[頭数]人头(人の数) 
\\	人数众多(多くの人) 【名词】 
\\	人数(人間の数) 
\\	(算作一个)人(人並み)	にんずう/にんず ひとかず	
\\	手狭	【名词·形动】 (房间等)狭窄(暮らしたり仕事をしたりするには場所が狭いこと。そのさま)	てぜま	
\\	遣って来る	【惯用句/自动词・サ变/三类】 来,到来	やってくる	
\\	実家	【名词】 父母之家,老家;娘家;婆家(その人の生家)	じっか	
\\	無理	【名・形容动词/ナ形容词・自动词・サ变/三类】 
\\	无理,不讲理(道理に反すること。筋道の通らないこと) 
\\	强干,强行,硬逼,不顾一切地干(行うのが難しいこと) 
\\	难以办成,办不到,勉强(困難を承知で強引に行う・こと)	むり	
\\	合格	【自动词・サ变/三类】 
\\	及格,考上(入学試験·採用試験·資格試験などに及第すること) 試験に~/考试及格 
\\	合格(ある条件や格式に適合すること)	ごうかく	
\\	郵送料	【名】 邮费	ゆうそうりょう	
\\	船便	【名词】 
\\	通航,通船,便船。船舶的班次(船舶の便。便船。せんびん) 
\\	海运,海上运输,用船邮寄	ふなびん せんびん	
\\	輸入品	【名词】 进口商品;进口货(外国から輸入した物品)	ゆにゅうひん	
\\	謝り	【名词】 谢罪,道歉(わびること。わび。謝罪)	あやまり	
\\	雪だるま	【名词】 雪人;堆雪人(雪を固めて達磨の形にこしらえたもの)	ゆきだるま	
\\	利く	【自动词・五段/一类】 
\\	灵敏,好使。充分或超常发挥其部分的功能(ある部分の機能が十分に、あるいは普通以上に発揮される) 気が〜/机灵 ブレーキが〜ない/刹车不灵 
\\	能够,经得住(…ができる…に耐えられる) 無理が〜ない/不能勉强 見晴らしが〜/能眺望很远 
\\	《相关惯用句》 口を〜/说话 無駄口を〜/闲聊 間に立って口を〜/居间斡旋	きく	
\\	留守	【名词】 
\\	看家,看门(主人や家人などの外出中、その家を守る。留守居) 
\\	不在家(外出し家にいないこと) 
\\	忽略;开小差,思想不集中(ほかのことに気を取られて、すべきことをおそろかにすること)	るす	
\\	行事	【名词】 仪式;活动(一定の日を決めて行う儀式や催し)	ぎょうじ	
\\	存在	【自动词・サ变/三类】 
\\	存在,有(人や物事があること、いること。また、その人や事物) 
\\	存在的理由,存在的意义	そんざい	
\\	文化的	【形容动词/な形容词】 文化(的),文化上(的)。(文化にかかわりのあるさま。文化を取り入れているさま。文化にかなっているさま。)	ぶんかてき	
\\	前年	【名词】 前一年,上一年。此前的一年。前年,前几年,前些年。以前的年月,过去的某年(前の年。以前の年。過ぎ去ったある年。先年)	ぜんねん	
\\	無事	【名词/形动】 
\\	平安,太平无事 
\\	健康(健康で元気なこと。つつがないこと) 
\\	最好,没毛病(何もしないこと) 
\\	没有过失(過失や事故のないこと) 
\\	无聊,闲散(なすべき事がないこと。ひまな状態)	ぶじ	
\\	祝う	【他动词・五段/一类】 
\\	祝贺 
\\	送贺礼(物を贈る) 
\\	祝愿,祝福(将来を幸福を願い、めでたい言葉を述べる)	いわう	
\\	元日	【名词】 元旦(1年の最初の日)	がんじつ	
\\	祝日	【名词】 节日。庆祝的日子,尤指国家规定的休息日(祝いの日。特に国が定めた祝日)	しゅくじつ	
\\	三が日	【名词】 正月的头三天(正月の元日から三日まで)	さんがにち	
\\	準備	【名・他动词・サ变/三类】 准备,预备;筹备(物事をする前に、あらかじめ必要なものをそろえたり態勢を整えたりして用意をすること)	じゅんび	
\\	大掃除	【名・他动词・サ变/三类】 大扫除,大清扫(平常は手のとどかない家の隅々までも、大規模に掃除すること)	おおそうじ	
\\	知人	【名词】 相识(知りあい);熟人(親しい人)	ちじん	
\\	年賀状	【名词】 贺年片,贺年信。为庆祝新年而寄出的明信片、书信(新年を祝って書き送る葉書・手紙。賀状、年始状)	ねんがじょう	
\\	飾り	【名词】 
\\	装饰;装饰品(かざること。よそおい。装飾。装飾品。転じて、見かけをととのえるためだけにあるもの) 
\\	徒饰表面,华而不实;粉饰,摆设(表面だけの美しさ。虚飾)	かざり	
\\	盛大	【形容动词/な形容词】 盛大;隆重;宏大(集会・儀式などが、盛んで大きいさま。極めて盛んなこと)	せいだい	
\\	お年玉	【名词】 新年礼物,压岁钱(新年を祝って贈る金品。主に子供や目下の者などに贈る)	おとしだま	
\\	与える	【他动词・一段/二类】 
\\	给与,给予;供给(授ける) 
\\	使蒙受,使遭到;使承受((影響・効果などを)相手に蒙らせる;(特別の配慮を)相手に施す) 
\\	分配;布置((仕事・課題などを)課する。宛がう)	あたえる	
\\	日頃	【名词/副】 平时,素日,平常(ふだん。へいぜい) 〜の行い/平时的行为	ひごろ	
\\	年始	【名词】 
\\	年初(年の初め) 年末~の休暇/年终年初假期 
\\	贺年,拜年(年賀)	ねんし	
\\	場合	【名词】 
\\	场合,时候,情况(その時の事情や状況。局面) 
\\	情况,状态(ある状況になった時)	ばあい	
\\	町中	【名词】 市内,街里(住宅や商店が集まっている所;町の中で人家·商店が立ち並んでいる所)	まちなか まちじゅう	
\\	ムード	【名词】 【英】
\\	心情,心绪,情绪,气氛(気分。情調)		
\\	盛り上がる	【一类】 
\\	鼓,凸起,隆起(盛ったように高くなる) 
\\	涌起,涌上来(わきあがる) 泡が〜/往上冒泡 
\\	(气氛)热烈,高涨起来(物事の勢いが高まる。興趣が高まる) 世論が〜/舆论沸腾	もりあがる	
\\	普段	【名词・副词】 平素,平时,平常,平日,日常(平生。平常) 〜はじょうぶだ/平时身体很好 〜の状態に戻る/回到正常的状态	ふだん	不断
\\	宅	【名词】 
\\	家,住所,居处(住居。住み家) 
\\	舍下,本宅,我(们)家(自分の家。自宅) 
\\	丈夫,爱人(妻が他人に対して自分の夫をいう語)	たく	
\\	急度	【副词】 一定;必定;锐利;严峻(必ず、ぜひ) 〜彼は来たくないはずだ/他肯定不愿意来	きっと	きっ度
\\	旧暦	【名词】 旧历;阴历;农历(1872年(明治5)の太陽暦採用以前に使用されていた暦法)	きゅうれき	
\\	西暦	【名词】 西历,公历(キリストが誕生したとされる年を元年とする年代の数え方。西紀)	せいれき	
\\	明ける	【自动词・一段/二类】 
\\	明,亮(明るくなる。夜が終わって朝になる) 
\\	过年(日や年があらたまる)年が〜とすぐ新しい仕事を始める予定だ/预定一过了年就开始新的工作 
\\	结束,终了;满期(期限が満了する。終わる) 年期はこの月末で〜/学徒期本月底就要满了	あける	
\\	纏める	【他动词・一段/二类】 
\\	集中(ばらばらあだった物を一つの整った状態にする) 経験を~/总结经验 
\\	完结(完成する) 
\\	做成,完成(望ましい状態に成就させる) 論文を~/完成论文	まとめる	
\\	現実的	【形容动词/な形容词】 现实的,实际的(考え方などが現実に即しているさま)	げんじつてき	
\\	繰り返す	【他动词・五段/一类】 反复;重复(同じことをもう一度、或いは何度もする。反復する) 歴史は決して~ことはない/历史决不可能重演	くりかえす	
\\	テーマ	【名词】 
\\	【德】
\\	中心思想,主题,题目(行動や創作などの基調となる考え) 
\\	【德】
\\	(音乐中的)主题音乐,主旋律。(楽曲の主旋律など)		
\\	発表	【名词・他动·三类】 发表;发布;宣布,揭晓(世間へ表向きに知らせること。広く知らせること)	はっぴょう	
\\	グループ	【名词】 
\\	【英】
\\	根据共通点组成的集团(似通った点によって分ける人(物)の集まり) 
\\	【英】
\\	共同行动的集合,伙伴,企业组织的通称((その人と)行動を共にする集団。仲間。企業集団の通称にも用いられる)		
\\	調査	【名・他动词・サ变/三类】 查,调查(ある事項を明確にするためにしらべること。とりしらべ)	ちょうさ	
\\	ダイエット	【自动词・サ变/三类】 (英)
\\	特种饮食。食物疗法。减肥		
\\	植える	【他动词・一段/二类】 
\\	栽。植(その根を土の中に埋める) 
\\	嵌入(固定して立てる) 
\\	接种。培育(他から移して育つようにする)	うえる	
\\	山水画	【名词】 山水画	さんすいが	
\\	似る	【自动词・一段/二类】 像,似(互いに同じ様に見える)	にる	
\\	心掛け	【名词】 留心;注意;品行;品德;作风;努力(平素からの心の持ち方) ~がよい/心地善良的	こころがけ	
\\	詰まらない	【形容词/い形容词】 
\\	无聊的,无趣的;没有意思的〔興味が湧かない。退屈だ〕 
\\	没用;没有意义,胡乱;不起劲〔取るにたりない。言うに値しない。無意味だ〕 
\\	没有价值的,不值钱的〔価値がない〕 
\\	不合算的,吃亏的;愚蠢的〔馬鹿馬鹿しい。役だたない。意味がない。張り合いがない)	つまらない	
\\	お歳暮	【名词】 年末互相馈赠的礼物。如果是初夏赠送的话则叫做“お中元「おちゅうげん」”(年末にお世話になった方々に贈るもの。 初夏に贈る同様なものはお中元と言う)	おせいぼ	
\\	挨拶	【自动词・サ变/三类】 
\\	寒暄语。与人见面或分别等时的寒暄及其礼貌动作。打招呼 
\\	贺辞或谢辞 
\\	回答,回话(受け答え。返事) 
\\	对对方的态度表示无可奈何时讥讽的说法 
\\	(对争议的)仲裁。调解。斡旋(仲裁) ~切(き)る/绝交。与人断绝关系	あいさつ	
\\	期末	【名词】 期末(ある一定期間の終り)	きまつ	
\\	案内書	【名词】 旅行指南,参考手册(事情やようすなどを紹介する説明書。旅行・遊覧などの手引き書。ガイドブック)	あんないしょ	
\\	汚れる	【自动词・一段/二类】 脏了(きたなくなる。汚らわしくなる) 【自动词・一段/二类】 
\\	污染(汚くなる) 
\\	失去神圣性(名誉・名声などに傷が付く) 
\\	失去贞操(女子の貞操が傷つく) 
\\	在服丧期、产期、经期等忌讳朝拜神佛	よごれる けがれる	
\\	事故	【名词】 
\\	事故(思いがけが起こった悪い出来事) 
\\	事由(事柄の理由)	じこ	
\\	貼る	【他动词・五段/一类】 
\\	粘,贴,糊(のりなどで、くっつける) 
\\	钉上去(板などを打ちつける)	はる	
\\	集まり	【名词】 
\\	集会,聚会(会合。寄り合い) 
\\	聚集,集合,集体。汇集,亦指汇集在一起的人或物(集まること。また、集まったもの)	あつまり	
\\	家庭的	【形容动词/な形容词】 
\\	家庭式的,家庭型的(家庭にいる時のような、うちとけた気分にひたれるさま) 
\\	顾家的,念家的(家庭生活に向いているさま。また、家庭を大事にするさま)	かていてき	
\\	旅	【名词】 旅行(住む土地を離れて、一時他の土地に行くこと) 海の~/海上旅行 空の旅~/坐飞机旅行 ~に出る/出去旅行 ~は情け人は心/出门靠情谊,处世讲诚意	たび	
\\	叶う	【自动词・五段/一类】 能实现,能如愿以偿(願望が実現する。思うようになる)	かなう	
\\	表現	【名・他动词・サ变/三类】 表现;表达(心的状態・過程または性格・志向・意味など総じて内面的・精神的・主体的なものを、外面的・感性的形象として表すこと。また、この客観的・感性的形象そのもの、すなわち表情・身振り・動作・言語・作品など。表出)	ひょうげん	
\\	分ける	【他动词・一段/二类】 
\\	分,分开(一つの物をいくつかの部分にする) 
\\	分类,区别(沢山ある物を、基準によって区別する) 
\\	分配,分派(一部分を人に与える。配る。分配する) 
\\	调停,仲裁,排解(争いの仲裁をする。また、引き分けにする) 
\\	向左右拨开,分开(通るために、邪魔者を押し開く) 
\\	辨明是非,判断对错(理非を区別する。判断する。弁別する。判別する)	わける	
\\	日常	【名词】 日常,平时(ふだん;つね)	にちじょう	
\\	実に	【副词】 
\\	实在,确实,的确,真(ほんとうに。うたがいなく) ~美しい/非常美丽 
\\	竟,仅仅,足(まったく。非常に) 【副词】 确实,诚然,实在(まことに。いかにもそのように)	じつに げに	
\\	スムーズ	【形容动词/な形容词】 【英】
\\	光滑,平滑;流畅,顺畅(物事が支障なく滑らかに運ぶさま)		
\\	謝る	【他动词・五段/一类】 
\\	道歉,赔礼,谢罪(自分の非を認め、相手に許しを認める、詫びる) 被害者に~/向受害者道歉 
\\	认输,折服,服,没有办法(閉口する、恐れ入る) 
\\	谢绝,辞退(閉口して断る、ごめんこうむる)	あやまる	
\\	問う	【他动词・五段/一类】 
\\	问,询问(聞く) 道を〜/问路 
\\	调查,追究(聞きただす) 
\\	当做问题(問題にする)	とう	
\\	感謝	【名词・自他动·三类】 感谢(ありがたく感じて礼を表すこと)	かんしゃ	
\\	親しみ	【名词】 
\\	亲密感情,亲近,亲切(〔動詞「親しむ」の連用形から〕親しく思う気持ち。親近感) 
\\	爱情;亲密〔愛情。親近感〕 ~を覚える/感到爱情;感到亲密	したしみ	
\\	親しむ	【自动词・五段/一类】 
\\	亲近,亲密,接近(親近感をもつ) 友と~/和朋友亲近 
\\	爱好,喜好 スポーツに~/喜好运动 
\\	接近,不离。持续不断地,职业般地做某事 医薬に~/经常吃药,离不开药	したしむ	
\\	湧く	【自动词・五段/一类】 
\\	涌出,冒出,喷出(水などが地中から噴き出る) 温泉が~/温泉涌出 
\\	涌现,产生,发生(ある考えや感情が生じる) 歴史に興味が~/对历史产生兴趣 
\\	大量涌现,孳生(虫などが、一時に発生する) うじが~/生蛆	わく	
\\	人間関係	【名】 人际关系;人与人的关系(人間同士の勢力の配分のぐあいやかかわり方) 会社での~がうまくいかない/跟同事的关系老搞不好	にんげんかんけい	
\\	言い方	【名词】 
\\	说法(ものの言いよう) ~がまずい/说话不得体 
\\	表现方法(言葉づかい)	いいかた	
\\	気を配る	留神,照顾,注意,警惕(方々に注意を払う。あれこれ気を遣う) 細かいことに~/注意细节 【惯用句】 
\\	留神(さまざまに注意を払う。配慮する) 
\\	关心(気配り) 常に他人に~/经常关心他人	きをくばる	
\\	通り	【名词】 
\\	流通;通(往来、行き来、通行、通過) 
\\	道路,大街;马路 
\\	声音传送,声音传递方式 
\\	声誉,名声;评价 
\\	通顺;易懂;理解 
\\	同样;原样;照样,表示与前面所述状态相同 【量词】 计算组、种类或次数等的量词	とおり どおり	
\\	童話	【名词】 童话(子供のために作られた話。昔からのおとぎ話や寓話などのほか、創作童話がある)	どうわ	
\\	達する	【自他・サ变/三类】 
\\	到,到达(到着する) 頂上に~/到达山顶 
\\	达,达到(届く) 
\\	达到,完成(やり遂げる)	たっする	
\\	駄目	【名词】 
\\	〈碁〉在终局时不属于任何一方的地方;单官(空所) 
\\	【形动】 白费,无用(むだ);无望(望みなし) 
\\	不行,不可以(いけない) 
\\	不好,坏(わるい状態・不適) 
\\	不行,不可能(できない)	だめ	
\\	過ち	【名词】 
\\	错误,错儿,失败〔まちがい〕 ~に気づく/发觉错误 
\\	过错,过失,罪过〔過失〕 ~を犯す/犯过;犯错误;弄错;失败	あやまち	
\\	過つ	【他动词・五段/一类】 弄错,做错,犯错误,失败,不留神(やりそこなう。過失を犯す)	あやまつ	
\\	認める	【他动词・一段/二类】 
\\	看见,看到(目にとめる) 
\\	认识,赏识,重视(みどころがあると考える。高く評価する) 
\\	承认(承認する) 
\\	断定,认为(見て判断する) 
\\	准许,同意(見てよしとする。かまわないとして許す。受け入れる。許可する) 【他动词・一段/二类】 
\\	书;写;书写(書類・手紙を書きととのえる。書きしるす。記す) 
\\	吃(食事する。食べる) 
\\	处置;处理;整理;准备	みとめる したためる	
\\	熱心	【名词/形动】 [对人或事物]热心;热诚;热情	ねっしん	
\\	姑	【名词】 婆婆,岳母(夫の母。妻の母)	しゅうとめ しゅうと	
\\	チンパンジー	【名词】 【英】
\\	〈動〉黑猩猩(ショウジョウ科の哺乳類。全身黒い毛に覆われ、ゴリラに似るが小形で、雄は背の高さ1.5メートルほど。サル類の中では最も知能の高いものの一種。熱帯アフリカの森林地帯にすみ、夜は木の上で眠り、昼間、果実、木の芽・葉、昆虫などを食べる。黒猩々(くろしようじよう)。広義には、別種のピグミー-チンパンジーを含む)		
\\	太る	【自动词・五段/一类】 
\\	胖,发福;肥(人や動物が肉や脂肪が付いて、体が太く、あつくなる。こえる) 
\\	发财,增加(増える。大きく、または多くなる)	ふとる	肥る
\\	家鴨	【名词】 鸭子,家鸭	あひる	
\\	鴨	【名词】 
\\	野鸭子(カモ目カモ科のうち、ハクチョウ類・ガン類・アイサ類を除いたものの総称。中形の水鳥。雄は派手な色合い、雌は地味な茶褐色のものが多い。マガモ・コガモ・オナガガモ・ハシビロガモなど。日本ではカルガモを除き、多くは冬鳥) 
\\	容易上当受骗的对手。大头,冤大头(勝負事などで、くみしやすい相手。また、だましやすい相手) 
\\	日本地名。贺茂。京都市鸭川流域的上贺茂、下鸭的总称(京都市鴨川流域の上賀茂・下鴨の総称)	かも	
\\	家畜	【名词】 家畜,牲口。人类为生活需要而饲养的动物,如牛、马、羊、猪等(人間が生活に役立てるために飼う動物)	かちく	
\\	包む	【他动词・五段/一类】 
\\	包,裹,包上,穿上(物を中に入れて、外から覆う) 
\\	笼罩,覆盖,隐没,沉浸(取り囲む、すっかり覆う) 
\\	藏在心里,隐藏,隐瞒(心の中に込めておく、秘める) 
\\	束起(紙などに包むことから、人に金品を贈る)	つつむ くるむ	
\\	品物	【名词】 物品,东西;实物;商品,货。货物(物品。特に商品。品)	しなもの	
\\	途中	【名词】 
\\	途中,路上(目的地に着くまでの間) 
\\	中途(物事が終わらないうち)	とちゅう	
\\	移る	【自动词・五段/一类】 
\\	移动(ものがある場所から他の場所へ置き換わる) 
\\	变心(人の心や関心の対象などが変わる) 
\\	染上(香りや色が他の物に染み付く) 
\\	感染(病気などほかに感染する) 
\\	时光流逝(時間が経過する) 
\\	转移(前と違った状態に変わる)	うつる	
\\	遠く	【名词・副词】 
\\	远方,远处(とおい所) 
\\	远远。距离、差距很大(はるかに)	とおく	
\\	都会	【名词】 都会,都市,城市(人々が多く集まり住んで、商工業や文化の発達した土地)	とかい	
\\	大分	【副词】 
\\	数量多,相当(たくさん。だいぶん) 小遣いを~もらった/得到很多零花钱 
\\	不少,大为,大有(かなり。だいぶん) ~寒くなった/已经很冷了	だいぶ だいぶん	
\\	偉い	【形容词/い形容词】 
\\	伟大;卓越;了不起(すぐれている。人に尊敬されるべき立場にある。卓越している。すごい) 
\\	高;高贵(地位・身分が高い) 
\\	严重,厉害,不得了(普通あるべき状態より程度が甚だしい。程度や情況がひどい) 
\\	吃力,累人,劳累(仕事などが大変だ)	えらい	豪い
\\	日々	【名词】 天天,每天(毎日。にちにち。一日一日) ~の生活/每天的生活 ~の仕事/每天的工作	ひび	日日
\\	続く	【自动词・五段/一类】 
\\	持续;连续;连绵(物事が途切れずに同じ状態が保たれる) お天気が~/连续是好天气 
\\	接连发生;连续不断(物事が間を置かずに起きる) 
\\	紧连着;接着;跟着(引き続き) 
\\	坚持;维持;支撑得住(維持する)	つづく	
\\	立派	【形容动词/な形容词】 
\\	漂亮,美观,美丽,华丽(美しいこと) ~な邸宅/漂亮的宅邸 
\\	壮丽,宏伟,盛大(規模が大きくて、見事なこと) ~な儀式/盛大的仪式 
\\	优秀,出色,杰出,卓越(優れていること。非難すべき点が見出せない様子) ~な青年/优秀的青年 
\\	庄严,堂堂(公明正大) 
\\	〔気高い〕高尚,崇高(崇高)	りっぱ	
\\	一歩	【自动词・サ变/三类】 
\\	一步,走一步(ひと足) ~も動くな/一步也不准动 
\\	一步,一个阶段,稍许,一点儿(ほんの少しの程度) 
\\	向…进一步(あゆみ)	いっぽ	
\\	振り出す	【他动词・五段/一类】 
\\	摇出,晃出,簸出。摇晃容器使里面的东西出来(容器を振って中にある物を出す) おみくじを~/摇出签 
\\	开始摇,摇晃起来,开始簸动(振りはじめる) 鈴を~/摇铃铛 
\\	开票,发票,出票。开出汇票、票据、支票等(為替・手形・小切手を発行する)	ふりだす	
\\	喜び	【名词】 
\\	喜悦,愉快,高兴,欢喜(よろこぶこと) ~にたえない/不胜高兴〔愉快〕 
\\	祝贺,道喜(祝い) 
\\	喜事,喜庆(めでたいこと) 結婚の~/新婚之喜	よろこび	
\\	充実	【名/自サ3段】 充实(中身がいっぱい入っていること。内容が満ちて豊富なこと) 気力~/精力充沛 ~した知識/丰富的知识	じゅうじつ	
\\	伝える	【他动词・一段/二类】 
\\	传,传导〔伝導する〕 
\\	传〔後に残す〕 
\\	让,让给;传,传给〔譲り与える) 
\\	传授〔教えさずける〕 
\\	传达,转告,转达。〔知らせる〕;告诉,告知(言う)	つたえる	
\\	報告	【名・他动词・サ变/三类】 报告;汇报;告知(告げ知らせること。研究や調査の結果、与えられた任務の結果などについて述べること。また、その内容) 最終~/最后报告 事件のいきさつを~する/报告事件的经过	ほうこく	
\\	東北	【名词】 
\\	东北,东北方(東と北との中間の方角) 
\\	日本本州东北部地方(東北地方) 
\\	〔中国的〕东北(中国北東部の地区名)	とうほく	
\\	仙台	【名词】 日本地名,仙台市,宫城县中央的一个市,宫城县厅所在地(宮城県中央部にある市。県庁所在地)	せんだい	
\\	魯迅	【人名】 鲁迅
\\	中国的文学家,浙江人。本命周树人,鲁迅为笔名。曾在日本留过学,后来弃医从文	ろじん	
\\	文章	【名词】 文章;散文(思想や感情を表したひとまとまりの統一ある言語表現。ひとつもしくは複数の文から成るもの)	ぶんしょう	
\\	知識	【名词】 
\\	知识(ある物事について知っていること。また、その内容。知恵と見識) 
\\	名僧,高僧(仏道に教え導く指導者)	ちしき	智識
\\	曇る	【自动词・五段/一类】 
\\	阴天(雲が出て、空を覆う) 
\\	变模糊不清,朦胧(鏡、ガラスなどが光をよく通したり反射したりしなくなる。また、涙などで、かすんでみえる) 
\\	暗淡,忧郁不乐(不安、心配、悲しみなどで心がふさぐ)	くもる	
\\	到頭	【副词】 到底;终于;终究;结局(最終的な結果として物事が実現した、あるいは実現しなかったという意を表す。ついに。結局、最後に)	とうとう	
\\	エネルギー	【名词】 
\\	【德】
\\	能,能量,基本物理量之一,可有各种形态,也能相互转换,但总量不变,种类有尔格、焦耳、卡、电子伏等(物体がもっている、仕事をすることができる能力。また、その量) 
\\	【德】
\\	经历,气力,完成工作的身心活力(仕事や活動をするための力。元気、精力)		
\\	真っ赤	【名/な形容词】 
\\	通红,鲜红,极其红(非常に赤いこと。充血して赤いこと。また、そのさま) 
\\	纯粹,完全。不含杂质(全くそうであるさま。まるっきり)	まっか	
\\	グーグー	【拟音/副】 打呼噜,打鼾,形容酣睡		
\\	閉店	【自动词・サ变/三类】 
\\	关门,上板儿,收市,打烊,停止营业(その日の商売を終えて店を閉めること) 店は8時に~する/店铺八点关门 
\\	歇业,废业,倒闭〔廃業〕	へいてん	
\\	郊外	【名词】 郊外,城外(都市に隣接した地域。市街地周辺の田園地帯)	こうがい	
\\	ライバル	【名词】 
\\	【英】
\\	竞争对手(競争者、競争相手、好敵手) 
\\	【英】
\\	情敌(恋敵)		
\\	クラスメート	【名词】 【英】
\\	同班生,同班同学,同学(同じクラスで学ぶ生徒)		
\\	学位	【名词】 学位(学術上、一定の能力または業績を示した者に授与される称号。学士・修士と博士とがある。旧制では博士のみで、文部省の認可を受けて大学が授与していた)	がくい	
\\	大都会	【名词】 大城市。大都会(大きな都会。大都市)	だいとかい	
\\	世話	【名・他动词・サ变/三类】 
\\	帮助,帮忙,援助(手助けする) 彼の~をする/帮助他 
\\	照料,照顾,照应,照看,照管(面倒を見る) 病人の~をする/照料病人 
\\	推荐,周旋,斡旋,介绍(取り持つこと) 就職の~をする/介绍就业	せわ	
\\	職場	【名词】 工作单位;工作岗位(持ち場);车间(現場) ~に出て働く/到车间去工作 ~を守る/坚守工作岗位	しょくば	
\\	一応	【副词】 大致,大体;首先;暂且(十分といえないが、とりあえず;ひとまず;ともかく) ~準備はできた/大致准备好了	いちおう	
\\	面倒	【名/形动】 
\\	麻烦,费事,棘手(手数がかかること。物事をするのが煩わしいこと) 事件が~になってきた/事情麻烦起来了 ~な仕事/繁难的工作 
\\	照顾,照料(人の世話をすることを表す) 遺族の~を見る/照料遗属	めんどう	
\\	尽くす	
\\	2语法】 
\\	…光;…尽;…完(すっかり…、…する) 言い~/说尽 【他动词・五段/一类】 
\\	尽,尽力,力(能力、手段を全て使い切る) 全力を~/竭尽全力 
\\	为…尽力,报效,贡献力量(他のために献身的に努力すること) 国のために~/为国效力 
\\	达到极点(十分に表現し切る) 贅沢を~/极端奢侈	つくす	
\\	対する	【自动词・サ变/三类】 
\\	对,面对,面向(二つの物が向かい合う。ある物に向かう) 
\\	对于,关于(対象とする。関わる) 文学に~興味/对文学的兴趣 
\\	对待(人と応対する) 客に~時は丁寧に接待客人要适时地表示恭敬 
\\	对抗(敵として相手にする) 難敵に~/与难以制胜的对手对阵 
\\	对比,对照,相对,成双成对(比較する。対照される) 真に~偽/与真相对的伪	たいする	
\\	見知らぬ	【接续】 未见过的,不认识的(今まであったこともなく、知らない) ~人/陌生人 ~人に声をかけられる/一个陌生人向我打招呼	みしらぬ	
\\	他人	【名词】 
\\	别人(自分以外の人) ~はいざ知らず/别人怎样姑且不论 
\\	(没有血亲关系的)外人(血筋のつながらない人) 赤の~/毫无关系的人 
\\	局外者(そのことに関係のない人) ~扱い/当外人看待	たにん	
\\	認知	【名・他动词・サ变/三类】 
\\	(亲生父母)对非婚生子(私生子)的承认,认领(法律上の婚姻関係にない男女の間に生まれた子を、親が戸籍法の手続きによって、自分の子と認めること) 私生児~/私生子的认领 自分の息子であると~する/承认是自己的儿子 
\\	认定,认知,认识,识别。清楚地确认(それとしてはっきりと認めること) 目標を~する/认定目标	にんち	
\\	暗に	【副词】 暗中,私下,背地(はっきり口に出さずに、なんとなくにおわせるさま。それとなく) ~言う/背地里说	あんに	
\\	用	【名词】 
\\	事情(しなくてはならない事柄。用事) 人の~をする/给别人办事情 
\\	用途,用处(役に立つこと。はたらきをすること) 種々の~に供せられる/供多种用途 
\\	大小便(大小便をすること。用便) ~を足す/解手	よう	
\\	所属	【自动词・サ变/三类】 
\\	所属;附属(人やものが団体や組織に属していること) 政府~の機関/属于政府的机关 
\\	参加,加入(入る) テニス部に~する	しょぞく	
\\	未知	【名词】 未知,不知道(まだ知られていないこと) ~の世界/未知的世界 ~の人/不认识的人 そのことは~の領域に属している/那件事还属于未知范畴	みち	
\\	ほっと	【副词】 叹气或喘粗气貌(一息、息をつく様子) 【副词・自动词・サ变/三类】 放心貌;从担心、操劳中解放出来的安心状(それまでの心配や緊張などから解放されて、ああよかったと安心する様子) ~胸を撫で下ろす/松了口气 ~した/舒了一口气		
\\	都合	【名・他动词・サ变/三类】 
\\	准备;安排;设法;通融;挪用;调度 ~をつけて出席する/安排时间出席 
\\	障碍(差し支え) 
\\	机会;凑巧,顺利(与否) ~悪く、留守だった/不凑巧,(他)没在家 
\\	〔状況〕方便合适(与否) 
\\	(某种)情况,关系;理由,原因(具合)	つごう	
\\	勤め先	【名词】 工作地点;工作岗位(勤めているところ。勤務先。職場)	つとめさき	
\\	両国	【名词】 两国(二つの国。両方の国)	りょうこく	
\\	交流	
\\	交流;往来(ことなった所や組織などに属している人びとが互いにまじわること) 
\\	〈電〉交流(きまった時間ごとに、流れるむきが変化する電流。交流電流)	こうりゅう	
\\	盛ん	【形容动词/ナ形容词】 
\\	(气势)盛,强盛,强健(勢いの酣なこと) 雨が~に降る/雨下得很大 
\\	繁荣,昌盛(盛大に行われるさま。繁盛しているさま。また、広く行われるさま) 工業の~な町/工业繁盛的城镇 
\\	热心,积极(積極的に繰り返し行われるさま。熱心)	さかん	
\\	誇り	【名词】 骄傲,荣誉,自豪(誇ること。名誉に感じること。また、その心) ~を感ずる/感到自豪	ほこり	
\\	架け橋	【名词】 
\\	吊桥、栈桥、栈道。沿着险峻的山崖用木头和藤蔓等搭设的像棚架一样的通道 
\\	吊桥、浮桥、渡桥。在山谷或河流上临时搭起的桥 
\\	媒介、媒人。使两人间建立起某种关系,也指其人(なかだち)	かけはし	懸け橋 掛け橋
\\	努力	【自动词・サ变/三类】 努力,奋勉(ある目標を現実させるために一生懸命につとめ励むこと) ~の人/很努力的人	どりょく	
\\	必要	【名・形容动词/ナ形容词】 必需,需要(それがなくてはやっていけないこと、どうしてもしなければならないこと) 彼にぜひ知らせる~がある/必须通知他	ひつよう	
\\	硬い	【形容词/い形容词】 
\\	硬 
\\	坚实,坚固(崩れにくい) 
\\	坚定;坚决 
\\	紧;用力 
\\	可靠;正派;说话算数 
\\	生硬(かたくるしい) 
\\	顽固(頑固である) 
\\	拘谨,僵硬 
\\	严厉(厳しい。強い) 
\\	(想象)无差,(确信)无疑(確実である)	かたい	
\\	最新	【名词】 最新(最も新しいこと)	さいしん	
\\	データ	【名词】 【英】
\\	(作为立论根据的)事实,资料 【英】
\\	(输入计算机进行程序处理的,或从计算机中调出的)数据,形式有数字、文字和符号等		
\\	参考	【名·他动·三类】 参考;借鉴(考えをまとめる上で、いろいろなものを利用すること。また、そのための材料。また、自分の戒めとする) ~にする/作为参考;以资借鉴	さんこう	
\\	作品	【名词】 作品,创作,著作,艺术制品(製作物。主に、芸術活動によって作られた物)	さくひん	
\\	紛争	【自动词・サ变/三类】 纠纷;争端(事がもつれて争いになること。個人や集団の間で,対立する利益や価値をめぐって起きる行動や緊張状態をいう。もめごと)	ふんそう	
\\	効率	【名词】 效率(のうりつ) ~のよい機械/效率好的机器	こうりつ	
\\	欠席	【自动词・サ变/三类】 缺席;缺课(会合や授業などに出ないこと) 病気で学校を~する/因病不到校	けっせき	
\\	安全	【名・形容动词/ナ形容词】 安全,保险,平安(危険がないところ)	あんぜん	
\\	再建	【名・他动词・サ变/三类】 
\\	重新建筑,重新建造。重建失去了的建筑物 焼けた校舎を~する/重建烧毁的校舍 
\\	重新建设;重新建立。重建衰败的团体、组织、文明等	さいけん さいこん	
\\	辞意	【名】 词意,辞职之意(辞退・辞職したいという気持ち) ~を表明する/表明辞职之意	じい	
\\	漏らす	【他动词・五段/一类】 
\\	漏,漏掉;露出;洒。遗尿;尿床,尿炕(こぼす) 明かりを~な/别透亮光 
\\	遗漏 
\\	泄漏,走漏,透露 
\\	流露,发出,发泄(感情などを思わず外に表し出す) 不満を友人に~/向朋友吐露不满	もらす	
\\	急用	【名词】 急事。紧急的事情(急ぎの用件) ~ができる/有了急事	きゅうよう	
\\	喧嘩	【自动词・サ变/三类】 吵嘴,口角,吵闹,吵架;喧嚷,打架(言い争ったり腕力で争ったりすること。いさかい) 人と~をする/跟人吵架	けんか	
\\	停電	【自动词・サ变/三类】 停电,停止供电(送電画一時とまって、電灯が消えること) ~の知らせ/停电通知	ていでん	
\\	外食	【自动词・サ变/三类】 在外吃饭(食堂など家庭外で食事をとること。また、その食事) 間借りして~する/租一间房住,在外边吃饭(不包伙)	がいしょく	
\\	大量	【名・形动】 
\\	大量,多量,大批(量の多い・こと〔さま〕。多量) ~の品物/大量的物品;大批货物 
\\	大度量,大气量,宽宏大量『成』	たいりょう	
\\	文書	【名词】 文书,文件(文字で人の意思を書き表した物)	ぶんしょ	
\\	関係者	【名词】 有关人员,…工作者(ある事柄に関係のある人)	かんけいしゃ	
\\	少年	【名词】 少年 ~文学/儿童文学	しょうねん	
\\	中年	【名词】 中年(成人として中くらいの年齢。すなわち壮年期を過ぎたころから高年 期の域に入る前までを指す)	ちゅうねん	
\\	ゲーム	【英】
\\	竞技,比赛
\\	競技) 
\\	游戏(遊び);游艺;娱乐(道具を使う) 
\\	场;局,盘(試合数)		
\\	持ち掛ける	【他动词・一段/二类】 (主动)提出,先开口(相談などを切り出す。誘いかける) 縁談を~/提出婚事	もちかける	
\\	兎に角	【副词】 无论如何,不管怎样;总之;好歹,不论好坏;反正;姑且,姑且不论(不说) ~事実だ/反正是事实	とにかく	
\\	適当	【名・自动词・サ变/三类】 
\\	适当。适合。恰当。适宜(ある状態や目的などに、ほどよくあてはまること) ~な語を補え/填一下恰当的词 
\\	正好。恰当。适度(その場に合せて要領よくやること) 
\\	酌情。随意。随便。马虎。敷衍(いい加減)	てきとう	
\\	断る	【他动词・五段/一类】 
\\	不应允;拒绝〔相手の希望や申し入れを拒む。拒絶する〕 きっぱり~/断然拒绝 
\\	事先打招呼;预告〔事前に知らせる)	ことわる	
\\	提案	【名词・他动·三类】 提案;建议(議案や意見を提出すること。また、その議案や意見) 憲法改正について~する/就修改宪法提出议案	ていあん	
\\	ドル	【英】
\\	美元(アメリカ合衆国・カナダ・オーストラリアなどの貨幣単位。普通はアメリカ合衆国のドル
\\	を指す。これはアメリカ合衆国のドルが国際決済通貨・国際準備通貨として、世界で最も多く利用されてきたためである。ダラー。記号
\\	太陽	【名词】 
\\	太阳,日 ~が沈む/太阳落山 
\\	希望,期望 心に~を持つ/心中充满希望 
\\	1895年创刊、1928年废刊的日本综合类杂志。杂志名	たいよう	
\\	黙る	【自动词・五段/一类】 
\\	不说话,不作声,沉默(話さない);缄默(口をとじる) 急に~/突然沉默起来 
\\	不理(ほうっておく);不问不管(干渉しない)	だまる	
\\	見付かる	【自动词・五段/一类】 
\\	被看到,被发现(人にみつけられる。人の目にとまる) 
\\	能找出,找到(探しもとめていたものを見つけることに成功する)	みつかる	
\\	一言	【名词】 一言,一句话;三言两语(一つの言葉。いちごん。わずかの言葉。また、ちょっと言うこと) ~も言わない/一言不发 【自动词・サ变/三类】 一言,一句话(ひとつのことば。一語) あえて君に~する/大胆向你进一言 【自动词・サ变/三类】 一言,一句话	ひとこと いちげんいちごん	
\\	口にする	【惯用句】 
\\	吃,喝(口に入れる。飲み食いする。また口にくわえる) 
\\	说,讲(言葉に出して言う、話す) ~のも不愉快な話だ/说出口的话也是不愉快的话	くちにする	
\\	取り出す	【他动词・五段/一类】 拿出,取出;掏出;选出,挑出;抽出,取出(取って外に出す。あるものの中からぬき出す) かばんから書類を~/从皮包中拿出文件	とりだす	
\\	手渡す	【他动词・五段/一类】 面交,亲手交给,递给,传递(直接自分の手から相手の手に渡す) 順々に~/依次传递	てわたす	
\\	興味	【名词】 兴趣,兴味,兴致;兴头;兴会(気のり) ~がある/有兴趣	きょうみ	
\\	どんどん	【副词】 
\\	事务进展顺利,顺畅貌(物事を次から次へと積極的に行うことを表わす) ~売れる/非常畅销 
\\	连续不断貌(止めることが出来ないほど、物事の進行が勢い付いていることを表わす) 金を~使う/一个劲儿地花钱 
\\	咚咚;强烈的敲击声		
\\	即ち	【接续词】 
\\	于是〔そこで。そうして〕 ~引き返した/于是就回去了 
\\	即。正是,就是;换言之〔まさに。そのときは。つまり)	すなわち	
\\	柴	【名词】 柴,小杂树(山野に生える小さい雑木。また、それを折って薪や垣にするもの。そだ。しばき。ふし) ~を刈る/砍柴	しば	
\\	長持ち	【名·自サ】 
\\	耐久,耐用,持久(34物が長くもつこと) ~のする布/耐穿的衣料 
\\	带盖的长方形衣箱	ながもち	
\\	詰まり	【名词】 
\\	缩小,缩短(縮小すること) ~具合/缩水情况 
\\	终点,尽头(果て;終わり) 【副词】 
\\	归根到底。总之(結局。要するに) ~、失敗だった/总之是失败了 
\\	就是说。用以加强语气,即(強めのために使う。すなわち)	つまり	
\\	有益	【な形容词】 有益,有意义,有好处(利益のあること。ためになること。また、そのさま) ~な話/有益处的话 金を~に使う/花钱花得有意义	ゆうえき	
\\	誘う	【他动词・五段/一类】 
\\	促使;引起;引发(ある行為をするような気分にさせる。促す) 
\\	劝诱;邀请;约(自分と同じ行動をとるように勧める) 
\\	引诱;诱惑(そうする(なる)ように働きかける) 悪事に~/引诱做坏事 【他动词・五段/一类】 邀请(さそう。さそい連れて行く) 自宅に~/请到家来	さそう いざなう	
\\	誤魔化す	【他动词・五段/一类】 
\\	欺骗,欺瞒,蒙混,愚弄〔だます〕 年を~/隐瞒年龄;谎报年龄 
\\	蒙蔽,掩盖,搪塞,敷衍〔とりつくろう) 事実を~/掩饰事实 
\\	弄虚作假,捣鬼;舞弊;侵吞〔いんちきをする) 勘定を~/报假帐	ごまかす	
\\	済ませる	【他动词・一段/二类】 做完,完成(すます) 夕食を~/吃完晚饭	すませる	
\\	お互いに	【副词】 彼此,互相(双方が同じようなことをしあうさま) ~助け合う/互相帮助	おたがいに	
\\	向ける	【他动词・一段/二类】 
\\	向(対象が正面にあるように回す) 水を~/引诱;刺探 
\\	派遣(派遣する) 代理の者を~/派代理人去 
\\	归入,挪用(ふりあてる) 工業用に~/划归工业使用 
\\	前往(赴く)	むける	
\\	責める	【他动词・一段/二类】 
\\	责备,责难(咎める〕 人の怠慢を~/责备别人的怠慢 
\\	苛责,折磨,使对方感到痛苦(苦しめる) 
\\	拷打〔厳しく折檻する〕 罪人をむちで~/用鞭子拷打罪人 
\\	严加催促,催逼(せがむ) 
\\	调教马匹,使之驯服	せめる	
\\	慰める	【他动词・一段/二类】 
\\	轻松。使愉快 心を~/使心情安适 
\\	安慰 病気の友を~/安慰患病中的友人	なぐさめる	
\\	束の間	【名词】 一刹那,转眼之间,转瞬,瞬息间(わずかの時間。ほんのちょっとのあいだ。ごく短い時間、瞬間) ~の喜び/瞬间的欢喜	つかのま	
\\	景品	【名词】 
\\	赠品(商品に添える) 
\\	纪念品,奖品,礼品(催し物の参加者に与える品) ~を受け取る/接受纪念品	けいひん	
\\	操る	【他动词・五段/一类】 
\\	〔船、机械等〕开动,驾驶,驾驭。耍,耍弄。善于操作 人形を~/耍木偶 
\\	〔暗地里〕操纵,控制 世論を~/操纵舆论 
\\	操,掌握(言葉をたくみに使う) 英語を上手に~/擅长英语	あやつる	
\\	率直	【形動】 直率;坦率,爽直,直爽;[腹蔵なく]坦率 ~な人/爽直的人;直肠子	そっちょく	
\\	大まか	【形容动词/な形容词】 
\\	粗枝大叶;粗略,草率,马马虎虎,大大咧咧 ~な人間/粗率的人 仕事が~だ/工作不过细 
\\	粗略;笼统;大致;扼要 ~な計画を立てる/制定笼统的计划 
\\	草率;粗率;大方;宽宏大量;心胸开阔(緻密でないさま。おおざっぱ)	おおまか	
\\	擬音語	【名词】 拟声词,象声词(擬音語) 幼児期において~をもって対象物を表現する/幼儿时期用拟声词来表达对象	ぎおんご	
\\	擬態語	【名词】 状态词,拟态词,象态词	ぎたいご	
\\	踏む	【他动词・五段/一类】 
\\	踏,踩,践踏;跺脚(足で押し付ける) 人の足を~踩别人的脚 
\\	走上,踏上(至る) 外国の地を~踏上外国的土地 
\\	实践,经验(経験する) 
\\	估价,估计,评价(予想する・見積もる) 
\\	经历,经过(経る)	ふむ	
\\	小さな	【连体词】 小,微小(小さい) ~進歩/微小的进步 ~親切/作件小小的好事	ちいさな	
\\	近年	【名词】 近几年,最近(最近の数年。近頃)	きんねん	
\\	浦安市	【日本地名】 浦安市	うらやすし	
\\	埋め立てる	【他动词・一段/二类】 填拓,填海(川·沼·海などを土で埋めて、陸地にする) 海を~/填海造地	うめたてる	
\\	住宅地	【名词】 住宅区(住宅をつくるための土地。また、住宅の集まっている地域)	じゅうたくち	
\\	賑わう	【自动词・五段/一类】 
\\	热闹,拥挤,繁华。人来的很多而热闹(人が大勢出てにぎやかになる) 
\\	繁荣,兴旺,兴隆,兴盛,繁盛(富み栄える。繁盛する) 市が~/市场旺盛 
\\	丰盛。变丰富(豊かになる) 食卓が~/餐桌丰盛	にぎわう	
\\	排気	【名词・自サ】 
\\	排除空气(空気を除くこと) 
\\	内燃机燃烧完毕后排出的蒸汽,气体(エンジンからはき出されること。また、そのガス)	はいき	
\\	汚れ	【名词】 
\\	污垢;污渍;脏(よごれること。よごれた痕) ~を取る/去污 
\\	月经(月経) 【名词】 
\\	污秽。肮脏(きたないこと) ~を落とす/去掉不洁的东西;荡除污秽 
\\	污点(名誉を傷つけられること。汚点) 
\\	丑恶,耻辱 
\\	在服丧期或产期等忌讳朝拜山神的一种迷信说法	よごれ けがれ	
\\	震動	【名・自他・サ变/三类】 震动,晃荡(揺れ動くこと。また、揺り動かすこと)	しんどう	
\\	悩み	【名词】 烦恼;苦恼;痛苦(悩むこと。思い苦しむこと)	なやみ	
\\	づくり	【接头/接尾】 ……制。表示以不同的样式、材料和方法等建造、制成的 神殿~/神殿结构		
\\	下水道	【名词】 下水道(下水を流すための排水設備。法律では、下水処理施設を含めていう)	げすいどう	
\\	少子化	【名词】 少子化。出生率降低,儿童人数日渐减少的现象或倾向(出生率が低下し、子供の数が減少し続けること。先進諸国にほぼ共通する。1992年度の国民生活白書で使われた語)	しょうしか	
\\	一人っ子	【名词】 独生子女。独生子,独生女(兄弟姉妹のいない)	ひとりっこ	
\\	政策	【名词】 政策(政治の方策)	せいさく	
\\	保護	【名词・他动·三类】 保护(危険・破壊・困難などが及ばないように,かばい守ること)	ほご	
\\	増加	【名・自他・サ变/三类】 增加,增多。数量增长(数量がふえること。ふやすこと)	ぞうか	
\\	身の回り	【名词】 
\\	身边衣物〔身につけるもの・手回りの品〕 ~を整える/收拾好身边衣物 
\\	日常生活〔日常の雑事〕 ~の世話をする/照料日常生活 
\\	应由自己处理的事情〔仕事・つき合い〕 ~をきれいにしておく/把该办的事办利落	みのまわり	
\\	我が国	【名词】 我国,我们的国家(わたし(たち)の国)	わがくに	
\\	総合的	【形容动词/な形容词】 综合性的(ある体系によって全体が一つにまとめられているさま) ~な学習の時間/综合的学习时间	そうごうてき	
\\	国力	【名词】 国力(国家の勢力。特に、国の経済力)	こくりょく	
\\	及び	【接续词】 (连接并列的体言或与体言同等资格的词语时使用)以及,还有,和,与(名詞や名詞と同じ資格をもつ句に付く)並べて挙げる時用いる。並びに。かつ。…も…も 
\\	北京~天津/北京和天津	および	
\\	理念	【名词】 理念;最高意境;根本想法(物事のあるべき状態についての基本的な考え)	りねん	
\\	機構	【名词】 
\\	组织;机构(機械の内部の構造。しくみ。メカニズム) 
\\	构造;结构(会社、団体などの組織。また、その仕組み) 
\\	结构(機械の内部の構造) 
\\	机构,组织(機械的に構成されている仕組み)	きこう	
\\	史上	【名词】 史上(歴史が推移して来た途上)	しじょう	
\\	会場	【名词】 会场;会议地点(会議や催し物などが開かれる場所)	かいじょう	
\\	パビリオン	
\\	【英】
\\	亭子(庭園のあずまや。園亭) 
\\	帐幕(天幕) 
\\	展览会馆(博覧会の展示用に一時的に設けた建物。展示館)		
\\	演芸	【名词】 表演(公衆の前で演じられる落語・漫才・講談・浪曲・曲芸・奇術その他、庶民的な演劇・音楽・舞踊などの芸)	えんげい	
\\	主催	【名・他动词・サ变/三类】 主办;举办(中心となってあることを催すこと)	しゅさい	
\\	シンボル	【名词】 
\\	【英】
\\	象征(象徴) 
\\	象征。根据习惯和社会上的约定,表示某种事物的符号(ある意味を表す記号)		
\\	訪れる	【自动词・一段/二类】 
\\	通信问候〔便りをする〕 
\\	到来;来临〔やってくる〕 春が~/春天来临 【自他・一段/二类】 访问,过访(ある場所や人の家に行く。訪問する) 新居を~/访问新住宅	おとずれる	
\\	提供	【名·他动·三类】 
\\	提供,供给,让对方自由运用(さし出して相手の用に供すること) 
\\	赞助播出(商業放送番組に出資すること)	ていきょう	
\\	規模	【名词】 
\\	规模(しくみ) 
\\	范围(範囲)	きぼ	
\\	促す	【他动词・五段/一类】 
\\	促使;促进(促進する) 成長を~/促进生长 
\\	催促;促使(早くするように急き立てる。催促する。相手がそれをする気になるよう仕向ける)	うながす	
\\	機会	【名词】 机会(何かをするのに好都合な時機)	きかい	
\\	端午	【名词】 端午,端阳(五節句の一つ)	たんご	
\\	イベント	【名词】 
\\	【英】
\\	事件,事变,大事(できごと;事件) 
\\	集会(文娱)活动 (催し;行事) 
\\	〈体〉比赛项目,比赛(運動競技・試合の種目)		
\\	国際化	【名・他动词・サ变/三类】 (使)国际化。扩大到国际性规模(国際的な規模に広がること)	こくさいか	
\\	杭州	【名词】 杭州。浙江省省会	こうしゅう	
\\	進出	【名·サ変自】 进入,打入;挤进;参加;向…发展,进展(進み出すこと。新しい分野に発展して出ること)	しんしゅつ	
\\	火事	【名词】 火灾,失火,走火(建築物や山林などが焼けること)	かじ	
\\	励む	【自动词・五段/一类】 努力,刻苦,辛勤,奋勉,勤勉(やる気を出して、一生懸命頑張る) 勉強に~/刻苦学习	はげむ	
\\	歩む	【自动词・五段/一类】 
\\	行,走(あるく) 
\\	前进,进展,经历过(経て来る。経験する) 苦難の道を~/在艰苦的道路上前进	あゆむ	
\\	隣国	【名词】 邻国;邻邦(となりの国。隣邦)	りんこくりんごく	
\\	大金	【名词】 巨款,大钱(多額の金銭)	たいきん	
\\	水源	【名】 水源(川などの水の流れ出るおおもと。みなもと。上水道として利用する水の供給源)	すいげん	
\\	インか	【名词】 【西】
\\	印加人,印加王族		
\\	工業	【名词】 工业(自然の原料に人力や機械力を加え、商品価値のある生産物を製造する産業)	こうぎょう	
\\	繁栄	【自动词・サ变/三类】 繁荣,昌盛,兴旺(栄えること。繁盛)	はんえい	
\\	珍しい	【形容词/イ形容词】 
\\	新奇,新颖(聞いたり見たりすることが稀で、興味をひかれる) 
\\	珍奇,稀奇(稀にしかなく貴重である) 
\\	少有,罕见(めったにないことである) 
\\	非常好,珍贵 (素晴らしい)	めずらしい	
\\	行方不明	【名词】 去向不明,失踪,下落不明(どこへ行ったかわからないこと。安否がわからないこと) 出水による~者は50名にのぼった/因涨大水而失踪者达五十名	ゆくえふめい	
\\	外科医	【名词】 外科医生	げかい	
\\	継ぐ	【他动词・五段/一类】 
\\	[密着させて]接,接上;[つなぎあわす]连接,连上〔つなぐ〕 骨を~/接骨 
\\	〔続ける〕继续 
\\	〔加える〕添加,续上 炭を~/添炭;续炭 
\\	继承,承袭 
\\	继承;嗣(継しょうする)继续	つぐ	
\\	水の泡	【名词】 
\\	水泡,也可指虚幻的事物(水面に浮かぶ泡。転じて、はかなく消えやすいもののたとえ。すいほう) 
\\	白费,化为泡影(努力・苦心がすべてむだになること) 苦労のかいもなく~となる/白白付出了艰辛 せっかくの苦心も~/一片苦心白搭了	みずのあわ	
\\	愛用	【名词・他サ】 爱用,喜欢用(いい物と思い、いつもそれを使うこと)	あいよう	
\\	芥川龍之介賞	【名】 芥川龙之介奖,简称芥川奖。是纪念日本大正时代的文豪芥川龙之介
\\	所设立的文学奖,并由主办单位文艺春秋颁发给纯文学新人作家的一个奖项;现今的主办单位已改为日本文学振兴会	あくたがわりゅうのすけしょう	
\\	上げる	【他动词・一段/二类】 
\\	举起;使升高(高い所に移動する) 旗を~/升旗 
\\	抬起;扬起(上の方に動かす) 
\\	提高;改善;进步(高くする。よくする) 
\\	得到;产生效果(良い成果を出す) 
\\	使上学;提升级别(入学させる) 
\\	做完;建成(終える。仕上げる) 
\\	对付过去;够用(済むようにする。まかなう) 
\\	呕吐(胃から吐く) 
\\	给神明上供。(神仏に供えたり、祈ったりする) 【自动词・一段/二类】 潮水上涨(潮が満ちて、水位が上がる) 【接头/接尾】 (接在其他动词的连用形下) 
\\	完全…;彻底…(すっかり…する) 
\\	做完…;完成…(…し終える) 
\\	清楚地…;明确地…(はっきり…する) 書き~/清楚地写出来 
\\	表示对动作接受者的尊敬之意(動作の受け手に対する尊敬の意を表す) では、お暇申し~/那么(我)告辞了 【接续词】 接在「て」的下面,表示该动作对于接受者来说将是一种好处 この小説を貸して~/这本小说借你	あげる	
\\	朝寝坊	【名词・自动・三类】 好睡懒觉的人;起床晚的人(朝,起きるのが遅い人。また,朝寝をすること) よいっぱりの~/晚上不睡,早上不起;属夜猫子的	あさねぼう	
\\	味わい	【名词】 
\\	风味。味道(飲食物が舌に与える感じ。風味) 
\\	趣味,有趣(おもむき。妙味) ~のある言葉/有趣的话 
\\	食物的味道。风味(食べ物の味。うまみ) 
\\	事物的乐趣。趣味。妙趣。风趣(物事の趣。おもしろみ) ~のある言葉/妙语	あじわい	
\\	褪せる	【自动词・一段/二类】 
\\	退色,掉色,走色(時間が経過したり、光線が当たったりして、色が薄くなる) 
\\	减弱,衰退(元の美しさや力が失う。衰える) 色香が~/颜色和香味都减弱了	あせる	
\\	遊び人	【名词】 
\\	游手好闲的人,好赌博的人,赌徒(何も職を持たずに暮らしている人。狭義では、ばくち打ちを指す) 
\\	喜欢游玩的人,花花公子(遊ぶことが好きな人)	あそびにん	
\\	頭を下げる	【惯用句】 
\\	鞠躬行礼 ~て礼をする/鞠躬行礼 
\\	屈服,认输,表示歉意 あいつにだけは~たくない/就是不想对他屈服 
\\	钦佩,佩服	あたまをさげる	
\\	厚紙	【名】 厚纸;马粪纸	あつがみ	
\\	あっさり	【副词・自动词・サ变/三类】 
\\	轻易,简单的样子(簡単なさま) ~負ける/轻易地输掉 
\\	清淡的样子,淡泊的样子(さっぱりとしたさま) ~した味付け/味道调的清淡		
\\	宛	【造语・接头/接尾】 
\\	寄给…(郵送で);发给…(交付・電報で);送给…(配達で);汇给…(為替で送り出す) 国際書店~の荷物/发给国际书店的货 
\\	每(配分する数量・割合を表す。あたり) ひとり5個~/每人各五个	あて	
\\	アナウンサー	【名词】 【英】
\\	广播员;报告员。〔电台〕播音员,播送员(テレビやラジオ放送で、ニュースを報じたり、司会・実況放送することを職とする人)		
\\	亜熱帯	【名词】 亚热带(気候帯の一。熱帯と温帯との中間の地域で、緯度でほぼ20~30度の間にある)	あねったい	
\\	雨垂れ	【名词】 从房檐流下的雨水(軒先などからしたたり落ちる雨のしずく)	あまだれ	
\\	甘味	【名词】 
\\	甜味(甘さの程度) このお菓子は~がない/这个点心不甜 
\\	甜食(お菓子などの甘い食品)	あまみ	
\\	雨水	【名词】 雨水(降る雨の水。また、雨が降って溜まった水)	あまみず	
\\	有様	【名词】 样子,光景,情况,状态,存在状态。事物的状态(物事の状態。ようす) 海底の~/海底的光景 見るもみじめな~/目不忍赌的凄惨情景	ありさま	
\\	アルコール	【名词】 
\\	【英】
\\	酒精,乙醇(溶剤など) 
\\	酒(飲料)		
\\	安価	【名・形动】 
\\	廉价,便宜(値段の安い) ~に仕入れる/廉价买进 
\\	没有价值,肤浅,浅薄〔浅薄なこと〕 ~な同情は受けたくない/不愿意接受浅薄的同情	あんか	
\\	矢張り	【副词】 
\\	仍然,依然,还是,照旧〔依然として〕 彼はいまでも~勉強家です/他至今仍然是个用功的人 
\\	也,同样,又(前もってした予想や判断と同様であるさま。また,他の例から類推される状況と現実が同じであるさま) わたしも~そうだと思う/我也那么想 
\\	毕竟还是,归根结底〔けっきょく〕 暖かくても~冬だ/虽然暖和,毕竟还是冬天 
\\	(虽然)…但仍,(尽管)…还是〔しかし〕 
\\	果然〔案の定〕 ~あなただったのか/果然是你啊	やはり	
\\	居合わせる	【自动词・一段/二类】 正好在场(ちょうど うまいぐあいに、そこに居る)	いあわせる	
\\	言い過ぎる	【他动词・一段/二类】 说得过火,表现过火(度を越えて言う。必要以上に言う。言い過ごす)	いいすぎる	
\\	委員会	【名词】 委员会(委員によって構成される合議制の機関。また、その会議)	いいんかい	
\\	生き返る	【自动词・五段/一类】 复活,苏醒(一度死んだものが息を吹きかえす。よみがえる。活力をとりもどす。再び元気になる) 人工呼吸で~/靠人工呼吸苏醒过来	いきかえる	
\\	不明	【名・形容动词/ナ形容词】 
\\	不详,不明(明らかでないこと) 
\\	蠢,无能(才知の足りないこと)	ふめい	
\\	炒める	【他动词・一段/二类】 炒,煎,爆。在锅里放入少量的油,加热后放入材料,一边翻动,一边快速地高温加热(なべに油を引き、加熱したところへ材料を入れ、かき混ぜながら高温で手早く火を通す) 白菜を油で~/用油炒白菜	いためる	
\\	傷める	【他动词・一段/二类】 
\\	弄坏,损坏(器物・建物などに傷をつける。破損させる) 乱暴をしてかさを~/胡闹弄坏了伞 
\\	伤害,弄疼(からだを傷つける。痛くする) 足を~/脚弄疼了 
\\	惹恼,伤脑筋(精神的に苦痛を与える。悩ます) 借金で頭を~/因借钱伤脑筋 
\\	(水果、青菜等)使腐烂,使坏掉(食物などに傷をつけたり、腐らせたりする) 輸送中に花を~/运送途中花弄烂了 
\\	使蒙受损失(経済的に痛手を与える。損をさせる) 父の懐を~/让父亲破费	いためる	
\\	一見	【名・他动词・サ变/三类】 
\\	一见,一看,看一次(一度見る) ~の価値がある/值得看一下 
\\	看一眼,一瞥,乍一看,初看(ひと目みる) ~したところ…/乍一看来 
\\	一看,乍一看(~して)	いっけん	
\\	一刻	【名·形动】 
\\	短时间,片刻,一刻〔わずかな時〕 ~を争う/分秒必争 
\\	刻板,冥顽不灵,顽固听不进别人意见的人(一国とも書く。頑固で人の意見を聞き入れないこと) ~な男/冥顽不化的家伙	いっこく	
\\	一戸建て	【名词】 独幢楼房(構造・設備ともに独立した一棟の建物からなる一戸の住宅) こんど建つ住宅はみな~だ/这次盖的住宅都是单门独户的房子	いっこだて	
\\	一切	【名词・副词】 
\\	一切,全部,所有,一点儿不剩(全部。すべて。残らず) ~の責任を負う/负全部责任 
\\	下接表示否定的词带有强调的语气。全然不,完全不(下に打ち消しの語を伴って、打ち消しを強調する気持ちを表す。全然。全く。例外なく) 遅刻は~許さない/都不准迟到	いっさい	
\\	一点	【名词】 
\\	一点,一个点(一つの点) 
\\	一点点,一点儿,少微,微小(ほんの少し。わずか) 
\\	一件(品物や作品一つ) 
\\	(分数或评分)一分(得点や評点一つ) 
\\	一刻。把一个时辰(相当于今2小时)四等分的第一段时间	いってん	
\\	一泊	【名词・自动·三类】 一宿。住一夜(一晩よそでとまること) 箱根で~する/在箱根住一宿 ~料金は税込みで1万円/一宿的费用包括税金一万日元 ~旅行/宿一夜的旅行	いっぱく	
\\	一般的	【形容动词/な形容词】 一般的(広く全体を取り上げるさま。広く行き渡っているさま) ~に論ずる/一般而论 ~な情勢/一般的情况	いっぱんてき	
\\	否応なし	【惯用句】 不管愿意不愿意,不容分说(承知も不承知もないようす。有無を言わせないようす) ~に連れ出す/不容分说的被带出去了	いやおうなし	
\\	癒す	【他动词・五段/一类】 
\\	治疗,医治(病気や傷などをなおす) 病を~/治病 
\\	解除,去除悲伤、痛苦(悲しみや苦痛をなくす) 恋の痛手を~/解除失恋的打击	いやす	
\\	受け入れる	【他动词・一段/二类】 
\\	接受,接收,收纳,收进(受け取って収める) プリントを~/接受材料 
\\	接受,听从他人意见(人の言うことや要求などを聞き入れる) アドバイスを~/接受建议 
\\	接纳,迎接(自分の方へ引き取る) 被害者を~/接纳受害者	うけいれる	
\\	受け持ち	【名词】 主管,担任;主管〔担任〕的人;主管〔担任〕的事(自分の仕事として責任をもって行うこと。また、その仕事を行う人) 彼女は僕の一年生の時の~の先生だった/她是我一年级时的班主任	うけもち	
\\	打ち明ける	【他动词・一段/二类】 毫不隐瞒地说出,坦率说出(秘密や、思っていたことなどを、包み隠さず人に話す) 心を~/说出心里话 秘密を~/吐露秘密	うちあける	
\\	写し	【名词】 
\\	抄本,本,摹本,誊本(書き写すこと。また、書き写したもの) 
\\	抄本,抄件,誊本,副本。抄写下来备用的文书(控えに写しておく文書。謄本。副書) ~をとる/抄副本 
\\	临摹画,仿制品。仿照原品制作,亦指其制品(原品になぞらえて作ること。また、その品。模造品)	うつし	
\\	器	【名词】 
\\	容器,器皿;器具(容器) 
\\	才干,人才(能力・人物)	うつわ	
\\	奪い合い	【名词】 (互相)争夺(互いに奪うこと) 1の位置の~/争夺第一位	うばいあい	
\\	裏通り	【名词】 后街,后巷,小胡同(裏の方の通り。うらみち) 北京の~はだんだん昔の様子を失ってしまった/北京的小胡同渐渐失去了过去的模样	うらどおり	
\\	憂い	【名词】 
\\	忧,忧虑,担忧,忧愁(予測される悪い事態に対する心配・気づかい) 後顧の~/后顾之忧 
\\	悲伤,哀愁,忧愁(心中にいだくのも悲しい思い) ~を帯びた顔/面带愁容	うれい	
\\	売れ残る	【自动词・五段/一类】 
\\	卖剩下(売れずに残る) 商品が~た/商品卖剩下了 
\\	嫁不出去,没人要(俗に、女性が婚期をのがして独身でいる)	うれのこる	
\\	云々	【名・他动词・サ变/三类】 
\\	一言难尽(一言では言い切れないさまざまのこと) ~の事情がある/有一言难尽的事 
\\	云云,等等。使引用文章、词句的后部含糊或省略时使用的词语(引用した文を中途で切り、以下を省略する時に用いる語。しかじか。これこれ) 美人を、立てば芍薬~と形容する/将美人形容成站立的芍药等等	うんぬん	
\\	エイズ	【名词】 〈医〉爱滋(病),艾滋(病),获得性免疫缺损综合症(後天性免疫不全症候群。一九八〇年代になって発見されたエイズウイルスの感染によって起こる疾患。性行為感染症の一つ)		
\\	円高	【名词】 日元上涨,日元汇价高
\\	日元对外比价偏高(為替相場で、相手の外貨に対する日本の円の価値の高い場合を言う) ~によって倒産した/由于日元升值而导致破产	えんだか	
\\	追い込む	【他动词・五段/一类】 
\\	赶进,撵进。 逼入,使陷入。(赛跑等)接近终点最后加劲;(工作等)到紧要关头作最后努力(追い立てて、中へ入れる。追いつめる) 人を死に~/把人整死 
\\	紧排,挤排,移前(組版で、改行や改丁などをせず詰めて活字を組む) この行を前ページに~こと/这一行要挤进前页里 
\\	使…内攻(病を内攻させる) 病を~/使病内攻	おいこむ	
\\	仰せ	【名词】 
\\	指示,命令,吩咐(目上の人からの「言いつけ」「命令」の尊敬語。お言いつけ。ご命令) ~の通りに従う/遵从命令 
\\	说的话(おっしゃること。お言葉) ~のとおりです/正如您所说	おおせ	
\\	大粒	【名词】 大粒,大颗(粒が大きいこと。また、大きな粒) ~の汗がほおを伝わった/大颗的汗珠顺着双颊流下来	おおつぶ	
\\	起き上がる	【自动词・五段/一类】 起来,爬起来(横になっていたものがからだを起こす。また、立ち上がる) 寝床から~/从床上爬起来	おきあがる	
\\	置き引き	【名词・他动・三类】 顺手牵羊,顺包贼。把隔在那儿的他人行李物品等偷走,亦指顺手偷东西的人(置いてある他人の荷物などを、盗み去ること。また、その者) ~にあう/贼把包偷了	おきびき	
\\	陥る	【自动词・五段/一类】 
\\	落入,陷入(落ちて、その中に入る) 穴に~/陷进坑里 
\\	陷入不良状态(よくない状態になる) 
\\	中计谋,上圈套(計略に引っかかる) 
\\	城池陷落(城が攻め落とされて、敵の物となる)	おちいる	
\\	衰える	【自动词・一段/二类】 
\\	势头消失。衰弱(生命力・活動力がすっかり弱った状態になる) 体が~/身体衰弱 
\\	衰亡。衰败(勢力をなくした状態になる。落ちぶれる)	おとろえる	
\\	御礼	【名词】 
\\	感谢,谢意(感謝の意を表すこと。また、その言葉) ~を述べる/道谢,致谢 
\\	回敬,回礼,还礼,答礼。谢礼,报酬,酬谢(感謝を表すために贈る品物)	おれい	
\\	お詫び	【自动词・サ变/三类】 道歉,赔罪,表示歉意,赔不是,请求原谅(謝ること) 彼に~をせねばならない/我该向他道歉 ごぶさたを~申しあげます/久疏问候向您表示歉意	おわび	
\\	恩師	【名词】 恩师(教わった先生のうちで、その人の生涯を決定するような影響・感化を与えた人) ~への手紙/给恩师的一封信	おんし	
\\	快感	【名词】 快感(気持ちのいい感じ) ~を覚える/有快感,感觉愉快,感觉痛快 ~を味わう/尝到快感	かいかん	
\\	解明	【名・他动词・サ变/三类】 阐明,解释清楚,弄清(解き明かすこと。不明な点をはっきりさせること) 謎を~する/解谜	かいめい	
\\	回路	【名词】 回路,电路,线路(電源から出た電流が、再び電源に入るまでの道筋。電気回路)	かいろ	
\\	瓦解	【名词・自动·三类】 瓦解,崩溃(一部の瓦のくずれ落ちることが屋根全体に及ぶように、ある一部の乱れ・破れ目が広がって組織全体がこわれること)	がかい	
\\	屈める	【他动词・一段/二类】 弯腰,屈身(からだを前に折り曲げてやや低い姿勢をとる。かがむようにする)	かがめる	
\\	書き直す	【他动词・五段/一类】 改写,重新写(一度書いたものを,訂正や浄書のためにもう一度書く) 報告書を~/把报告书重写一遍	かきなおす	
\\	斯くて	【接续】 于是,就这样(前に述べたことを受けて、新たに説き起こす。このようにして。かくして) ~ローマは滅びた/于是罗马就灭亡了 ~私の休暇は終わった/于是,我的假期就结束了	かくて	
\\	駆け落ち	【自动词・サ变/三类】 
\\	私奔(親から結婚の許しを得られない男女が,しめし合わせてひそかによそへ逃げ隠れること) ~結婚/私奔结婚 
\\	逃往他乡,出奔,失踪(逃げて行方をくらますこと。戦国時代,農民が戦乱・重税などのために散発的あるいは組織的に離村・離郷すること) ~もの/逃难者	かけおち	
\\	貸切	【名词】 包租,整租(貸し切ること) ~バス/包租公共汽车 ~温泉/包租温泉	かしきり	
\\	過信	【名・他动词・サ变/三类】 过于相信(物事や人物、能力などを信じすぎること) 体力を~する/太相信自己的体力	かしん	
\\	片隅	【名词】 一隅,(一个)角落(中央部から離れた目立たないところ) へやの~/屋子的一个角落 町の~に住む/住在城镇的偏僻角落里	かたすみ	
\\	偏り	【名词】 偏,偏向一方,偏颇;偏倚,偏重(一方に寄ること。一方に偏すること) 栄養の~がひどい/营养失衡	かたより	
\\	カルチャーショック	文化差异。文化冲击(自分とは異なる考え方・慣習・生活様式)		
\\	枯れ枯れ	【形動】 (即将)枯萎,凋零	かれがれ	
\\	交わす	【他动词・五段/一类】 
\\	交,交换(やりとりする) 意見を~/交换意见 
\\	交结,交叉;交错,互相…(まじりあわせる) ひそかにまなざしを~/偷偷地互递眼色 情を~/(男女)相爱;发生肉体关系	かわす	
\\	皮袋	【名词】 皮口袋,皮囊,钱包(皮革で作った袋、または財布)	かわぶくろ	
\\	変わる	【自动词・五段/一类】 
\\	改变地点,迁居,迁移(場所、住所などが移る。移転する) 新しい家に~/迁入新房 
\\	不同,与众不同;奇怪,出奇(普通と違う) 
\\	变,变化;改变,转变(形、性質、様子などがそれまでと異なったものとなる。変化する) 風向きが~/风向改变	かわる	
\\	癌	【名词】 
\\	癌、癌瘤、癌肿、癌症(多細胞生物の細胞分裂が不規則になって無制限に増殖し、周囲の組織を侵したり他の臓器に移転したりして生体を死に至らしめる病気) 
\\	症结、要害、祸根(組織全体に障害を及ぼしている事柄)	がん	
\\	考え込む	【自动词・五段/一类】 沉思,苦想(ほかの事を忘れて、その事を考えることに集中する) 1日中~/一整天沉思苦想	かんがえこむ	
\\	消え失せる	【自动词・一段/二类】 
\\	消失,溜走(そこにあったものが見えなくなる) 
\\	咽气,死掉(死ぬ)	きえうせる	
\\	企画	【名词・他动・三类】 规划,计划(計画を立てること)	きかく	
\\	飢渇	【名词】 饥渴(飢えとかわき。特に、飲食物の欠乏すること)	きかつ	
\\	議事堂	【名】 
\\	〔会議をする建物〕会议厅 
\\	〔国会議事堂〕国会议事堂,国会大厦	ぎじどう	
\\	規制	【名・他动词・サ变/三类】 规定(章则);限制,控制(規則によって物事を制限すること。物事をなす際に従うべき決まり。規定) 自主~/主动限制	きせい	
\\	喫煙	【名・自サ】 吸烟,抽烟(タバコを吸うこと)	きつえん	
\\	黄ばむ	【自动词・五段/一类】 带黄色;呈黄色(黄色みを帯びた色になる。黄色みがかる)	きばむ	
\\	逆転	【名词・自他动・三类】 
\\	逆转,反转,倒转(それまでとは逆の向きに回転すること) 
\\	倒过来,反过来;倒退;恶化(事のなりゆきや優劣の関係が今までとは逆になること) ~ホームラン/扭转局势的本垒打 
\\	(飞机的)空中翻跟头(戦闘機などの宙返り)	ぎゃくてん	
\\	九死	【名词】 九死。几乎难免一死的危险情形(ほとんど死を避けがたい危険な場合) ~に一生を得る/九死一生	きゅうし	
\\	休日	【名词】 休息日,假日(その組織体において、その日は全員が業務を休むように前もって定めた日)	きゅうじつ	
\\	教育費	【名词】 教育费	きょういくひ	
\\	競争	【名・自他・サ变/三类】 竞争,竞赛(争う)	きょうそう	
\\	驕慢	【名词・形动】 骄傲,傲慢(自分だけが偉いと思い、他人を見くだして、かってな事をする様子)	きょうまん	
\\	強烈	【形容动词/ナ形容词】 强烈。力量、作用、刺激等强大而猛烈(力・作用・刺激が強く激しいこと。また、そのさま)	きょうれつ	
\\	煌びやか	【形容动词/ナ形容词】 光辉灿烂,灿烂夺目,华丽(輝くばかりに美しいさま) ~な衣装/华丽的衣着 【古语】 干脆,清爽(きっぱり。はっきり)	きらびやか	
\\	キログラム	【名词】 【仏】
\\	千克;公斤(国際単位系
\\	の基本単位の一で、質量の単位)		
\\	近所	【名词】 近处,附近,左近;近邻,邻居,街坊,四邻(となり近所)	きんじょ	
\\	金賞	【名词】 金奖	きんしょう	
\\	禁物	【名词】 
\\	严禁,忌讳(用いることを禁じられた物事) その話はいま彼らのあいだでは~になっている/现在他们之间忌讳那句话 
\\	切忌(避けるべき物事) 夜ふかしは~だ/切忌熬夜	きんもつ	
\\	空海作戦	【名词】 海空作战	くうかいさくせん	
\\	鯨	【名词】 鲸,鲸鱼(動物中最大の海獣)	クジラ	
\\	口先	【名词】 
\\	吻,嘴;喙;嘴尖,嘴边,唇边(口の先端) ことばが~に出かかった/话到嘴边 
\\	口头上的(心のこもらないうわべだけの言葉や話ぶり〕 ~だけの約束/口头上的约定	くちさき	
\\	国々	【名词】 各国,各地	くにぐに	
\\	雲	【名词】 云,云彩(大気中の水分が細かな粒となって、白く見えるもの)	くも	
\\	燻る	【自动词・五段/一类】 冒烟。烟熏(炎を出さずに、煙が立つ) 煙が~/烟熏人	くゆる	
\\	暗闇	【名词】 
\\	漆黑,黑暗(暗いこと。また、暗い場所) 外はまっ~だ/外边黑咕隆咚的 
\\	暗处(人の目につかないところ) ~の恥を明るみへ出す/隐私〔丑事〕外扬 
\\	黑暗状态,乱世(前途の見通しがはっきりせず、将来に希望が持てないこと)	くらやみ	
\\	クラブ	【名词】 
\\	【英】
\\	俱乐部。同好会。亦指其集会场所(研究や趣味や社交など共通の目的をもつ人々の集まり。同好会。またその集会所) 
\\	课外活动小组(学校の課外活動での集まり) 
\\	俱乐部。以会员制为原则的酒吧或娱乐场(会員制を建て前とするバーや娯楽場) 
\\	(高尔夫)球棒(ゴルフで球を打つための棒状の用具) 
\\	(扑克牌)梅花(トランプの種類の一) 
\\	【英】
\\	蟹,螃蟹(蟹,かに)		
\\	狂い	【名词】 
\\	翘棱,歪斜,弯曲〔ゆがみ・そり〕 この板には~がきた/这块木板翘棱了 
\\	紊乱,错乱,失常,失调〔順調でない,平常でない〕 手順に~がある/程序有打乱的地方 時計の~/表走得不准 
\\	不准确,错误(まちがい)	くるい	
\\	黒々	【副词・自动词・サ变/三类】 乌黑、漆黑。黑得很明显的样子。漆黑的样子(いかにも黒く際立っているさま。まっくろなさま)	くろぐろ	
\\	桑	【名】 〈植〉桑,桑树	くわ	
\\	計画	【名・他动词・サ变/三类】 计划,谋划,规划(事を行うにあたり、その方法や手順などをあらかじめ考えること。また、その考えの内容。もくろみ。プラン)	けいかく	
\\	形成	【名词・他动·三类】 形成(整ったものにつくり上げること。形づくること)	けいせい	
\\	警報	【名词】 警报(災害の起こりそうな時、それに対する人々の注意を促し警戒させるための知らせ)	けいほう	
\\	桂林	【名词】 桂林(中国広西チワン族自治区の観光都市。珠江支流に臨む水陸交通の要地で、石灰岩地域特有の奇峰が多い。コイリン)	けいりん	
\\	決起	【名词・自サ】 奋起;蹶起(ある目的のために、決意を固めて行動を起こすこと)	けっき	
\\	結構	【名词】 结构,布局(一つの纏まった物に作り上げること。組み立て。構成) 【形容动词/ナ形容词】 
\\	漂亮的,完美的(素晴らしくて難点が特別見当たらない様) ~な出来栄え/做得完美无缺 
\\	温和的,温柔的(気立てのよい様。お人好し) お元気そうで~ですね/您看上去很精神 
\\	可喜的,令人满意的(満足できる状態である様。その状態で十分である様) お元気で~です/您身体健康,真让人高兴 
\\	足够的,充分的(それ以上を必要としない様。十分) もう~だ/已经够了 【副词】 相当,颇(完全ではないが、それなりに十分である様) ~眺めのいいところ/风景相当不错的地方	けっこう	
\\	原因	【名・自动词・サ变/三类】 原因(ものごとのおこるもと)	げんいん	
\\	嫌煙	【名】 他人の吸うタバコの煙から受ける害を嫌うこと	けんえん	
\\	原子力	【名词】 原子能(原子エネルギー。核エネルギー)	げんしりょく	
\\	現地	【名词】 
\\	现场(あることが実際に行われている場所。現場) 
\\	当地,现住地方(自分が現在住んでいる土地)	げんち	
\\	原発	【名词】 
\\	「原子力発電」「原子力発電所」之略(「原子力発電」「原子力発電所」の略) 
\\	[医]原发(肿瘤,症状等)由病因直接或初次表现出来(腫瘍・症状等が)病因から直接または最初に現れること	げんぱつ	
\\	見聞	【名・他动词・サ变/三类】 见闻,见识。实际看到、听到的事物,亦指通过所见所闻得到的经验、知识(実際に見たり聞いたりすること。また、それで得た経験・知識)	けんぶん けんもん	
\\	原文	【名】 原文(翻訳・校訂・改作・加筆などをする前の、もとの文章)	げんぶん	
\\	剣幕	【名词】 气势汹汹,怒气冲天(怒って興奮しているようす。いきり立った、荒々しい態度や顔つき) すごい~で怒鳴る/气势汹汹地叫嚷	けんまく	
\\	弧	【名词】 
\\	〈数〉弧(円周または曲線の一部分。弧
\\	は記号で表される) 
\\	弧形(弦を張ったときの、弓のように曲がった形)	こ	
\\	交渉	【名・他动词・サ变/三类】 
\\	交涉;谈判(特定の問題について相手と話し合うこと。掛け合うこと) 
\\	关系;联系(交際や接触によって生じる関係。かかわり合い。関係)	こうしょう	
\\	構図	【名词】 
\\	构图(絵や写真などの画面の、全体の構成) 
\\	构图(平面的な造形美術で、全体の効果を高めるための諸要素・諸部分の配置。コンポジション) 
\\	(比喻)事物整体形象(物事全体のすがた、かたち)	こうず	
\\	厚生	【名词】 
\\	提高生活;增进健康;保健,卫生(生活を健康で豊かなものにすること) ~施設/福利(保健)设施 
\\	使人民的生活富足、充裕;保障;福利(生活を健康で豊かなものにすること)	こうせい	
\\	抗戦	【名词・自动・三类】 抗战(抵抗して戦うこと)	こうせん	
\\	行動力	【名词】 行动力。执行力(何かを思い立った際に、実際に行動を起こし、それを実現する力。または、そうした実現に向けて行動する勇気や度胸などのことを幅広く指す表現)	こうどうりょく	
\\	購読者	【名】 (报纸等)订阅者 定期的な~ 
\\	定期订阅者	こうどくしゃ	
\\	神戸	【名词】 神户。兵库县东南部的市,为县厅所在地,指定城市。面临大阪湾,在背靠六甲山地的狭长地上形成中心市街地。日本数一数二的贸易港,阪神工业地带的中心 【名词】 神户。附属于神社,向神社缴纳租税,受神社课役的民户(神社に付属して、租税・課役を神社に納めた民。神封戸)	こうべ かんべ	
\\	後方	【名词】 后方(うしろのほう)	こうほう	
\\	黄浦江	【名词】 黄浦江,长江支流。黄浦江是上海的地标河流,流经上海市区,将上海分成浦西和浦东	こうほこう	
\\	古雅	【形容动词/ナ形容词・名】 古雅	こが	
\\	獄窓	【名】 狱窗,牢狱	ごくそう	
\\	国内	【名词】 国内(一国の領土内。国の中)	こくない	
\\	国連	【名词】 联合国的略语(国際連合の略) ~に加盟する/加入联合国	こくれん	
\\	国論	【名词】 国论,舆论(国民一般の論・意見。世論)	こくろん	
\\	個々	【名词】 每个。各个,个别,一个一个,各自(いくつかあるうちの一つ一つ)	ここ	
\\	心掛ける	【他动词・一段/二类】 留心;注意;记在心里(気をつける。その事を忘れずに、常に念頭に置く。どんな事態にも対処出来るような用意を忘れずにする)	こころがける	
\\	試みる	【他动词・一段/二类】 试验一下(能力・効力などを実地について試験する) 勧められた薬を~/试服别人介绍的药	こころみる	
\\	誤差	【名】 
\\	〈数・測〉误差(測定値・理論的推定値また近似計算によって得られた値と、真の値との差) ~を測る/量误差 ~が生じる/产生误差 
\\	偏差(食い違い。ずれ)	ごさ	
\\	腰掛ける	【自动词・一段/二类】 坐下(椅子や台などの上に腰をおろす。腰をかける) 縁側に~/坐在外廊	こしかける	
\\	御仁	【名词】 人(人を敬っていう語。おかた。現在では、ひやかしの気持ちを含んで用いることもある) 立派な~だ/真是个善人;真是个好人(多讽刺意味)	ごじん	
\\	個性	【名】 个性(その個人にしかない性格や性質)	こせい	
\\	木立	【名词】 树木;树丛(何本かまとまって生えている木)	こだち きだち	
\\	殊に	【副词】 特别,格外,分外,尤其(その物事の程度が、同種の他のものとひどくかけ離れていることを表わす) ~すぐれている/格外优秀	ことに	
\\	好む	【他动词・五段/一类】 
\\	喜欢,称心如意,热爱(興味を持つ。気に入って味わい楽しむ) 
\\	希望,欲望(えらび望む。所望する) 立身出世を~/企盼飞黄腾达	このむ	
\\	コメンテーター	【名词】 【英】
\\	解说员,评论家(批評・解説を加える人。また、ニュース解説者)		
\\	ころころ	【副词・自动词・サ变/三类】 
\\	滚动、滚转(的样子)(小物品滚动的状态和声音)骨碌碌(まるい物、小さい物などが軽快に転がるさま) ~倒れる/叽里咕噜地倒下 
\\	(朗朗的笑声和铃等的响声)格格。丁丁。丁零(笑声)朗朗(高く澄んで、響きのよい音を表す語。鈴の音・虫の声・笑い声など) ~と笑う/格格地笑 
\\	想法和话题的转变(簡単にひっくり返ったりころがったりするさま) 話が~(と)変わる/转变话题 
\\	(一个接一个地)倒下(ものが容易に倒れるさま。ころりころり) 上位陣が~敗れる/败下阵来 
\\	溜圆『方』,滚圆。胖乎乎(中身が詰まって、いかにも丸いさま。丸々として、かわいらしいさま) 太って~した子犬/胖得滚圆的小狗 
\\	表示青蛙的叫声(蛙の鳴く声を表す語)		
\\	根拠	【名词】 根据,依据(ある言動のよりどころ。もと。また、議論などのよりどころ) …を根拠として/根据… 事実を根拠とする/以事实为根据 ~になる材料/成为根据的材料	こんきょ	
\\	根性	【名词】 
\\	脾气;性情,秉性,根性;赋性(性質。性格) 卑しい~/卑鄙的根性 
\\	骨气,斗志,毅力(強い性質) ~がある男/有骨气的汉子;硬汉	こんじょう	
\\	材質	【名词】 
\\	木材的性质,木质(材木の性質) 
\\	材料的性质,木材的性质,木质,材料的性质(材料の性質)	ざいしつ	
\\	最大	【名】 最大(最も大きいこと)	さいだい	
\\	再度	【名词・副词】 再度,第二次(再び、2度)	さいど	
\\	栽培	【名・他动词・サ变/三类】 栽培,种植(植え育てること)	さいばい	
\\	再販	【名词】 再卖,转卖《转卖价格不变合约》之略(以前売られていたものを再び販売すること「再販売価格維持契約」の略)	さいはん	
\\	遮る	【他动词・五段/一类】 
\\	遮挡,遮住,遮蔽(向こうが見えないように間に物を置いて隔てる) カーテンで~/用窗帘挡住 
\\	遮断,遮拦,阻挡(邪魔をして止める) 行く手を~/挡住去路	さえぎる	
\\	錯覚	【自动词・サ变/三类】 错觉;错误的观念;误会,错认为(事実とは異なるが、そうであるかのように思うこと。思い違い) ~に陥る/陷入错觉 ~を起こす/产生错觉; 产生错误的观念	さっかく	
\\	昨今	【名词】 近来,最近(このごろ。近頃)	さっこん	
\\	妨げ	【名词】 妨碍,阻碍。阻挠,障碍(妨げになること。邪魔。障害。触り) 通行の~をする/妨碍交通;阻挡通行 遊びが仕事の~となってはいけない/游戏不可妨碍工作 なんの~もなく事が順調に運んだ/事情毫无阻碍地顺利地进行了	さまたげ	
\\	死因	【名词】 死因,死亡原因(死亡の原因)	しいん	
\\	紫煙	【名】 (香烟的)烟((紫色の煙の意から)タバコの煙) ~をくゆらす/吐着烟雾	しえん	
\\	自国	【名词】 本国(自分の国)	じこく	
\\	事実	【名词】 事实;真实(実際の事がら) 赤裸々な事実/赤裸裸的事实 ~に反する/与事实相反 【副词】 事实上,实际上(ほんとうに・まったく);确实(確かに) ~そうなんだ/(事情)确实是那样	じじつ	
\\	自主	【名词】 自主(他からの干渉や保護を受けず、独立して事を行うこと) ~規制する/主动限制	じしゅ	
\\	視聴者	【名】 收看者;观众(テレビやラジオの番組を見たり聞いたりする人) ~が少ない/观众少	しちょうしゃ	
\\	実験	【名词】 
\\	实验,实地试验〔実際に試み、考え方の正否を調べること〕 
\\	经验,体验,实际经验〔実際の経験〕	じっけん	
\\	視点	【名词】 
\\	视线的集中点(視線の注がれるところ) 
\\	观点(ものを見る立場。観点) 
\\	绘画上的视点(絵画の遠近法で、画面上において遠方へ向かう平行線が集結する一点。消点。焦点)	してん	
\\	凌ぐ	【他动词・五段/一类】 
\\	忍耐,忍受;维持,抵御;躲避,排除(苦痛や困難に屈しないで、耐えしのぶ。また、防ぎ止める) 暑さを~/避暑 
\\	闯过;摆脱;应付,对付;冒着,顶住(障害・困難などと闘って、それをのりこえる) 困難を~/克服〔顶住〕困难 
\\	凌驾,超过(数量・程度・力量などが、あるところをこえる。凌駕する)	しのぐ	
\\	地べた	【名词】 地面(土地の表面。地面のくだけた言い方) ~に寝ころぶ/躺在地上	じべた	
\\	島国	【名词】 岛国。四面环海的国家(四方を海で囲まれている国)	しまぐに	
\\	染み	【名词】 
\\	污垢,污点;斑痕,污痕(液体などがしみこんでできた汚れ) ~を抜く/除去污垢 
\\	玷污(名誉の汚れ) 名前に~がつく/玷污名声 
\\	〔斑点;老人斑(顔面や手の甲などに生じる褐色の色素斑)	しみ	
\\	従業員	【名词】 工作人员,业务员,职工(業務に従事している人)	じゅうぎょういん	
\\	終結	【名・自动・三类】 终结,完结,结束(物事のおさまりがつくこと。おわり)	しゅうけつ	
\\	集積	【名・自他・サ变/三类】 集积,集聚(集めて積み重ねること。また、そうなること)	しゅうせき	
\\	醜聞	【名词】 丑闻(その人の名誉や人格を傷つけるような、よくないうわさ。男女関係や金銭に関する評判など。スキャンダル) ~が広まる/丑闻传开	しゅうぶん	
\\	粥	【名词】 粥。稀饭(水を多くして米を軟らかく煮たもの)	かゆ	
\\	主食	【名词】 主食(品)(日常の食事の中心となる食品。日本では,米飯・めん類・パンなどの類)	しゅしょく	
\\	出世	【自动词・サ变/三类】 成功,出息,发迹(立派な地位・身分となること) 立身~/出息发迹 ~が早い/发迹得快;地位升得快	しゅっせ	
\\	受動	【名词】 被动(他からの動作・作用を受けること。受け身)	じゅどう	
\\	順応	
\\	顺应,适应,合乎。习惯于环境或境遇的变化(環境や境遇の変化になれること) 
\\	顺应。生物体的机能、性质、状态随外部条件的持续变化而发生变化 
\\	〔心〕顺应。感觉器官连续接受同一刺激,并出现感受性下降的现象	じゅんのう じゅんおう	
\\	主	【名词】 
\\	主人(家の主人) 
\\	一国之长 
\\	招待客人的人(客に対して、主人) 【名词】 
\\	主人((一家の主人。あるじ) 
\\	主,者,人(所有者) 
\\	长居者	あるじ ぬし	
\\	賞賛	【名・他动词・サ变/三类】 称赞、赞扬、赞赏(ほめたたえること)	しょうさん	称賛
\\	正体	【名词】 
\\	原形,真面目,本来面目(うわべの姿のおくにかくされているほんとうの姿) 
\\	意识,神志(正常な状態でいるときの、しっかりした気持ち)	しょうたい	
\\	常套	【名词】 常规,老一套(きまりきった、または、ありふれたやり方)	じょうとう	
\\	乗務員	【名词】 乘务员(電車・バス・航空機などの交通機関に乗って、車内・機内で職務を行う人。運転士・車掌・操縦士・スチュワーデスなど)	じょうむいん	
\\	醤油	【名词】 酱油。调味品的一种,用大豆和小麦制取,再将其放入盐水中发酵熟成后榨出的黑褐色液体。日本独特的调味品之一(むらさき。したじ)	しょうゆ	
\\	勝率	【名词】 (比赛中)取胜的比率,胜利所占的比率(試合などに勝った割合)	しょうりつ	
\\	食生活	【名词】 (每天的)伙食,饮食生活(人間の生活のうち、食事に関する分野) ~を改善する/改善伙食	しょくせいかつ	
\\	食文化	【名】 饮食文化	しょくぶんか	
\\	文化圏	【名词】 文化圈。具有相同文化特征,或包含相同文化要素的地理区域的最大范围	ぶんかけん	
\\	諸国	【名词】 
\\	诸国。各国(多くの国。方々の国) 
\\	各地方(国内各地)	しょこく	
\\	初心	【名】 
\\	初志,初愿,初衷〔初志〕 ~に帰れ/勿改初志 
\\	初学,没有经验,不成熟(習いはじめ) 【な形容词】 
\\	纯真;纯洁;幼稚;没经验;处女(年が若く世間ずれしていない・こと(さま)。純情なさま) ~な青年/纯情的青年 
\\	不懂男女之情的(男女の情に通じていないさま)	しょしんうぶ	
\\	ショック	【名词】 
\\	【英】
\\	冲击(人体や物が受ける物理的な衝撃) 
\\	【英】
\\	吃惊(予期しない事態にあい、心が動揺すること。衝撃) 
\\	【英】
\\	休克状态,身体出现的一种症状,由于种种原因,使流向脏器的血流量减少,体内组织不能维持正常功能,呈现出面色苍白、出冷汗、四肢冰凉、意识消失等症状		
\\	所有	【名・他动词・サ变/三类】 所有(自分のものとして持っていること)	しょゆう	
\\	白鷺	【日本地名】 白鷺	しらさぎ	
\\	調べ	【名词】 
\\	调查,检查(調べること。調査。検査) ~がつく/调查清楚 
\\	审查,审问。调查。讯问,查问(尋問。取~) 
\\	演奏,乐曲。亦指其旋律(音楽を奏でること。演奏。また、そのメロディー) 
\\	(诗歌、音乐等的)调子,音调((音楽や詩歌の持つ)調子)	しらべ	
\\	辞令	【名词】 
\\	辞令,措辞(応対のことば) 
\\	任免证书,任免命令(役職の任免に際して、その旨を書いて本人に渡す書類)	じれい	
\\	真相	【名】 真相(ある物事の真実のすがた)	しんそう	
\\	辛抱強い	【形容词/い形容词】 有耐心;能忍耐(がまんづよい。たえられる)	しんぼうづよい	
\\	随意	随意,任意,随便,由(你)。指没有束缚或限制(束縛や制限のないこと。思いのままであるさま。かって)	ずいい	
\\	推進	【名・他动词・サ变/三类】 推进;推动(物を前へ進めること。物事を目的に向かって、はかどらせること)	すいしん	
\\	衰退	【自动词・サ变/三类】 衰退。衰颓。因衰弱而失去势力(衰えて、勢いを失うこと。凋落。衰微)	すいたい	
\\	睡眠不足	【名词】 睡眠不足(眠る時間が充分でないこと。睡眠がたりないこと。また、そのさま。ねぶそく)	すいみんぶそく	
\\	水力	【名】 水力(流水の力。また、動力として利用される水の運動エネルギーや位置エネルギー)	すいりょく	
\\	雀	【名词】 
\\	麻雀,家雀儿 
\\	喋喋不休的人(おしゃべりな人) 
\\	知道内情的人(ある所によく出入りして事情にくわしい人)	すずめ	
\\	すっかり	【副词】 
\\	全,都,全都;完全,全部;已经(その事、状態が余すところなく全体に及んでいると認められる様子) ~忘れた/全都忘了;忘得精光;忘得干干净净 
\\	〈商〉脱销;卖光; 售完全,都,全都,完全(すっかりになる)	すっかり	
\\	スーツケース	【名词】 【英】
\\	旅行用(手提式)衣箱.(旅行用の衣類かばん。旅行かばん)		
\\	ストーブ	【名词】 【英】
\\	炉子,火炉,暖炉		
\\	スニーカー	【名词】 【英】
\\	轻便运动鞋(底がゴム製の運動靴)		
\\	巣箱	【名词】 巢箱,鸟巢箱,蜂巢箱(野鳥や蜜蜂が巣をつくやすいように作った木箱)	すばこ	
\\	スパスパ	【副词】 
\\	一口接一口地吸烟的样子(たて続けにタバコを吸うさま) タバコを~(と)吸う/一口接一口地吸烟 
\\	喀嚓喀嚓。连续、麻利地切东西状(手際よく、たて続けに切るさま) 大根を~(と)切る/喀嚓喀嚓地切萝卜 
\\	轻松顺畅地处理事情状(ためらわずにどんどん物事を行うさま)		
\\	滑る	【自动词・五段/一类】 
\\	(在物体表面)滑行,滑动(物の上をなめらかに移動する) 
\\	站不住脚,打滑(にじり移る) 
\\	不及格,没考上(落第する) 
\\	下跌,跌落(位を降りる) 
\\	说漏嘴(思わず言う)	すべる	
\\	李	【名词】 李树;李子(果樹の一つ。春、白い花を開く落葉高木。小形の実は、桃よりすっぱい。中国原産)	すもも	
\\	せかせか	【副词】 急急忙忙,慌慌张张(動作・態度が忙しそうで落ち着きのないさま) 中年の婦人が~と店に入って行った/一个中年妇女急急忙忙地进了商店		
\\	世論	【名词】 世论,舆论(世間一般の論) 【名词】 舆论(世間一般の人が唱える論。社会大衆に共通な意見)	せろん よろん	
\\	背を向ける	【惯用句】 不服从,不理睬,违规,态度冷淡(無関心な態度をとる。そむく。従わない)	せをむける	
\\	宣教	【自动词・サ变/三类】 传教,布道(宗教を教え広めること。伝道。布教)	せんきょう	
\\	戦時	【名词】 战时。战争时期(戦争をしている時期)	せんじ	
\\	喘息	【名词】 喘息,哮喘(あえぐこと。いきのせくこと。または、発作的に呼吸困難を起こす病気)	ぜんそく	
\\	センチ	【名词】 【英】
\\	公分(メートル法の単位に冠して100分の一の意を表す語)		
\\	草原	【名词】 草原;草原地带(草の茂っている野原)	そうげん くさはら	
\\	創造	【名・他动词・サ变/三类】 创造(新たに造ること)	そうぞう	
\\	想像力	【名词】 
\\	想象力。想象的能力或功能,大致区分为再现过去的表象、创造全新的形象等 
\\	想象力。在康德理论中以感性和悟性为媒介使认识成立的能力	そうぞうりょく	
\\	増設	【名・他动词・サ变/三类】 增设(設備をさらに追加してつくること)	ぞうせつ	
\\	添える	【他动词・一段/二类】 
\\	添,附加,附上(くわえる);配上(組み合わせる) 景品を~/附带赠品 錦上に花を~/锦上添花 
\\	伴随,陪同(付きそわせる)	そえる	
\\	損なう	【他动词・五段/一类】 
\\	【接尾】损坏,破损(物をこわす。傷つける) 器物を~/打破器皿 
\\	伤害,损害(悪い状態にする。害する) 感情を~/伤害感情 
\\	(接尾)没成功,失败,错误,…坏;…错。失掉时机,耽误。差一点,险些(…するのに失敗する。…する機会をうしなう) 
\\	损伤,死伤(人を殺傷する) 一兵も~ことなく勝つ/未损一兵而胜	そこなう	
\\	備え	【名词】 
\\	准备。预备(あらかじめしておく用意。準備) 老後の~に貯金をする/为了晚年的生活而存款 
\\	警备,戒备,防备(守り。防備。また、軍の構えや隊列) ~を厳重にする/严加防备	そなえ	
\\	備わる	【自动词・五段/一类】 设有,具有,具备(必要なものが不足なくそろい、整っている) 品位が~/很有品格(体面)	そなわる	
\\	備え付ける	【他动词・一段/二类】 设置,备置,装置(ある場所に置いて使えるようにしておく。設けておく) 教室にテレビを~/在教室装一台电视	そなえつける	
\\	存続	【名・自他・サ变/三类】 长存,继续存在,永存(引き続き存在すること。また、引き続き残しておくこと)	そんぞく	
\\	台無し	
\\	弄坏,糟蹋,白费,事情完全变糟。断送(物事がすっかりだめになること) 彼の一生を~にしてしまった/把他的一生断送掉了 
\\	作践。糟蹋。不能使用(物事がすっかりだめになる・こと(さま)) せっかくの苦労が~だ/一番辛苦白费了	だいなし	
\\	ダイナミズム	【名词】 
\\	【英】
\\	感染力。强大的力量。迫力(そのものがもつ力強さ。迫力) 
\\	表现机器和人的有力动作的艺术流派。未来派的主张 
\\	物力论。力本说(力本説)		
\\	タイプ	【名词・他动词・サ变/三类】 【英】
\\	型;类型,共同的特性;样式,格式;打字机 【英】
\\	(用打字机)打字		
\\	台風	【名词】 台风(北太平洋西部の熱帯海上、北緯5~20度付近で発生し、最大風速が毎秒17。2メートル以上の熱帯低気圧)	たいふう	
\\	高が	【副词】 充其量;顶多((「全額を数えても」の意から)たかだか。せいぜい。見くびった気持で使う) ~百円ぐらいの品/顶多一百日元左右的东西	たかが	
\\	宅配	【名・他动词・サ变/三类】 送到家里,(由专人把商品、杂志、节目表、招待券等)送到家里(新聞、荷物、牛乳などを家まで配達すること) ~便/(把商品等)送货上门	たくはい	
\\	企み	【名词】 企图,阴谋,坏主意(たくらむこと。計略。陰謀) 暗殺の~/暗杀的阴谋 それにはなにか~があるらしい/这里面象是有什么阴谋	たくらみ	
\\	佇む	【自动词・五段/一类】 伫立『書』,站着(立ち去ることが出来ないで、そこにしばらく居る) 庭に~/伫立在庭园里	たたずむ	
\\	尊い	【形容词/い形容词】 珍贵的,宝贵的,尊贵的,高贵的(崇高で近寄りがたい。神聖である。また、高貴である) ~仏像/尊贵的佛像	たっとい とうとい	
\\	縦	【名词】 
\\	纵,竖(上下の方向。垂直の方向。また,その長さ) ~に書く/竖着写 
\\	长,宽(長い方向)	たて	
\\	譬える	【他动词・一段/二类】 比方,比喻。为通俗易懂,用相类似的事物或具体的事物进行比拟说明(わかりやすく説明するために、ある物事を引き合いに出していう。なぞらえる) 人生を旅に~/人生如旅	たとえる	例える
\\	七夕	【名词】 
\\	织女星(織女星) 
\\	七夕,乞巧节(五節句の一。七月七日に行う牽牛星と織女星を祭る行事)	たなばた	
\\	田畑	【名词】 水田和旱田,田地,耕地(田と畑。耕作地)	たはた でんぱた	
\\	旅先	【名词】 旅行目的地。旅途中到达的地方(旅行をしている土地。旅行先)	たびさき	
\\	多忙	繁忙,忙碌(事が多くて忙しいこと) ~な1週間/繁忙的一个星期 ~をきわめる/十分忙;异常繁忙	たぼう	
\\	魂	【名词】 
\\	魂,灵魂,魂魄(人の肉体に宿り,生命を保ち,心の働きをつかさどると考えられているもの。肉体から離れても存在し,死後も不滅で祖霊を経て神霊になるとされる。霊魂) 死者の~をなぐさめる/安慰亡魂 
\\	精神;精力;心魂(気力。精神。心) ~を奪う/吸引住;迷住;夺人心魂	たましい	
\\	タレント	【名词】 
\\	【英】
\\	才能,技能,本事(才能、技量) 
\\	【英】
\\	明星,电视、广播等大众传媒中的演员、歌星、主持人等		
\\	短時間	【名词】 短时间,短期(短い時間。わずかの間) ~で完成する/短时间完成	たんじかん	
\\	単調	【名词・形動】 单调,平庸,无变化;无抑扬顿挫(起伏・変化に乏しくて、がまんがならない様子) ~な色/单调的色彩 ここの景色は~だ/这里景色单调	たんちょう	
\\	血	【名词】 
\\	血,血液(けつえき) 
\\	血缘,血脉(ちすじ)	ち	
\\	違う	【自动词・五段/一类】 
\\	不同 
\\	差别,不一样(普通と異なる) 
\\	错误,不对(合わない。相違する。誤る。たがう) 
\\	不合,不应该,有出入(互いに行き外れる) 【接头/接尾】 〈接动词连用形后〉交叉,交错(交差する) 擦れ~/擦肩而过 【自动词・五段/一类】 
\\	(老)不一致,不符合〔ちがう〕 
\\	违背,违反〔そむく〕 法に~/违反法律;违法 約束に~/违约	ちがう たがう	
\\	地下室	【名词】 地下室(地面より低く設けた部屋)	ちかしつ	
\\	縮こまる	【自动词・一段/二类】 抽缩,卷曲。蜷曲身体而变小(寒さや緊張などのために、からだや気持ちが小さくなる)	ちぢこまる	
\\	着実	【名词・形動】 踏〔塌〕实;扎实,牢靠;稳健(地道で危なげがないこと) ~な考え/稳健的想法 ~に仕事をする/踏踏实实地工作	ちゃくじつ	
\\	注意深い	【形容词/い形容词】 谨慎的,注意的(注意する度合が深いさま)	ちゅういぶかい	
\\	張本人	【名词】 祸首,罪魁,肇事者(その事件を起こすそもそもの原因となった人。または、悪事などのたくらみをした首謀者) 事件の~/肇事者 今回の爆発を企てた~/策划这次爆炸案的祸首	ちょうほんにん	
\\	調理	【名・他动词・サ变/三类】 烹调,烹饪,做(菜),煮菜(料理すること。割烹。物事をととのえおさめること)	ちょうり	
\\	塵	【名词】 
\\	尘土,尘埃,尘垢,细小的垃圾(小さなごみ。ほこり) 
\\	微小,微不足道〔取るに足らぬもの〕 
\\	少许,一点,丝毫〔ほんのわずかなこと。ほんの少し〕 
\\	世俗,尘世,红尘(世俗のよごれ) 
\\	肮脏,污垢〔汚れ〕	ちり	
\\	追求	【名・他动词・サ变/三类】 追求(目的とするものをねばり強く追い求めること)	ついきゅう	
\\	追放	【名・他动词・サ变/三类】 
\\	放逐,驱逐(出境);赶出,赶走,赶出去(追い払うこと。追い出すこと。危険人物または不法入国者を国外に退去させること) ~された/被赶出去了 
\\	驱除,肃清,洗刷,扫除(不要または有害なものとして、その社会から追い払うこと) 
\\	开除,革职,清洗(一定の職業・地位から退かせること)	ついほう	
\\	使い付ける	【他カ下二】 使いなれている。しばしば使う	つかいつける	
\\	突き進む	【自动词・五段/一类】 突破,冲破;冒着,顶着;奋勇前进,一往无前,猛冲(障害をものともせず勢いよく進む) 敵陣を~/冲过敌阵 敵陣めざして~/冲向敌人阵地	つきすすむ	
\\	捏ねんと	
\\	〔ぼんやりと〕呆然 ~座っている/呆呆地坐着 
\\	〔ひとりで〕孤零零地.呆然,孤零零地	つくねんと	
\\	伝え	【名词】 
\\	传言,口信,传话,通知(ことづて。伝言) 彼の~はもう~てきた/他的口信已经传到了 
\\	传说;传闻(言い伝え。伝説) 昔の~によれば/据古来的传说	つたえ	
\\	躑躅	【名词】 踯躅,杜鹃花,映山红。杜鹃花科杜鹃花属植物的总称,常绿或落叶性灌木,自生于山地,也作栽培。4-5月份在枝端生数朵漏斗形的美丽花朵。园艺品种很多,如野杜鹃、九州杜鹃、五月杜鹃等	ツツジ	
\\	露	【名词】 
\\	露,露水(空気中の水蒸気が地面近くの冷たい物体の表面に凝結して水滴となったもの。温度が露点以下になるとできる) 
\\	泪(涙のたとえ) 
\\	短暂,无常(打ち消しを強調する気持ちを表す。 少しも。夢にも) 【形容动词/ナ形容词】 
\\	袒露,裸露(むき出しなさま。多く、人間の肉体についていう) 
\\	显露,公然(気持ちや意見を隠さないさま) 
\\	公然,大白(明らかになるさま。公になるさま)	つゆ あらわ	
\\	貫く	【他动词・五段/一类】 
\\	贯通,穿透,穿过(突き通す。端から端へ通す。つきぬく) 山を~工事/贯通大山的工程 
\\	贯彻(始终),达到(目的),坚持到底(始めから終りまでし遂げる。果たす) 初志を~/贯彻初衷	つらぬく	
\\	吊り下げる	【他动词・一段/二类】 挂,悬挂(何かで吊るようにして、下げる) 軒に提灯を~/屋檐上吊着灯笼	つりさげる	
\\	丁重	【名・形动】 
\\	很有礼貌,郑重其事,彬彬有礼(礼儀正しい) 
\\	恳挚,诚恳(まごころがこもる) 
\\	殷勤(ねんごろ) ~なあいさつ/郑重的寒暄 ~にほうむる/厚葬 ~に断る/郑重回绝;婉言谢绝	ていちょう	
\\	手軽	【名词・形动】 简单,简便,轻易(手間のかからないこと。楽に処理できること。また,そのさま。安直) ~な食事/便饭;小吃 ~な方法/简便的方法 ~に引き受ける/痛快地答应下来	てがる	
\\	手数	【名词】 
\\	费事(ことをするのに必要な労力。それが多くて面倒なこと) ~を省く/省事 ~のいらない仕事/不费事的工作 
\\	费心,麻烦(煩わしいこと)	てすう	
\\	掌	【名词】 手掌。从手腕起手的内侧面(手首から先の、握った時に内側になる面。たなごころ) ~に載せて持つ/放在手掌上托着 ~を返す/态度突然变;翻脸不认人	てのひら	手の平
\\	転換	【名词・自他动·三类】 转换,转变(別のものに変えること。向きを変えること。また、変わること)	てんかん	
\\	天使	【名词】 天使,安琪儿,温柔可爱的人(天界にあり、神の使者として人間に神意を伝えたり、人間を守護したりすると信じられるもの。心の清らかな、やさしい人のたとえ)	てんし	
\\	天成	【名】 
\\	天然形成、自然形成(天然にできあがること) 
\\	天生(生まれつき)	てんせい	
\\	伝達	【名・他动词・サ变/三类】 传达,转达(命令・連絡事項などを伝えること。つぎつぎに伝え届けること)	でんたつ	
\\	転倒	【名・自他・サ变/三类】 
\\	跌倒(倒れること) 
\\	颠倒(さかさまになること) 
\\	惊慌失措,(吓得)神魂颠倒(あわてふためくこと。動転)	てんとう	
\\	店内	【名词】 店铺内部(みせのうち)	てんない	
\\	伝来	【自动词・サ变/三类】 
\\	(从外国)传来,传入(外国から伝わってくること。渡来) 
\\	祖传,世传(代々うけつぎつたえること。相伝)	でんらい	
\\	東京タワー	【名词】 东京塔(東京都港区芝公園にある電波塔。放送・通信用アンテナとして使用。展望台もあり、東京名所として有名)	とうきょうたわー	
\\	同質	【名词・形动】 同质,同一性质(二つ以上のものの質が同じであること。また、そのさま)	どうしつ	
\\	同席	【自动词・サ变/三类】 
\\	同席,同桌。(同じ席に連なること) 
\\	同席位,同等地位(同じ席次や地位)	どうせき	
\\	到底	【副词】 无论如何也,怎么也,无法(あとに否定の語を伴う。いかにしても。どうしても。とても) ~まにあわない/怎么也赶不上 ~逃げきれない/怎么也逃脱不了 ~信じられない/无论如何也不相信 ~彼には及ばない/跟他简直无法相比	とうてい	
\\	尊ぶ	【他动词・五段/一类】 
\\	敬重,尊崇。认为神圣而崇拜敬重(神聖なものとして、あがめ敬う) 先祖を~/敬重祖先 
\\	尊重,珍视。珍重值得尊敬的、有价值的东西(尊敬すべきもの、価値あるものとして、重んじる) 勤労を~/珍视劳动	とうとぶ たっとぶ	
\\	糖尿病	【名词】 〈医〉糖尿病	とうにょうびょう	
\\	豆腐	【名词】 豆腐(大豆の蛋白質を固めた白く柔らかい食品。豆乳ににがりなどを入れて凝固させたもの)	とうふ	
\\	当分	【名词・副】 暂时,一时(将来のある時期までをばくぜんと表す語。しばらくの間) ここ~/目前 ~の間休業する/暂时停业一下 雨は~やむまい/雨一时不会停	とうぶん	
\\	道楽	【自动词・サ变/三类】 
\\	(业余的)爱好,嗜好,癖好〔本職以外の趣味にふけること。趣味を楽しむこと。また,その趣味 
\\	着~/讲究穿 ~に絵をかく/作为消遣而画画儿 
\\	吃喝嫖赌,放荡,不务正业,堕落〔身持ちの悪い。酒色・ばくちなどの遊興にふける・こと(さま)〕 ~をしつくす/吃喝嫖赌全都干 ~息子/放荡的儿子,浪子	どうらく	
\\	道路工事	【名】 修路作业	どうろこうじ	
\\	どぎまぎ	【副词】 慌张,慌神,惊慌,张皇失措(平静さを失ってうろたえてあわてるさま。どまどま) 指名されて~する/突然被点名有点惊慌失措 突然のことで~した/因突发事件而慌神		
\\	読者	【名】 读者(新聞・雑誌・書籍などを読む人。読み手)	どくしゃ	
\\	独占	【名・他动词・サ变/三类】 
\\	独占(ひとりじめにすること) 
\\	垄断,专营(市場における売手または買手がただ一人の状態)	どくせん	
\\	閉ざす	【他动词・五段/一类】 
\\	关闭(閉める) 
\\	封,封闭,封锁(さえぎる) 道を~/封锁道路 
\\	封上,封住(閉じこめる) 
\\	憋在心里(気分がおもくるしい)	とざす	
\\	富む	【自动词・五段/一类】 
\\	富裕(財産が増える) 
\\	丰富(たくさん持っている) 魚介類に~海/盛产鱼类和贝类的海洋 示唆に~/富有启发性	とむ	
\\	名	【名词】 
\\	名,姓名,名字(人が認識した事物に,他の事物と区別するために言葉で言い表した呼称。名前) 
\\	名称(物の名称) 
\\	名分(名分) 
\\	名声,名誉,名气(その呼び名とともに世にあらわれた評判;よい評判。名声) 【名词】 知名,有名(名詞の上に付いて,すぐれている,評判が高い,などの意を表す)	な めい	
\\	ナイロビ	【名】 奈洛比(肯尼亚首都)		
\\	中庭	【名词】 中庭;(建筑物之间的)院子;中庭院子;里院(屋敷の中の建物や塀で囲われた庭)	なかにわ	
\\	眺め入る	【自他・五段/一类】 凝视、用心眺望(長時間熱心に見入る。じっと見る)	ながめいる	
\\	馴染み	【名词】 
\\	熟人,熟识,相好(慣れ親しんで知っていること。また、その人) 
\\	熟客(同じ遊女の下に通い慣れること。また、その客。遊女の側からも言う)	なじみ	
\\	夏目漱石	【名词】 夏目漱石。人名。小说家,英国文学研究家。本名金之助,生于江户,东京大学毕业。与森鸥外齐名的近代文学巨匠,曾在《朝日新闻》上连续发表作品,也精通俳句,汉诗和书画	なつめそうせき	
\\	ニーズ	【名词】 【英】
\\	需要,必要,要求(必要)		
\\	肉食	【自动词・サ变/三类】 
\\	吃肉,食肉〔菜食に対して鳥獣の肉を食べること。にくじき〕 
\\	肉食〔草食に対して動物が他の動物を食べて生きていくこと〕	にくしょく にくじき	
\\	肉体	【名】 肉体,血肉的身体(なまみの体)	にくたい	
\\	逃げ込む	【自动词・五段/一类】 逃进,躲入,(比赛中)甩开追上来的对手而取胜(逃げて、ある場所へ入り込む。競技などで、逃げ切って勝ちをおさめる) 雨に降られて軒下へ~/下雨躲入廊下避雨	にげこむ	
\\	偽	【名词】 假,假冒;赝品。仿真东西进行制造,亦指这种制品(本物に似せて作ること。また、そのもの)	にせ	贋
\\	煮物	【名词】 煮食,炖菜(食物を調味して煮ること。また、そうして作られた料理)	にもの	
\\	乳房	【名词】 乳房(哺乳類の胸·腹部の左右に対をなしてある)	にゅうぼう ちぶさ	
\\	縫いぐるみ	【名词】 
\\	(内填棉絮的)布制(动物)玩偶 
\\	兽形罩衣(演劇で、犬・猿・狐などの動物や怪獣の形に作った衣装)	ぬいぐるみ	
\\	泥濘	【名词】 泥泞(ぬかっているところ。泥深いところ)	ぬかるみ でいねい	
\\	値上げ	【名・他动词・サ变/三类】 提高价格,加价,加薪(物の値段や料金を高くすること) 水道料金の~/自来水费加价 賃金の~を要求する/要求提高工资	ねあげ	
\\	寝惚け	【名词】 睡迷糊(了的人);迷糊;发呆	ねぼけ	
\\	眠気	【名词】 睡意,困,困倦(眠いという感じ) ~を催す/发困,想睡 ~が襲う/(突然)睡意袭来	ねむけ	
\\	根元	【名词】 
\\	根(根のもと) 
\\	根本(物事の基本) 【名词】 根源(物事の一番もとになっているもの)	ねもと こんげん	
\\	年配	【名词】 
\\	大致的年龄(年のころ) 
\\	相当大的年龄;通晓世故的年龄(相当の年齢)	ねんぱい	
\\	燃費	【名词】 耗油量,每公升燃料行驶的公里数(ある距離を走ったり、ある仕事をしたりするのに必要な燃料の量。燃料消費量)	ねんぴ	
\\	燃料	【名词】 燃料(燃焼させて熱源とする材料)	ねんりょう	
\\	伸び伸び	【副词・自动词・サ变/三类】 
\\	生长茂盛,欣欣向荣(すくすく伸びるさま) 輝く春の日の下に青草が~と萌えていた/春日下青草萌发出嫩芽 
\\	悠闲,悠然自得(押さえられることもなく、ゆったりと落ち着きのあるさま) ~と仕事をする/悠闲地工作	のびのび	
\\	乗り合わせる	【自动词・一段/二类】 (和别人)偶然同乘(一辆车或一艘船)(同じ乗り物に乗る。たまたま同じ乗り物に,一緒に乗る)	のりあわせる	
\\	媒体	【名词】 媒介物,手段,(理)介质〔伝達などの手段。メディア〕	ばいたい	
\\	這い蹲う	【自动词・五段/一类】 匍匐(在地),拜倒在地(よつんばいに伏す。這うようにしてかしこまる。平伏する) 地面に~/匍匐在地	はいつくばう	
\\	派遣	【名・他动词・サ变/三类】 派遣,派出	はけん	
\\	発音	【名・他动词・サ变/三类】 发音(音を発すること。音を出すこと。特に、音声を発生調整する作用)	はつおん	
\\	発電所	【名词】 发电站,发电厂(発電機を設置し、水力・火力・原子力・地熱・風力などを利用して発電機を回転し電力を発生するところ)	はつでんしょ	
\\	発露	【名词】 流露,表现(心の中にあるものや隠していたことがおもてに現れ出ること。また、現し出すこと)	はつろ	
\\	花畑	【名词】 花田;花圃;天然的花卉地带(草花を植えた畑;山で、草花が一面に生えている所)	はなばたけ	
\\	食み出す	【自动词・五段/一类】 (由内部、当中)溢出,挤出;超出限度;超出范围(一定の範囲・枠から外に出る。中に入りきれないで外へあふれ出る。はみでる)	はみだす	
\\	ばら撒く	【他动・一类】 
\\	撒播,散布(ばらばらに散らしてまく。方々にまき散らす) うわさを~/传播谣言 
\\	到处花钱(金銭や物品を多くの人に配る) 金を~/到处花钱	ばらまく	ばら蒔く
\\	遥か	【形容动词/ナ形容词/副】 遥远,远远,远为(距離・時間の非常に隔たっているさま) ~な世界/遥远的世界 ~に富士山を望む/遥望富士山	はるか	
\\	ばれる	【自动词・一段/二类】 
\\	(谎言等)揭穿。露马脚(嘘や隠し事などが他人に知られてしまう。露見する。発覚する) 嘘をついても、すぐ~よ/就算你撒谎,也会马上败露的 
\\	钓到的鱼逃走(釣りで、針にかかった魚が逃げる) 
\\	约定的事情泡汤 (約束事などがだめになる。破談になる) 
\\	说猥琐的话(みだらな話をする。しもがかる)		
\\	ハンモック	【名词】 【英】
\\	吊床(丈夫なひもを網状に編み、木や柱に両端をつって用いる寝床。つり床)		
\\	反論	【名词・自他サ】 反论,反驳(他人の意見や批判に対し論じ返すこと。また、その論)	はんろん	
\\	冷え込む	【自动词・五段/一类】 
\\	骤冷,气温急剧下降(すっかり冷たくなる) みぞれが降って~/下了雨雪骤然冷起来 けさはひどく~/今天早晨冷得很 
\\	着凉受寒,发冷(からだの中まで冷えてしまう) 体のしんまで~/冷得彻骨 
\\	没有活力(活気などがなくなる) 景気が~/不景气	ひえこむ	
\\	東アジア	东亚(アジア大陸の東部、太平洋に面する地域。極東とほぼ同義)	ひがしあじあ	
\\	引き離す	【他动词・五段/一类】 
\\	使分离,拉开,拆开。使疏远,离间(引っ張って離す。無理に離れさせる) 親と子を~/使父子疏远 ふたりの仲を~/离间两个人的关系 
\\	拉开,领先,拉大差距。遥遥领先,(将后面的人)拉下(很远)(後ろに続くものとの距離・間隔を大きくあける)	ひきはなす	
\\	非行	【名词】 不正当的行为,违背道德规范的行为,流氓行为。严重不良行为,失足。特指青少年的违法行为和违反社会规范的行为等(よくない行い。特に青少年が法律で禁じられたことや社会規範に反した行いなどをすること) 友の~をいさめる/劝阻朋友的不正行为 他人の~を暴く/揭发别人的不正当行为 ~化/流氓化	ひこう	
\\	久しい	【形容词/い形容词】 
\\	好久,许久。过去很长时间(長い時間がたつ) ~昔/很久以前 
\\	久违。好久才见面(しばらくぶりである) ~間ごぶさたいたしまして申しわけございません/久疏问候,真对不起	ひさしい	
\\	左回り	【名】 左旋;逆时针转;反转	ひだりまわり	
\\	浸る	【自动词・五段/一类】 
\\	浸,泡,浸湿(水などの中に入る。つかる) 川に~/浸在河里 
\\	沉浸,沉湎,沉醉,陶醉(在…)(ある状態や心境にはいりきる) 幸福に~/沉浸在幸福中	ひたる	
\\	筆舌	【名词】 笔墨和言词(文章に書くことと、口で言うこと。文章と言葉) ~に尽くしがたい/非笔墨言词所能形容	ひつぜつ	
\\	一雨	【名词】 一场雨,一阵雨(ひとしきり降る雨。ひとしめり)	ひとあめ	
\\	一口	【名词】 
\\	一口(一回口に入れること) 
\\	一点(ちょっと口を動かして食べること。また,それだけのわずかな量) 
\\	一言,一句话;三言两语(手みじかにかいつまんで言うこと。また,異なる物事を同じように扱って言うこと) 
\\	一股,一份(あるひとまとまり。株・寄付・出資などの一単位)	ひとくち	
\\	秘める	【他动词・一段/二类】 藏隐秘,隐藏,隐瞒(内に隠して人に知られないようにする。また、内部にもつ) 胸に~/藏在心里 可能性を~/隐瞒可能性	ひめる	
\\	表示	【名词・他动・三类】 表示,表达,表明(外部にはっきりとあらわし示すこと;表にして示すこと)	ひょうじ	
\\	病状	【名词】 病状,病情,病况(病気の状態。病人のようす)	びょうじょう	
\\	広まる	【自动词・五段/一类】 
\\	扩大(ひろくなる) 勢力範囲が~/势力范围扩大了 
\\	传播,蔓延(ひろく伝わる。あまねく行われる) うわさが~/谣言四起 仏教が~/传播佛教	ひろまる	
\\	ブーメラン	【名词】 【英】
\\	飞镖,飞来回镖。澳大利亚土人狩猎及礼仪用的木制投镖,有一种飞镖呈「へ」字形,投出后一边旋转一边飞行,如果击不中猎物则划一条曲线又回到手边		
\\	副食	【名词】 副食品((同おかず)副食,副食品。中国では,“蔬菜”などを除く副食品を売る食料品店を“副食商店”という)	ふくしょく	
\\	縁取る	【他动词・五段/一类】 加(上)边,饰边,镶边(もののまわりを、ほかのものでかこったりしてかざる) カーテンをレースで~/用蕾丝为窗帘镶边	ふちどる	
\\	不透明	
\\	不透明,浑浊(すきとおっていないさま。物質が光を通さないこと) 
\\	可疑,蹊跷(先行きが見通せなかったり、不審な点があったりするさま)	ふとうめい	
\\	赴任	【自动词・サ变/三类】 赴任,上任(任地へ赴くこと)	ふにん	
\\	ブラス	【名词】 
\\	【英】
\\	黄铜(真鍮(しんちゆう)。黄銅) 
\\	銅管乐器
\\	の略〕金管楽器。真鍮楽器)		
\\	降り注ぐ	【自动词・五段/一类】 (雨、阳光等)落下来,降下来((雨や日光などが)降りかかる。そそぐように降る) 陽光が~/阳光照射进来	ふりそそぐ	
\\	プロジェクト	【名词】 【英】
\\	研究课题;计划,设计;项目(研究や開発の計画。企画。研究課題)		
\\	付録	【名・他动词・サ变/三类】 
\\	附带的东西,增刊(書籍·新聞·雑誌に添えてある紙面または冊子その他のもの。おまけ) 
\\	(另外的)记录(付け加えて記録すること。また、その記録)	ふろく	
\\	文通	【自动词・サ变/三类】 通信,写信联系(文書で通信すること。手紙のやりとり。書信)	ぶんつう	
\\	閉鎖	【名・他动词・サ变/三类】 封闭,关闭(とざすこと)	へいさ	
\\	米食	【名词】 以大米为主食(米を食べること。また、米を主食とすること)	べいしょく	
\\	弁解	【名・自他・サ变/三类】 辩解,分辩,辩明。辩白(言い訳をすること。言い訳)	べんかい	
\\	変革	【名词・自动・三类】 变革,改革,变化(物事を変えて新しくすること。また、変わること)	へんかく	
\\	細長い	【形容词/い形容词】 细长的(ほそくてながい)	ほそながい	
\\	北海道	【名词】 北海道(日本列島の四大島の一。日本列島の北端)	ほっかいどう	
\\	マーケティング	【名词】 【英】
\\	[经营]市场学;销售学(消費者の求めている商品、サービスを調査して、供給する商品や販売活動の方法などを決定することで、生産者から消費者への流通を円滑化する活動)		
\\	薪	【名词】 木柴;劈柴,薪柴;柴火。为作燃料而截成适当长度的木头(燃料にするため適当な長さに切ったり割ったりした木。たきぎ。わりき)	まき	
\\	紛れる	【自动词・一段/二类】 
\\	混同,混淆,难以辨别(入りまじって区別がつかなくなる。また、はっきりしなくなる) 
\\	搀混,混杂,混进(入り混じる) 
\\	忘怀,忘(忧)(移って、悲しみなどを忘れる) 気が~/解闷,排遣,忘忧	まぎれる	
\\	おまけ	【名・他动词・サ变/三类】 
\\	(作为赠品)另外奉送,白送给(的东西)(景品や附録として付けること) 
\\	另外附加(的东西),附带(的东西)(後から付け加えた物) 
\\	减价,让价(商品の値段を安くすること)		
\\	勝る	【自动词・五段/一类】 比…好,胜过,强过,凌驾(他と比べて価値や能力などが上である。すぐれる。ひいでる) 健康は富に~/健康胜于财富	まさる	優る
\\	増し	【名・形动】 
\\	增,增加,增多〔増すこと〕 ~払い金/附加付款;追加付款 
\\	(比)好〔强〕些,(与其…)不如(宁可)…;胜过,胜于(他と比べて勝っていること) ないよりは~だ/有胜于无	まし	
\\	交じり	【名词・接尾】 混合物,杂质;混杂(まじること。また、まじったもの) 雨~の雪/雨夹雪	まじり	混じり
\\	マッチ	【自动词・サ变/三类】 
\\	【英】
\\	火柴,洋火『口』(摩擦によって発火する薬剤を細い軸木先端につけた発火具) 
\\	比赛,竞赛(勝負。試合) 
\\	调和,适称,相称;般配,谐调(調和が取れていつこと)		
\\	真夏	【名词】 盛夏,仲夏。夏天最热的时候(夏の盛り。夏のいちばん暑い時季) ~の果実/盛夏的果实	まなつ	
\\	満開	【自动词・サ变/三类】 (花)盛开,满开(花が十分に開くこと) 桜が~だ/樱花盛开	まんかい	
\\	右回り	【名词】 右旋转(右の方へ向かって回ること。時計の針の進む方向に回ること)	みぎまわり	
\\	見定める	【他动词・一段/二类】 看准,看清(見て確かにそれと決める。見きわめる) 真偽のほどを~/分不清真伪	みさだめる	
\\	見知る	【他动词・五段/一类】 见过,认识,熟识(前に会って知っている。面識がある)	みしる	
\\	蜜蜂	【名词】 蜜蜂(ミツバチ科の蜂の総称。蜜蝋・ロイヤル‐ゼリーを採るために広く飼養され、品種が多い)	みつばち	
\\	見守る	【他动词・五段/一类】 
\\	注视。监视(大きな関心を寄せて、ものごとの進みぐあいを見つづける) 相手の行動を~/监视对方行动 
\\	监护;照看;[子どもや病人を]照料(問題がおこらないように気をつける。注意して見る)	みまもる	
\\	身持ち	【名词】 
\\	品行,操行(品行。多く異性との交際にいう) ~のよい男/品行端正的男人 
\\	怀孕,妊娠(子を孕むこと。妊娠) ~の女/孕妇	みもち	
\\	見破る	【他动词・五段/一类】 看破,识破,看穿,看透(かくしていることを見ぬく) 変装を~/识破化装	みやぶる	
\\	無	【名词】 
\\	无(ないこと) 皆~/全无 
\\	无(哲学)(万有を生み出し、万有の根源とまるもの) ~の哲学/无的哲学 
\\	没有(あるものがないこと) ~益/无益	む	
\\	向こう	【名词】 
\\	前面,正面,对面(前方),正对面(正面。前方。また、前方の比較的離れた場所) 
\\	另一侧,另一边(物を隔てた反対側) 学校は駅の~にある/学校在火车站的那边 
\\	那边,那儿(自分からやや離れている方向・方面。あちらの方) 
\\	对方(相手。先方) ~が悪いのです/是对方的不是 
\\	从现在起,从今以后,今后(今後。これから先)	むこう	
\\	無心	【名・他动词・サ变/三类】 
\\	无心。和歌、连歌中,追求机智、滑稽的歌体(和歌・連歌で、卑俗・滑稽さを求めたもの) 
\\	(日本)狂歌(狂歌のこと。和歌を有心というのに対していう) 
\\	要,求,讨,索取。不客气地讨要(遠慮なく人に金品をねだること) 
\\	天真(無邪気であること。また、そのさま) 
\\	热中,专心致志『成』,一心一意『成』(心にわだかまりのないこと。雑念や欲心のないこと。また、そのさま〕 
\\	无心。像非生物或植物似地没有心(無生物や植物のように、心をもたないこと) 
\\	〈数〉无中心(中心がないこと)	むしん	
\\	無論	【副词】 不用说,当然(言うまでもなく。もちろん) ~彼も来ます/他当然要来 夫は~、子どもも賛成してくれた/不用说我丈夫,孩子也同意了	むろん	
\\	メディア	【名词】 
\\	【英】
\\	手段,方法,媒体(手段、方法、媒体。特に、新聞・テレビ・ラジオなどの情報媒体) 
\\	外存储媒体(情報を保存する外部記憶装置の媒体) 
\\	介质(情報を頒布する手段)		
\\	燃え移る	【自动词・五段/一类】 延烧;蔓延燃烧开来	もえうつる	
\\	目礼	【名词・自动・三类】 点头礼,点头致意(目を交わして礼をすること)	もくれい	
\\	模型	【名词】 模型,仿照实物制作的物品(実物にまねて作ったもの)	もけい	
\\	持ち込む	【他动词・五段/一类】 
\\	带入,拿进(運び入れる) 危険物を汽車に~/把危险物品带进火车 
\\	提出(意見・問題を持ってくる) 縁談を~/提亲事	もちこむ	
\\	持ち主	【名词】 持有者,所有人;物主(その物を所有している人。所有主) 家の~/房屋的主人	もちぬし	
\\	野外	【名词】 
\\	野外,郊外,原野(野原・郊外〕 
\\	户外,室外(建物の外。屋外〕	やがい	
\\	夜景	【名词】 夜景,夜色(夜の景色)	やけい	
\\	夜色	【名词】 夜色,夜景(夜の景色。夜景。また、夜の気配。夜の風情)	やしょく	
\\	安上がり	【名词・形動】 省钱;便宜(安い値段で済むこと) ~の旅行法/省钱的旅行方法	やすあがり	
\\	野党	【名词】 在野党(政党政治において、現在、政権にあずかっていない政党。在野の党)	やとう	
\\	柳	【名词】 柳,柳树(ヤナギ科ヤナギ属の落葉樹の総称)	やなぎ	
\\	野暮	
\\	庸俗,俗气,土气〔洗練されていないこと〕 ~な男/土包子 
\\	不知趣,不知好歹。不懂世路人情〔気がきかないこと〕	やぼ	
\\	闇	【名词】 
\\	黑暗(暗黒);黑夜(やみ夜) 
\\	(心中)无数,糊涂,辨别不清(心が迷う);不知所措『成』,不知如何是好(先の見通しがつかない) 
\\	黑暗(希望のない) 
\\	黑市;黑市交易;黑货(闇取引をやること)	やみ	
\\	弥生	【名词】 
\\	阴历三月〔陰暦3月〕 3月は~と呼ばれる由来はご存知ですが/您知道为什么阴历3月被称为弥生吗? 
\\	弥生(時代の名前。また、その時代の文化) ~時代/弥生时代 
\\	日本地名以及人名(日本の地名と人の名前)	やよい	
\\	遣り切れない	【接续】 
\\	完成不了,做不过来(やり遂げることができない) 
\\	受不了,应付不了,忍受不住(がまんできない。耐えられない) 暑くて~/热得让人难以忍受	やりきれない	
\\	幽霊	【名词】 
\\	幽灵,鬼魂(死者の霊。亡魂。死者が成仏(じようぶつ)できないでこの世に現すという姿。おばけ) 
\\	有名无实的(事物)(実際には存在しないものを形の上だけで存在するように見せかけたもの)	ゆうれい	
\\	誘惑	【名・他动词・サ变/三类】 诱惑。蛊惑人心并引入邪道,亦指这种引诱(心を惑わせ、悪い道へ誘い込むこと。また、その誘い)	ゆうわく	
\\	雪解け	【名词】 
\\	雪融、雪融时期(積もった雪が溶け出して水になること。また、その頃) 
\\	缓解、解冻、消解(対立・諍いが緩和すること)	ゆきどけ	
\\	行き届く	【自动词・五段/一类】 周到,周密,彻底,无微不至(隅々までよく注意が行き渡る)	ゆきとどく	
\\	ゆったり	【副词・自动词・サ变/三类】 
\\	有余地;宽敞(ゆとりのある) ~した着物/宽舒的衣服 
\\	舒畅;舒适(落ち着いた) 家に帰ると~とした気分になる/一回到家里就觉着心情舒畅		
\\	揚子江	【名词】 长江,扬子江(長江の通称。本来は揚州付近の局部的名称)	ようすこう	
\\	要請	【名词・他动・三类】 
\\	请求,要求(必要なこととして,実現を願い求めること。乞い求めること) 
\\	先决条件,必要,条件请求(公理ほど自明ではないが,証明なしに原理として立てられる命題)	ようせい	
\\	要望	【名词・他动・三类】 要求,迫切期望(物事の実現を強くのぞむこと)	ようぼう	
\\	予感	【名词・他サ】 预感,预兆(ある結果になることを前もってちょっとした事で感じること)	よかん	
\\	与党	【名词】 
\\	执政的政党,执政党(政党政治で政権を担当している政党または政権を支持している政党) 
\\	志同道合的伙伴(くみする仲間。同志。徒党。一味)	よとう	
\\	蘇る	【自动词・五段/一类】 
\\	〈意为从黄泉归来〉苏醒,复活((黄泉から帰る意)生き返る。蘇生する。失っていた活力を取り戻す) 
\\	恢复,失而复得(以前の積極的な状態に戻る) 記憶が~/过去的记忆被唤醒	よみがえる	
\\	弱まる	【自动词・五段/一类】 变弱,衰弱(弱くなる) 体力が~/体力衰弱	よわまる	
\\	ランプ	【名词】 
\\	【英】
\\	煤油灯;电球;电灯(石油を燃料とし、灯心に火をつけて、ほやをかぶせて用いる灯心) 
\\	(英) 
\\	灯管。电灯等的总称(電灯などの総称)		
\\	利子	【名词】 利息,利钱(金銭の貸借が行われた場合、その使用の対価として借り手が貸し手に支払う金銭。利息) ~を取る/要利钱 ~を生む/生息 ~を払う/付利息	りし	
\\	リスク	【名词】 【英】
\\	风险;危险〔危険、事業などにおける損害の可能性〕		
\\	理由	【名词】 
\\	理由,缘故(なぜそうなったかという筋道。また,なぜそうするかという根拠。わけ。事情) 
\\	借口(言いわけ。口実)	りゆう	
\\	隣家	【名词】 邻家、隔壁(となりの家)	りんか	
\\	例年	【名词】 往年,历年,常年(いつもの年。毎年)	れいねん	
\\	レコード	【名词】 
\\	【英】
\\	成绩,记录;最高记录(スポーツなどの記録) 
\\	唱片(音盤)		
\\	労働省	【名词】 劳动省,劳动部(労働に関する事項を担当した国の行政機関。外局として中央労働委員会などが置かれた)	ろうどうしょう	
\\	沸かす	【他动词・五段/一类】 
\\	烧开,烧热(熱を加えて水などを熱くする。また、煮えたたせる) お茶を~/煮茶;烹茶 
\\	使沸腾,使狂热,使兴高采烈(熱狂させる。夢中にさせる) 青年の血を~/使青年的热血沸腾	わかす	
\\	和食	【名词】 日餐,日式饭菜(日本風の食事。日本料理)	わしょく	
\\	渡す	【他动词・五段/一类】 
\\	交付,交给(人、物を別の人の手に移す) 家を人手に~/把房子交给别人 
\\	遍及;表示把一个动作一直做下去之意(その動作が全体に及ぶようにする意を表す) 
\\	赐予;授予;让给;转给(ほかの人の所有物とする) 
\\	架设(離れた2点に、物を跨らせて、つなぐ) 
\\	渡,送过河(水の上を、船、人手などによって、対岸へ移す)	わたす	
\\	である	【接续】 是,为。表示断定。比「だ」「です」稍显庄重的说法(主語と述語の間に論理的、整合性(関連性)が有ることを表す言葉。「だ」の荘重表現)		
\\	難産	【名词・自动・三类】 (胎儿)难产,(事情)难产(出産で、胎児がなかなか生まれないこと。物事がたやすく成立しないこと)	なんざん	
\\	シルエット	【名词】 
\\	【法】
\\	人影,影像(かげぼうし) 壁に彼女の~が映っている/墙上映出她的影子 
\\	黑色侧面投影头像,剪影(横顔の) ベートーベンの~/贝多芬的侧面投影头像 
\\	轮廓,外形(服装の) ~の美しい服/轮廓美丽的衣服		
\\	大勢	【名词】 许多人,很多人,众人(多くの人。多人数。副詞的にも用いる) 【名词】 大局,总的趋势;大势(おおよその形勢) 【名词】 多数的人;人数众多	おおぜい たいせい たいぜい	
\\	ゆとり	【名词】 宽裕,余裕,余地(物事に余裕があって、ゆったりとしていること)		
\\	一番	【名词】 
\\	最初,第一,最前列(順番、番号などの最初。また、最初のもの) 
\\	最好,最妙(多くの中で最も良いもの。最上) 
\\	一局,一盘,一场;一曲,一出(能、狂言、将棋、相撲など「番」で数えるものの一つ)【副词】 
\\	试试,一下;先(思い切って) 
\\	最,顶(最も。この上なく) 【名词】 
\\	(鸟等雌雄)一对(雌雄一対) 
\\	组队(番舞で左方と右方が各一曲ずつ演じること)	いちばん ひとつがい	
\\	行う	【他动词・五段/一类】 
\\	实行;进行;处理〔合目的的(有意的)に〕、何かをする) 
\\	举行〔あるルールに従って、何かをする〕 
\\	修行佛道;修行(「仏道を修行する」意の雅語的表現)	おこなう	
\\	真面	【名词・形动】 
\\	正面(まっすぐに向かい合うこと。正しく向かい合うこと。また、そのさま。真正面) 
\\	正经,正直,正派(その社会で認められる規定·方式にかなっている様子)	まとも	正面
\\	海鮮	【名词】 海鲜(海で獲れた新鮮な魚介類)	かいせん	
\\	春休み	【名词】 春假(春のお休み)	はるやすみ	
\\	抜け出す	【自动词・五段/一类】 
\\	溜;摆脱,脱出(ある場所・集まり・状態などからのがれ出る。抜けて出る) 
\\	开始脱落(抜け始める)	ぬけだす	
\\	早春	【名词】 早春,初春(春のはじめ。初春。浅春)	そうしゅん	
\\	葦	【名词】 芦苇(イネ科の多年草。根茎は地中をはい、沼や川の岸に大群落をつくる。高さは2~3メートルになり、茎は堅く、円柱形で、細長い葉が互生する)	あし	芦
\\	畦	【名词】 田埂;田界 【名词】 
\\	〔畑の〕垄 
\\	〔模様の〕垄状的棱纹;凸条 
\\	〔波の〕涟漪『書』;波峰	あぜ うね	畔
\\	悪化	【自动词・サ变/三类】 恶化;变坏(わるくなること)	あっか	
\\	圧倒	【名・他动词・サ变/三类】 压倒,胜过,超过(押し倒すこと。また、他を押さえつけること)	あっとう	
\\	アマチュア	【名词】 
\\	【英】
\\	业余爱好者,外行(職業ではなく、趣味や余技として物事を行う人。素人。対義語は「プロフェッショナル」) 
\\	【英】
\\	业余爱好者,业余艺术家,业余运动员(職業としてでなしに、趣味や余技として携わる人。愛好家。アマ) 
\\	并非该行道专家的人,外行,门外汉(その道の専門家ではない人)		
\\	案内状	【名词】 
\\	通知书。通知聚会或文娱活动的日期、时间和内容等的信函(催し物の日時・内容等を知らせる書状) 
\\	邀请函,请柬。请帖。邀请客人来的信函(招待客への書状。招待状)	あんないじょう	
\\	意	【名词】 
\\	意,心意,心情,意志,想法(心、気持ち、考え) 
\\	意义,意思(物事の内容。意味)	い	
\\	威圧	【名词】 威压,威迫,威逼;威慑;压制(威光や威力で、相手をおさえつけること)	いあつ	
\\	言い伝え	【名词】 传说(昔から口づてに伝えられてきた事柄。口碑。伝説・ことわざなど。ことづて。伝言) 地方に残っている~を集める/搜集地方上遗留下来的传说	いいつたえ	
\\	幾度	【名词】 
\\	好几次(若干の回数) ~戸をたたいても返事がなかった/敲了好几次门也没有人答应 
\\	多少次(どのくらいの回数)	いくど いくたび	
\\	池袋	【名词】 池袋。日本地名。东京丰岛区的一个区域(東京都豊島区の地名)	いけぶくろ	
\\	移行	【自动词・サ变/三类】 过渡,转移,移交(制度などが移り行くこと)	いこう	
\\	いざこざ	【名词】 (亲人或熟人之间的)纠纷,不和(ちょっとした言葉のやり取りや感情的な行き違いなどが原因で起こる争い事) 夫婦の~/夫妻间的小争吵		
\\	石	【名词】 
\\	石头;岩石;围棋子;石板;钻石;墓石;划拳时的石头;结石;坚硬 
\\	岩石。钻。石见国。围棋子 
\\	日制度量衡的容积单位。船或木材的体积单位。江户时代俸禄的单位	いし せき こく	
\\	痛み止め	【名词】 镇痛、止痛药(痛みをやわらげ、また除く薬)	いたみどめ	
\\	痛める	【他动词・一段/二类】 
\\	使(肉体)疼痛;损伤(〔体の部位に〕痛みを伴うような異状や故障を生じさせる) 
\\	使(精神、心灵)痛苦(心・頭に苦しめる) 
\\	弄坏,损坏(壊す)	いためる	
\\	一過性	【名词】 一时性,短暂性(現象が一時的ですぐ消えること) ~の熱病/一时性的热病	いっかせい	
\\	一神教	【名词】 一神教(一切を創造して支配する唯一絶対の神のみを認めて信仰する宗教。ユダヤ教・キリスト教・イスラム教など。唯一神教)	いっしんきょう	
\\	体感	【名词】 身体所受的感觉;(对内脏的刺激所引起的)体感(饥饿,呕吐等)(体に受ける感じ。体で感じること。内臓諸器官が受ける刺激によって起こる,飢え・吐き気・性欲などの感覚)	たいかん	
\\	一変	【名・自他・サ变/三类】 一变,完全改变。突然改变(すっかり変わること。また、変えること)	いっぺん	
\\	一遍	【名词・副词】 
\\	一遍。一回(1度。1回) 
\\	纯粹,仅仅是((名詞の下に付き、接尾語的に用いて)表面のみで、内実のこもらないさまを表す)	いっぺん	
\\	以南	【名词】 以南(その地点を含めて、それより南)	いなん	
\\	訝しい	【形容词/い形容词】 奇怪、可疑、令人诧异(疑わしい。不審だ)	いぶかしい	
\\	戒める	【他动词・一段/二类】 
\\	劝戒,劝告,规劝〔注意する〕 
\\	禁,戒,戒除〔禁止する〕 
\\	警戒,警惕〔用心する〕	いましめる	警める
\\	意味合い	【名词】 意义,含义。情况,详情,来龙去脉『成』。理由,用意((背後の事情を含めた)意味。理由。わけ) 微妙な~を含む/包含着微妙的含义	いみあい	
\\	入り江	【名词】 海湾、湖岔、小海湾、入水口(海や湖が陸地にはいり込んでいる所)	いりえ	
\\	入り組む	【自动词・五段/一类】 错综复杂『成』,互相纠缠,头绪纷繁『成』(物事がさまざまに交じり合って、複雑になる)	いりくむ	
\\	入り込む	【自动词・五段/一类】 
\\	进入,挤进,钻进去。强行进到里面去(強引に中にはいって行く。はいりこむ) 
\\	混入,潜入(まぎれこむ) 
\\	深入,伸入(奥へ入る) 
\\	错综复杂。交错。事物复杂地相互纠缠(物事が複雑に絡みあっている。入り組む)	いりこむ	
\\	入り日	【名词】 落日,夕阳。将要西下的夕阳(夕暮に西に沈もうとする太陽。また、その光。沈もうとする太陽。夕日。落日。西日)	いりひ	入日
\\	動き	【名词】 
\\	动,活动,移动(動くこと。また、動くようす) 
\\	动向,动态,变化,变动(状態・情勢・内容などが変わること。変動。変化)	うごき	
\\	後ろ姿	【名词】 后影,背影(後ろから見た姿。後ろ手)	うしろすがた	後姿
\\	薄	【接头/接尾】 那样的成分并不算很大,不大,少(そのことが、あまり多いとはいえない) 気のり~/兴趣不大 見込み~/希望不大 手持ち~/存货少	うす	
\\	鬱陶しい	【形容词/イ形容词】 
\\	厌烦,不痛快,因为碍事而讨厌的,令人厌烦,麻烦的,腻烦的〔わずらわしい。妨げになってうるさい〕 
\\	郁闷,阴郁,沉闷,沉重而阴暗,心情不愉快〔気分・天候がよくない。重苦しく陰気である〕	うっとうしい	
\\	俯く	【自动词・五段/一类】 
\\	俯首,垂头,低头,脸朝下(首を少し折って、頭を垂れる。下を向く) 恥ずかしくて~/羞得低下头 
\\	向下倾斜(直立したものの先端が曲がって下を向く) 花が~/花搭拉下来	うつむく	
\\	生まれ立て	【名词】 刚生下(生まれて間もないこと。また、そのころ)	うまれたて	
\\	生み出す	【他动词・五段/一类】 
\\	生(出),产(出)(胎児または卵を生む) 
\\	产生出,创造出(新しく作り出す) 
\\	开始产(生み始める)	うみだす	産み出す
\\	羽毛	【名词】 羽毛,羽翎,绒毛。生于鸟类体表面毛的一种,表皮的变形物,年脱换一二次(鳥類の体の表面に生える毛の一種。表皮の変形したもので、皮膚の保護や保温に役立ち、一年に一、二回抜けかわる)	うもう	
\\	エキスポ	博览会之简称 (英) 
\\	エレクトロニクス	【名词】 【英】
\\	电子学,电子工学(電子伝導、およびその現象を応用する装置・技術についての学問)		
\\	遠近	【名词】 远近,遐迩,到处;现在与未来(遠い所と近い所。あちらこちら;将来と現在。昔と今) ~人/远近的人们 【名词】 远近(遠い所と近い所。遠いことと近いこと) ~を問わず,無数のファンが集まってきた/无数的粉丝不论距离远近都集合起来了	おちこちえんきん	
\\	縁結び	【名词】 结婚,结亲(男女の縁を結ぶこと。縁組み) ~びの神/月下老人	えんむすび	
\\	欧州	【名词】 欧洲(ヨーロッパ)	おうしゅう	
\\	大慌て	【名词】 非常惊慌,非常着急(大いにあわてること) ~で出掛ける/着急地出门了	おおあわて	
\\	大方	【名词】 
\\	大部分,大体,大致(全体のほとんどの部分を指す表現) 
\\	一般人,大家,诸位(世間一般) 【副词】 
\\	大概(確かではないが,そうとらえてまず間違いないさま。恐らく。多分) 
\\	大约,大致(全体の大部分に及ぶさま。だいたい。あらかた) 
\\	(后接否定)丝毫(也不)((打ち消しの語を伴って)全然。少しも) 【名词】 高明的人,有学识的人,大方(度量の大きいこと。また、そういう人。学問・見識の高い人)	おおかた たいほう	
\\	大型	【名词】 
\\	在同类物品中形状大,大型(同類のものの中で、形が大きいこと。また、大きいもの) 
\\	形状或规模大的东西,大型(形や規模の大きいこと。また、大きいもの)	おおがた	
\\	大手	【名词】 
\\	臂,胳膊 
\\	大摇大摆 (大手を振る) 
\\	明目张胆 【名词】 
\\	大户头;大企业,大公司〔大手筋〕 
\\	正面进攻部队 
\\	前门,正门〔城の表門)	おおで おおて	
\\	大らか	【形容动词/ナ形容词】 
\\	落落大方,胸襟开阔,豁达,胸襟豁达。大方。心胸开阔(心がゆったりとして、こせこせしないさま) 
\\	分量多(分量が多いさま)	おおらか	多らか
\\	オギャア	【副词】 呱呱(赤ん坊の泣き声を表す語) ~と産声をあげる/呱呱落地	おぎゃあ	
\\	重ね	【名词】 
\\	重叠(重ねること) 
\\	重叠地穿衣服(袍の下に重ねてきた衣服) 
\\	内衣,外衣成套的衣服(衣の上着と下着との備わった物) 
\\	套,层(重なった物を数える語)	かさね	
\\	押し切る	【他动词・五段/一类】 
\\	压着切,切断,铡(押し付けて断ち切る) 
\\	敢,排除(反对,困难);坚持到底,硬干到底(困難を排して目的を果たす。おしとおす)	おしきる	
\\	押し止める	[動マ下一] 制止、阻止。拦住。扣住。ある行動に出ようとするのを、押さえてやめさせる。制止する 「転職しようとする娘を―・める」	おしとどめる	
\\	惜しむ	【他动词・五段/一类】 
\\	珍惜,爱惜
\\	無駄に失われないよう、大切にする) 命を~惜命;珍惜性命 
\\	惋惜(価値あるものが失われたこと、また、活用されずに終わることを残念に思う) 
\\	吝惜,小气(自分の金銭や物品を大切に思い、使わずに済ませようとする)	おしむ	
\\	押し寄せる	【自他・一段/二类】 
\\	涌来,涌过来,涌上来,蜂拥而至,麇集而来〔どっとせまる〕 大ぜいの人が~/很多人蜂拥而来 
\\	推到一旁,挪到一边〔押して近づける〕	おしよせる	
\\	お節介	【名・形容动词/ナ形容词】 好处风头爱管闲事,好管闲事的人(かえって迷惑になるような余計な世話をやくこと。また、そのような人やさま) ~な人/爱管闲事的人 いらぬ~/多管闲事,好管闲事	おせっかい	
\\	節介	【名词】 管闲事,多嘴多舌(よけいな世話をやくこと) よけいなお~はよしてくれ/少管闲事! ~焼き/爱管闲事的人	せっかい	
\\	躍らす	【他动词・五段/一类】 
\\	跳,蹦跳(おどるようにさせる) 
\\	使之跳动,使之受到鼓舞(他人を思いどおりに動かす) 
\\	欢悦(期待や喜びでわくわくする) 
\\	跳起来(素早くからだを跳躍するように動かす)	おどらす	踊らす
\\	踊らす	【他动词・五段/一类】 
\\	使跳舞,让跳舞(踊るようにさせる) 
\\	操纵,摆弄(人を操って、自分の思いどおりにさせる)	おどらす	
\\	衰え	【名】 衰弱,衰老;衰落;衰败;衰亡;衰微(衰えること) 健康の~がめだってきた/健康状况显著地衰弱了 頭の~/脑筋衰退;记忆衰退	おとろえ	
\\	彼岸	【名词】 
\\	春分(秋分)周;春分(秋分)季节(春分や秋分の日を中日とした、前後各三日を合わせた七日間。このあいだに法要や墓まいりをする) 
\\	对岸,对面(対岸。むこう岸) 
\\	彼岸,涅槃岸(仏教で、さとりをひらいた境地)	ひがん	
\\	おまけに	【副词】 又加上,更加上,而且,况且,加之『書』(さらにその上) きょうは非常に暑い、~風がちっともない/今天很热,并且一点风也没有 彼は貧乏で~病人ときている/他既穷又有病		
\\	御守り	【名词】 护身符(神や仏の守り札。おふだ)	おまもり	
\\	思い付き	【名词】 
\\	一时想起,随便一想,灵机一动。偶然浮现在心头。偶然的想法,凭一时高兴,主意,设想(思いついたこと。ふと浮かんだ考え) ~出物を言うな/不要想起来就说 
\\	好主意,好办法。设想。好的提案(いい考え。おもしろい着想) 私~をした/我想到一个好办法	おもいつき	
\\	思い煩う	【自动词・五段/一类】 愁思,烦恼(あれこれ考えて悩む)	おもいわずらう	
\\	思える	【自动词・一段/二类】 总觉得。看来(自然とそう思うようになる) 人生は楽しいもののようにも~/看来人生还是快乐的	おもえる	
\\	面白がる	【自他・五段/一类】 感觉有趣(面白いと思う。おかしがる) しゃれを聞いて~/听俏皮话觉得有趣	おもしろがる	
\\	親不孝	【名词・形动・他动・三类】 不孝,不孝顺父母者(親を大切にしないで、心配や迷惑をかけること。また、そのさまや、その人) ~な娘/不孝女	おやふこう	
\\	織り込む	【他动词・五段/一类】 
\\	织入,织进去(金銀糸や模様などを織り入れる) 銀糸を~/织上银线 
\\	穿插,编入,采纳(一つの物事の中に他の物事を組み入れる。加味する) ストーリーの中に~/穿插在故事里	おりこむ	
\\	折り込む	【他动词・五段/一类】 
\\	向里折,折入,折进(内側に折る。また、折り曲げて中の方へ入れる) 端を2寸~/把边儿折进去二寸 
\\	夹入,叠入(折ってはさみこむ) 広告ビラを新聞に~/把广告宣传单夹在报纸里	おりこむ	
\\	おんぼろ	【名词】 破旧,破烂不堪(ぼろぼろなこと。ぼろぼろなもの) ~の着物を着る/穿破烂的衣服 ~電車/破旧的电车		
\\	向き合う	【自五】 相对,相向,面对面(互いに正面をむいて相対する)	むきあう	
\\	妄想	【名词】 妄想,邪念,胡思乱想。幻想(根拠のない誤った判断に基づいて作られた主観的な信念。統合失調症・進行麻痺などで特徴的に見られ,その内容があり得ないものであっても経験や他人の説得によっては容易に訂正されない) 〜にふける/沉浸在妄想中	もうそう	
\\	ブーム	【名词】 【英】
\\	流行,暂时的景气,突然繁荣,过分繁荣(一時に、非常な勢いで流行ること)		
\\	階下	【名词】 
\\	楼下(2階以上の建物で、1階または下の方の階) 
\\	楼梯底下,楼梯下(階段の下。きざはしのもと)	かいか	
\\	開会式	【名词】 
\\	开幕式(開会にあたって行う儀式) 
\\	国会,天皇参加的议会的开始仪式(国会が、天皇の臨席を得て会期のはじめに行う式典)	かいかいしき	
\\	開催	【名・他动词・サ变/三类】 开(会);召开;举办(会や催しものなどを開くこと)	かいさい	
\\	介する	【他动词・サ变/三类】 
\\	通过.....,介于.....之间(間にたてる) 友人を~して知りあう/经朋友介绍而认识 彼を~して希望を申し込んだ/通过他提出了希望 
\\	介意(心にかける) 意に~/介意	かいする	
\\	階層	【名词】 
\\	阶层。社会地位相当的人组成的集团。依据职业、学历、财产、年龄、身份、人种等各种标准或综合标准构成(社会的地位が大体等しい人々の集団。職業・学歴・財産・年齢・身分・人種などさまざまな基準やその重なったもので作られる) 
\\	层(建物の層の重なり) 
\\	分层,分级。数据等构成上下层的重叠(データなどの上下に層を成した重なり)	かいそう	
\\	外泊	【名词・自サ】 在外过夜,夜不归宿,外宿(いつも寝る宿所、特に自分の家以外の場所に泊まること) 【日本地名】 外泊(愛媛県南宇和郡愛南町にある大字)	がいはく そとどまり	
\\	掲げる	【他动词・一段/二类】 
\\	悬挂,高举(人目につくように高く上げる) 看板を~/挂招牌 
\\	刊登,登载(新聞・書物などに載せる) 
\\	张贴(標語や張り紙などを貼る) 
\\	提出(主义、方针等)(主義・方針などを示す) 
\\	拨亮(灯心)(灯火をかきたてて明るくする) 
\\	往上提(布状の垂れ下がったものの下端を上げる)	かかげる	
\\	関わる	【自动词・五段/一类】 
\\	拘泥(拘る。泥む) 
\\	纠缠到,有瓜葛(付き纏う) 
\\	关系到,涉及到(関係する。及ぶ)	かかわる	
\\	格言	【名词】 格言(短い言葉で,人生の真理や処世術などを述べ,教えや戒めとした言葉)	かくげん	
\\	獲得	【名・他动词・サ变/三类】 获得。取得。争取(手に入れること。得ること)	かくとく	
\\	確立	【名・自他・サ变/三类】 确立,确定(制度・組織・計画・思想などを、しっかりと打ち立てること。また、しっかりと定めること)	かくりつ	
\\	閣僚	【名词】 阁僚,阁员,部长(内閣を組織する各大臣)	かくりょう	
\\	梶棒	【名词】 车把,车辕,舵柄(人力車・荷車などを引くための長い柄。かじ)	かじぼう	
\\	数々	【形容动词/ナ形容词・名词・副】 
\\	很多(数が多いこと。多数) ~の作品/很多的作品 
\\	种种;许多。形容数量之多(数え上げる数の多いこと。種類が多いことを表わす) ~の贈り物/各种各样的赠品	かずかず	
\\	過疎	【名词】 (人口)过稀,过少(その地域の人口が他に流出して異常に少ない様子)	かそ	
\\	語り草	【名词】 话题,谈话的资料(いつまでも人の口に上るような話題。話のたね。かたらいぐさ) ~にする/作为话题	かたりぐさ	
\\	ガチョウ	【名词】 鹅(雁鴨目の家禽)	がちょう	鵞鳥
\\	かっきり	【副词】 
\\	恰好,正好,整(数量・時間などに、端数のないさま。ぴったり。きっかり) ~五千円持っている/正好带着五千日円 
\\	明确地,清楚地,截然(区切りがはっきりしているさま。くっきり。きっぱり) ~と区切る/截然分开		
\\	葛藤	【名词】 
\\	心里矛盾(心の中に、それぞれ違った方向あるいは相反する方向の力があって、その選択に迷う状態) 
\\	纠葛(もつれ)	かっとう	
\\	がてら	【接续】 ~的时候,顺便~;~的同时~;借…之便…((接助)動詞の連用形および体言に付き、ある事柄をしながら、同時に他の 事柄をもする意を表す。动词连用形;体言+~(に) 注:前后两个动作的时间带是一样的,但ついでに前后两个时间带是不一样的) 散歩~買い物をしてきた/散步时顺便买了东西回来 日本の留学生と一緒に遊び~会話を練習する/跟日本留学生一起玩的时候顺便练习会话 遊び~、建設中の別荘を見に行こう/玩的同时,顺便去别墅看看吧		
\\	奏でる	【他动词・一段/二类】 
\\	奏乐,演奏(音楽を奏する。楽器、特に弦楽器を鳴らす) 
\\	手脚舞动(手足を動かして舞う) 
\\	挥鞭(鞭などを振る)	かなでる	
\\	被さる	【自动词・五段/一类】 
\\	盖到…上,把…蒙上(かぶせたようになる。覆いかかる) 
\\	落在肩上(負担がわが身に及ぶ) 責任が~/担负责任	かぶさる	
\\	上座	【名词】 
\\	上座,上席。地位高的人坐的座席,最上首的座位(地位の高い人や客が座る座席。最も上手の座) 
\\	(舞台的)上场处,左方(从观众席看是右侧)(舞台の、客席から見て右の方。上手) 
\\	上座。禅宗中对上位修行僧的敬称,亦为对同伴僧人的敬称(〔仏〕禅宗で、修行僧の上位にある者に対する敬称。また、相手の僧に対する敬称 
\\	かみざ じょうざ	
\\	神棚	【名词】 神龛(家の中で、大神宮などの神符を祭る棚)	かみだな	
\\	過密	【名・形容动词/ナ形容词】 过密,过于集中(人口などがある地域·範囲に集中しすぎていること)	かみつ	
\\	亀	【名词】 龟,乌龟,海龟(カメ目の爬虫類の総称)	かめ	
\\	火薬	【名】 火药(衝撃・摩擦・熱などによって急激な化学変化を起こし、多量の気体と熱とを発生することにより、推進・破壊などの作用を行う化合物、または混合物。ダイナマイト・ペントリット・ヘキソーゲンの類)	かやく	
\\	川端	【名词】 河边,河畔(川ぶち。川べ。川のほとり)	かわばた	
\\	代わる代わる	【副词】 交替,轮流,轮换(複数の人が同じ行為を入れ替わりながら行うさま。入れ替わりに。互いにかわりあって) ~意見を述べた/轮流阐述意见	かわるがわる	
\\	還元	【名・自他・サ变/三类】 
\\	还原,返回本来面目,返回原样,恢复原状 
\\	〈化〉还原	かんげん	
\\	干渉	【名・自动词・サ变/三类】 
\\	干涉,干预(他人の物事に強いて立ち入り、自己の意思に従わせようとすること。立ち入る。口出しする) 
\\	干扰,干涉(波動に特有な現象で、二つ以上の同一種類の波動が同一点に会した時、その点において起こる相互作用。波動が同位相では互いに強め合い、反対の位相では互いに弱め合う)	かんしょう	
\\	癇癖	【名词】 脾气暴躁(怒りっぽい性質) ~が強い/脾气暴躁很强烈	かんぺき	
\\	管理職	【名词】 (公司的)管理人员(管理・監督の任にある職種。また、その任にある人)	かんりしょく	
\\	聞き下手	【形容动词/ナ形容词】 不善于听别人讲话的人	ききべた	
\\	気障	【形容动词/ナ形容词】 
\\	让人觉得不舒服,呕心(相手にいやな感じを与えること。また、そのものやそのさま) 
\\	矫揉做作(気がかりなこと。また、そのものやそのさま) 
\\	装模做样,装腔作势(服装・態度やものの言い方などが気取っていて、いやみなこと(さま))	きざ	
\\	岸辺	【名词】 岸边(岸に沿った所。岸のあたり)	きしべ	
\\	記す	【他动词・五段/一类】 
\\	做记号,加上符号(文字・記号や文章を書きつける) 
\\	记住,铭记(印象などを記憶する) 
\\	预示前兆(徴候をあらわす。前兆を示す)	しるすきす	
\\	傷付ける	【他动词・一段/二类】 
\\	弄伤,损伤(身体・器物などに傷を負わせる) 
\\	败坏,损坏(人の名誉・心情を損なう) 感情を~/伤害感情	きずつける	
\\	既成	【名词】 既成,现有,原有(既に現実の存在となっていること)	きせい	
\\	基点	【名】 基点(距離や時間を計るとき、もとになる点や場所。考えや行動のもととなるところ)	きてん	
\\	希薄	【名・形容动词/ナ形容词】 
\\	稀薄(気体の密度や液体の濃度のうすいこと(さま)) 
\\	缺乏(乏しいこと。欠けること(さま))	きはく	
\\	初	【名・形容动词/ナ形容词】 
\\	纯真,纯朴,纯情。纯洁;幼稚;没经验;处女。不懂男女之情的〔年が若く世間ずれしていないこと。純情なさま) 〜な青年/纯情的青年 
\\	天生,保持自然的样子(生まれたときのままであること。自然のままであること) 【名词】 最初,第一次(最初、初めて) 【接头/接尾】 接在名词前,表示第一次,新的开始(名詞の上に付いて、接頭語的に用いて、初めての、或いは新しいの意を表す) 【名词】 初	うぶ はつ しょ	初心
\\	ギプス	【名词】 【德】
\\	〈医〉石膏(石膏) ~包帯/石膏绷带		
\\	急所	【名词】 
\\	弱点,疼处(弱点のこと) 
\\	要害,要点,关键,事物最重要的部分(物事の最も大事な所。要所。要点) 
\\	要害,致命处,一旦被击中或受伤会危及性命的身体重要部位(からだの中で、生命にかかわる大事な所)	きゅうしょ	
\\	強制	【名・他动词・サ变/三类】 强制,强迫(力によって他人を従わせること、むりじい)	きょうせい	
\\	競技場	【名词】 竞技场(各種のスポーツを行うための施設。スタジアム)	きょうぎじょう	
\\	教訓	【名・他动词・サ变/三类】 教训(教えさとすこと。また、その教え)	きょうくん	
\\	去来	【自动词・サ变/三类】 去和来。也指心中的意念时隐时现。来去。联翩(去ることと来ること。心の中に思いがうかんだり、消えたりすること) 様々な思いが脳裏に~して、なかなかねつけなかった/万千思绪浮现脑海,怎么也不能入睡	きょらい	
\\	キリスト教	【名词】 【葡】
\\	基督教;罗马公教;基督徒;文明人;基督的;基督教的;基督教徒;信徒;基督教精神	キリストきょう	
\\	キリスト	【名词】 【葡】
\\	基督,耶稣。原为被涂油者之意,曾为王赐予的称号。在耶稣时代,成为今世末日出现的救世主的意思。耶稣的敬称,后指耶稣本人(元来、油を塗られたものの意で、王に与えられた称号だった。イエスの時代には、この世の終末に現れる救世主の意味になる。イエスの敬称。のちイエスそのものを指す。クリスト)		
\\	切り開く	【他动词・五段/一类】 
\\	凿开,开垦〔木を切り倒したり山を切り崩したりして,田畑・宅地・道などにする〕 
\\	开辟,开创,开拓,创造〔新しい道をつける〕 
\\	杀开〔敵の囲みから逃げ出す出口をつくる〕	きりひらく	
\\	ぎりぎり	【副词】 
\\	紧紧地缠绕(強く力を入れるさま;きつく巻き締めたりこすり合わせたりするさま) 
\\	嘎吱嘎吱(滑りの悪い堅い物がこすれ合ってきしる連続音;軋る音) 
\\	最大限度,极限(最大または最小の限度で、それ以上に余地のないさま;極限)		
\\	決まりが悪い	【惯用句】 不好意思(に対して面目が立たない。恥ずかしい。ばつが悪い) ~思いをする/觉得很不好意思	きまりがわるい	
\\	逆効果	【名词】 适得其反的效果(期待したのとは反対の結果が出ること)	ぎゃくこうか	
\\	境遇	【名词】 境遇,处境,环境,遭遇(その人の置かれた環境や身辺の諸事情。身の上。境涯)	きょうぐう	
\\	狂言	【名词】 
\\	狂言(日本の伝統芸能の一) 
\\	歌舞伎狂言,歌舞伎剧〔歌舞伎狂言〕 
\\	诡计,骗局(人を欺くために仕組むたくらみ。お芝居) 
\\	戏言,诨语(道理にはずれた言葉や行為。たわごと)	きょうげん	
\\	強靭	【形容动词/ナ形容词】 坚韧,坚强,刚强,结实(強くしなやかで粘りのある・こと(さま)) ~な神経/坚强的神经	きょうじん	
\\	行政機構	【名词】 行政机构,行政组织(行政を担当する機関)	ぎょうせいきこう	同:行政機関
\\	曲折	【自动词・サ变/三类】 
\\	弯曲,曲折(曲がりくねること。折れ曲がること) 
\\	错综复杂(物事がさまざまに入り組んで変化をすること。また、込み入った事情)	きょくせつ	
\\	極端	【名・形容动词/ナ形容词】 
\\	极限(ものの一番はし。極限) 
\\	极端(はなはだしく一方にかたよっていること)	きょくたん	
\\	極致	【名词】 极致,绝顶(力をつくして最終的に到達するところ。この上ないおもむき)	きょくち	
\\	極点	【名词】 
\\	极点,极限(行きついた究極の点。どんづまり。はて) 
\\	(南北)极点(北極点または南極点)	きょくてん	
\\	極まる	【自动词・五段/一类】 
\\	达到极限,极其,极,到了限度(極限に達する。限度に行きつく) 
\\	极其,透顶,至极,再也没有超过的(形容動詞の語幹に付いて、この上なく…である、きわめて…であるなどの意を表す) 
\\	穷尽,尽头,到了终点,不能再前进(終わる。尽きる。先に進めなくなる) 
\\	穷,窘境,困境,困窘,走投无路而一筹莫展((「谷まる」とも書く)行き詰まって困りはてる)	きわまる	窮まる
\\	極める	【他动词・五段/一类】 穷其究竟,达到极限。研究,探明,探求,阐明,穷原竟委(学問や技芸などを深く研究して、基本的なものを体得する。極限にまで行きつく)	きわめる	窮める 究める
\\	銀河	【名】 〈天〉银河,天河(天の川。また、銀河系の別称。天漢。銀漢)	ぎんが	
\\	近世	【名词】 近世,近代(今に近い世。近時。近頃)	きんせい	
\\	勤続	【自动词・サ变/三类】 继续工作(同じ勤務先に連続してつとめること)	きんぞく	
\\	空間	【名词】 
\\	空间(ものがなく、空いているところ) 
\\	空隙,空的地方,空子(上下、四方の広がり)	くうかん	
\\	空白	【名・形容动词/ナ形容词】 
\\	空白(紙面などの何も書いてない所) 
\\	空白,空虚,空空度过(的时间、日子),无所作为(的日子)(実質的な内容のないこと。また、そのさま。ブランク)	くうはく	
\\	古今東西	【名词】 古今中外(昔から今までと、東西四方のあらゆる所。いつでも、どこでも)	ここんとうざい	
\\	口拍子	【名词】 
\\	用嘴打拍子(口で拍子をとること) ~に合わせて踊る/合着用嘴打的拍子跳舞 
\\	越说越上劲(言葉の調子のよさ) ~に乗る/说上劲了	くちびょうし	
\\	寛ぐ	【自动词・五段/一类】 
\\	舒畅,轻松愉快,惬意,轻松地休息(休憩する。緩む。ゆったりとした気分になる。心身が休まる) 
\\	放松,松弛(ゆったりできる余地がある。余裕がある) 音楽でも聴いて~/可以听听音乐放松一下 
\\	不拘礼节,随便(ゆったりする。のびのびする。気楽にする)	くつろぐ	
\\	クライマックス	【名词】 
\\	【英】
\\	最高潮,最高点,最高峰(山場) 
\\	【英】
\\	极顶期,植物群落发展到极点的状态(極相)		
\\	グリーン車	软席
\\	、旧日本国有鉄道(国鉄)または
\\	グループの旅客列車の車両のうち、普通車に比して乗客1人当たりの占有面積が広く、設備が豪華であるなどの理由で別途の料金がかかる特別車両の名称である	グリーンしゃ ぐりーんくるま	
\\	くるり	【副词】 
\\	迅速回转,快速转身(軽く一回転するさま。一巻きするさま) ~と背を向ける/一个转身背对着我 
\\	事物向另一个方向急剧变化(物事が急に違った方向に変わるさま) 方針を~と変更する/一下子改变了方针 
\\	眼睛等圆圆的很可爱的样子(目などが、丸く愛らしいさま) ~とした目/眼珠滴溜溜转 
\\	圆形物体表面的皮剥落或脱落(丸い物の表面がむけたりはげたりするさま) 
\\	把东西团起(物を丸めるさまをいう語)		
\\	くよくよ	【副词・自他・サ变/三类】 想不开,耿耿于怀,闷闷不乐(心を悩ませても仕方のないことに、いつまでもこだわって、あれこれ心配する様子、くやくや) 大したことではないから、~するな/不是什么大问题,所以不要闷闷不乐了		
\\	企て	【名词】 计划,企图,阴谋(計画を立てること。また、その内容。もくろみ。計画) 謀反を~る/图谋不轨;谋反	くわだて	
\\	君子	【名词】 君子(学識・人格ともにすぐれた、りっぱな人。高位・高官の人)	くんし	
\\	経緯	【名词】 
\\	经线和纬线(経糸と緯糸) 
\\	经度和纬度。南北和东西(南北と東西。経度と緯度) 
\\	事情的经过,原委(物事がそうなったわけ) 【名词】 
\\	事情的经过,事情的原委(事件の経過) 
\\	错综复杂的情况(こみいった関係)	けいい いきさつ	
\\	経過	【自动词・サ变/三类】 
\\	时间流逝,过去(時間が過ぎていくこと) 
\\	经过,过程(事柄の時間的な変わりよう、成り行き)	けいか	
\\	景観	【名词】 景观,景色,风景,景致(風景外観)	けいかん	
\\	激励	【名・他动词・サ变/三类】 激励,鼓励,鼓舞,鞭策(はげまして気を引き立たせること、べんたつする)	げきれい	
\\	怪訝	诧异,惊讶,感到奇怪;莫名其妙『成』(そんな事が実際にあるのかというように不思議がる様子) ~そうに見る/惊讶地看着 ~な顔つきをしている/露出莫名其妙的神色 【名词・自サ】 诧异,惊异(「そんな事が有っていいものかと、限り無く疑わしく思う」意の漢語的表現) 梅は~の目を睜った/小梅睁着诧异的双眼	けげん かいが	
\\	下座	【名词】 
\\	平伏礼,跪拜礼(敬礼) 
\\	末座,末席,下位 
\\	〈剧〉舞台(右方)伴奏者(的席位)(歌舞伎劇場で、囃子(はやし)方のいる席。また、囃子方) 【名词】 下座,末座(人々が並んですわるとき、目下の者がすわる座席。末席)	げざ しもざ	
\\	堅実	【名・形容动词/ナ形容词】 牢靠,稳固,稳健,稳重,踏实(手がたくあぶなげのないこと。しっかりしていて、確かなこと。また、そのさま)	けんじつ	
\\	原始林	【名詞】 原始森林	げんしりん	
\\	妍を競う	大勢の美しい女性が人目をひきつける様子「ミスコンテストで美女がけんを競う」	けんをきそう	
\\	理屈	【名词】 
\\	理论;道理;理;理由(物事のすじみち。道理。ことわり) 
\\	歪理;诡辩;借口(こじつけの理由。現実を無視した条理。また、それを言い張ること)	りくつ	理窟
\\	割り込む	【他动词・五段/一类】 
\\	挤进,加塞儿(無理に中に入り込む) 
\\	硬加入,插嘴(人と人との間にむりに押し入る) 
\\	跌破,股票下跌到某价格之下(株式で、ある値段より下がる)	わりこむ	
\\	然程	【副词】 (后接否定)并非如此(それほど。そんなに) ~おもしろくもない/并不那么有趣 酒は~好きではない/我不那么喜欢喝酒	さほど	
\\	団栗	【名词】 
\\	[コルクがしの]栓皮栎;[かしわぎの]青冈栎的果实;[くぬぎの]橡实、橡子 
\\	~の背くらべ:半斤八两『成』两个一样平庸无奇 彼らの実力は~の背くらべだ/他们的实力分不出高低.栓皮栎,青冈栎的果实,橡实,橡子	どんぐり	
\\	占拠	【名・他动词・サ变/三类】 占据,占领(ある場所を自分のものにして,いすわること)	せんきょ	
\\	細やか	【形容动词/ナ形容词】 
\\	微薄,简单(取るに足らないさま) 
\\	细小,小规模(規模が小さい。こぢんまりして目立たないさま) 【形容动词/ナ形容词】 
\\	浓厚,细腻,深厚(心のこもっている) 
\\	细致,入微(小さいさま。微小である) 
\\	意味深长(微妙で奥深い趣のあるさま) 【形容动词/ナ形容词】 纤细;苗条(ほっそりしたさま)	ささやか こまやか ほそやか	
\\	引き摺る	【自他・五段/一类】 
\\	使物体的下侧贴着地面或地板移动,拖;垂地,拖地(ものの下がわの面が、地面やゆかについた状態で、動かしていく。長く垂らして地に触れさせる) 着物の裾を〜/衣服下摆拖着地了 
\\	尽管本人不愿意却硬拉着去,拖,拉,拽;强拉硬拽(無理に引っぱってゆく;地をすって引いてゆく) 
\\	不立即把事做完,而是使其拖延(ものごとをすぐにやってしまわないで、長びかせる) 
\\	无法忘怀,无法舍弃(捨てきれずに今なおもちつづける。いつまでも忘れないでいる) 
\\	表示影响他人行动、行为,给予影响((普通「ひきずられる」の形で用いる)他人の行動や考えを引っ張る。影響を与える)	ひきずる	
\\	功	【名词】 
\\	功,功劳,功绩,功勋(成しとげた仕事。特に成功をおさめた立派な仕事をいう。手柄。いさお。功績) 
\\	功效,成效(ききめ〕 
\\	经验的积累,资历业绩(経験の蓄積。年功) 【名词】 功,勋,功勋。功绩,功劳(国家・民族・社会などに対する功績。手柄。いさおし)	こう いさお	
\\	甲	【名词】 
\\	甲壳(カメ・カニなどの体をおおう堅い殻。甲羅) 
\\	〔手脚的〕外侧面,手背,脚背((手足的表面)手の甲。足の甲 
\\	甲,十干的第一(十干の第一。きのえ) 
\\	甲,第一位(等級・成績などをつけるときに用いて、一番上位を表す) 
\\	甲,甲方(二人以上の人、または二つ以上の物事がある時、その一つの名にかえて用いる) 【名词】 (天干的第一位)甲(十干の第一。木の兄の意) 【名词】 兜鍪,头盔(頭部を守る武具)	こう きのえ かぶと	
\\	行為	【名词】 
\\	行为,行动,举动(おこない) 
\\	不法行為;不法行为	こうい	
\\	後遺症	【名词】 
\\	后遗症。患病初期的急性症状消失后,长期遗留的非进行性功能障碍
\\	疾病の初期の急性症状が消失したあとに長く残る非進行性の機能障害。脳卒中後の手足の麻痺など) 
\\	后遗症。转指事件造成的影响继续残留到以后(転じて,表面的には一段落した事柄の影響が後々まで尾をひくこと)	こういしょう	
\\	光栄	【名词/形动】 光荣;荣誉(人に認められたり,ほめられたり,大役を任せられたりして,名誉に思うこと)	こうえい	
\\	抗議	【名词・自サ】 抗议(相手の発言/決定/行為などに対して反対の意見を申し立てること) ~を申し込む/提出抗议	こうぎ	
\\	ゴーゴリ	
\\	尼古拉·瓦西里耶维奇·果戈里-亚诺夫斯基 俄罗斯作家,生于现在的乌克兰的一个地主家庭。他自幼爱好文学,深受启蒙运动的影响。1831年发表的《狄康卡近乡夜话》使他受到了亚历山大·普希金的赞誉		
\\	恒久	【名词】 恒久,永久,持久(久しくかわらないこと。永久)	こうきゅう	
\\	広大	【名・形容动词/ナ形容词】 广大,广阔,宏大(ひろく大きいこと(さま))	こうだい	
\\	弘法	【名词】 弘法大师(の略) ~にも筆の誤り/智者千虑必有一失『成』 ~筆を選ばず/善书者不择笔	こうぼう	
\\	巧妙	【形容动词/ナ形容词】 巧妙,做事的方法巧妙(物事のやり方などが、優れてたくみな・こと(さま))	こうみょう	
\\	凍り付く	【自动词・五段/一类】 
\\	结冰,冻住,冻结,粘住其他东西(凍って、他の物にくっつく) バケツが地面に~/桶被冻在地上了 
\\	彻底冻硬。上冻结冰。冻结(完全に凍って固くなる。いてつく。凍結する) 
\\	由于紧张和恐惧身体变的僵硬(緊張や恐怖でからだが固くこわばる) 恐怖で身も~た/由于恐惧,身体都僵住了	こおりつく	
\\	湖岸	【名】 湖畔、湖边	こがん	
\\	ごった返す	【自动词・五段/一类】 杂乱无章,乱七八糟,十分拥挤(思うように動きが取れないほど混雑する) 車内が~/车内十分拥挤 大掃除で家の中が~/因为大扫除家里弄得杂乱无章	ごったがえす	
\\	固定	【名・自他・サ变/三类】 
\\	固定;不从一定的状态发生变化(一定の状態から変化しないこと) 
\\	固定;坚持不从一个场所移动(一つの場所から動かないようにすること)	こてい	
\\	事柄	【名词】 事情,事体;事态(ものごと。また、ものごとの性質や内容) それは容易ならぬ~だ/那是非同小可的事情。 見てきた~/亲眼看到的事情	ことがら	
\\	小旗	【名】 小旗	こばた	
\\	湖畔	【名】 湖畔;湖滨(その湖のほとり)	こはん	
\\	小判	【名词】 
\\	小张(纸)(判の小さいもの) 
\\	一两金币,小金币(むかしの貨幣)	こばん	
\\	湖面	【名词】 湖面(湖の表面)	こめん	
\\	小止み	【名词】 (风雨等)暂停。暂息 雪が~なく降っている/雪一刻不停地下着;雪下个不停 雨が~になった/雨暂停了	こやみ	
\\	懇願	【名・他动词・サ变/三类】 恳求,恳请(ねんごろに願うこと。切に願うこと) 助力を~する/恳求援助 …の~を容れる/接受…的恳求 切に~する/殷切恳请	こんがん	
\\	混沌	【名词】 
\\	混沌(入りまじって区別がつかず、はっきりしないさま) ~たる状態にある/处于混沌状态 形勢が~としている/局势混沌不清 
\\	混沌。在开天辟地的神话中,天和地尚未分开,模糊一团的状态(天地創造の神話で、天と地がまだ分かれず、まじり合っている状態。カオス)	こんとん	渾沌
\\	罪悪	【名词】 罪恶(道徳や宗教の教えに背くこと。つみ。とが)	ざいあく	
\\	再審	【名・他サ】 复审,重新审查(二度目の審査、審判をすること) ~を決める/决定重新审查	さいしん	
\\	冴える	【自动词・一段/二类】 
\\	寒冷,寒凉,冷峭,清寒料峭(冷えびえする) 冬の夜はしんしんと~/冬天的夜晚冷峭逼人 
\\	清澈,鲜明(澄みきる) 
\\	清爽,清醒(頭がはっきりする) 
\\	清晰,灵敏,精巧,纯熟(あざやかな) 
\\	挺棒(すばらしい)	さえる	
\\	探し求める	【他动词・一段/二类】 追求,寻觅,搜寻(目的の物を手に入れようと方々をさがす) 真理を~/追求真理	さがしもとめる	捜し求める
\\	差し掛かる	【自动词・五段/一类】 
\\	到达,走到,路过。来到某场所(ある場所に来る。通りかかる) 急坂に~/走到陡坡 
\\	逼近,临近,紧迫(ある時期に入る。その時期・状況になろうとする) 雨期に~/临近雨期 
\\	遮盖,笼罩(上から覆いかぶさる) 庭木が茂って屋根に~/茂盛的庭院树木遮盖着屋顶	さしかかる	
\\	雑踏	【自动词・サ变/三类】 人多拥挤,人山人海,混乱拥挤;熙熙攘攘;喧闹(人々が大勢集まってこみあうこと。人ごみ)	ざっとう	雑沓
\\	裁く	【他动词・五段/一类】 裁判,审判,排解,从中调停;评理(善悪・理非の判断をする) 事件を~/审理案件	さばく	
\\	作用	【自动词・サ变/三类】 作用,起作用(働きを及ぼすこと)	さよう	
\\	ざるを得ない	接在动词未然形之后,表示“不得不……”((動詞の未然形に続けて)「…しなければならない」意を表す。…せねばならぬ) 途中バスが故障したので、戻ら~た/半路上车坏了,不得不又回来了 あなたの言葉なら、信用せ~ね/你说的话,我不得不相信呀!	ざるをえない	
\\	惨事	【名词】 悲惨事件,惨案(むごたらしい出来事)	さんじ	
\\	サンテグジュペリ	
\\	安托万·德圣埃克絮佩里(他以于1943年出版的童话《小王子》而闻名于世)(フランスの作家、操縦士。代表作『夜間飛行』『人間の土地』『星の王子さま』)		
\\	桟橋	【名词】 
\\	(为了上下高处而架设的带斜坡的)跳板(高所に登るために組み立てた勾配つきの作業用の足場) 
\\	码头(船舶をつなぎ、貨物の積み卸しまたは船客の昇降のため、水面から突出させた構造物で、脚柱で水底に支持する形式のもの)	さんばし	
\\	国民生产总值
\\	の略称で、「国民総生産」とも呼ばれています)	ジーエヌピー	
\\	仕返し	【自动词・サ变/三类】 
\\	报复,回击;报仇,复仇(報復すること。復讐) 
\\	改做,重做(やりなおし)	しかえし	
\\	視線	【名词】 视线。眼睛和所视对象之间的。(假想)连线,注视的方向(目の向き。目で見ている方向。目の中心と、見ている対象とを結ぶ線。視軸。他人を、また、他人が見る目付き。ある気持ちの表れた目付き)	しせん	
\\	自体	【名词・副词】 
\\	自己身体(自分の体) 
\\	本身,自身(それ自身) 
\\	原来;究竟(いったい)	じたい	
\\	仕立てる	【他动词・一段/二类】 
\\	缝制,制作,尤指缝制衣服(作り上げる。特に、衣服を縫い上げる) 背広を~/做西装 
\\	培养,造就;掌握技术等(養成する。技術などを身につけさせる) 一人前の人間に~/培养成人 大工に~/培养成木匠 
\\	特别准备(特別に用意する) 
\\	乔装(もともとは違うものをそれらしく見えるようにする)	したてる	
\\	実業家	【名词】 实业家。企业家(商業や工業など、生産・経済に関係した事業を営む人)	じつぎょうか	
\\	じっくり	【副词】 慢慢地;仔细地;不慌不忙;踏踏实实(落ち着いて、ゆっくり物事をする様) 慌てないで、~やればなんとかなる/不要慌,慢慢来总会有办法 ~考える/踏实地思考 あなたと~お話したい/想和你好好谈一谈		
\\	実質	【名】 实质,本质。事物的内容,本质(物事の中身・本質)	じっしつ	
\\	絞り出す	【他动词・五段/一类】 拧出,挤出,榨出(しぼって中の液体などを外に出す。比喩的に、考えや声などを苦心して出す) 汁を~/榨出汁液 声を~/勉强〔费力〕发出声音 海綿から水を~/从海绵中挤出水来	しぼりだす	搾り出す
\\	紙面	【名】 
\\	纸面,篇幅(紙の表面) 
\\	报纸上(手紙。書面。新聞)	しめん	
\\	地元	【名词】 
\\	当地。与某事直接相关之地(その事に直接関係のある地方(土地)) 
\\	本地。自己居住之地(自分の住んでいる土地)	じもと	
\\	社交	【名】 社交,交际(その社会で生きてゆく上に必要な、人との付き合い)	しゃこう	
\\	謝罪	【名・自他・サ变/三类】 谢罪,赔礼,道歉(罪や過ちを詫びること)	しゃざい	
\\	車夫	【名词】 车夫(人力車を引く人。車引き) お抱えの~/自家雇佣的车夫	しゃふ	
\\	秋分	【名词】 秋分。秋分日。二十四节气之一(二十四節気の一。八月中気。太陽の黄経が一八〇度に達した時をいい、毎年9月23日頃。すなわち秋の彼岸の中日。太陽は天の赤道上にあり、ほぼ真東から昇ってほぼ真西に沈む。昼夜はほぼ同時間)	しゅうぶん	
\\	宿願	【名词】 
\\	夙愿,宿愿(年来の願い。宿望) ~を果たす/偿宿愿 
\\	〈佛〉前世的誓愿(前世におこした誓願)	しゅくがん	
\\	塾	【名词】 塾,私塾;补习班,培训班(児童・生徒に学問・技術を教えるための私設の教育機関)	じゅく	
\\	種目	【名词】 项目,小项(種類によって分けた項目。種類の名)	しゅもく	
\\	順々に	【副词】 
\\	按顺序,依次,顺次(先を争わずに(順序正しく計画的に)次つぎと何かをすることを表わす) ~問題を解いていく/一次解决问题 
\\	逐渐(順を追って少しずつ行われることを表わす)	じゅんじゅんに	
\\	春分	【名词】 春分(二十四節気の一)	しゅんぶん	
\\	生涯	【名词】 
\\	生涯;一生,终生,毕生,一辈子,终身(人が生きている間) 
\\	一生中的某一阶段(一生の一時期)	しょうがい	
\\	招じ入れる	【他动词・一段/二类】 请入,请进(人を家の中や室内に招き入れる)	しょうじいれる	請じ入れる
\\	上昇	【自动词・サ变/三类】 上升;上涨(のぼること。あがること)	じょうしょう	
\\	象徴	【名・他动词・サ变/三类】 象征(本来かかわりのない二つの物を何らかの類似性をもとに関連付ける作用)	しょうちょう	
\\	縄文	【名】 绳文(撚りひもを、押しつけたり転がしたりしてつけた土器の文様)	じょうもん	
\\	消費財	【名词】 消费品,生活资料(経済財のうち、人の欲望を直接に満たす財)	しょうひざい	
\\	食物繊維	食物纤维,有调整肠胃的作用	しょくもつせんい	
\\	所得	【名词】 所得,收入,收益,所得,所有(物)(得をすること。もうけること)	しょとく	
\\	所要	【名词】 所要,所需,需要,必要(的事物)(必要なこと)	しょよう	
\\	白樺	【名词】 〈植〉白桦(シラカバの別称。カバノキ科の落葉高木)	しらかば	
\\	神器	【名词】 神传的宝器,(特指象征皇位的)三种神器(「三種の神器」の略)	じんぎ	
\\	真実	【名・副・形容动词/ナ形容词】 
\\	真实,真话,实际,事实(うそいつわりのないこと。ほんとうのこと) 
\\	实在地,真的,真心(ほんとうに。こころから)	しんじつ	
\\	信者	【名词】 信徒,追随者,崇拜者(その宗教を信仰する人。信徒。また、ある人物・主義などの信奉者の意にも使う)	しんじゃ	
\\	真珠	【名词】 珍珠(貝類の殻の中にできる光沢のある玉) 豚に~/对牛弹琴	しんじゅ	
\\	神道	【名】 〈宗〉(以崇拜皇室祖先为中心的日本民族固有的)神道(日本に発生した民族信仰。祖先神や自然神への尊崇を中心とする古来の民間信仰が、外来思想である仏教・儒教などの影響を受けつつ理論化されたもの。平安時代には神仏習合・本地垂迹があらわれ、両部神道・山王神道が成立、中世には伊勢神道・吉田神道、江戸時代には垂加神道・吉川神道などが流行した。明治以降は神社神道と教派神道(神道十三派)とに分かれ、前者は太平洋戦争終了まで政府の大きな保護を受けた。かんながらの道)	しんとう	
\\	身辺	【名词】 身边(身のまわり)	しんぺん	
\\	辛抱	【名・自动词・サ变/三类】 
\\	耐心工作,耐(がまんしてつとめる) 
\\	忍,忍耐,忍受(がまん)	しんぼう	
\\	法案	【名词】 法案,法律草案(法律の案文として、条文の形式に整えられた文書。特に、法律として制定されることを求めて、国会に提出されるもの。法案は、国会で可決されて法律となる。法律案)	ほうあん	
\\	針葉樹	【名词】 〈植〉针叶树(葉が針のように細長い種類の木)	しんようじゅ	
\\	人力車	【名词】 人力车,黄包车(後ろの座席に人を乗せ、梶棒を両手で持ち、人の力で引く二輪車) ~を引く/拉人力车;拉黄包车	じんりきしゃ	
\\	透かす	【他动词・五段/一类】 
\\	留出空隙,留出间隔(すき間をこしらえる 
\\	使稀疏,间伐(たてこんでいるものの一部を取り去り,まばらにする) 
\\	透过亮光,迎着光亮(观看)(光を通して中の物や向う側を見る) 
\\	透过…(看)(すき間を通して物が見えるようにする) 
\\	疏忽大意(油断をする) 
\\	放无声屁(音を立てないで屁をする)	すかす	
\\	ずかずか	【副词】 无礼貌地,鲁莽地,冒冒失失地,冒昧地(遠慮なく乱暴に入ったり近寄ったりするさま) ~上がり込む/冒冒失失地闯进来		
\\	過ぎ行く	【自动词・五段/一类】 
\\	走过去(通ってゆく。また、通り越してゆく) 
\\	过去(時が移ってゆく) 時は~/时光飞逝	すぎゆく	
\\	砂浜	【名词】 海滨沙滩(砂の堆積した海岸) ~に寝ころんで休む/躺在海边沙滩上休息	すなはま	
\\	ツアー	【名词】 
\\	【英】
\\	旅行,旅游,游览(周遊旅行。団体旅行。小旅行) 
\\	【英】
\\	(歌手等的)巡回演出(劇団や楽団、スポーツチームなどの地方巡行)		
\\	スマッシュ	【名词】 【英】
\\	〈网球・排球・乒乓球〉(猛烈)扣杀,猛扣,扣球(相手のコートに、たたきつけるように激しくボールを打ちおろすこと。打込み)		
\\	ずるずる	【形容动词/ナ形容词】 
\\	松松垮垮的样子(物が引きずられたり、滑り落ちたりするさま) 
\\	拖延不决(しまりのないさま) 
\\	一点点地滑动状(しまりのないさま) 
\\	拖拉着(物が引きずられたり、滑り落ちたりするさま) 
\\	抽吸液体状物的声音(汁などを音を立てて吸い込むさま) 【副词】 
\\	抽吸液状物声音(汁などを音を立てて吸い込むさま) 
\\	拖延不决(しまりのないさま) 
\\	一点点地滑动状(しまりのないさま) 
\\	拖拉着(物が引きずられたり、滑り落ちたりするさま)		
\\	整然	【形容动词/ナ形容词】 有条不紊,井井有条,整齐(きちんとして、正しくととのったさま) ~たる秩序/秩序井然	せいぜん	
\\	静止	【名・自动词・サ变/三类】 静止(動きを止めた状態を保ち、前と同じ位置に在ること)	せいし	
\\	正反対	正相反,完全相反(全く反対である・こと(さま)。あべこべ) 目的と~の方向に進む/朝着和目的完全相反的方向前进 彼女の性格はわたしと~だ/她的性格和我正相反	せいはんたい	
\\	比べ	【接头/接尾】 比,比较;比赛,竞赛(くらべること。競い合うこと。競走。多く複合語として用いる) 高さ~/比高矮 背~/比身高;比个子	くらべ	
\\	折衷	【名・他动词・サ变/三类】 折中,折衷(あれこれと取捨して適当なところをとること)	せっちゅう	
\\	節分	【名词】 
\\	立春、立夏、立秋、立冬的前一天(季節の移り変わる時、すなわち立春・立夏・立秋・立冬の前日の称) 
\\	立春的前一天(特に立春の前日の称。この日の夕暮、柊ひいらぎの枝に鰯いわしの頭を刺したものを戸口に立て、鬼打豆と称して炒いった大豆をまく習慣がある) ~祭/节分祭。立春前夜及当天在神社和寺院举行的祭奠仪式	せつぶん	
\\	摂理	【名词】 天理,天命(最終的に人を善へ導く、神の意志) 神の~/天理	せつり	
\\	宣言	【名・他动词・サ变/三类】 宣言,宣布,宣告(広く述べること)	せんげん	
\\	選手村	【名词】 奥运村(オリンピックなどの大きなスポーツ競技会で,出場する選手などのための宿泊施設を設けた地域)	せんしゅむら	
\\	全通	【名词・自动・三类】 全线通车,全线畅通(路線の全部が開通すること) 工事も終わり、上下線とも~した/工程已完,上下行车线均已全线通车	ぜんつう	
\\	前半	【名词】 前半,前一半;上(一)半(前後二つに分けたうちの前の半分) ~の戦いは優勢のうちに終わった/上半场比赛以优势结束 ~だけ読んだ/只看前半部分〔前一半〕	ぜんはん ぜんぱん	
\\	船尾	【名词】 船尾(船の後端部。とも)	せんび	
\\	先方	【名词】 
\\	对方(相手の人、相手方) 
\\	那里,那方,对面(先のほう、むこう) 【名词】 刚才,方才,不久以前那人(取引や交渉などの相手。せんぽう) お客さまは~お帰りになりました/客人方才走了	せんぽう さきかた	
\\	千里	【名词】 千里,一千里,远处(非常に遠い所のたとえ)	せんり	
\\	丘陵	【名词】 丘陵(なだらかな起伏や、小山(丘)の続く地形のことである)	きゅうりょう	
\\	造花	【名词】 人造花,假花,纸花,绢花,塑料花(紙・布などを使って、生花に似せてつくった花。つくりばな) ~を作る/做花	ぞうか	
\\	造形	【名・自サ】 造型(形のあるものを作り上げること) 都市空間の~/城市空间的造型	ぞうけい	造型
\\	俗世間	【名词】 俗世,世上(もと、出家しない者たちで構成される世の中の意。醜い事や情実の方が、まっとうな人や物を圧倒していることが多いものと考えられる実社会。俗世)	ぞくせけん	
\\	染まる	【自动词・五段/一类】 
\\	染上。着色(色が付く。染色される。染む) 手が黒く~/手染黑了 血に~/被血染 
\\	受影响,沾染(よくないものの影響を受けて、それになじむ。染む)	そまる	
\\	それでは	【接续词】 
\\	那么,那么说,那就(では) ~、これで失礼します/那么我就失陪了 
\\	如果是那样,要是那样的话(それならば) ~、私が困ります/要是那样的话,我就不好办了 【连体词】 假使那样的话,既然是那样(それなら) ~、あまりにも可哀想です/假使那样的话就太可怜了		
\\	損気	【名词】 损失,吃亏(損をする気質) 短気は~/性急好吃亏;心急吃不了热豆腐	そんき	
\\	大規模	大规模,规模宏大(仕組み・構想が大きい) ~な建築物/规模宏大的建筑	だいきぼ	
\\	耐久	【自他・サ变/三类】 耐久;持久;持续(長い間もちこたえること。同:ながもち)	たいきゅう	
\\	形式	【名词】 
\\	形式(外形) 
\\	方式,手续(方法)	けいしき	
\\	大車輪	【名】 
\\	〔大きな車輪〕大的车轮 
\\	〈体操〉大回环 
\\	〔いっしょうけんめい〕拼命,努力	だいしゃりん	
\\	耐性	【名词】 抗性,(细菌的)抗药性,耐受性(環境条件の変化に耐えうる生物の性質。薬物の反復使用によって薬効が低下する現象)	たいせい	
\\	大雪山	【名词】 大雪山。日本北海道中部的火山群的总称(北海道中央部にある火山群)	だいせつざん	
\\	変動	【名词・自动・三类】 变动,波动,改变,变化(物事が変わり動くこと)	へんどう	
\\	怠慢	【名・形容动词/ナ形容词】 怠慢。懈怠,玩忽(职务),懒散而不做该做的事(なまけおこたること。なまけてするべき事をしないこと。また、そのさま)	たいまん	
\\	鷹	【名词】 鹰(ワシタカ目ワシタカ科に属する鳥のうち中・小形のものの総称)	たか	
\\	高々	【副词】 
\\	高高地(きわだって高いさま) 旗を~と掲げる/旗子高高挂起 
\\	高声地;大声地(声や音が大きく響き渡るさま) サイレンが~と鳴る/警笛高鸣 
\\	至多(不过);充其量;顶多(最高に見積もっても大したことではないさま。せいぜい) ~一万円が関の山だ/顶多1万日元	たかだか	
\\	宝物	【名词】 宝物,宝贝,财宝	たからもの ほうもつ	
\\	託する	【他动词・サ变/三类】 
\\	托,托付(たのむ。あずける。ことづける) 
\\	寄托(気持ちや意見などを他の物にことよせて表す) 思いのたけを歌に~/将思念倾注在歌声中	たくする	托する
\\	多神教	【名词】 多神教。崇拜多神的宗教,常见于原始社会后期和古代社会(多数の神々を崇拝する宗教。未開社会・古代社会で広く見られる)	たしんきょう	
\\	襷	【名词】 
\\	束衣袖的带子(和服の袖やたもとがじゃまにならないようにたくし上げるためのひも。背中で斜め十文字に交差させ両肩にまわして結ぶ) 
\\	斜挂在肩上的窄布条(一方の肩から他方の腰のあたりに斜めにかける、輪にした細長いひも)	たすき	
\\	絶つ	【他动词・五段/一类】 
\\	剪断,割断,斩断(ひも状のものを切る。切断する。比喩的にも用いる) 
\\	截断(必要な通路などを途中で通れないようにする。遮断する) 
\\	切断,阻断(资金等) 
\\	断绝,解除关系(関係を解消する) 
\\	绝望(執着・欲望などをなくする) 
\\	戒断(ある飲食物をとることやある行為をやめる) 【自动词・五段/一类】 失去联络(連絡がとれなくなることをいう) 消息/連絡を~	たつ	
\\	脱出	【名・自动词・サ变/三类】 逃出,逃脱,逃亡(抜け出ること。抜け出すこと)	だっしゅつ	
\\	達成	【名・他动词・サ变/三类】 达成,告成(成し遂げること。目的を果たすこと)	たっせい	
\\	縦割り	【名词・他サ】 
\\	竖着切(劈)开(たてに切りさくこと) ~にする/竖着劈开 
\\	直线领导,条条领导(組織が上下関係を中心に運営されること) ~行政/条条领导主义的行政方式	たてわり	
\\	他方	【名词・副】 
\\	他方,另一方向,另一方面(他の方面。他の方向) ~の言い分もきく/也听取其它方面的意见 
\\	另一方面,从另一方面(一方では。別の面から見ると)	たほう	
\\	団子	【名词】 
\\	米粉团,江米团(穀類の粉をこねて丸め、蒸したり、茹でたりした食品) 
\\	丸子(丸く固めたもの)	だんご	
\\	近場	【名】 比较近的地方,附近	ちかば	
\\	中継	【名・他动词・サ变/三类】 
\\	中继(途中でうけつぐこと) 
\\	转播(「~放送」の略)	ちゅうけい	
\\	忠誠	【名】 忠诚(まごころ。また、まごころをもって尽くすこと)	ちゅうせい	
\\	中日	【名】 
\\	〔中国と日本〕中国和日本;中日 
\\	〔彼岸の中日〕春分;秋分 【名词】 正当中的一天(一定期間のまんなかにあたる日)	ちゅうにち なかび	
\\	調和	【名・自动词・サ变/三类】 调和,颜色配合,声音和谐,关系协调(うまくつりあい、全体が整っていること)	ちょうわ	
\\	賃金	【名词】 工资,薪金,作为劳动报酬支付的金钱(労働者が労働の対価として受け取る報酬。貨幣で表示された労働力の価値。給料。労賃)	ちんぎん	
\\	突き出る	【自动词・一段/二类】 
\\	扎出来
\\	突き破って出る) 釘が板から~/钉子刺出了板 
\\	突出,伸出(ある部分が外側や前方に出っ張る) 桟橋が海に~/栈桥伸到了海面	つきでる	
\\	付き纏う	【自动词・五段/一类】 
\\	伴随,缠住。总是跟随在身边不离开(いつもそばについて離れないでいる) 小犬が~/小狗伴随在身边 
\\	纠缠,缠住。某件事自己始终摆脱不了(好ましくない事がいつも離れずについている) 不安が脳裏に~/不安萦绕心头	つきまとう	
\\	包み込む	【他动词・五段/一类】 包进去,包在里面(つつみこむ。また、ふくみこむ) 本をふろしきに~/把书放进包里	つつみこむ	
\\	募る	【自动词・五段/一类】 激化;越来越严重(だんだん勢いが激しくなる。強さが増す) 【他动词・五段/一类】 招;招募;募集(広く呼びかけて集める) 寄付を~/募捐	つのる	
\\	適応	【自动词・サ变/三类】 
\\	适应,顺应(その状況によくかなうこと) 
\\	生物的诸性状适应在生活环境下的生存和繁殖(生物の形態・習性などの形質が、その環境で生活・繁殖するのに適合していること、あるいはそう判断できること)	てきおう	
\\	手付き	【名词】 手的姿势,手的动作(手のかっこう。手のようす。また、手の動かし方)	てつき	
\\	天	【名词】 
\\	天,天空(空) 
\\	天国,天堂(天国) 
\\	天道,天理(摂理) 
\\	天命,运命(天命) 
\\	天,苍天,上帝(神) 
\\	(字画、货物等的)上部;(书的)天头(上部)	てん	
\\	展示場	【名】 展览会,展览现场	てんじじょう	
\\	伝説	【名词】 传说,口传(うわさ。風説)	でんせつ	
\\	天命	【名词】 天命,命运,天年(天の命令。天が人間に与えた使命。人の力で変えることのできない運命。宿命。天の定めた寿命。天寿。 天の与える罰。天罰) 人事を尽くして~を待つ/尽人事听天命	てんめい	
\\	投影	【名・他动词・サ变/三类】 
\\	投影。投射物体的影子,亦指映出的影子(影を映すこと。また、映った影) 
\\	投影;(比喻性的)使某事物反映在另一事物上(比喩的にある物事を他に反映させて表し出すこと) 
\\	[数]投影。向平面图形或立体照射平行光线,在平面上映出其影像(平面図形あるいは立体に平行光線を当てて、平面上にその影を映したもの)	とうえい	
\\	同音	【名词】 
\\	同音,字音相同(同じ発音。同じ読み) 
\\	同样音高(同じ高さの音、声) 
\\	同声发言(声をそろえて言うこと。同時に言うこと)	どうおん	
\\	銅貨	【名词】 铜币、铜钱(銅を主材料とした貨幣)	どうか	
\\	東海道	【名词】 东海道(五畿七道の一。伊賀・伊勢・志摩・尾張・三河・遠江(とおとうみ)・駿河・甲斐・伊豆・相模・武蔵・安房・上総・下総・常陸の15か国)	とうかいどう	
\\	倒産	【名・自动・三类】 倒闭,破产(経営が破綻して会社がつぶれること)	とうさん	
\\	闘病	【自动词・サ变/三类】 和疾病作斗争(なおそうという強い意志をもって積極的に療養すること)	とうびょう	
\\	解きほぐす	【他动词・五段/一类】 
\\	解开,松开(もつれたり、固く結ばれたりしているものを少しずつ解いて、ばらばらにする) 
\\	解,解开,解决(複雑にからんだ問題を少しずつほどくようにして解明する) 事件の謎を~/解开事件之谜	ときほぐす	
\\	徳川家康	【名词】 德川家康(徳川初代将軍(在職1603-1605)。松平(まつだいら)広忠の長子)	とくがわいえやす	
\\	閉じ籠る	【自动词・一段/二类】 闷在家〔房间〕里(戸を閉じて中にこもっている。籠居する)	とじこもる	閉じこもる
\\	途上	【名词】 
\\	道上,路上,中途
\\	目的地に行く途中) 
\\	事情正在进行中(事業、計画などが目的に従って進行している途中)	とじょう	
\\	どっこい	【感叹词】 
\\	用力时的吆喝(力を入れて何かをするときのかけ声。どっこいしょ。) うんとこ~/嘿哟 
\\	且慢。用来阻止对方
\\	相手の意図をさえぎり止めるときに発する語) ~、それはいかないよ/慢着!那可不行		
\\	突拍子もない	【惯用句】 离奇,异常(とんでもなく調子はずれである。突飛である) ~ことを考える人だ/一个异想天开的人	とっぴょうしもない	
\\	届出	【名词】 呈报,呈报物(役所・学校・会社または上司などに、所定の手続きに従って申し出ること 
\\	とどけで	
\\	怒濤	【名词】 怒涛(荒れ狂う大波。はげしく打ち寄せる波)	どとう	怒涛
\\	跳び上がる	【自动词・五段/一类】 
\\	飞起;飞向天空(飛んで空へ上がる。空中へ舞いあがる) 
\\	跳上去(跳ねて上にあがる。跳ね上がる) 岩の上に~/跳到岩石的上面 
\\	(因吃惊或高兴而)跳起来(驚きや喜びを禁じ得ず、思わずおどり上がる) 
\\	不按顺序地晋级(順番を飛びこして進む)	とびあがる	飛び上がる
\\	取扱い	【名词】 
\\	对待;接待,待遇(世話をすること。面倒をみること。よそから来た人に接して、用が足りるように取り計らう。待遇する。応対する) 
\\	使用;操作(物を手で持って動かしたり使ったりすること。使用。操作。手で持って使ったり、動かしたりする) 
\\	处理,办理(物事を処理すること。扱うこと。とりはからうこと。とりさばくこと。受付けて、処理する)	とりあつかい	取り扱い
\\	取り組む	【自动词・五段/一类】 
\\	着手,对付,解决,应对(真剣に事をする) 
\\	较量,与…为对手,同…比赛(試合をするために、ある相手と組み合わせられる) 
\\	全力处理,埋头苦干,下大气力去做(解決・処理するため、一生懸命に取り掛かる) 
\\	(相扑)扭住(对方)((相撲などで)組み付き合う。互いに組み合う。相手となって争う)	とりくむ	
\\	取り込む	【自动词・五段/一类】 (婚丧节庆或发生意外时的)忙乱(家内に冠婚葬祭など不時のことがあり、ごたごたする) 【他动词・五段/一类】 
\\	收进;拿入(取り入れる) 
\\	拉拢;笼络(人の機嫌をとって丸め込む) 
\\	诈骗;骗取金钱(不正な方法で自分の手に収める)	とりこむ	
\\	取り散らかす	【他动词・五段/一类】 弄得乱七八糟	とりちらかす	
\\	取り残す	【他动词・五段/一类】 
\\	剩下,留下〔そのまま留める〕 きょう~た草はあすまた取ろう/今天拔剩下的草明天再拔吧 
\\	落到后面,落伍,掉队〔置き去りにする〕 時代に~される/落后于时代	とりのこす	
\\	取り分け	【副词】 特别,尤其,格外(同じようなものの中でも、その程度が際立ってはなはだしいさま) 彼女は今日は~しょげていた/她今天特别的没精神 彼女は踊っている時の姿が~美しい/她舞蹈时的姿态特别美 今夏は~暑い/今年夏天特别热	とりわけ	
\\	とろりと	【副词】 
\\	打盹,打瞌睡(眠る) ~するうちに夢を見た/刚一打盹儿做了个梦 
\\	粘糊,稠糊,浓糊(ものが溶けてやわらかくなるさま。また、液体にやや粘り気があって、滑らかなさま)		
\\	内閣	【名词】 内阁,政府(大臣で組織する、国の最高行政機関)	ないかく	
\\	投げ掛ける	【他动词・一段/二类】 
\\	投,扔过去,投到 
\\	提出(相手や関係のある方面に問題などを提起する) 
\\	偎靠(寄りかかるようにする) 身を~/依偎着	なげかける	
\\	成し遂げる	【他动词・一段/二类】 完成(物事を最後までやりとげる) 目的を~/达到目的	なしとげる	為し遂げる
\\	馴染む	【自动词・五段/一类】 
\\	熟识,亲密(人になれて親しくなる。また、物事や場所になれて親しみをもつ) 新しい環境にすぐ~/很快适应新的环境 
\\	融合,(味道)调和
\\	味わいや調子などが一つに溶けあう。ほどよく調和する)	なじむ	
\\	何食わぬ顔	【惯用句】 佯装不知,若无其事(何も知らない、自分には関係ないという顔つき、またはそのように振る舞うようす。そしらぬ顔) 人を裏切りながら~でつき合う/背叛了别人,还装作若无其事地交往	なにくわぬかお	
\\	何もかも	【名词・惯用句】 完全,什么都,一切(どれもこれも。すべて) ~失ってしまった/什么都失去了	なにもかも	
\\	波打ち際	【名】 岸边;水边;海滩(同なぎさ)	なみうちぎわ	
\\	成程	【副词】 诚然,的确;果然;怪不得(なるたけ。できるだけ。まことに。いかにも。言ったとおり。どおりで) 【感叹词】 诚然;的确;可不是;表示同意,赞同之意,也可用于表示附和(合点がいった時、または相手の話に相づちを打つ時に発する話)	なるほど	成る程
\\	憎まれ口	【名词】 招人厌恶的话,讨厌的话(人に憎まれるようなことを言うこと。また、その言葉) ~をたたく/净说令人讨厌的话	にくまれぐち	
\\	拭う	【他动词・五段/一类】 
\\	擦掉,拭去,揩掉(ふいてきれいにする。ふき取る。ふく) 汗を~/擦汗 
\\	消除,洗刷(恥・汚点・印象などを消し去る。取り去る) 汚名を~/洗掉坏名声	ぬぐう	
\\	盗人	【名词】 贼,偷窃他人东西的人(他人の物を盗む人。どろぼう。盗賊。同:ぬすっと、ぬすと、とうじん) ~に追い銭/赔了夫人又折兵。被小偷偷了东西又给小偷前,比喻损失重重	ぬすびと	
\\	値上がり	【自动词・サ变/三类】 价格上涨,涨价(物の値段や料金が高くなること) ~する/价格上涨 食糧の~は国民の生活に直接ひびく/粮价上涨直接影响国民的生活	ねあがり	
\\	寝静まる	【自动词・五段/一类】 入睡后夜深人静(夜が更け、人々が皆寝てしまってあたりがしんと静かになる) 人々は~ていた/人们都沉睡了	ねしずまる	
\\	ネットワーク	【名词】 
\\	【英】
\\	广播网;电视网。
\\	网络织网工艺;网状组织;[广电]广播网(テレビ、ラジオで、番組を送り出す局を中心に、中継回線によって結ばれた、全国的な放送局の組織。放送網。ネット。コンビューターネットワークの略) 
\\	网络		
\\	寝床	【名词】 睡铺,被窝(寝る床。また、寝る場所。ねや。ねま。ねどころ) ~にもぐり込む/钻进被窝里 ~を離れる/离开被窝(起床)	ねどこ	
\\	年功序列	【名词】 按工龄,能力,贡献而进级,提薪(勤続年数や年齢によって,職場での地位や賃金が上がること) 若手の意欲を低下させる~のジレンマ/年功序列降低年轻人的积极性,陷入困境 ~崩壊/年功序列崩塌	ねんこうじょれつ	
\\	年俸	【名词】 年俸,年薪(一年単位で支払われる給与。また、一年分の給与の総額)	ねんぽう	
\\	軒下	【名词】 屋檐下(軒の下) ~で雨宿りする/在房檐下避雨	のきした	
\\	把握	【名・他动词・サ变/三类】 
\\	握在手中(手で握ること。しっかりつかむこと) 
\\	掌握。充分理解(よく理解すること)	はあく	
\\	排除	【名・他动词・サ变/三类】 排除(おしのけてそこからなくすこと) 暴力を~する/排除暴力	はいじょ	
\\	倍増	【名词・自サ】 倍增;成倍增长(倍にふえること)	ばいぞう	
\\	バイロン	
\\	イギリスの詩人		
\\	儚い	【形容词/イ形容词】 
\\	靠不住的。虚幻的,渺茫无常的,不可靠的(しっかりしていなくて頼りにならない。不確かであてにならない。実現の可能性が乏しい) ~望み/幻想 
\\	脆弱的,无常的(将来確実にどうなるという目当てがない。 消えてなくなりやすい。もろくて長続きしない) ~命/脆弱的生命	はかない	
\\	馬鹿馬鹿しい	【形容词/イ形容词】 
\\	无聊,毫无价值;荒谬,愚蠢(全くばかげている) ~お話で恐れ入ります/实出唐突 
\\	胡闹;(超出限度)很大的,厉害的(はなはだしい)~値段/天价	ばかばかしい	
\\	計り知れない	【连体词】 (指数量特别大)不可估量的(推し量ることができない。見当がつけられない(ほどはなはだしい)。計り知れぬ)	はかりしれない	
\\	履き替える	【他动词・一段/二类】 换鞋(穿),换穿别的鞋(袜)(履物やズボンなどを別のものに替える) 濡れた靴下を~/换下湿了的袜子	はきかえる	
\\	波及	【名・自动词・サ变/三类】 波及,影响(波紋が広がるように、影響が徐々に広い範囲に及んで行くこと)	はきゅう	
\\	白銀	【名词】 
\\	银,白银,也用来比喻白雪(しろがね。ぎん。また、降り積もった雪のたとえ) 
\\	银币(江戸時代、銀を9センチメートルほどの平たい楕円形に延ばして紙に包んだもの。通用銀の3分に当たり、多くは贈答などに用いた)	はくぎん	
\\	爆発的	【形容动词/ナ形容词】 爆炸性的,惊人的(物事が急激な勢いで起きたり行われたりするさま)	ばくはつてき	
\\	恥	【名词】 耻;耻辱(不名誉な);羞耻(不体裁な);丢人(はじをかく)	はじ	
\\	はしたない	【形容词/イ形容词】 
\\	粗鲁的,下流的(礼儀に外れていて品がない。上品ではない。下品だ。卑しい。情けない) 
\\	半途而废的(どっちつかずのさまである。中途半端である) 
\\	羞耻,不体面,难为情,不凑巧,不走运(決まりが悪い。間が悪い) 
\\	薄情,冷淡
\\	素っ気無い。つれない) 
\\	麻烦、困窘的,不愉快的(迷惑である) 
\\	程度远远超过一般的状态(程度がはなはだしい)		
\\	派出所	【名词】 
\\	办事处,事务所(派出された人が詰めて業務をする所) 
\\	警察派出所(交番の旧称)	はしゅつじょ	
\\	パスカル	【人名】 【英】
\\	帕斯卡(法国哲学,数学,物理学家)		
\\	外れ	【名词】 
\\	未中,不中(当たらないこと) 
\\	尽头(はし) 
\\	未中;不中;尽头;脱落	はずれ	
\\	八分目	【名词】 八分、八成。也指有节制(全体の8割程度。また、内輪にとどめておくこと) 腹~/吃到八分饱	はちぶんめ	
\\	初詣	【名词】 新年第一次参拜神社(新年に初めて社寺に参詣すること。はつまいり。一年の感謝を捧げたり、新年の無事と平安を祈願したりする)	はつもうで	
\\	果て	【名词】 
\\	[境界的]边,边际;尽头(終わること。尽きること) 
\\	最后,末了,结局;下场(物事の結末;行きつく最後の所) なれの~/落魄的下场;末路	はて	
\\	果てる	【自动词・一段/二类】 
\\	…之极(限界まで~する) 
\\	死(死ぬ。命が尽きる) 
\\	终,尽,完(続いていたことが終わる)	はてる	
\\	放つ	【他动词・五段/一类】 
\\	放走,放出(自由に行かせる) 
\\	舍弃,流放(追放する) 罪人を東国に~/把犯人流放到关东地方 
\\	发射,发出(光・矢などを発する) 
\\	表现,放出(あらわす) 
\\	放火(火をつける) 家に火を~/放火烧房 
\\	派遣,派出(刺客などを送り込む)	はなつ	
\\	万国博覧会	【名词】 
\\	万国博览会 
\\	世博会	ばんこくはくらんかい	
\\	煩雑	【名・形容动词/ナ形容词】 繁杂,复杂,麻烦(事柄がこみいっていてまとまりがつかず,わずらわしい・こと(さま)。こみいっていてわずらわしいこと。また、そのさま)	はんざつ	繁雑
\\	反射的	【形容动词/ナ形容词】 反射的(刺激に対して瞬間的に反応するさま「~に飛び起きる」)	はんしゃてき	
\\	万人	【名词】 万人,众人;很多人;大众(すべての人;百人や千人どころではない、ちょっと算えられないくらいたくさんの人)	ばんにん ばんじん まんにん	
\\	半々	【名词】 一半一半,各半(半分ずつ。半分半分。五分五分)	はんはん	
\\	引き込む	【他动词・五段/一类】 
\\	引进来,拉进来(引いて内へ入れる) 水道を~/安自来水(管道) 
\\	拉拢,引诱进来(誘って仲間にする) 
\\	伤风,感冒(本格的に風邪を引く) 
\\	吸引。夺人心魄(人の心・魂を引き寄せる) 
\\	从河流或其他线状物引出一条分支到自己的所在地(川の流れや線状の物を分岐させて自分のほうまで延長する) 
\\	凹进,陷下。瘪进去(へこむ) 
\\	隐退,隐居。隐遁(隠遁する)	ひきこむ	
\\	引き摺り込む	【他动词・五段/一类】 拖入,拽进,硬拉进去;强迫加入自己的集团(無理に中へ入れる) 不良仲間にお~まれる/被拉入流氓团伙	ひきずりこむ	
\\	比重	【名词】 
\\	比重(その物質の質量の、同体積のセ氏四度の水の質量に対する比) 
\\	(所占的)比例,对比(他の物と比べた時の大きさ・重要さなどの割合)	ひじゅう	
\\	卑小	【名词】 卑微,微贱(取るに足りないこと。ちっぽけで価値の低いこと。また、そのさま) ~な存在/卑贱的存在	ひしょう	
\\	歪み	【名词】 
\\	歪斜,翘曲(形がゆがんでいること。いびつ) 
\\	不良影响〔后果〕(比喩的に、ある事の結果として現れた悪い影響。しわよせ) 
\\	形变(物体が外力の作用を受けた時に生ずる形や体積の変化。変形。ゆがみ) 【名词】 
\\	歪斜;歪曲(ゆがんでいること。曲がっていること。ひずみ) 
\\	不爽直;不直爽;乖僻(よこしまなこと。不正)	ひずみ ゆがみ	
\\	潜む	【自动词・五段/一类】 
\\	隐藏,潜藏,潜伏下来(見えないように隠れている) 魚が水底に~/鱼潜藏在水底 
\\	藏在(心里),隐含,潜在(潜在する)	ひそむ	
\\	顰める	【他动词・一段/二类】 皱眉。蹙眉(痛みや不快のために、まゆのあたりにしわを寄せる; [怒りや痛みで]皱眉;[憂いで]颦蹙『書』;厳密に言えば「眉をしかめる」という表現はない。心配や不安があり、表情が暗くなる様子を言いたいなら、「眉をひそめる」というのが正しい) 痛みで顔を~/疼得皱眉 眉を~/皱起眉头;皱眉;双眉颦蹙	ひそめる	
\\	悲嘆	【自动词・サ变/三类】 悲叹(ショックを感じる大変な出来事にあって、身の不幸を大いに悲しむこと) ~にくれる/日夜悲伤 ~の極み/极其悲叹	ひたん	悲歎
\\	引っ込める	【他动词・一段/二类】 
\\	缩入,缩回(もとへもどす〕 手を~/缩回手 亀が頭を~/乌龟缩回头 
\\	撤回(とりさげる〕 要求を~/撤回要求	ひっこめる	
\\	ひっそり	【副词・自动词・サ变/三类】 沉寂,寂静,静悄悄
\\	動くものがなく物音のしないさま。静かなさま) 町は~としている/街上鸦雀无声 あたりは~と静まり返っている/四周万籁俱寂 【副词】 默默地,不显眼地(控えめに物事が行われるさま。ひそかに) ~と余生を送る/默默地度过余生		
\\	ヒット	【名词】 
\\	【英】
\\	打中,大成功,大受欢迎(大成功。大当たり) 
\\	【英】
\\	(棒球)安全打(野球で安打)		
\\	一際	【副词】 格外,分外(他とくらべて程度が際立っているさま。一段と。ひとしお。いっそう) ~高く聳える山/特别高的山 ~声が大きい/声音特别大	ひときわ	
\\	一掴み	【名词】 
\\	一把。少量〔片手でつかむこと。また、その程度の分量。わずかな量) ~の豆しかない/只有很少的豆 
\\	轻易打败对方(簡単に相手を負かすこと) 上から飛び下りて~にする/从上面跳下来,轻松打败对方	ひとつかみ	
\\	人っ子一人	【接续】 (接否定)连一个人。任何人(人ひとり。だれひとり。下に打消しの語を伴って用いる) ~通らない/连一个来往的行人都没有 ~居ない/连一个人影都没有	ひとっこひとり	
\\	人波	【名词】 潮水般的人群,人潮(大勢の人が波のように押し合って動くこと) ~にのまれる/被埋没在人群里	ひとなみ	
\\	一眠り	【名词】 睡一会儿,打盹儿,打瞌睡(しばらくの間、眠ること。ひとねいり) 車中で~する/在车里打个盹儿	ひとねむり	
\\	日の丸	【名词】 
\\	太阳形(太陽をかたどった赤色や金色の丸。古くから、扇や旗さし物などに用いられた) 
\\	太阳旗,日本国旗(白地に、太陽をかたどった赤い丸を描いた旗。日の丸の旗) ~の旗/日本国旗	ひのまる	
\\	日増し	
\\	日益((多く「~に」の形で)日数がたつにつれて,その傾向・程度が強まること) ~に大きくなっていく柿の実/日益长大的柿子 ~に暖かになる/日益变暖 
\\	陈旧,不新鲜(日数がたって古くなっていること。また,そのもの) ~のようかん/不新鲜的羊羹	ひまし	
\\	びゅうびゅう	【副词】 呼呼地,嗖嗖地(激しく風が吹いたり、風を切ったりするさまを表わす語) 風が~(と)吹きつける/风呼呼地吹着	ビュービュー	
\\	標高	【名词】 〈地〉标高,海拔(一定の海面を基準として表した、山や陸地などの高さ)	ひょうこう	
\\	瓢箪	【名词】 
\\	〈植〉葫芦(ウリ科の一年草。ユウガオの変種。茎は長く伸び、巻きひげで他物にからまる。葉は心臓形。花は白色。果実は中間部がくびれ、熟すと果皮が硬くなる。観賞用・日除け用に植える。実を容器にする。[季]秋) 
\\	将葫芦的挖去瓤后用于盛酒等的容器
\\	のよく熟した果実の中身をくりぬき、十分に乾燥させたもの。酒・水や七味唐辛子の容器とする。ひさご。ふくべ) ~から駒/葫芦中跑出马来。事出意外。戏言成现实	ひょうたん	
\\	平	【名词】 平;一般;平椀的略语;平入り的略语;平面部分;一个劲地 【名词】 平坦;平安;平定;寻常;平氏 【接续词】 坪(地名の下に付いて,山間の平地であることを表す) 【日本地名】 平市(福島県いわき市内の地名。いわき市の中心地区で,商業・行政・文教機能が集中。旧,平市。江戸時代は鳥居氏・安藤氏などの城下町)	ひら へい だいら たいら	
\\	ファッション	【名词】 
\\	【英】
\\	流行,时兴(流行。はやり) 
\\	样式;剪裁式样,时装(服装。髪型などについていう。また、流行の服装)		
\\	不意	【名・形容动词/ナ形容词】 冷不防,忽然,突然,抽冷子,意外,想不到,出其不意(思いもよらないこと。思いがけないこと。意外)	ふい	
\\	不可解	【名・形容动词/ナ形容词】 不可理解,不可思议((複雑または神秘的すぎて)理解ができないさま。わけがわからないこと。怪しいこと)	ふかかい	
\\	武漢	【中国地名】 武汉市(中国、湖北省の省都。武昌・漢陽・漢口の3市が1949年に統合されて成立)	ぶかん	
\\	吹き飛ぶ	【自动词・五段/一类】 
\\	吹跑,刮跑(風などに吹かれて物が飛ぶ) 強風で看板が~/强风把看板吹跑 
\\	消散,消失,云消雾散(いっぺんに消えてなくなる)	ふきとぶ	
\\	吹き捲る	【自他・サ变/三类】 
\\	狂吹,狂风大作(風が激しく吹きつづける) 
\\	夸夸其谈,大吹大擂,吹起来没完(大げさなことや自慢話などを、とめどなくしゃべる)	ふきまくる	
\\	伏する	【自他・サ变/三类】 
\\	伏、俯。弯下腰去,拜倒(かがむ。平伏する) 
\\	潜伏、隐藏(かくれる。ひそむ) 
\\	屈服、降服(負けて、従う。降伏する。また、降伏させる)	ふくする	
\\	無作法	【名・形容动词/ナ形容词】 没规矩,粗鲁,没礼貌(礼儀作法にはずれていること。また、そのさま。ぶしつけ) ~な振る舞い/粗鲁的举止	ぶさほう	不作法
\\	不始末	【名・形容动词/ナ形容词】 
\\	不经心,不注意,不在意〔始末の悪いこと〕 たき火の~から山火事を起こす/因为没管好篝火引起了山林火灾 
\\	不检点,不规矩〔ふしだら〕 ~をしでかす/闯祸,干出不检点的事来	ふしまつ	
\\	不信	【名・形容动词/ナ形容词】 
\\	不诚实,不守信用
\\	誠実でないこと。偽りの多いこと) 
\\	不相信(信じないこと。信用できないこと) 
\\	(宗教)没有信仰心,不诚心(信仰心がないこと。不信心)	ふしん	
\\	不断	【名词・副词】 不断(絶えないこと。絶え間の無いこと)	ふだん	
\\	仏教	【名词】 佛教(宗教の一つである。この世の苦しみ、迷いの世界と見、正しい実験によってそこから脱出すること、迷いに沈む生きとし生けるものを救うことを目指す)	ぶっきょう	
\\	仏壇	【名词】 (放佛像,牌位的)佛龛(仏像・位牌を安置する壇)	ぶつだん	
\\	踏み締める	【他动词・一段/二类】 
\\	用力踩(力を入れてしっかり踏む) 
\\	踩结实(踏んでかためる。踏みかためる)	ふみしめる	
\\	プライベート	【形容动词/ナ形容词】 【英】
\\	个人的,私人的,仅与己有关的(個人的。私的)		
\\	振り子	【名词】 摆(定点を中心にして一定周期で振動する物体) 時計の~が止まった/钟停摆了 ~時計/摆钟	ふりこ	
\\	不慮	【名词】 意外,不测(思いがけないこと。不意。意外。よくないことについていう)	ふりょ	
\\	古く	【名词・副】 以前(ずっと以前の時期。昔) ~さかのぼれば/追溯从前 ~からのつきあい/老交情 ~からの友人/老朋友	ふるく	
\\	無礼	【名・形容动词/ナ形容词】 没有礼貌,不恭敬,失礼(礼儀に反すること)	ぶれい	
\\	踏ん付ける	【他动词・一段/二类】 
\\	踩住,踩上(踏んで押さえる。強く踏む) 
\\	藐视,轻视,欺侮,践踏(人の面目を無視して、事をする。人の気持ちをないがしろにする)	ふんづける	
\\	返答	【名词】 回答,回话(問いや呼びかけなどに対して答えること。また、その言葉。返事) ~を求める/要求(对方)回答 彼の~を待つ/等他回话	へんとう	
\\	変貌	【自动词・サ变/三类】 变形,改观,改变面貌(姿や様子などがすっかり変わること。変容) 農村は今日大きく~した/农村(地区)今天面貌大变了	へんぼう	
\\	防寒	【名词】 防寒,御寒(寒さを防ぐこと)	ぼうかん	
\\	暴走	【自动词・サ变/三类】 
\\	乱跑,狂跑(常軌や規則を無視して乱暴に走ること) 
\\	失去控制突然驶出(運転者のいない車両が走りだすこと) 
\\	鲁莽从事,随心所欲(周囲の状況や他の人の思惑を考えずに勝手にどんどん事を進めること)	ぼうそう	
\\	北上	【名词】 北上,北进(北に向かって進むこと)	ほくじょう	
\\	北部	【名词】 北部(その地域の北よりの部分)	ほくぶ	
\\	北風	【名词】 北风(北の方から吹いてくる風) ~が吹く/刮北风	ほくふう	
\\	ホック	【名词】 【荷】
\\	挂钩,摁扣儿,子母扣(洋服の合わせ目などをひっかけてとめる、かぎ形の止め金。フック)		
\\	微笑み	【名词】 微笑(ほほえむこと) ~をうかべる/面泛微笑;浮出微笑	ほほえみ	
\\	本気	【名・形容动词/ナ形容词】 真实,认真,当真,真的(まじめな心。冗談や遊びでない真剣な気持。また、そのような気持で取り組むさま) ~で言っているのか/此话当真?不是开玩笑? 嘘を~にする/把谎言当作真实	ほんき	
\\	本州	【地名】 本州;日本列岛中的主岛(日本列島最大の島、四大島の中の一島)	ほんしゅう	
\\	巻き起こる	【自动词・五段/一类】 
\\	产生,发生 拍手があらしのように~/掌声雷鸣 
\\	刮起,扬起,吹起 ブームを~/掀起风潮 
\\	掀起	まきおこる	
\\	瞬く間	【词组】 瞬時。瞬間。眨眼之间,一瞬间,转眼间 (まばたくほどのごく短い間。まばたきをするほどのきわめて短い時間。あっという間。瞬間。瞬く中。瞬く隙) それは~の出来事だった ~に時がたつ ~に作り上げる	またたくま	
\\	松葉杖	【名词】 (架在腋下用的)拐杖,丁字拐(足の不自由な人が用いる杖。上辺が松葉のように二またになっているもの) ~をついて歩く/拄着拐杖走路	まつばづえ	
\\	間取り	【名】 房间布局(配置),平面布置(住宅の部屋の配置) ~のよい家/房间布局好的房子 ~図/房屋平面布置图	まどり	
\\	ままならない	【接续】 不随心,不可心,不如意,不如愿(思うようにならない。自由にできない。ままにならない) ~は浮き世/世事常不如意;不如意事常八九		
\\	丸々	【名词】 
\\	双圈儿(二重の圏点) 
\\	连续两个
\\	符号(ある事物を伏せて暗示するときに用いる符号。
\\	と書く) 【副词】 
\\	胖的溜圆(よく太っているさま) ~と太った赤ちゃん/胖墩墩的婴儿。 
\\	全部。整个(ある数量や事柄の全体に及ぶさま。完全に) ~食べ残す/一点也没吃	まるまる	
\\	満腹	【自动词・サ变/三类】 吃饱(腹がいっぱいになること。それ以上食べられないほど腹がいっぱいであること) もう~です/已经吃饱了 ~感を与える食べ物/饱人〔饱肚子〕的食品 見ただけで~だ/一瞧见就觉得饱了〔厌腻了〕	まんぷく	
\\	水際	【名词】 水边,水滨;河滨;湖滨;海滨(川・海・湖の波が打ち寄せて来る、常に濡れた一帯の砂地。水面が陸地と接している所)	みずぎわ みぎわ	
\\	妄り	【形容动词/ナ形容词】 
\\	〔むやみやたら〕[道理にあわない]不合情理;[でたらめに]胡乱;[かってに]随便 ~に鳥をとってはならない/不得任意捕鸟 ~に動物に餌を与えないでください/请勿随意给动物乱喂食 ~に入るべからず/不准擅入 
\\	〔でたらめ〕狂妄;[ゆきすぎた]过分 ~なふるまい/狂妄行为;胡作非为;不合情理,胡乱,随便,狂妄	みだり	濫り
\\	満ち溢れる	【自动词・一段/二类】 洋溢,饱满(いっぱいになってこぼれる。あふれ出さんばかりである) 幸せに~た顔/洋溢幸福的脸庞	みちあふれる	
\\	道草を食う	【惯用句】 
\\	路上闲逛。在途中(闲逛)耽误(由马吃路边的草而迟迟不前行之意,引申指做某事因偏离轨道走弯路而浪费时间)(目的地に行く途中で、他のことに関わって時間を費やす) 学校からの帰りに~子供が多い/在放学回家路上闲逛的孩子很多 
\\	走上弯路耽误时间(馬が路傍の草を食って進行が遅れる) 子供のとき、演劇に夢中になって、二年ほど~た/小时候迷上了演戏,浪费了两年时光	みちくさをくう	
\\	道端	【名词】 道旁,路边。道路的旁边(道のほとり。路傍)	みちばた	
\\	密度	【名词】 
\\	物质单位体积的质量(物質の単位体積の質量) 
\\	密度(粗密の度合) 
\\	文章,讲话等内容的充实程度(内容の充実している度合)	みつど	
\\	耳学問	【名词】 道听途说之学,一知半解的学问(自分で修得したものでなく、人から聞いて得た知識。聞きかじった知識)	みみがくもん	
\\	夢幻	【名词】 梦幻,梦想,虚幻,无常(夢とまぼろし。非常にはかないことのたとえ。はかないことのたとえ)	むげん ゆめまぼろし	
\\	無策	【名词】 无策,无对策,没办法(何の方策も対策もないこと。前もって何の策もたてていないこと) 無為~/束手无策	むさく	
\\	むさ苦しい	【形】 肮脏,邋遢(ごちゃごちゃときたならしい。だらしなく不潔である。むさくろしい) ~身なり/邋遢的打扮	むさくるしい	
\\	虫が良い	【惯用句】 自私(自分の都合ばかり考えて他を顧みない。身がってである) ~考え/自私的想法 ~提案/自私自利的意见	むしがいい	
\\	無神経	【形容动词/ナ形容词】 反应迟钝,反应慢,没反应,没感觉(感覚が鈍いこと。 恥や外聞、他人の気持ちなどを気にしないこと) 彼は~な男だ/他是个反应迟钝(满不在乎,不体谅,不顾及别人)的人 ことばづかいに~な人はきらわれる/说话不注意轻重的人令人讨厌	むしんけい	
\\	空しい	【形容词/い形容词】 
\\	空虚,空洞。仅有外形而没有实际内容(体だけで中身がない。うつろ。形だけで中身がない。形式だけで実質が伴わない。うつろである) ~笑い/虚伪的笑 
\\	空虚,徒然,徒劳。不起任何作用,不留任何结果。无事实的(根拠がない。無実である;何の役にも立たない。何の結果も残らない) 
\\	虚幻。不确实,不可靠;无意义的,徒劳的,变幻无常的(確実でない。頼りにならない。はかない)	むなしい	虚しい
\\	無法	【名词・形动】 
\\	不执法(法のないこと。法律があっても守られないこと。また、そのさま) ~地帯/无法地带 
\\	不讲道理,无法无天,蛮横,粗暴(道理にはずれていること。乱暴なこと。また、そのさま) ~者/无法无天者	むほう	
\\	無闇	【形容动词/ナ形容词】 
\\	胡乱,随便,轻率,瞎(そうすべき理由もないのに結果を考えずに行うこと) ~に人のうわさをするものではない/不要乱议论人 ~に山の木を切る/滥伐山林树木 
\\	过度,过分,太,乱(度を超しているさま)	むやみ	無暗
\\	明白	【名・形容动词/ナ形容词】 明白,明显(はっきりしていて疑う余地の全くない・こと(さま);同:明瞭)	めいはく	
\\	名場面	【名词】 影像中为人所熟知的经典片段(すぐれた場面。有名な場面。感動的な場面)	めいばめん	
\\	目薬	【名词】 
\\	眼药(眼病をなおすため,目につける薬。点眼薬。点眼水。同:眼薬) ~をさす/上〔点〕眼药 
\\	小恩,小惠,小礼物,小贿赂
\\	賄賂としてのきわめてわずかの金銭や贈り物。鼻薬) ~がきく/贿赂有效果了	めぐすり	
\\	目先	【名词】 
\\	目前;眼前(すぐ目の前) 
\\	预见(先をよく見通すことができる)	めさき	
\\	目の前	【名词】 
\\	眼前,面前(見ている人の前。眼前) ~にある本、持ってきてくれない/能把你面前的那本书拿给我吗? 死んだはずの人が突然~に現れたら、どんなに怖いだろう/应该已经死了的人突然出现在眼前那该多可怕啊 
\\	迫在目前,最近(ごく近い将来。目前) 入試が~に迫る/入学考试迫在眉睫。 ~が暗くなる/前途一片黑暗	めのまえ	
\\	目紛しい	【形容词/い形容词】 眼花缭乱,天旋地转;瞬息万变的(目の前をいろいろなものが次から次へと通りすぎて、目のまわるような感じである;物の動きが早くて目が回るようだ。変化が激しくて,対応できない) ~車の往来/令人眼花缭乱的车水马龙	めまぐるしい	目紛るしい
\\	燃え上がる	【自动词・五段/一类】 燃起,烧起(燃えて、炎が高く上がる。一時的に特定の感情が強まる意にも用いられる)	もえあがる	
\\	黙然	【形容动词/ナ形容词】 默然(黙っている様子。口をつぐんでいるさま) ~と座っている/静静地坐着	もくぜん もくねん	
\\	黙認	【名词・他サ】 默认,默许(はっきりいいとは言わないで、見逃すこと) あくどいいたずらは~できない/不能纵容恶劣的恶作剧	もくにん	
\\	黙許	【名词・他动・三类】 默许,默认(知らないふりをしてそのまま許すこと。黙認) ~できない不正/无法容忍的恶劣行径	もっきょ	
\\	持って生まれた	【惯用句・连语・连体】 生来就有的,天生的(生まれつき備わっている) 人間には、それぞれ~性分があります/每个人都有着与生俱来的性情 ~せっかちな性分/生就的急性子 ~才能。/天赋(的才能)	もってうまれた	
\\	弄ぶ	【他动词・五段/一类】 
\\	摆弄,把玩;(拿着)玩(いじくる。手に持って遊ぶ。手であれこれいじる) ハンカチを~/摆弄着手帕 
\\	赏玩;玩赏,欣赏(観賞して楽しむ。心の慰みとして愛する。賞翫する) 書画骨董を~/赏玩书画古董 
\\	操纵;摆布,玩忽,任意处置,随意对待(思うままにあやつる。弄する;好き勝手に扱う。楽しむかのように、思いのままに操る) 
\\	戏弄,捉弄。玩弄(人を慰みものにする。なぶる) 他人の気持ちを~/玩弄他人的感情	もてあそぶ	玩ぶ
\\	物持ち	【名词】 
\\	财主,富人(財産を多く持っている人。財産家) 村一番の~/村里的首富 
\\	使用物品经心〔在意〕(物を大切にしていつまでも持っていること) 彼は~がいい/他使用东西很经心	ものもち	
\\	最早	【副词】 
\\	(事到如今)已经(今となっては。もう) ~間にあわない/(事到如今)已经来不及了 医者を呼んでも~手おくれだ/就是请医生来,也已经晚了 
\\	再快也…,已经…(早くも。すでに) ~12時だ/已经十二点了	もはや	
\\	靄	【名词】 霭、云气、烟霭、轻雾(空気中に小さい水滴や吸湿性の粒子などが浮遊している状態)	もや	
\\	安らぎ	【名词】 安乐,平静,安稳(心がゆったりと落ち着いて穏やかなこと) 心の~を覚える/觉得心情平静	やすらぎ	
\\	宿	【名词】 
\\	房屋,家(すみか。家) 
\\	宿,过夜,下榻,旅馆,旅店(一夜を過ごす所) 【名词】 
\\	宿驿,驿站(宿場) 
\\	星宿(星座)	やど しゅく	
\\	山道	【名词】 山道,山路(山の中の小さな道)	やまみち さんどう	
\\	有限	有限(限度・限界のあること)	ゆうげん	
\\	裕福	【名・形容动词/ナ形容词】 富裕(財産があり、生活がゆたかであること)	ゆうふく	
\\	夕闇	【名词】 暮色,薄暮,黄昏,夜色苍茫(夕方のうす暗さ。宵闇)	ゆうやみ	
\\	遊覧船	客船の一種で、川・湖・沼・港湾など景観の良好な観光地の水域で航行される旅客船。主に観光客のために航行され、広義には水上バスや船下りも遊覧船の一種といえる。なお、レストランを備えた大型遊覧船はレストラン船に分類される	ゆうらんせん	
\\	行き着く	【自动词・五段/一类】 走到,到达(目的地)(行こうと目指した目的地に着く。目的地にたどりつく) 道に迷ったすえ、やっと~/在经历迷路后终于到达目的地	ゆきつく いきつく	
\\	溶液	【名词】 溶液(2種またはそれ以上の物質が溶けて均質になっている液体。溶けている物質を溶質、溶かしている液体を溶媒という)	ようえき	
\\	要する	【他动词・サ变/三类】 
\\	必须,需要(そのこと、あるいはそのものを必要とする) 急を~問題/急待解决的问题 
\\	埋伏,伏击(人を道に待ち受ける。待ち伏せする) 
\\	要之,总而言之(要約する)	ようする	
\\	用足し	【名词・自动・三类】 
\\	办事(用をすます) ~に出かける/出去办事 
\\	解手(用便) ちょっと~しをしてくる/我去解一下手	ようたし	用達
\\	余寒	【名词】 春寒(立春後の寒さ。寒が明けてもなお残る寒さ。残寒) ~厳しき折から/时值春寒料峭	よかん	
\\	横好き	【名词】 对专业以外的爱好,外行的酷好(じょうずでもないのに、むやみに好むこと) 下手の~/不熟练的业余爱好	よこずき	
\\	横たわる	【自动词・五段/一类】 
\\	躺卧(横になる) ベッドに~/躺在床上 
\\	横放,横亘〔横に長くのびる〕 山脈が~/山脉横亘 
\\	迫在眼前,横在前面〔前にひかえている〕 目の前に危険が~/危险就在眼前	よこたわる	
\\	良し悪し	【名词】 
\\	善恶,好歹。好与坏,善与恶(善いことと悪いこと。よいかわるいか。ぜんあく) ~の見分けがつく/能辨别好歹 
\\	有利有弊,也好也不好。很难说是好是坏。有优点也有缺点(よいとも悪いともすぐには判断できかねる状態であること。良い点も悪い点もあって、一概にどちらとも言えないこと。よしわるし) 慎重過ぎるのも~だ/谨慎过头了也不好	よしあし	善し悪し
\\	予想外	【名词・形动】 出乎预料(予想と違った成り行きとなること。また、そのさま。思いのほか。意外) 結果は~だった/结局出乎意料	よそうがい	
\\	余地	【名词】 
\\	空地〔あいた土地〕 建て増しをするだけの~はある/还有可以扩建的空地 
\\	余地,宽裕。剩余部分,尚有些富余(残された部分。何かできるゆとり) 弁解の~を与えない/不容辩白	よち	
\\	予兆	【名词】 预兆,兆头	よちょう	
\\	世に言う	【惯用句】 常言道,俗话说(いわゆる。世の中で言い習わされている) これが~革命というものか/这就是常说的革命吗	よにいう	
\\	弱々しい	【形容词/い形容词】孱弱,软弱(いかにも弱そうに見える。力がないようすである。また、弱りきったありさまである)	よわよわしい	
\\	羅針盤	【名】 罗盘,指南针(磁針がほとんど南北を指す特性を利用し、船舶・航空機などで方位を測定する用具。羅針儀。磁気羅針儀。羅盤。コンパス。ジャイロ‐コンパス)	らしんばん	
\\	利己的	【形容动词/ナ形容词】 利己的,自私自利的(自分の利益だけを追求しようとするさま)	りこてき	
\\	流通	【自动词・サ变/三类】 流通(空気や水などが、滞らずに流れかようこと。広く通用すること。また、広く行われること)	りゅうつう	
\\	流動性	【名】 
\\	流动性。如液体或气体等非固定而流动的性质(固定しないで流れ動く性質。不定形に変化する性質) 
\\	流动性。将某种资产无损地转变为货币的难易程度(ある資産について、損失を被ることなく、しかも直ちに貨幣にかえることができる可能性の度合い)	りゅうどうせい	
\\	両岸	【名词】 两岸(川などの両方の岸。りょうぎし)	りょうがん	
\\	レジャー	【名词】 
\\	【英】
\\	空闲,闲暇,业余时间(余暇) 
\\	【英】
\\	业余时间的娱乐(また、それを使ってする娯楽)		
\\	レントゲン	【名词】 
\\	【独】
\\	伦,伦琴(エックス線量の単位) 
\\	【独】
\\	伦琴射线,爱克斯射线,
\\	光线
\\	線・γ線の照射線量の単位)		
\\	ローマ	【名词】 
\\	【拉】
\\	古罗马(イタリア半島中部に興ったラテン人の都市国家。古代ローマ) 
\\	罗马(イタリア共和国の首都。テベレ川下流に臨む七つの丘を中心に発達。古代ローマ帝国の首都。中世以降はローマ教皇庁の所在地として繁栄。コロセウム・カラカラ浴場跡・パンテオンなどの遺跡に富む世界的な観光都市)		
\\	若々しい	【形容词/い形容词】 年轻而有朝气的;朝气蓬勃的(元気にみちていて、いかにも若さを感じさせる様子だ)	わかわかしい	
\\	和気藹々	【形動】 和蔼可亲(楽しい気分が満ちているさま;親しい)	わきあいあい	
\\	沸き起こる	【自动词・五段/一类】 涌现,呈现(感情などがこみ上げてくる) 悲しみが~/悲伤涌上心头	わきおこる	湧き起こる
\\	枠	【名词】 
\\	纺线,篗子,(缠线的)木制线框(糸を巻く道具) 
\\	边,缘,框,轮廓(物の輪郭や範囲を定めるために設けた仕切りや線) 
\\	框,框子(木・竹・金属など細い材で組んだ、物の骨組みや囲み) 
\\	限制,范围,框框(物事の制約。範囲)	わく	
\\	ありとあらゆる	【惯用句】 所有,一切。各种,各式各样的(「あらゆる」を強めた言い方。すべての。ある限りの。連体詞的に用いる) ~手段を講ずる/使用各种手段		
\\	暑い	【形容词/イ形容词】 热(空気や物の温度または体温が著しく高い状態)	あつい	
\end{CJK}
\end{document}