\documentclass[8pt]{extreport} 
\usepackage{hyperref}
\usepackage{CJKutf8}
\begin{document}
\begin{CJK}{UTF8}{min}
\\	誰
\\	毎日
\\	誰の 
\\	そのほかに、それ以上に、その上に
\\	暗い、黒い、黒っぽい
\\	目 複数形:
\\	花瓶
\\	どこへ、どこを目指して
\\	おはよう
\\	こんにちは
\\	同僚、仲間、同志
\\	小さい
\\	娘
\\	の愛称形
\\	煙
\\	買う(完了体) 
\\	→不完了体:
\\	ハエ
\\	[挿入語]恐らく、たぶん、ひょっとすると、 (疑問や勧誘の意味を和らげて)なんなら、もし良かったら
\\	(花が)咲く、咲いている(不完了体) 
\\	縫う、縫い物をする(不完了体)
\\	指 複数形:
\\	白い
\\	百合
\\	血の繋がった、実の、肉親の、親類の、親愛な
\\	歯
\\	耳 複数形:
\\	オス猫
\\	サーカス
\\	ピリオド 点、地点 点、斑点、点在するもの
\\	モミの木、クリスマスツリー クリスマス中の子供のための祭り
\\	にわとり、おんどり 指小形の
\\	もよく使われる
\\	どうぞ! どういたしまして! どうか〜してください こちらへどうぞ ください
\\	望む、願う(不完了体)  →完了体:
\\	快適な、快い、愉快な、楽しい 魅力ある、好感を与える
\\	敵、仇
\\	スペイン人(男)
\\	イタリア人(男)
\\	杖、棒、ステッキ 棒状のもの、(口)痩せこけた人
\\	与える、貸す、知らせる、教える 〜させる(完了体) 
\\	→不完了体:
\\	祝う、祝賀する(不完了体)  →完了体:
\\	産むこと、出産、誕生 誕生日
\\	奇妙な、不思議な、普通と違った
\\	地球 (天国に対する)地上、この世、現世 土地、地面、地表、大地
\\	整理・整頓された(状態)、(人体・機械などの)よい状態 規律、秩序、きまり、(社会的・経済的)体制 
\\	立ち上がる、(ある場所に行って)立つ (助動詞として)…になる (完了体の過去形のみ)…するようになった、…し始めた(完了体) 
\\	→不完了体:
\\	悪魔、悪霊 (ふつうよからぬことをしようとする)欲望
\\	嘘、虚偽、欺瞞 虚構、作り話
\\	仕事、活動、事業、事務 専門、業種
\\	ポケット (口)財産、金
\\	民話、昔話、御伽噺、童話
\\	助詞:ただ…だけ、…のみ 接続詞:(文または文の成分を付加して)ただ…だけ、(時の従属文を導いて)…するやいなや
\\	ほこり、ちり、粉塵状のもの 花粉
\\	(人・動物、まれに草・木の)頭、首 頭脳、判断力、意識
\\	打つ、叩く、ぶつかる (光などが)当たる、差す、(においなどが)刺激する(不完了体) 
\\	努力する、精を出す、骨折る …しようと試みる、努める(不完了体)  →完了体:
\\	与える、やる、あげる、貸す、貸与する、(使用・利用のために)提供する、知らせる、教える(不完了体) 
\\	→完了体:
\\	眠る、睡眠する、寝入っている (死者が)眠る、永眠する(不完了体)
\\	さようなら(別れの挨拶) (直訳:また会うときまで)
\\	さようなら、またね、じゃあね(別れの挨拶、親しい間柄で使う)
\\	重要な
\\	いかに、どのように
\\	(句読点としての)横線、ダッシュ
\\	こんばんは!
\\	健康 健康状態、身体の具合
\\	健康を祝して!(乾杯の挨拶) どういたしまして(客にごちそうでもてなし、お礼を言われた時の返事) (直訳:健康のために)
\\	噛む、噛み付く (虫が)刺す・食う、(鳥が)つつく・ついばむ 噛み切る、齧り取る (口)(寒気・植物の棘などが)刺す、ちくちくさせる(不完了体)  →完了体:
\\	雄羊
\\	見せる、披露する、公開する、提示する(完了体)  →不完了体:
\\	飢えた、空腹な
\\	オオカミ
\\	野ウサギ 
\\	キツネ
\\	リス
\\	(乗り物の)車輪、車 (機械の部品としての)輪、車 複:
\\	[関係][疑問]何の目的で、何のために
\\	金
\\	雄牛
\\	雌牛
\\	雄熊 熊
\\	の指小形 ネズミ
\\	小さな店/(古いロシアの家屋・農家の壁に取り付けた)寝台兼用腰掛板 (公園などの)ベンチ
\\	座る (乗り物に)乗る、乗り込む(完了体) 
\\	(「座る」という動作を指す、「座っている」状態を指す場合は
\\	→不完了体:
\\	鼻
\\	食べる、食う、食い尽くす、平らげる(完了体) 
\\	過去形:
\\	→不完了体:
\\	(容器・箱などの)蓋、覆い、カバー、(本の)表紙
\\	(複は単と同義)草、草木、植物 草地、草むら
\\	受け取る、もらう、受ける(不完了体)  →完了体:
\\	似たような、…に似ている
\\	(名詞)	全て、一切、何もかも
\\	(副詞)	いつも、絶えず
\\	ヘビ 性悪な女性のこと
\\	生格	~のうちの ~から〜製の
\\	(乗って)出発する、出かける (乗って)向かう、赴く(完了体)
\\	時間を表す語・数詞	~時に、~(の時)に
\\	与格	~(事柄や人)に向けて、~に向かって、(ある人の)ところへ ~のほうへ 〜へ、〜に(接近・接触) 〜へ至る
\\	与格 
\\	動詞・原形	(…は)~しなければならない、〜する必要がある
\\	全部の、統計で (限定助詞として、しばしば
\\	と共に)たった、わずかに
\\	(自分の言説の)誤りを改める、言い直す、書き直す 健康を回復する 好転する、よくなる (口)太る(不完了体)  →完了体:
\\	造格	~と ~と一緒に
\\	小さな、少ない 弱い (時間が)短い
\\	造格	~を学んでいる、〜に従事している、〜を(趣味や専門で)やっている (勉強、スポーツを)する (娯楽、スポーツ、勉強、研究などの話題でよく用いられる)
\\	造格	~に興味がある、~に興味をもつ
\\	前置格	~について 
\\	と一緒に使う場合は
\\	になる
\\	[序]十一番目の
\\	[序]十二番目の
\\	[序]十三番目の
\\	[序]十四番目の
\\	[序]十五番目の
\\	[序]十六番目の
\\	[序]十七番目の
\\	[序]十八番目の
\\	[序]十九番目の
\\	[序]二十番目の
\\	[序]三十番目の
\\	順序数詞の中性の形	日付を表す(月を付ける場合は、月を生格にする 例:
\\	5月5日)
\\	順序数詞・中性形・生格+月名の生格	~月~日
\\	静かな、穏やかな、のどかな、平穏な (心が)落ち着いた、心の安まった、平安な
\\	吠える、うなる(不完了体) 
\\	風
\\	本当に?
\\	対象、目的物
\\	ちょっと伺いますが、… 
\\	よりも丁寧な表現、「すみません」、「恐れ入りますが」ぐらいの意味)
\\	はい
\\	はじめまして! とても嬉しいです!
\\	どうもすいません(丁寧な表現) ごめんなさい(丁寧な表現) 
\\	以下を続けて述べ、そのことについて謝る表現にすることも可能
\\	良いですおかげさまで、順調です(お元気?に対する返答として)
\\	端、ふち、へり、外れ、果て 地方、土地、国
\\	基本的な、根本的な、根元的な、主要な
\\	ギター
\\	星 運、運勢、星回り スター、花形 複:
\\	戸口、戸、扉、ドア
\\	傷、外傷、創傷 心の傷、痛み、悩み
\\	(人・動物の)身体、(精神に対する)肉体
\\	苦しむ、悩む、心を痛める、困る 欠点を持つ、…を患っている、…にかかっている(不完了体)
\\	(飲食物を)煮て作る(不完了体)  →完了体:
\\	肩 複数形:
\\	腹、腹部
\\	肘 生:
\\	複:
\\	複生:
\\	頬
\\	石 単数形:
\\	複数形:
\\	マント
\\	切手、商標、ブランド (口)評判、名声、品質、種類
\\	(落ちないように)手に持っている、握っている、抱えている(不完了体)
\\	射撃する、撃つ、射る (鉄砲を)撃つような音を立てる(不完了体)
\\	[不定](誰のか不明だが)誰かの
\\	(ふつう木の)塀、柵、囲い 障害、障壁
\\	汚れていない、綺麗な、清潔な、きちんとした、さっぱりした 比較級:
\\	(失われたものを)惜しむ気持ち、哀惜の念 (行為などへの)後悔の悔恨、惜しむ気持ち
\\	[定](手で掴まり足を掛けながら)登る、降りる、よじ登る (口)(物を取るために)手を知れる(不完了体) →[不定] 
\\	満足、喜び、娯楽、快楽、遊び、楽しみ
\\	天国
\\	こんにちは! やぁ!
\\	スネージナヤ・バーバ(雪だるま、基本的には三つの玉から成る)
\\	(一般に)球形のもの、玉
\\	何歳ですか? あなたはおいくつですか?
\\	クラブ、同好会 (クラブ活動、文化活動を行う)施設、集会所、公民館
\\	匙、スプーン
\\	あなたのお名前は? (直訳:あなたはどう呼ばれていますか?)
\\	(写真・映画の)撮影
\\	(時期・季節が)始まる、来る/踏む、踏みつける、片足に自分の体重をかける(完了体)  →不完了体:
\\	太陽
\\	バーバ・ヤガー(ヤガー婆さん) ロシア民話に登場する妖婆、鬼婆
\\	(理解・解決・実現が)困難な、難しい、複雑な 多様な
\\	呼吸、息 息吹、匂い、気配
\\	光線、光
\\	持つ、有する(不完了体)
\\	のこぎり、(刃がギザギザの)パン切りナイフ、爪やすり
\\	樽
\\	屋敷番、門番、掃除番
\\	囁き、ひそひそ声、微かな声、さらさらいう音
\\	薪 (単数形のない単語)
\\	交換する 変更する、改める、変える(不完了体)
\\	流行の、人気のある、有名な
\\	職業、業務、作業、仕事 (複)授業
\\	歌、(鳥の)さえずり
\\	造格	~の下で、〜の下に (背番号)を付けて
\\	ケフィール
\\	ごめんなさい すみません
\\	(定語を伴って)料理、食事 台所、炊事場
\\	アライグマ
\\	石鹸、石鹸の泡
\\	島
\\	花
\\	メロン
\\	さくらんぼ
\\	速度、スピード、速さ
\\	袋
\\	ジャズ
\\	冗談、洒落、軽口、おどけ
\\	同様に、均等に、等しく
\\	招き、招待 依頼、勧め、要請
\\	しずく
\\	大きさ、規模、程度 措置、処置、対策、方策、手段
\\	過度に、あまりにも
\\	召し上がれ! おいしく召し上がれ! たんと召し上がれ! 食事を始めましょう! (直訳:心地よい食欲を)
\\	スープ
\\	カワカマス
\\	(口)休日
\\	ざわめく、騒音を立てる(不完了体)
\\	棚 (列車の)寝台
\\	終端、末端、端、果て、はずれ
\\	忙しい
\\	日曜日に
\\	月曜日に
\\	火曜日に(前置詞は
\\	ではなく
\\	になる)
\\	水曜日に(アクセントが
\\	に移動)
\\	木曜日に
\\	金曜日に
\\	土曜日に
\\	ご機嫌いかがですか? 調子はどうですか?
\\	(調子を尋ねられて)まぁまぁです。 何でもありません、大丈夫です。 (相手の謝罪に対して)気にしないでください。
\\	たまねぎ
\\	食物、食料、食品
\\	(調理法として)焼く (不完了体のみ)
\\	(太陽が)焼く、照りつける(不完了体) 
\\	過:
\\	→完了体:
\\	炒める、揚げる(不完了体) →完了体:
\\	空、(地に対しての)天
\\	飲む (不完のみ)(空気などを)吸い込む(不完了体)
\\	命:
\\	→完了体:
\\	幅、横
\\	壁
\\	遅く、遅れて
\\	何故? なんで? どうして?
\\	蒸気、湯気
\\	塩 要点、急所、ポイント
\\	(クシャミをした人に、相手を気遣って)お大事に! (直訳:元気でいてね!)
\\	悪魔、鬼 (罵り言葉で)悪魔、畜生
\\	バーニャ(ロシアの蒸し風呂)
\\	鍵、キー(複数形
\\	では末尾にアクセントが移動)
\\	便利な、使いやすい、快適な、心地よい 都合の良い、有利な
\\	歩く、歩を運ぶ、(歩調をとって力強く)歩く(不完了体)
\\	影、投影
\\	オーブン、(部屋・台所用の)ストーブ、ペチカ、ストーブ、暖炉
\\	ガマズミ [集合]ガマズミの赤い苦い実
\\	(複)涙 (複)泣くこと 複数形:
\\	雲 雲状のもの 暗い表情、影
\\	太鼓、ドラム
\\	それほど~ではない、あまり~ではない
\\	キリン
\\	カバ
\\	ピョートル大帝、ピョートル1世
\\	皇帝、ツァーリ
\\	波
\\	溢れるばかりの、(なにかで)いっぱいの、山盛りの 
\\	満員の、すし詰めの 
\\	(ある感情・気分などに)満ち溢れた、(ある特徴・性質の)著しい 完璧な、完全な、絶対的な 全部の、欠けるところのない、完結した
\\	ピロシキ
\\	負担、重圧
\\	種子 原因、もと、芽、種
\\	お誕生日おめでとう!
\\	レモン、レモンの木
\\	分 ちょっとの間、瞬時
\\	記念日
\\	傑作、名作
\\	楽器の前置格	~を演奏する
\\	(俳優が演じる)役、役割、(見せかけの)演技
\\	あなたのご職業は?
\\	鳥
\\	地域の、地方の、現地の 局所的な、局部的な
\\	ウラジオストック
\\	より頻繁に 
\\	の比較級
\\	食料品、食材
\\	カニ
\\	ヨガ
\\	ウサギの (口)内気な、小心な
\\	声
\\	悲しみ、寂しさ、憂鬱、憂愁
\\	喜んで!
\\	今何時ですか?
\\	これはいくらですか?
\\	大聖堂 (帝政ロシアで)大会議
\\	堀、溝 
\\	リツェイ(学習院) 1811年に設立された教育機関
\\	温泉
\\	気晴らし、なぐさみ、娯楽、遊び
\\	(長尾)肉や乳製品を含まない 精進の
\\	フォーク
\\	アスファルト アスファルト舗装道路
\\	バレーボール
\\	どうぞ(ドアをノックされた時の返事)
\\	がんばれ!
\\	たいしたことないさ!
\\	(水道・ガスの)栓、蛇口
\\	[不定](誰だか分からない・言えない)誰か、ある人 (任意の)誰か
\\	どこか、どこかに、どこかで (口語)だいたい、確か
\\	ありがとう!
\\	絹、シルク
\\	ブラシ
\\	シャワー
\\	いいえ。
\\	(建物の)玄関、車寄せ、(集合住宅の建物の)入口
\\	上昇、高揚 (精神的)高まり
\\	オオトカゲ
\\	より合わせる、なう、より合わせて作る、編む(不完了体)  →完了体:
\\	包帯 ゲートル
\\	ねじ、ボルト スクリュー、プロペラ
\\	梁、桁、ビーム
\\	(犬などの断続的な)吠え声 (廃)(俗)ばり、ののしり言葉
\\	ラマ、アメリカラクダ
\\	枠、フレーム 額縁
\\	ニス、ラッカー、うるし 光沢、輝き、艶
\\	癌/ザリガニ
\\	(廃)雄ギツネ 狡猾な人、おべっか使い
\\	(数量を表す語を伴って)回、度 ~の時、機会、場合 (完了体動詞を伴って)一旦、ひとたび (数量を表す語と前置詞
\\	を伴って)…倍、…分の1 ※不変化 (数を数えるときの)1、ひとつ
\\	(通り抜けられる)穴、隙間、通路 獣道 (機械などの)掃除口、マンホール
\\	祝祭、宴会 ばか騒ぎ、大はしゃぎ
\\	氏族
\\	エゾイチゴ [述語](俗)素晴らしい
\\	(木の枝・藁で覆った)仮小屋、掘立小屋
\\	スイバ、ギシギシ、スカンボ
\\	こぶ、隆起、でっぱり (俗)大物、偉い人
\\	盾 保護者、防護者、後ろ盾 守り
\\	(ふつう複)野菜、青物
\\	(スカート・ズボンなどの)すその広い※不変化
\\	ダニ
\\	(ふつう複)(口)はした金 2カペイカ玉
\\	、半カペイカ玉
\\	穴(虫喰い穴、皮革の穴など) (口)くだらない、つまらない人間
\\	入ること、乗り入れること
\\	遠方、遠隔、遠望、遠景
\\	男性・女性名詞:(口)告げ口をする人、密告者、中傷者 女性名詞:(口)告げ口、中傷   (廃)(口)訴状、密告状   (廃)難癖、言いがかり
\\	ポーランド人/ポルカ(舞踏曲)
\\	(口)熊(の愛称)
\\	屏風、衝立 (文)(真の目的・意図を)隠蔽するもの
\\	ポーランド人
\\	ビザ、査証
\\	私はふくろう。 (ふくろうのように夜更かしをしていることを意味する)
\\	いったん口に出した言葉は、もはや取り返すことができない。 (直訳:言葉は雀ではない)
\\	オオカミと違って森の中に逃げていかないから、仕事はいつでも間に合う。 (直訳:仕事はオオカミではない……)
\\	まだ夜ではない(夢を諦めようとしている人にかける言葉)
\\	彼は何も取り柄がない人物だ。 (直訳:彼は魚でも肉でもない)
\\	彼は自分の価値観に従って生きている。 (直訳:彼は白いカラスだ)
\\	はかり
\\	エゴイスト
\\	鋼鉄
\\	この方はどなたですか?
\\	あの方はどなたですか?
\\	これは何ですか?
\\	これは何ですか?
\\	あれは何ですか?
\\	あれは何ですか?
\\	ここにあるのは何ですか?
\\	ここにあるのは何ですか?
\\	(湿地・沼地に散在する草・苔に覆われた)小丘、土塊 (ふつう複)道路・地面のでこぼこ
\\	フクロウ (口)夜型人間
\\	トルストイ(人名)
\\	(油絵の)画布、キャンバス 麻布、亜麻布、カンバス
\\	至る所に
\\	まもなく、急に
\\	擬声語で「寒い!」を意味する
\\	シマウマ (口)(街路上の)歩行者横断歩道標示
\\	コブラ
\\	(ひとつも)…せぬ、なにも…ない …だろうと
\\	カラス (口)うすのろ、まぬけ
\\	雨裂、谷間(雨水や雪解け水の浸食によってできた深くて長い窪地)
\\	では、また明日!また明日ね!
\\	ガチョウ
\\	中庭、庭、構内、(個々の)屋敷
\\	(パンなどの)硬い表層、皮 硬くなった表層、硬皮、外皮
\\	飲み込む、(口)がつがつ食べる、一気に飲み干す (口)じっと我慢する、(感情などを)表に出さない、押し隠す(完了体)  →不完了体:
\\	唾を吐く (唾とともに)…を吐き出す(不完了体) 
\\	→完了体・一回:
\\	(不完了体動詞の不定形を伴って)(〜するのを)やめる (不要なものとして)捨てる 投げる ほうる、投げつける(不完了体) →完了体:
\\	労働、勤労、仕事、職業、業務
\\	誕生日
\\	彼は豊富な経験を積んだ人物だ。 (直訳:彼は撃たれたことのある雀だ)
\\	いろいろとお世話になり、ありがとうございました。 (お世話になった人へのお礼の言葉)
\\	彼はキツネのようにずる賢い。
\\	形容詞	なんて〜
\\	どういたしまして。 (ありがとうと言われた時の返事)
\\	彼はオオカミのようにお腹を空かせている。
\\	絶対しない、絶対許せない 断じて、決して
\\	悲しい気持ちの、寂しげな、沈んだ 悲しげな、悲しみに満ちた
\\	サル
\\	の指小・表愛形:ハリネズミ 短く刈り込んだ男、スポーツ刈りの頭
\\	熊の、熊のような
\\	[比較級]
\\	よりよい、より優れた、より見事な [最上級]
\\	最もよい
\\	[比較級]
\\	より小さい、より少ない、より弱い [最上級]
\\	狼の 悪意に満ちた、残忍な、強欲な
\\	非常に痩せた、やつれた、衰弱した (髪・髭などが)薄い、まだらな
\\	ゾウ 大男
\\	ゴムタイヤ (馬車などの)輪金、リム
\\	円盤、円盤状のもの、(太陽・月などの)輪郭
\\	(口)(装置・設備などの)据え付け、組立て、設置
\\	万事順調です。 (お元気?に対する返答として)
\\	ありがとう、万事順調です。 (お元気?の返答として)
\\	スズメ
\\	射撃を受けたことのある、戦場経験豊かな、百戦錬磨の、世故に長けた (口)(鳥が)狩猟でしとめられた、射止められた
\\	は
\\	のようです。
\\	彼はロバのように頑固だ。
\\	彼はウサギのように臆病だ。
\\	夫婦は性格や好みが似ている。 (直訳:夫婦は一つの悪魔だ)
\\	それはどうでもいいことだ。 (直訳:これは10番目のことだ)
\\	彼はあまりにも口が軽い。 (直訳:彼は舌が長い)
\\	彼は少し抜けている。 (直訳:彼の家には全員が揃っていない)
\\	彼はなんでもできる達人だ。 (直訳:彼は黄金の腕を持っている)
\\	私は重要な仕事を終えて、ほっとした。 (直訳:私の肩から山が下りた)
\\	これは作り話だ。 (直訳:これはおばあさんの昔話だ)
\\	ごめんね。 
\\	のくだけた言い方、親しい人に使う)
\\	[不定](疑問詞として)いくら、いくつ、どれだけ (感嘆文で)なんと多くの (関係詞として)…だけの
\\	市長
\\	予算、(個人の)生計、収支 計算、配分
\\	イメージ、印象
\\	頑固な、強情な、意固地な
\\	サイ
\\	ロバ (俗)愚かで強情な人間
\\	杖、(複)松葉杖
\\	(牛乳、バター、卵など入りの)味付きパン
\\	サタン、悪魔 ※男性名詞
\\	(本当に)ごめんなさい、すみません 
\\	よりも丁寧な言い方)
\\	(本当に)ごめんなさい、すみません
\\	のくだけた言い方、親しい人に用いる)
\\	シカ、トナカイ
\\	生格+
\\	〜はいません、ありません
\\	(生格)+
\\	(イベントなどの主格)	
\\	には
\\	がある。
\\	(持ち主・生格) 
\\	(物・主格)	
\\	には
\\	がある。 (「物」の存在することは当然で、どんなものかが問題になる時は
\\	を使わない)
\\	(持ち主・生格) 
\\	(物・主格)	
\\	は
\\	を持っていない。
\\	それぞれの家族
\\	それぞれの単語
\\	(生格)	
\\	は
\\	のところにいる
\\	エカチェリーナ2世(1729-1796年)
\\	もう一度お願いします。
\\	[不定]ある、さる、なんとかいう、なにかの、どこかの、…とかいう
\\	鉢、深皿
\\	ミルクキャラメル
\\	ヘルメット、かぶと
\\	の表愛形:カーシャ、粥
\\	面、仮面
\\	の表愛形:ネズミ
\\	ネズミ
\\	機械仕掛けの、ぜんまい仕掛けの
\\	取り巻く、取り囲む 迂回用の
\\	胸の 授乳期の
\\	豚 雌豚
\\	天賦の才、能力 進物、贈物、寄付
\\	イブニングドレス
\\	早口 早口言葉
\\	きれいなお名前ですね。
\\	彼はよく考えずに行動する。 (直訳:彼の頭の中には皇帝がいない)
\\	君のドレスの山
\\	主格.	私の名前は…です。
\\	彼はどんな仕打ちをされても全く平気でいる。 (直訳:彼からはガチョウのように水が落ちている)
\\	何のために? なぜ?(目的を問う)
\\	ありがとう、とても嬉しいです。
\\	彼は気が変わりやすい。 (直訳:彼の一週間には7つの金曜日がある)
\\	埃の、埃だらけの、埃っぽい 埃を通して見るような、ぼんやりと霞んだ
\\	[表愛]
\\	太陽 木漏れ日、漏れ入る日差し
\\	(口)ヒビの入った、割れ目のできた (声が)潰れた、ガラガラの
\\	忘れられた、記憶にない、廃れた 置き忘れた
\\	イグナーチー(男性の名前)
\\	社会的地位の高い、高貴な身分の
\\	(廃)思う、考える(不完了体)
\\	彼は1人ぼっちです。
\\	ある女性
\\	同じドレス
\\	言葉ばかり
\\	日付	〜から
\\	私の知ったことではない。 (直訳:私の家は外れにある)
\\	もう遅いです。 時間切れだよ。 もう遅い時間だから、そろそろ失礼します。
\\	21分
\\	22分
\\	25分
\\	5時59分
\\	彼らは2滴の水のように似ている。
\\	まだ早いです。 (遊びに来た相手が帰りそうな時などに)
\\	何の手応えもなく張り合いがない。 (直訳:エンドウ豆が壁から跳ね返るように)
\\	の小指形:壁 (箱・樽などの種々の容器の)内壁、(内蔵の)壁
\\	[集合]エンドウ豆
\\	(粗)密かに、こっそりと
\\	神経 (複)苛立ち
\\	缶詰、瓶詰 (埃・光などを防ぐ)保護メガネ、ゴーグル
\\	(口)苛立ち
\\	(ウクライナ・ベラルーシ・南ロシアなどの)百姓家、農家 (隠語で)居室、住宅
\\	小川
\\	の小形形・表愛形:カブ
\\	ハンチング、鳥打帽
\\	あなたは私を覚えていますか?
\\	彼女は色っぽい目つきをする。 (直訳:彼女は眼を作る)
\\	私は分かりません。
\\	彼は何かを逃して残念がっている。 (直訳:彼は肘を噛んでいる)
\\	苦笑い
\\	お手伝いありがとうございます。
\\	彼は口から出任せを言う。 (直訳:彼は天井から取る)
\\	天井
\\	(愛撫・キスなどによる)愛情表現、愛撫、優しさ 厚意、親切
\\	踊ること 踊り、踊りの曲・歌
\\	広葉樹(主に菩提樹)の若木の靭皮、同上の一片・一筋(縄・かご・むしろなどを作るのに用いる)
\\	マンホール(甲板の)艙口、ハッチ
\\	クローン/傾斜、勾配
\\	カエデ
\\	歯医者
\\	恐怖させる、怯えさせる、心配させる、不安を与える(不完了体)
\\	試しの、試験用の、見本の
\\	悪意のある、害心のある、敵意も露わな、意地悪そうな 
\\	滑稽な、おかしい、面白い、興味深い 
\\	栄誉ある、光栄ある、栄えある、名声の高い、高名な
\\	の指小・表愛形:目 生格:
\\	複数形:
\\	与格	〜のおかげで
\\	私は君を愛しています。
\\	彼は小さいことで大騒ぎする。 (直訳:彼はハエからゾウを作る)
\\	お誘いありがとうございます。
\\	たいへんやっかいなことをしでかす。 (直訳:お粥を作る)
\\	名詞の対格	〜に歩いて行く
\\	今どこに向かっていますか?
\\	チャイコフスキー(人名)
\\	ドストエフスキー(人名)
\\	ムソルグスキー(人名)
\\	卑劣なことをする。 (直訳:豚を置く)
\\	密な、密生した、目の細かい テンポの速い、絶え間ない たびたびの、頻繁な、常連の 比較級:
\\	山の、山にある、山に住む、山に多い 岩石の
\\	リズミカルな、規則正しい、律動的な
\\	(口)ごみの、ごみため用の (口)ごみため用の
\\	(口)相棒、コンビの相手、二人で組んで働く相手
\\	撃鉄、撃針 (ピアノなどの)ハンマー
\\	(紙ばさみ用の)クリップ 留めること
\\	着色、彩色 いろどり、色合い
\\	(俗)(方)煮て作る、熱湯に浸して作る 煮出す、熱湯を注ぐ(完了体) →不完了体:
\\	大きな息、溜め息、嘆息
\\	大きく嘆息する、溜め息をつく、嘆息する(不完了体)
\\	値が上がる、高くなる(完了体) →不完了体:
\\	リズム、律動 (仕事などの)調子、リズム
\\	カルム、キャラウェイ
\\	(岸のそばの)浅瀬、州
\\	印をつけること しるし、記号、(書類などの)確認中 (学校での)点数評価、成績
\\	下へ置く (もう少し)加える、足す、付加する、添加する(完了体) →不完了体:
\\	天才、非凡な才能 天才の人 権化、化身
\\	(小さな)縁、枠
\\	8月1日
\\	私たちは誰を待っているの?
\\	少数派の都合には合わせない (直訳:七人は一人を待たず)
\\	彼は空約束をする (直訳:彼は風に言葉を投げている)
\\	私たちは遅れそうじゃありませんか?
\\	お互いの悪事をかばい合う (直訳:手が手を洗う)
\\	遅れてすみません。 遅刻を許してください。
\\	彼は軽薄だ (直訳:彼の頭の中には風が(吹いている))
\\	送付、配達
\\	甘いもの、菓子
\\	食べ放題、バイキング
\\	[集合]7人、7個
\\	南の、南側の、南方の、南部の
\\	カナダの
\\	地獄の 地獄さながらの、恐ろしい、嫌な
\\	隣人の、(口)隣に住んでいる、近所に住んでいる
\\	洗う、洗浄する(不完了体) 
\\	エストニア人
\\	物事にきちんと取り組む (直訳:牡牛の角をつかむ)
\\	ヘアサロン、美容院
\\	には
\\	が必要です。
\\	とてもおいしい!
\\	指を舐めるほど美味しい。
\\	ごちそうさまでした (直訳:ごちそうをありがとう)
\\	彼女は人を言いくるめている (直訳:彼女は脳にファンデーションを付けている)
\\	(書物の)巻、冊 (蔵書・図書館などの)個々の本
\\	(牛・羊などの)角 (火薬・たばこなどの)角製の容器、角の杯 角笛
\\	(突き抜けた)穴、孔、裂け目、割れ目、隙間 欠乏、不足、ブランク (口)片田舎、辺地
\\	芝生、芝地
\\	の指小・表愛形:(人間の手・足の)指
\\	舐め回す、舐めて綺麗にする (話し方・文体などに)磨きをかける、(文芸作品の)個性を失わせる(完了体)(一回) →不完了体:
\\	ご馳走すること、もてなし、奢り ごちそう(の品)
\\	の指小形:ねじ、ボルト 細部、小部分
\\	の指小形:(複)罰金遊び(くじで当たった人が何かを見せる遊びのこと) (口)(子供が遊びに使う)チョコレートの綺麗な包み紙
\\	(複)罰金遊び(くじで当たった人が何かをして見せる遊び) 罰金遊びをする人が引くくじ、罰金遊びでさせられる芸
\\	湿気、水分、水気 涙、唾 水面、水 水蒸気、水滴、露
\\	(平たい・携帯用)容器、飯ごう、水筒 (大型の取っ手付き)液体容器
\\	(くるぶしまでの)浅い靴、パンプス
\\	細首のガラス瓶、水差し
\\	絨毯、カーペット
\\	(ふつう複)戦利品 勝利・偉業などを記念する品、トロフィー
\\	…に白粉をつける、…にパウダーをふる(不完了体) →完了体:
\\	脳 頭脳、知恵、頭
\\	ご成功をお祈りして!
\\	ご多幸をお祈りして!
\\	命を賭けられるほど確信している (直訳:切り落とすために頭を差し出す)
\\	与格 
\\	動詞の不定形	〜できる、〜してもいい
\\	与格 
\\	動詞の不定形	〜してはいけない、〜できない
\\	(液体が)流れる、流れ出る (雲・天体などが)漂う、ゆっくり動く(不完了体) 
\\	知り合い、面識、知り合いになること 知己、知り合いの人々 知識、知識を持っていること
\\	芽、つぼみ
\\	塔、タワー、櫓 高層ビル、タワービル
\\	耕地、畑 (廃)耕作
\\	空腹でつらい生活をする (直訳:棚の上に歯を置く)
\\	与格	〜がうれしい
\\	お会いできて嬉しいです
\\	過去の過ちを責め立てる (直訳:傷口に塩を振りかける)
\\	間投詞:(トランプで)パス 名詞:(サッカーなどで)パス
\\	(口)(足や膝などで)蹴ること、足蹴
\\	学校用腰掛つき机
\\	双眼鏡
\\	切断、分断
\\	調和、仲の良さ、和合、親和、平和、親睦 様式、風、やり方、調子
\\	麦芽、麦麹 もやし
\\	錠、錠前 (首飾り・腕輪などの)留め金、鍵
\\	シャベル、スコップ
\\	穀粒、実、種子 [集合]穀類、穀物
\\	(廃)(公共施設の)集会場、ホール (住宅の)大広間、サロン、ダンス室
\\	穂
\\	飛脚、急使、使者
\\	の指小形:小山、丘 (食器用の)飾り棚、食器棚
\\	(粉状・粒状のものを)ぱらぱらと入れる、ふりかける、(木の葉などを)散らす、ばらまく (粉状・粒状のものを)投げつける、撒き散らす (細かな雨・雪が)降る、(涙・汗が)ぽろぽろと流れ落ちる(不完了体)
\\	(果実・花などの)房
\\	骨 (複)体、手足
\\	物をすくうような形にした手 (量について)一掬い、一握り
\\	時間・日付	〜までに
\\	私はどんな仕事でも引き受けます。
\\	彼はぼーっとしている (直訳:彼はカラスを数えている)
\\	給料はいくらもらっているの?
\\	彼はしつこく同じことを言う (直訳:彼は脳にしずくを垂らしている)
\\	無意味な交換をする 選択を誤る、誤算する (錐を石鹸と交換する)
\\	(口)ぱちん・かちゃり・ぱしっ・ぽんという短く鋭い音
\\	シェルパ族
\\	木片、こっぱ、木屑
\\	木片、こっぱ、木屑
\\	しゅうしゅう言う、(宥めたり・黙らせたりするために)しーっと言う(不完了体)
\\	抓る、つまむ ひりひり刺激する、痛みを与える(不完了体)
\\	(ツバメ・ヒワなどが)さえずる声
\\	(ツバメ・ヒワなどが)囀る (口)(女性・子供が)ぺちゃくちゃ喋る(不完了体)
\\	一滴ずつ注ぐ、垂らす、零す(不完のみ)滴る、雫となって落ちる (不完のみ)漏る、雨漏りがする(不完了体)  →完了体:
\\	→完了体・一回:
\\	気前の良さ、鷹揚さ
\\	三本指、ひとつまみ(の量)
\\	くすぐること くすぐったいこと、むずむずすること
\\	ルーブル(パリにある美術館)
\\	小川 (液体の)流れ
\\	手の、腕の 手動の、機械によらない
\\	弓の、アーチェリーの
\\	ブブリク(輪形のパン、大きくて柔らかめ)
\\	の指小形:マット、絨毯
\\	計器盤、信号パネル、スコアボード ※不変化
\\	錐
\\	不気味に、気味悪く、ひどく、恐ろしく
\\	[述]…が可哀想だ、…が哀れだ、…が気の毒だ (…がいなくなって・なくなって)悲しい、嘆かわしい、(…なのが)残念だ、遺憾だ
\\	ショール、肩掛け
\\	暑さ、高温、熱、熱気、暑気、暑い所
\\	さらさら鳴る、かすかにざわめく(不完了体)
\\	囁く、かすかな・さらさらいう音を立てる(不完了体) →完了体・一回:
\\	(水の流れが)音を立てる、さらさら・ちょろちょろと音を立てる(不完了体)
\\	8月31日 20以上の順序数詞を作る場合は、十の位は個数詞、一の位は順序数詞を使う
\\	話し声、人声 しゃべり方、話しぶり、発音 地域方言
\\	活動領域・範囲(与格)	〜に関する
\\	空間・場所(与格)	〜を通って、〜中
\\	通信手段(与格)	〜によって、〜で
\\	(歯・毛髪・羽などが)抜ける (雨・雪などが)降る 滑り落ちる、転げ落ちる(不完了体)  →完了体:
\\	鼓動を打つ、鼓動する 戦う、打ちあう、試合をする 痙攣する、震える、身を震わせる、もがく(不完了体)
\\	花(ふつう複は
\\	ただし一輪ずつの花や器官としての花は
\\	与格 
\\	形容詞・短語尾中性形	与格にとって…だ
\\	私には高くて手が出ない (直訳:私にはポケットの中にある金額では足りない)
\\	がっかりする (直訳:鼻を掛ける)
\\	私はどちらでもいいですよ どちらでも構わない(投げやりなニュアンスもある) (直訳:私にとってはすべて同じです)
\\	能力がまさる (直訳:鼻を拭う)
\\	私がおごります あなたに手料理をご馳走します
\\	彼はささいなことで腹を立てる (直訳:彼はビンの中に入り込んでいる)
\\	どうかしましたか? どうなさったのですか?
\\	どうしたの?
\\	小さな目的に大げさな手段を用いる (直訳:大砲で雀を撃つ)
\\	(固いものを砕くための)かなてこ、バール
\\	ライムギ、ライムギの穀粒(黒パンの原料)
\\	ヤナギ類 (ブドウその他の低木の)枝、つる 笞刑用の細い樹枝
\\	大砲、火砲、(俗)ピストル 号砲 (俗)うそ
\\	細首のガラス瓶、瓶一本の量 ブトゥイルカ(ロシアの古い液量単位、ワインで0.77
\\	、ビール・ウォッカで0.615
\\	(ふつう複)(肉の詰め物が表面に出ている)丸いピロシキ
\\	吹き出物、にきび
\\	パンチ(ラム酒・ウイスキー・コニャックなどに果汁・砂糖などを入れた)飲み物
\\	キヅタ
\\	プラッシュ、フラシ天(ビロードよりも手足の長い絹布、木綿布)
\\	アセリナ(ひれに大きなトゲを持つパーチ科の淡水魚)
\\	ブリーム(コイ科)
\\	ひき肉 (料理で)細かく刻んだ詰め物
\\	軟骨/(廃)砂礫、粗粒砂、砂利
\\	柄杓、杓子
\\	(ユダヤ教の)安息日(土曜日) (中世の迷信で)魔女の夜会、乱痴気騒ぎ/(廃)(俗)仕事の終わり、(安息のための)中休み (漕ぎ手に対する号令で)漕ぎ方やめ
\\	(口)無秩序、混乱、乱雑
\\	かける、吊るす(不完了体) →完了体:
\\	(口)嘘をつく、でたらめを言う (口)貶す、腐す (口)くだらないおしゃべりをする、無駄口を叩く(不完了体)
\\	(国家・政党などの)連合、ブロック ブロック、ユニット、組み立て部品/滑車、プーリ
\\	結婚、婚姻/不合格品、傷物、(製品・商品の)欠点、疵
\\	基礎、基盤 基地 倉庫、貯蔵所
\\	拭う ふく(完了体) →不完了体:
\\	体力があれば、知恵はいらない (体力はすごいが頭はさっぱりだ)
\\	残念ですが、できません (相手の要望に応えられない時に使う、時間がない時や面倒な時にも)
\\	彼は大げさに物を言う (直訳:彼は絵の具を濃くする)
\\	造格	〜のように見える
\\	前置詞 
\\	数字	〜歳に見える
\\	おめでとう!
\\	彼は怠けて暮らしている (直訳:彼は天井につばを吐く)
\\	造格	〜と見なされる
\\	記念日おめでとう!
\\	彼は仕返しの機会をうかがっている (直訳:彼は懐に石を入れている)
\\	(方)(草原・森林の)火事 (方)(草原・森林の)焼跡 (方)野焼、畑焼
\\	脂肪、(料理用の)油
\\	杭 (5点満点の)1点、最低点
\\	非難、責め、叱責 恥辱、不面目
\\	合唱団、唱歌隊、合唱曲、合唱
\\	縒り紐、細縄 (電気の)コード 導火線
\\	(完了体)	(口)1つずつ・数回で全部を移動させる (母親が胎児を)月満ちるまで胎内に宿す、月満ちて産む、(思想・構想などを)熟するまで抱く、練る 愛育する、育て上げる(完了体) →不完了体:
\\	ごみ、埃、がらくた、くず(もの) (口)(一般に)無益・無用なもの
\\	(ロシアの伝統的な丸太作りの)家、百姓家 (モスクワ公国時代の)事務所、役所、官庁
\\	ひたすら頼む、懇願する(不完了体)
\\	駆除剤で殺す 苛む、へとへとにさせる 弱らせる、ふらふらにさせる(不完了体)
\\	(蜂などが)刺す、(口)(蛇などが)噛む (棘のある植物が)刺す(不完了体) →完了体:
\\	手探りする、(光が)探る (口)捜す、探索する、(盗む目的で)探る(不完了体)
\\	塩を入れる、塩をかける、塩味をつける 塩漬けにする、塩でしめる(不完了体)
\\	(ごみなどを)投げ捨てる、(ごみなどで)汚す (不完のみ)(口)浪費する(不完了体) →完了体:
\\	白くする、白く塗る 白粉をつける(不完了体)
\\	そる (不完了体のみ)当意即妙にうまく的確に答える(不完了体) →完了体:
\\	はかる、(大きさ・規模を)推し量る、推量する 歩き回る、行ったり来たりする、旅する (寸法に合うかどうか)試着する、履いてみる、かぶってみる(不完了体)
\\	濃縮する、(濃度を)濃くする 集める、集中させる、密集させる (音・声を)より低く響かせる(不完了体) →完了体:
\\	絵の具、塗料、染料、顔料、ペンキ (ふつう複)色彩、色調、色合、トーン
\\	(廃)(俗)蛇、大蛇 (おとぎ話・神話の)邪身有翼の怪物、飛竜(凶悪・凶暴の象徴とされた)
\\	物置、納屋、(口)大きな住み心地の悪い家
\\	侍僕、従僕、給仕
\\	浅瀬、砂洲
\\	助詞:[挿入語](俗)(他人の言葉を引く際に用いて)…ということだ、…だそうだ、(自分が前に言った言葉を引く際に用いて)…だと 名詞:防波堤
\\	石炭、木炭 (原材料としての)炭、カーボン
\\	トウヒ、エゾマツ
\\	懐、懐中 (口)くぼみ、凹所
\\	動詞の不定形	〜したい
\\	彼は意図的に邪魔をする (直訳:彼は車輪に棒を突っ込んでいる)
\\	(造格) 
\\	(主格)	
\\	の下に
\\	がある
\\	造格	〜の前に
\\	造格	~の後ろで、〜の後ろに ~を求めて ~の向こうで(場所・目的を示す)
\\	あけましておめでとう! 新年おめでとう!
\\	彼は一生懸命に頑張っている (直訳:彼は鼻で土を掘っている)
\\	ご健康をお祈りします!
\\	彼は頭の中が混乱している (直訳:彼の頭の中にはお粥がある)
\\	ちょっとお時間ありますか? (誰かを呼び止めたい時に)
\\	生活のやりくりに苦しむ (直訳:魚のように氷に身を打ち付ける)
\\	圧力鍋、圧力釜
\\	(口)フライパン
\\	(古文書の)草書 (廃)速記
\\	はめる、入れる 挿入する、書き入れる、口を挟む(不完了体) →完了体:
\\	(犬・蛇などが)咬む、(虫が)刺す (口)咬み切る、噛み取る(完了体)
\\	蚊
\\	ツツイカ類(ヤリイカ、スルメイカなど)
\\	マカロニ
\\	掘る 掘り返す、ほじくる (地中の物を)掘り出す(不完了体)
\\	ペリメニ(水餃子に似た食べ物)
\\	[表愛](口)
\\	おばあちゃん
\\	首飾り、ネックレス、ビーズ飾り
\\	(郵便その他で送られる)小包 送ること、送達、派遣、向けること
\\	財政、金融 財務、会計
\\	(ゴム底・ズックの)運動靴、バスケットシューズ、スニーカー (一足は複数形で)
\\	スズラン スズラン香水
\\	(弾丸・球などが)当たる、届く、入る、(たまたま跳ねたり零れたりして)入る、落ちる 
\\	当てる、ぶつける、入れる、突っ込む 
\\	の形で
\\	などの代名詞、
\\	などの副詞の後に置かれて)…でも構わずに(完了体) 
\\	過:
\\	→不完了体:
\\	前置詞:…に向かい合って、…の向かいに、…に面して   (風・流れ・光などに)逆らって、向かって   …に反して、…に背いて   …に対抗して、…に反抗して、…に反対して 副詞:反対して
\\	(不完了体)	
\\	持ち出す、運び出す、持ち込む 別にする、脇へ移す 提出する(不完了体) 
\\	→完了体:
\\	造格	〜を使う
\\	造格	〜を誇りに思う
\\	とんちんかんなことを言う (直訳:空に指を差し込む)
\\	(クシャミをした男友達に)お大事に! (直訳:元気でいてね!)
\\	(クシャミをした女友達に)お大事に! (直訳:元気でいてね!)
\\	彼の言いなりになる (直訳:彼の笛につられて踊る)
\\	お大事に!
\\	お大事に!(親しい人に対して)
\\	ひどい目に遭わせる (直訳:ザリガニが越冬している場所を見せる)
\\	大丈夫ですか?(困っている人に対して) 返答としては
\\	(ありがとう、大丈夫です)
\\	すべてがうまくいった (直訳:仕事は帽子の中にある)
\\	ありがとう、大丈夫です 
\\	(大丈夫ですか?)に対しての返答
\\	お休みなさい!
\\	机の上に
\\	机の中に
\\	年老いた時に、老後
\\	私は構いません(賛成の意思を示す)
\\	私は反対です
\\	実物を見ずに買う (直訳:袋に入った猫を買う)
\\	舞踏会で
\\	地獄で
\\	彼は人を苛立たせる (直訳:彼は神経を演奏している)
\\	喧嘩しないでください!
\\	喧嘩しないで!(親しい人に言う場合)
\\	彼には特別な才能がない (直訳:彼は空の星をつかめない)
\\	服装・前置格	〜を着ている
\\	欲張らないでください!
\\	欲張らないで!(親しい人に言う場合)
\\	彼は音痴だ (直訳:熊が彼の耳を踏んだ)
\\	妬まないでください!
\\	妬まないで!(親しい人に言う場合)
\\	彼は体ばかり大きくて邪魔になる (直訳:彼は食器屋さんの中にいるゾウのようだ)
\\	名詞・対格	〜が欲しい
\\	(のこぎりで)ひく (口)(非難)うるさく小言をいって悩ます(不完了体)
\\	(木の)株、切り株 (俗)(罵)でくの坊
\\	(船の)舵を回す、舵を切る、(自動車などを)運転する、ハンドルを握る(不完了体)
\\	(炙って毛や羽毛を)焼く、焼きとる (口)(照明・暖房などのために)燃やす (太陽が)焼けるように照る、焼く(不完了体)
\\	とげ・針のある、ちくちくする、(寒さ・風が)刺すような、身を切るような 辛辣な、皮肉な、毒々しい
\\	(民族舞踊を)踊る (不完のみ)
\\	の形で)ぴょんぴょん跳びはねる、震える、ぴくぴく動く(不完了体) →完了体:
\\	笛、笛の音
\\	コクマルガラス
\\	(エチル)アルコール、酒精 (口)ウォッカ、アルコール飲料
\\	女占い師、女易者
\\	ゴールキーパー
\\	(獣を捕らえる)罠、詭計
\\	ブルカ(ふつう丸型、あるいは楕円形の小さい白パン)
\\	越冬する、冬ごもりする、冬眠する(不完了体)
\\	ピローグ(ロシア風パイ) 利得の源泉、甘い汁
\\	犬、雄犬生:
\\	ペンギン
\\	工兵
\\	写し、模写、複写、コピー
\\	(つばのある)帽子 ぐうたら、無気力な人、無精者
\\	場所	~の中に ~の上に ~で、~に
\\	(俗)うっかりして見逃す・見落とす(完了体) →不完了体:
\\	子ガラス生:
\\	複:
\\	襟、カラー
\\	通行証 通すこと、入れること、見落とし、欠席、(好機・期限などを)逃すこと
\\	(道・川の)カーブ、曲がり角 回転、方向転換、旋回、回すこと 転換、変転、転向
\\	取引所
\\	(高燥地の)針葉樹林、松林
\\	合札、預かり札 木札(量目・送り先などを記して商品や荷車に付ける荷札)
\\	ステッキ、杖 (管楽器の)リード
\\	ツグミ
\\	(手当たり次第に)取る、(口に)くわえる、(俗)(素早く)つかむ 届く、達する (口)捕らえる、逮捕する(不完了体) →完了体:
\\	(口)欲張る、がっつく、けちる (廃)切望する、渇望する(不完了体)
\\	花壇、花園 (女性や子供の)美しい群れ (昔の)説教集、教訓集
\\	色のついた、着色された、色刷りの、色とりどりの、色彩豊かな 天然色の、カラーの
\\	花序
\\	雄花 役立たず、あだ花
\\	(光に輝く)宝石、天然石
\\	名詞:脱走者、逃亡者形容詞:脱走した、逃亡した   ゆらゆらする、絶えず動く   すばやい、一瞬の   よどみない、流暢な
\\	(野球の)バット
\\	大型ボート (港内用の)ランチ
\\	食器の、容器の
\\	[集合]ビーズ (廃)(聖書で)真珠
\\	くよくよしないでください!
\\	くよくよしないで!(親しい相手に言う場合)
\\	私はヒバリ (朝方の人の例え)
\\	移動の手段・前置格	〜に乗って、〜で
\\	頑張ってください!
\\	頑張って!(親しい人に言う場合)
\\	意気消沈する、気を落とす(不完了体)  →完了体:(廃)
\\	(パンなどの)生地、生パン、ドー (セメント・石灰の)練ったもの
\\	謎、得体のしれない物 判じ物、判じ絵(単語や句が絵や記号で表されているもの)
\\	ヒバリ 朝型の人
\\	[指小形・表愛形]
\\	マヒワ 棒打ち遊び(両端のとがった短い棒を長い棒で輪の中へ打ち込む子供の遊び)
\\	マヒワ
\\	(ふつう取っ手のついた陶磁器の)茶碗、カップ 皿状のもの
\\	日本料理
\end{CJK}
\end{document}