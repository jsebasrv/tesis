\documentclass[8pt]{extreport} 
\usepackage{hyperref}
\usepackage{CJKutf8}
\begin{document}
\begin{CJK}{UTF8}{min}
\\	伊能忠敬
\\	石原純
\\	【テキスト中に現れる記号について】
\\	ルビ
\\	(例)殆《ほと》んど
\\	ルビの付く文字列の始まりを特定する記号
\\	(例)高橋作左衞門|至時《よしとき》
\\	入力者注 主に外字の説明や、傍点の位置の指定
\\	(例)[#7字下げ]
\\	[#7字下げ]地図の作製[#「地図の作製」は中見出し]
\\	どこの国でも、その国の全体の有様を知るのには、地図がつくられていなければなりませんが、正しい地図をつくるのには、すべての場処に出かけて行って土地の測量を正確に行わなければならないのは、言うまでもありません。ところが、我が国においてそのような正確な土地の測量は、昔は殆《ほと》んど行われていなかったので、従って正しい地図もまるでなかったのでした。それと云《い》うのも、このような測量をするのにはいろいろの精密な器械も必要でありましたし、また土地測量の基準として星の位置を正しく観測することも必要であったからです。そこで、このような仕事が、我が国では最初に誰によってなされたのかと云《い》いますと、それはここにお話ししようとする伊能忠敬《いのうただたか》に依るのでありまして、しかもその測量は日本全国に及んでいるのですから、実に驚くべき事がらでもあるのです。それは今から百数十年も前のことでありますし、その時代にはどこへ旅をするのにも、すべて自分で足を運ばなくてはならなかったので、全国の地図を完成するのにも、二十年に近い歳月を費さなくてはならなかったのでした。そのようなことを思うと、この大きな仕事を自分一人でなし遂げた伊能忠敬の功績はまことにすばらしいものであったと云《い》わなければなりますまい。そのほかに、ちょうどこの時代にはわが国の北辺がようやく騒がしくなり始め、それに伴れて林子平《はやししへい》の『海国兵談』なども出て、国防の問題もいろいろ議論せられるようになっていましたので、それにつけても正確な地図が必要とされたに違いないのですから、この点から見ても忠敬の仕事は大きな意味をもっていたと云《い》わなければならないのでしょう。
\\	ところで、忠敬がどのようにしてこの土地測量の仕事を始めるようになったかと云《い》うことについても、ともかくも古い昔の時代であっただけに、特別な決心が必要であったのに違いないので、それらの事がらについて、次に少しくお話しして見たいと思います。
\\	[#7字下げ]忠敬の前半生[#「忠敬の前半生」は中見出し]
\\	伊能忠敬は、幼名を三治郎、後に佐忠太と云《い》いましたが、成人して通称三郎右衞門と称し、字は子齊、東河と号し、晩年には勘解由《かげゆ》とも称しました。上総国《かずさのくに》山武郡《さんぶぐん》[#「山武郡」はママ]小関村《こぜきむら》で延享二年一月十一日に神保利左衞門貞恒の第三男として生まれたのでした。もっともこの時に父は小関村の小關家を継いでいたのでしたが、忠敬が七歳のときに妻の死歿に遭い神保家に戻りましたので、それでも、忠敬は幼かったのでその儘《まま》小關家に留まり、十一歳になってようやく父の許に帰ったと云《い》うことです。ですから、忠敬の幼時は言わば不遇の境地に置かれていたのでしたが、その頃から学問を好んでいたということは、後に自分で記している処によっても確かであったのでした。しかしそれでもなかなかその方に向うことなどは思いもよらない処であったので、十八歳になった際には、下総《しもうさ》佐原町《さわらまち》の伊能家に婿養子に遣られ、その時忠敬と名のることとなったのでした。ところで伊能家は元来は佐原町《さわらまち》の豪家であったのでしたが、この頃家運が甚だ衰えていましたので、忠敬はそこへ赴くと共に、まず家運を恢復《かいふく》することに全力を尽さなくてはならなかったのです。それでこの時から実に三十年の長い間、この事に熱心に従い、産業の発展に努めたのでした。この産業という中には、米穀を豊作の土池から買って来て、それを他に売りさばくことや、また醸造《じょうぞう》や薪問屋の営業などもあったと云《い》うことです。ともかくそのようにして忠敬の一生懸命の努力のおかげで家運も再び盛んになることができたので、それに伴れて忠敬は救民の事業などをも興したので、終《つい》には尊敬されて名主ともなり、また幕府からも大いに賞《ほ》められて、苗字《みょうじ》、佩刀《はいとう》をも許されました。この事は忠敬が自分の仕事に対していつも忠実にはたらく人物であることを既に十分に示しているのであります。
\\	ところが、この間に忠敬は妻の死歿に二度も遭っていたと云《い》うので、彼の前半生は決して幸福とは云《い》われなかったのでしたが、それでも自分の仕事に屈することなく励んで来たので、ようやく家運も盛んになったのでした。そこで彼の年齢も五十歳に達して隠居が許されるようになると、さっそくに家督を長子景敬に譲り、自分は江戸に出て、かねてから望んでいた学問の道を修めようと決心したのでした。これはその頃としてもまことに特別な心がけで、忠敬のような人物でなければとても出来なかったところであると思われるのです。
\\	[#7字下げ]忠敬の学問修業[#「忠敬の学問修業」は中見出し]
\\	忠敬が隠居したのは寛政六年のことでありましたが、翌七年の五月には江戸に出て、深川の黒江町に居住し、それから学問を修めようとしたのでした。ところが、ちょうどこの時に彼は幸運にめぐまれました。それはこの年の三月に幕府が暦法改正の仕事を始めるために大阪から暦学天文の大家として知られている高橋作左衞門|至時《よしとき》、ならびに間《はざま》五郎兵衞|重富《しげとみ》を江戸に呼びよせたことで、高橋は四月に、間は六月に江戸に到着したからです。この高橋と間とは共に大阪で名高かった麻田剛立の門弟であって、既に十分の実力を具《そな》えていたのでしたが、若《も》しそのまま大阪に居住していたとしたならば、忠敬もたやすくその教えを乞うことはできなかったに違いないのでした。ところが、この両人が忠敬の江戸に出るのと時を同じうして江戸に来合わせたということは、忠敬にとってまことに得難い奇遇であったと云《い》わなければなりません。ともかくも忠敬はこの事を聞いて大いに喜び、さっそくに高橋作左衞門の許《もと》を訪ずれて、鄭重に入門を請いました。そして測量、地理、暦術を熱心に学びました。この時、忠敬は五十一歳であったのに対し、師の高橋は三十二歳であったのですが、忠敬は高橋を師とあがめて、いろいろな知識や技術を学んだと云《い》うことを思うと、これも実に一つの美談であると云《い》わなければなりますまい。
\\	高橋作左衛門はその頃暦学では他に並ぶものがないと云《い》われたほどの人で、寛政丁巳暦と称せられたのは彼と間重富《はざましげとみ》との方寸《ほうすん》によって成り立ったものであったのでしたが、それだけに門弟に対してもなかなかに厳しく教えたということで、それがしかし忠敬には却って幸いであったのでした。忠敬は暦学天文と共に、それを利用して行う土地測量の方法をも熱心に研究しました。土地を測量するのには、或《あ》る位置に機械を据えつけて、それで目標の観測を行わなくてはならないのですが、それぞれの土地には傾斜があったり凹凸があるのですから、実際にはいろいろの苦心が要るのです。それで方位を測る器械や、傾斜を測る器械などを工夫して、これを行わなければなりません。それはともかくも西洋で行われている方法を詳しくしらべて、それに依るのがよいと考えて、そこでいろいろな測量の器械をつくって見ました。そのなかには、ものさし(尺度)、間棹《けんざお》、間縄《けんなわ》、量程車《りょうていしゃ》、羅鍼《らしん》、方位盤、象限儀《しょうげんぎ》、時計、測量定分儀[#「測量定分儀」はママ]、圭表儀《けいひょうぎ》、望遠鏡などがありました。ここではこれらの器械について一々説明しているわけにもゆきませんが、これらに対して忠敬はこまかい注意を加えてできるだけ精密な測量をめざしたのでした。これらの器械のことについては、後に忠敬の門弟の渡邊愼という人が書きのこした「伊能東河先生量地伝習録」という書物にかなり詳しく記されているのですが、それを読んで見ても、忠敬がいかにこれについて苦心を重ねたかがはっきりとわかるのです。
\\	その一つの例をとり出して見ますと、これらの器械のうちで最も簡単なものさし[#「ものさし」に傍点]にしましても、その頃我が国ではこれが精密には定まっていなかったのでした。まず比較的に広く行われていた物さしとしては、享保尺《きょうほうじゃく》というのと、又四郎尺《またしろうじゃく》というのとありましたが、それらも幾らか長さのちがいがありました。そこで忠敬はこの二つの物さしの平均をとって新しい尺度を定め、これを折衷尺《せっちゅうじゃく》と名づけ、これを測量の土台にしたのでした。後に明治の時代になって度量衡法《どりょうこうほう》を定める場合に、やはりこの忠敬の折衷尺《せっちゅうじゃく》を基として、一メートルが三尺三寸に当ると定められたのですが、ともかく測量を正しく行うのには物さしの寸法をはっきりと定めておかなくてはならないのですから、それを最初に行う人の苦心はこのような処にもあったのでした。忠敬はこの物さしを使って後に地球の緯度の一度が二十八里二分に当るという結果を出しているのですが、これは現在の測定に比べて見ても僅かに千分の二ほどしか異っていないということで、忠敬の測量がその時代としていかに精密なものであったかが、この一事でも知られるのであります。
\\	[#7字下げ]日本全国の測量[#「日本全国の測量」は中見出し]
\\	前にも述べたように、ちょうどこの頃我が国の沿海にロシヤの艦船などが出没し、ようやく騒がしくなって来ましたので、寛政十二年になると、幕府が忠敬に命じてまず蝦夷《えぞ》の測量を行わせることになりました。この頃の蝦夷《えぞ》と云《い》えば、まだまるで拓《ひら》けてもいなかったので、その地を旅するだけでもなかなかの難事であったのでしたが、忠敬は既に五十六歳にもなる身で殆《ほとん》ど一年間を費してその土地測量を行い、その年の十二月に蝦夷《えぞ》の地図をつくり上げたということです。この蝦夷《えぞ》の地で、忠敬は間宮倫宗に出遇い、それから倫宗と親しく交友したのでした。
\\	蝦夷《えぞ》の測量を終ってから、忠敬は更に日本全国の測量を志し、それから実に十八年の長い間到るところに旅してこの大きな仕事を果したというのは、まことに驚くべきことであると云《い》わなければなりますまい。その間に文化元年には尾張、越前より東に当る地図を完成し、同四年にはその後の測量にかかる地図をつくり、文化六年に大体において日本輿地全図《にほんよちぜんず》をつくり上げました。この中には全国の大図、中図、小図の三種類のものがありましたが、それらは夫々《それぞれ》三万六千分の一、二十一万六千分の一、四十三万二千分の一の大いさに相当するものです。何《いず》れにしてもこれだけのものを、僅かに幾たりかの門弟と共に完全につくり上げた功績はまことにすばらしいことであると云《い》わなければなりますまい。
\\	忠敬はともかくもこのようにして自分の志した大きな事業を成し遂げた上で、文政元年の四月十三日に江戸八丁堀亀島町の邸で歿しました。その際には、特に遺言して、自分がこのように日本全国を測量するという大きな仕事をなし遂げることのできたのも、全く高橋作左衞門師のおかげであったのであるから、その恩を深く謝するためにせめてその墓側に葬ってくれと云《い》ったとのことです。高橋|至時《よしとき》は既にそれ以前の文化元年に歿くなって、浅草の源空寺に葬られていましたので、忠敬の遺骸もこの遺言に従ってその墓側に葬られました。しかしこの時には、その日本輿地全図《にほんよちぜんず》と、ならびにそれに附隨《ふずい》している輿地実測録《よちじっそくろく》とがまだ完全に出来上っていなかったので、その完成を見るまでは忠敬の喪を公けに発表しないでおいたと云《い》うことで、これらが出来上った後に、文政四年の九月四日に喪を発したのでした。
\\	忠敬の著した書物としては、「国郡昼夜時刻対数表」、「記源術並びに用法」、「求割円八線表」、「割円八線表源法」、「地球測遠術問答」、「仏国暦衆編斥妄」などというのがあります。この外に「測量日記」二十八冊、「大日本沿海実測録」十四冊などがあり、これらはその測量の実際を知る上に、特に重要なものであります。下総《しもうさ》の佐原町《さわらまち》には、忠敬の旧宅が今でも残っていて、これらの書物や、測量に使った器械道具なども保存されているので、これはまことに貴重な記念物であります。
\\	忠敬のすばらしい功績については、今日一般によく認められているのですが、明治十六年にはそれをよみして正四位を追贈せられましたし、また明治二十二年には東京地学協会で芝公園の円山に記念碑を立て、それには「贈正四位伊能忠敬先生遺功碑」としるしてあります。またその後、帝国学士院では、大谷亮吉氏に依嘱《いしょく》して、忠敬の事蹟《じせき》を詳しく調査し、これが「伊能忠敬」と題する一書となって刊行されています。このようにして忠敬の遺した仕事はいつまでも大きな意味をもって記憶されてゆくことを考えますと、夙《はや》く学問の道に志した彼もまた安んじて瞑《めい》するに足りるのでありましょう。
\\	底本:「偉い科學者」實業之日本社
\\	1942(昭和17)年10月10日発行
\\	※「旧字、旧仮名で書かれた作品を、現代表記にあらためる際の作業指針」に基づいて、底本の表記をあらためました。
\\	「於て」は「おいて」に、「漸く」は「ようやく」に、「之」は「これ」に、「之等」は「これら」に、「併し」は「しかし」に、「先づ」は「まず」に、「早速に」は「さっそくに」に、「並びに」は「ならびに」に、置き換えました。
\\	※読みにくい言葉、読み誤りやすい言葉に振り仮名を付しました。底本には振り仮名が付されていません。
\\	※「殆《ほと》んど」と「殆《ほとん》ど」、「器械」と「機械」の混在は、底本通りです。
\\	※国立国会図書館デジタルコレクション
\\	で公開されている当該書籍画像に基づいて、作業しました。
\\	※表題は底本では、「伊能忠敬《いのうただたか》」となっています。
\\	入力:高瀬竜一
\\	校正:
\\	2018年12月24日作成
\\	青空文庫作成ファイル:
\\	このファイルは、インターネットの図書館、青空文庫
\\	で作られました。入力、校正、制作にあたったのは、ボランティアの皆さんです。
\end{CJK}
\end{document}