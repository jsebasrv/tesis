\documentclass[8pt]{extreport} 
\usepackage{hyperref}
\usepackage{CJKutf8}
\begin{document}
\begin{CJK}{UTF8}{min}
\\	ジェームズ・ワット
\\	石原純
\\	【テキスト中に現れる記号について】
\\	ルビ
\\	(例)云《い》っても
\\	ルビの付く文字列の始まりを特定する記号
\\	(例)その後|諸所《しょしょ》
\\	入力者注 主に外字の説明や、傍点の位置の指定
\\	(例)[#7字下げ]
\\	[#7字下げ]機械の発明[#「機械の発明」は中見出し]
\\	科学が昔からどんな風に進んで、またその影響をどう社会に及ぼして来たかという歴史をよく見てゆくと、なかなかおもしろいのです。科学は学問として絶えず進んでゆくのですが、それに伴れていろいろな便利な機械が発明されて来るので、そうすれば世のなかの人々はどうしてもそういう機械を利用しないではいられなくなります。すると機械の発達に従ってすべての物の生産や、また交通の有様などがちがって来ますから、つまりはそれが人間の社会の有様を変えてゆくことになるので、このようにして昔から今日に至るまで文化が発達して来たのだと云《い》ってもよいのでしょう。このように考えてゆくと、人間の社会にとって科学がどれほど重要なものであるかと云《い》うことも、始めて本当にわかって来るのでしょう。
\\	さて科学の発達につれて社会の有様がいちじるしく変ったのは、いつ頃からであるかと云《い》いますと、それは勿論《もちろん》古い時代にもいくらかずつはあったに違いありませんが、歴史の上で最も目立っているのは、十八世紀の末から十九世紀の初めにかけてのことであります。それ以前にも機械の利用はいろいろあったのは確かですが、ちょうどこの頃に以前よりもずっと便利ないろいろの機械が発明されたので、それを盛んに産業の上で使うようになったからであります。そのなかには織物の紡績につかう機械やそのほかのものもありましたが、なかでもいちばん重要なのは蒸気のはたらきを利用する蒸気機関であって、これがここでお話ししようとするジェームズ・ワットによってその当時としてはいかにもすばらしいものに改良されたのでした。それ以前には水や風の力を利用する水車や風車が主な機械であって、その外には人間の手足の力とか、牛馬をつかうぐらいがせいぜいであったので、これでは大きな工業などが発達しなかったのも当然なのでありましたが、一度蒸気機関が現れると、これはいかにも便利なものでありましたから、忽《たちま》ちあらゆる方面に利用せられて、諸所《しょしょ》に大きな工場が建てられるようにもなりました。そうなると今までは多くの人々が主に家庭のなかで働いていたのに引きかえて、工場に雇われていろいろな生産労働に従事するようになりましたから、社会の全体の有様もそれに伴れて大いに変って来ました。それから蒸気機関を据えつけて、汽車や汽船が動き出すようになると、諸所《しょしょ》の交通もそれまでとは比べられないほど便利になり、これも社会の有様を変えてゆきました。世界の歴史の上では、この著しい変りかたを名づけて、産業革命と言っているのですが、ともかくもこれはそれ以前に見られなかった重要な変化なのでありました。ところがそれがワットのつくった蒸気機関のおかげであったことを思って見ますと、ワットが世界の歴史をこの方向に向けたのだと云《い》ってもよいほどに、それは重要な発明であったと見なければなりますまい。ですからすばらしい機械の発明というものは、それ自身のすぐれた価値のほかに、世界の歴史をも運命づけるというほどに重大な意味をもっていることが、これではっきりとわかるのでしょう。
\\	[#7字下げ]蒸気の利用[#「蒸気の利用」は中見出し]
\\	ワットの蒸気機関が蒸気の力を利用したものであることは、誰も知っている通りですが、蒸気が物を動かすだけの力をもっていると云《い》うことは非常に古くから知られていたのですし、それを利用しようと考えた人々もたくさんにあったのでした。それでも一つの立派な機械が発明されるまでには、いろいろな段階を経て、だんだんにそこに到達しなくてはならないので、蒸気機関のでき上って来た道筋などは、それのよい例を示していると云《い》ってよいのでしょう。
\\	蒸気を利用したものの中で最も古く知られているのは、西洋紀元前百数十年という頃にアレキサンドリアのヘロンという人のつくった一つの装置であります。この人はいろいろな機械を工夫したので名だかいのですが、蒸気をつかって球をぐるぐる廻《まわ》して見せたのでした。この球の内部は中空になっているので、その両側にそれぞれ管をつけておいて、水の沸騰する器のなかから出てくる蒸気が管を通って球に入りこむようにし、別にこの球には蒸気のふき出る短かい二つの出口をつけておくのです。そうすると蒸気がはげしい勢で球にはいって、それから出口から噴き出るのにつれて、球はぐるぐる廻《まわ》り出すのです。
\\	これは玩具のようなものでありますが、ともかく蒸気の力で球が廻《まわ》り出すので、そこで後になってこれを機械に利用しようという考えの起って来たのも当然であったのでしょう。もっともそれはようやく十六世紀から十七世紀にわたる頃になってからのことなのですが、イタリヤのブランカとか、フランスのド・コウというような人たちがそれについていろいろ工夫を凝らしたということです。それでもこれ等はまだ実際の役に立つまでにゆかなかったので、さらにもっとよい考えを必要としたのですが、それについて最初にいくらかの成功を果したのはフランスのドニ・パパンでした。パパンは一六四七年の生まれで、アンジェの大学で医学を修めた後にパリで学位を得たのでしたが、宗教の上で新教を信じていたのでその頃の世間から迫害を受け、イギリスのロンドンに赴いて学問を続けました。そして間もなく偉い学者を集めている王立協会の幹事にもなったのを見れば、その才能のすぐれていたこともわかるのです。この頃から彼は蒸気の利用を考えていたとの事ですが、一六八七年にはドイツのヘッセンの領主に招かれて、マールブルグの大学の数学の教授になりました。しかし彼の関心はやはり蒸気の問題にあったので、それについていろいろ実験を行った結果、次のようなものをつくり上げました。
\\	それは気筒の底に水を入れ、それを熱して蒸気としたので、そうすると筒の上の方からはめられたピストンがこの蒸気の力で持ち上げられますが、蒸気が冷えて再び凝結するとこのピストンは外からの大気の圧力でまた下って来るのです。
\\	今から見れば、これはごく幼稚な機械にちがいありませんが、しかし蒸気のはたらきを利用した最初のものと見なされる点で重要な意味をもっているのです。その頃鉱山の採掘が盛んに行われ出すにつれて、深く掘りさげた鉱坑のなかに水がたまって、その水を汲《く》み出すのにどこでも苦しんでいたので、パパンはこの機械を排水に使ったらよかろうと考えたのでしたが、これはまだうまく成功しませんでした。その後パパンは一七〇七年にロンドンへ戻りましたけれども、その頃の世間にさほど認められないで、一七一二年に不遇のうちに亡くなったのでした。
\\	ところで実際に鉱山の排水につかうことのできるような蒸気機関を始めてつくり出したのは、イギリスのトーマス・セーヴァリーという人でした。この人は一六五〇年頃に生まれ、長じてからは軍隊に入っていろいろな技術に熟達するようになったとのことです。蒸気機関をつくってその特許をとったのは一六九八年でありましたが、パパンのよりは構造もいくらか複雑になっていて、うまくつくられていました。それでこれをつかってかなり深い坑から水を汲《く》み出すことも出来、その頃の鉱山所有者たちに大いに喜ばれたと云《い》うことです。もっともセーヴァリーの最初の機関はいくらか不完全であったので、それを改良するのにニューコメンという人の力を借りたという話がありますが、それはどれほど確かであるかわかりません。
\\	それにしても、ともかくセーヴァリーのよりも一層完全なものをこのニューコメンがつくったことだけは確かなことでした。ニューコメンは一六六三年に生まれて一七二九年に亡くなった人ですが、鍜冶屋を商売にしていましたし、また生地がセーヴァリーの住んでいた場所とはさほど離れてもいなかったので、セーヴァリーの蒸気機関のことを知っていたのには違いないのです。それでさらにこれに工夫を加えて改良を施し、一層役に立つものにすることができたのでした。もっともそれには当時の王立協会の会員で、名だかい科学者であったロバート・フックという人がいろいろと助言をしてくれたという話も伝わっています。そうして一七〇五年にその機関がうまく出来上ったので、その後|諸所《しょしょ》の鉱山でそれが使われることとなったのでした。
\\	[#7字下げ]ワットの発明[#「ワットの発明」は中見出し]
\\	蒸気機関がこのようにしてだんだんに実用に供せられるようになりましたが、この頃のものはいかにも不経済のものでありました。なぜと云《い》えば気筒のなかの熱い蒸気を外から水を注ぎ込んでさましてしまうと云《い》うのですから、また次に蒸気を送るのにつけても一時冷たくしてしまうのはそれだけ熱をむだにすることになるからです。そのほかに蒸気の力の利用も十分にはされていないので、その強い力を直接につかう方法がなお考えられなくてはならなかったのでした。しかしこれ等の点の改良はなかなかむずかしかったので、その後いくらかずつの考案を行った人たちもあったのでしたが、結局ワットの出るまではうまく成功しなかったのでした。
\\	ジェームズ・ワットは一七三六年の一月十九日にイギリスのグリノックという小さな港町で生まれました。父親は大工でしたが、そのほかの商売をも同時に行っていたとのことです。ワットは幼い頃身体が弱かったのですが、それでも土地の学校を終えて、その後は父親の仕事場で働いていました。ところが、その時分からどことなく指先が器用で、それが職人たちの間でも評判になったと云《い》われています。つまりこの幼い頃からの彼の天分がそういう処《ところ》に現れていたのにちがいないのです。
\\	ところが一七五三年には母親が亡くなり、その上に父の商売がどうしてもうまくゆかなくなったので、その翌年ワットはグラスゴーにある母方の親戚のもとへ送られてしまいました。彼はそこで何とかして身を立てたいと思って、親戚を通じてグラスゴー大学のディック教授に頼みましたところが、教授は大いにこれに同情してロンドンの知人に紹介してくれました。その頃のワットの希望は数学器具の製造を行いたいことにあったので、いろいろ探しまわってようやくそういう場所を見つけ出して徒弟になりました。そして熱心にそれを学んで、普通には三、四年もかかることを一年足らずで覚えてしまうという有様でした。
\\	かくて一七五七年には再びグラスゴーに戻りましたが、そこでは大学のなかに店を開いて数学器具製造を行うことを許されたので、彼は大いに喜び、それに励みました。それでもその生計はかなりに困難であったらしいので、最初はずいぶん苦しんだようでしたが、数年後にはジョン・クレイグという人と共に共同営業をするようになり、漸次《ぜんじ》それも楽になったのでした。そして一七六三年にはトロンゲート街に自分だけの店を開くことさえできるようになりました。
\\	ワットはその後いろいろな仕事を行ったのですが、蒸気機関に興味をもつようになったのは、ちょうど一七六三年の頃にグラスゴー大学からニューコメンの蒸気機関の模型を修繕するように頼まれたことにあったのでした。つまりこの事が後のワットの発明を結果することにもなったのですから、まことに運命というものは不思議でもあります。
\\	ワットはそれ迄《まで》は蒸気のはたらきに就《つ》いていくらの知識ももっていなかったのでしたが、この模型をしらべてゆくうちにいろいろな事を知ったのでした。それでともかく模型を修理したものの、この機関にはいろいろの欠点があることもわかり、それを何とか改良したいと思い立ったのでした。最もいけないことは、上にも述べましたように気筒のなかの熱い蒸気に水を注ぎ込んで冷やしてしまうことですから、ワットはその蒸気をよその場所に導いて来て、そこで冷やすようにしたならよかろうと考え、気筒から別に管をつけて他の器につなぎ、その器を水で冷やすようにしました。これは今では凝結器と称《とな》えているものです。この方法で蒸気はよそで冷えることになりますから、気筒の温度は下らないですむことになりました。つまりそれだけ蒸気を出させるための燃料が節約されることになるので、これは実に蒸気機関にとって重要な意味をもっているのです。ワットはこの考案に対して一七六五年に特許を得ました。この外に、ワットの改良のもう一つの大切な事がらは、ピストンを蒸気の力で一方にだけ動かす代りに、両方に往復するように工夫したことです。それには、つまり蒸気をピストンの両側に交互に入れるようにすればよいので、そのようにして一度は蒸気がビストンの下側に入ってそれを上方に押し上げ、次には上側に入って下方に押しさげるようにするのです。おまけに、若《も》しそれが出来さえすれば、気筒を上下に立てておくことも必要でなくなり、これを横にしておいてピストンを左右に動かすこともでき、それに伴れて蒸気の力を十分に強くしてピストンの運動を活溌《かっぱつ》にさせることができるわけです。しかしこれを実際に行って、それに成功するまでには、さまざまの苦心を必要としたので、それでも一七六九年になってようやく満足なものが出来上り、その特許を取ることもできました。もちろんそれから後にもワットはいろいろな点で、その機関を改良したので、それはほとんどそのために一生涯を費したと云《い》ってもよいくらいでありました。
\\	ところがこれ等の発明に凝っていた一方では、それだけ生活にも困ることになったので、土地測量の仕事を行ったり、そのほかいろいろの事をもしなければならなかったのでした。それでもワットの熱心に感じ入った人たちが彼を助けてくれたので、そのおかげで成功したと云《い》ってもよいのでしょう。最初は硫酸製造をやり、又《また》炭坑の持主でもあったレーバックという人が協力してくれましたが、数年後にはこの人も財産をなくしてしまったので、それからはブールトンという人に頼って共同事業を行うことになりました。
\\	ワットはこの外にいろいろの特許をも得ましたし、それらによって名声が漸次《ぜんじ》に高まったので、晩年には幸福に過ごすことができました。また一七八四年にはエディンバラ王立協会の会員に選ばれ、その翌年にはロンドン王立協会の会員となり、さらに一八一四年にはフランスのアカデミーの外国会員にも選ばれて、多大の名声を一身に担いました。しかし老年になったので、一八〇〇年には隠退して故郷に帰り、静かな生活を送りましたが、一八一九年の八月二十五日に八十四歳の高齢で亡くなりました。この頃には既《すで》にこの蒸気機関を利用した汽船や汽車もつくられて世を驚かしていたので、ワットの死は一層すべての人々から惜しまれたことでもあったのでした。彼の遺骸はハンズワース教区にある教会の墓地に葬られましたが、一八二四年にはウェストミンスター寺院のなかにその彫像が建てられ、イギリスでの最大の栄誉をもになったのでした。そればかりでなく、最初にも述べたように、その蒸気機関のおかげで産業革命というような大きな変化を社会に与えたということは、実に驚くべきことで、それに比較される程の人はほとんど他にはないと云《い》ってもよいのでありましょう。これもワットのひたすら真摯な努力が実を結んだものに外ならないのです。
\\	底本:「偉い科學者」實業之日本社
\\	1942(昭和17)年10月10日発行
\\	※「旧字、旧仮名で書かれた作品を、現代表記にあらためる際の作業指針」に基づいて、底本の表記をあらためました。
\\	「漸く」は「ようやく」に、「更に」は「さらに」に、「併し」は「しかし」に、「殆んど」は「ほとんど」に、「ハンヅワース」は「ハンズワース」に、「ウエストミンスター」は「ウェストミンスター」に、置き換えました。
\\	※読みにくい言葉、読み誤りやすい言葉に振り仮名を付しました。底本には振り仮名が付されていません。
\\	※国立国会図書館デジタルコレクション
\\	で公開されている当該書籍画像に基づいて、作業しました。
\\	入力:高瀬竜一
\\	校正:
\\	2018年7月27日作成
\\	青空文庫作成ファイル:
\\	このファイルは、インターネットの図書館、青空文庫
\\	で作られました。入力、校正、制作にあたったのは、ボランティアの皆さんです。
\end{CJK}
\end{document}