\documentclass[8pt]{extreport} 
\usepackage{hyperref}
\usepackage{CJKutf8}
\begin{document}
\begin{CJK}{UTF8}{min}
\\	先生、お鞄を (持つ: 
\\	。	先生[せんせい]、お 鞄[かばん]をお持ちします。	お持ちします	お~する	
\\	東京では 
\\	家なんか買えません。	東京[とうきょう]ではとても 家[いえ]なんか 買[か]えません。	とても	とても~ない	
\\	多くのオーストラリア
\\	人 
\\	彼は気さくだ。	多[おお]くのアメリカ 人[じん]と 同様[どうよう]に 彼[かれ]は 気[き]さくだ。	と同様に	と同様に	
\\	いくら安く (ても 
\\	、好きじゃないものは買わない。	いくら 安[やす]くたって、 好[す]きじゃないものは 買[か]わない。	たって	たって	
\\	事故は起こるものだ。	どうしても 事故[じこ]は 起[お]こるものだ。	どうしても	どうしても~	
\\	会いに行ったのに、友達はいませんでした。	せっかく 会[あ]いに 行[い]ったのに、 友達[ともだち]はいませんでした。	せっかく	せっかく~	
\\	急いでいる時 
\\	バスが遅れる。	急[いそ]いでいる 時[とき]に 限[かぎ]ってバスが 遅[おく]れる。	に限って	に限って	
\\	人間は 
\\	死ぬんだから、あくせく働いても仕方がない。	人間[にんげん]はどうせ 死[し]ぬんだから、あくせく 働[はたら]いても 仕方[しかた]がない。	どうせ	どうせ~	
\\	両親の期待 
\\	女の子が生まれた。	両親[りょうしん]の 期待[きたい]に 反[はん]して 女の子[おんなのこ]が 生[う]まれた。	に反して	に反して	
\\	社長、奥様にお電話を (かけた: 
\\	社長[しゃちょう]、 奥様[おくさま]にお 電話[でんわ]をおかけになりましたか?	おかけになりましたか	お~になる	
\\	田中さんには (言う: 
\\	。	フレッドには 言[い]わないでおきましょう。	言わないでおきましょう。	ないでおく	
\\	アメリカの大統領は来日した (とき)、わたしたちの大学でスピーチを行った。	アメリカの大統領は来日した 際[さい]に、わたしたちの大学でスピーチを行った。	際に	際(に)	
\\	ずっと本を読んでいて急に立ち上がった 
\\	、めまいがしました。	ずっと 本[ほん]を 読[よ]んでいて 急[きゅう]に 立ち上[たちあ]がったとたん、めまいがしました。	とたん	たとたん(に)	
\\	電車のドアが (開く: 
\\	乗客が飛び出して来る	電車[でんしゃ]のドアが 開[ひら]くか 開[ひら]かないかのうちに 乗客[じょうきゃく]が 飛び出[とびだ]して 来[く]る	開くか開かないかのうちに	か~ないかのうちに	
\\	卒業 
\\	、私たちは感謝の印として彼に置き時計を贈りました。	卒業[そつぎょう]に 際[さい]して、 私[わたし]たちは 感謝[かんしゃ]の 印[しるし]として 彼[かれ]に 置き時計[おきどけい]を 贈[おく]りました。	に際して	に際して	
\\	雨が (降ってくる: 
\\	もうやんだ。	雨[あめ]が 降[ふ]ってきたかと 思[おも]うともうやんだ。	降ってきたかと思うと	かと思うと~	
\\	家事は、子供が眠っている 
\\	、全部やってしまった。	家事は、子供が眠っているうちに、全部やってしまった。	うちに	うちに	
\\	この携帯電話は、長い間使っている 
\\	、もう自分の体の一部のようになった。	この携帯電話は、長い間使っているうちに、もう自分の体の一部のようになった。	うちに	うちに	
\\	桜が満開に (なる:
\\	とき、雪が降った。	桜が満開になろうとしているとき、雪が降った。	うとしている	うとしている	
\\	考えている 
\\	、話しかけられて困った。	考[かんが]えている 最中[さいちゅう]に、 話[はな]しかけられて 困[こま]った。	最中に	最中に	
\\	このままではジムの日本語の成績は下がる 
\\	。	このままではジムの 日本語[にほんご]の 成績[せいせき]は 下[さ]がるばかりだ。	ばかりだ	ばかりだ	
\\	その運動は急速に (進む: ている, 
\\	。	その 運動[うんどう]は 急速[きゅうそく]に 進[すす]みつつある。	つつある	つつある	
\\	、アメリカ人は大きな成功を夢見て、危険を冒す傾向が強い。	一方[いっぽう]、アメリカ 人[じん]は 大[おお]きな 成功[せいこう]を 夢見[ゆめみ]て、 危険[きけん]を 冒[おか]す 傾向[けいこう]が 強[つよ]い。	一方・その一方では	一方	
\\	携帯電話を使う人の数は増える 
\\	です	携帯[けいたい] 電話[でんわ]を 使[つか]う 人[ひと]の 数[かず]は 増[ふ]える 一方[いっぽう]です	一方	一方	
\\	山に(登る: ~ながら, [書き言葉])、人は人生についてさまざまなことを考える。	山[やま]に 登[のぼ]りつつ、 人[ひと]は 人生[じんせい]についてさまざまなことを 考[かんが]える。	登りつつ	つつ	
\\	詳しいことが (わかる: 
\\	、ご連絡いたします。	詳しいことがわかり次第、ご連絡いたします。	わかり次第	次第	
\\	あの山の写真 (見る:からずっと)、いつか登ってみたいとずっと思い続けてきた。	あの山の写真を見て以来、いつか登ってみたいとずっと思い続けてきた。	を見て以来	て以来	
\\	母がいなく (なる:ずっと、以来)、母のことを考えない日はない。	母がいなくなってこのかた、母のことを考えない日はない。	なってこのかた	てこのかた	
\\	上司に相談 (する: 
\\	お返事できません。	上司[じょうし]に 相談[そうだん]してからでないとお 返事[へんじ]できません。	してからでないと	てからでないと	
\\	大きな仕事は十分な準備が (ある: 
\\	、成功するのだ。	大[おお]きな 仕事[しごと]は 十分[じゅうぶん]な 準備[じゅんび]があってはじめて、 成功[せいこう]するのだ。	あってはじめて	てはじめて	
\\	よく (考える: 
\\	ご返事いたします。	よく 考[かんが]えた 上[うえ]でご 返事[へんじ]いたします。	考えた上で	上で	
\\	この体育館では水泳 
\\	、色々なスポーツが楽しめる。	この体育館では水泳をはじめ、色々なスポーツが楽しめる。	をはじめ	をはじめ(として)	
\\	三日間 
\\	研究発表大会が、無事終了しました。	三日間にわたる研究発表大会が、無事終了しました。	にわたる	にわたる	
\\	連休の最終日、高速道路は20キロ 
\\	、渋滞が続いた。	連休の最終日、高速道路は20キロにわたって、渋滞が続いた。	にわたって	にわたって	
\\	君の知っている 
\\	のことを全部私に話してほしい。	君の知っている限りのことを全部私に話してほしい。	限り	限り	
\\	父は働く 
\\	限界まで)働いて、定年前に退職してしまった。	父は働くだけ働いて、定年前に退職してしまった。	だけ	だけ	
\\	彼の健康状態 
\\	スポーツは無理でしょう。	彼[かれ]の 健康[けんこう] 状態[じょうたい]からしてスポーツは 無理[むり]でしょう。	からして	からして	
\\	本日、退社することになりました社長 
\\	皆様には本当にお世話になりました	本日[ほんじつ]、 退社[たいしゃ]することになりました 社長[しゃちょう]を 初[はじ]め 皆様[みなさま]には 本当[ほんとう]にお 世話[せわ]になりました	を初め	を初め	
\\	会話 (で, 
\\	日本語を覚えています。	会話[かいわ]を 通[つう]じて 日本語[にほんご] を覚[おぼ]えています。	を通じて	を通じて	
\\	急を要する患者さん 
\\	、時間外でも診察いたします。	急を要する患者さんに限り、時間外でも診察いたします。	に限り	に限り	
\\	親である
\\	、子供に対する責任があると思う。	親である限りは、子供に対する責任があると思う。	限り(は)	限り(は)	
\\	私が知っている 
\\	、この近所に花屋はありません。	私が知っている限りでは、この近所に花屋はありません。	限りでは	限りでは	
\\	あのレストラン (信じているから)、古い食材などを使うはずはないと思う。	あのレストランに限って、古い食材などを使うはずはないと思う。	に限って	に限って	
\\	ちょっと話した 
\\	、彼はいつもとまったく変わらないように思えた。	ちょっと 話[はな]したかぎりでは、 彼[かれ]はいつもとまったく 変[か]わらないように 思[おも]えた。	かぎりでは	かぎりでは	
\\	近年、地方の町 (だけでなく、
\\	大都会でも書店の閉店が相次いでいる。	近年、地方の町に限らず大都会でも書店の閉店が相次いでいる。	に限らず	に限らず	
\\	柔道は若者の健康によい (だけじゃなくて 
\\	、人格形成にも大いに役立つ。	柔道[じゅうどう]は 若者[わかもの]の 健康[けんこう]によいばかりか、 人格[じんかく] 形成[けいせい]にも大い[おおい]に 役立[やくだ]つ。	ばかりか	ばかりか	
\\	彼女はうまく歌えない (だけじゃなくて 
\\	ピアノもよく弾けません。	彼女[かのじょ]はうまく 歌[うた]えないばかりかピアノもよく 弾[ひ]けません。	ばかりか	ばかりか	
\\	寒かった 
\\	、風が強かった。	寒[さむ]かった上に、 風[かぜ]が 強[つよ]かった。	上に	上に	
\\	彼が日本 (だけでなく 
\\	、世界においても有名な物理学者である。	彼[かれ]が 日本[にっぽん]のみならず、 世界[せかい]においても 有名[ゆうめい]な 物理[ぶつり] 学者[がくしゃ]である。	のみならず	のみならず	
\\	彼は料理 (~はもちろん 
\\	食器も洗う。	彼[かれ]は 料理[りょうり]はもとより 食器[しょっき]も 洗[あら]う。	はもとより	はもとより	
\\	今回の事件 
\\	の内容について)、何か詳しいことがわかりましたか。	今回の事件に関して、何か詳しいことがわかりましたか。	に関して	に関して、に関する	
\\	親が残した土地 
\\	、親族が争っている。	親が残した土地をめぐって、親族が争っている。	をめぐって	をめぐって	
\\	政府の案 
\\	、住民は大反対した。	政府の案に対して、住民は大反対した。	に対して	に対して	
\\	うちでは、父は感情が激しい 
\\	、母は冷静で穏やかな性格です。	うちでは、父は感情が激しいのに対して、母は冷静で穏やかな性格です。	のに対して	に対して	
\\	応援してくれる人の気持ち 
\\	、立派な試合をしよう!	応援してくれる人の気持ちにこたえて、立派な試合をしよう!	にこたえて	にこたえて	
\\	数学 (に関して, 
\\	彼女はクラスでいつも一番だった	数学[すうがく]にかけては 彼女[かのじょ]はクラスでいつも 一番[いちばん]だった	にかけては	にかけては	
\\	この小説は作者自身の個人的な体験 
\\	書いたものだそうだ。	この小説は作者自身の個人的な体験をもとに書いたものだそうだ。	をもとに	をもとに(して)	
\\	あの飛行機事故 
\\	映画はこれだけじゃない。	あの飛行機事故をもとにした映画はこれだけじゃない。	をもとにした	をもとに(して)	
\\	このジムでは、科学的な実験結果 
\\	筋肉トレーニングを行っている。	このジムでは、科学的な実験結果に基づいた筋肉トレーニングを行っている。	に基づいた	に基づいて	
\\	政府の基本方針 
\\	、今年度の予算案を作っている。	政府の基本方針に沿って、今年度の予算案を作っている。	に沿って	に沿って	
\\	チームは今、新しい監督 
\\	練習に励んでいる。	チームは今、新しい監督のもとで練習に励んでいる。	のもとで	のもとで・に	
\\	このマンションは一人暮らしの高齢者 
\\	設計されています。	このマンションは一人暮らしの高齢者向けに設計されています。	向けに	向けだ	
\\	本日は夏 
\\	冷たいお料理をいくつかご紹介します。	本日[ほんじつ]は 夏向[なつむ]きに 冷[つめ]たいお 料理[りょうり]をいくつかご 紹介[しょうかい]します。	向きに	向きに	
\\	先生の指導 
\\	試験に受かる事が出来ました。	先生[せんせい]の 指導[しどう]のもとに 試験[しけん]に 受[う]かる 事[こと]が 出来[でき]ました。	のもとに・のもとで	のもとに・~のもとで	
\\	息子は成長する 
\\	、口数が少なくなった。	息子は成長するにしたがって、口数が少なくなった。	にしたがって	にしたがって	
\\	経済の回復 (に伴って, 
\\	、人々の表情も明るくなってきた。	経済の回復とともに、人々の表情も明るくなってきた。	とともに	とともに	
\\	レストランは、雰囲気 (によって, 
\\	、お客様が増えたり、減ったりするのです。	レストランは、雰囲気次第で、お客様が増えたり、減ったりするのです。	次第で	次第だ	
\\	父は何か (すると 
\\	、若いころ外国で過ごした思い出を語る。	父は何かにつけて、若いころ外国で過ごした思い出を語る。	につけて	につけて	
\\	収入の多寡 
\\	出資額を決める。	収入[しゅうにゅう]の 多寡[たか]に 応[おう]じて 出資[しゅっし] 額[がく]を 決[き]める。	に応じて	に応じて	
\\	人口の急増 
\\	につれて)、いろいろの問題が出てきた。	人口[じんこう]の 急増[きゅうぞう]に 伴[ともな]って、いろいろの 問題[もんだい]が 出[で]てきた。	に伴って	に伴って	
\\	山は高くなる 
\\	、気温が下がる。	山[やま]は 高[たか]くなるにつれて、 気温[きおん]が 下[さ]がる。	につれて	につれて	
\\	インド料理やタイ料理 
\\	とか) 南の国の食べ物には辛いものが多い。	インド 料理[りょうり]やタイ 料理[りょうり]といった 南[みなみ]の 国[くに]の 食べ物[たべもの]には 辛[つら]いものが 多[おお]い。	といった	といった	
\\	宿題 (や, 
\\	レポート (や, 
\\	で、のんびりする暇もない。	宿題[しゅくだい]やらレポートやらで、のんびりする 暇[ひま]もない。	やら	やら~やら	
\\	論文を書く 
\\	討論をする 
\\	、十分にデータを集めておく必要がある。	論文を書くにせよ討論をするにせよ、十分にデータを集めておく必要がある。	~にせよ~にせよ	~にしても~にしても・~にしろ~にしろ・~にせよ~にせよ	
\\	京都とか鎌倉 (例をまとめる)古い街には寺が多い。	京都とか 鎌倉[かまくら]といった古い街には寺が多い。	といった	といった	
\\	理由が何であるか (に関係なく)、一度納入した入学金はお返しできません。	理由が 何[なに]であるかにかかわりなく、一度 納入[のうにゅう]した入学金はお返しできません。	にかかわりなく・にかかわらず	にかかわりなく・にかかわらず	
\\	費用の問題 
\\	、旅行の目的地を決める方が先です。	費用[ひよう]の 問題[もんだい]はともかく、 旅行[りょこう]の 目的[もくてき] 地[ち]を 決[き]める 方[ほう]が 先[さき]です。	はともかく(として)	はともかく	
\\	金の問題 
\\	、そういう旅行は疲れるよ。	金[かね]の 問題[もんだい]はさておき、そういう 旅行[りょこう]は 疲[つか]れるよ。	はさておき	はさておき	
\\	学歴、経験は 
\\	、やる気のある社員を募集します。	学歴[がくれき]、 経験[けいけん]は 問[と]わず、やる 気[き]のある 社員[しゃいん]を 募集[ぼしゅう]します。	問わず	問わず	
\\	人の迷惑 
\\	電車の中で携帯電話で話している人がいる。	人[ひと]の 迷惑[めいわく]もかまわず 電車[でんしゃ]の 中[なか]で 携帯[けいたい] 電話[でんわ]で 話[はな]している 人[ひと]がいる。	も構わず	も構わず	
\\	どんなこともお金で解決できる 
\\	。	どんなこともお 金[かね]で 解決[かいけつ]できる、というものではありません。	というものではありません	というものではない	
\\	彼はよく間違いをするが、馬鹿 
\\	。	彼[かれ]はよく 間違[まちが]いをするが、 馬鹿[ばか]どころではない。	どころではない	どころではない	
\\	忙しくて遊ぶ 
\\	。	忙[いそが]しくて 遊[あそ]ぶどころではない。	どころではない	どころではない	
\\	あきらめる 
\\	。最後までがんばりぬくぞ。	あきらめるものか。 最後[さいご]までがんばりぬくぞ。	ものか	ものか	
\\	彼女がそんなひどいことを言う 
\\	。	彼女[かのじょ]がそんなひどいことを 言[い]うわけがない。	わけがない	わけがない	
\\	今のアパート、駅から遠いので不便 (一応認めておく)不便ですが、静かでいいですよ。	今のアパート、駅から遠いので不便といえば不便ですが、静かでいいですよ。	といえば	といえば	
\\	水蒸気 
\\	気体の状態に変わった水のことである。	水蒸気[すいじょうき]とは 気体[きたい]の 状態[じょうたい]に 変[か]わった 水[みず]のことである。	とは	とは	
\\	父も母ものんびり過ごしています。私 
\\	、毎日ただ忙しく働いています。	父[ちち]も 母[はは]ものんびり 過[す]ごしています。 私[わたし]はというと、 毎日[まいにち]ただ 忙[いそが]しく 働[はたら]いています。	はというと	はというと	
\\	花子は食べる 
\\	、急に元気になる。	花子[はなこ]は 食[た]べることとなると、 急[きゅう]に 元気[げんき]になる。	こととなると	こととなると	
\\	「日本」
\\	、何を連想しますか。	
\\	日本[にっぽん]」というと、 何[なに]を 連想[れんそう]しますか。	というと	というと	
\\	オーストラリア 
\\	、すぐにコアラとかカンガルーを思い浮かべる。	オーストラリアというと、すぐにコアラとかカンガルーを思い浮かべる。	~というと・~といえば・~といったら	~というと・~といえば・~といったら	
\\	富士山の頂上から見た景色 (の程度が普通ではないということ、強調する)、思い出すだけで感動する。	富士山の頂上から見た景色といったら、思い出すだけで感動する。	といったら	といったら	
\\	水道工事の人は悪天候 
\\	、作業を続けている。	水道工事の人は悪天候にもかかわらず、作業を続けている。	にもかかわらず	にもかかわらず	
\\	高価な着物を買った 
\\	、着るチャンスがない。	高価な着物を買ったものの、着るチャンスがない。	ものの	ものの、とはいうものの	
\\	この部屋は狭い (ても, 
\\	居心地がよい。	この 部屋[へや]は 狭[せま]いながらも 居心地[いごこち]がよい。	ながらも	ながらも	
\\	大学を出た 
\\	、必ずしも教養があるわけではない。	大学を出たからといって、必ずしも教養があるわけではない。	からといって	からといって	
\\	流言飛語に惑わされるべからず、
\\	、言うは易く行うは難し、と思わない?	流言飛語[りゅうげんひご]に 惑[まど]わされるべからず、とはいうものの、 言[い]うは 易[やす]く 行[おこな]うは 難[かた]し、と 思[おも]わない?	とはいうものの	とはいうものの	
\\	悪いと (知る: 
\\	も、ごみを分別せずに捨ててしまう。	悪[わる]いと 知[し]りつつも、ごみを 分別[ふんべつ]せずに 捨[す]ててしまう。	知りつつ	つつ	
\\	料理ができる 
\\	、私が作れるのは簡単なものだけです。	料理ができるといっても、私が作れるのは簡単なものだけです。	といっても	といっても	
\\	南極旅行は、気持ち的には、借金してでも行ける 
\\	行きたいんだが。	南極[なんきょく] 旅行[りょこう]は、 気持[きも]ち 的[てき]には、 借金[しゃっきん]してでも 行[い]けるものなら 行[い]きたいんだが。	ものなら	ものなら	
\\	貧乏 (たとえ~ても, ~でも, 
\\	盗みはいけないよ!	貧乏[びんぼう]にせよ 盗[ぬす]みはいけないよ!	にしても・にしろ
\\	・にせよ
\\	にしても・にしろ・にせよ	
\\	情報を教えて (頂く: 
\\	検討のしようがありません。	情報[じょうほう]を 教[おし]えていただけないことには 検討[けんとう]のしようがありません。	いただけないことには	ないことには	
\\	この魚は猛毒を持っていて、間違ってそのトゲに (刺される: 
\\	、患部が大きく腫れ上がってしまう。	この 魚[さかな]は 猛毒[もうどく]を 持[も]っていて、 間違[まちが]ってそのトゲに 刺[さ]されようものなら、 患部[かんぶ]が 大[おお]きく 腫れ上[はれあ]がってしまう。	刺されようものなら	ようものなら	
\\	日本経済がこのまま下降線をたどる 
\\	、日本語ブームもやがては終わるかもしれない。	日本[にっぽん] 経済[けいざい]がこのまま 下降[かこう] 線[せん]をたどるとなると、 日本語[にほんご]ブームもやがては 終[お]わるかもしれない。	となると	となると	
\\	今日は月曜日 
\\	明日は火曜日だ。	今日[きょう]は 月曜日[げつようび]だとしたら, 明日[あした]は 火曜日[かようび]だ。	だとしたら	としたら	
\\	インターネットとメール (がなければ、実現しない)、現代は成り立たないと言ってもいい。	インターネットとメールを抜きにしては、現代は成り立たないと言ってもいい。	を抜きにしては	を抜きにしては	
\\	ショック 
\\	、涙も出ませんでしたよ。	ショックのあまり、 涙[なみだ]も 出[で]ませんでしたよ。	の余り	の余り	
\\	彼女は失敗を私
\\	責める)。	彼女[かのじょ]は 失敗[しっぱい]を 私[わたし]のせいにした。	のせいにした	のせいにした	
\\	いつもの列車に乗り遅れた 
\\	。	いつもの 列車[れっしゃ]に 乗り遅[のりおく]れたものですから。	ものですから	ものだから・もので・もの	
\\	この仕事は一時間 
\\	十ドル貰える。	この 仕事[しごと]は一 時間[じかん]につき十ドル 貰[もら]える。	につき	につき	
\\	工事 (のため, 
\\	電車は不通です。	工事[こうじ]につき 電車[でんしゃ]は 不通[ふつう]です。	につき	につき	
\\	来週はお客様が来る (とにかく~だからあることをする)、家の中の大掃除をしなきゃ。	来週はお客様が来ることだし、家の中の 大掃除[おおそうじ]をしなきゃ。	~ことだし	~ことだし	
\\	母は今年93歳になった。今は元気だが、 (歳~歳~: 
\\	、病気をすると心配だ。	母[はは]は 今年[ことし]93 歳[さい]になった。 今[いま]は 元気[げんき]だが、 歳[とし]が 歳[とし]だけに、 病気[びょうき]をすると 心配[しんぱい]だ。	歳が歳だけに	が~だけに	
\\	田中 (やっぱり、
\\	、遅れるでしょう.	田中[たなか]、 遅[おく]れるでしょう.	のことだから	のことだから	
\\	新車を(買う: 
\\	、お金がありません。	新車[しんしゃ]を 買[か]ったばかりに、お 金[かね]がありません。	
\\	買ったばかりに
\\	買ったばかりだから	たばかりに	
\\	やると決めた 
\\	最後までやれ	やると 決[き]めたからには 最後[さいご]までやれ	からには・からは	からには	
\\	両親は年をとっている
\\	、娘としては心配だ。	両親[りょうしん]は 年[とし]をとっているだけに、 娘[むすめ]としては 心配[しんぱい]だ。	だけに	だけに	
\\	動物を飼う 
\\	責任をもって世話をするべきです。	動物[どうぶつ]を 飼[か]う 上[うえ]では 責任[せきにん]をもって 世話[せわ]をするべきです。	上では	上は	
\\	あの優しい彼がそんなひどいことをしたとは (信じる: 難しい)。	あの優しい彼がそんなひどいことをしたとは信じがたい。	~がたい	~がたい	
\\	道は通行止めだし。これではそこへ (行く: 
\\	。	道は通行止めだし。これではそこへ行きようがない。	行きようがない	ようがない	
\\	そのような条件には (応じる: 
\\	。	そのような 条件[じょうけん]には 応[おう]じかねる。	応じかねる	かねる	
\\	癌は誰でも (かかる: 可能性がある)病気だ。	癌[がん]は誰でもかかり 得る[うる]病気だ。	~得る	~得る・~得ない	
\\	人間が100メートルを5秒で走るなんて (ある:可能性がない)話だ。	人間が100メートルを5秒で走るなんてあり 得ない[えない]話だ。	~得ない	~得る・~得ない	
\\	さすが毎日トレーニングしている 
\\	すごい体力。	さすが 毎日[まいにち]トレーニングしているだけあってすごい 体力[たいりょく]。	だけあって	だけあって	
\\	このメロンはおいしい。高かった 
\\	。	このメロンはおいしい。 高[たか]かっただけのことはある。	だけのことはある	だけのことはある	
\\	弟 
\\	、私のようなうるさい姉はいやでたまらない。	弟[おとうと]にしたら、 私[わたし]のようなうるさい 姉[あね]はいやでたまらない。	にしたら	にしたら	
\\	6月 
\\	寒い朝だ。	6月[ろくがつ]にしては 寒[さむ]い 朝[あさ]だ。	にしては	にしては	
\\	喧嘩 (する: 
\\	どうにもならないだろ。	喧嘩[けんか]したところでどうにもならないだろ。	したところで	たところで	
\\	留学について父に相談して (みる: 
\\	、喜んで賛成してくれた。	留学[りゅうがく]について 父[ちち]に 相談[そうだん]して みたところ、 喜[よろこ]んで 賛成[さんせい]してくれた。	みたところ	たところ	
\\	手紙を貰った (まだそのままの状態)、連絡がない。	手紙[てがみ]を 貰[もら]ったきり、 連絡[れんらく]がない。	きり	きり	
\\	散々(考える: 
\\	、 その計画を実行にうつした	散々[さんざん] 考[かんが]えた 挙げ句[あげく]、その 計画[けいかく]を 実行[じっこう]にうつした	考えたあげく	たあげく	
\\	あの映画も終わってしまった。あんなに見たいと思っていたのに、とうとう (見る: 
\\	だった。	あの 映画[えいが]も 終[お]わってしまった。あんなに 見[み]たいと 思[おも]っていたのに、とうとう 見[み]ずじまいだった。	見ずじまい	ずじまい	
\\	うちに帰ってきたら、自分の靴 (程度が軽い)ちゃんと並べなさい。	うちに帰ってきたら、自分の靴ぐらいちゃんと並べなさい。	~ぐらい・~くらい	~ぐらい・~くらい	
\\	あの人の言ったこと (価値が低い)、気にすることはありません。	あの人の言ったことなど、気にすることはありません。	~など・~なんか・~なんて	~など・~なんか・~なんて	
\\	このサイトは検索はおろか、本の内容紹介 
\\	くれるので非常に便利だ。	このサイトは 検索[けんさく]はおろか、 本[ほん]の 内容[ないよう] 紹介[しょうかい]までしてくれるので 非常[ひじょう]に 便利[べんり]だ。	までして	までして	
\\	家族や家庭を犠牲に (する: 
\\	会社のために働くなんてとんでもない。	家族[かぞく]や 家庭[かてい]を 犠牲[ぎせい]にしてまで 会社[かいしゃ]のために 働[はたら]くなんてとんでもない。	してまで	てまで	
\\	南極旅行は、気持ち的には、借金 (する: 
\\	行けるものなら行きたいんだが。	南極[なんきょく] 旅行[りょこう]は、 気持[きも]ち 的[てき]には、 借金[しゃっきん]してでも 行[い]けるものなら 行[い]きたいんだが。	してでも	てでも	
\\	私の子供のころの写真は一枚 (とうえども~ない, 
\\	残ってない。	私[わたし]の 子供[こども]のころの 写真[しゃしん]は一 枚[まい]として 残[のこ]ってない。	として	として~ない	
\\	外国で病気になる (不安: 
\\	。	外国[がいこく]で 病気[びょうき]になるくらい 不安[ふあん]なことはない。	くらい不安なことはない	くらい~ことはない	
\\	仕事がとても忙しいときは、会社に泊まること (極端な例もそうだから...)ある。	仕事がとても忙しいときは、会社に泊まることさえある。	さえ	さえ	
\\	年をとっても体 (これが満たされれば問題ない)丈夫なら、他に望むことはない。	年をとっても体さえ丈夫なら、他に望むことはない。	さえ	さえ	
\\	今夜から明日にかけて東日本で大雨の (悪いことが起こる可能性がある)。	今夜から明日にかけて東日本で大雨のおそれがあります。	おそれがある	おそれがある	
\\	夜遅く雨が降った (らしい, 
\\	、庭が濡れている。	夜[よる] 遅[おそ]く 雨[あめ]が 降[ふ]ったとみえて、 庭[にわ]が 濡[ぬ]れている。	とみえて	とみえて	
\\	彼は天才である (に違いない 
\\	。	彼[かれ]は 天才[てんさい]であるに 相違[そうい]ない。	に相違ない	に相違ない	
\\	誰も信じて (くれる: ないだろう)が、これは本当の話だ。	誰も信じてくれまいが、これは本当の話だ。	~まい	~まい・~ではあるまいか	
\\	勝手にお父さんの車を使ったりしたら、叱られる (絶対に~だと思う)。	勝手にお父さんの車を使ったりしたら、叱られるにきまっている。	にきまっている	にきまっている	
\\	人間というのは本来一人では生きられない (一般的な常識.. 真理..)。	人間というのは本来一人では生きられないものだ。	ものだ	ものだ	
\\	私は当たり前のことをした (ただ~だけで)。	私は当たり前のことをしたにすぎません。	~にすぎない	~にすぎない	
\\	勉強する (しかない 
\\	。	勉強[べんきょう]するよりほかない。	(より)ほかない	(より)ほかない	
\\	戦争は死 
\\	。	戦争[せんそう]は 死[し]にほかならない。	にほかならない	にほかならない	
\\	決められた時間より早めに着く 
\\	。	決[き]められた 時間[じかん]より 早[はや]めに 着[つ]くにこしたことはない。	にこしたことはない	にこしたことはない	
\\	太りたくなければ、夜遅く食べない (~する・~しないことが大切だ)。	太りたくなければ、夜遅く食べないことです。	ことです	ことだ	
\\	人との出会いは大切にする (した方がいい, 
\\	。	人との出会いは大切にするものだ。	~ものだ	~ものだ・~ものではない	
\\	これからは少しでも人の役に立つことを (考える: 
\\	。	これからは 少[すこ]しでも 人[ひと]の 役に立[やくにた]つことを 考[かんが]えようではないか。	考えようではないか	ようではないか	
\\	彼の言葉など気にする (~する必要がない)よ。いつもきつい言い方をする人だから。	彼の言葉など気にすることはないよ。いつもきつい言い方をする人だから。	ことはない	ことはない	
\\	会を欠席する 
\\	。	会[かい]を 欠席[けっせき]するわけにはいきません。	わけにはいかない	わけにはいかない	
\\	大学に合格して、(嬉しい: 
\\	。	大学[だいがく]に 合格[ごうかく]して、嬉しくてたまらない。	嬉しくてたまらない	たまらない	
\\	東京の夏は (暑い: 
\\	。	東京[とうきょう]の 夏[なつ]は 暑[あつ]くてならない。	暑くてならない	てならない	
\\	子供のことが心配 
\\	。	子供[こども]のことが 心配[しんぱい]でたまらない。	で堪らない	てたまらない	
\\	昨夜は徹夜したので、今日は (眠い: 
\\	。	昨夜[さくや]は 徹夜[てつや]したので、 今日[きょう]は 眠[ねむ]くてしかたがない。	眠くてしかたがない	てしかたがない	
\\	親友の結婚式だから、忙しくても出席 (する: 事情があって~しなきゃ)。	親友の結婚式だから、忙しくても出席しないわけにはいかない。	~ないわけに(は)いかない	~ないわけに(は)いかない	
\\	このまま赤字が続けば、経営方針を変更 (する: 
\\	だろう。	このまま赤字が続けば、経営方針を変更せざるを得ないだろう。	~ざるを得ない	~ざるを得ない	
\\	今度こそ実験が成功 (する: したい・してほしいと強く思う)。	今度こそ実験が成功してほしいものだ。	~たいものだ・てほしいものだ	~たいものだ・てほしいものだ	
\\	卒業してからもう10年か。時間が過ぎるのは早い (を非常に強く感じる)。	卒業してからもう10年か。時間が過ぎるのは早いものだ。	~ものだ	~ものだ	
\\	子供のとき、よく川で (遊ぶ: 
\\	。	子供[こども]のとき、よく 川[かわ]で 遊[あそ]んだものだ。	遊んだものだ	たものだ	
\\	誰かこの仕事を引き受けてくれる人はいない 
\\	。	誰かこの仕事を引き受けてくれる人はいないものだろうか。	~ないもの(だろう)か	~ないもの(だろう)か	
\\	毎日2時間もかけて通勤するのは、かなりつらい (~という感じがある)。	毎日2時間もかけて通勤するのは、かなりつらいものがある。	~ものがある	~ものがある	
\\	困ったとき助けてくれる友達がいる。ありがたい (本当に~だ)。	困ったとき助けてくれる友達がいる。ありがたいことだ。	ことだ	ことだ	
\\	そんなことは有り得ない	よもやそんなことは 有り得[ありえ]ない	よもや	よもや	
\\	日本へ (行く: 
\\	迷った。	日本へ行こうか行くまいか迷った。	行こうか行くまいか	ようか~まいか	
\\	(ハンサム: 
\\	恋人を探しやすいです。	ハンサムならばハンサムなほど 恋人[こいびと]を 探[さが]しやすいです。	ハンサムならばハンサムなほど	ば~ほど	
\\	雑誌を (読む: 
\\	、そのままうとうと寝てしまった。	雑誌[ざっし]を 読[よ]みかけて、そのままうとうと 寝[ね]てしまった。	読みかけて	かける	
\\	ここにゴミを捨てる 
\\	。	ここにゴミを 捨[す]てるべからず。	べからず	べからず	
\\	このことは今度お目にかかった 
\\	詳しくお話しいたします。	このことは 今度[こんど]お 目[め]にかかった 折[おり]に 詳[くわ]しくお 話[はな]しいたします。	折に	折に	
\\	課長にこう毎晩のように飲みに (誘われる: 
\\	。	課長[かちょう]にこう 毎晩[まいばん]のように 飲[の]みに 誘[さそ]われてはかなわない。	誘われてはかなわない。	てはかなわない。	
\\	(頑張る: 
\\	頑張[がんば]りぬく	頑張りぬく	ぬく	
\\	父は (お酒を飲む: 
\\	タバコも吸うので、健康が心配だ。	父[ちち]はお お酒[おさけ]も飲めばタバコも 吸[す]うので、 健康[けんこう]が 心配[しんぱい]だ。お	お酒も飲めば	も~ば~も	
\\	この方が 
\\	ほど優れている。	この方が 比[くら]べものにならないほど 優[すぐ]れている。	比べものにならない	比べものにならない	
\\	白 
\\	団体は去年ニュースになっていた。	白[しろ]ずくめ 団体[だんたい]は 去年[きょねん]ニュースになっていた。	ずくめ	ずくめ	
\\	いいこと 
\\	、ばかり) ですよ。	いいことずくめですよ。	ずくめ	ずくめ	
\\	私が料理すると、(味が濃い、薄い: 
\\	家族から文句がでる。	私[わたし]が 料理[りょうり]すると、 味[あじ]が 濃[こ]いにつけ、 薄[うす]いにつけ 家族[かぞく]から 文句[もんく]がでる。	味が濃いにつけ、薄いにつけ	につけ	
\\	(早い: 
\\	ありがたいです。	早[はや]ければ 早[はや]いほどありがたいです	早ければ早いほど	ば~ほど	
\\	妻の病気 
\\	、家族の結束は再び固まった。	妻[つま]の 病気[びょうき]をきっかけに、 家族[かぞく]の 結束[けっそく]は 再[ふたた]び 固[かた]まった。	をきっかけに	をきっかけに	
\\	テーブルの上の (飲む: 
\\	コーヒー、誰の?	テーブルの 上[うえ]の 飲[の]みかけのコーヒー、 誰[だれ]の?	飲みかけの	かけの~	
\\	仕事 
\\	、子供達にフランス語を教えるつもりです。	仕事[しごと]のかたわら、 子供[こども] 達[たち]に フランス語[ふらんすご]を 教[おし]えるつもりです。	のかたわら	かたわら	
\\	支払いは今すぐで (ない: なくても, 
\\	。後でもいいですよ。	支払[しはら]いは 今[いま]すぐでなくてもさしつかえありません。 後[のち]でもいいですよ。	なくてもさしつかえありません	なくてもさしつかえない	
\\	失礼 (の極まり, 
\\	。	失礼[しつれい] 極[きわ]まりない。	極まりない	極まりない	
\\	1日の仕事を終えた後は、冷えたビール 
\\	よ。	
\\	日[にち]の 仕事[しごと]を 終[お]えた 後[のち]は、 冷[ひ]えたビールに 限[かぎ]るよ。	に限る	に限る	
\\	最近、忙しい仕事が続いたので少し (疲れる: 
\\	です。	最近[さいきん]、 忙[いそが]しい 仕事[しごと]が 続[つづ]いたので 少[すこ]し 疲[つか]れ 気味[ぎみ]です。	疲れ気味	気味	
\\	信頼する 
\\	人物だと思っていた、その人にだまされてしまった	信頼[しんらい]するに 足[た]る 人物[じんぶつ]だと 思[おも]っていた、その 人[ひと]にだまされてしまった	に足る	に足る	
\\	帰国するというのは、さんざん (迷う: あげく, 
\\	出した結論です。	帰国[きこく]するというのは、さんざん 迷[まよ]った 末[すえ]に 出[だ]した 結論[けつろん]です。	迷った末に	すえに	
\\	試合 
\\	開会式を行います。	試合[しあい]に 先立[さきだ]って 開会[かいかい] 式[しき]を 行[おこな]います。	に先立って	に先立って	
\\	それ 
\\	、なかなかユニークなキャラクターだ。	それに 加[くわ]えて、なかなかユニークなキャラクターだ。	に加えて	に加えて	
\\	あの男が犯人だ 
\\	、警察はつかまえるチャンスを逃したことになる。	あの 男[おとこ]が 犯人[はんにん]だとすると、 警察[けいさつ]はつかまえるチャンスを 逃[のが]したことになる。	とすると	とすると	
\\	これは何かの間違い (ではないか 
\\	。	これは 何[なに]かの 間違[まちが]いではあるまいか。	ではあるまいか	ではあるまいか	
\\	背が高い (のに 
\\	早く走れない。	背[せ]が 高[たか]いくせに 早[はや]く 走[はし]れない。	くせに	くせに	
\\	完全を求めると 
\\	失敗する。	完全[かんぜん]を 求[もと]めるとかえって 失敗[しっぱい]する。	かえって	かえって~	
\\	昼飯 
\\	仕事をした.	昼飯[ひるめし]ぬきで 仕事[しごと]をした.	ぬきで	ぬきで	
\\	運動 (する: 
\\	、歩いて行った。	運動[うんどう]したつもりで、 歩[ある]いて 行[い]った。	したつもりで	たつもりで	
\\	ドアが (開く: 
\\	が、また閉まってしまった。	ドアが 開[ひら]きかかったが、また 閉[し]まってしまった。	開きかかった	かかる	
\\	山田さんの部屋は何か月も掃除していない 
\\	汚い。	山田[やまだ]さんの 部屋[へや]は 何[なん]か 月[げつ]も 掃除[そうじ]していないかのように 汚[きたな]い。	かのように	かのよう	
\\	彼の引っ越し先はなんと人口1000人の小さな孤島 
\\	。	彼[かれ]の 引っ越[ひっこ]し 先[さき]はなんと 人口[じんこう] 
\\	人[にん]の 小[ちい]さな 孤島[ことう]ではないか。	ではないか	ではないか	
\\	生徒は学校もともかく (休む: 
\\	だ。	生徒[せいと]は 学校[がっこう]もともかく 休[やす]みがちだ。	休みがち	がち	
\\	友達を傷つける 
\\	、 (自分が我慢する: 
\\	。	友達[ともだち]を 傷[きず]つけるくらいなら、 自分[じぶん]が 我慢[がまん]したほうがいい。	くらいなら、自分が我慢したほうがいい	くらいなら~ほうがいい	
\\	たった1キロを走っただけで、汗 
\\	になるのは情けない。	たった1キロを 走[はし]っただけで、 汗[あせ]まみれになるのは 情[なさ]けない。	まみれ	まみれ	
\\	私はあなたを助けてくれる人が(一人: 
\\	いる事を祈ります。	私[わたし]はあなたを 助[たす]けてくれる 人[ひと]が 一人[ひとり]でも 多[おお]く いる 事[こと]を 祈[いの]ります。	一人でも多く	一人でも多く	
\\	運動をしてもっと健康になる 
\\	。	運動[うんどう]をしてもっと 健康[けんこう]になるようにする。	ようにする	ようにする	
\\	この条件では拒絶 
\\	。	この 条件[じょうけん]では 拒絶[きょぜつ]に 等[ひと]しい。	に等しい	に等しい	
\\	上司 (の代わりに 
\\	会議に出た。	上司[じょうし]に 代[か]わって 会議[かいぎ]に 出[で]た。	に代わって	に代わって	
\\	ダイエットしていたのに、痩せる 
\\	、太ってしまいました。	ダイエットしていたのに、 痩[や]せるどころか、 太[ふと]ってしまいました。	どころか	どころか	
\\	善人が 
\\	成功するとは限らない。	善人[ぜんにん]が 必[かなら]ずしも 成功[せいこう]するとは 限[かぎ]らない。	必ずしも	必ずしも	
\\	身を (捨てる: 
\\	浮かぶ、瀬もあれ。	身[み]を 捨[す]ててこそ 浮[う]かぶ、 瀬[せ]もあれ。	捨ててこそ	てこそ	
\\	正月は(食べる: 
\\	寝,(飲む: 
\\	寝の日を過ごしていた。	正月[しょうがつ]は 食[た]べては 寝[ね], 飲[の]んでは 寝[ね]の 日[ひ]を 過[す]ごしていた。	食べては寝, 飲んでは寝	ては	
\\	彼がそのゲームを (やる: 
\\	。	彼[かれ]がそのゲームをやりこなしました。	こなしました	こなす	
\\	ノーベル賞なんて、僕は (貰える: 
\\	。	ノーベル 賞[しょう]なんて、 僕[ぼく]は 貰[もらい]えっこない。	貰えっこない	っこない	
\\	うそを (つく: 
\\	彼に嫌われた。	うそをついたばかりに 彼[かれ]に 嫌[きら]われた。	ついたばかりに	~ばかりに	
\\	彼は年 
\\	老けて見える。	彼[かれ]は 年[とし]のわりに 老[ふ]けて 見[み]える。	のわりに	のわりに	
\\	実家に帰った 
\\	昔の学校に行ってみた。	実家[じっか]に 帰[かえ]ったついでに 昔[むかし]の 学校[がっこう]に 行[い]ってみた。	ついでに	ついでに	
\\	この布は水に強い 
\\	、熱には弱い.	この 布[ぬの]は 水[みず]に 強[つよ]い 反面[はんめん]、 熱[ねつ]には 弱[よわ]い.	反面	反面	
\\	どんな人(でも, 
\\	、この映画は楽しめます。	どんな 人[ひと]であろうと、この 映画[えいが]は 楽[たの]しめます。	であろうと	であろうと	
\\	天気予報がいつも当たる 
\\	。	天気[てんき] 予報[よほう]がいつも 当[あ]たるとは 限[かぎ]らない。	とは限らない	とは限らない	
\\	買い物 (する: 
\\	。	買い物[かいもの]せずにはいられない。	しないでいられない	ないでいられない	
\\	この本には金を払った 
\\	価値がない。	この 本[ほん]には 金[きん]を 払[はら]っただけの 価値[かち]がない。	だけの	だけの	
\\	公園へ行くより (実は, 
\\	動物園へ行きたい。	公園[こうえん]へ 行[い]くよりむしろ 動物[どうぶつ] 園[えん]へ 行[い]きたい。	むしろ	むしろ	
\\	このドキュメントは間違え 
\\	で、全然役に立たない。	このドキュメントは 間違[まちが]えだらけで、 全然[ぜんぜん] 役に立[やくにた]たない。	だらけ	だらけ	
\\	我々は決定 
\\	偏見に左右されない。	我々[われわれ]は 決定[けってい]にあたって 偏見[へんけん]に 左右[さゆう]されない。	にあたって	にあたって・にあたり	
\\	5月末までに問題を解決すると約束 (する: 
\\	、どうしても頑張らなければならない。	
\\	月[つき] 末[まつ]までに 問題[もんだい]を 解決[かいけつ]すると 約束[やくそく]した 手前[てまえ]、どうしても 頑張[がんば]らなければならない。	した手前	手前	
\\	ぼくをおいて (行く: 
\\	ぼくをおいて 行[い]きやがったんだ!	行ってやがったんだ
\\	行きやがったんだ	てやがる	
\\	冗談は 
\\	冗談[じょうだん]は 抜[ぬ]きにして…	抜きにして	抜きにして	
\\	日本語は (勉強する: 
\\	面白くなります。	日本語[にほんご]は 勉強[べんきょう]すれば 勉強[べんきょう]するほど 面白[おもしろ]くなります。	勉強すれば勉強するほど	ば~ほど	
\\	(決める: 
\\	、やれるところまでやった方がいい。	一度[いちど] 決[き]めたら、やれるところまでやった 方[ほう]がいい。	一度決めたら	一度~たら~	
\\	嬉しい 
\\	4月から、給料が1割りほど上がる。	嬉[うれ]しいことに 4月[しがつ]から、 給料[きゅうりょう]が1 割[わ]りほど 上[あ]がる。	ことに	ことに	
\\	あの男はどんなばかなことでも (やる: 
\\	。	あの 男[おとこ]はどんなばかなことでもやりかねない。	やりかねない	かねない	
\\	大統領は中国訪問
\\	、万里の長城を訪れました。	大統領は中国 訪問[のうもん]を契機に、万里の長城を訪れました。	を契機に	を契機に	
\\	部屋の電気がまだついている 
\\	、森さんはまだ起きているようだ。	部屋[へや]の 電気[でんき]がまだついているところをみると、 森[もり]さんはまだ 起[お]きているようだ。	ところをみると	ところをみると	
\\	買っちゃった!	つい 買[か]っちゃった!	つい	つい~	
\\	英語は商業 
\\	役に立つ。	英語[えいご]は 商業[しょうぎょう]において 役に立[やくにた]つ。	において	において	
\\	警察 (にしたって 
\\	取り締まりの方法がないだろう。	警察[けいさつ]にしたところで 取り締[とりし]まりの 方法[ほうほう]がないだろう。	にしたところで	にしたところで	
\\	(たとえ)どんなことがあった 
\\	。	たとえどんなことがあったとしても....	たとえ~ても	たとえ~ても	
\\	そういう明白な事実があるのに、非を (認める: 
\\	なんてとんでもない奴だ。	そういう 明白[めいはく]な 事実[じじつ]があるのに、 非[ひ]を 認[みと]めようとしないなんてとんでもない 奴[やつ]だ。	認めようとしない	ようとしない	
\\	命令の仕方が人によって違うのでは、命令される犬 
\\	困ってしまうだろう。	命令[めいれい]の仕方が 人[ひと]によって 違[ちが]うのでは、 命令[めいれい]される 犬[いぬ]にしたって 困[こま]ってしまうだろう。	にしたって	にしたって	
\\	顔 (からすると 
\\	、彼女がミス日本になると思います。	顔[かお]からいうと、 彼女[かのじょ]がミス 日本[にっぽん]になると 思[おも]います。	から言うと・から言えば・から言って	からいうと	
\\	母の手紙にはいつも体を大切に (する: 
\\	書いてあります。	母[はは]の 手紙[てがみ]にはいつも 体[からだ]を 大切[たいせつ]にしろと 書[か]いてあります。	しろと	しろと	
\\	ホストファミリーは成田空港まで迎えに来てくれた。それ (だけでなく, 
\\	毎日いろいろ面白い所に連れて行ってくれた。	ホストファミリーは 成田空港[なりたくうこう]まで 迎[むか]えに 来[き]てくれた。そればかりでなく 毎日[まいにち]いろいろ 面白[おもしろ]い 所[ところ]に 連[つ]れて 行[い]ってくれた。	ばかりでなく	ばかりでなく	
\\	私の立場 
\\	当然でしょう。	私[わたし]の 立場[たちば]からみると 当然[とうぜん]でしょう。	からみると	からみると	
\\	彼は学生時代は抜群の成績だったが、会社に (入る: 
\\	パッとしない。	彼[かれ]は 学生[がくせい] 時代[じだい]は 抜群[ばつぐん]の 成績[せいせき]だったが、 会社[かいしゃ]に 入[はい]ってからはパッとしない。	入ってからは	てからは	
\\	彼女は「えい!」
\\	侵入者を投げ飛ばした。	彼女[かのじょ]は「えい!」とばかりに 侵入[しんにゅう] 者[しゃ]を 投げ飛[なげと]ばした。	とばかりに	とばかりに	
\\	課長は余計な質問はするな 
\\	私をにらみつけた。	課長[かちょう]は 余計[よけい]な 質問[しつもん]はするな
\\	とばかりに
\\	私[わたし]をにらみつけた。	とばかりに	とばかりに	
\\	毎日水をやった 
\\	、10月になって庭の花が綺麗に咲いた。	毎日[まいにち] 水[すい]をやったかいがあって、 10月[じゅうがつ]になって 庭[にわ]の 花[はな]が 綺麗[きれい]に 咲[さ]いた。	かいがあって	かいがあって~	
\\	国が違うと、言葉や文化 
\\	、人々の考え方まで違う。	国[くに]が 違[ちが]うと、 言葉[ことば]や 文化[ぶんか]はもちろん、 人々[ひとびと]の 考え方[かんがえかた]まで 違[ちが]う。	はもちろん	はもちろん	
\\	そういうところが実際にあるかどうか 
\\	、自分の理想とする会社はどんなところかを書いて下さい。	そういうところが 実際[じっさい]にあるかどうかは 別[べつ]として、 自分[じぶん]の 理想[りそう]とする 会社[かいしゃ]はどんなところかを 書[か]いて 下[くだ]さい。	は別として	は別として	
\\	この空模様 (から言うと 
\\	、雨になりそうだ。	この 空模様[そらもよう]からすると、 雨[あめ]になりそうだ。	からすると	からすると	
\\	その子供はまるで大人である 
\\	話し方をする。	その 子供[こども]はまるで 大人[おとな]であるかのような 話し方[はなしかた]をする。	かのような	かのよう	
\\	大木君、会議だっていうのに、外出しちゃいましたよ。
\\	書類をそろえてくれた 
\\	だよ。	
\\	大木[おおき] 君[くん]、 会議[かいぎ]だっていうのに、 外出[がいしゅつ]しちゃいましたよ。
\\	書類[しょるい]をそろえてくれただけましだよ。	だけまし	だけまし	
\\	犯人は、誰にも気づかれる 
\\	[硬い言い方])、その場所を離れた。	犯人[はんにん]は、誰にも 気[き]づかれることなく、その 場所[ばしょ]を 離[はな]れた。	ことなく	ことなく	
\\	彼は時が経つのはなんと早い 
\\	とつくづく考えた。	彼[かれ]は 時[じ]が経つのはなんと 早[はや]いことかとつくづく 考[かんが]えた。	ことか	ことか	
\\	ひろ子は自転車に乗る 
\\	、どんどん行ってしまった。	ひろ 子[こ]は 自転車[じてんしゃ]に 乗[の]るが 早[はや]いか、どんどん 行[い]ってしまった。	が早いか	が早いか	
\\	当地は夏もそれほど暑くありません。最高に暑い日でも26、7度 (程度は最高でも~で、あまり高くない)。	当地は夏もそれほど暑くありません。最高に暑い日でも26、7度といったところです。	といったところだ	といったところだ	
\\	今日 
\\	禁煙することにしていました。	今日[きょう]を 限[かぎ]りに 禁煙[きんえん]することにしていました。	を限りに	を限りに	
\\	本日 
\\	今年の研究会は終了いたします。	本日[ほんじつ]をもって 今年[ことし]の 研究[けんきゅう] 会[かい]は 終了[しゅうりょう]いたします。	をもって	をもって	
\\	これほど多くの作品を手がけた監督を
\\	氏 
\\	ほかにいない。	これほど 多[おお]くの 作品[さくひん]を 手[て]がけた 監督[かんとく]を 
\\	氏[し]をおいてほかにいない。	をおいて	をおいて	
\\	マスメディアによる情報というものは、今や一国 
\\	、世界中に伝わる。	マスメディアによる 情報[じょうほう]というものは、 今[いま]や一 国[こく]にとどまらず、 世界中[せかいじゅう]に 伝[つた]わる。	にとどまらず	にとどまらず	
\\	(地震、火事: 同じことが言える)、緊急の場合に冷静になれる人は少ないだろう。	地震であれ、火事であれ、緊急の場合に冷静になれる人は少ないだろう。	であれ~であれ	であれ~であれ・であろうと~であろうと	
\\	(デザイン, 色: 
\\	を見ても、を見ても)、彼の作品が最優秀だと思う。	デザインといい 色[しょく]といい、 彼[かれ]の 作品[さくひん]が 最優秀[さいゆうしゅう]だと 思[おも]う。	デザインといい、色といい	といい~といい	
\\	本の売れ行き 
\\	、すぐに増刷ということもあるでしょう。	本[ほん]の 売れ行[うれゆ]きいかんでは、すぐに 増刷[ぞうさつ]ということもあるでしょう。	いかんでは	いかんでは	
\\	安いホテル (はそうかもしれない..)、一流ホテルでこんなにサービスが悪いなんて許せない。	安いホテルならいざしらず、一流ホテルでこんなにサービスが悪いなんて許せない。	ならいざしらず	ならいざしらず	
\\	民衆は軍事政権の弾圧 
\\	、デモ行進を続けた。	民衆[みんしゅう]は 軍事[ぐんじ] 政権[せいけん]の 弾圧[だんあつ]をものともせず、デモ 行進[こうしん]を 続[つづ]けた。	をものともせず(に)	をものともせず	
\\	彼は力強くうなずいた。任せろと (言う: しそうなほどの状態だ)。	彼は力強くうなずいた。任せろと言わんばかりだった。	んばかりだ	んばかりだ	
\\	この子は生まれ (~のまま変わらない状態だ)優れた音感を持っていた。	この子は生まれながらに優れた音感を持っていた。	ながらに(して)	ながらに(して)	
\\	彼はどうも物事を悲観的に考える (よくない傾向・性質・くせがある)。	彼はどうも物事を悲観的に考えるきらいがある。	きらいがある	きらいがある	
\\	ごぶさたのおわび 
\\	、近況報告に先生をお尋ねした。	ごぶさたのおわびかたがた、 近況[きんきょう] 報告[ほうこく]に 先生[せんせい]をお 尋[たず]ねした。	かたがた	かたがた	
\\	小さな失敗をいちいち気にする (よくない状態では、よくない結果になるだろう)、この会社ではやっていけないよ。	小さな失敗をいちいち気にするようでは、この会社ではやっていけないよ。	ようでは	ようでは	
\\	子供の教育費 
\\	、多少の出費もしかたがない。	子供[こども]の 教育[きょういく] 費[ひ]とあれば、 多少[たしょう]の 出費[しゅっぴ]もしかたがない。	とあれば	とあれば	
\\	まさおは遊びに (出かける: 
\\	、暗くなるまで戻ってきません。	まさおは 遊[あそ]びに 出[で]かけたら 最後[さいご]、 暗[くら]くなるまで 戻[もど]ってきません。	出かけたら最後	たら最後	
\\	法律で決められて (いる: 
\\	当然の義務です。	法律[ほうりつ]で 決[き]められていようがいまいが 当然[とうぜん]の 義務[ぎむ]です。	いようがいまいが・いようといまいと	ようが~まいが・~ようと~まいと	
\\	いくら (謝る: 無意味・無駄)、彼女との関係は元には戻らないと思う。	いくら謝ったところで、彼女との関係は元には戻らないと思う。	たところで	たところで	
\\	家は広い方がいいが、(広い: 想像しているようにはよくない・悪くない)、掃除が大変だろう。	家は広い方がいいが、広ければ広いで、掃除が大変だろう。	ば~で・なら~で・たら~たで	ば~で・なら~で・たら~たで	
\\	介護ロボットを開発 (する: ために行為をする)、私たちは今日も実験を続ける。	介護ロボットを開発すべく、私たちは今日も実験を続ける。	べく	べく	
\\	何 (を手段としてある行為をする)人の価値を評価するかは難しい問題だ。	何をもって人の価値を評価するかは難しい問題だ。	をもって	をもって	
\\	子供に大切な書類を汚されて、(泣く: したいが、何かの事情があってできない)心境だ。	子供に大切な書類を汚されて、泣くに泣けない心境だ。	に~ない・(よ)うにも~ない	に~ない・(よ)うにも~ない	
\\	子供がおもちゃを欲しがったが、お金がなくて (買ってやる: したいが、何かの事情があってできない)。	子供がおもちゃを欲しがったが、お金がなくて買ってやろうにも買ってやれなかった。	に~ない・(よ)うにも~ない	に~ない・(よ)うにも~ない	
\\	彼は学生として許す (べからず 
\\	行為を行ったとして大学を退学させられた。	彼[かれ]は 学生[がくせい]として 許[ゆる]すべからざる行為[こうい]を 行[おこな]ったとして 大学[だいがく]を 退学[たいがく]させられた。	べからざる	べからざる	
\\	紳士 (責任のある立場にはそのようなことがふさわしい)、強く優しくなければならない。	紳士たるもの、強く優しくなければならない。	たるもの(は)	たるもの(は)	
\\	死者が出る 
\\	、国は初めて病気の感染拡大の深刻さに気がついたのだ。	死者[ししゃ]が 出[で]るに 至[いた]って、 国[くに]は 初[はじ]めて 病気[びょうき]の 感染[かんせん] 拡大[かくだい]の 深刻[しんこく]さに 気[き]がついたのだ。	に至って	に至って	
\\	この天才の私 (でも, 
\\	わからなかった。	この 天才[てんさい]の 私[わたし]ですらわからなかった。	で「すら」	ですら	
\\	この画家の作品は小さいものでも10万円 
\\	。	この 画家[がか]の 作品[さくひん]は 小[ちい]さいものでも10 万[まん] 円[えん]からする。	からする	からする (からある)	
\\	日々の練習 
\\	勝利だ	日々[ひび]の 練習[れんしゅう]あっての 勝利[しょうり]だ	あっての	あっての	
\\	インターネットで得たその情報は、信頼 (あるもの・あることが十分~できる)ものとは思えない。	インターネットで得たその情報は、信頼に足るものとは思えない。	に足る	に足る	
\\	人の悪口は聞く (するだけの価値がある・ない)。	人の悪口は聞くに堪えない。	に堪える・に堪えない	に堪える・に堪えない	
\\	事故現場はまったく見る 
\\	ありさまだった。	事故[じこ] 現場[げんば]はまったく 見[み]るに 耐[た]えないありさまだった。	に耐えない	に耐えない	
\\	徹夜続き、彼らは苦労 (の至り, 
\\	に達していた。	徹夜[てつや] 続[つづ]き、 彼[かれ]らは 苦労[くろう]の 極[きわ]みに 達[たっ]していた。	の極み	の極み	
\\	何しに来たって?飲みに来た 
\\	じゃないか。	何[なに]しに 来[き]たって? 飲[の]みに 来[き]たに 決[き]まっているじゃないか。	に決まっている	に決まっている	
\\	シベリアがどんなに (寒い: ~ても, 
\\	、このコートと帽子があれば大丈夫だ。	シベリアがどんなに 寒[さむ]かろうと、このコートと 帽子[ぼうし]があれば 大丈夫[だいじょうぶ]だ。	寒かろうと(も)	ようと	
\\	弱い者いじめをする等、許す 
\\	ことです。	弱[よわ]い 者[もの]いじめをする 等[ひとし]、 許[ゆる]すまじきことです。	まじき	まじき	
\\	報酬が (多い・少ない: 
\\	、私はやるべきことをやるだけだ。	報酬[ほうしゅう]が 多[おお]かろう(と/が) 少[すく]なかろう(と/が) 私[わたし]はやるべきことをやるだけだ。	多かろう(と/が)・少なかろう(と/が)	ようが・ようと	
\\	昨日は嫌な日だった。会社では社長に注意されるし、夜は友人と喧嘩してしまうし、最後は帰りの電車の中にカバンを忘れてきてしまう 
\\	だ。	昨日[きのう]は 嫌[いや]な 日[ひ]だった。 会社[かいしゃ]では 社長[しゃちょう]に 注意[ちゅうい]されるし、 夜[よる]は 友人[ゆうじん]とけんかしてしまうし、 最後[さいご]は 帰[かえ]りの 電車[でんしゃ]の 中[なか]にカバンを 忘[わす]れてきてしまうしまつだ。	しまつ	しまつ	
\\	道具が (出す: 
\\	だよ。使ったら、ちゃんと片付けなさい。	道具[どうぐ]が 出[だ]しっぱなしだよ。 使[つか]ったら、ちゃんと 片付[かたづ]けなさい。	出しっぱなし	っぱなし	
\\	あいつ 
\\	、いつも遅れてくるんだから、頭にくるよ。	あいつときたら、いつも 遅[おく]れてくるんだから、 頭[あたま]にくるよ。	ときたら	ときたら	
\\	愛惜の情 
\\	。	愛惜[あいせき]の 情[じょう]を 禁[きん]じ 得[え]ない。	を禁じ得ない	を禁じ得ない	
\\	(東京都民: 
\\	、全国民が今度の都知事選に関心を持っている。	ただ 東京[とうきょう] 都民[とみん]だけでなく、 全[ぜん] 国民[こくみん]が 今度[こんど]の 都知事[とちじ] 選[せん]に 関心[かんしん]を 持[も]っている。	ただ東京都民だけでなく・ひとり東京都民のみならず	ただ~だけでなく・ひとり~のみならず	
\\	そのテレビドラマは1回目が放映される 
\\	、国中に大きな韓国ブームを引き起こした。	そのテレビドラマは1 回[かい] 目[め]が 放映[ほうえい]されるや 否[いな]や、 国[くに] 中[ちゅう]に 大[おお]きな 韓国[かんこく]ブームを 引き起[ひきお]こした。	や否や	や否や	
\\	父は頑固だから兄の結婚には反対するか 
\\	、何も言わずに賛成した。	父[ちち]は 頑固[がんこ]だから 兄[あに]の 結婚[けっこん]には 反対[はんたい]するかと 思[おも]いきや、 何[なに]も 言[い]わずに 賛成[さんせい]した。	と思いきや	と思いきや	
\\	村で花の 
\\	美人に出会った。	村[むら]で 花[はな]のごとき 美人[びじん]に 出会[であ]った。	ごとき	ごとき	
\\	朝の連続ドラマに主演したの 
\\	、彼女はスターへの道を歩み始めた。	朝[あさ]の 連続[れんぞく]ドラマに 主演[しゅえん]したのを 皮切[かわき]りに、 彼女[かのじょ]はスターへの 道[みち]を 歩[あゆ]み 始[はじ]めた。	を皮切りに	を皮切りに	
\\	人がなんと (言う: 
\\	、私は自分の子供を信じている。	人[ひと]がなんと 言[い]おうが、 私[わたし]は 自分[じぶん]の 子供[こども]を 信[しん]じている。	言おうが	ようが	
\\	長年勤めた会社だが、退職してしまえば (したら最後~ 
\\	だ。	長年[ながねん] 勤[つと]めた 会社[かいしゃ]だが、 退職[たいしょく]してしまえばそれまでだ。	それまで	ばそれまでだ	
\\	大切なものを壊してしまったのです。買って (返す: べき, 
\\	でしょう。	大切[たいせつ]なものを 壊[こわ]してしまったのです。 買[か]って 返[かえ]さずにはすまないでしょう。	返さずにはすまない	ずにはすまない	
\\	近年、医学が目覚ましく進歩した 
\\	、病人の数は減少傾向にはない。	近年[きんねん]、 医学[いがく]が 目覚[めざ]ましく 進歩[しんぽ]したといえども、 病人[びょうにん]の 数[かず]は 減少[げんしょう] 傾向[けいこう]にはない。	といえども	といえども	
\\	日本は物価が高いから、1円 (として~ない, 
\\	無駄に使うことは出来ない。	日本[にっぽん]は 物価[ぶっか]が 高[たか]いから、 
\\	円[えん]といえども 無駄[むだ]に 使[つか]うことは 出来[でき]ない。	といえども	といえども~ない	
\\	長引く不況のため、労働者は賃金カット 
\\	しまった。	長引[ながび]く 不況[ふきょう]のため、 労働[ろうどう] 者[しゃ]は 賃金[ちんぎん]カットを 余儀[よぎ]なくされてしまった。	を余儀なくされて	を余儀なくさせる	
\\	両親の意向 
\\	、子供自身が有名校にあこがれている。	両親[りょうしん]の 意向[いこう]もさることながら、 子供[こども] 自身[じしん]が 有名[ゆうめい] 校[こう]にあこがれている。	もさることながら	もさることながら	
\\	多くの国々では、水や空気の汚染 
\\	、環境破壊が進んでいる。	多[おお]くの 国々[くにぐに]では、 水[みず]や 空気[くうき]の 汚染[おせん]と 相[あい]まって、 環境[かんきょう] 破壊[はかい]が 進[すす]んでいる。	と相まって	と相まって	
\\	土日 
\\	、ウイークデーだから大丈夫だよ。	土日[どにち]はいざしらず、ウイークデーだから 大丈夫[だいじょうぶ]だよ。	はいざ知らず	はいざ知らず	
\\	結婚して以来30年 
\\	、刺激に満ちた楽しい日々であった。	結婚[けっこん]して 以来[いらい] 
\\	年[ねん]というもの、 刺激[しげき]に 満[み]ちた 楽[たの]しい 日々[ひび]であった。	というもの	というもの	
\\	先月の食料品の売り上げが著しく伸びた 
\\	、衣料品の売り上げが落ち込んだ。	先月[せんげつ]の 食料[しょくりょう] 品[ひん]の 売り上[うりあ]げが 著[いちじる]しく 伸[の]びたのにひきかえ、 衣料[いりょう] 品[ひん]の 売り上[うりあ]げが 落ち込[おちこ]んだ。	のにひきかえ	のにひきかえ	
\\	弟は内気で、人前でスピーチ 
\\	簡単な挨拶さえできない。	弟[おとうと]は 内気[うちき]で、 人前[ひとまえ]でスピーチはおろか 簡単[かんたん]な 挨拶[あいさつ]さえできない。	はおろか	はおろか	
\\	あの人は氷の 
\\	[書き言葉]) 冷たい人だ。	あの 人[ひと]は 氷[こおり]のごとく 冷[つめ]たい 人[ひと]だ。	ごとく	ごとき	
\\	たとえ国王 (にしても, 
\\	、国民すべてを従わせることができるわけではない	たとえ 国王[こくおう]であれ、 国民[こくみん]すべてを 従[したが]わせることができるわけではない	であれ	であれ	
\\	10年前に別れた女性の居どころなど知る (わけない, 
\\	年[ねん] 前[まえ]に 別[わか]れた 女性[じょせい]の 居[きょ]どころなど 知[し]るべくもない。	べくもない	べくもない	
\\	黙っていないで、(反対する, 賛成する: 
\\	意見を言ってください。	黙[だま]っていないで、 反対[はんたい]するなり 賛成[さんせい]するなり 意見[いけん]を 言[い]ってください。	反対するなり賛成するなり	なり~なり	
\\	砂浜で遊んでいた子供たちは、(手,足: 
\\	全身砂だらけだ。	砂浜[すなはま]で 遊[あそ]んでいた 子供[こども]たちは、 手[て]といわず 足[あし]といわず 全身[ぜんしん] 砂[すな]だらけだ。	手といわず足といわず	といわず~といわず	
\\	こんな立派な賞を頂きまして、光栄 (の極み, 
\\	です。	こんな 立派[りっぱ]な 賞[しょう]を 頂[いただ]きまして、 光栄[こうえい]の 至[いた]りです。	の至り	の至り	
\\	社は人件費を削減す 
\\	大規模なリストラを行う予定だ。	
\\	社[しゃ]は 人件[じんけん] 費[ひ]を 削減[さくげん]すべく 大[だい] 規模[きぼ]なリストラを 行[おこな]う予定だ。	べく	べく	
\\	相手の言い分もわかった 
\\	、心から納得したわけではない。	相手[あいて]の 言い分[いいぶん]もわかったとはいえ、 心[こころ]から 納得[なっとく]したわけではない。	とはいえ	とはいえ	
\\	彼は交通事故に遭って、命 
\\	大怪我をしたそうです	彼[かれ]は 交通[こうつう] 事故[じこ]に 遭[あ]って、 命[いのち]にかかわる 大[だい]けがをしたそうです	にかかわる	にかかわる	
\\	変な男の人がうちの家の前を (行く, 戻る: 
\\	している。何をしているんだろう。	変[へん]な 男[おとこ]の 人[ひと]がうちの 家[いえ]の 前[まえ]を 行[い]きつ 戻[もど]りつしている。 何[なに]をしているんだろう。	行きつ戻りつ	つ~つ	
\\	どの演奏も若さと情熱が (あふれる: 
\\	満ちていた。	どの 演奏[えんそう]も 若[わか]さと 情熱[じょうねつ]があふれんばかりに 満[み]ちていた。	あふれんばかりに	んばかりに	
\\	首相は国民の批判 
\\	イラク戦争支持を表明した。	首相[しゅしょう]は 国民[こくみん]の 批判[ひはん]をよそにイラク 戦争[せんそう] 支持[しじ]を 表明[ひょうめい]した。	をよそに	をよそに	
\\	コミュニケーション (ないと~, 
\\	、人間が関与する一切の活動は成立し得ない。	コミュニケーションなくしては、 人間[にんげん]が 関与[かんよ]する 一切[いっさい]の 活動[かつどう]は 成立[せいりつ]し 得[え]ない。	なくしては	なくしては	
\\	彼は正直者 
\\	、嘘つきじゃない。	彼[かれ]は 正直[しょうじき] 者[しゃ]じゃないまでも、 嘘[うそ]つきじゃない。	じゃないまでも	ないまでも	
\\	飛行機の窓から見えたオーロラの美しさ 
\\	。	飛行機[ひこうき]の 窓[まど]から 見[み]えたオーロラの 美[うつく]しさといったらなかった。	といったらなかった	といったらない	
\\	彼の実力を考えると、今回の受賞も驚く 
\\	。	彼[かれ]の 実力[じつりょく]を 考[かんが]えると、 今回[こんかい]の 受賞[じゅしょう]も 驚[おどろ]くにあたらない。	に当たらない	に当たらない	
\\	あの犬を描いた映画は、見る人を感動 (させる: ずにいられない, 
\\	。	あの 犬[いぬ]を 描[えが]いた 映画[えいが]は、 見[み]る 人[ひと]を 感動[かんどう]させずにはおかない。	させずにはおかない	ずにはおかない	
\\	散歩 
\\	、ちょっと郵便局まで行ってきます。	散歩[さんぽ]がてら、ちょっと 郵便[ゆうびん] 局[きょく]まで 行[い]ってきます。	がてら	がてら	
\\	この祭りごとは京都 
\\	の光景です。	この 祭[まつ]りごとは 京都[きょうと]ならではの 光景[こうけい]です。	ならでは	ならでは	
\\	娘が帰って (来る: ~してから、その後ずっと~)、年老いた父親を見違えるほど元気になった。	娘[むすめ]が 帰[かえ]って 来[き]てからというもの、 年老[としお]いた 父親[ちちおや]を 見違[みちが]えるほど 元気[げんき]になった。	来てからというもの	てからというもの	
\\	集めたデータ 
\\	レポートを作成する。	集[あつ]めたデータを 踏[ふ]まえてレポートを 作成[さくせい]する。	を踏まえて	を踏まえて	
\\	兄が起こした会社は発展を続け、海外に支店を出す 
\\	。	兄[あに]が 起[お]こした 会社[かいしゃ]は 発展[はってん]を 続[つづ]け、 海外[かいがい]に 支店[してん]を 出[だ]すに 至[いた]った。	に至った	に至る	
\\	突然の事故で子供を亡くした親の悲しみは、察する 
\\	。	突然[とつぜん]の 事故[じこ]で 子供[こども]を 亡[な]くした 親[おや]の 悲[かな]しみは、 察[さっ]するにかたくない。	にかたくない	にかたくない	
\\	デパートの閉店が相次いでいる。Aデパート 
\\	すでに三つの支店が閉店した。	デパートの 閉店[へいてん]が 相次[あいつ]いでいる。 
\\	[えい]デパートに 至[いた]ってはすでに 三[みっ]つの 支店[してん]が 閉店[へいてん]した。	に至っては	に至っては	
\\	国会議員 
\\	人が、差別的な発言をするなんて信じられない。	国会[こっかい] 議員[ぎいん]ともあろう 人[ひと]が、 差別[さべつ] 的[てき]な 発言[はつげん]をするなんて 信[しん]じられない。	ともあろう	ともあろう	
\\	商品の説明のし方 
\\	、売れ行きに大きく差が出てきてしまう。	商品[しょうひん]の 説明[せつめい]のし 方[かた]いかんで、 売れ行[うれゆ]きに 大[おお]きく 差[さ]が 出[で]てきてしまう。	いかんで	いかんで	
\\	(厳しい: だけで、
\\	ではいい教育とは言えない。	ただ厳しいのみではいい 教育[きょういく]とは 言[い]えない。	ただ厳しいのみ	ただ~のみ	
\\	野山が薄緑になり、すっかり春 
\\	きた	野山[のやま]が 薄[うす] 緑[みどり]になり、すっかり 春[はる]めいてきた	めいて	めく	
\\	スミスさんは週刊誌から学術書 
\\	、幅広い日本語が読める。	スミスさんは 週刊[しゅうかん] 誌[し]から 学術[がくじゅつ] 書[しょ]に 至[いた]るまで、 幅広[はばひろ]い 日本語[にほんご]が 読[よ]める。	に至るまで	に至るまで	
\\	自分の命を犠牲にして多くの人を救った。あの男が英雄 
\\	。	自分[じぶん]の 命[いのち]を 犠牲[ぎせい]にして 多[おお]くの 人[ひと]を 救[すく]った。あの 男[おとこ]が 英雄[えいゆう]でなくてなんだろう。	でなくてなんだろう	でなくてなんだろう	
\\	先輩があんなに親切に言ってくれる 
\\	、彼はどうして断るのだろう。	先輩[せんぱい]があんなに 親切[しんせつ]に 言[い]ってくれるものを、 彼[かれ]はどうして 断[ことわ]るのだろう。	ものを	ものを	
\\	ひどく頭が痛くて (起きる: 
\\	。	ひどく 頭[あたま]が 痛[いた]くて 起[お]きようにも 起[お]きられない。	起きようにも起きられない	ようにも~ない	
\\	何を考えている (か 
\\	、息子の心の中はさっぱり分からない。	何[なに]を 考[かんが]えているやら、 息子[むすこ]の 心[こころ]の 中[なか]はさっぱり 分[わ]からない。	やら	やら	
\\	この区では、場所 
\\	路上喫煙は禁止です。	この 区[く]では、 場所[ばしょ]のいかんにかかわらず 路上[ろじょう] 喫煙[きつえん]は 禁止[きんし]です。	のいかんにかかわらず・のいかんによらず	いかんにかかわらず・いかんによらず	
\\	事情 
\\	、欠席は欠席だ。	事情[じじょう]のいかんによらず、 欠席[けっせき]は 欠席[けっせき]だ。	のいかんによらず・のいかんにかかわらず	いかんによらず・いかんにかかわらず	
\\	優勝できたのは、チーム全員の協力が (ある: 
\\	だ	優勝[ゆうしょう]できたのは、チーム 全員[ぜんいん]の 協力[きょうりょく]があればこそだ	あればこそ	ばこそ	
\\	人間90歳 
\\	初めて分かることもある。	人間[にんげん] 
\\	歳[さい]にして 初[はじ]めて 分[わ]かることもある。	にして	にして	
\\	店が狭い (のゆえに, 
\\	、お客様には窮屈な思いをさせて申し訳ございません。	店[みせ]が 狭[せま]いこととて、お 客様[きゃくさま]には 窮屈[きゅうくつ]な 思[おも]いをさせて 申し訳[もうしわけ]ございません。	こととて	こととて	
\\	貧しさ (こととて, 
\\	、子供が働かなければならない社会もある。	貧[まず]しさのゆえに、 子供[こども]が 働[はたら]かなければならない 社会[しゃかい]もある。	のゆえ(に)	ゆえに	
\\	試験にパスできなくて、悔しい (非常に...) だ。	試験[しけん]にパスできなくて、 悔[くや]しい 限[かぎ]りだ。	限り	限りだ	
\\	子供も小さいときは素直だが、中学生 
\\	、親の言うことを聞かなくなる。	子供[こども]も 小[ちい]さいときは 素直[すなお]だが、 中学生[ちゅうがくせい]ともなると、 親[おや]の 言[い]うことを 聞[き]かなくなる。	ともなると	ともなると	
\\	雪解けの水は指が切れる 
\\	冷たい。	雪解[ゆきど]けの 水[みず]は 指[ゆび]が 切[き]れるかと 思[おも]うほど 冷[つめ]たい。	かと思うほど	かと思うほど	
\\	試合に勝った
\\	、全力を出し切れたことを誇りに思う。	試合[しあい]に 勝[か]ったのにもまして、 全力[ぜんりょく]を 出し切[だしき]れたことを 誇[ほこ]りに 思[おも]う。	のにもまして	にもまして	
\\	これは簡単な計算だから計算器を使う (必要ない, 
\\	。	これは 簡単[かんたん]な 計算[けいさん]だから 計算[けいさん] 器[き]を 使[つか]うまでもない。	までもない	までもない	
\\	小さい子供は、お母さんが洗濯する 
\\	、服を汚してしまいます。	小[ちい]さい 子供[こども]は、お 母[かあ]さんが 洗濯[せんたく]するそばから、 服[ふく]を 汚[よご]してしまいます。	そばから	そばから	
\\	信じられないなあ! 私が東大に入学できた 
\\	。	信[しん]じられないなあ! 私[わたし]が 東大[とうだい]に 入学[にゅうがく]できたなんて。	なんて	なんて	
\\	見る 
\\	ぼんやり外を見ていたら、不意に大きなカラスが飛んで来た	見[み]るともなくぼんやり 外[そと]を 見[み]ていたら、 不意[ふい]に 大[おお]きなカラスが 飛[と]んで 来[き]た	ともなく・ともなしに	ともなく・ともなしに	
\\	もう時間がない。1分 (といえども~ない, 
\\	無駄にできない。	もう 時間[じかん]がない。1 分[ふん]たりとも 無駄[むだ]にできない。	たりとも	たりとも~ない	
\\	冬山登山をするの 
\\	、そんな大げさな格好は要りません。	冬山[ふゆやま] 登山[とざん]をするのではあるまいし、そんな 大[おお]げさ 格好[かっこう]は 要[い]りません。	ではあるまいし	ではあるまいし	
\\	田中さんがはっきりと反対の意見を言う 
\\	意外でした。	田中[たなか]さんがはっきりと 反対[はんたい]の 意見[いけん]を 言[い]うとは 意外[いがい]でした。	とは	とは	
\\	優勝を争うチーム同士の対決 
\\	、スタジアム超満員だ。	優勝[ゆうしょう]を 争[あらそ]うチーム 同士[どうし]の 対決[たいけつ]とあって、スタジアム 超[ちょう] 満員[まんいん]だ。	とあって	とあって	
\\	敬慕 (する: 
\\	師を失った。	敬慕[けいぼ]してやまない 師[し]を 失[うしな]った。	してやまない	てやまない	
\\	そのニュースが報道される 
\\	、放送局に問い合わせの電話が殺到した。	そのニュースが 報道[ほうどう]されるなり、 放送[ほうそう] 局[きょく]に 問[と]い 合[あ]わせの 電話[でんわ]が 殺到[さっとう]した。	なり	なり	
\\	この非常時 
\\	、あなたはどうしてそんなに平気でいられるのですか。	この 非常時[ひじょうじ]にあって、あなたはどうしてそんなに 平気[へいき]でいられるのですか。	にあって	にあって	
\\	試験中の不正行為は、校則 
\\	処理する。	試験[しけん] 中[ちゅう]の 不正[ふせい] 行為[こうい]は、 校則[こうそく]に 即[そく]して 処理[しょり]する。	に即して	に即して	
\\	来年度私がもらえそうな奨学金はせいぜい5万円 
\\	。	来年度[らいねんど]わたしがもらえそうな 奨学[しょうがく] 金[きん]はせいぜい 
\\	万[まん] 円[えん]というところだ。	というところだ	というところだ	
\\	お忙しい 
\\	多くの方にお集まりいただき、感激に堪えません。	お 忙[いそが]しいところを 多[おお]くの 方[ほう]にお 集[あつ]まりいただき、 感激[かんげき]に 堪[こた]えません。	ところを	ところを	
\\	私の言葉に特別な意味はない。ただ、彼を慰めようと思って言った (だけ, 
\\	だ。	私[わたし]の 言葉[ことば]に 特別[とくべつ]な 意味[いみ]はない。ただ、 彼[かれ]を 慰[なぐさ]めようと 思[おも]って 言[い]ったまでだ。	まで	までだ	
\\	(学生:
\\	、勉強を第一にしなさい。	学生[がくせい]である 以上[いじょう]、 勉強[べんきょう]を 第[だい]一にしなさい。	である以上	以上(は)	
\\	第一子御出産の 
\\	、心からお祝い申し上げます。	第[だい] 一子[いっし] 御[ご] 出産[しゅっさん]の 由[よし]、 心[こころ]からお 祝[いわ]い 申し上[もうしあ]げます。	由	由	
\\	この台風で家までの交通機関がストップしてしまったら、歩いて帰る 
\\	だ。	この 台風[たいふう]で 家[いえ]までの 交通[こうつう] 機関[きかん]がストップしてしまったら、 歩[ある]いて 帰[かえ]るまでだ。	まで	までだ	
\\	彼女の話は涙 (~なく, 
\\	は、聞けない。	彼女[かのじょ]の 話[はなし]は 涙[なみだ]なしには、 聞[き]けない。	なしに	なしに	
\\	約束した 
\\	、守るべきだと思う。	約束[やくそく]した 以上[いじょう]、 守[まも]るべきだと 思[おも]う。	以上	以上(は)	
\\	何とか父に気づかれる (なく, 
\\	、家を出ることが出来た。	何[なん]とか 父[ちち]に 気[き]づかれることなしに、 家[いえ]を 出[で]ることが 出来[でき]た。	ことなしに	なしに	
\\	試験に合格せん 
\\	、この1年間努力を続けてきた。	試験[しけん]に 合格[ごうかく]せんがため、この1 年間[ねんかん] 努力[どりょく]を 続[つづ]けてきた。	がため	んがため	
\\	あの子供も子供 
\\	いろいろ心配しているのだ。	あの 子供[こども]も 子供[こども]なりにいろいろ 心配[しんぱい]しているのだ。	なりに	なりに	
\\	20キロ 
\\	荷物を背負って7階まで階段を上った。	20キロからある 荷物[にもつ]を 背負[せお]って7 階[かい]まで 階段[かいだん]を 上[のぼ]った。	からある	からある	
\\	このように地球温暖化が進むとは、30年前には、想像 (さえ, 
\\	しなかった。	このように 地球[ちきゅう] 温暖[おんだん] 化[か]が 進[すす]むとは、 
\\	年[ねん] 前[まえ]には、 想像[そうぞう]だにしなかった。	だに	だに	
\end{CJK}
\end{document}